\documentclass[10pt,a4paper,onecolumn,titlepage,twoside,openany]{book}
% \usepackage[utf8]{vietnam}
%\usepackage{fouriernc}
\usepackage{tasks}
%%%%%%%%%%%%% KIỂU MÀU VÀ ĐỘ RỘNG NOTE
\def\kieumau{Y} %Y: Màu; N: đen-trắng
% \def\kieumau{N} %Y: Màu; N: đen-trắng
\usepackage{xcolor} 
\def\leftnote{5} %Độ rộng cột Note
%%%%%%%%%%%%% ĐN CƠ BẢN
\input{cautrucDT/color\kieumau} %MÀU
%=====================================
% Khai báo nhóm Tex (cơ bản)
%=====================================
\usepackage{amsmath,amssymb,mathrsfs,maybemath,xlop,polynom,slashbox}
\usepackage{yhmath} %\let\widering\relax %cần khi sd với font fouriernc

\usepackage{enumerate}
\usepackage{tikz} 
\usepackage{tkz-euclide}
%\usepackage{ex_tkz-euclide}
%\usetkzobj{all}
\usepackage{tikz-3dplot}
\usepackage{tkz-tab}
\usepackage{pifont} %kí hiệu đặc biệt
% \usepackage{xcolor}
%\usepackage{bbding}
%\usepackage{array}
\usepackage{tasks}
% \usepackage{casiovn}
%==========
\usetikzlibrary{math,through,calc,intersections,angles,quotes,shapes,shapes.geometric,arrows,patterns,snakes,matrix,chains,arrows.meta,decorations.shapes,decorations.fractals,decorations.markings,shadows}
\usetikzlibrary{positioning,decorations.text,decorations.pathmorphing}% Để uốn cong văn bản 
\usetikzlibrary{shadings,fadings} %ĐỔ BÓNG
\usepackage{pgfplots}
\usepackage{pgfornament}
\usepgfplotslibrary{fillbetween}
\pgfplotsset{compat=1.9}
\usepackage[hidelinks,unicode]{hyperref}
\usepackage{currfile}
\usepackage[outline]{contour} %viền
\usepackage{fontawesome} % Gói kí hiệu
\usepackage{lipsum} %Lấy text
\usepackage{tabularx}
%%---------
%\usepackage{setspace}
%\usepackage{scrextend}
\usepackage{varwidth}
%===========Bảng
\usepackage{longtable,multirow,makecell}
\usepackage{diagbox}
\renewcommand{\tabcolsep}{3mm}
\newcolumntype{C}[1]{>{\centering\arraybackslash}p{#1}}
\newcolumntype{L}[1]{>{\raggedright\arraybackslash}p{#1}}
%-----------Trang vb


%%%%%%%%%%%%% Các thông số trang tài liệu
\def\tren{1.5}\def\duoi{1.5}\def\trai{1.25}\def\phai{0.75} %cách lề
\def\topset{0.75} %kc giữa đáy header và vùng vb
\def\botset{0.75} %kc giữa đỉnh footer và vùng vb
%\usepackage{ifoddpage}
\pgfmathsetmacro{\mepphai}{\phai+\leftnote} 
%\usepackage[top=\tren cm, bottom=\duoi cm, left=\trai cm, right=\mepphai cm] {geometry}
%%%%%%%%%%%%%
%---------------------------------Các thông số trang tài liệu
\pgfmathsetmacro{\so}{\leftnote - 0.5} 
\usepackage[top=\tren cm, bottom=\duoi cm, left=\trai cm, right=\mepphai cm,
marginparwidth=\so cm, marginparsep=5mm,
%,headsep=6mm
%,footskip=10mm
] {geometry}
%-------------------------------------
\usepackage{marginnote}
\setlength{\marginparwidth}{\so cm}
\renewcommand*{\marginfont}{\small}
%--------------Gói trắc nghiệm EX-TEST
% \usepackage[dethi]{ex_test}
\usepackage[loigiai]{ex_test} 
% \usepackage[solcolor]{ex_test}
%----Lời giải, Hiền thị tên EX; Dấu kết thúc
\font\damEX=ugqb8v at 11pt
\def\loigiaiEX{\color{\mauLG}\damEX\strut\faCommenting\ Lời giải.}
%lời giải EXS
\def\loigiaiEXS{\loigiaiEX{\fontsize{8}{16}\selectfont\color{\maucham}\dotfill}}
%--
\renewcommand{\nameex}{\damEX\color{\mauEX} CÂU}
\newtheorem{EX}{\nameex} %MÔI TRƯỜNG PHỤ CHO TÁCH CÂU
\def\mauVuong{cyan}
\def\qedEX{\color{\mauVuong}\ensuremath{\square}}
%--------------Cài đặt lại dòng kẻ \dotline
\renewcommand{\dotlineEX}[1]{
	\def\numlinedot{#1}
	\par
	\foreach \dotline in{1,...,\numlinedot}
	{
		\noindent
		\fontsize{8}{16}\selectfont
		\color{\maucham}\dotfill
		\par
	}
}
% sd cho \dongcham
\newcommand{\dotlineEXS}[1]{
	\def\numlinedot{#1}
	\foreach \dotline in{1,...,\numlinedot}
	{
		\noindent
		\fontsize{8}{16}\selectfont
		\color{\maucham}\dotfill
		\par
	}
}
%---------- Khai báo viết tắt, in đáp án
\newcommand{\hoac}[1]{ %hệ hoặc
	\left[\begin{aligned}#1\end{aligned}\right.}
\newcommand{\heva}[1]{ %hệ và
	\left\{\begin{aligned}#1\end{aligned}\right.}
%--In đáp án
\newcommand{\indapan}[2]{
	\addcontentsline{toc}{subsection}{\sf Bảng đáp án} % đưa MT vào mục lục
	\begin{center}
		\begin{tikzpicture}%
			\node[thick,scale=1,fill=\mauEX!2,draw=\maufoot,minimum width=3.5cm,minimum height=0.1cm,rounded corners=2mm]{\fontfamily{qag}\fontsize{11}{11}\selectfont\bfseries\color{\mauEX} BẢNG ĐÁP ÁN};
		\end{tikzpicture}%
	\end{center}
	\inputansbox{#1}{#2}
}
%----------
\usepackage{esvect}
\def\vec{\vv} %vecto
\def\overrightarrow{\vv}
%Lệnh song song
\DeclareSymbolFont{symbolsC}{U}{txsyc}{m}{n}
\DeclareMathSymbol{\varparallel}{\mathrel}{symbolsC}{9}
\DeclareMathSymbol{\parallel}{\mathrel}{symbolsC}{9}
%--------------------------
% HEADER AND FOOTER STYLING
%--------------------------
%--------------------------
\newcommand{\myfancyhead}{% trên và chấm trái
		\boldmath
\begin{tikzpicture}[remember picture,overlay,>=stealth]
		\path ([yshift=-\tren cm+0.5*\topset cm]current page.north west) coordinate (AA)
		++(\paperwidth,0)coordinate (BB); 
\checkoddpage\ifoddpage %nếu trang lẻ
		%-----đường kẻ
		\draw[\maufoot, line width=2pt] 
		([xshift=\trai cm]AA) --([xshift=-\phai cm]BB);
		%-----bên phải
		\node[text=\maufoot, anchor=south east,inner sep=0pt] at ([xshift=-\phai cm,yshift=4pt]BB){
			\fontfamily{qag}\fontsize{8.5pt}{12pt}\selectfont 
			{\color{\mauSO}\faMapMarker}\,\, \diachi\,\,{\color{\mauSO}\faMapMarker}
		};
		%-----bên trái
		\node[text=\maufoot, anchor=south west,inner sep=0pt] at ([xshift=\trai cm,yshift=4pt]AA){
			\fontfamily{qag}\fontsize{10pt}{10pt}\selectfont\bfseries\faEdit\, \tenchuyende
		};
		%----- Kẻ đứng
		\draw[\maufoot] ([xshift=-\mepphai cm+2.5mm]BB)--([yshift=\duoi cm-0.75*\botset cm,xshift=-\mepphai cm+2.5mm]current page.south east);
		%--		
		\path ([yshift=-\tren cm+0.5*\topset cm-0.5cm,xshift=-\phai cm-0.5*\leftnote cm+2.5mm]current page.north east) coordinate (DDD); 
		\begin{scope}
			\clip ([yshift=-\tren cm+0.5*\topset cm-1pt,xshift=-\phai cm]current page.north east) rectangle ([yshift=\duoi cm-0.5*\botset cm,xshift=-\mepphai cm+5mm]current page.south east);% cắt chấm
			\node[inner sep =0pt,scale=1,anchor=north] at ([yshift=0cm,xshift=0pt]DDD) {
				\parbox{\leftnote cm}{\centering
					\def\maucham{\maufoot}\dotlineEX{60}
				}
			};
			%--note dưới
			\node[inner sep =6pt, text=white,scale=1,anchor=north,fill=\maufoot] (noteduoi) at ([yshift=2.5mm]DDD) {
				\parbox{\leftnote cm-5mm-12pt}{ \fontsize{11}{1}\fontfamily{qag}\selectfont\bfseries\centering
					QUICK NOTE
				}
			};
			\draw[\maufoot, line width=0.4pt] ([yshift=-2pt]noteduoi.south west)--([yshift=-2pt]noteduoi.south east);
		\end{scope}
\else %chẵn
		%-----đường kẻ
		\draw[\maufoot, line width=2pt] 
		([xshift=\phai cm]AA) --([xshift=-\trai cm]BB);
		%-----bên trái
		\node[text=\maufoot, anchor=south west,inner sep=0pt] at ([xshift=\phai cm,yshift=4pt]AA){
			\fontfamily{qag}\fontsize{8.5pt}{12pt}\selectfont 
			{\color{\mauSO}\faMapMarker}\,\, \diachi\,\,{\color{\mauSO}\faMapMarker}
		};
		%-----bên phải
		\node[text=\maufoot, anchor=south east,inner sep=0pt] at ([xshift=-\trai cm,yshift=4pt]BB){
			\fontfamily{qag}\fontsize{10pt}{10pt}\selectfont\bfseries\faEdit\, \tenchuyende
		};
		%----- Kẻ đứng
		\draw[\maufoot] ([xshift=\mepphai cm-2.5mm]AA)--([yshift=\duoi cm-0.75*\botset cm,xshift=\mepphai cm-2.5mm]current page.south west);
		%--		
		\path ([yshift=-\tren cm+0.5*\topset cm-0.5cm,xshift=\phai cm+0.5*\leftnote cm-2.5mm]current page.north west) coordinate (DDD); 
		\begin{scope}
			\clip ([yshift=-\tren cm+0.5*\topset cm-1pt,xshift=\phai cm]current page.north west) rectangle ([yshift=\duoi cm-0.5*\botset cm,xshift=\mepphai cm-5mm]current page.south west);% cắt chấm
			\node[inner sep =0pt,scale=1,anchor=north] at ([yshift=0cm,xshift=0pt]DDD) {
				\parbox{\leftnote cm}{\centering
					\def\maucham{\maufoot}\dotlineEX{60}
				}
			};
			%--note dưới
			\node[inner sep =6pt, text=white,scale=1,anchor=north,fill=\maufoot] (noteduoi) at ([yshift=2.5mm]DDD) {
				\parbox{\leftnote cm-5mm-12pt}{ \fontsize{11}{1}\fontfamily{qag}\selectfont\bfseries\centering
					QUICK NOTE
				}
			};
			\draw[\maufoot, line width=0.4pt] ([yshift=-2pt]noteduoi.south west)--([yshift=-2pt]noteduoi.south east);
		\end{scope}
\fi
\end{tikzpicture}%
}
% trên mục lục
\newcommand{\headmucluc}{%
	\boldmath
\begin{tikzpicture}[remember picture,overlay,>=stealth]
	\path ([yshift=-\tren cm+0.5*\topset cm]current page.north west) coordinate (AA)
	++(\paperwidth,0)coordinate (BB); 
\checkoddpage\ifoddpage %nếu trang lẻ
	%-----đường kẻ
	\draw[\maufoot, line width=2pt] 
	([xshift=\trai cm]AA) --([xshift=-\phai cm]BB);
	%-----bên phải
	\node[text=\maufoot, anchor=south east,inner sep=0pt] at ([xshift=-\phai cm,yshift=4pt]BB){
		\fontfamily{qag}\fontsize{8.5pt}{12pt}\selectfont 
		{\color{\mauSO}\faMapMarker}\,\, \diachi\,\,{\color{\mauSO}\faMapMarker}
	};
	%-----bên trái
	\node[text=\maufoot, anchor=south west,inner sep=0pt] at ([xshift=\trai cm,yshift=4pt]AA){
		\fontfamily{qag}\fontsize{12pt}{12pt}\selectfont\bfseries\faEdit\, \tenchuyende
	};
\else %chẵn
	%-----đường kẻ
	\draw[\maufoot, line width=2pt] 
	([xshift=\phai cm]AA) --([xshift=-\trai cm]BB);
	%-----bên trái
	\node[text=\maufoot, anchor=south west,inner sep=0pt] at ([xshift=\phai cm,yshift=4pt]AA){
		\fontfamily{qag}\fontsize{8.5pt}{12pt}\selectfont 
		{\color{\mauSO}\faMapMarker}\,\, \diachi\,\,{\color{\mauSO}\faMapMarker}
	};
	%-----bên phải
	\node[text=\maufoot, anchor=south east,inner sep=0pt] at ([xshift=-\trai cm,yshift=4pt]BB){
		\fontfamily{qag}\fontsize{12pt}{12pt}\selectfont\bfseries\faEdit\, \tenchuyende
	};
\fi
\end{tikzpicture}%
}
%===========================
\newcommand{\myfancyfoot}{% dưới
	\begin{tikzpicture}[remember picture,overlay]
	\path ([yshift=\duoi cm-0.75*\botset cm]current page.south west) coordinate (AA)
	++(\paperwidth,0)coordinate (BB); 
	\checkoddpage\ifoddpage %nếu trang lẻ
		%---kẻ
		\draw[\maufoot, line width=2pt] ([xshift=2*\trai cm+4pt]AA)--([xshift=-\phai cm+3pt]BB);
		%-----bên trái
		\fill[fill=\maufoot, rounded corners=2mm] ([xshift=2*\trai cm,yshift=0.25 cm]AA) rectangle +(-3*\trai cm,-0.5cm);
		%-----trang
		\node[anchor=west,text=white,inner sep=0pt,xshift=-0.75cm] at ([xshift=2*\trai cm]AA) {\fontfamily{put}\bfseries\thepage};
		%-----tên tg
		\node[anchor=west,text=\maufoot,inner sep=0pt,fill=white] at ([xshift=2*\trai cm]AA){\fontfamily{qag}\fontsize{9pt}{1pt}\selectfont\bfseries \,\,\, \tentacgia \,\,\, };
	\else %chẵn
		%---kẻ
		\draw[\maufoot, line width=2pt] ([xshift=-2*\trai cm+4pt]BB)--([xshift=\phai cm-3pt]AA);
		%-----bên trái
		\fill[fill=\maufoot, rounded corners=2mm] ([xshift=-2*\trai cm,yshift=0.25 cm]BB) rectangle +(3*\trai cm,-0.5cm);
		%-----trang
		\node[anchor=east,text=white,inner sep=0pt,xshift=0.75cm] at ([xshift=-2*\trai cm]BB) {\fontfamily{put}\bfseries\thepage};
		%-----tên tg
		\node[anchor=east,text=\maufoot,inner sep=0pt,fill=white] at ([xshift=-2*\trai cm]BB){\fontfamily{qag}\fontsize{9pt}{1pt}\selectfont\bfseries \,\,\, \tentacgia \,\,\,};
	\fi
	\end{tikzpicture}%
}
%======================Head chapter theo note, fullwidth
%----------------------
\usepackage{changepage}
\strictpagecheck
\usepackage{lastpage}
\usepackage{fancyhdr,lastpage}
\pagestyle{fancy}
\fancyhf{}
\fancypagestyle{plain}{
	\fancyhead[LO,RE]{\headmucluc}
	\fancyfoot[LO,RE]{\myfancyfoot}
}
\fancyhead[LO,RE]{\myfancyhead}
\fancyfoot[LO,RE]{\myfancyfoot}
\renewcommand{\footrulewidth}{0pt}
\renewcommand{\headrulewidth}{0pt}
%--------------4.2
\usepackage[most]{tcolorbox}
\colorlet{tcbcol@back}{tcbcolback}
\colorlet{tcbcol@frame}{tcbcolframe}
%---------------------------------------------------------------
% ĐỊNH NGHĨA SECTION. SUBSECTION, SUBSUBSECTION ... THEO Ý RIÊNG
%---------------------------------------------------------------
\usepackage[explicit]{titlesec} % để gọi #1
\usepackage{titledot} % gói lệnh chứa cả titlesec và titletoc
%=====================================
\setcounter{secnumdepth}{4} %độ sâu
\renewcommand{\thechapter}{\Roman{chapter}}
\renewcommand{\thesection}{\arabic{section}}
\renewcommand{\thesubsection}{\Alph{subsection}}
\renewcommand{\thesubsubsection}{\arabic{subsubsection}}
%--------------Tròn
\newcommand{\tron}[1]{
	\begin{tikzpicture}[baseline=(A.base)]%
		\node[circle,draw=\mauSUBSEC,line width=0.5pt,fill=white,inner sep=2pt,outer sep=1pt] (A) {\color{white} #1};
		\node[circle,draw=none,fill=\mauSUBSEC,inner sep=1pt,outer sep=1pt] (A) {\color{white} #1};
	\end{tikzpicture}%
}
%================= Đn chương
\font\fontchap=ugqb8v at 21pt
\titlespacing{\chapter}{0cm}{0cm}{0.5cm}[0cm] %1: , 2: Trên, 3: dưới
\titleformat{\chapter}[display]
{\fontsize{20pt}{20pt}\fontfamily{qag}\selectfont\bfseries\color{\mauCHUONG}} %định dạng chung
{\fontsize{16pt}{20pt}\selectfont\chaptername\, \thechapter.} %đánh số
{1mm}
{\fontchap\centering\MakeUppercase{#1}}
[\vspace{0cm}]
%============================Mục lục - Chapter*
\titleformat{name=\chapter,numberless}[display]
{\fontsize{14pt}{16pt}\fontfamily{qag}\selectfont\bfseries\color{\mauCHUONG}} %định dạng chung
{}
{-1em}
{%
	\begin{tikzpicture}
		%-----Nội dung
		\node[inner sep=0pt,right] (ndchuong) at (0,0){\fontchap \MakeUppercase{#1}};
%		%-----Đường kẻ ngang
%		\begin{scope}
%			\clip (0,-0.75) rectangle +(\textwidth,1.5);
%			\draw[\mauchuong,line width=2pt] (ndchuong.south east)++(10pt,8pt) --++(\linewidth,0);
%		\end{scope}
	\end{tikzpicture}
}
[
\vspace{-3mm}
%\thispagestyle{empty}
]
%--------Đn Section---------------------------
%\titlespacing*{\section}{0cm}{0cm}{0cm}[0cm]
\titleformat
{\section}
{\color{\mauSEC}\fontfamily{qag}\fontsize{16pt}{1pt}\selectfont\bfseries\centering}
{Bài\,\thesection.}
{3mm}
{\MakeUppercase{#1}}
[]
%-------Đn subsection---------------------------
\titlespacing{\subsection}{0cm}{0cm}{0cm}[0cm]
\titleformat{\subsection}
{\normalfont\fontsize{15pt}{20pt}\fontfamily{put}\selectfont\bfseries\color{\mausubsec}}
{\thesubsection.}
{3mm}
{\MakeUppercase{#1}}
[]
%----------ĐN subsubsection-----------------------
\titlespacing{\subsubsection}{0pt}{0mm}{0mm}[0cm]
\titleformat{\subsubsection}
{\fontsize{13pt}{18pt}\fontfamily{put}\selectfont\bfseries\color{\mausubsubsec}}
{\thesubsubsection.}
{3mm}
{#1}
[]
%----------ĐN paragraph-----------------------
\titlespacing{\paragraph}{0pt}{0mm}{0mm}[0cm]
\titleformat{\paragraph}
{\fontsize{11.5pt}{17pt}\fontfamily{put}\selectfont\bfseries\color{\mausubsubsec}}
{\theparagraph.}
{3mm}
{#1}
[]
%============================
\def\itemKN{\color{\mauitemKN}\faCheckSquareO}
\def\itemCI{\color{\mauitemCI}\faCheckCircleO}
%%======= Thiết lập labelitem, labelenumerate
%\renewcommand{\labelitemi}{\color{red}\faCheckSquareO}
\renewcommand{\labelitemi}{\color{\mauitem}\faCheckCircleO}
\renewcommand{\labelitemii}{\color{\mauitem}\bf ---}
\renewcommand{\labelitemiii}{\color{\mauitem}\bf +}
\renewcommand{\labelenumi}{\alph{enumi})}
%\renewcommand{\labelenumii}{\color{blue}\bf\arabic{enumi}.\arabic{enumii}}
%============================
%============================
% Canh chỉnh mục lục chính
\setcounter{secnumdepth}{4} %Độ sâu đánh số
\setcounter{tocdepth}{2} %Độ sâu mục lục
\contentsmargin{0cm}
%~~~~~~~~~~~~~~~~~~~~~
\renewcommand*\l@part[2]{%
	\ifnum \c@tocdepth >-2\relax
	\addpenalty{-\@highpenalty}%
	\addvspace{10pt \@plus\p@}%
	\setlength\@tempdima{3em}%
	\begingroup
	\hypersetup{linkcolor=violet}
	\tikz[remember picture, overlay]{
		\fill[\mauPHAN] (0,0) rectangle +(\textwidth,1);
		\draw (0,0.5) node[right=5pt]
		{\color{white}\fontsize{16pt}{1pt}\fontfamily{qag}\selectfont\bfseries  {\scshape Phần} #1};
	}
	\par\smallskip
	\penalty\@highpenalty
	\endgroup
	\fi
}
%------------------------
%\titlecontents{part}[0pc]
%{\addvspace{10pt}%
%	\color{red!70!black}\fontsize{18pt}{1pt}\fontfamily{qag}\selectfont\bfseries 
%}%
%{}
%{}
%{}%
%~~~~~~~~~~~~~~~~~~~~~
\titlecontents{chapter}[6.5pc] % nd cách trái
{\addvspace{5pt}%
	\color{\mauCHUONG}\fontsize{13pt}{16pt}\fontfamily{put}\selectfont\bfseries
}%
{\contentslabel[\chaptertitlename\,\thecontentslabel.]{6.5pc}} %nhãn
{}
{\hfill\bfseries\thecontentspage
}%
[\vspace*{5pt}]
%~~~~~~~~~~~~~~~~~~~~~
\titlecontents{section}[10pc]
{\addvspace{0pt}\bfseries\color{\mauSEC}}
{\fontsize{12.5pt}{15pt}\selectfont\sffamily\contentslabel[{Bài\,\thecontentslabel.}]{3.5pc}}
{}
{\hfill
	\thecontentspage
}
[]
%~~~~~~~~~~~~~~~~~~~~~
\titlecontents{subsection}[10pc]
{\addvspace{0pt}\color{\mauSUBSEC}}
{\fontsize{12pt}{15pt}\selectfont\sffamily\contentslabel[\tron{\thecontentslabel}]{2.8pc}}
{}
{{\tiny\dotfill}\thecontentspage}
[]
%~~~~~~~~~~~~~~~~~~~~~
%--------------------------
% ĐỊNH NGHĨA CÁC MÔI TRƯỜNG 
%--------------------------
\listenumerate{dn,dl,tc,nx,ex}%xuống dòng khi liệt kê
\theoremstyle{plain} %
\theoremheaderfont{\scshape} %đầu
\theorembodyfont{\normalfont} % thân
\theoremseparator {.} % Ngăn cách
\newtheorem{dn}{\color{\maudn}\faBolt\, Định nghĩa}[section]
%===================================ĐNghĩa
\theoremstyle{plain} %
\theoremheaderfont{\fontfamily{put}\bfseries} %đầu
\theorembodyfont{\normalfont} % thân
\theoremseparator {.} % Ngăn cách
\newtheorem{vd}{\color{\mauVD}\damEX
	%\faToggleOn\ 
	%\faUnlink\ 
	VÍ DỤ}%[section]
\newtheorem{bt}{\color{\mauBT}\damEX BÀI}
%===================================
\theoremstyle{plain} %
\theoremheaderfont{\scshape} %đầu
\theorembodyfont{\slshape} % thân 
\theoremseparator {.} % Ngăn cách 
\newtheorem{dl}{\color{\maudl}\faBolt\, Định lí}[section]
\newtheorem{tc}{\color{\mauhq!70!black}\faBolt\, Tính chất}[section]
\newtheorem{hq}{\color{\mauhq!70!black}\faBolt\, Hệ quả}[section]
%====================================
\theoremstyle{nonumberplain} %ko đánh số, ko xuống dòng
\theoremheaderfont{\scshape} %đầu
\theorembodyfont{\normalfont} %phần thân
\theoremseparator {.} %ngăn cách
\newtheorem{nx}{\color{\mauhq!70!black}\faBolt\, Nhận xét}
\newtheorem{tomtat}{\!\!\!\!\!\!\!\!}
%====================================Hộp
%--------------------Chú ý
\newenvironment{note}
{\begin{tcolorbox}
		[enhanced jigsaw,breakable,pad at break*=1mm,
		opacityback=0,boxrule=0pt,frame hidden,
		left=8mm, right=0pt, bottom=0pt, top=0pt,
		before skip=1mm,
		after skip=1mm,
		underlay unbroken and first={
			\draw ([xshift=0.3cm,yshift=-0.32cm]interior.north west) node[\mauly]{\large\bfseries \faExclamationTriangle};
		},
		fontupper=\it,
		]}
	{\end{tcolorbox}}
%\let\mynote\note
%\renewcommand{\note}{\mynote{\bfseries\color{\mauly}Lưu ý:}} 
%---------------Dạng toán
\newcounter{dang}\setcounter{dang}{0}
\renewcommand{\thedang}{\arabic{dang}}
%---Dạng 1
\newtcolorbox{dang}[1]{
	fonttitle=\fontfamily{qag}\bfseries,%fontupper=\itshape,
	colframe=\maudang,colback=yellow!20,coltitle=white,
	sharp corners, breakable, halign title=center,%adjusted title=center, %canh giữa DẠNG
	before skip=2mm,after skip=3mm,
	left=2mm,right=2mm,top=2mm,bottom=2mm,
	boxrule=1pt,
	title={\faFolderOpen\ Dạng~\stepcounter{dang}\thedang.\ #1}
		\addcontentsline{toc}{subsection}{\it\sffamily \faFolderOpen\ Dạng~\thedang.#1}
		\setcounter{subsubsection}{0}
		\setcounter{vd}{0}
		\setcounter{ex}{0}
		\setcounter{bt}{0}
}
%====================
\setlength{\parindent}{0pt} %không thụt đầu dòng
%--Name
\newcounter{deso}
\font\dam=ugqb8v at 13pt
\font\damTT=ugqb8v at 18pt
%================đn notenam
\def\notename{
		\begin{tikzpicture}[remember picture,overlay,>=stealth]
		\checkoddpage\ifoddpage %nếu trang lẻ
		%--tiêu đề phải
		\path ([yshift=-\tren cm+0.5*\topset cm-0.5cm,xshift=-\phai cm-0.5*\leftnote cm+2.5mm]current page.north east) coordinate (DD); 
		%--
		\fill[white] ([yshift=-\tren cm+0.5*\topset cm-5pt,xshift=\trai cm-2pt]current page.north east) rectangle ([yshift=\duoi cm-0.5*\botset cm-12cm,xshift=-\mepphai cm+3mm]current page.north east);
		\node[inner sep =0pt,anchor=north] (thanhta) at ([yshift=-2mm]DD) {
			\includegraphics[width=4.5cm]{logo/logo.jpg}		
		};	
		\else
		%--tiêu đề phải
		\path ([yshift=-\tren cm+0.5*\topset cm-0.5cm,xshift=\phai cm+0.5*\leftnote cm-2.5mm]current page.north west) coordinate (DD); 
		%--
		\fill[white] ([yshift=-\tren cm+0.5*\topset cm-5pt,xshift=\phai cm-2pt]current page.north west) rectangle ([yshift=\duoi cm-0.5*\botset cm-12cm,xshift=\mepphai cm-3mm]current page.north west);
		\node[inner sep =0pt,anchor=north] (thanhta) at ([yshift=-2mm]DD) {
			\includegraphics[width=4.5cm]{logo/logo.jpg}		
		};
		\fi
		%\draw (thanhta.south) node[below=0pt,xscale=0.8]{\small\normalfont\color{\mauname} Sưu tầm \& Biên tập};
		%---note
		\node[inner sep =6pt, text=black,scale=1,anchor=north,fill=\maufoot!3,draw=\maufoot] (bon) at ([yshift=-1cm]thanhta.south) {
			\parbox{\leftnote cm-5mm-12pt}{ \fontsize{10}{15}\selectfont\normalfont
				\vspace*{2pt}
				\chamngon%
			}
		};
		\draw[\maufoot, line width=5pt] (bon.north west)--(bon.north east);
		\draw[\maufoot!50] ([yshift=9pt,line width=0.4pt]bon.north east)--([yshift=9pt]bon.north west)
		node[fill=white,inner sep=2pt,anchor=south west,yshift=-2pt,xshift=-2pt]{\bfseries\color{\maufoot}ĐIỂM:}
		;
		%--note dưới
		\node[inner sep =6pt, text=white,scale=1,anchor=north,fill=\maufoot] (noteduoi) at ([yshift=-0.25cm]bon.south) {
			\parbox{\leftnote cm-5mm-12pt}{ \fontsize{11}{1}\selectfont\bfseries\centering
				QUICK NOTE
			}
		};
		\draw[\maufoot, line width=0.4pt] ([yshift=-2pt]noteduoi.south west)--([yshift=-2pt]noteduoi.south east);
	\end{tikzpicture}
}
%===================note và nonote
%FULL WIDTH
\def\FULLWIDTH{
	\newpage
	\fancyhead[LO,RE]{\headmucluc}
	\def\notename{}
	\newgeometry{top=\tren cm, bottom=\duoi cm, left=\trai cm, right=\phai cm}
}
\def\NOTE{
	\newpage
	\fancyhead[LO,RE]{\myfancyhead}
	\def\notename{
		\begin{tikzpicture}[remember picture,overlay,>=stealth]
			\checkoddpage\ifoddpage %nếu trang lẻ
			%--tiêu đề phải
			\path ([yshift=-\tren cm+0.5*\topset cm-0.5cm,xshift=-\phai cm-0.5*\leftnote cm+2.5mm]current page.north east) coordinate (DD); 
			%--
			\fill[white] ([yshift=-\tren cm+0.5*\topset cm-5pt,xshift=\trai cm-2pt]current page.north east) rectangle ([yshift=\duoi cm-0.5*\botset cm-12cm,xshift=-\mepphai cm+3mm]current page.north east);
			\node[inner sep =0pt,anchor=north] (thanhta) at ([yshift=-2mm]DD) {
				\includegraphics[width=4.5cm]{logo/logo.jpg}		
			};	
			\else
			%--tiêu đề phải
			\path ([yshift=-\tren cm+0.5*\topset cm-0.5cm,xshift=\phai cm+0.5*\leftnote cm-2.5mm]current page.north west) coordinate (DD); 
			%--
			\fill[white] ([yshift=-\tren cm+0.5*\topset cm-5pt,xshift=\phai cm-2pt]current page.north west) rectangle ([yshift=\duoi cm-0.5*\botset cm-12cm,xshift=\mepphai cm-3mm]current page.north west);
			\node[inner sep =0pt,anchor=north] (thanhta) at ([yshift=-2mm]DD) {
				\includegraphics[width=4.5cm]{logo/logo.jpg}		
			};
			\fi
			%\draw (thanhta.south) node[below=0pt,xscale=0.8]{\small\normalfont\color{\mauname} Sưu tầm \& Biên tập};
			%---note
			\node[inner sep =6pt, text=black,scale=1,anchor=north,fill=\maufoot!3,draw=\maufoot] (bon) at ([yshift=-1cm]thanhta.south) {
				\parbox{\leftnote cm-5mm-12pt}{ \fontsize{10}{15}\selectfont\normalfont
					\vspace*{2pt}
					\chamngon%
				}
			};
			\draw[\maufoot, line width=5pt] (bon.north west)--(bon.north east);
			\draw[\maufoot!50] ([yshift=9pt,line width=0.4pt]bon.north east)--([yshift=9pt]bon.north west)
			node[fill=white,inner sep=2pt,anchor=south west,yshift=-2pt,xshift=-2pt]{\bfseries\color{\maufoot}ĐIỂM:}
			;
			%--note dưới
			\node[inner sep =6pt, text=white,scale=1,anchor=north,fill=\maufoot] (noteduoi) at ([yshift=-0.25cm]bon.south) {
				\parbox{\leftnote cm-5mm-12pt}{ \fontsize{11}{1}\selectfont\bfseries\centering
					QUICK NOTE
				}
			};
			\draw[\maufoot, line width=0.4pt] ([yshift=-2pt]noteduoi.south west)--([yshift=-2pt]noteduoi.south east);
		\end{tikzpicture}
	}
	\newgeometry{top=\tren cm, bottom=\duoi cm, left=\trai cm, right=\mepphai cm}
}
%===================đn name
\newcommand{\name}[4]{
%	\NOTE
%	\newpage
	\setcounter{ex}{0}\setcounter{bt}{0}%\setcounter{EX}{0}
	\boldmath\fontfamily{qag}\selectfont\color{\mauname}
\hoten \dotfill {\fontsize{10}{11}\selectfont \ngaylamde}
	\begin{tcolorbox}[boxrule=0.7pt,arc=0mm,breakable,colframe=\mauSO,colback=\mauname!2,before skip=2mm,after skip=2mm]\color{\mauname}
	\begin{center}
		%%---
		{\damTT \MakeUppercase{#1}}\\[1pt]
		{\dam \MakeUppercase{#2 --- Đề} \stepcounter{deso}\thedeso}\\[1pt]
		% {\dam \MakeUppercase{#2}}\\[1pt]
		{\dam\color{\mauSO} \MakeUppercase{#3}}\\[1pt]
		{\fontsize{10}{10}\selectfont \textit{#4}}%\\[-1mm]
	\end{center}
	\end{tcolorbox}
	%%--- Phần note đầu đề
	\notename
\vspace*{0.5cm}
	\addcontentsline{toc}{section}{\hspace*{-4.2cm}\sf Đề \thedeso: #2 --- #3} % đưa MT vào mục lục
}
%--Sang trang 
\BeforeBeginEnvironment{name}{
	\ifnum\the\value{deso}>0
	\newpage
	\fi
}
%%---Đánh số trang
%\AtEndEnvironment{name}{
%	\ifnum\the\value{deso}=1
%	\pagenumbering{arabic}%đánh số trang dạng 1,2,...
%	\fi
%}
%---------
\def\chap#1{
	\begin{center}
		\fontchap\color{\mauCHUONG} #1
	\end{center}
	\addcontentsline{toc}{chapter}{\hspace*{-2.75cm}#1}
}
%---Hiện bảng ĐA
\newcommand{\hienDA}{
	\renewcommand{\indapan}[2]{
		\addcontentsline{toc}{subsection}{\hspace*{-4.2cm}\sf Bảng đáp án} % đưa MT vào mục lục
		%		\begin{center}
		\par\vspace*{5mm}
		\begin{tikzpicture}%
			\draw (0,0)++(0.5*\textwidth,0) node[thick,scale=1,fill=\mauEX!2,draw=\maufoot,minimum width=3.5cm,minimum height=0.1cm,rounded corners=2mm] {\damEX\color{\mauname} BẢNG ĐÁP ÁN};
		\end{tikzpicture}%
		%		\end{center}
		\vspace*{-2mm}
		\inputansbox{##1}{##2}
	}
}
%---Ẩn bảng ĐA
\newcommand{\anDA}{
	\renewcommand{\indapan}[2]{}
}
%---Dòng chấm từng câu
\newcommand{\dongchamEX}[1]{
%	\hideansEX{ex}
	\anLG
	\AfterEndEnvironment{ex}{%
		\foreach \cauEX/\dongEX in {#1}{
			\ifnum\dongEX=0
			\else
			\ifnum\the\value{ex}=\cauEX
			\par\noindent\loigiaiEXS\par
			\dotlineEX{\dongEX}
			\fi
			\fi
		}
	}
}
%---Dòng chấm nhiều câu
\newcommand{\dongchamEXS}[2]{
%	\hideansEX{ex}
	\anLG
	\AfterEndEnvironment{ex}{%
		\foreach \cauEX in {#1}{
			\ifnum#2=0
			\else
			\ifnum\the\value{ex}=\cauEX
			\par\noindent\loigiaiEXS\par
			\dotlineEXS{#2}
			\fi
			\fi
		}
	}
}
%---Dòng chấm từng câu theo đề
\newcommand{\DEdongchamEX}[2]{
%	\hideansEX{ex}
	\anLG
	\AfterEndEnvironment{ex}{%
		\foreach \cauEX/\dongEX in {#2}{
			\ifnum\dongEX=0
			\else
			\ifnum\the\value{deso}=#1
			\ifnum\the\value{ex}=\cauEX
			\par\noindent\loigiaiEXS\par
			\dotlineEX{\dongEX}
			\fi
			\fi
			\fi
		}
	}
}
%---Dòng chấm nhiều câu theo đề
\newcommand{\DEdongchamEXS}[3]{
%	\hideansEX{ex}
	\anLG
	\AfterEndEnvironment{ex}{%
		\foreach \cauEX in {#2}{
			\ifnum#3=0
			\else
			\ifnum\the\value{deso}=#1
			\ifnum\the\value{ex}=\cauEX
			\par\noindent\loigiaiEXS\par
			\dotlineEX{#3}
			\fi
			\fi
			\fi
		}
	}
}
%---Ẩn LG
\newcommand{\anLG}{
	\renewcommand{\loigiai}[1]{	}%
	% \chooseNSA
	\renewcommand{\TrueTF}{\FalseTF}
	\renewcommand{\TrueEX}{\FalseEX}
	\renewcommand{\writekeyTFone}{\gdef\TrueX{}\gdef\FalseX{}}
	\renewcommand{\writekeyTF}{&&}
}
%---Hiện LG
\newcommand{\hienLG}{
	%Xuất hiện chữ Lời giải trong môi trường onlysolution
	\renewcommand{\loigiai}[1]{%
		\begin{onlysolution}%
			##1
		\end{onlysolution}%
	}%
	%---
	\def\loigiaiEXS{}
	%\choiceTF
	\renewcommand{\writekeyTFone}{\gdef\TrueX{}\gdef\FalseX{\tickF}}
	\renewcommand{\writekeyTF}{%
		&\centering\leavevmode\TrueX%
		&\parbox[t]{\linewidth}{\centering\leavevmode\FalseX}%
			\gdef\TrueX{}\gdef\FalseX{\tickF}%
	}
	\def\kindSA{ShowSAKeyColor}
	\showanswers
	% \SAOPTN{kindSA=oly}
	\renewcommand{\dotlineEXS}[1]{}	
}
%=======Đn các phương án
\def\khoanhtrondapan{
	\renewcommand*\circled[1]{\tikz[baseline=(char.base)]{
			\node[shape=circle,draw=\mauDA,inner sep=1pt] (char) {##1};}}
	\renewcommand{\TrueEX}{\stepcounter{dapan}
		{\squareEX{\textbf{\damEX\color{\mauDA}\Alph{dapan}}}} \ignorespaces}
	\renewcommand{\FalseEX}{\stepcounter{dapan}
		{\circled{\textbf{\damEX\color{\mauDA}\Alph{dapan}}}} \ignorespaces}
	%---Chọn đáp án
	\renewcommand{\circEX}[2][fill=\mauTrue!3,draw=\mauTrue]{%
	\tikz[baseline=(char.base)]{\node[shape=circle,inner sep=1pt,##1] (char) {\color{red}##2};}}
}
%----
\def\khongkhoanhtrondapan{
	\renewcommand{\TrueEX}{\stepcounter{dapan}
		{\squareEX{\textbf{\damEX\color{\mauDA}\Alph{dapan}}}} \ignorespaces}
	\renewcommand{\FalseEX}{\stepcounter{dapan}
		{\textbf{\damEX\color{\mauDA}\Alph{dapan}.}} \ignorespaces}
	%---Chọn đáp án
	\renewcommand{\circEX}[2][fill=\mauTrue!3,draw=\mauTrue]{%
	\tikz[baseline=(char.base)]{\node[shape=circle,inner sep=1pt,##1] (char) {\color{red}##2};}}
}
%=============ĐN HIỆN CÂU EX CẦN THIẾT if
\newcommand{\hienEXS}[2]{
	%\foreach \bdem in {#1,...,#2}{%11-30
	\def\biendau{#1}\def\biencuoi{#2}%
	\pgfmathsetmacro{\sodau}{\fpeval{round(\biendau-1,0)}}
	\pgfmathsetmacro{\socuoi}{\fpeval{round(\biencuoi+1,0)}}
	\setcounter{EX}{#1-1}
	\RenewEnviron{ex}{
		\stepcounter{ex}%
		\ifnum\value{ex}<\socuoi
		\ifnum\value{ex}>\sodau
		\par%
		\begin{EX}
			\BODY% 
		\end{EX}
		\fi\fi
	}
	%
	\AtEndEnvironment{name}{\setcounter{EX}{#1-1}}
	%
	\AtEndEnvironment{EX}{
		\ifnum\the\value{numTrue}=1
		\scantokens{\begin{EXsol}A\end{EXsol}}
		\fi
		\ifnum\the\value{numTrue}=2
		\scantokens{\begin{EXsol}B\end{EXsol}}
		\fi
		\ifnum\the\value{numTrue}=3
		\scantokens{\begin{EXsol}C\end{EXsol}}
		\fi
		\ifnum\the\value{numTrue}=4
		\scantokens{\begin{EXsol}D\end{EXsol}}
		\fi
		\setcounter{numTrue}{0}
	}
}
%%%%%%%%%%%%%%%%%%%%%%%

%============================ Khung
\newenvironment{khung}
{\begin{tcolorbox}[
		enhanced,breakable,
		colback=yellow!10,
		colframe=blue,
		boxrule=0.5pt,
		%		drop fuzzy shadow=gray,
		left=5pt,right=5pt,top=5pt,bottom=5pt,
		arc=0mm
		]}
	{\end{tcolorbox}}
%-----------------------------Mục con = subsub
\newcounter{muccon}
\newcommand{\muccon}[1]{%
	\stepcounter{muccon}
	{%\setcounter{bt}{0}\setcounter{vd}{0}\setcounter{ex}{0}
		%\fontsize{13pt}{15pt}\selectfont
		%		\color{violet!70!black}\sffamily
		\bfseries\sffamily\bfseries\hspace*{0mm}\themuccon.\  
		#1}
}
%----------------------------------------------------

% Hộp định nghĩa
\newenvironment{boxdn}
{\begin{tcolorbox}
		[enhanced jigsaw,breakable,pad at break*=1mm,
		colback=cyan!2,
		%standard jigsaw, 
		opacityback=0, %ko nền
		boxrule=0pt,frame hidden, left=0.7cm, right=0pt, bottom=2pt, top=0pt,
		borderline west={1mm}{0.5cm}{cyan},
		overlay={
			\fill[fill=cyan!20,draw=none] ([xshift=0.6cm]interior.north west) rectangle (interior.south east)
			;
		}
		\setcounter{muccon}{0}
		]}%0mm lề trái
	{\end{tcolorbox}}
%===============================================
\theoremstyle{nonumberbreak} % ko đánh số
\theoremheaderfont{\sffamily\bfseries} %tên
\theorembodyfont{\normalfont} %thân
\theoremsymbol{\ensuremath{_\blacksquare}} %Dấu kết thúc là ô vuông đen.
\theoremseparator {:} % Dấu ngăn cách
\newtheorem{myphantich}{\color{violet}%\faServer\ 
	\faFileText\ PHÂN TÍCH}
%===============================================
\newenvironment{phantich}{\begin{boxdn}\begin{myphantich}}{\end{myphantich}\end{boxdn}}
%-------------- Khung (Trong main này ko sd)
\newtcolorbox[auto counter]{khung4}[1]{enhanced, breakable,
	before skip=1mm,after skip=1mm,
	left=1mm,right=1mm,top=2mm,bottom=1mm,
	colframe=myblue,colback=cyan!0,colbacktitle=cyan!6,coltitle=myblue,colupper=black,sharp corners,
	,boxrule=0.4mm,
	coltext=mauE,
	attach boxed title to top center=
	{yshift=-0.1mm-\tcboxedtitleheight/2,yshifttext=2mm-\tcboxedtitleheight/2},
	varwidth boxed title*=-3cm,
	boxed title style={boxrule=0.3mm,
		frame code={ \path[tcb fill frame] ([xshift=-4mm]frame.west)
			-- (frame.north west) -- (frame.north east) -- ([xshift=4mm]frame.east)
			-- (frame.south east) -- (frame.south west) -- cycle; },
		interior code={ \path[tcb fill interior] ([xshift=-2mm]interior.west)
			-- (interior.north west) -- (interior.north east)
			-- ([xshift=2mm]interior.east) -- (interior.south east) -- (interior.south west)
			-- cycle;} 
	},
	fonttitle=\fontsize{10}{0}
	\bfseries,
	fontupper=\fontsize{10}{0},
	title={#1}
}

\newcommand{\boxmini}[1]{
	\vspace*{-2mm}
	\begin{center}
		\begin{tikzpicture}[outline/.style={draw=##1,thick,fill=##1!3},outline/.default=myblue]
			\node [outline,
			sharp corners] at (0,0) {\fontfamily{qag} \selectfont\bfseries\color{\mauEX} #1};
		\end{tikzpicture}
	\end{center}
	\vspace*{0mm}
}

%-----------------------
\newcommand{\inden}[1]{
	{\fontsize{11pt}{9pt}\sffamily \selectfont\bfseries\color{\maudn} #1}
}
\newcommand{\indam}[1]{
	{\fontsize{11.5pt}{9pt}\sffamily \selectfont\bfseries\color{\maudn} #1}
}
\newcommand{\indamm}[1]{
	{\fontsize{11.5pt}{9pt}\sffamily \selectfont\bfseries\color{\maudl} #1}
}
\newcommand{\ind}[1]{
	{\fontsize{11.5pt}{9pt}\sffamily \selectfont\bfseries\color{\maucham} #1}
}
%%%===Các biểu tượng===
\def\iconGN{{\color{magenta}\faPencilSquareO}}
\def\iconNS{{\color{gray}\faStar}}
\def\iconQS{{\color{magenta}\faFolderOpen}}
\def\iconMT{{\color{magenta!80!black}\faSunO}}
\def\iconX{{\color{red}\faClose}}
\def\iconCH{{\color{myblue}\faCheckCircle}}
\def\iconVD{\faCubes}
\def\iconCV{{\color{myblue}\faCubes}}

\newcommand{\dongcham}[1]{
	\def\sod{#1}
	\pgfmathsetmacro{\sodong}{2*\sod -1} 
	\columnsep=10pt
	\vspace*{-3.5mm}
	\begin{multicols}{2}
		\foreach \dotline in{1,...,\sodong}
		{\noindent\color{gray}{\dotfill}\\[1mm]
		}\noindent\color{gray}{\dotfill}\\[-4mm]
	\end{multicols}
}


\def\TNTF{
    {\bfseries Phần II. Trong mỗi ý a), b), c) và d) ở mỗi câu, học sinh chọn đúng hoặc sai.}
}
\def\TN{
    {\bfseries Phần I. Mỗi câu hỏi học sinh chọn một trong bốn phương án A, B, C, D.}
}
\def\TNSA{
    {\bfseries Phần III. Học sinh điền kết quả vào ô trống.}
}
\def\BTTL{
    \begin{center}
        \fcolorbox{black}{white}{{\bfseries BÀI TẬP TỰ LUẬN TRẢ LỜI NGẮN}}
    \end{center}
}
\def\BTTF{
    \begin{center}
        \fcolorbox{black}{white}{{\bfseries BÀI TẬP TRẮC NGHIỆM ĐÚNG SAI}}
    \end{center}
%    \TNTF
}
\def\BTTN{
    \begin{center}
        \fcolorbox{black}{white}{{\bfseries BÀI TẬP TRẮC NGHIỆM 4 PHƯƠNG ÁN}}
    \end{center}
}
\def\TL{
    {\bfseries Phần II. Câu hỏi tự luận.}
}
 %Khai báo cơ bản
\usepackage{tkz-euclide,circuitikz,casio580x}
%%%%%%%%%%%%% ĐIỀU KHIỂN LỜI GIẢI ,DÒNG CHẤM, ĐÁP SỐ
%------------Dòng chấm bằng chiều dài LG (bật)
% \dotlinefull{ex}
%------------Thay Loi giải bằng n dòng kẻ (bật)
% \dotlineans{2}{ex}
%------------Ẩn lời giải
%\hideansEX{ex}
%------------Dòng chấm tùy ý (ko cần \loigiai{})
%---Nhiều câu cùng dòng chấm (tách dụng lên mọi đề)
%\dongchamEXS{1,...,20}{2}
%\dongchamEXS{21,...,40}{5}
%\dongchamEXS{41,...,50}{10}
%---Nhiều câu cùng dòng chấm (tách dụng lên MỘT ĐỀ đc chọn)
%\DEdongchamEXS{3}{1,...,20}{2} %{3} là đề thứ 3
%---Dòng chấm từng câu, tác dụng lên mọi đề
%\dongchamEX{1/3,2/5,3/7} % câu / số dòng chấm của câu đó
%---Dòng chấm từng câu, tác dụng lên 1 đê
%\DEdongchamEX{3}{1/3,2/5,3/7} % câu / số dòng chấm của câu đó, {3} là đề số 3 
\renewcommand{\dongcham}[1]{}
%------------Ẩn đáp số (bật), đáp án
\exitdapso %ẩn đs
%\renewcommand{\indapan}[2]{} %ẩn đáp án
%%%%%%%%%%%%% khung NAME
\def\ngaylamde{Ngày làm đề: ...../...../........} %để {} nếu ko muốn
\def\tenchude{ỨNG DỤNG ĐẠO HÀM}
% \def\tendethi{ }
\def\tentruong{PHedu}
\def\thoigian{Thời gian làm bài: 90 phút, không kể thời gian phát đề}
%%%%%%%%%%%%% Nội dung head & foot
% \def\diachi{ }
\def\diachi{VNPmath - 0962940819}
\def\tenchuyende{\tenchude}
\def\tentacgia{GV.VŨ NGỌC PHÁT}
\def\chamngon{\lq\lq It's not how much time you have, it's how you use it.\rq\rq
}
%%%%%%%%%%%%% Đn lại A.B.C.D
\khoanhtrondapan
% \khongkhoanhtrondapan
%%%%%%%%%%%%%
\renewcommand{\arraystretch}{1}
\newcommand{\viduminhhoa}{\subsubsection{Ví dụ minh hoạ}}
\newcommand{\baitaptl}{\subsubsection{Bài tập tự luận}}
%TF Option
\TFOPTN{kindTF=t,dapanTF=a,boldTF=1,phatbieu=Mệnh đề,viettat=1}
%%===================================================
%=================BẮT ĐẦU TÀI LIỆU===================
\begin{document}
\renewcommand{\chaptername}{Chương}
\pagenumbering{arabic}%đánh số trang dạng 1,2,...
%====================================================
%==================BẮT ĐẦU TÀI LIỆU==================
%\hienEXS{41}{50} %chỉ hiện câu từ 41 đến 50 của đề
%--------Đề bài
% \NOTE \anLG \anDA
%--
% \notename
%%Bài 1. Đơn điệu, Cực trị
% \section{TÍNH ĐƠN ĐIỆU VÀ CỰC TRỊ CỦA HÀM SỐ}
\subsection{LÝ THUYẾT CẦN NHỚ}
\subsubsection{Tính đơn điệu của hàm số}
\begin{enumerate}[\iconMT]
	\item \indam{Định nghĩa:} Cho hàm số $y=f(x)$ xác định trên $K$ ($K$ là khoảng, đoạn hoặc nửa khoảng). \\
\begin{minipage}[b]{6cm}
\begin{khung4}{Ghi nhớ 1}
	Hàm số đồng biến trên $K$ nếu
	$\forall x_1,\,x_2 \in K$, $$ x_1<x_2 \Rightarrow f(x_1)<f(x_2)$$
	\centerline{\begin{tikzpicture}[>=stealth,scale=0.6]
		\draw[->] (-1,0)--(0,0)%
		node[below left]{$O$}--(5,0) node[below]{$x$};
		\draw[->] (0,-1) --(0,3) node[right]{$y$};
		\draw [black,thick, domain=0.2:4, samples=100] %
		plot (\x, {0.1*(\x)^2+1});
		\draw [dashed] (1,0)node[below]{\footnotesize$x_1$} --(1,1.1)--(0,1.1)node[left]{\footnotesize$f(x_1)$};
		\draw [dashed] (3,0)node[below]{\footnotesize$x_2$} --(3,1.9)--(0,1.9)node[left]{\footnotesize$f(x_2)$};
		\draw[fill=blue] (1,1.1) circle(2pt);
		\draw[fill=blue] (3,1.9) circle(2pt);
	\end{tikzpicture}}\\
Trên $K$, đồ thị là một "\textbf{đường đi lên}" khi xét từ trái sang phải.
\end{khung4}
\end{minipage}\hspace{0.5cm}
\begin{minipage}[b]{6cm}
\begin{khung4}{Ghi nhớ 2}
		Hàm số nghịch biến trên $K$ nếu
		$\forall x_1,\,x_2 \in K$, $$ x_1<x_2 \Rightarrow f(x_1)>f(x_2)$$
		\centerline{\begin{tikzpicture}[>=stealth,scale=0.6]
			\draw[->] (-1,0)--(0,0)%
			node[below left]{$O$}--(5,0) node[below]{$x$};
			\draw[->] (0,-1) --(0,3) node[right]{$y$};
			\draw [thick, domain=0.2:4, samples=100] %
			plot (\x, {-0.1*(\x)^2+2.5});
			\draw [dashed] (1,0)node[below]{\footnotesize$x_1$} --(1,2.4)--(0,2.4)node[left]{\footnotesize$f(x_1)$};
			\draw [dashed] (3,0)node[below]{\footnotesize$x_2$} --(3,1.6)--(0,1.6)node[left]{\footnotesize$f(x_2)$};
			\draw[fill=blue] (1,2.4) circle(2pt);
			\draw[fill=blue] (3,1.6) circle(2pt);
		\end{tikzpicture}}\\
	Trên $K$, đồ thị là một "\textbf{đường đi xuống}" khi xét từ trái sang phải.
\end{khung4}
\end{minipage}
	\item \indam{Liên hệ giữa đạo hàm và tính đơn điệu:}
	Cho hàm số $y=f(x)$ có đạo hàm trên khoảng $(a;b)$.
	\begin{boxdn}
	\begin{listEX}[1]
		\item [$\bullet$] Nếu $y'\ge 0$, $\forall x \in (a;b)$ và dấu bằng chỉ xảy ra tại hữu hạn điểm thì hàm số $y=f(x)$ đồng biến trên $(a;b)$.
		\item [$\bullet$] Nếu $y'\le 0$, $\forall x \in (a;b)$ và dấu bằng chỉ xảy ra tại hữu hạn điểm thì hàm số  $y=f(x)$ nghịch biến trên $(a;b)$.
	\end{listEX}
	\end{boxdn}
\end{enumerate}
\subsubsection{Cực trị của hàm số}
\begin{enumerate}[\iconMT]
	\item \indam{Định nghĩa:} Cho hàm số $y=f(x)$ xác định và liên tục trên khoảng $(a ; b)$ ( $a$ có thể là $-\infty, b$ có thể là $+\infty)$ và điểm $x_0 \in(a ; b)$.
	\begin{boxdn}
	\begin{itemize}
		\item [$\bullet$] Nếu tồn tại số $h>0$ sao cho $f(x)<f\left(x_0\right)$ với mọi $x \in\left(x_0-h ; x_0+h\right) \subset(a ; b)$ và $x \neq x_0$ thì ta nói hàm số $f(x)$ đạt cực đại tại $x_0$.
		\item [$\bullet$] Nếu tồn tại số $h>0$ sao cho $f(x)>f\left(x_0\right)$ với mọi $x \in\left(x_0-h ; x_0+h\right) \subset(a ; b)$ và $x \neq x_0$ thì ta nói hàm số $f(x)$ đạt cực tiểu tại $x_0$.
	\end{itemize}
	\end{boxdn}
	\item \indam{Định lý:} Giả sử hàm số $y=f(x)$ liên tục trên khoảng $(a ; b)$ chứa điểm $x_0$ và có đạo hàm trên các khoảng $\left(a ; x_0\right)$ và $\left(x_0 ; b\right)$. Khi đó:
	\begin{boxdn}
	\begin{itemize}
		\item [$\bullet$] Nếu $f^{\prime}(x)<0$ với mọi $x \in\left(a ; x_0\right)$ và $f^{\prime}(x)>0$ với mọi $x \in\left(x_0 ; b\right)$ thì $x_0$ là một điểm cực tiểu của hàm số $f(x)$.
		\item [$\bullet$] Nếu $f^{\prime}(x)>0$ với mọi $x \in\left(a ; x_0\right)$ và $f^{\prime}(x)<0$ với mọi $x \in\left(x_0 ; b\right)$ thì $x_0$ là một điểm cực đại của hàm số $f(x)$.
	\end{itemize}
	\end{boxdn}
	\item \indam{Các tên gọi:}\\
		\begin{tikzpicture}[smooth,samples=300,scale=1.15,>=stealth]
			\draw[->,>=stealth] (-2.5,0)--(2.7,0) node[below]{$x$};
			\draw[->,>=stealth] (0,-1.5)--(0,4) node[right]{$y$};
			\draw (0,0) node[above left]{$O$};
			\draw[blue,domain=-2:2,line width = 1.2pt] plot(\x,{(\x)^3-3*(\x)+1})node[right]{$y=f(x)$};
			\draw[fill=black] (1,0) circle(1pt) (1,-1) circle(2pt) (0,-1) circle(1pt) (-1,0) circle(1pt) (-1,3) circle(2pt) (0,3) circle(1pt);
			\draw[dashed] (1,0)node[above]{\small$x_2$}--(1,-1)--(0,-1)node[left]{\small$y_2$} (-1,0)node[below]{\small$x_1$}--(-1,3)--(0,3)node[right]{\small$y_1$};
			
			\draw[-,dotted] (-0.5,3.7)--(4,3.7)node[right]{$(x_1;y_1)$ là điểm cực đại của đồ thị hàm số;}; 
			\draw[->,dotted] (-0.5,3.7)--(-1,3.15);
			\node[right] at (4.5,3.1) {$\bullet$ $x_1$ là điểm cực đại của hàm số;};
			\node[right] at (4.5,2.5) {$\bullet$ $y_1$ là giá trị cực đại của hàm số.};
			
			\draw[-,dotted] (2,-1)--(2,1)--(4,1)node[right]{$(x_2;y_2)$ là điểm cực tiểu của đồ thị hàm số;}; \draw[->,dotted] (2,-1)--(1.15,-1);
			\node[right] at (4.5,0.4) {$\bullet$ $x_2$ là điểm cực tiểu của hàm số;};
			\node[right] at (4.5,-0.2) {$\bullet$ $y_2$ là giá trị cực tiểu của hàm số.};
		\end{tikzpicture}
\end{enumerate}
\subsection{PHÂN LOẠI VÀ PHƯƠNG PHÁP GIẢI TOÁN}
\begin{dang}{Bài toán tìm khoảng đơn điệu và cực trị của hàm số cho trước}
	\begin{listEX}[1]
		\item [\ding{172}] Tìm tập xác định $\mathscr{D}$ của hàm số $y=f(x)$ .
		\item [\ding{173}] Tính đạo hàm $f'(x)$. Tìm các điểm $x_i \,(i = 1, 2, ..., n)$ thuộc $\mathscr{D}$ mà tại đó đạo hàm bằng $0$ hoặc không xác định.
		\item [\ding{174}] Sắp xếp các điểm $x_i$ theo thứ tự tăng dần, xét dấu $y'$ và lập bảng biến thiên. Từ đây, nêu các khoảng đồng biến, nghịch biến và các điểm cực trị.
	\end{listEX}
\end{dang}
\indamm{Ghi nhớ cách xét dấu:}
\begin{note}
\begin{enumerate}[\iconCH]
		% \item Nếu $$f'(x)=(x-a)(x-b)^2(x-c)^{2n}(x-d)^{2n+1},\,\forall n \in \mathbb{N}*$$
		% thì phương trình $f'(x)=0$ có
		% \begin{itemize}
		% 	\item 	$x=a$ là nghiệm đơn;
		% 	\item  $x=b$ là nghiệm kép;
		% 	\item  $x=c$ là nghiệm bội chẵn;
		% 	\item  $x=d$ là nghiệm bội lẻ.
		% \end{itemize}
		\item Khi xét dấu $f'(x)$ thì $f'(x)$ sẽ không đổi dấu khi qua nghiệm kép (nghiệm bội chẵn) và đổi dấu khi qua nghiệm đơn (nghiệm bội lẻ).
	\end{enumerate}
	% \begin{tikzpicture}[smooth,samples=300,scale=0.8,>=stealth,font=\footnotesize]
	% 	\draw[->] (-3.5,0)--(6,0) node[below]{$x$};
	% 	\draw[->] (0,-1.5)--(0,4) node[left]{$y$};
	% 	\draw (0,0) node[above left]{$O$};
	% 	\draw[blue,line width=0.7pt,domain=-2.15:1.5] plot(\x,{(\x+2)*(\x-1)^2});
	% 	\draw[blue,line width=0.7pt,domain=1.5:4.7] plot(\x,{-1*(\x-1.64)*(\x-4)^2})node[below]{$y=f'(x)$};
	% 	\draw[fill=red] (-2,0)node[above left]{$x_1$} circle(1.5pt) (1,0)node[below]{$x_2$} circle(1.5pt) (4,0)node[above right]{$x_4$} circle(1.5pt) (1.64,0)node[above right]{$x_3$} circle(1.5pt);
	% 	\draw[dashed,<-] (-1.8,-0.2)--(0.5,-2.3)node[below]{\fbox{\scriptsize\text{Nghiệm bội lẻ}}};
	% 	\draw[dashed,->](0.5,-2.3)--(1.58,-0.2);
	% 	\draw[dashed,<-] (1,0.2)--(2,3)node[above]{\fbox{\scriptsize\text{Nghiệm bội chẵn}}};
	% 	\draw[dashed,->](2,3)--(3.9,0.1);
	% 	\end{tikzpicture}
\end{note}
\boxmini{BÀI TẬP TỰ LUẬN}

\begin{vd}
	Tìm các khoảng đơn điệu và các điểm cực trị của hàm số sau
	\begin{tasks}(3)
		\task $ y=-x^3+3x^2-4$;
		\task $ y=x^3-3x^2+1$;
		\task $y=x^3+3x^2+3x+2$;
		\task $y=-2x^4+4x^2$;
		\task $y=x^4+4x^3-1$;
		\task $y=-16x^4+x-1$.
	\end{tasks}
	\loigiai{
	\begin{enumEX}[a)]{1}
		\item Tập xác định: $\mathscr{D}=\mathbb{R}$. \\
		Đạo hàm: $y'=-3x^2+6x$.\\
		Xét $y'=0 \Leftrightarrow -3x^2+6x=0 \Leftrightarrow
		\hoac{
			& x=0 \\
			& x=2 }$
		Bảng biến thiên:\begin{center}
			\begin{tikzpicture}
				\tkzTabInit[nocadre=false,lgt=0.7,espcl=2.1,deltacl=0.6]
				{$x$ /0.6,$y'$ /0.6,$y$ /2}
				{$-\infty$,$0$,$2$,$+\infty$}
				\tkzTabLine{,-,$0$,+,$0$,-,}
				\tkzTabVar{+/$+\infty$, -/$-4$,+/$0$,-/$-\infty$}
			\end{tikzpicture}
		\end{center}
		\item Ta có: $ y'=3x^2-6x\Rightarrow y'=0\Leftrightarrow \hoac{&x=0\\&x=2.} $\\
		Từ bảng biến thiên suy ra hàm số đồng biến trên khoảng $ (-\infty;0) $ và $ (2;+\infty). $
		\begin{center}
			\begin{tikzpicture}
				\tkzTabInit[nocadre=false,lgt=1,espcl=3]
				{$x$ /1,$y'$ /1,$y$ /2}
				{$-\infty$,$0$, $2$,$+\infty$}
				\tkzTabLine{,+,$0$,-,$0$,+, }
				\tkzTabVar{-/ $-\infty$,+/$1 $ ,-/$-3$,+/$+\infty$}
			\end{tikzpicture}
		\end{center}
		\item Hàm số đã cho xác định trên $\mathscr{D}=\mathbb{R}$.\\
		Ta có $y'=3x^2+6x+3$. Cho $y'=0 \Leftrightarrow 3x^2+6x+3=0 \Leftrightarrow x=-1$.\\
		Bảng biến thiên
		\begin{center}
			\begin{tikzpicture}
				\tkzTabInit[lgt=1,espcl=3]
				{$x$/0.7,$y'$/0.7,$y$/2}
				{$-\infty$,$-1$,$+\infty$}
				\tkzTabLine{,+,0,+,}
				\tkzTabVar{-/$-\infty$,R/,+/$+\infty$}
				
			\end{tikzpicture}
		\end{center}
		Vậy hàm số đồng biến trên $\mathbb{R}$.
		\item Tập xác định của hàm số là $ \mathscr{D}=\mathbb{R}$.\\
		Ta có $y'=-8x^3+8x$.
		Cho $y'=0 \Leftrightarrow -8x^3+8x=0 \Leftrightarrow 8x(-x^2+1)=0$\\
		\centerline{$ \Leftrightarrow \left[\begin{aligned}
				&8x=0 \\
				&-x^2+1=0
			\end{aligned}\right. \Leftrightarrow \left[\begin{aligned}
				&x=0 \\
				&x^2=1
			\end{aligned}\right. \Leftrightarrow \left[\begin{aligned}
				&x=0 \\
				&x=\pm 1.
			\end{aligned}\right. $}
		Bảng biến thiên
		\begin{center}
			\begin{tikzpicture}
				\tkzTabInit[lgt=1,espcl=3]
				{$x$/0.7,$y'$/0.7,$y$/2}
				{$-\infty$,$-1$,$0$,$1$,$+\infty$}
				\tkzTabLine{,+,0,-,0,+,0,-,}
				\tkzTabVar{-/$-\infty$,+/ $2$/,-/$0$,+/$2$,-/$-\infty$}
			\end{tikzpicture}
		\end{center}
		Vậy hàm số đồng biến trên mỗi khoảng $(-\infty;-1)$ và $(0;1)$,\\
		\indent{ } hàm số nghịch biến trên mỗi khoảng $(-1;0)$ và $(1;+\infty)$.
		\item Hàm số đã cho xác định trên $\mathscr{D}=\mathbb{R}$.\\
		Ta có $y'=4x^3+12x^2=0=4x^2(x+3)$.\\
		Cho $y'=0 \Leftrightarrow 4x^2(x+3)=0 \Leftrightarrow \left[\begin{aligned}
			&x=0 \\
			&x=-3.
		\end{aligned}\right.$\\
		Bảng biến thiên
		\begin{center}
			\begin{tikzpicture}
				\tkzTabInit[lgt=1,espcl=3]
				{$x$/0.7,$y'$/0.7,$y$/2}
				{$-\infty$,$-3$,$0$,$+\infty$}
				\tkzTabLine{,-,0,+,0,+,}
				\tkzTabVar{+/$+\infty$,-/$-28$ /,R,+/$+\infty$}
			\end{tikzpicture}
		\end{center}
		Vậy hàm số nghịch biến trên khoảng $(-\infty;-3)$ và đồng biến trên khoảng $(-3;+\infty)$.
		\item Ta có $y'=-64x^3+1<0\Leftrightarrow x>\dfrac{1}{4}$ nên hàm số nghịch biến trên khoảng $\left(\dfrac{1}{4};+\infty\right)$.
\end{enumEX}}
\end{vd}

\begin{vd}
	Tìm các khoảng đơn điệu và cực trị của các hàm số sau:
	\begin{tasks}(3)
		\task $y=\dfrac{2x+1}{x+1}$;
		\task $y=\dfrac{3x+1}{x-1}$;
		\task $y=\dfrac{x^2+2x+2}{x+1}$;
		\task $y=x+\dfrac{4}{x}$;
		\task $y=\sqrt{x^2-2x}$;
		\task $y=x-3\sqrt[3]{x^2}$ .
	\end{tasks}
	\loigiai{
		\begin{enumEX}[a)]{1}
			\item Ta có $y'=\dfrac{1}{(x+1)^2} > 0, \forall x \in \mathbb{R} \backslash \{-1\}$.\\
			Vậy hàm số đồng biến trên $(-\infty ;-1)$ và $(-1 ;+\infty)$.\\
			Hàm số không có cực trị.
			\item Ta có $y'=\dfrac{-4}{(x-1)^2}>0,\,\forall x\in\mathbb{R}\setminus\{1\}$.\\
			Do vậy hàm số nghịch biến trên các khoảng  $(-\infty;1)$; $(1;+\infty)$.\\
			Hàm số không có cực trị.
			\item \begin{itemize}
				\item TXĐ: $\mathscr{D}=\mathbb{R}\setminus \left\{-1\right\}$.
				\item $y'=\dfrac{x^2+2x}{(x+1)^2}$, $y'=0\Leftrightarrow \hoac{& x=-2 \\ & x=0.}$\\
				Ta có bảng biến thiên
				\begin{center}
					\begin{center}
						\begin{tikzpicture}
							\tkzTabInit[nocadre=True,lgt=1,espcl=2]
							{$x$ /0.7,$y'$ /0.7,$y$ /2}
							{$-\infty$,$-2$,$-1$,$0$,$+\infty$}
							\tkzTabLine{,+,$0$,-,d,-,$0$,+,}
							\tkzTabVar{-/$-\infty$,+/$-2$,-D+/$-\infty$/$+\infty$,-/$2$,+/$+\infty$}
						\end{tikzpicture}
					\end{center}
				\end{center}
				Hàm số đồng biến trên khoảng $\left( -\infty;-2\right)$ và $\left( 0;+\infty\right)$;  nghịch biến trên $(-2;-1)$ và $(-1;0)$.\\
				Hàm số đạt cực đại tại $x=-2$, giá trị cực đại $y=-2$\\
				Hàm số đạt cực tiểu tại $x=0$, giá trị cực tiểu $y=2$.\\
			\end{itemize}
			\item Tập xác định $\mathscr{D}=\mathbb{R}\setminus\{0\}$.\\
			Ta có $y'=1-\dfrac{4}{x^2}=\dfrac{x^2-4}{x^2}$, $y'=0\Leftrightarrow x=\pm 2$.\\
			Bảng biến thiên
			\begin{center}
				\begin{tikzpicture}[yscale=.8,xscale=1.5,]
					\begin{scope}[shift={(-.5,.5)}]
						\draw
						(0,0) rectangle +(8,-5)
						(0,-1)--+(0:8) (0,-2)--+(0:8) (1,0)--+(-90:5);
					\end{scope}
					\path
					(0,0) node{$x$}          % <<< dòng 1
					++(0:1) node{$-\infty$}
					++(0:2) node{$-2$}
					++(0:1) node{$0$}
					++(0:1) node{$2$}
					++(0:2) node{$+\infty$}
					(0,-1)   node{$y'$}         % <<< dòng 2
					++(0:2) node{$+$}
					++(0:1) node{$0$}
					++(0:.5) node{$-$}
					++(0:1) node{$-$}
					++(0:.5) node{$0$}
					++(0:1) node{$+$}
					(0,-3)   node{$y$}       % <<< dòng 3
					++(0:1) ++(-90:1)  node (A) {$-\infty$}
					++(0:2) ++(90:2) node (B) {$-4$}
					++(0:1) ++(-90:2) node (C)[left]
					{$-\infty$}
					++(90:2) node (D)[right]{$+\infty$}
					++(0:1) ++(-90:2) node (E) {$4$}
					++(0:2) ++(90:2) node (F) {$+\infty$};
					\draw[-stealth] (A)--(B);
					\draw[-stealth] (B)--(C);
					\draw[-stealth] (D)--(E);
					\draw[-stealth] (E)--(F);
					\draw[double] (4,-.5)--(4,-4.5);
				\end{tikzpicture}
			\end{center}
			Hàm số đồng biến trên khoảng $\left( -\infty;-4\right)$ và $\left( 2;+\infty\right)$; nghịch biến trên các khoảng $(-2;0)$ và $(0;2)$.\\
			Hàm số đạt cực đại tại $x=-2$, giá trị cực đại $y=-4$\\
			Hàm số đạt cực tiểu tại $x=2$, giá trị cực tiểu $y=4$\\
			\item Tập xác định: $\mathscr{D}=(-\infty;0]\cup [2;+\infty)$.\\
			Ta có $y'=\dfrac{x-1}{\sqrt{x^2-2x}},\forall x\in (-\infty;0)\cup (2;+\infty)$.\\
			$y'=0 \Leftrightarrow \dfrac{x-1}{\sqrt{x^2-2x}}=0 \Rightarrow x-1=0 \Leftrightarrow x=1 \notin \mathscr{D}$.\\
			Bảng biến thiên:
			\begin{center}
				\begin{tikzpicture}
					\tkzTabInit[lgt=1,espcl=3]
					{$x$/0.7,$y'$/0.7,$y$/2}
					{$-\infty$,$0$,$2$,$+\infty$}
					\tkzTabLine{,-,d,h,d,+,}
					\tkzTabVar{+/$+\infty$,-H/$0$/,-/$0$,+/$+\infty$}
				\end{tikzpicture}
			\end{center}
			Vậy hàm số nghịch biến trên khoảng $(-\infty;0)$ và đồng biến trên khoảng $(2;+\infty)$.\\
			Hàm số không có cực trị.
			\item Tập xác định: $\mathscr{D}=\mathbb{R}$.\\
			Đạo hàm $y'=1-\dfrac{2}{\sqrt[3]{x}}$, xác định với mọi $x\neq 0$.\\
			$y'=0\Leftrightarrow \sqrt[3]{x}=2\Leftrightarrow x=8$.\\
			Đạo hàm không xác định tại $x=0$.\\
			Bảng biến thiên
			\begin{center}
				\begin{tikzpicture}
					\tkzTabInit[nocadre,lgt=1,espcl=2]{$x$/0.7,$y'$/0.7,$y$/2}{$-\infty$,$0$,$8$,$+\infty$}%
					\tkzTabLine{,+,d,-,z,+,}
					\tkzTabVar{-/$-\infty$ , +/$0$,-/$-4$, +/$+\infty$}%
				\end{tikzpicture}
			\end{center}
	\end{enumEX}}
\end{vd}

\begin{vd}
	Thể tích $V$ (đơn vị: centimét khối) của $1 \mathrm{~kg}$ nước tại nhiệt độ $T\,\left(0^{\circ} \mathrm{C} \leq T \leq 30^{\circ} \mathrm{C}\right)$ được tính bởi công thức	$$	V(T)=999,87-0,06426 T+0,0085043 T^2-0,0000679 T^3$$
	 Hỏi thể tích $V(T), \,0^{\circ} \mathrm{C} \leq T \leq 30^{\circ} \mathrm{C}$, giảm trong khoảng nhiệt độ nào?
	\loigiai{
		Xét hàm số  $V(T)=999{,}87-0{,}06426T+0{,}0085043T^2-0{,}0000679T^3$, với $T\in [0;30]$.\\
	Ta có $V'(T)=-0{,}0002037T^2+0{,}0170086T-0{,}06426$.\\
	$V'(T)=0\Leftrightarrow T=3{,}966514624=T_1$ hoặc $T=79{,}53176716\not\in [0;30]$.\\
	Bảng biến thiên của hàm số $V(T)$ như sau
	\begin{center}
		\begin{tikzpicture}[font=\footnotesize,thick,>=stealth]
			\tikzset{double style/.append style={double distance=1.5pt}}
			\tkzTabInit[nocadre=false,lgt=1.2,espcl=2.5,deltacl=0.6,lw=.75pt,color,colorL=green!50,colorV=green!50]
			{$T$ /0.7, $V'(T)$ /0.8, $V(T)$ /2}
			{$0$,$T_1$,$30$}
			\tkzTabLine{ ,-,$0$,+, }
			\tkzTabVar{+/$V(0)$,-/$V(T_1)$,+/$V(30)$}
		\end{tikzpicture}
	\end{center}
	Từ bảng biến thiên suy ra, thể tích $V(T), 0^{\circ}\mathrm{C}\leq T \leq 30^{\circ}\mathrm{C}$, giảm trong khoảng nhiệt độ từ $0^\circ$C đến $3{,}966514624^\circ$C.}
\end{vd}

\boxmini{BÀI TẬP TRẮC NGHIỆM}
\ind{PHẦN I.} \inden{Câu trắc nghiệm nhiều phương án lựa chọn. Mỗi câu hỏi học sinh chỉ chọn một phương án.}\\
\setcounter{ex}{0}
\Opensolutionfile{ans}[ans/2D1-B1-d1-1]

\begin{ex}%[KSCL, Sở GD \& ĐT Hà Nam, 2018]%[Lê Quốc Hiệp, dự án 12EX10-18]%[2D1B1-2]%
	\immini
	{Cho hàm số $y=f(x)$ có đồ thị như hình vẽ bên. Hàm số $y=f(x)$ nghịch biến trên khoảng nào dưới đây?
		\haicot
		{$(\sqrt{2};+\infty)$}
		{$(-2;2)$}
		{$(-\infty;0)$}
		{\True $(0;\sqrt{2})$}
	}
	{\begin{tikzpicture}[line cap=round,line join=round,x=1.0cm,y=1.0cm,>=stealth,scale=0.7]
			\draw[->,color=black,smooth,samples=100] (-2.5,0.) -- (2.5,0.) node[below] {\footnotesize $x$};
			\draw[->,color=black,smooth,samples=100] (0.,-2.5) -- (0.,3) node[left] {\footnotesize $y$};
			\draw plot[smooth,tension=.7] coordinates {(-2,3) (-1.41,-2)  (0,2) (1.41,-2) (2,3)};
			\draw[fill=black] (0,0) circle [radius=1pt] node[above left] {\footnotesize $O$};
			\fill (-1.41,0) node[shift={(90:2ex)}]{\footnotesize $-\sqrt{2}$} circle(1pt);
			\fill (1.41,0) node[shift={(90:2ex)}]{\footnotesize $\sqrt{2}$} circle(1pt);
			\fill (0,-2) node[shift={(-45:1.5ex)}]{\footnotesize $-2$} circle(1pt);
			\fill (0,2) node[shift={(45:1.5ex)}]{\footnotesize $2$} circle(1pt);
			\draw[dashed] (-1.41,0)|-(0,-2)-|(1.41,0);
	\end{tikzpicture}}
	\loigiai
	{
		Dựa vào đồ thị, ta thấy trên khoảng $(0;\sqrt{2})$ đồ thị đi xuống nên hàm số $y=f(x)$ nghịch biến trên khoảng đó.
	}
\end{ex}

\begin{ex}
	\immini{Cho hàm số $y=f(x)$ có đồ thị như hình vẽ bên. Mệnh đề nào sau đây là mệnh đề \textbf{sai}?
		\choice
		{Hàm số đạt cực đại tại $x=0$}
		{Hàm số có giá trị cực tiểu bằng $-2$}
		{\True Hàm số đồng biến trên $(-\infty; 2)$}
		{Hàm số nghịch biến trên $(0; 2)$}
	}
	{
		
		\begin{tikzpicture}[smooth,samples=300,scale=0.7,>=stealth]
			\draw[->] (-2,0)--(3.7,0) node[below]{$x$};
			\draw[->] (0,-2.5)--(0,2.5) node[right]{$y$};
			\draw (0,0) node[below right]{$O$};
			\draw[smooth,samples=100,domain=-1:3]
			plot(\x,{(\x)^3-3*(\x)^2+2});
			\draw[fill=black] (2,0) circle(1.5pt) (0,2) circle(1.5pt) (0,-2) circle(1.5pt);
			\draw[dashed] (2,0)node[above]{\small$2$}--(2,-2)--(0,-2)node[left]{\small$-2$} (0,2.1)node[left]{\small$2$};
		\end{tikzpicture}
	}
	
	\loigiai{
	}
	
\end{ex}

\begin{ex}
	\immini{
		Hàm số $y=f(x)$ có đồ thị là đường cong trong hình vẽ bên. Hàm số $y=f(x)$ đạt cực tiểu tại điểm nào dưới đây?
		\haicot
		{$x=2$}
		{\True $x=0$}
		{$x=-2$}
		{$x=4$}
	}{
		\begin{tikzpicture}[xscale=.7,yscale=.6, font=\footnotesize, line join=round, line cap=round, >=stealth]
			\draw[->] (-2.7,0)--(3,0) node[below]{$x$};
			\draw[->] (0,-1.25)--(0,5) node[left]{$y$};
			\draw[dashed] (-2^.5,0)--(-2^.5,4)--(2^.5,4)--(2^.5,0);
			\draw[domain=-2.05:2.05] plot(\x,{-(\x)^2*((\x)^2-4)});
			\path
			(0,0) node[below right]{$O$}
			(2,0) node[above right]{$2$}
			(-2,0) node[above left]{$-2$}
			(0,4) node[below right]{$4$}
			(-2^.5,0) node[below]{$-\sqrt{2}$}
			(2^.5,0) node[below]{$\sqrt{2}$};
		\end{tikzpicture}
	}
	\loigiai{
		Dựa vào đồ thị hàm số ta thấy hàm số đạt cực tiểu tại $x=0$.}
\end{ex}

\begin{ex}
	\immini{Cho hàm số $y=f(x)$ có bảng biến thiên như hình bên. Mệnh đề nào sau đây là mệnh đề đúng?
	\choice
	{Hàm số đồng biến trên khoảng $(-\infty;3)$}
	{Hàm số nghịch biến trên khoảng $(-2;+\infty)$}
	{Hàm số đạt cực đại tại $x=3$}
	{\True Hàm số đạt cực tiểu tại $x=2$}}{
	\begin{tikzpicture}
	\tkzTabInit[lgt=1.2,espcl=1.8,nocadre=True]
	{$x$/0.6,$f'(x)$/0.6,$f(x)$/2}{$-\infty$,$-2$,$2$,$+\infty$}
	\tkzTabLine{,+,0,-,0,+,}
	\tkzTabVar{-/$-\infty$,+/$3$,-/$0$,+/$+\infty$}
\end{tikzpicture}}
	\loigiai{
	}
	
\end{ex}

\begin{ex}
	Cho hàm số $y=f(x)$ có bảng biến thiên bên dưới
	\begin{center}
		\begin{tikzpicture}
			\tikzset{double style/.append style = {draw=\tkzTabDefaultWritingColor,double=\tkzTabDefaultBackgroundColor,double distance=2pt}}
			\tkzTabInit[lgt=1.2,espcl=2,nocadre=True]
			{$x$ /.7, $f’(x)$ /.7,$f(x)$ /2}
			{$-\infty$ , $-2$, $0$ ,$2$ ,$+\infty$}
			\tkzTabLine{ ,+,$0$,-,d,-,$0$,+, }
			\tkzTabVar{ -/ $-\infty$,+/ $-4$/,-D+/$- \infty$ /$+\infty$,-/ $4$,+ /$+\infty$}
		\end{tikzpicture}
	\end{center}
Khẳng định nào sau đây là khẳng định \textbf{sai}?
	\choice
	{Hàm số có hai điểm cực trị}
	{Tọa độ điểm cực đại của đồ thị hàm số là $(-2;-4)$}
	{\True Hàm số nghịch biến trên khoảng $(-2;2)$}
	{Hàm số đồng biến trên khoảng $(3;+\infty)$}
	\loigiai{
	}
\end{ex}


\begin{ex}
	Cho hàm số $y= - \dfrac{1}{3} x^3 - x -3 $. Mệnh đề nào dưới đây đúng?
	\choice
	{Hàm số đồng biến trên $(-\infty; 1)$ và trên $(1; +\infty)$}
	{\True Hàm số nghịch biến trên $\mathbb{R}$}
	{Hàm số đồng biến trên $(-1;1)$}
	{Hàm số đồng biến trên $\mathbb{R}$}
	\loigiai{
		Tập xác định $\mathscr D = \mathbb{R}$.\\
		$y'=-x^2 -1<0 $ với mọi $x$.\\
		Suy ra  hàm số đã cho nghịch biến trên $\mathbb{R}$.}
\end{ex} 


\begin{ex}
	Gọi $x_1$ là điểm cực đại $x_2$ là điểm cực tiểu của hàm số $y=-x^3+3x+2$. Tính $x_1+2x_2$.
	\choice{$2$}
	{$1$}
	{\True $-1$}
	{$0$}
	\loigiai{
		Ta có $y'=-3x^2+3$, $y'=0\Leftrightarrow x=\pm 1$.\\
		Vì $y'$ đổi dấu từ âm sang dương khi qua $x=-1$ và đổi dấu từ dương sang âm khi qua $x=1$ nên $x_2=-1$ là điểm cực tiểu và $x_1=1$ là điểm cực đại của hàm số. Do đó $x_1+2x_2=1-2=-1$.
	}
\end{ex} 


\begin{ex}
	Khoảng cách giữa hai điểm cực trị của đồ thị hàm số $y=x^3-3x^2+4$ bằng
	\choice
	{\True $2\sqrt{5}$}
	{$2\sqrt{2}$}
	{$2$}
	{$ 4 $}
	\loigiai{
		Ta có $y'=3x^2-6x$, $ y'=0\Rightarrow \hoac{&x=0\Rightarrow y=4\\&x=2\Rightarrow y=0.} $\\
		Suy ra hai điểm cực trị của đồ thị hàm số là $A(0;4),B(2;0)$.\\
		Do đó $AB=\sqrt{2^2+(-4)^2}=2\sqrt{5}$.
	}
\end{ex} 

\begin{ex}%[2D1B1]
	Hàm số $y=x^4-2x^2+1$ đồng biến trên khoảng nào dưới đây?
	\choice
	{\True $(-1;0)$}
	{$(-1;+ \infty)$}
	{$(-3;8)$}
	{$(- \infty ; -1)$}
	\loigiai
	{
		$y'= 4x^3-4x$ $\Rightarrow y'=0 \Leftrightarrow 4x^3-4x=0$ $\Leftrightarrow \hoac{x&= -1 \\ x&=0 \\ x &= 1}$\\
		Bảng xét dấu
		\begin{center}
			\begin{tikzpicture}
				\tkzTabInit[nocadre=false, lgt=1, espcl=2.5]{$x$ /1,$y$ /1}{$-\infty$,$-1$,$0$,$1$,$+\infty$}
				\tkzTabLine{,-,$0$,+,$0$,-,$0$,+}
			\end{tikzpicture}
		\end{center}
	}
	
\end{ex} 

\begin{ex}%[2HK1-13-ChuyenLeQuyDon-QuangTri]%[2D1B2-1]%
	Cho hàm số $ y = - \dfrac{1}{4}x^4 + \dfrac{1}{2}x^2 - 3 $. Khẳng định nào sau đây là khẳng định đúng?
	\choice
	{Hàm số đạt cực tiểu tại $ x = -3 $}
	{ \True Hàm số đạt cực tiểu tại $ x = 0 $}
	{Hàm số đạt cực đại tại $ x = 0 $}
	{Hàm số đạt cực tiểu tại $ x = -1 $}
	\loigiai{
		Ta có $ y' = - x^3 + x = - x (x^2 - 1) $.
		Ta có bảng biến thiên như hình bên
		\begin{center}
			\begin{tikzpicture}[scale=1]
				\tkzTabInit[lgt=1.5,espcl=2.5]{$x$  /1,$y'$  /1,$y$ /2}
				{$-\infty$,$ -1 $,$ 0 $,$ 1 $,$+\infty$}%
				\tkzTabLine{,+,z,-,z,+,z,-,}
				\tkzTabVar{-/$ -\infty $,+/   $\dfrac{-11}{4}$ /,-/ $-3$,+/$ \dfrac{-11}{4} $,-/$ -\infty $}
				%\tkzTabIma{1}{3}{2}{$ 0 $}
			\end{tikzpicture}
		\end{center}
	}
\end{ex} 

\begin{ex}
	Cho hàm số $y=\dfrac{3x-1}{x-2}$. Mệnh đề nào dưới đây là đúng?
	\choice
	{Hàm số nghịch biến trên $\mathbb{R}$}
	{Hàm số đồng biến trên các khoảng $(-\infty;2)$ và $(2;+\infty)$}
	{\True Hàm số nghịch biến trên các khoảng $(-\infty;2)$ và $(2;+\infty)$}
	{Hàm số đồng biến trên $\mathbb{R}\setminus\{2\}$}
	\loigiai{Tập xác định là $\mathscr{D}=\mathbb{R}\setminus\{2\}$.\\
		Có $y'=\dfrac{-5}{(x-2)^2}<0$, $\forall x\in\mathscr{D}$ nên hàm số nghịch biến trên các khoảng $(-\infty;2)$ và $(2;+\infty)$.}
	
\end{ex} 

\begin{ex}
	Cho hàm số $y=\dfrac{x-2}{x+3}$. Mệnh đề nào dưới đây đúng?
	\choice
	{Hàm số nghịch biến trên khoảng $(-\infty;-3)\cup (-3;+\infty) $}
	{\True Hàm số đồng biến trên khoảng $(-\infty;-3) $ và $(-3;+\infty)$}
	{Hàm số nghịch biến trên khoảng $(-\infty;-3)$ và $(-3;+\infty)$}
	{Hàm số đồng biến trên khoảng $(-\infty;-3)\cup (-3;+\infty) $}
	\loigiai{
		Tập xác định $\mathscr{D}=\mathbb{R}\setminus \{-3\}$. Ta có $y'=\dfrac{5}{(x+3)^2}>0$, $\forall x\in\mathscr{D}$.\\ Suy ra hàm số đồng biến trên khoảng $(-\infty;-3)$ và $(-3;+\infty)$.
	}
\end{ex} 

\begin{ex}
	Gọi $y_{\text{CĐ}},\,y_{\text{CT}}$ lần lượt là giá trị cực đại và giá trị cực tiểu của hàm số $y=\dfrac{x^2+3x+3}{x+2}$. Giá trị của biểu thức $y_{\text{CĐ}}^2-2y_{\text{CT}}^2$ bằng
	\choice
	{$8$}
	{\True $7$}
	{$9$}
	{$6$}
	\loigiai{
		Ta có $y'=\dfrac{x^2+4x+3}{(x+2)^2}$; $y'=0 \Leftrightarrow \left[\begin{aligned}
			&x=-1 \\
			&x=-3
		\end{aligned}\right. $. \\
		Bảng biến thiên
		\begin{center}
			\begin{tikzpicture}
				\tkzTab
				[lgt=1,espcl=2] % tùy chọn
				{$x$/0.7, $y'$/0.7, $y$/2} % cột đầu tiên
				{$-\infty$, $-3$, $-2$, $-1$, $+\infty$} % hàng 1 cột 2
				{,+,0,-,d,-,0,+,} % hàng 2 cột 2
				{-/ $-\infty$, +/ $-3$, -D+/ $-\infty$ / $+\infty$, -/ $1$, +/ $+\infty$} % hàng 3 cột 2
			\end{tikzpicture}
		\end{center}
		Từ bảng biến thiên ta tìm được $y_{\text{CĐ}}=-3;\,y_{\text{CT}}=1$ $ \Rightarrow $ $y_{\text{CĐ}}^2-2y_{\text{CT}}^2$ $=9-2=7$.}
\end{ex} 

\begin{ex}
	Tìm điểm cực tiểu của hàm số $f(x)=(x-3)\mathrm{e}^x$.
	\choice
	{$x=3$}
	{$x=0$}
	{\True $x=2$}
	{$x=1$}
	\loigiai{
		\begin{itemize}
			\item Ta có $f'(x)=\mathrm{e}^x(x-2)$, $f''(x)=\mathrm{e}^x(x-1)$.
			\item $f'(x)=0\Rightarrow x=2$ và $f''(2)=\mathrm{e}^2>0$.
		\end{itemize}
		Vậy hàm số đã cho đạt cực tiểu tại $x=2$.}
\end{ex} 

\begin{ex}
	Cho hàm số $y=x^2+4\ln(3-x)$. Tìm giá trị cực đai $y_\text{CĐ}$ của hàm số đã cho.
	\choice
	{$y_\text{CĐ}=2$}
	{\True $y_\text{CĐ}=4$}
	{$y_\text{CĐ}=1+4\ln2$}
	{$y_\text{CĐ}=1$}
	\loigiai{
		Tập xác định $\mathscr{D}=(-\infty;3)$.\\
		Đạo hàm $y'=2x-\dfrac{4}{3-x}=\dfrac{-2x^2+6x-4}{3-x}$.\\
		$y'=0\Leftrightarrow -2x^2+6x-4=0\Leftrightarrow \hoac{&x=1\\&x=2}$.\\
		Bảng biến thiên
		\begin{center}
			\begin{tikzpicture}[>=stealth]
				\tkzTabInit[nocadre=false,lgt=1,espcl=2,deltacl=0.5]{$x$/.7,$y'$/.7,$y$/2}
				{$-\infty$,$1$,$2$,$3$}
				\tkzTabLine{,-,0,+,0,-,d}
				\tkzTabVar{+/$+\infty$,-/$1+4\ln 2$,+/$4$,-D/$-\infty$}
			\end{tikzpicture}
		\end{center}
		Hàm số đạt cực đại tại $x=2$, $y_\text{CĐ}=4$.
	}
\end{ex} 


\begin{ex}%[2D1K2]
	Cho hàm số $y = f(x)$ xác định trên $\mathbb{R}$ và có đạo hàm $y' = f'(x) = 3x^3 - 3x^2$. Mệnh đề nào sau đây \textbf{sai}?
	\choice
	{Trên khoảng $(1;+\infty)$ hàm số đồng biến}
	{Trên khoảng $(-1;1)$ hàm số nghịch biến}
	{\True Đồ thị hàm số có hai điểm cực trị}
	{Đồ thị hàm số có một điểm cực tiểu}
	\loigiai
	{
		Ta có: $y' = 0 \Leftrightarrow 3x^3 - 3x^2 = 0 \Leftrightarrow \hoac{& x = 0 \\& x = 1.}$\\
		Bảng biến thiên:
		\begin{center}
			\begin{tikzpicture}[>=stealth]
				\tkzTabInit[nocadre, lgt=1, espcl=2.5]
				{$x$ /0.7,$y'$ /0.7,$y$ /1.7}
				{$-\infty$,$0$,$1$,$+\infty$}
				\tkzTabLine{,-,$0$,-,$0$,+,}
				\tkzTabVar{+/ $+\infty$, R, -/{\text{CT}}, +/ $+\infty$}
			\end{tikzpicture}
		\end{center}
		Hàm số đồng biến trên khoảng $(1;+\infty)$.\\
		Hàm số nghịch biến trên khoảng $(-\infty;1)$.\\
		Hàm số đạt cực tiểu tại $x = 1$.
	}
\end{ex} 

\begin{ex}%[2D1B2]
	Cho hàm số $ y=f(x) $ liên tục trên $ \mathbb{R} $ và có đạo hàm $ f'(x)=x(x-1)^2(x-2)^3 $. Số điểm cực trị của hàm số $ y=f(x) $ là
	\choice{1}{\True 2}{0}{3}
	\loigiai{	Ta có bảng xét dấu của $ f'(x) $:
		\begin{center}
			\begin{tikzpicture}
				\tkzTabInit[lgt=2,espcl=1.5]%
				{$x$ /1,$f'(x)$ /1}
				{$-\infty$ , $0$ , $1$ , $2$ ,$+\infty$}
				\tkzTabLine{ ,+,0,-,0,-,0,+,}
			\end{tikzpicture}
		\end{center}
		Dựa vào bảng xét dấu ta thấy $ f(x) $ có 2 điểm cực trị.
}\end{ex} 



\begin{ex}%[2D1K2-2]%
	\immini{Cho hàm số bậc bốn $ y=f(x) $. Biết $f'(x) $ có đồ thị như hình bên. Khẳng định nào sau đây là khẳng định đúng?
		\choice
		{Hàm số $f(x)$ đồng biến trên khoảng $(-\infty;0)$}
		{Hàm số $f(x)$ nghịch biến trên khoảng $(-1;1)$}
		{Hàm số $f(x)$ có đúng một điểm cực tiểu}
		{\True Hàm số $f(x)$ có đúng một điểm cực đại}
	}{
		\begin{tikzpicture}[>=stealth,line join=round,line cap=round,font=\footnotesize,scale=0.7,smooth]
			\draw[->] (-3,0)--(7,0)node[below]{$x$};
			\foreach \x in {-2,-1,1,2,3,4}\draw[shift={(\x,0)}] (0,2pt)--(0,-2pt) node[below]{\scriptsize $\x$};
			\draw[->] (0,-2)--(0,3)node[right]{$y$};
			\draw[] plot[smooth,tension=.65] coordinates{(-1.7,-2) (-1,0) (0,.7) (1,0)(2.7,-1.2)(4,0) (5,2.5)}node[right]{$y=f'(x)$};
		\end{tikzpicture}
	}
	\loigiai{
		\immini{Dựa vào đồ thị, ta có bảng biến thiên như hình vẽ. \\
		}{% Cần khai báo \usepackage{tkz-tab}
			\begin{tikzpicture}[scale=.8, font=\footnotesize, line join=round, line cap=round, >=stealth]
				\tkzTabInit[nocadre=false,lgt=1,espcl=2,deltacl=0.5]{$x$/.7 ,$y'$/.7,$y$/2}
				{$-\infty$ , $-1$ , $1$, $ 4 $, $+\infty$}
				\tkzTabLine{ , - , $0$ ,+, $ 0 $, -, $0$ , + , }
				\tkzTabVar{+/$+\infty$ , -/$f(-1)$ ,+/$f(-1)$ ,-/$ f(4) $, +/$+\infty$}
		\end{tikzpicture}}
	}
\end{ex} 

\begin{ex}
	\immini{
		Cho hàm số $y=f(x)$ xác định và liên tục trên $\mathbb{R}$. Biết rằng hàm số $f(x)$ có đạo hàm $f'(x)$ và hàm số $y=f'(x)$ có đồ thị như hình vẽ. Khi đó nhận xét nào sau đây đúng?
		\choice
		{\True Hàm số $f(x)$ không có cực trị}
		{Đồ thị hàm số $f(x)$ có đúng $2$ điểm cực tiểu}
		{Đồ thị hàm số $f(x)$ có đúng một cực đại}
		{Hàm số $f(x)$ có $3$ cực trị}
	}{
		\begin{tikzpicture}[scale=.8,font=\footnotesize, line join=round,line cap=round,>=stealth]
			\draw[->] (-2.5,0)--(2.5,0)node[below]{$x$};
			\draw[->] (0,-1)--(0,3.5)node[left]{$y$};
			\draw[samples=100,domain=-1.7:1.7] plot(\x,{(\x)^4-2*(\x)^2+1});
			\draw[dashed] (-1,0)node[below]{$-1$}circle(1pt) (1,0)node[below]{$1$}circle(1pt) (0,1)node[above right]{$1$}circle(1pt);
		\end{tikzpicture}
	}
	\loigiai{
		Dựa vào đồ thị ta thấy $f'(x)\geq 0$, với mọi $x\in\mathbb{R}$.\\
		Suy ra, hàm số $f(x)$ không có cực trị.
	}
\end{ex} 


\Closesolutionfile{ans}

\ind{PHẦN II.} \inden{Câu trắc nghiệm đúng sai. Trong mỗi ý a), b), c), d) ở mỗi câu, học sinh chọn đúng hoặc sai.}\\
\Opensolutionfile{ans}[ans/2D1-B1-d1-2]

\begin{ex}
	Cho hàm số $y=f(x)$ liên tục trên $\mathbb{R}$ và có bảng xét dấu đạo hàm như hình bên.
	\begin{center}
		\begin{tikzpicture}
			\tikzset{double style/.append style = {draw=\tkzTabDefaultWritingColor,double=\tkzTabDefaultBackgroundColor,double distance=2pt}}
			\tkzTabInit[nocadre=false, lgt=1, espcl=1.2]{$x$ /0.7,$y'$ /1}{$-\infty$,$0$,$1$,$2$,$+\infty$}
			\tkzTabLine{,+,$0$,-,d,+,$0$,+,}
		\end{tikzpicture}
	\end{center}
	% \immini{
		\choiceTF
		{Hàm số đồng biến trên khoảng $(-\infty;1)$}
		{\True Hàm số đồng biến trên khoảng $(1;+\infty)$}
		{Hàm số đạt cực đại tại $x=2$}
		{Hàm số có một điểm cực đại và hai điểm cực tiểu}
	% }{\vspace{0.1cm}
		%}
	\loigiai{
		Ta có bảng biến thiên như sau:
		\begin{center}
			\begin{tikzpicture}
				\tikzset{double style/.append style = {draw=\tkzTabDefaultWritingColor,double=\tkzTabDefaultBackgroundColor,double distance=2pt}}
				\tkzTabInit[lgt=1.1,espcl=2,nocadre=True]
				{$x$ /.7, $y'$ /.7,$y$ /2}
				{$-\infty$ , $0$, $1$ ,$2$ ,$+\infty$}
				\tkzTabLine{ ,+,$0$,-,d,+,$0$,+, }
				\tkzTabVar{ -/,+/ /,-/,R,+/$+\infty$}
			\end{tikzpicture}
		\end{center}
	Từ đây, suy ra:
		\begin{enumerate}[a)]
			\item Hàm số đồng biến trên khoảng $(-\infty;1)$ là khẳng định sai.
			\item Hàm số đồng biến trên khoảng $(1;+\infty)$ là khẳng định đúng.
			\item Hàm số đạt cực đại tại $x=2$ là khẳng định sai.
			\item Hàm số có một điểm cực đại và hai điểm cực tiểu là khẳng định sai.
		\end{enumerate}
	}
	
\end{ex} 

\begin{ex}
	Cho hàm số $y=x^3-3x^2+4$ có đồ thị $(C)$. Gọi $A$, $B$ là hai điểm cực trị của $(C)$.
	\choiceTF
	{\True Tập xác định của hàm số là $\mathbb{R}$}
	{Hàm số đồng biến trên khoảng $(0;2)$}
	{\True PTĐT qua hai điểm cực trị của đồ thị hàm số là $2x+y-4=0$}
	{\True Diện tích của tam giác $OAB$ bằng $4$, với $O$ là gốc tọa độ}
	\loigiai{
		\begin{enumerate}[a)]
			\item Hàm số đa thức nên có tập xác định là $D=\mathbb{R}$.
			\item Ta có 
			\begin{itemize}
				\item [$\bullet$] $y'=3x^2-6x$ và $y'=0 \Leftrightarrow x=0$ hoặc $x=2$.
			\end{itemize}
			Bảng biến thiên:
			\begin{center}
				\begin{tikzpicture}
					\tkzTabInit[lgt=1,espcl=3]
					{$x$ /0.7, $y'$ /0.7, $y$ /2.5}
					{$-\infty$,$0$,$2$,$+\infty$}
					\tkzTabLine{,+,$0$,-,$0$,+,}
					\tkzTabVar{-/$-\infty$,+/$4$,-/$0$,+/$+\infty$}
				\end{tikzpicture}
			\end{center}
		Suy ra hàm nghịch biến trên $(0;2)$.
			\item Tọa độ $A(0;4)$, $B(2;0)$. PTĐT $AB$ là
			$$\dfrac{x-0}{2-0}=\dfrac{y-4}{0-4} \Leftrightarrow 2x+y-4=0$$
			\item Diện tích tam giác vuông $OAB$ là $S_{OAB}=\dfrac{1}{2}OA \cdot OB=4$.
		\end{enumerate}

	}
\end{ex} 

\begin{ex}
	Cho hàm số $y=\dfrac{x^2+2x+2}{x+1}$ có đồ thị $(C)$. Gọi $A$, $B$ lần lượt là điểm cực tiểu và điểm cực đại của $(C)$.
	\choiceTF
	{Tập xác định của hàm số là $\mathbb{R}$}
	{Hàm số nghịch biến trên khoảng $(-2;0)$}
	{Tọa độ điểm $A(-2;-2)$, $B(0;2)$}
	{Khoảng cách giữa hai điểm cực trị là $AB=2\sqrt{5}$}
	\loigiai{
		\begin{enumerate}[a)]
			\item Đặt điều kiện mẫu số khác 0, ta được $x+1 \ne 0 \Leftrightarrow x \ne -1$. Suy ra $\mathscr{D}=\mathbb{R}\setminus \left\{-1\right\}$.
			\item $y'=\dfrac{x^2+2x}{(x+1)^2}\Rightarrow y'=0\Leftrightarrow \hoac{& x=-2 \\ & x=0.}$\\
			Ta có bảng xét dấu của hàm $f'(x)$ như sau
			\begin{center}
					\begin{tikzpicture}
					\tkzTabInit[nocadre=false,lgt=1,espcl=3]
					{$x$ /0.7,$y'$ /0.7,$y$ /2}
					{$-\infty$,$-2$,$-1$,$0$,$+\infty$}
					\tkzTabLine{,+,$0$,-,d,-,$0$,+,}
					\tkzTabVar{-/$-\infty$,+/$-2$,-D+/$-\infty$/$+\infty$,-/$2$,+/$+\infty$}
				\end{tikzpicture}
			\end{center}
			Dựa vào bảng xét dấu ta thấy rằng hàm số $y=f'(x)$ nghịch biến trên $(-2;-1)$ và $(-1;0)$.
			\item Tọa độ điểm $A(0;2)$, $B(-2;-2)$
			\item Độ dài $AB=\sqrt{(-2-0)^2+(-2-2)^2}=2\sqrt{5}$.
		\end{enumerate}

	}
\end{ex} 


\begin{ex}
	Xét một chất điểm chuyển động dọc theo trục $Ox$. Toạ độ của chất điểm tại thời điểm $t$ được xác định bởi hàm số $x(t)=t^3-6t^2+9t$ với $t\geq 0$. Khi đó $x'(t)$ là vận tốc của chất điểm tại thời điểm $t$, kí hiệu $v(t)$; $v'(t)$ là gia tốc chuyển động của chất điểm tại thời điểm $t$, kí hiệu $a(t)$.
	\choiceTF
	{Phương trình hàm vận tốc là $v(t)=3t^2-6t+9$}
	{\True Phương trình hàm gia tốc là $a(t)=6t-12$}
	{Vận tốc của chất điểm tăng khi $t\in (0;1)$ hoặc  $t \in (3;+\infty)$}
	{Vận tốc của chất điểm giảm khi $t\in (1;3)$}
	\loigiai{
		\begin{enumerate}
			\item $v(t)=x'(t)=3t^2-12t+9$
			\item $a(t)=v'(t)=6t-12$.
			\item Xét $v'(t)=6t-12$, $v'(t)=0\Leftrightarrow t=2$\\
			Bảng xét dấu
			\begin{center}
				\begin{tikzpicture}
					\tkzTabInit[nocadre=false,lgt=2,espcl=2.1]
					{$t$ /0.6,$v'(t)$ /0.6}
					{$0$,$2$,$+\infty$}
					\tkzTabLine{,-,$0$,+,}
				\end{tikzpicture}
			\end{center}
			Suy ra vận tốc của chất điểm tăng khi $t\in (2;+\infty) $, giảm khi $t\in (0;2)$.
		\end{enumerate}
	}
\end{ex} 

\Closesolutionfile{ans}
% \begin{dang}{Bài toán tìm m để hàm số đồng biến (nghịch biến) trên khoảng cho trước}
\begin{enumerate}[\iconCV]
\item Xét hàm số bậc ba $y=ax^3+bx^2+cx+d$ có $y'=3ax^2+2bx+c$.
	\begin{listEX}[1]
		\item [\ding{172}] Hàm số đồng biến trên  $\mathbb{R}$ khi và chỉ khi $$y' \ge 0,\,\forall x \in \mathbb{R} \Leftrightarrow \heva{&a>0\\&\Delta_{y'}\le 0}.$$
		\item [\ding{173}] Hàm số nghịch biến trên  $\mathbb{R}$ khi và chỉ khi $$y' \le 0, \,\forall x \in \mathbb{R} \Leftrightarrow \heva{&a<0\\&\Delta_{y'}\le 0}.$$
	\end{listEX}
\textit{Trường hợp hệ số $a$ có chứa tham số, ta kiểm tra thêm trường hợp $a=0$.}
\item Xét hàm phân thức $y=\displaystyle\frac{ax+b}{cx+d}$ có $y'=\dfrac{ad-cb}{(cx+d)^2}$, với $ad-cb \ne 0$ và $c \ne 0$.
\begin{itemize}
	\item [\ding{172}] Hàm số đồng biến trên từng khoảng xác định của nó khi và chỉ khi
	$$y'>0,\, \forall x \ne -\dfrac{d}{c}\Leftrightarrow ad-cb>0.$$
	\item [\ding{173}]  Hàm số nghịch biến trên từng khoảng xác định của nó khi và chỉ khi
	$$y'<0,\, \forall x \ne -\dfrac{d}{c}\Leftrightarrow ad-cb<0.$$
\end{itemize}
\item Xét hàm phân thức $y=\displaystyle\frac{ax^2+bx+c}{dx+e}$ có $y'=\dfrac{adx^2+2aex+be-dc}{(dx+e)^2}$, với $ad \ne 0$.
\begin{itemize}
	\item [\ding{172}] Hàm số đồng biến trên từng khoảng xác định của nó khi và chỉ khi
	$$y'\ge 0,\, \forall x \ne -\dfrac{e}{d}\Leftrightarrow adx^2+2aex+be-dc\ge 0,\, \forall x \ne -\dfrac{e}{d}.$$
	\item [\ding{173}]  Hàm số nghịch biến trên từng khoảng xác định của nó khi và chỉ khi
	$$y'\le 0,\, \forall x \ne -\dfrac{e}{d}\Leftrightarrow adx^2+2aex+be-dc\le 0,\, \forall x \ne -\dfrac{e}{d}.$$
\end{itemize}
\end{enumerate}
\end{dang}
\boxmini{BÀI TẬP TỰ LUẬN}
\setcounter{vd}{0}

\begin{vd}
	Tìm tất cả giá trị của tham số $m$ để hàm số
	\begin{tasks}
		\task $y=x^3+mx^2+2mx+2$ đồng biến trên $(-\infty;+\infty)$.
		\task $y=-\dfrac{1}{3}x^3-mx^2+\left(2m-3\right)x-m+2$ nghịch biến trên $\mathbb{R}$.
		\task $ y=\dfrac{1}{3}x^3-mx^2-(2m+1)x+1$ nghịch biến trên khoảng $(0;5)$.
		\task $y=x^3-3x^2+(5-m)x$ đồng biến trên khoảng $(2;+\infty)$.
	\end{tasks}
\loigiai{
\begin{enumerate}[a)]
	\item Hàm số đã cho có tập xác định $\mathscr{D}=\mathbb{R}$ và $y'=3x^2+2mx+2m$.\\
	Hàm số đã cho đồng biến trên $\mathbb{R}$ khi và chỉ khi
	\[y'\ge0,~\forall x\in\mathbb{R}\Leftrightarrow m^2-6m\le0\Leftrightarrow 0\le m\le6.\]
	\item Tập xác định: $D=\mathbb{R}$. Ta có $y'=-x^2-2mx+2m-3$.\\
	Để hàm số nghịch biến trên $\mathbb{R}$ thì:\\
	$y'\le 0,\forall x\in\mathbb{R} \Leftrightarrow\left\{
	\begin{aligned}
		&a_{y'}<0\\
		&\Delta'\le 0
	\end{aligned}
	\right.
	\Leftrightarrow \left\{
	\begin{aligned}
		&-1<0\\
		&m^2+2m-3\le0
	\end{aligned}
	\right.
	\Leftrightarrow -3\le m\le 1$.
	\item Tập xác định $\mathscr{D}=\mathbb{R}$.\\
	Ta có $y'=x^2-2mx-(2m+1)$, $ y'=0\Leftrightarrow\hoac{&x=-1\\&x=2m+1.}$\\
	Nếu $2m+1\leq-1\Leftrightarrow m\leq-1$ thì $y'\leq 0\Leftrightarrow x\in\left[2m+1;-1\right]$.\\
	Suy ra hàm số không nghịch biến trên khoảng $(0;5)$. \\
	$\Rightarrow m\leq-1$ không thỏa mãn.\\
	Nếu $2m+1>-1\Leftrightarrow m>-1$ thì $y'\leq 0\Leftrightarrow x\in\left[-1;2m+1\right]$.\\
	Để hàm số nghịch biến trên khoảng $(0;5)$ thì ta có $2m+1\geq 5\Leftrightarrow m\geq 2$.
	\item \textbf{\underline{Cách 1:}} Tập xác định $\mathscr{D}=\mathbb{R}$.\\
	Ta có $y'=3x^2-6x+5-m$.\\
	Hàm số $y=x^3-3x^2+(5-m)x$ đồng biến trên khoảng $(2;+\infty)$ khi và chỉ khi
	\allowdisplaybreaks
	\begin{eqnarray*}
		&&y'\ge 0,\,\forall x\in (2;+\infty)\\
		&\Leftrightarrow& 3x^2-6x+5-m\ge 0,\,\forall x\in (2;+\infty)\\
		&\Leftrightarrow& m\le 3x^2-6x+5, \,\forall x\in (2;+\infty)
	\end{eqnarray*}
	Xét hàm $g(x)=3x^2-6x+5$ trên $(2;+\infty)$ có $g'(x)=6x-6$ và $g'(x)=0\Leftrightarrow x=1$.\\
	Bảng biến thiên của $g(x)$
	\begin{center}
		\begin{tikzpicture}
			\tkzTabInit[nocadre=false,lgt=1.5,espcl=2,deltacl=0.5]
			{$x$/0.6,$g'(x)$/0.6,$g(x)$/1.5}
			{$2$,$+\infty$}
			\tkzTabLine{,+,}
			\tkzTabVar{-/$5$,+/$+\infty$}
		\end{tikzpicture}
	\end{center}
	Dựa vào bảng biến thiên của $g(x)$, ta được
	$$m\le 3x^2-6x+5, \,\forall x\in (2;+\infty) \Leftrightarrow m\le 5.$$
	\textbf{\underline{Cách 2:}} Tập xác định $\mathscr{D}=\mathbb{R}$.\\
	Ta có $y'=3x^2-6x+5-m$.\\
	Hàm số $y=x^3-3x^2+(5-m)x$ đồng biến trên khoảng $(2;+\infty)$ khi và chỉ khi
	$$y'\ge 0,\,\forall x\in (2;+\infty) 
	\Leftrightarrow \heva{& y'(2)\ge 0 \\ & -\dfrac{b}{2a} \le 2} 
	\Leftrightarrow \heva{& 5-m\ge 0 \\ & 1 \le 2}
	\Leftrightarrow m \le 5. $$
\end{enumerate}}
\end{vd}

\begin{vd}
	Tìm tất cả giá trị của tham số $m$ để hàm số
	\begin{tasks}
		\task $y=\dfrac{mx+2}{x+1}$ đồng biến trên từng khoảng xác định.
		\task $y=\dfrac{mx-2}{x+m-3}$ nghịch biến trên các khoảng xác định
		\task $y = \dfrac{mx-8}{x-2m}$ đồng biến trên $(3;+\infty )$.
		\task $y=\dfrac{mx+9}{4x+m}$ nghịch biến trên khoảng $(0;4)$.
	\end{tasks}
\loigiai{
\begin{enumerate}[a)]
	\item Từ yêu cầu bài toán, $\forall x \neq -1$ ta xét $y'>0$ $\Leftrightarrow m-2>0 \Leftrightarrow m>2$.
	\item Tập xác định $\mathbb{R}\setminus\{3-m\}$.\\
	$y' = \dfrac{m(m - 3) + 2}{\left( x + m - 3\right)^2} = \dfrac{m^2 - 3m + 2}{\left(x + m - 3\right)^2}$. \\
	Điều kiện để hàm số nghịch biến trên các khoảng xác định của nó là $y' < 0,\,\forall x \ne 3 - m$ hay $m^2 - 3m + 2 < 0 \Leftrightarrow m \in (1;2)$.
	\item Tập xác định: $\mathscr{D} = \mathbb{R} \setminus \{2m\}$.\\
	$y' = \dfrac{-2m^2+8}{(x-2m)^2}$.\\
	Hàm số luôn đơn điệu trên từng khoảng xác định $(-\infty; 2m)$ và $(2m; +\infty)$ khi $-2m^2 + 8 \ne 0$.\\
	Vậy hàm số đồng biến trên $(3;+\infty)$ khi và chỉ khi $-2m^2+8 > 0$ và $(3;+\infty) \subset (2m ;+\infty)$. \\
	Điều này tương đương $\heva{&-2<m<2\\&2m \le 3}$, hay $-2 < m \le \dfrac{3}{2}$.
	\item Tập xác định $\mathscr{D}=\mathbb{R}\setminus\left\{-\dfrac{m}{4}\right\}$.\\
	Ta có $y=\dfrac{mx+9}{4x+m}\Rightarrow y'=\dfrac{m^2-36}{(4x+m)^2}$.\\
	Để hàm số nghịch biến trên khoảng $(0;4)$ thì
	$$\heva{& y'<0 ,\forall x\in(0;4)\\ & -\dfrac{m}{4}\notin (0;4)}\Leftrightarrow\heva{& m^2-36<0 \\ &\hoac{&-\dfrac{m}{4}\geq4\\&-\dfrac{m}{4}\leq 0}}\Leftrightarrow\heva{& -6<m<6 \\ &\hoac{&m\leq-16\\&m\geq 0}}\Leftrightarrow 0\leq m<6.$$
\end{enumerate}}
\end{vd}

\begin{vd}
	Tìm tất cả giá trị của tham số $m$ để hàm số
	\begin{tasks}
		\task $ y = \dfrac{2x^2+3x+m+1}{x+1} $ đồng biến trên các khoảng xác định.
		\task $y=\dfrac{x^2+(m+1)x-1}{2-x}$ ($m$ là tham số) nghịch biến trên mỗi khoảng xác định.
	\end{tasks}
	\loigiai{
		\begin{enumerate}[a)]
			\item Tập xác định: $\mathbb{R}\setminus\{-1\}$.\\
			Ta có $y'=\dfrac{2x^2+4x+2-m}{(x+1)^2}$. Hàm số đồng biến trên các khoảng xác định khi 
			$$2x^2+4x+2-m\ge 0, \forall x\in \mathbb{R} \Leftrightarrow m\le \min\limits{\mathbb{R}\setminus \{-1\} } (2x^2+4x+2) = 0.$$
			\item Tập xác định $\mathscr{D}=\mathbb{R}\backslash\{2\}$.\\
			Đạo hàm: $y'=\dfrac{-x^2+4x+2m+1}{(2-x)^2}=\dfrac{g(x)}{(2-x)^2}$.\\
			Hàm số nghịch biến trên mỗi khoảng xác định của nó khi và chỉ khi $y'\le 0,\forall x\in \mathscr{D}$ (Dấu \lq\lq $=$\rq\rq~ chỉ xảy ra tại hữu hạn điểm thuộc $\mathscr{D}$).\\
			$\Leftrightarrow g(x)=-x^2+4x+2m+1\le 0,$  $\forall x\in \mathbb{R}$\\
			Điều kiện: ${\Delta}'\le 0$ (vì $a=-1<0$) $\Leftrightarrow 4-(-1)\cdot(2m+1)\le 0\Leftrightarrow 2m+5\le 0\Leftrightarrow m\le -\dfrac{5}{2}$.
	\end{enumerate}}
\end{vd}

\boxmini{BÀI TẬP TRẮC NGHIỆM}
\ind{PHẦN I.} \inden{Câu trắc nghiệm nhiều phương án lựa chọn. Học sinh trả lời từ câu 1 đến câu 17. Mỗi câu hỏi học sinh chỉ chọn một phương án.}\\
\setcounter{ex}{0}
\Opensolutionfile{ans}[ans/2D1-B1-d2-1]

\begin{ex}%[Nguyễn Trung Kiên, dự án 12-EX-7-2020]%[2D1B1-3]%
	Tất cả giá trị của $m$ để hàm số $y=\dfrac{x+m}{x-2}$ nghịch biến trên từng khoảng xác định là
	\choice
	{\True $m>-2$}
	{$m<-2$}
	{$m\leq -2$}
	{$m\geq -2$}
	\loigiai
	{Tập xác định $\mathscr{D}=\mathbb{R}\setminus \{2\}$ và $y'=\dfrac{-2-m}{(x-2)^2}$.\\
		Hàm số nghịch biến trên các khoảng $(-\infty;2)$ và $(2;+\infty)$ khi và chỉ khi
		\[y'<0,\, \forall x\neq 2\Leftrightarrow -2-m<0 \Leftrightarrow m>-2.\]}
\end{ex} 

\begin{ex}
	Cho hàm số $y=\dfrac{mx-2}{x+1-m}$. Tìm tất cả giá trị của tham số $m$ để hàm số đồng biến trên từng khoảng xác định.
	\choice
	{$\hoac{& m> 2\\& m< -1}$}
	{\True $-1<m<2$}
	{$-1\le m\le 2$}
	{$\hoac{& m\ge 2\\ &m\le -1}$}
	\loigiai{
		Yêu cầu bài toán $\Leftrightarrow ad-bc>0 \Leftrightarrow m(1-m)+2>0 \Leftrightarrow -1<m<2$.
	}
\end{ex} 

\begin{ex}
	Cho hàm số $ y=\dfrac{x+m}{x+2} $. Tập hợp tất cả các giá trị của $ m $ để hàm số đồng biến trên khoảng $ \left(0;+\infty\right)  $ là
	\choice
	{$ \left[2;+\infty\right) $}
	{$ \left(2;+\infty\right)  $}
	{$ \left(-\infty;2\right ]  $}
	{\True $\left(-\infty;2\right)   $}
	\loigiai{
		Hàm số xác định khi $ x\ne -2. $\\
		Có $ y'=\dfrac{2-m}{\left(x+2\right)^2 }, x\ne -2 $.\\
		Hàm số đồng biến trên $ (0;+\infty) $ khi và chỉ khi $ 2-m>0\Leftrightarrow m<2. $
	}
\end{ex} 

\begin{ex}
	Cho hàm số $f(x)=\dfrac{mx-4}{x-m}$ ( $m$ là tham số thực). Có bao nhiêu giá trị nguyên của $m$ để hàm số đồng biến trên khoảng $\left( 0;+\infty  \right)$?  
	\choice
	{$5$}
	{$4$}
	{$3$}
	{\True  $2$}
	\loigiai{
		Ta có $f'(x)=\dfrac{-m^2+4}{{{\left( x-m \right)}^{2}}}$\\
		Hàm số đồng biến trên khoảng $\left( 0;+\infty  \right)$ $\Leftrightarrow$ $\dfrac{-m^2+4}{\left( x-m \right)^2}>0,\,\, \forall x\in \left( 0;+\infty  \right)$\\
		$\Rightarrow \heva{
			& -m^2+4>0 \\ 
			& x\ne m\ \ \forall x\in \left( 0;+\infty  \right) \\ 
		}\Leftrightarrow \heva{
			& m\in \left( -2;2 \right) \\ 
			& m\in \left( -\infty ;0 \right] \\ 
		}\Leftrightarrow m\in \left( -2;0 \right]$\\
		Vậy có hai giá trị nguyên của $m$ là $-1$ và $0$.      
	}
\end{ex} 

\begin{ex}
	Tìm tất cả các giá trị của $m$ để hàm số $y=\dfrac{mx+4}{x+m}$ nghịch biến trên $(-\infty;1)$.
	\choice
	{$-2<m<2$}
	{$-2<m <-1$}
	{$-2\leq m <-1$}
	{\True $-2<m\leq-1$}
	\loigiai{
		ĐKXĐ: $x\neq-m$.\\
		Hàm số $y=\dfrac{mx+4}{x+m}$ nghịch biến trên $(-\infty;1)$\\$\Leftrightarrow y'=\dfrac{m^2-4}{(x+m)^2}<0$, $\forall x\in(-\infty;1)$
		$ \Leftrightarrow\heva{&m^2-4<0\\&-m\geq 1}\Leftrightarrow\heva{&-2<m<2\\&m\leq-1}\Leftrightarrow-2<m\leq-1 $.}
\end{ex} 

\begin{ex}%[THPT Tĩnh Gia - Thanh Hóa, 2020]%[Bùi Mạnh Tiến, 12EX7]%[2D1B1-3]%
	Số giá trị nguyên của tham số $m$ để hàm số $y=\dfrac{mx+10}{2x+m}$ nghịch biến trên khoảng $(0;2)$ là
	\choice
	{\True $6$}
	{$5$}
	{$4$}
	{$9$}
	\loigiai
	{
		Ta có $y'=\dfrac{m^2-20}{(2x+m)^2}$.\\
		Do đó hàm số $y=\dfrac{mx+10}{2x+m}$ nghịch biến trên $(0;2)$ khi và chỉ khi
		\begin{align*}
			\heva{& m^2-20<0 \\ & -\dfrac{m}{2}\notin (0;2)}\Leftrightarrow \heva{& -2\sqrt{5}<m<2\sqrt{5} \\ & \hoac{& -\dfrac{m}{2}\le 0 \\ & -\dfrac{m}{2}\ge 2}}\Leftrightarrow \hoac{& 0\le m<2\sqrt{5} \\ & -2\sqrt{5}<m\le -4.}
		\end{align*}
		Vì $m\in \mathbb{Z}$ nên $m\in \left\{-4;0;1;2;3;4\right\}$.\\
		Vậy có tất cả $6$ giá trị nguyên của $m$ thỏa mãn yêu cầu bài toán.
	}
\end{ex} 

\begin{ex}
	Có bao nhiêu giá trị nguyên của tham số $m$ để hàm số $y=x^3-2mx^2+\left(m^2+3\right)x$ đồng biến trên $\mathbb{R}$?
	\choice
	{$8$}
	{$6$}
	{\True $7$}
	{$0$}
	\loigiai{
		Hàm số $y=x^3-2mx^2+\left(m^2+3\right)x$ đồng biến trên $\mathbb{R}$
		\begin{eqnarray*}
			&\Leftrightarrow &y'=3x^2-4mx+m^2+3\ge 0, \, \forall x\in \mathbb{R}\\
			&\Leftrightarrow & \Delta'=4m^2-3\left(m^2+3\right)\le 0\\
			&\Leftrightarrow & m^2-9\le 0\Leftrightarrow-3\le m\le 3.
		\end{eqnarray*}
		Do $m$ là số nguyên nên $m\in \left\lbrace -3;-2;-1;0;1;2;3\right\rbrace $.\\
		Vậy có $7$ giá trị nguyên của tham số $m$.
	}
\end{ex} 

\begin{ex}
	Cho hàm số $y=-x^3-mx^2+(4m+9)x+5$. Có bao nhiêu giá trị nguyên của $m$ để hàm số nghịch biến trên $\mathbb{R}$?
	\choice
	{\True $7$}
	{$4$}
	{$5$}
	{$6$}
	\loigiai{
		Ta có $y'=-3x^2-2mx+(4m+9)$. Hàm số đã cho nghịch biến trên $\mathbb{R}$ khi và chỉ khi
		\[ \Delta'\le 0 \Leftrightarrow m^2+12m+27\le 0 \Leftrightarrow -9\le m\le -3. \]
		Vậy có tất cả $7$ giá trị nguyên của $m$ thỏa mãn bài toán.
	}
\end{ex} 

\begin{ex}
	Cho hàm số $y=(m-1)x^3 + (m-1)x^2 -2x+5$ với $m$ là tham số. Có bao nhiêu giá trị nguyên của $m$ để hàm số nghịch biến trên khoảng $(-\infty;+\infty)$?
	\choice
	{$5$}
	{\True $7$}
	{$8$}
	{$6$}
	\loigiai{
		\textbf{Trường hợp 1:} $m-1=0 \Leftrightarrow m=1$ khi đó $y=-2x+5$ nghịch biến trên $\mathbb{R}$. Do đó nhận $m=1$.\\
		\textbf{Trường hợp 2:} $m-1\ne 0 \Leftrightarrow m\ne 1$.\\
		Ta có $y'=3(m-1)x^2+2(m-1)x-2$. \\
		Hàm số nghịch biến trên $(-\infty;+\infty) $ $\Leftrightarrow y' \le 0 $, $\forall x\in (-\infty;+\infty)$
		$$\Leftrightarrow \heva{& 3(m-1)<0 \\ & (m-1)^2-3(m-1)\cdot (-2) \le 0} \Leftrightarrow \heva{& m<1 \\ & -5 \le m \le 1} \Leftrightarrow -5 \le m <1.$$.\\
		Do $m \in \mathbb{Z} \Rightarrow m\in \{-5;-4;-3;-2;-1;0\}$.\\
		Vậy cả $2$ trường hợp thì ta có tất cả $7$ giá trị $m$ thỏa yêu cầu bài toán là $\{-5;-4;-3;-2;-1;0;1\}$.
	}
\end{ex} 

\begin{ex}
	Tìm tất cả các giá trị thực của tham số $m$ để hàm số $y=x^3-3mx^2-9m^2x$ nghịch biến trên khoảng $(0;1)$.
	\choice
	{$-1<m<\dfrac{1}{3}$}
	{$m<-1$}
	{$m>\dfrac{1}{3}$}
	{\True $m\ge \dfrac{1}{3}$ hoặc $m\le -1$}
	\loigiai{
		Đặt $f(x)=y'=3x^2-6mx -9m^2$.\\
		Vì $y'$ là hàm số bậc hai với hệ số $a=3>0$ nên để hàm số nghịch biến trên $(0;1)$ thì phương trình $y'=0$ có hai nghiệm phân biệt $x_1, x_2$ thỏa mãn $x_1\le 0<1 \le x_2$ $$\Leftrightarrow \heva{&af(0)\le 0\\&af(1) \le0} \Leftrightarrow \heva{&-9m^2\le 0\\&3-6m-9m^2 \le 0} \Leftrightarrow \hoac{&x\le -1\\&x\ge \dfrac{1}{3}.}$$
	}
\end{ex} 

\begin{ex}
	Có bao nhiêu giá trị nguyên của tham số $ m$ thuộc khoảng $( -2019;2020 )$ để hàm số $ y=2x^3-3( 2m+1 )x^2+6m(m+1)x+2019$ đồng biến trên khoảng $(2;+\infty )$?
	\choice
	{\True $2020$}
	{$2018$}
	{$2021$}
	{$2019$}
	\loigiai{
		Ta có $y'=6x^2-6(2m+1)x+6m^2+6m$.\\
		Xét $y'=0$ $\Leftrightarrow x^2-(2m+1)x+m^2+m=0$, có $\Delta =(2m+1)^2-4\left( m^2+m \right)$ $=1>0$, $\forall m\in \mathbb{R}$. Suy ra phương trình $y'=0$ luôn có hai nghiệm phân biệt: $x_1=m$; $x_2=m+1$. Dễ thấy $x_1<x_2$.\\
		Bảng biến thiên
		\begin{center}
			\begin{tikzpicture}
				\tkzTabInit[nocadre=true,lgt=0.7,espcl=2.1]
				{$x$ /0.6,$y'$ /0.6,$y$ /2}
				{$-\infty$,$m$,$m+1$,$+\infty$}
				\tkzTabLine{,+,$0$,-,$0$,+,}
				\tkzTabVar{-/$-\infty$, +/$y(m)$,-/$y(m+1)$,+/$+\infty$}
			\end{tikzpicture}
		\end{center}
		Dựa vào bảng biến thiên ta thấy hàm số đồng biến trên mỗi khoảng $( -\infty ;m )$; $( m+1;+\infty )$. Vì thế, hàm số đồng biến trên $( 2:+\infty )$ khi $ m+1\le 2\Leftrightarrow m\le 1$.\\
		Suy ra có $2020$ giá trị nguyên của $ m$ thỏa mãn yêu cầu đề bài. }
\end{ex} 

\begin{ex}
	Tập hợp các giá trị thực của tham số $m$ để hàm số $y = - x^3 - 6x^2 + \left(4m - 9\right)x + 4$ nghịch biến trên khoảng $\left(- \infty; - 1\right)$ là
	\choice
	{$\left(- \infty; 0\right]$}
	{$\left[-\dfrac{3}{4}; +\infty\right)$}
	{\True $\left(- \infty; -\dfrac{3}{4}\right]$}
	{$\left[0; +\infty \right)$}
	\loigiai{ 
		Ta có $y'=-3x^2-12x+4m-9$. \\
		Hàm số đã cho nghịch biến trên khoảng $(-\infty;-1)$ khi và chỉ khi $y'\le 0$, $\forall x\in (-\infty;-1)$
		\begin{center}
			$\Leftrightarrow -3x^2-12x+4m-9\le 0\Leftrightarrow 4m\le 3x^2+12x+9$, $\forall x\in (-\infty;-1)$.
		\end{center}
		Đặt $g(x)=3x^2+12x+9\Rightarrow g'(x)=6x+12$. Giải $g'(x)=0\Leftrightarrow x=-2$.\\
		Bảng biến thiên của hàm số $g(x)$ trên $(-\infty;-1)$.
		\begin{center}
			\begin{tikzpicture}
				\tkzTabInit[nocadre=false,lgt=2,espcl=3.5,deltacl=0.6] %phần bắt buộc
				{$x$ /0.6,$g'(x)$ /0.6,$g(x)$ /2}%phần bắt buộc
				{$-\infty$,$-2$,$-1$}
				\tkzTabLine{,-,$0$,+,}
				\tkzTabVar{+/$+\infty$, -/$-3$,+/$0$}
			\end{tikzpicture}
		\end{center}
		Dựa vào bảng biến thiên suy  ra $4m\le g(x)$, $\forall x\in (-\infty;-1)\Leftrightarrow 4m\le -3\Leftrightarrow m\le -\dfrac{3}{4}$.
	}
\end{ex} 

\begin{ex}
	Tìm tất cả các giá trị thực của tham số $m$ sao cho hàm số $y=x^3-6x^2+mx+1$ đồng biến trên khoảng $\left(0;+\infty\right)$.
	\choice
	{$m\leq 12$}
	{\True $m\geq 12$}
	{$m\leq 0$}
	{$m\geq 0$}
	\loigiai{
		Tập xác định $\mathscr{D} =\mathbb{R}$.\\
		$y'=3x^2-12x+m$.\\
		Hàm số đồng biến trên khoảng $\left(0;+\infty\right)$ khi và chỉ khi
		{\allowdisplaybreaks
			\begin{eqnarray*}
				& & f'(x)\geq 0 , \forall x\in \left(0;+\infty\right) \\
				& \Leftrightarrow & 3x^2-12x+m \geq 0 , \forall x\in \left(0;+\infty\right) \\
				& \Leftrightarrow & m \geq -3x^2+12x , \forall x\in \left(0;+\infty\right).
		\end{eqnarray*}}
		Xét hàm số $g(x)= -3x^2+12x$ trên $\left(0;+\infty\right)$.
		Ta có $g'(x)=-6x+12 \Leftrightarrow x=2$.\\
		Bảng biến thiên của hàm số $g(x)$
		\begin{center}
			\begin{tikzpicture}
				\tkzTabInit[lgt=1.2,espcl=3]{$x$ /1, $y'$ /1,$y$ /2}{
					$0$,$2$,$+\infty$}
				\tkzTabLine{,+,0 ,-, }
				\tkzTabVar{-/$0$, +/$12$ ,-/$-\infty$ }
			\end{tikzpicture}
		\end{center}
		Suy ra hàm số đồng biến trên khoảng $\left(0;+\infty\right)$ khi $m \geq 12$.
	}
\end{ex} 

\begin{ex}
	Tìm tất cả các giá trị $m$ để hàm số $y=\dfrac{x^2-8x}{x+m}$ đồng biến trên mỗi khoảng xác định.
	\choice
	{$(-8;0)$}
	{$(0;8)$}
	{$[0;8]$}
	{\True $[-8;0]$}
	\loigiai{
		Ta có $y'=\dfrac{x^2+2mx-8m}{(x+m)^2}$. Khi đó
		\allowdisplaybreaks
		\begin{eqnarray*}
			\text{YCBT} &\Leftrightarrow & x^2+2mx-8m\ge 0, \forall x \Leftrightarrow \Delta' \le 0\\
			&\Leftrightarrow & m^2+8m\le 0\Leftrightarrow -8\le m\le 0.
		\end{eqnarray*}
	}
\end{ex} 

\begin{ex}
	Tập hợp các giá trị thực của tham số $m$ để hàm số $y=x+1+\dfrac{m}{x-2}$ đồng biến trên mỗi khoảng xác định của nó là
	\choice
	{$\left(-\infty;0\right)$}
	{$\left[0;1\right)$}
	{$\left[0;+\infty \right)\backslash \left\{1\right\}$}
	{\True $\left(-\infty;0\right]$}
	\loigiai{
		Tập xác định $\mathscr{D}=\mathbb{R}\backslash \left\{2\right\}$.
		Ta có $y'=1-\dfrac{m}{\left(x-2\right)^2}$.\\
		Hàm số đồng biến trên mỗi khoảng các định của nó khi và chỉ khi
		\begin{eqnarray*}
			&&y'\geq 0,\;\forall x\in \mathbb{R}\backslash \left\{2\right\}\Leftrightarrow 1-\dfrac{m}{\left(x-2\right)^2}\geq 0,\;\forall x\in \mathbb{R}\backslash \left\{2\right\}\\
			&\Leftrightarrow &m\le {\left(x-2\right)}^2,\;\forall x\in \mathbb{R}\backslash \left\{2\right\}\Leftrightarrow m\leq 0.
		\end{eqnarray*}
	}
\end{ex} 

\begin{ex}%[2D1K1-3]%
	Tìm tất cả các giá trị thực của tham số $ m $ để hàm số $ f(x)=2^{x^3-x^2+mx+1}$ đồng biến trên khoảng $(1; 2)$.
	\choice
	{$m\leq-8$}
	{$m>-8$}
	{\True $m\geq-1$}
	{$m<-1$}
	\loigiai{
		Ta có $ f'(x)=(3x^2-2x+m)\cdot 2^{x^3-x^2+mx+1}\cdot\ln 2 $.\\
		Ta thấy\allowdisplaybreaks{
			\begin{eqnarray*}
				&& f(x) \textrm{ đồng biến trên } (1; 2)\\
				\Leftrightarrow && (3x^2-2x+m)\cdot 2^{x^3-x^2+mx+1}\cdot\ln 2\geq 0,\forall x\in (1; 2)\\
				\Leftrightarrow && (3x^2-2x+m)\geq 0,\forall x\in (1; 2)\\
				\Leftrightarrow && m\geq (-3x^2+2x),\forall x\in (1; 2)\\
				\Leftrightarrow && m\geq\max\limits_{[1; 2]} (-3x^2+2x)\\
				\Leftrightarrow && m\geq-1.
			\end{eqnarray*}
		}
	}
\end{ex} 

\begin{ex}
	Có bao nhiêu giá trị nguyên dương của tham số $m$ để hàm số $f(x)=(x+1)\ln x+(2-m)x$ đồng biến trên khoảng $(0;\mathrm{e}^2)$?
	\choice
	{0}
	{3}
	{5}
	{\True 4}
	\loigiai
	{Hàm số đã cho xác định khi $x>0$ hay $D=\big(0;+\infty\big)$\\
			Với $x>0$, ta có $f'(x)=\ln x+\dfrac{x+1}{x}+2-m$.\\
			Hàm số đã cho đồng biến trên khoảng $(0;\mathrm{e}^2)$ khi
			\allowdisplaybreaks
			\begin{align*}
				f'(x) \geq 0, \forall x \in (0;\mathrm{e}^2) &\Leftrightarrow \ln x+\dfrac{x+1}{x}+2-m \geq 0, \forall x \in (0;\mathrm{e}^2)\\
				&\Leftrightarrow m \leq \ln x+\dfrac{x+1}{x}+2, \forall x \in (0;\mathrm{e}^2). \tag{$*$}
			\end{align*}
			Xét hàm số $g(x)=\ln x+\dfrac{x+1}{x}+2, \forall x \in (0;\mathrm{e}^2)$.\\
			Ta có $g'(x)=\dfrac{1}{x}-\dfrac{1}{x^2}=\dfrac{x-1}{x^2}$. Khi đó $g'(x)=0$ có nghiệm $x=1 \in (0;\mathrm{e}^2)$.\\
			Bảng biến thiên của hàm số $g$
			\begin{center}
				\begin{tikzpicture}
					\tkzTabInit[nocadre=false,lgt=1.5,espcl=3.5,deltacl=0.6] %phần bắt buộc
					{$x$/0.6, $g'(x)$/0.6, $g(x)$/2} %phần bắt buộc
					{$0$, $1$, $\mathrm{e}^2$} % hàng 1 cột 2
					\tkzTabLine{,-,z,+,}
					\tkzTabVar{+/$+\infty$,-/$4$,+/$g(\mathrm{e}^2)$}
				\end{tikzpicture}
			\end{center}
			Từ bảng biến thiên trên, bất phương trình $(*)$ thỏa mãn khi $m \leq 4$.
	}
\end{ex} 


\Closesolutionfile{ans}

\ind{PHẦN II.} \inden{Câu trắc nghiệm đúng sai. Học sinh trả lời từ câu 18 đến câu 20. Trong mỗi ý a), b), c), d) ở mỗi câu, học sinh chọn đúng hoặc sai.}\\
	
\Opensolutionfile{ans}[ans/2D1-B1-d2-2]

\begin{ex}
	Cho hàm số $ y=mx^3+mx^2-(m+1)x+1 $, với $m$ là tham số.
	\choiceTF
	{\True Hàm số là hàm số bậc ba khi $m \ne 0$}
	{\True Tập xác định của hàm số là $\mathbb{R}$}
	{Hàm số đồng biến trên $\mathbb{R}$ khi và chỉ khi $m<-\dfrac{3}{4}$ hoặc $m \ge 0$}
	{Hàm số nghịch biến trên $\mathbb{R}$ khi và chỉ khi $-\dfrac{3}{4}\leq m<0$}
	\loigiai{
		\begin{enumerate}[a)]
			\item Với $m \ne 0$ thì hàm số đã cho là một hàm số bậc ba.
			\item Hàm số là hàm đa thức nên có tập xác định là $\mathbb{R}$.
			\item Ta có $ y'=3mx^2+2mx-(m+1)$.
			\begin{itemize}
				\item [$\bullet$] Với $m=0$ thì $y'=-1<0$ (không thỏa)
				\item [$\bullet$] Với $m \ne 0$, yêu cầu bài toán tương đương với
				$\heva{&m>0\\&\Delta \le 0} \Leftrightarrow \heva{&m>0\\&4m^2+3m \le 0}$ (không tồn tại $m$)
			\end{itemize}
			\item 
			\begin{itemize}
				\item [$\bullet$] Với $m=0$ thì $y'=-1<0$ (thỏa)
				\item [$\bullet$] Với $m \ne 0$, yêu cầu bài toán tương đương với
				$$\heva{&m<0\\&\Delta \le 0} \Leftrightarrow \heva{&m<0\\&4m^2+3m \le 0} \Leftrightarrow -\dfrac{3}{4}\leq m<0$$
			\end{itemize}
		Suy ra $-\dfrac{3}{4}\leq m \leq 0$.
		\end{enumerate}
	}
\end{ex} 

\begin{ex}
	Cho hàm số $y=\dfrac{1}{3}x^3 + (m + 1)x^2 + \left(m^2 + 2m\right)x - 3$, với $m$ là tham số.
	\choiceTF
	{Tập xác định của hàm số là $\mathbb{R}$}
	{\True Phương trình $y'=0$ có hai nghiệm phân biệt $x_1=-m$ và $x_2=-m-2$}
	{\True Không tồn tại giá trị của tham số $m$ để hàm số đồng biến trên $\mathbb{R}$}
	{Hàm số nghịch biến trên khoảng $(- 1; 1)$ khi và chỉ khi $m \ge -1$}
	\loigiai{
		\begin{enumerate}[a)]
			\item Hàm số là hàm đa thức nên có tập xác định là $\mathbb{R}$
			\item Ta có $y'=x^2+2(m+1)x+m^2+2m$. Do $\Delta'=b'^2-ac=(m+1)^2-(m^2+2m)=1>0$ nên phương trình có hai nghiệm phân biệt
			$x_1=\dfrac{-b'+\sqrt{\Delta'}}{a}=-m$ và $x_2=\dfrac{-b'-\sqrt{\Delta'}}{a}=-m-2$.
			\item Bảng biến thiên
				\begin{center}
					\begin{tikzpicture}
						\tkzTabInit[lgt=1,espcl=3,nocadre=True]
						{$x$ /0.7, $y'$ /0.7, $y$ /2.5}
						{$-\infty$,$-m-2$,$-m$,$+\infty$}
						\tkzTabLine{,+,$0$,-,$0$,+,}
						\tkzTabVar{-/$-\infty$,+/$y(-m-2)$,-/$y(-m)$,+/$+\infty$}
					\end{tikzpicture}
				\end{center}
			Từ bảng biến thiên, suy ra không tồn tại giá trị của tham số $m$ để hàm số đồng biến trên $\mathbb{R}$
			\item Bảng biến thiên
			\begin{center}
				\begin{tikzpicture}
					\tkzTabInit[lgt=1,espcl=3,nocadre=True]
					{$x$ /0.7, $y'$ /0.7, $y$ /2.5}
					{$-\infty$,$-m-2$,$-m$,$+\infty$}
					\tkzTabLine{,+,$0$,-,$0$,+,}
					\tkzTabVar{-/$-\infty$,+/$y(-m-2)$,-/$y(-m)$,+/$+\infty$}
				\end{tikzpicture}
			\end{center}
			Từ bảng biến thiên, suy ra hàm số nghịch biến trên khoảng $(- 1; 1)$ khi và chỉ khi 
			$$\heva{&-m-2 \le -1\\& -m \ge 1} \Leftrightarrow m = -1.$$
		\end{enumerate}
		
	}
\end{ex} 

\begin{ex}
	Cho hàm số $ y=\dfrac{x+5}{x+m}$, với $m$ là tham số.
	\choiceTF
	{Tập xác định của hàm số là $\mathbb{R}$}
	{Hàm số đồng biến trên từng khoảng xác định khi và chỉ khi $m \ge 5$}
	{\True Hàm số nghịch biến trên từng khoảng xác định khi và chỉ khi $m < 5$}
	{Hàm số đồng biến trên khoảng $\left(-\infty ;\, -8\right)$ khi và chỉ khi $\left(5;\, 8\right)$}
	\loigiai{
		\begin{enumerate}[a)]
			\item Điều kiện $x+m \ne 0 \Leftrightarrow x \ne -m$. Tập xác định là $D=\mathbb{R} \backslash\{-m\}$.
			\item Ta có $y'=\dfrac{m-5}{\left( x+m \right)^2},\forall x\in \mathbb{R}\backslash \left\{ -m \right\}.$\\
			Hàm số đồng biến trên từng khoảng xác định $\Leftrightarrow m-5>0 \Leftrightarrow m>5$.
			\item Ta có $y'=\dfrac{m-5}{\left( x+m \right)^2},\forall x\in \mathbb{R}\backslash \left\{ -m \right\}.$\\
			Hàm số nghịch biến trên từng khoảng xác định $\Leftrightarrow m-5<0 \Leftrightarrow m<5$.
			\item 	Hàm số $ y=\dfrac{x+5}{x+m}$ đồng biến trên khoảng $\left(-\infty ;\, -8\right)$ khi và chỉ khi 
			$$\heva{
				&\dfrac{m-5}{\left(x+m\right)^2}> 0\\
				&-m\notin\left(-\infty ;\, -8\right)
			}\Leftrightarrow \heva{
				&m > 5\\
				&-m\ge-8
			} \Leftrightarrow 5 < m\le 8.$$
		\end{enumerate}
	}
\end{ex} 

\Closesolutionfile{ans}


% \begin{dang}{Bài toán tìm m để hàm số có cực trị hoặc đạt cực trị tại điểm cho trước}
	\begin{enumerate}[\iconCV]
		\item Tìm $m$ để hàm số $y=f(x)$ đạt cực trị tại điểm $x_0$ cho trước ($f(x)$ có đạo hàm tại $x_0$):
		\begin{listEX}[1]
			\item [\ding{172}] Giải điều kiện $y'(x_0)=0$, tìm $m$.
			\item [\ding{173}] Lập bảng biến thiên với $m $ vừa tìm được và chọn giá trị $m$ nào thỏa yêu cầu.
				\end{listEX}
	\item Biện luận cực trị hàm số $y=ax^3+bx^2+cx+d$.\\
	Tính $y'=3ax^2+2bx+c$ với $\Delta_{y'}=b^2-3ac$
	\begin{itemize}
		\item[\ding{172}] $\heva{&\Delta_{y'} >0\\&a \ne 0}$: Hàm số có hai điểm cực trị
		\item[\ding{173}]  $\Delta_{y'} \le 0$ hoặc suy biến $\heva{&a=0\\&b=0}$: Hàm số không có cực trị.
	\end{itemize}
	% \begin{note}
		\begin{enumerate}[\iconMT]
				\item Gọi $x_1$, $x_2$ là hai nghiệm phân biệt của $y'=0$ thì $x_1+x_2=-\dfrac{2b}{3a}$ và $x_1\cdot x_2 =\dfrac{c}{3a}$.
			\begin{itemize}
				\item [$\bullet$] $x_1^2+x_2^2=(x_1+x_2)^2-2x_1 x_2$
				\item [$\bullet$] $(x_1-x_2)^2=(x_1+x_2)^2-4x_1 x_2$
				\item [$\bullet$] $x_1^3+x_2^3=(x_1+x_2)^3-3x_1x_2(x_1+x_2)$.
			\end{itemize}
			\item Các công thức tính toán thường gặp:
			\begin{itemize}
				\item [$\bullet$] Độ dài $MN=\sqrt{(x_N-x_M)^2+(y_N-y_M)^2}$
				\item [$\bullet$]  Khoảng cách từ $M$ đến $\Delta$: $d(M,\Delta)=\dfrac{|Ax_M+By_M+C|}{\sqrt{A^2+B^2}}$, với $\Delta \colon Ax+By+C=0$.
				\item [$\bullet$] Tam giác $ABC$ vuông tại $A \Leftrightarrow \overrightarrow{AB} \cdot \overrightarrow{AC}=0 \Leftrightarrow \text{hoành}\cdot\text{hoành}+\text{tung}\cdot\text{tung}=0$.
				\item [$\bullet$] Diện tích tam giác $ABC$ là  $S=\dfrac{1}{2}|a_1b_2-a_2b_1|$, với $\overrightarrow{AB}=(a_1;b_1)$, $\overrightarrow{AC}=(a_2;b_2)$.
			\end{itemize}
			\item PTĐT qua hai điểm cực trị là $y=-\dfrac{2}{9a}(b^2-3ac)x+d-\dfrac{bc}{9a}$.
		\end{enumerate}
	% \end{note}
	\end{enumerate}
\end{dang}
\boxmini{BÀI TẬP TỰ LUẬN}
\setcounter{vd}{0}
\begin{vd}
	Tìm $m$ để hàm số
	\begin{tasks}
		\task  $y=\dfrac{x^3}{3}-mx^2+(m^2-m+1)x+1$ đạt cực tiểu tại $x=3$.
		\task  $y=x^3-3mx^2+3(m^2-1)x$ đạt cực đại tại $x_0=1$.
	\end{tasks}
\loigiai{
\begin{enumerate}[a)]
	\item Ta có $y'=x^2-2mx+m^2-m+1$. Hàm số đạt cực tiểu tại $x=3$ thì
	$$y'(3)=0 \Leftrightarrow 9-6m+m^2-m+1=0 \Leftrightarrow \hoac{&m=2\\&m=5}.$$
	Lập bảng biến thiên của hàm số với lần lượt hai giá trị $m$ vừa tìm được, ta thấy $m=2$ thỏa.\\
	Vậy $m=2$.
	\item Ta có $y'=3x^2-6mx+3(m^2-1)$\\
	Điều kiện cần và đủ để thỏa điều kiện bài toán là
	\begin{eqnarray*}
		\heva{&y'(1)=0 \\&y''(1)<0}
		\Leftrightarrow \heva{&3m^2-6m=0 \\&6-6m<0}
		\Leftrightarrow \heva{&m=0 \vee m=2 \\&m>1}
		\Leftrightarrow m=2.
	\end{eqnarray*}
	Vậy $m=2$ thì thỏa bài toán.
\end{enumerate}}
\end{vd}

\begin{vd}
	Tìm tất cả giá trị của tham số $m$ để hàm số (đồ thị hàm số)
	\begin{tasks}
		\task $ y=x^3-3x^2+2mx+m+2024$ có hai điểm cực trị.
		\task $ y=\dfrac{1}{3}x^3-mx^2+\left(m+2\right)x+2019$ không có cực trị.
		\task $y=x^3-3(m+1)x^2+12mx+2019$ có hai điểm cực trị $x_1,\ x_2$ thỏa mãn $x_1+x_2+2x_1x_2=-8$.
		\task $y=-x^3-3mx^2+m-2$ với $m$ là tham số có hai điểm cực trị $A,B$ sao cho $AB=2$.
	\end{tasks}
\loigiai{
\begin{enumerate}[a)]
	\item Ta có $y’=3x^2-6x+2m$.\\
	Hàm số có cực đại, cực tiểu khi và chỉ khi phương trình $y’=0$ có hai nghiệm phân biệt $\Leftrightarrow {\Delta }’_{y’}>0$ $\Leftrightarrow 9-6m>0$ $\Leftrightarrow m<\dfrac{3}{2}$.
	\item Ta có $y’=x^2-2mx+m+2$\\
	Hàm số đã cho không có cực trị $\Leftrightarrow$ phương trình $y’=0$ vô nghiệm hoặc có nghiệm kép hay ${\Delta }’_{y’} \le 0$ $\Leftrightarrow m^2-\left( m+2 \right)\le 0$ $\Leftrightarrow -1\le m\le 2$.
	\item Ta có $y'=3x^2-6(m+1)x+12m,\ y'=0\Leftrightarrow 3x^2-6(m+1)x+12m=0$. \\
	Hàm số có hai điểm cực trị $\Leftrightarrow \Delta '=9m^2-18m+9>0\Leftrightarrow m\ne 1$.\tagEX{1}
	Giả sử $x_1,\ x_2$ là hai nghiệm của phương trình $y'=0$, theo định lí Vi-ét ta có
	$$\heva{&x_1+x_2=-\dfrac{b}{a}=2(m+1)\\&x_1\cdot x_2=\dfrac{c}{a}=4m.}$$
	Do đó $x_1+x_2+2x_1\cdot x_2=-8\Leftrightarrow 2(m+1)+8m=-8\Leftrightarrow 10m=-10\Leftrightarrow m=-1$ thỏa mãn $(1)$.\\
	Vậy $m=-1$ là giá trị cần tìm của $m$.
	\item Ta có $y'=-3x^2-6mx$; $y'=0\Leftrightarrow
	\hoac{&x=0\\&x=-2m.\\}$\\
	Hàm số có hai điểm cực trị khi và chỉ khi $m\ne 0$.\\
	Gọi hai điểm cực trị của đồ thị hàm số là $A$, $B$.\\
	Ta có $A\left(0;m-2\right)$, $ B\left(-2m;-4{m}^{3}+m-2\right)$.\\
	Do đó
	{\allowdisplaybreaks
		\begin{align*}
			AB^2=4m^2+16m^6=4&\Leftrightarrow 4m^6+m^2-1=0\\
			&\Leftrightarrow m^2=\dfrac{1}{2}\Leftrightarrow m=\pm \dfrac{1}{\sqrt{2}}.
	\end{align*}}
\end{enumerate}}
\end{vd}

\boxmini{BÀI TẬP TRẮC NGHIỆM}
\ind{PHẦN I.} \inden{Câu trắc nghiệm nhiều phương án lựa chọn. Mỗi câu hỏi học sinh chỉ chọn một phương án.}\\
\setcounter{ex}{0}
\Opensolutionfile{ans}[ans/2D1-B1-d3-1]

\begin{ex}
	Tìm tất cả giá trị của tham số $m$ để hàm số $y=\dfrac{1}{3}x^3+(m+1)x^2+(1-3m)x+2$ có cực đại và cực tiểu.
	\choice
	{$m\leq-5;m\geq 0$}
	{\True $m <-5$; $m>0$}
	{$-5<m<0$}
	{$-5\leq m\leq 0$}
	\loigiai{
		Tập xác định $\mathscr{D}=\mathbb{R}$.\\
		Ta có $y’=x^2+2(m+1)x+1-3m$.\\
		Hàm số có cực đại và cực tiểu khi phương trình $y’=0$ có hai nghiệm phân biệt và đổi dấu qua các nghiệm đó.\\
		Khi đó $\Delta’_{y’}=(m+1)^2-(1-3m)>0\Leftrightarrow m^2+5m>0\Leftrightarrow \hoac{&m<-5\\&m>0.}$}
\end{ex} 

\begin{ex}
	Tìm tất cả các giá trị của tham số $ m $ để hàm số $ y=-x^3-3x^2+mx+2 $ có cực đại và cực tiểu.
	\choice
	{\True $m>-3$}
	{$m\geq 3$}
	{$m\geq-3$}
	{$m>3$}
	\loigiai{
		Ta có $ y'=-3x^2-6x+m $. Hàm số đã cho có cực đại và cực tiểu khi và chỉ khi phương trình $ y'=0 $ có $ 2 $ nghiệm phân biệt $\Leftrightarrow\Delta'>0\Leftrightarrow 9+3m>0\Leftrightarrow m>-3 $.
	}
\end{ex} 

\begin{ex}
	Cho hàm số $y=x^3-3(m+1)x^2+3(7m-3)x$. Số giá trị nguyên của tham số $m$ để hàm số không có cực trị là
	\choice
	{$2$}
	{$1$}
	{\True $4$}
	{$3$}
	\loigiai{
		Hàm số bậc $3$ không có cực trị khi và chỉ khi phương trình $y'=0 \Leftrightarrow 3x^2-6(m+1)x+3(7m-3)=0$ có nghiệm kép hoặc vô nghiệm hay
		$$\Delta' \le 0 \Leftrightarrow 9(m+1)^2-9(7m-3)\le 0 \Leftrightarrow m^2-5m+4 \le 0 \Leftrightarrow 1 \le m \le 4.$$
		Mà $m \in \mathbb{Z}$ nên $ m \in \{1;2;3;4\}$.\\
		Vậy có $4$ giá trị nguyên của $m$ thỏa mãn yêu cầu bài toán.
	}
\end{ex} 

\begin{ex}
	Cho hàm số $y=x^3-3(m+1)x^2+3(7m-3)x$. Gọi $S$ là tập hợp tất cả các giá trị nguyên của tham số $m$ để hàm số không có cực trị. Số phần tử của $S$ là
	\choice
	{$2$}
	{\True $4$}
	{$0$}
	{Vô số}
	\loigiai{
		Tập xác định là $\mathscr{D}=\mathbb{R}$.\\
		$y'=3x^2-6(m+1)x+3(7m-3)$.\\
		Hàm số không có cực trị khi và chỉ khi $\Delta'=9(m+1)^2-9(7m-3)\le 0\Leftrightarrow m^2-5m+4\le 0\Leftrightarrow 1\le m \le 4.$\\
		Vậy có $m\in \{1;2;3;4\}$.
	}
\end{ex} 

\begin{ex}
	Giả sử hàm số $ y=\dfrac{1}{3}x^3-x^2-\dfrac{1}{3}mx$ có hai điểm cực trị $x_1, x_2$ thỏa mãn $x_1+ x_2+2x_1x_2=0$. Giá trị của $ m$ là
	\choice
	{$ m=\dfrac{4}{3}$}
	{$ m=-3$}
	{\True $ m=3$}
	{$ m=2$}
	\loigiai{
		Ta có $y’=x^2-2x-\dfrac{1}{3}m$.\\
		$y’=0\Leftrightarrow  3x^2-6x-m=0$.\\
		Hàm số có hai cực trị $\Leftrightarrow y'=0$ có hai nghiệm phân biệt $\Leftrightarrow 9+3m>0 \Leftrightarrow m>-3$.\\
		Khi đó $x_1+ x_2 + 2x_1x_2=0 \Leftrightarrow 2-\dfrac{2m}{3}=0 \Leftrightarrow m=3$ (TM).}
\end{ex} 

\begin{ex}
	Cho hàm số $ f\left( x \right)=x^3-3x^2+mx-1$. Tìm giá trị của tham số $m$ để hàm số có hai cực trị $x_1, x_2$ thỏa $x_1^2+x_2^2=3$.
	\choice
	{$ m=\dfrac{1}{2}$}
	{$ m=-2$}
	{$ m=1$}
	{\True $ m=\dfrac{3}{2}$}
	\loigiai{
		TXĐ $D=\mathbb{R}$.\\
		${f}’\left( x \right)=3x^2-6x+m$.\\
		Hàm số có hai điểm cực trị $x_1, x_2 \Leftrightarrow {f}’\left( x \right)=0$ có hai nghiệm phân biệt $\Leftrightarrow 9-3m>0  \Leftrightarrow m<3$.\\
		Theo hệ thức Vi-et: $x_1+ x_2=2$; $x_1.x_2=\dfrac{m}{3}$.\\
		Khi đó: $x_1^2+x_2^2=3  \Leftrightarrow  \left( {x_1+ x_2} \right)^2 - 2x_1x_2=3 \Leftrightarrow 2^2-\dfrac{2m}{3}=3 \Leftrightarrow m=\dfrac{3}{2}$.}
\end{ex} 

\begin{ex}
	Tìm tất cả các giá trị của tham số $m$ để đồ thị hàm số $y=x^3-12x+m+2$ có hai cực trị và hai điểm cực
	trị này nằm về hai phía trục hoành?
	\choice
	{$m=-2$}
	{\True $-18<m<14$}
	{$\forall m\in \mathbb{R}$}
	{$m\neq 1$}
	\loigiai{
		Ta có $y'=3x^2-12$. Suy ra $y'=0\Leftrightarrow \hoac{& x=2\Rightarrow y=m-14 \\ & x=-2\Rightarrow y=m+18.}$\\
		Đồ thị hàm số có hai điểm cực trị nằm về hai phía trục hoành khi và chỉ khi
		$$(m-14)(m+18)<0\Leftrightarrow -18<m<14.$$
	}
\end{ex} 

\begin{ex}
	Tập hợp các giá trị của $m$ để đồ thị hàm số $y=x^3+mx^2-\left(m^2-4\right)x+1$ có hai điểm cực trị nằm ở hai phía của trục $Oy$ là
	\choice
	{$(-\infty;2)$}
	{\True $\mathbb{R}\setminus[-2;2]$}
	{$(-2;2)$}
	{$(2;+\infty)$}
	\loigiai{
		Ta có $y'=3x^2+2mx+4-m^2$.\\
		Đồ thị hàm số có hai cực trị nằm hai phía đối với trục $Oy$ khi và chỉ khi $y'=0$ có hai nghiệm trái dấu $\Leftrightarrow P=\dfrac{4-m^2}{3}<0\Leftrightarrow\hoac{&m>2\\&m<-2.}$}
\end{ex} 

\begin{ex}
	Cho hàm số $y=x^3+3mx^2+3(m^2-1)x+m^3.$ Tìm $m$ để hàm số đạt cực tiểu tại điểm $x=0.$
	\choice
	{$m=-1$}
	{\True $m=1$}
	{$m=0$}
	{$m=2$}
	\loigiai{
		Ta có $y'=3x^2+6mx+3(m^2-1)$ và $y''=6x+6m\Rightarrow y''(0)=6m.$\\
		Hàm số đạt cực tiểu tại $x=0\Rightarrow y'(0)=0\Leftrightarrow 3(m^2-1)=0\Leftrightarrow m=\pm 1.$\\
		Với $m=1\Rightarrow y''(0)=6>0\Rightarrow$  hàm số đạt cực tiểu tại $x=0.$\\
		Với $m=-1\Rightarrow y''(0)=-6<0\Rightarrow$  hàm số đạt cực đại tại $x=0.$\\
		Vậy $m=1$ thỏa mãn bài.
	}
\end{ex} 

\begin{ex}
	Hàm số $ y=x^3-2mx^2+m^2x-2 $ đạt cực tiểu tại $ x=1 $ khi
	\choice
	{$ m=3 $}
	{$ m=-3 $}
	{\True $ m=1 $}
	{$ m=-1 $}
	\loigiai{
		Ta có: $ y'=3x^2-4mx+m^2 ,
		y''=6x-4m. $\\
		Hàm số đạt cực tiểu tại $ x=1 $, suy ra $y'(1)=0\Leftrightarrow m^2-4m+3=0 \Leftrightarrow \hoac{&m=1\\&m=3.}$
		\begin{itemize}
			\item Với $m=1$ ta có $y'(1)=0, y''(1)=2>0$ nên hàm số đạt cực tiểu tại $x=1$.
			\item Với $m=3$ ta có $y'(1)=0, y''(1)=-6<0$ nên hàm số đạt cực đại tại $x=1$.
	\end{itemize}}
\end{ex} 

\begin{ex}
	Tìm giá trị thực của tham số $m$ để hàm số $y=\dfrac{1}{3}x^3-mx^2+(m^2-4)x+3$ đạt cực tiểu tại $x=3$.
	\choice
	{$m=-1$}
	{\True $m=1$}
	{$m=-7$}
	{$m=5$}
	\loigiai{
		Ta có $y'=x^2-2mx+m^2-4$ và $y''=2x-2m$.\\
		Hàm số đạt cực tiểu tại $x=3$ nên $y'(3)=0 \Leftrightarrow 9-6m+m^2-4=0 \Leftrightarrow \hoac{&m=5 \\ &m=1.}$\\
		Với $m=5$ thì $y''(3)=-4<0$, loại.\\
		Với $m=1$ thì $y''(3)=4>0$, thỏa mãn.
	}
\end{ex} 

\begin{ex}
	Đồ thị hàm số $y=x^3-3x^2+2ax+b$ (với $a, b \in \mathbb{R}$) có điểm cực tiểu $A(2;-2)$. Khi đó $a+b$ bằng
	\choice
	{$-4$}
	{$4$}
	{\True $2$}
	{$-2$}
	\loigiai{
		Ta có: $y’=3x^2-6x+2a; y''=6x-6$.\\
		Đồ thị hàm số có điểm cực tiểu $A(2;-2)$ nên ta có:
		$\heva{&y’(2)=0\\&y(2)=-2} \Leftrightarrow \heva{&2a=0\\&4a+b=2} \Leftrightarrow \heva{&a=0\\&b=2.}$\\
		Với $a=2,b=0$ ta thấy $y''(2)=6.2-6=6>0$ nên hàm số đạt cực tiểu tại $x=2$, thỏa yêu cầu bài toán.\\
		Suy ra $a+b=2$.
	}
\end{ex} 

\begin{ex}
	Gọi $m_1, m_2$ là các giá trị của tham số $m$ để đồ thị hàm số $y=2x^{3}-3x^{2}+m-1$ có hai điểm cực trị $B, C$ sao cho tam giác $OBC$ có diện tích bằng 2, với $O$ là gốc tọa độ. Tích $m_{1} \cdot m_{2}$ bằng
	\choice
	{$12$}
	{$6$}
	{\True $-15$}
	{$-20$}
	\loigiai{
		Tập xác định: $\mathscr{D}=\mathbb{R}$.\\
		Ta có \begin{eqnarray*}
			y'=6 x^{2}-6 x=0 &\Leftrightarrow&
			\hoac{x=0 \Rightarrow y=m-1 \Rightarrow B(0 ; m-1) \\ x=1 \Rightarrow y=m-2 \Rightarrow C(1 ; m-2)}\\
			&\Rightarrow& S_{\triangle OBC}=\dfrac{1}{2} d(C ; O B) \cdot O B=\dfrac{1}{2} \cdot 1 \cdot |m-1|=2\\
			&\Leftrightarrow& |m-1|=4
			\Leftrightarrow \hoac{&m_1=5 \\ &m_2=-3.}
		\end{eqnarray*}
		Vậy $m_1 \cdot m_2 = -15$.
	}
\end{ex} 

\begin{ex}
	Cho hàm số $y=x^3-3mx^2+3m^3$. Biết rằng có hai giá trị của tham số $m$ để đồ thị hàm số có hai điểm cực trị $A,B$ và tam giác $OAB$ có diện tích bằng $48$. Khi đó tổng các giá trị của $m$ là
	\choice
	{\True $0$}
	{$2$}
	{$\sqrt{2}$}
	{$-2$}
	\loigiai{
		Tập xác định $\mathscr{D}=\mathbb{R}$.\\
		Đạo hàm $y'=3x^2-6mx$, xác định với mọi $x\in\mathbb{R}$.\\
		$y'=0\Leftrightarrow\hoac{&x=0\\&x=2m.}$ \\
		Do đó hàm số có hai cực trị khi và chỉ khi $m\neq 0$.\\
		Khi đó $A\left(0;3m^3\right)$, $B\left(2m;-m^3\right)$.\\
		Suy ra $\overrightarrow{OA}=\left(0;3m^3\right)$, $\overrightarrow{OB}=\left(2m;-m^3\right)$.\\
		$S_{\triangle OAB}=48\Leftrightarrow \dfrac{1}{2}\left|\left[\overrightarrow{OA},\overrightarrow{OB}\right]\right|=48\Leftrightarrow \left|-6m^4\right|=96\Leftrightarrow m=\pm 2$.\\
		Vậy tổng các giá trị của $m$ là $0$.
	}
\end{ex} 
\Closesolutionfile{ans}

\ind{PHẦN II.} \inden{Câu trắc nghiệm đúng sai. Trong mỗi ý a), b), c), d) ở mỗi câu, học sinh chọn đúng hoặc sai.}\\
\Opensolutionfile{ans}[ans/2D1-B1-d3-2]

\begin{ex}
	Cho hàm số $ y=\dfrac{m}{3}x^3+2x^2+mx+1$, với $m$ là tham số.
	\choiceTF
	{Hàm số có hai điểm cực trị khi $-2<m<2$}
	{Hàm số có đúng một điểm cực trị khi $m=0$ hoặc $m=2$}
	{\True Hàm số không có cực trị khi $m \le -2$ hoặc $m \ge 2$}
	{\True Hàm số có $2$ điểm cực trị thỏa mãn $x_\text{CĐ}<x_{CT}$ khi $0<m<2$}
	\loigiai{
		\begin{enumerate}[a)]
			\item Ta có $y’=mx^2+4x+m$.\\
			Hàm số có $2$ điểm cực trị $\Leftrightarrow y’=0$ có $2$ nghiệm phân biệt $\Leftrightarrow \left\{ \begin{aligned}
				& m\ne 0 \\
				& 4-m^2>0 \\
			\end{aligned} \right.\Leftrightarrow \left\{ \begin{aligned}
				& m\ne 0 \\
				& -2<m<2 \\
			\end{aligned} \right.\quad(1)$.
			\item Hàm số có đúng 1 cực trị khi hàm số này bị suy biến về hàm bậc hai, nghĩa là $\dfrac{m}{3}=0 \Leftrightarrow m=0$.
			\item Với $m=0$ thì hàm số trở thành $y=2x^2+1$. Hàm số này có 1 điểm cực tiểu. Điều này không thỏa yêu cầu bài toán\\
			Với $m \ne 0$: Hàm số không có cực trị $\Leftrightarrow y’=0$ có vô nghiệm hoặc nghiệm kép. $\Leftrightarrow \left\{ \begin{aligned}
				& m\ne 0 \\
				& 4-m^2 \le 0 \\
			\end{aligned} \right.\Leftrightarrow \left\{ \begin{aligned}
				& m\ne 0 \\
				&  m \le -2,\,m \ge 2\\
			\end{aligned} \right.$.
			\item Dựa vào dạng đồ thị hàm số bậc $3$, hàm số có $2$ điểm cực trị thỏa mãn $x_\text{CĐ}<x_{CT}$ khi $ m>0$ $(2)$\\
			Từ $\left(1\right)$ và $\left(2\right)$ suy ra giá trị $ m$ cần tìm là $0<m<2$.
		\end{enumerate}
}
\end{ex} 

\begin{ex}
	Cho hàm số $y=x^3-3mx^2+3\left(m^2-1\right)x-m^3$ với $m$ là tham số.
	\choiceTF
	{\True Hàm số luôn có hai điểm cực trị với mọi $m$}
	{\True Hàm số đạt cực tiểu tại $x=3$ khi $m=2$}
	{\True Khi đồ thị hàm số có hai điểm cực trị thì khoảng cách giữa hai điểm cực trị bằng $2\sqrt{5}$}
	{\True Điểm cực tiểu của đồ thị hàm số luôn thuộc đường thẳng cố định với hệ số góc $k=-3$}
	\loigiai
	{
		\begin{enumerate}[a)]
			\item Ta có $y'=3x^2-6mx+3\left(m^2-1\right). y'=0\Leftrightarrow \hoac{&x_1=m-1\\&x_2=m+1}$.\\
			Do $x_1 \ne x_2, \,\forall m$ nên hàm số luôn có hai điểm cực trị.
			\item Dễ thấy $x=m+1$ là điểm cực tiểu. Suy ra, hàm số đạt cực tiểu tại $x=3$ khi $m+1=3 \Leftrightarrow m=2$.
			\item Với mọi $m$, tọa độ hai điểm cực trị là $A(m+1;-3m-2)$ và $B(m-1;-3m+2)$.\\
			Khoảng cách giữa hai điểm cực trị là $AB=\sqrt{(x_B-x_A)^2+(y_B-y_A)^2}=2\sqrt{5}$.
			\item Ta có $y'=3x^2-6mx+3\left(m^2-1\right). y'=0\Leftrightarrow \hoac{&x=m-1\\&x=m+1}$\\
			Vì là hàm số bậc ba với hệ số $a=1>0$ nên điểm cực tiểu của hàm số là $A\left(m+1;-3m-2\right)$. \\
			Lại có $-3m-2=-3\left(m+1\right)+1$ nên điểm cực tiểu của hàm số luôn thuộc đường thẳng $d:y=-3x+1$, hệ số góc $k=-3$.
		\end{enumerate}
	}
\end{ex} 

\begin{ex}
	Cho hàm số $y=\dfrac{x^2-2mx +m +2}{x-m}$, với $m$ là tham số.
	\choiceTF
	{\True Tập xác định của hàm số là $\mathbb{R}\backslash\{m\}$}
	{\True Có hai giá trị nguyên của tham số $m$ để hàm số có hai điểm cực trị}
	{\True Hàm số đạt cực đại tại $x=-1$ khi $m=\dfrac{1}{2}$}
	{Khi đồ thị hàm số có hai điểm cực trị thì đường thẳng qua hai điểm cực trị của đồ thị có phương trình là $y=2x-2m$}
	\loigiai{
		\begin{enumerate}[a)]
			\item Hàm số xác định khi $x-m \ne 0 \Leftrightarrow x \ne m$. Suy ra $\mathscr{D}=\mathbb{R}\backslash\{m\}$.
			\item $y'=\dfrac{x^2-2mx+2m^2-m-2}{(x-m)^2}$.\\
			Để hàm số có hai điểm cực trị thì $y'=0$ có hai nghiệm phân biệt khác $m$ hay $g(x)=x^2-2mx+2m^2-m-2$ có hai nghiệm phân biệt khác $m$.
			$$\Leftrightarrow \heva{&\Delta'>0\\&g(m) \ne 0} \Leftrightarrow \heva{&-m^2+m+2>0\\&m^2-m-2 \ne 0}  \Leftrightarrow m \in (-1;2).$$
			Vì $m$ nguyên nên $m \in \{0;1\}$.
			\item Hàm số đạt cực trị tại $x=-1$ thì $y'(-1)=0 \Leftrightarrow 2m^2+m-1 =0 \Leftrightarrow m=-1$ hoặc $m=\dfrac{1}{2}$.\\
			Thử lại với $m=\dfrac{1}{2}$, ta có $y'=\dfrac{x^2-x-2}{x-\dfrac{1}{2}}$.\\
				Bảng biến thiên
				\begin{center}
					\begin{tikzpicture}
						\tkzTabInit[nocadre=false,lgt=1,espcl=3]
						{$x$ /0.7,$y'$ /0.7,$y$ /2}
						{$-\infty$,$-1$,$0.5$,$2$,$+\infty$}
						\tkzTabLine{,+,$0$,-,d,-,$0$,+,}
						\tkzTabVar{-/$-\infty$,+/$y_1$,-D+/$-\infty$/$+\infty$,-/$y_2$,+/$+\infty$}
					\end{tikzpicture}
				\end{center}
			Suy ra $m=\dfrac{1}{2}$ thỏa yêu cầu bài toán.
			\item Cho hàm số $y=\dfrac{u(x)}{v(x)}$. Nếu đồ thị hàm số có hai điểm cực trị thì đường thẳng qua hai điểm cực trị có dạng $y=\dfrac{u'(x)}{v'(x)}$.\\
			Áp dụng, ta được $y=\dfrac{(x^2-2mx+m+2)'}{(x-m)'}=2x-2m$
		\end{enumerate}
	}
\end{ex} 

\Closesolutionfile{ans}

% \input{data/12/2D1-B1-4.tex}
% \begin{dang}{Cực trị hàm hợp, hàm chứa trị tuyệt đối}
    \begin{itemize}
        \item Các phép biến đổi đồ thị
        \begin{itemize}
            \item Đồ thị hàm $y=f(x+a)$ vẽ bằng cách dời đồ thị $y=f(x)$ sang trái $a$ đơn vị.
            \item Đồ thị hàm $y=f(x)+b$ vẽ bằng cách dời đồ thị $y=f(x)$ lên trên $b$ đơn vị.
            \item Đồ thị hàm $y=f(|x|)$ vẽ bằng cách "lật qua trái".
            \item Đồ thị hàm $y=|f(x)|$ vẽ bằng cách "lật lên".
            \item Đồ thị hàm $y=|f(|x|)|$ vẽ bằng cách "lật lên rồi lật qua trái".
        \end{itemize}
        \begin{note} Hàm $y=f(x)$ có $m$ điểm cực trị, $n$ nghiệm bội lẻ, $p$ điểm cực trị dương. Khi đó
            \begin{itemize}
                \item[-]Hàm $y=f(ax+b)+c$ cũng có $m$ điểm cực trị.
                \item[-]	Hàm $y=|f(x)|$ có $m+n$  điểm cực trị.
                \item[-] 	Hàm $y=f(|x|)$ có $2p+1$  điểm cực trị.
            \end{itemize}
        \end{note}
        \item Hàm $y=f(u)$.
        \begin{itemize}
            \item \textbf{Bước 1: } Tính đạo hàm $y'=u'f'(u)$.
            \item \textbf{Bước 2: } Lập bảng xét dấu của $y'$ hoặc đếm số nghiệm bội lẻ của $y'=0$.
            \item \textbf{Bước 3: } Kết luận.
        \end{itemize}
        \item Hàm $y=f(u)+g(x)$.
        \begin{itemize}
            \item \textbf{Bước 1: } Tính đạo hàm $y'=u'f'(u)+g'$.
            \item \textbf{Bước 2: } Lập bảng xét dấu của $y'$ hoặc đếm số nghiệm bội lẻ của $y'=0$ (dựa vào tương giao giữa hai đồ thị).
            \item \textbf{Bước 3: } Kết luận.
        \end{itemize}
    \end{itemize}
\end{dang}
\begin{vd}
    \immini{Cho hàm số $y=f(x)$ có bảng biến thiên như hình vẽ. Tìm các điểm cực trị, các cực trị của hàm số sau
        \begin{listEX}[1]
            \item $y=f(x+2)$
            \item $y=f(x)-3$
            \item $y=f(2x-3)+1$
            \item $y=f(1-2x)+2025$
    \end{listEX}}{\begin{tikzpicture}[>=stealth]
            \tkzTabInit[nocadre=false,lgt=1,espcl=2,deltacl=0.5]{$x$/.6 ,$y'$/.6,$y$/1.8}
            {$-\infty$ , $0$ , $2$ , $+\infty$}
            \tkzTabLine{ , - , $0$ , + , $0$ , - , }
            \tkzTabVar{+/$+\infty$ , -/$1$ , +/$5$ , -/$-\infty$}
    \end{tikzpicture}}
    \loigiai{}
\end{vd}
\begin{vd}
    \immini{Cho hàm số $y=f(x)$ có bảng biến thiên như hình vẽ.Tìm các điểm cực trị của hàm số sau
        \begin{listEX}[1]
            \item $y=f(x^2)$
            \item $y=f(3x^2-2x)$
            \item $y=f(\sqrt{x^2+2x+2})$
    \end{listEX}}{
        \begin{tikzpicture}[>=stealth]
            \tkzTabInit[nocadre=false,lgt=1,espcl=2,deltacl=0.5]{$x$/.6,$y'$/.6,$y$/1.8}
            {$-\infty$ , $0$ , $2$ , $+\infty$}
            \tkzTabLine{ , - , $0$ , + , $0$ , - , }
            \tkzTabVar{+/$+\infty$ , -/$1$ , +/$5$ , -/$-\infty$}
        \end{tikzpicture}
    }
    \loigiai{}
\end{vd}
\begin{vd}%[2D1G5-5]
    \immini{Cho hàm số $y=f(x)$ có đồ thị $y=f'(x)$ như hình vẽ. Tìm số điểm cực trị của các hàm số sau
        \begin{listEX}[2]
            \item $y=f(x)$
            \item $y=2f(x)-x$
            \item $y=f(3x)+2x$
            \item $y=f(x)+\dfrac{x^2}{2}-x$
            \item $y=3f(x)-2x^3$
            \item $y=f(2x+1)-4x$
    \end{listEX}}{\begin{tikzpicture}[smooth, >=latex, line cap =round, line join =round,font=\scriptsize,x=1.4cm]
            \begin{scope}[scale=.5]
                \draw[->] (-3,0)--(3,0) node[below]{$x$};
                \draw[->] (0,-2.5) -- (0,3) node[left] {$y$};
                \draw[ name path=dcong] (-2,-2)..controls +(90:0.3) and +(180:0.3)..(-0.7,2.7)..controls +(0:0.2) and +(180:0.3)..(0,0.5)..controls +(30:0.2) and +(180:0.3)..(1,2)..controls +(0:0.3) and +(90:0.1).. (2,-2);
                \draw[thick,dashed] (-1,0) node[below] {$-1$} --(-1,2) --(1,2) -- (1,0) node[below] {$1$} (0,2) node[above right] {$2$};
            \end{scope}
    \end{tikzpicture}}
    \loigiai{}
\end{vd}
\begin{vd}
    \immini{Cho hàm số $y=f(x)$ có đồ thị như hình vẽ. Tìm số điểm cực trị của hàm số
        \begin{listEX}[2]
            \item $y=f(|x|)$
            \item $y=|f(x)|$
            \item $y=|f(|x|)|$
            \item $y=f(|x|-a)$
            \item $y=f(|x+b|)$
            \item $y=|f(x+2025)|$
    \end{listEX}}{
        \begin{tikzpicture}[>=stealth,line join=round, line cap=round, font=\scriptsize]
            \begin{scope}[scale=.8]
                \draw[-stealth](-4,0)--(0,0)node[below right]{$O$}--(4,0)node[below left]{$x$};
                \draw[-stealth](0,-2)--(0,3)node[below left]{$y$};
                \draw[dashed]
                (-3,0)node[above]{$a$}--(-3,-2)
                (3,0)node[below]{$b$}--(3,3)
                ;
                \draw[smooth]
                (-3,-2)..controls+(85:3) and+(180:.5)..(-2,2)
                ..controls+(0:.5) and+(180:.5)..(-1,1)
                ..controls+(0:.5)and+(180:.5)..(0.5,2)
                ..controls+(0:.5)and+(180:.75)..(1.5,-1.5)
                ..controls+(0:.75)and+(-95:.3)..(3,3)
                ;
            \end{scope}
    \end{tikzpicture}}
    \loigiai{}
\end{vd}
\begin{vd}
    Tìm $m$ để
    \begin{listEX}
        \item  Hàm số $y=|f(x)|$ có $5$ điểm cực trị, với  $f(x)= 3x^3+3x^2+mx+m$
        \item Hàm số $y=f\left(\vert x\vert\right)$ có $5$ điểm cực trị, với $f(x)=x^3-(2m-1)x^2+(2-m)x+2$.
    \end{listEX}
    \loigiai{
        \begin{listEX}
            \item Đặt $f(x)=3x^3+3x^2+mx+m=3x^2(x+1)+m(x+1)=(x+1)(3x^2+m)$.\\
            Suy ra $f'(x)=9x^2+6x+m$.\\
            Phương trình $f'(x)=0$ có $2$ nghiệm phân biệt $x_1$, $x_2$ khi và chỉ khi $\Delta'=9-9m>0\Leftrightarrow m<1$. Khi đó ta có $x_1+x_2=-\dfrac{2}{3}$, $x_1x_2=\dfrac{m}{9}$.\\
            Hàm số $y=|f(x)|$ có $5$ điểm cực trị khi và chỉ khi $\heva{&\Delta'>0\\&y(x_1)\cdot y(x_2)<0.}$\\
            Thực hiện biến đổi
            \allowdisplaybreaks
            \begin{eqnarray*}
                y(x_1)\cdot y(x_2) &=&\ (x_1+1)(3x_1^2+m)\cdot(x_2+1)(3x_2^2+m)\\
                &=&\ \left[9(x_1x_2)^2+3m(x_1^2+x_2^2)+m^2\right]\left(x_1x_2+x_1+x_2+1\right)\\
                &=&\ \left[\dfrac{m^2}{9}+3m\left[\left(-\dfrac{2}{3}\right)^2-\dfrac{2m}{9}\right]+m^2\right]\left(\dfrac{m}{9}-\dfrac{2}{3}+1\right)\\
                &=&\ \dfrac{1}{9}(4m^2+12m)(m+3).
            \end{eqnarray*}
            Suy ra $y(x_1)\cdot y(x_2)<0\Leftrightarrow (4m^2+12m)(m+3)<0\Leftrightarrow -3\neq m<0$.\\
            Kết hợp với điều kiện $m$ là số nguyên thỏa $|m|<10$ ta được $m\in\{-1;-2;-4;-5;-6;-7;-8;-9\}$.\\
            Vậy có $8$ giá trị nguyên của tham số $m$.
            \item Tập xác định $\mathscr{D}=\mathbb{R}$.\\
            Ta có $f\left(|-x|\right)=f\left(|x|\right)$, $\forall x\in\mathbb{R}$ nên $y=f\left(|x|\right)$ là hàm số chẵn. \\
            Do đó, đồ thị hàm số $y=f\left(|x|\right)$ đối xứng qua trục tung.\\
            Suy ra hàm số $y=f\left(|x|\right)$ luôn có một điểm cực trị là $x=0$.\\
            Do đó, $y=f\left(|x|\right)$ có $5$ điểm cực trị $\Leftrightarrow$ hàm số $y=f(x)$ có $2$ điểm cực trị dương.\\
            \phantom{Do đó, số $y=f\left(|x|\right)$ có $5$ điểm cực trị} $\Leftrightarrow$ $f'(x)=0$ có hai nghiệm dương phân biệt.\\
            Ta có $f'(x)=3x^2-2(m-1)x+2-m$.\\
            Yêu cầu bài toán $\Leftrightarrow\heva{&\Delta'>0 \\ &S>0 \\ &P>0}\Leftrightarrow\heva{&4m^2-m-5>0 \\ &2m-1>0 \\ &2-m>0}\Leftrightarrow\heva{&m<-1\;\text{hoặc}\;m>\dfrac{5}{4} \\ &m>\dfrac{1}{2} \\ &m<3}\Leftrightarrow \dfrac{5}{4}<m<2$.
        \end{listEX}
    }
\end{vd}
\boxmini{BÀI TẬP TRẮC NGHIỆM}
\Opensolutionfile{ans}[ans/2D1-2-DANG-3]
\begin{ex}%[2D1K2-6]
    \immini
    {Cho hàm số $f(x)$ có đồ thị $f'(x)$ có đồ thị như hình vẽ bên dưới.\\ Hàm số $y=f(1-2x)$ có bao nhiêu cực trị ?
        \choice[2]
        {$4$}
        {$7$}
        {\True $3$}
        {$9$}
    }
    {
        \begin{tikzpicture}[>=stealth,font=\scriptsize]
            \begin{scope}[scale=0.55]
                \draw[->] (0,-1)--(0,3.5)node[right]{\scriptsize $y$};
                \draw[->] (-2,0)--(5,0)node[below]{\scriptsize $x$};
                \fill (0,0) node[below left]{\scriptsize  $O$} circle(1.5pt);
                \draw (-0.8,0) node[below left]{ $-1$} (0.9,0) node[below left]{ $1$} (2,0) node[below]{ $2$} (4,0) node[below]{ $4$};
                \clip (-2,-1) rectangle (5,3.5);
                \draw[] plot[smooth,tension=.65] coordinates{(-1.05,-0.9) (-0.3,2.5) (1.2,-0.5) (2.7,0.7) (4.2,0.2) (4.8,3.5)};
            \end{scope}
        \end{tikzpicture}
    }
    \loigiai{
        Đặt $g(x)=f(1-2x)$\\
        Dựa vào đồ thị, ta thấy $f'(x)=0$ có nghiệm $x_1=-1,x_2=1,x_3=2$ và $x_4=4$ nên $f'(x)$ có dạng $$f'(x)=k(x+1)(x-1)(x-2)(x-4)$$
        Khi đó $g'(x)=-2f'(1-2x)=-2k(2-2x)(-2x)(-1-2x)(-3-2x)^2$
        $$g'(x)=0 \Leftrightarrow \hoac{&x=1\\&x=0\\&x=-\dfrac{1}{2}\\&x=-\dfrac{3}{2} \text{ (kép)}}$$
        Bảng xét dấu $g'(x)$
        \begin{center}
            \begin{tikzpicture}[every node/.style={circle,fill=white,inner sep=0pt},arrow/.style={>=stealth,->,shorten <= 0.3cm,shorten >= 0.3cm},font=\footnotesize,xscale=1,yscale=1]
                \def\mnumline{1} %Số dòng
                \def\mnumcol{11} %Số cột
                \foreach \j in {0,...,\mnumline}
                \foreach \i in {0,...,\mnumcol}{
                    \coordinate (\j\i) at (\i,-\j);
                }
                \pgfmathsetmacro\yline{\mnumline/2-1}
                \path node at (00){$x$} node at (10){$g'(x)$};
                \foreach \x/\mnamex in {01/$-\infty$,03/$-\dfrac{3}{2}$,05/$-\dfrac{1}{2}$,07/$0$,09/$1$,0\mnumcol/$+\infty$} \path node at (\x) {\mnamex};
                \foreach \dy/\mnamedy in {12/$-$,13/$0$,14/$-$,15/$0$,16/$+$,17/$0$,18/$-$,19/$0$,110/$+$} \path node at (\dy) {\mnamedy};
                \draw[thick] (-.5,.5)rectangle([xshift=0.5cm,yshift=-0.5cm]\mnumline\mnumcol) ([xshift=-0.5cm,yshift=-0.5cm]00)--([xshift=0.5cm,yshift=-0.5cm]0\mnumcol)  ([xshift=0.5cm,yshift=0.5cm]00)--([xshift=0.5cm,yshift=-0.5cm]\mnumline0);
            \end{tikzpicture}
        \end{center}
        Dựa vào bảng xét dấu, ta thấy $g'(x)$ đổi dấu 3 lần nên $y=f(1-2x)$ có 3 cực trị.
    }
\end{ex}
\begin{ex}%[2D1K2-2]
    \immini{Cho hàm số $ f(x) $ có đạo hàm là $ f'(x) $. Đồ thị của hàm số $ y=f'(x) $ như hình vẽ bên. Khi đó hàm số $ y=f(x^2) $ có bao nhiêu điểm cực trị?
        \choice[2]
        {$2$}
        {$4$}
        {\True $3$}
        {$5$}}
    {
        \begin{tikzpicture}[line join=round, line cap=round,>=stealth,font=\scriptsize]
            \begin{scope}[scale=0.35]
                \tikzset{label style/.style={font=\footnotesize}}
                \def \xmin{-1.5}
                \def \xmax{6.5}
                \def \ymin{-2}
                \def \ymax{5.5}
                \def \hamso{-0.11*(\x)^3+1.09*(\x)^2-1.73*(\x)}
                \draw[->] (\xmin,0)--(\xmax,0) node[below left] {$x$};
                \draw[->] (0,\ymin)--(0,\ymax) node[below left] {$y$};
                \draw (0,0) node [below left] {$O$};
                \begin{scope}
                    \clip (\xmin+0.01,\ymin+0.01) rectangle (\xmax-0.01,\ymax-0.01);
                    \draw[samples=350,domain=-1.2:5.5,smooth,variable=\x] plot (\x,{\hamso});
                \end{scope}
                \draw [dashed] (5,4.6)--(5,0) node[below]{$5$} (2,0) node[below]{$2$};
            \end{scope}
        \end{tikzpicture}
    }
    \loigiai{$ y'=2xf'(x^2) $. Cho $ y'=0 \Leftrightarrow \hoac{&x=0\\&f'(x^2)=0} \Leftrightarrow \hoac{&x=0\\&x^2=0\\&x^2=2} \Leftrightarrow \hoac{&x=0\\&x=0 \text{ (nghiệm kép)}\\&x=\pm \sqrt{2}} $.\\
        $ y'=0 $ có 3 nghiệm bội bậc lẻ nên hàm số có 3 điểm cực trị.
    }
\end{ex}
\begin{ex}%[2D1K2-6]
    \immini{	Cho hàm số $y=f(x)$ xác định trên $\mathbb{R}$ và hàm số $y=f'(x)$ có đồ thị như hình vẽ. Hàm số $y=f(1-x^2)$ đạt cực đại tại điểm nào sau đây?
        \choice[2]
        {$x=-1$}
        {\True $x=\pm \sqrt{2}$}
        {$x=3$}
        {$x=0$}}{
        \begin{tikzpicture}[>=stealth, font=\scriptsize, line join=round, line cap=round,y=0.7cm]
            \begin{scope}[scale=.5]
                \def\a{1} \def\b{-2} \def\c{-2.5} % Hệ số
                \def\xmin{-2} \def\xmax{4}
                \def\ymin{-4} \def\ymax{1.5}
                %\draw[color=gray!50,dashed] (\xmin,\ymin) grid (\xmax,\ymax);
                \draw[->] (\xmin,0)--(\xmax,0);
                \draw[->] (0,\ymin)--(0,\ymax);
                \node at (0,0) [below right]{$O$};
                \node at (-1,0) [below left]{$-1$};
                \node at (3,0) [below right]{$3$};
                \clip (\xmin+0.1,\ymin+0.1) rectangle (\xmax-0.5,\ymax-0.1);
                \draw[smooth,samples=300,domain=-1.3:3.3] plot(\x,{\a*(\x)^2+\b*(\x)+\c});
            \end{scope}
    \end{tikzpicture}}
    \loigiai{Đặt $g(x)=f(1-x^2)$\\
        Khi đó $g'(x)=-2x\cdot f'(1-x^2)$\\
        Cho $g'(x)=0 \Leftrightarrow -2x \cdot f'(1-x^2) =0$
        $$ \Leftrightarrow \hoac{&x=0\\&f'(1-x^2)=0 \Leftrightarrow \hoac{&1-x^2=-1\Leftrightarrow x^2=2 \Leftrightarrow x=\pm \sqrt{2}\\&1-x^2=3}}$$
        Bảng xét dấu
        \begin{center}
            \begin{tikzpicture}[every node/.style={circle,fill=white,inner sep=0pt},arrow/.style={>=stealth,->,shorten <= 0.3cm,shorten >= 0.3cm},font=\footnotesize,xscale=1.4,yscale=.8]
                \def\mnumline{3} %Số dòng
                \def\mnumcol{9} %Số cột
                \foreach \j in {0,...,\mnumline}
                \foreach \i in {0,...,\mnumcol}{
                    \coordinate (\j\i) at (\i,-\j);
                    %	\draw[gray!30] ([xshift=-0.5cm,yshift=0.5cm]\j\i)--([xshift=0.5cm,yshift=0.5cm]\j\i)--([xshift=0.5cm,yshift=-0.5cm]\j\i)--([xshift=-0.5cm,yshift=-0.5cm]\j\i)--cycle (\j\i)node[]{\j\i}; %Ẩn lệnh này sau khi hoàn thành BBT
                }
                \pgfmathsetmacro\yline{\mnumline/2-1}
                \path node at (00){$x$} node at (10){$-x$} node at (20){\scriptsize $f'(1-x^2)$} node at (30){$g'(x)$};
                \foreach \x/\mnamex in {01/$-\infty$,03/$-\sqrt{2}$,05/$0$,07/$\sqrt{2}$,0\mnumcol/$+\infty$} \path node at (\x) {\mnamex};
                \foreach \dy/\mnamedy in {12/$-$,13/$0$,14/$+$,16/$+$} \path node at (\dy) {\mnamedy};
                \path node at ($(12)$){$+$} node at ($(13)$){$|$} node at ($(14)$){$+$} node at ($(15)$){$0$} node at ($(16)$){$-$} node at ($(17)$){$|$} node at ($(18)$){$-$} node at ($(22)$){$+$} node at ($(23)$){$0$} node at ($(24)$){$-$} node at ($(25)$){$|$} node at ($(26)$){$-$} node at ($(27)$){$0$} node at ($(28)$){$+$} node at ($(32)$){$+$} node at ($(33)$){$0$} node at ($(34)$){$-$} node at ($(35)$){$0$} node at ($(36)$){$+$} node at ($(37)$){$0$} node at ($(38)$){$-$};
                \draw[thick] (-.5,.5)rectangle([xshift=0.5cm,yshift=-0.5cm]\mnumline\mnumcol) ([xshift=-0.5cm,yshift=-0.5cm]00)--([xshift=0.5cm,yshift=-0.5cm]0\mnumcol) ([xshift=-0.5cm,yshift=-0.5cm]10)--([xshift=0.5cm,yshift=-0.5cm]1\mnumcol)

                ([xshift=-0.5cm,yshift=-0.5cm]20)--([xshift=0.5cm,yshift=-0.5cm]2\mnumcol)

                ([xshift=0.5cm,yshift=0.5cm]00)--([xshift=0.5cm,yshift=-0.5cm]\mnumline0); %Lệnh tự động kẻ bảng
            \end{tikzpicture}
        \end{center}
        Dựa vào bảng xét dấu ta xác định được hàm số đạt cực đại tại $x=\pm \sqrt{2}$.}
\end{ex}
\begin{ex}%[2D1K2-6]
    \immini{Cho hàm số $y=f(x)$ có đồ thị hàm $f'(x)=ax^2+bx+c$ như hình bên dưới. Hỏi hàm số $y=f(x-x^2)$ có bao nhiêu cực trị?
        \choice[2]
        {$0$}
        {\True $1$}
        {$2$}
        {$3$}}{
        \begin{tikzpicture}[>=stealth,x=1.2cm,y=0.7cm,font=\scriptsize]
            \begin{scope}[scale=0.35]
                \clip (-2,-2) rectangle (5,5.5);
                \def\a{1}
                \def\b{-3}
                \def\c{2}
                \draw[->] (-2,0) -- (4,0) node[below] { $x$};
                \draw[->] (0,-1) -- (0,5) node[left] {$y$};
                \draw (0,0)node[below left]{ $O$} circle(1.5pt);
                \draw (1,0) node[below]{$1$} (2,0) node[below]{  $2$} (0,2) node[left]{$2$};
                \pgfmathsetmacro\xdinh{-(\b)/2*(\a)}
                \pgfmathsetmacro\ydinh{(4*(\a)*(\c)-(\b)^2)/(4*(\a))}
                \draw[samples=150,smooth,domain=-5:5] plot(\x,{\a*(\x)^2+(\b)*\x+(\c)});
            \end{scope}
        \end{tikzpicture}
    }
    \loigiai{
        Đặt $g(x)=f\left(x-x^2\right)$\\
        Dựa vào đồ thị ta thấy $f'(x)=0$ có hai nghiệm $x_1=1,x_2=2$ nên $f'(x)$ có dạng $$f'(x)=k(x-1)(x-2)$$
        Khi đó $g'(x)=(1-2x)f'\left(x-x^2\right)=0$
        $$ \Leftrightarrow \hoac{&1-2x=0\\&f'\left(x-x^2\right)=0} \Leftrightarrow \hoac{&x=\dfrac{1}{2}\\&x-x^2=1\\&x-x^2=2} \Leftrightarrow \hoac{&x=\dfrac{1}{2}\\& \text{ vô nghiệm}\\&\text{ vô nghiệm.}}$$
        Bảng xét dấu
        \begin{center}
            \begin{tikzpicture}[every node/.style={circle,fill=white,inner sep=0pt},arrow/.style={>=stealth,->,shorten <= 0.3cm,shorten >= 0.3cm},font=\footnotesize,xscale=1,yscale=1]
                \def\mnumline{1} %Số dòng
                \def\mnumcol{5} %Số cột
                \foreach \j in {0,...,\mnumline}
                \foreach \i in {0,...,\mnumcol}{
                    \coordinate (\j\i) at (\i,-\j);
                }
                \pgfmathsetmacro\yline{\mnumline/2-1}
                \path node at (00){$x$} node at (10){$g'(x)$};
                \foreach \x/\mnamex in {01/$-\infty$,03/$\dfrac{1}{2}$,0\mnumcol/$+\infty$} \path node at (\x) {\mnamex};
                \foreach \dy/\mnamedy in {12/$+$,13/$0$,14/$-$} \path node at (\dy) {\mnamedy};
                \draw[thick] (-.5,.5)rectangle([xshift=0.5cm,yshift=-0.5cm]\mnumline\mnumcol) ([xshift=-0.5cm,yshift=-0.5cm]00)--([xshift=0.5cm,yshift=-0.5cm]0\mnumcol)  ([xshift=0.5cm,yshift=0.5cm]00)--([xshift=0.5cm,yshift=-0.5cm]\mnumline0);
            \end{tikzpicture}
        \end{center}
        Dựa vào bảng xét dấu, ta thấy $g(x)$ có 1 cực đại.
    }
\end{ex}
\begin{ex}%[2D1K2-2]
    \immini{Cho hàm số bậc bốn $y=f(x)$. Hàm số $y=f'(x)$
        có đồ thị như hình bên. Số điểm cực trị của hàm số $y=f\left(\sqrt{x^{2}+2 x+2}\right)$ là
        \choice[2]
        {$1$}
        {$2$}
        {$4$}
        {\True $3$}}
    {
        \begin{tikzpicture}[line join=round, line cap=round,>=stealth,font=\scriptsize]
            \begin{scope}[scale=0.5]
                \tikzset{label style/.style={font=\footnotesize}}
                \def \xmin{-2}
                \def \xmax{4.5}
                \def \ymin{-2}
                \def \ymax{3.5}
                \def \hamso{0.55*(\x)^3-1.76*(\x)^2-0.31*(\x)+2}
                \draw[->] (\xmin,0)--(\xmax,0) node[below left] {$x$};
                \draw[->] (0,\ymin)--(0,\ymax) node[below left] {$y$};
                \draw (0,0) node [below left] {$O$};
                \begin{scope}
                    \clip (\xmin+0.01,\ymin+0.01) rectangle (\xmax-0.01,\ymax-0.01);
                    \draw[samples=350,domain=-1.3:3.3,smooth,variable=\x] plot (\x,{\hamso});
                \end{scope}
                \draw (-1,0) node[below left]{$-1$} (1,0) node[below]{$1$} (3,0) node[below right]{$3$} (0,2) node[above left]{$2$};
            \end{scope}
        \end{tikzpicture}
    }
    \loigiai{
        $ y'=\dfrac{x+1}{\sqrt{x^2+2x+2}}f'(\sqrt{x^2+2x+2}) $.\\$ y'=0 \Leftrightarrow \hoac{&x=-1\\&f'(\sqrt{x^2+2x+2})=0} \Leftrightarrow \hoac{&x=-1\\&\sqrt{x^2+2x+2}=-1\\&\sqrt{x^2+2x+2}=1\\&\sqrt{x^2+2x+2}=3} \Leftrightarrow \hoac{&x=-1\\&x^2+2x+1=0\\&x^2+2x-7=0}\Leftrightarrow \hoac{&x=-1\\&x=-1 \text{ (nghiệm kép)}\\&x=-1\pm 2\sqrt{2}} $\\
        $ y'=0 $ có 3 nghiệm bội bậc lẻ nên hàm số có 3 điểm cực trị.
    }
\end{ex}
\begin{ex}%[2D1K2-2]
    \immini{Cho hàm số $ y=f(x) $ liên tục trên $ (a,b) $ và có đồ thị như hình bên. Số điểm cực trị của hàm số $ y=\left[f(x)\right]^2 $ trên $ (a;b) $ là
        \choice[2]
        {$4$}
        {$6$}
        {$2$}
        {\True $5$}}
    {
        \begin{tikzpicture}[line join=round, line cap=round,>=stealth,font=\scriptsize]
            \begin{scope}[scale=.35]
                \def \xmin{-3.5}
                \def \xmax{4.5}
                \def \ymin{-4}
                \def \ymax{3.5}
                \def \hamso{-0.37*(\x)^3+0.15*(\x)^2+2.41*(\x)-1}
                \draw[->] (\xmin,0)--(\xmax,0) node[below] {$x$};
                \draw[->] (0,\ymin)--(0,\ymax) node[left] {$y$};
                \draw (0,0) node [below left] {$O$};
                \clip (\xmin+0.01,\ymin+0.01) rectangle (\xmax-0.01,\ymax-0.01);
                \draw[samples=350,domain=-3:4,smooth,variable=\x] plot (\x,{\hamso});
                \draw[dashed] (-3,3.11)--(-3,0) node[below]{$a$} (3,-2.41)--(3,0) node[above]{$b$};
            \end{scope}
        \end{tikzpicture}
    }
    \loigiai{\immini{$ y=\left(f(x)\right)^2 $ nên $ y'=2f(x)f'(x) $.\\$ y'=0 \Leftrightarrow \hoac{&f(x)=0\\&f'(x)=0} \Leftrightarrow \hoac{&x=x_1,\ x=x_2,\ x=x_3\\&x=c,\ x=d}$.\\
            $ y'=0 $ có 5 nghiệm bội bậc lẻ thuộc $ (a,b) $ nên Số điểm cực trị của hàm số $ y=\left(f(x)\right)^2 $ trên $ (a;b) $ là 5.}
        {
            \begin{tikzpicture}[line join=round, line cap=round,>=stealth,thick,scale=0.8]
                \tikzset{label style/.style={font=\footnotesize}}
                \def \xmin{-3.5}
                \def \xmax{4.5}
                \def \ymin{-4}
                \def \ymax{3.5}
                \def \hamso{-0.37*(\x)^3+0.15*(\x)^2+2.41*(\x)-1}
                \draw[->] (\xmin,0)--(\xmax,0) node[below left] {$x$};
                \draw[->] (0,\ymin)--(0,\ymax) node[below left] {$y$};
                \draw (0,0) node [below left] {$O$};
                \begin{scope}
                    \clip (\xmin+0.01,\ymin+0.01) rectangle (\xmax-0.01,\ymax-0.01);
                    \draw[samples=350,domain=-3:4,smooth,variable=\x] plot (\x,{\hamso});
                \end{scope}
                \draw[dashed] (-3,3.11)--(-3,0) node[below]{\footnotesize $a$} (3,-2.41)--(3,0) node[above]{\footnotesize $b$} (2.54,0) node[below left]{\footnotesize $x_3$} (0.41,0) node[below right]{\footnotesize $x_2$} (-2.55,0) node[above right]{\footnotesize $x_1$} (-1.34,-3.07) -- (-1.34,0) node[above]{\footnotesize $c$} (1.61,1.72)--(1.61,0) node[below]{\footnotesize $d$};
            \end{tikzpicture}
        }
    }
\end{ex}
\begin{ex}%[2D1G2-1]
    \immini{Cho hàm số $y=f(x)$ có đạo hàm trên $\mathbb{R}$ và có bảng xét dấu $f'(x)$ như hình bên. Hàm số $y=f\left(x^{2}-2 x\right)$ có bao nhiêu điểm cực tiểu?
        \choice
        {\True $1$}
        {$2$}
        {$3$}
        {$4$}}{\begin{tikzpicture}
            \tkzTabInit[lgt=1,espcl=1.2]
            {$x$ /.7, $y'$ /.7}
            {$-\infty$,$-2$,$1$,$3$,$+\infty$}
            \tkzTabLine{ ,-,0,+,0,+,0,-, }
    \end{tikzpicture}}
    \loigiai{$ y'=(2x-2)f'(x^2-2x) $.
        \begin{eqnarray*}
            y'=0 	&\Leftrightarrow& \hoac{&x=1\\&f'(x^2-2x)=0}\\
            &\Leftrightarrow& \hoac{&x=1\\&x^2-2x=-2 \text{ (vô nghiệm)}\\&x^2-2x=1 \text{ (nghiệm bội bậc chẵn)}\\&x^2-2x=3} \\
            &\Leftrightarrow& \hoac{&x=1\\&x=1-\sqrt{2} \text{ (nghiệm bội bậc chẵn)}\\&x=1+\sqrt{2} \text{ (nghiệm bội bậc chẵn)}\\&x=3, \ x=-1.}
        \end{eqnarray*}
        $ y'=0 $ có 3 nghiệm bội bậc lẻ, khi đó $ y' $ đổi dấu qua các nghiệm này.\\
        $ y'=0 $ có 2 nghiệm bội bậc chẵn và $ y' $ sẽ không đổi dấu qua các nghiệm này.\\
        Tại $ x=4 $ thì $ y'(4)=(2\cdot 4 -2)f'(4^2-2\cdot 4)=6f'(8)<0 $.\\
        Bảng xét dấu
        \begin{center}
            \begin{tikzpicture}
                \tkzTabInit[lgt=1,espcl=1.2]
                {$x$ /1, $y'$ /1}
                {$-\infty$,$-1$,$1-\sqrt{2}$,$1+\sqrt{2}$,$3$,$+\infty$}
                \tkzTabLine{ ,-,0,+,0,+,0,+,0,-, }
            \end{tikzpicture}
        \end{center}
        Vậy hàm số có 1 điểm cực tiểu.
    }
\end{ex}
\begin{ex}%[2D1K2-6]
    \immini{Cho hàm số $f(x)$ có bảng biến thiên bên dưới. Trên khoảng $(-\sqrt{5};\sqrt{5})$ thì hàm số $y=f(x^2)$ đạt cực đại tại điểm nào sau đây?\choice
        {$x=\sqrt{2}$}
        {$x=-\sqrt{2}$}
        {\True $x=0$}
        {$x=2$}}{\begin{tikzpicture}
            \tkzTabInit[nocadre=false,lgt=1,espcl=1.6,deltacl=0.5]{$x$/.7 ,$f$/.7}
            {$-\infty$ , $0$ , $2$ , $+\infty$}
            \tkzTabLine{  , + , 0, - , 0 , +  }
    \end{tikzpicture}}
    \loigiai{Đặt $g(x)=f(x^2)$.\\
        Khi đó $g'(x)=2x \cdot f'(x^2)$.\\
        Cho $g'(x)=0 \Leftrightarrow 2x \cdot f'(x^2) =0 \Leftrightarrow
        \hoac{&x=0\\&f'(x^2)=0 \Leftrightarrow \hoac{x^2=0\\x^2=2} \Leftrightarrow \hoac{x=0\\x=\pm \sqrt{2}}}$\\
        Bảng xét dấu
        \begin{center}
            \begin{tikzpicture}[every node/.style={circle,fill=white,inner sep=0pt},arrow/.style={>=stealth,->,shorten <= 0.3cm,shorten >= 0.3cm},font=\footnotesize,xscale=1,yscale=.7]
                \def\mnumline{3} %Số dòng
                \def\mnumcol{9} %Số cột
                \foreach \j in {0,...,\mnumline}
                \foreach \i in {0,...,\mnumcol}{
                    \coordinate (\j\i) at (\i,-\j);
                }
                \pgfmathsetmacro\yline{\mnumline/2-1}
                \path node at (00){$x$} node at (10){$x$} node at (20){$f'(x^2)$} node at (30){$g'(x)$};
                \foreach \x/\mnamex in {01/$-\sqrt{5}$,03/$-\sqrt{2}$,05/$0$,07/$\sqrt{2}$,0\mnumcol/$\sqrt{5}$} \path node at (\x) {\mnamex};
                \foreach \dy/\mnamedy in {12/$-$,13/$0$,14/$+$,16/$+$} \path node at (\dy) {\mnamedy};
                \path node at ($(12)$){$-$} node at ($(13)$){$|$} node at ($(14)$){$-$} node at ($(15)$){$0$} node at ($(16)$){$+$} node at ($(17)$){$|$} node at ($(18)$){$+$} node at ($(22)$){$+$} node at ($(23)$){$0$} node at ($(24)$){$-$} node at ($(25)$){$0$} node at ($(26)$){$-$} node at ($(27)$){$0$} node at ($(28)$){$+$} node at ($(32)$){$-$} node at ($(33)$){$0$} node at ($(34)$){$+$} node at ($(35)$){$0$} node at ($(36)$){$-$} node at ($(37)$){$0$} node at ($(38)$){$+$};
                \draw[thick] (-.5,.5)rectangle([xshift=0.5cm,yshift=-0.5cm]\mnumline\mnumcol) ([xshift=-0.5cm,yshift=-0.5cm]00)--([xshift=0.5cm,yshift=-0.5cm]0\mnumcol) ([xshift=-0.5cm,yshift=-0.5cm]10)--([xshift=0.5cm,yshift=-0.5cm]1\mnumcol)

                ([xshift=-0.5cm,yshift=-0.5cm]20)--([xshift=0.5cm,yshift=-0.5cm]2\mnumcol)

                ([xshift=0.5cm,yshift=0.5cm]00)--([xshift=0.5cm,yshift=-0.5cm]\mnumline0); %Lệnh tự động kẻ bảng
            \end{tikzpicture}
        \end{center}
        Dựa vào bảng xét dấu ta xác định được hàm số đạt cực đại tại $x=0$.
    }
\end{ex}
\begin{ex}%[2D1K2-6]
    \immini{Cho hàm số $f(x)$ có bảng biến thiên bên dưới. Hàm số $y=f(x^2-2)$ đạt cực đại tại điểm nào sau đây?
        \choice
        {$x=-2$}
        {$x=-1$}
        {\True $x=0$}
        {$x=2$}}{\begin{tikzpicture}
            \tkzTabInit[nocadre=false,lgt=1,espcl=1.6,deltacl=0.5]{$x$/.7 ,$f$/.7}
            {$-\infty$ , $-1$ , $2$ , $+\infty$}
            \tkzTabLine{  , - , 0, - , 0 , +  }
    \end{tikzpicture}}
    \loigiai{Đặt $g(x)=f(x^2-2)$\\
        Khi đó $g'(x)=2x \cdot f'(x^2-2)$\\
        Cho $g'(x)=0 \Leftrightarrow 2x \cdot f'(x^2-2) =0$
        $$ \Leftrightarrow \hoac{&x=0\\&f'(x^2-2)=0 \Leftrightarrow \hoac{&x^2-2=-1\\&x^2-2=2} \Leftrightarrow \hoac{x^2=1\\x^2=4} \Leftrightarrow \hoac{x=\pm 1\\x=\pm 2}}$$
        Bảng xét dấu
        \begin{center}
            \begin{tikzpicture}[every node/.style={circle,fill=white,inner sep=0pt},arrow/.style={>=stealth,->,shorten <= 0.3cm,shorten >= 0.3cm},font=\footnotesize,xscale=1,yscale=1]
                \def\mnumline{3} %Số dòng
                \def\mnumcol{14} %Số cột
                \foreach \j in {0,...,\mnumline}
                \foreach \i in {0,...,\mnumcol}{
                    \coordinate (\j\i) at (\i,-\j);
                }
                \pgfmathsetmacro\yline{\mnumline/2-1}
                \path node at ([xshift=0.5cm]00){$x$} node at ([xshift=0.5cm]10){$x$}  node at ([xshift=0.5cm]20){$f'\left(x^2-2\right)$} node at ([xshift=0.5cm]\mnumline0){$g'(x)$};
                \foreach \x/\mnamex in {02/$-\infty$,04/$-2$,06/$-1$,08/$0$,010/$1$,012/$2$,0\mnumcol/$+\infty$} \path node at (\x) {\mnamex};
                \foreach \dy/\mnamedy in {13/$-$,14/$0$,15/$+$,16/$+$} \path node at (\dy) {\mnamedy};
                \path node at ($(13)$){$-$} node at ($(14)$){$|$} node at ($(15)$){$-$} node at ($(16)$){$|$} node at ($(17)$){$-$} node at ($(18)$){$0$} node at ($(19)$){$+$} node at ($(110)$){$|$} node at ($(111)$){$+$} node at ($(112)$){$|$} node at ($(113)$){$+$}
                node at ($(23)$){$+$} node at ($(24)$){$0$} node at ($(25)$){$-$} node at ($(26)$){$0$} node at ($(27)$){$-$} node at ($(28)$){$|$} node at ($(29)$){$-$} node at ($(210)$){$0$} node at ($(211)$){$-$} node at ($(212)$){$0$} node at ($(213)$){$+$}
                node at ($(33)$){$-$} node at ($(34)$){$0$} node at ($(35)$){$+$} node at ($(36)$){$0$} node at ($(37)$){$+$} node at ($(38)$){$0$} node at ($(39)$){$-$} node at ($(310)$){$0$} node at ($(311)$){$-$} node at ($(312)$){$0$} node at ($(313)$){$+$};
                \draw[thick] (-.5,.5)rectangle([xshift=0.5cm,yshift=-0.5cm]\mnumline\mnumcol) ([xshift=-0.5cm,yshift=-0.5cm]00)--([xshift=0.5cm,yshift=-0.5cm]0\mnumcol)
                ([xshift=-0.5cm,yshift=-0.5cm]20)--([xshift=0.5cm,yshift=-0.5cm]2\mnumcol)
                ([xshift=-0.5cm,yshift=-0.5cm]10)--([xshift=0.5cm,yshift=-0.5cm]1\mnumcol) ([xshift=0.5cm,yshift=0.5cm]01)--([xshift=0.5cm,yshift=-0.5cm]\mnumline1); %Lệnh tự động kẻ bảng
            \end{tikzpicture}
        \end{center}
        Dựa vào bảng xét dấu ta xác định được hàm số đạt cực đại tại $x=0$.
    }
\end{ex}
\begin{ex}%[2D1K2-1]
    Cho hàm số $ y=f(x) $ có đạo hàm $ f'(x)=x^2(x-1)(x-4)^2 $. Khi đó hàm số $ y=f(x^2) $ có bao nhiêu điểm cực trị?
    \choice
    {$4$}
    {\True $3$}
    {$5$}
    {$2$}
    \loigiai{$ f'(x)=0 \Leftrightarrow x=1 $ (nghiệm đơn), $ x=0 $ (nghiệm kép), $ x=4 $ (nghiệm kép).\\
        $ y=f(x^2) $ thì $ y'=2xf'(x^2) $.\\$y'=0 \Leftrightarrow \hoac{&x=0\\&x^2=1\\&x^2=0 \text{ (nghiệm kép)}\\&x^2=4 \text{ (nghiệm kép)}} \Leftrightarrow \hoac{&x=0\\&x=\pm 1\\&x=0 \text{ (nghiệm bội chẵn)}\\&x=\pm 2 \text{ (nghiệm bội chẵn).}} $\\
        Vậy hàm số có 3 điểm cực trị.
    }
\end{ex}
\begin{ex}%[2D1K2-6]
    Cho hàm $f(x)$ có đạo hàm $f'(x)=x^2-2x,\forall x\in \mathbb{R}$. Hàm số $y=f\left(1-\dfrac{1}{2}x\right)+4x$ có bao nhiêu điểm cực trị?
    \choice
    {0}
    {1}
    {\True 2}
    {3}
    \loigiai{Ta có $y'=-\dfrac{1}{2}f'\left(1-\dfrac{1}{2}x\right)+4$\\
        $y'=0 \Leftrightarrow
        f'\left(1-\dfrac{1}{2}x\right)=8\Leftrightarrow \left(1-\dfrac{1}{2}x\right)^2-2\left(1-\dfrac{1}{2}x\right)=8 \Leftrightarrow \dfrac{1}{4}x^2-9=0
        \Leftrightarrow \hoac{&x=-6\\&x=6}$\\
        Bảng xét dấu
        \begin{center}
            \begin{tikzpicture}[every node/.style={circle,fill=white,inner sep=0pt},arrow/.style={>=stealth,->,shorten <= 0.3cm,shorten >= 0.3cm},font=\footnotesize,xscale=1,yscale=1]
                \def\mnumline{1} %Số dòng
                \def\mnumcol{7} %Số cột
                \foreach \j in {0,...,\mnumline}
                \foreach \i in {0,...,\mnumcol}{
                }
                \pgfmathsetmacro\yline{\mnumline/2-1}
                \path node at (00){$x$} node at (10){$y'$};
                \foreach \x/\mnamex in {01/$-\infty$,03/$-6$,05/$6$,0\mnumcol/$+\infty$} \path node at (\x) {\mnamex};
                \foreach \dy/\mnamedy in {12/$+$,13/$0$,14/$-$,15/$0$,16/$+$} \path node at (\dy) {\mnamedy};
                \draw[thick] (-.5,.5)rectangle([xshift=0.5cm,yshift=-0.5cm]\mnumline\mnumcol) ([xshift=-0.5cm,yshift=-0.5cm]00)--([xshift=0.5cm,yshift=-0.5cm]0\mnumcol)  ([xshift=0.5cm,yshift=0.5cm]00)--([xshift=0.5cm,yshift=-0.5cm]\mnumline0); %Lệnh tự động kẻ bảng
            \end{tikzpicture}
        \end{center}
        Vậy hàm số $y=f\left(1-\dfrac{1}{2}x\right)+4x$ có 2 điểm cực trị.}
\end{ex}
\begin{ex}%[2D1G2-1]
    Cho hàm số $ y=f(x) $ có đạo hàm $ f'(x)=(x-1)^2(x^2-2x) $, với mọi $ x \in \mathbb{R} $. Có bao nhiêu giá trị nguyên dương của tham số $m$ để hàm số $ y=f(x^2-8x+m) $ có 5 điểm cực trị?
    \choice
    {\True $15$}
    {$16$}
    {$17$}
    {$18$}
    \loigiai{$ f'(x)=0 \Leftrightarrow x=1 $ (nghiệm kép), $ x=0 $ (nghiệm đơn), $ x=2 $ (nghiệm đơn).\\
        $ y=f(x^2-8x+m) $ thì $ y'=(2x-8)f'(x^2-8x+m) $.\\$y'=0 \Leftrightarrow \hoac{&x=4\\&x^2-8x+m=1 \text{ (nghiệm kép)}\\&x^2-8x+m=0 \quad (1)\\&x^2-8x+m=2 \quad (2)} $.\\
        Hàm số có 5 điểm cực trị $ \Leftrightarrow (1) $ có 2 nghiệm phân biệt khác 4 và $ (2) $ có 2 nghiệm phân biệt khác 4.\\$(1) $ có 2 nghiệm phân biệt khác 4 $ \Leftrightarrow \heva{&16-32+m \ne 0\\&\Delta'=16-m>0} \Leftrightarrow \heva{&m \ne 16\\&m<16}\Leftrightarrow m<16$.\\
        $(2) $ có 2 nghiệm phân biệt khác 4 $ \Leftrightarrow \heva{&16-32+m \ne 2\\&\Delta'=16-m+2>0} \Leftrightarrow \heva{&m \ne 18\\&m<18} \Leftrightarrow m<18$.\\
        Vậy ta có $ m<16 $ mà $ m $ nguyên dương nên $ m \in \{1,2,\cdots,15\} $ (15 số $ m $ thỏa mãn).
    }
\end{ex}
\begin{ex}%[2D1Y2-2]
    \immini
    {
        Cho hàm số $y=f(x)$ có đạo hàm liên tục trên $\mathbb{R}$. Đồ thị hàm số $y=f'(x)$ như hình vẽ bên. Số điểm cực trị của hàm số $y=f(x)-5x$ là
        \choice
        {$2$}
        {$3$}
        {$4$}
        {\True $1$}
    }
    {
        \begin{tikzpicture}[font=\scriptsize, line join=round, line cap=round, >=stealth,y=.8cm]
            \begin{scope}[scale=.6]
                \draw[->,>=latex](-3,0)--(3,0)node[above]{$x$};
                \draw[->,>=latex](0,-1)--(0,5)node[right]{$y$};
                \node[above left] at (0,0){$O$};
                \draw plot [samples=100,domain=-2.1:2.1] (\x,{(\x)^3-3*(\x)+2});
                \foreach\i in{-1,1}{\node[below] at (\i,0){$\i$};}
                \foreach\i in{4,2}{\node[right] at (0,\i){$\i$};}
                \draw[dashed](-1,0)--(-1,4)--(0,4);
            \end{scope}
        \end{tikzpicture}
    }
    \loigiai
    {
        Gọi $g(x)=f(x)-5x$. Ta có đạo hàm $g'(x)=f'(x)-5$. Bảng biến thiên của $g'(x)$ như hình dưới.
        \begin{center}
            \begin{tikzpicture}
                \tkzTabInit[nocadre=false,lgt=1.2,espcl=2.5,deltacl=0.6]
                {$x$/1, $f'(x)$/2, $g'(x)$/2}
                {$-\infty$, $-1$, $1$, $+\infty$}
                \tkzTabVar{-/ $-\infty$, +/$4$, -/$0$, +/$+\infty$}
                \tkzTabVar{-/ $-\infty$, +/$-1$, -/$-5$, +/$+\infty$}
            \end{tikzpicture}
        \end{center}
        Ta thấy $g'(x)$ chỉ đổi dấu một lần từ âm sang dương.\\
        Vì vậy hàm số $y=f(x)-5x$ có một điểm cực trị.
    }
\end{ex}
\begin{ex}%[2D1Y2-2]
    \immini
    {
        Cho hàm số $y=f(x)$ có đạo hàm trên $\mathbb{R}$. Biết hàm số $y=f'(x)$ có đồ thị như hình vẽ. Khẳng định nào sau đây đúng về cực trị của hàm số $g(x)=f(x)+x$?
        \choice
        {Hàm số có một điểm cực đại và một điểm cực tiểu}
        {Hàm số không có điểm cực đại và một điểm cực tiểu}
        {Hàm số có một điểm cực đại và hai điểm cực tiểu}
        {\True Hàm số có hai điểm cực đại và một điểm cực tiểu}
    }
    {
        \begin{tikzpicture}[ font=\scriptsize, line join=round, line cap=round, >=stealth]
            \begin{scope}[scale=.5]
                \foreach\x in{-1,0,...,3}{\draw[color=gray!30](\x,-3)--(\x,3.3);}
                \foreach\y in{-2,-1,...,3}{\draw[color=gray!30](-2,\y)--(4,\y);}
                \draw[->,>=latex](-2,0)--(4,0)node[above]{$x$};
                \draw[->,>=latex](0,-3)--(0,3.3)node[right]{$y$};
                \node[above left] at (0,0){$O$};
                \draw plot [samples=100,domain=-1.12:3.1] (\x,{-(\x)^3+3*(\x)^2-2});
            \end{scope}
        \end{tikzpicture}
    }
    \loigiai
    {
        Ta có $g'(x)=f'(x)+1$. Bảng biến thiên của $g'(x)$ như hình dưới.
        \begin{center}
            \begin{tikzpicture}
                \tkzTabInit[nocadre=false,lgt=1.2,espcl=2.5,deltacl=0.6]
                {$x$/1, $f'(x)$/2, $g'(x)$/2}
                {$-\infty$, $0$, $2$, $+\infty$}
                \tkzTabVar{+/ $+\infty$, -/$-2$, +/$2$, -/$-\infty$}
                \tkzTabVar{+/ $+\infty$, -/$-1$, +/$3$, -/$-\infty$}
            \end{tikzpicture}
        \end{center}
        Dựa vào bảng biến thiên của $g'(x)$, ta thấy đạo hàm đổi dấu từ dương sang âm hai lần, từ âm sang dương một lần.\\
        Do đó hàm số $g(x)$ có hai điểm cực đại và một điểm cực tiểu.
    }
\end{ex}
\begin{ex}%[2D1K2-6]
    \immini{	Cho hàm số $y=f(x)$ có đạo hàm trên $\mathbb{R}$ và có đồ thị hàm số $f'(x)$ như hình vẽ. Hàm số $y=2f(x)+x^2$ đạt cực đại tại điểm nào sau đây ?
        \choice[2]
        {\True $x=-1$}
        {$x=0$}
        {$x=1$}
        {$x=2$}}{\begin{tikzpicture}[>=stealth,font=\scriptsize,x=1.3cm]
            \begin{scope}[scale=.7]
                \draw[->] (-2,0) -- (3,0) node[below] {\scriptsize $x$};
                \draw[->] (0,-3) -- (0,2.5) node[left] { $y$};
                \draw (0,0)node[below left]{$O$} (-1.2,0) node[below]{ $-1$} (0,-2) node[below right]{ $-2$} (0,1) node[above left]{$1$} (0,-1) node[below right]{  $-1$} (1,0) node[above]{$1$} (2,0) node[above]{ $2$};
                \draw plot[smooth,tension=.65] coordinates{(-1.05,1.7) (-0.5,-2.6) (0.17,0.5) (0.9,-0.9) (1.5,-1.1) (2.1,-1.9) (2.3,2)};
                \draw[dashed] (-1,0) -- (-1,1) -- (0,1) (1,0)--(1,-1)--(0,-1) (2,0)--(2,-2)--(0,-2);
            \end{scope}
    \end{tikzpicture}}
    \loigiai{
        Đặt $g(x)=2f(x)+x^2$\\
        Khi đó $g'(x)=2f'(x)+2x=0 \Leftrightarrow 2\left(f'(x)+x\right)=0 \Leftrightarrow f'(x)=-x \quad (*)$\\
        Số nghiệm của phương trình $(*)$ là số giao điểm của đồ thị hàm số $y=f'(x)$ và $y=-x$\\
        Dựa vào hình bên ta thấy có $4$ giao điểm lần lượt có tọa độ là $(-1;1),(0;0),(1;-1)$ và $(2;-2)$ \\ $ \Rightarrow (*)  \Leftrightarrow \hoac{&x=-1 \quad \text{(đơn)}\\&x=0 \quad \text{(đơn)}\\&x=1 \quad \text{(kép)}\\&x=2 \quad \text{(kép)}.}$\\
        Bảng xét dấu
        \begin{center}
            \begin{tikzpicture}[every node/.style={circle,fill=white,inner sep=0pt},arrow/.style={>=stealth,->,shorten <= 0.3cm,shorten >= 0.3cm},font=\footnotesize,xscale=1,yscale=1]
                \def\mnumline{1} %Số dòng
                \def\mnumcol{11} %Số cột
                \foreach \j in {0,...,\mnumline}
                \foreach \i in {0,...,\mnumcol}{
                    \coordinate (\j\i) at (\i,-\j);
                }
                \pgfmathsetmacro\yline{\mnumline/2-1}
                \path node at (00){$x$} node at (10){$g'(x)$};
                \foreach \x/\mnamex in {01/$-\infty$,03/$-1$,05/$0$,07/$1$,09/$2$,0\mnumcol/$+\infty$} \path node at (\x) {\mnamex};
                \foreach \dy/\mnamedy in {12/$+$,13/$0$,14/$-$,15/$0$,16/$+$,17/$0$,18/$+$,19/$0$,110/$+$} \path node at (\dy) {\mnamedy};
                \draw[thick] (-.5,.5)rectangle([xshift=0.5cm,yshift=-0.5cm]\mnumline\mnumcol) ([xshift=-0.5cm,yshift=-0.5cm]00)--([xshift=0.5cm,yshift=-0.5cm]0\mnumcol)  ([xshift=0.5cm,yshift=0.5cm]00)--([xshift=0.5cm,yshift=-0.5cm]\mnumline0);
            \end{tikzpicture}
        \end{center}
        Dựa vào bảng xét dấu, ta thấy $g(x)$ đạt cực đại tại $x=-1$.
    }
\end{ex}
\begin{ex}%[2D1K2-6]
    \immini{Hàm số $y=f(x)$ liên tục trên $\mathbb{R}$ và có đồ thị hàm số $f'(x)$ như hình vẽ bên dưới. Hàm số $y=f(x)-\dfrac{1}{3}x^3+x^2-x+2$ đạt cực đại tại điểm nào sau đây ?
        \choice[2]
        {\True $x=1$}
        {$x=-1$}
        {$x=0$}
        {$x=2$}}{\begin{tikzpicture}[>=stealth,font=\scriptsize,y=.7cm]
            \begin{scope}[scale=.8]
                \draw[->] (-2,0) -- (3,0) node[below] {\scriptsize $x$};
                \draw[->] (0,-3) -- (0,2.5) node[left] {\scriptsize $y$};
                \draw (0,0)node[below left]{ $O$}  (-1.2,0) node[below left]{ $-1$} (0,-2) node[right]{ $-2$} (0,1) node[above left]{  $1$} (1,0) node[below]{ $1$} (2,0) node[below]{ $2$};
                \draw plot [samples=100,domain=-1.1:2.2] (\x,{(\x)^3-2*(\x)^2+1});
                \draw[dashed] (-1,0) -- (-1,-2) -- (0,-2) (0,1)--(2,1)--(2,0);
            \end{scope}
    \end{tikzpicture}}
    \loigiai{
        Đặt $g(x)=f(x)-\dfrac{1}{3}x^3+x^2-x+2$	\\
        Khi đó $g'(x)=f'(x)-x^2+2x-1$.\\
        $g'(x)=0 \Leftrightarrow f'(x)=x^2-2x+1 \quad (*)$
        \immini{Số nghiệm của $(*)$ cũng chính là số giao điểm của đồ thị hàm số $y=f'(x)$ với $y=x^2-2x+1$\\
            Dựa vào hình vẽ bên, ta thấy có $3$ giao điểm lần lượt có tọa độ là $(1;0),(2;1),(0;1)$. Khi đó,
            $(*) \Leftrightarrow \hoac{&x=1\\&x=0\\&x=2.}$
        }
        {\begin{tikzpicture}[>=stealth,x=1.0cm,y=1.0cm,scale=0.6]
                \draw[->] (-2,0) -- (3,0) node[below] {\scriptsize $x$};
                \draw[->] (0,-3) -- (0,2.5) node[left] {\scriptsize $y$};
                \draw (0,0)node[below right]{\scriptsize $O$} circle(1.5pt) (-1.2,0) node[below]{\scriptsize $-1$} (0,-2) node[right]{\scriptsize  $-2$} (0,1) node[left]{\scriptsize  $1$} (1,0) node[below]{\scriptsize  $1$} (2,0) node[below]{\scriptsize  $2$};
                \def\a{1}
                \def\b{-2}
                \def\c{1}
                \pgfmathsetmacro\xdinh{-(\b)/2*(\a)}
                \pgfmathsetmacro\ydinh{(4*(\a)*(\c)-(\b)^2)/(4*(\a))}
                \fill[dashed] (\xdinh,\ydinh)circle(2pt) edge (\xdinh,0) edge (0,\ydinh);
                \clip (-2,-3)rectangle(3,3);
                \draw[thick,samples=150,smooth,domain=-5:5] plot(\x,{\a*(\x)^2+(\b)*\x+(\c)});
                \draw[thick] plot[smooth,tension=.65] coordinates{(-1.1,-2.2) (-0.1,1) (1.4,-0.2) (2.5,2.8)};
                \draw[dashed] (-1,0) -- (-1,-2) -- (0,-2) (0,1)--(2,1)--(2,0);
        \end{tikzpicture}}
        \noindent
        Bảng xét dấu
        \begin{center}
            \begin{tikzpicture}[every node/.style={circle,fill=white,inner sep=0pt},arrow/.style={>=stealth,->,shorten <= 0.3cm,shorten >= 0.3cm},font=\footnotesize,xscale=1,yscale=1]
                \def\mnumline{1} %Số dòng
                \def\mnumcol{9} %Số cột
                \foreach \j in {0,...,\mnumline}
                \foreach \i in {0,...,\mnumcol}{
                    \coordinate (\j\i) at (\i,-\j);
                }
                \pgfmathsetmacro\yline{\mnumline/2-1}
                \path node at (00){$x$} node at (10){$g'(x)$};
                \foreach \x/\mnamex in {01/$-\infty$,03/$0$,05/$1$,07/$2$,0\mnumcol/$+\infty$} \path node at (\x) {\mnamex};
                \foreach \dy/\mnamedy in {12/$-$,13/$0$,14/$+$,15/$0$,16/$-$,17/$0$,18/$+$} \path node at (\dy) {\mnamedy};
                \draw[thick] (-.5,.5)rectangle([xshift=0.5cm,yshift=-0.5cm]\mnumline\mnumcol) ([xshift=-0.5cm,yshift=-0.5cm]00)--([xshift=0.5cm,yshift=-0.5cm]0\mnumcol)  ([xshift=0.5cm,yshift=0.5cm]00)--([xshift=0.5cm,yshift=-0.5cm]\mnumline0);
            \end{tikzpicture}
        \end{center}
        Hàm số đạt cực đại tại $x=1$.
    }
\end{ex}
\begin{ex}%[2D1G2-6]
    \immini{	Cho hàm số $f(x)$ có đạo hàm liên tục trên $\mathbb{R}$ và đồ thị $y=f'(x)$ như hình vẽ dưới đây. Xét trên khoảng $(-\pi;2\pi)$, số điểm cực trị của hàm số $g(x)=f(2\cos x)+2\cos2x$ là
        \choice[2]
        {$13$}
        {$10$}
        {\True $11$}
        {$9$}}{\begin{tikzpicture}[>=stealth,font=\scriptsize,x=1.3cm]
            \begin{scope}[scale=.5]
                \draw[->] (-2.5,0) -- (2.5,0) node[below] { $x$};
                \draw[->] (0,-2.5) -- (0,2.5) node[left] {$y$};
                \draw (0,0)node[below left]{ $O$};
                \draw (0,-2)node[below right]{$-2$} (0,2)node[above right]{\scriptsize $2$} (1,0) node[above]{ $1$} (-2,0) node[above]{ $-2$} (-1,0) node[below]{ $-1$} (2,0) node[below]{$2$};
                \draw[dashed] (-2,0)--(-2,-2)--(1,-2)--(1,0) (-1,0)--(-1,2)--(2,2)--(2,0);
                \clip (-2.5,-2.5)rectangle(2.5,3);
                \draw[samples=150,smooth,domain=-2.1:2.1] plot(\x,{(\x)^3-3*\x});

                \fill[black] (-2,0) circle(1.5pt) (-1,0) circle(1.5pt) (1,0) circle(1.5pt) (2,0) circle(1.5pt)(-2,-2) circle(1.5pt)(0,-2) circle(1.5pt)(1,-2) circle(1.5pt)(-1,2) circle(1.5pt)(0,2) circle(1.5pt)(2,2) circle(1.5pt);
            \end{scope}
    \end{tikzpicture}}
    \loigiai{
        Ta có $g'(x)=f'(2\cos x)\cdot(-2\sin x)-2\sin{2x}\cdot2=-2\sin{x}\left[f'(2\cos x)+4\cos x\right]$.\\
        Suy ra $g'(x)=0 \Leftrightarrow \hoac{&\sin x=0\\&f'(2\cos x)=-4\cos x.}$\\
        \begin{itemize}
            \item $\sin x=0 \Leftrightarrow x\in\{0;\pi\}$ vì $x\in(-\pi;2\pi)$.
            \item $f'(2\cos x)=-4\cos x$.\\
            Đặt $t=2\cos x$, vì $x\in(-\pi;2\pi)$ nên $t\in(-1;1)$.\\
            Phương trình trở thành $f'(t)=-2t$. Nghiệm của phương trình này là hoành độ giao điểm của đồ thị hàm số $y=f'(t)$ và đường thẳng $y=-2t$.\\
            \begin{center}
                \begin{tikzpicture}[>=stealth,x=1cm,y=1cm,scale=1]
                    \draw[->] (-2.5,0) -- (2.5,0) node[below] {\scriptsize $t$};
                    \draw[->] (0,-2.5) -- (0,2.5) node[left] {\scriptsize $y$};
                    \draw (0,0)node[below left]{\scriptsize $O$};
                    \draw (0,-2)node[below right]{\scriptsize $-2$} (0,2)node[above right]{\scriptsize $2$} (1,0) node[above]{\scriptsize $1$} (-2,0) node[above]{\scriptsize $-2$} (-1,0) node[below]{\scriptsize $-1$} (2,0) node[below]{\scriptsize $2$};
                    \draw[dashed] (-2,0)--(-2,-2)--(1,-2)--(1,0) (-1,0)--(-1,2)--(2,2)--(2,0);
                    \clip (-2.5,-2.5)rectangle(2.5,2.5);
                    \draw[thick,samples=150,smooth,domain=-2.1:2.1] plot(\x,{(\x)^3-3*\x}) node[right]{$(l)$};
                    \node[above left] at (2,2){\scriptsize $y=f'(t)$};
                    \fill[black] (-2,0) circle(1.5pt) (-1,0) circle(1.5pt) (1,0) circle(1.5pt) (2,0) circle(1.5pt)(-2,-2) circle(1.5pt)(0,-2) circle(1.5pt)(1,-2) circle(1.5pt)(-1,2) circle(1.5pt)(0,2) circle(1.5pt)(2,2) circle(1.5pt);
                    \draw[thick,samples=150,smooth,domain=-2.1:2.1] plot(\x,{-2*(\x)});
                \end{tikzpicture}
            \end{center}
            Suy ra $f'(t)=-2t \Leftrightarrow \hoac{&t=-1\\&t=0\\&t=1.}$

            \begin{itemize}
                \item Với $t=-1 \Rightarrow 2\cos x=-1 \Leftrightarrow \cos x=-\dfrac{1}{2} \Leftrightarrow x\in\left\{-\dfrac{2\pi}{3};\dfrac{2\pi}{3};\dfrac{4\pi}{3}
                \right\}$ vì $x\in(-\pi;2\pi)$.
                \item Với $t=0 \Rightarrow \cos x=0 \Leftrightarrow x\in\left\{-\dfrac{\pi}{2};\dfrac{\pi}{2};\dfrac{3\pi}{2}
                \right\}$ vì $x\in(-\pi;2\pi)$.
                \item Với $t=1 \Rightarrow 2\cos x=1 \Leftrightarrow \cos x=\dfrac{1}{2} \Leftrightarrow x\in\left\{-\dfrac{\pi}{3};\dfrac{\pi}{3};\dfrac{5\pi}{3}
                \right\}$ vì $x\in(-\pi;2\pi)$.
            \end{itemize}
        \end{itemize}
        Và
        \begin{itemize}
            \item $f'(t)+2t>0\Leftrightarrow f'(t)>-2t\Leftrightarrow \hoac{&-1<t<0\\&t>1}\\
            \Rightarrow \hoac{&-\dfrac{1}{2}<\cos x<0\\&\cos x>\dfrac{1}{2}}\Leftrightarrow \hoac{&-\dfrac{2\pi}{3}<x<-\dfrac{\pi}{3}\\&\dfrac{4\pi}{3}<x<\dfrac{5\pi}{3}\\&\dfrac{\pi}{3}<x<\dfrac{2\pi}{3}}$ (vì $x\in(-\pi;2\pi)$).
            \item $f'(t)+2t<0\Leftrightarrow f'(t)<-2t\Leftrightarrow \hoac{&t<-1\\&0<t<1}\\
            \Rightarrow \hoac{&\cos x<-\dfrac{1}{2}\\&0<\cos x<\dfrac{1}{2}} \Leftrightarrow \hoac{&-\pi<x<-\dfrac{2\pi}{3}\\&-\dfrac{\pi}{3}<x<\dfrac{\pi}{3}\\&\dfrac{2\pi}{3}<x<\dfrac{4\pi}{3}}$ (vì $x\in(-\pi;2\pi)$).
        \end{itemize}
        Bảng biến thiên hàm số $y=g(x)$
        \begin{center}
            \begin{tikzpicture}
                \tkzTabInit[nocadre=false,lgt=4,espcl=1]
                {$x$ /1.1,$-2\sin x$ /0.7,$f'(2\cos x)+4\cos x$ /0.7,$g'(x)$ /0.7,$g(x)$ /2}
                {$-\pi$,$-\dfrac{2\pi}{3}$,$-\dfrac{\pi}{2}$,$-\dfrac{\pi}{3}$,$0$,$\dfrac{\pi}{3}$,$\dfrac{\pi}{2}$,$\dfrac{2\pi}{3}$,$\pi$,$\dfrac{4\pi}{3}$,$\dfrac{3\pi}{2}$,$\dfrac{5\pi}{3}$,$2\pi$}
                \tkzTabLine{,+,|,+,|,+,|,+,$0$,-,|,-,|,-,|,-,$0$,+,|,+,|,+,|,+,}
                \tkzTabLine{,-,$0$,+,$0$,-,$0$,+,|,+,$0$,-,$0$,+,$0$,-,|,-,$0$,+,$0$,-,$0$,+,}
                \tkzTabLine{,-,$0$,+,$0$,-,$0$,+,|,-,$0$,+,$0$,-,$0$,+,|,-,$0$,+,$0$,-,$0$,+,}
                \tkzTabVar{+/,-/,+/,-/,+/,-/,+/,-/,+/,-/,+/,-/,+/,}
            \end{tikzpicture}
        \end{center}
        Từ bảng biến thiên ta suy ra hàm số $y=g(x)$ có $11$ điểm cực trị trên khoảng $(-\pi;2\pi)$.
    }
\end{ex}
\begin{ex}%[2D1G2-6]
    \immini{	Cho hàm số $y=f(x)$ có đồ thị của $y=f'(x)$ có đồ thị như hình vẽ bên dưới. Hàm số $g(x)=f(x^3-3x)-x^3+3x$ có bao nhiêu điểm cực tiểu? \choice[2]
        {$2$}
        {$4$}
        {$3$}
        {\True $5$}}{\begin{tikzpicture}[>=stealth,font=\scriptsize]
            \begin{scope}[scale=.5]
                \draw[->,line width = 1pt] (-2,0)--(0,0) node[below left]{$O$}--(5,0) node[below]{$x$};
                \draw[->,line width = 1pt] (0,-2) --(0,3) node[right]{$y$};
                \draw (-1,0) node[below left]{$-1$} circle (1pt);
                \draw (0,2) node[above right]{$2$} circle (1pt);
                \draw (2,0) node[below left]{$2$} circle (1pt);
                \draw (4,0) node[below right]{$4$} circle (1pt);
                \draw [ domain=-1.3:4.6, samples=100] %
                plot (\x, {0.25*(\x)^3-1.25*(\x)^2+0.5*(\x)+2});
            \end{scope}
    \end{tikzpicture}}
    \loigiai{
        $g'(x)=f'(x^3-3x)\cdot (3x^2-3)-3x^2+3=3(x^2-1)\left[f'(x^3-3x)-1\right].\\
        \Rightarrow g'(x)=0\Leftrightarrow \hoac{&x^2=1\\&f'(x^2-3x)=1} \Leftrightarrow \hoac{&x=\pm1\\&x^3-3x=a\quad (-1<a<0)\\&x^3-3x=b\quad (0<b<2)\\&x^3-3x=c\quad (c>4).}$\\
        \begin{center}
            \begin{tikzpicture}[>=stealth]
                \draw[->,line width = 1pt] (-2,0)--(0,0) node[below left]{$O$}--(5,0) node[below]{$x$};
                \draw[->,line width = 1pt] (0,-2) --(0,3) node[right]{$y$};
                \draw (-1,0) node[below left]{$-1$} circle (1pt);
                \draw (0,2) node[above right]{$2$} circle (1pt);
                \draw (2,0) node[below left]{$2$} circle (1pt);
                \draw (4,0) node[below right]{$4$} circle (1pt);
                \draw [thick, domain=-1.3:4.6, samples=100] %
                plot (\x, {0.25*(\x)^3-1.25*(\x)^2+0.5*(\x)+2});
                \draw [thick, domain=-2:5, samples=100] %
                plot (\x, {0*(\x)+1});
                \draw (-1,1) node[above left]{$y=1$};
                \draw (1,1.8) node[right]{$y=f'(x)$};
                \draw (4,0) node[below right]{$4$} circle(1pt);
                \draw (4.323404276086477,1) node[below right] {$c$} circle(1pt);
                \draw (1.3579263675184994,1) node[below left] {$b$} circle(1pt);
                \draw (-0.6813306436049771,1) node[below right] {$a$} circle(1pt);
            \end{tikzpicture}
        \end{center}
        \begin{itemize}
            \item Phương trình $x^3-3x=a$ có $3$ nghiệm $x_1$, $x_2$, $x_3$ với $x_1<x_2<x_3$.
            \item Phương trình $x^3-3x=b$ có $3$ nghiệm $x_4$, $x_5$, $x_6$ với $x_4<x_5<x_6$.
            \item Phương trình $x^3-3x=c$ có $1$ nghiệm $x_7\quad(x_7>x_6)$.
        \end{itemize}
        \begin{center}
            \begin{tikzpicture}[>=stealth]
                \draw[->,line width = 1pt] (-3,0)--(0,0) node[below left]{$O$}--(3,0) node[below]{$x$};
                \draw[->,line width = 1pt] (0,-3) --(0,6) node[right]{$y$};
                \draw (-1,0) node[below left]{$-1$} circle (1pt);
                \draw (0,2) node[above right]{$2$} circle (1pt);
                \draw (1,0) node[above left]{$1$} circle (1pt);
                \draw (2,0) node[below right]{$2$} circle (1pt);
                \draw (0,-2) node[below left]{$-2$} circle (1pt);
                \draw [thick,color=red, domain=-2.1:2.3, samples=100] %
                plot (\x, {(\x)^3-3*(\x)});
                \draw [thick, domain=-3:3, samples=100] plot (\x, {0*(\x)+4.32});
                \draw [thick, domain=-3:3, samples=100] plot (\x, {0*(\x)+1.36});
                \draw [thick, domain=-3:3, samples=100] plot (\x, {0*(\x)-0.68});
                \draw (2.3,5) node[above right]{$y=x^3-3x$};
                \draw (-2,-0.7) node[below left]{$y=a$};
                \draw (-2,1.4) node[below left]{$y=b$};
                \draw (-2,4) node[left]{$y=c$};
                \draw[dashed](-1,0)--(-1,2)--(0,2);
                \draw[dashed](0,-2)--(1,-2)--(1,0);
                \draw (-1.84,-0.68) node[below right] {$x_1$};
                \draw (0.23,-0.68) node[below right] {$x_2$};
                \draw (1.61,-0.68) node[below right] {$x_3$};
                \draw (-1.43,1.36) node[above left] {$x_4$};
                \draw (-0.56,1.36) node[above right] {$x_5$};
                \draw (1.93,1.36) node[below right] {$x_6$};
                \draw (2.22,4.32) node[below right] {$x_7$};
            \end{tikzpicture}
        \end{center}
        Bảng xét dấu $g'(x)$
        \begin{center}
            \begin{tikzpicture}
                \tkzTabInit[nocadre=false,lgt=2.3,espcl=1.3]
                {$x$ /1.1,$x^2-1$ /0.7,$f'(x^3-3x)$ /0.7,$g'(x)$ /0.7,$g(x)$ /2}
                {$-\infty$,$x_1$,$x_4$,$-1$,$x_5$,$x_2$,$1$,$x_3$,$x_6$,$x_7$,$+\infty$}
                \tkzTabLine{,+,|,+,|,+,$0$,-,|,-,|,-,$0$,+,|,+,|,+,|,+,}
                \tkzTabLine{,-,$0$,+,$0$,-,|,-,$0$,+,$0$,-,|,-,$0$,+,$0$,-,$0$,+,}
                \tkzTabLine{,-,$0$,+,$0$,-,$0$,+,$0$,-,$0$,+,$0$,-,$0$,+,$0$,-,$0$,+,}
                \tkzTabVar{+/,-/,+/,-/,+/,-/,+/,-/,+/,-/,+/}
            \end{tikzpicture}
        \end{center}
        Dựa vào bảng xét dấu ta kết luận hàm số $y=g(x)$ có $5$ điểm cực tiểu.
    }
\end{ex}
\begin{ex}%[2D1G2-2]
    \immini{Cho hàm số $y=f(x)$ có đạo hàm và liên tục trên $\mathbb{R}$ và có đồ thị $y=f'(x)$ như hình vẽ. Hàm $y=f(x^2-2)-\dfrac{1}{2}x^4+\dfrac{3}{2}x^2$ có bao nhiêu điểm cực tiểu?
        \choice[2]
        {$4$}
        {$1$}
        {$2$}
        {\True$3$}}
    {\begin{tikzpicture}[>=stealth,font=\scriptsize,x=1.2cm]
            \begin{scope}[scale=.7]
                \def\mx{-1} \def\max{3}
                \def\my{-1} \def\may{3.5}
                \def\hamso(#1,#2){plot [samples=200,smooth,domain=#1:#2](\x,{
                        2*(\x)^4-5*(\x)^3+1.5*(\x)^2+2.25*(\x)
                    })}
                \draw[fill=black]
                (-0.5,0)circle (.7pt)node[shift={(-110:.5)}]{$-\dfrac{1}{2}$}
                (0.5,0)circle (.7pt)node[shift={(-90:.5)}]{$\dfrac{1}{2}$}
                (2,0)circle (.7pt)node[shift={(-90:.5)}]{$2$}
                (1.5,0)circle (.7pt)node[shift={(-90:.5)}]{$\dfrac{3}{2}$}
                (0,1)circle (.7pt)node[shift={(180:.3)}]{$1$}
                (0,2.5)circle (.7pt)node[shift={(180:.3)}]{$\dfrac{5}{2}$}
                (0.5,1)circle (.7pt)
                (2,2.5)circle (.7pt)
                ;
                \draw[dashed,thin] (0.5,0)|-(0,1)(2,0)|-(0,2.5);
                %===========================================
                \draw[->] (\mx,0)--(0,0) node [below right] {$O$}--(\max,0) node[below] {$x$};
                \draw[->] (0,\my)--(0,\may) node[left] {$y$};
                \clip (\mx,\my) rectangle (\max,\may);
                \draw \hamso(\mx,\max);
            \end{scope}
    \end{tikzpicture}}
    \loigiai{
        \immini{
            Ta có $y'=2xf'(x^2-2)-2x^3+3x=2x\left(f'(x^2-2)-x^2+\dfrac{3}{2}\right)$.\\
            $y'=0\Leftrightarrow \hoac{&x=0\\&f'(x^2-2)-x^2+\dfrac{3}{2}=0.\quad(*)}$\\
            Đặt $t=x^2-2$ ta có $(*)\Leftrightarrow f'(t)-t-\dfrac{1}{2}=0\Leftrightarrow f'(t)=t+\dfrac{1}{2}$.\\
            Dựa vào đồ thị hàm số bên ta có
            $$f'(t)=t+\dfrac{1}{2}\Leftrightarrow \hoac{&t=-\dfrac{1}{2}\\&t=\dfrac{1}{2}\\&t=2.}$$
        }{
            \begin{tikzpicture}[scale=1.2,>=stealth,font=\footnotesize,y=.8cm]
                \def\mx{-1} \def\max{3}
                \def\my{-1} \def\may{3.5}
                \def\hamso(#1,#2){plot [samples=200,smooth,domain=#1:#2](\x,{
                        (\x)+0.5
                    })}
                \def\ham(#1,#2){plot [samples=200,smooth,domain=#1:#2](\x,{
                        2*(\x)^4-5*(\x)^3+1.5*(\x)^2+2.25*(\x)
                    })}
                \draw[fill=black]
                (-0.5,0)circle (.7pt)node[shift={(-110:.5)}]{$-\dfrac{1}{2}$}
                (0.5,0)circle (.7pt)node[shift={(-90:.5)}]{$\dfrac{1}{2}$}
                (2,0)circle (.7pt)node[shift={(-90:.5)}]{$2$}
                (1.5,0)circle (.7pt)node[shift={(-90:.5)}]{$\dfrac{3}{2}$}
                (0,1)circle (.7pt)node[shift={(180:.3)}]{$1$}
                (0,2.5)circle (.7pt)node[shift={(180:.3)}]{$\dfrac{5}{2}$}
                (0.5,1)circle (.7pt)
                (2,2.5)circle (.7pt)
                ;
                \draw[dashed,thin] (0.5,0)|-(0,1)(2,0)|-(0,2.5);
                %===========================================
                \draw[->] (\mx,0)--(0,0) node [below right] {$O$}--(\max,0) node[below] {$t$};
                \draw[->] (0,\my)--(0,\may) node[left] {$y$};
                \clip (\mx,\my) rectangle (\max,\may);
                \draw \hamso(\mx,\max)\ham(\mx,\max);
        \end{tikzpicture}}
        \noindent
        Suy ra $\hoac{&x^2-2=-\dfrac{1}{2}\\&x^2-2=\dfrac{1}{2}\\&x^2-2=2}\Leftrightarrow\hoac{&x=\pm\dfrac{\sqrt{6}}{2}\\&x=\pm\dfrac{\sqrt{10}}{2} &\text{ (nghiệm kép)}\\&x=\pm 2.}$\\
        Bảng xét dấu
        \begin{center}
            \begin{tikzpicture}
                \tkzTabInit[nocadre=false, lgt=0.7,espcl=1.6,deltacl=0.5]
                {$x$/1.2, $y'$ /0.6}
                {$-\infty$ , $-2$ , $-\dfrac{\sqrt{10}}{2}$ ,$-\dfrac{\sqrt{6}}{2}$,$0$,$\dfrac{\sqrt{6}}{2}$,$\dfrac{\sqrt{10}}{2}$ ,$2$, $+\infty$}
                \tkzTabLine{ ,-,0,+, 0 ,+, 0 ,-,0,+,0,-,0,-,0,+ }
            \end{tikzpicture}
        \end{center}
        Suy ra hàm số có $3$ điểm cực tiểu.
    }
\end{ex}
\begin{ex}%[2D1G2-6]
    \immini{Cho hàm số $y=f(x)$ có bảng biến thiên bên dưới. Số điểm cực đại và số điểm cực tiểu của hàm số $y=f^2(2x)-2f(2x)+1$ lần lượt là
        \choice
        {\True $2$ và $3$}
        {$3$ và $2$}
        {$1$ và $1$}
        {$2$ và $2$}}{\begin{tikzpicture}
            \tkzTabInit[nocadre=false,lgt=1.2,espcl=2,deltacl=0.6]
            {$x$ /0.6,$f'(x)$ /0.6,$f(x)$ /2}
            {$-\infty$,$-1$,$2$,$+\infty$}
            \tkzTabLine{,-,$0$,+,$0$,-,}
            \tkzTabVar{+/$+\infty$,-/$0$,+/$3$,-/$-\infty$}
    \end{tikzpicture}}
    \loigiai{
        Đặt $g(x)=	f^2(2x)-2f(2x)+1=\left[f(2x)-1\right]^2$.\\
        $\Rightarrow g'(x)=2\cdot\left[f(2x)-1\right]\cdot f'(2x)$.\\
        $\Rightarrow g'(x)=0\Leftrightarrow \hoac{&f(2x)=1\\&f'(2x)=0.}$\\
        \begin{itemize}
            \item $f(2x)=1\Leftrightarrow \hoac{&2x=a\quad(a<-1)\\&2x=b\quad(-1<b<2)\\&2x=c\quad(2<c)}\Leftrightarrow \hoac{&x=\dfrac{a}{2}\quad(\dfrac{a}{2}<-\dfrac{1}{2})\\&x=\dfrac{b}{2}\quad(-\dfrac{1}{2}<\dfrac{b}{2}<1)\\&x=\dfrac{c}{2}\quad(1<\dfrac{c}{2}).}$
            \item $f'(2x)=0\Leftrightarrow\hoac{&2x=-1\\&2x=2}\Leftrightarrow\hoac{&x=-\dfrac{1}{2}\\&x=1.}$
        \end{itemize}
        Bảng biến thiên hàm số $y=g(x)$
        \begin{center}
            \begin{tikzpicture}
                \tkzTabInit[nocadre=false,lgt=2.1,espcl=2,deltacl=0.6]
                {$x$ /1.1,$f'(2x)$ /0.7,$f(2x)-1$ /0.7,$g'(x)$ /0.7,$g(x)$ /2}
                {$-\infty$,$\dfrac{a}{2}$,$-\dfrac{1}{2}$,$\dfrac{b}{2}$,$1$,$\dfrac{c}{2}$,$+\infty$}
                \tkzTabLine{,-,|,-,$0$,+,|,+,$0$,-,|,-,}
                \tkzTabLine{,+,$0$,-,|,-,$0$,+,|,+,$0$,-,}
                \tkzTabLine{,-,$0$,+,$0$,-,$0$,+,$0$,-,$0$,+,}
                \tkzTabVar{+/,-/,+/,-/,+/,-/,+/}
            \end{tikzpicture}
        \end{center}
        Dựa vào bảng biến thiên ta thấy hàm số $y=g(x)$ có $2$ điểm cực đại và $3$ điểm cực tiểu.
    }
\end{ex}

\begin{ex}%[2D1G2-6]
    Cho hàm số bậc ba $y=f(x)$ có đồ thị như hình bên. Có bao nhiêu giá trị nguyên của tham số $m$ để hàm số $y=\vert f^2(x)+2f(x)+m\vert$ có $9$ điểm cực trị?
    \choice
    {\True $24$}
    {Vô số}
    {$25$}
    {$23$}
    \loigiai{
        Đặt $y=g(x)=f^2(x)+2f(x)+m=\left[f(x)+1\right]^2+m-1.\\
        \Rightarrow g'(x)=2\left[f(x)+1\right]\cdot f'(x).\\
        \Rightarrow g'(x)=0\Leftrightarrow \hoac{&f'(x)=0\\&f(x)=-1}\Leftrightarrow \hoac{&x=1\\&x=3\\&x=a\quad(0<a<1)\\&x=b\quad(1<b<3)\\&x=c\quad(3<c).}$\\
        Từ đồ thị ta suy ra
        \begin{itemize}
            \item $f'(x)+1>0\Leftrightarrow f'(x)>-1\Leftrightarrow a<x<b \text{ hoặc } x>c$.
            \item $f'(x)+1<0\Leftrightarrow f'(x)<-1\Leftrightarrow x<a \text{ hoặc } b<x<c$.
        \end{itemize}
        Bảng biến thiên hàm số $y=g(x)$
        \begin{center}
            \begin{tikzpicture}
                \tkzTabInit[nocadre=false,lgt=2.3,espcl=1.8]
                {$x$ /1.1,$f'(x)$ /0.7,$f(x)+1$ /0.7,$g'(x)$ /0.7,$g(x)$ /2}
                {$-\infty$,$0$,$a$,$1$,$b$,$3$,$c$,$+\infty$}
                \tkzTabLine{,+,t,+,|,+,$0$,-,|,-,$0$,+,|,+,}
                \tkzTabLine{,-,t,-,$0$,+,|,+,$0$,-,|,-,$0$,+,}
                \tkzTabLine{,-,t,-,$0$,+,$0$,-,$0$,+,$0$,-,$0$,+,}
                \tkzTabVar{+/,R,-/$m-1$,+/$m+24$,-/$m-1$,+/$m$,-/$m-1$,+/}
            \end{tikzpicture}
        \end{center}
        Đồ thị hàm số $y=|g(x)|$ gồm có $2$ phần như sau:
        \begin{itemize}
            \item Phần 1: Trùng với đồ thị hàm số $y=g(x)$ với $g(x)\ge0$.
            \item Phần 2: Là phần đối xứng với phần đồ thị của hàm số $y=g(x)$ với $g(x)<0$ qua trục $\text{Ox}$.
        \end{itemize}
        Kết hợp với bảng biến thiên hàm số $y=g(x)$ ta suy ra hàm số $y=|g(x)|$ có $9$ điểm cực trị khi và chỉ khi $m\le 0<m+24 \Leftrightarrow -24<m\le0$. Mà $m$ là số nguyên nên ta được $24$ giá trị của $m$.
    }
\end{ex}
\begin{ex}%[2D1K2-6]
    Có bao giá trị nguyên của tham số $m$ thoả mãn $\vert m\vert<10$ sao cho hàm số $y=\vert x^3-(m-2)x^2-mx-m^2\vert$ có $3$ điểm cực tiểu?
    \choice
    {$9$}
    {$10$}
    {\True $8$}
    {$16$}
    \loigiai{
        Đặt hàm số $y=f(x)=x^3-(m-2)x^2-mx-m^2=(x-m)(x^2+2x+m)=(x-m)\left[x(x+2)+m\right]$.
        Suy ra $f'(x)=3x^2-2(m-2)x-m=0$ có $\Delta'=(m-2)^2+3m=m^2-m+4>0$ với mọi $m$.\\
        Theo định lí Vi-ét ta có $\heva{&x_1+x_2=\dfrac{2(m-2)}{3}\\&x_1x_2=-\dfrac{m}{3}}$.\\
        Hàm số $y=|f(x)|$ có $3$ điểm cực tiểu khi và chỉ khi $y(x_1)\cdot y(x_2)<0$.\\
        Thực hiện biến đổi\\
        $y(x_1)\cdot y(x_2)=  (x_1-m)(x_2-m)\left[x_1(x_1+2)+m\right]\left[x_2(x_2+2)+m\right]\\
        = (x_1-m)(x_2-m)\left[x_1x_2(x_1+2)(x_2+2)+m(x_1^2+x_2^2)+2m(x_1+x_2)+m^2\right]\\
        = \left[x_1x_2-m(x_1+x_2)+m^2\right]\left[x_1x_2\left(x_1x_2+2(x_1+x_2)+4\right)+m(x_1^2+x_2^2)+2m(x_1+x_2)+m^2\right]\\
        = \left(\dfrac{m^2}{3}+m\right)\left[-\dfrac{m}{3}\left(m+\dfrac{4}{3}\right)+m\left(\dfrac{4m^2}{9}-\dfrac{10m}{9}+\dfrac{16}{9}\right)+\dfrac{4m^2}{3}-\dfrac{8m}{3}+m^2\right]\\
        = \dfrac{2}{27}m^2(m+3)(2m^2+4m-5)$.\\
        Suy ra $y(x_1)\cdot y(x_2)<0\Leftrightarrow m^2(m+3)(2m^2+4m-5)<0\Leftrightarrow \hoac{&m<-3\\&\dfrac{-2-\sqrt{14}}{2}<m<0\\&0<m<\dfrac{-2+\sqrt{14}}{2}}$.\\
        Kết hợp với điều kiện $m$ là số nguyên thỏa $|m|<10$ ta được $m\in\{-9;-8;-7;-6;-5;-4;-2;-1\}$.\\
        Vậy có $8$ giá trị nguyên của tham số $m$.
    }
\end{ex}

\begin{ex}%[2D1G2-6]
    \immini{	Cho hàm số $f(x)=ax^4+bx^3+cx^2+dx+e, (ae<0)$. Đồ thì hàm số $y=f'(x)$ như hình bên dưới. Hàm số $y=\left|4f(x)-x^2\right|$ có bao nhiêu điểm cực tiểu?
        \choice[2]
        {$4$}
        {$5$}
        {\True $3$}
        {$2$}
    }{\begin{tikzpicture}[>=stealth,font=\scriptsize,x=1.3cm,y=1.5cm]
            \begin{scope}[scale=0.5]
                \draw[->] (-2,0) -- (3,0) node[below] { $x$};
                \draw[->] (0,-1) -- (0,3) node[left] {\ $y$};
                \draw (0,0) node[above left] { $O$} circle (1pt);
                \draw[smooth](0,0) parabola bend (-0.6,-1)(-1.7,1.7);
                \draw(0,0) parabola bend (1.3,2.5)(2,1);
                \draw [dashed] (0,1)--(2,1)--(2,0);
                \draw [dashed] (-1.05,0)--(-1.05,-0.5)--(0,-0.5);
                \draw (-1,0) node[above] { $-1$};
                \draw (0,-0.5) node[right] { $-\dfrac{1}{2}$};
                \draw (0,1) node[left] { $1$};
                \draw (2,0) node[below] { $2$};
            \end{scope}
    \end{tikzpicture}}
    \loigiai{
        Ta có $f'(x)=4ax^3+3bx^2+2cx+d$. Từ đồ thị hàm số $f'(x)$ suy ra $a<0$, do đó $e>0$.\\
        Đặt $y=g(x)=4f(x)-x^2\Rightarrow g'(x)	=4f'(x)-2x=4\left[f'(x)-\dfrac{x}{2}\right]$.\\
        \begin{center}
            \begin{tikzpicture}[>=stealth,x=1.0cm,y=1.0cm,thick, scale=1.0]
                \draw[->] (-2,0) -- (3,0) node[below] {\footnotesize $x$};
                \draw[->] (0,-1) -- (0,3) node[left] {\footnotesize $y$};
                \draw (0,0) node[below left] {\footnotesize $O$} circle (1pt);
                \draw[smooth](0,0) parabola bend (-0.6,-1)(-1.7,1.7);
                \draw(0,0) parabola bend (1.3,2.5)(2,1);
                \draw [dashed] (0,1)--(2,1)--(2,0);
                \draw [dashed] (-1.05,0)--(-1.05,-0.5)--(0,-0.5);
                \draw (-1,0) node[above] {\footnotesize $-1$};
                \draw (0,-0.5) node[right] {\footnotesize $-\dfrac{1}{2}$};
                \draw (0,1) node[left] {\footnotesize $1$};
                \draw (2,0) node[below] {\footnotesize $2$};
                \draw [thick, domain=-2.1:2.5, samples=100] %
                plot (\x, {0.5*(\x)});
            \end{tikzpicture}
        \end{center}
        Suy ra $g'(x)=0\Leftrightarrow f'(x)-\dfrac{x}{2}=0\Leftrightarrow f'(x)=\dfrac{x}{2}\Leftrightarrow \hoac{&x=-1\\&x=0\\&x=2}$.\\
        Bảng biến thiên
        \begin{center}
            \begin{tikzpicture}
                \tkzTabInit[nocadre=false,lgt=1.2,espcl=2.3]
                {$x$ /0.6,$g'(x)$ /0.6,$g(x)$ /2}
                {$-\infty$,$-1$,$0$,$2$,$+\infty$}
                \tkzTabLine{,+,$0$,-,$0$,+,$0$,-,}
                \tkzTabVar{-/,+/,-/$4e$,+/,-/}
            \end{tikzpicture}
        \end{center}
        Vì $4e>0$ nên từ bảng biến thiên hàm số $g(x)$ ta suy ra hàm số $y=\left|g(x)\right|$ có $3$ điểm cực tiểu.
    }
\end{ex}

\begin{ex}%[2D1G2-6]
    \immini{	Cho hàm số bậc bốn $f(x)$ có $f(0)=-1$. Hàm số $y=f'(x)$ có đồ thị là hình bên. Số điểm cực trị của hàm số $y=\vert 4f(x+1)+x^2+2x\vert$ là
        \choice[2]
        {$3$}
        {\True $5$}
        {$4$}
        {$6$}}{\begin{tikzpicture}[>=stealth,font=\scriptsize,y=.8cm]
            \begin{scope}[scale=.5]
                \draw[->] (-3.5,0) -- (5,0) node[below] {$x$};
                \draw[->] (0,-4) -- (0,2) node[left] { $y$};
                \draw (0,0) node[below left] {$O$} circle (1pt);
                \draw (-3.6,-3.6) ..controls +(60:0.2) and +(-180:1.3).. (-1.5,1.2) ..controls +(0:1.6) and +(-180:1.6) .. (2.8,-3.4)..controls +(0:0.5) and +(-100:4.5) .. (4.95,1.8);
                \draw [dashed] (-2,0)--(-2,1)--(0,1);
                \draw [dashed] (0,-2)--(4,-2)--(4,0);
                \draw (-2,0) node[below] { $-2$};
                \draw (0,-2) node[left] { $-2$};
                \draw (0,1) node[right] { $1$};
                \draw (4,0) node[above] {$4$};
            \end{scope}
    \end{tikzpicture}}
    \loigiai{
        Đặt $y=g(x)=4f(x+1)+x^2+2x\Rightarrow g'(x)=4f'(x+1)+2x+2=4\left[f'(x+1)+\dfrac{x+1}{2}\right]$.\\
        Suy ra $g'(x)=0\Leftrightarrow f'(x+1)=-\dfrac{x+1}{2}$.\\
        Đặt $t=x+1$ thì phương trình trở thành $f'(t)=-\dfrac{t}{2}$. Nghiệm của phương trình này là hoành độ giao điểm của đồ thị hàm số $y=f'(t)$ và $y=-\dfrac{t}{2}$.
        \begin{center}
            \begin{tikzpicture}[>=stealth,x=1.0cm,y=1.0cm,thick, scale=0.8]
                \draw[->] (-3.5,0) -- (5,0) node[below] {\footnotesize $t$};
                \draw[->] (0,-4) -- (0,2) node[left] {\footnotesize $y$};
                \draw (0,0) node[below left] {\footnotesize $O$} circle (1pt);
                \draw[very thick] (-3.6,-3.6) ..controls +(60:0.2) and +(-180:1.3).. (-1.5,1.2) ..controls +(0:1.6) and +(-180:1.6) .. (2.8,-3.4)..controls +(0:0.5) and +(-100:4.5) .. (4.95,1.8);
                \draw [thick, domain=-3.5:5, samples=100] %
                plot (\x, {-0.5*(\x)});
                \draw [dashed] (-2,0)--(-2,1)--(0,1);
                \draw [dashed] (0,-2)--(4,-2)--(4,0);
                \draw (-2,0) node[below] {\footnotesize $-2$};
                \draw (0,-2) node[left] {\footnotesize $-2$};
                \draw (0,1) node[right] {\footnotesize $1$};
                \draw (4,0) node[above] {\footnotesize $4$};
            \end{tikzpicture}
        \end{center}
        Do đó\\
        $$f'(t)=-\dfrac{t}{2}\Leftrightarrow\hoac{&t=-2\\&t=0\\&t=4}\Rightarrow \hoac{&x+1=-2\\&x+1=0\\&x+1=4}\Leftrightarrow \hoac{&x=-3\\&x=-1\\&x=3.}$$
        Bảng biến thiên
        \begin{center}
            \begin{tikzpicture}
                \tkzTabInit[nocadre=false,lgt=1.2,espcl=2.3]
                {$x$ /0.6,$g'(x)$ /0.6,$g(x)$ /2}
                {$-\infty$,$-3$,$-1$,$3$,$+\infty$}
                \tkzTabLine{,-,$0$,+,$0$,-,$0$,+,}
                \tkzTabVar{+/,-/,+/$-5$,-/,+/}
            \end{tikzpicture}
        \end{center}
        Từ bảng biến thiên suy ra hàm số $y=g(x)$ có $3$ cực trị âm, do đó hàm số $y=\left|g(x)\right|$ có $5$ điểm cực trị.
    }
\end{ex}
\begin{ex}%[2D1Y2-6]
    \immini{	Cho hàm số $y=f(x)$ có bảng biến thiên như hình vẽ. Hàm số $y=f\left(|x|\right)$ đạt cực đại tại.
        \choice[2]
        {$x=-1$}
        {\True $x=0$}
        {$x=2$}
        {$x=-2$}}{\begin{tikzpicture}[>=stealth,scale=0.9]
            \tkzTabInit[nocadre=false,lgt=1,espcl=2.6,deltacl=0.5]{$x$/.7 ,$y'$/.7,$y$/2}
            {$-\infty$ , $-1$ , $2$ , $+\infty$}
            \tkzTabLine{ , + , $0$ , - , $0$ , + , }
            \tkzTabVar{-/$-\infty$ , +/$3$ , -/$1$ , +/$+\infty$}
    \end{tikzpicture}}
    \loigiai{Từ bảng biến thiên của hàm số $y=f(x)$ ta có bảng biến thiên của hàm số $y=f\left(|x|\right)$ như sau
        \begin{center}
            \begin{tikzpicture}[scale=0.9]
                \foreach \x/\texn in {0/x,2/-\infty,4/-2,6/0,8/2,10/+\infty} \path (\x,3.5)node{$\texn$};
                \foreach \x/\texn in
                {0/y',3/-,4/0,5/+,6/0,7/-,8/0,9/+} \path (\x,2.5)node{$\texn$};
                \foreach \x/\y/\texn in {0/1/y,
                    2/1.5/+\infty,4/0/1,6/1.0/f(0),8/0/1,10/1.5/+\infty}
                \path (\x,\y) node(\x){$\texn$};
                \foreach \x/\y/\texn in {2/4,4/6,6/8,8/10}
                \draw[-stealth] (\x)--(\y);
                \draw
                (-.5,3)--(10.5,3) (1,4)--(1,0)
                (-.5,2)--(10.5,2);
            \end{tikzpicture}
        \end{center}
        Từ bảng biến thiên ta thấy hàm số $y=f\left(|x|\right)$ đạt cực đại tại $x=0$.
    }
\end{ex}
%%=====Câu 70
\begin{ex}%[2D1Y2-6]
    \immini{	Cho hàm số $y=f(x)$ có bảng biến thiên như hình vẽ. Tổng các giá trị cực đại của hàm số $y=\left|f(x)\right|$ là
        \choice[2]
        {\True $9$}
        {$-3$}
        {$3$}
        {$7$}}{\begin{tikzpicture}[>=stealth]
            \tkzTabInit[nocadre=false,lgt=1,espcl=1.7,deltacl=0.5]{$x$/.7 ,$y'$/.7,$y$/2}
            {$-\infty$ , $-1$ , $0$ , $1$ , $+\infty$}
            \tkzTabLine{ , - , $0$ , + , $0$ , - , $0$ , + , }
            \tkzTabVar{+/$+\infty$ , -/$-2$, +/$3$ , -/$-4$ , +/$+\infty$}
    \end{tikzpicture}}
    \loigiai{Từ bảng biến thiên của hàm số $y=f(x)$ ta có bảng biến thiên của hàm số $y=\left|f(x)\right|$ như sau
        \begin{center}
            \begin{tikzpicture}[scale=0.8]
                \foreach \x/\texn in {0/x,2/-\infty,4/x_1,6/-1,8/x_2,10/0,12/x_3,14/1,16/x_4,18/+\infty} \path (\x,3.5)node{$\texn$};
                \foreach \x/\texn in
                {0/y',3/-,4/||,5/+,6/0,7/-,8/||,9/+,10/0,11/-,12/||,13/+,14/0,15/-,16/||,17/+} \path (\x,2.5)node{$\texn$};
                \foreach \x/\y/\texn in {0/1/y,
                    2/1.5/+\infty,4/0/0,6/0.5/2,8/0/0,10/0.7/3,12/0/0,14/1.0/4,16/0/0,18/1.5/+\infty}
                \path (\x,\y) node(\x){$\texn$};
                \foreach \x/\y/\texn in {2/4,4/6,6/8,8/10,10/12,12/14,14/16,16/18}
                \draw[-stealth] (\x)--(\y);
                \foreach \x in {4,8,12}\draw[dashed,red] (\x)--+(3.7,0);
                \draw[dashed,red] (1.2,0)--+(2.7,0) (16)--+(2.0,0) node[below=-.1]{\small $y=0$};
                \draw
                (-.5,3)--(18.5,3) (1,4)--(1,0)
                (-.5,2)--(18.5,2);
            \end{tikzpicture}
        \end{center}
        Từ bảng biến thiên ta thấy hàm số $y=\left|f(x)\right|$ có $3$ giá trị cực đại lần lượt là $2$, $3$, $4$.\\
        Tổng các giá trị cực đại là $9$.
    }
\end{ex}
\begin{ex}%[2D1Y2-6]
    Cho hàm số $y=f(x)$ có đạo hàm $y=f'(x)=(x-1)(x-2)^4(x^2-4)$. Số điểm cực trị của hàm số $y=f(|x|)$ là
    \choice
    {$3$}
    {$2$}
    {$4$}
    {\True $5$}
\end{ex}
\begin{ex}%[2D1Y2-6]
    Cho hàm số $y=f(x)$ có đạo hàm $y=f'(x)=(x^3-2x^2)(x^3-2x)$ trên $\mathbb{R}$. Hàm số $y=|f(4-2021x)|$ có nhiều nhất bao nhiêu điểm cực trị?
    \choice
    {\True$9$}
    {$11$}
    {$2021$}
    { $5$}
\end{ex}
\begin{ex}%[2D1B2-6]
    Có bao nhiêu giá trị nguyên của tham số $m$ để hàm số $y=|3x^4-4x^3-12x^2+m|$ có $7$ điểm cực trị?
    \choice
    {$3$}
    {$5$}
    {$6$}
    {\True $4$}
    \loigiai{
        Đặt $f(x)=3x^4-4x^3-12x^2+m$ $\Rightarrow f'(x)=12x^3-12x^2-24x=0 \Rightarrow x=0; x=-1; x=2$.\\
        Qua BBT của $y=f(x)$ ta suy ra $y=|f(x)|$ có $7$ điểm cực trị $\Rightarrow \heva{&m>0\\&m-5<0} \Rightarrow 0<m<5$. Vậy có $4$ giá trị nguyên $m$ thỏa yêu cầu bài toán.
    }
\end{ex}
\begin{ex}%[2D1B2-6]
    Tìm các giá trị của $m$ để hàm số $f(x)=|x^3+3x^2+m-3|$ có ba điểm cực trị.
    \choice
    {$m=3; m=-1$}
    {$m\ge 1; m \le-3$}
    {$1\le m \le 3$}
    {\True $m\ge 3; m \le -1$}
\end{ex}
\begin{ex}%[2D1B2-6]
    Cho hàm số $y=f(x)=x^3-3mx^2+3(m^2-4)x+1$, có bao nhiêu số nguyên $m \in (-10;10)$  để hàm số $y=f(|x|)$ có đúng $5$ điểm cực trị.
    \choice
    {$3$}
    {$6$}
    {$8$}
    {\True $7$}
    \loigiai{
        $y=f(|x|)$ có đúng $5$ điểm cực trị $\Rightarrow y=f(x)$ có hai điểm cực trị dương.\\
        $f'(x)=3x^2-6mx+3(m^2-4)=0 \Rightarrow x=m-2; x=m+2$ có hai nghiệm dương $\Leftrightarrow m-2>0 \Leftrightarrow m >2$.\\ Vậy có $7$ giá trị $m$ thỏa yêu cầu bài toán.
    }
\end{ex}
\begin{ex}%[2D1G2-6]
    Cho hàm số $f(x)=\dfrac{1}{3}x^3-(2m-1)x^2+(8-m)x+2020$ với $m$ là tham số. Tập hợp tất cả các giá trị của tham số $m$ để hàm số $y=f\left(\vert x\vert\right)$ có điểm $5$ cực trị là khoảng $(a;b)$. Tích $a\cdot b$ bằng
    \choice
    {$12$}
    {$16$}
    {$10$}
    {\True $14$}
    \loigiai{
        Tập xác định $\mathscr{D}=\mathbb{R}$.\\
        Ta có $f\left(|-x|\right)=f\left(|x|\right)$, $\forall x\in\mathbb{R}$ nên $y=f\left(|x|\right)$ là hàm số chẵn. \\
        Do đó, đồ thị hàm số $y=f\left(|x|\right)$ đối xứng qua trục tung.\\
        Suy ra hàm số $y=f\left(|x|\right)$ luôn có một điểm cực trị là $x=0$.\\
        Do đó, $y=f\left(|x|\right)$ có $5$ điểm cực trị $\Leftrightarrow$ hàm số $y=f(x)$ có $2$ điểm cực trị dương.\\
        \phantom{Do đó, $y=f\left(|x|\right)$ có $5$ điểm cực trị} $\Leftrightarrow$  $f'(x)=0$ có hai nghiệm dương phân biệt.\\
        Ta có $f'(x)=x^2-2(m-1)x+8-m$.\\
        Yêu cầu bài toán $\Leftrightarrow\heva{&\Delta'>0 \\ &S>0 \\ &P>0}\Leftrightarrow\heva{&4m^2-3m-7>0 \\ &2m-1>0 \\ &8-m>0}\Leftrightarrow\heva{&m<-1\;\text{hoặc}\;m>\dfrac{7}{4} \\ &m>\dfrac{1}{2} \\ &m<8}\Leftrightarrow \dfrac{7}{4}<m<8$.
        Suy ra $a\cdot b=14$.
    }
\end{ex}
\begin{ex}%[2D1G2-6]
    \immini{	Cho hàm số $f(x)$ có đạo hàm liên tục trên $\mathbb{R}$ và đồ thị hàm số $f'(x)$ như hình vẽ. Hàm số $y=f\left(x^2-2\vert x\vert\right)$ có bao nhiêu điểm cực tiểu?
        \choice[2]
        {$1$}
        {\True $2$}
        {$5$}
        {$3$}}{\begin{tikzpicture}[>=stealth,font=\scriptsize]
            \draw[->] (-2,0) -- (2,0) node[below] {$x$};
            \draw[->] (0,-1) -- (0,2) node[left] {$y$};
            \draw (0,0) node[below right] {$O$} circle(1pt) (-1,0) node[above left]{ $-1$} (1,0) node[below]{ $1$} (0,1) node[right]{ $1$} (-1/3,0) node[below]{\ $-\dfrac{1}{3}$};
            \draw [dashed] (-1/3,0)--(-1/3,1.2);
            \draw plot[smooth,tension=.65] coordinates{(-1.2,-0.5) (-1/3,1.2) (1,0) (1.8,1.5)};
    \end{tikzpicture}}
    \loigiai{
        Đặt $g(x)=f(x^2-2x)\Rightarrow g'(x)=2(x-1)f'(x^2-2x).\\
        g'(x)=0\Leftrightarrow \hoac{&x=1\\&f'(x^2-2x)=0}\Leftrightarrow \hoac{&x=1\\&x^2-2x=-1\\&x^2-2x=1}\Leftrightarrow \hoac{&x=1\text{ (bội 3)}\\&x=1-\sqrt{2}\\&x=1+\sqrt{2}.}$\\
        Ta có
        \begin{itemize}
            \item $f'(x)>0\Leftrightarrow \heva{&x>-1\\&x\neq1}$ nên $f'(x^2-2x)>0\Leftrightarrow \heva{&x^2-2x>-1\\&x^2-2x\neq -1}\Leftrightarrow \heva{&x\neq 1\\&x=1\pm\sqrt{2}.}$
            \item $f'(x)<0\Leftrightarrow x<-1$ nên $f'(x^2-2x)<0\Leftrightarrow x^2-2x<-1 \text{ (Vô nghiệm)}.$
        \end{itemize}
        Bảng biến thiên hàm số $y=g(x)$
        \begin{center}
            \begin{tikzpicture}
                \tkzTabInit[nocadre=false,lgt=2.5,espcl=2.1,deltacl=0.6]
                {$x$ /0.7,$x-1$ /0.7,$f'(x^2-2x)$ /0.7,$g'(x)$ /0.7,$g(x)$ /2}
                {$-\infty$,$1-\sqrt{2}$,$0$,$1$,$1+\sqrt{2}$,$+\infty$}
                \tkzTabLine{,-,|,-,t,-,$0$,+,|,+,}
                \tkzTabLine{,+,$0$,+,t,+,$0$,+,$0$,+,}
                \tkzTabLine{,-,$0$,-,t,-,$0$,+,$0$,+,}
                \tkzTabVar{+/,R,R,-/,R,+/}
            \end{tikzpicture}
        \end{center}
        Do hàm số $y=f(x^2-2\vert x\vert)$ là hàm số chẵn nên từ bảng biến thiên trên ta suy ra đồ thị hàm số $y=f(x^2-2\vert x\vert)$ gồm hai nhánh như sau
        \begin{itemize}
            \item Nhánh thứ nhất là phần đồ thị hàm số $y=g(x)$ với $x\ge 0$.
            \item Nhánh thứ hai là phần đối xứng với nhánh thức nhất qua trục $Oy$
        \end{itemize}
        Do đó hàm số $y=f(x^2-2\vert x\vert)$ có $2$ điểm cực tiểu.

    }
\end{ex}
\begin{ex}%[2D1G2-6]
    \immini{	Cho hàm bậc bốn $y=f(x)$ có đồ thị như hình vẽ dưới đây. Số điểm cực trị của hàm số $g(x)=f\left(\vert x\vert^3-3\vert x\vert\right)$ là
        \choice[2]
        {$5$}
        {$3$}
        {\True $7$}
        {$11$}}{\begin{tikzpicture}[>=stealth,font=\scriptsize,y=.6cm,x=.7cm]
            \begin{scope}[scale=.5]
                \draw[->] (-5,0) -- (5,0) node[below] {\footnotesize $x$};
                \draw[->] (0,-4.5) -- (0,4) node[left] {\footnotesize $y$};
                \draw (0,0) node[below left] {\footnotesize $O$} circle (1pt);
                \draw[smooth](-1.1,0) parabola bend (-2,-2)(-4,4);
                \draw(-1.1,0) parabola bend (0,2)(1.5,-2.1);
                \draw[smooth](1.5,-2.1) parabola bend (2.6,-4)(4,4);
                \draw (1.5,0) node[above] {\footnotesize $2$};
                \draw (-1.1,0) node[above left] {\footnotesize $-2$};
            \end{scope}
    \end{tikzpicture}}
    \loigiai{
        Đặt $g(x)=f(x^3-3x)\Rightarrow g'(x)=3(x^2-1)f'(x^3-3x)$.\\
        Suy ra $g'(x)=0\Leftrightarrow \hoac{&x^2=1\\&f'(x^3-3x)=0}\Leftrightarrow \hoac{&x=\pm1\\&x^3-3x=2\\&x^3-3x=-2\\&x^3-3x=a\quad(a<-2)\\&x^3-3x=b\quad(b>2).}$\\
        Ta có
        \begin{itemize}
            \item $x^3-3x=2\Leftrightarrow \hoac{&x=2\\&x=-1.}$
            \item $x^3-3x=-2\Leftrightarrow \hoac{&x=-2\\&x=1.}$
            \item $x^3-3x=a\Leftrightarrow x=m \text{ (với $m<-2$)}$.
            \item $x^3-3x=b\Leftrightarrow x=n \text{ (với $n>2$)}$.
        \end{itemize}
        Từ đồ thị hàm số $f'(x)$ ta có $f'(x)>0\Leftrightarrow \hoac{&x<a\\&-2<x<2\\&x>b.}$\\
        Suy ra $h'(x)>0\Leftrightarrow \hoac{&x^3-3x<a\\&-2<x^3-3x<2\\&x^3-3x>b}\Leftrightarrow \hoac{&x<m\\&-2<x<-1\\&-1<x<2\\&x>n.}$\\
        Bảng biến thiên hàm số $y=g(x)$
        \begin{center}
            \begin{tikzpicture}
                \tkzTabInit[nocadre=false,lgt=2.5,espcl=1.7,deltacl=0.5]
                {$x$ /0.7,$x^2-1$ /0.7,$f'(x^3-3x)$ /0.7,$g'(x)$ /0.7,$g(x)$ /2}
                {$-\infty$,$m$,$-2$,$-1$,$0$,$1$,$2$,$n$,$+\infty$}
                \tkzTabLine{,+,|,+,|,+,$0$,-,t,-,$0$,+,|,+,|,+,}
                \tkzTabLine{,+,|,-,|,+,$0$,+,t,+,$0$,+,|,-,|,+,}
                \tkzTabLine{,+,|,-,|,+,$0$,-,t,-,$0$,+,|,-,|,+,}
                \tkzTabVar{-/,+/,-/,+/,R,-/,+/,-/,+/}
            \end{tikzpicture}
        \end{center}
        Từ bảng biến thiên suy ra hàm số $y=g(x)$ có $3$ điểm cực trị ứng với $x>0$ nên hàm số $y=f(|x|^3-3|x|)$ có $7$ điểm cực trị.

    }
\end{ex}
\begin{ex}%[2D1K2-2]
    \immini{
        Hình vẽ dưới đây là đồ thị của hàm số $y=f(x)$.
        Có bao nhiêu giá trị nguyên dương của tham số $m$ để hàm số $y=\left|f(x+1)+m\right|$ có $5$ cực trị?
        \choice
        {$0$}
        {\True $3$}
        {$2$}
        {$1$}
    }{
        \begin{tikzpicture}[>=stealth, font=\footnotesize, line join=round, line cap=round,y=.7cm]
            \begin{scope}[scale=.5]
                \def\xmin{-4} \def\xmax{3}
                \def\ymin{-5.5} \def\ymax{4}
                %\draw[color=gray!50,dashed] (\xmin,\ymin) grid (\xmax,\ymax);
                \draw[->] (\xmin,0)--(\xmax,0) node [below]{$x$};
                \draw[->] (0,\ymin)--(0,\ymax) node [left]{$y$};
                \node at (0,0) [below right]{$O$};
                \clip (\xmin+0.1,\ymin+0.1) rectangle (\xmax-0.5,\ymax-0.1);
                \draw[smooth,samples=300,domain=-3.5:0] plot(\x,{-1.24*(\x)^3-5.74*(\x)^2-5.78*(\x)});
                \draw[smooth,samples=300,domain=0:2.3] plot(\x,{1.78*(\x)^3-1.61*(\x)^2-4.57*(\x)-0.04});
                \draw[dashed](-2.5,0)|-(0,-2.07)  (-0.7,0)|-(0,1.7) (1.3,0)|-(0,-4.8);
                \draw[fill=black](0,-2.07)node[below left]{$-3$}circle(1pt)
                (0,-4.8)node[left]{$-6$}circle(1pt)
                (0,1.7)node[right]{$2$}circle(1pt);
            \end{scope}
        \end{tikzpicture}
    }
    \loigiai{
        Nhận xét
        \begin{itemize}
            \item  Hàm số $y=\left|f(x)-\alpha\right|$ có số điểm cực trị bằng số cực trị của hàm $y=f(x)$ và số giao điểm của đồ thị hàm $y=f(x)$ với đường thẳng $y=\alpha$ (không tính giao điểm là các điểm cực trị).
            \item  Số điểm cực trị của hàm $y=f(x)$ bằng số điểm cực trị của hàm $y=f(x+a)$.
        \end{itemize}
        Từ nhận xét trên ta có: Hàm số $y=f(x+1)$ có $3$ cực trị.\\
        Vậy ta cần đường thẳng $y=-m$ cắt đồ thị hàm số $y=f(x+1)$ tại 2 điểm khác cực trị.\\
        Từ đồ thị ta suy ra: $\hoac{&-6 <-m\leq-3\\&-m\geq 2}\Leftrightarrow\hoac{&3\leq m<6\\&m\leq-2.}$ \\
        Do $m\in\mathbb{N}^*$ nên $m\in\{3,4,5\}$.
    }
\end{ex}
\Closesolutionfile{ans}
%% \indapan{10}{ans/2D1-2-DANG-3}
%%Bài 2. Max-min
% \setcounter{section}{1}
\section{GIÁ TRỊ LỚN NHẤT - NHỎ NHẤT CỦA HÀM SỐ}

\subsection{LÝ THUYẾT CẦN NHỚ}
Cho hàm số $y=f(x)$ xác định trên tập $\mathscr{D}$. Ta có
\immini{\begin{itemize}
		\item[\ding{172}] $M$ là giá trị lớn nhất của hàm số nếu $\heva{&f(x) \le M,\forall x \in \mathscr{D}\\& \exists x_0 \in \mathscr{D}: f(x_0)=M.}$\\
		Kí hiệu \fbox{$\displaystyle\max_{x \in \mathscr{D}}f(x)=M$}
		\vskip 0.5cm
		\item[\ding{173}] $n$ là giá trị nhỏ nhất của hàm số nếu $\heva{&f(x) \ge n,\forall x \in \mathscr{D}\\& \exists x_0 \in \mathscr{D}: f(x_0)=n.}$\\
		Kí hiệu \fbox{$\displaystyle\min_{x \in \mathscr{D}}f(x)=n$}
\end{itemize}
}{
\begin{tikzpicture}[smooth,samples=300,scale=0.7,>=stealth]
	\draw[->] (-1.5,0)--(4.8,0) node[below]{$x$};
	\draw[->] (0,-2)--(0,4) node[right]{$y$};
	\draw (0,0) node[above left]{$O$};
	\draw[line width = 1.2pt,domain=-1:4,blue] plot(\x,{0.5*((\x)^2-4*(\x)+1)});
	\draw[fill=black] (-1,0) circle(1.5pt) (-1,3) circle(2pt) (0,3) circle(1.5pt) (0,-1.5) circle(1.5pt) (2,0) circle(1.5pt) (2,-1.5) circle(2pt) (4,0) circle(1.5pt) (4,0.5) circle(1.5pt);
	\draw[dashed] (-1,0)node[below]{\small$a$}--(-1,3)--(0,3)node[right]{\small$f(a)$} (2,0)node[above]{\small$x_0$}--(2,-1.5)--(0,-1.5)node[left]{\small$f(x_0)$}
	(4,0)node[below]{\small$b$}--(4,0.5);
	\node[above] at (-1,3) {\small $y_{\max}$};
	\node[below] at (2,-1.5) {\small $y_{\min}$};
\end{tikzpicture}}
\begin{note}
	\begin{listEX}[1]
		\item [\ding{172}] Khi yêu cầu tìm max min của hàm số mà không nói rõ xét trên tập nào, thì ta hiểu là tìm max min trên miền xác định của hàm số đó.
		\item [\ding{173}] Để tìm max min của hàm số $y=f(x)$ trên miền $\mathscr{D}$, ta thường lập bảng biến thiên của hàm số $y=f(x)$ trên $\mathscr{D}$. Từ bảng biến thiên, ta kết luận:
		\begin{itemize}
			\item [$\bullet$] Điểm ở vị trí cao nhất $\longrightarrow$ Kết luận max;
			\item [$\bullet$] Điểm ở vị trí thấp nhất $\longrightarrow$ Kết luận min.
		\end{itemize}
		\item [\ding{174}] Để tìm max min của hàm số $y=f(x)$ trên đoạn $[a;b]$ (\textit{$f(x)$ liên tục trên đoạn $[a ; b]$ và có đạo hàm trên $(a ; b)$ (có thể trừ một số hữu hạn các điểm) và $f^{\prime}(x)=0$ chỉ tại một số hữu hạn các điểm trong $(a ; b)$}), thì ta có thể giải như sau:
		\begin{itemize}
			\item [$\bullet$] Giải $f'(x)=0$ tìm các nghiệm $x_0 \in (a;b)$; 
			\item [$\bullet$] Tìm các điểm $x_i\in (a;b)$ mà tại đó đạo hàm không xác định (nếu có).
			\item [$\bullet$] Tính toán $f(a)$, $f(x_0)$, $f(x_i)$, $f(b)$ \quad ($\star$)
			\item [$\bullet$]  Gọi $M$, $n$ lần lượt là số lớn nhất và số nhỏ nhất của các kết quả tính toán ở bước ($\star$) thì
			$$M=\displaystyle\max_{[a;b]}f(x); \quad n=\displaystyle\min_{[a;b]}f(x)$$
		\end{itemize}
	\item [\ding{175}] Ta có thể dùng các bất đẳng thức có sẵn để đánh giá biểu thức cần tìm max, min. 
	% \begin{itemize}
	% 	\item [$\bullet$] Bất đẳng thức Cauchy cho hai số không âm $a$, $b$:
	% 	$$a+b \ge 2\sqrt{ab}$$
	% 	Dấu "=" xảy ra khi $a=b$.
	% 	\item [$\bullet$]  Bất đẳng thức Cauchy cho ba số không âm $a$, $b$, $c$:
	% 	$$a+b +c\ge 3\sqrt[3]{abc}$$
	% 	Dấu "=" xảy ra khi $a=b=c$.
	% 	\item [$\bullet$]  Bất đẳng thức Cauchy cho $n$ số không âm $a_1$, $a_2$,..., $a_n$:
	% 	$$a_1+a_2 +...+a_n \ge n\sqrt[n]{a_1a_2...a_n}$$
	% 	Dấu "=" xảy ra khi $a_1=a_2=...=a_n$.
	% \end{itemize}
	\end{listEX}
\end{note}
% \newpage
\subsection{PHÂN LOẠI VÀ PHƯƠNG PHÁP GIẢI TOÁN}
\begin{dang} {Bài toán tìm max, min của hàm số $y=f(x)$ trên miền $\mathscr{D}$}
	\begin{enumerate}[\iconMT]
		\item \indam{Phương pháp giải:} 
		\begin{listEX}[1]
			\item [\ding{172}] Tính $y'$. Giải phương trình $y'=0$ tìm các nghiệm $x_i \in \mathscr{D}$ và tìm các điểm $x_j \in \mathscr{D}$ mà tại đó $y'$ không xác định.
			\item [\ding{173}] Lập bảng biến thiên của hàm số trên $\mathscr{D}$.
			\item [\ding{174}] Từ bảng biến thiên, kết luận:
			\begin{itemize}
				\item [$\bullet$] Điểm ở vị trí cao nhất $\longrightarrow$ Kết luận max;
				\item [$\bullet$] Điểm ở vị trí thấp nhất $\longrightarrow$ Kết luận min.
			\end{itemize}
		\end{listEX}
		\item \indam{Lưu ý:} Nếu $\mathscr{D}$ là đoạn $\left[a;b\right]$ và hàm số $y=f(x)$ liên tục trên đoạn $\left[a;b\right]$ thì ta có thể làm như sau:
		\begin{listEX}[1]
			\item [\ding{172}] Giải $f'(x)=0$ tìm các nghiệm $x_0 \in (a;b)$;
			\item [\ding{173}] Tìm các điểm $x_i\in (a;b)$ mà tại đó đạo hàm không xác định (nếu có).
			\item [\ding{174}] Tính toán $f(a)$, $f(x_0)$, $f(x_i)$, $f(b)$ \quad ($\star$)
			\item [\ding{175}] Gọi $M$, $n$ lần lượt là số lớn nhất và số nhỏ nhất của các kết quả tính toán ở bước ($\star$) thì
			$$M=\displaystyle\max_{[a;b]}f(x); \quad n=\displaystyle\min_{[a;b]}f(x)$$
		\end{listEX}
		\begin{note}
			\begin{itemize}
				\item[\iconCH] Nếu hàm số $y=f(x)$ đồng biến trên đoạn $\left[a;b\right]$ thì $\min\limits_{[a;b]} f(x)=f(a)$ và $\max\limits_{[a;b]} f(x)=f(b)$.
				\item[\iconCH]  Nếu hàm số $y=f(x)$ nghịch biến trên đoạn $\left[a;b\right]$ thì $\min\limits_{[a;b]} f(x)=f(b)$ và $\max\limits_{[a;b]} f(x)=f(a)$.
			\end{itemize}
		\end{note}
	\end{enumerate}
\end{dang}

\boxmini{BÀI TẬP TỰ LUẬN}
\begin{vd}
	Tìm giá trị lớn nhất và nhỏ nhất (nếu có) của hàm số sau trên đoạn đã chỉ ra.
	\begin{tasks}(2)
		\task $f(x)=-x^3+3x^2+10$ trên đoạn $[-3;1]$.
		\task $f(x)=\dfrac{x^3}{3}-2x^2+3x+1$ trên đoạn $[-3;2]$.
		\task $f(x) = - 2x^4 + 4x^2 + 3$ trên đoạn $\left[0;2\right]$
		\task $f(x)=\dfrac{2x+3}{x+1}$ trên đoạn $[0;4]$.
		\task $f(x)=x+\dfrac{4}{x}$ trên khoảng $(0;+\infty)$;
		\task $f(x)=3x+\dfrac{4}{x^2}$ trên $(0;+\infty)$.
		\task $f(x)=\dfrac{2x^2+4x+5}{x^2+1}$ trên $\mathbb{R}$.
		\task $f(x)=\sqrt{-x^2+2x}$ trên miền xác định.
	\end{tasks}
\loigiai{
\begin{enumerate}[a)]
	\item Hàm số liên tục trên $[-3;1]$. Ta có $f'(x)=-3x^2+6x$; $f'(x)=0 \,\Leftrightarrow \hoac{&x=0 \in [-3;1]\\&x=2 \notin [-3;1]}$.\\
	Ta có $f(-3)=64$, $f(0)=10$, $f(1)=12$. Suy ra, $\max\limits_{[-3;1]} f(x)=f(-3)=64$; $\min\limits_{[-3;1]} f(x)=f(0)=10$.
	
	\item Hàm số liên tục trên $[-3;2]$
	Ta có $f^{\prime}(x)=x^2-4x+3$; $f^{\prime}(x)=0\Leftrightarrow \heva{&x=1\\x=3\notin[-3;2]}$.\\
	$f(1)=\dfrac{7}{3}$, $f(3)=-35$, $f(2)=\dfrac{5}{3}$.\\
	Vậy 
	$\max\limits_{[-3;2]}f(x)=\dfrac{7}{3}$ và
	$\min\limits_{[-3;2]}f(x)=-35$.	
	\item Ta có $f'(x)=- 8x^3 + 8x =- 8x(x^2 - 1) =- 8x(x - 1)(x + 1)$.\\
	Xét $f(0) = 3$, $f(1) = 5$ và $f(2) =- 13$.\\
	Vậy 
	$\max\limits_{[0;2]}f(x)=5$ và
	$\min\limits_{[0;2]}f(x)=-13$.	
	\item Hàm số đã cho liên tục trên đoạn $[0;4]$.\\
	Ta có $y'=-\dfrac{1}{(x+1)^2} < 0$, $\forall x \in [0;4]$. Suy ra hàm số đã cho nghịch biến trên đoạn $[0;4]$.\\
	Vậy $\max\limits_{[0;4]} y = y(0) = 3$ và $\min\limits_{[0;4]} y = y(4) = \dfrac{11}{5}$.
	
	\item Xét hàm số $f(x)=x+\dfrac{4}{x}$ trên khoảng $(0;+\infty)$.\\
	Đạo hàm $f'(x)=1-\dfrac{4}{x^2}=\dfrac{x^2-4}{x^2}$.\\
	Cho $f'(x)=0 \Leftrightarrow x^2-4=0 \Leftrightarrow \hoac{&x=2\in(0;+\infty)\\&x=-2\notin(0;+\infty).}$\\
	Bảng biến thiên
	\begin{center}
		\begin{tikzpicture}[font=\footnotesize,thick,>=stealth]
			\tikzset{double style/.append style={double distance=1.75pt}}
			\tkzTabInit[nocadre=false,lgt=1.2,espcl=2.5,deltacl=0.6,lw=.5pt,color,colorL=green!50,colorV=green!50]
			{$x$ /0.6,$f'(x)$ /0.6,$f(x)$ /2}
			{$-\infty$,$-2$,$0$,$2$,$+\infty$}
			\tkzTabLine{,+,$0$,-,d,-,$0$,+,}
			\tkzTabVar{-/$-\infty$,+/$-4$,-D+/$-\infty$/$+\infty$,-/$4$,+/$+\infty$}
			%\draw[pattern={Lines[angle=60,distance=1.25mm]},pattern color=blue,thin] (N11)--(N31)--(N33)--(N13);
		\end{tikzpicture}
	\end{center}
	Căn cứ vào bảng biến thiên, ta có $\min\limits_{(0;+\infty)}f(x)=4$.
	
	\item Áp dụng bất đẳng thức Cauchy cho $3$ số dương, ta có
	$$y=\dfrac{3x}{2}+\dfrac{3x}{2}+\dfrac{4}{x^2} \geq 3\sqrt[3]{\dfrac{3x}{2}\cdot \dfrac{3x}{2}\cdot \dfrac{4}{x^2}}=3\sqrt[3]{9}.$$
	Đẳng thức xảy ra khi $\dfrac{3x}{2}=\dfrac{4}{x^2} \Leftrightarrow x=\dfrac{2}{\sqrt[3]{2}}=2\sqrt[3]{2}$.
	
	\item Tập xác định $\mathscr{D}= \mathbb{R}$.\\
	Ta có $y'= \dfrac{-4x^2-6x+4}{(x^2+1)^2}$, \; $y'=0 \Leftrightarrow -4x^2-6x+4=0\Leftrightarrow \hoac{&x=-2\\&x=\dfrac{1}{2}.}$
	\begin{center}
		\begin{tikzpicture}\tkzTabInit[nocadre=false,lgt=1.2,espcl=2.5,deltacl=0.6]
			{$x$ /1, $y'$ /0.6, $y$ /2.5}
			{$-\infty$,$-2$,$\dfrac{1}{2}$,$+\infty$}
			\tkzTabLine{,-,$0$,+,$0$,-,}
			\tkzTabVar{+/$2$,-/$1$,+/$6$,-/$2$}
		\end{tikzpicture}
	\end{center}
	Suy ra $M=6$ và $m=1$.
	
	\item Hàm số $f(x)=\sqrt{-x^2+2x}$ liên tục trên $[0;2]$.\\
	$f'(x)=\dfrac{1-x}{\sqrt{-x^2+2x}}$, $f'(x)=0 \Leftrightarrow x=1$.\\
	Ta có $f(0)=0$, $f(2)=0$, $f(1)=1$.\\
	Vậy $\displaystyle\max_{x\in [0;2]}f(x)=1$ và $\displaystyle\min_{x\in [0;2]}f(x)=0$.
\end{enumerate}}
\end{vd}
\dongcham{54}
\begin{vd}
	Tìm giá trị lớn nhất và nhỏ nhất của hàm số sau trên miền đã chỉ ra.
	\begin{tasks}(2)
		\task $y=x-\sin 2x$ trên đoạn $\left[-\dfrac{\pi}{2};\pi\right]$
		\task $y = \mathrm{e}^{x^3 - 3x + 3}$ trên đoạn $[0; 2]$
		\task $y=\mathrm{e}^{x}(x^{2}-3)$ trên đoạn $[-2;2]$
		\task $y=\dfrac{\ln^2x}{x}$ trên đoạn $\left[1;\mathrm{e}^5\right]$
	\end{tasks}
	\loigiai{
		\begin{enumerate}[a)]
			\item Ta có
			\begin{itemize}
				\item $y'=1-2\cos 2x$.
				\item $\heva{&x\in \left(-\dfrac{\pi}{2};\pi\right)\\& y'=0}\Leftrightarrow \heva{&x\in \left(-\dfrac{\pi}{2};\pi\right)\\&\cos 2x=\dfrac{1}{2}}\Leftrightarrow \heva{&x\in \left(-\dfrac{\pi}{2};\pi\right)\\&x=\pm \dfrac{\pi}{6}+k\pi} \Leftrightarrow \hoac {&x=\pm\dfrac{\pi}{6}\\& x=\dfrac{5\pi}{6}.}$
				\item $f\left(-\dfrac{\pi}{2}\right)=-\dfrac{\pi}{2}$,  $f(\pi)=\pi$,
				$f\left(-\dfrac{\pi}{6}\right)=-\dfrac{\pi}{6}+\dfrac{\sqrt{3}}{2}$,  $f\left(\dfrac{\pi}{6}\right)=\dfrac{\pi}{6}-\dfrac{\sqrt{3}}{2}$,  $f\left(\dfrac{5\pi}{2}\right)=\dfrac{5\pi}{6}+\dfrac{\sqrt{3}}{2}$.
			\end{itemize}
			Vậy giá trị lớn nhất và giá trị nhỏ nhất của hàm số $y=x-\sin 2x$ trên đoạn $\left[-\dfrac{\pi}{2};\pi\right]$ lần lượt là $\dfrac{5\pi+3\sqrt{3}}{6}$ và $-\dfrac{\pi}{2}$.
			\item Ta có $y' = (3x^2 - 3)\cdot \mathrm{e}^{x^3 - 3x + 3}$.\\
			$y' = 0 \Leftrightarrow 3x^2 - 3 = 0 \Leftrightarrow x = 1$ do $x \in [0;2]$.\\
			Khi đó $y(0) = \mathrm{e}^3$; $y(2) = \mathrm{e}^5$; $y(1) = \mathrm{e}$.
			Vậy $ \max \limits_{[0; 2]} y = \mathrm{e}^5 $ khi $x = 2$.
			\item Ta có $y'=\mathrm{e}^x(x^2+2x-3)=0\Leftrightarrow \hoac{&x=-3\\ &x=1}$.
			Xét các giá trị: $f(-2)=\mathrm{e}^{-2}$; $f(1)=-2\mathrm{e}$; $f(2)=\mathrm{e}^2$, từ đó suy ra $y_{\min}=-2\mathrm{e}$.
			\item $y'=\dfrac{2\ln x-\ln^2x}{x^2}$, $y'=0\Leftrightarrow \hoac{
				& \ln x=0 \\
				& \ln x=2 \\} \Leftrightarrow \hoac{
				& x=1 \\
				& x=\mathrm{e}^2. \\}$\\
			Tính $y(1)=0$, $y(\mathrm{e}^2)=\dfrac{4}{\mathrm{e}^2}\approx 0{,}54$, $y(\mathrm{e}^5)=\dfrac{9}{\mathrm{e}^5}\approx 0{,}16$.\\
			Vậy $\max\limits_{x \in \left[1;\mathrm{e}^5\right]}y=\dfrac{4}{\mathrm{e}^2}$
	\end{enumerate}}
\end{vd}
\dongcham{40}
\begin{vd}
	Tìm giá trị lớn nhất và nhỏ nhất (nếu có) của hàm số sau trên miền đã chỉ ra.
	\begin{tasks}(2)
		\task $f(x)=\dfrac{5\sin x+1}{\sin x+2}$ trên đoạn $\left[0;\dfrac{\pi}{6}\right]$.
		\task $ y=\cos^3x +2\sin^2x+\cos x$ trên miền xác định.
	\end{tasks}
	\loigiai{
		\begin{enumerate}[a)]
			\item Đặt $t=\sin x,\; x\in \left[0;\dfrac{\pi}{6}\right]\Rightarrow t \in \left[0;\dfrac{1}{2}\right]$.\\
			Ta được hàm số $y=g(t)=\dfrac{5t+1}{t+2}$.\\
			$g'(t)=\dfrac{9}{(t+2)^2}>0,\forall t \in \left[0;\dfrac{1}{2}\right]$.\\
			Vì $g(0)=\dfrac{1}{2}$, $g\left(\dfrac{1}{2}\right)=\dfrac{7}{5}$ nên $\min\limits_{\left[0;\tfrac{1}{2}\right]}g(t)=g(0)=\dfrac{1}{2}.$\\
			Vậy $\min\limits_{\left[0;\tfrac{\pi}{6}\right]}f(x)=\min\limits_{\left[0;\tfrac{1}{2}\right]}g(t)=\dfrac{1}{2}$
			\item Ta có $ y=\cos^3x +2\sin^2x+\cos x =\cos^3x +2(1-\cos^2x)+\cos x =\cos^3x-2\cos^2x+\cos x+2$.\\
			Đặt $ t=\cos x,\, t\in [-1;1] $. Ta được $ f(t)=t^3-2t^2 +t+2$.\\
			Ta có $ f'(t)=3t^2-4t+1;\,y'=0\Leftrightarrow \hoac{&t=1\in[-1;1]\\&t=\dfrac{1}{3}\in[-1;1].} $\\
			Mà $f\left(-1\right)=-2$, $ f\left(\dfrac{1}{3}\right)=\dfrac{58}{27} $, $f(1)=2$ nên $\max\limits_{x\in \mathbb{R}}y=\max\limits_{\left[-1;1\right]}f(t)=\dfrac{58}{27}$
			\item
			\item
	\end{enumerate}}
\end{vd}
\dongcham{43}
\boxmini{BÀI TẬP TRẮC NGHIỆM}
\ind{PHẦN I.} \inden{Câu trắc nghiệm nhiều phương án lựa chọn. Mỗi câu hỏi học sinh chỉ chọn một phương án.}\\
\setcounter{ex}{0}
\Opensolutionfile{ans}[ans/2D1-B2-d1-1]

\begin{ex}
	\immini
	{Hàm số $y=f(x)$ liên tục trên đoạn $[-1;3]$ và có bảng biến thiên như sau.\\
		Gọi $M$ là giá trị lớn nhất của hàm số $y=f(x)$ trên đoạn $[-1;3]$. Khẳng định nào sau đây là khẳng định đúng?
		\choice
		{\True $M=f(0)$}
		{$M=f(-1)$}
		{$M=f(3)$}
		{$M=f(2)$}
	}
	{\begin{tikzpicture}
			\tkzTabInit[nocadre=True,lgt=1.2,espcl=2]
			{$x$ /0.7,$y'$ /0.7,$y$ /2.1}
			{$-1$,$0$,$2$,$3$}
			\tkzTabLine{,+,$0$,-,$0$,+,}
			\tkzTabVar{-/$0$, +/$5$,-/$1$,+/$4$}
	\end{tikzpicture}}
	\loigiai
	{Dựa vào bảng biến thiên ta có $M=f(0)=5$.}
\end{ex} \dongcham{1}

\begin{ex}
	\immini{Cho hàm số $f(x)$ liên tục trên đoạn $[-1;5]$ và có đồ thị như hình vẽ bên. Gọi $M$ và $m$ lần lượt là giá trị lớn nhất và nhỏ nhất của hàm số đã cho trên $[-1;5]$. Giá trị của $M+m$ bằng
		\choice
		{$5$}
		{$6$}
		{$3$}
		{\True $1$}
	}{
		\begin{tikzpicture}[scale=0.65, font=\footnotesize, line join=round, line cap=round, >=stealth]
			%%Nhập giới hạn đồ thị và hàm số cần vẽ
			\def \xmin{-1.5}
			\def \xmax{6.3}
			\def \ymin{-2.8}
			\def \ymax{4}
			%%Tự động
			\draw[->] (\xmin,0)--(\xmax,0) node[below left] {$x$};
			\draw[->] (0,\ymin)--(0,\ymax) node[below left] {$y$};
			\draw[fill=black] (0,0) circle(1pt) node [below right] {$O$};
			%%Vẽ các điểm trên 2 hệ trục
			\foreach \x in {3,4,5}
			\draw[fill=black] (\x,0) circle(1pt) node [below] {$\x$};
			\foreach \x in {-1,2}
			\draw[fill=black] (\x,0) circle(1pt) node [above] {$\x$};
			\foreach \y in {-2,1,3}
			\draw[fill=black] (0,\y) circle(1pt) node [above right] {$\y$};
			\draw[dashed](-1,0)--(-1,-2)--(0,-2)--(2,-2)--(2,0) (5,0)--(5,1)--(0,1) (3,0)--(3,1) (4,0)--(4,3)--(0,3);
			%%Tự động
			\draw
			(-1.1,-2.7) to[out=80, in=-100] (-1,-2)
			..controls +(80:1.2) and +(180:.5)..(0,1)
			..controls +(0:.6) and +(180:0.7)..(2,-2)
			..controls +(0:0.4) and +(-100:1.2)..(2.8,0)
			to[in=80, out=-100] (3,1)
			..controls +(75:1.5) and +(180:0.3)..(4,3)
			..controls +(0:0.5) and +(-75:1)..(5,1)
			to[in=105, out=-75] (6,-2.7);
			\fill[black]
			(-1,-2) circle(1pt)
			(2,-2) circle(1pt)
			(3,1) circle(1pt)
			(4,3) circle(1pt)
			(5,1) circle(1pt)
			;
		\end{tikzpicture}
	}
	\loigiai{
		Dựa vào đồ thị, suy ra $m=\min\limits_{[-1;5]} f(x)=f(-1)=-2$, $M=\max\limits_{[-1;5]} f(x)=f(4)=3$.\\
		Do đó $M+m=3-2=1$.
	}
\end{ex} \dongcham{1}

\begin{ex}
	\immini{Cho hàm số $y=f(x)$ có đồ thị là đường cong ở hình bên. Tìm giá trị nhỏ nhất $m$ của hàm số $y=f(x)$ trên đoạn $[-1;1] $.
		\haicot
		{$m=2 $}
		{\True $m=-2 $}
		{$m=1 $}
		{$m=-1 $}}{\vspace{-0.5cm}
		\begin{tikzpicture}[smooth,samples=300,scale=0.68,>=stealth]
			\draw[->] (-2.3,0)--(2.3,0) node[below]{$x$};
			\draw[->] (0,-2.5)--(0,2.5) node[right]{$y$};
			\draw (0,0) node[above right]{$O$};
			\draw[line width = 1pt,domain=-2:2] plot(\x,{(\x)^(3)-3*(\x)});
			\draw[fill=black] (-1,2) circle(1.5pt) (1,-2) circle(1.5pt);
			\draw[dashed] (-1,0)node[below]{\small$-1$}--(-1,2)--(0,2)node[right]{\small$2$} (1,0)node[above]{\small$1$}--(1,-2)--(0,-2)node[left]{\small$-2$};
	\end{tikzpicture}}
	\loigiai{
		Dựa vào đồ thị ta có giá trị nhỏ nhất của hàm số trên đoạn $[-1;1] $ bằng $-2$.	
		
	}
	
\end{ex} \dongcham{1}

\begin{ex}
	Cho hàm số $y=f(x)$ có bảng biến thiên trên đoạn $[-2;3]$ như hình bên dưới.
	\begin{center}
		\begin{tikzpicture}[>=stealth,scale=1]
			\tkzTabInit[nocadre=false,lgt=1.2,espcl=2,deltacl=0.6]
			{$x$/0.6,$f’(x)$/0.6,$f(x)$/2}
			{$-\infty$,$-2$,$-1$,$1$,$3$,$+\infty$}
			\tkzTabLine{,h,,+,z,-,d,+,,h}
			\tkzTabVar{+H/,-/$0$,+/$1$,-/$-2$,+H/$5$}
		\end{tikzpicture}
	\end{center}
	Gọi $M$ và $m$ lần lượt là giá trị lớn nhất và giá trị nhỏ nhất của hàm số đã cho trên đoạn $[-1;3]$. Giá trị của biểu thức $M-m$ là
	\choice
	{$5$}
	{\True $7$}
	{$-1$}
	{$3$}
	\loigiai{
		Dựa vào bảng biến thiên ta thấy giá trị lớn nhất của hàm số là $M=5$ và giá trị nhỏ nhất của hàm số là $m=-2$ nên $M-m=7$.}
\end{ex}
% \newpage
\begin{ex}
	Giá trị lớn nhất và nhỏ nhất của hàm số $y=x^3-12x+1$ trên đoạn $[-2;3]$ lần lượt là
	\choice
	{\True $17$, $-15$}
	{$10$, $-26$}
	{$-15$, $17$}
	{$6$, $-26$}
	\loigiai{
		Ta có $y'=3x^2-12$, do đó $y'=0\Leftrightarrow 3x^2-12=0\Leftrightarrow x=\pm 2\in [-2;3]$.\\
		Mặt khác $f(-2)=17$, $f(2)=-15$, $f(3)=-8$.\\
		Vậy giá trị lớn nhất và nhỏ nhất cần tìm lần lượt là $17$, $-15$.
	}
\end{ex} \dongcham{12}

\begin{ex}
	Gọi $ M, m $ lần lượt là giá trị lớn nhất và giá trị nhỏ nhất của hàm số $ y = x^3 + 3x^2 - 9x + 1 $ trên $ [-4;4] $. Tính tổng $ M + m. $
	\choice 
	{$ 12 $}
	{$ 98 $}
	{$ 17 $}
	{\True $ 73 $}
	\loigiai
	{
		Ta có $ y' = 3x^2 + 6x - 9 = 0 \Leftrightarrow \hoac{&x = 1\\ &x = -3.} $\\
		Khi đó: $ y(-4) = 21 $,\, $ y(-3) = 28, $
		\, $ y(1) = -4, $
		\, $ y(4) = 77. $\\
		Do đó $ M + m = 77 + (-4) = 73. $
	}
\end{ex} \dongcham{12}

\begin{ex}
	Giá trị lớn nhất của hàm số $f(x)=-x^4+12x^2+1$ trên đoạn $\left[ -1;2\right] $ bằng
	\choice
	{\True $33$}
	{$37$}
	{$12$}
	{$1$}
	\loigiai{
		Hàm số $f(x)=-x^4+12x^2+1$ liên tục trên đoạn $\left[ -1;2\right] $.\\
		Ta có $f'(x)=-4x^3+24x=-4x(x^2-6)$.\\
		$f'(x)=0 \Leftrightarrow \hoac{& x=-\sqrt{6} \not \in \left[ -1;2\right] \\ &x=0 \in \left[ -1;2\right] \\&x=\sqrt{6} \not \in \left[ -1;2\right]. }$\\
		Ta có $f(-1)=12; f(0)=1; f(2)=33$.\\
		Vậy $\max\limits_{\left[ -1;2\right] } f(x)=33$.
	}
\end{ex} \dongcham{12}
\begin{ex}
	Giá trị lớn nhất của hàm số $y=x^4-3x^2+2$ trên đoạn $\left[ 0;3\right] $ bằng
	\choice
	{ $ 57 $}
	{\True $ 56 $}
	{$ 54$}
	{$ 55 $}
	\loigiai{
		Hàm số $y$ liên tục trên đoạn $\left[ 0;3\right] $ và có đạo hàm $y'=4x^3-6x$.\\
		Ta có $y'=0 \Leftrightarrow 4x^3-6x=0 \Leftrightarrow \hoac{& x=0 \in \left[ 0;3\right]  \\&x=\sqrt{\dfrac{3}{2}} \in \left[ 0;3\right]\\ & x=- \sqrt{\dfrac{3}{2}}\notin \left[ 0;3\right].}$\\
		Ta có $y(0)=2$, $y(3)=56$, $y\left(\sqrt{\dfrac{3}{2}}\right) =-\dfrac{1}{4} $.\\
		Do đó giá trị lớn nhất của hàm số $y=x^4-3x^2+2$ trên đoạn $\left[ 0;3\right] $ bằng $56$.
	}
\end{ex} \dongcham{7}

\begin{ex}%[2D1Y3-1]
	Giá trị nhỏ nhất của hàm số $y=\dfrac{x-1}{x+1}$ trên đoạn $[0;3]$ là
	\choice
	{$\min\limits_{[0;3]}y=\dfrac{1}{2}$}
	{$\min\limits_{[0;3]}y=-3$}
	{$\min\limits_{[0;3]}y=1$}
	{\True $\min\limits_{[0;3]}y=-1$}	
	\loigiai{
		Trên đoạn $[0;3]$ hàm số luôn xác định.\\
		Ta có $y'=\dfrac{2}{(x+1)^2}>0$, $\forall x \in [0;3]$ nên hàm số đã cho đồng biến trên đoạn $[0;3]$.\\
		Do đó $\min\limits_{[0;3]}y=y(0)=-1$.
	}	
\end{ex} \dongcham{7}

\begin{ex}%
	Giá trị nhỏ nhất của hàm số $y=\dfrac{2x+3}{x+1}$ trên đoạn $[0;4]$ là
	\choice
	{$2$}
	{$\dfrac{7}{5}$}
	{$3$}
	{\True $\dfrac{11}{5}$}
	\loigiai
	{
		Ta có $y'=\dfrac{-1}{(x+1)^2}<0$ nên $\min\limits_{[0;4]} y=y(4)=\dfrac{11}{5}$.
	}
\end{ex} \dongcham{7}

\begin{ex}%[2D1B3]
	Giá trị lớn nhất của hàm số $y=\dfrac{x^2-3x+3}{x-1}$ trên đoạn $\left[-2;\dfrac{1}{2}\right]$ bằng
	\choice
	{$4$}
	{\True $-3$}
	{$-\dfrac{7}{2}$}
	{$-\dfrac{13}{3}$}
	\loigiai{
		Ta có $y'=\dfrac{x^2-2x}{(x-1)^2}$. Xét $y'=0\Leftrightarrow  x^2-2x=0\Leftrightarrow \hoac{&x=0\in \left[-2;\dfrac{1}{2}\right]\\&x=2\notin\left[-2;\dfrac{1}{2}\right]}$.\\
		Ta có $y(0)=-3$, $y(-2)=\dfrac{-13}{3}$, $y\left(\dfrac{1}{2}\right)=\dfrac{-7}{2}$.\\
		Suy ra $\underset{x\in \left[-2;\dfrac{1}{2}\right]}{\max y}=-3$
	}
\end{ex} \dongcham{10}

\begin{ex}
	Giá trị lớn nhất của hàm số $y=\sqrt{4-x^2}$ là
	\choice
	{$M=-2$}
	{\True $M=2$ }
	{$M=4$}
	{$M=0$}
	\loigiai
	{
		Tập xác định: $\mathscr{D}=\left[-2;2\right]$.\\
		Đạo hàm $y'=\dfrac{-x}{\sqrt{4-x^{2}}}$; $y'=0 \Leftrightarrow x=0 \in \left[-2;2\right]$.\\
		Ta có $y(2)=y(-2)=0$; $y(0)=2$.\\
		Vậy giá trị lớn nhất của hàm số đã cho bằng $2$.
	}
\end{ex} \dongcham{8}

\begin{ex}%[2D1B3-1]
	Tìm giá trị lớn nhất $M$ của hàm số $y=\sqrt{7+6x-x^2}$.
	\choice
	{\True $M=4$}
	{$M=\sqrt{7}$}
	{$M=7$}
	{$M=3$}
	\loigiai{
		Tập xác định $\mathscr{D}=[-1;7]$.\\
		$y'=\dfrac{-x+3}{\sqrt{7+6x-x^2}}$.\\
		Cho $y'=0\Leftrightarrow x= 3\in \mathscr{D}$.\\
		Có $y(3)=4, y(-1)=0, y(7)=0$. Vậy $M=4$.	
	}
\end{ex} \dongcham{8}

\begin{ex}%[Nguyễn Quang Hiệp - Phát triển đề minh họa 2021]%[2D2B4-4]%
	Tính giá trị lớn nhất của hàm số $y=x-\ln x$ trên $\left[\dfrac{1}{2};\mathrm{e}\right]$.\\
	\choice
	{$\max\limits_{x \in \left[\frac{1}{2};\mathrm{e}\right]}y=1$}
	{\True $\max\limits_{x \in \left[\frac{1}{2};\mathrm{e}\right]}y=\mathrm{e}-1$}
	{$\max\limits_{x \in \left[\frac{1}{2};\mathrm{e}\right]}y=\mathrm{e}$}
	{$\max\limits_{x \in \left[\frac{1}{2};\mathrm{e}\right]}y=\dfrac{1}{2}+\ln 2$}
	\loigiai{
		Hàm số $y=x-\ln x$liên tục trên đoạn $\left[\dfrac{1}{2};\mathrm{e}\right]$.\\
		Ta có $y'=1-\dfrac{1}{x}\Rightarrow y'=0\Leftrightarrow x=1\in \left[\dfrac{1}{2};\mathrm{e}\right]$.\\
		Do $y\left(\dfrac{1}{2}\right)=\dfrac{1}{2}+\ln 2$; $y(\mathrm{e})=\mathrm{e}-1$; $y(1)=1$ nên $\max\limits_{x \in \left[\frac{1}{2};\mathrm{e}\right]}y=\mathrm{e}-1$.}
\end{ex} \dongcham{8}

\begin{ex}
	Gọi $M, N$ lần lượt là giá trị lớn nhất và nhỏ nhất của hàm số $y = x^2 - 4\ln (1 - x)$ trên đoạn $[-2;0]$. Tính $M - N$.
	\choice
	{$M - N = 4\ln 2$}
	{$M - N = -1$}
	{\True $M - N = 4\ln 2 -1$}
	{$M - N = 4\ln 3 -4$}
	\loigiai{
		Tập xác định: $\mathscr{D} = (-\infty;1)$.\\
		Ta có $y' = 2x + \dfrac{4}{1 - x} = \dfrac{-2x^2 + 2x + 4}{1 - x}$.\\
		Khi đó $y' = 0 \Leftrightarrow -2x^2 + 2x + 4 = 0 \Leftrightarrow \hoac{&x = -1 \quad \mbox{(nhận)} \\&x = 2 \quad \mbox{(loại)}. }$\\
		Khi đó $\heva{& y(-2) = 4 - 4\ln 3 \approx -0{,}4 \\& y(-1) = 1 - 4\ln 2  \approx -1{,}7\\& y(0) = 0.} \Rightarrow M = 0, N = 1 -4\ln 2$\\
		Vậy $M - N = 4\ln 2 -1$.
	}
\end{ex} \dongcham{8}

\begin{ex}
	Cho hàm số $f(x)$ nghịch biến trên $\mathbb{R}$. Giá trị nhỏ nhất của hàm số $g(x)=\mathrm{e}^{3x^2-2x^3}-f(x)$ trên đoạn $[0;1]$ bằng
	\choice
	{$\mathrm{e}-f(1)$}
	{$f(1)$}
	{$f(0)$}
	{\True $1-f(0)$}
	\loigiai{
		Ta có $g'(x)=(6x-6x^2)\mathrm{e}^{3x^2-2x^3}-f'(x)$.\\
		Trên đoạn $[0;1]$ thì $6x-6x^2\ge 0$, $f'(x)\le 0$ nên $g'(x)\ge 0$, suy ra hàm số $g(x)$ đồng biến, suy ra giá trị nhỏ nhất là $g(0)=1-f(0)$.
	}
\end{ex} \dongcham{8}

\begin{ex}
	\immini{Cho hàm số $y=f(x)$ xác định và liên tục trên đoạn $\left[0;\dfrac{7}{2}\right]$, có
		đồ thị của hàm số $y=f'(x)$ như hình vẽ. Hỏi hàm số $y=f(x)$ đạt giá trị nhỏ nhất trên đoạn $\left[0;\dfrac{7}{2}\right]$ tại điểm $x_0$ nào dưới đây?
		\choice
		{\True $x_0=3$}
		{$x_0=2$}
		{$x_0=1$}
		{$x_0=0$}
	}
	{\begin{tikzpicture}[scale=1,font=\footnotesize, line join=round,line cap=round,>=stealth]
			\draw [->] (-1,0)--(0,0)
			node[below left]{$O$}--(4.5,0)node[below]{$x$}; % Hệ trục tọa độ
			\draw[->] (0,-1.5) --(0,2) node[left]{$y$};
			\draw[dashed](3.5,0)node[below]{$\tfrac{7}{2}$}--(3.5,25/16);
			\draw(1,0)node[above]{$1$}(3,0)node[above left]{$3$};
			\draw [domain=0:3.5,samples=100] plot (\x, {(\x-1)^2*(\x-3)/2});
	\end{tikzpicture}}
	\loigiai{
		Từ đồ thị hàm số ta có $f'(x)=0 \Leftrightarrow \hoac{&x=1\\&x=3.}$\\
		Bảng biến thiên của hàm số $y=f(x)$ trên đoạn $\left[0;\dfrac{7}{2}\right]$
		\begin{center}
			\begin{tikzpicture}
				\tkzTabInit[nocadre=false,lgt=1.2,espcl=2.5,deltacl=0.7]{$x$ / 1.1 , $f’(x)$ /0.7, $f(x)$ / 2}
				{$0$,$1$, $3$ , $\dfrac{7}{2}$}%
				\tkzTabLine{,-,0,-,0,+,}%
				\tkzTabVar{+ /$f(0)$,R/,-/$f(3)$,+ / $f\left(\dfrac{7}{2}\right)$}%
				\tkzTabIma{1}{3}{2}{$f(0)$}
			\end{tikzpicture}
		\end{center}
		Từ bảng biến thiên ta có hàm số $y=f(x)$ đạt giá trị nhỏ nhất trên đoạn $\left[0;\dfrac{7}{2}\right]$ tại điểm $x_0=3.$
	}
\end{ex} \dongcham{10}

\begin{ex}%[2D1K3-1]
	\immini{Cho hàm số $y=f(x)$, biết hàm số $y=f'(x)$ có đồ thị như hình vẽ dưới đây. Hàm số $y=f(x)$ đạt giá trị nhỏ nhất trên đoạn $\left[\dfrac{1}{2};\dfrac{3}{2} \right]$ tại điểm nào sau đây?
		\choicew{0,25 \textwidth}
		\choice
		{$x=\dfrac{3}{2}$}
		{$x=\dfrac{1}{2}$}
		{\True $x=1$}
		{$x=0$}}{\vspace{-0.5cm}\begin{tikzpicture}[>=stealth,scale=1.5]
			\draw[->] (-0.5,0)--(2.5,0) node[below]{\footnotesize $x$};
			\draw[->] (0,-0.5)--(0,1.5) node[right]{\footnotesize $y$};
			\draw (0,0) node[below left]{\footnotesize $O$};
			\draw[line width = 1pt,smooth,domain=-0.4:1.7] plot({\x},{(\x)^2-\x});
			\draw[fill=black] (1.5,0.75) circle(1pt);
			\draw [dashed] (1.5,0)
			node[below]{\footnotesize$\dfrac{3}{2}$}--(1.5,0.75)--(0,0.75)(1,0)node[below]{$1$};
	\end{tikzpicture}}
	\loigiai{
		Dựa vào đồ thị hàm số $y=f'(x)$. Ta có bảng biến thiên
		\begin{center}\begin{tikzpicture}
				\tkzTabInit[nocadre=false,lgt=1,espcl=2.1]
				{$x$ /1,$y'$ /0.6,$y$ /2}
				{$\dfrac{1}{2}$,$1$,$\dfrac{3}{2}$}
				\tkzTabLine{,-,$0$,+,$0$,}
				\tkzTabVar{+/, -/,+/}
			\end{tikzpicture}
		\end{center}
		Suy ra hàm số đạt giá trị nhỏ nhất trên $\left[\dfrac{1}{2};\dfrac{3}{2} \right]$ tại $x=1$.}
\end{ex} \dongcham{10}

\begin{ex}
	\immini{ Cho hàm số $f(x)$ có đồ thị của hàm số $y=f'(x)$ như hình vẽ. Biết $f(0)+f(1)-2f(2)=f(4)-f(3)$. Giá trị nhỏ nhất $m$, giá trị lớn nhất $M$ của hàm số $f(x)$ trên đoạn $[0;4]$ là
		\choice
		{$m=f(4)$, $M=f(1)$}
		{\True $m=f(4)$, $M=f(2)$}
		{$m=f(1)$, $M=f(2)$}
		{$m=f(0)$, $M=f(2)$}
	}{
		\begin{tikzpicture}[scale=0.79, >=stealth]
			\draw[->] (-0.6,0.) -- (5.3,0.);
			\draw[->] (0.,-1.7) -- (0.,1.6);
			\draw[dashed] (4,0) -- (4,-1.2);
			\clip(-0.6,-1.7) rectangle (5.3,1.7);
			\draw[smooth,samples=100,domain=0:2] plot(\x,{-0.8*((\x)^2-2*(\x))});
			\draw[smooth,samples=100,domain=2:4.5] plot(\x,{0.2*(((\x)-2)*((\x)-7)});
			\draw (-0.3,-0.3) node {$O$} (5.2,0.3) node {$x$} (0.35,1.5) node {$y$} (1.9,-0.3) node {$2$} (4.0,0.3) node {$4$} (2.0,1.15) node {$y=f'(x)$};
			\fill (0,0) circle(1pt) (2,0) circle(1pt) (4,0) circle(1pt); 
		\end{tikzpicture}
	}
	\loigiai{
		Từ đồ thị hàm số $y=f'(x)$ ta suy ra $f'(x)=0 \Leftrightarrow \hoac{&x=0\\&x=2.}$\\
		Ta có bảng biến thiên: 
		\begin{center}\begin{tikzpicture}[>=stealth,scale=1]
				\tkzTabInit[lgt=1.2,espcl=2.5]
				{$x$/1,$f'(x)$/1,$f(x)$/2.5}
				{$0$,$2$,$4$}
				\tkzTabLine{$0$,+,$0$,- }
				\tkzTabVar{-/$f(0)$,+/$f(2)$,-/$f(4)$}
		\end{tikzpicture}\end{center}
		Từ bảng biến thiên ta thấy $M=f(2)$.\\
		Mặt khác, từ bảng biến thiên ta có $\heva{&f(1)<f(2)\\&f(3)<f(2)}\Rightarrow f(1)+f(3)<2f(2)$.\\
		Do đó $f(4)=f(0)+f(1)+f(3)-2f(2)<f(0)+f(2)+f(2)-2f(2)=f(0) \Rightarrow m=f(4)$.	
	}
\end{ex} \dongcham{10}


\begin{ex}
	Giá trị lớn nhất, giá trị nhỏ nhất của hàm số $y={\sin}^3x-3{\sin}^2x+2$ lần lượt là $M$, $m$. Tổng $M+m$ bằng
	\choice
	{\True $0$}
	{$4$}
	{$1$}
	{$3$}
	\loigiai{
		Đặt $t=\sin x \, (-1\le t\le 1)$. Ta có $y=f(t)=t^3-3t^2+2 \, (-1\le t\le 1)$.\\$y'=3t^2-6t=0\Leftrightarrow\hoac{&t=0\in \left[-1;1\right]\\&t=2\notin \left[-1;1\right].}$\\
		Ta có $f(-1)=-2,\,f(1)=0, \,f(0)=2$. Vậy $M=2$ và $m=-2\Rightarrow M+m=0$.}
\end{ex} \dongcham{11}

\begin{ex}
	Giá trị nhỏ nhất của hàm số $f(x)=(x+1)(x+2)(x+3)(x+4)+2019$ là
	\choice
	{$2017$}
	{$2020$}
	{\True $2018$}
	{$2019$}
	\loigiai{
		Tập xác định $\mathscr{D}=\mathbb{R}$.\\
		Biến đổi $f(x)=(x+1)(x+2)(x+3)(x+4)+2019=(x^2+5x+4)(x^2+5x+6)+2019$.\\
		Đặt $t=x^2+5x+4\Rightarrow t=\left( x+\dfrac{5}{2} \right)^2-\dfrac{9}{4}\Rightarrow t\ge-\dfrac{9}{4}\,\forall x\,\in \,\mathbb{R}$.\\
		Hàm số đã cho trở thành $f(x)=t^2+2t+2019=(t+1)^2+2018\ge 2018 \,\forall t\ge -\dfrac{9}{4}$.\\
		Vậy giá trị nhỏ nhất của hàm số đã cho bằng $2018$ tại $t=-1\in \left[-\dfrac{9}{4};+\infty \right)$.
	}
\end{ex} \dongcham{11}

\Closesolutionfile{ans}

\ind{PHẦN II.} \inden{Câu trắc nghiệm đúng sai. Trong mỗi ý a), b), c), d) ở mỗi câu, học sinh chọn đúng hoặc sai.}\\
\Opensolutionfile{ans}[ans/2D1-B2-d1-2]

\begin{ex}%[2D1Y3]
		Cho hàm số $y=f(x)$ là hàm số liên tục trên $\mathbb{R}$ và có bảng biến thiên như hình vẽ dưới đây. 
		\begin{center}
			\begin{tikzpicture}
				\tkzTabInit[nocadre=false, lgt=1.2, espcl=1.5]{$x$ /0.6,$f'(x)$ /0.6,$f(x)$ /1.7}{$-\infty$,$-1$,$0$,$1$,$+\infty$}
				\tkzTabLine{,+,$0$,-,$0$,+,$0$,-,}
				\tkzTabVar{-/ $-\infty$ ,+/$4$,-/$3$,+/$4$,-/$-\infty$}
			\end{tikzpicture}
		\end{center}
	Xét tính đúng, sai của các khẳng định sau:
		\choiceTF
		{\True Cực đại của hàm số là $4$}
		{\True Cực tiểu của hàm số là $3$}
		{\True $\max\limits_{\mathbb{R}}{y}=4$}
		{$\min\limits_{\mathbb{R}}{y}=3$}
	\loigiai{
		Tử bảng biến thiên ta thấy $\lim\limits_{x\to+\infty}f(x)=-\infty$ nên hàm số không có giá trị nhỏ nhất trên $\mathbb{R}.$}
\end{ex} \dongcham{8} 

\begin{ex}
	Hình bên cho biết sự thay đổi của nhiệt độ ở một thành phố trong một ngày. Xét tính đúng, sai của các khẳng định sau:
	\begin{center}
		\begin{tikzpicture}[>=stealth,x=0.25cm,y=0.15cm]
			\draw[->] (-2,0)--(0,0) node[below left]{$O$}--(28,0) node[below right]{$x$ (giờ)};
			\draw[->] (0,-4)--(0,40) node[left]{$t$ ($^\circ C$)};
			\foreach \x/\g in {4/-90,8/-90,12/-90,16/-90,20/-90,24/-90}
			\draw[thin] (\x,2pt)--(\x,-2pt) + (\g:3mm) node {$\x$};
			%Vẽ các điểm trên trục Oy
			\foreach \y/\g in {25/180}
			\draw[thin] (2pt,\y)--(-2pt,\y) + (\g:3mm) node {$\y$};
			\path
			(0,25) coordinate (25)
			(4,20) coordinate (20)
			(8,31) coordinate (31)
			(12,28) coordinate (28)
			(16,34) coordinate (34)
			(20,27) coordinate (27)
			(24,24) coordinate (24);
			\draw [dashed] (4,0)--(4,20) (8,0)--(8,31) (12,0)--(12,28) (16,0)--(16,34) (20,0)--(20,27) (24,0)--(24,24); 
			\draw[smooth, thick, red]
			(25) .. controls +(-10:1) and +(-180:1) .. (20)
			(20) .. controls +(0:1) and +(-180:1) .. (31)
			(31) .. controls +(0:1) and +(160:1) .. (28)
			(28) .. controls +(0:1) and +(-180:2) .. (34)
			(34) .. controls +(0:1.5) and +(130:1.5) .. (27)
			(27) .. controls +(-60:1.5) and +(-180:2) .. (24);
			\foreach \x in {20,31,28,34,27,24}
			\fill (\x) +(90:3mm) node {$\x$};
		\end{tikzpicture}
	\end{center}
		\choiceTF
		{Nhiệt độ cao nhất trong ngày là $28^{\circ} \mathrm{C}$}
		{\True Nhiệt độ thấp nhất trong ngày là $20^{\circ} \mathrm{C}$}
		{\True Thời điểm có nhiệt độ cao nhất trong ngày là lúc $16$ giờ}
		{\True Thời điểm có nhiệt độ thấp nhất trong ngày là lúc $4$ giờ}
	\loigiai{}
\end{ex} \dongcham{8}

\begin{ex}
	Cho hàm số $y=f\left(x\right)$ có đạo hàm $y=f'\left(x\right)$ liên tục trên $\mathbb{R}$ và đồ thị của hàm số $f'\left(x\right)$ trên đoạn $\left[-2;6\right]$ như hình vẽ bên. 	Xét tính đúng, sai của các khẳng định sau:
	\begin{center}
		\begin{tikzpicture}[line join=round, line cap=round,>=stealth,scale=.7]
			\def\xmin{-3}\def\xmax{6.5}\def\ymin{-1}\def\ymax{3.5}
			\draw[->] (\xmin-0.2,0)--(\xmax+0.2,0) node[below] {\small $x$};
			\draw[->] (0,\ymin-0.2)--(0,\ymax+0.2) node[right] {\small $y$};
			\draw (0,0) node [below left] {\footnotesize $O$};
			\foreach \x in {-2,-1,2,6}\draw (\x,0.05)--(\x,-0.05) node [below] {\scriptsize $\x$};
			\foreach \y in {-1,1,2,3}\draw (0.05,\y)--(-0.05,\y) node [left] {\scriptsize $\y$};
			\clip (\xmin,\ymin) rectangle (\xmax,\ymax);
			\draw[thick,smooth,samples=200,domain=-2:6] plot (\x,{13/3360*(\x)^4-61/672*(\x)^3+173/336*(\x)^2-11/42*(\x)-61/70});
			\draw[dashed](-2,0)|-(0,2.5)(6,0)|-(0,1.5);
		\end{tikzpicture}
	\end{center}
		\choiceTF
		{$\max\limits_{\left[-2;6\right]}f\left(x\right)=f\left(-1\right)$}
		{$\max\limits_{\left[-2;6\right]}f\left(x\right)=f\left(6\right)$}
		{$\max\limits_{\left[-2;6\right]}f\left(x\right)=f\left(-2\right)$}
		{\True $\max\limits_{\left[-2;6\right]}f\left(x\right)=\max\left\{ f\left(-1\right),f\left(6\right)\right\}$}
	
	\loigiai{
		\begin{center}
			\begin{tikzpicture}
				\tkzTabInit[nocadre,,lgt=1.2,espcl=2.5,deltacl=0.6]
				{$x$/0.6,$y'$/0.6,$y$/2}
				{$-2$,$-1$,$2$,$6$}
				\tkzTabLine{,+,0,-,0,+,}
				\tkzTabVar{-/$f(-2)$,+/$f(-1)$,-/$f(2)$,+/$f(6)$}
			\end{tikzpicture}
		\end{center}
		Dựa vào bảng biến thiên, ta thấy
		\begin{itemize}
			\item Hàm số đồng biến trên $\left( { - 2; - 1} \right)$ và $\left( {2;6} \right)$ do $f'\left( x \right) > 0$, suy ra
			\begin{center}
				$f\left( { - 1} \right) > f\left( { - 2} \right)$ và $f\left( 6 \right) > f\left( 2 \right)$ (1).
			\end{center}
			\item Hàm số nghịch biến trên $\left( { - 1;2} \right)$ do $f'\left( x \right) < 0$, suy ra
			\begin{center}
				$f\left( { - 1} \right) > f\left( 2 \right)$  $ (2) $.
			\end{center}
		\end{itemize}
		Từ $ (1) $, $ (2) $ suy ra $\mathop {\max }\limits_{\left[ { - 2;6} \right]} f\left( x \right) = \max \left\{ {f\left( { - 2} \right),f\left( { - 1} \right),f\left( 2 \right),f\left( 6 \right)} \right\} = \max \left\{ {f\left( { - 1} \right),f\left( 6 \right)} \right\}$.
	}
\end{ex} \dongcham{13}

\begin{ex}
	Cho hàm số $f(x)$ có đạo hàm là $f'(x)$. Đồ thị $y=f'(x)$ được cho như hình vẽ. Biết rằng $f(0)+f(3)=f(2)+f(5)$. Xét tính đúng, sai của các khẳng định sau:
	\begin{center}
		\begin{tikzpicture}[scale=1, font=\footnotesize, line join=round, line cap=round, >=stealth]
			\draw[->](-1,0)--(5.5,0) node[right] {$x$};
			\draw[->](0,-1)--(0,2.5) node[right] {$y$};
			\node (0,0) [below left]{$O$};
			\foreach \x in {1,...,5}
			\draw[shift={(\x,0)},color=black] (0pt,2pt) -- (0pt,-2pt);
			\foreach \y in {1,...,2}
			\draw[shift={(0,\y)},color=black] (2pt,0pt) -- (-2pt,0pt);
			\draw (-0.3,1.2) .. controls (0.1,-1.8) and (1.5,-0.5) .. (2,0) .. controls (3,1.) and (4,1.2) .. (5,1.3) .. controls (5.1,1.3) and (5.3,1.3) .. (5.5,1.3);
			\clip (-1,-1) rectangle (5.5,2.5);
			\draw[dashed](5,0)--(5,1.3);
			\fill (0,0) circle(1pt) (2,0) circle(1pt) node[below right]{$2$} (5,0) circle(1pt) node[below]{$5$};
		\end{tikzpicture}
	\end{center}
	\choiceTF
	{Hàm số nghịch biến trên khoảng $(-\infty;0)$}
	{\True Hàm số nghịch biến trên khoảng $(0;2)$}
	{$\min\limits_{[0;5]}f(x)=f(0)$ và $\max\limits_{[0;5]}f(x)=f(5)$}
	{\True $\min\limits_{[0;5]}f(x)=f(2)$ và $\max\limits_{[0;5]}f(x)=f(5)$}
	
	\loigiai
	{
		Bảng biến thiên của hàm số trên đoạn $[0;5]$
		\begin{center}
			\begin{tikzpicture}
				\tkzTabInit[lgt=1.5,espcl=3,deltacl=0.6]
				{$x$/0.6, $f'(x)$/0.6, $f(x)$/2}
				{$0$, $2$, $3$, $5$}
				\tkzTabLine{,-,z,+, ,+,}
				\tkzTabVar{+/$f(0)$, -/$f(2)$, R, +/$f(5)$}
				\tkzTabVal[draw]{2}{4}{0.5}{}{$f(3)$}
			\end{tikzpicture}
		\end{center}
		Từ bảng biến thiên suy ra $\min\limits_{[0;5]}f(x)=f(2)$ và $\max\limits_{[0;5]}f(x)=\max\{f(0);f(5)\}$.\\
		Theo bảng biến thiên thì $f(3)>f(2)$ nên $f(3)-f(2)>0$.\\
		Theo giả thiết, ta có
		\[f(0)+f(3)=f(2)+f(5) \Leftrightarrow f(5)=f(0)+\left[f(3)-f(2)\right]>f(0).\]
		Suy ra $\max\limits_{[0;5]}f(x)=f(5)$.\\
		Vậy $\min\limits_{[0;5]}f(x)=f(2)$ và $\max\limits_{[0;5]}f(x)=f(5)$.
	}
\end{ex} \dongcham{13}

\Closesolutionfile{ans}
\begin{dang}{Bài toán max, min có chứa tham số $m$}
\end{dang}
\boxmini{BÀI TẬP TỰ LUẬN}
\begin{vd}
	Tìm tất cả giá trị của tham số $m$ để 
	\begin{tasks}
		\task giá trị lớn nhất của hàm số $f(x)= - x^3 -3x^2 +m$ trên $[-1;1]$ bằng $0$.
		\task giá trị nhỏ nhất của hàm số $ f(x)=\dfrac{x+5m}{x-3} $ trên $[1;2]$ bằng $4$.
	\end{tasks}
	\loigiai{
		\begin{enumerate}[a)]
			\item Hàm số liên tục và xác định trên đoạn $[-1;1]$.\\
			Ta có $f'(x)= -3x^2 -6x$.\\
			Cho $f'(x)=0 \Leftrightarrow \hoac{& x=0 \in [-1;1]\\& x= -2 \notin [-1;1].}$\\
			Xét $f(-1)= -2 + m $; $f(1)= -4 + m$.\\
			Suy ra $\displaystyle \max_{[-1;1]} f(x) = -2 + m$.\\
			Theo đề bài, $-2+ m=0 \Leftrightarrow m=2.$
			\item Ta có $ y'=\dfrac{-3-5m}{(x-3)^2} $.
			\begin{itemize}
				\item Trường hợp $ -3-5m>0\Leftrightarrow m<-\dfrac{3}{5}$\\
				$\Rightarrow y'>0 $ thì $ \displaystyle\min_{[1;2]}y=y(1)\Leftrightarrow -\dfrac{1}{2}(1+5m)=4\Leftrightarrow m=-\dfrac{9}{5}$ (nhận vì $ -\dfrac{9}{5}<-\dfrac{3}{5} $).
				\item Trường hợp $ -3-5m<0\Leftrightarrow m>-\dfrac{3}{5}$\\
				$ \Rightarrow y'<0 $ thì $ \displaystyle\min_{[1;4]}y=y(2)\Leftrightarrow -(2+5m)=4\Leftrightarrow m=-\dfrac{6}{5} $ (loại vì $ -\dfrac{6}{5}<-\dfrac{3}{5} $).
			\end{itemize}
			Vậy  $m=-\dfrac{9}{5}$.
		\end{enumerate}
		}
\end{vd}
\dongcham{20}
\boxmini{BÀI TẬP TRẮC NGHIỆM}

\setcounter{ex}{0}
\Opensolutionfile{ans}[ans/2D1-B2-d2-1]

\begin{ex}
	Cho hàm số $f(x) = 2x^3 -3x^2 + m$ thoả mãn $\displaystyle \min_{[0;5]} f(x) = 5$. Khi đó giá trị của $m$ bằng
	\choice
	{$10 $}
	{$ 5$}
	{\True $ 6$}
	{$ 7$}
	\loigiai{
		Ta có $f'(x)= 6x^2 -6x$.\\
		Cho $f'(x)=0 \Leftrightarrow \hoac{&x=0 \in [0;5] \\& x=1 \in [0;5].}$\\
		Xét $f(0)= m$; $f(1)= -1+ m$; $f(5)= 175 +m$.\\
		Suy ra $\displaystyle \min_{[0;5]} f(x)= -1+m$.\\
		Theo giả thiết $-1+ m= 5 \Leftrightarrow m=6$.}
\end{ex} \dongcham{10}

\begin{ex}
	Tìm $m$ để giá trị nhỏ nhất của hàm số $f(x) = 3x^3 - 4x^2 + 2(m - 10)$ trên đoạn $[1; 3]$ bằng $-5$.
	\choice
	{$m = \dfrac{15}{2}$}
	{$m = - 15$}
	{\True $m = 8$}
	{$m = -8$}
	\loigiai{
		$\bullet$ $f'(x) = 9x^2-8x$. Ta có $f'(x) = 0 \Leftrightarrow \hoac{&x = 0\\&x = \dfrac{8}{9}.}$\\
		$\bullet$ Ta có bảng biến thiên
		\begin{center}
			\begin{tikzpicture}
				\tkzTabInit[espcl=4,lgt=2,deltacl=1]{$x$/1,$f'(x)$/1,$f(x)$/2}
				{$1$,$3$}
				\tkzTabLine{,+,}
				\tkzTabVar{-/$2m-21$,+/$2m+25$}
			\end{tikzpicture}
		\end{center}
		$\bullet$ Giá trị nhỏ nhất của $f(x)$ trên đoạn $[1;3]$ bằng $-5 \Leftrightarrow 2m - 21 = -5 \Leftrightarrow m= 8$.
	}
\end{ex} \dongcham{12}

\begin{ex}
	Tìm $m$ để giá trị nhỏ nhất của hàm số $f(x)=\dfrac{x-m^2+m}{x+1}$ trên đoạn $[0;1]$ bằng $-2$.
	\choice
	{$\hoac{&m=1\\&m=-2}$}
	{$\hoac{&m=1\\&m=2}$}
	{$m=\dfrac{1\pm\sqrt{21}}{2}$}
	{\True $\hoac{&m=-1\\&m=2}$}
	\loigiai{
		$\mathscr{D}=\mathbb{R}\setminus\{-1\}$.\\
		Ta có $f'(x)=\dfrac{m^2-m+1}{(x+1)^2}>0$, $\forall x\in\mathscr{D}$.\\
		Khi đó $\min\limits_{x\in[0;1]}f(x)=f(0)\Leftrightarrow -2=-m^2+m\Leftrightarrow \hoac{&m=-1\\&m=2}$.
	}
\end{ex} \dongcham{12}

\begin{ex}
	Hàm số $y=\dfrac{x-m}{x+2}$ thỏa mãn $\min \limits_{x\in[0;3]}y+\max \limits_{x\in[0;3]}y=\dfrac{7}{6}$. Hỏi giá trị $m$ thuộc khoảng nào trong các khoảng dưới đây?
	\choice
	{$(2;+\infty)$}
	{$(0;2)$}
	{$(-\infty;-1)$}
	{\True $(-1;0)$}
	\loigiai{
		Do hàm số $y=\dfrac{x-m}{x+2}$ luôn đơn điệu trên đoạn $[0;3]$.\\
		Do đó $\min \limits_{x\in[0;3]}y+\max \limits_{x\in[0;3]}y=y(0)+y(3)=\dfrac{-m}{2}+\dfrac{3-m}{5}=\dfrac{7}{6}\Leftrightarrow\dfrac{-7m}{10}=\dfrac{17}{30}\Leftrightarrow m=\dfrac{-17}{21}$.}
\end{ex} \dongcham{11}

\begin{ex}
	Cho hàm số $y=\dfrac{x+m}{x+1}$ ($m$ là tham số thực) thỏa mãn $\min\limits_{[1;2]} y+\max\limits_{[1;2]} y=\dfrac{16}{3}$. Mệnh đề nào dưới đây đúng?
	\choice
	{\True $m>4$}
	{$m\le 0$}
	{$0<m\le 2$}
	{$2<m\le 4$}
	\loigiai{
		Tập xác định $\mathscr{D}=\mathbb{R}$.\\
		Ta có $y'=\dfrac{1-m}{(x+1)^2}$.
		\begin{itemize}
			\item Với $m=1$ thì $y=1$ nên $\min\limits_{[1;2]} y+\max\limits_{[1;2]} y=2$ (không thỏa mãn).
			\item Với $m\neq 1$ thì hàm số đơn điệu trên $[1;2]$ nên
			\begin{eqnarray*}
				&& \min\limits_{[1;2]} y+\max\limits_{[1;2]} y=\dfrac{16}{3}\\
				& \Leftrightarrow & y(1)+y(2)=\dfrac{16}{3}\\
				& \Leftrightarrow & \dfrac{m+1}{2}+\dfrac{m+2}{3}=\dfrac{16}{3}\\
				& \Leftrightarrow & m=5>4.
			\end{eqnarray*}
		\end{itemize}
	}
\end{ex} \dongcham{11}

\begin{ex}
	Cho hàm số $ f(x)=\dfrac{x+m}{x-1} $ ($ m $ là tham số thực) thỏa mãn $ \min\limits_{[2 ; 4]} f(x)=3 $. Mệnh đề nào dưới đây đúng ?
	\choice
	{$1\leq m<3$}
	{$m < -1$}
	{$3<m\leq 4$}
	{\True$m>4$}
	\loigiai{
		Tập xác định $ \mathscr{D} = \mathbb{R} \setminus\{1\}$.\\
		Ta có $ f'(x)=\dfrac{-1-m}{(x-1)^{2}} $.\\
		\underline{\textbf{TH1}}: $ -1-m<0 \Leftrightarrow m >-1 $.\\
		Ta có $ \min\limits_{[2 ; 4]}y=y(4)=\dfrac{4+m}{4-1}=3\Leftrightarrow m=5$ (thỏa mãn).\\
		\underline{\textbf{TH2}}: $  -1-m>0 \Leftrightarrow m <-1 $.\\
		Ta có $ \min\limits_{[2 ; 4]}y=y(2)=\dfrac{2+m}{2-1}=3\Leftrightarrow m=1$ (loại).\\
		Vậy $ m=5>4 $.
	}
\end{ex} \dongcham{11}

\begin{ex}
	Gọi $S$ là tổng giá trị của $m$ để hàm số $f(x) = \dfrac{x - m^2 - m}{x+1}$ có giá trị nhỏ nhất trên $[0;1]$ bằng $-2$. Mệnh đề nào sau đây đúng?
	\choice
	{\True $S=-1 $}
	{$S=1 $}
	{$S=-2 $}
	{$ S=-3$}
	\loigiai{
		Ta có $f'(x)= \dfrac{m^2 + m -1 }{(x+1)^2}$.
		\begin{itemize}
			\item Trường hợp $1$: $y'<0 \Leftrightarrow m^2 + m -1 <0$.\\
			Khi đó hàm số nghịch biến trên $[0;1]$.\\
			Suy ra $\displaystyle \min_{[0;1]} f(x) = f(1)= \dfrac{-m^2 -m +1}{2}$.\\
			Theo giả thiết $\dfrac{-m^2 -m +1}{2} = -2 \Leftrightarrow m^2 + m =5$ (không thoả điều kiện $m^2 +m <1$).
			\item Trường hợp $2$: $y'>0 \Leftrightarrow m^2 + m -1>0$.\\
			Khi đó $\displaystyle \min_{[0;1]} f(x) = f(0)=-m^2 -m$.\\
			Theo giả thiết $-m^2 -m =-2  \Leftrightarrow \hoac{&m= 1 \text{ (nhận) }\\& m=-2 \text{ (nhận).}}$
		\end{itemize}
		Vậy tổng các giá trị của $m$ là $-2+1 =-1.$
	}
\end{ex} \dongcham{11}

\begin{ex}
	Cho hàm số $f(x)=x^3+m x^2-m^2x+2$ với tham số $m>0$. Biết $\min\limits_{[-m ; m]}f(x)=\dfrac{14}{ 27}$. Mệnh đề nào dưới đây đúng
	\choice
	{$m\in (-\infty;-3)$}
	{$m\in (3;+\infty)$}
	{\True $m\in (1;3)$}
	{$m\in (-3;-1)$}
	\loigiai{
		Ta có $f'(x)=3x^2+2mx-m^2=(x+m)(3x-m)$.\\
		$f'(x)=0\Leftrightarrow \hoac{& x=-m \\ & x=\dfrac{m}{3}}$. Suy ra $\heva{& f(-m)=m^3 +2\\ & f(m)=m^3+2\\ &f\left(\dfrac{m}{3}\right)=-\dfrac{5m^3}{27}+2.}$\\
		Vì $m>0$ nên $f(m)=f(-m)>f\left(\dfrac{m}{3}\right)$, suy ra $\min\limits_{  [-m;m]} f(x)=f\left(\dfrac{m}{3}\right)=\dfrac{14}{27}$.\\
		Do đó $m=2$, vậy $m\in(1;3)$.
	}
\end{ex} \dongcham{11}

\begin{ex}%[2D1K3-1]
	Có tất cả bao nhiêu giá trị nguyên của tham số $m$ để giá trị nhỏ nhất của hàm số $y=x^3+\left(m^2-m+1\right)x+m^3-4m^2+m+2025$ trên đoạn $[0;2]$ bằng $2019$?
	\choice
	{$0$}
	{$1$}
	{$2$}
	{\True $3$}
	\loigiai{
		Ta có $y'=f'(x)=3x^2+\left(m^2-m+1\right)$ trên đoạn $[0;2]$.\\
		Ta có $y'=3x^2+\left(m-\dfrac{1}{2}\right)^2+\dfrac{3}{4}>0,\forall x\in\mathbb{R}$.\\
		Do đó hàm số đồng biến trên $\mathbb{R}\Rightarrow$ ta có $\min\limits_{[0;2]}y=f(0)=m^3-4m^2+m+2025$.\\
		Ta có $f(0)=2019\Leftrightarrow m^3-4m^2+m+2025=2019\Leftrightarrow m^3-4m^2+m+6=0\Leftrightarrow\hoac{&m=-1\\&m=2\\&m=3.}$\\
		Vậy tập các giá trị $m$ thỏa mãn là $\{-1;2;3\}$. Hay có tất cả $3$ giá trị $m$ thỏa mãn.}
\end{ex} \dongcham{11}

\begin{ex}
	Gọi $S$ là tập tất cả các giá trị của $m$ sao cho giá trị nhỏ nhất của hàm số $y=\left(x^3-3x+m \right)^2 $ trên
	đoạn $[-1;1]$ bằng $4$. Tính tổng các phần tử của $S$.
	\choice
	{\True  $ 0 $}
	{$ 6 $}
	{$ -5 $}
	{$ 3 $}
	\loigiai{
		\immini{Ta có  $\displaystyle\min\limits_{[-1;1]}\left(x^3-3x+m \right)^2=4  \Leftrightarrow \displaystyle\min\limits_{[-1;1]}\left|x^3-3x+m \right|=2$.\\Xét hàm số $y=f(x)=x^3-3x+m$ trên $[-1;1]$.\\
			Ta có $y'=3x^2-3=3(x^2-1)$, $y'=0\Leftrightarrow x=\pm1$.\\
			Bảng biến thiên hàm số như hình bên.
		}{\begin{tikzpicture}[scale=.8,line join=round, line cap=round,font=\footnotesize,>=stealth]
				\def\a{6}
				\def\b{3.7}
				\draw[shift={(-.5,.5)},blue!50!black]
				(0,0) rectangle +(\a,-\b)
				(0,-1)--+(0:\a)
				(0,-2)--+(0:\a)
				(1,0)--+(-90:\b)
				;
				\path
				(0,0) node{$x$}
				(0,-1) node{$y'$}
				(0,-2.3) node{$y$}
				(1,0) node{$-1$}
				(5,0) node{$1$}
				(1,-1) node{$0$}
				(3,-1) node{$-$}
				(5,-1) node{$0$}
				(1.2,-1.8) node (A) {$m+2$}
				(4.8,-3) node (C){$m-2$}
				;
				\draw[->] (A)--(C);
		\end{tikzpicture}}
		\noindent Từ bảng biến thiên của hàm số $y=f(x)$, ta có $\displaystyle\min\limits_{[-1;1]}\left|x^3-3x+m \right|=2$ khi và chỉ khi
		\begin{enumerate}[TH1.]
			\item $\heva{&m+2<0\\&m+2=-2}\Leftrightarrow m=-4$.
			\item $\heva{&m-2>0\\&m-2=2}\Leftrightarrow m=4$.
		\end{enumerate}
		Vậy $S=\{-4,4\}\Rightarrow $ Tổng các phần tử của $S$ bằng $0$.
	}
\end{ex} \dongcham{12}

\Closesolutionfile{ans}



% 
\begin{dang}{Bài toán vận dụng, thực tiễn có liên quan đến max min}
	\begin{enumerate}[\iconMT]
		\item \indam{Bài toán chuyển động:}
		\begin{itemize}
			\item [$\bullet$] Gọi $s(t)$ là hàm quãng đường; $v(t)$ là hàm vận tốc; $a(t)$ là hàm giá tốc;
			\item [$\bullet$] Khi đó $s'(t)=v(t)$; $v'(t)=a(t)$.
		\end{itemize}
		\item \indam{Bài toán thực tế -- tối ưu:}
		\begin{itemize}
			\item[$\bullet$] Biểu diễn dữ kiện cần đạt max -- min qua một hàm $f(t)$. 
			\item[$\bullet$] Khảo sát hàm $f(t)$ trên miền điều kiện của hàm và suy ra kết quả.
		\end{itemize}
	\end{enumerate}
\end{dang}
\boxmini{BÀI TẬP TỰ LUẬN}
\begin{vd}%[2D1B3-6]
	Một chất điểm chuyển động có vận tốc tức thời $v(t)$ phụ thuộc vào thời gian $t$ theo hàm số $v(t)=-t^4+24t^2+500$ (m/s). Trong khoảng thời gian từ $t=0$ (s) đến $t=5$ (s) chất điểm đạt vận tốc lớn nhất tại thời điểm nào?
	\loigiai{Ta có $v'(t)=-4t^3+48t=-4t(t^2-12)$\\
		$v'(t)=0\Leftrightarrow \hoac{&t=0\\&t=\pm 2\sqrt{3}}$.\\
		Bài toán trở thành tìm giá trị lớn nhất của hàm số $v(t)$ trên đoạn $[0;10]$, ta có:\\
		$v(0)=500$, $v(2\sqrt{3})=664$, $v(5)=475$.\\
		Vậy vận tốc lớn nhất khi $t=2\sqrt{3}\approx 4$ (s).
	}
\end{vd}
\dongcham{8}
\begin{vd}
	\immini{
		Sự phân huỷ của rác thải hữu cơ có trong nước sẽ làm tiêu hao oxygen hoà tan trong nước. Nồng độ oxygen (mg/l) trong một hồ nước sau $t$ giờ $(t \geq 0)$ khi một lượng rác thải hữu cơ bị xả vào hồ được xấp xỉ bởi hàm số (có đồ thị như đường màu đỏ ở hình bên)
		$$
		y(t)=5-\frac{15 t}{9 t^2+1}.
		$$
	}{
		\begin{tikzpicture}[>=stealth,x=1cm,y=0.3cm,scale=1.5,font=\footnotesize]
			\draw[->] (-0.5,0) -- (4,0) node[below] {$t$};
			\draw[->] (0,-1) -- (0,6) node[left] {$y$};
			\filldraw (0,0) circle (1pt)node[below left]{$O$};
			\draw[domain=0:4,samples=200,red] plot (\x,{5-(15*(\x))/(9*(\x)^2+1)});
			\draw[dashed] (0,5) node [left] {$5$}--(4,5);
			\foreach \x/\g in {1/-90,2/-90,3/-90}
			\draw[thin] (\x,2pt)--(\x,-2pt) + (\g:3mm) node {$\x$};
		\end{tikzpicture}
	}
	\noindent
	Vào các thời điểm nào nồng độ oxygen trong nước cao nhất và thấp nhất?\
	\loigiai{
		Xét hàm số $y(t)=5-\dfrac{15t}{9t^2+1}$ xác định và liên tục trên khoảng $[0;+\infty)$ .\\
		Ta có $y'(t)=\dfrac{135t^2-15}{(9t^2+1)^2}=0\Leftrightarrow t=\dfrac{1}{3}$ (giờ).\\
		Mặt khác $\lim\limits_{t\to+\infty}y(t)=\lim\limits_{t\to+\infty}\left[5-\dfrac{15t}{9t^2+1}\right]=5$ và $\lim\limits_{t\to 0}y(t)=\lim\limits_{t\to 0}\left[5-\dfrac{15t}{9t^2+1}\right]=5$.\\
		Bảng biến thiên
		\begin{center}
			\begin{tikzpicture}
				\tkzTabInit[espcl=3,lgt=1.5]
				{$t$/0.6,$y'(t)$/0.6,$y(t)$/1.5}
				{$0$,$\frac{1}{3}$,$+\infty$}
				\tkzTabLine{,-,0,+,}
				\tkzTabVar{+/$5$,-/$0$,+/$5$}
			\end{tikzpicture}
		\end{center}
		Từ bảng biến thiên, ta thấy $\min\limits_{[0;+\infty)}y(x)=0$ và $\mathop{\rm{max}}\limits_{[0;+\infty)}y(x)=5$.
	}
\end{vd}
\dongcham{10}
\begin{vd}%[2D1T3-6]
	\immini[0.02]{
		Tính diện tích lớn nhất $S_{\max}$ của một hình chữ nhật nội tiếp trong nửa đường tròn bán kính $R=6$ cm nếu một cạnh của hình chữ nhật nằm dọc theo đường kính của hình tròn mà hình chữ nhật đó nội tiếp.
	}{
		\begin{tikzpicture}[line join = round, line cap = round,>=stealth,font=\footnotesize,scale=1] 
			\def\R{2}
			\coordinate[label = below:$O$] (O) at (0,0); 
			\coordinate (A) at (-\R,0); 
			\coordinate (B) at ($(A)!2!(O)$);
			\coordinate[label = above right:$C$] (C) at (50:\R); 
			\coordinate[label = above left:$D$] (D) at (130:\R);
			\coordinate[label = below:$A$] (AA) at ($(A)!(D)!(B)$); 
			\coordinate[label = below:$B$] (BB) at ($(A)!(C)!(B)$); 
			\draw (A) arc(180:0:\R)--cycle;
			\draw[fill=cyan!20] (BB)--(C)--(D)--(AA)--cycle;
			\foreach \x in {AA,O,BB} \fill[black] (\x) circle (1.5pt); 
		\end{tikzpicture}
	}
	\loigiai{
		\immini{
			{\bf Cách 1.}\\
			Gọi chiều dài $AD=2x$ ($0<x<6$)\\
			$\Rightarrow AB=\sqrt{36-x^{2}}$.\\
			Diện tích hình chữ nhật là $S=2x\sqrt{36-x^{2}}$.\\
			Xét $f(x)=x\sqrt{36-x^{2}}$ trên $(0;6)$, ta có $$f'(x)=\sqrt{36-x^{2}}-\dfrac{x^{2}}{\sqrt{36-x^{2}}}=0\Leftrightarrow x=\pm 3\sqrt{2}.$$
		}{
			\begin{tikzpicture}
				\tikzset{on double/.style = {fill = \tkzTabDefaultBackgroundColor}} 
				\tikzset{h style/.style = {pattern=north west lines}} 
				\tkzTabInit[lgt=1.2,espcl=2]
				{$x$ /.6,$f'(x)$ /.6, $f(x)$ /1.5}
				{$0$,$3\sqrt{2}$,$6$}
				\tkzTabLine{d,+,0,-,d}
				\tkzTabVar{-/$0$,+/$36$,-/$0$}
			\end{tikzpicture}
		}
		Bảng biến thiên hàm số $f(x)$ trên $(0,6)$ ở hình bên\\
		Vậy giá trị lớn nhất của diện tích hình chữ nhật $ABCD$ là $36$ cm$^2$.\\
		{\bf Cách 2.}\\
		Đặt $AB=CD=2x$ ($0<x<6$). Khi đó $AD=\sqrt{DO^2-AO^2}=\sqrt{36-x^2}$. Suy ra
		\begin{align*}
			S_{ABCD}=2x\sqrt{36-x^2}\le 2\cdot \dfrac{x^2+36-x^2}{2}=36.
		\end{align*}
		Dấu bằng xảy ra khi $x=\sqrt{36-x^2}$ hay $x=3\sqrt{2}$.\\
		Vậy giá trị lớn nhất của diện tích hình chữ nhật $ABCD$ là $36$ cm$^2$.
	}
\end{vd}
\dongcham{14}
\begin{vd}%[2D1K3-6]
	\immini{Một người muốn xây một cái bể chứa nước, dạng một khối hộp chữ nhật không nắp có thể tích
	bằng $288$ dm$^3$. Đáy bể là hình chữ nhật có chiều dài gấp đôi chiều rộng, giá thuê nhân công để xây bể là
	$500000$ đồng/ m$^2$. Nếu người đó biết xác định các kích thước của bể hợp lí thì chi phí thuê nhân công sẽ
	thấp nhất. Hỏi người đó trả chi phí thấp nhất để thuê nhân công xây dựng bể đó là bao nhiêu?}{\hspace{1cm}
	\begin{tikzpicture}[scale=0.8, line join=round, line cap=round]
		\tkzDefPoints{0/0/A,-1.3/-1.1/B,2/-1.1/C}
		\coordinate (D) at ($(A)+(C)-(B)$);
		\coordinate (A') at ($(A)+(0,2.5)$);
		\tkzDefPointsBy[translation=from A to A'](B,C,D){B'}{C'}{D'}
		\tkzDrawPolygon(A',B',B,C,D,D')
		\tkzDrawSegments(B',C' C',D' C,C')
		\tkzDrawSegments[dashed](A,B A,D A,A')
\end{tikzpicture}}
	\loigiai{
		Gọi $x(x>0)$ là chiều rộng của đáy bể. Khi đó, chiều dài của bể là $2x$ và chiều cao của bể là $\dfrac{0,144}{x^2}$.\\
		Diện tích cần xây $2x^2+\dfrac{0,864}{x}$\\
		Xét $f(x) = 2x^2 + \dfrac{0,864}{x}$, có
		$f'(x) = 4x - \dfrac{0,864}{x^2}$\\
		$f'(x) = 0 \Leftrightarrow 4x - \dfrac{0,864}{x^2} \Leftrightarrow x=0,6.$\\
		Bảng biến thiên
		\begin{center}
			\begin{tikzpicture}
				\tkzTabInit[nocadre=false, lgt=1.2, espcl=3]
				{$x$ /0.6,$f'(x)$ /0.6,$f(x)$ /1.5} 	
				{$0$, $0{,}6$, $+\infty$}
				\tkzTabLine{,-,$0$,+}
				\tkzTabVar{+/ $+\infty$ ,-/$2{,}16$,+/$+\infty$}
			\end{tikzpicture}
		\end{center}
		Từ bảng biến thiên ta có $\min f(x)= 2,16.$\\
		Vậy chi phí thấp nhất để thuê nhân công xây bể là $2,16 \times 500000 = 1080000$ đồng.
	}
\end{vd}
\dongcham{18}
\begin{vd}%[2D1T3-2]
	\immini{Một nhà sản xuất cần làm ra những chiếc bình có dạng hình trụ với dung tích $1000\mathrm{~cm}^3$. Mặt trên và mặt dưới của bình được làm bằng vật liệu có giá 1,2 nghìn đồng$/\mathrm{cm}^2$, trong khi mặt bên của bình được làm bằng vật liệu có giá $0{,}75$ nghìn đồng$/\mathrm{cm}^2$. Tìm các kích thước của bình để chi phí vật liệu sản xuất mỗi chiếc bình là nhỏ nhất.}{\hspace{1cm}
	\begin{tikzpicture}[line join=round,line cap=round,line width=.6pt,font=\footnotesize,scale=0.46,>=stealth]
		\coordinate[label=right:$A$] (A) at (3,0);
		\coordinate[label=left:$O$] (O) at (0,0);
		\coordinate[label=right:$A'$] (A1) at ($(A)+(90:6)$);
		\coordinate[label=left:$O'$] (O1) at ($(O)+(90:6)$);
		\draw (A) arc (0:-180:3 and 3/4)--($(A1)!2!(O1)$) arc (180:0:3 and 3/4) arc (0:-180:3 and 3/4) (A)--(A1)--(O1);
		\draw[dashed] (O1)--(O)--(A) arc (0:180:3 and 3/4);
		\fill (O)circle(1.5pt) (O1)circle(1.5pt) (A)circle(1.5pt) (A1)circle(1.5pt);
\end{tikzpicture}}
	\loigiai{
			Gọi bán kính đáy của bình là $x$ (cm), ($x > 0$).\\
			Chiều cao của bình là $\dfrac{1000}{\pi \cdot x^2}$ (cm).\\
			Chi phí để sản xuất một chiếc bình là 
			\[
			T(x)=2\cdot1{,}2\cdot\pi \cdot x^2+0{,}75\cdot \dfrac{2000}{x}=2{,}4\pi \cdot x^2+\dfrac{1500}{x}~\text{(nghìn đồng)}.
			\]
			Để chi phí sản xuất mỗi chiếc bình là thấp nhất thì $T(x)$ là nhỏ nhất.\\
			$T^{\prime}(x)=4,8\pi x-\dfrac{1500}{x^2}, T^{\prime}(x)=0\Leftrightarrow x=\sqrt[3]{\dfrac{625}{2\pi}}$ (thỏa mãn).\\
			Bảng biến thiên:
			\begin{center}
				\begin{tikzpicture}[scale=1, font=\footnotesize]
					\tkzTabInit[nocadre=false, lgt=1.2, espcl=2, deltacl=0.6]
					{$x$/0.8,$T'(x)$/0.6,$T(x)$/2}
					{$0$,$\sqrt[3]{\frac{625}{2\pi}}$,$12$};
					\tkzTabLine{,-,$0$,+,};
					\tkzTabVar{+/$+\infty$,-/$T\left(\sqrt[3]{\frac{625}{2\pi}}\right)$,+/$T(12)$};
				\end{tikzpicture}
			\end{center}
			Để chi phí sản xuất mỗi chiếc bình là nhỏ nhất thì bán kính đáy của bình là $\sqrt[3]{\dfrac{625}{2\pi}}$ cm và chiều cao của bình là $\dfrac{1000}{\pi \cdot\left(\sqrt[3]{\dfrac{625}{2\pi}}\right)^2}$ cm.
	}
\end{vd}
\dongcham{20}
\boxmini{BÀI TẬP TRẮC NGHIỆM}
\ind{PHẦN I.} \inden{Câu trắc nghiệm nhiều phương án lựa chọn. Mỗi câu hỏi học sinh chỉ chọn một phương án.}\\
\setcounter{ex}{0}
\Opensolutionfile{ans}[ans/2D1-B2-d3-1]
\begin{ex}%[2D1K3]
	Một chất điểm chuyển động với quãng đường $s(t)$ cho bởi công thức $s(t)=6t^2-t^3$, $t$ (giây) là thời gian. Hỏi trong khoảng thời gian từ $0$ đến $4$ giây, vận tốc tức thời của chất điểm đạt giá trị lớn nhất tại thời điểm  $t$ (giây) bằng bao nhiêu?
	\choice
	{$t=3$ s}
	{$t=4$ s}
	{\True $t=2$ s}
	{$t=6$ s}
	\loigiai{Ta có $v(t)=s'(t)=12t-3t^2$.\\
		$v'(t)=12-6t$, $v'(t)=0\Leftrightarrow t=2$. \\
		Lập bảng biến thiên ta thấy $v(t)$ đạt giá trị lớn nhất tại $t=2$.
	}
\end{ex} \dongcham{7}

\begin{ex}
	Trong $3$ giây đầu tiên, một chất điểm chuyển động theo phương trình $s(t)=-t^3+6t^2+t+5,$ trong đó $t$ tính bằng giây và $s$ tính bằng mét. Chất điểm có vận tốc tức thời lớn nhất bằng bao nhiêu trong $3$ giây đầu tiên đó?
	\choice
	{\True 13 m/s}
	{10 m/s}
	{9 m/s}
	{12 m/s}
	\loigiai{
		Ta có $v(t)=s'(t)=-3t^2+12t+1.$ Xét hàm số $v(t)=-3t^2+12t+1$ trên đoạn $[0;5]$.\\
		$v'(t)=-6t+12$; $v'(t)=0 \Leftrightarrow t=2$.\\
		Tính các giá trị $v(0)=1$, $v\left(2\right)=13$, $v(3)=10$.\\
		So sánh các giá trị, ta có $\max\limits_{[0;3]}v(t)=13$.
	}
\end{ex}
\dongcham{7}
\begin{ex}
	Độ giảm huyết áp của một bệnh nhân được cho bởi công thức $G(x)=0{,}025x^2(30-x)$, trong đó $x$ là liều lượng thuốc được tiêm cho bệnh nhân ($x$ được tính bằng miligam). Liều lượng thuốc cần tiêm cho bệnh nhân là bao nhiêu để huyết áp được giảm nhanh nhất?
	\choice
	{$24$ mg}
	{\True $20$ mg}
	{$15$ mg}
	{$10$ mg}
	\loigiai
	{ 
		Bài toán trở thành: Tìm $x\in[0;30]$ để hàm số $G(x)=0{,}025x^2(30-x)$ đạt giá trị lớn nhất. \\
		Ta có $G(x)=0{,}025\left(30x^2-x^3\right) \Rightarrow G'(x)=0{,}025\left(60x-3x^2\right)$. \\
		Xét $G'(x)=0 \Leftrightarrow \hoac{ & x=0 \\ & x=20.}$ \\
		Bảng biến thiên hàm số $G(x)$
		\begin{center}
			\begin{tikzpicture}[scale=1]
				\tkzTabInit[nocadre=false, lgt=1.2, espcl=3.5, deltacl=0.6]{$x$/0.6, $G'(x)$/0.6, $G(x)$/2}{$0$, $20$, $30$}
				\tkzTabLine{0,+,0,-,}
				\tkzTabVar{-/$0$, +/$100$, -/$0$}
			\end{tikzpicture}
		\end{center}
		Từ bảng biến thiên ta có $\max\limits_{[0;30]} G(x)=G(20)=100$. \\
		Vậy liều lượng thuốc cần tiêm cho bệnh nhân để huyết áp giảm nhanh nhất là $20$ mg.
	}
\end{ex}
\dongcham{7}
\begin{ex}
	Trong thí nghiệm y học, người ta cấy $1\,000$ vi khuẩn vào môi trường dinh dưỡng. Bằng thực nghiệm, người ta xác định số lượng vi khuẩn thay đổi theo thời gian bởi công thức \[N(t)=1\,000+\dfrac{100t}{100+t^2}\,\text(con).\]
	trong đó $t$ là thời gian tính bằng giây. Tính số lượng vi khuẩn lớn nhất kể từ khi thực hiện cấy vi khuẩn vào môi trường dinh dưỡng.
	\choice
	{$1\,008$ con}
	{$1\,012$ con}
	{\True $1\,005$ con}
	{$1\,020$ con}
	\loigiai{
		Xét hàm số $N(t)=1\,000+\dfrac{100t}{100+t^2}$ ($t>0$).\\
		Ta có $N'(t)=\dfrac{100\cdot (100+t^2)-100t\cdot 2t}{\left(100+t^2\right)^2}=\dfrac{100\cdot (100-t^2)}{\left(100+t^2\right)^2}$.\\
		Khi đó, với $t>0$, $N'(t)=0\Leftrightarrow 100-t^2=0\Leftrightarrow t^2=100\Leftrightarrow t=10$.\\
		Bảng biến thiên của hàm số $N(t)$ như sau
		\begin{center}
			\begin{tikzpicture}[>=stealth]
				\tkzTabInit[nocadre=false,lgt=1.5,espcl=3,deltacl=0.6]{$t$/.6 ,$N'(t)$/.6,$N(t)$/1.5}
				{$0$ , $10$ , $+\infty$}
				\tkzTabLine{ ,+ , $0$ , - , }
				\tkzTabVar{-/$1\,000$ , +/$1\,005$ , -/$1\,000$}
			\end{tikzpicture}
		\end{center}
		Căn cứ vào bảng biến thiên, ta thấy trên khoảng $(0;+\infty)$, hàm số $N(t)$ đạt giá trị lớn nhất bằng $1\,005$ tại $t=10$.\\
		Vậy số lượng vi khuẩn lớn nhất kể từ khi thực hiện nuôi cấy vi khuẩn vào môi trường dinh dưỡng là $1\,005$ con.
	}
\end{ex}
\dongcham{14}
\begin{ex}
	Tam giác vuông có cạnh huyền bằng $5 \mathrm{~cm}$ có thể có diện tích lớn nhất bằng bao nhiêu?
	\choice
	{25 $\text{cm}^2$}
	{$\dfrac{125}{4}\,\text{cm}^2$}
	{$\dfrac{625}{4}\,\text{cm}^2$}
	{$125 \text{cm}^2$}
	\loigiai{Gọi một cạnh góc vuông là $x$ ($0<x<5$) thì cạnh góc vuông còn lại là $\sqrt{25-x^2}$.\\ Như vậy, diện tích tam giác là $S=\dfrac{x\cdot\sqrt{25-x^2}}{2}$.
		Đặt $f(x)=25x^2-x^4$. 
		\\Ta có $f'(x)=50x-4x^3$. Khi đó
		$f'(x)=0 \Leftrightarrow x=\dfrac{5\sqrt{2}}{2}$.\\
		Vì vậy $\displaystyle\max _{(0;5)} f(x)=f\left( \dfrac{5\sqrt{2}}{2}\right) =\dfrac{625}{4}$.\\
		Vậy tam giác vuông có cạnh huyền bằng $5 \mathrm{~cm}$ có thể có diện tích lớn nhất bằng $\dfrac{625}{4}$.}
\end{ex}
\dongcham{18}
\begin{ex}
	\immini{
		Từ một tấm tôn có hình dạng là nửa hình tròn bán kính $R=3$, người ta muốn cắt ra một hình chữ nhật (hình vẽ bên). Diện tích lớn nhất có thể của tấm tôn hình chữ nhật là
		\choice
		{$\dfrac{9}{2}$}
		{$6\sqrt2$}
		{\True $9$}
		{$9\sqrt2$}
	}
	{
		\begin{tikzpicture}[thick,scale=0.57]
			\draw [-] (-4,0)--(4,0);
			\draw [-] (-3,0)--(-2.99,2.65)--(2.99,2.65)--(3,0);
			\draw[smooth,samples=200,variable=\t,domain=0:180] plot({(4)*cos (\t)},{(4)*sin(\t)});
			\draw (0,0) [fill=black] circle (1pt) node[below]{$O$};
			\draw (-3,0) [fill=black] circle (1pt) node[below]{$Q$};
			\draw (3,0) [fill=black] circle (1pt) node[below]{$P$};
			\draw (-2.99,2.65) [fill=black] circle (1pt) node[left]{$M$};
			\draw (2.99,2.65) [fill=black] circle (1pt) node[right]{$N$};
			\draw[pattern=north east lines,pattern color=black!50!] (-3,0)--(-2.99,2.65)--(2.99,2.65)--(3,0);
		\end{tikzpicture}
	}
	\loigiai{
		Đặt $OQ=x,\ (0<x<3) \Rightarrow MQ=\sqrt{MO^2-OQ^2}=\sqrt{9-x^2}$.\\
		Ta có  $S_{MNPQ}=PQ\cdot MQ=2x\cdot\sqrt{9-x^2}\le 2\cdot\dfrac{x^2+9-x^2}{2}=9.$\\
		Dấu $=$ xảy ra khi $x=\dfrac{3\sqrt2}{2}.$
	}
\end{ex} \dongcham{18}

\begin{ex}%[2D1K3-6]
	Cho một tấm tôn hình chữ nhật có kích thước $10$ cm $\times$ $16$ cm. Người ta cắt bỏ $4$ góc của tấm tôn $4$ miếng hình vuông bằng nhau rồi gò lại thành một hình hộp chữ nhật không có nắp. Để thể tích của hình hộp đó lớn nhất thì độ dài cạnh hình vuông của các miếng tôn bị cắt bỏ bằng
	\choice
	{$3$ m}
	{$4$ m}
	{$5$ m}
	{\True $2$ m}
	\loigiai{
		\immini
		{Giả sử độ dài cạnh hình vuông của các miếng tôn bị cắt bỏ bằng $x$ $(0<2x<10\Leftrightarrow 0<x<5)$. Khi đó hình hộp chữ nhật có chiều cao bằng $x$, chiều rộng bằng $10-2x$ và chiều dài bằng $16-2x$. Suy ra hình hộp chữ nhật có thể tích $V=x(10-2x)(16-2x)=4x^3-52x^2+160x$.}
		{
			\begin{tikzpicture}[scale=0.7]
				\tkzInit[xmin=-5,xmax=6,ymin=-3,ymax=6]
				\tkzDefPoints{0/0/A, 8/0/D, 8/6/C, 0/6/B, 0/1/E, 0/5/F, 1/6/G, 7/6/H, 8/5/I, 8/1/J, 1/0/M, 7/0/N}
				\tkzDrawPoints(A,B,C,D,M,N,E,F,G,H,I,J)
				\tkzLabelSegments[above](B,G H,C){$x$}
				\tkzLabelSegments[right](C,I J,D){$x$}
				\tkzLabelSegment[left](A,B){$10$}
				\tkzLabelSegment[below](A,D){$16$}
				\tkzDrawSegments[thin](A,B A,D B,C C,D F,I E,J G,M H,N)
			\end{tikzpicture}
		}
		Xét hàm $f(x)=4x^3-52x^2+160x$ trên $(0; 5)$. Tập xác định $\mathscr{D}=\mathbb{R}$,\\ $f'(x)=12x^2-104x+160=0\Leftrightarrow\hoac{&x=2\\&x=\dfrac{20}{3}.}$
		Bảng biến thiên hàm số $f(x)$ trên $(0; 5)$:
		\begin{center}
			\begin{tikzpicture}
				\tkzTabInit%
				{$x$/1,%
					$f’(x)$ /1,%
					$f(x)$ /2}%
				{$0$ ,$2$ , $5$}%
				\tkzTabLine{ ,+, 0 ,-,}
				\tkzTabVar %
				{
					-/,+/ ,-/
				}
			\end{tikzpicture}
		\end{center}
		Dựa vào bảng biến thiên ta có hàm số đạt giá trị lớn nhất trên $(0; 5)$ tại $x=2$ hay hình hộp chữ nhật có thể tích lớn nhất khi độ dài cạnh hình vuông của miếng tôn bị cắt bỏ bằng $2$ m.
	}
\end{ex}
\dongcham{18}

\begin{ex}%[2H1K3-6]
	Ông Bình dự định sử dụng hết $5,5\,\mathrm{m^2}$ kính để làm một bể cá bằng kính có dạng hình hộp chữ nhật không nắp, chiều dài gấp đôi chiều rộng (các mối ghép có kích thước không đáng kể). Bể cá có dung tích lớn nhất bằng bao nhiêu (làm tròn đến hàng phần trăm)?
	\choice
	{ $1{,}01\,\mathrm{m^3}$}
	{\True $1{,}17\,\mathrm{m^3}$}
	{ $1{,}51\,\mathrm{m^3}$}
	{ $1{,}40\,\mathrm{m^3}$}
	\loigiai{
		\immini{
			Gọi $x,2x,y$ với $x,y>0$  lần lượt là chiều rộng, chiều dài, chiều cao của bể cá.
			Theo giả thiết ta có: $$2\cdot 2xy+2\cdot xy+2x^2=5{,}5\Leftrightarrow 6xy+2x^2=5{,}5\Rightarrow y=\dfrac{5{,}5-2x^2}{6x}.$$
			Do $y>0$ nên $5,5 - 2x^2 >0 \Rightarrow 0<x<\dfrac{\sqrt{11}}{2}$.\\
			Thể tích bể cá là $$V(x)=2x^2y=2x^2\cdot \dfrac{5{,}5-2x^2}{6x}=-\dfrac{2}{3}{x^3}+\dfrac{11}{6}x.$$
			Khảo sát hàm số $V(x)=-\dfrac{2}{3}{x^3}+\dfrac{11}{6}x$ trên khoảng $\left( 0;\dfrac{\sqrt{11}}{2} \right) $
			\begin{itemize}
				\item [$\bullet$] $V'(x)=-2x^2+\dfrac{11}{6}$; $V'(x)=0\Leftrightarrow x=\sqrt{\dfrac{11}{12}}$.
				\item [$\bullet$] Bảng biến thiên:
				\begin{center}
					\begin{tikzpicture}
						\tkzTabInit[nocadre=True,lgt=1,espcl=3]
						{$x$ /1,$V'$ /0.6,$V$ /2}
						{$0$,$\sqrt{\frac{11}{12}}$,$\frac{\sqrt{11}}{2}$}
						\tkzTabLine{,+,$0$,-,}
						\tkzTabVar{-/, +/$y_0$,-/}
					\end{tikzpicture}
				\end{center}
			\end{itemize}
			Thể tích lớn nhất của bể cá là $V\left( \sqrt{\dfrac{11}{12}} \right)=1{,}17\,\mathrm{m^3}$.}{
			\begin{tikzpicture}
				\def\tls{.4}
				\path
				(0,0) coordinate (A)
				++ (0:4)coordinate (B)
				++ (30:2.3)coordinate (C)
				($(A)+(C)-(B)$)coordinate (D)
				\foreach \x in {A,B,C,D}{(\x)++(90:2.5) coordinate (\x_1)}
				;
				\draw[dashed]
				(A)--(D) node[pos=.5,sloped,above]{$x$}
				(D)--(C) node[pos=.4,sloped,above]{$2x$}
				(D)--(D_1) node[pos=.4, right]{$y$}
				;
				\draw
				(A)--(B)--(C)
				(A_1)--(B_1)--(C_1)--(D_1)--cycle
				(A)--(A_1) (B)--(B_1) (C)--(C_1)
				;
		\end{tikzpicture}}
	}
\end{ex} \dongcham{20}

\begin{ex}%[2D1T3-6]
	Người ta muốn xây một chiếc bể nước có hình dạng là	một khối hộp chữ nhật không nắp có thể tích bằng $\dfrac {500}{3}$ m$^3$. Biết đáy bể là một hình chữ nhật có chiều dài gấp đôi chiều rộng và giá thuê thợ xây là $700.000$ đồng/m$^2$. Để chi phí thuê nhân công ít nhất thì chi phí thuê nhân công là
	\choice
	{$120$ triệu đồng}	
	{\True $105$ triệu đồng}
	{$115$ triệu đồng}	
	{$110$ triệu đồng}
	\loigiai{
		Gọi $x,y$ lần lượt là chiều rộng và chiều cao của bể cá (điều kiện $x,y>0$ ).
		\immini{Với giả thiết của bài toán, thể tích bể cá là $$V=2x^2y=\dfrac {500}{3}\Rightarrow y=\dfrac {250}{3x^2}.$$
			Để chi phí thuê nhân công ít nhất thì tổng diện tích các mặt của bể cá phải nhỏ nhất. Tổng diện tích các mặt của bể cá} 
		{\begin{tikzpicture}[scale=0.8, font=\footnotesize, line join=round, line cap=round, >=stealth]
				\tkzDefPoints{0/0/A,-1.3/-1.1/B,2/-1.1/C}
				\coordinate (D) at ($(A)+(C)-(B)$);
				\coordinate (A') at ($(A)+(0,2.5)$);
				\tkzDefPointsBy[translation=from A to A'](B,C,D){B'}{C'}{D'}
				\tkzDrawPolygon(A',B',B,C,D,D')
				\tkzDrawSegments(B',C' C',D' C,C')
				\tkzDrawSegments[dashed](A,B A,D A,A')
				\tkzDrawPoints[fill=black](A,B,D,C,A',B',C',D')
				\tkzLabelSegment[left](B',B){$y$}
				\tkzLabelSegment[below](B,C){$2x$}
				\tkzLabelSegment[right](A,B){$x$}
		\end{tikzpicture}}
		$S=2xy+2\cdot 2xy+2x^2=6xy+2x^2=\dfrac {500}{x}+2x^2$.\\
		Xét hàm số $S(x)=\dfrac {500}{x}+2x^2$ trên khoảng $(0;+\infty)$.\\
		$\Rightarrow S'(x)=-\dfrac {500}{x^2}+4x$.\\
		$S'(x)=0\Leftrightarrow -500+4x^3=0\Leftrightarrow x=5$.\\
		Bảng biến thiên
		\begin{center}
			\begin{tikzpicture}[scale=1, font=\footnotesize, line join=round, line cap=round, >=stealth]
				\tkzTabInit[nocadre=false,lgt=1.2,espcl=2, deltacl=0.5]
				{$x$/0.6,$S’(x)$/0.6,$S(x)$/1.5}
				{$0$,$5$,$+\infty$}
				\tkzTabLine{,-,z,+,}
				\tkzTabVar{+/$+\infty$,-/$150$,+/$+\infty$}
			\end{tikzpicture}
		\end{center}
		Do đó $\min S=150$ tại $x=5$. \\
		Khi đó, chi phí thuê nhân công là $150\cdot 700000=105$ triệu đồng.\\Vậy chi phí thuê nhân công ít nhất là $105$ triệu đồng.}
\end{ex}
\dongcham{13}
\begin{ex}%[2D1V3-6]
	Từ một tấm bìa hình chữ nhật có chiều rộng $30 \mathrm{~cm}$ và chiều dài $80 \mathrm{~cm}$ (Hình a), người ta cắt ở bốn góc bốn hình vuông có cạnh $x(\mathrm{~cm})$ với $5 \leq x \leq 10$ và gấp lại để tạo thành chiếc hộp có dạng hình hộp chữ nhật không nắp như Hình b. Tìm $x$ để thể tích chiếc hộp là lớn nhất (kết quả làm tròn đến hàng phần trăm).
	\begin{center}
		\begin{tikzpicture}[line join=round, line cap=round,scale=0.9]
			\coordinate (A) at (0,3);
			\coordinate (B) at (5,3);
			\coordinate (D) at (0,0);
			\coordinate (C) at ($(B)+(D)-(A)$);
			\draw(A)--(B)--(C)--(D)--cycle;
			\draw (0,0) rectangle (1,1) (A) rectangle (1,2) (B) rectangle (4,2) (4,1) rectangle (C);
			\draw[dashed] (1,1) rectangle (4,2);
			%	\foreach \i/\g in {A/90,B/90,C/-90,D/-90}{\draw[fill=black](\i) circle (1pt) ($(\i)+(\g:3mm)$) node[scale=1]{$\i$};}
			\draw (0,.5) node [left] {$x$};
			\draw (.5,0) node [below] {$x$};
			\draw (0,2.5) node [left] {$x$};
			\draw (0.5,3) node [above] {$x$};
			%%%%%%%%%
			\draw (4.5,0) node [below] {$x$};
			\draw (5,0.5) node [right] {$x$};
			\draw (5,2.5) node [right] {$x$};
			\draw (4.5,3) node [above] {$x$};
			%%%%%%%%
			\draw[<->] (-1,0)--(-1,3) node[above,midway,sloped] {$30$cm};
			\draw[<->] (0,-1)--(5,-1) node[above,midway] {$80$cm};
			\path (current bounding box.south) node[below, black]{a)}; %dưới
		\end{tikzpicture}
		\hspace*{1cm}
		\begin{tikzpicture}[scale=0.9, font=\footnotesize, line join=round, line cap=round, >=stealth]
			\def\bc{3} % cạnh BC
			\def\ba{1} % cạnh BA
			\def\h{1.5} % đường cao
			\def\gocnghieng{90} % góc nghiêng
			\def\gocB{35} % góc B của đáy
			\coordinate (B) at (0,0);
			\coordinate (A) at (\gocB:\ba);
			\coordinate (C) at (\bc,0);
			\coordinate (D) at ($(C)-(B)+(A)$);
			\coordinate (A') at ($(A)+(\gocnghieng:\h)$);
			\coordinate (B') at ($(B)-(A)+(A')$);
			\coordinate (C') at ($(C)-(A)+(A')$);
			\coordinate (D') at ($(D)-(A)+(A')$);
			\draw (B')--(B)--(C)--(D)--(D')--(A')--(B')--(C')--(D') (C)--(C');
			\draw[dashed] (A')--(A)--(D) (A)--(B);
			\path (current bounding box.south) node[below, black]{b)}; %dưới
		\end{tikzpicture}
	\end{center}
	\choice
	{\True $x=\dfrac{20}{3} \mathrm{~cm}$}
	{$x=\dfrac{20}{7} \mathrm{~cm}$}
	{$x=\dfrac{25}{3} \mathrm{~cm}$}
	{$x=\dfrac{25}{7} \mathrm{~cm}$}
	\loigiai{
		Thể tích chiếc hộp là $V(x)=x(30-2 x)(80-2 x)=2400 x-220 x^2+4 x^3$ với $5 \leq x \leq 10$.\\
		Ta có: $V'(x)=12 x^2-440 x+2400$;\\
		$V'(x)=0 \Leftrightarrow x=\dfrac{20}{3}$ hoặc $x=30$ (loại vì không thuộc $[5 ; 10]$);
		\begin{center}
			$V(5)=7000 ; V\left(\dfrac{20}{3}\right)=\dfrac{200000}{27} ; V(10)=6000$.
		\end{center}
		Do đó $\max \limits_{[5 ; 10]} V(x)=\dfrac{200000}{27}$ khi $x=\dfrac{20}{3}$.
		Vậy để thể tích chiếc hộp là lớn nhất thì $x=\dfrac{20}{3} \mathrm{~cm}$.}
\end{ex}
\dongcham{13}
\begin{ex}%[2D1K3]
	Một sợi dây có chiều dài là $6$ m, được chia thành $2$ phần. Phần thứ nhất được uốn thành hình tam giác đều, phần thứ hai uốn thành hình vuông. Hỏi độ dài của cạnh hình tam giác đều bằng bao nhiêu để tổng diện tích $2$ hình thu được là nhỏ nhất?
	\begin{center}
		\begin{tikzpicture}[scale=0.8,>=stealth]
			\draw(0,0)--(7,0);
			\draw (0,0)circle (1pt)(7,0) circle (1pt)(3,0) circle (1pt);
			\draw[->](1.5,-0.3)--(1.5,-0.7);
			\draw[->](5,-0.3)--(5,-0.7);
			\draw(4.5,-0.9)--(5.5,-0.9)--(5.5,-1.9)--(4.5,-1.9)--(4.5,-0.9);
			\draw(1.5,-0.9)--(2,-1.9)--(1,-1.9)--(1.5,-0.9);
		\end{tikzpicture}
	\end{center}
	\choice
	{$\dfrac{12}{4+\sqrt{3}}$ m}
	{$\dfrac{18\sqrt{3}}{4+\sqrt{3}}$ m}
	{$\dfrac{36\sqrt{3}}{4+\sqrt{3}}$ m}
	{\True $\dfrac{18}{9+4\sqrt{3}}$ m}
	\loigiai{
		Gọi độ dài cạnh hình tam giác đều là $x$ (m). Khi đó độ dài cạnh hình vuông là $\dfrac{6-3x}{4}$.\\
		Tổng diện tích khi đó là $S =\dfrac{\sqrt{3}}{4}x^2 + \left(\dfrac{{6 - 3x}}{4}\right)^2 =\dfrac{1}{16}\left[\left(9+4\sqrt{3}\right)x^2 - 36x + 36 \right)]$.\\
		Xét hàm số $f(x)=\left(9+4\sqrt{3}\right)x^2-36x+36, x\in(0;6)$.\\
		Ta có $f(x)$ là tam thức bậc $2$ có $-\dfrac{b}{2a}=\dfrac{18}{9+4\sqrt{3}} \in (0;6)$ và $a>0$.\\
		Suy ra $f(x)$ đạt giá trị nhỏ nhất tại
		$x=-\dfrac{b}{2a}\dfrac{18}{9+4\sqrt{3}}$.\\
		Vậy diện tích nhỏ nhất khi $x=\dfrac{18}{9+4\sqrt{3}}$ m.
	}
\end{ex}
\dongcham{14}
\begin{ex}
	Một doanh nghiệp tư nhân $A$ chuyên kinh doanh xe gắn máy các loại. Hiện nay doanh nghiệp đang tập trung vào chiến lược kinh doanh xe $X$ với chi phí mua vào một chiếc là 27 triệu đồng và bán ra với giá 31 triệu đồng. Với giá bán này, số lượng xe mà khách hàng đã mua trong một năm là 600 chiếc. Nhằm mục tiêu đẩy mạnh hơn nữa lượng tiêu thụ dòng xe đang bán chạy này, doanh nghiệp dự định giảm giá bán. Bộ phận nghiên cứu thị trường ước tính rằng nếu giảm 1 triệu đồng mỗi chiếc xe thì số lượng xe bán ra trong một năm sẽ tăng thêm 200 chiếc. Hỏi theo đó, giá bán mới là bao nhiêu thì lợi nhuận thu được cao nhất?
	\choice
	{$30$ triệu đồng}
	{\True $30,5$ triệu đồng}
	{$29,5$ triệu đồng}
	{$32$ triệu đồng}
	\loigiai{
		Gọi giá bán mới là $x$ (triệu đồng) với $x \in [27;31]$.\\
		Khi đó số xe bán ra là $600+(31-x) \cdot 200$.\\
		Lợi nhuận thu được là 
		\begin{eqnarray*}
			f(x) &=& [600+(31-x) \cdot 200](x-27)\\
			&=& (-200x+6800)(x-27)\\
			&=& -200x^2+12200x-183600\\
			&=& -200\left(x-\dfrac{61}{2}\right)^2+2450\\
			&\leq&2450.
		\end{eqnarray*}
		Vậy giá bán mới là $30,5$ triệu đồng thì lợi nhuận thu được là lớn nhất là $2\,450$ (triệu đông).
	}
\end{ex}
\dongcham{14}
\Closesolutionfile{ans}
\ind{PHẦN II.} \inden{Câu trắc nghiệm đúng sai. Trong mỗi ý a), b), c), d) ở mỗi câu, học sinh chọn đúng hoặc sai.}\\
\Opensolutionfile{ans}[ans/2D1-B2-d3-2]

\begin{ex}
	Người ta bơm xăng vào bình xăng của một xe ô tô. Biết rằng thể tích $V$ (lít) của lượng xăng trong bình xăng tính theo thời gian bơm xăng $t$ (phút) được cho bởi công thức $$V(t)=300(t^2-t^3)+4 \text{ với } 0\le t\le 0{,}5.$$
Gọi $V'(t)$ là tốc độ tăng thể tích tại thời điểm $t$ với $0\le t\le 0{,}5$.
\choiceTF
{Lượng xăng trong bình ban đầu là $1$ lít}
{\True Lượng xăng lớn nhất bơm vào bình xăng là $41{,}5$ lít}
{$V'(t)=300(2t-3t^2)+4$, với $0\le t\le 0{,}5$}
{\True Xăng chảy vào bình xăng vào thời điểm ở giây thứ $30$ có tốc độ tăng thể tích là lớn nhất}
	\loigiai{
		\begin{enumerate}[a)]
			\item Số xăng trong bình ban đầu là $V(0)=4$ lít.
			\item Lượng xăng lớn nhất bơm vào bình xăng là $V=V\left(\dfrac{1}{2}\right)=41{,}5$ lít.
			\item Xét hàm số $V(t)=300(t^2-t^3)+4 \text{ với } 0\le t\le 0{,}5.$\\
			Đạo hàm $V'(t)=300t(2-3t)$.\\
			\item Cho $V'(t)=0 \Leftrightarrow 300t(t-3t)=0 \Leftrightarrow \hoac{&t=0\in[0;0{,}5]\\&t=\dfrac{2}{3}\notin[0;0{,5}].}$\\
			Các giá trị $V(0)=4$, $V\left(\dfrac{1}{2}\right)=41{,}5$.\\
			Xăng chảy vào bình xăng vào thời điểm ở giây thứ $30$ có tốc độ tăng thể tích là lớn nhất.
		\end{enumerate}
	}
\end{ex}
\dongcham{20}
\begin{ex}
	Tại một xí nghiệp chuyên sản xuất vật liệu xây dựng, nếu trong một ngày xí nghiệp sản xuất $x$ (m$^3$) sản phẩm thì phải bỏ ra các khoản chi phí bao gồm: $4$ triệu đồng chi phí cố định; $0{,}2$ triệu đồng chi phí cho mỗi mét khối sản phẩm và $0{,}001 x^2$ triệu đồng chi phí bảo dưỡng máy móc. Biết rằng, mỗi ngày xí nghiệp sản xuất được tối đa $100$ m$^3$ sản phẩm. Goi $C(x)$ là tổng chi phí để xí nghiệp sản xuất $x$ (m$^3$) sản phẩm trong một ngày và $\overline{C}$ là chi phí trung bình  trên mỗi mét khối sản phẩm.
	\choiceTF
	{$C=0{,}2 x+0{,}001 x^2 \quad \text { với } 0 \leq x \leq 100$}
	{\True Tổng chi phí khi sản xuất 100 m$^3$ sản phẩm là 34 triệu đồng}
	{\True $\overline{C}=0{,}001 x+\dfrac{4}{x}+0{,}2 \quad\text { với } 0<x \leq 100$}
	{\True $\overline{C}$ có giá trị thấp nhất bằng 0,326 triệu đồng (\textit{kết quả làm tròn 3 chữ số thập phân})}
	\loigiai{
		\begin{enumerate}
			\item Tổng chi phí (triệu đồng) để xí nghiệp sản xuất $x$ (m$^3$) sản phẩm trong một ngày là
			$$
			C=C(x)=4+0{,}2 x+0{,}001 x^2 \text { với } 0 \leq x \leq 100.
			$$
			\item Thay $x=100$ vào hàm $C(x)$, ta được kết quả 34 (triệu đồng).
			\item Chi phí trung bình (triệu đồng) trên mỗi mét khối sản phẩm là
			$$
			\overline{C}=\overline{C}(x)=\dfrac{C(x)}{x}=\dfrac{4+0{,}2 x+0{,}001 x^2}{x}=0{,}001 x+\dfrac{4}{x}+0{,}2 \text { với } 0<x \leq 100.
			$$
			\item Ta có $\bar{C}'(x)=0{,}001-\dfrac{4}{x^2}$;
			$$
			\overline{C}'(x)=0 \Leftrightarrow 0{,}001-\dfrac{4}{x^2}=0 \Leftrightarrow x^2=4\,000 \Leftrightarrow x=20 \sqrt{10} \in(0 ; 100].
			$$
			
			Ta có $\overline{C}(20 \sqrt{10})=\dfrac{\sqrt{10}}{25}+\dfrac{1}{5} \approx 0,326$.\\
			Bảng biến thiên
			\begin{center}
				\begin{tikzpicture}
					\tikzset{double style/.append style={double distance=2pt}}
					\tkzTabInit[lgt=1.2, espcl=2]
					{$x$/0.6,$\overline{C'}(x)$/0.6,$\overline{C}(x)$/2.5}{$0$,$20\sqrt{10}$,$100$}
					\tkzTabLine{,-,0,+,}
					\tkzTabVar{+/$+\infty$,-/$\dfrac{\sqrt{10}}{25}+\dfrac{1}{5}$,+/$0{,}34$}
				\end{tikzpicture}
			\end{center}
			Từ bảng biến thiên, ta thấy chi phí trung bình thấp nhất là $\bar{C}(20 \sqrt{10}) \approx 0{,}326$ (triệu đồng/m$^3$ sản phẩm), đạt được khi $x=20 \sqrt{10} \approx 63$ (m$^3$).
		\end{enumerate}
	}
\end{ex}
\dongcham{20}
\begin{ex}
	Nhà máy $A$ chuyên sản xuất một loại sản phẩm cung cấp cho nhà máy $B$. Hai nhà máy thoả thuận rằng, hằng tháng $A$ cung cấp cho $B$ số lượng sản phẩm theo đơn đặt hàng của $B$ (tối đa $100$ tấn sản phẩm). Nếu số lượng đặt hàng là $x$ tấn sản phẩm thì giá bán cho mỗi tấn sản phẩm là $P(x)=45-0{,}001 x^2$ (triệu đồng). Chi phí để $A$ sản xuất $x$ tấn sản phẩm trong một tháng là $C(x)=100+30 x$ (triệu đồng) (gồm $100$ triệu đồng chi phí cố định và $30$ triệu đồng cho mỗi tấn sản phẩm).
	\choiceTF
	{\True Chi phí để  A sản xuất 10 tấn sảm phẩm trong một tháng là 400 triệu đồng}
	{Số tiền  A thu được khi bán 10 tấn sản phẩm cho B là 600 triệu đồng}
	{\True Lợi nhuận mà A thu được khi bán $x$ tấn sản phẩm ($0\le x \le 100)$ cho  B là $-0{,}001 x^3+15 x-100$}
	{\True A bán cho $B$ khoảng 70,7 tấn sản phẩm mỗi tháng thì thu được lợi nhuận lớn nhất}
	\loigiai{
		\begin{enumerate}[a)]
			\item Chi phí để  A sản xuất 10 tấn sảm phẩm trong một tháng là $C(10)=100+30\cdot 10=400$ (triệu)
			\item Số tiền mà $A$ thu được (gọi là doanh thu) từ việc bán $x$ tấn sản phẩm $(0 \leq x \leq 100)$ cho $B$ là
			$$
			R(x)=x \cdot P(x)=x\left(45-0{,}001 x^2\right)=45 x-0{,}001 x^3 \text { (triệu đồng). }
			$$
			Thay $x=10$, ta được $R(10)=449$ (triệu đồng).
			\item Lợi nhuận (triệu đồng) mà $A$ thu được là
			$$
			P(x)=R(x)-C(x)=x\left(45-0{,}001 x^2\right)-(100+30 x)=-0{,}001 x^3+15 x-100.
			$$
			\item Xét hàm số $P(x)=-0{,}001 x^3+15 x-100$ với $0 \leq x \leq 100$, ta có
			$$
			\begin{aligned}
				& P'(x)=-0{,}003 x^2+15; \\
				& P'(x)=0 \Leftrightarrow-0{,}003 x^2+15=0 \Leftrightarrow x^2=5\,000 \Leftrightarrow x=50 \sqrt{2} \in[0 ; 100].
			\end{aligned}
			$$
			
			Ta có $P(0)=-100$; $P(50 \sqrt{2})=500 \sqrt{2}-100 \approx 607$; $P(100)=400$.\\
			Bảng biến thiên
			\begin{center}
				\begin{tikzpicture}
					\tkzTabInit[lgt=1, espcl=4]
					{$x$/1,$y'$/0.6,$y$/3}{$0$,$50\sqrt{2}$,$100$}
					\tkzTabLine{,+,0,-,}
					\tkzTabVar{-/$100$,+/$500\sqrt{2}-100$,-/$400$}
				\end{tikzpicture}
			\end{center}
			
			Từ bảng biến thiên, ta có $\max \limits_{[0 ; 100]} P=P(50 \sqrt{2})=500 \sqrt{2}-100 \approx 607$.\\
			Vậy $A$ thu được lợi nhuận lớn nhất khi bán $50 \sqrt{2} \approx 70{,}7$ tấn sản phẩm cho $B$ mỗi tháng và lợi nhuận lớn nhất thu được khoảng $607$ triệu đồng.
		\end{enumerate}
	}
\end{ex}
\dongcham{20}
\Closesolutionfile{ans}
%%Bài 3. Tiệm cận
% \setcounter{section}{2}
\section{ĐƯỜNG TIỆM CẬN CỦA ĐỒ THỊ HÀM SỐ}
\subsection{LÝ THUYẾT CẦN NHỚ}
\subsubsection{Đường tiệm cận ngang (TCN):}
\begin{enumerate}[\iconMT]
	\item \indam{Định nghĩa:} Đường thẳng $y=m$ được gọi là một \inden{đường tiệm cận ngang} (hay \inden{tiệm cận ngang}) của đồ thị hàm số $y=f(x)$ nếu 
	$$\lim\limits_{x \rightarrow-\infty} f(x)=m \text{ hoặc }\lim\limits_{x \rightarrow+\infty} f(x)=m.$$
Đường thẳng $y=m$ là tiệm cận ngang của đồ thị hàm số $y=f(x)$ được minh hoạ như hình bên dưới\\
	\begin{tikzpicture}[scale=1,>=stealth, font=\footnotesize, line join=round, line cap=round]
		\def\xmin{-4} \def\xmax{2}
		\def\ymin{-0.5} \def\ymax{3}
		%\draw[color=gray!50,dashed] (\xmin,\ymin) grid (\xmax,\ymax);
		\draw[->] (\xmin,0)--(\xmax,0) node [below]{$x$};
		\draw[->] (0,\ymin)--(0,\ymax) node [left]{$y$};
		\fill (0,0) circle (1pt) node[shift={(-135:2.5mm)}]{$O$};
		\node at (current bounding box.south) [below=-2pt] {a) $\lim\limits_{x \rightarrow-\infty} f(x)=m$};
		\clip (\xmin+0.1,\ymin+0.1) rectangle (\xmax-0.1,\ymax-0.1);
		\draw[red,thick,smooth,samples=300,domain=\xmin:\xmax]
		(-4,0.9)..controls +(0:2) and +(180:0.5)
		..(-1.5,0.5)..controls +(0:0.5) and +(180:0.5)
		..(-0.3,1.4)..controls +(0:0.5) and +(135:1)
		..(1.8,0.3);
		\draw [blue](\xmin,1)--(\xmax,1);
		\path[blue] (-3,1)node[above]{$y=m$};
		\path[red] (0,1.3)node[above left]{$y=f(x)$};
		\fill (0,1) circle (1pt) node[shift={(-135:3mm)}]{$m$};
	\end{tikzpicture}\hspace*{.5cm}
	\begin{tikzpicture}[scale=1,>=stealth, font=\footnotesize, line join=round, line cap=round]
		\def\xmin{-1.5} \def\xmax{4}
		\def\ymin{-0.5} \def\ymax{3}
		%\draw[color=gray!50,dashed] (\xmin,\ymin) grid (\xmax,\ymax);
		\draw[->] (\xmin,0)--(\xmax,0) node [below]{$x$};
		\draw[->] (0,\ymin)--(0,\ymax) node [left]{$y$};
		\fill (0,0) circle (1pt) node[shift={(-135:2.5mm)}]{$O$};
		\node at (current bounding box.south) [below=-2pt] {b) $\lim\limits_{x \rightarrow+\infty} f(x)=m$};
		\clip (\xmin+0.1,\ymin+0.1) rectangle (\xmax-0.1,\ymax-0.1);
		\draw[red,thick,smooth,samples=300,domain=\xmin:\xmax]
		(-1,3)..controls +(-80:1) and +(170:1)
		..(0.5,1.1)..controls +(170:-1) and +(180:-0.5)
		..(3.9,0.8);
		\draw [blue](\xmin,0.7)--(\xmax,0.7);
		\path[blue] (4,0.7)node[below left]{$y=m$};
		\path[red] (0.5,1)node[above right]{$y=f(x)$};
		\fill (0,0.7) circle (1pt) node[shift={(-135:3mm)}]{$m$};
	\end{tikzpicture}
	\item \indam{Các bước tìm TCN:}
	\begin{boxdn}
		\begin{listEX}[1]
			\item [\ding{172}] Tính $\lim \limits_{x \to +\infty} f(x)$ và $\lim \limits_{x \to -\infty} f(x)$.
			\item [\ding{173}] Xem ở "vị trí" nào ra kết quả hữu hạn thì ta kết luận có tiệm cận ngang ở "vị trí" đó.
		\end{listEX}
	\end{boxdn}
\end{enumerate}
\subsubsection{Đường tiệm cận đứng (TCĐ)}
\begin{enumerate}[\iconMT]
	\item \indam{Định nghĩa:}	Đường thẳng $x=a$ được gọi là một \inden{đường tiệm cận đứng} (hay \inden{tiệm cận đứng}) của đồ thị hàm số $y=f(x)$ nếu ít nhất một trong các điều kiện sau thoả mãn:		
	$$
	\lim\limits_{x \rightarrow a^{-}} f(x)=+\infty,\,\, \lim\limits_{x \rightarrow a^{+}} f(x)=+\infty,\,\, \lim\limits_{x \rightarrow a^{-}} f(x)=-\infty,\,\, \lim\limits_{x \rightarrow a^{+}} f(x)=-\infty \text {. }
	$$
		Đường thẳng $x=a$ là tiệm cận đứng của đồ thị hàm số $y=f(x)$ được minh hoạ như hình bên dưới.\\
		\begin{center}
		\begin{tikzpicture}[scale=.7,>=stealth, font=\footnotesize, line join=round, line cap=round]
			%Hình a
			\def\xmin{-2.2} \def\xmax{3.5}
			\def\ymin{-2} \def\ymax{2} 
			%\draw[color=gray!50,dashed] (\xmin,\ymin) grid (\xmax,\ymax); 
			\draw[->] (\xmin,0)--(\xmax,0) node [below]{$x$};
			\draw[->] (0,\ymin)--(0,\ymax) node [left]{$y$};
			\fill (0,0) circle (1pt) node[shift={(-45:2.5mm)}]{$O$};
			\draw (2.1,\ymin)--(2.1,\ymax)node[below right]{$x=a$};
			\fill (2.1,0) circle (1pt) node[shift={(-45:3mm)}]{$a$};
			%\clip (\xmin+0.1,\ymin+0.1) rectangle (\xmax-0.1,\ymax-0.1);
			\draw[red] (-2,-1)..controls +(80:0.5) and +(0:-.5)..(-1,0.5)node[above]{$y=f(x)$}
			..controls +(0:0.5) and +(180:0.5)..(0.5,-1.5)
			..controls +(0:0.5) and +(87:-0.2)..(1.6,0)
			..controls +(87:-.2) and +(90:-0.2)
			..(2,1.85);
			\node at (current bounding box.south) [below=-2pt] {a) $\lim\limits_{x \rightarrow a^{-}} f(x)=+\infty$};
		\end{tikzpicture}
		\begin{tikzpicture}[scale=.7,>=stealth, font=\footnotesize, line join=round, line cap=round]
			%Hình b
			\def\xmin{-1.2} \def\xmax{4}
			\def\ymin{-2} \def\ymax{2} 
			%\draw[color=gray!50,dashed] (\xmin,\ymin) grid (\xmax,\ymax); 
			\draw[->] (\xmin,0)--(\xmax,0) node [below]{$x$};
			\draw[->] (0,\ymin)--(0,\ymax) node [left]{$y$};
			\fill (0,0) circle (1pt) node[shift={(-45:2.5mm)}]{$O$};
			\draw (1,\ymin)node[above right]{$x=a$}--(1,\ymax);
			\fill (1,0) circle (1pt) node[shift={(-135:3mm)}]{$a$};
			\path[red] (1.25,1)node[above right]{$y=f(x)$};
			%\clip (\xmin+0.1,\ymin+0.1) rectangle (\xmax-0.1,\ymax-0.1);
			\draw[red] (1.2,2)..controls +(80:0) and +(0:-1.4)..(2.5,-0.8)
			..controls +(0:0.1) and +(-80:-0.6)
			..(3.5,-1.5);
			\node at (current bounding box.south) [below=-2pt] {b) $\lim\limits_{x \rightarrow a^{+}} f(x)=+\infty$};		
		\end{tikzpicture}\\
		\begin{tikzpicture}[scale=.7,>=stealth, font=\footnotesize, line join=round, line cap=round]
			%Hình c
			\def\xmin{-2.2} \def\xmax{3.5}
			\def\ymin{-2} \def\ymax{2} 
			%\draw[color=gray!50,dashed] (\xmin,\ymin) grid (\xmax,\ymax); 
			\draw[->] (\xmin,0)--(\xmax,0) node [below]{$x$};
			\draw[->] (0,\ymin)--(0,\ymax) node [left]{$y$};
			\fill (0,0) circle (1pt) node[shift={(-45:2.5mm)}]{$O$};
			\draw (2,\ymin)--(2,\ymax)node[below right]{$x=a$};
			\fill (2,0) circle (1pt) node[shift={(-45:3mm)}]{$a$};
			\path[red] (-2.25,1.2)node[below right]{$y=f(x)$};
			%\clip (\xmin+0.1,\ymin+0.1) rectangle (\xmax-0.1,\ymax-0.1);
			\draw[red] (-2,1.4)..controls +(-10:-0.2) and +(-55:-.7)
			..(1.3,0.65)..controls +(-50:0.4) and +(-90:0)
			..(1.8,-2)
			;
			\node at (current bounding box.south) [below=-2pt] {c) $\lim\limits_{x \rightarrow a^{-}} f(x)=-\infty$};		
		\end{tikzpicture}
		\begin{tikzpicture}[scale=.7,>=stealth, font=\footnotesize, line join=round, line cap=round]
			%Hình d
			\def\xmin{-2.2} \def\xmax{3.5}
			\def\ymin{-2} \def\ymax{2} 
			%\draw[color=gray!50,dashed] (\xmin,\ymin) grid (\xmax,\ymax); 
			\draw[->] (\xmin,0)--(\xmax,0) node [below]{$x$};
			\draw[->] (0,\ymin)--(0,\ymax) node [left]{$y$};
			\fill (0,0) circle (1pt) node[shift={(-135:2.5mm)}]{$O$};
			\draw (.6,\ymin)--(.6,\ymax)node[below right]{$x=a$};
			\fill (.6,0) circle (1pt) node[shift={(-135:3mm)}]{$a$};
			%\clip (\xmin+0.1,\ymin+0.1) rectangle (\xmax-0.1,\ymax-0.1);
			\draw[red] (0.7,-2)..controls +(85:0.2) and +(180:0.2)
			..(1.2,-0.3)..controls +(0:0.2) and +(180:0.2)
			..(1.7,-0.6)..controls +(0:0.4) and +(90:0)
			..(2.5,2)
			;
			\node at (current bounding box.south) [below=-2pt] {d) $\lim\limits_{x \rightarrow a^{+}} f(x)=-\infty$};		
		\end{tikzpicture}
	\end{center}
	\item \indam{Các bước tìm TCĐ:}
	\begin{boxdn}
		\begin{listEX}[1]
			\item [\ding{172}] Tìm nghiệm của mẫu, giả sử nghiệm đó là $x=x_0$.
			\item [\ding{173}] Tính giới hạn một bên tại $x_0$. Nếu xảy ra $\lim \limits_{x \to x_0^{-}} f(x) =\infty \text{ hoặc} \lim \limits_{x \to x_0^{+}} f(x) =\infty$
			thì ta kết luận $x=x_0$ là đường tiệm cận đứng.
		\end{listEX}
	\end{boxdn}
\end{enumerate}
\subsubsection{Đường tiệm cận xiên}
\begin{enumerate}[\iconMT]
	\item \indam{Định nghĩa:} Đường thẳng $y=ax+b$, $a \neq 0$, được gọi là \inden{đường tiệm cận xiên} (hay \inden{tiệm cận xiên}) của đồ thị hàm số $y=f(x)$ nếu 
	$$\lim\limits_{x \rightarrow-\infty}[f(x)-(ax+b)]=0 \text{ hoặc }\lim\limits_{x \rightarrow+\infty}[f(x)-(ax+b)]=0.$$
	Đường thẳng $y=ax+b$ là tiệm cận xiên của đồ thị hàm số $y=f(x)$ được minh hoạ như hình bên dưới:\\	
		\begin{tikzpicture}[scale=1,>=stealth, font=\footnotesize, line join=round, line cap=round]
			\def\xmin{-4} \def\xmax{2.5}
			\def\ymin{-0.5} \def\ymax{3}
			%\draw[color=gray!50,dashed] (\xmin,\ymin) grid (\xmax,\ymax);
			\draw[->] (\xmin,0)--(\xmax,0) node [below]{$x$};
			\draw[->] (0,\ymin)--(0,\ymax) node [left]{$y$};
			\fill (0,0) circle (1pt) node[shift={(-135:2.5mm)}]{$O$};
			\node at (current bounding box.south) [below=-2pt] {a) $\lim\limits_{x \rightarrow-\infty}\left[f(x)-(ax+b)\right]=0$};
			\clip (\xmin+0.1,\ymin+0.1) rectangle (\xmax-0.1,\ymax-0.1);
			\draw[red,thick,smooth,samples=300,domain=\xmin:\xmax]
			(-3.8,-0.6)..controls +(34:0.5) and +(180:.75)
			..(-0.2,1.2)..controls +(0:0.75) and +(180:.75)
			..(1,0.3)..controls +(0:0.5) and +(80:0)
			..(2.2,1);
			\draw[blue,smooth,samples=300,domain=\xmin:\xmax] plot(\x,{2/3*(\x)+2});
			\path[blue] (-3,0)--(0,2)node[below,sloped,pos=1.3]{$y=ax+b$};
			\path[red] (0.5,1)node[above right]{$y=f(x)$};
		\end{tikzpicture}\hspace{.5cm}
		\begin{tikzpicture}[scale=1,>=stealth, font=\footnotesize, line join=round, line cap=round]
			\def\xmin{-3.5} \def\xmax{3}
			\def\ymin{-0.5} \def\ymax{3}
			%\draw[color=gray!50,dashed] (\xmin,\ymin) grid (\xmax,\ymax);
			\draw[->] (\xmin,0)--(\xmax,0) node [below]{$x$};
			\draw[->] (0,\ymin)--(0,\ymax) node [left]{$y$};
			\fill (0,0) circle (1pt) node[shift={(-135:2.5mm)}]{$O$};
			\node at (current bounding box.south) [below=-2pt] {a) $\lim\limits_{x \rightarrow+\infty}\left[f(x)-(ax+b)\right]=0$};
			\clip (\xmin+0.1,\ymin+0.1) rectangle (\xmax-0.1,\ymax-0.1);
			\draw[red,thick,smooth,samples=300,domain=\xmin:\xmax]
			(-3,0.8)..controls +(60:0.5) and +(180:.75)
			..(-1.5,2)..controls +(0:.5) and +(180:.75)
			..(0.5,1.3)..controls +(0:.75) and +(-160:.5)
			..(2.8,1.8);
			\draw[blue,smooth,samples=300,domain=\xmin:\xmax] plot(\x,{1/3*(\x)+0.75});
			\path[blue] (-3,-0.25)--(0,0.75)node[below,sloped,pos=1.6]{$y=ax+b$};
			\path[red] (-2.5,2)node[above right]{$y=f(x)$};
		\end{tikzpicture}
	\item \indam{Các bước tìm TCX y = ax + b:}
	Ta xác định hệ số của $a$ và $b$ trong 2 trường hợp sau:
	\begin{boxdn}
		\begin{listEX}[1]
			\item [\ding{172}] Tính $a=\lim\limits_{x \rightarrow+\infty} \dfrac{f(x)}{x}$, $b=\lim\limits_{x \rightarrow+\infty}[f(x)-ax]$.
			\item [\ding{173}] Tính $a=\lim\limits_{x \rightarrow-\infty} \dfrac{f(x)}{x}$, $b=\lim\limits_{x \rightarrow-\infty}[f(x)-ax]$.
		\end{listEX}
	\end{boxdn}
\end{enumerate}
\subsection{PHÂN LOẠI VÀ PHƯƠNG PHÁP GIẢI TOÁN}
\begin{dang}{Bài toán tìm tiệm cận đứng và tiệm cận ngang của đồ thị hàm số}
	Cho hàm số $y=f(x)$. Để tìm tiệm cận đứng và tiệm cận ngang, ta làm như sau:
	\begin{enumerate}[\iconCH]
		\item \indamm{Các bước tìm tiệm cận đứng:}
		\begin{listEX}[1]
			\item [\ding{172}] Tìm nghiệm của mẫu, giả sử nghiệm đó là $x=x_0$.
			\item [\ding{173}] Tính giới hạn một bên tại $x_0$. Nếu xảy ra $\lim \limits_{x \to x_0^{-}} f(x) =\infty \text{ hoặc} \lim \limits_{x \to x_0^{+}} f(x) =\infty$
			thì ta kết luận $x=x_0$ là đường tiệm cận đứng.
		\end{listEX}
		\item \indamm{Các bước tìm tiệm cận ngang:}
		\begin{listEX}[1]
			\item [\ding{172}] Tính $\lim \limits_{x \to +\infty} f(x)$ và $\lim \limits_{x \to -\infty} f(x)$.
			\item [\ding{173}] Xem ở "vị trí" nào ra kết quả hữu hạn thì ta kết luận có tiệm cận ngang ở "vị trí" đó.
		\end{listEX}
		\item \indamm{Lưu ý:} Đồ thị hàm số $y=\dfrac{ax+b}{cx+d}$ luôn có TCĐ $x=-\dfrac{d}{c}$ và TCN: $y=\dfrac{a}{c}$.
	\end{enumerate}
\end{dang}
\boxmini{BÀI TẬP TỰ LUẬN}
\begin{vd}
	Xác định tiệm cận đứng và tiệm cận ngang của đồ thị hàm số cho bởi công thức sau:
	\begin{enumEX}[a)]{4}
		\item $y=\dfrac{2x-1}{x+1}$;
		\item $y=\dfrac{2 x-3}{1-2 x}$;
		\item $y=\dfrac{x^2-5x+4}{x^2-1}$;
		\item $y=\dfrac{2x-1}{x^2-3x+2}$.
	\end{enumEX}
\loigiai{
\begin{enumerate}[a)]
	\item Xét $\lim\limits_{x \to -1^+} \dfrac{2x-1}{x+1}=-\infty$ (hoặc $\lim\limits_{x \to -1^-} \dfrac{2x-1}{x+1}=+\infty$) nên đường thẳng $x=-1$ là tiệm cận đứng.\\
	Xét $\lim\limits_{x \to \pm \infty } \dfrac{2x-1}{x+1}=2$ nên đường thẳng $y=2$ là tiệm cận ngang.
	\item Ta có
	\begin{itemize}
		\item $\lim\limits_{x \to \pm\infty} y=\lim\limits_{x \to \pm\infty} \dfrac{2x-3}{1-2x}=-1$ suy ra $y=-1$ là tiệm cận ngang.
		\item $\heva{& \lim\limits_{x \to \tfrac{1}{2}^+} \dfrac{2x-3}{1-2x}=+\infty \\ & \lim\limits_{x \to \tfrac{1}{2}^-} \dfrac{2x-3}{1-2x}=-\infty}$ suy ra $x=\dfrac{1}{2}$ là tiệm cận đứng.
	\end{itemize}
	\item Điều kiện xác định: $\heva{&x\neq-1\\ &x\neq1.}$
	\begin{itemize}
		\item $\lim\limits_{x\to\pm\infty}\dfrac{x^2-5x+4}{x^2-1}=1$
		\item $\lim\limits_{x\to(-1)^-}\dfrac{x^2-5x+4}{x^2-1}=+\infty$
		\item $\lim\limits_{x\to1}\dfrac{x^2-5x+4}{x^2-1}=-\dfrac{3}{2}$
	\end{itemize}
	Vậy đồ thị hàm số có một tiệm cận ngang $y=1$ và một tiệm cận đứng $x=-1$.
	\item Tập xác định $\mathscr{D}=\mathbb{R}\setminus\{1; 2\}$.\\
	Ta có\begin{itemize}
		\item $\lim\limits_{x\to\pm\infty}y=\lim\limits_{x\to\pm\infty}\dfrac{2x-1}{x^2-3x+2}=0$ nên $y=0$ là đường tiệm cận ngang.
		\item $\lim\limits_{x\to 1^-}y=\lim\limits_{x\to 1^-}\dfrac{2x-1}{x^2-3x+2}=\lim\limits_{x\to 1^-}\dfrac{2x-1}{(x-1)(x-2)}=+\infty$ và $\lim\limits_{x\to 1^+}y=-\infty$ nên $x=1$ là đường tiệm cận đứng.
		\item $\lim\limits_{x\to 2^-}y=\lim\limits_{x\to 2^-}\dfrac{2x-1}{x^2-3x+2}=\lim\limits_{x\to 2^-}\dfrac{2x-1}{(x-1)(x-2)}=-\infty$ và $\lim\limits_{x\to 2^+}y=2=+\infty$ nên $x=2$ là đường tiệm cận đứng.
	\end{itemize}
\end{enumerate}}
\end{vd}

\boxmini{BÀI TẬP TRẮC NGHIỆM}
\ind{PHẦN I.} \inden{Câu trắc nghiệm nhiều phương án lựa chọn. Mỗi câu hỏi học sinh chỉ chọn một phương án.}\\
\setcounter{ex}{0}
\Opensolutionfile{ans}[ans/2D1-B3-d1-1]
\begin{ex}
	Đường tiệm cận ngang của đồ thị hàm số $y=\dfrac{2x-4}{x+2}$ là
	\choice
	{\True $y=2$}
	{  $x=2$}
	{ $x=-2$}
	{$y=-2$}
	
	\loigiai{$\underset{x\to -\infty }{\mathop{\lim \limits_{n \to +\infty}}}\,\dfrac{2x-4}{x+2}=2$ và $\underset{x\to +\infty }{\mathop{\lim \limits_{n \to +\infty}}}\,\dfrac{2x-4}{x+2}=2$ nên hàm số có tiệm cận ngang là $y=2$.
	}
\end{ex}

\begin{ex}
	Tìm tiệm cận ngang của đồ thị hàm số $ y = \dfrac{2x + 1}{ x +1} $.
	\choice
	{ \True $  y = -2 $}
	{$  x = -2 $}
	{ $  y = 2 $}
	{$ x = 1 $}
	\loigiai{
		Ta có $ \displaystyle \lim_{ x \rightarrow \pm \infty } \dfrac{2x + 1}{-x + 1} = -2  $.	
	}		
\end{ex}

\begin{ex}
	Đường thẳng $y=3$ là tiệm cận ngang của đồ thị hàm số nào sau đây?
	\choice
	{$y=\dfrac{1-3x}{2+x}$}
	{$y=\dfrac{x^2+3x+2}{x-2}$}
	{\True $y=\dfrac{1+3x}{1+x}$}
	{$y=\dfrac{3x^2+2}{2-x}$}
	\loigiai{
		Ta có $\lim\limits_{x\to \pm \infty}\dfrac{1+3x}{1+x}=3$ nên $y=3$ là tiệm cận ngang của đồ thị hàm số $y=\dfrac{1+3x}{1+x}$.}
\end{ex}

\begin{ex}
	Hàm số nào có đồ thị nhận đường thẳng $x = 2$ làm đường tiệm cận đứng?
	\choice
	{$y=x-2+\dfrac{1}{x+1}$}
	{$y=\dfrac{1}{x+1}$}
	{$y=\dfrac{2}{x+2}$}
	{\True $y=\dfrac{5x}{2-x}$}
	\loigiai{ Xét hàm số $y=\dfrac{5x}{2-x}$\\
		Ta có $\lim\limits_{x\to 2^+}5x=10>0$; $\lim\limits_{x\to 2^+}(2-x)$ và $x-2<0$ khi $x>2$ suy ra $\lim\limits_{x\to 2^+}\dfrac{5x}{2-x}=-\infty$.\\
		Vậy đồ thị hàm số $y=\dfrac{5x}{2-x}$ nhận đường thẳng $x=2$ làm tiệm cận đứng.
	}
\end{ex}

\begin{ex}
	Đường tiệm cận đứng của đồ thị hàm số $y=\dfrac{3x+1}{x-2}$ là đường thẳng
	\choice
	{$x=-2$}
	{\True $x=2$}
	{$y=3$}
	{$y=-\dfrac{1}{2}$}
	\loigiai{Ta có: $\lim \limits_{x\to 2^+}{\dfrac{3x+1}{x-2}}=+\infty$.
	}
\end{ex}

\begin{ex}
	Đường tiệm cận đứng của đồ thị hàm số $y=\dfrac{x+1}{x^2+4x-5}$ có phương trình là
	\choice
	{$x=-1$}
	{$y=1;y=-5$}
	{\True $x=1;x=-5$}
	{$x=\pm 5$}
	\loigiai{
		Ta có $\mathop{\lim}\limits_{x\rightarrow 1^+}y=+\infty$, $\mathop{\lim}\limits_{x\rightarrow 1^-}y=-\infty$, $\mathop{\lim}\limits_{x\rightarrow 5^+}y=+\infty$, $\mathop{\lim}\limits_{x\rightarrow 5^-}y=-\infty$.\\
		Vậy đồ thị hàm số có hai đường tiệm cận đứng là $x=1$ và $x=-5$.}
\end{ex}

\begin{ex}
	Tìm số đường tiệm cận của đồ thị hàm số $ y = \dfrac{x^2 - 3x + 2}{x^2 - 4}. $
	\choice
	{$1$}
	{$ 0$}
	{\True $2$}
	{$3$}
	\loigiai
	{
		Tập xác định: $ \mathscr D = \mathbb{R} \backslash \{\pm2 \} $.\\
		Ta có $ \lim \limits_{x \to \pm  \infty} y = 1 \Rightarrow  $ đồ thị hàm số có 1 tiệm cận ngang là $ y = 1. $\\
		Ta lại có $\lim \limits_{x \to 2} y =  \lim \limits_{x \to 2} \dfrac{x-1}{x+2} = \dfrac{1}{4} $ và $\lim \limits_{x \to -2^+} y =  \lim \limits_{x \to -2^+} \dfrac{x-1}{x+2} = -\infty$ nên đồ thị hàm số có 1 tiệm cận đứng là $ x = -2. $\\
		Vậy đồ thị hàm số đã cho có 2 đường tiệm cận.
	}
\end{ex}

\begin{ex}
	Số đường tiệm cận của đồ thị hàm số $y=\dfrac{3}{x-2}$ là
	\choice
	{$1$}
	{\True $2$}
	{$0$}
	{$3$}
	\loigiai{
		Tiệm cận đứng $x=2$.\\
		Tiệm cận ngang $y=0$.
	}
\end{ex}

\begin{ex}
	Cho hàm số $y=f(x)$ có đồ thị là đường cong $(C)$ và các giới hạn $\lim\limits_{x\to 2^{+}}f(x)=1$, $\lim\limits_{x\to 2^{-}}f(x)=1$, $\lim\limits_{x\to +\infty}f(x)=2$, $\lim\limits_{x\to -\infty}f(x)=2$. Hỏi mệnh đề nào sau đây đúng?
	\choice
	{\True Đường thẳng $y=2$ là tiệm cận ngang của $(C)$}
	{Đường thẳng $y=1$ là tiệm cận ngang của $(C)$}
	{Đường thẳng $x=2$ là tiệm cận ngang của $(C)$}
	{Đường thẳng $x=2$ là tiệm cận đứng của $(C)$}
	\loigiai{
		Ta có $\lim\limits_{x\to +\infty}f(x)=2$, $\lim\limits_{x\to -\infty}f(x)=2\Rightarrow y=2$ là tiệm cận ngang của $(C)$.
	}
\end{ex}

\begin{ex}
	Số tiệm cận đứng của đồ thị hàm số $y=\dfrac{\sqrt{x+9}-3}{x^2+x}$ là
	\choice
	{$3$}
	{$2$}
	{$0$}
	{\True $1$}
	\loigiai{
		Tập xác định $\mathscr{D}=[-9;+\infty)\setminus \{-1;0\}$. \\
		Ta có $\left\{\begin{aligned}
			&\lim\limits_{x\to -1^+} \dfrac{\sqrt{x+9}-3}{x^2+x}=+\infty \\
			&\lim\limits_{x\to -1^-} \dfrac{\sqrt{x+9}-3}{x^2+x}=-\infty
		\end{aligned}\right. \Rightarrow x=-1$ là tiệm cận đứng. \\
		Ngoài ra $\lim\limits_{x\to 0} \dfrac{\sqrt{x+9}-3}{x^2+x}=\dfrac{1}{6}$ nên $x=0$ không thể là một tiệm cận được.}
\end{ex} 

\begin{ex}%[2D1B4]
	\immini{Cho hàm số $y=f(x)$ xác định trên $\mathbb{R}\setminus\left\{\pm1\right\}$ liên tục trên mỗi khoảng xác định và có bảng biến thiên như hình vẽ. Số đường tiệm cận của đồ thị hàm số là
	\choice
	{$1$}
	{$2$}
	{\True $3$}
	{$4$}}{\hspace{0.5cm}
\begin{tikzpicture}
	\tikzset{double style/.append style = {draw=\tkzTabDefaultWritingColor,double=\tkzTabDefaultBackgroundColor,double distance=2pt}}
	\tikzset{double style/.append style = {double distance=0.5pt}} 
	\tkzTabInit[nocadre=false,lgt=1,espcl=1.7]
	{$x$/.7,$y'$ /.7, $y$ /2.3}
	{$-\infty$ ,$-1$,$0$,$1$,$+\infty$}
	\tkzTabLine{,-,d,-,0,+,d,+,}
	\tkzTabVar {+/$-2$,-D+/$-\infty$/$+\infty$,-/$1$,+D-/$+\infty$/$-\infty$,+/$-2$}
\end{tikzpicture}}
	\loigiai{
		Dựa vào bảng biến thiên ta có:\\
		$\lim\limits_{x\to -1^\pm}f(x)=\pm\infty$. 
		$\lim\limits_{x\to 1^\pm}f(x)=\mp\infty$.\\
		Do đó $x=1$ và $x=-1$ là các đường tiệm cận đứng của đồ thị hàm số.\\
		Lại có $\lim\limits_{x\to \pm\infty}f(x)=-2$. Do đó $y=-2$ là tiệm cận ngang của đồ thị hàm số.\\
		Vậy đồ thị hàm số có $3$ đường tiệm cận.
	}
\end{ex}

\begin{ex}
	\immini{Cho hàm số $y=f(x)$ xác định trên $\mathbb{R}\backslash \left\{0\right\},$ liên tục trên mỗi khoảng xác định và có bảng biến thiên như hình bên. Chọn khẳng định đúng.
	\choice
	{Đồ thị hàm số có đúng một tiệm cận ngang}
	{Đồ thị hàm số có hai tiệm cận ngang}
	{\True Đồ thị hàm số có đúng một tiệm cận đứng}
	{Đồ thị hàm số không có tiệm đứng và tiệm cận ngang}}{
	\begin{tikzpicture}[>=stealth]
		\tikzset{double style/.append style = {draw=\tkzTabDefaultWritingColor,double=\tkzTabDefaultBackgroundColor,double distance=2pt}}
		\tkzTabInit[nocadre=false,lgt=1,espcl=2]{$x$/.6,$y'$/.7,$y$/2}{$-\infty$,$0$,$1$,$+\infty$}
		\tkzTabLine{,-, d ,+,z,-,} 
		\tkzTabVar{+/$+\infty$ / , -D- / $-1$ /$-\infty$,+/$2$,-/$-\infty$}
\end{tikzpicture}}
	\loigiai{
		Do $\lim\limits_{x \to +\infty} y=-\infty$ và $\lim\limits_{x \to -\infty} y=+\infty$  nên đồ thị hàm số không có tiệm cận ngang.\\
		Do $\lim\limits_{x \to 0^+} y=+\infty$ suy ra $x=0$ là tiệm cận đứng của đồ thị hàm số.
	} 
\end{ex}

\begin{ex}
	\immini{Cho hàm số $ y=f(x) $ có bảng biến thiên như hình bên. Hỏi đồ thị hàm số đã cho có bao nhiêu đường tiệm cận?
	\choice
	{\True $ 2 $}
	{$ 3 $}
	{$ 4 $}
	{$ 1 $}}{
\begin{tikzpicture}[yscale=.8,xscale=1.15,
	kxd/.pic={\draw[double] (90:.4)--(-90:.4);}]
	\begin{scope}[shift={(-.5,.5)}]
		\fill[pattern=north east lines,pattern color=black]
		(1,-1) rectangle +(1.45,-4);
		\draw
		(0,0) rectangle +(7,-5)
		(0,-1)--+(0:7) (0,-2)--+(0:7) (1,0)--+(-90:5);
	\end{scope}
	\path
	(0,0) node{$ x $}
	++ (0:1) node{$ -\infty $}
	++(0:1)node{$ -2 $}
	++(0:2)node{$ 0 $}
	++(0:2)node{$ +\infty $}
	(0,-1)node{$ y' $}
	++(0:2)pic{kxd}
	++(0:1)node{$ + $}
	++(0:1)pic{kxd}
	++(0:1)node{$ - $}
	(0,-3)node{$ y $}
	++(0:2)pic[yscale=3]{kxd}
	+(-90:1)node[below right](A){$ -\infty $}
	++(0:2) pic[yscale=3]{kxd}
	node[above right](C){$ 1 $}
	+(90:1)node[left](B){$ 2 $}
	++(0:2)node[below](D){$ 0 $};
	\draw[-stealth,black](A)--(B)
	;
	\draw[-stealth,black] (C)--(D);
\end{tikzpicture}}
	\loigiai{
		Dựa vào bảng biến thiên của hàm số, suy ra
		\begin{itemize}
			\item  $ \lim\limits_{x \to +\infty} f(x)=0 $, đồ thị hàm số có tiệm cận ngang là $ y=0 $.
			\item $ \lim\limits_{x \to (-2)^+} f(x)=-\infty $, đồ thị hàm số có tiệm cận đứng là $ x=-2 $.
			Vậy đồ thị hàm số đã cho có $ 2 $ đường tiệm cận.
		\end{itemize}
	}
\end{ex}

\Closesolutionfile{ans}

\ind{PHẦN II.} \inden{Câu trắc nghiệm đúng sai. Trong mỗi ý a), b), c), d) ở mỗi câu, học sinh chọn đúng hoặc sai.}\\
\Opensolutionfile{ans}[ans/2D1-B3-d1-2]
\begin{ex}
	Cho hàm số $y=f(x)$ có bảng biến thiên như hình bên. Xét tính đúng, sai của các khẳng định sau:
	\begin{center}
		\begin{tikzpicture}
			\tikzset{double style/.append style = {draw=\tkzTabDefaultWritingColor,double=\tkzTabDefaultBackgroundColor,double distance=2pt}}
			\tkzTab[nocadre=false,lgt=1.2,espcl=1.7,deltacl=0.6]
			{$x$/0.6, $y'$/0.6, $y$/2}
			{$-\infty$, $0$, $2$, $+\infty$}
			{,-,d,-,$0$,+,}
			{+/ $2$, -D+/ $-\infty$ / $+\infty$, -/ $2$,+/$+\infty$}
		\end{tikzpicture}
	\end{center}
	\choiceTF
	{\True $f(-5)<f(4)$}
	{Hàm số có giá trị nhỏ nhất bằng $2$}
	{\True Đồ thị hàm số có tiệm cận đứng $x=0$}
	{Đồ thị hàm số không có tiệm cận ngang}
	\loigiai{
		\begin{enumerate}[a)]
			\item Từ bảng biến thiên ta thấy $f(-5)<2$ và $f(4)>2$ nên $f(-5)<f(4)$.
			\item Do $\lim \limits_{x\to 0^-}y=-\infty$ nên hàm số không có giá trị nhỏ nhất.
			\item Do $\lim \limits_{x\to 0^-}y=-\infty$ nên đồ thị hàm số có tiệm cận đứng $x=0$.
			\item Do $\lim \limits_{x\to -\infty}y=2y$ nên đồ thị hàm số có tiệm cận ngang $y=2$.
		\end{enumerate}
}
\end{ex}


\begin{ex}
	Cho hàm số hàm số $y=\dfrac{-4x+5}{2x+3}$ có đồ thị $(C)$.
	Xét tính đúng sai của các khẳng định sau:
	\choiceTF
	{\True Hàm số không có cực trị}
	{Đồ thị hàm số có tiệm cận đứng $x=-3$}
	{Đồ thị hàm số có tiệm cận ngang $y=-2$}
	{\True Các đường tiệm cận của đồ thị tạo với hai trục toạ độ một hình chữ nhật có diện tích bằng $3$}
	\loigiai{
		Tập xác định $\mathscr D=\mathbb{R}\setminus \left \{-\dfrac{3}{2}\right \}$\\
		$\lim \limits_{x\to \left (-\frac{3}{2}\right )^+}y=+\infty; \ \lim \limits_{x\to \left (-\frac{3}{2}\right )^-}y=-\infty$ nên đồ thị hàm số có tiệm cận đứng $x=-\dfrac{3}{2}$\\
		$\lim \limits_{x\to -\infty}y=-2, \ \lim \limits_{x\to +\infty}y=-2$ nên đồ thị hàm số có một tiệm cận ngang là $y=-2$\\
		Diện tích hình chữ nhật cần tìm là $S=\left |-\dfrac{3}{2}\right |\cdot \left |-2\right |=3$
	}
\end{ex}

\Closesolutionfile{ans}

\begin{dang}{Bài toán tìm tiệm cận đứng và tiệm cận xiên của đồ thị hàm số}
	\begin{enumerate}[\iconCH]
		\item \indamm{Các bước tìm TCX y = ax + b:}
		Ta xác định hệ số của $a$ và $b$ trong 2 trường hợp sau:
			\begin{listEX}[1]
				\item [\ding{172}] Tính $a=\lim\limits_{x \rightarrow+\infty} \dfrac{f(x)}{x}$, $b=\lim\limits_{x \rightarrow+\infty}[f(x)-ax]$.
				\item [\ding{173}] Tính $a=\lim\limits_{x \rightarrow-\infty} \dfrac{f(x)}{x}$, $b=\lim\limits_{x \rightarrow-\infty}[f(x)-ax]$.
			\end{listEX}
		\item \indamm{Lưu ý:} 
		\begin{listEX}[1]
			\item [\ding{172}] Nếu $a=0$ thì tiệm cận xiên chính là tiệm cận ngang.
			\item [\ding{173}] Đối với hàm số phân thức $f(x)=\dfrac{ax^2+bx+c}{mx+n}$, ta có thể chia đa thức, biến đổi về dạng
			$$f(x)=a'x+b'+\dfrac{e}{mx+n}, \, \text{ với } e \ne0$$
			Suy ra $y=a'x+b'$ là đường tiệm cận xiên của đồ thị hàm số.
		\end{listEX}
	\end{enumerate}
	
\end{dang}
\boxmini{BÀI TẬP TỰ LUẬN}

\begin{vd}
	Tìm các tiệm cận đứng và tiệm cận xiên của đồ thị hàm số sau:
	\begin{listEX}[3]
		\item $y=\dfrac{x^{2}+2}{2x-4}$;
		\item $y=\dfrac{2x^{2}-3x-6}{x+2}$;
		\item $y=\dfrac{2x^{2}+9x+11}{2x+5}$.
	\end{listEX}
	\loigiai{
		\begin{listEX}
			\item Hàm số $y=f(x)=\dfrac{x^{2}+2}{2x-4}$ có tập xác định $\mathscr{D}=\mathbb{R} \setminus \left\lbrace 2\right\rbrace$.
			\begin{itemize}
				\item Ta có $\lim\limits_{x \rightarrow 2^{-}} \dfrac{x^{2}+2}{2x-4}=-\infty$; $\lim\limits_{x \rightarrow 2^{+}} \dfrac{x^{2}+2}{2x-4}=+\infty$.\\
				Suy ra đường thẳng $x=2$ là một tiệm cận đứng của đồ thị hàm số.
				\item Ta có $\begin{aligned}[t]
					a&=\lim\limits_{x \rightarrow+\infty} \dfrac{f(x)}{x}=\lim\limits_{x \rightarrow+\infty} \dfrac{x+\dfrac{2}{x}}{2x-4}=\dfrac{1}{2};\\
					b&=\lim\limits_{x \rightarrow+\infty}[f(x)-ax]=\lim\limits_{x \rightarrow+\infty}\left(\dfrac{x^{2}+2}{2x-4}-\dfrac{1}{2}x\right)=\lim\limits_{x \rightarrow+\infty} \dfrac{2x+2}{2x-4}=1.
				\end{aligned}$\\
				Ta cũng có $\lim\limits_{x \rightarrow-\infty} \dfrac{f(x)}{x}=\dfrac{1}{2}$; $\lim\limits_{x \rightarrow-\infty}[f(x)-\dfrac{1}{2}x]=1$.
				\\
				Do đó, đồ thị hàm số có tiệm cận xiên là đường thẳng $y=\dfrac{1}{2}x+1$.
			\end{itemize}	
			\item Hàm số $y=f(x)=\dfrac{2x^{2}-3x-6}{x+2}$ có tập xác định $\mathscr{D}=\mathbb{R} \setminus \left\lbrace -2\right\rbrace$.
			\begin{itemize}
				\item Ta có $\lim\limits_{x \rightarrow \left(-2\right)^{-}} \dfrac{2x^{2}-3x-6}{x+2}=-\infty$; $\lim\limits_{x \rightarrow \left(-2\right)^{+}} \dfrac{2x^{2}-3x-6}{x+2}=+\infty$.\\
				Suy ra đường thẳng $x=-2$ là một tiệm cận đứng của đồ thị hàm số.
				\item Ta có $\begin{aligned}[t]
					a&=\lim\limits_{x \rightarrow+\infty} \dfrac{f(x)}{x}=\lim\limits_{x \rightarrow+\infty} \dfrac{2x-3-\dfrac{6}{x}}{x+2}=2;\\
					b&=\lim\limits_{x \rightarrow+\infty}[f(x)-ax]=\lim\limits_{x \rightarrow+\infty}\left(\dfrac{2x^{2}-3x-6}{x+2}-2x\right)=\lim\limits_{x \rightarrow+\infty} \dfrac{-7x-6}{x+2}=-7.
				\end{aligned}$\\
				Ta cũng có $\lim\limits_{x \rightarrow-\infty} \dfrac{f(x)}{x}=2$; $\lim\limits_{x \rightarrow-\infty}[f(x)-2x]=-7$.\\
				Do đó, đồ thị hàm số có tiệm cận xiên là đường thẳng $y=2x-7$.
			\end{itemize}
			\item Hàm số $y=f(x)=\dfrac{2x^{2}+9x+11}{2x+5}$ có tập xác định $\mathscr{D}=\mathbb{R} \setminus \left\lbrace -\dfrac{5}{2}\right\rbrace$.
			\begin{itemize}
				\item 
				Ta có $\lim\limits_{x \rightarrow \left(-\tfrac{5}{2}\right)^{-}} \dfrac{2x^{2}+9x+11}{2x+5}=-\infty$; $\lim\limits_{x \rightarrow \left(-\tfrac{5}{2}\right)^{+}} \dfrac{2x^{2}+9x+11}{2x+5}=+\infty$.\\
				Suy ra đường thẳng $x=-\dfrac{5}{2}$ là một tiệm cận đứng của đồ thị hàm số.
				\item Ta có $\begin{aligned}[t]
					a&=\lim\limits_{x \rightarrow+\infty} \dfrac{f(x)}{x}=\lim\limits_{x \rightarrow+\infty} \dfrac{2x+9+\dfrac{11}{x}}{2x+5}=1;\\
					b&=\lim\limits_{x \rightarrow+\infty}[f(x)-ax]=\lim\limits_{x \rightarrow+\infty}\left(\dfrac{2x^{2}+9x+11}{2x+5}-x\right)=\lim\limits_{x \rightarrow+\infty} \dfrac{4x+11}{2x+5}=2.
				\end{aligned}$\\
				Ta cũng có $\lim\limits_{x \rightarrow-\infty} \dfrac{f(x)}{x}=1$; $\lim\limits_{x \rightarrow-\infty}[f(x)-x]=2$.\\
				Do đó, đồ thị hàm số có tiệm cận xiên là đường thẳng $y=x+2$.
			\end{itemize}
		\end{listEX}	
	}
\end{vd}

\boxmini{BÀI TẬP TRẮC NGHIỆM}
\ind{PHẦN I.} \inden{Câu trắc nghiệm nhiều phương án lựa chọn. Mỗi câu hỏi học sinh chỉ chọn một phương án.}\\
\setcounter{ex}{0}
\Opensolutionfile{ans}[ans/2D1-B3-d2-1]
\begin{ex}
	Đường tiệm cận xiên của đồ thị hàm số $y=f(x)=2x-1-\dfrac{1}{x+1}$ có phương trình là
	\choice
	{$y=x+1$}
	{\True $y=2x-1$}
	{$y=x-1$}
	{$y=2x+1$}
	\loigiai{
		Do $\lim\limits_{x\to +\infty}[f(x)-(2x-1)]=\lim\limits_{x\to +\infty}\dfrac{-1}{x+1}=0$ nên đường thẳng $y=2x-1$
		là tiệm cận xiên của đồ thị hàm số đã cho.}
\end{ex}

\begin{ex}
	Đường tiệm cận xiên của đồ thị hàm số $y=f(x)=x+3+\dfrac{1}{2x+1}$ có phương trình là
	\choice
	{$y=2x+1$}
	{$y=x-3$}
	{\True $y=x+3$}
	{$y=2x-1$}
	\loigiai{
		Do $\lim\limits_{x\to \pm\infty}[f(x)-(x+3)]=\lim\limits_{x\to \pm\infty}\dfrac{1}{2x+1}=0$ nên đường thẳng $y=x+3$
		là tiệm cận xiên của đồ thị hàm số đã cho.}
\end{ex}

\begin{ex}
	Tìm tiệm cận xiên của đồ thị hàm số $y=f(x)=\dfrac{x^2+3x}{x-2}$.
	\choice
	{$y=2x-5$}
	{$y=x-2$}
	{\True $y=x+5$}
	{$y=x-5$}
	\loigiai{
	Ta có
	\begin{itemize}
		\item $a=\lim\limits_{x\to +\infty}\dfrac{f(x)}{x}=\lim\limits_{x\to +\infty}\dfrac{x^2+3x}{x(x-2)}=1$
		\item và $b=\lim\limits_{x\to +\infty}[f(x)-x]=\lim\limits_{x\to +\infty}\dfrac{5x}{x-2}=5$.
	\end{itemize}
	Vậy đường thẳng $y=x+5$ là tiệm cận xiên của đồ thị hàm số đã cho (khi $x \to +\infty$).\\
	Tương tự, do $\lim\limits_{x\to -\infty}\dfrac{f(x)}{x}=1$ và $\lim\limits_{x\to -\infty}[f(x)-x]=5$ nên đường thẳng $y=x+5$ cũng là tiệm cận xiên của đồ thị hàm số đã cho (khi $x \to -\infty$).}
\end{ex}

\begin{ex}%[2D1H4-1]
	Tiệm cận xiên của đồ thị hàm số $y=\dfrac{x^2+2x-2}{x+2}$ là
	\choice
	{$y=-2$}
	{$y=1$}
	{$y=x+2$}
	{\True $y=x$}
	\loigiai{
		Ta có $y=\dfrac{x^2+2x-2}{x+2}=\dfrac{x(x+2)-2}{x+2}=x-\dfrac{2}{x+2}$.\\
		$\underset{x\to +\infty}{\mathop{\lim}} [ y-x ] =\underset{x\to +\infty}{\mathop{\lim}}\dfrac{-2}{x+2}=0$ và $\underset{x\to -\infty}{\mathop{\lim}} [ y-x ] =\underset{x\to -\infty}{\mathop{\lim}}\dfrac{-2}{x+2}=0$. \\ 
		Vậy đồ thị hàm số có tiệm cận xiên là đường thẳng $y=x$. 
	}
\end{ex}

\begin{ex}
	Tìm tiệm cận xiên của đồ thị hàm số $f(x)=\dfrac{x^{2}-3 x+1}{x-2}$.
	\choice
	{$y=x+1$}
	{$y=-3x+1$}
	{$y=x-2$}
	{\True $y=x-1$}
	\loigiai{
		Tập xác định: $\mathscr{D}=\mathbb{R} \setminus\{2\}$.
		\\
		Ta có $\begin{aligned}[t]
			a&=\lim\limits_{x \rightarrow+\infty} \dfrac{f(x)}{x}=\lim\limits_{x \rightarrow+\infty} \dfrac{x^{2}-3 x+1}{x^{2}-2 x}=1;\\
			b&=\lim\limits_{x \rightarrow+\infty}[f(x)-a x]=\lim\limits_{x \rightarrow+\infty}\left(\dfrac{x^{2}-3 x+1}{x-2}-x\right)=\lim\limits_{x \rightarrow+\infty} \dfrac{-x+1}{x-2}=-1.
		\end{aligned}$\\
		Ta cũng có $\lim\limits_{x \rightarrow-\infty} \dfrac{f(x)}{x}=1$; $\lim\limits_{x \rightarrow-\infty}[f(x)-x]=-1$.
		\\
		Do đó, đồ thị hàm số có tiệm cận xiên là đường thẳng $y=x-1$.
	}
\end{ex}

\begin{ex}
	Đường tiệm cận xiên của đồ thị hàm số $y=\dfrac{2x^{2}-3x}{x+5}$ đi qua điểm nào sau đây?
	\choice
	{$(5;3)$}
	{$(-4;-5)$}
	{\True $(6;-1)$}
	{$(2;-10)$}
	\loigiai{
		Tập xác định: $\mathscr{D}=\mathbb{R} \setminus\{-5\}$.
		\\
		Ta có $\begin{aligned}[t]
			a&=\lim\limits_{x \rightarrow+\infty} \dfrac{f(x)}{x}=\lim\limits_{x \rightarrow+\infty} \dfrac{2x^{2}-3x}{x^2+5x}=2;\\
			b&=\lim\limits_{x \rightarrow+\infty}[f(x)-ax]=\lim\limits_{x \rightarrow+\infty}\left(\dfrac{2x^{2}-3x}{x+5}-2x\right)=\lim\limits_{x \rightarrow+\infty} \dfrac{-13x}{x+5}=-13.
		\end{aligned}$\\
		Ta cũng có $\lim\limits_{x \rightarrow-\infty} \dfrac{f(x)}{x}=2$; $\lim\limits_{x \rightarrow-\infty}[f(x)-x]=-13$.
		\\
		Do đó, đồ thị hàm số có tiệm cận xiên là đường thẳng $y=2x-13$.	Đường thẳng này qua $(6;-1)$.
	}
\end{ex}

\begin{ex}
	Giao điểm của đường tiệm cận đứng và đường tiệm cận xiên của đồ thị hàm số $y=\dfrac{2x^2-3x+2}{x-1}$ là
	\choice
	{$(1;2)$}
	{\True $(1;1)$}
	{$(1;-1)$}
	{$(1;0)$}
	\loigiai{
	Ta viết lại $y=\dfrac{2x^2-3x+2}{x-1}=2x-1+\dfrac{1}{x-1}$. Suy ra
	\begin{itemize}
		\item [$\bullet$] Tiệm cận đứng $x=1$;
		\item [$\bullet$] Tiệm cận ngang $y=2x-1$.
	\end{itemize}
Xét hệ $\heva{&x=1\\&y=2x-1} \Leftrightarrow \heva{&x=1\\&y=1}$}
\end{ex}

\Closesolutionfile{ans}

\ind{PHẦN II.} \inden{Câu trắc nghiệm đúng sai. Trong mỗi ý a), b), c), d) ở mỗi câu, học sinh chọn đúng hoặc sai.}\\
\Opensolutionfile{ans}[ans/2D1-B3-d2-2]

\begin{ex}
	\immini{Cho hàm số $y=f(x)=\dfrac{ax^2+bx+c}{dx+e}$ có đồ thị như hình bên. 
		\choiceTF
		{Tập xác định của hàm số là $\mathbb{R}$}
		{\True Hàm số có hai điểm cực trị}
		{Đồ thị hàm số có đường tiệm cận đứng là $x=0$}
		{Đồ thị hàm số có đường tiệm cận xiên là $y=x+1$}
	}{
		\begin{tikzpicture}[scale=.4, font=\footnotesize, line join=round, line cap=round, >=stealth]
			\draw[->] (-6,0)--(0,0) node[below left]{$O$}--(6,0) node[below]{$x$};
			\draw[->] (0,-8) --(0,6) node[right]{$y$};
			\clip (-6,-8) rectangle (6,6);
			\draw[violet] [domain=-0.8:6, samples=100,thick] %
			plot (\x, {\x-1+ (2)/((\x)+1)});
			\draw[violet] [domain=-6:-1.3, samples=100,thick] %
			plot (\x, {\x-1+ (2)/((\x)+1)});
			\draw[fill] (0,0) circle (1pt) (-1,0) circle (1pt) (-1,-2) circle (1pt) (1,0) circle (1pt)node[above] {$1$} (0,-1) circle (1pt)node[right] {$-1$};
			\draw[domain=-8:7, samples=100] %
			plot (\x, {\x-1});
			\draw (-1,-8)--(-1,0)node[above left] {$-1$}--(-1,6);
	\end{tikzpicture}}
\end{ex}

\begin{ex}
	\immini{Cho đồ thị của hàm số $y=f(x)=\dfrac{2 x^2}{x^2-1}$. Xét tính đúng sai của các khẳng định sau:
	\choiceTF
	{Đồ thị hàm số có 3 điểm cực trị}
	{$\lim \limits_{x \rightarrow-\infty} f(x)=2$ ; $\lim \limits_{x \rightarrow 1^{-}} f(x)=-\infty$}
	{Đồ thị hàm số có 3 đường tiệm cận đứng $x=-1$, $x=0$, $x=1$} 
	{Đồ thị hàm số có hai đường tiệm cận ngang $y=2$ và $y=0$} 
	}{
\begin{tikzpicture}[scale=.5,>=stealth, font=\footnotesize, line join=round, line cap=round]
	\def\xmin{-6} \def\xmax{6}
	\def\ymin{-5} \def\ymax{7}
	%\draw[color=gray!50,dashed] (\xmin,\ymin) grid (\xmax,\ymax);
	\draw[->] (\xmin,0)--(\xmax,0) node [below]{$x$};
	\draw[->] (0,\ymin)--(0,\ymax) node [left]{$y$};
	\fill (0,0) circle (1pt) node[shift={(135:2.5mm)}]{$O$};
	\clip (\xmin+0.1,\ymin+0.1) rectangle (\xmax-0.1,\ymax-0.1);
	\draw[thick,smooth,violet,samples=300,domain=(\xmin:-1.01)] plot(\x,{(2*(\x)^2)/((\x)^2-1)});		
	\draw[thick,smooth,violet,samples=300,domain=(-0.9:0.9)] plot(\x,{(2*(\x)^2)/((\x)^2-1)});
	\draw[thick,smooth,violet,samples=300,domain=(1.1:\xmax)] plot(\x,{(2*(\x)^2)/((\x)^2-1)});
	\draw[blue] (\xmin,2)--(\xmax,2);	
	\draw[blue] (-1,\ymin)--(-1,\ymax);	
	\draw[blue] (1,\ymin)--(1,\ymax);		
	\foreach \x in {\xmin,...,\xmax}
	\draw (\x,-0.1)--(\x,0.1);
	\foreach \y in {\ymin,...,\ymax}
	\draw (-0.1,\y)--(0.1,\y);
	\node at (-5,2)[below]{$y=2$};
	\node at (-1.2,-4)[left]{$x=-1$};
	\node at (1.2,-4)[right]{$x=1$};
	%\node at (-1,0)[shift={(-135:2.5mm)}]{$-1$};
	%\node at (.5,0)[shift={(-75:2.5mm)}]{$\dfrac{1}{2}$};
	%\node at (0,-1)[left]{$-1$};
	%\node at (0,2)[shift={(135:2.5mm)}]{$2$};		
\end{tikzpicture}}
	\loigiai{
	\begin{enumerate}[a)]
		\item Đồ thị hàm số có một điểm cực trị $(0;0)$.
		\item Theo hình vẽ thì $\lim \limits_{x \rightarrow-\infty} f(x)=2$; $\lim \limits_{x \rightarrow 1^{-}} f(x)=-\infty$.
		\item Đồ thị hàm số có 2 đường tiệm cận đứng $x= \pm 1$.
		\item Đồ thị hàm số có 1 đường tiệm cận ngang $y= 2$.
\end{enumerate}}
\end{ex}

\Closesolutionfile{ans}
\begin{dang}{Bài toán về đường tiệm cận có chứa tham số}
\end{dang}
\boxmini{BÀI TẬP TỰ LUẬN}
\begin{vd}%[2D1Y4-2]
	Tìm tham số $m$ để đồ thị hàm số 
	\begin{tasks}
		\task $y=\dfrac{3x-1}{x-m}$ có đường tiệm cận đứng là $x=5$.
		\task $y=\dfrac{(m+1)x-5m}{2x-m}$ có tiệm cận ngang là đường thẳng $y=1$.
	\end{tasks}
	\loigiai{
		\begin{enumerate}[a)]
			\item Điều kiện để đồ thị hàm số có tiệm cận đứng là $-3m+1\neq 0\Leftrightarrow m\neq \dfrac{1}{3}$.\\
			Đồ thị hàm số có tiệm cận đứng $x=m$.\\
			Theo đề bài ta có $m=5$ (thoả mãn).
			\item Điều kiện để đồ thị hàm số có tiệm cận ngang là $-m(m+1)+10m\neq 0$.\\
			Tiệm cận ngang là $y=\dfrac{a}{c}=\dfrac{m+1}{2}.$\\
			Theo đề bài ta có $\dfrac{m+1}{2}=1\Leftrightarrow m+1=2\Leftrightarrow m=1$ (thoả mãn).
		\end{enumerate}
	}
\end{vd}

\begin{vd}%[2D1K4-2]
	Tìm $m$ để đồ thị hàm số 
	\begin{tasks}
		\task $y=\dfrac{x-2}{x^2-mx+1}$ có hai đường tiệm cận đứng.
		\task $y=\dfrac{2x^2-3x+m}{x-m}$ có đường tiệm cận xiên.
	\end{tasks}
	\loigiai{
		\begin{enumerate}[a)]
			\item Đồ thị hàm số có hai tiệm cận đứng $\Leftrightarrow$ phương trình $g(x)=x^2-mx+1=0$ có hai nghiệm phân biệt khác $2$.
			$$\Leftrightarrow\heva{&a=1\neq 0 \, (\textrm{LĐ})\\ & \Delta =m^2-4>0\\&g(2)=2^2-2m+1\neq 0} \Leftrightarrow \heva{&\hoac{&m<-2\\&m>2}\\& m\neq \dfrac{5}{2}}.$$
			Vậy $m\in\left(-\infty; -2\right) \cup \left(2; +\infty\right) \setminus \left\{\dfrac{5}{2}\right\}$.
			\item 	Đồ thị hàm số có đường tiệm cận xiên khi và chỉ khi phương trình $g(x)=2x^2-3x+m=0$ không có nghiệm $x=m$. Tức là:
			$$g(m)\neq 0 \Leftrightarrow 2m^2-2m\neq 0 \Leftrightarrow \heva{&m\neq 0\\ &n\neq 1}.$$
			Vậy $m\in\mathbb{R}\setminus\left\{0; 1\right\}$ là các giá trị cần tìm.
		\end{enumerate}
	}	
\end{vd}


\boxmini{BÀI TẬP TRẮC NGHIỆM}
\ind{PHẦN I.} \inden{Câu trắc nghiệm nhiều phương án lựa chọn. Mỗi câu hỏi học sinh chỉ chọn một phương án.}\\
\setcounter{ex}{0}
\Opensolutionfile{ans}[ans/2D1-B3-d3-1]

\begin{ex}
	Tìm tất cả các giá trị của $m$ để đồ thị hàm số $y=\dfrac{mx+2}{x-5}$ có đường tiệm cận ngang đi qua điểm $A(1; 3)$.
	\choice
	{$m=-3$}
	{$m=1$}
	{$m=-1$}
	{\True $m=3$}
	\loigiai{
		Tiệm cận ngang $y=m$ đi qua điểm $A(1; 3)$ nên $m=3$.
	}
\end{ex} 

\begin{ex}
	Tìm tham số thực $m$ để đồ thị hàm số $y=\dfrac{mx+3}{x-m}$ có tiệm cận đứng là đường $x=1$, tiệm cận ngang là đường $y=1$.
	\choice
	{\True $m=1$}
	{$m=2$}
	{$m=-1$}
	{$m=3$}
	\loigiai{
		\begin{itemize}
			\item Điều kiện để đồ thị hàm số có tiệm cận là $-m^2-3\ne 0 \ \forall m$
			\item Phương trình đường tiệm cận đứng là $x=m$ nên có $m=1$
			\item Phương trình đường tiệm cận ngang là $y=m$ nên có $m=1$\\
			Vậy $m=1$.
		\end{itemize}
	}
\end{ex}

\begin{ex}
	Biết rằng hai đường tiệm cận của đồ thị hàm số $y=\dfrac{2x+1}{x-m}$ (với $m$ là tham số) tạo với hai trục tọa độ một hình chữ nhật có diện tích bằng $2$. Giá trị của $m$ là
	\choice
	{$m=\pm 2$}
	{$m=-1$}
	{$m=2$}
	{\True $m=\pm 1$}
	\loigiai{
		Điều kiện $ m\neq -\dfrac{1}{2} $.\\
		Ta có $\lim\limits_{x\to+\infty}\dfrac{2x+1}{x-m}=2$ và $\lim\limits_{x\to-\infty}\dfrac{2x+1}{x-m}=2\Rightarrow y=2$ là tiệm cận ngang của đồ thị hàm số.\\
		\begin{itemize}
			\item Xét $ m>-\dfrac{1}{2} $, ta có $\lim\limits_{x\to m^{+}}\dfrac{2x+1}{x-m}=+\infty$, $\lim\limits_{x\to m^{-}}\dfrac{2x+1}{x-m}=-\infty\Rightarrow x=m$ là tiệm cận đứng của đồ thị hàm số.
			\item Xét $ m<-\dfrac{1}{2} $, ta có $\lim\limits_{x\to m^{+}}\dfrac{2x+1}{x-m}=-\infty$, $\lim\limits_{x\to m^{-}}\dfrac{2x+1}{x-m}=+\infty\Rightarrow x=m$ là tiệm cận đứng của đồ thị hàm số.
		\end{itemize}
		Diện tích hình chữ nhật là $|2m|=2\Rightarrow m=\pm 1$ (thỏa mãn).
	}
\end{ex} 


\begin{ex}
	Tìm giá trị của $m$ để đồ thị hàm số $y=\dfrac{2x^2-5x+m}{x-m}$ có tiệm cận đứng.
	\choice
	{$\hoac{&m=0\\&m=2}$}
	{$m\ne 0$}
	{$m\ne 2$}
	{\True $\heva{&m\ne 0\\&m\ne 2}$}
	\loigiai{
		Ta có $x-m=0\Leftrightarrow x=m$. \\
		Để đồ thị hàm số có tiệm cận đứng thì $2(m)^2-5(m)+m\ne 0\Leftrightarrow 2m^2-4m\ne 0\Leftrightarrow \heva{&m\ne 0\\&m\ne 2}$.
	}
\end{ex} 

\begin{ex}%[2D1Y4-1]
	Tìm tất cả các giá trị thực của tham số $m$ để đồ thị hàm số $y=\dfrac{x-4}{x^2-mx+4}$ có hai đường tiệm cận đứng?
	\choice
	{$m \in \left (-\infty;-4\right] \cup \left [4;+\infty \right )$}
	{$m \ne 5$}
	{\True $m \in \left (-\infty;-4\right) \cup \left (4;+\infty \right ) \setminus \left \{5\right \}$}
	{$m \in \left (-\infty;-4\right) \cup \left (4;+\infty \right )$}
	\loigiai{
		Đồ thị hàm số có hai tiệm cận đứng khi phương trình $x^2-mx+4=0$ có hai nghiệm phân biệt khác $4\Leftrightarrow \heva{&m^2-16>0\\&16-4m+4\ne 0}\Leftrightarrow m \in \left (-\infty;-4\right) \cup \left (4;+\infty \right ) \setminus \left \{5\right \}$ 
	}
\end{ex}

\begin{ex}%[2D1B4-2]
	Cho hàm số $ y = \dfrac{2x^2-3x+m}{x-m} $ có đồ thị $ (C) $. Tìm tất cả các giá trị của tham số $ m $ để $ (C) $ không có tiệm cận đứng.
	\choice
	{\True $ m = 0 $ hoặc $ m = 1 $}
	{$ m = 2 $}
	{$ m = 1 $}
	{$ m = 0 $}
	\loigiai{
		Đồ thị $ (C) $ không có tiệm cận đứng khi $ m $ là nghiệm của $ 2x^2-3x+m $
		\begin{align*}
			\Leftrightarrow 2m^2 - 3m + m = 0 \Leftrightarrow \hoac{& m = 0 \\& m = 1.}
		\end{align*}
	}
\end{ex}

\begin{ex}
	Tìm tất cả các giá trị của tham số thực $m$ để đồ thị hàm số $y=\dfrac{x-2}{x^2-mx+1}$ có đúng $3$ đường tiệm cận.
	\choice
	{\True $\left[\begin{aligned}
			&\left\{\begin{aligned}
				&m>2 \\
				&m\ne \dfrac{5}{2}
			\end{aligned}\right. \\
			&m<-2
		\end{aligned}\right. $}
	{$\left[\begin{aligned}
			&m>2 \\
			&\left\{\begin{aligned}
				&m<-2 \\
				&m\ne -\dfrac{5}{2}
			\end{aligned}\right.
		\end{aligned}\right. $}
	{$\left[\begin{aligned}
			&m>2 \\
			&m<-2
		\end{aligned}\right. $}
	{$-2<m<2$}
	\loigiai{
		ĐKXĐ : $x^2-mx+1\ne 0$ \\
		Ta có $\displaystyle\lim \limits_{x\to \pm \infty}y=\displaystyle\lim \limits_{x\to \pm \infty}\dfrac{x-2}{x^2-mx+1}=0$ $ \Rightarrow y=0$ là tiệm cận ngang. \\
		Do đó đồ thị hàm số $y=\dfrac{x-2}{x^2-mx+1}$ có đúng $3$ đường tiệm cận khi và chỉ khi phương trình $x^2-mx+1=0$ có hai nghiệm phân biệt khác $2$. \\
		$ \Leftrightarrow \left\{\begin{aligned}
			& \Delta =m^2-4>0 \\
			&2^2-2m+1\ne 0
		\end{aligned}\right. \Leftrightarrow \left\{\begin{aligned}
			&\left[\begin{aligned}
				&m>2 \\
				&m<-2
			\end{aligned}\right. \\
			&m\ne \dfrac{5}{2}
		\end{aligned}\right. $. }
\end{ex} 

\begin{ex}
	Cho hàm số $y=\dfrac{ax+1}{bx-2}$, xác định $a$ và $b$ để đồ thị của hàm số trên nhận đường thẳng $x=1$ làm tiệm cận đứng và đường thẳng $y=\dfrac{1}{2}$ làm tiệm cận ngang.
	\choice
	{$ \heva{&a=-1\\&b=-2} $}
	{\True $ \heva{&a=1\\&b=2} $}
	{$ \heva{&a=2\\&b=2} $}
	{$ \heva{&a=2\\&b=-2} $}
	\loigiai{Yêu cầu bài toán $\Leftrightarrow\heva{&\dfrac{a}{b}=\dfrac{1}{2}\\&\dfrac{2}{b}=1}\Leftrightarrow\heva{&b=2\\&a=1}$.}
\end{ex} 


\begin{ex}%[2D1Y4-1]
	Cho hàm số $y=\dfrac{mx+1}{x+3n+1}$. Đồ thị hàm số nhận trục hoành và trục tung làm tiệm cận ngang và tiệm cận đứng. Tính $m+n$.
	\choice
	{\True $m+n=-\dfrac{1}{3}$}
	{$m+n=\dfrac{1}{3}$}
	{$m+n=\dfrac{2}{3}$}
	{$m+n=0$}
	\loigiai{
		\begin{itemize}
			\item Điều kiện để đồ thị hàm số có tiệm cận là $m\left (3n+1\right )\ne 0$
			\item Phương trình đường tiệm cận đứng là $x=-3n-1$ nên có $n=-\dfrac{1}{3}$
			\item Phương trình đường tiệm cận ngang là $y=m$ nên có $m=0$\\
			Vậy $m+n=-\dfrac{1}{3}$.
		\end{itemize}
	}
\end{ex}

\begin{ex}%[2D1K4-2]
	Đồ thị hàm số $y=\dfrac{(4a-b)x^2+ax+1}{x^2+ax+b-12}$ nhận trục hoành và trục tung làm hai tiệm cận. Tính giá trị của $a+b$.
	\choice
	{$a+b=10$}
	{$a+b=12$}
	{$a+b=18$}
	{\True $a+b=15$}
	\loigiai{
		Tiệm cận đứng $x=0 \Rightarrow 0^2+a.0+b-12=0\Leftrightarrow b=12.$\\
		Tiệm cận ngang $y=0 \Rightarrow 4a-b=0\Leftrightarrow 4a-12=0 \Leftrightarrow a=3.$\\
		\textbf{Kết luận:} $a+b=15.$
	}
\end{ex}

\Closesolutionfile{ans}

\ind{PHẦN II.} \inden{Câu trắc nghiệm đúng sai. Trong mỗi ý a), b), c), d) ở mỗi câu, học sinh chọn đúng hoặc sai.}\\
\Opensolutionfile{ans}[ans/2D1-B3-d3-2]
\begin{ex}%[2D1B4-2]
	Cho hàm số $y=\dfrac{mx^2+6x-2}{x+2}$, với $m$ là tham số.
	\choiceTF
	{\True Tập xác định của hàm số là $\mathbb{R}\backslash\{-2\}$}
	{Đồ thị hàm số có tiệm cận ngang khi $m>0$}
	{Đồ thị hàm số có tiệm cận đứng khi $m\ne 0$}
	{\True Tập hợp tất cả giá trị của $m$ đề đồ thị có hai đường tiệm cận là $\mathbb{R}\setminus\left\{\dfrac{7}{2}\right\}$}
	\loigiai
	{
		\begin{enumerate}[a)]
			\item Điều kiện $x+2 \ne 0 \Leftrightarrow x \ne -2$. Vậy Tập xác định là $\mathbb{R}\backslash\{-2\}$
			\item Đồ thị hàm số có tiệm cận ngang khi hệ số của $x^2$ trên tử số phải bằng 0. Suy ra $m=0$.
			\item Đồ thị hàm số có tiệm cận đứng khi $x=-2$ không là nghiệm của tam thức $g(x)=mx^2+6x-2$. Suy ra
			$$g(-2)\ne 0 \Leftrightarrow m \ne \dfrac{7}{2}$$
			\item Đồ thị hàm số chắc chắn có 1 tiệm cận xiên (hoặc ngang). Suy ra, để đồ thị có hai đường tiệm cận thì nó phải có 1 tiệm cận đứng. Điều này tương đương với $m \ne \dfrac{7}{2}$.
		\end{enumerate}
	}
\end{ex}

\Closesolutionfile{ans}

% \section[TIỆM CẬN]{ĐƯỜNG TIỆM CẬN CỦA ĐỒ THỊ HÀM SỐ}
\subsection{TÓM TẮT LÝ THUYẾT}
\subsubsection{Đường tiệm cận ngang}%[Lý Văn Hoàng, Dự án TeX hóa Lý Thuyết]
\begin{dn}
    Đường thẳng $y=m$ là đường tiệm cận ngang (hay tiệm cận ngang)
    của đồ thị hàm số $y=f(x)$ nếu ít nhất một trong các điều kiện sau được thỏa mãn:\\
    \centerline{$\lim\limits_{x\to+\infty}f(x)=m, \quad\lim\limits_{x\to-\infty}f(x)=m $.}
\end{dn}
\begin{center}
    \begin{tikzpicture}[scale=1,>=stealth, font=\footnotesize, line join=round, line cap=round]
        \def\xmin{-4} \def\xmax{2}
        \def\ymin{-0.5} \def\ymax{3}
        %\draw[color=gray!50,dashed] (\xmin,\ymin) grid (\xmax,\ymax);
        \draw[->] (\xmin,0)--(\xmax,0) node [below]{$x$};
        \draw[->] (0,\ymin)--(0,\ymax) node [left]{$y$};
        \fill (0,0) circle (1pt) node[shift={(-135:2.5mm)}]{$O$};
        \node at (current bounding box.south) [below=-2pt] {a) $\lim\limits_{x \rightarrow-\infty} f(x)=m$};
        \clip (\xmin+0.1,\ymin+0.1) rectangle (\xmax-0.1,\ymax-0.1);
        \draw[red,thick,smooth,samples=300,domain=\xmin:\xmax]
        (-4,0.9)..controls +(0:2) and +(180:0.5)
        ..(-1.5,0.5)..controls +(0:0.5) and +(180:0.5)
        ..(-0.3,1.4)..controls +(0:0.5) and +(135:1)
        ..(1.8,0.3);
        \draw [blue](\xmin,1)--(\xmax,1);
        \path[blue] (-3,1)node[above]{$y=m$};
        \path[red] (0,1.3)node[above left]{$y=f(x)$};
        \fill (0,1) circle (1pt) node[shift={(-135:3mm)}]{$m$};
    \end{tikzpicture}\hspace*{1cm}
    \begin{tikzpicture}[scale=1,>=stealth, font=\footnotesize, line join=round, line cap=round]
        \def\xmin{-1.5} \def\xmax{4}
        \def\ymin{-0.5} \def\ymax{3}
        %\draw[color=gray!50,dashed] (\xmin,\ymin) grid (\xmax,\ymax);
        \draw[->] (\xmin,0)--(\xmax,0) node [below]{$x$};
        \draw[->] (0,\ymin)--(0,\ymax) node [left]{$y$};
        \fill (0,0) circle (1pt) node[shift={(-135:2.5mm)}]{$O$};
        \node at (current bounding box.south) [below=-2pt] {b) $\lim\limits_{x \rightarrow+\infty} f(x)=m$};
        \clip (\xmin+0.1,\ymin+0.1) rectangle (\xmax-0.1,\ymax-0.1);
        \draw[red,thick,smooth,samples=300,domain=\xmin:\xmax]
        (-1,3)..controls +(-80:1) and +(170:1)
        ..(0.5,1.1)..controls +(170:-1) and +(180:-0.5)
        ..(3.9,0.8);
        \draw [blue](\xmin,0.7)--(\xmax,0.7);
        \path[blue] (4,0.7)node[below left]{$y=m$};
        \path[red] (0.5,1)node[above right]{$y=f(x)$};
        \fill (0,0.7) circle (1pt) node[shift={(-135:3mm)}]{$m$};
    \end{tikzpicture}
\end{center}
\begin{nx} \quad
    \begin{itemize}
        \item Để tìm tiệm cận ngang của đồ thị hàm số ta cần tính giới hạn của hàm số tại vô cực $(\pm \infty)$.
        \item Tìm giới hạn ở vô cực của hàm $y=\dfrac{P(x)}{Q(x)}$ với $P(x)$, $Q(x)$ là các đa thức không căn.
        \begin{enumerate}[i)]
            \item Bậc của $P(x)$ nhỏ hơn bậc của $Q(x) \Rightarrow \lim\limits_{x\to \pm\infty} y =0 \Rightarrow$ Tiệm cận ngang $Ox \colon y=0$.
            \item Bậc của $P(x)$ bằng bậc của $Q(x) \Rightarrow \lim\limits_{x\to \pm\infty} y = \dfrac{\text{Hệ số x bậc cao của P(x) }}{\text{Hệ số x bậc cao của Q(x)}} = \alpha$ (một số cụ thể) $\Rightarrow y= \alpha$ là tiệm cận ngang.
            \item Bậc của $P(x)$ lớn hơn bậc của $Q(x) \Rightarrow \lim\limits_{x\to \pm\infty} y = \pm \infty \Rightarrow$ Không có tiệm cận ngang.
        \end{enumerate}
    \end{itemize}
\end{nx}
\subsubsection{Đường tiệm cận đứng}%[Lý Văn Hoàng, Dự án TeX hóa Lý Thuyết]
\begin{dn}
    Đường thẳng $x=a$ được gọi là đường tiệm cận đứng (hay tiệm cận đứng) của đồ thị hàm số $y=f(x)$ nếu ít nhất một trong các điều kiện sau được thỏa mãn:
    $$ \lim\limits_{x \to a^{+} } f(x)= + \infty; \lim\limits_{x \to a^{+} } f(x)= - \infty ;$$ $$ \lim\limits_{x \to a^{-} } f(x)= + \infty; \lim\limits_{x \to a^{-}} f(x)= - \infty.$$
\end{dn}
\begin{center}
    \begin{tikzpicture}[scale=.7,>=stealth, font=\footnotesize, line join=round, line cap=round]
        %Hình a
        \def\xmin{-2.2} \def\xmax{3.5}
        \def\ymin{-2} \def\ymax{2}
        %\draw[color=gray!50,dashed] (\xmin,\ymin) grid (\xmax,\ymax);
        \draw[->] (\xmin,0)--(\xmax,0) node [below]{$x$};
        \draw[->] (0,\ymin)--(0,\ymax) node [left]{$y$};
        \fill (0,0) circle (1pt) node[shift={(-45:2.5mm)}]{$O$};
        \draw (2.1,\ymin)--(2.1,\ymax)node[below right]{$x=a$};
        \fill (2.1,0) circle (1pt) node[shift={(-45:3mm)}]{$a$};
        %\clip (\xmin+0.1,\ymin+0.1) rectangle (\xmax-0.1,\ymax-0.1);
        \draw[red] (-2,-1)..controls +(80:0.5) and +(0:-.5)..(-1,0.5)node[above]{$y=f(x)$}
        ..controls +(0:0.5) and +(180:0.5)..(0.5,-1.5)
        ..controls +(0:0.5) and +(87:-0.2)..(1.6,0)
        ..controls +(87:-.2) and +(90:-0.2)
        ..(2,1.85);
        \node at (current bounding box.south) [below=-2pt] {a) $\lim\limits_{x \rightarrow a^{-}} f(x)=+\infty$};
    \end{tikzpicture}
    \begin{tikzpicture}[scale=.7,>=stealth, font=\footnotesize, line join=round, line cap=round]
        %Hình b
        \def\xmin{-1.2} \def\xmax{4}
        \def\ymin{-2} \def\ymax{2}
        %\draw[color=gray!50,dashed] (\xmin,\ymin) grid (\xmax,\ymax);
        \draw[->] (\xmin,0)--(\xmax,0) node [below]{$x$};
        \draw[->] (0,\ymin)--(0,\ymax) node [left]{$y$};
        \fill (0,0) circle (1pt) node[shift={(-45:2.5mm)}]{$O$};
        \draw (1,\ymin)node[above right]{$x=a$}--(1,\ymax);
        \fill (1,0) circle (1pt) node[shift={(-135:3mm)}]{$a$};
        \path[red] (1.25,1)node[above right]{$y=f(x)$};
        %\clip (\xmin+0.1,\ymin+0.1) rectangle (\xmax-0.1,\ymax-0.1);
        \draw[red] (1.2,2)..controls +(80:0) and +(0:-1.4)..(2.5,-0.8)
        ..controls +(0:0.1) and +(-80:-0.6)
        ..(3.5,-1.5);
        \node at (current bounding box.south) [below=-2pt] {b) $\lim\limits_{x \rightarrow a^{+}} f(x)=+\infty$};
    \end{tikzpicture}
    \begin{tikzpicture}[scale=.7,>=stealth, font=\footnotesize, line join=round, line cap=round]
        %Hình c
        \def\xmin{-2.2} \def\xmax{3.5}
        \def\ymin{-2} \def\ymax{2}
        %\draw[color=gray!50,dashed] (\xmin,\ymin) grid (\xmax,\ymax);
        \draw[->] (\xmin,0)--(\xmax,0) node [below]{$x$};
        \draw[->] (0,\ymin)--(0,\ymax) node [left]{$y$};
        \fill (0,0) circle (1pt) node[shift={(-45:2.5mm)}]{$O$};
        \draw (2,\ymin)--(2,\ymax)node[below right]{$x=a$};
        \fill (2,0) circle (1pt) node[shift={(-45:3mm)}]{$a$};
        \path[red] (-2.25,1.2)node[below right]{$y=f(x)$};
        %\clip (\xmin+0.1,\ymin+0.1) rectangle (\xmax-0.1,\ymax-0.1);
        \draw[red] (-2,1.4)..controls +(-10:-0.2) and +(-55:-.7)
        ..(1.3,0.65)..controls +(-50:0.4) and +(-90:0)
        ..(1.8,-2)
        ;
        \node at (current bounding box.south) [below=-2pt] {c) $\lim\limits_{x \rightarrow a^{-}} f(x)=-\infty$};
    \end{tikzpicture}
    \begin{tikzpicture}[scale=.7,>=stealth, font=\footnotesize, line join=round, line cap=round]
        %Hình d
        \def\xmin{-2.2} \def\xmax{3.5}
        \def\ymin{-2} \def\ymax{2}
        %\draw[color=gray!50,dashed] (\xmin,\ymin) grid (\xmax,\ymax);
        \draw[->] (\xmin,0)--(\xmax,0) node [below]{$x$};
        \draw[->] (0,\ymin)--(0,\ymax) node [left]{$y$};
        \fill (0,0) circle (1pt) node[shift={(-135:2.5mm)}]{$O$};
        \draw (.6,\ymin)--(.6,\ymax)node[below right]{$x=a$};
        \fill (.6,0) circle (1pt) node[shift={(-135:3mm)}]{$a$};
        %\clip (\xmin+0.1,\ymin+0.1) rectangle (\xmax-0.1,\ymax-0.1);
        \draw[red] (0.7,-2)..controls +(85:0.2) and +(180:0.2)
        ..(1.2,-0.3)..controls +(0:0.2) and +(180:0.2)
        ..(1.7,-0.6)..controls +(0:0.4) and +(90:0)
        ..(2.5,2)
        ;
        \node at (current bounding box.south) [below=-2pt] {d) $\lim\limits_{x \rightarrow a^{+}} f(x)=-\infty$};
    \end{tikzpicture}
\end{center}
\immini{\textbf{Đặc biệt} Đối với hàm số $y= \dfrac{ax+b}{cx+d}$ có tiệm cận ngang $y=\dfrac{a}{c}$ và tiệm cận đứng $x= -\dfrac{d}{c}$. Tâm đối xứng là giao điểm của hai đường tiệm cận.
}{
    \begin{tikzpicture}[>=stealth, line join=round, line cap=round, font=\scriptsize,x=.8cm,y=.7cm]
        \begin{scope}[scale=.7]
            \def\a{1}
            \def\b{1}
            \def\c{1}
            \def\d{-2}
            \def\mau{red}
            \draw[->] (-5,0) -- (8,0) node[below] {$x$};
            \draw[->] (0,-5) -- (0,5) node[left] {$y$};
            \draw (0,0)node[below left]{$O$};
            \draw[dashed,blue] ({-\d/\c},-5)--({-\d/\c},5) (-5,{\a/\c})--(8,{\a/\c}); % Vẽ TCĐ và TCN
            \clip (-5,-5)rectangle(8,5);
            \draw ({-\d/\c},0)node[below right]{$2$};
            \draw (0,{\a/\c})node[above left]{$1$};
            \draw (7,-4)node[above left]{{\normalsize }$y=\dfrac{x+1}{x-2}$};
            \pgfmathsetmacro{\can}{-(\d)/(\c)}
            \draw[\mau,samples=150,smooth,domain=-5:{\can-.1}] plot(\x,{(\a*\x+(\b))/(\c*\x+(\d))}); % Vẽ nhánh bên trái TCĐ
            \draw[\mau,samples=150,smooth,domain={\can+.1}:8] plot(\x,{(\a*\x+(\b))/(\c*\x+(\d))}); % Vẽ nhánh bên phải TCĐ
        \end{scope}
\end{tikzpicture}}
\begin{nx} \quad
    \begin{itemize}
        \item Để tìm tiệm cận đứng của đồ thị hàm số, ta cần tính giới hạn một bên của $x_0$, với $x_0$ thường là điều kiện của hàm số (hay tại $x_0$ thì hàm số không xác định).
        \item Kỹ năng sử dụng máy tính (tham khảo):
        \begin{enumerate}[i)]
            \item Tính $\lim\limits_{x \to x_0^+} f(x)$ thì nhập $f(x)$ và CALC $x= x_0 + 10^{-9}$.
            \item Tính $\lim\limits_{x \to x_0^-} f(x)$ thì nhập $f(x)$ và CALC $x= x_0 - 10^{-9}$.
        \end{enumerate}
    \end{itemize}
\end{nx}
\subsubsection{Đường tiệm cận xiên}
\begin{dn}
    Đường thẳng $y=ax+b$ được gọi là đường tiệm cận xiên của đồ thị $(C):y=f(x)$ nếu \[\lim \limits_{x \to -\infty} \left[f(x)-(ax+b)\right]=0 \text{ hoặc }\lim \limits_{x \to +\infty} \left[f(x)-(ax+b)\right]=0\]
\end{dn}
\begin{center}
    \begin{tikzpicture}[scale=0.8,>=stealth, font=\footnotesize, line join=round, line cap=round]
        \def\xmin{-4} \def\xmax{2.5}
        \def\ymin{-0.5} \def\ymax{3}
        %\draw[color=gray!50,dashed] (\xmin,\ymin) grid (\xmax,\ymax);
        \draw[->] (\xmin,0)--(\xmax,0) node [below]{$x$};
        \draw[->] (0,\ymin)--(0,\ymax) node [left]{$y$};
        \fill (0,0) circle (1pt) node[shift={(-135:2.5mm)}]{$O$};
        \node at (current bounding box.south) [below=-2pt] {a) $\lim\limits_{x \rightarrow-\infty}\left[f(x)-(ax+b)\right]=0$};
        \clip (\xmin+0.1,\ymin+0.1) rectangle (\xmax-0.1,\ymax-0.1);
        \draw[red,thick,smooth,samples=300,domain=\xmin:\xmax]
        (-3.8,-0.6)..controls +(34:0.5) and +(180:.75)
        ..(-0.2,1.2)..controls +(0:0.75) and +(180:.75)
        ..(1,0.3)..controls +(0:0.5) and +(80:0)
        ..(2.2,1);
        \draw[blue,smooth,samples=300,domain=\xmin:\xmax] plot(\x,{2/3*(\x)+2});
        \path[blue] (-3,0)--(0,2)node[below,sloped,pos=1.3]{$y=ax+b$};
        \path[red] (0.5,1)node[above right]{$y=f(x)$};
    \end{tikzpicture}\hspace{1cm}
    \begin{tikzpicture}[scale=0.8,>=stealth, font=\footnotesize, line join=round, line cap=round]
        \def\xmin{-3.5} \def\xmax{3}
        \def\ymin{-0.5} \def\ymax{3}
        %\draw[color=gray!50,dashed] (\xmin,\ymin) grid (\xmax,\ymax);
        \draw[->] (\xmin,0)--(\xmax,0) node [below]{$x$};
        \draw[->] (0,\ymin)--(0,\ymax) node [left]{$y$};
        \fill (0,0) circle (1pt) node[shift={(-135:2.5mm)}]{$O$};
        \node at (current bounding box.south) [below=-2pt] {a) $\lim\limits_{x \rightarrow+\infty}\left[f(x)-(ax+b)\right]=0$};
        \clip (\xmin+0.1,\ymin+0.1) rectangle (\xmax-0.1,\ymax-0.1);
        \draw[red,thick,smooth,samples=300,domain=\xmin:\xmax]
        (-3,0.8)..controls +(60:0.5) and +(180:.75)
        ..(-1.5,2)..controls +(0:.5) and +(180:.75)
        ..(0.5,1.3)..controls +(0:.75) and +(-160:.5)
        ..(2.8,1.8);
        \draw[blue,smooth,samples=300,domain=\xmin:\xmax] plot(\x,{1/3*(\x)+0.75});
        \path[blue] (-3,-0.25)--(0,0.75)node[below,sloped,pos=1.6]{$y=ax+b$};
        \path[red] (-2.5,2)node[above right]{$y=f(x)$};
    \end{tikzpicture}
\end{center}
\begin{nx}\quad
    \begin{itemize}
        \item Để tìm TCX của đồ thị hàm số $y=f(x)$ ta giải hệ phương trình: $\heva{& \lim \limits_{x \to +\infty} \dfrac{f(x)}{x}=a \ne 0 \\ & \lim \limits_{x \to +\infty} \left[f(x)-ax\right]=b}$ hoặc $\heva{& \lim \limits_{x \to -\infty} \dfrac{f(x)}{x}=a \ne 0 \\ & \lim \limits_{x \to -\infty} \left[f(x)-ax\right]=b}$, khi đó tiệm cận xiên của đồ thị hàm số $y=f(x)$ là đường thẳng $y=ax+b$.
        \item Đồ thị hàm số $y=\dfrac{mx^2+nx+p}{cx+d}=ax+b+\dfrac{r}{cx+d}$ có đường tiệm cận xiên là đường thẳng $y=ax+b$.
        \item Hàm phân thức có bậc tử bé hơn hoặc bằng bậc mẫu, bậc tử lớn hơn bậc mẫu 2 bậc thì không có tiệm cận xiên.
    \end{itemize}
\end{nx}
%\subsection{CÁC DẠNG TOÁN}
\begin{dang}{Tìm các đường tiệm cận qua biểu thức hàm số, bảng biến thiên}
\end{dang}
\begin{vd} Tìm các đường tiệm cận đứng, ngang, xiên (nếu có) của đồ thị hàm số sau
    \begin{listEX}[3]
        \item $y=\dfrac{2x+1}{x+1}$.
        \item $y=\dfrac{x}{2x-1}$.
        \item $y=\dfrac{3-x}{x+1}$.
        \item $y=2x+1+\dfrac{1}{x-3}$
        \item $y=\dfrac{4x^2-3x+10}{x-1}$.
        \item $y=\dfrac{x^2-4x+3}{x^2-1}$.
        \item $y=\dfrac{2x+4}{x^2+x-2}$.
        \item $y=\dfrac{\sqrt{9-x^2}}{x-1}$.
        \item $y=x+\sqrt{x^2-1}$
        %	\item $y=\dfrac{x}{\sqrt{x^2+1}}$.
        %	\item $y=\dfrac{\sqrt{x+25}-5}{x^2+x}$.
    \end{listEX}
    \loigiai{}
\end{vd}
\begin{vd}
    Tìm các đường tiệm cận của đồ thị hàm số $y=f(x)$, biết
    \begin{listEX}[2]
        \item \begin{tikzpicture}[scale=.7,>=stealth, font=\footnotesize, line join=round, line cap=round]
            \def\xmin{-2} \def\xmax{4}
            \def\ymin{-3} \def\ymax{3}
            %\draw[color=gray!50,dashed] (\xmin,\ymin) grid (\xmax,\ymax);
            \draw[->] (\xmin,0)--(\xmax,0) node [below]{$x$};
            \draw[->] (0,\ymin)--(0,\ymax) node [left]{$y$};
            \fill (0,0) circle (1pt) node[shift={(135:2.5mm)}]{$O$};
            %\node at (current bounding box.south) [below=-2pt] {a) $y=\dfrac{2x-3}{5x^{2}-15x+10}$};
            \clip (\xmin+0.1,\ymin+0.1) rectangle (\xmax-0.1,\ymax-0.1);
            \draw[thick,smooth,samples=300,domain=\xmin:0.99] plot(\x,{(2*(\x)-3)/(5*(\x)^2-15*(\x)+10)});
            \draw[thick,smooth,samples=300,domain=1.01:1.99] plot(\x,{(2*(\x)-3)/(5*(\x)^2-15*(\x)+10)});
            \draw[thick,smooth,samples=300,domain=2.01:\xmax] plot(\x,{(2*(\x)-3)/(5*(\x)^2-15*(\x)+10)});
            \draw[dashed] (1,\ymin)--(1,\ymax);
            \draw[dashed] (2,\ymin)--(2,\ymax);
            \foreach \s/\t in {2/-45,1/-45}
            \fill (\s,0) circle (1pt) node[shift={(\t:3mm)}]{$\s$};
        \end{tikzpicture}
        \item \begin{tikzpicture}[scale=.5,>=stealth, font=\footnotesize, line join=round, line cap=round]
            \def\xmin{-4} \def\xmax{4}
            \def\ymin{-3} \def\ymax{5}
            %\draw[color=gray!50,dashed] (\xmin,\ymin) grid (\xmax,\ymax);
            \draw[->] (\xmin,0)--(\xmax,0) node [below]{$x$};
            \draw[->] (0,\ymin)--(0,\ymax) node [right]{$y$};
            \fill (0,0) circle (1pt) node[shift={(-135:2.5mm)}]{$O$};
            %\node at (current bounding box.south) [below=-2pt] {c) $y=\dfrac{16x^{2}-8x}{16x^{2}+1}$};
            \clip (\xmin+0.1,\ymin+0.1) rectangle (\xmax-0.1,\ymax-0.1);
            \draw[thick,smooth,samples=300,domain=\xmin:\xmax] plot(\x,{(16*(\x)^2-8*(\x))/(16*(\x)^2+1)});
            \draw[dashed](\xmin,1)--(\xmax,1);
            \foreach \p/\r in {1/45}
            \fill (0,\p) circle (1pt) node[shift={(\r:3mm)}]{$\p$};
        \end{tikzpicture}
        \item 	\begin{tikzpicture}[scale=.7,>=stealth, font=\footnotesize, y=.7cm]
            \def\xmin{-.5} \def\xmax{6}
            \def\ymin{-.5} \def\ymax{5}
            \draw[->] (\xmin,0)--(\xmax,0) node [below]{$x$};
            \draw[->] (0,\ymin)--(0,\ymax) node [left]{$y$};
            \fill (0,0) circle (1pt) node[shift={(-135:2.5mm)}]{$O$};
            \node at (1,-.5)[right]{$x=1$};
            \clip (\xmin+0.1,\ymin+0.1) rectangle (\xmax-0.1,\ymax-0.1);
            \draw[smooth,thick,samples=300,domain=(1.01:\xmax)] plot(\x,{2/sqrt(\x-1)});
            \draw[blue,dashed] (1,\ymin)--(1,\ymax);
            \draw[blue,dashed] (-1,.8)--(6,.8)node[below left]{$y=0.5$};
            \foreach \x in {\xmin,...,\xmax}
            \draw (\x,-0.1)--(\x,0.1);
            \foreach \y in {\ymin,...,\ymax}
            \draw (-0.1,\y)--(0.1,\y);
        \end{tikzpicture}
        \item \begin{tikzpicture}[scale=0.5, font=\footnotesize, line join=round, line cap=round, >=stealth]
            \clip(-3,-2) rectangle (5.1,4.1);
            \draw[->] (-3,0) -- (5,0);\draw (4.9,0) node[below] { $x$};
            \draw[->] (0,-2) -- (0,4);\draw (0,3.9) node[right] { $y$};
            \draw[fill=black] (0,0) node[below right]{$O$} circle (1pt);
            \draw (1,0) node[below right]{$2$};
            \draw (0,1) node[above left]{$1$};
            \draw[thick] plot[domain=-3:0.5,samples=100] (\x, {(1 + \x)/(\x - 1)});
            \draw[thick] plot[domain= 1.5:5,samples=100] (\x, {(1 + \x)/(\x - 1)});
            \draw [-,dashed] (-3,1)--(5,1); %TCN
            \draw [-,dashed] (1,-2)--(1,4); %TCĐ
            \draw[fill=black] (0,0) circle(1pt);
        \end{tikzpicture}
        \item \begin{tikzpicture}[scale=.7,>=stealth, font=\footnotesize, line join=round, line cap=round]
            \def\xmin{-4} \def\xmax{4}
            \def\ymin{-3} \def\ymax{5}
            %\draw[color=gray!50,dashed] (\xmin,\ymin) grid (\xmax,\ymax);
            \draw[->] (\xmin,0)--(\xmax,0) node [below]{$x$};
            \draw[->] (0,\ymin)--(0,\ymax) node [right]{$y$};
            \fill (0,0) circle (1pt) node[shift={(-135:2.5mm)}]{$O$};
            %\node at (current bounding box.south) [below=-2pt] {b) $y=\dfrac{x^{2}+x-1}{x}$};
            \clip (\xmin+0.1,\ymin+0.1) rectangle (\xmax-0.1,\ymax-0.1);
            \draw[thick,smooth,samples=300,domain=\xmin:-0.01] plot(\x,{((\x)^2+(\x)-1)/(\x)});
            \draw[thick,smooth,samples=300,domain=0.01:\xmax] plot(\x,{((\x)^2+(\x)-1)/(\x)});
            \draw[dashed,smooth,samples=300,domain=\xmin:\xmax] plot(\x,{(\x)+1});
            \foreach \s/\t in {-1/-90}
            \fill (\s,0) circle (1pt) node[shift={(\t:3mm)}]{$\s$};
            \foreach \p/\r in {1/-20}
            \fill (0,\p) circle (1pt) node[shift={(\r:3mm)}]{$\p$};
        \end{tikzpicture}
        \item \begin{tikzpicture}[scale=.7,>=stealth, font=\footnotesize,x=.7cm,y=.7cm]
            \def\xmin{-6} \def\xmax{6}
            \def\ymin{-5} \def\ymax{7}
            %\draw[color=gray!50,dashed] (\xmin,\ymin) grid (\xmax,\ymax);
            \draw[->] (\xmin,0)--(\xmax,0) node [below]{$x$};
            \draw[->] (0,\ymin)--(0,\ymax) node [left]{$y$};
            \fill (0,0) circle (1pt) node[shift={(135:2.5mm)}]{$O$};
            \clip (\xmin+0.1,\ymin+0.1) rectangle (\xmax-0.1,\ymax-0.1);
            \draw[smooth,thick,samples=300,domain=(\xmin:-1.01)] plot(\x,{(2*(\x)^2)/((\x)^2-1)});
            \draw[smooth,thick,samples=300,domain=(-0.9:0.9)] plot(\x,{(2*(\x)^2)/((\x)^2-1)});
            \draw[smooth,thick,samples=300,domain=(1.1:\xmax)] plot(\x,{(2*(\x)^2)/((\x)^2-1)});
            \draw[dashed] (\xmin,2)--(\xmax,2);
            \draw[dashed] (-1,\ymin)--(-1,\ymax);
            \draw[dashed] (1,\ymin)--(1,\ymax);
            \foreach \x in {\xmin,...,\xmax}
            \draw (\x,-0.1)--(\x,0.1);
            \foreach \y in {\ymin,...,\ymax}
            \draw (-0.1,\y)--(0.1,\y);
            \node at (-5,2)[below]{$y=2$};
            \node at (-1.2,-4)[left]{$x=-1$};
            \node at (1.2,-4)[right]{$x=1$};
        \end{tikzpicture}
        \item
        \begin{tikzpicture}[>=stealth]
            \tkzTabInit[nocadre=false,lgt=1,espcl=2,deltacl=0.5]{$x$/.7 ,$y'$/.7,$y$/2}
            {$-\infty$ , $1$ , $+\infty$}
            \tkzTabLine{ , - , d , - , }
            \tkzTabVar{+/$2$ ,-D+/$-\infty$/$+\infty$ , -/$2$}
        \end{tikzpicture}
        \item
        \begin{tikzpicture}[>=stealth]
            \tkzTabInit[nocadre=false,lgt=1,espcl=1.5,deltacl=0.5]{$x$/.7 ,$y'$/.7,$y$/2}
            {$-\infty$ , $0$,$1$ , $+\infty$}
            \tkzTabLine{ , + , 0,-, d , + , }
            \tkzTabVar{-/$0$, +/$2$ ,-D-/$-\infty$/$3$ , +/$5$}
        \end{tikzpicture}
        % \item
        % \begin{tikzpicture}[>=stealth]
        %     \tkzTabInit[nocadre=false,lgt=1,espcl=1.8,deltacl=0.5]{$x$/.7 ,$y'$/.7,$y$/2}
        %     {$-\infty$ , $-1$,$1$ , $+\infty$}
        %     \tkzTabLine{ , - , d,-, 0 , + , }
        %     \tkzTabVar{+/$2$ ,-D+/$-5$/$3$, -/$-1$ , +/$+\infty$}
        % \end{tikzpicture}
        % \item
        % \begin{tikzpicture}[>=stealth]
        %     \tkzTabInit[nocadre=false,lgt=1,espcl=1.4,deltacl=0.5]{$x$/.7 ,$y'$/.7,$y$/2}
        %     {$-\infty$ , $-2$, $0$,$1$ , $+\infty$}
        %     \tkzTabLine{ , - , d,-, 0 , + ,d,-, }
        %     \tkzTabVar{+/$-1$ ,-D+/$-\infty$/$2$, -/$-4$, +/$3$ , -/$0$}
        % \end{tikzpicture}
    \end{listEX}
    \loigiai{}
\end{vd}
\begin{vd}
    Một bể bơi chứa $5\,000$ lít nước tinh khiết. Người ta bơm vào bể đó nước muối có nồng đồ $30$ gam muối cho mỗi lít nước với tốc độ $25$ lít/phút.
    \begin{listEX}
        \item Lập hàm số biểu diễn nồng độ muối trong bể sau $t$ phút.
        \item Tìm tiệm cận ngang của hàm số vừa tìm được.
        \item Nêu nhận xét về nồng độ muối trong bể khi thời gian $t$ ngày càng lớn.
    \end{listEX}
    \loigiai{
        \begin{enumerate}[a)]
            \immini{\item Sau $t$ phút, ta có: khối lượng muối trong bể là $25\cdot 30\cdot t=750t$ (gam); thể tích của lượng nước trong bể là $5\,000+25t$ (lít). Vậy nồng độ muối sau $t$ phút là
                $$f(t)=\dfrac{750t}{5\,000+25t}=\dfrac{30t}{200+t}\,\text{(gam/lít)}.$$
                \item Ta có\\
                $\lim\limits_{t\to +\infty}f(t)=\lim\limits_{t \to +\infty}\dfrac{30t}{200+t}=\lim\limits_{t\to +\infty}\left(30-\dfrac{6\,000}{200+t}\right)=30$.\\
                Vậy đường thẳng $y=30$ là tiệm cận ngang của đồ thị hàm số $f(t)$ \texttt{(Hình 17).}}{\begin{tikzpicture}[scale=.1,xscale=0.1, font=\footnotesize, line join=round, line cap=round, >=stealth]
                    \draw[->] (-.5,0)--(0,0) node[below right]{$O$}--(500,0) node[below]{$x$};
                    \draw[->] (0,-1) --(0,34) node[right]{$y$};
                    \draw[blue] [domain=0:500, samples=100] %
                    plot (\x, {(30*(\x))/((\x)+200)});
                    \draw[fill] (0,0) circle (1pt);
                    \foreach \y/\g in {30/180}
                    \draw[fill] (0,\y) circle(1pt)node [shift={(\g:.3)}] {$\y$};
                    \draw[thick] (-.1,30)--(500,30);
                    \draw (250,-5) node{Hình 17};
            \end{tikzpicture}}
            \item Ta có đồ thị hàm số $y=f(t)$ nhận đường thẳng $y=30$ làm đường tiệm cận ngang, tức là khi $t$ càng lớn thì nồng độ muối trong bể sẽ tiến gần đến mức $30$ (gam/lít). Lúc đó, nồng độ muối trong bể sẽ gần như bằng nồng độ nước muối bơm vào bể.
        \end{enumerate}
    }
\end{vd}
\begin{vd}
    Một mô hình kinh tế mô tả lượng cung cầu theo giá cả được cho bởi hàm:
    \[
    Q(p) = \frac{k}{p - p_0}
    \]
    trong đó \( Q(p) \) là lượng cung cầu, \( p \) là giá cả, \( p_0 \) là mức giá tối thiểu, và \( k \) là hằng số tỷ lệ. Xác định tiệm cận đứng của hàm số này và nêu ý nghĩa của nó.
    % \shortans{$Q=p$, khi giá giảm về mức tối thiểu thì nhu cầu tăng lên vô hạn}
    \loigiai{
        Để tìm tiệm cận đứng, ta xem xét các giá trị của \( p \) làm cho mẫu số của phương trình bằng 0:
        \[
        p - p_0 = 0 \Rightarrow p = p_0
        \]
        Vậy đường thẳng \( p = p_0 \) là tiệm cận đứng của đồ thị hàm số.
        \textbf{Ý nghĩa:} Từ đó ta suy ra khi giá cả \( p \) càng sát với \( p_0 \), lượng cung cầu \( Q(p) \) sẽ tăng lên vô hạn. Điều này có nghĩa là nếu giá cả của sản phẩm giảm gần bằng mức giá tối thiểu \( p_0 \), thì nhu cầu đối với sản phẩm đó sẽ tăng lên vô hạn.}
\end{vd}
\BTTN
\Opensolutionfile{ans}[ans/2D1-4-DANG-1]
\begin{ex}%[Nguyễn Văn Sang, dự án Tex hoá đề cương trường Marie Curie - Lần 6]%[2D1Y4-1]
    Đường thẳng nào dưới đây là tiệm cận ngang của đồ thị hàm số $y=\dfrac{x-1}{x+1}$?
    \choice
    {$y=-1$}
    {$x=-1$}
    {\True $y=1$}
    {$x=1$}
    \loigiai{
        Tập xác định $\mathscr{D}=\mathbb{R}\setminus\left\lbrace -1\right\rbrace$.
        \begin{itemize}
            \item $\lim\limits_{x \to \pm\infty} y=\lim\limits_{x \to \pm\infty} \dfrac{x-1}{x+1}=1$ suy ra $y=1$ là tiệm cận ngang.
            \item $\heva{& \lim\limits_{x \to -1^+} \dfrac{x-1}{x+1}=-\infty \\ & \lim\limits_{x \to -1^-} \dfrac{x-1}{x+1}=+\infty}$ suy ra $x=-1$ là tiệm cận đứng.
        \end{itemize}
    }
\end{ex}
%%=====Câu 13
\begin{ex}%[Nguyễn Văn Sang, dự án Tex hoá đề cương trường Marie Curie - Lần 6]%[2D1Y4-1]
    Đồ thị hàm số $y=\dfrac{2 x-3}{1-2 x}$ có tiệm cận đứng là đường thẳng
    \choice
    {$x=3$}
    {$x=2$}
    {\True $x=\dfrac{1}{2}$}
    {$x=\dfrac{3}{2}$}
    \loigiai{
        Tập xác định $\mathscr{D}=\mathbb{R}\setminus\left\lbrace \dfrac{1}{2}\right\rbrace$.
        \begin{itemize}
            \item $\lim\limits_{x \to \pm\infty} y=\lim\limits_{x \to \pm\infty} \dfrac{2x-3}{1-2x}=-1$ suy ra $y=-1$ là tiệm cận ngang.
            \item $\heva{& \lim\limits_{x \to \tfrac{1}{2}^+} \dfrac{2x-3}{1-2x}=+\infty \\ & \lim\limits_{x \to \tfrac{1}{2}^-} \dfrac{2x-3}{1-2x}=-\infty}$ suy ra $x=\dfrac{1}{2}$ là tiệm cận đứng.
        \end{itemize}
    }
\end{ex}
\begin{ex}%[Nguyễn Văn Sang, dự án Tex hoá đề cương trường Marie Curie - Lần 6]%[2D1Y4-1]
    Đồ thị hàm số $y=\dfrac{2-3 x}{2 x-3}$ có tiệm cận đứng và ngang lần lượt là
    \choice
    {\True $x=\dfrac{3}{2}$ và $y=-\dfrac{3}{2}$}
    {$x=\dfrac{3}{2}$ và $y=1$}
    {$x=\dfrac{2}{3}$ và $y=-\dfrac{3}{2}$}
    {$x=\dfrac{2}{3}$ và $y=1$}
    \loigiai{
        Tập xác định $\mathscr{D}=\mathbb{R}\setminus\left\lbrace \dfrac{3}{2}\right\rbrace$.
        \begin{itemize}
            \item $\lim\limits_{x \to \pm\infty} y=\lim\limits_{x \to \pm\infty} \dfrac{2-3 x}{2 x-3}=\dfrac{-3}{2}$ suy ra $y=-\dfrac{3}{2}$ là tiệm cận ngang.
            \item $\heva{& \lim\limits_{x \to \tfrac{3}{2}^+} \dfrac{2-3 x}{2 x-3}=-\infty \\ & \lim\limits_{x \to \tfrac{3}{2}^-} \dfrac{2-3 x}{2 x-3}=+\infty}$ suy ra $x=\dfrac{3}{2}$ là tiệm cận đứng.
        \end{itemize}
    }
\end{ex}
\begin{ex}%[BGD-THPT-2020-104-L2]%[2D1Y4-1]
    Tiệm cận đứng của đồ thị hàm số $y=\dfrac{x+1}{x+3}$ có phương trình là
    \choice
    {$x=-1$}
    {$x=1$}
    {\True $x=-3$}
    {$x=3$}
    \loigiai{
        Tập xác định của hàm số đã cho $\mathscr{D}=\mathbb{R}\setminus\{-3\}$.\\
        Ta có $\lim\limits_{x\rightarrow-3^-}y=\lim\limits_{x\rightarrow-3^-}\dfrac{x+1}{x+3}=+\infty$ và $\lim\limits_{x\rightarrow-3^+}y=\lim\limits_{x\rightarrow-3^+}\dfrac{x+1}{x+3}=-\infty$.\\
        Khi đó đường tiệm cận đứng của đồ thị hàm số đã cho là $x=-3$.
    }
\end{ex}
%%==========Câu 11
\begin{ex}%[BGD-Minh Họa-2020-L2]%[2D1Y4-1]
    Tiệm cận ngang của đồ thị hàm số $y=\dfrac{x-2}{x+1}$ có phương trình là
    \choice
    {$y=-2$}
    {\True $y=1$}
    {$x=-1$}
    {$x=2$}
    \loigiai
    {
        Tập xác định: $\mathscr{D}=\mathbb{R}\setminus \{-1\}$.\\
        Ta có $\lim \limits_{x \to +\infty} y=\lim \limits_{x \to +\infty} \dfrac{x-2}{x+1}=\lim \limits_{x \to +\infty} \dfrac{1-\dfrac{2}{x}}{1+\dfrac{1}{x}}=1$ và $\lim \limits_{x \to -\infty} y=\lim \limits_{x \to -\infty} \dfrac{x-2}{x+1}=\lim \limits_{x \to -\infty} \dfrac{1-\dfrac{2}{x}}{1+\dfrac{1}{x}}=1$ nên đường thẳng $y=1$ là đường tiệm cận ngang của đồ thị.
    }
\end{ex}
%%==========Câu 12
\begin{ex}%[BGD-THPT-2021-101-L1]%[2D1Y4-1]
    Tiệm cận đứng của đồ thị hàm số $ y=\dfrac{2x-1}{x-1}$ là đường thẳng có phương trình là
    \choice
    {\True $x=1$}
    {$x=-1$}
    {$x=2$}
    {$x=\dfrac{1}{2}$}
    \loigiai{
        Vì $\lim\limits_{x\to 1^+}\dfrac{2x-1}{x-1}=+\infty $ và $\lim\limits_{x\to 1^-}\dfrac{2x-1}{x-1}=-\infty $ nên đồ thị hàm số $ y=\dfrac{2x-1}{x-1}$ có một tiệm cận đứng là đường thẳng $ x=1 $.
    }
\end{ex}
%%==========Câu 13
\begin{ex}%[BGD-THPT-2021-102-L1]%[2D1Y4-1]
    Tiệm cận đứng của đồ thị hàm số $ y=\dfrac{x+1}{x-2}$ là đường thẳng có phương trình là
    \choice
    {$x=-1$}
    {$x=-2$}
    {\True $x=2$}
    {$x=1$}
    \loigiai{
        Ta có $\displaystyle\lim\limits_{x\to 2^+}\dfrac{x+1}{x-2}=+\infty $; $\displaystyle\lim\limits_{x\to 2^-}\dfrac{x+1}{x-2}=-\infty $.\\
        Vậy đồ thị hàm số $ y=\dfrac{x+1}{x-2}$ có tiệm cận đứng là đường thẳng $ x=2 $.
    }
\end{ex}
%%==========Câu 14
\begin{ex}%[BGD-THPT-2021-103-L1]%[2D1B4-1]
    Tiệm cận đứng của đồ thị hàm số $ y=\dfrac{2x+1}{x-1}$ là đường thẳng có phương trình là
    \choice
    {$x=2$}
    {\True $x=1$}
    {$x=-\dfrac{1}{2}$}
    {$x=-1$}
    \loigiai{
        Ta có $\lim\limits_{x\to 1^+}y=\lim\limits_{x\to 1^+}\dfrac{2x+1}{x-1}=+\infty $ nên tiệm cận đứng của đồ thị hàm số là đường thẳng $ x=1 $.
    }
\end{ex}
%%==========Câu 15
\begin{ex}%[BGD-THPT-2021-104-L1]%[2D1Y4-1]
    Tiệm cận đứng của đồ thị hàm số $ y=\dfrac{x-1}{x+2}$ là đường thẳng có phương trình là
    \choice
    {$x=2$}
    {$x=-1$}
    {\True $x=-2$}
    {$x=1$}
    \loigiai{
        Ta có $\lim\limits_{x\to (-2)^{+}}\dfrac{x-1}{x+2}=-\infty $, $\lim\limits_{x\to (-2)^{-}}\dfrac{x-1}{x+2}=+\infty $.\\
        Đồ thị hàm số có tiệm cận đứng là đường thẳng có phương trình $ x=-2 $.
    }
\end{ex}
\begin{ex}%[2D1Y4-1]
    Giao điểm của tiệm cận đứng và tiệm cận ngang của đồ thị hàm số $y=\dfrac{-2}{3x-1}$ là điểm
    \choice
    {$Q\left(\dfrac{1}{3};-2\right)$}
    {$M\left(\dfrac{1}{3};-\dfrac{2}{3}\right)$}
    {$N\left(\dfrac{1}{3};2\right)$}
    {\True $P\left(\dfrac{1}{3};0\right)$}
    \loigiai{
        Tiệm cận đứng, tiệm cận ngang của đồ thị hàm số lần lượt là $x=\dfrac{1}{3}$ và $y=0$. Giao điểm của $2$ tiệm cận là $P\left(\dfrac{1}{3};0\right)$.
    }
\end{ex}
\begin{ex}%[2D1Y4-1]
    Đồ thị hàm số $y=\dfrac{3-4x}{x-5}$ có tâm đối xứng là điểm
    \choice
    {$M\left(5;-\dfrac{3}{5}\right)$}
    {$P\left(5;\dfrac{4}{5}\right)$}
    {$Q(5;3)$}
    {\True $N(5;-4)$}
    \loigiai{
        Tiệm cận đứng, tiệm cận ngang của đồ thị hàm số lần lượt là $x=5$ và $y=-4$. Tâm đối xứng là điểm $N(5;-4)$.
    }
\end{ex}
\begin{ex}%[2D1B4-1]
    Đồ thị hàm số nào dưới đây có tiệm cận đứng?
    \choice
    {$y=\dfrac{x^2-3x+2}{x-1}$}
    {$y=\dfrac{x^2}{x^2+1}$}
    {$y=\sqrt{x^2-1}$}
    {\True $y=\dfrac{x}{x+1}$}
    \loigiai{
    }
\end{ex}
\begin{ex}%[2D1B4-1]
    Cho hàm số $y=f(x)$ có bảng biến thiên như hình bên. Tổng số tiệm cận đứng và tiệm cận ngang của đồ thị hàm số đã cho là
    \begin{center}
        \begin{tikzpicture}
            \tkzTabInit[nocadre=false,lgt=1.5,espcl=3,deltacl=0.6]
            {$x$ /0.6,$y’$ /0.6,$y$ /2}
            {$-\infty$ ,$0$, $1$, $+\infty$}
            \tkzTabLine{,-,d,+,0,-,}
            \tkzTabVar{+/$+\infty$,-D-/$-\infty$/$-1$,+/$2$,-/$-3$}
        \end{tikzpicture}
    \end{center}
    \choice
    {$1$}
    {$3$}
    {\True $2$}
    {$4$}
    \loigiai{Dựa vào bảng biến thiên ta thấy đồ thị hàm số có tiệm cận đứng $x=0$ và tiệm cận ngang $y=-3$.}
\end{ex}
\begin{ex}%[2D1B4-1]
    Cho hàm số $y=f(x)$ có bảng biến thiên như hình bên. Tổng số tiệm cận đứng và tiệm cận ngang của đồ thị hàm số đã cho là
    \begin{center}
        \begin{tikzpicture}[scale=0.8]
            \tkzTabInit[nocadre=false,lgt=1.5,espcl=3,deltacl=0.6]
            {$x$ /0.6,$y’$ /0.6,$y$ /2}
            {$-\infty$ , $0$,$2$, $+\infty$}
            \tkzTabLine{,-,0,+,d,-,}
            \tkzTabVar{+/$8$,-/$1$,+/$4$,-/$2$}
        \end{tikzpicture}
    \end{center}
    \choice
    {$1$}
    {$3$}
    {\True $2$}
    {$4$}
    \loigiai{
        Dựa vào bảng biến thiên ta thấy đồ thị hàm số có tiệm cận ngang $y=8$ và $y=2$.
    }
\end{ex}
\begin{ex}%[2D1B4-1]
    Cho hàm số $y=f(x)$ có bảng biến thiên như hình bên. Tổng số tiệm cận đứng và tiệm cận ngang của đồ thị hàm số đã cho là
    \begin{center}
        \begin{tikzpicture}[scale=0.8]
            \tkzTabInit[nocadre=false,lgt=1.5,espcl=3,deltacl=0.6]
            {$x$ /0.6,$y’$ /0.6,$y$ /2}
            {$-\infty$ ,$1$, $2$, $+\infty$}
            \tkzTabLine{,+,d,-,d,+,}
            \tkzTabVar{-/$-4$,+/$3$,-/$-5$,+/$+\infty$}
        \end{tikzpicture}
    \end{center}
    \choice
    {\True $1$}
    {$3$}
    {$2$}
    {$0$}
    \loigiai{
        Dựa vào bảng biến thiên ta thấy đồ thị hàm số có một tiệm cận ngang $y=-4$.
    }
\end{ex}
\begin{ex}%[2D1B4-1]
    Cho hàm số $y=f(x)$ có bảng biến thiên như hình bên. Đồ thị hàm số đã cho có tiệm cận đứng là đường thẳng
    \begin{center}
        \begin{tikzpicture}[scale=0.8, font=\footnotesize, line join=round, line
            cap=round, >=stealth]
            \tkzTabInit[espcl=2.5,lgt=1,nocadre=false]
            {$x$/0.7,$f(x)$/2.1}
            {$-\infty$,$0$,$1$,$2$,$+\infty$}
            \tkzTabVar{-/$-\infty$,+/$2$,-D+/$-\infty$/$+\infty$,-/$4$,+/$+\infty$}
        \end{tikzpicture}
    \end{center}
    \choice
    {$x=0$}
    {\True $x=1$}
    {$x=2$}
    {$x=4$}
    \loigiai{Dựa vào bảng biến thiên ta thấy đồ thị hàm có tiệm cận đứng $x=1$.}
\end{ex}
%%==========Câu 16
\begin{ex}%[BGD-THPT-2019-103]%[2D1B4-1]
    Cho hàm số $y=f(x)$ có bảng biến thiên như sau
    \begin{center}
        \begin{tikzpicture}[scale=1, font=\footnotesize,line join=round, >=stealth]
            \tkzTabInit[nocadre=false,lgt=1.5,espcl=3]{$x$/.7,$y'$/.7,$y$/2.5}{$-\infty$,$0$,$3$,$+\infty$}%
            \tkzTabLine{,-,d,+,0,-,}%
            \tkzTabVar{+/$1$ , -D+/$-\infty$/$2$,-/$-3$, +/$3$}%
        \end{tikzpicture}
    \end{center}
    Tổng số tiệm cận đứng và tiệm cận ngang của đồ thị hàm số đã cho là
    \choice
    {1}
    {2}
    {\True 3}
    {4}
    \loigiai{
        Nhìn bảng biến thiên ta thấy\\
        $\lim\limits_{x \to 0^-} f(x)=-\infty \Rightarrow x=0$ là TCĐ của đồ thị hàm số.\\
        $\lim\limits_{x \to +\infty} f(x)=3 \Rightarrow y=3$ là TCN của đồ thị hàm số.\\
        $\lim\limits_{x \to -\infty} f(x)=1 \Rightarrow y=1$ là TCN của đồ thị hàm số.\\
        Vậy hàm số có 3 tiệm cận.}
\end{ex}
%%==========Câu 17
\begin{ex}%[BGD-THPT-2019-102]%[2D1B4-1]
    Cho hàm số $f(x)$ có bảng biến thiên như sau
    \begin{center}
        \begin{tikzpicture}[scale=1, font=\footnotesize,line join=round, >=stealth]
            \tkzTabInit[lgt=1.2,espcl=3]
            {$x$/0.8,$f’(x)$/0.8,$f(x)$/2}
            {$-\infty$,$0$,$1$,$+\infty$}
            \tkzTabLine{ ,-,d,-,0,+,}
            \tkzTabVar{+/$0$,-D+/$-\infty$/$2$,-/$-2$,+/$+\infty$}
        \end{tikzpicture}
    \end{center}
    Tổng số tiệm cận đứng và tiệm cận ngang của đồ thị hàm số đã cho là
    \choice
    {$3$}
    {$1$}
    {\True $2$}
    {$4$}
    \loigiai{
        Từ bảng biến thiên đã cho ta có\\
        $\lim\limits_{x \to -\infty} f(x)=0$ nên đường thẳng $y=0$ là một tiệm cận ngang của đồ thị hàm số.\\
        $\lim\limits_{x \to 0^-} f(x)=-\infty$ nên đường thẳng $x=0$ là một tiệm cận đứng của đồ thị hàm số.\\
        Vậy đồ thị hàm số đã cho có hai đường tiệm cận.}
\end{ex}
\begin{ex}%[2D1B4-1]
    Cho hàm số $y=f(x)$ có bảng biến thiên như hình bên. Tổng số tiệm cận đứng và tiệm cận ngang của đồ thị hàm số đã cho là
    \begin{center}
        \begin{tikzpicture}[scale=0.8]
            \tkzTabInit[nocadre=false,lgt=1.5,espcl=3,deltacl=0.6]
            {$x$ /0.6,$y’$ /0.6,$y$ /2}
            {$-\infty$ ,$0$, $1$, $+\infty$}
            \tkzTabLine{,+,0,-,d,-,}
            \tkzTabVar{-/$4$,+/$2$,-D+/$-\infty$/$5$,-/$-3$}
        \end{tikzpicture}
    \end{center}
    \choice
    {$1$}
    {\True $3$}
    {$2$}
    {$4$}
    \loigiai{
        Dựa vào bảng biến thiên ta thấy đồ thị hàm số có tiệm cận đứng $x=1$, tiệm cận ngang $y=4$ và $y=-3$.
    }
\end{ex}
\begin{ex}%[2D1B4-1]
    Cho hàm số $y=f\left(x\right)$ có bảng biến thiên như sau
    \begin{center}
        \begin{tikzpicture}[scale=1,line join=round,>=stealth]\tikzset{double style/.append style={double distance=2pt}}
            \tkzTabInit[nocadre=false,lgt=1.2,espcl=2.2,deltacl=0.6]
            {$x$ /.6,$y'$ /.6,$y$ /2.2}
            {$ -\infty $,$-2$,$0$,$+\infty$}
            \tkzTabLine{,-,d,+,d,-}
            \tkzTabVar{+/$+\infty$,-D-/$1$/$-\infty$,+D+/$+\infty$/$1$,-/$0$,}
        \end{tikzpicture}
    \end{center}
    Tổng số đường tiệm cận đứng và tiệm cận ngang của đồ thị hàm số đã cho bằng
    \choice
    {$2$}
    {$1$}
    {$0$}
    {\True $3$}
    \loigiai{
        Ta có
        \begin{itemize}
            \item $\lim\limits_{x \to -2^{+}} y=-\infty \Rightarrow x=-2$ là tiệm cận đứng.
            \item $\lim\limits_{x \to 0^{-}} y=+\infty \Rightarrow x=0$ là tiệm cận đứng.
            \item $\lim\limits_{x \to +\infty} y=0 \Rightarrow y=0$ là tiệm cận ngang.
        \end{itemize}
        Vậy đồ thị hàm số đã cho có tổng đường tiệm cận đứng và tiệm cận ngang là $3$.}
\end{ex}
\begin{ex}%[2D1B4-1]
    Cho hàm số $y=f\left(x\right)$ liên tục trên $\mathbb{R} \backslash\{1\}$ có bảng biến thiên như bảng sau:
    \begin{center}
        \begin{tikzpicture}[scale=1,line join=round,>=stealth]
            \tikzset{double style/.append style={double distance=2pt}}
            \tkzTabInit[nocadre=false,lgt=1.2,espcl=2.8,deltacl=0.6]
            {$x$ /0.6,$y'$ /0.6,$y$ /2.2}
            {$ -\infty $,$-1$,$1$,$+\infty$}
            \tkzTabLine{,-,0,+,d,+}
            \tkzTabVar{+/$1$,-/$-\sqrt 2$,+D-/$+\infty$/$-\infty$,+/$-1$,}
        \end{tikzpicture}
    \end{center}
    Tổng số đường tiệm cận đứng và đường tiệm cận ngang của đồ thị hàm số $y=f\left(x\right)$ là
    \choice
    {$1$}
    {$4$}
    {$2$}
    {\True $3$}
    \loigiai{
        Do $\lim\limits_{x \to 1^{+}} y=-\infty \Rightarrow$ Tiệm cận đứng $x=1$.\\
        Lại có $\lim\limits_{x \to +\infty} y=-1 ; \lim\limits_{x \to -\infty} y=1 \Rightarrow$ Đồ thị có $2$ tiệm cận ngang là $y=\pm 1$.\\
        Vậy, đồ thị hàm số đã cho có tổng số tiệm cận là $3$.}
\end{ex}
\begin{ex}%[2D1B4-1]
    Cho hàm số $y=f\left(x\right)$ có bảng biến như sau:
    \begin{center}
        \begin{tikzpicture}[scale=1,line join=round,>=stealth]
            \tikzset{double style/.append style={double distance=2pt}}
            \tkzTabInit[nocadre=false,lgt=1.2,espcl=2.5,deltacl=0.6]{$x$ /.6,$y'$ /.6,$y$ /2}
            {$ -\infty $,$-3$,$3$,$+\infty$}
            \tkzTabLine{,+,d,+,d,+}
            \tkzTabVar{-/$0$,+D-/$+\infty$/$-\infty$,+D-/$+\infty$/$-\infty$,+/$0$,}
        \end{tikzpicture}
    \end{center}
    Số đường tiệm cận của đồ thị hàm số là
    \choice
    {\True $3$}
    {$1$}
    {$4$}
    {$2$}
    \loigiai{
        Từ bảng biến thiên của hàm số ta có
        \begin{itemize}
            \item $\lim\limits_{x \to -\infty} y=0 ; \lim\limits_{x \to +\infty} y=0 \Rightarrow$ Đường thẳng $y=0$ là tiệm cận ngang.
            \item $\lim\limits_{x \to (-3)^{-}} y=+\infty \Rightarrow$ Đường thẳng $x=-3$ là tiệm cận đứng.
            \item $+\lim\limits_{x \to 3^{-}} y=+\infty \Rightarrow$ Đường thẳng $x=3$ là tiệm cận đứng.
        \end{itemize}
        Vậy số đường tiệm cận của đồ thị hàm số là $3$.}
\end{ex}
\begin{ex}%[2D1B4-1]
    Cho hàm số $y=f\left(x\right)$ có bảng biến thiên như sau
    \begin{center}
        \begin{tikzpicture}[scale=1,line join=round,>=stealth]
            \tikzset{double style/.append style={double distance=2pt}}
            \tkzTabInit[nocadre=false,lgt=1.2,espcl=2.5,deltacl=0.6]
            {$x$ /.6,$y'$ /.6,$y$ /2.2}
            {$ -\infty $,$-2$,$2$,$+\infty$}
            \tkzTabLine{,-,d,-,d,-}
            \tkzTabVar{+/$0$,-D+/$-\infty$/$+\infty$,-D+/$-\infty$/$+\infty$,-/$-\infty$,}
        \end{tikzpicture}
    \end{center}
    Tổng số tiệm cận đứng và tiệm cận ngang của đồ thị hàm số đã cho là
    \choice
    {$4$}
    {$2$}
    {\True $3$}
    {$1$}
    \loigiai{
        Dựa vào bảng biến thiên, ta có:
        \begin{itemize}
            \item $\lim\limits_{x \to -\infty} f(x)=0$ nên đường thẳng $y=0$ là đường tiệm cận ngang.
            \item $\lim\limits_{x \to -2^{+}} f(x)=+\infty $ nên đường thẳng $x=-2$ là đường tiệm cận đứng.
            \item $\lim\limits_{x \to 2^{+}} f(x)=+\infty$ nên đường thẳng $x=2$ là đường tiệm cận đứng.
        \end{itemize}
        Vậy, tổng số tiệm cận đứng và tiệm cận ngang của đồ thị hàm số đã cho là $3$.}
\end{ex}
%%==========Câu 20
\begin{ex}%[THPT Yên Định - Thanh Hóa 2019]%[2D1B4-1]
    Cho hàm số $ y=f(x) $ xác định và có đạo hàm trên $ \mathbb{R}\setminus\{\pm 1\} $. Hàm số có bảng biến thiên như hình vẽ dưới đây.
    \begin{center}
        \begin{tikzpicture}[scale=1, font=\footnotesize,line join=round, >=stealth]
            \tkzTabInit[nocadre=false,lgt=1.2,espcl=2.5,deltacl=0.6]{$x$/.6 ,$y'$/.6,$y$/2.5} {$-\infty$ , $-1$ , $0$ , $1$ , $+\infty$}
            \tkzTabLine{ , + , d , - , d , + , d , + , }
            \tkzTabVar{-/$-4$ , +D-/$+\infty$/$-\infty$ , +/$2$,-D-/$-\infty$/$-\infty$,+/$-1$}
        \end{tikzpicture}
    \end{center}
    Tổng số đường tiệm cận đứng và tiệm cận ngang của đồ thị hàm số đã cho là
    \choice
    { $ 1 $ }
    { $ 2 $ }
    { $ 3 $ }
    {\True $ 4 $ }
    \loigiai{
        Dựa vào bảng biến thiên, suy ra:\\
        $ \lim \limits_{x \to - \infty} y=-4 $, $ \lim \limits_{x \to + \infty} y=-1$. Đồ thị có hai tiệm cận ngang là $ y=-4 $ và $ y=-1 $.\\
        Lại có $ \lim \limits_{x \to (-1)^+} y=+\infty $ và $ \lim \limits_{x \to 1^-} y=+\infty $, $ \lim \limits_{x \to 1^-} y=-\infty $. Đồ thị hàm số có hai đường tiệm cận đứng là $ x=1 $ và $ x=-1 $.
    }
\end{ex}
\begin{ex}%[2D1B4-1]
    Cho hàm số $y=f\left(x\right)$ có bảng biến thiên như hình vẽ dưới đây.
    \begin{center}
        \begin{tikzpicture}[line cap=round,line join=round,>=triangle 45,x=1.0cm,y=1.0cm]
            \clip(-1.58,-2.4) rectangle (12.58,2.);
            \fill[line width=1.2pt,dash pattern=on 15 pt off 5pt,color=white,fill=black,pattern=north east lines,pattern color=black] (0.,1.) -- (2.84,1.) -- (2.84,-1.96) -- (0.,-1.96) -- cycle;
            \draw (-1.,1.)-- (12.,1.);
            \draw (-1.,0.)-- (12.,0.);
            \draw (0.,1.62)-- (0.,-1.96);
            \draw (0.08,1.5) node[anchor=north west] {$-\infty$};
            \draw (2.46,1.5) node[anchor=north west] {$-2$};
            \draw (6.85,1.5) node[anchor=north west] {$0$};
            \draw (11.14,1.5) node[anchor=north west] {$+\infty$};
            \draw (2.84,1.)-- (2.84,-1.96);
            \draw (3.,1.)-- (3.,-1.96);
            \draw (7.,1.)-- (7.,-1.96);
            \draw (7.14,1.)-- (7.14,-1.96);
            \draw (-0.7,1.5) node[anchor=north west] {$x$};
            \draw (-0.72,0.78) node[anchor=north west] {$y'$};
            \draw (-0.64,-0.8) node[anchor=north west] {$y$};
            \draw [->] (3.96,-1.54) -- (6.24,-0.64);
            \draw [->] (7.64,-0.58) -- (11.12,-1.7);
            \draw (11.34,-1.5) node[anchor=north west] {$0$};
            \draw (8.98,0.7) node[anchor=north west] {$-$};
            \draw (4.68,0.7) node[anchor=north west] {$+$};
            \draw (6.0,-0.2) node[anchor=north west] {$+\infty$};
            \draw (7.22,-0.2) node[anchor=north west] {$1$};
            \draw (3.08,-1.5) node[anchor=north west] {$-\infty$};
        \end{tikzpicture}
    \end{center}
    Hỏi đồ thị của hàm số đã cho có bao nhiêu đường tiệm cận?
    \choice
    {\True $3$}
    {$2$}
    {$4$}
    {$1$}
    \loigiai{
        Dựa vào bảng biến thiên ta có:\\
        $\lim\limits_{x \to -2^{+}} f(x)=-\infty$, suy ra đường thẳng $x=-2$ là tiệm cận đứng của đồ thị hàm số.\\
        $\lim\limits_{x \to 0^{-}} f(x)=+\infty$, suy ra đường thẳng $x=0$ là tiệm cận đứng của đồ thị hàm số.\\
        $\lim\limits_{x \to +\infty} f(x)=0$, suy ra đường thẳng $y=0$ là tiệm cận ngang của đồ thị hàm số.\\
        Vậy đồ thị hàm số có $3$ đường tiệm cận.}
\end{ex}
%%=====Câu 5
\begin{ex}%[Nguyễn Văn Sang, dự án Tex hoá đề cương trường Marie Curie - Lần 6]%[2D1Y4-1]
    Cho hàm số $y=f(x)$ có $\lim\limits_{x \rightarrow 3} f(x)=+\infty$, $\lim\limits_{x \rightarrow+\infty} f(x)=-\infty$, $\lim\limits_{x \rightarrow-\infty} f(x)=8$ và $\lim\limits_{x \rightarrow 7} f(x)=5 $. Tổng số tiệm cận ngang và tiệm cận đứng của đồ thị hàm số đã cho là
    \choice
    {$4$}
    {\True $2$}
    {$1$}
    {$3$}
    \loigiai{
        Ta có
        \begin{itemize}
            \item $\lim\limits_{x \rightarrow-\infty} f(x)=8$, suy ra $y=8$ là tiệm cận ngang.
            \item $\lim\limits_{x \rightarrow 3} f(x)=+\infty$, suy ra $x=3$ là tiệm cận đứng.
            \item $\lim\limits_{x \rightarrow 7} f(x)=5 $, suy ra $x=7$ không là tiệm cận đứng.
        \end{itemize}
        Vậy đồ thị hàm số có $1$ tiệm cận đứng và $1$ tiệm cận ngang.
    }
\end{ex}
%%=====Câu 7
\begin{ex}%[Nguyễn Văn Sang, dự án Tex hoá đề cương trường Marie Curie - Lần 6]%[2D1Y4-1]
    Cho hàm số $y=f(x)$ có $\lim\limits_{x \rightarrow 1^{+}} f(x)=+\infty$ và $\lim\limits_{x \rightarrow 1^{-}} f(x)=2$. Mệnh đề nào sau đây đúng?
    \choice
    {Đồ thị hàm số không có tiệm cận}
    {\True Đồ thị hàm số có tiệm cận đứng $x=1$}
    {Đồ thị hàm số có hai tiệm cận}
    {Đồ thị hàm số tiệm cận ngang $y=2$}
    \loigiai{
        Ta có $\lim\limits_{x \rightarrow 1^{-}} f(x)=2$, suy ra $x=1$ là tiệm cận đứng.
    }
\end{ex}
\begin{ex}%[2D1B4-1]
    Cho hàm số $y=\dfrac{\sqrt{x+1}}{\sqrt{x^2-4}}$ mệnh đề nào sau đây đúng?
    \choice
    {\True Đồ thị hàm số có một tiệm cận đứng và một tiệm cận ngang}
    {Đồ thị hàm số có một tiệm cận đứng và hai tiệm cận ngang}
    {Đồ thị hàm số có hai tiệm cận đứng và hai tiệm cận ngang}
    {Đồ thị hàm số có hai tiệm cận đứng và một tiệm cận ngang}
    \loigiai{
        Tập xác định $\mathscr{D}=[-1;+\infty) \setminus \{2\}$. \\
        Đồ thị hàm số có một tiệm cận đứng $x=2$, tiệm cận ngang là $y=0$.
    }
\end{ex}
\begin{ex}%[2-HK1-49-THPT-NKKN-TPHCM, 12EX5]%[Nhật Thiện, ID6]%[2D1K4-2]%
    Với giá trị nào của $m$ thì đồ thị hàm số $y=\dfrac{mx-1}{2x+m}$ có tiệm cận đứng là đường thẳng $x=-1$?
    \choice
    {$m=2$}
    {\True $m=-2$}
    {$m=\dfrac{1}{2}$}
    {$m=0$}
    \loigiai{
        Đồ thị hàm số $y=\dfrac{mx-1}{2x+m}$ có tiệm cận đứng là đường thẳng $x=-1$ khi và chỉ khi $$\heva{&m(-1)-1\ne 0\\&2(-1)+m=0}\Leftrightarrow \heva{&m\ne -1\\&m=-2(n).}$$
    }
\end{ex}
\begin{ex}%[2D1K4-1]
    Đồ thị hàm số $y=\dfrac{2x-1-\sqrt{x^2+x+3}}{x^2-5x+6}$ có tất cả đường tiệm cận đứng là đường thẳng
    \choice
    {$x=-3$ và $x=-2$}
    {$x=-3$}
    {$x=3$ và $x=-2$}
    {\True $x=3$}
    \loigiai{
        Điều kiện xác định $x \ne 3$, $x \ne 2$.\\
        Với điều kiện xác định trên, ta có
        {\allowdisplaybreaks
            \begin{eqnarray*}
                y&=&\dfrac{2x-1-\sqrt{x^2+x+3}}{x^2-5x+6}=\dfrac{(3x+1)(x-2)}{(x-2)(x-3)\left(2x-1+\sqrt{x^2+x+3}\right)}\\
                &=&\dfrac{3x+1}{(x-3)\left(2x-1+\sqrt{x^2+x+3}\right)}.
        \end{eqnarray*} }
        Tiệm cận đứng của đồ thị hàm số là $x=3$.
    }
\end{ex}
\begin{ex}%[2D1K4-1]
    Số tiệm cận đứng của đồ thị hàm số $y=\dfrac{\sqrt{x+9}-3}{x^2+x}$ là
    \choice
    {$3$}
    {$2$}
    {$0$}
    {\True $1$}
    \loigiai{
        Tập xác định $\mathscr{D}=[-9;+\infty)\setminus \{-1;0\}$. \\
        Ta có $\left\{\begin{aligned}
            &\lim\limits_{x\to -1^+} y=\lim\limits_{x\to -1^+} \dfrac{\sqrt{x+9}-3}{x^2+x}=+\infty \\
            &\lim\limits_{x\to -1^-} y =\lim\limits_{x\to -1^-} \dfrac{\sqrt{x+9}-3}{x^2+x}=-\infty
        \end{aligned}\right. \Rightarrow x=-1$ là tiệm cận đứng. \\
        Ngoài ra $\lim\limits_{x\to 0} y =\lim\limits_{x\to 0} \dfrac{\sqrt{x+9}-3}{x^2+x}=\dfrac{1}{6}$ nên $x=0$ không phải là một tiệm cận đứng.}
\end{ex}
\BTTF
\begin{ex}%[EX-TF-2024, Lê Đạt]%[2D1N4-1]
    Cho hàm số $y=\dfrac{2x-3}{x-1}$. Xét tính đúng sai các khẳng định dưới đây
    \choiceTF
    {\True Đường tiệm cận đứng của đồ thị hàm số là $ x=1 $}
    {Đường tiệm cận đứng của đồ thị hàm số là $ y=2 $}
    {Đường tiệm cận ngang của đồ thị hàm số là $ x=1 $}
    {\True Đường tiệm cận ngang của đồ thj hàm số là $ y=2 $}
    \loigiai{
        Ta có $\lim\limits_{x\to -\infty}y=\lim\limits_{x\to +\infty}y=2$ nên đồ thị hàm số đã cho có tiệm cận ngang là $y=2$.\\
        Ta có $\lim\limits_{x\to 1^+}y=-\infty$ nên đồ thị hàm số đã cho có tiệm cận ngang là $ x=1 $.
        \begin{itemchoice}
            \itemch Đường tiệm cận đứng của đồ thị hàm số là $ x=1 $.
            \itemch Đường tiệm cận đứng của đồ thị hàm số là $ x=1 $.
            \itemch Đường tiệm cận ngang của đồ thj hàm số là $ y=2 $.
            \itemch Đường tiệm cận ngang của đồ thj hàm số là $ y=2 $.
        \end{itemchoice}
    }
\end{ex}
\begin{ex}%[EX-TF-2024, Lê Đạt]%[2D1N4-1]
    Cho hàm số $y=f(x)$ có bảng biến thiên như sau
    \begin{center}
        \begin{tikzpicture}[>=stealth]
            \tkzTabInit[nocadre=false,lgt=1,espcl=3,deltacl=0.6]
            {$x$/.7 ,$y'$/.7,$y$/2}
            {$-\infty$ , $-2$ , $0$, $+\infty$}
            \tkzTabLine{ , - , d , + , d , -, }
            \tkzTabVar{+/$+\infty$ , -D-/$1$/$-\infty$ , +D+/$+\infty$ /$1$, -/$0$}
        \end{tikzpicture}
    \end{center}
    Xét tính đúng sai của các khẳng định sau
    \choiceTF
    {\True $ x=0 $ là tiệm cận đứng của đồ thị hàm số $ y=f(x) $}
    {\True $ x=-2 $ là tiệm cận đứng của đồ thị hàm số $ y=f(x) $}
    {$ x=1 $ là tiệm cận đứng của đồ thị hàm số $ y=f(x) $}
    {\True $ y=0 $ là tiệm cận ngang của đồ thị hàm số $ y=f(x) $}
    \loigiai{
        \begin{itemchoice}
            \itemch $\lim \limits_{x \to 0^-} f(x)=+\infty\Rightarrow x=0$ là đường tiệm cận đứng của đồ thị hàm số $f(x)$.
            \itemch $\lim \limits_{x \to (-2)^+} f(x)=-\infty\Rightarrow x=-2$ là đường tiệm cận đứng của đồ thị hàm số $f(x)$.
            \itemch Đồ thị hàm số chỉ có hai tiệm cận đứng là $ x=0 $ và $ x=-2 $.
            \itemch $\lim \limits_{x \to +\infty} f(x)=0\Rightarrow y=0$ là đường tiệm cận ngang của đồ thị hàm số $f(x)$.
        \end{itemchoice}
    }
\end{ex}
%===== DẠNG 2
\begin{ex}%[EX-TF-2024, Lê Đạt]%[2D1H4-2]
    Cho hàm số $ y=\dfrac{m^2x+1}{x-1} $. Xét tính đúng sai của các khẳng định sau
    \choiceTF
    {\True Đồ thị hàm số luôn có tiệm cận ngang}
    {\True Đồ thị hàm số luôn có tiệm cận đứng}
    {\True Khi $ m=1$ đồ thị hàm số có $ 2 $ đường tiệm cận}
    {Khi $ m=0 $ đồ thị hàm số có $ 1 $ đường tiệm cận}
    \loigiai{
        \begin{itemchoice}
            \itemch $\lim\limits_{x\to -\infty}y=\lim\limits_{x\to +\infty}y=m^2$ suy ra hàm số luôn có tiệm cận ngang.
            \itemch $\lim\limits_{x\to 1^+}y=+\infty$ nên đồ thị hàm số đã cho có tiệm cận ngang là $ x=1 $.
            \itemch Khi $ m=1 $ ta được hàm số $ y=\dfrac{x+1}{x-1} $ suy ra đồ thì hàm số có $ x=1 $ là tiệm cận đứng và $ y=1 $ là tiệm cận ngang nên đồ thị hàm số có $ 2 $ tiệm cận.
            \itemch Khi $ m=0 $ ta được hàm số $ y=\dfrac{1}{x-1} $ suy ra đồ thì hàm số có $ x=1 $ là tiệm cận đứng và $ y=0 $ là tiệm cận ngang nên đồ thị hàm số có $ 2 $ tiệm cận.
        \end{itemchoice}
    }
\end{ex}
\begin{ex}%[EX-TF-2024, Lê Đạt]%[2D1H4-2]
    Cho hàm số $y=\dfrac{m x^{2}+6 x-2}{x+2}$. Xét tính đúng sai của các khẳng định sau
    \choiceTF
    {Đồ thị hàm số luôn có tiệm cận đứng với mọi $ m $}
    {Đồ thị hàm số không có tiệm cận ngang với mọi $ m $}
    {\True Khi $ m=1 $ đồ thị hàm số có một tiệm cận xiên là $ y=x+4 $ }
    {Đồ thị hàm số luôn có tiệm cận xiên}
    \loigiai{
        \begin{itemchoice}
            \itemch Khi $ m=\dfrac{7}{2} $ hàm số trở thành $y=\dfrac{\dfrac{7}{2} x^{2}+6 x-2}{x+2}=\dfrac{7}{2}\left(x-\dfrac{2}{7} \right) $ suy ra đồ thị hàm số không có tiệm cận đứng.
            \itemch Khi $ m=0 $ hàm số trở thành $ y=\dfrac{6x-2}{x+2} $ từ đó suy ra đồ thị hàm số có $ y=6 $ là tiệm cận ngang.
            \itemch Khi $ m=1 $ hàm số trở thành $ y=\dfrac{x^2+6x-2}{x+2}=x+4-\dfrac{10}{x+2} $ từ đó suy ra $ y=x+4 $ là một tiệm cận ngang.
            \itemch Khi $ m=0 $ hàm số trở thành $ y=\dfrac{6x-2}{x+2} $ từ đó suy ra đồ thị hàm số có $ y=6 $ là tiệm cận ngang, $ x=-2 $ là tiệm cận đứng và không có tiệm cận xiên.
        \end{itemchoice}
    }
\end{ex}
\begin{ex}
    Cho hàm số $y=\dfrac{x-1}{x^2-8 x+m}$, $m$ là tham số. Các mệnh đề sau đúng hay sai?
    \choiceTF
    {\True Đồ thị hàm số có 1 đường tiệm cận ngang}
    {Khi $m<16$ thì đồ thị hàm số có 3 đường tiệm cận}
    {Khi $m=16$ thì đồ thị hàm số có 2 đường tiệm cận đứng}
    {\True Có 14 giá trị nguyên dương của $m$ để đồ thị hàm số có 3 đường tiệm cận}
    \loigiai{
        Ta có $\lim \limits{n \to +\infty}_{x \rightarrow-\infty} \frac{x-1}{x^2-8 x+m}=\lim \limits{n \to +\infty}_{x \rightarrow+\infty} \frac{x-1}{x^2-8 x+m}=0$ nên hàm số có một tiện cận ngang $y=0$.
        Hàm số có 3 đường tiệm cận khi và chỉ khi hàm số có hai đường tiệm cận đứng $\Leftrightarrow$ phương trình $x^2-8 x+m=0$ có hai nghiệm phân biệt khác $1 \Leftrightarrow\left\{\begin{array}{l}\Delta^{\prime}=16-m>0 \\ m-7 \neq 0\end{array} \Leftrightarrow\left\{\begin{array}{l}m<16 \\ m \neq 7\end{array}\right.\right.$.
        Kết hợp với điều kiện $m$ nguyên dương ta có $\quad m \in\{1 ; 2 ; 3 ; \ldots ; 6 ; 8 ; \ldots ; 15\}$. Vậy có 14 giá trị của $m$ thỏa mãn đề bài.}
\end{ex}
\begin{ex}
    Cho hàm số $y=\dfrac{x^2+m x-1}{x-1}\left(C_m\right)$ ( $m$ là tham số). Các mệnh đề sau đúng hay sai?
    \choiceTF
    {\True Để đồ thị $\left(C_m\right)$ của hàm số có tiệm cận xiên thì $m \neq 0$.}
    {\True Để tiệm cận xiên của $\left(C_m\right)$ đi qua $M(2,-5)$ thì $m=-8$}
    { Để tiệm cận xiên của $\left(C_m\right)$ tạo với hai trục toạ độ một tam giác có diện tích bằng 8 thì tổng tất cả các giá trị $m$ tìm được bằng 2}
    { Với $m=3$ thì giao điểm của hai đường tiệm cận của $\left(C_m\right)$ nằm trên Parapol $y=x^2+3$}
    \loigiai{
        Hàm số xác định trên $\mathbb{R} \backslash\{1\}$.
        \begin{listEX}
            \item Ta có $y=x+m+1+\frac{m}{x-1}$
            Để đồ thị $\left(C_m\right)$ của hàm số có tiệm cận xiên thì $m \neq 0$.
            - Với $m \neq 0,\left(C_m\right)$ có tiệm cận xiên
            $y=x+m+1\left(\Delta_m\right)$ vì $\lim \limits{n \to +\infty}_{x \rightarrow \infty}[y-(x+m+1)]=\lim \limits{n \to +\infty}_{x \rightarrow \infty} \frac{m}{x-1}=0$.
            \item Để $\left(\Delta_m\right)$ qua $M(2,-5)$ thì $-5=2+m+1 \Leftrightarrow m=-8$. (thỏa mãn $m \neq 0$ ).
            \item Gọi $A$ là giao điểm của $\Delta_m$ với $O x$. Khi đó $A(-m-1 ; 0)$
            Gọi $B$ là giao điểm của $\Delta_m$ với $O y$. Khi đó $B(0 ; m+1)$.
            Suy ra $S_{\triangle O A B}=\frac{1}{2} O A \cdot O B=\frac{1}{2}|-m-1||m+1|=\frac{1}{2}(m+1)^2$
            Để $S_{\triangle O A B}=8 \Leftrightarrow \frac{1}{2}(m+1)^2=8 \Leftrightarrow\left[\begin{array}{l}m=-5 \\ m=3\end{array}\right.$ (thỏa mãn $m \neq 0$ ).
            \item Ta có với $m \neq 0, x=1$ là tiệm cận đứng vì $\lim \limits{n \to +\infty}_{x \rightarrow 1} y=\infty$ nên $y=x+m+1$ là tiệm cận xiên.
            Khi đó giao điểm của 2 tiệm cận là $I(1, m+2)$.
            Để $I$ nằm trên Parabol $y=x^2+3$ thì $m+2=1+3 \Leftrightarrow m=2(\mathrm{t} / \mathrm{m} m \neq 0)$.
        \end{listEX}
    }
\end{ex}
%===== DẠNG 3
\begin{ex}%[EX-TF-2024, Lê Đạt]%[2D1N4-3]
    \immini{Cho hàm số $y=f(x)$ có đồ thị như hình bên. Xét tính đúng sai của các khẳng định sau
        \choiceTF
        {$ x=2 $ là đường tiệm cận ngang của đồ thị hàm số}
        {\True $ x=-1 $ là đường tiệm cận đứng của đồ thị hàm số}
        {\True Đồ thị hàm số có hai đường tiệm cận}
        {\True Đồ thị hàm số không có tiệm cận xiên}
    }{
        \begin{tikzpicture}[scale=0.5, font=\footnotesize, line join=round, line cap=round, >=stealth]
            \draw[->](-5,0)--(5,0)node[below]{ $x$};
            \draw[->](0,-4)--(0,5)node[right]{ $y$};
            \draw [fill=black,draw=black] (0,0) circle (1pt)node[above left] { $O$};
            \foreach \x in {-1}\draw[shift={(\x,0)}](0pt,-2pt)--(0pt,2pt) node[below left]{ $\x$};
            \foreach \y in {2}\draw[shift={(0,\y)}](-2pt,0pt)--(2pt,0pt)node[above right]{ $\y$};
            \clip(-5,-4) rectangle (5,5);
            \draw[smooth,samples=100,domain=-5:-1.1] plot(\x,{(2*(\x)-1)/((\x)+1)});
            \draw[smooth,samples=100,domain=-0.9:5] plot(\x,{(2*(\x)-1)/((\x)+1)});
            \draw[dashed](-5,2)--(5,2) (-1,-4)--(-1,5);
        \end{tikzpicture}
    }
    \loigiai{
        \begin{itemchoice}
            \itemch $ y=2 $ là đường tiệm cận ngang của đồ thị hàm số.
            \itemch $ x=-1 $ là đường tiệm cận đứng của đồ thị hàm số.
            \itemch $ x=-1 $ là đường tiệm cận đứng và $ y=2 $ là đường tiệm cận ngang của đồ thị hàm số suy ra đồ thị hàm số có hai đường tiệm cận.
            \itemch Đồ thị hàm số không có tiệm cận xiên.
        \end{itemchoice}
    }
\end{ex}
\begin{ex}%[EX-TF-2024, Lê Đạt]%[2D1H4-3]
    \immini{Cho hàm số $y=f(x)$ có đồ thị như hình bên. Xét tính đúng sai của các khẳng định sau
        \choiceTF
        {\True $ x=0 $ là một đường tiệm cận đứng của đồ thị hàm số}
        {$ y=-x $ là một đường tiệm cận xiên của đồ thị hàm số}
        {\True $ y=x $ là một đường tiệm cận xiên của đồ thị hàm số}
        {Đồ thị hàm số có ba đường tiệm cận}
    }{
        \begin{tikzpicture}[scale=.9, font=\footnotesize, line join=round, line cap=round,>=stealth]
            \def\a{0} \def\b{1} \def\c{1} \def\d{-1} % Hệ số
            \def\xmin{-3} \def\xmax{3.5}
            \def\ymin{-2.8} \def\ymax{3.3}
            \draw[color=gray!50,dashed] (\xmin,\ymin) grid (\xmax,\ymax);
            \draw[->] (\xmin,0)--(\xmax,0) node [below]{$x$};
            \draw[->] (0,\ymin)--(0,\ymax) node [left]{$y$};
            \fill (0,0) circle(1pt) node[shift=(-45:0.25)]{$O$};
            \clip (\xmin+0.1,\ymin+0.1) rectangle (\xmax-0.1,\ymax-0.1);
            \draw[smooth,samples=300,domain=-3:3] plot(\x,{\x+1/(7*\x)});
            \draw[dashed,smooth,samples=300,domain=-3:3] plot(\x,{\x});
            %	\fill (-1,0) circle (1.0pt) node[below]{$-1$} (1,0) circle (1.0pt) node[below right]{$1$};
    \end{tikzpicture}}
    \loigiai{
        \begin{itemchoice}
            \itemch $ x=0 $ là một đường tiệm cận đứng của đồ thị hàm số.
            \itemch	$ y=x $ là một đường tiệm cận xiên của đồ thị hàm số.
            \itemch $ y=x $ là một đường tiệm cận xiên của đồ thị hàm số.
            \itemch Đồ thị hàm số có $ x=0 $ là tiệm cận đứng và $ y=x $ là tiệm cận xiên nên có hai tiệm cận.
        \end{itemchoice}
    }
\end{ex}
\BTTL
\begin{ex}%[2D1K4-2]
    Nếu đồ thị hàm số $y=\dfrac{(m+1)x+2}{x-n+1}$ lần lượt nhận trục hoành và trục tung làm đường đường tiệm cận ngang và tiệm cận đứng thì $m+n$ bằng bao nhiêu?
    \shortans{$0$}
    \loigiai{
        Theo đề bài, ta có $\heva{&m+1=0\\&n-1=0} \Leftrightarrow \heva{&m=-1\\&n=1.}$\\
        Suy ra $m+n=0$.
    }
\end{ex}
\begin{ex}%[2D1K4-2]
    Tìm giá trị của $m$ để đồ thị hàm số $y=\dfrac{(2m+1)x+3}{x+1}$ có đường đường tiệm cận đi qua điểm $A(-2;7)$.
    \shortans{m=3}
    \loigiai{
        Từ đề bài, suy ra $2m+1=7 \Leftrightarrow m=3$.\\
        Suy ra $m+n=0$.
    }
\end{ex}
\begin{ex}%[2D1K4-2]
    Cho hàm số $y=\dfrac{-3+mx}{x+n}$. Tìm giá trị của $m$ và $n$ để đồ thị hàm số đã cho có tiệm cận đứng $x=2$ và tiệm cận ngang $y=2$.
    \shortans{$m=2, n=-2$}
    \loigiai{
        Từ yêu cầu đề bài, suy ra $\heva{&m=2\\&-n=2} \Leftrightarrow \heva{&m=2\\&n=-2.}$
    }
\end{ex}
\begin{ex}%[2D1K4-2]
    Để đường tiệm cận đứng và tiệm cận ngang của đồ thị hàm số $y=\dfrac{mx+1}{2m+1-x}$ cùng với hai trục tọa độ tạo thành một hình chữ nhật có diện tích bằng $3$ thì giá trị của $m$ bằng bao nhiêu?
    \shortans{$1$ hay $-\dfrac{3}{2}$}
    \loigiai{
        Từ yêu cầu đề bài, suy ra $|-m| \cdot |2m+1|=3 \Leftrightarrow \hoac{&m=1\\&m=-\dfrac{3}{2}.}$
    }
\end{ex}
\begin{ex}%[2D1K4-2]%[Thầy Hải Toán]%Câu 2.
    Đường tiệm cận đứng và đường tiệm cận ngang của đồ thị hàm số $y=\dfrac{mx+1}{2m+1-x}$ cùng với hai trục tọa độ tạo thành một hình chữ nhật có diện tích bằng $3$. Tính giá trị của $m$.
    \shortans{$m=1$; $m=-\dfrac{3}{2}$}
    \loigiai{
        Ta có $\lim\limits_{x\to+\infty}\dfrac{mx+1}{2m+1-x}=-m$; $\lim\limits_{x\to(2m+1)^+}\dfrac{mx+1}{2m+1-x} =\lim\limits_{x\to(2m+1)^+}\dfrac{m(2m+1)+1}{2m+1-x} =\lim\limits_{x\to(2m+1)^+}\dfrac{2m^2+m+1}{2m+1-x}$
        $\lim\limits_{x\to(2m+1)^+}\left(2m^2+m+1\right)=2m^2+m+1>0$; $\lim\limits_{x\to(2m+1)^+}(2m+1-x)=0$ và $2m+1-x<0\forall x>2m+1$ \\
        $ \Rightarrow\lim\limits_{x\to(2m+1)^+}\dfrac{mx+1}{2m+1-x}=-\infty $.\\
        Vậy đồ thị hàm số có hai đường tiệm cận $x=2m+1$ và $y=-m$.\\
        Hai đường tiệm cận tạo với hai trục tọa độ một hình chữ nhật có diện tích bằng $3$ suy ra $|2m+1|\cdot|m|=3\Leftrightarrow\hoac{&2m^2+m=3\\&2m^2+m=-3(PTVN)}\Leftrightarrow 2m^2+m-3=0\Leftrightarrow\hoac{&m=1\\&m=\dfrac{-3}{2}}$.}
\end{ex}
\begin{ex}%[KSCL L1, THPT Nhã Nam - Bắc Giang, 2019]%[Phạm An Bình, 12EX3]%[2D1K4-2]%
    Biết rằng đồ thị của hàm số $y=\dfrac{(n-3)x+n-2017}{x+m+3}$ ($m$, $n$ là tham số) nhận trục hoành làm tiệm cận ngang và trục tung làm tiệm cận đứng. Tính tổng $m-2n$.
    \shortans{$-9$}
    \loigiai{
        $\bullet$ $\lim\limits_{x\to +\infty}y=\lim\limits_{x\to +\infty}\dfrac{n-3+\dfrac{n-2017}{x}}{1+\dfrac{m+3}{x}} =n-3$.\\
        Vì đồ thị nhận trục hoành làm tiệm cận ngang nên $n-3=0\Leftrightarrow n=3$.\\
        $\bullet$ Vì đồ thị hàm số nhận trục tung làm tiệm cận đứng nên $\heva{&n-2017\ne 0\\&m+3=0}\Leftrightarrow \heva{&n\ne 2017\\&m= -3.}$\\
        Vậy $m-2n=-9$.
    }
\end{ex}
\begin{ex}%[TT Nguyễn Đăng Đạo, Bắc Ninh, lần 3, đề 152, 2018]%[2D1K4-2]%[Nguyễn Vân Trường, 12EX-8]%
    Tìm $m$ để tiệm cận đứng của đồ thị hàm số $y = \dfrac{m^2x-4m}{2x-m^2}$ đi qua điểm $A(2;1)$.
    \shortans{$m=-2$}
    \loigiai{
        Để hàm số có tiệm cận đứng thì \\
        $\hoac{& m \ne 0 \\ & m^2\cdot \dfrac{m^2}{2} - 4m \ne 0} \Leftrightarrow \hoac{& m \ne 0 \\ & m(m^3-8) \ne 0} \Leftrightarrow \hoac{& m \ne 0 \\ & m \ne 2}.$\\
        Khi đó tiệm cận đứng của hàm số là $x = \dfrac{m^2}{2}.$ Theo giả thiết ta có $ \dfrac{m^2}{2} = 2 \Leftrightarrow \hoac{& m =2 \text{ (loại)} \\ & m=-2 \text{ (thỏa mãn).}}$ Vậy $m=-2$.
    }
\end{ex}
\begin{ex}%[TT, Chuyên Lê Quý Đôn, Lai Châu, 2018]%[2D1K4-2]%[Nguyễn Tiến Thùy, 12EX-8]%
    Tìm $m$ để đồ thị hàm số $y=\dfrac{(m+1)x-5m}{2x-m}$ có tiệm cận ngang là đường thẳng $y=1$.
    \shortans{$m=1$}
    \loigiai{
        Ta có $\lim\limits_{x\rightarrow \pm\infty}f(x)=\lim\limits_{x\rightarrow \pm\infty}\dfrac{(m+1)x-5m}{2x-m}=\dfrac{m+1}{2}$, suy ra $y=\dfrac{m+1}{2}$ là tiệm cận ngang.\\
        Theo bài ra ta có $y=\dfrac{m+1}{2}=1\Leftrightarrow m=1$.
    }
\end{ex}
\begin{ex}%[2D1K4-1]
    Tìm tất cả các đường tiệm cận ngang của đồ thị hàm số $y=\dfrac{\sqrt{4x^2-x+1}}{2x+1}$.
    \shortans{$y=1$ và $y=-1$}
    \loigiai{
        Điều kiện xác định $x \ne \dfrac{-1}{2}$.\\
        Ta có $\lim\limits_{x \to +\infty} \dfrac{\sqrt{4x^2-x+1}}{2x+1}=\lim\limits_{x \to +\infty} \dfrac{|2x|\sqrt{1-\dfrac{1}{4x}+\dfrac{1}{4x^2}}}{2x\left(1+\dfrac{1}{2x}\right)}=1$.\\
        $\lim\limits_{x \to -\infty} \dfrac{\sqrt{4x^2-x+1}}{2x+1}=\lim\limits_{x \to -\infty} \dfrac{|2x|\sqrt{1-\dfrac{1}{4x}+\dfrac{1}{4x^2}}}{2x\left(1+\dfrac{1}{2x}\right)}=-1$.\\
        Tiệm cận ngang của đồ thị hàm số là $y= \pm 1$.
    }
\end{ex}
\begin{ex}%[2D1K4-1]
    Đồ thị hàm số $y=\dfrac{1-\sqrt{x^2+x+1}}{x^3+1}$ có tất cả bao nhiêu tiệm cận đứng và ngang?
    \shortans{$1$}
    \loigiai{
        Tập xác định $\mathscr{D}=\mathbb{R} \setminus \{-1\}$.
        \begin{itemize}
            \item
            {\allowdisplaybreaks
                \begin{eqnarray*}
                    \lim\limits_{x\to -1} \dfrac{1-\sqrt{x^2+x+1}}{x^3+1}&=&\lim\limits_{x\to -1} \dfrac{-x(x+1)}{(x+1)\left(x^2-x+1\right)\left(1+\sqrt{x^2+x+1}\right)}\\
                    &=&\lim\limits_{x\to -1} \dfrac{-x}{\left(x^2-x+1\right)\left(1+\sqrt{x^2+x+1}\right)}\\
                    &=&\dfrac{1}{6}.
            \end{eqnarray*} }
            \item $\lim\limits_{x\to +\infty}\dfrac{1-\sqrt{x^2+x+1}}{x^3+1}=0$.
        \end{itemize}
        Đồ thị hàm số không có tiệm cận đứng, tiệm cận ngang là $y=0$.
    }
\end{ex}
\begin{ex}%[2D1K4-1]
    Đồ thị hàm số $y=\dfrac{|x|}{\sqrt{x^2-1}}$ có tất cả bao nhiêu tiệm cận đứng và ngang?
    \shortans{$3$}
    \loigiai{
        Tập xác định $\mathscr{D}=(-\infty;-1) \cup (1;+\infty)$.
        \begin{itemize}
            \item $\lim\limits_{x\to-1^-}\dfrac{|x|}{\sqrt{x^2-1}}=+\infty$.
            \item $\lim\limits_{x\to 1^+}\dfrac{|x|}{\sqrt{x^2-1}}=+\infty$.
            \item $\lim\limits_{x\to +\infty}\dfrac{|x|}{\sqrt{x^2-1}}=1$.
            \item $\lim\limits_{x\to -\infty}\dfrac{|x|}{\sqrt{x^2-1}}=1$.
        \end{itemize}
        Đồ thị hàm số có $2$ tiệm cận đứng là $x=\pm 1$, tiệm cận ngang là $y=1$.
    }
\end{ex}
\begin{ex}%[2D1K4-1]
    Đồ thị hàm số $y=\dfrac{x}{\sqrt{x^2+1}}$ có tất cả bao nhiêu tiệm cận đứng và ngang?
    \shortans{$2$}
    \loigiai{
        Tập xác định $\mathscr{D}=\mathbb{R}$.
        \begin{itemize}
            \item $\lim\limits_{x\to +\infty}\dfrac{x}{\sqrt{x^2+1}}=1$.
            \item $\lim\limits_{x\to -\infty}\dfrac{x}{\sqrt{x^2+1}}=-1$.
        \end{itemize}
        Đồ thị hàm số không có tiệm cận đứng, tiệm cận ngang là $y=\pm 1$.
    }
\end{ex}
\begin{ex}%[2D1K4-1]
    Đồ thị hàm số $y=\dfrac{\sqrt{x^2-4}}{x^2-5x+6}$ có tất cả bao nhiêu tiệm cận đứng và ngang?
    \shortans{$3$}
    \loigiai{
        Tập xác định $\mathscr{D}=(-\infty;-2] \cup (2;+\infty) \setminus \{3\}$.
        \begin{itemize}
            \item $\lim\limits_{x\to 2^+}\dfrac{\sqrt{x^2-4}}{x^2-5x+6}=-\infty$.
            \item $\lim\limits_{x\to -2^-}\dfrac{\sqrt{x^2-4}}{x^2-5x+6}=-\infty$.
            \item $\lim\limits_{x\to +\infty}\dfrac{\sqrt{x^2-4}}{x^2-5x+6}=0$.
            \item $\lim\limits_{x\to -\infty}\dfrac{\sqrt{x^2-4}}{x^2-5x+6}=0$.
        \end{itemize}
        Đồ thị hàm số có $2$ tiệm cận đứng là $x=\pm 2$, tiệm cận ngang là $y=0$.
    }
\end{ex}
\begin{ex}
    Nồng độ thuốc trong máu $C(t)$ sau $t$ giờ khi uống một liều thuốc có thể được mô tả bởi hàm $C(t) = \dfrac{3}{1 + 2t}$. Tìm đường tiệm cận của nồng độ thuốc khi thời gian tăng lên rất lớn.
    \shortans{$0$}
\end{ex}
\begin{ex}
    Tốc độ (km/h) của một chiếc xe hơi tăng theo thời gian được mô tả bởi hàm $ v(t) = \dfrac{120t}{3+ t}$. Tìm đường tiệm cận của tốc độ khi thời gian tăng lên rất lớn.
    \shortans{$120$}
\end{ex}
\begin{ex}%[TeX hóa SGK CTST 12]%[Phạm Phương]%[2D1V4-4]
    Nồng độ oxygen trong hồ theo thời gian $t$ cho bởi công thức $y(t)=5-\dfrac{15t}{9t^{2}+1}$, với $y$ được tính theo mg/l và $t$ được tính theo giờ, $t \geq 0$. Tìm các đường tiệm cận của đồ thị hàm số $y(t)$. Từ đó, có nhận xét gì về nồng độ oxygen trong hồ khi thời gian $t$ trở nên rất lớn?
    \shortans{$y=5$, nồng độ tiến về $5$ mg/l}
    \loigiai{
        Hàm số $y(t)=5-\dfrac{15t}{9t^{2}+1}$ có tập xác định $\mathscr{D}=\mathbb{R}$.\\
        Ta có $\lim\limits_{x \rightarrow+\infty} \left(5-\dfrac{15t}{9t^{2}+1}\right)=5$.\\
        Vậy đồ thị hàm số có tiệm cận ngang là đường thẳng $y=5$.\\
        Khi thời gian $t$ trở nên rất lớn thì nồng độ oxygen trong hồ sẽ tiến dần về $5$ mg/l.
    }
\end{ex}
\begin{ex}
    Mô hình phát triển số lượng lợi khuẩn $P(t)$ theo thời gian có thể được mô tả bởi hàm $P(t) = \dfrac{100}{1 + 5e^{-2t}}$. Tính số lượng lợi khuẩn khi thời gian tăng lên rất lớn.\\
    \shortans{$100$}
\end{ex}
\begin{ex}
    Đáp ứng xung của một hệ thống điện tử the thời gian $t$ được mô tả bởi hàm \( h(t) = 120 e^{-\sqrt{3}t} \sin(2 t + \pi) \). Tìm và nêu ý nghĩa của đường tiệm cận của đáp ứng xung khi thời gian tăng.
    \shortans{$0$}
\end{ex}
\begin{ex}
    Điện áp của pin sạc theo thời gian được mô tả bởi hàm \( V(t) = 220 \left(1 - e^{-\dfrac{t}{\tau}}\right) \), trong đó \( \tau \) là hằng số thời gian. Tìm và nêu ý nghĩa của đường tiệm cận của điện áp khi thời gian tăng.
    \shortans{$220$}
\end{ex}
\begin{ex}%[0D1K1-4]
    Số lượng sản phẩm bán được của một công ty trong $x$ (tháng) được tính theo công thức $S(x)=200\left(5-\dfrac{9}{2+x}\right)$, trong đó $x\ge 1$ \emph{(Nguồn: R.Larson and B.Edwards, Calculus 10e, Cengage 2014).}
    \begin{enumerate}[a)]
        \item Xem $y=S(x)$ là một hàm số xác định trên nửa khoảng $[1;+\infty)$, hãy tìm tiệm cận ngang của đồ thị hàm số đó.
        \item Nêu nhận xét về số lượng sản phẩm bán được của công ty trong $x$ (tháng) khi $x$ đủ lớn.
    \end{enumerate}
    \shortans{$y=1$, sản phẩm gần $1\,000$}
    \loigiai{
        \begin{enumerate}[a)]
            \item Ta có $\lim\limits_{x\to +\infty}\left[200\left(5+\dfrac{9}{2-x}\right)\right]=200\cdot 5=1000$.\\
            Vậy $y=1\,000$ là tiệm cận ngang của đồ thị hàm số $y=S(x)$.
            \item Từ phần trên ta có thể rút ra nhận xét: khi số tháng đủ lớn thì công ty có thể bán được số sản phẩm gần bằng $1\,000$.
        \end{enumerate}
    }
\end{ex}
\begin{ex}
    Công ty cung cấp dịch vụ internet tính $75\$$ phí lắp đặt thiết bị ban đầu và phí sử dụng internet $40\$$ mỗi tháng
    \begin{listEX}
        \item Lập hàm số thể hiện chi phí sử dụng trung bình mỗi tháng sau $x$ tháng sử dụng
        \item Chi phí sử dụng trung bình thay đổi thế nào khi số tháng sử dụng tăng lên rất nhiều.
    \end{listEX}
    \shortans{$y=\dfrac{40x+75}{40}$, chi phí tiến về $40\$$}
\end{ex}
\begin{ex}
    Nhà trường dự định tổ chức tiệc liên hoan chào mừng lớp 12, tiền thuê hội trường là $1$ tỷ. Cứ mỗi người tham gia sẽ tính thêm phí phục vụ là $2$ triệu mỗi người. Gọi $x$ là số người tham gia bữa tiệc
    \begin{listEX}
        \item Lập hàm số thể hiện tổng chi phí của bữa tiệc
        \item Lập hàm số thể hiện chi phí trung bình của mỗi người bỏ ra cho bữa tiệc
        \item Chi phí trung bình của mỗi người thay đổi thế nào khi số người tham gia tăng lên rất lớn.
    \end{listEX}
    \shortans{$y=\dfrac{0,02x+1}{x}$, tiến về $2$ triệu}
\end{ex}
\begin{ex}
    Số lượng vi khuẩn trong một môi trường dinh dưỡng có thể được mô tả bởi hàm:
    \[
    N(t) = \dfrac{N_0}{1 - \dfrac{t}{T}}
    \]
    trong đó \( N(t) \) là số lượng vi khuẩn tại thời gian \( t \), \( N_0 \) là số lượng vi khuẩn ban đầu, và \( T \) là thời gian mà môi trường dinh dưỡng không còn đủ để hỗ trợ sự tăng trưởng của vi khuẩn. Xác định tiệm cận đứng của hàm số này và nêu ý nghĩa của nó.
    \shortans{$t=T$, khi $t$ tiến về $T$ thì số lượng vi khuẩn tăng lên vô hạn}
    \loigiai{
        Để tìm tiệm cận đứng, ta xem xét các giá trị của \( t \) làm cho mẫu số của phương trình bằng 0:
        \[
        1 - \frac{t}{T} = 0 \Rightarrow t = T
        \]
        Vậy đường thẳng \( t = T \) là tiệm cận đứng của đồ thị hàm số.
        \textbf{Ý nghĩa:} Từ đó ta suy ra khi thời gian \( t \) càng sát với \( T \), số lượng vi khuẩn \( N(t) \) sẽ tăng lên vô hạn. Điều này có nghĩa là khi thời gian tiếp cận \( T \) thì số lượng vi khuẩn sẽ tăng lên nhanh chóng đến mức vô hạn.}
\end{ex}
\begin{ex}
    Trong vật lý, vận tốc tối đa \(V\) của một vật rơi qua một chất lỏng được mô tả bằng phương trình:
    \[
    V(t) = \frac{mg}{b} \left(1 - e^{-\dfrac{bt}{m}}\right)
    \]
    trong đó \(m\) là khối lượng của vật, \(g\) là gia tốc trọng trường, \(b\) là hệ số ma sát, và \(t\) là thời gian. Xác định tiệm cận đứng của hàm số này và nêu ý nghĩa của nó.
    \shortans{Không có TCĐ}
    \loigiai{Để tìm tiệm cận đứng, ta xem xét các giá trị của \(t\) làm cho mẫu số của phương trình bằng 0. Tuy nhiên, trong trường hợp này, hàm số không có tiệm cận đứng vì biểu thức mũ đảm bảo hàm số được xác định cho tất cả các số thực.
        \textbf{Ý nghĩa:} Điều này ngụ ý rằng không có giới hạn về thời gian để vật đạt đến vận tốc tối đa. Khi thời gian tăng lên, vận tốc của vật sẽ tiệm cận đến vận tốc tối đa, nhưng vật không bao giờ thực sự đạt được nó.}
\end{ex}
\begin{ex}
    Trong sinh học, sự tăng trưởng của dân số \(P\) theo thời gian \(t\) có thể được mô hình bằng hàm số:
    \[
    P(t) = \frac{P_0}{1 - kP_0t}
    \]
    trong đó \(P_0\) là kích thước dân số ban đầu và \(k\) là hằng số tốc độ tăng trưởng. Xác định tiệm cận đứng của hàm số này và nêu ý nghĩa của nó.
    \shortans{$t = \frac{1}{kP_0}$}
    \loigiai{Để tìm tiệm cận đứng, ta xem xét các giá trị của \(t\) làm cho mẫu số của phương trình bằng 0:
        \[
        1 - kP_0t = 0 \Rightarrow t = \frac{1}{kP_0}
        \]
        Vậy tiệm cận đứng là \(t = \frac{1}{kP_0}\).
        \textbf{Ý nghĩa:} Điều này ngụ ý rằng hàm số tăng trưởng dân số có một tiệm cận đứng tại thời điểm \(t\) bằng nghịch đảo của tích của hằng số tốc độ tăng trưởng \(k\) và kích thước dân số ban đầu \(P_0\). Điều này chỉ ra một giới hạn cho tốc độ tăng trưởng dân số theo thời gian.}
\end{ex}
\begin{ex}
    Trong khoa học máy tính, độ phức tạp thời gian \(T(n)\) của một thuật toán với kích thước đầu vào \(n\) có thể được biểu diễn bằng hàm số:
    \[
    T(n) = \frac{an^2 + bn + c}{n}
    \]
    trong đó \(a\), \(b\), và \(c\) là các hằng số. Xác định tiệm cận đứng của hàm số này và nêu ý nghĩa của nó.
    \shortans{$T=0$}
    \loigiai{Để tìm tiệm cận đứng, ta xem xét các giá trị của \(n\) làm cho mẫu số của phương trình bằng 0:
        \[
        n = 0
        \]
        Vậy tiệm cận đứng là \(n = 0\).
        \textbf{Ý nghĩa:} Điều này ngụ ý rằng hàm số độ phức tạp thời gian không có tiệm cận đứng. Trong phân tích tính toán, một tiệm cận đứng tại \(n = 0\) sẽ ngụ ý rằng thuật toán có độ phức tạp thời gian vô hạn cho các đầu vào có kích thước bằng 0, điều này không có ý nghĩa trong hầu hết các trường hợp.}
\end{ex}
\Closesolutionfile{ans}
\begin{dang}{Đường tiệm cận liên quan tham số $m$}
\end{dang}
\begin{vd}
    Tìm $m$ để đồ thị hàm số
    \begin{listEX}[2]
        \item $y=\dfrac{x-2}{x^2-mx+1}$ có hai đường tiệm cận đứng.
        \item $y=\dfrac{x-1}{x^2-mx+1}$ có đúng ba đường tiệm cận.
        \item $y=\dfrac{\sqrt{x-3}}{x^2+x-m}$ có đúng hai đường tiệm cận.
        \item $y=\dfrac{\sqrt{1-x}}{x^2+4x+m}$ có ba đường tiệm cận.
        %	\item* $y=\dfrac{x}{x^2-2(m+1)x+m^2}$ có đúng hai đường tiệm cận.
    \end{listEX}
    \loigiai{}
\end{vd}
\BTTN
\Opensolutionfile{ans}[ans/2D1-4-DANG-2]
\begin{ex}%[Phạm Văn Long]%[Latex-HK2-TT-2020-2021]%[2D1K4-2]%
    Tìm $m$ để đồ thị hàm số $y=\dfrac{2x^2-3x+4}{x^2+mx+1}$ có duy nhất một đường tiệm cận?
    \choice
    {\True $m\in (-2;2)$}
    {$m\in [-2;2]$}
    {$m\in \{-2;2\}$}
    {$m\in (2;+\infty)$}
    \loigiai{
        Ta thấy đồ thị hàm số đã cho luôn có một tiệm cận ngang là đường $y=2$.\\
        Do đó, để đồ thị hàm số đã cho có duy nhất một đường tiệm cận thì đồ thị hàm số đã cho không có tiệm cận đứng.\\
        $\Rightarrow$ Phương trình $x^2+mx+1=0$ vô nghiệm $\Leftrightarrow \Delta <0 \Leftrightarrow m^2-4<0\Leftrightarrow m\in (-2;2)$.
    }
\end{ex}
\begin{ex}%[2D1K4-2]%[Đề GHK1, THPT Trần Nhân Tông, Hà Nội 2018]%[WTT2D1-128]%
    Tìm giá trị thực của tham số $m$ để đồ thị hàm số $y=\dfrac{x-4}{m-x^2}$ có đường tiệm cận đứng.
    \choice
    {$m\ge0;\,m\ne16$}
    {\True $m\ge0$}
    {$m>0$}
    {$m>0;\,m\ne16$}
    \loigiai{
        Điều kiện xác định: $m-x^2\neq0$.\\
        Để đồ thị hàm số có đường tiệm cận đứng thì phương trình $m-x^2=0$ có nghiệm, tức là $m\ge0$.\\
        Với $m=16$ thì $y=\dfrac{-1}{4+x}$ có một tiệm cận đứng là $x=-4$. Vậy giá trị $m$ cần tìm là $m\ge0$.
    }
\end{ex}
\begin{ex}%[2D1K4-2]%
    Có bao nhiêu giá trị của tham số $m$ thoả mãn đồ thị hàm số $y=\dfrac{x+3}{x^2-x-m}$ có đúng hai đường tiệm cận?
    \choice
    {$1$}
    {$4$}
    {\True $2$}
    {$3$}
    \loigiai{
        Đồ thị hàm số có đúng hai đường tiệm cận khi phương trình $x^2-x-m=0$ có nghiệm kép hoặc có hai nghiệm phân biệt với một nghiệm bằng $-3$. Khi đó
        \[\hoac{&\Delta=0\\&\heva{&\Delta>0\\&g(-3)=0}}\Leftrightarrow\hoac{&4m+1=0\\&\heva{&4m+1>0\\&m=12}}\hoac{&m=-\dfrac{1}{4}\\&m=12.}\]
        Vậy có hai giá trị của m.}
\end{ex}
\begin{ex}%[2D1K4-2]%[Đề kiểm tra giữa học kì I, 2017 - 2018 trường THPT Chu Văn An, Hà Nội]%[WTT2D1-156]%
    Tìm tất cả các giá trị thưc của tham số $m$ để đồ thị hàm số $y=\dfrac{x^2+m}{x^2-3x+2}$ có đúng hai tiệm cận.
    \choice
    {$m=-1$}
    {$m\in\left\{1;4\right\}$}
    {\True $m\in\left\{-1;-4\right\}$}
    {$m=4$}
    \loigiai{
        Vì $\lim\limits_{x\to\pm\infty}\dfrac{x^2+m}{x^2-3x+2}=1,\,\forall m$ nên đồ thị hàm số luôn có một tiêm cận ngang là $y=1$.\\
        Để đồ thị hàm số có đúng hai tiệm cận thì đồ thị hàm số có thêm một tiệm cận đứng là $x=1$ hoặc là $x=2$.
        \begin{itemize}
            \item Đồ thị hàm số có một tiệm cận đứng $x=1$, suy ra pt $x^2+m=0$ và phương trình $x^2-3x+2=0$ có nghiệm chung là $x=1\Rightarrow m=-1$.
            \item Đồ thị hàm số có một tiệm cận đứng $x=2$, suy ra pt $x^2+m=0$ và phương trình $x^2-3x+2=0$ có nghiệm chung là $x=2\Rightarrow m=-4$.
        \end{itemize}
        Vậy $m\in\left\{-1;4\right\}$ thỏa yêu cầu bài toán.
    }
\end{ex}
\begin{ex}%[Thi thử, THPT Lục Ngạn - Bắc Giang, 2019]%[Trần Như Ngọc, 12EX3-2019]%[2D1K4-2]%
    Có bao nhiêu giá trị nguyên dương của tham số $m$ để đồ thị hàm số
    $y=\dfrac{\sqrt{9-x}}{x^2-2(m+1)x+m^2+2m}$
    có đúng hai đường tiệm cận.
    \choice
    {\True $2$}
    {$1$}
    {$4$}
    {$3$}
    \loigiai{
        Ta có $ x^2-2(m+1)x+m^2+2m = 0 \Leftrightarrow \hoac{& x=m \\ & x=m+2}$
        $( \Delta ' = 1 )$. \\
        Hàm số xác định khi $ \heva{& x \le 9 \\ & x \ne m \\ & x \ne m+2.} $\\
        Ta có $\lim \limits_{x\to -\infty}y = 0$ nên đồ thị hàm số có một tiệm cận ngang là $ y = 0 $.\\
        Đồ thị hàm số có đúng hai tiệm cận khi và chỉ khi nó có đúng một tiệm cận đứng \\
        $\Leftrightarrow$ phương trình trên có một nghiệm nhỏ hơn hoặc bằng $ 9 $.\\
        $\Leftrightarrow m \le 9 < m+2 \Leftrightarrow 7 < m \le 9 $.\\
        Vậy có $ 2 $ giá trị $ m $ nguyên dương thỏa mãn điều kiện bài toán.
    }
\end{ex}
\begin{ex}%[Thi học kỳ I, Trường THPT Chuyên Lê Quý Đôn - Khánh Hòa, 2021]%[Lê Hồng Phi, 12EX5]%[2D1K4-2]%
    Cho hàm số $y=\dfrac{2x-3}{\sqrt{x^2+2(m-2)x+m^2}}$ với $m$ là tham số thực và $m>1$. Hỏi đồ thị hàm số có bao nhiêu đường tiệm cận (tiệm cận ngang và tiệm cận đứng)?
    \choice
    {$1$}
    {\True $2$}
    {$3$}
    {$4$}
    \loigiai
    {Phương trình $x^2+2(m-2)x+m^2=0$ có $\Delta'=(m-2)^2-m^2=-2(2m-2)=-4(m-1)<0,\ \forall m>1$ nên vô nghiệm.\\
        Do đó tập xác định của hàm số là $\mathscr{D}=\mathbb{R}$.\\
        Như thế đồ thị hàm số không có đường tiệm cận đứng.\\
        Ta tính được
        \begin{itemize}
            \item $\lim\limits_{x\to +\infty}y=\lim\limits_{x\to +\infty}\dfrac{2-\dfrac{3}{x}}{\sqrt{1+\dfrac{2(m-2)}{x}+\dfrac{m^2}{x}}}=2$ nên $y=2$ là đường tiệm cận ngang.
            \item $\lim\limits_{x\to -\infty}y=\lim\limits_{x\to -\infty}\dfrac{2-\dfrac{3}{x}}{-\sqrt{1+\dfrac{2(m-2)}{x}+\dfrac{m^2}{x}}}=-2$ nên $y=-2$ là đường tiệm cận ngang.
        \end{itemize}
        Vậy đồ thị hàm số đã cho có $2$ đường tiệm cận.
    }
\end{ex}
\begin{ex}%[Đề Khảo sát lần 1 THPT Quang Hà - Vĩnh Phúc, 2021]%[Trần Nhân Kiệt, 12EX4-2021]%[2D1K4-2]%
    Có bao nhiêu giá trị nguyên của tham số $m$ để đồ thị tham số $y=\dfrac{1+\sqrt{x+1}}{\sqrt{x^2-(1-m)x+2m}}$ có hai tiệm cận đứng?
    \choice
    {$2$}
    {\True $3$}
    {$1$}
    {$0$}
    \loigiai{
        Điều kiện $\heva{& x\ge -1 \\ & x^2-(1-m)x+2m>0.}$\\
        Đồ thị hàm số có hai tiệm cận đứng khi và chỉ khi phương trình $x^2-(1-m)x+2m=0$ có hai nghiệm phân biệt lớn hơn hoặc bằng $-1$.\\
        Ta có $x^2-(1-m)x+2m=0\Leftrightarrow x^2-x+m(x+2)=0\Leftrightarrow m=\dfrac{-x^2+x}{x+2}$.\\
        Đặt $f(x)=\dfrac{-x^2+x}{x+2}$, $x\ge -1$.\\
        Ta có $f'(x)=\dfrac{-x^2-4x+2}{(x+2)^2}$, suy ra $f'(x)=0\Leftrightarrow -x^2-4x+2=0\Leftrightarrow x=-2\pm \sqrt{6}$.
        \begin{center}
            \begin{tikzpicture}[>=stealth]
                \tkzTabInit[nocadre=false,lgt=1.2,espcl=3,deltacl=0.5]
                {$x$/.7 ,$f'(x)$/.7,$f(x)$/2}
                {$-1$ , $-2+\sqrt{6}$ , $+\infty$}
                \tkzTabLine{ , - , $0$ , + , }
                \tkzTabVar{-/$-2$ , +/$5-2\sqrt{6}$ , -/$-\infty$}
            \end{tikzpicture}
        \end{center}
        Từ bảng biến thiên suy ra $m\in [-2;5-2\sqrt{6})$.\\
        Vì $m$ nguyên nên $m\in \{-2;-1;0\}$.\\
        Vậy có $3$ giá trị nguyên của $m$ thỏa mãn bài.
    }
\end{ex}
\BTTL
\begin{ex}%[2D1K4-2]%
    Cho hàm số $y=\dfrac{2x^2-3x+m}{x-m}$ có đồ thị $(C)$. Với tất cả các giá trị thực nào của tham số $m$ thì đồ thị $(C)$ không có tiệm cận đứng?
    \shortans{$m=0$ hoặc $m=1$}
    %	\choice
    %	{$m=2$}
    %	{$m=0$}
    %	{$m=1$}
    %	{\True $m=0$ hoặc $m=1$}
    \loigiai{
        Đồ thị không có tiệm cận đứng khi $x=m$ là nghiệm của phương trình $2x^2-3x+m=0$, suy ra $2m^2-3m+m=0 \Leftrightarrow \hoac{&m=0\\&m=1}$.
    }
\end{ex}
\begin{ex}%[2D1K4-2]%
    Với tất cả các giá trị thực nào của tham số $m$ thì đồ thị hàm số $y=\dfrac{x^2+x-2}{x^2+x+m}$ có ba đường tiệm cận?
    \shortans{$m<\dfrac{1}{4}$ và $m\ne -2$}
    %	\choice
    %	{$m>\dfrac{1}{4}$ và $m\ne 2$}
    %	{$m>\dfrac{1}{4}$}
    %	{$m<\dfrac{1}{4}$}
    %	{\True $m<\dfrac{1}{4}$ và $m\ne -2$}
    \loigiai{
        Đồ thị hàm số chỉ có $1$ tiệm cận ngang là $y=1$.\\
        Ta có $x^2+x-2 \Leftrightarrow \hoac{&x=1\\&x=-2.}$\\
        Đồ thị hàm số có ba đường tiệm cận khi và chỉ khi có $2$ tiệm cận đứng. Điều này tương đương với phương trình $x^2+x+m=0$ có $2$ nghiệm phân biệt khác $1$ và $-2$, nghĩa là\\
        $\heva{&1-4m>0\\&1^2+1+m \ne 0\\& (-2)^2-2+m\ne 0} \Leftrightarrow \heva{&m<\dfrac{1}{4}\\&m\ne -2.}$
    }
\end{ex}
\begin{ex}%[đề thi thử THPT Quốc gia, đề số 3, nguyễn hoàng thanh]%[2D1K4-2]%
    Tìm số giá trị nguyên thuộc đoạn $ [-2025;2025] $ của tham số $ m $ để đồ thị hàm số $ y=\dfrac{\sqrt{x-3}}{x^2+x-m} $ có đúng hai đường tiệm cận.
    \shortans{$2014$}
    %	\choice
    %	{$ 2007 $}
    %	{$ 2010 $}
    %	{$ 2009 $}
    %	{\True $ 2008 $}
    \loigiai{
        Điều kiện xác định của hàm số $ \heva{& x\ge 3\\& x^2+x-m\ne 0.} $\\
        Vì $ \lim\limits_{x\to +\infty}\dfrac{\sqrt{x-3}}{x^2+x-m}=\lim\limits_{x\to +\infty}\dfrac{\sqrt{\frac{1}{x}-\frac{3}{x^2 }}}{1+\frac{1}{x}-\frac{m}{x^2}}=0 $, suy ra $ y=0 $ là tiệm cận ngang.\\
        Để đồ thị hàm số có đúng hai tiệm cận thì đồ thị hàm số chỉ có thêm một tiệm cận đứng, tương đương $ f(x)=x^2+x-m $ có đúng một nghiệm lớn hơn $ 3 $. Xét các trường hợp xảy ra như sau
        \begin{enumerate}
            \item $ f(x)=0 $ có nghiệm kép $ x_{1}=x_2=-\dfrac{1}{2}<3 $ (không thỏa mãn).
            \item $ f(x)=0 $ có hai nghiệm thỏa $ x_1<3\le x_2\Leftrightarrow a\cdot f(3)\le 0\Leftrightarrow 12-m\le 0\Leftrightarrow m\ge 12 $.
        \end{enumerate}
        Kết hợp với yêu cầu bài toán ta suy ra $ \heva{&m\in \mathbb{Z}\\ &m\in[12;2025]} $, suy ra có $ 2025-12+1=2014 $ giá trị nguyên của $ m $ thỏa mãn bài toán.
    }
\end{ex}
\begin{ex}%[Đề tham khảo THPT Quốc gia 2021 - Đề 5]%[Đoàn Minh Tân]%[2D1K4-2]%
    Tìm tất cả giá trị thực của tham số $m$ để đồ thị hàm số $y=\dfrac{3x+2018}{\sqrt{mx^2+5x+6}}$ có hai đường tiệm cận ngang.
    \shortans{$m>0$}
    %	\choice
    %	{$m\in \varnothing$}
    %	{$m<0$}
    %	{$m=0$}
    %	{\True $m>0$}
    \loigiai{
        Ta có $\displaystyle \lim \limits_{x\to +\infty} y=\displaystyle\lim\limits_{x\to +\infty}\dfrac{3x+2018}{\sqrt{mx^2+5x+6}}=\displaystyle\lim\limits_{x\to +\infty}\dfrac{3+\dfrac{2018}{x}}{\sqrt{m+\dfrac{5}{x}+\dfrac{6}{x^2}}}=\dfrac{3}{\sqrt{m}}$ tồn tại khi $m>0$.\\
        $\displaystyle\lim\limits_{x\to -\infty}=\displaystyle\lim\limits_{x\to -\infty}\dfrac{3x+2018}{\sqrt{mx^2+5x+6}}=\displaystyle\lim\limits_{x\to -\infty}\dfrac{3+\dfrac{2018}{x}}{-\sqrt{m+\dfrac{5}{x}+\dfrac{6}{x^2}}}=-\dfrac{3}{\sqrt{m}}$ tồn tại khi $m>0$.\\
        Hiên nhiên $\displaystyle\lim\limits_{x\to +\infty}y\ne \displaystyle \lim \limits_{x\to -\infty}y$.\\
        Vậy đồ thị hàm số đã cho có hai tiệm cận ngang khi và chỉ khi $m>0$.
    }
\end{ex}
\begin{ex}%[2D1K4-2]%
    Có bao nhiêu giá trị nguyên của tham số thực $m$ thuộc đoạn $[-20; 10]$ để đồ thị hàm số $y=\dfrac{x+2}{\sqrt{x^2-4x+m}}$ có hai đường tiệm cận đứng?
    \shortans{$23$}
    %	\choice
    %	{$20$}
    %	{$21$}
    %	{$22$}
    %	{\True $23$}
    \loigiai{
        Đồ thị hàm số có hai đường tiệm cận đứng $\Leftrightarrow$ phương trình $x^2-4x+m=0$ có hai nghiệm phân biệt khác $-2$ \\
        $ \Leftrightarrow\heva{&2^2-m>0\\&(-2)^2-4\cdot (-2)+m\neq 0}\Leftrightarrow\heva{&m<4\\&m\neq-12.} $ \\
        Do $m$ nguyên và $m\in[-20; 10]$ nên $m\in\left\{-20;-19;\ldots;-13;-11;\ldots; 2; 3\right\}$, gồm $23$ giá trị thỏa mãn.}
\end{ex}
\Closesolutionfile{ans}
\begin{dang}{Tìm các đường tiệm cận đồ thị hàm ẩn}
\end{dang}
\begin{vd}
    Cho hàm số $y=f(x)$ có bảng biến thiên như hình vẽ sau
    \begin{center}
        \begin{tikzpicture}[>=stealth]
            \tkzTabInit[nocadre=false,lgt=1,espcl=1.5,deltacl=0.5]{$x$/.7 ,$y'$/.7,$y$/2}
            {$-\infty$ , $-1$ , $2$ , $+\infty$}
            \tkzTabLine{ , + , $0$ , - , d , + , }
            \tkzTabVar{-/$1$ , +/$4$ , -/$-5$ , +/$+\infty$}
        \end{tikzpicture}
    \end{center}
    Tìm TCĐ, TCN của đồ thị hàm số
    \begin{listEX}[3]
        \item $y=\dfrac{2}{f(x)-3}$
        \item $y=\dfrac{-3}{f(x)+2}$
        \item $y=\dfrac{x-2}{f(x)+5}$
        \item $y=\dfrac{x+1}{f(x)-4}$
        \item $y=\dfrac{2}{f(x^2)+3}$
        \item $y=\dfrac{4f(x)-5}{3f(x)+1}$
    \end{listEX}
    \loigiai{}
\end{vd}
\begin{vd}\immini{Cho hàm bậc ba $y=f(x)$ có đồ thị như hình vẽ. Tìm số tiệm cận đứng của đồ thị hàm số
        \begin{listEX}[2]
            \item $y=\dfrac{\sqrt{x+3}}{(x-1)f(x)}$
            \item $g(x)=\dfrac{(x^2+4x+3)\sqrt{x^2+x}}{x\left[f^2(x)-2f(x)\right]}$ .
    \end{listEX}}{\begin{tikzpicture}[line cap=round,line join=round, >=stealth,font=\footnotesize]
            \begin{scope}[scale=.5]
                \def\a{-1} % Hệ số a phải khác 0
                \def\b{-13/2}
                \def\c{-12}
                \def\d{-9/2}
                \draw[->] (-5,0) -- (2,0)node[below]{$x$};
                \draw[->] (0,-3) -- (0,4) node[left] {$y$};
                \draw (0,0)node[below right]{$O$} (-3,0)node[below]{$-3$};
                \draw[dashed] (-1,0)node[below]{$-1$}|-(0,2)node[right]{$2$};
                \draw[samples=150,smooth,domain=-4:.-.2] plot(\x,{\a*(\x)^3+(\b)*(\x)^2+(\c)*\x+(\d)});
            \end{scope}
    \end{tikzpicture}}
    \loigiai{
        \begin{center}
            \begin{tikzpicture}[line cap=round,line join=round, >=stealth,font=\footnotesize,scale=1]
                \def\a{-1} % Hệ số a phải khác 0
                \def\b{-13/2}
                \def\c{-12}
                \def\d{-9/2}
                \draw[->] (-5,0) -- (2,0)node[below]{$x$};
                \draw[->] (0,-3) -- (0,4) node[left] {$y$};
                \draw (0,0)node[below right]{$O$} (-3,0)node[below]{$-3$} (-.3,0)node[above]{$a$};
                \draw[dashed] (-3.78,0)node[below]{$c$}|-(0,2)|-(-1.71,0)node[below]{$b$}|-(0,2) (-1,0)node[below]{$-1$}|-(0,2)node[right]{$2$};
                \draw[samples=150,smooth,domain=-4:.-.2] plot(\x,{\a*(\x)^3+(\b)*(\x)^2+(\c)*\x+(\d)});
            \end{tikzpicture}
        \end{center}
        $g(x)=\dfrac{(x^2+4x+3)\sqrt{x^2+x}}{x\left[f^2(x)-2f(x)\right]}=\dfrac{(x+1)(x+3)\sqrt{x(x+1)}}{x\left[f^2(x)-2f(x)\right]}$.\\
        Điều kiện của căn là $x\le -1; x\ge 0$.\\
        Dựa vào đồ thị ta có \[x\left[f^2(x)-2f(x)\right]=0 \Leftrightarrow \hoac{&x=0\\&f(x)=0\\& f(x)=2} \Leftrightarrow \hoac{&x=0\text{ (nhận)}\\&x=-3\text{ (nhận)};\ x=a \text{ (loại)} \\&x=-1\text{ (nhận)};\ x=b\text{ (nhận)};\ x=c\text{ (nhận)}}\]\\
        Số TCĐ lúc này chính là số nghiệm không bị rút gọn của mẫu, vậy có bốn TCĐ là $x=0; x=-3; x=b; x=c$.
    }
\end{vd}
\BTTN
\Opensolutionfile{ans}[ans/2D1-4-DANG-3]
\begin{ex}%[2D1K4-1]
    Cho hàm số $y=f(x)$ có bảng biến thiên như hình bên. Đồ thị hàm số $y=\dfrac{-5}{f(x)+4}$ có bao nhiêu tiệm cận đứng?
    \begin{center}
        \begin{tikzpicture}[scale=0.8]
            \tkzTabInit[nocadre=false,lgt=1.5,espcl=3,deltacl=0.6]
            {$x$ /0.6,$y’$ /0.6,$y$ /2}
            {$-\infty$ ,$1$, $2$, $+\infty$}
            \tkzTabLine{,+,d,-,d,+,}
            \tkzTabVar{-/$-4$,+/$3$,-/$-5$,+/$+\infty$}
        \end{tikzpicture}
    \end{center}
    \choice
    {$1$}
    {$3$}
    {\True $2$}
    {$4$}
    \loigiai{
        Dựa vào bảng biến thiên suy ra
        $f(x)+4=0 \Leftrightarrow f(x) =-4$, phương trình này có $2$ nghiệm phân biệt nên đồ thị hàm số $y=\dfrac{-5}{f(x)+4}$ có $2$ tiệm cận đứng.
    }
\end{ex}
\begin{ex}%[2D1K4-1]
    Cho hàm số $y=f(x)$ có bảng biến thiên như hình bên. Đồ thị hàm số $y=\dfrac{x+2}{2f(x)-1}$ có bao nhiêu tiệm cận đứng?
    \begin{center}
        \begin{tikzpicture}[scale=0.8]
            \tkzTabInit[nocadre=false,lgt=1.5,espcl=3,deltacl=0.6]
            {$x$ /0.6,$y’$ /0.6,$y$ /2}
            {$-\infty$ ,$-1$, $0$, $1$, $+\infty$}
            \tkzTabLine{,+,0,-,0,+,0,-,}
            \tkzTabVar{-/$-\infty$,+/$0$,-/$-\dfrac{5}{3}$,+/$0$,-/$-\infty$}
        \end{tikzpicture}
    \end{center}
    \choice
    {$1$}
    {$3$}
    {$2$}
    {\True $0$}
    \loigiai{
        Dựa vào bảng biến thiên suy ra
        $2f(x)-1=0 \Leftrightarrow f(x) =\dfrac{1}{2}$, phương trình này có $0$ nghiệm nên đồ thị hàm số $y=\dfrac{x+2}{2f(x)-1}$ không có tiệm cận đứng.
    }
\end{ex}
%69
\begin{ex}%[2D1K4-1]
    Cho hàm số $y=f(x)$ có bảng biến thiên như hình bên. Đồ thị hàm số $y=\dfrac{1}{2f(x)-3}$ có bao nhiêu tiệm cận đứng?
    \begin{center}
        \begin{tikzpicture}[scale=0.8]
            \tkzTabInit[nocadre=false,lgt=1.5,espcl=3,deltacl=0.6]
            {$x$ /0.6,$y’$ /0.6,$y$ /2}
            {$-\infty$ ,$0$, $1$, $+\infty$}
            \tkzTabLine{,+,0,-,0,+,}
            \tkzTabVar{-/$-\infty$,+/$5$,-/$-1$,+/$+\infty$}
        \end{tikzpicture}
    \end{center}
    \choice
    {$1$}
    {\True $3$}
    {$2$}
    {$0$}
    \loigiai{
        Dựa vào bảng biến thiên suy ra
        $2f(x)-3=0 \Leftrightarrow f(x) =-\dfrac{3}{2}$, phương trình này có $3$ nghiệm phân biệt nên đồ thị hàm số $y=\dfrac{1}{2f(x)-3}$ có ba tiệm cận đứng.
    }
\end{ex}
%70
%71
%72
\begin{ex}%[2D1K4-1]
    Cho hàm số $y=f(x)$ có bảng biến thiên như hình bên. Đồ thị hàm số $y=\dfrac{x}{f(x)-3}$ có bao nhiêu tiệm cận đứng?
    \begin{center}
        \begin{tikzpicture}[scale=0.8]
            \tkzTabInit[nocadre=false,lgt=1.5,espcl=3,deltacl=0.6]
            {$x$ /0.6,$y’$ /0.6,$y$ /2}
            {$-\infty$ ,$-1$, $0$, $1$, $+\infty$}
            \tkzTabLine{,-,0,+,0,-,0,+,}
            \tkzTabVar{+/$+\infty$,-/$0$,+/$3$,-/$0$,+/$+\infty$}
        \end{tikzpicture}
    \end{center}
    \choice
    {$1$}
    {\True $3$}
    {$2$}
    {$4$}
    \loigiai{
        Dựa vào bảng biến thiên suy ra
        $f(x)-3=0 \Leftrightarrow f(x) =3$, phương trình này có $2$ nghiệm phân biệt khác $0$ và một nghiệm bội chẵn $x=0$ nên đồ thị hàm số $y=\dfrac{x}{f(x)-3}$ có ba tiệm cận đứng.
    }
\end{ex}
\begin{ex}%[2D1K4-1]
    Cho hàm số $y=f(x)$ có bảng biến thiên như hình bên. Đồ thị hàm số $y=\dfrac{4}{f(x)+1}$ có tiệm cận ngang là đường thẳng
    \begin{center}
        \begin{tikzpicture}[scale=0.8]
            \tkzTabInit[nocadre=false,lgt=1.5,espcl=3,deltacl=0.6]
            {$x$ /0.6,$y’$ /0.6,$y$ /2}
            {$-\infty$ ,$-1$, $2$, $+\infty$}
            \tkzTabLine{,+,0,-,0,+,}
            \tkzTabVar{-/$1$,+/$4$,-/$-5$,+/$1$}
        \end{tikzpicture}
    \end{center}
    \choice
    {$y=1$}
    {$y=-5$}
    {\True $y=2$}
    {$y=4$}
    \loigiai{
        Dựa vào bảng biến thiên suy ra
        $\lim \limits_{x \to \pm \infty} f(x)=1 \Leftrightarrow \lim \limits_{x \to \pm \infty} \dfrac{4}{f(x)+1} =2$ nên đồ thị hàm số đã cho có tiệm cận ngang là $y=2$.
    }
\end{ex}
\begin{ex}%[2D1K4-1]
    Cho hàm số $y=f(x)$ có bảng biến thiên như hình bên. Đồ thị hàm số $y=\dfrac{2-f(x)}{f(x)+3}$ có tiệm cận ngang là đường thẳng
    \begin{center}
        \begin{tikzpicture}[scale=0.8]
            \tkzTabInit[nocadre=false,lgt=1.5,espcl=3,deltacl=0.6]
            {$x$ /0.6,$y’$ /0.6,$y$ /2}
            {$-\infty$ ,$0$, $2$, $+\infty$}
            \tkzTabLine{,-,0,+,0,-,}
            \tkzTabVar{+/$+\infty$,-/$1$,+/$5$,-/$-\infty$}
        \end{tikzpicture}
    \end{center}
    \choice
    {$y=1$}
    {$y=-3$}
    {$y=2$}
    {\True $y=-1$}
    \loigiai{
        Dựa vào bảng biến thiên suy ra
        $\lim \limits_{x \to \pm \infty} f(x)=\pm \infty \Leftrightarrow \lim \limits_{x \to \pm \infty} \dfrac{2-f(x)}{f(x)+3} =-1$ nên đồ thị hàm số $y=\dfrac{2-f(x)}{f(x)+3}$ có tiệm cận ngang là $y=-1$.
    }
\end{ex}
\begin{ex}%[2D1K4-1]
    Cho hàm số $y=f(x)$ có bảng biến thiên như hình bên. Đồ thị hàm số $y=\dfrac{1}{f^2(x)-4f(x)+4}$ có bao nhiêu tiệm cận đứng?
    \begin{center}
        \begin{tikzpicture}[scale=0.8]
            \tkzTabInit[nocadre=false,lgt=1.5,espcl=3,deltacl=0.6]
            {$x$ /0.6,$y’$ /0.6,$y$ /2}
            {$-\infty$, $2$, $+\infty$}
            \tkzTabLine{,-,0,+,}
            \tkzTabVar{+/$1$,-/$-3$,+/$1$}
        \end{tikzpicture}
    \end{center}
    \choice
    {$1$}
    {$3$}
    {$2$}
    {$0$}
    \loigiai{
        Dựa vào bảng biến thiên suy ra $f^2(x)-4f(x)+4=0 \Leftrightarrow f(x)=2$, phương trình $f(x)=2$ vô nghiệm nên đồ thị hàm số đã cho không có tiệm cận đứng.
    }
\end{ex}
%83
\begin{ex}%[2D1K4-1]
    Cho hàm số $y=f(x)$ có bảng biến thiên như hình bên. Đồ thị hàm số $y=\dfrac{1}{f(3-x)-2}$ có bao nhiêu tiệm cận đứng?
    \begin{center}
        \begin{tikzpicture}[scale=0.8]
            \tkzTabInit[nocadre=false,lgt=1.5,espcl=3,deltacl=0.6]
            {$x$ /0.6,$y’$ /0.6,$y$ /2}
            {$-\infty$ ,$-2$, $2$, $+\infty$}
            \tkzTabLine{,+,0,-,0,+,}
            \tkzTabVar{-/$-\infty$,+/$3$,-/$0$,+/$+\infty$}
        \end{tikzpicture}
    \end{center}
    \choice
    {$1$}
    {\True $3$}
    {$2$}
    {$0$}
    \loigiai{
        Dựa vào bảng biến thiên suy ra $f(3-x)-2=0 \Leftrightarrow f(3-x)=2$, phương trình này có $3$ nghiệm phân biệt nên đồ thị hàm số đã cho có $3$ tiệm cận đứng.
    }
\end{ex}
\begin{ex}%[2D1G4-1]
    Cho hàm số $y=f(x)$ có bảng biến thiên như hình bên. Đồ thị hàm số $y=\dfrac{4}{f(x^2)-2}$ có bao nhiêu tiệm cận đứng?
    \begin{center}
        \begin{tikzpicture}[scale=0.8]
            \tkzTabInit[nocadre=false,lgt=1.5,espcl=3,deltacl=0.6]
            {$x$ /0.6,$y’$ /0.6,$y$ /2}
            {$-\infty$ ,$0$, $3$, $+\infty$}
            \tkzTabLine{,-,0,+,d,-,}
            \tkzTabVar{+/$8$,-/$1$,+/$4$,-/$2$}
        \end{tikzpicture}
    \end{center}
    \choice
    {$5$}
    {$3$}
    {\True $2$}
    {$4$}
    \loigiai{
        Dựa vào bảng biến thiên suy ra
        $f(x^2)-2=0 \Leftrightarrow f(x^2) =2$. Kẻ đường thẳng $y=2$ ta thấy đường thẳng cắt đồ thị hàm số tại hai điểm phân biệt. Suy ra
        $$\hoac{&x^2=a \; (a<0)\\&x^2=b \; (b >0)} \Rightarrow x=\pm \sqrt{b}.$$
        Do đó đồ thị hàm số đã cho có $2$ tiệm cận đứng.
    }
\end{ex}%89
\begin{ex}%[2D1G4-1]
    Cho hàm số $y=f(x)$ có bảng biến thiên như hình bên. Đồ thị hàm số $y=\dfrac{2}{f(|x|)-3}$ có bao nhiêu tiệm cận ngang?
    \begin{center}
        \begin{tikzpicture}[scale=0.8]
            \tkzTabInit[nocadre=false,lgt=1.5,espcl=3,deltacl=0.6]
            {$x$ /0.6,$y’$ /0.6,$y$ /2}
            {$-\infty$ ,$0$, $2$, $+\infty$}
            \tkzTabLine{,+,0,-,0,+,}
            \tkzTabVar{-/$-\infty$,+/$3$,-/$-1$,+/$+\infty$}
        \end{tikzpicture}
    \end{center}
    \choice
    {$4$}
    {\True $3$}
    {$5$}
    {$6$}
    \loigiai{
        Dựa vào bảng biến thiên suy ra
        $f(|x|)-3=0 \Leftrightarrow f(|x|) =3$.\\
        Bảng biến thiên hàm số $y=f(|x|)$ như sau
        \begin{center}
            \begin{tikzpicture}[scale=0.8]
                \tkzTabInit[nocadre=false,lgt=1.5,espcl=3,deltacl=0.6]
                {$x$ /0.6,$y’$ /0.6,$y$ /2}
                {$-\infty$ ,$-2$, $0$, $2$, $+\infty$}
                \tkzTabLine{,-,0,+,0,-,0,+,}
                \tkzTabVar{+/$+\infty$,-/$-1$,+/$3$,-/$-1$,+/$+\infty$}
            \end{tikzpicture}
        \end{center}
        Dựa vào bảng biến thiên hàm số $y=f(|x|)$, phương trình $f(|x|) =3$ có ba nghiệm phân biệt, do đó đồ thị hàm số $y=\dfrac{2}{f(|x|)-3}$ có $3$ tiệm cận đứng.
    }
\end{ex}
\begin{ex}
    \immini{ %Câu 90
        Cho hàm số bậc ba $f(x)= ax^3 +bx^2 +cx +d$ có đồ thị như hình vẽ bên. Đồ thị hàm số $g(x) = \dfrac{\sqrt{x+1}}{(x-3)\cdot f(x)}$ có bao nhiêu đường tiệm cận đứng?
        \choice
        {5}
        {2}
        {4}
        {\True 3}}{\begin{tikzpicture}[scale=.5, font=\footnotesize, line join=round, line cap=round, >=stealth]
            \def\xmin{-3}\def\xmax{3}\def\ymin{-5}\def\ymax{1}
            \draw[->] (\xmin-0.2,0)--(\xmax+0.2,0) node[below] {\footnotesize $x$};
            \draw[->] (0,\ymin-0.2)--(0,\ymax+0.2) node[right] {\footnotesize $y$};
            \draw (0,0) node [below left] {\footnotesize $O$};
            \foreach \x in {-1}\draw (\x,-0.1)--(\x,0.1) node [above] {\footnotesize $\x$};
            \foreach \x in {2}\draw (\x,-0.1)--(\x,0.1) node [above right] {\footnotesize $\x$};
            \foreach \y in {}\draw (-0.1,\y)--(0.1,\y) node [right] {\footnotesize $\y$};
            \clip (\xmin,\ymin) rectangle (\xmax,\ymax);
            \draw[smooth,samples=200,domain=\xmin:\xmax] plot (\x,{1*((\x)^3)+0*((\x)^2)+-3*(\x)+-2});
        \end{tikzpicture}
    }
    \loigiai{
        * Điều kiện: $\heva{&x \ne 3\\&f(x) \ne 0\\&x \ge -1.}$\\
        Nhìn hình vẽ ta thấy
        $f(x)=0\Leftrightarrow \hoac{&x=-1&(\text{nghiệm kép}) \\&x=2&(\text{nghiệm đơn}).}$\\
        Vậy $g(x) = \dfrac{\sqrt{x+1}}{(x-3)\cdot a(x+1)^2 (x-2)}.$ \\
        Đồ thị hàm số $g(x)$ có 3 đường tiệm cận đứng.}
\end{ex}
\begin{ex}
    \immini{ %Câu 92.
        Đường cong ở hình bên là đồ thị của hàm số $y = ax^3 +bx^2 +cx+d$. Đồ thị hàm số $y =\dfrac{(2x+1)\sqrt{x-1}}{x\cdot f(x-2)}$ có tất cả bao nhiêu tiệm cận đứng?
        \choice
        {1}
        {3}
        {4}
        {\True 2}}{\begin{tikzpicture}[scale=.6, font=\footnotesize, line join=round, line cap=round, >=stealth]
            \def\xmin{-3}\def\xmax{3}\def\ymin{-3}\def\ymax{3}
            \draw[->] (\xmin-0.2,0)--(\xmax+0.2,0) node[below] {\footnotesize $x$};
            \draw[->] (0,\ymin-0.2)--(0,\ymax+0.2) node[right] {\footnotesize $y$};
            \draw (0,0) node [below left] {\footnotesize $O$};
            \foreach \x in {-2}\draw (\x,-0.1)--(\x,0.1) node [above left] {\footnotesize $\x$};
            \foreach \x in {2}\draw (\x,-0.1)--(\x,0.1) node [above right] {\footnotesize $\x$};
            \foreach \y in {}\draw (-0.1,\y)--(0.1,\y) node [right] {\footnotesize $\y$};
            \clip (\xmin,\ymin) rectangle (\xmax,\ymax);
            \draw[smooth,samples=200,domain=\xmin:\xmax] plot (\x,{(2/3)*((\x)^3)+0*((\x)^2)+-(8/3)*(\x)});
    \end{tikzpicture}}
    \loigiai{
        * Điều kiện: $\heva{&x \ne 0\\&f(x-2) \ne 0\\&x \ge 1.}$\\
        Nhìn hình vẽ ta thấy
        $f(x-2)=0\Leftrightarrow \hoac{&x-2=-2\\&x-2=0\\&x-2=2}\Leftrightarrow \hoac{&x=0&(\text{không thỏa mãn})\\&x=2&(\text{nghiệm đơn})\\&x=4&(\text{nghiệm đơn}).}$\\
        Vậy $g(x) =\dfrac{(2x+1)\sqrt{x-1}}{x\cdot f(x-2)}=\dfrac{(x-1)\sqrt{x+2}}{x\cdot ax(x-2)(x-4)}.$ \\
        Đồ thị hàm số $g(x)$ có 2 đường tiệm cận đứng.}
\end{ex}
\begin{ex}
    \immini{ %Câu 93.
        Cho hàm số $y= f(x)$ có đồ thị cắt trục hoành tại đúng 3 điểm như hình bên. Đồ thị hàm số $y =\dfrac{(x+2)\sqrt{3-x}}{f(|x|)}$
        có tất cả bao nhiêu tiệm cận đứng?
        \choice
        {1}
        {3}
        {4}
        {\True 2}}{\begin{tikzpicture}[scale=.5, font=\footnotesize, line join=round, line cap=round, >=stealth]
            \def\xmin{-2}\def\xmax{5}\def\ymin{-3}\def\ymax{5}
            \draw[->] (\xmin-0.2,0)--(\xmax+0.2,0) node[below] {\footnotesize $x$};
            \draw[->] (0,\ymin-0.2)--(0,\ymax+0.2) node[right] {\footnotesize $y$};
            \draw (0,0) node [below left] {\footnotesize $O$};
            \foreach \x in {-1,2,4}\draw (\x,-0.1)--(\x,0.1) node [above left] {\footnotesize $\x$};
            \foreach \y in {}\draw (-0.1,\y)--(0.1,\y) node [right] {\footnotesize $\y$};
            \clip (\xmin,\ymin) rectangle (\xmax,\ymax);
            \draw[smooth,samples=200,domain=-1.2:0] plot(\x,{0-8.48*(\x)^(2.0)-5.48*(\x)+3.0});
            \draw[smooth,samples=200,domain=0:2]
            plot(\x,{0-2.7989489689153735*(\x)^(3.0)+8.326740175055514*(\x)^(2.0)-6.957684474449535*(\x)+3.0});
            \draw[smooth,samples=200,domain=2:5]
            plot(\x,{2.395330112721417*(\x)^(2.0)-14.371980676328501*(\x)+19.162640901771336});
    \end{tikzpicture}}
    \loigiai{
        * Điều kiện: $\heva{&f(|x|) \ne 0\\&x \le 3.}$\\
        Nhìn hình vẽ ta thấy
        $f(|x|)=0\Leftrightarrow \hoac{&|x|=-1\\&|x|=2\\&|x|=4}\Leftrightarrow \hoac{&x=\pm 2&(\text{nghiệm đơn})\\&x=- 4&(\text{nghiệm đơn})\\&x=4&(\text{không thỏa mãn}).}$\\
        Vậy $y =\dfrac{(x+2)\sqrt{3-x}}{a(x-2)(x+2)(x+4)(x-4)}$ \\
        Đồ thị hàm số có 2 đường tiệm cận đứng.}
\end{ex}
\begin{ex}
    \immini{ %Câu 94.
        Đường cong ở hình bên là đồ thị của hàm số $y = ax^3 +bx^2 +cx+d$. Đồ thị hàm số $y =\dfrac{(2x+1)\sqrt{1-x}}{f(|x|)}$ có tất cả bao nhiều tiệm cận đứng?
        \choice
        { 1}
        {3}
        {4}
        {\True 2}}{\begin{tikzpicture}[scale=.8, font=\footnotesize, line join=round, line cap=round, >=stealth]
            \def\xmin{-1}\def\xmax{2}\def\ymin{-1.5}\def\ymax{1.5}
            \draw[->] (\xmin-0.2,0)--(\xmax+0.2,0) node[below] {\footnotesize $x$};
            \draw[->] (0,\ymin-0.2)--(0,\ymax+0.2) node[right] {\footnotesize $y$};
            \draw (0.15,0) node [below left] {\footnotesize $O$};
            \foreach \x in {}\draw (\x,0.1)--(\x,-0.1) node [below] {\footnotesize $\x$};
            \foreach \y in {-1,1}\draw (0.1,\y)--(-0.1,\y) node [left] {\footnotesize $\y$};
            \clip (\xmin,\ymin) rectangle (\xmax,\ymax);
            \draw[smooth,samples=200,domain=\xmin:\xmax] plot (\x,{4*((\x)^3)+-6*((\x)^2)+0*(\x)+1});
            \draw[dashed] (0.5,0)--(0.5,0.0)--(0,0.0);
            \draw (0.5,-1pt)--(0.5,1pt) node [above] {\footnotesize $\frac{1}{2}$};
            \draw (-0.7,-1pt)--(-0.7,1pt) node [above] {\footnotesize $-\frac{1}{2}$};
            \draw (1,-1pt)--(1,1pt) node [above] {\footnotesize $1$};
            \draw[dashed] (0.0,0)--(0.0,1.0)--(0,1.0);
            \draw[dashed] (1.0,0)--(1.0,-1.0)--(0,-1.0);
    \end{tikzpicture}}
    \loigiai{
        * Điều kiện: $\heva{&f(|x|) \ne 0\\&x \le 1.}$\\
        Nhìn hình vẽ ta thấy
        $f(|x|)=0\Leftrightarrow \hoac{&|x|=-\dfrac{1}{2}\\&|x|=\dfrac{1}{2}\\&|x|=x_1>1}\Leftrightarrow \hoac{&x=\pm \dfrac{1}{2}&(\text{hai nghiệm đơn})\\&x=- x_1&(\text{nghiệm đơn})\\&x=x_1&(\text{không thỏa mãn}).}$\\
        Vậy $y =\dfrac{(2x+1)\sqrt{1-x}}{f(|x|)}=\dfrac{(2x+1)\sqrt{1-x}}{a\left(x-\dfrac{1}{2}\right)\left(x+\dfrac{1}{2}\right)(x+x_1)(x-x_1)}$ \\
        Đồ thị hàm số có 2 đường tiệm cận đứng.}
\end{ex}
\begin{ex}
    \immini{ %Câu 96.
        Cho đồ thị hàm số $y =f(x)$ và trục hoành có đúng 2 điểm chung như hình bên. Đồ thị hàm số $y =\dfrac{(x-1)\sqrt{3-x}}{f(x^2)}$ có tất cả bao nhiêu tiệm cận đứng?
        \choice
        {1}
        {3}
        {4}
        {\True 2}}{\begin{tikzpicture}[scale=.8, font=\footnotesize, line join=round, line cap=round, >=stealth]
            \def\xmin{-1.5}\def\xmax{2}\def\ymin{-1}\def\ymax{4.5}
            \draw[->] (\xmin-0.2,0)--(\xmax+0.2,0) node[below] {\footnotesize $x$};
            \draw[->] (0,\ymin-0.2)--(0,\ymax+0.2) node[right] {\footnotesize $y$};
            \draw (0,0) node [below left] {\footnotesize $O$};
            \foreach \x in {1}\draw (\x,0.1)--(\x,-0.1) node [below] {\footnotesize $\x$};
            \foreach \x in {-1}\draw (\x,0.1)--(\x,-0.1) node [below left] {\footnotesize $\x$};
            \clip (\xmin,\ymin) rectangle (\xmax,\ymax);
            \draw[smooth,samples=200,domain=-1.1:0] plot(\x,{21.044670464836045*(\x)^(3.0)+24.701786337609526*(\x)^(2.0)+5.65711587277348*(\x)+2.0});
            \draw[smooth,samples=200,domain=0:\xmax] plot(\x,{10.591704641658401*(\x)^(3.0)-19.26315454354621*(\x)^(2.0)+6.6714499018878115*(\x)+2.0});
    \end{tikzpicture}}
    \loigiai{
        * Điều kiện: $\heva{&f(x^2) \ne 0\\&x \le 3.}$\\
        Nhìn hình vẽ ta thấy
        $f(x^2)=0\Leftrightarrow \hoac{&x^2=-1\\&x^2=1}\Leftrightarrow x=\pm 1\,(\text{nghiệm kép}).$\\
        Vậy $y=\dfrac{(x-1)\sqrt{3-x}}{f(x^2)}=\dfrac{(x-1)\sqrt{3-x}}{(x-1)^2(x+1)^2}$ \\
        Đồ thị hàm số có 2 đường tiệm cận đứng.}
\end{ex}
\begin{ex}%[2D1G4-3]%Câu 52
    Cho hàm số $y=ax^3+bx^2+cx+d$ có đồ thị như hình vẽ. Đồ thị của hàm số $g(x)=\dfrac{x^2-x}{f^2(x)-2f(x)}$ có bao nhiêu đường tiệm cận đứng?
    \choice
    {$2$}
    {$3$}
    {\True $4$}
    {$5$}
    \begin{center}
        \begin{tikzpicture}[thick,>=stealth,x=1cm,y=1cm,scale=.7]
            \draw[thin,color=gray!50] (-3.3,-1.3) grid (3.9,5.9);
            \draw[->] (-3.2,0) -- (4.2,0) node[right] {$x$};
            \draw[->] (0,-1.2) -- (0,5.2) node[above] {$y$};
            \draw[color=blue, domain=-2.15:2.15,samples=300] plot (\x,{(\x)^3-3*(\x)+2}) node[right] {$y=f(x)$};
            \draw (-2,0) circle (1.5pt) node[below left]{$-2$};
            \draw (-1,0) circle (1.5pt) node[below]{$-1$};
            \draw (0,0) circle (1.5pt) node[above left]{$O$};
            \draw (1,0) circle (1.5pt) node[below]{$1$};
            \draw (0,4) circle (1.5pt) node[right]{$4$};
            \draw (-1,4) circle (1.5pt);
            \draw[dashed] (-1,0)--(-1,4)--(0,4);
            \draw[red] (-3,2)--(3.2,2);
            \draw[red] (3.5,2) node[right]{$f(x)=2$};
        \end{tikzpicture}
    \end{center}
    \loigiai{
        Xét phương trình $f^2(x)-2f(x)=0 \Leftrightarrow \hoac{&f(x)=0\\&f(x)=2}\Leftrightarrow \hoac{&x=1 \, (\textrm{nghiệm kép trùng nghiệm đơn ở tử số})\\&x=-2\, (\textrm{nghiệm đơn khác nghiệm của tử})\\&x=a\in(-2; -1)\\&x=0\, (\textrm{nghiệm đơn trùng nghiệm ở tử})\\&x=b\in(1; 2)}$\\
        \textbf{Kết luận:} Đồ thị hàm số có $4$ đường tiệm cận đứng.
    }
\end{ex}
\begin{ex}%[Thi thử L3, Lương Thế Vinh, Hà Nội, 2018]%[Phạm Toàn, Dự án (12EX-10)]%[2D1G4-3]%
    \immini{Cho hàm số $y=f(x)$ có đạo hàm liên tục trên $\mathbb{R}$. Đồ thị hàm $f(x)$ như hình vẽ. Số đường tiệm cận đứng của đồ thị hàm số $y=\dfrac{x^2-1}{f^2(x)-4f(x)}$ bằng
        \choice
        {$3$}
        {$1$}
        {$2$}
        {\True $4$}
    }{\begin{tikzpicture}[>=stealth,x=1cm,y=0.75cm,scale=0.7]
            \draw[->] (-2.5,0)--(0,0)%
            node[below right]{$O$}--(2.5,0) node[below]{$x$};
            \draw[->] (0,-2) --(0,5) node[right]{$y$};
            \foreach \x in {-1,1}{
                \draw (\x,0) node[below]{\footnotesize $\x$} circle (1pt);%Ox
            }
            \foreach \y in {2,4}{
                \draw (0,\y) node[right]{\footnotesize $\y$} circle (1pt);%Oy
            }
            \draw[samples=100,domain=-2.05:2] plot (\x,{(\x -1)^2*(\x+2)});
            \draw [dashed] (-1,0)--(-1,4)--(0,4);
            \draw(-1,4) circle (1pt);
    \end{tikzpicture}}
    \loigiai{Xét $f^2(x)-4f(x)=0\Leftrightarrow \hoac{& f(x)=0\\ &f(x)=4.}$\\
        Xét $f(x)=0$ có hai nghiệm, nghiệm $x_1\ne \pm 1$ và nghiệm $x_2=1$ là nghiệm bội (do đồ thị tiếp xúc với trục hoành tại $x=1$. Trường hợp này có $2$ tiệm cận đứng.\\
        Xét $f(x)=4$ có hai nghiệm, nghiệm $x_3\ne \pm 1$ và nghiệm $x_4=-1$ là nghiệm bội (do đồ thị tiếp xúc với đường thẳng $y=4$ tại $x=-1$. Trường hợp này có $2$ tiệm cận đứng.\\
        Vậy đồ thị có $4$ tiệm cận đứng.}
\end{ex}
\begin{ex}%[Thi thử, Trường THPT Lý Thái Tổ - Bắc Ninh, 2019]%[Duong Xuan Loi, 12EX3]%[2D1G4-3]%
    \immini{
        Cho hàm số $f(x)$ có đồ thị như hình bên. Số đường tiệm cận đứng của đồ thị hàm số
        $y=\dfrac{(x^2-4)(x^2+2x)}{[f(x)]^2+2f(x)-3}$ là
        \choice
        {\True $4$}
        {$5$}
        {$3$}
        {$2$}
    }{
        \begin{tikzpicture}[scale=0.5, font=\footnotesize, line join=round, line cap=round, >=stealth]
            \def\a{1} \def\b{-8} \def\c{1} % Hệ số
            \def\xt{-3.7} \def\xp{4} \def\yt{2} \def\yd{-3.7} % x_trái, x_phải, y_trên, y_dưới (giới hạn)
            \draw[->] (\xt,0)--(\xp,0) node [below]{$x$};
            \draw[->] (0,\yd)--(0,\yt) node [left]{$y$};
            \node at (0,0) [below left]{$O$};
            \clip (\xt-0.1,\yd+0.1) rectangle (\xp-0.1,\yt-0.1);
            \draw[smooth,samples=300] plot(\x,{1/4*(\a*(\x)^4+\b*(\x)^2)+\c});
            \draw[dashed] (-2,0)node[above]{$-2$}--(-2,-3)--(2,-3)--(2,0)node[above]{$2$};
            \node at (0,-3)[above left]{$-3$};
            \node at (-3,0)[above left]{$-3$};
            \node at (0,1)[above right]{$1$};
            \node at (3,0)[above right]{$3$};
            \fill (0,0) circle (1pt) (0,-3) circle (1pt) (2,0) circle (1pt) (-2,0) circle (1pt) (-3,0) circle (1pt) (0,1) circle (1pt) (3,0) circle (1pt);
        \end{tikzpicture}
    }
    \loigiai{
        Ta có $y=\dfrac{(x^2-4)(x^2+2x)}{[f(x)]^2+2f(x)-3}$ có các nghiệm ở tử là $x=0$ (bội $1$), $x=2$ (bội $1$), $x=-2$ (bội $2$).\\
        Mặt khác, từ đồ thị $f(x)$ ta thấy hàm số $y=\dfrac{(x^2-4)(x^2+2x)}{[f(x)]^2+2f(x)-3}$ có các nghiệm ở mẫu là
        $f^2(x)+2f(x)-3=0\Leftrightarrow \hoac{& f(x)=1 \\ & f(x)=-3}
        \Leftrightarrow \hoac{& x=0,x=x_1,x=x_2 \\ & x=-2,x=2.}$\\
        Trong đó nghiệm $x=0$, $x=-2$, $x=2$ đều có bội $2$ và $x_1$, $x_2$ khác các nghiệm của tử.\\
        So sánh bội nghiệm ở mẫu và bội nghiệm ở tử thì thấy đồ thị có các tiệm cận đứng là $x=0$, $x=2$; $x=x_1$; $x=x_2$.
    }
\end{ex}
\begin{ex}%[Thi thử, THPT Sơn Tây, Hà Nội, 2019]%[Huỳnh Xuân Tín, 12EX3]%[2D1G4-3]%
    \immini{Cho hàm số $ f(x)=(x+3)(x+1)^2(x-1)(x-3)$ có đồ thị như hình vẽ. Đồ thị hàm số $ g(x)=\dfrac{\sqrt{x-1}}{f^2(x)-9f(x)}$ có bao nhiêu tiệm cận đứng và tiệm cận ngang?
        \choice
        {$3$}
        {\True$ 4$}
        {$ 9$}
        { $8$}
    }{\begin{tikzpicture}[scale=0.3, font=\footnotesize, line join=round, line cap=round, >=stealth]
            %\draw[dashed, line width=0.1pt, gray] (-3.2,-5.5) grid (5.2,4.5);
            \draw[->] (-3.5,0)--(0,0) node[below right]{$O$}--(3.6,0) node[below]{$x$};
            \draw[fill=black] (0,0) circle (1pt);
            \draw[->] (0,-7.7) --(0,6.5) node[right]{$y$};
            \foreach \x in {-3,-1,3}{
                \draw[fill=black] (\x,0) node[below left]{$\x$} circle (1pt);}
            \draw[fill=black] (1,0) node[below right]{$1$} circle (1pt);
            \draw[fill=black] (0,1.35) node[above left]{$9$} circle (1pt);
            \draw [black, domain=-3.2:3.18, samples=100] %
            plot(\x,{0.15*(\x+3)*(\x+1)^2*(\x-1)*(\x-3)});
    \end{tikzpicture}}
    \loigiai{Điều kiện xác định của hàm số $g(x)$ là $\heva{&x\ge1\\ &f^2(x)-9f(x)\not=0.}$\\
        Từ $f^2(x)-9f(x)=0\Leftrightarrow \hoac{&f(x)=0\\&f(x)=9.}$\\
        Với $f(x)=0$ có nghiệm là $x=\pm 1, x=\pm 3$.\\
        Dựa vào đồ thị ta thấy nghiệm của phương trình $f(x)=9$ là hoành độ giao điểm của đường thẳng $y=9$ với đồ thị hàm số $y=f(x)$ nên có nghiệm là $-3<x_3<x_2<-1<0<x_1<1<3<x_0$.\\
        Do đó tập xác định của hàm số $y=g(x)$ là $\mathscr{D}=\left[1;+\infty \right)\setminus\left\lbrace1;3;x_0 \right\rbrace $.\\
        Khi đó ta có \begin{itemize}
            \item $\lim\limits_{x\rightarrow1^+ } g(x)=\lim\limits_{x\rightarrow1^+ }\dfrac{\sqrt{x-1}}{f(x)\left(f(x)-9 \right)}=+\infty$ (vì $x$ tiến gần bên phải $1$ thì $f(x)<0, f(x)-9<0$), suy ra đường thẳng $x=1$ là tiệm cận đứng.
            \item $\lim\limits_{x\rightarrow3^+ } g(x)=\lim\limits_{x\rightarrow3^+ }\dfrac{\sqrt{x-1}}{f(x)\left(f(x)-9 \right)}=-\infty$ (vì $x$ tiến gần bên phải $3$ thì $f(x)>0, f(x)-9<0$), suy ra đường thẳng $x=3$ là tiệm cận đứng.
            \item $\lim\limits_{x\rightarrow x_0^+} g(x)=\lim\limits_{x\rightarrow x_0^+ }\dfrac{\sqrt{x-1}}{f(x)\left(f(x)-9 \right)}=+\infty$ (vì $x$ tiến gần bên phải $x_0$ thì $f(x)>0, f(x)-9>0$), suy ra đường thẳng $x=x_0$ là tiệm cận đứng.
        \end{itemize}
        Và $\lim\limits_{x\rightarrow +\infty} g(x)=\lim\limits_{x\rightarrow +\infty }\dfrac{\sqrt{x-1}}{f(x)\left(f(x)-9 \right)}=0$ (vì bậc ở mẫu của $y=g(x)$ là $10$ và bậc tử của nó là $\dfrac{1}{2}$). Do vậy đồ thị hàm số $y=g(x)$ có một tiệm cận ngang là đường thẳng $y=0$.\\
        Vậy đồ thị hàm số $y=g(x)$ có bốn tiệm cận ngang và đứng. }
\end{ex}
\begin{ex}%[Thi thử, Chuyên Quang Trung-Bình Phước, 2021,lần 1]%[Trần Hòa, 12EX6]%[2D1G4-3]%
    \immini{Cho hàm số $y=f(x)=ax^3+bx^2+cx+d$, có đồ thị như hình vẽ. Số đường tiệm cận đứng của đồ thị hàm số $y=\dfrac{x^2+x-2}{f^2(x)-f(x)}$ là
        \choice
        {$3$}
        {$2$}
        {\True $4$}
        {$5$}}
    {\begin{tikzpicture}[scale=.5, font=\footnotesize, line join=round, line cap=round, >=stealth]
            \draw[->] (-2.5,0)--(0,0) node[below right]{$O$}--(2,0) node[below]{$x$};
            \draw[->] (0,-.5) --(0,4.5) node[right]{$y$};
            \draw [domain=-2.05:2.05, samples=100] %
            plot (\x, {(\x+2)*(\x-1)^2});
            \draw[fill] (0,0) circle (1pt);
            \foreach \x/\g in {-2/140,-1/-90,1/-90}
            \draw[fill] (\x,0) circle(.5pt)node [shift={(\g:.3)}] {$\x$};
            \foreach \y/\g in {2/0,4/0}
            \draw[fill] (0,\y) circle(.5pt)node [shift={(\g:.3)}] {$\y$};
            \draw[dashed] (-1,0)--(-1,4)--(0,4);
    \end{tikzpicture}}
    \loigiai{
        \begin{itemize}
            \item $x^2+x-2=(x-1)(x+2)$.\\
            \item Dựa vào đồ thị hàm số $y=f(x)$ ta có $f^2(x)-f(x)=0\Leftrightarrow\hoac{&f(x)=0\\&f(x)=1.}$\\
            $f(x)=0\Leftrightarrow x=-2$, $x=1$ (nghiệm kép).\\
            $f(x)=1\Leftrightarrow\hoac{&x=x_1,(x_1\in (-2;-1))\\&x=x_2,(x_2\in (0;1))\\&x=x_3,(x_3>1). }$
            \item Do đó $y=\dfrac{(x-1)(x+2)}{a^2(x+2)(x-1)^2(x-x_1)(x-x_2)(x-x_3)}$.
        \end{itemize}
        Suy ra đồ thị có các đườn tiệm cận đứng $x=1$, $x=x_1$, $x=x_2$, $x=x_3$.
    }
\end{ex}
\begin{ex}%[Đề thi hết học kì 2, Bình Minh, Ninh Bình 2018]%[Nguyễn Tuấn Anh, dự án EX9]%[2D1G4-3]%
    \immini{Cho hàm số bậc ba $f(x)=ax^3+bx^2+cx+d$ có đồ thị như hình vẽ bên dưới. Hỏi đồ thị hàm số $g(x)=\dfrac{(x^2-3x+2)\sqrt{x-1}}{x[f^2(x)-f(x)]}$ có bao nhiêu tiệm cận đứng?
        \choice
        {$5$}
        {$6$}
        {\True $3$}
        {$4$}
    }{
        \begin{tikzpicture}[line width=1.0pt,line join=round,>=stealth,x=1cm,y=1cm,scale=1.0]
            \draw[->,line width = 1pt] (-1,0)--(0,0) node[below right]{$O$}--(4,0) node[below]{$x$};
            \draw[->,line width = 1pt] (0,-1.5) --(0,2.5) node[right]{$y$};
            \foreach \x in {1,2}{
                \draw (\x,0) node[below]{$\x$} circle (1pt);
            }
            \foreach \y in {1}{
                \draw (0,\y) node[left]{$\y$} circle (1pt);
            }
            \clip(-0.8,-1) rectangle (3.8,2.3);
            \draw [line width=1.0pt, thick, domain=-0.5:3.5, samples=100]%,domain=-1.5:3] %
            plot (\x, {(5*(\x)-4)*((\x)-2)^2});
            \draw [dash pattern=on 4pt off 4pt] (1.,0.)-- (1.,1.)-- (0.,1.);
            \draw (1,1) circle (1pt);
        \end{tikzpicture}
    }
    \loigiai{
        Điều kiện $\heva{&x\geq 1\\ &x\ne 0\\ &f^2(x)-f(x)\ne 0}\Leftrightarrow \heva{&x\geq 1\\ &f(x)\ne 0\\ & f(x)\ne 1.}$\\
        Dựa vào đồ thị hàm số $y=f(x)$, ta thấy $f(x)=0$ có hai nghiệm, một nghiệm $x_1<1$ và một nghiệm kép bằng $2$. Do đó ta biểu diễn được $f(x)$ dưới dạng
        $$ f(x)=a(x-x_1)(x-2)^2. $$
        Dựa vào đồ thị hàm số $y=f(x)$, ta thấy phương trình $f(x)=1$ có ba nghiệm $1,x_2, x_3$, với $1<x_2<2<x_3$. Do đó ta biểu diễn được $f(x)-1$ dưới dạng
        $$ f(x)-1=a(x-1)(x-x_2)(x-x_3). $$
        Lúc này điều kiện được viết lại như sau $\heva{&x>1\\ &x\ne x_2, x\ne 2, x\ne x_3.}$\\
        Với điều kiện đó thì $g(x)$ được viết lại là
        $$ g(x)=\dfrac{\sqrt{x-1}}{a^2x(x-x_1)(x-x_2)(x-2)(x-x_3)}. $$
        Ta có
        \begin{align*}
            &\lim\limits_{x\to 1^+}g(x)=\lim\limits_{x\to 1^+}\dfrac{\sqrt{x-1}}{a^2x(x-x_1)(x-x_2)(x-2)(x-x_3)}=0,\\
            & (x=1\mbox{ \textbf{không} là tiệm cận đứng}) \\
            &\lim\limits_{x\to x_2^+}g(x)=\lim\limits_{x\to x_2^+}\dfrac{\sqrt{x-1}}{a^2x(x-x_1)(x-x_2)(x-2)(x-x_3)}=+\infty,\\
            & (x=x_2\mbox{ là tiệm cận đứng}) \\
            &\lim\limits_{x\to 2^+}g(x)=\lim\limits_{x\to 2^+}\dfrac{\sqrt{x-1}}{a^2x(x-x_1)(x-x_2)(x-2)(x-x_3)}=-\infty,\\
            & (x=2\mbox{ là tiệm cận đứng}) \\
            &\lim\limits_{x\to x_3^+}g(x)=\lim\limits_{x\to x_3^+}\dfrac{\sqrt{x-1}}{a^2x(x-x_1)(x-x_2)(x-2)(x-x_3)}=+\infty,\\
            & (x=x_3\mbox{ là tiệm cận đứng}) \\
        \end{align*}
        Vậy đồ thị hàm số $g(x)$ có tất cả $3$ tiệm cận đứng.
    }
\end{ex}
\begin{ex}%[VDC5-Đỗ Đường Hiếu]%[2D1G4-3]%
    \immini{Cho hàm số $f(x)=(x+3)(x+1)^2(x-1)(x-3)$ có đồ thị như hình vẽ. Đồ thị hàm số $g(x)=\dfrac{\sqrt{x-1}}{f^2(x)-9f(x)}$ có bao nhiêu tiệm cận đứng và tiệm cận ngang?
        \choice
        {$3$}
        {\True $4$}
        {$9$}
        {$8$}}
    {\begin{tikzpicture}[xscale=0.8,yscale=0.05, line join=round, line cap=round,font=\footnotesize,>=stealth]
            \draw[->] (-4,0)--(4,0) node[below]{$x$};
            \draw[->] (0,-56)--(0,30) node[left]{$y$};
            \coordinate[label=below left:$O$] (O) at (0,0);
            \draw (-1,0) node[below] { $-1$}(1,0) node[below] { $1$};
            \draw (-3,0) node[below left] { $-3$};
            \draw (3,0) node[below right] { $3$};
            \clip (-3.3,-60) rectangle (3.5,26);
            \draw[smooth,samples=300,domain=-3.5:3.5] plot(\x,{(\x+3)*(\x+1)^2*(\x-1)*(\x-3)});
            \foreach \x in {-3,-1,1,3}
            \draw[shift={(\x,0)},color=black] (0pt,20pt) -- (0pt,-20pt);
            \draw[shift={(0,9)},color=black] (2pt,0pt) -- (-2pt,0pt) node[left] {$9$};
        \end{tikzpicture}
    }
    \loigiai{%GV tổng quát hóa bài toán:
        Cho hàm số đa thức $y=f(x)$ có đồ thị $(C)$. Tìm số đường tiệm cận của đồ thị hàm số $g(x)=\dfrac{\sqrt{ax+b}}{P\left(f(x) \right) }$, trong đó $P\left(f(x) \right)$ là một đa thức của $f(x)$.
        Nếu $a>0$ thì $\lim\limits_{x\to +\infty}g(x)=0$.\\
        Nếu $a<0$ thì $\lim\limits_{x\to -\infty}g(x)=0$.\\
        Do đó đồ thị hàm số $y=g(x)$ luôn có duy nhất một đường tiệm cận ngang là $y=0$.\\
        Gọi $x=x_0$ là một nghiệm của phương trình $P\left(f(x) \right) =0$ thỏa mãn điều kiện $ax+b\ge 0$. Rõ ràng khi đó $\lim\limits_{x\to x_0^+}g(x)=+\infty$ hoặc $\lim\limits_{x\to x_0^+}g(x)=-\infty$.\\
        Bởi vậy, số đường tiệm cận đứng của đồ thị hàm số $y=g(x)$ chính là số nghiệm của phương trình $P\left(f(x) \right) =0$ thỏa mãn điều kiện $ax+b\ge 0$.
        \immini{Ta có $f^2(x)-9f(x)=0\Leftrightarrow \hoac{&f(x)=0\\&f(x)=9.}$\\
            \begin{itemize}
                \item $f(x)=0$ có các nghiệm thuộc $\left[1;+\infty\right)$ là $x=1$ và $x=3$.
                \item Đường thẳng $y=9$ cắt đồ thị hàm số $y=f(x)$ tại duy nhất một điểm có hoành độ thuộc $\left[1;+\infty\right)$ là $x=a>3$.
            \end{itemize}
        }
        {\begin{tikzpicture}[xscale=0.8,yscale=0.05, line join=round, line cap=round,font=\footnotesize,>=stealth]
                \draw[->] (-4,0)--(4,0) node[below]{$x$};
                \draw[->] (0,-56)--(0,30) node[left]{$y$};
                \coordinate[label=below left:$O$] (O) at (0,0);
                \draw (-4,9)--(4,9);
                \draw (-1,0) node[below] { $-1$}(1,0) node[below] { $1$};
                \draw (-3,0) node[below left] { $-3$};
                \draw (3,0) node[below right] { $3$};
                \clip (-3.3,-60) rectangle (3.5,26);
                \draw[smooth,samples=300,domain=-3.5:3.5] plot(\x,{(\x+3)*(\x+1)^2*(\x-1)*(\x-3)});
                \foreach \x in {-3,-1,1,3}
                \draw[shift={(\x,0)},color=black] (0pt,20pt) -- (0pt,-20pt);
                \draw[shift={(0,9)},color=black] (2pt,0pt) -- (-2pt,0pt) node[above left] {$9$};
        \end{tikzpicture}}
        \noindent
        Bởi vậy, hàm số $g(x)=\dfrac{\sqrt{x-1}}{f^2(x)-9f(x)}$ có tập xác định là $\mathscr D=\left[1;3\right) \cup \left(3;a\right) \cup\left( a;+\infty\right)$.\\
        Khi đó ta có
        \begin{itemize}
            \item $\lim\limits_{x\to+\infty}g(x)=0$ nên đồ thị hàm số $y=g(x)$ có một đường tiệm cận ngang là đường thẳng $y=0$.
            \item $\lim\limits_{x\to 1^+}g(x)=\lim\limits_{x\to 1^+}\dfrac{\sqrt{x-1}}{f(x)\left[f(x)-9\right] }=+\infty$;\\
            $\lim\limits_{x\to 3^+}g(x)=\lim\limits_{x\to 3^+}\dfrac{\sqrt{x-1}}{f(x)\left[f(x)-9\right] }=-\infty$;\\
            $\lim\limits_{x\to a^+}g(x)=\lim\limits_{x\to a^+}\dfrac{\sqrt{x-1}}{f(x)\left[f(x)-9\right] }=+\infty$.\\
            Do đó nên đồ thị hàm số $y=g(x)$ có $3$ đường tiệm cận đứng là các đường thẳng $x=1$, $x=3$ và $x=a$.
        \end{itemize}
        Như vậy, đồ thị hàm số $y=g(x)$ có $4$ đường tiệm cận, trong đó có $1$ đường tiệm cận ngang và $3$ đường tiệm cận đứng.
    }
\end{ex}
\begin{ex}%[VDC5-Đỗ Đường Hiếu]%[2D1G4-3]%
    \immini{Cho hàm số bậc ba $y=f(x)$ có đồ thị như hình vẽ bên. Đồ thị hàm số $g(x)=\dfrac{x\sqrt{x+1}}{f(x)\left[f^2(x)-16 \right] }$ có bao nhiêu tiệm cận đứng?
        \choice
        {\True $4$}
        {$5$}
        {$6$}
        {$7$}}
    {\begin{tikzpicture}[scale=0.6,line join=round, line cap=round,font=\footnotesize,>=stealth]
            \draw[->] (-2.5,0)--(4,0) node[below]{$x$};
            \draw[->] (0,-5)--(0,2.5) node[left]{$y$};
            \coordinate[label=below left:$O$] (O) at (0,0);
            \draw[dashed] (-1,0)--(-1,-4)--(0,-4);
            \clip (-2.3,-5) rectangle (3.5,2.5);
            \draw[smooth,samples=300,domain=-3.5:3.5] plot(\x,{-0.5*(\x+2)*(\x-1)*(\x-3)});
            \foreach \x in {-2,-1,1,3}
            \draw[shift={(\x,0)},color=black] (0pt,2pt) -- (0pt,-2pt) node[above] { $\x$};
            \foreach \y in {-4,-3,1}
            \draw[shift={(0,\y)},color=black] (2pt,0pt) -- (-2pt,0pt) node[right] {$\y$};
        \end{tikzpicture}
    }
    \loigiai{
        Xét phương trình $f(x)\left[f^2(x)-16 \right]=0$ \, $(*)$, với điều kiện $x\in\left[-1;+\infty \right) $.\\
        Ta có $f(x)\left[f^2(x)-16 \right]=0\Leftrightarrow \hoac{&f(x)=0\\&f(x)=4\\&f(x)=-4.}$\\
        \begin{itemize}
            \item Phương trình $f(x)=0$ có hai nghiệm $x\in\left[-1;+\infty \right) $ là $x=1$ và $x=3$.
            \item Phương trình $f(x)=4$ có không có nghiệm $x\in\left[-1;+\infty \right) $.
            \item Phương trình $f(x)=-4$ có hai nghiệm $x\in\left[-1;+\infty \right) $ là $-1<x_1<0$ và $x_2>3$.
        \end{itemize}
        Rõ ràng $\lim\limits_{x\to x_0^+}g(x)=+\infty$ hoặc $\lim\limits_{x\to x_0^+}g(x)=-\infty$, trong đó $x=x_0$ là nghiệm thuộc $\left[-1;+\infty \right) $ của phương trình $(*)$. Do đó đường thẳng $x=x_0$ là tiệm cận đứng của đồ thị hàm số $y=g(x)$.\\
        Từ đó suy ra đồ thị hàm số $g(x)=\dfrac{x\sqrt{x+1}}{f(x)\left[f^2(x)-16 \right] }$ có $4$ tiệm cận đứng.
    }
\end{ex}
\begin{ex}%[VDC5-Đỗ Đường Hiếu]%[2D1G4-3]%
    \immini{Cho $y=f(x)$ là hàm số đa thức có đồ thị như hình vẽ bên. Đặt $g(x)=\dfrac{\sqrt{x-1}}{\left[f(x)\right]^2-2f(x)}$ có bao nhiêu đường tiệm cận đứng?
        \choice
        {$5$}
        {$3$}
        {$4$}
        {\True $2$}}
    {\begin{tikzpicture}[scale=0.6,line join=round, line cap=round,font=\footnotesize,>=stealth]
            \draw[->] (-3,0)--(2.5,0) node[below]{$x$};
            \draw[->] (0,-1)--(0,5) node[left]{$y$};
            \coordinate[label=above left:$O$] (O) at (0,0);
            \draw[dashed] (-1,0)--(-1,4)--(0,4);
            \clip (-2.3,-1) rectangle (2.5,4.5);
            \draw[smooth,samples=300,domain=-3.5:3.5] plot(\x,{(\x)^3-3*(\x)+2});
            \foreach \x in {-2,-1,1}
            \draw[shift={(\x,0)},color=black] (0pt,2pt) -- (0pt,-2pt) node[below] { $\x$};
            \foreach \y in {2,4}
            \draw[shift={(0,\y)},color=black] (2pt,0pt) -- (-2pt,0pt) node[right] {$\y$};
        \end{tikzpicture}
    }
    \loigiai{
        Xét phương trình $\left[f(x)\right]^2-2f(x)=0$ \, $(*)$, với điều kiện $x\in\left[1;+\infty \right) $.\\
        Ta có $\left[f(x)\right]^2-2f(x)=0\Leftrightarrow \hoac{&f(x)=0\\&f(x)=2.}$\\
        \begin{itemize}
            \item Phương trình $f(x)=0$ có một nghiệm $x\in\left[1;+\infty \right) $ là $x=1$.
            \item Phương trình $f(x)=2$ có một nghiệm $x\in\left[1;+\infty \right) $ là $x=x_1>1$.
        \end{itemize}
        Rõ ràng $\lim\limits_{x\to x_0^+}g(x)=+\infty$ hoặc $\lim\limits_{x\to x_0^+}g(x)=-\infty$, trong đó $x=x_0$ là nghiệm thuộc $\left[1;+\infty \right) $ của phương trình $(*)$. Do đó đường thẳng $x=x_0$ là tiệm cận đứng của đồ thị hàm số $y=g(x)$.\\
        Từ đó suy ra đồ thị hàm số $g(x)=\dfrac{\sqrt{x-1}}{\left[f(x)\right]^2-2f(x)}$ có $2$ tiệm cận đứng.
    }
\end{ex}
\begin{ex}%[VDC5-NgocDungHo]%[2D1G4-3]%
    \immini
    {
        Cho hàm số $f(x)$ có đồ thị như hình bên. Số đường tiệm cận đứng của đồ thị hàm số $y=\dfrac{(x^2-4)(x^2+2x)}{[f(x)]^2-4f(x)+3}$ là
        \choice
        {$4$}
        {\True $5$}
        {$3$}
        {$2$}
    }
    {\begin{tikzpicture}[>=stealth,scale=0.5, line join=round, line cap=round]
            \def\f[#1]{-0.25*((#1)^4-8*(#1)^2+4)}
            \draw[->] (-4.1,0)--(4,0) node [below]{$x$};
            \draw[->] (0,-2)--(0,4) node [left]{$y$};
            \node at (0,0) [above left]{$O$};
            % \clip;
            \draw[domain=-2.9:2.9,samples=300,thick] plot (\x,{\f[\x]});
            \foreach \x in {-2,2} \filldraw (\x,0) node[below]{\x} circle (2pt);
            %\foreach \x in {-3,3} \filldraw (\x,0) node[below left]{\x} circle (2pt);
            \filldraw (-3,0) node[below left]{$-3$} circle (2pt);
            \filldraw (3,0) node[below right]{$3$} circle (2pt);
            \filldraw (0,1) node[left]{$1$} circle (2pt);
            \filldraw (0,3) node[above left]{$3$} circle (2pt);
            \draw[dashed](-2,0)--(-2,3)--(2,3)--(2,0);
            \draw (3,-1.75) node[right]{$y=f(x)$};
        \end{tikzpicture}
    }
    \loigiai{
        Xét hàm số $y=g(x)=\dfrac{(x^2-4 )(x^2+2x)}{[f(x)]^2-4f(x)+3}$.
        \immini
        {
            Giải phương trình $(x^2-4)(x^2+2x)=0 $\\
            $\Leftrightarrow \hoac{& x^2-4=0 \\ & x^2+2x=0}\Leftrightarrow \hoac{& x=\pm 2 \\ & x=0.}$\\
            Giải phương trình $[f(x)]^2-4f(x)+3=0$\\
            $ \Leftrightarrow \hoac{& f(x)=1 \\ & f(x)=3} \Leftrightarrow \hoac{& x = \pm 2 \\ & x=a\\&x=b\\&x=c\\&x=d.}$\\ với $-3<a<-2<b<c<2<d<3$.\\
        }
        {\begin{tikzpicture}[>=stealth,scale=0.8, line join=round, line cap=round]
                \def\f[#1]{-0.25*((#1)^4-8*(#1)^2+4)}
                \def\g[#1]{1}
                \def\h[#1]{3}
                \draw[->] (-4.1,0)--(4,0) node [below]{$x$};
                \draw[->] (0,-2)--(0,4) node [left]{$y$};
                \node at (0,0) [above left]{$O$};
                % \clip;
                \draw[domain=-2.9:2.9,samples=300,thick] plot (\x,{\f[\x]});
                \draw[domain=-4:4,samples=300,thick] plot (\x,{\g[\x]});
                \draw[domain=-4:4,samples=300,thick] plot (\x,{\h[\x]});
                \foreach \x in {-3,-2,2,3} \filldraw (\x,0) node[below]{\x} circle (2pt);
                % \filldraw (-3,0) node[above left]{$-3$} circle (2pt);
                % \filldraw (3,0) node[above ]{$3$} circle (2pt);
                \filldraw (0,1) node[below left]{$1$} circle (2pt);
                \filldraw (0,-1) node[below left]{$-1$} circle (2pt);
                \filldraw (0,3) node[above left]{$3$} circle (2pt);
                \draw[dashed](-2,0)--(-2,3) (2,3)--(2,0) (2.61,0)node[below]{$d$}--(2.61,1) (-2.61,0)node[below]{$a$}--(-2.61,1) (1.08,0)node[below]{$c$}--(1.08,1)(-1.08,0)node[below]{$b$}--(-1.08,1);
                \draw (3,2.75) node[right]{$y=f(x)$};
            \end{tikzpicture}
        }
        Trong điều kiện xác định của hàm số $y=g(x)$ ta có thể viết $$y=g(x)=\dfrac{x(x-2)(x+2)^2}{(x-a)(x-b)(x-c)(x-d) (x-2)^2(x+2)^2}=\dfrac{x}{(x-a)(x-b)(x-c)(x-d)(x-2)}$$
        Vậy số tiệm cận đứng của đồ thị hàm số $y=g(x)$ bằng $5$.
    }
\end{ex}
\Closesolutionfile{ans}
%\subsection{ĐỀ ÔN LUYỆN}
%\boxde
\BTTN
\begin{ex}%[2D1N3-1]Câu 2
 Đường thẳng $x = a$ là một đường tiệm cận đứng của
 đồ thị hàm số $ y = f (x)$ nếu điều kiện sau thoả mãn
 \choice
 {$\displaystyle\lim_{x\to +\infty }f(x)=a$}
 {\True $\displaystyle\lim_{x\to a^-}f(x)=+\infty $}
 {$\displaystyle\lim_{x\to -\infty }f(x)=a$}
 {$\displaystyle\lim_{x\to a^-}f(x)=a $}
 \loigiai{ Đường thẳng $x = a$ được gọi là một đường tiệm cận đứng (hay tiệm cận đứng) của đồ thị hàm số $ y = f (x)$ nếu ít nhất một trong các điều kiện sau thoả mãn: \\$\displaystyle\lim_{x\to a^+}f(x)=+\infty $, $\displaystyle\lim_{x\to a^+}f(x)=-\infty $, $\displaystyle\lim_{x\to a^-}f(x)=-\infty $, $\displaystyle\lim_{x\to a^-}f(x)=+\infty $.}
\end{ex}
\begin{ex}%[2D1N3-1]Câu 4
 Đường thẳng $y = ax + b$ ($a \neq 0$) được gọi là đường tiệm cận xiên của đồ thị hàm số $y = f(x)$ nếu
 \choice
 {\True $\displaystyle\lim_{x\to -\infty }\big(f(x)-ax-b\big)=0$ hoặc $\displaystyle\lim_{x\to +\infty }\big(f(x)-ax-b\big)=0$}
 {$\displaystyle\lim_{x\to -\infty }\big(f(x)-ax+b\big)=0$ hoặc $\displaystyle\lim_{x\to +\infty }\big(f(x)-ax+b\big)=0$}
 {$\displaystyle\lim_{x\to 0 }\big(f(x)-ax+b\big)=+\infty$ hoặc $\displaystyle\lim_{x\to 0 }\big(f(x)-ax+b\big)=+\infty$}
 {$\displaystyle\lim_{x\to 0 }\big(f(x)-ax-b\big)=-\infty$ hoặc $\displaystyle\lim_{x\to 0 }\big(f(x)-ax-b\big)=-\infty$}
 \loigiai{Đường thẳng $y = ax + b, a \neq 0$, được gọi là đường tiệm cận xiên (hay tiệm cận xiên) của đồ thị hàm số $y = f(x)$ nếu\\ $\displaystyle\lim_{x\to -\infty }[f(x)-(ax+b)]=\displaystyle\lim_{x\to -\infty }(f(x)-ax-b)=0$ hoặc\\ $\displaystyle\lim_{x\to +\infty }[f(x)-(ax+b)]=\displaystyle\lim_{x\to +\infty }(f(x)-ax-b)=0$.
 }
\end{ex}
\begin{ex}
 Tiệm cận ngang của đồ thị hàm số $ y=\dfrac{2x-1}{x+1} $ là đường thẳng
 \choice
 {$y=-1$}
 {$ x=-1 $}
 {\True $ y=2 $}
 {$ x=2 $}
 \loigiai
 {
 Ta có $ \lim\limits_{x\to \pm\infty}y=2$ suy ra đường thẳng $ y=2 $ là tiệm cận ngang của đồ thị hàm số $ y=\dfrac{2x-1}{x+1} $.
 }
\end{ex}
\begin{ex}
 Tiệm cận ngang của đồ thị hàm số $y=\dfrac{1}{2x-3}$ là đường thẳng
 \choice
 {$y=\dfrac{3}{2}$}
 {$x=\dfrac{3}{2}$}
 {\True $y=0$}
 {$y=\dfrac{1}{2}$}
 \loigiai{
 Vì $\lim\limits_{x\to -\infty} \dfrac{1}{2x-3}=\lim\limits_{x\to +\infty} \dfrac{1}{2x-3}=0$ nên đồ thị hàm số có tiệm cận ngang $y=0$.
 }
\end{ex}
\begin{ex}
 Đồ thị hàm số $f(x)=\dfrac{2x-3}{x+1}$ có đường tiệm cận đứng là
 \choice
 {$y=2$}
 {\True $x=-1$}
 {$y=-1$}
 {$x=2$}
 \loigiai{
 Ta có $\displaystyle \lim_{x \to (-1)^-}f(x)=\displaystyle \lim_{x \to (-1)^-}\dfrac{2x-3}{x+1}=+\infty $; $\displaystyle \lim_{x \to (-1)^+}f(x)=\displaystyle \lim_{x \to (-1)^+}\dfrac{2x-3}{x+1}=-\infty$ nên đường thẳng $x=-1$ là đường tiệm cận đứng của đồ thị hàm số.}
\end{ex}
\begin{ex}
 Hàm số nào sau đây có đồ thị nhận đường thẳng $x=2$ là đường tiệm cận đứng?
 \choice
 {$y=\dfrac{2}{x+2}$}
 {\True $y=\dfrac{5x}{2-x}$}
 {$y=\dfrac{1}{x+1}$}
 {$y=x-2+\dfrac{1}{x+1}$}
 \loigiai{
 Ta có $\lim\limits_{x\to 2^+} \dfrac{5x}{2-x}=-\infty $ và $\lim\limits_{x\to 2^-} \dfrac{5x}{2-x}=+\infty $ nên đồ thị hàm số $y=\dfrac{5x}{2-x}$ nhận $x=2$ làm tiệm cận đứng.}
\end{ex}
\begin{ex}%[2D1V3-1]Câu 12
 Đồ thị của hàm số nào sau đây có giao điểm của hai đường tiệm cận thuộc đường thẳng $y=x$?
 \choice
 {$y=\dfrac{2x-1}{x+3}$}
 {\True$y=\dfrac{x+4}{x-1}$}
 {$y=\dfrac{2x+1}{x+2}$}
 {$\dfrac{1}{x+3}$}
 \loigiai{
 Đáp án $y=\dfrac{2x-1}{x+3}$ có giao hai đường tiệm tiệm cận là $(-3;2)\notin d$\\
 Đáp án $y=\dfrac{x+4}{x-1}$ có giao hai đường tiệm cận là $(1;1)\in d$\\
 Đáp án $y=\dfrac{2x+1}{x+2}$ có giao hai đường tiệm cận là $(-2;2)\notin d$\\
 Đáp án $\dfrac{1}{x+3}$ có giao hai đường tiệm cận là $(-3;0)\notin d$\\
 }
\end{ex}
\begin{ex}%[2D1N3-1]Câu 6
 Đồ thị hàm số $y=\dfrac{x-2}{x^{2}-4}$ có mấy đường tiệm cận?
 \choice
 {$3$}
 {$1$}
 {\True$2$}
 {$0$}
 \loigiai{ Hàm số $y=\dfrac{x-2}{x^{2}-4}=\dfrac{x-2}{(x-2)(x+2)}=\dfrac{1}{x+2}$.\\
 $\heva{&\displaystyle\lim_{x\to +\infty }\dfrac{1}{x+2}=0\\&
 \displaystyle\lim_{x\to -\infty }\dfrac{1}{x+2}=0.}$\\
 Nên $y=0$ là đường tiệm cận ngang của hàm số, hàm số có tiệm cận ngang thì không có tiệm cận xiên.\\
 $\heva{&\displaystyle\lim_{x\to -2^- }\dfrac{1}{x+2}= - \infty \\&
 \displaystyle\lim_{x\to -2^+ }\dfrac{1}{x+2}= + \infty.}$\\
 Nên $x=-2$ là đường tiệm cận đứng của hàm số.\\
 Vậy hàm số có hai đường tiệm cận.
 }
\end{ex}
\begin{ex}
 Tiệm cận xiên của đồ thị hàm số $y=\dfrac{x^2+x-1}{x}$ có phương trình là
 \choice
 {$y=x-1$}
 {$y=x-2$}
 {$y=x-3$}
 {\True$y=x+1$}
 \loigiai{
 Ta có $y=\dfrac{x^2+x-1}{x}=x+1-\dfrac{1}{x}$.\\
 Xét $$\displaystyle\lim_{x\to \pm \infty }\big(y-(x+1)\big)=\displaystyle\lim_{x\to \pm \infty }\dfrac{-1}{x}=0$$
 Vậy đường tiệm cận xiên cần tìm của hàm số $f(x)$ có phương trình $y=x+1$.}
\end{ex}
\begin{ex}
 Tiệm cận xiên của đồ thị hàm số $y=\dfrac{2x^2-3x+4}{x-1}$ có phương trình là
 \choice
 {$y=x-1$}
 {\True $y=2x-1$}
 {$y=2x+1$}
 {$y=x+1$}
 \loigiai{
 Ta có $y=\dfrac{2x^2-3x+4}{x-1}=2x-1+\dfrac{3}{x-1}$. Suy ra $y=2x-1$ là đường tiệm cận xiên của đồ thị hàm số.
 }
\end{ex}
\begin{ex}
 Cho hàm số $y=f(x)$ xác định $ \mathbb{R} \setminus \left\lbrace 0\right\rbrace $, liên tục trên mỗi khoảng xác định và có bảng biến thiên như sau.\\
 \begin{center}
 \begin{tikzpicture}[>=stealth,font=\footnotesize,scale=1]
 \tikzset{double style/.append style = {draw=\tkzTabDefaultWritingColor,double=\tkzTabDefaultBackgroundColor,double distance=2pt}}
 \tkzTabInit[nocadre=false,lgt=1.2,espcl=2.5,deltacl=0.6]
 {$x$ /0.6,$y'$ /0.6,$y$ /2}
 {$-\infty$,$0$,$ 1 $,$+\infty$}
 \tkzTabLine{,-,d,+,$ 0 $,- }
 \tkzTabVar{+/ $+\infty$,-D- /$-1$/$-\infty$,+/$2$,-/$ -\infty $}
 \end{tikzpicture}
 \end{center}
 Chọn khẳng định đúng
 \choice
 {Đồ thị hàm số có hai tiệm cận ngang}
 {\True Đồ thị hàm số có đúng một tiệm cận đứng}
 {Đồ thị hàm số không có tiệm cận đứng và tiệm cận ngang}
 {Đồ thị hàm số có đúng một tiệm cận ngang}
 \loigiai
 {
 Dựa vào bảng biến thiên ta thấy
 $ \lim\limits_{x \to + \infty} f(x)=+\infty$; $ \lim\limits_{x \to -\infty}f(x)=-\infty$; $ \lim\limits_{x \to 0^+}f(x)=-\infty$.\\
 Suy ra đồ thị hàm số có đúng một tiệm cận đứng.
 }
\end{ex}
\begin{ex}%[2D1N3-1]Câu 5
 \immini{Cho hàm số $y=f(x)$ có đồ thị như hình bên dưới. Khẳng định nào sau đây là khẳng định đúng?
 \choice
 {Đồ thị hàm số chỉ có 2 đường tiệm cận đứng $x=-1$ và $x=1$}
 {\True Đồ thị hàm số có 3 đường tiệm cận}
 {Đồ thị hàm số có 4 đường tiệm cận}
 {Đồ thị hàm số có 2 đường tiệm cận đứng và 1 đường tiệm cận xiên}}{
 \begin{tikzpicture}[>=stealth]
 \draw[->] (-4,0) --(4,0);
 \draw[->](0,-4)--(0,4);
 \draw (0,0) node[below left]{$O$};
 \draw (4,0) node[below]{$x$};
 \draw (0,4) node[left]{$y$};
 \draw (1,0) node[above left]{$1$};
 \draw (-1,0) node[above left]{$-1$};
 \clip (-4,-4) rectangle(4,4);
 \draw[thick,samples=100] plot[domain=-4:4]
 (\x,{(\x)/((\x)^(2)-1)});
 \draw (-1.7,1.5) node
 {$x=-1$};
 \draw (1.5,-1.5) node
 {$x=1$};
 \end{tikzpicture}}
 \loigiai{ Đồ thị hàm số có 2 đường tiệm cận đứng $x=-1$ và $x=1$ và một đường tiệm cận ngang $y=0$, hàm số không có đường tiệm cận xiên.}
\end{ex}
\begin{ex}
 Biết rằng đồ thị hàm số $ y=\dfrac{ax+1}{bx-2}$ có tiệm cận đứng là $x=2$ và tiệm cận ngang là $y=3$. Giá trị của $a+b$ bằng
 \choice
 {$0$}
 {\True $4$}
 {$5$}
 {$1$}
 \loigiai{
 Điều kiện để đồ thị hàm số $ y=\dfrac{ax+1}{bx-2}$ có tiệm cận đứng và tiệm cận ngang là $-2a-b\ne 0$. \quad$(*)$\\
 $b\ne 0$ vì nếu $ b=0$, đồ thị hàm số $ y=\dfrac{ax+1}{-2}$ không có tiệm cận.\\
 Tập xác định của hàm số $y=\dfrac{ax+1}{bx-2}$ là $\mathscr{D}=\left(-\infty;\dfrac{2}{b}\right)\cup\left(\dfrac{2}{b};+\infty\right)$.\\
 $\lim\limits_{x\to\pm\infty}\dfrac{ax+1}{bx-2}=\dfrac{a}{b}\Rightarrow y=\dfrac{a}{b}$ là đường tiệm cận ngang của đồ thị hàm số.\\
 Theo giả thiết ta có $\dfrac{a}{b}=3\Leftrightarrow a=3b$.\\
 Đồ thị hàm số $y=\dfrac{ax+1}{bx-2}$ có $ x=\dfrac{2}{b}$ là đường tiệm cận đứng.\\
 Theo giả thiết ta có $\dfrac{2}{b}=2\Leftrightarrow b=1\Rightarrow a=3$ (thỏa mãn điều kiện $(*)$).\\
 Vậy $a+b=4$.
 }
\end{ex}
\begin{ex}
 Tìm tất cả giá trị của tham số $m$ để đường tiệm cận xiên của đồ thị hàm số $y=2mx+3-\dfrac{4}{x+1}$ đi qua điểm $M(1;7)$.
 \choice
 {$m=1$}
 {$m=3$}
 {\True $m=2$}
 {$m=-2$}
 \loigiai{
 Xét $\displaystyle\lim_{x\to \pm \infty }\left( y-\left( 2mx+3\right) \right) =\displaystyle\lim_{x\to \pm \infty }\dfrac{-4}{x+1}=0$.\\
 Vậy đường tiệm cận xiên có phương trình $y=2mx+3$.\\
 Đường thẳng này qua điểm $M(1;7)$, suy ra $2m \cdot 1 ++3=7 \Leftrightarrow m=2$.
 }
\end{ex}
\begin{ex}
 Tại một công ty sản xuất đồ chơi A, công ty phải chi 50000 USD để thiết lập dây chuyền sản xuất ban đầu. Sau đó, cứ sản xuất được một sản phẩm đồ chơi A, công ty phải trả 5 USD cho nguyên liệu thô và nhân công. Gọi $x\,(x \geq 1)$ là số đồ chơi A mà công ty đã sản xuất và $T(x)$ (đơn vị USD) là tổng số tiền bao gồm cả chi phí ban đầu mà công ty phải chi trả khi sản xuất $x$ đồ chơi A. Người ta xác định chi phí trung bình cho mỗi sản phẩm đồ chơi A là $M(x)=\dfrac{T(x)}{x}$. Khi $x$ đủ lớn ($x\to +\infty$) thì chi phí trung bình (USD) cho mỗi sản phẩm đồ chơi $A$ gần nhất với kết quả nào sau đây?
 \choice
 {$50\,000$}
 {$50\,005$}
 {10}
 {\True $5$}
 \loigiai{
 Gọi $T(x)$ (đơn vị USD) là tổng số tiền bao gồm cả chi phí ban đầu mà công ty phải chi trả khi sản xuất $x$ đồ chơi A thì $T(x)=50\,000 + 5x$.\\
 Ta có $$\displaystyle\lim_{x\to + \infty }\dfrac{T(x)}{x} =\displaystyle\lim_{x\to + \infty }\left(\dfrac{50\,000}{x}+5\right) =5.$$
 }
\end{ex}
\BTTF
\begin{ex}
 Cho hàm số $y=f(x)$ có $\displaystyle\lim_{x\rightarrow 3^{-}}f(x)=1$, $\displaystyle\lim\limits_{x\rightarrow 3^{+}}f(x)=+\infty$ và $\displaystyle\lim_{x\rightarrow -\infty}f(x)=1$, $\displaystyle\lim\limits_{x\rightarrow +\infty}f(x)=+\infty$. Xét tính đúng sai của các khẳng định sau:
 \choiceTF
 {\True Đồ thị của hàm số $y=f(x)$ có tiệm cận ngang là đường thẳng $y=1$}
 {\True Đồ thị của hàm số $y=f(x)$ có tiệm cận đứng là đường thẳng $x=3$}
 {Đồ thị của hàm số $y=f(x)$ không có tiệm cận ngang}
 {Đồ thị của hàm số $y=f(x)$ không có tiệm cận đứng}
 \loigiai{
 \begin{itemchoice}
 \itemch Do $\displaystyle\lim_{x\rightarrow -\infty}f(x)=1$ nên $y=1$ là đường tiệm cận ngang của đồ thị hàm số. (1)
 \itemch Do $\displaystyle\lim\limits_{x\rightarrow 3^{+}}f(x)=+\infty$ nên $x=3$ là đường tiệm cận đứng của đồ thị hàm số. (2)
 \itemch Từ (1) suy ra khẳng định này sai.
 \itemch Từ (2) suy ra khẳng định này sai.
 \end{itemchoice}
 }
\end{ex}
\begin{ex}
 Cho hàm số $y=f(x)$ xác định trên $\mathbb{R}\backslash\{\pm 2\}$ và có bảng biến thiên như hình vẽ bên dưới.
 \begin{center}
 \begin{tikzpicture}[scale=0.8,>=stealth]
 \tikzset{double style/.append style = {draw=\tkzTabDefaultWritingColor,double=\tkzTabDefaultBackgroundColor,double distance=2pt}}
 \tkzTabInit[nocadre=false, lgt=1, espcl=4,deltacl=1pt]{$x$ /1,$y'$ /1,$y$ /2.2}{$-\infty$,$-2$,$2$,$+\infty$}
 \tkzTabLine{,-,d,-,d,-,}
 \tkzTabVar{+/ $0$ ,-D+/ $-10$/$+\infty$ , -D+/ $-\infty$/$+\infty$,-/$0$}
 \end{tikzpicture}
 \end{center}
 Xét tính đúng sai của các khẳng định sau:
 \choiceTF
 {\True Hàm số không có điểm cực trị}
 {$\lim\limits_{x\to -2^{-}}f(x)=+\infty$}
 {\True Đồ thị hàm số có đúng 1 tiệm cận ngang}
 {Đồ thị hàm số có đúng $1$ tiệm cận đứng}
 \loigiai{
 Dựa vào bảng biến thiên ta thấy
 \begin{itemchoice}
 \itemch Hàm số không có điểm cực trị;
 \itemch $\lim\limits_{x\to -2^{-}}f(x)=-10$;
 \itemch $\lim\limits_{x\to \pm \infty}f(x)=0$. Suy ra đồ thị có đúng 1 đường tiệm cận ngang là $y=0$.
 \itemch $\lim\limits_{x\to -2^{+}}f(x)=+\infty$ và $\lim\limits_{x\to 2^{+}}f(x)=+\infty$ nên đồ thị hàm số có đúng 2 đường tiệm cận đứng $x = \pm 2$.
 \end{itemchoice}
 }
\end{ex}
\begin{ex}
 Cho hàm số $y=\dfrac{\sqrt{x^2-x+2}}{x-1}$. Xét tính đúng sai của các khẳng định sau:
 \choiceTF
 {\True Tập xác định của hàm số là $\mathbb{R} \backslash\{1\}$}
 {\True Đồ thị hàm số có các đường tiệm cận ngang là $y=1,\,y=-1$}
 {Đồ thị hàm số đã cho có tất cả 2 đường tiệm cận}
 {Các đường tiệm cận của đồ thị cùng với trục $O y$ tạo thành 1 đa giác có diện tích bằng 1}
 \loigiai{
 \begin{itemchoice}
 \itemch Điều kiện xác định $\heva{&x^2-x+2>0\text{ luôn đúng}\\& x-1 \ne 0} \Leftrightarrow x \ne 1$. Vậy tập xác định của hàm số là $\mathbb{R} \backslash\{1\}$
 \itemch Ta có
 \begin{itemize}
 \item [$\bullet$] $\displaystyle\lim_{x\rightarrow -\infty}f(x)=-1$ nên $y=-1$ là đường tiệm cận ngang;
 \item [$\bullet$] $\displaystyle\lim_{x\rightarrow +\infty}f(x)=1$ nên $y=1$ là đường tiệm cận ngang;
 \end{itemize}
 \itemch Do $\displaystyle\lim_{x\rightarrow 1^+}f(x)=+\infty$ nên $x=1$ là đường tiệm cận đứng. Vậy đồ thị hàm số có tất cả 3 đường tiệm cận (2 TCN và 1 TCĐ).
 \itemch Minh họa miền giới hạn của các đường tiệm cận và trục $Oy$ như sau:
 \begin{center}
 \begin{tikzpicture}[smooth,samples=300,scale=0.8,>=stealth]
 \draw[->] (-3,0)--(6,0) node[below]{$x$};
 \draw[->] (0,-3)--(0,3) node[right]{$y$};
 \draw (0,0) node[below left]{$O$};
 \draw[pattern = north west lines] (0,-1)--(1,-1)--(1,1)--(0,1);
 \draw
 (-3,-1)--(4,-1)node[below]{\scriptsize TCN $y=-1$}
 (-3,1)--(4,1)node[above]{\scriptsize TCN $y=1$}
 (1,-3)--(1,3)node[above right]{\scriptsize TCĐ $x=1$};
 \end{tikzpicture}
 \end{center}
 Miền giới hạn là hình chữ nhật có diện tích là $S=2 \cdot 1 =2$.
 \end{itemchoice}
 }
\end{ex}
\begin{ex}
 Cho hàm số $y=f(x)=\dfrac{2 x^2+2 x+5}{2 x+1}$. Xét tính đúng sai của các khẳng định sau:
 \choiceTF
 {\True Đạo hàm của hàm số đã cho là $y'=\dfrac{4\left(x^2+x-2\right)}{(2 x+1)^2}$}
 {\True Các điểm cực trị của đồ thị hàm số có toạ độ là $(-2 ;-3)$ và $(1 ; 3)$}
 {\True Đường tiệm cận đứng của đồ thị hàm số có phương trình là $x=-\dfrac{1}{2}$}
 {\True Đường tiệm cận xiên của đồ thị hàm số có phương trình là $y=x+\dfrac{1}{2}$}
 \loigiai{
 \begin{itemchoice}
 \itemch Ta có $y'=\dfrac{(2 x^2+2 x+5)'(2 x+1)-(2 x+1)'(2 x^2+2 x+5)}{(2x+1)^2}=\dfrac{4\left(x^2+x-2\right)}{(2 x+1)^2}$.
 \itemch $y'=0 \Leftrightarrow x^2+x-2 =0 \Leftrightarrow \hoac{&x=1\\&x=-2}$.\\
 Thay vào hàm số, ta tính được toạ độ các điểm cực trị là $(-2 ;-3)$ và $(1 ; 3)$.
 \itemch Điều kiện xác định $x \ne -\dfrac{1}{2}$.\\
 $\displaystyle\lim_{x\rightarrow -\frac{1}{2}^+}f(x)=+\infty$ nên $x=-\dfrac{1}{2}$ là đường tiệm cận đứng;
 \itemch $y=\dfrac{2 x^2+2 x+5}{2 x+1}=x+\dfrac{1}{2}+\dfrac{9}{2(2x+1)}$. \\
 Suy ra đồ thị có đường tiệm cận xiên là $y=x+\dfrac{1}{2}$.
 \end{itemchoice}
 }
\end{ex}
\BTTL
\begin{ex}%[2D1B4-1]%
 Các đường tiệm cận của đồ thị hàm số $ y=\dfrac{2x+3}{x-1}$ tạo với hai trục tọa độ một hình chữ nhật có
 diện tích bằng bao nhiêu?\\
 \shortans[3]{$2$}
 \loigiai{
 Tập xác định $\mathscr{D}=\mathbb{R}\setminus\{1\}$.
 \begin{itemize}
 \item $\lim\limits_{x\to 1} y=\lim\limits_{x\to 1^+}\dfrac{2x+3}{x-1}=+\infty\Rightarrow x=1 $ là tiệm cận đứng của đồ thị hàm số.
 \item $\lim\limits_{x\to+\infty} y=\lim\limits_{x\to+\infty}\dfrac{2x+1}{x-1}=2\Rightarrow y=2 $ là tiệm cận ngang của đồ thị hàm số.
 \end{itemize}
 Hai đường tiệm cận của đồ thị hàm số tạo với hai trục tọa độ một hình chữ nhật có diện tích $ S=1\cdot 2=2 $.
 }
\end{ex}
\begin{ex}%[2D1B4-1]%
 Cho hàm số $y=\dfrac{x+1}{x-3}$ có đồ thị $(C)$ và đường thẳng $\Delta: y=mx+m-3.$ Biết đường thẳng $\Delta$ đi qua giao điểm hai đường tiệm cận của
 $(C).$ Khi đó giá trị của $m$ bằng bao nhiêu?\\
 \shortans[3]{$1$}
 \loigiai{
 Đồ thị (C) có TCĐ là $x=3$ và TCN là $y=1$, suy ra $I(3 ; 1)$ là giao điểm hai tiệm cận của $(C)$.\\
 Do $I \in \Delta \Rightarrow 1=3m+m-3 \Leftrightarrow 4m-4=0 \Leftrightarrow m=1$.
 }
\end{ex}
\begin{ex}
 Cho hàm số $y=\dfrac{3x^2+2x}{4x+4}$. Khoảng cách từ điểm $M(3;-2)$ đến đường tiệm cận xiên của đồ thị hàm số này bằng bao nhiêu?\\
 \shortans[3]{$3{,}2$}
 \loigiai{
 $y=\dfrac{3x^2+2x}{4x+4}=\dfrac{3}{4}x-\dfrac{1}{4}+\dfrac{1}{4x+4}$.\\
 Xét $\displaystyle\lim_{x\to \pm \infty }\left( y-\left( \dfrac{3}{4}x-\dfrac{1}{4}\right) \right) =\displaystyle\lim_{x\to \pm \infty }\dfrac{1}{4x+4}=0$.\\
 Vậy đường tiệm cận xiên có phương trình $y=\dfrac{3}{4}x-\dfrac{1}{4} \Leftrightarrow 3x-4y-1=0$.\\
 Khoảng cách từ điểm $M$ đến đường tiệm cận xiên là
 $$d=\dfrac{\big|3 \cdot 3 -4 \cdot (-2)-1\big|}{\sqrt{3^2+(-4)^2}}=\dfrac{16}{5}=3,2$$
 }
\end{ex}
\begin{ex}
 Nồng độ oxygen trong hồ theo thời gian $t$ cho bởi công thức $y(t)=5-\dfrac{15 t}{9 t^2+1}$, với $y$ được tính theo $\mathrm{mg} / l$ và $t$ được tính theo giờ, $t \geq 0$. Đường tiệm cận ngang của đồ thị hàm số $y=y(t)$ khi $t \to +\infty$ có dạng $y=a$. Giá trị của $a$ bằng bao nhiêu?\\
 \shortans[3]{$5$}
 \loigiai{
 $\displaystyle\lim_{t\rightarrow +\infty}y(t)=\lim_{t\rightarrow +\infty}\left( 5-\dfrac{15 t}{9 t^2+1}\right) =5$ nên $y=5$ là đường tiệm cận ngang.
 }
\end{ex}
\begin{ex}
 Số lượng sản phẩm bán được của một công ty trong $x$ (tháng) được tính theo công thức $S(x)=200\left(5-\dfrac{9}{2+x}\right)$, trong đó $x \geq 1$. Xem $y=S(x)$ là một hàm số xác định trên nửa khoảng $[1 ;+\infty)$. Biết $y=a$ là tiệm cận ngang của đồ thị hàm số đó. Giá trị của $a$ bằng bao nhiêu?\\
 \shortans[3]{$1000$}
 \loigiai{
 Ta có
 $S(x)=200\left(5-\dfrac{9}{2+x}\right)=1000-\dfrac{1800}{2+x}.$\\
 Vì $\displaystyle\lim \limits_{n \to +\infty}_{x \rightarrow\pm\infty} S(x)=\lim \limits_{n \to +\infty}_{x \rightarrow\pm\infty} \left(1000-\dfrac{1800}{2+x}\right)=1000$
 nên đường thẳng $y=1000$ là tiệm cận ngang của đồ thị hàm số đã cho.
 }
\end{ex}
\begin{ex}%
 \immini{Cho hàm đa thức bậc ba $y=f(x)$ có đồ thị như hình vẽ.	Đồ thị hàm số $y=\dfrac{(x+1)(x^2-1)}{f(x)}$ có bao nhiêu đường tiệm cận (đứng và ngang)?\\
 \shortans[3]{$3$}}{
 \begin{tikzpicture}[>=stealth]
 \draw[->] (-3,0) --(3,0)node[below]{$x$};
 \draw[->](0,-3)--(0,2.5)node[left]{$y$};
 \draw (0,0.5) node[below left]{$O$};
 \draw (2,0) node[above left]{$2$};
 \draw (-1,0) node[above left]{$-1$};
 \draw[dashed] (0,-2)
 node[left]{$-2$} -- (1,-2) --
 (1,0) node[above]{$1$};
 \draw[thick,samples=100] plot[domain=-2.2:2.3]
 (\x,{(1/2)*(\x)^3-(3/2)*\x-1})node[above]{$y=f(x)$};
 \end{tikzpicture}}
 \loigiai{ Hàm số có dạng $f(x)=ax^3+bx^2+cx-1$ (vì là hàm bậc ba cắt trục tung tại điểm có tung độ $-1$)\\
 Đồ thị hàm số đã cho đi qua các điểm có tọa độ là $(-1;0)$, $(1;-2)$, $(2;0)$ \\
 $\to \heva{&8a+4b+2c=1\\& -a+b-c=1 \\&a+b+c=-1}\Leftrightarrow \heva{&a=\dfrac{1}{2}\\&b=0 \\&c=\dfrac{-3}{2}.}$\\
 $\to f(x)=\dfrac{1}{2}x^3-\dfrac{3}{2}x-1=\dfrac{1}{2}(x-1)^2(x-2)$.\\
 Khi đó $y=\dfrac{(x+1)(x^2-1)}{f(x)}=\dfrac{(x+1)(x^2-1)}{\dfrac{1}{2}(x-1)^2(x-2)}=\dfrac{2(x+1)^2}{(x-1)(x-2)}$.\\
 Đồ thị hàm số trên có tiệm cận ngang $y=2$ và tiệm cận đứng là $x=1,x=2$.\\
 Vậy đồ thị hàm số $y=\dfrac{(x+1)(x^2-1)}{f(x)}$ có 3 đường tiệm cận.
 }
\end{ex}
%\boxde
\BTTN
\begin{ex}%[2D1B4-1]
    Phương trình đường tiệm cận ngang của đồ thị hàm số $y=\dfrac{x-3}{x-1}$ là
    \choice
    {$y=5$}
    {$y=0$}
    {$x=1$}
    {\True $y=1$}
    \loigiai
    {
        Ta có $\lim\limits_{x \to \pm \infty}\dfrac{x-3}{x-1} = \lim\limits_{x \to \pm \infty}\dfrac{1-\dfrac{3}{x}}{1-\dfrac{1}{x}}=1$, nên đường thẳng $y=1$ là tiệm cận ngang của đồ thị hàm số đã cho.
    }
\end{ex}


\begin{ex}
    Đường thẳng nào dưới đây là tiệm cận đứng của đồ thị hàm số $y=\dfrac{2x}{x+2}$?
    \choice
    {$x=2$}
    {$x=0$}
    {\True $x=-2$}
    {$x=1$}
    \loigiai{
        Tập xác định $\mathscr{D}=\mathbb{R}\setminus\{-2\}$.
        \begin{itemize}
            \item $\lim\limits_{x\to -2^+}\dfrac{2x}{x+2}=-\infty$.
            \item $\lim\limits_{x\to -2^-}\dfrac{2x}{x+2}=+\infty$.
        \end{itemize}
        Vậy $x=2$ là đường tiệm cận đứng của đồ thị hàm số.
    }
\end{ex}

\begin{ex}%[2D1Y4-1]
    Cho hàm số $y=\dfrac{x+1}{2x-2}$. Khẳng định nào sau đây đúng?
    \choice
    {Đồ thị hàm số có tiệm cận đứng là $x=\dfrac{1}{2}$}
    {Đồ thị hàm số có tiệm cận ngang là $y=-\dfrac{1}{2}$}
    {\True Đồ thị hàm số có tiệm cận ngang là $y=\dfrac{1}{2}$}
    {Đồ thị hàm số có tiệm cận đứng là $x=2$}
    \loigiai{
        Đồ thị hàm số $y=\dfrac{x+1}{2x-2}$ có tiệm cận đứng $x=1$ và tiệm cận ngang $y=\dfrac{1}{2}$.
    }
\end{ex}


\begin{ex}%[2D1Y4-1]
    Cho hàm số $y=f(x)$ có $\lim\limits_{x\to -\infty} f(x)= -2$ và $\lim\limits_{x\to +\infty} f(x)= 2$. Khẳng định nào sau đây đúng?
    \choice
    {Đồ thị hàm số đã cho có đúng một tiệm cận ngang}
    {Đồ thị hàm số đã cho không có tiệm cận ngang}
    {Đồ thị hàm số đã cho có hai tiệm cận ngang là hai đường thẳng $x=-2$ và $x=2$}
    {\True Đồ thị hàm số đã cho có hai tiệm cận ngang là hai đường thẳng $y=-2$ và $y=2$}
    \loigiai{
        $\lim\limits_{x\to -\infty} f(x)= -2$ nên $y=-2$ là tiệm cận ngang.\\
        $\lim\limits_{x\to +\infty} f(x)= 2$ nên $y=2$ là tiệm cận ngang.
    }
\end{ex}

\begin{ex}%[2D1Y4-1]
    Cho hàm số $y= \dfrac{2017}{x-2}$ có đồ thị $(H)$. Số đường tiệm cận của $(H)$ là
    \choice
    {$0$}
    {\True $2$}
    {$3$}
    {$4$}
    \loigiai{
        Đồ thị $(H)$ có tiệm cận đứng là $x=2$, tiệm cận ngang là $y=0$.\\
        Vậy số đường tiệm cận của $(H)$ là $2$.
    }
\end{ex}

\begin{ex}
    Tìm số đường tiệm cận của đồ thị hàm số $ y = \dfrac{x^2 - 3x + 2}{x^2 - 4}. $
    \choice
    {$1$}
    {$ 0$}
    {\True $2$}
    {$3$}
    \loigiai
    {
        Tập xác định: $ \mathscr D = \mathbb{R} \backslash \{\pm2 \} $.\\
        Ta có $ \lim \limits_{x \to \pm  \infty} y = 1 \Rightarrow  $ đồ thị hàm số có 1 tiệm cận ngang là $ y = 1. $\\
        Ta lại có $\lim \limits_{x \to 2} y =  \lim \limits_{x \to 2} \dfrac{x-1}{x+2} = \dfrac{1}{4} $ và $\lim \limits_{x \to -2^+} y =  \lim \limits_{x \to -2^+} \dfrac{x-1}{x+2} = -\infty$ nên đồ thị hàm số có 1 tiệm cận đứng là $ x = -2. $\\
        Vậy đồ thị hàm số đã cho có 2 đường tiệm cận.
    }
\end{ex}


\begin{ex}%[Đề minh họa BGD 2018-2019]%[2D1B4-1]
    \immini[thm]{Cho hàm số $y=f(x)$ có bảng biến thiên như sau. Tổng số tiệm cận ngang và tiệm cận đứng của đồ thị hàm số đã cho là
        \haicot
        {$4$}
        {$1$}
        {\True $3$}
        {$2$}
    }{
        \begin{tikzpicture}
            \tikzset{double style/.append style = {draw=\tkzTabDefaultWritingColor,double=\tkzTabDefaultBackgroundColor,double distance=2pt}}
            \tkzTabInit[nocadre=false, lgt=1.2, espcl=2.5,deltacl=0.6
            ]{$x$ /0.6,$y'$ /0.6,$y$ /1.6}{$-\infty$,$1$,$+\infty$}
            \tkzTabLine{,+, d ,+,}
            \tkzTabVar{-/ $2$ / , +D-/ $+\infty$ / $3$ , +/ $5$ /}
    \end{tikzpicture}}
    \loigiai{
        Từ bảng biến thiên ta có
        \begin{itemize}
            \item $\lim\limits_{x \to -\infty} y =2$ suy ra $y=2$ là tiệm cận ngang.
            \item $\lim\limits_{x \to +\infty} y =5$ suy ra $y=5$ là tiệm cận ngang.
            \item $\lim\limits_{x \to 1^-} y = +\infty$ suy ra $x=1$ là tiệm cận đứng.
        \end{itemize}
        Vậy đồ thị hàm số tổng cộng có $3$ đường tiệm cận ngang và tiệm cận đứng.

    }
\end{ex}

\begin{ex}%[Đề tập huấn Sở Ninh Bình, 2019]%[Nguyễn Văn Hải, dự án(12EX-5-2019)]%[2D1B4-1]
    \immini[thm]{Cho hàm số $y=f(x)$ có bảng biến thiên như hình bên. Hỏi đồ thị hàm số $y=f(x)$ có tổng số bao nhiêu tiệm cận (tiệm cận đứng và tiệm cận ngang)?
        \haicot
        {$0$}
        {\True $2$}
        {$3$}
        {$1$}
    }{
        \begin{tikzpicture}
            \tikzset{double style/.append style = {draw=\tkzTabDefaultWritingColor,double=\tkzTabDefaultBackgroundColor,double distance=2pt}}
            \tkzTabInit[nocadre=false,lgt=1,espcl=2,deltacl=0.6]
            {$x$ /0.6,$y’$ /0.6,$y$ /2.2}
            {$-\infty$ , $1$ , $3$ , $+\infty$}
            \tkzTabLine{,+,d,+,0,-,}
            \tkzTabVar{-/$-1$ ,+D- / $+\infty$ /  $-\infty$,+/ $2$, -/$-\infty$}
    \end{tikzpicture}}
    \loigiai{
        Ta có $\lim\limits_{x\to -\infty}f(x)=-1$, $\lim\limits_{x\to +\infty}f(x)=-\infty$ nên $y=-1$ là tiệm cận ngang.\\
        Ta có $\lim\limits_{x\to 1^+}f(x)=-\infty$ nên $x=1$ là tiệm cận đứng.\\
        Vậy đồ thị hàm số có $2$ đường tiệm cận.
    }
\end{ex}


\begin{ex}%[GHK1, THCS - THPT Nguyễn Khuyến, HCM, 2019]%[Vinh Vo, 12Ex3-2019]%[2D1B4-1]
    \immini[thm]{Cho hàm số $ y = f(x) $ xác định trên $ (-2;0) \cup (0;+\infty) $ và có bảng biến thiên như hình vẽ. Số đường tiệm cận của đồ thị hàm số $ f(x) $ là
        \haicot
        {$ 4 $}
        {$ 2 $}
        {$ 1 $}
        {\True $ 3 $}
    }{
        \begin{tikzpicture}
            \tikzset{double style/.append style = {double distance=2pt}}
            \tkzTabInit[lgt=1.2,espcl=3.5,nocadre=false]
            {$x$ /0.6, $f’(x)$ /0.7,$f(x)$ /1.7}
            { $-2$ , $0$ , $+\infty$}
            \tkzTabLine{d,+,d,-, }
            \tkzTabVar{D-/ /$ -\infty $, +D+/$ +\infty $/ $ 1 $, -/ $ 0 $}
            \draw[pattern = north west lines] ($(N13)-(0.2ex,0)$) rectangle (T11);
    \end{tikzpicture}}
    \loigiai{
        Từ bảng biến thiên, ta thấy $ \heva{& \lim \limits_{ x \to -2^{+} } f(x) = - \infty \\ & \lim \limits_{x \to 0^{-} } f(x) = + \infty \\ & \lim \limits_{x \to + \infty} f(x) = 0  } $, suy ra đồ thị hàm số $ f(x) $ có $ 3 $ tiệm cận trong đó có $ 2 $ tiệm cận đứng và $ 1 $ tiệm cận ngang.
    }
\end{ex}

\begin{ex}%[Đề thi khảo sát chất lượng trường THCS-THPT Lômônôxốp, Hà Nội 2018 ,Nhật Thiện 12EX1-2019]%[2D1Y4-1]
    \immini[thm]{Cho hàm số $y=f(x)$ xác định trên $\mathbb{R}\backslash \left\{{0}\right\}$, liên tục trên mỗi khoảng xác định và có bảng biến thiên như hình bên. Hỏi đồ thị hàm số có bao nhiêu đường tiệm cận?
        \haicot
        {$1$}
        {\True $2$}
        {$3$}
        {$4$}}{\begin{tikzpicture}
            \tikzset{double style/.append style = {draw=\tkzTabDefaultWritingColor,double=\tkzTabDefaultBackgroundColor,double distance=2pt}}
            \tkzTabInit[nocadre=false,lgt=1,espcl=2.3,deltacl=0.6]{$x$ /0.6,$y'$ /0.6,$y$ /1.8}{$-\infty$,$0$,$1$,$+\infty$}
            \tkzTabLine{,+,d,-,0,+}
            \tkzTabVar{+/ $2$ ,-D-/ $-\infty$/$-\infty$, +/ $1$ ,-/ $-\infty$ /}
    \end{tikzpicture}}
    \loigiai{
        Dựa vào bảng biến thiên, ta có $$\lim\limits_{x\to -\infty}y=2;\qquad \lim\limits_{x\to 0^{\pm}}y=-\infty$$
        Vậy hàm số có một tiệm cận ngang $y=2$, một tiệm cận đứng $x=0$.
    }
\end{ex}

\begin{ex}
    Số tiệm cận đứng của đồ thị hàm số $y=\dfrac{\sqrt{x+9}-3}{x^2+x}$ là
    \choice
    {$3$}
    {$2$}
    {$0$}
    {\True $1$}
    \loigiai{
        Tập xác định $\mathscr{D}=[-9;+\infty)\setminus \{-1;0\}$. \\
        Ta có $\left\{\begin{aligned}
            &\lim\limits_{x\to -1^+} \dfrac{\sqrt{x+9}-3}{x^2+x}=+\infty \\
            &\lim\limits_{x\to -1^-} \dfrac{\sqrt{x+9}-3}{x^2+x}=-\infty
        \end{aligned}\right. \Rightarrow x=-1$ là tiệm cận đứng. \\
        Ngoài ra $\lim\limits_{x\to 0} \dfrac{\sqrt{x+9}-3}{x^2+x}=\dfrac{1}{6}$ nên $x=0$ không là tiệm cận.}
\end{ex}

\begin{ex}
    Phương trình đường tiệm cận xiên của đồ thị hàm số $y=\dfrac{2x^2-x+1}{x-1}$ là
    \choice
    {$y=x-1$}
    {\True $y=2x+1$}
    {$y=2x+3$}
    {$y=x+1$}
    \loigiai{
        Sau khi chia đa thức, ta viết lại hàm số $y=2x+1+\dfrac{2}{x-1}$.\\
        Do $\lim\limits_{x\to \pm \infty}\left[y-(2x+1)\right]=\lim\limits_{x\to \pm \infty}\dfrac{2}{x-1}=0$ nên $y=2x+1$ là đường tiệm cận xiên.}
\end{ex}

\begin{ex}
    Giao điểm của đường tiệm cận đứng và đường tiệm cận xiên của đồ thị hàm số $y=\dfrac{x^2-3x+5}{x-2}$ có tọa độ là
    \choice
    {$(2;3)$}
    {$(-2;1)$}
    {\True $(2;1)$}
    {$(-2;3)$}
    \loigiai{
        Sau khi chia đa thức, ta viết lại hàm số $y=x-1+\dfrac{3}{x-2}$.
        \begin{itemize}
            \item [$\bullet$] Đồ thị hàm số có tiệm cận đứng là $x=2$
            \item [$\bullet$] Do $\lim\limits_{x\to \pm \infty}\left[y-(x-1)\right]=\lim\limits_{x\to \pm \infty}\dfrac{3}{x-2}=0$ nên $y=x-1$ là đường tiệm cận xiên.
        \end{itemize}
        Giải hệ $\heva{&x=2\\&y=x-1} \Leftrightarrow \heva{&x=2\\&y=1}$. Suy ra, giao hai đường tiệm cận có tọa độ $(2;1)$.
    }
\end{ex}

\begin{ex}%[Đề thi giữa HK1, THPT Bình Sơn Đồng Nai, 2019]%[Phan Minh Tâm, dự án EX3]%[2D1B4-2]
    Tiệm cận đứng của đồ thị hàm số $ y=\dfrac{2x+1}{x-m} $ đi qua điểm $ M(2;5) $ khi $ m $ bằng bao nhiêu?
    \choice
    {$ m=-2 $}
    {$ m=-5 $}
    {$ m=5 $}
    {\True $ m=2 $}
    \loigiai{
        Với $m \ne -\dfrac{1}{2}$ đồ thị có tiệm cận đứng là đường thẳng $ x=m $. Tiệm cận đứng $ x=m $ đi qua $ M(2;5) $ khi chỉ khi $ m=2 $.
    }
\end{ex}

\begin{ex}%[GHK1, THPT Quế Võ 2-Bắc Ninh, 2019]%[TranTony,12EX2]%[2D1B4-2]
    Cho hàm số $ y = \dfrac{2x^2-3x+m}{x-m} $ có đồ thị $ (C) $. Tìm tất cả các giá trị của tham số $ m $ để $ (C) $ không có tiệm cận đứng.
    \choice
    {\True $ m = 0 $ hoặc $ m = 1 $}
    {$ m = 2 $}
    {$ m = 1 $}
    {$ m = 0 $}
    \loigiai{
        Đồ thị $ (C) $ không có tiệm cận đứng khi $ m $ là nghiệm của $ 2x^2-3x+m $
        \begin{align*}
            \Leftrightarrow 2m^2 - 3m + m = 0 \Leftrightarrow \hoac{& m = 0 \\& m = 1.}
        \end{align*}
    }
\end{ex}

\BTTF

\begin{ex}
    Cho hàm số $y=f(x)=\dfrac{3-2x}{x+1}$. Xét tính đúng sai của các khẳng định sau:
    \choiceTF
    {\True Tập xác định của hàm số là $\mathbb{R}\backslash\{-1\}$}
    {\True  Đồ thị hàm số có đường tiệm cận đứng là $x=-1$}
    {Đồ thị hàm số có đường tiệm cận ngang là $y=3$}
    {Hai đường tiệm cận (đứng và ngang) của đồ thị tạo với hai trục tọa độ một hình phẳng có diện tích bằng $3$}
    \loigiai{
        \begin{itemchoice}
            \itemch Điều kiện xác định $x+1 \ne 0 \Leftrightarrow x \ne -1$. Suy ra $D=\mathbb{R}\backslash\{-1\}$.
            \itemch Đồ thị hàm số có tiệm cận đứng là $x=-1$.
            \itemch Đồ thị hàm số có tiệm cận ngang là $y=\dfrac{-2}{1}=-2$.
            \itemch Hai đường tiệm cận (đứng và ngang) của đồ thị tạo với hai trục tọa độ một hình chữ nhật như hình vẽ
            \begin{center}
                \begin{tikzpicture}[smooth,samples=300,scale=0.8,>=stealth]
                    \draw[->] (-3,0)--(2,0) node[below]{$x$};
                    \draw[->] (0,-3)--(0,1) node[right]{$y$};
                    \draw (0,0) node[below right]{$O$};
                    \draw[pattern = north west lines] (0,0)--(0,-2)--(-1,-2)--(-1,0);
                    \draw (-3,-2)--(2,-2)node[above]{\scriptsize TCN $y=-2$} (-1,-3)--(-1,1)node[above]{\scriptsize TCĐ $x=-1$};
                    \draw[fill=black] (-1,0) circle(1.5pt) (-1,-2) circle(1pt) (0,-2) circle(1.5pt);
                    \node[right] at (0,-2.3) {$A$};
                    \node[left] at (-1,0.3) {$B$};
                \end{tikzpicture}
            \end{center}
            Diện tích hình chữ nhật này là
            $$S=OA \cdot OB=2 \cdot 1=2.$$
    \end{itemchoice}}
\end{ex}

\begin{ex}%[2D1K4]
    Cho hàm số $y=f(x)$ xác định trên $(-\infty;2) \backslash\{-2\}$ và có bảng biến thiên như hình vẽ dưới đây.
    \begin{center}
        \begin{tikzpicture}
            \tikzset{double style/.append style = {draw=\tkzTabDefaultWritingColor,double=\tkzTabDefaultBackgroundColor,double distance=2pt}}
            \tkzTabInit[nocadre=false,lgt=1,espcl=3]
            {$x$ /0.7,$y'$ /0.7,$y$ /2.1}
            {$-\infty$,$-2$,$0$,$2$,$+\infty$}
            \tkzTabLine{,+,d,-,0,+,d,}
            \tkzTabVar{-/$-2$/,+D+/ $3$ / $+\infty$,-/$-2$/,+D/ $+\infty$ / }
            \draw[pattern = north west lines] ($(N43)+(0.1ex,0)$) rectangle (T21);
        \end{tikzpicture}
    \end{center}
    Xét tính đúng sai của các khẳng định sau:
    \choiceTF
    {\True Hàm số có giá trị nhỏ nhất bằng $-2$}
    {Hàm số có giá trị lớn nhất bằng $3$}
    {\True Đồ thị hàm số có hai đường tiệm cận đứng là $x=-2$ và $x=2$}
    {Đồ thị hàm số có hai đường tiệm cận ngang là $y=-2$ và $y=3$}
    \loigiai
    {
        Căn cứ vào bảng biến thiên của hàm số, ta có
        \begin{itemchoice}
            \itemch Hàm số đạt giá trị nhỏ nhất bằng $-2$ khi $x=0$.
            \itemch Do $\lim\limits_{x\to 2^{-}}f(x)=+\infty$ nên hàm số không có giá trị lớn nhất.
            \itemch Do $\lim\limits_{x\to -2^{+}}f(x)=+\infty$ và $\lim\limits_{x\to 2^{-}}f(x)=+\infty$ nên đồ thị hàm số có hai đường tiệm cận đứng là $x=-2$ và $x=2$.
            \itemch Do $\lim\limits_{x\to -\infty}f(x)=-2$ nên đồ thị hàm số có hai đường tiệm cận ngang là $y=-2$.
    \end{itemchoice}}
\end{ex}

\begin{ex}
    Cho hàm số $y=f(x)=\dfrac{5 x^2+9 x+9}{x-4}$. Xét tính đúng sai của các khẳng định sau:
    \choiceTF
    {Tập xác định của hàm số là $\mathbb{R}\backslash\{-4\}$}
    {\True Đường tiệm cận đứng của đồ thị hàm số có phương trình là $x=4$}
    {\True Đường tiệm cận xiên của đồ thị hàm số có phương trình là $y=5x+29$}
    {Giao điểm hai đường tiệm cận của đồ thị hàm số có toạ độ là $(4;29)$}
    \loigiai{
        Hàm số được viết thành $y=5x+29+\dfrac{125}{x-4}$.
        \begin{itemchoice}
            \itemch Điều kiện $x-4 \ne 0 \Leftrightarrow x \ne 4$. Suy ra $D=\mathbb{R}\backslash\{4\}$.
            \itemch Đường tiệm cận đứng của đồ thị hàm số có phương trình là $x=4$
            \itemch Đường tiệm cận xiên của đồ thị hàm số có phương trình là $y=5x+29$
            \itemch Giải hệ $\heva{&x=4\\&y=5x+29}\Leftrightarrow \heva{&x=4\\&y=49}$. Giao điểm hai đường tiệm cận của đồ thị hàm số có toạ độ là $(4;49)$.
    \end{itemchoice}}
\end{ex}

\begin{ex}
    Cho hàm số $y=f(x)=\dfrac{x^2-4 x+7}{x-1}$. Xét tính đúng sai của các khẳng định sau:
    \choiceTF
    {\True Đường tiệm cận đứng của đồ thị hàm số có phương trình là $x=1$}
    {\True Đường tiệm cận xiên của đồ thị hàm số có phương trình là $y=x-3$}
    {\True Giao điểm hai đường tiệm cận của đồ thị hàm số có toạ độ là $(1 ;-2)$}
    {Diện tích tam giác tạo bởi đường tiệm cận xiên của đồ thị hàm số và hai trục toạ độ là $\dfrac{9}{4}$}
    \loigiai{
        Hàm số được viết thành $y=x-3+\dfrac{4}{x-1}$.
        \begin{itemchoice}
            \itemch Đường tiệm cận đứng của đồ thị hàm số có phương trình là $x=1$
            \itemch Đường tiệm cận xiên của đồ thị hàm số có phương trình là $y=x-3$
            \itemch Giải hệ $\heva{&x=1\\&y=x-3}\Leftrightarrow \heva{&x=1\\&y=-2}$. Giao điểm hai đường tiệm cận của đồ thị hàm số có toạ độ là $(1;-2)$.
            \itemch Giao của đường thẳng $d \colon y=x-3$ với các trục tọa độ lần lượt tại $A(0;-3)$ và $B(3;0)$.\\
            Diện tích tam giác $OAB$ là $S=\dfrac{1}{2} OA \cdot OB=\dfrac{9}{2}$.
    \end{itemchoice}}
\end{ex}

\BTTL

\begin{ex}%[2D1N4-1]Câu 1
    Cho hàm số $y=\dfrac{2x-1}{x+3}$. Gọi $x=m$ và $y=n$ lần lượt là đường tiệm cận đứng và tiệm cận ngang của đồ thị hàm số. Tính giá trị của biểu thức $P=\dfrac{2m-1}{n+3}$.\\
    \shortans[3]{$-1{,}4$}
    \loigiai{Ta có:\\
        $\bullet \underset{x \to -3}{\lim}\,y=\infty \Rightarrow x=-3$ là đường tiệm cận đứng.\\
        $\bullet \underset{x \to +\infty}{\lim}\,y=2$ và $\underset{x \to -\infty}{\lim}\,y=2 \Rightarrow y=2$ là đường tiệm cận ngang.\\
        Vậy $m=-3; n=2 \Rightarrow P=\dfrac{2\cdot(-3)-1}{2+3}=\dfrac{-7}{5}=-1{,}4$.}
\end{ex}

\begin{ex}%[2D1B4-3]
    Cho đồ thị $(C)\colon y=\dfrac{x-3}{x+2}$ có hai đường tiệm cận cắt nhau tại $I$. Với $O$ là gốc tọa độ, hãy tính độ dài đoạn thẳng $OI$ (làm tròn đến hàng phần trăm).\\
    \shortans[3]{$2{,}24$}
    \loigiai{
        Ta có tiệm cận đứng của đồ thị $(C)$ là $x=-2$ và tiệm cận ngang là $y=1$. Do đó $I(-2;1)$ là giao điểm của hai đường tiệm cận của đồ thị $(C)$.\\
        Ta có $OI=\sqrt{ (-2-0)^2+(1-0)^2}=\sqrt{5} \approx 2{,}24$.
    }
\end{ex}

\begin{ex}
    Nếu trong một ngày, một xưởng sản xuất được $x$ kilôgam sản phẩm thì chi phí trung bình (tính bằng nghìn đồng) cho một sản phẩm được cho bởi công thức:
    $$
    y=C(x)=\dfrac{50x+2000}{x}
    $$
    Đồ thị hàm số $C(x)$ có một đường tiệm cận ngang (khi $x \to +\infty$) là $y=y_0$. Giá trị $y_0$ bằng bao nhiêu?\\
    \shortans[3]{$50$}
    \loigiai{
        Ta có $\lim\limits_{x \rightarrow+\infty} \dfrac{50x+2000}{x}=\lim\limits_{x \rightarrow+\infty} \left(50+\dfrac{2000}{x}\right)=50$.\\
        Vậy đường thẳng $y=50$ là tiệm cận ngang của đồ thị hàm số.
    }
\end{ex}

\begin{ex}%[2D1C4-3]Câu 6
    Cho hàm số $y=\dfrac{x-2}{x^2-3mx+m}$ tìm $m$ để đồ thị hàm số có đúng một tiệm cận đứng. Biết tổng các giá trị của tham số $m$ có dạng phân số $\dfrac{a}{b}$, tính tổng $S=a+b$.\\
    \shortans[3]{$101$}
    \loigiai{Dễ thấy tử thức có một nghiệm là $x=2$ do đó để đồ thị hàm số có đúng một tiệm cận đứng thì phương trình $x^2-3mx+m=0$ có nghiệm kép hoặc có hai nghiệm phân biệt trong đó có một nghiệm bằng $2$.\\
        $\Rightarrow \hoac{&\Delta=0\\&\heva{&\Delta>0\\&4-3\cdot2m+m=0}} \Leftrightarrow \hoac{&9m^2-4m=0\\&\heva{&9m^2-4m>0\\&4-3\cdot2m+m=0}} \Leftrightarrow \hoac{&m=0\\&m=\dfrac{4}{9}\\&\heva{&\hoac{&m<0\\&m>\dfrac{4}{9}}\\&m=\dfrac{4}{5}}} \Leftrightarrow \hoac{&m=0\\&m=\dfrac{4}{9}\\&m=\dfrac{4}{5}}$\\
        Vậy tổng các giá trị của tham số $m$ bằng $\dfrac{56}{45} \Rightarrow S=101$.
    }
\end{ex}

\begin{ex}%[2D1H4-1]Câu 4
    Cho hàm số $y=\dfrac{x^2+2x-3}{x-2}$, đồ thị hàm số có đường tiệm cận xiên có dạng $(C) \colon y=ax+b$. Tính giá trị của biểu thức $P=\dfrac{a}{b}$.\\
    \shortans[3]{$0{,}25$}
    \loigiai{Ta xét $y=\dfrac{x^2+2x-3}{x-2}=x+4+\dfrac{5}{x-2} \Rightarrow (C) \colon y=x+4$ là đường tiệm cận xiên của đồ thị hàm số. Vậy $P=\dfrac{1}{4}$.}
\end{ex}

\begin{ex}
    Gọi $d$ là đường tiệm cận xiên của đồ thị hàm số $y=mx+4-3m+\dfrac{3}{x+2}$, $m$ là tham số. Đường thẳng $d$ luôn qua điểm cố định $M$. Tính độ dài đoạn $OM$, với $O$ là gốc tọa độ.\\
    \shortans[3]{$5$}
    \loigiai{
        Đường tiệm cần xiên của đồ thị là $y=mx+4-3m \Leftrightarrow (x-3)m+4-y=0$.\\
        Ta có $(x-3)m+4-y=0,\,\forall m \Leftrightarrow \heva{&x-3=0\\&4-y=0} \Leftrightarrow \heva{&x=3\\&y=4}$.\\
        Đường thẳng này luôn qua điểm cố định $M(3;4)$. Khi đó $OM=\sqrt{3^2+4^2}=5$.
    }
\end{ex}
%\boxde
\BTTN
\Opensolutionfile{ans}[ans/2D1-4-DEON-1]
\begin{ex}%[2D1B4-1]
    Cho hàm số $y=f(x)$ có bảng biến thiên như hình bên. Đồ thị hàm số đã cho có tiệm cận ngang là đường thẳng
    \begin{center}
        \begin{tikzpicture}[scale=0.8, font=\footnotesize, line join=round, line
            cap=round, >=stealth]
            \tkzTabInit[espcl=2.5,lgt=1,nocadre=false]
            {$x$/0.7,$f(x)$/2.1}
            {$-\infty$,$0$,$1$,$2$,$+\infty$}
            \tkzTabVar{-/$-\infty$,+/$2$,-D+/$-\infty$/$+\infty$,-/$4$,+/$6$}
        \end{tikzpicture}
    \end{center}
    \choice
    {$y=2$}
    {$y=1$}
    {\True $y=6$}
    {$y=4$}
    \loigiai{Dựa vào bảng biến thiên ta thấy đồ thị hàm có tiệm cận ngang $y=6$.
    }
\end{ex}
%56
\begin{ex}%[2D1B4-1]
    Cho hàm số $y=f(x)$ có bảng biến thiên như hình bên. Tổng số tiệm cận đứng và tiệm cận ngang của đồ thị hàm số đã cho là
    \begin{center}
        \begin{tikzpicture}[scale=0.8]
            \tkzTabInit[nocadre=false,lgt=1.5,espcl=3,deltacl=0.6]
            {$x$ /0.6,$y’$ /0.6,$y$ /2}
            {$-\infty$ , $1$, $+\infty$}
            \tkzTabLine{,+,d,+,}
            \tkzTabVar{-/$2$,+D-/$+\infty$/$3$,+/$5$}
        \end{tikzpicture}
    \end{center}
    \choice
    {$1$}
    {\True $3$}
    {$2$}
    {$4$}
    \loigiai{Dựa vào bảng biến thiên ta thấy đồ thị hàm số có tiệm cận đứng $x=1$ và tiệm cận ngang $y=2$ và $y=5$.}
\end{ex}
\begin{ex}%[2D1B4-1]
    Cho hàm số $y=f(x)$ có bảng biến thiên như hình bên. Tổng số tiệm cận đứng và tiệm cận ngang của đồ thị hàm số đã cho là
    \begin{center}
        \begin{tikzpicture}[scale=0.8]
            \tkzTabInit[nocadre=false,lgt=1.5,espcl=3,deltacl=0.6]
            {$x$ /0.6,$y’$ /0.6,$y$ /2}
            {$-\infty$ ,$0$, $1$, $+\infty$}
            \tkzTabLine{,+,0,-,d,-,}
            \tkzTabVar{-/$4$,+/$2$,-D+/$-1$/$+\infty$,-/$-3$}
        \end{tikzpicture}
    \end{center}
    \choice
    {$1$}
    {\True $3$}
    {$2$}
    {$4$}
    \loigiai{
        Dựa vào bảng biến thiên ta thấy đồ thị hàm số có tiệm cận đứng $x=1$, tiệm cận ngang $y=4$ và $y=-3$.
    }
\end{ex}
%61
\begin{ex}%[2D1B4-1]
    Cho hàm số $y=f(x)$ có bảng biến thiên như hình bên. Tổng số tiệm cận đứng và tiệm cận ngang của đồ thị hàm số đã cho là
    \begin{center}
        \begin{tikzpicture}[scale=0.8]
            \tkzTabInit[nocadre=false,lgt=1.5,espcl=3,deltacl=0.6]
            {$x$ /0.6,$y’$ /0.6,$y$ /2}
            {$-\infty$ ,$0$, $1$, $+\infty$}
            \tkzTabLine{,-,0,+,d,+,}
            \tkzTabVar{+/$5$,-/$-4$,+D-/$+\infty$/$-\infty$,+/$2$}
        \end{tikzpicture}
    \end{center}
    \choice
    {$1$}
    {\True $3$}
    {$2$}
    {$4$}
    \loigiai{Dựa vào bảng biến thiên ta thấy đồ thị hàm số có một tiệm cận đứng $x=1$, hai tiệm cận ngang $y=5$ và $y=2$.}
\end{ex}
\begin{ex}%[2D1B4-1]
    Đồ thị hàm số nào trong các hàm số dưới đây có tiệm cận đứng?
    \choice
    {\True $y=\dfrac{1}{\sqrt{x}}$}
    {$y=\dfrac{1}{x^2+x+1}$}
    {$y=\dfrac{1}{x^4+1}$}
    {$y=\dfrac{1}{x^2+1}$}
    \loigiai{
    }
\end{ex}
\begin{ex}%[2D1K4-1]
    Số tiệm cận đứng của đồ thị hàm số $y=\dfrac{\sqrt{x+4}-2}{x^2+x}$ là
    \choice
    {$3$}
    {$0$}
    {$2$}
    {\True $1$}
    \loigiai{
        Tập xác định hàm số $ \mathscr{D}=[-4;+\infty)\setminus\lbrace -1;0\rbrace
        $.\\
        Ta có $ \lim\limits_{x\to -1^{+}}y=+\infty $, $ \lim\limits_{x\to 0^{+}}y=1
        $ và $ \lim\limits_{x\to 0^{-}}y=1 $.\\
        Suy ra đồ thị hàm số chỉ có $ 1 $ tiệm cận đứng là $ x=-1 $.
    }
\end{ex}
\begin{ex}%[Nguyễn Văn Sang, dự án Tex hoá đề cương trường Marie Curie - Lần 6]%[2D1Y4-1]
    Đường thẳng nào dưới đây là tiệm cận ngang của đồ thị hàm số $y=\dfrac{3+2 x}{x+1}$?
    \choice
    {$y=3$}
    {$x=-1$}
    {\True $y=2$}
    {$x=2$}
    \loigiai{
        Tập xác định $\mathscr{D}=\mathbb{R}\setminus\left\lbrace -1\right\rbrace$.
        \begin{itemize}
            \item $\lim\limits_{x \to \pm\infty} y=\lim\limits_{x \to \pm\infty} \dfrac{3+2 x}{x+1}=2$ suy ra $y=2$ là tiệm cận ngang.
            \item $\heva{& \lim\limits_{x \to -1^+} \dfrac{3+2 x}{x+1}=+\infty \\ & \lim\limits_{x \to -1^-} \dfrac{3+2 x}{x+1}=-\infty}$ suy ra $x=-1$ là tiệm cận đứng.
        \end{itemize}
    }
\end{ex}
%%=====Câu 15
\begin{ex}%[2D1Y4-1]
    Giao điểm của tiệm cận đứng và tiệm cận ngang của đồ thị hàm số $y=\dfrac{3x-2}{1-x}$ là điểm
    \choice
    {$M(1;3)$}
    {$P(-3;1)$}
    {\True $Q(1;-3)$}
    {$N\left(\dfrac{2}{3};3\right)$}
    \loigiai{
        Tiệm cận đứng, tiệm cận ngang của đồ thị hàm số lần lượt là $x=1$ và $y=-3$. Giao điểm của $2$ tiệm cận là $Q(1;-3)$.
    }
\end{ex}
\begin{ex}%[2D1K4-2]%
    Nếu đồ thị hàm số $y=\dfrac{(m+1)x+2}{x-n+1}$ lần lượt nhận trục hoành và trục tung làm đường đường tiệm cận ngang và tiệm cận đứng thì $m+n$ bằng bao nhiêu?
    \choice
    {\True $m+n=0$}
    {$m+n=2$}
    {$m+n=-1$}
    {$m+n=1$}
    \loigiai{
        Theo đề bài, ta có $\heva{&m+1=0\\&n-1=0} \Leftrightarrow \heva{&m=-1\\&n=1.}$\\
        Suy ra $m+n=0$.
    }
\end{ex}
\begin{ex}%[2D1K4-1]
    Cho hàm số $y=f(x)$ có bảng biến thiên như hình bên. Đồ thị hàm số $y=\dfrac{x-2}{f(x)-1}$ có bao nhiêu tiệm cận đứng?
    \begin{center}
        \begin{tikzpicture}
            \tkzTabInit[espcl=3]{$x$ / 1 , $f’(x)$ / 1, $f(x)$ / 2}
            {$-\infty$, $-1$ , $5$, $+\infty$}%
            \tkzTabLine{,-,0,+,0,-,}%
            \tkzTabVar{+/ $+\infty$, - / $-1$, + / $3$,-/$-2$}%
            \tkzTabVal[draw]{2}{3}{0.4}{$2$}{$1$}
        \end{tikzpicture}
    \end{center}
    \choice
    {$1$}
    {$3$}
    {\True $2$}
    {$4$}
    \loigiai{
        Dựa vào bảng biến thiên suy ra
        $f(x)-1=0 \Leftrightarrow f(x) =1$, phương trình này có $2$ nghiệm phân biệt khác $2$ và một nghiệm $x=2$ nên đồ thị hàm số $y=\dfrac{x-2}{f(x)-1}$ có hai tiệm cận đứng.
    }
\end{ex}
%68
\begin{ex}%[2D1K4-1]
    Cho hàm số $y=f(x)$ có bảng biến thiên như hình bên. Đồ thị hàm số $y=\dfrac{1}{2f(x)+1}$ có bao nhiêu tiệm cận đứng?
    \begin{center}
        \begin{tikzpicture}[scale=0.8]
            \tkzTabInit[nocadre=false,lgt=1.5,espcl=3,deltacl=0.6]
            {$x$ /0.6,$y’$ /0.6,$y$ /2}
            {$-\infty$ ,$-2$, $2$, $+\infty$}
            \tkzTabLine{,+,0,-,0,+,}
            \tkzTabVar{-/$-\infty$,+/$3$,-/$0$,+/$+\infty$}
        \end{tikzpicture}
    \end{center}
    \choice
    {\True $1$}
    {$3$}
    {$2$}
    {$0$}
    \loigiai{
        Dựa vào bảng biến thiên suy ra
        $2f(x)+1=0 \Leftrightarrow f(x) =-\dfrac{1}{2}$, phương trình này có $1$ nghiệm nên đồ thị hàm số $y=\dfrac{1}{2f(x)+1}$ có một tiệm cận đứng.
    }
\end{ex}
\begin{ex}%[2D1K4-1]
    Cho hàm số $y=f(x)$ có bảng biến thiên như hình bên. Đồ thị hàm số $y=\dfrac{1}{2f(x)-1}$ có bao nhiêu tiệm cận ngang?
    \begin{center}
        \begin{tikzpicture}[scale=0.8]
            \tkzTabInit[nocadre=false,lgt=1.5,espcl=3,deltacl=0.6]
            {$x$ /0.6,$y’$ /0.6,$y$ /2}
            {$-\infty$, $2$, $+\infty$}
            \tkzTabLine{,-,0,+,}
            \tkzTabVar{+/$1$,-/$-3$,+/$1$}
        \end{tikzpicture}
    \end{center}
    \choice
    {$1$}
    {\True $3$}
    {$2$}
    {$0$}
    \loigiai{
        Dựa vào bảng biến thiên suy ra
        \begin{itemize}
            \item 	$\lim \limits_{x \to \pm \infty} f(x)=1 \Leftrightarrow \lim \limits_{x \to \pm \infty}\dfrac{1}{2f(x)-1} =1$ nên đồ thị hàm số đã cho có tiệm cận ngang là $y=1$.
            \item $2f(x)-1=0 \Leftrightarrow f(x)=\dfrac{1}{2}$, phương trình này có $2$ nghiệm phân biệt nên đồ thị hàm số đã cho có hai tiệm cận đứng.
        \end{itemize}
    }
\end{ex}
%80
\begin{ex}%[2D1K4-1]
    Cho hàm số $y=f(x)$ có bảng biến thiên như hình bên. Đồ thị hàm số $y=\dfrac{1}{f^2(x)+f(x)}$ có bao nhiêu tiệm cận đứng?
    \begin{center}
        \begin{tikzpicture}[scale=0.8]
            \tkzTabInit[nocadre=false,lgt=1.5,espcl=3,deltacl=0.6]
            {$x$ /0.6,$y’$ /0.6,$y$ /2}
            {$-\infty$ ,$-4$, $6$, $+\infty$}
            \tkzTabLine{,-,0,+,0,-,}
            \tkzTabVar{+/$+\infty$,-/$-2$,+/$5$,-/$-\infty$}
        \end{tikzpicture}
    \end{center}
    \choice
    {$4$}
    {$3$}
    {$2$}
    {\True $6$}
    \loigiai{
        Dựa vào bảng biến thiên suy ra $f^2(x)+f(x)=0 \Leftrightarrow \hoac{&f(x)=0\\&f(x)=-1}$, mỗi phương trình này có $3$ nghiệm phân biệt nên đồ thị hàm số đã cho có $6$ tiệm cận đứng.
    }
\end{ex}
\begin{ex}%[2D1K4-1]
    Cho hàm số $y=f(x)$ có bảng biến thiên như hình bên. Đồ thị hàm số $y=\dfrac{3}{f(x^2)+1}$ có bao nhiêu tiệm cận đứng?
    \begin{center}
        \begin{tikzpicture}[scale=0.8]
            \tkzTabInit[nocadre=false,lgt=1.5,espcl=3,deltacl=0.6]
            {$x$ /0.6,$y’$ /0.6,$y$ /2}
            {$-\infty$ ,$0$, $2$, $+\infty$}
            \tkzTabLine{,+,d,-,0,+,}
            \tkzTabVar{-/$-\infty$,+/$1$,-/$-2$,+/$+\infty$}
        \end{tikzpicture}
    \end{center}
    \choice
    {\True $4$}
    {$3$}
    {$6$}
    {$0$}
    \loigiai{
        Dựa vào bảng biến thiên suy ra
        $f(x^2)+1=0 \Leftrightarrow f(x^2) =-1$. Kẻ đường thẳng $y=-1$ ta thấy đường thẳng cắt đồ thị hàm số tại 3 điểm phân biệt. Suy ra
        $$\hoac{&x^2=a \; (a<0)\\&x^2=b \; (b \in (0;2)\\&x^2=c \; (c>2)} \Rightarrow \hoac{&x=\pm \sqrt{b}\\&x=\pm \sqrt{c}.}$$
        Do đó đồ thị hàm số $y=\dfrac{2}{f(x^2)+1}$ có $4$ tiệm cận đứng.
    }
\end{ex}
\BTTF
\begin{ex}%[EX-TF-2024, Lê Đạt]%[2D1N4-1]
    Cho hàm số $y=\dfrac{2x-3}{x-1}$. Xét tính đúng sai các khẳng định dưới đây
    \choiceTF
    {\True Đường tiệm cận đứng của đồ thị hàm số là $ x=1 $}
    {Đường tiệm cận đứng của đồ thị hàm số là $ y=2 $}
    {Đường tiệm cận ngang của đồ thị hàm số là $ x=1 $}
    {\True Đường tiệm cận ngang của đồ thj hàm số là $ y=2 $}
    \loigiai{
        Ta có $\lim\limits_{x\to -\infty}y=\lim\limits_{x\to +\infty}y=2$ nên đồ thị hàm số đã cho có tiệm cận ngang là $y=2$.\\
        Ta có $\lim\limits_{x\to 1^+}y=-\infty$ nên đồ thị hàm số đã cho có tiệm cận ngang là $ x=1 $.
        \begin{itemchoice}
            \itemch Đường tiệm cận đứng của đồ thị hàm số là $ x=1 $.
            \itemch Đường tiệm cận đứng của đồ thị hàm số là $ x=1 $.
            \itemch Đường tiệm cận ngang của đồ thj hàm số là $ y=2 $.
            \itemch Đường tiệm cận ngang của đồ thj hàm số là $ y=2 $.
        \end{itemchoice}
    }
\end{ex}
%===== DẠNG 2
\begin{ex}%[EX-TF-2024, Lê Đạt]%[2D1H4-2]
    Cho hàm số $ y=\dfrac{m^2x+1}{x-1} $. Xét tính đúng sai của các khẳng định sau
    \choiceTF
    {\True Đồ thị hàm số luôn có tiệm cận ngang}
    {\True Đồ thị hàm số luôn có tiệm cận đứng}
    {\True Khi $ m=1$ đồ thị hàm số có $ 2 $ đường tiệm cận}
    {Khi $ m=0 $ đồ thị hàm số có $ 1 $ đường tiệm cận}
    \loigiai{
        \begin{itemchoice}
            \itemch $\lim\limits_{x\to -\infty}y=\lim\limits_{x\to +\infty}y=m^2$ suy ra hàm số luôn có tiệm cận ngang.
            \itemch $\lim\limits_{x\to 1^+}y=+\infty$ nên đồ thị hàm số đã cho có tiệm cận ngang là $ x=1 $.
            \itemch Khi $ m=1 $ ta được hàm số $ y=\dfrac{x+1}{x-1} $ suy ra đồ thì hàm số có $ x=1 $ là tiệm cận đứng và $ y=1 $ là tiệm cận ngang nên đồ thị hàm số có $ 2 $ tiệm cận.
            \itemch Khi $ m=0 $ ta được hàm số $ y=\dfrac{1}{x-1} $ suy ra đồ thì hàm số có $ x=1 $ là tiệm cận đứng và $ y=0 $ là tiệm cận ngang nên đồ thị hàm số có $ 2 $ tiệm cận.
        \end{itemchoice}
    }
\end{ex}
%===== DẠNG 3
\begin{ex}%[EX-TF-2024, Lê Đạt]%[2D1N4-3]
    \immini{Cho hàm số $y=f(x)$ có đồ thị như hình bên. Xét tính đúng sai của các khẳng định sau
        \choiceTF
        {$ x=2 $ là đường tiệm cận ngang của đồ thị hàm số}
        {\True $ x=-1 $ là đường tiệm cận đứng của đồ thị hàm số}
        {\True Đồ thị hàm số có hai đường tiệm cận}
        {\True Đồ thị hàm số không có tiệm cận xiên}
    }{
        \begin{tikzpicture}[scale=0.5, font=\footnotesize, line join=round, line cap=round, >=stealth]
            \draw[->](-5,0)--(5,0)node[below]{ $x$};
            \draw[->](0,-4)--(0,5)node[right]{ $y$};
            \draw [fill=black,draw=black] (0,0) circle (1pt)node[above left] { $O$};
            \foreach \x in {-1}\draw[shift={(\x,0)}](0pt,-2pt)--(0pt,2pt) node[below left]{ $\x$};
            \foreach \y in {2}\draw[shift={(0,\y)}](-2pt,0pt)--(2pt,0pt)node[above right]{ $\y$};
            \clip(-5,-4) rectangle (5,5);
            \draw[smooth,samples=100,domain=-5:-1.1] plot(\x,{(2*(\x)-1)/((\x)+1)});
            \draw[smooth,samples=100,domain=-0.9:5] plot(\x,{(2*(\x)-1)/((\x)+1)});
            \draw[dashed](-5,2)--(5,2) (-1,-4)--(-1,5);
        \end{tikzpicture}
    }
    \loigiai{
        \begin{itemchoice}
            \itemch $ y=2 $ là đường tiệm cận ngang của đồ thị hàm số.
            \itemch $ x=-1 $ là đường tiệm cận đứng của đồ thị hàm số.
            \itemch $ x=-1 $ là đường tiệm cận đứng và $ y=2 $ là đường tiệm cận ngang của đồ thị hàm số suy ra đồ thị hàm số có hai đường tiệm cận.
            \itemch Đồ thị hàm số không có tiệm cận xiên.
        \end{itemchoice}
    }
\end{ex}
\BTTL
\begin{ex}%[2D1K4-2]%
    Đường tiệm cận đứng và tiệm cận ngang của đồ thị hàm số $y=\dfrac{mx+1}{2m+1-x}$ cùng với hai trục tọa độ tạo thành một hình chữ nhật có diện tích bằng $3$. Khi đó $m$ bằng
    \shortans{$1$ hay $-\dfrac{3}{2}$}
    % \choice
    % {$1$ hay $\dfrac{3}{2}$}
    % {$-1$ hay $-\dfrac{3}{2}$}
    % {\True $1$ hay $-\dfrac{3}{2}$}
    % {$-1$ hay $3$}
    \loigiai{
        Từ yêu cầu đề bài, suy ra $|-m| \cdot |2m+1|=3 \Leftrightarrow \hoac{&m=1\\&m=-\dfrac{3}{2}.}$
    }
\end{ex}
\begin{ex}%[2D1K4-2]%
    Tìm tất cả các giá trị thực $m$ sao cho đồ thị hàm số $y=\dfrac{5x-3}{x^2-2mx+1}$ không có tiệm cận đứng.
    \shortans{$-1<m<1$}
    % \choice
    % {\True $-1<m<1$}
    % {$m=1$}
    % {$m=-1$}
    % {$m <-1$ hoặc $m>1$}
    \loigiai{
        Xét $f(x)=5x-3$, có $f(x)=0\Leftrightarrow x=\dfrac{3}{5}$; $g(x)=x^2-2mx+1$ có $\Delta’=m^2-1$.\\
        Đồ thị hàm số không có tiệm cận đứng khi phương trình $g(x)=0$ vô nghiệm $\Leftrightarrow m^2-1<0\Leftrightarrow-1<m<1$.\\
        Vậy với $-1<m<1$ thì đồ thị hàm số đã cho không có tiệm cận đứng.}
\end{ex}
\begin{ex}%[Nguyễn Văn Sang, dự án VDC-Hàm số 2020 - Lần 2]%[2D1K4-2]%
    Cho hàm số $y=\dfrac{1+\sqrt{x+1}}{\sqrt{x^2-mx-3m}}$ với $m$ là tham số. Tìm tập hợp các giá trị của tham số $m$ để đồ thị hàm số có hai tiệm cận đứng.
    \shortans{$\left(0;\dfrac{1}{2}\right)$}
    % \choice
    % {\True $\left(0;\dfrac{1}{2}\right)$}
    % {$\left(\left. 0;\dfrac{1}{2}\right]\right.$}
    % {$\left(0;+\infty \right)$}
    % {$\left(-\infty;-12\right)\cup \left(0;+\infty \right)$}
    \loigiai{
        Ta có $\sqrt{x+1}$ xác định khi $x\ge-1.$\\
        Yêu cầu bài toán $\Leftrightarrow $ phương trình $x^2-mx-3m=0$ có hai nghiệm phân biệt $x_1$, $x_2$ thỏa mãn $$-1<x_1<x_2\Leftrightarrow \heva{
            & \Delta >0 \\
            & x_1+x_2>-2 \\
            & a\cdot f\left(-1\right)>0 \\}\Leftrightarrow \heva{
            & m^2+12m>0 \\
            & m>-2 \\
            & 1\cdot \left(1-2m\right)>0 \\}\Leftrightarrow 0<m<\dfrac{1}{2}.$$
    }
\end{ex}
\begin{ex}%[VDC5-NgocDungHo]%[2D1G4-3]%
    \immini{Cho hàm số $f(x)$ có đồ thị như hình bên. Số đường tiệm cận đứng của đồ thị hàm số $y=\dfrac{(x^2-4)(x^2+2x)}{[f(x)]^2+2f(x)-3}$ là bao nhiêu?
        \shortans{$4$}
        % \choice
        % {\True $4$}
        % {$5$}
        % {$3$}
        % {$2$}
    }{\begin{tikzpicture}[>=stealth,scale=0.5, line join=round, line cap=round]
            \def\f[#1]{0.25*((#1)^4-8*(#1)^2+4)}
            \draw[->] (-4.1,0)--(4,0) node [below]{$x$};
            \draw[->] (0,-3.5)--(0,4) node [left]{$y$};
            \node at (0,0) [below left]{$O$};
            % \clip;
            \draw[domain=-3:3,samples=300,thick] plot (\x,{\f[\x]});
            \foreach \x in {-2,2} \filldraw (\x,0) node[above]{\x} circle (2pt);
            \foreach \x in {-3,3} \filldraw (\x,0) node[below]{\x} circle (2pt);
            \filldraw (0,1) node[above left]{$1$} circle (2pt);
            \filldraw (0,-3) node[below left]{$-3$} circle (2pt);
            \draw[dashed](-2,0)--(-2,-3)--(2,-3)--(2,0);
            \draw (3,2.75) node[right]{$y=f(x)$};
    \end{tikzpicture}}
    \loigiai{%GV tổng quát hóa bài toán:
        Cho hàm số $f(x)$ có đồ thị $(C)$ cho trước. Xác định số đường tiệm cận đứng của đồ thị hàm số $y=\dfrac{u(x)}{v[f(x)]}$.
        \begin{enumerate}
            \item Tìm tập xác định của hàm số $y=\dfrac{u(x)}{v[f(x)]}$.\\
            \item Tìm nghiệm của phương trình $u(x)=0\quad (1)$.\\
            \item Tìm nghiệm của phương trình $v[f(x)]=0\quad (2)$. Giả sử $f(x)=m_1$, $f(x)=m_2,\ldots$.
        \end{enumerate}
        Dựa vào đồ thị $(C)$, xác định hoành độ giao điểm của $(C)$ với các đường thẳng $d_1\colon f(x)=m_1$, $d_2\colon f(x)=m_2,\ldots$.\\
        Số đường tiệm cận đứng của đồ thị hàm số $y=\dfrac{u(x)}{v[f(x)]}$ chính là tổng của:
        \begin{itemize}
            \item Số nghiệm riêng của phương trình $(2)$.
            \item Số nghiệm chung $x=x_0$ của $(1) $ và $(2)$ mà bậc của $(x-x_0)$ ở mẫu lớn hơn bậc của $(x-x_0)$ ở tử.
        \end{itemize}
        \noindent
        Xét hàm số $y=g(x)=\dfrac{(x^2-4)(x^2+2x)}{[f(x)]^2+2f(x)-3}$.
        \immini
        {
            Giải phương trình $(x^2-4)(x^2+2x)=0\,(1)$\\$ \Leftrightarrow \hoac{& x^2-4=0 \\ & x^2+2x=0}\Leftrightarrow \hoac{& x=\pm 2 \\ & x=0.}$\\
            Giải phương trình $[f(x)]^2+2f(x)-3=0\,(2)$\\
            $ \Leftrightarrow \hoac{& f(x)=1 \\ & f(x)=-3.}$\\
        }
        {\begin{tikzpicture}[>=stealth,scale=0.7, line join=round, line cap=round]
                \def\f[#1]{0.25*((#1)^4-8*(#1)^2+4)}
                \def\g[#1]{1}
                \def\h[#1]{-3}
                \draw[->] (-4.1,0)--(4,0) node [below]{$x$};
                \draw[->] (0,-3.5)--(0,4) node [left]{$y$};
                \node at (0,0) [below left]{$O$};
                % \clip;
                \draw[domain=-3:3,samples=300,thick] plot (\x,{\f[\x]});
                \draw[domain=-4:4,samples=300,thick] plot (\x,{\g[\x]});
                \draw[domain=-4:4,samples=300,thick] plot (\x,{\h[\x]});
                \foreach \x in {-2,2} \filldraw (\x,0) node[above]{\x} circle (2pt);
                \filldraw (-3,0) node[above left]{$-3$} circle (2pt);
                \filldraw (3,0) node[above right]{$3$} circle (2pt);
                \filldraw (-2.85,0) node[below]{$a$} circle (2pt);
                \filldraw (2.85,0) node[below]{$b$} circle (2pt);
                \filldraw (0,1) node[above left]{$1$} circle (2pt);
                \filldraw (0,-3) node[below left]{$-3$} circle (2pt);
                \draw[dashed](-2,0)--(-2,-3)--(2,-3)--(2,0) (-2.85,0)--(-2.85,1) (2.85,0)--(2.85,1);
                \draw (3,2.75) node[right]{$(C):y=f(x)$};
                \draw (4.2,1) node[above]{$d_1:y=1$};
                \draw (4,-3) node[below]{$d_2:y=-3$};
            \end{tikzpicture}
        }
        Dựa vào đồ thị đã cho $(2)\Leftrightarrow \hoac{& x = \pm 2 \\ & x=0\\&x=a\\&x=b.}$
        với $-3<a<-2<2<b<3$.\\
        Trong điều kiện xác định của hàm số $y=g(x)$ ta có thể viết $$y=g(x)=\dfrac{x(x-2)(x+2)^2}{x^2(x-a)(x-b)(x-2)^2(x+2)^2}=\dfrac{1}{x(x-a)(x-b)(x-2)}$$
        Vậy số tiệm cận đứng của đồ thị hàm số $y=g(x)$ bằng $4$.
    }
\end{ex}
\begin{ex}
    \immini{%Câu 97.
        Đường cong ở hình bên là đồ thị của hàm số $y = ax^3 +bx^2 +cx+d$. Đồ thị hàm số $y =\dfrac{(x+1)\sqrt{1-x}}{f(x^2)}$ có tất cả bao nhiêu tiệm cận đứng?
        \shortans{$2$}
        % \choice
        % {1}
        % {6}
        % {4}
        % {\True 2}
    }{\begin{tikzpicture}[scale=.6, font=\footnotesize, line join=round, line cap=round, >=stealth]
            \def\xmin{-2}\def\xmax{4}\def\ymin{-3}\def\ymax{3}
            \draw[->] (\xmin-0.2,0)--(\xmax+0.2,0) node[below] {\footnotesize $x$};
            \draw[->] (0,\ymin-0.2)--(0,\ymax+0.2) node[right] {\footnotesize $y$};
            \draw (0,0) node [below left] {\footnotesize $O$};
            \foreach \x in {1,2}\draw (\x,-0.1)--(\x,0.1) node [above ] {\footnotesize $\x$};
            \foreach \x in {-1,3}\draw (\x,-0.1)--(\x,0.1) node [above left] {\footnotesize $\x$};
            \foreach \y in {-2}\draw (0.1,\y)--(-0.1,\y) node [left] {\footnotesize $\y$};
            \foreach \y in {2}\draw (-0.1,\y)--(0.1,\y) node [right] {\footnotesize $\y$};
            \clip (\xmin,\ymin) rectangle (\xmax,\ymax);
            \draw[smooth,samples=200,domain=\xmin:\xmax] plot (\x,{0.6666666666666666*((\x)^3)+-2*((\x)^2)+-0.6666666666666666*(\x)+2});
            \draw[dashed] (1.0,0)--(1.0,0.0)--(0,0.0);\fill (1.0,0.0) circle (1pt);
            \draw[dashed] (2,0)--(2,-2)--(0,-2);
    \end{tikzpicture}}
    \loigiai{
        * Điều kiện: $\heva{&f(x^2) \ne 0\\&x \le 1.}$\\
        Nhìn hình vẽ ta thấy
        $f(x^2)=0\Leftrightarrow \hoac{&x^2=-1\\&x^2=1\\&x^2=3}\Leftrightarrow \hoac{&x=\pm 1\,(\text{nghiệm đơn})\\&x=- \sqrt{3}\,(\text{nghiệm đơn})\\&x= \sqrt{3}\,(\text{không thỏa mãn})}.$\\
        Vậy $y=\dfrac{(x+1)\sqrt{1-x}}{f(x^2)}=\dfrac{(x+1)\sqrt{1-x}}{(x - 1)(x + 1)(x + \sqrt{3})}$ \\
        Đồ thị hàm số có 2 đường tiệm cận đứng.}
\end{ex}
\begin{ex}
    \immini{ %Câu 95.
        Đường cong ở hình bên là đồ thị của hàm số $y = ax^3 +bx^2 +cx+d$. Đồ thị hàm số $y =\dfrac{(2x+1)\sqrt{x-1}}{f(|x|)}$ có tất cả bao nhiêu tiệm cận đứng?
        \shortans{$1$}
        % \choice
        % {\True 1}
        % {3}
        % {4}
        % {2}
    }{\begin{tikzpicture}[scale=.5, font=\footnotesize, line join=round, line cap=round, >=stealth]
            \def\xmin{-3}\def\xmax{3}\def\ymin{-5}\def\ymax{5}
            \draw[->] (\xmin-0.2,0)--(\xmax+0.2,0) node[below] {\footnotesize $x$};
            \draw[->] (0,\ymin-0.2)--(0,\ymax+0.2) node[right] {\footnotesize $y$};
            \draw (0,0) node [below left] {\footnotesize $O$};
            \foreach \x in {-1,2}\draw (\x,0.1)--(\x,-0.1) node [below] {\footnotesize $\x$};
            \foreach \x in {-2,1}\draw (\x,-0.1)--(\x,0.1) node [above] {\footnotesize $\x$};
            \foreach \y in {-4,2}\draw (-0.1,\y)--(0.1,\y) node [right] {\footnotesize $\y$};
            \foreach \y in {-2,4}\draw (0.1,\y)--(-0.1,\y) node [left] {\footnotesize $\y$};
            \clip (\xmin,\ymin) rectangle (\xmax,\ymax);
            \draw[smooth,samples=200,domain=\xmin:\xmax] plot (\x,{1.3333333333333333*((\x)^3)+0*((\x)^2)+-3.3333333333333335*(\x)+0});
            \draw[dashed] (-2,0)--(-2,-4)--(0,-4);
            \draw[dashed] (2,0)--(2,4)--(0,4);
            \draw[dashed] (1,0)--(1,-2)--(0,-2);
            \draw[dashed] (-1,0)--(-1,2)--(0,2);
    \end{tikzpicture}}
    \loigiai{
        * Điều kiện: $\heva{&f(|x|) \ne 0\\&x \ge 1.}$\\
        Nhìn hình vẽ ta thấy
        $f(|x|)=0\Leftrightarrow \hoac{&|x|=x_1\,(-2<x_1<-1)\\&|x|=0\\&|x|=x_2\,(1<x_2<2)}\Leftrightarrow \hoac{&x=0&(\text{không thỏa mãn})\\&x=- x_2&(\text{không thỏa mãn})\\&x=x_2&(\text{nghiệm đơn}).}$\\
        Vậy $y =\dfrac{(2x+1)\sqrt{x-1}}{f(|x|)}=\dfrac{(2x+1)\sqrt{x-1}}{ax(x+x_2)(x-x_2)}.$ \\
        Đồ thị hàm số có 1 đường tiệm cận đứng.}
\end{ex}
\begin{ex}
    Đáp ứng tần số của một hệ thống điều khiển có thể được mô tả bởi hàm truyền \( H(s) = \dfrac{\omega_n^2}{s^2 + 2\zeta\omega_ns + \omega_n^2} \), trong đó \( \omega_n \) là tần số tự nhiên và \( \zeta \) là hệ số tắt dần. Tìm đường tiệm cận ngang của đáp ứng tần số khi tần số góc \( s \) tăng và nêu ý nghĩa của nó.
    \shortans{$y=0$}
    \loigiai{
        Khi \( s \) tăng vô hạn, các thành phần bậc cao trong mẫu số chiếm ưu thế:
        \[
        H(s) \approx \frac{\omega_n^2}{s^2}
        \]
        Đường tiệm cận ngang của \( H(s) \) khi \( s \to \infty \) là:
        \[
        |H(s)| \approx \frac{\omega_n^2}{s^2} \to 0
        \]}
\end{ex}
\begin{ex}
    Trong thuyết tương đối của Einstein, khối lượng của vật chuyển động với vận tốc $v$ được cho bởi công thức:
    $$m(v)=\dfrac{m_0}{\sqrt{1-\dfrac{v^2}{c^2}}},$$
    trong đó $m_0$ là khối lượng của vật khi nó đứng yên, $c$ là vận tốc ánh sáng.\\
    (nguồn: https://www.britannica.com/science/relativity/Relativistic-mass)\\
    Xem $m$ là hàm số theo vận tốc $v$, tìm đường tiệm cận đứng của đồ thị hàm số. Từ đó nhận xét khối lượng của vật khi vận tốc của nó càng gần với vận tốc ánh sáng.
    \shortans{$v=c$, khối lượng tăng lên vô hạn}
    \loigiai{
        Điều kiện xác định: $\heva{&1-\dfrac{v^2}{c^2}>0\\
            &v>0}\Leftrightarrow\heva{& -c<v<c\\
            &v>0}\Leftrightarrow 0<v<c$.\\
        Ta có $\lim\limits_{v\to c^{-}} m(v)=\lim\limits_{v\to c^{-}}\dfrac{m_0}{\sqrt{1-\dfrac{v^2}{c^2}}}=+\infty$ nên đường thẳng $v=c$ là tiệm cận đứng của đồ thị hàm số.\\
        Từ đó ta suy ra khi vận tốc của vật càng sát với vận tốc ánh sáng thì khối lượng của vật tăng lên vô hạn.
    }
\end{ex}
\Closesolutionfile{ans}
%\boxde
\BTTN
\Opensolutionfile{ans}[ans/2D1-4-DEON-2]
\begin{ex}%[2D1B4-1]
    Cho hàm số $y=f(x)$ có bảng biến thiên như hình bên. Tổng số tiệm cận đứng và tiệm cận ngang của đồ thị hàm số đã cho là
    \begin{center}
        \begin{tikzpicture}
            \tkzTabInit[nocadre=false,lgt=1.5,espcl=3,deltacl=0.6]
            {$x$ /0.6,$y’$ /0.6,$y$ /2}
            {$-\infty$ ,$0$, $1$, $+\infty$}
            \tkzTabLine{,-,d,+,0,-,}
            \tkzTabVar{+/$+\infty$,-D-/$-1$/$-\infty$,+/$2$,-/$-\infty$}
        \end{tikzpicture}
    \end{center}
    \choice
    {\True $1$}
    {$3$}
    {$2$}
    {$4$}
    \loigiai{
        Dựa vào bảng biến thiên ta thấy đồ thị hàm số có một tiệm cận đứng $x=0$.
    }
\end{ex}
%62
%63
\begin{ex}%[2D1B4-1]
    Cho hàm số $y=f(x)$ xác định, liên tục trên $\mathbb{R} \backslash \{0;1\}$ và có bảng biến thiên như hình bên. Đồ thị hàm số $y=f(x)$ có
    \begin{center}
        \begin{tikzpicture}
            \tkzTabInit[nocadre=false,lgt=1.5,espcl=3,deltacl=0.6]
            {$x$ /0.6,$y’$ /0.6,$y$ /2}
            {$-\infty$ ,$0$, $1$, $+\infty$}
            \tkzTabLine{,+,d,+,d,+,}
            \tkzTabVar{-/$-5$,+D-/$+\infty$/$-\infty$,+D-/$3$/$-\infty$,+/$+\infty$}
        \end{tikzpicture}
    \end{center}
    \choice
    {\True $2$ tiệm cận đứng và $1$ tiệm cận ngang}
    {$2$ tiệm cận đứng và $2$ tiệm cận ngang}
    {$1$ tiệm cận đứng và $1$ tiệm cận ngang}
    {$1$ tiệm cận đứng và $2$ tiệm cận ngang}
    \loigiai{
        Dựa vào bảng biến thiên ta thấy đồ thị hàm số có hai tiệm cận đứng $x=0$ và $x=1$; một tiệm cận ngang $y=-5$.
    }
\end{ex}
%64
\begin{ex}%[2D1B4-1]
    Cho hàm số $y=f(x)$ có bảng biến thiên như hình bên. Tổng số tiệm cận đứng và tiệm cận ngang của đồ thị hàm số đã cho là
    \begin{center}
        \begin{tikzpicture}[scale=0.8]
            \tkzTabInit[nocadre=false,lgt=1.5,espcl=3,deltacl=0.6]
            {$x$ /0.6,$y’$ /0.6,$y$ /2}
            {$-\infty$ ,$-1$, $1$, $+\infty$}
            \tkzTabLine{,+,d,+,0,-,}
            \tkzTabVar{-/$2$,+D-/$4$/$-\infty$,+/$3$,-/$-1$}
        \end{tikzpicture}
    \end{center}
    \choice
    {$1$}
    {\True $3$}
    {$2$}
    {$4$}
    \loigiai{Dựa vào bảng biến thiên ta thấy đồ thị hàm số có một tiệm cận đứng $x=-1$; hai tiệm cận ngang $y=-1$ và $y=2$.
    }
\end{ex}
%65
\begin{ex}%[2D1B4-1]
    Cho hàm số $y=f(x)$ có bảng biến thiên như hình bên. Tổng số tiệm cận đứng và tiệm cận ngang của đồ thị hàm số đã cho là
    \begin{center}
        \begin{tikzpicture}[scale=0.8]
            \tkzTabInit[nocadre=false,lgt=1.5,espcl=3,deltacl=0.6]
            {$x$ /0.6,$y’$ /0.6,$y$ /2}
            {$-\infty$ ,$0$, $3$, $+\infty$}
            \tkzTabLine{,-,0,+,d,-,}
            \tkzTabVar{+/$8$,-/$1$,+/$4$,-/$2$}
        \end{tikzpicture}
    \end{center}
    \choice
    {$1$}
    {$3$}
    {\True $2$}
    {$4$}
    \loigiai{
        Dựa vào bảng biến thiên ta thấy đồ thị hàm số có hai tiệm cận ngang $y=2$ và $y=8$.
    }
\end{ex}
\begin{ex}%[Nguyễn Văn Sang, dự án Tex hoá đề cương trường Marie Curie - Lần 6]%[2D1Y4-1]
    Đường thẳng nào dưới đây là tiệm cận đứng của đồ thị hàm số $y=\dfrac{2 x+1}{x+1}$?
    \choice
    {$x=1$}
    {$y=-1$}
    {$y=2$}
    {\True $x=-1$}
    \loigiai{
        Tập xác định $\mathscr{D}=\mathbb{R}\setminus\left\lbrace -1\right\rbrace$.
        \begin{itemize}
            \item $\lim\limits_{x \to \pm\infty} y=\lim\limits_{x \to \pm\infty} \dfrac{2x+1}{x+1}=2$ suy ra $y=2$ là tiệm cận ngang.
            \item $\heva{& \lim\limits_{x \to -1^+} \dfrac{2x+1}{x+1}=-\infty \\ & \lim\limits_{x \to -1^-} \dfrac{2x+1}{x+1}=+\infty}$ suy ra $x=-1$ là tiệm cận đứng.
        \end{itemize}
    }
\end{ex}
\begin{ex}%[2D1Y4-1]
    Đồ thị hàm số $y=\dfrac{2x-3}{2x+1}$ có tâm đối xứng là điểm
    \choice
    {\True $M\left(-\dfrac{1}{2};1\right)$}
    {$P\left(-\dfrac{1}{2};2\right)$}
    {$Q\left(-\dfrac{1}{2};-3\right)$}
    {$N\left(1;-\dfrac{1}{2}\right)$}
    \loigiai{
        Tiệm cận đứng, tiệm cận ngang của đồ thị hàm số lần lượt là $x=-\dfrac{1}{2}$ và $y=3$. Tâm đối xứng là điểm $M\left(-\dfrac{1}{2};1\right)$.
    }
\end{ex}
\begin{ex}%[2D1K4-1]
    Đồ thị hàm số $y=\dfrac{\sqrt{x}}{x+1}-\dfrac{1}{x}$ có tất cả bao nhiêu tiệm cận đứng và ngang?
    \choice
    {$0$}
    {$3$}
    {\True $2$}
    {$1$}
    \loigiai{
        Tập xác định $\mathscr{D}=(0;+\infty)$.
        \begin{itemize}
            \item $\lim\limits_{x\to 0^+} \left(\dfrac{\sqrt{x}}{x+1}-\dfrac{1}{x}\right)=-\infty$.
            \item $\lim\limits_{x\to +\infty}\left(\dfrac{\sqrt{x}}{x+1}-\dfrac{1}{x}\right)=0$.
        \end{itemize}
        Suy ra đồ thị hàm số có tiệm cận đứng $x=0$, tiệm cận ngang $y=0$.
    }
\end{ex}
\begin{ex}%[2D1K4-1]
    Số tiệm cận đứng của đồ thị hàm số $y=\dfrac{x^2-3x-4}{x^2-16}$ là
    \choice
    {$2$}
    {$3$}
    {\True $1$}
    {$0$}
    \loigiai{
        Điều kiện xác định $x \ne \pm 4$.\\
        Với điều kiện xác định trên, ta có $y=\dfrac{x^2-3x-4}{x^2-16}=\dfrac{(x+1)(x-4)}{(x-4)(x+4)}=\dfrac{x+1}{x+4}$.\\
        Tiệm cận đứng của đồ thị hàm số là $x=-4$.
    }
\end{ex}
\begin{ex}%[2D1K4-1]
    Số đường tiệm cận đứng và ngang của đồ thị hàm số $y=\dfrac{x-1}{x^2-x-2}$ là
    \choice
    {\True $3$}
    {$1$}
    {$0$}
    {$2$}
    \loigiai{
        Điều kiện xác định $x \ne -1$, $x \ne 2$.\\
        Với điều kiện xác định trên, ta có $y=\dfrac{x-1}{x^2-x-2}=\dfrac{x-1}{(x+1)(x-2)}$.\\
        Tiệm cận đứng của đồ thị hàm số là $x=-1$, $x=2$, tiệm cận ngang của đồ thị hàm số là $y=0$.
    }
\end{ex}
%81
\begin{ex}%[2D1K4-1]
    Cho hàm số $y=f(x)$ có bảng biến thiên như hình bên. Đồ thị hàm số $y=\dfrac{1}{f^2(x)-2f(x)}$ có bao nhiêu tiệm cận đứng?
    \begin{center}
        \begin{tikzpicture}[scale=0.8]
            \tkzTabInit[nocadre=false,lgt=1.5,espcl=3,deltacl=0.6]
            {$x$ /0.6,$y’$ /0.6,$y$ /2}
            {$-\infty$ ,$-1$, $2$, $+\infty$}
            \tkzTabLine{,+,d,-,0,+,}
            \tkzTabVar{-/$-\infty$,+/$1$,-/$-2$,+/$+\infty$}
        \end{tikzpicture}
    \end{center}
    \choice
    {\True $4$}
    {$3$}
    {$2$}
    {$6$}
    \loigiai{
        Dựa vào bảng biến thiên suy ra $f^2(x)-2f(x)=0 \Leftrightarrow \heva{&f(x)=0\\&f(x)=2}$, phương trình $f(x)=0$ có $3$ nghiệm phân biệt và phương trình $f(x)=2$ có $1$ nghiệm nên đồ thị hàm số đã cho có $4$ tiệm cận đứng.
    }
\end{ex}
%79
\begin{ex}%[2D1K4-1]
    Cho hàm số $y=f(x)$ có bảng biến thiên như hình bên. Tổng số tiệm cận ngang và tiệm cận đứng của đồ thị hàm số $y=\dfrac{2}{f(x)+3}$ là
    \begin{center}
        \begin{tikzpicture}[scale=0.8]
            \tkzTabInit[nocadre=false,lgt=1.5,espcl=3,deltacl=0.6]
            {$x$ /0.6,$y’$ /0.6,$y$ /2}
            {$-\infty$ ,$-4$, $6$, $+\infty$}
            \tkzTabLine{,-,0,+,0,-,}
            \tkzTabVar{+/$+\infty$,-/$-2$,+/$5$,-/$-\infty$}
        \end{tikzpicture}
    \end{center}
    \choice
    {$4$}
    {$3$}
    {\True $2$}
    {$1$}
    \loigiai{
        Dựa vào bảng biến thiên suy ra
        \begin{itemize}
            \item 	$\lim \limits_{x \to \pm \infty} f(x)=\pm \infty \Leftrightarrow \lim \limits_{x \to \pm \infty}\dfrac{2}{f(x)+3} =0$ nên đồ thị hàm số đã cho có tiệm cận ngang là $y=0$.
            \item $f(x)+3=0 \Leftrightarrow f(x) =-3$, phương trình này có $1$ nghiệm $x=a>6$ nên đồ thị hàm số đã cho có một tiệm cận đứng.
        \end{itemize}
    }
\end{ex}
\begin{ex}%[2D1K4-1]
    Cho hàm số $y=f(x)$ có bảng biến thiên như hình bên. Đồ thị hàm số $y=\dfrac{x+1}{f(x)-4}$ có bao nhiêu tiệm cận đứng?
    \begin{center}
        \begin{tikzpicture}[scale=0.8]
            \tkzTabInit[nocadre=false,lgt=1.5,espcl=3,deltacl=0.6]
            {$x$ /0.6,$y’$ /0.6,$y$ /2}
            {$-\infty$ ,$-1$, $2$, $+\infty$}
            \tkzTabLine{,+,0,-,0,+,}
            \tkzTabVar{-/$1$,+/$4$,-/$-5$,+/$+\infty$}
        \end{tikzpicture}
    \end{center}
    \choice
    {$1$}
    {$3$}
    {\True $2$}
    {$4$}
    \loigiai{
        Dựa vào bảng biến thiên suy ra
        $f(x)-4=0 \Leftrightarrow f(x) =4$, phương trình này có $1$ nghiệm khác $-1$ và một nghiệm bội chẵn $x=-1$ nên đồ thị hàm số $y=\dfrac{x+1}{f(x)-4}$ có hai tiệm cận đứng.
    }
\end{ex}
\begin{ex}%[2D1K4-1]
    Cho hàm số $y=f(x)$ có bảng biến thiên như hình bên. Đồ thị hàm số $y=\dfrac{x-5}{f(x)-1}$ có bao nhiêu tiệm cận đứng?
    \begin{center}
        \begin{tikzpicture}
            \tkzTabInit[espcl=3]{$x$ / 1 , $f’(x)$ / 1, $f(x)$ / 2}
            {$-\infty$, $-1$ , $2$, $+\infty$}%
            \tkzTabLine{,-,0,+,d,-,}%
            \tkzTabVar{+/ $+\infty$, - / $-1$, + / $3$,-/$-\infty$}%
            \tkzTabVal[draw]{3}{4}{0.4}{$5$}{$1$}%
            %\tkzTabVal[draw]{2}{3}{0.4}{$e^2$}{$1$}%
        \end{tikzpicture}
    \end{center}
    \choice
    {$1$}
    {$3$}
    {\True $2$}
    {$4$}
    \loigiai{
        Dựa vào bảng biến thiên suy ra
        $f(x)-1=0 \Leftrightarrow f(x) =1$, phương trình này có $2$ nghiệm phân biệt khác $5$ và một nghiệm $x=5$ nên đồ thị hàm số $y=\dfrac{x-5}{f(x)-1}$ có hai tiệm cận đứng.
    }
\end{ex}
\begin{ex}%[VDC5-NgocDungHo]%[2D1G4-3]%
    \immini
    {
        Cho hàm số $f(x)$ có đồ thị như hình bên. Số đường tiệm cận đứng của đồ thị hàm số $y=\dfrac{(x^2-1)(x^2+x)}{[f(x)]^2-2f(x)-3}$ là
        \choice
        {$4$}
        {$5$}
        {\True $3$}
        {$2$}
    }
    {
        \begin{tikzpicture}[>=stealth,scale=0.7, line join=round, line cap=round]
            \def\f[#1]{(#1)^3-3*(#1)+1)}
            \draw[->] (-2.2,0)--(2.4,0) node [below]{$x$};
            \draw[->] (0,-1.5)--(0,3.5) node [left]{$y$};
            \node at (0,0) [below left]{$O$};
            % \clip;
            \draw[domain=-2.1:2.1,samples=300,thick] plot (\x,{\f[\x]});
            \filldraw (-1,0) node[below]{$-1$} circle (2pt);
            \filldraw (1,0) node[above]{$1$} circle (2pt);
            \filldraw (0,-1) node[ left]{$1$} circle (2pt);
            \filldraw (0,3) node[ right]{$3$} circle (2pt);
            \draw[dashed](-1,0)--(-1,3)--(0,3) (1,0)--(1,-1)--(0,-1);
            \draw (2,3) node[right]{$y=f(x)$};
        \end{tikzpicture}
    }
    \loigiai{
        Xét hàm số $y=g(x)=\dfrac{(x^2-1)(x^2+x)}{[f(x)]^2-2f(x)-3}$.
        \immini
        {
            Giải phương trình $(x^2-1)(x^2+x)=0 \Leftrightarrow \hoac{& x^2-1=0 \\ & x^2+x=0}\Leftrightarrow \hoac{& x=\pm 1 \\ & x=0.}$\\
            Giải phương trình $[f(x)]^2-2f(x)-3=0$\\$ \Leftrightarrow \hoac{& f(x)=-1 \\ & f(x)=3} \Leftrightarrow \hoac{& x = \pm 1 \\ & x=a\\&x=b\;(a<-1<1<b).}$
        }
        {
            \begin{tikzpicture}[>=stealth,scale=0.7, line join=round, line cap=round]
                \def\f[#1]{(#1)^3-3*(#1)+1)}
                \def\g[#1]{3}
                \def\h[#1]{-1}
                \draw[->] (-2.5,0)--(4,0) node [below]{$x$};
                \draw[->] (0,-1.5)--(0,3.5) node [left]{$y$};
                \node at (0,0) [below left]{$O$};
                % \clip;
                \draw[domain=-2.5:4,samples=300,thick] plot (\x,{\g[\x]});
                \draw[domain=-2.5:4,samples=300,thick] plot (\x,{\h[\x]});
                \draw[domain=-2.1:2.1,samples=300,thick] plot (\x,{\f[\x]});
                \filldraw (-1,0) node[below]{$-1$} circle (2pt);
                \filldraw (1,0) node[above]{$1$} circle (2pt);
                \filldraw (0,-1) node[ left]{$1$} circle (2pt);
                \filldraw (0,3) node[ right]{$3$} circle (2pt);
                \draw[dashed](-1,0)--(-1,3)--(0,3) (1,0)--(1,-1) (2,0)node[below]{$b$}--(2,3) (-2,0)node[above]{$a$}--(-2,-1);
                \draw (2,2) node[right]{$y=f(x)$};
                \draw (3.3,3) node[above]{$d_1:y=3$};
                \draw (3,-1) node[below]{$d_2:y=-1$};
            \end{tikzpicture}
        }
        Trong điều kiện xác định của hàm số $y=g(x)$ ta có thể viết
        $$y=g(x)=\dfrac{x(x-1)(x+1)^2}{(x-a)(x-b)(x-1)^2(x+1)^2}=\dfrac{x}{(x-a)(x-b)(x-1)}$$
        Vậy số tiệm cận đứng của đồ thị hàm số $y=g(x)$ bằng $3$.
    }
\end{ex}
\begin{ex}
    \immini{ %Câu 91.
        Đường cong ở hình bên là đồ thị của hàm số $y = ax^4 + bx^2 +c$. Đồ thị hàm số $g(x) =\dfrac{(x^2-x)\sqrt{x+2}}{(x-2)\cdot f(x+1)}$
        có bao nhiêu đường tiệm cận đứng?
        \choice
        {1}
        {3}
        {4}
        {2}}{
        \begin{tikzpicture}[scale=.8, font=\footnotesize, line join=round, line cap=round, >=stealth]
            \def\xmin{-2}\def\xmax{2}\def\ymin{-3}\def\ymax{1}
            \draw[->] (\xmin-0.2,0)--(\xmax+0.2,0) node[below] {\footnotesize $x$};
            \draw[->] (0,\ymin-0.2)--(0,\ymax+0.2) node[right] {\footnotesize $y$};
            \draw (0,0) node [below left] {\footnotesize $O$};
            \foreach \x in {-1,1}\draw (\x,-0.1)--(\x,0.1) node [above left] {\footnotesize $\x$};
            \foreach \y in {-2}\draw (0.1,\y)--(-0.1,\y) node [ below left] {\footnotesize $\y$};
            \clip (\xmin,\ymin) rectangle (\xmax,\ymax);
            \draw[smooth,samples=200,domain=\xmin:\xmax] plot (\x,{((\x)^4)+((\x)^2)+-2});
        \end{tikzpicture}
    }
    \loigiai{
        * Điều kiện: $\heva{&x \ne 2\\&f(x+1) \ne 0\\&x \ge -2.}$\\
        Nhìn hình vẽ ta thấy
        $f(x+1)=0\Leftrightarrow \hoac{&x+1=-1\\&x+1=1}\Leftrightarrow \hoac{&x=-2&(\text{nghiệm đơn})\\&x=0&(\text{nghiệm đơn}).}$\\
        Vậy $g(x) = \dfrac{(x^2-x)\sqrt{x+2}}{(x-2)\cdot ax^2(x^2+2) }=\dfrac{(x-1)\sqrt{x+2}}{(x-2)\cdot ax(x^2+2)}.$ \\
        Đồ thị hàm số $g(x)$ có 2 đường tiệm cận đứng.}
\end{ex}
\begin{ex}%[Thi thử THPT Yên Phong 1 - Bắc Ninh, 2021]%[Duong Xuan Loi,12EX 6- 2021]%[2D1G4-3]%
    \immini{
        Cho hàm số $y=f(x)$ có đồ thị như hình vẽ. Biết $f'(x)<0$, $\forall x <-1$ và $f'(x)>0$, $\forall x>1$. Khi đó, tổng số tiệm cận của đồ thị hàm số $y=\dfrac{2024}{\sqrt{xf(x+1)}[xf(x+1)+1]-2}$ là
        \choice
        {$1$}
        {$3$}
        {$4$}
        {\True $2$}
    }{
        \begin{tikzpicture}[scale=0.7, font=\footnotesize, line join=round, line cap=round,>=stealth]
            \def\xmin{-2} \def\xmax{2}
            \def\ymin{-2} \def\ymax{3.3}
            \draw[color=gray!50,dashed] (\xmin,\ymin) grid (\xmax,\ymax);
            \draw[->] (\xmin,0)--(\xmax,0) node [below]{$x$};
            \draw[->] (0,\ymin)--(0,\ymax) node [left]{$y$};
            \node at (0,0) [above right]{$O$};
            \clip (\xmin+0.1,\ymin+0.1) rectangle (\xmax-0.1,\ymax-0.1);
            \draw[smooth,samples=300,domain=-1.8:1.4] plot(\x,{(\x+1)*(\x+1)*(\x)*(\x-1)});
            \fill (-1,0) circle (1.0pt) node[below]{$-1$} (1,0) circle (1.0pt) node[below right]{$1$};
        \end{tikzpicture}
    }
    \loigiai{
        Xét phương trình $\sqrt{xf(x+1)}[xf(x+1)+1]-2=0.\quad(1)$\\
        Đặt $t=\sqrt{xf(x+1)}(t\geq 0)$, ta được phương trình $t\left(t^2+1\right)=2\Leftrightarrow t^3+t-2=0\Leftrightarrow t=1$.\\
        Với $t=1\Rightarrow\sqrt{xf(x+1)}=1\Leftrightarrow xf(x+1)-1=0$.\\
        Đặt $u=x+1\Rightarrow x=u-1$, ta được phương trình $(u-1)f(u)-1=0\Leftrightarrow f(u)=\dfrac{1}{u-1}.\quad(2)$
        \begin{center}
            \begin{tikzpicture}[scale=1, font=\footnotesize, line join=round, line cap=round,>=stealth]
                \def\a{0} \def\b{1} \def\c{1} \def\d{-1} % Hệ số
                \def\xmin{-3} \def\xmax{3.5}
                \def\ymin{-2.8} \def\ymax{3.3}
                \draw[color=gray!50,dashed] (\xmin,\ymin) grid (\xmax,\ymax);
                \draw[->] (\xmin,0)--(\xmax,0) node [below]{$u$};
                \draw[->] (0,\ymin)--(0,\ymax) node [left]{$y$};
                \node at (0,0) [above right]{$O$};
                \clip (\xmin+0.1,\ymin+0.1) rectangle (\xmax-0.1,\ymax-0.1);
                \draw[smooth,samples=300,domain=-1.8:1.4] plot(\x,{(\x+1)*(\x+1)*(\x)*(\x-1)});
                \draw[smooth,samples=300,domain=\xmin:(-\d/\c-0.1)] plot(\x,{(\a*(\x)+\b)/(\c*(\x)+\d)});
                \draw[smooth,samples=300,domain=(-\d/\c+0.1:\xmax)] plot(\x,{(\a*(\x)+\b)/(\c*(\x)+\d)});
                \fill (-1,0) circle (1.0pt) node[below]{$-1$} (1,0) circle (1.0pt) node[below right]{$1$};
            \end{tikzpicture}
        \end{center}
        Nhận thấy đồ thị của các hàm số $y=f(u)$, $y=\dfrac{1}{u-1}$ chỉ cắt nhau tại $1$ điểm do đó phương trình $(2)$ có nghiệm duy nhất $\Rightarrow(1)$ có nghiệm duy nhất, suy ra đồ thị có $1$ tiệm cận đứng.\\
        Mặt khác: $\lim\limits_{x\to+\infty} f(x+1)=+\infty\Rightarrow\lim\limits_{x\to+\infty}\dfrac{2021}{\sqrt{xf(x+1)}[xf(x+1)+1]-2}=a>0$.\\
        $\lim\limits_{x\to-\infty} xf(x+1)=-\infty\Rightarrow\lim\limits_{x\to+\infty}\dfrac{2021}{\sqrt{xf(x+1)}[xf(x+1)+1]-2}$ không tồn tại.\\
        Do đó đường thẳng $y=a$ là tiệm cận ngang.}
\end{ex}
\BTTF
\begin{ex}%[EX-TF-2024, Lê Đạt]%[2D1N4-1]
    Cho hàm số $y=f(x)$ có bảng biến thiên như sau
    \begin{center}
        \begin{tikzpicture}[>=stealth]
            \tkzTabInit[nocadre=false,lgt=1,espcl=3,deltacl=0.6]
            {$x$/.7 ,$y'$/.7,$y$/2}
            {$-\infty$ , $-2$ , $0$, $+\infty$}
            \tkzTabLine{ , - , d , + , d , -, }
            \tkzTabVar{+/$+\infty$ , -D-/$1$/$-\infty$ , +D+/$+\infty$ /$1$, -/$0$}
        \end{tikzpicture}
    \end{center}
    Xét tính đúng sai của các khẳng định sau
    \choiceTF
    {\True $ x=0 $ là tiệm cận đứng của đồ thị hàm số $ y=f(x) $}
    {\True $ x=-2 $ là tiệm cận đứng của đồ thị hàm số $ y=f(x) $}
    {$ x=1 $ là tiệm cận đứng của đồ thị hàm số $ y=f(x) $}
    {\True $ y=0 $ là tiệm cận ngang của đồ thị hàm số $ y=f(x) $}
    \loigiai{
        \begin{itemchoice}
            \itemch $\lim \limits_{x \to 0^-} f(x)=+\infty\Rightarrow x=0$ là đường tiệm cận đứng của đồ thị hàm số $f(x)$.
            \itemch $\lim \limits_{x \to (-2)^+} f(x)=-\infty\Rightarrow x=-2$ là đường tiệm cận đứng của đồ thị hàm số $f(x)$.
            \itemch Đồ thị hàm số chỉ có hai tiệm cận đứng là $ x=0 $ và $ x=-2 $.
            \itemch $\lim \limits_{x \to +\infty} f(x)=0\Rightarrow y=0$ là đường tiệm cận ngang của đồ thị hàm số $f(x)$.
        \end{itemchoice}
    }
\end{ex}
\begin{ex}%[EX-TF-2024, Lê Đạt]%[2D1H4-2]
    Cho hàm số $y=\dfrac{m x^{2}+6 x-2}{x+2}$. Xét tính đúng sai của các khẳng định sau
    \choiceTF
    {Đồ thị hàm số luôn có tiệm cận đứng với mọi $ m $}
    {Đồ thị hàm số không có tiệm cận ngang với mọi $ m $}
    {\True Khi $ m=1 $ đồ thị hàm số có một tiệm cận xiên là $ y=x+4 $ }
    {Đồ thị hàm số luôn có tiệm cận xiên}
    \loigiai{
        \begin{itemchoice}
            \itemch Khi $ m=\dfrac{7}{2} $ hàm số trở thành $y=\dfrac{\dfrac{7}{2} x^{2}+6 x-2}{x+2}=\dfrac{7}{2}\left(x-\dfrac{2}{7} \right) $ suy ra đồ thị hàm số không có tiệm cận đứng.
            \itemch Khi $ m=0 $ hàm số trở thành $ y=\dfrac{6x-2}{x+2} $ từ đó suy ra đồ thị hàm số có $ y=6 $ là tiệm cận ngang.
            \itemch Khi $ m=1 $ hàm số trở thành $ y=\dfrac{x^2+6x-2}{x+2}=x+4-\dfrac{10}{x+2} $ từ đó suy ra $ y=x+4 $ là một tiệm cận ngang.
            \itemch Khi $ m=0 $ hàm số trở thành $ y=\dfrac{6x-2}{x+2} $ từ đó suy ra đồ thị hàm số có $ y=6 $ là tiệm cận ngang, $ x=-2 $ là tiệm cận đứng và không có tiệm cận xiên.
        \end{itemchoice}
    }
\end{ex}
\begin{ex}%[EX-TF-2024, Lê Đạt]%[2D1H4-3]
    \immini{Cho hàm số $y=f(x)$ có đồ thị như hình bên. Xét tính đúng sai của các khẳng định sau
        \choiceTF
        {\True $ x=0 $ là một đường tiệm cận đứng của đồ thị hàm số}
        {$ y=-x $ là một đường tiệm cận xiên của đồ thị hàm số}
        {\True $ y=x $ là một đường tiệm cận xiên của đồ thị hàm số}
        {Đồ thị hàm số có ba đường tiệm cận}
    }{
        \begin{tikzpicture}[scale=.9, font=\footnotesize, line join=round, line cap=round,>=stealth]
            \def\a{0} \def\b{1} \def\c{1} \def\d{-1} % Hệ số
            \def\xmin{-3} \def\xmax{3.5}
            \def\ymin{-2.8} \def\ymax{3.3}
            \draw[color=gray!50,dashed] (\xmin,\ymin) grid (\xmax,\ymax);
            \draw[->] (\xmin,0)--(\xmax,0) node [below]{$x$};
            \draw[->] (0,\ymin)--(0,\ymax) node [left]{$y$};
            \fill (0,0) circle(1pt) node[shift=(-45:0.25)]{$O$};
            \clip (\xmin+0.1,\ymin+0.1) rectangle (\xmax-0.1,\ymax-0.1);
            \draw[smooth,samples=300,domain=-3:3] plot(\x,{\x+1/(7*\x)});
            \draw[dashed,smooth,samples=300,domain=-3:3] plot(\x,{\x});
            %	\fill (-1,0) circle (1.0pt) node[below]{$-1$} (1,0) circle (1.0pt) node[below right]{$1$};
    \end{tikzpicture}}
    \loigiai{
        \begin{itemchoice}
            \itemch $ x=0 $ là một đường tiệm cận đứng của đồ thị hàm số.
            \itemch	$ y=x $ là một đường tiệm cận xiên của đồ thị hàm số.
            \itemch $ y=x $ là một đường tiệm cận xiên của đồ thị hàm số.
            \itemch Đồ thị hàm số có $ x=0 $ là tiệm cận đứng và $ y=x $ là tiệm cận xiên nên có hai tiệm cận.
        \end{itemchoice}
    }
\end{ex}
\begin{ex}
    \immini
    {
        Cho hàm số $y=f(x)$ có đạo hàm liên tục trên $R$. Hàm số $y=f^{\prime}(x)$ có đồ thị như hình bên. Xác định tính đúng, sai của các mệnh đề sau
        \choiceTF
        {Hàm số $y=f(x)$ có hai cực trị}
        {Hàm số $y=f(x)$ đồng biến trên khoảng $(1 ;+\infty)$}
        {\True $f(1)>f(2)>f(4)$.}
        {\True Trên đoạn $[-1 ; 4]$, giá trị lớn nhất của hàm số $y=f(x)$ là $f(1)$.}
    }
    {
        \begin{tikzpicture}[line join=round, line cap=round,>=stealth,font=\scriptsize]
            \begin{scope}[scale=0.5]
                \tikzset{label style/.style={font=\footnotesize}}
                \def \xmin{-2}
                \def \xmax{4.5}
                \def \ymin{-2}
                \def \ymax{3.5}
                \def \hamso{0.55*(\x)^3-1.76*(\x)^2-0.31*(\x)+2}
                \draw[->] (\xmin,0)--(\xmax,0) node[below left] {$x$};
                \draw[->] (0,\ymin)--(0,\ymax) node[below left] {$y$};
                \draw (0,0) node [below left] {$O$};
                \begin{scope}
                    \clip (\xmin+0.01,\ymin+0.01) rectangle (\xmax-0.01,\ymax-0.01);
                    \draw[samples=350,domain=-1.3:3.3,smooth,variable=\x] plot (\x,{\hamso});
                \end{scope}
                \draw (-1,0) node[below left]{$-1$} (1,0) node[below]{$1$} (3,0) node[below right]{$4$} (0,2) node[above left]{$2$};
            \end{scope}
        \end{tikzpicture}
    }
    \loigiai{}
\end{ex}
\begin{ex}
    \immini{Cho hàm số $y=f(x)$ liên tục trên đoạn $\left[0 ; \frac{7}{2}\right]$ có đồ thị hàm số $y=f^{\prime}(x)$ như hình vẽ.
        \choiceTF
        {\True Hàm số $y=f(x)$ đồng biến trên khoảng $\left(3 ; \frac{7}{2}\right)$}
        {\True $f(0)>f(3)$}
        {$f(3)>f\left(\frac{7}{2}\right)$}
        {Hàm số $y=f(x)$ đạt giá trị nhỏ nhất trên đoạn $\left[0 ; \frac{7}{2}\right]$ tại điểm $x_0=\frac{7}{2}$}
    }{\begin{tikzpicture}[>=stealth, samples=100,smooth,y=.7cm,font=\scriptsize]
            \begin{scope}[scale=.7]
                \draw[->] (-1,0)--(4.5,0) node[below] {$x$};
                \draw[->] (0,-2)--(0,4) node[right] {$y$};
                \draw (0,0) node [below left] {$O$};
                \draw[dashed] (3.6,0)--(3.6,4);
                \draw[samples=200,domain=0.2:3.6,smooth,variable=\x]
                plot (\x,{1.06*(\x)^3-5.3*(\x)^2+7.23*(\x)-3});
                \path
                (3.6,0)node[below]{$\dfrac{7}{2}$}
                (3,0)node[above left]{$3$}
                (1,0)node[above]{$1$}
                ;
            \end{scope}
    \end{tikzpicture}}
    \loigiai{}
\end{ex}
\BTTL
\begin{ex}%[2D1K4-2]%
    Đồ thị hàm số $y=\dfrac{(2m+1)x+3}{x+1}$ có đường đường tiệm cận đi qua điểm $A(-2;7)$ khi và chỉ khi
    \shortans{$m=3$}
    %	\choice
    %	{\True $m=3$}
    %	{$m=1$}
    %	{$m=-1$}
    %	{$m=-3$}
    \loigiai{
        Từ đề bài, suy ra $2m+1=7 \Leftrightarrow m=3$.\\
        Suy ra $m+n=0$.
    }
\end{ex}
\begin{ex}%[Học kì 1, THPT Nguyễn Thi Minh Khai - Hà Nội, 2020-2021]%[Bùi Mạnh Tiến, 12EX5]%[2D1K4-2]%
    Cho hàm số $ y=\dfrac{2mx+m}{x-1}$. Với giá trị nào của tham số $m$ thì đường tiệm cận đứng, tiệm cận ngang của đồ thị hàm số cùng hai trục tọa độ tạo thành một hình chữ nhật có diện tích bằng $8$?
    \shortans{$m=\pm 4$}
    %	\choice
    %	{$ m=2$}
    %	{ $m=\pm 2$}
    %	{\True $m=\pm 4$}
    %	{$ m=\pm\dfrac{1}{2}$}
    \loigiai{
        Hàm số $y=\dfrac{2mx+m}{x-1}$ có $a=2m$, $b=m$, $c=1$, $d=-1$.\\
        Tiệm cận ngang $y=\dfrac{a}{c}=2m$.\\
        Tiệm cận đứng $x=-\dfrac{d}{c}=1$.\\
        Diện tích hình chữ nhật tạo thành bởi hai đường tiệm cận và hai trục tọa độ có diện tích
        \begin{align*}
            |2m|\cdot 1=8\Leftrightarrow m=\pm 4.
        \end{align*}
    }
\end{ex}
\begin{ex}%[KSCL lần 1, Liễn Sơn - Vĩnh Phúc, 2021]%[Phạm Doãn Lê Bình, 12EX4-2021]%[2D1K4-2]%
    Cho hàm số $y=\dfrac{x-\sqrt{x^2+2x}}{x^2+mx-m-3}$ có đồ thị $(C)$. Giá trị của $m$ để $(C)$ có đúng hai tiệm cận thuộc tập nào sau đây?
    \shortans{$(-5;2)$}
    %	\choice
    %	{$(-2;1)$}
    %	{$(1;5)$}
    %	{$(5;8)$}
    %	{\True $(-5;2)$}
    \loigiai{
        Điều kiện xác định của hàm số đã cho $\heva{& \hoac{ & x\ge 0 \\ & x\le -2}\\ & x^2+mx-m-3\ne 0.}$\\
        Ta có $\lim \limits_{x\to +\infty} y = \lim \limits_{x\to -\infty} y = 0$ nên $(C)$ có một tiệm cận ngang $y=0$.\\
        Xét phương trình $x^2+mx-m-3=0$.\hfill $(1)$\\
        Ta có
        \begin{itemize}
            \item $\Delta = m^2+4m+12>0$, $\forall m \in \mathbb{R}$.\\ Vậy phương trình $(1)$ luôn có hai nghiệm phân biệt $x_1,x_2$ ($x_1<x_2$).
            \item $x-\sqrt{x^2+2x}=0 \Leftrightarrow \heva{& x\ge 0 \\ & x^2=x^2+2x} \Leftrightarrow x=0$.
            \item Phương trình $(1)$ có nghiệm $x=0 \Leftrightarrow m=-3$. Với $m=-3$ ta có
            $ y =\dfrac{x-\sqrt{x^2+2x}}{x^2-3x}.$
            Khi đó
            \begin{eqnarray*}
                & \lim \limits_{x\to 0^+} y & =\lim \limits_{x\to 0^+} \dfrac{x-\sqrt{x^2+2x}}{x^2-3x}\\
                & & =\lim \limits_{x\to 0^+} \dfrac{-2x}{(x^2-3x)\left( x+\sqrt{x^2+2x}\right)}\\
                & & = \lim \limits_{x\to 0^+} \dfrac{-2}{(x-3)\left( x+\sqrt{x^2+2x}\right)}=+\infty
            \end{eqnarray*}
            và $\lim \limits_{x\to 3^+} y =-\infty$
            nên $(C)$ có thêm hai tiệm cận đứng $x=0$ và $x=3$ (không thỏa yêu cầu bài toán).
            \item Với $m\ne -3$ thì $(C)$ có đúng hai tiệm cận khi và chỉ khi $\hoac{& x_1<-2<x_2<0 &(2)\\ & -2<x_1<0<x_2. & (3)}$
            \item Đặt $f(x)=x^2+mx-m-3$. Ta có
            $(2)\Leftrightarrow \heva{& f(-2)< 0 \\ & f(0) >0 \\ & 0>-m} \Leftrightarrow \heva{& m>\dfrac{1}{3}\\ & m<-3 \\ & m>0}\Leftrightarrow m \in \varnothing.$
            \item $(3)\Leftrightarrow \heva{& f(-2)> 0 \\ & f(0) <0 \\ & -2<-m} \Leftrightarrow \heva{& m<\dfrac{1}{3}\\ & m>-3 \\ & m<2}\Leftrightarrow -3<m<\dfrac{1}{3}.$
        \end{itemize}
        Vậy $m\in \left(-3;\dfrac{1}{3}\right)$.
    }
\end{ex}
\begin{ex}%[kiểm tra GHK1, Sở GD và ĐT - Vĩnh Phúc, 2021]%[Huỳnh Xuân Tín, 12EX4]%[2D1K4-2]%
    Gọi $S$ là tập tất cả các giá trị của tham số $m$ để đồ thị hàm số	$y=\dfrac{x-3}{x^2-2x-m}$ có đúng một đường
    tiệm cận đứng. Tính tổng các phần tử của tập $S$.
    \shortans{$2$}
    %	\choice
    %	{$-1$}
    %	{\True $2$}
    %	{$-6$}
    %	{$1$}
    \loigiai{
        Để đồ thị hàm số	$y=\dfrac{x-3}{x^2-2x-m}$ có đúng một đường
        tiệm cận đứng, ta có hai trường hợp sau
        \begin{enumerate}[TH 1.]
            \item $x^2-2x-m=0$ có nghiệm kép $\Leftrightarrow \Delta'=1+m=0\Leftrightarrow m=-1$.
            \item $x^2-2x-m=0$ có hai nghiệm phân biệt trong đó có một nghiệm bằng $3$
            \[\Leftrightarrow \heva{&\Delta'=1+m>0\\& 3^2-6-m=0}\Leftrightarrow\heva{&m>-1\\&m=3}\Leftrightarrow m=3.\]
        \end{enumerate}
        Khi đó $S=\{-1;3\}$ và có tổng là $2$.
    }
\end{ex}
\begin{ex}
    Tốc độ phản ứng của enzyme theo nồng độ cơ chất \( S \) được mô tả bởi phương trình Michaelis-Menten: $v(S) = \dfrac{V_{\text{max}} S}{K_m + S}$,
    trong đó \( v(S) \) là tốc độ phản ứng, \( S \) là nồng độ cơ chất, \( V_{\text{max}} \) là tốc độ tối đa, và \( K_m \) là hằng số Michaelis. Xác định và nêu ý nghĩa của đường tiệm cận đứng của hàm số này.
    \shortans{không có TCĐ, tốc độ phản ứng không thể tới vô hạn}
    \loigiai{
        Để tìm tiệm cận đứng, ta xét các giá trị của \( S \) làm cho mẫu số của phương trình bằng 0:
        \[
        K_m + S = 0 \Rightarrow S = -K_m
        \]
        Vì nồng độ cơ chất \( S \) không thể âm, không có tiệm cận đứng trong trường hợp này.
        \textbf{Ý nghĩa:} Điều này có nghĩa là tốc độ phản ứng enzyme không có giá trị nào dẫn đến tốc độ phản ứng tiến đến vô hạn trong phạm vi các giá trị hợp lý của \( S \).}
\end{ex}
\begin{ex}
    \immini{Một ống khói của nhà máy điện hạt nhân có mặt cắt là một hypebol $(H)$ có phương trình chính tắc là $\dfrac{x^2}{27^2}-\dfrac{y^2}{40^2}=1$ (Hình $1.25$). Xét hai nhánh bên trên $Ox$ của $(H)$ là đồ thị $(C)$ của hàm số $y=\dfrac{40}{27}\sqrt{x^2-27^2}$ (phần nét liền đậm). Tìm tất cả các đường tiệm cận xiên của $(C)$.}{\begin{tikzpicture}[>=latex,line join=round, line cap=round, scale=.04, font=\footnotesize]
            \draw[->] (-90,0)--(90,0) node[above]{$x$};
            \draw[->] (0,-130)--(0,80) node[left]{$y$};
            \foreach \x in {-80,-60,-40,-20,20,40,60,80}
            \draw[fill=black] (\x,0) circle (15pt) node[below, fill=white]{$\x$};
            \foreach \y in {-120,-100,-80,-60,-40,-20,20,40,60}
            \draw[fill=black] (0,\y) circle (15pt) node[left]{$\y$};
            \clip (-90,-130) rectangle (90,80);
            \draw[samples=200,smooth,blue,line width=1] plot[domain=-90:-27] (\x,{40*sqrt((\x)^2-27^2)/27});
            \draw[samples=200,smooth,blue,line width=1] plot[domain=27:90] (\x,{40*sqrt((\x)^2-27^2)/27});
            \draw[samples=200,smooth,blue,line width=1, dashed] plot[domain=-90:-27] (\x,{-40*sqrt((\x)^2-27^2)/27});
            \draw[samples=200,smooth,blue,line width=1, dashed] plot[domain=27:90] (\x,{-40*sqrt((\x)^2-27^2)/27});
            \draw (0,0) node[above right]{$O$};
    \end{tikzpicture}}
    \shortans{$y=\pm \dfrac{40}{27}$}
    \loigiai{
        Ta có
        \allowdisplaybreaks
        \begin{eqnarray*}
            a&=&\lim\limits_{x\to+\infty}\dfrac{f(x)}{x}	 =\lim\limits_{x\to+\infty}\dfrac{\dfrac{40}{27}\sqrt{x^2-27^2}}{x}\\
            &=&\lim\limits_{x\to+\infty}\dfrac{\dfrac{40}{27}x\sqrt{1-\dfrac{27^2}{x^2}}}{x}
            =\lim\limits_{x\to+\infty}\dfrac{40}{27}\sqrt{1-\dfrac{27^2}{x^2}}
            =\dfrac{40}{27}.\\
            b&=&\lim\limits_{x\to+\infty}\left[f(x)-ax\right]
            =\lim\limits_{x\to+\infty}\left[\dfrac{40}{27}\sqrt{x^2-27^2}-\dfrac{40}{27}x\right]\\
            &=&\lim\limits_{x\to+\infty}\dfrac{40}{27}\left(\sqrt{x^2-27^2}-x\right)
            =\lim\limits_{x\to+\infty}\dfrac{40}{27}\cdot\dfrac{x^2-27^2-x^2}{\sqrt{x^2-27^2}+x}\\
            &=&\lim\limits_{x\to+\infty}\dfrac{40}{27}\cdot\dfrac{-27^2}{x\left(\sqrt{1-\dfrac{27}{x^2}}+1\right)}
            =0.
        \end{eqnarray*}
        Vậy đường thẳng $y=\dfrac{40}{27}x$ là một tiệm cận xiên của đồ thị.\\
        Tương tự, $\lim\limits_{x\to-\infty}\dfrac{f(x)}{x}=-\dfrac{40}{27}\Rightarrow a=-\dfrac{40}{27}$; $\lim\limits_{x\to-\infty} \left[f(x)-ax\right]=0\Rightarrow b=0$.\\
        Vậy đường thẳng $y=-\dfrac{40}{27}x$ là tiệm cận xiên của đồ thị.
    }
\end{ex}

\Closesolutionfile{ans}
% \begin{dang}{Tìm các đường tiệm cận đồ thị hàm ẩn}
\end{dang}
\begin{vd}
    Cho hàm số $y=f(x)$ có bảng biến thiên như hình vẽ sau
    \begin{center}
        \begin{tikzpicture}[>=stealth]
            \tkzTabInit[nocadre=false,lgt=1,espcl=1.5,deltacl=0.5]{$x$/.7 ,$y'$/.7,$y$/2}
            {$-\infty$ , $-1$ , $2$ , $+\infty$}
            \tkzTabLine{ , + , $0$ , - , d , + , }
            \tkzTabVar{-/$1$ , +/$4$ , -/$-5$ , +/$+\infty$}
        \end{tikzpicture}
    \end{center}
    Tìm TCĐ, TCN của đồ thị hàm số
    \begin{listEX}[3]
        \item $y=\dfrac{2}{f(x)-3}$
        \item $y=\dfrac{-3}{f(x)+2}$
        \item $y=\dfrac{x-2}{f(x)+5}$
        \item $y=\dfrac{x+1}{f(x)-4}$
        \item $y=\dfrac{2}{f(x^2)+3}$
        \item $y=\dfrac{4f(x)-5}{3f(x)+1}$
    \end{listEX}
    \loigiai{}
\end{vd}
\begin{vd}\immini{Cho hàm bậc ba $y=f(x)$ có đồ thị như hình vẽ. Tìm số tiệm cận đứng của đồ thị hàm số
        \begin{listEX}[2]
            \item $y=\dfrac{\sqrt{x+3}}{(x-1)f(x)}$
            \item $g(x)=\dfrac{(x^2+4x+3)\sqrt{x^2+x}}{x\left[f^2(x)-2f(x)\right]}$ .
    \end{listEX}}{\begin{tikzpicture}[line cap=round,line join=round, >=stealth,font=\footnotesize]
            \begin{scope}[scale=.5]
                \def\a{-1} % Hệ số a phải khác 0
                \def\b{-13/2}
                \def\c{-12}
                \def\d{-9/2}
                \draw[->] (-5,0) -- (2,0)node[below]{$x$};
                \draw[->] (0,-3) -- (0,4) node[left] {$y$};
                \draw (0,0)node[below right]{$O$} (-3,0)node[below]{$-3$};
                \draw[dashed] (-1,0)node[below]{$-1$}|-(0,2)node[right]{$2$};
                \draw[samples=150,smooth,domain=-4:.-.2] plot(\x,{\a*(\x)^3+(\b)*(\x)^2+(\c)*\x+(\d)});
            \end{scope}
    \end{tikzpicture}}
    \loigiai{
        \begin{center}
            \begin{tikzpicture}[line cap=round,line join=round, >=stealth,font=\footnotesize,scale=1]
                \def\a{-1} % Hệ số a phải khác 0
                \def\b{-13/2}
                \def\c{-12}
                \def\d{-9/2}
                \draw[->] (-5,0) -- (2,0)node[below]{$x$};
                \draw[->] (0,-3) -- (0,4) node[left] {$y$};
                \draw (0,0)node[below right]{$O$} (-3,0)node[below]{$-3$} (-.3,0)node[above]{$a$};
                \draw[dashed] (-3.78,0)node[below]{$c$}|-(0,2)|-(-1.71,0)node[below]{$b$}|-(0,2) (-1,0)node[below]{$-1$}|-(0,2)node[right]{$2$};
                \draw[samples=150,smooth,domain=-4:.-.2] plot(\x,{\a*(\x)^3+(\b)*(\x)^2+(\c)*\x+(\d)});
            \end{tikzpicture}
        \end{center}
        $g(x)=\dfrac{(x^2+4x+3)\sqrt{x^2+x}}{x\left[f^2(x)-2f(x)\right]}=\dfrac{(x+1)(x+3)\sqrt{x(x+1)}}{x\left[f^2(x)-2f(x)\right]}$.\\
        Điều kiện của căn là $x\le -1; x\ge 0$.\\
        Dựa vào đồ thị ta có \[x\left[f^2(x)-2f(x)\right]=0 \Leftrightarrow \hoac{&x=0\\&f(x)=0\\& f(x)=2} \Leftrightarrow \hoac{&x=0\text{ (nhận)}\\&x=-3\text{ (nhận)};\ x=a \text{ (loại)} \\&x=-1\text{ (nhận)};\ x=b\text{ (nhận)};\ x=c\text{ (nhận)}}\]\\
        Số TCĐ lúc này chính là số nghiệm không bị rút gọn của mẫu, vậy có bốn TCĐ là $x=0; x=-3; x=b; x=c$.
    }
\end{vd}
\BTTN
\Opensolutionfile{ans}[ans/2D1-4-DANG-3]
\begin{ex}%[2D1K4-1]
    Cho hàm số $y=f(x)$ có bảng biến thiên như hình bên. Đồ thị hàm số $y=\dfrac{-5}{f(x)+4}$ có bao nhiêu tiệm cận đứng?
    \begin{center}
        \begin{tikzpicture}[scale=0.8]
            \tkzTabInit[nocadre=false,lgt=1.5,espcl=3,deltacl=0.6]
            {$x$ /0.6,$y’$ /0.6,$y$ /2}
            {$-\infty$ ,$1$, $2$, $+\infty$}
            \tkzTabLine{,+,d,-,d,+,}
            \tkzTabVar{-/$-4$,+/$3$,-/$-5$,+/$+\infty$}
        \end{tikzpicture}
    \end{center}
    \choice
    {$1$}
    {$3$}
    {\True $2$}
    {$4$}
    \loigiai{
        Dựa vào bảng biến thiên suy ra
        $f(x)+4=0 \Leftrightarrow f(x) =-4$, phương trình này có $2$ nghiệm phân biệt nên đồ thị hàm số $y=\dfrac{-5}{f(x)+4}$ có $2$ tiệm cận đứng.
    }
\end{ex}
\begin{ex}%[2D1K4-1]
    Cho hàm số $y=f(x)$ có bảng biến thiên như hình bên. Đồ thị hàm số $y=\dfrac{x+2}{2f(x)-1}$ có bao nhiêu tiệm cận đứng?
    \begin{center}
        \begin{tikzpicture}[scale=0.8]
            \tkzTabInit[nocadre=false,lgt=1.5,espcl=3,deltacl=0.6]
            {$x$ /0.6,$y’$ /0.6,$y$ /2}
            {$-\infty$ ,$-1$, $0$, $1$, $+\infty$}
            \tkzTabLine{,+,0,-,0,+,0,-,}
            \tkzTabVar{-/$-\infty$,+/$0$,-/$-\dfrac{5}{3}$,+/$0$,-/$-\infty$}
        \end{tikzpicture}
    \end{center}
    \choice
    {$1$}
    {$3$}
    {$2$}
    {\True $0$}
    \loigiai{
        Dựa vào bảng biến thiên suy ra
        $2f(x)-1=0 \Leftrightarrow f(x) =\dfrac{1}{2}$, phương trình này có $0$ nghiệm nên đồ thị hàm số $y=\dfrac{x+2}{2f(x)-1}$ không có tiệm cận đứng.
    }
\end{ex}
%69
\begin{ex}%[2D1K4-1]
    Cho hàm số $y=f(x)$ có bảng biến thiên như hình bên. Đồ thị hàm số $y=\dfrac{1}{2f(x)-3}$ có bao nhiêu tiệm cận đứng?
    \begin{center}
        \begin{tikzpicture}[scale=0.8]
            \tkzTabInit[nocadre=false,lgt=1.5,espcl=3,deltacl=0.6]
            {$x$ /0.6,$y’$ /0.6,$y$ /2}
            {$-\infty$ ,$0$, $1$, $+\infty$}
            \tkzTabLine{,+,0,-,0,+,}
            \tkzTabVar{-/$-\infty$,+/$5$,-/$-1$,+/$+\infty$}
        \end{tikzpicture}
    \end{center}
    \choice
    {$1$}
    {\True $3$}
    {$2$}
    {$0$}
    \loigiai{
        Dựa vào bảng biến thiên suy ra
        $2f(x)-3=0 \Leftrightarrow f(x) =-\dfrac{3}{2}$, phương trình này có $3$ nghiệm phân biệt nên đồ thị hàm số $y=\dfrac{1}{2f(x)-3}$ có ba tiệm cận đứng.
    }
\end{ex}
%70
%71
%72
\begin{ex}%[2D1K4-1]
    Cho hàm số $y=f(x)$ có bảng biến thiên như hình bên. Đồ thị hàm số $y=\dfrac{x}{f(x)-3}$ có bao nhiêu tiệm cận đứng?
    \begin{center}
        \begin{tikzpicture}[scale=0.8]
            \tkzTabInit[nocadre=false,lgt=1.5,espcl=3,deltacl=0.6]
            {$x$ /0.6,$y’$ /0.6,$y$ /2}
            {$-\infty$ ,$-1$, $0$, $1$, $+\infty$}
            \tkzTabLine{,-,0,+,0,-,0,+,}
            \tkzTabVar{+/$+\infty$,-/$0$,+/$3$,-/$0$,+/$+\infty$}
        \end{tikzpicture}
    \end{center}
    \choice
    {$1$}
    {\True $3$}
    {$2$}
    {$4$}
    \loigiai{
        Dựa vào bảng biến thiên suy ra
        $f(x)-3=0 \Leftrightarrow f(x) =3$, phương trình này có $2$ nghiệm phân biệt khác $0$ và một nghiệm bội chẵn $x=0$ nên đồ thị hàm số $y=\dfrac{x}{f(x)-3}$ có ba tiệm cận đứng.
    }
\end{ex}
\begin{ex}%[2D1K4-1]
    Cho hàm số $y=f(x)$ có bảng biến thiên như hình bên. Đồ thị hàm số $y=\dfrac{4}{f(x)+1}$ có tiệm cận ngang là đường thẳng
    \begin{center}
        \begin{tikzpicture}[scale=0.8]
            \tkzTabInit[nocadre=false,lgt=1.5,espcl=3,deltacl=0.6]
            {$x$ /0.6,$y’$ /0.6,$y$ /2}
            {$-\infty$ ,$-1$, $2$, $+\infty$}
            \tkzTabLine{,+,0,-,0,+,}
            \tkzTabVar{-/$1$,+/$4$,-/$-5$,+/$1$}
        \end{tikzpicture}
    \end{center}
    \choice
    {$y=1$}
    {$y=-5$}
    {\True $y=2$}
    {$y=4$}
    \loigiai{
        Dựa vào bảng biến thiên suy ra
        $\lim \limits_{x \to \pm \infty} f(x)=1 \Leftrightarrow \lim \limits_{x \to \pm \infty} \dfrac{4}{f(x)+1} =2$ nên đồ thị hàm số đã cho có tiệm cận ngang là $y=2$.
    }
\end{ex}
\begin{ex}%[2D1K4-1]
    Cho hàm số $y=f(x)$ có bảng biến thiên như hình bên. Đồ thị hàm số $y=\dfrac{2-f(x)}{f(x)+3}$ có tiệm cận ngang là đường thẳng
    \begin{center}
        \begin{tikzpicture}[scale=0.8]
            \tkzTabInit[nocadre=false,lgt=1.5,espcl=3,deltacl=0.6]
            {$x$ /0.6,$y’$ /0.6,$y$ /2}
            {$-\infty$ ,$0$, $2$, $+\infty$}
            \tkzTabLine{,-,0,+,0,-,}
            \tkzTabVar{+/$+\infty$,-/$1$,+/$5$,-/$-\infty$}
        \end{tikzpicture}
    \end{center}
    \choice
    {$y=1$}
    {$y=-3$}
    {$y=2$}
    {\True $y=-1$}
    \loigiai{
        Dựa vào bảng biến thiên suy ra
        $\lim \limits_{x \to \pm \infty} f(x)=\pm \infty \Leftrightarrow \lim \limits_{x \to \pm \infty} \dfrac{2-f(x)}{f(x)+3} =-1$ nên đồ thị hàm số $y=\dfrac{2-f(x)}{f(x)+3}$ có tiệm cận ngang là $y=-1$.
    }
\end{ex}
\begin{ex}%[2D1K4-1]
    Cho hàm số $y=f(x)$ có bảng biến thiên như hình bên. Đồ thị hàm số $y=\dfrac{1}{f^2(x)-4f(x)+4}$ có bao nhiêu tiệm cận đứng?
    \begin{center}
        \begin{tikzpicture}[scale=0.8]
            \tkzTabInit[nocadre=false,lgt=1.5,espcl=3,deltacl=0.6]
            {$x$ /0.6,$y’$ /0.6,$y$ /2}
            {$-\infty$, $2$, $+\infty$}
            \tkzTabLine{,-,0,+,}
            \tkzTabVar{+/$1$,-/$-3$,+/$1$}
        \end{tikzpicture}
    \end{center}
    \choice
    {$1$}
    {$3$}
    {$2$}
    {$0$}
    \loigiai{
        Dựa vào bảng biến thiên suy ra $f^2(x)-4f(x)+4=0 \Leftrightarrow f(x)=2$, phương trình $f(x)=2$ vô nghiệm nên đồ thị hàm số đã cho không có tiệm cận đứng.
    }
\end{ex}
%83
\begin{ex}%[2D1K4-1]
    Cho hàm số $y=f(x)$ có bảng biến thiên như hình bên. Đồ thị hàm số $y=\dfrac{1}{f(3-x)-2}$ có bao nhiêu tiệm cận đứng?
    \begin{center}
        \begin{tikzpicture}[scale=0.8]
            \tkzTabInit[nocadre=false,lgt=1.5,espcl=3,deltacl=0.6]
            {$x$ /0.6,$y’$ /0.6,$y$ /2}
            {$-\infty$ ,$-2$, $2$, $+\infty$}
            \tkzTabLine{,+,0,-,0,+,}
            \tkzTabVar{-/$-\infty$,+/$3$,-/$0$,+/$+\infty$}
        \end{tikzpicture}
    \end{center}
    \choice
    {$1$}
    {\True $3$}
    {$2$}
    {$0$}
    \loigiai{
        Dựa vào bảng biến thiên suy ra $f(3-x)-2=0 \Leftrightarrow f(3-x)=2$, phương trình này có $3$ nghiệm phân biệt nên đồ thị hàm số đã cho có $3$ tiệm cận đứng.
    }
\end{ex}
\begin{ex}%[2D1G4-1]
    Cho hàm số $y=f(x)$ có bảng biến thiên như hình bên. Đồ thị hàm số $y=\dfrac{4}{f(x^2)-2}$ có bao nhiêu tiệm cận đứng?
    \begin{center}
        \begin{tikzpicture}[scale=0.8]
            \tkzTabInit[nocadre=false,lgt=1.5,espcl=3,deltacl=0.6]
            {$x$ /0.6,$y’$ /0.6,$y$ /2}
            {$-\infty$ ,$0$, $3$, $+\infty$}
            \tkzTabLine{,-,0,+,d,-,}
            \tkzTabVar{+/$8$,-/$1$,+/$4$,-/$2$}
        \end{tikzpicture}
    \end{center}
    \choice
    {$5$}
    {$3$}
    {\True $2$}
    {$4$}
    \loigiai{
        Dựa vào bảng biến thiên suy ra
        $f(x^2)-2=0 \Leftrightarrow f(x^2) =2$. Kẻ đường thẳng $y=2$ ta thấy đường thẳng cắt đồ thị hàm số tại hai điểm phân biệt. Suy ra
        $$\hoac{&x^2=a \; (a<0)\\&x^2=b \; (b >0)} \Rightarrow x=\pm \sqrt{b}.$$
        Do đó đồ thị hàm số đã cho có $2$ tiệm cận đứng.
    }
\end{ex}%89
\begin{ex}%[2D1G4-1]
    Cho hàm số $y=f(x)$ có bảng biến thiên như hình bên. Đồ thị hàm số $y=\dfrac{2}{f(|x|)-3}$ có bao nhiêu tiệm cận ngang?
    \begin{center}
        \begin{tikzpicture}[scale=0.8]
            \tkzTabInit[nocadre=false,lgt=1.5,espcl=3,deltacl=0.6]
            {$x$ /0.6,$y’$ /0.6,$y$ /2}
            {$-\infty$ ,$0$, $2$, $+\infty$}
            \tkzTabLine{,+,0,-,0,+,}
            \tkzTabVar{-/$-\infty$,+/$3$,-/$-1$,+/$+\infty$}
        \end{tikzpicture}
    \end{center}
    \choice
    {$4$}
    {\True $3$}
    {$5$}
    {$6$}
    \loigiai{
        Dựa vào bảng biến thiên suy ra
        $f(|x|)-3=0 \Leftrightarrow f(|x|) =3$.\\
        Bảng biến thiên hàm số $y=f(|x|)$ như sau
        \begin{center}
            \begin{tikzpicture}[scale=0.8]
                \tkzTabInit[nocadre=false,lgt=1.5,espcl=3,deltacl=0.6]
                {$x$ /0.6,$y’$ /0.6,$y$ /2}
                {$-\infty$ ,$-2$, $0$, $2$, $+\infty$}
                \tkzTabLine{,-,0,+,0,-,0,+,}
                \tkzTabVar{+/$+\infty$,-/$-1$,+/$3$,-/$-1$,+/$+\infty$}
            \end{tikzpicture}
        \end{center}
        Dựa vào bảng biến thiên hàm số $y=f(|x|)$, phương trình $f(|x|) =3$ có ba nghiệm phân biệt, do đó đồ thị hàm số $y=\dfrac{2}{f(|x|)-3}$ có $3$ tiệm cận đứng.
    }
\end{ex}
\begin{ex}
    \immini{ %Câu 90
        Cho hàm số bậc ba $f(x)= ax^3 +bx^2 +cx +d$ có đồ thị như hình vẽ bên. Đồ thị hàm số $g(x) = \dfrac{\sqrt{x+1}}{(x-3)\cdot f(x)}$ có bao nhiêu đường tiệm cận đứng?
        \choice
        {5}
        {2}
        {4}
        {\True 3}}{\begin{tikzpicture}[scale=.5, font=\footnotesize, line join=round, line cap=round, >=stealth]
            \def\xmin{-3}\def\xmax{3}\def\ymin{-5}\def\ymax{1}
            \draw[->] (\xmin-0.2,0)--(\xmax+0.2,0) node[below] {\footnotesize $x$};
            \draw[->] (0,\ymin-0.2)--(0,\ymax+0.2) node[right] {\footnotesize $y$};
            \draw (0,0) node [below left] {\footnotesize $O$};
            \foreach \x in {-1}\draw (\x,-0.1)--(\x,0.1) node [above] {\footnotesize $\x$};
            \foreach \x in {2}\draw (\x,-0.1)--(\x,0.1) node [above right] {\footnotesize $\x$};
            \foreach \y in {}\draw (-0.1,\y)--(0.1,\y) node [right] {\footnotesize $\y$};
            \clip (\xmin,\ymin) rectangle (\xmax,\ymax);
            \draw[smooth,samples=200,domain=\xmin:\xmax] plot (\x,{1*((\x)^3)+0*((\x)^2)+-3*(\x)+-2});
        \end{tikzpicture}
    }
    \loigiai{
        * Điều kiện: $\heva{&x \ne 3\\&f(x) \ne 0\\&x \ge -1.}$\\
        Nhìn hình vẽ ta thấy
        $f(x)=0\Leftrightarrow \hoac{&x=-1&(\text{nghiệm kép}) \\&x=2&(\text{nghiệm đơn}).}$\\
        Vậy $g(x) = \dfrac{\sqrt{x+1}}{(x-3)\cdot a(x+1)^2 (x-2)}.$ \\
        Đồ thị hàm số $g(x)$ có 3 đường tiệm cận đứng.}
\end{ex}
\begin{ex}
    \immini{ %Câu 92.
        Đường cong ở hình bên là đồ thị của hàm số $y = ax^3 +bx^2 +cx+d$. Đồ thị hàm số $y =\dfrac{(2x+1)\sqrt{x-1}}{x\cdot f(x-2)}$ có tất cả bao nhiêu tiệm cận đứng?
        \choice
        {1}
        {3}
        {4}
        {\True 2}}{\begin{tikzpicture}[scale=.6, font=\footnotesize, line join=round, line cap=round, >=stealth]
            \def\xmin{-3}\def\xmax{3}\def\ymin{-3}\def\ymax{3}
            \draw[->] (\xmin-0.2,0)--(\xmax+0.2,0) node[below] {\footnotesize $x$};
            \draw[->] (0,\ymin-0.2)--(0,\ymax+0.2) node[right] {\footnotesize $y$};
            \draw (0,0) node [below left] {\footnotesize $O$};
            \foreach \x in {-2}\draw (\x,-0.1)--(\x,0.1) node [above left] {\footnotesize $\x$};
            \foreach \x in {2}\draw (\x,-0.1)--(\x,0.1) node [above right] {\footnotesize $\x$};
            \foreach \y in {}\draw (-0.1,\y)--(0.1,\y) node [right] {\footnotesize $\y$};
            \clip (\xmin,\ymin) rectangle (\xmax,\ymax);
            \draw[smooth,samples=200,domain=\xmin:\xmax] plot (\x,{(2/3)*((\x)^3)+0*((\x)^2)+-(8/3)*(\x)});
    \end{tikzpicture}}
    \loigiai{
        * Điều kiện: $\heva{&x \ne 0\\&f(x-2) \ne 0\\&x \ge 1.}$\\
        Nhìn hình vẽ ta thấy
        $f(x-2)=0\Leftrightarrow \hoac{&x-2=-2\\&x-2=0\\&x-2=2}\Leftrightarrow \hoac{&x=0&(\text{không thỏa mãn})\\&x=2&(\text{nghiệm đơn})\\&x=4&(\text{nghiệm đơn}).}$\\
        Vậy $g(x) =\dfrac{(2x+1)\sqrt{x-1}}{x\cdot f(x-2)}=\dfrac{(x-1)\sqrt{x+2}}{x\cdot ax(x-2)(x-4)}.$ \\
        Đồ thị hàm số $g(x)$ có 2 đường tiệm cận đứng.}
\end{ex}
\begin{ex}
    \immini{ %Câu 93.
        Cho hàm số $y= f(x)$ có đồ thị cắt trục hoành tại đúng 3 điểm như hình bên. Đồ thị hàm số $y =\dfrac{(x+2)\sqrt{3-x}}{f(|x|)}$
        có tất cả bao nhiêu tiệm cận đứng?
        \choice
        {1}
        {3}
        {4}
        {\True 2}}{\begin{tikzpicture}[scale=.5, font=\footnotesize, line join=round, line cap=round, >=stealth]
            \def\xmin{-2}\def\xmax{5}\def\ymin{-3}\def\ymax{5}
            \draw[->] (\xmin-0.2,0)--(\xmax+0.2,0) node[below] {\footnotesize $x$};
            \draw[->] (0,\ymin-0.2)--(0,\ymax+0.2) node[right] {\footnotesize $y$};
            \draw (0,0) node [below left] {\footnotesize $O$};
            \foreach \x in {-1,2,4}\draw (\x,-0.1)--(\x,0.1) node [above left] {\footnotesize $\x$};
            \foreach \y in {}\draw (-0.1,\y)--(0.1,\y) node [right] {\footnotesize $\y$};
            \clip (\xmin,\ymin) rectangle (\xmax,\ymax);
            \draw[smooth,samples=200,domain=-1.2:0] plot(\x,{0-8.48*(\x)^(2.0)-5.48*(\x)+3.0});
            \draw[smooth,samples=200,domain=0:2]
            plot(\x,{0-2.7989489689153735*(\x)^(3.0)+8.326740175055514*(\x)^(2.0)-6.957684474449535*(\x)+3.0});
            \draw[smooth,samples=200,domain=2:5]
            plot(\x,{2.395330112721417*(\x)^(2.0)-14.371980676328501*(\x)+19.162640901771336});
    \end{tikzpicture}}
    \loigiai{
        * Điều kiện: $\heva{&f(|x|) \ne 0\\&x \le 3.}$\\
        Nhìn hình vẽ ta thấy
        $f(|x|)=0\Leftrightarrow \hoac{&|x|=-1\\&|x|=2\\&|x|=4}\Leftrightarrow \hoac{&x=\pm 2&(\text{nghiệm đơn})\\&x=- 4&(\text{nghiệm đơn})\\&x=4&(\text{không thỏa mãn}).}$\\
        Vậy $y =\dfrac{(x+2)\sqrt{3-x}}{a(x-2)(x+2)(x+4)(x-4)}$ \\
        Đồ thị hàm số có 2 đường tiệm cận đứng.}
\end{ex}
\begin{ex}
    \immini{ %Câu 94.
        Đường cong ở hình bên là đồ thị của hàm số $y = ax^3 +bx^2 +cx+d$. Đồ thị hàm số $y =\dfrac{(2x+1)\sqrt{1-x}}{f(|x|)}$ có tất cả bao nhiều tiệm cận đứng?
        \choice
        { 1}
        {3}
        {4}
        {\True 2}}{\begin{tikzpicture}[scale=.8, font=\footnotesize, line join=round, line cap=round, >=stealth]
            \def\xmin{-1}\def\xmax{2}\def\ymin{-1.5}\def\ymax{1.5}
            \draw[->] (\xmin-0.2,0)--(\xmax+0.2,0) node[below] {\footnotesize $x$};
            \draw[->] (0,\ymin-0.2)--(0,\ymax+0.2) node[right] {\footnotesize $y$};
            \draw (0.15,0) node [below left] {\footnotesize $O$};
            \foreach \x in {}\draw (\x,0.1)--(\x,-0.1) node [below] {\footnotesize $\x$};
            \foreach \y in {-1,1}\draw (0.1,\y)--(-0.1,\y) node [left] {\footnotesize $\y$};
            \clip (\xmin,\ymin) rectangle (\xmax,\ymax);
            \draw[smooth,samples=200,domain=\xmin:\xmax] plot (\x,{4*((\x)^3)+-6*((\x)^2)+0*(\x)+1});
            \draw[dashed] (0.5,0)--(0.5,0.0)--(0,0.0);
            \draw (0.5,-1pt)--(0.5,1pt) node [above] {\footnotesize $\frac{1}{2}$};
            \draw (-0.7,-1pt)--(-0.7,1pt) node [above] {\footnotesize $-\frac{1}{2}$};
            \draw (1,-1pt)--(1,1pt) node [above] {\footnotesize $1$};
            \draw[dashed] (0.0,0)--(0.0,1.0)--(0,1.0);
            \draw[dashed] (1.0,0)--(1.0,-1.0)--(0,-1.0);
    \end{tikzpicture}}
    \loigiai{
        * Điều kiện: $\heva{&f(|x|) \ne 0\\&x \le 1.}$\\
        Nhìn hình vẽ ta thấy
        $f(|x|)=0\Leftrightarrow \hoac{&|x|=-\dfrac{1}{2}\\&|x|=\dfrac{1}{2}\\&|x|=x_1>1}\Leftrightarrow \hoac{&x=\pm \dfrac{1}{2}&(\text{hai nghiệm đơn})\\&x=- x_1&(\text{nghiệm đơn})\\&x=x_1&(\text{không thỏa mãn}).}$\\
        Vậy $y =\dfrac{(2x+1)\sqrt{1-x}}{f(|x|)}=\dfrac{(2x+1)\sqrt{1-x}}{a\left(x-\dfrac{1}{2}\right)\left(x+\dfrac{1}{2}\right)(x+x_1)(x-x_1)}$ \\
        Đồ thị hàm số có 2 đường tiệm cận đứng.}
\end{ex}
\begin{ex}
    \immini{ %Câu 96.
        Cho đồ thị hàm số $y =f(x)$ và trục hoành có đúng 2 điểm chung như hình bên. Đồ thị hàm số $y =\dfrac{(x-1)\sqrt{3-x}}{f(x^2)}$ có tất cả bao nhiêu tiệm cận đứng?
        \choice
        {1}
        {3}
        {4}
        {\True 2}}{\begin{tikzpicture}[scale=.8, font=\footnotesize, line join=round, line cap=round, >=stealth]
            \def\xmin{-1.5}\def\xmax{2}\def\ymin{-1}\def\ymax{4.5}
            \draw[->] (\xmin-0.2,0)--(\xmax+0.2,0) node[below] {\footnotesize $x$};
            \draw[->] (0,\ymin-0.2)--(0,\ymax+0.2) node[right] {\footnotesize $y$};
            \draw (0,0) node [below left] {\footnotesize $O$};
            \foreach \x in {1}\draw (\x,0.1)--(\x,-0.1) node [below] {\footnotesize $\x$};
            \foreach \x in {-1}\draw (\x,0.1)--(\x,-0.1) node [below left] {\footnotesize $\x$};
            \clip (\xmin,\ymin) rectangle (\xmax,\ymax);
            \draw[smooth,samples=200,domain=-1.1:0] plot(\x,{21.044670464836045*(\x)^(3.0)+24.701786337609526*(\x)^(2.0)+5.65711587277348*(\x)+2.0});
            \draw[smooth,samples=200,domain=0:\xmax] plot(\x,{10.591704641658401*(\x)^(3.0)-19.26315454354621*(\x)^(2.0)+6.6714499018878115*(\x)+2.0});
    \end{tikzpicture}}
    \loigiai{
        * Điều kiện: $\heva{&f(x^2) \ne 0\\&x \le 3.}$\\
        Nhìn hình vẽ ta thấy
        $f(x^2)=0\Leftrightarrow \hoac{&x^2=-1\\&x^2=1}\Leftrightarrow x=\pm 1\,(\text{nghiệm kép}).$\\
        Vậy $y=\dfrac{(x-1)\sqrt{3-x}}{f(x^2)}=\dfrac{(x-1)\sqrt{3-x}}{(x-1)^2(x+1)^2}$ \\
        Đồ thị hàm số có 2 đường tiệm cận đứng.}
\end{ex}
\begin{ex}%[2D1G4-3]%Câu 52
    Cho hàm số $y=ax^3+bx^2+cx+d$ có đồ thị như hình vẽ. Đồ thị của hàm số $g(x)=\dfrac{x^2-x}{f^2(x)-2f(x)}$ có bao nhiêu đường tiệm cận đứng?
    \choice
    {$2$}
    {$3$}
    {\True $4$}
    {$5$}
    \begin{center}
        \begin{tikzpicture}[thick,>=stealth,x=1cm,y=1cm,scale=.7]
            \draw[thin,color=gray!50] (-3.3,-1.3) grid (3.9,5.9);
            \draw[->] (-3.2,0) -- (4.2,0) node[right] {$x$};
            \draw[->] (0,-1.2) -- (0,5.2) node[above] {$y$};
            \draw[color=blue, domain=-2.15:2.15,samples=300] plot (\x,{(\x)^3-3*(\x)+2}) node[right] {$y=f(x)$};
            \draw (-2,0) circle (1.5pt) node[below left]{$-2$};
            \draw (-1,0) circle (1.5pt) node[below]{$-1$};
            \draw (0,0) circle (1.5pt) node[above left]{$O$};
            \draw (1,0) circle (1.5pt) node[below]{$1$};
            \draw (0,4) circle (1.5pt) node[right]{$4$};
            \draw (-1,4) circle (1.5pt);
            \draw[dashed] (-1,0)--(-1,4)--(0,4);
            \draw[red] (-3,2)--(3.2,2);
            \draw[red] (3.5,2) node[right]{$f(x)=2$};
        \end{tikzpicture}
    \end{center}
    \loigiai{
        Xét phương trình $f^2(x)-2f(x)=0 \Leftrightarrow \hoac{&f(x)=0\\&f(x)=2}\Leftrightarrow \hoac{&x=1 \, (\textrm{nghiệm kép trùng nghiệm đơn ở tử số})\\&x=-2\, (\textrm{nghiệm đơn khác nghiệm của tử})\\&x=a\in(-2; -1)\\&x=0\, (\textrm{nghiệm đơn trùng nghiệm ở tử})\\&x=b\in(1; 2)}$\\
        \textbf{Kết luận:} Đồ thị hàm số có $4$ đường tiệm cận đứng.
    }
\end{ex}
\begin{ex}%[Thi thử L3, Lương Thế Vinh, Hà Nội, 2018]%[Phạm Toàn, Dự án (12EX-10)]%[2D1G4-3]%
    \immini{Cho hàm số $y=f(x)$ có đạo hàm liên tục trên $\mathbb{R}$. Đồ thị hàm $f(x)$ như hình vẽ. Số đường tiệm cận đứng của đồ thị hàm số $y=\dfrac{x^2-1}{f^2(x)-4f(x)}$ bằng
        \choice
        {$3$}
        {$1$}
        {$2$}
        {\True $4$}
    }{\begin{tikzpicture}[>=stealth,x=1cm,y=0.75cm,scale=0.7]
            \draw[->] (-2.5,0)--(0,0)%
            node[below right]{$O$}--(2.5,0) node[below]{$x$};
            \draw[->] (0,-2) --(0,5) node[right]{$y$};
            \foreach \x in {-1,1}{
                \draw (\x,0) node[below]{\footnotesize $\x$} circle (1pt);%Ox
            }
            \foreach \y in {2,4}{
                \draw (0,\y) node[right]{\footnotesize $\y$} circle (1pt);%Oy
            }
            \draw[samples=100,domain=-2.05:2] plot (\x,{(\x -1)^2*(\x+2)});
            \draw [dashed] (-1,0)--(-1,4)--(0,4);
            \draw(-1,4) circle (1pt);
    \end{tikzpicture}}
    \loigiai{Xét $f^2(x)-4f(x)=0\Leftrightarrow \hoac{& f(x)=0\\ &f(x)=4.}$\\
        Xét $f(x)=0$ có hai nghiệm, nghiệm $x_1\ne \pm 1$ và nghiệm $x_2=1$ là nghiệm bội (do đồ thị tiếp xúc với trục hoành tại $x=1$. Trường hợp này có $2$ tiệm cận đứng.\\
        Xét $f(x)=4$ có hai nghiệm, nghiệm $x_3\ne \pm 1$ và nghiệm $x_4=-1$ là nghiệm bội (do đồ thị tiếp xúc với đường thẳng $y=4$ tại $x=-1$. Trường hợp này có $2$ tiệm cận đứng.\\
        Vậy đồ thị có $4$ tiệm cận đứng.}
\end{ex}
\begin{ex}%[Thi thử, Trường THPT Lý Thái Tổ - Bắc Ninh, 2019]%[Duong Xuan Loi, 12EX3]%[2D1G4-3]%
    \immini{
        Cho hàm số $f(x)$ có đồ thị như hình bên. Số đường tiệm cận đứng của đồ thị hàm số
        $y=\dfrac{(x^2-4)(x^2+2x)}{[f(x)]^2+2f(x)-3}$ là
        \choice
        {\True $4$}
        {$5$}
        {$3$}
        {$2$}
    }{
        \begin{tikzpicture}[scale=0.5, font=\footnotesize, line join=round, line cap=round, >=stealth]
            \def\a{1} \def\b{-8} \def\c{1} % Hệ số
            \def\xt{-3.7} \def\xp{4} \def\yt{2} \def\yd{-3.7} % x_trái, x_phải, y_trên, y_dưới (giới hạn)
            \draw[->] (\xt,0)--(\xp,0) node [below]{$x$};
            \draw[->] (0,\yd)--(0,\yt) node [left]{$y$};
            \node at (0,0) [below left]{$O$};
            \clip (\xt-0.1,\yd+0.1) rectangle (\xp-0.1,\yt-0.1);
            \draw[smooth,samples=300] plot(\x,{1/4*(\a*(\x)^4+\b*(\x)^2)+\c});
            \draw[dashed] (-2,0)node[above]{$-2$}--(-2,-3)--(2,-3)--(2,0)node[above]{$2$};
            \node at (0,-3)[above left]{$-3$};
            \node at (-3,0)[above left]{$-3$};
            \node at (0,1)[above right]{$1$};
            \node at (3,0)[above right]{$3$};
            \fill (0,0) circle (1pt) (0,-3) circle (1pt) (2,0) circle (1pt) (-2,0) circle (1pt) (-3,0) circle (1pt) (0,1) circle (1pt) (3,0) circle (1pt);
        \end{tikzpicture}
    }
    \loigiai{
        Ta có $y=\dfrac{(x^2-4)(x^2+2x)}{[f(x)]^2+2f(x)-3}$ có các nghiệm ở tử là $x=0$ (bội $1$), $x=2$ (bội $1$), $x=-2$ (bội $2$).\\
        Mặt khác, từ đồ thị $f(x)$ ta thấy hàm số $y=\dfrac{(x^2-4)(x^2+2x)}{[f(x)]^2+2f(x)-3}$ có các nghiệm ở mẫu là
        $f^2(x)+2f(x)-3=0\Leftrightarrow \hoac{& f(x)=1 \\ & f(x)=-3}
        \Leftrightarrow \hoac{& x=0,x=x_1,x=x_2 \\ & x=-2,x=2.}$\\
        Trong đó nghiệm $x=0$, $x=-2$, $x=2$ đều có bội $2$ và $x_1$, $x_2$ khác các nghiệm của tử.\\
        So sánh bội nghiệm ở mẫu và bội nghiệm ở tử thì thấy đồ thị có các tiệm cận đứng là $x=0$, $x=2$; $x=x_1$; $x=x_2$.
    }
\end{ex}
\begin{ex}%[Thi thử, THPT Sơn Tây, Hà Nội, 2019]%[Huỳnh Xuân Tín, 12EX3]%[2D1G4-3]%
    \immini{Cho hàm số $ f(x)=(x+3)(x+1)^2(x-1)(x-3)$ có đồ thị như hình vẽ. Đồ thị hàm số $ g(x)=\dfrac{\sqrt{x-1}}{f^2(x)-9f(x)}$ có bao nhiêu tiệm cận đứng và tiệm cận ngang?
        \choice
        {$3$}
        {\True$ 4$}
        {$ 9$}
        { $8$}
    }{\begin{tikzpicture}[scale=0.3, font=\footnotesize, line join=round, line cap=round, >=stealth]
            %\draw[dashed, line width=0.1pt, gray] (-3.2,-5.5) grid (5.2,4.5);
            \draw[->] (-3.5,0)--(0,0) node[below right]{$O$}--(3.6,0) node[below]{$x$};
            \draw[fill=black] (0,0) circle (1pt);
            \draw[->] (0,-7.7) --(0,6.5) node[right]{$y$};
            \foreach \x in {-3,-1,3}{
                \draw[fill=black] (\x,0) node[below left]{$\x$} circle (1pt);}
            \draw[fill=black] (1,0) node[below right]{$1$} circle (1pt);
            \draw[fill=black] (0,1.35) node[above left]{$9$} circle (1pt);
            \draw [black, domain=-3.2:3.18, samples=100] %
            plot(\x,{0.15*(\x+3)*(\x+1)^2*(\x-1)*(\x-3)});
    \end{tikzpicture}}
    \loigiai{Điều kiện xác định của hàm số $g(x)$ là $\heva{&x\ge1\\ &f^2(x)-9f(x)\not=0.}$\\
        Từ $f^2(x)-9f(x)=0\Leftrightarrow \hoac{&f(x)=0\\&f(x)=9.}$\\
        Với $f(x)=0$ có nghiệm là $x=\pm 1, x=\pm 3$.\\
        Dựa vào đồ thị ta thấy nghiệm của phương trình $f(x)=9$ là hoành độ giao điểm của đường thẳng $y=9$ với đồ thị hàm số $y=f(x)$ nên có nghiệm là $-3<x_3<x_2<-1<0<x_1<1<3<x_0$.\\
        Do đó tập xác định của hàm số $y=g(x)$ là $\mathscr{D}=\left[1;+\infty \right)\setminus\left\lbrace1;3;x_0 \right\rbrace $.\\
        Khi đó ta có \begin{itemize}
            \item $\lim\limits_{x\rightarrow1^+ } g(x)=\lim\limits_{x\rightarrow1^+ }\dfrac{\sqrt{x-1}}{f(x)\left(f(x)-9 \right)}=+\infty$ (vì $x$ tiến gần bên phải $1$ thì $f(x)<0, f(x)-9<0$), suy ra đường thẳng $x=1$ là tiệm cận đứng.
            \item $\lim\limits_{x\rightarrow3^+ } g(x)=\lim\limits_{x\rightarrow3^+ }\dfrac{\sqrt{x-1}}{f(x)\left(f(x)-9 \right)}=-\infty$ (vì $x$ tiến gần bên phải $3$ thì $f(x)>0, f(x)-9<0$), suy ra đường thẳng $x=3$ là tiệm cận đứng.
            \item $\lim\limits_{x\rightarrow x_0^+} g(x)=\lim\limits_{x\rightarrow x_0^+ }\dfrac{\sqrt{x-1}}{f(x)\left(f(x)-9 \right)}=+\infty$ (vì $x$ tiến gần bên phải $x_0$ thì $f(x)>0, f(x)-9>0$), suy ra đường thẳng $x=x_0$ là tiệm cận đứng.
        \end{itemize}
        Và $\lim\limits_{x\rightarrow +\infty} g(x)=\lim\limits_{x\rightarrow +\infty }\dfrac{\sqrt{x-1}}{f(x)\left(f(x)-9 \right)}=0$ (vì bậc ở mẫu của $y=g(x)$ là $10$ và bậc tử của nó là $\dfrac{1}{2}$). Do vậy đồ thị hàm số $y=g(x)$ có một tiệm cận ngang là đường thẳng $y=0$.\\
        Vậy đồ thị hàm số $y=g(x)$ có bốn tiệm cận ngang và đứng. }
\end{ex}
\begin{ex}%[Thi thử, Chuyên Quang Trung-Bình Phước, 2021,lần 1]%[Trần Hòa, 12EX6]%[2D1G4-3]%
    \immini{Cho hàm số $y=f(x)=ax^3+bx^2+cx+d$, có đồ thị như hình vẽ. Số đường tiệm cận đứng của đồ thị hàm số $y=\dfrac{x^2+x-2}{f^2(x)-f(x)}$ là
        \choice
        {$3$}
        {$2$}
        {\True $4$}
        {$5$}}
    {\begin{tikzpicture}[scale=.5, font=\footnotesize, line join=round, line cap=round, >=stealth]
            \draw[->] (-2.5,0)--(0,0) node[below right]{$O$}--(2,0) node[below]{$x$};
            \draw[->] (0,-.5) --(0,4.5) node[right]{$y$};
            \draw [domain=-2.05:2.05, samples=100] %
            plot (\x, {(\x+2)*(\x-1)^2});
            \draw[fill] (0,0) circle (1pt);
            \foreach \x/\g in {-2/140,-1/-90,1/-90}
            \draw[fill] (\x,0) circle(.5pt)node [shift={(\g:.3)}] {$\x$};
            \foreach \y/\g in {2/0,4/0}
            \draw[fill] (0,\y) circle(.5pt)node [shift={(\g:.3)}] {$\y$};
            \draw[dashed] (-1,0)--(-1,4)--(0,4);
    \end{tikzpicture}}
    \loigiai{
        \begin{itemize}
            \item $x^2+x-2=(x-1)(x+2)$.\\
            \item Dựa vào đồ thị hàm số $y=f(x)$ ta có $f^2(x)-f(x)=0\Leftrightarrow\hoac{&f(x)=0\\&f(x)=1.}$\\
            $f(x)=0\Leftrightarrow x=-2$, $x=1$ (nghiệm kép).\\
            $f(x)=1\Leftrightarrow\hoac{&x=x_1,(x_1\in (-2;-1))\\&x=x_2,(x_2\in (0;1))\\&x=x_3,(x_3>1). }$
            \item Do đó $y=\dfrac{(x-1)(x+2)}{a^2(x+2)(x-1)^2(x-x_1)(x-x_2)(x-x_3)}$.
        \end{itemize}
        Suy ra đồ thị có các đườn tiệm cận đứng $x=1$, $x=x_1$, $x=x_2$, $x=x_3$.
    }
\end{ex}
\begin{ex}%[Đề thi hết học kì 2, Bình Minh, Ninh Bình 2018]%[Nguyễn Tuấn Anh, dự án EX9]%[2D1G4-3]%
    \immini{Cho hàm số bậc ba $f(x)=ax^3+bx^2+cx+d$ có đồ thị như hình vẽ bên dưới. Hỏi đồ thị hàm số $g(x)=\dfrac{(x^2-3x+2)\sqrt{x-1}}{x[f^2(x)-f(x)]}$ có bao nhiêu tiệm cận đứng?
        \choice
        {$5$}
        {$6$}
        {\True $3$}
        {$4$}
    }{
        \begin{tikzpicture}[line width=1.0pt,line join=round,>=stealth,x=1cm,y=1cm,scale=1.0]
            \draw[->,line width = 1pt] (-1,0)--(0,0) node[below right]{$O$}--(4,0) node[below]{$x$};
            \draw[->,line width = 1pt] (0,-1.5) --(0,2.5) node[right]{$y$};
            \foreach \x in {1,2}{
                \draw (\x,0) node[below]{$\x$} circle (1pt);
            }
            \foreach \y in {1}{
                \draw (0,\y) node[left]{$\y$} circle (1pt);
            }
            \clip(-0.8,-1) rectangle (3.8,2.3);
            \draw [line width=1.0pt, thick, domain=-0.5:3.5, samples=100]%,domain=-1.5:3] %
            plot (\x, {(5*(\x)-4)*((\x)-2)^2});
            \draw [dash pattern=on 4pt off 4pt] (1.,0.)-- (1.,1.)-- (0.,1.);
            \draw (1,1) circle (1pt);
        \end{tikzpicture}
    }
    \loigiai{
        Điều kiện $\heva{&x\geq 1\\ &x\ne 0\\ &f^2(x)-f(x)\ne 0}\Leftrightarrow \heva{&x\geq 1\\ &f(x)\ne 0\\ & f(x)\ne 1.}$\\
        Dựa vào đồ thị hàm số $y=f(x)$, ta thấy $f(x)=0$ có hai nghiệm, một nghiệm $x_1<1$ và một nghiệm kép bằng $2$. Do đó ta biểu diễn được $f(x)$ dưới dạng
        $$ f(x)=a(x-x_1)(x-2)^2. $$
        Dựa vào đồ thị hàm số $y=f(x)$, ta thấy phương trình $f(x)=1$ có ba nghiệm $1,x_2, x_3$, với $1<x_2<2<x_3$. Do đó ta biểu diễn được $f(x)-1$ dưới dạng
        $$ f(x)-1=a(x-1)(x-x_2)(x-x_3). $$
        Lúc này điều kiện được viết lại như sau $\heva{&x>1\\ &x\ne x_2, x\ne 2, x\ne x_3.}$\\
        Với điều kiện đó thì $g(x)$ được viết lại là
        $$ g(x)=\dfrac{\sqrt{x-1}}{a^2x(x-x_1)(x-x_2)(x-2)(x-x_3)}. $$
        Ta có
        \begin{align*}
            &\lim\limits_{x\to 1^+}g(x)=\lim\limits_{x\to 1^+}\dfrac{\sqrt{x-1}}{a^2x(x-x_1)(x-x_2)(x-2)(x-x_3)}=0,\\
            & (x=1\mbox{ \textbf{không} là tiệm cận đứng}) \\
            &\lim\limits_{x\to x_2^+}g(x)=\lim\limits_{x\to x_2^+}\dfrac{\sqrt{x-1}}{a^2x(x-x_1)(x-x_2)(x-2)(x-x_3)}=+\infty,\\
            & (x=x_2\mbox{ là tiệm cận đứng}) \\
            &\lim\limits_{x\to 2^+}g(x)=\lim\limits_{x\to 2^+}\dfrac{\sqrt{x-1}}{a^2x(x-x_1)(x-x_2)(x-2)(x-x_3)}=-\infty,\\
            & (x=2\mbox{ là tiệm cận đứng}) \\
            &\lim\limits_{x\to x_3^+}g(x)=\lim\limits_{x\to x_3^+}\dfrac{\sqrt{x-1}}{a^2x(x-x_1)(x-x_2)(x-2)(x-x_3)}=+\infty,\\
            & (x=x_3\mbox{ là tiệm cận đứng}) \\
        \end{align*}
        Vậy đồ thị hàm số $g(x)$ có tất cả $3$ tiệm cận đứng.
    }
\end{ex}
\begin{ex}%[VDC5-Đỗ Đường Hiếu]%[2D1G4-3]%
    \immini{Cho hàm số $f(x)=(x+3)(x+1)^2(x-1)(x-3)$ có đồ thị như hình vẽ. Đồ thị hàm số $g(x)=\dfrac{\sqrt{x-1}}{f^2(x)-9f(x)}$ có bao nhiêu tiệm cận đứng và tiệm cận ngang?
        \choice
        {$3$}
        {\True $4$}
        {$9$}
        {$8$}}
    {\begin{tikzpicture}[xscale=0.8,yscale=0.05, line join=round, line cap=round,font=\footnotesize,>=stealth]
            \draw[->] (-4,0)--(4,0) node[below]{$x$};
            \draw[->] (0,-56)--(0,30) node[left]{$y$};
            \coordinate[label=below left:$O$] (O) at (0,0);
            \draw (-1,0) node[below] { $-1$}(1,0) node[below] { $1$};
            \draw (-3,0) node[below left] { $-3$};
            \draw (3,0) node[below right] { $3$};
            \clip (-3.3,-60) rectangle (3.5,26);
            \draw[smooth,samples=300,domain=-3.5:3.5] plot(\x,{(\x+3)*(\x+1)^2*(\x-1)*(\x-3)});
            \foreach \x in {-3,-1,1,3}
            \draw[shift={(\x,0)},color=black] (0pt,20pt) -- (0pt,-20pt);
            \draw[shift={(0,9)},color=black] (2pt,0pt) -- (-2pt,0pt) node[left] {$9$};
        \end{tikzpicture}
    }
    \loigiai{%GV tổng quát hóa bài toán:
        Cho hàm số đa thức $y=f(x)$ có đồ thị $(C)$. Tìm số đường tiệm cận của đồ thị hàm số $g(x)=\dfrac{\sqrt{ax+b}}{P\left(f(x) \right) }$, trong đó $P\left(f(x) \right)$ là một đa thức của $f(x)$.
        Nếu $a>0$ thì $\lim\limits_{x\to +\infty}g(x)=0$.\\
        Nếu $a<0$ thì $\lim\limits_{x\to -\infty}g(x)=0$.\\
        Do đó đồ thị hàm số $y=g(x)$ luôn có duy nhất một đường tiệm cận ngang là $y=0$.\\
        Gọi $x=x_0$ là một nghiệm của phương trình $P\left(f(x) \right) =0$ thỏa mãn điều kiện $ax+b\ge 0$. Rõ ràng khi đó $\lim\limits_{x\to x_0^+}g(x)=+\infty$ hoặc $\lim\limits_{x\to x_0^+}g(x)=-\infty$.\\
        Bởi vậy, số đường tiệm cận đứng của đồ thị hàm số $y=g(x)$ chính là số nghiệm của phương trình $P\left(f(x) \right) =0$ thỏa mãn điều kiện $ax+b\ge 0$.
        \immini{Ta có $f^2(x)-9f(x)=0\Leftrightarrow \hoac{&f(x)=0\\&f(x)=9.}$\\
            \begin{itemize}
                \item $f(x)=0$ có các nghiệm thuộc $\left[1;+\infty\right)$ là $x=1$ và $x=3$.
                \item Đường thẳng $y=9$ cắt đồ thị hàm số $y=f(x)$ tại duy nhất một điểm có hoành độ thuộc $\left[1;+\infty\right)$ là $x=a>3$.
            \end{itemize}
        }
        {\begin{tikzpicture}[xscale=0.8,yscale=0.05, line join=round, line cap=round,font=\footnotesize,>=stealth]
                \draw[->] (-4,0)--(4,0) node[below]{$x$};
                \draw[->] (0,-56)--(0,30) node[left]{$y$};
                \coordinate[label=below left:$O$] (O) at (0,0);
                \draw (-4,9)--(4,9);
                \draw (-1,0) node[below] { $-1$}(1,0) node[below] { $1$};
                \draw (-3,0) node[below left] { $-3$};
                \draw (3,0) node[below right] { $3$};
                \clip (-3.3,-60) rectangle (3.5,26);
                \draw[smooth,samples=300,domain=-3.5:3.5] plot(\x,{(\x+3)*(\x+1)^2*(\x-1)*(\x-3)});
                \foreach \x in {-3,-1,1,3}
                \draw[shift={(\x,0)},color=black] (0pt,20pt) -- (0pt,-20pt);
                \draw[shift={(0,9)},color=black] (2pt,0pt) -- (-2pt,0pt) node[above left] {$9$};
        \end{tikzpicture}}
        \noindent
        Bởi vậy, hàm số $g(x)=\dfrac{\sqrt{x-1}}{f^2(x)-9f(x)}$ có tập xác định là $\mathscr D=\left[1;3\right) \cup \left(3;a\right) \cup\left( a;+\infty\right)$.\\
        Khi đó ta có
        \begin{itemize}
            \item $\lim\limits_{x\to+\infty}g(x)=0$ nên đồ thị hàm số $y=g(x)$ có một đường tiệm cận ngang là đường thẳng $y=0$.
            \item $\lim\limits_{x\to 1^+}g(x)=\lim\limits_{x\to 1^+}\dfrac{\sqrt{x-1}}{f(x)\left[f(x)-9\right] }=+\infty$;\\
            $\lim\limits_{x\to 3^+}g(x)=\lim\limits_{x\to 3^+}\dfrac{\sqrt{x-1}}{f(x)\left[f(x)-9\right] }=-\infty$;\\
            $\lim\limits_{x\to a^+}g(x)=\lim\limits_{x\to a^+}\dfrac{\sqrt{x-1}}{f(x)\left[f(x)-9\right] }=+\infty$.\\
            Do đó nên đồ thị hàm số $y=g(x)$ có $3$ đường tiệm cận đứng là các đường thẳng $x=1$, $x=3$ và $x=a$.
        \end{itemize}
        Như vậy, đồ thị hàm số $y=g(x)$ có $4$ đường tiệm cận, trong đó có $1$ đường tiệm cận ngang và $3$ đường tiệm cận đứng.
    }
\end{ex}
\begin{ex}%[VDC5-Đỗ Đường Hiếu]%[2D1G4-3]%
    \immini{Cho hàm số bậc ba $y=f(x)$ có đồ thị như hình vẽ bên. Đồ thị hàm số $g(x)=\dfrac{x\sqrt{x+1}}{f(x)\left[f^2(x)-16 \right] }$ có bao nhiêu tiệm cận đứng?
        \choice
        {\True $4$}
        {$5$}
        {$6$}
        {$7$}}
    {\begin{tikzpicture}[scale=0.6,line join=round, line cap=round,font=\footnotesize,>=stealth]
            \draw[->] (-2.5,0)--(4,0) node[below]{$x$};
            \draw[->] (0,-5)--(0,2.5) node[left]{$y$};
            \coordinate[label=below left:$O$] (O) at (0,0);
            \draw[dashed] (-1,0)--(-1,-4)--(0,-4);
            \clip (-2.3,-5) rectangle (3.5,2.5);
            \draw[smooth,samples=300,domain=-3.5:3.5] plot(\x,{-0.5*(\x+2)*(\x-1)*(\x-3)});
            \foreach \x in {-2,-1,1,3}
            \draw[shift={(\x,0)},color=black] (0pt,2pt) -- (0pt,-2pt) node[above] { $\x$};
            \foreach \y in {-4,-3,1}
            \draw[shift={(0,\y)},color=black] (2pt,0pt) -- (-2pt,0pt) node[right] {$\y$};
        \end{tikzpicture}
    }
    \loigiai{
        Xét phương trình $f(x)\left[f^2(x)-16 \right]=0$ \, $(*)$, với điều kiện $x\in\left[-1;+\infty \right) $.\\
        Ta có $f(x)\left[f^2(x)-16 \right]=0\Leftrightarrow \hoac{&f(x)=0\\&f(x)=4\\&f(x)=-4.}$\\
        \begin{itemize}
            \item Phương trình $f(x)=0$ có hai nghiệm $x\in\left[-1;+\infty \right) $ là $x=1$ và $x=3$.
            \item Phương trình $f(x)=4$ có không có nghiệm $x\in\left[-1;+\infty \right) $.
            \item Phương trình $f(x)=-4$ có hai nghiệm $x\in\left[-1;+\infty \right) $ là $-1<x_1<0$ và $x_2>3$.
        \end{itemize}
        Rõ ràng $\lim\limits_{x\to x_0^+}g(x)=+\infty$ hoặc $\lim\limits_{x\to x_0^+}g(x)=-\infty$, trong đó $x=x_0$ là nghiệm thuộc $\left[-1;+\infty \right) $ của phương trình $(*)$. Do đó đường thẳng $x=x_0$ là tiệm cận đứng của đồ thị hàm số $y=g(x)$.\\
        Từ đó suy ra đồ thị hàm số $g(x)=\dfrac{x\sqrt{x+1}}{f(x)\left[f^2(x)-16 \right] }$ có $4$ tiệm cận đứng.
    }
\end{ex}
\begin{ex}%[VDC5-Đỗ Đường Hiếu]%[2D1G4-3]%
    \immini{Cho $y=f(x)$ là hàm số đa thức có đồ thị như hình vẽ bên. Đặt $g(x)=\dfrac{\sqrt{x-1}}{\left[f(x)\right]^2-2f(x)}$ có bao nhiêu đường tiệm cận đứng?
        \choice
        {$5$}
        {$3$}
        {$4$}
        {\True $2$}}
    {\begin{tikzpicture}[scale=0.6,line join=round, line cap=round,font=\footnotesize,>=stealth]
            \draw[->] (-3,0)--(2.5,0) node[below]{$x$};
            \draw[->] (0,-1)--(0,5) node[left]{$y$};
            \coordinate[label=above left:$O$] (O) at (0,0);
            \draw[dashed] (-1,0)--(-1,4)--(0,4);
            \clip (-2.3,-1) rectangle (2.5,4.5);
            \draw[smooth,samples=300,domain=-3.5:3.5] plot(\x,{(\x)^3-3*(\x)+2});
            \foreach \x in {-2,-1,1}
            \draw[shift={(\x,0)},color=black] (0pt,2pt) -- (0pt,-2pt) node[below] { $\x$};
            \foreach \y in {2,4}
            \draw[shift={(0,\y)},color=black] (2pt,0pt) -- (-2pt,0pt) node[right] {$\y$};
        \end{tikzpicture}
    }
    \loigiai{
        Xét phương trình $\left[f(x)\right]^2-2f(x)=0$ \, $(*)$, với điều kiện $x\in\left[1;+\infty \right) $.\\
        Ta có $\left[f(x)\right]^2-2f(x)=0\Leftrightarrow \hoac{&f(x)=0\\&f(x)=2.}$\\
        \begin{itemize}
            \item Phương trình $f(x)=0$ có một nghiệm $x\in\left[1;+\infty \right) $ là $x=1$.
            \item Phương trình $f(x)=2$ có một nghiệm $x\in\left[1;+\infty \right) $ là $x=x_1>1$.
        \end{itemize}
        Rõ ràng $\lim\limits_{x\to x_0^+}g(x)=+\infty$ hoặc $\lim\limits_{x\to x_0^+}g(x)=-\infty$, trong đó $x=x_0$ là nghiệm thuộc $\left[1;+\infty \right) $ của phương trình $(*)$. Do đó đường thẳng $x=x_0$ là tiệm cận đứng của đồ thị hàm số $y=g(x)$.\\
        Từ đó suy ra đồ thị hàm số $g(x)=\dfrac{\sqrt{x-1}}{\left[f(x)\right]^2-2f(x)}$ có $2$ tiệm cận đứng.
    }
\end{ex}
\begin{ex}%[VDC5-NgocDungHo]%[2D1G4-3]%
    \immini
    {
        Cho hàm số $f(x)$ có đồ thị như hình bên. Số đường tiệm cận đứng của đồ thị hàm số $y=\dfrac{(x^2-4)(x^2+2x)}{[f(x)]^2-4f(x)+3}$ là
        \choice
        {$4$}
        {\True $5$}
        {$3$}
        {$2$}
    }
    {\begin{tikzpicture}[>=stealth,scale=0.5, line join=round, line cap=round]
            \def\f[#1]{-0.25*((#1)^4-8*(#1)^2+4)}
            \draw[->] (-4.1,0)--(4,0) node [below]{$x$};
            \draw[->] (0,-2)--(0,4) node [left]{$y$};
            \node at (0,0) [above left]{$O$};
            % \clip;
            \draw[domain=-2.9:2.9,samples=300,thick] plot (\x,{\f[\x]});
            \foreach \x in {-2,2} \filldraw (\x,0) node[below]{\x} circle (2pt);
            %\foreach \x in {-3,3} \filldraw (\x,0) node[below left]{\x} circle (2pt);
            \filldraw (-3,0) node[below left]{$-3$} circle (2pt);
            \filldraw (3,0) node[below right]{$3$} circle (2pt);
            \filldraw (0,1) node[left]{$1$} circle (2pt);
            \filldraw (0,3) node[above left]{$3$} circle (2pt);
            \draw[dashed](-2,0)--(-2,3)--(2,3)--(2,0);
            \draw (3,-1.75) node[right]{$y=f(x)$};
        \end{tikzpicture}
    }
    \loigiai{
        Xét hàm số $y=g(x)=\dfrac{(x^2-4 )(x^2+2x)}{[f(x)]^2-4f(x)+3}$.
        \immini
        {
            Giải phương trình $(x^2-4)(x^2+2x)=0 $\\
            $\Leftrightarrow \hoac{& x^2-4=0 \\ & x^2+2x=0}\Leftrightarrow \hoac{& x=\pm 2 \\ & x=0.}$\\
            Giải phương trình $[f(x)]^2-4f(x)+3=0$\\
            $ \Leftrightarrow \hoac{& f(x)=1 \\ & f(x)=3} \Leftrightarrow \hoac{& x = \pm 2 \\ & x=a\\&x=b\\&x=c\\&x=d.}$\\ với $-3<a<-2<b<c<2<d<3$.\\
        }
        {\begin{tikzpicture}[>=stealth,scale=0.8, line join=round, line cap=round]
                \def\f[#1]{-0.25*((#1)^4-8*(#1)^2+4)}
                \def\g[#1]{1}
                \def\h[#1]{3}
                \draw[->] (-4.1,0)--(4,0) node [below]{$x$};
                \draw[->] (0,-2)--(0,4) node [left]{$y$};
                \node at (0,0) [above left]{$O$};
                % \clip;
                \draw[domain=-2.9:2.9,samples=300,thick] plot (\x,{\f[\x]});
                \draw[domain=-4:4,samples=300,thick] plot (\x,{\g[\x]});
                \draw[domain=-4:4,samples=300,thick] plot (\x,{\h[\x]});
                \foreach \x in {-3,-2,2,3} \filldraw (\x,0) node[below]{\x} circle (2pt);
                % \filldraw (-3,0) node[above left]{$-3$} circle (2pt);
                % \filldraw (3,0) node[above ]{$3$} circle (2pt);
                \filldraw (0,1) node[below left]{$1$} circle (2pt);
                \filldraw (0,-1) node[below left]{$-1$} circle (2pt);
                \filldraw (0,3) node[above left]{$3$} circle (2pt);
                \draw[dashed](-2,0)--(-2,3) (2,3)--(2,0) (2.61,0)node[below]{$d$}--(2.61,1) (-2.61,0)node[below]{$a$}--(-2.61,1) (1.08,0)node[below]{$c$}--(1.08,1)(-1.08,0)node[below]{$b$}--(-1.08,1);
                \draw (3,2.75) node[right]{$y=f(x)$};
            \end{tikzpicture}
        }
        Trong điều kiện xác định của hàm số $y=g(x)$ ta có thể viết $$y=g(x)=\dfrac{x(x-2)(x+2)^2}{(x-a)(x-b)(x-c)(x-d) (x-2)^2(x+2)^2}=\dfrac{x}{(x-a)(x-b)(x-c)(x-d)(x-2)}$$
        Vậy số tiệm cận đứng của đồ thị hàm số $y=g(x)$ bằng $5$.
    }
\end{ex}
\Closesolutionfile{ans}
%--------Lời giải chi tiết
\FULLWIDTH \hienLG \anDA % ẩn bảng đáp án
% % % --
% % % \setcounter{deso}{0}
\chap{LỜI GIẢI CHI TIẾT} \setcounter{dang}{0} \setcounter{section}{0}

%%Bài 1. Đơn điệu, Cực trị
% \section{TÍNH ĐƠN ĐIỆU VÀ CỰC TRỊ CỦA HÀM SỐ}
\subsection{LÝ THUYẾT CẦN NHỚ}
\subsubsection{Tính đơn điệu của hàm số}
\begin{enumerate}[\iconMT]
	\item \indam{Định nghĩa:} Cho hàm số $y=f(x)$ xác định trên $K$ ($K$ là khoảng, đoạn hoặc nửa khoảng). \\
\begin{minipage}[b]{6cm}
\begin{khung4}{Ghi nhớ 1}
	Hàm số đồng biến trên $K$ nếu
	$\forall x_1,\,x_2 \in K$, $$ x_1<x_2 \Rightarrow f(x_1)<f(x_2)$$
	\centerline{\begin{tikzpicture}[>=stealth,scale=0.6]
		\draw[->] (-1,0)--(0,0)%
		node[below left]{$O$}--(5,0) node[below]{$x$};
		\draw[->] (0,-1) --(0,3) node[right]{$y$};
		\draw [black,thick, domain=0.2:4, samples=100] %
		plot (\x, {0.1*(\x)^2+1});
		\draw [dashed] (1,0)node[below]{\footnotesize$x_1$} --(1,1.1)--(0,1.1)node[left]{\footnotesize$f(x_1)$};
		\draw [dashed] (3,0)node[below]{\footnotesize$x_2$} --(3,1.9)--(0,1.9)node[left]{\footnotesize$f(x_2)$};
		\draw[fill=blue] (1,1.1) circle(2pt);
		\draw[fill=blue] (3,1.9) circle(2pt);
	\end{tikzpicture}}\\
Trên $K$, đồ thị là một "\textbf{đường đi lên}" khi xét từ trái sang phải.
\end{khung4}
\end{minipage}\hspace{0.5cm}
\begin{minipage}[b]{6cm}
\begin{khung4}{Ghi nhớ 2}
		Hàm số nghịch biến trên $K$ nếu
		$\forall x_1,\,x_2 \in K$, $$ x_1<x_2 \Rightarrow f(x_1)>f(x_2)$$
		\centerline{\begin{tikzpicture}[>=stealth,scale=0.6]
			\draw[->] (-1,0)--(0,0)%
			node[below left]{$O$}--(5,0) node[below]{$x$};
			\draw[->] (0,-1) --(0,3) node[right]{$y$};
			\draw [thick, domain=0.2:4, samples=100] %
			plot (\x, {-0.1*(\x)^2+2.5});
			\draw [dashed] (1,0)node[below]{\footnotesize$x_1$} --(1,2.4)--(0,2.4)node[left]{\footnotesize$f(x_1)$};
			\draw [dashed] (3,0)node[below]{\footnotesize$x_2$} --(3,1.6)--(0,1.6)node[left]{\footnotesize$f(x_2)$};
			\draw[fill=blue] (1,2.4) circle(2pt);
			\draw[fill=blue] (3,1.6) circle(2pt);
		\end{tikzpicture}}\\
	Trên $K$, đồ thị là một "\textbf{đường đi xuống}" khi xét từ trái sang phải.
\end{khung4}
\end{minipage}
	\item \indam{Liên hệ giữa đạo hàm và tính đơn điệu:}
	Cho hàm số $y=f(x)$ có đạo hàm trên khoảng $(a;b)$.
	\begin{boxdn}
	\begin{listEX}[1]
		\item [$\bullet$] Nếu $y'\ge 0$, $\forall x \in (a;b)$ và dấu bằng chỉ xảy ra tại hữu hạn điểm thì hàm số $y=f(x)$ đồng biến trên $(a;b)$.
		\item [$\bullet$] Nếu $y'\le 0$, $\forall x \in (a;b)$ và dấu bằng chỉ xảy ra tại hữu hạn điểm thì hàm số  $y=f(x)$ nghịch biến trên $(a;b)$.
	\end{listEX}
	\end{boxdn}
\end{enumerate}
\subsubsection{Cực trị của hàm số}
\begin{enumerate}[\iconMT]
	\item \indam{Định nghĩa:} Cho hàm số $y=f(x)$ xác định và liên tục trên khoảng $(a ; b)$ ( $a$ có thể là $-\infty, b$ có thể là $+\infty)$ và điểm $x_0 \in(a ; b)$.
	\begin{boxdn}
	\begin{itemize}
		\item [$\bullet$] Nếu tồn tại số $h>0$ sao cho $f(x)<f\left(x_0\right)$ với mọi $x \in\left(x_0-h ; x_0+h\right) \subset(a ; b)$ và $x \neq x_0$ thì ta nói hàm số $f(x)$ đạt cực đại tại $x_0$.
		\item [$\bullet$] Nếu tồn tại số $h>0$ sao cho $f(x)>f\left(x_0\right)$ với mọi $x \in\left(x_0-h ; x_0+h\right) \subset(a ; b)$ và $x \neq x_0$ thì ta nói hàm số $f(x)$ đạt cực tiểu tại $x_0$.
	\end{itemize}
	\end{boxdn}
	\item \indam{Định lý:} Giả sử hàm số $y=f(x)$ liên tục trên khoảng $(a ; b)$ chứa điểm $x_0$ và có đạo hàm trên các khoảng $\left(a ; x_0\right)$ và $\left(x_0 ; b\right)$. Khi đó:
	\begin{boxdn}
	\begin{itemize}
		\item [$\bullet$] Nếu $f^{\prime}(x)<0$ với mọi $x \in\left(a ; x_0\right)$ và $f^{\prime}(x)>0$ với mọi $x \in\left(x_0 ; b\right)$ thì $x_0$ là một điểm cực tiểu của hàm số $f(x)$.
		\item [$\bullet$] Nếu $f^{\prime}(x)>0$ với mọi $x \in\left(a ; x_0\right)$ và $f^{\prime}(x)<0$ với mọi $x \in\left(x_0 ; b\right)$ thì $x_0$ là một điểm cực đại của hàm số $f(x)$.
	\end{itemize}
	\end{boxdn}
	\item \indam{Các tên gọi:}\\
		\begin{tikzpicture}[smooth,samples=300,scale=1.15,>=stealth]
			\draw[->,>=stealth] (-2.5,0)--(2.7,0) node[below]{$x$};
			\draw[->,>=stealth] (0,-1.5)--(0,4) node[right]{$y$};
			\draw (0,0) node[above left]{$O$};
			\draw[blue,domain=-2:2,line width = 1.2pt] plot(\x,{(\x)^3-3*(\x)+1})node[right]{$y=f(x)$};
			\draw[fill=black] (1,0) circle(1pt) (1,-1) circle(2pt) (0,-1) circle(1pt) (-1,0) circle(1pt) (-1,3) circle(2pt) (0,3) circle(1pt);
			\draw[dashed] (1,0)node[above]{\small$x_2$}--(1,-1)--(0,-1)node[left]{\small$y_2$} (-1,0)node[below]{\small$x_1$}--(-1,3)--(0,3)node[right]{\small$y_1$};
			
			\draw[-,dotted] (-0.5,3.7)--(4,3.7)node[right]{$(x_1;y_1)$ là điểm cực đại của đồ thị hàm số;}; 
			\draw[->,dotted] (-0.5,3.7)--(-1,3.15);
			\node[right] at (4.5,3.1) {$\bullet$ $x_1$ là điểm cực đại của hàm số;};
			\node[right] at (4.5,2.5) {$\bullet$ $y_1$ là giá trị cực đại của hàm số.};
			
			\draw[-,dotted] (2,-1)--(2,1)--(4,1)node[right]{$(x_2;y_2)$ là điểm cực tiểu của đồ thị hàm số;}; \draw[->,dotted] (2,-1)--(1.15,-1);
			\node[right] at (4.5,0.4) {$\bullet$ $x_2$ là điểm cực tiểu của hàm số;};
			\node[right] at (4.5,-0.2) {$\bullet$ $y_2$ là giá trị cực tiểu của hàm số.};
		\end{tikzpicture}
\end{enumerate}
\subsection{PHÂN LOẠI VÀ PHƯƠNG PHÁP GIẢI TOÁN}
\begin{dang}{Bài toán tìm khoảng đơn điệu và cực trị của hàm số cho trước}
	\begin{listEX}[1]
		\item [\ding{172}] Tìm tập xác định $\mathscr{D}$ của hàm số $y=f(x)$ .
		\item [\ding{173}] Tính đạo hàm $f'(x)$. Tìm các điểm $x_i \,(i = 1, 2, ..., n)$ thuộc $\mathscr{D}$ mà tại đó đạo hàm bằng $0$ hoặc không xác định.
		\item [\ding{174}] Sắp xếp các điểm $x_i$ theo thứ tự tăng dần, xét dấu $y'$ và lập bảng biến thiên. Từ đây, nêu các khoảng đồng biến, nghịch biến và các điểm cực trị.
	\end{listEX}
\end{dang}
\indamm{Ghi nhớ cách xét dấu:}
\begin{note}
\begin{enumerate}[\iconCH]
		% \item Nếu $$f'(x)=(x-a)(x-b)^2(x-c)^{2n}(x-d)^{2n+1},\,\forall n \in \mathbb{N}*$$
		% thì phương trình $f'(x)=0$ có
		% \begin{itemize}
		% 	\item 	$x=a$ là nghiệm đơn;
		% 	\item  $x=b$ là nghiệm kép;
		% 	\item  $x=c$ là nghiệm bội chẵn;
		% 	\item  $x=d$ là nghiệm bội lẻ.
		% \end{itemize}
		\item Khi xét dấu $f'(x)$ thì $f'(x)$ sẽ không đổi dấu khi qua nghiệm kép (nghiệm bội chẵn) và đổi dấu khi qua nghiệm đơn (nghiệm bội lẻ).
	\end{enumerate}
	% \begin{tikzpicture}[smooth,samples=300,scale=0.8,>=stealth,font=\footnotesize]
	% 	\draw[->] (-3.5,0)--(6,0) node[below]{$x$};
	% 	\draw[->] (0,-1.5)--(0,4) node[left]{$y$};
	% 	\draw (0,0) node[above left]{$O$};
	% 	\draw[blue,line width=0.7pt,domain=-2.15:1.5] plot(\x,{(\x+2)*(\x-1)^2});
	% 	\draw[blue,line width=0.7pt,domain=1.5:4.7] plot(\x,{-1*(\x-1.64)*(\x-4)^2})node[below]{$y=f'(x)$};
	% 	\draw[fill=red] (-2,0)node[above left]{$x_1$} circle(1.5pt) (1,0)node[below]{$x_2$} circle(1.5pt) (4,0)node[above right]{$x_4$} circle(1.5pt) (1.64,0)node[above right]{$x_3$} circle(1.5pt);
	% 	\draw[dashed,<-] (-1.8,-0.2)--(0.5,-2.3)node[below]{\fbox{\scriptsize\text{Nghiệm bội lẻ}}};
	% 	\draw[dashed,->](0.5,-2.3)--(1.58,-0.2);
	% 	\draw[dashed,<-] (1,0.2)--(2,3)node[above]{\fbox{\scriptsize\text{Nghiệm bội chẵn}}};
	% 	\draw[dashed,->](2,3)--(3.9,0.1);
	% 	\end{tikzpicture}
\end{note}
\boxmini{BÀI TẬP TỰ LUẬN}

\begin{vd}
	Tìm các khoảng đơn điệu và các điểm cực trị của hàm số sau
	\begin{tasks}(3)
		\task $ y=-x^3+3x^2-4$;
		\task $ y=x^3-3x^2+1$;
		\task $y=x^3+3x^2+3x+2$;
		\task $y=-2x^4+4x^2$;
		\task $y=x^4+4x^3-1$;
		\task $y=-16x^4+x-1$.
	\end{tasks}
	\loigiai{
	\begin{enumEX}[a)]{1}
		\item Tập xác định: $\mathscr{D}=\mathbb{R}$. \\
		Đạo hàm: $y'=-3x^2+6x$.\\
		Xét $y'=0 \Leftrightarrow -3x^2+6x=0 \Leftrightarrow
		\hoac{
			& x=0 \\
			& x=2 }$
		Bảng biến thiên:\begin{center}
			\begin{tikzpicture}
				\tkzTabInit[nocadre=false,lgt=0.7,espcl=2.1,deltacl=0.6]
				{$x$ /0.6,$y'$ /0.6,$y$ /2}
				{$-\infty$,$0$,$2$,$+\infty$}
				\tkzTabLine{,-,$0$,+,$0$,-,}
				\tkzTabVar{+/$+\infty$, -/$-4$,+/$0$,-/$-\infty$}
			\end{tikzpicture}
		\end{center}
		\item Ta có: $ y'=3x^2-6x\Rightarrow y'=0\Leftrightarrow \hoac{&x=0\\&x=2.} $\\
		Từ bảng biến thiên suy ra hàm số đồng biến trên khoảng $ (-\infty;0) $ và $ (2;+\infty). $
		\begin{center}
			\begin{tikzpicture}
				\tkzTabInit[nocadre=false,lgt=1,espcl=3]
				{$x$ /1,$y'$ /1,$y$ /2}
				{$-\infty$,$0$, $2$,$+\infty$}
				\tkzTabLine{,+,$0$,-,$0$,+, }
				\tkzTabVar{-/ $-\infty$,+/$1 $ ,-/$-3$,+/$+\infty$}
			\end{tikzpicture}
		\end{center}
		\item Hàm số đã cho xác định trên $\mathscr{D}=\mathbb{R}$.\\
		Ta có $y'=3x^2+6x+3$. Cho $y'=0 \Leftrightarrow 3x^2+6x+3=0 \Leftrightarrow x=-1$.\\
		Bảng biến thiên
		\begin{center}
			\begin{tikzpicture}
				\tkzTabInit[lgt=1,espcl=3]
				{$x$/0.7,$y'$/0.7,$y$/2}
				{$-\infty$,$-1$,$+\infty$}
				\tkzTabLine{,+,0,+,}
				\tkzTabVar{-/$-\infty$,R/,+/$+\infty$}
				
			\end{tikzpicture}
		\end{center}
		Vậy hàm số đồng biến trên $\mathbb{R}$.
		\item Tập xác định của hàm số là $ \mathscr{D}=\mathbb{R}$.\\
		Ta có $y'=-8x^3+8x$.
		Cho $y'=0 \Leftrightarrow -8x^3+8x=0 \Leftrightarrow 8x(-x^2+1)=0$\\
		\centerline{$ \Leftrightarrow \left[\begin{aligned}
				&8x=0 \\
				&-x^2+1=0
			\end{aligned}\right. \Leftrightarrow \left[\begin{aligned}
				&x=0 \\
				&x^2=1
			\end{aligned}\right. \Leftrightarrow \left[\begin{aligned}
				&x=0 \\
				&x=\pm 1.
			\end{aligned}\right. $}
		Bảng biến thiên
		\begin{center}
			\begin{tikzpicture}
				\tkzTabInit[lgt=1,espcl=3]
				{$x$/0.7,$y'$/0.7,$y$/2}
				{$-\infty$,$-1$,$0$,$1$,$+\infty$}
				\tkzTabLine{,+,0,-,0,+,0,-,}
				\tkzTabVar{-/$-\infty$,+/ $2$/,-/$0$,+/$2$,-/$-\infty$}
			\end{tikzpicture}
		\end{center}
		Vậy hàm số đồng biến trên mỗi khoảng $(-\infty;-1)$ và $(0;1)$,\\
		\indent{ } hàm số nghịch biến trên mỗi khoảng $(-1;0)$ và $(1;+\infty)$.
		\item Hàm số đã cho xác định trên $\mathscr{D}=\mathbb{R}$.\\
		Ta có $y'=4x^3+12x^2=0=4x^2(x+3)$.\\
		Cho $y'=0 \Leftrightarrow 4x^2(x+3)=0 \Leftrightarrow \left[\begin{aligned}
			&x=0 \\
			&x=-3.
		\end{aligned}\right.$\\
		Bảng biến thiên
		\begin{center}
			\begin{tikzpicture}
				\tkzTabInit[lgt=1,espcl=3]
				{$x$/0.7,$y'$/0.7,$y$/2}
				{$-\infty$,$-3$,$0$,$+\infty$}
				\tkzTabLine{,-,0,+,0,+,}
				\tkzTabVar{+/$+\infty$,-/$-28$ /,R,+/$+\infty$}
			\end{tikzpicture}
		\end{center}
		Vậy hàm số nghịch biến trên khoảng $(-\infty;-3)$ và đồng biến trên khoảng $(-3;+\infty)$.
		\item Ta có $y'=-64x^3+1<0\Leftrightarrow x>\dfrac{1}{4}$ nên hàm số nghịch biến trên khoảng $\left(\dfrac{1}{4};+\infty\right)$.
\end{enumEX}}
\end{vd}

\begin{vd}
	Tìm các khoảng đơn điệu và cực trị của các hàm số sau:
	\begin{tasks}(3)
		\task $y=\dfrac{2x+1}{x+1}$;
		\task $y=\dfrac{3x+1}{x-1}$;
		\task $y=\dfrac{x^2+2x+2}{x+1}$;
		\task $y=x+\dfrac{4}{x}$;
		\task $y=\sqrt{x^2-2x}$;
		\task $y=x-3\sqrt[3]{x^2}$ .
	\end{tasks}
	\loigiai{
		\begin{enumEX}[a)]{1}
			\item Ta có $y'=\dfrac{1}{(x+1)^2} > 0, \forall x \in \mathbb{R} \backslash \{-1\}$.\\
			Vậy hàm số đồng biến trên $(-\infty ;-1)$ và $(-1 ;+\infty)$.\\
			Hàm số không có cực trị.
			\item Ta có $y'=\dfrac{-4}{(x-1)^2}>0,\,\forall x\in\mathbb{R}\setminus\{1\}$.\\
			Do vậy hàm số nghịch biến trên các khoảng  $(-\infty;1)$; $(1;+\infty)$.\\
			Hàm số không có cực trị.
			\item \begin{itemize}
				\item TXĐ: $\mathscr{D}=\mathbb{R}\setminus \left\{-1\right\}$.
				\item $y'=\dfrac{x^2+2x}{(x+1)^2}$, $y'=0\Leftrightarrow \hoac{& x=-2 \\ & x=0.}$\\
				Ta có bảng biến thiên
				\begin{center}
					\begin{center}
						\begin{tikzpicture}
							\tkzTabInit[nocadre=True,lgt=1,espcl=2]
							{$x$ /0.7,$y'$ /0.7,$y$ /2}
							{$-\infty$,$-2$,$-1$,$0$,$+\infty$}
							\tkzTabLine{,+,$0$,-,d,-,$0$,+,}
							\tkzTabVar{-/$-\infty$,+/$-2$,-D+/$-\infty$/$+\infty$,-/$2$,+/$+\infty$}
						\end{tikzpicture}
					\end{center}
				\end{center}
				Hàm số đồng biến trên khoảng $\left( -\infty;-2\right)$ và $\left( 0;+\infty\right)$;  nghịch biến trên $(-2;-1)$ và $(-1;0)$.\\
				Hàm số đạt cực đại tại $x=-2$, giá trị cực đại $y=-2$\\
				Hàm số đạt cực tiểu tại $x=0$, giá trị cực tiểu $y=2$.\\
			\end{itemize}
			\item Tập xác định $\mathscr{D}=\mathbb{R}\setminus\{0\}$.\\
			Ta có $y'=1-\dfrac{4}{x^2}=\dfrac{x^2-4}{x^2}$, $y'=0\Leftrightarrow x=\pm 2$.\\
			Bảng biến thiên
			\begin{center}
				\begin{tikzpicture}[yscale=.8,xscale=1.5,]
					\begin{scope}[shift={(-.5,.5)}]
						\draw
						(0,0) rectangle +(8,-5)
						(0,-1)--+(0:8) (0,-2)--+(0:8) (1,0)--+(-90:5);
					\end{scope}
					\path
					(0,0) node{$x$}          % <<< dòng 1
					++(0:1) node{$-\infty$}
					++(0:2) node{$-2$}
					++(0:1) node{$0$}
					++(0:1) node{$2$}
					++(0:2) node{$+\infty$}
					(0,-1)   node{$y'$}         % <<< dòng 2
					++(0:2) node{$+$}
					++(0:1) node{$0$}
					++(0:.5) node{$-$}
					++(0:1) node{$-$}
					++(0:.5) node{$0$}
					++(0:1) node{$+$}
					(0,-3)   node{$y$}       % <<< dòng 3
					++(0:1) ++(-90:1)  node (A) {$-\infty$}
					++(0:2) ++(90:2) node (B) {$-4$}
					++(0:1) ++(-90:2) node (C)[left]
					{$-\infty$}
					++(90:2) node (D)[right]{$+\infty$}
					++(0:1) ++(-90:2) node (E) {$4$}
					++(0:2) ++(90:2) node (F) {$+\infty$};
					\draw[-stealth] (A)--(B);
					\draw[-stealth] (B)--(C);
					\draw[-stealth] (D)--(E);
					\draw[-stealth] (E)--(F);
					\draw[double] (4,-.5)--(4,-4.5);
				\end{tikzpicture}
			\end{center}
			Hàm số đồng biến trên khoảng $\left( -\infty;-4\right)$ và $\left( 2;+\infty\right)$; nghịch biến trên các khoảng $(-2;0)$ và $(0;2)$.\\
			Hàm số đạt cực đại tại $x=-2$, giá trị cực đại $y=-4$\\
			Hàm số đạt cực tiểu tại $x=2$, giá trị cực tiểu $y=4$\\
			\item Tập xác định: $\mathscr{D}=(-\infty;0]\cup [2;+\infty)$.\\
			Ta có $y'=\dfrac{x-1}{\sqrt{x^2-2x}},\forall x\in (-\infty;0)\cup (2;+\infty)$.\\
			$y'=0 \Leftrightarrow \dfrac{x-1}{\sqrt{x^2-2x}}=0 \Rightarrow x-1=0 \Leftrightarrow x=1 \notin \mathscr{D}$.\\
			Bảng biến thiên:
			\begin{center}
				\begin{tikzpicture}
					\tkzTabInit[lgt=1,espcl=3]
					{$x$/0.7,$y'$/0.7,$y$/2}
					{$-\infty$,$0$,$2$,$+\infty$}
					\tkzTabLine{,-,d,h,d,+,}
					\tkzTabVar{+/$+\infty$,-H/$0$/,-/$0$,+/$+\infty$}
				\end{tikzpicture}
			\end{center}
			Vậy hàm số nghịch biến trên khoảng $(-\infty;0)$ và đồng biến trên khoảng $(2;+\infty)$.\\
			Hàm số không có cực trị.
			\item Tập xác định: $\mathscr{D}=\mathbb{R}$.\\
			Đạo hàm $y'=1-\dfrac{2}{\sqrt[3]{x}}$, xác định với mọi $x\neq 0$.\\
			$y'=0\Leftrightarrow \sqrt[3]{x}=2\Leftrightarrow x=8$.\\
			Đạo hàm không xác định tại $x=0$.\\
			Bảng biến thiên
			\begin{center}
				\begin{tikzpicture}
					\tkzTabInit[nocadre,lgt=1,espcl=2]{$x$/0.7,$y'$/0.7,$y$/2}{$-\infty$,$0$,$8$,$+\infty$}%
					\tkzTabLine{,+,d,-,z,+,}
					\tkzTabVar{-/$-\infty$ , +/$0$,-/$-4$, +/$+\infty$}%
				\end{tikzpicture}
			\end{center}
	\end{enumEX}}
\end{vd}

\begin{vd}
	Thể tích $V$ (đơn vị: centimét khối) của $1 \mathrm{~kg}$ nước tại nhiệt độ $T\,\left(0^{\circ} \mathrm{C} \leq T \leq 30^{\circ} \mathrm{C}\right)$ được tính bởi công thức	$$	V(T)=999,87-0,06426 T+0,0085043 T^2-0,0000679 T^3$$
	 Hỏi thể tích $V(T), \,0^{\circ} \mathrm{C} \leq T \leq 30^{\circ} \mathrm{C}$, giảm trong khoảng nhiệt độ nào?
	\loigiai{
		Xét hàm số  $V(T)=999{,}87-0{,}06426T+0{,}0085043T^2-0{,}0000679T^3$, với $T\in [0;30]$.\\
	Ta có $V'(T)=-0{,}0002037T^2+0{,}0170086T-0{,}06426$.\\
	$V'(T)=0\Leftrightarrow T=3{,}966514624=T_1$ hoặc $T=79{,}53176716\not\in [0;30]$.\\
	Bảng biến thiên của hàm số $V(T)$ như sau
	\begin{center}
		\begin{tikzpicture}[font=\footnotesize,thick,>=stealth]
			\tikzset{double style/.append style={double distance=1.5pt}}
			\tkzTabInit[nocadre=false,lgt=1.2,espcl=2.5,deltacl=0.6,lw=.75pt,color,colorL=green!50,colorV=green!50]
			{$T$ /0.7, $V'(T)$ /0.8, $V(T)$ /2}
			{$0$,$T_1$,$30$}
			\tkzTabLine{ ,-,$0$,+, }
			\tkzTabVar{+/$V(0)$,-/$V(T_1)$,+/$V(30)$}
		\end{tikzpicture}
	\end{center}
	Từ bảng biến thiên suy ra, thể tích $V(T), 0^{\circ}\mathrm{C}\leq T \leq 30^{\circ}\mathrm{C}$, giảm trong khoảng nhiệt độ từ $0^\circ$C đến $3{,}966514624^\circ$C.}
\end{vd}

\boxmini{BÀI TẬP TRẮC NGHIỆM}
\ind{PHẦN I.} \inden{Câu trắc nghiệm nhiều phương án lựa chọn. Mỗi câu hỏi học sinh chỉ chọn một phương án.}\\
\setcounter{ex}{0}
\Opensolutionfile{ans}[ans/2D1-B1-d1-1]

\begin{ex}%[KSCL, Sở GD \& ĐT Hà Nam, 2018]%[Lê Quốc Hiệp, dự án 12EX10-18]%[2D1B1-2]%
	\immini
	{Cho hàm số $y=f(x)$ có đồ thị như hình vẽ bên. Hàm số $y=f(x)$ nghịch biến trên khoảng nào dưới đây?
		\haicot
		{$(\sqrt{2};+\infty)$}
		{$(-2;2)$}
		{$(-\infty;0)$}
		{\True $(0;\sqrt{2})$}
	}
	{\begin{tikzpicture}[line cap=round,line join=round,x=1.0cm,y=1.0cm,>=stealth,scale=0.7]
			\draw[->,color=black,smooth,samples=100] (-2.5,0.) -- (2.5,0.) node[below] {\footnotesize $x$};
			\draw[->,color=black,smooth,samples=100] (0.,-2.5) -- (0.,3) node[left] {\footnotesize $y$};
			\draw plot[smooth,tension=.7] coordinates {(-2,3) (-1.41,-2)  (0,2) (1.41,-2) (2,3)};
			\draw[fill=black] (0,0) circle [radius=1pt] node[above left] {\footnotesize $O$};
			\fill (-1.41,0) node[shift={(90:2ex)}]{\footnotesize $-\sqrt{2}$} circle(1pt);
			\fill (1.41,0) node[shift={(90:2ex)}]{\footnotesize $\sqrt{2}$} circle(1pt);
			\fill (0,-2) node[shift={(-45:1.5ex)}]{\footnotesize $-2$} circle(1pt);
			\fill (0,2) node[shift={(45:1.5ex)}]{\footnotesize $2$} circle(1pt);
			\draw[dashed] (-1.41,0)|-(0,-2)-|(1.41,0);
	\end{tikzpicture}}
	\loigiai
	{
		Dựa vào đồ thị, ta thấy trên khoảng $(0;\sqrt{2})$ đồ thị đi xuống nên hàm số $y=f(x)$ nghịch biến trên khoảng đó.
	}
\end{ex}

\begin{ex}
	\immini{Cho hàm số $y=f(x)$ có đồ thị như hình vẽ bên. Mệnh đề nào sau đây là mệnh đề \textbf{sai}?
		\choice
		{Hàm số đạt cực đại tại $x=0$}
		{Hàm số có giá trị cực tiểu bằng $-2$}
		{\True Hàm số đồng biến trên $(-\infty; 2)$}
		{Hàm số nghịch biến trên $(0; 2)$}
	}
	{
		
		\begin{tikzpicture}[smooth,samples=300,scale=0.7,>=stealth]
			\draw[->] (-2,0)--(3.7,0) node[below]{$x$};
			\draw[->] (0,-2.5)--(0,2.5) node[right]{$y$};
			\draw (0,0) node[below right]{$O$};
			\draw[smooth,samples=100,domain=-1:3]
			plot(\x,{(\x)^3-3*(\x)^2+2});
			\draw[fill=black] (2,0) circle(1.5pt) (0,2) circle(1.5pt) (0,-2) circle(1.5pt);
			\draw[dashed] (2,0)node[above]{\small$2$}--(2,-2)--(0,-2)node[left]{\small$-2$} (0,2.1)node[left]{\small$2$};
		\end{tikzpicture}
	}
	
	\loigiai{
	}
	
\end{ex}

\begin{ex}
	\immini{
		Hàm số $y=f(x)$ có đồ thị là đường cong trong hình vẽ bên. Hàm số $y=f(x)$ đạt cực tiểu tại điểm nào dưới đây?
		\haicot
		{$x=2$}
		{\True $x=0$}
		{$x=-2$}
		{$x=4$}
	}{
		\begin{tikzpicture}[xscale=.7,yscale=.6, font=\footnotesize, line join=round, line cap=round, >=stealth]
			\draw[->] (-2.7,0)--(3,0) node[below]{$x$};
			\draw[->] (0,-1.25)--(0,5) node[left]{$y$};
			\draw[dashed] (-2^.5,0)--(-2^.5,4)--(2^.5,4)--(2^.5,0);
			\draw[domain=-2.05:2.05] plot(\x,{-(\x)^2*((\x)^2-4)});
			\path
			(0,0) node[below right]{$O$}
			(2,0) node[above right]{$2$}
			(-2,0) node[above left]{$-2$}
			(0,4) node[below right]{$4$}
			(-2^.5,0) node[below]{$-\sqrt{2}$}
			(2^.5,0) node[below]{$\sqrt{2}$};
		\end{tikzpicture}
	}
	\loigiai{
		Dựa vào đồ thị hàm số ta thấy hàm số đạt cực tiểu tại $x=0$.}
\end{ex}

\begin{ex}
	\immini{Cho hàm số $y=f(x)$ có bảng biến thiên như hình bên. Mệnh đề nào sau đây là mệnh đề đúng?
	\choice
	{Hàm số đồng biến trên khoảng $(-\infty;3)$}
	{Hàm số nghịch biến trên khoảng $(-2;+\infty)$}
	{Hàm số đạt cực đại tại $x=3$}
	{\True Hàm số đạt cực tiểu tại $x=2$}}{
	\begin{tikzpicture}
	\tkzTabInit[lgt=1.2,espcl=1.8,nocadre=True]
	{$x$/0.6,$f'(x)$/0.6,$f(x)$/2}{$-\infty$,$-2$,$2$,$+\infty$}
	\tkzTabLine{,+,0,-,0,+,}
	\tkzTabVar{-/$-\infty$,+/$3$,-/$0$,+/$+\infty$}
\end{tikzpicture}}
	\loigiai{
	}
	
\end{ex}

\begin{ex}
	Cho hàm số $y=f(x)$ có bảng biến thiên bên dưới
	\begin{center}
		\begin{tikzpicture}
			\tikzset{double style/.append style = {draw=\tkzTabDefaultWritingColor,double=\tkzTabDefaultBackgroundColor,double distance=2pt}}
			\tkzTabInit[lgt=1.2,espcl=2,nocadre=True]
			{$x$ /.7, $f’(x)$ /.7,$f(x)$ /2}
			{$-\infty$ , $-2$, $0$ ,$2$ ,$+\infty$}
			\tkzTabLine{ ,+,$0$,-,d,-,$0$,+, }
			\tkzTabVar{ -/ $-\infty$,+/ $-4$/,-D+/$- \infty$ /$+\infty$,-/ $4$,+ /$+\infty$}
		\end{tikzpicture}
	\end{center}
Khẳng định nào sau đây là khẳng định \textbf{sai}?
	\choice
	{Hàm số có hai điểm cực trị}
	{Tọa độ điểm cực đại của đồ thị hàm số là $(-2;-4)$}
	{\True Hàm số nghịch biến trên khoảng $(-2;2)$}
	{Hàm số đồng biến trên khoảng $(3;+\infty)$}
	\loigiai{
	}
\end{ex}


\begin{ex}
	Cho hàm số $y= - \dfrac{1}{3} x^3 - x -3 $. Mệnh đề nào dưới đây đúng?
	\choice
	{Hàm số đồng biến trên $(-\infty; 1)$ và trên $(1; +\infty)$}
	{\True Hàm số nghịch biến trên $\mathbb{R}$}
	{Hàm số đồng biến trên $(-1;1)$}
	{Hàm số đồng biến trên $\mathbb{R}$}
	\loigiai{
		Tập xác định $\mathscr D = \mathbb{R}$.\\
		$y'=-x^2 -1<0 $ với mọi $x$.\\
		Suy ra  hàm số đã cho nghịch biến trên $\mathbb{R}$.}
\end{ex} 


\begin{ex}
	Gọi $x_1$ là điểm cực đại $x_2$ là điểm cực tiểu của hàm số $y=-x^3+3x+2$. Tính $x_1+2x_2$.
	\choice{$2$}
	{$1$}
	{\True $-1$}
	{$0$}
	\loigiai{
		Ta có $y'=-3x^2+3$, $y'=0\Leftrightarrow x=\pm 1$.\\
		Vì $y'$ đổi dấu từ âm sang dương khi qua $x=-1$ và đổi dấu từ dương sang âm khi qua $x=1$ nên $x_2=-1$ là điểm cực tiểu và $x_1=1$ là điểm cực đại của hàm số. Do đó $x_1+2x_2=1-2=-1$.
	}
\end{ex} 


\begin{ex}
	Khoảng cách giữa hai điểm cực trị của đồ thị hàm số $y=x^3-3x^2+4$ bằng
	\choice
	{\True $2\sqrt{5}$}
	{$2\sqrt{2}$}
	{$2$}
	{$ 4 $}
	\loigiai{
		Ta có $y'=3x^2-6x$, $ y'=0\Rightarrow \hoac{&x=0\Rightarrow y=4\\&x=2\Rightarrow y=0.} $\\
		Suy ra hai điểm cực trị của đồ thị hàm số là $A(0;4),B(2;0)$.\\
		Do đó $AB=\sqrt{2^2+(-4)^2}=2\sqrt{5}$.
	}
\end{ex} 

\begin{ex}%[2D1B1]
	Hàm số $y=x^4-2x^2+1$ đồng biến trên khoảng nào dưới đây?
	\choice
	{\True $(-1;0)$}
	{$(-1;+ \infty)$}
	{$(-3;8)$}
	{$(- \infty ; -1)$}
	\loigiai
	{
		$y'= 4x^3-4x$ $\Rightarrow y'=0 \Leftrightarrow 4x^3-4x=0$ $\Leftrightarrow \hoac{x&= -1 \\ x&=0 \\ x &= 1}$\\
		Bảng xét dấu
		\begin{center}
			\begin{tikzpicture}
				\tkzTabInit[nocadre=false, lgt=1, espcl=2.5]{$x$ /1,$y$ /1}{$-\infty$,$-1$,$0$,$1$,$+\infty$}
				\tkzTabLine{,-,$0$,+,$0$,-,$0$,+}
			\end{tikzpicture}
		\end{center}
	}
	
\end{ex} 

\begin{ex}%[2HK1-13-ChuyenLeQuyDon-QuangTri]%[2D1B2-1]%
	Cho hàm số $ y = - \dfrac{1}{4}x^4 + \dfrac{1}{2}x^2 - 3 $. Khẳng định nào sau đây là khẳng định đúng?
	\choice
	{Hàm số đạt cực tiểu tại $ x = -3 $}
	{ \True Hàm số đạt cực tiểu tại $ x = 0 $}
	{Hàm số đạt cực đại tại $ x = 0 $}
	{Hàm số đạt cực tiểu tại $ x = -1 $}
	\loigiai{
		Ta có $ y' = - x^3 + x = - x (x^2 - 1) $.
		Ta có bảng biến thiên như hình bên
		\begin{center}
			\begin{tikzpicture}[scale=1]
				\tkzTabInit[lgt=1.5,espcl=2.5]{$x$  /1,$y'$  /1,$y$ /2}
				{$-\infty$,$ -1 $,$ 0 $,$ 1 $,$+\infty$}%
				\tkzTabLine{,+,z,-,z,+,z,-,}
				\tkzTabVar{-/$ -\infty $,+/   $\dfrac{-11}{4}$ /,-/ $-3$,+/$ \dfrac{-11}{4} $,-/$ -\infty $}
				%\tkzTabIma{1}{3}{2}{$ 0 $}
			\end{tikzpicture}
		\end{center}
	}
\end{ex} 

\begin{ex}
	Cho hàm số $y=\dfrac{3x-1}{x-2}$. Mệnh đề nào dưới đây là đúng?
	\choice
	{Hàm số nghịch biến trên $\mathbb{R}$}
	{Hàm số đồng biến trên các khoảng $(-\infty;2)$ và $(2;+\infty)$}
	{\True Hàm số nghịch biến trên các khoảng $(-\infty;2)$ và $(2;+\infty)$}
	{Hàm số đồng biến trên $\mathbb{R}\setminus\{2\}$}
	\loigiai{Tập xác định là $\mathscr{D}=\mathbb{R}\setminus\{2\}$.\\
		Có $y'=\dfrac{-5}{(x-2)^2}<0$, $\forall x\in\mathscr{D}$ nên hàm số nghịch biến trên các khoảng $(-\infty;2)$ và $(2;+\infty)$.}
	
\end{ex} 

\begin{ex}
	Cho hàm số $y=\dfrac{x-2}{x+3}$. Mệnh đề nào dưới đây đúng?
	\choice
	{Hàm số nghịch biến trên khoảng $(-\infty;-3)\cup (-3;+\infty) $}
	{\True Hàm số đồng biến trên khoảng $(-\infty;-3) $ và $(-3;+\infty)$}
	{Hàm số nghịch biến trên khoảng $(-\infty;-3)$ và $(-3;+\infty)$}
	{Hàm số đồng biến trên khoảng $(-\infty;-3)\cup (-3;+\infty) $}
	\loigiai{
		Tập xác định $\mathscr{D}=\mathbb{R}\setminus \{-3\}$. Ta có $y'=\dfrac{5}{(x+3)^2}>0$, $\forall x\in\mathscr{D}$.\\ Suy ra hàm số đồng biến trên khoảng $(-\infty;-3)$ và $(-3;+\infty)$.
	}
\end{ex} 

\begin{ex}
	Gọi $y_{\text{CĐ}},\,y_{\text{CT}}$ lần lượt là giá trị cực đại và giá trị cực tiểu của hàm số $y=\dfrac{x^2+3x+3}{x+2}$. Giá trị của biểu thức $y_{\text{CĐ}}^2-2y_{\text{CT}}^2$ bằng
	\choice
	{$8$}
	{\True $7$}
	{$9$}
	{$6$}
	\loigiai{
		Ta có $y'=\dfrac{x^2+4x+3}{(x+2)^2}$; $y'=0 \Leftrightarrow \left[\begin{aligned}
			&x=-1 \\
			&x=-3
		\end{aligned}\right. $. \\
		Bảng biến thiên
		\begin{center}
			\begin{tikzpicture}
				\tkzTab
				[lgt=1,espcl=2] % tùy chọn
				{$x$/0.7, $y'$/0.7, $y$/2} % cột đầu tiên
				{$-\infty$, $-3$, $-2$, $-1$, $+\infty$} % hàng 1 cột 2
				{,+,0,-,d,-,0,+,} % hàng 2 cột 2
				{-/ $-\infty$, +/ $-3$, -D+/ $-\infty$ / $+\infty$, -/ $1$, +/ $+\infty$} % hàng 3 cột 2
			\end{tikzpicture}
		\end{center}
		Từ bảng biến thiên ta tìm được $y_{\text{CĐ}}=-3;\,y_{\text{CT}}=1$ $ \Rightarrow $ $y_{\text{CĐ}}^2-2y_{\text{CT}}^2$ $=9-2=7$.}
\end{ex} 

\begin{ex}
	Tìm điểm cực tiểu của hàm số $f(x)=(x-3)\mathrm{e}^x$.
	\choice
	{$x=3$}
	{$x=0$}
	{\True $x=2$}
	{$x=1$}
	\loigiai{
		\begin{itemize}
			\item Ta có $f'(x)=\mathrm{e}^x(x-2)$, $f''(x)=\mathrm{e}^x(x-1)$.
			\item $f'(x)=0\Rightarrow x=2$ và $f''(2)=\mathrm{e}^2>0$.
		\end{itemize}
		Vậy hàm số đã cho đạt cực tiểu tại $x=2$.}
\end{ex} 

\begin{ex}
	Cho hàm số $y=x^2+4\ln(3-x)$. Tìm giá trị cực đai $y_\text{CĐ}$ của hàm số đã cho.
	\choice
	{$y_\text{CĐ}=2$}
	{\True $y_\text{CĐ}=4$}
	{$y_\text{CĐ}=1+4\ln2$}
	{$y_\text{CĐ}=1$}
	\loigiai{
		Tập xác định $\mathscr{D}=(-\infty;3)$.\\
		Đạo hàm $y'=2x-\dfrac{4}{3-x}=\dfrac{-2x^2+6x-4}{3-x}$.\\
		$y'=0\Leftrightarrow -2x^2+6x-4=0\Leftrightarrow \hoac{&x=1\\&x=2}$.\\
		Bảng biến thiên
		\begin{center}
			\begin{tikzpicture}[>=stealth]
				\tkzTabInit[nocadre=false,lgt=1,espcl=2,deltacl=0.5]{$x$/.7,$y'$/.7,$y$/2}
				{$-\infty$,$1$,$2$,$3$}
				\tkzTabLine{,-,0,+,0,-,d}
				\tkzTabVar{+/$+\infty$,-/$1+4\ln 2$,+/$4$,-D/$-\infty$}
			\end{tikzpicture}
		\end{center}
		Hàm số đạt cực đại tại $x=2$, $y_\text{CĐ}=4$.
	}
\end{ex} 


\begin{ex}%[2D1K2]
	Cho hàm số $y = f(x)$ xác định trên $\mathbb{R}$ và có đạo hàm $y' = f'(x) = 3x^3 - 3x^2$. Mệnh đề nào sau đây \textbf{sai}?
	\choice
	{Trên khoảng $(1;+\infty)$ hàm số đồng biến}
	{Trên khoảng $(-1;1)$ hàm số nghịch biến}
	{\True Đồ thị hàm số có hai điểm cực trị}
	{Đồ thị hàm số có một điểm cực tiểu}
	\loigiai
	{
		Ta có: $y' = 0 \Leftrightarrow 3x^3 - 3x^2 = 0 \Leftrightarrow \hoac{& x = 0 \\& x = 1.}$\\
		Bảng biến thiên:
		\begin{center}
			\begin{tikzpicture}[>=stealth]
				\tkzTabInit[nocadre, lgt=1, espcl=2.5]
				{$x$ /0.7,$y'$ /0.7,$y$ /1.7}
				{$-\infty$,$0$,$1$,$+\infty$}
				\tkzTabLine{,-,$0$,-,$0$,+,}
				\tkzTabVar{+/ $+\infty$, R, -/{\text{CT}}, +/ $+\infty$}
			\end{tikzpicture}
		\end{center}
		Hàm số đồng biến trên khoảng $(1;+\infty)$.\\
		Hàm số nghịch biến trên khoảng $(-\infty;1)$.\\
		Hàm số đạt cực tiểu tại $x = 1$.
	}
\end{ex} 

\begin{ex}%[2D1B2]
	Cho hàm số $ y=f(x) $ liên tục trên $ \mathbb{R} $ và có đạo hàm $ f'(x)=x(x-1)^2(x-2)^3 $. Số điểm cực trị của hàm số $ y=f(x) $ là
	\choice{1}{\True 2}{0}{3}
	\loigiai{	Ta có bảng xét dấu của $ f'(x) $:
		\begin{center}
			\begin{tikzpicture}
				\tkzTabInit[lgt=2,espcl=1.5]%
				{$x$ /1,$f'(x)$ /1}
				{$-\infty$ , $0$ , $1$ , $2$ ,$+\infty$}
				\tkzTabLine{ ,+,0,-,0,-,0,+,}
			\end{tikzpicture}
		\end{center}
		Dựa vào bảng xét dấu ta thấy $ f(x) $ có 2 điểm cực trị.
}\end{ex} 



\begin{ex}%[2D1K2-2]%
	\immini{Cho hàm số bậc bốn $ y=f(x) $. Biết $f'(x) $ có đồ thị như hình bên. Khẳng định nào sau đây là khẳng định đúng?
		\choice
		{Hàm số $f(x)$ đồng biến trên khoảng $(-\infty;0)$}
		{Hàm số $f(x)$ nghịch biến trên khoảng $(-1;1)$}
		{Hàm số $f(x)$ có đúng một điểm cực tiểu}
		{\True Hàm số $f(x)$ có đúng một điểm cực đại}
	}{
		\begin{tikzpicture}[>=stealth,line join=round,line cap=round,font=\footnotesize,scale=0.7,smooth]
			\draw[->] (-3,0)--(7,0)node[below]{$x$};
			\foreach \x in {-2,-1,1,2,3,4}\draw[shift={(\x,0)}] (0,2pt)--(0,-2pt) node[below]{\scriptsize $\x$};
			\draw[->] (0,-2)--(0,3)node[right]{$y$};
			\draw[] plot[smooth,tension=.65] coordinates{(-1.7,-2) (-1,0) (0,.7) (1,0)(2.7,-1.2)(4,0) (5,2.5)}node[right]{$y=f'(x)$};
		\end{tikzpicture}
	}
	\loigiai{
		\immini{Dựa vào đồ thị, ta có bảng biến thiên như hình vẽ. \\
		}{% Cần khai báo \usepackage{tkz-tab}
			\begin{tikzpicture}[scale=.8, font=\footnotesize, line join=round, line cap=round, >=stealth]
				\tkzTabInit[nocadre=false,lgt=1,espcl=2,deltacl=0.5]{$x$/.7 ,$y'$/.7,$y$/2}
				{$-\infty$ , $-1$ , $1$, $ 4 $, $+\infty$}
				\tkzTabLine{ , - , $0$ ,+, $ 0 $, -, $0$ , + , }
				\tkzTabVar{+/$+\infty$ , -/$f(-1)$ ,+/$f(-1)$ ,-/$ f(4) $, +/$+\infty$}
		\end{tikzpicture}}
	}
\end{ex} 

\begin{ex}
	\immini{
		Cho hàm số $y=f(x)$ xác định và liên tục trên $\mathbb{R}$. Biết rằng hàm số $f(x)$ có đạo hàm $f'(x)$ và hàm số $y=f'(x)$ có đồ thị như hình vẽ. Khi đó nhận xét nào sau đây đúng?
		\choice
		{\True Hàm số $f(x)$ không có cực trị}
		{Đồ thị hàm số $f(x)$ có đúng $2$ điểm cực tiểu}
		{Đồ thị hàm số $f(x)$ có đúng một cực đại}
		{Hàm số $f(x)$ có $3$ cực trị}
	}{
		\begin{tikzpicture}[scale=.8,font=\footnotesize, line join=round,line cap=round,>=stealth]
			\draw[->] (-2.5,0)--(2.5,0)node[below]{$x$};
			\draw[->] (0,-1)--(0,3.5)node[left]{$y$};
			\draw[samples=100,domain=-1.7:1.7] plot(\x,{(\x)^4-2*(\x)^2+1});
			\draw[dashed] (-1,0)node[below]{$-1$}circle(1pt) (1,0)node[below]{$1$}circle(1pt) (0,1)node[above right]{$1$}circle(1pt);
		\end{tikzpicture}
	}
	\loigiai{
		Dựa vào đồ thị ta thấy $f'(x)\geq 0$, với mọi $x\in\mathbb{R}$.\\
		Suy ra, hàm số $f(x)$ không có cực trị.
	}
\end{ex} 


\Closesolutionfile{ans}

\ind{PHẦN II.} \inden{Câu trắc nghiệm đúng sai. Trong mỗi ý a), b), c), d) ở mỗi câu, học sinh chọn đúng hoặc sai.}\\
\Opensolutionfile{ans}[ans/2D1-B1-d1-2]

\begin{ex}
	Cho hàm số $y=f(x)$ liên tục trên $\mathbb{R}$ và có bảng xét dấu đạo hàm như hình bên.
	\begin{center}
		\begin{tikzpicture}
			\tikzset{double style/.append style = {draw=\tkzTabDefaultWritingColor,double=\tkzTabDefaultBackgroundColor,double distance=2pt}}
			\tkzTabInit[nocadre=false, lgt=1, espcl=1.2]{$x$ /0.7,$y'$ /1}{$-\infty$,$0$,$1$,$2$,$+\infty$}
			\tkzTabLine{,+,$0$,-,d,+,$0$,+,}
		\end{tikzpicture}
	\end{center}
	% \immini{
		\choiceTF
		{Hàm số đồng biến trên khoảng $(-\infty;1)$}
		{\True Hàm số đồng biến trên khoảng $(1;+\infty)$}
		{Hàm số đạt cực đại tại $x=2$}
		{Hàm số có một điểm cực đại và hai điểm cực tiểu}
	% }{\vspace{0.1cm}
		%}
	\loigiai{
		Ta có bảng biến thiên như sau:
		\begin{center}
			\begin{tikzpicture}
				\tikzset{double style/.append style = {draw=\tkzTabDefaultWritingColor,double=\tkzTabDefaultBackgroundColor,double distance=2pt}}
				\tkzTabInit[lgt=1.1,espcl=2,nocadre=True]
				{$x$ /.7, $y'$ /.7,$y$ /2}
				{$-\infty$ , $0$, $1$ ,$2$ ,$+\infty$}
				\tkzTabLine{ ,+,$0$,-,d,+,$0$,+, }
				\tkzTabVar{ -/,+/ /,-/,R,+/$+\infty$}
			\end{tikzpicture}
		\end{center}
	Từ đây, suy ra:
		\begin{enumerate}[a)]
			\item Hàm số đồng biến trên khoảng $(-\infty;1)$ là khẳng định sai.
			\item Hàm số đồng biến trên khoảng $(1;+\infty)$ là khẳng định đúng.
			\item Hàm số đạt cực đại tại $x=2$ là khẳng định sai.
			\item Hàm số có một điểm cực đại và hai điểm cực tiểu là khẳng định sai.
		\end{enumerate}
	}
	
\end{ex} 

\begin{ex}
	Cho hàm số $y=x^3-3x^2+4$ có đồ thị $(C)$. Gọi $A$, $B$ là hai điểm cực trị của $(C)$.
	\choiceTF
	{\True Tập xác định của hàm số là $\mathbb{R}$}
	{Hàm số đồng biến trên khoảng $(0;2)$}
	{\True PTĐT qua hai điểm cực trị của đồ thị hàm số là $2x+y-4=0$}
	{\True Diện tích của tam giác $OAB$ bằng $4$, với $O$ là gốc tọa độ}
	\loigiai{
		\begin{enumerate}[a)]
			\item Hàm số đa thức nên có tập xác định là $D=\mathbb{R}$.
			\item Ta có 
			\begin{itemize}
				\item [$\bullet$] $y'=3x^2-6x$ và $y'=0 \Leftrightarrow x=0$ hoặc $x=2$.
			\end{itemize}
			Bảng biến thiên:
			\begin{center}
				\begin{tikzpicture}
					\tkzTabInit[lgt=1,espcl=3]
					{$x$ /0.7, $y'$ /0.7, $y$ /2.5}
					{$-\infty$,$0$,$2$,$+\infty$}
					\tkzTabLine{,+,$0$,-,$0$,+,}
					\tkzTabVar{-/$-\infty$,+/$4$,-/$0$,+/$+\infty$}
				\end{tikzpicture}
			\end{center}
		Suy ra hàm nghịch biến trên $(0;2)$.
			\item Tọa độ $A(0;4)$, $B(2;0)$. PTĐT $AB$ là
			$$\dfrac{x-0}{2-0}=\dfrac{y-4}{0-4} \Leftrightarrow 2x+y-4=0$$
			\item Diện tích tam giác vuông $OAB$ là $S_{OAB}=\dfrac{1}{2}OA \cdot OB=4$.
		\end{enumerate}

	}
\end{ex} 

\begin{ex}
	Cho hàm số $y=\dfrac{x^2+2x+2}{x+1}$ có đồ thị $(C)$. Gọi $A$, $B$ lần lượt là điểm cực tiểu và điểm cực đại của $(C)$.
	\choiceTF
	{Tập xác định của hàm số là $\mathbb{R}$}
	{Hàm số nghịch biến trên khoảng $(-2;0)$}
	{Tọa độ điểm $A(-2;-2)$, $B(0;2)$}
	{Khoảng cách giữa hai điểm cực trị là $AB=2\sqrt{5}$}
	\loigiai{
		\begin{enumerate}[a)]
			\item Đặt điều kiện mẫu số khác 0, ta được $x+1 \ne 0 \Leftrightarrow x \ne -1$. Suy ra $\mathscr{D}=\mathbb{R}\setminus \left\{-1\right\}$.
			\item $y'=\dfrac{x^2+2x}{(x+1)^2}\Rightarrow y'=0\Leftrightarrow \hoac{& x=-2 \\ & x=0.}$\\
			Ta có bảng xét dấu của hàm $f'(x)$ như sau
			\begin{center}
					\begin{tikzpicture}
					\tkzTabInit[nocadre=false,lgt=1,espcl=3]
					{$x$ /0.7,$y'$ /0.7,$y$ /2}
					{$-\infty$,$-2$,$-1$,$0$,$+\infty$}
					\tkzTabLine{,+,$0$,-,d,-,$0$,+,}
					\tkzTabVar{-/$-\infty$,+/$-2$,-D+/$-\infty$/$+\infty$,-/$2$,+/$+\infty$}
				\end{tikzpicture}
			\end{center}
			Dựa vào bảng xét dấu ta thấy rằng hàm số $y=f'(x)$ nghịch biến trên $(-2;-1)$ và $(-1;0)$.
			\item Tọa độ điểm $A(0;2)$, $B(-2;-2)$
			\item Độ dài $AB=\sqrt{(-2-0)^2+(-2-2)^2}=2\sqrt{5}$.
		\end{enumerate}

	}
\end{ex} 


\begin{ex}
	Xét một chất điểm chuyển động dọc theo trục $Ox$. Toạ độ của chất điểm tại thời điểm $t$ được xác định bởi hàm số $x(t)=t^3-6t^2+9t$ với $t\geq 0$. Khi đó $x'(t)$ là vận tốc của chất điểm tại thời điểm $t$, kí hiệu $v(t)$; $v'(t)$ là gia tốc chuyển động của chất điểm tại thời điểm $t$, kí hiệu $a(t)$.
	\choiceTF
	{Phương trình hàm vận tốc là $v(t)=3t^2-6t+9$}
	{\True Phương trình hàm gia tốc là $a(t)=6t-12$}
	{Vận tốc của chất điểm tăng khi $t\in (0;1)$ hoặc  $t \in (3;+\infty)$}
	{Vận tốc của chất điểm giảm khi $t\in (1;3)$}
	\loigiai{
		\begin{enumerate}
			\item $v(t)=x'(t)=3t^2-12t+9$
			\item $a(t)=v'(t)=6t-12$.
			\item Xét $v'(t)=6t-12$, $v'(t)=0\Leftrightarrow t=2$\\
			Bảng xét dấu
			\begin{center}
				\begin{tikzpicture}
					\tkzTabInit[nocadre=false,lgt=2,espcl=2.1]
					{$t$ /0.6,$v'(t)$ /0.6}
					{$0$,$2$,$+\infty$}
					\tkzTabLine{,-,$0$,+,}
				\end{tikzpicture}
			\end{center}
			Suy ra vận tốc của chất điểm tăng khi $t\in (2;+\infty) $, giảm khi $t\in (0;2)$.
		\end{enumerate}
	}
\end{ex} 

\Closesolutionfile{ans}
% \begin{dang}{Bài toán tìm m để hàm số đồng biến (nghịch biến) trên khoảng cho trước}
\begin{enumerate}[\iconCV]
\item Xét hàm số bậc ba $y=ax^3+bx^2+cx+d$ có $y'=3ax^2+2bx+c$.
	\begin{listEX}[1]
		\item [\ding{172}] Hàm số đồng biến trên  $\mathbb{R}$ khi và chỉ khi $$y' \ge 0,\,\forall x \in \mathbb{R} \Leftrightarrow \heva{&a>0\\&\Delta_{y'}\le 0}.$$
		\item [\ding{173}] Hàm số nghịch biến trên  $\mathbb{R}$ khi và chỉ khi $$y' \le 0, \,\forall x \in \mathbb{R} \Leftrightarrow \heva{&a<0\\&\Delta_{y'}\le 0}.$$
	\end{listEX}
\textit{Trường hợp hệ số $a$ có chứa tham số, ta kiểm tra thêm trường hợp $a=0$.}
\item Xét hàm phân thức $y=\displaystyle\frac{ax+b}{cx+d}$ có $y'=\dfrac{ad-cb}{(cx+d)^2}$, với $ad-cb \ne 0$ và $c \ne 0$.
\begin{itemize}
	\item [\ding{172}] Hàm số đồng biến trên từng khoảng xác định của nó khi và chỉ khi
	$$y'>0,\, \forall x \ne -\dfrac{d}{c}\Leftrightarrow ad-cb>0.$$
	\item [\ding{173}]  Hàm số nghịch biến trên từng khoảng xác định của nó khi và chỉ khi
	$$y'<0,\, \forall x \ne -\dfrac{d}{c}\Leftrightarrow ad-cb<0.$$
\end{itemize}
\item Xét hàm phân thức $y=\displaystyle\frac{ax^2+bx+c}{dx+e}$ có $y'=\dfrac{adx^2+2aex+be-dc}{(dx+e)^2}$, với $ad \ne 0$.
\begin{itemize}
	\item [\ding{172}] Hàm số đồng biến trên từng khoảng xác định của nó khi và chỉ khi
	$$y'\ge 0,\, \forall x \ne -\dfrac{e}{d}\Leftrightarrow adx^2+2aex+be-dc\ge 0,\, \forall x \ne -\dfrac{e}{d}.$$
	\item [\ding{173}]  Hàm số nghịch biến trên từng khoảng xác định của nó khi và chỉ khi
	$$y'\le 0,\, \forall x \ne -\dfrac{e}{d}\Leftrightarrow adx^2+2aex+be-dc\le 0,\, \forall x \ne -\dfrac{e}{d}.$$
\end{itemize}
\end{enumerate}
\end{dang}
\boxmini{BÀI TẬP TỰ LUẬN}
\setcounter{vd}{0}

\begin{vd}
	Tìm tất cả giá trị của tham số $m$ để hàm số
	\begin{tasks}
		\task $y=x^3+mx^2+2mx+2$ đồng biến trên $(-\infty;+\infty)$.
		\task $y=-\dfrac{1}{3}x^3-mx^2+\left(2m-3\right)x-m+2$ nghịch biến trên $\mathbb{R}$.
		\task $ y=\dfrac{1}{3}x^3-mx^2-(2m+1)x+1$ nghịch biến trên khoảng $(0;5)$.
		\task $y=x^3-3x^2+(5-m)x$ đồng biến trên khoảng $(2;+\infty)$.
	\end{tasks}
\loigiai{
\begin{enumerate}[a)]
	\item Hàm số đã cho có tập xác định $\mathscr{D}=\mathbb{R}$ và $y'=3x^2+2mx+2m$.\\
	Hàm số đã cho đồng biến trên $\mathbb{R}$ khi và chỉ khi
	\[y'\ge0,~\forall x\in\mathbb{R}\Leftrightarrow m^2-6m\le0\Leftrightarrow 0\le m\le6.\]
	\item Tập xác định: $D=\mathbb{R}$. Ta có $y'=-x^2-2mx+2m-3$.\\
	Để hàm số nghịch biến trên $\mathbb{R}$ thì:\\
	$y'\le 0,\forall x\in\mathbb{R} \Leftrightarrow\left\{
	\begin{aligned}
		&a_{y'}<0\\
		&\Delta'\le 0
	\end{aligned}
	\right.
	\Leftrightarrow \left\{
	\begin{aligned}
		&-1<0\\
		&m^2+2m-3\le0
	\end{aligned}
	\right.
	\Leftrightarrow -3\le m\le 1$.
	\item Tập xác định $\mathscr{D}=\mathbb{R}$.\\
	Ta có $y'=x^2-2mx-(2m+1)$, $ y'=0\Leftrightarrow\hoac{&x=-1\\&x=2m+1.}$\\
	Nếu $2m+1\leq-1\Leftrightarrow m\leq-1$ thì $y'\leq 0\Leftrightarrow x\in\left[2m+1;-1\right]$.\\
	Suy ra hàm số không nghịch biến trên khoảng $(0;5)$. \\
	$\Rightarrow m\leq-1$ không thỏa mãn.\\
	Nếu $2m+1>-1\Leftrightarrow m>-1$ thì $y'\leq 0\Leftrightarrow x\in\left[-1;2m+1\right]$.\\
	Để hàm số nghịch biến trên khoảng $(0;5)$ thì ta có $2m+1\geq 5\Leftrightarrow m\geq 2$.
	\item \textbf{\underline{Cách 1:}} Tập xác định $\mathscr{D}=\mathbb{R}$.\\
	Ta có $y'=3x^2-6x+5-m$.\\
	Hàm số $y=x^3-3x^2+(5-m)x$ đồng biến trên khoảng $(2;+\infty)$ khi và chỉ khi
	\allowdisplaybreaks
	\begin{eqnarray*}
		&&y'\ge 0,\,\forall x\in (2;+\infty)\\
		&\Leftrightarrow& 3x^2-6x+5-m\ge 0,\,\forall x\in (2;+\infty)\\
		&\Leftrightarrow& m\le 3x^2-6x+5, \,\forall x\in (2;+\infty)
	\end{eqnarray*}
	Xét hàm $g(x)=3x^2-6x+5$ trên $(2;+\infty)$ có $g'(x)=6x-6$ và $g'(x)=0\Leftrightarrow x=1$.\\
	Bảng biến thiên của $g(x)$
	\begin{center}
		\begin{tikzpicture}
			\tkzTabInit[nocadre=false,lgt=1.5,espcl=2,deltacl=0.5]
			{$x$/0.6,$g'(x)$/0.6,$g(x)$/1.5}
			{$2$,$+\infty$}
			\tkzTabLine{,+,}
			\tkzTabVar{-/$5$,+/$+\infty$}
		\end{tikzpicture}
	\end{center}
	Dựa vào bảng biến thiên của $g(x)$, ta được
	$$m\le 3x^2-6x+5, \,\forall x\in (2;+\infty) \Leftrightarrow m\le 5.$$
	\textbf{\underline{Cách 2:}} Tập xác định $\mathscr{D}=\mathbb{R}$.\\
	Ta có $y'=3x^2-6x+5-m$.\\
	Hàm số $y=x^3-3x^2+(5-m)x$ đồng biến trên khoảng $(2;+\infty)$ khi và chỉ khi
	$$y'\ge 0,\,\forall x\in (2;+\infty) 
	\Leftrightarrow \heva{& y'(2)\ge 0 \\ & -\dfrac{b}{2a} \le 2} 
	\Leftrightarrow \heva{& 5-m\ge 0 \\ & 1 \le 2}
	\Leftrightarrow m \le 5. $$
\end{enumerate}}
\end{vd}

\begin{vd}
	Tìm tất cả giá trị của tham số $m$ để hàm số
	\begin{tasks}
		\task $y=\dfrac{mx+2}{x+1}$ đồng biến trên từng khoảng xác định.
		\task $y=\dfrac{mx-2}{x+m-3}$ nghịch biến trên các khoảng xác định
		\task $y = \dfrac{mx-8}{x-2m}$ đồng biến trên $(3;+\infty )$.
		\task $y=\dfrac{mx+9}{4x+m}$ nghịch biến trên khoảng $(0;4)$.
	\end{tasks}
\loigiai{
\begin{enumerate}[a)]
	\item Từ yêu cầu bài toán, $\forall x \neq -1$ ta xét $y'>0$ $\Leftrightarrow m-2>0 \Leftrightarrow m>2$.
	\item Tập xác định $\mathbb{R}\setminus\{3-m\}$.\\
	$y' = \dfrac{m(m - 3) + 2}{\left( x + m - 3\right)^2} = \dfrac{m^2 - 3m + 2}{\left(x + m - 3\right)^2}$. \\
	Điều kiện để hàm số nghịch biến trên các khoảng xác định của nó là $y' < 0,\,\forall x \ne 3 - m$ hay $m^2 - 3m + 2 < 0 \Leftrightarrow m \in (1;2)$.
	\item Tập xác định: $\mathscr{D} = \mathbb{R} \setminus \{2m\}$.\\
	$y' = \dfrac{-2m^2+8}{(x-2m)^2}$.\\
	Hàm số luôn đơn điệu trên từng khoảng xác định $(-\infty; 2m)$ và $(2m; +\infty)$ khi $-2m^2 + 8 \ne 0$.\\
	Vậy hàm số đồng biến trên $(3;+\infty)$ khi và chỉ khi $-2m^2+8 > 0$ và $(3;+\infty) \subset (2m ;+\infty)$. \\
	Điều này tương đương $\heva{&-2<m<2\\&2m \le 3}$, hay $-2 < m \le \dfrac{3}{2}$.
	\item Tập xác định $\mathscr{D}=\mathbb{R}\setminus\left\{-\dfrac{m}{4}\right\}$.\\
	Ta có $y=\dfrac{mx+9}{4x+m}\Rightarrow y'=\dfrac{m^2-36}{(4x+m)^2}$.\\
	Để hàm số nghịch biến trên khoảng $(0;4)$ thì
	$$\heva{& y'<0 ,\forall x\in(0;4)\\ & -\dfrac{m}{4}\notin (0;4)}\Leftrightarrow\heva{& m^2-36<0 \\ &\hoac{&-\dfrac{m}{4}\geq4\\&-\dfrac{m}{4}\leq 0}}\Leftrightarrow\heva{& -6<m<6 \\ &\hoac{&m\leq-16\\&m\geq 0}}\Leftrightarrow 0\leq m<6.$$
\end{enumerate}}
\end{vd}

\begin{vd}
	Tìm tất cả giá trị của tham số $m$ để hàm số
	\begin{tasks}
		\task $ y = \dfrac{2x^2+3x+m+1}{x+1} $ đồng biến trên các khoảng xác định.
		\task $y=\dfrac{x^2+(m+1)x-1}{2-x}$ ($m$ là tham số) nghịch biến trên mỗi khoảng xác định.
	\end{tasks}
	\loigiai{
		\begin{enumerate}[a)]
			\item Tập xác định: $\mathbb{R}\setminus\{-1\}$.\\
			Ta có $y'=\dfrac{2x^2+4x+2-m}{(x+1)^2}$. Hàm số đồng biến trên các khoảng xác định khi 
			$$2x^2+4x+2-m\ge 0, \forall x\in \mathbb{R} \Leftrightarrow m\le \min\limits{\mathbb{R}\setminus \{-1\} } (2x^2+4x+2) = 0.$$
			\item Tập xác định $\mathscr{D}=\mathbb{R}\backslash\{2\}$.\\
			Đạo hàm: $y'=\dfrac{-x^2+4x+2m+1}{(2-x)^2}=\dfrac{g(x)}{(2-x)^2}$.\\
			Hàm số nghịch biến trên mỗi khoảng xác định của nó khi và chỉ khi $y'\le 0,\forall x\in \mathscr{D}$ (Dấu \lq\lq $=$\rq\rq~ chỉ xảy ra tại hữu hạn điểm thuộc $\mathscr{D}$).\\
			$\Leftrightarrow g(x)=-x^2+4x+2m+1\le 0,$  $\forall x\in \mathbb{R}$\\
			Điều kiện: ${\Delta}'\le 0$ (vì $a=-1<0$) $\Leftrightarrow 4-(-1)\cdot(2m+1)\le 0\Leftrightarrow 2m+5\le 0\Leftrightarrow m\le -\dfrac{5}{2}$.
	\end{enumerate}}
\end{vd}

\boxmini{BÀI TẬP TRẮC NGHIỆM}
\ind{PHẦN I.} \inden{Câu trắc nghiệm nhiều phương án lựa chọn. Học sinh trả lời từ câu 1 đến câu 17. Mỗi câu hỏi học sinh chỉ chọn một phương án.}\\
\setcounter{ex}{0}
\Opensolutionfile{ans}[ans/2D1-B1-d2-1]

\begin{ex}%[Nguyễn Trung Kiên, dự án 12-EX-7-2020]%[2D1B1-3]%
	Tất cả giá trị của $m$ để hàm số $y=\dfrac{x+m}{x-2}$ nghịch biến trên từng khoảng xác định là
	\choice
	{\True $m>-2$}
	{$m<-2$}
	{$m\leq -2$}
	{$m\geq -2$}
	\loigiai
	{Tập xác định $\mathscr{D}=\mathbb{R}\setminus \{2\}$ và $y'=\dfrac{-2-m}{(x-2)^2}$.\\
		Hàm số nghịch biến trên các khoảng $(-\infty;2)$ và $(2;+\infty)$ khi và chỉ khi
		\[y'<0,\, \forall x\neq 2\Leftrightarrow -2-m<0 \Leftrightarrow m>-2.\]}
\end{ex} 

\begin{ex}
	Cho hàm số $y=\dfrac{mx-2}{x+1-m}$. Tìm tất cả giá trị của tham số $m$ để hàm số đồng biến trên từng khoảng xác định.
	\choice
	{$\hoac{& m> 2\\& m< -1}$}
	{\True $-1<m<2$}
	{$-1\le m\le 2$}
	{$\hoac{& m\ge 2\\ &m\le -1}$}
	\loigiai{
		Yêu cầu bài toán $\Leftrightarrow ad-bc>0 \Leftrightarrow m(1-m)+2>0 \Leftrightarrow -1<m<2$.
	}
\end{ex} 

\begin{ex}
	Cho hàm số $ y=\dfrac{x+m}{x+2} $. Tập hợp tất cả các giá trị của $ m $ để hàm số đồng biến trên khoảng $ \left(0;+\infty\right)  $ là
	\choice
	{$ \left[2;+\infty\right) $}
	{$ \left(2;+\infty\right)  $}
	{$ \left(-\infty;2\right ]  $}
	{\True $\left(-\infty;2\right)   $}
	\loigiai{
		Hàm số xác định khi $ x\ne -2. $\\
		Có $ y'=\dfrac{2-m}{\left(x+2\right)^2 }, x\ne -2 $.\\
		Hàm số đồng biến trên $ (0;+\infty) $ khi và chỉ khi $ 2-m>0\Leftrightarrow m<2. $
	}
\end{ex} 

\begin{ex}
	Cho hàm số $f(x)=\dfrac{mx-4}{x-m}$ ( $m$ là tham số thực). Có bao nhiêu giá trị nguyên của $m$ để hàm số đồng biến trên khoảng $\left( 0;+\infty  \right)$?  
	\choice
	{$5$}
	{$4$}
	{$3$}
	{\True  $2$}
	\loigiai{
		Ta có $f'(x)=\dfrac{-m^2+4}{{{\left( x-m \right)}^{2}}}$\\
		Hàm số đồng biến trên khoảng $\left( 0;+\infty  \right)$ $\Leftrightarrow$ $\dfrac{-m^2+4}{\left( x-m \right)^2}>0,\,\, \forall x\in \left( 0;+\infty  \right)$\\
		$\Rightarrow \heva{
			& -m^2+4>0 \\ 
			& x\ne m\ \ \forall x\in \left( 0;+\infty  \right) \\ 
		}\Leftrightarrow \heva{
			& m\in \left( -2;2 \right) \\ 
			& m\in \left( -\infty ;0 \right] \\ 
		}\Leftrightarrow m\in \left( -2;0 \right]$\\
		Vậy có hai giá trị nguyên của $m$ là $-1$ và $0$.      
	}
\end{ex} 

\begin{ex}
	Tìm tất cả các giá trị của $m$ để hàm số $y=\dfrac{mx+4}{x+m}$ nghịch biến trên $(-\infty;1)$.
	\choice
	{$-2<m<2$}
	{$-2<m <-1$}
	{$-2\leq m <-1$}
	{\True $-2<m\leq-1$}
	\loigiai{
		ĐKXĐ: $x\neq-m$.\\
		Hàm số $y=\dfrac{mx+4}{x+m}$ nghịch biến trên $(-\infty;1)$\\$\Leftrightarrow y'=\dfrac{m^2-4}{(x+m)^2}<0$, $\forall x\in(-\infty;1)$
		$ \Leftrightarrow\heva{&m^2-4<0\\&-m\geq 1}\Leftrightarrow\heva{&-2<m<2\\&m\leq-1}\Leftrightarrow-2<m\leq-1 $.}
\end{ex} 

\begin{ex}%[THPT Tĩnh Gia - Thanh Hóa, 2020]%[Bùi Mạnh Tiến, 12EX7]%[2D1B1-3]%
	Số giá trị nguyên của tham số $m$ để hàm số $y=\dfrac{mx+10}{2x+m}$ nghịch biến trên khoảng $(0;2)$ là
	\choice
	{\True $6$}
	{$5$}
	{$4$}
	{$9$}
	\loigiai
	{
		Ta có $y'=\dfrac{m^2-20}{(2x+m)^2}$.\\
		Do đó hàm số $y=\dfrac{mx+10}{2x+m}$ nghịch biến trên $(0;2)$ khi và chỉ khi
		\begin{align*}
			\heva{& m^2-20<0 \\ & -\dfrac{m}{2}\notin (0;2)}\Leftrightarrow \heva{& -2\sqrt{5}<m<2\sqrt{5} \\ & \hoac{& -\dfrac{m}{2}\le 0 \\ & -\dfrac{m}{2}\ge 2}}\Leftrightarrow \hoac{& 0\le m<2\sqrt{5} \\ & -2\sqrt{5}<m\le -4.}
		\end{align*}
		Vì $m\in \mathbb{Z}$ nên $m\in \left\{-4;0;1;2;3;4\right\}$.\\
		Vậy có tất cả $6$ giá trị nguyên của $m$ thỏa mãn yêu cầu bài toán.
	}
\end{ex} 

\begin{ex}
	Có bao nhiêu giá trị nguyên của tham số $m$ để hàm số $y=x^3-2mx^2+\left(m^2+3\right)x$ đồng biến trên $\mathbb{R}$?
	\choice
	{$8$}
	{$6$}
	{\True $7$}
	{$0$}
	\loigiai{
		Hàm số $y=x^3-2mx^2+\left(m^2+3\right)x$ đồng biến trên $\mathbb{R}$
		\begin{eqnarray*}
			&\Leftrightarrow &y'=3x^2-4mx+m^2+3\ge 0, \, \forall x\in \mathbb{R}\\
			&\Leftrightarrow & \Delta'=4m^2-3\left(m^2+3\right)\le 0\\
			&\Leftrightarrow & m^2-9\le 0\Leftrightarrow-3\le m\le 3.
		\end{eqnarray*}
		Do $m$ là số nguyên nên $m\in \left\lbrace -3;-2;-1;0;1;2;3\right\rbrace $.\\
		Vậy có $7$ giá trị nguyên của tham số $m$.
	}
\end{ex} 

\begin{ex}
	Cho hàm số $y=-x^3-mx^2+(4m+9)x+5$. Có bao nhiêu giá trị nguyên của $m$ để hàm số nghịch biến trên $\mathbb{R}$?
	\choice
	{\True $7$}
	{$4$}
	{$5$}
	{$6$}
	\loigiai{
		Ta có $y'=-3x^2-2mx+(4m+9)$. Hàm số đã cho nghịch biến trên $\mathbb{R}$ khi và chỉ khi
		\[ \Delta'\le 0 \Leftrightarrow m^2+12m+27\le 0 \Leftrightarrow -9\le m\le -3. \]
		Vậy có tất cả $7$ giá trị nguyên của $m$ thỏa mãn bài toán.
	}
\end{ex} 

\begin{ex}
	Cho hàm số $y=(m-1)x^3 + (m-1)x^2 -2x+5$ với $m$ là tham số. Có bao nhiêu giá trị nguyên của $m$ để hàm số nghịch biến trên khoảng $(-\infty;+\infty)$?
	\choice
	{$5$}
	{\True $7$}
	{$8$}
	{$6$}
	\loigiai{
		\textbf{Trường hợp 1:} $m-1=0 \Leftrightarrow m=1$ khi đó $y=-2x+5$ nghịch biến trên $\mathbb{R}$. Do đó nhận $m=1$.\\
		\textbf{Trường hợp 2:} $m-1\ne 0 \Leftrightarrow m\ne 1$.\\
		Ta có $y'=3(m-1)x^2+2(m-1)x-2$. \\
		Hàm số nghịch biến trên $(-\infty;+\infty) $ $\Leftrightarrow y' \le 0 $, $\forall x\in (-\infty;+\infty)$
		$$\Leftrightarrow \heva{& 3(m-1)<0 \\ & (m-1)^2-3(m-1)\cdot (-2) \le 0} \Leftrightarrow \heva{& m<1 \\ & -5 \le m \le 1} \Leftrightarrow -5 \le m <1.$$.\\
		Do $m \in \mathbb{Z} \Rightarrow m\in \{-5;-4;-3;-2;-1;0\}$.\\
		Vậy cả $2$ trường hợp thì ta có tất cả $7$ giá trị $m$ thỏa yêu cầu bài toán là $\{-5;-4;-3;-2;-1;0;1\}$.
	}
\end{ex} 

\begin{ex}
	Tìm tất cả các giá trị thực của tham số $m$ để hàm số $y=x^3-3mx^2-9m^2x$ nghịch biến trên khoảng $(0;1)$.
	\choice
	{$-1<m<\dfrac{1}{3}$}
	{$m<-1$}
	{$m>\dfrac{1}{3}$}
	{\True $m\ge \dfrac{1}{3}$ hoặc $m\le -1$}
	\loigiai{
		Đặt $f(x)=y'=3x^2-6mx -9m^2$.\\
		Vì $y'$ là hàm số bậc hai với hệ số $a=3>0$ nên để hàm số nghịch biến trên $(0;1)$ thì phương trình $y'=0$ có hai nghiệm phân biệt $x_1, x_2$ thỏa mãn $x_1\le 0<1 \le x_2$ $$\Leftrightarrow \heva{&af(0)\le 0\\&af(1) \le0} \Leftrightarrow \heva{&-9m^2\le 0\\&3-6m-9m^2 \le 0} \Leftrightarrow \hoac{&x\le -1\\&x\ge \dfrac{1}{3}.}$$
	}
\end{ex} 

\begin{ex}
	Có bao nhiêu giá trị nguyên của tham số $ m$ thuộc khoảng $( -2019;2020 )$ để hàm số $ y=2x^3-3( 2m+1 )x^2+6m(m+1)x+2019$ đồng biến trên khoảng $(2;+\infty )$?
	\choice
	{\True $2020$}
	{$2018$}
	{$2021$}
	{$2019$}
	\loigiai{
		Ta có $y'=6x^2-6(2m+1)x+6m^2+6m$.\\
		Xét $y'=0$ $\Leftrightarrow x^2-(2m+1)x+m^2+m=0$, có $\Delta =(2m+1)^2-4\left( m^2+m \right)$ $=1>0$, $\forall m\in \mathbb{R}$. Suy ra phương trình $y'=0$ luôn có hai nghiệm phân biệt: $x_1=m$; $x_2=m+1$. Dễ thấy $x_1<x_2$.\\
		Bảng biến thiên
		\begin{center}
			\begin{tikzpicture}
				\tkzTabInit[nocadre=true,lgt=0.7,espcl=2.1]
				{$x$ /0.6,$y'$ /0.6,$y$ /2}
				{$-\infty$,$m$,$m+1$,$+\infty$}
				\tkzTabLine{,+,$0$,-,$0$,+,}
				\tkzTabVar{-/$-\infty$, +/$y(m)$,-/$y(m+1)$,+/$+\infty$}
			\end{tikzpicture}
		\end{center}
		Dựa vào bảng biến thiên ta thấy hàm số đồng biến trên mỗi khoảng $( -\infty ;m )$; $( m+1;+\infty )$. Vì thế, hàm số đồng biến trên $( 2:+\infty )$ khi $ m+1\le 2\Leftrightarrow m\le 1$.\\
		Suy ra có $2020$ giá trị nguyên của $ m$ thỏa mãn yêu cầu đề bài. }
\end{ex} 

\begin{ex}
	Tập hợp các giá trị thực của tham số $m$ để hàm số $y = - x^3 - 6x^2 + \left(4m - 9\right)x + 4$ nghịch biến trên khoảng $\left(- \infty; - 1\right)$ là
	\choice
	{$\left(- \infty; 0\right]$}
	{$\left[-\dfrac{3}{4}; +\infty\right)$}
	{\True $\left(- \infty; -\dfrac{3}{4}\right]$}
	{$\left[0; +\infty \right)$}
	\loigiai{ 
		Ta có $y'=-3x^2-12x+4m-9$. \\
		Hàm số đã cho nghịch biến trên khoảng $(-\infty;-1)$ khi và chỉ khi $y'\le 0$, $\forall x\in (-\infty;-1)$
		\begin{center}
			$\Leftrightarrow -3x^2-12x+4m-9\le 0\Leftrightarrow 4m\le 3x^2+12x+9$, $\forall x\in (-\infty;-1)$.
		\end{center}
		Đặt $g(x)=3x^2+12x+9\Rightarrow g'(x)=6x+12$. Giải $g'(x)=0\Leftrightarrow x=-2$.\\
		Bảng biến thiên của hàm số $g(x)$ trên $(-\infty;-1)$.
		\begin{center}
			\begin{tikzpicture}
				\tkzTabInit[nocadre=false,lgt=2,espcl=3.5,deltacl=0.6] %phần bắt buộc
				{$x$ /0.6,$g'(x)$ /0.6,$g(x)$ /2}%phần bắt buộc
				{$-\infty$,$-2$,$-1$}
				\tkzTabLine{,-,$0$,+,}
				\tkzTabVar{+/$+\infty$, -/$-3$,+/$0$}
			\end{tikzpicture}
		\end{center}
		Dựa vào bảng biến thiên suy  ra $4m\le g(x)$, $\forall x\in (-\infty;-1)\Leftrightarrow 4m\le -3\Leftrightarrow m\le -\dfrac{3}{4}$.
	}
\end{ex} 

\begin{ex}
	Tìm tất cả các giá trị thực của tham số $m$ sao cho hàm số $y=x^3-6x^2+mx+1$ đồng biến trên khoảng $\left(0;+\infty\right)$.
	\choice
	{$m\leq 12$}
	{\True $m\geq 12$}
	{$m\leq 0$}
	{$m\geq 0$}
	\loigiai{
		Tập xác định $\mathscr{D} =\mathbb{R}$.\\
		$y'=3x^2-12x+m$.\\
		Hàm số đồng biến trên khoảng $\left(0;+\infty\right)$ khi và chỉ khi
		{\allowdisplaybreaks
			\begin{eqnarray*}
				& & f'(x)\geq 0 , \forall x\in \left(0;+\infty\right) \\
				& \Leftrightarrow & 3x^2-12x+m \geq 0 , \forall x\in \left(0;+\infty\right) \\
				& \Leftrightarrow & m \geq -3x^2+12x , \forall x\in \left(0;+\infty\right).
		\end{eqnarray*}}
		Xét hàm số $g(x)= -3x^2+12x$ trên $\left(0;+\infty\right)$.
		Ta có $g'(x)=-6x+12 \Leftrightarrow x=2$.\\
		Bảng biến thiên của hàm số $g(x)$
		\begin{center}
			\begin{tikzpicture}
				\tkzTabInit[lgt=1.2,espcl=3]{$x$ /1, $y'$ /1,$y$ /2}{
					$0$,$2$,$+\infty$}
				\tkzTabLine{,+,0 ,-, }
				\tkzTabVar{-/$0$, +/$12$ ,-/$-\infty$ }
			\end{tikzpicture}
		\end{center}
		Suy ra hàm số đồng biến trên khoảng $\left(0;+\infty\right)$ khi $m \geq 12$.
	}
\end{ex} 

\begin{ex}
	Tìm tất cả các giá trị $m$ để hàm số $y=\dfrac{x^2-8x}{x+m}$ đồng biến trên mỗi khoảng xác định.
	\choice
	{$(-8;0)$}
	{$(0;8)$}
	{$[0;8]$}
	{\True $[-8;0]$}
	\loigiai{
		Ta có $y'=\dfrac{x^2+2mx-8m}{(x+m)^2}$. Khi đó
		\allowdisplaybreaks
		\begin{eqnarray*}
			\text{YCBT} &\Leftrightarrow & x^2+2mx-8m\ge 0, \forall x \Leftrightarrow \Delta' \le 0\\
			&\Leftrightarrow & m^2+8m\le 0\Leftrightarrow -8\le m\le 0.
		\end{eqnarray*}
	}
\end{ex} 

\begin{ex}
	Tập hợp các giá trị thực của tham số $m$ để hàm số $y=x+1+\dfrac{m}{x-2}$ đồng biến trên mỗi khoảng xác định của nó là
	\choice
	{$\left(-\infty;0\right)$}
	{$\left[0;1\right)$}
	{$\left[0;+\infty \right)\backslash \left\{1\right\}$}
	{\True $\left(-\infty;0\right]$}
	\loigiai{
		Tập xác định $\mathscr{D}=\mathbb{R}\backslash \left\{2\right\}$.
		Ta có $y'=1-\dfrac{m}{\left(x-2\right)^2}$.\\
		Hàm số đồng biến trên mỗi khoảng các định của nó khi và chỉ khi
		\begin{eqnarray*}
			&&y'\geq 0,\;\forall x\in \mathbb{R}\backslash \left\{2\right\}\Leftrightarrow 1-\dfrac{m}{\left(x-2\right)^2}\geq 0,\;\forall x\in \mathbb{R}\backslash \left\{2\right\}\\
			&\Leftrightarrow &m\le {\left(x-2\right)}^2,\;\forall x\in \mathbb{R}\backslash \left\{2\right\}\Leftrightarrow m\leq 0.
		\end{eqnarray*}
	}
\end{ex} 

\begin{ex}%[2D1K1-3]%
	Tìm tất cả các giá trị thực của tham số $ m $ để hàm số $ f(x)=2^{x^3-x^2+mx+1}$ đồng biến trên khoảng $(1; 2)$.
	\choice
	{$m\leq-8$}
	{$m>-8$}
	{\True $m\geq-1$}
	{$m<-1$}
	\loigiai{
		Ta có $ f'(x)=(3x^2-2x+m)\cdot 2^{x^3-x^2+mx+1}\cdot\ln 2 $.\\
		Ta thấy\allowdisplaybreaks{
			\begin{eqnarray*}
				&& f(x) \textrm{ đồng biến trên } (1; 2)\\
				\Leftrightarrow && (3x^2-2x+m)\cdot 2^{x^3-x^2+mx+1}\cdot\ln 2\geq 0,\forall x\in (1; 2)\\
				\Leftrightarrow && (3x^2-2x+m)\geq 0,\forall x\in (1; 2)\\
				\Leftrightarrow && m\geq (-3x^2+2x),\forall x\in (1; 2)\\
				\Leftrightarrow && m\geq\max\limits_{[1; 2]} (-3x^2+2x)\\
				\Leftrightarrow && m\geq-1.
			\end{eqnarray*}
		}
	}
\end{ex} 

\begin{ex}
	Có bao nhiêu giá trị nguyên dương của tham số $m$ để hàm số $f(x)=(x+1)\ln x+(2-m)x$ đồng biến trên khoảng $(0;\mathrm{e}^2)$?
	\choice
	{0}
	{3}
	{5}
	{\True 4}
	\loigiai
	{Hàm số đã cho xác định khi $x>0$ hay $D=\big(0;+\infty\big)$\\
			Với $x>0$, ta có $f'(x)=\ln x+\dfrac{x+1}{x}+2-m$.\\
			Hàm số đã cho đồng biến trên khoảng $(0;\mathrm{e}^2)$ khi
			\allowdisplaybreaks
			\begin{align*}
				f'(x) \geq 0, \forall x \in (0;\mathrm{e}^2) &\Leftrightarrow \ln x+\dfrac{x+1}{x}+2-m \geq 0, \forall x \in (0;\mathrm{e}^2)\\
				&\Leftrightarrow m \leq \ln x+\dfrac{x+1}{x}+2, \forall x \in (0;\mathrm{e}^2). \tag{$*$}
			\end{align*}
			Xét hàm số $g(x)=\ln x+\dfrac{x+1}{x}+2, \forall x \in (0;\mathrm{e}^2)$.\\
			Ta có $g'(x)=\dfrac{1}{x}-\dfrac{1}{x^2}=\dfrac{x-1}{x^2}$. Khi đó $g'(x)=0$ có nghiệm $x=1 \in (0;\mathrm{e}^2)$.\\
			Bảng biến thiên của hàm số $g$
			\begin{center}
				\begin{tikzpicture}
					\tkzTabInit[nocadre=false,lgt=1.5,espcl=3.5,deltacl=0.6] %phần bắt buộc
					{$x$/0.6, $g'(x)$/0.6, $g(x)$/2} %phần bắt buộc
					{$0$, $1$, $\mathrm{e}^2$} % hàng 1 cột 2
					\tkzTabLine{,-,z,+,}
					\tkzTabVar{+/$+\infty$,-/$4$,+/$g(\mathrm{e}^2)$}
				\end{tikzpicture}
			\end{center}
			Từ bảng biến thiên trên, bất phương trình $(*)$ thỏa mãn khi $m \leq 4$.
	}
\end{ex} 


\Closesolutionfile{ans}

\ind{PHẦN II.} \inden{Câu trắc nghiệm đúng sai. Học sinh trả lời từ câu 18 đến câu 20. Trong mỗi ý a), b), c), d) ở mỗi câu, học sinh chọn đúng hoặc sai.}\\
	
\Opensolutionfile{ans}[ans/2D1-B1-d2-2]

\begin{ex}
	Cho hàm số $ y=mx^3+mx^2-(m+1)x+1 $, với $m$ là tham số.
	\choiceTF
	{\True Hàm số là hàm số bậc ba khi $m \ne 0$}
	{\True Tập xác định của hàm số là $\mathbb{R}$}
	{Hàm số đồng biến trên $\mathbb{R}$ khi và chỉ khi $m<-\dfrac{3}{4}$ hoặc $m \ge 0$}
	{Hàm số nghịch biến trên $\mathbb{R}$ khi và chỉ khi $-\dfrac{3}{4}\leq m<0$}
	\loigiai{
		\begin{enumerate}[a)]
			\item Với $m \ne 0$ thì hàm số đã cho là một hàm số bậc ba.
			\item Hàm số là hàm đa thức nên có tập xác định là $\mathbb{R}$.
			\item Ta có $ y'=3mx^2+2mx-(m+1)$.
			\begin{itemize}
				\item [$\bullet$] Với $m=0$ thì $y'=-1<0$ (không thỏa)
				\item [$\bullet$] Với $m \ne 0$, yêu cầu bài toán tương đương với
				$\heva{&m>0\\&\Delta \le 0} \Leftrightarrow \heva{&m>0\\&4m^2+3m \le 0}$ (không tồn tại $m$)
			\end{itemize}
			\item 
			\begin{itemize}
				\item [$\bullet$] Với $m=0$ thì $y'=-1<0$ (thỏa)
				\item [$\bullet$] Với $m \ne 0$, yêu cầu bài toán tương đương với
				$$\heva{&m<0\\&\Delta \le 0} \Leftrightarrow \heva{&m<0\\&4m^2+3m \le 0} \Leftrightarrow -\dfrac{3}{4}\leq m<0$$
			\end{itemize}
		Suy ra $-\dfrac{3}{4}\leq m \leq 0$.
		\end{enumerate}
	}
\end{ex} 

\begin{ex}
	Cho hàm số $y=\dfrac{1}{3}x^3 + (m + 1)x^2 + \left(m^2 + 2m\right)x - 3$, với $m$ là tham số.
	\choiceTF
	{Tập xác định của hàm số là $\mathbb{R}$}
	{\True Phương trình $y'=0$ có hai nghiệm phân biệt $x_1=-m$ và $x_2=-m-2$}
	{\True Không tồn tại giá trị của tham số $m$ để hàm số đồng biến trên $\mathbb{R}$}
	{Hàm số nghịch biến trên khoảng $(- 1; 1)$ khi và chỉ khi $m \ge -1$}
	\loigiai{
		\begin{enumerate}[a)]
			\item Hàm số là hàm đa thức nên có tập xác định là $\mathbb{R}$
			\item Ta có $y'=x^2+2(m+1)x+m^2+2m$. Do $\Delta'=b'^2-ac=(m+1)^2-(m^2+2m)=1>0$ nên phương trình có hai nghiệm phân biệt
			$x_1=\dfrac{-b'+\sqrt{\Delta'}}{a}=-m$ và $x_2=\dfrac{-b'-\sqrt{\Delta'}}{a}=-m-2$.
			\item Bảng biến thiên
				\begin{center}
					\begin{tikzpicture}
						\tkzTabInit[lgt=1,espcl=3,nocadre=True]
						{$x$ /0.7, $y'$ /0.7, $y$ /2.5}
						{$-\infty$,$-m-2$,$-m$,$+\infty$}
						\tkzTabLine{,+,$0$,-,$0$,+,}
						\tkzTabVar{-/$-\infty$,+/$y(-m-2)$,-/$y(-m)$,+/$+\infty$}
					\end{tikzpicture}
				\end{center}
			Từ bảng biến thiên, suy ra không tồn tại giá trị của tham số $m$ để hàm số đồng biến trên $\mathbb{R}$
			\item Bảng biến thiên
			\begin{center}
				\begin{tikzpicture}
					\tkzTabInit[lgt=1,espcl=3,nocadre=True]
					{$x$ /0.7, $y'$ /0.7, $y$ /2.5}
					{$-\infty$,$-m-2$,$-m$,$+\infty$}
					\tkzTabLine{,+,$0$,-,$0$,+,}
					\tkzTabVar{-/$-\infty$,+/$y(-m-2)$,-/$y(-m)$,+/$+\infty$}
				\end{tikzpicture}
			\end{center}
			Từ bảng biến thiên, suy ra hàm số nghịch biến trên khoảng $(- 1; 1)$ khi và chỉ khi 
			$$\heva{&-m-2 \le -1\\& -m \ge 1} \Leftrightarrow m = -1.$$
		\end{enumerate}
		
	}
\end{ex} 

\begin{ex}
	Cho hàm số $ y=\dfrac{x+5}{x+m}$, với $m$ là tham số.
	\choiceTF
	{Tập xác định của hàm số là $\mathbb{R}$}
	{Hàm số đồng biến trên từng khoảng xác định khi và chỉ khi $m \ge 5$}
	{\True Hàm số nghịch biến trên từng khoảng xác định khi và chỉ khi $m < 5$}
	{Hàm số đồng biến trên khoảng $\left(-\infty ;\, -8\right)$ khi và chỉ khi $\left(5;\, 8\right)$}
	\loigiai{
		\begin{enumerate}[a)]
			\item Điều kiện $x+m \ne 0 \Leftrightarrow x \ne -m$. Tập xác định là $D=\mathbb{R} \backslash\{-m\}$.
			\item Ta có $y'=\dfrac{m-5}{\left( x+m \right)^2},\forall x\in \mathbb{R}\backslash \left\{ -m \right\}.$\\
			Hàm số đồng biến trên từng khoảng xác định $\Leftrightarrow m-5>0 \Leftrightarrow m>5$.
			\item Ta có $y'=\dfrac{m-5}{\left( x+m \right)^2},\forall x\in \mathbb{R}\backslash \left\{ -m \right\}.$\\
			Hàm số nghịch biến trên từng khoảng xác định $\Leftrightarrow m-5<0 \Leftrightarrow m<5$.
			\item 	Hàm số $ y=\dfrac{x+5}{x+m}$ đồng biến trên khoảng $\left(-\infty ;\, -8\right)$ khi và chỉ khi 
			$$\heva{
				&\dfrac{m-5}{\left(x+m\right)^2}> 0\\
				&-m\notin\left(-\infty ;\, -8\right)
			}\Leftrightarrow \heva{
				&m > 5\\
				&-m\ge-8
			} \Leftrightarrow 5 < m\le 8.$$
		\end{enumerate}
	}
\end{ex} 

\Closesolutionfile{ans}


% \begin{dang}{Bài toán tìm m để hàm số có cực trị hoặc đạt cực trị tại điểm cho trước}
	\begin{enumerate}[\iconCV]
		\item Tìm $m$ để hàm số $y=f(x)$ đạt cực trị tại điểm $x_0$ cho trước ($f(x)$ có đạo hàm tại $x_0$):
		\begin{listEX}[1]
			\item [\ding{172}] Giải điều kiện $y'(x_0)=0$, tìm $m$.
			\item [\ding{173}] Lập bảng biến thiên với $m $ vừa tìm được và chọn giá trị $m$ nào thỏa yêu cầu.
				\end{listEX}
	\item Biện luận cực trị hàm số $y=ax^3+bx^2+cx+d$.\\
	Tính $y'=3ax^2+2bx+c$ với $\Delta_{y'}=b^2-3ac$
	\begin{itemize}
		\item[\ding{172}] $\heva{&\Delta_{y'} >0\\&a \ne 0}$: Hàm số có hai điểm cực trị
		\item[\ding{173}]  $\Delta_{y'} \le 0$ hoặc suy biến $\heva{&a=0\\&b=0}$: Hàm số không có cực trị.
	\end{itemize}
	% \begin{note}
		\begin{enumerate}[\iconMT]
				\item Gọi $x_1$, $x_2$ là hai nghiệm phân biệt của $y'=0$ thì $x_1+x_2=-\dfrac{2b}{3a}$ và $x_1\cdot x_2 =\dfrac{c}{3a}$.
			\begin{itemize}
				\item [$\bullet$] $x_1^2+x_2^2=(x_1+x_2)^2-2x_1 x_2$
				\item [$\bullet$] $(x_1-x_2)^2=(x_1+x_2)^2-4x_1 x_2$
				\item [$\bullet$] $x_1^3+x_2^3=(x_1+x_2)^3-3x_1x_2(x_1+x_2)$.
			\end{itemize}
			\item Các công thức tính toán thường gặp:
			\begin{itemize}
				\item [$\bullet$] Độ dài $MN=\sqrt{(x_N-x_M)^2+(y_N-y_M)^2}$
				\item [$\bullet$]  Khoảng cách từ $M$ đến $\Delta$: $d(M,\Delta)=\dfrac{|Ax_M+By_M+C|}{\sqrt{A^2+B^2}}$, với $\Delta \colon Ax+By+C=0$.
				\item [$\bullet$] Tam giác $ABC$ vuông tại $A \Leftrightarrow \overrightarrow{AB} \cdot \overrightarrow{AC}=0 \Leftrightarrow \text{hoành}\cdot\text{hoành}+\text{tung}\cdot\text{tung}=0$.
				\item [$\bullet$] Diện tích tam giác $ABC$ là  $S=\dfrac{1}{2}|a_1b_2-a_2b_1|$, với $\overrightarrow{AB}=(a_1;b_1)$, $\overrightarrow{AC}=(a_2;b_2)$.
			\end{itemize}
			\item PTĐT qua hai điểm cực trị là $y=-\dfrac{2}{9a}(b^2-3ac)x+d-\dfrac{bc}{9a}$.
		\end{enumerate}
	% \end{note}
	\end{enumerate}
\end{dang}
\boxmini{BÀI TẬP TỰ LUẬN}
\setcounter{vd}{0}
\begin{vd}
	Tìm $m$ để hàm số
	\begin{tasks}
		\task  $y=\dfrac{x^3}{3}-mx^2+(m^2-m+1)x+1$ đạt cực tiểu tại $x=3$.
		\task  $y=x^3-3mx^2+3(m^2-1)x$ đạt cực đại tại $x_0=1$.
	\end{tasks}
\loigiai{
\begin{enumerate}[a)]
	\item Ta có $y'=x^2-2mx+m^2-m+1$. Hàm số đạt cực tiểu tại $x=3$ thì
	$$y'(3)=0 \Leftrightarrow 9-6m+m^2-m+1=0 \Leftrightarrow \hoac{&m=2\\&m=5}.$$
	Lập bảng biến thiên của hàm số với lần lượt hai giá trị $m$ vừa tìm được, ta thấy $m=2$ thỏa.\\
	Vậy $m=2$.
	\item Ta có $y'=3x^2-6mx+3(m^2-1)$\\
	Điều kiện cần và đủ để thỏa điều kiện bài toán là
	\begin{eqnarray*}
		\heva{&y'(1)=0 \\&y''(1)<0}
		\Leftrightarrow \heva{&3m^2-6m=0 \\&6-6m<0}
		\Leftrightarrow \heva{&m=0 \vee m=2 \\&m>1}
		\Leftrightarrow m=2.
	\end{eqnarray*}
	Vậy $m=2$ thì thỏa bài toán.
\end{enumerate}}
\end{vd}

\begin{vd}
	Tìm tất cả giá trị của tham số $m$ để hàm số (đồ thị hàm số)
	\begin{tasks}
		\task $ y=x^3-3x^2+2mx+m+2024$ có hai điểm cực trị.
		\task $ y=\dfrac{1}{3}x^3-mx^2+\left(m+2\right)x+2019$ không có cực trị.
		\task $y=x^3-3(m+1)x^2+12mx+2019$ có hai điểm cực trị $x_1,\ x_2$ thỏa mãn $x_1+x_2+2x_1x_2=-8$.
		\task $y=-x^3-3mx^2+m-2$ với $m$ là tham số có hai điểm cực trị $A,B$ sao cho $AB=2$.
	\end{tasks}
\loigiai{
\begin{enumerate}[a)]
	\item Ta có $y’=3x^2-6x+2m$.\\
	Hàm số có cực đại, cực tiểu khi và chỉ khi phương trình $y’=0$ có hai nghiệm phân biệt $\Leftrightarrow {\Delta }’_{y’}>0$ $\Leftrightarrow 9-6m>0$ $\Leftrightarrow m<\dfrac{3}{2}$.
	\item Ta có $y’=x^2-2mx+m+2$\\
	Hàm số đã cho không có cực trị $\Leftrightarrow$ phương trình $y’=0$ vô nghiệm hoặc có nghiệm kép hay ${\Delta }’_{y’} \le 0$ $\Leftrightarrow m^2-\left( m+2 \right)\le 0$ $\Leftrightarrow -1\le m\le 2$.
	\item Ta có $y'=3x^2-6(m+1)x+12m,\ y'=0\Leftrightarrow 3x^2-6(m+1)x+12m=0$. \\
	Hàm số có hai điểm cực trị $\Leftrightarrow \Delta '=9m^2-18m+9>0\Leftrightarrow m\ne 1$.\tagEX{1}
	Giả sử $x_1,\ x_2$ là hai nghiệm của phương trình $y'=0$, theo định lí Vi-ét ta có
	$$\heva{&x_1+x_2=-\dfrac{b}{a}=2(m+1)\\&x_1\cdot x_2=\dfrac{c}{a}=4m.}$$
	Do đó $x_1+x_2+2x_1\cdot x_2=-8\Leftrightarrow 2(m+1)+8m=-8\Leftrightarrow 10m=-10\Leftrightarrow m=-1$ thỏa mãn $(1)$.\\
	Vậy $m=-1$ là giá trị cần tìm của $m$.
	\item Ta có $y'=-3x^2-6mx$; $y'=0\Leftrightarrow
	\hoac{&x=0\\&x=-2m.\\}$\\
	Hàm số có hai điểm cực trị khi và chỉ khi $m\ne 0$.\\
	Gọi hai điểm cực trị của đồ thị hàm số là $A$, $B$.\\
	Ta có $A\left(0;m-2\right)$, $ B\left(-2m;-4{m}^{3}+m-2\right)$.\\
	Do đó
	{\allowdisplaybreaks
		\begin{align*}
			AB^2=4m^2+16m^6=4&\Leftrightarrow 4m^6+m^2-1=0\\
			&\Leftrightarrow m^2=\dfrac{1}{2}\Leftrightarrow m=\pm \dfrac{1}{\sqrt{2}}.
	\end{align*}}
\end{enumerate}}
\end{vd}

\boxmini{BÀI TẬP TRẮC NGHIỆM}
\ind{PHẦN I.} \inden{Câu trắc nghiệm nhiều phương án lựa chọn. Mỗi câu hỏi học sinh chỉ chọn một phương án.}\\
\setcounter{ex}{0}
\Opensolutionfile{ans}[ans/2D1-B1-d3-1]

\begin{ex}
	Tìm tất cả giá trị của tham số $m$ để hàm số $y=\dfrac{1}{3}x^3+(m+1)x^2+(1-3m)x+2$ có cực đại và cực tiểu.
	\choice
	{$m\leq-5;m\geq 0$}
	{\True $m <-5$; $m>0$}
	{$-5<m<0$}
	{$-5\leq m\leq 0$}
	\loigiai{
		Tập xác định $\mathscr{D}=\mathbb{R}$.\\
		Ta có $y’=x^2+2(m+1)x+1-3m$.\\
		Hàm số có cực đại và cực tiểu khi phương trình $y’=0$ có hai nghiệm phân biệt và đổi dấu qua các nghiệm đó.\\
		Khi đó $\Delta’_{y’}=(m+1)^2-(1-3m)>0\Leftrightarrow m^2+5m>0\Leftrightarrow \hoac{&m<-5\\&m>0.}$}
\end{ex} 

\begin{ex}
	Tìm tất cả các giá trị của tham số $ m $ để hàm số $ y=-x^3-3x^2+mx+2 $ có cực đại và cực tiểu.
	\choice
	{\True $m>-3$}
	{$m\geq 3$}
	{$m\geq-3$}
	{$m>3$}
	\loigiai{
		Ta có $ y'=-3x^2-6x+m $. Hàm số đã cho có cực đại và cực tiểu khi và chỉ khi phương trình $ y'=0 $ có $ 2 $ nghiệm phân biệt $\Leftrightarrow\Delta'>0\Leftrightarrow 9+3m>0\Leftrightarrow m>-3 $.
	}
\end{ex} 

\begin{ex}
	Cho hàm số $y=x^3-3(m+1)x^2+3(7m-3)x$. Số giá trị nguyên của tham số $m$ để hàm số không có cực trị là
	\choice
	{$2$}
	{$1$}
	{\True $4$}
	{$3$}
	\loigiai{
		Hàm số bậc $3$ không có cực trị khi và chỉ khi phương trình $y'=0 \Leftrightarrow 3x^2-6(m+1)x+3(7m-3)=0$ có nghiệm kép hoặc vô nghiệm hay
		$$\Delta' \le 0 \Leftrightarrow 9(m+1)^2-9(7m-3)\le 0 \Leftrightarrow m^2-5m+4 \le 0 \Leftrightarrow 1 \le m \le 4.$$
		Mà $m \in \mathbb{Z}$ nên $ m \in \{1;2;3;4\}$.\\
		Vậy có $4$ giá trị nguyên của $m$ thỏa mãn yêu cầu bài toán.
	}
\end{ex} 

\begin{ex}
	Cho hàm số $y=x^3-3(m+1)x^2+3(7m-3)x$. Gọi $S$ là tập hợp tất cả các giá trị nguyên của tham số $m$ để hàm số không có cực trị. Số phần tử của $S$ là
	\choice
	{$2$}
	{\True $4$}
	{$0$}
	{Vô số}
	\loigiai{
		Tập xác định là $\mathscr{D}=\mathbb{R}$.\\
		$y'=3x^2-6(m+1)x+3(7m-3)$.\\
		Hàm số không có cực trị khi và chỉ khi $\Delta'=9(m+1)^2-9(7m-3)\le 0\Leftrightarrow m^2-5m+4\le 0\Leftrightarrow 1\le m \le 4.$\\
		Vậy có $m\in \{1;2;3;4\}$.
	}
\end{ex} 

\begin{ex}
	Giả sử hàm số $ y=\dfrac{1}{3}x^3-x^2-\dfrac{1}{3}mx$ có hai điểm cực trị $x_1, x_2$ thỏa mãn $x_1+ x_2+2x_1x_2=0$. Giá trị của $ m$ là
	\choice
	{$ m=\dfrac{4}{3}$}
	{$ m=-3$}
	{\True $ m=3$}
	{$ m=2$}
	\loigiai{
		Ta có $y’=x^2-2x-\dfrac{1}{3}m$.\\
		$y’=0\Leftrightarrow  3x^2-6x-m=0$.\\
		Hàm số có hai cực trị $\Leftrightarrow y'=0$ có hai nghiệm phân biệt $\Leftrightarrow 9+3m>0 \Leftrightarrow m>-3$.\\
		Khi đó $x_1+ x_2 + 2x_1x_2=0 \Leftrightarrow 2-\dfrac{2m}{3}=0 \Leftrightarrow m=3$ (TM).}
\end{ex} 

\begin{ex}
	Cho hàm số $ f\left( x \right)=x^3-3x^2+mx-1$. Tìm giá trị của tham số $m$ để hàm số có hai cực trị $x_1, x_2$ thỏa $x_1^2+x_2^2=3$.
	\choice
	{$ m=\dfrac{1}{2}$}
	{$ m=-2$}
	{$ m=1$}
	{\True $ m=\dfrac{3}{2}$}
	\loigiai{
		TXĐ $D=\mathbb{R}$.\\
		${f}’\left( x \right)=3x^2-6x+m$.\\
		Hàm số có hai điểm cực trị $x_1, x_2 \Leftrightarrow {f}’\left( x \right)=0$ có hai nghiệm phân biệt $\Leftrightarrow 9-3m>0  \Leftrightarrow m<3$.\\
		Theo hệ thức Vi-et: $x_1+ x_2=2$; $x_1.x_2=\dfrac{m}{3}$.\\
		Khi đó: $x_1^2+x_2^2=3  \Leftrightarrow  \left( {x_1+ x_2} \right)^2 - 2x_1x_2=3 \Leftrightarrow 2^2-\dfrac{2m}{3}=3 \Leftrightarrow m=\dfrac{3}{2}$.}
\end{ex} 

\begin{ex}
	Tìm tất cả các giá trị của tham số $m$ để đồ thị hàm số $y=x^3-12x+m+2$ có hai cực trị và hai điểm cực
	trị này nằm về hai phía trục hoành?
	\choice
	{$m=-2$}
	{\True $-18<m<14$}
	{$\forall m\in \mathbb{R}$}
	{$m\neq 1$}
	\loigiai{
		Ta có $y'=3x^2-12$. Suy ra $y'=0\Leftrightarrow \hoac{& x=2\Rightarrow y=m-14 \\ & x=-2\Rightarrow y=m+18.}$\\
		Đồ thị hàm số có hai điểm cực trị nằm về hai phía trục hoành khi và chỉ khi
		$$(m-14)(m+18)<0\Leftrightarrow -18<m<14.$$
	}
\end{ex} 

\begin{ex}
	Tập hợp các giá trị của $m$ để đồ thị hàm số $y=x^3+mx^2-\left(m^2-4\right)x+1$ có hai điểm cực trị nằm ở hai phía của trục $Oy$ là
	\choice
	{$(-\infty;2)$}
	{\True $\mathbb{R}\setminus[-2;2]$}
	{$(-2;2)$}
	{$(2;+\infty)$}
	\loigiai{
		Ta có $y'=3x^2+2mx+4-m^2$.\\
		Đồ thị hàm số có hai cực trị nằm hai phía đối với trục $Oy$ khi và chỉ khi $y'=0$ có hai nghiệm trái dấu $\Leftrightarrow P=\dfrac{4-m^2}{3}<0\Leftrightarrow\hoac{&m>2\\&m<-2.}$}
\end{ex} 

\begin{ex}
	Cho hàm số $y=x^3+3mx^2+3(m^2-1)x+m^3.$ Tìm $m$ để hàm số đạt cực tiểu tại điểm $x=0.$
	\choice
	{$m=-1$}
	{\True $m=1$}
	{$m=0$}
	{$m=2$}
	\loigiai{
		Ta có $y'=3x^2+6mx+3(m^2-1)$ và $y''=6x+6m\Rightarrow y''(0)=6m.$\\
		Hàm số đạt cực tiểu tại $x=0\Rightarrow y'(0)=0\Leftrightarrow 3(m^2-1)=0\Leftrightarrow m=\pm 1.$\\
		Với $m=1\Rightarrow y''(0)=6>0\Rightarrow$  hàm số đạt cực tiểu tại $x=0.$\\
		Với $m=-1\Rightarrow y''(0)=-6<0\Rightarrow$  hàm số đạt cực đại tại $x=0.$\\
		Vậy $m=1$ thỏa mãn bài.
	}
\end{ex} 

\begin{ex}
	Hàm số $ y=x^3-2mx^2+m^2x-2 $ đạt cực tiểu tại $ x=1 $ khi
	\choice
	{$ m=3 $}
	{$ m=-3 $}
	{\True $ m=1 $}
	{$ m=-1 $}
	\loigiai{
		Ta có: $ y'=3x^2-4mx+m^2 ,
		y''=6x-4m. $\\
		Hàm số đạt cực tiểu tại $ x=1 $, suy ra $y'(1)=0\Leftrightarrow m^2-4m+3=0 \Leftrightarrow \hoac{&m=1\\&m=3.}$
		\begin{itemize}
			\item Với $m=1$ ta có $y'(1)=0, y''(1)=2>0$ nên hàm số đạt cực tiểu tại $x=1$.
			\item Với $m=3$ ta có $y'(1)=0, y''(1)=-6<0$ nên hàm số đạt cực đại tại $x=1$.
	\end{itemize}}
\end{ex} 

\begin{ex}
	Tìm giá trị thực của tham số $m$ để hàm số $y=\dfrac{1}{3}x^3-mx^2+(m^2-4)x+3$ đạt cực tiểu tại $x=3$.
	\choice
	{$m=-1$}
	{\True $m=1$}
	{$m=-7$}
	{$m=5$}
	\loigiai{
		Ta có $y'=x^2-2mx+m^2-4$ và $y''=2x-2m$.\\
		Hàm số đạt cực tiểu tại $x=3$ nên $y'(3)=0 \Leftrightarrow 9-6m+m^2-4=0 \Leftrightarrow \hoac{&m=5 \\ &m=1.}$\\
		Với $m=5$ thì $y''(3)=-4<0$, loại.\\
		Với $m=1$ thì $y''(3)=4>0$, thỏa mãn.
	}
\end{ex} 

\begin{ex}
	Đồ thị hàm số $y=x^3-3x^2+2ax+b$ (với $a, b \in \mathbb{R}$) có điểm cực tiểu $A(2;-2)$. Khi đó $a+b$ bằng
	\choice
	{$-4$}
	{$4$}
	{\True $2$}
	{$-2$}
	\loigiai{
		Ta có: $y’=3x^2-6x+2a; y''=6x-6$.\\
		Đồ thị hàm số có điểm cực tiểu $A(2;-2)$ nên ta có:
		$\heva{&y’(2)=0\\&y(2)=-2} \Leftrightarrow \heva{&2a=0\\&4a+b=2} \Leftrightarrow \heva{&a=0\\&b=2.}$\\
		Với $a=2,b=0$ ta thấy $y''(2)=6.2-6=6>0$ nên hàm số đạt cực tiểu tại $x=2$, thỏa yêu cầu bài toán.\\
		Suy ra $a+b=2$.
	}
\end{ex} 

\begin{ex}
	Gọi $m_1, m_2$ là các giá trị của tham số $m$ để đồ thị hàm số $y=2x^{3}-3x^{2}+m-1$ có hai điểm cực trị $B, C$ sao cho tam giác $OBC$ có diện tích bằng 2, với $O$ là gốc tọa độ. Tích $m_{1} \cdot m_{2}$ bằng
	\choice
	{$12$}
	{$6$}
	{\True $-15$}
	{$-20$}
	\loigiai{
		Tập xác định: $\mathscr{D}=\mathbb{R}$.\\
		Ta có \begin{eqnarray*}
			y'=6 x^{2}-6 x=0 &\Leftrightarrow&
			\hoac{x=0 \Rightarrow y=m-1 \Rightarrow B(0 ; m-1) \\ x=1 \Rightarrow y=m-2 \Rightarrow C(1 ; m-2)}\\
			&\Rightarrow& S_{\triangle OBC}=\dfrac{1}{2} d(C ; O B) \cdot O B=\dfrac{1}{2} \cdot 1 \cdot |m-1|=2\\
			&\Leftrightarrow& |m-1|=4
			\Leftrightarrow \hoac{&m_1=5 \\ &m_2=-3.}
		\end{eqnarray*}
		Vậy $m_1 \cdot m_2 = -15$.
	}
\end{ex} 

\begin{ex}
	Cho hàm số $y=x^3-3mx^2+3m^3$. Biết rằng có hai giá trị của tham số $m$ để đồ thị hàm số có hai điểm cực trị $A,B$ và tam giác $OAB$ có diện tích bằng $48$. Khi đó tổng các giá trị của $m$ là
	\choice
	{\True $0$}
	{$2$}
	{$\sqrt{2}$}
	{$-2$}
	\loigiai{
		Tập xác định $\mathscr{D}=\mathbb{R}$.\\
		Đạo hàm $y'=3x^2-6mx$, xác định với mọi $x\in\mathbb{R}$.\\
		$y'=0\Leftrightarrow\hoac{&x=0\\&x=2m.}$ \\
		Do đó hàm số có hai cực trị khi và chỉ khi $m\neq 0$.\\
		Khi đó $A\left(0;3m^3\right)$, $B\left(2m;-m^3\right)$.\\
		Suy ra $\overrightarrow{OA}=\left(0;3m^3\right)$, $\overrightarrow{OB}=\left(2m;-m^3\right)$.\\
		$S_{\triangle OAB}=48\Leftrightarrow \dfrac{1}{2}\left|\left[\overrightarrow{OA},\overrightarrow{OB}\right]\right|=48\Leftrightarrow \left|-6m^4\right|=96\Leftrightarrow m=\pm 2$.\\
		Vậy tổng các giá trị của $m$ là $0$.
	}
\end{ex} 
\Closesolutionfile{ans}

\ind{PHẦN II.} \inden{Câu trắc nghiệm đúng sai. Trong mỗi ý a), b), c), d) ở mỗi câu, học sinh chọn đúng hoặc sai.}\\
\Opensolutionfile{ans}[ans/2D1-B1-d3-2]

\begin{ex}
	Cho hàm số $ y=\dfrac{m}{3}x^3+2x^2+mx+1$, với $m$ là tham số.
	\choiceTF
	{Hàm số có hai điểm cực trị khi $-2<m<2$}
	{Hàm số có đúng một điểm cực trị khi $m=0$ hoặc $m=2$}
	{\True Hàm số không có cực trị khi $m \le -2$ hoặc $m \ge 2$}
	{\True Hàm số có $2$ điểm cực trị thỏa mãn $x_\text{CĐ}<x_{CT}$ khi $0<m<2$}
	\loigiai{
		\begin{enumerate}[a)]
			\item Ta có $y’=mx^2+4x+m$.\\
			Hàm số có $2$ điểm cực trị $\Leftrightarrow y’=0$ có $2$ nghiệm phân biệt $\Leftrightarrow \left\{ \begin{aligned}
				& m\ne 0 \\
				& 4-m^2>0 \\
			\end{aligned} \right.\Leftrightarrow \left\{ \begin{aligned}
				& m\ne 0 \\
				& -2<m<2 \\
			\end{aligned} \right.\quad(1)$.
			\item Hàm số có đúng 1 cực trị khi hàm số này bị suy biến về hàm bậc hai, nghĩa là $\dfrac{m}{3}=0 \Leftrightarrow m=0$.
			\item Với $m=0$ thì hàm số trở thành $y=2x^2+1$. Hàm số này có 1 điểm cực tiểu. Điều này không thỏa yêu cầu bài toán\\
			Với $m \ne 0$: Hàm số không có cực trị $\Leftrightarrow y’=0$ có vô nghiệm hoặc nghiệm kép. $\Leftrightarrow \left\{ \begin{aligned}
				& m\ne 0 \\
				& 4-m^2 \le 0 \\
			\end{aligned} \right.\Leftrightarrow \left\{ \begin{aligned}
				& m\ne 0 \\
				&  m \le -2,\,m \ge 2\\
			\end{aligned} \right.$.
			\item Dựa vào dạng đồ thị hàm số bậc $3$, hàm số có $2$ điểm cực trị thỏa mãn $x_\text{CĐ}<x_{CT}$ khi $ m>0$ $(2)$\\
			Từ $\left(1\right)$ và $\left(2\right)$ suy ra giá trị $ m$ cần tìm là $0<m<2$.
		\end{enumerate}
}
\end{ex} 

\begin{ex}
	Cho hàm số $y=x^3-3mx^2+3\left(m^2-1\right)x-m^3$ với $m$ là tham số.
	\choiceTF
	{\True Hàm số luôn có hai điểm cực trị với mọi $m$}
	{\True Hàm số đạt cực tiểu tại $x=3$ khi $m=2$}
	{\True Khi đồ thị hàm số có hai điểm cực trị thì khoảng cách giữa hai điểm cực trị bằng $2\sqrt{5}$}
	{\True Điểm cực tiểu của đồ thị hàm số luôn thuộc đường thẳng cố định với hệ số góc $k=-3$}
	\loigiai
	{
		\begin{enumerate}[a)]
			\item Ta có $y'=3x^2-6mx+3\left(m^2-1\right). y'=0\Leftrightarrow \hoac{&x_1=m-1\\&x_2=m+1}$.\\
			Do $x_1 \ne x_2, \,\forall m$ nên hàm số luôn có hai điểm cực trị.
			\item Dễ thấy $x=m+1$ là điểm cực tiểu. Suy ra, hàm số đạt cực tiểu tại $x=3$ khi $m+1=3 \Leftrightarrow m=2$.
			\item Với mọi $m$, tọa độ hai điểm cực trị là $A(m+1;-3m-2)$ và $B(m-1;-3m+2)$.\\
			Khoảng cách giữa hai điểm cực trị là $AB=\sqrt{(x_B-x_A)^2+(y_B-y_A)^2}=2\sqrt{5}$.
			\item Ta có $y'=3x^2-6mx+3\left(m^2-1\right). y'=0\Leftrightarrow \hoac{&x=m-1\\&x=m+1}$\\
			Vì là hàm số bậc ba với hệ số $a=1>0$ nên điểm cực tiểu của hàm số là $A\left(m+1;-3m-2\right)$. \\
			Lại có $-3m-2=-3\left(m+1\right)+1$ nên điểm cực tiểu của hàm số luôn thuộc đường thẳng $d:y=-3x+1$, hệ số góc $k=-3$.
		\end{enumerate}
	}
\end{ex} 

\begin{ex}
	Cho hàm số $y=\dfrac{x^2-2mx +m +2}{x-m}$, với $m$ là tham số.
	\choiceTF
	{\True Tập xác định của hàm số là $\mathbb{R}\backslash\{m\}$}
	{\True Có hai giá trị nguyên của tham số $m$ để hàm số có hai điểm cực trị}
	{\True Hàm số đạt cực đại tại $x=-1$ khi $m=\dfrac{1}{2}$}
	{Khi đồ thị hàm số có hai điểm cực trị thì đường thẳng qua hai điểm cực trị của đồ thị có phương trình là $y=2x-2m$}
	\loigiai{
		\begin{enumerate}[a)]
			\item Hàm số xác định khi $x-m \ne 0 \Leftrightarrow x \ne m$. Suy ra $\mathscr{D}=\mathbb{R}\backslash\{m\}$.
			\item $y'=\dfrac{x^2-2mx+2m^2-m-2}{(x-m)^2}$.\\
			Để hàm số có hai điểm cực trị thì $y'=0$ có hai nghiệm phân biệt khác $m$ hay $g(x)=x^2-2mx+2m^2-m-2$ có hai nghiệm phân biệt khác $m$.
			$$\Leftrightarrow \heva{&\Delta'>0\\&g(m) \ne 0} \Leftrightarrow \heva{&-m^2+m+2>0\\&m^2-m-2 \ne 0}  \Leftrightarrow m \in (-1;2).$$
			Vì $m$ nguyên nên $m \in \{0;1\}$.
			\item Hàm số đạt cực trị tại $x=-1$ thì $y'(-1)=0 \Leftrightarrow 2m^2+m-1 =0 \Leftrightarrow m=-1$ hoặc $m=\dfrac{1}{2}$.\\
			Thử lại với $m=\dfrac{1}{2}$, ta có $y'=\dfrac{x^2-x-2}{x-\dfrac{1}{2}}$.\\
				Bảng biến thiên
				\begin{center}
					\begin{tikzpicture}
						\tkzTabInit[nocadre=false,lgt=1,espcl=3]
						{$x$ /0.7,$y'$ /0.7,$y$ /2}
						{$-\infty$,$-1$,$0.5$,$2$,$+\infty$}
						\tkzTabLine{,+,$0$,-,d,-,$0$,+,}
						\tkzTabVar{-/$-\infty$,+/$y_1$,-D+/$-\infty$/$+\infty$,-/$y_2$,+/$+\infty$}
					\end{tikzpicture}
				\end{center}
			Suy ra $m=\dfrac{1}{2}$ thỏa yêu cầu bài toán.
			\item Cho hàm số $y=\dfrac{u(x)}{v(x)}$. Nếu đồ thị hàm số có hai điểm cực trị thì đường thẳng qua hai điểm cực trị có dạng $y=\dfrac{u'(x)}{v'(x)}$.\\
			Áp dụng, ta được $y=\dfrac{(x^2-2mx+m+2)'}{(x-m)'}=2x-2m$
		\end{enumerate}
	}
\end{ex} 

\Closesolutionfile{ans}

% \input{data/12/2D1-B1-4.tex}
% \begin{dang}{Cực trị hàm hợp, hàm chứa trị tuyệt đối}
    \begin{itemize}
        \item Các phép biến đổi đồ thị
        \begin{itemize}
            \item Đồ thị hàm $y=f(x+a)$ vẽ bằng cách dời đồ thị $y=f(x)$ sang trái $a$ đơn vị.
            \item Đồ thị hàm $y=f(x)+b$ vẽ bằng cách dời đồ thị $y=f(x)$ lên trên $b$ đơn vị.
            \item Đồ thị hàm $y=f(|x|)$ vẽ bằng cách "lật qua trái".
            \item Đồ thị hàm $y=|f(x)|$ vẽ bằng cách "lật lên".
            \item Đồ thị hàm $y=|f(|x|)|$ vẽ bằng cách "lật lên rồi lật qua trái".
        \end{itemize}
        \begin{note} Hàm $y=f(x)$ có $m$ điểm cực trị, $n$ nghiệm bội lẻ, $p$ điểm cực trị dương. Khi đó
            \begin{itemize}
                \item[-]Hàm $y=f(ax+b)+c$ cũng có $m$ điểm cực trị.
                \item[-]	Hàm $y=|f(x)|$ có $m+n$  điểm cực trị.
                \item[-] 	Hàm $y=f(|x|)$ có $2p+1$  điểm cực trị.
            \end{itemize}
        \end{note}
        \item Hàm $y=f(u)$.
        \begin{itemize}
            \item \textbf{Bước 1: } Tính đạo hàm $y'=u'f'(u)$.
            \item \textbf{Bước 2: } Lập bảng xét dấu của $y'$ hoặc đếm số nghiệm bội lẻ của $y'=0$.
            \item \textbf{Bước 3: } Kết luận.
        \end{itemize}
        \item Hàm $y=f(u)+g(x)$.
        \begin{itemize}
            \item \textbf{Bước 1: } Tính đạo hàm $y'=u'f'(u)+g'$.
            \item \textbf{Bước 2: } Lập bảng xét dấu của $y'$ hoặc đếm số nghiệm bội lẻ của $y'=0$ (dựa vào tương giao giữa hai đồ thị).
            \item \textbf{Bước 3: } Kết luận.
        \end{itemize}
    \end{itemize}
\end{dang}
\begin{vd}
    \immini{Cho hàm số $y=f(x)$ có bảng biến thiên như hình vẽ. Tìm các điểm cực trị, các cực trị của hàm số sau
        \begin{listEX}[1]
            \item $y=f(x+2)$
            \item $y=f(x)-3$
            \item $y=f(2x-3)+1$
            \item $y=f(1-2x)+2025$
    \end{listEX}}{\begin{tikzpicture}[>=stealth]
            \tkzTabInit[nocadre=false,lgt=1,espcl=2,deltacl=0.5]{$x$/.6 ,$y'$/.6,$y$/1.8}
            {$-\infty$ , $0$ , $2$ , $+\infty$}
            \tkzTabLine{ , - , $0$ , + , $0$ , - , }
            \tkzTabVar{+/$+\infty$ , -/$1$ , +/$5$ , -/$-\infty$}
    \end{tikzpicture}}
    \loigiai{}
\end{vd}
\begin{vd}
    \immini{Cho hàm số $y=f(x)$ có bảng biến thiên như hình vẽ.Tìm các điểm cực trị của hàm số sau
        \begin{listEX}[1]
            \item $y=f(x^2)$
            \item $y=f(3x^2-2x)$
            \item $y=f(\sqrt{x^2+2x+2})$
    \end{listEX}}{
        \begin{tikzpicture}[>=stealth]
            \tkzTabInit[nocadre=false,lgt=1,espcl=2,deltacl=0.5]{$x$/.6,$y'$/.6,$y$/1.8}
            {$-\infty$ , $0$ , $2$ , $+\infty$}
            \tkzTabLine{ , - , $0$ , + , $0$ , - , }
            \tkzTabVar{+/$+\infty$ , -/$1$ , +/$5$ , -/$-\infty$}
        \end{tikzpicture}
    }
    \loigiai{}
\end{vd}
\begin{vd}%[2D1G5-5]
    \immini{Cho hàm số $y=f(x)$ có đồ thị $y=f'(x)$ như hình vẽ. Tìm số điểm cực trị của các hàm số sau
        \begin{listEX}[2]
            \item $y=f(x)$
            \item $y=2f(x)-x$
            \item $y=f(3x)+2x$
            \item $y=f(x)+\dfrac{x^2}{2}-x$
            \item $y=3f(x)-2x^3$
            \item $y=f(2x+1)-4x$
    \end{listEX}}{\begin{tikzpicture}[smooth, >=latex, line cap =round, line join =round,font=\scriptsize,x=1.4cm]
            \begin{scope}[scale=.5]
                \draw[->] (-3,0)--(3,0) node[below]{$x$};
                \draw[->] (0,-2.5) -- (0,3) node[left] {$y$};
                \draw[ name path=dcong] (-2,-2)..controls +(90:0.3) and +(180:0.3)..(-0.7,2.7)..controls +(0:0.2) and +(180:0.3)..(0,0.5)..controls +(30:0.2) and +(180:0.3)..(1,2)..controls +(0:0.3) and +(90:0.1).. (2,-2);
                \draw[thick,dashed] (-1,0) node[below] {$-1$} --(-1,2) --(1,2) -- (1,0) node[below] {$1$} (0,2) node[above right] {$2$};
            \end{scope}
    \end{tikzpicture}}
    \loigiai{}
\end{vd}
\begin{vd}
    \immini{Cho hàm số $y=f(x)$ có đồ thị như hình vẽ. Tìm số điểm cực trị của hàm số
        \begin{listEX}[2]
            \item $y=f(|x|)$
            \item $y=|f(x)|$
            \item $y=|f(|x|)|$
            \item $y=f(|x|-a)$
            \item $y=f(|x+b|)$
            \item $y=|f(x+2025)|$
    \end{listEX}}{
        \begin{tikzpicture}[>=stealth,line join=round, line cap=round, font=\scriptsize]
            \begin{scope}[scale=.8]
                \draw[-stealth](-4,0)--(0,0)node[below right]{$O$}--(4,0)node[below left]{$x$};
                \draw[-stealth](0,-2)--(0,3)node[below left]{$y$};
                \draw[dashed]
                (-3,0)node[above]{$a$}--(-3,-2)
                (3,0)node[below]{$b$}--(3,3)
                ;
                \draw[smooth]
                (-3,-2)..controls+(85:3) and+(180:.5)..(-2,2)
                ..controls+(0:.5) and+(180:.5)..(-1,1)
                ..controls+(0:.5)and+(180:.5)..(0.5,2)
                ..controls+(0:.5)and+(180:.75)..(1.5,-1.5)
                ..controls+(0:.75)and+(-95:.3)..(3,3)
                ;
            \end{scope}
    \end{tikzpicture}}
    \loigiai{}
\end{vd}
\begin{vd}
    Tìm $m$ để
    \begin{listEX}
        \item  Hàm số $y=|f(x)|$ có $5$ điểm cực trị, với  $f(x)= 3x^3+3x^2+mx+m$
        \item Hàm số $y=f\left(\vert x\vert\right)$ có $5$ điểm cực trị, với $f(x)=x^3-(2m-1)x^2+(2-m)x+2$.
    \end{listEX}
    \loigiai{
        \begin{listEX}
            \item Đặt $f(x)=3x^3+3x^2+mx+m=3x^2(x+1)+m(x+1)=(x+1)(3x^2+m)$.\\
            Suy ra $f'(x)=9x^2+6x+m$.\\
            Phương trình $f'(x)=0$ có $2$ nghiệm phân biệt $x_1$, $x_2$ khi và chỉ khi $\Delta'=9-9m>0\Leftrightarrow m<1$. Khi đó ta có $x_1+x_2=-\dfrac{2}{3}$, $x_1x_2=\dfrac{m}{9}$.\\
            Hàm số $y=|f(x)|$ có $5$ điểm cực trị khi và chỉ khi $\heva{&\Delta'>0\\&y(x_1)\cdot y(x_2)<0.}$\\
            Thực hiện biến đổi
            \allowdisplaybreaks
            \begin{eqnarray*}
                y(x_1)\cdot y(x_2) &=&\ (x_1+1)(3x_1^2+m)\cdot(x_2+1)(3x_2^2+m)\\
                &=&\ \left[9(x_1x_2)^2+3m(x_1^2+x_2^2)+m^2\right]\left(x_1x_2+x_1+x_2+1\right)\\
                &=&\ \left[\dfrac{m^2}{9}+3m\left[\left(-\dfrac{2}{3}\right)^2-\dfrac{2m}{9}\right]+m^2\right]\left(\dfrac{m}{9}-\dfrac{2}{3}+1\right)\\
                &=&\ \dfrac{1}{9}(4m^2+12m)(m+3).
            \end{eqnarray*}
            Suy ra $y(x_1)\cdot y(x_2)<0\Leftrightarrow (4m^2+12m)(m+3)<0\Leftrightarrow -3\neq m<0$.\\
            Kết hợp với điều kiện $m$ là số nguyên thỏa $|m|<10$ ta được $m\in\{-1;-2;-4;-5;-6;-7;-8;-9\}$.\\
            Vậy có $8$ giá trị nguyên của tham số $m$.
            \item Tập xác định $\mathscr{D}=\mathbb{R}$.\\
            Ta có $f\left(|-x|\right)=f\left(|x|\right)$, $\forall x\in\mathbb{R}$ nên $y=f\left(|x|\right)$ là hàm số chẵn. \\
            Do đó, đồ thị hàm số $y=f\left(|x|\right)$ đối xứng qua trục tung.\\
            Suy ra hàm số $y=f\left(|x|\right)$ luôn có một điểm cực trị là $x=0$.\\
            Do đó, $y=f\left(|x|\right)$ có $5$ điểm cực trị $\Leftrightarrow$ hàm số $y=f(x)$ có $2$ điểm cực trị dương.\\
            \phantom{Do đó, số $y=f\left(|x|\right)$ có $5$ điểm cực trị} $\Leftrightarrow$ $f'(x)=0$ có hai nghiệm dương phân biệt.\\
            Ta có $f'(x)=3x^2-2(m-1)x+2-m$.\\
            Yêu cầu bài toán $\Leftrightarrow\heva{&\Delta'>0 \\ &S>0 \\ &P>0}\Leftrightarrow\heva{&4m^2-m-5>0 \\ &2m-1>0 \\ &2-m>0}\Leftrightarrow\heva{&m<-1\;\text{hoặc}\;m>\dfrac{5}{4} \\ &m>\dfrac{1}{2} \\ &m<3}\Leftrightarrow \dfrac{5}{4}<m<2$.
        \end{listEX}
    }
\end{vd}
\boxmini{BÀI TẬP TRẮC NGHIỆM}
\Opensolutionfile{ans}[ans/2D1-2-DANG-3]
\begin{ex}%[2D1K2-6]
    \immini
    {Cho hàm số $f(x)$ có đồ thị $f'(x)$ có đồ thị như hình vẽ bên dưới.\\ Hàm số $y=f(1-2x)$ có bao nhiêu cực trị ?
        \choice[2]
        {$4$}
        {$7$}
        {\True $3$}
        {$9$}
    }
    {
        \begin{tikzpicture}[>=stealth,font=\scriptsize]
            \begin{scope}[scale=0.55]
                \draw[->] (0,-1)--(0,3.5)node[right]{\scriptsize $y$};
                \draw[->] (-2,0)--(5,0)node[below]{\scriptsize $x$};
                \fill (0,0) node[below left]{\scriptsize  $O$} circle(1.5pt);
                \draw (-0.8,0) node[below left]{ $-1$} (0.9,0) node[below left]{ $1$} (2,0) node[below]{ $2$} (4,0) node[below]{ $4$};
                \clip (-2,-1) rectangle (5,3.5);
                \draw[] plot[smooth,tension=.65] coordinates{(-1.05,-0.9) (-0.3,2.5) (1.2,-0.5) (2.7,0.7) (4.2,0.2) (4.8,3.5)};
            \end{scope}
        \end{tikzpicture}
    }
    \loigiai{
        Đặt $g(x)=f(1-2x)$\\
        Dựa vào đồ thị, ta thấy $f'(x)=0$ có nghiệm $x_1=-1,x_2=1,x_3=2$ và $x_4=4$ nên $f'(x)$ có dạng $$f'(x)=k(x+1)(x-1)(x-2)(x-4)$$
        Khi đó $g'(x)=-2f'(1-2x)=-2k(2-2x)(-2x)(-1-2x)(-3-2x)^2$
        $$g'(x)=0 \Leftrightarrow \hoac{&x=1\\&x=0\\&x=-\dfrac{1}{2}\\&x=-\dfrac{3}{2} \text{ (kép)}}$$
        Bảng xét dấu $g'(x)$
        \begin{center}
            \begin{tikzpicture}[every node/.style={circle,fill=white,inner sep=0pt},arrow/.style={>=stealth,->,shorten <= 0.3cm,shorten >= 0.3cm},font=\footnotesize,xscale=1,yscale=1]
                \def\mnumline{1} %Số dòng
                \def\mnumcol{11} %Số cột
                \foreach \j in {0,...,\mnumline}
                \foreach \i in {0,...,\mnumcol}{
                    \coordinate (\j\i) at (\i,-\j);
                }
                \pgfmathsetmacro\yline{\mnumline/2-1}
                \path node at (00){$x$} node at (10){$g'(x)$};
                \foreach \x/\mnamex in {01/$-\infty$,03/$-\dfrac{3}{2}$,05/$-\dfrac{1}{2}$,07/$0$,09/$1$,0\mnumcol/$+\infty$} \path node at (\x) {\mnamex};
                \foreach \dy/\mnamedy in {12/$-$,13/$0$,14/$-$,15/$0$,16/$+$,17/$0$,18/$-$,19/$0$,110/$+$} \path node at (\dy) {\mnamedy};
                \draw[thick] (-.5,.5)rectangle([xshift=0.5cm,yshift=-0.5cm]\mnumline\mnumcol) ([xshift=-0.5cm,yshift=-0.5cm]00)--([xshift=0.5cm,yshift=-0.5cm]0\mnumcol)  ([xshift=0.5cm,yshift=0.5cm]00)--([xshift=0.5cm,yshift=-0.5cm]\mnumline0);
            \end{tikzpicture}
        \end{center}
        Dựa vào bảng xét dấu, ta thấy $g'(x)$ đổi dấu 3 lần nên $y=f(1-2x)$ có 3 cực trị.
    }
\end{ex}
\begin{ex}%[2D1K2-2]
    \immini{Cho hàm số $ f(x) $ có đạo hàm là $ f'(x) $. Đồ thị của hàm số $ y=f'(x) $ như hình vẽ bên. Khi đó hàm số $ y=f(x^2) $ có bao nhiêu điểm cực trị?
        \choice[2]
        {$2$}
        {$4$}
        {\True $3$}
        {$5$}}
    {
        \begin{tikzpicture}[line join=round, line cap=round,>=stealth,font=\scriptsize]
            \begin{scope}[scale=0.35]
                \tikzset{label style/.style={font=\footnotesize}}
                \def \xmin{-1.5}
                \def \xmax{6.5}
                \def \ymin{-2}
                \def \ymax{5.5}
                \def \hamso{-0.11*(\x)^3+1.09*(\x)^2-1.73*(\x)}
                \draw[->] (\xmin,0)--(\xmax,0) node[below left] {$x$};
                \draw[->] (0,\ymin)--(0,\ymax) node[below left] {$y$};
                \draw (0,0) node [below left] {$O$};
                \begin{scope}
                    \clip (\xmin+0.01,\ymin+0.01) rectangle (\xmax-0.01,\ymax-0.01);
                    \draw[samples=350,domain=-1.2:5.5,smooth,variable=\x] plot (\x,{\hamso});
                \end{scope}
                \draw [dashed] (5,4.6)--(5,0) node[below]{$5$} (2,0) node[below]{$2$};
            \end{scope}
        \end{tikzpicture}
    }
    \loigiai{$ y'=2xf'(x^2) $. Cho $ y'=0 \Leftrightarrow \hoac{&x=0\\&f'(x^2)=0} \Leftrightarrow \hoac{&x=0\\&x^2=0\\&x^2=2} \Leftrightarrow \hoac{&x=0\\&x=0 \text{ (nghiệm kép)}\\&x=\pm \sqrt{2}} $.\\
        $ y'=0 $ có 3 nghiệm bội bậc lẻ nên hàm số có 3 điểm cực trị.
    }
\end{ex}
\begin{ex}%[2D1K2-6]
    \immini{	Cho hàm số $y=f(x)$ xác định trên $\mathbb{R}$ và hàm số $y=f'(x)$ có đồ thị như hình vẽ. Hàm số $y=f(1-x^2)$ đạt cực đại tại điểm nào sau đây?
        \choice[2]
        {$x=-1$}
        {\True $x=\pm \sqrt{2}$}
        {$x=3$}
        {$x=0$}}{
        \begin{tikzpicture}[>=stealth, font=\scriptsize, line join=round, line cap=round,y=0.7cm]
            \begin{scope}[scale=.5]
                \def\a{1} \def\b{-2} \def\c{-2.5} % Hệ số
                \def\xmin{-2} \def\xmax{4}
                \def\ymin{-4} \def\ymax{1.5}
                %\draw[color=gray!50,dashed] (\xmin,\ymin) grid (\xmax,\ymax);
                \draw[->] (\xmin,0)--(\xmax,0);
                \draw[->] (0,\ymin)--(0,\ymax);
                \node at (0,0) [below right]{$O$};
                \node at (-1,0) [below left]{$-1$};
                \node at (3,0) [below right]{$3$};
                \clip (\xmin+0.1,\ymin+0.1) rectangle (\xmax-0.5,\ymax-0.1);
                \draw[smooth,samples=300,domain=-1.3:3.3] plot(\x,{\a*(\x)^2+\b*(\x)+\c});
            \end{scope}
    \end{tikzpicture}}
    \loigiai{Đặt $g(x)=f(1-x^2)$\\
        Khi đó $g'(x)=-2x\cdot f'(1-x^2)$\\
        Cho $g'(x)=0 \Leftrightarrow -2x \cdot f'(1-x^2) =0$
        $$ \Leftrightarrow \hoac{&x=0\\&f'(1-x^2)=0 \Leftrightarrow \hoac{&1-x^2=-1\Leftrightarrow x^2=2 \Leftrightarrow x=\pm \sqrt{2}\\&1-x^2=3}}$$
        Bảng xét dấu
        \begin{center}
            \begin{tikzpicture}[every node/.style={circle,fill=white,inner sep=0pt},arrow/.style={>=stealth,->,shorten <= 0.3cm,shorten >= 0.3cm},font=\footnotesize,xscale=1.4,yscale=.8]
                \def\mnumline{3} %Số dòng
                \def\mnumcol{9} %Số cột
                \foreach \j in {0,...,\mnumline}
                \foreach \i in {0,...,\mnumcol}{
                    \coordinate (\j\i) at (\i,-\j);
                    %	\draw[gray!30] ([xshift=-0.5cm,yshift=0.5cm]\j\i)--([xshift=0.5cm,yshift=0.5cm]\j\i)--([xshift=0.5cm,yshift=-0.5cm]\j\i)--([xshift=-0.5cm,yshift=-0.5cm]\j\i)--cycle (\j\i)node[]{\j\i}; %Ẩn lệnh này sau khi hoàn thành BBT
                }
                \pgfmathsetmacro\yline{\mnumline/2-1}
                \path node at (00){$x$} node at (10){$-x$} node at (20){\scriptsize $f'(1-x^2)$} node at (30){$g'(x)$};
                \foreach \x/\mnamex in {01/$-\infty$,03/$-\sqrt{2}$,05/$0$,07/$\sqrt{2}$,0\mnumcol/$+\infty$} \path node at (\x) {\mnamex};
                \foreach \dy/\mnamedy in {12/$-$,13/$0$,14/$+$,16/$+$} \path node at (\dy) {\mnamedy};
                \path node at ($(12)$){$+$} node at ($(13)$){$|$} node at ($(14)$){$+$} node at ($(15)$){$0$} node at ($(16)$){$-$} node at ($(17)$){$|$} node at ($(18)$){$-$} node at ($(22)$){$+$} node at ($(23)$){$0$} node at ($(24)$){$-$} node at ($(25)$){$|$} node at ($(26)$){$-$} node at ($(27)$){$0$} node at ($(28)$){$+$} node at ($(32)$){$+$} node at ($(33)$){$0$} node at ($(34)$){$-$} node at ($(35)$){$0$} node at ($(36)$){$+$} node at ($(37)$){$0$} node at ($(38)$){$-$};
                \draw[thick] (-.5,.5)rectangle([xshift=0.5cm,yshift=-0.5cm]\mnumline\mnumcol) ([xshift=-0.5cm,yshift=-0.5cm]00)--([xshift=0.5cm,yshift=-0.5cm]0\mnumcol) ([xshift=-0.5cm,yshift=-0.5cm]10)--([xshift=0.5cm,yshift=-0.5cm]1\mnumcol)

                ([xshift=-0.5cm,yshift=-0.5cm]20)--([xshift=0.5cm,yshift=-0.5cm]2\mnumcol)

                ([xshift=0.5cm,yshift=0.5cm]00)--([xshift=0.5cm,yshift=-0.5cm]\mnumline0); %Lệnh tự động kẻ bảng
            \end{tikzpicture}
        \end{center}
        Dựa vào bảng xét dấu ta xác định được hàm số đạt cực đại tại $x=\pm \sqrt{2}$.}
\end{ex}
\begin{ex}%[2D1K2-6]
    \immini{Cho hàm số $y=f(x)$ có đồ thị hàm $f'(x)=ax^2+bx+c$ như hình bên dưới. Hỏi hàm số $y=f(x-x^2)$ có bao nhiêu cực trị?
        \choice[2]
        {$0$}
        {\True $1$}
        {$2$}
        {$3$}}{
        \begin{tikzpicture}[>=stealth,x=1.2cm,y=0.7cm,font=\scriptsize]
            \begin{scope}[scale=0.35]
                \clip (-2,-2) rectangle (5,5.5);
                \def\a{1}
                \def\b{-3}
                \def\c{2}
                \draw[->] (-2,0) -- (4,0) node[below] { $x$};
                \draw[->] (0,-1) -- (0,5) node[left] {$y$};
                \draw (0,0)node[below left]{ $O$} circle(1.5pt);
                \draw (1,0) node[below]{$1$} (2,0) node[below]{  $2$} (0,2) node[left]{$2$};
                \pgfmathsetmacro\xdinh{-(\b)/2*(\a)}
                \pgfmathsetmacro\ydinh{(4*(\a)*(\c)-(\b)^2)/(4*(\a))}
                \draw[samples=150,smooth,domain=-5:5] plot(\x,{\a*(\x)^2+(\b)*\x+(\c)});
            \end{scope}
        \end{tikzpicture}
    }
    \loigiai{
        Đặt $g(x)=f\left(x-x^2\right)$\\
        Dựa vào đồ thị ta thấy $f'(x)=0$ có hai nghiệm $x_1=1,x_2=2$ nên $f'(x)$ có dạng $$f'(x)=k(x-1)(x-2)$$
        Khi đó $g'(x)=(1-2x)f'\left(x-x^2\right)=0$
        $$ \Leftrightarrow \hoac{&1-2x=0\\&f'\left(x-x^2\right)=0} \Leftrightarrow \hoac{&x=\dfrac{1}{2}\\&x-x^2=1\\&x-x^2=2} \Leftrightarrow \hoac{&x=\dfrac{1}{2}\\& \text{ vô nghiệm}\\&\text{ vô nghiệm.}}$$
        Bảng xét dấu
        \begin{center}
            \begin{tikzpicture}[every node/.style={circle,fill=white,inner sep=0pt},arrow/.style={>=stealth,->,shorten <= 0.3cm,shorten >= 0.3cm},font=\footnotesize,xscale=1,yscale=1]
                \def\mnumline{1} %Số dòng
                \def\mnumcol{5} %Số cột
                \foreach \j in {0,...,\mnumline}
                \foreach \i in {0,...,\mnumcol}{
                    \coordinate (\j\i) at (\i,-\j);
                }
                \pgfmathsetmacro\yline{\mnumline/2-1}
                \path node at (00){$x$} node at (10){$g'(x)$};
                \foreach \x/\mnamex in {01/$-\infty$,03/$\dfrac{1}{2}$,0\mnumcol/$+\infty$} \path node at (\x) {\mnamex};
                \foreach \dy/\mnamedy in {12/$+$,13/$0$,14/$-$} \path node at (\dy) {\mnamedy};
                \draw[thick] (-.5,.5)rectangle([xshift=0.5cm,yshift=-0.5cm]\mnumline\mnumcol) ([xshift=-0.5cm,yshift=-0.5cm]00)--([xshift=0.5cm,yshift=-0.5cm]0\mnumcol)  ([xshift=0.5cm,yshift=0.5cm]00)--([xshift=0.5cm,yshift=-0.5cm]\mnumline0);
            \end{tikzpicture}
        \end{center}
        Dựa vào bảng xét dấu, ta thấy $g(x)$ có 1 cực đại.
    }
\end{ex}
\begin{ex}%[2D1K2-2]
    \immini{Cho hàm số bậc bốn $y=f(x)$. Hàm số $y=f'(x)$
        có đồ thị như hình bên. Số điểm cực trị của hàm số $y=f\left(\sqrt{x^{2}+2 x+2}\right)$ là
        \choice[2]
        {$1$}
        {$2$}
        {$4$}
        {\True $3$}}
    {
        \begin{tikzpicture}[line join=round, line cap=round,>=stealth,font=\scriptsize]
            \begin{scope}[scale=0.5]
                \tikzset{label style/.style={font=\footnotesize}}
                \def \xmin{-2}
                \def \xmax{4.5}
                \def \ymin{-2}
                \def \ymax{3.5}
                \def \hamso{0.55*(\x)^3-1.76*(\x)^2-0.31*(\x)+2}
                \draw[->] (\xmin,0)--(\xmax,0) node[below left] {$x$};
                \draw[->] (0,\ymin)--(0,\ymax) node[below left] {$y$};
                \draw (0,0) node [below left] {$O$};
                \begin{scope}
                    \clip (\xmin+0.01,\ymin+0.01) rectangle (\xmax-0.01,\ymax-0.01);
                    \draw[samples=350,domain=-1.3:3.3,smooth,variable=\x] plot (\x,{\hamso});
                \end{scope}
                \draw (-1,0) node[below left]{$-1$} (1,0) node[below]{$1$} (3,0) node[below right]{$3$} (0,2) node[above left]{$2$};
            \end{scope}
        \end{tikzpicture}
    }
    \loigiai{
        $ y'=\dfrac{x+1}{\sqrt{x^2+2x+2}}f'(\sqrt{x^2+2x+2}) $.\\$ y'=0 \Leftrightarrow \hoac{&x=-1\\&f'(\sqrt{x^2+2x+2})=0} \Leftrightarrow \hoac{&x=-1\\&\sqrt{x^2+2x+2}=-1\\&\sqrt{x^2+2x+2}=1\\&\sqrt{x^2+2x+2}=3} \Leftrightarrow \hoac{&x=-1\\&x^2+2x+1=0\\&x^2+2x-7=0}\Leftrightarrow \hoac{&x=-1\\&x=-1 \text{ (nghiệm kép)}\\&x=-1\pm 2\sqrt{2}} $\\
        $ y'=0 $ có 3 nghiệm bội bậc lẻ nên hàm số có 3 điểm cực trị.
    }
\end{ex}
\begin{ex}%[2D1K2-2]
    \immini{Cho hàm số $ y=f(x) $ liên tục trên $ (a,b) $ và có đồ thị như hình bên. Số điểm cực trị của hàm số $ y=\left[f(x)\right]^2 $ trên $ (a;b) $ là
        \choice[2]
        {$4$}
        {$6$}
        {$2$}
        {\True $5$}}
    {
        \begin{tikzpicture}[line join=round, line cap=round,>=stealth,font=\scriptsize]
            \begin{scope}[scale=.35]
                \def \xmin{-3.5}
                \def \xmax{4.5}
                \def \ymin{-4}
                \def \ymax{3.5}
                \def \hamso{-0.37*(\x)^3+0.15*(\x)^2+2.41*(\x)-1}
                \draw[->] (\xmin,0)--(\xmax,0) node[below] {$x$};
                \draw[->] (0,\ymin)--(0,\ymax) node[left] {$y$};
                \draw (0,0) node [below left] {$O$};
                \clip (\xmin+0.01,\ymin+0.01) rectangle (\xmax-0.01,\ymax-0.01);
                \draw[samples=350,domain=-3:4,smooth,variable=\x] plot (\x,{\hamso});
                \draw[dashed] (-3,3.11)--(-3,0) node[below]{$a$} (3,-2.41)--(3,0) node[above]{$b$};
            \end{scope}
        \end{tikzpicture}
    }
    \loigiai{\immini{$ y=\left(f(x)\right)^2 $ nên $ y'=2f(x)f'(x) $.\\$ y'=0 \Leftrightarrow \hoac{&f(x)=0\\&f'(x)=0} \Leftrightarrow \hoac{&x=x_1,\ x=x_2,\ x=x_3\\&x=c,\ x=d}$.\\
            $ y'=0 $ có 5 nghiệm bội bậc lẻ thuộc $ (a,b) $ nên Số điểm cực trị của hàm số $ y=\left(f(x)\right)^2 $ trên $ (a;b) $ là 5.}
        {
            \begin{tikzpicture}[line join=round, line cap=round,>=stealth,thick,scale=0.8]
                \tikzset{label style/.style={font=\footnotesize}}
                \def \xmin{-3.5}
                \def \xmax{4.5}
                \def \ymin{-4}
                \def \ymax{3.5}
                \def \hamso{-0.37*(\x)^3+0.15*(\x)^2+2.41*(\x)-1}
                \draw[->] (\xmin,0)--(\xmax,0) node[below left] {$x$};
                \draw[->] (0,\ymin)--(0,\ymax) node[below left] {$y$};
                \draw (0,0) node [below left] {$O$};
                \begin{scope}
                    \clip (\xmin+0.01,\ymin+0.01) rectangle (\xmax-0.01,\ymax-0.01);
                    \draw[samples=350,domain=-3:4,smooth,variable=\x] plot (\x,{\hamso});
                \end{scope}
                \draw[dashed] (-3,3.11)--(-3,0) node[below]{\footnotesize $a$} (3,-2.41)--(3,0) node[above]{\footnotesize $b$} (2.54,0) node[below left]{\footnotesize $x_3$} (0.41,0) node[below right]{\footnotesize $x_2$} (-2.55,0) node[above right]{\footnotesize $x_1$} (-1.34,-3.07) -- (-1.34,0) node[above]{\footnotesize $c$} (1.61,1.72)--(1.61,0) node[below]{\footnotesize $d$};
            \end{tikzpicture}
        }
    }
\end{ex}
\begin{ex}%[2D1G2-1]
    \immini{Cho hàm số $y=f(x)$ có đạo hàm trên $\mathbb{R}$ và có bảng xét dấu $f'(x)$ như hình bên. Hàm số $y=f\left(x^{2}-2 x\right)$ có bao nhiêu điểm cực tiểu?
        \choice
        {\True $1$}
        {$2$}
        {$3$}
        {$4$}}{\begin{tikzpicture}
            \tkzTabInit[lgt=1,espcl=1.2]
            {$x$ /.7, $y'$ /.7}
            {$-\infty$,$-2$,$1$,$3$,$+\infty$}
            \tkzTabLine{ ,-,0,+,0,+,0,-, }
    \end{tikzpicture}}
    \loigiai{$ y'=(2x-2)f'(x^2-2x) $.
        \begin{eqnarray*}
            y'=0 	&\Leftrightarrow& \hoac{&x=1\\&f'(x^2-2x)=0}\\
            &\Leftrightarrow& \hoac{&x=1\\&x^2-2x=-2 \text{ (vô nghiệm)}\\&x^2-2x=1 \text{ (nghiệm bội bậc chẵn)}\\&x^2-2x=3} \\
            &\Leftrightarrow& \hoac{&x=1\\&x=1-\sqrt{2} \text{ (nghiệm bội bậc chẵn)}\\&x=1+\sqrt{2} \text{ (nghiệm bội bậc chẵn)}\\&x=3, \ x=-1.}
        \end{eqnarray*}
        $ y'=0 $ có 3 nghiệm bội bậc lẻ, khi đó $ y' $ đổi dấu qua các nghiệm này.\\
        $ y'=0 $ có 2 nghiệm bội bậc chẵn và $ y' $ sẽ không đổi dấu qua các nghiệm này.\\
        Tại $ x=4 $ thì $ y'(4)=(2\cdot 4 -2)f'(4^2-2\cdot 4)=6f'(8)<0 $.\\
        Bảng xét dấu
        \begin{center}
            \begin{tikzpicture}
                \tkzTabInit[lgt=1,espcl=1.2]
                {$x$ /1, $y'$ /1}
                {$-\infty$,$-1$,$1-\sqrt{2}$,$1+\sqrt{2}$,$3$,$+\infty$}
                \tkzTabLine{ ,-,0,+,0,+,0,+,0,-, }
            \end{tikzpicture}
        \end{center}
        Vậy hàm số có 1 điểm cực tiểu.
    }
\end{ex}
\begin{ex}%[2D1K2-6]
    \immini{Cho hàm số $f(x)$ có bảng biến thiên bên dưới. Trên khoảng $(-\sqrt{5};\sqrt{5})$ thì hàm số $y=f(x^2)$ đạt cực đại tại điểm nào sau đây?\choice
        {$x=\sqrt{2}$}
        {$x=-\sqrt{2}$}
        {\True $x=0$}
        {$x=2$}}{\begin{tikzpicture}
            \tkzTabInit[nocadre=false,lgt=1,espcl=1.6,deltacl=0.5]{$x$/.7 ,$f$/.7}
            {$-\infty$ , $0$ , $2$ , $+\infty$}
            \tkzTabLine{  , + , 0, - , 0 , +  }
    \end{tikzpicture}}
    \loigiai{Đặt $g(x)=f(x^2)$.\\
        Khi đó $g'(x)=2x \cdot f'(x^2)$.\\
        Cho $g'(x)=0 \Leftrightarrow 2x \cdot f'(x^2) =0 \Leftrightarrow
        \hoac{&x=0\\&f'(x^2)=0 \Leftrightarrow \hoac{x^2=0\\x^2=2} \Leftrightarrow \hoac{x=0\\x=\pm \sqrt{2}}}$\\
        Bảng xét dấu
        \begin{center}
            \begin{tikzpicture}[every node/.style={circle,fill=white,inner sep=0pt},arrow/.style={>=stealth,->,shorten <= 0.3cm,shorten >= 0.3cm},font=\footnotesize,xscale=1,yscale=.7]
                \def\mnumline{3} %Số dòng
                \def\mnumcol{9} %Số cột
                \foreach \j in {0,...,\mnumline}
                \foreach \i in {0,...,\mnumcol}{
                    \coordinate (\j\i) at (\i,-\j);
                }
                \pgfmathsetmacro\yline{\mnumline/2-1}
                \path node at (00){$x$} node at (10){$x$} node at (20){$f'(x^2)$} node at (30){$g'(x)$};
                \foreach \x/\mnamex in {01/$-\sqrt{5}$,03/$-\sqrt{2}$,05/$0$,07/$\sqrt{2}$,0\mnumcol/$\sqrt{5}$} \path node at (\x) {\mnamex};
                \foreach \dy/\mnamedy in {12/$-$,13/$0$,14/$+$,16/$+$} \path node at (\dy) {\mnamedy};
                \path node at ($(12)$){$-$} node at ($(13)$){$|$} node at ($(14)$){$-$} node at ($(15)$){$0$} node at ($(16)$){$+$} node at ($(17)$){$|$} node at ($(18)$){$+$} node at ($(22)$){$+$} node at ($(23)$){$0$} node at ($(24)$){$-$} node at ($(25)$){$0$} node at ($(26)$){$-$} node at ($(27)$){$0$} node at ($(28)$){$+$} node at ($(32)$){$-$} node at ($(33)$){$0$} node at ($(34)$){$+$} node at ($(35)$){$0$} node at ($(36)$){$-$} node at ($(37)$){$0$} node at ($(38)$){$+$};
                \draw[thick] (-.5,.5)rectangle([xshift=0.5cm,yshift=-0.5cm]\mnumline\mnumcol) ([xshift=-0.5cm,yshift=-0.5cm]00)--([xshift=0.5cm,yshift=-0.5cm]0\mnumcol) ([xshift=-0.5cm,yshift=-0.5cm]10)--([xshift=0.5cm,yshift=-0.5cm]1\mnumcol)

                ([xshift=-0.5cm,yshift=-0.5cm]20)--([xshift=0.5cm,yshift=-0.5cm]2\mnumcol)

                ([xshift=0.5cm,yshift=0.5cm]00)--([xshift=0.5cm,yshift=-0.5cm]\mnumline0); %Lệnh tự động kẻ bảng
            \end{tikzpicture}
        \end{center}
        Dựa vào bảng xét dấu ta xác định được hàm số đạt cực đại tại $x=0$.
    }
\end{ex}
\begin{ex}%[2D1K2-6]
    \immini{Cho hàm số $f(x)$ có bảng biến thiên bên dưới. Hàm số $y=f(x^2-2)$ đạt cực đại tại điểm nào sau đây?
        \choice
        {$x=-2$}
        {$x=-1$}
        {\True $x=0$}
        {$x=2$}}{\begin{tikzpicture}
            \tkzTabInit[nocadre=false,lgt=1,espcl=1.6,deltacl=0.5]{$x$/.7 ,$f$/.7}
            {$-\infty$ , $-1$ , $2$ , $+\infty$}
            \tkzTabLine{  , - , 0, - , 0 , +  }
    \end{tikzpicture}}
    \loigiai{Đặt $g(x)=f(x^2-2)$\\
        Khi đó $g'(x)=2x \cdot f'(x^2-2)$\\
        Cho $g'(x)=0 \Leftrightarrow 2x \cdot f'(x^2-2) =0$
        $$ \Leftrightarrow \hoac{&x=0\\&f'(x^2-2)=0 \Leftrightarrow \hoac{&x^2-2=-1\\&x^2-2=2} \Leftrightarrow \hoac{x^2=1\\x^2=4} \Leftrightarrow \hoac{x=\pm 1\\x=\pm 2}}$$
        Bảng xét dấu
        \begin{center}
            \begin{tikzpicture}[every node/.style={circle,fill=white,inner sep=0pt},arrow/.style={>=stealth,->,shorten <= 0.3cm,shorten >= 0.3cm},font=\footnotesize,xscale=1,yscale=1]
                \def\mnumline{3} %Số dòng
                \def\mnumcol{14} %Số cột
                \foreach \j in {0,...,\mnumline}
                \foreach \i in {0,...,\mnumcol}{
                    \coordinate (\j\i) at (\i,-\j);
                }
                \pgfmathsetmacro\yline{\mnumline/2-1}
                \path node at ([xshift=0.5cm]00){$x$} node at ([xshift=0.5cm]10){$x$}  node at ([xshift=0.5cm]20){$f'\left(x^2-2\right)$} node at ([xshift=0.5cm]\mnumline0){$g'(x)$};
                \foreach \x/\mnamex in {02/$-\infty$,04/$-2$,06/$-1$,08/$0$,010/$1$,012/$2$,0\mnumcol/$+\infty$} \path node at (\x) {\mnamex};
                \foreach \dy/\mnamedy in {13/$-$,14/$0$,15/$+$,16/$+$} \path node at (\dy) {\mnamedy};
                \path node at ($(13)$){$-$} node at ($(14)$){$|$} node at ($(15)$){$-$} node at ($(16)$){$|$} node at ($(17)$){$-$} node at ($(18)$){$0$} node at ($(19)$){$+$} node at ($(110)$){$|$} node at ($(111)$){$+$} node at ($(112)$){$|$} node at ($(113)$){$+$}
                node at ($(23)$){$+$} node at ($(24)$){$0$} node at ($(25)$){$-$} node at ($(26)$){$0$} node at ($(27)$){$-$} node at ($(28)$){$|$} node at ($(29)$){$-$} node at ($(210)$){$0$} node at ($(211)$){$-$} node at ($(212)$){$0$} node at ($(213)$){$+$}
                node at ($(33)$){$-$} node at ($(34)$){$0$} node at ($(35)$){$+$} node at ($(36)$){$0$} node at ($(37)$){$+$} node at ($(38)$){$0$} node at ($(39)$){$-$} node at ($(310)$){$0$} node at ($(311)$){$-$} node at ($(312)$){$0$} node at ($(313)$){$+$};
                \draw[thick] (-.5,.5)rectangle([xshift=0.5cm,yshift=-0.5cm]\mnumline\mnumcol) ([xshift=-0.5cm,yshift=-0.5cm]00)--([xshift=0.5cm,yshift=-0.5cm]0\mnumcol)
                ([xshift=-0.5cm,yshift=-0.5cm]20)--([xshift=0.5cm,yshift=-0.5cm]2\mnumcol)
                ([xshift=-0.5cm,yshift=-0.5cm]10)--([xshift=0.5cm,yshift=-0.5cm]1\mnumcol) ([xshift=0.5cm,yshift=0.5cm]01)--([xshift=0.5cm,yshift=-0.5cm]\mnumline1); %Lệnh tự động kẻ bảng
            \end{tikzpicture}
        \end{center}
        Dựa vào bảng xét dấu ta xác định được hàm số đạt cực đại tại $x=0$.
    }
\end{ex}
\begin{ex}%[2D1K2-1]
    Cho hàm số $ y=f(x) $ có đạo hàm $ f'(x)=x^2(x-1)(x-4)^2 $. Khi đó hàm số $ y=f(x^2) $ có bao nhiêu điểm cực trị?
    \choice
    {$4$}
    {\True $3$}
    {$5$}
    {$2$}
    \loigiai{$ f'(x)=0 \Leftrightarrow x=1 $ (nghiệm đơn), $ x=0 $ (nghiệm kép), $ x=4 $ (nghiệm kép).\\
        $ y=f(x^2) $ thì $ y'=2xf'(x^2) $.\\$y'=0 \Leftrightarrow \hoac{&x=0\\&x^2=1\\&x^2=0 \text{ (nghiệm kép)}\\&x^2=4 \text{ (nghiệm kép)}} \Leftrightarrow \hoac{&x=0\\&x=\pm 1\\&x=0 \text{ (nghiệm bội chẵn)}\\&x=\pm 2 \text{ (nghiệm bội chẵn).}} $\\
        Vậy hàm số có 3 điểm cực trị.
    }
\end{ex}
\begin{ex}%[2D1K2-6]
    Cho hàm $f(x)$ có đạo hàm $f'(x)=x^2-2x,\forall x\in \mathbb{R}$. Hàm số $y=f\left(1-\dfrac{1}{2}x\right)+4x$ có bao nhiêu điểm cực trị?
    \choice
    {0}
    {1}
    {\True 2}
    {3}
    \loigiai{Ta có $y'=-\dfrac{1}{2}f'\left(1-\dfrac{1}{2}x\right)+4$\\
        $y'=0 \Leftrightarrow
        f'\left(1-\dfrac{1}{2}x\right)=8\Leftrightarrow \left(1-\dfrac{1}{2}x\right)^2-2\left(1-\dfrac{1}{2}x\right)=8 \Leftrightarrow \dfrac{1}{4}x^2-9=0
        \Leftrightarrow \hoac{&x=-6\\&x=6}$\\
        Bảng xét dấu
        \begin{center}
            \begin{tikzpicture}[every node/.style={circle,fill=white,inner sep=0pt},arrow/.style={>=stealth,->,shorten <= 0.3cm,shorten >= 0.3cm},font=\footnotesize,xscale=1,yscale=1]
                \def\mnumline{1} %Số dòng
                \def\mnumcol{7} %Số cột
                \foreach \j in {0,...,\mnumline}
                \foreach \i in {0,...,\mnumcol}{
                }
                \pgfmathsetmacro\yline{\mnumline/2-1}
                \path node at (00){$x$} node at (10){$y'$};
                \foreach \x/\mnamex in {01/$-\infty$,03/$-6$,05/$6$,0\mnumcol/$+\infty$} \path node at (\x) {\mnamex};
                \foreach \dy/\mnamedy in {12/$+$,13/$0$,14/$-$,15/$0$,16/$+$} \path node at (\dy) {\mnamedy};
                \draw[thick] (-.5,.5)rectangle([xshift=0.5cm,yshift=-0.5cm]\mnumline\mnumcol) ([xshift=-0.5cm,yshift=-0.5cm]00)--([xshift=0.5cm,yshift=-0.5cm]0\mnumcol)  ([xshift=0.5cm,yshift=0.5cm]00)--([xshift=0.5cm,yshift=-0.5cm]\mnumline0); %Lệnh tự động kẻ bảng
            \end{tikzpicture}
        \end{center}
        Vậy hàm số $y=f\left(1-\dfrac{1}{2}x\right)+4x$ có 2 điểm cực trị.}
\end{ex}
\begin{ex}%[2D1G2-1]
    Cho hàm số $ y=f(x) $ có đạo hàm $ f'(x)=(x-1)^2(x^2-2x) $, với mọi $ x \in \mathbb{R} $. Có bao nhiêu giá trị nguyên dương của tham số $m$ để hàm số $ y=f(x^2-8x+m) $ có 5 điểm cực trị?
    \choice
    {\True $15$}
    {$16$}
    {$17$}
    {$18$}
    \loigiai{$ f'(x)=0 \Leftrightarrow x=1 $ (nghiệm kép), $ x=0 $ (nghiệm đơn), $ x=2 $ (nghiệm đơn).\\
        $ y=f(x^2-8x+m) $ thì $ y'=(2x-8)f'(x^2-8x+m) $.\\$y'=0 \Leftrightarrow \hoac{&x=4\\&x^2-8x+m=1 \text{ (nghiệm kép)}\\&x^2-8x+m=0 \quad (1)\\&x^2-8x+m=2 \quad (2)} $.\\
        Hàm số có 5 điểm cực trị $ \Leftrightarrow (1) $ có 2 nghiệm phân biệt khác 4 và $ (2) $ có 2 nghiệm phân biệt khác 4.\\$(1) $ có 2 nghiệm phân biệt khác 4 $ \Leftrightarrow \heva{&16-32+m \ne 0\\&\Delta'=16-m>0} \Leftrightarrow \heva{&m \ne 16\\&m<16}\Leftrightarrow m<16$.\\
        $(2) $ có 2 nghiệm phân biệt khác 4 $ \Leftrightarrow \heva{&16-32+m \ne 2\\&\Delta'=16-m+2>0} \Leftrightarrow \heva{&m \ne 18\\&m<18} \Leftrightarrow m<18$.\\
        Vậy ta có $ m<16 $ mà $ m $ nguyên dương nên $ m \in \{1,2,\cdots,15\} $ (15 số $ m $ thỏa mãn).
    }
\end{ex}
\begin{ex}%[2D1Y2-2]
    \immini
    {
        Cho hàm số $y=f(x)$ có đạo hàm liên tục trên $\mathbb{R}$. Đồ thị hàm số $y=f'(x)$ như hình vẽ bên. Số điểm cực trị của hàm số $y=f(x)-5x$ là
        \choice
        {$2$}
        {$3$}
        {$4$}
        {\True $1$}
    }
    {
        \begin{tikzpicture}[font=\scriptsize, line join=round, line cap=round, >=stealth,y=.8cm]
            \begin{scope}[scale=.6]
                \draw[->,>=latex](-3,0)--(3,0)node[above]{$x$};
                \draw[->,>=latex](0,-1)--(0,5)node[right]{$y$};
                \node[above left] at (0,0){$O$};
                \draw plot [samples=100,domain=-2.1:2.1] (\x,{(\x)^3-3*(\x)+2});
                \foreach\i in{-1,1}{\node[below] at (\i,0){$\i$};}
                \foreach\i in{4,2}{\node[right] at (0,\i){$\i$};}
                \draw[dashed](-1,0)--(-1,4)--(0,4);
            \end{scope}
        \end{tikzpicture}
    }
    \loigiai
    {
        Gọi $g(x)=f(x)-5x$. Ta có đạo hàm $g'(x)=f'(x)-5$. Bảng biến thiên của $g'(x)$ như hình dưới.
        \begin{center}
            \begin{tikzpicture}
                \tkzTabInit[nocadre=false,lgt=1.2,espcl=2.5,deltacl=0.6]
                {$x$/1, $f'(x)$/2, $g'(x)$/2}
                {$-\infty$, $-1$, $1$, $+\infty$}
                \tkzTabVar{-/ $-\infty$, +/$4$, -/$0$, +/$+\infty$}
                \tkzTabVar{-/ $-\infty$, +/$-1$, -/$-5$, +/$+\infty$}
            \end{tikzpicture}
        \end{center}
        Ta thấy $g'(x)$ chỉ đổi dấu một lần từ âm sang dương.\\
        Vì vậy hàm số $y=f(x)-5x$ có một điểm cực trị.
    }
\end{ex}
\begin{ex}%[2D1Y2-2]
    \immini
    {
        Cho hàm số $y=f(x)$ có đạo hàm trên $\mathbb{R}$. Biết hàm số $y=f'(x)$ có đồ thị như hình vẽ. Khẳng định nào sau đây đúng về cực trị của hàm số $g(x)=f(x)+x$?
        \choice
        {Hàm số có một điểm cực đại và một điểm cực tiểu}
        {Hàm số không có điểm cực đại và một điểm cực tiểu}
        {Hàm số có một điểm cực đại và hai điểm cực tiểu}
        {\True Hàm số có hai điểm cực đại và một điểm cực tiểu}
    }
    {
        \begin{tikzpicture}[ font=\scriptsize, line join=round, line cap=round, >=stealth]
            \begin{scope}[scale=.5]
                \foreach\x in{-1,0,...,3}{\draw[color=gray!30](\x,-3)--(\x,3.3);}
                \foreach\y in{-2,-1,...,3}{\draw[color=gray!30](-2,\y)--(4,\y);}
                \draw[->,>=latex](-2,0)--(4,0)node[above]{$x$};
                \draw[->,>=latex](0,-3)--(0,3.3)node[right]{$y$};
                \node[above left] at (0,0){$O$};
                \draw plot [samples=100,domain=-1.12:3.1] (\x,{-(\x)^3+3*(\x)^2-2});
            \end{scope}
        \end{tikzpicture}
    }
    \loigiai
    {
        Ta có $g'(x)=f'(x)+1$. Bảng biến thiên của $g'(x)$ như hình dưới.
        \begin{center}
            \begin{tikzpicture}
                \tkzTabInit[nocadre=false,lgt=1.2,espcl=2.5,deltacl=0.6]
                {$x$/1, $f'(x)$/2, $g'(x)$/2}
                {$-\infty$, $0$, $2$, $+\infty$}
                \tkzTabVar{+/ $+\infty$, -/$-2$, +/$2$, -/$-\infty$}
                \tkzTabVar{+/ $+\infty$, -/$-1$, +/$3$, -/$-\infty$}
            \end{tikzpicture}
        \end{center}
        Dựa vào bảng biến thiên của $g'(x)$, ta thấy đạo hàm đổi dấu từ dương sang âm hai lần, từ âm sang dương một lần.\\
        Do đó hàm số $g(x)$ có hai điểm cực đại và một điểm cực tiểu.
    }
\end{ex}
\begin{ex}%[2D1K2-6]
    \immini{	Cho hàm số $y=f(x)$ có đạo hàm trên $\mathbb{R}$ và có đồ thị hàm số $f'(x)$ như hình vẽ. Hàm số $y=2f(x)+x^2$ đạt cực đại tại điểm nào sau đây ?
        \choice[2]
        {\True $x=-1$}
        {$x=0$}
        {$x=1$}
        {$x=2$}}{\begin{tikzpicture}[>=stealth,font=\scriptsize,x=1.3cm]
            \begin{scope}[scale=.7]
                \draw[->] (-2,0) -- (3,0) node[below] {\scriptsize $x$};
                \draw[->] (0,-3) -- (0,2.5) node[left] { $y$};
                \draw (0,0)node[below left]{$O$} (-1.2,0) node[below]{ $-1$} (0,-2) node[below right]{ $-2$} (0,1) node[above left]{$1$} (0,-1) node[below right]{  $-1$} (1,0) node[above]{$1$} (2,0) node[above]{ $2$};
                \draw plot[smooth,tension=.65] coordinates{(-1.05,1.7) (-0.5,-2.6) (0.17,0.5) (0.9,-0.9) (1.5,-1.1) (2.1,-1.9) (2.3,2)};
                \draw[dashed] (-1,0) -- (-1,1) -- (0,1) (1,0)--(1,-1)--(0,-1) (2,0)--(2,-2)--(0,-2);
            \end{scope}
    \end{tikzpicture}}
    \loigiai{
        Đặt $g(x)=2f(x)+x^2$\\
        Khi đó $g'(x)=2f'(x)+2x=0 \Leftrightarrow 2\left(f'(x)+x\right)=0 \Leftrightarrow f'(x)=-x \quad (*)$\\
        Số nghiệm của phương trình $(*)$ là số giao điểm của đồ thị hàm số $y=f'(x)$ và $y=-x$\\
        Dựa vào hình bên ta thấy có $4$ giao điểm lần lượt có tọa độ là $(-1;1),(0;0),(1;-1)$ và $(2;-2)$ \\ $ \Rightarrow (*)  \Leftrightarrow \hoac{&x=-1 \quad \text{(đơn)}\\&x=0 \quad \text{(đơn)}\\&x=1 \quad \text{(kép)}\\&x=2 \quad \text{(kép)}.}$\\
        Bảng xét dấu
        \begin{center}
            \begin{tikzpicture}[every node/.style={circle,fill=white,inner sep=0pt},arrow/.style={>=stealth,->,shorten <= 0.3cm,shorten >= 0.3cm},font=\footnotesize,xscale=1,yscale=1]
                \def\mnumline{1} %Số dòng
                \def\mnumcol{11} %Số cột
                \foreach \j in {0,...,\mnumline}
                \foreach \i in {0,...,\mnumcol}{
                    \coordinate (\j\i) at (\i,-\j);
                }
                \pgfmathsetmacro\yline{\mnumline/2-1}
                \path node at (00){$x$} node at (10){$g'(x)$};
                \foreach \x/\mnamex in {01/$-\infty$,03/$-1$,05/$0$,07/$1$,09/$2$,0\mnumcol/$+\infty$} \path node at (\x) {\mnamex};
                \foreach \dy/\mnamedy in {12/$+$,13/$0$,14/$-$,15/$0$,16/$+$,17/$0$,18/$+$,19/$0$,110/$+$} \path node at (\dy) {\mnamedy};
                \draw[thick] (-.5,.5)rectangle([xshift=0.5cm,yshift=-0.5cm]\mnumline\mnumcol) ([xshift=-0.5cm,yshift=-0.5cm]00)--([xshift=0.5cm,yshift=-0.5cm]0\mnumcol)  ([xshift=0.5cm,yshift=0.5cm]00)--([xshift=0.5cm,yshift=-0.5cm]\mnumline0);
            \end{tikzpicture}
        \end{center}
        Dựa vào bảng xét dấu, ta thấy $g(x)$ đạt cực đại tại $x=-1$.
    }
\end{ex}
\begin{ex}%[2D1K2-6]
    \immini{Hàm số $y=f(x)$ liên tục trên $\mathbb{R}$ và có đồ thị hàm số $f'(x)$ như hình vẽ bên dưới. Hàm số $y=f(x)-\dfrac{1}{3}x^3+x^2-x+2$ đạt cực đại tại điểm nào sau đây ?
        \choice[2]
        {\True $x=1$}
        {$x=-1$}
        {$x=0$}
        {$x=2$}}{\begin{tikzpicture}[>=stealth,font=\scriptsize,y=.7cm]
            \begin{scope}[scale=.8]
                \draw[->] (-2,0) -- (3,0) node[below] {\scriptsize $x$};
                \draw[->] (0,-3) -- (0,2.5) node[left] {\scriptsize $y$};
                \draw (0,0)node[below left]{ $O$}  (-1.2,0) node[below left]{ $-1$} (0,-2) node[right]{ $-2$} (0,1) node[above left]{  $1$} (1,0) node[below]{ $1$} (2,0) node[below]{ $2$};
                \draw plot [samples=100,domain=-1.1:2.2] (\x,{(\x)^3-2*(\x)^2+1});
                \draw[dashed] (-1,0) -- (-1,-2) -- (0,-2) (0,1)--(2,1)--(2,0);
            \end{scope}
    \end{tikzpicture}}
    \loigiai{
        Đặt $g(x)=f(x)-\dfrac{1}{3}x^3+x^2-x+2$	\\
        Khi đó $g'(x)=f'(x)-x^2+2x-1$.\\
        $g'(x)=0 \Leftrightarrow f'(x)=x^2-2x+1 \quad (*)$
        \immini{Số nghiệm của $(*)$ cũng chính là số giao điểm của đồ thị hàm số $y=f'(x)$ với $y=x^2-2x+1$\\
            Dựa vào hình vẽ bên, ta thấy có $3$ giao điểm lần lượt có tọa độ là $(1;0),(2;1),(0;1)$. Khi đó,
            $(*) \Leftrightarrow \hoac{&x=1\\&x=0\\&x=2.}$
        }
        {\begin{tikzpicture}[>=stealth,x=1.0cm,y=1.0cm,scale=0.6]
                \draw[->] (-2,0) -- (3,0) node[below] {\scriptsize $x$};
                \draw[->] (0,-3) -- (0,2.5) node[left] {\scriptsize $y$};
                \draw (0,0)node[below right]{\scriptsize $O$} circle(1.5pt) (-1.2,0) node[below]{\scriptsize $-1$} (0,-2) node[right]{\scriptsize  $-2$} (0,1) node[left]{\scriptsize  $1$} (1,0) node[below]{\scriptsize  $1$} (2,0) node[below]{\scriptsize  $2$};
                \def\a{1}
                \def\b{-2}
                \def\c{1}
                \pgfmathsetmacro\xdinh{-(\b)/2*(\a)}
                \pgfmathsetmacro\ydinh{(4*(\a)*(\c)-(\b)^2)/(4*(\a))}
                \fill[dashed] (\xdinh,\ydinh)circle(2pt) edge (\xdinh,0) edge (0,\ydinh);
                \clip (-2,-3)rectangle(3,3);
                \draw[thick,samples=150,smooth,domain=-5:5] plot(\x,{\a*(\x)^2+(\b)*\x+(\c)});
                \draw[thick] plot[smooth,tension=.65] coordinates{(-1.1,-2.2) (-0.1,1) (1.4,-0.2) (2.5,2.8)};
                \draw[dashed] (-1,0) -- (-1,-2) -- (0,-2) (0,1)--(2,1)--(2,0);
        \end{tikzpicture}}
        \noindent
        Bảng xét dấu
        \begin{center}
            \begin{tikzpicture}[every node/.style={circle,fill=white,inner sep=0pt},arrow/.style={>=stealth,->,shorten <= 0.3cm,shorten >= 0.3cm},font=\footnotesize,xscale=1,yscale=1]
                \def\mnumline{1} %Số dòng
                \def\mnumcol{9} %Số cột
                \foreach \j in {0,...,\mnumline}
                \foreach \i in {0,...,\mnumcol}{
                    \coordinate (\j\i) at (\i,-\j);
                }
                \pgfmathsetmacro\yline{\mnumline/2-1}
                \path node at (00){$x$} node at (10){$g'(x)$};
                \foreach \x/\mnamex in {01/$-\infty$,03/$0$,05/$1$,07/$2$,0\mnumcol/$+\infty$} \path node at (\x) {\mnamex};
                \foreach \dy/\mnamedy in {12/$-$,13/$0$,14/$+$,15/$0$,16/$-$,17/$0$,18/$+$} \path node at (\dy) {\mnamedy};
                \draw[thick] (-.5,.5)rectangle([xshift=0.5cm,yshift=-0.5cm]\mnumline\mnumcol) ([xshift=-0.5cm,yshift=-0.5cm]00)--([xshift=0.5cm,yshift=-0.5cm]0\mnumcol)  ([xshift=0.5cm,yshift=0.5cm]00)--([xshift=0.5cm,yshift=-0.5cm]\mnumline0);
            \end{tikzpicture}
        \end{center}
        Hàm số đạt cực đại tại $x=1$.
    }
\end{ex}
\begin{ex}%[2D1G2-6]
    \immini{	Cho hàm số $f(x)$ có đạo hàm liên tục trên $\mathbb{R}$ và đồ thị $y=f'(x)$ như hình vẽ dưới đây. Xét trên khoảng $(-\pi;2\pi)$, số điểm cực trị của hàm số $g(x)=f(2\cos x)+2\cos2x$ là
        \choice[2]
        {$13$}
        {$10$}
        {\True $11$}
        {$9$}}{\begin{tikzpicture}[>=stealth,font=\scriptsize,x=1.3cm]
            \begin{scope}[scale=.5]
                \draw[->] (-2.5,0) -- (2.5,0) node[below] { $x$};
                \draw[->] (0,-2.5) -- (0,2.5) node[left] {$y$};
                \draw (0,0)node[below left]{ $O$};
                \draw (0,-2)node[below right]{$-2$} (0,2)node[above right]{\scriptsize $2$} (1,0) node[above]{ $1$} (-2,0) node[above]{ $-2$} (-1,0) node[below]{ $-1$} (2,0) node[below]{$2$};
                \draw[dashed] (-2,0)--(-2,-2)--(1,-2)--(1,0) (-1,0)--(-1,2)--(2,2)--(2,0);
                \clip (-2.5,-2.5)rectangle(2.5,3);
                \draw[samples=150,smooth,domain=-2.1:2.1] plot(\x,{(\x)^3-3*\x});

                \fill[black] (-2,0) circle(1.5pt) (-1,0) circle(1.5pt) (1,0) circle(1.5pt) (2,0) circle(1.5pt)(-2,-2) circle(1.5pt)(0,-2) circle(1.5pt)(1,-2) circle(1.5pt)(-1,2) circle(1.5pt)(0,2) circle(1.5pt)(2,2) circle(1.5pt);
            \end{scope}
    \end{tikzpicture}}
    \loigiai{
        Ta có $g'(x)=f'(2\cos x)\cdot(-2\sin x)-2\sin{2x}\cdot2=-2\sin{x}\left[f'(2\cos x)+4\cos x\right]$.\\
        Suy ra $g'(x)=0 \Leftrightarrow \hoac{&\sin x=0\\&f'(2\cos x)=-4\cos x.}$\\
        \begin{itemize}
            \item $\sin x=0 \Leftrightarrow x\in\{0;\pi\}$ vì $x\in(-\pi;2\pi)$.
            \item $f'(2\cos x)=-4\cos x$.\\
            Đặt $t=2\cos x$, vì $x\in(-\pi;2\pi)$ nên $t\in(-1;1)$.\\
            Phương trình trở thành $f'(t)=-2t$. Nghiệm của phương trình này là hoành độ giao điểm của đồ thị hàm số $y=f'(t)$ và đường thẳng $y=-2t$.\\
            \begin{center}
                \begin{tikzpicture}[>=stealth,x=1cm,y=1cm,scale=1]
                    \draw[->] (-2.5,0) -- (2.5,0) node[below] {\scriptsize $t$};
                    \draw[->] (0,-2.5) -- (0,2.5) node[left] {\scriptsize $y$};
                    \draw (0,0)node[below left]{\scriptsize $O$};
                    \draw (0,-2)node[below right]{\scriptsize $-2$} (0,2)node[above right]{\scriptsize $2$} (1,0) node[above]{\scriptsize $1$} (-2,0) node[above]{\scriptsize $-2$} (-1,0) node[below]{\scriptsize $-1$} (2,0) node[below]{\scriptsize $2$};
                    \draw[dashed] (-2,0)--(-2,-2)--(1,-2)--(1,0) (-1,0)--(-1,2)--(2,2)--(2,0);
                    \clip (-2.5,-2.5)rectangle(2.5,2.5);
                    \draw[thick,samples=150,smooth,domain=-2.1:2.1] plot(\x,{(\x)^3-3*\x}) node[right]{$(l)$};
                    \node[above left] at (2,2){\scriptsize $y=f'(t)$};
                    \fill[black] (-2,0) circle(1.5pt) (-1,0) circle(1.5pt) (1,0) circle(1.5pt) (2,0) circle(1.5pt)(-2,-2) circle(1.5pt)(0,-2) circle(1.5pt)(1,-2) circle(1.5pt)(-1,2) circle(1.5pt)(0,2) circle(1.5pt)(2,2) circle(1.5pt);
                    \draw[thick,samples=150,smooth,domain=-2.1:2.1] plot(\x,{-2*(\x)});
                \end{tikzpicture}
            \end{center}
            Suy ra $f'(t)=-2t \Leftrightarrow \hoac{&t=-1\\&t=0\\&t=1.}$

            \begin{itemize}
                \item Với $t=-1 \Rightarrow 2\cos x=-1 \Leftrightarrow \cos x=-\dfrac{1}{2} \Leftrightarrow x\in\left\{-\dfrac{2\pi}{3};\dfrac{2\pi}{3};\dfrac{4\pi}{3}
                \right\}$ vì $x\in(-\pi;2\pi)$.
                \item Với $t=0 \Rightarrow \cos x=0 \Leftrightarrow x\in\left\{-\dfrac{\pi}{2};\dfrac{\pi}{2};\dfrac{3\pi}{2}
                \right\}$ vì $x\in(-\pi;2\pi)$.
                \item Với $t=1 \Rightarrow 2\cos x=1 \Leftrightarrow \cos x=\dfrac{1}{2} \Leftrightarrow x\in\left\{-\dfrac{\pi}{3};\dfrac{\pi}{3};\dfrac{5\pi}{3}
                \right\}$ vì $x\in(-\pi;2\pi)$.
            \end{itemize}
        \end{itemize}
        Và
        \begin{itemize}
            \item $f'(t)+2t>0\Leftrightarrow f'(t)>-2t\Leftrightarrow \hoac{&-1<t<0\\&t>1}\\
            \Rightarrow \hoac{&-\dfrac{1}{2}<\cos x<0\\&\cos x>\dfrac{1}{2}}\Leftrightarrow \hoac{&-\dfrac{2\pi}{3}<x<-\dfrac{\pi}{3}\\&\dfrac{4\pi}{3}<x<\dfrac{5\pi}{3}\\&\dfrac{\pi}{3}<x<\dfrac{2\pi}{3}}$ (vì $x\in(-\pi;2\pi)$).
            \item $f'(t)+2t<0\Leftrightarrow f'(t)<-2t\Leftrightarrow \hoac{&t<-1\\&0<t<1}\\
            \Rightarrow \hoac{&\cos x<-\dfrac{1}{2}\\&0<\cos x<\dfrac{1}{2}} \Leftrightarrow \hoac{&-\pi<x<-\dfrac{2\pi}{3}\\&-\dfrac{\pi}{3}<x<\dfrac{\pi}{3}\\&\dfrac{2\pi}{3}<x<\dfrac{4\pi}{3}}$ (vì $x\in(-\pi;2\pi)$).
        \end{itemize}
        Bảng biến thiên hàm số $y=g(x)$
        \begin{center}
            \begin{tikzpicture}
                \tkzTabInit[nocadre=false,lgt=4,espcl=1]
                {$x$ /1.1,$-2\sin x$ /0.7,$f'(2\cos x)+4\cos x$ /0.7,$g'(x)$ /0.7,$g(x)$ /2}
                {$-\pi$,$-\dfrac{2\pi}{3}$,$-\dfrac{\pi}{2}$,$-\dfrac{\pi}{3}$,$0$,$\dfrac{\pi}{3}$,$\dfrac{\pi}{2}$,$\dfrac{2\pi}{3}$,$\pi$,$\dfrac{4\pi}{3}$,$\dfrac{3\pi}{2}$,$\dfrac{5\pi}{3}$,$2\pi$}
                \tkzTabLine{,+,|,+,|,+,|,+,$0$,-,|,-,|,-,|,-,$0$,+,|,+,|,+,|,+,}
                \tkzTabLine{,-,$0$,+,$0$,-,$0$,+,|,+,$0$,-,$0$,+,$0$,-,|,-,$0$,+,$0$,-,$0$,+,}
                \tkzTabLine{,-,$0$,+,$0$,-,$0$,+,|,-,$0$,+,$0$,-,$0$,+,|,-,$0$,+,$0$,-,$0$,+,}
                \tkzTabVar{+/,-/,+/,-/,+/,-/,+/,-/,+/,-/,+/,-/,+/,}
            \end{tikzpicture}
        \end{center}
        Từ bảng biến thiên ta suy ra hàm số $y=g(x)$ có $11$ điểm cực trị trên khoảng $(-\pi;2\pi)$.
    }
\end{ex}
\begin{ex}%[2D1G2-6]
    \immini{	Cho hàm số $y=f(x)$ có đồ thị của $y=f'(x)$ có đồ thị như hình vẽ bên dưới. Hàm số $g(x)=f(x^3-3x)-x^3+3x$ có bao nhiêu điểm cực tiểu? \choice[2]
        {$2$}
        {$4$}
        {$3$}
        {\True $5$}}{\begin{tikzpicture}[>=stealth,font=\scriptsize]
            \begin{scope}[scale=.5]
                \draw[->,line width = 1pt] (-2,0)--(0,0) node[below left]{$O$}--(5,0) node[below]{$x$};
                \draw[->,line width = 1pt] (0,-2) --(0,3) node[right]{$y$};
                \draw (-1,0) node[below left]{$-1$} circle (1pt);
                \draw (0,2) node[above right]{$2$} circle (1pt);
                \draw (2,0) node[below left]{$2$} circle (1pt);
                \draw (4,0) node[below right]{$4$} circle (1pt);
                \draw [ domain=-1.3:4.6, samples=100] %
                plot (\x, {0.25*(\x)^3-1.25*(\x)^2+0.5*(\x)+2});
            \end{scope}
    \end{tikzpicture}}
    \loigiai{
        $g'(x)=f'(x^3-3x)\cdot (3x^2-3)-3x^2+3=3(x^2-1)\left[f'(x^3-3x)-1\right].\\
        \Rightarrow g'(x)=0\Leftrightarrow \hoac{&x^2=1\\&f'(x^2-3x)=1} \Leftrightarrow \hoac{&x=\pm1\\&x^3-3x=a\quad (-1<a<0)\\&x^3-3x=b\quad (0<b<2)\\&x^3-3x=c\quad (c>4).}$\\
        \begin{center}
            \begin{tikzpicture}[>=stealth]
                \draw[->,line width = 1pt] (-2,0)--(0,0) node[below left]{$O$}--(5,0) node[below]{$x$};
                \draw[->,line width = 1pt] (0,-2) --(0,3) node[right]{$y$};
                \draw (-1,0) node[below left]{$-1$} circle (1pt);
                \draw (0,2) node[above right]{$2$} circle (1pt);
                \draw (2,0) node[below left]{$2$} circle (1pt);
                \draw (4,0) node[below right]{$4$} circle (1pt);
                \draw [thick, domain=-1.3:4.6, samples=100] %
                plot (\x, {0.25*(\x)^3-1.25*(\x)^2+0.5*(\x)+2});
                \draw [thick, domain=-2:5, samples=100] %
                plot (\x, {0*(\x)+1});
                \draw (-1,1) node[above left]{$y=1$};
                \draw (1,1.8) node[right]{$y=f'(x)$};
                \draw (4,0) node[below right]{$4$} circle(1pt);
                \draw (4.323404276086477,1) node[below right] {$c$} circle(1pt);
                \draw (1.3579263675184994,1) node[below left] {$b$} circle(1pt);
                \draw (-0.6813306436049771,1) node[below right] {$a$} circle(1pt);
            \end{tikzpicture}
        \end{center}
        \begin{itemize}
            \item Phương trình $x^3-3x=a$ có $3$ nghiệm $x_1$, $x_2$, $x_3$ với $x_1<x_2<x_3$.
            \item Phương trình $x^3-3x=b$ có $3$ nghiệm $x_4$, $x_5$, $x_6$ với $x_4<x_5<x_6$.
            \item Phương trình $x^3-3x=c$ có $1$ nghiệm $x_7\quad(x_7>x_6)$.
        \end{itemize}
        \begin{center}
            \begin{tikzpicture}[>=stealth]
                \draw[->,line width = 1pt] (-3,0)--(0,0) node[below left]{$O$}--(3,0) node[below]{$x$};
                \draw[->,line width = 1pt] (0,-3) --(0,6) node[right]{$y$};
                \draw (-1,0) node[below left]{$-1$} circle (1pt);
                \draw (0,2) node[above right]{$2$} circle (1pt);
                \draw (1,0) node[above left]{$1$} circle (1pt);
                \draw (2,0) node[below right]{$2$} circle (1pt);
                \draw (0,-2) node[below left]{$-2$} circle (1pt);
                \draw [thick,color=red, domain=-2.1:2.3, samples=100] %
                plot (\x, {(\x)^3-3*(\x)});
                \draw [thick, domain=-3:3, samples=100] plot (\x, {0*(\x)+4.32});
                \draw [thick, domain=-3:3, samples=100] plot (\x, {0*(\x)+1.36});
                \draw [thick, domain=-3:3, samples=100] plot (\x, {0*(\x)-0.68});
                \draw (2.3,5) node[above right]{$y=x^3-3x$};
                \draw (-2,-0.7) node[below left]{$y=a$};
                \draw (-2,1.4) node[below left]{$y=b$};
                \draw (-2,4) node[left]{$y=c$};
                \draw[dashed](-1,0)--(-1,2)--(0,2);
                \draw[dashed](0,-2)--(1,-2)--(1,0);
                \draw (-1.84,-0.68) node[below right] {$x_1$};
                \draw (0.23,-0.68) node[below right] {$x_2$};
                \draw (1.61,-0.68) node[below right] {$x_3$};
                \draw (-1.43,1.36) node[above left] {$x_4$};
                \draw (-0.56,1.36) node[above right] {$x_5$};
                \draw (1.93,1.36) node[below right] {$x_6$};
                \draw (2.22,4.32) node[below right] {$x_7$};
            \end{tikzpicture}
        \end{center}
        Bảng xét dấu $g'(x)$
        \begin{center}
            \begin{tikzpicture}
                \tkzTabInit[nocadre=false,lgt=2.3,espcl=1.3]
                {$x$ /1.1,$x^2-1$ /0.7,$f'(x^3-3x)$ /0.7,$g'(x)$ /0.7,$g(x)$ /2}
                {$-\infty$,$x_1$,$x_4$,$-1$,$x_5$,$x_2$,$1$,$x_3$,$x_6$,$x_7$,$+\infty$}
                \tkzTabLine{,+,|,+,|,+,$0$,-,|,-,|,-,$0$,+,|,+,|,+,|,+,}
                \tkzTabLine{,-,$0$,+,$0$,-,|,-,$0$,+,$0$,-,|,-,$0$,+,$0$,-,$0$,+,}
                \tkzTabLine{,-,$0$,+,$0$,-,$0$,+,$0$,-,$0$,+,$0$,-,$0$,+,$0$,-,$0$,+,}
                \tkzTabVar{+/,-/,+/,-/,+/,-/,+/,-/,+/,-/,+/}
            \end{tikzpicture}
        \end{center}
        Dựa vào bảng xét dấu ta kết luận hàm số $y=g(x)$ có $5$ điểm cực tiểu.
    }
\end{ex}
\begin{ex}%[2D1G2-2]
    \immini{Cho hàm số $y=f(x)$ có đạo hàm và liên tục trên $\mathbb{R}$ và có đồ thị $y=f'(x)$ như hình vẽ. Hàm $y=f(x^2-2)-\dfrac{1}{2}x^4+\dfrac{3}{2}x^2$ có bao nhiêu điểm cực tiểu?
        \choice[2]
        {$4$}
        {$1$}
        {$2$}
        {\True$3$}}
    {\begin{tikzpicture}[>=stealth,font=\scriptsize,x=1.2cm]
            \begin{scope}[scale=.7]
                \def\mx{-1} \def\max{3}
                \def\my{-1} \def\may{3.5}
                \def\hamso(#1,#2){plot [samples=200,smooth,domain=#1:#2](\x,{
                        2*(\x)^4-5*(\x)^3+1.5*(\x)^2+2.25*(\x)
                    })}
                \draw[fill=black]
                (-0.5,0)circle (.7pt)node[shift={(-110:.5)}]{$-\dfrac{1}{2}$}
                (0.5,0)circle (.7pt)node[shift={(-90:.5)}]{$\dfrac{1}{2}$}
                (2,0)circle (.7pt)node[shift={(-90:.5)}]{$2$}
                (1.5,0)circle (.7pt)node[shift={(-90:.5)}]{$\dfrac{3}{2}$}
                (0,1)circle (.7pt)node[shift={(180:.3)}]{$1$}
                (0,2.5)circle (.7pt)node[shift={(180:.3)}]{$\dfrac{5}{2}$}
                (0.5,1)circle (.7pt)
                (2,2.5)circle (.7pt)
                ;
                \draw[dashed,thin] (0.5,0)|-(0,1)(2,0)|-(0,2.5);
                %===========================================
                \draw[->] (\mx,0)--(0,0) node [below right] {$O$}--(\max,0) node[below] {$x$};
                \draw[->] (0,\my)--(0,\may) node[left] {$y$};
                \clip (\mx,\my) rectangle (\max,\may);
                \draw \hamso(\mx,\max);
            \end{scope}
    \end{tikzpicture}}
    \loigiai{
        \immini{
            Ta có $y'=2xf'(x^2-2)-2x^3+3x=2x\left(f'(x^2-2)-x^2+\dfrac{3}{2}\right)$.\\
            $y'=0\Leftrightarrow \hoac{&x=0\\&f'(x^2-2)-x^2+\dfrac{3}{2}=0.\quad(*)}$\\
            Đặt $t=x^2-2$ ta có $(*)\Leftrightarrow f'(t)-t-\dfrac{1}{2}=0\Leftrightarrow f'(t)=t+\dfrac{1}{2}$.\\
            Dựa vào đồ thị hàm số bên ta có
            $$f'(t)=t+\dfrac{1}{2}\Leftrightarrow \hoac{&t=-\dfrac{1}{2}\\&t=\dfrac{1}{2}\\&t=2.}$$
        }{
            \begin{tikzpicture}[scale=1.2,>=stealth,font=\footnotesize,y=.8cm]
                \def\mx{-1} \def\max{3}
                \def\my{-1} \def\may{3.5}
                \def\hamso(#1,#2){plot [samples=200,smooth,domain=#1:#2](\x,{
                        (\x)+0.5
                    })}
                \def\ham(#1,#2){plot [samples=200,smooth,domain=#1:#2](\x,{
                        2*(\x)^4-5*(\x)^3+1.5*(\x)^2+2.25*(\x)
                    })}
                \draw[fill=black]
                (-0.5,0)circle (.7pt)node[shift={(-110:.5)}]{$-\dfrac{1}{2}$}
                (0.5,0)circle (.7pt)node[shift={(-90:.5)}]{$\dfrac{1}{2}$}
                (2,0)circle (.7pt)node[shift={(-90:.5)}]{$2$}
                (1.5,0)circle (.7pt)node[shift={(-90:.5)}]{$\dfrac{3}{2}$}
                (0,1)circle (.7pt)node[shift={(180:.3)}]{$1$}
                (0,2.5)circle (.7pt)node[shift={(180:.3)}]{$\dfrac{5}{2}$}
                (0.5,1)circle (.7pt)
                (2,2.5)circle (.7pt)
                ;
                \draw[dashed,thin] (0.5,0)|-(0,1)(2,0)|-(0,2.5);
                %===========================================
                \draw[->] (\mx,0)--(0,0) node [below right] {$O$}--(\max,0) node[below] {$t$};
                \draw[->] (0,\my)--(0,\may) node[left] {$y$};
                \clip (\mx,\my) rectangle (\max,\may);
                \draw \hamso(\mx,\max)\ham(\mx,\max);
        \end{tikzpicture}}
        \noindent
        Suy ra $\hoac{&x^2-2=-\dfrac{1}{2}\\&x^2-2=\dfrac{1}{2}\\&x^2-2=2}\Leftrightarrow\hoac{&x=\pm\dfrac{\sqrt{6}}{2}\\&x=\pm\dfrac{\sqrt{10}}{2} &\text{ (nghiệm kép)}\\&x=\pm 2.}$\\
        Bảng xét dấu
        \begin{center}
            \begin{tikzpicture}
                \tkzTabInit[nocadre=false, lgt=0.7,espcl=1.6,deltacl=0.5]
                {$x$/1.2, $y'$ /0.6}
                {$-\infty$ , $-2$ , $-\dfrac{\sqrt{10}}{2}$ ,$-\dfrac{\sqrt{6}}{2}$,$0$,$\dfrac{\sqrt{6}}{2}$,$\dfrac{\sqrt{10}}{2}$ ,$2$, $+\infty$}
                \tkzTabLine{ ,-,0,+, 0 ,+, 0 ,-,0,+,0,-,0,-,0,+ }
            \end{tikzpicture}
        \end{center}
        Suy ra hàm số có $3$ điểm cực tiểu.
    }
\end{ex}
\begin{ex}%[2D1G2-6]
    \immini{Cho hàm số $y=f(x)$ có bảng biến thiên bên dưới. Số điểm cực đại và số điểm cực tiểu của hàm số $y=f^2(2x)-2f(2x)+1$ lần lượt là
        \choice
        {\True $2$ và $3$}
        {$3$ và $2$}
        {$1$ và $1$}
        {$2$ và $2$}}{\begin{tikzpicture}
            \tkzTabInit[nocadre=false,lgt=1.2,espcl=2,deltacl=0.6]
            {$x$ /0.6,$f'(x)$ /0.6,$f(x)$ /2}
            {$-\infty$,$-1$,$2$,$+\infty$}
            \tkzTabLine{,-,$0$,+,$0$,-,}
            \tkzTabVar{+/$+\infty$,-/$0$,+/$3$,-/$-\infty$}
    \end{tikzpicture}}
    \loigiai{
        Đặt $g(x)=	f^2(2x)-2f(2x)+1=\left[f(2x)-1\right]^2$.\\
        $\Rightarrow g'(x)=2\cdot\left[f(2x)-1\right]\cdot f'(2x)$.\\
        $\Rightarrow g'(x)=0\Leftrightarrow \hoac{&f(2x)=1\\&f'(2x)=0.}$\\
        \begin{itemize}
            \item $f(2x)=1\Leftrightarrow \hoac{&2x=a\quad(a<-1)\\&2x=b\quad(-1<b<2)\\&2x=c\quad(2<c)}\Leftrightarrow \hoac{&x=\dfrac{a}{2}\quad(\dfrac{a}{2}<-\dfrac{1}{2})\\&x=\dfrac{b}{2}\quad(-\dfrac{1}{2}<\dfrac{b}{2}<1)\\&x=\dfrac{c}{2}\quad(1<\dfrac{c}{2}).}$
            \item $f'(2x)=0\Leftrightarrow\hoac{&2x=-1\\&2x=2}\Leftrightarrow\hoac{&x=-\dfrac{1}{2}\\&x=1.}$
        \end{itemize}
        Bảng biến thiên hàm số $y=g(x)$
        \begin{center}
            \begin{tikzpicture}
                \tkzTabInit[nocadre=false,lgt=2.1,espcl=2,deltacl=0.6]
                {$x$ /1.1,$f'(2x)$ /0.7,$f(2x)-1$ /0.7,$g'(x)$ /0.7,$g(x)$ /2}
                {$-\infty$,$\dfrac{a}{2}$,$-\dfrac{1}{2}$,$\dfrac{b}{2}$,$1$,$\dfrac{c}{2}$,$+\infty$}
                \tkzTabLine{,-,|,-,$0$,+,|,+,$0$,-,|,-,}
                \tkzTabLine{,+,$0$,-,|,-,$0$,+,|,+,$0$,-,}
                \tkzTabLine{,-,$0$,+,$0$,-,$0$,+,$0$,-,$0$,+,}
                \tkzTabVar{+/,-/,+/,-/,+/,-/,+/}
            \end{tikzpicture}
        \end{center}
        Dựa vào bảng biến thiên ta thấy hàm số $y=g(x)$ có $2$ điểm cực đại và $3$ điểm cực tiểu.
    }
\end{ex}

\begin{ex}%[2D1G2-6]
    Cho hàm số bậc ba $y=f(x)$ có đồ thị như hình bên. Có bao nhiêu giá trị nguyên của tham số $m$ để hàm số $y=\vert f^2(x)+2f(x)+m\vert$ có $9$ điểm cực trị?
    \choice
    {\True $24$}
    {Vô số}
    {$25$}
    {$23$}
    \loigiai{
        Đặt $y=g(x)=f^2(x)+2f(x)+m=\left[f(x)+1\right]^2+m-1.\\
        \Rightarrow g'(x)=2\left[f(x)+1\right]\cdot f'(x).\\
        \Rightarrow g'(x)=0\Leftrightarrow \hoac{&f'(x)=0\\&f(x)=-1}\Leftrightarrow \hoac{&x=1\\&x=3\\&x=a\quad(0<a<1)\\&x=b\quad(1<b<3)\\&x=c\quad(3<c).}$\\
        Từ đồ thị ta suy ra
        \begin{itemize}
            \item $f'(x)+1>0\Leftrightarrow f'(x)>-1\Leftrightarrow a<x<b \text{ hoặc } x>c$.
            \item $f'(x)+1<0\Leftrightarrow f'(x)<-1\Leftrightarrow x<a \text{ hoặc } b<x<c$.
        \end{itemize}
        Bảng biến thiên hàm số $y=g(x)$
        \begin{center}
            \begin{tikzpicture}
                \tkzTabInit[nocadre=false,lgt=2.3,espcl=1.8]
                {$x$ /1.1,$f'(x)$ /0.7,$f(x)+1$ /0.7,$g'(x)$ /0.7,$g(x)$ /2}
                {$-\infty$,$0$,$a$,$1$,$b$,$3$,$c$,$+\infty$}
                \tkzTabLine{,+,t,+,|,+,$0$,-,|,-,$0$,+,|,+,}
                \tkzTabLine{,-,t,-,$0$,+,|,+,$0$,-,|,-,$0$,+,}
                \tkzTabLine{,-,t,-,$0$,+,$0$,-,$0$,+,$0$,-,$0$,+,}
                \tkzTabVar{+/,R,-/$m-1$,+/$m+24$,-/$m-1$,+/$m$,-/$m-1$,+/}
            \end{tikzpicture}
        \end{center}
        Đồ thị hàm số $y=|g(x)|$ gồm có $2$ phần như sau:
        \begin{itemize}
            \item Phần 1: Trùng với đồ thị hàm số $y=g(x)$ với $g(x)\ge0$.
            \item Phần 2: Là phần đối xứng với phần đồ thị của hàm số $y=g(x)$ với $g(x)<0$ qua trục $\text{Ox}$.
        \end{itemize}
        Kết hợp với bảng biến thiên hàm số $y=g(x)$ ta suy ra hàm số $y=|g(x)|$ có $9$ điểm cực trị khi và chỉ khi $m\le 0<m+24 \Leftrightarrow -24<m\le0$. Mà $m$ là số nguyên nên ta được $24$ giá trị của $m$.
    }
\end{ex}
\begin{ex}%[2D1K2-6]
    Có bao giá trị nguyên của tham số $m$ thoả mãn $\vert m\vert<10$ sao cho hàm số $y=\vert x^3-(m-2)x^2-mx-m^2\vert$ có $3$ điểm cực tiểu?
    \choice
    {$9$}
    {$10$}
    {\True $8$}
    {$16$}
    \loigiai{
        Đặt hàm số $y=f(x)=x^3-(m-2)x^2-mx-m^2=(x-m)(x^2+2x+m)=(x-m)\left[x(x+2)+m\right]$.
        Suy ra $f'(x)=3x^2-2(m-2)x-m=0$ có $\Delta'=(m-2)^2+3m=m^2-m+4>0$ với mọi $m$.\\
        Theo định lí Vi-ét ta có $\heva{&x_1+x_2=\dfrac{2(m-2)}{3}\\&x_1x_2=-\dfrac{m}{3}}$.\\
        Hàm số $y=|f(x)|$ có $3$ điểm cực tiểu khi và chỉ khi $y(x_1)\cdot y(x_2)<0$.\\
        Thực hiện biến đổi\\
        $y(x_1)\cdot y(x_2)=  (x_1-m)(x_2-m)\left[x_1(x_1+2)+m\right]\left[x_2(x_2+2)+m\right]\\
        = (x_1-m)(x_2-m)\left[x_1x_2(x_1+2)(x_2+2)+m(x_1^2+x_2^2)+2m(x_1+x_2)+m^2\right]\\
        = \left[x_1x_2-m(x_1+x_2)+m^2\right]\left[x_1x_2\left(x_1x_2+2(x_1+x_2)+4\right)+m(x_1^2+x_2^2)+2m(x_1+x_2)+m^2\right]\\
        = \left(\dfrac{m^2}{3}+m\right)\left[-\dfrac{m}{3}\left(m+\dfrac{4}{3}\right)+m\left(\dfrac{4m^2}{9}-\dfrac{10m}{9}+\dfrac{16}{9}\right)+\dfrac{4m^2}{3}-\dfrac{8m}{3}+m^2\right]\\
        = \dfrac{2}{27}m^2(m+3)(2m^2+4m-5)$.\\
        Suy ra $y(x_1)\cdot y(x_2)<0\Leftrightarrow m^2(m+3)(2m^2+4m-5)<0\Leftrightarrow \hoac{&m<-3\\&\dfrac{-2-\sqrt{14}}{2}<m<0\\&0<m<\dfrac{-2+\sqrt{14}}{2}}$.\\
        Kết hợp với điều kiện $m$ là số nguyên thỏa $|m|<10$ ta được $m\in\{-9;-8;-7;-6;-5;-4;-2;-1\}$.\\
        Vậy có $8$ giá trị nguyên của tham số $m$.
    }
\end{ex}

\begin{ex}%[2D1G2-6]
    \immini{	Cho hàm số $f(x)=ax^4+bx^3+cx^2+dx+e, (ae<0)$. Đồ thì hàm số $y=f'(x)$ như hình bên dưới. Hàm số $y=\left|4f(x)-x^2\right|$ có bao nhiêu điểm cực tiểu?
        \choice[2]
        {$4$}
        {$5$}
        {\True $3$}
        {$2$}
    }{\begin{tikzpicture}[>=stealth,font=\scriptsize,x=1.3cm,y=1.5cm]
            \begin{scope}[scale=0.5]
                \draw[->] (-2,0) -- (3,0) node[below] { $x$};
                \draw[->] (0,-1) -- (0,3) node[left] {\ $y$};
                \draw (0,0) node[above left] { $O$} circle (1pt);
                \draw[smooth](0,0) parabola bend (-0.6,-1)(-1.7,1.7);
                \draw(0,0) parabola bend (1.3,2.5)(2,1);
                \draw [dashed] (0,1)--(2,1)--(2,0);
                \draw [dashed] (-1.05,0)--(-1.05,-0.5)--(0,-0.5);
                \draw (-1,0) node[above] { $-1$};
                \draw (0,-0.5) node[right] { $-\dfrac{1}{2}$};
                \draw (0,1) node[left] { $1$};
                \draw (2,0) node[below] { $2$};
            \end{scope}
    \end{tikzpicture}}
    \loigiai{
        Ta có $f'(x)=4ax^3+3bx^2+2cx+d$. Từ đồ thị hàm số $f'(x)$ suy ra $a<0$, do đó $e>0$.\\
        Đặt $y=g(x)=4f(x)-x^2\Rightarrow g'(x)	=4f'(x)-2x=4\left[f'(x)-\dfrac{x}{2}\right]$.\\
        \begin{center}
            \begin{tikzpicture}[>=stealth,x=1.0cm,y=1.0cm,thick, scale=1.0]
                \draw[->] (-2,0) -- (3,0) node[below] {\footnotesize $x$};
                \draw[->] (0,-1) -- (0,3) node[left] {\footnotesize $y$};
                \draw (0,0) node[below left] {\footnotesize $O$} circle (1pt);
                \draw[smooth](0,0) parabola bend (-0.6,-1)(-1.7,1.7);
                \draw(0,0) parabola bend (1.3,2.5)(2,1);
                \draw [dashed] (0,1)--(2,1)--(2,0);
                \draw [dashed] (-1.05,0)--(-1.05,-0.5)--(0,-0.5);
                \draw (-1,0) node[above] {\footnotesize $-1$};
                \draw (0,-0.5) node[right] {\footnotesize $-\dfrac{1}{2}$};
                \draw (0,1) node[left] {\footnotesize $1$};
                \draw (2,0) node[below] {\footnotesize $2$};
                \draw [thick, domain=-2.1:2.5, samples=100] %
                plot (\x, {0.5*(\x)});
            \end{tikzpicture}
        \end{center}
        Suy ra $g'(x)=0\Leftrightarrow f'(x)-\dfrac{x}{2}=0\Leftrightarrow f'(x)=\dfrac{x}{2}\Leftrightarrow \hoac{&x=-1\\&x=0\\&x=2}$.\\
        Bảng biến thiên
        \begin{center}
            \begin{tikzpicture}
                \tkzTabInit[nocadre=false,lgt=1.2,espcl=2.3]
                {$x$ /0.6,$g'(x)$ /0.6,$g(x)$ /2}
                {$-\infty$,$-1$,$0$,$2$,$+\infty$}
                \tkzTabLine{,+,$0$,-,$0$,+,$0$,-,}
                \tkzTabVar{-/,+/,-/$4e$,+/,-/}
            \end{tikzpicture}
        \end{center}
        Vì $4e>0$ nên từ bảng biến thiên hàm số $g(x)$ ta suy ra hàm số $y=\left|g(x)\right|$ có $3$ điểm cực tiểu.
    }
\end{ex}

\begin{ex}%[2D1G2-6]
    \immini{	Cho hàm số bậc bốn $f(x)$ có $f(0)=-1$. Hàm số $y=f'(x)$ có đồ thị là hình bên. Số điểm cực trị của hàm số $y=\vert 4f(x+1)+x^2+2x\vert$ là
        \choice[2]
        {$3$}
        {\True $5$}
        {$4$}
        {$6$}}{\begin{tikzpicture}[>=stealth,font=\scriptsize,y=.8cm]
            \begin{scope}[scale=.5]
                \draw[->] (-3.5,0) -- (5,0) node[below] {$x$};
                \draw[->] (0,-4) -- (0,2) node[left] { $y$};
                \draw (0,0) node[below left] {$O$} circle (1pt);
                \draw (-3.6,-3.6) ..controls +(60:0.2) and +(-180:1.3).. (-1.5,1.2) ..controls +(0:1.6) and +(-180:1.6) .. (2.8,-3.4)..controls +(0:0.5) and +(-100:4.5) .. (4.95,1.8);
                \draw [dashed] (-2,0)--(-2,1)--(0,1);
                \draw [dashed] (0,-2)--(4,-2)--(4,0);
                \draw (-2,0) node[below] { $-2$};
                \draw (0,-2) node[left] { $-2$};
                \draw (0,1) node[right] { $1$};
                \draw (4,0) node[above] {$4$};
            \end{scope}
    \end{tikzpicture}}
    \loigiai{
        Đặt $y=g(x)=4f(x+1)+x^2+2x\Rightarrow g'(x)=4f'(x+1)+2x+2=4\left[f'(x+1)+\dfrac{x+1}{2}\right]$.\\
        Suy ra $g'(x)=0\Leftrightarrow f'(x+1)=-\dfrac{x+1}{2}$.\\
        Đặt $t=x+1$ thì phương trình trở thành $f'(t)=-\dfrac{t}{2}$. Nghiệm của phương trình này là hoành độ giao điểm của đồ thị hàm số $y=f'(t)$ và $y=-\dfrac{t}{2}$.
        \begin{center}
            \begin{tikzpicture}[>=stealth,x=1.0cm,y=1.0cm,thick, scale=0.8]
                \draw[->] (-3.5,0) -- (5,0) node[below] {\footnotesize $t$};
                \draw[->] (0,-4) -- (0,2) node[left] {\footnotesize $y$};
                \draw (0,0) node[below left] {\footnotesize $O$} circle (1pt);
                \draw[very thick] (-3.6,-3.6) ..controls +(60:0.2) and +(-180:1.3).. (-1.5,1.2) ..controls +(0:1.6) and +(-180:1.6) .. (2.8,-3.4)..controls +(0:0.5) and +(-100:4.5) .. (4.95,1.8);
                \draw [thick, domain=-3.5:5, samples=100] %
                plot (\x, {-0.5*(\x)});
                \draw [dashed] (-2,0)--(-2,1)--(0,1);
                \draw [dashed] (0,-2)--(4,-2)--(4,0);
                \draw (-2,0) node[below] {\footnotesize $-2$};
                \draw (0,-2) node[left] {\footnotesize $-2$};
                \draw (0,1) node[right] {\footnotesize $1$};
                \draw (4,0) node[above] {\footnotesize $4$};
            \end{tikzpicture}
        \end{center}
        Do đó\\
        $$f'(t)=-\dfrac{t}{2}\Leftrightarrow\hoac{&t=-2\\&t=0\\&t=4}\Rightarrow \hoac{&x+1=-2\\&x+1=0\\&x+1=4}\Leftrightarrow \hoac{&x=-3\\&x=-1\\&x=3.}$$
        Bảng biến thiên
        \begin{center}
            \begin{tikzpicture}
                \tkzTabInit[nocadre=false,lgt=1.2,espcl=2.3]
                {$x$ /0.6,$g'(x)$ /0.6,$g(x)$ /2}
                {$-\infty$,$-3$,$-1$,$3$,$+\infty$}
                \tkzTabLine{,-,$0$,+,$0$,-,$0$,+,}
                \tkzTabVar{+/,-/,+/$-5$,-/,+/}
            \end{tikzpicture}
        \end{center}
        Từ bảng biến thiên suy ra hàm số $y=g(x)$ có $3$ cực trị âm, do đó hàm số $y=\left|g(x)\right|$ có $5$ điểm cực trị.
    }
\end{ex}
\begin{ex}%[2D1Y2-6]
    \immini{	Cho hàm số $y=f(x)$ có bảng biến thiên như hình vẽ. Hàm số $y=f\left(|x|\right)$ đạt cực đại tại.
        \choice[2]
        {$x=-1$}
        {\True $x=0$}
        {$x=2$}
        {$x=-2$}}{\begin{tikzpicture}[>=stealth,scale=0.9]
            \tkzTabInit[nocadre=false,lgt=1,espcl=2.6,deltacl=0.5]{$x$/.7 ,$y'$/.7,$y$/2}
            {$-\infty$ , $-1$ , $2$ , $+\infty$}
            \tkzTabLine{ , + , $0$ , - , $0$ , + , }
            \tkzTabVar{-/$-\infty$ , +/$3$ , -/$1$ , +/$+\infty$}
    \end{tikzpicture}}
    \loigiai{Từ bảng biến thiên của hàm số $y=f(x)$ ta có bảng biến thiên của hàm số $y=f\left(|x|\right)$ như sau
        \begin{center}
            \begin{tikzpicture}[scale=0.9]
                \foreach \x/\texn in {0/x,2/-\infty,4/-2,6/0,8/2,10/+\infty} \path (\x,3.5)node{$\texn$};
                \foreach \x/\texn in
                {0/y',3/-,4/0,5/+,6/0,7/-,8/0,9/+} \path (\x,2.5)node{$\texn$};
                \foreach \x/\y/\texn in {0/1/y,
                    2/1.5/+\infty,4/0/1,6/1.0/f(0),8/0/1,10/1.5/+\infty}
                \path (\x,\y) node(\x){$\texn$};
                \foreach \x/\y/\texn in {2/4,4/6,6/8,8/10}
                \draw[-stealth] (\x)--(\y);
                \draw
                (-.5,3)--(10.5,3) (1,4)--(1,0)
                (-.5,2)--(10.5,2);
            \end{tikzpicture}
        \end{center}
        Từ bảng biến thiên ta thấy hàm số $y=f\left(|x|\right)$ đạt cực đại tại $x=0$.
    }
\end{ex}
%%=====Câu 70
\begin{ex}%[2D1Y2-6]
    \immini{	Cho hàm số $y=f(x)$ có bảng biến thiên như hình vẽ. Tổng các giá trị cực đại của hàm số $y=\left|f(x)\right|$ là
        \choice[2]
        {\True $9$}
        {$-3$}
        {$3$}
        {$7$}}{\begin{tikzpicture}[>=stealth]
            \tkzTabInit[nocadre=false,lgt=1,espcl=1.7,deltacl=0.5]{$x$/.7 ,$y'$/.7,$y$/2}
            {$-\infty$ , $-1$ , $0$ , $1$ , $+\infty$}
            \tkzTabLine{ , - , $0$ , + , $0$ , - , $0$ , + , }
            \tkzTabVar{+/$+\infty$ , -/$-2$, +/$3$ , -/$-4$ , +/$+\infty$}
    \end{tikzpicture}}
    \loigiai{Từ bảng biến thiên của hàm số $y=f(x)$ ta có bảng biến thiên của hàm số $y=\left|f(x)\right|$ như sau
        \begin{center}
            \begin{tikzpicture}[scale=0.8]
                \foreach \x/\texn in {0/x,2/-\infty,4/x_1,6/-1,8/x_2,10/0,12/x_3,14/1,16/x_4,18/+\infty} \path (\x,3.5)node{$\texn$};
                \foreach \x/\texn in
                {0/y',3/-,4/||,5/+,6/0,7/-,8/||,9/+,10/0,11/-,12/||,13/+,14/0,15/-,16/||,17/+} \path (\x,2.5)node{$\texn$};
                \foreach \x/\y/\texn in {0/1/y,
                    2/1.5/+\infty,4/0/0,6/0.5/2,8/0/0,10/0.7/3,12/0/0,14/1.0/4,16/0/0,18/1.5/+\infty}
                \path (\x,\y) node(\x){$\texn$};
                \foreach \x/\y/\texn in {2/4,4/6,6/8,8/10,10/12,12/14,14/16,16/18}
                \draw[-stealth] (\x)--(\y);
                \foreach \x in {4,8,12}\draw[dashed,red] (\x)--+(3.7,0);
                \draw[dashed,red] (1.2,0)--+(2.7,0) (16)--+(2.0,0) node[below=-.1]{\small $y=0$};
                \draw
                (-.5,3)--(18.5,3) (1,4)--(1,0)
                (-.5,2)--(18.5,2);
            \end{tikzpicture}
        \end{center}
        Từ bảng biến thiên ta thấy hàm số $y=\left|f(x)\right|$ có $3$ giá trị cực đại lần lượt là $2$, $3$, $4$.\\
        Tổng các giá trị cực đại là $9$.
    }
\end{ex}
\begin{ex}%[2D1Y2-6]
    Cho hàm số $y=f(x)$ có đạo hàm $y=f'(x)=(x-1)(x-2)^4(x^2-4)$. Số điểm cực trị của hàm số $y=f(|x|)$ là
    \choice
    {$3$}
    {$2$}
    {$4$}
    {\True $5$}
\end{ex}
\begin{ex}%[2D1Y2-6]
    Cho hàm số $y=f(x)$ có đạo hàm $y=f'(x)=(x^3-2x^2)(x^3-2x)$ trên $\mathbb{R}$. Hàm số $y=|f(4-2021x)|$ có nhiều nhất bao nhiêu điểm cực trị?
    \choice
    {\True$9$}
    {$11$}
    {$2021$}
    { $5$}
\end{ex}
\begin{ex}%[2D1B2-6]
    Có bao nhiêu giá trị nguyên của tham số $m$ để hàm số $y=|3x^4-4x^3-12x^2+m|$ có $7$ điểm cực trị?
    \choice
    {$3$}
    {$5$}
    {$6$}
    {\True $4$}
    \loigiai{
        Đặt $f(x)=3x^4-4x^3-12x^2+m$ $\Rightarrow f'(x)=12x^3-12x^2-24x=0 \Rightarrow x=0; x=-1; x=2$.\\
        Qua BBT của $y=f(x)$ ta suy ra $y=|f(x)|$ có $7$ điểm cực trị $\Rightarrow \heva{&m>0\\&m-5<0} \Rightarrow 0<m<5$. Vậy có $4$ giá trị nguyên $m$ thỏa yêu cầu bài toán.
    }
\end{ex}
\begin{ex}%[2D1B2-6]
    Tìm các giá trị của $m$ để hàm số $f(x)=|x^3+3x^2+m-3|$ có ba điểm cực trị.
    \choice
    {$m=3; m=-1$}
    {$m\ge 1; m \le-3$}
    {$1\le m \le 3$}
    {\True $m\ge 3; m \le -1$}
\end{ex}
\begin{ex}%[2D1B2-6]
    Cho hàm số $y=f(x)=x^3-3mx^2+3(m^2-4)x+1$, có bao nhiêu số nguyên $m \in (-10;10)$  để hàm số $y=f(|x|)$ có đúng $5$ điểm cực trị.
    \choice
    {$3$}
    {$6$}
    {$8$}
    {\True $7$}
    \loigiai{
        $y=f(|x|)$ có đúng $5$ điểm cực trị $\Rightarrow y=f(x)$ có hai điểm cực trị dương.\\
        $f'(x)=3x^2-6mx+3(m^2-4)=0 \Rightarrow x=m-2; x=m+2$ có hai nghiệm dương $\Leftrightarrow m-2>0 \Leftrightarrow m >2$.\\ Vậy có $7$ giá trị $m$ thỏa yêu cầu bài toán.
    }
\end{ex}
\begin{ex}%[2D1G2-6]
    Cho hàm số $f(x)=\dfrac{1}{3}x^3-(2m-1)x^2+(8-m)x+2020$ với $m$ là tham số. Tập hợp tất cả các giá trị của tham số $m$ để hàm số $y=f\left(\vert x\vert\right)$ có điểm $5$ cực trị là khoảng $(a;b)$. Tích $a\cdot b$ bằng
    \choice
    {$12$}
    {$16$}
    {$10$}
    {\True $14$}
    \loigiai{
        Tập xác định $\mathscr{D}=\mathbb{R}$.\\
        Ta có $f\left(|-x|\right)=f\left(|x|\right)$, $\forall x\in\mathbb{R}$ nên $y=f\left(|x|\right)$ là hàm số chẵn. \\
        Do đó, đồ thị hàm số $y=f\left(|x|\right)$ đối xứng qua trục tung.\\
        Suy ra hàm số $y=f\left(|x|\right)$ luôn có một điểm cực trị là $x=0$.\\
        Do đó, $y=f\left(|x|\right)$ có $5$ điểm cực trị $\Leftrightarrow$ hàm số $y=f(x)$ có $2$ điểm cực trị dương.\\
        \phantom{Do đó, $y=f\left(|x|\right)$ có $5$ điểm cực trị} $\Leftrightarrow$  $f'(x)=0$ có hai nghiệm dương phân biệt.\\
        Ta có $f'(x)=x^2-2(m-1)x+8-m$.\\
        Yêu cầu bài toán $\Leftrightarrow\heva{&\Delta'>0 \\ &S>0 \\ &P>0}\Leftrightarrow\heva{&4m^2-3m-7>0 \\ &2m-1>0 \\ &8-m>0}\Leftrightarrow\heva{&m<-1\;\text{hoặc}\;m>\dfrac{7}{4} \\ &m>\dfrac{1}{2} \\ &m<8}\Leftrightarrow \dfrac{7}{4}<m<8$.
        Suy ra $a\cdot b=14$.
    }
\end{ex}
\begin{ex}%[2D1G2-6]
    \immini{	Cho hàm số $f(x)$ có đạo hàm liên tục trên $\mathbb{R}$ và đồ thị hàm số $f'(x)$ như hình vẽ. Hàm số $y=f\left(x^2-2\vert x\vert\right)$ có bao nhiêu điểm cực tiểu?
        \choice[2]
        {$1$}
        {\True $2$}
        {$5$}
        {$3$}}{\begin{tikzpicture}[>=stealth,font=\scriptsize]
            \draw[->] (-2,0) -- (2,0) node[below] {$x$};
            \draw[->] (0,-1) -- (0,2) node[left] {$y$};
            \draw (0,0) node[below right] {$O$} circle(1pt) (-1,0) node[above left]{ $-1$} (1,0) node[below]{ $1$} (0,1) node[right]{ $1$} (-1/3,0) node[below]{\ $-\dfrac{1}{3}$};
            \draw [dashed] (-1/3,0)--(-1/3,1.2);
            \draw plot[smooth,tension=.65] coordinates{(-1.2,-0.5) (-1/3,1.2) (1,0) (1.8,1.5)};
    \end{tikzpicture}}
    \loigiai{
        Đặt $g(x)=f(x^2-2x)\Rightarrow g'(x)=2(x-1)f'(x^2-2x).\\
        g'(x)=0\Leftrightarrow \hoac{&x=1\\&f'(x^2-2x)=0}\Leftrightarrow \hoac{&x=1\\&x^2-2x=-1\\&x^2-2x=1}\Leftrightarrow \hoac{&x=1\text{ (bội 3)}\\&x=1-\sqrt{2}\\&x=1+\sqrt{2}.}$\\
        Ta có
        \begin{itemize}
            \item $f'(x)>0\Leftrightarrow \heva{&x>-1\\&x\neq1}$ nên $f'(x^2-2x)>0\Leftrightarrow \heva{&x^2-2x>-1\\&x^2-2x\neq -1}\Leftrightarrow \heva{&x\neq 1\\&x=1\pm\sqrt{2}.}$
            \item $f'(x)<0\Leftrightarrow x<-1$ nên $f'(x^2-2x)<0\Leftrightarrow x^2-2x<-1 \text{ (Vô nghiệm)}.$
        \end{itemize}
        Bảng biến thiên hàm số $y=g(x)$
        \begin{center}
            \begin{tikzpicture}
                \tkzTabInit[nocadre=false,lgt=2.5,espcl=2.1,deltacl=0.6]
                {$x$ /0.7,$x-1$ /0.7,$f'(x^2-2x)$ /0.7,$g'(x)$ /0.7,$g(x)$ /2}
                {$-\infty$,$1-\sqrt{2}$,$0$,$1$,$1+\sqrt{2}$,$+\infty$}
                \tkzTabLine{,-,|,-,t,-,$0$,+,|,+,}
                \tkzTabLine{,+,$0$,+,t,+,$0$,+,$0$,+,}
                \tkzTabLine{,-,$0$,-,t,-,$0$,+,$0$,+,}
                \tkzTabVar{+/,R,R,-/,R,+/}
            \end{tikzpicture}
        \end{center}
        Do hàm số $y=f(x^2-2\vert x\vert)$ là hàm số chẵn nên từ bảng biến thiên trên ta suy ra đồ thị hàm số $y=f(x^2-2\vert x\vert)$ gồm hai nhánh như sau
        \begin{itemize}
            \item Nhánh thứ nhất là phần đồ thị hàm số $y=g(x)$ với $x\ge 0$.
            \item Nhánh thứ hai là phần đối xứng với nhánh thức nhất qua trục $Oy$
        \end{itemize}
        Do đó hàm số $y=f(x^2-2\vert x\vert)$ có $2$ điểm cực tiểu.

    }
\end{ex}
\begin{ex}%[2D1G2-6]
    \immini{	Cho hàm bậc bốn $y=f(x)$ có đồ thị như hình vẽ dưới đây. Số điểm cực trị của hàm số $g(x)=f\left(\vert x\vert^3-3\vert x\vert\right)$ là
        \choice[2]
        {$5$}
        {$3$}
        {\True $7$}
        {$11$}}{\begin{tikzpicture}[>=stealth,font=\scriptsize,y=.6cm,x=.7cm]
            \begin{scope}[scale=.5]
                \draw[->] (-5,0) -- (5,0) node[below] {\footnotesize $x$};
                \draw[->] (0,-4.5) -- (0,4) node[left] {\footnotesize $y$};
                \draw (0,0) node[below left] {\footnotesize $O$} circle (1pt);
                \draw[smooth](-1.1,0) parabola bend (-2,-2)(-4,4);
                \draw(-1.1,0) parabola bend (0,2)(1.5,-2.1);
                \draw[smooth](1.5,-2.1) parabola bend (2.6,-4)(4,4);
                \draw (1.5,0) node[above] {\footnotesize $2$};
                \draw (-1.1,0) node[above left] {\footnotesize $-2$};
            \end{scope}
    \end{tikzpicture}}
    \loigiai{
        Đặt $g(x)=f(x^3-3x)\Rightarrow g'(x)=3(x^2-1)f'(x^3-3x)$.\\
        Suy ra $g'(x)=0\Leftrightarrow \hoac{&x^2=1\\&f'(x^3-3x)=0}\Leftrightarrow \hoac{&x=\pm1\\&x^3-3x=2\\&x^3-3x=-2\\&x^3-3x=a\quad(a<-2)\\&x^3-3x=b\quad(b>2).}$\\
        Ta có
        \begin{itemize}
            \item $x^3-3x=2\Leftrightarrow \hoac{&x=2\\&x=-1.}$
            \item $x^3-3x=-2\Leftrightarrow \hoac{&x=-2\\&x=1.}$
            \item $x^3-3x=a\Leftrightarrow x=m \text{ (với $m<-2$)}$.
            \item $x^3-3x=b\Leftrightarrow x=n \text{ (với $n>2$)}$.
        \end{itemize}
        Từ đồ thị hàm số $f'(x)$ ta có $f'(x)>0\Leftrightarrow \hoac{&x<a\\&-2<x<2\\&x>b.}$\\
        Suy ra $h'(x)>0\Leftrightarrow \hoac{&x^3-3x<a\\&-2<x^3-3x<2\\&x^3-3x>b}\Leftrightarrow \hoac{&x<m\\&-2<x<-1\\&-1<x<2\\&x>n.}$\\
        Bảng biến thiên hàm số $y=g(x)$
        \begin{center}
            \begin{tikzpicture}
                \tkzTabInit[nocadre=false,lgt=2.5,espcl=1.7,deltacl=0.5]
                {$x$ /0.7,$x^2-1$ /0.7,$f'(x^3-3x)$ /0.7,$g'(x)$ /0.7,$g(x)$ /2}
                {$-\infty$,$m$,$-2$,$-1$,$0$,$1$,$2$,$n$,$+\infty$}
                \tkzTabLine{,+,|,+,|,+,$0$,-,t,-,$0$,+,|,+,|,+,}
                \tkzTabLine{,+,|,-,|,+,$0$,+,t,+,$0$,+,|,-,|,+,}
                \tkzTabLine{,+,|,-,|,+,$0$,-,t,-,$0$,+,|,-,|,+,}
                \tkzTabVar{-/,+/,-/,+/,R,-/,+/,-/,+/}
            \end{tikzpicture}
        \end{center}
        Từ bảng biến thiên suy ra hàm số $y=g(x)$ có $3$ điểm cực trị ứng với $x>0$ nên hàm số $y=f(|x|^3-3|x|)$ có $7$ điểm cực trị.

    }
\end{ex}
\begin{ex}%[2D1K2-2]
    \immini{
        Hình vẽ dưới đây là đồ thị của hàm số $y=f(x)$.
        Có bao nhiêu giá trị nguyên dương của tham số $m$ để hàm số $y=\left|f(x+1)+m\right|$ có $5$ cực trị?
        \choice
        {$0$}
        {\True $3$}
        {$2$}
        {$1$}
    }{
        \begin{tikzpicture}[>=stealth, font=\footnotesize, line join=round, line cap=round,y=.7cm]
            \begin{scope}[scale=.5]
                \def\xmin{-4} \def\xmax{3}
                \def\ymin{-5.5} \def\ymax{4}
                %\draw[color=gray!50,dashed] (\xmin,\ymin) grid (\xmax,\ymax);
                \draw[->] (\xmin,0)--(\xmax,0) node [below]{$x$};
                \draw[->] (0,\ymin)--(0,\ymax) node [left]{$y$};
                \node at (0,0) [below right]{$O$};
                \clip (\xmin+0.1,\ymin+0.1) rectangle (\xmax-0.5,\ymax-0.1);
                \draw[smooth,samples=300,domain=-3.5:0] plot(\x,{-1.24*(\x)^3-5.74*(\x)^2-5.78*(\x)});
                \draw[smooth,samples=300,domain=0:2.3] plot(\x,{1.78*(\x)^3-1.61*(\x)^2-4.57*(\x)-0.04});
                \draw[dashed](-2.5,0)|-(0,-2.07)  (-0.7,0)|-(0,1.7) (1.3,0)|-(0,-4.8);
                \draw[fill=black](0,-2.07)node[below left]{$-3$}circle(1pt)
                (0,-4.8)node[left]{$-6$}circle(1pt)
                (0,1.7)node[right]{$2$}circle(1pt);
            \end{scope}
        \end{tikzpicture}
    }
    \loigiai{
        Nhận xét
        \begin{itemize}
            \item  Hàm số $y=\left|f(x)-\alpha\right|$ có số điểm cực trị bằng số cực trị của hàm $y=f(x)$ và số giao điểm của đồ thị hàm $y=f(x)$ với đường thẳng $y=\alpha$ (không tính giao điểm là các điểm cực trị).
            \item  Số điểm cực trị của hàm $y=f(x)$ bằng số điểm cực trị của hàm $y=f(x+a)$.
        \end{itemize}
        Từ nhận xét trên ta có: Hàm số $y=f(x+1)$ có $3$ cực trị.\\
        Vậy ta cần đường thẳng $y=-m$ cắt đồ thị hàm số $y=f(x+1)$ tại 2 điểm khác cực trị.\\
        Từ đồ thị ta suy ra: $\hoac{&-6 <-m\leq-3\\&-m\geq 2}\Leftrightarrow\hoac{&3\leq m<6\\&m\leq-2.}$ \\
        Do $m\in\mathbb{N}^*$ nên $m\in\{3,4,5\}$.
    }
\end{ex}
\Closesolutionfile{ans}
%% \indapan{10}{ans/2D1-2-DANG-3}
%%Bài 2. Max-min
% \setcounter{section}{1}
\section{GIÁ TRỊ LỚN NHẤT - NHỎ NHẤT CỦA HÀM SỐ}

\subsection{LÝ THUYẾT CẦN NHỚ}
Cho hàm số $y=f(x)$ xác định trên tập $\mathscr{D}$. Ta có
\immini{\begin{itemize}
		\item[\ding{172}] $M$ là giá trị lớn nhất của hàm số nếu $\heva{&f(x) \le M,\forall x \in \mathscr{D}\\& \exists x_0 \in \mathscr{D}: f(x_0)=M.}$\\
		Kí hiệu \fbox{$\displaystyle\max_{x \in \mathscr{D}}f(x)=M$}
		\vskip 0.5cm
		\item[\ding{173}] $n$ là giá trị nhỏ nhất của hàm số nếu $\heva{&f(x) \ge n,\forall x \in \mathscr{D}\\& \exists x_0 \in \mathscr{D}: f(x_0)=n.}$\\
		Kí hiệu \fbox{$\displaystyle\min_{x \in \mathscr{D}}f(x)=n$}
\end{itemize}
}{
\begin{tikzpicture}[smooth,samples=300,scale=0.7,>=stealth]
	\draw[->] (-1.5,0)--(4.8,0) node[below]{$x$};
	\draw[->] (0,-2)--(0,4) node[right]{$y$};
	\draw (0,0) node[above left]{$O$};
	\draw[line width = 1.2pt,domain=-1:4,blue] plot(\x,{0.5*((\x)^2-4*(\x)+1)});
	\draw[fill=black] (-1,0) circle(1.5pt) (-1,3) circle(2pt) (0,3) circle(1.5pt) (0,-1.5) circle(1.5pt) (2,0) circle(1.5pt) (2,-1.5) circle(2pt) (4,0) circle(1.5pt) (4,0.5) circle(1.5pt);
	\draw[dashed] (-1,0)node[below]{\small$a$}--(-1,3)--(0,3)node[right]{\small$f(a)$} (2,0)node[above]{\small$x_0$}--(2,-1.5)--(0,-1.5)node[left]{\small$f(x_0)$}
	(4,0)node[below]{\small$b$}--(4,0.5);
	\node[above] at (-1,3) {\small $y_{\max}$};
	\node[below] at (2,-1.5) {\small $y_{\min}$};
\end{tikzpicture}}
\begin{note}
	\begin{listEX}[1]
		\item [\ding{172}] Khi yêu cầu tìm max min của hàm số mà không nói rõ xét trên tập nào, thì ta hiểu là tìm max min trên miền xác định của hàm số đó.
		\item [\ding{173}] Để tìm max min của hàm số $y=f(x)$ trên miền $\mathscr{D}$, ta thường lập bảng biến thiên của hàm số $y=f(x)$ trên $\mathscr{D}$. Từ bảng biến thiên, ta kết luận:
		\begin{itemize}
			\item [$\bullet$] Điểm ở vị trí cao nhất $\longrightarrow$ Kết luận max;
			\item [$\bullet$] Điểm ở vị trí thấp nhất $\longrightarrow$ Kết luận min.
		\end{itemize}
		\item [\ding{174}] Để tìm max min của hàm số $y=f(x)$ trên đoạn $[a;b]$ (\textit{$f(x)$ liên tục trên đoạn $[a ; b]$ và có đạo hàm trên $(a ; b)$ (có thể trừ một số hữu hạn các điểm) và $f^{\prime}(x)=0$ chỉ tại một số hữu hạn các điểm trong $(a ; b)$}), thì ta có thể giải như sau:
		\begin{itemize}
			\item [$\bullet$] Giải $f'(x)=0$ tìm các nghiệm $x_0 \in (a;b)$; 
			\item [$\bullet$] Tìm các điểm $x_i\in (a;b)$ mà tại đó đạo hàm không xác định (nếu có).
			\item [$\bullet$] Tính toán $f(a)$, $f(x_0)$, $f(x_i)$, $f(b)$ \quad ($\star$)
			\item [$\bullet$]  Gọi $M$, $n$ lần lượt là số lớn nhất và số nhỏ nhất của các kết quả tính toán ở bước ($\star$) thì
			$$M=\displaystyle\max_{[a;b]}f(x); \quad n=\displaystyle\min_{[a;b]}f(x)$$
		\end{itemize}
	\item [\ding{175}] Ta có thể dùng các bất đẳng thức có sẵn để đánh giá biểu thức cần tìm max, min. 
	% \begin{itemize}
	% 	\item [$\bullet$] Bất đẳng thức Cauchy cho hai số không âm $a$, $b$:
	% 	$$a+b \ge 2\sqrt{ab}$$
	% 	Dấu "=" xảy ra khi $a=b$.
	% 	\item [$\bullet$]  Bất đẳng thức Cauchy cho ba số không âm $a$, $b$, $c$:
	% 	$$a+b +c\ge 3\sqrt[3]{abc}$$
	% 	Dấu "=" xảy ra khi $a=b=c$.
	% 	\item [$\bullet$]  Bất đẳng thức Cauchy cho $n$ số không âm $a_1$, $a_2$,..., $a_n$:
	% 	$$a_1+a_2 +...+a_n \ge n\sqrt[n]{a_1a_2...a_n}$$
	% 	Dấu "=" xảy ra khi $a_1=a_2=...=a_n$.
	% \end{itemize}
	\end{listEX}
\end{note}
% \newpage
\subsection{PHÂN LOẠI VÀ PHƯƠNG PHÁP GIẢI TOÁN}
\begin{dang} {Bài toán tìm max, min của hàm số $y=f(x)$ trên miền $\mathscr{D}$}
	\begin{enumerate}[\iconMT]
		\item \indam{Phương pháp giải:} 
		\begin{listEX}[1]
			\item [\ding{172}] Tính $y'$. Giải phương trình $y'=0$ tìm các nghiệm $x_i \in \mathscr{D}$ và tìm các điểm $x_j \in \mathscr{D}$ mà tại đó $y'$ không xác định.
			\item [\ding{173}] Lập bảng biến thiên của hàm số trên $\mathscr{D}$.
			\item [\ding{174}] Từ bảng biến thiên, kết luận:
			\begin{itemize}
				\item [$\bullet$] Điểm ở vị trí cao nhất $\longrightarrow$ Kết luận max;
				\item [$\bullet$] Điểm ở vị trí thấp nhất $\longrightarrow$ Kết luận min.
			\end{itemize}
		\end{listEX}
		\item \indam{Lưu ý:} Nếu $\mathscr{D}$ là đoạn $\left[a;b\right]$ và hàm số $y=f(x)$ liên tục trên đoạn $\left[a;b\right]$ thì ta có thể làm như sau:
		\begin{listEX}[1]
			\item [\ding{172}] Giải $f'(x)=0$ tìm các nghiệm $x_0 \in (a;b)$;
			\item [\ding{173}] Tìm các điểm $x_i\in (a;b)$ mà tại đó đạo hàm không xác định (nếu có).
			\item [\ding{174}] Tính toán $f(a)$, $f(x_0)$, $f(x_i)$, $f(b)$ \quad ($\star$)
			\item [\ding{175}] Gọi $M$, $n$ lần lượt là số lớn nhất và số nhỏ nhất của các kết quả tính toán ở bước ($\star$) thì
			$$M=\displaystyle\max_{[a;b]}f(x); \quad n=\displaystyle\min_{[a;b]}f(x)$$
		\end{listEX}
		\begin{note}
			\begin{itemize}
				\item[\iconCH] Nếu hàm số $y=f(x)$ đồng biến trên đoạn $\left[a;b\right]$ thì $\min\limits_{[a;b]} f(x)=f(a)$ và $\max\limits_{[a;b]} f(x)=f(b)$.
				\item[\iconCH]  Nếu hàm số $y=f(x)$ nghịch biến trên đoạn $\left[a;b\right]$ thì $\min\limits_{[a;b]} f(x)=f(b)$ và $\max\limits_{[a;b]} f(x)=f(a)$.
			\end{itemize}
		\end{note}
	\end{enumerate}
\end{dang}

\boxmini{BÀI TẬP TỰ LUẬN}
\begin{vd}
	Tìm giá trị lớn nhất và nhỏ nhất (nếu có) của hàm số sau trên đoạn đã chỉ ra.
	\begin{tasks}(2)
		\task $f(x)=-x^3+3x^2+10$ trên đoạn $[-3;1]$.
		\task $f(x)=\dfrac{x^3}{3}-2x^2+3x+1$ trên đoạn $[-3;2]$.
		\task $f(x) = - 2x^4 + 4x^2 + 3$ trên đoạn $\left[0;2\right]$
		\task $f(x)=\dfrac{2x+3}{x+1}$ trên đoạn $[0;4]$.
		\task $f(x)=x+\dfrac{4}{x}$ trên khoảng $(0;+\infty)$;
		\task $f(x)=3x+\dfrac{4}{x^2}$ trên $(0;+\infty)$.
		\task $f(x)=\dfrac{2x^2+4x+5}{x^2+1}$ trên $\mathbb{R}$.
		\task $f(x)=\sqrt{-x^2+2x}$ trên miền xác định.
	\end{tasks}
\loigiai{
\begin{enumerate}[a)]
	\item Hàm số liên tục trên $[-3;1]$. Ta có $f'(x)=-3x^2+6x$; $f'(x)=0 \,\Leftrightarrow \hoac{&x=0 \in [-3;1]\\&x=2 \notin [-3;1]}$.\\
	Ta có $f(-3)=64$, $f(0)=10$, $f(1)=12$. Suy ra, $\max\limits_{[-3;1]} f(x)=f(-3)=64$; $\min\limits_{[-3;1]} f(x)=f(0)=10$.
	
	\item Hàm số liên tục trên $[-3;2]$
	Ta có $f^{\prime}(x)=x^2-4x+3$; $f^{\prime}(x)=0\Leftrightarrow \heva{&x=1\\x=3\notin[-3;2]}$.\\
	$f(1)=\dfrac{7}{3}$, $f(3)=-35$, $f(2)=\dfrac{5}{3}$.\\
	Vậy 
	$\max\limits_{[-3;2]}f(x)=\dfrac{7}{3}$ và
	$\min\limits_{[-3;2]}f(x)=-35$.	
	\item Ta có $f'(x)=- 8x^3 + 8x =- 8x(x^2 - 1) =- 8x(x - 1)(x + 1)$.\\
	Xét $f(0) = 3$, $f(1) = 5$ và $f(2) =- 13$.\\
	Vậy 
	$\max\limits_{[0;2]}f(x)=5$ và
	$\min\limits_{[0;2]}f(x)=-13$.	
	\item Hàm số đã cho liên tục trên đoạn $[0;4]$.\\
	Ta có $y'=-\dfrac{1}{(x+1)^2} < 0$, $\forall x \in [0;4]$. Suy ra hàm số đã cho nghịch biến trên đoạn $[0;4]$.\\
	Vậy $\max\limits_{[0;4]} y = y(0) = 3$ và $\min\limits_{[0;4]} y = y(4) = \dfrac{11}{5}$.
	
	\item Xét hàm số $f(x)=x+\dfrac{4}{x}$ trên khoảng $(0;+\infty)$.\\
	Đạo hàm $f'(x)=1-\dfrac{4}{x^2}=\dfrac{x^2-4}{x^2}$.\\
	Cho $f'(x)=0 \Leftrightarrow x^2-4=0 \Leftrightarrow \hoac{&x=2\in(0;+\infty)\\&x=-2\notin(0;+\infty).}$\\
	Bảng biến thiên
	\begin{center}
		\begin{tikzpicture}[font=\footnotesize,thick,>=stealth]
			\tikzset{double style/.append style={double distance=1.75pt}}
			\tkzTabInit[nocadre=false,lgt=1.2,espcl=2.5,deltacl=0.6,lw=.5pt,color,colorL=green!50,colorV=green!50]
			{$x$ /0.6,$f'(x)$ /0.6,$f(x)$ /2}
			{$-\infty$,$-2$,$0$,$2$,$+\infty$}
			\tkzTabLine{,+,$0$,-,d,-,$0$,+,}
			\tkzTabVar{-/$-\infty$,+/$-4$,-D+/$-\infty$/$+\infty$,-/$4$,+/$+\infty$}
			%\draw[pattern={Lines[angle=60,distance=1.25mm]},pattern color=blue,thin] (N11)--(N31)--(N33)--(N13);
		\end{tikzpicture}
	\end{center}
	Căn cứ vào bảng biến thiên, ta có $\min\limits_{(0;+\infty)}f(x)=4$.
	
	\item Áp dụng bất đẳng thức Cauchy cho $3$ số dương, ta có
	$$y=\dfrac{3x}{2}+\dfrac{3x}{2}+\dfrac{4}{x^2} \geq 3\sqrt[3]{\dfrac{3x}{2}\cdot \dfrac{3x}{2}\cdot \dfrac{4}{x^2}}=3\sqrt[3]{9}.$$
	Đẳng thức xảy ra khi $\dfrac{3x}{2}=\dfrac{4}{x^2} \Leftrightarrow x=\dfrac{2}{\sqrt[3]{2}}=2\sqrt[3]{2}$.
	
	\item Tập xác định $\mathscr{D}= \mathbb{R}$.\\
	Ta có $y'= \dfrac{-4x^2-6x+4}{(x^2+1)^2}$, \; $y'=0 \Leftrightarrow -4x^2-6x+4=0\Leftrightarrow \hoac{&x=-2\\&x=\dfrac{1}{2}.}$
	\begin{center}
		\begin{tikzpicture}\tkzTabInit[nocadre=false,lgt=1.2,espcl=2.5,deltacl=0.6]
			{$x$ /1, $y'$ /0.6, $y$ /2.5}
			{$-\infty$,$-2$,$\dfrac{1}{2}$,$+\infty$}
			\tkzTabLine{,-,$0$,+,$0$,-,}
			\tkzTabVar{+/$2$,-/$1$,+/$6$,-/$2$}
		\end{tikzpicture}
	\end{center}
	Suy ra $M=6$ và $m=1$.
	
	\item Hàm số $f(x)=\sqrt{-x^2+2x}$ liên tục trên $[0;2]$.\\
	$f'(x)=\dfrac{1-x}{\sqrt{-x^2+2x}}$, $f'(x)=0 \Leftrightarrow x=1$.\\
	Ta có $f(0)=0$, $f(2)=0$, $f(1)=1$.\\
	Vậy $\displaystyle\max_{x\in [0;2]}f(x)=1$ và $\displaystyle\min_{x\in [0;2]}f(x)=0$.
\end{enumerate}}
\end{vd}
\dongcham{54}
\begin{vd}
	Tìm giá trị lớn nhất và nhỏ nhất của hàm số sau trên miền đã chỉ ra.
	\begin{tasks}(2)
		\task $y=x-\sin 2x$ trên đoạn $\left[-\dfrac{\pi}{2};\pi\right]$
		\task $y = \mathrm{e}^{x^3 - 3x + 3}$ trên đoạn $[0; 2]$
		\task $y=\mathrm{e}^{x}(x^{2}-3)$ trên đoạn $[-2;2]$
		\task $y=\dfrac{\ln^2x}{x}$ trên đoạn $\left[1;\mathrm{e}^5\right]$
	\end{tasks}
	\loigiai{
		\begin{enumerate}[a)]
			\item Ta có
			\begin{itemize}
				\item $y'=1-2\cos 2x$.
				\item $\heva{&x\in \left(-\dfrac{\pi}{2};\pi\right)\\& y'=0}\Leftrightarrow \heva{&x\in \left(-\dfrac{\pi}{2};\pi\right)\\&\cos 2x=\dfrac{1}{2}}\Leftrightarrow \heva{&x\in \left(-\dfrac{\pi}{2};\pi\right)\\&x=\pm \dfrac{\pi}{6}+k\pi} \Leftrightarrow \hoac {&x=\pm\dfrac{\pi}{6}\\& x=\dfrac{5\pi}{6}.}$
				\item $f\left(-\dfrac{\pi}{2}\right)=-\dfrac{\pi}{2}$,  $f(\pi)=\pi$,
				$f\left(-\dfrac{\pi}{6}\right)=-\dfrac{\pi}{6}+\dfrac{\sqrt{3}}{2}$,  $f\left(\dfrac{\pi}{6}\right)=\dfrac{\pi}{6}-\dfrac{\sqrt{3}}{2}$,  $f\left(\dfrac{5\pi}{2}\right)=\dfrac{5\pi}{6}+\dfrac{\sqrt{3}}{2}$.
			\end{itemize}
			Vậy giá trị lớn nhất và giá trị nhỏ nhất của hàm số $y=x-\sin 2x$ trên đoạn $\left[-\dfrac{\pi}{2};\pi\right]$ lần lượt là $\dfrac{5\pi+3\sqrt{3}}{6}$ và $-\dfrac{\pi}{2}$.
			\item Ta có $y' = (3x^2 - 3)\cdot \mathrm{e}^{x^3 - 3x + 3}$.\\
			$y' = 0 \Leftrightarrow 3x^2 - 3 = 0 \Leftrightarrow x = 1$ do $x \in [0;2]$.\\
			Khi đó $y(0) = \mathrm{e}^3$; $y(2) = \mathrm{e}^5$; $y(1) = \mathrm{e}$.
			Vậy $ \max \limits_{[0; 2]} y = \mathrm{e}^5 $ khi $x = 2$.
			\item Ta có $y'=\mathrm{e}^x(x^2+2x-3)=0\Leftrightarrow \hoac{&x=-3\\ &x=1}$.
			Xét các giá trị: $f(-2)=\mathrm{e}^{-2}$; $f(1)=-2\mathrm{e}$; $f(2)=\mathrm{e}^2$, từ đó suy ra $y_{\min}=-2\mathrm{e}$.
			\item $y'=\dfrac{2\ln x-\ln^2x}{x^2}$, $y'=0\Leftrightarrow \hoac{
				& \ln x=0 \\
				& \ln x=2 \\} \Leftrightarrow \hoac{
				& x=1 \\
				& x=\mathrm{e}^2. \\}$\\
			Tính $y(1)=0$, $y(\mathrm{e}^2)=\dfrac{4}{\mathrm{e}^2}\approx 0{,}54$, $y(\mathrm{e}^5)=\dfrac{9}{\mathrm{e}^5}\approx 0{,}16$.\\
			Vậy $\max\limits_{x \in \left[1;\mathrm{e}^5\right]}y=\dfrac{4}{\mathrm{e}^2}$
	\end{enumerate}}
\end{vd}
\dongcham{40}
\begin{vd}
	Tìm giá trị lớn nhất và nhỏ nhất (nếu có) của hàm số sau trên miền đã chỉ ra.
	\begin{tasks}(2)
		\task $f(x)=\dfrac{5\sin x+1}{\sin x+2}$ trên đoạn $\left[0;\dfrac{\pi}{6}\right]$.
		\task $ y=\cos^3x +2\sin^2x+\cos x$ trên miền xác định.
	\end{tasks}
	\loigiai{
		\begin{enumerate}[a)]
			\item Đặt $t=\sin x,\; x\in \left[0;\dfrac{\pi}{6}\right]\Rightarrow t \in \left[0;\dfrac{1}{2}\right]$.\\
			Ta được hàm số $y=g(t)=\dfrac{5t+1}{t+2}$.\\
			$g'(t)=\dfrac{9}{(t+2)^2}>0,\forall t \in \left[0;\dfrac{1}{2}\right]$.\\
			Vì $g(0)=\dfrac{1}{2}$, $g\left(\dfrac{1}{2}\right)=\dfrac{7}{5}$ nên $\min\limits_{\left[0;\tfrac{1}{2}\right]}g(t)=g(0)=\dfrac{1}{2}.$\\
			Vậy $\min\limits_{\left[0;\tfrac{\pi}{6}\right]}f(x)=\min\limits_{\left[0;\tfrac{1}{2}\right]}g(t)=\dfrac{1}{2}$
			\item Ta có $ y=\cos^3x +2\sin^2x+\cos x =\cos^3x +2(1-\cos^2x)+\cos x =\cos^3x-2\cos^2x+\cos x+2$.\\
			Đặt $ t=\cos x,\, t\in [-1;1] $. Ta được $ f(t)=t^3-2t^2 +t+2$.\\
			Ta có $ f'(t)=3t^2-4t+1;\,y'=0\Leftrightarrow \hoac{&t=1\in[-1;1]\\&t=\dfrac{1}{3}\in[-1;1].} $\\
			Mà $f\left(-1\right)=-2$, $ f\left(\dfrac{1}{3}\right)=\dfrac{58}{27} $, $f(1)=2$ nên $\max\limits_{x\in \mathbb{R}}y=\max\limits_{\left[-1;1\right]}f(t)=\dfrac{58}{27}$
			\item
			\item
	\end{enumerate}}
\end{vd}
\dongcham{43}
\boxmini{BÀI TẬP TRẮC NGHIỆM}
\ind{PHẦN I.} \inden{Câu trắc nghiệm nhiều phương án lựa chọn. Mỗi câu hỏi học sinh chỉ chọn một phương án.}\\
\setcounter{ex}{0}
\Opensolutionfile{ans}[ans/2D1-B2-d1-1]

\begin{ex}
	\immini
	{Hàm số $y=f(x)$ liên tục trên đoạn $[-1;3]$ và có bảng biến thiên như sau.\\
		Gọi $M$ là giá trị lớn nhất của hàm số $y=f(x)$ trên đoạn $[-1;3]$. Khẳng định nào sau đây là khẳng định đúng?
		\choice
		{\True $M=f(0)$}
		{$M=f(-1)$}
		{$M=f(3)$}
		{$M=f(2)$}
	}
	{\begin{tikzpicture}
			\tkzTabInit[nocadre=True,lgt=1.2,espcl=2]
			{$x$ /0.7,$y'$ /0.7,$y$ /2.1}
			{$-1$,$0$,$2$,$3$}
			\tkzTabLine{,+,$0$,-,$0$,+,}
			\tkzTabVar{-/$0$, +/$5$,-/$1$,+/$4$}
	\end{tikzpicture}}
	\loigiai
	{Dựa vào bảng biến thiên ta có $M=f(0)=5$.}
\end{ex} \dongcham{1}

\begin{ex}
	\immini{Cho hàm số $f(x)$ liên tục trên đoạn $[-1;5]$ và có đồ thị như hình vẽ bên. Gọi $M$ và $m$ lần lượt là giá trị lớn nhất và nhỏ nhất của hàm số đã cho trên $[-1;5]$. Giá trị của $M+m$ bằng
		\choice
		{$5$}
		{$6$}
		{$3$}
		{\True $1$}
	}{
		\begin{tikzpicture}[scale=0.65, font=\footnotesize, line join=round, line cap=round, >=stealth]
			%%Nhập giới hạn đồ thị và hàm số cần vẽ
			\def \xmin{-1.5}
			\def \xmax{6.3}
			\def \ymin{-2.8}
			\def \ymax{4}
			%%Tự động
			\draw[->] (\xmin,0)--(\xmax,0) node[below left] {$x$};
			\draw[->] (0,\ymin)--(0,\ymax) node[below left] {$y$};
			\draw[fill=black] (0,0) circle(1pt) node [below right] {$O$};
			%%Vẽ các điểm trên 2 hệ trục
			\foreach \x in {3,4,5}
			\draw[fill=black] (\x,0) circle(1pt) node [below] {$\x$};
			\foreach \x in {-1,2}
			\draw[fill=black] (\x,0) circle(1pt) node [above] {$\x$};
			\foreach \y in {-2,1,3}
			\draw[fill=black] (0,\y) circle(1pt) node [above right] {$\y$};
			\draw[dashed](-1,0)--(-1,-2)--(0,-2)--(2,-2)--(2,0) (5,0)--(5,1)--(0,1) (3,0)--(3,1) (4,0)--(4,3)--(0,3);
			%%Tự động
			\draw
			(-1.1,-2.7) to[out=80, in=-100] (-1,-2)
			..controls +(80:1.2) and +(180:.5)..(0,1)
			..controls +(0:.6) and +(180:0.7)..(2,-2)
			..controls +(0:0.4) and +(-100:1.2)..(2.8,0)
			to[in=80, out=-100] (3,1)
			..controls +(75:1.5) and +(180:0.3)..(4,3)
			..controls +(0:0.5) and +(-75:1)..(5,1)
			to[in=105, out=-75] (6,-2.7);
			\fill[black]
			(-1,-2) circle(1pt)
			(2,-2) circle(1pt)
			(3,1) circle(1pt)
			(4,3) circle(1pt)
			(5,1) circle(1pt)
			;
		\end{tikzpicture}
	}
	\loigiai{
		Dựa vào đồ thị, suy ra $m=\min\limits_{[-1;5]} f(x)=f(-1)=-2$, $M=\max\limits_{[-1;5]} f(x)=f(4)=3$.\\
		Do đó $M+m=3-2=1$.
	}
\end{ex} \dongcham{1}

\begin{ex}
	\immini{Cho hàm số $y=f(x)$ có đồ thị là đường cong ở hình bên. Tìm giá trị nhỏ nhất $m$ của hàm số $y=f(x)$ trên đoạn $[-1;1] $.
		\haicot
		{$m=2 $}
		{\True $m=-2 $}
		{$m=1 $}
		{$m=-1 $}}{\vspace{-0.5cm}
		\begin{tikzpicture}[smooth,samples=300,scale=0.68,>=stealth]
			\draw[->] (-2.3,0)--(2.3,0) node[below]{$x$};
			\draw[->] (0,-2.5)--(0,2.5) node[right]{$y$};
			\draw (0,0) node[above right]{$O$};
			\draw[line width = 1pt,domain=-2:2] plot(\x,{(\x)^(3)-3*(\x)});
			\draw[fill=black] (-1,2) circle(1.5pt) (1,-2) circle(1.5pt);
			\draw[dashed] (-1,0)node[below]{\small$-1$}--(-1,2)--(0,2)node[right]{\small$2$} (1,0)node[above]{\small$1$}--(1,-2)--(0,-2)node[left]{\small$-2$};
	\end{tikzpicture}}
	\loigiai{
		Dựa vào đồ thị ta có giá trị nhỏ nhất của hàm số trên đoạn $[-1;1] $ bằng $-2$.	
		
	}
	
\end{ex} \dongcham{1}

\begin{ex}
	Cho hàm số $y=f(x)$ có bảng biến thiên trên đoạn $[-2;3]$ như hình bên dưới.
	\begin{center}
		\begin{tikzpicture}[>=stealth,scale=1]
			\tkzTabInit[nocadre=false,lgt=1.2,espcl=2,deltacl=0.6]
			{$x$/0.6,$f’(x)$/0.6,$f(x)$/2}
			{$-\infty$,$-2$,$-1$,$1$,$3$,$+\infty$}
			\tkzTabLine{,h,,+,z,-,d,+,,h}
			\tkzTabVar{+H/,-/$0$,+/$1$,-/$-2$,+H/$5$}
		\end{tikzpicture}
	\end{center}
	Gọi $M$ và $m$ lần lượt là giá trị lớn nhất và giá trị nhỏ nhất của hàm số đã cho trên đoạn $[-1;3]$. Giá trị của biểu thức $M-m$ là
	\choice
	{$5$}
	{\True $7$}
	{$-1$}
	{$3$}
	\loigiai{
		Dựa vào bảng biến thiên ta thấy giá trị lớn nhất của hàm số là $M=5$ và giá trị nhỏ nhất của hàm số là $m=-2$ nên $M-m=7$.}
\end{ex}
% \newpage
\begin{ex}
	Giá trị lớn nhất và nhỏ nhất của hàm số $y=x^3-12x+1$ trên đoạn $[-2;3]$ lần lượt là
	\choice
	{\True $17$, $-15$}
	{$10$, $-26$}
	{$-15$, $17$}
	{$6$, $-26$}
	\loigiai{
		Ta có $y'=3x^2-12$, do đó $y'=0\Leftrightarrow 3x^2-12=0\Leftrightarrow x=\pm 2\in [-2;3]$.\\
		Mặt khác $f(-2)=17$, $f(2)=-15$, $f(3)=-8$.\\
		Vậy giá trị lớn nhất và nhỏ nhất cần tìm lần lượt là $17$, $-15$.
	}
\end{ex} \dongcham{12}

\begin{ex}
	Gọi $ M, m $ lần lượt là giá trị lớn nhất và giá trị nhỏ nhất của hàm số $ y = x^3 + 3x^2 - 9x + 1 $ trên $ [-4;4] $. Tính tổng $ M + m. $
	\choice 
	{$ 12 $}
	{$ 98 $}
	{$ 17 $}
	{\True $ 73 $}
	\loigiai
	{
		Ta có $ y' = 3x^2 + 6x - 9 = 0 \Leftrightarrow \hoac{&x = 1\\ &x = -3.} $\\
		Khi đó: $ y(-4) = 21 $,\, $ y(-3) = 28, $
		\, $ y(1) = -4, $
		\, $ y(4) = 77. $\\
		Do đó $ M + m = 77 + (-4) = 73. $
	}
\end{ex} \dongcham{12}

\begin{ex}
	Giá trị lớn nhất của hàm số $f(x)=-x^4+12x^2+1$ trên đoạn $\left[ -1;2\right] $ bằng
	\choice
	{\True $33$}
	{$37$}
	{$12$}
	{$1$}
	\loigiai{
		Hàm số $f(x)=-x^4+12x^2+1$ liên tục trên đoạn $\left[ -1;2\right] $.\\
		Ta có $f'(x)=-4x^3+24x=-4x(x^2-6)$.\\
		$f'(x)=0 \Leftrightarrow \hoac{& x=-\sqrt{6} \not \in \left[ -1;2\right] \\ &x=0 \in \left[ -1;2\right] \\&x=\sqrt{6} \not \in \left[ -1;2\right]. }$\\
		Ta có $f(-1)=12; f(0)=1; f(2)=33$.\\
		Vậy $\max\limits_{\left[ -1;2\right] } f(x)=33$.
	}
\end{ex} \dongcham{12}
\begin{ex}
	Giá trị lớn nhất của hàm số $y=x^4-3x^2+2$ trên đoạn $\left[ 0;3\right] $ bằng
	\choice
	{ $ 57 $}
	{\True $ 56 $}
	{$ 54$}
	{$ 55 $}
	\loigiai{
		Hàm số $y$ liên tục trên đoạn $\left[ 0;3\right] $ và có đạo hàm $y'=4x^3-6x$.\\
		Ta có $y'=0 \Leftrightarrow 4x^3-6x=0 \Leftrightarrow \hoac{& x=0 \in \left[ 0;3\right]  \\&x=\sqrt{\dfrac{3}{2}} \in \left[ 0;3\right]\\ & x=- \sqrt{\dfrac{3}{2}}\notin \left[ 0;3\right].}$\\
		Ta có $y(0)=2$, $y(3)=56$, $y\left(\sqrt{\dfrac{3}{2}}\right) =-\dfrac{1}{4} $.\\
		Do đó giá trị lớn nhất của hàm số $y=x^4-3x^2+2$ trên đoạn $\left[ 0;3\right] $ bằng $56$.
	}
\end{ex} \dongcham{7}

\begin{ex}%[2D1Y3-1]
	Giá trị nhỏ nhất của hàm số $y=\dfrac{x-1}{x+1}$ trên đoạn $[0;3]$ là
	\choice
	{$\min\limits_{[0;3]}y=\dfrac{1}{2}$}
	{$\min\limits_{[0;3]}y=-3$}
	{$\min\limits_{[0;3]}y=1$}
	{\True $\min\limits_{[0;3]}y=-1$}	
	\loigiai{
		Trên đoạn $[0;3]$ hàm số luôn xác định.\\
		Ta có $y'=\dfrac{2}{(x+1)^2}>0$, $\forall x \in [0;3]$ nên hàm số đã cho đồng biến trên đoạn $[0;3]$.\\
		Do đó $\min\limits_{[0;3]}y=y(0)=-1$.
	}	
\end{ex} \dongcham{7}

\begin{ex}%
	Giá trị nhỏ nhất của hàm số $y=\dfrac{2x+3}{x+1}$ trên đoạn $[0;4]$ là
	\choice
	{$2$}
	{$\dfrac{7}{5}$}
	{$3$}
	{\True $\dfrac{11}{5}$}
	\loigiai
	{
		Ta có $y'=\dfrac{-1}{(x+1)^2}<0$ nên $\min\limits_{[0;4]} y=y(4)=\dfrac{11}{5}$.
	}
\end{ex} \dongcham{7}

\begin{ex}%[2D1B3]
	Giá trị lớn nhất của hàm số $y=\dfrac{x^2-3x+3}{x-1}$ trên đoạn $\left[-2;\dfrac{1}{2}\right]$ bằng
	\choice
	{$4$}
	{\True $-3$}
	{$-\dfrac{7}{2}$}
	{$-\dfrac{13}{3}$}
	\loigiai{
		Ta có $y'=\dfrac{x^2-2x}{(x-1)^2}$. Xét $y'=0\Leftrightarrow  x^2-2x=0\Leftrightarrow \hoac{&x=0\in \left[-2;\dfrac{1}{2}\right]\\&x=2\notin\left[-2;\dfrac{1}{2}\right]}$.\\
		Ta có $y(0)=-3$, $y(-2)=\dfrac{-13}{3}$, $y\left(\dfrac{1}{2}\right)=\dfrac{-7}{2}$.\\
		Suy ra $\underset{x\in \left[-2;\dfrac{1}{2}\right]}{\max y}=-3$
	}
\end{ex} \dongcham{10}

\begin{ex}
	Giá trị lớn nhất của hàm số $y=\sqrt{4-x^2}$ là
	\choice
	{$M=-2$}
	{\True $M=2$ }
	{$M=4$}
	{$M=0$}
	\loigiai
	{
		Tập xác định: $\mathscr{D}=\left[-2;2\right]$.\\
		Đạo hàm $y'=\dfrac{-x}{\sqrt{4-x^{2}}}$; $y'=0 \Leftrightarrow x=0 \in \left[-2;2\right]$.\\
		Ta có $y(2)=y(-2)=0$; $y(0)=2$.\\
		Vậy giá trị lớn nhất của hàm số đã cho bằng $2$.
	}
\end{ex} \dongcham{8}

\begin{ex}%[2D1B3-1]
	Tìm giá trị lớn nhất $M$ của hàm số $y=\sqrt{7+6x-x^2}$.
	\choice
	{\True $M=4$}
	{$M=\sqrt{7}$}
	{$M=7$}
	{$M=3$}
	\loigiai{
		Tập xác định $\mathscr{D}=[-1;7]$.\\
		$y'=\dfrac{-x+3}{\sqrt{7+6x-x^2}}$.\\
		Cho $y'=0\Leftrightarrow x= 3\in \mathscr{D}$.\\
		Có $y(3)=4, y(-1)=0, y(7)=0$. Vậy $M=4$.	
	}
\end{ex} \dongcham{8}

\begin{ex}%[Nguyễn Quang Hiệp - Phát triển đề minh họa 2021]%[2D2B4-4]%
	Tính giá trị lớn nhất của hàm số $y=x-\ln x$ trên $\left[\dfrac{1}{2};\mathrm{e}\right]$.\\
	\choice
	{$\max\limits_{x \in \left[\frac{1}{2};\mathrm{e}\right]}y=1$}
	{\True $\max\limits_{x \in \left[\frac{1}{2};\mathrm{e}\right]}y=\mathrm{e}-1$}
	{$\max\limits_{x \in \left[\frac{1}{2};\mathrm{e}\right]}y=\mathrm{e}$}
	{$\max\limits_{x \in \left[\frac{1}{2};\mathrm{e}\right]}y=\dfrac{1}{2}+\ln 2$}
	\loigiai{
		Hàm số $y=x-\ln x$liên tục trên đoạn $\left[\dfrac{1}{2};\mathrm{e}\right]$.\\
		Ta có $y'=1-\dfrac{1}{x}\Rightarrow y'=0\Leftrightarrow x=1\in \left[\dfrac{1}{2};\mathrm{e}\right]$.\\
		Do $y\left(\dfrac{1}{2}\right)=\dfrac{1}{2}+\ln 2$; $y(\mathrm{e})=\mathrm{e}-1$; $y(1)=1$ nên $\max\limits_{x \in \left[\frac{1}{2};\mathrm{e}\right]}y=\mathrm{e}-1$.}
\end{ex} \dongcham{8}

\begin{ex}
	Gọi $M, N$ lần lượt là giá trị lớn nhất và nhỏ nhất của hàm số $y = x^2 - 4\ln (1 - x)$ trên đoạn $[-2;0]$. Tính $M - N$.
	\choice
	{$M - N = 4\ln 2$}
	{$M - N = -1$}
	{\True $M - N = 4\ln 2 -1$}
	{$M - N = 4\ln 3 -4$}
	\loigiai{
		Tập xác định: $\mathscr{D} = (-\infty;1)$.\\
		Ta có $y' = 2x + \dfrac{4}{1 - x} = \dfrac{-2x^2 + 2x + 4}{1 - x}$.\\
		Khi đó $y' = 0 \Leftrightarrow -2x^2 + 2x + 4 = 0 \Leftrightarrow \hoac{&x = -1 \quad \mbox{(nhận)} \\&x = 2 \quad \mbox{(loại)}. }$\\
		Khi đó $\heva{& y(-2) = 4 - 4\ln 3 \approx -0{,}4 \\& y(-1) = 1 - 4\ln 2  \approx -1{,}7\\& y(0) = 0.} \Rightarrow M = 0, N = 1 -4\ln 2$\\
		Vậy $M - N = 4\ln 2 -1$.
	}
\end{ex} \dongcham{8}

\begin{ex}
	Cho hàm số $f(x)$ nghịch biến trên $\mathbb{R}$. Giá trị nhỏ nhất của hàm số $g(x)=\mathrm{e}^{3x^2-2x^3}-f(x)$ trên đoạn $[0;1]$ bằng
	\choice
	{$\mathrm{e}-f(1)$}
	{$f(1)$}
	{$f(0)$}
	{\True $1-f(0)$}
	\loigiai{
		Ta có $g'(x)=(6x-6x^2)\mathrm{e}^{3x^2-2x^3}-f'(x)$.\\
		Trên đoạn $[0;1]$ thì $6x-6x^2\ge 0$, $f'(x)\le 0$ nên $g'(x)\ge 0$, suy ra hàm số $g(x)$ đồng biến, suy ra giá trị nhỏ nhất là $g(0)=1-f(0)$.
	}
\end{ex} \dongcham{8}

\begin{ex}
	\immini{Cho hàm số $y=f(x)$ xác định và liên tục trên đoạn $\left[0;\dfrac{7}{2}\right]$, có
		đồ thị của hàm số $y=f'(x)$ như hình vẽ. Hỏi hàm số $y=f(x)$ đạt giá trị nhỏ nhất trên đoạn $\left[0;\dfrac{7}{2}\right]$ tại điểm $x_0$ nào dưới đây?
		\choice
		{\True $x_0=3$}
		{$x_0=2$}
		{$x_0=1$}
		{$x_0=0$}
	}
	{\begin{tikzpicture}[scale=1,font=\footnotesize, line join=round,line cap=round,>=stealth]
			\draw [->] (-1,0)--(0,0)
			node[below left]{$O$}--(4.5,0)node[below]{$x$}; % Hệ trục tọa độ
			\draw[->] (0,-1.5) --(0,2) node[left]{$y$};
			\draw[dashed](3.5,0)node[below]{$\tfrac{7}{2}$}--(3.5,25/16);
			\draw(1,0)node[above]{$1$}(3,0)node[above left]{$3$};
			\draw [domain=0:3.5,samples=100] plot (\x, {(\x-1)^2*(\x-3)/2});
	\end{tikzpicture}}
	\loigiai{
		Từ đồ thị hàm số ta có $f'(x)=0 \Leftrightarrow \hoac{&x=1\\&x=3.}$\\
		Bảng biến thiên của hàm số $y=f(x)$ trên đoạn $\left[0;\dfrac{7}{2}\right]$
		\begin{center}
			\begin{tikzpicture}
				\tkzTabInit[nocadre=false,lgt=1.2,espcl=2.5,deltacl=0.7]{$x$ / 1.1 , $f’(x)$ /0.7, $f(x)$ / 2}
				{$0$,$1$, $3$ , $\dfrac{7}{2}$}%
				\tkzTabLine{,-,0,-,0,+,}%
				\tkzTabVar{+ /$f(0)$,R/,-/$f(3)$,+ / $f\left(\dfrac{7}{2}\right)$}%
				\tkzTabIma{1}{3}{2}{$f(0)$}
			\end{tikzpicture}
		\end{center}
		Từ bảng biến thiên ta có hàm số $y=f(x)$ đạt giá trị nhỏ nhất trên đoạn $\left[0;\dfrac{7}{2}\right]$ tại điểm $x_0=3.$
	}
\end{ex} \dongcham{10}

\begin{ex}%[2D1K3-1]
	\immini{Cho hàm số $y=f(x)$, biết hàm số $y=f'(x)$ có đồ thị như hình vẽ dưới đây. Hàm số $y=f(x)$ đạt giá trị nhỏ nhất trên đoạn $\left[\dfrac{1}{2};\dfrac{3}{2} \right]$ tại điểm nào sau đây?
		\choicew{0,25 \textwidth}
		\choice
		{$x=\dfrac{3}{2}$}
		{$x=\dfrac{1}{2}$}
		{\True $x=1$}
		{$x=0$}}{\vspace{-0.5cm}\begin{tikzpicture}[>=stealth,scale=1.5]
			\draw[->] (-0.5,0)--(2.5,0) node[below]{\footnotesize $x$};
			\draw[->] (0,-0.5)--(0,1.5) node[right]{\footnotesize $y$};
			\draw (0,0) node[below left]{\footnotesize $O$};
			\draw[line width = 1pt,smooth,domain=-0.4:1.7] plot({\x},{(\x)^2-\x});
			\draw[fill=black] (1.5,0.75) circle(1pt);
			\draw [dashed] (1.5,0)
			node[below]{\footnotesize$\dfrac{3}{2}$}--(1.5,0.75)--(0,0.75)(1,0)node[below]{$1$};
	\end{tikzpicture}}
	\loigiai{
		Dựa vào đồ thị hàm số $y=f'(x)$. Ta có bảng biến thiên
		\begin{center}\begin{tikzpicture}
				\tkzTabInit[nocadre=false,lgt=1,espcl=2.1]
				{$x$ /1,$y'$ /0.6,$y$ /2}
				{$\dfrac{1}{2}$,$1$,$\dfrac{3}{2}$}
				\tkzTabLine{,-,$0$,+,$0$,}
				\tkzTabVar{+/, -/,+/}
			\end{tikzpicture}
		\end{center}
		Suy ra hàm số đạt giá trị nhỏ nhất trên $\left[\dfrac{1}{2};\dfrac{3}{2} \right]$ tại $x=1$.}
\end{ex} \dongcham{10}

\begin{ex}
	\immini{ Cho hàm số $f(x)$ có đồ thị của hàm số $y=f'(x)$ như hình vẽ. Biết $f(0)+f(1)-2f(2)=f(4)-f(3)$. Giá trị nhỏ nhất $m$, giá trị lớn nhất $M$ của hàm số $f(x)$ trên đoạn $[0;4]$ là
		\choice
		{$m=f(4)$, $M=f(1)$}
		{\True $m=f(4)$, $M=f(2)$}
		{$m=f(1)$, $M=f(2)$}
		{$m=f(0)$, $M=f(2)$}
	}{
		\begin{tikzpicture}[scale=0.79, >=stealth]
			\draw[->] (-0.6,0.) -- (5.3,0.);
			\draw[->] (0.,-1.7) -- (0.,1.6);
			\draw[dashed] (4,0) -- (4,-1.2);
			\clip(-0.6,-1.7) rectangle (5.3,1.7);
			\draw[smooth,samples=100,domain=0:2] plot(\x,{-0.8*((\x)^2-2*(\x))});
			\draw[smooth,samples=100,domain=2:4.5] plot(\x,{0.2*(((\x)-2)*((\x)-7)});
			\draw (-0.3,-0.3) node {$O$} (5.2,0.3) node {$x$} (0.35,1.5) node {$y$} (1.9,-0.3) node {$2$} (4.0,0.3) node {$4$} (2.0,1.15) node {$y=f'(x)$};
			\fill (0,0) circle(1pt) (2,0) circle(1pt) (4,0) circle(1pt); 
		\end{tikzpicture}
	}
	\loigiai{
		Từ đồ thị hàm số $y=f'(x)$ ta suy ra $f'(x)=0 \Leftrightarrow \hoac{&x=0\\&x=2.}$\\
		Ta có bảng biến thiên: 
		\begin{center}\begin{tikzpicture}[>=stealth,scale=1]
				\tkzTabInit[lgt=1.2,espcl=2.5]
				{$x$/1,$f'(x)$/1,$f(x)$/2.5}
				{$0$,$2$,$4$}
				\tkzTabLine{$0$,+,$0$,- }
				\tkzTabVar{-/$f(0)$,+/$f(2)$,-/$f(4)$}
		\end{tikzpicture}\end{center}
		Từ bảng biến thiên ta thấy $M=f(2)$.\\
		Mặt khác, từ bảng biến thiên ta có $\heva{&f(1)<f(2)\\&f(3)<f(2)}\Rightarrow f(1)+f(3)<2f(2)$.\\
		Do đó $f(4)=f(0)+f(1)+f(3)-2f(2)<f(0)+f(2)+f(2)-2f(2)=f(0) \Rightarrow m=f(4)$.	
	}
\end{ex} \dongcham{10}


\begin{ex}
	Giá trị lớn nhất, giá trị nhỏ nhất của hàm số $y={\sin}^3x-3{\sin}^2x+2$ lần lượt là $M$, $m$. Tổng $M+m$ bằng
	\choice
	{\True $0$}
	{$4$}
	{$1$}
	{$3$}
	\loigiai{
		Đặt $t=\sin x \, (-1\le t\le 1)$. Ta có $y=f(t)=t^3-3t^2+2 \, (-1\le t\le 1)$.\\$y'=3t^2-6t=0\Leftrightarrow\hoac{&t=0\in \left[-1;1\right]\\&t=2\notin \left[-1;1\right].}$\\
		Ta có $f(-1)=-2,\,f(1)=0, \,f(0)=2$. Vậy $M=2$ và $m=-2\Rightarrow M+m=0$.}
\end{ex} \dongcham{11}

\begin{ex}
	Giá trị nhỏ nhất của hàm số $f(x)=(x+1)(x+2)(x+3)(x+4)+2019$ là
	\choice
	{$2017$}
	{$2020$}
	{\True $2018$}
	{$2019$}
	\loigiai{
		Tập xác định $\mathscr{D}=\mathbb{R}$.\\
		Biến đổi $f(x)=(x+1)(x+2)(x+3)(x+4)+2019=(x^2+5x+4)(x^2+5x+6)+2019$.\\
		Đặt $t=x^2+5x+4\Rightarrow t=\left( x+\dfrac{5}{2} \right)^2-\dfrac{9}{4}\Rightarrow t\ge-\dfrac{9}{4}\,\forall x\,\in \,\mathbb{R}$.\\
		Hàm số đã cho trở thành $f(x)=t^2+2t+2019=(t+1)^2+2018\ge 2018 \,\forall t\ge -\dfrac{9}{4}$.\\
		Vậy giá trị nhỏ nhất của hàm số đã cho bằng $2018$ tại $t=-1\in \left[-\dfrac{9}{4};+\infty \right)$.
	}
\end{ex} \dongcham{11}

\Closesolutionfile{ans}

\ind{PHẦN II.} \inden{Câu trắc nghiệm đúng sai. Trong mỗi ý a), b), c), d) ở mỗi câu, học sinh chọn đúng hoặc sai.}\\
\Opensolutionfile{ans}[ans/2D1-B2-d1-2]

\begin{ex}%[2D1Y3]
		Cho hàm số $y=f(x)$ là hàm số liên tục trên $\mathbb{R}$ và có bảng biến thiên như hình vẽ dưới đây. 
		\begin{center}
			\begin{tikzpicture}
				\tkzTabInit[nocadre=false, lgt=1.2, espcl=1.5]{$x$ /0.6,$f'(x)$ /0.6,$f(x)$ /1.7}{$-\infty$,$-1$,$0$,$1$,$+\infty$}
				\tkzTabLine{,+,$0$,-,$0$,+,$0$,-,}
				\tkzTabVar{-/ $-\infty$ ,+/$4$,-/$3$,+/$4$,-/$-\infty$}
			\end{tikzpicture}
		\end{center}
	Xét tính đúng, sai của các khẳng định sau:
		\choiceTF
		{\True Cực đại của hàm số là $4$}
		{\True Cực tiểu của hàm số là $3$}
		{\True $\max\limits_{\mathbb{R}}{y}=4$}
		{$\min\limits_{\mathbb{R}}{y}=3$}
	\loigiai{
		Tử bảng biến thiên ta thấy $\lim\limits_{x\to+\infty}f(x)=-\infty$ nên hàm số không có giá trị nhỏ nhất trên $\mathbb{R}.$}
\end{ex} \dongcham{8} 

\begin{ex}
	Hình bên cho biết sự thay đổi của nhiệt độ ở một thành phố trong một ngày. Xét tính đúng, sai của các khẳng định sau:
	\begin{center}
		\begin{tikzpicture}[>=stealth,x=0.25cm,y=0.15cm]
			\draw[->] (-2,0)--(0,0) node[below left]{$O$}--(28,0) node[below right]{$x$ (giờ)};
			\draw[->] (0,-4)--(0,40) node[left]{$t$ ($^\circ C$)};
			\foreach \x/\g in {4/-90,8/-90,12/-90,16/-90,20/-90,24/-90}
			\draw[thin] (\x,2pt)--(\x,-2pt) + (\g:3mm) node {$\x$};
			%Vẽ các điểm trên trục Oy
			\foreach \y/\g in {25/180}
			\draw[thin] (2pt,\y)--(-2pt,\y) + (\g:3mm) node {$\y$};
			\path
			(0,25) coordinate (25)
			(4,20) coordinate (20)
			(8,31) coordinate (31)
			(12,28) coordinate (28)
			(16,34) coordinate (34)
			(20,27) coordinate (27)
			(24,24) coordinate (24);
			\draw [dashed] (4,0)--(4,20) (8,0)--(8,31) (12,0)--(12,28) (16,0)--(16,34) (20,0)--(20,27) (24,0)--(24,24); 
			\draw[smooth, thick, red]
			(25) .. controls +(-10:1) and +(-180:1) .. (20)
			(20) .. controls +(0:1) and +(-180:1) .. (31)
			(31) .. controls +(0:1) and +(160:1) .. (28)
			(28) .. controls +(0:1) and +(-180:2) .. (34)
			(34) .. controls +(0:1.5) and +(130:1.5) .. (27)
			(27) .. controls +(-60:1.5) and +(-180:2) .. (24);
			\foreach \x in {20,31,28,34,27,24}
			\fill (\x) +(90:3mm) node {$\x$};
		\end{tikzpicture}
	\end{center}
		\choiceTF
		{Nhiệt độ cao nhất trong ngày là $28^{\circ} \mathrm{C}$}
		{\True Nhiệt độ thấp nhất trong ngày là $20^{\circ} \mathrm{C}$}
		{\True Thời điểm có nhiệt độ cao nhất trong ngày là lúc $16$ giờ}
		{\True Thời điểm có nhiệt độ thấp nhất trong ngày là lúc $4$ giờ}
	\loigiai{}
\end{ex} \dongcham{8}

\begin{ex}
	Cho hàm số $y=f\left(x\right)$ có đạo hàm $y=f'\left(x\right)$ liên tục trên $\mathbb{R}$ và đồ thị của hàm số $f'\left(x\right)$ trên đoạn $\left[-2;6\right]$ như hình vẽ bên. 	Xét tính đúng, sai của các khẳng định sau:
	\begin{center}
		\begin{tikzpicture}[line join=round, line cap=round,>=stealth,scale=.7]
			\def\xmin{-3}\def\xmax{6.5}\def\ymin{-1}\def\ymax{3.5}
			\draw[->] (\xmin-0.2,0)--(\xmax+0.2,0) node[below] {\small $x$};
			\draw[->] (0,\ymin-0.2)--(0,\ymax+0.2) node[right] {\small $y$};
			\draw (0,0) node [below left] {\footnotesize $O$};
			\foreach \x in {-2,-1,2,6}\draw (\x,0.05)--(\x,-0.05) node [below] {\scriptsize $\x$};
			\foreach \y in {-1,1,2,3}\draw (0.05,\y)--(-0.05,\y) node [left] {\scriptsize $\y$};
			\clip (\xmin,\ymin) rectangle (\xmax,\ymax);
			\draw[thick,smooth,samples=200,domain=-2:6] plot (\x,{13/3360*(\x)^4-61/672*(\x)^3+173/336*(\x)^2-11/42*(\x)-61/70});
			\draw[dashed](-2,0)|-(0,2.5)(6,0)|-(0,1.5);
		\end{tikzpicture}
	\end{center}
		\choiceTF
		{$\max\limits_{\left[-2;6\right]}f\left(x\right)=f\left(-1\right)$}
		{$\max\limits_{\left[-2;6\right]}f\left(x\right)=f\left(6\right)$}
		{$\max\limits_{\left[-2;6\right]}f\left(x\right)=f\left(-2\right)$}
		{\True $\max\limits_{\left[-2;6\right]}f\left(x\right)=\max\left\{ f\left(-1\right),f\left(6\right)\right\}$}
	
	\loigiai{
		\begin{center}
			\begin{tikzpicture}
				\tkzTabInit[nocadre,,lgt=1.2,espcl=2.5,deltacl=0.6]
				{$x$/0.6,$y'$/0.6,$y$/2}
				{$-2$,$-1$,$2$,$6$}
				\tkzTabLine{,+,0,-,0,+,}
				\tkzTabVar{-/$f(-2)$,+/$f(-1)$,-/$f(2)$,+/$f(6)$}
			\end{tikzpicture}
		\end{center}
		Dựa vào bảng biến thiên, ta thấy
		\begin{itemize}
			\item Hàm số đồng biến trên $\left( { - 2; - 1} \right)$ và $\left( {2;6} \right)$ do $f'\left( x \right) > 0$, suy ra
			\begin{center}
				$f\left( { - 1} \right) > f\left( { - 2} \right)$ và $f\left( 6 \right) > f\left( 2 \right)$ (1).
			\end{center}
			\item Hàm số nghịch biến trên $\left( { - 1;2} \right)$ do $f'\left( x \right) < 0$, suy ra
			\begin{center}
				$f\left( { - 1} \right) > f\left( 2 \right)$  $ (2) $.
			\end{center}
		\end{itemize}
		Từ $ (1) $, $ (2) $ suy ra $\mathop {\max }\limits_{\left[ { - 2;6} \right]} f\left( x \right) = \max \left\{ {f\left( { - 2} \right),f\left( { - 1} \right),f\left( 2 \right),f\left( 6 \right)} \right\} = \max \left\{ {f\left( { - 1} \right),f\left( 6 \right)} \right\}$.
	}
\end{ex} \dongcham{13}

\begin{ex}
	Cho hàm số $f(x)$ có đạo hàm là $f'(x)$. Đồ thị $y=f'(x)$ được cho như hình vẽ. Biết rằng $f(0)+f(3)=f(2)+f(5)$. Xét tính đúng, sai của các khẳng định sau:
	\begin{center}
		\begin{tikzpicture}[scale=1, font=\footnotesize, line join=round, line cap=round, >=stealth]
			\draw[->](-1,0)--(5.5,0) node[right] {$x$};
			\draw[->](0,-1)--(0,2.5) node[right] {$y$};
			\node (0,0) [below left]{$O$};
			\foreach \x in {1,...,5}
			\draw[shift={(\x,0)},color=black] (0pt,2pt) -- (0pt,-2pt);
			\foreach \y in {1,...,2}
			\draw[shift={(0,\y)},color=black] (2pt,0pt) -- (-2pt,0pt);
			\draw (-0.3,1.2) .. controls (0.1,-1.8) and (1.5,-0.5) .. (2,0) .. controls (3,1.) and (4,1.2) .. (5,1.3) .. controls (5.1,1.3) and (5.3,1.3) .. (5.5,1.3);
			\clip (-1,-1) rectangle (5.5,2.5);
			\draw[dashed](5,0)--(5,1.3);
			\fill (0,0) circle(1pt) (2,0) circle(1pt) node[below right]{$2$} (5,0) circle(1pt) node[below]{$5$};
		\end{tikzpicture}
	\end{center}
	\choiceTF
	{Hàm số nghịch biến trên khoảng $(-\infty;0)$}
	{\True Hàm số nghịch biến trên khoảng $(0;2)$}
	{$\min\limits_{[0;5]}f(x)=f(0)$ và $\max\limits_{[0;5]}f(x)=f(5)$}
	{\True $\min\limits_{[0;5]}f(x)=f(2)$ và $\max\limits_{[0;5]}f(x)=f(5)$}
	
	\loigiai
	{
		Bảng biến thiên của hàm số trên đoạn $[0;5]$
		\begin{center}
			\begin{tikzpicture}
				\tkzTabInit[lgt=1.5,espcl=3,deltacl=0.6]
				{$x$/0.6, $f'(x)$/0.6, $f(x)$/2}
				{$0$, $2$, $3$, $5$}
				\tkzTabLine{,-,z,+, ,+,}
				\tkzTabVar{+/$f(0)$, -/$f(2)$, R, +/$f(5)$}
				\tkzTabVal[draw]{2}{4}{0.5}{}{$f(3)$}
			\end{tikzpicture}
		\end{center}
		Từ bảng biến thiên suy ra $\min\limits_{[0;5]}f(x)=f(2)$ và $\max\limits_{[0;5]}f(x)=\max\{f(0);f(5)\}$.\\
		Theo bảng biến thiên thì $f(3)>f(2)$ nên $f(3)-f(2)>0$.\\
		Theo giả thiết, ta có
		\[f(0)+f(3)=f(2)+f(5) \Leftrightarrow f(5)=f(0)+\left[f(3)-f(2)\right]>f(0).\]
		Suy ra $\max\limits_{[0;5]}f(x)=f(5)$.\\
		Vậy $\min\limits_{[0;5]}f(x)=f(2)$ và $\max\limits_{[0;5]}f(x)=f(5)$.
	}
\end{ex} \dongcham{13}

\Closesolutionfile{ans}
\begin{dang}{Bài toán max, min có chứa tham số $m$}
\end{dang}
\boxmini{BÀI TẬP TỰ LUẬN}
\begin{vd}
	Tìm tất cả giá trị của tham số $m$ để 
	\begin{tasks}
		\task giá trị lớn nhất của hàm số $f(x)= - x^3 -3x^2 +m$ trên $[-1;1]$ bằng $0$.
		\task giá trị nhỏ nhất của hàm số $ f(x)=\dfrac{x+5m}{x-3} $ trên $[1;2]$ bằng $4$.
	\end{tasks}
	\loigiai{
		\begin{enumerate}[a)]
			\item Hàm số liên tục và xác định trên đoạn $[-1;1]$.\\
			Ta có $f'(x)= -3x^2 -6x$.\\
			Cho $f'(x)=0 \Leftrightarrow \hoac{& x=0 \in [-1;1]\\& x= -2 \notin [-1;1].}$\\
			Xét $f(-1)= -2 + m $; $f(1)= -4 + m$.\\
			Suy ra $\displaystyle \max_{[-1;1]} f(x) = -2 + m$.\\
			Theo đề bài, $-2+ m=0 \Leftrightarrow m=2.$
			\item Ta có $ y'=\dfrac{-3-5m}{(x-3)^2} $.
			\begin{itemize}
				\item Trường hợp $ -3-5m>0\Leftrightarrow m<-\dfrac{3}{5}$\\
				$\Rightarrow y'>0 $ thì $ \displaystyle\min_{[1;2]}y=y(1)\Leftrightarrow -\dfrac{1}{2}(1+5m)=4\Leftrightarrow m=-\dfrac{9}{5}$ (nhận vì $ -\dfrac{9}{5}<-\dfrac{3}{5} $).
				\item Trường hợp $ -3-5m<0\Leftrightarrow m>-\dfrac{3}{5}$\\
				$ \Rightarrow y'<0 $ thì $ \displaystyle\min_{[1;4]}y=y(2)\Leftrightarrow -(2+5m)=4\Leftrightarrow m=-\dfrac{6}{5} $ (loại vì $ -\dfrac{6}{5}<-\dfrac{3}{5} $).
			\end{itemize}
			Vậy  $m=-\dfrac{9}{5}$.
		\end{enumerate}
		}
\end{vd}
\dongcham{20}
\boxmini{BÀI TẬP TRẮC NGHIỆM}

\setcounter{ex}{0}
\Opensolutionfile{ans}[ans/2D1-B2-d2-1]

\begin{ex}
	Cho hàm số $f(x) = 2x^3 -3x^2 + m$ thoả mãn $\displaystyle \min_{[0;5]} f(x) = 5$. Khi đó giá trị của $m$ bằng
	\choice
	{$10 $}
	{$ 5$}
	{\True $ 6$}
	{$ 7$}
	\loigiai{
		Ta có $f'(x)= 6x^2 -6x$.\\
		Cho $f'(x)=0 \Leftrightarrow \hoac{&x=0 \in [0;5] \\& x=1 \in [0;5].}$\\
		Xét $f(0)= m$; $f(1)= -1+ m$; $f(5)= 175 +m$.\\
		Suy ra $\displaystyle \min_{[0;5]} f(x)= -1+m$.\\
		Theo giả thiết $-1+ m= 5 \Leftrightarrow m=6$.}
\end{ex} \dongcham{10}

\begin{ex}
	Tìm $m$ để giá trị nhỏ nhất của hàm số $f(x) = 3x^3 - 4x^2 + 2(m - 10)$ trên đoạn $[1; 3]$ bằng $-5$.
	\choice
	{$m = \dfrac{15}{2}$}
	{$m = - 15$}
	{\True $m = 8$}
	{$m = -8$}
	\loigiai{
		$\bullet$ $f'(x) = 9x^2-8x$. Ta có $f'(x) = 0 \Leftrightarrow \hoac{&x = 0\\&x = \dfrac{8}{9}.}$\\
		$\bullet$ Ta có bảng biến thiên
		\begin{center}
			\begin{tikzpicture}
				\tkzTabInit[espcl=4,lgt=2,deltacl=1]{$x$/1,$f'(x)$/1,$f(x)$/2}
				{$1$,$3$}
				\tkzTabLine{,+,}
				\tkzTabVar{-/$2m-21$,+/$2m+25$}
			\end{tikzpicture}
		\end{center}
		$\bullet$ Giá trị nhỏ nhất của $f(x)$ trên đoạn $[1;3]$ bằng $-5 \Leftrightarrow 2m - 21 = -5 \Leftrightarrow m= 8$.
	}
\end{ex} \dongcham{12}

\begin{ex}
	Tìm $m$ để giá trị nhỏ nhất của hàm số $f(x)=\dfrac{x-m^2+m}{x+1}$ trên đoạn $[0;1]$ bằng $-2$.
	\choice
	{$\hoac{&m=1\\&m=-2}$}
	{$\hoac{&m=1\\&m=2}$}
	{$m=\dfrac{1\pm\sqrt{21}}{2}$}
	{\True $\hoac{&m=-1\\&m=2}$}
	\loigiai{
		$\mathscr{D}=\mathbb{R}\setminus\{-1\}$.\\
		Ta có $f'(x)=\dfrac{m^2-m+1}{(x+1)^2}>0$, $\forall x\in\mathscr{D}$.\\
		Khi đó $\min\limits_{x\in[0;1]}f(x)=f(0)\Leftrightarrow -2=-m^2+m\Leftrightarrow \hoac{&m=-1\\&m=2}$.
	}
\end{ex} \dongcham{12}

\begin{ex}
	Hàm số $y=\dfrac{x-m}{x+2}$ thỏa mãn $\min \limits_{x\in[0;3]}y+\max \limits_{x\in[0;3]}y=\dfrac{7}{6}$. Hỏi giá trị $m$ thuộc khoảng nào trong các khoảng dưới đây?
	\choice
	{$(2;+\infty)$}
	{$(0;2)$}
	{$(-\infty;-1)$}
	{\True $(-1;0)$}
	\loigiai{
		Do hàm số $y=\dfrac{x-m}{x+2}$ luôn đơn điệu trên đoạn $[0;3]$.\\
		Do đó $\min \limits_{x\in[0;3]}y+\max \limits_{x\in[0;3]}y=y(0)+y(3)=\dfrac{-m}{2}+\dfrac{3-m}{5}=\dfrac{7}{6}\Leftrightarrow\dfrac{-7m}{10}=\dfrac{17}{30}\Leftrightarrow m=\dfrac{-17}{21}$.}
\end{ex} \dongcham{11}

\begin{ex}
	Cho hàm số $y=\dfrac{x+m}{x+1}$ ($m$ là tham số thực) thỏa mãn $\min\limits_{[1;2]} y+\max\limits_{[1;2]} y=\dfrac{16}{3}$. Mệnh đề nào dưới đây đúng?
	\choice
	{\True $m>4$}
	{$m\le 0$}
	{$0<m\le 2$}
	{$2<m\le 4$}
	\loigiai{
		Tập xác định $\mathscr{D}=\mathbb{R}$.\\
		Ta có $y'=\dfrac{1-m}{(x+1)^2}$.
		\begin{itemize}
			\item Với $m=1$ thì $y=1$ nên $\min\limits_{[1;2]} y+\max\limits_{[1;2]} y=2$ (không thỏa mãn).
			\item Với $m\neq 1$ thì hàm số đơn điệu trên $[1;2]$ nên
			\begin{eqnarray*}
				&& \min\limits_{[1;2]} y+\max\limits_{[1;2]} y=\dfrac{16}{3}\\
				& \Leftrightarrow & y(1)+y(2)=\dfrac{16}{3}\\
				& \Leftrightarrow & \dfrac{m+1}{2}+\dfrac{m+2}{3}=\dfrac{16}{3}\\
				& \Leftrightarrow & m=5>4.
			\end{eqnarray*}
		\end{itemize}
	}
\end{ex} \dongcham{11}

\begin{ex}
	Cho hàm số $ f(x)=\dfrac{x+m}{x-1} $ ($ m $ là tham số thực) thỏa mãn $ \min\limits_{[2 ; 4]} f(x)=3 $. Mệnh đề nào dưới đây đúng ?
	\choice
	{$1\leq m<3$}
	{$m < -1$}
	{$3<m\leq 4$}
	{\True$m>4$}
	\loigiai{
		Tập xác định $ \mathscr{D} = \mathbb{R} \setminus\{1\}$.\\
		Ta có $ f'(x)=\dfrac{-1-m}{(x-1)^{2}} $.\\
		\underline{\textbf{TH1}}: $ -1-m<0 \Leftrightarrow m >-1 $.\\
		Ta có $ \min\limits_{[2 ; 4]}y=y(4)=\dfrac{4+m}{4-1}=3\Leftrightarrow m=5$ (thỏa mãn).\\
		\underline{\textbf{TH2}}: $  -1-m>0 \Leftrightarrow m <-1 $.\\
		Ta có $ \min\limits_{[2 ; 4]}y=y(2)=\dfrac{2+m}{2-1}=3\Leftrightarrow m=1$ (loại).\\
		Vậy $ m=5>4 $.
	}
\end{ex} \dongcham{11}

\begin{ex}
	Gọi $S$ là tổng giá trị của $m$ để hàm số $f(x) = \dfrac{x - m^2 - m}{x+1}$ có giá trị nhỏ nhất trên $[0;1]$ bằng $-2$. Mệnh đề nào sau đây đúng?
	\choice
	{\True $S=-1 $}
	{$S=1 $}
	{$S=-2 $}
	{$ S=-3$}
	\loigiai{
		Ta có $f'(x)= \dfrac{m^2 + m -1 }{(x+1)^2}$.
		\begin{itemize}
			\item Trường hợp $1$: $y'<0 \Leftrightarrow m^2 + m -1 <0$.\\
			Khi đó hàm số nghịch biến trên $[0;1]$.\\
			Suy ra $\displaystyle \min_{[0;1]} f(x) = f(1)= \dfrac{-m^2 -m +1}{2}$.\\
			Theo giả thiết $\dfrac{-m^2 -m +1}{2} = -2 \Leftrightarrow m^2 + m =5$ (không thoả điều kiện $m^2 +m <1$).
			\item Trường hợp $2$: $y'>0 \Leftrightarrow m^2 + m -1>0$.\\
			Khi đó $\displaystyle \min_{[0;1]} f(x) = f(0)=-m^2 -m$.\\
			Theo giả thiết $-m^2 -m =-2  \Leftrightarrow \hoac{&m= 1 \text{ (nhận) }\\& m=-2 \text{ (nhận).}}$
		\end{itemize}
		Vậy tổng các giá trị của $m$ là $-2+1 =-1.$
	}
\end{ex} \dongcham{11}

\begin{ex}
	Cho hàm số $f(x)=x^3+m x^2-m^2x+2$ với tham số $m>0$. Biết $\min\limits_{[-m ; m]}f(x)=\dfrac{14}{ 27}$. Mệnh đề nào dưới đây đúng
	\choice
	{$m\in (-\infty;-3)$}
	{$m\in (3;+\infty)$}
	{\True $m\in (1;3)$}
	{$m\in (-3;-1)$}
	\loigiai{
		Ta có $f'(x)=3x^2+2mx-m^2=(x+m)(3x-m)$.\\
		$f'(x)=0\Leftrightarrow \hoac{& x=-m \\ & x=\dfrac{m}{3}}$. Suy ra $\heva{& f(-m)=m^3 +2\\ & f(m)=m^3+2\\ &f\left(\dfrac{m}{3}\right)=-\dfrac{5m^3}{27}+2.}$\\
		Vì $m>0$ nên $f(m)=f(-m)>f\left(\dfrac{m}{3}\right)$, suy ra $\min\limits_{  [-m;m]} f(x)=f\left(\dfrac{m}{3}\right)=\dfrac{14}{27}$.\\
		Do đó $m=2$, vậy $m\in(1;3)$.
	}
\end{ex} \dongcham{11}

\begin{ex}%[2D1K3-1]
	Có tất cả bao nhiêu giá trị nguyên của tham số $m$ để giá trị nhỏ nhất của hàm số $y=x^3+\left(m^2-m+1\right)x+m^3-4m^2+m+2025$ trên đoạn $[0;2]$ bằng $2019$?
	\choice
	{$0$}
	{$1$}
	{$2$}
	{\True $3$}
	\loigiai{
		Ta có $y'=f'(x)=3x^2+\left(m^2-m+1\right)$ trên đoạn $[0;2]$.\\
		Ta có $y'=3x^2+\left(m-\dfrac{1}{2}\right)^2+\dfrac{3}{4}>0,\forall x\in\mathbb{R}$.\\
		Do đó hàm số đồng biến trên $\mathbb{R}\Rightarrow$ ta có $\min\limits_{[0;2]}y=f(0)=m^3-4m^2+m+2025$.\\
		Ta có $f(0)=2019\Leftrightarrow m^3-4m^2+m+2025=2019\Leftrightarrow m^3-4m^2+m+6=0\Leftrightarrow\hoac{&m=-1\\&m=2\\&m=3.}$\\
		Vậy tập các giá trị $m$ thỏa mãn là $\{-1;2;3\}$. Hay có tất cả $3$ giá trị $m$ thỏa mãn.}
\end{ex} \dongcham{11}

\begin{ex}
	Gọi $S$ là tập tất cả các giá trị của $m$ sao cho giá trị nhỏ nhất của hàm số $y=\left(x^3-3x+m \right)^2 $ trên
	đoạn $[-1;1]$ bằng $4$. Tính tổng các phần tử của $S$.
	\choice
	{\True  $ 0 $}
	{$ 6 $}
	{$ -5 $}
	{$ 3 $}
	\loigiai{
		\immini{Ta có  $\displaystyle\min\limits_{[-1;1]}\left(x^3-3x+m \right)^2=4  \Leftrightarrow \displaystyle\min\limits_{[-1;1]}\left|x^3-3x+m \right|=2$.\\Xét hàm số $y=f(x)=x^3-3x+m$ trên $[-1;1]$.\\
			Ta có $y'=3x^2-3=3(x^2-1)$, $y'=0\Leftrightarrow x=\pm1$.\\
			Bảng biến thiên hàm số như hình bên.
		}{\begin{tikzpicture}[scale=.8,line join=round, line cap=round,font=\footnotesize,>=stealth]
				\def\a{6}
				\def\b{3.7}
				\draw[shift={(-.5,.5)},blue!50!black]
				(0,0) rectangle +(\a,-\b)
				(0,-1)--+(0:\a)
				(0,-2)--+(0:\a)
				(1,0)--+(-90:\b)
				;
				\path
				(0,0) node{$x$}
				(0,-1) node{$y'$}
				(0,-2.3) node{$y$}
				(1,0) node{$-1$}
				(5,0) node{$1$}
				(1,-1) node{$0$}
				(3,-1) node{$-$}
				(5,-1) node{$0$}
				(1.2,-1.8) node (A) {$m+2$}
				(4.8,-3) node (C){$m-2$}
				;
				\draw[->] (A)--(C);
		\end{tikzpicture}}
		\noindent Từ bảng biến thiên của hàm số $y=f(x)$, ta có $\displaystyle\min\limits_{[-1;1]}\left|x^3-3x+m \right|=2$ khi và chỉ khi
		\begin{enumerate}[TH1.]
			\item $\heva{&m+2<0\\&m+2=-2}\Leftrightarrow m=-4$.
			\item $\heva{&m-2>0\\&m-2=2}\Leftrightarrow m=4$.
		\end{enumerate}
		Vậy $S=\{-4,4\}\Rightarrow $ Tổng các phần tử của $S$ bằng $0$.
	}
\end{ex} \dongcham{12}

\Closesolutionfile{ans}



% 
\begin{dang}{Bài toán vận dụng, thực tiễn có liên quan đến max min}
	\begin{enumerate}[\iconMT]
		\item \indam{Bài toán chuyển động:}
		\begin{itemize}
			\item [$\bullet$] Gọi $s(t)$ là hàm quãng đường; $v(t)$ là hàm vận tốc; $a(t)$ là hàm giá tốc;
			\item [$\bullet$] Khi đó $s'(t)=v(t)$; $v'(t)=a(t)$.
		\end{itemize}
		\item \indam{Bài toán thực tế -- tối ưu:}
		\begin{itemize}
			\item[$\bullet$] Biểu diễn dữ kiện cần đạt max -- min qua một hàm $f(t)$. 
			\item[$\bullet$] Khảo sát hàm $f(t)$ trên miền điều kiện của hàm và suy ra kết quả.
		\end{itemize}
	\end{enumerate}
\end{dang}
\boxmini{BÀI TẬP TỰ LUẬN}
\begin{vd}%[2D1B3-6]
	Một chất điểm chuyển động có vận tốc tức thời $v(t)$ phụ thuộc vào thời gian $t$ theo hàm số $v(t)=-t^4+24t^2+500$ (m/s). Trong khoảng thời gian từ $t=0$ (s) đến $t=5$ (s) chất điểm đạt vận tốc lớn nhất tại thời điểm nào?
	\loigiai{Ta có $v'(t)=-4t^3+48t=-4t(t^2-12)$\\
		$v'(t)=0\Leftrightarrow \hoac{&t=0\\&t=\pm 2\sqrt{3}}$.\\
		Bài toán trở thành tìm giá trị lớn nhất của hàm số $v(t)$ trên đoạn $[0;10]$, ta có:\\
		$v(0)=500$, $v(2\sqrt{3})=664$, $v(5)=475$.\\
		Vậy vận tốc lớn nhất khi $t=2\sqrt{3}\approx 4$ (s).
	}
\end{vd}
\dongcham{8}
\begin{vd}
	\immini{
		Sự phân huỷ của rác thải hữu cơ có trong nước sẽ làm tiêu hao oxygen hoà tan trong nước. Nồng độ oxygen (mg/l) trong một hồ nước sau $t$ giờ $(t \geq 0)$ khi một lượng rác thải hữu cơ bị xả vào hồ được xấp xỉ bởi hàm số (có đồ thị như đường màu đỏ ở hình bên)
		$$
		y(t)=5-\frac{15 t}{9 t^2+1}.
		$$
	}{
		\begin{tikzpicture}[>=stealth,x=1cm,y=0.3cm,scale=1.5,font=\footnotesize]
			\draw[->] (-0.5,0) -- (4,0) node[below] {$t$};
			\draw[->] (0,-1) -- (0,6) node[left] {$y$};
			\filldraw (0,0) circle (1pt)node[below left]{$O$};
			\draw[domain=0:4,samples=200,red] plot (\x,{5-(15*(\x))/(9*(\x)^2+1)});
			\draw[dashed] (0,5) node [left] {$5$}--(4,5);
			\foreach \x/\g in {1/-90,2/-90,3/-90}
			\draw[thin] (\x,2pt)--(\x,-2pt) + (\g:3mm) node {$\x$};
		\end{tikzpicture}
	}
	\noindent
	Vào các thời điểm nào nồng độ oxygen trong nước cao nhất và thấp nhất?\
	\loigiai{
		Xét hàm số $y(t)=5-\dfrac{15t}{9t^2+1}$ xác định và liên tục trên khoảng $[0;+\infty)$ .\\
		Ta có $y'(t)=\dfrac{135t^2-15}{(9t^2+1)^2}=0\Leftrightarrow t=\dfrac{1}{3}$ (giờ).\\
		Mặt khác $\lim\limits_{t\to+\infty}y(t)=\lim\limits_{t\to+\infty}\left[5-\dfrac{15t}{9t^2+1}\right]=5$ và $\lim\limits_{t\to 0}y(t)=\lim\limits_{t\to 0}\left[5-\dfrac{15t}{9t^2+1}\right]=5$.\\
		Bảng biến thiên
		\begin{center}
			\begin{tikzpicture}
				\tkzTabInit[espcl=3,lgt=1.5]
				{$t$/0.6,$y'(t)$/0.6,$y(t)$/1.5}
				{$0$,$\frac{1}{3}$,$+\infty$}
				\tkzTabLine{,-,0,+,}
				\tkzTabVar{+/$5$,-/$0$,+/$5$}
			\end{tikzpicture}
		\end{center}
		Từ bảng biến thiên, ta thấy $\min\limits_{[0;+\infty)}y(x)=0$ và $\mathop{\rm{max}}\limits_{[0;+\infty)}y(x)=5$.
	}
\end{vd}
\dongcham{10}
\begin{vd}%[2D1T3-6]
	\immini[0.02]{
		Tính diện tích lớn nhất $S_{\max}$ của một hình chữ nhật nội tiếp trong nửa đường tròn bán kính $R=6$ cm nếu một cạnh của hình chữ nhật nằm dọc theo đường kính của hình tròn mà hình chữ nhật đó nội tiếp.
	}{
		\begin{tikzpicture}[line join = round, line cap = round,>=stealth,font=\footnotesize,scale=1] 
			\def\R{2}
			\coordinate[label = below:$O$] (O) at (0,0); 
			\coordinate (A) at (-\R,0); 
			\coordinate (B) at ($(A)!2!(O)$);
			\coordinate[label = above right:$C$] (C) at (50:\R); 
			\coordinate[label = above left:$D$] (D) at (130:\R);
			\coordinate[label = below:$A$] (AA) at ($(A)!(D)!(B)$); 
			\coordinate[label = below:$B$] (BB) at ($(A)!(C)!(B)$); 
			\draw (A) arc(180:0:\R)--cycle;
			\draw[fill=cyan!20] (BB)--(C)--(D)--(AA)--cycle;
			\foreach \x in {AA,O,BB} \fill[black] (\x) circle (1.5pt); 
		\end{tikzpicture}
	}
	\loigiai{
		\immini{
			{\bf Cách 1.}\\
			Gọi chiều dài $AD=2x$ ($0<x<6$)\\
			$\Rightarrow AB=\sqrt{36-x^{2}}$.\\
			Diện tích hình chữ nhật là $S=2x\sqrt{36-x^{2}}$.\\
			Xét $f(x)=x\sqrt{36-x^{2}}$ trên $(0;6)$, ta có $$f'(x)=\sqrt{36-x^{2}}-\dfrac{x^{2}}{\sqrt{36-x^{2}}}=0\Leftrightarrow x=\pm 3\sqrt{2}.$$
		}{
			\begin{tikzpicture}
				\tikzset{on double/.style = {fill = \tkzTabDefaultBackgroundColor}} 
				\tikzset{h style/.style = {pattern=north west lines}} 
				\tkzTabInit[lgt=1.2,espcl=2]
				{$x$ /.6,$f'(x)$ /.6, $f(x)$ /1.5}
				{$0$,$3\sqrt{2}$,$6$}
				\tkzTabLine{d,+,0,-,d}
				\tkzTabVar{-/$0$,+/$36$,-/$0$}
			\end{tikzpicture}
		}
		Bảng biến thiên hàm số $f(x)$ trên $(0,6)$ ở hình bên\\
		Vậy giá trị lớn nhất của diện tích hình chữ nhật $ABCD$ là $36$ cm$^2$.\\
		{\bf Cách 2.}\\
		Đặt $AB=CD=2x$ ($0<x<6$). Khi đó $AD=\sqrt{DO^2-AO^2}=\sqrt{36-x^2}$. Suy ra
		\begin{align*}
			S_{ABCD}=2x\sqrt{36-x^2}\le 2\cdot \dfrac{x^2+36-x^2}{2}=36.
		\end{align*}
		Dấu bằng xảy ra khi $x=\sqrt{36-x^2}$ hay $x=3\sqrt{2}$.\\
		Vậy giá trị lớn nhất của diện tích hình chữ nhật $ABCD$ là $36$ cm$^2$.
	}
\end{vd}
\dongcham{14}
\begin{vd}%[2D1K3-6]
	\immini{Một người muốn xây một cái bể chứa nước, dạng một khối hộp chữ nhật không nắp có thể tích
	bằng $288$ dm$^3$. Đáy bể là hình chữ nhật có chiều dài gấp đôi chiều rộng, giá thuê nhân công để xây bể là
	$500000$ đồng/ m$^2$. Nếu người đó biết xác định các kích thước của bể hợp lí thì chi phí thuê nhân công sẽ
	thấp nhất. Hỏi người đó trả chi phí thấp nhất để thuê nhân công xây dựng bể đó là bao nhiêu?}{\hspace{1cm}
	\begin{tikzpicture}[scale=0.8, line join=round, line cap=round]
		\tkzDefPoints{0/0/A,-1.3/-1.1/B,2/-1.1/C}
		\coordinate (D) at ($(A)+(C)-(B)$);
		\coordinate (A') at ($(A)+(0,2.5)$);
		\tkzDefPointsBy[translation=from A to A'](B,C,D){B'}{C'}{D'}
		\tkzDrawPolygon(A',B',B,C,D,D')
		\tkzDrawSegments(B',C' C',D' C,C')
		\tkzDrawSegments[dashed](A,B A,D A,A')
\end{tikzpicture}}
	\loigiai{
		Gọi $x(x>0)$ là chiều rộng của đáy bể. Khi đó, chiều dài của bể là $2x$ và chiều cao của bể là $\dfrac{0,144}{x^2}$.\\
		Diện tích cần xây $2x^2+\dfrac{0,864}{x}$\\
		Xét $f(x) = 2x^2 + \dfrac{0,864}{x}$, có
		$f'(x) = 4x - \dfrac{0,864}{x^2}$\\
		$f'(x) = 0 \Leftrightarrow 4x - \dfrac{0,864}{x^2} \Leftrightarrow x=0,6.$\\
		Bảng biến thiên
		\begin{center}
			\begin{tikzpicture}
				\tkzTabInit[nocadre=false, lgt=1.2, espcl=3]
				{$x$ /0.6,$f'(x)$ /0.6,$f(x)$ /1.5} 	
				{$0$, $0{,}6$, $+\infty$}
				\tkzTabLine{,-,$0$,+}
				\tkzTabVar{+/ $+\infty$ ,-/$2{,}16$,+/$+\infty$}
			\end{tikzpicture}
		\end{center}
		Từ bảng biến thiên ta có $\min f(x)= 2,16.$\\
		Vậy chi phí thấp nhất để thuê nhân công xây bể là $2,16 \times 500000 = 1080000$ đồng.
	}
\end{vd}
\dongcham{18}
\begin{vd}%[2D1T3-2]
	\immini{Một nhà sản xuất cần làm ra những chiếc bình có dạng hình trụ với dung tích $1000\mathrm{~cm}^3$. Mặt trên và mặt dưới của bình được làm bằng vật liệu có giá 1,2 nghìn đồng$/\mathrm{cm}^2$, trong khi mặt bên của bình được làm bằng vật liệu có giá $0{,}75$ nghìn đồng$/\mathrm{cm}^2$. Tìm các kích thước của bình để chi phí vật liệu sản xuất mỗi chiếc bình là nhỏ nhất.}{\hspace{1cm}
	\begin{tikzpicture}[line join=round,line cap=round,line width=.6pt,font=\footnotesize,scale=0.46,>=stealth]
		\coordinate[label=right:$A$] (A) at (3,0);
		\coordinate[label=left:$O$] (O) at (0,0);
		\coordinate[label=right:$A'$] (A1) at ($(A)+(90:6)$);
		\coordinate[label=left:$O'$] (O1) at ($(O)+(90:6)$);
		\draw (A) arc (0:-180:3 and 3/4)--($(A1)!2!(O1)$) arc (180:0:3 and 3/4) arc (0:-180:3 and 3/4) (A)--(A1)--(O1);
		\draw[dashed] (O1)--(O)--(A) arc (0:180:3 and 3/4);
		\fill (O)circle(1.5pt) (O1)circle(1.5pt) (A)circle(1.5pt) (A1)circle(1.5pt);
\end{tikzpicture}}
	\loigiai{
			Gọi bán kính đáy của bình là $x$ (cm), ($x > 0$).\\
			Chiều cao của bình là $\dfrac{1000}{\pi \cdot x^2}$ (cm).\\
			Chi phí để sản xuất một chiếc bình là 
			\[
			T(x)=2\cdot1{,}2\cdot\pi \cdot x^2+0{,}75\cdot \dfrac{2000}{x}=2{,}4\pi \cdot x^2+\dfrac{1500}{x}~\text{(nghìn đồng)}.
			\]
			Để chi phí sản xuất mỗi chiếc bình là thấp nhất thì $T(x)$ là nhỏ nhất.\\
			$T^{\prime}(x)=4,8\pi x-\dfrac{1500}{x^2}, T^{\prime}(x)=0\Leftrightarrow x=\sqrt[3]{\dfrac{625}{2\pi}}$ (thỏa mãn).\\
			Bảng biến thiên:
			\begin{center}
				\begin{tikzpicture}[scale=1, font=\footnotesize]
					\tkzTabInit[nocadre=false, lgt=1.2, espcl=2, deltacl=0.6]
					{$x$/0.8,$T'(x)$/0.6,$T(x)$/2}
					{$0$,$\sqrt[3]{\frac{625}{2\pi}}$,$12$};
					\tkzTabLine{,-,$0$,+,};
					\tkzTabVar{+/$+\infty$,-/$T\left(\sqrt[3]{\frac{625}{2\pi}}\right)$,+/$T(12)$};
				\end{tikzpicture}
			\end{center}
			Để chi phí sản xuất mỗi chiếc bình là nhỏ nhất thì bán kính đáy của bình là $\sqrt[3]{\dfrac{625}{2\pi}}$ cm và chiều cao của bình là $\dfrac{1000}{\pi \cdot\left(\sqrt[3]{\dfrac{625}{2\pi}}\right)^2}$ cm.
	}
\end{vd}
\dongcham{20}
\boxmini{BÀI TẬP TRẮC NGHIỆM}
\ind{PHẦN I.} \inden{Câu trắc nghiệm nhiều phương án lựa chọn. Mỗi câu hỏi học sinh chỉ chọn một phương án.}\\
\setcounter{ex}{0}
\Opensolutionfile{ans}[ans/2D1-B2-d3-1]
\begin{ex}%[2D1K3]
	Một chất điểm chuyển động với quãng đường $s(t)$ cho bởi công thức $s(t)=6t^2-t^3$, $t$ (giây) là thời gian. Hỏi trong khoảng thời gian từ $0$ đến $4$ giây, vận tốc tức thời của chất điểm đạt giá trị lớn nhất tại thời điểm  $t$ (giây) bằng bao nhiêu?
	\choice
	{$t=3$ s}
	{$t=4$ s}
	{\True $t=2$ s}
	{$t=6$ s}
	\loigiai{Ta có $v(t)=s'(t)=12t-3t^2$.\\
		$v'(t)=12-6t$, $v'(t)=0\Leftrightarrow t=2$. \\
		Lập bảng biến thiên ta thấy $v(t)$ đạt giá trị lớn nhất tại $t=2$.
	}
\end{ex} \dongcham{7}

\begin{ex}
	Trong $3$ giây đầu tiên, một chất điểm chuyển động theo phương trình $s(t)=-t^3+6t^2+t+5,$ trong đó $t$ tính bằng giây và $s$ tính bằng mét. Chất điểm có vận tốc tức thời lớn nhất bằng bao nhiêu trong $3$ giây đầu tiên đó?
	\choice
	{\True 13 m/s}
	{10 m/s}
	{9 m/s}
	{12 m/s}
	\loigiai{
		Ta có $v(t)=s'(t)=-3t^2+12t+1.$ Xét hàm số $v(t)=-3t^2+12t+1$ trên đoạn $[0;5]$.\\
		$v'(t)=-6t+12$; $v'(t)=0 \Leftrightarrow t=2$.\\
		Tính các giá trị $v(0)=1$, $v\left(2\right)=13$, $v(3)=10$.\\
		So sánh các giá trị, ta có $\max\limits_{[0;3]}v(t)=13$.
	}
\end{ex}
\dongcham{7}
\begin{ex}
	Độ giảm huyết áp của một bệnh nhân được cho bởi công thức $G(x)=0{,}025x^2(30-x)$, trong đó $x$ là liều lượng thuốc được tiêm cho bệnh nhân ($x$ được tính bằng miligam). Liều lượng thuốc cần tiêm cho bệnh nhân là bao nhiêu để huyết áp được giảm nhanh nhất?
	\choice
	{$24$ mg}
	{\True $20$ mg}
	{$15$ mg}
	{$10$ mg}
	\loigiai
	{ 
		Bài toán trở thành: Tìm $x\in[0;30]$ để hàm số $G(x)=0{,}025x^2(30-x)$ đạt giá trị lớn nhất. \\
		Ta có $G(x)=0{,}025\left(30x^2-x^3\right) \Rightarrow G'(x)=0{,}025\left(60x-3x^2\right)$. \\
		Xét $G'(x)=0 \Leftrightarrow \hoac{ & x=0 \\ & x=20.}$ \\
		Bảng biến thiên hàm số $G(x)$
		\begin{center}
			\begin{tikzpicture}[scale=1]
				\tkzTabInit[nocadre=false, lgt=1.2, espcl=3.5, deltacl=0.6]{$x$/0.6, $G'(x)$/0.6, $G(x)$/2}{$0$, $20$, $30$}
				\tkzTabLine{0,+,0,-,}
				\tkzTabVar{-/$0$, +/$100$, -/$0$}
			\end{tikzpicture}
		\end{center}
		Từ bảng biến thiên ta có $\max\limits_{[0;30]} G(x)=G(20)=100$. \\
		Vậy liều lượng thuốc cần tiêm cho bệnh nhân để huyết áp giảm nhanh nhất là $20$ mg.
	}
\end{ex}
\dongcham{7}
\begin{ex}
	Trong thí nghiệm y học, người ta cấy $1\,000$ vi khuẩn vào môi trường dinh dưỡng. Bằng thực nghiệm, người ta xác định số lượng vi khuẩn thay đổi theo thời gian bởi công thức \[N(t)=1\,000+\dfrac{100t}{100+t^2}\,\text(con).\]
	trong đó $t$ là thời gian tính bằng giây. Tính số lượng vi khuẩn lớn nhất kể từ khi thực hiện cấy vi khuẩn vào môi trường dinh dưỡng.
	\choice
	{$1\,008$ con}
	{$1\,012$ con}
	{\True $1\,005$ con}
	{$1\,020$ con}
	\loigiai{
		Xét hàm số $N(t)=1\,000+\dfrac{100t}{100+t^2}$ ($t>0$).\\
		Ta có $N'(t)=\dfrac{100\cdot (100+t^2)-100t\cdot 2t}{\left(100+t^2\right)^2}=\dfrac{100\cdot (100-t^2)}{\left(100+t^2\right)^2}$.\\
		Khi đó, với $t>0$, $N'(t)=0\Leftrightarrow 100-t^2=0\Leftrightarrow t^2=100\Leftrightarrow t=10$.\\
		Bảng biến thiên của hàm số $N(t)$ như sau
		\begin{center}
			\begin{tikzpicture}[>=stealth]
				\tkzTabInit[nocadre=false,lgt=1.5,espcl=3,deltacl=0.6]{$t$/.6 ,$N'(t)$/.6,$N(t)$/1.5}
				{$0$ , $10$ , $+\infty$}
				\tkzTabLine{ ,+ , $0$ , - , }
				\tkzTabVar{-/$1\,000$ , +/$1\,005$ , -/$1\,000$}
			\end{tikzpicture}
		\end{center}
		Căn cứ vào bảng biến thiên, ta thấy trên khoảng $(0;+\infty)$, hàm số $N(t)$ đạt giá trị lớn nhất bằng $1\,005$ tại $t=10$.\\
		Vậy số lượng vi khuẩn lớn nhất kể từ khi thực hiện nuôi cấy vi khuẩn vào môi trường dinh dưỡng là $1\,005$ con.
	}
\end{ex}
\dongcham{14}
\begin{ex}
	Tam giác vuông có cạnh huyền bằng $5 \mathrm{~cm}$ có thể có diện tích lớn nhất bằng bao nhiêu?
	\choice
	{25 $\text{cm}^2$}
	{$\dfrac{125}{4}\,\text{cm}^2$}
	{$\dfrac{625}{4}\,\text{cm}^2$}
	{$125 \text{cm}^2$}
	\loigiai{Gọi một cạnh góc vuông là $x$ ($0<x<5$) thì cạnh góc vuông còn lại là $\sqrt{25-x^2}$.\\ Như vậy, diện tích tam giác là $S=\dfrac{x\cdot\sqrt{25-x^2}}{2}$.
		Đặt $f(x)=25x^2-x^4$. 
		\\Ta có $f'(x)=50x-4x^3$. Khi đó
		$f'(x)=0 \Leftrightarrow x=\dfrac{5\sqrt{2}}{2}$.\\
		Vì vậy $\displaystyle\max _{(0;5)} f(x)=f\left( \dfrac{5\sqrt{2}}{2}\right) =\dfrac{625}{4}$.\\
		Vậy tam giác vuông có cạnh huyền bằng $5 \mathrm{~cm}$ có thể có diện tích lớn nhất bằng $\dfrac{625}{4}$.}
\end{ex}
\dongcham{18}
\begin{ex}
	\immini{
		Từ một tấm tôn có hình dạng là nửa hình tròn bán kính $R=3$, người ta muốn cắt ra một hình chữ nhật (hình vẽ bên). Diện tích lớn nhất có thể của tấm tôn hình chữ nhật là
		\choice
		{$\dfrac{9}{2}$}
		{$6\sqrt2$}
		{\True $9$}
		{$9\sqrt2$}
	}
	{
		\begin{tikzpicture}[thick,scale=0.57]
			\draw [-] (-4,0)--(4,0);
			\draw [-] (-3,0)--(-2.99,2.65)--(2.99,2.65)--(3,0);
			\draw[smooth,samples=200,variable=\t,domain=0:180] plot({(4)*cos (\t)},{(4)*sin(\t)});
			\draw (0,0) [fill=black] circle (1pt) node[below]{$O$};
			\draw (-3,0) [fill=black] circle (1pt) node[below]{$Q$};
			\draw (3,0) [fill=black] circle (1pt) node[below]{$P$};
			\draw (-2.99,2.65) [fill=black] circle (1pt) node[left]{$M$};
			\draw (2.99,2.65) [fill=black] circle (1pt) node[right]{$N$};
			\draw[pattern=north east lines,pattern color=black!50!] (-3,0)--(-2.99,2.65)--(2.99,2.65)--(3,0);
		\end{tikzpicture}
	}
	\loigiai{
		Đặt $OQ=x,\ (0<x<3) \Rightarrow MQ=\sqrt{MO^2-OQ^2}=\sqrt{9-x^2}$.\\
		Ta có  $S_{MNPQ}=PQ\cdot MQ=2x\cdot\sqrt{9-x^2}\le 2\cdot\dfrac{x^2+9-x^2}{2}=9.$\\
		Dấu $=$ xảy ra khi $x=\dfrac{3\sqrt2}{2}.$
	}
\end{ex} \dongcham{18}

\begin{ex}%[2D1K3-6]
	Cho một tấm tôn hình chữ nhật có kích thước $10$ cm $\times$ $16$ cm. Người ta cắt bỏ $4$ góc của tấm tôn $4$ miếng hình vuông bằng nhau rồi gò lại thành một hình hộp chữ nhật không có nắp. Để thể tích của hình hộp đó lớn nhất thì độ dài cạnh hình vuông của các miếng tôn bị cắt bỏ bằng
	\choice
	{$3$ m}
	{$4$ m}
	{$5$ m}
	{\True $2$ m}
	\loigiai{
		\immini
		{Giả sử độ dài cạnh hình vuông của các miếng tôn bị cắt bỏ bằng $x$ $(0<2x<10\Leftrightarrow 0<x<5)$. Khi đó hình hộp chữ nhật có chiều cao bằng $x$, chiều rộng bằng $10-2x$ và chiều dài bằng $16-2x$. Suy ra hình hộp chữ nhật có thể tích $V=x(10-2x)(16-2x)=4x^3-52x^2+160x$.}
		{
			\begin{tikzpicture}[scale=0.7]
				\tkzInit[xmin=-5,xmax=6,ymin=-3,ymax=6]
				\tkzDefPoints{0/0/A, 8/0/D, 8/6/C, 0/6/B, 0/1/E, 0/5/F, 1/6/G, 7/6/H, 8/5/I, 8/1/J, 1/0/M, 7/0/N}
				\tkzDrawPoints(A,B,C,D,M,N,E,F,G,H,I,J)
				\tkzLabelSegments[above](B,G H,C){$x$}
				\tkzLabelSegments[right](C,I J,D){$x$}
				\tkzLabelSegment[left](A,B){$10$}
				\tkzLabelSegment[below](A,D){$16$}
				\tkzDrawSegments[thin](A,B A,D B,C C,D F,I E,J G,M H,N)
			\end{tikzpicture}
		}
		Xét hàm $f(x)=4x^3-52x^2+160x$ trên $(0; 5)$. Tập xác định $\mathscr{D}=\mathbb{R}$,\\ $f'(x)=12x^2-104x+160=0\Leftrightarrow\hoac{&x=2\\&x=\dfrac{20}{3}.}$
		Bảng biến thiên hàm số $f(x)$ trên $(0; 5)$:
		\begin{center}
			\begin{tikzpicture}
				\tkzTabInit%
				{$x$/1,%
					$f’(x)$ /1,%
					$f(x)$ /2}%
				{$0$ ,$2$ , $5$}%
				\tkzTabLine{ ,+, 0 ,-,}
				\tkzTabVar %
				{
					-/,+/ ,-/
				}
			\end{tikzpicture}
		\end{center}
		Dựa vào bảng biến thiên ta có hàm số đạt giá trị lớn nhất trên $(0; 5)$ tại $x=2$ hay hình hộp chữ nhật có thể tích lớn nhất khi độ dài cạnh hình vuông của miếng tôn bị cắt bỏ bằng $2$ m.
	}
\end{ex}
\dongcham{18}

\begin{ex}%[2H1K3-6]
	Ông Bình dự định sử dụng hết $5,5\,\mathrm{m^2}$ kính để làm một bể cá bằng kính có dạng hình hộp chữ nhật không nắp, chiều dài gấp đôi chiều rộng (các mối ghép có kích thước không đáng kể). Bể cá có dung tích lớn nhất bằng bao nhiêu (làm tròn đến hàng phần trăm)?
	\choice
	{ $1{,}01\,\mathrm{m^3}$}
	{\True $1{,}17\,\mathrm{m^3}$}
	{ $1{,}51\,\mathrm{m^3}$}
	{ $1{,}40\,\mathrm{m^3}$}
	\loigiai{
		\immini{
			Gọi $x,2x,y$ với $x,y>0$  lần lượt là chiều rộng, chiều dài, chiều cao của bể cá.
			Theo giả thiết ta có: $$2\cdot 2xy+2\cdot xy+2x^2=5{,}5\Leftrightarrow 6xy+2x^2=5{,}5\Rightarrow y=\dfrac{5{,}5-2x^2}{6x}.$$
			Do $y>0$ nên $5,5 - 2x^2 >0 \Rightarrow 0<x<\dfrac{\sqrt{11}}{2}$.\\
			Thể tích bể cá là $$V(x)=2x^2y=2x^2\cdot \dfrac{5{,}5-2x^2}{6x}=-\dfrac{2}{3}{x^3}+\dfrac{11}{6}x.$$
			Khảo sát hàm số $V(x)=-\dfrac{2}{3}{x^3}+\dfrac{11}{6}x$ trên khoảng $\left( 0;\dfrac{\sqrt{11}}{2} \right) $
			\begin{itemize}
				\item [$\bullet$] $V'(x)=-2x^2+\dfrac{11}{6}$; $V'(x)=0\Leftrightarrow x=\sqrt{\dfrac{11}{12}}$.
				\item [$\bullet$] Bảng biến thiên:
				\begin{center}
					\begin{tikzpicture}
						\tkzTabInit[nocadre=True,lgt=1,espcl=3]
						{$x$ /1,$V'$ /0.6,$V$ /2}
						{$0$,$\sqrt{\frac{11}{12}}$,$\frac{\sqrt{11}}{2}$}
						\tkzTabLine{,+,$0$,-,}
						\tkzTabVar{-/, +/$y_0$,-/}
					\end{tikzpicture}
				\end{center}
			\end{itemize}
			Thể tích lớn nhất của bể cá là $V\left( \sqrt{\dfrac{11}{12}} \right)=1{,}17\,\mathrm{m^3}$.}{
			\begin{tikzpicture}
				\def\tls{.4}
				\path
				(0,0) coordinate (A)
				++ (0:4)coordinate (B)
				++ (30:2.3)coordinate (C)
				($(A)+(C)-(B)$)coordinate (D)
				\foreach \x in {A,B,C,D}{(\x)++(90:2.5) coordinate (\x_1)}
				;
				\draw[dashed]
				(A)--(D) node[pos=.5,sloped,above]{$x$}
				(D)--(C) node[pos=.4,sloped,above]{$2x$}
				(D)--(D_1) node[pos=.4, right]{$y$}
				;
				\draw
				(A)--(B)--(C)
				(A_1)--(B_1)--(C_1)--(D_1)--cycle
				(A)--(A_1) (B)--(B_1) (C)--(C_1)
				;
		\end{tikzpicture}}
	}
\end{ex} \dongcham{20}

\begin{ex}%[2D1T3-6]
	Người ta muốn xây một chiếc bể nước có hình dạng là	một khối hộp chữ nhật không nắp có thể tích bằng $\dfrac {500}{3}$ m$^3$. Biết đáy bể là một hình chữ nhật có chiều dài gấp đôi chiều rộng và giá thuê thợ xây là $700.000$ đồng/m$^2$. Để chi phí thuê nhân công ít nhất thì chi phí thuê nhân công là
	\choice
	{$120$ triệu đồng}	
	{\True $105$ triệu đồng}
	{$115$ triệu đồng}	
	{$110$ triệu đồng}
	\loigiai{
		Gọi $x,y$ lần lượt là chiều rộng và chiều cao của bể cá (điều kiện $x,y>0$ ).
		\immini{Với giả thiết của bài toán, thể tích bể cá là $$V=2x^2y=\dfrac {500}{3}\Rightarrow y=\dfrac {250}{3x^2}.$$
			Để chi phí thuê nhân công ít nhất thì tổng diện tích các mặt của bể cá phải nhỏ nhất. Tổng diện tích các mặt của bể cá} 
		{\begin{tikzpicture}[scale=0.8, font=\footnotesize, line join=round, line cap=round, >=stealth]
				\tkzDefPoints{0/0/A,-1.3/-1.1/B,2/-1.1/C}
				\coordinate (D) at ($(A)+(C)-(B)$);
				\coordinate (A') at ($(A)+(0,2.5)$);
				\tkzDefPointsBy[translation=from A to A'](B,C,D){B'}{C'}{D'}
				\tkzDrawPolygon(A',B',B,C,D,D')
				\tkzDrawSegments(B',C' C',D' C,C')
				\tkzDrawSegments[dashed](A,B A,D A,A')
				\tkzDrawPoints[fill=black](A,B,D,C,A',B',C',D')
				\tkzLabelSegment[left](B',B){$y$}
				\tkzLabelSegment[below](B,C){$2x$}
				\tkzLabelSegment[right](A,B){$x$}
		\end{tikzpicture}}
		$S=2xy+2\cdot 2xy+2x^2=6xy+2x^2=\dfrac {500}{x}+2x^2$.\\
		Xét hàm số $S(x)=\dfrac {500}{x}+2x^2$ trên khoảng $(0;+\infty)$.\\
		$\Rightarrow S'(x)=-\dfrac {500}{x^2}+4x$.\\
		$S'(x)=0\Leftrightarrow -500+4x^3=0\Leftrightarrow x=5$.\\
		Bảng biến thiên
		\begin{center}
			\begin{tikzpicture}[scale=1, font=\footnotesize, line join=round, line cap=round, >=stealth]
				\tkzTabInit[nocadre=false,lgt=1.2,espcl=2, deltacl=0.5]
				{$x$/0.6,$S’(x)$/0.6,$S(x)$/1.5}
				{$0$,$5$,$+\infty$}
				\tkzTabLine{,-,z,+,}
				\tkzTabVar{+/$+\infty$,-/$150$,+/$+\infty$}
			\end{tikzpicture}
		\end{center}
		Do đó $\min S=150$ tại $x=5$. \\
		Khi đó, chi phí thuê nhân công là $150\cdot 700000=105$ triệu đồng.\\Vậy chi phí thuê nhân công ít nhất là $105$ triệu đồng.}
\end{ex}
\dongcham{13}
\begin{ex}%[2D1V3-6]
	Từ một tấm bìa hình chữ nhật có chiều rộng $30 \mathrm{~cm}$ và chiều dài $80 \mathrm{~cm}$ (Hình a), người ta cắt ở bốn góc bốn hình vuông có cạnh $x(\mathrm{~cm})$ với $5 \leq x \leq 10$ và gấp lại để tạo thành chiếc hộp có dạng hình hộp chữ nhật không nắp như Hình b. Tìm $x$ để thể tích chiếc hộp là lớn nhất (kết quả làm tròn đến hàng phần trăm).
	\begin{center}
		\begin{tikzpicture}[line join=round, line cap=round,scale=0.9]
			\coordinate (A) at (0,3);
			\coordinate (B) at (5,3);
			\coordinate (D) at (0,0);
			\coordinate (C) at ($(B)+(D)-(A)$);
			\draw(A)--(B)--(C)--(D)--cycle;
			\draw (0,0) rectangle (1,1) (A) rectangle (1,2) (B) rectangle (4,2) (4,1) rectangle (C);
			\draw[dashed] (1,1) rectangle (4,2);
			%	\foreach \i/\g in {A/90,B/90,C/-90,D/-90}{\draw[fill=black](\i) circle (1pt) ($(\i)+(\g:3mm)$) node[scale=1]{$\i$};}
			\draw (0,.5) node [left] {$x$};
			\draw (.5,0) node [below] {$x$};
			\draw (0,2.5) node [left] {$x$};
			\draw (0.5,3) node [above] {$x$};
			%%%%%%%%%
			\draw (4.5,0) node [below] {$x$};
			\draw (5,0.5) node [right] {$x$};
			\draw (5,2.5) node [right] {$x$};
			\draw (4.5,3) node [above] {$x$};
			%%%%%%%%
			\draw[<->] (-1,0)--(-1,3) node[above,midway,sloped] {$30$cm};
			\draw[<->] (0,-1)--(5,-1) node[above,midway] {$80$cm};
			\path (current bounding box.south) node[below, black]{a)}; %dưới
		\end{tikzpicture}
		\hspace*{1cm}
		\begin{tikzpicture}[scale=0.9, font=\footnotesize, line join=round, line cap=round, >=stealth]
			\def\bc{3} % cạnh BC
			\def\ba{1} % cạnh BA
			\def\h{1.5} % đường cao
			\def\gocnghieng{90} % góc nghiêng
			\def\gocB{35} % góc B của đáy
			\coordinate (B) at (0,0);
			\coordinate (A) at (\gocB:\ba);
			\coordinate (C) at (\bc,0);
			\coordinate (D) at ($(C)-(B)+(A)$);
			\coordinate (A') at ($(A)+(\gocnghieng:\h)$);
			\coordinate (B') at ($(B)-(A)+(A')$);
			\coordinate (C') at ($(C)-(A)+(A')$);
			\coordinate (D') at ($(D)-(A)+(A')$);
			\draw (B')--(B)--(C)--(D)--(D')--(A')--(B')--(C')--(D') (C)--(C');
			\draw[dashed] (A')--(A)--(D) (A)--(B);
			\path (current bounding box.south) node[below, black]{b)}; %dưới
		\end{tikzpicture}
	\end{center}
	\choice
	{\True $x=\dfrac{20}{3} \mathrm{~cm}$}
	{$x=\dfrac{20}{7} \mathrm{~cm}$}
	{$x=\dfrac{25}{3} \mathrm{~cm}$}
	{$x=\dfrac{25}{7} \mathrm{~cm}$}
	\loigiai{
		Thể tích chiếc hộp là $V(x)=x(30-2 x)(80-2 x)=2400 x-220 x^2+4 x^3$ với $5 \leq x \leq 10$.\\
		Ta có: $V'(x)=12 x^2-440 x+2400$;\\
		$V'(x)=0 \Leftrightarrow x=\dfrac{20}{3}$ hoặc $x=30$ (loại vì không thuộc $[5 ; 10]$);
		\begin{center}
			$V(5)=7000 ; V\left(\dfrac{20}{3}\right)=\dfrac{200000}{27} ; V(10)=6000$.
		\end{center}
		Do đó $\max \limits_{[5 ; 10]} V(x)=\dfrac{200000}{27}$ khi $x=\dfrac{20}{3}$.
		Vậy để thể tích chiếc hộp là lớn nhất thì $x=\dfrac{20}{3} \mathrm{~cm}$.}
\end{ex}
\dongcham{13}
\begin{ex}%[2D1K3]
	Một sợi dây có chiều dài là $6$ m, được chia thành $2$ phần. Phần thứ nhất được uốn thành hình tam giác đều, phần thứ hai uốn thành hình vuông. Hỏi độ dài của cạnh hình tam giác đều bằng bao nhiêu để tổng diện tích $2$ hình thu được là nhỏ nhất?
	\begin{center}
		\begin{tikzpicture}[scale=0.8,>=stealth]
			\draw(0,0)--(7,0);
			\draw (0,0)circle (1pt)(7,0) circle (1pt)(3,0) circle (1pt);
			\draw[->](1.5,-0.3)--(1.5,-0.7);
			\draw[->](5,-0.3)--(5,-0.7);
			\draw(4.5,-0.9)--(5.5,-0.9)--(5.5,-1.9)--(4.5,-1.9)--(4.5,-0.9);
			\draw(1.5,-0.9)--(2,-1.9)--(1,-1.9)--(1.5,-0.9);
		\end{tikzpicture}
	\end{center}
	\choice
	{$\dfrac{12}{4+\sqrt{3}}$ m}
	{$\dfrac{18\sqrt{3}}{4+\sqrt{3}}$ m}
	{$\dfrac{36\sqrt{3}}{4+\sqrt{3}}$ m}
	{\True $\dfrac{18}{9+4\sqrt{3}}$ m}
	\loigiai{
		Gọi độ dài cạnh hình tam giác đều là $x$ (m). Khi đó độ dài cạnh hình vuông là $\dfrac{6-3x}{4}$.\\
		Tổng diện tích khi đó là $S =\dfrac{\sqrt{3}}{4}x^2 + \left(\dfrac{{6 - 3x}}{4}\right)^2 =\dfrac{1}{16}\left[\left(9+4\sqrt{3}\right)x^2 - 36x + 36 \right)]$.\\
		Xét hàm số $f(x)=\left(9+4\sqrt{3}\right)x^2-36x+36, x\in(0;6)$.\\
		Ta có $f(x)$ là tam thức bậc $2$ có $-\dfrac{b}{2a}=\dfrac{18}{9+4\sqrt{3}} \in (0;6)$ và $a>0$.\\
		Suy ra $f(x)$ đạt giá trị nhỏ nhất tại
		$x=-\dfrac{b}{2a}\dfrac{18}{9+4\sqrt{3}}$.\\
		Vậy diện tích nhỏ nhất khi $x=\dfrac{18}{9+4\sqrt{3}}$ m.
	}
\end{ex}
\dongcham{14}
\begin{ex}
	Một doanh nghiệp tư nhân $A$ chuyên kinh doanh xe gắn máy các loại. Hiện nay doanh nghiệp đang tập trung vào chiến lược kinh doanh xe $X$ với chi phí mua vào một chiếc là 27 triệu đồng và bán ra với giá 31 triệu đồng. Với giá bán này, số lượng xe mà khách hàng đã mua trong một năm là 600 chiếc. Nhằm mục tiêu đẩy mạnh hơn nữa lượng tiêu thụ dòng xe đang bán chạy này, doanh nghiệp dự định giảm giá bán. Bộ phận nghiên cứu thị trường ước tính rằng nếu giảm 1 triệu đồng mỗi chiếc xe thì số lượng xe bán ra trong một năm sẽ tăng thêm 200 chiếc. Hỏi theo đó, giá bán mới là bao nhiêu thì lợi nhuận thu được cao nhất?
	\choice
	{$30$ triệu đồng}
	{\True $30,5$ triệu đồng}
	{$29,5$ triệu đồng}
	{$32$ triệu đồng}
	\loigiai{
		Gọi giá bán mới là $x$ (triệu đồng) với $x \in [27;31]$.\\
		Khi đó số xe bán ra là $600+(31-x) \cdot 200$.\\
		Lợi nhuận thu được là 
		\begin{eqnarray*}
			f(x) &=& [600+(31-x) \cdot 200](x-27)\\
			&=& (-200x+6800)(x-27)\\
			&=& -200x^2+12200x-183600\\
			&=& -200\left(x-\dfrac{61}{2}\right)^2+2450\\
			&\leq&2450.
		\end{eqnarray*}
		Vậy giá bán mới là $30,5$ triệu đồng thì lợi nhuận thu được là lớn nhất là $2\,450$ (triệu đông).
	}
\end{ex}
\dongcham{14}
\Closesolutionfile{ans}
\ind{PHẦN II.} \inden{Câu trắc nghiệm đúng sai. Trong mỗi ý a), b), c), d) ở mỗi câu, học sinh chọn đúng hoặc sai.}\\
\Opensolutionfile{ans}[ans/2D1-B2-d3-2]

\begin{ex}
	Người ta bơm xăng vào bình xăng của một xe ô tô. Biết rằng thể tích $V$ (lít) của lượng xăng trong bình xăng tính theo thời gian bơm xăng $t$ (phút) được cho bởi công thức $$V(t)=300(t^2-t^3)+4 \text{ với } 0\le t\le 0{,}5.$$
Gọi $V'(t)$ là tốc độ tăng thể tích tại thời điểm $t$ với $0\le t\le 0{,}5$.
\choiceTF
{Lượng xăng trong bình ban đầu là $1$ lít}
{\True Lượng xăng lớn nhất bơm vào bình xăng là $41{,}5$ lít}
{$V'(t)=300(2t-3t^2)+4$, với $0\le t\le 0{,}5$}
{\True Xăng chảy vào bình xăng vào thời điểm ở giây thứ $30$ có tốc độ tăng thể tích là lớn nhất}
	\loigiai{
		\begin{enumerate}[a)]
			\item Số xăng trong bình ban đầu là $V(0)=4$ lít.
			\item Lượng xăng lớn nhất bơm vào bình xăng là $V=V\left(\dfrac{1}{2}\right)=41{,}5$ lít.
			\item Xét hàm số $V(t)=300(t^2-t^3)+4 \text{ với } 0\le t\le 0{,}5.$\\
			Đạo hàm $V'(t)=300t(2-3t)$.\\
			\item Cho $V'(t)=0 \Leftrightarrow 300t(t-3t)=0 \Leftrightarrow \hoac{&t=0\in[0;0{,}5]\\&t=\dfrac{2}{3}\notin[0;0{,5}].}$\\
			Các giá trị $V(0)=4$, $V\left(\dfrac{1}{2}\right)=41{,}5$.\\
			Xăng chảy vào bình xăng vào thời điểm ở giây thứ $30$ có tốc độ tăng thể tích là lớn nhất.
		\end{enumerate}
	}
\end{ex}
\dongcham{20}
\begin{ex}
	Tại một xí nghiệp chuyên sản xuất vật liệu xây dựng, nếu trong một ngày xí nghiệp sản xuất $x$ (m$^3$) sản phẩm thì phải bỏ ra các khoản chi phí bao gồm: $4$ triệu đồng chi phí cố định; $0{,}2$ triệu đồng chi phí cho mỗi mét khối sản phẩm và $0{,}001 x^2$ triệu đồng chi phí bảo dưỡng máy móc. Biết rằng, mỗi ngày xí nghiệp sản xuất được tối đa $100$ m$^3$ sản phẩm. Goi $C(x)$ là tổng chi phí để xí nghiệp sản xuất $x$ (m$^3$) sản phẩm trong một ngày và $\overline{C}$ là chi phí trung bình  trên mỗi mét khối sản phẩm.
	\choiceTF
	{$C=0{,}2 x+0{,}001 x^2 \quad \text { với } 0 \leq x \leq 100$}
	{\True Tổng chi phí khi sản xuất 100 m$^3$ sản phẩm là 34 triệu đồng}
	{\True $\overline{C}=0{,}001 x+\dfrac{4}{x}+0{,}2 \quad\text { với } 0<x \leq 100$}
	{\True $\overline{C}$ có giá trị thấp nhất bằng 0,326 triệu đồng (\textit{kết quả làm tròn 3 chữ số thập phân})}
	\loigiai{
		\begin{enumerate}
			\item Tổng chi phí (triệu đồng) để xí nghiệp sản xuất $x$ (m$^3$) sản phẩm trong một ngày là
			$$
			C=C(x)=4+0{,}2 x+0{,}001 x^2 \text { với } 0 \leq x \leq 100.
			$$
			\item Thay $x=100$ vào hàm $C(x)$, ta được kết quả 34 (triệu đồng).
			\item Chi phí trung bình (triệu đồng) trên mỗi mét khối sản phẩm là
			$$
			\overline{C}=\overline{C}(x)=\dfrac{C(x)}{x}=\dfrac{4+0{,}2 x+0{,}001 x^2}{x}=0{,}001 x+\dfrac{4}{x}+0{,}2 \text { với } 0<x \leq 100.
			$$
			\item Ta có $\bar{C}'(x)=0{,}001-\dfrac{4}{x^2}$;
			$$
			\overline{C}'(x)=0 \Leftrightarrow 0{,}001-\dfrac{4}{x^2}=0 \Leftrightarrow x^2=4\,000 \Leftrightarrow x=20 \sqrt{10} \in(0 ; 100].
			$$
			
			Ta có $\overline{C}(20 \sqrt{10})=\dfrac{\sqrt{10}}{25}+\dfrac{1}{5} \approx 0,326$.\\
			Bảng biến thiên
			\begin{center}
				\begin{tikzpicture}
					\tikzset{double style/.append style={double distance=2pt}}
					\tkzTabInit[lgt=1.2, espcl=2]
					{$x$/0.6,$\overline{C'}(x)$/0.6,$\overline{C}(x)$/2.5}{$0$,$20\sqrt{10}$,$100$}
					\tkzTabLine{,-,0,+,}
					\tkzTabVar{+/$+\infty$,-/$\dfrac{\sqrt{10}}{25}+\dfrac{1}{5}$,+/$0{,}34$}
				\end{tikzpicture}
			\end{center}
			Từ bảng biến thiên, ta thấy chi phí trung bình thấp nhất là $\bar{C}(20 \sqrt{10}) \approx 0{,}326$ (triệu đồng/m$^3$ sản phẩm), đạt được khi $x=20 \sqrt{10} \approx 63$ (m$^3$).
		\end{enumerate}
	}
\end{ex}
\dongcham{20}
\begin{ex}
	Nhà máy $A$ chuyên sản xuất một loại sản phẩm cung cấp cho nhà máy $B$. Hai nhà máy thoả thuận rằng, hằng tháng $A$ cung cấp cho $B$ số lượng sản phẩm theo đơn đặt hàng của $B$ (tối đa $100$ tấn sản phẩm). Nếu số lượng đặt hàng là $x$ tấn sản phẩm thì giá bán cho mỗi tấn sản phẩm là $P(x)=45-0{,}001 x^2$ (triệu đồng). Chi phí để $A$ sản xuất $x$ tấn sản phẩm trong một tháng là $C(x)=100+30 x$ (triệu đồng) (gồm $100$ triệu đồng chi phí cố định và $30$ triệu đồng cho mỗi tấn sản phẩm).
	\choiceTF
	{\True Chi phí để  A sản xuất 10 tấn sảm phẩm trong một tháng là 400 triệu đồng}
	{Số tiền  A thu được khi bán 10 tấn sản phẩm cho B là 600 triệu đồng}
	{\True Lợi nhuận mà A thu được khi bán $x$ tấn sản phẩm ($0\le x \le 100)$ cho  B là $-0{,}001 x^3+15 x-100$}
	{\True A bán cho $B$ khoảng 70,7 tấn sản phẩm mỗi tháng thì thu được lợi nhuận lớn nhất}
	\loigiai{
		\begin{enumerate}[a)]
			\item Chi phí để  A sản xuất 10 tấn sảm phẩm trong một tháng là $C(10)=100+30\cdot 10=400$ (triệu)
			\item Số tiền mà $A$ thu được (gọi là doanh thu) từ việc bán $x$ tấn sản phẩm $(0 \leq x \leq 100)$ cho $B$ là
			$$
			R(x)=x \cdot P(x)=x\left(45-0{,}001 x^2\right)=45 x-0{,}001 x^3 \text { (triệu đồng). }
			$$
			Thay $x=10$, ta được $R(10)=449$ (triệu đồng).
			\item Lợi nhuận (triệu đồng) mà $A$ thu được là
			$$
			P(x)=R(x)-C(x)=x\left(45-0{,}001 x^2\right)-(100+30 x)=-0{,}001 x^3+15 x-100.
			$$
			\item Xét hàm số $P(x)=-0{,}001 x^3+15 x-100$ với $0 \leq x \leq 100$, ta có
			$$
			\begin{aligned}
				& P'(x)=-0{,}003 x^2+15; \\
				& P'(x)=0 \Leftrightarrow-0{,}003 x^2+15=0 \Leftrightarrow x^2=5\,000 \Leftrightarrow x=50 \sqrt{2} \in[0 ; 100].
			\end{aligned}
			$$
			
			Ta có $P(0)=-100$; $P(50 \sqrt{2})=500 \sqrt{2}-100 \approx 607$; $P(100)=400$.\\
			Bảng biến thiên
			\begin{center}
				\begin{tikzpicture}
					\tkzTabInit[lgt=1, espcl=4]
					{$x$/1,$y'$/0.6,$y$/3}{$0$,$50\sqrt{2}$,$100$}
					\tkzTabLine{,+,0,-,}
					\tkzTabVar{-/$100$,+/$500\sqrt{2}-100$,-/$400$}
				\end{tikzpicture}
			\end{center}
			
			Từ bảng biến thiên, ta có $\max \limits_{[0 ; 100]} P=P(50 \sqrt{2})=500 \sqrt{2}-100 \approx 607$.\\
			Vậy $A$ thu được lợi nhuận lớn nhất khi bán $50 \sqrt{2} \approx 70{,}7$ tấn sản phẩm cho $B$ mỗi tháng và lợi nhuận lớn nhất thu được khoảng $607$ triệu đồng.
		\end{enumerate}
	}
\end{ex}
\dongcham{20}
\Closesolutionfile{ans}
%%Bài 3. Tiệm cận
% \setcounter{section}{2}
\section{ĐƯỜNG TIỆM CẬN CỦA ĐỒ THỊ HÀM SỐ}
\subsection{LÝ THUYẾT CẦN NHỚ}
\subsubsection{Đường tiệm cận ngang (TCN):}
\begin{enumerate}[\iconMT]
	\item \indam{Định nghĩa:} Đường thẳng $y=m$ được gọi là một \inden{đường tiệm cận ngang} (hay \inden{tiệm cận ngang}) của đồ thị hàm số $y=f(x)$ nếu 
	$$\lim\limits_{x \rightarrow-\infty} f(x)=m \text{ hoặc }\lim\limits_{x \rightarrow+\infty} f(x)=m.$$
Đường thẳng $y=m$ là tiệm cận ngang của đồ thị hàm số $y=f(x)$ được minh hoạ như hình bên dưới\\
	\begin{tikzpicture}[scale=1,>=stealth, font=\footnotesize, line join=round, line cap=round]
		\def\xmin{-4} \def\xmax{2}
		\def\ymin{-0.5} \def\ymax{3}
		%\draw[color=gray!50,dashed] (\xmin,\ymin) grid (\xmax,\ymax);
		\draw[->] (\xmin,0)--(\xmax,0) node [below]{$x$};
		\draw[->] (0,\ymin)--(0,\ymax) node [left]{$y$};
		\fill (0,0) circle (1pt) node[shift={(-135:2.5mm)}]{$O$};
		\node at (current bounding box.south) [below=-2pt] {a) $\lim\limits_{x \rightarrow-\infty} f(x)=m$};
		\clip (\xmin+0.1,\ymin+0.1) rectangle (\xmax-0.1,\ymax-0.1);
		\draw[red,thick,smooth,samples=300,domain=\xmin:\xmax]
		(-4,0.9)..controls +(0:2) and +(180:0.5)
		..(-1.5,0.5)..controls +(0:0.5) and +(180:0.5)
		..(-0.3,1.4)..controls +(0:0.5) and +(135:1)
		..(1.8,0.3);
		\draw [blue](\xmin,1)--(\xmax,1);
		\path[blue] (-3,1)node[above]{$y=m$};
		\path[red] (0,1.3)node[above left]{$y=f(x)$};
		\fill (0,1) circle (1pt) node[shift={(-135:3mm)}]{$m$};
	\end{tikzpicture}\hspace*{.5cm}
	\begin{tikzpicture}[scale=1,>=stealth, font=\footnotesize, line join=round, line cap=round]
		\def\xmin{-1.5} \def\xmax{4}
		\def\ymin{-0.5} \def\ymax{3}
		%\draw[color=gray!50,dashed] (\xmin,\ymin) grid (\xmax,\ymax);
		\draw[->] (\xmin,0)--(\xmax,0) node [below]{$x$};
		\draw[->] (0,\ymin)--(0,\ymax) node [left]{$y$};
		\fill (0,0) circle (1pt) node[shift={(-135:2.5mm)}]{$O$};
		\node at (current bounding box.south) [below=-2pt] {b) $\lim\limits_{x \rightarrow+\infty} f(x)=m$};
		\clip (\xmin+0.1,\ymin+0.1) rectangle (\xmax-0.1,\ymax-0.1);
		\draw[red,thick,smooth,samples=300,domain=\xmin:\xmax]
		(-1,3)..controls +(-80:1) and +(170:1)
		..(0.5,1.1)..controls +(170:-1) and +(180:-0.5)
		..(3.9,0.8);
		\draw [blue](\xmin,0.7)--(\xmax,0.7);
		\path[blue] (4,0.7)node[below left]{$y=m$};
		\path[red] (0.5,1)node[above right]{$y=f(x)$};
		\fill (0,0.7) circle (1pt) node[shift={(-135:3mm)}]{$m$};
	\end{tikzpicture}
	\item \indam{Các bước tìm TCN:}
	\begin{boxdn}
		\begin{listEX}[1]
			\item [\ding{172}] Tính $\lim \limits_{x \to +\infty} f(x)$ và $\lim \limits_{x \to -\infty} f(x)$.
			\item [\ding{173}] Xem ở "vị trí" nào ra kết quả hữu hạn thì ta kết luận có tiệm cận ngang ở "vị trí" đó.
		\end{listEX}
	\end{boxdn}
\end{enumerate}
\subsubsection{Đường tiệm cận đứng (TCĐ)}
\begin{enumerate}[\iconMT]
	\item \indam{Định nghĩa:}	Đường thẳng $x=a$ được gọi là một \inden{đường tiệm cận đứng} (hay \inden{tiệm cận đứng}) của đồ thị hàm số $y=f(x)$ nếu ít nhất một trong các điều kiện sau thoả mãn:		
	$$
	\lim\limits_{x \rightarrow a^{-}} f(x)=+\infty,\,\, \lim\limits_{x \rightarrow a^{+}} f(x)=+\infty,\,\, \lim\limits_{x \rightarrow a^{-}} f(x)=-\infty,\,\, \lim\limits_{x \rightarrow a^{+}} f(x)=-\infty \text {. }
	$$
		Đường thẳng $x=a$ là tiệm cận đứng của đồ thị hàm số $y=f(x)$ được minh hoạ như hình bên dưới.\\
		\begin{center}
		\begin{tikzpicture}[scale=.7,>=stealth, font=\footnotesize, line join=round, line cap=round]
			%Hình a
			\def\xmin{-2.2} \def\xmax{3.5}
			\def\ymin{-2} \def\ymax{2} 
			%\draw[color=gray!50,dashed] (\xmin,\ymin) grid (\xmax,\ymax); 
			\draw[->] (\xmin,0)--(\xmax,0) node [below]{$x$};
			\draw[->] (0,\ymin)--(0,\ymax) node [left]{$y$};
			\fill (0,0) circle (1pt) node[shift={(-45:2.5mm)}]{$O$};
			\draw (2.1,\ymin)--(2.1,\ymax)node[below right]{$x=a$};
			\fill (2.1,0) circle (1pt) node[shift={(-45:3mm)}]{$a$};
			%\clip (\xmin+0.1,\ymin+0.1) rectangle (\xmax-0.1,\ymax-0.1);
			\draw[red] (-2,-1)..controls +(80:0.5) and +(0:-.5)..(-1,0.5)node[above]{$y=f(x)$}
			..controls +(0:0.5) and +(180:0.5)..(0.5,-1.5)
			..controls +(0:0.5) and +(87:-0.2)..(1.6,0)
			..controls +(87:-.2) and +(90:-0.2)
			..(2,1.85);
			\node at (current bounding box.south) [below=-2pt] {a) $\lim\limits_{x \rightarrow a^{-}} f(x)=+\infty$};
		\end{tikzpicture}
		\begin{tikzpicture}[scale=.7,>=stealth, font=\footnotesize, line join=round, line cap=round]
			%Hình b
			\def\xmin{-1.2} \def\xmax{4}
			\def\ymin{-2} \def\ymax{2} 
			%\draw[color=gray!50,dashed] (\xmin,\ymin) grid (\xmax,\ymax); 
			\draw[->] (\xmin,0)--(\xmax,0) node [below]{$x$};
			\draw[->] (0,\ymin)--(0,\ymax) node [left]{$y$};
			\fill (0,0) circle (1pt) node[shift={(-45:2.5mm)}]{$O$};
			\draw (1,\ymin)node[above right]{$x=a$}--(1,\ymax);
			\fill (1,0) circle (1pt) node[shift={(-135:3mm)}]{$a$};
			\path[red] (1.25,1)node[above right]{$y=f(x)$};
			%\clip (\xmin+0.1,\ymin+0.1) rectangle (\xmax-0.1,\ymax-0.1);
			\draw[red] (1.2,2)..controls +(80:0) and +(0:-1.4)..(2.5,-0.8)
			..controls +(0:0.1) and +(-80:-0.6)
			..(3.5,-1.5);
			\node at (current bounding box.south) [below=-2pt] {b) $\lim\limits_{x \rightarrow a^{+}} f(x)=+\infty$};		
		\end{tikzpicture}\\
		\begin{tikzpicture}[scale=.7,>=stealth, font=\footnotesize, line join=round, line cap=round]
			%Hình c
			\def\xmin{-2.2} \def\xmax{3.5}
			\def\ymin{-2} \def\ymax{2} 
			%\draw[color=gray!50,dashed] (\xmin,\ymin) grid (\xmax,\ymax); 
			\draw[->] (\xmin,0)--(\xmax,0) node [below]{$x$};
			\draw[->] (0,\ymin)--(0,\ymax) node [left]{$y$};
			\fill (0,0) circle (1pt) node[shift={(-45:2.5mm)}]{$O$};
			\draw (2,\ymin)--(2,\ymax)node[below right]{$x=a$};
			\fill (2,0) circle (1pt) node[shift={(-45:3mm)}]{$a$};
			\path[red] (-2.25,1.2)node[below right]{$y=f(x)$};
			%\clip (\xmin+0.1,\ymin+0.1) rectangle (\xmax-0.1,\ymax-0.1);
			\draw[red] (-2,1.4)..controls +(-10:-0.2) and +(-55:-.7)
			..(1.3,0.65)..controls +(-50:0.4) and +(-90:0)
			..(1.8,-2)
			;
			\node at (current bounding box.south) [below=-2pt] {c) $\lim\limits_{x \rightarrow a^{-}} f(x)=-\infty$};		
		\end{tikzpicture}
		\begin{tikzpicture}[scale=.7,>=stealth, font=\footnotesize, line join=round, line cap=round]
			%Hình d
			\def\xmin{-2.2} \def\xmax{3.5}
			\def\ymin{-2} \def\ymax{2} 
			%\draw[color=gray!50,dashed] (\xmin,\ymin) grid (\xmax,\ymax); 
			\draw[->] (\xmin,0)--(\xmax,0) node [below]{$x$};
			\draw[->] (0,\ymin)--(0,\ymax) node [left]{$y$};
			\fill (0,0) circle (1pt) node[shift={(-135:2.5mm)}]{$O$};
			\draw (.6,\ymin)--(.6,\ymax)node[below right]{$x=a$};
			\fill (.6,0) circle (1pt) node[shift={(-135:3mm)}]{$a$};
			%\clip (\xmin+0.1,\ymin+0.1) rectangle (\xmax-0.1,\ymax-0.1);
			\draw[red] (0.7,-2)..controls +(85:0.2) and +(180:0.2)
			..(1.2,-0.3)..controls +(0:0.2) and +(180:0.2)
			..(1.7,-0.6)..controls +(0:0.4) and +(90:0)
			..(2.5,2)
			;
			\node at (current bounding box.south) [below=-2pt] {d) $\lim\limits_{x \rightarrow a^{+}} f(x)=-\infty$};		
		\end{tikzpicture}
	\end{center}
	\item \indam{Các bước tìm TCĐ:}
	\begin{boxdn}
		\begin{listEX}[1]
			\item [\ding{172}] Tìm nghiệm của mẫu, giả sử nghiệm đó là $x=x_0$.
			\item [\ding{173}] Tính giới hạn một bên tại $x_0$. Nếu xảy ra $\lim \limits_{x \to x_0^{-}} f(x) =\infty \text{ hoặc} \lim \limits_{x \to x_0^{+}} f(x) =\infty$
			thì ta kết luận $x=x_0$ là đường tiệm cận đứng.
		\end{listEX}
	\end{boxdn}
\end{enumerate}
\subsubsection{Đường tiệm cận xiên}
\begin{enumerate}[\iconMT]
	\item \indam{Định nghĩa:} Đường thẳng $y=ax+b$, $a \neq 0$, được gọi là \inden{đường tiệm cận xiên} (hay \inden{tiệm cận xiên}) của đồ thị hàm số $y=f(x)$ nếu 
	$$\lim\limits_{x \rightarrow-\infty}[f(x)-(ax+b)]=0 \text{ hoặc }\lim\limits_{x \rightarrow+\infty}[f(x)-(ax+b)]=0.$$
	Đường thẳng $y=ax+b$ là tiệm cận xiên của đồ thị hàm số $y=f(x)$ được minh hoạ như hình bên dưới:\\	
		\begin{tikzpicture}[scale=1,>=stealth, font=\footnotesize, line join=round, line cap=round]
			\def\xmin{-4} \def\xmax{2.5}
			\def\ymin{-0.5} \def\ymax{3}
			%\draw[color=gray!50,dashed] (\xmin,\ymin) grid (\xmax,\ymax);
			\draw[->] (\xmin,0)--(\xmax,0) node [below]{$x$};
			\draw[->] (0,\ymin)--(0,\ymax) node [left]{$y$};
			\fill (0,0) circle (1pt) node[shift={(-135:2.5mm)}]{$O$};
			\node at (current bounding box.south) [below=-2pt] {a) $\lim\limits_{x \rightarrow-\infty}\left[f(x)-(ax+b)\right]=0$};
			\clip (\xmin+0.1,\ymin+0.1) rectangle (\xmax-0.1,\ymax-0.1);
			\draw[red,thick,smooth,samples=300,domain=\xmin:\xmax]
			(-3.8,-0.6)..controls +(34:0.5) and +(180:.75)
			..(-0.2,1.2)..controls +(0:0.75) and +(180:.75)
			..(1,0.3)..controls +(0:0.5) and +(80:0)
			..(2.2,1);
			\draw[blue,smooth,samples=300,domain=\xmin:\xmax] plot(\x,{2/3*(\x)+2});
			\path[blue] (-3,0)--(0,2)node[below,sloped,pos=1.3]{$y=ax+b$};
			\path[red] (0.5,1)node[above right]{$y=f(x)$};
		\end{tikzpicture}\hspace{.5cm}
		\begin{tikzpicture}[scale=1,>=stealth, font=\footnotesize, line join=round, line cap=round]
			\def\xmin{-3.5} \def\xmax{3}
			\def\ymin{-0.5} \def\ymax{3}
			%\draw[color=gray!50,dashed] (\xmin,\ymin) grid (\xmax,\ymax);
			\draw[->] (\xmin,0)--(\xmax,0) node [below]{$x$};
			\draw[->] (0,\ymin)--(0,\ymax) node [left]{$y$};
			\fill (0,0) circle (1pt) node[shift={(-135:2.5mm)}]{$O$};
			\node at (current bounding box.south) [below=-2pt] {a) $\lim\limits_{x \rightarrow+\infty}\left[f(x)-(ax+b)\right]=0$};
			\clip (\xmin+0.1,\ymin+0.1) rectangle (\xmax-0.1,\ymax-0.1);
			\draw[red,thick,smooth,samples=300,domain=\xmin:\xmax]
			(-3,0.8)..controls +(60:0.5) and +(180:.75)
			..(-1.5,2)..controls +(0:.5) and +(180:.75)
			..(0.5,1.3)..controls +(0:.75) and +(-160:.5)
			..(2.8,1.8);
			\draw[blue,smooth,samples=300,domain=\xmin:\xmax] plot(\x,{1/3*(\x)+0.75});
			\path[blue] (-3,-0.25)--(0,0.75)node[below,sloped,pos=1.6]{$y=ax+b$};
			\path[red] (-2.5,2)node[above right]{$y=f(x)$};
		\end{tikzpicture}
	\item \indam{Các bước tìm TCX y = ax + b:}
	Ta xác định hệ số của $a$ và $b$ trong 2 trường hợp sau:
	\begin{boxdn}
		\begin{listEX}[1]
			\item [\ding{172}] Tính $a=\lim\limits_{x \rightarrow+\infty} \dfrac{f(x)}{x}$, $b=\lim\limits_{x \rightarrow+\infty}[f(x)-ax]$.
			\item [\ding{173}] Tính $a=\lim\limits_{x \rightarrow-\infty} \dfrac{f(x)}{x}$, $b=\lim\limits_{x \rightarrow-\infty}[f(x)-ax]$.
		\end{listEX}
	\end{boxdn}
\end{enumerate}
\subsection{PHÂN LOẠI VÀ PHƯƠNG PHÁP GIẢI TOÁN}
\begin{dang}{Bài toán tìm tiệm cận đứng và tiệm cận ngang của đồ thị hàm số}
	Cho hàm số $y=f(x)$. Để tìm tiệm cận đứng và tiệm cận ngang, ta làm như sau:
	\begin{enumerate}[\iconCH]
		\item \indamm{Các bước tìm tiệm cận đứng:}
		\begin{listEX}[1]
			\item [\ding{172}] Tìm nghiệm của mẫu, giả sử nghiệm đó là $x=x_0$.
			\item [\ding{173}] Tính giới hạn một bên tại $x_0$. Nếu xảy ra $\lim \limits_{x \to x_0^{-}} f(x) =\infty \text{ hoặc} \lim \limits_{x \to x_0^{+}} f(x) =\infty$
			thì ta kết luận $x=x_0$ là đường tiệm cận đứng.
		\end{listEX}
		\item \indamm{Các bước tìm tiệm cận ngang:}
		\begin{listEX}[1]
			\item [\ding{172}] Tính $\lim \limits_{x \to +\infty} f(x)$ và $\lim \limits_{x \to -\infty} f(x)$.
			\item [\ding{173}] Xem ở "vị trí" nào ra kết quả hữu hạn thì ta kết luận có tiệm cận ngang ở "vị trí" đó.
		\end{listEX}
		\item \indamm{Lưu ý:} Đồ thị hàm số $y=\dfrac{ax+b}{cx+d}$ luôn có TCĐ $x=-\dfrac{d}{c}$ và TCN: $y=\dfrac{a}{c}$.
	\end{enumerate}
\end{dang}
\boxmini{BÀI TẬP TỰ LUẬN}
\begin{vd}
	Xác định tiệm cận đứng và tiệm cận ngang của đồ thị hàm số cho bởi công thức sau:
	\begin{enumEX}[a)]{4}
		\item $y=\dfrac{2x-1}{x+1}$;
		\item $y=\dfrac{2 x-3}{1-2 x}$;
		\item $y=\dfrac{x^2-5x+4}{x^2-1}$;
		\item $y=\dfrac{2x-1}{x^2-3x+2}$.
	\end{enumEX}
\loigiai{
\begin{enumerate}[a)]
	\item Xét $\lim\limits_{x \to -1^+} \dfrac{2x-1}{x+1}=-\infty$ (hoặc $\lim\limits_{x \to -1^-} \dfrac{2x-1}{x+1}=+\infty$) nên đường thẳng $x=-1$ là tiệm cận đứng.\\
	Xét $\lim\limits_{x \to \pm \infty } \dfrac{2x-1}{x+1}=2$ nên đường thẳng $y=2$ là tiệm cận ngang.
	\item Ta có
	\begin{itemize}
		\item $\lim\limits_{x \to \pm\infty} y=\lim\limits_{x \to \pm\infty} \dfrac{2x-3}{1-2x}=-1$ suy ra $y=-1$ là tiệm cận ngang.
		\item $\heva{& \lim\limits_{x \to \tfrac{1}{2}^+} \dfrac{2x-3}{1-2x}=+\infty \\ & \lim\limits_{x \to \tfrac{1}{2}^-} \dfrac{2x-3}{1-2x}=-\infty}$ suy ra $x=\dfrac{1}{2}$ là tiệm cận đứng.
	\end{itemize}
	\item Điều kiện xác định: $\heva{&x\neq-1\\ &x\neq1.}$
	\begin{itemize}
		\item $\lim\limits_{x\to\pm\infty}\dfrac{x^2-5x+4}{x^2-1}=1$
		\item $\lim\limits_{x\to(-1)^-}\dfrac{x^2-5x+4}{x^2-1}=+\infty$
		\item $\lim\limits_{x\to1}\dfrac{x^2-5x+4}{x^2-1}=-\dfrac{3}{2}$
	\end{itemize}
	Vậy đồ thị hàm số có một tiệm cận ngang $y=1$ và một tiệm cận đứng $x=-1$.
	\item Tập xác định $\mathscr{D}=\mathbb{R}\setminus\{1; 2\}$.\\
	Ta có\begin{itemize}
		\item $\lim\limits_{x\to\pm\infty}y=\lim\limits_{x\to\pm\infty}\dfrac{2x-1}{x^2-3x+2}=0$ nên $y=0$ là đường tiệm cận ngang.
		\item $\lim\limits_{x\to 1^-}y=\lim\limits_{x\to 1^-}\dfrac{2x-1}{x^2-3x+2}=\lim\limits_{x\to 1^-}\dfrac{2x-1}{(x-1)(x-2)}=+\infty$ và $\lim\limits_{x\to 1^+}y=-\infty$ nên $x=1$ là đường tiệm cận đứng.
		\item $\lim\limits_{x\to 2^-}y=\lim\limits_{x\to 2^-}\dfrac{2x-1}{x^2-3x+2}=\lim\limits_{x\to 2^-}\dfrac{2x-1}{(x-1)(x-2)}=-\infty$ và $\lim\limits_{x\to 2^+}y=2=+\infty$ nên $x=2$ là đường tiệm cận đứng.
	\end{itemize}
\end{enumerate}}
\end{vd}

\boxmini{BÀI TẬP TRẮC NGHIỆM}
\ind{PHẦN I.} \inden{Câu trắc nghiệm nhiều phương án lựa chọn. Mỗi câu hỏi học sinh chỉ chọn một phương án.}\\
\setcounter{ex}{0}
\Opensolutionfile{ans}[ans/2D1-B3-d1-1]
\begin{ex}
	Đường tiệm cận ngang của đồ thị hàm số $y=\dfrac{2x-4}{x+2}$ là
	\choice
	{\True $y=2$}
	{  $x=2$}
	{ $x=-2$}
	{$y=-2$}
	
	\loigiai{$\underset{x\to -\infty }{\mathop{\lim \limits_{n \to +\infty}}}\,\dfrac{2x-4}{x+2}=2$ và $\underset{x\to +\infty }{\mathop{\lim \limits_{n \to +\infty}}}\,\dfrac{2x-4}{x+2}=2$ nên hàm số có tiệm cận ngang là $y=2$.
	}
\end{ex}

\begin{ex}
	Tìm tiệm cận ngang của đồ thị hàm số $ y = \dfrac{2x + 1}{ x +1} $.
	\choice
	{ \True $  y = -2 $}
	{$  x = -2 $}
	{ $  y = 2 $}
	{$ x = 1 $}
	\loigiai{
		Ta có $ \displaystyle \lim_{ x \rightarrow \pm \infty } \dfrac{2x + 1}{-x + 1} = -2  $.	
	}		
\end{ex}

\begin{ex}
	Đường thẳng $y=3$ là tiệm cận ngang của đồ thị hàm số nào sau đây?
	\choice
	{$y=\dfrac{1-3x}{2+x}$}
	{$y=\dfrac{x^2+3x+2}{x-2}$}
	{\True $y=\dfrac{1+3x}{1+x}$}
	{$y=\dfrac{3x^2+2}{2-x}$}
	\loigiai{
		Ta có $\lim\limits_{x\to \pm \infty}\dfrac{1+3x}{1+x}=3$ nên $y=3$ là tiệm cận ngang của đồ thị hàm số $y=\dfrac{1+3x}{1+x}$.}
\end{ex}

\begin{ex}
	Hàm số nào có đồ thị nhận đường thẳng $x = 2$ làm đường tiệm cận đứng?
	\choice
	{$y=x-2+\dfrac{1}{x+1}$}
	{$y=\dfrac{1}{x+1}$}
	{$y=\dfrac{2}{x+2}$}
	{\True $y=\dfrac{5x}{2-x}$}
	\loigiai{ Xét hàm số $y=\dfrac{5x}{2-x}$\\
		Ta có $\lim\limits_{x\to 2^+}5x=10>0$; $\lim\limits_{x\to 2^+}(2-x)$ và $x-2<0$ khi $x>2$ suy ra $\lim\limits_{x\to 2^+}\dfrac{5x}{2-x}=-\infty$.\\
		Vậy đồ thị hàm số $y=\dfrac{5x}{2-x}$ nhận đường thẳng $x=2$ làm tiệm cận đứng.
	}
\end{ex}

\begin{ex}
	Đường tiệm cận đứng của đồ thị hàm số $y=\dfrac{3x+1}{x-2}$ là đường thẳng
	\choice
	{$x=-2$}
	{\True $x=2$}
	{$y=3$}
	{$y=-\dfrac{1}{2}$}
	\loigiai{Ta có: $\lim \limits_{x\to 2^+}{\dfrac{3x+1}{x-2}}=+\infty$.
	}
\end{ex}

\begin{ex}
	Đường tiệm cận đứng của đồ thị hàm số $y=\dfrac{x+1}{x^2+4x-5}$ có phương trình là
	\choice
	{$x=-1$}
	{$y=1;y=-5$}
	{\True $x=1;x=-5$}
	{$x=\pm 5$}
	\loigiai{
		Ta có $\mathop{\lim}\limits_{x\rightarrow 1^+}y=+\infty$, $\mathop{\lim}\limits_{x\rightarrow 1^-}y=-\infty$, $\mathop{\lim}\limits_{x\rightarrow 5^+}y=+\infty$, $\mathop{\lim}\limits_{x\rightarrow 5^-}y=-\infty$.\\
		Vậy đồ thị hàm số có hai đường tiệm cận đứng là $x=1$ và $x=-5$.}
\end{ex}

\begin{ex}
	Tìm số đường tiệm cận của đồ thị hàm số $ y = \dfrac{x^2 - 3x + 2}{x^2 - 4}. $
	\choice
	{$1$}
	{$ 0$}
	{\True $2$}
	{$3$}
	\loigiai
	{
		Tập xác định: $ \mathscr D = \mathbb{R} \backslash \{\pm2 \} $.\\
		Ta có $ \lim \limits_{x \to \pm  \infty} y = 1 \Rightarrow  $ đồ thị hàm số có 1 tiệm cận ngang là $ y = 1. $\\
		Ta lại có $\lim \limits_{x \to 2} y =  \lim \limits_{x \to 2} \dfrac{x-1}{x+2} = \dfrac{1}{4} $ và $\lim \limits_{x \to -2^+} y =  \lim \limits_{x \to -2^+} \dfrac{x-1}{x+2} = -\infty$ nên đồ thị hàm số có 1 tiệm cận đứng là $ x = -2. $\\
		Vậy đồ thị hàm số đã cho có 2 đường tiệm cận.
	}
\end{ex}

\begin{ex}
	Số đường tiệm cận của đồ thị hàm số $y=\dfrac{3}{x-2}$ là
	\choice
	{$1$}
	{\True $2$}
	{$0$}
	{$3$}
	\loigiai{
		Tiệm cận đứng $x=2$.\\
		Tiệm cận ngang $y=0$.
	}
\end{ex}

\begin{ex}
	Cho hàm số $y=f(x)$ có đồ thị là đường cong $(C)$ và các giới hạn $\lim\limits_{x\to 2^{+}}f(x)=1$, $\lim\limits_{x\to 2^{-}}f(x)=1$, $\lim\limits_{x\to +\infty}f(x)=2$, $\lim\limits_{x\to -\infty}f(x)=2$. Hỏi mệnh đề nào sau đây đúng?
	\choice
	{\True Đường thẳng $y=2$ là tiệm cận ngang của $(C)$}
	{Đường thẳng $y=1$ là tiệm cận ngang của $(C)$}
	{Đường thẳng $x=2$ là tiệm cận ngang của $(C)$}
	{Đường thẳng $x=2$ là tiệm cận đứng của $(C)$}
	\loigiai{
		Ta có $\lim\limits_{x\to +\infty}f(x)=2$, $\lim\limits_{x\to -\infty}f(x)=2\Rightarrow y=2$ là tiệm cận ngang của $(C)$.
	}
\end{ex}

\begin{ex}
	Số tiệm cận đứng của đồ thị hàm số $y=\dfrac{\sqrt{x+9}-3}{x^2+x}$ là
	\choice
	{$3$}
	{$2$}
	{$0$}
	{\True $1$}
	\loigiai{
		Tập xác định $\mathscr{D}=[-9;+\infty)\setminus \{-1;0\}$. \\
		Ta có $\left\{\begin{aligned}
			&\lim\limits_{x\to -1^+} \dfrac{\sqrt{x+9}-3}{x^2+x}=+\infty \\
			&\lim\limits_{x\to -1^-} \dfrac{\sqrt{x+9}-3}{x^2+x}=-\infty
		\end{aligned}\right. \Rightarrow x=-1$ là tiệm cận đứng. \\
		Ngoài ra $\lim\limits_{x\to 0} \dfrac{\sqrt{x+9}-3}{x^2+x}=\dfrac{1}{6}$ nên $x=0$ không thể là một tiệm cận được.}
\end{ex} 

\begin{ex}%[2D1B4]
	\immini{Cho hàm số $y=f(x)$ xác định trên $\mathbb{R}\setminus\left\{\pm1\right\}$ liên tục trên mỗi khoảng xác định và có bảng biến thiên như hình vẽ. Số đường tiệm cận của đồ thị hàm số là
	\choice
	{$1$}
	{$2$}
	{\True $3$}
	{$4$}}{\hspace{0.5cm}
\begin{tikzpicture}
	\tikzset{double style/.append style = {draw=\tkzTabDefaultWritingColor,double=\tkzTabDefaultBackgroundColor,double distance=2pt}}
	\tikzset{double style/.append style = {double distance=0.5pt}} 
	\tkzTabInit[nocadre=false,lgt=1,espcl=1.7]
	{$x$/.7,$y'$ /.7, $y$ /2.3}
	{$-\infty$ ,$-1$,$0$,$1$,$+\infty$}
	\tkzTabLine{,-,d,-,0,+,d,+,}
	\tkzTabVar {+/$-2$,-D+/$-\infty$/$+\infty$,-/$1$,+D-/$+\infty$/$-\infty$,+/$-2$}
\end{tikzpicture}}
	\loigiai{
		Dựa vào bảng biến thiên ta có:\\
		$\lim\limits_{x\to -1^\pm}f(x)=\pm\infty$. 
		$\lim\limits_{x\to 1^\pm}f(x)=\mp\infty$.\\
		Do đó $x=1$ và $x=-1$ là các đường tiệm cận đứng của đồ thị hàm số.\\
		Lại có $\lim\limits_{x\to \pm\infty}f(x)=-2$. Do đó $y=-2$ là tiệm cận ngang của đồ thị hàm số.\\
		Vậy đồ thị hàm số có $3$ đường tiệm cận.
	}
\end{ex}

\begin{ex}
	\immini{Cho hàm số $y=f(x)$ xác định trên $\mathbb{R}\backslash \left\{0\right\},$ liên tục trên mỗi khoảng xác định và có bảng biến thiên như hình bên. Chọn khẳng định đúng.
	\choice
	{Đồ thị hàm số có đúng một tiệm cận ngang}
	{Đồ thị hàm số có hai tiệm cận ngang}
	{\True Đồ thị hàm số có đúng một tiệm cận đứng}
	{Đồ thị hàm số không có tiệm đứng và tiệm cận ngang}}{
	\begin{tikzpicture}[>=stealth]
		\tikzset{double style/.append style = {draw=\tkzTabDefaultWritingColor,double=\tkzTabDefaultBackgroundColor,double distance=2pt}}
		\tkzTabInit[nocadre=false,lgt=1,espcl=2]{$x$/.6,$y'$/.7,$y$/2}{$-\infty$,$0$,$1$,$+\infty$}
		\tkzTabLine{,-, d ,+,z,-,} 
		\tkzTabVar{+/$+\infty$ / , -D- / $-1$ /$-\infty$,+/$2$,-/$-\infty$}
\end{tikzpicture}}
	\loigiai{
		Do $\lim\limits_{x \to +\infty} y=-\infty$ và $\lim\limits_{x \to -\infty} y=+\infty$  nên đồ thị hàm số không có tiệm cận ngang.\\
		Do $\lim\limits_{x \to 0^+} y=+\infty$ suy ra $x=0$ là tiệm cận đứng của đồ thị hàm số.
	} 
\end{ex}

\begin{ex}
	\immini{Cho hàm số $ y=f(x) $ có bảng biến thiên như hình bên. Hỏi đồ thị hàm số đã cho có bao nhiêu đường tiệm cận?
	\choice
	{\True $ 2 $}
	{$ 3 $}
	{$ 4 $}
	{$ 1 $}}{
\begin{tikzpicture}[yscale=.8,xscale=1.15,
	kxd/.pic={\draw[double] (90:.4)--(-90:.4);}]
	\begin{scope}[shift={(-.5,.5)}]
		\fill[pattern=north east lines,pattern color=black]
		(1,-1) rectangle +(1.45,-4);
		\draw
		(0,0) rectangle +(7,-5)
		(0,-1)--+(0:7) (0,-2)--+(0:7) (1,0)--+(-90:5);
	\end{scope}
	\path
	(0,0) node{$ x $}
	++ (0:1) node{$ -\infty $}
	++(0:1)node{$ -2 $}
	++(0:2)node{$ 0 $}
	++(0:2)node{$ +\infty $}
	(0,-1)node{$ y' $}
	++(0:2)pic{kxd}
	++(0:1)node{$ + $}
	++(0:1)pic{kxd}
	++(0:1)node{$ - $}
	(0,-3)node{$ y $}
	++(0:2)pic[yscale=3]{kxd}
	+(-90:1)node[below right](A){$ -\infty $}
	++(0:2) pic[yscale=3]{kxd}
	node[above right](C){$ 1 $}
	+(90:1)node[left](B){$ 2 $}
	++(0:2)node[below](D){$ 0 $};
	\draw[-stealth,black](A)--(B)
	;
	\draw[-stealth,black] (C)--(D);
\end{tikzpicture}}
	\loigiai{
		Dựa vào bảng biến thiên của hàm số, suy ra
		\begin{itemize}
			\item  $ \lim\limits_{x \to +\infty} f(x)=0 $, đồ thị hàm số có tiệm cận ngang là $ y=0 $.
			\item $ \lim\limits_{x \to (-2)^+} f(x)=-\infty $, đồ thị hàm số có tiệm cận đứng là $ x=-2 $.
			Vậy đồ thị hàm số đã cho có $ 2 $ đường tiệm cận.
		\end{itemize}
	}
\end{ex}

\Closesolutionfile{ans}

\ind{PHẦN II.} \inden{Câu trắc nghiệm đúng sai. Trong mỗi ý a), b), c), d) ở mỗi câu, học sinh chọn đúng hoặc sai.}\\
\Opensolutionfile{ans}[ans/2D1-B3-d1-2]
\begin{ex}
	Cho hàm số $y=f(x)$ có bảng biến thiên như hình bên. Xét tính đúng, sai của các khẳng định sau:
	\begin{center}
		\begin{tikzpicture}
			\tikzset{double style/.append style = {draw=\tkzTabDefaultWritingColor,double=\tkzTabDefaultBackgroundColor,double distance=2pt}}
			\tkzTab[nocadre=false,lgt=1.2,espcl=1.7,deltacl=0.6]
			{$x$/0.6, $y'$/0.6, $y$/2}
			{$-\infty$, $0$, $2$, $+\infty$}
			{,-,d,-,$0$,+,}
			{+/ $2$, -D+/ $-\infty$ / $+\infty$, -/ $2$,+/$+\infty$}
		\end{tikzpicture}
	\end{center}
	\choiceTF
	{\True $f(-5)<f(4)$}
	{Hàm số có giá trị nhỏ nhất bằng $2$}
	{\True Đồ thị hàm số có tiệm cận đứng $x=0$}
	{Đồ thị hàm số không có tiệm cận ngang}
	\loigiai{
		\begin{enumerate}[a)]
			\item Từ bảng biến thiên ta thấy $f(-5)<2$ và $f(4)>2$ nên $f(-5)<f(4)$.
			\item Do $\lim \limits_{x\to 0^-}y=-\infty$ nên hàm số không có giá trị nhỏ nhất.
			\item Do $\lim \limits_{x\to 0^-}y=-\infty$ nên đồ thị hàm số có tiệm cận đứng $x=0$.
			\item Do $\lim \limits_{x\to -\infty}y=2y$ nên đồ thị hàm số có tiệm cận ngang $y=2$.
		\end{enumerate}
}
\end{ex}


\begin{ex}
	Cho hàm số hàm số $y=\dfrac{-4x+5}{2x+3}$ có đồ thị $(C)$.
	Xét tính đúng sai của các khẳng định sau:
	\choiceTF
	{\True Hàm số không có cực trị}
	{Đồ thị hàm số có tiệm cận đứng $x=-3$}
	{Đồ thị hàm số có tiệm cận ngang $y=-2$}
	{\True Các đường tiệm cận của đồ thị tạo với hai trục toạ độ một hình chữ nhật có diện tích bằng $3$}
	\loigiai{
		Tập xác định $\mathscr D=\mathbb{R}\setminus \left \{-\dfrac{3}{2}\right \}$\\
		$\lim \limits_{x\to \left (-\frac{3}{2}\right )^+}y=+\infty; \ \lim \limits_{x\to \left (-\frac{3}{2}\right )^-}y=-\infty$ nên đồ thị hàm số có tiệm cận đứng $x=-\dfrac{3}{2}$\\
		$\lim \limits_{x\to -\infty}y=-2, \ \lim \limits_{x\to +\infty}y=-2$ nên đồ thị hàm số có một tiệm cận ngang là $y=-2$\\
		Diện tích hình chữ nhật cần tìm là $S=\left |-\dfrac{3}{2}\right |\cdot \left |-2\right |=3$
	}
\end{ex}

\Closesolutionfile{ans}

\begin{dang}{Bài toán tìm tiệm cận đứng và tiệm cận xiên của đồ thị hàm số}
	\begin{enumerate}[\iconCH]
		\item \indamm{Các bước tìm TCX y = ax + b:}
		Ta xác định hệ số của $a$ và $b$ trong 2 trường hợp sau:
			\begin{listEX}[1]
				\item [\ding{172}] Tính $a=\lim\limits_{x \rightarrow+\infty} \dfrac{f(x)}{x}$, $b=\lim\limits_{x \rightarrow+\infty}[f(x)-ax]$.
				\item [\ding{173}] Tính $a=\lim\limits_{x \rightarrow-\infty} \dfrac{f(x)}{x}$, $b=\lim\limits_{x \rightarrow-\infty}[f(x)-ax]$.
			\end{listEX}
		\item \indamm{Lưu ý:} 
		\begin{listEX}[1]
			\item [\ding{172}] Nếu $a=0$ thì tiệm cận xiên chính là tiệm cận ngang.
			\item [\ding{173}] Đối với hàm số phân thức $f(x)=\dfrac{ax^2+bx+c}{mx+n}$, ta có thể chia đa thức, biến đổi về dạng
			$$f(x)=a'x+b'+\dfrac{e}{mx+n}, \, \text{ với } e \ne0$$
			Suy ra $y=a'x+b'$ là đường tiệm cận xiên của đồ thị hàm số.
		\end{listEX}
	\end{enumerate}
	
\end{dang}
\boxmini{BÀI TẬP TỰ LUẬN}

\begin{vd}
	Tìm các tiệm cận đứng và tiệm cận xiên của đồ thị hàm số sau:
	\begin{listEX}[3]
		\item $y=\dfrac{x^{2}+2}{2x-4}$;
		\item $y=\dfrac{2x^{2}-3x-6}{x+2}$;
		\item $y=\dfrac{2x^{2}+9x+11}{2x+5}$.
	\end{listEX}
	\loigiai{
		\begin{listEX}
			\item Hàm số $y=f(x)=\dfrac{x^{2}+2}{2x-4}$ có tập xác định $\mathscr{D}=\mathbb{R} \setminus \left\lbrace 2\right\rbrace$.
			\begin{itemize}
				\item Ta có $\lim\limits_{x \rightarrow 2^{-}} \dfrac{x^{2}+2}{2x-4}=-\infty$; $\lim\limits_{x \rightarrow 2^{+}} \dfrac{x^{2}+2}{2x-4}=+\infty$.\\
				Suy ra đường thẳng $x=2$ là một tiệm cận đứng của đồ thị hàm số.
				\item Ta có $\begin{aligned}[t]
					a&=\lim\limits_{x \rightarrow+\infty} \dfrac{f(x)}{x}=\lim\limits_{x \rightarrow+\infty} \dfrac{x+\dfrac{2}{x}}{2x-4}=\dfrac{1}{2};\\
					b&=\lim\limits_{x \rightarrow+\infty}[f(x)-ax]=\lim\limits_{x \rightarrow+\infty}\left(\dfrac{x^{2}+2}{2x-4}-\dfrac{1}{2}x\right)=\lim\limits_{x \rightarrow+\infty} \dfrac{2x+2}{2x-4}=1.
				\end{aligned}$\\
				Ta cũng có $\lim\limits_{x \rightarrow-\infty} \dfrac{f(x)}{x}=\dfrac{1}{2}$; $\lim\limits_{x \rightarrow-\infty}[f(x)-\dfrac{1}{2}x]=1$.
				\\
				Do đó, đồ thị hàm số có tiệm cận xiên là đường thẳng $y=\dfrac{1}{2}x+1$.
			\end{itemize}	
			\item Hàm số $y=f(x)=\dfrac{2x^{2}-3x-6}{x+2}$ có tập xác định $\mathscr{D}=\mathbb{R} \setminus \left\lbrace -2\right\rbrace$.
			\begin{itemize}
				\item Ta có $\lim\limits_{x \rightarrow \left(-2\right)^{-}} \dfrac{2x^{2}-3x-6}{x+2}=-\infty$; $\lim\limits_{x \rightarrow \left(-2\right)^{+}} \dfrac{2x^{2}-3x-6}{x+2}=+\infty$.\\
				Suy ra đường thẳng $x=-2$ là một tiệm cận đứng của đồ thị hàm số.
				\item Ta có $\begin{aligned}[t]
					a&=\lim\limits_{x \rightarrow+\infty} \dfrac{f(x)}{x}=\lim\limits_{x \rightarrow+\infty} \dfrac{2x-3-\dfrac{6}{x}}{x+2}=2;\\
					b&=\lim\limits_{x \rightarrow+\infty}[f(x)-ax]=\lim\limits_{x \rightarrow+\infty}\left(\dfrac{2x^{2}-3x-6}{x+2}-2x\right)=\lim\limits_{x \rightarrow+\infty} \dfrac{-7x-6}{x+2}=-7.
				\end{aligned}$\\
				Ta cũng có $\lim\limits_{x \rightarrow-\infty} \dfrac{f(x)}{x}=2$; $\lim\limits_{x \rightarrow-\infty}[f(x)-2x]=-7$.\\
				Do đó, đồ thị hàm số có tiệm cận xiên là đường thẳng $y=2x-7$.
			\end{itemize}
			\item Hàm số $y=f(x)=\dfrac{2x^{2}+9x+11}{2x+5}$ có tập xác định $\mathscr{D}=\mathbb{R} \setminus \left\lbrace -\dfrac{5}{2}\right\rbrace$.
			\begin{itemize}
				\item 
				Ta có $\lim\limits_{x \rightarrow \left(-\tfrac{5}{2}\right)^{-}} \dfrac{2x^{2}+9x+11}{2x+5}=-\infty$; $\lim\limits_{x \rightarrow \left(-\tfrac{5}{2}\right)^{+}} \dfrac{2x^{2}+9x+11}{2x+5}=+\infty$.\\
				Suy ra đường thẳng $x=-\dfrac{5}{2}$ là một tiệm cận đứng của đồ thị hàm số.
				\item Ta có $\begin{aligned}[t]
					a&=\lim\limits_{x \rightarrow+\infty} \dfrac{f(x)}{x}=\lim\limits_{x \rightarrow+\infty} \dfrac{2x+9+\dfrac{11}{x}}{2x+5}=1;\\
					b&=\lim\limits_{x \rightarrow+\infty}[f(x)-ax]=\lim\limits_{x \rightarrow+\infty}\left(\dfrac{2x^{2}+9x+11}{2x+5}-x\right)=\lim\limits_{x \rightarrow+\infty} \dfrac{4x+11}{2x+5}=2.
				\end{aligned}$\\
				Ta cũng có $\lim\limits_{x \rightarrow-\infty} \dfrac{f(x)}{x}=1$; $\lim\limits_{x \rightarrow-\infty}[f(x)-x]=2$.\\
				Do đó, đồ thị hàm số có tiệm cận xiên là đường thẳng $y=x+2$.
			\end{itemize}
		\end{listEX}	
	}
\end{vd}

\boxmini{BÀI TẬP TRẮC NGHIỆM}
\ind{PHẦN I.} \inden{Câu trắc nghiệm nhiều phương án lựa chọn. Mỗi câu hỏi học sinh chỉ chọn một phương án.}\\
\setcounter{ex}{0}
\Opensolutionfile{ans}[ans/2D1-B3-d2-1]
\begin{ex}
	Đường tiệm cận xiên của đồ thị hàm số $y=f(x)=2x-1-\dfrac{1}{x+1}$ có phương trình là
	\choice
	{$y=x+1$}
	{\True $y=2x-1$}
	{$y=x-1$}
	{$y=2x+1$}
	\loigiai{
		Do $\lim\limits_{x\to +\infty}[f(x)-(2x-1)]=\lim\limits_{x\to +\infty}\dfrac{-1}{x+1}=0$ nên đường thẳng $y=2x-1$
		là tiệm cận xiên của đồ thị hàm số đã cho.}
\end{ex}

\begin{ex}
	Đường tiệm cận xiên của đồ thị hàm số $y=f(x)=x+3+\dfrac{1}{2x+1}$ có phương trình là
	\choice
	{$y=2x+1$}
	{$y=x-3$}
	{\True $y=x+3$}
	{$y=2x-1$}
	\loigiai{
		Do $\lim\limits_{x\to \pm\infty}[f(x)-(x+3)]=\lim\limits_{x\to \pm\infty}\dfrac{1}{2x+1}=0$ nên đường thẳng $y=x+3$
		là tiệm cận xiên của đồ thị hàm số đã cho.}
\end{ex}

\begin{ex}
	Tìm tiệm cận xiên của đồ thị hàm số $y=f(x)=\dfrac{x^2+3x}{x-2}$.
	\choice
	{$y=2x-5$}
	{$y=x-2$}
	{\True $y=x+5$}
	{$y=x-5$}
	\loigiai{
	Ta có
	\begin{itemize}
		\item $a=\lim\limits_{x\to +\infty}\dfrac{f(x)}{x}=\lim\limits_{x\to +\infty}\dfrac{x^2+3x}{x(x-2)}=1$
		\item và $b=\lim\limits_{x\to +\infty}[f(x)-x]=\lim\limits_{x\to +\infty}\dfrac{5x}{x-2}=5$.
	\end{itemize}
	Vậy đường thẳng $y=x+5$ là tiệm cận xiên của đồ thị hàm số đã cho (khi $x \to +\infty$).\\
	Tương tự, do $\lim\limits_{x\to -\infty}\dfrac{f(x)}{x}=1$ và $\lim\limits_{x\to -\infty}[f(x)-x]=5$ nên đường thẳng $y=x+5$ cũng là tiệm cận xiên của đồ thị hàm số đã cho (khi $x \to -\infty$).}
\end{ex}

\begin{ex}%[2D1H4-1]
	Tiệm cận xiên của đồ thị hàm số $y=\dfrac{x^2+2x-2}{x+2}$ là
	\choice
	{$y=-2$}
	{$y=1$}
	{$y=x+2$}
	{\True $y=x$}
	\loigiai{
		Ta có $y=\dfrac{x^2+2x-2}{x+2}=\dfrac{x(x+2)-2}{x+2}=x-\dfrac{2}{x+2}$.\\
		$\underset{x\to +\infty}{\mathop{\lim}} [ y-x ] =\underset{x\to +\infty}{\mathop{\lim}}\dfrac{-2}{x+2}=0$ và $\underset{x\to -\infty}{\mathop{\lim}} [ y-x ] =\underset{x\to -\infty}{\mathop{\lim}}\dfrac{-2}{x+2}=0$. \\ 
		Vậy đồ thị hàm số có tiệm cận xiên là đường thẳng $y=x$. 
	}
\end{ex}

\begin{ex}
	Tìm tiệm cận xiên của đồ thị hàm số $f(x)=\dfrac{x^{2}-3 x+1}{x-2}$.
	\choice
	{$y=x+1$}
	{$y=-3x+1$}
	{$y=x-2$}
	{\True $y=x-1$}
	\loigiai{
		Tập xác định: $\mathscr{D}=\mathbb{R} \setminus\{2\}$.
		\\
		Ta có $\begin{aligned}[t]
			a&=\lim\limits_{x \rightarrow+\infty} \dfrac{f(x)}{x}=\lim\limits_{x \rightarrow+\infty} \dfrac{x^{2}-3 x+1}{x^{2}-2 x}=1;\\
			b&=\lim\limits_{x \rightarrow+\infty}[f(x)-a x]=\lim\limits_{x \rightarrow+\infty}\left(\dfrac{x^{2}-3 x+1}{x-2}-x\right)=\lim\limits_{x \rightarrow+\infty} \dfrac{-x+1}{x-2}=-1.
		\end{aligned}$\\
		Ta cũng có $\lim\limits_{x \rightarrow-\infty} \dfrac{f(x)}{x}=1$; $\lim\limits_{x \rightarrow-\infty}[f(x)-x]=-1$.
		\\
		Do đó, đồ thị hàm số có tiệm cận xiên là đường thẳng $y=x-1$.
	}
\end{ex}

\begin{ex}
	Đường tiệm cận xiên của đồ thị hàm số $y=\dfrac{2x^{2}-3x}{x+5}$ đi qua điểm nào sau đây?
	\choice
	{$(5;3)$}
	{$(-4;-5)$}
	{\True $(6;-1)$}
	{$(2;-10)$}
	\loigiai{
		Tập xác định: $\mathscr{D}=\mathbb{R} \setminus\{-5\}$.
		\\
		Ta có $\begin{aligned}[t]
			a&=\lim\limits_{x \rightarrow+\infty} \dfrac{f(x)}{x}=\lim\limits_{x \rightarrow+\infty} \dfrac{2x^{2}-3x}{x^2+5x}=2;\\
			b&=\lim\limits_{x \rightarrow+\infty}[f(x)-ax]=\lim\limits_{x \rightarrow+\infty}\left(\dfrac{2x^{2}-3x}{x+5}-2x\right)=\lim\limits_{x \rightarrow+\infty} \dfrac{-13x}{x+5}=-13.
		\end{aligned}$\\
		Ta cũng có $\lim\limits_{x \rightarrow-\infty} \dfrac{f(x)}{x}=2$; $\lim\limits_{x \rightarrow-\infty}[f(x)-x]=-13$.
		\\
		Do đó, đồ thị hàm số có tiệm cận xiên là đường thẳng $y=2x-13$.	Đường thẳng này qua $(6;-1)$.
	}
\end{ex}

\begin{ex}
	Giao điểm của đường tiệm cận đứng và đường tiệm cận xiên của đồ thị hàm số $y=\dfrac{2x^2-3x+2}{x-1}$ là
	\choice
	{$(1;2)$}
	{\True $(1;1)$}
	{$(1;-1)$}
	{$(1;0)$}
	\loigiai{
	Ta viết lại $y=\dfrac{2x^2-3x+2}{x-1}=2x-1+\dfrac{1}{x-1}$. Suy ra
	\begin{itemize}
		\item [$\bullet$] Tiệm cận đứng $x=1$;
		\item [$\bullet$] Tiệm cận ngang $y=2x-1$.
	\end{itemize}
Xét hệ $\heva{&x=1\\&y=2x-1} \Leftrightarrow \heva{&x=1\\&y=1}$}
\end{ex}

\Closesolutionfile{ans}

\ind{PHẦN II.} \inden{Câu trắc nghiệm đúng sai. Trong mỗi ý a), b), c), d) ở mỗi câu, học sinh chọn đúng hoặc sai.}\\
\Opensolutionfile{ans}[ans/2D1-B3-d2-2]

\begin{ex}
	\immini{Cho hàm số $y=f(x)=\dfrac{ax^2+bx+c}{dx+e}$ có đồ thị như hình bên. 
		\choiceTF
		{Tập xác định của hàm số là $\mathbb{R}$}
		{\True Hàm số có hai điểm cực trị}
		{Đồ thị hàm số có đường tiệm cận đứng là $x=0$}
		{Đồ thị hàm số có đường tiệm cận xiên là $y=x+1$}
	}{
		\begin{tikzpicture}[scale=.4, font=\footnotesize, line join=round, line cap=round, >=stealth]
			\draw[->] (-6,0)--(0,0) node[below left]{$O$}--(6,0) node[below]{$x$};
			\draw[->] (0,-8) --(0,6) node[right]{$y$};
			\clip (-6,-8) rectangle (6,6);
			\draw[violet] [domain=-0.8:6, samples=100,thick] %
			plot (\x, {\x-1+ (2)/((\x)+1)});
			\draw[violet] [domain=-6:-1.3, samples=100,thick] %
			plot (\x, {\x-1+ (2)/((\x)+1)});
			\draw[fill] (0,0) circle (1pt) (-1,0) circle (1pt) (-1,-2) circle (1pt) (1,0) circle (1pt)node[above] {$1$} (0,-1) circle (1pt)node[right] {$-1$};
			\draw[domain=-8:7, samples=100] %
			plot (\x, {\x-1});
			\draw (-1,-8)--(-1,0)node[above left] {$-1$}--(-1,6);
	\end{tikzpicture}}
\end{ex}

\begin{ex}
	\immini{Cho đồ thị của hàm số $y=f(x)=\dfrac{2 x^2}{x^2-1}$. Xét tính đúng sai của các khẳng định sau:
	\choiceTF
	{Đồ thị hàm số có 3 điểm cực trị}
	{$\lim \limits_{x \rightarrow-\infty} f(x)=2$ ; $\lim \limits_{x \rightarrow 1^{-}} f(x)=-\infty$}
	{Đồ thị hàm số có 3 đường tiệm cận đứng $x=-1$, $x=0$, $x=1$} 
	{Đồ thị hàm số có hai đường tiệm cận ngang $y=2$ và $y=0$} 
	}{
\begin{tikzpicture}[scale=.5,>=stealth, font=\footnotesize, line join=round, line cap=round]
	\def\xmin{-6} \def\xmax{6}
	\def\ymin{-5} \def\ymax{7}
	%\draw[color=gray!50,dashed] (\xmin,\ymin) grid (\xmax,\ymax);
	\draw[->] (\xmin,0)--(\xmax,0) node [below]{$x$};
	\draw[->] (0,\ymin)--(0,\ymax) node [left]{$y$};
	\fill (0,0) circle (1pt) node[shift={(135:2.5mm)}]{$O$};
	\clip (\xmin+0.1,\ymin+0.1) rectangle (\xmax-0.1,\ymax-0.1);
	\draw[thick,smooth,violet,samples=300,domain=(\xmin:-1.01)] plot(\x,{(2*(\x)^2)/((\x)^2-1)});		
	\draw[thick,smooth,violet,samples=300,domain=(-0.9:0.9)] plot(\x,{(2*(\x)^2)/((\x)^2-1)});
	\draw[thick,smooth,violet,samples=300,domain=(1.1:\xmax)] plot(\x,{(2*(\x)^2)/((\x)^2-1)});
	\draw[blue] (\xmin,2)--(\xmax,2);	
	\draw[blue] (-1,\ymin)--(-1,\ymax);	
	\draw[blue] (1,\ymin)--(1,\ymax);		
	\foreach \x in {\xmin,...,\xmax}
	\draw (\x,-0.1)--(\x,0.1);
	\foreach \y in {\ymin,...,\ymax}
	\draw (-0.1,\y)--(0.1,\y);
	\node at (-5,2)[below]{$y=2$};
	\node at (-1.2,-4)[left]{$x=-1$};
	\node at (1.2,-4)[right]{$x=1$};
	%\node at (-1,0)[shift={(-135:2.5mm)}]{$-1$};
	%\node at (.5,0)[shift={(-75:2.5mm)}]{$\dfrac{1}{2}$};
	%\node at (0,-1)[left]{$-1$};
	%\node at (0,2)[shift={(135:2.5mm)}]{$2$};		
\end{tikzpicture}}
	\loigiai{
	\begin{enumerate}[a)]
		\item Đồ thị hàm số có một điểm cực trị $(0;0)$.
		\item Theo hình vẽ thì $\lim \limits_{x \rightarrow-\infty} f(x)=2$; $\lim \limits_{x \rightarrow 1^{-}} f(x)=-\infty$.
		\item Đồ thị hàm số có 2 đường tiệm cận đứng $x= \pm 1$.
		\item Đồ thị hàm số có 1 đường tiệm cận ngang $y= 2$.
\end{enumerate}}
\end{ex}

\Closesolutionfile{ans}
\begin{dang}{Bài toán về đường tiệm cận có chứa tham số}
\end{dang}
\boxmini{BÀI TẬP TỰ LUẬN}
\begin{vd}%[2D1Y4-2]
	Tìm tham số $m$ để đồ thị hàm số 
	\begin{tasks}
		\task $y=\dfrac{3x-1}{x-m}$ có đường tiệm cận đứng là $x=5$.
		\task $y=\dfrac{(m+1)x-5m}{2x-m}$ có tiệm cận ngang là đường thẳng $y=1$.
	\end{tasks}
	\loigiai{
		\begin{enumerate}[a)]
			\item Điều kiện để đồ thị hàm số có tiệm cận đứng là $-3m+1\neq 0\Leftrightarrow m\neq \dfrac{1}{3}$.\\
			Đồ thị hàm số có tiệm cận đứng $x=m$.\\
			Theo đề bài ta có $m=5$ (thoả mãn).
			\item Điều kiện để đồ thị hàm số có tiệm cận ngang là $-m(m+1)+10m\neq 0$.\\
			Tiệm cận ngang là $y=\dfrac{a}{c}=\dfrac{m+1}{2}.$\\
			Theo đề bài ta có $\dfrac{m+1}{2}=1\Leftrightarrow m+1=2\Leftrightarrow m=1$ (thoả mãn).
		\end{enumerate}
	}
\end{vd}

\begin{vd}%[2D1K4-2]
	Tìm $m$ để đồ thị hàm số 
	\begin{tasks}
		\task $y=\dfrac{x-2}{x^2-mx+1}$ có hai đường tiệm cận đứng.
		\task $y=\dfrac{2x^2-3x+m}{x-m}$ có đường tiệm cận xiên.
	\end{tasks}
	\loigiai{
		\begin{enumerate}[a)]
			\item Đồ thị hàm số có hai tiệm cận đứng $\Leftrightarrow$ phương trình $g(x)=x^2-mx+1=0$ có hai nghiệm phân biệt khác $2$.
			$$\Leftrightarrow\heva{&a=1\neq 0 \, (\textrm{LĐ})\\ & \Delta =m^2-4>0\\&g(2)=2^2-2m+1\neq 0} \Leftrightarrow \heva{&\hoac{&m<-2\\&m>2}\\& m\neq \dfrac{5}{2}}.$$
			Vậy $m\in\left(-\infty; -2\right) \cup \left(2; +\infty\right) \setminus \left\{\dfrac{5}{2}\right\}$.
			\item 	Đồ thị hàm số có đường tiệm cận xiên khi và chỉ khi phương trình $g(x)=2x^2-3x+m=0$ không có nghiệm $x=m$. Tức là:
			$$g(m)\neq 0 \Leftrightarrow 2m^2-2m\neq 0 \Leftrightarrow \heva{&m\neq 0\\ &n\neq 1}.$$
			Vậy $m\in\mathbb{R}\setminus\left\{0; 1\right\}$ là các giá trị cần tìm.
		\end{enumerate}
	}	
\end{vd}


\boxmini{BÀI TẬP TRẮC NGHIỆM}
\ind{PHẦN I.} \inden{Câu trắc nghiệm nhiều phương án lựa chọn. Mỗi câu hỏi học sinh chỉ chọn một phương án.}\\
\setcounter{ex}{0}
\Opensolutionfile{ans}[ans/2D1-B3-d3-1]

\begin{ex}
	Tìm tất cả các giá trị của $m$ để đồ thị hàm số $y=\dfrac{mx+2}{x-5}$ có đường tiệm cận ngang đi qua điểm $A(1; 3)$.
	\choice
	{$m=-3$}
	{$m=1$}
	{$m=-1$}
	{\True $m=3$}
	\loigiai{
		Tiệm cận ngang $y=m$ đi qua điểm $A(1; 3)$ nên $m=3$.
	}
\end{ex} 

\begin{ex}
	Tìm tham số thực $m$ để đồ thị hàm số $y=\dfrac{mx+3}{x-m}$ có tiệm cận đứng là đường $x=1$, tiệm cận ngang là đường $y=1$.
	\choice
	{\True $m=1$}
	{$m=2$}
	{$m=-1$}
	{$m=3$}
	\loigiai{
		\begin{itemize}
			\item Điều kiện để đồ thị hàm số có tiệm cận là $-m^2-3\ne 0 \ \forall m$
			\item Phương trình đường tiệm cận đứng là $x=m$ nên có $m=1$
			\item Phương trình đường tiệm cận ngang là $y=m$ nên có $m=1$\\
			Vậy $m=1$.
		\end{itemize}
	}
\end{ex}

\begin{ex}
	Biết rằng hai đường tiệm cận của đồ thị hàm số $y=\dfrac{2x+1}{x-m}$ (với $m$ là tham số) tạo với hai trục tọa độ một hình chữ nhật có diện tích bằng $2$. Giá trị của $m$ là
	\choice
	{$m=\pm 2$}
	{$m=-1$}
	{$m=2$}
	{\True $m=\pm 1$}
	\loigiai{
		Điều kiện $ m\neq -\dfrac{1}{2} $.\\
		Ta có $\lim\limits_{x\to+\infty}\dfrac{2x+1}{x-m}=2$ và $\lim\limits_{x\to-\infty}\dfrac{2x+1}{x-m}=2\Rightarrow y=2$ là tiệm cận ngang của đồ thị hàm số.\\
		\begin{itemize}
			\item Xét $ m>-\dfrac{1}{2} $, ta có $\lim\limits_{x\to m^{+}}\dfrac{2x+1}{x-m}=+\infty$, $\lim\limits_{x\to m^{-}}\dfrac{2x+1}{x-m}=-\infty\Rightarrow x=m$ là tiệm cận đứng của đồ thị hàm số.
			\item Xét $ m<-\dfrac{1}{2} $, ta có $\lim\limits_{x\to m^{+}}\dfrac{2x+1}{x-m}=-\infty$, $\lim\limits_{x\to m^{-}}\dfrac{2x+1}{x-m}=+\infty\Rightarrow x=m$ là tiệm cận đứng của đồ thị hàm số.
		\end{itemize}
		Diện tích hình chữ nhật là $|2m|=2\Rightarrow m=\pm 1$ (thỏa mãn).
	}
\end{ex} 


\begin{ex}
	Tìm giá trị của $m$ để đồ thị hàm số $y=\dfrac{2x^2-5x+m}{x-m}$ có tiệm cận đứng.
	\choice
	{$\hoac{&m=0\\&m=2}$}
	{$m\ne 0$}
	{$m\ne 2$}
	{\True $\heva{&m\ne 0\\&m\ne 2}$}
	\loigiai{
		Ta có $x-m=0\Leftrightarrow x=m$. \\
		Để đồ thị hàm số có tiệm cận đứng thì $2(m)^2-5(m)+m\ne 0\Leftrightarrow 2m^2-4m\ne 0\Leftrightarrow \heva{&m\ne 0\\&m\ne 2}$.
	}
\end{ex} 

\begin{ex}%[2D1Y4-1]
	Tìm tất cả các giá trị thực của tham số $m$ để đồ thị hàm số $y=\dfrac{x-4}{x^2-mx+4}$ có hai đường tiệm cận đứng?
	\choice
	{$m \in \left (-\infty;-4\right] \cup \left [4;+\infty \right )$}
	{$m \ne 5$}
	{\True $m \in \left (-\infty;-4\right) \cup \left (4;+\infty \right ) \setminus \left \{5\right \}$}
	{$m \in \left (-\infty;-4\right) \cup \left (4;+\infty \right )$}
	\loigiai{
		Đồ thị hàm số có hai tiệm cận đứng khi phương trình $x^2-mx+4=0$ có hai nghiệm phân biệt khác $4\Leftrightarrow \heva{&m^2-16>0\\&16-4m+4\ne 0}\Leftrightarrow m \in \left (-\infty;-4\right) \cup \left (4;+\infty \right ) \setminus \left \{5\right \}$ 
	}
\end{ex}

\begin{ex}%[2D1B4-2]
	Cho hàm số $ y = \dfrac{2x^2-3x+m}{x-m} $ có đồ thị $ (C) $. Tìm tất cả các giá trị của tham số $ m $ để $ (C) $ không có tiệm cận đứng.
	\choice
	{\True $ m = 0 $ hoặc $ m = 1 $}
	{$ m = 2 $}
	{$ m = 1 $}
	{$ m = 0 $}
	\loigiai{
		Đồ thị $ (C) $ không có tiệm cận đứng khi $ m $ là nghiệm của $ 2x^2-3x+m $
		\begin{align*}
			\Leftrightarrow 2m^2 - 3m + m = 0 \Leftrightarrow \hoac{& m = 0 \\& m = 1.}
		\end{align*}
	}
\end{ex}

\begin{ex}
	Tìm tất cả các giá trị của tham số thực $m$ để đồ thị hàm số $y=\dfrac{x-2}{x^2-mx+1}$ có đúng $3$ đường tiệm cận.
	\choice
	{\True $\left[\begin{aligned}
			&\left\{\begin{aligned}
				&m>2 \\
				&m\ne \dfrac{5}{2}
			\end{aligned}\right. \\
			&m<-2
		\end{aligned}\right. $}
	{$\left[\begin{aligned}
			&m>2 \\
			&\left\{\begin{aligned}
				&m<-2 \\
				&m\ne -\dfrac{5}{2}
			\end{aligned}\right.
		\end{aligned}\right. $}
	{$\left[\begin{aligned}
			&m>2 \\
			&m<-2
		\end{aligned}\right. $}
	{$-2<m<2$}
	\loigiai{
		ĐKXĐ : $x^2-mx+1\ne 0$ \\
		Ta có $\displaystyle\lim \limits_{x\to \pm \infty}y=\displaystyle\lim \limits_{x\to \pm \infty}\dfrac{x-2}{x^2-mx+1}=0$ $ \Rightarrow y=0$ là tiệm cận ngang. \\
		Do đó đồ thị hàm số $y=\dfrac{x-2}{x^2-mx+1}$ có đúng $3$ đường tiệm cận khi và chỉ khi phương trình $x^2-mx+1=0$ có hai nghiệm phân biệt khác $2$. \\
		$ \Leftrightarrow \left\{\begin{aligned}
			& \Delta =m^2-4>0 \\
			&2^2-2m+1\ne 0
		\end{aligned}\right. \Leftrightarrow \left\{\begin{aligned}
			&\left[\begin{aligned}
				&m>2 \\
				&m<-2
			\end{aligned}\right. \\
			&m\ne \dfrac{5}{2}
		\end{aligned}\right. $. }
\end{ex} 

\begin{ex}
	Cho hàm số $y=\dfrac{ax+1}{bx-2}$, xác định $a$ và $b$ để đồ thị của hàm số trên nhận đường thẳng $x=1$ làm tiệm cận đứng và đường thẳng $y=\dfrac{1}{2}$ làm tiệm cận ngang.
	\choice
	{$ \heva{&a=-1\\&b=-2} $}
	{\True $ \heva{&a=1\\&b=2} $}
	{$ \heva{&a=2\\&b=2} $}
	{$ \heva{&a=2\\&b=-2} $}
	\loigiai{Yêu cầu bài toán $\Leftrightarrow\heva{&\dfrac{a}{b}=\dfrac{1}{2}\\&\dfrac{2}{b}=1}\Leftrightarrow\heva{&b=2\\&a=1}$.}
\end{ex} 


\begin{ex}%[2D1Y4-1]
	Cho hàm số $y=\dfrac{mx+1}{x+3n+1}$. Đồ thị hàm số nhận trục hoành và trục tung làm tiệm cận ngang và tiệm cận đứng. Tính $m+n$.
	\choice
	{\True $m+n=-\dfrac{1}{3}$}
	{$m+n=\dfrac{1}{3}$}
	{$m+n=\dfrac{2}{3}$}
	{$m+n=0$}
	\loigiai{
		\begin{itemize}
			\item Điều kiện để đồ thị hàm số có tiệm cận là $m\left (3n+1\right )\ne 0$
			\item Phương trình đường tiệm cận đứng là $x=-3n-1$ nên có $n=-\dfrac{1}{3}$
			\item Phương trình đường tiệm cận ngang là $y=m$ nên có $m=0$\\
			Vậy $m+n=-\dfrac{1}{3}$.
		\end{itemize}
	}
\end{ex}

\begin{ex}%[2D1K4-2]
	Đồ thị hàm số $y=\dfrac{(4a-b)x^2+ax+1}{x^2+ax+b-12}$ nhận trục hoành và trục tung làm hai tiệm cận. Tính giá trị của $a+b$.
	\choice
	{$a+b=10$}
	{$a+b=12$}
	{$a+b=18$}
	{\True $a+b=15$}
	\loigiai{
		Tiệm cận đứng $x=0 \Rightarrow 0^2+a.0+b-12=0\Leftrightarrow b=12.$\\
		Tiệm cận ngang $y=0 \Rightarrow 4a-b=0\Leftrightarrow 4a-12=0 \Leftrightarrow a=3.$\\
		\textbf{Kết luận:} $a+b=15.$
	}
\end{ex}

\Closesolutionfile{ans}

\ind{PHẦN II.} \inden{Câu trắc nghiệm đúng sai. Trong mỗi ý a), b), c), d) ở mỗi câu, học sinh chọn đúng hoặc sai.}\\
\Opensolutionfile{ans}[ans/2D1-B3-d3-2]
\begin{ex}%[2D1B4-2]
	Cho hàm số $y=\dfrac{mx^2+6x-2}{x+2}$, với $m$ là tham số.
	\choiceTF
	{\True Tập xác định của hàm số là $\mathbb{R}\backslash\{-2\}$}
	{Đồ thị hàm số có tiệm cận ngang khi $m>0$}
	{Đồ thị hàm số có tiệm cận đứng khi $m\ne 0$}
	{\True Tập hợp tất cả giá trị của $m$ đề đồ thị có hai đường tiệm cận là $\mathbb{R}\setminus\left\{\dfrac{7}{2}\right\}$}
	\loigiai
	{
		\begin{enumerate}[a)]
			\item Điều kiện $x+2 \ne 0 \Leftrightarrow x \ne -2$. Vậy Tập xác định là $\mathbb{R}\backslash\{-2\}$
			\item Đồ thị hàm số có tiệm cận ngang khi hệ số của $x^2$ trên tử số phải bằng 0. Suy ra $m=0$.
			\item Đồ thị hàm số có tiệm cận đứng khi $x=-2$ không là nghiệm của tam thức $g(x)=mx^2+6x-2$. Suy ra
			$$g(-2)\ne 0 \Leftrightarrow m \ne \dfrac{7}{2}$$
			\item Đồ thị hàm số chắc chắn có 1 tiệm cận xiên (hoặc ngang). Suy ra, để đồ thị có hai đường tiệm cận thì nó phải có 1 tiệm cận đứng. Điều này tương đương với $m \ne \dfrac{7}{2}$.
		\end{enumerate}
	}
\end{ex}

\Closesolutionfile{ans}

% \section[TIỆM CẬN]{ĐƯỜNG TIỆM CẬN CỦA ĐỒ THỊ HÀM SỐ}
\subsection{TÓM TẮT LÝ THUYẾT}
\subsubsection{Đường tiệm cận ngang}%[Lý Văn Hoàng, Dự án TeX hóa Lý Thuyết]
\begin{dn}
    Đường thẳng $y=m$ là đường tiệm cận ngang (hay tiệm cận ngang)
    của đồ thị hàm số $y=f(x)$ nếu ít nhất một trong các điều kiện sau được thỏa mãn:\\
    \centerline{$\lim\limits_{x\to+\infty}f(x)=m, \quad\lim\limits_{x\to-\infty}f(x)=m $.}
\end{dn}
\begin{center}
    \begin{tikzpicture}[scale=1,>=stealth, font=\footnotesize, line join=round, line cap=round]
        \def\xmin{-4} \def\xmax{2}
        \def\ymin{-0.5} \def\ymax{3}
        %\draw[color=gray!50,dashed] (\xmin,\ymin) grid (\xmax,\ymax);
        \draw[->] (\xmin,0)--(\xmax,0) node [below]{$x$};
        \draw[->] (0,\ymin)--(0,\ymax) node [left]{$y$};
        \fill (0,0) circle (1pt) node[shift={(-135:2.5mm)}]{$O$};
        \node at (current bounding box.south) [below=-2pt] {a) $\lim\limits_{x \rightarrow-\infty} f(x)=m$};
        \clip (\xmin+0.1,\ymin+0.1) rectangle (\xmax-0.1,\ymax-0.1);
        \draw[red,thick,smooth,samples=300,domain=\xmin:\xmax]
        (-4,0.9)..controls +(0:2) and +(180:0.5)
        ..(-1.5,0.5)..controls +(0:0.5) and +(180:0.5)
        ..(-0.3,1.4)..controls +(0:0.5) and +(135:1)
        ..(1.8,0.3);
        \draw [blue](\xmin,1)--(\xmax,1);
        \path[blue] (-3,1)node[above]{$y=m$};
        \path[red] (0,1.3)node[above left]{$y=f(x)$};
        \fill (0,1) circle (1pt) node[shift={(-135:3mm)}]{$m$};
    \end{tikzpicture}\hspace*{1cm}
    \begin{tikzpicture}[scale=1,>=stealth, font=\footnotesize, line join=round, line cap=round]
        \def\xmin{-1.5} \def\xmax{4}
        \def\ymin{-0.5} \def\ymax{3}
        %\draw[color=gray!50,dashed] (\xmin,\ymin) grid (\xmax,\ymax);
        \draw[->] (\xmin,0)--(\xmax,0) node [below]{$x$};
        \draw[->] (0,\ymin)--(0,\ymax) node [left]{$y$};
        \fill (0,0) circle (1pt) node[shift={(-135:2.5mm)}]{$O$};
        \node at (current bounding box.south) [below=-2pt] {b) $\lim\limits_{x \rightarrow+\infty} f(x)=m$};
        \clip (\xmin+0.1,\ymin+0.1) rectangle (\xmax-0.1,\ymax-0.1);
        \draw[red,thick,smooth,samples=300,domain=\xmin:\xmax]
        (-1,3)..controls +(-80:1) and +(170:1)
        ..(0.5,1.1)..controls +(170:-1) and +(180:-0.5)
        ..(3.9,0.8);
        \draw [blue](\xmin,0.7)--(\xmax,0.7);
        \path[blue] (4,0.7)node[below left]{$y=m$};
        \path[red] (0.5,1)node[above right]{$y=f(x)$};
        \fill (0,0.7) circle (1pt) node[shift={(-135:3mm)}]{$m$};
    \end{tikzpicture}
\end{center}
\begin{nx} \quad
    \begin{itemize}
        \item Để tìm tiệm cận ngang của đồ thị hàm số ta cần tính giới hạn của hàm số tại vô cực $(\pm \infty)$.
        \item Tìm giới hạn ở vô cực của hàm $y=\dfrac{P(x)}{Q(x)}$ với $P(x)$, $Q(x)$ là các đa thức không căn.
        \begin{enumerate}[i)]
            \item Bậc của $P(x)$ nhỏ hơn bậc của $Q(x) \Rightarrow \lim\limits_{x\to \pm\infty} y =0 \Rightarrow$ Tiệm cận ngang $Ox \colon y=0$.
            \item Bậc của $P(x)$ bằng bậc của $Q(x) \Rightarrow \lim\limits_{x\to \pm\infty} y = \dfrac{\text{Hệ số x bậc cao của P(x) }}{\text{Hệ số x bậc cao của Q(x)}} = \alpha$ (một số cụ thể) $\Rightarrow y= \alpha$ là tiệm cận ngang.
            \item Bậc của $P(x)$ lớn hơn bậc của $Q(x) \Rightarrow \lim\limits_{x\to \pm\infty} y = \pm \infty \Rightarrow$ Không có tiệm cận ngang.
        \end{enumerate}
    \end{itemize}
\end{nx}
\subsubsection{Đường tiệm cận đứng}%[Lý Văn Hoàng, Dự án TeX hóa Lý Thuyết]
\begin{dn}
    Đường thẳng $x=a$ được gọi là đường tiệm cận đứng (hay tiệm cận đứng) của đồ thị hàm số $y=f(x)$ nếu ít nhất một trong các điều kiện sau được thỏa mãn:
    $$ \lim\limits_{x \to a^{+} } f(x)= + \infty; \lim\limits_{x \to a^{+} } f(x)= - \infty ;$$ $$ \lim\limits_{x \to a^{-} } f(x)= + \infty; \lim\limits_{x \to a^{-}} f(x)= - \infty.$$
\end{dn}
\begin{center}
    \begin{tikzpicture}[scale=.7,>=stealth, font=\footnotesize, line join=round, line cap=round]
        %Hình a
        \def\xmin{-2.2} \def\xmax{3.5}
        \def\ymin{-2} \def\ymax{2}
        %\draw[color=gray!50,dashed] (\xmin,\ymin) grid (\xmax,\ymax);
        \draw[->] (\xmin,0)--(\xmax,0) node [below]{$x$};
        \draw[->] (0,\ymin)--(0,\ymax) node [left]{$y$};
        \fill (0,0) circle (1pt) node[shift={(-45:2.5mm)}]{$O$};
        \draw (2.1,\ymin)--(2.1,\ymax)node[below right]{$x=a$};
        \fill (2.1,0) circle (1pt) node[shift={(-45:3mm)}]{$a$};
        %\clip (\xmin+0.1,\ymin+0.1) rectangle (\xmax-0.1,\ymax-0.1);
        \draw[red] (-2,-1)..controls +(80:0.5) and +(0:-.5)..(-1,0.5)node[above]{$y=f(x)$}
        ..controls +(0:0.5) and +(180:0.5)..(0.5,-1.5)
        ..controls +(0:0.5) and +(87:-0.2)..(1.6,0)
        ..controls +(87:-.2) and +(90:-0.2)
        ..(2,1.85);
        \node at (current bounding box.south) [below=-2pt] {a) $\lim\limits_{x \rightarrow a^{-}} f(x)=+\infty$};
    \end{tikzpicture}
    \begin{tikzpicture}[scale=.7,>=stealth, font=\footnotesize, line join=round, line cap=round]
        %Hình b
        \def\xmin{-1.2} \def\xmax{4}
        \def\ymin{-2} \def\ymax{2}
        %\draw[color=gray!50,dashed] (\xmin,\ymin) grid (\xmax,\ymax);
        \draw[->] (\xmin,0)--(\xmax,0) node [below]{$x$};
        \draw[->] (0,\ymin)--(0,\ymax) node [left]{$y$};
        \fill (0,0) circle (1pt) node[shift={(-45:2.5mm)}]{$O$};
        \draw (1,\ymin)node[above right]{$x=a$}--(1,\ymax);
        \fill (1,0) circle (1pt) node[shift={(-135:3mm)}]{$a$};
        \path[red] (1.25,1)node[above right]{$y=f(x)$};
        %\clip (\xmin+0.1,\ymin+0.1) rectangle (\xmax-0.1,\ymax-0.1);
        \draw[red] (1.2,2)..controls +(80:0) and +(0:-1.4)..(2.5,-0.8)
        ..controls +(0:0.1) and +(-80:-0.6)
        ..(3.5,-1.5);
        \node at (current bounding box.south) [below=-2pt] {b) $\lim\limits_{x \rightarrow a^{+}} f(x)=+\infty$};
    \end{tikzpicture}
    \begin{tikzpicture}[scale=.7,>=stealth, font=\footnotesize, line join=round, line cap=round]
        %Hình c
        \def\xmin{-2.2} \def\xmax{3.5}
        \def\ymin{-2} \def\ymax{2}
        %\draw[color=gray!50,dashed] (\xmin,\ymin) grid (\xmax,\ymax);
        \draw[->] (\xmin,0)--(\xmax,0) node [below]{$x$};
        \draw[->] (0,\ymin)--(0,\ymax) node [left]{$y$};
        \fill (0,0) circle (1pt) node[shift={(-45:2.5mm)}]{$O$};
        \draw (2,\ymin)--(2,\ymax)node[below right]{$x=a$};
        \fill (2,0) circle (1pt) node[shift={(-45:3mm)}]{$a$};
        \path[red] (-2.25,1.2)node[below right]{$y=f(x)$};
        %\clip (\xmin+0.1,\ymin+0.1) rectangle (\xmax-0.1,\ymax-0.1);
        \draw[red] (-2,1.4)..controls +(-10:-0.2) and +(-55:-.7)
        ..(1.3,0.65)..controls +(-50:0.4) and +(-90:0)
        ..(1.8,-2)
        ;
        \node at (current bounding box.south) [below=-2pt] {c) $\lim\limits_{x \rightarrow a^{-}} f(x)=-\infty$};
    \end{tikzpicture}
    \begin{tikzpicture}[scale=.7,>=stealth, font=\footnotesize, line join=round, line cap=round]
        %Hình d
        \def\xmin{-2.2} \def\xmax{3.5}
        \def\ymin{-2} \def\ymax{2}
        %\draw[color=gray!50,dashed] (\xmin,\ymin) grid (\xmax,\ymax);
        \draw[->] (\xmin,0)--(\xmax,0) node [below]{$x$};
        \draw[->] (0,\ymin)--(0,\ymax) node [left]{$y$};
        \fill (0,0) circle (1pt) node[shift={(-135:2.5mm)}]{$O$};
        \draw (.6,\ymin)--(.6,\ymax)node[below right]{$x=a$};
        \fill (.6,0) circle (1pt) node[shift={(-135:3mm)}]{$a$};
        %\clip (\xmin+0.1,\ymin+0.1) rectangle (\xmax-0.1,\ymax-0.1);
        \draw[red] (0.7,-2)..controls +(85:0.2) and +(180:0.2)
        ..(1.2,-0.3)..controls +(0:0.2) and +(180:0.2)
        ..(1.7,-0.6)..controls +(0:0.4) and +(90:0)
        ..(2.5,2)
        ;
        \node at (current bounding box.south) [below=-2pt] {d) $\lim\limits_{x \rightarrow a^{+}} f(x)=-\infty$};
    \end{tikzpicture}
\end{center}
\immini{\textbf{Đặc biệt} Đối với hàm số $y= \dfrac{ax+b}{cx+d}$ có tiệm cận ngang $y=\dfrac{a}{c}$ và tiệm cận đứng $x= -\dfrac{d}{c}$. Tâm đối xứng là giao điểm của hai đường tiệm cận.
}{
    \begin{tikzpicture}[>=stealth, line join=round, line cap=round, font=\scriptsize,x=.8cm,y=.7cm]
        \begin{scope}[scale=.7]
            \def\a{1}
            \def\b{1}
            \def\c{1}
            \def\d{-2}
            \def\mau{red}
            \draw[->] (-5,0) -- (8,0) node[below] {$x$};
            \draw[->] (0,-5) -- (0,5) node[left] {$y$};
            \draw (0,0)node[below left]{$O$};
            \draw[dashed,blue] ({-\d/\c},-5)--({-\d/\c},5) (-5,{\a/\c})--(8,{\a/\c}); % Vẽ TCĐ và TCN
            \clip (-5,-5)rectangle(8,5);
            \draw ({-\d/\c},0)node[below right]{$2$};
            \draw (0,{\a/\c})node[above left]{$1$};
            \draw (7,-4)node[above left]{{\normalsize }$y=\dfrac{x+1}{x-2}$};
            \pgfmathsetmacro{\can}{-(\d)/(\c)}
            \draw[\mau,samples=150,smooth,domain=-5:{\can-.1}] plot(\x,{(\a*\x+(\b))/(\c*\x+(\d))}); % Vẽ nhánh bên trái TCĐ
            \draw[\mau,samples=150,smooth,domain={\can+.1}:8] plot(\x,{(\a*\x+(\b))/(\c*\x+(\d))}); % Vẽ nhánh bên phải TCĐ
        \end{scope}
\end{tikzpicture}}
\begin{nx} \quad
    \begin{itemize}
        \item Để tìm tiệm cận đứng của đồ thị hàm số, ta cần tính giới hạn một bên của $x_0$, với $x_0$ thường là điều kiện của hàm số (hay tại $x_0$ thì hàm số không xác định).
        \item Kỹ năng sử dụng máy tính (tham khảo):
        \begin{enumerate}[i)]
            \item Tính $\lim\limits_{x \to x_0^+} f(x)$ thì nhập $f(x)$ và CALC $x= x_0 + 10^{-9}$.
            \item Tính $\lim\limits_{x \to x_0^-} f(x)$ thì nhập $f(x)$ và CALC $x= x_0 - 10^{-9}$.
        \end{enumerate}
    \end{itemize}
\end{nx}
\subsubsection{Đường tiệm cận xiên}
\begin{dn}
    Đường thẳng $y=ax+b$ được gọi là đường tiệm cận xiên của đồ thị $(C):y=f(x)$ nếu \[\lim \limits_{x \to -\infty} \left[f(x)-(ax+b)\right]=0 \text{ hoặc }\lim \limits_{x \to +\infty} \left[f(x)-(ax+b)\right]=0\]
\end{dn}
\begin{center}
    \begin{tikzpicture}[scale=0.8,>=stealth, font=\footnotesize, line join=round, line cap=round]
        \def\xmin{-4} \def\xmax{2.5}
        \def\ymin{-0.5} \def\ymax{3}
        %\draw[color=gray!50,dashed] (\xmin,\ymin) grid (\xmax,\ymax);
        \draw[->] (\xmin,0)--(\xmax,0) node [below]{$x$};
        \draw[->] (0,\ymin)--(0,\ymax) node [left]{$y$};
        \fill (0,0) circle (1pt) node[shift={(-135:2.5mm)}]{$O$};
        \node at (current bounding box.south) [below=-2pt] {a) $\lim\limits_{x \rightarrow-\infty}\left[f(x)-(ax+b)\right]=0$};
        \clip (\xmin+0.1,\ymin+0.1) rectangle (\xmax-0.1,\ymax-0.1);
        \draw[red,thick,smooth,samples=300,domain=\xmin:\xmax]
        (-3.8,-0.6)..controls +(34:0.5) and +(180:.75)
        ..(-0.2,1.2)..controls +(0:0.75) and +(180:.75)
        ..(1,0.3)..controls +(0:0.5) and +(80:0)
        ..(2.2,1);
        \draw[blue,smooth,samples=300,domain=\xmin:\xmax] plot(\x,{2/3*(\x)+2});
        \path[blue] (-3,0)--(0,2)node[below,sloped,pos=1.3]{$y=ax+b$};
        \path[red] (0.5,1)node[above right]{$y=f(x)$};
    \end{tikzpicture}\hspace{1cm}
    \begin{tikzpicture}[scale=0.8,>=stealth, font=\footnotesize, line join=round, line cap=round]
        \def\xmin{-3.5} \def\xmax{3}
        \def\ymin{-0.5} \def\ymax{3}
        %\draw[color=gray!50,dashed] (\xmin,\ymin) grid (\xmax,\ymax);
        \draw[->] (\xmin,0)--(\xmax,0) node [below]{$x$};
        \draw[->] (0,\ymin)--(0,\ymax) node [left]{$y$};
        \fill (0,0) circle (1pt) node[shift={(-135:2.5mm)}]{$O$};
        \node at (current bounding box.south) [below=-2pt] {a) $\lim\limits_{x \rightarrow+\infty}\left[f(x)-(ax+b)\right]=0$};
        \clip (\xmin+0.1,\ymin+0.1) rectangle (\xmax-0.1,\ymax-0.1);
        \draw[red,thick,smooth,samples=300,domain=\xmin:\xmax]
        (-3,0.8)..controls +(60:0.5) and +(180:.75)
        ..(-1.5,2)..controls +(0:.5) and +(180:.75)
        ..(0.5,1.3)..controls +(0:.75) and +(-160:.5)
        ..(2.8,1.8);
        \draw[blue,smooth,samples=300,domain=\xmin:\xmax] plot(\x,{1/3*(\x)+0.75});
        \path[blue] (-3,-0.25)--(0,0.75)node[below,sloped,pos=1.6]{$y=ax+b$};
        \path[red] (-2.5,2)node[above right]{$y=f(x)$};
    \end{tikzpicture}
\end{center}
\begin{nx}\quad
    \begin{itemize}
        \item Để tìm TCX của đồ thị hàm số $y=f(x)$ ta giải hệ phương trình: $\heva{& \lim \limits_{x \to +\infty} \dfrac{f(x)}{x}=a \ne 0 \\ & \lim \limits_{x \to +\infty} \left[f(x)-ax\right]=b}$ hoặc $\heva{& \lim \limits_{x \to -\infty} \dfrac{f(x)}{x}=a \ne 0 \\ & \lim \limits_{x \to -\infty} \left[f(x)-ax\right]=b}$, khi đó tiệm cận xiên của đồ thị hàm số $y=f(x)$ là đường thẳng $y=ax+b$.
        \item Đồ thị hàm số $y=\dfrac{mx^2+nx+p}{cx+d}=ax+b+\dfrac{r}{cx+d}$ có đường tiệm cận xiên là đường thẳng $y=ax+b$.
        \item Hàm phân thức có bậc tử bé hơn hoặc bằng bậc mẫu, bậc tử lớn hơn bậc mẫu 2 bậc thì không có tiệm cận xiên.
    \end{itemize}
\end{nx}
%\subsection{CÁC DẠNG TOÁN}
\begin{dang}{Tìm các đường tiệm cận qua biểu thức hàm số, bảng biến thiên}
\end{dang}
\begin{vd} Tìm các đường tiệm cận đứng, ngang, xiên (nếu có) của đồ thị hàm số sau
    \begin{listEX}[3]
        \item $y=\dfrac{2x+1}{x+1}$.
        \item $y=\dfrac{x}{2x-1}$.
        \item $y=\dfrac{3-x}{x+1}$.
        \item $y=2x+1+\dfrac{1}{x-3}$
        \item $y=\dfrac{4x^2-3x+10}{x-1}$.
        \item $y=\dfrac{x^2-4x+3}{x^2-1}$.
        \item $y=\dfrac{2x+4}{x^2+x-2}$.
        \item $y=\dfrac{\sqrt{9-x^2}}{x-1}$.
        \item $y=x+\sqrt{x^2-1}$
        %	\item $y=\dfrac{x}{\sqrt{x^2+1}}$.
        %	\item $y=\dfrac{\sqrt{x+25}-5}{x^2+x}$.
    \end{listEX}
    \loigiai{}
\end{vd}
\begin{vd}
    Tìm các đường tiệm cận của đồ thị hàm số $y=f(x)$, biết
    \begin{listEX}[2]
        \item \begin{tikzpicture}[scale=.7,>=stealth, font=\footnotesize, line join=round, line cap=round]
            \def\xmin{-2} \def\xmax{4}
            \def\ymin{-3} \def\ymax{3}
            %\draw[color=gray!50,dashed] (\xmin,\ymin) grid (\xmax,\ymax);
            \draw[->] (\xmin,0)--(\xmax,0) node [below]{$x$};
            \draw[->] (0,\ymin)--(0,\ymax) node [left]{$y$};
            \fill (0,0) circle (1pt) node[shift={(135:2.5mm)}]{$O$};
            %\node at (current bounding box.south) [below=-2pt] {a) $y=\dfrac{2x-3}{5x^{2}-15x+10}$};
            \clip (\xmin+0.1,\ymin+0.1) rectangle (\xmax-0.1,\ymax-0.1);
            \draw[thick,smooth,samples=300,domain=\xmin:0.99] plot(\x,{(2*(\x)-3)/(5*(\x)^2-15*(\x)+10)});
            \draw[thick,smooth,samples=300,domain=1.01:1.99] plot(\x,{(2*(\x)-3)/(5*(\x)^2-15*(\x)+10)});
            \draw[thick,smooth,samples=300,domain=2.01:\xmax] plot(\x,{(2*(\x)-3)/(5*(\x)^2-15*(\x)+10)});
            \draw[dashed] (1,\ymin)--(1,\ymax);
            \draw[dashed] (2,\ymin)--(2,\ymax);
            \foreach \s/\t in {2/-45,1/-45}
            \fill (\s,0) circle (1pt) node[shift={(\t:3mm)}]{$\s$};
        \end{tikzpicture}
        \item \begin{tikzpicture}[scale=.5,>=stealth, font=\footnotesize, line join=round, line cap=round]
            \def\xmin{-4} \def\xmax{4}
            \def\ymin{-3} \def\ymax{5}
            %\draw[color=gray!50,dashed] (\xmin,\ymin) grid (\xmax,\ymax);
            \draw[->] (\xmin,0)--(\xmax,0) node [below]{$x$};
            \draw[->] (0,\ymin)--(0,\ymax) node [right]{$y$};
            \fill (0,0) circle (1pt) node[shift={(-135:2.5mm)}]{$O$};
            %\node at (current bounding box.south) [below=-2pt] {c) $y=\dfrac{16x^{2}-8x}{16x^{2}+1}$};
            \clip (\xmin+0.1,\ymin+0.1) rectangle (\xmax-0.1,\ymax-0.1);
            \draw[thick,smooth,samples=300,domain=\xmin:\xmax] plot(\x,{(16*(\x)^2-8*(\x))/(16*(\x)^2+1)});
            \draw[dashed](\xmin,1)--(\xmax,1);
            \foreach \p/\r in {1/45}
            \fill (0,\p) circle (1pt) node[shift={(\r:3mm)}]{$\p$};
        \end{tikzpicture}
        \item 	\begin{tikzpicture}[scale=.7,>=stealth, font=\footnotesize, y=.7cm]
            \def\xmin{-.5} \def\xmax{6}
            \def\ymin{-.5} \def\ymax{5}
            \draw[->] (\xmin,0)--(\xmax,0) node [below]{$x$};
            \draw[->] (0,\ymin)--(0,\ymax) node [left]{$y$};
            \fill (0,0) circle (1pt) node[shift={(-135:2.5mm)}]{$O$};
            \node at (1,-.5)[right]{$x=1$};
            \clip (\xmin+0.1,\ymin+0.1) rectangle (\xmax-0.1,\ymax-0.1);
            \draw[smooth,thick,samples=300,domain=(1.01:\xmax)] plot(\x,{2/sqrt(\x-1)});
            \draw[blue,dashed] (1,\ymin)--(1,\ymax);
            \draw[blue,dashed] (-1,.8)--(6,.8)node[below left]{$y=0.5$};
            \foreach \x in {\xmin,...,\xmax}
            \draw (\x,-0.1)--(\x,0.1);
            \foreach \y in {\ymin,...,\ymax}
            \draw (-0.1,\y)--(0.1,\y);
        \end{tikzpicture}
        \item \begin{tikzpicture}[scale=0.5, font=\footnotesize, line join=round, line cap=round, >=stealth]
            \clip(-3,-2) rectangle (5.1,4.1);
            \draw[->] (-3,0) -- (5,0);\draw (4.9,0) node[below] { $x$};
            \draw[->] (0,-2) -- (0,4);\draw (0,3.9) node[right] { $y$};
            \draw[fill=black] (0,0) node[below right]{$O$} circle (1pt);
            \draw (1,0) node[below right]{$2$};
            \draw (0,1) node[above left]{$1$};
            \draw[thick] plot[domain=-3:0.5,samples=100] (\x, {(1 + \x)/(\x - 1)});
            \draw[thick] plot[domain= 1.5:5,samples=100] (\x, {(1 + \x)/(\x - 1)});
            \draw [-,dashed] (-3,1)--(5,1); %TCN
            \draw [-,dashed] (1,-2)--(1,4); %TCĐ
            \draw[fill=black] (0,0) circle(1pt);
        \end{tikzpicture}
        \item \begin{tikzpicture}[scale=.7,>=stealth, font=\footnotesize, line join=round, line cap=round]
            \def\xmin{-4} \def\xmax{4}
            \def\ymin{-3} \def\ymax{5}
            %\draw[color=gray!50,dashed] (\xmin,\ymin) grid (\xmax,\ymax);
            \draw[->] (\xmin,0)--(\xmax,0) node [below]{$x$};
            \draw[->] (0,\ymin)--(0,\ymax) node [right]{$y$};
            \fill (0,0) circle (1pt) node[shift={(-135:2.5mm)}]{$O$};
            %\node at (current bounding box.south) [below=-2pt] {b) $y=\dfrac{x^{2}+x-1}{x}$};
            \clip (\xmin+0.1,\ymin+0.1) rectangle (\xmax-0.1,\ymax-0.1);
            \draw[thick,smooth,samples=300,domain=\xmin:-0.01] plot(\x,{((\x)^2+(\x)-1)/(\x)});
            \draw[thick,smooth,samples=300,domain=0.01:\xmax] plot(\x,{((\x)^2+(\x)-1)/(\x)});
            \draw[dashed,smooth,samples=300,domain=\xmin:\xmax] plot(\x,{(\x)+1});
            \foreach \s/\t in {-1/-90}
            \fill (\s,0) circle (1pt) node[shift={(\t:3mm)}]{$\s$};
            \foreach \p/\r in {1/-20}
            \fill (0,\p) circle (1pt) node[shift={(\r:3mm)}]{$\p$};
        \end{tikzpicture}
        \item \begin{tikzpicture}[scale=.7,>=stealth, font=\footnotesize,x=.7cm,y=.7cm]
            \def\xmin{-6} \def\xmax{6}
            \def\ymin{-5} \def\ymax{7}
            %\draw[color=gray!50,dashed] (\xmin,\ymin) grid (\xmax,\ymax);
            \draw[->] (\xmin,0)--(\xmax,0) node [below]{$x$};
            \draw[->] (0,\ymin)--(0,\ymax) node [left]{$y$};
            \fill (0,0) circle (1pt) node[shift={(135:2.5mm)}]{$O$};
            \clip (\xmin+0.1,\ymin+0.1) rectangle (\xmax-0.1,\ymax-0.1);
            \draw[smooth,thick,samples=300,domain=(\xmin:-1.01)] plot(\x,{(2*(\x)^2)/((\x)^2-1)});
            \draw[smooth,thick,samples=300,domain=(-0.9:0.9)] plot(\x,{(2*(\x)^2)/((\x)^2-1)});
            \draw[smooth,thick,samples=300,domain=(1.1:\xmax)] plot(\x,{(2*(\x)^2)/((\x)^2-1)});
            \draw[dashed] (\xmin,2)--(\xmax,2);
            \draw[dashed] (-1,\ymin)--(-1,\ymax);
            \draw[dashed] (1,\ymin)--(1,\ymax);
            \foreach \x in {\xmin,...,\xmax}
            \draw (\x,-0.1)--(\x,0.1);
            \foreach \y in {\ymin,...,\ymax}
            \draw (-0.1,\y)--(0.1,\y);
            \node at (-5,2)[below]{$y=2$};
            \node at (-1.2,-4)[left]{$x=-1$};
            \node at (1.2,-4)[right]{$x=1$};
        \end{tikzpicture}
        \item
        \begin{tikzpicture}[>=stealth]
            \tkzTabInit[nocadre=false,lgt=1,espcl=2,deltacl=0.5]{$x$/.7 ,$y'$/.7,$y$/2}
            {$-\infty$ , $1$ , $+\infty$}
            \tkzTabLine{ , - , d , - , }
            \tkzTabVar{+/$2$ ,-D+/$-\infty$/$+\infty$ , -/$2$}
        \end{tikzpicture}
        \item
        \begin{tikzpicture}[>=stealth]
            \tkzTabInit[nocadre=false,lgt=1,espcl=1.5,deltacl=0.5]{$x$/.7 ,$y'$/.7,$y$/2}
            {$-\infty$ , $0$,$1$ , $+\infty$}
            \tkzTabLine{ , + , 0,-, d , + , }
            \tkzTabVar{-/$0$, +/$2$ ,-D-/$-\infty$/$3$ , +/$5$}
        \end{tikzpicture}
        % \item
        % \begin{tikzpicture}[>=stealth]
        %     \tkzTabInit[nocadre=false,lgt=1,espcl=1.8,deltacl=0.5]{$x$/.7 ,$y'$/.7,$y$/2}
        %     {$-\infty$ , $-1$,$1$ , $+\infty$}
        %     \tkzTabLine{ , - , d,-, 0 , + , }
        %     \tkzTabVar{+/$2$ ,-D+/$-5$/$3$, -/$-1$ , +/$+\infty$}
        % \end{tikzpicture}
        % \item
        % \begin{tikzpicture}[>=stealth]
        %     \tkzTabInit[nocadre=false,lgt=1,espcl=1.4,deltacl=0.5]{$x$/.7 ,$y'$/.7,$y$/2}
        %     {$-\infty$ , $-2$, $0$,$1$ , $+\infty$}
        %     \tkzTabLine{ , - , d,-, 0 , + ,d,-, }
        %     \tkzTabVar{+/$-1$ ,-D+/$-\infty$/$2$, -/$-4$, +/$3$ , -/$0$}
        % \end{tikzpicture}
    \end{listEX}
    \loigiai{}
\end{vd}
\begin{vd}
    Một bể bơi chứa $5\,000$ lít nước tinh khiết. Người ta bơm vào bể đó nước muối có nồng đồ $30$ gam muối cho mỗi lít nước với tốc độ $25$ lít/phút.
    \begin{listEX}
        \item Lập hàm số biểu diễn nồng độ muối trong bể sau $t$ phút.
        \item Tìm tiệm cận ngang của hàm số vừa tìm được.
        \item Nêu nhận xét về nồng độ muối trong bể khi thời gian $t$ ngày càng lớn.
    \end{listEX}
    \loigiai{
        \begin{enumerate}[a)]
            \immini{\item Sau $t$ phút, ta có: khối lượng muối trong bể là $25\cdot 30\cdot t=750t$ (gam); thể tích của lượng nước trong bể là $5\,000+25t$ (lít). Vậy nồng độ muối sau $t$ phút là
                $$f(t)=\dfrac{750t}{5\,000+25t}=\dfrac{30t}{200+t}\,\text{(gam/lít)}.$$
                \item Ta có\\
                $\lim\limits_{t\to +\infty}f(t)=\lim\limits_{t \to +\infty}\dfrac{30t}{200+t}=\lim\limits_{t\to +\infty}\left(30-\dfrac{6\,000}{200+t}\right)=30$.\\
                Vậy đường thẳng $y=30$ là tiệm cận ngang của đồ thị hàm số $f(t)$ \texttt{(Hình 17).}}{\begin{tikzpicture}[scale=.1,xscale=0.1, font=\footnotesize, line join=round, line cap=round, >=stealth]
                    \draw[->] (-.5,0)--(0,0) node[below right]{$O$}--(500,0) node[below]{$x$};
                    \draw[->] (0,-1) --(0,34) node[right]{$y$};
                    \draw[blue] [domain=0:500, samples=100] %
                    plot (\x, {(30*(\x))/((\x)+200)});
                    \draw[fill] (0,0) circle (1pt);
                    \foreach \y/\g in {30/180}
                    \draw[fill] (0,\y) circle(1pt)node [shift={(\g:.3)}] {$\y$};
                    \draw[thick] (-.1,30)--(500,30);
                    \draw (250,-5) node{Hình 17};
            \end{tikzpicture}}
            \item Ta có đồ thị hàm số $y=f(t)$ nhận đường thẳng $y=30$ làm đường tiệm cận ngang, tức là khi $t$ càng lớn thì nồng độ muối trong bể sẽ tiến gần đến mức $30$ (gam/lít). Lúc đó, nồng độ muối trong bể sẽ gần như bằng nồng độ nước muối bơm vào bể.
        \end{enumerate}
    }
\end{vd}
\begin{vd}
    Một mô hình kinh tế mô tả lượng cung cầu theo giá cả được cho bởi hàm:
    \[
    Q(p) = \frac{k}{p - p_0}
    \]
    trong đó \( Q(p) \) là lượng cung cầu, \( p \) là giá cả, \( p_0 \) là mức giá tối thiểu, và \( k \) là hằng số tỷ lệ. Xác định tiệm cận đứng của hàm số này và nêu ý nghĩa của nó.
    % \shortans{$Q=p$, khi giá giảm về mức tối thiểu thì nhu cầu tăng lên vô hạn}
    \loigiai{
        Để tìm tiệm cận đứng, ta xem xét các giá trị của \( p \) làm cho mẫu số của phương trình bằng 0:
        \[
        p - p_0 = 0 \Rightarrow p = p_0
        \]
        Vậy đường thẳng \( p = p_0 \) là tiệm cận đứng của đồ thị hàm số.
        \textbf{Ý nghĩa:} Từ đó ta suy ra khi giá cả \( p \) càng sát với \( p_0 \), lượng cung cầu \( Q(p) \) sẽ tăng lên vô hạn. Điều này có nghĩa là nếu giá cả của sản phẩm giảm gần bằng mức giá tối thiểu \( p_0 \), thì nhu cầu đối với sản phẩm đó sẽ tăng lên vô hạn.}
\end{vd}
\BTTN
\Opensolutionfile{ans}[ans/2D1-4-DANG-1]
\begin{ex}%[Nguyễn Văn Sang, dự án Tex hoá đề cương trường Marie Curie - Lần 6]%[2D1Y4-1]
    Đường thẳng nào dưới đây là tiệm cận ngang của đồ thị hàm số $y=\dfrac{x-1}{x+1}$?
    \choice
    {$y=-1$}
    {$x=-1$}
    {\True $y=1$}
    {$x=1$}
    \loigiai{
        Tập xác định $\mathscr{D}=\mathbb{R}\setminus\left\lbrace -1\right\rbrace$.
        \begin{itemize}
            \item $\lim\limits_{x \to \pm\infty} y=\lim\limits_{x \to \pm\infty} \dfrac{x-1}{x+1}=1$ suy ra $y=1$ là tiệm cận ngang.
            \item $\heva{& \lim\limits_{x \to -1^+} \dfrac{x-1}{x+1}=-\infty \\ & \lim\limits_{x \to -1^-} \dfrac{x-1}{x+1}=+\infty}$ suy ra $x=-1$ là tiệm cận đứng.
        \end{itemize}
    }
\end{ex}
%%=====Câu 13
\begin{ex}%[Nguyễn Văn Sang, dự án Tex hoá đề cương trường Marie Curie - Lần 6]%[2D1Y4-1]
    Đồ thị hàm số $y=\dfrac{2 x-3}{1-2 x}$ có tiệm cận đứng là đường thẳng
    \choice
    {$x=3$}
    {$x=2$}
    {\True $x=\dfrac{1}{2}$}
    {$x=\dfrac{3}{2}$}
    \loigiai{
        Tập xác định $\mathscr{D}=\mathbb{R}\setminus\left\lbrace \dfrac{1}{2}\right\rbrace$.
        \begin{itemize}
            \item $\lim\limits_{x \to \pm\infty} y=\lim\limits_{x \to \pm\infty} \dfrac{2x-3}{1-2x}=-1$ suy ra $y=-1$ là tiệm cận ngang.
            \item $\heva{& \lim\limits_{x \to \tfrac{1}{2}^+} \dfrac{2x-3}{1-2x}=+\infty \\ & \lim\limits_{x \to \tfrac{1}{2}^-} \dfrac{2x-3}{1-2x}=-\infty}$ suy ra $x=\dfrac{1}{2}$ là tiệm cận đứng.
        \end{itemize}
    }
\end{ex}
\begin{ex}%[Nguyễn Văn Sang, dự án Tex hoá đề cương trường Marie Curie - Lần 6]%[2D1Y4-1]
    Đồ thị hàm số $y=\dfrac{2-3 x}{2 x-3}$ có tiệm cận đứng và ngang lần lượt là
    \choice
    {\True $x=\dfrac{3}{2}$ và $y=-\dfrac{3}{2}$}
    {$x=\dfrac{3}{2}$ và $y=1$}
    {$x=\dfrac{2}{3}$ và $y=-\dfrac{3}{2}$}
    {$x=\dfrac{2}{3}$ và $y=1$}
    \loigiai{
        Tập xác định $\mathscr{D}=\mathbb{R}\setminus\left\lbrace \dfrac{3}{2}\right\rbrace$.
        \begin{itemize}
            \item $\lim\limits_{x \to \pm\infty} y=\lim\limits_{x \to \pm\infty} \dfrac{2-3 x}{2 x-3}=\dfrac{-3}{2}$ suy ra $y=-\dfrac{3}{2}$ là tiệm cận ngang.
            \item $\heva{& \lim\limits_{x \to \tfrac{3}{2}^+} \dfrac{2-3 x}{2 x-3}=-\infty \\ & \lim\limits_{x \to \tfrac{3}{2}^-} \dfrac{2-3 x}{2 x-3}=+\infty}$ suy ra $x=\dfrac{3}{2}$ là tiệm cận đứng.
        \end{itemize}
    }
\end{ex}
\begin{ex}%[BGD-THPT-2020-104-L2]%[2D1Y4-1]
    Tiệm cận đứng của đồ thị hàm số $y=\dfrac{x+1}{x+3}$ có phương trình là
    \choice
    {$x=-1$}
    {$x=1$}
    {\True $x=-3$}
    {$x=3$}
    \loigiai{
        Tập xác định của hàm số đã cho $\mathscr{D}=\mathbb{R}\setminus\{-3\}$.\\
        Ta có $\lim\limits_{x\rightarrow-3^-}y=\lim\limits_{x\rightarrow-3^-}\dfrac{x+1}{x+3}=+\infty$ và $\lim\limits_{x\rightarrow-3^+}y=\lim\limits_{x\rightarrow-3^+}\dfrac{x+1}{x+3}=-\infty$.\\
        Khi đó đường tiệm cận đứng của đồ thị hàm số đã cho là $x=-3$.
    }
\end{ex}
%%==========Câu 11
\begin{ex}%[BGD-Minh Họa-2020-L2]%[2D1Y4-1]
    Tiệm cận ngang của đồ thị hàm số $y=\dfrac{x-2}{x+1}$ có phương trình là
    \choice
    {$y=-2$}
    {\True $y=1$}
    {$x=-1$}
    {$x=2$}
    \loigiai
    {
        Tập xác định: $\mathscr{D}=\mathbb{R}\setminus \{-1\}$.\\
        Ta có $\lim \limits_{x \to +\infty} y=\lim \limits_{x \to +\infty} \dfrac{x-2}{x+1}=\lim \limits_{x \to +\infty} \dfrac{1-\dfrac{2}{x}}{1+\dfrac{1}{x}}=1$ và $\lim \limits_{x \to -\infty} y=\lim \limits_{x \to -\infty} \dfrac{x-2}{x+1}=\lim \limits_{x \to -\infty} \dfrac{1-\dfrac{2}{x}}{1+\dfrac{1}{x}}=1$ nên đường thẳng $y=1$ là đường tiệm cận ngang của đồ thị.
    }
\end{ex}
%%==========Câu 12
\begin{ex}%[BGD-THPT-2021-101-L1]%[2D1Y4-1]
    Tiệm cận đứng của đồ thị hàm số $ y=\dfrac{2x-1}{x-1}$ là đường thẳng có phương trình là
    \choice
    {\True $x=1$}
    {$x=-1$}
    {$x=2$}
    {$x=\dfrac{1}{2}$}
    \loigiai{
        Vì $\lim\limits_{x\to 1^+}\dfrac{2x-1}{x-1}=+\infty $ và $\lim\limits_{x\to 1^-}\dfrac{2x-1}{x-1}=-\infty $ nên đồ thị hàm số $ y=\dfrac{2x-1}{x-1}$ có một tiệm cận đứng là đường thẳng $ x=1 $.
    }
\end{ex}
%%==========Câu 13
\begin{ex}%[BGD-THPT-2021-102-L1]%[2D1Y4-1]
    Tiệm cận đứng của đồ thị hàm số $ y=\dfrac{x+1}{x-2}$ là đường thẳng có phương trình là
    \choice
    {$x=-1$}
    {$x=-2$}
    {\True $x=2$}
    {$x=1$}
    \loigiai{
        Ta có $\displaystyle\lim\limits_{x\to 2^+}\dfrac{x+1}{x-2}=+\infty $; $\displaystyle\lim\limits_{x\to 2^-}\dfrac{x+1}{x-2}=-\infty $.\\
        Vậy đồ thị hàm số $ y=\dfrac{x+1}{x-2}$ có tiệm cận đứng là đường thẳng $ x=2 $.
    }
\end{ex}
%%==========Câu 14
\begin{ex}%[BGD-THPT-2021-103-L1]%[2D1B4-1]
    Tiệm cận đứng của đồ thị hàm số $ y=\dfrac{2x+1}{x-1}$ là đường thẳng có phương trình là
    \choice
    {$x=2$}
    {\True $x=1$}
    {$x=-\dfrac{1}{2}$}
    {$x=-1$}
    \loigiai{
        Ta có $\lim\limits_{x\to 1^+}y=\lim\limits_{x\to 1^+}\dfrac{2x+1}{x-1}=+\infty $ nên tiệm cận đứng của đồ thị hàm số là đường thẳng $ x=1 $.
    }
\end{ex}
%%==========Câu 15
\begin{ex}%[BGD-THPT-2021-104-L1]%[2D1Y4-1]
    Tiệm cận đứng của đồ thị hàm số $ y=\dfrac{x-1}{x+2}$ là đường thẳng có phương trình là
    \choice
    {$x=2$}
    {$x=-1$}
    {\True $x=-2$}
    {$x=1$}
    \loigiai{
        Ta có $\lim\limits_{x\to (-2)^{+}}\dfrac{x-1}{x+2}=-\infty $, $\lim\limits_{x\to (-2)^{-}}\dfrac{x-1}{x+2}=+\infty $.\\
        Đồ thị hàm số có tiệm cận đứng là đường thẳng có phương trình $ x=-2 $.
    }
\end{ex}
\begin{ex}%[2D1Y4-1]
    Giao điểm của tiệm cận đứng và tiệm cận ngang của đồ thị hàm số $y=\dfrac{-2}{3x-1}$ là điểm
    \choice
    {$Q\left(\dfrac{1}{3};-2\right)$}
    {$M\left(\dfrac{1}{3};-\dfrac{2}{3}\right)$}
    {$N\left(\dfrac{1}{3};2\right)$}
    {\True $P\left(\dfrac{1}{3};0\right)$}
    \loigiai{
        Tiệm cận đứng, tiệm cận ngang của đồ thị hàm số lần lượt là $x=\dfrac{1}{3}$ và $y=0$. Giao điểm của $2$ tiệm cận là $P\left(\dfrac{1}{3};0\right)$.
    }
\end{ex}
\begin{ex}%[2D1Y4-1]
    Đồ thị hàm số $y=\dfrac{3-4x}{x-5}$ có tâm đối xứng là điểm
    \choice
    {$M\left(5;-\dfrac{3}{5}\right)$}
    {$P\left(5;\dfrac{4}{5}\right)$}
    {$Q(5;3)$}
    {\True $N(5;-4)$}
    \loigiai{
        Tiệm cận đứng, tiệm cận ngang của đồ thị hàm số lần lượt là $x=5$ và $y=-4$. Tâm đối xứng là điểm $N(5;-4)$.
    }
\end{ex}
\begin{ex}%[2D1B4-1]
    Đồ thị hàm số nào dưới đây có tiệm cận đứng?
    \choice
    {$y=\dfrac{x^2-3x+2}{x-1}$}
    {$y=\dfrac{x^2}{x^2+1}$}
    {$y=\sqrt{x^2-1}$}
    {\True $y=\dfrac{x}{x+1}$}
    \loigiai{
    }
\end{ex}
\begin{ex}%[2D1B4-1]
    Cho hàm số $y=f(x)$ có bảng biến thiên như hình bên. Tổng số tiệm cận đứng và tiệm cận ngang của đồ thị hàm số đã cho là
    \begin{center}
        \begin{tikzpicture}
            \tkzTabInit[nocadre=false,lgt=1.5,espcl=3,deltacl=0.6]
            {$x$ /0.6,$y’$ /0.6,$y$ /2}
            {$-\infty$ ,$0$, $1$, $+\infty$}
            \tkzTabLine{,-,d,+,0,-,}
            \tkzTabVar{+/$+\infty$,-D-/$-\infty$/$-1$,+/$2$,-/$-3$}
        \end{tikzpicture}
    \end{center}
    \choice
    {$1$}
    {$3$}
    {\True $2$}
    {$4$}
    \loigiai{Dựa vào bảng biến thiên ta thấy đồ thị hàm số có tiệm cận đứng $x=0$ và tiệm cận ngang $y=-3$.}
\end{ex}
\begin{ex}%[2D1B4-1]
    Cho hàm số $y=f(x)$ có bảng biến thiên như hình bên. Tổng số tiệm cận đứng và tiệm cận ngang của đồ thị hàm số đã cho là
    \begin{center}
        \begin{tikzpicture}[scale=0.8]
            \tkzTabInit[nocadre=false,lgt=1.5,espcl=3,deltacl=0.6]
            {$x$ /0.6,$y’$ /0.6,$y$ /2}
            {$-\infty$ , $0$,$2$, $+\infty$}
            \tkzTabLine{,-,0,+,d,-,}
            \tkzTabVar{+/$8$,-/$1$,+/$4$,-/$2$}
        \end{tikzpicture}
    \end{center}
    \choice
    {$1$}
    {$3$}
    {\True $2$}
    {$4$}
    \loigiai{
        Dựa vào bảng biến thiên ta thấy đồ thị hàm số có tiệm cận ngang $y=8$ và $y=2$.
    }
\end{ex}
\begin{ex}%[2D1B4-1]
    Cho hàm số $y=f(x)$ có bảng biến thiên như hình bên. Tổng số tiệm cận đứng và tiệm cận ngang của đồ thị hàm số đã cho là
    \begin{center}
        \begin{tikzpicture}[scale=0.8]
            \tkzTabInit[nocadre=false,lgt=1.5,espcl=3,deltacl=0.6]
            {$x$ /0.6,$y’$ /0.6,$y$ /2}
            {$-\infty$ ,$1$, $2$, $+\infty$}
            \tkzTabLine{,+,d,-,d,+,}
            \tkzTabVar{-/$-4$,+/$3$,-/$-5$,+/$+\infty$}
        \end{tikzpicture}
    \end{center}
    \choice
    {\True $1$}
    {$3$}
    {$2$}
    {$0$}
    \loigiai{
        Dựa vào bảng biến thiên ta thấy đồ thị hàm số có một tiệm cận ngang $y=-4$.
    }
\end{ex}
\begin{ex}%[2D1B4-1]
    Cho hàm số $y=f(x)$ có bảng biến thiên như hình bên. Đồ thị hàm số đã cho có tiệm cận đứng là đường thẳng
    \begin{center}
        \begin{tikzpicture}[scale=0.8, font=\footnotesize, line join=round, line
            cap=round, >=stealth]
            \tkzTabInit[espcl=2.5,lgt=1,nocadre=false]
            {$x$/0.7,$f(x)$/2.1}
            {$-\infty$,$0$,$1$,$2$,$+\infty$}
            \tkzTabVar{-/$-\infty$,+/$2$,-D+/$-\infty$/$+\infty$,-/$4$,+/$+\infty$}
        \end{tikzpicture}
    \end{center}
    \choice
    {$x=0$}
    {\True $x=1$}
    {$x=2$}
    {$x=4$}
    \loigiai{Dựa vào bảng biến thiên ta thấy đồ thị hàm có tiệm cận đứng $x=1$.}
\end{ex}
%%==========Câu 16
\begin{ex}%[BGD-THPT-2019-103]%[2D1B4-1]
    Cho hàm số $y=f(x)$ có bảng biến thiên như sau
    \begin{center}
        \begin{tikzpicture}[scale=1, font=\footnotesize,line join=round, >=stealth]
            \tkzTabInit[nocadre=false,lgt=1.5,espcl=3]{$x$/.7,$y'$/.7,$y$/2.5}{$-\infty$,$0$,$3$,$+\infty$}%
            \tkzTabLine{,-,d,+,0,-,}%
            \tkzTabVar{+/$1$ , -D+/$-\infty$/$2$,-/$-3$, +/$3$}%
        \end{tikzpicture}
    \end{center}
    Tổng số tiệm cận đứng và tiệm cận ngang của đồ thị hàm số đã cho là
    \choice
    {1}
    {2}
    {\True 3}
    {4}
    \loigiai{
        Nhìn bảng biến thiên ta thấy\\
        $\lim\limits_{x \to 0^-} f(x)=-\infty \Rightarrow x=0$ là TCĐ của đồ thị hàm số.\\
        $\lim\limits_{x \to +\infty} f(x)=3 \Rightarrow y=3$ là TCN của đồ thị hàm số.\\
        $\lim\limits_{x \to -\infty} f(x)=1 \Rightarrow y=1$ là TCN của đồ thị hàm số.\\
        Vậy hàm số có 3 tiệm cận.}
\end{ex}
%%==========Câu 17
\begin{ex}%[BGD-THPT-2019-102]%[2D1B4-1]
    Cho hàm số $f(x)$ có bảng biến thiên như sau
    \begin{center}
        \begin{tikzpicture}[scale=1, font=\footnotesize,line join=round, >=stealth]
            \tkzTabInit[lgt=1.2,espcl=3]
            {$x$/0.8,$f’(x)$/0.8,$f(x)$/2}
            {$-\infty$,$0$,$1$,$+\infty$}
            \tkzTabLine{ ,-,d,-,0,+,}
            \tkzTabVar{+/$0$,-D+/$-\infty$/$2$,-/$-2$,+/$+\infty$}
        \end{tikzpicture}
    \end{center}
    Tổng số tiệm cận đứng và tiệm cận ngang của đồ thị hàm số đã cho là
    \choice
    {$3$}
    {$1$}
    {\True $2$}
    {$4$}
    \loigiai{
        Từ bảng biến thiên đã cho ta có\\
        $\lim\limits_{x \to -\infty} f(x)=0$ nên đường thẳng $y=0$ là một tiệm cận ngang của đồ thị hàm số.\\
        $\lim\limits_{x \to 0^-} f(x)=-\infty$ nên đường thẳng $x=0$ là một tiệm cận đứng của đồ thị hàm số.\\
        Vậy đồ thị hàm số đã cho có hai đường tiệm cận.}
\end{ex}
\begin{ex}%[2D1B4-1]
    Cho hàm số $y=f(x)$ có bảng biến thiên như hình bên. Tổng số tiệm cận đứng và tiệm cận ngang của đồ thị hàm số đã cho là
    \begin{center}
        \begin{tikzpicture}[scale=0.8]
            \tkzTabInit[nocadre=false,lgt=1.5,espcl=3,deltacl=0.6]
            {$x$ /0.6,$y’$ /0.6,$y$ /2}
            {$-\infty$ ,$0$, $1$, $+\infty$}
            \tkzTabLine{,+,0,-,d,-,}
            \tkzTabVar{-/$4$,+/$2$,-D+/$-\infty$/$5$,-/$-3$}
        \end{tikzpicture}
    \end{center}
    \choice
    {$1$}
    {\True $3$}
    {$2$}
    {$4$}
    \loigiai{
        Dựa vào bảng biến thiên ta thấy đồ thị hàm số có tiệm cận đứng $x=1$, tiệm cận ngang $y=4$ và $y=-3$.
    }
\end{ex}
\begin{ex}%[2D1B4-1]
    Cho hàm số $y=f\left(x\right)$ có bảng biến thiên như sau
    \begin{center}
        \begin{tikzpicture}[scale=1,line join=round,>=stealth]\tikzset{double style/.append style={double distance=2pt}}
            \tkzTabInit[nocadre=false,lgt=1.2,espcl=2.2,deltacl=0.6]
            {$x$ /.6,$y'$ /.6,$y$ /2.2}
            {$ -\infty $,$-2$,$0$,$+\infty$}
            \tkzTabLine{,-,d,+,d,-}
            \tkzTabVar{+/$+\infty$,-D-/$1$/$-\infty$,+D+/$+\infty$/$1$,-/$0$,}
        \end{tikzpicture}
    \end{center}
    Tổng số đường tiệm cận đứng và tiệm cận ngang của đồ thị hàm số đã cho bằng
    \choice
    {$2$}
    {$1$}
    {$0$}
    {\True $3$}
    \loigiai{
        Ta có
        \begin{itemize}
            \item $\lim\limits_{x \to -2^{+}} y=-\infty \Rightarrow x=-2$ là tiệm cận đứng.
            \item $\lim\limits_{x \to 0^{-}} y=+\infty \Rightarrow x=0$ là tiệm cận đứng.
            \item $\lim\limits_{x \to +\infty} y=0 \Rightarrow y=0$ là tiệm cận ngang.
        \end{itemize}
        Vậy đồ thị hàm số đã cho có tổng đường tiệm cận đứng và tiệm cận ngang là $3$.}
\end{ex}
\begin{ex}%[2D1B4-1]
    Cho hàm số $y=f\left(x\right)$ liên tục trên $\mathbb{R} \backslash\{1\}$ có bảng biến thiên như bảng sau:
    \begin{center}
        \begin{tikzpicture}[scale=1,line join=round,>=stealth]
            \tikzset{double style/.append style={double distance=2pt}}
            \tkzTabInit[nocadre=false,lgt=1.2,espcl=2.8,deltacl=0.6]
            {$x$ /0.6,$y'$ /0.6,$y$ /2.2}
            {$ -\infty $,$-1$,$1$,$+\infty$}
            \tkzTabLine{,-,0,+,d,+}
            \tkzTabVar{+/$1$,-/$-\sqrt 2$,+D-/$+\infty$/$-\infty$,+/$-1$,}
        \end{tikzpicture}
    \end{center}
    Tổng số đường tiệm cận đứng và đường tiệm cận ngang của đồ thị hàm số $y=f\left(x\right)$ là
    \choice
    {$1$}
    {$4$}
    {$2$}
    {\True $3$}
    \loigiai{
        Do $\lim\limits_{x \to 1^{+}} y=-\infty \Rightarrow$ Tiệm cận đứng $x=1$.\\
        Lại có $\lim\limits_{x \to +\infty} y=-1 ; \lim\limits_{x \to -\infty} y=1 \Rightarrow$ Đồ thị có $2$ tiệm cận ngang là $y=\pm 1$.\\
        Vậy, đồ thị hàm số đã cho có tổng số tiệm cận là $3$.}
\end{ex}
\begin{ex}%[2D1B4-1]
    Cho hàm số $y=f\left(x\right)$ có bảng biến như sau:
    \begin{center}
        \begin{tikzpicture}[scale=1,line join=round,>=stealth]
            \tikzset{double style/.append style={double distance=2pt}}
            \tkzTabInit[nocadre=false,lgt=1.2,espcl=2.5,deltacl=0.6]{$x$ /.6,$y'$ /.6,$y$ /2}
            {$ -\infty $,$-3$,$3$,$+\infty$}
            \tkzTabLine{,+,d,+,d,+}
            \tkzTabVar{-/$0$,+D-/$+\infty$/$-\infty$,+D-/$+\infty$/$-\infty$,+/$0$,}
        \end{tikzpicture}
    \end{center}
    Số đường tiệm cận của đồ thị hàm số là
    \choice
    {\True $3$}
    {$1$}
    {$4$}
    {$2$}
    \loigiai{
        Từ bảng biến thiên của hàm số ta có
        \begin{itemize}
            \item $\lim\limits_{x \to -\infty} y=0 ; \lim\limits_{x \to +\infty} y=0 \Rightarrow$ Đường thẳng $y=0$ là tiệm cận ngang.
            \item $\lim\limits_{x \to (-3)^{-}} y=+\infty \Rightarrow$ Đường thẳng $x=-3$ là tiệm cận đứng.
            \item $+\lim\limits_{x \to 3^{-}} y=+\infty \Rightarrow$ Đường thẳng $x=3$ là tiệm cận đứng.
        \end{itemize}
        Vậy số đường tiệm cận của đồ thị hàm số là $3$.}
\end{ex}
\begin{ex}%[2D1B4-1]
    Cho hàm số $y=f\left(x\right)$ có bảng biến thiên như sau
    \begin{center}
        \begin{tikzpicture}[scale=1,line join=round,>=stealth]
            \tikzset{double style/.append style={double distance=2pt}}
            \tkzTabInit[nocadre=false,lgt=1.2,espcl=2.5,deltacl=0.6]
            {$x$ /.6,$y'$ /.6,$y$ /2.2}
            {$ -\infty $,$-2$,$2$,$+\infty$}
            \tkzTabLine{,-,d,-,d,-}
            \tkzTabVar{+/$0$,-D+/$-\infty$/$+\infty$,-D+/$-\infty$/$+\infty$,-/$-\infty$,}
        \end{tikzpicture}
    \end{center}
    Tổng số tiệm cận đứng và tiệm cận ngang của đồ thị hàm số đã cho là
    \choice
    {$4$}
    {$2$}
    {\True $3$}
    {$1$}
    \loigiai{
        Dựa vào bảng biến thiên, ta có:
        \begin{itemize}
            \item $\lim\limits_{x \to -\infty} f(x)=0$ nên đường thẳng $y=0$ là đường tiệm cận ngang.
            \item $\lim\limits_{x \to -2^{+}} f(x)=+\infty $ nên đường thẳng $x=-2$ là đường tiệm cận đứng.
            \item $\lim\limits_{x \to 2^{+}} f(x)=+\infty$ nên đường thẳng $x=2$ là đường tiệm cận đứng.
        \end{itemize}
        Vậy, tổng số tiệm cận đứng và tiệm cận ngang của đồ thị hàm số đã cho là $3$.}
\end{ex}
%%==========Câu 20
\begin{ex}%[THPT Yên Định - Thanh Hóa 2019]%[2D1B4-1]
    Cho hàm số $ y=f(x) $ xác định và có đạo hàm trên $ \mathbb{R}\setminus\{\pm 1\} $. Hàm số có bảng biến thiên như hình vẽ dưới đây.
    \begin{center}
        \begin{tikzpicture}[scale=1, font=\footnotesize,line join=round, >=stealth]
            \tkzTabInit[nocadre=false,lgt=1.2,espcl=2.5,deltacl=0.6]{$x$/.6 ,$y'$/.6,$y$/2.5} {$-\infty$ , $-1$ , $0$ , $1$ , $+\infty$}
            \tkzTabLine{ , + , d , - , d , + , d , + , }
            \tkzTabVar{-/$-4$ , +D-/$+\infty$/$-\infty$ , +/$2$,-D-/$-\infty$/$-\infty$,+/$-1$}
        \end{tikzpicture}
    \end{center}
    Tổng số đường tiệm cận đứng và tiệm cận ngang của đồ thị hàm số đã cho là
    \choice
    { $ 1 $ }
    { $ 2 $ }
    { $ 3 $ }
    {\True $ 4 $ }
    \loigiai{
        Dựa vào bảng biến thiên, suy ra:\\
        $ \lim \limits_{x \to - \infty} y=-4 $, $ \lim \limits_{x \to + \infty} y=-1$. Đồ thị có hai tiệm cận ngang là $ y=-4 $ và $ y=-1 $.\\
        Lại có $ \lim \limits_{x \to (-1)^+} y=+\infty $ và $ \lim \limits_{x \to 1^-} y=+\infty $, $ \lim \limits_{x \to 1^-} y=-\infty $. Đồ thị hàm số có hai đường tiệm cận đứng là $ x=1 $ và $ x=-1 $.
    }
\end{ex}
\begin{ex}%[2D1B4-1]
    Cho hàm số $y=f\left(x\right)$ có bảng biến thiên như hình vẽ dưới đây.
    \begin{center}
        \begin{tikzpicture}[line cap=round,line join=round,>=triangle 45,x=1.0cm,y=1.0cm]
            \clip(-1.58,-2.4) rectangle (12.58,2.);
            \fill[line width=1.2pt,dash pattern=on 15 pt off 5pt,color=white,fill=black,pattern=north east lines,pattern color=black] (0.,1.) -- (2.84,1.) -- (2.84,-1.96) -- (0.,-1.96) -- cycle;
            \draw (-1.,1.)-- (12.,1.);
            \draw (-1.,0.)-- (12.,0.);
            \draw (0.,1.62)-- (0.,-1.96);
            \draw (0.08,1.5) node[anchor=north west] {$-\infty$};
            \draw (2.46,1.5) node[anchor=north west] {$-2$};
            \draw (6.85,1.5) node[anchor=north west] {$0$};
            \draw (11.14,1.5) node[anchor=north west] {$+\infty$};
            \draw (2.84,1.)-- (2.84,-1.96);
            \draw (3.,1.)-- (3.,-1.96);
            \draw (7.,1.)-- (7.,-1.96);
            \draw (7.14,1.)-- (7.14,-1.96);
            \draw (-0.7,1.5) node[anchor=north west] {$x$};
            \draw (-0.72,0.78) node[anchor=north west] {$y'$};
            \draw (-0.64,-0.8) node[anchor=north west] {$y$};
            \draw [->] (3.96,-1.54) -- (6.24,-0.64);
            \draw [->] (7.64,-0.58) -- (11.12,-1.7);
            \draw (11.34,-1.5) node[anchor=north west] {$0$};
            \draw (8.98,0.7) node[anchor=north west] {$-$};
            \draw (4.68,0.7) node[anchor=north west] {$+$};
            \draw (6.0,-0.2) node[anchor=north west] {$+\infty$};
            \draw (7.22,-0.2) node[anchor=north west] {$1$};
            \draw (3.08,-1.5) node[anchor=north west] {$-\infty$};
        \end{tikzpicture}
    \end{center}
    Hỏi đồ thị của hàm số đã cho có bao nhiêu đường tiệm cận?
    \choice
    {\True $3$}
    {$2$}
    {$4$}
    {$1$}
    \loigiai{
        Dựa vào bảng biến thiên ta có:\\
        $\lim\limits_{x \to -2^{+}} f(x)=-\infty$, suy ra đường thẳng $x=-2$ là tiệm cận đứng của đồ thị hàm số.\\
        $\lim\limits_{x \to 0^{-}} f(x)=+\infty$, suy ra đường thẳng $x=0$ là tiệm cận đứng của đồ thị hàm số.\\
        $\lim\limits_{x \to +\infty} f(x)=0$, suy ra đường thẳng $y=0$ là tiệm cận ngang của đồ thị hàm số.\\
        Vậy đồ thị hàm số có $3$ đường tiệm cận.}
\end{ex}
%%=====Câu 5
\begin{ex}%[Nguyễn Văn Sang, dự án Tex hoá đề cương trường Marie Curie - Lần 6]%[2D1Y4-1]
    Cho hàm số $y=f(x)$ có $\lim\limits_{x \rightarrow 3} f(x)=+\infty$, $\lim\limits_{x \rightarrow+\infty} f(x)=-\infty$, $\lim\limits_{x \rightarrow-\infty} f(x)=8$ và $\lim\limits_{x \rightarrow 7} f(x)=5 $. Tổng số tiệm cận ngang và tiệm cận đứng của đồ thị hàm số đã cho là
    \choice
    {$4$}
    {\True $2$}
    {$1$}
    {$3$}
    \loigiai{
        Ta có
        \begin{itemize}
            \item $\lim\limits_{x \rightarrow-\infty} f(x)=8$, suy ra $y=8$ là tiệm cận ngang.
            \item $\lim\limits_{x \rightarrow 3} f(x)=+\infty$, suy ra $x=3$ là tiệm cận đứng.
            \item $\lim\limits_{x \rightarrow 7} f(x)=5 $, suy ra $x=7$ không là tiệm cận đứng.
        \end{itemize}
        Vậy đồ thị hàm số có $1$ tiệm cận đứng và $1$ tiệm cận ngang.
    }
\end{ex}
%%=====Câu 7
\begin{ex}%[Nguyễn Văn Sang, dự án Tex hoá đề cương trường Marie Curie - Lần 6]%[2D1Y4-1]
    Cho hàm số $y=f(x)$ có $\lim\limits_{x \rightarrow 1^{+}} f(x)=+\infty$ và $\lim\limits_{x \rightarrow 1^{-}} f(x)=2$. Mệnh đề nào sau đây đúng?
    \choice
    {Đồ thị hàm số không có tiệm cận}
    {\True Đồ thị hàm số có tiệm cận đứng $x=1$}
    {Đồ thị hàm số có hai tiệm cận}
    {Đồ thị hàm số tiệm cận ngang $y=2$}
    \loigiai{
        Ta có $\lim\limits_{x \rightarrow 1^{-}} f(x)=2$, suy ra $x=1$ là tiệm cận đứng.
    }
\end{ex}
\begin{ex}%[2D1B4-1]
    Cho hàm số $y=\dfrac{\sqrt{x+1}}{\sqrt{x^2-4}}$ mệnh đề nào sau đây đúng?
    \choice
    {\True Đồ thị hàm số có một tiệm cận đứng và một tiệm cận ngang}
    {Đồ thị hàm số có một tiệm cận đứng và hai tiệm cận ngang}
    {Đồ thị hàm số có hai tiệm cận đứng và hai tiệm cận ngang}
    {Đồ thị hàm số có hai tiệm cận đứng và một tiệm cận ngang}
    \loigiai{
        Tập xác định $\mathscr{D}=[-1;+\infty) \setminus \{2\}$. \\
        Đồ thị hàm số có một tiệm cận đứng $x=2$, tiệm cận ngang là $y=0$.
    }
\end{ex}
\begin{ex}%[2-HK1-49-THPT-NKKN-TPHCM, 12EX5]%[Nhật Thiện, ID6]%[2D1K4-2]%
    Với giá trị nào của $m$ thì đồ thị hàm số $y=\dfrac{mx-1}{2x+m}$ có tiệm cận đứng là đường thẳng $x=-1$?
    \choice
    {$m=2$}
    {\True $m=-2$}
    {$m=\dfrac{1}{2}$}
    {$m=0$}
    \loigiai{
        Đồ thị hàm số $y=\dfrac{mx-1}{2x+m}$ có tiệm cận đứng là đường thẳng $x=-1$ khi và chỉ khi $$\heva{&m(-1)-1\ne 0\\&2(-1)+m=0}\Leftrightarrow \heva{&m\ne -1\\&m=-2(n).}$$
    }
\end{ex}
\begin{ex}%[2D1K4-1]
    Đồ thị hàm số $y=\dfrac{2x-1-\sqrt{x^2+x+3}}{x^2-5x+6}$ có tất cả đường tiệm cận đứng là đường thẳng
    \choice
    {$x=-3$ và $x=-2$}
    {$x=-3$}
    {$x=3$ và $x=-2$}
    {\True $x=3$}
    \loigiai{
        Điều kiện xác định $x \ne 3$, $x \ne 2$.\\
        Với điều kiện xác định trên, ta có
        {\allowdisplaybreaks
            \begin{eqnarray*}
                y&=&\dfrac{2x-1-\sqrt{x^2+x+3}}{x^2-5x+6}=\dfrac{(3x+1)(x-2)}{(x-2)(x-3)\left(2x-1+\sqrt{x^2+x+3}\right)}\\
                &=&\dfrac{3x+1}{(x-3)\left(2x-1+\sqrt{x^2+x+3}\right)}.
        \end{eqnarray*} }
        Tiệm cận đứng của đồ thị hàm số là $x=3$.
    }
\end{ex}
\begin{ex}%[2D1K4-1]
    Số tiệm cận đứng của đồ thị hàm số $y=\dfrac{\sqrt{x+9}-3}{x^2+x}$ là
    \choice
    {$3$}
    {$2$}
    {$0$}
    {\True $1$}
    \loigiai{
        Tập xác định $\mathscr{D}=[-9;+\infty)\setminus \{-1;0\}$. \\
        Ta có $\left\{\begin{aligned}
            &\lim\limits_{x\to -1^+} y=\lim\limits_{x\to -1^+} \dfrac{\sqrt{x+9}-3}{x^2+x}=+\infty \\
            &\lim\limits_{x\to -1^-} y =\lim\limits_{x\to -1^-} \dfrac{\sqrt{x+9}-3}{x^2+x}=-\infty
        \end{aligned}\right. \Rightarrow x=-1$ là tiệm cận đứng. \\
        Ngoài ra $\lim\limits_{x\to 0} y =\lim\limits_{x\to 0} \dfrac{\sqrt{x+9}-3}{x^2+x}=\dfrac{1}{6}$ nên $x=0$ không phải là một tiệm cận đứng.}
\end{ex}
\BTTF
\begin{ex}%[EX-TF-2024, Lê Đạt]%[2D1N4-1]
    Cho hàm số $y=\dfrac{2x-3}{x-1}$. Xét tính đúng sai các khẳng định dưới đây
    \choiceTF
    {\True Đường tiệm cận đứng của đồ thị hàm số là $ x=1 $}
    {Đường tiệm cận đứng của đồ thị hàm số là $ y=2 $}
    {Đường tiệm cận ngang của đồ thị hàm số là $ x=1 $}
    {\True Đường tiệm cận ngang của đồ thj hàm số là $ y=2 $}
    \loigiai{
        Ta có $\lim\limits_{x\to -\infty}y=\lim\limits_{x\to +\infty}y=2$ nên đồ thị hàm số đã cho có tiệm cận ngang là $y=2$.\\
        Ta có $\lim\limits_{x\to 1^+}y=-\infty$ nên đồ thị hàm số đã cho có tiệm cận ngang là $ x=1 $.
        \begin{itemchoice}
            \itemch Đường tiệm cận đứng của đồ thị hàm số là $ x=1 $.
            \itemch Đường tiệm cận đứng của đồ thị hàm số là $ x=1 $.
            \itemch Đường tiệm cận ngang của đồ thj hàm số là $ y=2 $.
            \itemch Đường tiệm cận ngang của đồ thj hàm số là $ y=2 $.
        \end{itemchoice}
    }
\end{ex}
\begin{ex}%[EX-TF-2024, Lê Đạt]%[2D1N4-1]
    Cho hàm số $y=f(x)$ có bảng biến thiên như sau
    \begin{center}
        \begin{tikzpicture}[>=stealth]
            \tkzTabInit[nocadre=false,lgt=1,espcl=3,deltacl=0.6]
            {$x$/.7 ,$y'$/.7,$y$/2}
            {$-\infty$ , $-2$ , $0$, $+\infty$}
            \tkzTabLine{ , - , d , + , d , -, }
            \tkzTabVar{+/$+\infty$ , -D-/$1$/$-\infty$ , +D+/$+\infty$ /$1$, -/$0$}
        \end{tikzpicture}
    \end{center}
    Xét tính đúng sai của các khẳng định sau
    \choiceTF
    {\True $ x=0 $ là tiệm cận đứng của đồ thị hàm số $ y=f(x) $}
    {\True $ x=-2 $ là tiệm cận đứng của đồ thị hàm số $ y=f(x) $}
    {$ x=1 $ là tiệm cận đứng của đồ thị hàm số $ y=f(x) $}
    {\True $ y=0 $ là tiệm cận ngang của đồ thị hàm số $ y=f(x) $}
    \loigiai{
        \begin{itemchoice}
            \itemch $\lim \limits_{x \to 0^-} f(x)=+\infty\Rightarrow x=0$ là đường tiệm cận đứng của đồ thị hàm số $f(x)$.
            \itemch $\lim \limits_{x \to (-2)^+} f(x)=-\infty\Rightarrow x=-2$ là đường tiệm cận đứng của đồ thị hàm số $f(x)$.
            \itemch Đồ thị hàm số chỉ có hai tiệm cận đứng là $ x=0 $ và $ x=-2 $.
            \itemch $\lim \limits_{x \to +\infty} f(x)=0\Rightarrow y=0$ là đường tiệm cận ngang của đồ thị hàm số $f(x)$.
        \end{itemchoice}
    }
\end{ex}
%===== DẠNG 2
\begin{ex}%[EX-TF-2024, Lê Đạt]%[2D1H4-2]
    Cho hàm số $ y=\dfrac{m^2x+1}{x-1} $. Xét tính đúng sai của các khẳng định sau
    \choiceTF
    {\True Đồ thị hàm số luôn có tiệm cận ngang}
    {\True Đồ thị hàm số luôn có tiệm cận đứng}
    {\True Khi $ m=1$ đồ thị hàm số có $ 2 $ đường tiệm cận}
    {Khi $ m=0 $ đồ thị hàm số có $ 1 $ đường tiệm cận}
    \loigiai{
        \begin{itemchoice}
            \itemch $\lim\limits_{x\to -\infty}y=\lim\limits_{x\to +\infty}y=m^2$ suy ra hàm số luôn có tiệm cận ngang.
            \itemch $\lim\limits_{x\to 1^+}y=+\infty$ nên đồ thị hàm số đã cho có tiệm cận ngang là $ x=1 $.
            \itemch Khi $ m=1 $ ta được hàm số $ y=\dfrac{x+1}{x-1} $ suy ra đồ thì hàm số có $ x=1 $ là tiệm cận đứng và $ y=1 $ là tiệm cận ngang nên đồ thị hàm số có $ 2 $ tiệm cận.
            \itemch Khi $ m=0 $ ta được hàm số $ y=\dfrac{1}{x-1} $ suy ra đồ thì hàm số có $ x=1 $ là tiệm cận đứng và $ y=0 $ là tiệm cận ngang nên đồ thị hàm số có $ 2 $ tiệm cận.
        \end{itemchoice}
    }
\end{ex}
\begin{ex}%[EX-TF-2024, Lê Đạt]%[2D1H4-2]
    Cho hàm số $y=\dfrac{m x^{2}+6 x-2}{x+2}$. Xét tính đúng sai của các khẳng định sau
    \choiceTF
    {Đồ thị hàm số luôn có tiệm cận đứng với mọi $ m $}
    {Đồ thị hàm số không có tiệm cận ngang với mọi $ m $}
    {\True Khi $ m=1 $ đồ thị hàm số có một tiệm cận xiên là $ y=x+4 $ }
    {Đồ thị hàm số luôn có tiệm cận xiên}
    \loigiai{
        \begin{itemchoice}
            \itemch Khi $ m=\dfrac{7}{2} $ hàm số trở thành $y=\dfrac{\dfrac{7}{2} x^{2}+6 x-2}{x+2}=\dfrac{7}{2}\left(x-\dfrac{2}{7} \right) $ suy ra đồ thị hàm số không có tiệm cận đứng.
            \itemch Khi $ m=0 $ hàm số trở thành $ y=\dfrac{6x-2}{x+2} $ từ đó suy ra đồ thị hàm số có $ y=6 $ là tiệm cận ngang.
            \itemch Khi $ m=1 $ hàm số trở thành $ y=\dfrac{x^2+6x-2}{x+2}=x+4-\dfrac{10}{x+2} $ từ đó suy ra $ y=x+4 $ là một tiệm cận ngang.
            \itemch Khi $ m=0 $ hàm số trở thành $ y=\dfrac{6x-2}{x+2} $ từ đó suy ra đồ thị hàm số có $ y=6 $ là tiệm cận ngang, $ x=-2 $ là tiệm cận đứng và không có tiệm cận xiên.
        \end{itemchoice}
    }
\end{ex}
\begin{ex}
    Cho hàm số $y=\dfrac{x-1}{x^2-8 x+m}$, $m$ là tham số. Các mệnh đề sau đúng hay sai?
    \choiceTF
    {\True Đồ thị hàm số có 1 đường tiệm cận ngang}
    {Khi $m<16$ thì đồ thị hàm số có 3 đường tiệm cận}
    {Khi $m=16$ thì đồ thị hàm số có 2 đường tiệm cận đứng}
    {\True Có 14 giá trị nguyên dương của $m$ để đồ thị hàm số có 3 đường tiệm cận}
    \loigiai{
        Ta có $\lim \limits{n \to +\infty}_{x \rightarrow-\infty} \frac{x-1}{x^2-8 x+m}=\lim \limits{n \to +\infty}_{x \rightarrow+\infty} \frac{x-1}{x^2-8 x+m}=0$ nên hàm số có một tiện cận ngang $y=0$.
        Hàm số có 3 đường tiệm cận khi và chỉ khi hàm số có hai đường tiệm cận đứng $\Leftrightarrow$ phương trình $x^2-8 x+m=0$ có hai nghiệm phân biệt khác $1 \Leftrightarrow\left\{\begin{array}{l}\Delta^{\prime}=16-m>0 \\ m-7 \neq 0\end{array} \Leftrightarrow\left\{\begin{array}{l}m<16 \\ m \neq 7\end{array}\right.\right.$.
        Kết hợp với điều kiện $m$ nguyên dương ta có $\quad m \in\{1 ; 2 ; 3 ; \ldots ; 6 ; 8 ; \ldots ; 15\}$. Vậy có 14 giá trị của $m$ thỏa mãn đề bài.}
\end{ex}
\begin{ex}
    Cho hàm số $y=\dfrac{x^2+m x-1}{x-1}\left(C_m\right)$ ( $m$ là tham số). Các mệnh đề sau đúng hay sai?
    \choiceTF
    {\True Để đồ thị $\left(C_m\right)$ của hàm số có tiệm cận xiên thì $m \neq 0$.}
    {\True Để tiệm cận xiên của $\left(C_m\right)$ đi qua $M(2,-5)$ thì $m=-8$}
    { Để tiệm cận xiên của $\left(C_m\right)$ tạo với hai trục toạ độ một tam giác có diện tích bằng 8 thì tổng tất cả các giá trị $m$ tìm được bằng 2}
    { Với $m=3$ thì giao điểm của hai đường tiệm cận của $\left(C_m\right)$ nằm trên Parapol $y=x^2+3$}
    \loigiai{
        Hàm số xác định trên $\mathbb{R} \backslash\{1\}$.
        \begin{listEX}
            \item Ta có $y=x+m+1+\frac{m}{x-1}$
            Để đồ thị $\left(C_m\right)$ của hàm số có tiệm cận xiên thì $m \neq 0$.
            - Với $m \neq 0,\left(C_m\right)$ có tiệm cận xiên
            $y=x+m+1\left(\Delta_m\right)$ vì $\lim \limits{n \to +\infty}_{x \rightarrow \infty}[y-(x+m+1)]=\lim \limits{n \to +\infty}_{x \rightarrow \infty} \frac{m}{x-1}=0$.
            \item Để $\left(\Delta_m\right)$ qua $M(2,-5)$ thì $-5=2+m+1 \Leftrightarrow m=-8$. (thỏa mãn $m \neq 0$ ).
            \item Gọi $A$ là giao điểm của $\Delta_m$ với $O x$. Khi đó $A(-m-1 ; 0)$
            Gọi $B$ là giao điểm của $\Delta_m$ với $O y$. Khi đó $B(0 ; m+1)$.
            Suy ra $S_{\triangle O A B}=\frac{1}{2} O A \cdot O B=\frac{1}{2}|-m-1||m+1|=\frac{1}{2}(m+1)^2$
            Để $S_{\triangle O A B}=8 \Leftrightarrow \frac{1}{2}(m+1)^2=8 \Leftrightarrow\left[\begin{array}{l}m=-5 \\ m=3\end{array}\right.$ (thỏa mãn $m \neq 0$ ).
            \item Ta có với $m \neq 0, x=1$ là tiệm cận đứng vì $\lim \limits{n \to +\infty}_{x \rightarrow 1} y=\infty$ nên $y=x+m+1$ là tiệm cận xiên.
            Khi đó giao điểm của 2 tiệm cận là $I(1, m+2)$.
            Để $I$ nằm trên Parabol $y=x^2+3$ thì $m+2=1+3 \Leftrightarrow m=2(\mathrm{t} / \mathrm{m} m \neq 0)$.
        \end{listEX}
    }
\end{ex}
%===== DẠNG 3
\begin{ex}%[EX-TF-2024, Lê Đạt]%[2D1N4-3]
    \immini{Cho hàm số $y=f(x)$ có đồ thị như hình bên. Xét tính đúng sai của các khẳng định sau
        \choiceTF
        {$ x=2 $ là đường tiệm cận ngang của đồ thị hàm số}
        {\True $ x=-1 $ là đường tiệm cận đứng của đồ thị hàm số}
        {\True Đồ thị hàm số có hai đường tiệm cận}
        {\True Đồ thị hàm số không có tiệm cận xiên}
    }{
        \begin{tikzpicture}[scale=0.5, font=\footnotesize, line join=round, line cap=round, >=stealth]
            \draw[->](-5,0)--(5,0)node[below]{ $x$};
            \draw[->](0,-4)--(0,5)node[right]{ $y$};
            \draw [fill=black,draw=black] (0,0) circle (1pt)node[above left] { $O$};
            \foreach \x in {-1}\draw[shift={(\x,0)}](0pt,-2pt)--(0pt,2pt) node[below left]{ $\x$};
            \foreach \y in {2}\draw[shift={(0,\y)}](-2pt,0pt)--(2pt,0pt)node[above right]{ $\y$};
            \clip(-5,-4) rectangle (5,5);
            \draw[smooth,samples=100,domain=-5:-1.1] plot(\x,{(2*(\x)-1)/((\x)+1)});
            \draw[smooth,samples=100,domain=-0.9:5] plot(\x,{(2*(\x)-1)/((\x)+1)});
            \draw[dashed](-5,2)--(5,2) (-1,-4)--(-1,5);
        \end{tikzpicture}
    }
    \loigiai{
        \begin{itemchoice}
            \itemch $ y=2 $ là đường tiệm cận ngang của đồ thị hàm số.
            \itemch $ x=-1 $ là đường tiệm cận đứng của đồ thị hàm số.
            \itemch $ x=-1 $ là đường tiệm cận đứng và $ y=2 $ là đường tiệm cận ngang của đồ thị hàm số suy ra đồ thị hàm số có hai đường tiệm cận.
            \itemch Đồ thị hàm số không có tiệm cận xiên.
        \end{itemchoice}
    }
\end{ex}
\begin{ex}%[EX-TF-2024, Lê Đạt]%[2D1H4-3]
    \immini{Cho hàm số $y=f(x)$ có đồ thị như hình bên. Xét tính đúng sai của các khẳng định sau
        \choiceTF
        {\True $ x=0 $ là một đường tiệm cận đứng của đồ thị hàm số}
        {$ y=-x $ là một đường tiệm cận xiên của đồ thị hàm số}
        {\True $ y=x $ là một đường tiệm cận xiên của đồ thị hàm số}
        {Đồ thị hàm số có ba đường tiệm cận}
    }{
        \begin{tikzpicture}[scale=.9, font=\footnotesize, line join=round, line cap=round,>=stealth]
            \def\a{0} \def\b{1} \def\c{1} \def\d{-1} % Hệ số
            \def\xmin{-3} \def\xmax{3.5}
            \def\ymin{-2.8} \def\ymax{3.3}
            \draw[color=gray!50,dashed] (\xmin,\ymin) grid (\xmax,\ymax);
            \draw[->] (\xmin,0)--(\xmax,0) node [below]{$x$};
            \draw[->] (0,\ymin)--(0,\ymax) node [left]{$y$};
            \fill (0,0) circle(1pt) node[shift=(-45:0.25)]{$O$};
            \clip (\xmin+0.1,\ymin+0.1) rectangle (\xmax-0.1,\ymax-0.1);
            \draw[smooth,samples=300,domain=-3:3] plot(\x,{\x+1/(7*\x)});
            \draw[dashed,smooth,samples=300,domain=-3:3] plot(\x,{\x});
            %	\fill (-1,0) circle (1.0pt) node[below]{$-1$} (1,0) circle (1.0pt) node[below right]{$1$};
    \end{tikzpicture}}
    \loigiai{
        \begin{itemchoice}
            \itemch $ x=0 $ là một đường tiệm cận đứng của đồ thị hàm số.
            \itemch	$ y=x $ là một đường tiệm cận xiên của đồ thị hàm số.
            \itemch $ y=x $ là một đường tiệm cận xiên của đồ thị hàm số.
            \itemch Đồ thị hàm số có $ x=0 $ là tiệm cận đứng và $ y=x $ là tiệm cận xiên nên có hai tiệm cận.
        \end{itemchoice}
    }
\end{ex}
\BTTL
\begin{ex}%[2D1K4-2]
    Nếu đồ thị hàm số $y=\dfrac{(m+1)x+2}{x-n+1}$ lần lượt nhận trục hoành và trục tung làm đường đường tiệm cận ngang và tiệm cận đứng thì $m+n$ bằng bao nhiêu?
    \shortans{$0$}
    \loigiai{
        Theo đề bài, ta có $\heva{&m+1=0\\&n-1=0} \Leftrightarrow \heva{&m=-1\\&n=1.}$\\
        Suy ra $m+n=0$.
    }
\end{ex}
\begin{ex}%[2D1K4-2]
    Tìm giá trị của $m$ để đồ thị hàm số $y=\dfrac{(2m+1)x+3}{x+1}$ có đường đường tiệm cận đi qua điểm $A(-2;7)$.
    \shortans{m=3}
    \loigiai{
        Từ đề bài, suy ra $2m+1=7 \Leftrightarrow m=3$.\\
        Suy ra $m+n=0$.
    }
\end{ex}
\begin{ex}%[2D1K4-2]
    Cho hàm số $y=\dfrac{-3+mx}{x+n}$. Tìm giá trị của $m$ và $n$ để đồ thị hàm số đã cho có tiệm cận đứng $x=2$ và tiệm cận ngang $y=2$.
    \shortans{$m=2, n=-2$}
    \loigiai{
        Từ yêu cầu đề bài, suy ra $\heva{&m=2\\&-n=2} \Leftrightarrow \heva{&m=2\\&n=-2.}$
    }
\end{ex}
\begin{ex}%[2D1K4-2]
    Để đường tiệm cận đứng và tiệm cận ngang của đồ thị hàm số $y=\dfrac{mx+1}{2m+1-x}$ cùng với hai trục tọa độ tạo thành một hình chữ nhật có diện tích bằng $3$ thì giá trị của $m$ bằng bao nhiêu?
    \shortans{$1$ hay $-\dfrac{3}{2}$}
    \loigiai{
        Từ yêu cầu đề bài, suy ra $|-m| \cdot |2m+1|=3 \Leftrightarrow \hoac{&m=1\\&m=-\dfrac{3}{2}.}$
    }
\end{ex}
\begin{ex}%[2D1K4-2]%[Thầy Hải Toán]%Câu 2.
    Đường tiệm cận đứng và đường tiệm cận ngang của đồ thị hàm số $y=\dfrac{mx+1}{2m+1-x}$ cùng với hai trục tọa độ tạo thành một hình chữ nhật có diện tích bằng $3$. Tính giá trị của $m$.
    \shortans{$m=1$; $m=-\dfrac{3}{2}$}
    \loigiai{
        Ta có $\lim\limits_{x\to+\infty}\dfrac{mx+1}{2m+1-x}=-m$; $\lim\limits_{x\to(2m+1)^+}\dfrac{mx+1}{2m+1-x} =\lim\limits_{x\to(2m+1)^+}\dfrac{m(2m+1)+1}{2m+1-x} =\lim\limits_{x\to(2m+1)^+}\dfrac{2m^2+m+1}{2m+1-x}$
        $\lim\limits_{x\to(2m+1)^+}\left(2m^2+m+1\right)=2m^2+m+1>0$; $\lim\limits_{x\to(2m+1)^+}(2m+1-x)=0$ và $2m+1-x<0\forall x>2m+1$ \\
        $ \Rightarrow\lim\limits_{x\to(2m+1)^+}\dfrac{mx+1}{2m+1-x}=-\infty $.\\
        Vậy đồ thị hàm số có hai đường tiệm cận $x=2m+1$ và $y=-m$.\\
        Hai đường tiệm cận tạo với hai trục tọa độ một hình chữ nhật có diện tích bằng $3$ suy ra $|2m+1|\cdot|m|=3\Leftrightarrow\hoac{&2m^2+m=3\\&2m^2+m=-3(PTVN)}\Leftrightarrow 2m^2+m-3=0\Leftrightarrow\hoac{&m=1\\&m=\dfrac{-3}{2}}$.}
\end{ex}
\begin{ex}%[KSCL L1, THPT Nhã Nam - Bắc Giang, 2019]%[Phạm An Bình, 12EX3]%[2D1K4-2]%
    Biết rằng đồ thị của hàm số $y=\dfrac{(n-3)x+n-2017}{x+m+3}$ ($m$, $n$ là tham số) nhận trục hoành làm tiệm cận ngang và trục tung làm tiệm cận đứng. Tính tổng $m-2n$.
    \shortans{$-9$}
    \loigiai{
        $\bullet$ $\lim\limits_{x\to +\infty}y=\lim\limits_{x\to +\infty}\dfrac{n-3+\dfrac{n-2017}{x}}{1+\dfrac{m+3}{x}} =n-3$.\\
        Vì đồ thị nhận trục hoành làm tiệm cận ngang nên $n-3=0\Leftrightarrow n=3$.\\
        $\bullet$ Vì đồ thị hàm số nhận trục tung làm tiệm cận đứng nên $\heva{&n-2017\ne 0\\&m+3=0}\Leftrightarrow \heva{&n\ne 2017\\&m= -3.}$\\
        Vậy $m-2n=-9$.
    }
\end{ex}
\begin{ex}%[TT Nguyễn Đăng Đạo, Bắc Ninh, lần 3, đề 152, 2018]%[2D1K4-2]%[Nguyễn Vân Trường, 12EX-8]%
    Tìm $m$ để tiệm cận đứng của đồ thị hàm số $y = \dfrac{m^2x-4m}{2x-m^2}$ đi qua điểm $A(2;1)$.
    \shortans{$m=-2$}
    \loigiai{
        Để hàm số có tiệm cận đứng thì \\
        $\hoac{& m \ne 0 \\ & m^2\cdot \dfrac{m^2}{2} - 4m \ne 0} \Leftrightarrow \hoac{& m \ne 0 \\ & m(m^3-8) \ne 0} \Leftrightarrow \hoac{& m \ne 0 \\ & m \ne 2}.$\\
        Khi đó tiệm cận đứng của hàm số là $x = \dfrac{m^2}{2}.$ Theo giả thiết ta có $ \dfrac{m^2}{2} = 2 \Leftrightarrow \hoac{& m =2 \text{ (loại)} \\ & m=-2 \text{ (thỏa mãn).}}$ Vậy $m=-2$.
    }
\end{ex}
\begin{ex}%[TT, Chuyên Lê Quý Đôn, Lai Châu, 2018]%[2D1K4-2]%[Nguyễn Tiến Thùy, 12EX-8]%
    Tìm $m$ để đồ thị hàm số $y=\dfrac{(m+1)x-5m}{2x-m}$ có tiệm cận ngang là đường thẳng $y=1$.
    \shortans{$m=1$}
    \loigiai{
        Ta có $\lim\limits_{x\rightarrow \pm\infty}f(x)=\lim\limits_{x\rightarrow \pm\infty}\dfrac{(m+1)x-5m}{2x-m}=\dfrac{m+1}{2}$, suy ra $y=\dfrac{m+1}{2}$ là tiệm cận ngang.\\
        Theo bài ra ta có $y=\dfrac{m+1}{2}=1\Leftrightarrow m=1$.
    }
\end{ex}
\begin{ex}%[2D1K4-1]
    Tìm tất cả các đường tiệm cận ngang của đồ thị hàm số $y=\dfrac{\sqrt{4x^2-x+1}}{2x+1}$.
    \shortans{$y=1$ và $y=-1$}
    \loigiai{
        Điều kiện xác định $x \ne \dfrac{-1}{2}$.\\
        Ta có $\lim\limits_{x \to +\infty} \dfrac{\sqrt{4x^2-x+1}}{2x+1}=\lim\limits_{x \to +\infty} \dfrac{|2x|\sqrt{1-\dfrac{1}{4x}+\dfrac{1}{4x^2}}}{2x\left(1+\dfrac{1}{2x}\right)}=1$.\\
        $\lim\limits_{x \to -\infty} \dfrac{\sqrt{4x^2-x+1}}{2x+1}=\lim\limits_{x \to -\infty} \dfrac{|2x|\sqrt{1-\dfrac{1}{4x}+\dfrac{1}{4x^2}}}{2x\left(1+\dfrac{1}{2x}\right)}=-1$.\\
        Tiệm cận ngang của đồ thị hàm số là $y= \pm 1$.
    }
\end{ex}
\begin{ex}%[2D1K4-1]
    Đồ thị hàm số $y=\dfrac{1-\sqrt{x^2+x+1}}{x^3+1}$ có tất cả bao nhiêu tiệm cận đứng và ngang?
    \shortans{$1$}
    \loigiai{
        Tập xác định $\mathscr{D}=\mathbb{R} \setminus \{-1\}$.
        \begin{itemize}
            \item
            {\allowdisplaybreaks
                \begin{eqnarray*}
                    \lim\limits_{x\to -1} \dfrac{1-\sqrt{x^2+x+1}}{x^3+1}&=&\lim\limits_{x\to -1} \dfrac{-x(x+1)}{(x+1)\left(x^2-x+1\right)\left(1+\sqrt{x^2+x+1}\right)}\\
                    &=&\lim\limits_{x\to -1} \dfrac{-x}{\left(x^2-x+1\right)\left(1+\sqrt{x^2+x+1}\right)}\\
                    &=&\dfrac{1}{6}.
            \end{eqnarray*} }
            \item $\lim\limits_{x\to +\infty}\dfrac{1-\sqrt{x^2+x+1}}{x^3+1}=0$.
        \end{itemize}
        Đồ thị hàm số không có tiệm cận đứng, tiệm cận ngang là $y=0$.
    }
\end{ex}
\begin{ex}%[2D1K4-1]
    Đồ thị hàm số $y=\dfrac{|x|}{\sqrt{x^2-1}}$ có tất cả bao nhiêu tiệm cận đứng và ngang?
    \shortans{$3$}
    \loigiai{
        Tập xác định $\mathscr{D}=(-\infty;-1) \cup (1;+\infty)$.
        \begin{itemize}
            \item $\lim\limits_{x\to-1^-}\dfrac{|x|}{\sqrt{x^2-1}}=+\infty$.
            \item $\lim\limits_{x\to 1^+}\dfrac{|x|}{\sqrt{x^2-1}}=+\infty$.
            \item $\lim\limits_{x\to +\infty}\dfrac{|x|}{\sqrt{x^2-1}}=1$.
            \item $\lim\limits_{x\to -\infty}\dfrac{|x|}{\sqrt{x^2-1}}=1$.
        \end{itemize}
        Đồ thị hàm số có $2$ tiệm cận đứng là $x=\pm 1$, tiệm cận ngang là $y=1$.
    }
\end{ex}
\begin{ex}%[2D1K4-1]
    Đồ thị hàm số $y=\dfrac{x}{\sqrt{x^2+1}}$ có tất cả bao nhiêu tiệm cận đứng và ngang?
    \shortans{$2$}
    \loigiai{
        Tập xác định $\mathscr{D}=\mathbb{R}$.
        \begin{itemize}
            \item $\lim\limits_{x\to +\infty}\dfrac{x}{\sqrt{x^2+1}}=1$.
            \item $\lim\limits_{x\to -\infty}\dfrac{x}{\sqrt{x^2+1}}=-1$.
        \end{itemize}
        Đồ thị hàm số không có tiệm cận đứng, tiệm cận ngang là $y=\pm 1$.
    }
\end{ex}
\begin{ex}%[2D1K4-1]
    Đồ thị hàm số $y=\dfrac{\sqrt{x^2-4}}{x^2-5x+6}$ có tất cả bao nhiêu tiệm cận đứng và ngang?
    \shortans{$3$}
    \loigiai{
        Tập xác định $\mathscr{D}=(-\infty;-2] \cup (2;+\infty) \setminus \{3\}$.
        \begin{itemize}
            \item $\lim\limits_{x\to 2^+}\dfrac{\sqrt{x^2-4}}{x^2-5x+6}=-\infty$.
            \item $\lim\limits_{x\to -2^-}\dfrac{\sqrt{x^2-4}}{x^2-5x+6}=-\infty$.
            \item $\lim\limits_{x\to +\infty}\dfrac{\sqrt{x^2-4}}{x^2-5x+6}=0$.
            \item $\lim\limits_{x\to -\infty}\dfrac{\sqrt{x^2-4}}{x^2-5x+6}=0$.
        \end{itemize}
        Đồ thị hàm số có $2$ tiệm cận đứng là $x=\pm 2$, tiệm cận ngang là $y=0$.
    }
\end{ex}
\begin{ex}
    Nồng độ thuốc trong máu $C(t)$ sau $t$ giờ khi uống một liều thuốc có thể được mô tả bởi hàm $C(t) = \dfrac{3}{1 + 2t}$. Tìm đường tiệm cận của nồng độ thuốc khi thời gian tăng lên rất lớn.
    \shortans{$0$}
\end{ex}
\begin{ex}
    Tốc độ (km/h) của một chiếc xe hơi tăng theo thời gian được mô tả bởi hàm $ v(t) = \dfrac{120t}{3+ t}$. Tìm đường tiệm cận của tốc độ khi thời gian tăng lên rất lớn.
    \shortans{$120$}
\end{ex}
\begin{ex}%[TeX hóa SGK CTST 12]%[Phạm Phương]%[2D1V4-4]
    Nồng độ oxygen trong hồ theo thời gian $t$ cho bởi công thức $y(t)=5-\dfrac{15t}{9t^{2}+1}$, với $y$ được tính theo mg/l và $t$ được tính theo giờ, $t \geq 0$. Tìm các đường tiệm cận của đồ thị hàm số $y(t)$. Từ đó, có nhận xét gì về nồng độ oxygen trong hồ khi thời gian $t$ trở nên rất lớn?
    \shortans{$y=5$, nồng độ tiến về $5$ mg/l}
    \loigiai{
        Hàm số $y(t)=5-\dfrac{15t}{9t^{2}+1}$ có tập xác định $\mathscr{D}=\mathbb{R}$.\\
        Ta có $\lim\limits_{x \rightarrow+\infty} \left(5-\dfrac{15t}{9t^{2}+1}\right)=5$.\\
        Vậy đồ thị hàm số có tiệm cận ngang là đường thẳng $y=5$.\\
        Khi thời gian $t$ trở nên rất lớn thì nồng độ oxygen trong hồ sẽ tiến dần về $5$ mg/l.
    }
\end{ex}
\begin{ex}
    Mô hình phát triển số lượng lợi khuẩn $P(t)$ theo thời gian có thể được mô tả bởi hàm $P(t) = \dfrac{100}{1 + 5e^{-2t}}$. Tính số lượng lợi khuẩn khi thời gian tăng lên rất lớn.\\
    \shortans{$100$}
\end{ex}
\begin{ex}
    Đáp ứng xung của một hệ thống điện tử the thời gian $t$ được mô tả bởi hàm \( h(t) = 120 e^{-\sqrt{3}t} \sin(2 t + \pi) \). Tìm và nêu ý nghĩa của đường tiệm cận của đáp ứng xung khi thời gian tăng.
    \shortans{$0$}
\end{ex}
\begin{ex}
    Điện áp của pin sạc theo thời gian được mô tả bởi hàm \( V(t) = 220 \left(1 - e^{-\dfrac{t}{\tau}}\right) \), trong đó \( \tau \) là hằng số thời gian. Tìm và nêu ý nghĩa của đường tiệm cận của điện áp khi thời gian tăng.
    \shortans{$220$}
\end{ex}
\begin{ex}%[0D1K1-4]
    Số lượng sản phẩm bán được của một công ty trong $x$ (tháng) được tính theo công thức $S(x)=200\left(5-\dfrac{9}{2+x}\right)$, trong đó $x\ge 1$ \emph{(Nguồn: R.Larson and B.Edwards, Calculus 10e, Cengage 2014).}
    \begin{enumerate}[a)]
        \item Xem $y=S(x)$ là một hàm số xác định trên nửa khoảng $[1;+\infty)$, hãy tìm tiệm cận ngang của đồ thị hàm số đó.
        \item Nêu nhận xét về số lượng sản phẩm bán được của công ty trong $x$ (tháng) khi $x$ đủ lớn.
    \end{enumerate}
    \shortans{$y=1$, sản phẩm gần $1\,000$}
    \loigiai{
        \begin{enumerate}[a)]
            \item Ta có $\lim\limits_{x\to +\infty}\left[200\left(5+\dfrac{9}{2-x}\right)\right]=200\cdot 5=1000$.\\
            Vậy $y=1\,000$ là tiệm cận ngang của đồ thị hàm số $y=S(x)$.
            \item Từ phần trên ta có thể rút ra nhận xét: khi số tháng đủ lớn thì công ty có thể bán được số sản phẩm gần bằng $1\,000$.
        \end{enumerate}
    }
\end{ex}
\begin{ex}
    Công ty cung cấp dịch vụ internet tính $75\$$ phí lắp đặt thiết bị ban đầu và phí sử dụng internet $40\$$ mỗi tháng
    \begin{listEX}
        \item Lập hàm số thể hiện chi phí sử dụng trung bình mỗi tháng sau $x$ tháng sử dụng
        \item Chi phí sử dụng trung bình thay đổi thế nào khi số tháng sử dụng tăng lên rất nhiều.
    \end{listEX}
    \shortans{$y=\dfrac{40x+75}{40}$, chi phí tiến về $40\$$}
\end{ex}
\begin{ex}
    Nhà trường dự định tổ chức tiệc liên hoan chào mừng lớp 12, tiền thuê hội trường là $1$ tỷ. Cứ mỗi người tham gia sẽ tính thêm phí phục vụ là $2$ triệu mỗi người. Gọi $x$ là số người tham gia bữa tiệc
    \begin{listEX}
        \item Lập hàm số thể hiện tổng chi phí của bữa tiệc
        \item Lập hàm số thể hiện chi phí trung bình của mỗi người bỏ ra cho bữa tiệc
        \item Chi phí trung bình của mỗi người thay đổi thế nào khi số người tham gia tăng lên rất lớn.
    \end{listEX}
    \shortans{$y=\dfrac{0,02x+1}{x}$, tiến về $2$ triệu}
\end{ex}
\begin{ex}
    Số lượng vi khuẩn trong một môi trường dinh dưỡng có thể được mô tả bởi hàm:
    \[
    N(t) = \dfrac{N_0}{1 - \dfrac{t}{T}}
    \]
    trong đó \( N(t) \) là số lượng vi khuẩn tại thời gian \( t \), \( N_0 \) là số lượng vi khuẩn ban đầu, và \( T \) là thời gian mà môi trường dinh dưỡng không còn đủ để hỗ trợ sự tăng trưởng của vi khuẩn. Xác định tiệm cận đứng của hàm số này và nêu ý nghĩa của nó.
    \shortans{$t=T$, khi $t$ tiến về $T$ thì số lượng vi khuẩn tăng lên vô hạn}
    \loigiai{
        Để tìm tiệm cận đứng, ta xem xét các giá trị của \( t \) làm cho mẫu số của phương trình bằng 0:
        \[
        1 - \frac{t}{T} = 0 \Rightarrow t = T
        \]
        Vậy đường thẳng \( t = T \) là tiệm cận đứng của đồ thị hàm số.
        \textbf{Ý nghĩa:} Từ đó ta suy ra khi thời gian \( t \) càng sát với \( T \), số lượng vi khuẩn \( N(t) \) sẽ tăng lên vô hạn. Điều này có nghĩa là khi thời gian tiếp cận \( T \) thì số lượng vi khuẩn sẽ tăng lên nhanh chóng đến mức vô hạn.}
\end{ex}
\begin{ex}
    Trong vật lý, vận tốc tối đa \(V\) của một vật rơi qua một chất lỏng được mô tả bằng phương trình:
    \[
    V(t) = \frac{mg}{b} \left(1 - e^{-\dfrac{bt}{m}}\right)
    \]
    trong đó \(m\) là khối lượng của vật, \(g\) là gia tốc trọng trường, \(b\) là hệ số ma sát, và \(t\) là thời gian. Xác định tiệm cận đứng của hàm số này và nêu ý nghĩa của nó.
    \shortans{Không có TCĐ}
    \loigiai{Để tìm tiệm cận đứng, ta xem xét các giá trị của \(t\) làm cho mẫu số của phương trình bằng 0. Tuy nhiên, trong trường hợp này, hàm số không có tiệm cận đứng vì biểu thức mũ đảm bảo hàm số được xác định cho tất cả các số thực.
        \textbf{Ý nghĩa:} Điều này ngụ ý rằng không có giới hạn về thời gian để vật đạt đến vận tốc tối đa. Khi thời gian tăng lên, vận tốc của vật sẽ tiệm cận đến vận tốc tối đa, nhưng vật không bao giờ thực sự đạt được nó.}
\end{ex}
\begin{ex}
    Trong sinh học, sự tăng trưởng của dân số \(P\) theo thời gian \(t\) có thể được mô hình bằng hàm số:
    \[
    P(t) = \frac{P_0}{1 - kP_0t}
    \]
    trong đó \(P_0\) là kích thước dân số ban đầu và \(k\) là hằng số tốc độ tăng trưởng. Xác định tiệm cận đứng của hàm số này và nêu ý nghĩa của nó.
    \shortans{$t = \frac{1}{kP_0}$}
    \loigiai{Để tìm tiệm cận đứng, ta xem xét các giá trị của \(t\) làm cho mẫu số của phương trình bằng 0:
        \[
        1 - kP_0t = 0 \Rightarrow t = \frac{1}{kP_0}
        \]
        Vậy tiệm cận đứng là \(t = \frac{1}{kP_0}\).
        \textbf{Ý nghĩa:} Điều này ngụ ý rằng hàm số tăng trưởng dân số có một tiệm cận đứng tại thời điểm \(t\) bằng nghịch đảo của tích của hằng số tốc độ tăng trưởng \(k\) và kích thước dân số ban đầu \(P_0\). Điều này chỉ ra một giới hạn cho tốc độ tăng trưởng dân số theo thời gian.}
\end{ex}
\begin{ex}
    Trong khoa học máy tính, độ phức tạp thời gian \(T(n)\) của một thuật toán với kích thước đầu vào \(n\) có thể được biểu diễn bằng hàm số:
    \[
    T(n) = \frac{an^2 + bn + c}{n}
    \]
    trong đó \(a\), \(b\), và \(c\) là các hằng số. Xác định tiệm cận đứng của hàm số này và nêu ý nghĩa của nó.
    \shortans{$T=0$}
    \loigiai{Để tìm tiệm cận đứng, ta xem xét các giá trị của \(n\) làm cho mẫu số của phương trình bằng 0:
        \[
        n = 0
        \]
        Vậy tiệm cận đứng là \(n = 0\).
        \textbf{Ý nghĩa:} Điều này ngụ ý rằng hàm số độ phức tạp thời gian không có tiệm cận đứng. Trong phân tích tính toán, một tiệm cận đứng tại \(n = 0\) sẽ ngụ ý rằng thuật toán có độ phức tạp thời gian vô hạn cho các đầu vào có kích thước bằng 0, điều này không có ý nghĩa trong hầu hết các trường hợp.}
\end{ex}
\Closesolutionfile{ans}
\begin{dang}{Đường tiệm cận liên quan tham số $m$}
\end{dang}
\begin{vd}
    Tìm $m$ để đồ thị hàm số
    \begin{listEX}[2]
        \item $y=\dfrac{x-2}{x^2-mx+1}$ có hai đường tiệm cận đứng.
        \item $y=\dfrac{x-1}{x^2-mx+1}$ có đúng ba đường tiệm cận.
        \item $y=\dfrac{\sqrt{x-3}}{x^2+x-m}$ có đúng hai đường tiệm cận.
        \item $y=\dfrac{\sqrt{1-x}}{x^2+4x+m}$ có ba đường tiệm cận.
        %	\item* $y=\dfrac{x}{x^2-2(m+1)x+m^2}$ có đúng hai đường tiệm cận.
    \end{listEX}
    \loigiai{}
\end{vd}
\BTTN
\Opensolutionfile{ans}[ans/2D1-4-DANG-2]
\begin{ex}%[Phạm Văn Long]%[Latex-HK2-TT-2020-2021]%[2D1K4-2]%
    Tìm $m$ để đồ thị hàm số $y=\dfrac{2x^2-3x+4}{x^2+mx+1}$ có duy nhất một đường tiệm cận?
    \choice
    {\True $m\in (-2;2)$}
    {$m\in [-2;2]$}
    {$m\in \{-2;2\}$}
    {$m\in (2;+\infty)$}
    \loigiai{
        Ta thấy đồ thị hàm số đã cho luôn có một tiệm cận ngang là đường $y=2$.\\
        Do đó, để đồ thị hàm số đã cho có duy nhất một đường tiệm cận thì đồ thị hàm số đã cho không có tiệm cận đứng.\\
        $\Rightarrow$ Phương trình $x^2+mx+1=0$ vô nghiệm $\Leftrightarrow \Delta <0 \Leftrightarrow m^2-4<0\Leftrightarrow m\in (-2;2)$.
    }
\end{ex}
\begin{ex}%[2D1K4-2]%[Đề GHK1, THPT Trần Nhân Tông, Hà Nội 2018]%[WTT2D1-128]%
    Tìm giá trị thực của tham số $m$ để đồ thị hàm số $y=\dfrac{x-4}{m-x^2}$ có đường tiệm cận đứng.
    \choice
    {$m\ge0;\,m\ne16$}
    {\True $m\ge0$}
    {$m>0$}
    {$m>0;\,m\ne16$}
    \loigiai{
        Điều kiện xác định: $m-x^2\neq0$.\\
        Để đồ thị hàm số có đường tiệm cận đứng thì phương trình $m-x^2=0$ có nghiệm, tức là $m\ge0$.\\
        Với $m=16$ thì $y=\dfrac{-1}{4+x}$ có một tiệm cận đứng là $x=-4$. Vậy giá trị $m$ cần tìm là $m\ge0$.
    }
\end{ex}
\begin{ex}%[2D1K4-2]%
    Có bao nhiêu giá trị của tham số $m$ thoả mãn đồ thị hàm số $y=\dfrac{x+3}{x^2-x-m}$ có đúng hai đường tiệm cận?
    \choice
    {$1$}
    {$4$}
    {\True $2$}
    {$3$}
    \loigiai{
        Đồ thị hàm số có đúng hai đường tiệm cận khi phương trình $x^2-x-m=0$ có nghiệm kép hoặc có hai nghiệm phân biệt với một nghiệm bằng $-3$. Khi đó
        \[\hoac{&\Delta=0\\&\heva{&\Delta>0\\&g(-3)=0}}\Leftrightarrow\hoac{&4m+1=0\\&\heva{&4m+1>0\\&m=12}}\hoac{&m=-\dfrac{1}{4}\\&m=12.}\]
        Vậy có hai giá trị của m.}
\end{ex}
\begin{ex}%[2D1K4-2]%[Đề kiểm tra giữa học kì I, 2017 - 2018 trường THPT Chu Văn An, Hà Nội]%[WTT2D1-156]%
    Tìm tất cả các giá trị thưc của tham số $m$ để đồ thị hàm số $y=\dfrac{x^2+m}{x^2-3x+2}$ có đúng hai tiệm cận.
    \choice
    {$m=-1$}
    {$m\in\left\{1;4\right\}$}
    {\True $m\in\left\{-1;-4\right\}$}
    {$m=4$}
    \loigiai{
        Vì $\lim\limits_{x\to\pm\infty}\dfrac{x^2+m}{x^2-3x+2}=1,\,\forall m$ nên đồ thị hàm số luôn có một tiêm cận ngang là $y=1$.\\
        Để đồ thị hàm số có đúng hai tiệm cận thì đồ thị hàm số có thêm một tiệm cận đứng là $x=1$ hoặc là $x=2$.
        \begin{itemize}
            \item Đồ thị hàm số có một tiệm cận đứng $x=1$, suy ra pt $x^2+m=0$ và phương trình $x^2-3x+2=0$ có nghiệm chung là $x=1\Rightarrow m=-1$.
            \item Đồ thị hàm số có một tiệm cận đứng $x=2$, suy ra pt $x^2+m=0$ và phương trình $x^2-3x+2=0$ có nghiệm chung là $x=2\Rightarrow m=-4$.
        \end{itemize}
        Vậy $m\in\left\{-1;4\right\}$ thỏa yêu cầu bài toán.
    }
\end{ex}
\begin{ex}%[Thi thử, THPT Lục Ngạn - Bắc Giang, 2019]%[Trần Như Ngọc, 12EX3-2019]%[2D1K4-2]%
    Có bao nhiêu giá trị nguyên dương của tham số $m$ để đồ thị hàm số
    $y=\dfrac{\sqrt{9-x}}{x^2-2(m+1)x+m^2+2m}$
    có đúng hai đường tiệm cận.
    \choice
    {\True $2$}
    {$1$}
    {$4$}
    {$3$}
    \loigiai{
        Ta có $ x^2-2(m+1)x+m^2+2m = 0 \Leftrightarrow \hoac{& x=m \\ & x=m+2}$
        $( \Delta ' = 1 )$. \\
        Hàm số xác định khi $ \heva{& x \le 9 \\ & x \ne m \\ & x \ne m+2.} $\\
        Ta có $\lim \limits_{x\to -\infty}y = 0$ nên đồ thị hàm số có một tiệm cận ngang là $ y = 0 $.\\
        Đồ thị hàm số có đúng hai tiệm cận khi và chỉ khi nó có đúng một tiệm cận đứng \\
        $\Leftrightarrow$ phương trình trên có một nghiệm nhỏ hơn hoặc bằng $ 9 $.\\
        $\Leftrightarrow m \le 9 < m+2 \Leftrightarrow 7 < m \le 9 $.\\
        Vậy có $ 2 $ giá trị $ m $ nguyên dương thỏa mãn điều kiện bài toán.
    }
\end{ex}
\begin{ex}%[Thi học kỳ I, Trường THPT Chuyên Lê Quý Đôn - Khánh Hòa, 2021]%[Lê Hồng Phi, 12EX5]%[2D1K4-2]%
    Cho hàm số $y=\dfrac{2x-3}{\sqrt{x^2+2(m-2)x+m^2}}$ với $m$ là tham số thực và $m>1$. Hỏi đồ thị hàm số có bao nhiêu đường tiệm cận (tiệm cận ngang và tiệm cận đứng)?
    \choice
    {$1$}
    {\True $2$}
    {$3$}
    {$4$}
    \loigiai
    {Phương trình $x^2+2(m-2)x+m^2=0$ có $\Delta'=(m-2)^2-m^2=-2(2m-2)=-4(m-1)<0,\ \forall m>1$ nên vô nghiệm.\\
        Do đó tập xác định của hàm số là $\mathscr{D}=\mathbb{R}$.\\
        Như thế đồ thị hàm số không có đường tiệm cận đứng.\\
        Ta tính được
        \begin{itemize}
            \item $\lim\limits_{x\to +\infty}y=\lim\limits_{x\to +\infty}\dfrac{2-\dfrac{3}{x}}{\sqrt{1+\dfrac{2(m-2)}{x}+\dfrac{m^2}{x}}}=2$ nên $y=2$ là đường tiệm cận ngang.
            \item $\lim\limits_{x\to -\infty}y=\lim\limits_{x\to -\infty}\dfrac{2-\dfrac{3}{x}}{-\sqrt{1+\dfrac{2(m-2)}{x}+\dfrac{m^2}{x}}}=-2$ nên $y=-2$ là đường tiệm cận ngang.
        \end{itemize}
        Vậy đồ thị hàm số đã cho có $2$ đường tiệm cận.
    }
\end{ex}
\begin{ex}%[Đề Khảo sát lần 1 THPT Quang Hà - Vĩnh Phúc, 2021]%[Trần Nhân Kiệt, 12EX4-2021]%[2D1K4-2]%
    Có bao nhiêu giá trị nguyên của tham số $m$ để đồ thị tham số $y=\dfrac{1+\sqrt{x+1}}{\sqrt{x^2-(1-m)x+2m}}$ có hai tiệm cận đứng?
    \choice
    {$2$}
    {\True $3$}
    {$1$}
    {$0$}
    \loigiai{
        Điều kiện $\heva{& x\ge -1 \\ & x^2-(1-m)x+2m>0.}$\\
        Đồ thị hàm số có hai tiệm cận đứng khi và chỉ khi phương trình $x^2-(1-m)x+2m=0$ có hai nghiệm phân biệt lớn hơn hoặc bằng $-1$.\\
        Ta có $x^2-(1-m)x+2m=0\Leftrightarrow x^2-x+m(x+2)=0\Leftrightarrow m=\dfrac{-x^2+x}{x+2}$.\\
        Đặt $f(x)=\dfrac{-x^2+x}{x+2}$, $x\ge -1$.\\
        Ta có $f'(x)=\dfrac{-x^2-4x+2}{(x+2)^2}$, suy ra $f'(x)=0\Leftrightarrow -x^2-4x+2=0\Leftrightarrow x=-2\pm \sqrt{6}$.
        \begin{center}
            \begin{tikzpicture}[>=stealth]
                \tkzTabInit[nocadre=false,lgt=1.2,espcl=3,deltacl=0.5]
                {$x$/.7 ,$f'(x)$/.7,$f(x)$/2}
                {$-1$ , $-2+\sqrt{6}$ , $+\infty$}
                \tkzTabLine{ , - , $0$ , + , }
                \tkzTabVar{-/$-2$ , +/$5-2\sqrt{6}$ , -/$-\infty$}
            \end{tikzpicture}
        \end{center}
        Từ bảng biến thiên suy ra $m\in [-2;5-2\sqrt{6})$.\\
        Vì $m$ nguyên nên $m\in \{-2;-1;0\}$.\\
        Vậy có $3$ giá trị nguyên của $m$ thỏa mãn bài.
    }
\end{ex}
\BTTL
\begin{ex}%[2D1K4-2]%
    Cho hàm số $y=\dfrac{2x^2-3x+m}{x-m}$ có đồ thị $(C)$. Với tất cả các giá trị thực nào của tham số $m$ thì đồ thị $(C)$ không có tiệm cận đứng?
    \shortans{$m=0$ hoặc $m=1$}
    %	\choice
    %	{$m=2$}
    %	{$m=0$}
    %	{$m=1$}
    %	{\True $m=0$ hoặc $m=1$}
    \loigiai{
        Đồ thị không có tiệm cận đứng khi $x=m$ là nghiệm của phương trình $2x^2-3x+m=0$, suy ra $2m^2-3m+m=0 \Leftrightarrow \hoac{&m=0\\&m=1}$.
    }
\end{ex}
\begin{ex}%[2D1K4-2]%
    Với tất cả các giá trị thực nào của tham số $m$ thì đồ thị hàm số $y=\dfrac{x^2+x-2}{x^2+x+m}$ có ba đường tiệm cận?
    \shortans{$m<\dfrac{1}{4}$ và $m\ne -2$}
    %	\choice
    %	{$m>\dfrac{1}{4}$ và $m\ne 2$}
    %	{$m>\dfrac{1}{4}$}
    %	{$m<\dfrac{1}{4}$}
    %	{\True $m<\dfrac{1}{4}$ và $m\ne -2$}
    \loigiai{
        Đồ thị hàm số chỉ có $1$ tiệm cận ngang là $y=1$.\\
        Ta có $x^2+x-2 \Leftrightarrow \hoac{&x=1\\&x=-2.}$\\
        Đồ thị hàm số có ba đường tiệm cận khi và chỉ khi có $2$ tiệm cận đứng. Điều này tương đương với phương trình $x^2+x+m=0$ có $2$ nghiệm phân biệt khác $1$ và $-2$, nghĩa là\\
        $\heva{&1-4m>0\\&1^2+1+m \ne 0\\& (-2)^2-2+m\ne 0} \Leftrightarrow \heva{&m<\dfrac{1}{4}\\&m\ne -2.}$
    }
\end{ex}
\begin{ex}%[đề thi thử THPT Quốc gia, đề số 3, nguyễn hoàng thanh]%[2D1K4-2]%
    Tìm số giá trị nguyên thuộc đoạn $ [-2025;2025] $ của tham số $ m $ để đồ thị hàm số $ y=\dfrac{\sqrt{x-3}}{x^2+x-m} $ có đúng hai đường tiệm cận.
    \shortans{$2014$}
    %	\choice
    %	{$ 2007 $}
    %	{$ 2010 $}
    %	{$ 2009 $}
    %	{\True $ 2008 $}
    \loigiai{
        Điều kiện xác định của hàm số $ \heva{& x\ge 3\\& x^2+x-m\ne 0.} $\\
        Vì $ \lim\limits_{x\to +\infty}\dfrac{\sqrt{x-3}}{x^2+x-m}=\lim\limits_{x\to +\infty}\dfrac{\sqrt{\frac{1}{x}-\frac{3}{x^2 }}}{1+\frac{1}{x}-\frac{m}{x^2}}=0 $, suy ra $ y=0 $ là tiệm cận ngang.\\
        Để đồ thị hàm số có đúng hai tiệm cận thì đồ thị hàm số chỉ có thêm một tiệm cận đứng, tương đương $ f(x)=x^2+x-m $ có đúng một nghiệm lớn hơn $ 3 $. Xét các trường hợp xảy ra như sau
        \begin{enumerate}
            \item $ f(x)=0 $ có nghiệm kép $ x_{1}=x_2=-\dfrac{1}{2}<3 $ (không thỏa mãn).
            \item $ f(x)=0 $ có hai nghiệm thỏa $ x_1<3\le x_2\Leftrightarrow a\cdot f(3)\le 0\Leftrightarrow 12-m\le 0\Leftrightarrow m\ge 12 $.
        \end{enumerate}
        Kết hợp với yêu cầu bài toán ta suy ra $ \heva{&m\in \mathbb{Z}\\ &m\in[12;2025]} $, suy ra có $ 2025-12+1=2014 $ giá trị nguyên của $ m $ thỏa mãn bài toán.
    }
\end{ex}
\begin{ex}%[Đề tham khảo THPT Quốc gia 2021 - Đề 5]%[Đoàn Minh Tân]%[2D1K4-2]%
    Tìm tất cả giá trị thực của tham số $m$ để đồ thị hàm số $y=\dfrac{3x+2018}{\sqrt{mx^2+5x+6}}$ có hai đường tiệm cận ngang.
    \shortans{$m>0$}
    %	\choice
    %	{$m\in \varnothing$}
    %	{$m<0$}
    %	{$m=0$}
    %	{\True $m>0$}
    \loigiai{
        Ta có $\displaystyle \lim \limits_{x\to +\infty} y=\displaystyle\lim\limits_{x\to +\infty}\dfrac{3x+2018}{\sqrt{mx^2+5x+6}}=\displaystyle\lim\limits_{x\to +\infty}\dfrac{3+\dfrac{2018}{x}}{\sqrt{m+\dfrac{5}{x}+\dfrac{6}{x^2}}}=\dfrac{3}{\sqrt{m}}$ tồn tại khi $m>0$.\\
        $\displaystyle\lim\limits_{x\to -\infty}=\displaystyle\lim\limits_{x\to -\infty}\dfrac{3x+2018}{\sqrt{mx^2+5x+6}}=\displaystyle\lim\limits_{x\to -\infty}\dfrac{3+\dfrac{2018}{x}}{-\sqrt{m+\dfrac{5}{x}+\dfrac{6}{x^2}}}=-\dfrac{3}{\sqrt{m}}$ tồn tại khi $m>0$.\\
        Hiên nhiên $\displaystyle\lim\limits_{x\to +\infty}y\ne \displaystyle \lim \limits_{x\to -\infty}y$.\\
        Vậy đồ thị hàm số đã cho có hai tiệm cận ngang khi và chỉ khi $m>0$.
    }
\end{ex}
\begin{ex}%[2D1K4-2]%
    Có bao nhiêu giá trị nguyên của tham số thực $m$ thuộc đoạn $[-20; 10]$ để đồ thị hàm số $y=\dfrac{x+2}{\sqrt{x^2-4x+m}}$ có hai đường tiệm cận đứng?
    \shortans{$23$}
    %	\choice
    %	{$20$}
    %	{$21$}
    %	{$22$}
    %	{\True $23$}
    \loigiai{
        Đồ thị hàm số có hai đường tiệm cận đứng $\Leftrightarrow$ phương trình $x^2-4x+m=0$ có hai nghiệm phân biệt khác $-2$ \\
        $ \Leftrightarrow\heva{&2^2-m>0\\&(-2)^2-4\cdot (-2)+m\neq 0}\Leftrightarrow\heva{&m<4\\&m\neq-12.} $ \\
        Do $m$ nguyên và $m\in[-20; 10]$ nên $m\in\left\{-20;-19;\ldots;-13;-11;\ldots; 2; 3\right\}$, gồm $23$ giá trị thỏa mãn.}
\end{ex}
\Closesolutionfile{ans}
\begin{dang}{Tìm các đường tiệm cận đồ thị hàm ẩn}
\end{dang}
\begin{vd}
    Cho hàm số $y=f(x)$ có bảng biến thiên như hình vẽ sau
    \begin{center}
        \begin{tikzpicture}[>=stealth]
            \tkzTabInit[nocadre=false,lgt=1,espcl=1.5,deltacl=0.5]{$x$/.7 ,$y'$/.7,$y$/2}
            {$-\infty$ , $-1$ , $2$ , $+\infty$}
            \tkzTabLine{ , + , $0$ , - , d , + , }
            \tkzTabVar{-/$1$ , +/$4$ , -/$-5$ , +/$+\infty$}
        \end{tikzpicture}
    \end{center}
    Tìm TCĐ, TCN của đồ thị hàm số
    \begin{listEX}[3]
        \item $y=\dfrac{2}{f(x)-3}$
        \item $y=\dfrac{-3}{f(x)+2}$
        \item $y=\dfrac{x-2}{f(x)+5}$
        \item $y=\dfrac{x+1}{f(x)-4}$
        \item $y=\dfrac{2}{f(x^2)+3}$
        \item $y=\dfrac{4f(x)-5}{3f(x)+1}$
    \end{listEX}
    \loigiai{}
\end{vd}
\begin{vd}\immini{Cho hàm bậc ba $y=f(x)$ có đồ thị như hình vẽ. Tìm số tiệm cận đứng của đồ thị hàm số
        \begin{listEX}[2]
            \item $y=\dfrac{\sqrt{x+3}}{(x-1)f(x)}$
            \item $g(x)=\dfrac{(x^2+4x+3)\sqrt{x^2+x}}{x\left[f^2(x)-2f(x)\right]}$ .
    \end{listEX}}{\begin{tikzpicture}[line cap=round,line join=round, >=stealth,font=\footnotesize]
            \begin{scope}[scale=.5]
                \def\a{-1} % Hệ số a phải khác 0
                \def\b{-13/2}
                \def\c{-12}
                \def\d{-9/2}
                \draw[->] (-5,0) -- (2,0)node[below]{$x$};
                \draw[->] (0,-3) -- (0,4) node[left] {$y$};
                \draw (0,0)node[below right]{$O$} (-3,0)node[below]{$-3$};
                \draw[dashed] (-1,0)node[below]{$-1$}|-(0,2)node[right]{$2$};
                \draw[samples=150,smooth,domain=-4:.-.2] plot(\x,{\a*(\x)^3+(\b)*(\x)^2+(\c)*\x+(\d)});
            \end{scope}
    \end{tikzpicture}}
    \loigiai{
        \begin{center}
            \begin{tikzpicture}[line cap=round,line join=round, >=stealth,font=\footnotesize,scale=1]
                \def\a{-1} % Hệ số a phải khác 0
                \def\b{-13/2}
                \def\c{-12}
                \def\d{-9/2}
                \draw[->] (-5,0) -- (2,0)node[below]{$x$};
                \draw[->] (0,-3) -- (0,4) node[left] {$y$};
                \draw (0,0)node[below right]{$O$} (-3,0)node[below]{$-3$} (-.3,0)node[above]{$a$};
                \draw[dashed] (-3.78,0)node[below]{$c$}|-(0,2)|-(-1.71,0)node[below]{$b$}|-(0,2) (-1,0)node[below]{$-1$}|-(0,2)node[right]{$2$};
                \draw[samples=150,smooth,domain=-4:.-.2] plot(\x,{\a*(\x)^3+(\b)*(\x)^2+(\c)*\x+(\d)});
            \end{tikzpicture}
        \end{center}
        $g(x)=\dfrac{(x^2+4x+3)\sqrt{x^2+x}}{x\left[f^2(x)-2f(x)\right]}=\dfrac{(x+1)(x+3)\sqrt{x(x+1)}}{x\left[f^2(x)-2f(x)\right]}$.\\
        Điều kiện của căn là $x\le -1; x\ge 0$.\\
        Dựa vào đồ thị ta có \[x\left[f^2(x)-2f(x)\right]=0 \Leftrightarrow \hoac{&x=0\\&f(x)=0\\& f(x)=2} \Leftrightarrow \hoac{&x=0\text{ (nhận)}\\&x=-3\text{ (nhận)};\ x=a \text{ (loại)} \\&x=-1\text{ (nhận)};\ x=b\text{ (nhận)};\ x=c\text{ (nhận)}}\]\\
        Số TCĐ lúc này chính là số nghiệm không bị rút gọn của mẫu, vậy có bốn TCĐ là $x=0; x=-3; x=b; x=c$.
    }
\end{vd}
\BTTN
\Opensolutionfile{ans}[ans/2D1-4-DANG-3]
\begin{ex}%[2D1K4-1]
    Cho hàm số $y=f(x)$ có bảng biến thiên như hình bên. Đồ thị hàm số $y=\dfrac{-5}{f(x)+4}$ có bao nhiêu tiệm cận đứng?
    \begin{center}
        \begin{tikzpicture}[scale=0.8]
            \tkzTabInit[nocadre=false,lgt=1.5,espcl=3,deltacl=0.6]
            {$x$ /0.6,$y’$ /0.6,$y$ /2}
            {$-\infty$ ,$1$, $2$, $+\infty$}
            \tkzTabLine{,+,d,-,d,+,}
            \tkzTabVar{-/$-4$,+/$3$,-/$-5$,+/$+\infty$}
        \end{tikzpicture}
    \end{center}
    \choice
    {$1$}
    {$3$}
    {\True $2$}
    {$4$}
    \loigiai{
        Dựa vào bảng biến thiên suy ra
        $f(x)+4=0 \Leftrightarrow f(x) =-4$, phương trình này có $2$ nghiệm phân biệt nên đồ thị hàm số $y=\dfrac{-5}{f(x)+4}$ có $2$ tiệm cận đứng.
    }
\end{ex}
\begin{ex}%[2D1K4-1]
    Cho hàm số $y=f(x)$ có bảng biến thiên như hình bên. Đồ thị hàm số $y=\dfrac{x+2}{2f(x)-1}$ có bao nhiêu tiệm cận đứng?
    \begin{center}
        \begin{tikzpicture}[scale=0.8]
            \tkzTabInit[nocadre=false,lgt=1.5,espcl=3,deltacl=0.6]
            {$x$ /0.6,$y’$ /0.6,$y$ /2}
            {$-\infty$ ,$-1$, $0$, $1$, $+\infty$}
            \tkzTabLine{,+,0,-,0,+,0,-,}
            \tkzTabVar{-/$-\infty$,+/$0$,-/$-\dfrac{5}{3}$,+/$0$,-/$-\infty$}
        \end{tikzpicture}
    \end{center}
    \choice
    {$1$}
    {$3$}
    {$2$}
    {\True $0$}
    \loigiai{
        Dựa vào bảng biến thiên suy ra
        $2f(x)-1=0 \Leftrightarrow f(x) =\dfrac{1}{2}$, phương trình này có $0$ nghiệm nên đồ thị hàm số $y=\dfrac{x+2}{2f(x)-1}$ không có tiệm cận đứng.
    }
\end{ex}
%69
\begin{ex}%[2D1K4-1]
    Cho hàm số $y=f(x)$ có bảng biến thiên như hình bên. Đồ thị hàm số $y=\dfrac{1}{2f(x)-3}$ có bao nhiêu tiệm cận đứng?
    \begin{center}
        \begin{tikzpicture}[scale=0.8]
            \tkzTabInit[nocadre=false,lgt=1.5,espcl=3,deltacl=0.6]
            {$x$ /0.6,$y’$ /0.6,$y$ /2}
            {$-\infty$ ,$0$, $1$, $+\infty$}
            \tkzTabLine{,+,0,-,0,+,}
            \tkzTabVar{-/$-\infty$,+/$5$,-/$-1$,+/$+\infty$}
        \end{tikzpicture}
    \end{center}
    \choice
    {$1$}
    {\True $3$}
    {$2$}
    {$0$}
    \loigiai{
        Dựa vào bảng biến thiên suy ra
        $2f(x)-3=0 \Leftrightarrow f(x) =-\dfrac{3}{2}$, phương trình này có $3$ nghiệm phân biệt nên đồ thị hàm số $y=\dfrac{1}{2f(x)-3}$ có ba tiệm cận đứng.
    }
\end{ex}
%70
%71
%72
\begin{ex}%[2D1K4-1]
    Cho hàm số $y=f(x)$ có bảng biến thiên như hình bên. Đồ thị hàm số $y=\dfrac{x}{f(x)-3}$ có bao nhiêu tiệm cận đứng?
    \begin{center}
        \begin{tikzpicture}[scale=0.8]
            \tkzTabInit[nocadre=false,lgt=1.5,espcl=3,deltacl=0.6]
            {$x$ /0.6,$y’$ /0.6,$y$ /2}
            {$-\infty$ ,$-1$, $0$, $1$, $+\infty$}
            \tkzTabLine{,-,0,+,0,-,0,+,}
            \tkzTabVar{+/$+\infty$,-/$0$,+/$3$,-/$0$,+/$+\infty$}
        \end{tikzpicture}
    \end{center}
    \choice
    {$1$}
    {\True $3$}
    {$2$}
    {$4$}
    \loigiai{
        Dựa vào bảng biến thiên suy ra
        $f(x)-3=0 \Leftrightarrow f(x) =3$, phương trình này có $2$ nghiệm phân biệt khác $0$ và một nghiệm bội chẵn $x=0$ nên đồ thị hàm số $y=\dfrac{x}{f(x)-3}$ có ba tiệm cận đứng.
    }
\end{ex}
\begin{ex}%[2D1K4-1]
    Cho hàm số $y=f(x)$ có bảng biến thiên như hình bên. Đồ thị hàm số $y=\dfrac{4}{f(x)+1}$ có tiệm cận ngang là đường thẳng
    \begin{center}
        \begin{tikzpicture}[scale=0.8]
            \tkzTabInit[nocadre=false,lgt=1.5,espcl=3,deltacl=0.6]
            {$x$ /0.6,$y’$ /0.6,$y$ /2}
            {$-\infty$ ,$-1$, $2$, $+\infty$}
            \tkzTabLine{,+,0,-,0,+,}
            \tkzTabVar{-/$1$,+/$4$,-/$-5$,+/$1$}
        \end{tikzpicture}
    \end{center}
    \choice
    {$y=1$}
    {$y=-5$}
    {\True $y=2$}
    {$y=4$}
    \loigiai{
        Dựa vào bảng biến thiên suy ra
        $\lim \limits_{x \to \pm \infty} f(x)=1 \Leftrightarrow \lim \limits_{x \to \pm \infty} \dfrac{4}{f(x)+1} =2$ nên đồ thị hàm số đã cho có tiệm cận ngang là $y=2$.
    }
\end{ex}
\begin{ex}%[2D1K4-1]
    Cho hàm số $y=f(x)$ có bảng biến thiên như hình bên. Đồ thị hàm số $y=\dfrac{2-f(x)}{f(x)+3}$ có tiệm cận ngang là đường thẳng
    \begin{center}
        \begin{tikzpicture}[scale=0.8]
            \tkzTabInit[nocadre=false,lgt=1.5,espcl=3,deltacl=0.6]
            {$x$ /0.6,$y’$ /0.6,$y$ /2}
            {$-\infty$ ,$0$, $2$, $+\infty$}
            \tkzTabLine{,-,0,+,0,-,}
            \tkzTabVar{+/$+\infty$,-/$1$,+/$5$,-/$-\infty$}
        \end{tikzpicture}
    \end{center}
    \choice
    {$y=1$}
    {$y=-3$}
    {$y=2$}
    {\True $y=-1$}
    \loigiai{
        Dựa vào bảng biến thiên suy ra
        $\lim \limits_{x \to \pm \infty} f(x)=\pm \infty \Leftrightarrow \lim \limits_{x \to \pm \infty} \dfrac{2-f(x)}{f(x)+3} =-1$ nên đồ thị hàm số $y=\dfrac{2-f(x)}{f(x)+3}$ có tiệm cận ngang là $y=-1$.
    }
\end{ex}
\begin{ex}%[2D1K4-1]
    Cho hàm số $y=f(x)$ có bảng biến thiên như hình bên. Đồ thị hàm số $y=\dfrac{1}{f^2(x)-4f(x)+4}$ có bao nhiêu tiệm cận đứng?
    \begin{center}
        \begin{tikzpicture}[scale=0.8]
            \tkzTabInit[nocadre=false,lgt=1.5,espcl=3,deltacl=0.6]
            {$x$ /0.6,$y’$ /0.6,$y$ /2}
            {$-\infty$, $2$, $+\infty$}
            \tkzTabLine{,-,0,+,}
            \tkzTabVar{+/$1$,-/$-3$,+/$1$}
        \end{tikzpicture}
    \end{center}
    \choice
    {$1$}
    {$3$}
    {$2$}
    {$0$}
    \loigiai{
        Dựa vào bảng biến thiên suy ra $f^2(x)-4f(x)+4=0 \Leftrightarrow f(x)=2$, phương trình $f(x)=2$ vô nghiệm nên đồ thị hàm số đã cho không có tiệm cận đứng.
    }
\end{ex}
%83
\begin{ex}%[2D1K4-1]
    Cho hàm số $y=f(x)$ có bảng biến thiên như hình bên. Đồ thị hàm số $y=\dfrac{1}{f(3-x)-2}$ có bao nhiêu tiệm cận đứng?
    \begin{center}
        \begin{tikzpicture}[scale=0.8]
            \tkzTabInit[nocadre=false,lgt=1.5,espcl=3,deltacl=0.6]
            {$x$ /0.6,$y’$ /0.6,$y$ /2}
            {$-\infty$ ,$-2$, $2$, $+\infty$}
            \tkzTabLine{,+,0,-,0,+,}
            \tkzTabVar{-/$-\infty$,+/$3$,-/$0$,+/$+\infty$}
        \end{tikzpicture}
    \end{center}
    \choice
    {$1$}
    {\True $3$}
    {$2$}
    {$0$}
    \loigiai{
        Dựa vào bảng biến thiên suy ra $f(3-x)-2=0 \Leftrightarrow f(3-x)=2$, phương trình này có $3$ nghiệm phân biệt nên đồ thị hàm số đã cho có $3$ tiệm cận đứng.
    }
\end{ex}
\begin{ex}%[2D1G4-1]
    Cho hàm số $y=f(x)$ có bảng biến thiên như hình bên. Đồ thị hàm số $y=\dfrac{4}{f(x^2)-2}$ có bao nhiêu tiệm cận đứng?
    \begin{center}
        \begin{tikzpicture}[scale=0.8]
            \tkzTabInit[nocadre=false,lgt=1.5,espcl=3,deltacl=0.6]
            {$x$ /0.6,$y’$ /0.6,$y$ /2}
            {$-\infty$ ,$0$, $3$, $+\infty$}
            \tkzTabLine{,-,0,+,d,-,}
            \tkzTabVar{+/$8$,-/$1$,+/$4$,-/$2$}
        \end{tikzpicture}
    \end{center}
    \choice
    {$5$}
    {$3$}
    {\True $2$}
    {$4$}
    \loigiai{
        Dựa vào bảng biến thiên suy ra
        $f(x^2)-2=0 \Leftrightarrow f(x^2) =2$. Kẻ đường thẳng $y=2$ ta thấy đường thẳng cắt đồ thị hàm số tại hai điểm phân biệt. Suy ra
        $$\hoac{&x^2=a \; (a<0)\\&x^2=b \; (b >0)} \Rightarrow x=\pm \sqrt{b}.$$
        Do đó đồ thị hàm số đã cho có $2$ tiệm cận đứng.
    }
\end{ex}%89
\begin{ex}%[2D1G4-1]
    Cho hàm số $y=f(x)$ có bảng biến thiên như hình bên. Đồ thị hàm số $y=\dfrac{2}{f(|x|)-3}$ có bao nhiêu tiệm cận ngang?
    \begin{center}
        \begin{tikzpicture}[scale=0.8]
            \tkzTabInit[nocadre=false,lgt=1.5,espcl=3,deltacl=0.6]
            {$x$ /0.6,$y’$ /0.6,$y$ /2}
            {$-\infty$ ,$0$, $2$, $+\infty$}
            \tkzTabLine{,+,0,-,0,+,}
            \tkzTabVar{-/$-\infty$,+/$3$,-/$-1$,+/$+\infty$}
        \end{tikzpicture}
    \end{center}
    \choice
    {$4$}
    {\True $3$}
    {$5$}
    {$6$}
    \loigiai{
        Dựa vào bảng biến thiên suy ra
        $f(|x|)-3=0 \Leftrightarrow f(|x|) =3$.\\
        Bảng biến thiên hàm số $y=f(|x|)$ như sau
        \begin{center}
            \begin{tikzpicture}[scale=0.8]
                \tkzTabInit[nocadre=false,lgt=1.5,espcl=3,deltacl=0.6]
                {$x$ /0.6,$y’$ /0.6,$y$ /2}
                {$-\infty$ ,$-2$, $0$, $2$, $+\infty$}
                \tkzTabLine{,-,0,+,0,-,0,+,}
                \tkzTabVar{+/$+\infty$,-/$-1$,+/$3$,-/$-1$,+/$+\infty$}
            \end{tikzpicture}
        \end{center}
        Dựa vào bảng biến thiên hàm số $y=f(|x|)$, phương trình $f(|x|) =3$ có ba nghiệm phân biệt, do đó đồ thị hàm số $y=\dfrac{2}{f(|x|)-3}$ có $3$ tiệm cận đứng.
    }
\end{ex}
\begin{ex}
    \immini{ %Câu 90
        Cho hàm số bậc ba $f(x)= ax^3 +bx^2 +cx +d$ có đồ thị như hình vẽ bên. Đồ thị hàm số $g(x) = \dfrac{\sqrt{x+1}}{(x-3)\cdot f(x)}$ có bao nhiêu đường tiệm cận đứng?
        \choice
        {5}
        {2}
        {4}
        {\True 3}}{\begin{tikzpicture}[scale=.5, font=\footnotesize, line join=round, line cap=round, >=stealth]
            \def\xmin{-3}\def\xmax{3}\def\ymin{-5}\def\ymax{1}
            \draw[->] (\xmin-0.2,0)--(\xmax+0.2,0) node[below] {\footnotesize $x$};
            \draw[->] (0,\ymin-0.2)--(0,\ymax+0.2) node[right] {\footnotesize $y$};
            \draw (0,0) node [below left] {\footnotesize $O$};
            \foreach \x in {-1}\draw (\x,-0.1)--(\x,0.1) node [above] {\footnotesize $\x$};
            \foreach \x in {2}\draw (\x,-0.1)--(\x,0.1) node [above right] {\footnotesize $\x$};
            \foreach \y in {}\draw (-0.1,\y)--(0.1,\y) node [right] {\footnotesize $\y$};
            \clip (\xmin,\ymin) rectangle (\xmax,\ymax);
            \draw[smooth,samples=200,domain=\xmin:\xmax] plot (\x,{1*((\x)^3)+0*((\x)^2)+-3*(\x)+-2});
        \end{tikzpicture}
    }
    \loigiai{
        * Điều kiện: $\heva{&x \ne 3\\&f(x) \ne 0\\&x \ge -1.}$\\
        Nhìn hình vẽ ta thấy
        $f(x)=0\Leftrightarrow \hoac{&x=-1&(\text{nghiệm kép}) \\&x=2&(\text{nghiệm đơn}).}$\\
        Vậy $g(x) = \dfrac{\sqrt{x+1}}{(x-3)\cdot a(x+1)^2 (x-2)}.$ \\
        Đồ thị hàm số $g(x)$ có 3 đường tiệm cận đứng.}
\end{ex}
\begin{ex}
    \immini{ %Câu 92.
        Đường cong ở hình bên là đồ thị của hàm số $y = ax^3 +bx^2 +cx+d$. Đồ thị hàm số $y =\dfrac{(2x+1)\sqrt{x-1}}{x\cdot f(x-2)}$ có tất cả bao nhiêu tiệm cận đứng?
        \choice
        {1}
        {3}
        {4}
        {\True 2}}{\begin{tikzpicture}[scale=.6, font=\footnotesize, line join=round, line cap=round, >=stealth]
            \def\xmin{-3}\def\xmax{3}\def\ymin{-3}\def\ymax{3}
            \draw[->] (\xmin-0.2,0)--(\xmax+0.2,0) node[below] {\footnotesize $x$};
            \draw[->] (0,\ymin-0.2)--(0,\ymax+0.2) node[right] {\footnotesize $y$};
            \draw (0,0) node [below left] {\footnotesize $O$};
            \foreach \x in {-2}\draw (\x,-0.1)--(\x,0.1) node [above left] {\footnotesize $\x$};
            \foreach \x in {2}\draw (\x,-0.1)--(\x,0.1) node [above right] {\footnotesize $\x$};
            \foreach \y in {}\draw (-0.1,\y)--(0.1,\y) node [right] {\footnotesize $\y$};
            \clip (\xmin,\ymin) rectangle (\xmax,\ymax);
            \draw[smooth,samples=200,domain=\xmin:\xmax] plot (\x,{(2/3)*((\x)^3)+0*((\x)^2)+-(8/3)*(\x)});
    \end{tikzpicture}}
    \loigiai{
        * Điều kiện: $\heva{&x \ne 0\\&f(x-2) \ne 0\\&x \ge 1.}$\\
        Nhìn hình vẽ ta thấy
        $f(x-2)=0\Leftrightarrow \hoac{&x-2=-2\\&x-2=0\\&x-2=2}\Leftrightarrow \hoac{&x=0&(\text{không thỏa mãn})\\&x=2&(\text{nghiệm đơn})\\&x=4&(\text{nghiệm đơn}).}$\\
        Vậy $g(x) =\dfrac{(2x+1)\sqrt{x-1}}{x\cdot f(x-2)}=\dfrac{(x-1)\sqrt{x+2}}{x\cdot ax(x-2)(x-4)}.$ \\
        Đồ thị hàm số $g(x)$ có 2 đường tiệm cận đứng.}
\end{ex}
\begin{ex}
    \immini{ %Câu 93.
        Cho hàm số $y= f(x)$ có đồ thị cắt trục hoành tại đúng 3 điểm như hình bên. Đồ thị hàm số $y =\dfrac{(x+2)\sqrt{3-x}}{f(|x|)}$
        có tất cả bao nhiêu tiệm cận đứng?
        \choice
        {1}
        {3}
        {4}
        {\True 2}}{\begin{tikzpicture}[scale=.5, font=\footnotesize, line join=round, line cap=round, >=stealth]
            \def\xmin{-2}\def\xmax{5}\def\ymin{-3}\def\ymax{5}
            \draw[->] (\xmin-0.2,0)--(\xmax+0.2,0) node[below] {\footnotesize $x$};
            \draw[->] (0,\ymin-0.2)--(0,\ymax+0.2) node[right] {\footnotesize $y$};
            \draw (0,0) node [below left] {\footnotesize $O$};
            \foreach \x in {-1,2,4}\draw (\x,-0.1)--(\x,0.1) node [above left] {\footnotesize $\x$};
            \foreach \y in {}\draw (-0.1,\y)--(0.1,\y) node [right] {\footnotesize $\y$};
            \clip (\xmin,\ymin) rectangle (\xmax,\ymax);
            \draw[smooth,samples=200,domain=-1.2:0] plot(\x,{0-8.48*(\x)^(2.0)-5.48*(\x)+3.0});
            \draw[smooth,samples=200,domain=0:2]
            plot(\x,{0-2.7989489689153735*(\x)^(3.0)+8.326740175055514*(\x)^(2.0)-6.957684474449535*(\x)+3.0});
            \draw[smooth,samples=200,domain=2:5]
            plot(\x,{2.395330112721417*(\x)^(2.0)-14.371980676328501*(\x)+19.162640901771336});
    \end{tikzpicture}}
    \loigiai{
        * Điều kiện: $\heva{&f(|x|) \ne 0\\&x \le 3.}$\\
        Nhìn hình vẽ ta thấy
        $f(|x|)=0\Leftrightarrow \hoac{&|x|=-1\\&|x|=2\\&|x|=4}\Leftrightarrow \hoac{&x=\pm 2&(\text{nghiệm đơn})\\&x=- 4&(\text{nghiệm đơn})\\&x=4&(\text{không thỏa mãn}).}$\\
        Vậy $y =\dfrac{(x+2)\sqrt{3-x}}{a(x-2)(x+2)(x+4)(x-4)}$ \\
        Đồ thị hàm số có 2 đường tiệm cận đứng.}
\end{ex}
\begin{ex}
    \immini{ %Câu 94.
        Đường cong ở hình bên là đồ thị của hàm số $y = ax^3 +bx^2 +cx+d$. Đồ thị hàm số $y =\dfrac{(2x+1)\sqrt{1-x}}{f(|x|)}$ có tất cả bao nhiều tiệm cận đứng?
        \choice
        { 1}
        {3}
        {4}
        {\True 2}}{\begin{tikzpicture}[scale=.8, font=\footnotesize, line join=round, line cap=round, >=stealth]
            \def\xmin{-1}\def\xmax{2}\def\ymin{-1.5}\def\ymax{1.5}
            \draw[->] (\xmin-0.2,0)--(\xmax+0.2,0) node[below] {\footnotesize $x$};
            \draw[->] (0,\ymin-0.2)--(0,\ymax+0.2) node[right] {\footnotesize $y$};
            \draw (0.15,0) node [below left] {\footnotesize $O$};
            \foreach \x in {}\draw (\x,0.1)--(\x,-0.1) node [below] {\footnotesize $\x$};
            \foreach \y in {-1,1}\draw (0.1,\y)--(-0.1,\y) node [left] {\footnotesize $\y$};
            \clip (\xmin,\ymin) rectangle (\xmax,\ymax);
            \draw[smooth,samples=200,domain=\xmin:\xmax] plot (\x,{4*((\x)^3)+-6*((\x)^2)+0*(\x)+1});
            \draw[dashed] (0.5,0)--(0.5,0.0)--(0,0.0);
            \draw (0.5,-1pt)--(0.5,1pt) node [above] {\footnotesize $\frac{1}{2}$};
            \draw (-0.7,-1pt)--(-0.7,1pt) node [above] {\footnotesize $-\frac{1}{2}$};
            \draw (1,-1pt)--(1,1pt) node [above] {\footnotesize $1$};
            \draw[dashed] (0.0,0)--(0.0,1.0)--(0,1.0);
            \draw[dashed] (1.0,0)--(1.0,-1.0)--(0,-1.0);
    \end{tikzpicture}}
    \loigiai{
        * Điều kiện: $\heva{&f(|x|) \ne 0\\&x \le 1.}$\\
        Nhìn hình vẽ ta thấy
        $f(|x|)=0\Leftrightarrow \hoac{&|x|=-\dfrac{1}{2}\\&|x|=\dfrac{1}{2}\\&|x|=x_1>1}\Leftrightarrow \hoac{&x=\pm \dfrac{1}{2}&(\text{hai nghiệm đơn})\\&x=- x_1&(\text{nghiệm đơn})\\&x=x_1&(\text{không thỏa mãn}).}$\\
        Vậy $y =\dfrac{(2x+1)\sqrt{1-x}}{f(|x|)}=\dfrac{(2x+1)\sqrt{1-x}}{a\left(x-\dfrac{1}{2}\right)\left(x+\dfrac{1}{2}\right)(x+x_1)(x-x_1)}$ \\
        Đồ thị hàm số có 2 đường tiệm cận đứng.}
\end{ex}
\begin{ex}
    \immini{ %Câu 96.
        Cho đồ thị hàm số $y =f(x)$ và trục hoành có đúng 2 điểm chung như hình bên. Đồ thị hàm số $y =\dfrac{(x-1)\sqrt{3-x}}{f(x^2)}$ có tất cả bao nhiêu tiệm cận đứng?
        \choice
        {1}
        {3}
        {4}
        {\True 2}}{\begin{tikzpicture}[scale=.8, font=\footnotesize, line join=round, line cap=round, >=stealth]
            \def\xmin{-1.5}\def\xmax{2}\def\ymin{-1}\def\ymax{4.5}
            \draw[->] (\xmin-0.2,0)--(\xmax+0.2,0) node[below] {\footnotesize $x$};
            \draw[->] (0,\ymin-0.2)--(0,\ymax+0.2) node[right] {\footnotesize $y$};
            \draw (0,0) node [below left] {\footnotesize $O$};
            \foreach \x in {1}\draw (\x,0.1)--(\x,-0.1) node [below] {\footnotesize $\x$};
            \foreach \x in {-1}\draw (\x,0.1)--(\x,-0.1) node [below left] {\footnotesize $\x$};
            \clip (\xmin,\ymin) rectangle (\xmax,\ymax);
            \draw[smooth,samples=200,domain=-1.1:0] plot(\x,{21.044670464836045*(\x)^(3.0)+24.701786337609526*(\x)^(2.0)+5.65711587277348*(\x)+2.0});
            \draw[smooth,samples=200,domain=0:\xmax] plot(\x,{10.591704641658401*(\x)^(3.0)-19.26315454354621*(\x)^(2.0)+6.6714499018878115*(\x)+2.0});
    \end{tikzpicture}}
    \loigiai{
        * Điều kiện: $\heva{&f(x^2) \ne 0\\&x \le 3.}$\\
        Nhìn hình vẽ ta thấy
        $f(x^2)=0\Leftrightarrow \hoac{&x^2=-1\\&x^2=1}\Leftrightarrow x=\pm 1\,(\text{nghiệm kép}).$\\
        Vậy $y=\dfrac{(x-1)\sqrt{3-x}}{f(x^2)}=\dfrac{(x-1)\sqrt{3-x}}{(x-1)^2(x+1)^2}$ \\
        Đồ thị hàm số có 2 đường tiệm cận đứng.}
\end{ex}
\begin{ex}%[2D1G4-3]%Câu 52
    Cho hàm số $y=ax^3+bx^2+cx+d$ có đồ thị như hình vẽ. Đồ thị của hàm số $g(x)=\dfrac{x^2-x}{f^2(x)-2f(x)}$ có bao nhiêu đường tiệm cận đứng?
    \choice
    {$2$}
    {$3$}
    {\True $4$}
    {$5$}
    \begin{center}
        \begin{tikzpicture}[thick,>=stealth,x=1cm,y=1cm,scale=.7]
            \draw[thin,color=gray!50] (-3.3,-1.3) grid (3.9,5.9);
            \draw[->] (-3.2,0) -- (4.2,0) node[right] {$x$};
            \draw[->] (0,-1.2) -- (0,5.2) node[above] {$y$};
            \draw[color=blue, domain=-2.15:2.15,samples=300] plot (\x,{(\x)^3-3*(\x)+2}) node[right] {$y=f(x)$};
            \draw (-2,0) circle (1.5pt) node[below left]{$-2$};
            \draw (-1,0) circle (1.5pt) node[below]{$-1$};
            \draw (0,0) circle (1.5pt) node[above left]{$O$};
            \draw (1,0) circle (1.5pt) node[below]{$1$};
            \draw (0,4) circle (1.5pt) node[right]{$4$};
            \draw (-1,4) circle (1.5pt);
            \draw[dashed] (-1,0)--(-1,4)--(0,4);
            \draw[red] (-3,2)--(3.2,2);
            \draw[red] (3.5,2) node[right]{$f(x)=2$};
        \end{tikzpicture}
    \end{center}
    \loigiai{
        Xét phương trình $f^2(x)-2f(x)=0 \Leftrightarrow \hoac{&f(x)=0\\&f(x)=2}\Leftrightarrow \hoac{&x=1 \, (\textrm{nghiệm kép trùng nghiệm đơn ở tử số})\\&x=-2\, (\textrm{nghiệm đơn khác nghiệm của tử})\\&x=a\in(-2; -1)\\&x=0\, (\textrm{nghiệm đơn trùng nghiệm ở tử})\\&x=b\in(1; 2)}$\\
        \textbf{Kết luận:} Đồ thị hàm số có $4$ đường tiệm cận đứng.
    }
\end{ex}
\begin{ex}%[Thi thử L3, Lương Thế Vinh, Hà Nội, 2018]%[Phạm Toàn, Dự án (12EX-10)]%[2D1G4-3]%
    \immini{Cho hàm số $y=f(x)$ có đạo hàm liên tục trên $\mathbb{R}$. Đồ thị hàm $f(x)$ như hình vẽ. Số đường tiệm cận đứng của đồ thị hàm số $y=\dfrac{x^2-1}{f^2(x)-4f(x)}$ bằng
        \choice
        {$3$}
        {$1$}
        {$2$}
        {\True $4$}
    }{\begin{tikzpicture}[>=stealth,x=1cm,y=0.75cm,scale=0.7]
            \draw[->] (-2.5,0)--(0,0)%
            node[below right]{$O$}--(2.5,0) node[below]{$x$};
            \draw[->] (0,-2) --(0,5) node[right]{$y$};
            \foreach \x in {-1,1}{
                \draw (\x,0) node[below]{\footnotesize $\x$} circle (1pt);%Ox
            }
            \foreach \y in {2,4}{
                \draw (0,\y) node[right]{\footnotesize $\y$} circle (1pt);%Oy
            }
            \draw[samples=100,domain=-2.05:2] plot (\x,{(\x -1)^2*(\x+2)});
            \draw [dashed] (-1,0)--(-1,4)--(0,4);
            \draw(-1,4) circle (1pt);
    \end{tikzpicture}}
    \loigiai{Xét $f^2(x)-4f(x)=0\Leftrightarrow \hoac{& f(x)=0\\ &f(x)=4.}$\\
        Xét $f(x)=0$ có hai nghiệm, nghiệm $x_1\ne \pm 1$ và nghiệm $x_2=1$ là nghiệm bội (do đồ thị tiếp xúc với trục hoành tại $x=1$. Trường hợp này có $2$ tiệm cận đứng.\\
        Xét $f(x)=4$ có hai nghiệm, nghiệm $x_3\ne \pm 1$ và nghiệm $x_4=-1$ là nghiệm bội (do đồ thị tiếp xúc với đường thẳng $y=4$ tại $x=-1$. Trường hợp này có $2$ tiệm cận đứng.\\
        Vậy đồ thị có $4$ tiệm cận đứng.}
\end{ex}
\begin{ex}%[Thi thử, Trường THPT Lý Thái Tổ - Bắc Ninh, 2019]%[Duong Xuan Loi, 12EX3]%[2D1G4-3]%
    \immini{
        Cho hàm số $f(x)$ có đồ thị như hình bên. Số đường tiệm cận đứng của đồ thị hàm số
        $y=\dfrac{(x^2-4)(x^2+2x)}{[f(x)]^2+2f(x)-3}$ là
        \choice
        {\True $4$}
        {$5$}
        {$3$}
        {$2$}
    }{
        \begin{tikzpicture}[scale=0.5, font=\footnotesize, line join=round, line cap=round, >=stealth]
            \def\a{1} \def\b{-8} \def\c{1} % Hệ số
            \def\xt{-3.7} \def\xp{4} \def\yt{2} \def\yd{-3.7} % x_trái, x_phải, y_trên, y_dưới (giới hạn)
            \draw[->] (\xt,0)--(\xp,0) node [below]{$x$};
            \draw[->] (0,\yd)--(0,\yt) node [left]{$y$};
            \node at (0,0) [below left]{$O$};
            \clip (\xt-0.1,\yd+0.1) rectangle (\xp-0.1,\yt-0.1);
            \draw[smooth,samples=300] plot(\x,{1/4*(\a*(\x)^4+\b*(\x)^2)+\c});
            \draw[dashed] (-2,0)node[above]{$-2$}--(-2,-3)--(2,-3)--(2,0)node[above]{$2$};
            \node at (0,-3)[above left]{$-3$};
            \node at (-3,0)[above left]{$-3$};
            \node at (0,1)[above right]{$1$};
            \node at (3,0)[above right]{$3$};
            \fill (0,0) circle (1pt) (0,-3) circle (1pt) (2,0) circle (1pt) (-2,0) circle (1pt) (-3,0) circle (1pt) (0,1) circle (1pt) (3,0) circle (1pt);
        \end{tikzpicture}
    }
    \loigiai{
        Ta có $y=\dfrac{(x^2-4)(x^2+2x)}{[f(x)]^2+2f(x)-3}$ có các nghiệm ở tử là $x=0$ (bội $1$), $x=2$ (bội $1$), $x=-2$ (bội $2$).\\
        Mặt khác, từ đồ thị $f(x)$ ta thấy hàm số $y=\dfrac{(x^2-4)(x^2+2x)}{[f(x)]^2+2f(x)-3}$ có các nghiệm ở mẫu là
        $f^2(x)+2f(x)-3=0\Leftrightarrow \hoac{& f(x)=1 \\ & f(x)=-3}
        \Leftrightarrow \hoac{& x=0,x=x_1,x=x_2 \\ & x=-2,x=2.}$\\
        Trong đó nghiệm $x=0$, $x=-2$, $x=2$ đều có bội $2$ và $x_1$, $x_2$ khác các nghiệm của tử.\\
        So sánh bội nghiệm ở mẫu và bội nghiệm ở tử thì thấy đồ thị có các tiệm cận đứng là $x=0$, $x=2$; $x=x_1$; $x=x_2$.
    }
\end{ex}
\begin{ex}%[Thi thử, THPT Sơn Tây, Hà Nội, 2019]%[Huỳnh Xuân Tín, 12EX3]%[2D1G4-3]%
    \immini{Cho hàm số $ f(x)=(x+3)(x+1)^2(x-1)(x-3)$ có đồ thị như hình vẽ. Đồ thị hàm số $ g(x)=\dfrac{\sqrt{x-1}}{f^2(x)-9f(x)}$ có bao nhiêu tiệm cận đứng và tiệm cận ngang?
        \choice
        {$3$}
        {\True$ 4$}
        {$ 9$}
        { $8$}
    }{\begin{tikzpicture}[scale=0.3, font=\footnotesize, line join=round, line cap=round, >=stealth]
            %\draw[dashed, line width=0.1pt, gray] (-3.2,-5.5) grid (5.2,4.5);
            \draw[->] (-3.5,0)--(0,0) node[below right]{$O$}--(3.6,0) node[below]{$x$};
            \draw[fill=black] (0,0) circle (1pt);
            \draw[->] (0,-7.7) --(0,6.5) node[right]{$y$};
            \foreach \x in {-3,-1,3}{
                \draw[fill=black] (\x,0) node[below left]{$\x$} circle (1pt);}
            \draw[fill=black] (1,0) node[below right]{$1$} circle (1pt);
            \draw[fill=black] (0,1.35) node[above left]{$9$} circle (1pt);
            \draw [black, domain=-3.2:3.18, samples=100] %
            plot(\x,{0.15*(\x+3)*(\x+1)^2*(\x-1)*(\x-3)});
    \end{tikzpicture}}
    \loigiai{Điều kiện xác định của hàm số $g(x)$ là $\heva{&x\ge1\\ &f^2(x)-9f(x)\not=0.}$\\
        Từ $f^2(x)-9f(x)=0\Leftrightarrow \hoac{&f(x)=0\\&f(x)=9.}$\\
        Với $f(x)=0$ có nghiệm là $x=\pm 1, x=\pm 3$.\\
        Dựa vào đồ thị ta thấy nghiệm của phương trình $f(x)=9$ là hoành độ giao điểm của đường thẳng $y=9$ với đồ thị hàm số $y=f(x)$ nên có nghiệm là $-3<x_3<x_2<-1<0<x_1<1<3<x_0$.\\
        Do đó tập xác định của hàm số $y=g(x)$ là $\mathscr{D}=\left[1;+\infty \right)\setminus\left\lbrace1;3;x_0 \right\rbrace $.\\
        Khi đó ta có \begin{itemize}
            \item $\lim\limits_{x\rightarrow1^+ } g(x)=\lim\limits_{x\rightarrow1^+ }\dfrac{\sqrt{x-1}}{f(x)\left(f(x)-9 \right)}=+\infty$ (vì $x$ tiến gần bên phải $1$ thì $f(x)<0, f(x)-9<0$), suy ra đường thẳng $x=1$ là tiệm cận đứng.
            \item $\lim\limits_{x\rightarrow3^+ } g(x)=\lim\limits_{x\rightarrow3^+ }\dfrac{\sqrt{x-1}}{f(x)\left(f(x)-9 \right)}=-\infty$ (vì $x$ tiến gần bên phải $3$ thì $f(x)>0, f(x)-9<0$), suy ra đường thẳng $x=3$ là tiệm cận đứng.
            \item $\lim\limits_{x\rightarrow x_0^+} g(x)=\lim\limits_{x\rightarrow x_0^+ }\dfrac{\sqrt{x-1}}{f(x)\left(f(x)-9 \right)}=+\infty$ (vì $x$ tiến gần bên phải $x_0$ thì $f(x)>0, f(x)-9>0$), suy ra đường thẳng $x=x_0$ là tiệm cận đứng.
        \end{itemize}
        Và $\lim\limits_{x\rightarrow +\infty} g(x)=\lim\limits_{x\rightarrow +\infty }\dfrac{\sqrt{x-1}}{f(x)\left(f(x)-9 \right)}=0$ (vì bậc ở mẫu của $y=g(x)$ là $10$ và bậc tử của nó là $\dfrac{1}{2}$). Do vậy đồ thị hàm số $y=g(x)$ có một tiệm cận ngang là đường thẳng $y=0$.\\
        Vậy đồ thị hàm số $y=g(x)$ có bốn tiệm cận ngang và đứng. }
\end{ex}
\begin{ex}%[Thi thử, Chuyên Quang Trung-Bình Phước, 2021,lần 1]%[Trần Hòa, 12EX6]%[2D1G4-3]%
    \immini{Cho hàm số $y=f(x)=ax^3+bx^2+cx+d$, có đồ thị như hình vẽ. Số đường tiệm cận đứng của đồ thị hàm số $y=\dfrac{x^2+x-2}{f^2(x)-f(x)}$ là
        \choice
        {$3$}
        {$2$}
        {\True $4$}
        {$5$}}
    {\begin{tikzpicture}[scale=.5, font=\footnotesize, line join=round, line cap=round, >=stealth]
            \draw[->] (-2.5,0)--(0,0) node[below right]{$O$}--(2,0) node[below]{$x$};
            \draw[->] (0,-.5) --(0,4.5) node[right]{$y$};
            \draw [domain=-2.05:2.05, samples=100] %
            plot (\x, {(\x+2)*(\x-1)^2});
            \draw[fill] (0,0) circle (1pt);
            \foreach \x/\g in {-2/140,-1/-90,1/-90}
            \draw[fill] (\x,0) circle(.5pt)node [shift={(\g:.3)}] {$\x$};
            \foreach \y/\g in {2/0,4/0}
            \draw[fill] (0,\y) circle(.5pt)node [shift={(\g:.3)}] {$\y$};
            \draw[dashed] (-1,0)--(-1,4)--(0,4);
    \end{tikzpicture}}
    \loigiai{
        \begin{itemize}
            \item $x^2+x-2=(x-1)(x+2)$.\\
            \item Dựa vào đồ thị hàm số $y=f(x)$ ta có $f^2(x)-f(x)=0\Leftrightarrow\hoac{&f(x)=0\\&f(x)=1.}$\\
            $f(x)=0\Leftrightarrow x=-2$, $x=1$ (nghiệm kép).\\
            $f(x)=1\Leftrightarrow\hoac{&x=x_1,(x_1\in (-2;-1))\\&x=x_2,(x_2\in (0;1))\\&x=x_3,(x_3>1). }$
            \item Do đó $y=\dfrac{(x-1)(x+2)}{a^2(x+2)(x-1)^2(x-x_1)(x-x_2)(x-x_3)}$.
        \end{itemize}
        Suy ra đồ thị có các đườn tiệm cận đứng $x=1$, $x=x_1$, $x=x_2$, $x=x_3$.
    }
\end{ex}
\begin{ex}%[Đề thi hết học kì 2, Bình Minh, Ninh Bình 2018]%[Nguyễn Tuấn Anh, dự án EX9]%[2D1G4-3]%
    \immini{Cho hàm số bậc ba $f(x)=ax^3+bx^2+cx+d$ có đồ thị như hình vẽ bên dưới. Hỏi đồ thị hàm số $g(x)=\dfrac{(x^2-3x+2)\sqrt{x-1}}{x[f^2(x)-f(x)]}$ có bao nhiêu tiệm cận đứng?
        \choice
        {$5$}
        {$6$}
        {\True $3$}
        {$4$}
    }{
        \begin{tikzpicture}[line width=1.0pt,line join=round,>=stealth,x=1cm,y=1cm,scale=1.0]
            \draw[->,line width = 1pt] (-1,0)--(0,0) node[below right]{$O$}--(4,0) node[below]{$x$};
            \draw[->,line width = 1pt] (0,-1.5) --(0,2.5) node[right]{$y$};
            \foreach \x in {1,2}{
                \draw (\x,0) node[below]{$\x$} circle (1pt);
            }
            \foreach \y in {1}{
                \draw (0,\y) node[left]{$\y$} circle (1pt);
            }
            \clip(-0.8,-1) rectangle (3.8,2.3);
            \draw [line width=1.0pt, thick, domain=-0.5:3.5, samples=100]%,domain=-1.5:3] %
            plot (\x, {(5*(\x)-4)*((\x)-2)^2});
            \draw [dash pattern=on 4pt off 4pt] (1.,0.)-- (1.,1.)-- (0.,1.);
            \draw (1,1) circle (1pt);
        \end{tikzpicture}
    }
    \loigiai{
        Điều kiện $\heva{&x\geq 1\\ &x\ne 0\\ &f^2(x)-f(x)\ne 0}\Leftrightarrow \heva{&x\geq 1\\ &f(x)\ne 0\\ & f(x)\ne 1.}$\\
        Dựa vào đồ thị hàm số $y=f(x)$, ta thấy $f(x)=0$ có hai nghiệm, một nghiệm $x_1<1$ và một nghiệm kép bằng $2$. Do đó ta biểu diễn được $f(x)$ dưới dạng
        $$ f(x)=a(x-x_1)(x-2)^2. $$
        Dựa vào đồ thị hàm số $y=f(x)$, ta thấy phương trình $f(x)=1$ có ba nghiệm $1,x_2, x_3$, với $1<x_2<2<x_3$. Do đó ta biểu diễn được $f(x)-1$ dưới dạng
        $$ f(x)-1=a(x-1)(x-x_2)(x-x_3). $$
        Lúc này điều kiện được viết lại như sau $\heva{&x>1\\ &x\ne x_2, x\ne 2, x\ne x_3.}$\\
        Với điều kiện đó thì $g(x)$ được viết lại là
        $$ g(x)=\dfrac{\sqrt{x-1}}{a^2x(x-x_1)(x-x_2)(x-2)(x-x_3)}. $$
        Ta có
        \begin{align*}
            &\lim\limits_{x\to 1^+}g(x)=\lim\limits_{x\to 1^+}\dfrac{\sqrt{x-1}}{a^2x(x-x_1)(x-x_2)(x-2)(x-x_3)}=0,\\
            & (x=1\mbox{ \textbf{không} là tiệm cận đứng}) \\
            &\lim\limits_{x\to x_2^+}g(x)=\lim\limits_{x\to x_2^+}\dfrac{\sqrt{x-1}}{a^2x(x-x_1)(x-x_2)(x-2)(x-x_3)}=+\infty,\\
            & (x=x_2\mbox{ là tiệm cận đứng}) \\
            &\lim\limits_{x\to 2^+}g(x)=\lim\limits_{x\to 2^+}\dfrac{\sqrt{x-1}}{a^2x(x-x_1)(x-x_2)(x-2)(x-x_3)}=-\infty,\\
            & (x=2\mbox{ là tiệm cận đứng}) \\
            &\lim\limits_{x\to x_3^+}g(x)=\lim\limits_{x\to x_3^+}\dfrac{\sqrt{x-1}}{a^2x(x-x_1)(x-x_2)(x-2)(x-x_3)}=+\infty,\\
            & (x=x_3\mbox{ là tiệm cận đứng}) \\
        \end{align*}
        Vậy đồ thị hàm số $g(x)$ có tất cả $3$ tiệm cận đứng.
    }
\end{ex}
\begin{ex}%[VDC5-Đỗ Đường Hiếu]%[2D1G4-3]%
    \immini{Cho hàm số $f(x)=(x+3)(x+1)^2(x-1)(x-3)$ có đồ thị như hình vẽ. Đồ thị hàm số $g(x)=\dfrac{\sqrt{x-1}}{f^2(x)-9f(x)}$ có bao nhiêu tiệm cận đứng và tiệm cận ngang?
        \choice
        {$3$}
        {\True $4$}
        {$9$}
        {$8$}}
    {\begin{tikzpicture}[xscale=0.8,yscale=0.05, line join=round, line cap=round,font=\footnotesize,>=stealth]
            \draw[->] (-4,0)--(4,0) node[below]{$x$};
            \draw[->] (0,-56)--(0,30) node[left]{$y$};
            \coordinate[label=below left:$O$] (O) at (0,0);
            \draw (-1,0) node[below] { $-1$}(1,0) node[below] { $1$};
            \draw (-3,0) node[below left] { $-3$};
            \draw (3,0) node[below right] { $3$};
            \clip (-3.3,-60) rectangle (3.5,26);
            \draw[smooth,samples=300,domain=-3.5:3.5] plot(\x,{(\x+3)*(\x+1)^2*(\x-1)*(\x-3)});
            \foreach \x in {-3,-1,1,3}
            \draw[shift={(\x,0)},color=black] (0pt,20pt) -- (0pt,-20pt);
            \draw[shift={(0,9)},color=black] (2pt,0pt) -- (-2pt,0pt) node[left] {$9$};
        \end{tikzpicture}
    }
    \loigiai{%GV tổng quát hóa bài toán:
        Cho hàm số đa thức $y=f(x)$ có đồ thị $(C)$. Tìm số đường tiệm cận của đồ thị hàm số $g(x)=\dfrac{\sqrt{ax+b}}{P\left(f(x) \right) }$, trong đó $P\left(f(x) \right)$ là một đa thức của $f(x)$.
        Nếu $a>0$ thì $\lim\limits_{x\to +\infty}g(x)=0$.\\
        Nếu $a<0$ thì $\lim\limits_{x\to -\infty}g(x)=0$.\\
        Do đó đồ thị hàm số $y=g(x)$ luôn có duy nhất một đường tiệm cận ngang là $y=0$.\\
        Gọi $x=x_0$ là một nghiệm của phương trình $P\left(f(x) \right) =0$ thỏa mãn điều kiện $ax+b\ge 0$. Rõ ràng khi đó $\lim\limits_{x\to x_0^+}g(x)=+\infty$ hoặc $\lim\limits_{x\to x_0^+}g(x)=-\infty$.\\
        Bởi vậy, số đường tiệm cận đứng của đồ thị hàm số $y=g(x)$ chính là số nghiệm của phương trình $P\left(f(x) \right) =0$ thỏa mãn điều kiện $ax+b\ge 0$.
        \immini{Ta có $f^2(x)-9f(x)=0\Leftrightarrow \hoac{&f(x)=0\\&f(x)=9.}$\\
            \begin{itemize}
                \item $f(x)=0$ có các nghiệm thuộc $\left[1;+\infty\right)$ là $x=1$ và $x=3$.
                \item Đường thẳng $y=9$ cắt đồ thị hàm số $y=f(x)$ tại duy nhất một điểm có hoành độ thuộc $\left[1;+\infty\right)$ là $x=a>3$.
            \end{itemize}
        }
        {\begin{tikzpicture}[xscale=0.8,yscale=0.05, line join=round, line cap=round,font=\footnotesize,>=stealth]
                \draw[->] (-4,0)--(4,0) node[below]{$x$};
                \draw[->] (0,-56)--(0,30) node[left]{$y$};
                \coordinate[label=below left:$O$] (O) at (0,0);
                \draw (-4,9)--(4,9);
                \draw (-1,0) node[below] { $-1$}(1,0) node[below] { $1$};
                \draw (-3,0) node[below left] { $-3$};
                \draw (3,0) node[below right] { $3$};
                \clip (-3.3,-60) rectangle (3.5,26);
                \draw[smooth,samples=300,domain=-3.5:3.5] plot(\x,{(\x+3)*(\x+1)^2*(\x-1)*(\x-3)});
                \foreach \x in {-3,-1,1,3}
                \draw[shift={(\x,0)},color=black] (0pt,20pt) -- (0pt,-20pt);
                \draw[shift={(0,9)},color=black] (2pt,0pt) -- (-2pt,0pt) node[above left] {$9$};
        \end{tikzpicture}}
        \noindent
        Bởi vậy, hàm số $g(x)=\dfrac{\sqrt{x-1}}{f^2(x)-9f(x)}$ có tập xác định là $\mathscr D=\left[1;3\right) \cup \left(3;a\right) \cup\left( a;+\infty\right)$.\\
        Khi đó ta có
        \begin{itemize}
            \item $\lim\limits_{x\to+\infty}g(x)=0$ nên đồ thị hàm số $y=g(x)$ có một đường tiệm cận ngang là đường thẳng $y=0$.
            \item $\lim\limits_{x\to 1^+}g(x)=\lim\limits_{x\to 1^+}\dfrac{\sqrt{x-1}}{f(x)\left[f(x)-9\right] }=+\infty$;\\
            $\lim\limits_{x\to 3^+}g(x)=\lim\limits_{x\to 3^+}\dfrac{\sqrt{x-1}}{f(x)\left[f(x)-9\right] }=-\infty$;\\
            $\lim\limits_{x\to a^+}g(x)=\lim\limits_{x\to a^+}\dfrac{\sqrt{x-1}}{f(x)\left[f(x)-9\right] }=+\infty$.\\
            Do đó nên đồ thị hàm số $y=g(x)$ có $3$ đường tiệm cận đứng là các đường thẳng $x=1$, $x=3$ và $x=a$.
        \end{itemize}
        Như vậy, đồ thị hàm số $y=g(x)$ có $4$ đường tiệm cận, trong đó có $1$ đường tiệm cận ngang và $3$ đường tiệm cận đứng.
    }
\end{ex}
\begin{ex}%[VDC5-Đỗ Đường Hiếu]%[2D1G4-3]%
    \immini{Cho hàm số bậc ba $y=f(x)$ có đồ thị như hình vẽ bên. Đồ thị hàm số $g(x)=\dfrac{x\sqrt{x+1}}{f(x)\left[f^2(x)-16 \right] }$ có bao nhiêu tiệm cận đứng?
        \choice
        {\True $4$}
        {$5$}
        {$6$}
        {$7$}}
    {\begin{tikzpicture}[scale=0.6,line join=round, line cap=round,font=\footnotesize,>=stealth]
            \draw[->] (-2.5,0)--(4,0) node[below]{$x$};
            \draw[->] (0,-5)--(0,2.5) node[left]{$y$};
            \coordinate[label=below left:$O$] (O) at (0,0);
            \draw[dashed] (-1,0)--(-1,-4)--(0,-4);
            \clip (-2.3,-5) rectangle (3.5,2.5);
            \draw[smooth,samples=300,domain=-3.5:3.5] plot(\x,{-0.5*(\x+2)*(\x-1)*(\x-3)});
            \foreach \x in {-2,-1,1,3}
            \draw[shift={(\x,0)},color=black] (0pt,2pt) -- (0pt,-2pt) node[above] { $\x$};
            \foreach \y in {-4,-3,1}
            \draw[shift={(0,\y)},color=black] (2pt,0pt) -- (-2pt,0pt) node[right] {$\y$};
        \end{tikzpicture}
    }
    \loigiai{
        Xét phương trình $f(x)\left[f^2(x)-16 \right]=0$ \, $(*)$, với điều kiện $x\in\left[-1;+\infty \right) $.\\
        Ta có $f(x)\left[f^2(x)-16 \right]=0\Leftrightarrow \hoac{&f(x)=0\\&f(x)=4\\&f(x)=-4.}$\\
        \begin{itemize}
            \item Phương trình $f(x)=0$ có hai nghiệm $x\in\left[-1;+\infty \right) $ là $x=1$ và $x=3$.
            \item Phương trình $f(x)=4$ có không có nghiệm $x\in\left[-1;+\infty \right) $.
            \item Phương trình $f(x)=-4$ có hai nghiệm $x\in\left[-1;+\infty \right) $ là $-1<x_1<0$ và $x_2>3$.
        \end{itemize}
        Rõ ràng $\lim\limits_{x\to x_0^+}g(x)=+\infty$ hoặc $\lim\limits_{x\to x_0^+}g(x)=-\infty$, trong đó $x=x_0$ là nghiệm thuộc $\left[-1;+\infty \right) $ của phương trình $(*)$. Do đó đường thẳng $x=x_0$ là tiệm cận đứng của đồ thị hàm số $y=g(x)$.\\
        Từ đó suy ra đồ thị hàm số $g(x)=\dfrac{x\sqrt{x+1}}{f(x)\left[f^2(x)-16 \right] }$ có $4$ tiệm cận đứng.
    }
\end{ex}
\begin{ex}%[VDC5-Đỗ Đường Hiếu]%[2D1G4-3]%
    \immini{Cho $y=f(x)$ là hàm số đa thức có đồ thị như hình vẽ bên. Đặt $g(x)=\dfrac{\sqrt{x-1}}{\left[f(x)\right]^2-2f(x)}$ có bao nhiêu đường tiệm cận đứng?
        \choice
        {$5$}
        {$3$}
        {$4$}
        {\True $2$}}
    {\begin{tikzpicture}[scale=0.6,line join=round, line cap=round,font=\footnotesize,>=stealth]
            \draw[->] (-3,0)--(2.5,0) node[below]{$x$};
            \draw[->] (0,-1)--(0,5) node[left]{$y$};
            \coordinate[label=above left:$O$] (O) at (0,0);
            \draw[dashed] (-1,0)--(-1,4)--(0,4);
            \clip (-2.3,-1) rectangle (2.5,4.5);
            \draw[smooth,samples=300,domain=-3.5:3.5] plot(\x,{(\x)^3-3*(\x)+2});
            \foreach \x in {-2,-1,1}
            \draw[shift={(\x,0)},color=black] (0pt,2pt) -- (0pt,-2pt) node[below] { $\x$};
            \foreach \y in {2,4}
            \draw[shift={(0,\y)},color=black] (2pt,0pt) -- (-2pt,0pt) node[right] {$\y$};
        \end{tikzpicture}
    }
    \loigiai{
        Xét phương trình $\left[f(x)\right]^2-2f(x)=0$ \, $(*)$, với điều kiện $x\in\left[1;+\infty \right) $.\\
        Ta có $\left[f(x)\right]^2-2f(x)=0\Leftrightarrow \hoac{&f(x)=0\\&f(x)=2.}$\\
        \begin{itemize}
            \item Phương trình $f(x)=0$ có một nghiệm $x\in\left[1;+\infty \right) $ là $x=1$.
            \item Phương trình $f(x)=2$ có một nghiệm $x\in\left[1;+\infty \right) $ là $x=x_1>1$.
        \end{itemize}
        Rõ ràng $\lim\limits_{x\to x_0^+}g(x)=+\infty$ hoặc $\lim\limits_{x\to x_0^+}g(x)=-\infty$, trong đó $x=x_0$ là nghiệm thuộc $\left[1;+\infty \right) $ của phương trình $(*)$. Do đó đường thẳng $x=x_0$ là tiệm cận đứng của đồ thị hàm số $y=g(x)$.\\
        Từ đó suy ra đồ thị hàm số $g(x)=\dfrac{\sqrt{x-1}}{\left[f(x)\right]^2-2f(x)}$ có $2$ tiệm cận đứng.
    }
\end{ex}
\begin{ex}%[VDC5-NgocDungHo]%[2D1G4-3]%
    \immini
    {
        Cho hàm số $f(x)$ có đồ thị như hình bên. Số đường tiệm cận đứng của đồ thị hàm số $y=\dfrac{(x^2-4)(x^2+2x)}{[f(x)]^2-4f(x)+3}$ là
        \choice
        {$4$}
        {\True $5$}
        {$3$}
        {$2$}
    }
    {\begin{tikzpicture}[>=stealth,scale=0.5, line join=round, line cap=round]
            \def\f[#1]{-0.25*((#1)^4-8*(#1)^2+4)}
            \draw[->] (-4.1,0)--(4,0) node [below]{$x$};
            \draw[->] (0,-2)--(0,4) node [left]{$y$};
            \node at (0,0) [above left]{$O$};
            % \clip;
            \draw[domain=-2.9:2.9,samples=300,thick] plot (\x,{\f[\x]});
            \foreach \x in {-2,2} \filldraw (\x,0) node[below]{\x} circle (2pt);
            %\foreach \x in {-3,3} \filldraw (\x,0) node[below left]{\x} circle (2pt);
            \filldraw (-3,0) node[below left]{$-3$} circle (2pt);
            \filldraw (3,0) node[below right]{$3$} circle (2pt);
            \filldraw (0,1) node[left]{$1$} circle (2pt);
            \filldraw (0,3) node[above left]{$3$} circle (2pt);
            \draw[dashed](-2,0)--(-2,3)--(2,3)--(2,0);
            \draw (3,-1.75) node[right]{$y=f(x)$};
        \end{tikzpicture}
    }
    \loigiai{
        Xét hàm số $y=g(x)=\dfrac{(x^2-4 )(x^2+2x)}{[f(x)]^2-4f(x)+3}$.
        \immini
        {
            Giải phương trình $(x^2-4)(x^2+2x)=0 $\\
            $\Leftrightarrow \hoac{& x^2-4=0 \\ & x^2+2x=0}\Leftrightarrow \hoac{& x=\pm 2 \\ & x=0.}$\\
            Giải phương trình $[f(x)]^2-4f(x)+3=0$\\
            $ \Leftrightarrow \hoac{& f(x)=1 \\ & f(x)=3} \Leftrightarrow \hoac{& x = \pm 2 \\ & x=a\\&x=b\\&x=c\\&x=d.}$\\ với $-3<a<-2<b<c<2<d<3$.\\
        }
        {\begin{tikzpicture}[>=stealth,scale=0.8, line join=round, line cap=round]
                \def\f[#1]{-0.25*((#1)^4-8*(#1)^2+4)}
                \def\g[#1]{1}
                \def\h[#1]{3}
                \draw[->] (-4.1,0)--(4,0) node [below]{$x$};
                \draw[->] (0,-2)--(0,4) node [left]{$y$};
                \node at (0,0) [above left]{$O$};
                % \clip;
                \draw[domain=-2.9:2.9,samples=300,thick] plot (\x,{\f[\x]});
                \draw[domain=-4:4,samples=300,thick] plot (\x,{\g[\x]});
                \draw[domain=-4:4,samples=300,thick] plot (\x,{\h[\x]});
                \foreach \x in {-3,-2,2,3} \filldraw (\x,0) node[below]{\x} circle (2pt);
                % \filldraw (-3,0) node[above left]{$-3$} circle (2pt);
                % \filldraw (3,0) node[above ]{$3$} circle (2pt);
                \filldraw (0,1) node[below left]{$1$} circle (2pt);
                \filldraw (0,-1) node[below left]{$-1$} circle (2pt);
                \filldraw (0,3) node[above left]{$3$} circle (2pt);
                \draw[dashed](-2,0)--(-2,3) (2,3)--(2,0) (2.61,0)node[below]{$d$}--(2.61,1) (-2.61,0)node[below]{$a$}--(-2.61,1) (1.08,0)node[below]{$c$}--(1.08,1)(-1.08,0)node[below]{$b$}--(-1.08,1);
                \draw (3,2.75) node[right]{$y=f(x)$};
            \end{tikzpicture}
        }
        Trong điều kiện xác định của hàm số $y=g(x)$ ta có thể viết $$y=g(x)=\dfrac{x(x-2)(x+2)^2}{(x-a)(x-b)(x-c)(x-d) (x-2)^2(x+2)^2}=\dfrac{x}{(x-a)(x-b)(x-c)(x-d)(x-2)}$$
        Vậy số tiệm cận đứng của đồ thị hàm số $y=g(x)$ bằng $5$.
    }
\end{ex}
\Closesolutionfile{ans}
%\subsection{ĐỀ ÔN LUYỆN}
%\boxde
\BTTN
\begin{ex}%[2D1N3-1]Câu 2
 Đường thẳng $x = a$ là một đường tiệm cận đứng của
 đồ thị hàm số $ y = f (x)$ nếu điều kiện sau thoả mãn
 \choice
 {$\displaystyle\lim_{x\to +\infty }f(x)=a$}
 {\True $\displaystyle\lim_{x\to a^-}f(x)=+\infty $}
 {$\displaystyle\lim_{x\to -\infty }f(x)=a$}
 {$\displaystyle\lim_{x\to a^-}f(x)=a $}
 \loigiai{ Đường thẳng $x = a$ được gọi là một đường tiệm cận đứng (hay tiệm cận đứng) của đồ thị hàm số $ y = f (x)$ nếu ít nhất một trong các điều kiện sau thoả mãn: \\$\displaystyle\lim_{x\to a^+}f(x)=+\infty $, $\displaystyle\lim_{x\to a^+}f(x)=-\infty $, $\displaystyle\lim_{x\to a^-}f(x)=-\infty $, $\displaystyle\lim_{x\to a^-}f(x)=+\infty $.}
\end{ex}
\begin{ex}%[2D1N3-1]Câu 4
 Đường thẳng $y = ax + b$ ($a \neq 0$) được gọi là đường tiệm cận xiên của đồ thị hàm số $y = f(x)$ nếu
 \choice
 {\True $\displaystyle\lim_{x\to -\infty }\big(f(x)-ax-b\big)=0$ hoặc $\displaystyle\lim_{x\to +\infty }\big(f(x)-ax-b\big)=0$}
 {$\displaystyle\lim_{x\to -\infty }\big(f(x)-ax+b\big)=0$ hoặc $\displaystyle\lim_{x\to +\infty }\big(f(x)-ax+b\big)=0$}
 {$\displaystyle\lim_{x\to 0 }\big(f(x)-ax+b\big)=+\infty$ hoặc $\displaystyle\lim_{x\to 0 }\big(f(x)-ax+b\big)=+\infty$}
 {$\displaystyle\lim_{x\to 0 }\big(f(x)-ax-b\big)=-\infty$ hoặc $\displaystyle\lim_{x\to 0 }\big(f(x)-ax-b\big)=-\infty$}
 \loigiai{Đường thẳng $y = ax + b, a \neq 0$, được gọi là đường tiệm cận xiên (hay tiệm cận xiên) của đồ thị hàm số $y = f(x)$ nếu\\ $\displaystyle\lim_{x\to -\infty }[f(x)-(ax+b)]=\displaystyle\lim_{x\to -\infty }(f(x)-ax-b)=0$ hoặc\\ $\displaystyle\lim_{x\to +\infty }[f(x)-(ax+b)]=\displaystyle\lim_{x\to +\infty }(f(x)-ax-b)=0$.
 }
\end{ex}
\begin{ex}
 Tiệm cận ngang của đồ thị hàm số $ y=\dfrac{2x-1}{x+1} $ là đường thẳng
 \choice
 {$y=-1$}
 {$ x=-1 $}
 {\True $ y=2 $}
 {$ x=2 $}
 \loigiai
 {
 Ta có $ \lim\limits_{x\to \pm\infty}y=2$ suy ra đường thẳng $ y=2 $ là tiệm cận ngang của đồ thị hàm số $ y=\dfrac{2x-1}{x+1} $.
 }
\end{ex}
\begin{ex}
 Tiệm cận ngang của đồ thị hàm số $y=\dfrac{1}{2x-3}$ là đường thẳng
 \choice
 {$y=\dfrac{3}{2}$}
 {$x=\dfrac{3}{2}$}
 {\True $y=0$}
 {$y=\dfrac{1}{2}$}
 \loigiai{
 Vì $\lim\limits_{x\to -\infty} \dfrac{1}{2x-3}=\lim\limits_{x\to +\infty} \dfrac{1}{2x-3}=0$ nên đồ thị hàm số có tiệm cận ngang $y=0$.
 }
\end{ex}
\begin{ex}
 Đồ thị hàm số $f(x)=\dfrac{2x-3}{x+1}$ có đường tiệm cận đứng là
 \choice
 {$y=2$}
 {\True $x=-1$}
 {$y=-1$}
 {$x=2$}
 \loigiai{
 Ta có $\displaystyle \lim_{x \to (-1)^-}f(x)=\displaystyle \lim_{x \to (-1)^-}\dfrac{2x-3}{x+1}=+\infty $; $\displaystyle \lim_{x \to (-1)^+}f(x)=\displaystyle \lim_{x \to (-1)^+}\dfrac{2x-3}{x+1}=-\infty$ nên đường thẳng $x=-1$ là đường tiệm cận đứng của đồ thị hàm số.}
\end{ex}
\begin{ex}
 Hàm số nào sau đây có đồ thị nhận đường thẳng $x=2$ là đường tiệm cận đứng?
 \choice
 {$y=\dfrac{2}{x+2}$}
 {\True $y=\dfrac{5x}{2-x}$}
 {$y=\dfrac{1}{x+1}$}
 {$y=x-2+\dfrac{1}{x+1}$}
 \loigiai{
 Ta có $\lim\limits_{x\to 2^+} \dfrac{5x}{2-x}=-\infty $ và $\lim\limits_{x\to 2^-} \dfrac{5x}{2-x}=+\infty $ nên đồ thị hàm số $y=\dfrac{5x}{2-x}$ nhận $x=2$ làm tiệm cận đứng.}
\end{ex}
\begin{ex}%[2D1V3-1]Câu 12
 Đồ thị của hàm số nào sau đây có giao điểm của hai đường tiệm cận thuộc đường thẳng $y=x$?
 \choice
 {$y=\dfrac{2x-1}{x+3}$}
 {\True$y=\dfrac{x+4}{x-1}$}
 {$y=\dfrac{2x+1}{x+2}$}
 {$\dfrac{1}{x+3}$}
 \loigiai{
 Đáp án $y=\dfrac{2x-1}{x+3}$ có giao hai đường tiệm tiệm cận là $(-3;2)\notin d$\\
 Đáp án $y=\dfrac{x+4}{x-1}$ có giao hai đường tiệm cận là $(1;1)\in d$\\
 Đáp án $y=\dfrac{2x+1}{x+2}$ có giao hai đường tiệm cận là $(-2;2)\notin d$\\
 Đáp án $\dfrac{1}{x+3}$ có giao hai đường tiệm cận là $(-3;0)\notin d$\\
 }
\end{ex}
\begin{ex}%[2D1N3-1]Câu 6
 Đồ thị hàm số $y=\dfrac{x-2}{x^{2}-4}$ có mấy đường tiệm cận?
 \choice
 {$3$}
 {$1$}
 {\True$2$}
 {$0$}
 \loigiai{ Hàm số $y=\dfrac{x-2}{x^{2}-4}=\dfrac{x-2}{(x-2)(x+2)}=\dfrac{1}{x+2}$.\\
 $\heva{&\displaystyle\lim_{x\to +\infty }\dfrac{1}{x+2}=0\\&
 \displaystyle\lim_{x\to -\infty }\dfrac{1}{x+2}=0.}$\\
 Nên $y=0$ là đường tiệm cận ngang của hàm số, hàm số có tiệm cận ngang thì không có tiệm cận xiên.\\
 $\heva{&\displaystyle\lim_{x\to -2^- }\dfrac{1}{x+2}= - \infty \\&
 \displaystyle\lim_{x\to -2^+ }\dfrac{1}{x+2}= + \infty.}$\\
 Nên $x=-2$ là đường tiệm cận đứng của hàm số.\\
 Vậy hàm số có hai đường tiệm cận.
 }
\end{ex}
\begin{ex}
 Tiệm cận xiên của đồ thị hàm số $y=\dfrac{x^2+x-1}{x}$ có phương trình là
 \choice
 {$y=x-1$}
 {$y=x-2$}
 {$y=x-3$}
 {\True$y=x+1$}
 \loigiai{
 Ta có $y=\dfrac{x^2+x-1}{x}=x+1-\dfrac{1}{x}$.\\
 Xét $$\displaystyle\lim_{x\to \pm \infty }\big(y-(x+1)\big)=\displaystyle\lim_{x\to \pm \infty }\dfrac{-1}{x}=0$$
 Vậy đường tiệm cận xiên cần tìm của hàm số $f(x)$ có phương trình $y=x+1$.}
\end{ex}
\begin{ex}
 Tiệm cận xiên của đồ thị hàm số $y=\dfrac{2x^2-3x+4}{x-1}$ có phương trình là
 \choice
 {$y=x-1$}
 {\True $y=2x-1$}
 {$y=2x+1$}
 {$y=x+1$}
 \loigiai{
 Ta có $y=\dfrac{2x^2-3x+4}{x-1}=2x-1+\dfrac{3}{x-1}$. Suy ra $y=2x-1$ là đường tiệm cận xiên của đồ thị hàm số.
 }
\end{ex}
\begin{ex}
 Cho hàm số $y=f(x)$ xác định $ \mathbb{R} \setminus \left\lbrace 0\right\rbrace $, liên tục trên mỗi khoảng xác định và có bảng biến thiên như sau.\\
 \begin{center}
 \begin{tikzpicture}[>=stealth,font=\footnotesize,scale=1]
 \tikzset{double style/.append style = {draw=\tkzTabDefaultWritingColor,double=\tkzTabDefaultBackgroundColor,double distance=2pt}}
 \tkzTabInit[nocadre=false,lgt=1.2,espcl=2.5,deltacl=0.6]
 {$x$ /0.6,$y'$ /0.6,$y$ /2}
 {$-\infty$,$0$,$ 1 $,$+\infty$}
 \tkzTabLine{,-,d,+,$ 0 $,- }
 \tkzTabVar{+/ $+\infty$,-D- /$-1$/$-\infty$,+/$2$,-/$ -\infty $}
 \end{tikzpicture}
 \end{center}
 Chọn khẳng định đúng
 \choice
 {Đồ thị hàm số có hai tiệm cận ngang}
 {\True Đồ thị hàm số có đúng một tiệm cận đứng}
 {Đồ thị hàm số không có tiệm cận đứng và tiệm cận ngang}
 {Đồ thị hàm số có đúng một tiệm cận ngang}
 \loigiai
 {
 Dựa vào bảng biến thiên ta thấy
 $ \lim\limits_{x \to + \infty} f(x)=+\infty$; $ \lim\limits_{x \to -\infty}f(x)=-\infty$; $ \lim\limits_{x \to 0^+}f(x)=-\infty$.\\
 Suy ra đồ thị hàm số có đúng một tiệm cận đứng.
 }
\end{ex}
\begin{ex}%[2D1N3-1]Câu 5
 \immini{Cho hàm số $y=f(x)$ có đồ thị như hình bên dưới. Khẳng định nào sau đây là khẳng định đúng?
 \choice
 {Đồ thị hàm số chỉ có 2 đường tiệm cận đứng $x=-1$ và $x=1$}
 {\True Đồ thị hàm số có 3 đường tiệm cận}
 {Đồ thị hàm số có 4 đường tiệm cận}
 {Đồ thị hàm số có 2 đường tiệm cận đứng và 1 đường tiệm cận xiên}}{
 \begin{tikzpicture}[>=stealth]
 \draw[->] (-4,0) --(4,0);
 \draw[->](0,-4)--(0,4);
 \draw (0,0) node[below left]{$O$};
 \draw (4,0) node[below]{$x$};
 \draw (0,4) node[left]{$y$};
 \draw (1,0) node[above left]{$1$};
 \draw (-1,0) node[above left]{$-1$};
 \clip (-4,-4) rectangle(4,4);
 \draw[thick,samples=100] plot[domain=-4:4]
 (\x,{(\x)/((\x)^(2)-1)});
 \draw (-1.7,1.5) node
 {$x=-1$};
 \draw (1.5,-1.5) node
 {$x=1$};
 \end{tikzpicture}}
 \loigiai{ Đồ thị hàm số có 2 đường tiệm cận đứng $x=-1$ và $x=1$ và một đường tiệm cận ngang $y=0$, hàm số không có đường tiệm cận xiên.}
\end{ex}
\begin{ex}
 Biết rằng đồ thị hàm số $ y=\dfrac{ax+1}{bx-2}$ có tiệm cận đứng là $x=2$ và tiệm cận ngang là $y=3$. Giá trị của $a+b$ bằng
 \choice
 {$0$}
 {\True $4$}
 {$5$}
 {$1$}
 \loigiai{
 Điều kiện để đồ thị hàm số $ y=\dfrac{ax+1}{bx-2}$ có tiệm cận đứng và tiệm cận ngang là $-2a-b\ne 0$. \quad$(*)$\\
 $b\ne 0$ vì nếu $ b=0$, đồ thị hàm số $ y=\dfrac{ax+1}{-2}$ không có tiệm cận.\\
 Tập xác định của hàm số $y=\dfrac{ax+1}{bx-2}$ là $\mathscr{D}=\left(-\infty;\dfrac{2}{b}\right)\cup\left(\dfrac{2}{b};+\infty\right)$.\\
 $\lim\limits_{x\to\pm\infty}\dfrac{ax+1}{bx-2}=\dfrac{a}{b}\Rightarrow y=\dfrac{a}{b}$ là đường tiệm cận ngang của đồ thị hàm số.\\
 Theo giả thiết ta có $\dfrac{a}{b}=3\Leftrightarrow a=3b$.\\
 Đồ thị hàm số $y=\dfrac{ax+1}{bx-2}$ có $ x=\dfrac{2}{b}$ là đường tiệm cận đứng.\\
 Theo giả thiết ta có $\dfrac{2}{b}=2\Leftrightarrow b=1\Rightarrow a=3$ (thỏa mãn điều kiện $(*)$).\\
 Vậy $a+b=4$.
 }
\end{ex}
\begin{ex}
 Tìm tất cả giá trị của tham số $m$ để đường tiệm cận xiên của đồ thị hàm số $y=2mx+3-\dfrac{4}{x+1}$ đi qua điểm $M(1;7)$.
 \choice
 {$m=1$}
 {$m=3$}
 {\True $m=2$}
 {$m=-2$}
 \loigiai{
 Xét $\displaystyle\lim_{x\to \pm \infty }\left( y-\left( 2mx+3\right) \right) =\displaystyle\lim_{x\to \pm \infty }\dfrac{-4}{x+1}=0$.\\
 Vậy đường tiệm cận xiên có phương trình $y=2mx+3$.\\
 Đường thẳng này qua điểm $M(1;7)$, suy ra $2m \cdot 1 ++3=7 \Leftrightarrow m=2$.
 }
\end{ex}
\begin{ex}
 Tại một công ty sản xuất đồ chơi A, công ty phải chi 50000 USD để thiết lập dây chuyền sản xuất ban đầu. Sau đó, cứ sản xuất được một sản phẩm đồ chơi A, công ty phải trả 5 USD cho nguyên liệu thô và nhân công. Gọi $x\,(x \geq 1)$ là số đồ chơi A mà công ty đã sản xuất và $T(x)$ (đơn vị USD) là tổng số tiền bao gồm cả chi phí ban đầu mà công ty phải chi trả khi sản xuất $x$ đồ chơi A. Người ta xác định chi phí trung bình cho mỗi sản phẩm đồ chơi A là $M(x)=\dfrac{T(x)}{x}$. Khi $x$ đủ lớn ($x\to +\infty$) thì chi phí trung bình (USD) cho mỗi sản phẩm đồ chơi $A$ gần nhất với kết quả nào sau đây?
 \choice
 {$50\,000$}
 {$50\,005$}
 {10}
 {\True $5$}
 \loigiai{
 Gọi $T(x)$ (đơn vị USD) là tổng số tiền bao gồm cả chi phí ban đầu mà công ty phải chi trả khi sản xuất $x$ đồ chơi A thì $T(x)=50\,000 + 5x$.\\
 Ta có $$\displaystyle\lim_{x\to + \infty }\dfrac{T(x)}{x} =\displaystyle\lim_{x\to + \infty }\left(\dfrac{50\,000}{x}+5\right) =5.$$
 }
\end{ex}
\BTTF
\begin{ex}
 Cho hàm số $y=f(x)$ có $\displaystyle\lim_{x\rightarrow 3^{-}}f(x)=1$, $\displaystyle\lim\limits_{x\rightarrow 3^{+}}f(x)=+\infty$ và $\displaystyle\lim_{x\rightarrow -\infty}f(x)=1$, $\displaystyle\lim\limits_{x\rightarrow +\infty}f(x)=+\infty$. Xét tính đúng sai của các khẳng định sau:
 \choiceTF
 {\True Đồ thị của hàm số $y=f(x)$ có tiệm cận ngang là đường thẳng $y=1$}
 {\True Đồ thị của hàm số $y=f(x)$ có tiệm cận đứng là đường thẳng $x=3$}
 {Đồ thị của hàm số $y=f(x)$ không có tiệm cận ngang}
 {Đồ thị của hàm số $y=f(x)$ không có tiệm cận đứng}
 \loigiai{
 \begin{itemchoice}
 \itemch Do $\displaystyle\lim_{x\rightarrow -\infty}f(x)=1$ nên $y=1$ là đường tiệm cận ngang của đồ thị hàm số. (1)
 \itemch Do $\displaystyle\lim\limits_{x\rightarrow 3^{+}}f(x)=+\infty$ nên $x=3$ là đường tiệm cận đứng của đồ thị hàm số. (2)
 \itemch Từ (1) suy ra khẳng định này sai.
 \itemch Từ (2) suy ra khẳng định này sai.
 \end{itemchoice}
 }
\end{ex}
\begin{ex}
 Cho hàm số $y=f(x)$ xác định trên $\mathbb{R}\backslash\{\pm 2\}$ và có bảng biến thiên như hình vẽ bên dưới.
 \begin{center}
 \begin{tikzpicture}[scale=0.8,>=stealth]
 \tikzset{double style/.append style = {draw=\tkzTabDefaultWritingColor,double=\tkzTabDefaultBackgroundColor,double distance=2pt}}
 \tkzTabInit[nocadre=false, lgt=1, espcl=4,deltacl=1pt]{$x$ /1,$y'$ /1,$y$ /2.2}{$-\infty$,$-2$,$2$,$+\infty$}
 \tkzTabLine{,-,d,-,d,-,}
 \tkzTabVar{+/ $0$ ,-D+/ $-10$/$+\infty$ , -D+/ $-\infty$/$+\infty$,-/$0$}
 \end{tikzpicture}
 \end{center}
 Xét tính đúng sai của các khẳng định sau:
 \choiceTF
 {\True Hàm số không có điểm cực trị}
 {$\lim\limits_{x\to -2^{-}}f(x)=+\infty$}
 {\True Đồ thị hàm số có đúng 1 tiệm cận ngang}
 {Đồ thị hàm số có đúng $1$ tiệm cận đứng}
 \loigiai{
 Dựa vào bảng biến thiên ta thấy
 \begin{itemchoice}
 \itemch Hàm số không có điểm cực trị;
 \itemch $\lim\limits_{x\to -2^{-}}f(x)=-10$;
 \itemch $\lim\limits_{x\to \pm \infty}f(x)=0$. Suy ra đồ thị có đúng 1 đường tiệm cận ngang là $y=0$.
 \itemch $\lim\limits_{x\to -2^{+}}f(x)=+\infty$ và $\lim\limits_{x\to 2^{+}}f(x)=+\infty$ nên đồ thị hàm số có đúng 2 đường tiệm cận đứng $x = \pm 2$.
 \end{itemchoice}
 }
\end{ex}
\begin{ex}
 Cho hàm số $y=\dfrac{\sqrt{x^2-x+2}}{x-1}$. Xét tính đúng sai của các khẳng định sau:
 \choiceTF
 {\True Tập xác định của hàm số là $\mathbb{R} \backslash\{1\}$}
 {\True Đồ thị hàm số có các đường tiệm cận ngang là $y=1,\,y=-1$}
 {Đồ thị hàm số đã cho có tất cả 2 đường tiệm cận}
 {Các đường tiệm cận của đồ thị cùng với trục $O y$ tạo thành 1 đa giác có diện tích bằng 1}
 \loigiai{
 \begin{itemchoice}
 \itemch Điều kiện xác định $\heva{&x^2-x+2>0\text{ luôn đúng}\\& x-1 \ne 0} \Leftrightarrow x \ne 1$. Vậy tập xác định của hàm số là $\mathbb{R} \backslash\{1\}$
 \itemch Ta có
 \begin{itemize}
 \item [$\bullet$] $\displaystyle\lim_{x\rightarrow -\infty}f(x)=-1$ nên $y=-1$ là đường tiệm cận ngang;
 \item [$\bullet$] $\displaystyle\lim_{x\rightarrow +\infty}f(x)=1$ nên $y=1$ là đường tiệm cận ngang;
 \end{itemize}
 \itemch Do $\displaystyle\lim_{x\rightarrow 1^+}f(x)=+\infty$ nên $x=1$ là đường tiệm cận đứng. Vậy đồ thị hàm số có tất cả 3 đường tiệm cận (2 TCN và 1 TCĐ).
 \itemch Minh họa miền giới hạn của các đường tiệm cận và trục $Oy$ như sau:
 \begin{center}
 \begin{tikzpicture}[smooth,samples=300,scale=0.8,>=stealth]
 \draw[->] (-3,0)--(6,0) node[below]{$x$};
 \draw[->] (0,-3)--(0,3) node[right]{$y$};
 \draw (0,0) node[below left]{$O$};
 \draw[pattern = north west lines] (0,-1)--(1,-1)--(1,1)--(0,1);
 \draw
 (-3,-1)--(4,-1)node[below]{\scriptsize TCN $y=-1$}
 (-3,1)--(4,1)node[above]{\scriptsize TCN $y=1$}
 (1,-3)--(1,3)node[above right]{\scriptsize TCĐ $x=1$};
 \end{tikzpicture}
 \end{center}
 Miền giới hạn là hình chữ nhật có diện tích là $S=2 \cdot 1 =2$.
 \end{itemchoice}
 }
\end{ex}
\begin{ex}
 Cho hàm số $y=f(x)=\dfrac{2 x^2+2 x+5}{2 x+1}$. Xét tính đúng sai của các khẳng định sau:
 \choiceTF
 {\True Đạo hàm của hàm số đã cho là $y'=\dfrac{4\left(x^2+x-2\right)}{(2 x+1)^2}$}
 {\True Các điểm cực trị của đồ thị hàm số có toạ độ là $(-2 ;-3)$ và $(1 ; 3)$}
 {\True Đường tiệm cận đứng của đồ thị hàm số có phương trình là $x=-\dfrac{1}{2}$}
 {\True Đường tiệm cận xiên của đồ thị hàm số có phương trình là $y=x+\dfrac{1}{2}$}
 \loigiai{
 \begin{itemchoice}
 \itemch Ta có $y'=\dfrac{(2 x^2+2 x+5)'(2 x+1)-(2 x+1)'(2 x^2+2 x+5)}{(2x+1)^2}=\dfrac{4\left(x^2+x-2\right)}{(2 x+1)^2}$.
 \itemch $y'=0 \Leftrightarrow x^2+x-2 =0 \Leftrightarrow \hoac{&x=1\\&x=-2}$.\\
 Thay vào hàm số, ta tính được toạ độ các điểm cực trị là $(-2 ;-3)$ và $(1 ; 3)$.
 \itemch Điều kiện xác định $x \ne -\dfrac{1}{2}$.\\
 $\displaystyle\lim_{x\rightarrow -\frac{1}{2}^+}f(x)=+\infty$ nên $x=-\dfrac{1}{2}$ là đường tiệm cận đứng;
 \itemch $y=\dfrac{2 x^2+2 x+5}{2 x+1}=x+\dfrac{1}{2}+\dfrac{9}{2(2x+1)}$. \\
 Suy ra đồ thị có đường tiệm cận xiên là $y=x+\dfrac{1}{2}$.
 \end{itemchoice}
 }
\end{ex}
\BTTL
\begin{ex}%[2D1B4-1]%
 Các đường tiệm cận của đồ thị hàm số $ y=\dfrac{2x+3}{x-1}$ tạo với hai trục tọa độ một hình chữ nhật có
 diện tích bằng bao nhiêu?\\
 \shortans[3]{$2$}
 \loigiai{
 Tập xác định $\mathscr{D}=\mathbb{R}\setminus\{1\}$.
 \begin{itemize}
 \item $\lim\limits_{x\to 1} y=\lim\limits_{x\to 1^+}\dfrac{2x+3}{x-1}=+\infty\Rightarrow x=1 $ là tiệm cận đứng của đồ thị hàm số.
 \item $\lim\limits_{x\to+\infty} y=\lim\limits_{x\to+\infty}\dfrac{2x+1}{x-1}=2\Rightarrow y=2 $ là tiệm cận ngang của đồ thị hàm số.
 \end{itemize}
 Hai đường tiệm cận của đồ thị hàm số tạo với hai trục tọa độ một hình chữ nhật có diện tích $ S=1\cdot 2=2 $.
 }
\end{ex}
\begin{ex}%[2D1B4-1]%
 Cho hàm số $y=\dfrac{x+1}{x-3}$ có đồ thị $(C)$ và đường thẳng $\Delta: y=mx+m-3.$ Biết đường thẳng $\Delta$ đi qua giao điểm hai đường tiệm cận của
 $(C).$ Khi đó giá trị của $m$ bằng bao nhiêu?\\
 \shortans[3]{$1$}
 \loigiai{
 Đồ thị (C) có TCĐ là $x=3$ và TCN là $y=1$, suy ra $I(3 ; 1)$ là giao điểm hai tiệm cận của $(C)$.\\
 Do $I \in \Delta \Rightarrow 1=3m+m-3 \Leftrightarrow 4m-4=0 \Leftrightarrow m=1$.
 }
\end{ex}
\begin{ex}
 Cho hàm số $y=\dfrac{3x^2+2x}{4x+4}$. Khoảng cách từ điểm $M(3;-2)$ đến đường tiệm cận xiên của đồ thị hàm số này bằng bao nhiêu?\\
 \shortans[3]{$3{,}2$}
 \loigiai{
 $y=\dfrac{3x^2+2x}{4x+4}=\dfrac{3}{4}x-\dfrac{1}{4}+\dfrac{1}{4x+4}$.\\
 Xét $\displaystyle\lim_{x\to \pm \infty }\left( y-\left( \dfrac{3}{4}x-\dfrac{1}{4}\right) \right) =\displaystyle\lim_{x\to \pm \infty }\dfrac{1}{4x+4}=0$.\\
 Vậy đường tiệm cận xiên có phương trình $y=\dfrac{3}{4}x-\dfrac{1}{4} \Leftrightarrow 3x-4y-1=0$.\\
 Khoảng cách từ điểm $M$ đến đường tiệm cận xiên là
 $$d=\dfrac{\big|3 \cdot 3 -4 \cdot (-2)-1\big|}{\sqrt{3^2+(-4)^2}}=\dfrac{16}{5}=3,2$$
 }
\end{ex}
\begin{ex}
 Nồng độ oxygen trong hồ theo thời gian $t$ cho bởi công thức $y(t)=5-\dfrac{15 t}{9 t^2+1}$, với $y$ được tính theo $\mathrm{mg} / l$ và $t$ được tính theo giờ, $t \geq 0$. Đường tiệm cận ngang của đồ thị hàm số $y=y(t)$ khi $t \to +\infty$ có dạng $y=a$. Giá trị của $a$ bằng bao nhiêu?\\
 \shortans[3]{$5$}
 \loigiai{
 $\displaystyle\lim_{t\rightarrow +\infty}y(t)=\lim_{t\rightarrow +\infty}\left( 5-\dfrac{15 t}{9 t^2+1}\right) =5$ nên $y=5$ là đường tiệm cận ngang.
 }
\end{ex}
\begin{ex}
 Số lượng sản phẩm bán được của một công ty trong $x$ (tháng) được tính theo công thức $S(x)=200\left(5-\dfrac{9}{2+x}\right)$, trong đó $x \geq 1$. Xem $y=S(x)$ là một hàm số xác định trên nửa khoảng $[1 ;+\infty)$. Biết $y=a$ là tiệm cận ngang của đồ thị hàm số đó. Giá trị của $a$ bằng bao nhiêu?\\
 \shortans[3]{$1000$}
 \loigiai{
 Ta có
 $S(x)=200\left(5-\dfrac{9}{2+x}\right)=1000-\dfrac{1800}{2+x}.$\\
 Vì $\displaystyle\lim \limits_{n \to +\infty}_{x \rightarrow\pm\infty} S(x)=\lim \limits_{n \to +\infty}_{x \rightarrow\pm\infty} \left(1000-\dfrac{1800}{2+x}\right)=1000$
 nên đường thẳng $y=1000$ là tiệm cận ngang của đồ thị hàm số đã cho.
 }
\end{ex}
\begin{ex}%
 \immini{Cho hàm đa thức bậc ba $y=f(x)$ có đồ thị như hình vẽ.	Đồ thị hàm số $y=\dfrac{(x+1)(x^2-1)}{f(x)}$ có bao nhiêu đường tiệm cận (đứng và ngang)?\\
 \shortans[3]{$3$}}{
 \begin{tikzpicture}[>=stealth]
 \draw[->] (-3,0) --(3,0)node[below]{$x$};
 \draw[->](0,-3)--(0,2.5)node[left]{$y$};
 \draw (0,0.5) node[below left]{$O$};
 \draw (2,0) node[above left]{$2$};
 \draw (-1,0) node[above left]{$-1$};
 \draw[dashed] (0,-2)
 node[left]{$-2$} -- (1,-2) --
 (1,0) node[above]{$1$};
 \draw[thick,samples=100] plot[domain=-2.2:2.3]
 (\x,{(1/2)*(\x)^3-(3/2)*\x-1})node[above]{$y=f(x)$};
 \end{tikzpicture}}
 \loigiai{ Hàm số có dạng $f(x)=ax^3+bx^2+cx-1$ (vì là hàm bậc ba cắt trục tung tại điểm có tung độ $-1$)\\
 Đồ thị hàm số đã cho đi qua các điểm có tọa độ là $(-1;0)$, $(1;-2)$, $(2;0)$ \\
 $\to \heva{&8a+4b+2c=1\\& -a+b-c=1 \\&a+b+c=-1}\Leftrightarrow \heva{&a=\dfrac{1}{2}\\&b=0 \\&c=\dfrac{-3}{2}.}$\\
 $\to f(x)=\dfrac{1}{2}x^3-\dfrac{3}{2}x-1=\dfrac{1}{2}(x-1)^2(x-2)$.\\
 Khi đó $y=\dfrac{(x+1)(x^2-1)}{f(x)}=\dfrac{(x+1)(x^2-1)}{\dfrac{1}{2}(x-1)^2(x-2)}=\dfrac{2(x+1)^2}{(x-1)(x-2)}$.\\
 Đồ thị hàm số trên có tiệm cận ngang $y=2$ và tiệm cận đứng là $x=1,x=2$.\\
 Vậy đồ thị hàm số $y=\dfrac{(x+1)(x^2-1)}{f(x)}$ có 3 đường tiệm cận.
 }
\end{ex}
%\boxde
\BTTN
\begin{ex}%[2D1B4-1]
    Phương trình đường tiệm cận ngang của đồ thị hàm số $y=\dfrac{x-3}{x-1}$ là
    \choice
    {$y=5$}
    {$y=0$}
    {$x=1$}
    {\True $y=1$}
    \loigiai
    {
        Ta có $\lim\limits_{x \to \pm \infty}\dfrac{x-3}{x-1} = \lim\limits_{x \to \pm \infty}\dfrac{1-\dfrac{3}{x}}{1-\dfrac{1}{x}}=1$, nên đường thẳng $y=1$ là tiệm cận ngang của đồ thị hàm số đã cho.
    }
\end{ex}


\begin{ex}
    Đường thẳng nào dưới đây là tiệm cận đứng của đồ thị hàm số $y=\dfrac{2x}{x+2}$?
    \choice
    {$x=2$}
    {$x=0$}
    {\True $x=-2$}
    {$x=1$}
    \loigiai{
        Tập xác định $\mathscr{D}=\mathbb{R}\setminus\{-2\}$.
        \begin{itemize}
            \item $\lim\limits_{x\to -2^+}\dfrac{2x}{x+2}=-\infty$.
            \item $\lim\limits_{x\to -2^-}\dfrac{2x}{x+2}=+\infty$.
        \end{itemize}
        Vậy $x=2$ là đường tiệm cận đứng của đồ thị hàm số.
    }
\end{ex}

\begin{ex}%[2D1Y4-1]
    Cho hàm số $y=\dfrac{x+1}{2x-2}$. Khẳng định nào sau đây đúng?
    \choice
    {Đồ thị hàm số có tiệm cận đứng là $x=\dfrac{1}{2}$}
    {Đồ thị hàm số có tiệm cận ngang là $y=-\dfrac{1}{2}$}
    {\True Đồ thị hàm số có tiệm cận ngang là $y=\dfrac{1}{2}$}
    {Đồ thị hàm số có tiệm cận đứng là $x=2$}
    \loigiai{
        Đồ thị hàm số $y=\dfrac{x+1}{2x-2}$ có tiệm cận đứng $x=1$ và tiệm cận ngang $y=\dfrac{1}{2}$.
    }
\end{ex}


\begin{ex}%[2D1Y4-1]
    Cho hàm số $y=f(x)$ có $\lim\limits_{x\to -\infty} f(x)= -2$ và $\lim\limits_{x\to +\infty} f(x)= 2$. Khẳng định nào sau đây đúng?
    \choice
    {Đồ thị hàm số đã cho có đúng một tiệm cận ngang}
    {Đồ thị hàm số đã cho không có tiệm cận ngang}
    {Đồ thị hàm số đã cho có hai tiệm cận ngang là hai đường thẳng $x=-2$ và $x=2$}
    {\True Đồ thị hàm số đã cho có hai tiệm cận ngang là hai đường thẳng $y=-2$ và $y=2$}
    \loigiai{
        $\lim\limits_{x\to -\infty} f(x)= -2$ nên $y=-2$ là tiệm cận ngang.\\
        $\lim\limits_{x\to +\infty} f(x)= 2$ nên $y=2$ là tiệm cận ngang.
    }
\end{ex}

\begin{ex}%[2D1Y4-1]
    Cho hàm số $y= \dfrac{2017}{x-2}$ có đồ thị $(H)$. Số đường tiệm cận của $(H)$ là
    \choice
    {$0$}
    {\True $2$}
    {$3$}
    {$4$}
    \loigiai{
        Đồ thị $(H)$ có tiệm cận đứng là $x=2$, tiệm cận ngang là $y=0$.\\
        Vậy số đường tiệm cận của $(H)$ là $2$.
    }
\end{ex}

\begin{ex}
    Tìm số đường tiệm cận của đồ thị hàm số $ y = \dfrac{x^2 - 3x + 2}{x^2 - 4}. $
    \choice
    {$1$}
    {$ 0$}
    {\True $2$}
    {$3$}
    \loigiai
    {
        Tập xác định: $ \mathscr D = \mathbb{R} \backslash \{\pm2 \} $.\\
        Ta có $ \lim \limits_{x \to \pm  \infty} y = 1 \Rightarrow  $ đồ thị hàm số có 1 tiệm cận ngang là $ y = 1. $\\
        Ta lại có $\lim \limits_{x \to 2} y =  \lim \limits_{x \to 2} \dfrac{x-1}{x+2} = \dfrac{1}{4} $ và $\lim \limits_{x \to -2^+} y =  \lim \limits_{x \to -2^+} \dfrac{x-1}{x+2} = -\infty$ nên đồ thị hàm số có 1 tiệm cận đứng là $ x = -2. $\\
        Vậy đồ thị hàm số đã cho có 2 đường tiệm cận.
    }
\end{ex}


\begin{ex}%[Đề minh họa BGD 2018-2019]%[2D1B4-1]
    \immini[thm]{Cho hàm số $y=f(x)$ có bảng biến thiên như sau. Tổng số tiệm cận ngang và tiệm cận đứng của đồ thị hàm số đã cho là
        \haicot
        {$4$}
        {$1$}
        {\True $3$}
        {$2$}
    }{
        \begin{tikzpicture}
            \tikzset{double style/.append style = {draw=\tkzTabDefaultWritingColor,double=\tkzTabDefaultBackgroundColor,double distance=2pt}}
            \tkzTabInit[nocadre=false, lgt=1.2, espcl=2.5,deltacl=0.6
            ]{$x$ /0.6,$y'$ /0.6,$y$ /1.6}{$-\infty$,$1$,$+\infty$}
            \tkzTabLine{,+, d ,+,}
            \tkzTabVar{-/ $2$ / , +D-/ $+\infty$ / $3$ , +/ $5$ /}
    \end{tikzpicture}}
    \loigiai{
        Từ bảng biến thiên ta có
        \begin{itemize}
            \item $\lim\limits_{x \to -\infty} y =2$ suy ra $y=2$ là tiệm cận ngang.
            \item $\lim\limits_{x \to +\infty} y =5$ suy ra $y=5$ là tiệm cận ngang.
            \item $\lim\limits_{x \to 1^-} y = +\infty$ suy ra $x=1$ là tiệm cận đứng.
        \end{itemize}
        Vậy đồ thị hàm số tổng cộng có $3$ đường tiệm cận ngang và tiệm cận đứng.

    }
\end{ex}

\begin{ex}%[Đề tập huấn Sở Ninh Bình, 2019]%[Nguyễn Văn Hải, dự án(12EX-5-2019)]%[2D1B4-1]
    \immini[thm]{Cho hàm số $y=f(x)$ có bảng biến thiên như hình bên. Hỏi đồ thị hàm số $y=f(x)$ có tổng số bao nhiêu tiệm cận (tiệm cận đứng và tiệm cận ngang)?
        \haicot
        {$0$}
        {\True $2$}
        {$3$}
        {$1$}
    }{
        \begin{tikzpicture}
            \tikzset{double style/.append style = {draw=\tkzTabDefaultWritingColor,double=\tkzTabDefaultBackgroundColor,double distance=2pt}}
            \tkzTabInit[nocadre=false,lgt=1,espcl=2,deltacl=0.6]
            {$x$ /0.6,$y’$ /0.6,$y$ /2.2}
            {$-\infty$ , $1$ , $3$ , $+\infty$}
            \tkzTabLine{,+,d,+,0,-,}
            \tkzTabVar{-/$-1$ ,+D- / $+\infty$ /  $-\infty$,+/ $2$, -/$-\infty$}
    \end{tikzpicture}}
    \loigiai{
        Ta có $\lim\limits_{x\to -\infty}f(x)=-1$, $\lim\limits_{x\to +\infty}f(x)=-\infty$ nên $y=-1$ là tiệm cận ngang.\\
        Ta có $\lim\limits_{x\to 1^+}f(x)=-\infty$ nên $x=1$ là tiệm cận đứng.\\
        Vậy đồ thị hàm số có $2$ đường tiệm cận.
    }
\end{ex}


\begin{ex}%[GHK1, THCS - THPT Nguyễn Khuyến, HCM, 2019]%[Vinh Vo, 12Ex3-2019]%[2D1B4-1]
    \immini[thm]{Cho hàm số $ y = f(x) $ xác định trên $ (-2;0) \cup (0;+\infty) $ và có bảng biến thiên như hình vẽ. Số đường tiệm cận của đồ thị hàm số $ f(x) $ là
        \haicot
        {$ 4 $}
        {$ 2 $}
        {$ 1 $}
        {\True $ 3 $}
    }{
        \begin{tikzpicture}
            \tikzset{double style/.append style = {double distance=2pt}}
            \tkzTabInit[lgt=1.2,espcl=3.5,nocadre=false]
            {$x$ /0.6, $f’(x)$ /0.7,$f(x)$ /1.7}
            { $-2$ , $0$ , $+\infty$}
            \tkzTabLine{d,+,d,-, }
            \tkzTabVar{D-/ /$ -\infty $, +D+/$ +\infty $/ $ 1 $, -/ $ 0 $}
            \draw[pattern = north west lines] ($(N13)-(0.2ex,0)$) rectangle (T11);
    \end{tikzpicture}}
    \loigiai{
        Từ bảng biến thiên, ta thấy $ \heva{& \lim \limits_{ x \to -2^{+} } f(x) = - \infty \\ & \lim \limits_{x \to 0^{-} } f(x) = + \infty \\ & \lim \limits_{x \to + \infty} f(x) = 0  } $, suy ra đồ thị hàm số $ f(x) $ có $ 3 $ tiệm cận trong đó có $ 2 $ tiệm cận đứng và $ 1 $ tiệm cận ngang.
    }
\end{ex}

\begin{ex}%[Đề thi khảo sát chất lượng trường THCS-THPT Lômônôxốp, Hà Nội 2018 ,Nhật Thiện 12EX1-2019]%[2D1Y4-1]
    \immini[thm]{Cho hàm số $y=f(x)$ xác định trên $\mathbb{R}\backslash \left\{{0}\right\}$, liên tục trên mỗi khoảng xác định và có bảng biến thiên như hình bên. Hỏi đồ thị hàm số có bao nhiêu đường tiệm cận?
        \haicot
        {$1$}
        {\True $2$}
        {$3$}
        {$4$}}{\begin{tikzpicture}
            \tikzset{double style/.append style = {draw=\tkzTabDefaultWritingColor,double=\tkzTabDefaultBackgroundColor,double distance=2pt}}
            \tkzTabInit[nocadre=false,lgt=1,espcl=2.3,deltacl=0.6]{$x$ /0.6,$y'$ /0.6,$y$ /1.8}{$-\infty$,$0$,$1$,$+\infty$}
            \tkzTabLine{,+,d,-,0,+}
            \tkzTabVar{+/ $2$ ,-D-/ $-\infty$/$-\infty$, +/ $1$ ,-/ $-\infty$ /}
    \end{tikzpicture}}
    \loigiai{
        Dựa vào bảng biến thiên, ta có $$\lim\limits_{x\to -\infty}y=2;\qquad \lim\limits_{x\to 0^{\pm}}y=-\infty$$
        Vậy hàm số có một tiệm cận ngang $y=2$, một tiệm cận đứng $x=0$.
    }
\end{ex}

\begin{ex}
    Số tiệm cận đứng của đồ thị hàm số $y=\dfrac{\sqrt{x+9}-3}{x^2+x}$ là
    \choice
    {$3$}
    {$2$}
    {$0$}
    {\True $1$}
    \loigiai{
        Tập xác định $\mathscr{D}=[-9;+\infty)\setminus \{-1;0\}$. \\
        Ta có $\left\{\begin{aligned}
            &\lim\limits_{x\to -1^+} \dfrac{\sqrt{x+9}-3}{x^2+x}=+\infty \\
            &\lim\limits_{x\to -1^-} \dfrac{\sqrt{x+9}-3}{x^2+x}=-\infty
        \end{aligned}\right. \Rightarrow x=-1$ là tiệm cận đứng. \\
        Ngoài ra $\lim\limits_{x\to 0} \dfrac{\sqrt{x+9}-3}{x^2+x}=\dfrac{1}{6}$ nên $x=0$ không là tiệm cận.}
\end{ex}

\begin{ex}
    Phương trình đường tiệm cận xiên của đồ thị hàm số $y=\dfrac{2x^2-x+1}{x-1}$ là
    \choice
    {$y=x-1$}
    {\True $y=2x+1$}
    {$y=2x+3$}
    {$y=x+1$}
    \loigiai{
        Sau khi chia đa thức, ta viết lại hàm số $y=2x+1+\dfrac{2}{x-1}$.\\
        Do $\lim\limits_{x\to \pm \infty}\left[y-(2x+1)\right]=\lim\limits_{x\to \pm \infty}\dfrac{2}{x-1}=0$ nên $y=2x+1$ là đường tiệm cận xiên.}
\end{ex}

\begin{ex}
    Giao điểm của đường tiệm cận đứng và đường tiệm cận xiên của đồ thị hàm số $y=\dfrac{x^2-3x+5}{x-2}$ có tọa độ là
    \choice
    {$(2;3)$}
    {$(-2;1)$}
    {\True $(2;1)$}
    {$(-2;3)$}
    \loigiai{
        Sau khi chia đa thức, ta viết lại hàm số $y=x-1+\dfrac{3}{x-2}$.
        \begin{itemize}
            \item [$\bullet$] Đồ thị hàm số có tiệm cận đứng là $x=2$
            \item [$\bullet$] Do $\lim\limits_{x\to \pm \infty}\left[y-(x-1)\right]=\lim\limits_{x\to \pm \infty}\dfrac{3}{x-2}=0$ nên $y=x-1$ là đường tiệm cận xiên.
        \end{itemize}
        Giải hệ $\heva{&x=2\\&y=x-1} \Leftrightarrow \heva{&x=2\\&y=1}$. Suy ra, giao hai đường tiệm cận có tọa độ $(2;1)$.
    }
\end{ex}

\begin{ex}%[Đề thi giữa HK1, THPT Bình Sơn Đồng Nai, 2019]%[Phan Minh Tâm, dự án EX3]%[2D1B4-2]
    Tiệm cận đứng của đồ thị hàm số $ y=\dfrac{2x+1}{x-m} $ đi qua điểm $ M(2;5) $ khi $ m $ bằng bao nhiêu?
    \choice
    {$ m=-2 $}
    {$ m=-5 $}
    {$ m=5 $}
    {\True $ m=2 $}
    \loigiai{
        Với $m \ne -\dfrac{1}{2}$ đồ thị có tiệm cận đứng là đường thẳng $ x=m $. Tiệm cận đứng $ x=m $ đi qua $ M(2;5) $ khi chỉ khi $ m=2 $.
    }
\end{ex}

\begin{ex}%[GHK1, THPT Quế Võ 2-Bắc Ninh, 2019]%[TranTony,12EX2]%[2D1B4-2]
    Cho hàm số $ y = \dfrac{2x^2-3x+m}{x-m} $ có đồ thị $ (C) $. Tìm tất cả các giá trị của tham số $ m $ để $ (C) $ không có tiệm cận đứng.
    \choice
    {\True $ m = 0 $ hoặc $ m = 1 $}
    {$ m = 2 $}
    {$ m = 1 $}
    {$ m = 0 $}
    \loigiai{
        Đồ thị $ (C) $ không có tiệm cận đứng khi $ m $ là nghiệm của $ 2x^2-3x+m $
        \begin{align*}
            \Leftrightarrow 2m^2 - 3m + m = 0 \Leftrightarrow \hoac{& m = 0 \\& m = 1.}
        \end{align*}
    }
\end{ex}

\BTTF

\begin{ex}
    Cho hàm số $y=f(x)=\dfrac{3-2x}{x+1}$. Xét tính đúng sai của các khẳng định sau:
    \choiceTF
    {\True Tập xác định của hàm số là $\mathbb{R}\backslash\{-1\}$}
    {\True  Đồ thị hàm số có đường tiệm cận đứng là $x=-1$}
    {Đồ thị hàm số có đường tiệm cận ngang là $y=3$}
    {Hai đường tiệm cận (đứng và ngang) của đồ thị tạo với hai trục tọa độ một hình phẳng có diện tích bằng $3$}
    \loigiai{
        \begin{itemchoice}
            \itemch Điều kiện xác định $x+1 \ne 0 \Leftrightarrow x \ne -1$. Suy ra $D=\mathbb{R}\backslash\{-1\}$.
            \itemch Đồ thị hàm số có tiệm cận đứng là $x=-1$.
            \itemch Đồ thị hàm số có tiệm cận ngang là $y=\dfrac{-2}{1}=-2$.
            \itemch Hai đường tiệm cận (đứng và ngang) của đồ thị tạo với hai trục tọa độ một hình chữ nhật như hình vẽ
            \begin{center}
                \begin{tikzpicture}[smooth,samples=300,scale=0.8,>=stealth]
                    \draw[->] (-3,0)--(2,0) node[below]{$x$};
                    \draw[->] (0,-3)--(0,1) node[right]{$y$};
                    \draw (0,0) node[below right]{$O$};
                    \draw[pattern = north west lines] (0,0)--(0,-2)--(-1,-2)--(-1,0);
                    \draw (-3,-2)--(2,-2)node[above]{\scriptsize TCN $y=-2$} (-1,-3)--(-1,1)node[above]{\scriptsize TCĐ $x=-1$};
                    \draw[fill=black] (-1,0) circle(1.5pt) (-1,-2) circle(1pt) (0,-2) circle(1.5pt);
                    \node[right] at (0,-2.3) {$A$};
                    \node[left] at (-1,0.3) {$B$};
                \end{tikzpicture}
            \end{center}
            Diện tích hình chữ nhật này là
            $$S=OA \cdot OB=2 \cdot 1=2.$$
    \end{itemchoice}}
\end{ex}

\begin{ex}%[2D1K4]
    Cho hàm số $y=f(x)$ xác định trên $(-\infty;2) \backslash\{-2\}$ và có bảng biến thiên như hình vẽ dưới đây.
    \begin{center}
        \begin{tikzpicture}
            \tikzset{double style/.append style = {draw=\tkzTabDefaultWritingColor,double=\tkzTabDefaultBackgroundColor,double distance=2pt}}
            \tkzTabInit[nocadre=false,lgt=1,espcl=3]
            {$x$ /0.7,$y'$ /0.7,$y$ /2.1}
            {$-\infty$,$-2$,$0$,$2$,$+\infty$}
            \tkzTabLine{,+,d,-,0,+,d,}
            \tkzTabVar{-/$-2$/,+D+/ $3$ / $+\infty$,-/$-2$/,+D/ $+\infty$ / }
            \draw[pattern = north west lines] ($(N43)+(0.1ex,0)$) rectangle (T21);
        \end{tikzpicture}
    \end{center}
    Xét tính đúng sai của các khẳng định sau:
    \choiceTF
    {\True Hàm số có giá trị nhỏ nhất bằng $-2$}
    {Hàm số có giá trị lớn nhất bằng $3$}
    {\True Đồ thị hàm số có hai đường tiệm cận đứng là $x=-2$ và $x=2$}
    {Đồ thị hàm số có hai đường tiệm cận ngang là $y=-2$ và $y=3$}
    \loigiai
    {
        Căn cứ vào bảng biến thiên của hàm số, ta có
        \begin{itemchoice}
            \itemch Hàm số đạt giá trị nhỏ nhất bằng $-2$ khi $x=0$.
            \itemch Do $\lim\limits_{x\to 2^{-}}f(x)=+\infty$ nên hàm số không có giá trị lớn nhất.
            \itemch Do $\lim\limits_{x\to -2^{+}}f(x)=+\infty$ và $\lim\limits_{x\to 2^{-}}f(x)=+\infty$ nên đồ thị hàm số có hai đường tiệm cận đứng là $x=-2$ và $x=2$.
            \itemch Do $\lim\limits_{x\to -\infty}f(x)=-2$ nên đồ thị hàm số có hai đường tiệm cận ngang là $y=-2$.
    \end{itemchoice}}
\end{ex}

\begin{ex}
    Cho hàm số $y=f(x)=\dfrac{5 x^2+9 x+9}{x-4}$. Xét tính đúng sai của các khẳng định sau:
    \choiceTF
    {Tập xác định của hàm số là $\mathbb{R}\backslash\{-4\}$}
    {\True Đường tiệm cận đứng của đồ thị hàm số có phương trình là $x=4$}
    {\True Đường tiệm cận xiên của đồ thị hàm số có phương trình là $y=5x+29$}
    {Giao điểm hai đường tiệm cận của đồ thị hàm số có toạ độ là $(4;29)$}
    \loigiai{
        Hàm số được viết thành $y=5x+29+\dfrac{125}{x-4}$.
        \begin{itemchoice}
            \itemch Điều kiện $x-4 \ne 0 \Leftrightarrow x \ne 4$. Suy ra $D=\mathbb{R}\backslash\{4\}$.
            \itemch Đường tiệm cận đứng của đồ thị hàm số có phương trình là $x=4$
            \itemch Đường tiệm cận xiên của đồ thị hàm số có phương trình là $y=5x+29$
            \itemch Giải hệ $\heva{&x=4\\&y=5x+29}\Leftrightarrow \heva{&x=4\\&y=49}$. Giao điểm hai đường tiệm cận của đồ thị hàm số có toạ độ là $(4;49)$.
    \end{itemchoice}}
\end{ex}

\begin{ex}
    Cho hàm số $y=f(x)=\dfrac{x^2-4 x+7}{x-1}$. Xét tính đúng sai của các khẳng định sau:
    \choiceTF
    {\True Đường tiệm cận đứng của đồ thị hàm số có phương trình là $x=1$}
    {\True Đường tiệm cận xiên của đồ thị hàm số có phương trình là $y=x-3$}
    {\True Giao điểm hai đường tiệm cận của đồ thị hàm số có toạ độ là $(1 ;-2)$}
    {Diện tích tam giác tạo bởi đường tiệm cận xiên của đồ thị hàm số và hai trục toạ độ là $\dfrac{9}{4}$}
    \loigiai{
        Hàm số được viết thành $y=x-3+\dfrac{4}{x-1}$.
        \begin{itemchoice}
            \itemch Đường tiệm cận đứng của đồ thị hàm số có phương trình là $x=1$
            \itemch Đường tiệm cận xiên của đồ thị hàm số có phương trình là $y=x-3$
            \itemch Giải hệ $\heva{&x=1\\&y=x-3}\Leftrightarrow \heva{&x=1\\&y=-2}$. Giao điểm hai đường tiệm cận của đồ thị hàm số có toạ độ là $(1;-2)$.
            \itemch Giao của đường thẳng $d \colon y=x-3$ với các trục tọa độ lần lượt tại $A(0;-3)$ và $B(3;0)$.\\
            Diện tích tam giác $OAB$ là $S=\dfrac{1}{2} OA \cdot OB=\dfrac{9}{2}$.
    \end{itemchoice}}
\end{ex}

\BTTL

\begin{ex}%[2D1N4-1]Câu 1
    Cho hàm số $y=\dfrac{2x-1}{x+3}$. Gọi $x=m$ và $y=n$ lần lượt là đường tiệm cận đứng và tiệm cận ngang của đồ thị hàm số. Tính giá trị của biểu thức $P=\dfrac{2m-1}{n+3}$.\\
    \shortans[3]{$-1{,}4$}
    \loigiai{Ta có:\\
        $\bullet \underset{x \to -3}{\lim}\,y=\infty \Rightarrow x=-3$ là đường tiệm cận đứng.\\
        $\bullet \underset{x \to +\infty}{\lim}\,y=2$ và $\underset{x \to -\infty}{\lim}\,y=2 \Rightarrow y=2$ là đường tiệm cận ngang.\\
        Vậy $m=-3; n=2 \Rightarrow P=\dfrac{2\cdot(-3)-1}{2+3}=\dfrac{-7}{5}=-1{,}4$.}
\end{ex}

\begin{ex}%[2D1B4-3]
    Cho đồ thị $(C)\colon y=\dfrac{x-3}{x+2}$ có hai đường tiệm cận cắt nhau tại $I$. Với $O$ là gốc tọa độ, hãy tính độ dài đoạn thẳng $OI$ (làm tròn đến hàng phần trăm).\\
    \shortans[3]{$2{,}24$}
    \loigiai{
        Ta có tiệm cận đứng của đồ thị $(C)$ là $x=-2$ và tiệm cận ngang là $y=1$. Do đó $I(-2;1)$ là giao điểm của hai đường tiệm cận của đồ thị $(C)$.\\
        Ta có $OI=\sqrt{ (-2-0)^2+(1-0)^2}=\sqrt{5} \approx 2{,}24$.
    }
\end{ex}

\begin{ex}
    Nếu trong một ngày, một xưởng sản xuất được $x$ kilôgam sản phẩm thì chi phí trung bình (tính bằng nghìn đồng) cho một sản phẩm được cho bởi công thức:
    $$
    y=C(x)=\dfrac{50x+2000}{x}
    $$
    Đồ thị hàm số $C(x)$ có một đường tiệm cận ngang (khi $x \to +\infty$) là $y=y_0$. Giá trị $y_0$ bằng bao nhiêu?\\
    \shortans[3]{$50$}
    \loigiai{
        Ta có $\lim\limits_{x \rightarrow+\infty} \dfrac{50x+2000}{x}=\lim\limits_{x \rightarrow+\infty} \left(50+\dfrac{2000}{x}\right)=50$.\\
        Vậy đường thẳng $y=50$ là tiệm cận ngang của đồ thị hàm số.
    }
\end{ex}

\begin{ex}%[2D1C4-3]Câu 6
    Cho hàm số $y=\dfrac{x-2}{x^2-3mx+m}$ tìm $m$ để đồ thị hàm số có đúng một tiệm cận đứng. Biết tổng các giá trị của tham số $m$ có dạng phân số $\dfrac{a}{b}$, tính tổng $S=a+b$.\\
    \shortans[3]{$101$}
    \loigiai{Dễ thấy tử thức có một nghiệm là $x=2$ do đó để đồ thị hàm số có đúng một tiệm cận đứng thì phương trình $x^2-3mx+m=0$ có nghiệm kép hoặc có hai nghiệm phân biệt trong đó có một nghiệm bằng $2$.\\
        $\Rightarrow \hoac{&\Delta=0\\&\heva{&\Delta>0\\&4-3\cdot2m+m=0}} \Leftrightarrow \hoac{&9m^2-4m=0\\&\heva{&9m^2-4m>0\\&4-3\cdot2m+m=0}} \Leftrightarrow \hoac{&m=0\\&m=\dfrac{4}{9}\\&\heva{&\hoac{&m<0\\&m>\dfrac{4}{9}}\\&m=\dfrac{4}{5}}} \Leftrightarrow \hoac{&m=0\\&m=\dfrac{4}{9}\\&m=\dfrac{4}{5}}$\\
        Vậy tổng các giá trị của tham số $m$ bằng $\dfrac{56}{45} \Rightarrow S=101$.
    }
\end{ex}

\begin{ex}%[2D1H4-1]Câu 4
    Cho hàm số $y=\dfrac{x^2+2x-3}{x-2}$, đồ thị hàm số có đường tiệm cận xiên có dạng $(C) \colon y=ax+b$. Tính giá trị của biểu thức $P=\dfrac{a}{b}$.\\
    \shortans[3]{$0{,}25$}
    \loigiai{Ta xét $y=\dfrac{x^2+2x-3}{x-2}=x+4+\dfrac{5}{x-2} \Rightarrow (C) \colon y=x+4$ là đường tiệm cận xiên của đồ thị hàm số. Vậy $P=\dfrac{1}{4}$.}
\end{ex}

\begin{ex}
    Gọi $d$ là đường tiệm cận xiên của đồ thị hàm số $y=mx+4-3m+\dfrac{3}{x+2}$, $m$ là tham số. Đường thẳng $d$ luôn qua điểm cố định $M$. Tính độ dài đoạn $OM$, với $O$ là gốc tọa độ.\\
    \shortans[3]{$5$}
    \loigiai{
        Đường tiệm cần xiên của đồ thị là $y=mx+4-3m \Leftrightarrow (x-3)m+4-y=0$.\\
        Ta có $(x-3)m+4-y=0,\,\forall m \Leftrightarrow \heva{&x-3=0\\&4-y=0} \Leftrightarrow \heva{&x=3\\&y=4}$.\\
        Đường thẳng này luôn qua điểm cố định $M(3;4)$. Khi đó $OM=\sqrt{3^2+4^2}=5$.
    }
\end{ex}
%\boxde
\BTTN
\Opensolutionfile{ans}[ans/2D1-4-DEON-1]
\begin{ex}%[2D1B4-1]
    Cho hàm số $y=f(x)$ có bảng biến thiên như hình bên. Đồ thị hàm số đã cho có tiệm cận ngang là đường thẳng
    \begin{center}
        \begin{tikzpicture}[scale=0.8, font=\footnotesize, line join=round, line
            cap=round, >=stealth]
            \tkzTabInit[espcl=2.5,lgt=1,nocadre=false]
            {$x$/0.7,$f(x)$/2.1}
            {$-\infty$,$0$,$1$,$2$,$+\infty$}
            \tkzTabVar{-/$-\infty$,+/$2$,-D+/$-\infty$/$+\infty$,-/$4$,+/$6$}
        \end{tikzpicture}
    \end{center}
    \choice
    {$y=2$}
    {$y=1$}
    {\True $y=6$}
    {$y=4$}
    \loigiai{Dựa vào bảng biến thiên ta thấy đồ thị hàm có tiệm cận ngang $y=6$.
    }
\end{ex}
%56
\begin{ex}%[2D1B4-1]
    Cho hàm số $y=f(x)$ có bảng biến thiên như hình bên. Tổng số tiệm cận đứng và tiệm cận ngang của đồ thị hàm số đã cho là
    \begin{center}
        \begin{tikzpicture}[scale=0.8]
            \tkzTabInit[nocadre=false,lgt=1.5,espcl=3,deltacl=0.6]
            {$x$ /0.6,$y’$ /0.6,$y$ /2}
            {$-\infty$ , $1$, $+\infty$}
            \tkzTabLine{,+,d,+,}
            \tkzTabVar{-/$2$,+D-/$+\infty$/$3$,+/$5$}
        \end{tikzpicture}
    \end{center}
    \choice
    {$1$}
    {\True $3$}
    {$2$}
    {$4$}
    \loigiai{Dựa vào bảng biến thiên ta thấy đồ thị hàm số có tiệm cận đứng $x=1$ và tiệm cận ngang $y=2$ và $y=5$.}
\end{ex}
\begin{ex}%[2D1B4-1]
    Cho hàm số $y=f(x)$ có bảng biến thiên như hình bên. Tổng số tiệm cận đứng và tiệm cận ngang của đồ thị hàm số đã cho là
    \begin{center}
        \begin{tikzpicture}[scale=0.8]
            \tkzTabInit[nocadre=false,lgt=1.5,espcl=3,deltacl=0.6]
            {$x$ /0.6,$y’$ /0.6,$y$ /2}
            {$-\infty$ ,$0$, $1$, $+\infty$}
            \tkzTabLine{,+,0,-,d,-,}
            \tkzTabVar{-/$4$,+/$2$,-D+/$-1$/$+\infty$,-/$-3$}
        \end{tikzpicture}
    \end{center}
    \choice
    {$1$}
    {\True $3$}
    {$2$}
    {$4$}
    \loigiai{
        Dựa vào bảng biến thiên ta thấy đồ thị hàm số có tiệm cận đứng $x=1$, tiệm cận ngang $y=4$ và $y=-3$.
    }
\end{ex}
%61
\begin{ex}%[2D1B4-1]
    Cho hàm số $y=f(x)$ có bảng biến thiên như hình bên. Tổng số tiệm cận đứng và tiệm cận ngang của đồ thị hàm số đã cho là
    \begin{center}
        \begin{tikzpicture}[scale=0.8]
            \tkzTabInit[nocadre=false,lgt=1.5,espcl=3,deltacl=0.6]
            {$x$ /0.6,$y’$ /0.6,$y$ /2}
            {$-\infty$ ,$0$, $1$, $+\infty$}
            \tkzTabLine{,-,0,+,d,+,}
            \tkzTabVar{+/$5$,-/$-4$,+D-/$+\infty$/$-\infty$,+/$2$}
        \end{tikzpicture}
    \end{center}
    \choice
    {$1$}
    {\True $3$}
    {$2$}
    {$4$}
    \loigiai{Dựa vào bảng biến thiên ta thấy đồ thị hàm số có một tiệm cận đứng $x=1$, hai tiệm cận ngang $y=5$ và $y=2$.}
\end{ex}
\begin{ex}%[2D1B4-1]
    Đồ thị hàm số nào trong các hàm số dưới đây có tiệm cận đứng?
    \choice
    {\True $y=\dfrac{1}{\sqrt{x}}$}
    {$y=\dfrac{1}{x^2+x+1}$}
    {$y=\dfrac{1}{x^4+1}$}
    {$y=\dfrac{1}{x^2+1}$}
    \loigiai{
    }
\end{ex}
\begin{ex}%[2D1K4-1]
    Số tiệm cận đứng của đồ thị hàm số $y=\dfrac{\sqrt{x+4}-2}{x^2+x}$ là
    \choice
    {$3$}
    {$0$}
    {$2$}
    {\True $1$}
    \loigiai{
        Tập xác định hàm số $ \mathscr{D}=[-4;+\infty)\setminus\lbrace -1;0\rbrace
        $.\\
        Ta có $ \lim\limits_{x\to -1^{+}}y=+\infty $, $ \lim\limits_{x\to 0^{+}}y=1
        $ và $ \lim\limits_{x\to 0^{-}}y=1 $.\\
        Suy ra đồ thị hàm số chỉ có $ 1 $ tiệm cận đứng là $ x=-1 $.
    }
\end{ex}
\begin{ex}%[Nguyễn Văn Sang, dự án Tex hoá đề cương trường Marie Curie - Lần 6]%[2D1Y4-1]
    Đường thẳng nào dưới đây là tiệm cận ngang của đồ thị hàm số $y=\dfrac{3+2 x}{x+1}$?
    \choice
    {$y=3$}
    {$x=-1$}
    {\True $y=2$}
    {$x=2$}
    \loigiai{
        Tập xác định $\mathscr{D}=\mathbb{R}\setminus\left\lbrace -1\right\rbrace$.
        \begin{itemize}
            \item $\lim\limits_{x \to \pm\infty} y=\lim\limits_{x \to \pm\infty} \dfrac{3+2 x}{x+1}=2$ suy ra $y=2$ là tiệm cận ngang.
            \item $\heva{& \lim\limits_{x \to -1^+} \dfrac{3+2 x}{x+1}=+\infty \\ & \lim\limits_{x \to -1^-} \dfrac{3+2 x}{x+1}=-\infty}$ suy ra $x=-1$ là tiệm cận đứng.
        \end{itemize}
    }
\end{ex}
%%=====Câu 15
\begin{ex}%[2D1Y4-1]
    Giao điểm của tiệm cận đứng và tiệm cận ngang của đồ thị hàm số $y=\dfrac{3x-2}{1-x}$ là điểm
    \choice
    {$M(1;3)$}
    {$P(-3;1)$}
    {\True $Q(1;-3)$}
    {$N\left(\dfrac{2}{3};3\right)$}
    \loigiai{
        Tiệm cận đứng, tiệm cận ngang của đồ thị hàm số lần lượt là $x=1$ và $y=-3$. Giao điểm của $2$ tiệm cận là $Q(1;-3)$.
    }
\end{ex}
\begin{ex}%[2D1K4-2]%
    Nếu đồ thị hàm số $y=\dfrac{(m+1)x+2}{x-n+1}$ lần lượt nhận trục hoành và trục tung làm đường đường tiệm cận ngang và tiệm cận đứng thì $m+n$ bằng bao nhiêu?
    \choice
    {\True $m+n=0$}
    {$m+n=2$}
    {$m+n=-1$}
    {$m+n=1$}
    \loigiai{
        Theo đề bài, ta có $\heva{&m+1=0\\&n-1=0} \Leftrightarrow \heva{&m=-1\\&n=1.}$\\
        Suy ra $m+n=0$.
    }
\end{ex}
\begin{ex}%[2D1K4-1]
    Cho hàm số $y=f(x)$ có bảng biến thiên như hình bên. Đồ thị hàm số $y=\dfrac{x-2}{f(x)-1}$ có bao nhiêu tiệm cận đứng?
    \begin{center}
        \begin{tikzpicture}
            \tkzTabInit[espcl=3]{$x$ / 1 , $f’(x)$ / 1, $f(x)$ / 2}
            {$-\infty$, $-1$ , $5$, $+\infty$}%
            \tkzTabLine{,-,0,+,0,-,}%
            \tkzTabVar{+/ $+\infty$, - / $-1$, + / $3$,-/$-2$}%
            \tkzTabVal[draw]{2}{3}{0.4}{$2$}{$1$}
        \end{tikzpicture}
    \end{center}
    \choice
    {$1$}
    {$3$}
    {\True $2$}
    {$4$}
    \loigiai{
        Dựa vào bảng biến thiên suy ra
        $f(x)-1=0 \Leftrightarrow f(x) =1$, phương trình này có $2$ nghiệm phân biệt khác $2$ và một nghiệm $x=2$ nên đồ thị hàm số $y=\dfrac{x-2}{f(x)-1}$ có hai tiệm cận đứng.
    }
\end{ex}
%68
\begin{ex}%[2D1K4-1]
    Cho hàm số $y=f(x)$ có bảng biến thiên như hình bên. Đồ thị hàm số $y=\dfrac{1}{2f(x)+1}$ có bao nhiêu tiệm cận đứng?
    \begin{center}
        \begin{tikzpicture}[scale=0.8]
            \tkzTabInit[nocadre=false,lgt=1.5,espcl=3,deltacl=0.6]
            {$x$ /0.6,$y’$ /0.6,$y$ /2}
            {$-\infty$ ,$-2$, $2$, $+\infty$}
            \tkzTabLine{,+,0,-,0,+,}
            \tkzTabVar{-/$-\infty$,+/$3$,-/$0$,+/$+\infty$}
        \end{tikzpicture}
    \end{center}
    \choice
    {\True $1$}
    {$3$}
    {$2$}
    {$0$}
    \loigiai{
        Dựa vào bảng biến thiên suy ra
        $2f(x)+1=0 \Leftrightarrow f(x) =-\dfrac{1}{2}$, phương trình này có $1$ nghiệm nên đồ thị hàm số $y=\dfrac{1}{2f(x)+1}$ có một tiệm cận đứng.
    }
\end{ex}
\begin{ex}%[2D1K4-1]
    Cho hàm số $y=f(x)$ có bảng biến thiên như hình bên. Đồ thị hàm số $y=\dfrac{1}{2f(x)-1}$ có bao nhiêu tiệm cận ngang?
    \begin{center}
        \begin{tikzpicture}[scale=0.8]
            \tkzTabInit[nocadre=false,lgt=1.5,espcl=3,deltacl=0.6]
            {$x$ /0.6,$y’$ /0.6,$y$ /2}
            {$-\infty$, $2$, $+\infty$}
            \tkzTabLine{,-,0,+,}
            \tkzTabVar{+/$1$,-/$-3$,+/$1$}
        \end{tikzpicture}
    \end{center}
    \choice
    {$1$}
    {\True $3$}
    {$2$}
    {$0$}
    \loigiai{
        Dựa vào bảng biến thiên suy ra
        \begin{itemize}
            \item 	$\lim \limits_{x \to \pm \infty} f(x)=1 \Leftrightarrow \lim \limits_{x \to \pm \infty}\dfrac{1}{2f(x)-1} =1$ nên đồ thị hàm số đã cho có tiệm cận ngang là $y=1$.
            \item $2f(x)-1=0 \Leftrightarrow f(x)=\dfrac{1}{2}$, phương trình này có $2$ nghiệm phân biệt nên đồ thị hàm số đã cho có hai tiệm cận đứng.
        \end{itemize}
    }
\end{ex}
%80
\begin{ex}%[2D1K4-1]
    Cho hàm số $y=f(x)$ có bảng biến thiên như hình bên. Đồ thị hàm số $y=\dfrac{1}{f^2(x)+f(x)}$ có bao nhiêu tiệm cận đứng?
    \begin{center}
        \begin{tikzpicture}[scale=0.8]
            \tkzTabInit[nocadre=false,lgt=1.5,espcl=3,deltacl=0.6]
            {$x$ /0.6,$y’$ /0.6,$y$ /2}
            {$-\infty$ ,$-4$, $6$, $+\infty$}
            \tkzTabLine{,-,0,+,0,-,}
            \tkzTabVar{+/$+\infty$,-/$-2$,+/$5$,-/$-\infty$}
        \end{tikzpicture}
    \end{center}
    \choice
    {$4$}
    {$3$}
    {$2$}
    {\True $6$}
    \loigiai{
        Dựa vào bảng biến thiên suy ra $f^2(x)+f(x)=0 \Leftrightarrow \hoac{&f(x)=0\\&f(x)=-1}$, mỗi phương trình này có $3$ nghiệm phân biệt nên đồ thị hàm số đã cho có $6$ tiệm cận đứng.
    }
\end{ex}
\begin{ex}%[2D1K4-1]
    Cho hàm số $y=f(x)$ có bảng biến thiên như hình bên. Đồ thị hàm số $y=\dfrac{3}{f(x^2)+1}$ có bao nhiêu tiệm cận đứng?
    \begin{center}
        \begin{tikzpicture}[scale=0.8]
            \tkzTabInit[nocadre=false,lgt=1.5,espcl=3,deltacl=0.6]
            {$x$ /0.6,$y’$ /0.6,$y$ /2}
            {$-\infty$ ,$0$, $2$, $+\infty$}
            \tkzTabLine{,+,d,-,0,+,}
            \tkzTabVar{-/$-\infty$,+/$1$,-/$-2$,+/$+\infty$}
        \end{tikzpicture}
    \end{center}
    \choice
    {\True $4$}
    {$3$}
    {$6$}
    {$0$}
    \loigiai{
        Dựa vào bảng biến thiên suy ra
        $f(x^2)+1=0 \Leftrightarrow f(x^2) =-1$. Kẻ đường thẳng $y=-1$ ta thấy đường thẳng cắt đồ thị hàm số tại 3 điểm phân biệt. Suy ra
        $$\hoac{&x^2=a \; (a<0)\\&x^2=b \; (b \in (0;2)\\&x^2=c \; (c>2)} \Rightarrow \hoac{&x=\pm \sqrt{b}\\&x=\pm \sqrt{c}.}$$
        Do đó đồ thị hàm số $y=\dfrac{2}{f(x^2)+1}$ có $4$ tiệm cận đứng.
    }
\end{ex}
\BTTF
\begin{ex}%[EX-TF-2024, Lê Đạt]%[2D1N4-1]
    Cho hàm số $y=\dfrac{2x-3}{x-1}$. Xét tính đúng sai các khẳng định dưới đây
    \choiceTF
    {\True Đường tiệm cận đứng của đồ thị hàm số là $ x=1 $}
    {Đường tiệm cận đứng của đồ thị hàm số là $ y=2 $}
    {Đường tiệm cận ngang của đồ thị hàm số là $ x=1 $}
    {\True Đường tiệm cận ngang của đồ thj hàm số là $ y=2 $}
    \loigiai{
        Ta có $\lim\limits_{x\to -\infty}y=\lim\limits_{x\to +\infty}y=2$ nên đồ thị hàm số đã cho có tiệm cận ngang là $y=2$.\\
        Ta có $\lim\limits_{x\to 1^+}y=-\infty$ nên đồ thị hàm số đã cho có tiệm cận ngang là $ x=1 $.
        \begin{itemchoice}
            \itemch Đường tiệm cận đứng của đồ thị hàm số là $ x=1 $.
            \itemch Đường tiệm cận đứng của đồ thị hàm số là $ x=1 $.
            \itemch Đường tiệm cận ngang của đồ thj hàm số là $ y=2 $.
            \itemch Đường tiệm cận ngang của đồ thj hàm số là $ y=2 $.
        \end{itemchoice}
    }
\end{ex}
%===== DẠNG 2
\begin{ex}%[EX-TF-2024, Lê Đạt]%[2D1H4-2]
    Cho hàm số $ y=\dfrac{m^2x+1}{x-1} $. Xét tính đúng sai của các khẳng định sau
    \choiceTF
    {\True Đồ thị hàm số luôn có tiệm cận ngang}
    {\True Đồ thị hàm số luôn có tiệm cận đứng}
    {\True Khi $ m=1$ đồ thị hàm số có $ 2 $ đường tiệm cận}
    {Khi $ m=0 $ đồ thị hàm số có $ 1 $ đường tiệm cận}
    \loigiai{
        \begin{itemchoice}
            \itemch $\lim\limits_{x\to -\infty}y=\lim\limits_{x\to +\infty}y=m^2$ suy ra hàm số luôn có tiệm cận ngang.
            \itemch $\lim\limits_{x\to 1^+}y=+\infty$ nên đồ thị hàm số đã cho có tiệm cận ngang là $ x=1 $.
            \itemch Khi $ m=1 $ ta được hàm số $ y=\dfrac{x+1}{x-1} $ suy ra đồ thì hàm số có $ x=1 $ là tiệm cận đứng và $ y=1 $ là tiệm cận ngang nên đồ thị hàm số có $ 2 $ tiệm cận.
            \itemch Khi $ m=0 $ ta được hàm số $ y=\dfrac{1}{x-1} $ suy ra đồ thì hàm số có $ x=1 $ là tiệm cận đứng và $ y=0 $ là tiệm cận ngang nên đồ thị hàm số có $ 2 $ tiệm cận.
        \end{itemchoice}
    }
\end{ex}
%===== DẠNG 3
\begin{ex}%[EX-TF-2024, Lê Đạt]%[2D1N4-3]
    \immini{Cho hàm số $y=f(x)$ có đồ thị như hình bên. Xét tính đúng sai của các khẳng định sau
        \choiceTF
        {$ x=2 $ là đường tiệm cận ngang của đồ thị hàm số}
        {\True $ x=-1 $ là đường tiệm cận đứng của đồ thị hàm số}
        {\True Đồ thị hàm số có hai đường tiệm cận}
        {\True Đồ thị hàm số không có tiệm cận xiên}
    }{
        \begin{tikzpicture}[scale=0.5, font=\footnotesize, line join=round, line cap=round, >=stealth]
            \draw[->](-5,0)--(5,0)node[below]{ $x$};
            \draw[->](0,-4)--(0,5)node[right]{ $y$};
            \draw [fill=black,draw=black] (0,0) circle (1pt)node[above left] { $O$};
            \foreach \x in {-1}\draw[shift={(\x,0)}](0pt,-2pt)--(0pt,2pt) node[below left]{ $\x$};
            \foreach \y in {2}\draw[shift={(0,\y)}](-2pt,0pt)--(2pt,0pt)node[above right]{ $\y$};
            \clip(-5,-4) rectangle (5,5);
            \draw[smooth,samples=100,domain=-5:-1.1] plot(\x,{(2*(\x)-1)/((\x)+1)});
            \draw[smooth,samples=100,domain=-0.9:5] plot(\x,{(2*(\x)-1)/((\x)+1)});
            \draw[dashed](-5,2)--(5,2) (-1,-4)--(-1,5);
        \end{tikzpicture}
    }
    \loigiai{
        \begin{itemchoice}
            \itemch $ y=2 $ là đường tiệm cận ngang của đồ thị hàm số.
            \itemch $ x=-1 $ là đường tiệm cận đứng của đồ thị hàm số.
            \itemch $ x=-1 $ là đường tiệm cận đứng và $ y=2 $ là đường tiệm cận ngang của đồ thị hàm số suy ra đồ thị hàm số có hai đường tiệm cận.
            \itemch Đồ thị hàm số không có tiệm cận xiên.
        \end{itemchoice}
    }
\end{ex}
\BTTL
\begin{ex}%[2D1K4-2]%
    Đường tiệm cận đứng và tiệm cận ngang của đồ thị hàm số $y=\dfrac{mx+1}{2m+1-x}$ cùng với hai trục tọa độ tạo thành một hình chữ nhật có diện tích bằng $3$. Khi đó $m$ bằng
    \shortans{$1$ hay $-\dfrac{3}{2}$}
    % \choice
    % {$1$ hay $\dfrac{3}{2}$}
    % {$-1$ hay $-\dfrac{3}{2}$}
    % {\True $1$ hay $-\dfrac{3}{2}$}
    % {$-1$ hay $3$}
    \loigiai{
        Từ yêu cầu đề bài, suy ra $|-m| \cdot |2m+1|=3 \Leftrightarrow \hoac{&m=1\\&m=-\dfrac{3}{2}.}$
    }
\end{ex}
\begin{ex}%[2D1K4-2]%
    Tìm tất cả các giá trị thực $m$ sao cho đồ thị hàm số $y=\dfrac{5x-3}{x^2-2mx+1}$ không có tiệm cận đứng.
    \shortans{$-1<m<1$}
    % \choice
    % {\True $-1<m<1$}
    % {$m=1$}
    % {$m=-1$}
    % {$m <-1$ hoặc $m>1$}
    \loigiai{
        Xét $f(x)=5x-3$, có $f(x)=0\Leftrightarrow x=\dfrac{3}{5}$; $g(x)=x^2-2mx+1$ có $\Delta’=m^2-1$.\\
        Đồ thị hàm số không có tiệm cận đứng khi phương trình $g(x)=0$ vô nghiệm $\Leftrightarrow m^2-1<0\Leftrightarrow-1<m<1$.\\
        Vậy với $-1<m<1$ thì đồ thị hàm số đã cho không có tiệm cận đứng.}
\end{ex}
\begin{ex}%[Nguyễn Văn Sang, dự án VDC-Hàm số 2020 - Lần 2]%[2D1K4-2]%
    Cho hàm số $y=\dfrac{1+\sqrt{x+1}}{\sqrt{x^2-mx-3m}}$ với $m$ là tham số. Tìm tập hợp các giá trị của tham số $m$ để đồ thị hàm số có hai tiệm cận đứng.
    \shortans{$\left(0;\dfrac{1}{2}\right)$}
    % \choice
    % {\True $\left(0;\dfrac{1}{2}\right)$}
    % {$\left(\left. 0;\dfrac{1}{2}\right]\right.$}
    % {$\left(0;+\infty \right)$}
    % {$\left(-\infty;-12\right)\cup \left(0;+\infty \right)$}
    \loigiai{
        Ta có $\sqrt{x+1}$ xác định khi $x\ge-1.$\\
        Yêu cầu bài toán $\Leftrightarrow $ phương trình $x^2-mx-3m=0$ có hai nghiệm phân biệt $x_1$, $x_2$ thỏa mãn $$-1<x_1<x_2\Leftrightarrow \heva{
            & \Delta >0 \\
            & x_1+x_2>-2 \\
            & a\cdot f\left(-1\right)>0 \\}\Leftrightarrow \heva{
            & m^2+12m>0 \\
            & m>-2 \\
            & 1\cdot \left(1-2m\right)>0 \\}\Leftrightarrow 0<m<\dfrac{1}{2}.$$
    }
\end{ex}
\begin{ex}%[VDC5-NgocDungHo]%[2D1G4-3]%
    \immini{Cho hàm số $f(x)$ có đồ thị như hình bên. Số đường tiệm cận đứng của đồ thị hàm số $y=\dfrac{(x^2-4)(x^2+2x)}{[f(x)]^2+2f(x)-3}$ là bao nhiêu?
        \shortans{$4$}
        % \choice
        % {\True $4$}
        % {$5$}
        % {$3$}
        % {$2$}
    }{\begin{tikzpicture}[>=stealth,scale=0.5, line join=round, line cap=round]
            \def\f[#1]{0.25*((#1)^4-8*(#1)^2+4)}
            \draw[->] (-4.1,0)--(4,0) node [below]{$x$};
            \draw[->] (0,-3.5)--(0,4) node [left]{$y$};
            \node at (0,0) [below left]{$O$};
            % \clip;
            \draw[domain=-3:3,samples=300,thick] plot (\x,{\f[\x]});
            \foreach \x in {-2,2} \filldraw (\x,0) node[above]{\x} circle (2pt);
            \foreach \x in {-3,3} \filldraw (\x,0) node[below]{\x} circle (2pt);
            \filldraw (0,1) node[above left]{$1$} circle (2pt);
            \filldraw (0,-3) node[below left]{$-3$} circle (2pt);
            \draw[dashed](-2,0)--(-2,-3)--(2,-3)--(2,0);
            \draw (3,2.75) node[right]{$y=f(x)$};
    \end{tikzpicture}}
    \loigiai{%GV tổng quát hóa bài toán:
        Cho hàm số $f(x)$ có đồ thị $(C)$ cho trước. Xác định số đường tiệm cận đứng của đồ thị hàm số $y=\dfrac{u(x)}{v[f(x)]}$.
        \begin{enumerate}
            \item Tìm tập xác định của hàm số $y=\dfrac{u(x)}{v[f(x)]}$.\\
            \item Tìm nghiệm của phương trình $u(x)=0\quad (1)$.\\
            \item Tìm nghiệm của phương trình $v[f(x)]=0\quad (2)$. Giả sử $f(x)=m_1$, $f(x)=m_2,\ldots$.
        \end{enumerate}
        Dựa vào đồ thị $(C)$, xác định hoành độ giao điểm của $(C)$ với các đường thẳng $d_1\colon f(x)=m_1$, $d_2\colon f(x)=m_2,\ldots$.\\
        Số đường tiệm cận đứng của đồ thị hàm số $y=\dfrac{u(x)}{v[f(x)]}$ chính là tổng của:
        \begin{itemize}
            \item Số nghiệm riêng của phương trình $(2)$.
            \item Số nghiệm chung $x=x_0$ của $(1) $ và $(2)$ mà bậc của $(x-x_0)$ ở mẫu lớn hơn bậc của $(x-x_0)$ ở tử.
        \end{itemize}
        \noindent
        Xét hàm số $y=g(x)=\dfrac{(x^2-4)(x^2+2x)}{[f(x)]^2+2f(x)-3}$.
        \immini
        {
            Giải phương trình $(x^2-4)(x^2+2x)=0\,(1)$\\$ \Leftrightarrow \hoac{& x^2-4=0 \\ & x^2+2x=0}\Leftrightarrow \hoac{& x=\pm 2 \\ & x=0.}$\\
            Giải phương trình $[f(x)]^2+2f(x)-3=0\,(2)$\\
            $ \Leftrightarrow \hoac{& f(x)=1 \\ & f(x)=-3.}$\\
        }
        {\begin{tikzpicture}[>=stealth,scale=0.7, line join=round, line cap=round]
                \def\f[#1]{0.25*((#1)^4-8*(#1)^2+4)}
                \def\g[#1]{1}
                \def\h[#1]{-3}
                \draw[->] (-4.1,0)--(4,0) node [below]{$x$};
                \draw[->] (0,-3.5)--(0,4) node [left]{$y$};
                \node at (0,0) [below left]{$O$};
                % \clip;
                \draw[domain=-3:3,samples=300,thick] plot (\x,{\f[\x]});
                \draw[domain=-4:4,samples=300,thick] plot (\x,{\g[\x]});
                \draw[domain=-4:4,samples=300,thick] plot (\x,{\h[\x]});
                \foreach \x in {-2,2} \filldraw (\x,0) node[above]{\x} circle (2pt);
                \filldraw (-3,0) node[above left]{$-3$} circle (2pt);
                \filldraw (3,0) node[above right]{$3$} circle (2pt);
                \filldraw (-2.85,0) node[below]{$a$} circle (2pt);
                \filldraw (2.85,0) node[below]{$b$} circle (2pt);
                \filldraw (0,1) node[above left]{$1$} circle (2pt);
                \filldraw (0,-3) node[below left]{$-3$} circle (2pt);
                \draw[dashed](-2,0)--(-2,-3)--(2,-3)--(2,0) (-2.85,0)--(-2.85,1) (2.85,0)--(2.85,1);
                \draw (3,2.75) node[right]{$(C):y=f(x)$};
                \draw (4.2,1) node[above]{$d_1:y=1$};
                \draw (4,-3) node[below]{$d_2:y=-3$};
            \end{tikzpicture}
        }
        Dựa vào đồ thị đã cho $(2)\Leftrightarrow \hoac{& x = \pm 2 \\ & x=0\\&x=a\\&x=b.}$
        với $-3<a<-2<2<b<3$.\\
        Trong điều kiện xác định của hàm số $y=g(x)$ ta có thể viết $$y=g(x)=\dfrac{x(x-2)(x+2)^2}{x^2(x-a)(x-b)(x-2)^2(x+2)^2}=\dfrac{1}{x(x-a)(x-b)(x-2)}$$
        Vậy số tiệm cận đứng của đồ thị hàm số $y=g(x)$ bằng $4$.
    }
\end{ex}
\begin{ex}
    \immini{%Câu 97.
        Đường cong ở hình bên là đồ thị của hàm số $y = ax^3 +bx^2 +cx+d$. Đồ thị hàm số $y =\dfrac{(x+1)\sqrt{1-x}}{f(x^2)}$ có tất cả bao nhiêu tiệm cận đứng?
        \shortans{$2$}
        % \choice
        % {1}
        % {6}
        % {4}
        % {\True 2}
    }{\begin{tikzpicture}[scale=.6, font=\footnotesize, line join=round, line cap=round, >=stealth]
            \def\xmin{-2}\def\xmax{4}\def\ymin{-3}\def\ymax{3}
            \draw[->] (\xmin-0.2,0)--(\xmax+0.2,0) node[below] {\footnotesize $x$};
            \draw[->] (0,\ymin-0.2)--(0,\ymax+0.2) node[right] {\footnotesize $y$};
            \draw (0,0) node [below left] {\footnotesize $O$};
            \foreach \x in {1,2}\draw (\x,-0.1)--(\x,0.1) node [above ] {\footnotesize $\x$};
            \foreach \x in {-1,3}\draw (\x,-0.1)--(\x,0.1) node [above left] {\footnotesize $\x$};
            \foreach \y in {-2}\draw (0.1,\y)--(-0.1,\y) node [left] {\footnotesize $\y$};
            \foreach \y in {2}\draw (-0.1,\y)--(0.1,\y) node [right] {\footnotesize $\y$};
            \clip (\xmin,\ymin) rectangle (\xmax,\ymax);
            \draw[smooth,samples=200,domain=\xmin:\xmax] plot (\x,{0.6666666666666666*((\x)^3)+-2*((\x)^2)+-0.6666666666666666*(\x)+2});
            \draw[dashed] (1.0,0)--(1.0,0.0)--(0,0.0);\fill (1.0,0.0) circle (1pt);
            \draw[dashed] (2,0)--(2,-2)--(0,-2);
    \end{tikzpicture}}
    \loigiai{
        * Điều kiện: $\heva{&f(x^2) \ne 0\\&x \le 1.}$\\
        Nhìn hình vẽ ta thấy
        $f(x^2)=0\Leftrightarrow \hoac{&x^2=-1\\&x^2=1\\&x^2=3}\Leftrightarrow \hoac{&x=\pm 1\,(\text{nghiệm đơn})\\&x=- \sqrt{3}\,(\text{nghiệm đơn})\\&x= \sqrt{3}\,(\text{không thỏa mãn})}.$\\
        Vậy $y=\dfrac{(x+1)\sqrt{1-x}}{f(x^2)}=\dfrac{(x+1)\sqrt{1-x}}{(x - 1)(x + 1)(x + \sqrt{3})}$ \\
        Đồ thị hàm số có 2 đường tiệm cận đứng.}
\end{ex}
\begin{ex}
    \immini{ %Câu 95.
        Đường cong ở hình bên là đồ thị của hàm số $y = ax^3 +bx^2 +cx+d$. Đồ thị hàm số $y =\dfrac{(2x+1)\sqrt{x-1}}{f(|x|)}$ có tất cả bao nhiêu tiệm cận đứng?
        \shortans{$1$}
        % \choice
        % {\True 1}
        % {3}
        % {4}
        % {2}
    }{\begin{tikzpicture}[scale=.5, font=\footnotesize, line join=round, line cap=round, >=stealth]
            \def\xmin{-3}\def\xmax{3}\def\ymin{-5}\def\ymax{5}
            \draw[->] (\xmin-0.2,0)--(\xmax+0.2,0) node[below] {\footnotesize $x$};
            \draw[->] (0,\ymin-0.2)--(0,\ymax+0.2) node[right] {\footnotesize $y$};
            \draw (0,0) node [below left] {\footnotesize $O$};
            \foreach \x in {-1,2}\draw (\x,0.1)--(\x,-0.1) node [below] {\footnotesize $\x$};
            \foreach \x in {-2,1}\draw (\x,-0.1)--(\x,0.1) node [above] {\footnotesize $\x$};
            \foreach \y in {-4,2}\draw (-0.1,\y)--(0.1,\y) node [right] {\footnotesize $\y$};
            \foreach \y in {-2,4}\draw (0.1,\y)--(-0.1,\y) node [left] {\footnotesize $\y$};
            \clip (\xmin,\ymin) rectangle (\xmax,\ymax);
            \draw[smooth,samples=200,domain=\xmin:\xmax] plot (\x,{1.3333333333333333*((\x)^3)+0*((\x)^2)+-3.3333333333333335*(\x)+0});
            \draw[dashed] (-2,0)--(-2,-4)--(0,-4);
            \draw[dashed] (2,0)--(2,4)--(0,4);
            \draw[dashed] (1,0)--(1,-2)--(0,-2);
            \draw[dashed] (-1,0)--(-1,2)--(0,2);
    \end{tikzpicture}}
    \loigiai{
        * Điều kiện: $\heva{&f(|x|) \ne 0\\&x \ge 1.}$\\
        Nhìn hình vẽ ta thấy
        $f(|x|)=0\Leftrightarrow \hoac{&|x|=x_1\,(-2<x_1<-1)\\&|x|=0\\&|x|=x_2\,(1<x_2<2)}\Leftrightarrow \hoac{&x=0&(\text{không thỏa mãn})\\&x=- x_2&(\text{không thỏa mãn})\\&x=x_2&(\text{nghiệm đơn}).}$\\
        Vậy $y =\dfrac{(2x+1)\sqrt{x-1}}{f(|x|)}=\dfrac{(2x+1)\sqrt{x-1}}{ax(x+x_2)(x-x_2)}.$ \\
        Đồ thị hàm số có 1 đường tiệm cận đứng.}
\end{ex}
\begin{ex}
    Đáp ứng tần số của một hệ thống điều khiển có thể được mô tả bởi hàm truyền \( H(s) = \dfrac{\omega_n^2}{s^2 + 2\zeta\omega_ns + \omega_n^2} \), trong đó \( \omega_n \) là tần số tự nhiên và \( \zeta \) là hệ số tắt dần. Tìm đường tiệm cận ngang của đáp ứng tần số khi tần số góc \( s \) tăng và nêu ý nghĩa của nó.
    \shortans{$y=0$}
    \loigiai{
        Khi \( s \) tăng vô hạn, các thành phần bậc cao trong mẫu số chiếm ưu thế:
        \[
        H(s) \approx \frac{\omega_n^2}{s^2}
        \]
        Đường tiệm cận ngang của \( H(s) \) khi \( s \to \infty \) là:
        \[
        |H(s)| \approx \frac{\omega_n^2}{s^2} \to 0
        \]}
\end{ex}
\begin{ex}
    Trong thuyết tương đối của Einstein, khối lượng của vật chuyển động với vận tốc $v$ được cho bởi công thức:
    $$m(v)=\dfrac{m_0}{\sqrt{1-\dfrac{v^2}{c^2}}},$$
    trong đó $m_0$ là khối lượng của vật khi nó đứng yên, $c$ là vận tốc ánh sáng.\\
    (nguồn: https://www.britannica.com/science/relativity/Relativistic-mass)\\
    Xem $m$ là hàm số theo vận tốc $v$, tìm đường tiệm cận đứng của đồ thị hàm số. Từ đó nhận xét khối lượng của vật khi vận tốc của nó càng gần với vận tốc ánh sáng.
    \shortans{$v=c$, khối lượng tăng lên vô hạn}
    \loigiai{
        Điều kiện xác định: $\heva{&1-\dfrac{v^2}{c^2}>0\\
            &v>0}\Leftrightarrow\heva{& -c<v<c\\
            &v>0}\Leftrightarrow 0<v<c$.\\
        Ta có $\lim\limits_{v\to c^{-}} m(v)=\lim\limits_{v\to c^{-}}\dfrac{m_0}{\sqrt{1-\dfrac{v^2}{c^2}}}=+\infty$ nên đường thẳng $v=c$ là tiệm cận đứng của đồ thị hàm số.\\
        Từ đó ta suy ra khi vận tốc của vật càng sát với vận tốc ánh sáng thì khối lượng của vật tăng lên vô hạn.
    }
\end{ex}
\Closesolutionfile{ans}
%\boxde
\BTTN
\Opensolutionfile{ans}[ans/2D1-4-DEON-2]
\begin{ex}%[2D1B4-1]
    Cho hàm số $y=f(x)$ có bảng biến thiên như hình bên. Tổng số tiệm cận đứng và tiệm cận ngang của đồ thị hàm số đã cho là
    \begin{center}
        \begin{tikzpicture}
            \tkzTabInit[nocadre=false,lgt=1.5,espcl=3,deltacl=0.6]
            {$x$ /0.6,$y’$ /0.6,$y$ /2}
            {$-\infty$ ,$0$, $1$, $+\infty$}
            \tkzTabLine{,-,d,+,0,-,}
            \tkzTabVar{+/$+\infty$,-D-/$-1$/$-\infty$,+/$2$,-/$-\infty$}
        \end{tikzpicture}
    \end{center}
    \choice
    {\True $1$}
    {$3$}
    {$2$}
    {$4$}
    \loigiai{
        Dựa vào bảng biến thiên ta thấy đồ thị hàm số có một tiệm cận đứng $x=0$.
    }
\end{ex}
%62
%63
\begin{ex}%[2D1B4-1]
    Cho hàm số $y=f(x)$ xác định, liên tục trên $\mathbb{R} \backslash \{0;1\}$ và có bảng biến thiên như hình bên. Đồ thị hàm số $y=f(x)$ có
    \begin{center}
        \begin{tikzpicture}
            \tkzTabInit[nocadre=false,lgt=1.5,espcl=3,deltacl=0.6]
            {$x$ /0.6,$y’$ /0.6,$y$ /2}
            {$-\infty$ ,$0$, $1$, $+\infty$}
            \tkzTabLine{,+,d,+,d,+,}
            \tkzTabVar{-/$-5$,+D-/$+\infty$/$-\infty$,+D-/$3$/$-\infty$,+/$+\infty$}
        \end{tikzpicture}
    \end{center}
    \choice
    {\True $2$ tiệm cận đứng và $1$ tiệm cận ngang}
    {$2$ tiệm cận đứng và $2$ tiệm cận ngang}
    {$1$ tiệm cận đứng và $1$ tiệm cận ngang}
    {$1$ tiệm cận đứng và $2$ tiệm cận ngang}
    \loigiai{
        Dựa vào bảng biến thiên ta thấy đồ thị hàm số có hai tiệm cận đứng $x=0$ và $x=1$; một tiệm cận ngang $y=-5$.
    }
\end{ex}
%64
\begin{ex}%[2D1B4-1]
    Cho hàm số $y=f(x)$ có bảng biến thiên như hình bên. Tổng số tiệm cận đứng và tiệm cận ngang của đồ thị hàm số đã cho là
    \begin{center}
        \begin{tikzpicture}[scale=0.8]
            \tkzTabInit[nocadre=false,lgt=1.5,espcl=3,deltacl=0.6]
            {$x$ /0.6,$y’$ /0.6,$y$ /2}
            {$-\infty$ ,$-1$, $1$, $+\infty$}
            \tkzTabLine{,+,d,+,0,-,}
            \tkzTabVar{-/$2$,+D-/$4$/$-\infty$,+/$3$,-/$-1$}
        \end{tikzpicture}
    \end{center}
    \choice
    {$1$}
    {\True $3$}
    {$2$}
    {$4$}
    \loigiai{Dựa vào bảng biến thiên ta thấy đồ thị hàm số có một tiệm cận đứng $x=-1$; hai tiệm cận ngang $y=-1$ và $y=2$.
    }
\end{ex}
%65
\begin{ex}%[2D1B4-1]
    Cho hàm số $y=f(x)$ có bảng biến thiên như hình bên. Tổng số tiệm cận đứng và tiệm cận ngang của đồ thị hàm số đã cho là
    \begin{center}
        \begin{tikzpicture}[scale=0.8]
            \tkzTabInit[nocadre=false,lgt=1.5,espcl=3,deltacl=0.6]
            {$x$ /0.6,$y’$ /0.6,$y$ /2}
            {$-\infty$ ,$0$, $3$, $+\infty$}
            \tkzTabLine{,-,0,+,d,-,}
            \tkzTabVar{+/$8$,-/$1$,+/$4$,-/$2$}
        \end{tikzpicture}
    \end{center}
    \choice
    {$1$}
    {$3$}
    {\True $2$}
    {$4$}
    \loigiai{
        Dựa vào bảng biến thiên ta thấy đồ thị hàm số có hai tiệm cận ngang $y=2$ và $y=8$.
    }
\end{ex}
\begin{ex}%[Nguyễn Văn Sang, dự án Tex hoá đề cương trường Marie Curie - Lần 6]%[2D1Y4-1]
    Đường thẳng nào dưới đây là tiệm cận đứng của đồ thị hàm số $y=\dfrac{2 x+1}{x+1}$?
    \choice
    {$x=1$}
    {$y=-1$}
    {$y=2$}
    {\True $x=-1$}
    \loigiai{
        Tập xác định $\mathscr{D}=\mathbb{R}\setminus\left\lbrace -1\right\rbrace$.
        \begin{itemize}
            \item $\lim\limits_{x \to \pm\infty} y=\lim\limits_{x \to \pm\infty} \dfrac{2x+1}{x+1}=2$ suy ra $y=2$ là tiệm cận ngang.
            \item $\heva{& \lim\limits_{x \to -1^+} \dfrac{2x+1}{x+1}=-\infty \\ & \lim\limits_{x \to -1^-} \dfrac{2x+1}{x+1}=+\infty}$ suy ra $x=-1$ là tiệm cận đứng.
        \end{itemize}
    }
\end{ex}
\begin{ex}%[2D1Y4-1]
    Đồ thị hàm số $y=\dfrac{2x-3}{2x+1}$ có tâm đối xứng là điểm
    \choice
    {\True $M\left(-\dfrac{1}{2};1\right)$}
    {$P\left(-\dfrac{1}{2};2\right)$}
    {$Q\left(-\dfrac{1}{2};-3\right)$}
    {$N\left(1;-\dfrac{1}{2}\right)$}
    \loigiai{
        Tiệm cận đứng, tiệm cận ngang của đồ thị hàm số lần lượt là $x=-\dfrac{1}{2}$ và $y=3$. Tâm đối xứng là điểm $M\left(-\dfrac{1}{2};1\right)$.
    }
\end{ex}
\begin{ex}%[2D1K4-1]
    Đồ thị hàm số $y=\dfrac{\sqrt{x}}{x+1}-\dfrac{1}{x}$ có tất cả bao nhiêu tiệm cận đứng và ngang?
    \choice
    {$0$}
    {$3$}
    {\True $2$}
    {$1$}
    \loigiai{
        Tập xác định $\mathscr{D}=(0;+\infty)$.
        \begin{itemize}
            \item $\lim\limits_{x\to 0^+} \left(\dfrac{\sqrt{x}}{x+1}-\dfrac{1}{x}\right)=-\infty$.
            \item $\lim\limits_{x\to +\infty}\left(\dfrac{\sqrt{x}}{x+1}-\dfrac{1}{x}\right)=0$.
        \end{itemize}
        Suy ra đồ thị hàm số có tiệm cận đứng $x=0$, tiệm cận ngang $y=0$.
    }
\end{ex}
\begin{ex}%[2D1K4-1]
    Số tiệm cận đứng của đồ thị hàm số $y=\dfrac{x^2-3x-4}{x^2-16}$ là
    \choice
    {$2$}
    {$3$}
    {\True $1$}
    {$0$}
    \loigiai{
        Điều kiện xác định $x \ne \pm 4$.\\
        Với điều kiện xác định trên, ta có $y=\dfrac{x^2-3x-4}{x^2-16}=\dfrac{(x+1)(x-4)}{(x-4)(x+4)}=\dfrac{x+1}{x+4}$.\\
        Tiệm cận đứng của đồ thị hàm số là $x=-4$.
    }
\end{ex}
\begin{ex}%[2D1K4-1]
    Số đường tiệm cận đứng và ngang của đồ thị hàm số $y=\dfrac{x-1}{x^2-x-2}$ là
    \choice
    {\True $3$}
    {$1$}
    {$0$}
    {$2$}
    \loigiai{
        Điều kiện xác định $x \ne -1$, $x \ne 2$.\\
        Với điều kiện xác định trên, ta có $y=\dfrac{x-1}{x^2-x-2}=\dfrac{x-1}{(x+1)(x-2)}$.\\
        Tiệm cận đứng của đồ thị hàm số là $x=-1$, $x=2$, tiệm cận ngang của đồ thị hàm số là $y=0$.
    }
\end{ex}
%81
\begin{ex}%[2D1K4-1]
    Cho hàm số $y=f(x)$ có bảng biến thiên như hình bên. Đồ thị hàm số $y=\dfrac{1}{f^2(x)-2f(x)}$ có bao nhiêu tiệm cận đứng?
    \begin{center}
        \begin{tikzpicture}[scale=0.8]
            \tkzTabInit[nocadre=false,lgt=1.5,espcl=3,deltacl=0.6]
            {$x$ /0.6,$y’$ /0.6,$y$ /2}
            {$-\infty$ ,$-1$, $2$, $+\infty$}
            \tkzTabLine{,+,d,-,0,+,}
            \tkzTabVar{-/$-\infty$,+/$1$,-/$-2$,+/$+\infty$}
        \end{tikzpicture}
    \end{center}
    \choice
    {\True $4$}
    {$3$}
    {$2$}
    {$6$}
    \loigiai{
        Dựa vào bảng biến thiên suy ra $f^2(x)-2f(x)=0 \Leftrightarrow \heva{&f(x)=0\\&f(x)=2}$, phương trình $f(x)=0$ có $3$ nghiệm phân biệt và phương trình $f(x)=2$ có $1$ nghiệm nên đồ thị hàm số đã cho có $4$ tiệm cận đứng.
    }
\end{ex}
%79
\begin{ex}%[2D1K4-1]
    Cho hàm số $y=f(x)$ có bảng biến thiên như hình bên. Tổng số tiệm cận ngang và tiệm cận đứng của đồ thị hàm số $y=\dfrac{2}{f(x)+3}$ là
    \begin{center}
        \begin{tikzpicture}[scale=0.8]
            \tkzTabInit[nocadre=false,lgt=1.5,espcl=3,deltacl=0.6]
            {$x$ /0.6,$y’$ /0.6,$y$ /2}
            {$-\infty$ ,$-4$, $6$, $+\infty$}
            \tkzTabLine{,-,0,+,0,-,}
            \tkzTabVar{+/$+\infty$,-/$-2$,+/$5$,-/$-\infty$}
        \end{tikzpicture}
    \end{center}
    \choice
    {$4$}
    {$3$}
    {\True $2$}
    {$1$}
    \loigiai{
        Dựa vào bảng biến thiên suy ra
        \begin{itemize}
            \item 	$\lim \limits_{x \to \pm \infty} f(x)=\pm \infty \Leftrightarrow \lim \limits_{x \to \pm \infty}\dfrac{2}{f(x)+3} =0$ nên đồ thị hàm số đã cho có tiệm cận ngang là $y=0$.
            \item $f(x)+3=0 \Leftrightarrow f(x) =-3$, phương trình này có $1$ nghiệm $x=a>6$ nên đồ thị hàm số đã cho có một tiệm cận đứng.
        \end{itemize}
    }
\end{ex}
\begin{ex}%[2D1K4-1]
    Cho hàm số $y=f(x)$ có bảng biến thiên như hình bên. Đồ thị hàm số $y=\dfrac{x+1}{f(x)-4}$ có bao nhiêu tiệm cận đứng?
    \begin{center}
        \begin{tikzpicture}[scale=0.8]
            \tkzTabInit[nocadre=false,lgt=1.5,espcl=3,deltacl=0.6]
            {$x$ /0.6,$y’$ /0.6,$y$ /2}
            {$-\infty$ ,$-1$, $2$, $+\infty$}
            \tkzTabLine{,+,0,-,0,+,}
            \tkzTabVar{-/$1$,+/$4$,-/$-5$,+/$+\infty$}
        \end{tikzpicture}
    \end{center}
    \choice
    {$1$}
    {$3$}
    {\True $2$}
    {$4$}
    \loigiai{
        Dựa vào bảng biến thiên suy ra
        $f(x)-4=0 \Leftrightarrow f(x) =4$, phương trình này có $1$ nghiệm khác $-1$ và một nghiệm bội chẵn $x=-1$ nên đồ thị hàm số $y=\dfrac{x+1}{f(x)-4}$ có hai tiệm cận đứng.
    }
\end{ex}
\begin{ex}%[2D1K4-1]
    Cho hàm số $y=f(x)$ có bảng biến thiên như hình bên. Đồ thị hàm số $y=\dfrac{x-5}{f(x)-1}$ có bao nhiêu tiệm cận đứng?
    \begin{center}
        \begin{tikzpicture}
            \tkzTabInit[espcl=3]{$x$ / 1 , $f’(x)$ / 1, $f(x)$ / 2}
            {$-\infty$, $-1$ , $2$, $+\infty$}%
            \tkzTabLine{,-,0,+,d,-,}%
            \tkzTabVar{+/ $+\infty$, - / $-1$, + / $3$,-/$-\infty$}%
            \tkzTabVal[draw]{3}{4}{0.4}{$5$}{$1$}%
            %\tkzTabVal[draw]{2}{3}{0.4}{$e^2$}{$1$}%
        \end{tikzpicture}
    \end{center}
    \choice
    {$1$}
    {$3$}
    {\True $2$}
    {$4$}
    \loigiai{
        Dựa vào bảng biến thiên suy ra
        $f(x)-1=0 \Leftrightarrow f(x) =1$, phương trình này có $2$ nghiệm phân biệt khác $5$ và một nghiệm $x=5$ nên đồ thị hàm số $y=\dfrac{x-5}{f(x)-1}$ có hai tiệm cận đứng.
    }
\end{ex}
\begin{ex}%[VDC5-NgocDungHo]%[2D1G4-3]%
    \immini
    {
        Cho hàm số $f(x)$ có đồ thị như hình bên. Số đường tiệm cận đứng của đồ thị hàm số $y=\dfrac{(x^2-1)(x^2+x)}{[f(x)]^2-2f(x)-3}$ là
        \choice
        {$4$}
        {$5$}
        {\True $3$}
        {$2$}
    }
    {
        \begin{tikzpicture}[>=stealth,scale=0.7, line join=round, line cap=round]
            \def\f[#1]{(#1)^3-3*(#1)+1)}
            \draw[->] (-2.2,0)--(2.4,0) node [below]{$x$};
            \draw[->] (0,-1.5)--(0,3.5) node [left]{$y$};
            \node at (0,0) [below left]{$O$};
            % \clip;
            \draw[domain=-2.1:2.1,samples=300,thick] plot (\x,{\f[\x]});
            \filldraw (-1,0) node[below]{$-1$} circle (2pt);
            \filldraw (1,0) node[above]{$1$} circle (2pt);
            \filldraw (0,-1) node[ left]{$1$} circle (2pt);
            \filldraw (0,3) node[ right]{$3$} circle (2pt);
            \draw[dashed](-1,0)--(-1,3)--(0,3) (1,0)--(1,-1)--(0,-1);
            \draw (2,3) node[right]{$y=f(x)$};
        \end{tikzpicture}
    }
    \loigiai{
        Xét hàm số $y=g(x)=\dfrac{(x^2-1)(x^2+x)}{[f(x)]^2-2f(x)-3}$.
        \immini
        {
            Giải phương trình $(x^2-1)(x^2+x)=0 \Leftrightarrow \hoac{& x^2-1=0 \\ & x^2+x=0}\Leftrightarrow \hoac{& x=\pm 1 \\ & x=0.}$\\
            Giải phương trình $[f(x)]^2-2f(x)-3=0$\\$ \Leftrightarrow \hoac{& f(x)=-1 \\ & f(x)=3} \Leftrightarrow \hoac{& x = \pm 1 \\ & x=a\\&x=b\;(a<-1<1<b).}$
        }
        {
            \begin{tikzpicture}[>=stealth,scale=0.7, line join=round, line cap=round]
                \def\f[#1]{(#1)^3-3*(#1)+1)}
                \def\g[#1]{3}
                \def\h[#1]{-1}
                \draw[->] (-2.5,0)--(4,0) node [below]{$x$};
                \draw[->] (0,-1.5)--(0,3.5) node [left]{$y$};
                \node at (0,0) [below left]{$O$};
                % \clip;
                \draw[domain=-2.5:4,samples=300,thick] plot (\x,{\g[\x]});
                \draw[domain=-2.5:4,samples=300,thick] plot (\x,{\h[\x]});
                \draw[domain=-2.1:2.1,samples=300,thick] plot (\x,{\f[\x]});
                \filldraw (-1,0) node[below]{$-1$} circle (2pt);
                \filldraw (1,0) node[above]{$1$} circle (2pt);
                \filldraw (0,-1) node[ left]{$1$} circle (2pt);
                \filldraw (0,3) node[ right]{$3$} circle (2pt);
                \draw[dashed](-1,0)--(-1,3)--(0,3) (1,0)--(1,-1) (2,0)node[below]{$b$}--(2,3) (-2,0)node[above]{$a$}--(-2,-1);
                \draw (2,2) node[right]{$y=f(x)$};
                \draw (3.3,3) node[above]{$d_1:y=3$};
                \draw (3,-1) node[below]{$d_2:y=-1$};
            \end{tikzpicture}
        }
        Trong điều kiện xác định của hàm số $y=g(x)$ ta có thể viết
        $$y=g(x)=\dfrac{x(x-1)(x+1)^2}{(x-a)(x-b)(x-1)^2(x+1)^2}=\dfrac{x}{(x-a)(x-b)(x-1)}$$
        Vậy số tiệm cận đứng của đồ thị hàm số $y=g(x)$ bằng $3$.
    }
\end{ex}
\begin{ex}
    \immini{ %Câu 91.
        Đường cong ở hình bên là đồ thị của hàm số $y = ax^4 + bx^2 +c$. Đồ thị hàm số $g(x) =\dfrac{(x^2-x)\sqrt{x+2}}{(x-2)\cdot f(x+1)}$
        có bao nhiêu đường tiệm cận đứng?
        \choice
        {1}
        {3}
        {4}
        {2}}{
        \begin{tikzpicture}[scale=.8, font=\footnotesize, line join=round, line cap=round, >=stealth]
            \def\xmin{-2}\def\xmax{2}\def\ymin{-3}\def\ymax{1}
            \draw[->] (\xmin-0.2,0)--(\xmax+0.2,0) node[below] {\footnotesize $x$};
            \draw[->] (0,\ymin-0.2)--(0,\ymax+0.2) node[right] {\footnotesize $y$};
            \draw (0,0) node [below left] {\footnotesize $O$};
            \foreach \x in {-1,1}\draw (\x,-0.1)--(\x,0.1) node [above left] {\footnotesize $\x$};
            \foreach \y in {-2}\draw (0.1,\y)--(-0.1,\y) node [ below left] {\footnotesize $\y$};
            \clip (\xmin,\ymin) rectangle (\xmax,\ymax);
            \draw[smooth,samples=200,domain=\xmin:\xmax] plot (\x,{((\x)^4)+((\x)^2)+-2});
        \end{tikzpicture}
    }
    \loigiai{
        * Điều kiện: $\heva{&x \ne 2\\&f(x+1) \ne 0\\&x \ge -2.}$\\
        Nhìn hình vẽ ta thấy
        $f(x+1)=0\Leftrightarrow \hoac{&x+1=-1\\&x+1=1}\Leftrightarrow \hoac{&x=-2&(\text{nghiệm đơn})\\&x=0&(\text{nghiệm đơn}).}$\\
        Vậy $g(x) = \dfrac{(x^2-x)\sqrt{x+2}}{(x-2)\cdot ax^2(x^2+2) }=\dfrac{(x-1)\sqrt{x+2}}{(x-2)\cdot ax(x^2+2)}.$ \\
        Đồ thị hàm số $g(x)$ có 2 đường tiệm cận đứng.}
\end{ex}
\begin{ex}%[Thi thử THPT Yên Phong 1 - Bắc Ninh, 2021]%[Duong Xuan Loi,12EX 6- 2021]%[2D1G4-3]%
    \immini{
        Cho hàm số $y=f(x)$ có đồ thị như hình vẽ. Biết $f'(x)<0$, $\forall x <-1$ và $f'(x)>0$, $\forall x>1$. Khi đó, tổng số tiệm cận của đồ thị hàm số $y=\dfrac{2024}{\sqrt{xf(x+1)}[xf(x+1)+1]-2}$ là
        \choice
        {$1$}
        {$3$}
        {$4$}
        {\True $2$}
    }{
        \begin{tikzpicture}[scale=0.7, font=\footnotesize, line join=round, line cap=round,>=stealth]
            \def\xmin{-2} \def\xmax{2}
            \def\ymin{-2} \def\ymax{3.3}
            \draw[color=gray!50,dashed] (\xmin,\ymin) grid (\xmax,\ymax);
            \draw[->] (\xmin,0)--(\xmax,0) node [below]{$x$};
            \draw[->] (0,\ymin)--(0,\ymax) node [left]{$y$};
            \node at (0,0) [above right]{$O$};
            \clip (\xmin+0.1,\ymin+0.1) rectangle (\xmax-0.1,\ymax-0.1);
            \draw[smooth,samples=300,domain=-1.8:1.4] plot(\x,{(\x+1)*(\x+1)*(\x)*(\x-1)});
            \fill (-1,0) circle (1.0pt) node[below]{$-1$} (1,0) circle (1.0pt) node[below right]{$1$};
        \end{tikzpicture}
    }
    \loigiai{
        Xét phương trình $\sqrt{xf(x+1)}[xf(x+1)+1]-2=0.\quad(1)$\\
        Đặt $t=\sqrt{xf(x+1)}(t\geq 0)$, ta được phương trình $t\left(t^2+1\right)=2\Leftrightarrow t^3+t-2=0\Leftrightarrow t=1$.\\
        Với $t=1\Rightarrow\sqrt{xf(x+1)}=1\Leftrightarrow xf(x+1)-1=0$.\\
        Đặt $u=x+1\Rightarrow x=u-1$, ta được phương trình $(u-1)f(u)-1=0\Leftrightarrow f(u)=\dfrac{1}{u-1}.\quad(2)$
        \begin{center}
            \begin{tikzpicture}[scale=1, font=\footnotesize, line join=round, line cap=round,>=stealth]
                \def\a{0} \def\b{1} \def\c{1} \def\d{-1} % Hệ số
                \def\xmin{-3} \def\xmax{3.5}
                \def\ymin{-2.8} \def\ymax{3.3}
                \draw[color=gray!50,dashed] (\xmin,\ymin) grid (\xmax,\ymax);
                \draw[->] (\xmin,0)--(\xmax,0) node [below]{$u$};
                \draw[->] (0,\ymin)--(0,\ymax) node [left]{$y$};
                \node at (0,0) [above right]{$O$};
                \clip (\xmin+0.1,\ymin+0.1) rectangle (\xmax-0.1,\ymax-0.1);
                \draw[smooth,samples=300,domain=-1.8:1.4] plot(\x,{(\x+1)*(\x+1)*(\x)*(\x-1)});
                \draw[smooth,samples=300,domain=\xmin:(-\d/\c-0.1)] plot(\x,{(\a*(\x)+\b)/(\c*(\x)+\d)});
                \draw[smooth,samples=300,domain=(-\d/\c+0.1:\xmax)] plot(\x,{(\a*(\x)+\b)/(\c*(\x)+\d)});
                \fill (-1,0) circle (1.0pt) node[below]{$-1$} (1,0) circle (1.0pt) node[below right]{$1$};
            \end{tikzpicture}
        \end{center}
        Nhận thấy đồ thị của các hàm số $y=f(u)$, $y=\dfrac{1}{u-1}$ chỉ cắt nhau tại $1$ điểm do đó phương trình $(2)$ có nghiệm duy nhất $\Rightarrow(1)$ có nghiệm duy nhất, suy ra đồ thị có $1$ tiệm cận đứng.\\
        Mặt khác: $\lim\limits_{x\to+\infty} f(x+1)=+\infty\Rightarrow\lim\limits_{x\to+\infty}\dfrac{2021}{\sqrt{xf(x+1)}[xf(x+1)+1]-2}=a>0$.\\
        $\lim\limits_{x\to-\infty} xf(x+1)=-\infty\Rightarrow\lim\limits_{x\to+\infty}\dfrac{2021}{\sqrt{xf(x+1)}[xf(x+1)+1]-2}$ không tồn tại.\\
        Do đó đường thẳng $y=a$ là tiệm cận ngang.}
\end{ex}
\BTTF
\begin{ex}%[EX-TF-2024, Lê Đạt]%[2D1N4-1]
    Cho hàm số $y=f(x)$ có bảng biến thiên như sau
    \begin{center}
        \begin{tikzpicture}[>=stealth]
            \tkzTabInit[nocadre=false,lgt=1,espcl=3,deltacl=0.6]
            {$x$/.7 ,$y'$/.7,$y$/2}
            {$-\infty$ , $-2$ , $0$, $+\infty$}
            \tkzTabLine{ , - , d , + , d , -, }
            \tkzTabVar{+/$+\infty$ , -D-/$1$/$-\infty$ , +D+/$+\infty$ /$1$, -/$0$}
        \end{tikzpicture}
    \end{center}
    Xét tính đúng sai của các khẳng định sau
    \choiceTF
    {\True $ x=0 $ là tiệm cận đứng của đồ thị hàm số $ y=f(x) $}
    {\True $ x=-2 $ là tiệm cận đứng của đồ thị hàm số $ y=f(x) $}
    {$ x=1 $ là tiệm cận đứng của đồ thị hàm số $ y=f(x) $}
    {\True $ y=0 $ là tiệm cận ngang của đồ thị hàm số $ y=f(x) $}
    \loigiai{
        \begin{itemchoice}
            \itemch $\lim \limits_{x \to 0^-} f(x)=+\infty\Rightarrow x=0$ là đường tiệm cận đứng của đồ thị hàm số $f(x)$.
            \itemch $\lim \limits_{x \to (-2)^+} f(x)=-\infty\Rightarrow x=-2$ là đường tiệm cận đứng của đồ thị hàm số $f(x)$.
            \itemch Đồ thị hàm số chỉ có hai tiệm cận đứng là $ x=0 $ và $ x=-2 $.
            \itemch $\lim \limits_{x \to +\infty} f(x)=0\Rightarrow y=0$ là đường tiệm cận ngang của đồ thị hàm số $f(x)$.
        \end{itemchoice}
    }
\end{ex}
\begin{ex}%[EX-TF-2024, Lê Đạt]%[2D1H4-2]
    Cho hàm số $y=\dfrac{m x^{2}+6 x-2}{x+2}$. Xét tính đúng sai của các khẳng định sau
    \choiceTF
    {Đồ thị hàm số luôn có tiệm cận đứng với mọi $ m $}
    {Đồ thị hàm số không có tiệm cận ngang với mọi $ m $}
    {\True Khi $ m=1 $ đồ thị hàm số có một tiệm cận xiên là $ y=x+4 $ }
    {Đồ thị hàm số luôn có tiệm cận xiên}
    \loigiai{
        \begin{itemchoice}
            \itemch Khi $ m=\dfrac{7}{2} $ hàm số trở thành $y=\dfrac{\dfrac{7}{2} x^{2}+6 x-2}{x+2}=\dfrac{7}{2}\left(x-\dfrac{2}{7} \right) $ suy ra đồ thị hàm số không có tiệm cận đứng.
            \itemch Khi $ m=0 $ hàm số trở thành $ y=\dfrac{6x-2}{x+2} $ từ đó suy ra đồ thị hàm số có $ y=6 $ là tiệm cận ngang.
            \itemch Khi $ m=1 $ hàm số trở thành $ y=\dfrac{x^2+6x-2}{x+2}=x+4-\dfrac{10}{x+2} $ từ đó suy ra $ y=x+4 $ là một tiệm cận ngang.
            \itemch Khi $ m=0 $ hàm số trở thành $ y=\dfrac{6x-2}{x+2} $ từ đó suy ra đồ thị hàm số có $ y=6 $ là tiệm cận ngang, $ x=-2 $ là tiệm cận đứng và không có tiệm cận xiên.
        \end{itemchoice}
    }
\end{ex}
\begin{ex}%[EX-TF-2024, Lê Đạt]%[2D1H4-3]
    \immini{Cho hàm số $y=f(x)$ có đồ thị như hình bên. Xét tính đúng sai của các khẳng định sau
        \choiceTF
        {\True $ x=0 $ là một đường tiệm cận đứng của đồ thị hàm số}
        {$ y=-x $ là một đường tiệm cận xiên của đồ thị hàm số}
        {\True $ y=x $ là một đường tiệm cận xiên của đồ thị hàm số}
        {Đồ thị hàm số có ba đường tiệm cận}
    }{
        \begin{tikzpicture}[scale=.9, font=\footnotesize, line join=round, line cap=round,>=stealth]
            \def\a{0} \def\b{1} \def\c{1} \def\d{-1} % Hệ số
            \def\xmin{-3} \def\xmax{3.5}
            \def\ymin{-2.8} \def\ymax{3.3}
            \draw[color=gray!50,dashed] (\xmin,\ymin) grid (\xmax,\ymax);
            \draw[->] (\xmin,0)--(\xmax,0) node [below]{$x$};
            \draw[->] (0,\ymin)--(0,\ymax) node [left]{$y$};
            \fill (0,0) circle(1pt) node[shift=(-45:0.25)]{$O$};
            \clip (\xmin+0.1,\ymin+0.1) rectangle (\xmax-0.1,\ymax-0.1);
            \draw[smooth,samples=300,domain=-3:3] plot(\x,{\x+1/(7*\x)});
            \draw[dashed,smooth,samples=300,domain=-3:3] plot(\x,{\x});
            %	\fill (-1,0) circle (1.0pt) node[below]{$-1$} (1,0) circle (1.0pt) node[below right]{$1$};
    \end{tikzpicture}}
    \loigiai{
        \begin{itemchoice}
            \itemch $ x=0 $ là một đường tiệm cận đứng của đồ thị hàm số.
            \itemch	$ y=x $ là một đường tiệm cận xiên của đồ thị hàm số.
            \itemch $ y=x $ là một đường tiệm cận xiên của đồ thị hàm số.
            \itemch Đồ thị hàm số có $ x=0 $ là tiệm cận đứng và $ y=x $ là tiệm cận xiên nên có hai tiệm cận.
        \end{itemchoice}
    }
\end{ex}
\begin{ex}
    \immini
    {
        Cho hàm số $y=f(x)$ có đạo hàm liên tục trên $R$. Hàm số $y=f^{\prime}(x)$ có đồ thị như hình bên. Xác định tính đúng, sai của các mệnh đề sau
        \choiceTF
        {Hàm số $y=f(x)$ có hai cực trị}
        {Hàm số $y=f(x)$ đồng biến trên khoảng $(1 ;+\infty)$}
        {\True $f(1)>f(2)>f(4)$.}
        {\True Trên đoạn $[-1 ; 4]$, giá trị lớn nhất của hàm số $y=f(x)$ là $f(1)$.}
    }
    {
        \begin{tikzpicture}[line join=round, line cap=round,>=stealth,font=\scriptsize]
            \begin{scope}[scale=0.5]
                \tikzset{label style/.style={font=\footnotesize}}
                \def \xmin{-2}
                \def \xmax{4.5}
                \def \ymin{-2}
                \def \ymax{3.5}
                \def \hamso{0.55*(\x)^3-1.76*(\x)^2-0.31*(\x)+2}
                \draw[->] (\xmin,0)--(\xmax,0) node[below left] {$x$};
                \draw[->] (0,\ymin)--(0,\ymax) node[below left] {$y$};
                \draw (0,0) node [below left] {$O$};
                \begin{scope}
                    \clip (\xmin+0.01,\ymin+0.01) rectangle (\xmax-0.01,\ymax-0.01);
                    \draw[samples=350,domain=-1.3:3.3,smooth,variable=\x] plot (\x,{\hamso});
                \end{scope}
                \draw (-1,0) node[below left]{$-1$} (1,0) node[below]{$1$} (3,0) node[below right]{$4$} (0,2) node[above left]{$2$};
            \end{scope}
        \end{tikzpicture}
    }
    \loigiai{}
\end{ex}
\begin{ex}
    \immini{Cho hàm số $y=f(x)$ liên tục trên đoạn $\left[0 ; \frac{7}{2}\right]$ có đồ thị hàm số $y=f^{\prime}(x)$ như hình vẽ.
        \choiceTF
        {\True Hàm số $y=f(x)$ đồng biến trên khoảng $\left(3 ; \frac{7}{2}\right)$}
        {\True $f(0)>f(3)$}
        {$f(3)>f\left(\frac{7}{2}\right)$}
        {Hàm số $y=f(x)$ đạt giá trị nhỏ nhất trên đoạn $\left[0 ; \frac{7}{2}\right]$ tại điểm $x_0=\frac{7}{2}$}
    }{\begin{tikzpicture}[>=stealth, samples=100,smooth,y=.7cm,font=\scriptsize]
            \begin{scope}[scale=.7]
                \draw[->] (-1,0)--(4.5,0) node[below] {$x$};
                \draw[->] (0,-2)--(0,4) node[right] {$y$};
                \draw (0,0) node [below left] {$O$};
                \draw[dashed] (3.6,0)--(3.6,4);
                \draw[samples=200,domain=0.2:3.6,smooth,variable=\x]
                plot (\x,{1.06*(\x)^3-5.3*(\x)^2+7.23*(\x)-3});
                \path
                (3.6,0)node[below]{$\dfrac{7}{2}$}
                (3,0)node[above left]{$3$}
                (1,0)node[above]{$1$}
                ;
            \end{scope}
    \end{tikzpicture}}
    \loigiai{}
\end{ex}
\BTTL
\begin{ex}%[2D1K4-2]%
    Đồ thị hàm số $y=\dfrac{(2m+1)x+3}{x+1}$ có đường đường tiệm cận đi qua điểm $A(-2;7)$ khi và chỉ khi
    \shortans{$m=3$}
    %	\choice
    %	{\True $m=3$}
    %	{$m=1$}
    %	{$m=-1$}
    %	{$m=-3$}
    \loigiai{
        Từ đề bài, suy ra $2m+1=7 \Leftrightarrow m=3$.\\
        Suy ra $m+n=0$.
    }
\end{ex}
\begin{ex}%[Học kì 1, THPT Nguyễn Thi Minh Khai - Hà Nội, 2020-2021]%[Bùi Mạnh Tiến, 12EX5]%[2D1K4-2]%
    Cho hàm số $ y=\dfrac{2mx+m}{x-1}$. Với giá trị nào của tham số $m$ thì đường tiệm cận đứng, tiệm cận ngang của đồ thị hàm số cùng hai trục tọa độ tạo thành một hình chữ nhật có diện tích bằng $8$?
    \shortans{$m=\pm 4$}
    %	\choice
    %	{$ m=2$}
    %	{ $m=\pm 2$}
    %	{\True $m=\pm 4$}
    %	{$ m=\pm\dfrac{1}{2}$}
    \loigiai{
        Hàm số $y=\dfrac{2mx+m}{x-1}$ có $a=2m$, $b=m$, $c=1$, $d=-1$.\\
        Tiệm cận ngang $y=\dfrac{a}{c}=2m$.\\
        Tiệm cận đứng $x=-\dfrac{d}{c}=1$.\\
        Diện tích hình chữ nhật tạo thành bởi hai đường tiệm cận và hai trục tọa độ có diện tích
        \begin{align*}
            |2m|\cdot 1=8\Leftrightarrow m=\pm 4.
        \end{align*}
    }
\end{ex}
\begin{ex}%[KSCL lần 1, Liễn Sơn - Vĩnh Phúc, 2021]%[Phạm Doãn Lê Bình, 12EX4-2021]%[2D1K4-2]%
    Cho hàm số $y=\dfrac{x-\sqrt{x^2+2x}}{x^2+mx-m-3}$ có đồ thị $(C)$. Giá trị của $m$ để $(C)$ có đúng hai tiệm cận thuộc tập nào sau đây?
    \shortans{$(-5;2)$}
    %	\choice
    %	{$(-2;1)$}
    %	{$(1;5)$}
    %	{$(5;8)$}
    %	{\True $(-5;2)$}
    \loigiai{
        Điều kiện xác định của hàm số đã cho $\heva{& \hoac{ & x\ge 0 \\ & x\le -2}\\ & x^2+mx-m-3\ne 0.}$\\
        Ta có $\lim \limits_{x\to +\infty} y = \lim \limits_{x\to -\infty} y = 0$ nên $(C)$ có một tiệm cận ngang $y=0$.\\
        Xét phương trình $x^2+mx-m-3=0$.\hfill $(1)$\\
        Ta có
        \begin{itemize}
            \item $\Delta = m^2+4m+12>0$, $\forall m \in \mathbb{R}$.\\ Vậy phương trình $(1)$ luôn có hai nghiệm phân biệt $x_1,x_2$ ($x_1<x_2$).
            \item $x-\sqrt{x^2+2x}=0 \Leftrightarrow \heva{& x\ge 0 \\ & x^2=x^2+2x} \Leftrightarrow x=0$.
            \item Phương trình $(1)$ có nghiệm $x=0 \Leftrightarrow m=-3$. Với $m=-3$ ta có
            $ y =\dfrac{x-\sqrt{x^2+2x}}{x^2-3x}.$
            Khi đó
            \begin{eqnarray*}
                & \lim \limits_{x\to 0^+} y & =\lim \limits_{x\to 0^+} \dfrac{x-\sqrt{x^2+2x}}{x^2-3x}\\
                & & =\lim \limits_{x\to 0^+} \dfrac{-2x}{(x^2-3x)\left( x+\sqrt{x^2+2x}\right)}\\
                & & = \lim \limits_{x\to 0^+} \dfrac{-2}{(x-3)\left( x+\sqrt{x^2+2x}\right)}=+\infty
            \end{eqnarray*}
            và $\lim \limits_{x\to 3^+} y =-\infty$
            nên $(C)$ có thêm hai tiệm cận đứng $x=0$ và $x=3$ (không thỏa yêu cầu bài toán).
            \item Với $m\ne -3$ thì $(C)$ có đúng hai tiệm cận khi và chỉ khi $\hoac{& x_1<-2<x_2<0 &(2)\\ & -2<x_1<0<x_2. & (3)}$
            \item Đặt $f(x)=x^2+mx-m-3$. Ta có
            $(2)\Leftrightarrow \heva{& f(-2)< 0 \\ & f(0) >0 \\ & 0>-m} \Leftrightarrow \heva{& m>\dfrac{1}{3}\\ & m<-3 \\ & m>0}\Leftrightarrow m \in \varnothing.$
            \item $(3)\Leftrightarrow \heva{& f(-2)> 0 \\ & f(0) <0 \\ & -2<-m} \Leftrightarrow \heva{& m<\dfrac{1}{3}\\ & m>-3 \\ & m<2}\Leftrightarrow -3<m<\dfrac{1}{3}.$
        \end{itemize}
        Vậy $m\in \left(-3;\dfrac{1}{3}\right)$.
    }
\end{ex}
\begin{ex}%[kiểm tra GHK1, Sở GD và ĐT - Vĩnh Phúc, 2021]%[Huỳnh Xuân Tín, 12EX4]%[2D1K4-2]%
    Gọi $S$ là tập tất cả các giá trị của tham số $m$ để đồ thị hàm số	$y=\dfrac{x-3}{x^2-2x-m}$ có đúng một đường
    tiệm cận đứng. Tính tổng các phần tử của tập $S$.
    \shortans{$2$}
    %	\choice
    %	{$-1$}
    %	{\True $2$}
    %	{$-6$}
    %	{$1$}
    \loigiai{
        Để đồ thị hàm số	$y=\dfrac{x-3}{x^2-2x-m}$ có đúng một đường
        tiệm cận đứng, ta có hai trường hợp sau
        \begin{enumerate}[TH 1.]
            \item $x^2-2x-m=0$ có nghiệm kép $\Leftrightarrow \Delta'=1+m=0\Leftrightarrow m=-1$.
            \item $x^2-2x-m=0$ có hai nghiệm phân biệt trong đó có một nghiệm bằng $3$
            \[\Leftrightarrow \heva{&\Delta'=1+m>0\\& 3^2-6-m=0}\Leftrightarrow\heva{&m>-1\\&m=3}\Leftrightarrow m=3.\]
        \end{enumerate}
        Khi đó $S=\{-1;3\}$ và có tổng là $2$.
    }
\end{ex}
\begin{ex}
    Tốc độ phản ứng của enzyme theo nồng độ cơ chất \( S \) được mô tả bởi phương trình Michaelis-Menten: $v(S) = \dfrac{V_{\text{max}} S}{K_m + S}$,
    trong đó \( v(S) \) là tốc độ phản ứng, \( S \) là nồng độ cơ chất, \( V_{\text{max}} \) là tốc độ tối đa, và \( K_m \) là hằng số Michaelis. Xác định và nêu ý nghĩa của đường tiệm cận đứng của hàm số này.
    \shortans{không có TCĐ, tốc độ phản ứng không thể tới vô hạn}
    \loigiai{
        Để tìm tiệm cận đứng, ta xét các giá trị của \( S \) làm cho mẫu số của phương trình bằng 0:
        \[
        K_m + S = 0 \Rightarrow S = -K_m
        \]
        Vì nồng độ cơ chất \( S \) không thể âm, không có tiệm cận đứng trong trường hợp này.
        \textbf{Ý nghĩa:} Điều này có nghĩa là tốc độ phản ứng enzyme không có giá trị nào dẫn đến tốc độ phản ứng tiến đến vô hạn trong phạm vi các giá trị hợp lý của \( S \).}
\end{ex}
\begin{ex}
    \immini{Một ống khói của nhà máy điện hạt nhân có mặt cắt là một hypebol $(H)$ có phương trình chính tắc là $\dfrac{x^2}{27^2}-\dfrac{y^2}{40^2}=1$ (Hình $1.25$). Xét hai nhánh bên trên $Ox$ của $(H)$ là đồ thị $(C)$ của hàm số $y=\dfrac{40}{27}\sqrt{x^2-27^2}$ (phần nét liền đậm). Tìm tất cả các đường tiệm cận xiên của $(C)$.}{\begin{tikzpicture}[>=latex,line join=round, line cap=round, scale=.04, font=\footnotesize]
            \draw[->] (-90,0)--(90,0) node[above]{$x$};
            \draw[->] (0,-130)--(0,80) node[left]{$y$};
            \foreach \x in {-80,-60,-40,-20,20,40,60,80}
            \draw[fill=black] (\x,0) circle (15pt) node[below, fill=white]{$\x$};
            \foreach \y in {-120,-100,-80,-60,-40,-20,20,40,60}
            \draw[fill=black] (0,\y) circle (15pt) node[left]{$\y$};
            \clip (-90,-130) rectangle (90,80);
            \draw[samples=200,smooth,blue,line width=1] plot[domain=-90:-27] (\x,{40*sqrt((\x)^2-27^2)/27});
            \draw[samples=200,smooth,blue,line width=1] plot[domain=27:90] (\x,{40*sqrt((\x)^2-27^2)/27});
            \draw[samples=200,smooth,blue,line width=1, dashed] plot[domain=-90:-27] (\x,{-40*sqrt((\x)^2-27^2)/27});
            \draw[samples=200,smooth,blue,line width=1, dashed] plot[domain=27:90] (\x,{-40*sqrt((\x)^2-27^2)/27});
            \draw (0,0) node[above right]{$O$};
    \end{tikzpicture}}
    \shortans{$y=\pm \dfrac{40}{27}$}
    \loigiai{
        Ta có
        \allowdisplaybreaks
        \begin{eqnarray*}
            a&=&\lim\limits_{x\to+\infty}\dfrac{f(x)}{x}	 =\lim\limits_{x\to+\infty}\dfrac{\dfrac{40}{27}\sqrt{x^2-27^2}}{x}\\
            &=&\lim\limits_{x\to+\infty}\dfrac{\dfrac{40}{27}x\sqrt{1-\dfrac{27^2}{x^2}}}{x}
            =\lim\limits_{x\to+\infty}\dfrac{40}{27}\sqrt{1-\dfrac{27^2}{x^2}}
            =\dfrac{40}{27}.\\
            b&=&\lim\limits_{x\to+\infty}\left[f(x)-ax\right]
            =\lim\limits_{x\to+\infty}\left[\dfrac{40}{27}\sqrt{x^2-27^2}-\dfrac{40}{27}x\right]\\
            &=&\lim\limits_{x\to+\infty}\dfrac{40}{27}\left(\sqrt{x^2-27^2}-x\right)
            =\lim\limits_{x\to+\infty}\dfrac{40}{27}\cdot\dfrac{x^2-27^2-x^2}{\sqrt{x^2-27^2}+x}\\
            &=&\lim\limits_{x\to+\infty}\dfrac{40}{27}\cdot\dfrac{-27^2}{x\left(\sqrt{1-\dfrac{27}{x^2}}+1\right)}
            =0.
        \end{eqnarray*}
        Vậy đường thẳng $y=\dfrac{40}{27}x$ là một tiệm cận xiên của đồ thị.\\
        Tương tự, $\lim\limits_{x\to-\infty}\dfrac{f(x)}{x}=-\dfrac{40}{27}\Rightarrow a=-\dfrac{40}{27}$; $\lim\limits_{x\to-\infty} \left[f(x)-ax\right]=0\Rightarrow b=0$.\\
        Vậy đường thẳng $y=-\dfrac{40}{27}x$ là tiệm cận xiên của đồ thị.
    }
\end{ex}

\Closesolutionfile{ans}
% \begin{dang}{Tìm các đường tiệm cận đồ thị hàm ẩn}
\end{dang}
\begin{vd}
    Cho hàm số $y=f(x)$ có bảng biến thiên như hình vẽ sau
    \begin{center}
        \begin{tikzpicture}[>=stealth]
            \tkzTabInit[nocadre=false,lgt=1,espcl=1.5,deltacl=0.5]{$x$/.7 ,$y'$/.7,$y$/2}
            {$-\infty$ , $-1$ , $2$ , $+\infty$}
            \tkzTabLine{ , + , $0$ , - , d , + , }
            \tkzTabVar{-/$1$ , +/$4$ , -/$-5$ , +/$+\infty$}
        \end{tikzpicture}
    \end{center}
    Tìm TCĐ, TCN của đồ thị hàm số
    \begin{listEX}[3]
        \item $y=\dfrac{2}{f(x)-3}$
        \item $y=\dfrac{-3}{f(x)+2}$
        \item $y=\dfrac{x-2}{f(x)+5}$
        \item $y=\dfrac{x+1}{f(x)-4}$
        \item $y=\dfrac{2}{f(x^2)+3}$
        \item $y=\dfrac{4f(x)-5}{3f(x)+1}$
    \end{listEX}
    \loigiai{}
\end{vd}
\begin{vd}\immini{Cho hàm bậc ba $y=f(x)$ có đồ thị như hình vẽ. Tìm số tiệm cận đứng của đồ thị hàm số
        \begin{listEX}[2]
            \item $y=\dfrac{\sqrt{x+3}}{(x-1)f(x)}$
            \item $g(x)=\dfrac{(x^2+4x+3)\sqrt{x^2+x}}{x\left[f^2(x)-2f(x)\right]}$ .
    \end{listEX}}{\begin{tikzpicture}[line cap=round,line join=round, >=stealth,font=\footnotesize]
            \begin{scope}[scale=.5]
                \def\a{-1} % Hệ số a phải khác 0
                \def\b{-13/2}
                \def\c{-12}
                \def\d{-9/2}
                \draw[->] (-5,0) -- (2,0)node[below]{$x$};
                \draw[->] (0,-3) -- (0,4) node[left] {$y$};
                \draw (0,0)node[below right]{$O$} (-3,0)node[below]{$-3$};
                \draw[dashed] (-1,0)node[below]{$-1$}|-(0,2)node[right]{$2$};
                \draw[samples=150,smooth,domain=-4:.-.2] plot(\x,{\a*(\x)^3+(\b)*(\x)^2+(\c)*\x+(\d)});
            \end{scope}
    \end{tikzpicture}}
    \loigiai{
        \begin{center}
            \begin{tikzpicture}[line cap=round,line join=round, >=stealth,font=\footnotesize,scale=1]
                \def\a{-1} % Hệ số a phải khác 0
                \def\b{-13/2}
                \def\c{-12}
                \def\d{-9/2}
                \draw[->] (-5,0) -- (2,0)node[below]{$x$};
                \draw[->] (0,-3) -- (0,4) node[left] {$y$};
                \draw (0,0)node[below right]{$O$} (-3,0)node[below]{$-3$} (-.3,0)node[above]{$a$};
                \draw[dashed] (-3.78,0)node[below]{$c$}|-(0,2)|-(-1.71,0)node[below]{$b$}|-(0,2) (-1,0)node[below]{$-1$}|-(0,2)node[right]{$2$};
                \draw[samples=150,smooth,domain=-4:.-.2] plot(\x,{\a*(\x)^3+(\b)*(\x)^2+(\c)*\x+(\d)});
            \end{tikzpicture}
        \end{center}
        $g(x)=\dfrac{(x^2+4x+3)\sqrt{x^2+x}}{x\left[f^2(x)-2f(x)\right]}=\dfrac{(x+1)(x+3)\sqrt{x(x+1)}}{x\left[f^2(x)-2f(x)\right]}$.\\
        Điều kiện của căn là $x\le -1; x\ge 0$.\\
        Dựa vào đồ thị ta có \[x\left[f^2(x)-2f(x)\right]=0 \Leftrightarrow \hoac{&x=0\\&f(x)=0\\& f(x)=2} \Leftrightarrow \hoac{&x=0\text{ (nhận)}\\&x=-3\text{ (nhận)};\ x=a \text{ (loại)} \\&x=-1\text{ (nhận)};\ x=b\text{ (nhận)};\ x=c\text{ (nhận)}}\]\\
        Số TCĐ lúc này chính là số nghiệm không bị rút gọn của mẫu, vậy có bốn TCĐ là $x=0; x=-3; x=b; x=c$.
    }
\end{vd}
\BTTN
\Opensolutionfile{ans}[ans/2D1-4-DANG-3]
\begin{ex}%[2D1K4-1]
    Cho hàm số $y=f(x)$ có bảng biến thiên như hình bên. Đồ thị hàm số $y=\dfrac{-5}{f(x)+4}$ có bao nhiêu tiệm cận đứng?
    \begin{center}
        \begin{tikzpicture}[scale=0.8]
            \tkzTabInit[nocadre=false,lgt=1.5,espcl=3,deltacl=0.6]
            {$x$ /0.6,$y’$ /0.6,$y$ /2}
            {$-\infty$ ,$1$, $2$, $+\infty$}
            \tkzTabLine{,+,d,-,d,+,}
            \tkzTabVar{-/$-4$,+/$3$,-/$-5$,+/$+\infty$}
        \end{tikzpicture}
    \end{center}
    \choice
    {$1$}
    {$3$}
    {\True $2$}
    {$4$}
    \loigiai{
        Dựa vào bảng biến thiên suy ra
        $f(x)+4=0 \Leftrightarrow f(x) =-4$, phương trình này có $2$ nghiệm phân biệt nên đồ thị hàm số $y=\dfrac{-5}{f(x)+4}$ có $2$ tiệm cận đứng.
    }
\end{ex}
\begin{ex}%[2D1K4-1]
    Cho hàm số $y=f(x)$ có bảng biến thiên như hình bên. Đồ thị hàm số $y=\dfrac{x+2}{2f(x)-1}$ có bao nhiêu tiệm cận đứng?
    \begin{center}
        \begin{tikzpicture}[scale=0.8]
            \tkzTabInit[nocadre=false,lgt=1.5,espcl=3,deltacl=0.6]
            {$x$ /0.6,$y’$ /0.6,$y$ /2}
            {$-\infty$ ,$-1$, $0$, $1$, $+\infty$}
            \tkzTabLine{,+,0,-,0,+,0,-,}
            \tkzTabVar{-/$-\infty$,+/$0$,-/$-\dfrac{5}{3}$,+/$0$,-/$-\infty$}
        \end{tikzpicture}
    \end{center}
    \choice
    {$1$}
    {$3$}
    {$2$}
    {\True $0$}
    \loigiai{
        Dựa vào bảng biến thiên suy ra
        $2f(x)-1=0 \Leftrightarrow f(x) =\dfrac{1}{2}$, phương trình này có $0$ nghiệm nên đồ thị hàm số $y=\dfrac{x+2}{2f(x)-1}$ không có tiệm cận đứng.
    }
\end{ex}
%69
\begin{ex}%[2D1K4-1]
    Cho hàm số $y=f(x)$ có bảng biến thiên như hình bên. Đồ thị hàm số $y=\dfrac{1}{2f(x)-3}$ có bao nhiêu tiệm cận đứng?
    \begin{center}
        \begin{tikzpicture}[scale=0.8]
            \tkzTabInit[nocadre=false,lgt=1.5,espcl=3,deltacl=0.6]
            {$x$ /0.6,$y’$ /0.6,$y$ /2}
            {$-\infty$ ,$0$, $1$, $+\infty$}
            \tkzTabLine{,+,0,-,0,+,}
            \tkzTabVar{-/$-\infty$,+/$5$,-/$-1$,+/$+\infty$}
        \end{tikzpicture}
    \end{center}
    \choice
    {$1$}
    {\True $3$}
    {$2$}
    {$0$}
    \loigiai{
        Dựa vào bảng biến thiên suy ra
        $2f(x)-3=0 \Leftrightarrow f(x) =-\dfrac{3}{2}$, phương trình này có $3$ nghiệm phân biệt nên đồ thị hàm số $y=\dfrac{1}{2f(x)-3}$ có ba tiệm cận đứng.
    }
\end{ex}
%70
%71
%72
\begin{ex}%[2D1K4-1]
    Cho hàm số $y=f(x)$ có bảng biến thiên như hình bên. Đồ thị hàm số $y=\dfrac{x}{f(x)-3}$ có bao nhiêu tiệm cận đứng?
    \begin{center}
        \begin{tikzpicture}[scale=0.8]
            \tkzTabInit[nocadre=false,lgt=1.5,espcl=3,deltacl=0.6]
            {$x$ /0.6,$y’$ /0.6,$y$ /2}
            {$-\infty$ ,$-1$, $0$, $1$, $+\infty$}
            \tkzTabLine{,-,0,+,0,-,0,+,}
            \tkzTabVar{+/$+\infty$,-/$0$,+/$3$,-/$0$,+/$+\infty$}
        \end{tikzpicture}
    \end{center}
    \choice
    {$1$}
    {\True $3$}
    {$2$}
    {$4$}
    \loigiai{
        Dựa vào bảng biến thiên suy ra
        $f(x)-3=0 \Leftrightarrow f(x) =3$, phương trình này có $2$ nghiệm phân biệt khác $0$ và một nghiệm bội chẵn $x=0$ nên đồ thị hàm số $y=\dfrac{x}{f(x)-3}$ có ba tiệm cận đứng.
    }
\end{ex}
\begin{ex}%[2D1K4-1]
    Cho hàm số $y=f(x)$ có bảng biến thiên như hình bên. Đồ thị hàm số $y=\dfrac{4}{f(x)+1}$ có tiệm cận ngang là đường thẳng
    \begin{center}
        \begin{tikzpicture}[scale=0.8]
            \tkzTabInit[nocadre=false,lgt=1.5,espcl=3,deltacl=0.6]
            {$x$ /0.6,$y’$ /0.6,$y$ /2}
            {$-\infty$ ,$-1$, $2$, $+\infty$}
            \tkzTabLine{,+,0,-,0,+,}
            \tkzTabVar{-/$1$,+/$4$,-/$-5$,+/$1$}
        \end{tikzpicture}
    \end{center}
    \choice
    {$y=1$}
    {$y=-5$}
    {\True $y=2$}
    {$y=4$}
    \loigiai{
        Dựa vào bảng biến thiên suy ra
        $\lim \limits_{x \to \pm \infty} f(x)=1 \Leftrightarrow \lim \limits_{x \to \pm \infty} \dfrac{4}{f(x)+1} =2$ nên đồ thị hàm số đã cho có tiệm cận ngang là $y=2$.
    }
\end{ex}
\begin{ex}%[2D1K4-1]
    Cho hàm số $y=f(x)$ có bảng biến thiên như hình bên. Đồ thị hàm số $y=\dfrac{2-f(x)}{f(x)+3}$ có tiệm cận ngang là đường thẳng
    \begin{center}
        \begin{tikzpicture}[scale=0.8]
            \tkzTabInit[nocadre=false,lgt=1.5,espcl=3,deltacl=0.6]
            {$x$ /0.6,$y’$ /0.6,$y$ /2}
            {$-\infty$ ,$0$, $2$, $+\infty$}
            \tkzTabLine{,-,0,+,0,-,}
            \tkzTabVar{+/$+\infty$,-/$1$,+/$5$,-/$-\infty$}
        \end{tikzpicture}
    \end{center}
    \choice
    {$y=1$}
    {$y=-3$}
    {$y=2$}
    {\True $y=-1$}
    \loigiai{
        Dựa vào bảng biến thiên suy ra
        $\lim \limits_{x \to \pm \infty} f(x)=\pm \infty \Leftrightarrow \lim \limits_{x \to \pm \infty} \dfrac{2-f(x)}{f(x)+3} =-1$ nên đồ thị hàm số $y=\dfrac{2-f(x)}{f(x)+3}$ có tiệm cận ngang là $y=-1$.
    }
\end{ex}
\begin{ex}%[2D1K4-1]
    Cho hàm số $y=f(x)$ có bảng biến thiên như hình bên. Đồ thị hàm số $y=\dfrac{1}{f^2(x)-4f(x)+4}$ có bao nhiêu tiệm cận đứng?
    \begin{center}
        \begin{tikzpicture}[scale=0.8]
            \tkzTabInit[nocadre=false,lgt=1.5,espcl=3,deltacl=0.6]
            {$x$ /0.6,$y’$ /0.6,$y$ /2}
            {$-\infty$, $2$, $+\infty$}
            \tkzTabLine{,-,0,+,}
            \tkzTabVar{+/$1$,-/$-3$,+/$1$}
        \end{tikzpicture}
    \end{center}
    \choice
    {$1$}
    {$3$}
    {$2$}
    {$0$}
    \loigiai{
        Dựa vào bảng biến thiên suy ra $f^2(x)-4f(x)+4=0 \Leftrightarrow f(x)=2$, phương trình $f(x)=2$ vô nghiệm nên đồ thị hàm số đã cho không có tiệm cận đứng.
    }
\end{ex}
%83
\begin{ex}%[2D1K4-1]
    Cho hàm số $y=f(x)$ có bảng biến thiên như hình bên. Đồ thị hàm số $y=\dfrac{1}{f(3-x)-2}$ có bao nhiêu tiệm cận đứng?
    \begin{center}
        \begin{tikzpicture}[scale=0.8]
            \tkzTabInit[nocadre=false,lgt=1.5,espcl=3,deltacl=0.6]
            {$x$ /0.6,$y’$ /0.6,$y$ /2}
            {$-\infty$ ,$-2$, $2$, $+\infty$}
            \tkzTabLine{,+,0,-,0,+,}
            \tkzTabVar{-/$-\infty$,+/$3$,-/$0$,+/$+\infty$}
        \end{tikzpicture}
    \end{center}
    \choice
    {$1$}
    {\True $3$}
    {$2$}
    {$0$}
    \loigiai{
        Dựa vào bảng biến thiên suy ra $f(3-x)-2=0 \Leftrightarrow f(3-x)=2$, phương trình này có $3$ nghiệm phân biệt nên đồ thị hàm số đã cho có $3$ tiệm cận đứng.
    }
\end{ex}
\begin{ex}%[2D1G4-1]
    Cho hàm số $y=f(x)$ có bảng biến thiên như hình bên. Đồ thị hàm số $y=\dfrac{4}{f(x^2)-2}$ có bao nhiêu tiệm cận đứng?
    \begin{center}
        \begin{tikzpicture}[scale=0.8]
            \tkzTabInit[nocadre=false,lgt=1.5,espcl=3,deltacl=0.6]
            {$x$ /0.6,$y’$ /0.6,$y$ /2}
            {$-\infty$ ,$0$, $3$, $+\infty$}
            \tkzTabLine{,-,0,+,d,-,}
            \tkzTabVar{+/$8$,-/$1$,+/$4$,-/$2$}
        \end{tikzpicture}
    \end{center}
    \choice
    {$5$}
    {$3$}
    {\True $2$}
    {$4$}
    \loigiai{
        Dựa vào bảng biến thiên suy ra
        $f(x^2)-2=0 \Leftrightarrow f(x^2) =2$. Kẻ đường thẳng $y=2$ ta thấy đường thẳng cắt đồ thị hàm số tại hai điểm phân biệt. Suy ra
        $$\hoac{&x^2=a \; (a<0)\\&x^2=b \; (b >0)} \Rightarrow x=\pm \sqrt{b}.$$
        Do đó đồ thị hàm số đã cho có $2$ tiệm cận đứng.
    }
\end{ex}%89
\begin{ex}%[2D1G4-1]
    Cho hàm số $y=f(x)$ có bảng biến thiên như hình bên. Đồ thị hàm số $y=\dfrac{2}{f(|x|)-3}$ có bao nhiêu tiệm cận ngang?
    \begin{center}
        \begin{tikzpicture}[scale=0.8]
            \tkzTabInit[nocadre=false,lgt=1.5,espcl=3,deltacl=0.6]
            {$x$ /0.6,$y’$ /0.6,$y$ /2}
            {$-\infty$ ,$0$, $2$, $+\infty$}
            \tkzTabLine{,+,0,-,0,+,}
            \tkzTabVar{-/$-\infty$,+/$3$,-/$-1$,+/$+\infty$}
        \end{tikzpicture}
    \end{center}
    \choice
    {$4$}
    {\True $3$}
    {$5$}
    {$6$}
    \loigiai{
        Dựa vào bảng biến thiên suy ra
        $f(|x|)-3=0 \Leftrightarrow f(|x|) =3$.\\
        Bảng biến thiên hàm số $y=f(|x|)$ như sau
        \begin{center}
            \begin{tikzpicture}[scale=0.8]
                \tkzTabInit[nocadre=false,lgt=1.5,espcl=3,deltacl=0.6]
                {$x$ /0.6,$y’$ /0.6,$y$ /2}
                {$-\infty$ ,$-2$, $0$, $2$, $+\infty$}
                \tkzTabLine{,-,0,+,0,-,0,+,}
                \tkzTabVar{+/$+\infty$,-/$-1$,+/$3$,-/$-1$,+/$+\infty$}
            \end{tikzpicture}
        \end{center}
        Dựa vào bảng biến thiên hàm số $y=f(|x|)$, phương trình $f(|x|) =3$ có ba nghiệm phân biệt, do đó đồ thị hàm số $y=\dfrac{2}{f(|x|)-3}$ có $3$ tiệm cận đứng.
    }
\end{ex}
\begin{ex}
    \immini{ %Câu 90
        Cho hàm số bậc ba $f(x)= ax^3 +bx^2 +cx +d$ có đồ thị như hình vẽ bên. Đồ thị hàm số $g(x) = \dfrac{\sqrt{x+1}}{(x-3)\cdot f(x)}$ có bao nhiêu đường tiệm cận đứng?
        \choice
        {5}
        {2}
        {4}
        {\True 3}}{\begin{tikzpicture}[scale=.5, font=\footnotesize, line join=round, line cap=round, >=stealth]
            \def\xmin{-3}\def\xmax{3}\def\ymin{-5}\def\ymax{1}
            \draw[->] (\xmin-0.2,0)--(\xmax+0.2,0) node[below] {\footnotesize $x$};
            \draw[->] (0,\ymin-0.2)--(0,\ymax+0.2) node[right] {\footnotesize $y$};
            \draw (0,0) node [below left] {\footnotesize $O$};
            \foreach \x in {-1}\draw (\x,-0.1)--(\x,0.1) node [above] {\footnotesize $\x$};
            \foreach \x in {2}\draw (\x,-0.1)--(\x,0.1) node [above right] {\footnotesize $\x$};
            \foreach \y in {}\draw (-0.1,\y)--(0.1,\y) node [right] {\footnotesize $\y$};
            \clip (\xmin,\ymin) rectangle (\xmax,\ymax);
            \draw[smooth,samples=200,domain=\xmin:\xmax] plot (\x,{1*((\x)^3)+0*((\x)^2)+-3*(\x)+-2});
        \end{tikzpicture}
    }
    \loigiai{
        * Điều kiện: $\heva{&x \ne 3\\&f(x) \ne 0\\&x \ge -1.}$\\
        Nhìn hình vẽ ta thấy
        $f(x)=0\Leftrightarrow \hoac{&x=-1&(\text{nghiệm kép}) \\&x=2&(\text{nghiệm đơn}).}$\\
        Vậy $g(x) = \dfrac{\sqrt{x+1}}{(x-3)\cdot a(x+1)^2 (x-2)}.$ \\
        Đồ thị hàm số $g(x)$ có 3 đường tiệm cận đứng.}
\end{ex}
\begin{ex}
    \immini{ %Câu 92.
        Đường cong ở hình bên là đồ thị của hàm số $y = ax^3 +bx^2 +cx+d$. Đồ thị hàm số $y =\dfrac{(2x+1)\sqrt{x-1}}{x\cdot f(x-2)}$ có tất cả bao nhiêu tiệm cận đứng?
        \choice
        {1}
        {3}
        {4}
        {\True 2}}{\begin{tikzpicture}[scale=.6, font=\footnotesize, line join=round, line cap=round, >=stealth]
            \def\xmin{-3}\def\xmax{3}\def\ymin{-3}\def\ymax{3}
            \draw[->] (\xmin-0.2,0)--(\xmax+0.2,0) node[below] {\footnotesize $x$};
            \draw[->] (0,\ymin-0.2)--(0,\ymax+0.2) node[right] {\footnotesize $y$};
            \draw (0,0) node [below left] {\footnotesize $O$};
            \foreach \x in {-2}\draw (\x,-0.1)--(\x,0.1) node [above left] {\footnotesize $\x$};
            \foreach \x in {2}\draw (\x,-0.1)--(\x,0.1) node [above right] {\footnotesize $\x$};
            \foreach \y in {}\draw (-0.1,\y)--(0.1,\y) node [right] {\footnotesize $\y$};
            \clip (\xmin,\ymin) rectangle (\xmax,\ymax);
            \draw[smooth,samples=200,domain=\xmin:\xmax] plot (\x,{(2/3)*((\x)^3)+0*((\x)^2)+-(8/3)*(\x)});
    \end{tikzpicture}}
    \loigiai{
        * Điều kiện: $\heva{&x \ne 0\\&f(x-2) \ne 0\\&x \ge 1.}$\\
        Nhìn hình vẽ ta thấy
        $f(x-2)=0\Leftrightarrow \hoac{&x-2=-2\\&x-2=0\\&x-2=2}\Leftrightarrow \hoac{&x=0&(\text{không thỏa mãn})\\&x=2&(\text{nghiệm đơn})\\&x=4&(\text{nghiệm đơn}).}$\\
        Vậy $g(x) =\dfrac{(2x+1)\sqrt{x-1}}{x\cdot f(x-2)}=\dfrac{(x-1)\sqrt{x+2}}{x\cdot ax(x-2)(x-4)}.$ \\
        Đồ thị hàm số $g(x)$ có 2 đường tiệm cận đứng.}
\end{ex}
\begin{ex}
    \immini{ %Câu 93.
        Cho hàm số $y= f(x)$ có đồ thị cắt trục hoành tại đúng 3 điểm như hình bên. Đồ thị hàm số $y =\dfrac{(x+2)\sqrt{3-x}}{f(|x|)}$
        có tất cả bao nhiêu tiệm cận đứng?
        \choice
        {1}
        {3}
        {4}
        {\True 2}}{\begin{tikzpicture}[scale=.5, font=\footnotesize, line join=round, line cap=round, >=stealth]
            \def\xmin{-2}\def\xmax{5}\def\ymin{-3}\def\ymax{5}
            \draw[->] (\xmin-0.2,0)--(\xmax+0.2,0) node[below] {\footnotesize $x$};
            \draw[->] (0,\ymin-0.2)--(0,\ymax+0.2) node[right] {\footnotesize $y$};
            \draw (0,0) node [below left] {\footnotesize $O$};
            \foreach \x in {-1,2,4}\draw (\x,-0.1)--(\x,0.1) node [above left] {\footnotesize $\x$};
            \foreach \y in {}\draw (-0.1,\y)--(0.1,\y) node [right] {\footnotesize $\y$};
            \clip (\xmin,\ymin) rectangle (\xmax,\ymax);
            \draw[smooth,samples=200,domain=-1.2:0] plot(\x,{0-8.48*(\x)^(2.0)-5.48*(\x)+3.0});
            \draw[smooth,samples=200,domain=0:2]
            plot(\x,{0-2.7989489689153735*(\x)^(3.0)+8.326740175055514*(\x)^(2.0)-6.957684474449535*(\x)+3.0});
            \draw[smooth,samples=200,domain=2:5]
            plot(\x,{2.395330112721417*(\x)^(2.0)-14.371980676328501*(\x)+19.162640901771336});
    \end{tikzpicture}}
    \loigiai{
        * Điều kiện: $\heva{&f(|x|) \ne 0\\&x \le 3.}$\\
        Nhìn hình vẽ ta thấy
        $f(|x|)=0\Leftrightarrow \hoac{&|x|=-1\\&|x|=2\\&|x|=4}\Leftrightarrow \hoac{&x=\pm 2&(\text{nghiệm đơn})\\&x=- 4&(\text{nghiệm đơn})\\&x=4&(\text{không thỏa mãn}).}$\\
        Vậy $y =\dfrac{(x+2)\sqrt{3-x}}{a(x-2)(x+2)(x+4)(x-4)}$ \\
        Đồ thị hàm số có 2 đường tiệm cận đứng.}
\end{ex}
\begin{ex}
    \immini{ %Câu 94.
        Đường cong ở hình bên là đồ thị của hàm số $y = ax^3 +bx^2 +cx+d$. Đồ thị hàm số $y =\dfrac{(2x+1)\sqrt{1-x}}{f(|x|)}$ có tất cả bao nhiều tiệm cận đứng?
        \choice
        { 1}
        {3}
        {4}
        {\True 2}}{\begin{tikzpicture}[scale=.8, font=\footnotesize, line join=round, line cap=round, >=stealth]
            \def\xmin{-1}\def\xmax{2}\def\ymin{-1.5}\def\ymax{1.5}
            \draw[->] (\xmin-0.2,0)--(\xmax+0.2,0) node[below] {\footnotesize $x$};
            \draw[->] (0,\ymin-0.2)--(0,\ymax+0.2) node[right] {\footnotesize $y$};
            \draw (0.15,0) node [below left] {\footnotesize $O$};
            \foreach \x in {}\draw (\x,0.1)--(\x,-0.1) node [below] {\footnotesize $\x$};
            \foreach \y in {-1,1}\draw (0.1,\y)--(-0.1,\y) node [left] {\footnotesize $\y$};
            \clip (\xmin,\ymin) rectangle (\xmax,\ymax);
            \draw[smooth,samples=200,domain=\xmin:\xmax] plot (\x,{4*((\x)^3)+-6*((\x)^2)+0*(\x)+1});
            \draw[dashed] (0.5,0)--(0.5,0.0)--(0,0.0);
            \draw (0.5,-1pt)--(0.5,1pt) node [above] {\footnotesize $\frac{1}{2}$};
            \draw (-0.7,-1pt)--(-0.7,1pt) node [above] {\footnotesize $-\frac{1}{2}$};
            \draw (1,-1pt)--(1,1pt) node [above] {\footnotesize $1$};
            \draw[dashed] (0.0,0)--(0.0,1.0)--(0,1.0);
            \draw[dashed] (1.0,0)--(1.0,-1.0)--(0,-1.0);
    \end{tikzpicture}}
    \loigiai{
        * Điều kiện: $\heva{&f(|x|) \ne 0\\&x \le 1.}$\\
        Nhìn hình vẽ ta thấy
        $f(|x|)=0\Leftrightarrow \hoac{&|x|=-\dfrac{1}{2}\\&|x|=\dfrac{1}{2}\\&|x|=x_1>1}\Leftrightarrow \hoac{&x=\pm \dfrac{1}{2}&(\text{hai nghiệm đơn})\\&x=- x_1&(\text{nghiệm đơn})\\&x=x_1&(\text{không thỏa mãn}).}$\\
        Vậy $y =\dfrac{(2x+1)\sqrt{1-x}}{f(|x|)}=\dfrac{(2x+1)\sqrt{1-x}}{a\left(x-\dfrac{1}{2}\right)\left(x+\dfrac{1}{2}\right)(x+x_1)(x-x_1)}$ \\
        Đồ thị hàm số có 2 đường tiệm cận đứng.}
\end{ex}
\begin{ex}
    \immini{ %Câu 96.
        Cho đồ thị hàm số $y =f(x)$ và trục hoành có đúng 2 điểm chung như hình bên. Đồ thị hàm số $y =\dfrac{(x-1)\sqrt{3-x}}{f(x^2)}$ có tất cả bao nhiêu tiệm cận đứng?
        \choice
        {1}
        {3}
        {4}
        {\True 2}}{\begin{tikzpicture}[scale=.8, font=\footnotesize, line join=round, line cap=round, >=stealth]
            \def\xmin{-1.5}\def\xmax{2}\def\ymin{-1}\def\ymax{4.5}
            \draw[->] (\xmin-0.2,0)--(\xmax+0.2,0) node[below] {\footnotesize $x$};
            \draw[->] (0,\ymin-0.2)--(0,\ymax+0.2) node[right] {\footnotesize $y$};
            \draw (0,0) node [below left] {\footnotesize $O$};
            \foreach \x in {1}\draw (\x,0.1)--(\x,-0.1) node [below] {\footnotesize $\x$};
            \foreach \x in {-1}\draw (\x,0.1)--(\x,-0.1) node [below left] {\footnotesize $\x$};
            \clip (\xmin,\ymin) rectangle (\xmax,\ymax);
            \draw[smooth,samples=200,domain=-1.1:0] plot(\x,{21.044670464836045*(\x)^(3.0)+24.701786337609526*(\x)^(2.0)+5.65711587277348*(\x)+2.0});
            \draw[smooth,samples=200,domain=0:\xmax] plot(\x,{10.591704641658401*(\x)^(3.0)-19.26315454354621*(\x)^(2.0)+6.6714499018878115*(\x)+2.0});
    \end{tikzpicture}}
    \loigiai{
        * Điều kiện: $\heva{&f(x^2) \ne 0\\&x \le 3.}$\\
        Nhìn hình vẽ ta thấy
        $f(x^2)=0\Leftrightarrow \hoac{&x^2=-1\\&x^2=1}\Leftrightarrow x=\pm 1\,(\text{nghiệm kép}).$\\
        Vậy $y=\dfrac{(x-1)\sqrt{3-x}}{f(x^2)}=\dfrac{(x-1)\sqrt{3-x}}{(x-1)^2(x+1)^2}$ \\
        Đồ thị hàm số có 2 đường tiệm cận đứng.}
\end{ex}
\begin{ex}%[2D1G4-3]%Câu 52
    Cho hàm số $y=ax^3+bx^2+cx+d$ có đồ thị như hình vẽ. Đồ thị của hàm số $g(x)=\dfrac{x^2-x}{f^2(x)-2f(x)}$ có bao nhiêu đường tiệm cận đứng?
    \choice
    {$2$}
    {$3$}
    {\True $4$}
    {$5$}
    \begin{center}
        \begin{tikzpicture}[thick,>=stealth,x=1cm,y=1cm,scale=.7]
            \draw[thin,color=gray!50] (-3.3,-1.3) grid (3.9,5.9);
            \draw[->] (-3.2,0) -- (4.2,0) node[right] {$x$};
            \draw[->] (0,-1.2) -- (0,5.2) node[above] {$y$};
            \draw[color=blue, domain=-2.15:2.15,samples=300] plot (\x,{(\x)^3-3*(\x)+2}) node[right] {$y=f(x)$};
            \draw (-2,0) circle (1.5pt) node[below left]{$-2$};
            \draw (-1,0) circle (1.5pt) node[below]{$-1$};
            \draw (0,0) circle (1.5pt) node[above left]{$O$};
            \draw (1,0) circle (1.5pt) node[below]{$1$};
            \draw (0,4) circle (1.5pt) node[right]{$4$};
            \draw (-1,4) circle (1.5pt);
            \draw[dashed] (-1,0)--(-1,4)--(0,4);
            \draw[red] (-3,2)--(3.2,2);
            \draw[red] (3.5,2) node[right]{$f(x)=2$};
        \end{tikzpicture}
    \end{center}
    \loigiai{
        Xét phương trình $f^2(x)-2f(x)=0 \Leftrightarrow \hoac{&f(x)=0\\&f(x)=2}\Leftrightarrow \hoac{&x=1 \, (\textrm{nghiệm kép trùng nghiệm đơn ở tử số})\\&x=-2\, (\textrm{nghiệm đơn khác nghiệm của tử})\\&x=a\in(-2; -1)\\&x=0\, (\textrm{nghiệm đơn trùng nghiệm ở tử})\\&x=b\in(1; 2)}$\\
        \textbf{Kết luận:} Đồ thị hàm số có $4$ đường tiệm cận đứng.
    }
\end{ex}
\begin{ex}%[Thi thử L3, Lương Thế Vinh, Hà Nội, 2018]%[Phạm Toàn, Dự án (12EX-10)]%[2D1G4-3]%
    \immini{Cho hàm số $y=f(x)$ có đạo hàm liên tục trên $\mathbb{R}$. Đồ thị hàm $f(x)$ như hình vẽ. Số đường tiệm cận đứng của đồ thị hàm số $y=\dfrac{x^2-1}{f^2(x)-4f(x)}$ bằng
        \choice
        {$3$}
        {$1$}
        {$2$}
        {\True $4$}
    }{\begin{tikzpicture}[>=stealth,x=1cm,y=0.75cm,scale=0.7]
            \draw[->] (-2.5,0)--(0,0)%
            node[below right]{$O$}--(2.5,0) node[below]{$x$};
            \draw[->] (0,-2) --(0,5) node[right]{$y$};
            \foreach \x in {-1,1}{
                \draw (\x,0) node[below]{\footnotesize $\x$} circle (1pt);%Ox
            }
            \foreach \y in {2,4}{
                \draw (0,\y) node[right]{\footnotesize $\y$} circle (1pt);%Oy
            }
            \draw[samples=100,domain=-2.05:2] plot (\x,{(\x -1)^2*(\x+2)});
            \draw [dashed] (-1,0)--(-1,4)--(0,4);
            \draw(-1,4) circle (1pt);
    \end{tikzpicture}}
    \loigiai{Xét $f^2(x)-4f(x)=0\Leftrightarrow \hoac{& f(x)=0\\ &f(x)=4.}$\\
        Xét $f(x)=0$ có hai nghiệm, nghiệm $x_1\ne \pm 1$ và nghiệm $x_2=1$ là nghiệm bội (do đồ thị tiếp xúc với trục hoành tại $x=1$. Trường hợp này có $2$ tiệm cận đứng.\\
        Xét $f(x)=4$ có hai nghiệm, nghiệm $x_3\ne \pm 1$ và nghiệm $x_4=-1$ là nghiệm bội (do đồ thị tiếp xúc với đường thẳng $y=4$ tại $x=-1$. Trường hợp này có $2$ tiệm cận đứng.\\
        Vậy đồ thị có $4$ tiệm cận đứng.}
\end{ex}
\begin{ex}%[Thi thử, Trường THPT Lý Thái Tổ - Bắc Ninh, 2019]%[Duong Xuan Loi, 12EX3]%[2D1G4-3]%
    \immini{
        Cho hàm số $f(x)$ có đồ thị như hình bên. Số đường tiệm cận đứng của đồ thị hàm số
        $y=\dfrac{(x^2-4)(x^2+2x)}{[f(x)]^2+2f(x)-3}$ là
        \choice
        {\True $4$}
        {$5$}
        {$3$}
        {$2$}
    }{
        \begin{tikzpicture}[scale=0.5, font=\footnotesize, line join=round, line cap=round, >=stealth]
            \def\a{1} \def\b{-8} \def\c{1} % Hệ số
            \def\xt{-3.7} \def\xp{4} \def\yt{2} \def\yd{-3.7} % x_trái, x_phải, y_trên, y_dưới (giới hạn)
            \draw[->] (\xt,0)--(\xp,0) node [below]{$x$};
            \draw[->] (0,\yd)--(0,\yt) node [left]{$y$};
            \node at (0,0) [below left]{$O$};
            \clip (\xt-0.1,\yd+0.1) rectangle (\xp-0.1,\yt-0.1);
            \draw[smooth,samples=300] plot(\x,{1/4*(\a*(\x)^4+\b*(\x)^2)+\c});
            \draw[dashed] (-2,0)node[above]{$-2$}--(-2,-3)--(2,-3)--(2,0)node[above]{$2$};
            \node at (0,-3)[above left]{$-3$};
            \node at (-3,0)[above left]{$-3$};
            \node at (0,1)[above right]{$1$};
            \node at (3,0)[above right]{$3$};
            \fill (0,0) circle (1pt) (0,-3) circle (1pt) (2,0) circle (1pt) (-2,0) circle (1pt) (-3,0) circle (1pt) (0,1) circle (1pt) (3,0) circle (1pt);
        \end{tikzpicture}
    }
    \loigiai{
        Ta có $y=\dfrac{(x^2-4)(x^2+2x)}{[f(x)]^2+2f(x)-3}$ có các nghiệm ở tử là $x=0$ (bội $1$), $x=2$ (bội $1$), $x=-2$ (bội $2$).\\
        Mặt khác, từ đồ thị $f(x)$ ta thấy hàm số $y=\dfrac{(x^2-4)(x^2+2x)}{[f(x)]^2+2f(x)-3}$ có các nghiệm ở mẫu là
        $f^2(x)+2f(x)-3=0\Leftrightarrow \hoac{& f(x)=1 \\ & f(x)=-3}
        \Leftrightarrow \hoac{& x=0,x=x_1,x=x_2 \\ & x=-2,x=2.}$\\
        Trong đó nghiệm $x=0$, $x=-2$, $x=2$ đều có bội $2$ và $x_1$, $x_2$ khác các nghiệm của tử.\\
        So sánh bội nghiệm ở mẫu và bội nghiệm ở tử thì thấy đồ thị có các tiệm cận đứng là $x=0$, $x=2$; $x=x_1$; $x=x_2$.
    }
\end{ex}
\begin{ex}%[Thi thử, THPT Sơn Tây, Hà Nội, 2019]%[Huỳnh Xuân Tín, 12EX3]%[2D1G4-3]%
    \immini{Cho hàm số $ f(x)=(x+3)(x+1)^2(x-1)(x-3)$ có đồ thị như hình vẽ. Đồ thị hàm số $ g(x)=\dfrac{\sqrt{x-1}}{f^2(x)-9f(x)}$ có bao nhiêu tiệm cận đứng và tiệm cận ngang?
        \choice
        {$3$}
        {\True$ 4$}
        {$ 9$}
        { $8$}
    }{\begin{tikzpicture}[scale=0.3, font=\footnotesize, line join=round, line cap=round, >=stealth]
            %\draw[dashed, line width=0.1pt, gray] (-3.2,-5.5) grid (5.2,4.5);
            \draw[->] (-3.5,0)--(0,0) node[below right]{$O$}--(3.6,0) node[below]{$x$};
            \draw[fill=black] (0,0) circle (1pt);
            \draw[->] (0,-7.7) --(0,6.5) node[right]{$y$};
            \foreach \x in {-3,-1,3}{
                \draw[fill=black] (\x,0) node[below left]{$\x$} circle (1pt);}
            \draw[fill=black] (1,0) node[below right]{$1$} circle (1pt);
            \draw[fill=black] (0,1.35) node[above left]{$9$} circle (1pt);
            \draw [black, domain=-3.2:3.18, samples=100] %
            plot(\x,{0.15*(\x+3)*(\x+1)^2*(\x-1)*(\x-3)});
    \end{tikzpicture}}
    \loigiai{Điều kiện xác định của hàm số $g(x)$ là $\heva{&x\ge1\\ &f^2(x)-9f(x)\not=0.}$\\
        Từ $f^2(x)-9f(x)=0\Leftrightarrow \hoac{&f(x)=0\\&f(x)=9.}$\\
        Với $f(x)=0$ có nghiệm là $x=\pm 1, x=\pm 3$.\\
        Dựa vào đồ thị ta thấy nghiệm của phương trình $f(x)=9$ là hoành độ giao điểm của đường thẳng $y=9$ với đồ thị hàm số $y=f(x)$ nên có nghiệm là $-3<x_3<x_2<-1<0<x_1<1<3<x_0$.\\
        Do đó tập xác định của hàm số $y=g(x)$ là $\mathscr{D}=\left[1;+\infty \right)\setminus\left\lbrace1;3;x_0 \right\rbrace $.\\
        Khi đó ta có \begin{itemize}
            \item $\lim\limits_{x\rightarrow1^+ } g(x)=\lim\limits_{x\rightarrow1^+ }\dfrac{\sqrt{x-1}}{f(x)\left(f(x)-9 \right)}=+\infty$ (vì $x$ tiến gần bên phải $1$ thì $f(x)<0, f(x)-9<0$), suy ra đường thẳng $x=1$ là tiệm cận đứng.
            \item $\lim\limits_{x\rightarrow3^+ } g(x)=\lim\limits_{x\rightarrow3^+ }\dfrac{\sqrt{x-1}}{f(x)\left(f(x)-9 \right)}=-\infty$ (vì $x$ tiến gần bên phải $3$ thì $f(x)>0, f(x)-9<0$), suy ra đường thẳng $x=3$ là tiệm cận đứng.
            \item $\lim\limits_{x\rightarrow x_0^+} g(x)=\lim\limits_{x\rightarrow x_0^+ }\dfrac{\sqrt{x-1}}{f(x)\left(f(x)-9 \right)}=+\infty$ (vì $x$ tiến gần bên phải $x_0$ thì $f(x)>0, f(x)-9>0$), suy ra đường thẳng $x=x_0$ là tiệm cận đứng.
        \end{itemize}
        Và $\lim\limits_{x\rightarrow +\infty} g(x)=\lim\limits_{x\rightarrow +\infty }\dfrac{\sqrt{x-1}}{f(x)\left(f(x)-9 \right)}=0$ (vì bậc ở mẫu của $y=g(x)$ là $10$ và bậc tử của nó là $\dfrac{1}{2}$). Do vậy đồ thị hàm số $y=g(x)$ có một tiệm cận ngang là đường thẳng $y=0$.\\
        Vậy đồ thị hàm số $y=g(x)$ có bốn tiệm cận ngang và đứng. }
\end{ex}
\begin{ex}%[Thi thử, Chuyên Quang Trung-Bình Phước, 2021,lần 1]%[Trần Hòa, 12EX6]%[2D1G4-3]%
    \immini{Cho hàm số $y=f(x)=ax^3+bx^2+cx+d$, có đồ thị như hình vẽ. Số đường tiệm cận đứng của đồ thị hàm số $y=\dfrac{x^2+x-2}{f^2(x)-f(x)}$ là
        \choice
        {$3$}
        {$2$}
        {\True $4$}
        {$5$}}
    {\begin{tikzpicture}[scale=.5, font=\footnotesize, line join=round, line cap=round, >=stealth]
            \draw[->] (-2.5,0)--(0,0) node[below right]{$O$}--(2,0) node[below]{$x$};
            \draw[->] (0,-.5) --(0,4.5) node[right]{$y$};
            \draw [domain=-2.05:2.05, samples=100] %
            plot (\x, {(\x+2)*(\x-1)^2});
            \draw[fill] (0,0) circle (1pt);
            \foreach \x/\g in {-2/140,-1/-90,1/-90}
            \draw[fill] (\x,0) circle(.5pt)node [shift={(\g:.3)}] {$\x$};
            \foreach \y/\g in {2/0,4/0}
            \draw[fill] (0,\y) circle(.5pt)node [shift={(\g:.3)}] {$\y$};
            \draw[dashed] (-1,0)--(-1,4)--(0,4);
    \end{tikzpicture}}
    \loigiai{
        \begin{itemize}
            \item $x^2+x-2=(x-1)(x+2)$.\\
            \item Dựa vào đồ thị hàm số $y=f(x)$ ta có $f^2(x)-f(x)=0\Leftrightarrow\hoac{&f(x)=0\\&f(x)=1.}$\\
            $f(x)=0\Leftrightarrow x=-2$, $x=1$ (nghiệm kép).\\
            $f(x)=1\Leftrightarrow\hoac{&x=x_1,(x_1\in (-2;-1))\\&x=x_2,(x_2\in (0;1))\\&x=x_3,(x_3>1). }$
            \item Do đó $y=\dfrac{(x-1)(x+2)}{a^2(x+2)(x-1)^2(x-x_1)(x-x_2)(x-x_3)}$.
        \end{itemize}
        Suy ra đồ thị có các đườn tiệm cận đứng $x=1$, $x=x_1$, $x=x_2$, $x=x_3$.
    }
\end{ex}
\begin{ex}%[Đề thi hết học kì 2, Bình Minh, Ninh Bình 2018]%[Nguyễn Tuấn Anh, dự án EX9]%[2D1G4-3]%
    \immini{Cho hàm số bậc ba $f(x)=ax^3+bx^2+cx+d$ có đồ thị như hình vẽ bên dưới. Hỏi đồ thị hàm số $g(x)=\dfrac{(x^2-3x+2)\sqrt{x-1}}{x[f^2(x)-f(x)]}$ có bao nhiêu tiệm cận đứng?
        \choice
        {$5$}
        {$6$}
        {\True $3$}
        {$4$}
    }{
        \begin{tikzpicture}[line width=1.0pt,line join=round,>=stealth,x=1cm,y=1cm,scale=1.0]
            \draw[->,line width = 1pt] (-1,0)--(0,0) node[below right]{$O$}--(4,0) node[below]{$x$};
            \draw[->,line width = 1pt] (0,-1.5) --(0,2.5) node[right]{$y$};
            \foreach \x in {1,2}{
                \draw (\x,0) node[below]{$\x$} circle (1pt);
            }
            \foreach \y in {1}{
                \draw (0,\y) node[left]{$\y$} circle (1pt);
            }
            \clip(-0.8,-1) rectangle (3.8,2.3);
            \draw [line width=1.0pt, thick, domain=-0.5:3.5, samples=100]%,domain=-1.5:3] %
            plot (\x, {(5*(\x)-4)*((\x)-2)^2});
            \draw [dash pattern=on 4pt off 4pt] (1.,0.)-- (1.,1.)-- (0.,1.);
            \draw (1,1) circle (1pt);
        \end{tikzpicture}
    }
    \loigiai{
        Điều kiện $\heva{&x\geq 1\\ &x\ne 0\\ &f^2(x)-f(x)\ne 0}\Leftrightarrow \heva{&x\geq 1\\ &f(x)\ne 0\\ & f(x)\ne 1.}$\\
        Dựa vào đồ thị hàm số $y=f(x)$, ta thấy $f(x)=0$ có hai nghiệm, một nghiệm $x_1<1$ và một nghiệm kép bằng $2$. Do đó ta biểu diễn được $f(x)$ dưới dạng
        $$ f(x)=a(x-x_1)(x-2)^2. $$
        Dựa vào đồ thị hàm số $y=f(x)$, ta thấy phương trình $f(x)=1$ có ba nghiệm $1,x_2, x_3$, với $1<x_2<2<x_3$. Do đó ta biểu diễn được $f(x)-1$ dưới dạng
        $$ f(x)-1=a(x-1)(x-x_2)(x-x_3). $$
        Lúc này điều kiện được viết lại như sau $\heva{&x>1\\ &x\ne x_2, x\ne 2, x\ne x_3.}$\\
        Với điều kiện đó thì $g(x)$ được viết lại là
        $$ g(x)=\dfrac{\sqrt{x-1}}{a^2x(x-x_1)(x-x_2)(x-2)(x-x_3)}. $$
        Ta có
        \begin{align*}
            &\lim\limits_{x\to 1^+}g(x)=\lim\limits_{x\to 1^+}\dfrac{\sqrt{x-1}}{a^2x(x-x_1)(x-x_2)(x-2)(x-x_3)}=0,\\
            & (x=1\mbox{ \textbf{không} là tiệm cận đứng}) \\
            &\lim\limits_{x\to x_2^+}g(x)=\lim\limits_{x\to x_2^+}\dfrac{\sqrt{x-1}}{a^2x(x-x_1)(x-x_2)(x-2)(x-x_3)}=+\infty,\\
            & (x=x_2\mbox{ là tiệm cận đứng}) \\
            &\lim\limits_{x\to 2^+}g(x)=\lim\limits_{x\to 2^+}\dfrac{\sqrt{x-1}}{a^2x(x-x_1)(x-x_2)(x-2)(x-x_3)}=-\infty,\\
            & (x=2\mbox{ là tiệm cận đứng}) \\
            &\lim\limits_{x\to x_3^+}g(x)=\lim\limits_{x\to x_3^+}\dfrac{\sqrt{x-1}}{a^2x(x-x_1)(x-x_2)(x-2)(x-x_3)}=+\infty,\\
            & (x=x_3\mbox{ là tiệm cận đứng}) \\
        \end{align*}
        Vậy đồ thị hàm số $g(x)$ có tất cả $3$ tiệm cận đứng.
    }
\end{ex}
\begin{ex}%[VDC5-Đỗ Đường Hiếu]%[2D1G4-3]%
    \immini{Cho hàm số $f(x)=(x+3)(x+1)^2(x-1)(x-3)$ có đồ thị như hình vẽ. Đồ thị hàm số $g(x)=\dfrac{\sqrt{x-1}}{f^2(x)-9f(x)}$ có bao nhiêu tiệm cận đứng và tiệm cận ngang?
        \choice
        {$3$}
        {\True $4$}
        {$9$}
        {$8$}}
    {\begin{tikzpicture}[xscale=0.8,yscale=0.05, line join=round, line cap=round,font=\footnotesize,>=stealth]
            \draw[->] (-4,0)--(4,0) node[below]{$x$};
            \draw[->] (0,-56)--(0,30) node[left]{$y$};
            \coordinate[label=below left:$O$] (O) at (0,0);
            \draw (-1,0) node[below] { $-1$}(1,0) node[below] { $1$};
            \draw (-3,0) node[below left] { $-3$};
            \draw (3,0) node[below right] { $3$};
            \clip (-3.3,-60) rectangle (3.5,26);
            \draw[smooth,samples=300,domain=-3.5:3.5] plot(\x,{(\x+3)*(\x+1)^2*(\x-1)*(\x-3)});
            \foreach \x in {-3,-1,1,3}
            \draw[shift={(\x,0)},color=black] (0pt,20pt) -- (0pt,-20pt);
            \draw[shift={(0,9)},color=black] (2pt,0pt) -- (-2pt,0pt) node[left] {$9$};
        \end{tikzpicture}
    }
    \loigiai{%GV tổng quát hóa bài toán:
        Cho hàm số đa thức $y=f(x)$ có đồ thị $(C)$. Tìm số đường tiệm cận của đồ thị hàm số $g(x)=\dfrac{\sqrt{ax+b}}{P\left(f(x) \right) }$, trong đó $P\left(f(x) \right)$ là một đa thức của $f(x)$.
        Nếu $a>0$ thì $\lim\limits_{x\to +\infty}g(x)=0$.\\
        Nếu $a<0$ thì $\lim\limits_{x\to -\infty}g(x)=0$.\\
        Do đó đồ thị hàm số $y=g(x)$ luôn có duy nhất một đường tiệm cận ngang là $y=0$.\\
        Gọi $x=x_0$ là một nghiệm của phương trình $P\left(f(x) \right) =0$ thỏa mãn điều kiện $ax+b\ge 0$. Rõ ràng khi đó $\lim\limits_{x\to x_0^+}g(x)=+\infty$ hoặc $\lim\limits_{x\to x_0^+}g(x)=-\infty$.\\
        Bởi vậy, số đường tiệm cận đứng của đồ thị hàm số $y=g(x)$ chính là số nghiệm của phương trình $P\left(f(x) \right) =0$ thỏa mãn điều kiện $ax+b\ge 0$.
        \immini{Ta có $f^2(x)-9f(x)=0\Leftrightarrow \hoac{&f(x)=0\\&f(x)=9.}$\\
            \begin{itemize}
                \item $f(x)=0$ có các nghiệm thuộc $\left[1;+\infty\right)$ là $x=1$ và $x=3$.
                \item Đường thẳng $y=9$ cắt đồ thị hàm số $y=f(x)$ tại duy nhất một điểm có hoành độ thuộc $\left[1;+\infty\right)$ là $x=a>3$.
            \end{itemize}
        }
        {\begin{tikzpicture}[xscale=0.8,yscale=0.05, line join=round, line cap=round,font=\footnotesize,>=stealth]
                \draw[->] (-4,0)--(4,0) node[below]{$x$};
                \draw[->] (0,-56)--(0,30) node[left]{$y$};
                \coordinate[label=below left:$O$] (O) at (0,0);
                \draw (-4,9)--(4,9);
                \draw (-1,0) node[below] { $-1$}(1,0) node[below] { $1$};
                \draw (-3,0) node[below left] { $-3$};
                \draw (3,0) node[below right] { $3$};
                \clip (-3.3,-60) rectangle (3.5,26);
                \draw[smooth,samples=300,domain=-3.5:3.5] plot(\x,{(\x+3)*(\x+1)^2*(\x-1)*(\x-3)});
                \foreach \x in {-3,-1,1,3}
                \draw[shift={(\x,0)},color=black] (0pt,20pt) -- (0pt,-20pt);
                \draw[shift={(0,9)},color=black] (2pt,0pt) -- (-2pt,0pt) node[above left] {$9$};
        \end{tikzpicture}}
        \noindent
        Bởi vậy, hàm số $g(x)=\dfrac{\sqrt{x-1}}{f^2(x)-9f(x)}$ có tập xác định là $\mathscr D=\left[1;3\right) \cup \left(3;a\right) \cup\left( a;+\infty\right)$.\\
        Khi đó ta có
        \begin{itemize}
            \item $\lim\limits_{x\to+\infty}g(x)=0$ nên đồ thị hàm số $y=g(x)$ có một đường tiệm cận ngang là đường thẳng $y=0$.
            \item $\lim\limits_{x\to 1^+}g(x)=\lim\limits_{x\to 1^+}\dfrac{\sqrt{x-1}}{f(x)\left[f(x)-9\right] }=+\infty$;\\
            $\lim\limits_{x\to 3^+}g(x)=\lim\limits_{x\to 3^+}\dfrac{\sqrt{x-1}}{f(x)\left[f(x)-9\right] }=-\infty$;\\
            $\lim\limits_{x\to a^+}g(x)=\lim\limits_{x\to a^+}\dfrac{\sqrt{x-1}}{f(x)\left[f(x)-9\right] }=+\infty$.\\
            Do đó nên đồ thị hàm số $y=g(x)$ có $3$ đường tiệm cận đứng là các đường thẳng $x=1$, $x=3$ và $x=a$.
        \end{itemize}
        Như vậy, đồ thị hàm số $y=g(x)$ có $4$ đường tiệm cận, trong đó có $1$ đường tiệm cận ngang và $3$ đường tiệm cận đứng.
    }
\end{ex}
\begin{ex}%[VDC5-Đỗ Đường Hiếu]%[2D1G4-3]%
    \immini{Cho hàm số bậc ba $y=f(x)$ có đồ thị như hình vẽ bên. Đồ thị hàm số $g(x)=\dfrac{x\sqrt{x+1}}{f(x)\left[f^2(x)-16 \right] }$ có bao nhiêu tiệm cận đứng?
        \choice
        {\True $4$}
        {$5$}
        {$6$}
        {$7$}}
    {\begin{tikzpicture}[scale=0.6,line join=round, line cap=round,font=\footnotesize,>=stealth]
            \draw[->] (-2.5,0)--(4,0) node[below]{$x$};
            \draw[->] (0,-5)--(0,2.5) node[left]{$y$};
            \coordinate[label=below left:$O$] (O) at (0,0);
            \draw[dashed] (-1,0)--(-1,-4)--(0,-4);
            \clip (-2.3,-5) rectangle (3.5,2.5);
            \draw[smooth,samples=300,domain=-3.5:3.5] plot(\x,{-0.5*(\x+2)*(\x-1)*(\x-3)});
            \foreach \x in {-2,-1,1,3}
            \draw[shift={(\x,0)},color=black] (0pt,2pt) -- (0pt,-2pt) node[above] { $\x$};
            \foreach \y in {-4,-3,1}
            \draw[shift={(0,\y)},color=black] (2pt,0pt) -- (-2pt,0pt) node[right] {$\y$};
        \end{tikzpicture}
    }
    \loigiai{
        Xét phương trình $f(x)\left[f^2(x)-16 \right]=0$ \, $(*)$, với điều kiện $x\in\left[-1;+\infty \right) $.\\
        Ta có $f(x)\left[f^2(x)-16 \right]=0\Leftrightarrow \hoac{&f(x)=0\\&f(x)=4\\&f(x)=-4.}$\\
        \begin{itemize}
            \item Phương trình $f(x)=0$ có hai nghiệm $x\in\left[-1;+\infty \right) $ là $x=1$ và $x=3$.
            \item Phương trình $f(x)=4$ có không có nghiệm $x\in\left[-1;+\infty \right) $.
            \item Phương trình $f(x)=-4$ có hai nghiệm $x\in\left[-1;+\infty \right) $ là $-1<x_1<0$ và $x_2>3$.
        \end{itemize}
        Rõ ràng $\lim\limits_{x\to x_0^+}g(x)=+\infty$ hoặc $\lim\limits_{x\to x_0^+}g(x)=-\infty$, trong đó $x=x_0$ là nghiệm thuộc $\left[-1;+\infty \right) $ của phương trình $(*)$. Do đó đường thẳng $x=x_0$ là tiệm cận đứng của đồ thị hàm số $y=g(x)$.\\
        Từ đó suy ra đồ thị hàm số $g(x)=\dfrac{x\sqrt{x+1}}{f(x)\left[f^2(x)-16 \right] }$ có $4$ tiệm cận đứng.
    }
\end{ex}
\begin{ex}%[VDC5-Đỗ Đường Hiếu]%[2D1G4-3]%
    \immini{Cho $y=f(x)$ là hàm số đa thức có đồ thị như hình vẽ bên. Đặt $g(x)=\dfrac{\sqrt{x-1}}{\left[f(x)\right]^2-2f(x)}$ có bao nhiêu đường tiệm cận đứng?
        \choice
        {$5$}
        {$3$}
        {$4$}
        {\True $2$}}
    {\begin{tikzpicture}[scale=0.6,line join=round, line cap=round,font=\footnotesize,>=stealth]
            \draw[->] (-3,0)--(2.5,0) node[below]{$x$};
            \draw[->] (0,-1)--(0,5) node[left]{$y$};
            \coordinate[label=above left:$O$] (O) at (0,0);
            \draw[dashed] (-1,0)--(-1,4)--(0,4);
            \clip (-2.3,-1) rectangle (2.5,4.5);
            \draw[smooth,samples=300,domain=-3.5:3.5] plot(\x,{(\x)^3-3*(\x)+2});
            \foreach \x in {-2,-1,1}
            \draw[shift={(\x,0)},color=black] (0pt,2pt) -- (0pt,-2pt) node[below] { $\x$};
            \foreach \y in {2,4}
            \draw[shift={(0,\y)},color=black] (2pt,0pt) -- (-2pt,0pt) node[right] {$\y$};
        \end{tikzpicture}
    }
    \loigiai{
        Xét phương trình $\left[f(x)\right]^2-2f(x)=0$ \, $(*)$, với điều kiện $x\in\left[1;+\infty \right) $.\\
        Ta có $\left[f(x)\right]^2-2f(x)=0\Leftrightarrow \hoac{&f(x)=0\\&f(x)=2.}$\\
        \begin{itemize}
            \item Phương trình $f(x)=0$ có một nghiệm $x\in\left[1;+\infty \right) $ là $x=1$.
            \item Phương trình $f(x)=2$ có một nghiệm $x\in\left[1;+\infty \right) $ là $x=x_1>1$.
        \end{itemize}
        Rõ ràng $\lim\limits_{x\to x_0^+}g(x)=+\infty$ hoặc $\lim\limits_{x\to x_0^+}g(x)=-\infty$, trong đó $x=x_0$ là nghiệm thuộc $\left[1;+\infty \right) $ của phương trình $(*)$. Do đó đường thẳng $x=x_0$ là tiệm cận đứng của đồ thị hàm số $y=g(x)$.\\
        Từ đó suy ra đồ thị hàm số $g(x)=\dfrac{\sqrt{x-1}}{\left[f(x)\right]^2-2f(x)}$ có $2$ tiệm cận đứng.
    }
\end{ex}
\begin{ex}%[VDC5-NgocDungHo]%[2D1G4-3]%
    \immini
    {
        Cho hàm số $f(x)$ có đồ thị như hình bên. Số đường tiệm cận đứng của đồ thị hàm số $y=\dfrac{(x^2-4)(x^2+2x)}{[f(x)]^2-4f(x)+3}$ là
        \choice
        {$4$}
        {\True $5$}
        {$3$}
        {$2$}
    }
    {\begin{tikzpicture}[>=stealth,scale=0.5, line join=round, line cap=round]
            \def\f[#1]{-0.25*((#1)^4-8*(#1)^2+4)}
            \draw[->] (-4.1,0)--(4,0) node [below]{$x$};
            \draw[->] (0,-2)--(0,4) node [left]{$y$};
            \node at (0,0) [above left]{$O$};
            % \clip;
            \draw[domain=-2.9:2.9,samples=300,thick] plot (\x,{\f[\x]});
            \foreach \x in {-2,2} \filldraw (\x,0) node[below]{\x} circle (2pt);
            %\foreach \x in {-3,3} \filldraw (\x,0) node[below left]{\x} circle (2pt);
            \filldraw (-3,0) node[below left]{$-3$} circle (2pt);
            \filldraw (3,0) node[below right]{$3$} circle (2pt);
            \filldraw (0,1) node[left]{$1$} circle (2pt);
            \filldraw (0,3) node[above left]{$3$} circle (2pt);
            \draw[dashed](-2,0)--(-2,3)--(2,3)--(2,0);
            \draw (3,-1.75) node[right]{$y=f(x)$};
        \end{tikzpicture}
    }
    \loigiai{
        Xét hàm số $y=g(x)=\dfrac{(x^2-4 )(x^2+2x)}{[f(x)]^2-4f(x)+3}$.
        \immini
        {
            Giải phương trình $(x^2-4)(x^2+2x)=0 $\\
            $\Leftrightarrow \hoac{& x^2-4=0 \\ & x^2+2x=0}\Leftrightarrow \hoac{& x=\pm 2 \\ & x=0.}$\\
            Giải phương trình $[f(x)]^2-4f(x)+3=0$\\
            $ \Leftrightarrow \hoac{& f(x)=1 \\ & f(x)=3} \Leftrightarrow \hoac{& x = \pm 2 \\ & x=a\\&x=b\\&x=c\\&x=d.}$\\ với $-3<a<-2<b<c<2<d<3$.\\
        }
        {\begin{tikzpicture}[>=stealth,scale=0.8, line join=round, line cap=round]
                \def\f[#1]{-0.25*((#1)^4-8*(#1)^2+4)}
                \def\g[#1]{1}
                \def\h[#1]{3}
                \draw[->] (-4.1,0)--(4,0) node [below]{$x$};
                \draw[->] (0,-2)--(0,4) node [left]{$y$};
                \node at (0,0) [above left]{$O$};
                % \clip;
                \draw[domain=-2.9:2.9,samples=300,thick] plot (\x,{\f[\x]});
                \draw[domain=-4:4,samples=300,thick] plot (\x,{\g[\x]});
                \draw[domain=-4:4,samples=300,thick] plot (\x,{\h[\x]});
                \foreach \x in {-3,-2,2,3} \filldraw (\x,0) node[below]{\x} circle (2pt);
                % \filldraw (-3,0) node[above left]{$-3$} circle (2pt);
                % \filldraw (3,0) node[above ]{$3$} circle (2pt);
                \filldraw (0,1) node[below left]{$1$} circle (2pt);
                \filldraw (0,-1) node[below left]{$-1$} circle (2pt);
                \filldraw (0,3) node[above left]{$3$} circle (2pt);
                \draw[dashed](-2,0)--(-2,3) (2,3)--(2,0) (2.61,0)node[below]{$d$}--(2.61,1) (-2.61,0)node[below]{$a$}--(-2.61,1) (1.08,0)node[below]{$c$}--(1.08,1)(-1.08,0)node[below]{$b$}--(-1.08,1);
                \draw (3,2.75) node[right]{$y=f(x)$};
            \end{tikzpicture}
        }
        Trong điều kiện xác định của hàm số $y=g(x)$ ta có thể viết $$y=g(x)=\dfrac{x(x-2)(x+2)^2}{(x-a)(x-b)(x-c)(x-d) (x-2)^2(x+2)^2}=\dfrac{x}{(x-a)(x-b)(x-c)(x-d)(x-2)}$$
        Vậy số tiệm cận đứng của đồ thị hàm số $y=g(x)$ bằng $5$.
    }
\end{ex}
\Closesolutionfile{ans}

PHẦN I. Câu trắc nghiệm nhiều phương án lựa chọn. Thí sinh trả lời từ câu 1 đến câu 12. Mỗi câu hỏi thí sinh chỉ chọn một phương án.
%Câu 1
\begin{ex}
	Tìm mệnh đề sai trong các mệnh đề sau:
	\choice
	{\True Hai vectơ được gọi là cùng phương nếu chúng có giá song song với nhau}
	{Nếu hai vectơ cùng phương thì chúng cùng hướng hoặc ngược hướng}
	{Hai vectơ được gọi là bằng nhau nếu chúng cùng độ dài và cùng hướng}
	{Nếu vectơ $\vec{a}$ và vectơ $\vec{b}$ cùng bằng vectơ $\vec{c}$ thì hai vectơ $\vec{a}$ và vectơ $\vec{b}$ bằng nhau}
	\loigiai{
		Hai vectơ được gọi là cùng phương nếu chúng có giá song song hoặc trùng nhau.
	}
\end{ex}
%Câu 2
\begin{ex}
	Cho hình hộp chữ nhật $ABCD.A'B'C'D'$. Khi đó, vectơ bằng vectơ $\vec{AB}$ là
	\choice
	{\True $\vec{D'C'}$}
	{$\vec{BA}$}
	{$\vec{CD}$}
	{$\vec{B'A'}$}
	\loigiai{
		\immini{
			Dễ thấy vectơ bằng với vectơ $\vec{AB}$ là vectơ nào $\vec{D'C'}$ vì chúng cùng hướng và có cùng độ dài.
		}{\begin{tikzpicture}[line cap=round,line join=round, >=stealth,scale=.6]
				\def \a{-1.5} \def \b{-1}\def \c{4.5} \def \h{4}
				\path (.5,.5)coordinate(A) 
				+(\a,\b)coordinate(B)
				+(\c,0)coordinate(D)
				($(B)+(D)-(A)$)coordinate(C)
				+(0,\h) coordinate(C')
				($(B)+(C')-(C)$)coordinate(B')
				($(A)+(C')-(C)$)coordinate(A')
				($(D)+(C')-(C)$)coordinate(D');
				\draw [dashed] (A)--(B)(D)--(A)--(A');
				\draw (B')--(B)--(C)(B')--(C')--(C)--(D)--(D')--(A')--(B')(C')--(D');
				\foreach \x/\g in {A/135,B/-135,C/-45,D/0,A'/135,B'/180,C'/-20,D'/0}\fill[red] (\x) circle (1pt)+(\g:3mm) node[black]{$\x$};
		\end{tikzpicture}}
	}
\end{ex}
%Câu 3
\begin{ex}
	Cho hình hộp $ABCD.A'B'C'D'$. Vectơ nào dưới đây cùng phương với vectơ $\vec{AB}$?
	\choice
	{\True $\vec{CD}$}
	{$\vec{B'C'}$}
	{$\vec{AD}$}
	{$\vec{AC'}$}
	\loigiai{
		\immini{
			Vectơ cùng phương với $\vec{AB}$ là $\vec{CD}$, vì hai vectơ này có giá song song với nhau.}
		{\begin{tikzpicture}[line cap=round,line join=round, >=stealth,scale=.6]
				\def \a{-1.5} \def \b{-1}\def \c{4.5} \def \h{4}
				\path (.5,.5)coordinate(A) 
				+(\a,\b)coordinate(B)
				+(\c,0)coordinate(D)
				($(B)+(D)-(A)$)coordinate(C)
				+(0,\h) coordinate(C')
				($(B)+(C')-(C)$)coordinate(B')
				($(A)+(C')-(C)$)coordinate(A')
				($(D)+(C')-(C)$)coordinate(D');
				\draw [dashed] (A)--(B)(D)--(A)--(A');
				\draw (B')--(B)--(C)(B')--(C')--(C)--(D)--(D')--(A')--(B')(C')--(D');
				\foreach \x/\g in {A/135,B/-135,C/-45,D/0,A'/135,B'/180,C'/-20,D'/0}\fill[red] (\x) circle (1pt)+(\g:3mm) node[black]{$\x$};
		\end{tikzpicture}}
	}
\end{ex}
%Câu 4
\begin{ex}
	Cho hình hộp $ABCD.A'B'C'D'$. Mệnh đề nào sau đây sai?
	\choice
	{$\vec{AC'}=\vec{AB}+\vec{AD}+\vec{AA'}$}
	{\True $\vec{BC'}=\vec{BC}+\vec{BD}+\vec{BB'}$}
	{$\vec{DB'}=\vec{DA}+\vec{DC}+\vec{DD'}$}
	{$\vec{BD'}=\vec{BA}+\vec{BC}+\vec{BB'}$}
	\loigiai{
		Theo quy tắc hình hộp, ta có mệnh đề sai là $\vec{BC'}=\vec{BC}+\vec{BD}+\vec{BB'}$.
	}
\end{ex}
%Câu 5
\begin{ex}
	\immini{Cho hình tứ diện $ABCD$. Gọi $M$, $N$ lần lượt là trung điểm $AB$, $CD$, $I$ là trung điểm của đoạn $MN$. Mệnh đề nào sau đây sai?
		\choice
		{\True $\vec{AN}=\left(\vec{AD}+\vec{AC}\right)$}
		{$\vec{IN}+\vec{IM}=\vec{0}$}
		{$\vec{MA}+\vec{MB}=\vec{0}$}
		{$\vec{NC}+\vec{ND}=\vec{0}$}
	}{
		\begin{tikzpicture}[scale=.6, font=\footnotesize, line join=round, line cap=round, >=stealth]
			\def\ac{4} % cạnh AC
			\def\ab{2} % cạnh AB
			\def\ad{4} % cạnh AD
			\def\gocA{50} % góc A của đáy
			\path
			(0,0) coordinate (A)
			(\ac,0) coordinate (C)
			(-\gocA:\ab) coordinate (B)
			(65:\ad) coordinate (D)
			($(A)!.5!(B)$) coordinate (M)
			($(C)!.5!(D)$) coordinate (N)
			($(M)!.5!(N)$) coordinate (I)
			;
			\draw (A)--(B)--(C)--(D)--cycle (D)--(B);
			\draw[dashed] (A)--(C) (M)--(N);
			\foreach \x/\g in {A/180,B/-90,C/0,D/90,M/-135,N/45,I/135}\fill (\x) circle (1pt)+(\g:3mm) node[black]{$\x$};
	\end{tikzpicture}}
	\loigiai{
		Đáp án B đúng: Vì $I$ là trung điểm $MN$ nên ta có: $\vec{IN}+\vec{IM}=\vec{0}$.\\
		Đáp án C đúng: Vì $M$ là trung điểm $AB$ nên ta có: $\vec{MA}+\vec{MB}=\vec{0}$.\\
		Đáp án D đúng. Vì N là trung điểm CD nên ta có $\vec{NC}+\vec{ND}=\vec{0}$.
	}
\end{ex}
%Câu 6
\begin{ex}
	\immini{Cho hình lập phương $ABCD.A'B'C'D'$ có cạnh bằng $a$. Hãy tìm mệnh đề đúng trong những mệnh đề sau đây
		\choice
		{\True $\vec{AB}+\vec{AD}+\vec{AA'}=\vec{AC'}$}
		{$\vec{AD}+\vec{DB'}=\vec{B'A}$}
		{$\vec{AB}+\vec{AD}=\vec{BD}$}
		{$\vec{AC}-\vec{AB'}=\vec{CB'}$}
	}{
		\begin{tikzpicture}[line cap=round,line join=round, >=stealth,scale=.6]
			\def \a{-1.5} \def \b{-1}\def \c{4} \def \h{4}
			\path (.5,.5)coordinate(A) 
			+(\a,\b)coordinate(B)
			+(\c,0)coordinate(D)
			($(B)+(D)-(A)$)coordinate(C)
			+(0,\h) coordinate(C')
			($(B)+(C')-(C)$)coordinate(B')
			($(A)+(C')-(C)$)coordinate(A')
			($(D)+(C')-(C)$)coordinate(D');
			\draw [dashed] (A)--(B)(D)--(A)--(A');
			\draw (B')--(B)--(C)(B')--(C')--(C)--(D)--(D')--(A')--(B')(C')--(D');
			\foreach \x/\g in {A/135,B/-135,C/-45,D/0,A'/135,B'/180,C'/-20,D'/0}\fill[red] (\x) circle (1pt)+(\g:3mm) node[black]{$\x$};
	\end{tikzpicture}}
	\loigiai{
		Theo quy tắc hình hộp ta có $\vec{AB}+\vec{AD}+\vec{AA'}=\vec{AC'}$.
	}
\end{ex}
%Câu 7
\begin{ex}
	Cho tứ diện $ABCD$, có bao nhiêu vectơ có điểm đầu là $A$ và điểm cuối là một trong các đỉnh còn lại của tứ diện?
	\choice
	{$1$}
	{\True $3$}
	{$2$}
	{$4$}
	\loigiai{
		Có ba vectơ là: $\vec{AB},\vec{AC},\vec{AD}$.
	}
\end{ex}
%Câu 8
\begin{ex}
	Cho hình hộp $ABCD.A'B'C'D'$. Hai vectơ nào sau đây cùng phương?
	\choice
	{$\vec{A'B}$ và $\vec{A'B'}$}
	{$\vec{B'C'}$ và $\vec{CD}$}
	{$\vec{AB}$ và $\vec{B'C'}$}
	{\True $\vec{AB}$ và $\vec{D'C'}$}
	\loigiai{
		\immini{
			Hai vectơ $\vec{AB}$ và $\vec{D'C'}$ có giá song song nên cùng phương.}
		{\begin{tikzpicture}[scale=.7, font=\footnotesize, line join=round, line cap=round, >=stealth]
				\def\bc{4} % cạnh BC
				\def\ba{2} % cạnh BA
				\def\h{3.5} % đường cao
				\def\gocB{35} % góc B của đáy
				\coordinate[label=below left:$B$] (B) at (0,0);
				\coordinate[label=above left:$A$] (A) at (\gocB:\ba);
				\coordinate[label=below:$C$] (C) at (\bc,0);
				\coordinate[label=right:$D$] (D) at ($(C)-(B)+(A)$);
				\coordinate[label=above left:$A'$] (A') at ($(A)+(80:\h)$);
				\coordinate[label=left:$B'$] (B') at ($(B)-(A)+(A')$);
				\coordinate[label=below right:$C'$] (C') at ($(C)-(A)+(A')$);
				\coordinate[label=right:$D'$] (D') at ($(D)-(A)+(A')$);
				\draw (B')--(B)--(C)--(D)--(D')--(A')--(B')--(C')--(D') (C)--(C');
				\draw[dashed] (A)--(D) (A')--(A)--(B);
				\foreach \diem in {A,B,C,D,A',B',C',D'}	\fill (\diem)circle(1.5pt);
		\end{tikzpicture}}
	}
\end{ex}
%Câu 9
\begin{ex}
	\immini{
		Cho tứ diện $ABCD$. Gọi $M$, $N$ lần lượt là trung điểm của $AB$, $CD$, $G$ là trung điểm của MN. Vectơ $\vec{GA}+\vec{GB}+\vec{GC}+\vec{GD}$ bằng Vectơ nào sau đây
		\choice
		{$4\vec{MG}$}
		{$\vec{GD}$}
		{\True $\vec{0 \cdot }$}
		{$\vec{MN}$}
	}{\begin{tikzpicture}[scale=.6, font=\footnotesize, line join=round, line cap=round, >=stealth]
			\def\ac{4} % cạnh AC
			\def\ab{2} % cạnh AB
			\def\ad{4} % cạnh AD
			\def\gocA{50} % góc A của đáy
			\path
			(0,0) coordinate (A)
			(\ac,0) coordinate (C)
			(-\gocA:\ab) coordinate (B)
			(65:\ad) coordinate (D)
			($(A)!.5!(B)$) coordinate (M)
			($(C)!.5!(D)$) coordinate (N)
			($(M)!.5!(N)$) coordinate (G)
			;
			\draw (A)--(B)--(C)--(D)--cycle (D)--(B);
			\draw[dashed] (A)--(C) (M)--(N);
			\foreach \x/\g in {A/180,B/-90,C/0,D/90,M/-135,N/45,G/-45}\fill (\x) circle (1pt)+(\g:3mm) node[black]{$\x$};
	\end{tikzpicture}}
	\loigiai{
		$\vec{GA}+\vec{GB}+\vec{GC}+\vec{GD}=\left(\vec{GA}+\vec{GB}\right)+\left(\vec{GC}+\vec{GD}\right)=2\vec{GM}+2\vec{GN}=2\left(\vec{GM}+\vec{GN}\right)=\vec{0}$.
	}
\end{ex}
%Câu 10
\begin{ex}
	Cho hình lập phương $ABCD.A'B'C'D'$. Chọn mệnh đề đúng?
	\choice
	{$\vec{AC}=\vec{C'A'}$}
	{$\vec{AB}+\vec{AD}+\vec{AC}=\vec{AA'}$}
	{$\vec{AB}=\vec{CD}$}
	{\True $\vec{AB}+\vec{C'D'}=\vec{0}$}
	\loigiai{
		Ta có $\vec{AB}+\vec{C'D'}=\vec{AB}+\vec{CD}=\vec{AB}+\vec{BA}=\vec{0}$.
	}
\end{ex}
%Câu 11
\begin{ex}
	Cho hình lăng trụ $ABC.A'B'C'$, $M$ là trung điểm của $BB'$. Đặt $\vec{CA}=\vec{a}$, $\vec{CB}=\vec{b}$, $\vec{AA'}=\vec{c}$. Khẳng định nào sau đây đúng?
	\choice
	{$\vec{AM}=\vec{b}+\vec{c}-\dfrac{1}{2}\vec{a}$}
	{$\vec{AM}=\vec{a}-\vec{c}+\dfrac{1}{2}\vec{b}$}
	{$\vec{AM}=\vec{a}+\vec{c}-\dfrac{1}{2}\vec{b}$}
	{\True $\vec{AM}=\vec{b}-\vec{a}+\dfrac{1}{2}\vec{c}$}
	\loigiai{
		\immini{
			Ta có\\
			$\vec{AM}=\vec{AB}+\vec{BM}=\vec{CB}-\vec{CA}+\dfrac{1}{2}\vec{BB'}$\\
			$=\vec{b}-\vec{a}+\dfrac{1}{2}\vec{AA'}=\vec{b}-\vec{a}+\dfrac{1}{2}\vec{c}$.
		}{
			\begin{tikzpicture}[scale=.6, font=\footnotesize, line join=round, line cap=round, >=stealth]
				\def\ac{4} % cạnh AC
				\def\ab{2} % cạnh AB
				\def\ben{4} % cạnh bên
				\def\gocnghieng{75} % góc nghiêng cạnh bên
				\def\gocA{50} % góc A của đáy
				\coordinate[label=left:$A$] (A) at (0,0);
				\coordinate[label=right:$C$] (C) at (\ac,0);
				\coordinate[label=below left:$B$] (B) at (-\gocA:\ab);
				\coordinate[label=left:$A'$] (A') at ($(A)+(\gocnghieng:\ben)$);
				\coordinate[label=below left:$B'$] (B') at ($(B)-(A)+(A')$);
				\coordinate[label=right:$C'$] (C') at ($(C)-(A)+(A')$);
				\draw (A')--(A)--(B)--(C)--(C')--(A')--(B')--(C') (B)--(B');
				\draw[dashed] (A)--(C);
				\foreach \diem in {A,B,C,A',B',C'} \fill (\diem)circle(1.5pt);
			\end{tikzpicture}
		}
	}
\end{ex}
%Câu 12
\begin{ex}
	\immini{
		Cho tứ diện $ABCD$ có $M$, $N$ lần lượt là trung điểm các cạnh $AC$ và $BD$. Gọi $G$ là trung điểm của đoạn thẳng $MN$. Hãy chọn khẳng định sai
		\choice
		{$\vec{GA}+\vec{GC}=2\vec{GM}$}
		{$\vec{GB}+\vec{GD}=\vec{MN}$}
		{$\vec{GA}+\vec{GB}+\vec{GC}+\vec{GD}=\vec{0}$}
		{\True $2\vec{NM}=\vec{AB}+\vec{CD}$}
	}{\begin{tikzpicture}[scale=.6, font=\footnotesize, line join=round, line cap=round, >=stealth]
			\def\ab{4}
			\def\ac{2}
			\def\ad{4}
			\def\gocA{55}
			\path
			(0,0) coordinate (A)
			(\ab,0) coordinate (B)
			(-\gocA:\ac) coordinate (C)
			(65:\ad) coordinate (D)
			($(A)!.5!(C)$) coordinate (M)
			($(B)!.5!(D)$) coordinate (N)
			($(M)!.5!(N)$) coordinate (G)
			;
			\draw (A)--(C)--(B)--(D)--cycle (D)--(C);
			\draw[dashed] (A)--(B) (M)--(N);
			\foreach \x/\g in {A/180,B/0,C/-90,D/90,M/-135,N/45,G/-45}\fill (\x) circle (1pt)+(\g:3mm) node[black]{$\x$};
	\end{tikzpicture}}
	\loigiai{
		\begin{itemize}
			\item $\vec{GA}+\vec{GC}=2\vec{GM}$ đúng vì $M$ là trung điểm $AC$.\\
			\item $\vec{GB}+\vec{GD}=\vec{MN}$ đúng vì $\vec{GB}+\vec{GD}=2\vec{GN}=\vec{MN}$\\
			\item $\vec{GA}+\vec{GB}+\vec{GC}+\vec{GD}=\vec{0}$ đúng vì $\vec{GA}+\vec{GB}+\vec{GC}+\vec{GD}=2\left(\vec{GM}+\vec{GN}\right)=\vec{0}$.\\
			\item $2\vec{NM}=\vec{AB}+\vec{CD}$ sai vì $\vec{AB}+\vec{CD} =\left(\vec{AM}+\vec{MN}+\vec{NB}\right)+\left(\vec{CM}+\vec{MN}+\vec{ND}\right) =2\vec{MN}+\vec{0}+\vec{0}=2\vec{MN}$.
		\end{itemize}
	}
\end{ex}
%Câu 13
\begin{ex}
	Cho tứ diện đều $SABC$ có cạnh $a$. Gọi $M$, $N$ lần lượt là trung điểm $SA$, $BC$. Các mệnh đề sau đúng hay sai?
	\begin{center}
		\begin{tikzpicture}[scale=1, font=\footnotesize, line join=round, line cap=round, >=stealth]
			\def\ac{4} % cạnh AC
			\def\ab{2} % cạnh AB
			\def\as{4} % cạnh AS
			\def\gocA{50} % góc A của đáy
			\path
			(0,0) coordinate (A)
			(\ac,0) coordinate (C)
			(-\gocA:\ab) coordinate (B)
			(70:\as) coordinate (S)
			($(S)!.5!(A)$) coordinate (M)
			($(B)!.5!(C)$) coordinate (N)
			;
			\draw (A)--(B)--(C)--(S)--cycle (S)--(B);
			\draw (S)--(A)node[midway,above left]{$a$};
			\draw[dashed] (A)--(C) (M)--(N);
			\foreach \x/\g in {A/180,B/-90,C/0,S/90}\fill (\x) circle (1pt)+(\g:3mm) node[black]{$\x$};
		\end{tikzpicture}
	\end{center}
	\choiceTF
	{\True Độ dài của vectơ $\vec{SA}$ bằng $a$.}
	{\True $\vec{SA} \cdot \vec{SB}=\dfrac{a^2\sqrt{3}}{2}$}
	{$\vec{SB}+\vec{AB}+\vec{SC}+\vec{AC}=4\vec{MN}$}
	{Gọi $I$ là trọng tâm của tứ diện. Khoảng cách từ $I$ đến $(ABC)$ bằng $\dfrac{3a\sqrt{6}}{4}$}
	\loigiai{
		\begin{enumerate}[a)]
			\item $|\vec{SA}|=SA=a$.
			\item $\vec{SA} \cdot \vec{SB}=\left| \vec{SA} \right| \cdot \left| \vec{SB} \right| \cdot \sin \widehat{ASB}=a \cdot a \cdot \sin 60^\circ=\dfrac{a^2\sqrt{3}}{2}$.
			\item Do $N$ là trung điểm của $BC$ nên $\vec{SB}+\vec{SC}=2\vec{SN}$ và $\vec{AB}+\vec{AC}=2\vec{MB}$.\\
			Suy ra $\vec{SB}+\vec{SC}+\vec{AB}+\vec{AC}=2\left(\vec{SN}+\vec{AN}\right)$\\
			Do $M$ là trung điểm của $SA$ nên $\vec{NA}+\vec{NS}=2\vec{NM}\Leftrightarrow \vec{AN}+\vec{SN}=2\vec{MN}$.\\
			Do đó $\vec{SB}+\vec{SC}+\vec{AB}+\vec{AC}=2 \cdot 2 \cdot \vec{MN}=4\vec{MN}$.
			\item Gọi $G$ là trọng tâm tam giác $ABC$.\\
			Do tứ diện $SABC$ là tứ diện đều và $I$ là trọng tâm tứ diện nên $d\left(I,(ABC)\right)=IG$\\
			Tam giác $ABC$ đều cạnh $a$, $N$ là trung điểm của $BC$, suy ra $AN=\dfrac{a\sqrt{3}}{2}$.\\
			Do $G$ là trọng tâm tam giác$ABC$ nên $AG=\dfrac{2}{3}AN=\dfrac{a\sqrt{3}}{3}$.\\
			Do tứ diện $SABC$ là tứ diện đều nên $SG\bot (ABC)\Rightarrow SG\bot AG$.\\
			Tam giác $SAG$ vuông tại $G$ nên $SG=\sqrt{SA^2-AG^2}=\sqrt{a^2-\dfrac{a^2}{3}}=\dfrac{a\sqrt{6}}{3}$.\\
			Do $I$ là trọng tâm tứ diện$SABC$ nên $IG=\dfrac{1}{4}SG=\dfrac{1}{4} \cdot \dfrac{a\sqrt{6}}{3}=\dfrac{a\sqrt{6}}{12}$.\\
			Vậy $d\left(I,(ABC)\right)=\dfrac{a\sqrt{6}}{12}$.
		\end{enumerate}
	}
\end{ex}
%Câu 14
\begin{ex}
	Cho hình lập phương $ABCD.A_1B_1C_1D_1$ có cạnh $a$. Gọi $M$ là trung điểm $AD$. Các mệnh đề sau đúng hay sai?
	\choiceTF
	{$\vec{A_1B_1}=\vec{CD}$}
	{\True $\vec{DC_1}=\vec{DC}+\vec{DD_1}$}
	{\True $\vec{AB_1} \cdot \vec{CD_1}=0$}
	{$\vec{B_1M} =\vec{BB_1}+\vec{A_1B_1
	}+\dfrac{1}{2}\vec{B_1C_1}$}
	\loigiai{
		{\centering\color{red} HINH O DAY}\\
		\begin{enumerate}[a)]
			\item $\vec{A_1B_1}=-\vec{CD}$.
			\item $\vec{DC}+\vec{DD_1}=\vec{DC}+\vec{CC_1}=\vec{DC_1}$.
			\item $\vec{AB_1} \cdot \vec{CD_1}=\vec{AB_1} \cdot \vec{BA_1}=0$
			\item $\begin{aligned}[t]
			\vec{B_1M} & =\vec{B_1B}+\vec{BM} =\vec{BB_1}+\dfrac{1}{2}\left(\vec{BA}+\vec{BD}\right) =\vec{BB_1}+\dfrac{1}{2}\left(\vec{B_1A_1}+\vec{B_1D_1}\right) \\
			& =\vec{BB_1}+\dfrac{1}{2}\left(\vec{B_1A_1}+\vec{B_1A_1}+\vec{B_1C_1}\right) =\vec{BB_1}+\vec{B_1A_1}+\dfrac{1}{2}\vec{B_1C_1} \end{aligned}$
		\end{enumerate}	
	}
\end{ex}
%Câu 15 
\begin{ex}
	Cho tứ diện $ABCD$ có cạnh $a$. Gọi $M$, $N$ lần lượt là trung điểm của $AB$, $CD$. Các mệnh đề sau đúng hay sai?
	\choiceTF
	{$\vec{AB}$ và $\vec{CD}$ cùng hướng}
	{\True $\vec{EA}+\vec{EB}+\vec{EC}+\vec{ED}=\vec{0}$ với $E$ là trung điểm $MN$}
	{\True $\vec{AB} \cdot \vec{CD}+\vec{AC} \cdot \vec{DB}+\vec{AD} \cdot \vec{BC}=\vec{0}$}
	{\True Điểm $I$ xác định bởi $P=3\vec{IA}^2+\vec{IB}^2+\vec{IC}^2+\vec{ID}^2$ có giá trị nhỏ nhất. Khi đó giá trị nhỏ nhất của $P$ là $2a^2$}
	\loigiai{
		\begin{enumerate}[a)]
			\item $\vec{AB}$ và $\vec{CD}$ ngược hướng.
			\item Vì $M$ là trung điểm $AB$ nên $\vec{EA}+\vec{EB}=2\vec{EM}$, $N$ là trung điểm $CD$ nên $\vec{EC}+\vec{ED}=2\vec{EN}$.\\
			Ta có $\vec{EA}+\vec{EB}+\vec{EC}+\vec{ED}=2\left(\vec{EM}+\vec{EN}\right)=\vec{0}$.
			\item $\begin{aligned}[t]
				&\vec{AB} \cdot \vec{CD}+\vec{AC} \cdot \vec{DB}+\vec{AD} \cdot \vec{BC}=\left(\vec{AC}+\vec{CB}\right) \cdot \vec{CD}+\vec{AC} \cdot \vec{DB}+\vec{AD} \cdot \vec{BC}\\
				 = & \vec{AC} \cdot \left(\vec{CD}+\vec{DB}\right)+\vec{AD} \cdot \vec{BC}+\vec{CB \cdot }\vec{CD}=\vec{AC} \cdot \vec{CB}+\vec{AD} \cdot \vec{BC}+\vec{CB \cdot }\vec{CD} \\
				 = &\vec{CB}\left(\vec{AC}-\vec{AD}\right)+\vec{CB \cdot }\vec{CD}=\vec{0} 
				\end{aligned}$
			\item Gọi $M$ là điểm thoả mãn hệ thức $3\vec{MA}+\vec{MB}+\vec{MC}+\vec{MD}=\vec{0}$ suy ra $M$ cố định vì $A$, $B,C$, $D$ cố định. Ta có
			\begin{align*}
				P& =3\vec{IA}^2+\vec{IB}^2+\vec{IC}^2+\vec{ID}^2 \\
				& =3\left(\vec{IM}+\vec{MA}\right)^2+\left(\vec{IM}+\vec{MB}\right)^2+\left(\vec{IM}+\vec{MC}\right)^2+\left(\vec{IM}+\vec{MD}\right)^2 \\
				& =6IM^2+3MA^2+MB^2+MC^2+MD^2+2\vec{IM}\left(3\vec{MA}+\vec{MB}+\vec{MC}+\vec{MD}\right) \\
				& =6IM^2+3MA^2+MB^2+MC^2+MD^2.
			\end{align*}
			Do đó để $P$ nhỏ nhất thì $I$ trùng với $M$. Gọi $G$ là trọng tâm tam giác $BCD$.\\
			Vì $3\vec{MA}+\vec{MB}+\vec{MC}+\vec{MD}=3\vec{MA}+\left(\vec{MB}+\vec{MC}+\vec{MD}\right) =3\vec{MA}+3\vec{MG}$ nên $\vec{MA}+\vec{MG}=\vec{0}$.\\			
			Suy ra $M$ là trung điểm của $AG$.\\
			Ta có $BG=\dfrac{2}{3} \cdot \dfrac{a\sqrt{3}}{2}=\dfrac{a}{\sqrt{3}}\Rightarrow AG=\sqrt{AB^2-BG^2}=\sqrt{a^2-{{\left(\dfrac{a}{\sqrt{3}}\right)}^2}}=\dfrac{a\sqrt{2}}{\sqrt{3}}$\\
			$\Rightarrow MA=\dfrac{1}{2}AG=\dfrac{a}{\sqrt{6}}\Rightarrow MA^2=\dfrac{a^2}{6}$.\\
			Lại có $MD^2=MC^2=MB^2=MG^2+BG^2=\dfrac{a^2}{6}+\dfrac{a^2}{3}=\dfrac{a^2}{2}$.\\
			Vậy giá trị nhỏ nhất là $P=3 \cdot \dfrac{a^2}{6}+3 \cdot \dfrac{a^2}{2}=2a^2$ khi $I$ trùng với $M$.
		\end{enumerate}
		
	}
\end{ex}
%Câu 16
\begin{ex}
	Cho tứ diện đều $ABCD$ cạnh $a$ có $G$ là trọng tâm của tam giác $BCD$ và $I$ là điểm thuộc đoạn thẳng $AG$ sao cho $\vec{AI}=3\vec{IG}$. Các mệnh đề sau đúng hay sai?
	\choiceTF
	{$\vec{GA}+\vec{GB}+\vec{GC}=\vec{0}$}
	{\True $\vec{IB}+\vec{IC}+\vec{ID}=3\vec{IG}$}
	{\True $\vec{GB}+\vec{GC}+\vec{GD}=\vec{IA}+\vec{IB}+\vec{IC}+\vec{ID}$}
	{\True $\vec{IB}=\dfrac{3}{4}\vec{AB}-\dfrac{1}{4}\vec{AC}-\dfrac{1}{4}\vec{AD}$}
	\loigiai{
		{\centering\color{red} HINH O DAY}\\
		\begin{enumerate}
			\item $G$ là trọng tâm của tam giác $BCD$ nên $\vec{GB}+\vec{GC}+\vec{GD}=\vec{0}$.
			\item $\vec{IB}+\vec{IC}+\vec{ID}=\vec{IG}+\vec{GB}+\vec{IG}+\vec{GC}+\vec{IG}+\vec{GD}=3\vec{IG}+\left(\vec{GB}+\vec{GC}+\vec{GD}\right)=3\vec{IG}$.
			\item $\vec{GB}+\vec{GC}+\vec{GD}=\vec{0}\vec{IA}+\vec{IB}+\vec{IC}+\vec{ID}=\vec{IA}+3\vec{IG}=\vec{IA}+\vec{AI}=\vec{0}$.
			\item $\vec{AI}=3\vec{IG}\Leftrightarrow \vec{IA}=-\dfrac{3}{4}\vec{AG}$.\\
			$\vec{IB}=\vec{IA}+\vec{AB}=-\dfrac{3}{4}\vec{AG}+\vec{AB}=-\dfrac{3}{4} \cdot \dfrac{1}{3}\left(\vec{AB}+\vec{AC}+\vec{AD}\right)+\vec{AB}=\dfrac{3}{4}\vec{AB}-\dfrac{1}{4}\vec{AC}-\dfrac{1}{4}\vec{AD}$.
		\end{enumerate}
	}
\end{ex}
%Câu 17
\begin{ex}
	Cho tứ diện $ABCD$. Gọi $E$ là trung điểm $AD$, $F$ là trung điểm $BC$. Ta có $\vec{AB}+\vec{DC}= k\vec{EF}$. Tìm giá trị của $k$.
	\loigiai{
		\shortans{2}
		Do $E$ là trung điểm $AD$, $F$ là trung điểm $BC$ nên $\vec{EA}+\vec{ED}=\vec{0}$; $\vec{FB}+\vec{FC}=-\left(\vec{BF}+\vec{CF}\right)=\vec{0}$.\\
		Có $\heva{& \vec{AB}=\vec{AE}+\vec{EF}+\vec{FB} \\& \vec{DC}=\vec{DE}+\vec{EF}+\vec{FB}}\Rightarrow \vec{AB}+\vec{DC}=2\vec{EF}$.
	}
\end{ex}
%Câu 18
\begin{ex}
	Cho hình hộp chữ nhật $ABCD.A'B'C'D'$ có $AB=2$, $AD=3$. Độ dài vectơ $\vec{B'D'}$ bằng bao nhiêu (làm tròn đến hàng phần trăm)?
	\loigiai{
		{\centering\color{red} HINH O DAY}\\
		Ta có: $\left| \vec{B'D'} \right|=B'D'=BD=\sqrt{AB^2+AD^2}=\sqrt{13}$.\\
		Vậy độ dài vectơ $\vec{B'D'}$ bằng $\sqrt{13} \approx 3,61$.
	}
\end{ex}
%Câu 19
\begin{ex}
	Cho hình lập phương $ABCD.A'B'C'D'$. Góc giữa hai vectơ $\vec{A'B}$ và $\vec{AC'}$ bằng
	\loigiai{
		{\centering\color{red} HINH O DAY}\\
		$\vec{A'B}=\vec{A'A}+\vec{AB}=\vec{AB}-\vec{AA'}$.\\
		$\vec{AC'}=\vec{AB}+\vec{AD}+\vec{AA'}$.\\
		$\Rightarrow \vec{A'B} \cdot \vec{AC'} = \left(\vec{AB}-\vec{AA'}\right) \cdot \left(\vec{AB}+\vec{AD}+\vec{AA'}\right)={{\vec{AB}}^2}-{{\vec{AA'}}^2}=0$.\\
		$\Rightarrow$ Góc giữa hai vectơ $\vec{A'B}$ và $\vec{AC'}$ bằng $90^\circ$.
	}
\end{ex}
%Câu 20
\begin{ex}
	Cho hình chóp $S.ABC$ có $SA$, $SB$, $SC$ đôi một vuông góc nhau và $SA=SB=SC=a$. Gọi $M$ là trung điểm của $AB$. Góc giữa hai vectơ $\vec{SM}$ và $\vec{BC}$ bằng
	\loigiai{
		{\centering\color{red} HINH O DAY}\\
		\shortans{120}
		Ta có $\cos \left(\vec{SM},\vec{BC}\right)=\dfrac{\vec{SM} \cdot \vec{BC}}{|\vec{SM}| \cdot |\vec{BC}|}=\dfrac{\vec{SM} \cdot \vec{BC}}{SM \cdot BC}$.\\
		\begin{align*}
			\vec{SM} \cdot \vec{BC} & =\dfrac{1}{2}\left(\vec{SA}+\vec{SB}\right) \cdot \left(\vec{SC}-\vec{SB}\right)\\
			& =\dfrac{1}{2}\left(\vec{SA} \cdot \vec{SC}-\vec{SA} \cdot \vec{SB}+\vec{SB} \cdot \vec{SC}-\vec{SB} \cdot \vec{SB}\right) \\
			& =-\dfrac{1}{2}\vec{SB} \cdot \vec{SB}=-\dfrac{1}{2}SB^2=-\dfrac{a^2}{2}.
		\end{align*}
		Tam giác $SAB$ và $SBC$ vuông cân tại $S$ nên $AB=BC=a\sqrt{2}$.\\
		$\Rightarrow SM=\dfrac{AB}{2}=\dfrac{a\sqrt{2}}{2}$.\\
		Do đó $\cos \left(\vec{SM},\vec{BC}\right)=\dfrac{-\dfrac{a^2}{2}}{\dfrac{a\sqrt{2}}{2} \cdot a\sqrt{2}}=-\dfrac{1}{2}$. Suy ra $\left(\vec{SM},\vec{BC}\right)={120}^\circ$.
	}
\end{ex}
%Câu 21
\begin{ex}
	Cho hình chóp $S.ABC$ có $SA$, $SB$, $SC$ đôi một vuông góc nhau và $SA=SB=SC=a$. Gọi $M$ là trung điểm của $AB$. Góc giữa hai vectơ $\vec{SM}$ và $\vec{BC}$ bằng
	\loigiai{
		\shortans{120}
		{\centering\color{red} HINH O DAY}\\
		Ta có $\cos \left(\vec{SM},\vec{BC}\right)=\dfrac{\vec{SM} \cdot \vec{BC}}{|\vec{SM}| \cdot |\vec{BC}|}=\dfrac{\vec{SM} \cdot \vec{BC}}{SM \cdot BC}$.\\
		\begin{align*}
			\vec{SM} \cdot \vec{BC} &=\dfrac{1}{2}\left(\vec{SA}+\vec{SB}\right) \cdot \left(\vec{SC}-\vec{SB}\right)\\
			& =\dfrac{1}{2}\left(\vec{SA} \cdot \vec{SC}-\vec{SA} \cdot \vec{SB}+\vec{SB} \cdot \vec{SC}-\vec{SB} \cdot \vec{SB}\right) \\
			& =-\dfrac{1}{2}\vec{SB} \cdot \vec{SB}=-\dfrac{1}{2}SB^2=-\dfrac{a^2}{2}.
		\end{align*}
		Tam giác $SAB$ và $SBC$ vuông cân tại $S$ nên $AB=BC=a\sqrt{2}$.\\
		Suy ra trung tuyến $SM=\dfrac{AB}{2}=\dfrac{a\sqrt{2}}{2}$.\\
		Do đó $\cos \left(\vec{SM},\vec{BC}\right)=\dfrac{-\dfrac{a^2}{2}}{\dfrac{a\sqrt{2}}{2} \cdot a\sqrt{2}}=-\dfrac{1}{2}$. Suy ra $\left(\vec{SM},\vec{BC}\right)={{120}^\circ}$.
	}
\end{ex}
%Câu 22
\begin{ex}
	Cho hình hộp $ABCD.A'B'C'D'$. Xét các điểm $M$, $N$ lần lượt thuộc các đường thẳng $A'C$, $C'D$ sao cho đường thẳng $MN$ song song với đường thẳng $BD'$. Khi đó tỉ số $\dfrac{MN}{BD'}$ bằng
	\loigiai{
		{\centering\color{red} HINH O DAY}\\
		Đặt $\vec{BA}=\vec{x}$, $\vec{BB'}=\vec{y}$, $\vec{BC}=\vec{z}$.\\
		Do $\vec{CM}$, $\vec{CA'}$ là hai vectơ cùng phương $\Rightarrow \exists \,k\in \mathbb{R}\colon \,\vec{CM}=k \cdot \vec{CA'}$.\\
		Và $\vec{C'N}$, $\vec{C'D}$ là hai vectơ cùng phương $\Rightarrow \exists \,h\in \mathbb{R}\colon \,\vec{C'N}=h \cdot \vec{C'D}$.\\
		Ta có: $\vec{BD'}=\vec{BA}+\vec{BC}+\vec{BB'}=\vec{x}+\vec{y}+\vec{z}$, (1)\\
		Ta lại có: $\vec{MN}=\vec{CN}-\vec{CM}=\vec{CC'}+\vec{C'N}-\vec{CM}=\vec{CC'}+h \cdot \vec{C'D}-k \cdot \vec{CA'}$\\
		$=\vec{y}+h \cdot (-\vec{y}+\vec{x})-k \cdot \left(\vec{y}-\vec{z}+\vec{x}\right)=(h-k) \cdot \vec{x}+(1-h-k) \cdot \vec{y}+k \cdot \vec{z}$, (2)\\
		Do $MN\parallel B'D$ nên tồn tại $t\in \mathbb{R} \colon \vec{MN}=t \cdot \vec{BD'}$. Từ (1) và (2) ta có$\heva{& h-k=t \\& 1-h-k=t \\& k=t}\Leftrightarrow \heva{& k=t \\& h=2t \\& 1-3t=t}\Rightarrow t=\dfrac{1}{4}\Rightarrow \vec{MN}=\dfrac{1}{4}\vec{BD'}$.\\
		Vậy $\dfrac{MN}{BD'}=\dfrac{1}{4}$.
	}
\end{ex}
% ---------Mục lục chính
% \FULLWIDTH
% \tableofcontents %lệnh in mục lục chính
% \begin{center}
% \includegraphics[width=5cm]{QRcode/12D1X3.png}
% \end{center}
\end{document}