Câu 49. Có bao nhiêu số nguyên dương nhỏ hơn 40 và nguyên tố cùng nhau với 33 (hai số nguyên tố cùng nhau nếu chúng có ước chung lớn nhất bằng 1)?
A. 25 số.   B. 26 số.   C. 24 số.   D. 36 số.
Lời giải
Chọn A.
Vì $33=3.11$ nên các số nguyên dương nhỏ hơn 40 và nguyên tố cùng nhau với 33 là các số không chia hết cho 3 hoặc 11.
Từ 1 đến 40 có $\frac{39}{3}=13$ số chia hết cho 3 và 3 số chia hết cho 11 là 11, 22, 33 (trong đó số 33 vừa chia hết cho 3 vừa chia hết cho 11).
Vậy có $40-13-3+1=25$ số nguyên dương nhỏ hơn 40 và nguyên tố cùng nhau với 33.

Câu 50. Một lớp học có 15 nam và 10 nữ. Số cách chọn hai bạn trực nhật có cả nam và nữ là
A. 300 cách.   B. 25 cách.   C. 150 cách.   D. 50 cách.
Lời giải
Chọn C.
Chọn một bạn nam có 15 cách, chọn một bạn nữ có 10 cách.
Vậy số cách chọn hai bạn trực nhật có cả nam và nữ là $15.10=150$ cách.

Câu 51. Từ các chữ số 0, 1, 2, 3, 5 có thể lập được bao nhiêu số tự nhiên gồm bốn chữ số đôi một khác nhau và không chia hết cho 5?
A. 120 số.   B. 72 số.   C. 69 số.   D. 54 số.
Lời giải
Chọn D.
Đặt $A=\{0,1,2,3,5\}$.
Gọi $\overline{abcd}$ là số cần lập.
Vì $\overline{abcd}$ không chia hết cho 5 nên chữ số $d$ có thể là 1, 2, 3.
Chọn $d \in \{1;2;3\}$ có 3 cách.
Chọn $a \in A\setminus \{0;d\}$ có 3 cách.
Chọn $b \in A\setminus \{a;d\}$ có 3 cách.
Chọn $c \in A\setminus \{a;b;d\}$ có 2 cách.
Vậy số các số tự nhiên cần lập là $3.3.3.2=54$ số.

Câu 52. Cho 30 thẻ đánh số từ 1 đến 30. Số cách chọn ra 1 thẻ hoặc là số chẵn hoặc là số chia hết cho 5 là 
A. 6.   B. 15.   C. 21.   D. 18.
Lời giải
Chọn D.
Từ 1 đến 30 có 15 số chẵn và 6 số chia hết cho 5 (trong đó có 3 số vừa chẵn vừa chia hết cho 5 là 10, 20, 30).
Vậy số cách chọn ra 1 thẻ hoặc là số chẵn hoặc là số chia hết cho 5 là $15+6-3=18$ cách.

Câu 53. Một người 