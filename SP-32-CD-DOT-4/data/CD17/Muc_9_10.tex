\Opensolutionfile{ans}[ans/CD17/Muc_9_10]
\setcounter{ex}{0}
\setcounter{dang}{0}
\section{Mức độ 9,10 điểm}
\begin{dang}
	{Một số bài toán khó}       
\end{dang}     
\begin{ex}%[2D2G3-2]
	[Chuyên Lam Sơn - 2020]
	Cho các số thực $a$, $b$ thỏa mãn $a>b>1$ và $\dfrac{1}{\log_ba}+\dfrac{1}{\log_ab}=\sqrt{2020}$. Giá trị của biểu thức $P=\dfrac{1}{\log_{ab}b}-\dfrac{1}{\log_{ab}a}$ bằng
	\choice
	{$\sqrt{2014}$}
	{\True $\sqrt{2016}$}
	{$\sqrt{2018}$}
	{$\sqrt{2020}$}
	\loigiai{
		Do $a>b>1$ nên $\log_ab>0$, $\log_ba>0$ và $\log_ba>\log_ab$.\\
		Ta có $\dfrac{1}{\log_ba}+\dfrac{1}{\log_ab}=\sqrt{2020}$\\
		$\Leftrightarrow{\log_b}a+\log_ab=\sqrt{2020}$\\
		$\Leftrightarrow\log_b^2a+\log_a^2b+2=2020$\\
		$\Leftrightarrow\log_b^2a+\log_a^2b=2018$\hfill$(*)$\\
		Khi đó, $P=\log_bab-\log_aab=\log_ba+\log_bb-\log_aa-\log_ab=\log_ba-\log_ab$.\\
		Suy ra $P^2=\left(\log_ba-\log_ab\right)^2=\log_b^2a+\log_a^2b-2=2018-2=2016\Rightarrow P=\sqrt{2016}$.
	}
\end{ex}
\begin{ex}%[2D2G3-2]
	[Liên Trường THPT Tp Vinh Nghệ 2019]
	Tìm số nguyên dương $n$ sao cho
	$$\log_{2018}2019+2^2\log_{\sqrt{2018}}2019+3^2\log_{\sqrt[3]{2018}}2019+\cdots+n^2\log_{\sqrt[n]{2018}}2019=1010^22021^2\log_{2018}2019.$$
	\choice
	{$n=2021$}
	{$n=2019$}
	{\True $n=2020$}
	{$n=2018$}
	\loigiai{
		$\log_{2018}2019+2^2\log_{\sqrt{2018}}2019+3^2\log_{\sqrt[3]{2018}}2019+\cdots+n^2\log_{\sqrt[n]{2018}}2019=1010^22021^2\log_{2018}2019$\\
		$\Leftrightarrow{\log_{2018}}2019+2^3\log_{2018}2019+3^3\log_{2018}2019+\cdots+n^3\log_{2018}2019=1010^22021^2\log_{2018}2019$ $\Leftrightarrow\left(1+2^3+3^3+\cdots+n^3\right){\log_{2018}}2019=1010^22021^2\log_{2018}2019$\\
		$\Leftrightarrow 1+2^3+3^3+\cdots+n^3=1010^22021^2$\\
		$\Leftrightarrow{\left(1+2+\cdots+n\right)^2}=1010^22021^2$\\
		$\Leftrightarrow{\left[\dfrac{n\left(n+1\right)}{2}\right]^2}=1010^22021^2$\\
		$\Leftrightarrow\dfrac{n\left(n+1\right)}{2}=1010\cdot2021$\\
		$\Leftrightarrow{n^2}+n-2020\cdot2021=0$\\
		$\Leftrightarrow\left[\begin{aligned}
			&n=2020\\ 
			&n=-2021\text{ (loại).}
		\end{aligned}\right.$
	}
\end{ex}
\begin{ex}%[2D2G3-2]
	Cho hàm số $f(x)=\log_2\left(x-\dfrac{1}{2}+\sqrt{x^2-x+\dfrac{17}{4}}\right)$. Tính
	$$T=f\left(\dfrac{1}{2019}\right)+f\left(\dfrac{2}{2019}\right)+\cdots+f\left(\dfrac{2018}{2019}\right).$$
	\choice
	{$T=\dfrac{2019}{2}$}
	{$T=2019$}
	{\True $T=2018$}
	{$T=1009$}
	\loigiai{
		Ta có
		\begin{align*}
			f(1-x)&=\log_2\left(1-x-\dfrac{1}{2}+\sqrt{\left(1-x\right)^2-\left(1-x\right)+\dfrac{17}{4}}\right)\cr
			&=\log_2\left(\sqrt{x^2-x+\dfrac{17}{4}}-\left(x-\dfrac{1}{2}\right)\right)\cr
			f(x)+f\left(1-x\right)&=\log_2\left(x-\dfrac{1}{2}+\sqrt{x^2-x+\dfrac{17}{4}}\right)+\log_2\left(\sqrt{x^2-x+\dfrac{17}{4}}-\left(x-\dfrac{1}{2}\right)\right)\cr
			&=\log_2\left[\left(x-\dfrac{1}{2}+\sqrt{x^2-x+\dfrac{17}{4}}\right)\left(\sqrt{x^2-x+\dfrac{17}{4}}-\left(x-\dfrac{1}{2}\right)\right)\right]=\log_24=2\cr
			\Rightarrow T&=f\left(\dfrac{1}{2019}\right)+f\left(\dfrac{2}{2019}\right)+\cdots+f\left(\dfrac{2018}{2019}\right)\cr
			&=f\left(\dfrac{1}{2019}\right)+f\left(\dfrac{2018}{2019}\right)+f\left(\dfrac{2}{2019}\right)+f\left(\dfrac{2017}{2019}\right)+\cdots+f\left(\dfrac{1009}{2019}\right)+f\left(\dfrac{1010}{2019}\right)\cr
			&=1009\cdot2=2018.
		\end{align*}
	}
\end{ex}
\begin{ex}%[2D2G3-3]
	[THPT Nguyễn Khuyến 2019]
	Gọi $a$ là giá trị nhỏ nhất của $f(n)=\dfrac{\log_32\cdot\log_33\cdot\log_34\cdot\cdots\cdot\log_3n}{9^n}$ với $n\in\mathbb{N}$ và $n\ge 2$. Hỏi có bao nhiêu giá trị của $n$ để $f(n)=a$.
	\choice
	{\True $2$}
	{$4$}
	{$1$}
	{vô số}
	\loigiai{
		Với $n\in\mathbb{N}$ và $n\ge2$ ta có
		$$f(n)=\dfrac{\log_32\cdot\log_33\cdot\log_34\cdot\cdots\cdot\log_3n}{9^n}=\dfrac{1}{9}{\log_{3^9}}2\cdot\log_{3^9}3\cdot\log_{3^9}4\cdot\cdots\cdot\log_{3^9}n.$$
		Ta có
		\begin{itemize}
			\item Nếu $2\le n\le{3^8}\Rightarrow 0<\log_{3^9}k<1\Rightarrow f(n)=\dfrac{1}{9}{\log_{3^9}}2\cdot\log_{3^9}3\cdot\log_{3^9}4\cdot\cdots\cdot\log_{3^9}n\ge f\left(3^8\right)$.
			\item Nếu $n=3^9\Rightarrow f\left(3^9\right)=f\left(3^8\right)\cdot\log_{3^9}{3^9}=f\left(3^8\right)$.
			\item Nếu $n>3^9\Rightarrow{\log_{3^9}}n>1\Rightarrow f(n)=f\left(3^9\right)\cdot\log_{3^9}\left(3^9+1\right)\cdot\cdots\cdot\log_{3^9}n>f\left(3^9\right)$.
		\end{itemize}
		Từ đó suy ra $\min f(n)=f\left(3^9\right)=f\left(3^8\right)$.
	}
\end{ex}
\begin{ex}%[2D2G3-2]
	[Chuyên Lê Quý Đôn Quảng Trị 2019]
	Cho $x$, $y$ và $z$ là các số thực lớn hơn $1$ và gọi $w$ là số thực dương sao cho $\log_xw=24$, $\log_yw=40$ và $\log_{xyz}w=12$. Tính $\log_zw$.
	\choice
	{$52$}
	{$-60$}
	{\True $60$}
	{$-52$}
	\loigiai{
		Ta có
		\begin{align*}
			\log_xw=24&\Rightarrow{\log_w}x=\dfrac{1}{24}\cr
			\log_yw=40&\Rightarrow{\log_w}y=\dfrac{1}{40}.
		\end{align*}
		Lại do
		\begin{align*}
			\log_{xyz}w=12&\Leftrightarrow\dfrac{1}{\log_{_w}\left(xyz\right)}=12\Leftrightarrow\dfrac{1}{\log_{_w}x+\log_{_w}y+\log_{_w}z}=12\cr
			&\Leftrightarrow\dfrac{1}{\dfrac{1}{24}+\dfrac{1}{40}+\log_{_w}z}=12\\
			&\Leftrightarrow{\log_{_w}}z=\dfrac{1}{60}\Rightarrow{\log_z}w=60.
		\end{align*}
	}
\end{ex}
\begin{ex}%[2D2G3-2]
	Cho $f(1)=1$, $f\left(m+n\right)=f(m)+f(n)+mn$ với mọi $m,n\in{\mathbb{N}^*}$. Tính giá trị của biểu thức $T=\log\left[\dfrac{f\left(96\right)-f\left(69\right)-241}{2}\right]$.
	\choice
	{$T=9$}
	{\True $T=3$}
	{$T=10$}
	{$T=4$}
	\loigiai{
		Vì $f(1)=1$, $f\left(m+n\right)=f(m)+f(n)+mn$ nên
		\begin{align*}
			f\left(96\right)&=f\left(95+1\right)=f\left(95\right)+f(1)+95=f\left(95\right)+96=f\left(94\right)+95+96\cr
			&=\cdots=f(1)+2+\cdots+95+96\cr
			&=1+2+\cdots+95+96=\dfrac{96\cdot97}{2}=4\ 656.
		\end{align*}
		Tương tự $f\left(69\right)=1+2+\cdots+68+69=\dfrac{69\cdot70}{2}=2\ 415$.\\
		Vậy $T=\log\left[\dfrac{f\left(96\right)-f\left(69\right)-241}{2}\right]=\log\left(\dfrac{4\ 656-2\ 415-241}{2}\right)=\log 1\ 000=3$.
	}
\end{ex}
\begin{ex}%[2D2G3-2]
	[Chuyên Lê Quý Dôn Quảng Trị 2019]
	Cho các số thực dương $x,y,z$ thỏa mãn đồng thời $\dfrac{1}{\log_2x}+\dfrac{1}{\log_2y}+\dfrac{1}{\log_2z}=\dfrac{1}{2020}$ và $\log_2(xyz)=~2020$. Tính $\log_2\left(xyz\left(x+y+z\right)-xy-yz-zx+1\right)$.
	\choice
	{\True $4040$}
	{$1010$}
	{$2020$}
	{$2020^2$}
	\loigiai{
		Đặt $a=\log_2x$; $b=\log_2y$; $c=\log_2z$. Ta có $\dfrac{1}{a}+\dfrac{1}{b}+\dfrac{1}{c}=\dfrac{1}{2020}$ và $a+b+c=2020$\\
		$\begin{aligned}
			&\left(\dfrac{1}{a}+\dfrac{1}{b}+\dfrac{1}{c}\right)\left(a+b+c\right)=1\Leftrightarrow\left(a+b+c\right)\left(ab+ac+bc\right)=abc\\ 
			&\Leftrightarrow{a^2}b+a{b^2}+abc+abc+b^2c+b{c^2}+a^2c+a{c^2}=0\\ 
			&\Leftrightarrow\left(a+b\right)\left(b+c\right)\left(c+a\right)=0.
		\end{aligned}$\\
		Vì vai trò $a,b,c$ như nhau nên giả sử $a+b=0\Rightarrow c=2020\Rightarrow z=2^{2020}$ và $xy=1$.\\
		$\begin{aligned}
			{\log_2}\left(xyz\left(x+y+z\right)-xy-yz-zx+1\right)&=\log_2\left(z(x+y+z)-1-yz-zx+1\right)\\ 
			&=\log_2\left(z^2\right)=2\log_2z=4040.
		\end{aligned}$
	}
\end{ex}
\begin{ex}%[2D2G3-2]
	[Bạc Liêu – Ninh Bình 2019]
	Cho ba số thực dương $x,y,z$ theo thứ tự lập thành một cấp số nhân, đồng thời với mỗi số thực dương $a$ $(a\ne 1)$ thì $\log_ax,{\log_{\sqrt{a}}}y,{\log_{\sqrt[3]{a}}}z$ theo thứ tự lập thành một cấp số cộng. Tính giá trị của biểu thức $P=\dfrac{1959x}{y}+\dfrac{2019y}{z}+\dfrac{60z}{x}$.
	\choice
	{$60$}
	{$2019$}
	{\True $4038$}
	{$\dfrac{2019}{2}$}
	\loigiai{
		Ta có $x,y,z$ là ba số thực dương, theo thứ tự lập thành một cấp số nhân nên $y^2=xz$.\hfill(1)\\
		Với mỗi số thực $a$ $(a\ne 1),$ $\log_ax,{\log_{\sqrt{a}}}y,{\log_{\sqrt[3]{a}}}z$ theo thứ tự lập thành một cấp số cộng nên
		$${2}{\log_{\sqrt{a}}}y=\log_ax+\log_{\sqrt[3]{a}}z\Leftrightarrow 4\log_ay=\log_ax+3\log_az.\eqno(2)$$
		Thay $(1)$ vào $(2)$ ta được $2\log_axz=\log_ax+3\log_az\Leftrightarrow{\log_a}x=\log_az\Leftrightarrow x=z$.\\
		Từ $(1)$ ta suy ra $y=x=z$.\\
		Từ đó $P=1\ 959+2\ 019+60=4\ 038$.
	}
\end{ex}
\begin{ex}%[2D2G3-2]
	[THPT Hai Bà Trưng-Huế-2019]
	Cho hàm số $f(x)=\dfrac{1}{2}{\log_2}\left(\dfrac{2x}{1-x}\right)$ và hai số thực $m,n$ thuộc khoảng $\left(0;1\right)$ sao cho $m+n=1$. Tính $f(m)+f(n)$.
	\choice
	{$2$}
	{$0$}
	{\True $1$}
	{$\dfrac{1}{2}$}
	\loigiai{
		$f(m)+f(n)=\dfrac{1}{2}{\log_2}\left(\dfrac{2m}{1-m}\right)+\dfrac{1}{2}{\log_2}\left(\dfrac{2n}{1-n}\right)$\\
		$=\dfrac{1}{2}\left[\log_2\left(\dfrac{2m}{1-m}\right)+\log_2\left(\dfrac{2n}{1-n}\right)\right]$\\
		$=\dfrac{1}{2}{\log_2}\left(\dfrac{2m}{1-m}\cdot\dfrac{2n}{1-n}\right)$\\
		$=\dfrac{1}{2}{\log_2}\left(\dfrac{4mn}{1-m-n+mn}\right)$, vì $m+n=1$\\
		$=\dfrac{1}{2}{\log_2}\left(\dfrac{4mn}{mn}\right)=\dfrac{1}{2}{\log_2}4=\dfrac{1}{2}\cdot2=1$.
	}
\end{ex}
\begin{ex}%[2D2G3-2]
	[Chuyên-Vĩnh Phúc-2019]
	Gọi $n$ là số nguyên dương sao cho $\dfrac{1}{\log_3x}+\dfrac{1}{\log_{3^2}x}+\dfrac{1}{\log_{3^3}x}+\cdots+\dfrac{1}{\log_{3^n}x}=\dfrac{190}{\log_3x}$ đúng với mọi $x$ dương, $x\ne 1$. Tìm giá trị của biểu thức $P=2n+3$.
	\choice
	{$P=32$}
	{$P=23$}
	{$P=43$}
	{\True $P=41$}
	\loigiai{
		$\begin{aligned}
			&\dfrac{1}{\log_3x}+\dfrac{1}{\log_{3^2}x}+\dfrac{1}{\log_{3^3}x}+\cdots+\dfrac{1}{\log_{3^n}x}=\dfrac{190}{\log_3x}\\ 
			&\Leftrightarrow{\log_x}3+2\log_x3+3\log_x3+\cdots+n{\log_x}3=190\log_x3\\ 
			&\Leftrightarrow{\log_x}3\left(1+2+3+\cdots+n\right)=190\log_x3\\ 
			&\Leftrightarrow 1+2+3+\cdots+n=190\\ 
			&\Leftrightarrow\dfrac{n\left(n+1\right)}{2}=190\\ 
		\end{aligned}$\\
		$\Leftrightarrow{n^2}+n-380=0$\\
		$\Leftrightarrow\left[\begin{aligned}
			& n=19\\ 
			& n=-20\\ 
		\end{aligned}\right.\Rightarrow n=19$ (do $n$ nguyên dương) $\Rightarrow P=2n+3=41$.
	}
\end{ex}
\begin{ex}%[2D2G3-2]
	Cho $x$, $y$, $z$ là ba số thực dương lập thành cấp số nhân; $\log_ax$, $\log_{\sqrt{a}}y$ , $\log_{\sqrt[3]{a}}z$ lập thành cấp số cộng, với $a$ là số thực dương khác 1. Giá trị của $p=\dfrac{9{x}}{y}+\dfrac{y}{z}+\dfrac{3{z}}{x}$ là
	\choice
	{\True $13$}
	{$3$}
	{$12$}
	{$10$}
	\loigiai{
		$x$, $y$, $z$ là ba số thực dương lập thành cấp số nhân nên ta có $xz=y^2$.\hfill$(1)$\\
		$\log_ax$, $\log_{\sqrt{a}}y$, $\log_{\sqrt[3]{a}}z$ lập thành cấp số cộng nên
		$$\log_ax+\log_{\sqrt[3]{a}}z=2\log_{\sqrt{a}}y\Leftrightarrow{\log_a}x+3\log_az=4\log_ay\Leftrightarrow x{{z}^3}=y^4.\eqno(2)$$
		Từ $(1)$ và $(2)$ ta suy ra $x=y=z$.\\
		Vậy $p=\dfrac{9{x}}{y}+\dfrac{y}{z}+\dfrac{3{z}}{x}=9+1+3=13$.
	}
\end{ex}
\begin{ex}%[2D2G3-2]
	[Chuyên Nguyễn Huệ 2019]
	Cho $f(1)=1$; $f(m+n)=f(m)+f(n)+mn$ với mọi $m,n\in{N^*}$. Tính giá trị của biểu thức\\
	$T=\log\left[\dfrac{f\left(2019\right)-f\left(2009\right)-145}{2}\right]$ 
	\choice
	{$3$}
	{\True $4$}
	{$5$}
	{$10$}
	\loigiai{
		Ta có $f(2019)=f(2009+10)=f(2009)+f(10)+20090$.\\
		Do đó $f(2019)-f(2009)-145=f(10)+20090-145$.\\
		$\begin{aligned}
			&f(10)=f(9)+f(1)+9\\ 
			&f(9)=f(8)+f(1)+8\\ 
			&\cdots\\ 
			&f(3)=f(2)+f(1)+2\\ 
			&f(2)=f(1)+f(1)+1\\ 
		\end{aligned}$\\
		Từ đó cộng vế với vế ta được: $f(10)=10\cdot f(1)+1+2+\cdots+8+9=55.$\\
		Vậy $\log\left[\dfrac{f(2019)-f(2009)-145}{2}\right]=\log\dfrac{20090-145+55}{2}=\log 10000=4$.
	}
\end{ex}
\begin{ex}%[2D2G3-2]
	Có bao nhiêu số nguyên dương $n$ để $\log_n256$ là một số nguyên dương?
	\choice
	{$2$}
	{$3$}
	{\True $4$}
	{$1$}
	\loigiai{
		$\log_n256=8\cdot\log_n2=\dfrac{8}{\log_2n}$ là số nguyên dương\\
		$\Leftrightarrow{\log_2}n\in\left\{1;2;4;8\right\}\Leftrightarrow n\in\left\{2;4;16;256\right\}$.\\
		Vậy có $4$ số nguyên dương.
	}
\end{ex}
\begin{ex}%[2D2G3-2]
	Cho tam giác $ABC$ có $BC=a$, $CA=b$, $AB=c$. Nếu $a$, $b$, $c$ theo thứ tự lập thành một cấp số nhân thì
	\choice
	{$\ln\sin A.\ln\sin C=\left(\ln\sin B\right)^2$}
	{$\ln\sin A.\ln\sin C=2\ln\sin B$}
	{\True $\ln\sin A+\ln\sin C=2\ln\sin B$}
	{$\ln\sin A+\ln\sin C=\ln\left(2\sin B\right)$}
	\loigiai{
		Theo định lý sin trong tam giác $ABC$ ta có: $\left\{\begin{aligned}
			&a=2R\sin A\\ 
			&b=2R\sin B\\ 
			&c=2R\sin C\\ 
		\end{aligned}\right.$, với $R$ là bán kính đường tròn ngoại tiếp tam giác $ABC$.\\
		Vì $a$, $b$, $c$ theo thứ tự lập thành một cấp số nhân nên ta có
		$$ac=b^2\Rightarrow\left(2R\sin A\right)\left(2R\sin C\right)=\left(2R\sin B\right)^2\Rightarrow\sin A\sin C=\left(\sin B\right)^2.$$
		Do $0^\circ <\sin A$, $\sin B$, $\sin C\le 180^\circ $ nên $\sin A$, $\sin B$, $\sin C>0$.\\
		Vì thế ta có thể suy ra $\ln\left(\sin A.\sin C\right)=\ln\left[\left(\sin B\right)^2\right]$ $\Rightarrow\ln\sin A+\ln\sin C=2\ln\sin B$.
	}
\end{ex}
\begin{ex}
	[Chuyên Lương Văn Chánh - Phú Yên - 2018]
	Cho $x=2018!$. Tính $A=\dfrac{1}{\log_{2^{2018}}x}+\dfrac{1}{\log_{3^{2018}}x}+\cdots+\dfrac{1}{\log_{2017^{2018}}x}+\dfrac{1}{\log_{2018^{2018}}x}$.
	\choice
	{$A=\dfrac{1}{2017}$}
	{\True $A=2018$}
	{$A=\dfrac{1}{2018}$}
	{$A=2017$}
	\loigiai{
		Ta  có
		\begin{align*}
			A&=\dfrac{1}{\log_{2^{2018}}x}+\dfrac{1}{\log_{3^{2018}}x}+\cdots+\dfrac{1}{\log_{2017^{2018}}x}+\dfrac{1}{\log_{2018^{2018}}x}\cr
			&=\log_x{2^{2018}}+\log_x{3^{2018}}+\cdots+\log_x{2017^{2018}}+\log_x{2018^{2018}}\cr
			&=2018\log_x2+2018\log_x3+\cdots+2018\log_x2017+2018\log_x2018\cr
			&=2018\left(\log_x2+\log_x3+\cdots+\log_x2017+\log_x2018\right)\cr
			&=2018\log_x\left(2\cdot3\cdot\cdots\cdot2017\cdot2018\right).
		\end{align*}
	}
\end{ex}
\begin{ex}%[2D2G3-2]
	[Chuyên Hùng Vương - Gia Lai - 2018]
	Tìm bộ ba số nguyên dương $(a;b;c)$ thỏa mãn
	$$\log 1+\log (1+3)+\log (1+3+5)+\cdots+\log (1+3+5+\cdots+19)-2\log 5040=a+b\log 2+c\log 3$$
	\choice
	{\True $(2;6;4)$}
	{$(1;3;2)$}
	{$(2;4;4)$}
	{$(2;4;3)$}
	\loigiai{
		Ta có\\
		$\log 1+\log (1+3)+\log (1+3+5)+\cdots+\log (1+3+5+\cdots+19)-2\log 5040=a+b\log 2+c\log 3$ $\Leftrightarrow\log 1+\log{2^2}+\log{3^2}+\cdots+\log{10^2}-2\log 5040=a+b\log 2+c\log 3$\\
		$\Leftrightarrow\log\left(1\cdot2^2\cdot3^2\cdot\cdots\cdot10^2\right)-2\log 5040=a+b\log 2+c\log 3$\\
		$\Leftrightarrow\log{\left(1\cdot2\cdot3\cdot\cdots\cdot10\right)^2}-2\log 5040=a+b\log 2+c\log 3$\\
		$\Leftrightarrow 2\log\left(1\cdot2\cdot3\cdot\cdots\cdot10\right)-2\log 5040=a+b\log 2+c\log 3$\\
		$\Leftrightarrow 2\left(\log 10!-\log 7!\right)=a+b\log 2+c\log 3\Leftrightarrow 2\log\left(8\cdot9\cdot10\right)=a+b\log 2+c\log 3$\\
		$\Leftrightarrow 2+6\log 2+4\log 3=a+b\log 2+c\log 3$.\\
		Vậy $a=2$, $b=6$, $c=4$.
	}
\end{ex}
\begin{ex}%[2D2G3-2]
	[Phan Đình Phùng - Hà Tĩnh - 2018]
	Tổng $S=1+2^2\log_{\sqrt{2}}2+3^2\log_{\sqrt[3]{2}}2+\cdots+2018^2\log_{\sqrt[2018]{2}}2$ dưới đây.
	\choice
	{$1008^22018^2$}
	{\True $1009^22019^2$}
	{$1009^22018^2$}
	{$2019^2$}
	\loigiai{
		Ta có $1^3+2^3+3^3+\cdots+n^3=\dfrac{\left(n\left(n+1\right)\right)^2}{4}$.\\
		Mặt khác
		\begin{align*}
			S&=1+2^2\log_{\sqrt{2}}2+3^2\log_{\sqrt[3]{2}}2+\cdots+2018^2\log_{\sqrt[2018]{2}}2\cr
			&=1+2^2\log_{2^{\frac{1}{2}}}2+3^2\log_{2^{\frac{1}{3}}}2+\cdots+2018^2\log_{2^{\frac{1}{2018}}}2\cr
			&=1+2^3\log_22+3^3\log_22+\cdots.+2018^3\log_22\cr
			&=1+2^3+3^3+\cdots+2018^3\cr
			&=\left[\dfrac{2018\left(2018+1\right)}{2}\right]^2=1009^22019^2.
		\end{align*}
	}
\end{ex}
\begin{ex}%[2D2G3-1]
	[ChuyêN KHTN - 2018]
	Số $20172018^{20162017}$ có bao nhiêu chữ số?
	\choice
	{\True $147278481$}
	{$147278480$}
	{$147347190$}
	{$147347191$}
	\loigiai{
		Số chữ số của một số tự nhiên $x$ là $\left[\log x\right]+1$ ($\left[\log x\right]$ là phần nguyên của $\log x$).\\
		Vậy số chữ số của số $20172018^{20162017}$ là\\
		$\left[\log{20172018^{20162017}}\right]+1=\left[20162017\log\left(20172018\right)\right]+1=147278481$.
	}
\end{ex}
\begin{ex}%[2D2G3-1]
	[THPT Quảng Xương 1 - Thanh Hóa - 2021]
	Cho các số thực $a,b,c$ thuộc khoảng $\left(1;+\infty\right)$ và $\log_{\sqrt{a}}^2b+\log_bc.\log_b\left(\dfrac{c^2}{b}\right)+9\log_ac=4\log_ab.$ Giá trị của biểu thức $\log_ab+\log_b{c^2}$ bằng
	\choice
	{\True $1$}
	{$\dfrac{1}{2}$}
	{$2$}
	{$3$}        
	\loigiai{                  
		Đặt $\log_ab=x,\log_bc=y\Rightarrow{\log_a}c=xy$. Điều kiện $x,y>0$.\\
		Bài toán trở thành\\
		Cho $4x^2+y(2y-1)+9xy-4x=0$. Tính $P=x+2y$.\\
		Rút $x=P-2y$ thay vào giả thiết, ta có\\
		$\begin{aligned}
			&4\left(P-2y\right)^2+y(2y-1)+9\left(P-2y\right)y-4\left(P-2y\right)=0\\ 
			&\Leftrightarrow 4P^2-7Py-4P+7y=0.\\ 
			&\Leftrightarrow\left(P-1\right)\left(4P-7y\right)=0\\ 
			&\Leftrightarrow\left[\begin{aligned}
				&P=1\\ 
				&4P-7y=0.\\      
			\end{aligned}\right.  
		\end{aligned}$\\
		Xét trường hợp: $4P-7y=0\Leftrightarrow4x+y=0$, loại vì $x,y>0$.\\
		Vậy $P=1$.
	}
\end{ex} 
\Closesolutionfile{ans}
\indapan{10}{ans/CD17/Muc_9_10}