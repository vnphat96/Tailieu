\Opensolutionfile{ans}[ans/CD17/Muc_7_8]
\setcounter{ex}{0}
\setcounter{dang}{0}
\section{Mức độ 7,8 điểm}
\begin{dang}
	{Biểu diễn biểu thức logarit này theo logarit khác}     
\end{dang}     
\begin{ex}%[2D2B3-2]
	[Đề Tham Khảo 2019]
	Đặt $\log_32=a$ khi đó $\log_{16}27$ bằng
	\choice
	{$\dfrac{3a}{4}$}
	{\True $\dfrac{3}{4a}$}
	{$\dfrac{4}{3a}$}
	{$\dfrac{4a}{3}$}
	\loigiai{
		Ta có $\log_{16}27=\dfrac{3}{4}{\log_2}3=\dfrac{3}{4.\log_32}=\dfrac{3}{4a}$.
	}
\end{ex}
\begin{ex}%[2D2B3-2]
	[Đề Minh Họa 2017]
	Đặt $a=\log_23,b=\log_53$. Hãy biểu diễn $\log_645$ theo $a$ và $b$.
	\choice
	{$\log_645=\dfrac{2a^2-2ab}{ab}$}
	{\True $\log_645=\dfrac{a+2ab}{ab+b}$}
	{$\log_645=\dfrac{2a^2-2ab}{ab+b}$}
	{$\log_645=\dfrac{a+2ab}{ab}$}
	\loigiai{
		$\log_645=\dfrac{\log_2\left(3^2.5\right)}{\log_2\left(2.3\right)}=\dfrac{2\log_23+\log_25}{1+\log_23}=\dfrac{2a+\log_23.\log_35}{1+a}=\dfrac{2a+\frac{\log_23}{\log_53}}{1+a}=\dfrac{2a+\frac{a}{b}}{1+a}=\dfrac{a+2ab}{ab+b}.$
	}
\end{ex}
\begin{ex}%[2D2B3-2]
	[Chuyên Đại Học Vinh 2019]
	Đặt $a=\log_32$, khi đó $\log_648$ bằng
	\choice
	{$\dfrac{3a-1}{a-1}$}
	{$\dfrac{3a+1}{a+1}$}
	{$\dfrac{4a-1}{a-1}$}
	{\True $\dfrac{4a+1}{a+1}$}
	\loigiai{
		\begin{align*}
			\log_648&=\log_6(6\cdot8)=\log_66+\log_68=1+\dfrac{1}{\log_86}=1+\dfrac{1}{\log_{2^3}(2\cdot3)}=1+\dfrac{1}{\dfrac{1}{3}\left(1+\log_23\right)}\cr
			&=\dfrac{1+\log_23+3}{1+\log_23}=\dfrac{4+\dfrac{1}{a}}{1+\dfrac{1}{a}}=\dfrac{4a+1}{a+1}.
		\end{align*}
	}
\end{ex}
\begin{ex}%[2D2K3-2]
	[Chuyên Phan Bội Châu-2019]
	Cho $\log_35=a$, $\log_36=b$, $\log_322=c$. Tính $P=\log_3\left(\dfrac{90}{11}\right)$ theo $a$, $b$, $c$?
	\choice
	{$P=2a-b+c$}
	{$P=2a+b+c$}
	{$P=2a+b-c$}
	{\True $P=a+2b-c$}
	\loigiai{
		Ta có
		\begin{align*}
			\log_36=b&\Leftrightarrow{\log_3}2+1=b\Leftrightarrow{\log_3}2=b-1,\cr
			\log_322=c&\Leftrightarrow{\log_3}2+\log_311=c\cr
			&\Rightarrow{\log_3}11=c-\log_32=c-b+1.
		\end{align*}
		Khi đó $P=\log_3\left(\dfrac{90}{11}\right)=\log_390-\log_311=2+\log_32+\log_35-\log_311=2b+a-c$.
	}
\end{ex}
\begin{ex}%[2D2K3-2]
	[Lương Thế Vinh Hà Nội 2019]
	Với $\log_{27}5=a$, $\log_37=b$ và $\log_23=c$, giá trị của $\log_635$ bằng
	\choice
	{\True $\dfrac{\left(3a+b\right)c}{1+c}$}
	{$\dfrac{\left(3a+b\right)c}{1+b}$}
	{$\dfrac{\left(3a+b\right)c}{1+a}$}
	{$\dfrac{\left(3b+a\right)c}{1+c}$}
	\loigiai{
		Ta có
		\begin{align*}
			\log_{27}5=a&\Rightarrow a=\dfrac{1}{3}{\log_3}5\Rightarrow 3a=\log_35\Rightarrow{\log_5}3=\dfrac{1}{3a},\cr
			\log_37=b&\Rightarrow{\log_7}3=\dfrac{1}{b}.\cr
		\end{align*}
		Khi đó
		\begin{align*}
			bc&=\log_23\cdot\log_37=\log_27\Rightarrow{\log_7}2=\dfrac{1}{bc},\cr
			3ac&=\log_35\cdot\log_23=\log_25\Rightarrow{\log_5}2=\dfrac{1}{3ac},\cr
			\log_635&=\log_65+\log_67=\dfrac{1}{\log_56}+\dfrac{1}{\log_76}\cr
			&=\dfrac{1}{\log_52+\log_53}+\dfrac{1}{\log_73+\log_72}=\dfrac{1}{\dfrac{1}{3ac}+\dfrac{1}{3a}}+\dfrac{1}{\dfrac{1}{b}+\dfrac{1}{bc}}=\dfrac{\left(3a+b\right)c}{c+1}.
		\end{align*}
	}
\end{ex}
\begin{ex}%[2D2K3-2]
	[THPT Nguyễn Khuyến 2019]
	Đặt $a=\log_23$; $b=\log_53$. Nếu biểu diễn $\log_645=\dfrac{a\left(m+nb\right)}{b\left(a+p\right)}$ thì $m+n+p$ bằng
	\choice
	{$3$}
	{\True $4$}
	{$6$}
	{$-3$}
	\loigiai{
		Ta có $\log_645=\dfrac{\log_345}{\log_36}=\dfrac{\log_39+\log_35}{\log_32+\log_33}=\dfrac{2+\dfrac{1}{b}}{\dfrac{1}{a}+1}=\dfrac{a\left(2b+1\right)}{b\left(1+a\right)}$.\\
		Suy ra $m=1$, $n=2$, $p=1\Rightarrow m+n+p=4$.
	}
\end{ex}
\begin{ex}%[2D2K3-2]
	[THPT Thiệu Hóa – Thanh Hóa 2019]
	Cho các số thực dương $a$, $b$ thỏa mãn $\log_3a=x$, $\log_3b=y$. Tính $P=\log_3\left(3a^4b^5\right)$.
	\choice
	{$P=3x^4y^5$}
	{$P=3+x^4+y^5$}
	{$P=60xy$}
	{\True $P=1+4x+5y$}
	\loigiai{
		$P=\log_3\left(3a^4b^5\right)=\log_33+\log_3a^4+\log_3b^5=1+4\log_3a+5\log_3b=1+4x+5y$.
	}
\end{ex}
\begin{ex}%[2D2K3-2]
	[THPT An Lão Hải Phòng 2019]
	Biết $\log_63=a,\log_65=b$. Tính $\log_35$ theo $a,b$.
	\choice
	{\True $\dfrac{b}{a}$}
	{$\dfrac{b}{1+a}$}
	{$\dfrac{b}{1-a}$}
	{$\dfrac{b}{a-1}$}
	\loigiai{
		Ta có ${\log_3}5=\dfrac{\log_65}{\log_63}=\dfrac{b}{a}$.
	}
\end{ex}
\begin{ex}%[2D2K3-2]
	Cho $\log_{12}3=a$. Tính $\log_{24}18$ theo $a$.
	\choice
	{$\dfrac{3a-1}{3-a}$}
	{\True $\dfrac{3a+1}{3-a}$}
	{$\dfrac{3a+1}{3+a}$}
	{$\dfrac{3a-1}{3+a}$}
	\loigiai{     
		Ta có $a=\log_{12}3=\dfrac{\log_23}{\log_212}=\dfrac{\log_23}{\log_2\left(2^2\cdot3\right)}=\dfrac{\log_23}{\log_2\left(2^2\right)+\log_23}=\dfrac{\log_23}{2+\log_23}\Rightarrow \log_23=\dfrac{2a}{1-a}$.\\
		Ta có $\log_{24}18=\dfrac{\log_218}{\log_224}=\dfrac{\log_2\left(2\cdot3^2\right)}{\log_2\left(2^3\cdot3\right)}=\dfrac{1+2\log_23}{3+\log_23}=\dfrac{1+2\cdot\dfrac{2a}{1-a}}{3+\dfrac{2a}{1-a}}=\dfrac{3a+1}{3-a}$.\\
		Vậy $\log_{24}18=\dfrac{3a+1}{3-a}$.
	}
\end{ex}       
\begin{ex}%[2D2K3-2]
	[THPT Gia Lộc Hải Dương 2019]
	Đặt $a=\log_23$ và $b=\log_53$. Hãy biểu diễn $\log_645$ theo $a$ và $b$.
	\choice
	{$\log_645=\dfrac{2a^2-2ab}{ab}$}
	{$\log_645=\dfrac{a+2ab}{ab}$}
	{\True $\log_645=\dfrac{a+2ab}{ab+b}$}
	{$\log_645=\dfrac{2a^2-2ab}{ab+b}$}
	\loigiai{
		Ta  có
		\begin{align*}
			{\log_6}45&=\dfrac{\log_345}{\log_36}=\dfrac{\log_33^2\cdot5}{\log_32\cdot3}=\dfrac{\log_33^2+\log_35}{\log_32+\log_33}\\ 
			&=\dfrac{2+\dfrac{1}{\log_53}}{\dfrac{1}{\log_23}+1}=\dfrac{2+\dfrac{1}{b}}{\dfrac{1}{a}+1}=\dfrac{\left(\dfrac{2b+1}{b}\right)}{\left(\dfrac{a+1}{a}\right)}=\dfrac{\left(2b+1\right)a}{b\left(a+1\right)}=\dfrac{a+2ab}{b+ab}.
		\end{align*}
	}
\end{ex}
\begin{ex}%[2D2K3-2]
	[HSG Bắc Ninh 2019]
	Đặt $a=\ln 2$, $b=\ln 5$, hãy biểu diễn
	$$I=\ln\dfrac{1}{2}+\ln\dfrac{2}{3}+\ln\dfrac{3}{4}+\cdots+\ln\dfrac{98}{99}+\ln\dfrac{99}{100}$$
	theo $a$ và $b$.
	\choice
	{\True $-2\left(a+b\right)$}
	{$-2\left(a-b\right)$}
	{$2\left(a+b\right)$}
	{$2\left(a-b\right)$}
	\loigiai{
		\begin{align*}
			I&=\ln\dfrac{1}{2}+\ln\dfrac{2}{3}+\ln\dfrac{3}{4}+\cdots+\ln\dfrac{98}{99}+\ln\dfrac{99}{100}\cr
			&=\ln\left(\dfrac{1}{2}\cdot\dfrac{2}{3}\cdot\dfrac{3}{4}\cdots\dfrac{98}{99}\cdot\dfrac{99}{100}\right)=\ln\dfrac{1}{100}=\ln{10^{-2}}\cr
			&=-2\ln 10=-2\left(\ln 2+\ln 5\right)=-2\left(a+b\right).
		\end{align*}
	}
\end{ex}
\begin{ex}%[2D2K3-2]
	[Chuyên Bắc Ninh 2019]
	Đặt $a=\log_23$; $b=\log_35$ Biểu diễn đúng của $\log_{20}12$ theo $a$, $b$ là
	\choice
	{$\dfrac{ab+1}{b-2}$}
	{$\dfrac{a+b}{b+2}$}
	{$\dfrac{a+1}{b-2}$}
	{\True $\dfrac{a+2}{ab+2}$}
	\loigiai{
		Ta có $\log_{20}12=\log_{20}3+2\log_{20}2=\dfrac{1}{2\log_32+\log_35}+\dfrac{2}{\log_25+2}=\dfrac{1}{2\cdot\dfrac{1}{a}+b}+\dfrac{2}{ab+2}=\dfrac{a+2}{ab+2}$.
	}
\end{ex}
\begin{ex}%[2D2K3-2]
	[Sở Bình Phước 2019]
	Cho $\log_23=a$, $\log_25=b$, khi đó $\log_{15}8$ bằng
	\choice
	{$\dfrac{a+b}{3}$}
	{$\dfrac{1}{3(a+b)}$}
	{$3(a+b)$}
	{\True $\dfrac{3}{a+b}$}
	\loigiai{
		$\log_{15}8=3\log_{15}2=\dfrac{3}{\log_215}=\dfrac{3}{\log_23+\log_25}=\dfrac{3}{a+b}$.
	}
\end{ex}
\begin{ex}%[2D2K3-2]
	[Chuyên Lê Quý Đôn Điện Biên 2019]
	Giả sử $\log_{27}5=a$; $\log_87=b$; $\log_23=c$. Hãy biểu diễn $\log_{12}35$ theo $a,b,c$.
	\choice
	{\True $\dfrac{3b+3ac}{c+2}$}
	{$\dfrac{3b+3ac}{c+1}$}
	{$\dfrac{3b+2ac}{c+3}$}
	{$\dfrac{3b+2ac}{c+2}$}
	\loigiai{
		$\log_{27}5=a\Leftrightarrow\dfrac{1}{3}{\log_3}5=a\Leftrightarrow\dfrac{\log_25}{\log_23}=3a\Leftrightarrow{\log_2}5=3ac$.\\
		$\log_87=b\Leftrightarrow\dfrac{1}{3}{\log_2}7=b\Leftrightarrow{\log_2}7=3b$.\\
		Xét $\log_{12}35=\dfrac{\log_235}{\log_212}=\dfrac{\log_2\left(5.7\right)}{\log_2\left(3.2^2\right)}=\dfrac{\log_25+\log_27}{\log_23+2}=\dfrac{3ac+3b}{c+2}$.
	}
\end{ex}
\begin{ex}%[2D2K3-2]
	[Chuyen Phan Bội Châu Nghệ An 2019]
	Cho $\log_35=a$, $\log_36=b$, $\log_322=c$. Tính $P=\log_3\left(\dfrac{90}{11}\right)$ theo $a$, $b$, $c$.
	\choice
	{$P=2a+b-c$}
	{\True $P=a+2b-c$}
	{$P=2a+b+c$}
	{$P=2a-b+c$}
	\loigiai{
		Ta có
		\begin{align*}
			P&=\log_3\left(\dfrac{90}{11}\right)=\log_3\left(\dfrac{180}{22}\right)=\log_3180-\log_322=\log_3\left(36\cdot5\right)-\log_322\cr
			&=\log_336+\log_35-\log_322=\log_3\left(6^2\right)+\log_35-\log_322=2\log_36+\log_35-\log_322\cr
			&=a+2b-c.
		\end{align*}
		Vậy $P=a+2b-c$.
	}
\end{ex}
\begin{ex}%[2D2K3-2]
	[THPT - Yên Định Thanh Hóa 2019]
	Đặt $a=\log_23$; $b=\log_35$. Biểu diễn $\log_{20}12$ theo $a,b$.
	\choice
	{$\log_{20}12=\dfrac{a+b}{b+2}$}
	{$\log_{20}12=\dfrac{ab+1}{b-2}$}
	{$\log_{20}12=\dfrac{a+1}{b-2}$}
	{\True $\log_{20}12=\dfrac{a+2}{ab+2}$}
	\loigiai{
		Ta có $\log_{20}12=\dfrac{\log_212}{\log_220}=\dfrac{\log_2(4\cdot3)}{\log_2(4\cdot5)}=\dfrac{2+\log_23}{2+\log_25}=\dfrac{2+\log_23}{2+\log_23\cdot\log_35}=\dfrac{a+2}{ab+2}$.
	}
\end{ex}
\begin{ex}%[2D2K3-2][Sở Hà Nội 2019]%Câu 17
	Nếu $\log_23=a$ thì $\log_{72}108$ bằng
	\choice
	{$\dfrac{2+a}{3+a}$}
	{\True $\dfrac{2+3a}{3+2a}$}
	{$\dfrac{3+a}{2+a}$}
	{\True $\dfrac{2+3a}{2+2a}$}
	\loigiai{
		Ta có $\log_{72}108=\dfrac{\log_2108}{\log_272}=\dfrac{\log_2\left(2^2\cdot3^3\right)}{\log_2\left(2^3\cdot3^2\right)}=\dfrac{2+3\log_23}{3+2\log_23}=\dfrac{2+3a}{3+2a}$.
	}
\end{ex}
\begin{ex}%[2D2K3-2]
	[Chuyên Trần Phú Hải Phòng 2019]
	Cho $\log_{30}3=a$; $\log_{30}5=b$. Tính $\log_{30}1350$ theo $a,b$; $\log_{30}1350$ bằng
	\choice
	{$2a+b$}
	{\True $2a+b+1$}
	{$2a+b-1$}
	{$2a+b-2$}
	\loigiai{
		Ta có $1350=30\cdot45=30\cdot9\cdot5=30\cdot3^2\cdot5$.\\
		Nên $\log_{30}1350=\log_{30}\left(30\cdot3^2\cdot5\right)=\log_{30}30+\log_{30}{3^2}+\log_{30}5=1+2\log_{30}3+\log_{30}5=1+2a+b$.
	}
\end{ex}
\begin{ex}%[2D2K3-2]
	[THPT Quang Trung Đống Đa Hà Nội 2019]
	Đặt $m=\log2$ và $n=\log7$. Hãy biểu diễn $\log6125\sqrt{7}$ theo $m$ và $n$.
	\choice
	{$\dfrac{6+6m+5n}{2}$}
	{$\dfrac{1}{2}(6-6n+5m)$}
	{$5m+6n-6$}
	{\True $\dfrac{6+5n-6m}{2}$}
	\loigiai{
		Ta có
		\begin{align*}
			\log 6125\sqrt{7}&=\log\left({5^3}\cdot{7^{\frac{5}{2}}}\right)=3\log 5+\dfrac{5}{2}\log 7=3\log\dfrac{10}{2}+\dfrac{5}{2}\log 7\cr
			&=3(\operatorname{l}-\log 2)+\dfrac{5}{2}\log 7=3\left(1-m\right)+\dfrac{5}{2}n=\dfrac{6+5n-6m}{2}.
		\end{align*}
		Vậy $\log 6125\sqrt{7}=\dfrac{6+5n-6m}{2}$.
	}
\end{ex}
\begin{ex}%[2D2K3-2]
	[Lương Thế Vinh Hà Nội 2019]
	Cho $\log_{27}5=a$, $\log_37=b$, $\log_23=c$. Tính $\log_635$ theo $a$, $b$ và $c$.
	\choice
	{$\dfrac{\left(3a+b\right)c}{1+c}$}
	{$\dfrac{\left(3a+b\right)c}{1+b}$}
	{$\dfrac{\left(3a+b\right)c}{1+a}$}
	{\True $\dfrac{\left(3b+a\right)c}{1+c}$}
	\loigiai{
		Theo giả thiết, ta có $\log_{27}5=a\Leftrightarrow\dfrac{1}{3}{\log_3}5=a\Leftrightarrow{\log_3}5=3a$.\\
		Ta có $\log_25=\log_23\cdot{\log_3}5=3ac$ và $\log_27=\log_23\cdot{\log_3}7=bc$.\\
		Vậy $\log_635=\dfrac{\log_235}{\log_26}=\dfrac{\log_25+\log_27}{\log_22+\log_23}=\dfrac{3ac+bc}{1+c}=\dfrac{\left(3a+b\right)c}{1+c}$.
	}
\end{ex}
\begin{ex}%[2D2K3-2]
	[Sở Thanh Hóa 2019]
	Cho $a=\log_2m$ và $A=\log_m16m$, với $0<m\ne 1$. Mệnh đề nào sau đây đúng?
	\choice
	{$A=\dfrac{4-a}{a}$}
	{\True $A=\dfrac{4+a}{a}$}
	{$A=(4+a)a$}
	{$A=(4-a)a$}
	\loigiai{
		Ta có $A=\log_m16m=\dfrac{\log_216m}{\log_2m}=\dfrac{\log_216+\log_2m}{\log_2m}=\dfrac{4+a}{a}$.
	}
\end{ex}
\begin{ex}%[2D2K3-2]      
	[THPT Ngô Sĩ Liên Bắc Giang 2019]
	Biết $\log_315=a$, tính $P=\log_{25}81$ theo $a$ ta được
	\choice
	{$P=2\left(a+1\right)$}
	{$P=2(a-1)$}
	{$P=\dfrac{2}{a+1}$}
	{\True $\dfrac{2}{a-1}$}
	\loigiai{
		Ta có $\log_315=a\Rightarrow 1+\log_35=a\Rightarrow{\log_3}5=a-1$.\\
		$P=\log_{25}81=\dfrac{\log_381}{\log_325}=\dfrac{4}{2\log_35}=\dfrac{4}{2\left(a-1\right)}=\dfrac{2}{a-1}$.
	}
\end{ex}
\begin{ex}%[2D2K3-2]
	[Chuyên Phan Bội Châu 2019]
	Cho $\log_35=a$, $\log_36=b$, $\log_322=c$. Tính $P=\log_3\dfrac{90}{11}$ theo $a,b,c$.
	\choice
	{$P=2a+b-c$}
	{\True $P=a+2b-c$}
	{$P=2a+b+c$}
	{$P=2a-b+c$}
	\loigiai{
		Ta có
		\begin{align*}
			P&=\log_390-\log_311=\log_390+\log_32-\log_311-\log_32\cr
			&=\log_3180-\log_322=\log_3\left(5\cdot36\right)-\log_322=\log_35+2\log_36-\log_322=a+2b-c.
		\end{align*}
	}
\end{ex}
\begin{ex}%[2D2K3-2]
	[Chuyên ĐHSP Hà Nội 2019]
	Nếu $\log_35=a$ thì $\log_{45}75$ bằng
	\choice
	{$\dfrac{2+a}{1+2a}$}
	{$\dfrac{1+a}{2+a}$}
	{\True $\dfrac{1+2a}{2+a}$}
	{$\dfrac{1+2a}{1+a}$}
	\loigiai{
		Ta có $\log_{45}75=2\log_{45}5+\log_{45}3$ và
		\begin{align*}
			\log_{45}5&=\dfrac{1}{\log_545}=\dfrac{1}{2\log_53+1}=\dfrac{1}{\dfrac{2}{a}+1}=\dfrac{a}{a+2},\cr
			{\log_{45}}3&=\dfrac{1}{\log_345}=\dfrac{1}{2+\log_35}=\dfrac{1}{a+2}.
		\end{align*}
		
		Do đó $\log_{45}75=\dfrac{2a}{a+2}+\dfrac{1}{a+2}=\dfrac{1+2a}{2+a}$.
	}
\end{ex}
\begin{ex}%[2D2K3-2]
	[Chuyên Phan Bội Châu Nghệ An 2019]
	Cho $\log_35=a$, $\log_36=b$, $\log_322=c$. Tính $P=\log_3\left(\dfrac{90}{11}\right)$ theo $a,b,c$. 
	\choice
	{$P=2a+b-c$}
	{\True $P=a+2b-c$}
	{$P=2a+b+c$}
	{$P=2a-b+c$}
	\loigiai{
		Ta có $P=\log_3\left(\dfrac{90}{11}\right)=\log_3\left(\dfrac{180}{22}\right)=\log_3\left(\dfrac{5\cdot6^2}{22}\right)=\log_35+2\log_36-\log_322=a+2b-c$.
	}
\end{ex}
\begin{ex}%[2D2K3-2]
	[Chuyên Nguyễn Tất Thành Yên Bái 2019]
	Cho $\log_{12}3=a$. Tính $\log_{24}18$ theo $a$.
	\choice
	{\True $\dfrac{3a+1}{3-a}$}
	{$\dfrac{3a+1}{3+a}$}
	{$\dfrac{3a-1}{3+a}$}
	{$\dfrac{3a-1}{3-a}$}
	\loigiai{
		Ta có $a=\log_{12}3=\dfrac{1}{\log_312}=\dfrac{1}{1+2\log_32}\Leftrightarrow{\log_2}3=\dfrac{2a}{1-a}$.\\
		Khi đó $\log_{24}18=\dfrac{\log_2\left(3^2\cdot2\right)}{\log_2\left(2^3\cdot3\right)}=\dfrac{1+2\log_23}{3+\log_23}=\dfrac{1+2\cdot\dfrac{2a}{1-a}}{3+\dfrac{2a}{1-a}}=\dfrac{1+3a}{3-a}$.
	}
\end{ex}
\begin{ex}%[2D2K3-2]
	[THPT Nghĩa Hưng Nđ-2019]
	Đặt $\log_ab=m$, $\log_bc=n$. Khi đó $\log_a\left(a{b^2}{c^3}\right)$ bằng
	\choice
	{$1+6mn$}
	{$1+2m+3n$}
	{$6mn$}
	{\True $1+2m+3mn$}
	\loigiai{
		\begin{align*}
			\log_a\left(a{b^2}{c^3}\right)&=\log_aa+2\log_ab+3\log_ac\cr
			&=1+2m+3\dfrac{\log_bc}{\log_ba}=1+2m+3\log_ab\cdot\log_bc=1+2m+3mn.
		\end{align*}
	}
\end{ex}
\begin{ex}%[2D2K3-2]
	[Cụm Liên Trường Hải Phòng 2019]
	Đặt $a=\log_23$ và $b=\log_53$. Hãy biểu diễn $\log_645$ theo $a$ và $b$.
	\choice
	{\True $\log_645=\dfrac{a+2{a}b}{ab+b}$}
	{$\log_645=\dfrac{a+2{a}b}{ab}$}
	{$\log_645=\dfrac{2a^2-2{a}b}{ab}$}
	{$\log_645=\dfrac{2a^2-2{a}b}{ab+b}$}
	\loigiai{
		$\log_645=\dfrac{\log_2\left(3^2\cdot5\right)}{\log_2\left(2\cdot3\right)}=\dfrac{2\log_23+\log_23\cdot\log_35}{1+\log_23}=\dfrac{2{a}+\dfrac{a}{b}}{1+a}=\dfrac{2{a}b+a}{ab+b}$.
	}
\end{ex}
\begin{ex}%[2D2K3-2]
	[THPT Thiệu Hóa – Thanh Hóa 2019]
	Cho $\log_95=a;\log_47=b;\log_23=c$. Biết $\log_{24}175=\dfrac{mb+nac}{pc+q}$. Tính $A=m+2n+3p+4q$.
	\choice
	{$27$}
	{\True $25$}
	{$23$}
	{$29$}
	\loigiai{
		Ta có
		\begin{align*}
			\log_{24}175&=\log_{24}{7\cdot5^2}=\log_{24}7+2\log_{24}{5^2}=\dfrac{1}{\log_724}+\dfrac{2}{\log_524}\cr
			&=\dfrac{1}{\log_73+\log_72^3}+\dfrac{2}{\log_53+\log_52^3}=\dfrac{1}{\dfrac{1}{\log_37}+\dfrac{3}{\log_27}}+\dfrac{2}{\dfrac{1}{\log_35}+\dfrac{3}{\log_25}}\cr
			&=\dfrac{1}{\dfrac{1}{\log_27\cdot\log_32}+\dfrac{3}{\log_27}}+\dfrac{2}{\dfrac{1}{\log_35}+\dfrac{3}{\log_23\cdot\log_35}}=\dfrac{1}{\dfrac{1}{2b\cdot\dfrac{1}{c}}+\dfrac{3}{2b}}+\dfrac{2}{\dfrac{1}{2{a}}+\dfrac{3}{c\cdot2{a}}}\cr
			&=\dfrac{1}{\dfrac{c}{2b}+\dfrac{3}{2b}}+\dfrac{2}{\dfrac{c}{2{ac}}+\dfrac{3}{2{ac}}}=\dfrac{2b}{c+3}+\dfrac{4{a}c}{c+3}=\dfrac{2b+4{a}c}{c+3}.
		\end{align*}
		Từ đó $A=m+2n+3p+4q=2+8+3+12=25$.
	}
\end{ex}
\begin{ex}%[2D2K3-2]
	[Chuyên KHTN 2019]
	Với các số $a,b>0$ thỏa mãn $a^2+b^2=6ab$, biểu thức $\log_2\left(a+b\right)$ bằng
	\choice
	{\True $\dfrac{1}{2}\left(3+\log_2a+\log_2b\right)$}
	{$\dfrac{1}{2}\left(1+\log_2a+\log_2b\right)$}
	{$1+\dfrac{1}{2}\left(\log_2a+\log_2b\right)$}
	{$2+\dfrac{1}{2}\left(\log_2a+\log_2b\right)$}
	\loigiai{
		Ta có $a^2+b^2=6ab\Leftrightarrow{a^2}+b^2+2ab=6ab+2ab\Leftrightarrow{\left(a+b\right)^2}=8ab$.\hfill(*)\\
		Do $a,b>0$ nên $ab>0$ và $a+b>0$. Lấy logarit cơ số $2$ hai vế của $(*)$ ta được
		\begin{align*}
			&\log_2\left(a+b\right)^2=\log_2\left(8ab\right)\Leftrightarrow 2\log_2\left(a+b\right)=3+\log_2a+\log_2b\cr
			\Leftrightarrow\ &{\log_2}\left(a+b\right)=\dfrac{1}{2}\left(3+\log_2a+\log_2b\right).
		\end{align*}
	}
\end{ex}
\begin{ex}%[2D2K3-2]
	[Chuyên Hạ Long - Quảng Ninh - 2021]     
	Biết $\log_712=a$; $\log_{12}24=b$. Giá trị của $\log_{54}168$ được tính theo $a$ và $b$ là
	\choice
	{\True $\dfrac{ab+1}{a\left(8-5b\right)}$}
	{$\dfrac{ab-1}{a\left(8+5b\right)}$}
	{$\dfrac{2ab+1}{8a-5b}$}
	{$\dfrac{2ab+1}{8a+5b}$}
	\loigiai{
		Do $\log_712=a$; $\log_{12}24=b\Rightarrow a;b>0$.\\
		$\log_712=a\Leftrightarrow{\log_7}\left(2^2\cdot3\right)=a\Leftrightarrow 2\log_72+\log_73=a$.\hfill(1)\\
		$\log_{12}24=b\Leftrightarrow\dfrac{\log_724}{\log_712}=b\Leftrightarrow\dfrac{3\log_72+\log_73}{a}=b\Leftrightarrow 3\log_72+\log_73=ab$.\hfill(2)\\
		Từ $(1)$ và $(2)$ ta có hệ phương trình: $\left\{\begin{aligned}
			&2\log_72+\log_73=a\\             
			&3\log_72+\log_73=ab\\ 
		\end{aligned}\right.\Leftrightarrow\left\{\begin{aligned}
			&{\log_7}2=ab-a\\ 
			&{\log_7}3=3a-2ab.
		\end{aligned}\right.$\\
		Mặt khác, $\log_{54}168=\dfrac{\log_7168}{\log_754}=\dfrac{\log_7\left(2^3.3.7\right)}{\log_7\left(2.3^3\right)}=\dfrac{3\log_72+\log_73+1}{\log_72+3\log_73}$. Do đó
		$${\log_{54}}168=\dfrac{3\left(ab-a\right)+3a-2ab+1}{ab-a+3\left(3a-2ab\right)}=\dfrac{3ab-3a+3a-2ab+1}{ab-a+9a-6ab}=\dfrac{ab+1}{8a-5ab}=\dfrac{ab+1}{a\left(8-5b\right)}.$$
		Vậy $\log_{54}168=\dfrac{ab+1}{a\left(8-5b\right)}$.
	}
\end{ex}
\Closesolutionfile{ans}
\indapan{10}{ans/CD17/Muc_7_8}