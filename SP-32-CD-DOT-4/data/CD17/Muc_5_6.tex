\Opensolutionfile{ans}[ans/CD17/Muc_5_6]
\setcounter{ex}{0}
\setcounter{dang}{0}
\section{Mức độ 5,6 điểm}            
\begin{dang} 
	{Câu hỏi lý thuyết}
\end{dang}          
\begin{ex}%[2D2B3-3]
	[Đề Minh Họa 2017]
	Cho hai số thực $a$ và $b$, với $1<a<b$. Khẳng định nào dưới đây là khẳng định đúng?
	\choice
	{\True $\log_ba<1<\log_ab$}
	{$ 1<\log_ab<\log_ba$}
	{$\log_ba<\log_ab<1$}
	{$\log_ab<1<\log_ba$}
	\loigiai{
		Vì $ b>a>1\Rightarrow\left\{\begin{aligned}
			&{\log_a}b>\log_aa\\ 
			&{\log_b}b>\log_ba\\ 
		\end{aligned}\right.\Leftrightarrow\left\{\begin{aligned}
			&{\log_a}b>1\\ 
			& 1>\log_ba\\ 
		\end{aligned}\right.\Rightarrow{\log_b}a<1<\log_ab.$
	}
\end{ex}
\begin{ex}%[2D2B3-2]
	[Mã 110 2017]
	Cho $a$ là số thực dương khác $1$. Mệnh đề nào dưới đây đúng với mọi số dương $x$, $y$?
	\choice
	{\True $\log_a\dfrac{x}{y}=\log_ax-\log_ay$}
	{$\log_a\dfrac{x}{y}=\log_a\left(x-y\right)$}
	{$\log_a\dfrac{x}{y}=\log_ax+\log_ay$}
	{$\log_a\dfrac{x}{y}=\dfrac{\log_ax}{\log_ay}$}
	\loigiai{
		Theo tính chất của logarit.
	}
\end{ex}
\begin{ex}%[2D2B3-2]
	[THPT Minh Khai Hà Tĩnh 2019]
	Với mọi số thực dương $a$, $b$, $x$, $y$ và $a,b\ne 1$, mệnh đề nào sau đây sai?
	\choice
	{\True $\log_a\dfrac{1}{x}=\dfrac{1}{\log_ax}$}
	{$\log_a\left(xy\right)=\log_ax+\log_ay$}
	{$\log_ba.\log_ax=\log_bx$}
	{$\log_a\dfrac{x}{y}=\log_ax-\log_ay$}
	\loigiai{
		Với mọi số thực dương $ a,b,x,y$ và $ a,b\ne 1$. 
		Ta có $\log_a\dfrac{1}{x}=\log_a{x^{-1}}\ne\dfrac{1}{\log_ax}$.
	}
\end{ex}
\begin{ex}%[2D2B3-2]
	[Chuyên Hạ Long 2019]
	Trong các mệnh đề sau, mệnh đề nào đúng?
	\choice
	{\True $\log_a{b^{\alpha}}=\alpha{\log_a}b$ với mọi số $a,b$ dương và $a\ne 1$}
	{$\log_ab=\dfrac{1}{\log_ba}$ với mọi số $a,b$ dương và $a\ne 1$}
	{$\log_ab+\log_ac=\log_abc$ với mọi số $a,b$ dương và $a\ne 1$}
	{$\log_ab=\dfrac{\log_ca}{\log_cb}$ với mọi số $a,b,c$ dương và $a\ne 1$}
	\loigiai{}
\end{ex}
\begin{ex}%[2D2B3-2]
	[THPT - Thăng - Long - Hà - Nội - 2019]
	Cho $a$, $b$ là hai số thực dương tùy ý và $ b\ne 1$.Tìm kết luận đúng.
	\choice
	{$\ln a+\ln b=\ln \left(a+b\right)$}
	{$\ln \left(a+b\right)=\ln a.\ln b$}
	{$\ln a-\ln b=\ln \left(a-b\right)$}
	{\True $\log_ba=\dfrac{\ln a}{\ln b}$}
	\loigiai{
	}
\end{ex}
\begin{ex}%[2D2B3-2]
	[THPT Yên Phong Số 1 Bắc Ninh 2019]
	Cho hai số dương $a,b\left(a\ne 1\right)$. Mệnh đề nào dưới đây SAI?
	\choice
	{\True $\log_aa=2a$}
	{$\log_a{a^{\alpha}}=\alpha $}
	{$\log_a1=0$}
	{$a^{\log_ab}=b$}
	\loigiai{}
\end{ex}
\begin{ex}%[2D2B3-2]
	[Sở Thanh Hóa 2019]
	Với các số thực dương $a$, $b$ bất kì. Mệnh đề nào dưới đây đúng?
	\choice
	{$\log\left(ab\right)=\log a.\log b$}
	{$\log\dfrac{a}{b}=\dfrac{\log a}{\log b}$}
	{\True $\log\left(ab\right)=\log a+\log b$}
	{$\log\dfrac{a}{b}=\operatorname{logb}-\operatorname{loga}$}
	\loigiai{
		Ta có $\log\left(ab\right)=\log a+\log b$.
	}
\end{ex}
\begin{ex}%[2D2B3-2]
	[VTED 03 2019]
	Với các số thực dương $a$, $b$ bất kì. Mệnh đề nào dưới đây đúng?
	\choice
	{\True $\ln \left(ab\right)=\ln a+\ln b$}
	{$\ln \left(\dfrac{a}{b}\right)=\dfrac{\ln a}{\ln b}$}
	{$\ln \left(ab\right)=\ln a.\ln b$}
	{$\ln \left(\dfrac{a}{b}\right)=\ln b-\ln a$}
	\loigiai{}
\end{ex}
\begin{ex}%[2D2B3-2]
	[Chuyên Lê Hồng Phong Nam Định 2019]
	Với các số thực dương $a$, $b$ bất kì. Mệnh đề nào sau đây đúng?
	\choice
	{$\log\left(ab\right)=\log a.logb$}
	{$\log\dfrac{a}{b}=\log b-\log a$}
	{$\log\dfrac{a}{b}=\dfrac{\log a}{\log b}$}
	{\True $\log\left(ab\right)=\log a+\log b$}
	\loigiai{
		Theo tính chất logarit
	}
\end{ex}
\begin{ex}%[2D2B3-2]
	Cho $a$, $b$, $c>0$, $a\ne 1$ và số $\alpha\in\mathbb{R}$, mệnh đề nào dưới đây sai?
	\choice
	{$\log_a{a^c}=c$}
	{$\log_aa=1$}
	{$\log_a{b^{\alpha}}=\alpha{\log_a}b$}
	{\True $\log_a\left| b-c\right|=\log_ab-\log_ac$}
	\loigiai{
		Theo tính chất của logarit, mệnh đề sai là $\log_a\left|b-c\right|=\log_ab-\log_ac$.
	}
\end{ex}
\begin{ex}%[2D2B3-2]
	[THPT An Lão Hải Phòng 2019] 
	Cho $a$, $b$, $c$ là các số dương $\left(a,b\ne1\right)$. Trong các mệnh đề sau, mệnh đề nào là mệnh đề đúng?
	\choice
	{$\log_a\left(\dfrac{b}{a^3}\right)=\dfrac{1}{3}{\log_a}b$}
	{$a^{\log_ba}=b$}
	{$\log_{a^{\alpha}}b=\alpha{\log_a}b\left(\alpha\ne0\right)$}
	{\True $\log_ac=\log_bc.\log_ab$}
	\loigiai{}
\end{ex}
\begin{dang}
	{Tính, rút gọn biểu thức chứa logarit}
\end{dang}
\begin{ex}%[2D2B3-2]
	[Đề minh họa 2022]
	Với mọi số thực $a$ dương, $\log_2\dfrac{a}{2}$ bằng 
	\choice
	{$\dfrac{1}{2}{\log_2}a$}
	{$\log_2a+1$}
	{\True $\log_2a-1$}
	{$\log_2a-2$}
	\loigiai{
		Ta có $\log_2\dfrac{a}{2}=\log_2a-\log_22=\log_2a-1$.
	}
\end{ex}
\begin{ex}%[2D2B3-2]
	[Đề minh họa 2022]
	Với mọi $a$, $b$ thỏa mãn $\log_2a-3\log_2b=2$, khẳng định nào dưới đây đúng?
	\choice
	{\True $a=4b^3$}
	{$a=3b+4$}
	{$a=3b+2$}
	{$a=\dfrac{4}{b^3}$}
	\loigiai{
		Điều kiện: $a,b>0$. Ta có
		$$\begin{aligned}
			{\log_2}a-3\log_2b=2&\Leftrightarrow{\log_2}a-\log_2b^3=2\Leftrightarrow{\log_2}\dfrac{a}{b^3}=2\\ 
			&\Leftrightarrow\dfrac{a}{b^3}=4\Leftrightarrow a=4b^3\\ 
		\end{aligned}$$
	}
\end{ex}
\begin{ex}%[2D2B3-2]
	[Mã 101 - 2022]
	Với $a$ là số thực dương tùy ý, $4\log\sqrt{a}$ bằng
	\choice
	{$-2\log a$}
	{\True $2\log a$}
	{$-4\log a$}
	{$8\log a$}
	\loigiai{
		Với $a>0$, ta có $4\log\sqrt{a}=4\log\left(a^{\frac{1}{2}}\right)=4.\dfrac{1}{2}\log a=2\log a$.
	}
\end{ex}
\begin{ex}%[2D2B3-2]
	[Mã 102 - 2022]
	Với $a$ là số thực dương tùy ý, $4\log\sqrt{a}$ bằng
	\choice
	{$-4\log a$}
	{$8\log a$}
	{\True $2\log a$}
	{$-2\log a$}
	\loigiai{
		Ta có $4\log\sqrt{a}=4\log{a^{\frac{1}{2}}}=2\log a$.
	}
\end{ex}
\begin{ex}%[2D2B3-2]
	[Mã 103 - 2022]
	Với $a$ là số thực dương tùy ý, $\log\left(100a\right)$ bằng 
	\choice
	{$1-\log a$}
	{\True $2+\log a$}
	{$2-\log a$}
	{$1+\log a$}
	\loigiai{
		$\log\left(100a\right)=\log\left(100\right)+\log a=2+\log a$.
	}
\end{ex}
\begin{ex}%[2D2B3-2]
	[Mã 103 - 2022]
	Với $a$, $b$ là các số thực dương tùy ý và $a\ne 1$, $\log_{\frac{1}{a}}\dfrac{1}{b^3}$ bằng
	\choice
	{\True $3\log_ab$}
	{$\log_ab$}
	{$-3\log_ab$}
	{$\dfrac{1}{3}{\log_a}b$}
	\loigiai{
		$\log_{\frac{1}{a}}\dfrac{1}{b^3}=-\log_a{b^{-3}}=3\log_ab$.
	}
\end{ex}
\begin{ex}%[2D2B3-2]
	[Mã 104 - 2022]
	Với $a$ là số thực dương tuỳ ý, $\log\left(100a\right)$ bằng
	\choice
	{$2-\log a$}
	{\True $2+\log a$}
	{$1-\log a$}
	{$1+\log a$}
	\loigiai{
		Với $ a>0$, ta có $\log\left(100a\right)=\log 100+\log a=\log{10^2}+\log a=2+\log a$.
	}
\end{ex}
\begin{ex}%[2D2B3-2]
	[Mã 104 - 2022]
	Với $a$, $b$ là các số thực dương tùy ý và $a\ne 1$, $\log_{\frac{1}{a}}\dfrac{1}{b^3}$ bằng
	\choice
	{$\log_ab$}
	{$-3\log_ab$}
	{$\dfrac{1}{3}{\log_a}b$}
	{\True $3\log_ab$}
	\loigiai{
		Ta có $\log_{\frac{1}{a}}\dfrac{1}{b^3}=\log_{a^{-1}}{b^{-3}}=3\log_ab$.
	}
\end{ex}
\begin{ex}%[2D2B3-2]
	[Mã 101 - 2021 Lần 1]
	Cho $a>0$ và $a\ne 1$, khi đó $\log_a\sqrt[4]
	{a}$ bằng 
	\choice
	{$4$}
	{\True $\dfrac{1}{4}$}
	{$-\dfrac{1}{4}$}
	{$-4$}
	\loigiai{
		Ta có $\log_a\sqrt[4]{a}=\log_a{a^{\frac{1}{4}}}=\dfrac{1}{4}$.
	}
\end{ex}
\begin{ex}%[2D2B3-2]
	[Mã 102 - 2021 Lần 1]\
	Cho $a>0$ và $a\ne 1$ khi đó $\log_a\sqrt[3]{a}$ bằng
	\choice
	{$-3$}
	{\True $\dfrac{1}{3}$}
	{$-\dfrac{1}{3}$}
	{$3$}
	\loigiai{
		$\log_a\sqrt[3]{a}=\dfrac{1}{3}{\log_a}a=\dfrac{1}{3}$.
	}
\end{ex}
\begin{ex}%[2D2B3-2]
	[Mã 104 - 2021 Lần 1]
	Cho $a>0$và $a\ne 1$, khi đó $\log_a\sqrt[5]{a}$ bằng
	\choice
	{\True $\dfrac{1}{5}$}
	{$-\dfrac{1}{5}$}
	{$5$}
	{$-5$}
	\loigiai{
		Ta có $\log_a\sqrt[5]{a}=\log_a{a^{\frac{1}{5}}}=\dfrac{1}{5}{\log_a}a=\dfrac{1}{5}$.
	}
\end{ex}
\begin{ex}%[2D2B3-2]
	[Mã 103 - 2021-Lần 1]
	Cho $a>0$ và $a\ne 1$, khi đó $\log_a\sqrt{a}$ bằng
	\choice
	{$2$}
	{$-2$}
	{$-\dfrac{1}{2}$}
	{\True $\dfrac{1}{2}$}
	\loigiai{
		Với $a>0$ và $a\ne 1$,ta có $\log_a\sqrt{a}=\log_a{a^{\frac{1}{2}}}=\dfrac{1}{2}{\log_a}a=\dfrac{1}{2}$.
	}
\end{ex}
\begin{ex}%[2D2B3-2]
	[Đề Minh Họa 2021]
	Với $ a$ là số thực dương tùy ý, $\log_3\left(9a\right)$ bằng
	\choice
	{$\dfrac{1}{2}+\log_3a$}
	{$2\log_3a$}
	{$\left(\log_3a\right)^2$}
	{\True $2+\log_3a$}
	\loigiai{
		Ta có $\log_3\left(9a\right)=\log_39+\log_3a=\log_33^2+\log_3a=2+\log_3a$.
	}
\end{ex}
\begin{ex}%[2D2B3-2]
	[Mã 101 - 2020 Lần 1]
	Với $a$, $b$ là các số thực dương tùy ý và $a\ne 1$, $\log_{a^5}b$ bằng
	\choice
	{$5\log_ab$}
	{$\dfrac{1}{5}+\log_ab$}
	{$5+\log_ab$}
	{\True $\dfrac{1}{5}{\log_a}b$}
	\loigiai{}
\end{ex}
\begin{ex}%[2D2B3-2]
	[Mã 102 - 2020 Lần 1]
	Với $a, b$ là các số thực dương tùy ý và $a\neq 1, \log _{a^2}b$ bằng
	\choice
	{$\dfrac{1}{2}+\log_a b$}
	{\True $\dfrac{1}{2}\log_a b$}
	{$2+\log_a b$}
	{$2\log_a b$}
	\loigiai{
		Ta có $\log _{a^2}b=\dfrac{1}{2}\log_a b$.
	}
\end{ex}
\begin{ex}%[2D2B3-2]
	[Mã 103 - 2020 Lần 1]
	Với $a,b$ là các số thực dương tùy ý và $a\ne 1$, $\log_{a^3}b$ bằng
	\choice
	{$3+\log_ab$}
	{$3\log_ab$}
	{$\dfrac{1}{3}+\log_ab$}
	{\True $\dfrac{1}{3}{\log_a}b$}
	\loigiai{
		Ta có $\log_{a^3}b=\dfrac{1}{3}{\log_a}b$.
	}
\end{ex}
\begin{ex}%[2D2B3-2]
	[Mã 102 - 2020 Lần 2]
	Với $ a$ là số thực dương tùy ý, $\log_5\left(5a\right)$ bằng
	\choice
	{$ 5+\log_5a$}
	{$ 5-\log_5a$}
	{\True $ 1+\log_5a$}
	{$ 1-\log_5a$}
	\loigiai{
		Ta có $\log_5\left(5a\right)$$=\log_55+\log_5a$$=1+\log_5a$.
	}
\end{ex}
\begin{ex}%[2D2B3-2]
	[Mã 103 - 2020 Lần 2]
	Với $a$ là số thực dương tùy ý, $\log_22a$ bằng
	\choice
	{\True $1+\log_2a$}
	{$1-\log_2a$}
	{$2-\log_2a$}
	{$2+\log_2a$}
	\loigiai{
		$\log_22a=\log_22+\log_2a=1+\log_2a$.
	}
\end{ex}
\begin{ex}%[2D2B3-2]
	[Đề Minh Họa 2020 Lần 1]
	Với $a$ là số thực dương tùy ý, $\log_2a^2$ bằng
	\choice
	{$2+\log_2a$}
	{$\dfrac{1}{2}+\log_2a$}
	{\True $2\log_2a$}
	{$\dfrac{1}{2}{\log_2}a$}
	\loigiai{
		Với $a>0;b>0;a\ne 1$. Với mọi $\alpha $. Ta có công thức $\log_a{b^{\alpha}}=\alpha{\log_a}b.$\\
		Vậy $\log_2a^2=2\log_2a$.
	}
\end{ex}
\begin{ex}%[2D2B3-2]
	[Đề Tham Khảo 2020 Lần 2]
	Với $a$ là hai số thực dương tùy ý, $\log_2\left(a^3\right)$ bằng
	\choice
	{$\dfrac{3}{2}{\log_2}a$}      
	{$\dfrac{1}{3}{\log_2}a$}
	{$3+\log_2a$}
	{\True $3\log_2a$}
	\loigiai{
		Ta có $\log_2\left(a^3\right)=3\log_2a$.
	}
\end{ex}
\begin{ex}%[2D2B3-2]
	[Mã 103 2019]
	Với $a$ là số thực dương tùy ý, $\log_2a^3$ bằng
	\choice
	{$3+\log_2a$}
	{\True $3\log_2a$}
	{$\dfrac{1}{3}{\log_2}a$}
	{$\dfrac{1}{3}+\log_2a$}
	\loigiai{
		Ta có $\log_2a^3=3\log_2a$.
	}
\end{ex}
\begin{ex}%[2D2B3-2]
	[Mã 102 2019]
	Với $ a$ là số thực dương tùy ý, $\log_5a^3$ bằng
	\choice
	{$\dfrac{1}{3}{\log_5}a$}
	{$\dfrac{1}{3}+\log_5a$}
	{$3+\log_5a$}
	{\True $ 3\log_5a$}
	\loigiai{
		$\log_5a^3=3\log_5a$.
	}
\end{ex}
\begin{ex}%[2D2B3-2]
	[Mã 104 2017]
	Cho $a$ là số thực dương tùy ý khác $1$. Mệnh đề nào dưới đây đúng?
	\choice
	{$\log_2a=\log_a2$}
	{$\log_2a=\dfrac{1}{\log_2a}$}
	{\True $\log_2a=\dfrac{1}{\log_a2}$}
	{$\log_2a=-\log_a2$}
	\loigiai{
		Áp dụng công thức đổi cơ số.
	}
\end{ex}
\begin{ex}%[2D2B3-2]
	[Mã 104 2019]
	Với $a$ là số thực dương tùy ý, $\log_2a^2$ bằng
	\choice
	{$\dfrac{1}{2}{\log_2}a$}
	{$2+\log_2a$}
	{\True $2\log_2a$}
	{$\dfrac{1}{2}+\log_2a$}
	\loigiai{
		Vì $a$ là số thực dương tùy ý nên $\log_2a^2=2\log_2a$.
	}
\end{ex}
\begin{ex}%[2D2B3-2]
	[Đề Tham Khảo 2019]
	Với $a$, $b$ là hai số dương tùy ý, $\log\left(a{b^2}\right)$ bằng
	\choice
	{$2\left(\log a+\log b\right)$}
	{$\log a+\dfrac{1}{2}\log b$}
	{$2\log a+\log b$}
	{\True $\log a+2\log b$}
	\loigiai{
		Ta có $\log\left(a{b^2}\right)=\log a+\log{b^2}=\log a+2\log b$.
	}
\end{ex}
\begin{ex}%[2D2B3-2]
	[Đề Tham Khảo 2017]
	Cho $a$ là số thực dương $a\ne 1$ và $\log_{\sqrt[3]{a}}{a^3}$. Mệnh đề nào sau đây đúng?
	\choice
	{$P=\dfrac{1}{3}$}
	{$P=3$}
	{$P=1$}
	{\True $P=9$}
	\loigiai{
		$\log_{\sqrt[3]{a}}{a^3}=\log_{a^{\frac{1}{3}}}{a^3}=9$.
	}
\end{ex}
\begin{ex}%[2D2B3-2]
	[Mã 101 2019]
	Với $a$ là số thực dương tùy ý, bằng $\log_5a^2$. 
	\choice
	{$\dfrac{1}{2}{\log_5}a$}
	{$2+\log_5a$}
	{$\dfrac{1}{2}+\log_5a$}
	{\True $2\log_5a$}
	\loigiai{
		Vì $a$ là số thực dương nên ta có $\log_5a^2=2\log_5a$.
	}
\end{ex}
\begin{ex}%[2D2B3-2]
	[Mã 103 2018]
	Với $a$ là số thực dương tùy ý, $\ln \left(7a\right)-\ln \left(3a\right)$ bằng
	\choice
	{$\dfrac{\ln 7}{\ln 3}$}
	{\True $\ln \dfrac{7}{3}$}
	{$\ln \left(4a\right)$}
	{$\dfrac{\ln \left(7a\right)}{\ln \left(3a\right)}$}
	\loigiai{
		$\ln \left(7a\right)-\ln \left(3a\right)$$=\ln \left(\dfrac{7a}{3a}\right)$$=\ln \dfrac{7}{3}$.
	}
\end{ex}
\begin{ex}%[2D2B3-2]
	[Mã 101 2018]
	Với $a$ là số thực dương tùy ý, $\ln \left(5a\right)-\ln \left(3a\right)$ bằng
	\choice
	{\True $\ln \dfrac{5}{3}$}
	{$\dfrac{\ln 5}{\ln 3}$}
	{$\dfrac{\ln \left(5a\right)}{\ln \left(3a\right)}$}
	{$\ln \left(2a\right)$}
	\loigiai{
		$\ln \left(5a\right)-\ln \left(3a\right)$$=\ln \dfrac{5}{3}$.
	}
\end{ex}
\begin{ex}%[2D2B3-2]
	[Mã 102 2018]
	Với $a$ là số thực dương tùy ý, $\log_3\left(3a\right)$ bằng
	\choice
	{$1-\log_3a$}
	{$3\log_3a$}
	{$3+\log_3a$}
	{\True $1+\log_3a$}
	\loigiai{}
\end{ex}
\begin{ex}%[2D2B3-2]
	Với các số thực dương $a$, $b$ bất kì. Mệnh đề nào dưới đây đúng.
	\choice
	{\True $\ln \left(ab\right)=\ln a+\ln b$}
	{$\ln \left(ab\right)=\ln a.\ln b$}
	{$\ln \dfrac{a}{b}=\dfrac{\ln a}{\ln b}$}
	{$\ln \dfrac{a}{b}=\ln b-\ln a$}
	\loigiai{
		Theo tính chất của lôgarit $\forall a>0,b>0:\ln \left(ab\right)=\ln a+\ln b$.
	}
\end{ex}
\begin{ex}%[2D2B3-2]
	[Mã 123 2017]
	Cho $a$ là số thực dương khác $1$. Tính $I=\log_{\sqrt{a}}a.$ 
	\choice
	{$ I=-2$}
	{\True $ I=2$}
	{$ I=\dfrac{1}{2}$}
	{$ I=0$}
	\loigiai{
		Với $ a$ là số thực dương khác $ 1$ ta được: $I=\log_{\sqrt{a}}a=\log_{a^{\frac{1}{2}}}a=2\log_aa=2$.
	}
\end{ex}
\begin{ex}%[2D2B3-2]
	[Mã 104 2018]
	Với $a$ là số thực dương tùy ý, $\log_3\left(\dfrac{3}{a}\right)$ bằng
	\choice
	{\True $1-\log_3a$}
	{$3-\log_3a$}
	{$\dfrac{1}{\log_3a}$}
	{$1+\log_3a$}
	\loigiai{
		Ta có $\log_3\left(\dfrac{3}{a}\right)=\log_33-\log_3a$$=1-\log_3a$.
	}
\end{ex}
\begin{ex}%[2D2B3-2]
	Với các số thực dương $a$, $b$ bất kì. Mệnh đề nào dưới đây đúng?
	\choice
	{\True $\log_2\left(\dfrac{2a^3}{b}\right)=1+3\log_2a+\log_2b$}
	{$\log_2\left(\dfrac{2a^3}{b}\right)=1+\dfrac{1}{3}{\log_2}a+\log_2b$}
	{$\log_2\left(\dfrac{2a^3}{b}\right)=1+3\log_2a-\log_2b$}
	{$\log_2\left(\dfrac{2a^3}{b}\right)=1+\dfrac{1}{3}{\log_2}a-\log_2b$}
	\loigiai{
		Ta có $\log_2\left(\dfrac{2a^3}{b}\right)=\log_2\left(2a^3\right)-\log_2(b)=\log_22+\log_2a^3-\log_2b=1+3\log_2a-\log_b$.
	}
\end{ex}
\begin{ex}%[2D2B3-2]
	[Mã 110 2017]
	Cho $\log_ab=2$ và $\log_ac=3$. Tính $P=\log_a\left(b^2c^3\right)$.
	\choice
	{\True $P=13$}
	{$P=31$}
	{$P=30$}
	{$P=108$}
	\loigiai{
		Ta có $\log_a\left(b^2c^3\right)=2\log_ab+3\log_ac=2.2+3.3=13$.
	}
\end{ex}
\begin{ex}%[2D2B3-2]
	[Mã 102 2019]
	Cho $a$ và $b$ là hai số thực dương thỏa mãn $a^3b^2=32$. Giá trị của $3\log_2a+2\log_2b$ bằng
	\choice
	{$4$}
	{\True $5$}
	{$2$}
	{$32$}
	\loigiai{
		Ta có $\log_2a^3b^2=\log_232\Leftrightarrow 3\log_2a+2\log_2b=5$.
	}
\end{ex}
\begin{ex}%[2D2B3-2]
	[Đề Tham Khảo 2017]
	Cho $a$, $b$ là các số thực dương thỏa mãn $a\ne 1$, $a\ne\sqrt{b}$ và $\log_ab=\sqrt{3}$. Tính $P=\log_{\frac{\sqrt{b}}{a}}\sqrt{\dfrac{b}{a}}$.
	\choice
	{$P=-5+3\sqrt{3}$}
	{$P=-1+\sqrt{3}$}
	{\True $P=-1-\sqrt{3}$}
	{$P=-5-3\sqrt{3}$}
	\loigiai{
		$P=\dfrac{\log_a\sqrt{\frac{b}{a}}}{\log_a\frac{\sqrt{b}}{a}}=\dfrac{\frac{1}{2}\left(\log_ab-1\right)}{\log_a\sqrt{b}-1}=\dfrac{\frac{1}{2}\left(\sqrt{3}-1\right)}{\frac{1}{2}{\log_a}b-1}=\dfrac{\sqrt{3}-1}{\sqrt{3}-2}=-1-\sqrt{3}$.
	}
\end{ex}
\begin{ex}%[2D2B3-2]
	[Mã 103 2019]
	Cho $a$ và $b$ là hai số thực dương thỏa mãn $a^2b^3=16$. Giá trị của $2\log_2a+3\log_2b$bằng
	\choice
	{$2$}
	{$8$}
	{$16$}
	{\True $4$}
	\loigiai{
		Ta có $2\log_2a+3\log_2b=\log_2\left(a^2b^3\right)=\log_216=4$.
	}
\end{ex}
\begin{ex}%[2D2B3-2]
	[Mã 104 2017]
	Với các số thực dương $x$, $y$ tùy ý, đặt $\log_3x=\alpha $, $\log_3y=\beta $. Mệnh đề nào dưới đây đúng?
	\choice
	{$\log_{27}{\left(\dfrac{\sqrt{x}}{y}\right)^3}=\dfrac{\alpha}{2}+\beta $}
	{$\log_{27}{\left(\dfrac{\sqrt{x}}{y}\right)^3}=9\left(\dfrac{\alpha}{2}+\beta\right)$}
	{$\log_{27}{\left(\dfrac{\sqrt{x}}{y}\right)^3}=\dfrac{\alpha}{2}-\beta $}
	{\True $\log_{27}{\left(\dfrac{\sqrt{x}}{y}\right)^3}=9\left(\dfrac{\alpha}{2}-\beta\right)$}
	\loigiai{
		$\log_{27}{\left(\dfrac{\sqrt{x}}{y}\right)^3}$ $=\dfrac{3}{2}{\log_{27}}x-3\log_{27}y$ $=\dfrac{1}{2}{\log_3}x-\log_3y=\dfrac{\alpha}{2}-\beta $.
	}
\end{ex}
\begin{ex}%[2D2B3-2]
	[Mã 101 2019]
	Cho $a$ và $b$ là hai số thực dương thỏa mãn $a^4b=16$. Giá trị của $4\log_2a+\log_2b$ bằng
	\choice
	{\True $4$}
	{$2$}
	{$16$}
	{$8$}
	\loigiai{
		$4\log_2a+\log_2b=\log_2a^4+\log_2b=\log_2\left(a^4b\right)=\log_216=\log_22^4=4$.
	}
\end{ex}
\begin{ex}%[2D2B3-2]
	[Đề Minh Họa 2017]
	Cho các số thực dương $a$, $b$ với $a\ne 1$. Khẳng định nào sau đây là khẳng định đúng?
	\choice
	{$\log_{a^2}\left(ab\right)=\dfrac{1}{4}{\log_a}b$}
	{\True $\log_{a^2}\left(ab\right)=\dfrac{1}{2}+\dfrac{1}{2}{\log_a}b$}
	{$\log_{a^2}\left(ab\right)=\dfrac{1}{2}{\log_a}b$}
	{$\log_{a^2}\left(ab\right)=2+2\log_ab$}
	\loigiai{
		Ta có $\log_{a^2}\left(ab\right)=\log_{a^2}a+\log_{a^2}b=\dfrac{1}{2}\cdot\log_aa+\dfrac{1}{2}\cdot\log_ab=\dfrac{1}{2}+\dfrac{1}{2}\cdot\log_ab$.
	}
\end{ex}
\begin{ex}%[2D2B3-2]
	[Mã 123 2017]
	Với $a$, $b$ là các số thực dương tùy ý và $a$ khác $1$, đặt $P=\log_a{b^3}+\log_{a^2}{b^6}$. Mệnh đề nào dưới đây đúng?
	\choice
	{\True $P=6\log_ab$}
	{$P=27\log_ab$}
	{$P=15\log_ab$}
	{$P=9\log_ab$}
	\loigiai{
		$P=\log_a{b^3}+\log_{a^2}{b^6}=3\log_ab+\dfrac{6}{2}{\log_a}b=6\log_ab$.
	}
\end{ex}
\begin{ex}%[2D2B3-2]
	[Đề Tham Khảo 2018]
	Với $a$ là số thực dương bất kì, mệnh đề nào dưới đây đúng?
	\choice
	{$\log\left(3a\right)=\dfrac{1}{3}\log a$}
	{$\log\left(3a\right)=3\log a$}
	{$\log{a^3}=\dfrac{1}{3}\log a$}
	{\True $\log{a^3}=3\log a$}
	\loigiai{}
\end{ex}
\begin{ex}%[2D2B3-2]
	[Mã 105 2017]
	Cho $\log_3a=2$ và $\log_2b=\dfrac{1}{2}$. Tính $I=2\log_3\left[\log_3\left(3a\right)\right]+\log_{\frac{1}{4}}{b^2}$.
	\choice
	{$I=\dfrac{5}{4}$}
	{$I=0$}
	{$I=4$}
	{\True $I=\dfrac{3}{2}$}
	\loigiai{
		$I=2\log_3\left[\log_3\left(3a\right)\right]+\log_{\frac{1}{4}}{b^2}=2\log_3\left(\log_33+\log_3a\right)+2\log_{2^{-2}}b=2-\dfrac{1}{2}=\dfrac{3}{2}$.
	}
\end{ex}
\begin{ex}%[2D2B3-2]
	[Mã 105 2017]
	Cho $a$ là số thực dương khác $2$. Tính $I=\log_{\frac{a}{2}}\left(\dfrac{a^2}{4}\right)$.
	\choice
	{\True $I=2$}
	{$I=-\dfrac{1}{2}$}
	{$I=-2$}
	{$I=\dfrac{1}{2}$}
	\loigiai{
		$I=\log_{\dfrac{a}{2}}\left(\dfrac{a^2}{4}\right)=\log_{\dfrac{a}{2}}{\left(\dfrac{a}{2}\right)^2}=2$.
	}
\end{ex}
\begin{ex}%[2D2B3-2]
	[Mã 104 2017]
	Với mọi $a$, $b$, $x$ là các số thực dương thoả mãn $\log_2x=5\log_2a+3\log_2b$. Mệnh đề nào dưới đây đúng?
	\choice
	{$x=5a+3b$}
	{$x=a^5+b^3$}
	{\True $x=a^5b^3$}
	{$x=3a+5b$}
	\loigiai{
		Có $\log_2x=5\log_2a+3\log_2b=\log_2a^5+\log_2b^3=\log_2a^5b^3\Leftrightarrow x=a^5b^3$.
	}
\end{ex}
\begin{ex}%[2D2B3-2]
	[Mã 104 2019]
	Cho $a$ và $b$ là hai số thực dương thỏa mãn $a{b^3}=8$. Giá trị của $\log_2a+3\log_2b$ bằng
	\choice
	{$6$}
	{$2$}
	{\True $3$}
	{$8$}
	\loigiai{
		Ta có $\log_2a+3\log_2b=\log_2a+\log_2b^3=\log_2\left(a{b^3}\right)=\log_28=3$.
	}
\end{ex}
\begin{ex}%[2D2B3-2]
	[Mã 105 2017]
	Với mọi số thực dương $a$ và $b$ thỏa mãn $a^2+b^2=8ab$, mệnh đề nào dưới đây đúng?
	\choice
	{$\log\left(a+b\right)=\dfrac{1}{2}\left(\log a+\log b\right)$}
	{$\log\left(a+b\right)=\dfrac{1}{2}+\log a+\log b$}
	{\True $\log\left(a+b\right)=\dfrac{1}{2}\left(1+\log a+\log b\right)$}
	{$\log\left(a+b\right)=1+\log a+\log b$}
	\loigiai{
		Ta có $a^2+b^2=8ab\Leftrightarrow{\left(a+b\right)^2}=10ab$.\\
		Lấy log cơ số $10$ hai vế ta được $$\log{\left(a+b\right)^2}=\log\left(10ab\right)\Leftrightarrow 2\log\left(a+b\right)=\log 10+\log a+\log b$$
		Suy ra $\log\left(a+b\right)=\dfrac{1}{2}\left(1+\log a+\log b\right)$.
	}
\end{ex}
\begin{ex}%[2D2B3-2]
	[Mã 123 2017]
	Cho $\log_ax=3,\log_bx=4$ với $a$, $b$ là các số thực lớn hơn $1$. Tính $ P=\log_{ab}x$.
	\choice
	{$P=12$}
	{\True $P=\dfrac{12}{7}$}
	{$P=\dfrac{7}{12}$}
	{$P=\dfrac{1}{12}$}
	\loigiai{
		$P=\log_{ab}x=\dfrac{1}{\log_xab}=\dfrac{1}{\log_xa+\log_xb}=\dfrac{1}{\frac{1}{3}+\frac{1}{4}}=\dfrac{12}{7}$.
	}
\end{ex}
\begin{ex}%[2D2B3-2]
	[Mã 110 2017] 
	Cho $x$, $y$ là các số thực lớn hơn $1$ thoả mãn $x^2+9y^2=6xy$. Tính $M=\dfrac{1+\log_{12}x+\log_{12}y}{2\log_{12}\left(x+3y\right)}$.
	\choice
	{$M=\dfrac{1}{2}$}
	{$M=\dfrac{1}{3}$}
	{$M=\dfrac{1}{4}$}
	{\True $M=1$}
	\loigiai{
		Ta có $x^2+9y^2=6xy\Leftrightarrow{\left(x-3y\right)^2}=0\Leftrightarrow x=3y$.\\
		Khi đó $M=\dfrac{1+\log_{12}x+\log_{12}y}{2\log_{12}\left(x+3y\right)}=\dfrac{\log_{12}\left(12xy\right)}{\log_{12}{\left(x+3y\right)^2}}=\dfrac{\log_{12}\left(36y^2\right)}{\log_{12}\left(36y^2\right)}=1$.
	}
\end{ex}
\begin{ex}%[2D2B3-2]
	[Đề Minh Họa 2020 Lần 1]
	Xét tất cả các số dương $a$ và $b$ thỏa mãn $\log_2a=\log_8(ab)$. Mệnh đề nào dưới đây đúng?
	\choice
	{$a=b^2$}
	{$a^3=b$}
	{$a=b$}
	{\True $a^2=b$}
	\loigiai{
		Theo đề ta có\\
		$\begin{aligned}
			&{\log_2}a=\log_8(ab)\Leftrightarrow{\log_2}a=\dfrac{1}{3}{\log_2}(ab)\Leftrightarrow 3\log_2a=\log_2(ab)\\ 
			&\Leftrightarrow{\log_2}{a^3}=\log_2(ab)\Leftrightarrow{a^3}=ab\Leftrightarrow{a^2}=b\\ 
		\end{aligned}$.
	}
\end{ex}
\begin{ex}%[2D2B3-2]
	[Đề Tham Khảo 2020 Lần 2]
	Xét số thực $a$ và $b$ thỏa mãn $\log_3\left(3^a{9^b}\right)=\log_93$. Mệnh đề nào dưới đây đúng
	\choice
	{$a+2b=2$}
	{$4a+2b=1$}
	{$4ab=1$}
	{\True $2a+4b=1$}
	\loigiai{
		Ta có
		$\begin{aligned}[t]
			{\log_3}\left(3^a{9^b}\right)=\log_93&\Leftrightarrow{\log_3}\left(3^a{3^{2b}}\right)=\log_{3^2}3\\ 
			&\Leftrightarrow{\log_3}{3^{a+2b}}=\log_33^{\frac{1}{2}}\Leftrightarrow a+2b=\dfrac{1}{2}\Leftrightarrow 2a+4b=1\\ 
		\end{aligned}$.
	}
\end{ex}
\begin{ex}%[2D2B3-2]
	[Mã 102 - 2020 Lần 1]
	Cho $a$ và $b$ là các số thực dương thỏa mãn $4^{\log_2(ab)}=3a$. Giá trị của $ab^2$ bằng
	\choice
	{\True $3$}
	{$6$}
	{$2$}
	{$12$}
	\loigiai{
		Từ giả thiết ta có
		$\begin{aligned}[t]
			4^{\log _2(a b)}=3 a&\Leftrightarrow \log _2(a b) \cdot \log _2 4=\log _2(3 a) \\
			& \Leftrightarrow 2\left(\log _2 a+\log _2 b\right)=\log _2 a+\log _2 3 \\
			&\Leftrightarrow \log _2 a+2 \log _2 b=\log _2 3 \\
			&\Leftrightarrow \log _2\left(a b^2\right)=\log _2 3 \\
			&\Leftrightarrow a b^2=3
		\end{aligned}$.
	}
\end{ex}
\begin{ex}%[2D2B3-2]
	[Mã 103 - 2020 Lần 1]
	Cho $a$ và $b$ là hai số thực dương thỏa mãn $9^{\log_3(ab)}=4a$. Giá trị của $a{b^2}$ bằng
	\choice
	{\True $3$}
	{$6$}
	{$2$}
	{$4$}
	\loigiai{
		Ta có $9^{\log_3\left(ab\right)}=4a\Leftrightarrow 2\log_3\left(ab\right)=\log_3\left(4a\right)$ $\Leftrightarrow{\log_3}\left(a^2b^2\right)=\log_3\left(4a\right)$ $\Rightarrow{a^2}{b^2}=4a$\\
		$\Leftrightarrow a{b^2}=4$.
	}
\end{ex}
\begin{ex}%[2D2B3-2]
	[Mã 102 - 2020 Lần 2]
	Với $a$, $b$ là các số thực dương tùy ý thỏa mãn $\log_3a-2\log_9b=2$, mệnh đề nào dưới đây đúng?
	\choice
	{$a=9b^2$}
	{\True $a=9b$}
	{$a=6b$}
	{$a=9b^2$}
	\loigiai{
		Ta có $\log_3a-2\log_9b=2$$\Leftrightarrow{\log_3}a-\log_3b=2$$\Leftrightarrow{\log_3}\left(\dfrac{a}{b}\right)=2$$\Leftrightarrow a=9b$.
	}
\end{ex}
\begin{ex}%[2D2B3-2]
	[Mã 103 - 2020 Lần 2]
	Với $a$, $b$ là các số thực dương tùy ý thỏa mãn $\log_3a-2\log_9b=3$, mệnh đề nào dưới đây đúng?
	\choice
	{\True $a=27b$}
	{$a=9b$}
	{$a=27b^4$}
	{$a=27b^2$}
	\loigiai{
		Ta có $\log_3a-2\log_9b=3\Leftrightarrow{\log_3}a-\log_3b=3\Leftrightarrow{\log_3}\dfrac{a}{b}=3\Leftrightarrow\dfrac{a}{b}=27\Leftrightarrow a=27b$.
	}
\end{ex}
\begin{ex}%[2D2B3-2]
	[Mã 104 - 2020 Lần 2]
	Với $a$, $b$ là các số thực dương tùy ý thỏa mãn $\log_2a-2\log_4b=4$, mệnh đề nào dưới đây đúng?
	\choice
	{$a=16b^2$}
	{$a=8b$}
	{\True $a=16b$}
	{$a=16b^4$}
	\loigiai{
		Ta có
		$\begin{aligned}[t]
			\log_2a-2\log_4b=4&\Leftrightarrow{\log_2}a-2\log_{2^2}b=4\\ 
			&\Leftrightarrow{\log_2}a-2.\dfrac{1}{2}{\log_2}b=4
			\Leftrightarrow{\log_2}a-\log_2b=4\\
			&\Leftrightarrow{\log_2}\dfrac{a}{b}=4
			\Leftrightarrow\dfrac{a}{b}=2^4\\ 
			&\Leftrightarrow a=16b\\ 
		\end{aligned}$.
	}
\end{ex}
\begin{ex}%[2D2B3-2]
	[Mã 101 - 2021 Lần 1]
	Với mọi $ a,b$ thỏa mãn $\log_2a^3+\log_2b=6$, khẳng định nào dưới đây đúng
	\choice
	{\True $a^3b=64$}
	{$a^3b=36$}
	{$a^3+b=64$}
	{$a^3+b=36$}
	\loigiai{
		Ta có $\log_2a^3+\log_2b=6\Leftrightarrow{a^3}b=2^6\Leftrightarrow{a^3}b=64$.
	}
\end{ex}
\begin{ex}%[2D2B3-2]
	[Mã 102 - 2021 Lần 1]
	Với mọi $a$, $b$ thỏa mãn $\log_2a^3+\log_2b=8$. Khẳng định nào dưới đây đúng?
	\choice
	{$a^3+b=64$}
	{\True $a^3b=256$}
	{$a^3b=64$}
	{$a^3+b=256$}
	\loigiai{
		Ta có $\log_2a^3+\log_2b=8\Rightarrow{\log_2}\left(a^3b\right)=8\Leftrightarrow{a^3}b=2^8=256$.\\
		Vậy $a^3b=256$.
	}
\end{ex}
\begin{ex}%[2D2B3-2]
	[Mã 104 - 2021 Lần 1]
	Với mọi $a$, $b$ thỏa mãn $\log_2a^3+\log_2b=5$, khẳng định nào dưới đây là đúng?
	\choice
	{\True $a^3b=32$}
	{$a^3b=25$}
	{$a^3+b=25$}
	{$a^3+b=32$}
	\loigiai{
		Ta có $\log_2a^3+\log_2b=5\Leftrightarrow{\log_2}\left(a^3b\right)=5\Leftrightarrow{a^3}b=32$.
	}
\end{ex}
\begin{ex}%[2D2B3-2]
	[Mã 103 - 2021 - Lần 1]
	Với mọi $a$, $b$ thỏa mãn $\log_2a^2+\log_2b=7$, khẳng định nào dưới đây đúng?
	\choice
	{$a^2+b=49$}
	{\True $a^2b=128$}
	{$a^2+b=128$}
	{$a^2b=49$}
	\loigiai{
		Ta có $\log_2a^2+\log_2b=7\Leftrightarrow{\log_2}\left(a^2b\right)=7\Leftrightarrow{a^2}b=2^7=128$.
	}
\end{ex}
\begin{ex}%[2D2B3-2]
	[Chuyên Bắc Giang 2019]
	Cho các số thực dương $a$, $b$ thỏa mãn $\ln a=x;\ln b=y$. Tính $\ln \left(a^3b^2\right)$
	\choice
	{$P=x^2y^3$}
	{$P=6xy$}
	{\True $P=3x+2y$}
	{$P=x^2+y^2$}
	\loigiai{
		Ta có $\ln \left(a^3b^2\right)=\ln {a^3}+\ln {b^2}=3\ln a+2\ln b=3x+2y$.
	}
\end{ex}
\begin{ex}%[2D2B3-2]
	[Chuyên Vĩnh Phúc 2019]
	Giá trị của biểu thức $M=\log_22+\log_24+\log_28+\ldots+\log_2256$ bằng
	\choice
	{$48$}
	{$56$}
	{\True $36$}
	{$8\log_2256$}
	\loigiai{
		Ta có
		$$\begin{aligned}[t]
			M=\log_22+\log_24+\log_28+\ldots+\log_2256&
			=\log_2\left(2.4.8\ldots256\right)=\log_2\left(2^12^22^3\ldots2^8\right)\\
			&=\log_2\left(2^{1+2+3+\ldots+8}\right)=\left(1+2+3+\ldots+8\right){\log_2}2\\
			&=1+2+3+\ldots+8=36
		\end{aligned}$$
	}
\end{ex}
\begin{ex}%[2D2B3-2]
	[THCS - THPT Nguyễn Khuyến 2019]
	Cho $\log_8c=m$ và $\log_{c^3}2=n$. Khẳng định đúng là
	\choice
	{$mn=\dfrac{1}{9}{\log_2}c$}
	{$mn=9$}
	{$mn=9\log_2c$}
	{\True $mn=\dfrac{1}{9}$}
	\loigiai{
		Ta có $mn=\log_8c.\log_{c^3}2=\left(\dfrac{1}{3}{\log_2}c\right).\left(\dfrac{1}{3}{\log_c}2\right)=\dfrac{1}{9}$.
	}
\end{ex}
\begin{ex}%[2D2B3-2]
	[THPT Lương Thế Vinh Hà Nội 2019]
	Cho $a>0$, $a\ne 1$ và $\log_ax=-1$, $\log_ay=4$. Tính $P=\log_a\left(x^2y^3\right)$.
	\choice
	{$P=18$}
	{$P=6$}
	{$P=14$}
	{\True $P=10$}
	\loigiai{
		Ta có $\log_a\left(x^2.y^3\right)=\log_a{x^2}+\log_a{y^3}=2\log_ax+3\log_ay=2.(-1)+3.4=10$.
	}
\end{ex}
\begin{ex}%[2D2B3-2]
	[Sở Bình Phước 2019]
	Với $a$ và $b$ là hai số thực dương tùy ý; $\log_2\left(a^3b^4\right)$ bằng
	\choice
	{$\dfrac{1}{3}{\log_2}a+\dfrac{1}{4}{\log_2}b$}
	{\True $3\log_2a+4\log_2b$}
	{$2\left(\log_2a+\log_4b\right)$}
	{$4\log_2a+3\log_2b$}
	\loigiai{
		Ta có $\log_2\left(a^3b^4\right)=\log_2a^3+\log_2b^4=3\log_2a+4\log_2b$ nên B đúng. 
	}
\end{ex}
\begin{ex}%[2D2B3-2]
	[Chuyên Hạ Long-2019]
	Cho $P=\sqrt[20]{3\sqrt[7]{27\sqrt[4]{243}}}$. Tính $\log_3P$?
	\choice
	{$\dfrac{45}{28}$}
	{\True $\dfrac{9}{112}$}
	{$\dfrac{45}{56}$}
	{Đáp án khác}
	\loigiai{
		Ta có $P=\sqrt[20]{3\sqrt[7]{27\sqrt[4]{243}}}\Rightarrow P=3^{\frac{1}{20}}{27^{\frac{1}{20}.\frac{1}{7}}}{243^{\frac{1}{20}.\frac{1}{7}.\frac{1}{4}}}=3^{\frac{9}{112}}$$\Rightarrow{\log_3}P=\log_33^{\frac{9}{112}}=\dfrac{9}{112}$.
	}
\end{ex}
\begin{ex}%[2D2B3-2]
	[THPT Cẩm Giàng 2 2019]
	Cho các số dương $a$, $b$, $c$, $d$. Biểu thức $S=\ln \dfrac{a}{b}+\ln \dfrac{b}{c}+\ln \dfrac{c}{d}+\ln \dfrac{d}{a}$ bằng
	\choice
	{$1$}
	{\True $0$}
	{$\ln \left(\dfrac{a}{b}+\dfrac{b}{c}+\dfrac{c}{d}+\dfrac{d}{a}\right)$}
	{$\ln \left(abcd\right)$}
	\loigiai{
		Ta có $S=\ln \dfrac{a}{b}+\ln \dfrac{b}{c}+\ln \dfrac{c}{d}+\ln \dfrac{d}{a}=\ln \left(\dfrac{a}{b}\cdot\dfrac{b}{c}\cdot\dfrac{c}{d}\cdot\dfrac{d}{a}\right)=\ln 1=0$.
	}
\end{ex}
\begin{ex}%[2D2B3-2]
	Cho $x$, $y$ là các số thực dương tùy ý, đặt $\log_3x=a$, $\log_3y=b$. Chọn mệnh đề đúng.
	\choice
	{$\log_{\frac{1}{27}}\left(\dfrac{x}{y^3}\right)=\dfrac{1}{3}a-b$}
	{$\log_{\frac{1}{27}}\left(\dfrac{x}{y^3}\right)=\dfrac{1}{3}a+b$}
	{$\log_{\frac{1}{27}}\left(\dfrac{x}{y^3}\right)=-\dfrac{1}{3}a-b$}
	{\True $\log_{\frac{1}{27}}\left(\dfrac{x}{y^3}\right)=-\dfrac{1}{3}a+b$}
	\loigiai{
		Do $x$, $y$ là các số thực dương nên ta có
		$$\begin{aligned}[t]
			\log_{\frac{1}{27}}\left(\dfrac{x}{y^3}\right)=-\dfrac{1}{3}{\log_3}\left(\dfrac{x}{y^3}\right)&=-\dfrac{1}{3}\left(\log_3x-\log_3y^3\right)\\
			&=-\dfrac{1}{3}\left(\log_3x-3\log_3y\right)=-\dfrac{1}{3}{\log_3}x+\log_3y=-\dfrac{1}{3}a+b.
		\end{aligned}$$
	}
\end{ex}
\begin{ex}%[2D2B3-2]
	[THPT Bạch Đằng Quảng Ninh 2019]
	Với $a$, $b$ là các số thực dương tùy ý và $a$ khác $1$, đặt $P=\log_a{b^3}+\log_{a^2}{b^6}$. Mệnh đề nào dưới đây đúng?
	\choice
	{$P=27\log_ab$}
	{$P=15\log_ab$}
	{$P=9\log_ab$}
	{\True $P=6\log_ab$}
	\loigiai{
		Ta có $P=\log_a{b^3}+\log_{a^2}{b^6}=3\log_ab+6.\dfrac{1}{2}{\log_a}b=6\log_ab$.
	}
\end{ex}
\begin{ex}%[2D2B3-2]
	[THPT Quang Trung Đống Đa Hà Nội 2019]
	Với các số thực dương$ a,b$ bất kỳ $ a\ne 1$. Mệnh đề nào dưới đây đúng?
	\choice
	{\True $\log_a\dfrac{\sqrt[3]{a}}{b^2}=\dfrac{1}{3}-2\log_ab$}
	{$\log_a\dfrac{\sqrt[3]{a}}{b^2}=3-\dfrac{1}{2}{\log_a}b$}
	{$\log_a\dfrac{\sqrt[3]{a}}{b^2}=\dfrac{1}{3}-\dfrac{1}{2}{\log_a}b$}
	{$\log_a\dfrac{\sqrt[3]{a}}{b^2}=3-2\log_ab$}
	\loigiai{
		Ta có
		$${\log_a}\dfrac{\sqrt[3]{a}}{b^2}=\log_a\sqrt[3]{a}-\log_a{b^2}{=}{\log_a}{a^{\frac{1}{3}}}-2\log_ab=\dfrac{1}{3}{\log_a}a-2\log_ab=\dfrac{1}{3}-2\log_ab$$	
	}
\end{ex}
\begin{ex}%[2D2B3-2]
	[Chuyên Lê Hồng Phong Nam Định 2019]
	Cho các số thực dương $a$, $b$, $c$ với $a$ và $b$ khác $1$. Khẳng định nào sau đây là đúng?
	\choice
	{$\log_a{b^2}.\log_{\sqrt{b}}c=\log_ac$}
	{$\log_a{b^2}.\log_{\sqrt{b}}c=\dfrac{1}{4}{\log_a}c$}
	{\True $\log_a{b^2}.\log_{\sqrt{b}}c=4\log_ac$}
	{$\log_a{b^2}.\log_{\sqrt{b}}c=2\log_ac$}
	\loigiai{
		Ta có $\log_a{b^2}.\log_{\sqrt{b}}c=2\log_ab.\log_{b^{\frac{1}{2}}}c$$=2\log_ab.2\log_bc$$=4\log_ab.\log_bc$$=4\log_ac$.
	}
\end{ex}
\begin{ex}%[2D2B3-2]
	[Chuyên Bắc Giang-2019]
	Giả sử $ a,b$ là các số thực dương bất kỳ. Mệnh đề nào sau đây sai?
	\choice
	{$\log{\left(10ab\right)^2}=2+\log{\left(ab\right)^2}$}
	{\True $\log{\left(10ab\right)^2}=\left(1+\log a+\log b\right)^2$}
	{$\log{\left(10ab\right)^2}=2+2\log\left(ab\right)$}
	{$\log{\left(10ab\right)^2}=2\left(1+\log a+\log b\right)$}
	\loigiai{
		Ta có $\log{\left(10ab\right)^2}=\log{10^2}+\log{\left(ab\right)^2}=2+2\log\left(ab\right)$.
	}
\end{ex}
\begin{ex}%[2D2B3-2]
	[THPT Lương Thế Vinh Hà Nội 2019]
	Cho $\log_ab=3,\log_ac=-2$. Khi đó $\log_a\left(a^3b^2\sqrt{c}\right)$ bằng bao nhiêu?
	\choice
	{$13$}
	{$5$}
	{\True $8$}
	{$10$}
	\loigiai{
		Ta có $\log_a\left(a^3b^2\sqrt{c}\right)$ $=\log_a{a^3}+\log_a{b^2}+\log_a\sqrt{c}$ $=3+2\log_ab+\dfrac{1}{2}{\log_a}c$ $=3+2.3-\dfrac{1}{2}.2=8$.
	}
\end{ex}
\begin{ex}%[2D2B3-2]
	[THPT Lê Quy Đôn Điện Biên 2019]
	Rút gọn biểu thức $M=3\log_{\sqrt{3}}\sqrt{x}-6\log_9\left(3x\right)+\log_{\dfrac{1}{3}}\dfrac{x}{9}$.
	\choice
	{\True $M=-\log_3\left(3x\right)$}
	{$M=2+\log_3\left(\dfrac{x}{3}\right)$}
	{$M=-\log_3\left(\dfrac{x}{3}\right)$}
	{$M=1+\log_3x$}
	\loigiai{
		Điều kiện $x>0$. Ta có
		$$M=3\log_3x-3\left(1+\log_3x\right)-\log_3x+2=-1-\log_3x=-\left(1+\log_3x\right)=-\log_3\left(3x\right).$$
	}
\end{ex}
\begin{ex}%[2D2B3-2]
	[Chuyên Lê Thánh Tông 2019]
	Cho $\log_8\left|x\right|+\log_4y^2=5$ và $\log_8\left|y\right|+\log_4x^2=7$. Tìm giá trị của biểu thức $P=\left|x\right|-\left|y\right|$.
	\choice
	{\True $P=56$}
	{$P=16$}
	{$P=8$}
	{$P=64$}
	\loigiai{
		Điều kiện $x,y\ne 0$\\
		Cộng vế với vế của hai phương trình, ta được $$\log_8\left|xy\right|+\log_4x^2y^2=12\Leftrightarrow{\log_2}\left| xy\right|=9\Leftrightarrow\left| xy\right|=512\quad\hfill (1)$$
		Trừ vế với vế của hai phương trình, ta được
		$$\log_8\left|\dfrac{x}{y}\right|+\log_4\dfrac{y^2}{x^2}=-2\Leftrightarrow{\log_2}\left|\dfrac{x}{y}\right|=3\Leftrightarrow\left|\dfrac{x}{y}\right|=8\Leftrightarrow\left| x\right|=8\left| y\right|\quad (2)$$
		Từ $(1)$ và $(2)$ suy ra $\left| y\right|=8\Rightarrow\left| x\right|=64\Leftrightarrow P=56$.
	}
\end{ex}
\begin{ex}%[2D2B3-2]
	[Hsg Bắc Ninh 2019]
	Cho hai số thực dương $a$, $b$ .Nếu viết $\log_2\dfrac{\sqrt[6]{64a^3b^2}}{ab}=1+x{\log_2}a+y{\log_4}b$ $(x,y\in\mathbb{Q})$ thì biểu thức $P=xy$ có giá trị bằng bao nhiêu?
	\choice
	{$P=\dfrac{1}{3}$}
	{\True $P=\dfrac{2}{3}$}
	{$P=-\dfrac{1}{12}$}
	{$P=\dfrac{1}{12}$}
	\loigiai{
		Ta có
		$$\begin{aligned}[t]
			\log_2\dfrac{\sqrt[6]{64a^3b^2}}{ab}&=\log_264^{\frac{1}{6}}+\dfrac{1}{2}{\log_2}a+\dfrac{1}{3}{\log_2}b-\log_2a-\log_2b\\
			&=1-\dfrac{1}{2}{\log_2}a-\dfrac{4}{3}{\log_4}b	
		\end{aligned}$$
		Khi đó $x=-\dfrac{1}{2};y=-\dfrac{4}{3}\Rightarrow P=xy=\dfrac{2}{3}$.
	}
\end{ex}
\begin{ex}%[2D2B3-2]
	Cho $\log_{700}490=a+\dfrac{b}{c+\log 7}$ với $a$, $b$, $c$ là các số nguyên. Tính tổng $T=a+b+c$.
	\choice
	{$T=7$}
	{$T=3$}
	{$T=2$}
	{\True $ =1$}
	\loigiai{
		Ta có $\log_{700}490=\dfrac{\log 490}{\log 700}=\dfrac{\log 10+\log 49}{\log 100+\log 7}=\dfrac{1+2\log 7}{2+\log 7}=\dfrac{4+2\log 7-3}{2+\log 7}=2+\dfrac{-3}{2+\log 7}$\\
		Suy ra $a=2$, $b=-3$, $c=2$\\
		Vậy $T=1$.
	}
\end{ex}
\begin{ex}%[2D2B3-2]
	Cho $a$, $b$ là hai số thưc dương thỏa mãn $a^2+b^2=14ab$. Khẳng định nào sau đây sai?
	\choice
	{$2\log_2\left(a+b\right)=4+\log_2a+\log_2b$}
	{$\ln \dfrac{a+b}{4}=\dfrac{\ln a+\ln b}{2}$}
	{$ 2\log\dfrac{a+b}{4}=\log a+\log b$}
	{\True $2\log_4\left(a+b\right)=4+\log_4a+\log_4b$}
	\loigiai{
		Ta có $a^2+b^2=14ab\Leftrightarrow{\left(a+b\right)^2}=16ab$.\\
		Suy ra $\log_4\left(a+b\right)^2=\log_4\left(16ab\right)\Leftrightarrow 2\log_4\left(a+b\right)=2+\log_4a+\log_4b$.
	}
\end{ex}
\begin{ex}%[2D2B3-2]  
	Cho $x$, $y$ là các số thực dương tùy ý, đặt $\log_3x=a$, $\log_3y=b$. Chọn mệnh đề đúng.
	\choice
	{$\log_{\frac{1}{27}}\left(\dfrac{x}{y^3}\right)=\dfrac{1}{3}a-b$}
	{$\log_{\frac{1}{27}}\left(\dfrac{x}{y^3}\right)=\dfrac{1}{3}a+b$}
	{$\log_{\frac{1}{27}}\left(\dfrac{x}{y^3}\right)=-\dfrac{1}{3}a-b$}
	{\True $\log_{\frac{1}{27}}\left(\dfrac{x}{y^3}\right)=-\dfrac{1}{3}a+b$}
	\loigiai{
		$\log_{\frac{1}{27}}\left(\dfrac{x}{y^3}\right)=\log_{3^{-3}}\left(\dfrac{x}{y^3}\right)=-\dfrac{1}{3}{\log_3}\left(\dfrac{x}{y^3}\right)=-\dfrac{1}{3}\left(\log_3x-\log_3y^3\right)=-\dfrac{1}{3}{\log_3}x+\log_3y=-\dfrac{1}{3}a+b$.
	}
\end{ex}
\begin{ex}%[2D2B3-2]
	[Sở Vĩnh Phúc 2019]
	Cho $\alpha=\log_ax$, $\beta=\log_bx$. Khi đó $\log_{a{b^2}}{x^2}$ bằng.
	\choice            
	{$\dfrac{\alpha\beta}{\alpha{+}\beta}$}
	{\True $\dfrac{{2}\alpha\beta}{{2}\alpha{+}\beta}$}
	{$\dfrac{{2}}{{2}\alpha{+}\beta}$}
	{$\dfrac{{2}\left(\alpha{+}\beta\right)}{\alpha{+2}\beta}$}
	\loigiai{
		Ta có $\begin{aligned}
			\log_{a{b^2}}{x^2}=2\log_{a{b^2}}x&=\dfrac{2}{\log_xa{b^2}}=\dfrac{2}{\log_xa+\log_x{b^2}}=\dfrac{2}{\frac{1}{\log_ax}+2.\frac{1}{\log_bx}}\\
			&=\dfrac{2}{\frac{1}{\alpha}+\frac{2}{\beta}}=\dfrac{2\alpha\beta}{\beta+2\alpha.}
		\end{aligned}$
	}
\end{ex}
\begin{ex}%[2D2B3-2]
	[THPT Bạch Đằng Quảng Ninh 2019]
	Tính giá trị biểu thức $P=\log_{a^2}\left(a^{10}{b^2}\right)+\log_{\sqrt{a}}\left(\dfrac{a}{\sqrt{b}}\right)+\log_{\sqrt[3]{b}}\left(b^{-2}\right)$ (với $0<a\ne 1$; $0<b\ne 1$).
	\choice
	{$\sqrt{3}$}
	{\True $1$}
	{$\sqrt{2}$}
	{$ 2$}
	\loigiai{
		Ta có $P=\log_{a^2}\left(a^{10}{b^2}\right)+\log_{\sqrt{a}}\left(\dfrac{a}{\sqrt{b}}\right)+\log_{\sqrt[3]{b}}\left(b^{-2}\right)=5+\log_ab+2-\log_ab-6=1$.
	}
\end{ex}
\begin{ex}%[2D2B3-2]
	[Toán Học Tuổi Trẻ 2019]
	Đặt $M=\log_656$, $N=a+\dfrac{\log_37-b}{\log_32+c}$ với $a,b,c\in\mathbb{R}$. Bộ số $a,b,c$ nào dưới đây để có $M=N?$
	\choice
	{\True $a=3$, $b=3$, $c=1$}
	{$a=3$, $b=\sqrt{2}$, $c=1$}
	{$a=1$, $b=2$, $c=3$}
	{$a=1$, $b=-3$, $c=2$}
	\loigiai{
		Ta có $\begin{aligned}
			M=\log_656=\dfrac{\log_356}{\log_36}&=\dfrac{\log_32^3.7}{1+\log_32}=\dfrac{3\log_32+\log_37}{1+\log_32}\\
			&=\dfrac{3\left(1+\log_32\right)+\log_37-3}{1+\log_32}=3+\dfrac{\log_37-3}{\log_32+1}
		\end{aligned}$\\
		Vậy $M=N\Leftrightarrow\left\{\begin{aligned}
			&a=3\\ 
			&b=3\\ 
			&c=1\\ 
		\end{aligned}\right.$.
	}
\end{ex}
\begin{ex}%[2D2B3-2]
	[THPT Yên Phong 1 Bắc Ninh 2019]
	Tính $T=\log\dfrac{1}{2}+\log\dfrac{2}{3}+\log\dfrac{3}{4}+\ldots+\log\dfrac{98}{99}+\log\dfrac{99}{100}$.
	\choice
	{$\dfrac{1}{10}$}
	{\True $-2$}
	{$\dfrac{1}{100}$}
	{$ 2$}
	\loigiai{
		$T=\log\dfrac{1}{2}+\log\dfrac{2}{3}+\log\dfrac{3}{4}+\ldots+\log\dfrac{98}{99}+\log\dfrac{99}{100}=\log\left(\dfrac{1}{2}\cdot\dfrac{2}{3}\cdot\dfrac{3}{4}\ldots\dfrac{98}{99}\cdot\dfrac{99}{100}\right)=\log\dfrac{1}{100}=\log{10^{-2}}=-2$.
	}
\end{ex}
\begin{ex}%[2D2B3-2]
	Cho $a,b,x>0;a>b$ và $b,x\ne 1$ thỏa mãn $\log_x\dfrac{a+2b}{3}=\log_x\sqrt{a}+\dfrac{1}{\log_b{x^2}}$.\\         
	Khi đó biểu thức $P=\dfrac{2a^2+3ab+b^2}{(a+2b)^2}$ có giá trị bằng     
	\choice
	{\True $P=\dfrac{5}{4}$}
	{$P=\dfrac{2}{3}$}
	{$P=\dfrac{16}{15}$}
	{$P=\dfrac{4}{5}$}
	\loigiai{
		$\log_x\dfrac{a+2b}{3}=\log_x\sqrt{a}+\dfrac{1}{\log_b{x^2}}\Leftrightarrow{\log_x}\dfrac{a+2b}{3}=\log_x\sqrt{a}+\log_x\sqrt{b}$\\
		$\Leftrightarrow a+2b=3\sqrt{ab}\Leftrightarrow{a^2}-5ab+4b^2=0\Leftrightarrow\left(a-b\right)\left(a-4b\right)=0\Leftrightarrow a=4b$ (do $ a>b$).\\
		$P=\dfrac{2a^2+3ab+b^2}{(a+2b)^2}=\dfrac{32b^2+12b^2+b^2}{36b^2}=\dfrac{5}{4}$.
	}
\end{ex}
\Closesolutionfile{ans}
\indapan{10}{ans/CD17/Muc_5_6}