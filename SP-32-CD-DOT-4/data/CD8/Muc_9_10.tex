\setcounter{dang}{0}
\setcounter{ex}{0} 
\Opensolutionfile{ans}[ans/CD8/Muc_9_10]
\begin{dang}{Khoảng cách từ điểm đến mặt phẳng}
\end{dang}
\begin{ex}%[1H3B5-3][Đề minh họa 2022]%Câu 1. 
	\immini{Cho hình lăng trụ đứng $ABC.A’B’C’$ có đáy $ABC$ là tam giác vuông cân tại $B$ và $AB=4$ (tham khảo hình bên). Khoảng cách từ $C$ đến mặt phẳng $(ABB’A’)$ bằng
		\choice
		{$2\sqrt{2}$}
		{$2$}
		{$4\sqrt{2}$}
		{\True $4$}}
	{\begin{tikzpicture}[scale=1,>=stealth, font=\footnotesize, line join=round, line cap=round]
			\path
			(0,0) coordinate (A)
			(1.1,-1.5) coordinate (B)
			(3.5,0) coordinate (C);
			\coordinate (A') at ($(A)+(0,3.2)$);
			\coordinate (B') at ($(B)+(0,3.2)$);
			\coordinate (C') at ($(C)+(0,3.2)$);
			\draw (A')--(B')--(C')--(A')--(A)--(B)--(C)--(C') (B)--(B');
			\draw[dashed] (A)--(C);
			\foreach \d/\g in{A'/135,B'/0,C'/45,A/180,B/-90,C/0}
			\draw[fill=black](\d)circle(1pt)node[shift={(\g:0.35)}]{$\d$};
			\tkzMarkRightAngles(A,B,C A',A,C)
			
	\end{tikzpicture}}
	\loigiai{
		Ta có
		$\heva{&CB\perp BB’\\&CB\perp AB}\Rightarrow CB\perp(ABB’A’)$ tại $B$. Vậy $\mathrm{d}\left(C;(ABB’A’)\right)=CB=AB=4$.}
\end{ex}
\begin{ex}%[1H3B5-3][Mã 104-2022]%Câu 2.
	\immini{Cho hình lập phương $ABCD.A’B’C’D’$ có cạnh bằng $3$ (tham khảo hình bên dưới).
		Khoảng cách từ $B$ đến mặt phẳng $(ACC’A’)$ bằng
		\choice
		{$3$}
		{$3\sqrt{2}$}
		{\True $\dfrac{3\sqrt{2}}{2}$}
		{$\dfrac{3}{2}$}}
	{\begin{tikzpicture}[scale=1,>=stealth, font=\footnotesize, line join=round, line cap=round]
			\path
			(0,0) coordinate (A)
			(-1,-1.1) coordinate (B)
			(2,-1.1) coordinate (C);
			\coordinate (D) at ($(A)+(C)-(B)$);
			\coordinate (A') at ($(A)+(0,3)$);
			\tkzDefPointsBy[translation=from A to A'](B,C,D){B'}{C'}{D'}
			\draw[dashed] (A')--(A)--(D) (A)--(B);
			\draw (B')--(C')--(D')--(A')--(B')--(B)--(C)--(C') (C)--(D)--(D');
			\foreach \d/\g in{A'/135,B'/180,C'/0,D'/45,A/135,B/-135,C/-45,D/0}
			\draw[fill=black](\d)circle(1pt)node[shift={(\g:0.35)}]{$\d$};
			\tkzMarkRightAngles(A',A,D)
			
	\end{tikzpicture}}
	\loigiai{
		\immini{
			Gọi $O$ là tâm hình vuông $ABCD$.\\
			Do $ABCD$ là hình vuông nên $BD\perp AC$ tại $O$.\\
			Do $ABCD.A’B’C’D’$ là hình lập phương nên $AA’\perp(ABCD)\Rightarrow AA’\perp BD$
			$ \Rightarrow BO\perp(ACC’A’) $ tại $O$\\
			$\Rightarrow\mathrm{d}\left(B,(ACC’A’)\right)=BO=\dfrac{1}{2}BD=\dfrac{3\sqrt{2}}{2}$.}
		{\begin{tikzpicture}[scale=1,>=stealth, font=\footnotesize, line join=round, line cap=round]
				\path
				(0,0) coordinate (A)
				(-1,-1.1) coordinate (B)
				(2,-1.1) coordinate (C);
				\coordinate (D) at ($(A)+(C)-(B)$);
				\coordinate (A') at ($(A)+(0,3)$);
				\coordinate (O) at ($(A)!0.5!(C)$);
				\tkzDefPointsBy[translation=from A to A'](B,C,D){B'}{C'}{D'}
				\draw[dashed] (A')--(A)--(D) (A)--(B) (A)--(C) (B)--(D);
				\draw (B')--(C')--(D')--(A')--(B')--(B)--(C)--(C') (C)--(D)--(D');
				\foreach \d/\g in{A'/135,B'/180,C'/0,D'/45,A/135,B/-135,C/-45,D/0,O/-90}
				\draw[fill=black](\d)circle(1pt)node[shift={(\g:0.35)}]{$\d$};
	\end{tikzpicture}}}
\end{ex}
\begin{ex}%[1H3B5-3][Mã 103 - 2022]%Câu 3.
	\immini{
		Cho hình lập phương $ABCD.A’B’C’D’$ có cạnh bằng 3 (tham khảo hình bên). Khoảng cách từ $B$ đến mặt phẳng $(ACC’A’)$ bằng}
	{\begin{tikzpicture}[scale=0.8,>=stealth, font=\footnotesize, line join=round, line cap=round]
			\path
			(0,0) coordinate (A)
			(-1,-1.1) coordinate (B)
			(2,-1.1) coordinate (C);
			\coordinate (D) at ($(A)+(C)-(B)$);
			\coordinate (A') at ($(A)+(0,3)$);
			%\coordinate (O) at ($(A)!0.5!(C)$);
			\tkzDefPointsBy[translation=from A to A'](B,C,D){B'}{C'}{D'}
			\draw[dashed] (A')--(A)--(D) (A)--(B);
			\draw (B')--(C')--(D')--(A')--(B')--(B)--(C)--(C') (C)--(D)--(D');
			\foreach \d/\g in{A'/135,B'/180,C'/0,D'/45,A/135,B/-135,C/-45,D/0}
			\draw[fill=black](\d)circle(1pt)node[shift={(\g:0.35)}]{$\d$};
	\end{tikzpicture}}
	\choice
	{\True $\dfrac{3\sqrt{2}}{2}$}
	{$\dfrac{3}{2}$}
	{$3\sqrt{2}$}
	{$3$}
	\loigiai{
		\immini{
			Gọi $H$ là trung điểm của $AC$.\\
			Vì $ABCD.A’B’C’D’$ là hình lập phương nên $BH\perp(ACC’A’)$ \\
			$ \Rightarrow\mathrm{d}\left(B;(ACC’A’)\right)=BH=\dfrac{1}{2}AC $.\\
			Mà $ABCD$ là hình vuông cạnh 3 nên $AC=3\sqrt{2}$ \\
			$ \Rightarrow\mathrm{d}\left(B;(ACC’A’)\right)=\dfrac{3\sqrt{2}}{2} $.}
		{\begin{tikzpicture}[scale=0.8,>=stealth, font=\footnotesize, line join=round, line cap=round]
				\path
				(0,0) coordinate (A)
				(-1,-1.1) coordinate (B)
				(2,-1.1) coordinate (C);
				\coordinate (D) at ($(A)+(C)-(B)$);
				\coordinate (A') at ($(A)+(0,3)$);
				\coordinate (O) at ($(A)!0.5!(C)$);
				\tkzDefPointsBy[translation=from A to A'](B,C,D){B'}{C'}{D'}
				\draw[dashed] (A')--(A)--(D) (A)--(B) (A)--(C) (B)--(O);
				\draw (B')--(C')--(D')--(A')--(B')--(B)--(C)--(C')--(A') (C)--(D)--(D');
				\foreach \d/\g in{A'/135,B'/180,C'/0,D'/45,A/135,B/-135,C/-45,D/0,O/-90}
				\draw[fill=black](\d)circle(1pt)node[shift={(\g:0.35)}]{$\d$};
		\end{tikzpicture}}
	}
\end{ex}
\begin{ex}%[1H3B5-3][Mã 101 - 2021 Lần 1]%Câu 4.
	Cho hình chóp $S.ABC$ có đáy $ABC$ là tam giác vuông cân tại $B, AB=2a$ và SA vuông góc với mặt phẳng đáy. Tính khoảng cách từ $C$ đến mặt phẳng $(SAB)$ bằng
	\choice
	{$\sqrt{2}a$}
	{\True $2a$}
	{$a$}
	{$2\sqrt{2}a$}
	\loigiai{
		\immini{
			Vì $SA\perp(ABC)$ suy ra $CB\perp SA$.\\
			Tam giác $ABC$ vuông tại $B,$ nên $CB\perp AB$ (2).\\
			Từ (1) và (2), ta suy ra $CB\perp(SAB)$ nên khoảng cách từ $C$ đến mặt phẳng $(SAB)$ bằng CB.\\
			Mà tam giác $ABC$ vuông cân tại $B,$ suy ra $AB=BC=2a$.\\
			Vậy $\mathrm{d}_{(C,(SAB))}=CB=2a$.}
		{\begin{tikzpicture}[scale=1,>=stealth, font=\footnotesize, line join=round, line cap=round]
				\path 
				(0,0) coordinate (A)
				(1.2,-1.5) coordinate (B)
				(4,0) coordinate (C);
				\coordinate (S) at ($(A)+(0,3)$);
				\draw (S)--(A)--(B)--(S)--(C)--(B);
				\draw[dashed] (A)--(C);
				\foreach \d/\g in{S/90,A/180,B/-90,C/0}
				\draw[fill=black](\d)circle(1pt)node[shift={(\g:0.35)}]{$\d$};
				\tkzMarkRightAngles(S,A,C)
		\end{tikzpicture}}
	}
\end{ex}
\begin{ex}%[1H3B5-3][Mã 103 - 2021 - Lần 1]%Câu 5.
	Cho hình chóp $S.ABC$ có đáy là tam giác vuông cân tại $C$, $AC=a$ và $SA$ vuông góc với mặt phẳng đáy. Khoảng cách từ $B$ đến mặt phẳng $(SAC)$ bằng
	\choice
	{$\dfrac{1}{2}a$}
	{$\sqrt{2}a$}
	{$\dfrac{\sqrt{2}}{2}a$}
	{\True $a$}
	\loigiai{
		\immini{
			Ta có: $SA$ vuông góc với mặt đáy suy ra $SA\perp BC$.\\
			Tam giác $ABC$ vuông cân tại $C$ suy ra $BC=a$ và $AC\perp BC$.\\
			Do đó ta có: $\heva{&SA\perp BC\\&CA\perp BC}\Rightarrow BC\perp (SAC)$.\\
			Vậy khoảng cách từ $B$ đến mặt phẳng $(SAC)$ bằng $BC=a$.}
		{\begin{tikzpicture}[scale=1,>=stealth, font=\footnotesize, line join=round, line cap=round]
				\path 
				(0,0) coordinate (A)
				(1.2,-1.5) coordinate (C)
				(4,0) coordinate (B);
				\coordinate (S) at ($(A)+(0,3)$);
				\draw (S)--(A)--(C)--(S)--(B)--(C);
				\draw[dashed] (A)--(B);
				\foreach \d/\g in{S/90,A/180,B/-90,C/-90}
				\draw[fill=black](\d)circle(1pt)node[shift={(\g:0.35)}]{$\d$};
				\tkzMarkRightAngles(A,C,B)
				\tkzMarkRightAngles(S,A,B)
	\end{tikzpicture}}}
\end{ex}
\begin{ex}%[1H3B5-3]%Câu 6.
	(Mã 102 - 2021 Lần 1) Cho hình chóp $S.ABC$ có đáy là tam giác vuông cân tại $C$, $AC=3a$ và $SA$ vuông góc với mặt phẳng đáy. Khoảng cách từ $B$ đến mặt phẳng $(SAC)$ bằng
	\choice
	{$\dfrac{3}{2}a$}
	{$\dfrac{3\sqrt{2}}{2}a$}
	{\True $3a$}
	{$3\sqrt{2}a$}
	\loigiai{
		\immini{
			Ta có $\triangle ABC$ vuông cân tại $C$ nên $BC\perp AC(1)$ và $AC=BC=3a$.\\
			Mặt khác $SA\perp(ABC)\Rightarrow SA\perp BC(2)$.\\
			Từ $(1)$ và $(2)$ suy ra $BC\perp(SAC)\Rightarrow\mathrm{d}\left(B,(SAC)\right)=BC=3a$.\\
			Vậy khoảng cách từ $B$ đến mặt phẳng $(SAC)$ bằng $3a$.}
		{\begin{tikzpicture}[scale=1,>=stealth, font=\footnotesize, line join=round, line cap=round]
				\path 
				(0,0) coordinate (A)
				(1.2,-1.5) coordinate (C)
				(4,0) coordinate (B);
				\coordinate (S) at ($(A)+(0,3)$);
				\draw (S)--(A)--(C)--(S)--(B)--(C);
				\draw[dashed] (A)--(B);
				\foreach \d/\g in{S/90,A/180,B/-90,C/-90}
				\draw[fill=black](\d)circle(1pt)node[shift={(\g:0.35)}]{$\d$};
				\tkzMarkRightAngles(A,C,B)
				\tkzMarkRightAngles(S,A,B)
	\end{tikzpicture}}}
\end{ex}
\begin{ex}%[1H3B5-3][Mã 104 - 2021 Lần 1]%Câu 7.
	Cho hình chóp $S.ABC$ có đáy là tam giác vuông cân tại $B$, $AB=4a$ và $SA$ vuông góc với mặt phẳng đáy. Khoảng cách từ $C$ đến mặt phẳng $(SAB)$ bằng
	\choice
	{\True $4a$}
	{$4\sqrt{2}a$}
	{$2\sqrt{2}a$}
	{$2a$}
	\loigiai{
		\immini{
			Ta có: $\heva{&BC\perp AB\\&BC\perp SA~\left( SA\perp(ABC)\right)\\&AB\cap SA=A}\Rightarrow BC\perp(SAB)$ tại $B$.\\
			Suy ra $\mathrm{d}\left(C,(SAB)\right)=CB$.\\
			Xét $\triangle ABC$ vuông cân tại $B$ có: $BC=AB=4a$.\\
			Vậy $\mathrm{d}\left(C,(SAB)\right)=4a$.}
		{\begin{tikzpicture}[scale=1,>=stealth, font=\footnotesize, line join=round, line cap=round]
				\path 
				(0,0) coordinate (A)
				(1.2,-1.5) coordinate (B)
				(4,0) coordinate (C);
				\coordinate (S) at ($(A)+(0,3)$);
				\draw (S)--(A)--(B)--(S)--(C)--(B);
				\draw[dashed] (A)--(C);
				\foreach \d/\g in{S/90,A/180,B/-90,C/0}
				\draw[fill=black](\d)circle(1pt)node[shift={(\g:0.35)}]{$\d$};
				\tkzMarkRightAngles(A,B,C)
				\tkzMarkRightAngles(S,A,C)
	\end{tikzpicture}}}
\end{ex}
\begin{ex}%[1H3B5-3][Đề minh họa 2021]%Câu 8.
	\immini{
		Cho hình chóp tứ giác đều $S.ABCD$ có độ dài cạnh đáy bằng $2$ và độ dài cạnh bên bằng $3$ (tham khảo hình vẽ bên). Khoảng cách từ $S$ đến mặt phẳng $ABCD$ bằng 
		\choice
		{\True $\sqrt{7}$}
		{$1$}
		{$7$}
		{$\sqrt{11}$}}
	{\begin{tikzpicture}[scale=1,>=stealth, font=\footnotesize, line join=round, line cap=round]
			\path
			(0,0) coordinate (A)
			(-1.9,-1.6) coordinate (B)
			(1.6,-1.6) coordinate (C);
			\coordinate (O) at ($(A)!0.5!(C)$);
			\coordinate (D) at ($(A)+(C)-(B)$);
			\coordinate (O) at ($(A)!1/2!(C)$);
			\coordinate (S) at ($(O)+(0,3.5)$);
			\draw (S)--(B)--(C)--(S)--(D)--(C);
			\draw[dashed] (S)--(A)--(C) (B)--(A)--(D)--(B) (S)--(O);
			\foreach \d/\g in{S/90,A/180,B/-135,C/-45,D/45,O/-90}
			\draw[fill=black](\d)circle(1pt)node[shift={(\g:0.35)}]{$\d$};
			\tkzMarkRightAngles(S,O,D)
			
	\end{tikzpicture}}
	\loigiai{
		\immini{
			Gọi $O$ là tâm đáy $ABCD$. Vì $S.ABCD$ là hình chóp đều nên $SO$ là đường cao khối chóp.\\
			Khi đó $\mathrm{d}(S;ABCD)=SO$.\\
			Ta có $AO=\dfrac{1}{2}AC=\dfrac{1}{2}\sqrt{AB^2+AC^2}=\sqrt{2}\Rightarrow SO=\sqrt{SA^2-AO^2}=\sqrt{3^2-2}=\sqrt{7}$.}
		{\begin{tikzpicture}[scale=1,>=stealth, font=\footnotesize, line join=round, line cap=round]
				\path
				(0,0) coordinate (A)
				(-1.9,-1.6) coordinate (B)
				(1.6,-1.6) coordinate (C);
				\coordinate (O) at ($(A)!0.5!(C)$);
				\coordinate (D) at ($(A)+(C)-(B)$);
				\coordinate (O) at ($(A)!1/2!(C)$);
				\coordinate (S) at ($(O)+(0,3.5)$);
				\draw (S)--(B)--(C)--(S)--(D)--(C);
				\draw[dashed] (S)--(A)--(C) (B)--(A)--(D)--(B) (S)--(O);
				\foreach \d/\g in{S/90,A/180,B/-135,C/-45,D/45,O/-90}
				\draw[fill=black](\d)circle(1pt)node[shift={(\g:0.35)}]{$\d$};
				\tkzMarkRightAngles(S,O,D)
				
		\end{tikzpicture}}
	}
\end{ex}
\begin{ex}%[1H3B5-3]%[Sở Lào Cai - 2021]%Câu 9.
	Cho tứ diện $OABC$ có $OA$, $OB$, $OC$ đôi một vuông góc và $OA=OB=2a$, $OC=a\sqrt{2}$. Khoảng cách từ điểm $O$ đến mặt phẳng $(ABC)$ bằng
	\choice
	{$a\sqrt{2}$}
	{\True $a$}
	{$\dfrac{a}{2}$}
	{$\dfrac{3a}{4}$}
	\loigiai{
		\immini{
			Xét hệ trục tọa độ $Oxyz$ như sau điểm $O$ là gốc tọa độ $OA\equiv Oz$; $OB\equiv Ox$ và $OC\equiv Oy$. Khi đó ta có $O(0; 0; 0)$; $A(0; 0; 2a)$; $B(2a; 0; 0)$ và $C\left(0; a\sqrt{2}; 0\right)$.\\
			Phương trình mặt phẳng $(ABC)$ là $$\dfrac{x}{2a}+\dfrac{y}{a\sqrt{2}}+\dfrac{z}{2a}=1\Leftrightarrow x+\sqrt{2}y+z-2a=0.$$
			Khoảng cách từ điểm $O$ đến mặt phẳng $(ABC)$ là $\mathrm{d}\left(O,(ABC)\right)=\dfrac{\left|0+\sqrt{2}\cdot 0+0-2a\right|}{\sqrt{1+2+1}}=a$.}
		{\begin{tikzpicture}[scale=1,>=stealth, font=\footnotesize, line join=round, line cap=round]
				\path
				(0,0) coordinate (O)
				(3,0) coordinate (A)
				(0,3) coordinate (B)
				(-2,-2) coordinate (C);
				\draw (A)--(B)--(C)--(A);
				\draw[dashed] (A)--(O)--(B) (O)--(C);
				\foreach \d/\g in{A/0,B/180,C/180,O/190}
				\draw[fill=black](\d)circle(1pt)node[shift={(\g:0.35)}]{$\d$};
				\tkzMarkRightAngles(A,O,B A,O,C B,O,C)
		\end{tikzpicture}}
	}
\end{ex}
\begin{ex}%[1H3B5-3][Liên trường Quỳnh Lưu - Hoàng Mai - Nghệ An - 2021]%Câu 10.
	Cho hình chóp $S.ABC$ có đáy $ABC$ là tam giác vuông tại $A$, $AB=a,AC=a\sqrt{2}$. Biết thể tích khối chóp $S.ABC$ bằng $\dfrac{a^3}{2}$. Khoảng cách từ $S$ đến mặt phẳng $(ABC)$ bằng
	\choice
	{$\dfrac{a\sqrt{2}}{2}$}
	{$\dfrac{a\sqrt{2}}{6}$}
	{$\dfrac{3a\sqrt{2}}{4}$}
	{\True $\dfrac{3a\sqrt{2}}{2}$}
	\loigiai{
		\immini{
			Ta có $S_{\triangle ABC}=\dfrac{1}{2}AB\cdot AC=\dfrac{a^2\sqrt{2}}{2}$ \\
			$\Rightarrow\mathrm{d}\left(S,(ABC)\right)=\dfrac{3V_{S.ABC}}{S_{\triangle ABC}}=\dfrac{3\cdot\dfrac{a^3}{2}}{\dfrac{a^2\sqrt{2}}{2}}=\dfrac{3a\sqrt{2}}{2}$.}
		{\begin{tikzpicture}[scale=1,>=stealth, font=\footnotesize, line join=round, line cap=round]
				\path 
				(0,0) coordinate (A)
				(1.2,-1.5) coordinate (B)
				(4,0) coordinate (C);
				\coordinate (S) at ($(A)+(0,3)$);
				\draw (S)--(A)--(B)--(S)--(C)--(B);
				\draw[dashed] (A)--(C);
				\foreach \d/\g in{S/90,A/180,B/-90,C/0}
				\draw[fill=black](\d)circle(1pt)node[shift={(\g:0.35)}]{$\d$};
				\tkzMarkRightAngles(S,A,C)
				
	\end{tikzpicture}}}
\end{ex}
\begin{ex}%[1H3B5-3][THPT Hoàng Hoa Thám - Đà Nẵng - 2021]%Câu 11.
	Cho khối chóp đều $S.ABC$ có cạnh đáy bằng $a$. Gọi $M$ là trung điểm của $SA$. Biết thể tích của khối chóp đó bằng $\dfrac{a^3}{2}$, khoảng cách từ điểm $M$ đến mặt phẳng $(ABC)$ bằng
	\choice
	{$a\sqrt{3}$}
	{\True $3a$}
	{$\dfrac{a\sqrt{3}}{3}$}
	{$2a\sqrt{3}$}
	\loigiai{
		\immini{
			Ta có $S_{ABC}=\dfrac{1}{2}AB\cdot AC\cdot\sin\widehat{BAC}=\dfrac{a^2\sqrt{3}}{4}$.\\
			Mà $V_{S.ABC}=\dfrac{1}{3}S_{ABC}\cdot\mathrm{d}\left(S,(ABC)\right)\Rightarrow\mathrm{d}\left(S,(ABC)\right)=\dfrac{3V_{S.ABC}}{S_{ABC}}=2a\sqrt{3}$.\\
			Khi đó $\mathrm{d}\left(M,(ABC)\right)=\dfrac{1}{2}\mathrm{d}\left(S,(ABC)\right)=a\sqrt{3}$.}
		{\begin{tikzpicture}[scale=0.8,>=stealth, font=\footnotesize, line join=round, line cap=round]
				\coordinate (A) at (-2,0);
				\coordinate (B) at (0,-3);
				\coordinate (C) at (4,0);
				\coordinate (M) at ($(A)!0.5!(B)$);
				\coordinate (O) at ($(C)!2/3!(M)$);
				\coordinate (S) at ($(O)+(0,5)$);
				\draw(S)--(A) (S)--(B) (S)--(C) (B)--(C) (A)--(B);
				\draw[dashed,thin](A)--(C) (A)--(O)--(B) (C)--(O)--(S);
				\foreach \d/\g in {S/90,A/180,B/-90,C/0,O/-70}
				\draw[fill=black](\d)circle(1pt)node[shift={(\g:0.35)}]{$\d$};
				\tkzMarkRightAngles(S,O,C)
		\end{tikzpicture}}
	}
\end{ex}
\begin{ex}%[1H3K5-3][THPT Chu Văn An - Thái Nguyên - 2021]%Câu 12.
	\immini{Cho lăng trụ đứng $ABC.A’B’C’$ có tất cả các cạnh đều bằng $a$. Gọi $M$ là trung điểm của $CC’$ (tham khảo hình bên). Khoảng cách từ $M$ đến mặt phẳng $(A’BC)$ bằng
		\choice
		{$\dfrac{\sqrt{21}a}{7}$}
		{$\dfrac{\sqrt{2}a}{4}$}
		{\True $\dfrac{\sqrt{21}a}{14}$}
		{$\dfrac{\sqrt{2}a}{2}$}}
	{\begin{tikzpicture}[scale=1,>=stealth, font=\footnotesize, line join=round, line cap=round]
			\path
			(0,0) coordinate (A)
			(1.1,-1.5) coordinate (B)
			(3.5,0) coordinate (C);
			\coordinate (A') at ($(A)+(0,3.2)$);
			\coordinate (B') at ($(B)+(0,3.2)$);
			\coordinate (C') at ($(C)+(0,3.2)$);
			\coordinate (M) at ($(C)!0.5!(C')$);
			\draw (A')--(B')--(C')--(A')--(A)--(B)--(C)--(C') (B)--(B');
			\draw[dashed] (A)--(C)--(A')--(B);
			\foreach \d/\g in{A'/135,B'/0,C'/45,A/180,B/-90,C/0,M/0}
			\draw[fill=black](\d)circle(1pt)node[shift={(\g:0.35)}]{$\d$};
			\tkzMarkRightAngles(A',A,C)
			
	\end{tikzpicture}}
	\loigiai{
		\immini{
			Gọi $H, K$ lần lượt là hình chiếu của $A$ lên $BC$ và $A’H$. 
			Ta có $\mathrm{d}\left(M,(A’BC)\right)=\dfrac{1}{2}\mathrm{d}\left(C’,(A’BC)\right)=\dfrac{1}{2}\mathrm{d}\left(A,(A’BC)\right)=\dfrac{1}{2}AK$.\\
			Mà $AH=\dfrac{a\sqrt{3}}{2}$; $AA’=a$ nên $AK=\dfrac{AH\cdot AA’}{\sqrt{AH^2+AA’^2}}=\dfrac{a\sqrt{21}}{7}$.\\
			Vậy $\mathrm{d}\left(M;(A’BC)\right)=\dfrac{a\sqrt{21}}{14}$.}
		{\begin{tikzpicture}[scale=1,>=stealth, font=\footnotesize, line join=round, line cap=round]
				\path
				(0,0) coordinate (A)
				(1.1,-1.5) coordinate (B)
				(3.5,0) coordinate (C);
				\coordinate (A') at ($(A)+(0,3.2)$);
				\coordinate (B') at ($(B)+(0,3.2)$);
				\coordinate (C') at ($(C)+(0,3.2)$);
				\coordinate (M) at ($(C)!0.5!(C')$);
				\coordinate (H) at ($(C)!0.5!(B)$);
				\coordinate (K) at ($(A')!0.6!(H)$);
				\draw (A')--(B')--(C')--(A')--(A)--(B)--(C)--(C') (B)--(B');
				\draw[dashed] (A)--(C)--(A')--(B) (A)--(K) (A')--(H)--(A);
				\foreach \d/\g in{A'/135,B'/0,C'/45,A/180,B/-90,C/0,M/0,H/-90,K/0}
				\draw[fill=black](\d)circle(1pt)node[shift={(\g:0.35)}]{$\d$};
				\tkzMarkRightAngles(A',A,C A,H,B A,K,H)
		\end{tikzpicture}}
	}
\end{ex}
\begin{ex}%[1H3B5-3][THPT Quảng Xương 1-Thanh Hóa - 2021]%Câu 13.
	Cho hình chóp $S.ABCD$ có đáy $ABCD$ là hình vuông cạnh $a$, cạnh bên $SA$ vuông góc với đáy và $SA=a\sqrt{3}$. Khoảng cách từ $A$ đến mặt phẳng $(SBC)$ bằng
	\choice
	{$\dfrac{2a\sqrt{5}}{5}$}
	{$a\sqrt{3}$}
	{$\dfrac{a}{2}$}
	{\True $\dfrac{a\sqrt{3}}{2}$}
	\loigiai{
		\immini{
			Kẻ $AH\perp SB (*)$.\\
			Ta có $BC\perp AB$ (Do $ABCD$ là hình vuông).\\
			$BC\perp SA$ (Do $SA\perp(ABCD)$).\\
			Suy ra $BC\perp(SAB)$.\\
			Suy ra $BC\perp AH (**)$.\\
			Từ $(*),(**)$ suy ra $AH\perp(SBC)$. Suy ra $\mathrm{d}\left(A,(SBC)\right)=AH$.\\
			$\dfrac{1}{AH^2}=\dfrac{1}{AB^2}+\dfrac{1}{SA^2}=\dfrac{1}{a^2}+\dfrac{1}{3a^2}=\dfrac{4}{3a^2}$.\\
			Suy ra $AH=\dfrac{a\sqrt{3}}{2}$.}
		{\begin{tikzpicture}[scale=1,>=stealth, font=\footnotesize, line join=round, line cap=round]
				\coordinate (A) at (0,0);
				\coordinate (B) at (-2,-2);
				\coordinate (D) at (4,0);
				\coordinate (C) at ($(B)+(D)-(A)$);
				\coordinate (S) at ($(A)+(0,3.5)$);
				\coordinate (H) at ($(S)!0.4!(D)$);
				\draw(S)--(B) (S)--(C) (S)--(D) (B)--(C)--(D);
				\draw[dashed,thin](S)--(A) (A)--(B) (H)--(A)--(D);
				\foreach \d/\g in {S/90,A/-90,B/-90,C/-90,D/0,H/45}
				\draw[fill=black](\d)circle(1pt)node[shift={(\g:0.35)}]{$\d$};
				\tkzMarkRightAngles(A,H,D)
				\tkzMarkRightAngles(S,A,D)
				
		\end{tikzpicture}}
	}
\end{ex}
\begin{ex}%[1H3B5-3][THPT Trần Phú - Đà Nẵng - 2021]%Câu 14.
	\immini{Cho hình lập phương $ABCD.A’B’C’D’$ cạnh $a\sqrt{3}$, $I$ là trung điểm $CD’$ (tham khảo hình vẽ). Khoảng cách từ $I$ đến mặt phẳng $(BDD’B’)$ bằng
		\choice
		{$\dfrac{a\sqrt{2}}{4}$}
		{$\dfrac{a}{4}$}
		{\True $\dfrac{a\sqrt{6}}{4}$}
		{$\dfrac{a\sqrt{3}}{4}$}}
	{\begin{tikzpicture}[scale=1,>=stealth, font=\footnotesize, line join=round, line cap=round]
			\path
			(0,0) coordinate (A)
			(-1,-1.1) coordinate (B)
			(2,-1.1) coordinate (C);
			\coordinate (D) at ($(A)+(C)-(B)$);
			\coordinate (A') at ($(A)+(0,3)$);
			\coordinate (I) at ($(C)!0.5!(D')$);
			\tkzDefPointsBy[translation=from A to A'](B,C,D){B'}{C'}{D'}
			\draw[dashed] (A')--(A)--(D) (A)--(B)--(D) ;
			\draw (B')--(C')--(D')--(A')--(B')--(B)--(C)--(C') (C)--(D)--(D')--(C) (B')--(D');
			\foreach \d/\g in{A'/135,B'/180,C'/0,D'/45,A/135,B/-135,C/-45,D/0,I/0}
			\draw[fill=black](\d)circle(1pt)node[shift={(\g:0.35)}]{$\d$};
			\tkzMarkRightAngles(S,A,D)
			
	\end{tikzpicture}}
	\loigiai{
		\immini{
			Do $CI\cap(BDD’B’)=D’$ nên ta có $$\dfrac{\mathrm{d}\left(I,(BDD’B’)\right)}{\mathrm{d}\left(C,(BDD’B’)\right)}=\dfrac{ID’}{CD’}=\dfrac{1}{2}.$$
			Gọi $\{M\}=BD\cap AC$. Khi đó $CM=\dfrac{AC}{2}=\dfrac{\sqrt{6}a}{2}$.\\
			Vậy $\mathrm{d}\left(I,(BDD’B’)\right)=\dfrac{1}{2}\cdot\mathrm{d}\left(C,(BDD’B’)\right)=\dfrac{CM}{2}=\dfrac{\sqrt{6}a}{4}$.}
		{\begin{tikzpicture}[scale=1,>=stealth, font=\footnotesize, line join=round, line cap=round]
				\path
				(0,0) coordinate (A)
				(-1,-1.1) coordinate (B)
				(2,-1.1) coordinate (C);
				\coordinate (D) at ($(A)+(C)-(B)$);
				\coordinate (A') at ($(A)+(0,3)$);
				\coordinate (I) at ($(C)!0.5!(D')$);
				\coordinate (M) at ($(A)!0.5!(C)$);
				\tkzDefPointsBy[translation=from A to A'](B,C,D){B'}{C'}{D'}
				\draw[dashed] (A')--(A)--(D) (C)--(A)--(B)--(D);
				\draw (B')--(C')--(D')--(A')--(B')--(B)--(C)--(C') (C)--(D)--(D')--(C) (B')--(D') ;
				\foreach \d/\g in{A'/135,B'/180,C'/0,D'/45,A/135,B/-135,C/-45,D/0,I/0,M/-90}
				\draw[fill=black](\d)circle(1pt)node[shift={(\g:0.35)}]{$\d$};
				\tkzMarkRightAngles(A',A,D)
		\end{tikzpicture}}
	}
\end{ex}
\begin{ex}%[1H3B5-3][Chuyên Tuyên Quang - 2021]%Câu 15.
	Cho hình lăng trụ đều $ABC.A’B’C’$ có tất cả các cạnh bằng $2022$. Khoảng cách từ điểm $A$ đến mặt phẳng $(BCC’B’)$ bằng
	\choice
	{\True $1011\sqrt{3}$}
	{$2022\sqrt{3}$}
	{$2022\sqrt{2}$}
	{$1011\sqrt{2}$}
	\loigiai{
		\immini{
			Gọi $H$ là trung điểm của $BC$.\\
			Ta có $\heva{&AH\perp BC\\&AH\perp BB’}\Rightarrow AH\perp(BB’C’C)$ \\
			$ \Rightarrow\mathrm{d}\left(A,(BCC’B’)\right)=AH=1011\sqrt{3} $.}
		{\begin{tikzpicture}[scale=1,>=stealth, font=\footnotesize, line join=round, line cap=round]
				\path
				(0,0) coordinate (A)
				(1.5,-1.5) coordinate (B)
				(3.5,0) coordinate (C);
				\coordinate (A') at ($(A)+(0,3.2)$);
				\coordinate (B') at ($(B)+(0,3.2)$);
				\coordinate (C') at ($(C)+(0,3.2)$);
				\coordinate (H) at ($(C)!0.5!(B)$);
				\draw (A')--(B')--(C')--(A')--(A)--(B)--(C)--(C') (B)--(B');
				\draw[dashed] (H)--(A)--(C);
				\foreach \d/\g in{A'/135,B'/0,C'/45,A/180,B/-90,C/0,H/-90}
				\draw[fill=black](\d)circle(1pt)node[shift={(\g:0.35)}]{$\d$};
				\tkzMarkRightAngles(A',A,C)
				
		\end{tikzpicture}}
	}
\end{ex}
\begin{ex}%[1H3B5-3][Cụm Ninh Bình – 2021]%Câu 16.
	Cho hình lăng trụ đứng $ABC.A’B’C’$ có đáy $ABC$ là tam giác vuông tại $B$, $AB=a$, $AA’=2a$. Tính theo $a$ khoảng cách từ điểm $A$ đến mặt phẳng $(A’BC)$. 
	\choice
	{$\dfrac{2a\sqrt{3}}{5}$}
	{$\dfrac{a\sqrt{5}}{3}$}
	{$\dfrac{a\sqrt{3}}{3}$}
	{$\dfrac{2a\sqrt{5}}{5}$}
	\loigiai{
		\immini{
			Vẽ $AH\perp A’B\Rightarrow AH\perp(A’BC)\Rightarrow\mathrm{d}\left(A,(A’BC)\right)=AH$.\\
			Ta có: $AH=\dfrac{AA’\cdot AB}{\sqrt{(AA’)^2+(AB)^2}}=\dfrac{2a\cdot a}{\sqrt{4a^2+a^2}}=\dfrac{2a\sqrt{5}}{5}$.}
		{\begin{tikzpicture}[scale=1,>=stealth, font=\footnotesize, line join=round, line cap=round]
				\path
				(0,0) coordinate (A)
				(1.5,-1.5) coordinate (B)
				(3.5,0) coordinate (C);
				\coordinate (A') at ($(A)+(0,3.2)$);
				\coordinate (B') at ($(B)+(0,3.2)$);
				\coordinate (C') at ($(C)+(0,3.2)$);
				\coordinate (H) at ($(A')!0.4!(B)$);
				\draw (A')--(B')--(C')--(A')--(A)--(B)--(C)--(C') (B)--(B') (A')--(B) (A)--(H);
				\draw[dashed] (A)--(C);
				\foreach \d/\g in{A'/135,B'/0,C'/45,A/180,B/-90,C/0,H/0}
				\draw[fill=black](\d)circle(1pt)node[shift={(\g:0.35)}]{$\d$};
				\tkzMarkRightAngles(A,H,B)
				\tkzMarkRightAngles(A',A,C)
		\end{tikzpicture}}
	}
\end{ex}
\begin{ex}%[1H3B5-3][Chuyên ĐHSP - 2021]%Câu 17.
	Cho hình chóp $S.ABCD$ có đáy $ABCD$ là hình chữ nhật, $SA\perp(ABCD)$. Biết $SA=a$, $AB=a$ và $AD=2a$. Gọi $G$ là trọng tâm tam giác $SAD$. Khoảng cách từ điểm $G$ đến mặt phẳng $(SBD)$ bằng
	\choice
	{$\dfrac{a}{3}$}
	{$\dfrac{2a}{9}$}
	{$\dfrac{a}{6}$}
	{\True $\dfrac{2a}{3}$}
	\loigiai{
		\immini{
			Gọi $M$ là trung điểm $SD$ $$\Rightarrow\mathrm{d}\left(G;(SBD)\right)=\dfrac{GM}{AM}\mathrm{d}\left(A;(SBD)\right)=\dfrac{1}{3}\mathrm{d}\left(A;(SBD)\right).$$
			Mà $SA$; $AB$; $AD$ đôi một vuông góc\\
			$ \Rightarrow\dfrac{1}{\left[\mathrm{d}\left(A;(SBD)\right)\right]^2}=\dfrac{1}{SA^2}+\dfrac{1}{AB^2}+\dfrac{1}{AD^2}$\\
			$\Rightarrow\mathrm{d}\left(A;(SBD)\right)=\dfrac{2a}{3} $.\\
			Vậy khoảng cách từ điểm $G$ đến mặt phẳng $(SBD)$ là $\mathrm{d}\left(G;(SBD)\right)=\dfrac{1}{3}\mathrm{d}\left(A;(SBD)\right)=\dfrac{2a}{9}$.}
		{\begin{tikzpicture}[scale=0.8,>=stealth, font=\footnotesize, line join=round, line cap=round]
				\coordinate (A) at (0,0);
				\coordinate (B) at (-2,-2);
				\coordinate (D) at (4,0);
				\coordinate (C) at ($(B)+(D)-(A)$);
				\coordinate (S) at ($(A)+(0,3.5)$);
				\coordinate (M) at ($(S)!0.5!(D)$);
				\coordinate (G) at ($(A)!0.66!(M)$);
				\coordinate (N) at ($(S)!1.5!(G)$);
				\draw(S)--(B) (S)--(C) (S)--(D) (B)--(C)--(D);
				\draw[dashed,thin](S)--(A) (A)--(B) (M)--(A)--(D)--(B) (S)--(N);
				\foreach \d/\g in {S/90,A/-90,B/-90,C/-90,D/0,M/0,G/0}
				\draw[fill=black](\d)circle(1pt)node[shift={(\g:0.35)}]{$\d$};
				\tkzMarkRightAngles(S,A,D)
				
		\end{tikzpicture}}
	}
\end{ex}
\begin{ex}%[1H3B5-3][Sở Hòa Bình - 2021]%Câu 18.
	\immini{Cho hình hộp chữ nhật $ABCD.A’B’C’D’$ có \\$AB=a,AD=2a$ (tham khảo hình vẽ bên). Khoảng cách từ điểm $A$ đến mặt phẳng $(BDD’B’)$ bằng
		\choice
		{$\dfrac{a\sqrt{5}}{2}$}
		{$a\sqrt{5}$}
		{$\dfrac{a\sqrt{5}}{5}$}
		{\True $\dfrac{2a\sqrt{5}}{5}$}}
	{\begin{tikzpicture}[scale=1,>=stealth, font=\footnotesize, line join=round, line cap=round]
			\path
			(0,0) coordinate (A)
			(-1,-1.1) coordinate (B)
			(3,-1.1) coordinate (C);
			\coordinate (D) at ($(A)+(C)-(B)$);
			\coordinate (A') at ($(A)+(0,3)$);
			\tkzDefPointsBy[translation=from A to A'](B,C,D){B'}{C'}{D'}
			\draw[dashed] (A')--(A)--(D) (A)--(B);
			\draw (B')--(C')--(D')--(A')--(B')--(B)--(C)--(C') (C)--(D)--(D');
			\foreach \d/\g in{A'/135,B'/180,C'/0,D'/45,A/135,B/-135,C/-45,D/0}
			\draw[fill=black](\d)circle(1pt)node[shift={(\g:0.35)}]{$\d$};
			\tkzMarkRightAngles(A',A,D)
	\end{tikzpicture}}
	\loigiai{
		\immini{
			Nhận thấy $(BDD’B’)\perp(ABCD)$.\\
			Trong mặt phẳng $(ABCD)$ kẻ $AH\perp BD(H\in BD)\Rightarrow AH\perp(BDD’B’)$ \\
			$ \Rightarrow\mathrm{d}\left(A,(BDD’B’)\right)=AH $.\\
			$\dfrac{1}{AH^2}=\dfrac{1}{AB^2}+\dfrac{1}{AD^2}=\dfrac{1}{a^2}+\dfrac{1}{4a^2}=\dfrac{5}{4a^2}\Rightarrow AH=\dfrac{2a\sqrt{5}}{5}$ \\
			$ \Rightarrow\mathrm{d}\left(A,(BDD’B’)\right)=\dfrac{2a\sqrt{5}}{5} $.}
		{\begin{tikzpicture}[scale=1,>=stealth, font=\footnotesize, line join=round, line cap=round]
				\path
				(0,0) coordinate (A)
				(-1,-1.3) coordinate (B)
				(3,-1.1) coordinate (C);
				\coordinate (D) at ($(A)+(C)-(B)$);
				\coordinate (A') at ($(A)+(0,3)$);
				\coordinate (H) at ($(B)!0.3!(D)$);
				\tkzDefPointsBy[translation=from A to A'](B,C,D){B'}{C'}{D'}
				\draw[dashed] (A')--(A)--(D) (H)--(A)--(B)--(D);
				\draw (B')--(C')--(D')--(A')--(B')--(B)--(C)--(C') (C)--(D)--(D')--(B');
				\foreach \d/\g in{A'/135,B'/180,C'/0,D'/45,A/135,B/-135,C/-45,D/0,H/-45}
				\draw[fill=black](\d)circle(1pt)node[shift={(\g:0.35)}]{$\d$};
				\tkzMarkRightAngles(A,H,B)
				\tkzMarkRightAngles(A',A,D)
				
		\end{tikzpicture}}
	}
\end{ex}
\begin{ex}%[1H3B5-3][Sở Nam Định - 2021]%Câu 19.
	\immini{Cho hình chóp $S.ABC$ có đáy là tam giác đều có cạnh bằng 3, mặt bên $(SAB)$ là tam giác đều và nằm trong mặt phẳng vuông góc với đáy (tham khảo hình vẽ bên). Khoảng cách từ đỉnh $S$ đến mặt phẳng $(ABC)$ bằng
		\choice
		{$\dfrac{\sqrt{3}}{2}$}
		{\True $\dfrac{3\sqrt{3}}{2}$}
		{$3$}
		{$\dfrac{3}{2}$}}
	{\begin{tikzpicture}[scale=1,>=stealth, font=\footnotesize, line join=round, line cap=round]
			\path
			(0,0) coordinate (A)
			(4,0) coordinate (B)
			(1.2,-1.2) coordinate (C)
			($(A)!0.5!(B)$) coordinate (H)
			($(H)+(0,4)$) coordinate (S)
			;
			\foreach \d/\g in{S/90,A/180,B/0,C/-90}
			\draw[fill=black](\d)circle(1pt)node[shift={(\g:0.35)}]{$\d$};
			\draw (S)--(A)--(C)--(B)--(S)--(C);
			\draw[dashed] (A)--(B);
		\end{tikzpicture}	
	}
	\loigiai{
		\immini{
			Gọi $H$ là trung điểm của $AB\Rightarrow SH\perp AB$ và $SH=\dfrac{3\sqrt{3}}{2}$ (do $SAB$ là tam giác đều có cạnh bằng 3).\\
			Ta có $\heva{&(SAB)\perp (ABC)\\&(SAB)\cap (ABC)=AB\\&SH\perp AB}\Rightarrow SH\perp (ABC)$.\\
			Khoảng cách từ đỉnh $S$ đến mặt phẳng $(ABC)$ bằng $SH=\dfrac{3\sqrt{3}}{2}$.}
		{\begin{tikzpicture}[scale=1,>=stealth, font=\footnotesize, line join=round, line cap=round]
				\path
				(0,0) coordinate (A)
				(4,0) coordinate (B)
				(1.2,-1.2) coordinate (C)
				($(A)!0.5!(B)$) coordinate (H)
				($(H)+(0,4)$) coordinate (S)
				;
				\foreach \d/\g in{S/90,A/180,B/0,C/-90,H/-90}
				\draw[fill=black](\d)circle(1pt)node[shift={(\g:0.35)}]{$\d$};
				\draw (S)--(A)--(C)--(B)--(S)--(C);
				\draw[dashed] (A)--(B) (S)--(H);
				\tkzMarkRightAngles(S,H,B)
				
		\end{tikzpicture}}
	}
\end{ex}
\begin{ex}%[1H3B5-3][Chuyên Vinh - 2021]%Câu 20.
	Cho hình chóp $S.ABCD$ có đáy $ABCD$ là hình chữ nhật $AB=a\sqrt{3}$, $BC=a$, các cạnh bên của hình chóp cùng bằng $a\sqrt{5}$. Gọi $M$ là trung điểm của $SC$. Tính khoảng cách từ $M$ đến mặt phẳng $(ABCD)$: 
	\choice
	{$2a$}
	{$a\sqrt{2}$}
	{$a\sqrt{3}$}
	{\True $a$}
	\loigiai{
		\immini{
			Gọi $O$ là giao của hai đường chéo.\\
			Dễ thấy cạnh bên của hình chóp bằng nhau nên chân đường cao của hình chóp chính là tâm của đáy.\\
			Ta có $AC=\sqrt{AB^2+BC^2}=\sqrt{3a^2+a^2}=2a$ \\
			$\Rightarrow AO=\dfrac{1}{2}AC=a$.\\
			Khi đó ta có $SO=\sqrt{SA^2-AO^2}=\sqrt{5a^2-a^2}=2a$.\\
			Gọi $H$ là chân đường cao hạ từ $M$ xuống $AC$ \\
			$\Rightarrow\mathrm{d}\left(M;(ABCD)\right)=MH$.\\
			Mặt khác $M$ là trung điểm của $SC$ nên $MH$ là đường trung bình của $\triangle SOC\Rightarrow MH=\dfrac{1}{2}SO=a$.\\
			Vậy $\mathrm{d}\left(M;(ABCD)\right)=a$.}
		{\begin{tikzpicture}[scale=1,>=stealth, font=\footnotesize, line join=round, line cap=round]
				\path
				(0,0) coordinate (A)
				(-2,-1.8) coordinate (B)
				(2,-1.8) coordinate (C);
				\coordinate (O) at ($(A)!0.5!(C)$);
				\coordinate (D) at ($(A)+(C)-(B)$);
				\coordinate (S) at ($(O)+(0,4)$);
				\coordinate (M) at ($(S)!0.5!(C)$);
				\coordinate (H) at ($(O)!0.5!(C)$);
				\draw (S)--(B)--(C)--(S)--(D)--(C);
				\draw[dashed] (S)--(A)--(C) (B)--(A)--(D)--(B) (S)--(O) (M)--(H);
				\foreach \d/\g in{S/90,A/180,B/-135,C/-45,D/45,O/-90,M/45,H/-90}
				\draw[fill=black](\d)circle(1pt)node[shift={(\g:0.35)}]{$\d$};
				\tkzMarkRightAngles(S,O,C M,H,C)
				
		\end{tikzpicture}}
	}
\end{ex}
\begin{ex}%[1H3B5-3][Mã 102 - 2020 Lần 1]%Câu 21.
	\immini{Cho lăng trụ đứng $ABC.A’B’C’$ có đáy $ABC$ là tam giác đều cạnh $a$ và $AA’=2a$. Gọi $M$ là trung điểm của $CC’$ (tham khảo hình bên). Khoảng cách từ $M$ đến mặt phẳng $(A’BC)$ bằng
		\choice
		{$\dfrac{a\sqrt{5}}{5}$}
		{$\dfrac{2\sqrt{5}a}{5}$}
		{$\dfrac{2\sqrt{57}a}{19}$}
		{\True $\dfrac{\sqrt{57}a}{19}$}}
	{\begin{tikzpicture}[scale=1,>=stealth, font=\footnotesize, line join=round, line cap=round]
			\path
			(0,0) coordinate (A)
			(1.1,-1.5) coordinate (B)
			(3.5,0) coordinate (C);
			\coordinate (A') at ($(A)+(0,3.2)$);
			\coordinate (B') at ($(B)+(0,3.2)$);
			\coordinate (C') at ($(C)+(0,3.2)$);
			\coordinate (M) at ($(C)!0.5!(C')$);
			\draw (A')--(B')--(C')--(A')--(A)--(B)--(C)--(C') (B)--(B');
			\draw[dashed] (A)--(C)--(A')--(B);
			\foreach \d/\g in{A'/135,B'/0,C'/45,A/180,B/-90,C/0,M/0}
			\draw[fill=black](\d)circle(1pt)node[shift={(\g:0.35)}]{$\d$};
			\tkzMarkRightAngles(A',A,C)
			
	\end{tikzpicture}}
	\loigiai{
		\immini{Gọi $H, K$ lần lượt là hình chiếu của $A$ lên $BC$ và $A’H$. 
			Ta có $\mathrm{d}\left(M,(A’BC)\right)=\dfrac{1}{2}\mathrm{d}\left(C’,(A’BC)\right)=\dfrac{1}{2}\mathrm{d}\left(A,(A’BC)\right)=\dfrac{1}{2}AK$.\\
			Mà $AH=\dfrac{a\sqrt{3}}{2}$; $AA’=2a$ nên $AK=\dfrac{AH\cdot AA’}{\sqrt{AH^2+AA’^2}}=\dfrac{2a\sqrt{57}}{19}$.\\
			Vậy $\mathrm{d}\left(M;(A’BC)\right)=\dfrac{a\sqrt{57}}{19}$.}
		{\begin{tikzpicture}[scale=1,>=stealth, font=\footnotesize, line join=round, line cap=round]
				\path
				(0,0) coordinate (A)
				(1.1,-1.5) coordinate (B)
				(3.5,0) coordinate (C);
				\coordinate (A') at ($(A)+(0,3.2)$);
				\coordinate (B') at ($(B)+(0,3.2)$);
				\coordinate (C') at ($(C)+(0,3.2)$);
				\coordinate (M) at ($(C)!0.5!(C')$);
				\coordinate (H) at ($(C)!0.5!(B)$);
				\coordinate (K) at ($(A')!0.6!(H)$);
				\draw (A')--(B')--(C')--(A')--(A)--(B)--(C)--(C') (B)--(B');
				\draw[dashed] (A)--(C)--(A')--(B) (A)--(K) (A')--(H)--(A);
				\foreach \d/\g in{A'/135,B'/0,C'/45,A/180,B/-90,C/0,M/0,H/-90,K/0}
				\draw[fill=black](\d)circle(1pt)node[shift={(\g:0.35)}]{$\d$};
				\tkzMarkRightAngles(A',A,C A,H,B A,K,H)
				
		\end{tikzpicture}}
	}
\end{ex}
\begin{ex}%[1H3B5-3][Mã 103 - 2020 Lần 1]%Câu 22.
	Cho hình lăng trụ đứng $ABC.A’B’C’$ có đáy $ABC$ là tam giác đều cạnh $a$ và $A’A=2a$. Gọi $M$ là trung điểm của $A’A$ (tham khảo hình vẽ bên). Khoảng cách từ $M$ đến mặt phẳng $(AB’C)$ bằng
	\choice
	{\True $\dfrac{\sqrt{57}a}{19}$}
	{$\dfrac{\sqrt{5}a}{5}$}
	{$\dfrac{2\sqrt{5}a}{5}$}
	{$\dfrac{2\sqrt{57}a}{19}$}
	\loigiai{
		\immini{
			Gọi $I=BM\cap AB’$ và $K$ là trung điểm $AC$.\\
			Ta có $\dfrac{\mathrm{d}\left(M,(AB’C)\right)}{\mathrm{d}\left(B,(AB’C)\right)}=\dfrac{MI}{BI}=\dfrac{MA}{BB’}=\dfrac{1}{2}$ \\
			$\Rightarrow\mathrm{d}\left(M,(AB’C)\right)=\dfrac{1}{2}\mathrm{d}\left(B,(AB’C)\right)=\dfrac{BH}{2}$.\\
			Xét tam giác $BB’K$ có $$\dfrac{1}{BH^2}=\dfrac{1}{B’B^2}+\dfrac{1}{BK^2}=\dfrac{1}{(2a)^2}+\dfrac{1}{\left(\dfrac{a\sqrt{3}}{2}\right)^2}\Rightarrow BH=\dfrac{2\sqrt{57}a}{19}.$$
			Vậy $\mathrm{d}\left(M,(AB’C)\right)=\dfrac{BH}{2}=\dfrac{\sqrt{57}a}{19}$.}
		{\begin{tikzpicture}[scale=1,>=stealth, font=\footnotesize, line join=round, line cap=round]
				\path
				(0,0) coordinate (A)
				(1.1,-1.5) coordinate (C)
				(3.5,0) coordinate (B);
				\coordinate (A') at ($(A)+(0,3.2)$);
				\coordinate (B') at ($(B)+(0,3.2)$);
				\coordinate (C') at ($(C)+(0,3.2)$);
				\coordinate (M) at ($(A)!0.5!(A')$);
				\coordinate (K) at ($(A)!0.5!(C)$);
				\coordinate (H) at ($(B')!0.5!(K)$);
				\tkzInterLL(A,B')(B,M)\tkzGetPoint{I}
				\draw (A')--(A)--(C)--(C')--(A')--(B')--(C')--(C)--(B)--(B')--(C);
				\draw[dashed] (A)--(B) (A)--(B') (B')--(K)--(B)--(H);
				\foreach \d/\g in{A'/135,B'/0,C'/90,A/180,B/-90,C/0,M/180,H/-90,K/180,I/90}
				\draw[fill=black](\d)circle(1pt)node[shift={(\g:0.35)}]{$\d$};
				\tkzMarkRightAngles(B,K,C B,H,B' A',A,B)
		\end{tikzpicture}}
	}
\end{ex}
\begin{ex}%[1H3B5-3][Mã 104 - 2020 Lần 1]%Câu 23.
	\immini{Cho hình lăng trụ đứng $ABC.A’B’C’$ có tất cả các cạnh bằng $a$. Gọi $M$ là trung điểm của $AA’$ (tham khảo hình vẽ). 
		Khoảng cách từ $M$ đến mặt phẳng $(AB’C)$ bằng
		\choice
		{$\dfrac{a\sqrt{2}}{4}$}
		{$\dfrac{a\sqrt{21}}{7}$}
		{$\dfrac{a\sqrt{2}}{2}$}
		{\True $\dfrac{a\sqrt{21}}{14}$}}
	{\begin{tikzpicture}[scale=1,>=stealth, font=\footnotesize, line join=round, line cap=round]
			\path
			(0,0) coordinate (A)
			(1.1,-1.5) coordinate (C)
			(3.5,0) coordinate (B);
			\coordinate (A') at ($(A)+(0,3.2)$);
			\coordinate (B') at ($(B)+(0,3.2)$);
			\coordinate (C') at ($(C)+(0,3.2)$);
			\coordinate (M) at ($(A)!0.5!(A')$);
			\draw (A')--(A)--(C)--(C')--(A')--(B')--(C')--(C)--(B)--(B')--(C);
			\draw[dashed] (A)--(B) (A)--(B');
			\foreach \d/\g in{A'/135,B'/0,C'/90,A/180,B/-90,C/0,M/180}
			\draw[fill=black](\d)circle(1pt)node[shift={(\g:0.35)}]{$\d$};
			\tkzMarkRightAngles(A',A,B)
	\end{tikzpicture}}
	\loigiai{
		\immini{
			Trong $(ABB’A’)$, gọi $E$ là giao điểm của $BM$ và $AB’$. Khi đó hai tam giác $EAM$ và $EB’B$ đồng dạng. \\
			Do đó $\dfrac{\mathrm{d}\left(M,(AB’C)\right)}{\mathrm{d}\left(B,(AB’C)\right)}=\dfrac{EM}{EB}=\dfrac{MA}{BB’}=\dfrac{1}{2}$ \\
			$\Rightarrow\mathrm{d}\left(M,(AB’C)\right)=\dfrac{1}{2}\cdot\mathrm{d}\left(B,(AB’C)\right).$ \\
			Từ $B$ kẻ $BN\perp AC$ thì $N$ là trung điểm của $AC$ và $BN=\dfrac{a\sqrt{3}}{2}$, $BB’=a$.\\
			Kẻ $BI\perp B’N$, ta có $$\mathrm{d}\left(B,(AB’C)\right)=BI=\dfrac{BB’\cdot BN}{\sqrt{BB’^2+BN^2}}=\dfrac{a\sqrt{21}}{7}.$$
			Vậy $\mathrm{d}\left(M,(AB’C)\right)=\dfrac{1}{2}\cdot\mathrm{d}\left(B,(AB’C)\right)=\dfrac{a\sqrt{21}}{14}$.}
		{\begin{tikzpicture}[scale=1,>=stealth, font=\footnotesize, line join=round, line cap=round]
				\path
				(0,0) coordinate (A)
				(1,-1.5) coordinate (C)
				(3.5,0) coordinate (B);
				\coordinate (A') at ($(A)+(0,3.2)$);
				\coordinate (B') at ($(B)+(0,3.2)$);
				\coordinate (C') at ($(C)+(0,3.2)$);
				\coordinate (M) at ($(A)!0.5!(A')$);
				\coordinate (N) at ($(A)!0.5!(C)$);
				\coordinate (I) at ($(B')!0.5!(K)$);
				\tkzInterLL(B,M)(A,B')\tkzGetPoint{E}
				\draw (A')--(A)--(C)--(C')--(A')--(B')--(C')--(C)--(B)--(B')--(C);
				\draw[dashed] (A)--(B) (A)--(B') (B')--(N)--(B)--(H) (B)--(M);
				\foreach \d/\g in{A'/135,B'/0,C'/90,A/180,B/-90,C/0,M/0,I/-90,N/180,E/-90}
				\draw[fill=black](\d)circle(1pt)node[shift={(\g:0.35)}]{$\d$};
				\tkzMarkRightAngles(A',A,B B,N,C B,I,B')
				
		\end{tikzpicture}}
	}
\end{ex}
\begin{ex}%[1H3B5-3][Mã 101 - 2020 Lần 1]%Câu 24.
	\immini{ Cho hình lăng trụ đứng $ABC.A’B’C’$ có tất cả các cạnh bằng $a$. Gọi $M$ là trung điểm của $CC’$ (tham khảo hình bên). Khoảng cách từ $M$ đến mặt phẳng $(A’BC)$ bằng
		\choice
		{\True $\dfrac{\sqrt{21}a}{14}$}
		{$\dfrac{\sqrt{2}a}{2}$}
		{$\dfrac{\sqrt{21}a}{7}$}
		{$\dfrac{\sqrt{2}a}{4}$}}
	{\begin{tikzpicture}[scale=1,>=stealth, font=\footnotesize, line join=round, line cap=round]
			\path
			(0,0) coordinate (A)
			(1.1,-1.5) coordinate (B)
			(3.5,0) coordinate (C);
			\coordinate (A') at ($(A)+(0,3.2)$);
			\coordinate (B') at ($(B)+(0,3.2)$);
			\coordinate (C') at ($(C)+(0,3.2)$);
			\coordinate (M) at ($(C)!0.5!(C')$);
			\draw (A')--(B')--(C')--(A')--(A)--(B)--(C)--(C') (B)--(B');
			\draw[dashed] (A)--(C)--(A')--(B);
			\foreach \d/\g in{A'/135,B'/0,C'/45,A/180,B/-90,C/0,M/0}
			\draw[fill=black](\d)circle(1pt)node[shift={(\g:0.35)}]{$\d$};
			\tkzMarkRightAngles(A',A,C)
			
	\end{tikzpicture}}
	\loigiai{
		\immini{
			$C’M\cap(A’BC)=C$, suy ra $\dfrac{\mathrm{d}\left(M,(A’BC)\right)}{\mathrm{d}\left(C’,(A’BC)\right)}=\dfrac{C’M}{C’C}=\dfrac{1}{2}$.\\
			Ta có $$V_{C’\cdot A’BC}=\dfrac{1}{3}V_{ABC.A’B’C’}=\dfrac{1}{3}\cdot C’C\cdot S_{\triangle ABC}=\dfrac{1}{3}\cdot a\cdot\dfrac{a^2\sqrt{3}}{4}=\dfrac{a^3\sqrt{3}}{12}.$$
			Lại có $A’B=a\sqrt{2}$, $CB=a$, $A’C=a\sqrt{2}\Rightarrow S_{A’BC}=\dfrac{a^2\sqrt{7}}{4}$.\\
			Suy ra $\mathrm{d}\left(C’,(A’BC)\right)=\dfrac{3V_{C’\cdot A’BC}}{S_{\triangle A’BC}}=\dfrac{3\cdot\dfrac{a^3\sqrt{3}}{12}}{\dfrac{a^2\sqrt{7}}{4}}=\dfrac{a\sqrt{21}}{7}$.\\
			Vậy $\mathrm{d}\left(M,(A’BC)\right)=\dfrac{1}{2}\mathrm{d}\left(C’,(A’BC)\right)=\dfrac{1}{2}\cdot\dfrac{a\sqrt{21}}{7}=\dfrac{a\sqrt{21}}{14}$.}
		{\begin{tikzpicture}[scale=1,>=stealth, font=\footnotesize, line join=round, line cap=round]
				\path
				(0,0) coordinate (A)
				(1.1,-1.5) coordinate (B)
				(3.5,0) coordinate (C);
				\coordinate (A') at ($(A)+(0,3.2)$);
				\coordinate (B') at ($(B)+(0,3.2)$);
				\coordinate (C') at ($(C)+(0,3.2)$);
				\coordinate (M) at ($(C)!0.5!(C')$);
				\draw (A')--(B')--(C')--(A')--(A)--(B)--(C)--(C') (B)--(B');
				\draw[dashed] (A)--(C)--(A')--(B);
				\foreach \d/\g in{A'/135,B'/0,C'/45,A/180,B/-90,C/0,M/0}
				\draw[fill=black](\d)circle(1pt)node[shift={(\g:0.35)}]{$\d$};
				\tkzMarkRightAngles(A',A,C)
				
		\end{tikzpicture}}
	}
\end{ex}
\begin{ex}%Câu 25.[Mã 101 2018][1H3B5-2]
	Cho hình chóp $S.ABC$ có đáy là tam giác vuông đỉnh $B$, $AB=a$, $SA$ vuông góc với mặt phẳng đáy và $SA=2a$. Khoảng cách từ $A$ đến mặt phẳng $(SBC)$ bằng
	\choice
	{\True $\dfrac{2\sqrt{5}a}{5}$}
	{$\dfrac{\sqrt{5}a}{3}$}
	{$\dfrac{2\sqrt{2}a}{3}$}
	{$\dfrac{\sqrt{5}a}{5}$}
	\loigiai{
		\immini{
			Ta có $\heva{&BC\perp AB\\&BC\perp SA}\Rightarrow BC\perp(SAB)$.\\
			Kẻ $AH\perp SB$. Khi đó $AH\perp BC\Rightarrow AH\perp(SBC)$ \\
			$ \Rightarrow AH $ là khoảng cách từ $A$ đến mặt phẳng $(SBC)$.\\
			Ta có $\dfrac{1}{AH^2}=\dfrac{1}{SA^2}+\dfrac{1}{AB^2}=\dfrac{1}{4a^2}+\dfrac{1}{a^2}=\dfrac{5}{4a^2}\\
			\Rightarrow AH^2=\dfrac{4a^2}{5}\Rightarrow AH=\dfrac{2\sqrt{5}a}{5}$.
		}
		{
			\begin{tikzpicture}[scale=0.7]
				\def\a{5}
				\def\h{5}
				\path 	
				(0:0) coordinate (A)
				++(0:\a) coordinate (C)
				++(-150:4*\a/5) coordinate (B)
				(A)+(90:4) coordinate (S)  ($(B)!0.5!(S)$) coordinate (H)
				;
				\pic[pic text= ,draw,thick,angle radius=2mm,angle eccentricity=1.5] {right angle = A--B--C};
				\pic[pic text= ,draw,thick,angle radius=2mm,angle eccentricity=1.5] {right angle = S--A--C};
				\pic[pic text= ,draw,thick,angle radius=2mm,angle eccentricity=1.5] {right angle = A--H--B};
				\draw (A)--(B)--(C) (S)--(A)--(H) (S)--(C) (S)--(B);
				\draw[dashed] (A)--(C) ;
				\foreach \x/\goc in {A/180,B/-90,C/-90,S/90,H/45}
				\fill (\x) circle (1.5pt)
				($(\x)+(\goc:3mm)$) node {$\x$};
			\end{tikzpicture}
		}
	}
\end{ex}
\begin{ex}%Câu 26.[Mã 102 2018][1H3B5-2]
	Cho hình chóp $S.ABC$ có đáy là tam giác vuông đỉnh $B$, $AB=a$, $SA$ vuông góc với mặt phẳng đáy và $SA=a$. Khoảng cách từ điểm $A$ đến mặt phẳng $(SBC)$ bằng
	\choice
	{$\dfrac{a\sqrt{6}}{3}$}
	{\True $\dfrac{a\sqrt{2}}{2}$}
	{$\dfrac{a}{2}$}
	{$a$}
	\loigiai{
		\immini{
			Kẻ $AH\perp SB$ trong mặt phẳng $(SBC)$.\\
			Ta có: $\heva{&BC\perp AB\\&BC\perp SA}\Rightarrow BC\perp(SAB)$ $\Rightarrow BC\perp AH$.\\
			Vì $\heva{&AH\perp BC\\&AH\perp SB}\Rightarrow AH\perp(SBC)$.\\
			Vậy $\mathrm{d}\left(A,(SBC)\right)=AH=\dfrac{1}{2}SB=\dfrac{a\sqrt{2}}{2}$.}
		{	\begin{tikzpicture}[scale=0.7]
				\def\a{5}
				\def\h{5}
				\path 	
				(0:0) coordinate (A)
				++(0:\a) coordinate (C)
				++(-150:4*\a/5) coordinate (B)
				(A)+(90:4) coordinate (S)  ($(B)!0.5!(S)$) coordinate (H)
				;
				\pic[pic text= ,draw,thick,angle radius=2mm,angle eccentricity=1.5] {right angle = A--B--C};
				\pic[pic text= ,draw,thick,angle radius=2mm,angle eccentricity=1.5] {right angle = A--H--B};
				\pic[pic text= ,draw,thick,angle radius=2mm,angle eccentricity=1.5] {right angle = S--A--C};
				
				\draw (A)--(B)--(C) (S)--(A)--(H) (S)--(C) (S)--(B);
				\draw[dashed] (A)--(C) ;
				\foreach \x/\goc in {A/180,B/-90,C/-90,S/90,H/45}
				\fill (\x) circle (1.5pt)
				($(\x)+(\goc:3mm)$) node {$\x$};
			\end{tikzpicture}
			
	}}
\end{ex}
\begin{ex}%Câu 27.[Mã 103 - 2019][1H3K5-2]
	\immini{Cho hình chóp $S.ABCD$ có đáy là hình vuông cạnh $a$, mặt bên $SAB$ là tam giác đều và nằm trong mặt phẳng vuông góc với mặt phẳng đáy (minh họa như hình vẽ bên). Khoảng cách từ $D$ đến mặt phẳng $(SAC)$ bằng
		\choice
		{$\dfrac{a\sqrt{2}}{2}$}
		{\True $\dfrac{a\sqrt{21}}{7}$}
		{$\dfrac{a\sqrt{21}}{14}$}
		{$\dfrac{a\sqrt{21}}{28}$}
	}
	{
		\begin{tikzpicture}[scale=0.8,>=stealth, font=\footnotesize, line join=round, line cap=round]
			
			\coordinate[label=above right:$A$] (A) at (0,0);
			\coordinate[label=above right:$D$] (D) at (3,0);
			\coordinate[label=below:$B$] (B) at (-1,-1.5);
			\coordinate (H) at ($(B)!0.5!(A)$);
			\coordinate[label=above:$S$] (S) at ($(H)+(0,3.5)$);
			\coordinate[label=right:$C$] (C) at ($(D)+(B)-(A)$);
			\draw (S)--(B)--(C)--(D)--(S)--(C);
			\draw[dashed] (C)--(A)--(S) (B)--(A)--(D);
			\pic[pic text= ,draw,thick,angle radius=2mm,angle eccentricity=1.5] {right angle = A--B--C};
		\end{tikzpicture}
		
	}
	\loigiai{
		\begin{center}
			\begin{tikzpicture}[scale=1,>=stealth, font=\footnotesize, line join=round, line cap=round]
				
				\coordinate[label=above right:$A$] (A) at (0,0);
				\coordinate[label=above right:$D$] (D) at (3,0);
				\coordinate[label=below:$B$] (B) at (-1,-1.5);
				\coordinate[label=right:$C$] (C) at ($(D)+(B)-(A)$);
				\coordinate[label=below right:$I$] (I) at ($(B)!0.5!(A)$);
				\coordinate[label=below:$O$] (O) at ($(C)!0.5!(A)$);
				\coordinate[label=above:$S$] (S) at ($(I)+(0,3.5)$);
				
				\coordinate[label=above:$G$] (G) at ($(I)!0.33!(D)$);
				\draw (S)--(B)--(C)--(D)--(S)--(C);
				\draw[dashed] (C)--(A)--(S)--(I)--(D)--(B)--(A)--(D);
				\foreach \diem in {A,B,C,D,S,I,G,O}	\fill (\diem)circle(1.5pt);
				\pic[pic text= ,draw,thick,angle radius=2mm,angle eccentricity=1.5] {right angle = A--B--C};
				\pic[pic text= ,draw,thick,angle radius=2mm,angle eccentricity=1.5] {right angle = S--I--D};
			\end{tikzpicture}
			\begin{tikzpicture}[scale=1,>=stealth, font=\footnotesize, line join=round, line cap=round]
				
				\coordinate[label=left:$A$] (A) at (0,0);
				\coordinate[label=below:$I$] (I) at (-1,-1);
				\coordinate[label=above right:$C$] (C) at (3,-1.5);
				\coordinate[label=above:$S$] (S) at ($(I)+(0,3.5)$);
				\coordinate[label=below:$O$] (O) at ($(C)!0.5!(A)$);
				\coordinate[label=right:$K$] (K) at ($(O)!0.5!(A)$);
				\coordinate[label=right:$H$] (H) at ($(S)!0.6!(K)$);
				\draw (S)--(I)--(C)--(S);
				\draw[dashed] (K)--(I)--(A)--(S)--(K) (A)--(C) (I)--(H);
				\foreach \diem in {S,A,I,C,O,K,H}	\fill (\diem)circle(1.5pt);
				
				\pic[pic text= ,draw,thick,angle radius=2mm,angle eccentricity=1.5] {right angle = S--I--K};
				\pic[pic text= ,draw,thick,angle radius=2mm,angle eccentricity=1.5] {right angle = I--H--K};
				\pic[pic text= ,draw,thick,angle radius=2mm,angle eccentricity=1.5] {right angle = I--K--O};
			\end{tikzpicture}
		\end{center}
		
		* Gọi $O=AC\cap BD$ và $G$ là trọng tâm tam giác $ABD$, $I$ là trung điểm của $AB$ ta có\\
		$SI\perp(ABCD)$ và $\dfrac{\mathrm{d}\left(D,(SAC)\right)}{\mathrm{d}\left(I,(SAC)\right)}=\dfrac{DG}{IG}=2\Rightarrow\mathrm{d}\left(D,(SAC)\right)=2\cdot\mathrm{d}\left(I,(SAC)\right)$.\\
		* Gọi $K$ là trung điểm của $AO$, $H$ là hình chiếu của $I$ lên $SK$ ta có $IK\perp AC; IH\perp(SAC)$ \\
		$ \Rightarrow\mathrm{d}\left(D,(SAC)\right)=2\cdot\mathrm{d}\left(I,(SAC)\right)=2\cdot IH $.\\
		* Xét tam giác $SIK$ vuông tại I ta có: $SI=\dfrac{a\sqrt{3}}{2}; IK=\dfrac{BO}{2}=\dfrac{a\sqrt{2}}{4}$ (do $IK$ là đường trung bình của tam giác $ABO$).\\
		$\dfrac{1}{IH^2}=\dfrac{1}{SI^2}+\dfrac{1}{IK^2}=\dfrac{4}{3a^2}+\dfrac{16}{2a^2}=\dfrac{28}{3a^2}\Rightarrow IH=\dfrac{a\sqrt{3}}{2\sqrt{7}}$ \\
		$ \Rightarrow\mathrm{d}\left(D,(SAC)\right)=2\cdot\mathrm{d}\left(I,(SAC)\right)=2\cdot IH=\dfrac{a\sqrt{21}}{7} $.}
\end{ex}
\begin{ex}%Câu 28.[Mã 101 -2019][1H3K5-2]
	\immini{
		Cho hình chóp $S.ABCD$ có đáy là hình vuông cạnh $a,$ mặt bên $SAB$ là tam giác đều và nằm trong mặt phẳng vuông góc với mặt phẳng đáy (minh họa như hình vẽ bên). Khoảng cách từ $A$ đến mặt phẳng $(SBD)$ bằng
		\choice
		{$\dfrac{\sqrt{21}a}{14}$}
		{\True $\dfrac{\sqrt{21}a}{7}$}
		{$\dfrac{\sqrt{2}a}{2}$}
		{$\dfrac{\sqrt{21}a}{28}$}
	}
	{\begin{tikzpicture}[scale=1,>=stealth, font=\footnotesize, line join=round, line cap=round]
			
			\coordinate[label=above right:$A$] (A) at (0,0);
			\coordinate[label=above right:$D$] (D) at (3,0);
			\coordinate[label=below:$B$] (B) at (-1,-1.5);
			\coordinate (H) at ($(B)!0.5!(A)$);
			\coordinate[label=above:$S$] (S) at ($(H)+(0,3.5)$);
			\coordinate[label=right:$C$] (C) at ($(D)+(B)-(A)$);
			\draw (S)--(B)--(C)--(D)--(S)--(C);
			\draw[dashed] (A)--(S) (D)--(B)--(A)--(D);
			\foreach \diem in {A,B,C,D,S}	\fill (\diem)circle(1.5pt);
			\pic[pic text= ,draw,thick,angle radius=2mm,angle eccentricity=1.5] {right angle = A--B--C};
		\end{tikzpicture}
		
		
	}
	\loigiai{
		\immini{
			Gọi $H$ là trung điểm của $AB$.\\
			Khi đó, $SH\perp(ABCD)$.\\
			Gọi $O$ là giao điểm của $AC$ và $BD$ suy ra $AC\perp BD$. \\
			Kẻ $HK\perp BD$ tại $K$ $\Leftrightarrow$ $K$ là trung điểm $BO$.\\
			Kẻ $HI\perp SH$ tại I. \\
			Khi đó: $\mathrm{d}\left(A,(SBD)\right)=2\mathrm{d}\left(H,(SBD)\right)=2HI$.\\
			Xét tam giác $SHK,$ có\\
			$SH=\dfrac{a\sqrt{3}}{2}, HK=\dfrac{1}{2}AO=\dfrac{a\sqrt{2}}{4}$.\\
			Khi đó: $\dfrac{1}{HI^2}=\dfrac{1}{SH^2}+\dfrac{1}{HK^2}=\dfrac{28}{3a^2}\Rightarrow HI=\dfrac{a\sqrt{21}}{14}$.\\
			Suy ra: $\mathrm{d}\left(A,(SBD)\right)=2HI=\dfrac{a\sqrt{21}}{7}$.}
		{
			\begin{tikzpicture}[scale=1,>=stealth, font=\footnotesize, line join=round, line cap=round]
				
				\coordinate[label=above right:$A$] (A) at (0,0);
				\coordinate[label=above right:$D$] (D) at (4,0);
				\coordinate[label=below:$B$] (B) at (-1.5,-2);
				\coordinate[label=left:$H$] (H) at ($(B)!0.5!(A)$);
				\coordinate[label=below:$O$] (O) at ($(C)!0.5!(A)$);
				\coordinate[label=above:$S$] (S) at ($(H)+(0,4)$);
				\coordinate[label=right:$C$] (C) at ($(D)+(B)-(A)$);
				\coordinate[label=below:$K$] (K) at ($(B)!0.5!(O)$);
				\coordinate[label=right:$I$] (I) at ($(S)!0.85!(K)$);
				\draw (S)--(B)--(C)--(D)--(S)--(C);
				\draw[dashed] (C)--(A)--(S)--(H)--(K)--(S) (I)--(H) (D)--(B)--(A)--(D);
				\foreach \diem in {A,B,C,D,S,H,O}	\fill (\diem)circle(1.5pt);
				\pic[pic text= ,draw,thick,angle radius=2mm,angle eccentricity=1.5] {right angle = A--B--C};
				\pic[pic text= ,draw,thick,angle radius=2mm,angle eccentricity=1.5] {right angle = S--H--A};
				\pic[pic text= ,draw,thick,angle radius=2mm,angle eccentricity=1.5] {right angle = H--I--K};
				\pic[pic text= ,draw,thick,angle radius=2mm,angle eccentricity=1.5] {right angle = H--K--B};
			\end{tikzpicture}
		}
	}
\end{ex}
\begin{ex}%Câu 29.[Đề Tham Khảo 2019][1H3K5-2]
	Cho hình chóp $S.ABCD$ có đáy là hình thoi cạnh $a$, $\widehat{BAD}=60^{\circ}$, $SA=a$ và $SA$ vuông góc với mặt phẳng đáy. Khoảng cách từ $B$ đến $(SCD)$ bằng
	\choice
	{ $\dfrac{\sqrt{21}a}{3}$}
	{$\dfrac{\sqrt{15}a}{3}$}
	{\True $\dfrac{\sqrt{21}a}{7}$}
	{$\dfrac{\sqrt{15}a}{7}$}
	\loigiai{
		\begin{center}
			\begin{tikzpicture}[=>stealth,line join=round,line cap=round,font=\footnotesize,scale=1]
				\def\a{5}
				\def\goc{210}
				\def\b{2.5}
				\def\h{4}
				\path
				(0,0)coordinate (A)++(0:\a)coordinate (D)++(\goc:\b)coordinate (C)++(180:\a)coordinate (B)
				(A)++(90:\h)coordinate (S)
				;
				\coordinate[label=above right:$M$] (M) at ($(C)!1.4!(D)$);
				\coordinate[label=above right:$H$] (H) at ($(S)!.3!(M)$);
				\draw (B)--(C)--(D)--(S)--cycle (D)--(M)--(S)--(C)
				;
				\draw[dashed] (B)--(A)--(D) (S)--(A)--(M) (A)--(H);
				\foreach \x/\g in{A/150,B/-90,C/-60,D/0,S/90}
				\fill[black](\x)circle(1pt) ($(\x)+(\g:3mm)$)node{$\x$};
				\pic[pic text= ,draw,thick,angle radius=2mm,angle eccentricity=1.5] {angle = B--A--D};
				\pic[pic text= ,draw,thick,angle radius=2mm,angle eccentricity=1.5] {right angle = S--A--D};
				\pic[pic text= ,draw,thick,angle radius=2mm,angle eccentricity=1.5] {right angle = A--M--D};
				\pic[pic text= ,draw,thick,angle radius=2mm,angle eccentricity=1.5] {right angle = A--H--D};
			\end{tikzpicture}
		\end{center}
		
		\textbf{CÁCH 1}.\\
		Ta có $AB\parallel CD\Rightarrow\mathrm{d}\left(B,(SCD)\right)=\mathrm{d}\left(A,(SCD)\right)$.\\
		Kẽ $MA\perp CD~(M\in CD)$, kẻ $AH\perp SM\Rightarrow AH\perp(SCD)\Rightarrow\mathrm{d}\left(A,(SCD)\right)=AH$.\\
		Ta có $SA=a$; $AM=\dfrac{2S_{ACD}}{CD}=\dfrac{S_{ABCD}}{CD}=\dfrac{a\sqrt{3}}{2}$.\\
		$\dfrac{1}{SH^2}=\dfrac{1}{SA^2}+\dfrac{1}{AM^2}\Rightarrow SM=\dfrac{\sqrt{21}}{7}a$.\\
		\textbf{CÁCH 2}. \\
		Ta có $AB\parallel CD\Rightarrow\mathrm{d}\left(B,(SCD)\right)=\mathrm{d}\left(A,(SCD)\right)=\dfrac{3V_{S.BCD}}{S_{SCD}}=\dfrac{3V_{S.ABCD}}{2S_{SCD}}=\dfrac{\sqrt{21}a}{7}$.\\
		($\triangle SCD$ có $SD=a\sqrt{2};SC=2a;CD=a$).}
\end{ex}
\begin{ex}%Câu 30.[Mã 102 - 2019][1H3K5-2]
	\immini{
		Cho hình chóp $S. ABCD$ có đáy là hình vuông cạnh $a$, mặt bên $SAB$ là tam giác đều và nằm trong mặt phẳng vuông góc với mặt phẳng đáy (minh họa như hình vẽ bên). Khoảng cách từ $C$ đến mặt phẳng $(SBD)$ bằng
		\choice
		{$\dfrac{\sqrt{21}a}{14}$}
		{$\dfrac{\sqrt{2}a}{2}$}
		{\True $\dfrac{\sqrt{21}a}{7}$}
		{$\dfrac{\sqrt{21}a}{28}$}
	}
	{
		\begin{tikzpicture}[scale=1,>=stealth, font=\footnotesize, line join=round, line cap=round]
			
			\coordinate[label=above right:$A$] (A) at (0,0);
			\coordinate[label=above right:$D$] (D) at (3,0);
			\coordinate[label=below:$B$] (B) at (-1,-1.5);
			\coordinate (H) at ($(B)!0.5!(A)$);
			\coordinate[label=above:$S$] (S) at ($(H)+(0,3.5)$);
			\coordinate[label=right:$C$] (C) at ($(D)+(B)-(A)$);
			\draw (S)--(B)--(C)--(D)--(S)--(C);
			\draw[dashed] (A)--(S) (D)--(B)--(A)--(D);
			\foreach \diem in {A,B,C,D,S}	\fill (\diem)circle(1.5pt);
			\pic[pic text= ,draw,thick,angle radius=2mm,angle eccentricity=1.5] {right angle = A--B--C};
		\end{tikzpicture}
	}
	\loigiai{
		\immini{
			Gọi $H$ là trung điểm của $AB$\\
			$\Rightarrow SH\perp AB\Rightarrow SH\perp (ABCD)$.\\
			Từ $H$ kẻ $HM\perp BD$, $M$ là trung điểm của $BI$ và $I$ là tâm của hình vuông.\\
			Ta có: $\heva{&BD\perp HM\\&BD\perp SH}\Rightarrow BD\perp (SHM)$.\\
			Từ $H$ kẻ $HK\perp SM\Rightarrow HK\perp BD$ (Vì $BD\perp (SHM)$)\\
			$ \Rightarrow HK\perp (SBD)\Rightarrow\mathrm{d}(H,(SBD))=HK $.\\
			Ta có: $HM=\dfrac{AI}{2}=\dfrac{AC}{4}=\dfrac{\sqrt{2}a}{4}$, $SH=\dfrac{\sqrt{3}a}{2}$.
		}
		{
			\begin{tikzpicture}[scale=0.8,>=stealth, font=\footnotesize, line join=round, line cap=round]
				
				\coordinate[label=above right:$A$] (A) at (0,0);
				\coordinate[label=above right:$D$] (D) at (4,0);
				\coordinate[label=below:$B$] (B) at (-1.5,-2);
				\coordinate[label=left:$H$] (H) at ($(B)!0.5!(A)$);
				\coordinate[label=below:$I$] (I) at ($(B)!0.5!(D)$);
				\coordinate[label=above:$S$] (S) at ($(H)+(0,4)$);
				\coordinate[label=right:$C$] (C) at ($(D)+(B)-(A)$);
				\coordinate[label=below:$M$] (M) at ($(B)!0.5!(I)$);
				\coordinate[label=right:$K$] (K) at ($(S)!0.85!(M)$);
				\draw (S)--(B)--(C)--(D)--(S)--(C);
				\draw[dashed] (C)--(A)--(S)--(H)--(M)--(S) (K)--(H) (D)--(B)--(A)--(D);
				\foreach \diem in {A,B,C,D,S,H,I,K,M}	\fill (\diem)circle(1.5pt);
				\pic[pic text= ,draw,thick,angle radius=2mm,angle eccentricity=1.5] {right angle = A--B--C};
				\pic[pic text= ,draw,thick,angle radius=2mm,angle eccentricity=1.5] {right angle = S--H--A};
				\pic[pic text= ,draw,thick,angle radius=2mm,angle eccentricity=1.5] {right angle = H--K--M};
			\end{tikzpicture}
		}
		$HK=\dfrac{HM\cdot HS}{\sqrt{HM^2+HS^2}}=\dfrac{\dfrac{\sqrt{2}a}{4}\cdot\dfrac{\sqrt{3}a}{2}}{\sqrt{\left(\dfrac{\sqrt{2}a}{4}\right)^2+\left(\dfrac{\sqrt{3}a}{2}\right)^2}}=\dfrac{\sqrt{21}a}{14}$.\\
		$\mathrm{d}(C,(SBD))=\mathrm{d}(A,(SBD))=2\mathrm{d}(H,(SBD))=2HK=2\cdot\dfrac{\sqrt{21}a}{14}=\dfrac{\sqrt{21}a}{7}$.\\
		Vậy $\mathrm{d}(C,(SBD)) =\dfrac{\sqrt{21}a}{7}$.}
\end{ex}
\begin{ex}%Câu 31.[Mã 103 2018][1H3B5-2]
	Cho hình chóp $S.ABCD$ có đáy là hình vuông cạnh $\sqrt{3}a$, $SA$ vuông góc với mặt phẳng đáy và $SA=a$. Khoảng cách từ A đến mặt phẳng $(SBC)$ bằng
	\choice
	{$\dfrac{\sqrt{6}a}{6}$}
	{$\dfrac{\sqrt{3}a}{3}$}
	{$\dfrac{\sqrt{5}a}{3}$}
	{\True $\dfrac{\sqrt{3}a}{2}$}
	\loigiai{
		\immini{
			Ta có: $\heva{&BC\perp AB\\&BC\perp SA}\Rightarrow BC\perp(SAB)$ \\
			$ \Rightarrow\heva{&(SAB)\perp(SBC)\\&(SAB)\cap(SBC)=SB.} $ \\
			Trong mặt phẳng $(SAB)$\\
			Kẻ $AH\perp SB\Rightarrow AH=\mathrm{d}\left(A,(SBC)\right)$.\\
			$\dfrac{1}{AH^2}=\dfrac{1}{SA^2}+\dfrac{1}{AB^2} =\dfrac{1}{a^2}+\dfrac{1}{3a^2} =\dfrac{4}{3a^2}$.\\
			$\Rightarrow\mathrm{d}\left(A,(SBC)\right)=AH=\dfrac{\sqrt{3}a}{2}$. }
		{
			\begin{tikzpicture}[scale=.8, font=\footnotesize, line join=round, line cap=round, >=stealth]
				\def\bc{4} % cạnh BC
				\def\ba{2} % cạnh BA
				\def\h{4} % đường cao
				\def\gocB{30} % góc B của đáy
				\coordinate[label=below left:$B$] (B) at (0,0);
				\coordinate[label=above left:$A$] (A) at (\gocB:\ba);
				\coordinate[label=below:$C$] (C) at (\bc,0);
				\coordinate[label=right:$D$] (D) at ($(C)-(B)+(A)$);
				\coordinate[label=above:$S$] (S) at ($(A)+(90:\h)$);
				\coordinate[label=above left:$H$] (H) at ($(B)!0.3!(S)$);
				\draw (B)--(C)--(D)--(S)--cycle (S)--(C);
				\draw[dashed] (H) -- (A)--(D) (S)--(A)--(B);
				\foreach \diem in {A,B,C,D,S,H}	\fill (\diem)circle(1.5pt);
				\pic[pic text= ,draw,thick,angle radius=2mm,angle eccentricity=1.5] {right angle = S--A--B};
				\pic[pic text= ,draw,thick,angle radius=2mm,angle eccentricity=1.5] {right angle = S--A--D};
				\pic[pic text= ,draw,thick,angle radius=2mm,angle eccentricity=1.5] {right angle = A--H--B};
			\end{tikzpicture}
		}
	}
\end{ex}
\begin{ex}%Câu 32.[Chuyên Vĩnh Phúc 2019][1H3B5-2]
	Cho tứ diện đều $ABCD$ có cạnh bằng $a$. Khoảng cách từ $A$ đến mặt phẳng $(BCD)$ bằng
	\choice
	{$\dfrac{a\sqrt{6}}{2}$}
	{\True $\dfrac{a\sqrt{6}}{3}$}
	{$\dfrac{3a}{2}$}
	{$2a$}
	\loigiai{
		\immini{	Gọi $E,F,G$ lần lượt là trung điểm của $BD,CD$ và trọng tâm tam giác $BCD$.\\
			Tam giác $BCD$ đều nên suy ra $CE=\dfrac{BC\sqrt{3}}{2}=\dfrac{a\sqrt{3}}{2}$,\\
			$CG=\dfrac{2}{3}CE=\dfrac{a\sqrt{3}}{3}$.\\
			Tam giác $ACG$ vuông tại $G$ nên\\ $AG^2=AC^2-CG^2=a^2-\dfrac{a^2}{3}=\dfrac{2a^2}{3}\Rightarrow AG=\dfrac{a\sqrt{6}}{3}$.\\
			Vậy $\mathrm{d}\left(A,(BCD)\right)=AG=\dfrac{a\sqrt{6}}{3}$.
		}
		{
			\begin{tikzpicture}[line join = round, line cap = round,>=stealth,scale=0.8]
				\coordinate[label=below left:$B$] (B) at (0,0);
				\coordinate[label=below left:$C$] (C) at (4,-2);
				\coordinate[label=below right:$D$] (D) at (6,0);
				\coordinate[label=above left:$E$] (E) at ($(B)!.5!(D)$);
				\coordinate[label=below right:$F$] (F) at ($(C)!.5!(D)$);   % M là trung điểm BC
				\coordinate[label=below left:$G$] (G) at ($(B)!.67!(F)$);
				\coordinate[label=above left:$A$] (A) at ($(G)+(0,5)$);
				\draw[dashed] (B)--(D) (A)--(G) (B)--(F) (C)--(E);
				\draw (F)--(A)--(B)--(C)--(D)--(A)--(C);
				\pic[pic text= ,draw,thick,angle radius=2mm,angle eccentricity=1.5] {right angle = A--G--F};
				\pic[pic text= ,draw,thick,angle radius=2mm,angle eccentricity=1.5] {right angle = B--F--C};
				\foreach \diem in {A,B,C,D,E,F,G}	\fill (\diem)circle(1.5pt); 
			\end{tikzpicture}
		}
	}
\end{ex}
\begin{ex}%Câu 33.[Chuyên Bắc Giang 2019][1H3B5-2]
	Cho hình chóp $S.ABCD$ có $SA\perp(ABCD)$, đáy $ABCD$ là hình chữ nhật. Biết $AD=2a$, $SA=a$. Khoảng cách từ $A$ đến $(SCD)$ bằng 
	\choice
	{$\dfrac{3a}{\sqrt{7}}$}
	{$\dfrac{3a\sqrt{2}}{2}$}
	{\True $\dfrac{2a}{\sqrt{5}}$}
	{$\dfrac{2a\sqrt{3}}{3}$}
	\loigiai{
		\immini{
			Gọi $H$ là hình chiếu của $A$ lên $SD$ ta chứng minh được $AH\perp(SCD)$.\\
			$\dfrac{1}{AH^2}=\dfrac{1}{SA^2}+\dfrac{1}{AD^2}\Rightarrow AH=\dfrac{2a}{\sqrt{5}}$.
		}
		{	\begin{tikzpicture}[scale=.8, font=\footnotesize, line join=round, line cap=round, >=stealth]
				\def\bc{4} % cạnh BC
				\def\ba{2} % cạnh BA
				\def\h{4} % đường cao
				\def\gocB{30} % góc B của đáy
				\coordinate[label=below left:$B$] (B) at (0,0);
				\coordinate[label=above left:$A$] (A) at (\gocB:\ba);
				\coordinate[label=below:$C$] (C) at (\bc,0);
				\coordinate[label=right:$D$] (D) at ($(C)-(B)+(A)$);
				\coordinate[label=above:$S$] (S) at ($(A)+(90:\h)$);
				\coordinate[label=above right:$H$] (H) at ($(S)!0.4!(D)$);
				\draw (B)--(C)--(D)--(S)--cycle (S)--(C);
				\draw[dashed] (H) -- (A)--(D) (S)--(A)--(B);
				\foreach \diem in {A,B,C,D,S,H}	\fill (\diem)circle(1.5pt);
				\pic[pic text= ,draw,thick,angle radius=2mm,angle eccentricity=1.5] {right angle = S--A--B};
				\pic[pic text= ,draw,thick,angle radius=2mm,angle eccentricity=1.5] {right angle = S--A--D};
				\pic[pic text= ,draw,thick,angle radius=2mm,angle eccentricity=1.5] {right angle = A--H--D};
			\end{tikzpicture}
	}}
\end{ex}
\begin{ex}%Câu 34.[Chuyên Lam Sơn Thanh Hóa 2019][1H3B5-2]
	Cho hình chóp $S.ABC$ có đáy là tam giác vuông tại $A$, $AB=a$, $AC=a\sqrt{3}$, $SA$ vuông góc với mặt phẳng đáy và $SA=2a$. Khoảng cách từ điểm $A$ đến mặt phẳng $(SBC)$ bằng 
	\choice
	{$\dfrac{a\sqrt{57}}{19}$}
	{\True $\dfrac{2a\sqrt{57}}{19}$}
	{$\dfrac{2a\sqrt{3}}{19}$}
	{$\dfrac{2a\sqrt{38}}{19}$}
	\loigiai{
		\immini{
			Kẻ $AH \perp BC, H \in BC$, $AK \perp SH, K \in SH$.\\
			Ta chứng minh được $\mathrm{d}(A,(SBC))=AK$.\\
			Ta có $\dfrac{1}{AK^2}=\dfrac{1}{AH^2}+\dfrac{1}{AS^2}\\
			=\dfrac{1}{AB^2}+\dfrac{1}{AC^2}+\dfrac{1}{AS^2}\\
			=\dfrac{1}{a^2}+\dfrac{1}{3a^2}+\dfrac{1}{4a^2}=\dfrac{19}{12a^2}$.\\
			Suy ra $AK=\dfrac{2a\sqrt{3}}{\sqrt{19}}$ hay $\mathrm{d}(A,(SBC))=\dfrac{2a\sqrt{57}}{19}$.
		}
		{
			\begin{tikzpicture}[scale =0.8]
				\def\a{5}
				\def\h{5}
				\path 	
				(0:0) coordinate (A)
				++(0:\a) coordinate (C)
				++(-150:4*\a/5) coordinate (B)
				(A)+(90:4) coordinate (S)  ($(B)!0.4!(C)$) coordinate (H) ($(S)!0.5!(H)$) coordinate (K)
				;
				\pic[pic text= ,draw,thick,angle radius=2mm,angle eccentricity=1.5] {right angle = B--A--C};
				\pic[pic text= ,draw,thick,angle radius=2mm,angle eccentricity=1.5] {right angle = A--H--B};
				\pic[pic text= ,draw,thick,angle radius=2mm,angle eccentricity=1.5] {right angle = A--K--H};
				\draw (A)--(B)--(C) (S)--(A) (S)--(C) (H)--(S)--(B);
				\draw[dashed] (A)--(C) (K)--(A)--(H);
				\foreach \x/\goc in {A/180,B/-90,C/-90,S/90,H/-45,K/30}
				\fill (\x) circle (1.5pt)
				($(\x)+(\goc:3mm)$) node {$\x$};
			\end{tikzpicture}
	}}
\end{ex}
\begin{ex}%Câu 35.[Hùng Vương Bình Phước 2019][1H3K5-2]
	Cho hình chóp tứ giác đều $S.ABCD$ có cạnh đáy bằng $a$ và chiều cao bằng $a\sqrt{2}$. Tính khoảng cách $d$ từ tâm $O$ của đáy $ABCD$ đến một mặt bên theo $a$. 
	\choice
	{$\mathrm{d}=\dfrac{2a\sqrt{5}}{3}$}
	{$\mathrm{d}=\dfrac{a\sqrt{3}}{2}$}
	{$\mathrm{d}=\dfrac{a\sqrt{5}}{2}$}
	{\True $\mathrm{d}=\dfrac{a\sqrt{2}}{3}$}
	\loigiai{
		\begin{center}
			\begin{tikzpicture}[scale=0.7, line join = round, line cap = round]
				\tikzset{label style/.style={font=\footnotesize}}
				
				\coordinate[label=below left:$B$] (B) at (0,0);
				\coordinate[label=below left:$C$] (C) at (5,0);
				\coordinate[label=above left:$A$] (A) at (2,2);
				
				\coordinate[label=below right:$D$] (D) at ($(A)+(C)-(B)$);
				\coordinate[label=below:$O$] (O) at ($(C)!0.5!(A)$);
				\coordinate[label=above:$S$] (S) at ($(O)+(0,6)$);
				\coordinate[label=below right:$H$] (H) at ($(C)!0.5!(D)$);
				\coordinate[label=below right:$K$] (K) at ($(S)!0.7!(H)$);
				\pic[pic text= ,draw,thick,angle radius=2mm,angle eccentricity=1.5] {right angle = S--O--H};
				\pic[pic text= ,draw,thick,angle radius=2mm,angle eccentricity=1.5] {right angle = O--K--H};
				\draw (H)--(S)--(B)--(C)--(D)--(S)--(C);
				\draw[dashed] (H)--(O)--(S)--(A)--(B)--(D)--(A)--(C) (O)--(K);
				\foreach \diem in {A,B,C,D,S,H,O,K}	\fill (\diem)circle(1.5pt);
			\end{tikzpicture}
		\end{center}
		$S.ABCD$ là hình chóp tứ giác đều nên $ABCD$ là hình vuông và $SO\perp(ABCD)$.\\
		Vẽ $OH$ vuông góc với $CD$ tại $H$ thì $H$ là trung điểm $CD$, $OH=\dfrac{a}{2}$.\\
		Dễ thấy $CD\perp(SOH)\Rightarrow(SCD)\perp(SOH)$.\\ Trong $(SOH)$, kẻ $OK$ vuông góc với $SH$ tại $K$ thì $OK\perp(SCD)$.\\
		Do đó $d\left( O,(SCD)\right)  =OK$.\\
		Tam giác vuông $SOH$ có $OK$ là đường cao nên $OK=\dfrac{OS\cdot OH}{\sqrt{OS^2+OH^2}}=\dfrac{a\sqrt{2}\cdot\dfrac{a}{2}}{\sqrt{2a^2+\dfrac{a^2}{4}}}=\dfrac{a\sqrt{2}}{3}$.\\
		Vậy $\mathrm{d}\left( O,(SCD)\right) =\dfrac{a\sqrt{2}}{3}$.}
\end{ex}
\begin{ex}%Câu 36.[Chuyên Trần Phú Hải Phòng 2019][1H3K5-2] 
	Cho khối chóp $S.ABCD$ có đáy $ABCD$ là hình vuông cạnh $a$, $SA\perp(ABCD)$ và $SA=a\sqrt{2}$. Gọi $M$ là trung điểm cạnh $SC$. Khoảng cách từ điểm $M$ đến mặt phẳng $(SBD)$ bằng
	\choice
	{$\dfrac{a\sqrt{2}}{4}$}
	{\True $\dfrac{a\sqrt{10}}{10}$}
	{$\dfrac{a\sqrt{2}}{2}$}
	{$\dfrac{a\sqrt{10}}{5}$}
	\loigiai{
		\begin{center}
			\begin{tikzpicture}[scale=.8, font=\footnotesize, line join=round, line cap=round, >=stealth]
				\def\bc{5} % cạnh BC
				\def\ba{3} % cạnh BA
				\def\h{5} % đường cao
				\def\gocB{30} % góc B của đáy
				\coordinate[label=below left:$B$] (B) at (0,0);
				\coordinate[label=above left:$A$] (A) at (\gocB:\ba);
				\coordinate[label=below:$C$] (C) at (\bc,0);
				\coordinate[label=right:$D$] (D) at ($(C)-(B)+(A)$);
				\coordinate[label=above:$S$] (S) at ($(A)+(90:\h)$);
				\coordinate[label=above right:$M$] (M) at ($(S)!0.5!(C)$);
				\coordinate[label=above right:$O$] (O) at ($(A)!0.5!(C)$);
				\coordinate[label=above right:$H$] (H) at ($(S)!0.75!(O)$);
				\draw (B)--(C)--(D)--(S)--cycle (S)--(C);
				\draw[dashed] (M) -- (A)--(D)--(B) (H)--(A)--(C) (O)--(S)--(A)--(B);
				\foreach \diem in {A,B,C,D,S,M,O,H}	\fill (\diem)circle(1.5pt);
				\pic[pic text= ,draw,thick,angle radius=2mm,angle eccentricity=1.5] {right angle = S--A--B};
				\pic[pic text= ,draw,thick,angle radius=2mm,angle eccentricity=1.5] {right angle = A--H--O};
			\end{tikzpicture}
		\end{center}
		Do $M$ là trung điểm $SC$ nên $\mathrm{d}\left(M,(SBD)\right)=\dfrac{1}{2}\mathrm{d}\left(C,(SBD)\right)=\dfrac{1}{2}\mathrm{d}\left(A,(SBD)\right)$.\\
		Gọi $H$ là hình chiếu của $A$ lên $(SBD)\Rightarrow\mathrm{d}\left(A,(SBD)\right)=AH$.\\
		Lại có $AS,AB,AD$ đôi một vuông góc nên $\dfrac{1}{AH^2}=\dfrac{1}{AS^2}+\dfrac{1}{AB^2}+\dfrac{1}{AD^2}=\dfrac{1}{a^2}+\dfrac{1}{a^2}+\dfrac{1}{(a\sqrt{2})^2}=\dfrac{5}{2a^2}$ \\
		$ \Rightarrow AH=\dfrac{a\sqrt{10}}{5}\Rightarrow\mathrm{d}\left(M,(SBD)\right)=\dfrac{a\sqrt{10}}{10} $.}
\end{ex}
\begin{ex}%[THPT Gang Thép Thái Nguyên 2019]%Câu 37.%[1H3B5-2]
	 Cho hình chóp $S.ABC$ có đáy $ABC$ là tam giác vuông tại $A$, $AB=a$, $AC=a\sqrt{3}$; $SA$ vuông góc với đáy, $SA=2a$. Khoảng cách từ điểm $A$ đến mặt phẳng $(SBC)$ bằng
	\choice
	{$\dfrac{2a\sqrt{3}}{\sqrt{7}}$}
	{$\dfrac{a\sqrt{3}}{\sqrt{7}}$}
	{$\dfrac{a\sqrt{3}}{\sqrt{19}}$}
	{\True $\dfrac{2a\sqrt{3}}{\sqrt{19}}$}
	\loigiai{
		\immini{
		Ta có\\
		$\heva{&SA\perp(ABC)\\&BC\subset(ABC)}\Rightarrow SA\perp BC$.\\
		Trong $(ABC)$, kẻ $AH\perp BC$, mà $BC\perp SA\Rightarrow BC\perp(SAH)\Rightarrow BC\perp SH$.\\
		Trong $(SAH)$, kẻ $AK\perp SH$, mà $SH\perp BC\Rightarrow AK\perp(SBC)$ hay $\mathrm{d}\left(A;(SBC)\right)=AK$.\\
		Vì $\triangle ABC$ vuông tại $A$ nên $BC=\sqrt{AB^2+AC^2}=2a$.\\
		Mặt khác có $AH$ là đường cao nên $AH=\dfrac{AB\cdot AC}{BC}=\dfrac{\sqrt{3}a}{2}$.\\
		Vì $\triangle SAH$ vuông tại $A$ nên $SH=\sqrt{SA^2+AH^2}=\dfrac{\sqrt{19}a}{2}$.\\
		Vậy có $AK$ là đường cao $AK=\dfrac{SA\cdot AH}{SH}=\dfrac{2a\sqrt{3}}{\sqrt{19}}$.\\
		Nhận xét. Trong thực hành làm toán trắc nghiệm ta nên áp dụng bài toán sau:\\
		Cho tứ diện $OABC$ có $OA, OB, OC$ đôi một vuông góc với nhau và $H$ là hình chiếu của $O$ lên mặt phẳng $(ABC)$. Khi đó $\dfrac{1}{OH^2}=\dfrac{1}{OA^2}+\dfrac{1}{OB^2}+\dfrac{1}{OC^2}$.}
		{\begin{tikzpicture}[line join=round, line cap=round]
				\coordinate (A) at (-2,0);
				\coordinate (B) at (0,-3);
				\coordinate (C) at (3,0);
				\coordinate (S) at ($(A)+(0,4)$);
				\coordinate (H) at ($(B)!0.3!(C)$);
				\coordinate (K) at ($(S)!0.5!(H)$);
				\draw(S)--(A) (S)--(B) (S)--(C) (B)--(C) (A)--(B) (S)--(H);
				\draw[dashed](A)--(C) (A)--(K) (A)--(H);
				\foreach \d/\g in{S/90,A/180,B/-90,C/0,H/-90,K/0}
				\draw[fill=black](\d)circle(1pt)node[shift={(\g:0.35)}]{$\d$};
				\tkzMarkRightAngles(A,H,B A,K,H)
				
		\end{tikzpicture}}
	}
\end{ex}
\begin{ex}%Câu 38.[Chuyên Sơn La 2019][1H3B5-2]
	Cho hình chóp $S.ABC$ có đáy $ABC$ là tam giác đều cạnh $a$, $SA=a$ và $SA$ vuông góc với mặt phẳng đáy. Khoảng cách từ điểm $A$ đến mặt phẳng $(SBC)$ bằng 
	\choice
	{$\dfrac{\sqrt{2}a}{2}$}
	{$\dfrac{\sqrt{3}a}{7}$}
	{\True $\dfrac{\sqrt{21}a}{7}$}
	{$\dfrac{\sqrt{15}a}{5}$}
	\loigiai{
		\immini{
			Gọi $M$ là trung điểm $BC$. Kẻ $AH\perp SM$ tại $H$.\\
			Ta có $AM\perp BC$ và $SA\perp BC$ nên $BC\perp(SAM)\\
			\Rightarrow BC\perp AH \qquad (1)$.\\
			Mà $AH\perp SM \qquad (2)$.\\
			Từ $(1)$ và $(2)$ suy ra $AH\perp(SBC)$.\\
			Do đó $\mathrm{d}\left(A,(SBC)\right)=AH$.\\
			Xét tam giác $SAM$ vuông tại $A$, có.\\
			$\dfrac{1}{AH^2}=\dfrac{1}{AM^2}+\dfrac{1}{AS^2} =\dfrac{1}{\left(\dfrac{a\sqrt{3}}{2}\right)^2}+\dfrac{1}{a^2} =\dfrac{7}{3a^2}\\
			\Rightarrow AH=a\sqrt{\dfrac{3}{7}} =\dfrac{\sqrt{21}a}{7}$.
		}
		{
			\begin{tikzpicture}
				\def\a{5}
				\def\h{5}
				\path 	
				(0:0) coordinate (A)
				++(0:\a) coordinate (C)
				++(-150:4*\a/5) coordinate (B)
				(A)+(90:4) coordinate (S)  ($(B)!0.5!(C)$) coordinate (M) ($(S)!0.5!(M)$) coordinate (H)
				;
				\pic[pic text= ,draw,thick,angle radius=2mm,angle eccentricity=1.5] {right angle = A--M--B};
				\pic[pic text= ,draw,thick,angle radius=2mm,angle eccentricity=1.5] {right angle = A--H--M};
				\draw (A)--(B)--(C) (S)--(A) (S)--(C) (M)--(S)--(B);
				\draw[dashed] (A)--(C) (H)--(A)--(M);
				\foreach \x/\goc in {A/180,B/-90,C/-90,S/90,M/-45,H/30}
				\fill (\x) circle (1.5pt)
				($(\x)+(\goc:3mm)$) node {$\x$};
			\end{tikzpicture}
	}}
\end{ex}
\begin{ex}%Câu 39.[Thpt Lê Văn Thịnh Bắc Ninh 2019][1H3K5-2] 
	Cho hình chóp đều $S.ABCD$, cạnh đáy bằng $a$, góc giữa mặt bên và mặt đáy là $60^{\circ}$. Tính khoảng cách từ điểm $B$ đến mặt phẳng $(SCD)$. 
	\choice
	{$\dfrac{a}{4}$}
	{$\dfrac{a\sqrt{3}}{4}$}
	{\True $\dfrac{a\sqrt{3}}{2}$}
	{$\dfrac{a}{2}$}
	\loigiai{
		* Ta có: $\dfrac{\mathrm{d}\left(B,(SCD)\right)}{\mathrm{d}\left(O,(SCD)\right)}=\dfrac{BD}{OD}=2\Rightarrow\mathrm{d}\left(B,(SCD)\right)=2\cdot\mathrm{d}\left(O,(SCD)\right)=2OH$. Trong đó $H$ là hình chiếu vuông góc của $O$ lên $(SCD)$. 
		\begin{center}
			\begin{tikzpicture}[scale=0.8, line join = round, line cap = round]
				\tikzset{label style/.style={font=\footnotesize}}
				
				\coordinate[label=below left:$B$] (B) at (0,0);
				\coordinate[label=below left:$C$] (C) at (5,0);
				\coordinate[label=above left:$A$] (A) at (2,2);
				
				\coordinate[label=below right:$D$] (D) at ($(A)+(C)-(B)$);
				\coordinate[label=below:$O$] (O) at ($(C)!0.5!(A)$);
				\coordinate[label=above:$S$] (S) at ($(O)+(0,6)$);
				\coordinate[label=below right:$I$] (I) at ($(C)!0.5!(D)$);
				\coordinate[label=below right:$H$] (H) at ($(S)!0.7!(I)$);
				\tkzMarkRightAngles[size=0.4](S,O,I O,H,I)
				\tkzMarkSegments[mark=|](I,D I,C)
				\tkzMarkAngles[size=0.4,arc=ll,mark=|](S,I,O)
				\tkzLabelAngles[pos=0.7,rotate=30](S,I,O){$60^{\circ}$}
				\draw (I)--(S)--(B)--(C)--(D)--(S)--(C);
				\draw[dashed] (I)--(O)--(S)--(A)--(B)--(D)--(A)--(C) (O)--(H);
				\foreach \diem in {A,B,C,D,S,I,O,H}	\fill (\diem)circle(1.5pt);
			\end{tikzpicture}
		\end{center}
		* Gọi $I$ là trung điểm của $CD$.\\
		Ta có:$\heva{&(SCD)\cap(ABCD)=CD\\&SI\perp CD\\&OI\perp CD}\Rightarrow\left((SCD);(ABCD)\right)=(OI;SI)=\widehat{SIO}=60^{\circ.}$ \\
		Xét tam giác $SOI$ vuông tại $O$ ta có: $SO=OI\cdot\tan 60^{\circ}=\dfrac{a\sqrt{3}}{2}$.\\
		Xét $\triangle SOI$, ta có $\dfrac{1}{OH^2}=\dfrac{1}{OI^2}+\dfrac{1}{OS^2}=\dfrac{4}{a^2}+\dfrac{4}{3a^2}=\dfrac{16}{3a^2} \Rightarrow OH=\dfrac{a\sqrt{3}}{4}$.\\
		Vậy $\mathrm{d}\left(B;(SCD)\right)=\dfrac{a\sqrt{3}}{2} $.}
\end{ex}
\begin{ex}%Câu 40.[Chuyên Vĩnh Phúc 2019][1H3K5-2]
	Cho hình chóp $S.ABCD$ có đáy là nửa lục giác đều $ABCD$ nội tiếp trong đường tròn đường kính $AD=2a$ và có cạnh $SA$ vuông góc với mặt phẳng đáy $(ABCD)$ với $SA=a\sqrt{6}$. Tính khoảng cách từ $B$ đến mặt phẳng $(SCD)$. 
	\choice
	{$a\sqrt{2}$}
	{$a\sqrt{3}$}
	{\True $\dfrac{a\sqrt{2}}{2}$}
	{$\dfrac{a\sqrt{3}}{2}$}
	\loigiai{
		\begin{center}
			\begin{tikzpicture}[scale=.8, font=\footnotesize, line join=round, line cap=round, >=stealth]
				\def\h{4} % đường cao
				\def\gocB{30} % góc B của đáy
				\coordinate[label=below left:$A$] (A) at (0,0);
				\coordinate[label= above right:$D$] (D) at (5,0);
				\coordinate[label=below left:$B$] (B) at (1,-2);
				\coordinate[label=below right:$C$] (C) at (3.5,-2);
				\coordinate[label=above:$S$] (S) at ($(A)+(90:\h)$);
				\coordinate[label=above left:$H$] (H) at ($(S)!0.4!(C)$);
				\coordinate[label=below:$E$] (E) at (2,-4);
				\draw (S)--(A)--(B)--(C)--(D)--(S)--(A) (B)--(S)--(C)--(E)--(B);
				\draw[dashed] (H) -- (A)--(D) (A)--(C);
				\foreach \diem in {A,B,C,D,S,H,E}	\fill (\diem)circle(1.5pt);
				\pic[pic text= ,draw,thick,angle radius=2mm,angle eccentricity=1.5] {right angle = S--A--B};
				\pic[pic text= ,draw,thick,angle radius=3mm,angle eccentricity=1.5] {right angle = S--A--D};
				\pic[pic text= ,draw,thick,angle radius=3mm,angle eccentricity=1.5] {right angle = A--H--C};
			\end{tikzpicture}
			\begin{tikzpicture}
				\coordinate (E) at (60:2);
				\draw (E) circle (0.02);
				\coordinate (D) at (0:2);
				\draw (D) circle (0.02) node[right]{$D$};
				\coordinate (C) at (-60:2);
				\draw (C) circle (0.02) node[right]{$C$};
				\coordinate (B) at (-120:2);
				\draw (B) circle (0.02) node[left]{$B$};
				\coordinate (A) at (-180:2) ;
				\draw (A) circle (0.02) node[left]{$A$};
				\coordinate (F) at (-240:2);
				\draw (F) circle (0.02);
				\draw (E)--(D)--(C)--(B)--(A)--(F)--(E);
				\draw (F)--(C);
				\draw (E)--(B);
				\draw (A)--(D);
				\draw (0,0.15) node [above]{$O$};
				\draw (0,0) circle (2);
			\end{tikzpicture}
			
		\end{center}
		Từ giả thiết suy ra: $AB=BC=CD=\dfrac{AD}{2}=a$, $AC=a\sqrt{3}$.\\
		Gọi $E=AB\cap CD$, suy ra tam giác $ADE$ đều.\\
		Khi đó $C$ là trung điểm của $ED$ và $AC\perp ED$.\\
		Dựng $AH\perp SC$ thì $AH\perp(SCD)$, suy ra $\mathrm{d} \left( A,(SCD)\right) =AH$.\\
		Xét tam giác $SAC$ vuông tại $A$, có $AH$ là đường cao.\\
		Suy ra: $\dfrac{1}{AH^2}=\dfrac{1}{SA^2}+\dfrac{1}{AC^2}\Rightarrow AH=\sqrt{2}a$.\\
		Mà $\mathrm{d} \left( B,(SCD)\right) =\dfrac{1}{2}\mathrm{d}\left( A,(SCD)\right) =\dfrac{1}{2}AH=\dfrac{a\sqrt{2}}{2}$.}
\end{ex}
\begin{ex}%Câu 41.[THPT Minh Châu Hưng Yên 2019][1H3K5-2]
	Cho khối chóp $S.ABCD$ có đáy $ABCD$ là hình thang vuông tại $A$ và $B$,\\
	$AD=2a$, $AB=BC=a$.  Hình chiếu của $S$ lên mặt phẳng đáy trùng với trung điểm $H$ của $AD$ và $SH=\dfrac{a\sqrt{6}}{2}$. Tính khoảng cách $\mathrm{d}$ từ $B$ đến mặt phẳng $(SCD)$. 
	\choice
	{$\mathrm{d}=\dfrac{\sqrt{6}a}{8}$}
	{$\mathrm{d}=a$}
	{\True $\mathrm{d}=\dfrac{\sqrt{6}a}{4}$}
	{$\mathrm{d}=\dfrac{\sqrt{15}a}{5}$}
	\loigiai{
		\immini{
			Gọi $M$ là trung điểm của $CD$, $K$ là hình chiếu của $H$ lên $SM$.\\
			Tam giác $HCD$ vuông tại $H$ có $CD=a\sqrt{2}$ và $HM=\dfrac{a\sqrt{2}}{2}$.\\
			Ta có $BH\parallel CD\Rightarrow\mathrm{d}\left(B,(SCD)\right)=\mathrm{d}\left(H,(SCD)\right)=HK$.\\
			Tam giác $SHM$ vuông tại $H$ có $HK=\dfrac{HM\cdot HS}{\sqrt{HM^2+HS^2}}=\dfrac{a\sqrt{6}}{4}$.\\
			Vậy $\mathrm{d}\left(B,(SCD)\right)=\dfrac{a\sqrt{6}}{4}$.
		}
		{
			\begin{tikzpicture}[scale=.8, font=\footnotesize, line join=round, line cap=round, >=stealth]
				\def\h{4} % đường cao
				\def\gocB{30} % góc B của đáy
				\coordinate[label=below left:$A$] (A) at (0,0);
				\coordinate[label= above right:$D$] (D) at (5,0);
				\coordinate[label=below left:$B$] (B) at (1,-2);
				\coordinate[label=below right:$C$] (C) at (3.5,-2);
				\coordinate[label=above left:$H$] (H) at ($(A)!0.5!(D)$);
				\coordinate[label=above:$S$] (S) at ($(H)+(90:\h)$);
				\coordinate[label=above right:$M$] (M) at ($(C)!0.5!(D)$);
				\coordinate[label=above right:$K$] (K) at ($(S)!0.6!(M)$);
				\draw (S)--(A)--(B)--(C)--(D)--(S)--(M) (B)--(S)--(C);
				\draw[dashed] (S)--(H)--(M)  (A)--(D) (C)--(H)--(K);
				\foreach \diem in {A,B,C,D,S,H,M}	\fill (\diem)circle(1.5pt);
				\pic[pic text= ,draw,thick,angle radius=2mm,angle eccentricity=1.5] {right angle = S--H--A};
				\pic[pic text= ,draw,thick,angle radius=2mm,angle eccentricity=1.5] {right angle = B--A--D};
				\pic[pic text= ,draw,thick,angle radius=2mm,angle eccentricity=1.5] {right angle = H--K--M};
			\end{tikzpicture}
	}}
\end{ex}
\begin{ex}%Câu 42.[Chuyên Quang Trung Bình Phước 2019][1H3K5-2]
	Tứ diện $O.ABC$ có $OA,OB,OC$ đôi một vuông góc với nhau $OA=OB=OC=\sqrt{3}$. Khoảng cách từ $O$ đến $(ABC)$ bằng
	\choice
	{$\dfrac{1}{\sqrt{3}}$}
	{\True $1$}
	{$\dfrac{1}{2}$}
	{$\dfrac{1}{3}$}
	\loigiai{
		\immini{
			Gọi $A'$ là chân đường cao kẻ từ $A$ lên $BC$, $C'$ là chân đường cao kẻ từ $C$ lên $AB$.\\
			Gọi $H$ là giao của $AA'$ với $CC'$ suy ra H là trực tâm của tam giác $ABC$. Ta dễ dàng chứng
			minh được $OH\perp (ABC)$.\\
			Do đó $\mathrm{d}(O;(ABC))=OH$.\\
			Tam giác $OAA’$ vuông tại $O,$ có $OH$ là đường cao. \\
			Suy ra $\dfrac{1}{OH^2}=\dfrac{1}{OA^2}+\dfrac{1}{OA'^2} \qquad(1)$ .\\
			Mà tam giác $OBC$ vuông tại $B,$ có $OA’$ là đường cao.\\ Nên $\dfrac{1}{OA'^2}=\dfrac{1}{OB^2}+\dfrac{1}{OC^2} \qquad (2)$.\\
			Từ $(1)$ và $(2)$ suy ra $\dfrac{1}{OH^2}=\dfrac{1}{OA^2}+\dfrac{1}{OB^2}+\dfrac{1}{OC^2}$.}
		{
			\begin{tikzpicture}[scale=.6, font=\footnotesize, line join=round, line cap=round, >=stealth]
				\def\h{5} % đường cao
				\def\gocB{30} % góc B của đáy
				\coordinate[label=left:$O$] (O) at (0,0);
				\coordinate[label= above right:$B$] (B) at (5,0);
				\coordinate[label=below left:$C$] (C) at (-1.5,-2);
				\coordinate[label=below:$A'$] (A') at ($(C)!0.45!(B)$);
				\coordinate[label=above:$A$] (A) at ($(O)+(90:\h)$);
				\coordinate[label=above right:$C'$] (C') at ($(A)!0.55!(B)$);
				\coordinate[label=below right:$H$] (H) at ($(A)!0.7!(A')$);
				\tkzInterLL(A,A')(C,C')\tkzGetPoint{H}
				\draw (C)--(A)--(B)--(C)--(C') (A)--(A');
				\draw[dashed] (A)--(O)--(C)  (O)--(B) (H)--(O)--(A');
				\foreach \diem in {O,A,B,C,A',C',H}	\fill (\diem)circle(1.5pt);
				\pic[pic text= ,draw,thick,angle radius=2mm,angle eccentricity=1.5] {right angle = A--O--C};
				\pic[pic text= ,draw,thick,angle radius=2mm,angle eccentricity=1.5] {right angle = A--O--B};
			\end{tikzpicture}
		}
		\noindent Thay $OA=OB=OC=\sqrt{3}$ vào, ta được
		$\dfrac{1}{OH^2}=\dfrac{1}{3}+\dfrac{1}{3}+\dfrac{1}{3}=1\Leftrightarrow OH=1$.\\
		Vậy $\mathrm{d}(O,(ABC))=OH=1$.}
\end{ex}
\begin{ex}%Câu 43.[Thpt Cẩm Giàng 2 2019][1H3K5-2]
	Cho hình chóp $S.ABCD$ có đáy $ABCD$ là hình thoi cạnh $a$, $\widehat{ABC}=60^{\circ}$. Cạnh bên $SA$ vuông góc với đáy, $SC=2a$. Khoảng cách từ $B$ đến mặt phẳng $(SCD)$ bằng
	\choice
	{\True $\dfrac{a\sqrt{15}}{5}$}
	{$\dfrac{a\sqrt{2}}{2}$}
	{$\dfrac{2a}{\sqrt{5}}$}
	{$\dfrac{5a\sqrt{30}}{3}$}
	\loigiai{
		\begin{center}
			\begin{tikzpicture}[=>stealth,line join=round,line cap=round,font=\footnotesize,scale=1]
				\def\a{5}
				\def\goc{210}
				\def\b{2.5}
				\def\h{4}
				\path
				(0,0)coordinate (A)++(0:\a)coordinate (D)++(\goc:\b)coordinate (C)++(180:\a)coordinate (B)
				(A)++(90:\h)coordinate (S)
				;
				\coordinate[label=below:$H$] (H) at ($(C)!0.5!(D)$);
				\coordinate[label=below right:$K$] (K) at ($(S)!0.5!(K)$);
				\draw (B)--(C)--(D)--(S)--cycle (S)--(C) (S)--(H)
				;
				\draw[dashed] (B)--(A)--(D) (S)--(A)--(C) (K)--(A)--(H);
				\foreach \x/\g in{A/150,B/-90,C/-60,D/0,S/90}
				\fill[black](\x)circle(1pt) ($(\x)+(\g:3mm)$)node{$\x$};
				
				\pic[pic text= ,draw,thick,angle radius=2mm,angle eccentricity=1.5] {right angle = S--A--D};
				\pic[pic text= ,draw,thick,angle radius=2mm,angle eccentricity=1.5] {right angle = A--H--C};
				\pic[pic text= ,draw,thick,angle radius=2mm,angle eccentricity=1.5] {right angle = A--K--H};
				
			\end{tikzpicture}
		\end{center}
		\textbf{Cách 1:} Sử dụng kiến thức ở lớp 11.\\
		$ABCD$ là hình thoi cạnh $a$, $\widehat{ABC}=60^{\circ}\Rightarrow\triangle ABC,\triangle ACD$ là các tam giác đều cạnh $a$.\\
		Xét $\triangle SAC$ vuông tại $A$ có: $SA=\sqrt{SC^2-AC^2} =\sqrt{4a^2-a^2} =a\sqrt{3}$.\\
		Vì $AB\parallel CD$ nên $AB\parallel (SCD)$. Do đó $\mathrm{d}\left(B,(SCD)\right)=\mathrm{d}\left(A,(SCD)\right)$.\\
		Kẻ $AH\perp CD ~ (H\in CD)$. Suy ra $H$ là trung điểm của cạnh $CD$, $AH=\dfrac{a\sqrt{3}}{2}$.\\
		Kẻ $AK\perp SH (K\in SH) \qquad (1)$.\\
		Ta có: $\heva{&CD\perp AH\\&CD\perp SA}\Rightarrow CD\perp(SAH)\Rightarrow CD\perp AK \qquad (2)$.\\
		Từ $(1)$ và $(2)$ suy ra: $AK\perp(SCD)\Rightarrow\mathrm{d}\left(A,(SCD)\right)=AK$.\\
		Xét $\triangle SAH$ vuông ở $A$: $\dfrac{1}{AK^2}=\dfrac{1}{AH^2}+\dfrac{1}{SA^2} =\dfrac{4}{3a^2}+\dfrac{1}{3a^2} =\dfrac{5}{3a^2}\Rightarrow AK=\dfrac{a\sqrt{15}}{5}$.\\
		Vậy $\mathrm{d}\left(B,(SCD)\right)=\dfrac{a\sqrt{15}}{5}$.\\
		\textbf{	Cách 2}: Tính khoảng cách thông qua tính thể tích.\\
		$ABCD$ là hình thoi cạnh $a$, $\widehat{ABC}=60^{\circ}\Rightarrow\triangle ABC,\triangle ACD$ là các tam giác đều cạnh $a$.\\
		Xét $\triangle SAC$ vuông tại $A$ có: $SA=\sqrt{SC^2-AC^2} =\sqrt{4a^2-a^2} =a\sqrt{3}$.\\
		Vì $AB\parallel DC$ nên $AB\parallel (SDC)$. Do đó $\mathrm{d}\left(B,(SCD)\right)=\mathrm{d}\left(A,(SCD)\right)=\dfrac{3V_{SACD}}{S_{\triangle SCD}}$.\\
		$V_{SACD}=\dfrac{1}{3}SA\cdot S_{\triangle ACD} =\dfrac{1}{3}a\sqrt{3}\cdot\dfrac{a^2\sqrt{3}}{4} =\dfrac{a^3}{4}$.\\
		Xét $\triangle SAC$ và $\triangle SAD$ có: $AD=AC=a$, $SA$ chung, $\widehat{SAC}=\widehat{SAD}=90^{\circ}$.\\
		Do đó $\triangle SAC=\triangle SAD\Rightarrow SC=SD\Rightarrow\triangle SCD$ cân tại $S$.\\
		Gọi $H$ là trung điểm $CD\Rightarrow SH\perp CD$.\\
		Xét $\triangle SHC$ vuông ở $H$: $SH=\sqrt{SC^2-CH^2} =\sqrt{4a^2-\dfrac{a^2}{4}} =\dfrac{a\sqrt{15}}{2}$.\\
		$S_{\triangle SCD}=\dfrac{1}{2}SH\cdot CD =\dfrac{1}{2}\cdot\dfrac{a\sqrt{15}}{2}\cdot a =\dfrac{a^2\sqrt{15}}{4}$.\\
		$\mathrm{d}\left(A,(SCD)\right) =\dfrac{3\cdot\dfrac{a^3}{4}}{\dfrac{a^2\sqrt{15}}{4}} =\dfrac{a\sqrt{15}}{5}$.\\
		Vậy $\mathrm{d}\left(B,(SCD)\right)=\dfrac{a\sqrt{15}}{5}$.}
\end{ex}
\begin{ex}%Câu 44.[Chuyên Biên Hòa - Hà Nam - 2020][1H3K5-2]
	Cho hình chóp $S.ABCD$ có đáy $ABCD$ là hình thang vuông tại $A$ và $D$; $AB=AD=2a; DC=a$. Điểm $I$ là trung điểm đoạn $AD,$ hai mặt phẳng $(SIB)$ và $(SIC)$ cùng vuông góc với mặt phẳng $(ABCD)$. Mặt phẳng $(SBC)$ tạo với mặt phẳng $(ABCD)$ một góc $60^{\circ}$. Tính khoảng cách từ $D$ đến $(SBC)$ theo $a$. 
	\choice
	{\True $\dfrac{a\sqrt{15}}{5}$}
	{$\dfrac{9a\sqrt{15}}{10}$}
	{$\dfrac{2a\sqrt{15}}{5}$}
	{$\dfrac{9a\sqrt{15}}{20}$}
	\loigiai{
		\immini{
			Theo đề ta có $SI\perp(ABCD)$.\\
			Gọi $K$ là hình chiếu vuông góc của $I$ trên $BC$.\\
			Suy ra góc giữa hai mặt phẳng $(SBC)$ và $(ABCD)$ là $\widehat{SKI}=60^{\circ}$.\\
			Gọi $E$ là trung điểm của $AB, M=IK\cap DE$.\\
			Do $BCDE$ là hình bình hành nên $DE\parallel(SBC)$ \\
			$ \Rightarrow\mathrm{d}\left(D,(SBC)\right)=\mathrm{d}\left(DE,(SBC)\right)=\mathrm{d}\left(M,(SBC)\right) $.\\
			Gọi $H$ là hình chiếu vuông góc của $M$ trên $SK$.\\
			Suy ra $\mathrm{d}\left(M,(SCD)\right)=MH$.\\
			
		}
		{
			\begin{tikzpicture}[scale=.8, font=\footnotesize, line join=round, line cap=round, >=stealth]
				\def\h{5} % đường cao
				\def\gocB{30} % góc B của đáy
				\coordinate[label=below left:$A$] (A) at (0,0);
				\coordinate[label= above right:$B$] (B) at (6,0);
				\coordinate[label=below left:$D$] (D) at (1.5,-2);
				\coordinate[label=below right:$C$] (C) at (4.5,-2);
				\coordinate[label=below left:$I$] (I) at ($(A)!0.5!(D)$);
				\coordinate[label=above:$S$] (S) at ($(I)+(90:\h)$);
				\coordinate[label=below right:$K$] (K) at ($(C)!0.35!(B)$);
				\coordinate[label=above :$E$] (E) at ($(A)!0.5!(B)$);
				\coordinate[label=above:$M$] (M) at ($(I)!0.35!(K)$);
				\coordinate[label=right:$H$] (H) at ($(S)!0.8!(K)$);
				\draw (D)--(A)--(S)--(D)--(C)--(B)--(S) (C)--(S)--(K) (S)--(I);
				\draw[dashed] (E)--(D)--(A)--(B)  (A)--(S) (M)--(H) (I)--(K);
				\foreach \diem in {A,B,C,D,S,I,H,E,K,M}	\fill (\diem)circle(1.5pt);
				\pic[pic text= ,draw,thick,angle radius=2mm,angle eccentricity=1.5] {right angle = S--I--K};
				\pic[pic text= ,draw,thick,angle radius=2mm,angle eccentricity=1.5] {right angle = I--K--C};
				\pic[pic text= ,draw,thick,angle radius=2mm,angle eccentricity=1.5] {right angle = M--H--K};
			\end{tikzpicture}
		}
		\immini{
			Dễ thấy: $IM=\dfrac{1}{2}AU=\dfrac{1}{2}KN=\dfrac{1}{2}MK$.\\
			$IN=IM+MK+KN=\dfrac{1}{2}MK+MK+MK=\dfrac{5}{2}MK$.\\
			Suy ra: $MK=\dfrac{2}{5}IN=\dfrac{2}{5}\sqrt{ID^2+DN^2}=\dfrac{2a\sqrt{5}}{5}$.\\
			Trong tam giác $MHK,$ ta có: $MH=MK\cdot\sin 60^{\circ}=\dfrac{a\sqrt{15}}{5}$.\\
			Vậy $\mathrm{d}\left(D,(SBC)\right)=\dfrac{a\sqrt{15}}{5}$.
		}
		{
			\begin{tikzpicture}[scale=.8, font=\footnotesize, line join=round, line cap=round, >=stealth]
				\def\h{5} % đường cao
				\def\gocB{30} % góc B của đáy
				\coordinate[label=below left:$A$] (A) at (0,0);
				\coordinate[label= above right:$B$] (B) at (4,0);
				\coordinate[label=below left:$C$] (C) at (2,-4);
				\coordinate[label=below right:$D$] (D) at (0,-4);
				\coordinate[label=below right:$N$] (N) at (4,-4);
				\coordinate[label=below left:$I$] (I) at ($(A)!0.5!(D)$);
				\coordinate[label=above:$E$] (E) at ($(A)!0.5!(B)$);
				\coordinate[label=right:$J$] (J) at ($(B)!0.5!(N)$);
				\coordinate[label=above:$U$] (U) at ($(A)!0.4!(J)$);
				\coordinate[label=above:$V$] (V) at ($(A)!0.8!(J)$);
				\coordinate[label=below :$K$] (K) at ($(N)!0.4!(I)$);
				\coordinate[label=below:$M$] (M) at ($(N)!0.8!(I)$);
				\draw (D)--(A)--(B)--(C)--(D) (D)--(E) (C)--(B) (I)--(N) (A)--(J);
				\draw[dashed] (C)--(N)--(B);
				\foreach \diem in {A,B,C,D,I,N,E,M,U,V,K}	\fill (\diem)circle(1.5pt);
				\pic[pic text= ,draw,thick,angle radius=2mm,angle eccentricity=1.5] {right angle = I--K--B};
			\end{tikzpicture}
		}	
	}
\end{ex}
\begin{ex}%Câu 45.[Chuyên ĐH Vinh - Nghệ An -2020][1H3K5-2]
	Cho hình chóp $S.ABC$ có đáy là tam giác vuông tại $A,AC=a$, $I$ là trung điểm $SC$. Hình chiếu vuông góc của $S$ lên $(ABC)$ là trung điểm $H$ của $BC$. Mặt phẳng $(SAB)$ tạo với $(ABC)$ một góc $60^{\circ}$. Tính khoảng cách từ $I$ đến mặt phẳng $(SAB)$. 
	\choice
	{\True $\dfrac{\sqrt{3}a}{4}$}
	{$\dfrac{\sqrt{3}a}{5}$}
	{$\dfrac{\sqrt{5}a}{4}$}
	{$\dfrac{\sqrt{2}a}{3}$}
	\loigiai{
		\immini{
			Gọi $M$ là trung điểm cạnh $AB$ thì $MH$ là đường trung bình của tam giác $ABC$ nên $MH=\dfrac{a}{2},MH\parallel AC$\\
			$\Rightarrow MH\perp AB$.\\
			Mặt khác, do $SH\perp(ABC)$ nên $(SMH)\perp BC$.\\ Suy ra góc giữa $(SAB)$ và $(ABC)$ là góc giữa $SM$ và $MH$; lại có $SH\perp MH$ nên góc này bằng góc $\widehat{SMH}$.\\
			Từ giả thiết suy ra $\widehat{SMH}=60^{\circ}$.\\
			Gọi $K$ là hình chiếu của $H$ lên $SM$ thì $HK\perp(SAB)$.\\
			Xét tam giác vuông $SMH$ có \\
			$SH=MH\cdot\tan 60^{\circ}=\dfrac{a\sqrt{3}}{2},MH=\dfrac{a}{2}\Rightarrow HK=\dfrac{a\sqrt{3}}{4}$.
		}
		{
			\begin{tikzpicture}[scale=.8, font=\footnotesize, line join=round, line cap=round, >=stealth]
				\def\h{5} % đường cao
				\def\gocB{30} % góc B của đáy
				\coordinate[label=below left:$B$] (B) at (0,0);
				\coordinate[label= above right:$C$] (C) at (6,0);
				\coordinate[label=below left:$A$] (A) at (2,-2);
				\coordinate[label=above right :$H$] (H) at ($(C)!0.5!(B)$);
				
				\coordinate[label=above:$S$] (S) at ($(H)+(90:\h)$);
				\coordinate[label=above right :$I$] (I) at ($(S)!0.5!(C)$);
				\coordinate[label=below left:$M$] (M) at ($(A)!0.5!(B)$);
				\coordinate[label=right:$K$] (K) at ($(S)!0.7!(M)$);
				\draw (A)--(S)--(B)--(A)--(C)--(S)--(M);
				\draw[dashed] (B)--(C) (M)--(H)  (K)--(H)--(S) (I)--(H);
				\foreach \diem in {A,B,C,S,H,K,M,I}	\fill (\diem)circle(1.5pt);
				\pic[pic text= ,draw,thick,angle radius=2mm,angle eccentricity=1.5] {right angle = S--H--B};
				\pic[pic text= ,draw,thick,angle radius=2mm,angle eccentricity=1.5] {right angle = B--A--C};
				\pic[pic text= ,draw,thick,angle radius=2mm,angle eccentricity=1.5] {right angle = H--K--M};
				\pic[pic text= ,draw,thick,angle radius=2mm,angle eccentricity=1.5] {right angle = H--M--A};
			\end{tikzpicture}
		}
		\noindent	Gọi khoảng cách từ $I,C,H$ đến mặt phẳng $(SAB)$ lần lượt là $\mathrm{d}\left(I,(SAB)\right),\mathrm{d}\left(C,(SAB)\right),\mathrm{d}\left(H,(SAB)\right)$.\\
		\textbf{Cách 1.}
		Ta có $\heva{&\mathrm{d}\left(I,(SAB)\right)=\dfrac{1}{2}\mathrm{d}\left(C,(SAB)\right)\\&\mathrm{d}\left(H,(SAB)\right)=\dfrac{1}{2}\mathrm{d}\left(C,(SAB)\right)}\Rightarrow\mathrm{d}\left(I,(SAB)\right)=\mathrm{d}\left(H,(SAB)\right)=\dfrac{a\sqrt{3}}{4.}$ \\
		\textbf{	Cách 2.}
		$IH$ là đường trung bình của tam giác $SBC$ nên $IH\parallel SB\Rightarrow IH\parallel(SAB)$ \\
		$ \Rightarrow\mathrm{d}\left(I,(SAB)\right)=\mathrm{d}\left(H,(SAB)\right)=\dfrac{a\sqrt{3}}{4} $.}
\end{ex}
\begin{ex}%Câu 46.[Chuyên Hưng Yên - 2020][1H3K5-2]
	Cho hình chóp $S.ABC$ có đáy $ABC$ là tam giác cân, $BA=BC=a$ và $\widehat{BAC}=30^{\circ}$. Cạnh bên $SA$ vuông góc với mặt phẳng đáy và $SA=a$. Gọi $D$ là điểm đối xứng với $B$ qua $AC$. Khoảng cách từ $B$ đến mặt phẳng $(SCD)$ bằng
	\choice
	{$\dfrac{2a\sqrt{21}}{7}$}
	{$\dfrac{a\sqrt{2}}{2}$}
	{$\dfrac{a\sqrt{21}}{14}$}
	{\True $\dfrac{a\sqrt{21}}{7}$}
	\loigiai{
		\immini{
			Tam giác $ABC$ cân tại $B$ có $\widehat{BAC}=30^{\circ}$ và $D$ đối xứng với $B$ qua $AC$ nên tứ giác $ABCD$ là hình thoi có $\widehat{ADC}=\widehat{ABC}=120^{\circ}$.\\
			Trong mặt phẳng $(ABC)$, kẻ $AH$ vuông góc với đường thẳng $CD$ tại $H$. Khi đó $CD\perp AH$ và $CD\perp SA$ nên $CD\perp(SAH)$. Do đó $(SCD)\perp(SAH)$.\\
			Trong mặt phẳng $(SAH)$, kẻ $AK\perp SH$ tại $K$. Khi đó, $AK\perp(SCD)$ và $AK=\mathrm{d}\left( A,(SCD)\right) $.\\
			Ta có $AH=AD\cdot\sin 60^{\circ}=\dfrac{a\sqrt{3}}{2}$.\\
			Áp dụng hệ thức lượng trong tam giác vuông $SAH$, ta có $\dfrac{1}{AK^2}=\dfrac{1}{AH^2}+\dfrac{1}{SA^2}=\dfrac{7}{3a^2}$.\\
			Từ đó, $AK=\dfrac{a\sqrt{21}}{7}$.
		}
		{
			\begin{tikzpicture}[scale=.8, font=\footnotesize, line join=round, line cap=round, >=stealth]
				\def\h{5} % đường cao
				\def\gocB{30} % góc B của đáy
				\coordinate[label=below left:$A$] (A) at (0,0);
				\coordinate[label= above right:$C$] (C) at (6,0);
				\coordinate[label=below left:$B$] (B) at (4.2,-1.5);
				\coordinate[label=below:$D$] (D) at (1.8,1.5);
				\coordinate[label=above:$S$] (S) at ($(A)+(90:\h)$);
				\coordinate[label=above right:$H$] (H) at ($(C)!1.2!(D)$);
				\coordinate[label=right:$K$] (K) at ($(S)!0.7!(H)$);
				\draw (S)--(A)--(B)--(C)--(S)--(B);
				\draw[dashed] (D)--(A)--(C)--(H)--(S) (K)--(A)--(H);
				\foreach \diem in {A,B,C,S,D,H,K}	\fill (\diem)circle(1.5pt);
				\pic[pic text= ,draw,thick,angle radius=3mm,angle eccentricity=1.5] {right angle = H--K--A};
				\pic[pic text= ,draw,thick,angle radius=3mm,angle eccentricity=1.5] {right angle = A--H--D};
				\pic[pic text= ,draw,thick,angle radius=3mm,angle eccentricity=1.5] {right angle = S--A--C};
			\end{tikzpicture}
		}
		\noindent	Vì $AB\parallel(SCD)$ nên $\mathrm{d}\left( B,(SCD)\right) =\mathrm{d}\left( A,(SCD)\right) =AK=\dfrac{a\sqrt{21}}{7}$.}
\end{ex}
\begin{ex}%Câu 47.[Chuyên Lam Sơn - 2020][1H3K5-2]
	Cho hình chóp $S.ABCD$ có đáy $ABCD$ là hình thoi cạnh $a$. Tam giác $ABC$ là tam giác đều, hình chiếu vuông góc của đỉnh $S$ lên mặt phẳng $(ABCD)$ trùng với trọng tâm tam giác $ABC$. Góc giữa đường thẳng $SD$ và mặt phẳng $(ABCD)$ bằng $30^{\circ}$. Khoảng cách từ điểm $B$ đến mặt phẳng $(SCD)$ bằng 
	\choice
	{\True $\dfrac{a\sqrt{21}}{7}$}
	{$a\sqrt{3}$}
	{$a$}
	{$\dfrac{2a\sqrt{21}}{3}$}
	\loigiai{
		\immini{
			Gọi $H$ là trọng tâm tam giác $\triangle ABC$, $O$ là tâm của hình thoi $ABCD$.\\
			Do $SH\perp(ABCD)$: $\left(\widehat{SD,(ABCD)}\right)=\widehat{SDH}=30^{\circ}$.\\
			Xét tam giác $\triangle SDH$ vuông tại $H$ có\\ $\widehat{SDH}=30^{\circ}$;\\ $HD=\dfrac{2}{3}BD=\dfrac{4}{3}BO=\dfrac{4}{3}\cdot\dfrac{a\sqrt{3}}{2}=\dfrac{2a\sqrt{3}}{3}$.\\
			Mà $\dfrac{SH}{HD}=\tan\widehat{SDH}$\\
			$\Rightarrow SH=HD\cdot\tan\widehat{SDH}=\dfrac{2a\sqrt{3}}{3}\cdot\tan 30^{\circ}=\dfrac{2a}{3}$.\\
			Từ $H$ hạ $HI\perp SC$ tại $I$.
		}
		{
			\begin{tikzpicture}[scale=1,>=stealth, font=\footnotesize, line join=round, line cap=round]
				
				\coordinate[label=above right:$A$] (A) at (0,0);
				\coordinate[label=above right:$D$] (D) at (4,0);
				\coordinate[label=below:$B$] (B) at (-1.5,-2);
				\coordinate[label=right:$C$] (C) at ($(D)+(B)-(A)$);
				%\coordinate[label=left:$H$] (H) at ($(B)!0.5!(A)$);
				\coordinate[label=below:$O$] (O) at ($(C)!0.5!(A)$);
				
				\coordinate[label=below:$H$] (H) at ($(B)!0.66!(O)$);
				\coordinate[label=above:$S$] (S) at ($(H)+(0,4)$);
				\coordinate[label=right:$I$] (I) at ($(S)!0.65!(C)$);
				\draw (S)--(B)--(C)--(D)--(S)--(C);
				\draw[dashed] (C)--(A)--(S)--(H)--(C) (I)--(H) (D)--(B)--(A)--(D);
				\foreach \diem in {A,B,C,D,S,H,O,I}	\fill (\diem)circle(1.5pt);
				\pic[pic text= ,draw,thick,angle radius=2mm,angle eccentricity=1.5] {angle = S--D--O};
				\pic[pic text= ,draw,thick,angle radius=2mm,angle eccentricity=1.5] {right angle = A--B--C};
				\pic[pic text= ,draw,thick,angle radius=2mm,angle eccentricity=1.5] {right angle = S--H--B};
				\pic[pic text= ,draw,thick,angle radius=2mm,angle eccentricity=1.5] {right angle = H--I--C};
			\end{tikzpicture}
		}
		Ta có: $\heva{&HI\perp SC\\&HI\perp CD\left(CD\perp(SHC)\right)\\&SC,CD\subset(SCD)\\&SC\cap CD=\{C\}}\Rightarrow HI\perp(SCD)$.\\
		Từ đó, $\mathrm{d}\left(H,(SCD)\right)=HI$.\\
		Xét tam giác $\triangle SHC$ vuông tại $H$, đường cao $HI$:\\
		$HI=\dfrac{HS\cdot HC}{\sqrt{HS^2+HC^2}}=\dfrac{\dfrac{2a}{3}\cdot\dfrac{a\sqrt{3}}{3}}{\sqrt{\left(\dfrac{2a}{3}\right)^2+\left(\dfrac{a\sqrt{3}}{3}\right)^2}}=\dfrac{2a\sqrt{21}}{21}$.\\
		Mặt khác: $\dfrac{\mathrm{d}\left(B,(SCD)\right)}{\mathrm{d}\left(H,(SCD)\right)}=\dfrac{DB}{DH}=\dfrac{3}{2}$.\\
		Vậy
		$\mathrm{d}\left(B,(SCD)\right)=\dfrac{3}{2}\mathrm{d}\left(H,(SCD)\right)=\dfrac{3}{2}HI=\dfrac{3}{2}\cdot\dfrac{2a\sqrt{21}}{21}=\dfrac{a\sqrt{21}}{7}$.\\
		\textbf{Cách khác:}\\
		Thể tích khối chóp $S.BCD$ là\\
		$V_{S.BCD}=\dfrac{1}{3}SH\cdot S_{BCD}=\dfrac{1}{3}SH\cdot\dfrac{1}{2}\cdot CB\cdot CD\cdot\sin\widehat{BCD}=\dfrac{1}{3}\cdot\dfrac{2a}{3}\cdot\dfrac{1}{2}\cdot a\cdot a\cdot\dfrac{\sqrt{3}}{2}=\dfrac{a^3\sqrt{3}}{18}$ (đvtt).\\
		Xét tam giác $\triangle SCD$ có $SD=\dfrac{SH}{\sin 30^{\circ}}=\dfrac{4a}{3}; CD=a; SC=\sqrt{SH^2+HC^2}=\dfrac{a\sqrt{7}}{3}$.\\
		Diện tích tam giác $\triangle SCD$ là $S_{\triangle SCD}=\sqrt{p(p-SC)(p-SD)(p-CD)}=\dfrac{a^2\sqrt{7}}{6}$ (đvdt)\\
		(với $p=\dfrac{SC+SD+CD}{2}$ là nửa chu vi tam giác $\triangle SCD$).\\
		Vậy khoảng cách từ điểm $B$ đến mặt phẳng $(SCD)$ là\\
		$\mathrm{d}\left(B,(SCD)\right)=\dfrac{3\cdot V_{B.SCD}}{S_{\triangle SCD}}=\dfrac{3\cdot V_{S.BCD}}{S_{\triangle SCD}}=\dfrac{3\cdot\dfrac{a^3\sqrt{3}}{18}}{\dfrac{a^2\sqrt{7}}{6}}=\dfrac{a\sqrt{21}}{7}$.
	}
\end{ex}
\begin{ex}%Câu 48.[Chuyên Lương Văn Chánh - Phú Yên - 2020][1H3K5-2]
	Cho hình chóp $S.ABCD$ có đáy là hình vuông, $AB=a$, $SA$ vuông góc với mặt phẳng đáy và $SA=2a$ (minh họa như hình vẽ bên dưới). Gọi $M$ là trung điểm của $CD$, khoảng cách giữa điểm $M$ và mặt phẳng $(SBD)$ bằng
	\begin{center}
		\begin{tikzpicture}[scale=.8, font=\footnotesize, line join=round, line cap=round, >=stealth]
			\def\bc{5} % cạnh BC
			\def\ba{3} % cạnh BA
			\def\h{5} % đường cao
			\def\gocB{30} % góc B của đáy
			\coordinate[label=below left:$B$] (B) at (0,0);
			\coordinate[label=above left:$A$] (A) at (\gocB:\ba);
			\coordinate[label=below:$C$] (C) at (\bc,0);
			\coordinate[label=right:$D$] (D) at ($(C)-(B)+(A)$);
			\coordinate[label=above:$S$] (S) at ($(A)+(90:\h)$);
			\coordinate[label=below right:$M$] (M) at ($(D)!0.5!(C)$);
			\coordinate[label=above right:$O$] (O) at ($(A)!0.5!(C)$);
			\draw (B)--(C)--(D)--(S)--cycle (S)--(C);
			\draw[dashed] (A)--(D)--(B) (A)--(C) (O)--(S)--(A)--(B);
			\foreach \diem in {A,B,C,D,S,M,O}	\fill (\diem)circle(1.5pt);
			\pic[pic text= ,draw,thick,angle radius=2mm,angle eccentricity=1.5] {right angle = S--A--B};
		\end{tikzpicture}
	\end{center}
	\choice
	{$\dfrac{2a}{3}$}
	{$\dfrac{a}{\sqrt{2}}$}
	{$\dfrac{a}{2}$}
	{\True $\dfrac{a}{3}$}
	\loigiai{
		\immini{
			Ta có $\mathrm{d}(M,(SBD)=\dfrac{1}{2} \mathrm{d}(C,(SBD)=\dfrac{1}{2}\mathrm{d}(A,(SBD))$.\\
			Dựng $AH$ vuông góc với $SO$ tại $H$.\\
			Ta có
			$\heva{&BD\perp SA\\&BD\perp AO}\Rightarrow BD\perp (SAO)\Rightarrow BD\perp AH$.\\
			Mà $\heva{&AH\perp SO\\&AH\perp BD}\Rightarrow AH\perp (SBD)$.\\
			Do đó, $\mathrm{d}(A,(SBD))=AH$.\\
		}
		{
			\begin{tikzpicture}[scale=.6, font=\footnotesize, line join=round, line cap=round, >=stealth]
				\def\bc{5} % cạnh BC
				\def\ba{3} % cạnh BA
				\def\h{5} % đường cao
				\def\gocB{30} % góc B của đáy
				\coordinate[label=below left:$B$] (B) at (0,0);
				\coordinate[label=above left:$A$] (A) at (\gocB:\ba);
				\coordinate[label=below:$C$] (C) at (\bc,0);
				\coordinate[label=right:$D$] (D) at ($(C)-(B)+(A)$);
				\coordinate[label=above:$S$] (S) at ($(A)+(90:\h)$);
				\coordinate[label=below right:$M$] (M) at ($(D)!0.5!(C)$);
				\coordinate[label=above right:$O$] (O) at ($(A)!0.5!(C)$);
				\coordinate[label=above right:$H$] (H) at ($(S)!0.75!(O)$);
				\draw (B)--(C)--(D)--(S)--cycle (S)--(C);
				\draw[dashed]  (A)--(D)--(B) (H)--(A)--(C) (O)--(S)--(A)--(B);
				\foreach \diem in {A,B,C,D,S,M,O,H}	\fill (\diem)circle(1.5pt);
				\pic[pic text= ,draw,thick,angle radius=2mm,angle eccentricity=1.5] {right angle = S--A--B};
				\pic[pic text= ,draw,thick,angle radius=2mm,angle eccentricity=1.5] {right angle = A--H--O};
			\end{tikzpicture}
		}
		\noindent
		$\dfrac{1}{AH^2}=\dfrac{1}{AS^2}+\dfrac{1}{AB^2}+\dfrac{1}{AD^2}
		=\dfrac{1}{4a^2}+\dfrac{1}{a^2}+\dfrac{1}{a^2}=\dfrac{9}{4a^2}
		\Rightarrow AH=\dfrac{2a}{3}$.\\
		Vậy $\mathrm{d}(M,(SBD)=\dfrac{a}{3}$.}
\end{ex}
\begin{dang}
	{Khoảng cách đường thẳng với đường thẳng}	
\end{dang}
%%%dạng 2 còn thiếu lý thuyết và bài tập từ câu 1 tới câu 33 do TV3 và TV4 chưa nộp%%%
\begin{ex}%[1H3K5-4]%[2H3K4-1]%Câu 34
	[(Chuyên Lê Hồng Phong - Nam Định - 2020)]
	\immini{
		Cho tứ diện đều $ABCD$ có cạnh bằng $a$. Gọi $M$ là trung điểm của cạnh $AD$. Tính khoảng cách giữa hai đường thẳng $AB$ và $CM$. 
		\choice
		{$\dfrac{a\sqrt{33}}{11}$}
		{$\dfrac{a}{\sqrt{33}}$}
		{$\dfrac{a}{\sqrt{22}}$}
		{\True $\dfrac{a\sqrt{22}}{11}$} }{
		\begin{tikzpicture}[scale=.7]
			\coordinate (A) at (1.7,4);
			\coordinate (B) at (0,0);
			\coordinate (D) at (5,0);
			\coordinate (C) at (2.2,-1.5);
			\coordinate (M) at ($(A)!.5!(D)$);
			%		\draw (a) node[left]{$8$ cm};
			%		\coordinate (b) at ($(E)!.5!(G)$);
			%		\draw (b) node[below]{$10$ cm};
			%		\coordinate (c) at ($(B)!.5!(G)$);
			%		\draw (c) node[below right]{$5$ cm};
			\draw (M)--(C)--(A)--(B)--(C)--(D)--(A)--(M);
			%	\tkzDrawSegments(D,A A,B B,C B,E C,F D,E E,F F,D)
			\draw[dashed] (D)--(B);
			%		%	\draw ([yshift=-0.15cm]E) -- ([yshift=-0.15cm]F) node[midway, below]{6 cm};
			%		\draw pic[draw,angle radius=2mm] {right angle = F--E--G};
			\draw[fill=black] (A) node[above]{$A$} circle (1pt)  
			(B) circle (1pt) node[left]{$B$} 
			(D) circle (1pt) node[right]{$D$} 
			(M) circle (1pt) node[right]{$M$}		 
			(C) circle (1pt) node[below]{$C$}; 
		\end{tikzpicture}
	}
	\loigiai{
		Ta có: $V_{ABCD}=\dfrac{a^3\sqrt{3}}{12}$; $\dfrac{V_{ABCD}}{V_{ABCM}}=\dfrac{1}{2}\Rightarrow V_{ABCM}=\dfrac{a^3\sqrt{2}}{24}$.\\
		$V_{ABCM}=\dfrac{1}{6}AB\cdot CM\cdot\mathrm{d}(AB,CM)\cdot\sin (AB,CM)$.\\
		$\cos(AB,CM)=\dfrac{\left|\overrightarrow{AB}\cdot\overrightarrow{CM}\right|}{AB\cdot CM}=\dfrac{\left|\overrightarrow{AB}\cdot\left(\overrightarrow{AM}-\overrightarrow{AC}\right)\right|}{AB\cdot CM}=\dfrac{\left|\dfrac{a^2}{4}-\dfrac{a^2}{2}\right|}{a\cdot a\dfrac{\sqrt{3}}{2}}=\dfrac{\sqrt{3}}{6}$.\\
		$\Rightarrow \sin (AB,CM)=\sqrt{1-\dfrac{1}{12}}=\sqrt{\dfrac{11}{12}}$. Vậy $\mathrm{d}(AB,CM)=\dfrac{6V_{ABCM}}{AB\cdot CM\cdot\sin (AB,CM)}=\dfrac{a\sqrt{22}}{11}$.}
\end{ex}
\begin{ex}%[1H3K5-4]%Câu 35
	[Chuyên Phan Bội Châu - Nghệ An - 2020]
	\immini{
		Cho hình lăng trụ đều có tất cả các cạnh có độ dài bằng $2$ (tham khảo hình vẽ). Tính khoảng cách giữa hai đường thẳng $AC'$ và $A'B$. 
		\choice
		{\True $\dfrac{2}{\sqrt{5}}$}
		{$\dfrac{\sqrt{3}}{2}$}
		{$\dfrac{1}{\sqrt{2}}$}
		{$\dfrac{\sqrt{3}}{\sqrt{5}}$}}{
		\begin{tikzpicture}[scale=.7]
			\coordinate (A) at (1.3,-1);
			\coordinate (C) at (0,0);
			\coordinate (B) at (5,0);
			\coordinate (z) at (0,4.5);
			\coordinate (A) at (1.3,-1);
			\coordinate (C') at ($(C)+(z)$);
			\coordinate (B') at ($(B)+(z)$);
			\coordinate (A') at ($(A)+(z)$);
			\draw (A')--(B')--(C')--(A')--(A)--(B)--(B') (C)--(A)--(C')--(C) (A')--(B);
			%	\tkzDrawSegments(D,A A,B B,C B,E C,F D,E E,F F,D)
			\draw[dashed] (C)--(B);
			%		%	\draw ([yshift=-0.15cm]E) -- ([yshift=-0.15cm]F) node[midway, below]{6 cm};
			%		\draw pic[draw,angle radius=2mm] {right angle = F--E--G};
			\draw[fill=black] (A) node[below]{$A$} circle (1pt)  
			(B) circle (1pt) node[below]{$B$} 
			(C) circle (1pt) node[below]{$C$} 
			(A') circle (1pt) node[above]{$A'$}		 
			(B') circle (1pt) node[above]{$B'$}
			(C') circle (1pt) node[above]{$C'$};
		\end{tikzpicture}
		
	}
	\loigiai{
		\immini{
			Gọi $D$ là điểm đối xứng của $C$ qua $A$ ta có tứ giác $ADA'C'$ là hình bình hành do đó $A'D\parallel AC'$, suy ra khoảng cách $\mathrm{d}(AC',BA')=\mathrm{d}(AC',(A'BD))=\mathrm{d}(A,(A'BD))$.\\
			Theo giả thiết $ABC.A'B'C'$ là lăng trụ đều nên $AA'\perp (ABC)$ hay $AA'\perp (ABCD)$ suy ra $A'A\perp BD \quad(1)$.\\
			Ta có $\triangle ABD$ có $AB=AD$ nên là tam giác cân tại $A$, gọi $I$ là trung điểm $BD$ ta có $AI\perp BD\quad (2)$.\\
			Xét tam giác $\triangle BCD$ có $A,I$ lần lượt là trung điểm của $DC,DB$ nên $AI=\dfrac{1}{2}BC=1$.\\
			Trong mặt phẳng $(A'AI)$ dựng $AH\perp A'I;H\in A'I (3)$.\\
			Từ (1) và (2) suy ra $BD\perp (A'AI)\Rightarrow BD\perp AH (4)$.\\
		}{
			\begin{tikzpicture}[scale=.7]
				\coordinate (A) at (1.3,-1);
				\coordinate (C) at (0,0);
				\coordinate (B) at (5,0);
				\coordinate (z) at (0,4.5);
				\coordinate (A) at (1.3,-1);
				\coordinate (D) at ($(C)!2!(A)$);
				\coordinate (I) at ($(B)!0.5!(D)$);
				\coordinate (H) at ($(A')!0.6!(I)$);
				\coordinate (C') at ($(C)+(z)$);
				\coordinate (B') at ($(B)+(z)$);
				\coordinate (A') at ($(A)+(z)$);
				\draw (I)--(A')--(B')--(C')--(A')--(A) (A')--(D) (B)--(B') (C)--(A)--(C')--(C) (A')--(B)--(D)--(A);
				%	\tkzDrawSegments(D,A A,B B,C B,E C,F D,E E,F F,D)
				\draw[dashed] (C)--(B)--(A) (I)--(A)--(H);
				%		%	\draw ([yshift=-0.15cm]E) -- ([yshift=-0.15cm]F) node[midway, below]{6 cm};
				
				\draw[fill=black] (A) node[below]{$A$} circle (1pt)  
				(B) circle (1pt) node[below]{$B$} 
				(C) circle (1pt) node[below]{$C$} 
				(D) circle (1pt) node[below]{$D$} 
				(I) circle (1pt) node[below]{$I$} 
				(H) circle (1pt) node[above right]{$H$} 
				(A') circle (1pt) node[above]{$A'$}		 
				(B') circle (1pt) node[above]{$B'$}
				(C') circle (1pt) node[above]{$C'$};
				\draw pic[draw,angle radius=2mm] {right angle = A--H--I};
				\draw pic[draw,angle radius=2mm] {right angle = A--I--D};
			\end{tikzpicture}
		}
		\noindent Từ (3) và (4) suy ra $AH\perp (A'BD)$ do đó khoảng cách $\mathrm{d}(A,(SBD))=AH$.\\
		Trong tam giác $A'AI$ vuông tại $A$ ta có $AH=\dfrac{AI\cdot AA'}{\sqrt{AI^2+(AA')^2}}=\dfrac{2}{\sqrt{5}}$.
	}
\end{ex}
\begin{ex}%[1H3K5-4]%[2H3K4-1]%Câu 36.
	[Đại Học Hà Tĩnh - 2020] 
	Cho lăng trụ đứng $ABC.A'B'C'$ có đáy là tam giác vuông và $AB=BC=a$, $AA'=a\sqrt{2}$, $M$ là trung điểm của $BC$. Tính khoảng cách $d$ của hai đường thẳng $AM$ và $B'C$. 
	\choice
	{$d=\dfrac{a\sqrt{6}}{6}$}
	{$d=\dfrac{a\sqrt{2}}{2}$}
	{\True $d=\dfrac{a\sqrt{7}}{7}$}
	{$d=\dfrac{a\sqrt{3}}{3}$}
	\loigiai{
		\immini{
			\textbf{Cách 1.}
			Do $\triangle ABC$ vuông và có $AB=BC$ nên $\triangle ABC$ vuông cân tại $B$.\\
			Gọi $N$ là trung điểm của $BB'$, ta có $B'C\parallel(AMN)$.\\
			Khi đó\\
			$\mathrm{d}(AM,B'C)=\mathrm{d}\left(B'C,(AMN)\right)=\mathrm{d}\left(C,(AMN)\right)\break=\mathrm{d}\left(B,(AMN)\right)$.\\
			Kẻ $BH\perp AM$ tại $H$ và kẻ $BK\perp NH$ tại $K$.\\
			Ta có $BH\perp AM,BN\perp AM\Rightarrow AM\perp(NBH)\Rightarrow BK\perp AM$.\\
			Do $BK\perp NH$, $BK\perp AM$ nên $BK\perp(AMN)$.\\
			Suy ra $\mathrm{d}\left(B,(AMN)\right)=BK$.\\
			Mặt khác }{
			\begin{tikzpicture}[scale=.7]
				\coordinate (A) at (1.3,-1);
				\coordinate (B) at (0,0);
				\coordinate (C) at (5,0);
				\coordinate (z) at (0,4.5);
				\coordinate (C') at ($(C)+(z)$);
				\coordinate (B') at ($(B)+(z)$);
				\coordinate (A') at ($(A)+(z)$);
				\coordinate (M) at ($(B)!0.5!(C)$);
				\coordinate (N) at ($(B)!0.5!(B')$);
				\coordinate (H) at ($(A)!0.6!(M)$);
				\coordinate (K) at ($(N)!0.3!(H)$);
				\draw (A')--(B')--(C')--(A')--(A)--(B)--(B') (C)--(A) (C')--(C) (A)--(N);
				%	\tkzDrawSegments(D,A A,B B,C B,E C,F D,E E,F F,D)
				\draw[dashed] (C)--(B) (A)--(M) (K)--(B)--(H) (M)--(N)--(H) (B')--(C);
				%		%	\draw ([yshift=-0.15cm]E) -- ([yshift=-0.15cm]F) node[midway, below]{6 cm};
				%		\draw pic[draw,angle radius=2mm] {right angle = F--E--G};
				\draw[fill=black] (A) node[below]{$A$} circle (1pt)  
				(B) circle (1pt) node[below]{$B$} 
				(C) circle (1pt) node[below]{$C$} 
				(A') circle (1pt) node[above]{$A'$}		 
				(B') circle (1pt) node[left]{$B'$}
				(C') circle (1pt) node[above]{$C'$}
				(M) circle (1pt) node[above]{$M$}
				(N) circle (1pt) node[left]{$N$}
				(K) circle (1pt) node[above right]{$K$}
				(H) circle (1pt) node[right]{$H$};
				\draw pic[draw,angle radius=2mm] {right angle = A--B--C};
				\draw pic[draw,angle radius=2mm] {right angle = B--K--H};
			\end{tikzpicture}
		}
		\noindent
		$BH=\dfrac{BM\cdot BA}{\sqrt{BM^2+BA^2}}=\dfrac{\sqrt{5}a}{5}$; $BK=\dfrac{BH\cdot BN}{\sqrt{BH^2+BN^2}}=\dfrac{a\sqrt{7}}{7}$.\\
		Vậy $\mathrm{d}(AM,B'C)=\mathrm{d}\left(B,(AMN)\right)=BK=\dfrac{a\sqrt{7}}{7}$.\\
		\textbf{Cách 2.}
		\immini{
			Do $\triangle ABC$ vuông và có $AB=BC$ nên $\triangle ABC$ vuông cân tại $B$.\\
			Chọn hệ trục tọa độ $Oxyz$ như hình vẽ. Không mất tính tổng quát, ta giả sử $a=1$.\\
			Ta có $A(1;0;0), M\left(0;\dfrac{1}{2};0\right), B'\left(0;0;\sqrt{2}\right), C(0;1;0)$.\\
			$\overrightarrow{AM}=\left(-1;\dfrac{1}{2};0\right),$
			$\overrightarrow{B'C}=\left(0;1;-\sqrt{2}\right),$$\\
			\overrightarrow{AC}=(-1;1;0)$ $ \Rightarrow \left[\overrightarrow{AM},\overrightarrow{B'C}\right]=\left(-\dfrac{\sqrt{2}}{2};-\sqrt{2};-1\right)$.\\
		}{
			
			\begin{tikzpicture}[scale=.7]
				\coordinate (A) at (1.3,-1);
				\coordinate (B) at (0,0);
				\coordinate (C) at (5,0);
				\coordinate (z) at (0,4.5);
				\coordinate (C') at ($(C)+(z)$);
				\coordinate (B') at ($(B)+(z)$);
				\coordinate (A') at ($(A)+(z)$);
				\coordinate (M) at ($(B)!0.5!(C)$);
				\coordinate (N) at ($(B)!0.5!(B')$);
				\coordinate (z) at ($(B)!1.2!(B')$);
				\coordinate (x) at ($(B)!1.2!(A)$);
				\coordinate (y) at ($(B)!1.2!(C)$);
				\draw (A')--(B')--(C')--(A')--(A)--(B)--(B') (C)--(A) (C')--(C) (A)--(N);
				\draw[->] (B')--(z) ;
				\draw[->] (A)--(x) ;
				\draw[->] (C)--(y);
				%	\tkzDrawSegments(D,A A,B B,C B,E C,F D,E E,F F,D)
				\draw[dashed] (C)--(B) (A)--(M)   (B')--(C);
				%		%	\draw ([yshift=-0.15cm]E) -- ([yshift=-0.15cm]F) node[midway, below]{6 cm};
				%		\draw pic[draw,angle radius=2mm] {right angle = F--E--G};
				\draw[fill=black] (A) node[below]{$A$} circle (1pt)  
				(B) circle (1pt) node[below left=0.1]{$O\equiv B$} 
				(C) circle (1pt) node[below]{$C$} 
				(A') circle (1pt) node[above]{$A'$}		 
				(B') circle (1pt) node[left]{$B'$}
				(C') circle (1pt) node[above]{$C'$}
				(M) circle (1pt) node[above]{$M$}
				(N) circle (1pt) node[left]{$N$}
				(z)  node[left]{$z$} 
				(y)  node[right]{$y$}
				(x)  node[right]{$x$}
				;
				\draw pic[draw,angle radius=2mm] {right angle = A--B--C};
				
			\end{tikzpicture}
			
		}
		\noindent	Khi đó
		$\mathrm{d}(AM,B'C)=\dfrac{\left|\left[\overrightarrow{AM},\overrightarrow{B'C}\right]\cdot\overrightarrow{AC}\right|}{\left|\left[\overrightarrow{AM},\overrightarrow{B'C}\right]\right|}=\dfrac{\left|\dfrac{\sqrt{2}}{2}-\sqrt{2}+0\right|}{\sqrt{\dfrac{1}{2}+2+1}}=\dfrac{\sqrt{7}}{7}$.\\
		Trong trường hợp tổng quát, ta có $\mathrm{d}(AM,B'C)=\dfrac{a\sqrt{7}}{7}$.	
	}
\end{ex}
\begin{ex}%[1H3K5-4]%[2H3K4-1]%Câu 37.
	[ĐHQG Hà Nội - 2020] 
	Cho lăng trụ đứng $ABC.A'B'C'$ có tất cả các cạnh bằng a. Gọi $M$ là trung điểm của $AA'$. Tính khoảng cách giữa hai đường thẳng $BM$ và $B'C$. 
	\choice
	{$\dfrac{\sqrt{3}}{\sqrt{5}}a$}
	{\True $\dfrac{\sqrt{3}}{\sqrt{10}}a$}
	{$\dfrac{\sqrt{3}}{2\sqrt{2}}a$}
	{$\dfrac{\sqrt{3}}{\sqrt{7}}a$}
	\loigiai{
		\immini{
			Gọi $O$ và $I$ lần lượt là trung điểm của $B'C'$, $BC$. Chọn hệ trục tọa độ $Oxyz$ sao cho\\
			$O(0;0;0);A'\in Ox\Rightarrow A'\left(\dfrac{-\sqrt{3}}{2};0;0\right);C'\in Oy\Rightarrow C'\left(0;\dfrac{1}{2};0\right);$ $ I\in Oz\Rightarrow I(0;0;1)$ \\
			$ \Rightarrow B'\left(0;\dfrac{-1}{2};0\right)$; $C\left(0;\dfrac{1}{2};1\right)$; $B\left(0;\dfrac{-1}{2};1\right)$; $A\left(\dfrac{-\sqrt{3}}{2};0;1\right)$; $M\left(\dfrac{-\sqrt{3}}{2};0;\dfrac{1}{2}\right)$ \\
			$ \Rightarrow\overrightarrow{B'C}=(0;1;1);$ $\overrightarrow{BM}=\left(\dfrac{-\sqrt{3}}{2};\dfrac{1}{2};\dfrac{-1}{2}\right)$ $\Rightarrow\left[\overrightarrow{BM},\overrightarrow{B'C}\right]=\left(1;\dfrac{\sqrt{3}}{2};\dfrac{-\sqrt{3}}{2}\right)$ \\
			$ \Rightarrow\overrightarrow{BC}=(0;1;0) $.\\
			Vậy khoảng cách giữa hai đường thẳng $BM$ và $B'C$ là\\
			$\mathrm{d}(BM;B'C)=\dfrac{\left|\overrightarrow{BC\cdot}\left[\overrightarrow{BM},\overrightarrow{B'C}\right]\right|}{\left|\left[\overrightarrow{BM},\overrightarrow{B'C}\right]\right|}=\dfrac{\dfrac{\sqrt{3}}{2}}{\sqrt{1+\dfrac{3}{4}+\dfrac{3}{4}}}=\dfrac{\sqrt{3}}{\sqrt{10}}$. }{
			\begin{tikzpicture}[scale=.7]
				\coordinate (B) at (1.3,1);
				\coordinate (A) at (0,0);
				\coordinate (C) at (5,0);
				\coordinate (z) at (0,-4.5);
				\coordinate (C') at ($(C)+(z)$);
				\coordinate (B') at ($(B)+(z)$);
				\coordinate (A') at ($(A)+(z)$);
				\coordinate (M) at ($(A)!0.5!(A')$);
				\coordinate (O) at ($(C')!0.5!(B')$);
				\coordinate (I) at ($(C)!0.5!(B)$);
				\coordinate (z) at ($(O)!1.3!(I)$);
				\coordinate (x) at ($(A')!2!(O)$);
				\coordinate (y) at ($(B')!1.3!(C')$);
				\coordinate (U) at ($(A)!0.5!(B)$);
				\coordinate (V) at ($(A')!0.5!(B')$);
				\coordinate (W) at (intersection of O--x and C--C');
				\draw[->] (I)--(z) ;
				\draw[->] (W)--(x) ;
				\draw[->] (C')--(y);
				\draw (A)--(B)--(C)--(A)--(A')--(C')--(C);
				\draw[dashed] (W)--(A')--(B')--(C') (B')--(B)   (O)--(I) (B)--(M);
				%		%	\draw ([yshift=-0.15cm]E) -- ([yshift=-0.15cm]F) node[midway, below]{6 cm};
				%		\draw pic[draw,angle radius=2mm] {right angle = F--E--G};
				\draw[fill=black] (A) node[left]{$A$} circle (1pt)  
				(B) circle (1pt) node[above]{$B$} 
				(C) circle (1pt) node[right]{$C$} 
				(A') circle (1pt) node[below]{$A'$}		 
				(B') circle (1pt) node[above left]{$B'$}
				(C') circle (1pt) node[below]{$C'$}
				(M) circle (1pt) node[left]{$M$}
				(O) circle (1pt) node[below]{$O$}
				(I) circle (1pt) node[above right]{$I$}
				(U) node[above]{$a$}
				(V)  node[above]{$a$}
				(z)  node[left]{$z$} 
				(y)  node[right]{$y$}
				(x)  node[right]{$x$}
				;
				%	\draw pic[draw,angle radius=2mm] {right angle = A--B--C};
				
			\end{tikzpicture}	
		}
		
	}
\end{ex}
\begin{ex}%[1H3K5-4]%[2H3K4-1]%Câu 38.
	[Sở Phú Thọ - 2020] 
	Cho hình chóp $S.ABCD$ có đáy $ABCD$ là hình vuông cạnh $a$. Hình chiếu vuông góc của $S$ trên mặt phẳng $(ABCD)$ là trung điểm của cạnh $AB$, góc giữa mặt phẳng $(SAC)$ và đáy bằng $45^{\circ}$. Gọi $M$ là trung điểm của cạnh $SD$. Khoảng cách giữa hai đường $AM$ và $SC$ bằng
	\choice
	{$a$}
	{$\dfrac{a\sqrt{2}}{4}$}
	{$\dfrac{a\sqrt{5}}{10}$}
	{\True $\dfrac{a\sqrt{5}}{5}$}
	\loigiai{
		\immini{
			Gọi $H$ là trung điểm cạnh $AB$, $I$ là trung điểm cạnh $AO$. Suy ra\\
			$SH\perp(ABCD)$, $\widehat{(SAC),(ABCD)}=\widehat{SIH}=45^{\circ}$. \\
			Do đó $SH=IH=\dfrac{1}{2}BO=\dfrac{a\sqrt{2}}{4}$.\\
			Gọi $N$ là trung điểm cạnh $CD$, khi đó $HN\perp AB$.\\
			Chọn hệ trục tọa độ trong không gian như hình vẽ bên, ta có tọa độ các điểm.\\
			$H(0;0;0)$, $A\left(0;-\dfrac{a}{2};0\right)$; $S\left(0;0;\dfrac{a\sqrt{2}}{4}\right)$; $D\left(a;-\dfrac{a}{2};0\right)$; $M\left(\dfrac{a}{2};-\dfrac{a}{4};\dfrac{a\sqrt{2}}{8}\right)$; $C\left(a;\dfrac{a}{2};0\right)$.\\
		}{
			\begin{tikzpicture}[scale=.7]
				\coordinate (A) at (0,0);
				\coordinate (D) at (5,0);
				\coordinate (B) at (-2,-1);
				\coordinate (C) at (3,-1);
				\coordinate (S) at (-1,4.5);
				\coordinate (H) at ($(A)!0.5!(B)$);
				\coordinate (O) at ($(A)!0.5!(C)$);
				\coordinate (N) at ($(C)!0.5!(D)$);
				\coordinate (M) at ($(S)!0.5!(D)$);
				\coordinate (I) at ($(A)!0.5!(O)$);
				
				
				\coordinate (z) at ($(H)!1.3!(S)$);
				\coordinate (x) at ($(H)!1.3!(N)$);
				\coordinate (y) at ($(A)!1.3!(B)$);
				
				%\coordinate (W) at (intersection of O--x and C--C');
				\draw[->] (S)--(z) ;
				\draw[->] (B)--(y) ;
				\draw[->] (N)--(x);
				\draw (C)--(S)--(B)--(C)--(D)--(S);
				\draw[dashed] (S)--(A)--(B)--(D)--(A)--(C) (H)--(S)--(I)--(H)--(N)  (A)--(M);
				%		%	\draw ([yshift=-0.15cm]E) -- ([yshift=-0.15cm]F) node[midway, below]{6 cm};
				%		\draw pic[draw,angle radius=2mm] {right angle = F--E--G};
				\draw[fill=black] (A) node[left]{$A$} circle (1pt)  
				(B) circle (1pt) node[above left]{$B$} 
				(C) circle (1pt) node[below]{$C$} 
				(D) circle (1pt) node[right]{$D$}		 
				(N) circle (1pt) node[below]{$N$}
				(I) circle (1pt) node[above]{$I$}
				(M) circle (1pt) node[right]{$M$}
				(O) circle (1pt) node[below]{$O$}
				(H) circle (1pt) node[above left]{$H$}
				(S) circle (1pt) node[above right]{$S$}
				(z)  node[left]{$z$} 
				(y)  node[below]{$y$}
				(x)  node[right]{$x$}
				;
				\draw pic[draw,angle radius=2mm] {right angle = S--H--B};
				
			\end{tikzpicture}	
			
		}
		\noindent Nên $\overrightarrow{AM}=\left(\dfrac{a}{2};\dfrac{a}{4};\dfrac{a\sqrt{2}}{8}\right);\overrightarrow{SC}=\left(a;\dfrac{a}{2};-\dfrac{a\sqrt{2}}{4}\right);\overrightarrow{AC}=(a;a;0)$.\\
		Khoảng cách giữa hai đường $AM$ và $SC$ là 
		$\mathrm{d}(AM,SC)=\dfrac{\left|\left[\overrightarrow{AM},\overrightarrow{SC}\right]\cdot\overrightarrow{AC}\right|}{\left|\left[\overrightarrow{AM},\overrightarrow{SC}\right]\right|}=\dfrac{a\sqrt{5}}{5}$.	
	}
\end{ex}
\begin{ex}%[1H3K5-4]%[2H3K4-1]%Câu 39.
	[Sở Hà Tĩnh - 2020] 
	Cho tứ diện $ABCD$ có $AB,AC,AD$ đôi một vuông góc với nhau và $AD=2,AB=AC=1$. Gọi $I$ là trung điểm của đoạn thẳng $BC$, khoảng cách giữa hai đường thẳng $AI$ và $BD$ bằng
	\choice
	{$\dfrac{3}{2}$}
	{$\dfrac{2}{\sqrt{5}}$}
	{$\dfrac{\sqrt{5}}{2}$}
	{\True $\dfrac{2}{3}$}
	\loigiai{
		\immini{
			Vì tứ diện $ABCD$ có $AB,AC,AD$ đôi một vuông góc với nhau, nên ta chọn hệ trục tọa độ $Axyz$ như hình vẽ (với $A$ là gốc tọa độ, đường thằng $AC$ nằm trên trục $Ax$, $AD$ nằm trên trục $Ay$ và $AB$ nằm trên trục $Az$).\\
			Từ đó suy ra: $A(0;0;0)$, $B(0;0;1)$ vì $B\in Az$, $C(1;0;0)$ vì $C\in Ax$, $D(0; 2;0)$ vì $D\in Ay$.\\
			Vì $I$ là trung điểm của $BC$ nên $I\left(\dfrac{1}{2};0;\dfrac{1}{2}\right)$.\\
			Từ đó, ta quay về bài toán tính khoảng cách giữa hai đường thẳng chéo nhau trong hệ tọa độ không gian $Axyz$.\\
		}{
			\begin{tikzpicture}[scale=.7]
				\coordinate (A) at (0,0);
				\coordinate (B) at (0,4.5);
				\coordinate (C) at (-2,-1);
				\coordinate (D) at (5,0);
				\coordinate (I) at ($(B)!0.5!(C)$);
				\coordinate (X) at ($(A)!0.5!(C)$);
				\coordinate (Y) at ($(A)!0.5!(D)$);
				\coordinate (Z) at ($(A)!0.5!(B)$);
				
				
				
				\coordinate (z) at ($(A)!1.3!(B)$);
				\coordinate (x) at ($(A)!1.5!(C)$);
				\coordinate (y) at ($(A)!1.3!(D)$);
				
				%\coordinate (W) at (intersection of O--x and C--C');
				\draw[->] (B)--(z) ;
				\draw[->] (C)--(x) ;
				\draw[->] (D)--(y);
				\draw (B)--(C)--(D)--(B);
				\draw[dashed] (C)--(A)--(D)   (I)--(A)--(B);
				%		%	\draw ([yshift=-0.15cm]E) -- ([yshift=-0.15cm]F) node[midway, below]{6 cm};
				%		\draw pic[draw,angle radius=2mm] {right angle = F--E--G};
				\draw[fill=black] 
				(A) circle (1pt)  node[left=0.2]{$A$} 
				(B) circle (1pt) node[above left]{$B$} 
				(C) circle (1pt) node[below right]{$C$} 
				(D) circle (1pt) node[below right]{$D$}		 
				(I) circle (1pt) node[above left]{$I$}
				
				(z)  node[left]{$z$} 
				(y)  node[below]{$y$}
				(x)  node[right]{$x$}
				
				(X)  node[above]{$1$} 
				(Y)  node[above]{$1$}
				(Z)  node[right]{$2$}
				;
				\draw pic[draw,angle radius=2mm] {right angle = B--A--D};
				\draw pic[draw,angle radius=2mm] {right angle = D--A--C};
				\draw pic[draw,angle radius=2mm] {right angle = C--A--B};
				\draw ($(I)!0.5!(B)$)node[rotate=40]{$/$} ($(I)!0.5!(C)$)node[rotate=40]{$/$} ;
			\end{tikzpicture}	
			
		}
		\noindent	Ta có $\overrightarrow{AI}=\left(\dfrac{1}{2};0;\dfrac{1}{2}\right),\overrightarrow{BD}=(0;2;-1)\Rightarrow\left[\overrightarrow{AI},\overrightarrow{BD}\right]=\left(-1;\dfrac{1}{2};1\right)$ và $\overrightarrow{AB}=(0;0;1)$.\\
		Ta có $\mathrm{d}(AI,BD)=\dfrac{\left|\left[\overrightarrow{AI},\overrightarrow{BD}\right]\cdot\overrightarrow{AB}\right|}{\left|\left[\overrightarrow{AI},\overrightarrow{BD}\right]\right|}=\dfrac{\left|(-1)\cdot 0+\dfrac{1}{2}\cdot 0+1\cdot 1\right|}{\sqrt{(-1)^2+\left(\dfrac{1}{2}\right)^2+1^2}}=\dfrac{2}{3}$.
	}
\end{ex}
\begin{ex}%[1H3K5-4]%Câu 40.
	[Sở Yên Bái - 2020] 
	Cho hình lăng trụ đứng $ABC{.}A'B'C'$ có đáy là tam giác vuông cân tại $B$, biết $AB=BC=a$, $AA'=a\sqrt{2}$, $M$ là trung điểm của $BC$. Tính khoảng cách giữa hai đường thẳng $AM$ và $B'C$. 
	\choice
	{\True $\dfrac{a\sqrt{7}}{7}$}
	{$\dfrac{2a\sqrt{5}}{5}$}
	{$\dfrac{a\sqrt{6}}{2}$}
	{$\dfrac{a\sqrt{15}}{5}$}
	\loigiai{
		\begin{center}
			\begin{tikzpicture}[scale=.7]
				\coordinate (A) at (0,0);
				\coordinate (B) at (2,-1);
				\coordinate (C) at (5,0);
				\coordinate (z) at (0,5);
				\coordinate (A') at ($(A)+(z)$);
				\coordinate (B') at ($(B)+(z)$);
				\coordinate (C') at ($(C)+(z)$);
				
				\coordinate (M) at ($(B)!0.5!(C)$);
				
				
				%		%	\draw ([yshift=-0.15cm]E) -- ([yshift=-0.15cm]F) node[midway, below]{6 cm};
				%		\draw pic[draw,angle radius=2mm] {right angle = F--E--G};
				\draw (A')--(A)--(B)--(C)--(C')--(B')--(A')--(C') (B)--(B')--(C);
				\draw[dashed] (C)--(A)--(M) ;
				\draw[fill=black] 
				(A) circle (1pt)  node[left]{$A$} 
				(B) circle (1pt) node[below]{$B$} 
				(C) circle (1pt) node[right]{$C$} 
				
				(A') circle (1pt)  node[left]{$A'$} 
				(B') circle (1pt) node[above]{$B'$} 
				(C') circle (1pt) node[right]{$C'$} 
				
				(M) circle (1pt) node[below]{$M$} 
				;
				%		\draw pic[draw,angle radius=2mm] {right angle = B--A--D};
				%		\draw pic[draw,angle radius=2mm] {right angle = D--A--C};
				%		\draw pic[draw,angle radius=2mm] {right angle = C--A--B};
				%		\draw ($(I)!0.5!(B)$)node[rotate=40]{$/$} ($(I)!0.5!(C)$)node[rotate=40]{$/$} ;
			\end{tikzpicture}
			\begin{tikzpicture}[scale=.7]
				\coordinate (B) at (0,0);
				\coordinate (A) at (0,4.5);
				\coordinate (B') at (2,-1);
				\coordinate (C) at (5,0);
				\coordinate (M) at ($(B)!0.5!(C)$);
				\coordinate (N) at ($(B)!0.5!(B')$);
				
				
				\draw (N)--(A)--(B)--(B')--(C)--(A)--(B');
				\draw[dashed] (B)--(C)   (A)--(M)--(N);
				%		%	\draw ([yshift=-0.15cm]E) -- ([yshift=-0.15cm]F) node[midway, below]{6 cm};
				%		\draw pic[draw,angle radius=2mm] {right angle = F--E--G};
				\draw[fill=black] 
				(A) circle (1pt)  node[left]{$A$} 
				(B) circle (1pt) node[left]{$B$} 
				(C) circle (1pt) node[right]{$C$} 
				(B') circle (1pt) node[below]{$B'$}		 
				(M) circle (1pt) node[above right]{$M$}
				(N) circle (1pt) node[below]{$N$}
				;
				\draw pic[draw,angle radius=2mm] {right angle = A--B--M};
				\draw pic[draw,angle radius=2mm] {right angle = M--B--N};
				\draw pic[draw,angle radius=2mm] {right angle = N--B--A};
				%	\draw ($(I)!0.5!(B)$)node[rotate=40]{$/$} ($(I)!0.5!(C)$)node[rotate=40]{$/$} ;
			\end{tikzpicture}
		\end{center}
		Kẻ $MN \parallel B'C' \Rightarrow B'C \parallel (AMN)$\\ $\Rightarrow \mathrm{d} = \mathrm{d}(B'C,MN)= \mathrm{d}(B'C,(AMN))= \mathrm{d}(C,(AMN))= \mathrm{d}(B,(AMN))$.\\
		Ta có tứ diện $BAMN$ là tứ diện vuông nên\\
		$\dfrac{1}{ \mathrm{d}^2}=\dfrac{1}{BA^2}+\dfrac{1}{BM^2}+\dfrac{1}{BN^2}=\dfrac{1}{a^2}+\dfrac{1}{\left(\dfrac{a}{2}\right)^2}+\dfrac{1}{\left(\dfrac{a\sqrt{2}}{2}\right)^2}=\dfrac{7}{a^2}$\\
		$\Rightarrow d = \dfrac{a\sqrt{7}}{7}$.
	}
\end{ex}
\begin{ex}%[1H3K5-4]%Câu 41.
	[Đặng Thúc Hứa - Nghệ An - 2020] 
	Cho hình chóp tứ giác đều $S.ABCD$ có cạnh đáy bằng $2a$, cạnh $SA$ tạo với mặt phẳng đáy một góc $30^{\circ}$. Khoảng cách giữa hai đường thẳng $SA$ và $CD$ bằng
	\choice
	{$\dfrac{2\sqrt{15}a}{5}$}
	{$\dfrac{3\sqrt{14}a}{5}$}
	{\True $\dfrac{2\sqrt{10}a}{5}$}
	{$\dfrac{4\sqrt{5}a}{5}$}
	\loigiai{
		\immini{
			Gọi $O$ là tâm của mặt đáy, $M$ là trung điểm của $AB$, $H$ là hình chiếu của $O$ trên $SM$.\\
			Ta có $\left(SA,(ABCD)\right)=(SA,OA)=\widehat{SAO}$\\
			$\Rightarrow\widehat{SAO}=30^{\circ}\Rightarrow SO=AO\tan 30^{\circ}=\dfrac{a\sqrt{2}}{\sqrt{3}}$.\\
			Ta có $AB\perp OM,AB\perp SO\Rightarrow AB\perp(SOM)$\\
			$\Rightarrow AB\perp OH$, mà $SM\perp OH\Rightarrow OH\perp(SAB)$.\\
			Tam giác $SOM$ vuông tại $O$ và có đường cao $OH$ nên $\dfrac{1}{OH^2}=\dfrac{1}{SO^2}+\dfrac{1}{OM^2}=\dfrac{3}{2a^2}+\dfrac{1}{a^2}=\dfrac{5}{2a^2}$\\
			$\Rightarrow OH=\dfrac{\sqrt{10}a}{5}$.\\
		}{
			\begin{tikzpicture}[scale=.7]
				\coordinate (S) at (0,5);
				\coordinate (O) at (0,0);
				\coordinate (D) at (-1,1);
				\coordinate (A) at (4,1);
				\coordinate (B) at ($(D)!2!(O)$);
				\coordinate (C) at ($(A)!2!(O)$);
				\coordinate (M) at ($(A)!0.5!(B)$);
				\coordinate (H) at ($(S)!0.4!(M)$);
				
				
				\draw (S)--(A)--(B)--(C)--(S)--(A) (M)--(S)--(B);
				\draw[dashed] (C)--(D)--(A)--(C)   (O)--(S)--(D)--(B) (H)--(O)--(M);
				%		%	\draw ([yshift=-0.15cm]E) -- ([yshift=-0.15cm]F) node[midway, below]{6 cm};
				%		\draw pic[draw,angle radius=2mm] {right angle = F--E--G};
				\draw[fill=black] 
				(A) circle (1pt)  node[right]{$A$} 
				(B) circle (1pt) node[right]{$B$} 
				(C) circle (1pt) node[left]{$C$} 
				(D) circle (1pt) node[below]{$D$}		 
				(O) circle (1pt) node[below]{$O$}
				(M) circle (1pt) node[right]{$M$}
				(H) circle (1pt) node[right]{$H$}
				(S) circle (1pt) node[right]{$S$}
				;
				\draw pic[draw,angle radius=2mm] {right angle = O--H--M};
				
				%	\draw ($(I)!0.5!(B)$)node[rotate=40]{$/$} ($(I)!0.5!(C)$)node[rotate=40]{$/$} ;
			\end{tikzpicture}	
		}
		\noindent Vì $CD\parallel AB$\\
		$\Rightarrow\mathrm{d}(CD,SA)=\mathrm{d}\left(CD,(SAB)\right)=\mathrm{d}\left(C,(SAB)\right)=2\mathrm{d}\left(O,(SAB)\right)=2OH=\dfrac{2\sqrt{10}a}{5}$.
	}
\end{ex}
\begin{ex}%[1H3K5-4]%Câu 42.
	[Kim Liên - Hà Nội - 2020]
	\immini{
		Cho hình chóp $S.ABC$ có đáy là tam giác đều cạnh $a$, $SA$ vuông góc với mặt phẳng đáy, góc giữa mặt phẳng $(SBC)$ và mặt phẳng đáy là $60^{\circ}$ (minh họa như hình dưới đây). Gọi $M, N$ lần lượt là trung điểm của $AB, AC$. 
		Khoảng cách giữa hai đường thẳng $SB$ và $MN$ bằng
		\choice
		{\True $\dfrac{3a}{8}$}
		{$\dfrac{a\sqrt{6}}{2}$}
		{$\dfrac{3a}{4}$}
		{$a\sqrt{6}$}}{
		\begin{tikzpicture}[scale=.7]
			\coordinate (A) at (0,0);
			\coordinate (S) at (0,4.5);
			\coordinate (B) at (2,-1);
			\coordinate (C) at (5,0);
			\coordinate (M) at ($(B)!0.5!(A)$);
			\coordinate (N) at ($(A)!0.5!(C)$);
			
			
			\draw (S)--(A)--(B)--(C)--(S)--(B);
			\draw[dashed] (A)--(C)  (M)--(N);
			%		%	\draw ([yshift=-0.15cm]E) -- ([yshift=-0.15cm]F) node[midway, below]{6 cm};
			%		\draw pic[draw,angle radius=2mm] {right angle = F--E--G};
			\draw[fill=black] 
			(A) circle (1pt)  node[left]{$A$} 
			(B) circle (1pt) node[below left]{$B$} 
			(C) circle (1pt) node[right]{$C$} 
			(S) circle (1pt) node[above]{$S$}		 
			(M) circle (1pt) node[below left]{$M$}
			(N) circle (1pt) node[above]{$N$}
			;
			%	\draw pic[draw,angle radius=2mm] {right angle = A--B--M};
			%	\draw pic[draw,angle radius=2mm] {right angle = M--B--N};
			%	\draw pic[draw,angle radius=2mm] {right angle = N--B--A};
			%	\draw ($(I)!0.5!(B)$)node[rotate=40]{$/$} ($(I)!0.5!(C)$)node[rotate=40]{$/$} ;
		\end{tikzpicture}
	}
	\loigiai{
		\immini{
			Gọi $E$ là trung điểm của $BC,$ vì tam giác $ABC$ đều\\ $\Rightarrow AE\perp BC$, lại có $SA\perp BC\Rightarrow BC\perp SE$.\\
			Mặt khác $(SBC)\cap(ABC)=BC$\\
			$\Rightarrow\left((SBC),(ABC)\right)=\widehat{SEA}=60^{\circ}$.\\
			Gọi $P$ là trung điểm của $SA$\\
			$\Rightarrow SB\parallel MP, MP\subset(MNP)\Rightarrow SB\parallel (MNP)$. Dẫn đến \\
			$ \mathrm{d}(SB,MN)=\mathrm{d}\left(SB,(MNP)\right)=\mathrm{d}\left(B,(MNP)\right)=\mathrm{d}\left(A,(MNP)\right) $.\\
			Gọi $AE\cap MN=I\Rightarrow\widehat{PIA}=\widehat{SEA}=60^{\circ}$ và $AI\perp MN$.\\
		}{
			\begin{tikzpicture}[scale=.7]
				\coordinate (A) at (0,0);
				\coordinate (S) at (0,4.5);
				\coordinate (B) at (1.5,-1);
				\coordinate (C) at (5,0);
				\coordinate (M) at ($(B)!0.5!(A)$);
				\coordinate (N) at ($(A)!0.5!(C)$);
				\coordinate (E) at ($(B)!0.5!(C)$);
				\coordinate (P) at ($(S)!0.5!(A)$);
				\coordinate (I) at ($(M)!0.5!(N)$);
				\coordinate (H) at ($(P)!0.4!(I)$);
				
				
				\draw (E)--(S)--(A)--(B)--(C)--(S)--(B) (P)--(M);
				\draw[dashed] (H)--(A)--(C)  (M)--(N)--(P)--(I) (A)--(E);
				%		%	\draw ([yshift=-0.15cm]E) -- ([yshift=-0.15cm]F) node[midway, below]{6 cm};
				%		\draw pic[draw,angle radius=2mm] {right angle = F--E--G};
				\draw[fill=black] 
				(A) circle (1pt)  node[left]{$A$} 
				(B) circle (1pt) node[below left]{$B$} 
				(C) circle (1pt) node[right]{$C$} 
				(S) circle (1pt) node[above]{$S$}		 
				(M) circle (1pt) node[below left]{$M$}
				(N) circle (1pt) node[above]{$N$}
				(P) circle (1pt) node[left]{$P$}
				(E) circle (1pt) node[below]{$E$}
				(I) circle (1pt) node[below]{$I$}
				(H) circle (1pt) node[above right]{$H$}
				;
				\draw pic[draw,angle radius=2mm] {right angle = P--H--I};
				\draw pic[draw,angle radius=2mm] {right angle = A--E--C};
				\draw pic[draw,angle radius=2mm] {right angle = S--E--C};
				\draw pic[draw,angle radius=2mm] {right angle = S--A--B};
				\draw pic[draw,angle radius=2mm] {right angle = S--A--C};
				\draw pic[draw,angle radius=2mm] {right angle = A--H--I};
				%	\draw ($(I)!0.5!(B)$)node[rotate=40]{$/$} ($(I)!0.5!(C)$)node[rotate=40]{$/$} ;
			\end{tikzpicture}	
			
		}
		\noindent	Ta có $MN\perp AI, MN\perp PI$
		$\Rightarrow MN\perp(API)\Rightarrow(PMN)\perp(API)$.\\
		Mà $(PMN)\cap(API)=PI$, kẻ $AH\perp PI$
		$\Rightarrow AH\perp(PMN)\Rightarrow\mathrm{d}\left(A,(PMN)\right)=AH$.\\
		Xét $\triangle API$ có $\widehat{AIP}=60^{\circ}, AI=\dfrac{1}{2}AE=\dfrac{a\sqrt{3}}{4}$
		$\Rightarrow AH=AI\cdot\sin\widehat{AIP}=\dfrac{a\sqrt{3}}{4}\cdot\dfrac{\sqrt{3}}{2}=\dfrac{3a}{8}$.\\
		Vậy $\mathrm{d}(SB,MN)=\dfrac{3a}{8}$.	
	}
\end{ex}
\begin{ex}%[1H3K5-4]%[2H3K4-1]%Câu 43.
	[Liên trường Nghệ An - 2020]
	Cho tứ diện $ABCD$ có $\widehat{ABC}=\widehat{ADC}=\widehat{ACD}=90^{\circ},BC=2a,CD=a$, góc giữa đường thẳng $AB$ và mặt phẳng $(BCD)$ bằng $60^{\circ}$. Tính khoảng cách giữa hai đường thẳng $AC$ và $BD$. 
	\choice
	{$\dfrac{a\sqrt{6}}{\sqrt{31}}$}
	{$\dfrac{2a\sqrt{6}}{\sqrt{31}}$}
	{\True $\dfrac{2a\sqrt{3}}{\sqrt{31}}$}
	{$\dfrac{a\sqrt{3}}{\sqrt{31}}$}
	\loigiai{
		\immini{
			Gọi $H$ là chân đường cao của tứ diện $ABCD$.\\
			Ta có: $\heva{&BC\perp AB\\&BC\perp AH}\Rightarrow BC\perp HB (1)$.\\
			Lại có: $\heva{&CD\perp AD\\&CD\perp AH}\Rightarrow CD\perp HD (2)$.\\
			Mà $\widehat{BCD}=90^{\circ}$.\\
			Từ đây ta suy ra $HBCD$ là hình chữ nhật.\\
			Mặt khác $\left(\widehat{AB,(BCD)}\right)=\widehat{ABH}=60^{\circ}$.\\
			Suy ra $AH=HB\cdot\tan 60^{\circ}=a\sqrt{3}$.\\
			Chọn hệ trục $Oxyz\equiv H\cdot DBA$ như hình vẽ.\\
		}{
			\begin{tikzpicture}[scale=.7]
				\coordinate (H) at (0,0);
				\coordinate (A) at (0,4.5);
				\coordinate (D) at (-2,-1);
				\coordinate (B) at (5,0);
				\coordinate (C) at ($(D)+(B)$);
				
				
				
				\coordinate (z) at ($(H)!1.3!(A)$);
				\coordinate (x) at ($(H)!1.5!(D)$);
				\coordinate (y) at ($(H)!1.3!(B)$);
				\coordinate (t) at ($(H)!0.8!(B)$);
				
				\coordinate (U) at ($(D)!0.5!(C)$);
				\coordinate (V) at ($(C)!0.5!(B)$);
				%\coordinate (W) at (intersection of O--x and C--C');
				\draw[->] (A)--(z) ;
				\draw[->] (D)--(x) ;
				\draw[->] (B)--(y);
				\draw (A)--(D)--(C)--(B)--(A)--(C);
				\draw[dashed] (A)--(H)--(D)--(B)--(H)--(C) ;
				%		%	\draw ([yshift=-0.15cm]E) -- ([yshift=-0.15cm]F) node[midway, below]{6 cm};
				%		\draw pic[draw,angle radius=2mm] {right angle = F--E--G};
				\draw[fill=black] 
				(A) circle (1pt)  node[left]{$A$} 
				(B) circle (1pt) node[above right]{$B$} 
				(C) circle (1pt) node[below right]{$C$} 
				(D) circle (1pt) node[below right]{$D$}		 
				(H) circle (1pt) node[above left]{$H$}
				
				(z)  node[left]{$z$} 
				(y)  node[below]{$y$}
				(x)  node[right]{$x$}
				(t)  node[above]{$60^\circ$}
				(U) node[below]{$a$}
				(V) node[right]{$2a$}
				;
				\draw pic[draw,angle radius=2mm] {right angle = A--H--D};
				\draw pic[draw,angle radius=2mm] {right angle = A--H--B};
				\draw pic[draw,angle radius=2mm] {right angle = B--H--D};
				\draw pic[draw,angle radius=2mm] {right angle = B--H--D};
				\draw pic[draw,angle radius=2mm] {right angle = H--D--C};
				\draw pic[draw,angle radius=2mm] {right angle = D--C--B};
				\draw pic[draw,angle radius=2mm] {right angle = C--B--H};
				\draw pic[draw,angle radius=3mm] {angle = S--B--H};
				%\draw ($(I)!0.5!(B)$)node[rotate=40]{$/$} ($(I)!0.5!(C)$)node[rotate=40]{$/$} ;
			\end{tikzpicture}		
		}
		\noindent	Ta có $H(0; 0; 0)$, $A\left(0; 0; a\sqrt{3}\right)$, $B(0; a; 0)$, $C(2a;a; 0)$, $D(2a; 0; 0)$.\\
		$\overrightarrow{AC}=\left(2a; a;-a\sqrt{3}\right)$, $\overrightarrow{BD}=(2a;-a;0)$, $\overrightarrow{AB}=\left(0; a;-a\sqrt{3}\right)$.\\
		Vậy $\mathrm{d}(AC,BC)=\dfrac{\left|\left[\overrightarrow{AC},\overrightarrow{BD}\right]\cdot\overrightarrow{AB}\right|}{\left|\left[\overrightarrow{AC},\overrightarrow{BD}\right]\right|}=\dfrac{2a^3\sqrt{3}}{\sqrt{\left(-a^2\sqrt{3}\right)^2+\left(-2a^2\sqrt{3}\right)^2+\left(-4a^2\right)^2}}=\dfrac{2a\sqrt{93}}{31}$.
	}
\end{ex}
\begin{ex}%[1H3K5-4]%Câu 44.
	[Lý Nhân Tông - Bắc Ninh - 2020]
	Cho tứ diện $OABC$ có $OA,OB,OC$ đôi một vuông góc với nhau và $OA=OB=a$, $OC=2a$. Gọi $M$ là trung điểm của $AB$. Khoảng cách giữa hai đường thẳng $OM$ và $AC$ bằng
	\choice
	{$\dfrac{\sqrt{2}a}{3}$}
	{$\dfrac{2\sqrt{5}a}{5}$}
	{$\dfrac{\sqrt{2}a}{2}$}
	{\True $\dfrac{2a}{3}$}
	\loigiai{
		\begin{center}
			\begin{tikzpicture}[scale=.7]
				\coordinate (O) at (0,0);
				\coordinate (C) at (0,4.5);
				\coordinate (A) at (3,-1);
				\coordinate (B) at (5,0);
				\coordinate (M) at ($(A)!0.5!(B)$);
				\coordinate (E) at ($(O)!-1!(B)$);
				\coordinate (K) at ($(E)!0.5!(A)$);
				\coordinate (H) at ($(C)!0.6!(K)$);
				
				
				\coordinate (W) at (intersection of E--B and C--K);
				\draw (W)--(E)--(A)--(B)--(C)--(K) (C)--(A);
				\draw[dashed] (C)--(O)--(K) (M)--(O)--(B) (W)--(O)--(H) ;
				%		%	\draw ([yshift=-0.15cm]E) -- ([yshift=-0.15cm]F) node[midway, below]{6 cm};
				%		\draw pic[draw,angle radius=2mm] {right angle = F--E--G};
				\draw[fill=black] 
				(A) circle (1pt)  node[below]{$A$} 
				(B) circle (1pt) node[right]{$B$} 
				(C) circle (1pt) node[above]{$C$} 
				(O) circle (1pt) node[above right]{$O$}		 
				(H) circle (1pt) node[above left]{$H$}
				(M) circle (1pt) node[below]{$M$}
				(K) circle (1pt) node[below]{$K$}
				(E) circle (1pt) node[below]{$E$}
				
				;
				\draw pic[draw,angle radius=2mm] {right angle = C--O--B};
				\draw pic[draw,angle radius=2mm] {right angle = C--O--A};
				\draw pic[draw,angle radius=2mm] {right angle = B--O--A};
				\draw pic[draw,angle radius=2mm] {right angle = O--H--K};
				\draw ($(M)!0.5!(B)$)node[rotate=40]{$/$} ($(M)!0.5!(A)$)node[rotate=40]{$/$} ;
			\end{tikzpicture}	
		\end{center}
		Dựng $AE\parallel OM$, khi đó $OM\parallel(CAE)$. Do đó $\mathrm{d}(OM,AC)=\mathrm{d}(OM,(CAE))=\mathrm{d}(O,(CAE))$.\\
		Dựng $OK\perp AE$, ta có\\
		$\heva{&AE\perp OK\\&AE\perp OC\left(\text{Vì } CO\perp(ABC)\right)}\Rightarrow AE\perp(COK)$.\\
		Mà $AE\subset(CAE)$ nên $(CAE)\perp(COK)$.\\
		Ta có $(CAE)\cap(COK)=CK$. Kẻ $OH\perp CK$, khi đó $OH\perp(COK)$.\\ Suy ra $\mathrm{d}(O,(CAE))=OH$.\\
		Xét tam giác $OAB$ ta có $AB=\sqrt{OA^2+OB^2}=a\sqrt{2}$.\\
		Dễ thấy $OKAM$ là hình chữ nhật nên $OK=AM=\dfrac{AB}{2}=\dfrac{a\sqrt{2}}{2}$.\\
		Xét tam giác $COK$ ta có\\
		$\dfrac{1}{OH^2}=\dfrac{1}{OK^2}+\dfrac{1}{OC^2}\Rightarrow\dfrac{1}{OH^2}=\dfrac{1}{\left(\dfrac{a\sqrt{2}}{2}\right)^2}+\dfrac{1}{(2a)^2}\Rightarrow OH=\dfrac{2}{3}a$.
		
	}
\end{ex}
\begin{ex}%[1H3K5-4]%Câu 45.
	[Nguyễn Huệ - Phú Yên - 2020]
	Cho hình chóp $S.ABC$ có đáy là tam giác vuông tại $A$, $AB=a, AC=2a, SA$ vuông góc với mặt phẳng đáy và $SA=2a$. Gọi $G$ là trọng tâm của $\triangle ABC$. Khoảng cách giữa hai đường thẳng $SG$ và $BC$ bằng
	\choice
	{\True $\dfrac{2a}{7}$}
	{$\dfrac{a\sqrt{6}}{3}$}
	{$\dfrac{2a\sqrt{6}}{9}$}
	{$\dfrac{4a}{7}$}
	\loigiai{
		\immini{
			Gọi $M$ là trung điểm của $BC$. Trong mp $(SAM)$ dựng $S'M\parallel SG$. Suy ra $S'A=\dfrac{3}{2}SA=3a$.\\
			Do đó $\mathrm{d}(SG,BC)=\mathrm{d}\left(SG,(S'BC)\right)=\mathrm{d}\left(G,(S'BC)\right)$.\\
			Vì $AM=3GM$ nên $\mathrm{d}\left(G,(S'BC)\right)=\dfrac{1}{3}\mathrm{d}\left(A,(S'BC)\right)$.\\
			Kẻ $AH\perp BC$ ta có $BC\perp(S'AH)$.\\
			Kẻ $AK\perp S'H\Rightarrow AK=\mathrm{d}\left(A,(S'BC)\right)$.\\
			Ta có $\dfrac{1}{AH^2}=\dfrac{1}{AB^2}+\dfrac{1}{AC^2}\Rightarrow AH=\dfrac{2a}{\sqrt{5}}$. \\
			Suy ra $\dfrac{1}{AK^2}=\dfrac{1}{S'A^2}+\dfrac{1}{AH^2}\Rightarrow AK=\dfrac{6a}{7}$.\\
			Do đó $\mathrm{d}\left(G,(S'BC)\right)=\dfrac{1}{3}AK=\dfrac{2a}{7}$.}{
			
			\begin{tikzpicture}[scale=.7]
				\coordinate (A) at (0,0);
				\coordinate (S) at (0,4.5);
				\coordinate (B) at (2,-1);
				\coordinate (C) at (7,0);
				\coordinate (M) at ($(B)!0.5!(C)$);
				\coordinate (G) at ($(A)!0.66667!(M)$);
				\coordinate (S') at ($(A)!1.5!(S)$);
				\coordinate (H) at ($(B)!0.23333!(C)$);
				\coordinate (K) at ($(S')!0.6!(H)$);
				
				%	\coordinate (W) at (intersection of E--B and C--K);
				\draw (B)--(S')--(A)--(B)--(C)--(S')--(M) (S')--(H) (S)--(B);
				\draw[dashed] (A)--(H) (A)--(K) (M)--(A)--(C)--(S)--(G) ;
				%		%	\draw ([yshift=-0.15cm]E) -- ([yshift=-0.15cm]F) node[midway, below]{6 cm};
				%		\draw pic[draw,angle radius=2mm] {right angle = F--E--G};
				\draw[fill=black] 
				(A) circle (1pt)  node[below]{$A$} 
				(B) circle (1pt) node[below right]{$B$} 
				(C) circle (1pt) node[right]{$C$} 
				(M) circle (1pt) node[below]{$M$}
				(H) circle (1pt) node[below]{$H$}		 
				(G) circle (1pt) node[above right]{$G$}
				(K) circle (1pt) node[right]{$K$}
				(S) circle (1pt) node[left]{$S$}
				(S') circle (1pt) node[left]{$S'$}
				
				;
				\draw pic[draw,angle radius=2mm] {right angle = A--K--H};
				\draw pic[draw,angle radius=2mm] {right angle = A--H--B};
				\draw pic[draw,angle radius=2mm] {right angle = S'--A--C};
				\draw pic[draw,angle radius=2mm] {right angle = B--A--C};
				\draw ($(M)!0.5!(B)$)node[rotate=40]{$/$} ($(M)!0.5!(C)$)node[rotate=40]{$/$} ;
			\end{tikzpicture}
			
		}
	}
\end{ex}
\begin{ex}%[1H3K5-4]%Câu 46.
	[Nguyễn Trãi - Thái Bình - 2020]
	Cho hình chóp $S.ABCD$ có đáy $ABCD$ là hình bình hành và $SA=SB=SC=11,$ góc $\angle SAB=30^{\circ},$ góc $\angle SBC=60^{\circ},$ góc $\angle SCA=45^{\circ}$. Tính khoảng cách $d$ giữa hai đường thẳng $AB$ và $SD$. 
	\choice
	{$2\sqrt{22}$}
	{\True $\sqrt{22}$}
	{$\dfrac{\sqrt{22}}{2}$}
	{$4\sqrt{11}$}
	\loigiai{
		\begin{center}
			\begin{tikzpicture}[scale=.7]
				\coordinate (A) at (0,0);
				\coordinate (C) at (6,0);
				\coordinate (D) at (0,-2);
				\coordinate (B) at ($(C)-(D)$);
				\coordinate (J) at ($(A)!(C)!(B)$);
				\coordinate (I) at ($(A)!0.5!(B)$);
				\coordinate (K) at ($(D)!(I)!(C)$);
				
				%	\coordinate (W) at (intersection of E--B and C--K);
				\draw (B)--(A)--(D)--(C)--(A)--(C)--(B) (I)--(K) (J)--(C) ;
				%	\draw[dashed] (D)--(A)--(S)--(I) (J)--(C)--(A)--(B) (H)--(I)--(K) ;
				%		%	\draw ([yshift=-0.15cm]E) -- ([yshift=-0.15cm]F) node[midway, below]{6 cm};
				%		\draw pic[draw,angle radius=2mm] {right angle = F--E--G};
				\draw[fill=black] 
				(A) circle (1pt)  node[left]{$A$} 
				(B) circle (1pt) node[right]{$B$} 
				(C) circle (1pt) node[right]{$C$} 
				(K) circle (1pt) node[below]{$K$}
				(J) circle (1pt) node[above]{$J$}
				(K) circle (1pt) node[below]{$K$}
				(I) circle (1pt) node[above]{$I$}
				(D) circle (1pt) node[left]{$D$}
				
				;
				\draw pic[draw,angle radius=2mm] {right angle = I--K--C};
				\draw pic[draw,angle radius=2mm] {right angle = C--J--B};
				%	\draw ($(M)!0.5!(B)$)node[rotate=40]{$/$} ($(M)!0.5!(A)$)node[rotate=40]{$/$} ;
			\end{tikzpicture}
			\begin{tikzpicture}[scale=.7]
				\coordinate (A) at (0,0);
				\coordinate (B) at (6,0);
				\coordinate (I) at (3,0);
				\coordinate (D) at (-2,-1);
				\coordinate (S) at (3,5);
				
				\coordinate (C) at ($(D)+(B)$);
				\coordinate (J) at ($(A)!0.8!(B)$);
				\coordinate (K) at ($(D)!0.7!(C)$);
				\coordinate (H) at ($(K)!0.4!(S)$);
				
				%	\coordinate (W) at (intersection of E--B and C--K);
				\draw (S)--(D)--(C)--(B)--(S)--(K) (S)--(C) ;
				\draw[dashed] (D)--(A)--(S)--(I) (J)--(C)--(A)--(B) (H)--(I)--(K) ;
				%		%	\draw ([yshift=-0.15cm]E) -- ([yshift=-0.15cm]F) node[midway, below]{6 cm};
				%		\draw pic[draw,angle radius=2mm] {right angle = F--E--G};
				\draw[fill=black] 
				(A) circle (1pt)  node[left]{$A$} 
				(B) circle (1pt) node[right]{$B$} 
				(C) circle (1pt) node[below]{$C$} 
				(K) circle (1pt) node[below]{$K$}
				(H) circle (1pt) node[left]{$H$}		 
				(J) circle (1pt) node[above]{$J$}
				(K) circle (1pt) node[below]{$K$}
				(S) circle (1pt) node[left]{$S$}
				(I) circle (1pt) node[below right]{$I$}
				(D) circle (1pt) node[below]{$D$}
				;		;
				\draw pic[draw,angle radius=2mm] {right angle = S--I--B};
				\draw pic[draw,angle radius=2mm] {right angle = I--H--K};
				\draw pic[draw,angle radius=2mm] {right angle = C--J--B};
				\draw pic[draw,angle radius=2mm] {right angle = I--K--C};
				%	\draw ($(M)!0.5!(B)$)node[rotate=40]{$/$} ($(M)!0.5!(A)$)node[rotate=40]{$/$} ;
			\end{tikzpicture}
		\end{center}
		Trong tam giác $\triangle SAB$ ta có $SB^2=SA^2+AB^2-2SA\cdot AB\cdot\cos 30^{\circ}\Leftrightarrow AB=11\sqrt{3}$.\\
		Trong tam giác $\triangle SBC$ ta có $SB=SC=11,\angle SBC=60^{\circ}$ nên $\triangle SBC$ đều suy ra $BC=11$.\\
		Trong tam giác $\triangle SCA$ ta có $SC=SA=11,\angle SCA=45^{\circ}$ nên $\triangle SCA$ vuông cân tại $S$ suy ra $AC=11\sqrt{2}$.\\
		Xét tam giác $ABC$ có $BC^2+AC^2=AB^2$ do vậy $\triangle ABC$ vuông tại $C$.\\
		Gọi $I$ là hình chiếu của $S$ lên mặt phẳng $(ABCD)$ vì $SA=SB=SC$ nên $I$ là tâm của đường tròn ngoại tiếp tam giác $ABC$, vì $\triangle ABC$ vuông tại $C$ nên $I$ là trung điểm của $AB$ và $SI\perp (ABCD)\Rightarrow SI\perp CD (1)$. Vẽ $IK\perp CD\quad  (2),IH\perp SK \quad(3)$.\\
		Từ (1) và (2) suy ra $CD\perp (SIK)\Rightarrow CD\perp IH \quad(4)$.\\
		Từ (3) và (4) suy ra $IH\perp (SCD)$ do đó khoảng cách $\mathrm{d}(I,(SCD))=IH$.\\
		Ta lại có $AB\parallel CD$ suy ra khoảng cách $\mathrm{d}(AB,SD)=\mathrm{d}(AB,(SCD))=\mathrm{d}(I,(SCD))=IH$.\\
		Trong mặt phẳng đáy vẽ $CJ\perp AB$ ta suy ra $IK=CJ=\dfrac{CA\cdot CB}{AB}=\dfrac{11\sqrt{6}}{3}$.\\
		Trong tam giác $SAB$ cân tại $S$ có $SI=\sqrt{SA^2-\dfrac{AB^2}{4}}=\dfrac{11}{2}$.\\
		Trong tam giác $SIK$ vuông tại $I$ ta có $IH=\dfrac{IK\cdot SI}{\sqrt{IK^2+SI^2}}=\sqrt{22}$.}
\end{ex}
\begin{ex}%[1H3K5-4]%Câu 47.
	[Tiên Du - Bắc Ninh - 2020] Cho hình lăng trụ tam giác $ABC.A'B'C'$ có cạnh bên bằng $a\sqrt{2}$, đáy $ABC$ là tam giác vuông tại $B,BC=a\sqrt{3},AB=a$. Biết hình chiếu vuông góc của đỉnh $A'$ lên mặt đáy là điểm $M$ thoả mãn $3\overrightarrow{AM}=\overrightarrow{AC}$. Khoảng cách giữa hai đường thẳng $AA'$ và $BC$ bằng
	\choice
	{\True $\dfrac{a\sqrt{210}}{15}$}
	{$\dfrac{a\sqrt{210}}{45}$}
	{$\dfrac{a\sqrt{714}}{17}$}
	{$\dfrac{a\sqrt{714}}{51}$}
	\loigiai{
		\immini{
			Dựng hình bình hành $ABCD$, vì tam giác $ABC$ là tam giác vuông tại $B$ nên $ABCD$ là hình chữ nhật.\\
			Suy ra $BC\parallel AD\Rightarrow BC\parallel(A'AD)$.\\
			Do đó $\mathrm{d}(BC,AA')=\mathrm{d}\left(BC,(A'AD)\right)=\mathrm{d}\left(C,(A'AD)\right)$.\\
			Mà $3\overrightarrow{AM}=\overrightarrow{AC}$ nên $\mathrm{d}\left(C,(A'AD)\right)=3\mathrm{d}\left(M,(A'AD)\right)$.\\
			Kẻ $MH\perp AD\Rightarrow(A'MH)\perp(A'AD)=A'H$.\\
			Kẻ $MK\perp A'H\Rightarrow MK\perp(A'AD)$\\
			$\Rightarrow MK=\mathrm{d}\left(M,(A'AD)\right)$.\\
			Mặt khác ta có $AC=\sqrt{AB^2+BC^2}=2a$\\
			$\Rightarrow AM=\dfrac{1}{3}AC=\dfrac{2a}{3}$.\\
			$\Rightarrow A'M=\sqrt{A'A^2-AM^2}=\dfrac{a\sqrt{14}}{3}$.\\
		}{
			\begin{tikzpicture}[scale=.7]
				\coordinate (A) at (0,0);
				\coordinate (C) at (6,0);
				\coordinate (M) at (2,0);
				\coordinate (B) at (1.5,-1);
				\coordinate (A') at (1,5);
				\coordinate (B') at ($(B)+(A')$);
				\coordinate (C') at ($(C)+(A')$);
				\coordinate (D) at ($(C)-(B)$);
				
				
				\coordinate (H) at ($(A)!0.333!(D)$);
				\coordinate (K) at ($(A')!0.7!(H)$);
				
				
				%	\coordinate (W) at (intersection of E--B and C--K);
				\draw (A')--(B')--(C')--(A')--(A)--(B)--(C)--(C') (B)--(B') ;
				\draw[dashed] (A)--(C)--(D)--(A')--(H)--(M)--(A') (K)--(M) (A)--(D);
				%		%	\draw ([yshift=-0.15cm]E) -- ([yshift=-0.15cm]F) node[midway, below]{6 cm};
				%		\draw pic[draw,angle radius=2mm] {right angle = F--E--G};
				\draw[fill=black] 
				(A) circle (1pt)  node[below]{$A$} 
				(B) circle (1pt) node[below]{$B$} 
				(C) circle (1pt) node[below]{$C$} 
				(K) circle (1pt) node[left]{$K$}
				(H) circle (1pt) node[above left]{$H$}		 
				(M) circle (1pt) node[below]{$M$}
				(A') circle (1pt) node[above]{$A'$} 
				(B') circle (1pt) node[above]{$B'$} 
				(C') circle (1pt) node[above]{$C'$} 
				(D) circle (1pt) node[right]{$D$} 
				;		
				\draw pic[draw,angle radius=2mm] {right angle = A'--M--C};
				\draw pic[draw,angle radius=2mm] {right angle = M--H--D};
				\draw pic[draw,angle radius=2mm] {right angle = M--K--A'};
				\draw pic[draw,angle radius=2mm] {right angle = A--B--C};
				%	\draw ($(M)!0.5!(B)$)node[rotate=40]{$/$} ($(M)!0.5!(A)$)node[rotate=40]{$/$} ;
			\end{tikzpicture}
		}	
		\noindent Và $MH\parallel CD\Rightarrow\dfrac{MH}{CD}=\dfrac{AM}{AC}=\dfrac{1}{3}\Rightarrow MH=\dfrac{1}{3}CD=\dfrac{1}{3}AB=\dfrac{a}{3}$.\\
		Suy ra
		\begin{eqnarray*}
			\dfrac{1}{MK^2}=\dfrac{1}{A'M^2}+\dfrac{1}{MH^2}&\Leftrightarrow &\dfrac{1}{MK^2}=\dfrac{1}{\left(\dfrac{a\sqrt{14}}{3}\right)^2}+\dfrac{1}{\left(\dfrac{a}{3}\right)^2}\\
			&\Leftrightarrow &\dfrac{1}{MK^2}=\dfrac{135}{14a^2}\Leftrightarrow MK=\dfrac{a\sqrt{210}}{45}.
		\end{eqnarray*}
		
		Vậy $\mathrm{d}(BC,AA')=\mathrm{d}\left(C,(A'AD)\right)=3\mathrm{d}\left(M,(A'AD)\right)=3MK=3\dfrac{a\sqrt{210}}{45}=\dfrac{a\sqrt{210}}{15}$.
	}
\end{ex}
\begin{ex}%[1H3K5-4]%Câu 48.
	[Hải Hậu - Nam Định - 2020]
	Cho hình chóp đều $S.ABCD$ có đáy $ABCD$ là hình vuông cạnh $a\sqrt{2}$. Biết rằng bán kính mặt cầu ngoại tiếp hình chóp bằng $\dfrac{9a\sqrt{2}}{8}$, độ dài cạnh bên lớn hơn độ dài cạnh đáy. Khoảng cách giữa hai đường thẳng $AB$ và $SD$ bằng
	\choice
	{$\dfrac{2a\sqrt{17}}{17}$}
	{$\dfrac{4a\sqrt{17}}{17}$}
	{$\dfrac{4a\sqrt{34}}{17}$}
	{\True $\dfrac{2a\sqrt{34}}{17}$}
	\loigiai{
		\immini{
			Gọi $O=AC\cap BD$, $M$ là trung điểm $SC$.\\
			Trong tam giác $SAC$, dựng đường trung trực của đoạn thẳng $SC$ cắt $SO$ tại $I$, $I$ là tâm mặt cầu ngoại tiếp hình chóp $S.ABCD$, bán kính $R=SI=\dfrac{9a\sqrt{2}}{8}$.\\
			Vì độ dài cạnh bên lớn hơn độ dài cạnh đáy nên tâm $I$ của mặt cầu ngoại tiếp hình chóp thuộc đoạn $SO$.\\
			Gọi $x$ là độ dài cạnh bên của hình chóp.\\
			Ta có $\triangle SOC$ đồng dạng với $\triangle SMI$.\\
		}{
			
			\begin{tikzpicture}[scale=.7]
				\coordinate (A) at (0,0);
				\coordinate (D) at (5,0);
				\coordinate (B) at (2,1.5);
				\coordinate (z) at (0,6);
				\coordinate (C) at ($(D)+(B)$);
				\coordinate (O) at ($(A)!0.5!(C)$);
				\coordinate (S) at ($(O)+(z)$);
				\coordinate (E) at ($(D)!0.5!(C)$);
				\coordinate (M) at ($(S)!0.5!(C)$);
				\coordinate (I) at ($(S)!0.6!(O)$);
				\coordinate (H) at ($(S)!0.6!(E)$);
				
				
				%	\coordinate (W) at (intersection of E--B and C--K);
				\draw (S)--(A)--(D)--(C)--(S)--(D) (S)--(E) ;
				\draw[dashed] (A)--(B)--(C)--(A) (D)--(B)--(S)--(O) (H)--(O)--(E) (I)--(M);
				%		%	\draw ([yshift=-0.15cm]E) -- ([yshift=-0.15cm]F) node[midway, below]{6 cm};
				%		\draw pic[draw,angle radius=2mm] {right angle = F--E--G};
				\draw[fill=black] 
				(A) circle (1pt)  node[below]{$A$} 
				(B) circle (1pt) node[above left]{$B$} 
				(C) circle (1pt) node[right]{$C$} 
				(E) circle (1pt) node[right]{$E$}
				(H) circle (1pt) node[below right]{$H$}		 
				(M) circle (1pt) node[above right]{$M$}
				(O) circle (1pt) node[below]{$O$} 
				(I) circle (1pt) node[left]{$I$} 
				(D) circle (1pt) node[below]{$D$}
				(S) circle (1pt) node[above]{$S$}
				;		
				%	\draw pic[draw,angle radius=2mm] {right angle = A'--M--C};
				%	\draw pic[draw,angle radius=2mm] {right angle = M--H--D};
				%	\draw pic[draw,angle radius=2mm] {right angle = M--K--A'};
				%	\draw pic[draw,angle radius=2mm] {right angle = A--B--C};
				%	\draw ($(M)!0.5!(B)$)node[rotate=40]{$/$} ($(M)!0.5!(A)$)node[rotate=40]{$/$} ;
			\end{tikzpicture}
			
		}
		Suy ra
		\begin{eqnarray*}
			\dfrac{SI}{SC}=\dfrac{SM}{SO}& \Leftrightarrow & \dfrac{\dfrac{9a\sqrt{2}}{8}}{x}=\dfrac{\dfrac{x}{2}}{\sqrt{x^2-a^2}}\\
			& \Leftrightarrow &\dfrac{9a\sqrt{2}}{8}\sqrt{x^2-a^2}=\dfrac{x^2}{2}\\
			&\Leftrightarrow & 9a\sqrt{x^2-a^2}=2\sqrt{2}x^2 \\
			&\Leftrightarrow & 81a^2\left(x^2-a^2\right)=8x^4  \\
			& \Leftrightarrow & 8x^4-81a^2x^2+81a^4=0 \\
			& \Leftrightarrow &  8\left(\dfrac{x^2}{a^2}\right)^2-81\left(\dfrac{x}{a}\right)+81=0\Leftrightarrow\hoac{&\left(\dfrac{x}{a}\right)^2=9\\&\left(\dfrac{x}{a}\right)^2=\dfrac{9}{8}.}  
		\end{eqnarray*} 
		$\left(\dfrac{x}{a}\right)^2=\dfrac{9}{8}$ không thỏa vì $x<a\sqrt{2}$.\\
		$\left(\dfrac{x}{a}\right)^2=9\Leftrightarrow x=3a$.\\
		Suy ra $SO^2=(3a)^2-a^2=8a^2$.\\
		$\mathrm{d}(AB;SD)=\mathrm{d}\left(AB,(SDC)\right)=\mathrm{d}\left(A;(SCD)\right)=2\mathrm{d}\left(O;(SCD)\right)$.\\
		Gọi $E$ là trung điểm $CD$, kẻ $OH\perp SE$, khi đó $\mathrm{d}\left(O,(SCD)\right)=OH$.\\
		$\dfrac{1}{OH^2}=\dfrac{1}{SO^2}+\dfrac{1}{OE^2}=\dfrac{1}{8a^2}+\dfrac{2}{a^2}\Rightarrow OH=\dfrac{2\sqrt{2}a}{17}$.\\
		.\\
		$\mathrm{d}(AB;SD)=2OH=\dfrac{4\sqrt{34}a}{17}$.	
	}
\end{ex}
\begin{ex}%[1H3K5-4]%[2H3K4-1]%Câu 49.
	[Lương Thế Vinh - Hà Nội - 2020]
	\immini{
		Cho hình chóp $S.ABCD$ có đáy $ABCD$ là hình chữ nhật với $AB=2a$, $AD=3a$ (tham khảo hình vẽ). Tam giác $SAB$ cân tại $S$ và nằm trong mặt phẳng vuông góc với mặt đáy; góc giữa mặt phẳng $(SCD)$ và mặt đáy là $45^{\circ}$. Gọi $H$ là trung điểm cạnh $AB$. Tính theo $a$ khoảng cách giữa hai đoạn thẳng $SD$ và $CH$. 
		\choice
		{$\dfrac{3\sqrt{11}a}{11}$}
		{\True $\dfrac{3\sqrt{14}a}{7}$}
		{$\dfrac{3\sqrt{10}a}{\sqrt{109}}$}
		{$\dfrac{3\sqrt{85}a}{17}$}}{
		\begin{tikzpicture}[scale=.7]
			\coordinate (A) at (0,0);
			\coordinate (D) at (5,0);
			\coordinate (B) at (-2,-1.5);
			\coordinate (z) at (0,5);
			\coordinate (C) at ($(B)+(D)$);
			\coordinate (H) at ($(A)!0.5!(B)$);
			\coordinate (S) at ($(H)+(z)$);
			
			
			
			%	\coordinate (W) at (intersection of E--B and C--K);
			\draw (S)--(B)--(C)--(D)--(S)--(C);
			\draw[dashed] (S)--(A)--(B) (D)--(A) (S)--(H)--(C);
			%		%	\draw ([yshift=-0.15cm]E) -- ([yshift=-0.15cm]F) node[midway, below]{6 cm};
			%		\draw pic[draw,angle radius=2mm] {right angle = F--E--G};
			\draw[fill=black] 
			(A) circle (1pt)  node[above right]{$A$} 
			(S) circle (1pt)  node[above right]{$S$} 
			(B) circle (1pt) node[left]{$B$} 
			(C) circle (1pt) node[right]{$C$} 
			(D) circle (1pt) node[right]{$D$}
			(H) circle (1pt) node[below]{$H$}		 
			
			
			;		
			%	\draw pic[draw,angle radius=2mm] {right angle = A'--M--C};
			%	\draw pic[draw,angle radius=2mm] {right angle = M--H--D};
			%	\draw pic[draw,angle radius=2mm] {right angle = M--K--A'};
			%	\draw pic[draw,angle radius=2mm] {right angle = A--B--C};
			%	\draw ($(S)!0.5!(B)$)node[rotate=40]{$/$} ($(S)!0.5!(A)$)node[rotate=-20]{$/$} ;
		\end{tikzpicture}
	}
	\loigiai{
		\textbf{Cách 1: }
		\immini{
			Ta có $\heva{&(SAB)\perp(ABCD)\\&(SAB)\cap(ABCD)\\&SH\perp AB; SH\subset(SAB)}\Rightarrow SH\perp(ABCD)$.\\
			Kẻ $HK\perp CD$ ($K$ là trung điểm của $CD$)\\
			$ \Rightarrow CD\perp(SHK)\Rightarrow CD\perp SK $ \\
			$ \Rightarrow\widehat{\left((SCD);(ABCD)\right)}=\widehat{(SK; HK)} =\widehat{SKH}=45^{\circ} $ \\
			$ \Rightarrow\triangle SHK $ vuông cân tại $H\Rightarrow SH=HK=3a$.\\
			Kẻ $d$ qua $D$ song song với $HC$ cắt $AB$ tại $E$ \\ $\Rightarrow ED=HC=a\sqrt{10}$ \\
			$ \Rightarrow\mathrm{d}(CH; SD)=\mathrm{d}\left(CH;(SED)\right) =\mathrm{d}\left(H;(SED)\right) $.
		}{
			\begin{tikzpicture}[scale=.7]
				\coordinate (A) at (0,0);
				\coordinate (D) at (5,0);
				\coordinate (B) at (-2,-1);
				\coordinate (z) at (0,5);
				\coordinate (C) at ($(B)+(D)$);
				\coordinate (H) at ($(A)!0.5!(B)$);
				\coordinate (K) at ($(D)!0.5!(C)$);
				\coordinate (S) at ($(H)+(z)$);
				\coordinate (E) at ($(B)!1.5!(A)$);
				\coordinate (F) at ($(D)!0.7!(E)$);
				\coordinate (G) at ($(S)!0.6!(F)$);
				
				
				
				%	\coordinate (W) at (intersection of E--B and C--K);
				\draw (K)--(S)--(B)--(C)--(D)--(S)--(C);
				\draw[dashed] (S)--(A)--(B) (D)--(A)--(E)--(D) (E)--(S)--(H)--(C) (H)--(F)--(S) (K)--(H)--(G);
				%		%	\draw ([yshift=-0.15cm]E) -- ([yshift=-0.15cm]F) node[midway, below]{6 cm};
				%		\draw pic[draw,angle radius=2mm] {right angle = F--E--G};
				\draw[fill=black] 
				(A) circle (1pt)  node[above]{$A$} 
				(S) circle (1pt)  node[above right]{$S$} 
				(B) circle (1pt) node[left]{$B$} 
				(C) circle (1pt) node[right]{$C$} 
				(D) circle (1pt) node[right]{$D$}
				(H) circle (1pt) node[left]{$H$}	
				
				(E) circle (1pt) node[above]{$E$} 
				(F) circle (1pt) node[above]{$F$} 
				(G) circle (1pt) node[below]{$G$}
				(K) circle (1pt) node[right]{$K$}		 
				
				
				;		
				%	\draw pic[draw,angle radius=2mm] {right angle = A'--M--C};
				%	\draw pic[draw,angle radius=2mm] {right angle = M--H--D};
				%	\draw pic[draw,angle radius=2mm] {right angle = M--K--A'};
				%	\draw pic[draw,angle radius=2mm] {right angle = A--B--C};
				%	\draw ($(S)!0.5!(B)$)node[rotate=40]{$/$} ($(S)!0.5!(A)$)node[rotate=-20]{$/$} ;
			\end{tikzpicture}	
		}
		\noindent 	Kẻ $HF\perp ED\Rightarrow ED\perp(SHF)$.\\
		Kẻ $HG\perp SF\Rightarrow HG\perp(SED)\Rightarrow\mathrm{d}\left(H;(SED)\right)=HG$.\\
		Ta có: $S_{\triangle HED}=\dfrac{1}{2}AD\cdot EH=\dfrac{1}{2}HF\cdot ED\Rightarrow HF=\dfrac{AD\cdot EH}{ED} =\dfrac{3a\cdot 2a}{a\sqrt{10}}=\dfrac{3\sqrt{10}a}{5}$.\\
		Xét tam giác $SHF$ vuông tại $H$ ta có\\
		$\dfrac{1}{HG^2}=\dfrac{1}{SH^2}+\dfrac{1}{HF^2}\Rightarrow HG=\dfrac{SH\cdot HF}{\sqrt{SH^2+HF^2}} =\dfrac{3a\cdot\dfrac{3\sqrt{10}a}{5}}{\sqrt{9a^2+\dfrac{18a^2}{5}}} =\dfrac{3\sqrt{14}a}{7}$ \\
		$ \Rightarrow\mathrm{d}(CH;SD)=\dfrac{3\sqrt{14}a}{7} $.
		
		\textbf{Cách 2: }
		\immini{
			Ta có $\heva{&(SAB)\perp(ABCD)\\&(SAB)\cap(ABCD)\\&SH\perp AB; SH\subset(SAB)}\Rightarrow SH\perp(ABCD)$.\\
			Kẻ $HK\perp CD$ ($K$ là trung điểm của $CD$)\\
			$ \Rightarrow CD\perp(SHK)\Rightarrow CD\perp SK $ \\
			$ \Rightarrow\widehat{\left((SCD);(ABCD)\right)}=\widehat{(SK; HK)} =\widehat{SKH}=45^{\circ} $ \\
			$ \Rightarrow\triangle SHK $ vuông cân tại $H\Rightarrow SH=HK=3a$.\\
			Chọn hệ trục tọa độ như hình vẽ $H\equiv O$, tia $Ox$ chứa $HK$, tia $Oy$ chứa $HA$, tia $Oz$ chứa $HS$.\\
			Khi đó $H(0; 0; 0)$; $C(3a;-a; 0)$; $D(3a; a; 0)$; $S(0; 0; 3a)$.\\
			Ta có  $\overrightarrow{HC}=(3a;-a; 0)$, $\overrightarrow{SD}=(3a; a;-3a)$, $\overrightarrow{SH}=(0; 0;-3a)$ 
		}{
			\begin{tikzpicture}[scale=.7]
				\coordinate (A) at (0,0);
				\coordinate (D) at (5,0);
				\coordinate (B) at (-2,-1);
				\coordinate (z) at (0,5);
				\coordinate (C) at ($(B)+(D)$);
				\coordinate (H) at ($(A)!0.5!(B)$);
				\coordinate (K) at ($(C)!0.5!(D)$);
				\coordinate (S) at ($(H)+(z)$);
				
				\coordinate (z) at ($(H)!1.3!(S)$);
				\coordinate (x) at ($(H)!1.3!(K)$);
				\coordinate (y) at ($(B)!1.5!(A)$);
				
				
				%	\coordinate (W) at (intersection of E--B and C--K);
				
				\draw[->] (S)--(z) ;
				\draw[->] (K)--(x) ;
				\draw[dashed,->] (A)--(y);
				
				
				\draw (S)--(B)--(C)--(D)--(S)--(C);
				\draw[dashed] (S)--(A)--(B) (D)--(A) (S)--(H)--(C) (H)--(K);
				%		%	\draw ([yshift=-0.15cm]E) -- ([yshift=-0.15cm]F) node[midway, below]{6 cm};
				%		\draw pic[draw,angle radius=2mm] {right angle = F--E--G};
				\draw[fill=black] 
				(A) circle (1pt)  node[above right]{$A$} 
				(S) circle (1pt)  node[above right]{$S$} 
				(B) circle (1pt) node[left]{$B$} 
				(C) circle (1pt) node[below right]{$C$} 
				(D) circle (1pt) node[right]{$D$}
				(H) circle (1pt) node[left]{$H$}
				(K) circle (1pt) node[below]{$K$}		 
				
				(z)  node[left]{$z$} 
				(y)  node[right]{$y$}
				(x)  node[right]{$x$}
				
				;		
				%	\draw pic[draw,angle radius=2mm] {right angle = A'--M--C};
				%	\draw pic[draw,angle radius=2mm] {right angle = M--H--D};
				%	\draw pic[draw,angle radius=2mm] {right angle = M--K--A'};
				%	\draw pic[draw,angle radius=2mm] {right angle = A--B--C};
				\draw ($(S)!0.5!(B)$)node[rotate=40]{$/$} ($(S)!0.5!(A)$)node[rotate=-20]{$/$} ;
			\end{tikzpicture}	
			
		}
		\noindent
		$ \Rightarrow\left[\overrightarrow{HC};\overrightarrow{SD}\right]=\left(3a^2; 9a^2; 6a^2\right) $ \\
		$ \Rightarrow\mathrm{d}(CH;SD)=\dfrac{\left|\overrightarrow{SH}\cdot\left[\overrightarrow{HC};\overrightarrow{SD}\right]\right|}{\left|\left[\overrightarrow{HC};\overrightarrow{SD}\right]\right|} =\dfrac{\left|6a^2\cdot (-3a)\right|}{\sqrt{\left(3a^2\right)^2+\left(9a^2\right)^2+\left(6a^2\right)^2}} =\dfrac{3\sqrt{14}a}{7} $.
	}
\end{ex}
\begin{dang}
	{Khoảng cách của đường với mặt, mặt với mặt}
\end{dang}
\begin{ex}%[1H3K5-3]%Câu 1.
	[Chuyên Nguyễn Bỉnh Khiêm - Quảng Nam - 2020]
	Cho hình chóp $S.ABCD$ có đáy $ABCD$ là hình thang vuông tại $A$ và $D$, $AB=3a, AD=DC=a$. Gọi $I$ là trung điểm của $AD$, biết hai mặt phẳng $(SBI)$ và $(SCI)$ cùng vuông góc với đáy và mặt phẳng $(SBC)$ tạo với đáy một góc $60^{\circ}$. Tính theo $a$ khoảng cách từ trung điểm cạnh $SD$ đến mặt phẳng $(SBC)$. 
	\choice
	{$\dfrac{a\sqrt{17}}{5}$}
	{\True $\dfrac{a\sqrt{6}}{19}$}
	{$\dfrac{a\sqrt{3}}{15}$}
	{$\dfrac{a\sqrt{15}}{20}$}
	\loigiai{
		\begin{center}
			\begin{tikzpicture}[scale=.9]
				\coordinate (A) at (0,0);
				\coordinate (B) at (6,0);
				\coordinate (D) at (-2,-1);
				\coordinate (I) at ($(A)!0.5!(D)$);
				\coordinate (z) at (0,5);
				\coordinate (C) at ($(D)+($(A)!0.33!(B)$)$);
				\coordinate (S) at ($(I)+(z)$);			
				
				\coordinate (M) at (intersection of A--D and B--C);
				\coordinate (K) at ($(M)!0.6!(C)$);
				\coordinate (H) at ($(S)!0.6!(K)$);
				\coordinate (E) at ($(S)!0.6!(D)$);
				\coordinate (X) at (intersection of D--C and I--K);
				%			
				\draw (K)--(S) (M)--(C)--(S)--(B)--(C)  (S)--(M);
				\draw[dashed] (S)--(A)--(B) (D)--(A) (S)--(I)--(X)--(I)--(K) (M)--(D)--(S) (D)--(C);
				%		%	\draw ([yshift=-0.15cm]E) -- ([yshift=-0.15cm]F) node[midway, below]{6 cm};
				%		\draw pic[draw,angle radius=2mm] {right angle = F--E--G};
				\draw[fill=black] 
				(S) circle (1pt)  node[above]{$S$} 
				(A) circle (1pt)  node[above right]{$A$} 
				(B) circle (1pt)  node[right]{$B$} 
				(C) circle (1pt)  node[below right]{$C$} 
				(D) circle (1pt)  node[above left]{$D$} 
				(I) circle (1pt)  node[right]{$I$} 
				(K) circle (1pt)  node[below right]{$K$} 
				(H) circle (1pt)  node[below left]{$H$}
				(E) circle (1pt)  node[left]{$E$}
				(M) circle (1pt)  node[left]{$M$}	
				%	(X) circle (1pt)  node[left]{$X$}		
				;		
				\draw pic[draw,angle radius=2mm] {right angle = I--H--K};
				%	\draw pic[draw,angle radius=2mm] {right angle = M--H--D};
				%	\draw pic[draw,angle radius=2mm] {right angle = M--K--A'};
				
				%			\draw pic[draw,angle radius=7mm] {angle = C--B--A};
				%			\draw pic[draw,angle radius=9mm] {angle = C'--B'--A'};
				%\draw ($(S)!0.5!(B)$)node[rotate=40]{$/$} ($(S)!0.5!(A)$)node[rotate=-20]{$/$} ;
			\end{tikzpicture}
			\begin{tikzpicture}[scale=.9]
				\coordinate (A) at (0,0);
				\coordinate (B) at (6,0);
				\coordinate (D) at (0,-2);
				\coordinate (M) at (0,-3);
				\coordinate (C) at (2,-2);
				\coordinate (I) at (0,-1);
				\coordinate (K) at ($(M)!0.4!(C)$);
				
				
				%			
				\draw (A)--(M)--(B)--(A) (I)--(K) (D)--(C)  ;
				%	\draw[dashed] (S)--(A)--(B) (D)--(A) (S)--(I)--(X)--(I)--(K) (M)--(D)--(S) (D)--(C);
				%		%	\draw ([yshift=-0.15cm]E) -- ([yshift=-0.15cm]F) node[midway, below]{6 cm};
				%		\draw pic[draw,angle radius=2mm] {right angle = F--E--G};
				\draw[fill=black] 
				(A) circle (1pt)  node[above left]{$A$} 
				(B) circle (1pt)  node[right]{$B$} 
				(C) circle (1pt)  node[below right]{$C$} 
				(D) circle (1pt)  node[left]{$D$} 
				(I) circle (1pt)  node[left]{$I$} 
				(K) circle (1pt)  node[below right]{$K$} 
				(M) circle (1pt)  node[left]{$M$}	
				
				
				%	(X) circle (1pt)  node[left]{$X$}		
				;		
				%	\draw pic[draw,angle radius=2mm] {right angle = I--H--K};
				%	\draw pic[draw,angle radius=2mm] {right angle = M--H--D};
				%	\draw pic[draw,angle radius=2mm] {right angle = M--K--A'};
				
				%			\draw pic[draw,angle radius=7mm] {angle = C--B--A};
				%			\draw pic[draw,angle radius=9mm] {angle = C'--B'--A'};
				%\draw ($(S)!0.5!(B)$)node[rotate=40]{$/$} ($(S)!0.5!(A)$)node[rotate=-20]{$/$} ;
			\end{tikzpicture}
		\end{center}
		Kẻ $IK\perp BC(K\in BC)\Rightarrow\left((SBC);(ABCD)\right)=\widehat{SKI}=60^{\circ}$.\\
		Gọi $M=AD\cap BC$. Ta có $\dfrac{MD}{MA}=\dfrac{1}{3}\Rightarrow MD=\dfrac{a}{2}$.\\
		Ta có $\triangle MIK$ đồng dạng với $\triangle MBA$ nên suy ra $\dfrac{IK}{BA}=\dfrac{MI}{MB}=\dfrac{a}{\sqrt{(3a)^2+\left(\dfrac{3a}{2}\right)^2}}=\dfrac{2\sqrt{5}}{15}$ \\
		$ \Rightarrow IK=\dfrac{2\sqrt{5}}{15}\cdot 3a=\dfrac{2a\sqrt{5}}{5} $.\\
		Gọi $N$ là trung điểm của $SD$.\\
		Ta có $\mathrm{d}\left(N,(SBC)\right)=\dfrac{1}{2}\mathrm{d}\left(D,(SBC)\right)=\dfrac{1}{4}\mathrm{d}\left(I,(SBC)\right)$.\\
		Từ $I$ kẻ $IH\perp SK$ suy ra $IH=\mathrm{d}\left(I,(SBC)\right)=IK\cdot\sin 60^{\circ}=\dfrac{a\sqrt{15}}{5}\Rightarrow\mathrm{d}\left(N,(SBC)\right)=\dfrac{a\sqrt{15}}{20}$.}
\end{ex}
\begin{ex}%[1H3K5-3]%Câu 2.
	[THPT Lê Xoay Vĩnh Phúc 2019]
	Cho hình chóp $S.ABCD$ có đáy là hình thang vuông tại $A$ và $D$, $SD$ vuông góc với mặt đáy $(ABCD)$, $AD=2a, SD=a\sqrt{2}$. Tính khoảng cách giữa đường thẳng $CD$ và mặt phẳng $(SAB)$ 
	\choice
	{$\dfrac{a}{\sqrt{2}}$}
	{$a\sqrt{2}$}
	{\True $\dfrac{2a}{\sqrt{3}}$}
	{$\dfrac{a\sqrt{3}}{2}$}
	\loigiai{
		\immini{
			Ta có $\heva{&AB\perp AD\\&AB\perp SD}$ nên $AB\perp(SAD)$.\\
			Kẻ $DH\perp SA$ tại $H$. Do $DH\subset(SAD)$ nên $AB\perp DH$.\\
			Ta có $\heva{&DH\perp SA\\&DH\perp AB}\Rightarrow DH\perp(SAB)$.\\
			Do $DC\parallel AB$ nên $DC\parallel(SAB)$.\\
			Vậy khoảng cách giữa đường thẳng $CD$ và mặt phẳng $(SAB)$ là $DH$.\\
			Xét $\triangle SAD$ vuông tại $D$ có\\ $\dfrac{1}{DH^2}=\dfrac{1}{SD^2}+\dfrac{1}{AD^2} =\dfrac{1}{(a\sqrt{2})^2}+\dfrac{1}{(2a)^2}=\dfrac{3}{4a^2}$ 
		}{
			
			\begin{tikzpicture}[scale=.7]
				\coordinate (A) at (0,0);
				\coordinate (B) at (4,0);
				\coordinate (D) at (2,1);
				\coordinate (I) at ($(A)!0.5!(D)$);
				\coordinate (z) at (0,5);
				\coordinate (C) at ($(D)+($(A)!1.33!(B)$)$);
				\coordinate (S) at ($(D)+(z)$);	
				\coordinate (H) at ($(S)!0.6!(A)$);		
				
				%\coordinate (M) at (intersection of A--D and B--C);
				
				%			
				\draw (S)--(A)--(B)--(C)--(S)--(B)  ;
				\draw[dashed] (S)--(D)--(A) (D)--(C) (D)--(H);
				%		%	\draw ([yshift=-0.15cm]E) -- ([yshift=-0.15cm]F) node[midway, below]{6 cm};
				%		\draw pic[draw,angle radius=2mm] {right angle = F--E--G};
				\draw[fill=black] 
				(S) circle (1pt)  node[above]{$S$} 
				(A) circle (1pt)  node[below]{$A$} 
				(B) circle (1pt)  node[below]{$B$} 
				(C) circle (1pt)  node[right]{$C$} 
				(D) circle (1pt)  node[above right]{$D$} 
				(H) circle (1pt)  node[left]{$I$} 
				
				;		
				\draw pic[draw,angle radius=2mm] {right angle = S--D--A};
				\draw pic[draw,angle radius=2mm] {right angle = S--D--C};
				\draw pic[draw,angle radius=2mm] {right angle = A--D--C};
				\draw pic[draw,angle radius=2mm] {right angle = D--H--A};
				
				%\draw ($(S)!0.5!(B)$)node[rotate=40]{$/$} ($(S)!0.5!(A)$)node[rotate=-20]{$/$} ;
			\end{tikzpicture}
			
		}
		\noindent $ \Rightarrow DH=\dfrac{2a}{\sqrt{3}} $. Khoảng cách giữa đường thẳng $CD$ và mặt phẳng $(SAB)$ là $\dfrac{2a}{\sqrt{3}}$.
	}
\end{ex}
\begin{ex}%[1H3K5-3]%Câu 3.
	Cho hình chóp $S.ABCD$ có đáy $ABCD$ là hình vuông cạnh $a$, $SA$ vuông góc với đáy và $SA=2a$. Gọi $M$ là trung điểm của $SD$. Tính khoảng cách $d$ giữa đường thẳng $SB$ và mặt phẳng $(ACM)$ 
	\choice
	{$d=\dfrac{3a}{2}$}
	{$d=a$}
	{\True $d=\dfrac{2a}{3}$}
	{$d=\dfrac{a}{3}$}
	\loigiai{
		\immini{
			Gọi $O$ là tâm hình vuông. \\
			Ta có $MO\parallel SB\Rightarrow SB\parallel (ACM)$ \\
			$ \Rightarrow\mathrm{d}(SB,(ACM))=\mathrm{d}(B,(ACM))=\mathrm{d}(D,(ACM)) $ (vì $O$ là trung điểm $BD$).\\
			Gọi I là trung điểm AD \\
			$\Rightarrow\heva{&MI\parallel SA\Rightarrow MI\perp (ABCD)\\&\mathrm{d}(D,(ACM))=2\mathrm{d}(I,(ACM)).}$ \\
			Trong $(ABCD)$ kẻ $IK\perp AC$ tại K.\\
			Trong $(MIK)$ kẻ $IH\perp MK$ tại H. (1)\\
			Ta có $AC\perp MI,AC\perp IK$\\
			$\Rightarrow AC\perp (MIK)\Rightarrow AC\perp IH$. (2)
		}{
			\begin{tikzpicture}[scale=.7]
				\coordinate (A) at (0,0);
				\coordinate (D) at (5,0);
				\coordinate (B) at (-2,-1);		
				\coordinate (z) at (0,5);
				\coordinate (C) at ($(B)+(D)$);
				\coordinate (S) at ($(A)+(z)$);	
				\coordinate (M) at ($(S)!0.5!(D)$);	
				\coordinate (O) at ($(B)!0.5!(D)$);
				\coordinate (I) at ($(A)!0.5!(D)$);	
				\coordinate (K) at ($(A)!0.5!(O)$);	
				\coordinate (H) at ($(M)!0.6!(K)$);			
				
				%	\coordinate (M) at (intersection of A--D and B--C);
				
				\draw  (S)--(B)--(C)--(D)--(S)--(C)--(M) ;
				\draw[dashed] (S)--(A)--(B) (B)--(D)--(A)--(C) (A)--(M)--(I)--(K)--(M)--(O) (I)--(H);
				%		%	\draw ([yshift=-0.15cm]E) -- ([yshift=-0.15cm]F) node[midway, below]{6 cm};
				%		\draw pic[draw,angle radius=2mm] {right angle = F--E--G};
				\draw[fill=black] 
				(S) circle (1pt)  node[above]{$S$} 
				(A) circle (1pt)  node[above left]{$A$} 
				(B) circle (1pt)  node[below]{$B$} 
				(C) circle (1pt)  node[below right]{$C$} 
				(D) circle (1pt)  node[right]{$D$} 
				(I) circle (1pt)  node[below left]{$I$} 
				(K) circle (1pt)  node[left]{$K$} 
				(O) circle (1pt)  node[below]{$O$} 
				(H) circle (1pt)  node[below left]{$H$}
				(M) circle (1pt)  node[right]{$M$}	
				;		
				%	\draw pic[draw,angle radius=2mm] {right angle = I--H--K};
				%	\draw pic[draw,angle radius=2mm] {right angle = M--H--D};
				%	\draw pic[draw,angle radius=2mm] {right angle = M--K--A'};
				
				%			\draw pic[draw,angle radius=7mm] {angle = C--B--A};
				%			\draw pic[draw,angle radius=9mm] {angle = C'--B'--A'};
				%\draw ($(S)!0.5!(B)$)node[rotate=40]{$/$} ($(S)!0.5!(A)$)node[rotate=-20]{$/$} ;
			\end{tikzpicture}	
			
		}
		\noindent Từ $(1)\&(2)\Rightarrow IH\perp (ACM)\Rightarrow\mathrm{d}(I,(ACM))=IH$.\\
		Trong tam giác $MIK$ ta có $IH=\dfrac{IM\cdot IK}{\sqrt{IM^2+IK^2}}$.\\
		Biết $MI=\dfrac{SA}{2}=a,IK=\dfrac{OD}{2}=\dfrac{BD}{4}=\dfrac{a\sqrt{2}}{4}\Rightarrow IH=\dfrac{a\cdot\dfrac{a\sqrt{2}}{4}}{\sqrt{a^2+\dfrac{a^2}{8}}}=\dfrac{a}{3}$.\\
		Vậy $\mathrm{d}(SB,(ACM))=\dfrac{2a}{3}$.
	}
\end{ex}
\begin{ex}%[1H3K5-3]%Câu 4.
	[THPT Lương Đắc Bằng - Thanh Hóa - 2018] Cho hình chóp $O.ABC$ có đường cao $OH=\dfrac{2a}{\sqrt{3}}$. Gọi $M$ và $N$ lần lượt là trung điểm $OA$ và $OB$. Khoảng cách giữa đường thẳng $MN$ và $(ABC)$ bằng 
	\choice
	{$\dfrac{a}{2}$}
	{\True $\dfrac{a\sqrt{3}}{3}$}
	{$\dfrac{a}{3}$}
	{$\dfrac{a\sqrt{2}}{2}$}
	\loigiai{
		\immini{
			Ta có $MN\parallel(ABC)\Rightarrow\mathrm{d}\left(MN;(ABC)\right)=\mathrm{d}\left(M;(ABC)\right)$.\\
			Mà $\dfrac{AM}{AO}=\dfrac{1}{2}=\dfrac{\mathrm{d}\left(M;(ABC)\right)}{\mathrm{d}\left(O;(ABC)\right)}$\\
			$\Rightarrow\mathrm{d}\left(M;(ABC)\right)=\dfrac{1}{2}\mathrm{d}\left(O;(ABC)\right)=\dfrac{1}{2}OH=\dfrac{a\sqrt{3}}{3}$.}{
			
			\begin{tikzpicture}[scale=.7]
				\coordinate (A) at (0,0);
				\coordinate (C) at (5,0);
				\coordinate (B) at (2,-1);		
				\coordinate (O) at (3.5,4);
				\coordinate (M) at ($(O)!0.5!(A)$);	
				\coordinate (N) at ($(O)!0.5!(B)$);
				
				
				%	\coordinate (M) at (intersection of A--D and B--C);
				
				\draw  (O)--(A)--(B)--(C)--(O)--(B) (N)--(M) ;
				\draw[dashed] (A)--(C);
				%		%	\draw ([yshift=-0.15cm]E) -- ([yshift=-0.15cm]F) node[midway, below]{6 cm};
				%		\draw pic[draw,angle radius=2mm] {right angle = F--E--G};
				\draw[fill=black] 
				(O) circle (1pt)  node[above]{$O$} 
				(A) circle (1pt)  node[left]{$A$} 
				(B) circle (1pt)  node[below]{$B$} 
				(C) circle (1pt)  node[right]{$C$} 
				(M) circle (1pt)  node[left]{$M$} 
				(N) circle (1pt)  node[right]{$N$} 
				;		
				%	\draw pic[draw,angle radius=2mm] {right angle = I--H--K};
				%	\draw pic[draw,angle radius=2mm] {right angle = M--H--D};
				%	\draw pic[draw,angle radius=2mm] {right angle = M--K--A'};
				
				%			\draw pic[draw,angle radius=7mm] {angle = C--B--A};
				%			\draw pic[draw,angle radius=9mm] {angle = C'--B'--A'};
				%\draw ($(S)!0.5!(B)$)node[rotate=40]{$/$} ($(S)!0.5!(A)$)node[rotate=-20]{$/$} ;
			\end{tikzpicture}	
		}
		
	}
\end{ex}
\begin{ex}%[1H3K5-3]%Câu 5.
	[Chuyên Nguyễn Quang Diêu - Đồng Tháp - 2018]
	Cho hình lập phương \break $ABCD.A'B'C'D'$ cạnh $a$. Gọi $I$, $J$ lần lượt là trung điểm của $BC$ và $AD$. Tính khoảng cách $d$ giữa hai mặt phẳng $(AIA')$ và $(CJC')$. 
	\choice
	{$d=2a\sqrt{\dfrac{5}{2}}$}
	{$d=2a\sqrt{5}$}
	{\True $d=\dfrac{a\sqrt{5}}{5}$}
	{$d=\dfrac{3a\sqrt{5}}{5}$}
	\loigiai{
		\immini{
			Ta có $\heva{&AA'\parallel CC'\\&AI\parallel CJ\\&AA'\cap AI=A\\&AA',AI\subset(AIA')}\Rightarrow(AIA')\parallel (CJC')$.\\
			$d=2(\sqrt{5}+2)a$.\\
			Kẻ $IK\perp CJ$. (1)\\
			Lại có $\heva{&CC'\perp IK\\&CC'\cap CJ=C\\&CC', CJ\subset(CJC').}$ (2)\\
			Từ $(1)$, $(2)$ suy ra $IK\perp(CJC')$ hay $\mathrm{d}\left(I,(CJC')\right)=IK$.\\
			Xét tam giác $CJI$ vuông tại $I$ }{
			\begin{tikzpicture}[scale=.7]
				\coordinate (A) at (0,0);
				\coordinate (D) at (5,0);
				\coordinate (B) at (2,1);		
				\coordinate (z) at (0,-5);
				\coordinate (C) at ($(B)+(D)$);
				\coordinate (A') at ($(A)+(z)$);
				\coordinate (B') at ($(B)+(z)$);
				\coordinate (C') at ($(C)+(z)$);
				\coordinate (D') at ($(D)+(z)$);	
				
				\coordinate (I) at ($(B)!0.5!(C)$);	
				\coordinate (J) at ($(A)!0.5!(D)$);
				\coordinate (K) at ($(J)!0.6!(C)$);	
				
				
				%	\coordinate (M) at (intersection of A--D and B--C);
				
				\draw  (A)--(B)--(C)--(D)--(A)--(A')--(D')--(C')--(C)--(J) (K)--(I)--(A) (D)--(D');
				\draw[dashed] (A')--(B')--(B) (J)--(C')--(B') (A')--(I) ;
				%		%	\draw ([yshift=-0.15cm]E) -- ([yshift=-0.15cm]F) node[midway, below]{6 cm};
				%		\draw pic[draw,angle radius=2mm] {right angle = F--E--G};
				\draw[fill=black] 
				(A) circle (1pt)  node[left]{$A$} 
				(B) circle (1pt)  node[above]{$B$} 
				(C) circle (1pt)  node[right]{$C$} 
				(D) circle (1pt)  node[below right]{$D$} 
				(A') circle (1pt)  node[left]{$A'$} 
				(B') circle (1pt)  node[below]{$B'$} 
				(C') circle (1pt)  node[right]{$C'$} 
				(D') circle (1pt)  node[below right]{$D'$} 
				
				
				(I) circle (1pt)  node[above]{$I$} 
				(K) circle (1pt)  node[below left]{$K$} 
				(J) circle (1pt)  node[below]{$J$} 
				;		
				\draw pic[draw,angle radius=2mm] {right angle = I--K--J};
				%	\draw pic[draw,angle radius=2mm] {right angle = M--H--D};
				%	\draw pic[draw,angle radius=2mm] {right angle = M--K--A'};
				
				%			\draw pic[draw,angle radius=7mm] {angle = C--B--A};
				%			\draw pic[draw,angle radius=9mm] {angle = C'--B'--A'};
				%\draw ($(S)!0.5!(B)$)node[rotate=40]{$/$} ($(S)!0.5!(A)$)node[rotate=-20]{$/$} ;
			\end{tikzpicture}	
			
		}
		$\dfrac{1}{IK^2}=\dfrac{1}{IC^2}+\dfrac{1}{IJ^2}\Leftrightarrow\dfrac{1}{IK^2}=\dfrac{1}{\left(\dfrac{a}{2}\right)^2}+\dfrac{1}{a^2}$ 
		$ \Leftrightarrow IK^2=\dfrac{a^2}{5}\Leftrightarrow IK=\dfrac{a\sqrt{5}}{5} $.	
	}
\end{ex}
%%%%% còn thiếu từ 49 đến 62 do TV3 chưa nộp %%%%
\Closesolutionfile{ans}
\indapan{10}{ans/CD8/Muc_9_10}