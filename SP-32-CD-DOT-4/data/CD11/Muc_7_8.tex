\setcounter{dang}{0}
\setcounter{ex}{0}
\section{Mức 7,8 điểm}
\Opensolutionfile{ans}[ans/Muc_7_8]
\begin{ex}%[2H1K3-2]%Câu 1.
	Cho hình lăng trụ đứng $ABC.A'B'C'$ có đáy $ABC$ là tam giác vuông cân tại $A$, $BC=a\sqrt{2}, A'B$ tạo với đáy một góc bằng $60^{\circ}$. Thể tích của khối lăng trụ bằng
	\choice
	{\True $\dfrac{\sqrt{3}a^3}{2}$}
	{$\dfrac{\sqrt{3}a^3}{4}$}
	{$\dfrac{3a^3}{2}$}
	{$\dfrac{a^3}{2}$}
	\loigiai{\begin{center}
			\begin{tikzpicture}[font=\footnotesize, line join=round, line cap=round, >=stealth,scale=1]
				\def\a{4}
				\def\h{4.5}
				\path 	(0:0) coordinate (A')
				++(0:\a) coordinate (C')
				++(-150:1*\a/2) coordinate (B')
				($(A')+(90:\h)$) coordinate (A)
				($(B')+(90:\h)$) coordinate (B)
				($(C')+(90:\h)$) coordinate (C)
				($(B)!.5!(C)$) coordinate (M);
				\draw[dashed,thick] (A')--(C');
				\draw[thick](C)--(C') 	(B)--(B')	(A)--(A') (A')--(B) (A)--(B)--(C)--cycle (A')--(B')--(C');
				\foreach \x/\g in {A/180,B/-45,C/0,A'/180,B'/-45,C'/0}
				\fill[black] (\x) circle (1pt)
				($(\g:4mm)+(\x)$) node {$\x$};	
				\foreach \x/\y/\z in {B/A/C,B/B'/A'}{\path pic[draw,angle radius=7pt]{right angle= \x--\y--\z};}
				\foreach \x/\y in {A/B,A/C}{\path (\x)--(\y) node[midway,sloped]{\tikz{\draw (-90:3pt)--(90:3pt);}};}
				\path pic["\scriptsize$60^\circ$",red,angle eccentricity=2,draw,angle radius=9pt]{angle= B'--A'--B};
				\draw (M) node[below] {$a\sqrt{2}$};
			\end{tikzpicture}
		\end{center}
		$\triangle ABC$ là tam giác vuông cân tại $A$, $BC=a\sqrt{2}\Rightarrow AB=AC=a\Rightarrow S_{\triangle ABC}=\dfrac{1}{2}a\cdot a=\dfrac{1}{2}a^2$.\\
		$A'B$ tạo với đáy một góc bằng $60^{\circ}\Rightarrow\widehat{BA'B'}=60^{\circ}$.\\
		$\triangle_vBA'B'\colon \tan\widehat{BA'B'}=\dfrac{BB'}{A'B'}=\sqrt{3}\Rightarrow BB'=\sqrt{3}A'B'=a\sqrt{3}$.\\
		Thể tích khối lăng trụ $ABC.A'B'C'$ là $V_{ABC.A'B'C'}=BB'\cdot S_{\triangle ABC}=a\sqrt{3}\cdot\dfrac{1}{2}a^2=\dfrac{\sqrt{3}a^3}{2}$.
	}
\end{ex}
\begin{ex}%[2H1K3-2]%Câu 2.[Lý Nhân Tông - Bắc Ninh 2019] 
	Cho khối lăng trụ đứng tam giác $ABC.A'B'C'$ có đáy là một tam giác vuông tại $A$. Cho $AC=AB=2a$, góc giữa $AC'$ và mặt phẳng $(ABC)$ bằng $30^{\circ}$. Tính thể tích khối lăng trụ $ABC.A'B'C'$. 
	\choice
	{$\dfrac{2a^3\sqrt{3}}{3}$}
	{$\dfrac{a^3\sqrt{3}}{3}$}
	{$\dfrac{5a^3\sqrt{3}}{3}$}
	{\True $\dfrac{4a^3\sqrt{3}}{3}$}
	\loigiai{\begin{center}
			\begin{tikzpicture}[font=\footnotesize, line join=round, line cap=round, >=stealth,scale=1]
				\def\a{4}
				\def\h{4.5}
				\path 	(0:0) coordinate (A)
				++(0:\a) coordinate (B)
				++(-150:1*\a/2) coordinate (C)
				($(A)+(90:\h)$) coordinate (A')
				($(B)+(90:\h)$) coordinate (B')
				($(C)+(90:\h)$) coordinate (C');
				\draw[dashed,thick] (A)--(B);
				\draw[thick](C)--(C') 	(B)--(B')	(A)--(A') (A)--(C') (A)--(C)--(B) (A')--(B')--(C')--cycle;
				\foreach \x/\g in {A/180,B/0,C/-30,A'/180,B'/0,C'/-30}
				\fill[black] 	(\x) circle (1pt)
				($(\g:4mm)+(\x)$) node {$\x$};	
				\path pic[draw,angle radius=7pt]{right angle= C'--C--A};
				\path pic["\scriptsize$30^\circ$", angle eccentricity=2.5,draw,angle radius=7pt]{angle= C--A--C'};
			\end{tikzpicture}
		\end{center}
		Diện tích tam giác $ABC$: $S_{\triangle ABC}=\dfrac{1}{2}AB\cdot AC=2a^2$.\\
		Hình chiếu vuông góc của $AC'$ lên $(ABC)$ là $AC$. \\
		$ \Rightarrow $ Góc giữa $AC'$ và mặt phẳng $(ABC)$ là góc tạo bởi giữa đường thẳng $AC'$ và $AC$ hay $\widehat{C'AC}$.\\
		Theo bài ra có $\widehat{C'AC}=30^{\circ}$.\\
		Xét tam giác $C'CA$ vuông tại $C$ có $CC'=AC\cdot\tan 30^{\circ}=\dfrac{2a\sqrt{3}}{3}$.\\
		Thể tích của khối lăng trụ $ABC.A'B'C'$ là $V_{ABC.A'B'C'}=CC'\cdot S_{\triangle ABC}=\dfrac{2a\sqrt{3}}{3}\cdot 2a^2=\dfrac{4a^3\sqrt{3}}{3}$.}
\end{ex}
\begin{ex}%[2H1K3-2]%Câu 3.
	Cho lăng trụ đứng tam giác $ABC.A'B'C'$ có đáy $ABC$ là tam giác vuông cân tại $B$ với $BA=BC=a$, biết $A'B$ tạo với mặt phẳng $(ABC)$ một góc $60^{\circ}$. Thể tích khối lăng trụ đã cho bằng
	\choice
	{$2a^3$}
	{$\dfrac{a^3\sqrt{3}}{6}$}
	{\True $\dfrac{a^3\sqrt{3}}{2}$}
	{$\dfrac{a^3}{2}$}
	\loigiai{\begin{center}
			\begin{tikzpicture}[font=\footnotesize, line join=round, line cap=round, >=stealth,scale=1]
				\def\a{4}
				\def\h{4.5}
				\path 	(0:0) coordinate (A)
				++(0:\a) coordinate (B)
				++(-150:2*\a/3) coordinate (C)
				($(A)+(90:\h)$) coordinate (A')
				($(B)+(90:\h)$) coordinate (B')
				($(C)+(90:\h)$) coordinate (C');
				\draw[dashed,thick] (A)--(B) (A')--(B);
				\draw[thick](C)--(C') 	(B)--(B')	(A)--(A') (A)--(C)--(B) (A')--(B')--(C')--cycle;
				\foreach \x/\g in {A/180,B/0,C/-30,A'/180,B'/0,C'/-30}
				\fill[black] 	(\x) circle (1pt)
				($(\g:4mm)+(\x)$) node {$\x$};	
				\path pic[draw,angle radius=7pt]{right angle= A--B--C};
				\foreach \x/\y in {A/B,B/C}{
					\path (\x)--(\y) node[midway,sloped]{\tikz{\draw (-90:3pt)--(90:3pt);}};}
				\path pic["\scriptsize$60^\circ$", angle eccentricity=2,draw,angle radius=9pt]{angle= A'--B--A};
			\end{tikzpicture}
		\end{center}
		Góc giữa đường thẳng $A'B$ và mặt phẳng $(ABC)$ là $\widehat{A'BA}=60^{\circ}\Rightarrow A'A=AB\cdot\tan 60^{\circ}=a\sqrt{3}$.\\
		Có $S_{\triangle ABC}=\dfrac{1}{2}BA\cdot BC=\dfrac{a^2}{2}\Rightarrow V_{ABC.A'B'C'}=S_{\triangle ABC}\cdot A'A=\dfrac{a^3\sqrt{3}}{2}$.}
\end{ex}
\begin{ex}%[2H1K3-2]%Câu 4.[SGD Nam Định]
	Cho hình lăng trụ đứng $ABC.A'B'C'$ có đáy $ABC$ là tam giác vuông tại $A$, $\widehat{ACB}=30^{\circ}$, biết góc giữa $B'C$ và mặt phẳng $(ACC'A')$ bằng $\alpha$ thỏa mãn $\sin\alpha=\dfrac{1}{2\sqrt{5}}$. Cho khoảng cách giữa hai đường thẳng $A'B$ và $CC'$ bằng $a\sqrt{3}$. Tính thể tích $V$ của khối lăng trụ $ABC.A'B'C'$. 
	\choice
	{$V=a^3\sqrt{6}$}
	{$V=\dfrac{3a^3\sqrt{6}}{2}$}
	{$V=a^3\sqrt{3}$}
	{\True $V=2a^3\sqrt{3}$}
	\loigiai{
		\begin{center}
			\begin{tikzpicture}[font=\footnotesize, line join=round, line cap=round, >=stealth,scale=1]
				\def\a{4}
				\def\h{4.5}
				\path 	(0:0) coordinate (A')
				++(0:\a) coordinate (C')
				++(-150:3*\a/4) coordinate (B')
				($(A')+(90:\h)$) coordinate (A)
				($(B')+(90:\h)$) coordinate (B)
				($(C')+(90:\h)$) coordinate (C);
				\draw[dashed,thick] 	(A')--(C') (A')--(C);
				\draw[thick]	(C)--(C') 	(B)--(B')	(A)--(A') (A')--(B) (B')--(C) (A)--(B)--(C)--cycle (A')--(B')--(C');
				\foreach \x/\g in {A/180,B/-45,C/0,A'/180,B'/-45,C'/0}
				\fill[black] 	(\x) circle (1pt)
				($(\g:4mm)+(\x)$) node {$\x$};
				\path pic[draw,angle radius=5pt]{right angle= C--A--B};
				\path pic["\scriptsize$30^\circ$", angle eccentricity=2.5,draw,angle radius=7pt]{angle= A--C--B};
			\end{tikzpicture}
		\end{center}
		Ta có: $CC'\parallel AA'\Rightarrow CC'\parallel(AA'B'B)$.
		Mà $A'B\subset(AA'B'B),$ nên:\\
		$\mathrm{d}(CC'; A'B)=\mathrm{d}\left(CC';(AA'B'B)\right)=C'A'=a\sqrt{3}$.\\
		Ta có: $AC=A'C'=a\sqrt{3}; AB=A'B'=a$.\\
		Diện tích đáy là $B=S_{\triangle ABC}=\dfrac{a^2\sqrt{3}}{2}$.\\
		Dễ thấy $A'B'\perp (ACC'A')$.\\
		Góc giữa $B'C$ và mặt phẳng $(ACC'A')$ là $\widehat{B'CA'}=\alpha$.\\
		$\sin\alpha=\dfrac{A'B'}{B'C}=\dfrac{1}{2\sqrt{5}}\Leftrightarrow B'C=2a\sqrt{5}$.\\
		$CC'=\sqrt{B'C^2-B'C'^2}=\sqrt{20a^2-4a^2}=4a$.\\
		Thể tích lăng trụ là: $V=B\cdot h$ với $h=CC'$ nên $V=\dfrac{a^2\sqrt{3}}{2}\cdot 4a=2a^3\sqrt{3}$.}
\end{ex}
\begin{ex}%[2H1K3-2]%Câu 5.[Chuyên Đại học Vinh - 2019]
	Cho hình lăng trụ tam giác đều $ABC.A'B'C'$ có $AB=a,$ góc giữa đường thẳng $A'C$ và mặt phẳng $(ABC)$ bằng $45^{\circ}$. Thể tích khối lăng trụ $ABC.A'B'C'$ bằng
	\choice
	{\True $\dfrac{a^3\sqrt{3}}{4}$}
	{$\dfrac{a^3\sqrt{3}}{2}$}
	{$\dfrac{a^3\sqrt{3}}{12}$}
	{$\dfrac{a^3\sqrt{3}}{6}$}
	\loigiai{\begin{center}
			\begin{tikzpicture}[font=\footnotesize, line join=round, line cap=round, >=stealth,scale=1]
				\def\a{4}
				\def\h{4}
				\path 	(0:0) coordinate (A)
				++(0:\a) coordinate (C)
				++(-150:2*\a/3) coordinate (B)
				($(A)+(90:\h)$) coordinate (A')
				($(B)+(90:\h)$) coordinate (B')
				($(C)+(90:\h)$) coordinate (C')
				($(A)!.5!(B)$) coordinate (M);
				\draw[dashed,thick] 	(A)--(C) (A')--(C);
				\draw[thick]	(C)--(C') 	(B)--(B')	(A)--(A') (A)--(B)--(C) (A')--(B')--(C')--cycle;
				\foreach \x/\g in {A/180,B/-30,C/0,A'/180,B'/-30,C'/0}
				\fill[black] 	(\x) circle (1pt)
				($(\g:4mm)+(\x)$) node {$\x$};	
				\path pic[draw,red,thick,angle radius=9pt]{angle= A'--C--A};
				\path pic[draw,blue,thick,angle radius=7pt]{right angle= A'--A--C};
				\draw(M) node [below left] {$a$};
			\end{tikzpicture}
		\end{center}
		Có: $\widehat{\left(A'C,(ABC)\right)}=\widehat{A'CA}=45^{\circ}$.\\
		Xét tam giác $A'AC$ vuông tại $A,$ ta có: $\tan\widehat{A'CA}=\dfrac{AA'}{AC}\Rightarrow AA'=a$.\\
		Thể tích khối lăng trụ $ABC.A'B'C'$ là $V=AA'\cdot S_{\triangle ABC}=a\cdot\dfrac{a^2\sqrt{3}}{4}=\dfrac{a^3\sqrt{3}}{4}$.}
\end{ex}
\begin{ex}%[2H1K3-2]%Câu 6.[Kinh Môn - Hải Dương 2019]
	Cho hình lăng trụ tam giác đều $ABC.A'B'C'$ có $AB=4a$, góc giữa đường thẳng $A'C$ và mặt phẳng $(ABC)$ bằng $45^{\circ}$. Thể tích khối lăng trụ $ABC.A'B'C'$ bằng
	\choice
	{$\dfrac{a^3\sqrt{3}}{4}$}
	{$\dfrac{a^3\sqrt{3}}{2}$}
	{\True $16a^3\sqrt{3}$}
	{$\dfrac{a^3\sqrt{3}}{6}$}
	\loigiai{\begin{center}
			\begin{tikzpicture}
				\def\a{4}
				\def\h{4}
				\path 	(0:0) coordinate (C)
				++(0:\a) coordinate (A)
				++(-150:2*\a/3) coordinate (B)
				($(A)+(90:\h)$) coordinate (A')
				($(B)+(90:\h)$) coordinate (B')
				($(C)+(90:\h)$) coordinate (C')
				($(A)!.5!(B)$) coordinate (M)
				($(B)!.5!(C)$) coordinate (N)
				($(A)!.5!(C)$) coordinate (P);
				\draw[dashed,thick] 	(A)--(C) (A')--(C);
				\draw[thick]	(C)--(C') 	(B)--(B')	(A)--(A') (A)--(B)--(C) (A')--(B')--(C')--cycle;
				\foreach \x/\g in {A/0,B/-45,C/180,A'/0,B'/-45,C'/180}
				\fill[black] 	(\x) circle (1pt)
				($(\g:4mm)+(\x)$) node {$\x$};
				\path pic["\scriptsize$45^\circ$", angle eccentricity=2.5,draw,angle radius=9pt]{angle= A--C--A'};
				\foreach \x/\g in {M/-30,N/-150,P/60}
				\draw ($(\g:4mm)+(\x)$) node {$4a$};
			\end{tikzpicture}
		\end{center}
		$ABC.A'B'C'$ là lăng trụ tam giác đều $\Rightarrow ABC.A'B'C'$ là lăng trụ đứng và đáy là tam giác đều.\\
		Ta có:\\
		$A'A\perp(ABC)\Rightarrow\left(\widehat{A'C,(ABC)}\right)=\widehat{A'CA}=45^{\circ}\Rightarrow\triangle A'AC$ vuông cân tại $A\Rightarrow A'A=AC=4a$.\\
		$S_{\triangle ABC}=\dfrac{(AB)^2\sqrt{3}}{4}=\dfrac{(4a)^2\sqrt{3}}{4}=4a^2\sqrt{3}\Rightarrow V_{ABC.A'B'C'}=AA'\cdot S_{\triangle ABC}=4a\cdot 4a^2\sqrt{3}=16a^3\sqrt{3}$.}
\end{ex}
\begin{ex}%[2H1K3-2]%Câu 7.[Mã 104 2017]
	Cho khối lăng trụ đứng $ABC.A'B'C'$ có đáy $ABC$ là tam giác cân với $AB=AC=a$, $\widehat{BAC}=120^{\circ}$. Mặt phẳng $(AB'C')$ tạo với đáy một góc $60^{\circ}$. Tính thể tích $V$ của khối lăng trụ đã cho. 
	\choice
	{\True $V=\dfrac{3a^3}{8}$}
	{$V=\dfrac{9a^3}{8}$}
	{$V=\dfrac{a^3}{8}$}
	{$V=\dfrac{3a^3}{4}$}
	\loigiai{\begin{center}
			\begin{tikzpicture}[font=\footnotesize, line join=round, line cap=round, >=stealth,scale=1]
				\def\a{4}
				\def\h{4}
				\path 	(0:0) coordinate (A')
				++(0:\a) coordinate (C')
				++(-150:7*\a/5) coordinate (B')
				($(A')+(90:\h)$) coordinate (A)
				($(B')+(90:\h)$) coordinate (B)
				($(C')+(90:\h)$) coordinate (C)
				($(B')!.5!(C')$) coordinate (H);
				\draw[dashed,thick] (C')--(A')--(B') (A)--(A') (A)--(B') (A)--(C') (A)--(H)--(A');
				\draw[thick]	(C)--(C') 	(B)--(B') (A)--(B)--(C)--cycle (B')--(C');
				\foreach \x/\g in {A/45,B/180,C/0,A'/45,B'/180,C'/0,H/0}
				\fill[black] (\x) circle (1pt)
				($(\g:4mm)+(\x)$) node {$\x$};
				\path pic["\scriptsize$60^\circ$", angle eccentricity=2,draw,red,angle radius=15pt]{angle= A--H--A'};
				\path pic["\scriptsize$120^\circ$", angle eccentricity=1.5,draw,red,right,angle radius=9pt]{angle= B--A--C};
				\foreach \x/\y in {A/B,A/C,A'/B',A'/C'}{
					\path (\x)--(\y) node[midway,sloped]{\tikz{\draw [shift={(-1.5pt,0)}](-90:3pt)--(90:3pt) [shift={(1.5pt,0)}](-90:3pt)--(90:3pt);}};}
			\end{tikzpicture}
		\end{center}
		Gọi $H$ là trung điểm của $B'C'$, khi đó góc giữa mp $(AB'C')$ và đáy là góc $\widehat{AHA'}=60^{\circ}$.\\
		Ta có $S_{\triangle ABC}=\dfrac{1}{2}AC\cdot AB\cdot\sin 120^{\circ}=\dfrac{a^2\sqrt{3}}{4}$.\\
		$B'C'=BC=\sqrt{AB^2+AC^2-2AB\cdot AC\cdot\cos 120^{\circ}}=\sqrt{a^2+a^2-2\cdot a\cdot a\cdot\dfrac{-1}{2}}=a\sqrt{3}\Rightarrow A'H=\dfrac{2S_{\triangle ABC}}{B'C'}=\dfrac{a}{2}\Rightarrow AA'=A'H\cdot\tan 60^{\circ}=\dfrac{a\sqrt{3}}{2}$.\\
		Vậy $V=S_{\triangle ABC}\cdot AA'=\dfrac{3a^3}{8}$.}
\end{ex}
\begin{ex}%[2H1K3-2]%Câu 8.[Chuyên Vĩnh Phúc 2019] 
	Cho lăng trụ đều $ABC.A'B'C'$. Biết rằng góc giữa $(A'BC)$ và $(ABC)$ là $30^{\circ}$, tam giác $A'BC$ có diện tích bằng $8$. Tính thể tích khối lăng trụ $ABC.A'B'C'$. 
	\choice
	{\True $8\sqrt{3}$}
	{$8$}
	{$3\sqrt{3}$}
	{$8\sqrt{2}$}
	\loigiai{\begin{center}
			\begin{tikzpicture}[font=\footnotesize, line join=round, line cap=round, >=stealth,scale=1]
				\def\a{5}
				\def\h{3.5}
				\path 	(0:0) coordinate (A)
				++(0:\a) coordinate (C)
				++(-150:3*\a/4) coordinate (B)
				($(A)+(90:\h)$) coordinate (A')
				($(B)+(90:\h)$) coordinate (B')
				($(C)+(90:\h)$) coordinate (C')
				($(B)!.5!(C)$) coordinate (M)
				($(A)!.5!(B)$) coordinate (N);
				\draw[dashed,thick] (A')--(C)--(A) (A')--(M)--(A);
				\draw[thick](C)--(C') 	(B)--(B') (A')--(B)	(A)--(A') (A)--(B)--(C) (A')--(B')--(C')--cycle;
				\foreach \x/\g in {A/180,B/-45,C/0,A'/180,B'/-45,C'/0,M/-30}
				\fill[black] 	(\x) circle (1pt)
				($(\g:4mm)+(\x)$) node {$\x$};
				\path pic["\scriptsize$30^\circ$", angle eccentricity=2,draw,red,angle radius=13pt]{angle= A'--M--A};
				\draw(N) node [below left] {$x$};
			\end{tikzpicture}
		\end{center}
		Đặt $AB=x\,(x>0)$, gọi $M$ là trung điểm $BC$.\\
		Ta có $\heva{&(A'BC)\cap(ABC)=BC\\&AM\perp BC\\&A'M\perp BC}\Rightarrow\widehat{\left((A'BC),(ABC)\right)}=\widehat{A'MA}=30^{\circ}$.\\
		Xét $\triangle A'AM$, có $A'M=\dfrac{AM}{\cos30^\circ}=\dfrac{x\sqrt{3}}{2}\cdot\dfrac{2}{\sqrt{3}}=x$.\\
		$S_{\triangle A'BC}=8\Leftrightarrow\dfrac{1}{2}A'M\cdot BC=8\Leftrightarrow x^2=16\Rightarrow x=4$.\\
		Suy ra $A'A=AM\cdot\tan 30^{\circ}=\dfrac{4\cdot\sqrt{3}}{2}\cdot\dfrac{1}{\sqrt{3}}=2$; $S_{\triangle ABC}=\dfrac{16\cdot\sqrt{3}}{4}=4\sqrt{3}$.\\
		Vậy $V_{ABC.A'B'C'}=A'A\cdot S_{\triangle ABC}=2\cdot 4\sqrt{3}=8\sqrt{3}$.}
\end{ex}
\begin{ex}%[2H1K3-2]%Câu 9.[THPT Thiệu Hóa – Thanh Hóa 2019] 
	Cho lăng trụ tam giác đều $ABC.A'B'C'$ có diện tích đáy bằng $\dfrac{a^2\sqrt{3}}{4}$. Mặt phẳng $(A'BC)$ hợp với mặt phẳng đáy một góc $60^{\circ}$. Tính thể tích khối lăng trụ $ABC.A'B'C'$. 
	\choice
	{\True $\dfrac{3a^3\sqrt{3}}{8}$}
	{$\dfrac{a^3\sqrt{3}}{8}$}
	{$\dfrac{5a^3\sqrt{3}}{12}$}
	{$\dfrac{3a^3\sqrt{2}}{8}$}
	\loigiai{\begin{center}
			\begin{tikzpicture}[font=\footnotesize, line join=round, line cap=round, >=stealth,scale=1]
				\def\a{5}
				\def\h{3.5}
				\path 	(0:0) coordinate (A)
				++(0:\a) coordinate (C)
				++(-120:2*\a/5) coordinate (B)
				($(A)+(90:\h)$) coordinate (A')
				($(B)+(90:\h)$) coordinate (B')
				($(C)+(90:\h)$) coordinate (C')
				($(B)!.5!(C)$) coordinate (M);
				\draw[dashed,thick] (A')--(C)--(A) (A')--(M)--(A);
				\draw[thick](C)--(C') (B)--(B') (A)--(A') (A')--(B) (A)--(B)--(C) (A')--(B')--(C')--cycle ;
				\foreach \x/\g in {A/180,B/0,C/-30,A'/180,B'/0,C'/-30,M/0}
				\fill[black] 	(\x) circle (1pt)
				($(\g:4mm)+(\x)$) node {$\x$};	
				\path pic["\scriptsize$60^\circ$",red,thick, angle eccentricity=2,draw,angle radius=13pt]{angle= A'--M--A};
				
			\end{tikzpicture}
		\end{center}
		Vì đáy $ABC$ là tam giác đều có diện tích bằng $\dfrac{a^2\sqrt{3}}{4}\Rightarrow$ Cạnh đáy bằng $a$.\\
		Gọi $M$ trung điểm $BC$, ta có $\heva{&BC\perp AM\\&BC\perp AA'}\Rightarrow BC\perp A'M$.\\
		Từ đó ta có $\widehat{\left((A'BC),(ABC)\right)}=\widehat{(A'M,AM)}=\widehat{A'MA}=60^{\circ}$.\\
		Xét $\triangle A'AM$ ta có $AA'=AM\cdot\tan 60^{\circ}=\dfrac{3a}{2}$.\\
		Thể tích lăng trụ $ABC.A'B'C'$ là $V_{ABC.A'B'C'}=AA'\cdot S_{\triangle ABC}=\dfrac{3a^3\sqrt{3}}{8}$.}
\end{ex}
\begin{ex}%[2H1K3-2]%Câu 10.[Hội 8 trường chuyên ĐBSH - 2019] 
	Cho lăng trụ tam giác đều $ABC.A'B'C'$ có cạnh đáy bằng $a$ và $AB'$ vuông góc với $BC'$. Tính thể tích $V$ của khối lăng trụ đã cho. 
	\choice
	{$V=\dfrac{a^3\sqrt{6}}{4}$}
	{\True $V=\dfrac{a^3\sqrt{6}}{8}$}
	{$V=a^3\sqrt{6}$}
	{$V=\dfrac{7a^3}{8}$}
	\loigiai{\begin{center}
			\begin{tikzpicture}[font=\footnotesize, line join=round, line cap=round, >=stealth,scale=1]
				\def\a{5}
				\def\h{3.5}
				\path 	(0:0) coordinate (B')
				++(0:\a) coordinate (A')
				++(-160:3*\a/4) coordinate (C')
				($(A')+(90:\h)$) coordinate (A)
				($(B')+(90:\h)$) coordinate (B)
				($(C')+(90:\h)$) coordinate (C);
				\draw[dashed,thick] (A')--(B')--(A) ;
				\draw[thick](C)--(C') (B)--(B') (A)--(A') (B)--(C') (A)--(B)--(C)--cycle
				(B')--(C')--(A');
				\foreach \x/\g in {A/0,B/180,C/-30,A'/0,B'/180,C'/-30}
				\fill[black] 	(\x) circle (1pt)
				($(\g:4mm)+(\x)$) node {$\x$};
				
			\end{tikzpicture}
		\end{center}
		Đặt $\overrightarrow{x}=\overrightarrow{BA},\overrightarrow{y}=\overrightarrow{BC},\overrightarrow{z}=\overrightarrow{BB'},$ theo giả thiết $AB'\perp BC'$ nên.\\
		$\begin{aligned}&\overrightarrow{AB'}\cdot\overrightarrow{BC'}=0\Leftrightarrow\left(\overrightarrow{z}-\overrightarrow{x}\right)\left(\overrightarrow{y}+\overrightarrow{z}\right)=0\Leftrightarrow\overrightarrow{z}\cdot\overrightarrow{y}+\left|\overrightarrow{z}\right|^2-\overrightarrow{x}\cdot\overrightarrow{y}-\overrightarrow{x}\cdot\overrightarrow{z}=0\Leftrightarrow\left|\overrightarrow{z}\right|^2=\overrightarrow{x}\cdot\overrightarrow{y}\\&\Leftrightarrow\left|\overrightarrow{z}\right|^2=\left|\overrightarrow{x}\right|\left|\overrightarrow{y}\right|\cos{60}^\circ=\dfrac{a^2}{2}\Rightarrow\left|\overrightarrow{z}\right|=\dfrac{a\sqrt{2}}{2}.\end{aligned}$\\
		Vậy $V_{ABC.A'B'C'}=S_{\triangle ABC}\cdot BB'=\dfrac{a^2\sqrt{3}}{4}\cdot\dfrac{a\sqrt{2}}{2}=\dfrac{a^3\sqrt{6}}{8}$.}
\end{ex}
\begin{ex}%[2H1K3-2]%Câu 11.
	Cho hình lăng trụ đứng $ABC.A'B'C'$ có đáy $ABC$ là tam giác đều cạnh bằng $a$ và $(A'BC)$ hợp với mặt đáy $ABC$ một góc $30^{\circ}$. Tính thể tích $V$ của khối lăng trụ $ABC.A'B'C'$. 
	\choice
	{\True $V=\dfrac{a^3\sqrt{3}}{8}$}
	{$V=\dfrac{a^3\sqrt{3}}{12}$}
	{$V=\dfrac{a^3\sqrt{3}}{24}$}
	{$V=\dfrac{3a^3}{8}$}
	\loigiai{
		\begin{center}
			\begin{tikzpicture}[font=\footnotesize, line join=round, line cap=round, >=stealth,scale=1]
				\def\a{4}
				\def\h{4.5}
				\path 	(0:0) coordinate (A)
				++(0:\a) coordinate (C)
				++(-150:3*\a/5) coordinate (B)
				($(A)+(90:\h)$) coordinate (A')
				($(B)+(90:\h)$) coordinate (B')
				($(C)+(90:\h)$) coordinate (C')
				($(B)!.5!(C)$) coordinate (H);
				\draw[dashed,thick] (A')--(C)--(A) (A')--(H)--(A);
				\draw[thick](C)--(C') 	(B)--(B') (A')--(B)	(A)--(A') (A)--(B)--(C) (A')--(B')--(C')--cycle;
				\foreach \x/\g in {A/180,B/-45,C/0,A'/180,B'/-45,C'/0,H/-30}
				\fill[black] 	(\x) circle (1pt)
				($(\g:4mm)+(\x)$) node {$\x$};
				\path pic["\scriptsize$30^\circ$", angle eccentricity=2,draw,red,angle radius=9pt]{angle= A'--H--A};
				\path pic[draw,angle radius=7pt]{right angle= A--H--B};
			\end{tikzpicture}
		\end{center}
		Gọi $H$ là hình chiếu vuông góc của $A$ trên $BC$. Suy ra $AH\perp BC$.\\
		$A'H\perp BC$.\\
		Mà $(ABC)\cap(A'BC)=BC$.\\
		$ \Rightarrow $ Góc giữa $(A'BC)$ và $(ABC)$ bằng góc $(AH; A'H)=\widehat{AHA'}=30^{\circ}$.\\
		Ta có: $ABC$ là tam giác đều cạnh bằng $a$ nên $AH=\dfrac{a\sqrt{3}}{2}$, $A'A=AH\cdot\tan 30^{\circ}=\dfrac{a}{2}$.\\
		Thể tích khối lăng trụ $ABC.A'B'C'$ là $V=A'A\cdot S_{\triangle ABC}=\dfrac{a}{2}\cdot\dfrac{a^2\sqrt{3}}{4}=\dfrac{a^3\sqrt{3}}{8}$.}
\end{ex}
\begin{ex}%[2H1K3-2]%Câu 12
	Cho lăng trụ đứng $ABC.A'B'C'$ có đáy $ABC$ là tam giác vuông tại $A$ và $AB=a$, $AC=a\sqrt{3}$, mặt phẳng $\left(A'BC\right)$ tạo với đáy một góc $30^\circ.$ Thể tích của khối lăng trụ $ABC.A'B'C'$ bằng
	\choice
	{$\dfrac{a^3\sqrt{3}}{12}$}{$\dfrac{a^3\sqrt{3}}{3}$}{$\dfrac{3\sqrt{3}\,a^3}{4}$}{\True $\dfrac{\sqrt{3}\,a^3}{4}$}
	\loigiai{
		\immini{* Xác định góc giữa mặt phẳng $\left(A'BC\right)$ và mặt phẳng đáy:\\
			Trong mặt phẳng $\left(ABC\right)$, dựng $AH \perp BC$ với $H$ nằm trên cạnh $BC.$\\
			Theo định lý ba đường vuông góc, ta có: $A'H \perp BC.$\\
			Vậy $\widehat{\left(\left(A'BC\right);\left(ABC\right)\right)}=\widehat{A'HA}=30^\circ.$\\
			* Xét tam giác $ABC$ có: $\dfrac{1}{AH^2}=\dfrac{1}{AB^2}+\dfrac{1}{AC^2}=\dfrac{1}{a^2}+\dfrac{1}{3a^2}\\
			\Rightarrow AH=\dfrac{a\sqrt{3}}{2}$\\
			Diện tích $B$ của tam giác $ABC$ là:$B=,\dfrac{AB.AC}{2}=\dfrac{a^2\sqrt{3}}{2}$}
		{\begin{tikzpicture}[line join=round,line cap=round, font=\footnotesize,scale=0.7,>=stealth]
				\def \a{5}
				\path
				(0,0) coordinate (A)
				(0:\a) coordinate (C)
				(-20:{0.75*\a}) coordinate (B)
				(90:{\a}) coordinate (A')
				($(B)+(A')$) coordinate (B')
				($(A')+(C)$) coordinate (C')
				($(B)!0.5!(C)$) coordinate (H);
				\draw (A)--(B)--(C)--(C')--(A')--(B')--(C') (A')--(A) (B')--(B)--(A');
				\draw[dashed] (A)--(C) (A)--(H)--(A')--(C);
				\draw pic[draw,angle radius=2mm] {right angle =A--H--B};	
				\foreach \x/\g in {A/180, B/-90, C/0, C'/0, A'/180, B'/90, H/-30 }
				\fill[black] (\x) circle (1pt)+(\g:.35)node{$\x$};
		\end{tikzpicture}}
		* Xét tam giác $A'HA$ vuông tại $A$, ta có: $A'A=AH.\tan 30^\circ =\dfrac{a}{2}$ \\
		Thể tích khối lăng trụ $ABC.A'B'C'$ bằng $V=B.h=\dfrac{a^2\sqrt{3}}{2}.\dfrac{a}{2}=\dfrac{\sqrt{3}a^3}{4}$
	}
\end{ex}
\begin{ex}%[2H1K3-2]%Câu 13
	Cho hình lăng trụ đứng, có đáy $ABC$ là tam giác vuông cân tại $A, AB=a\sqrt{2}$, góc giữa mp$\left(AB'C'\right)$ và mp$\left( ABC\right)$ bằng $60^\circ.$ Thể tích khối lăng trụ bằng
	\choice
	{$3a^3$}{$3\sqrt{3}a^3$}{$a^3$}{\True $\sqrt{3}a^3$}
	\loigiai{
		\immini{Gọi $I$ là trung điểm của cạnh $B'C'$\\
			Ta có $\widehat{\left(\left(AB'C'\right),\left(ABC\right)\right)}=\widehat{\left(\left(AB'C'\right),\left(A'B'C'\right)\right)}$ \\
			và $\left(AB'C'\right)\cap \left(A'B'C'\right)=B'C'$.\\
			Vì $ABC$ là tam giác vuông cân tại $A$ nên hai mặt bên $ABB'A'$ và $ACC'A'$ là hai hình chữ nhật bằng nhau.\\
			Do đó $AC'=AB'\Rightarrow \Delta AB'C'$ là tam giác cân tại $A$ $\Rightarrow AI\perp B'C'$\\
			Vì $\Delta A'B'C'$ là tam giác vuông cân tại $A'$ nên $A'I\perp B'C'.$\\
			Như vậy $\widehat{\left(\left(AB'C'\right),\left(ABC\right)\right)}=\widehat{AIA'}=60^\circ.$\\
			Ta có $A'I=\dfrac{1}{2}BC=a\Rightarrow AA'=A'I.\tan {{60}^\circ}=a\sqrt{3}$.\\
			$\Rightarrow {{V}_{ABC.A'B'C'}}=AA'.{{S}_{ABC}}=a\sqrt{3}.\dfrac{1}{2}{{\left(a\sqrt{2} \right)}^2}=a^3\sqrt{3}$}
		{\begin{tikzpicture}[line join=round,line cap=round, font=\footnotesize,scale=0.7,>=stealth]
				\def \a{5}
				\path
				(0,0) coordinate (B')
				(0:\a) coordinate (C')
				(-20:{0.75*\a}) coordinate (A')
				(90:{0.75*\a}) coordinate (B)
				($(A')+(B)$) coordinate (A)
				($(B)+(C')$) coordinate (C)
				($(B')!0.5!(C')$) coordinate (I);
				\draw (B')--(A')--(C')--(C)--(B)--(A)--(C) (B)--(B')--(A)--(C') (A)--(A');
				\draw[dashed] (B')--(C') (A')--(I)--(A);
				\draw pic[draw,angle radius=2mm] {angle =A'--I--A};	
				\foreach \x/\g in {B'/180, A'/-90, C'/0, C/0, B/180, A/90, I/150 }
				\fill[black] (\x) circle (1pt)+(\g:.45)node{$\x$};
		\end{tikzpicture}}
	}
\end{ex}
\begin{ex}%[2H1K3-2]%Câu 14
	Cho hình lăng trụ đều $ABC.A'B'C'.$ Biết khoảng cách từ điểm $C$ đến mặt phẳng $\left(ABC'\right)$ bằng $a$, góc giữa hai mặt phẳng $\left(ABC'\right)$ và $\left(BCC'B'\right)$ bằng $\alpha $ với $\cos \alpha =\dfrac{1}{2\sqrt{3}}.$ Tính thể tích khối lăng trụ $ABC.A'B'C'$
	\choice
	{$V=\dfrac{3a^3\sqrt{2}}{4}$}{\True $V=\dfrac{3a^3\sqrt{2}}{2}$}
	{$V=\dfrac{a^3\sqrt{2}}{2}$}{$V=\dfrac{3a^3\sqrt{2}}{8}$}
	\loigiai{
		\immini{Gọi $M,N$ lần lượt là trung điểm của $AB$ và $BC$\\
			Do $\heva{AB \perp CC'& \\ AB \perp CM} \Rightarrow AB\perp \left( MCC' \right)\Rightarrow \left( ABC' \right)\bot \left( MCC' \right)$\\
			Kẻ $CK$ vuông góc với $CM$ tại $K$ thì ta được $CK\perp \left(ABC'\right)$. \\
			Do đó $CK=d\left(C;\left(ABC'\right) \right)=a.$\\
			Đặt $BC=x,CC'=y,\left(x>0,y>0 \right)$, ta được: $CM=\dfrac{x\sqrt{3}}{2}$\\
			$\dfrac{1}{CM^2}+\dfrac{1}{C{{{C'}}^2}}=\dfrac{1}{CK^2}\Leftrightarrow \dfrac{4}{3x^2}+\dfrac{1}{y^2}=\dfrac{1}{a^2}\quad(1)$\\
			Kẻ $CE\perp BC'$ tại $E$, ta được $\widehat{KEC}=\alpha $,\\
			$EC=\dfrac{KC}{\sin \alpha }=\dfrac{a}{\sqrt{1-\dfrac{1}{12}}}=a\sqrt{\dfrac{12}{11}}$}			
		{\begin{tikzpicture}[line join=round,line cap=round, font=\footnotesize,scale=0.7,>=stealth]
				\def \a{5}
				\path
				(0,0) coordinate (A)
				(0:\a) coordinate (C)
				(-10:{0.75*\a}) coordinate (B)
				(90:{\a}) coordinate (A')
				($(B)+(A')$) coordinate (B')
				($(A')+(C)$) coordinate (C')
				($(A)!0.5!(B)$) coordinate (M)
				($(C')!(C)!(M)$) coordinate (K)
				($(C')!(C)!(B)$) coordinate (E)
				($(C')!(K)!(B)$) coordinate (E);
				\draw (A)--(B)--(C)--(C')--(A')--(B')--(C') (A')--(A) (B')--(B)--(C') (C)--(E);				
				\draw[dashed] (A)--(C)--(M)--(C')--cycle (C)--(K)--(E);
				\draw pic[draw,angle radius=2mm] {right angle =C--K--M}
				pic[draw,angle radius=1mm] {right angle =C'--E--K}
				pic[draw,angle radius=1mm] {right angle =C--E--C'}
				pic[draw,angle radius=1mm] {angle =K--E--C};	
				\foreach \x/\g in {A/180, B/-90, C/0, C'/0, A'/180, B'/90, M/-80, K/180,E/0 }
				\fill[black] (\x) circle (1pt)+(\g:.35)node{$\x$};
		\end{tikzpicture}}
		Lại có $\dfrac{1}{x^2}+\dfrac{1}{y^2}=\dfrac{1}{CE^2}=\dfrac{11}{12a^2}\quad(2)$\\
		Giải $\left( 1 \right),\left( 2 \right)$ ta được: $x=2a,y=\dfrac{a\sqrt{6}}{2}.$\\
		Thể tích khối lăng trụ $ABC.A'B'C'$ là:
		$V=y.\dfrac{x^2\sqrt{3}}{4}=\dfrac{a\sqrt{6}}{2}.\dfrac{4a^2\sqrt{3}}{4}=\dfrac{3\sqrt{2}a^3}{2}$
	}
\end{ex}
\begin{ex}%[2H1K3-2]%Câu 15
	(THPT Minh Khai - 2019) Cho khối lăng trụ tam giác đều $ABC.A'B'C'$ có $A'B=a\sqrt{6}$, đường thẳng $A'B$ vuông góc với đường thẳng $B'C.$ Tính thể tích khối lăng trụ đã cho theo $a$?
	\choice
	{\True $\dfrac{a^3\sqrt{6}}{3}$}{$a^3\sqrt{6}$}{$\dfrac{3a^3}{4}$}{$\dfrac{9a^3}{4}$}
	\loigiai{\immini{
			Dựng hình hộp $ABCD.ABCD$ khi đó tứ giác $ABCD$ là hình thoi.\\
			Đặt $AB=x,\Rightarrow AD=x.$\\
			Tam giác $ABD$ có góc $\widehat{BAD}=120^\circ;$ áp dụng định lý côsin ta có:\\
			$BD^2=AB^2+AD^2-2AB.AD.\cos \widehat{BAD}\\
			=x^2+x^2-2x.x.\cos 120^\circ=3x^2.$\\
			Ta có: $A'B=a\sqrt{6}\Rightarrow A'D=a\sqrt{6}$\\
			Ta có: $A'D \parallel B'C \Rightarrow A'B\perp A'D \Rightarrow \Delta A'BD$ vuông tại $A'$\\
			$\Rightarrow BD^2=A'B^2+A'D^2 \Leftrightarrow 3x^2=12a^2\Leftrightarrow x^2=4a^2\Rightarrow x=2a.$\\
			Chiều cao hình trụ $A{{A'}^2}=A'B^2-AB^2$$=6a^2-4a^2=2a^2\\
			\Rightarrow AA'=a\sqrt{2}.$\\
			$\Rightarrow V_{ABC.A'B'C'}=\dfrac{1}{3}AA'.S_{\Delta ABC}=\dfrac{1}{3}a\sqrt{2}.\dfrac{1}{2}.2a.2a.\dfrac{\sqrt{3}}{2}=\dfrac{\sqrt{6}a^3}{3}$.}
		{\begin{tikzpicture}[line join=round,line cap=round, font=\footnotesize,scale=.6,>=stealth]
				\def \a{4}
				\path
				(0,0) coordinate (A)
				(0:\a) coordinate (D)
				(-120:{0.6*\a}) coordinate (B)
				($(D)+(B)$) coordinate (C)
				(90:\a) coordinate (A')
				($(A')+(D)$) coordinate (D')
				($(A')+(B)$) coordinate (B')
				($(A')+(C)$) coordinate (C')
				;
				\draw
				(B')--(B)--(C)--(C')--(B')--(A')--(D')--(C')(C)--(D)--(D') (C)--(B') (A')--(C')
				;
				\draw[dashed](B)--(A)--(D)(A)--(A')--(D)--(B)--(A') (A)--(C);			
				\foreach \x/\g in {A/180,B/180,C/0,D/0,A'/180,B'/180,C'/0,D'/0}\fill[black] (\x) circle (2pt)+(\g:.4)node{$\x$};
		\end{tikzpicture}}
	}
\end{ex}
\begin{ex}%[2H1K3-2]%Câu 16
	(Chuyên Lam Sơn Thanh Hóa 2019) Cho khối lăng trụ đều $ABC.A'B'C'$ có cạnh đáy bằng $a.$ Khoảng cách từ điểm $A'$ đến mặt phẳng $\left(AB'C'\right)$ bằng $\dfrac{2a\sqrt{3}}{\sqrt{19}}.$ Thể tích của khối lăng trụ đã cho là
	\choice
	{$\dfrac{a^3\sqrt{3}}{4}$}{$\dfrac{a^3\sqrt{3}}{6}$}
	{\True $\dfrac{a^3\sqrt{3}}{2}$}{$\dfrac{3a^3}{2}$}
	\loigiai{\immini{
			Gọi $M$ là trung điểm của $B'C'.$\\
			Ta có $\heva{AA'\perp B'C'&\\A'M \perp B'C'}	\Rightarrow B'C'\perp \left(AA'M\right)\\
			\Rightarrow \left(AB'C'\right)\perp \left(AA'M\right)$ theo giao tuyến $AM.$\\
			Kẻ $A'H\perp AM$ trong mặt phẳng $\left(AA'M \right)$, suy ra $A'H\perp \left(AB'C'\right).$\\
			Vậy khoảng cách từ $A'$ đến mặt phẳng $\left(AB'C' \right)$ là $A'H=\dfrac{2a\sqrt{3}}{\sqrt{19}}.$\\
			Ta có $\dfrac{1}{A'H^2}=\dfrac{1}{A'A^2}+\dfrac{1}{A'M^2}\Rightarrow \dfrac{1}{A'A^2}=\dfrac{1}{A'H^2}-\dfrac{1}{A'M^2}=\dfrac{1}{4a^2}$\\
			$\Rightarrow A'A=2a.$\\
			Vậy thể tích khối lăng trụ là $V=AA'.{{S}_{A'B'C'}}=2a.\dfrac{a^2\sqrt{3}}{4}=\dfrac{a^3\sqrt{3}}{2}.$}
		{\begin{tikzpicture}[line join=round,line cap=round, font=\footnotesize,scale=0.7,>=stealth]
				\def \a{5}
				\path
				(0,0) coordinate (A')
				(0:\a) coordinate (C')
				(-20:{0.75*\a}) coordinate (B')
				(90:{\a}) coordinate (A)
				($(B')+(A)$) coordinate (B)
				($(A)+(C')$) coordinate (C)
				($(B')!0.5!(C')$) coordinate (M)
				($(A)!(A')!(M)$) coordinate (H);
				\draw (A')--(B')--(C')--(C)--(A)--(B)--(C) (A)--(A') (B')--(B);
				\draw[dashed] (H)--(A')--(C') (A')--(M)--(A)--(C');
				\draw pic[draw,angle radius=1.25mm] {right angle =A'--M--B'}
				pic[draw,angle radius=2mm] {right angle =A'--H--M};	
				\foreach \x/\g in {A'/180, B'/-90, C'/0, C/0, A/180, B/90, M/-30,H/0 }
				\fill[black] (\x) circle (1pt)+(\g:.35)node{$\x$};
		\end{tikzpicture}}
	}
\end{ex}
\begin{ex}%[2H1B3-2]%Câu 17
	(Chuyên Vĩnh Phúc - 2018)Cho lăng trụ đứng $ABC.A'B'C'$ đáy là tam giác vuông cân tại $B,AC=a\sqrt{2}$, biết góc giữa $\left( A'BC \right)$ và đáy bằng $60^\circ.$ Tính thể tích $V$ của khối lăng trụ.
	\choice
	{\True $V=\dfrac{a^3\sqrt{3}}{2}$}{$V=\dfrac{a^3\sqrt{3}}{3}$}
	{$V=\dfrac{a^3\sqrt{3}}{6}$}{$V=\dfrac{a^3\sqrt{6}}{6}$}
	\loigiai{\immini{
			Tam giác $ABC$ vuông cân tại $B,AC=a\sqrt{2}\\
			\Rightarrow AB=BC=a \Rightarrow S_{\Delta ABC}=\dfrac{a^2}{2}.$\\
			$\left(\widehat{\left(A'BC\right),\left(ABC\right)}\right)=\widehat{A'BA}=60^\circ \Rightarrow A'A=AB.\tan 60^\circ=a\sqrt{3}$.\\
			Vậy: $V_{ABC.A'B'C'}=S_{\Delta ABC}.A'A=\dfrac{a^2}{2}.a\sqrt{3}=\dfrac{a^3\sqrt{3}}{2}.$}
		{\begin{tikzpicture}[line join=round,line cap=round, font=\footnotesize,scale=0.7,>=stealth]
				\def \a{5}
				\path
				(0,0) coordinate (A)
				(0:\a) coordinate (C)
				(-20:{0.75*\a}) coordinate (B)
				(90:{\a}) coordinate (A')
				($(B)+(A')$) coordinate (B')
				($(A')+(C)$) coordinate (C');
				\draw (A)--(B)--(C)--(C')--(A')--(B')--(C') (B)--(A')--(A) (B)--(B');
				\draw[dashed] (A)--(C) (A')--(C);
				\draw pic[draw,angle radius=2mm] {right angle =A--B--C}
				pic[draw,angle radius=4mm] {angle =A'--B--A};	
				\foreach \x/\g in {A/180, B/-90, C/0, C'/0, A'/180, B'/90 }
				\fill[black] (\x) circle (1pt)+(\g:.35)node{$\x$};
		\end{tikzpicture}}
	}
\end{ex}
\begin{ex}%[2H1K3-2]%Câu 18
	(Liên Trường - Nghệ An 2018) Cho hình lăng trụ tam giác đều $ABC.A'B'C'$ có góc giữa hai mặt phẳng $\left(A'BC\right)$ và $\left( ABC \right)$ bằng $60^\circ$, cạnh $AB=a.$ Tính thể tích $V$ của khối lăng trụ $ABC.A'B'C'$
	\choice
	{$V=\dfrac{\sqrt{3}}{4}a^3$}{$V=\dfrac{3}{4}a^3$}
	{\True $V=\dfrac{3\sqrt{3}}{8}a^3$}{$V=\sqrt{3}a^3$}
	\loigiai{\immini{
			Gọi $M$ là trung điểm của $BC$ suy ra $AM\perp BC \quad (1)$\\
			Ta có $\heva{BC \perp AB&\\BC\perp AA'}\Rightarrow BC\perp A'M \quad(2).$\\
			Mặt khác $\left(ABC\right)\cap \left(A'BC\right)=BC, \quad (3)$\\
			Từ $\left(1\right), \left(2\right), \left(3\right)$ suy ra $\left(\widehat{\left(ABC\right);\left(A'BC\right)} \right)=\widehat{A'MA}=60^\circ.$\\
			Vì tam giác $ABC$ đều nên $S_{\Delta ABC}=\dfrac{a^2\sqrt{3}}{4}$ và $AM=\dfrac{a\sqrt{3}}{2}.$\\
			Ta có $AA'=AM.\tan 60^\circ =\dfrac{3a}{2}.$\\
			Vậy $V_{ABC.A'B'C'}=AA'.S_{\Delta ABC} = \dfrac{3a}{2}.\dfrac{a^2 \sqrt{3}}{4}=\dfrac{3a^3 \sqrt{3}}{8}.$}
		{\begin{tikzpicture}[line join=round,line cap=round, font=\footnotesize,scale=0.7,>=stealth]
				\def \a{5}
				\path
				(0,0) coordinate (A)
				(0:\a) coordinate (C)
				(-20:{0.75*\a}) coordinate (B)
				(90:{\a}) coordinate (A')
				($(B)+(A')$) coordinate (B')
				($(A')+(C)$) coordinate (C')
				($(B)!0.5!(C)$) coordinate (M);
				\draw (A)--(B)--(C)--(C')--(A')--(B')--(C') (B)--(A')--(A) (B)--(B');
				\draw[dashed] (A)--(C) (A)--(M)--(A')--(C);
				\draw pic[draw,angle radius=2mm] {right angle =A--M--B}
				pic[draw,angle radius=2mm] {right angle =A'--M--C}
				pic[draw,angle radius=3mm] {angle =A'--M--A};	
				\foreach \x/\g in {A/180, B/-90, C/0, C'/0, A'/180, B'/90, M/-30 }
				\fill[black] (\x) circle (1pt)+(\g:.35)node{$\x$};
		\end{tikzpicture}}
	}
\end{ex}
\begin{ex}%[2H1K3-2]%Câu 19
	(THPT Triệu Thị Trinh - 2018) Cho khối lăng trụ tam giác đều $ABC.A'B'C'$ có cạnh đáy là $a$ và khoảng cách từ $A$ đến mặt phẳng $\left(A'BC\right)$ bằng $\dfrac{a}{2}.$ Thể tích của khối lăng trụ bằng:
	\choice
	{$\dfrac{3\sqrt{2}a^3}{12}$}{$\dfrac{\sqrt{2}a^3}{16}$}
	{\True $\dfrac{3a^3\sqrt{2}}{16}$}{$\dfrac{3a^3\sqrt{2}}{48}$}
	\loigiai{\immini{
			Gọi $I$ là trung điểm của $BC$ và $H$ là hình chiếu vuông góc của $A$ trên $A'I.$\\
			Khi đó ta có:
			$d\left(A,\left(A'BC\right)\right)=AH=\dfrac{a}{2}.$\\
			Trong tam giác vuông $AA'I$ ta có:
			$\dfrac{1}{AH^2}=\dfrac{1}{A{{{A'}}^2}}+\dfrac{1}{AI^2}$$\\
			\Rightarrow \dfrac{1}{A{{{A'}}^2}}=\dfrac{1}{AH^2}-\dfrac{1}{AI^2}=\dfrac{1}{{{\left( \dfrac{a}{2} \right)}^2}}-\dfrac{1}{{{\left( \dfrac{a\sqrt{3}}{2} \right)}^2}}=\dfrac{4}{a^2}-\dfrac{4}{3a^2}=\dfrac{8}{3a^2}.$ Suy ra: $AA'=\dfrac{a\sqrt{6}}{4}$\\
			Thể tích khối lăng trụ là: $V={{S}_{\Delta ABC}}.AA'=\dfrac{a^2\sqrt{3}}{4}\cdot \dfrac{a\sqrt{6}}{4}=\dfrac{3a^3\sqrt{2}}{16}.$}
		{\begin{tikzpicture}[line join=round,line cap=round, font=\footnotesize,scale=0.7,>=stealth]
				\def \a{5}
				\path
				(0,0) coordinate (A')
				(0:\a) coordinate (C')
				(-60:{0.5*\a}) coordinate (B')
				(90:{\a}) coordinate (A)
				($(B')+(A)$) coordinate (B)
				($(A)+(C')$) coordinate (C)
				($(B)!0.5!(C)$) coordinate (I)
				($(A')!(A)!(I)$) coordinate (H);
				\fill[pattern=dots] (A')--(B)--(C);
				\draw (A')--(B')--(C')--(C)--(A)--(B)--(C) (I)--(A)--(A') (B')--(B)--(A');
				\draw[dashed] (A')--(C') (C)--(A')--(I) (A)--(H);
				\draw pic[draw,angle radius=2mm] {right angle =A--I--B}
				pic[draw,angle radius=2mm] {right angle =A'--H--A};	
				\foreach \x/\g in {A'/180, B'/-90, C'/0, C/0, A/180, B/180, I/90,H/0 }
				\fill[black] (\x) circle (1pt)+(\g:.35)node{$\x$};
		\end{tikzpicture}}
	}
\end{ex}
\begin{ex}%[2H1K3-2]%Câu 20
	(THPT Tứ Kỳ - Hải Dương - 2018) Cho khối lăng trụ đứng ${ABC.A'B'C'}$ có đáy ${ABC}$ là tam giác cân với $AB=AC=a$, $\widehat{BAC}=120^\circ$, mặt phẳng $(A'BC')$ tạo với đáy một góc $60{}^\circ.$ Tính thể tích của khối lăng trụ đã cho
	\choice
	{$V=\dfrac{3a^3}{8}$}{$V=\dfrac{9a^3}{8}$}
	{$\dfrac{\sqrt{3}a^3}{8}$}{\True $V=\dfrac{3\sqrt{3}a^3}{8}$}
	\loigiai{\immini{
			Hạ $B'I\perp A'C'.$ Khi đó ta có $\left(\widehat{\left(A'BC'\right),\left(ABC\right)} \right)=\widehat{B'IB}=60^\circ.$\\
			Vì $\widehat{B'A'C'}=120^\circ \Rightarrow \widehat{B'A'I}=60^\circ.$ Do đó $\sin 60^\circ =\dfrac{B'I}{B'A}\Leftrightarrow B'I=\dfrac{a\sqrt{3}}{2}.$\\
			Suy ra $\tan \widehat{B'IB}=\dfrac{BB'}{B'I}\Leftrightarrow \tan 60^\circ =\dfrac{BB'}{B'I}\Leftrightarrow BB'=\dfrac{a\sqrt{3}}{2}.\sqrt{3}=\dfrac{3a}{2}.$\\
			Mặt khác $S_{\Delta ABC}=\dfrac{1}{2}.AI.BC=\dfrac{1}{2}.\dfrac{a}{2}.a\sqrt{3}$$=\dfrac{a^2\sqrt{3}}{4}.$\\
			Vậy thể tích khối chóp là $V=B.h=\dfrac{a^2\sqrt{3}}{4}.\dfrac{a3}{2}=\dfrac{3\sqrt{3}a^3}{8}.$}
		{\begin{tikzpicture}[line join=round,line cap=round, font=\footnotesize,scale=0.7,>=stealth]
				\def \a{5}
				\path
				(0,0) coordinate (B')
				(0:\a) coordinate (C')
				(-20:{0.75*\a}) coordinate (A')
				(90:{\a}) coordinate (B)
				($(A')+(B)$) coordinate (A)
				($(B)+(C')$) coordinate (C)
				($(A')!(B')!(C')$) coordinate (I);
				\draw (C')--(C)--(B)--(A)--(C) (I)--(B)--(B') (C')--(A')--(A) (B')--(I)--(A');
				\draw[dashed] (A')--(B')--(C');
				\draw pic[draw,angle radius=2mm] {right angle =B'--I--A'};	
				\foreach \x/\g in {B'/180, A'/-90, C'/0, C/0, B/180, A/90,I/0}
				\fill[black] (\x) circle (1pt)+(\g:.35)node{$\x$};
		\end{tikzpicture}}
	}
\end{ex}
\begin{ex}%[2H1K3-2]%Câu 21
	(THPT Yên Lạc - 2018) Cho hình lăng trụ đều $ABC.A'B'C'$ có cạnh đáy bằng $a.$ Đường thẳng $AB'$ tạo với mặt phẳng $\left(BCC'B'\right)$ một góc $30^\circ.$ Thể tích khối lăng trụ $ABC.A'B'C'$ theo $a$
	\choice
	{$\dfrac{3a^3}{4}$}{$\dfrac{a^3}{4}$}
	{$\dfrac{a^3\sqrt{6}}{12}$}{\True $\dfrac{a^3\sqrt{6}}{4}$}
	\loigiai{\immini{
			Gọi $M$ là trung điểm của cạnh $BC$.\\
			Do $ABC.A'B'C'$ là hình lăng trụ tam giác đều nên ta có:\\
			$AM\perp \left(BCC'B'\right)\Rightarrow \left(AB',\left(BCC'B'\right)\right)=\widehat{AB'M}=30^\circ.$\\
			Xét tam giác vuông $AB'M$ ta có $\tan 30^\circ =\dfrac{AM}{AB'}\\
			\Leftrightarrow AB'=\dfrac{AM}{\tan 30^\circ }\Leftrightarrow AB'=\dfrac{3a}{2}.$\\
			Xét tam giác vuông $B'BM$ ta có\\ $BB'=\sqrt{B'M^2-BM^2}=\sqrt{\dfrac{9a^2}{4}-\dfrac{a^2}{4}}$$=a\sqrt{2}.$}
		{\begin{tikzpicture}[line join=round,line cap=round, font=\footnotesize,scale=0.7,>=stealth]
				\def \a{5}
				\path
				(0,0) coordinate (A)
				(0:\a) coordinate (C)
				(-20:{0.75*\a}) coordinate (B)
				(90:{0.75*\a}) coordinate (A')
				($(B)+(A')$) coordinate (B')
				($(A')+(C)$) coordinate (C')
				($(B)!0.5!(C)$) coordinate (M);
				\draw (A)--(B)--(C)--(C')--(A')--(B')--(C') (A')--(A)--(B')--(M) (B')--(B);
				\draw[dashed] (M)--(A)--(C) ;
				\draw pic[draw,angle radius=2mm] {right angle =A--M--B}
				pic[draw,angle radius=2mm] {right angle =B'--M--C}
				pic[draw,angle radius=3mm] {angle =A--B'--M};	
				\foreach \x/\g in {A/180, B/-90, C/0, C'/0, A'/180, B'/90, M/0 }
				\fill[black] (\x) circle (1pt)+(\g:.45)node{$\x$};
		\end{tikzpicture}}
		Thể tích khối lăng trụ $ABC.A'B'C'$ là $V_{ABC.A'B'C'}=\dfrac{1}{2}AB.AC.\sin 60^\circ .BB'=\dfrac{a^3\sqrt{6}}{4}.$
	}
\end{ex}
\begin{ex}%[2H1K3-2]%Câu 22
	(THPT Xuân Hòa - 2018) Cho hình lăng trụ đứng$ABC.A'B'C',$ biết đáy $ABC$ là tam giác đều cạnh $a.$ Khoảng cách từ tâm $O$ của tam giác $ABC$ đến mặt phẳng $\left(A'BC\right)$ bằng $\dfrac{a}{6}.$ Tính thể tích khối lăng trụ $ABC.A'B'C'$
	\choice
	{$\dfrac{3a^3\sqrt{2}}{8}$}{$\dfrac{3a^3\sqrt{2}}{28}$}
	{$\dfrac{3a^3\sqrt{2}}{4}$}{\True $\dfrac{3a^3\sqrt{2}}{16}$}
	\loigiai{\immini{
			Diện tích đáy là $B=S_{\Delta ABC}=\dfrac{a^2\sqrt{3}}{4}.$\\
			Chiều cao là $h=d\left(\left(ABC\right);\left(A'B'C'\right)\right)=AA'.$\\
			Do tam giác $ABC$ là tam giác đều nên $O$ là trọng tâm của tam giác $ABC.$\\
			Gọi $I$ là trung điểm của $BC$, $H$ là hình chiếu vuông góc của $A$ lên $A'I.$\\
			Ta có $AH\perp \left(A'BC \right)\Rightarrow d\left(A;\left(A'BC\right)\right)=AH.$\\
			$\dfrac{d\left(O;\left(A'BC\right)\right)}{d\left(A;\left(A'BC\right)\right)}=\dfrac{IO}{IA}=\dfrac{1}{3}\\
			\Rightarrow d\left(O;\left(A'BC\right)\right)=\dfrac{d\left(A;\left(A'BC\right)\right)}{3}=\dfrac{AH}{3}=\dfrac{a}{6}\Rightarrow AH=\dfrac{a}{2}.$}
		{\begin{tikzpicture}[line join=round,line cap=round, font=\footnotesize,scale=0.7,>=stealth]
				\def \a{5}
				\path
				(0,0) coordinate (A)
				(0:\a) coordinate (C)
				(-20:{0.75*\a}) coordinate (B)
				(90:{\a}) coordinate (A')
				($(B)+(A')$) coordinate (B')
				($(A')+(C)$) coordinate (C')
				($(B)!0.5!(C)$) coordinate (I)
				($(B)!0.5!(A)$) coordinate (N)
				(intersection of C--N and A--I) coordinate (O)
				($(A')!(O)!(I)$) coordinate (K)
				($(A')!(A)!(I)$) coordinate (H);
				\draw (A)--(B)--(C)--(C')--(A')--(B')--(C') (B)--(A')--(A) (B)--(B');
				\draw[dashed] (H)--(A)--(C) (A)--(I)--(A')--(C) (O)--(K);
				\draw pic[draw,angle radius=2mm] {right angle =A--H--A'}
				pic[draw,angle radius=2mm] {right angle =A'--K--O};	
				\foreach \x/\g in {A/180, B/-90, C/0, C'/0, A'/180, B'/90, I/-30,H/0,K/0 }
				\fill[black] (\x) circle (1pt)+(\g:.35)node{$\x$};
		\end{tikzpicture}}
		Xét tam giác $A'AI$ vuông tại $A$ ta có:\\
		$\dfrac{1}{AH^2}=\dfrac{1}{AA'^2}+\dfrac{1}{AI^2}\Rightarrow \dfrac{1}{AA'^2}=\dfrac{1}{AH^2}-\dfrac{1}{AI^2}\Rightarrow AA'=\dfrac{a\sqrt{3}}{2\sqrt{2}}\Rightarrow h=\dfrac{a\sqrt{3}}{2\sqrt{2}}\\
		\Rightarrow V_{ABC.A'B'C'}=\dfrac{3a^3\sqrt{2}}{16}$
	}
\end{ex}
\begin{ex}%[2H1K3-2]%Câu 23.%(THPT Hoàng Mai - Nghệ An - 2018) 
	Cho một lăng trụ tam giác đều $ABC.A'B'C'$ có cạnh đáy bằng $a$, góc giữa $A'C$ và mặt phẳng đáy bằng $60^{\circ}$. Tính diện tích xung quanh $S_{xq}$ của hình nón có đáy là đường tròn nội tiếp tam giác $ABC$ và đỉnh là trọng tâm của tam giác $A'B'C'$. 
	\begin{center}
		\begin{tikzpicture}
			\def\a{5}
			\def\h{4}
			\path 	(0:0) coordinate (A)
			++(0:\a) coordinate (C)
			++(-150:3*\a/4) coordinate (B)
			($(A)+(90:\h)$) coordinate (A')
			($(B)+(90:\h)$) coordinate (B')
			($(C)+(90:\h)$) coordinate (C');
			\draw[dashed] 	(A)--(C);
			\draw[thick]	(C)--(C') 	(B)--(B')	(A)--(A') (A)--(B)--(C) (A')--(B')--(C')--cycle;
			\foreach \x/\g in {A/180,B/-45,C/0,A'/180,B'/-45,C'/0}
			\fill[black] 	(\x) circle (1pt)
			($(\g:4mm)+(\x)$) node {$\x$};	
		\end{tikzpicture}
	\end{center}
	\choice
	{\True $S_{\text{xq}}=\dfrac{\pi a^2\sqrt{333}}{36}$}
	{$S_{\text{xq}}=\dfrac{\pi a^2\sqrt{333}}{6}$}
	{$S_{\text{xq}}=\dfrac{\pi a^2\sqrt{111}}{6}$}
	{$S_{\text{xq}}=\dfrac{\pi a^2\sqrt{111}}{36}$}
	\loigiai{
		\begin{center}
			\begin{tikzpicture}
				\begin{scope}
					%	\draw[color=gray!50,dashed] (-3,-3) grid (5,5);
					\def\x {2}
					\def\y {.7}
					\def\h {5}
					\path
					(0,0) coordinate (G)
					($(G)+(0,\h)$)coordinate (G')
					($(G)+(180:\x cm and \y cm)$) coordinate (M)
					%Vẽ tiếp tuyến với đường cong tại I,J,K
					(M) arc (180:-30:{\x} and {\y}) coordinate (I)--([turn]0:5) coordinate (b1)
					(M) arc (180:-165:{\x} and {\y}) coordinate (J)--([turn]0:4) coordinate (a1)
					(M) arc (180:90:{\x} and {\y}) coordinate (K)--([turn]0:6) coordinate (c1)
					($(b1)!2!(I)$) coordinate (b2)
					($(c1)!2!(K)$) coordinate (c2)
					($(a1)!2!(J)$) coordinate (a2)
					% Xác định giao điểm với tiếp tuyến
					(intersection of a1--a2 and c1--c2) coordinate (A)
					(intersection of a1--a2 and b1--b2) coordinate (B)
					(intersection of c1--c2 and b1--b2) coordinate (C)
					($(A)+(90:\h)$) coordinate (A')
					($(B)+(90:\h)$) coordinate (B')
					($(C)+(90:\h)$) coordinate (C')
					(intersection of A'--G' and B'--C') coordinate (N)
					%($(B')!0.5!(C')$)coordinate (N)
					;
					\draw (A)--(B)--(C) (A)--(A')--(B')--(B) (C)--(C') (A')--(B')--(C')--cycle;
					\draw[dashed,thick] (A)--(C) (C)--(G) (A)--(G)--(I) (B)--(G) (I)--(G')--(G) (G')--(J) (A')--(N);
					\draw  [dashed] (G) ellipse (\x cm and \y cm);
					\foreach \i/\j in{G'/60,G/-60,A/-120,B/-30,C/0,A'/180,B'/180,C'/0,I/-60}{\fill [black](\i) circle (1pt) ($(\i)+(\j:3mm)$) node {$\i$};}
				\end{scope}
			\end{tikzpicture}
		\end{center}
		Gọi $I$ là trung điểm của đoạn $BC$\\
		Ta có $\widehat{\left(A'C;(ABC)\right)}=\widehat{A'CA}=60^{\circ}$ suy ra $AA'=AC\cdot\tan 60^{\circ}=\sqrt{3}a$.\\
		Có $r=GI=\dfrac{1}{3}AI=\dfrac{1}{3}\cdot\dfrac{\sqrt{3}a}{2}=\dfrac{\sqrt{3}}{6}a$ và $l=G'I=\sqrt{G'G^2+GI^2} =\sqrt{3a^2+\dfrac{3a^2}{36}}=\dfrac{\sqrt{111}a}{6}$.\\
		Vậy $S_{xq}=\pi rl=\pi\cdot\dfrac{\sqrt{3}}{6}a\cdot\dfrac{\sqrt{111}}{6}a=\dfrac{\pi a^2\sqrt{333}}{36}$.}
\end{ex}
\begin{ex}%[2H1K3-2]%Câu 24.%(Mã 102 - 2021 Lần 1)
	Cho khối hộp chữ nhật $ABCD.A'B'C'D'$ có đáy là hình vuông, $BD=4a$, góc giữa hai mặt phẳng $(A'BD)$ và $(ABCD)$ bằng $30^{\circ}$. Thể tích của khối hộp chữ nhật bằng
	\choice
	{$\dfrac{16\sqrt{3}}{9}a^3$}
	{$48\sqrt{3}a^3$}
	{$\dfrac{16\sqrt{3}}{3}a^3$}
	{$16\sqrt{3}a^3$}
	\loigiai{
		%{\color{red} HINH O DAY}\\
		\begin{center}
			
			\begin{tikzpicture}
				\def\a{3.5}
				\def\b{2}
				\def\h{3}
				\path 	(0:0) coordinate (A)
				++(0:\a) coordinate (D)
				++(-130:\b) coordinate (C)
				($(A)+(C)-(D)$) coordinate (B)
				($(A)+(90:\h)$) coordinate (A')
				($(B)+(90:\h)$) coordinate (B')
				($(C)+(90:\h)$) coordinate (C')
				($(D)+(90:\h)$) coordinate (D')
				(intersection of A--C and B--D) coordinate (O);
				\draw[dashed,thick] 	(B)--(A)--(D)	(A)--(A') (A)--(C) (B)--(D) (A')--(O);
				\draw[thick] 	(C)--(C') 	(D)--(D') 	(B)--(B')	(C)--(C')
				(B)--(C)--(D) 
				(A')--(B')--(C')--(D')--cycle;
				\foreach \x/\g in {A/180,B/180,C/0,D/0,A'/180,B'/180,C'/0,D'/0,O/-90}
				\fill[black] 	(\x) circle (1pt)
				($(\g:4mm)+(\x)$) node {$\x$};	
			\end{tikzpicture}
		\end{center}
		Theo giả thiết $ABCD$ là hình vuông nên có $2AB^2=BD^2\Rightarrow AB=2\sqrt{2}a$.\\
		Do đó $S_{ABCD}=AB^2=8a^2$.\\
		Gọi $O$ là tâm của đáy $ABCD\Rightarrow OA\perp BD$ và $OA=\dfrac{1}{2}BD=2a$.\\
		Vì $ABCD.A'B'C'D'$ là hình hộp chữ nhật nên có $A'A\perp(ABCD)\Rightarrow A'A\perp BD\Rightarrow BD\perp(A'AO)$. Do đó góc giữa $(A'BD)$ và mặt phẳng $(ABCD)$ là góc $\widehat{A'OA}\Rightarrow\widehat{A'OA}=30^{\circ}$.\\
		Tam giác $A'OA$ vuông tại $A$ có $A'A=OA\cdot\tan\widehat{A'OA}=\dfrac{2a\sqrt{3}}{3}$.\\
		Vậy $V_{ABCD.A'B'C'D'}=8a^2\cdot\dfrac{2a\sqrt{3}}{3}=\dfrac{16\sqrt{3}}{3}a^3$.}
\end{ex}
\begin{ex}%[2H1K3-2]%Câu 25.%(Mã 103 - 2021 - Lần 1)
	Cho khối hộp chữ nhật $ABCD.A'B'C'D'$ có đáy là hình vuông, $BD=2a$, góc giữa hai mặt phẳng $(A'BD) v\`a(ABCD)$ bằng $60^{\circ}$. Thể tích của khối hộp đã cho bằng
	\choice
	{$\dfrac{2\sqrt{3}}{9}a^3$}
	{$6\sqrt{3}a^3$}
	{$\dfrac{2\sqrt{3}}{3}a^3$}
	{$2\sqrt{3}a^3$}
	\loigiai{
		%	{\color{red} HINH O DAY}\\
		\begin{center}
			\begin{tikzpicture}
				\def\a{5}
				\def\b{2.5}
				\def\h{3.5}
				\path 	(0:0) coordinate (A)
				++(0:\a) coordinate (D)
				++(-130:\b) coordinate (C)
				($(A)+(C)-(D)$) coordinate (B)
				($(A)+(90:\h)$) coordinate (A')
				($(B)+(90:\h)$) coordinate (B')
				($(C)+(90:\h)$) coordinate (C')
				($(D)+(90:\h)$) coordinate (D')
				(intersection of A--C and B--D) coordinate (O);
				\draw[dashed,thick] 	(B)--(A)--(D)	(A)--(A') (A)--(C) (O)--(A') (B)--(D)-- (A')--cycle;
				\draw[thick] 	(C)--(C') 	(D)--(D') 	(B)--(B')	(C)--(C')
				(B)--(C)--(D) 
				(A')--(B')--(C')--(D')--cycle;
				\draw pic[,draw,angle eccentricity=1.2,angle radius=0.5cm]{angle=A'--O--A};
				\foreach \x/\g in {A/180,B/180,C/0,D/0,A'/180,B'/180,C'/0,D'/0,O/-90}
				\fill[black] 	(\x) circle (1pt)
				($(\g:4mm)+(\x)$) node {$\x$};	
			\end{tikzpicture}
		\end{center}
		Gọi $O=AC\cap BD$.\\
		Ta có: $\heva{&(A'BD)\cap(ABCD)=BD\\&A'O\perp BD\\&AC\perp BD}\Rightarrow 60^{\circ}=\widehat{A'OA.}$ \\
		Tam giác $AA'O$ có: $AA'=\tan 60^{\circ}\cdot OA=\sqrt{3}a$ và $S_{ABCD}=2a^2$.\\
		Vậy $V_{ABCD.A'B'C'D'}=AA'\cdot S_{ABCD}=2\sqrt{3}a^3$.}
\end{ex}
\begin{ex}%[2H1K3-2]%Câu 26.%(Mã 101 - 2021 Lần 1)
	Cho khối hộp chữ nhật $ABCD.A'B'C'D'$ có đáy là hình vuông, $BD=2a,$ góc giữa hai mặt phẳng $(A'BD)$ và $(ABCD)$ bằng $30^{\circ}$. Thể tích khối hộp chữ nhật đã cho bằng
	\choice
	{$6\sqrt{3}a^3$}
	{$\dfrac{2\sqrt{3}}{9}a^3$}
	{$2\sqrt{3}a^3$}
	{$\dfrac{2\sqrt{3}}{3}a^3$}
	\loigiai{
		\begin{center}
			\begin{tikzpicture}
				\def\a{5}
				\def\b{2.5}
				\def\h{3.5}
				\path 	(0:0) coordinate (A)
				++(0:\a) coordinate (D)
				++(-130:\b) coordinate (C)
				($(A)+(C)-(D)$) coordinate (B)
				($(A)+(90:\h)$) coordinate (A')
				($(B)+(90:\h)$) coordinate (B')
				($(C)+(90:\h)$) coordinate (C')
				($(D)+(90:\h)$) coordinate (D')
				(intersection of A--C and B--D) coordinate (O);
				\draw[dashed,thick] 	(B)--(A)--(D)	(A)--(A') (A)--(C) (O)--(A') (B)--(D);
				\draw[thick] 	(C)--(C') 	(D)--(D') 	(B)--(B')	(C)--(C')
				(B)--(C)--(D) 
				(A')--(B')--(C')--(D')--cycle;
				\draw pic[,draw,angle eccentricity=1.2,angle radius=0.5cm]{angle=A'--O--A};
				\foreach \x/\g in {A/180,B/180,C/0,D/0,A'/180,B'/180,C'/0,D'/0,O/-90}
				\fill[black] 	(\x) circle (1pt)
				($(\g:4mm)+(\x)$) node {$\x$};	
			\end{tikzpicture}
		\end{center}
		Gọi $O=AC\cap BD$.\\
		Diện tích hình vuông $ABCD$ là $S_{ABCD}=AB^2=\left(\dfrac{BD}{\sqrt{2}}\right)^2=\left(\dfrac{2a}{\sqrt{2}}\right)^2=2a^2$.\\
		Ta có: $\left((A'BD),(ABCD)\right)=(A'O;AO)=30^{\circ}$.\\
		Xét tam giác $A'OA$ vuông tại $A,$ ta có: $A'A=\tan 30^{\circ}\cdot AO=\dfrac{\sqrt{3}}{3}a$.\\
		Thể tích khối hộp chữ nhật đã cho là $V=A'A\cdot S_{ABCD}=\dfrac{\sqrt{3}}{3}a\cdot 2a^2=\dfrac{2\sqrt{3}}{3}a^3$.}
\end{ex}
\begin{ex}%[2H1K3-2]%Câu 27.%(Mã 104 - 2021 Lần 1)
	Cho khối hộp chữ nhật $ABCD.A'B'C'D'$ có đáy là hình vuông, $BD=4a$, góc giữa hai mặt phẳng $(A'BD)$ và $(ABCD)$ bằng $60^{\circ}$. Thể tích của khối hộp chữ nhật đã cho bằng
	\choice
	{$48\sqrt{3}a^3$}
	{$\dfrac{16\sqrt{3}}{9}a^3$}
	{$\dfrac{16\sqrt{3}}{3}a^3$}
	{$16\sqrt{3}a^3$}
	\loigiai{
		\begin{center}
			\begin{tikzpicture}
				\def\a{5}
				\def\b{2.5}
				\def\h{3.5}
				\path 	(0:0) coordinate (A)
				++(0:\a) coordinate (D)
				++(-130:\b) coordinate (C)
				($(A)+(C)-(D)$) coordinate (B)
				($(A)+(90:\h)$) coordinate (A')
				($(B)+(90:\h)$) coordinate (B')
				($(C)+(90:\h)$) coordinate (C')
				($(D)+(90:\h)$) coordinate (D')
				(intersection of A--C and B--D) coordinate (O);
				\draw[dashed,thick] 	(B)--(A)--(D)	(A)--(A') (A)--(C) (O)--(A') (B)--(D)-- (A')--cycle;
				\draw[thick] 	(C)--(C') 	(D)--(D') 	(B)--(B')	(C)--(C')
				(B)--(C)--(D) 
				(A')--(B')--(C')--(D')--cycle;
				\draw pic[,draw,angle eccentricity=1.2,angle radius=0.5cm]{angle=A'--O--A};
				\foreach \x/\g in {A/180,B/180,C/0,D/0,A'/180,B'/180,C'/0,D'/0,O/-90}
				\fill[black] 	(\x) circle (1pt)
				($(\g:4mm)+(\x)$) node {$\x$};	
			\end{tikzpicture}
		\end{center}
		Gọi $O$ là giao điểm của hai đường chéo $AC$ và $BD$.\\
		Ta có:\\
		$\heva{&OA\perp BD\\&A'A\perp(ABCD)\Rightarrow A'A\perp BD}\Rightarrow BD\perp A'O$.\\
		Xét $(A'BD)$ và $(ABCD)$ có:\\
		$\heva{&(A'BD)\cap(ABCD)=BD\\&AO\perp BD\\&A'O\perp BD}\Rightarrow$ góc giữa hai mặt phẳng $(A'BD)$ và $(ABCD)$ là $\widehat{A'OA}$ \\
		$ \Rightarrow\widehat{A'OA}=60^{\circ} $.\\
		Ta có: $BD=4a\Rightarrow OA=2a$ và $\tan\widehat{A'OA}=\dfrac{AA'}{OA}\Leftrightarrow AA'=OA\cdot\tan 60^{\circ}=2\sqrt{3}a$.\\
		Vậy $V_{ABCD.A'B'C'D'}=AA'\cdot S_{ABCD} =AA'\cdot\dfrac{1}{2}AC\cdot BD =2\sqrt{3} a\cdot\dfrac{1}{2}\cdot (4a)^2=16\sqrt{3}a^3$.}
\end{ex}
\begin{ex}%[2H1K3-2]%Câu 28.%(Mã 101-2022)
	Cho khối lăng trụ đứng $ABC.A'B'C'$ có đáy $ABC$ là tam giác vuông cân tại $A$, $AB=2a$. Góc giữa đường thẳng $BC'$ và mặt phẳng $(ACC'A')$ bằng $30^{\circ}$. Thể tích của khối lăng trụ đã cho bằng
	\choice
	{$3a^3$}
	{$a^3$}
	{$12\sqrt{2}a^3$}
	{\True $4\sqrt{2}a^3$}
	\loigiai{
		%	{\color{red} HINH O DAY}\\
		\begin{center}
			
			\begin{tikzpicture}
				\def\a{4}
				\def\h{4}
				\path 	(0:0) coordinate (B)
				++(0:\a) coordinate (C)
				++(-150:3*\a/4) coordinate (A)
				($(A)+(90:\h)$) coordinate (A')
				($(B)+(90:\h)$) coordinate (B')
				($(C)+(90:\h)$) coordinate (C');
				\draw[dashed] 	(B)--(C) (B)--(C');
				\draw[thick]	(C)--(C') 	(B)--(B')	(A)--(A') (B)--(A)--(C) (A)--(C') (A')--(B')--(C')--cycle;
				\draw pic[,draw,angle eccentricity=1.2,angle radius=0.7cm]{angle=B--C'--A};
				\draw[dashed] pic[draw,angle radius=0.3cm]{right angle=B--A--C};
				\draw pic[draw,angle radius=0.3cm]{right angle=B--A--A'};
				\foreach \x/\g in {A/180,B/180,C/0,A'/180,B'/180,C'/0}
				\fill[black] 	(\x) circle (1pt)
				($(\g:4mm)+(\x)$) node {$\x$};	
			\end{tikzpicture}
		\end{center}
		Ta có: $\heva{&AB\perp AC\\&AB\perp AA'}\Rightarrow AB\perp(ACC'A')\Rightarrow AB\perp AC'$.\\
		Vậy góc giữa đường thẳng $BC'$ và mặt phẳng $(ACC'A')$ là góc $\widehat{BC'A}$.\\
		Trong tam giác vuông $BC'A$ ta có $\widehat{BC'A}=30^{\circ}; AB=2a\Rightarrow AC'=AB\cdot\cot\widehat{BC'A}=2a\cdot\sqrt{3}$.\\
		Trong tam giác vuông $ACC'$ ta có $CC'=\sqrt{AC'^2-AC^2}=2\sqrt{2}a$.\\
		Vậy thể tích khối lăng trụ đã cho là\\
		$V=CC'\cdot\dfrac{1}{2}AB^2=2\sqrt{2}a\cdot\dfrac{1}{2}\cdot 4a^2=4\sqrt{2}a^3$.}
\end{ex}
\begin{ex}%[2H1K3-2]%Câu 29.%(Mã 102 - 2022) 
	Cho khối lăng trụ đứng $ABC.A'B'C'$ có đáy $ABC$ là tam giác vuông cân tại $A$, $AB=a$. Góc giữa đường thẳng $BC'$ và mặt phẳng $(ACC'A')$ bằng $30^{\circ}$. Thể tích của khối lăng trụ đã cho bằng
	\choice
	{$\dfrac{1}{8}a^3$}
	{$\dfrac{3}{8}a^3$}
	{$\dfrac{3\sqrt{2}}{2}a^3$}
	{\True $\dfrac{\sqrt{2}}{2}a^3$}
	\loigiai{
		%{\color{red} HINH O DAY}\\
		\begin{center}
			
			\begin{tikzpicture}
				\def\a{4}
				\def\h{4}
				\path 	(0:0) coordinate (A')
				++(0:\a) coordinate (B')
				++(-150:3*\a/4) coordinate (C')
				($(A')+(90:\h)$) coordinate (A)
				($(B')+(90:\h)$) coordinate (B)
				($(C')+(90:\h)$) coordinate (C);
				\draw[dashed,thick] 	(A')--(B');
				\draw[thick]	(C)--(C') 	(B)--(B')	(A)--(A') (A')--(C')--(B') (C')--(A) (C')--(B) (A)--(B)--(C)--cycle;
				\foreach \x/\g in {A/180,B/-45,C/-30,A'/180,B'/-45,C'/-45}
				\fill[black] 	(\x) circle (1pt)
				($(\g:4mm)+(\x)$) node {$\x$};
				\draw[dashed,thick] pic[,draw,angle eccentricity=1.2,angle radius=0.7cm]{angle=B--C'--A};	
				\draw pic[draw,angle radius=0.3cm]{right angle=B--A--C};
			\end{tikzpicture}
		\end{center}
		Diện tích đáy: $S_{ABC}=\dfrac{1}{2}AB\cdot AC=\dfrac{a^2}{2}$.\\
		Ta có: $\heva{&AB\perp AC\\&AB\perp AA'}\Rightarrow AB\perp(ACC'A')\Rightarrow\widehat{\left(BC',(ACC'A')\right)}=\widehat{BC'A}=30^{\circ.}$ \\
		Khi đó $AC'=AB\cdot\cot 30^{\circ}=a\sqrt{3}\Rightarrow AA'=\sqrt{AC'^2-A'C'^2}=\sqrt{(a\sqrt{3})^2-a^2}=a\sqrt{2}$.\\
		Vậy, thể tích khối lăng trụ đã cho là $V=S_{ABC}\cdot AA'=\dfrac{a^2}{2}\cdot a\sqrt{2}=\dfrac{\sqrt{2}}{2}\cdot a^3$.}
\end{ex}
\begin{ex}%[2H1K3-2]%Câu 30.%(Mã 103 - 2022)
	Cho khối lăng trụ đứng $ABC.A'B'C'$ có đáy $ABC$ là tam giác vuông cân tại $A$, cạnh bên $AA'=2a$, góc giữa hai mặt phẳng $(A'BC)$ và $(ABC)$ bằng $30^{\circ}$. Thể tích của khối lăng trụ đã cho bằng
	\choice
	{\True $24a^3$}
	{$\dfrac{8}{3}a^3$}
	{$8a^3$}
	{$\dfrac{8}{9}a^3$}
	\loigiai{
		%{\color{red} HINH O DAY}\\
		\begin{center}
			
			\begin{tikzpicture}
				\def\a{4.5}
				\def\h{4}
				\path 	(0:0) coordinate (A)
				++(0:\a) coordinate (C)
				++(-150:3*\a/4) coordinate (B)
				($(A)+(90:\h)$) coordinate (A')
				($(B)+(90:\h)$) coordinate (B')
				($(C)+(90:\h)$) coordinate (C')
				($(B)!0.5!(C)$)coordinate (H);
				\draw[dashed,thick] 	(A)--(C) (H)--(A')--(C) (H)--(A);
				\draw[thick]	(C)--(C') 	(B)--(B')	(A)--(A')--(B) (A)--(B)--(C) (A')--(B')--(C')--cycle;
				\foreach \x/\g in {A/180,B/-45,C/0,A'/180,B'/-45,C'/0,H/-45}
				\fill[black] 	(\x) circle (1pt)
				($(\g:4mm)+(\x)$) node {$\x$};	
				\draw[thick] pic[,draw,angle eccentricity=1.2,angle radius=0.5cm]{angle=A'--H--A};
				\draw pic[draw,angle radius=0.3cm]{right angle=B--A--C};
				\draw pic[draw,angle radius=0.3cm]{right angle=A--H--B};
				\draw pic[draw,angle radius=0.3cm]{right angle=C--H--A'};		
			\end{tikzpicture}
		\end{center}
		Kẻ $AH\perp BC$, ta có $AA'\perp(ABC)$ nên $AA'\perp BC$.\\
		$AH\perp BC$ và $AA'\perp BC$ suy ra $BC\perp(AA'H)\Rightarrow A'H\perp BC$.\\
		Suy ra góc giữa $(A'BC)$ và $(ABC)$ là $\widehat{A'HA}\Rightarrow\widehat{A'HA}=30^{\circ}$.\\
		$\triangle A'AH$ vuông tại $A$ có.\\
		$\tan\widehat{A'HA}=\dfrac{AA'}{AH}\Rightarrow\tan 30^{\circ}=\dfrac{2a}{AH}\Leftrightarrow AH=\dfrac{2a}{\tan 30^{\circ}}=2a\sqrt{3}$.\\
		$\triangle ABC$ vuông cân tại $A$ nên $BC=2AH=4a\sqrt{3}$ \\
		$ \Rightarrow S_{ABC}=\dfrac{1}{2}AH\cdot BC=\dfrac{1}{2}2a\sqrt{3}\cdot 4a\sqrt{3}=12a^2 $.\\
		Vậy thể tích của khối lăng trụ đứng $ABC.A'B'C'$ là.}
\end{ex}
\begin{ex}%[2H1K3-2]%Câu 31.%(Mã 104-2022)
	Cho khối lăng trụ đứng $ABC.A'B'C'$ có đáy $ABC$ là tam giác vuông cân tại $A$, cạnh bên $AA'=2a$, góc giữa hai mặt phẳng $(A'BC)$ và $(ABC)$ bằng $60^{\circ}$. Thể tích của khối lăng trụ đã cho bằng
	\choice
	{$\dfrac{8}{9}a^3$}
	{$8a^3$}
	{\True $\dfrac{8}{3}a^3$}
	{$24a^3$}
	\loigiai{
		\begin{center}
			
			\begin{tikzpicture}
				\def\a{4.5}
				\def\h{4}
				\path 	(0:0) coordinate (A)
				++(0:\a) coordinate (C)
				++(-150:3*\a/4) coordinate (B)
				($(A)+(90:\h)$) coordinate (A')
				($(B)+(90:\h)$) coordinate (B')
				($(C)+(90:\h)$) coordinate (C')
				($(B)!0.5!(C)$)coordinate (I);
				\draw[dashed,thick] 	(A)--(C) (I)--(A')--(C) (I)--(A);
				\draw[thick]	(C)--(C') 	(B)--(B')	(A)--(A')--(B) (A)--(B)--(C) (A')--(B')--(C')--cycle;
				\foreach \x/\g in {A/180,B/-45,C/0,A'/180,B'/-45,C'/0,I/-45}
				\fill[black] 	(\x) circle (1pt)
				($(\g:4mm)+(\x)$) node {$\x$};	
				\draw[thick] pic[,draw,angle eccentricity=1.2,angle radius=0.5cm]{angle=A'--I--A};
				\draw pic[draw,angle radius=0.3cm]{right angle=B--A--C};
				\draw pic[draw,angle radius=0.3cm]{right angle=A--I--B};
				\draw pic[draw,angle radius=0.3cm]{right angle=C--I--A'};		
			\end{tikzpicture}
		\end{center}
		Gọi $I$ là trung điểm của $BC$.\\
		Ta có: + $ABC$ là tam giác vuông cân tại $A$ nên $AI\perp BC$.\\
		+ $ABC.A'B'C'$ là khối lăng trụ đứng nên $AA'\perp BC$.\\
		suy ra $BC\perp(AA'I)\Rightarrow BC\perp A'I$.\\
		Do đó, góc giữa hai mặt phẳng $(A'BC)$ và $(ABC)$ bằng góc giữa $A'I$ và $AI$, mà tam giác $AA'I$ vuông tại $A$ nên ta có $\widehat{AIA'}$ là góc nhọn. Suy ra góc giữa hai mặt phẳng $(A'BC)$ và $(ABC)$ bằng $\widehat{AIA'}=60^{\circ}$.\\
		Trong tam giác vuông $AA'I$, ta có $AI=\dfrac{AA'}{\tan 60^{\circ}}=\dfrac{2a}{\sqrt{3}}$.\\
		$ABC$ là tam giác vuông cân tại $A$ nên $BC=2AI=\dfrac{4a}{\sqrt{3}}$, $AB=AC=\dfrac{BC}{\sqrt{2}}=\dfrac{2a\sqrt{6}}{3}$.\\
		Vậy thể tích khối lăng trụ đã cho là $V=AA'\cdot S_{\triangle ABC}=AA'\cdot\dfrac{1}{2}AB\cdot AC=\dfrac{1}{2}\cdot 2 a\cdot\left(\dfrac{2 a\sqrt{6}}{3}\right)^2=\dfrac{8a^3}{3}$.\\
	}
\end{ex}
\subsection{Dạng 2: Thể tích khối lăng trụ xiên}
\begin{ex}%Câu 1.
	(Sở Bình Phước 2019) Cho hình lăng trụ $ABC.A'B'C'$ có tất cả các cạnh bằng $a$, các cạnh bên tạo với đáy góc $60^{\circ}$. Tính thể tích khối lăng trụ $ABC.A'B'C'$ bằng
	\choice
	{$\dfrac{a^3\sqrt{3}}{24}$}
	{\True $\dfrac{3a^3}{8}$}
	{$\dfrac{a^3\sqrt{3}}{8}$}
	{$\dfrac{a^3}{8}$}
	\loigiai{
		\immini{Kẻ $AH'\perp(ABC)\Rightarrow\left(A'A,(ABC)\right)=\widehat{A'AH}=60^{\circ}$.\\
		Xét $\triangle AHA'\colon\sin 60^{\circ}=\dfrac{A'H}{AA'}\Leftrightarrow A'H=AA'\cdot\sin 60^{\circ}=\dfrac{a\sqrt{3}}{2}$.\\
		Thể tích khối lăng trụ $ABC.A'B'C'$ là \\
		 $ V=S_{\triangle ABC}\cdot A'H=\dfrac{a^2\sqrt{3}}{4}\cdot\dfrac{a\sqrt{3}}{2}=\dfrac{3a^3}{8}$.}{\begin{tikzpicture}[declare function={a=2;b=4;h=3;gn=65;},line join=round]
			\path (0,0) coordinate (A)
			(b,0) coordinate (C)
			(-45:a) coordinate (B)
			($(A)+(gn:3)$) coordinate (A')
			($(B)-(A)+(A')$) coordinate (B')
			($(C)-(A)+(A')$) coordinate (C')
			($(C)!0.5!(B)$) coordinate (M)
			($(A)!0.4! (M)$) coordinate (H);
			\draw (A')--(A)--(B)--(C)--(C')--(A')--(B')--(C') (B)--(B');
			\draw[dashed] (A)--(C) (A)--(H)--(A');
			\path pic["\scriptsize$60^\circ$", angle eccentricity=2,draw,angle radius=7pt]{angle= H--A--A'};
			\path pic[angle radius=3mm,draw=blue] {right angle = A--H--A'};
			\foreach \t/\g in {A/180,B/-90,C/-90,A'/90,B'/90,C'/90,H/0}{
				\draw[fill=white] (\t) circle (1pt) node[shift={(\g:7pt)},font=\scriptsize]{$ \t $};
			}
	\end{tikzpicture}}}
\end{ex}
\begin{ex}%Câu 2.
	(THPT Thăng Long - Hà Nội - 2018) Cho lăng trụ $ABC.A'B'C'$ có đáy $ABC$ là tam giác đều cạnh bằng $a$, biết $A'A=A'B=A'C=a$. Tính thể tích khối lăng trụ $ABC.A'B'C'$?
	\choice
	{$\dfrac{3a^3}{4}$}
	{\True $\dfrac{a^3\sqrt{2}}{4}$}
	{$\dfrac{a^3\sqrt{3}}{4}$}
	{$\dfrac{a^3}{4}$}
	\loigiai{
	\begin{center}
		\begin{tikzpicture}[declare function={a=2;b=4;h=3;gn=65;},line join=round]
			\path (0,0) coordinate (A)
			(b,0) coordinate (C)
			(-45:a) coordinate (B)
			($(A)+(gn:3)$) coordinate (A')
			($(B)-(A)+(A')$) coordinate (B')
			($(C)-(A)+(A')$) coordinate (C')
			($(C)!0.5!(B)$) coordinate (M)
			($(A)!0.4! (M)$) coordinate (H);
			\draw (A')--(A)--(B)--(C)--(C')--(A')--(B')--(C') (B)--(B');
			\draw[dashed] (M)--(A)--(C) (H)--(A');
			\foreach \t/\g in {A/180,B/-90,C/-90,A'/90,B'/90,C'/90,H/0}{
				\draw[fill=white] (\t) circle (1pt) node[shift={(\g:7pt)},font=\scriptsize]{$ \t $};
			}
		\end{tikzpicture}
	\end{center}
		Gọi $H$ là trọng tâm tam giác $ABC$. Theo giả thiết ta có $ABC$ là tam giác đều cạnh bằng $a$ và $A'A=A'B=A'C=a$ nên $A'\cdot ABC$ là tứ diện đều cạnh $a\Rightarrow A'H\perp(ABC)$ hay $A'H$ là đường cao của khối chóp $A'\cdot ABC$.\\
		Xét tam giác vuông $A'HA$ ta có $A'H=\sqrt{A'A^2-AH^2} =\dfrac{a\sqrt{6}}{3}$.\\
		Diện tích tam giác $ABC$ là $S_{ABC}=\dfrac{1}{2}a\cdot a\cdot\sin 60^{\circ} =\dfrac{a^2\sqrt{3}}{4}$.\\
		Thể tích khối lăng trụ $ABC.A'B'C'$ là $V_{ABC.A'B'C'}=\dfrac{a^2\sqrt{3}}{4}\dfrac{a\sqrt{6}}{3} =\dfrac{a^3\sqrt{2}}{4}$.}
\end{ex}
\begin{ex}%Câu 3.
	(HSG Bắc Ninh 2019) Cho hình lăng trụ $ABC.A'B'C'$ có đáy $ABC$ là tam giác vuông cân tại $A, AC=2\sqrt{2}$, biết góc giữa $AC'$ và $(ABC)$ bằng $60^{\circ}$ và $AC'=4$. Tính thể tích $V$ của khối lăng trụ $ABC.A'B'C'$. 
	\choice
	{$V=\dfrac{8}{3}$}
	{$V=\dfrac{16}{3}$}
	{$V=\dfrac{8\sqrt{3}}{3}$}
	{\True $8\sqrt{3}$}
	\loigiai{
	\begin{center}
			\begin{tikzpicture}[declare function={a=2;b=4;h=3;gn=105;},line join=round]
			\path (0,0) coordinate (A)
			(b,0) coordinate (C)
			(-45:a) coordinate (B)
			($(A)+(gn:3)$) coordinate (A')
			($(B)-(A)+(A')$) coordinate (B')
			($(C)-(A)+(A')$) coordinate (C')
			(3.2,0.2) coordinate (H)
			;
			\draw (A')--(A)--(B)--(C)--(C')--(A')--(B')--(C') (B)--(B');
			\draw[dashed] (C')--(H)--(A)--(C) (A)--(C') ;
			\foreach \t/\g in {A/180,B/-90,C/-90,A'/90,B'/90,C'/90,H/0}{
				\draw[fill=white] (\t) circle (1pt) node[shift={(\g:7pt)},font=\scriptsize]{$ \t $};
			}
		\end{tikzpicture}
	\end{center}
		Gọi $H$ là hình chiếu của $C'$ lên mặt phẳng $(ABC)$, khi đó $C'H$ là đường cao $\Rightarrow\widehat{AC',(ABC)}=\widehat{C'AH}=60^{\circ}$.\\
		Xét tam giác vuông $AC'H$ ta có $C'H=C'A\cdot\sin 60^{\circ}=2\sqrt{3}$.\\
		Khi đó $V_{ABC.A'B'C}=S_d\cdot C'H=\dfrac{1}{2}(2\sqrt{2})^2\cdot 2\sqrt{3}=8\sqrt{3}$.}
\end{ex}
\begin{ex}%Câu 4.
	(Gia Bình 2019) Cho lăng trụ tam giác $ABC.A'B'C'$ có đáy là tam giác đều cạnh $a$, góc giữa cạnh bên và mặt đáy bằng $30^{\circ}$. Hình chiếu của $A'$ lên $(ABC)$ là trung điểm $I$ của $BC$. Tính thể tích khối lăng trụ
	\choice
	{$\dfrac{a^3\sqrt{3}}{2}$}
	{$\dfrac{a^3\sqrt{13}}{12}$}
	{\True $\dfrac{a^3\sqrt{3}}{8}$}
	{$\dfrac{a^3\sqrt{3}}{6}$}
	\loigiai{
		\begin{center}
				\begin{tikzpicture}[declare function={a=2;b=4;h=4;gn=45;},line join=round]
				\path (0,0) coordinate (A)
				(b,0) coordinate (C)
				(-65:a) coordinate (B)
				($(A)+(gn:h)$) coordinate (A')
				($(B)-(A)+(A')$) coordinate (B')
				($(C)-(A)+(A')$) coordinate (C')
				($(C)!0.5!(B)$) coordinate (I)	;
				\draw (A')--(A)--(B)--(C)--(C')--(A')--(B')--(C') (B)--(B');
				\draw[dashed] (I)--(A)--(C) (I)--(A');
				\path pic[angle radius=3mm,draw=blue,fill=blue] {angle = I--A--A'};
				\path pic[angle radius=3mm,draw=green,fill=green] {right angle = A--I--A'};
				
				\foreach \t/\g in {A/180,B/-90,C/-90,A'/90,B'/90,C'/90,I/0}{
					\draw[fill=white] (\t) circle (1pt) node[shift={(\g:7pt)},font=\scriptsize]{$ \t $};
				}
			\end{tikzpicture}
		\end{center}
		Ta có $A'I\perp(ABC)\Rightarrow AI$ là hình chiếu vuông góc của $AA'$ lên $(ABC)$.\\
		Nên $\left(\widehat{AA',(ABC)}\right)=\left(\widehat{AA',AI}\right)=\widehat{A'AI}=30^{\circ}$.\\
		Ta có $AI=\dfrac{a\sqrt{3}}{2}\Rightarrow A'I=AI\tan 30^{\circ}=\dfrac{a}{2},S_{\triangle ABC}=\dfrac{a^2\sqrt{3}}{4}$.\\
		Vậy $V_{ABC.A'B'C'}=\dfrac{a^2\sqrt{3}}{4}\cdot\dfrac{a}{2}=\dfrac{a^3\sqrt{3}}{8}$.}
\end{ex}
\begin{ex}%Câu 5.
	(Nguyễn Khuyến 2019) Một khối lăng trụ tam giác có đáy là tam giác đều cạnh bằng $3$, cạnh bên bằng $2\sqrt{3}$ tạo với mặt phẳng đáy một góc $30^{\circ}$. Khi đó thể tích khối lăng trụ là 
	\choice
	{$\dfrac{9}{4}$}
	{\True $\dfrac{27}{4}$}
	{$\dfrac{27\sqrt{3}}{4}$}
	{$\dfrac{9\sqrt{3}}{4}$}
	\loigiai{
	\begin{center}
			\begin{tikzpicture}[declare function={a=2;b=4;h=3;gn=65;},line join=round]
			\path (0,0) coordinate (A)
			(b,0) coordinate (C)
			(-45:a) coordinate (B)
			($(A)+(gn:3)$) coordinate (A')
			($(B)-(A)+(A')$) coordinate (B')
			($(C)-(A)+(A')$) coordinate (C')
			($(C)!0.5!(B)$) coordinate (M)
			($(A)!0.4! (M)$) coordinate (H);
			\draw (A')--(A)--(B)--(C)--(C')--(A')--(B')--(C') (B)--(B');
			\draw[dashed] (A)--(C) (A)--(H)--(A');
			\path pic["\scriptsize$30^\circ$", angle eccentricity=2,draw,angle radius=7pt]{angle= H--A--A'};
			\path pic[angle radius=3mm,draw=blue] {right angle = A--H--A'};
			\foreach \t/\g in {A/180,B/-90,C/-90,A'/90,B'/90,C'/90,H/0}{
				\draw[fill=white] (\t) circle (1pt) node[shift={(\g:7pt)},font=\scriptsize]{$ \t $};
			}
		\end{tikzpicture}
	\end{center}
		Gọi $H$ là hình chiếu của $A'$ lên mặt đáy. Suy ra góc $\widehat{A'AH}=30^{\circ}$.\\
		$\sin 30^{\circ}=\dfrac{A'H}{A'A}\Rightarrow A'H=A'A\cdot\sin 30^{\circ}=2\sqrt{3}\cdot\dfrac{1}{2}=\sqrt{3}$.\\
		Khi đó: $V_{ABC.A'B'C}=3^2\cdot\dfrac{\sqrt{3}}{4}\cdot\sqrt{3}=\dfrac{27}{4}$.}
\end{ex}
\begin{ex}%Câu 6.
	(Chuyên Bến Tre - 2020) Cho hình hộp $ABCD.A'B'C'D'$ có các cạnh bằng $2a$. Biết $\widehat{BAD}=60^{\circ}$, $\widehat{A'AB}=\widehat{A'AD}=120^{\circ}$. Tính thể tích $V$ của khối hộp $ABCD.A'B'C'D'$. 
	\choice
	{\True $4\sqrt{2}a^3$}
	{$2\sqrt{2}a^3$}
	{$8a^3$}
	{$\sqrt{2}a^3$}
	\loigiai{
	\begin{center}
		\begin{tikzpicture}[declare function={a=2;b=4;h=3;},line join=round]
			\path (0,0) coordinate (B)
			(35:a) coordinate (A)
			(b,0) coordinate (C)
			($(C)-(B)+(A)$) coordinate (D)
			($(A)!0.3!(C)$)	coordinate (H)
			($(H)+(90:h)$) coordinate (A')
			($(B)-(A)+(A')$)  coordinate (B')
			($(C)-(A)+(A')$) coordinate (C')
			($(D)-(A)+(A')$) coordinate (D');
			\draw (B')--(B)--(C)--(D)--(D')--(A')--(B')--(C')--(D') (C)--(C');
			\draw[dashed] (C)--(A)--(D)--(B) (H)--(A')--(A)--(B)--(A')--(D);
			\foreach \t/\g in {A/180,H/30,B/-90,C/-90,A'/90,B'/90,C'/90,D'/0,D/0}{
				\draw[fill=white] (\t) circle (1pt) node[shift={(\g:7pt)},font=\scriptsize]{$ \t $};
			}
		\end{tikzpicture}
	\end{center}
		Từ giả thuyết ta có các tam giác $\triangle ABD$, $\triangle A'AD$ và $A'AB$ là các tam giác đều\\
		$ \Rightarrow A'A=A'B=A'D $ nên hình chiếu $H$ của $A'$ trên mặt phẳng $(ABCD)$ là tâm đường tròn ngoại tiếp tam giác đều $ABD$ \\
		$ \Rightarrow AH=\dfrac{2}{3}\cdot 2a\cdot\dfrac{\sqrt{3}}{2}=\dfrac{2\sqrt{3}}{3}a $ \\
		$ \Rightarrow A'H=\sqrt{A'A^2-AH^2}=\dfrac{2\sqrt{6}}{3}a $.\\
		Thể tích của khối hộp $ABCD.A'B'C'D'$: $V=A'H\cdot S_{ABCD}=\dfrac{2\sqrt{6}}{3}a\cdot 2\cdot\dfrac{4a^2\cdot\sqrt{3}}{4}=4\sqrt{2}a^3$.}
\end{ex}
\begin{ex}%Câu 7.
	(SGD Gia Lai 2019) Cho hình lăng trụ $ABC.A'B'C'$ có đáy là tam giác đều cạnh bằng $2$. Hình chiếu vuống góc của $A'$ lên mặt phẳng $(ABC)$ trùng với trung điểm $H$ của cạnh $BC$. Góc tạo bởi cạnh bên $A'A$ với đáy bằng $45^{\circ}$ (hình vẽ bên). Tính thể tích $V$ của khối lăng trụ $ABC.A'B'C'$. 
	\begin{center}
		\begin{tikzpicture}[declare function={a=2;b=4;h=4;gn=45;},line join=round]
		\path (0,0) coordinate (A)
		(b,0) coordinate (C)
		(-65:a) coordinate (B)
		($(A)+(gn:h)$) coordinate (A')
		($(B)-(A)+(A')$) coordinate (B')
		($(C)-(A)+(A')$) coordinate (C')
		($(C)!0.5!(B)$) coordinate (H)	;
		\draw (A')--(A)--(B)--(C)--(C')--(A')--(B')--(C') (B)--(B');
		\draw[dashed] (H)--(A)--(C) (H)--(A');
		\path pic[angle radius=3mm,draw=blue,fill=blue!20] {right angle = A--H--B};
		\path pic[angle radius=3mm,draw=green,fill=green!20] {right angle = A--H--A'};
		
		\foreach \t/\g in {A/180,B/-90,C/-90,A'/90,B'/90,C'/90,H/0}{
			\draw[fill=white] (\t) circle (1pt) node[shift={(\g:7pt)},font=\scriptsize]{$ \t $};
		}
	\end{tikzpicture}
	\end{center}
	\choice
	{$V=\dfrac{\sqrt{6}}{24}$}
	{$V=1$}
	{$V=\dfrac{\sqrt{6}}{8}$}
	{\True $V=3$}
	\loigiai{
		Thể tích của khối lăng trụ $ABC.A'B'C'$: $V_{ABC.A'B'C'}=S_{ABC}\cdot A'H$.\\
		Ta có\\
		$S_{ABC}=\dfrac{4\sqrt{3}}{4}=\sqrt{3}$.\\
		$\heva{&AH=\dfrac{2\sqrt{3}}{2}=\sqrt{3}\\&\tan{45}^{\circ}=\dfrac{A'H}{AH}\Rightarrow A'H=AH=\sqrt{3}.}$ \\
		Vậy thể tích khối lăng trụ $ABC.A'B'C'$ bằng $V_{ABC.A'B'C'}=S_{ABC}\cdot A'H=\sqrt{3}\cdot\sqrt{3}=3$.}
\end{ex}
\begin{ex}%Câu 8.
	Cho lăng trụ tam giác $ABC.A'B'C'$ có đáy $ABC$ là tam giác đều cạnh $a$, hình chiếu của $A'$ xuống $(ABC)$ là tâm $O$ đường tròn ngoại tiếp tam giác $ABC$. Biết $AA'$ hợp với đáy $(ABC)$ một góc $60^{\circ}$, thể tích khối lăng trụ là
	\choice
	{\True $\dfrac{a^3\sqrt{3}}{4}$}
	{$\dfrac{3a^3\sqrt{3}}{4}$}
	{$\dfrac{a^3\sqrt{3}}{12}$}
	{$\dfrac{a^3\sqrt{3}}{36}$}
	\loigiai{
		\begin{center}
			\begin{tikzpicture}[declare function={a=2;b=4;h=3;gn=55;},line join=round]
				\path (0,0) coordinate (A)
				(b,0) coordinate (C)
				(-45:a) coordinate (B)
				($(A)+(gn:3)$) coordinate (A')
				($(B)-(A)+(A')$) coordinate (B')
				($(C)-(A)+(A')$) coordinate (C')
				($(C)!0.5!(B)$) coordinate (M)
				($(A)!0.6! (M)$) coordinate (O);
				\draw (A')--(A)--(B)--(C)--(C')--(A')--(B')--(C') (B)--(B');
				\draw[dashed] (M)--(A)--(C) (A)--(O)--(A');
				\path pic["\scriptsize$60^\circ$", angle eccentricity=2,draw,angle radius=7pt]{angle= O--A--A'};
				\foreach \t/\g in {A/180,B/-90,C/-90,A'/90,B'/90,C'/90,O/35,M/-35}{
					\draw[fill=white] (\t) circle (1pt) node[shift={(\g:7pt)},font=\scriptsize]{$ \t $};
				}
			\end{tikzpicture}
		\end{center}
		Gọi $M$ là trung điểm cạnh $BC$. Khi đó $AM=\dfrac{a\sqrt{3}}{2}$ và $AO=\dfrac{2}{3}AM=\dfrac{a\sqrt{3}}{3}$.\\
		Do $A'O\perp(ABC)$ tại điểm $O$ nên $AO$ là hình chiếu vuông góc của $AA'$ xuống $(ABC)$. Suy ra góc giữa đường thẳng $AA'$ và $(ABC)$ là góc $\widehat{A'AO}$, suy ra $\widehat{A'AO}=60^{\circ}$.\\
		Xét $\triangle A'AO$ vuông tại $O$ ta có $A'O=AO\cdot\tan 60=\dfrac{a\sqrt{3}}{3}\cdot\sqrt{3}=a$.\\
		Vậy thể tích khối lăng trụ là $V=A'O\cdot S_{\triangle ABC}=a\cdot\dfrac{a^2\sqrt{3}}{4}=\dfrac{a^3\sqrt{3}}{4}$.}
\end{ex}
\begin{ex}%Câu 9.
	(THPT Ngô Quyền - Ba Vì - Hải Phòng 2019) Cho lăng trụ tam giác $ABC.A'B'C'$ có đáy là tam giác đều cạnh $a$. Độ dài cạnh bên bằng 4 $a$. Mặt phẳng $(BCC'B')$ vuông góc với đáy và $\widehat{B'BC}=30^{\circ}$. Thể tích khối chóp $A.CC'B'$ là 
	\choice
	{$\dfrac{a^3\sqrt{3}}{2}$}
	{$\dfrac{a^3\sqrt{3}}{12}$}
	{$\dfrac{a^3\sqrt{3}}{18}$}
	{\True $\dfrac{a^3\sqrt{3}}{6}$}
	\loigiai{
	\begin{center}
		\begin{tikzpicture}[declare function={a=2;b=4;h=3;gn=75;},line join=round]
			\path (0,0) coordinate (A)
			(b,0) coordinate (C)
			(-45:a) coordinate (B)
			($(A)+(75:3)$) coordinate (A')
			($(B)-(A)+(A')$) coordinate (B')
			($(C)-(A)+(A')$) coordinate (C');
			\path ($(C)!0.8!(B)$) coordinate (H);
			
			\draw (A')--(A)--(B)--(C)--(C')--(A')--(B')--(C') (B)--(B')--(H);
			\draw[dashed] (A)--(C);
			\foreach \t/\g in {A/180,B/-90,C/-90,A'/90,B'/90,C'/90,H/-35}{
				\draw[fill=white] (\t) circle (1pt) node[shift={(\g:7pt)},font=\scriptsize]{$ \t $};
			}
		\end{tikzpicture}
	\end{center}
		Ta có $(BCC'B')\perp(ABC)$ (gt).\\
		Hạ $B'H\perp BC\Rightarrow B'H\perp(ABC)$ và $\widehat{B'BH}=\widehat{B'BC}=30^{\circ}$.\\
		Suy ra chiều cao của lăng trụ $ABC.A'B'C'$ là $h=B'H=BB'\cdot\sin 30^{\circ}=2a$.\\
		Diện tích đáy là\\
		Thể tích của khối lăng trụ là\\
		Thể tích khối chóp $A.CC'B'$ là $V=\dfrac{1}{3}V_{LT}=\dfrac{a^3\sqrt{3}}{6}$.}
\end{ex}
\begin{ex}%Câu 10.
	(Đề thử nghiệm 2017) Cho lăng trụ tam giác $ABC.A'B'C'$ có đáy $ABC$ là tam giác vuông cân tại $A$, cạnh $AC=2\sqrt{2}$. Biết $AC'$ tạo với mặt phẳng $(ABC)$ một góc $60^{\circ}$ và $AC'=4$. Tính thể tích $V$ của khối đa diện $ABCB'C'$. 
	\choice
	{$V=\dfrac{8}{3}$}
	{$V=\dfrac{16}{3}$}
	{$V=\dfrac{8\sqrt{3}}{3}$}
	{\True $V=\dfrac{16\sqrt{3}}{3}$}
	\loigiai{
		\begin{center}
			\begin{tikzpicture}[declare function={a=2;b=4;h=3;gn=75;},line join=round]
				\path (0,0) coordinate (B)
				(b,0) coordinate (C)
				(-45:a) coordinate (A)
				($(B)+(75:3)$) coordinate (B')
				($(A)-(B)+(B')$) coordinate (A')
				($(C)-(B)+(B')$) coordinate (C')
				(4.5,-1) coordinate (H);
				\path (5.2,1) node [scale=0.5,left]{$ 2\sqrt{3} $};
				\draw (B')--(B)--(A)--(H)--(C')--(B')--(A')--(C')--(A) (B')--(A)--(A');
				\draw[dashed] (B)--(C) (A)--(C)--(C');
				\path pic["\scriptsize$60^\circ$", angle eccentricity=2,draw,angle radius=9pt]{angle= H--A--C'};
				\path pic[draw,angle radius=5pt]{right angle= A--H--C'};
				\foreach \t/\g in {B/180,A/-90,C/-90,B'/90,A'/90,C'/90,H/0}{
					\draw[fill=white] (\t) circle (1pt) node[shift={(\g:7pt)},font=\scriptsize]{$ \t $};
				}
			\end{tikzpicture}
		\end{center}
		Phân tích: Tính thể tích của khối đa diện $ABCB'C'$ bằng thể tích khối của lăng trụ $ABC.A'B'C'$ trừ đi thể tích của khối chóp $A.A'B'C'$.\\
		Giả sử đường cao của lăng trụ là $C'H$. Khi đó góc giữa $AC'$ mặt phẳng $(ABC)$ là góc $\widehat{C'AH}=60^{\circ}$.\\
		Ta có: $\sin 60^{\circ}=\dfrac{C'H}{AC'}\Rightarrow C'H=2\sqrt{3};S_{\triangle ABC}=4$; $V_{ABC.A'B'C'}=C'H\cdot S_{\triangle ABC}=2\sqrt{3}\cdot\dfrac{1}{2}\cdot (2\sqrt{2})^2=8\sqrt{3}$.\\
		$V_{A.A'B'C'}=\dfrac{1}{3}C'H\cdot S_{\triangle ABC}=\dfrac{1}{3}\cdot V_{ABC.A'B'C'}=\dfrac{8\sqrt{3}}{3}$; $V_{ABB'C'C}=V_{ABC.A'B'C'}-V_{A.A'B'C'}=8\sqrt{3}-\dfrac{8\sqrt{3}}{3}=\dfrac{16\sqrt{3}}{3}$.}
\end{ex}
\begin{ex}%Câu 11.
	(THPT Hoàng Hoa Thám - Hưng Yên 2019) Cho lăng trụ tam giác $ABC.A'B'C'$ có độ dài cạnh bên bằng $8a$ và khoảng cách từ điểm $A$ đến các đường thẳng $BB'$, $CC'$ lần lượt bằng $2a$ và $4a$. Biết góc giữa hai mặt phẳng $(ABB'A')$ và $(ACC'A')$ bằng $60^{\circ}$. Tính thể tích khối lăng trụ $ABC.A'B'C'$. 
	\choice
	{$\dfrac{16}{3}\sqrt{3}a^3$}
	{$8\sqrt{3}a^3$}
	{$24\sqrt{3}a^3$}
	{\True $16\sqrt{3}a^3$}
	\loigiai{
		\begin{center}
			\begin{tikzpicture}[declare function={a=3;b=4;h=4;},line join=round]
				\path (0,0) coordinate (A)
				(b,0) coordinate (K)
				(-35:a) coordinate (H)
				($(A)+(90:h)$) coordinate (A')
				($(H)-(A)+(A')$) coordinate (H')
				($(K)-(A)+(A')$) coordinate (K');
				\path ($(H')!1.2!(H)$) coordinate (B)
				($(K')!1.3!(K)$) coordinate (C)
				($(H')!0.2!(H)$) coordinate (B')
				($(K')!0.3!(K)$) coordinate (C');	
				\draw (A')--(A)--(H)--(K)--(K')--(A')--(H')--(K') (H)--(H') (A')--(B')--(C') (A)--(B)--(C)--(K) (H)--(B);
				\draw[dashed] (C)--(A)--(K) (A')--(C');
				\path pic["\scriptsize$60^\circ$", angle eccentricity=2,draw,angle radius=11pt]{angle= H--A--K};
				\path pic[draw,angle radius=5pt]{right angle= A'--A--K};
				\foreach \t/\g in {A/180,H/-40,K/0,A'/90,H'/90,K'/90,B/0,C/0,B'/0,C'/0}{
					\draw[fill=white] (\t) circle (1pt) node[shift={(\g:7pt)},font=\scriptsize]{$ \t $};
				}
			\end{tikzpicture}
		\end{center}
		Gọi $H, K$ lần lượt là hình chiếu vuông góc của $A$ trên $BB',CC'$.\\
		Ta có $HA\perp BB', KA\perp CC'\Rightarrow A'A\perp(AHK)$ do đó $\angle AHK=60^{\circ}$.\\
		Khi đó $HK^2=AK^2+AH^2-2AK\cdot AH\cdot\cos 60^{\circ}=12a^2\Rightarrow AK^2=HK^2+AH^2$. Suy ra tam giác $AHK$ vuông tại $H$.\\
		Gọi $H', K'$ lần lượt là hình chiếu vuông góc của $A'$ trên $BB',CC'$. Ta có $V_{A.BCKH}=V_{A.B'C'K'H'}$.\\
		Khi đó $V_{ABC.A'B'C'}=V_{AHK\cdot A'H'K'}=AA'\cdot S_{AHK}=16\sqrt{3}a^3$.}
\end{ex}
\begin{ex}%Câu 12.
	(Chuyên - KHTN - Hà Nội - 2019) Cho hình lăng trụ $ABC.A'B'C'$ có đáy $ABC$ là tam giác đều cạnh $a$, hình chiếu vuông góc của $A'$ trên $(ABC)$ là trung điểm cạnh $AB$, góc giữa đường thẳng $A'C$ và mặt phẳng đáy bằng $60^{\circ}$. Thể tích khối lăng trụ $ABC.A'B'C'$ bằng
	\choice
	{$\dfrac{\sqrt{2}a^3}{4}$}
	{$\dfrac{\sqrt{3}a^3}{4}$}
	{\True $\dfrac{3\sqrt{3}a^3}{8}$}
	{$\dfrac{3\sqrt{3}a^3}{4}$}
	\loigiai{
	\begin{center}
		\begin{tikzpicture}[declare function={a=2;b=4;h=4;gn=55;},line join=round]
			\path (0,0) coordinate (A)
			(b,0) coordinate (B)
			(-35:a) coordinate (C)
			($(A)+(gn:h)$) coordinate (A')
			($(C)-(A)+(A')$) coordinate (C')
			($(B)-(A)+(A')$) coordinate (B');
			\path ($(A)!0.5!(B)$) coordinate (H);
			\path (A)--(H)node[pos=0.5,sloped,black]{$|$} (H)--(B)node[pos=0.5,sloped,black]{$|$} ;
			\path pic[angle radius=6mm,draw=blue,"$60^\circ$",angle eccentricity=1.4] {angle = H--C--A'};
			
			\draw (A')--(A)--(C)--(B)--(B')--(A')--(C')--(B') (A)--(C')--(C)--(A');
			\draw[dashed] (A)--(B) (C)--(H)--(A');
			\foreach \t/\g in {A/180,C/-90,B/-90,A'/90,C'/90,B'/90,H/30}{
				\draw[fill=white] (\t) circle (1pt) node[shift={(\g:7pt)},font=\scriptsize]{$ \t $};
			}
		\end{tikzpicture}
	\end{center}
		Gọi $H$ là hình chiếu vuông góc của $A'$ trên mặt phẳng $(ABC)$.\\
		Ta có: $A'H\perp(ABC)\Rightarrow HC$ là hình chiếu vuông góc của $A'C$ lên mặt phẳng $(ABC)$ \\
		$ \Rightarrow\widehat{\left(A'C,(ABC)\right)}=\widehat{(A'C,HC)}=\widehat{A'CH}=60^{\circ} $.\\
		$CH=\dfrac{a\sqrt{3}}{2}$.\\
		Xét tam giác vuông $A'HC$, ta có: $A'H=CH\cdot\tan 60^{\circ}=\dfrac{a\sqrt{3}}{2}\cdot\sqrt{3}=\dfrac{3a}{2}$, $S_{ABC}=\dfrac{a^2\sqrt{3}}{4}$.\\
		Vậy thể tích của khối lăng trụ $ABC.A'B'C'$ là $V_{ABC.A'B'C'}=S_{ABC}\cdot A'H=\dfrac{a^2\sqrt{3}}{4}\cdot\dfrac{3a}{2}=\dfrac{3\sqrt{3}a^3}{8}$.}
\end{ex}
\begin{ex}%Câu 13.
	(Hội 8 trường chuyên ĐBSH - 2019) Cho lăng trụ $ABC.A_1B_1C_1$ có diện tích mặt bên $(ABB_1A_1)$ bằng $4$, khoảng cách giữa cạnh $CC_1$ đến mặt phẳng $(ABB_1A_1)$ bằng 6. Tính thể tích khối lăng trụ $ABC.A_1B_1C_1$. 
	\choice
	{\True $12$}
	{$18$}
	{$24$}
	{$9$}
	\loigiai{
		\begin{center}
			\begin{tikzpicture}[declare function={a=2;b=4;h=3;gn=75;},line join=round]
				\path (0,0) coordinate (A_1)
				(b,0) coordinate (C_1)
				(-45:a) coordinate (B_1)
				($(A_1)+(75:3)$) coordinate (A)
				($(B_1)-(A_1)+(A)$) coordinate (B)
				($(C_1)-(A_1)+(A)$) coordinate (C);
				\draw (A)--(A_1)--(B_1)--(C_1)--(C)--(A)--(B)--(C)--(B_1)--(B);
				\draw[dashed] (C)--(A_1)--(C_1);
				\foreach \t/\g in {A_1/180,B_1/-90,C_1/-90,A/90,B/90,C/90}{
					\draw[fill=white] (\t) circle (1pt) node[shift={(\g:7pt)},font=\scriptsize]{$ \t $};
				}
			\end{tikzpicture}
		\end{center}
		Ta có: $V_{C.ABB_1A_1}=\dfrac{1}{3}\mathrm{d}\left(C,(ABB_1A_1)\right)\cdot S_{ABB_1A_1}=\dfrac{1}{3}\cdot 4\cdot 6=8$ (đvtt).\\
		$V_{C.ABB_1A_1}=V_{ABC.A_1B_1C_1}-V_{C.C_1B_1A_1}=V_{ABC.A_1B_1C_1}-\dfrac{1}{3}V_{ABC.A_1B_1C_1}=\dfrac{2}{3}V_{ABC.A_1B_1C_1}$ \\
		$ \Rightarrow V_{ABC.A_1B_1C_1}=\dfrac{3}{2}\cdot V_{C.ABB_1A_1}=\dfrac{3}{2}\cdot 8=12 $ (đvtt).}
\end{ex}
\begin{ex}%Câu 14.
	(chuyên Hùng Vương Gia Lai 2019) Cho khối lăng trụ $ABC.A'B'C',$ tam giác $A'BC$ có diện tích bằng 1 và khoảng cách từ $A$ đến mặt phẳng $(A'BC)$ bằng 2. Thể tích khối lăng trụ đã cho bằng
	\choice
	{6}
	{3}
	{\True 2}
	{1}
	\loigiai{
	\begin{center}
		\begin{tikzpicture}[declare function={a=2;b=4;h=3;gn=65;},line join=round]
			\path (0,0) coordinate (A)
			(b,0) coordinate (C)
			(-45:a) coordinate (B)
			($(A)+(gn:3)$) coordinate (A')
			($(B)-(A)+(A')$) coordinate (B')
			($(C)-(A)+(A')$) coordinate (C')
			($(C)!0.5!(B)$) coordinate (M)
			($(A)!0.4! (M)$) coordinate (H);
			\draw (A')--(A)--(B)--(C)--(C')--(A')--(B')--(C') (A')--(B)--(B');
			\draw[dashed] (A)--(C) (H)--(A')--(C);
			\foreach \t/\g in {A/180,B/-90,C/-90,A'/90,B'/90,C'/90,H/180}{
				\draw[fill=white] (\t) circle (1pt) node[shift={(\g:7pt)},font=\scriptsize]{$ \t $};
			}
		\end{tikzpicture}
	\end{center}
		Gọi $H$ là hình chiếu vuông góc của $A'$ trên mp $(ABC)$ suy ra $A'H$ là chiều cao của lăng trụ.\\
		Xét khối chóp $A.A'BC$ có diện tích đáy $B=S_{A'BC}=1$, chiều cao $h=\mathrm{d}\left(A,(A'BC)\right)=2$ suy ra thể.\\
		tích của khối chóp $A.A'BC$ là $V_{A.A'BC}=\dfrac{1}{3}Bh=\dfrac{1}{3}\cdot 1\cdot 2=\dfrac{2}{3}$.\\
		Mặt khác $\heva{&V_{A.A'BC}=V_{A'\cdot ABC}=\dfrac{1}{3}S_{ABC}\cdot A'H=\dfrac{2}{3}\\&V_{ABC.A'B'C'}=S_{ABC}\cdot A'H}\Rightarrow V_{ABC.A'B'C'}=3V_{A.A'BC}=3\cdot\dfrac{2}{3}=2$.\\
		* Cách khác.\\
		Ta thấy lăng trụ $ABC.A'B'C'$ được chia thành ba khối chóp có thể thích bằng nhau là\\
		$A'\cdot ABC, A'\cdot BCB', A'\cdot B'C'C$.\\
		Mà $V_{A'\cdot ABC}=V_{A.A'BC}=\dfrac{1}{3}Bh=\dfrac{1}{3}\cdot 1\cdot 2=\dfrac{2}{3}$ suy ra $V_{ABC.A'B'C'}=3V_{A.A'BC}=3\cdot\dfrac{2}{3}=2$.}
\end{ex}
\begin{ex}%Câu 15.
	(Đại học Hồng Đức –Thanh Hóa – 2019) Một khối lăng trụ tam giác có đáy là tam giác đều cạnh 3, cạnh bên bằng $2\sqrt{3}$ và tạo với mặt phẳng đáy một góc $60^{\circ}$. Khi đó thể tích khối lăng trụ là
	\choice
	{$\dfrac{27}{4}$}
	{$\dfrac{9\sqrt{3}}{4}$}
	{\True $\dfrac{27\sqrt{3}}{4}$}
	{$\dfrac{9}{4}$}
	\loigiai{
	\begin{center}
		\begin{tikzpicture}[declare function={a=2;b=4;h=3;gn=75;},line join=round]
			\path (0,0) coordinate (C)
			(b,0) coordinate (A)
			(-45:a) coordinate (B)
			($(C)+(65:3)$) coordinate (C')
			($(B)-(C)+(C')$) coordinate (B')
			($(A)-(C)+(C')$) coordinate (A')
			($(A)!0.5!(B)$) coordinate (M)
			($(C)!0.4! (M)$) coordinate (H);
			\draw (C')--(C)--(B)--(A)--(A')--(C')--(B')--(A') (B)--(B');
			\draw[dashed] (C')--(H)--(C)--(A) ;
			\foreach \t/\g in {C/180,B/-90,A/-90,C'/90,B'/90,A'/90,H/0}{
				\draw[fill=white] (\t) circle (1pt) node[shift={(\g:7pt)},font=\scriptsize]{$ \t $};
			}
		\end{tikzpicture}
	\end{center}
		Gọi $H$ là hình chiếu vuông góc của $C'$ xuống $mp(ABC)$, khi đó góc hợp bởi $CC'$ và $mp(ABC)$ là $\widehat{C'CH}$. Theo đề bài: $\widehat{C'CH}=60^{\circ}\Rightarrow C'H=C'C\cdot\sin 60^{\circ}=2\sqrt{3}\cdot\dfrac{\sqrt{3}}{2}=3$.\\
		Lại có $\triangle ABC$ đều cạnh bằng 3 nên $S_{ABC}=\dfrac{\sqrt{3}}{4}\cdot 3^2=\dfrac{9\sqrt{3}}{4}$.}
\end{ex}
\begin{ex}%Câu 16.
	Do đó $V_{ABC.A'B'C'}=S_{ABC}\cdot C'H=\dfrac{9\sqrt{3}}{4}\cdot 3=\dfrac{27\sqrt{3}}{4}$. (Sở Hà Nội 2019) Cho lăng trụ $ABC.A'B'C'$ có đáy $ABC$ là tam giác vuông tại $B$, đường cao $BH$. Biết $A'H\perp(ABC)$ và $AB=1,AC=2,AA'=\sqrt{2}$. Thể tích của khối lăng trụ đã cho bằng
	\choice
	{$\dfrac{\sqrt{21}}{12}$}
	{$\dfrac{\sqrt{7}}{4}$}
	{\True $\dfrac{\sqrt{21}}{4}$}
	{$\dfrac{3\sqrt{7}}{4}$}
	\loigiai{
	\begin{center}
		\begin{tikzpicture}[declare function={a=2;b=4;h=3;gn=75;},line join=round]
			\path (0,0) coordinate (A)
			(b,0) coordinate (C)
			(-45:a) coordinate (B)
			($(A)+(55:4)$) coordinate (A')
			($(B)-(A)+(A')$) coordinate (B')
			($(C)-(A)+(A')$) coordinate (C')
			($(A)!0.5!(C)$) coordinate (H);
			\draw (A')--(A)--(B)--(C)--(C')--(A')--(B')--(C') (B)--(B');
			\draw[dashed] (A)--(C) (A')--(H)--(B);
			\foreach \t/\g in {A/180,B/-90,C/-90,A'/90,B'/90,C'/90,H/150}{
				\draw[fill=white] (\t) circle (1pt) node[shift={(\g:7pt)},font=\scriptsize]{$ \t $};
			}
		\end{tikzpicture}
	\end{center}
		Tam giác $ABC$ vuông tại $B$ có $AB=1;AC=2$ nên $\mathrm{d}\left(A;(P)\right)=\dfrac{\sqrt{6}}{2};\mathrm{d}\left(A;(Q)\right)=\sqrt{6}\Rightarrow\mathrm{d}\left(A;(Q)\right)=\mathrm{d}\left(A;(P)\right)+\mathrm{d}\left((Q);(P)\right)$.\\
		Độ dài của đường cao $BH$: $BH=\dfrac{AB\cdot BC}{AC}=\dfrac{\sqrt{3}}{2}$. Suy ra $AH=\dfrac{\sqrt{3}}{2}\colon\sqrt{3}=\dfrac{1}{2}$.\\
		Khi đó độ dài đường cao $A'H$ của hình lăng trụ bằng $A'H=\sqrt{AA'^2-AH^2}=\sqrt{2-\dfrac{1}{4}}=\dfrac{\sqrt{7}}{2}$.\\
		Thể tích khối lăng trụ đã cho bằng $V=\dfrac{1}{2}AB\cdot BC\cdot A'H=\dfrac{1}{2}\cdot 1\cdot\sqrt{3}\dfrac{\sqrt{7}}{2}=\dfrac{\sqrt{21}}{4}$.}
\end{ex}
\begin{ex}%Câu 17.
	(THPT Lương Thế Vinh Hà Nội 2019) Cho hình lăng trụ $ABC.A'B'C'$ có đáy là tam giác đều cạnh $a$, góc giữa cạnh bên và mặt phẳng đáy bằng $30^{\circ}$. Hình chiếu của $A'$ xuống $(ABC)$ là trung điểm $BC$. Tính thể tích khối lăng trụ $ABC.A'B'C'$. 
	\choice
	{\True $\dfrac{a^3\sqrt{3}}{8}$}
	{$\dfrac{a^3}{8}$}
	{$\dfrac{a^3\sqrt{3}}{24}$}
	{$\dfrac{a^3\sqrt{3}}{4}$}
	\loigiai{
		\begin{center}
			\begin{tikzpicture}[declare function={a=2;b=4;h=3;gn=75;},line join=round]
				\path (0,0) coordinate (A)
				(b,0) coordinate (C)
				(-45:a) coordinate (B)
				($(A)+(75:3)$) coordinate (A')
				($(B)-(A)+(A')$) coordinate (B')
				($(C)-(A)+(A')$) coordinate (C')
				($(B)!0.5!(C)$) coordinate (H);
				\draw (A')--(A)--(B)--(C)--(C')--(A')--(B')--(C') (B)--(B');
				\draw[dashed] (A')--(H)--(A)--(C);
				\foreach \t/\g in {A/180,B/-90,C/-90,A'/90,B'/90,C'/90,H/-45}{
					\draw[fill=white] (\t) circle (1pt) node[shift={(\g:7pt)},font=\scriptsize]{$ \t $};
				}
			\end{tikzpicture}
		\end{center}
		Gọi $H$ là trung điểm $BC$ suy ra $A'H\perp(ABC)$.\\
		Ta có $\left(A'A,(ABC)\right)=(A'A,AH)=\widehat{A'AH}=30^{\circ}$.\\
		Ta có $AH=\dfrac{a\sqrt{3}}{2}$.\\
		Ta có $A'H=AH\cdot\tan 30^{\circ}=\dfrac{a}{2}$ và $S_{ABC}=\dfrac{a^2\sqrt{3}}{4}$.\\
		Vậy $V=A'H\cdot S_{ABC}=\dfrac{a^3\sqrt{3}}{8}$.}
\end{ex}
\begin{ex}%Câu 18.
	(THPT Việt Đức Hà Nội 2019) Cho hình lăng trụ $ABCD.A'B'C'D'$ có đáy $ABCD$ là hình thoi cạnh $a$, $\widehat{ABC}=60^{\circ}$. Chân đường cao hạ từ $B'$ trùng với tâm $O$ của đáy $ABCD$; góc giữa mặt phẳng $(BB'C'C)$ với đáy bằng $60^{\circ}$. Thể tích lăng trụ bằng 
	\choice
	{\True $\dfrac{3a^3\sqrt{3}}{8}$}
	{$\dfrac{2a^3\sqrt{3}}{9}$}
	{$\dfrac{3a^3\sqrt{2}}{8}$}
	{$\dfrac{3a^3}{4}$}
	\loigiai{
		\begin{center}
			\begin{tikzpicture}[declare function={a=2;b=4;h=3;},line join=round]
				\path (0,0) coordinate (A)
				(35:a) coordinate (B)
				(b,0) coordinate (D)
				($(D)-(A)+(B)$) coordinate (C)
				($(A)!0.5!(C)$)	coordinate (O)
				($(O)+(90:h)$) coordinate (B')
				($(A)-(B)+(B')$)  coordinate (A')
				($(D)-(B)+(B')$) coordinate (D')
				($(C)-(B)+(B')$) coordinate (C')
				($(B)!0.6!(C)$)	coordinate (H);
				\draw (A')--(A)--(D)--(C)--(C')--(B')--(A')--(D')--(C') (D)--(D');
				\draw[dashed] (D)--(B)--(C)--(A) (O)--(B')--(B)--(A) (O)--(H)--(B');
				\foreach \t/\g in {B/180,H/65,O/-45,A/-90,D/-90,B'/90,A'/90,D'/90,C'/0,C/0}{
					\draw[fill=white] (\t) circle (1pt) node[shift={(\g:7pt)},font=\scriptsize]{$ \t $};
				}
			\end{tikzpicture}
		\end{center}
		$ABCD$ là hình thoi nên $AB=BC$. Lại có $\widehat{ABC}=60^{\circ}$ nên $\triangle ABC$ là tam giác đều. $OH\perp BC$. Góc giữa mặt phẳng $(BB'C'C)$ với đáy khi đó là $\widehat{B'HO}=60^{\circ}$.\\
		Ta có $\dfrac{1}{OH^2}=\dfrac{1}{OB^2}+\dfrac{1}{OC^2}=\dfrac{1}{\dfrac{3a^2}{4}}+\dfrac{1}{\dfrac{a^2}{4}}=\dfrac{4}{3a^2}+\dfrac{4}{a^2}=\dfrac{16}{3a^2}$. $\Rightarrow OH=\dfrac{a\sqrt{3}}{4}$.\\
		Theo giả thiết, $B'O$ là đường cao lăng trụ $ABCD.A'B'C'D'$.\\
		$B'O=OH\cdot tan\widehat{B'HO}=\dfrac{a\sqrt{3}}{4}\tan 60^{\circ}=\dfrac{3a}{4}$.\\
		$V_{ABCD.A'B'C'D'}=S_{day}\cdot h=\dfrac{a^2\sqrt{3}}{2}\cdot\dfrac{3a}{4}=\dfrac{3a^3\sqrt{3}}{8}$.}
\end{ex}
\begin{ex}%Câu 19.
	(THPT Lê Quy Đôn Điện Biên 2019) Cho lăng trụ $ABC.A'B'C'$ có đáy là tam giác đều cạnh $a$, hình chiếu vuông góc của điểm lên mặt phẳng $(ABC)$ trùng với trọng tâm tam giác $ABC$. Biết khoảng cách giữa hai đường thẳng và $BC$ bằng $\dfrac{a\sqrt{3}}{4}$. Tính theo $a$ thể tích của khối lăng trụ đã cho. 
	\choice
	{$\dfrac{a^3\sqrt{3}}{3}$}
	{$\dfrac{a^3\sqrt{3}}{24}$}
	{$\dfrac{a^3\sqrt{3}}{6}$}
	{\True $\dfrac{a^3\sqrt{3}}{12}$}
	\loigiai{
	\begin{center}
		\begin{tikzpicture}[declare function={a=2;b=4;h=3;gn=75;},line join=round]
			\path (0,0) coordinate (A)
			(b,0) coordinate (C)
			(-45:a) coordinate (B)
			($(A)+(55:3)$) coordinate (A')
			($(B)-(A)+(A')$) coordinate (B')
			($(C)-(A)+(A')$) coordinate (C')
			($(B)!0.5!(C)$)	coordinate (M)
			($(A)!0.6!(M)$)	coordinate (G);
			\path ($(A)!(M)! (A')$) coordinate (H);	
			\draw (A')--(A)--(B)--(C)--(C')--(A')--(B')--(C') (B)--(B');
			\draw[dashed] (H)--(M)--(A)--(C) (A')--(G);
			\foreach \x/\y/\z in {M/H/A',M/G/A'}{
				\path pic[draw,angle radius=5pt]{right angle= \x--\y--\z};
			}
			\foreach \t/\g in {A/180,B/-90,C/-90,A'/90,B'/90,C'/90,M/0,G/-90,H/135}{
				\draw[fill=white] (\t) circle (1pt) node[shift={(\g:7pt)},font=\scriptsize]{$ \t $};
			}
		\end{tikzpicture}
	\end{center}
		Ta có $\heva{&BC\perp AM\\&BC\perp A'G}\Rightarrow BC\perp AA'$.\\
		Kẻ $MH\perp AA'$ tại $H$, suy ra $MH$ là đoạn vuông góc chung của giữa hai đường thẳng và $BC$.\\
		Tam giác $MHA$ vuông tại $H$ có $AH=\sqrt{AM^2-AH^2}=\dfrac{3}{4}a$.\\
		Tam giác $A'GA$ đồng dạng tam giác $MHA$ nên $\dfrac{A'G}{MH}=\dfrac{GA}{HA}\Rightarrow A'G=\dfrac{MH\cdot GA}{HA}=\dfrac{a}{3}$.\\
		Thể tích khối lăng trụ là $V=S_{ABC}\cdot A'G=\dfrac{a^3\sqrt{3}}{12}$.}
\end{ex}
\begin{ex}%Câu 20.
	(Toán Học Tuổi Trẻ 2019) Cho hình lăng trụ $ABC.A'B'C'$ có $AA'=2a$, tam giác $ABC$ vuông tại $C$ và $\widehat{BAC}=60^{\circ}$, góc giữa cạnh bên $BB'$ và mặt đáy $(ABC)$ bằng $60^{\circ}$. Hình chiếu vuông góc của $B'$ lên mặt phẳng $(ABC)$ trùng với trọng tâm của tam giác $ABC$. Thể tích của khối tứ diện $A'\cdot ABC$ theo $a$ bằng
	\choice
	{$\dfrac{9a^3}{208}$}
	{$\dfrac{3a^3}{26}$}
	{\True $\dfrac{9a^3}{26}$}
	{$\dfrac{27a^3}{208}$}
	\loigiai{
		\begin{center}
			\begin{tikzpicture}[declare function={a=2;b=4;h=3;gn=75;},line join=round]
				\path (0,0) coordinate (B)
				(b,0) coordinate (C)
				(-45:a) coordinate (A)
				($(B)+(50:3)$) coordinate (B')
				($(A)-(B)+(B')$) coordinate (A')
				($(C)-(B)+(B')$) coordinate (C')
				($(B)!0.5!(C)$)	coordinate (M)
				($(B)!0.5!(A)$)	coordinate (N)
				($(C)!0.5!(A)$)	coordinate (I);
				\path (intersection of A--M and C--N) coordinate (G);
				\path pic["\scriptsize$60^\circ$", angle eccentricity=2,draw,angle radius=7pt]{angle= I--B--B'};
				\path pic["\scriptsize$60^\circ$", angle eccentricity=2,draw,angle radius=11pt]{angle= B--C--A};
				\draw (B')--(B)--(A)--(C)--(C')--(B')--(A')--(C') (A)--(A');
				\draw[dashed] (I)--(B)--(C) (C)--(N) (A)--(M) (B')--(G);
				\foreach \t/\g in {B/180,I/-45,G/170,A/-90,C/-90,B'/90,A'/90,C'/90,M/45,N/-135}{
					\draw[fill=white] (\t) circle (1pt) node[shift={(\g:7pt)},font=\scriptsize]{$ \t $};
				}
			\end{tikzpicture}
		\end{center}
		Ta có\\
		$B'G=BB'\sin 60^{\circ}=2a\cdot\dfrac{\sqrt{3}}{2}=a\sqrt{3}$\\
		$BG=BB'\cos 60^{\circ}=2a\cdot\dfrac{1}{2}=a\Rightarrow BI=\dfrac{3}{2}BG=\dfrac{3a}{2}$.\\
		Đặt $AC=2x(x>0)\Rightarrow CI=x; BC=AC\cdot\tan 60^{\circ}=2x\sqrt{3}$.\\
		Khi đó\\
		$x^2+(2x\sqrt{3})^2=\left(\dfrac{3a}{2}\right)^2\Leftrightarrow x=\dfrac{3a\sqrt{13}}{26}\Rightarrow S_{\triangle ABC}=\dfrac{1}{2}AC\cdot BC=\dfrac{1}{2}\cdot 2\cdot\dfrac{3a\sqrt{13}}{26}\cdot 2\cdot\dfrac{3a\sqrt{13}}{26}\cdot\sqrt{3}=\dfrac{9a^2\sqrt{3}}{26}$.\\
		Vậy $V_{A'\cdot ABC}=\dfrac{1}{3}\cdot\dfrac{9a^2\sqrt{3}}{26}\cdot a\sqrt{3}=\dfrac{9a^3}{26}$.}
\end{ex}
\begin{ex}%Câu 21.
	(THPT Thiệu Hóa – Thanh Hóa 2019) Cho lăng trụ tam giác $ABC.A'B'C'$ có đáy $ABC$ là tam giác đều cạnh $a$. Hình chiếu của điểm $A'$ trên mặt phẳng $(ABC)$ trùng vào trọng tâm $G$ của tam giác $ABC$. Biết tam giác $A'BB'$ có diện tích bằng $\dfrac{2a^2\sqrt{3}}{3}$. Tính thể tích khối lăng trụ $ABC.A'B'C'$. 
	\choice
	{$\dfrac{6a^3\sqrt{2}}{7}$}
	{\True $\dfrac{3a^3\sqrt{7}}{8}$}
	{$\dfrac{3a^3\sqrt{5}}{8}$}
	{$\dfrac{3a^3\sqrt{3}}{8}$}
	\loigiai{
		\begin{center}
			\begin{tikzpicture}[declare function={a=2;b=4;h=3;gn=65;},line join=round]
				\path (0,0) coordinate (A)
				(b,0) coordinate (C)
				(-45:a) coordinate (B)
				($(A)+(50:3)$) coordinate (A')
				($(B)-(A)+(A')$) coordinate (B')
				($(C)-(A)+(A')$) coordinate (C')
				($(A)!0.5!(B)$)	coordinate (M)
				($(C)!0.5!(B)$)	coordinate (N);
				\path (intersection of A--N and C--M) coordinate (G);
				\draw (A')--(A)--(B)--(C)--(C')--(A')--(B')--(C') (B)--(B');
				\draw[dashed] (N)--(A)--(C)--(M) (A')--(G) ;
				\foreach \t/\g in {A/180,B/-90,C/-90,A'/90,B'/90,C'/90,M/210,N/-35,G/130}{
					\draw[fill=white] (\t) circle (1pt) node[shift={(\g:7pt)},font=\scriptsize]{$ \t $};
				}
			\end{tikzpicture}
		\end{center}
		+ Ta có $\heva{&AB\perp CM\\&AB\perp A'M}\Rightarrow AB\perp(A'CM)\Rightarrow AB\perp A'M$.\\
		Nên $S_{\triangle A'AB}=\dfrac{1}{2}A'M\cdot AB=\dfrac{2a^2\sqrt{3}}{3}\Leftrightarrow A'M=\dfrac{4a\sqrt{3}}{3}$.\\
		Do $\triangle ABC$ đều cạnh bằng $a$ nên $GM=\dfrac{1}{3}CM=\dfrac{a\sqrt{3}}{6}$.\\
		+ Trong $\triangle A'GM$ vuông tại $G$ ta có $A'G=\sqrt{A'M^2-GM^2}=\dfrac{a\sqrt{21}}{2}$.\\
		Vậy $V_{ABC.A'B'C'}=A'G\cdot\mathrm{\,d}t(\triangle ABC)=\dfrac{a\sqrt{21}}{2}\cdot\dfrac{a^2\sqrt{3}}{4}=\dfrac{3a^3\sqrt{7}}{8}$.}
\end{ex}
\begin{ex}%Câu 22.
	(Cụm liên trường Hải Phòng 2019) Cho hình lăng trụ $ABC.A'B'C'$ có đáy $ABC$ là tam giác vuông cân tại $B$ và $AC=2a$. Hình chiếu vuông góc của $A'$ trên mặt phẳng $(ABC)$ là trung điểm $H$ của cạnh $AB$ và $AA'=a\sqrt{2}$. Tính thể tích $V$ của khối lăng trụ đã cho. 
	\choice
	{$V=\dfrac{a^3\sqrt{6}}{6}$}
	{\True $V=\dfrac{a^3\sqrt{6}}{2}$}
	{$V=2a^2\sqrt{2}$}
	{$V=a^3\sqrt{3}$}
	\loigiai{
		\begin{center}
			\begin{tikzpicture}[declare function={a=2;b=4;h=3;gn=75;},line join=round]
				\path (0,0) coordinate (A)
				(b,0) coordinate (C)
				(-45:a) coordinate (B)
				($(A)+(75:3)$) coordinate (A')
				($(B)-(A)+(A')$) coordinate (B')
				($(C)-(A)+(A')$) coordinate (C');
				\path ($(A)!0.3!(B)$) coordinate (H);	
				\draw (H)--(A')--(A)--(B)--(C)--(C')--(A')--(B')--(C') (B)--(B');
				\draw[dashed] (A)--(C);
				\foreach \t/\g in {A/180,B/-90,C/-90,A'/90,B'/90,C'/90,H/-145}{
					\draw[fill=white] (\t) circle (1pt) node[shift={(\g:7pt)},font=\scriptsize]{$ \t $};
				}
			\end{tikzpicture}
		\end{center}
		Tam giác $ABC$ vuông cân tại $B$ cạnh $AC=2a$ nên suy ra $AB=a\sqrt{2}$, có diện tích đáy.\\
		$S_{\triangle ABC}=\dfrac{1}{2}AB^2=\dfrac{1}{2}(a\sqrt{2})^2=a^2$.\\
		$H$ là hình chiếu vuông góc của $A'$ trên mặt phẳng $(ABC)$ nên $A'H$ là chiều cao của khối lăng trụ. Thể tích là $V=A'H\cdot S_{\triangle ABC}$.\\
		$H$ là trung điểm của cạnh $AB\Rightarrow AH=\dfrac{a\sqrt{2}}{2}\Rightarrow A'H=\sqrt{AA'^2-AH^2}=\sqrt{2a^2-\dfrac{2a^2}{4}}=\dfrac{a\sqrt{6}}{2}$.\\
		Suy ra $V=A'H\cdot S_{\triangle ABC}=\dfrac{a\sqrt{6}}{2}\cdot a^2=\dfrac{a^3\sqrt{6}}{2}$.}
\end{ex}
\begin{ex}%Câu 23.
	(THPT Trần Phú 2019) Cho lăng trụ $ABC.A'B'C'$ có đáy là tam giác đều cạnh $2a$, cạnh bên $AA'=2a$. Hình chiếu vuông góc của $A'$ lên mặt phẳng $(ABC)$ là trung điểm $BC$. Thể tích của khối lăng trụ đã cho là
	\choice
	{\True $a^3\sqrt{3}$}
	{$2a^3\sqrt{3}$}
	{$3a^3\sqrt{2}$}
	{$2a^3\sqrt{6}$}
	\loigiai{
		\begin{center}
			\begin{tikzpicture}[declare function={a=2;b=4;h=h;gn=75;},line join=round]
				\path (0,0) coordinate (A)
				(b,0) coordinate (C)
				(-45:a) coordinate (B)
				($(A)+(45:4)$) coordinate (A')
				($(B)-(A)+(A')$) coordinate (B')
				($(C)-(A)+(A')$) coordinate (C');
				\path ($(C)!0.5!(B)$) coordinate (H);	
				\draw (A')--(A)--(B)--(C)--(C')--(A')--(B')--(C') (B)--(B');
				\draw[dashed] (A')--(H)--(A)--(C) ;
				\foreach \t/\g in {A/180,B/-90,C/-90,A'/90,B'/90,C'/90,H/-45}{
					\draw[fill=white] (\t) circle (1pt) node[shift={(\g:7pt)},font=\scriptsize]{$ \t $};
				}
			\end{tikzpicture}
		\end{center}
		Gọi $H$ là hình chiếu của $A'$ trên mặt phẳng $(ABC)$, suy ra $H$ là trung điểm của $BC$.\\
		Tam giác $ABC$ đều cạnh $2a$, suy ra $AH=a\sqrt{3}$.\\
		Đường cao hình lăng trụ: $h=A'H=\sqrt{4a^2-3a^2}=a$.\\
		Vậy thể tích lăng trụ: $V=S_{\triangle ABC}\cdot h=\dfrac{1}{2}AH\cdot BC\cdot A'H=\dfrac{1}{2}a\sqrt{3}\cdot 2a\cdot a=a^3\sqrt{3}$.}
\end{ex}
\begin{ex}%Câu 24.
	Cho hình lăng trụ $ABC.A'B'C'$ có đáy $ABC$ là tam giác đều cạnh $a$, $AA'=\dfrac{3a}{2}$. Biết rằng hình chiếu vuông góc của điểm $A'$ lên mặt phẳng $(ABC)$ là trung điểm của cạnh $BC$. Tính thể tích $V$ của khối lăng trụ đó theo $a$. 
	\choice
	{$V=a^3\sqrt{\dfrac{3}{2}}$}
	{$V=\dfrac{2a^3}{3}$}
	{\True $V=\dfrac{3a^3}{4\sqrt{2}}$}
	{$V=a^3$}
	\loigiai{
	\begin{center}
		\begin{tikzpicture}[declare function={a=2;b=4;h=h;gn=75;},line join=round]
			\path (0,0) coordinate (A)
			(b,0) coordinate (C)
			(-45:a) coordinate (B)
			($(A)+(45:4)$) coordinate (A')
			($(B)-(A)+(A')$) coordinate (B')
			($(C)-(A)+(A')$) coordinate (C');
			\path ($(C)!0.5!(B)$) coordinate (H);	
			\draw (A')--(A)--(B)--(C)--(C')--(A')--(B')--(C') (B)--(B');
			\draw[dashed] (A')--(H)--(A)--(C) ;
			\foreach \x/\y/\z in {A'/H/A,A'/H/C}{
				\path pic[draw,angle radius=5pt]{right angle= \x--\y--\z};
			}
			\foreach \t/\g in {A/180,B/-90,C/-90,A'/90,B'/90,C'/90,H/-45}{
				\draw[fill=white] (\t) circle (1pt) node[shift={(\g:7pt)},font=\scriptsize]{$ \t $};
			}
		\end{tikzpicture}
	\end{center}
		Gọi $M$ là trung điểm của $BC$.\\
		Theo bài ra $ABC$ là tam giác đều cạnh $a$ nên: $AM=\dfrac{a\sqrt{3}}{2}$; $S_{ABC}=\dfrac{a^2\sqrt{3}}{4}$.\\
		Hình chiếu vuông góc của điểm $A'$ lên mặt phẳng $(ABC)$ là trung điểm $M$ của cạnh $BC$ nên có: $A'M\perp(ABC)$; $A'M\perp BC$.\\
		Xét tam giác $A'MA$ vuông tại $M$: $A'M=\sqrt{AA'^2-AM^2}=\sqrt{\left(\dfrac{3a}{2}\right)^2-\left(\dfrac{a\sqrt{3}}{2}\right)^2}=\dfrac{a\sqrt{6}}{2}$.\\
		Thể tích của khối lăng trụ $ABC.A'B'C'$ là $V_{ABC.A'B'C'}=A'M\cdot S_{ABC}=\dfrac{a\sqrt{6}}{2}\cdot\dfrac{a^2\sqrt{3}}{4}=\dfrac{3a^3}{4\sqrt{2}}$.}
\end{ex}
\begin{ex}%Câu 25.
	(Ngô Quyền - Hải Phòng 2019) Cho hình lăng trụ $ABC.A'B'C'$ có đáy là tam giác vuông cân đỉnh $A, AB=a, AA'=2a,$ hình chiếu vuông góc của $A'$ lên mặt phẳng $(ABC)$ là trung điểm $H$ của cạnh $BC$. Thể tích của khối lăng trụ $ABC.A'B'C'$ bằng
	\choice
	{$\dfrac{a^3\sqrt{14}}{2}$}
	{\True $\dfrac{a^3\sqrt{14}}{4}$}
	{$\dfrac{a^3\sqrt{7}}{4}$}
	{$\dfrac{a^3\sqrt{3}}{2}$}
	\loigiai{
		\begin{center}
			\begin{tikzpicture}[declare function={a=2;b=4;h=h;gn=75;},line join=round]
				\path (0,0) coordinate (A)
				(b,0) coordinate (C)
				(-55:a) coordinate (B)
				($(A)+(45:4)$) coordinate (A')
				($(B)-(A)+(A')$) coordinate (B')
				($(C)-(A)+(A')$) coordinate (C');
				\path ($(C)!0.5!(B)$) coordinate (H);	
				\draw (A')--(A)--(B)--(C)--(C')--(A')--(B')--(C') (B)--(B');
				\draw[dashed] (A')--(H)--(A)--(C) ;
				%	\path (A)--(B)node[pos=0.5,sloped,black,left]{a};
				\path (A)--(B)node[pos=0.5,sloped,black,below,scale=0.6]{a}
				(A)--(A')node[pos=0.5,sloped,black,above,scale=0.6]{2a};
				
				\foreach \x/\y/\z in {B/A/C,A'/H/C}{
					\path pic[draw,angle radius=5pt]{right angle= \x--\y--\z};
				}
				\foreach \t/\g in {A/180,B/-90,C/-90,A'/90,B'/90,C'/90,H/-45}{
					\draw[fill=white] (\t) circle (1pt) node[shift={(\g:7pt)},font=\scriptsize]{$ \t $};
				}
			\end{tikzpicture}
		\end{center}
		Tam giác $ABC$ vuông cân tại $A\Rightarrow BC=a\sqrt{2};AH=\dfrac{1}{2}BC=\dfrac{a\sqrt{2}}{2}$.\\
		$A'H\perp(ABC)\Rightarrow A'H\perp AH$.\\
		Trong tam giác $AA'H$ vuông tại $H$ ta có: $A'H=\sqrt{AA'^2-AH^2}=\sqrt{4a^2-\dfrac{2a^2}{4}}=a\dfrac{\sqrt{14}}{2}$.\\
		Vậy $V_{ABC.A'B'C'}=A'H\cdot S_{ABC}=a\dfrac{\sqrt{14}}{2}\cdot\dfrac{1}{2}\cdot a\cdot a=\dfrac{a^3\sqrt{14}}{4}$.}
\end{ex}
\begin{ex}%Câu 26.
	(SGD Hưng Yên) Cho lăng trụ $ABC.A'B'C'$ có đáy $ABC$ là tam giác đều cạnh $a$, độ dài cạnh bên bằng $\dfrac{2a}{3}$, hình chiếu của đỉnh $A'$ trên mặt phẳng $(ABC)$ trùng với trọng tâm của tam giác $ABC$. Thể tích khối lăng trụ $ABC.A'B'C'$ bằng 
	\choice
	{$\dfrac{a^3\sqrt{3}}{36}$}
	{$\dfrac{a^3\sqrt{3}}{6}$}
	{\True $\dfrac{a^3\sqrt{3}}{12}$}
	{$\dfrac{a^3\sqrt{3}}{24}$}
	\loigiai{
		\begin{center}
			\begin{tikzpicture}[declare function={a=2;b=4;h=h;gn=75;},line join=round]
				\path (0,0) coordinate (A)
				(b,0) coordinate (C)
				(-45:a) coordinate (B)
				($(A)+(65:4)$) coordinate (A')
				($(B)-(A)+(A')$) coordinate (B')
				($(C)-(A)+(A')$) coordinate (C');
				\path ($(C)!0.5!(B)$) coordinate (I)
				($(A)!0.6!(I)$) coordinate (G);	
				\draw (A')--(A)--(B)--(C)--(C')--(A')--(B')--(C') (B)--(B');
				\draw[dashed] (A')--(G) (I)--(A)--(C) ;
				%	\path (A)--(B)node[pos=0.5,sloped,black,left]{a};
				\path (A)--(B)node[pos=0.4,sloped,scale=0.6,black]{$|$} (C)--(B)node[pos=0.4,sloped,scale=0.6,black]{$|$} (C)--(A)node[pos=0.4,sloped,scale=0.6,black]{$|$} ;
				
				\path (A)--(B)node[pos=0.5,sloped,black,below,scale=0.6]{$ a $}
				(A)--(A')node[pos=0.5,sloped,black,above,scale=0.6]{$ 2a\sqrt{3} $};				
				\foreach \t/\g in {A/180,B/-90,C/-90,A'/90,B'/90,C'/90,I/-45,G/-110}{
					\draw[fill=white] (\t) circle (1pt) node[shift={(\g:7pt)},font=\scriptsize]{$ \t $};
				}
			\end{tikzpicture}
		\end{center}
		Gọi $G$ là trọng tâm của tam giác $ABC$. Ta có:\\
		$AG=\dfrac{2}{3} AI=\dfrac{a\sqrt{3}}{3}$; $A'G^2=A'A^2-AG^2=\left(\dfrac{2a}{3}\right)^2-\left(\dfrac{a\sqrt{3}}{3}\right)^2=\dfrac{a^2}{9}\Rightarrow A'G=\dfrac{a}{3}$.\\
		$V=B\cdot h=\dfrac{a^2\sqrt{3}}{4}\cdot\dfrac{a}{3}=\dfrac{a^3\sqrt{3}}{12}$.}
\end{ex}
\begin{ex}%Câu 27.
	(SGD Bắc Ninh 2019) Cho hình lăng trụ $ABC.A'B'C'$ có đáy $ABC$ là tam giác đều cạnh $a$, $AA'=\dfrac{3a}{2}$. Biết rằng hình chiếu vuông góc của $A'$ lên $(ABC)$ là trung điểm $BC$. Thể tích của khối lăng trụ $ABC.A'B'C'$ là
	\choice
	{$\dfrac{a^3\cdot\sqrt{2}}{8}$}
	{\True $\dfrac{3a^3\cdot\sqrt{2}}{8}$}
	{$\dfrac{a^3\cdot\sqrt{6}}{2}$}
	{$\dfrac{2a^3}{3}$}
	\loigiai{
			\begin{center}
			\begin{tikzpicture}[declare function={a=2;b=4;h=h;gn=75;},line join=round]
				\path (0,0) coordinate (A)
				(b,0) coordinate (C)
				(-45:a) coordinate (B)
				($(A)+(45:4)$) coordinate (A')
				($(B)-(A)+(A')$) coordinate (B')
				($(C)-(A)+(A')$) coordinate (C');
				\path ($(C)!0.5!(B)$) coordinate (H);	
				\draw (A')--(A)--(B)--(C)--(C')--(A')--(B')--(C') (B)--(B');
				\draw[dashed] (A')--(H)--(A)--(C) ;
				\foreach \x/\y/\z in {A'/H/A,A'/H/C}{
					\path pic[draw,angle radius=5pt]{right angle= \x--\y--\z};
				}
				\foreach \t/\g in {A/180,B/-90,C/-90,A'/90,B'/90,C'/90,H/-45}{
					\draw[fill=white] (\t) circle (1pt) node[shift={(\g:7pt)},font=\scriptsize]{$ \t $};
				}
			\end{tikzpicture}
		\end{center}
		Gọi $H$ là trung điểm $BC$, vì tam giác $ABC$ đều nên ta có $AH=\dfrac{a\sqrt{3}}{2}\Rightarrow S_{\triangle ABC}=\dfrac{a^2\cdot\sqrt{3}}{4}$.\\
		Theo đề: $A'H\perp(ABC)\Rightarrow A'H\perp AH$. Trong tam giác vuông $A'AH$ có.\\
		$A'H=\sqrt{A'A^2-AH^2}=\sqrt{\dfrac{9a^2}{4}-\dfrac{3a^2}{4}}=\dfrac{a\sqrt{3}}{\sqrt{2}}$.\\
		Suy ra $V_{ABC.A'B'C'}=B\cdot h=\dfrac{a^2\sqrt{3}}{4}\cdot\dfrac{a\sqrt{3}}{\sqrt{2}}=\dfrac{3a^3\cdot\sqrt{2}}{8}$.}
\end{ex}
\begin{ex}%Câu 28.
	(THPT Cẩm Bình Hà Tỉnh 2019) Cho hình lăng trụ $ABC.A'B'C'$ có đáy là tam giác đều cạnh bằng $a$, hình chiếu vuông góc của $A'$ lên mặt phẳng $(ABC)$ trùng với trọng tâm $G$ của tam giác $ABC$. Biết khoảng cách giữa $BC$ và $AA'$ bằng $\dfrac{a\sqrt{3}}{4}$. Thể tích khối chóp $B'\cdot ABC$ bằng 
	\choice
	{\True $\dfrac{a^3\sqrt{3}}{36}$}
	{$\dfrac{a^3\sqrt{3}}{9}$}
	{$\dfrac{a^3\sqrt{3}}{18}$}
	{$\dfrac{a^3\sqrt{3}}{12}$}
	\loigiai{
		\begin{center}
			\begin{tikzpicture}[declare function={a=2;b=4;h=3;gn=75;},line join=round]
				\path (0,0) coordinate (A)
				(b,0) coordinate (C)
				(-45:a) coordinate (B)
				($(A)+(55:3)$) coordinate (A')
				($(B)-(A)+(A')$) coordinate (B')
				($(C)-(A)+(A')$) coordinate (C');
				\path ($(C)!0.5!(B)$) coordinate (M)
				($(A)!0.5!(A')$) coordinate (H)
				($(A)!0.6!(M)$) coordinate (G);
				\draw (A')--(A)--(B)--(C)--(C')--(A')--(B')--(C') (B)--(B');
				\draw[dashed] (H)--(M)--(A)--(C) (A')--(G);
				\foreach \t/\g in {A/180,B/-90,C/-90,A'/90,B'/90,C'/90,H/180,M/0,G/-90}{
					\draw[fill=white] (\t) circle (1pt) node[shift={(\g:7pt)},font=\scriptsize]{$ \t $};
				}
			\end{tikzpicture}
		\end{center}
		Gọi $M$ là trung điểm của $BC$, $MH\perp AA'$ tại $H$.\\
		Ta có $BC\perp(AA'M)\Rightarrow BC\perp HM$. Do đó $HM=\mathrm{d}(AA',BC)$.\\
		$\begin{aligned}&AM=\dfrac{a\sqrt{3}}{2}, AG=\dfrac{a\sqrt{3}}{3}\Rightarrow\sin\widehat{HAM}=\dfrac{HM}{AM}=\dfrac{1}{2}\Rightarrow\widehat{HAM}={30}^{\circ}\cdot\\&A'G=AG\cdot\tan{30}^{\circ}=\dfrac{a}{3}, S_{ABC}=\dfrac{1}{2}AM\cdot BC=\dfrac{a^2\sqrt{3}}{4}.\end{aligned}$\\
		$V_{B'\cdot ABC}=\dfrac{1}{3}A'G\cdot S_{ABC}=\dfrac{a^3\sqrt{3}}{36}$.}
\end{ex}
\begin{ex}%Câu 29.
	(TT Diệu Hiền - Cần Thơ - 2018) Cho lăng trụ $ABCD.A'B'C'D'$ có đáy $ACBD$ là hình thoi cạnh $a$, biết $A'\cdot ABC$ là hình chóp đều và $A'D$ hợp với mặt đáy một góc $45^{\circ}$. Thể tích khối lăng trụ $ABCD.A'B'C'D'$ là 
	\choice
	{\True $a^3$}
	{$\dfrac{a^3\sqrt{6}}{12}$}
	{$a^3\sqrt{3}$}
	{$\dfrac{a^3\sqrt{6}}{3}$}
	\loigiai{
	\begin{center}
		\begin{tikzpicture}[declare function={a=2;b=4;h=3;},line join=round]
			\path (0,0) coordinate (B)
			(35:a) coordinate (A)
			(b,0) coordinate (C)
			($(C)-(B)+(A)$) coordinate (D)
			($(B)!0.4!(D)$)	coordinate (G)
			($(G)+(90:h)$) coordinate (A')
			($(B)-(A)+(A')$)  coordinate (B')
			($(C)-(A)+(A')$) coordinate (C')
			($(D)-(A)+(A')$) coordinate (D');
			\path (intersection of A--C and B--D) coordinate (O);
			\foreach \x/\y/\z in {B/G/A',A/O/B}{
				\path pic[draw,angle radius=5pt]{right angle= \x--\y--\z};
			}
			\draw (B')--(B)--(C)--(D)--(D')--(A')--(B')--(C')--(D') (C)--(C');
			\draw[dashed] (C)--(A)--(D)--(B) (G)--(A')--(A)--(B) (C)--(A')--(D);
			\foreach \t/\g in {A/180,G/-45,B/-90,C/-90,A'/90,B'/90,C'/90,D'/0,D/0,O/-90}{
				\draw[fill=white] (\t) circle (1pt) node[shift={(\g:7pt)},font=\scriptsize]{$ \t $};
			}
		\end{tikzpicture}
	\end{center}
		Ta có $\widehat{\left(A'D,(ABCD)\right)}=\widehat{A'DG}=45^{\circ}$.\\
		Ta giác $ABC$ đều cạnh $a$ nên $BG=\dfrac{a\sqrt{3}}{3}$, $DB=a\sqrt{3}$, $DG=2BG=\dfrac{2a\sqrt{3}}{3}$.\\
		Tam giác $A'DG$ vuông cân tại $G$ nên $A'G=DG=\dfrac{2a\sqrt{3}}{3}$.\\
		$V_{ABCD.A'B'C'D'}=S_{ABCD}\cdot AG=\dfrac{1}{2}a\cdot a\sqrt{3}\cdot\dfrac{2a\sqrt{3}}{3}=a^3$.}
\end{ex}
\begin{ex}%Câu 30.
	(Chuyên Long An - 2018) Cho hình lăng trụ $ABC.A'B'C'$ có đáy là tam giác đều cạnh $m\in[-5;2)$. Hình chiếu vuông góc của điểm $A'$ lên mặt phẳng $(ABC)$ trùng với trọng tâm tam giác $ABC$. Biết khoảng cách giữa hai đường $AA'$ và $BC$ bằng $\dfrac{a\sqrt{3}}{4}$. Tính thể tích $V$ của khối lăng trụ $ABC.A'B'C'$. 
	\choice
	{$V=\dfrac{a^3\sqrt{3}}{6}$}
	{$V=\dfrac{a^3\sqrt{3}}{24}$}
	{\True $V=\dfrac{a^3\sqrt{3}}{12}$}
	{$V=\dfrac{a^3\sqrt{3}}{3}$}
	\loigiai{
	\begin{center}
		\begin{tikzpicture}[declare function={a=2;b=4;h=3;gn=75;},line join=round]
			\path (0,0) coordinate (A)
			(b,0) coordinate (C)
			(-45:a) coordinate (B)
			($(A)+(45:3)$) coordinate (A')
			($(B)-(A)+(A')$) coordinate (B')
			($(C)-(A)+(A')$) coordinate (C');
			\path ($(C)!0.5!(B)$) coordinate (E)
			($(A)!0.7!(E)$) coordinate (G)
			($(A)!0.5!(A')$) coordinate (F)
			($(A)!0.5!(B)$) coordinate (H);
			\path (F)--(E)node[pos=0.2,sloped,black,below,scale=0.4]{$ \tfrac{a\sqrt{3}}{4} $};
			\path (A)--(B)node[pos=0.5,sloped,black,below,scale=0.4]{$a$};
			\path (A)--(B)node[pos=0.4,sloped,scale=0.5,black]{$|$} (C)--(B)node[pos=0.4,sloped,scale=0.5,black]{$|$} (A)--(C)node[pos=0.4,sloped,scale=0.5,black]{$|$} ;
			
			\draw (A')--(A)--(B)--(C)--(C')--(A')--(B')--(C') (B)--(B');
			\draw[dashed] (F)--(E)--(A)--(C) (A')--(G);
			\foreach \t/\g in {A/180,B/-90,C/-90,A'/90,B'/90,C'/90,E/-45,G/-90,F/180}{
				\draw[fill=white] (\t) circle (1pt) node[shift={(\g:7pt)},font=\scriptsize]{$ \t $};
			}
		\end{tikzpicture}
	\end{center}
		Gọi $G$ là trọng tâm tam giác $ABC$. Vì $A'G\perp(ABC)$ và tam giác $ABC$ đều nên $A'ABC$ là hình chóp đều. Kẻ $EF\perp AA'$ và $BC\perp(AA'E)$ nên $\mathrm{d}(AA',BC)=EF=\dfrac{a\sqrt{3}}{4}$. Đặt $A'G=h$.\\
		Ta có $A'A=\sqrt{h^2+\left(\dfrac{a\sqrt{3}}{3}\right)^2}$.\\
		Tam giác $A'AG$ đồng dạng với tam giác $EAF$ nên.\\
		$\dfrac{A'A}{EA}=\dfrac{AG}{FA}=\dfrac{A'G}{FE}\Rightarrow A'G\cdot EA=A'A\cdot FE\Leftrightarrow h\cdot\dfrac{a\sqrt{3}}{2}=\sqrt{h^2+\left(\dfrac{a\sqrt{3}}{3}\right)^2}\cdot\dfrac{a\sqrt{3}}{4}\Leftrightarrow h=\dfrac{a}{3}$.\\
		Thể tích $V$ của khối lăng trụ $ABC.A'B'C'$ là $V=AG\cdot S_{ABC}=\dfrac{a}{3}\cdot\dfrac{a^2\sqrt{3}}{4}=\dfrac{a^3\sqrt{3}}{12}$.}
\end{ex}
\begin{ex}%Câu 31.
	(Lê Quý Đôn - Quảng Trị - 2018) Cho hình lăng trụ $C$ có đáy là tam giác đều cạnh $H$. Hình chiếu vuông góc của điểm $D$ lên mặt phẳng $M$ trùng với trọng tâm tam giác $ABC$. Biết khoảng cách giữa hai đường thẳng $AA'$ và $BC$ bằng $\dfrac{a\sqrt{3}}{4}$. Tính thể tích $V$ của khối lăng trụ $ABC.A'B'C'$. 
	\choice
	{$V=\dfrac{a^3\sqrt{3}}{6}$}
	{$V=\dfrac{a^3\sqrt{3}}{3}$}
	{$V=\dfrac{a^3\sqrt{3}}{24}$}
	{\True $V=\dfrac{a^3\sqrt{3}}{12}$}
	\loigiai{
		\begin{center}
			\begin{tikzpicture}[declare function={a=2;b=4;h=3;gn=75;},line join=round]
				\path (0,0) coordinate (A)
				(b,0) coordinate (C)
				(-45:a) coordinate (B)
				($(A)+(60:4)$) coordinate (A')
				($(B)-(A)+(A')$) coordinate (B')
				($(C)-(A)+(A')$) coordinate (C');
				\path ($(C)!0.5!(B)$) coordinate (M)
				($(A)!0.5!(C)$) coordinate (N)
				($(A)!0.5!(A')$) coordinate (H);
				\path (intersection of A--M and B--N) coordinate (G);
				
				\draw (A')--(A)--(B)--(C)--(C')--(A')--(B')--(C') (B)--(B');
				\draw[dashed] (A')--(G) (H)--(M)--(A)--(C) (B)--(N);
				\foreach \t/\g in {A/180,B/-90,C/-90,A'/90,B'/90,C'/90,M/0,G/-160,H/135}{
					\draw[fill=white] (\t) circle (1pt) node[shift={(\g:7pt)},font=\scriptsize]{$ \t $};
				}
			\end{tikzpicture}
		\end{center}
		Gọi $M$ là trung điểm của $BC$. Vẽ $MH\perp AA'(H\in BC)$.\\
		Ta có $AM\perp BC$, $A'G\perp BC\Rightarrow BC\perp(A'AG)\Rightarrow BC\perp MH\Rightarrow\mathrm{d}(AA',BC)=MH$.\\
		$AH=\sqrt{AM^2-MH^2} =\sqrt{\dfrac{3a^2}{4}-\dfrac{3a^2}{16}} =\dfrac{3a}{4}$.\\
		Ta có $\dfrac{MH}{AH}=\dfrac{A'G}{AG}=\tan\widehat{GAH}\Rightarrow A'G=\dfrac{MH\cdot AG}{AH} =\dfrac{\dfrac{a\sqrt{3}}{4}\cdot\dfrac{a\sqrt{3}}{3}}{\dfrac{3a}{4}} =\dfrac{a}{3}$.\\
		Vậy $V=S_{ABC}\cdot A'G =\dfrac{a^2\sqrt{3}}{4}\cdot\dfrac{a}{3} =\dfrac{a^3\sqrt{3}}{12}$.}
\end{ex}
\begin{ex}%Câu 32.
	(THPT Hà Huy Tập - Hà Tĩnh - 2018) Cho lăng trụ $ABCD.A'B'C'D'$ có đáy $ABCD$ là hình thoi cạnh $a$, tâm $O$ và $\widehat{ABC}=120^{\circ}$. Góc giữa cạnh bên $AA'$ và mặt đáy bằng $60^{\circ}$. Đỉnh $A'$ cách đều các điểm $A$, $B$, $D$. Tính theo $a$ thể tích $V$ của khối lăng trụ đã cho. 
	\choice
	{$V=\dfrac{3a^3}{2}$}
	{$V=\dfrac{a^3\sqrt{3}}{6}$}
	{\True $V=\dfrac{a^3\sqrt{3}}{2}$}
	{$V=a^3\sqrt{3}$}
	\loigiai{
		\begin{center}
			\begin{tikzpicture}[declare function={a=2;b=4;h=3;},line join=round]
				\path (0,0) coordinate (B)
				(35:a) coordinate (A)
				(b,0) coordinate (C)
				($(C)-(B)+(A)$) coordinate (D)
				($(A)!0.3!(C)$)	coordinate (H)
				($(H)+(90:h)$) coordinate (A')
				($(B)-(A)+(A')$)  coordinate (B')
				($(C)-(A)+(A')$) coordinate (C')
				($(D)-(A)+(A')$) coordinate (D');
				\path (intersection of A--C and B--D) coordinate (O);
				\path pic[draw,angle radius=5pt]{right angle= A'--H--A};
				\path pic["\scriptsize$60^\circ$", angle eccentricity=2,draw,angle radius=7pt]{angle= H--A--A'};
				\draw (B')--(B)--(C)--(D)--(D')--(A')--(B')--(C')--(D') (C)--(C');
				\draw[dashed] (C)--(A)--(D)--(B) (H)--(A')--(A)--(B) (B)--(A')--(D);
				\foreach \x/\y in {A'/B,A'/A,A'/D}{
					\path (\x)--(\y) node[midway,sloped]{\tikz{\draw [shift={(-0.65pt,0)}](-90:1pt)--(90:1pt) [shift={(0.65pt,0)}](-90:1pt)--(90:1pt);}};
				}
				\foreach \t/\g in {A/180,H/-145,B/-90,C/-90,A'/90,B'/90,C'/90,D'/0,D/0,O/-90}{
					\draw[fill=white] (\t) circle (1pt) node[shift={(\g:7pt)},font=\scriptsize]{$ \t $};
				}
			\end{tikzpicture}
		\end{center}
		Ta có tam giác $ABD$ cân tại $A$ và $\widehat{BAD}=60^{\circ}$ nên $ABD$ là tam giác đều.\\
		Gọi $H$ là trọng tâm tam giác $ABD$. Vì $A'$ cách đều $A$, $B$, $D$ nên $A'H$ là trục đường tròn ngoại tiếp tam giác $ABD$. Do đó $A'H\perp(ABD)$.\\
		Suy ra góc giữa $A'A$ và đáy $(ABCD)$ là góc $\widehat{A'AH}=60^{\circ}$.\\
		Ta có $AH=\dfrac{2}{3}AO=\dfrac{a\sqrt{3}}{2}$. Do đó $A'H=AH\cdot\tan 60^{\circ}=\dfrac{3a}{2}$.\\
		Ngoài ra $S_{ABCD}=2S_{ABD}=2\cdot\dfrac{a^2\sqrt{3}}{4}=\dfrac{a^2\sqrt{3}}{2}$.\\
		Thể tích khối lăng trụ $ABCD.A'B'C'D'$ là $V=S_{ABCD}\cdot A'H=\dfrac{a^2\sqrt{3}}{2}\cdot\dfrac{3a}{2}=\dfrac{3a^3\sqrt{3}}{8}$.}
\end{ex}
\begin{ex}%Câu 33.
	(THPT Trần Quốc Tuấn - 2018) Cho hình lăng trụ $ABC.A'B'C'$ có đáy $S.ABCD$ là tam giác vuông tại $A$, $AB=a$, $AC=a\sqrt{3}$. Hình chiếu vuông góc của đỉnh $A'$ lên $(ABC)$ trùng với tâm của đường tròn ngoại tiếp của tam giác $ABC$. Trên cạnh $AC$ lấy điểm $M$ sao cho $CM=2MA$. Biết khoảng cách giữa hai đường thẳng $A'M$ và $BC$ bằng $\dfrac{a}{2}$. Tính thể tích $V$ của khối lăng trụ đã cho. 
	\choice
	{\True $V=\dfrac{a^3\sqrt{3}}{2}$}
	{$V=a^3$}
	{$V=\dfrac{3a^3}{2}$}
	{$V=\dfrac{2a^3\sqrt{3}}{3}$}
	\loigiai{
		\begin{center}
				\begin{tikzpicture}[declare function={a=2;b=4;h=3;gn=75;},line join=round]
				\path (0,0) coordinate (A)
				(b,0) coordinate (C)
				(-45:a) coordinate (B)
				($(A)+(45:4)$) coordinate (A')
				($(B)-(A)+(A')$) coordinate (B')
				($(C)-(A)+(A')$) coordinate (C');
				\path ($(C)!0.5!(B)$) coordinate (H)
				($(C)!2/3!(A)$) coordinate (M)
				($(B)!2/3!(A)$) coordinate (N)
				($(M)!0.5!(N)$) coordinate (K)
				($(A')!0.7!(K)$) coordinate (I);		
				\draw (A')--(A)--(B)--(C)--(C')--(A')--(B')--(C') (B)--(B');
				\draw[dashed] (A)--(C) (N)--(M) (H)--(K)--(A')--(H)--(I);
				\foreach \x/\y/\z in {H/K/M,H/I/A'}{
					\path pic[draw,angle radius=5pt]{right angle= \x--\y--\z};
				}
				\foreach \t/\g in {A/180,B/-90,C/-90,A'/90,B'/90,C'/90,H/-30,M/45,N/-135,K/180,I/180}{
					\draw[fill=white] (\t) circle (1pt) node[shift={(\g:7pt)},font=\scriptsize]{$ \t $};
				}
			\end{tikzpicture}
		\end{center}
		Kẻ $MN\parallel BC$, $N\in AB$. $HK\perp MN$, $HI\perp A'K$.\\
		$\mathrm{d}(A'M;BC)=\mathrm{d}\left(BC;(A'MN)\right)=\mathrm{d}\left(H;(A'MN)\right)=HI\Rightarrow HI=\dfrac{a}{2}$.\\
		Kẻ $AT\parallel HK$, $AT\cap MN=P\Rightarrow HK=PT=\dfrac{2}{3}AT$.\\
		Tam giác $ABC$ vuông tại $A\Rightarrow\dfrac{1}{AT^2}=\dfrac{1}{AB^2}+\dfrac{1}{AC^2}=\dfrac{4}{3a^2}\Rightarrow HK=\dfrac{2}{3}AT=\dfrac{a}{\sqrt{3}}$.\\
		Tam giác $A'HK$ vuông tại $H\Rightarrow\dfrac{1}{A'H^2}=\dfrac{1}{HI^2}-\dfrac{1}{HK^2}=\dfrac{4}{a^2}-\dfrac{3}{a^2}=\dfrac{1}{a^2}\Rightarrow A'H=a$.\\
		Vậy thể tích khối lăng trụ đã cho là $V=A'H\cdot S_{ABC}=a\cdot\dfrac{1}{2}\cdot a\cdot a\sqrt{3}=\dfrac{a^3\sqrt{3}}{2}$.}
\end{ex}				
\Closesolutionfile{ans}
%\indapan{10}{ans/CD_11/Muc_7_8}