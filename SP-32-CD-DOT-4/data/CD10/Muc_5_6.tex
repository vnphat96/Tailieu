\setcounter{dang}{0}
\setcounter{ex}{0}
\section{Mức độ 5,6 điểm}
\Opensolutionfile{ans}[ans/CD10/Muc_5_6]
\begin{dang}
	{Cạnh bên vuông góc với đáy}
\end{dang}
\begin{ex}%[2H1Y3-2]
	(Mã 101-2022) Cho khối chóp $ S.ABC$ có chiều cao bằng $ 3$, đáy $ ABC$ có diện tích bằng $ 10$. Thể tích khối chóp $ S.ABC$ bằng
	\choice
	{$ 2$}
	{$ 15$}
	{\True $ 10$}
	{$ 30$}
	\loigiai{
		Thể tích khối chóp $ S.ABC$ là $ V=\dfrac{1}{3}\cdot S\cdot h=\dfrac{1}{3}\cdot10\cdot3=10$.}
\end{ex}

%[Câu 2]
\begin{ex}%[2H1Y3-2]
	(Mã 103-2022) Cho khối chóp $ S.ABC$ có chiều cao bằng 5, đáy $ ABC$ có diện tích bằng $ 6$. Thể tích khối chóp $ S.ABC$ bằng 
	\choice
	{$ 11$}
	{\True $ 10$}
	{$ 15$}
	{$ 30$}
	\loigiai{
		$V_{S.ABC}=\dfrac{1}{3}\cdot S\cdot h=\dfrac{1}{3}\cdot6\cdot5=10$.}
\end{ex}

%[Câu 3]
\begin{ex}%[2H1Y3-2]
	(Đề minh họa 2022) Cho khối chóp có diện tích đáy $ S=7$ và chiều cao $h=6$. Thể tích khối chóp đã cho bằng
	\choice
	{$ 42$}
	{$ 126$}
	{\True $ 14$}
	{$ 56$}
	\loigiai{
		Thể tích khối chóp là $V=\dfrac{1}{3}\cdot S\cdot h=\dfrac{1}{3}\cdot7\cdot6=14.$}
\end{ex}

%[Câu 4]
\begin{ex}%[2H1Y3-2]
	(Đề Minh Họa 2021) Một khối chóp có diện tích đáy bằng $6$ và chiều cao bằng $5$. Thể tích của khối chóp đã cho bằng
	\choice
	{\True $ 10$}
	{$ 30$}
	{$90$}
	{$15$}
	\loigiai{
		$$V=\dfrac{1}{3}\cdot S\cdot h=\dfrac{1}{3}\cdot6\cdot5=10.$$}
\end{ex}

%[Câu 5]
\begin{ex}%[2H1Y3-2]
	(Mã 104-2021 Lần 1) Cho khối chóp có diện tích đáy $ S=8a^2$ và chiều cao $h=a$. Thể tích của khối chóp đã cho bằng
	\choice
	{$ 8 a^3$}
	{$\dfrac{4}{3}a^3$}
	{$ 4 a^3$}
	{\True $\dfrac{8}{3}a^3$}
	\loigiai{
		Thể tích của khối chóp có diện tích đáy $S=8a^2$ và chiều cao $ h=a$ là $$V=\dfrac{1}{3}\cdot S\cdot h=\dfrac{1}{3}\cdot8a^2\cdot a=\dfrac{8}{3}{a^3}.$$}
\end{ex}

%[Câu 6]
\begin{ex}%[2H1Y3-2]
	(Mã 101-2021 Lần 1) Cho khối chóp có diện tích đáy $ S=5a^2$ và chiều cao $ h=a$. Thể tích của khối chóp đã cho bằng
	\choice
	{$\dfrac{5}{6}{a^3}$}
	{$\dfrac{5}{2}{a^3}$}
	{$ 5a^3$}
	{\True $\dfrac{5}{3}{a^3}$}
	\loigiai{
		Thể tích của khối chóp đã cho bằng $ V=\dfrac{1}{3}\cdot S\cdot h=\dfrac{1}{3}\cdot5a^2\cdot a=\dfrac{5}{3}{a^3}$.}
\end{ex}

%[Câu 7]
\begin{ex}%[2H1Y3-2]
	(Mã 103-2021-Lần 1) Cho khối chóp có diện tích đáy $ S=7a^2$ và chiều cao $ h=a$. Thể tích của khối chóp đã cho bằng
	\choice
	{$\dfrac{7}{6}{a^3}$}
	{$\dfrac{7}{2}{a^3}$}
	{\True $\dfrac{7}{3}{a^3}$}
	{$ 7a^3$}
	\loigiai{
		Thể tích khối chóp $ V=\dfrac{1}{3}\cdot S\cdot h=\dfrac{1}{3}\cdot7a^2\cdot a=\dfrac{7}{3}{a^3}$.}
\end{ex}

%[Câu 8]
\begin{ex}%[2H1Y3-2]
	(Mã 102-2021 Lần 1) Cho khối chóp có diện tích đáy $ S=3a^2$ và chiều cao $ h=a$. Thể tích của khối chóp đã cho bằng
	\choice
	{$\dfrac{3}{2}{a^3}$}
	{$ 3a^3$}
	{$\dfrac{1}{3}{a^3}$}
	{\True $a^3$}
	\loigiai{
		Thể tích của khối chóp đã cho bằng
		$$V=\dfrac{1}{3}\cdot S \cdot h=\dfrac{1}{3}\cdot3a^2\cdot a=a^3.$$}
\end{ex}

%[Câu 9]
\begin{ex}%[2H1Y3-2]
	(Đề Tham Khảo 2020 Lần 2) Cho khối chóp có diện tích đáy $ S=3$ và chiều cao $ h=4$. Thể tích của khối chóp đã cho bằng
	\choice
	{$ 6$}
	{$ 12$}
	{$ 36$}
	{\True $4$}
	\loigiai
	{
		Ta có công thức thể tích khối chóp
		$$ V=\dfrac{1}{3}\cdot S\cdot h=\dfrac{1}{3}\cdot3\cdot4=4.$$}
\end{ex}

%[Câu 10]
\begin{ex}%[2H1Y3-2]
	(Mã 101-2020 Lần 1) Cho khối chóp có diện tích đáy $ S=6$ và chiều cao $ h=2$. Thể tích của khối chóp đã cho bằng
	\choice
	{$ 6$}
	{$ 3$}
	{\True $ 4$}
	{$ 12$}
	\loigiai
	{
		Thể tích của khối chóp $$ V=\dfrac{1}{3}\cdot S \cdot h=\dfrac{1}{3}\cdot 6 \cdot 2=4.$$}
\end{ex}

%[Câu 11]
\begin{ex}%[2H1Y3-2]
	(Mã 102-2020 Lần 1) Cho khối chóp có diện tích đáy $S=3$ và chiều cao $h=2$. Thể tích khối chóp đã cho bằng
	\choice
	{6}
	{12}
	{\True 2}
	{3}
	\loigiai
	{
		Thể tích khối chóp đã cho là
		$$V=\dfrac{1}{3}\cdot B \cdot h=\dfrac{1}{3} \cdot 3 \cdot 2=2.$$
		
	}
\end{ex}

%[Câu 12]
\begin{ex}%[2H1Y3-2]
	(Mã 102-2020 Lần 2) Cho khối chóp có diện tích đáy $ S=6a^2$ và chiều cao $ h=2a$. Thể tích khối chóp đã cho bằng
	\choice
	{$ 2a^3$}
	{\True $ 4a^3$}
	{$ 6a^3$}
	{$ 12a^3$}
	\loigiai
	{
		$$V=\dfrac{1}{3}\cdot S\cdot h=\dfrac{1}{3}\cdot 6a^2\cdot 2a=4a^3.$$
	}
\end{ex}

%[Câu 13]
\begin{ex}%[2H1B3-2]
	(Đề Minh Họa 2017) Cho hình chóp tứ giác $S.ABCD$ có đáy $ABCD$ là hình vuông cạnh $a$, cạnh bên $SA$ vuông góc với mặt phẳng đáy và $ SA=a\sqrt{2}$. Tính thể tích $ V$ của khối chóp $S.ABCD$ 
	\choice
	{$ V=\dfrac{\sqrt{2}{a^3}}{6}$}
	{$ V=\dfrac{\sqrt{2}{a^3}}{4}$}
	{$ V=\sqrt{2}{a^3}$}
	{\True $V=\dfrac{\sqrt{2}{a^3}}{3}$}
	\loigiai
	{
		\begin{center}
			\begin{tikzpicture}
				\path 
				(0:0) coordinate (A)
				+(0:5) coordinate (B)
				+(-150:2.5) coordinate (D)
				+(90:4) coordinate (S)
				($(B)+(D)-(A)$) coordinate (C)
				;
				\draw[dashed] 
				(A)--(B) (A)--(D) (A)--(S)
				;
				\draw 
				(D)--(C)--(B)
				(S)--(B) (S)--(C) (S)--(D)
				;
				\foreach \x/\g in {A/135,B/0,C/-45,D/-135,S/90}
				\fill (\x) circle (1.5pt)
				+(\g:3.5mm) node {$\x$};
			\end{tikzpicture}
		\end{center}
		Ta có $ SA\perp\left(ABCD\right)\Rightarrow SA$ là đường cao của hình chóp.\\
		Thể tích khối chóp $S.ABCD$ $$ V=\dfrac{1}{3}\cdot SA\cdot S_{ABCD}=\dfrac{1}{3}\cdot a\sqrt{2}\cdot a^2=\dfrac{a^3\sqrt{2}}{3}.$$}
\end{ex}

%[Câu 14]
\begin{ex}%[2H1B3-2]
	(Mã 105 2017) Cho khối chóp $ S.ABC$ có $ SA$ vuông góc với đáy, $ SA=4$, $ AB=6$, $ BC=10$ và $ CA=8$. Tính thể tích $ V$ của khối chóp $ S.ABC$
	\choice
	{\True $ V=32$}
	{$ V=192$}
	{$ V=40$}
	{$ V=24$}
	\loigiai
	{
		\begin{center}
			\begin{tikzpicture}
				%		\draw[gray] (0,0) grid (10,10);
				\path 
				(0:0) coordinate (A)
				+(10:5) coordinate (C)
				+(-30:3) coordinate (B)
				+(90:4) coordinate (S)
				;
				\draw[dashed] 
				(A)--(C)
				;
				\draw 
				(S)--(B)
				(S)--(A)--(B)--(C)--cycle
				;
				\foreach \x/\g in {A/135,B/0,C/-45,S/90}
				\fill (\x) circle (1.5pt)
				+(\g:3.5mm) node {$\x$};
			\end{tikzpicture}
		\end{center}
		Ta có $ BC^2=AB^2+AC^2$ suy ra $\Delta ABC$ vuông tại $A$.\\
		Diện tích tam giác $ABC$ là $S_{ABC}=24.$\\
		Thể tích $V$ của khối chóp $S.ABC$ là
		$$ V=\dfrac{1}{3}\cdot {S_{ABC}}\cdot SA=32.$$}
\end{ex}
%
%[Câu 15]
\begin{ex}%[2H1B3-2]
	(THPT Nguyễn Khuyến 2019) Cho hình chóp tứ giác $S.ABCD$ có đáy $ABCD$ là hình vuông cạnh $a$, cạnh bên $SA$ vuông góc với mặt phẳng đáy và $SA=\sqrt{2}a$. Tính thể tích khối chóp $S.ABCD$
	\choice
	{$\dfrac{\sqrt{2}{a^3}}{6}$}
	{$\dfrac{\sqrt{2}{a^3}}{4}$}
	{$\sqrt{2}{a^3}$}
	{\True $\dfrac{\sqrt{2}{a^3}}{3}$}
	\loigiai
	{
		\begin{center}
			\begin{tikzpicture}
				\path 
				(0:0) coordinate (A)
				+(0:5) coordinate (B)
				+(-150:2.5) coordinate (D)
				+(90:4) coordinate (S)
				($(B)+(D)-(A)$) coordinate (C)
				;
				\draw[dashed] 
				(A)--(B) (A)--(D) (A)--(S)
				;
				\draw 
				(D)--(C)--(B)
				(S)--(B) (S)--(C) (S)--(D)
				;
				\foreach \x/\g in {A/135,B/0,C/-45,D/-135,S/90}
				\fill (\x) circle (1.5pt)
				+(\g:3.5mm) node {$\x$};
			\end{tikzpicture}
		\end{center}
		Ta có $S_{ABCD}=a^2$.\\
		Thể tích khối chóp $S.ABCD$ là $$V_{S.ABCD}=\dfrac{1}{3}\cdot SA\cdot S_{ABCD}=\dfrac{\sqrt{2}{a^3}}{3}.$$}
\end{ex}
%
%[Câu 16]
\begin{ex}%[2H1B3-2]
	(THPT Đoàn Thượng-Hải Dương 2019) Cho hình chóp $ S.ABC$ có đáy là tam giác đều cạnh $ a$, cạnh bên $ SA$ vuông góc với đáy và thể tích của khối chóp đó bằng $\dfrac{a^3}{4}$. Tính cạnh bên $SA$
	\choice 
	{$\dfrac{a\sqrt{3}}{2}$}
	{$\dfrac{a\sqrt{3}}{3}$}
	{\True $a\sqrt{3}$}
	{$2a\sqrt{3}$}
	\loigiai{
		\begin{center}
			\begin{tikzpicture}
				%		\draw[gray] (0,0) grid (10,10);
				\path 
				(0:0) coordinate (A)
				+(0:5) coordinate (C)
				+(-30:3) coordinate (B)
				+(90:4) coordinate (S)
				;
				\draw[dashed] 
				(A)--(C)
				;
				\draw 
				(S)--(B)
				(S)--(A)--(B)--(C)--cycle
				;
				\foreach \x/\g in {A/135,B/-90,C/0,S/90}
				\fill (\x) circle (1.5pt)
				+(\g:3.5mm) node {$\x$};
			\end{tikzpicture}
		\end{center}
		$$V_{S.ABC}=\dfrac{1}{3}\cdot S_{\Delta ABC}\cdot SA\Rightarrow SA=\dfrac{3V_{S.ABC}}{S_{\Delta ABC}}=\dfrac{3\cdot\dfrac{a^3}{4}}{\dfrac{a^2\sqrt{3}}{4}}=a\sqrt{3}.$$}
\end{ex}

%[Câu 17]
\begin{ex}%[2H1B3-2]
	(THPT Minh Châu Hưng Yên 2019) Cho hình chóp $S.ABC$ có đáy $ABC$ là tam giác đều cạnh $ a$. Biết $SA\perp\left(ABC\right)$ và $ SA=a\sqrt{3}$. Tính thể tích khối chóp $S.ABC$
	\choice
	{$\dfrac{a}{4}$}
	{$\dfrac{a^3}{2}$}
	{\True $\dfrac{a^3}{4}$}
	{$\dfrac{3a^3}{4}$}
	\loigiai
	{
		\begin{center}
			\begin{tikzpicture}
				%		\draw[gray] (0,0) grid (10,10);
				\path 
				(0:0) coordinate (A)
				+(0:5) coordinate (C)
				+(-30:3) coordinate (B)
				+(90:4) coordinate (S)
				;
				\draw[dashed] 
				(A)--(C)
				;
				\draw 
				(S)--(B)
				(S)--(A)--(B)--(C)--cycle
				;
				\path 
				(B)--(C) node[midway,below right] {$a$}
				(B)--(A) node[midway,below left] {$a$}
				(C)--(A) node[midway,above] {$a$}
				(S)--(A) node[midway,left] {$a\sqrt{3}$}
				;
				\foreach \x/\g in {A/180,B/-90,C/0,S/90}
				\fill (\x) circle (1.5pt)
				+(\g:3.5mm) node {$\x$};
			\end{tikzpicture}
		\end{center}
		Ta có $ SA$ là đường cao hình chóp.\\
		Tam giác $ ABC$ đều cạnh $ a$ nên $S_{\Delta ABC}=\dfrac{a^2\sqrt{3}}{4}.$\\
		Vậy thể tích cần tìm là $V_{S.ABC}=\dfrac{1}{3}\cdot \dfrac{a^2\sqrt{3}}{4}\cdot a\sqrt{3}=\dfrac{a^3}{4}$.}
\end{ex}
%
%[Câu 18]
\begin{ex}%[2H1B3-2]
	(THPT Việt Đức Hà Nội 2019) Cho hình chóp $S.ABC$ có đáy là tam giác đều cạnh $a$. Cạnh bên $SC$ vuông góc với mặt phẳng $\left(ABC\right)$, $SC=a$. Thể tích khối chóp $S.ABC$ bằng
	\choice
	{$\dfrac{a^3\sqrt{3}}{3}$}
	{$\dfrac{a^3\sqrt{2}}{12}$}
	{$\dfrac{a^3\sqrt{3}}{9}$}
	{\True $\dfrac{a^3\sqrt{3}}{12}$}
	\loigiai
	{
		\begin{center}
			\begin{tikzpicture}
				%		\draw[gray] (0,0) grid (10,10);
				\path 
				(0:0) coordinate (A)
				+(0:5) coordinate (C)
				+(-30:3) coordinate (B)
				+(90:4) coordinate (S)
				;
				\draw[dashed] 
				(A)--(C)
				;
				\draw 
				(S)--(B)
				(S)--(A)--(B)--(C)--cycle
				;
				\path 
				(S)--(A) node[midway,left] {$a$}
				;
				\path pic[draw,angle radius=8pt]{right angle= C--A--S};
				\foreach \x/\g in {A/135,B/-90,C/0,S/90}
				\fill (\x) circle (1.5pt)
				+(\g:3.5mm) node {$\x$};
			\end{tikzpicture}
		\end{center}
		$$S_{ABC}=\dfrac{a^2\sqrt{3}}{4} \Rightarrow{V_{S.ABC}}=\dfrac{1}{3}\cdot a\cdot \dfrac{a^2\sqrt{3}}{4}=\dfrac{a^3\sqrt{3}}{12}.$$}
\end{ex}
%
%[Câu 19]
\begin{ex}%[2H1B3-2]
	(THPT An Lão Hải Phòng 2019) Cho tứ diện $ABCD$ có $AD$ vuông góc với mặt phẳng $\left(ABC\right)$ biết đáy $ABC$ là tam giác vuông tại $B$ và $AD=10,\,\,AB=10,\,\,BC=24$. Tính thể tích của tứ diện $ABCD$
	\choice
	{$V=1200$}
	{$V=960$}
	{\True $V=400$}
	{$V=\dfrac{1300}{3}$}
	\loigiai
	{
		\begin{center}
			\begin{tikzpicture}
				%		\draw[gray] (0,0) grid (10,10);
				\path 
				(0:0) coordinate (D)
				+(0:5) coordinate (C)
				+(-30:3) coordinate (B)
				+(90:4) coordinate (A)
				;
				\draw[dashed] 
				(D)--(C)
				;
				\draw 
				(A)--(B)
				(A)--(D)--(B)--(C)--cycle
				;
				\path 
				(A)--(D)
				;
				\foreach \x/\g in {A/90,B/-90,C/0,D/180}
				\fill (\x) circle (1.5pt)
				+(\g:3.5mm) node {$\x$};
			\end{tikzpicture}
		\end{center}
		Ta có: $$V_{ABCD}=\dfrac{1}{3}\cdot AD\cdot\dfrac{1}{2}\cdot AB\cdot BC=\dfrac{1}{6}\cdot10\cdot10\cdot24=400.$$}
\end{ex}
%
%[Câu 20]
\begin{ex}%[2H1B3-2]
	(THPT Hùng Vương Bình Phước 2019) Cho hình chóp $S.ABC$ có cạnh bên $SA$ vuông góc với mặt phẳng đáy $\left(ABC\right)$. Biết $SA=a$, tam giác $ABC$ là tam giác vuông cân tại $A$, $AB=2a$. Tính theo $a$ thể tích $V$ của khối chóp $S.ABC$
	\choice
	{$V=\dfrac{a^3}{6}$}
	{$V=\dfrac{a^3}{2}$}
	{\True $V=\dfrac{2a^3}{3}$}
	{$V=2a^3$}
	\loigiai{
		\begin{center}
			\begin{center}
				\begin{tikzpicture}
					%		\draw[gray] (0,0) grid (10,10);
					\path 
					(0:0) coordinate (A)
					+(0:5) coordinate (C)
					+(-30:3) coordinate (B)
					+(90:4) coordinate (S)
					;
					\draw[dashed] 
					(A)--(C)
					;
					\draw 
					(S)--(B)
					(S)--(A)--(B)--(C)--cycle
					;
					\path 
					(S)--(A) node[midway,left] {$a$}
					(B)--(A) node[midway,below left] {$2a$}
					;
					\path pic[draw,angle radius=8pt]{right angle= C--A--S};
					\path pic[draw,angle radius=8pt]{right angle= B--A--C};
					\foreach \x/\g in {A/135,B/-90,C/0,S/90}
					\fill (\x) circle (1.5pt)
					+(\g:3.5mm) node {$\x$};
				\end{tikzpicture}
			\end{center}
		\end{center}
		Diện tích tam giác $ABC$ vuông cân tại $A$ là $$S_{ABC}=\dfrac{1}{2}\cdot AB\cdot AC=\dfrac{1}{2}\cdot 2a \cdot2a=2a^2.$$
		Thể tích khối chóp $S.ABC$ là $$V_{S.ABC}=\dfrac{1}{3}\cdot SA\cdot S_{ABC}=\dfrac{1}{3}\cdot a\cdot 2a^2=\dfrac{2a^3}{3}.$$}
\end{ex}

%[Câu 21]
\begin{ex}%[2H1B3-2]
	(Chuyên KHTN 2019) Cho khối chóp $ S.ABC$ có đáy $ ABC$ là tam giác vuông tại $B$, $ AB=a,\,AC=2a,\,SA\perp\left(ABC\right)$ và $ SA=a$. Thể tích của khối chóp đã cho bằng
	\choice
	{$\dfrac{a^3\sqrt{3}}{3}$}
	{\True $\dfrac{a^3\sqrt{3}}{6}$}
	{$\dfrac{a^3}{3}$}
	{$\dfrac{2a^3}{3}$}
	\loigiai{
		\begin{center}
			\begin{tikzpicture}
				%		\draw[gray] (0,0) grid (10,10);
				\path 
				(0:0) coordinate (A)
				+(0:5) coordinate (C)
				+(-30:3) coordinate (B)
				+(90:4) coordinate (S)
				;
				\draw[dashed] 
				(A)--(C)
				;
				\draw 
				(S)--(B)
				(S)--(A)--(B)--(C)--cycle
				;
				\path 
				(S)--(A) node[midway,left] {$a$}
				(B)--(A) node[midway,below left] {$a$}
				(C)--(A) node[midway,above] {$2a$}
				;
				\path pic[draw,angle radius=8pt]{right angle= C--A--S};
				\path pic[draw,angle radius=8pt]{right angle= C--B--A};
				\foreach \x/\g in {A/135,B/-90,C/0,S/90}
				\fill (\x) circle (1.5pt)
				+(\g:3.5mm) node {$\x$};
			\end{tikzpicture}
		\end{center}
		Ta có $ BC^2=AC^2-AB^2=3a^2\Rightarrow BC=a\sqrt{3}$.\\
		Vậy thể tích khối chóp $S.ABC$ là $$V_{S.ABC}=\dfrac{1}{3}{S_{\Delta ABC}}.SA=\dfrac{1}{3}\cdot\dfrac{1}{2}AB\cdot BC\cdot SA=\dfrac{1}{6}\cdot a\cdot a\sqrt{3}\cdot a=\dfrac{a^3\sqrt{3}}{6}.$$}
\end{ex}

%[Câu 22]
\begin{ex}%[2H1B3-2]
	(Sở Cần Thơ 2019) Cho hình chóp $S.ABCD$ có đáy $ABCD$ là hình chữ nhật, $ AB=3a$ và $ AD=4a$. Cạnh bên $SA$ vuông góc với mặt phẳng $\left(ABCD\right)$ và $ SA=a\sqrt{2}$. Thể tích của khối chóp $S.ABCD$ bằng
	\choice
	{\True $ 4\sqrt{2}{a^3}$}
	{$ 12\sqrt{2}{a^3}$}
	{$\dfrac{4\sqrt{2}{a^3}}{3}$}
	{$\dfrac{2\sqrt{2}{a^3}}{3}$}
	\loigiai
	{
		\begin{center}
			\begin{tikzpicture}
				\path 
				(0:0) coordinate (A)
				+(0:5) coordinate (B)
				+(-150:2.5) coordinate (D)
				+(90:4) coordinate (S)
				($(B)+(D)-(A)$) coordinate (C)
				;
				\draw[dashed] 
				(A)--(B) (A)--(D) (A)--(S)
				;
				\draw 
				(D)--(C)--(B)
				(S)--(B) (S)--(C) (S)--(D)
				;
				\path 
				(S)--(A) node[midway,right] {$a\sqrt{2}$}
				(D)--(A) node[midway,above left] {$4a$}
				(B)--(A) node[midway,above right] {$3a$}
				;
				\foreach \x/\g in {A/135,B/0,C/-45,D/-135,S/90}
				\fill (\x) circle (1.5pt)
				+(\g:3.5mm) node {$\x$};
			\end{tikzpicture}
		\end{center}
		Diện tích đáy hình chữ nhật là $ S=AB\cdot AD=3a\cdot 4a=12a^2$ (đvdt).\\
		Thể tích của hình chóp có đáy hình chữ nhật là $$ V=\dfrac{1}{3}Sh=\dfrac{1}{3}\cdot 12a^2\cdot a\sqrt{2}=4\sqrt{2}{a^3}.$$}
\end{ex}

%[Câu 23]
\begin{ex}%[2H1Y3-2]
	(Sở Cần Thơ 2019) Thể tích của khối chóp có diện tích đáy bằng $\dfrac{\sqrt{3}}{2}$ và chiều cao bằng $\dfrac{2\sqrt{3}}{3}$ là
	\choice
	{$\dfrac{\sqrt{6}}{6}$}
	{\True $\dfrac{1}{3}$}
	{$\dfrac{\sqrt{2}}{3}$}
	{$1$}
	\loigiai
	{
		Thể tích khối chóp là $V=\dfrac{1}{3} \cdot$ chiều cao $\cdot$  diện tích đáy $=\dfrac{1}{3}\cdot \dfrac{2\sqrt{3}}{3} \cdot \dfrac{\sqrt{3}}{2}=\dfrac{1}{3}.$}
\end{ex}
%
%[Câu 24]
\begin{ex}%[2H1B3-2]
	(Sở Nam Định 2019) Cho khối chóp $S.ABC$ có đáy $ABC$ là tam giác vuông cân tại $B$, độ dài cạnh $AB=BC=a$, cạnh bên $SA$ vuông góc với đáy và $SA=2a$. Tính thể tích $V$ của khối chóp $S.ABC$
	\choice
	{\True $V=\dfrac{a^3}{3}$}
	{$V=\dfrac{a^3}{2}$}
	{$V=a^3$}
	{$V=\dfrac{a^3}{6}$}
	\loigiai{
		\begin{center}
			\begin{tikzpicture}
				%		\draw[gray] (0,0) grid (10,10);
				\path 
				(0:0) coordinate (A)
				+(0:5) coordinate (C)
				+(-30:3) coordinate (B)
				+(90:4) coordinate (S)
				;
				\draw[dashed] 
				(A)--(C)
				;
				\draw 
				(S)--(B)
				(S)--(A)--(B)--(C)--cycle
				;
				\path 
				(S)--(A) node[midway,left] {$2a$}
				(B)--(A) node[midway,below left] {$a$}
				(B)--(C) node[midway,below right] {$a$}
				;
				\path pic[draw,angle radius=8pt]{right angle= C--A--S};
				\path pic[draw,angle radius=8pt]{right angle= C--B--A};
				\foreach \x/\g in {A/135,B/-90,C/0,S/90}
				\fill (\x) circle (1.5pt)
				+(\g:3.5mm) node {$\x$};
			\end{tikzpicture}
		\end{center}
		Thể tích khối chóp $S.ABC$ là $$V_{S.ABC}=\dfrac{1}{3}\cdot SA\cdot{S_{ABC}}=\dfrac{1}{3}\cdot 2a\cdot\dfrac{1}{2}\cdot{a^2}=\dfrac{a^3}{3}.$$}
\end{ex}
%
%[Câu 25]
\begin{ex}%[2H1B3-2]
	(Bạc Liêu – Ninh Bình 2019) Cho hình chóp $S.ABC$, có đáy $ABC$ là tam giác vuông cân tại $A$, $SA=AB=a$, $SA$ vuông góc với mặt phẳng $\left(ABC\right)$. Thể tích của khối chóp $S.ABC$ bằng
	\choice
	{$\dfrac{a^3}{3}$}
	{\True $\dfrac{a^3}{6}$}
	{$\dfrac{a^3}{2}$}
	{$\dfrac{3a^3}{2}$}
	\loigiai{
		\begin{center}
			\begin{tikzpicture}
				%		\draw[gray] (0,0) grid (10,10);
				\path 
				(0:0) coordinate (A)
				+(0:5) coordinate (C)
				+(-30:3) coordinate (B)
				+(90:4) coordinate (S)
				;
				\draw[dashed] 
				(A)--(C)
				;
				\draw 
				(S)--(B)
				(S)--(A)--(B)--(C)--cycle
				;
				\path 
				(S)--(A) node[midway,left] {$a$}
				(B)--(A) node[midway,below left] {$a$}
				%		(B)--(C) node[midway,below right] {$a$}
				;
				\path pic[draw,angle radius=8pt]{right angle= C--A--S};
				\path pic[draw,angle radius=8pt]{right angle= B--A--C};
				\foreach \x/\g in {A/135,B/-90,C/0,S/90}
				\fill (\x) circle (1.5pt)
				+(\g:3.5mm) node {$\x$};
			\end{tikzpicture}
		\end{center}
		Thể tích của khối chóp $S.ABC$ $$V_{S.ABC}=\dfrac{1}{3}\cdot SA\cdot S_{ABC}=\dfrac{a^3}{6}.$$}
\end{ex}

%[Câu 26]
\begin{ex}%[2H1B3-2]
	(Nguyễn Khuyến HCM-2019) Cho tứ diện $ OABC$ có $ OA,$ $ OB,$ $ OC$ đôi một vuông góc và $ OA=OB=OC=a$. Khi đó thể tích của tứ diện $ OABC$ là
	\choice
	{$\dfrac{a^3}{12}$}
	{\True $\dfrac{a^3}{6}$}
	{$\dfrac{a^3}{3}$}
	{$\dfrac{a^3}{2}$}
	\loigiai
	{
		\begin{center}
			\begin{tikzpicture}
				%		\draw[gray] (0,0) grid (10,10);
				\path 
				(0:0) coordinate (O)
				+(0:5) coordinate (C)
				+(-30:3) coordinate (B)
				+(90:4) coordinate (A)
				;
				\draw[dashed] 
				(O)--(C)
				;
				\draw 
				(A)--(B)
				(A)--(O)--(B)--(C)--cycle
				;
				\path 
				(A)--(O) node[midway,left] {$a$}
				(B)--(O) node[midway,below] {$a$}
				(C)--(O) node[midway,above] {$a$}
				;
				\path pic[draw,angle radius=10pt]{right angle= C--O--A};
				\path pic[draw,angle radius=10pt]{right angle= B--O--C};
				\path pic[draw,angle radius=10pt]{right angle= B--O--A};
				\foreach \x/\g in {A/135,B/-90,C/0,O/180}
				\fill (\x) circle (1.5pt)
				+(\g:3.5mm) node {$\x$};
			\end{tikzpicture}
		\end{center}
		Thể tích của tứ diện $OABC$ là $$ V=\dfrac{1}{3}{S_{OBC}}\cdot OA=\dfrac{1}{3}\cdot \dfrac{1}{2}\cdot OB\cdot OC\cdot OA=\dfrac{a^3}{6}.$$}
\end{ex}

%[Câu 27]
\begin{ex}%[2H1Y3-2]
	(THPT Minh Khai-2019) Cho hình chóp $ S.ABC$ có diện tích đáy là $a^2\sqrt{3}$, cạnh bên $ SA$ vuông góc với đáy, $ SA=a$. Tính thể tích khối chóp $ S.ABC$ theo $a$
	\choice
	{$a^3\sqrt{3}$}
	{\True $\dfrac{a^3\sqrt{3}}{3}$}
	{$\dfrac{a^3\sqrt{3}}{6}$}
	{$\dfrac{a^3\sqrt{3}}{2}$}
	\loigiai{
		Ta có: $$V=\dfrac{1}{3}Sh=\dfrac{a^3\sqrt{3}}{3}.$$}
\end{ex}
%
%[Câu 28]
\begin{ex}%[2H1B3-2]
	(Thpt Vĩnh Lộc-Thanh Hóa 2019) Cho hình chóp $S.ABCD$ có đáy là hình vuông $ABCD$ cạnh $a$, cạnh bên $SA$ vuông góc với mặt
	phẳng đáy và $ SA=a\sqrt{2}$. Thể tích của khối chóp $S.ABCD$ bằng
	\choice
	{$ V=\sqrt{2}{a^3}$}
	{$ V=\dfrac{\sqrt{2}{a^3}}{6}$}
	{$ V=\dfrac{\sqrt{2}{a^3}}{4}$}
	{\True $ V=\dfrac{\sqrt{2}{a^3}}{3}$}
	\loigiai{
		\begin{center}
			\begin{tikzpicture}
				\path 
				(0:0) coordinate (A)
				+(0:5) coordinate (B)
				+(-150:2.5) coordinate (D)
				+(90:4) coordinate (S)
				($(B)+(D)-(A)$) coordinate (C)
				;
				\draw[dashed] 
				(A)--(B) (A)--(D) (A)--(S)
				;
				\draw 
				(D)--(C)--(B)
				(S)--(B) (S)--(C) (S)--(D)
				;
				\path 
				(S)--(A) node[midway,right] {$a\sqrt{2}$}
				(D)--(A) node[midway,above left] {$a$}
				%		(B)--(A) node[midway,above right] {$3a$}
				;
				\foreach \x/\g in {A/135,B/0,C/-45,D/-135,S/90}
				\fill (\x) circle (1.5pt)
				+(\g:3.5mm) node {$\x$};
			\end{tikzpicture}
		\end{center}
		$$V_{S.ABCD}=\dfrac{1}{3}\cdot SA\cdot S_{ABCD}=\dfrac{1}{3}\cdot a\sqrt{2}\cdot a^2=\dfrac{a^3\sqrt{2}}{3}.$$}
\end{ex}
%
%[Câu 29]
\begin{ex}%[2H1B3-2]
	(Hội 8 trường chuyên ĐBSH-2019) Cho hình chóp tứ giác $ S.ABCD$ có đáy $ ABCD$ là hình vuông cạnh bằng $ a$, $ SA\perp\left(ABC\right)$, $ SA=3a$. Thể tích $ V$ của khối chóp $ S.ABCD$ là
	\choice
	{\True $ V=a^3$}
	{$ V=3a^3$}
	{$ V=\dfrac{1}{3}{a^3}$}
	{$ V=2a^3$}
	\loigiai
	{
		\begin{center}
			\begin{tikzpicture}
				\path 
				(0:0) coordinate (A)
				+(0:5) coordinate (B)
				+(-150:2.5) coordinate (D)
				+(90:4) coordinate (S)
				($(B)+(D)-(A)$) coordinate (C)
				;
				\draw[dashed] 
				(A)--(B) (A)--(D) (A)--(S)
				;
				\draw 
				(D)--(C)--(B)
				(S)--(B) (S)--(C) (S)--(D)
				;
				\path 
				(S)--(A) node[midway,right] {$3a$}
				(D)--(A) node[midway,above left] {$a$}
				%		(B)--(A) node[midway,above right] {$3a$}
				;
				\foreach \x/\g in {A/135,B/0,C/-45,D/-135,S/90}
				\fill (\x) circle (1.5pt)
				+(\g:3.5mm) node {$\x$};
			\end{tikzpicture}
		\end{center}
		Diện tích đáy $ ABCD$ là $S_{ABCD}=a^2$.\\
		Vì $ SA\perp\left(ABC\right)$ nên chiều cao của khối chóp là $ SA=3a$.\\
		Vậy thể tích khối chóp $ S.ABCD$ là $$V=\dfrac{1}{3}\cdot S_{ABCD}\cdot SA=\dfrac{1}{3}\cdot a^2\cdot 3a=a^3.$$}
\end{ex}

%[Câu 30]
\begin{ex}%[2H1B3-2]
	(THPT Hàm Rồng 2019) Cho hình chóp $ S.ABCD$ có đáy $ ABCD$ là hình vuông cạnh $a$. Biết $ SA\perp\left(ABCD\right)$ và $SA=a\sqrt{3}$. Thể tích của khối chóp $S.ABCD$ là
	\choice
	{$\dfrac{a^3\sqrt{3}}{12}$}
	{$a^3\sqrt{3}$}
	{\True $\dfrac{a^3\sqrt{3}}{3}$}
	{$\dfrac{a^3}{4}$}
	\loigiai
	{
		\begin{center}
			\begin{tikzpicture}
				\path 
				(0:0) coordinate (A)
				+(0:5) coordinate (B)
				+(-150:2.5) coordinate (D)
				+(90:4) coordinate (S)
				($(B)+(D)-(A)$) coordinate (C)
				;
				\draw[dashed] 
				(A)--(B) (A)--(D) (A)--(S)
				;
				\draw 
				(D)--(C)--(B)
				(S)--(B) (S)--(C) (S)--(D)
				;
				\path 
				(S)--(A) node[midway,right] {$a\sqrt{3}$}
				(D)--(A) node[midway,above left] {$a$}
				(B)--(A) node[midway,above right] {$a$}
				;
				\foreach \x/\g in {A/135,B/0,C/-45,D/-135,S/90}
				\fill (\x) circle (1.5pt)
				+(\g:3.5mm) node {$\x$};
			\end{tikzpicture}
		\end{center}
		Khối chóp $ S.ABCD$ có chiều cao $ h=a\sqrt{3}$ và diện tích đáy $ S=a^2$.\\
		Suy ra, ta có thể tích $$ V=\dfrac{1}{3}\cdot a^2\cdot a\sqrt{3}=\dfrac{a^3\sqrt{3}}{3}.$$}
\end{ex}

%[Câu 31]
\begin{ex}%[2H1Y3-2]
	(THPT Cộng Hiền-2019) Khẳng định nào sau đây là sai?
	\choice
	{Thể tích của khối chóp có diện tích đáy $S$ và chiều cao $h$ là $ V=\dfrac{1}{3}Sh$}
	{Thể tích của khối lăng trụ có diện tích đáy $S$ và chiều cao $h$ là $ V=Sh$}
	{Thể tích của một khối hộp chữ nhật bằng tích ba kính thước của nó}
	{\True Thể tích của khối chóp có diện tích đáy $S$ và chiều cao $ h$ là $ V=3Sh$}
	\loigiai
	{
		Thể tích của khối chóp có diện tích đáy $S$ và chiều cao $h$ là $ V=\dfrac{1}{3}Sh.$\\
		Thể tích của khối lăng trụ có diện tích đáy $S$ và chiều cao $h$ là $ V=Sh.$\\
		Thể tích của một khối hộp chữ nhật bằng tích ba kính thước của nó.
	}
	
\end{ex}
%
%[Câu 32]
\begin{ex}%[2H1B3-2]
	(Lý Nhân Tông-Bắc Ninh 2019) Cho hình chóp $ S.ABC$ có đáy $ ABC$ là tam giác vuông tại $ B.$ Cạnh bên $SA$ vuông góc với mặt phẳng đáy. Biết $ SA=AB=2a$, $ BC=3a$. Tính thể tích của $ S.ABC$ là
	\choice
	{$3a^3$}
	{$4a^3$}
	{\True $2a^3$}
	{$a^3$}
	\loigiai
	{
		\begin{center}
			\begin{tikzpicture}
				%		\draw[gray] (0,0) grid (10,10);
				\path 
				(0:0) coordinate (A)
				+(0:5) coordinate (C)
				+(-30:3) coordinate (B)
				+(90:4) coordinate (S)
				;
				\draw[dashed] 
				(A)--(C)
				;
				\draw 
				(S)--(B)
				(S)--(A)--(B)--(C)--cycle
				;
				\path 
				(S)--(A) node[midway,left] {$2a$}
				(B)--(A) node[midway,below left] {$2a$}
				(B)--(C) node[midway,below right] {$3a$}
				;
				\path pic[draw,angle radius=8pt]{right angle= C--A--S};
				\path pic[draw,angle radius=8pt]{right angle= C--B--A};
				\foreach \x/\g in {A/135,B/-90,C/0,S/90}
				\fill (\x) circle (1.5pt)
				+(\g:3.5mm) node {$\x$};
			\end{tikzpicture}
		\end{center}
		Thể tích của khối chóp $S.ABC$ là $$V=\dfrac{1}{3}\cdot \dfrac{1}{2}\cdot AB\cdot BC\cdot SA=2a^3.$$}
\end{ex}
%
%[Câu 33]
\begin{ex}%[2H1B3-2]
	(Kinh Môn-Hải Dương 2019) Cho hình chóp $ S.ABCD$ có đáy $ ABCD$ hình chữ nhật với $ AB=4a$, $ BC=a$, cạnh bên $ SD=2a$ và $ SD$ vuông góc với mặt phẳng đáy. Thể tích khối chóp $ S.ABCD$ bằng
	\choice
	{$ 6a^3$}
	{$ 3a^3$}
	{\True $\dfrac{8}{3}{a^3}$}
	{$\dfrac{2}{3}{a^3}$}
	\loigiai
	{
		\begin{center}
			\begin{tikzpicture}
				\path 
				(0:0) coordinate (D)
				+(0:5) coordinate (C)
				+(-150:2.5) coordinate (A)
				+(90:4) coordinate (S)
				($(C)+(A)-(D)$) coordinate (B)
				;
				\draw[dashed] 
				(D)--(C) (D)--(A) (D)--(S)
				;
				\draw 
				(A)--(B)--(C)
				(S)--(B) (S)--(C) (S)--(A)
				;
				\path 
				(S)--(D) node[midway,right] {$2a$}
				(B)--(C) node[midway,below right] {$a$}
				(B)--(A) node[midway,below] {$4a$}
				;
				\foreach \x/\g in {A/-90,B/-90,C/0,D/135,S/90}
				\fill (\x) circle (1.5pt)
				+(\g:3.5mm) node {$\x$};
			\end{tikzpicture}
		\end{center}
		Theo đề, ta có thể tích hình chóp: $ S.ABCD$ là $ V=\dfrac{1}{3}\cdot S_{ABCD}\cdot SD$.\\
		$ABCD$ là hình chữ nhật nên $S_{ABCD}=AB\cdot BC=4a^2$.\\ 
		Thể tích khối chóp $S.ABCD$ là $$V_{S.ABCD}=\dfrac{1}{3}\cdot 4a^2\cdot 2a=\dfrac{8}{3}{a^3}.$$}
\end{ex}
%
%[Câu 34]
\begin{ex}%[2H1B3-2]
	(Sở Điện Biên-2019) Tính thể tích của khối chóp $S.ABC$ có $ SA$ là đường cao, đáy là tam giác $ BAC$ vuông cân tại $ A$; $ SA=AB=a$.
	\choice
	{$ V=\dfrac{a^3}{3}$}
	{\True $ V=\dfrac{a^3}{6}$}
	{$ V=\dfrac{2a^3}{3}$}
	{$ V=\dfrac{a^3}{9}$}
	\loigiai
	{
		\begin{center}
			\begin{tikzpicture}
				%		\draw[gray] (0,0) grid (10,10);
				\path 
				(0:0) coordinate (A)
				+(0:5) coordinate (C)
				+(-30:3) coordinate (B)
				+(90:4) coordinate (S)
				;
				\draw[dashed] 
				(A)--(C)
				;
				\draw 
				(S)--(B)
				(S)--(A)--(B)--(C)--cycle
				;
				\path 
				(S)--(A) node[midway,left] {$a$}
				(B)--(A) node[midway,below left] {$a$}
				%		(B)--(C) node[midway,below right] {$3a$}
				;
				\path pic[draw,angle radius=8pt]{right angle= C--A--S};
				\path pic[draw,angle radius=8pt]{right angle= B--A--C};
				\foreach \x/\g in {A/135,B/-90,C/0,S/90}
				\fill (\x) circle (1.5pt)
				+(\g:3.5mm) node {$\x$};
			\end{tikzpicture}
		\end{center}
		Thể tích khối chóp $S.ABC$ là $$V_{S.ABC}=\dfrac{1}{3}\cdot SA\cdot S_{\Delta ABC}=\dfrac{1}{3}\cdot SA\cdot \dfrac{1}{2}\cdot AB\cdot BC=\dfrac{1}{6}\cdot a\cdot a\cdot a=\dfrac{a^3}{6}.$$
	}
\end{ex}
\begin{ex}%[2H1B3-2]
	[Sở Điện Biên-2019] Tính thể tích của khối chóp $ S.ABC$ có $ SA$ là đường cao, đáy là tam giác $ BAC$ vuông cân tại $ A$, $ SA=AB=a$
	\choice
	{$ V=\dfrac{a^3}{3}$}
	{\True $ V=\dfrac{a^3}{6}$}
	{$ V=\dfrac{2a^3}{3}$}
	{$ V=\dfrac{a^3}{9}$}
	\loigiai{
		\begin{center}
			\begin{tikzpicture}[scale=.8]
				\coordinate [label=left:$A$](A) at (0,0);
				\coordinate [label=above:$S$](S) at (0,5);
				\coordinate [label=below:$B$](B) at (3,-3);
				\coordinate [label=right:$C$](C) at (8,0);
				\foreach \diem in {A,B,C,S}	\fill (\diem)circle(1.5pt);		
				\draw[smooth](S)--(A)--(B)--(C)--(S)--(B);
				\draw[dashed](A)--(C);
				
			\end{tikzpicture}
		\end{center}
		Ta có $V_{S.ABC}=\dfrac{1}{3}\cdot SA \cdot S_{\triangle ABC}=\dfrac{1}{3}SA\cdot \dfrac{1}{2}\cdot AB\cdot BC=\dfrac{1}{6}\cdot a\cdot a\cdot a=\dfrac{a^3}{6}$.}
\end{ex}
\begin{dang}{Mặt bên vuông góc với đáy}
\end{dang}
\begin{ex}%[2H1B3-2]
	[THPT Lương Thế Vinh Hà Nội 2019] Cho hình chóp $ S.ABC$ có đáy $ ABC$ là tam giác vuông cân tại $ B$ và $ AB=2a$. Tam giác $ SAB$ đều và nằm trong mặt phẳng vuông góc với đáy. Tính thể tích $ V$ của khối chóp $ S.ABC$
	\choice
	{$ V=\dfrac{a^3\sqrt{3}}{4}$}
	{$ V=\dfrac{a^3\sqrt{3}}{3}$}
	{$ V=\dfrac{a^3\sqrt{3}}{12}$}
	{\True $ V=\dfrac{2a^3\sqrt{3}}{3}$}
	\loigiai{
		\begin{center}
			\begin{tikzpicture}[scale=.8]
				\coordinate [label=left:$A$](A) at (0,0);
				\coordinate [label=below:$B$](B) at (3,-3);
				\coordinate [label=right:$C$](C) at (8,0);
				\coordinate [label=below:$H$](H) at ($(A)! .5 ! (B)$);
				\coordinate[label=above left:$S$] (S) at ($(H)+(90:5)$);
				\foreach \diem in {A,B,C,S,H}	\fill (\diem)circle(1.5pt);		
				\draw[smooth](S)--(A)--(B)--(C)--(S)--(B);
				\draw (S)--(H);
				\draw[dashed](A)--(C);
				
			\end{tikzpicture}
		\end{center}
		Gọi $ H$ là trung điểm của $ AB$ suy ra $ SH=a\sqrt{3}$.\\
		$ AB=2a\Rightarrow BC=2a\Rightarrow{S_{\triangle ABC}}=\dfrac{1}{2}{\left(2a\right)^2}=2a^2$.\\
		$V_{S.ABC}=\dfrac{1}{3} S_{ABC} SH=\dfrac{1}{3}2a^2a\sqrt{3}=\dfrac{2a^3\sqrt{3}}{3}$.}
\end{ex}

\begin{ex}%[2H1B3-2]
	[Chuyên Bắc Ninh 2019] Cho khối chóp $ S.ABCD$ có đáy là hình vuông cạnh $ a\sqrt{2}$, tam giác $ SAC$ vuông tại $ S$ và nằm trong mặt phẳng vuông góc với đáy, cạnh bên $ SA$ tạo với đáy góc $60^{\circ}$. Tính thể tích $ V$ của khối chóp $ S.ABCD$
	\choice
	{$ V=\dfrac{a^3\sqrt{3}}{12}$}
	{\True $ V=\dfrac{a^3\sqrt{3}}{3}$}
	{$ V=\dfrac{a^3\sqrt{6}}{12}$}
	{$ V=\dfrac{a^3\sqrt{2}}{12}$}
	\loigiai{
		\begin{center}
			\begin{tikzpicture}[scale=.8]
				\coordinate [label=left:$A$](A) at (3,2);
				\coordinate [label=below:$B$](B) at (0,0);
				\coordinate [label=below:$C$](C) at (6,0);
				\coordinate[label=right:$D$] (D) at ($(C)-(B)+(A)$);
				\coordinate [label=below:$H$] (H) at ($(A)! .5 ! (C)$);
				\coordinate[label=above left:$S$] (S) at ($(H)+(90:6)$);
				
				\foreach \diem in {A,B,C,D,S,H}	\fill (\diem)circle(1.5pt);		
				\draw[smooth](B)--(C) (C)--(D) (S)--(B) (S)--(C) (S)--(D);
				\draw[dashed](A)--(S) (A)--(B) (A)--(D);
				\draw[dashed](S)--(H);
				\draw[dashed](A)--(C);
			\end{tikzpicture}
		\end{center}
		Kẻ $ SH\perp AC$, $ H\in AC$ suy ra $SH\perp\left(ABCD\right)$ .\\
		$ AC=2a$, tam giác $ SAC$ vuông ở $ S$, góc $ \widehat{SAC}=60^{\circ}$ \\
		nên $ SA=a,\,SC=a\sqrt{3},\,SH=\dfrac{a\sqrt{3}}{2}$.\\
		Thể tích hình chóp là $ V=\dfrac{1}{3}{\left(a\sqrt{2}\right)^2}\cdot \dfrac{a\sqrt{3}}{2}=\dfrac{a^3\sqrt{3}}{3}$.}
\end{ex}

\begin{ex}%[2H1B3-2]
	(SGD Nam Định 2019) Cho hình chóp $ S.ABCD$ có đáy là hình vuông cạnh bằng $ 2a$. Mặt bên $\left(SAB\right)$ là tam giác đều và nằm trong mặt phẳng vuông góc với mặt phẳng $\left(ABCD\right)$. Thể tích của khối chóp $ S.ABCD$ là
	\choice
	{$ 4a^3\sqrt{3}$}
	{$\dfrac{a^3\sqrt{3}}{2}$}
	{$\dfrac{a^3\sqrt{3}}{4}$}
	{\True $\dfrac{4a^3\sqrt{3}}{3}$}
	\loigiai{
		\begin{center}
			\begin{tikzpicture}[scale=.8]
				\coordinate [label=left:$A$](A) at (3,2);
				\coordinate [label=below:$B$](B) at (0,0);
				\coordinate [label=below:$C$](C) at (6,0);
				\coordinate [label=above:$S$](S) at (1.5,6);
				\coordinate[label=right:$D$] (D) at ($(C)-(B)+(A)$);
				\coordinate[label=below:$H$](H) at ($(A)! .5 ! (B)$);
				
				\foreach \diem in {A,B,C,D,S,H}	\fill (\diem)circle(1.5pt);		
				\draw[smooth](B)--(C) (C)--(D) (S)--(B) (S)--(C) (S)--(D);
				\draw[dashed](A)--(S) (A)--(B) (A)--(D);
				\draw[dashed](S)--(H);
			\end{tikzpicture}
		\end{center}
		Gọi $ H$ là trung điểm của $ AB$, ta có $ SH\perp AB$.\\
		Mà $\left(SAB\right)\perp\left(ABCD\right)$ theo giao tuyến là đường thẳng $ AB$ nên $ SH\perp\left(ABCD\right)$.\\
		Thể tích khối chóp $ S.ABCD$ bằng $ V=\dfrac{1}{3}{S_{ABCD}}\cdot SH=\dfrac{1}{3}\cdot \left(2a\right)^2\cdot \dfrac{2a\sqrt{3}}{2}=\dfrac{4a^3\sqrt{3}}{3}$.}
\end{ex}

\begin{ex}%[2H1B3-2]
	Cho khối chóp $ S.ABCD$ có đáy $ ABCD$ là hình vuông cạnh $ a$, tam giác $ SAB$ cân tại $ S$ và nằm trong mặt phẳng vuông góc với đáy, $ SA=2a$. Tính theo $ a$ thể tích khối chóp $ S.ABCD$
	\choice
	{$ V=2a^3$}
	{$ V=\dfrac{a^3\sqrt{15}}{12}$}
	{\True $ V=\dfrac{a^3\sqrt{15}}{6}$}
	{$ V=\dfrac{2a^3}{3}$}
	\loigiai{
		\begin{center}
			\begin{tikzpicture}[scale=.8]
				\coordinate [label=left:$A$](A) at (3,2);
				\coordinate [label=below:$B$](B) at (0,0);
				\coordinate [label=below:$C$](C) at (6,0);
				\coordinate [label=above:$S$](S) at (1.5,6);
				\coordinate[label=right:$D$] (D) at ($(C)-(B)+(A)$);
				\coordinate[label=below:$H$](H) at ($(A)! .5 ! (B)$);
				%kí hiệu độ dài	
				\coordinate[label=right:$2a$](1) at ($(S)! .5 ! (A)$);
				\coordinate[label=left:$2a$](2) at ($(S)! .5 ! (B)$);
				\coordinate[label=below:$a$](3) at ($(A)! .5 ! (D)$);
				\coordinate[label=below:$a$](3) at ($(C)! .5 ! (D)$);
				\coordinate[label=below:$a$](3) at ($(B)! .5 ! (C)$);
				
				\foreach \diem in {A,B,C,D,S,H}	\fill (\diem)circle(1.5pt);		
				\draw[smooth](B)--(C) (C)--(D) (S)--(B) (S)--(C) (S)--(D);
				\draw[dashed](A)--(S) (A)--(B) (A)--(D);
				\draw[dashed](S)--(H);
				\foreach \x/\y/\z in {S/H/A}{\draw pic[draw,angle radius=2mm]{right angle=\x--\y--\z};}
				\foreach \x/\y/\z in {A/B/C}{\draw pic[draw,angle radius=2mm]{right angle=\x--\y--\z};}
				\foreach \x/\y/\z in {A/D/C}{\draw pic[draw,angle radius=2mm]{right angle=\x--\y--\z};}
			\end{tikzpicture}
		\end{center}
		Gọi $ H$ là trung điểm $ AB$.\\
		Theo đề bài, tam giác $ SAB$ cân tại $ S$ nên suy ra $ SH\perp AB$.\\
		Mặt khác, tam giác $ SAB$ nằm trong mặt phẳng vuông góc với đáy nên suy ra $ SH\perp\left(ABCD\right)$.\\
		Xét tam giác $ SHA$ vuông tại $ H$.\\
		$ SH=\sqrt{S{A^2}-A{H^2}}=\sqrt{\left(2a\right)^2-\left(\dfrac{a}{2}\right)^2}=\dfrac{a\sqrt{15}}{2}$.\\
		Diện tích hình vuông là $S_{ABCD}=a^2$.\\
		Vậy thể tích khối chóp $ S.ABCD$ là $ V=\dfrac{1}{3}\cdot SH\cdot S_{ABCD}=\dfrac{a^3\sqrt{15}}{6}$.}
\end{ex}

\begin{ex}%[2H1B3-2]
	Cho hình chóp $ S.ABC$ có đáy $ ABC$ là tam giác vuông tại $ C$, tam giác $ SAB$ đều nằm trong mặt phẳng vuông góc với đáy. Tính theo $ a$ thể tích của khối chóp. Biết rằng $ AB=a\sqrt{3};AC=a$
	\choice
	{$\dfrac{a^3}{2}$}
	{\True $\dfrac{a^3\sqrt{2}}{4}$}
	{$\dfrac{a^3\sqrt{3}}{2}$}
	{$\dfrac{a^3\sqrt{2}}{2}$}
	\loigiai{
		\begin{center}
			\begin{tikzpicture}[scale=.8]
				\coordinate [label=left:$A$](A) at (0,0);
				\coordinate [label=above:$S$](S) at (4,5);
				\coordinate [label=below:$C$](C) at (3,-3);
				\coordinate [label=right:$B$](B) at (8,0);
				\coordinate [label=below:$H$](H) at ($(A)! .5 ! (B)$);
				\coordinate [label=below:$a\sqrt{3}$](1) at ($(H)! .5 ! (B)$);
				\coordinate [label=below:$a$](1) at ($(A)! .5 ! (C)$);
				
				\foreach \diem in {A,B,C,S}	\fill (\diem)circle(1.5pt);		
				\draw[smooth](S)--(A)--(C)--(S)--(B)--(C);
				\draw[dashed](A)--(B);
				\draw[dashed](S)--(H);
				\foreach \x/\y/\z in {A/C/B}{\draw pic[draw,angle radius=2mm]{right angle=\x--\y--\z};}
			\end{tikzpicture}
		\end{center}
		Trong mặt phẳng $\left(SAB\right)$. Gọi $ H$ là trung điểm của $ AB$.\\
		$\triangle SAB$ đều $\Rightarrow SH\perp AB.$\\
		Ta có
		$\heva{
			S H \perp A B \\
			(S A B) \cap(A B C)=A B \\
			(S A B) \perp(A B C)
		}$
		$\Rightarrow S H \perp(A B C)$.\\
		$\triangle SAB$ đều $ AB=a\sqrt{3}$ $\Rightarrow SH=\dfrac{3a}{2}$.\\
		$\triangle ABC$ là tam giác vuông cân tại $ C \Rightarrow A{B^2}=A{C^2}+B{C^2}\Rightarrow BC=\sqrt{3a^2-a^2}=a\sqrt{2}.$\\
		$V_{S.ABC}=\dfrac{1}{3}\cdot \dfrac{3a}{2}\cdot \dfrac{1}{2\cdot }a\sqrt{2}\cdot a=\dfrac{a^3\sqrt{2}}{4}$.}
\end{ex}

\begin{ex}%[2H1B3-2]
	Cho hình chóp tứ giác $ S.ABCD$ có đáy $ ABCD$ là hình vuông cạnh $a$ , mặt bên $ SAB$ là một tam giác đều và nằm trong mặt phẳng vuông góc với đáy $\left(ABCD\right)$. Tính thể tích khối chóp $S.ABCD$
	\choice
	{$\dfrac{a^3}{6}$}
	{\True $\dfrac{a^3\sqrt{3}}{6}$}
	{$\dfrac{a^3\sqrt{3}}{2}$}
	{$10$}
	\loigiai{
		\begin{center}
			\begin{tikzpicture}[scale=.8]
				\coordinate [label=left:$A$](A) at (3,2);
				\coordinate [label=below:$B$](B) at (0,0);
				\coordinate [label=below:$C$](C) at (6,0);
				\coordinate [label=above:$S$](S) at (1.5,6);
				\coordinate[label=right:$D$] (D) at ($(C)-(B)+(A)$);
				\coordinate[label=below:$H$](H) at ($(A)! .5 ! (B)$);
				\coordinate [label=below:$O$](O) at ($(A)! .5 ! (C)$);
				
				\foreach \diem in {A,B,C,D,S,H,O}	\fill (\diem)circle(1.5pt);		
				\draw[smooth](B)--(C) (C)--(D) (S)--(B) (S)--(C) (S)--(D);
				\draw[dashed](A)--(S) (A)--(B) (A)--(D);
				\draw[dashed](S)--(H);
				\draw[dashed](A)--(C);
				\draw[dashed](B)--(D);
				\foreach \x/\y/\z in {S/H/A}{\draw pic[draw,angle radius=2mm]{right angle=\x--\y--\z};}
			\end{tikzpicture}
		\end{center}
		Gọi $ H$ là trung điểm $ AB$ thì $SH\perp AB$ và $ SH=\dfrac{a\sqrt{3}}{2}$.\\
		Ta có 
		$\heva{
			&\left(SAB\right)\perp\left(ABCD\right)\\ 
			&\left(SAB\right)\cap\left(ABCD\right)=AB\\ 
			& SH\perp AB}
		\Rightarrow SH\perp\left(ABCD\right)$.\\
		Suy ra $ SH$ là đường cao của hình chóp.\\
		Diện tích đáy $S_{ABCD}=a^2$.\\
		Vậy thể tích khối chóp $ S.ABCD$ là $V_{A B C D}=\dfrac{1}{3} S H \cdot S_{A B C D}=\dfrac{1}{3} \dfrac{a \sqrt{3}}{2} \cdot a^2=\dfrac{a^3 \sqrt{3}}{6}$.}
\end{ex}

\begin{ex}%[2H1B3-2]
	(Chuyên ĐH Vinh 2019) Cho hình chóp $ S.ABCD$ có đáy $ ABCD$ là hình vuông cạnh $ a$, $ SA=\dfrac{a\sqrt{2}}{2}$, tam giác $ SAC$ vuông tại $ S$ và nằm trong mặt phẳng vuông góc với $\left(ABCD\right)$. Tính theo $ a$ thể tích $ V$ của khối chóp $ S.ABCD$
	\choice
	{\True $ V=\dfrac{\sqrt{6}{a^3}}{12}$}
	{$ V=\dfrac{\sqrt{6}{a^3}}{3}$}
	{$ V=\dfrac{\sqrt{6}{a^3}}{4}$}
	{$ V=\dfrac{\sqrt{2}{a^3}}{6}$}
	\loigiai{
		\begin{center}
			\begin{tikzpicture}[scale=.8]
				\coordinate [label=left:$A$](A) at (3,2);
				\coordinate [label=below:$B$](B) at (0,0);
				\coordinate [label=below:$C$](C) at (6,0);
				\coordinate [label=above:$S$](S) at (4,6);
				\coordinate[label=right:$D$] (D) at ($(C)-(B)+(A)$);
				\coordinate [label=below:$O$] (O) at ($(A)! .5 ! (C)$);
				\coordinate [label=below:$H$] (H) at ($(A)! .3 ! (C)$);
				
				\foreach \diem in {A,B,C,D,S}	\fill (\diem)circle(1.5pt);		
				\draw[smooth](B)--(C) (C)--(D) (S)--(B) (S)--(C) (S)--(D);
				\draw[dashed](A)--(S) (A)--(B) (A)--(D);
				\draw[dashed](S)--(H);
				\draw[dashed](A)--(C);
				\draw[dashed](B)--(D);
			\end{tikzpicture}
		\end{center}
		Gọi $ H$ là hình chiếu vuông góc của $ S$ lên $ AC$.\\
		Ta có $ SO=\dfrac{1}{2}AC=\dfrac{a\sqrt{2}}{2}$ suy ra $\triangle SAO$ là tam giác đều.\\
		$\Rightarrow SH=\dfrac{a\sqrt{6}}{4}$.\\
		Vậy $ V=\dfrac{1}{3}.\dfrac{a\sqrt{6}}{4}.a^2=\dfrac{a^3\sqrt{6}}{12}$.}
\end{ex}

\begin{ex}%[2H1B3-2]
	Cho hình chóp $ S.ABC$ có đáy là tam giác cân tại $ A$ $ ,\,\,AB=AC=a$, $\widehat{BAC}=120^\circ $. Tam giác $ SAB$ là tam giác đều và nằm trong mặt phẳng vuông góc với mặt đáy. Tính thể tích $ V$ của khối chóp $ S.ABC$
	\choice
	{$ V=\dfrac{a^3}{2}$}
	{$ V=2a^3$}
	{$ V=a^3$}
	{\True $ V=\dfrac{a^3}{8}$}
	\loigiai{
		\begin{center}
			\begin{tikzpicture}[scale=.8]
				\coordinate [label=left:$A$](A) at (0,0);
				\coordinate [label=above:$S$](S) at (4,5);
				\coordinate [label=below:$C$](C) at (3,-3);
				\coordinate [label=right:$B$](B) at (8,0);
				\coordinate [label=below:$H$](H) at ($(A)! .5 ! (B)$);			
				\foreach \diem in {A,B,C,S}	\fill (\diem)circle(1.5pt);		
				\draw[smooth](S)--(A)--(C)--(S)--(B)--(C);
				\draw[dashed](A)--(B);
				\draw[dashed](S)--(H);
				\foreach \x/\y/\z in {S/H/B}{\draw pic[draw,angle radius=2mm]{right angle=\x--\y--\z};}
			\end{tikzpicture}
		\end{center}
		Gọi $ H$ là trung điểm $ AB$, ta có $ SH\perp AB$ và $ SH=\dfrac{a\sqrt{3}}{2}$.\\
		Khi đó $\left\{\begin{aligned}
			&\left(SAB\right)\perp\left(ABC\right)\\ 
			&\left(SAB\right)\cap\left(ABC\right)=AB\\ 
			& SH\perp AB\\ 
		\end{aligned}\right.$ $\Rightarrow SH\perp\left(ABC\right)$.\\
		Thể tích khối chóp $ V=\dfrac{1}{3}SH\cdot S_{\triangle ABC}$ $=\dfrac{1}{3}\cdot \dfrac{a\sqrt{3}}{2}\cdot \dfrac{1}{2}\cdot a^2\cdot \sin\,120^\circ =\dfrac{a^3}{8}$.\\
		Vậy $ V=\dfrac{a^3}{8}$.}
\end{ex}

\begin{ex}%[2H1B3-2]
	Cho hình chóp $S.ABCD$ có đáy là hình vuông cạnh bằng $2a$ . Tam giác $SAB$ cân tại $S$ và nằm trong mặt phẳng vuông góc với đáy. Biết thể tích khối chóp $S.ABCD$ bằng $\dfrac{4a^3}{3}$ . Gọi $\alpha $ là góc giữa $SC$ và mặt đáy, tính $\tan\alpha $ \\
	\begin{center}
		\begin{tikzpicture}[scale=.8]
			\coordinate [label=left:$A$](A) at (3,2);
			\coordinate [label=below:$B$](B) at (0,0);
			\coordinate [label=below:$C$](C) at (6,0);
			\coordinate [label=above:$S$](S) at (1.5,6);
			\coordinate[label=right:$D$] (D) at ($(C)-(B)+(A)$);
			\coordinate[label=below:$H$](H) at ($(A)! .5 ! (B)$);
			\coordinate[label=below:$2a$](1) at ($(B)! .5 ! (C)$);
			\foreach \diem in {A,B,C,D,S,H}	\fill (\diem)circle(1.5pt);		
			\draw[smooth](B)--(C) (C)--(D) (S)--(B) (S)--(C) (S)--(D);
			\draw[dashed](A)--(S) (A)--(B) (A)--(D);
			\draw[dashed](S)--(H);
			\draw[dashed](H)--(C);
			\foreach \x/\y/\z in {S/H/A}{\draw pic[draw,angle radius=2mm]{right angle=\x--\y--\z};}
		\end{tikzpicture}
	\end{center}
	\choice
	{$\tan\alpha=\dfrac{\sqrt{3}}{3}$}
	{$\tan\alpha=\dfrac{2\sqrt{5}}{5}$}
	{$\tan\alpha=\dfrac{\sqrt{7}}{7}$}
	{\True $\tan\alpha=\dfrac{\sqrt{5}}{5}$}
	\loigiai{
		Dựng $SH\perp AB$, do $\left(SAB\right)\perp\left(ABCD\right)$ theo giao tuyến $ AB$ nên $SH\perp\left(ABCD\right)$ $\Rightarrow\alpha=\widehat{SCH}$ .\\
		Ta có $V_{S.ABCD}=\dfrac{1}{3}SH\cdot S_{ABCD}$ $\Rightarrow\dfrac{1}{3}SH\cdot 4a^2=\dfrac{4a^3}{3}$ $\Rightarrow SH=a$ .\\
		Do $\triangle SAB$ cân tại $S$ nên $H$ là trung điểm của $AB$ $\Rightarrow HC=\sqrt{B{H^2}+B{C^2}}=a\sqrt{5}$ .\\
		$\Rightarrow $ $\tan\alpha=\tan\widehat{SCH}$ $=\dfrac{SH}{HC}$ $=\dfrac{a}{a\sqrt{5}}$ $=\dfrac{\sqrt{5}}{5}$ .}
\end{ex}

\begin{ex}%[2H1B3-2]
	(Sở Bắc Giang 2019) Cho hình chóp $S.ABC$ có đáy $ ABC$ là tam giác vuông tại $ A$. Hình chiếu của $ S$ lên mặt phẳng $\left(ABC\right)$ là trung điểm $ H$ của $ BC$, $ AB=a$, $AC=a\sqrt{3}$ , $ SB=a\sqrt{2}$. Thể tích của khối chóp $S.ABC$ bằng
	\choice
	{$\dfrac{a^3\sqrt{3}}{2}$}
	{$\dfrac{a^3\sqrt{6}}{2}$}
	{\True $\dfrac{a^3\sqrt{3}}{6}$}
	{$\dfrac{a^3\sqrt{6}}{6}$}
	\loigiai{
		\begin{center}
			\begin{tikzpicture}[scale=.8]
				\coordinate [label=left:$A$](A) at (0,0);
				\coordinate [label=below:$C$](C) at (8,0);
				\coordinate [label=right:$B$](B) at (3,-3);
				\coordinate [label=below:$H$](H) at ($(B)! .5 ! (C)$);
				\coordinate[label=above left:$S$] (S) at ($(H)+(90:5)$);
				\foreach \diem in {A,B,C,S}	\fill (\diem)circle(1.5pt);		
				\draw[smooth](S)--(A)--(S)--(B)--(C);
				\draw[dashed](A)--(C);
				\draw (S)--(H);
				\draw (S)--(C);
				\draw (A)--(B);
				\foreach \x/\y/\z in {S/H/C}{\draw pic[draw,angle radius=2mm]{right angle=\x--\y--\z};}
			\end{tikzpicture}
		\end{center}
		Xét tam giác $ ABC$ vuông tại $ A$ có: $ BC=\sqrt{A{B^2}+A{C^2}}=\sqrt{a^2+\left(a\sqrt{3}\right)^2}=2a$.\\
		$ H$ là trung điểm của $ BC$ nên $ BH=a$.\\
		Xét tam giác $SBH$ vuông tại $ H$ có: $ SH=\sqrt{S{B^2}-H{B^2}}=\sqrt{\left(a\sqrt{2}\right)^2-a^2}=a$.\\
		Diện tích đáy $ ABC$ là $S_{ABC}=\dfrac{1}{2}AB\cdot AC=\dfrac{1}{2}{a^2}\sqrt{3}$ .\\
		Thể tích của khối chóp $S.ABC$ là: $ V=\dfrac{1}{3}SH\cdot S_{ABC}=\dfrac{1}{3}\cdot a\cdot \dfrac{1}{2}\cdot a^2\sqrt{3}=\dfrac{a^3\sqrt{3}}{6}$.}
\end{ex}
\begin{dang}{Thể tích khối chóp đều}
\end{dang}
\begin{ex}%[2H1B3-2]
	(Chuyên Hùng Vương Gia Lai 2019) Thể tích của khối chóp tứ giác đều có tất cả các cạnh bằng $ a$ là
	\choice
	{\True $\dfrac{a^3\sqrt{2}}{6}$}
	{$\dfrac{a^3\sqrt{2}}{3}$}
	{$a^3$}
	{$\dfrac{a^3\sqrt{2}}{2}$}
	\loigiai{
		\begin{center}
			\begin{tikzpicture}[line join=round,line cap=round,line width=.6pt,font=\footnotesize,scale=1]
				\coordinate[label=below left:$B$] (B) at (0,0);
				\coordinate[label=above right:$A$] (A) at (2,2);
				\coordinate[label=below right:$C$] (C) at (6,0);
				\coordinate[label=above right:$D$] (D) at ($(C)-(B)+(A)$);
				\coordinate[label=below:$H$] (H) at ($(A)!.5!(C)$);
				\coordinate[label=above left:$S$] (S) at ($(H)+(90:5)$);
				\draw (B)--(C)--(D)--(S)--cycle (S)--(C);
				\draw[dashed] (C)--(A)--(D)--(B) (H)--(S)--(A)--(B);
				\fill (A)circle(1pt) (B)circle(1pt) (C)circle(1pt) (D)circle(1pt) (S)circle(1pt) (H)circle(1pt);
			\end{tikzpicture}	
		\end{center}
		Giả sử khối chóp tứ giác đều đã cho là $ S.ABCD$.\\
		Khi đó $ ABCD$ là hình vuông cạnh $ a$ và $ SA=SB=SC=SD=a$.\\
		Gọi $ H$ là tâm của hình vuông $ ABCD$ thì $ SH\perp\left(ABCD\right)$ nên $ SH$ là chiều cao của khối chóp $ S.ABCD$.\\
		Tính $ SH$.
		Xét tam giác $ ABC$ vuông tại $ B$ ta có $ AC=\sqrt{A{B^2}+B{C^2}}$$=\sqrt{a^2+a^2}$$=a\sqrt{2}$.\\
		Nhận thấy $ A{C^2}=S{A^2}+S{C^2}$ nên tam giác $ SAC$ vuông tại $ S$. Suy ra $ SH=\dfrac{AC}{2}=\dfrac{a}{\sqrt{2}}$.\\
		Diện tích đáy của khối chóp $ S.ABCD$ là $S_{ABCD}=a^2$.\\
		Vậy thể tích khối chóp $ S.ABCD$ là: $ V=\dfrac{1}{3}\cdot S_{ABCD}\cdot SH=\dfrac{1}{3}\cdot a^2\cdot \dfrac{a}{\sqrt{2}}=\dfrac{a^3\sqrt{2}}{6}$.}
\end{ex}

\begin{ex}%[2H1B3-2]
	(Mã 104 2017) Cho khối chóp tam giác đều $S.ABC$ có cạnh đáy bằng $ a$ và cạnh bên bằng $2a$. Tính thể tích $ V$ của khối chóp $S.ABC$
	\choice
	{$ V=\dfrac{\sqrt{11}{a^3}}{6}$}
	{$ V=\dfrac{\sqrt{11}{a^3}}{4}$}
	{$ V=\dfrac{\sqrt{13}{a^3}}{12}$}
	{$ V=\dfrac{\sqrt{11}{a^3}}{12}$}
	\loigiai{
		\begin{center}
			\begin{tikzpicture}[line join=round,line cap=round,line width=.6pt,font=\footnotesize,scale=1]
				\coordinate[label=left:$A$] (A) at (0,0);
				\coordinate[label=below left:$B$] (B) at (2,-2);
				\coordinate[label=right:$C$] (C) at (7,0);
				\coordinate[label=below right:$I$] (I) at ($(B)!.5!(C)$);
				\coordinate[] (J) at ($(B)!.5!(A)$);
				\coordinate[label=below:$O$] (O) at ($(A)!2/3!(I)$);
				\coordinate[label=above left:$S$] (S) at ($(O)+(90:4)$);
				\draw (A)--(B)--(C)--(S)--cycle (S)--(B);
				\draw[dashed] (I)--(A)--(C)--(J) (O)--(S);
				\fill (A)circle(1pt) (B)circle(1pt) (C)circle(1pt) (S)circle(1pt) (O)circle(1pt) (I)circle(1pt);
			\end{tikzpicture}
		\end{center}
		Do đáy là tam giác đều nên gọi $ I$ là trung điểm cạnh $BC$, khi đó $AI$ là đường cao của tam giác $ABC$.\\
		Theo định lý Pitago ta có $ AI=\sqrt{a^2-\dfrac{a^2}{4}}=\dfrac{a\sqrt{3}}{2}$ và $ AO=\dfrac{2}{3}AI=\dfrac{2a\sqrt{3}}{3.2}=\dfrac{a\sqrt{3}}{3}$.\\
		Trong tam giác $SOA$ vuông tại $ O$ ta có $ SO=\sqrt{4a^2-\dfrac{a^2}{3}}=\dfrac{\sqrt{11}a}{\sqrt{3}}$.\\
		Vậy thể tích khối chóp $S.ABC$ là $ V=\dfrac{1}{3}\cdot \dfrac{1}{2}a\cdot \dfrac{a\sqrt{3}}{2}\cdot \dfrac{\sqrt{11}a}{\sqrt{3}}=\dfrac{\sqrt{11}{a^3}}{12}$.}
\end{ex}
\begin{ex}%[2H1B3-2]
	(Chuyên Vĩnh Phúc 2019) Cho một hình chóp tam giác đều có cạnh đáy bằng $ a$, góc giữa cạnh bên và mặt phẳng đáy bằng $45^\circ.$ Thể tích khối chóp đó là
	\choice
	{$\dfrac{a^3\sqrt{3}}{12}$}
	{\True $\dfrac{a^3}{12}$}
	{$\dfrac{a^3}{36}$}
	{$\dfrac{a^3\sqrt{3}}{36}$}
	\loigiai{
		\begin{center}
			\begin{tikzpicture}[line join=round,line cap=round,line width=.6pt,font=\footnotesize,scale=1]
				\coordinate[label=left:$A$] (A) at (0,0);
				\coordinate[label=below left:$B$] (B) at (2,-2);
				\coordinate[label=right:$C$] (C) at (7,0);
				\coordinate[label=below right:$M$] (M) at ($(B)!.5!(C)$);
				\coordinate[label=below:$O$] (O) at ($(A)!2/3!(M)$);
				\coordinate[label=above left:$S$] (S) at ($(O)+(90:4)$);
				\draw (A)--(B)--(C)--(S)--cycle (S)--(B);
				\draw[dashed] (M)--(A)--(C) (O)--(S);
				\draw pic[draw,blue,fill=green!50,opacity=.5,angle radius=5mm,angle eccentricity=1.7,"$45^\circ$"] {angle = M--A--S};
				\fill (A)circle(1pt) (B)circle(1pt) (C)circle(1pt) (S)circle(1pt) (O)circle(1pt) (M)circle(1pt);
			\end{tikzpicture}
		\end{center}
		$\left(SA;\left(ABC\right)\right)=\widehat{SAO}=45^\circ $.\\
		$ SO=AO.\tan 45^\circ=\dfrac{a\sqrt{3}}{3}$.\\
		$ V=\dfrac{1}{3}\cdot SO \cdot S_{ABC}=\dfrac{1}{3}\cdot \dfrac{a\sqrt{3}}{3} \cdot \dfrac{a^2\sqrt{3}}{4}=\dfrac{a^3}{12}$.}
\end{ex}

\begin{ex}%[2H1B3-2]
	(Đề Tham Khảo 2019) Cho khối chóp tứ giác đều có tất cả các cạnh bằng $ 2\text{a}$. Thể tích của khối chóp đã cho bằng
	\choice
	{$\dfrac{2\sqrt{2}{a^3}}{3}$}
	{$\dfrac{8\text{a}^3}{3}$}
	{$\dfrac{8\sqrt{2}{a^3}}{3}$}
	{$\dfrac{4\sqrt{2}{a^3}}{3}$}
	\loigiai{
		\begin{center}
			\begin{tikzpicture}[scale=.8]
				\coordinate [label=left:$A$](A) at (3,2);
				\coordinate [label=below:$B$](B) at (0,0);
				\coordinate [label=below:$C$](C) at (6,0);
				\coordinate [label=above:$S$](S) at (4.5,6);
				\coordinate[label=right:$D$] (D) at ($(C)-(B)+(A)$);
				\coordinate [label=below:$I$] (I) at ($(A)! .5 ! (C)$);
				\foreach \diem in {A,B,C,D,S,I}	\fill (\diem)circle(1.5pt);		
				\draw[smooth](B)--(C) (C)--(D) (S)--(B) (S)--(C) (S)--(D);
				\draw[dashed](A)--(S) (A)--(B) (A)--(D);
				\draw[dashed](S)--(I);
				\draw[dashed](A)--(C);
				\draw[dashed](B)--(D);
			\end{tikzpicture}
		\end{center}
		Gọi hình chóp tứ giác đều có tất cả các cạnh bằng $ 2a$ là $ S.ABCD$ và $ I$ tâm của đáy.\\
		Ta có $ SA=SC=BA=BC=DA=DC$ \\
		Suy ra $\triangle SAC=\triangle BAC=\triangle DBC$ $\Rightarrow\triangle SAC;\triangle BAC;\triangle DAC$lần lượt vuông tại $ S,B,D$.\\
		$ I$ là trung điểm của $ AC$ suy ra $ SI=\dfrac{1}{2}AC=\dfrac{1}{2}2\text{a}\cdot \sqrt{2}=a\sqrt{2}$.\\
		$V_{S.ABCD}=\dfrac{1}{3}{S_{ABCD}}.SI=\dfrac{1}{3}{\left(2a\right)^2}\cdot a\sqrt{2}=\dfrac{4\sqrt{2}{a^3}}{3}$.}
\end{ex}

\begin{ex}%[2H1B3-2]
	(Mã 123 2017) Cho khối chóp tứ giác đều có cạnh đáy bằng $ a,$ cạnh bên gấp hai lần cạnh đáy. Tính thể tích $ V$ của khối chóp đã cho
	\choice
	{$ V=\dfrac{\sqrt{2}{a^3}}{2}$}
	{$ V=\dfrac{\sqrt{14}{a^3}}{2}$}
	{$ V=\dfrac{\sqrt{2}{a^3}}{6}$}
	{$ V=\dfrac{\sqrt{14}{a^3}}{6}$}
	\loigiai{
		\begin{center}
			\begin{tikzpicture}[scale=.8]
				\coordinate [label=left:$A$](A) at (3,2);
				\coordinate [label=below:$B$](B) at (0,0);
				\coordinate [label=below:$C$](C) at (6,0);
				\coordinate [label=above:$S$](S) at (4.5,6);
				\coordinate[label=right:$D$] (D) at ($(C)-(B)+(A)$);
				\coordinate [label=below:$I$] (I) at ($(A)! .5 ! (C)$);
				\foreach \diem in {A,B,C,D,S,I}	\fill (\diem)circle(1.5pt);		
				\draw[smooth](B)--(C) (C)--(D) (S)--(B) (S)--(C) (S)--(D);
				\draw[dashed](A)--(S) (A)--(B) (A)--(D);
				\draw[dashed](S)--(I);
				\draw[dashed](A)--(C);
				\draw[dashed](B)--(D);
			\end{tikzpicture}
		\end{center}
		Chiều cao của khối chóp: $SI=\sqrt{S{A^2}-A{I^2}}=\sqrt{4a^2-\left(\dfrac{a\sqrt{2}}{2}\right)^2}=\dfrac{a\sqrt{14}}{2}$.\\
		Thể tích khối chóp: $ V=\dfrac{1}{3}SI.S_{ABCD}=\dfrac{1}{3}.\dfrac{a\sqrt{14}}{2}{a^2}=\dfrac{\sqrt{14}{a^3}}{6}$.}
\end{ex}

\begin{ex}%[2H1B3-2]
	(Liên Trường Thpt Tp Vinh Nghệ An 2019) Cho khối chóp tứ giác đều có cạnh đáy bằng $ 2a$ cạnh bên bằng $ a\sqrt{5}$. Thể tích của khối chóp đã cho bằng
	\choice
	{$ 4\sqrt{5}{a^3}$}
	{$ 4\sqrt{3}{a^3}$}
	{$\dfrac{4\sqrt{5}{a^3}}{3}$}
	{\True $\dfrac{4\sqrt{3}{a^3}}{3}$}
	\loigiai{
		\begin{center}
			\begin{tikzpicture}[scale=.8]
				\coordinate [label=left:$A$](A) at (3,2);
				\coordinate [label=below:$B$](B) at (0,0);
				\coordinate [label=below:$C$](C) at (6,0);
				\coordinate [label=above:$S$](S) at (4.5,6);
				\coordinate[label=right:$D$] (D) at ($(C)-(B)+(A)$);
				\coordinate [label=below:$O$] (O) at ($(A)! .5 ! (C)$);
				\foreach \diem in {A,B,C,D,S,O}	\fill (\diem)circle(1.5pt);		
				\draw[smooth](B)--(C) (C)--(D) (S)--(B) (S)--(C) (S)--(D);
				\draw[dashed](A)--(S) (A)--(B) (A)--(D);
				\draw[dashed](S)--(O);
				\draw[dashed](A)--(C);
				\draw[dashed](B)--(D);
			\end{tikzpicture}
		\end{center}
		Ta có $S_{ABCD}=4a^2$ ; $SO=\sqrt{S{B^2}-O{B^2}}=\sqrt{5a^2-2a^2}=a\sqrt{3}$.\\
		Vậy $V_{S.ABCD}=\dfrac{1}{3}SO\cdot S_{ABCD}=\dfrac{a\sqrt{3}\cdot 4a^2}{3}=\dfrac{4\sqrt{3}{a^3}}{3}$.}
\end{ex}

\begin{ex}%[2H1B3-2]
	(THPT Lương Tài Số 2 2019) Cho hình chóp tứ giác đều S.ABCD có cạnh đáy bằng $ a\sqrt{6}$, góc giữa cạnh bên và mặt đáy bằng $60^\circ$. Tính thể tích V của khối chóp S.ABC
	\choice
	{$ V=9a^3$}
	{$ V=2a^3$}
	{$ V=3a^3$}
	{\True $ V=6a^3$}
	\loigiai{
		\begin{center}
			\begin{tikzpicture}[scale=.8]
				\coordinate [label=left:$A$](A) at (3,2);
				\coordinate [label=below:$B$](B) at (0,0);
				\coordinate [label=below:$C$](C) at (6,0);
				\coordinate [label=above:$S$](S) at (4.5,6);
				\coordinate[label=right:$D$] (D) at ($(C)-(B)+(A)$);
				\coordinate [label=below:$O$] (O) at ($(A)! .5 ! (C)$);
				\foreach \diem in {A,B,C,D,S,O}	\fill (\diem)circle(1.5pt);		
				\draw[smooth](B)--(C) (C)--(D) (S)--(B) (S)--(C) (S)--(D);
				\draw[dashed](A)--(S) (A)--(B) (A)--(D);
				\draw[dashed](S)--(O);
				\draw[dashed](A)--(C);
				\draw[dashed](B)--(D);
			\end{tikzpicture}
		\end{center}
		Diện tích đáy là: $S_{ABCD}=A{B^2}=\left(a\sqrt{6}\right)^2=6a^2.$\\
		Góc giữa cạnh bên $ SB$ và mặt đáy $ \left(ABCD\right)$ là $\left(\widehat{SD,\left(ABCD\right)}\right)=\widehat{SDO}\Rightarrow\widehat{SDO}=60^\circ$\\
		$ ABCD$ là hình vuông suy ra $ DO=\dfrac{1}{2}BD=\dfrac{1}{2}AB\sqrt{2}=\dfrac{1}{2}a\sqrt{6}\cdot \sqrt{2}=a\sqrt{3}.$\\
		Xét tam giác vuông $ SOD:\,SO=DO\cdot\tan\widehat{SDO}=a\sqrt{3}\cdot\tan{60^\circ}=3a$.\\
		Vậy $V_{S.ABCD}=\dfrac{1}{3}\cdot SO\cdot S_{ABCD}=\dfrac{1}{3}\cdot 3a\cdot 6a^2=6a^3$.}
\end{ex}

\begin{ex}%[2H1B3-2]
	(THPT Gia Lộc Hải Dương 2019) Cho hình chóp tam giác đều $S.ABC$ có độ dài cạnh đáy bằng $a$ , góc hợp bởi cạnh bên và mặt đáy bằng $60^{^\circ}$ . Thể tích của khối chóp đã cho bằng
	\choice
	{\True $\dfrac{a^3\sqrt{3}}{12}$}
	{$\dfrac{a^3\sqrt{3}}{3}$}
	{$\dfrac{a^3\sqrt{3}}{6}$}
	{$\dfrac{a^3\sqrt{3}}{4}$}
	\loigiai{
		\begin{center}
			\begin{tikzpicture}[scale=.8]
				\coordinate [label=left:$A$](A) at (0,0);
				\coordinate [label=above:$S$](S) at (11/3,5);
				\coordinate [label=below:$B$](C) at (8,0);
				\coordinate [label=right:$C$](B) at (3,-4);
				\coordinate [label=below:$M$](M) at ($(B)! .5 ! (C)$);
				\coordinate [label=below:$N$](N) at ($(B)! .5 ! (A)$);
				\coordinate [label=below:$H$](H) at ($(A)! 2/3 ! (M)$);
				\foreach \diem in {A,B,C,S,N,M,H}	\fill (\diem)circle(1.5pt);		
				\draw[smooth](S)--(A)--(S)--(B)--(C);
				\draw[dashed](A)--(C);
				\draw[dashed](A)--(M);
				\draw[dashed](C)--(N);
				\draw (S)--(C);
				\draw (A)--(B);
				\draw[dashed](S)--(H);
			\end{tikzpicture}
		\end{center}
		Gọi $H$ là tâm của tam giác đều $ABC$ .\\
		Khi đó $SH\perp\left(ABC\right)$ , $BH=\dfrac{a\sqrt{3}}{3}$ .\\
		Theo đề bài ta có: $\widehat{\left(SB,\,\left(ABC\right)\right)}=\widehat{SBH}=60^\circ $ .\\
		Xét $\triangle SBH$ vuông tại $H$ . Có $SH=BH.\tan 60^\circ=\dfrac{a\sqrt{3}}{3}\cdot\sqrt{3}=a$ .\\
		Thể tích $V_{S.ABC}=\dfrac{1}{3}SH\cdot S_{\triangle ABC}=\dfrac{1}{3}a\cdot \dfrac{a^2\sqrt{3}}{4}=\dfrac{a^3\sqrt{3}}{12}$ .}
\end{ex}

\begin{ex}%[2H1B3-2] 
	[Chuyên Nguyễn Du ĐăkLăk] Cho hình chóp đều $S . A B C D$ có chiều cao bằng $a \sqrt{2}$ và độ dài cạnh bên bằng $a \sqrt{6}$. Thể tích khối chóp $S . A B C D$ bằng
	\choice
	{$\dfrac{10 a^{3} \sqrt{3}}{3}$}
	{ $\dfrac{10 a^{3} \sqrt{2}}{3}$}
	{ $\dfrac{8 a^{3} \sqrt{3}}{3}$}
	{\True $\dfrac{8 a^{3} \sqrt{2}}{3}$}
	\loigiai{
		\begin{center}
			\begin{tikzpicture}[scale=1]
				\coordinate [label=left:$A$](A) at (3,2);
				\coordinate [label=below:$B$](B) at (0,0);
				\coordinate [label=below:$C$](C) at (7,0);
				\coordinate [label=above:$S$](S) at (5,7);
				\coordinate[label=right:$D$] (D) at ($(C)-(B)+(A)$);
				\coordinate [label=below:$O$](O) at ($(A)! .5 ! (C)$);
				\foreach \diem in {A,B,C,D,S,O}	\fill (\diem)circle(1.5pt);		
				\draw[smooth](B)--(C) (C)--(D) (S)--(B) (S)--(C) (S)--(D);
				\draw[dashed](A)--(S) (A)--(B) (A)--(D) (A)--(C) (D)--(B) (S)--(O);
			\end{tikzpicture}
		\end{center}
		Gọi $O=A C \cap B D$ thì $S O=a \sqrt{2}$.\\
		Tam giác $S O A$ vuông tại $O$ và $S A=a \sqrt{6}$ nên $O A=\sqrt{S A^{2}-S O^{2}}=2 a \Rightarrow A C=B D=4 a$.\\
		Thể tích khối chóp $S . A B C D$ là $$V=\dfrac{1}{3} \cdot S O \cdot \dfrac{A C \cdot B D}{2}=\dfrac{1}{3} \cdot a \sqrt{2} \cdot \dfrac{4 a \cdot 4 a}{2}=\dfrac{8 a^{3} \sqrt{2}}{3}.$$}
\end{ex}
\begin{ex}%[2H1B3-2] 
	[Thi thử Lômônôxốp - Hà Nội 2019] Xét khối chóp tam giác đều cạnh đáy bằng $a$, cạnh bên bằng 2 lần chiều cao tam giác đáy. Tính thể tích khối chóp
	\choice
	{ $\dfrac{a^{3} \sqrt{3}}{2}$}
	{  $\dfrac{a^{3} \sqrt{6}}{18}$}
	{\True  $\dfrac{a^{3} \sqrt{2}}{6}$}
	{  $\dfrac{a^{3} \sqrt{2}}{4}$}
	\loigiai{
		\begin{center}
			\begin{tikzpicture}[scale=.8]
				\coordinate [label=left:$A$](A) at (0,0);
				\coordinate [label=above:$S$](S) at (3.75,5);
				\coordinate [label=below:$B$](B) at (3,-3);
				\coordinate [label=right:$C$](C) at (8,1);
				\coordinate [label=below right:$M$](M) at ($(B)! .5 ! (C)$);
				\coordinate [label=below:$H$](H) at ($(A)! 2/3 ! (M)$);
				\foreach \diem in {A,B,C,S,M,H}	\fill (\diem)circle(1.5pt);		
				\draw[smooth](S)--(A)--(B)--(C)--(S)--(B);
				\draw[dashed](S)--(H) (C)--(A)--(M);
				%	\foreach \x/\y/\z in {B/C/D}{\draw pic[draw,angle radius=2mm]{right angle=\x--\y--\z};}
			\end{tikzpicture}
		\end{center}
		Gọi $H$ là trọng tâm tam giác $A B C \Rightarrow S H \perp(A B C)$.\\
		Gọi $M$ là trung điểm của cạnh $B C \Rightarrow A M \perp B C, A M=\dfrac{a \sqrt{3}}{2} \Rightarrow S A=a \sqrt{3}$.\\
		Xét tam giác $S A H$ vuông tại $H \Rightarrow S H=\sqrt{S A^{2}-A H^{2}}=\sqrt{(a \sqrt{3})^{2}-\left(\dfrac{2}{3} \cdot \dfrac{a \sqrt{3}}{2}\right)^{2}}=\dfrac{2 a \sqrt{6}}{3}$.\\
		Ta có $$V_{S . A B C}=\dfrac{1}{3} \cdot S_{\triangle A B C} \cdot S H=\dfrac{1}{3} \cdot \dfrac{a^{2} \sqrt{3}}{4} \cdot \dfrac{2 a \sqrt{6}}{3}=\dfrac{a^{3} \sqrt{2}}{6}.$$ 
}\end{ex}
\begin{ex}%[2H1B3-2]
	[SP Đồng Nai - 2019] Thể tích khối tứ diện đều có cạnh bằng 3 là
	\choice
	{\True $\dfrac{9 \sqrt{2}}{4}$}
	{  $2 \sqrt{2}$}
	{  $\dfrac{4 \sqrt{2}}{9}$}
	{  $\sqrt{2}$}
	\loigiai{
		\begin{center}
			\begin{tikzpicture}[scale=.8]
				\coordinate [label=left:$B$](B) at (0,0);
				\coordinate [label=above:$A$](A) at (3.75,5);
				\coordinate [label=below:$C$](C) at (3,-3);
				\coordinate [label=right:$D$](D) at (8,1);
				\coordinate [label=below right:$E$](E) at ($(C)! .5 ! (D)$);
				\coordinate [label=below:$H$](H) at ($(B)! 2/3 ! (E)$);
				\foreach \diem in {A,B,C,D,E,H}	\fill (\diem)circle(1.5pt);		
				\draw[smooth](A)--(B)--(C)--(D)--(A)--(C) (A)--(E);
				\draw[dashed](A)--(H) (D)--(B)--(E);
				%	\foreach \x/\y/\z in {B/C/D}{\draw pic[draw,angle radius=2mm]{right angle=\x--\y--\z};}
			\end{tikzpicture}
		\end{center}
		Có $\triangle B C D$ đều cạnh $3 \Rightarrow B E=\dfrac{3 \sqrt{3}}{2} \Rightarrow B H=\sqrt{3}$.\\
		$\triangle A B H$ vuông tại $H \Rightarrow A H=\sqrt{A B^{2}-B H^{2}}=\sqrt{3^{2}-(\sqrt{3})^{2}}=\sqrt{6}$.\\
		$S_{\triangle B C D}=\dfrac{1}{2} \cdot B E \cdot C D=\dfrac{1}{2} \cdot \dfrac{3 \sqrt{3}}{2} \cdot 3=\dfrac{9 \sqrt{3}}{4}$.\\
		Vậy
		$$ V_{A B C D}=\dfrac{1}{3} \cdot A H \cdot S_{\triangle B C D}=\dfrac{1}{3} \cdot \sqrt{6} \cdot \dfrac{9 \sqrt{3}}{4}=\dfrac{9 \sqrt{2}}{4}.$$
}\end{ex}

\begin{ex}%[2H1B3-2] 
	Cho khối chóp tứ giác đều có cạnh đáy bằng $a$, cạnh bên gấp hai lần cạnh đáy. Tính thể tích $V$ của khối chóp đã cho
	\choice
	{\True $V=\dfrac{\sqrt{14} a^{3}}{6}$}
	{  $V=\dfrac{\sqrt{14} a^{3}}{2}$}
	{  $V=\dfrac{\sqrt{2} a^{3}}{2}$}
	{  $V=\dfrac{\sqrt{2} a^{3}}{6}$}
	\loigiai{
		\begin{center}
			\begin{tikzpicture}[scale=1]
				\coordinate [label=left:$A$](A) at (3,2);
				\coordinate [label=below:$B$](B) at (0,0);
				\coordinate [label=below:$C$](C) at (7,0);
				\coordinate [label=above:$S$](S) at (5,7);
				\coordinate[label=right:$D$] (D) at ($(C)-(B)+(A)$);
				\coordinate [label=below:$O$](O) at ($(A)! .5 ! (C)$);
				\foreach \diem in {A,B,C,D,S,O}	\fill (\diem)circle(1.5pt);		
				\draw[smooth](B)--(C) (C)--(D) (S)--(B) (S)--(C) (S)--(D);
				\draw[dashed](A)--(S) (A)--(B) (A)--(D) (A)--(C) (D)--(B) (O)--(S);
			\end{tikzpicture}
		\end{center}
		Gọi $O$ là tâm hình vuông $A B C D$, ta có $S O \perp(A B C D)$.\\
		Trong tam giác $S O C$ vuông tại $O$ có $S O=\sqrt{S C^{2}-O C^{2}}=\sqrt{(2 a)^{2}-\left(\dfrac{a \sqrt{2}}{2}\right)^{2}}=\dfrac{a \sqrt{14}}{2}$.\\
		Thể tích khối chóp $S . A B C D$ là $$V=\dfrac{1}{3} \cdot S O \cdot S_{A B C D}=\dfrac{1}{3} \cdot \dfrac{a \sqrt{14}}{2} \cdot a^{2}=\dfrac{a^{3} \sqrt{14}}{6}.$$ 
}\end{ex}
\begin{ex}%[2H1B3-2]
	[Nguyễn Huệ - Ninh Bình - 2019]Cho hình chóp đều $S . A B C D$ có đáy $A B C D$ là hình vuông cạnh $a$. Cạnh bên $S A$ tạo với đáy góc $60^\circ$. Tính thể tích khối $S B C D$
	\choice
	{ $\dfrac{a^{3} \sqrt{6}}{6}$}
	{\True  $\dfrac{a^{3} \sqrt{6}}{12}$}
	{  $\dfrac{a^{3} \sqrt{3}}{6}$}
	{  $\dfrac{a^{3} \sqrt{3}}{12}$}
	\loigiai{
		\begin{center}
			\begin{tikzpicture}[scale=1]
				\coordinate [label=left:$A$](A) at (3,2);
				\coordinate [label=below:$B$](B) at (0,0);
				\coordinate [label=below:$C$](C) at (7,0);
				\coordinate [label=above:$S$](S) at (5,7);
				\coordinate[label=right:$D$] (D) at ($(C)-(B)+(A)$);
				\coordinate [label=below:$O$](O) at ($(A)! .5 ! (C)$);
				\foreach \diem in {A,B,C,D,S,O}	\fill (\diem)circle(1.5pt);		
				\draw[smooth](B)--(C) (C)--(D) (S)--(B) (S)--(C) (S)--(D);
				\draw[dashed](A)--(S) (A)--(B) (A)--(D) (A)--(C) (D)--(B) (O)--(S);
			\end{tikzpicture}
		\end{center}
		Gọi $O=A C \cap B D$. Do hình chóp $S . A B C D$ đều nên $S O \perp(A B C D)$ suy ra $O A$ là hình chiếu vuông góc của $S A$ trên $\mathrm{mp}(A B C D) \Rightarrow(S A,(A B C D))=(S A, O A)=\widehat{S A O}=60^\circ$.\\
		Ta có $S O=A O \cdot \tan 60^\circ=\dfrac{a \sqrt{2}}{2} \cdot \sqrt{3}=\dfrac{a \sqrt{6}}{2} ;$ $ S_{B C D}=\dfrac{a^{2}}{2}$.\\
		Từ đó, $V_{S B C D}=\dfrac{1}{3} S O \cdot S_{B C D}=\dfrac{1}{3} \cdot \dfrac{a \sqrt{6}}{2} \cdot \dfrac{a^{2}}{2}=\dfrac{a^{3} \sqrt{6}}{12}$.
}\end{ex}
\begin{ex}%[2H1B3-2]
	Cho khối chóp đều $S . A B C D$ có cạnh đáy là $a$, các mặt bên tạo với đáy một góc $60^\circ$. Tính thể tích khối chóp đó
	\choice
	{ $\dfrac{a^{3} \sqrt{3}}{2}$}
	{  $\dfrac{a^{3} \sqrt{3}}{12}$}
	{\True  $\dfrac{a^{3} \sqrt{3}}{6}$}
	{  $\dfrac{a^{3} \sqrt{3}}{3}$}
	\loigiai{
		\begin{center}
			\begin{tikzpicture}[scale=1]
				\coordinate [label=left:$A$](A) at (3,2);
				\coordinate [label=below:$D$](D) at (0,0);
				\coordinate [label=below:$C$](C) at (7,0);
				\coordinate [label=above:$S$](S) at (5,7);
				\coordinate[label=right:$B$] (B) at ($(C)-(D)+(A)$);
				\coordinate [label=below:$O$](O) at ($(A)! .5 ! (C)$);
				\coordinate [label=below right:$M$](M) at ($(B)! .5 ! (C)$);
				\foreach \diem in {A,B,C,D,S,O,M}	\fill (\diem)circle(1.5pt);	
				%Vẽ kí hiệu góc VUÔNG
				\foreach \x/\y/\z in {S/M/B}{\draw pic[draw,angle radius=3mm]{right angle=\x--\y--\z};}	
				\foreach \x/\y/\z in {O/M/C}{\draw pic[draw,angle radius=3mm]{right angle=\x--\y--\z};}	
				\draw pic[draw, angle radius=6mm]{angle=S--M--O};	
				\draw[smooth](D)--(C) (C)--(B) (S)--(D) (S)--(C) (S)--(B);
				\draw[dashed](A)--(S) (A)--(D) (A)--(B) (A)--(C) (D)--(B) (O)--(S)--(M)--(O);
			\end{tikzpicture}
		\end{center}
		Gọi $M$ là trung điểm $B C$, Góc giữa mặt bên $(S B C)$ và mặt phẳng $(A B C D)$ là góc $\widehat{S M O}=60^\circ$.\\
		Xét $\triangle S O M$ có $O M=\dfrac{a}{2}, \widehat{S M O}=60^\circ$ thì $S O=O M \cdot \tan \widehat{S M O}=\dfrac{a}{2} \cdot \sqrt{3}=\dfrac{a \sqrt{3}}{2}$.\\
		Nên $V_{S . A B C D}=\dfrac{1}{3} \cdot S O \cdot S_{A B C D}=\dfrac{a^{3} \sqrt{3}}{6}($ đvtt $)$. }\end{ex}
\begin{ex}%[2H1B3-2]
	Cho khối chóp tứ giác đều $S \cdot A B C D$ có cạnh đáy bằng $a$. Biết $\widehat{A S C}=90^\circ$, tính thể tích $V$ của khối chóp đó
	\choice
	{ $V=\dfrac{a^{3}}{3}$}
	{  $V=\dfrac{a^{3} \sqrt{2}}{3}$}
	{\True  $V=\dfrac{a^{3} \sqrt{2}}{6}$}
	{  $V=\dfrac{a^{3} \sqrt{2}}{12}$}
	\loigiai{
		\begin{center}
			\begin{tikzpicture}[scale=1]
				\coordinate [label=left:$A$](A) at (3,2);
				\coordinate [label=below:$B$](B) at (0,0);
				\coordinate [label=below:$C$](C) at (7,0);
				\coordinate [label=above:$S$](S) at (5,7);
				\coordinate[label=right:$D$] (D) at ($(C)-(B)+(A)$);
				\coordinate [label=below:$H$](H) at ($(A)! .5 ! (C)$);
				\foreach \diem in {A,B,C,D,S,H}	\fill (\diem)circle(1.5pt);		
				\draw[smooth](B)--(C) (C)--(D) (S)--(B) (S)--(C) (S)--(D);
				\draw[dashed](A)--(S) (A)--(B) (A)--(D) (A)--(C) (D)--(B) (H)--(S);
			\end{tikzpicture}
		\end{center}
		Ta có $S_{A B C D}=a^{2}$.\\
		Gọi $H$ là tâm của hình vuông $A B C D$. Tam giác $A S C$ là tam giác vuông, $H$ là trung điểm của $A C$ nên $S H=\dfrac{A C}{2}=\dfrac{a \sqrt{2}}{2}$.\\
		Vậy $V_{S \cdot A B C D}=\dfrac{1}{3} S_{A B C D} \cdot S H=\dfrac{1}{3} \cdot a^{2} \cdot \dfrac{a \sqrt{2}}{2}=\dfrac{a^{3} \sqrt{2}}{6}$.
}\end{ex}
\begin{ex}%[2H1B3-2]
	Cho hình chóp tứ giác đều $S . A B C D$ có cạnh đáy bằng $a$, góc giữa cạnh bên và mặt đáy bằng $60^\circ$. Thể tích khối chóp $S . A B C D$ là
	\choice
	{\True $\dfrac{a^{3} \sqrt{6}}{6}$}
	{  $\dfrac{a^{3} \sqrt{3}}{6}$}
	{  $\dfrac{a^{3} \sqrt{6}}{12}$}
	{  $\dfrac{a^{3} \sqrt{6}}{2}$}
	\loigiai{
		\begin{center}
			\begin{tikzpicture}[scale=1]
				\coordinate [label=left:$A$](A) at (3,2);
				\coordinate [label=below:$B$](B) at (0,0);
				\coordinate [label=below:$C$](C) at (7,0);
				\coordinate [label=above:$S$](S) at (5,7);
				\coordinate[label=right:$D$] (D) at ($(C)-(B)+(A)$);
				\coordinate [label=below:$O$](O) at ($(A)! .5 ! (C)$);
				\foreach \diem in {A,B,C,D,S,O}	\fill (\diem)circle(1.5pt);	
				\draw pic[draw, angle radius=8mm]{angle=S--D--B};		
				\draw[smooth](B)--(C) (C)--(D) (S)--(B) (S)--(C) (S)--(D);
				\draw[dashed](A)--(S) (A)--(B) (A)--(D) (A)--(C) (D)--(B) (O)--(S);
			\end{tikzpicture}
		\end{center}
		Gọi $O$ là tâm của đáy thì $S O \perp(A B C D)$. Suy ra $\widehat{S D B}=60^\circ$.\\
		$\triangle S D B$ đều nên $S O=\dfrac{D B \sqrt{3}}{2}=\dfrac{a \sqrt{6}}{2}$.\\
		Thể tích khối chóp $S . A B C D$ là $V=\dfrac{1}{3} S_{A B C D} \cdot S O=\dfrac{1}{3} a^{2} \cdot \dfrac{a \sqrt{6}}{2}=\dfrac{a^{3} \sqrt{6}}{6}$.
}\end{ex}
\begin{ex}%[2H1B3-2]
	[Truờng THPT Thăng Long 2019] Hình chóp tam giác đều $S . A B C$ có cạnh đáy là $a$ và mặt bên tạo với đáy góc $45^\circ$. Tính theo $a$ thể tích khối chóp $S . A B C$
	\choice
	{ $\dfrac{a^{3}}{8}$}
	{ \True $\dfrac{a^{3}}{24}$}
	{  $\dfrac{a^{3}}{12}$}
	{  $\dfrac{a^{3}}{4}$}
	\loigiai{
		\begin{center}
			\begin{tikzpicture}[scale=.8]
				\coordinate [label=left:$A$](A) at (0,0);
				\coordinate [label=above:$S$](S) at (3.75,5);
				\coordinate [label=below:$B$](B) at (3,-3);
				\coordinate [label=right:$C$](C) at (8,1);
				\coordinate [label=below right:$M$](M) at ($(B)! .5 ! (C)$);
				\coordinate (N) at ($(B)! .5 ! (A)$);
				\coordinate [label=below:$G$](G) at ($(A)! 2/3 ! (M)$);
				\foreach \diem in {A,B,C,S,M,G,N}	\fill (\diem)circle(1.5pt);		
				\draw[smooth](S)--(A)--(B)--(C)--(S)--(B);
				\draw[dashed](S)--(G) (C)--(A)--(M) (C)--(N);
				%	\foreach \x/\y/\z in {B/C/D}{\draw pic[draw,angle radius=2mm]{right angle=\x--\y--\z};}
			\end{tikzpicture}
		\end{center}
		Gọi $G$ là tâm của tam giác đều $A B C$ và $M$ là trung điểm $B C$. Theo giả thiết góc giữa mặt bên và đáy bằng $45^\circ$ suy ra $\widehat{S M G}=45^\circ$.\\
		Tam giác $A B C$ đều cạnh $a$ nên $A M=\dfrac{\sqrt{3}}{2} a$ và $G M=\dfrac{1}{3} A M=\dfrac{a \sqrt{3}}{6}$.\\
		Xét tam giác $S G M$ có $\tan \widehat{S M G}=\dfrac{S G}{G M} \Rightarrow \tan 45^\circ=\dfrac{S G}{G M} \Rightarrow S G=G M=\dfrac{a \sqrt{3}}{6}$.\\
		Vậy thể tích khối chóp $S . A B C$ là $$V_{S . A B C}=\dfrac{1}{3} S_{A B C} \cdot S G=\dfrac{1}{3} \cdot \dfrac{\sqrt{3}}{4} a^{2} \cdot \dfrac{a \sqrt{3}}{6}=\dfrac{a^{3}}{24}.$$
}\end{ex}
\begin{ex}%[2H1B3-2]
	[THPT Quỳnh Lưu - Nghệ An - 2019] Cho khối chóp có đáy hình thoi cạnh $a,$ $(a>0)$, các cạnh bên bằng nhau và cùng tạo với đáy góc $45^\circ$. Thể tích của khối chóp đã cho bằng
	\choice
	{\True $\dfrac{1}{3 \sqrt{2}} a^{3}$}
	{  $\sqrt{2} a^{3}$}
	{  $\dfrac{3 a^{3}}{\sqrt{2}}$}
	{  $\dfrac{1}{\sqrt{2}} a^{3}$}
	\loigiai{
		Ta có hình vẽ dưới đây.
		\begin{center}
			\begin{tikzpicture}[scale=1]
				\coordinate [label=left:$A$](A) at (3,2);
				\coordinate [label=below:$B$](B) at (0,0);
				\coordinate [label=below:$C$](C) at (7,0);
				\coordinate [label=above:$S$](S) at (5,7);
				\coordinate[label=right:$D$] (D) at ($(C)-(B)+(A)$);
				\coordinate [label=below:$O$](O) at ($(A)! .5 ! (C)$);
				\foreach \diem in {A,B,C,D,S,O}	\fill (\diem)circle(1.5pt);	
				\draw[smooth](B)--(C) (C)--(D) (S)--(B) (S)--(C) (S)--(D);
				\draw[dashed](A)--(S) (A)--(B) (A)--(D) (A)--(C) (D)--(B) (O)--(S);
			\end{tikzpicture}
		\end{center}
		Xét khối chóp trên ta thấy hình chiếu vuông góc của $S$ lên mặt phẳng đáy trùng với tâm của hình thoi $A B C D$.\\
		Mặt khác $S A=S B=S C=S D$ và góc hợp bởi các cạnh bên bằng $45^\circ$ nên ta có các tam giác vuông cân tại $O$ bằng nhau: $\triangle S O A=\triangle S O B=\triangle S O C=\triangle S O D$.\\
		Suy ra hình thoi $A B C D$ là một hình vuông diện tích đáy bằng $S_{A B C D}=a^{2}$.\\
		Chiều cao của hình chóp trên là $S O=O D=\dfrac{1}{2} B D=\dfrac{a \sqrt{2}}{2}$.\\
		Suy ra thể tích khối chóp bằng $$V_{S \cdot A B C D}=\dfrac{1}{3} \cdot S O \cdot S_{A B C D}=\dfrac{1}{3} \cdot \dfrac{a \sqrt{2}}{2} \cdot a^{2}=\dfrac{a^{3}}{3 \sqrt{2}}.$$
}\end{ex}
\begin{ex}%[2H1B3-2]
	[Chuyên Quang Trung- Bình Phước 2019] Tính thể tích khối tứ diện đều có tất cả các cạnh bằng $a$ 
	\choice
	{ $a^{3}$}
	{\True  $\dfrac{\sqrt{2}}{12} a^{3}$}
	{  $\dfrac{1}{12} a^{3}$}
	{  $6 a^{3}$}
	\loigiai{
		\begin{center}
			\begin{tikzpicture}[scale=.8]
				\coordinate [label=left:$B$](B) at (0,0);
				\coordinate [label=above:$A$](A) at (3.75,5);
				\coordinate [label=below:$C$](C) at (3,-3);
				\coordinate [label=right:$D$](D) at (8,1);
				\coordinate [label=below right:$M$](M) at ($(C)! .5 ! (D)$);
				\coordinate [label=below:$H$](H) at ($(B)! 2/3 ! (M)$);
				\foreach \diem in {A,B,C,D,M,H}	\fill (\diem)circle(1.5pt);		
				\draw[smooth](A)--(B)--(C)--(D)--(A)--(C) (A)--(M);
				\draw[dashed](A)--(H) (D)--(B)--(M);
				\foreach \x/\y/\z in {A/H/B}{\draw pic[draw,angle radius=3mm]{right angle=\x--\y--\z};}
				\foreach \x/\y/\z in {D/M/B}{\draw pic[draw,angle radius=3mm]{right angle=\x--\y--\z};}
			\end{tikzpicture}
		\end{center}
		Gọi $M$ là trung điểm của $C D$. Ta có $B M=\dfrac{a \sqrt{3}}{2} \Rightarrow B H=\dfrac{a \sqrt{3}}{3} \cdot A H=\sqrt{A B^{2}-B H^{2}}=\dfrac{a \sqrt{6}}{3}$.\\ Do đáy $B C D$ là tam giác đều cạnh $a \Rightarrow S_{B C D}=\dfrac{a^{2} \sqrt{3}}{4}$.\\
		Vậy thể tích tứ diện đều là $$V_{A B C D}=\dfrac{1}{3} \dfrac{a^{2} \sqrt{3}}{4} \cdot \dfrac{a \sqrt{6}}{3}=\dfrac{\sqrt{2}}{12} a^{3}.$$
}\end{ex}
\begin{ex}%[2H1B3-2]
	[Hậu Lộc 2 - Thanh Hóa - 2019] Cho hình chóp tứ giác đều có cạnh đáy bằng $a$, góc giữa cạnh bên và mặt đáy bằng $60^\circ$. Thể tích khối chóp là
	\choice
	{\True $\dfrac{a^{3} \sqrt{6}}{6}$}
	{  $\dfrac{a^{3} \sqrt{6}}{2}$}
	{  $\dfrac{a^{3} \sqrt{3}}{6}$}
	{  $\dfrac{a^{3} \sqrt{6}}{3}$}
	\loigiai{
		\begin{center}
			\begin{tikzpicture}[scale=1]
				\coordinate [label=left:$A$](A) at (3,2);
				\coordinate [label=below:$B$](B) at (0,0);
				\coordinate [label=below:$C$](C) at (7,0);
				\coordinate [label=above:$S$](S) at (5,7);
				\coordinate[label=right:$D$] (D) at ($(C)-(B)+(A)$);
				\coordinate [label=below:$O$](O) at ($(A)! .5 ! (C)$);
				\foreach \diem in {A,B,C,D,S,O}	\fill (\diem)circle(1.5pt);	
				\draw[smooth](B)--(C) (C)--(D) (S)--(B) (S)--(C) (S)--(D);
				\draw[dashed](A)--(S) (A)--(B) (A)--(D) (A)--(C) (D)--(B) (O)--(S);
			\end{tikzpicture}
		\end{center}
		Giả sử hình chóp tứ giác đều là $S . A B C D$. Gọi $O$ là giao điểm của $B D$ và $A C$.\\
		Ta có $S O \perp(A B C D), \widehat{S A O}=60^\circ, A C=a \sqrt{2} \Rightarrow O A=\dfrac{a \sqrt{2}}{2}$.\\
		Khi đó $S O=A O \cdot \tan \widehat{S A O}=\dfrac{a \sqrt{6}}{2}, S_{A B C D}=a^{2}$.
		\\
		Thể tích khối chóp là $$V=\dfrac{1}{3} S O \cdot S_{A B C D}=\dfrac{a^{3} \sqrt{6}}{6}.$$
}\end{ex}
\begin{ex}%[2H1B3-2]
	Cho hình chóp tam giác đều $S . A B C$ có cạnh đáy bằng $2 a$, cạnh bên tạo với đáy một góc $60^\circ$. Thể tích khối chóp $S . A B C$ là 
	\choice
	{\True $\dfrac{2 a^{3} \sqrt{3}}{3}$}
	{  $\dfrac{a^{3} \sqrt{3}}{3}$}
	{  $\dfrac{a^{3} \sqrt{3}}{4}$}
	{  $a^{3} \sqrt{3}$}
	\loigiai{
		\begin{center}
			\begin{tikzpicture}[scale=.8]
				\coordinate [label=left:$A$](A) at (0,0);
				\coordinate [label=above:$S$](S) at (3.75,5);
				\coordinate [label=below:$B$](B) at (3,-3);
				\coordinate [label=right:$C$](C) at (8,1);
				\coordinate [label=below right:$I$](I) at ($(B)! .5 ! (C)$);
				\coordinate (N) at ($(B)! .5 ! (A)$);
				\coordinate [label=below:$O$](O) at ($(A)! 2/3 ! (I)$);
				\foreach \diem in {A,B,C,S,I,O,N}	\fill (\diem)circle(1.5pt);		
				\draw[smooth](S)--(A)--(B)--(C)--(S)--(B);
				\draw[dashed](S)--(O) (C)--(A)--(I) (C)--(N);
				%	\foreach \x/\y/\z in {B/C/D}{\draw pic[draw,angle radius=2mm]{right angle=\x--\y--\z};}
			\end{tikzpicture}
		\end{center}
		Gọi $O$ là tâm đường tròn ngoại tiếp $\triangle A B C$ thì $S O \perp(A B C)$. Suy ra $\widehat{S A O}=60^\circ$.\\
		$A O=\dfrac{2}{3} \cdot 2 a \cdot \dfrac{\sqrt{3}}{2}=\dfrac{2 a \sqrt{3}}{3}, S H=A O \cdot \tan 60^\circ=2 a$.\\
		Diện tích $\triangle A B C$ là $S_{A B C}=\dfrac{(2 a)^{2} \sqrt{3}}{4}=a^{2} \sqrt{3}$.\\
		Thể tích khối chóp $S \cdot A B C$ là $$V=\dfrac{1}{3} S_{A B C} \cdot S O=\dfrac{2 a^{3} \sqrt{3}}{3}.$$
}\end{ex}
\begin{ex}%[2H1B3-2]
	[SGD Điện Biên - 2019] Cho hình chóp tứ giác đều $S . A B C D$ có cạnh đáy bằng $2 a$, cạnh bên bằng $3 a$. Tính thể tích $V$ của khối chóp đã cho
	\choice
	{ $V=4 \sqrt{7} a^{3}$}
	{  $V=\dfrac{4 \sqrt{7} a^{3}}{9}$}
	{  $V=\dfrac{4 a^{3}}{3}$}
	{\True  $V=\dfrac{4 \sqrt{7} a^{3}}{3}$}
	\loigiai{
		\begin{center}
			\begin{tikzpicture}[scale=1]
				\coordinate [label=left:$A$](A) at (3,2);
				\coordinate [label=below:$B$](B) at (0,0);
				\coordinate [label=below:$C$](C) at (7,0);
				\coordinate [label=above:$S$](S) at (5,7);
				\coordinate[label=right:$D$] (D) at ($(C)-(B)+(A)$);
				\coordinate [label=below:$O$](O) at ($(A)! .5 ! (C)$);
				\foreach \diem in {A,B,C,D,S,O}	\fill (\diem)circle(1.5pt);	
				\draw[smooth](B)--(C) (C)--(D) (S)--(B) (S)--(C) (S)--(D);
				\draw[dashed](A)--(S) (A)--(B) (A)--(D) (A)--(C) (D)--(B) (O)--(S);
			\end{tikzpicture}
		\end{center}
		Diện tích đáy $S_{A B C D}=(2 a)^{2}=4 a^{2}$.\\
		$S . A B C D$ là hình chóp tứ giác đều nên $S O \perp(A B C D)$.\\
		$h=S O=\sqrt{S A^{2}-A O^{2}}=\sqrt{9 a^{2}-2 a^{2}}=a \sqrt{7}$.\\
		Vậy $V_{S . A B C D}=\dfrac{1}{3} S h=\dfrac{4 a^{3} \sqrt{7}}{3}$.
}\end{ex}
\begin{ex}%[2H1B3-2]
	[Nguyễn Huệ - Ninh Bình - 2019]Kim tự tháp Kê - ốp ở Ai Cập được xây dựng vào khoảng 2500 năm trước Công nguyên. Kim tự tháp này là một khối chóp tứ giác đều có chiều cao là $147 \mathrm{~m}$, cạnh đáy là $230 \mathrm{~m}$. Thể tích của nó là
	\choice
	{\True $2592100 \mathrm{~m}^{3}$}
	{  $2952100 \mathrm{~m}^{3}$}
	{  $2529100 \mathrm{~m}^{3}$}
	{  $2591200 \mathrm{~m}^{3}$}
	\loigiai{
		\begin{center}
			\begin{tikzpicture}[scale=1]
				\coordinate [label=left:$A$](A) at (3,2);
				\coordinate [label=below:$B$](B) at (0,0);
				\coordinate [label=below:$C$](C) at (7,0);
				\coordinate [label=above:$S$](S) at (5,7);
				\coordinate[label=right:$D$] (D) at ($(C)-(B)+(A)$);
				\coordinate [label=below:$H$](H) at ($(A)! .5 ! (C)$);
				\foreach \diem in {A,B,C,D,S,H}	\fill (\diem)circle(1.5pt);	
				\draw[smooth](B)--(C) (C)--(D) (S)--(B) (S)--(C) (S)--(D);
				\draw[dashed](A)--(S) (A)--(B) (A)--(D) (A)--(C) (D)--(B) (H)--(S);
				\draw (4.3,3)node {147 m};
				\draw (4,-.25)node {230 m};
			\end{tikzpicture}
		\end{center}
		Gọi khối chóp tứ giác đều là $S . A B C D$ có đáy là hình vuông cạnh $230 \mathrm{~m}$; chiều cao $S H=147 \mathrm{~m}$. \\Thể tích của nó là $$V_{S \cdot A B C D}=\dfrac{1}{3} \cdot S_{A B C D} \cdot S H=\dfrac{1}{3} \cdot\left(230^{2}\right) \cdot 147=2592100.$$
		Vậy thể tích Kim tự tháp là $2592100 \mathrm{~m}^{3}$.
	}
\end{ex}
\Closesolutionfile{ans}
\indapan{10}{ans/CD10/Muc_5_6}

