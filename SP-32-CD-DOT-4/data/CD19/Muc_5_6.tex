\Opensolutionfile{ans}[ans/CD19/Muc_5_6]
\setcounter{ex}{0}
\setcounter{dang}{0}
\section{Mức độ 5,6 điểm}
\begin{dang}
	{Phương trình logarit}
\end{dang}
\begin{ex}           
	[Mã 101 - 2021 Lần 1]%Câu 1
	Nghiệm của phương trình $\log_3\left(5x\right)=2$ là 
	\choice
	{$x=\dfrac{8}{5}$}
	{$x=9$}
	{\True $x=\dfrac{9}{5}$}
	{$x=8$}
	\loigiai{
		Tập xác định: $\mathscr{D}=\left(0;+\infty\right)$.\\
		Ta có $\log_3\left(5x\right)=2\Leftrightarrow 5x=3^2\Leftrightarrow x=\dfrac{9}{5}$.
	}
\end{ex}
\begin{ex}
	[Mã 104 - 2021 Lần 1]%Câu 2
	Nghiệm của phương trình $\log_2\left(5x\right)=3$ là:
	\choice
	{\True $x=\dfrac{8}{5}$}
	{$x=\dfrac{9}{5}$}
	{$x=8$}
	{$x=9$}
	\loigiai{
		Điều kiện $x>0$\\
		$\log_2\left(5x\right)=3$ $\Leftrightarrow 5x=2^3$ $\Leftrightarrow 5x=8$ $\Leftrightarrow x=\dfrac{8}{5}$ (nhận).
	}
\end{ex}
\begin{ex}
	[Đề Minh Họa 2021]%Câu 3
	Nghiệm của phương trình $\log_2\left(3x\right)=3$ là:
	\choice
	{$x=3$}
	{$x=2$}
	{\True $x=\dfrac{8}{3}$}
	{$x=\dfrac{1}{2}$}
	\loigiai{
		Ta có $\log_2\left(3x\right)=3\Leftrightarrow 3x=8\Leftrightarrow x=\dfrac{8}{3}$.
	}
\end{ex}
\begin{ex}
	[Mã 102 - 2021 Lần 1]%Câu 4
	Nghiệm của phương trình $\log_5\left(3x\right)=2$ là
	\choice
	{$x=25$}
	{$x=\dfrac{32}{3}$}
	{$x=32$}
	{\True $x=\dfrac{25}{3}$}
	\loigiai{
		Điều kiện: $x>0$.\\
		Với điều kiện phương trình đã cho tương đương $3x=5^2=25$$\Leftrightarrow x=\dfrac{25}{3}$.
	}
\end{ex}
\begin{ex}
	[Mã 103 - 2021 - Lần 1]%Câu 5
	Nghiệm của phương trình $\log_3\left(2x\right)=2$ là
	\choice
	{\True $x=\dfrac{9}{2}$}
	{$x=9$}
	{$x=4$}
	{$x=8$}
	\loigiai{
		$\log_3\left(2x\right)=2\Leftrightarrow 2x=9\Leftrightarrow x=\dfrac{9}{2}$.
	}
\end{ex}
\begin{ex}
	[Đề Minh Họa 2020 Lần 1]%Câu 6
	Nghiệm của phương trình $\log_3\left(2x-1\right)=2$ là
	\choice
	{$x=3$}
	{\True $x=5$}
	{$x=\dfrac{9}{2}$}
	{$x=\dfrac{7}{2}$}
	\loigiai{
		Điều kiện: $2x-1>0\Leftrightarrow x>\dfrac{1}{2}$\\
		Ta có $\log_3\left(2x-1\right)=2\Leftrightarrow\left\{\begin{aligned}
			&x>\dfrac{1}{2}\\ 
			&2x-1=3^2\\ 
		\end{aligned}\right.\Leftrightarrow\left\{\begin{aligned}
			&x>\dfrac{1}{2}\\ 
			&x=5\\ 
		\end{aligned}\right.\Leftrightarrow x=5$.\\
		Vậy phương trình có nghiệm $x=5$.
	}
\end{ex}
\begin{ex}
	[Mã 101-2020 Lần 1]%Câu 7
	Nghiệm của phương trình $\log_3\left(x-1\right)=2$ là
	\choice
	{$x=8$}
	{$x=9$}
	{$x=7$}
	{\True $x=10$}
	\loigiai{
		Tập xác định: $\mathscr{D}=\left(1;+\infty\right)$\\
		$\log_3\left(x-1\right)=2\Leftrightarrow x-1=3^2\Leftrightarrow x=10$.
	}
\end{ex}
\begin{ex}
	[Mã 102 - 2020 Lần 1]%Câu 8
	Nghiệm của phương trình $\log _2(x-1)=3$ là
	\choice
	{$x=10$}
	{$x=8$}
	{\True $x=9$}
	{$x=7$}
	\loigiai{
		Ta có $\log_2(x-1)=3\Leftrightarrow\left\{\begin{aligned}
			&x-1>0 \\ 
			&x-1=2^3\\
		\end{aligned}\Leftrightarrow\left\{\begin{aligned}
			&x>1 \\ 
			&x=9\\
		\end{aligned}\Leftrightarrow x=9\right.\right.$.
	}
\end{ex}
\begin{ex}
	[Mã 103 - 2020 Lần 1]%Câu 9
	Nghiệm của phương trình $\log_2\left(x-2\right)=3$ là:
	\choice
	{$x=6$}
	{$x=8$}
	{$x=11$}
	{\True $x=10$}
	\loigiai{
		Điều kiện: $x-2>0\Leftrightarrow x>2$.\\
		$\log_2\left(x-2\right)=3\Leftrightarrow x-2=8\Leftrightarrow x=10$(thỏa).\\
		Vậy phương trình có nghiệm $x=10$.
	}
\end{ex}
\begin{ex}
	[Mã 104 - 2020 Lần 1]%Câu 10
	Nghiệm của phương trình $\log_3^{}\left(x-2\right)=2$ là
	\choice
	{\True $x=11$}
	{$x=10$}
	{$x=7$}
	{$8$}
	\loigiai{
		Điều kiện: $x>2$\\
		Phương trình tương đương với $x-2=3^2\Leftrightarrow x=11$.
	}
\end{ex}
\begin{ex}
	[Mã 102 - 2020 Lần 2]%Câu 11
	Nghiệm của phương trình $\log_2\left(x+9\right)=5$ là
	\choice
	{$x=41$}
	{\True $x=23$}
	{$x=1$}
	{$x=16$}
	\loigiai{
		ĐK: $ x>-9$\\
		Ta có $\log_2\left(x+9\right)=5\Leftrightarrow x+9=2^5$$\Leftrightarrow x=23$.
	}
\end{ex}
\begin{ex}
	[Mã 103 - 2020 Lần 2]%Câu 12
	Nghiệm của phương trình $\log_2\left(x+6\right)=5$ là:
	\choice
	{$x=4$}
	{$x=19$}
	{$x=38$}
	{\True $x=26$}
	\loigiai{
		Điều kiện $x+6>0\Leftrightarrow x>-6$\\
		Ta có $\log_2\left(x+6\right)=5\Leftrightarrow{\log_2}\left(x+6\right)=\log_22^5\Leftrightarrow\left(x+6\right)=32$$\Leftrightarrow x=32-6z_2=\dfrac{1+i\sqrt{7}}{2}$\\
		Vậy nghiệm của phương trình: $x=26$.
	}
\end{ex}
\begin{ex}
	[Mã 104 - 2020 Lần 2]%Câu 13
	Nghiệm của phương trình $\log_2\left(x+7\right)=5$ là
	\choice
	{$x=18$}
	{\True $x=25$}
	{$x=39$}
	{$x=3$}
	\loigiai{
		$\log_2\left(x+7\right)=5\Leftrightarrow x+7=2^5\Leftrightarrow x=25$.
	}
\end{ex}
\begin{ex}
	[Mã 101 - 2020 Lần 2]%Câu 14
	Nghiệm của phương trình $\log_2(x+8)=5$ bằng
	\choice
	{$x=17$}
	{\True $x=24$}
	{$x=2$}
	{$x=40$}
	\loigiai{
		Ta có $\log_2(x+8)=5\Leftrightarrow x+8=2^5\Leftrightarrow x=24$.
	}
\end{ex}
\begin{ex}
	[Đề Tham Khảo 2019]%Câu 15
	Tập nghiệm của phương trình $\log_2\left(x^2-x+2\right)=1$ là
	\choice
	{$\left\{0\right\}$}
	{\True $\left\{0;1\right\}$}
	{$\left\{-1;0\right\}$}
	{$\left\{1\right\}$}
	\loigiai{
		$\log_2\left(x^2-x+2\right)=1\Leftrightarrow{x^2}-x+2=2\Leftrightarrow\left[\begin{aligned}
			&x=0\\ 
			&x=1\\ 
		\end{aligned}\right.$.
	}
\end{ex}
\begin{ex}
	[Đề Minh Họa 2017]%Câu 16
	Giải phương trình $\log_4(x-1)=3$.
	\choice
	{\True $x=65$}
	{$x=80$}
	{$x=82$}
	{$x=63$}
	\loigiai{
		Điều kiện: $\Leftrightarrow x-1>0\Leftrightarrow x>1$\\
		Phương trình $\log_4\left(x-1\right)=3$ $\Leftrightarrow x-1=4^3\Leftrightarrow x=65$.
	}
\end{ex}
\begin{ex}
	[Mã 110 2017]%Câu 17
	Tìm nghiệm của phương trình $\log_2\left(1-x\right)=2$.
	\choice
	{$x=5$}
	{\True $x=-3$}
	{$x=-4$}
	{$x=3$}
	\loigiai{
		Ta có $\log_2\left(1-x\right)=2$$\Leftrightarrow 1-x=4$$\Leftrightarrow x=-3$.
	}
\end{ex}
\begin{ex}
	[Mã 102 2018]%Câu 18
	Tập nghiệm của phương trình $\log_2\left(x^2-1\right)=3$ là
	\choice
	{$\left\{-\sqrt{10};\sqrt{10}\right\}$}
	{\True $\left\{-3;3\right\}$}
	{$\left\{-3\right\}$}
	{$\left\{3\right\}$}
	\loigiai{
		$\log_2\left(x^2-1\right)=3$$\Leftrightarrow{x^2}-1=8$$\Leftrightarrow{x^2}=9$$\Leftrightarrow x=\pm 3$.
	}
\end{ex}
\begin{ex}
	[Mã 104 2017]%Câu 19
	Tìm nghiệm của phương trình $\log_2\left(x-5\right)=4$.
	\choice
	{$x=11$}
	{$x=13$}
	{\True $x=21$}
	{$x=3$}
	\loigiai{
		Điều kiện: $x-5>0\Leftrightarrow x>5$\\
		Khi đó $\log_2\left(x-5\right)=4$ $\Leftrightarrow x-5=16\Leftrightarrow x=21$.
	}
\end{ex}
\begin{ex}
	[Mã 103 2018]%Câu 20
	Tập nghiệm của phương trình $\log_3(x^2-7)=2$ là
	\choice
	{$\left\{ 4\right\}$}
	{$\left\{-4\right\}$}
	{$\left\{-\sqrt{15};\sqrt{15}\right\}$}
	{\True $\{-4;4\}$}
	\loigiai{
		$\log_3(x^2-7)=2$$\Leftrightarrow{x^2}-7=9$$\Leftrightarrow\left[\begin{aligned}
			& x=4\\ 
			& x=-4\\ 
		\end{aligned}\right.$.
	}
\end{ex}
\begin{ex}
	[Mã 105 2017]%Câu 21      
	Tìm nghiệm của phương trình $\log_{25}\left(x+1\right)=\dfrac{1}{2}$.
	\choice
	{$x=6$}
	{\True $x=4$}
	{$x=\dfrac{23}{2}$}
	{$x=-6$}
	\loigiai{
		Điều kiện: $x>-1$\\
		Xét phương trình $\log_{25}\left(x+1\right)=\dfrac{1}{2}\Leftrightarrow{\log_5}\left(x+1\right)=1$ $\Leftrightarrow x+1=5\Leftrightarrow x=4$.
	}
\end{ex}
\begin{ex}
	[Chuyên Vĩnh Phúc 2019]%Câu 22
	Phương trình $\log_3\left(3x-2\right)=3$ có nghiệm là
	\choice
	{$x=\dfrac{25}{3}$}
	{$x=87$}
	{\True $x=\dfrac{29}{3}$}
	{$x=\dfrac{11}{3}$}
	\loigiai{
		Ta có $\log_3\left(3x-2\right)=3\Leftrightarrow 3x-2=3^3\Leftrightarrow 3x=29\Leftrightarrow x=\dfrac{29}{3}$.\\
		Vậy phương trình đã cho có nghiệm là $x=\dfrac{29}{3}$.
	}
\end{ex}
\begin{ex}
	[THPT Ba Đình 2019]%Câu 23
	Tập nghiệm của phương trình $\log_3\left(x^2-x+3\right)=1$ là
	\choice
	{$\left\{1\right\}$}
	{\True $\left\{0;1\right\}$}
	{$\left\{-1;0\right\}$}
	{$\left\{0\right\}$}
	\loigiai{
		Điều kiện xác định: $x^2-x+3>0 \Leftrightarrow x \in \mathbb{R}$.\\
		Ta có $\log _3\left(x^2-x+3\right)=1 \Leftrightarrow x^2-x+3=3 \Leftrightarrow\left[\begin{aligned}
			&x=0 \\ 
			&x=1\\
		\end{aligned}\right.$.\\
		Vậy tập nghiệm của phương trình là $S=\left\{0;1\right\}$.
	}
\end{ex}
\begin{ex}
	[THPT Cù Huy Cận 2019]%Câu 24
	Tập nghiệm của phương trình $\log_3\left(x^2+x+3\right)=1$ là:
	\choice
	{\True $\left\{-1;0\right\}$}
	{$\left\{0;1\right\}$}
	{$\left\{0\right\}$}
	{$\left\{-1\right\}$}
	\loigiai{
		$\log_3\left(x^2+x+3\right)=1\Leftrightarrow{x^2}+x+3=3\Leftrightarrow{x^2}+x=0\Leftrightarrow\left[\begin{aligned}
			&x=0\\ 
			&x=-1\\ 
		\end{aligned}\right.$.
	}
\end{ex}
\begin{ex}
	[Chuyên Vĩnh Phúc 2019]%Câu 25
	Phương trình $\log_3\left(3x-2\right)=3$ có nghiệm là:
	\choice
	{$x=\dfrac{25}{3}$}
	{$87$}
	{\True $x=\dfrac{29}{3}$}
	{$x=\dfrac{11}{3}$}
	\loigiai{
		Điều kiện: $x>\dfrac{2}{3}$.\\
		Phương trình tương đương $3x-2=3^3$$\Leftrightarrow x=\dfrac{29}{3}$(nhận).\\
		Vậy $S=\left\{\dfrac{29}{3}\right\}$.
	}
\end{ex}           
\begin{ex}
	[Chuyên Phan Bội Châu Nghệ An 2019]%Câu 26
	Tập nghiệm của phương trình $\log\left(x^2-2x+2\right)=1$ là
	\choice
	{$\varnothing $}
	{\True $\{-2;4\}$}
	{$\{4\}$}
	{$\{-2\}$}
	\loigiai{
		Ta có $\log\left(x^2-2x+2\right)=1\Leftrightarrow{x^2}-2x+2=10\Leftrightarrow{x^2}-2x-8=0\Leftrightarrow\left[\begin{aligned}
			&x=-2\\ 
			&x=4\\ 
		\end{aligned}\right.$.
	}
\end{ex}
\begin{ex}
	[Chuyên Lương Thế Vinh Đồng Nai 2019]%Câu 27
	Cho phương trình $\log_2(2x-1)^2=2\log_2(x-2).$Số nghiệm thực của phương trình là
	\choice
	{$1$}
	{\True $0$}
	{$3$}
	{$2$}
	\loigiai{
		Điều kiện: $x>2$.\\
		Phương trình đã cho tương đương với: $2\log_2(2x-1)=2\log_2(x-2)$\\
		$\Leftrightarrow 2x-1=x-2\Leftrightarrow x=-1$\\
		Nghiệm này không thỏa mãn điều kiện của phương trình nên phương trình đã cho vô nghiệm.
	}
\end{ex}
\begin{ex}
	[Chuyên Sơn La 2019]%Câu 28
	Tập nghiệm của phương trình $\log_3\left(x^2+2x\right)=1$ là
	\choice
	{\True $\left\{1;-3\right\}$}
	{$\left\{1;3\right\}$}
	{$\left\{0\right\}$}
	{$\left\{-3\right\}$}
	\loigiai{
		Phương trình $\log_3\left(x^2+2\text{x}\right)=1\Leftrightarrow{x^2}+2\text{x}=3^1\Leftrightarrow{x^2}+2\text{x}-3=0\Leftrightarrow\left[\begin{aligned}
			&x=1\\ 
			&x=-3\\ 
		\end{aligned}\right.$ .\\
		Tập nghiệm của phương trình là $\left\{ 1;-3\right\}$.
	}
\end{ex}
\begin{ex}
	[THPT Quỳnh Lưu 3 Nghệ An 2019]%Câu 29
	Tập hợp các số thực $ m$ để phương trình $\log_2x=m$ có nghiệm thực là
	\choice
	{$\left[0;+\infty\right)$}
	{$\left(-\infty ;0\right)$}
	{\True $\mathbb{R}$}
	{$\left(0;+\infty\right)$}
	\loigiai{
		Tập giá trị của hàm số $ y=\log_2x$ là $\mathbb{R}$ nên để phương trình có nghiệm thực thì $ m\in\mathbb{R}$.
	}
\end{ex}
\begin{ex}
	[Chuyên Bắc Giang 2019]%Câu 30
	Tổng bình phương các nghiệm của phương trình\\ $\log_{\frac{1}{2}}\left(x^2-5x+7\right)=0$ bằng
	\choice
	{$6$}
	{$5$}
	{\True $13$}
	{$7$}
	\loigiai{
		$\log_{\frac{1}{2}}\left(x^2-5x+7\right)=0\Leftrightarrow{x^2}-5x+7=1\Leftrightarrow{x^2}-5x+6=0\Leftrightarrow{x_1}=2\vee{x_2}=3\Rightarrow x_1^2+x_2^2=13$.
	}
\end{ex}
\begin{ex}
	[THPT - Thăng - Long - Hà - Nội - 2019]%Câu 31
	Tổng các nghiệm của phương trình $\log_4x^2-\log_23=1$ là
	\choice
	{$6$}
	{$5$}
	{$4$}
	{\True $0$}
	\loigiai{
		Điều kiện $x\ne 0$. Có $\log_4x^2-\log_23=1\Leftrightarrow\dfrac{1}{2}{\log_2}{x^2}=1+\log_23\Leftrightarrow{\log_2}{x^2}=2.\log_26\Leftrightarrow{x^2}=6^2$\\
		Dó đó, tổng các nghiệm sẽ bằng $0$.
	}
\end{ex}
\begin{ex}
	[THPT - Thăng - Long - Hà - Nội - 2019]%Câu 32
	Tập nghiệm của phương trình $\log_{0,25}\left(x^2-3x\right)=-1$ là:
	\choice
	{$\left\{4\right\}$}
	{$\left\{1;-4\right\}$}
	{$\left\{\dfrac{3-2\sqrt{2}}{2};\dfrac{3+2\sqrt{2}}{2}\right\}$}
	{\True $\left\{-1;4\right\}$}
	\loigiai{
		Ta có $\log_{0,25}\left(x^2-3x\right)=-1\Leftrightarrow\left\{\begin{aligned}
			&{x^2}-3x>0\\ 
			&{x^2}-3x=\left(0,25\right)^{-1}\\ 
		\end{aligned}\right.\Leftrightarrow\left\{\begin{aligned}
			&\left[\begin{aligned}
				&x<0\\ 
				&x>3\\ 
			\end{aligned}\right.\\ 
			&{x^2}-3x-4=0\\ 
		\end{aligned}\right.\Leftrightarrow\left\{\begin{aligned}
			&\left[\begin{aligned}
				&x<0\\ 
				&x>3\\ 
			\end{aligned}\right.\\ 
			&\left[\begin{aligned}
				&x=4(\mathrm{n})\\ 
				&x=-1(\mathrm{n})\\ 
			\end{aligned}\right.\\ 
		\end{aligned}\right.$\\
		Vậy tập nghiệm của phương trình là $S=\left\{-1;4\right\}$.
	}
\end{ex}
\begin{ex}
	[THPT Yên Phong 1 Bắc Ninh 2019]%Câu 33
	Nghiệm nhỏ nhất của phương trình $\log_5\left(x^2-3x+5\right)=1$ là
	\choice
	{$-3$}
	{$a$}
	{$3$}
	{\True $0$}
	\loigiai{
		$\log_5\left(x^2-3x+5\right)=1\Leftrightarrow{x^2}-3x+5=5\Leftrightarrow{x^2}-3x=0\Leftrightarrow\left[\begin{aligned}
			&x=3\\ 
			&x=0\\ 
		\end{aligned}\right.$. Vậy nghiệm nhỏ nhất của phương trình $\log_5\left(x^2-3x+5\right)=1$ là $0$.
	}
\end{ex}
\begin{ex}
	[Sở Hà Nội 2019]%Câu 34
	Số nghiệm dương của phương trình $\ln \left|x^2-5\right|=0$ là
	\choice
	{\True $2$}
	{$4$}
	{$0$}
	{$1$}
	\loigiai{
		Có $\ln \left|x^2-5\right|=0$$\Leftrightarrow\left|x^2-5\right|=1$$\Leftrightarrow\left[\begin{aligned}
			&{x^2}-5=1\\ 
			&{x^2}-5=-1\\ 
		\end{aligned}\right.$$\Leftrightarrow\left[\begin{aligned}
			&x=\sqrt{6}\\ 
			&x=-\sqrt{6}\\ 
			&x=2\\ 
			&x=-2\\ 
		\end{aligned}\right.$.\\
		Vậy phương trình có $2$ nghiệm dương là $x=\sqrt{6}$, $x=2$.
	}
\end{ex}
\begin{ex}
	[Chuyên Hạ Long 2019]%Câu 35
	Số nghiệm của phương trình $(x+3){\log_2}(5-x^2)=0$.
	\choice
	{\True $2$}
	{$0$}
	{$1$}
	{$3$}
	\loigiai{
		Điều kiện: $5-x^2>0\Leftrightarrow-\sqrt{5}<x<\sqrt{5}$.\\
		Phương trình $(x+3){\log_2}(5-x^2)=0\Leftrightarrow\left[\begin{aligned}
			&x+3=0\\
			&{\log_2}(5-x^2)=0\\
		\end{aligned}\right.\Leftrightarrow\left[\begin{aligned}
			&x=-3\\
			&5-x^2=1\\
		\end{aligned}\Leftrightarrow\left[\begin{aligned}
			&x=-3\\
			&x=\pm 2\\
		\end{aligned}\right.\right.$.\\
		Đối chiếu điều kiện ta có $x=\pm 2$ thỏa mãn yêu cầu bài toán. Vậy phương trình có $2$ nghiệm.
	}
\end{ex}
\begin{ex}
	[THPT Yên Khánh - Ninh Bình - 2019]%Câu 36
	Tổng tất cả các nghiệm của phương trình $\left(2x^2-5x+2\right)\left[\log_x\left(7x-6\right)-2\right]=0$ bằng
	\choice
	{$\dfrac{17}{2}$}
	{$9$}
	{\True $8$}
	{$\dfrac{19}{2}$}
	\loigiai{
		Điều kiện $\left\{\begin{aligned}
			&0<x\ne 1\\ 
			&x>\dfrac{6}{7}\\ 
		\end{aligned}\right.\Leftrightarrow\dfrac{6}{7}<x\ne 1(*)$.\\
		Phương trình $\left(2x^2-5x+2\right)\left[\log_x\left(7x-6\right)-2\right]=0\Leftrightarrow\left[\begin{aligned}
			& 2x^2-5x+2=0\\ 
			&{\log_x}\left(7x-6\right)-2=0\\ 
		\end{aligned}\right.$.\\
		Phương trình $2x^2-5x+2=0\Leftrightarrow\left[\begin{aligned}
			&x=2\\ 
			&x=\dfrac{1}{2}\\ 
		\end{aligned}\right.$. Kết hợp với điều kiện $(*)\Rightarrow x=2$.\\
		Phương trình $\log_x\left(7x-6\right)-2=0\Leftrightarrow 7x-6=x^2\Leftrightarrow{x^2}-7x+6=0\Leftrightarrow\left[\begin{aligned}
			&x=1\\ 
			&x=6\\ 
		\end{aligned}\right.$. Kết hợp với\\
		điều kiện $(*)\Rightarrow x=6$.\\
		Vậy phương trình đã cho có hai nghiệm $x=2;x=6$ suy ra tổng các nghiệm bằng $8$.
	}
\end{ex}
\begin{ex}
	[Chuyên ĐHSP Hà Nội 2019]%Câu 37
	Tập hợp các số thực $m$ để phương trình $\log_2x=m$ có nghiệm thực là
	\choice
	{$\left(0;+\infty\right)$}
	{$\left[0;+\infty\right)$}
	{$\left(-\infty ;0\right)$}
	{\True $\mathbb{R}$}
	\loigiai{
		{\color{red}HÌNH Ở ĐÂY}\\
		Điều kiện để phương trình đã cho có nghĩa là $x>0$.\\
		Dễ thấy $\forall m\in\mathbb{R}$ thì đường thẳng $y=m$ luôn cắt đồ thị hàm số $y=\log_2x$ tại đúng một điểm.\\
		Vậy tập hợp các số thực $ m$ để phương trình $\log_2x=m$ có nghiệm thực là $\forall m\in\mathbb{R}$.
	}
\end{ex}
\begin{ex}
	[Đề minh họa 2022]%Câu 38
	Nghiệm của phương trình $\log_2\left(x+4\right)=3$ là:
	\choice
	{$x=5$}
	{\True $x=4$}
	{$x=2$}
	{$x=12$}
	\loigiai{
		$\log_2\left(x+4\right)=3\Leftrightarrow x+4=2^3\Leftrightarrow x=4$.
	}
\end{ex}
\begin{ex}
	[Mã 103 - 2022]%Câu 39
	Nghiệm của phương trình $\log_{\frac{1}{2}}\left(2x-1\right)=0$ là
	\choice
	{$x=\dfrac{3}{4}$}
	{\True $x=1$}
	{$x=\dfrac{1}{2}$}
	{$x=\dfrac{2}{3}$}
	\loigiai{
		$\log_{\dfrac{1}{2}}\left(2x-1\right)=0\Leftrightarrow 2x-1=1\Leftrightarrow x=1$.\\
		Vậy nghiệm của phương trình là $ x=1$.
	}     
\end{ex}
\begin{ex}
	[Mã 103 - 2020 Lần 2]%Câu 40
	Hàm số $ y=\log_ax$ và $ y=\log_bx$ có đồ thị như hình bên.\\
	{\color{red}HÌNH Ở ĐÂY}\\
	Đường thẳng $ y=3$ cắt hai đồ thị tại các điểm có hoành độ là $x_1;{x_2}$. Biết rằng $x_1=2x_2$. Giá trị của $\dfrac{a}{b}$ bằng
	\choice
	{$\dfrac{1}{3}$}
	{$\sqrt{3}$}
	{$2$}
	{\True $\sqrt[3]{2}$}
	\loigiai{
		Xét phương trình hoành độ giao điểm $\log_ax=3\Leftrightarrow{x_1}=a^3$, và $\log_bx=3\Leftrightarrow{x_2}=b^3$.\\
		Ta có $x_1=2x_2\Leftrightarrow{a^3}=2b^3\Leftrightarrow{\left(\dfrac{a}{b}\right)^3}=2\Leftrightarrow\dfrac{a}{b}=\sqrt[3]{2}$.
	}        
\end{ex}
\begin{ex}
	[Đề Tham Khảo 2017]%Câu 41.
	Tìm tập nghiệm $S$ của phương trình $\log_2(x-1)+\log_2(x+1)=3$. 
	\choice
	{\True $S=\{3\}$}
	{$S=\left\{-\sqrt{10};\sqrt{10}\right\}$}
	{$S=\{-3;3\}$}
	{$S=\{4\}$}
	\loigiai{
		Điều kiện $x>1$. Phương trình đã cho trở thành $\log_2\left(x^2-1\right)=3\Leftrightarrow x^2-1=8\Leftrightarrow x=\pm 3$.\\
		Đối chiếu điều kiện, ta được nghiệm duy nhất của phương trình là $x=3\Rightarrow S=\{3\}$.
	}
\end{ex}
\begin{ex}
	[Mã 103 - 2019]%Câu 42.
	Nghiệm của phương trình $\log_2(x+1)+1=\log_2(3x-1)$ là
	\choice
	{$x=1$}
	{$x=2$}
	{$x=-1$}
	{\True $x=3$}
	\loigiai{
		Điều kiện phương trình: $x>\dfrac{1}{3}$.\\
		$\log_2(x+1)+1=\log_2(3x-1)\Leftrightarrow\log_2[(x+1)\cdot 2]=\log_2(3x-1)\Leftrightarrow 2(x+1)=3x-1\Leftrightarrow x=3$.\\
		Ta có $x=3$ (Thỏa mãn điều kiện phương trình).\\
		Vậy nghiệm phương trình là $x=3$.
	}
\end{ex}
\begin{ex}
	[Mã 105 2017]%Câu 43.
	Tìm tập nghiệm $S$ của phương trình $\log_3(2x+1)-\log_3(x-1)=1$. 
	\choice
	{$S=\{3\}$}
	{\True $S=\{4\}$}
	{$S=\{1\}$}
	{$S=\{-2\}$}
	\loigiai{
		ĐK: $\heva{&2x+1>0\\&x-1>0}\Leftrightarrow\heva{&x>\dfrac{-1}{2}\\&x>1}\Leftrightarrow x>1$.\\
		Ta có $\log_3(2x+1)-\log_3(x-1)=1\Leftrightarrow\log_3\dfrac{2x+1}{x-1}=1\Leftrightarrow\dfrac{2x+1}{x-1}=3\Leftrightarrow x=4$ (thỏa).
	}
\end{ex}
\begin{ex}
	[Mã 101 - 2019]%Câu 44.
	Nghiệm của phương trình $\log_3(x+1)+1=\log_3(4x+1)$ 
	\choice
	{$x=4$}
	{\True $x=2$}
	{$x=3$}
	{$x=-3$}
	\loigiai{
		Điều kiện: $x >-\dfrac{1}{4}$. Ta có:\\
		$\begin{aligned}&\log_3(x+1)+1=\log_3(4x+1)\\&\Leftrightarrow\heva{&x>\dfrac{-1}{4}\\&3(x+1)=4x+1}\Leftrightarrow\heva{&x>\dfrac{-1}{4}\\&x=2}\Leftrightarrow x=2.\end{aligned}$ \\
		Vậy: Nghiệm của phương trình là $x=2$.
	}
\end{ex}
\begin{ex}
	[Mã 104 - 2019]%Câu 45.
	Nghiệm của phương trình $\log_3(2x+1)=1+\log_3(x-1)$ là
	\choice
	{\True $x=4$}
	{$x=-2$}
	{$x=1$}
	{$x=2$}
	\loigiai{
		Điều kiện: $\heva{&2x+1>0\\&x-1>0}\Leftrightarrow x>1$.\\
		Ta có $\log_3(2x+1)=1+\log_3(x-1)$ \\
		$\Leftrightarrow\log_3(2x+1)=\log_3\left[3\cdot(x-1)\right] $ \\
		$\Leftrightarrow 2x+1=3x-3 $ \\
		$\Leftrightarrow x=4 $ (nhận).
	}
\end{ex}
\begin{ex}
	[Mã 102 -2019]%Câu 46.
	Nghiệm của phương trình $\log_2(x+1)=1+\log_2(x-1)$ là
	\choice
	{\True $x=3$}
	{$x=2$}
	{$x=1$}
	{$x=-2$}
	\loigiai{
		Điều kiện: $\heva{&x >-1\\&x>1}\Leftrightarrow x>1$.\\
		Phương trình đã cho tương đương với.\\
		$\log_2(x+1)=1+\log_2(x-1)$ \\
		$ \Leftrightarrow\log_2(x+1)=\log_22\cdot (x-1) $ \\
		$ \Leftrightarrow x+1=2x-2\Leftrightarrow x=3 $ (Thỏa mãn).
	}
\end{ex}
\begin{ex}[THPT Lê Quy Đôn Điện Biên 2019]%Câu 47.
	Số nghiệm của phương trình $\ln (x+1)+\ln (x+3)=\ln (x+7)$ là
	\choice
	{\True $1$}
	{$0$}
	{$2$}
	{$3$}
	\loigiai{
		Điều kiện: $x >-1$.\\
		Phương trình $\Leftrightarrow\ln [(x+1)(x+3)]=\ln (x+7)$ \\
		$ \Leftrightarrow(x+1)(x+3)=x+7 $ \\
		$ \Leftrightarrow x^2+3x-4=0 $ \\
		$ \Leftrightarrow\hoac{&x=1 (n)\\&x=-4 (\ell)} $.
	}
\end{ex}
\begin{ex}
	Tìm số nghiệm của phương trình $\log_2x+\log_2(x-1)=2$ 
	\choice
	{$0$}
	{\True $1$}
	{$3$}
	{$2$}
	\loigiai{
		Điều kiện: $x>1$.\\
		Ta có: $\log_2x+\log_2(x-1)=2$.\\
		$\begin{aligned}&\Leftrightarrow\log_2[x(x-1)]=2\Leftrightarrow x(x-1)=4\Leftrightarrow x^2-x-4=0\\&\Leftrightarrow\hoac{&x=\dfrac{1-\sqrt{17}}{2}\\&x=\dfrac{1+\sqrt{17}}{2}}.\end{aligned}$ \\
		Đối chiếu với điều kiện ta được nghiệm của phương trình là $x=\dfrac{1+\sqrt{17}}{2}$.
	}
\end{ex}
\begin{ex}[HSG Bắc Ninh 2019]%Câu 49.
	Số nghiệm của phương trình $\log_3(6+x)+\log_39x-5=0$. 
	\choice
	{$0$}
	{$2$}
	{\True $1$}
	{$3$}
	\loigiai{
		Điều kiện $x>0$.\\
		Phương trình $\Leftrightarrow\log_3(6+x)+\log_3x=3\Leftrightarrow\log_3x(6+x)=3\Leftrightarrow x^2+6x-27=0$ \\
		$ \Leftrightarrow\hoac{&x=3\\&x=-9(L)}\Leftrightarrow x=3 $. Vậy phương trình có $1$ nghiệm.\\
		Vậy số nghiệm của phương trình là $1$.
	}
\end{ex}
\begin{ex}
	[THPT Đoàn Thượng - Hải Dương - 2019]%Câu 50.
	Tìm tập nghiệm $S$ của phương trình: $\log_3(2x+1)-\log_3(x-1)=1$. 
	\choice
	{$S=\{3\}$}
	{$S=\{1\}$}
	{$S=\{2\}$}
	{\True $S=\{4\}$}
	\loigiai{
		Điều kiện: $\heva{&2x+1>0\\&x-1>0}\Leftrightarrow x>1$.\\
		Với điều kiện trên, $\log_3(2x+1)-\log_3(x-1)=1\Leftrightarrow\log_3(2x+1)=\log_3(x-1)+\log_33\Leftrightarrow\log_3(2x+1)=\log_3(3x-3)\Leftrightarrow 2x+1=3x-3\Leftrightarrow x=4$ (thỏa mãn điều kiện).\\
		Vậy tập nghiệm $S=\{4\}$.
	}
\end{ex}
\begin{ex}
	[Sở Bắc Giang 2019]%Câu 51.
	Phương trình $\log_2x+\log_2(x-1)=1$ có tập nghiệm là
	\choice
	{$S=\{-1;3\}$}
	{$S=\{1;3\}$}
	{\True $S=\{2\}$}
	{$S=\{1\}$}
	\loigiai{
		Điều kiện: $x>1$.\\
		Với điều kiện trên, ta có: $\log_2x+\log_2(x-1)=1\Leftrightarrow\log_2[x(x-1)]=1\Leftrightarrow x^2-x-2=0\Leftrightarrow\hoac{&x=-1\\&x=2.}$ \\
		Kết hợp với điều kiện ta được: $x=2$.\\
		Vậy tập nghiệm của phương trình là $S=\{2\}$.
	}
\end{ex}
\begin{ex}
	[THPT Gang Thép Thái Nguyên 2019]%Câu 52.
	Tổng các nghiệm của phương trình $\log_2(x-1)+\log_2(x-2)=\log_5125$ là
	\choice
	{\True $\dfrac{3+\sqrt{33}}{2}$}
	{$\dfrac{3-\sqrt{33}}{2}$}
	{$3$}
	{$\sqrt{33}$}
	\loigiai{
		Điều kiện: $x>2$.\\
		$\log_2(x-1)+\log_2(x-2)=\log_5125\Leftrightarrow\log_2\left(x^2-3x+2\right)=3$ \\
		$ \Leftrightarrow x^2-3x-6=0\Leftrightarrow\hoac{&x=\dfrac{3+\sqrt{33}}{2}\\&x=\dfrac{3-\sqrt{33}}{2}.} $ \\
		Đối chiếu điều kiện ta thấy nghiệm $x=\dfrac{3+\sqrt{33}}{2}$ thỏa mãn.\\
		Vậy tổng các nghiệm của phương trình là $\dfrac{3+\sqrt{33}}{2}$.
	}
\end{ex}
\begin{ex}
	[THPT Ngô Sĩ Liên Bắc Giang 2019]%Câu 53.
	Tập nghiệm của phương trình $\log_2x+\log_2(x-3)=2$ là
	\choice
	{\True $S=\{4\}$}
	{$S=\{-1,4\}$}
	{$S=\{-1\}$}
	{$S=\{4,5\}$}
	\loigiai{
		Điều kiện: $x\geq 3$.\\
		PT $\Leftrightarrow\log_2[x(x-3)]=2\Leftrightarrow x^2-3x-4=0\Leftrightarrow\hoac{&x=4\\&x=-1.}$ \\
		So sánh điều kiện ta được $x=4$.\\
		Vậy tập nghiệm của phương trình là $S=\{4\}$.
	}
\end{ex}
\begin{ex}
	[Chuyên Thái Nguyên 2019]%Câu 54.
	Số nghiệm của phương trình $\log_3x+\log_3(x-6)=\log_37$ là
	\choice
	{$0$}
	{$2$}
	{\True $1$}
	{$3$}
	\loigiai{
		Đk: $x>6$.\\
		Ta có: $\log_3x+\log_3(x-6)=\log_37\Leftrightarrow\log_3[x(x-6)]=\log_37\Leftrightarrow x^2-6x-7=0\Leftrightarrow\hoac{&x=-1\\&x=7.}$ \\
		So với điều kiên vậy phuiwng trình có một nghiệm $x=7$.
	}
\end{ex}
\begin{ex}
	[Chuyên Sơn La 2019]%Câu 55.
	Cho $x\in\left(0;\dfrac{\pi}{2}\right)$, biết rằng $\log_2(\sin x)+\log_2(\cos x)=-2$ và $\log_2\left(\sin x+\cos x\right)=\dfrac{1}{2}(\log_2n+1)$. Giá trị của $n$ bằng
	\choice
	{$\dfrac{1}{4}$}
	{$\dfrac{5}{2}$}
	{$\dfrac{1}{2}$}
	{\True $\dfrac{3}{4}$}
	\loigiai{
		Vì $x\in\left(0;\dfrac{\pi}{2}\right)$ nên $\sin x>0$ và $\cos x>0$.\\
		Ta có $\log_2(\sin x)+\log_2(\cos x)=-2\Leftrightarrow\log_2\left(\sin x\cdot\cos x\right)=-2\Leftrightarrow\sin x\cdot\cos x=\dfrac{1}{4}$ \\
		$ \Rightarrow\left(\sin x+\cos x\right)^2=1+2\sin x\cdot\cos x=\dfrac{3}{2} $.\\
		Suy ra: $\log_2\left(\sin x+\cos x\right)=\dfrac{1}{2}(\log_2n+1)\Leftrightarrow\log_2\left(\sin x+\cos x\right)^2=\log_2(2n)$ \\
		$ \Leftrightarrow\left(\sin x+\cos x\right)^2=2n\Leftrightarrow\dfrac{3}{2}=2n\Leftrightarrow n=\dfrac{3}{4} $.
	}
\end{ex}  
\begin{ex}
	[Mã 110 2017]%Câu 56.
	Tìm tập nghiệm $S$ của phương trình $\log_{\sqrt{2}}(x-1)+\log_{\tfrac{1}{2}}(x+1)=1$. 
	\choice
	{$S=\{3\}$}
	{$S=\left\{2-\sqrt{5};2+\sqrt{5}\right\}$}
	{\True $S=\{2+\sqrt{5}\}$}
	{$S=\left\{\dfrac{3+\sqrt{13}}{2}\right\}$}
	\loigiai{
		Điều kiện $\heva{&x-1>0\\&x+1>0}\Leftrightarrow x>1 (*)$.\\
		Phương trình $\Leftrightarrow 2\log_2(x-1)-\log_2(x+1)=1$ \\
		$ \Leftrightarrow 2\log_2(x-1)=\log_2(x+1)+\log_22 $ \\
		$ \Leftrightarrow\log_2(x-1)^2=\log_2[2(x+1)] $ \\
		$ \Leftrightarrow x^2-2x+1=2x+2 $ \\
		$ \Leftrightarrow x^2-4x-1=0\Leftrightarrow\hoac{&x=2-\sqrt{5}(L)\\&x=2+\sqrt{5}} $. Vậy tập nghiệm phương trình $S=\{2+\sqrt{5}\}$.
	}
\end{ex}
\begin{ex}
	[THPT Hàm Rồng Thanh Hóa 2019]%Câu 57.
	Số nghiệm của phương trình $\log_3\left(x^2+4x\right)+\log_{\tfrac{1}{3}}(2x+3)=0$ là
	\choice
	{$2$}
	{$3$}
	{$0$}
	{\True $1$}
	\loigiai{
		Viết lại phương trình ta được.\\
		$\log_3\left(x^2+4x\right)=\log_3(2x+3)\Leftrightarrow\heva{&2x+3>0\\&x^2+4x=2x+3}\Leftrightarrow\heva{&x >-\dfrac{3}{2}\\&\hoac{&x=1\\&x=-3}}\Leftrightarrow x=1$.
	}
\end{ex}
\begin{ex}
	[Đề Tham Khảo 2018]%Câu 58.
	Tổng giá trị tất cả các nghiệm của phương trình $\log_3x\cdot\log_9x\cdot\log_{27}x\cdot\log_{81}x=\dfrac{2}{3}$ bằng
	\choice
	{$0$}
	{$\dfrac{80}{9}$}
	{\True $9$}
	{$\dfrac{82}{9}$}
	\loigiai{
		Điều kiện $x>0$.\\
		Phương trình đã cho tương đương với.\\
		$\log_3\cdot\dfrac{1}{2}\cdot\log_3x\cdot\dfrac{1}{3}\log_3x\cdot\dfrac{1}{4}\log_3x=\dfrac{2}{3}\Leftrightarrow (\log_3x)^4=16\Leftrightarrow\hoac{&\log_3x=2\\&\log_3x=-2}\Leftrightarrow\hoac{&x=9\\&x=\dfrac{1}{9}}$.
	}
\end{ex}
\begin{ex}
	[VTED 2019]%Câu 59.
	Nghiệm của phương trình $\log_2x+\log_4x=\log_{\tfrac{1}{2}}\sqrt{3}$ là
	\choice
	{\True $x=\dfrac{1}{\sqrt[3]{3}}$}
	{$x=\sqrt[3]{3}$}
	{$x=\dfrac{1}{3}$}
	{$x=\dfrac{1}{\sqrt{3}}$}
	\loigiai{
		Điều kiện: $x>0$.\\
		Ta có $\log_2x+\log_4x=\log_{\tfrac{1}{2}}\sqrt{3}\Leftrightarrow\log_2x+\dfrac{1}{2}\log_2x=-\dfrac{1}{2}\log_23$ \\
		$ \Leftrightarrow 2\log_2x+\log_2x+\log_23=0\Leftrightarrow 3\log_2x+\log_23=0 $ \\
		$\Leftrightarrow\log_2x^3+\log_23=0\Leftrightarrow\log_2\left(3x^3\right)=0\Leftrightarrow 3x^3=1\Leftrightarrow x=\dfrac{1}{\sqrt[3]{3}} $.\\
		So với điều kiện, nghiệm phương trình là $x=\dfrac{1}{\sqrt[3]{3}}$.}
\end{ex}
\begin{ex}
	[THPT Lê Quý Dôn Dà Nẵng - 2019]%Câu 60.
	Gọi $S$ là tập nghiệm của phương trình $\log_{\sqrt{2}}(x+1)=\log_2\left(x^2+2\right)-1$. Số phần tử của tập $S$ là
	\choice
	{$2$}
	{$3$}
	{$1$}   
	{\True $0$}
	\loigiai{
		ĐK: $x >-1$.\\
		$\log_{\sqrt{2}}(x+1)=\log_2\left(x^2+2\right)-1\Rightarrow(x+1)^2=\dfrac{x^2+2}{2}\Rightarrow\hoac{&x=0(\mathrm{TM})\\&x=-4(\mathrm{L}).}$ \\
		Vậy tập nghiệm có một phần tử.
	}
\end{ex}
\begin{ex}
	[Chuyên Lam Sơn Thanh Hóa 2019]%Câu 61.
	Số nghiệm thục của phương trình $3\log_3(x-1)-\log_{\frac{1}{3}}(x-5)^3=3$ là
	\choice
	{$3$}
	{\True $1$}
	{$2$}
	{$0$}
	\loigiai{
		Điều kiện: $x>5$.\\
		$3\log_3(x-1)-\log_{\frac{1}{3}}(x-5)^3=3\Leftrightarrow 3\log_3(x-1)+3\log_3(x-5)=3\Leftrightarrow\log_3(x-1)+\log_3(x-5)=1\Leftrightarrow\log_3[(x-1)(x-5)]=1\Leftrightarrow(x-1)(x-5)=3$ \\
		$ \Leftrightarrow x^2-6x+2=0\Leftrightarrow x=3\pm\sqrt{7} $.\\
		Đối chiếu điều kiện suy ra phương trình có $1$ nghiệm $x=3+\sqrt{7}$.
	}
\end{ex}
\begin{ex}
	[Chuyên Lê Hồng Phong Nam Định 2019]%Câu 62.
	Tổng các nghiệm của phương trình $\log_{\sqrt{3}}(x-2)+\log_3(x-4)^2=0$ là $S=a+b\sqrt{2}$ (với $a, b$ là các số nguyên). Giá trị của biểu thức $Q=a\cdot b$ bằng
	\choice
	{$0$}
	{$3$}
	{$9$}
	{\True $6$}
	\loigiai{
		Điều kiện: $2<x\neq 4$.\\
		Với điều kiện trên, phương trình đã cho tương đương.\\
		$2\log_3(x-2)+2\log_3|x-4|=0\Leftrightarrow\log_3(x-2)|x-4|=0\Leftrightarrow(x-2)|x-4|=1$ \\
		$\Leftrightarrow\hoac{&(x-2)(x-4)=1\\&(x-2)(x-4)=-1}\Leftrightarrow\hoac{&x^2-6x+7=0\\&x^2-6x+9=0}\Leftrightarrow\hoac{&x=3\pm\sqrt{2}\\&x=3.} $ \\
		So lại điều kiện, ta nhận hai nghiệm $x_1=3+\sqrt{2}; x_2=3$.\\
		Ta được: $S=x_1+x_2=6+\sqrt{2}\Rightarrow a=6; b=1$. Vậy $Q=a\cdot b=6$.
	}
\end{ex}
\begin{dang}
	{Phương trình mũ}
\end{dang}
\begin{ex}
	[Đề Minh Họa 2021]%Câu 63.
	Nghiệm của phương trình $5^{2x-4}=25$ là
	\choice
	{\True $x=3$}
	{$x=2$}
	{$x=1$}
	{$x=-1$}
	\loigiai{
		Ta có $5^{2x-4}=25\Leftrightarrow 5^{2x-4}=5^2\Leftrightarrow 2x-4=2\Leftrightarrow x=3$.\\
		Vậy tập nghiệm của phương trình đã cho là $S=\{3\}$.
	}
\end{ex}
\begin{ex}
	[Đề Tham Khảo 2020 Lần 2]%Câu 64.
	Nghiệm của phương trình $3^{x-1}=27$ là
	\choice
	{\True $x=4$}
	{$x=3$}
	{$x=2$}
	{$x=1$}
	\loigiai{
		Ta có $3^{x-1}=27\Leftrightarrow 3^{x-1}=3^3\Leftrightarrow x-1=3\Leftrightarrow x=4$.\\
		Vậy nghiệm của phương trình là $x=4$.
	}
\end{ex}
\begin{ex}[Mã 101 - 2020 Lần 1]%Câu 65.
	Nghiệm của phương trình $3^{x-1}=9$ là 
	\choice
	{$x=-2$}
	{\True $x=3$}
	{$x=2$}
	{$x=-3$}
	\loigiai{
		$3^{x-1}=9\Leftrightarrow x-1=\log_39\Leftrightarrow x-1=2\Leftrightarrow x=3$.
	}
\end{ex}
\begin{ex}
	[Mã 102 - 2020 Lần 1]%Câu 66
	Nghiệm của phương trình $3^{x-2}=9$ là
	\choice
	{$x=-3$}
	{$x=3$}
	{\True $x=4$}
	{$x=-4$}
	\loigiai{
		Ta có $3^{x-2}=9 \Leftrightarrow x-2=2 \Leftrightarrow x=4$.
	}
\end{ex}
\begin{ex}
	[Mã 103 - 2020 Lần 1]%Câu 67.
	Nghiệm của phương trình $3^{x+1}=9$ là
	\choice
	{\True $x=1$}
	{$x=2$}
	{$x=-2$}
	{$x=-1$}
	\loigiai{
		Ta có $3^{x+1}=9\Leftrightarrow 3^{x+1}=3^2\Leftrightarrow x+1=2\Leftrightarrow x=1$.
	}
\end{ex}
\begin{ex}
	[Mã 104 - 2020 Lần 1]%Câu 68.
	Nghiệm của phương trình $3^{x+2}=27$ là
	\choice
	{$x=-2$}
	{$x=-1$}
	{$x=2$}
	{\True $x=1$}
	\loigiai{
		Ta có $3^{x+2}=27\Leftrightarrow 3^{x+2}=3^3\Leftrightarrow x+2=3\Leftrightarrow x=1$.
	}
\end{ex}
\begin{ex}
	[Mã 102 - 2020 Lần 2]%Câu 69.
	Nghiệm của phương trình $2^{2x-4}=2^x$ là
	\choice
	{$x=16$}
	{$x=-16$}
	{$x=-4$}
	{\True $x=4$}
	\loigiai{
		Ta có $2^{2x-4}=2^x\Leftrightarrow 2x-4=x\Leftrightarrow x=4$.
	}
\end{ex}
\begin{ex}
	[Mã 101 - 2020 Lần 2]%Câu 70.
	Nghiệm của phương trình $2^{2x-3}=2^x$ là
	\choice
	{$x=8$}
	{$x=-8$}
	{\True $x=3$}
	{$x=-3$}
	\loigiai{
		Ta có $2^{2x-3}=2^x\Leftrightarrow 2x-3=x\Leftrightarrow x=3$. Vậy phương trình đã cho có một nghiệm $x=3$.
	}
\end{ex}
\begin{ex}
	[Mã 104 - 2020 Lần 2]%Câu 71.
	Nghiệm của phương trình $Ox$ là
	\choice
	{$x=-2$}
	{\True $x=2$}
	{$x=-4$}
	{$x=4$}
	\loigiai{
		$2^{2x-2}=2^x\Leftrightarrow 2x-2=x\Leftrightarrow x=2$.
	}
\end{ex}
\begin{ex}
	[Mã 101 - 2019]%Câu 72.
	Nghiệm của phương trình: $3^{2x-1}=27$ là
	\choice
	{$x=1$}
	{\True $x=2$}
	{$x=4$}
	{$x=5$}
	\loigiai{
		Ta có $3^{2x-1}=27\Leftrightarrow 3^{2x-1}=3^3\Leftrightarrow 2x-1=3\Leftrightarrow x=2$.
	}
\end{ex}
\begin{ex}
	[Mã 102 - 2019]%Câu 73.
	Nghiệm của phương trình $3^{2x+1}=27$ là
	\choice
	{$5$}
	{$4$}
	{$2$}
	{\True $1$}
	\loigiai{
		Ta có $2x+1=3\Rightarrow x=1$.
	}
\end{ex}
\begin{ex}
	Tìm nghiệm của phương trình $3^{x-1}=27$ 
	\choice
	{$x=10$}
	{$x=9$}
	{$x=3$}
	{\True $x=4$}
	\loigiai{
		$3^{x-1}=3^3\Leftrightarrow x-1=3\Leftrightarrow x=4$.
	}
\end{ex}
\begin{ex}
	[Mã 104 2018]%Câu 75.
	Phương trình $5^{2x+1}=125$ có nghiệm là
	\choice
	{$x=\dfrac{5}{2}$}
	{\True $x=1$}
	{$x=3$}
	{$x=\dfrac{3}{2}$}
	\loigiai{
		Ta có $5^{2x+1}=125\Leftrightarrow 5^{2x+1}=5^3\Leftrightarrow 2x+1=3\Leftrightarrow x=1$.
	}
\end{ex}  
\begin{ex}
	[Mã 101 2018]%Câu 76.
	Phương trình $2^{2x+1}=32$ có nghiệm là
	\choice
	{$x=3$}
	{$x=\dfrac{5}{2}$}
	{\True $x=2$}
	{$x=\dfrac{3}{2}$}
	\loigiai{
		Ta có $2^{2x+1}=32\Leftrightarrow 2^{2x+1}=2^5\Leftrightarrow 2x+1=5\Leftrightarrow x=2$.
	}
\end{ex}
\begin{ex}
	[Mã 104 - 2019]%Câu 77.
	Nghiệm của phương trình $2^{2x-1}=32$ là
	\choice
	{$x=2$}
	{$x=\dfrac{17}{2}$}
	{$x=\dfrac{5}{2}$}
	{\True $x=3$}
	\loigiai{
		$2^{2x-1}=32\Leftrightarrow 2^{2x-1}=2^5\Leftrightarrow 2x-1=5\Leftrightarrow x=3$.
	}
\end{ex}
\begin{ex}
	[Mã 103 - 2019]%Câu 78.
	Nghiệm của phương trình $2^{2x-1}=8$ là
	\choice
	{\True $x=2$}
	{$x=\dfrac{5}{2}$}
	{$x=1$}
	{$x=\dfrac{3}{2}$}
	\loigiai{
		Ta có $2^{2x-1}=8\Leftrightarrow 2x-1=3\Leftrightarrow x=2$.
	}
\end{ex}
\begin{ex}
	[Mã 104 2017]%Câu 79.
	Tìm tất cả các giá trị thực của $m$ để phương trình $3^x=m$ có nghiệm thực. 
	\choice
	{$m\geq 1$}
	{$m\geq 0$}
	\True {$m>0$}
	{$m\neq 0$}
	\loigiai{
		Để phương trình $3^x=m$ có nghiệm thực thì $m>0$.
	}
\end{ex}
\begin{ex}
	[THPT An Lão Hải Phòng 2019]%Câu 80.
	Tìm tập nghiệm $S$ của phương trình $5^{2x^2-x}=5$. 
	\choice
	{$S=\varnothing$}
	{$S=\left\{0;\dfrac{1}{2}\right\}$}
	{$S=\{0;2\}$}
	{\True $S=\left\{1;-\dfrac{1}{2}\right\}$}
	\loigiai{
		$5^{2x^2-x}=5\Leftrightarrow 2x^2-x=1\Leftrightarrow 2x^2-x-1=0\Leftrightarrow\hoac{&x=1\\&x=-\dfrac{1}{2}}$.}
\end{ex}
\begin{ex}
	[Chuyên Bắc Ninh 2019]%Câu 81.
	Tìm tập nghiệm $S$ của phương trình $2^{x+1}=8$. 
	\choice
	{$S=\{4\}$}
	{$S=\{1\}$}
	{$S=\{3\}$}
	{\True $S=\{2\}$}
	\loigiai{
		Ta có $2^{x+1}=8\Leftrightarrow 2^{x+1}=2^3\Leftrightarrow x+1=3\Leftrightarrow x=2$.\\
		Vậy tập nghiệm của phương trình đã cho là $S=\{2\}$.
	}
\end{ex}
\begin{ex}
	[Liên Trường THPT Tp Vinh Nghệ An 2019]%Câu 82.
	Phương trình $(\sqrt{5})^{x^2+4x+6}=\log_2128$ có bao nhiêu nghiệm?
	\choice
	{$1$}
	{$3$}
	{\True $2$}
	{$0$}
	\loigiai{
		Phương trình đã cho tương đương với: $x^2+4x+6=\log_{\sqrt{5}}7\Leftrightarrow x^2+4x+6-\log_{\sqrt{5}}7=0$.\\
		Sử dụng máy tính bỏ túi ta thấy phương trình trên có hai nghiệm phân biệt.
	}
\end{ex}
\begin{ex}
	[THPT - Yên Định Thanh Hóa 2019]%Câu 83.
	Tập nghiệm $S$ của phương trình $3^{x^2-2x}=27$. 
	\choice
	{$S=\{1; 3\}$}
	{$S=\{-3; 1\}$}
	{$S=\{-3;-1\}$}
	{\True $S=\{-1; 3\}$}
	\loigiai{
		Ta có $3^{x^2-2x}=27\Leftrightarrow x^2-2x=3\Leftrightarrow\hoac{&x=-1\\&x=3.}$ \\
		Vậy tập nghiệm $S$ của phương trình $3^{x^2-2x}=27$ là $S=\{-1; 3\}$.
	}
\end{ex}
\begin{ex}
	[THPT Quỳnh Lưu 3 Nghệ An 2019]%Câu 84.
	Số nghiệm thực phân biệt của phương trình $\mathrm{e}^{x^2}=\sqrt{3}$ là 
	\choice
	{$1$}
	{$0$}
	{$3$}
	{\True $2$}
	\loigiai{
		Ta có $\mathrm{e}^{x^2}=\sqrt{3}\Leftrightarrow x^2=\ln \sqrt{3}\Leftrightarrow x=\pm\sqrt{\ln \sqrt{3}}$.\\
		Vậy phương trình có $2$ nghiệm thực phân biệt.
	}
\end{ex}
\begin{ex}
	[Sở Ninh Bình 2019]%Câu 85.
	Phương trình $5^{x+2}-1=0$ có tập nghiệm là
	\choice
	{$S=\{3\}$}
	{$S=\{2\}$}
	{$S=\{0\}$}
	{\True $S=\{-2\}$}
	\loigiai{
		Ta có $5^{x+2}-1=0\Leftrightarrow 5^{x+2}=1\Leftrightarrow x+2=0\Leftrightarrow x=-2$.\\
		Vậy $S=\{-2\}$.
	}
\end{ex}
\begin{ex}
	[THCS - THPT Nguyễn Khuyến 2019]%Câu 86.
	Họ nghiệm của phương trình $4^{\cos^2x}-1=0$ là
	\choice
	{$\left\{k\pi;k\in\mathbb{Z}\right\}$}
	{\True $\left\{\dfrac{\pi}{2}+k\pi;k\in\mathbb{Z}\right\}$}
	{$\left\{k2\pi;k\in\mathbb{Z}\right\}$}
	{$\left\{\dfrac{\pi}{3}+k\pi;k\in\mathbb{Z}\right\}$}
	\loigiai{
		Ta có: $4^{\cos^2x}-1=0\Leftrightarrow 4^{\cos^2x}=1\Leftrightarrow\cos^2x=0\Leftrightarrow x=\dfrac{\pi}{2}+k\pi$, $k\in\mathbb{Z}$.\\
		Vậy họ nghiệm của phương trình là $\dfrac{\pi}{2}+k\pi;k\in\mathbb{Z}$.
	}
\end{ex}
\begin{ex}
	[Chuyên Lê Thánh Tông 2019]%Câu 87.
	Cho biết $9^x-12^2=0$, tính giá trị của biểu thức $P=\dfrac{1}{3^{-x-1}}-8\cdot 9^{\frac{x-1}{2}}+19$. 
	\choice
	{$31$}
	{\True $23$}
	{$22$}
	{$15$}
	\loigiai{
		Ta có $9^x-12^2=0\Leftrightarrow 3^x=12$.\\
		$P=3^{x+1}-8\cdot 3^{x-1}+19=3\cdot 3^x-8\cdot\dfrac{3^x}{3}+19=3\cdot 12-8\cdot\dfrac{12}{3}+19=23$.
	}
\end{ex}
\begin{ex}
	[Chuyên Bắc Ninh 2019]%Câu 88.
	Tính tổng tất cả các nghiệm của phương trình $2^{2x^2+5x+4}=4$ 
	\choice
	{\True $-\dfrac{5}{2}$}
	{$-1$}
	{$1$}
	{$\dfrac{5}{2}$}
	\loigiai{
		$2^{2x^2+5x+4}=4\Leftrightarrow 2x^2+5x+2=0\Leftrightarrow\hoac{&x=-\dfrac{1}{2}\\&x=-2.}$ \\
		Vậy tổng hai nghiệm bằng $-\dfrac{5}{2}$.
	}
\end{ex}
\begin{ex}
	Tìm tất cả các giá trị thực của tham số $m$ để phương trình $3^{2x-1}+2m^2-m-3=0$ có nghiệm. 
	\choice
	{\True $m\in\left(-1;\dfrac{3}{2}\right)$}
	{$m\in\left(\dfrac{1}{2};+\infty\right)$}
	{$m\in(0;+\infty)$}
	{$m\in\left[-1;\dfrac{3}{2}\right]$}
	\loigiai{
		$3^{2x-1}+2m^2-m-3=0\Leftrightarrow 3^{2x-1}=3+m-2m^2$.\\
		Phương trình có nghiệm khi $3+m-2m^2>0\Leftrightarrow-1<m<\dfrac{3}{2}$.\\
		Vậy $m\in\left(-1;\dfrac{3}{2}\right)$.
	}
\end{ex}
\begin{ex}%Câu 90
	Cho $a$, $b$ là hai số thực khác $0$, biết: $\left(\dfrac{1}{125}\right)^{a^2+4 a b}=(\sqrt[3]{625})^{3 a^2-8 a b}$. Tỉ sô $\dfrac{a}{b}$ là
	\choice
	{$\dfrac{-8}{7}$}
	{$\dfrac{1}{7}$}
	{$\dfrac{4}{7}$}
	{\True $\dfrac{-4}{21}$}
	\loigiai{
		Ta có 
		$\begin{aligned}[t]
			& \left(\dfrac{1}{125}\right)^{a^2+4 a b}=(\sqrt[3]{625})^{3 a^2-8 a b}\Leftrightarrow 5^{-3\left(a^2+4 a b\right)}=5^{\frac{4}{3}\left(3 a^2-8 a b\right)}\\
			& \Leftrightarrow-3\left(a^2+4 a b\right)=\dfrac{4}{3}\left(3 a^2-8 a b\right) \Leftrightarrow 21 a^2=-4 a b \Leftrightarrow \dfrac{a}{b}=-\dfrac{4}{21}
		\end{aligned}$.
	}
\end{ex}
\begin{ex}
	Tổng các nghiệm của phương trình $2^{x^2-2x+1}=8$ bằng
	\choice
	{$0$}
	{$-2$}
	{\True $2$}
	{$1$}
	\loigiai{
		Ta có $2^{x^2-2x+1}=8\Leftrightarrow 2^{x^2-2x+1}=2^3\Leftrightarrow x^2-2x+1=3\Leftrightarrow x^2-2x-2=0\Leftrightarrow\hoac{&x=1-\sqrt{3}\\&x=1+\sqrt{3}.}$ \\
		Như vậy phương trình đã cho có hai nghiệm: $1-\sqrt{3}$; $1+\sqrt{3}$.\\
		Tổng hai nghiệm là $(1-\sqrt{3})+(1+\sqrt{3}) =2$.
	}
\end{ex}
\begin{ex}
	[KTNL GV Thuận Thành 2 Bắc Ninh 2019]%Câu 92.
	Phương trình $2^{2x^2+5x+4}=4$ có tổng tất cả các nghiệm bằng
	\choice
	{$1$}
	{$\dfrac{5}{2}$}
	{$-1$}
	{\True $-\dfrac{5}{2}$}
	\loigiai{
		Cách 1.
		Ta có $2^{2x^2+5x+4}=4\Leftrightarrow 2^{2x^2+5x+4}=2^2\Leftrightarrow 2x^2+5x+4=2\Leftrightarrow 2x^2+5x+2=0\Leftrightarrow\hoac{&x=-2\\&x=-\dfrac{1}{2}.}$ \\
		Tổng tất cả các nghiệm của phương trình đã cho là $-2+\left(-\dfrac{1}{2}\right)=-\dfrac{5}{2}$.\\
		Cách 2.
		Ta có $2^{2x^2+5x+4}=4\Leftrightarrow 2^{2x^2+5x+4}=2^2\Leftrightarrow 2x^2+5x+4=2\Leftrightarrow 2x^2+5x+2=0$ (1).\\
		Xét phương trình (1): $\Delta=9>0\Rightarrow$ Phương trình (1) có hai nghiệm phân biệt $x_1;x_2$.\\
		Theo định lý Vi-ét ta có: $x_1+x_2=-\dfrac{5}{2}$.\\
		Tổng tất cả các nghiệm của phương trình đã cho là $-\dfrac{5}{2}$.
	}
\end{ex}
\begin{ex}
	[THPT Ngô Sĩ Liên Bắc Giang 2019]%Câu 93.
	Phương trình $5^{2x^2+5x+4}=25$ có tổng tất cả các nghiệm bằng
	\choice
	{$1$}
	{$\dfrac{5}{2}$}
	{$-1$}
	{\True $-\dfrac{5}{2}$}
	\loigiai{
		$5^{2x^2+5x+4}=5^2\Leftrightarrow 2x^2+5x+4=2\Leftrightarrow 2x^2+5x+2=0$.\\
		Tổng các nghiệm là $-\dfrac{5}{2}$.
	}
\end{ex}
\begin{ex}
	[Sở Bắc Ninh 2019]%Câu 94.
	Phương trình $7^{2x^2+5x+4}=49$ có tổng tất cả các nghiệm bằng
	\choice
	{\True $-\dfrac{5}{2}$}
	{$1$}
	{$-1$}
	{$\dfrac{5}{2}$}
	\loigiai{
		$7^{2x^2+5x+4}=49\Leftrightarrow 7^{2x^2+5x+4}=7^2\Leftrightarrow 2x^2+5x+4=2\Leftrightarrow 2x^2+5x+2=0\Leftrightarrow\hoac{&x=-2\\&x=-\dfrac{1}{2}.}$ \\
		Vậy tổng tất cả các nghiệm của phương trình bằng $-2+(-\dfrac{1}{2})=-\dfrac{5}{2}$.
	}
\end{ex}
\begin{ex}
	[Mã 101-2022]%Câu 95.
	Nghiệm của phương trình $3^{2x+1}=3^{2-x}$ là 
	\choice
	{\True $x=\dfrac{1}{3}$}
	{$x=0$}
	{$x=-1$}
	{$x=1$}
	\loigiai{
		$3^{2x+1}=3^{2-x}\Leftrightarrow 2x+1=2-x\Leftrightarrow 3x=1\Leftrightarrow x=\dfrac{1}{3}$.
	}
\end{ex}
\begin{ex}
	[Mã 103 - 2022]%Câu 96.
	Số nghiệm thực của phương trình $2^{x^2+1}=4$ là
	\choice
	{$1$}
	{\True $2$}
	{$3$}
	{$0$}
	\loigiai{
		$2^{x^2+1}=2^2\Leftrightarrow x^2+1=2\Leftrightarrow x^2=1\Leftrightarrow\hoac{&x=1\\&x=-1.}$ \\
	}
\end{ex}
\begin{ex}
	Tập nghiệm của phương trình: $4^{x+1}+4^{x-1}=272$ là
	\choice
	{$\{3; 2\}$}
	{$\{2\}$}
	{\True $\{3\}$}
	{$\{3; 5\}$}
	\loigiai{
		$4^{x+1}+4^{x-1}=272\Leftrightarrow 4\cdot 4^x+\dfrac{4^x}{4}=272\Leftrightarrow 4^x=64\Leftrightarrow x=3$.\\
		Vậy phương trình có tập nghiệm $S=\{3\}$.
	}
\end{ex}
\begin{ex}
	[HKI - NK HCM - 2019]%Câu 98.
	Phương trình $27^{2x-3}=\left(\dfrac{1}{3}\right)^{x^2+2}$ có tập nghiệm là
	\choice
	{$\{-1;7\}$}
	{$\{-1;-7\}$}
	{$\{1;7\}$}
	{\True $\{1;-7\}$}
	\loigiai{
		Ta có $27^{2x-3}=\left(\dfrac{1}{3}\right)^{x^2+2}\Leftrightarrow 3^{6x-9}=3^{-x^2-2}$ \\
		$ \Leftrightarrow 6x-9=-x^2-2\Leftrightarrow x^2+6x-7=0\Leftrightarrow\hoac{&x=1\\&x=-7.} $ \\
		Vậy tập nghiệm của phương trình là $\{1;-7\}$.
	}
\end{ex}
\begin{ex}
	[THPT Quỳnh Lưu- Nghệ An- 2019]%Câu 99.
	Phương trình $3^x\cdot 2^{x+1}=72$ có nghiệm là
	\choice
	{$x=\dfrac{5}{2}$}
	{\True $x=2$}
	{$x=\dfrac{3}{2}$}
	{$x=3$}
	\loigiai{
		$3^x\cdot 2^{x+1}=72\Leftrightarrow 3^x\cdot 2^x\cdot 2=72\Leftrightarrow 6^x=36\Leftrightarrow x=2$.
	}
\end{ex}
\begin{ex}
	[Chuyên Bắc Giang 2019]%Câu 100.
	Nghiệm của phương trình $\left(\dfrac{1}{5}\right)^{x^2-2x-3}=5^{x+1}$ là
	\choice
	{\True $x=-1$; $x=2$}
	{$x=1$; $x=-2$}
	{$x=1$; $x=2$}
	{Vô nghiệm}
	\loigiai{
		Ta có\\
		$\left(\dfrac{1}{5}\right)^{x^2-2x-3}=5^{x+1}\Leftrightarrow5^{-(x^2-2x-3)}=5^{x+1}\Leftrightarrow-x^2+2x+3=x+1\Leftrightarrow-x^2+x+2=0\Leftrightarrow\hoac{&x=-1\\&x=2.}$ \\
		Vậy nghiệm của phương trình là $x=-1$; $x=2$.
	}
\end{ex}   
\begin{ex}
	Tập nghiệm của phương trình $\left(\dfrac{1}{7}\right)^{x^2-2x-3}=7^{x+1}$ là
	\choice
	{$\{-1\}$}
	{\True $\{-1;2\}$}
	{$\{-1;4\}$}
	{$\{2\}$}
	\loigiai{
		Ta có $\left(\dfrac{1}{7}\right)^{x^2-2x-3}=7^{x+1}\Leftrightarrow 7^{-x^2+2x+3}=7^{x+1}\Leftrightarrow-x^2+2x+3=x+1$ \\
		$ \Leftrightarrow x^2-x-2=0\Leftrightarrow\hoac{&x=-1\\&x=2}$.
	}
\end{ex}
\begin{ex}
	Tổng các nghiệm của phương trình $2^{x^2+2x}=8^{2-x}$ bằng
	\choice
	{$-6$}
	{\True $-5$}
	{$5$}
	{$6$}
	\loigiai{
		Ta có $2^{x^2+2x}=8^{2-x}\Leftrightarrow 2^{x^2+2x}=2^{6-3x}\Leftrightarrow x^2+5x-6=0\Leftrightarrow\hoac{&x=1\\&x=-6.}$ \\
		Vậy tổng hai nghiệm của phương trình bằng $-5$.
	}
\end{ex}
\begin{ex}
	[SGD Điện Biên - 2019]%Câu 103.
	Gọi $x_1$, $x_2$ là hai nghiệm của phương trình $7^{x+1}=\left(\dfrac{1}{7}\right)^{x^2-2x-3}$. Khi đó $x_1^2+x_2^2$ bằng 
	\choice
	{$17$}
	{$1$}
	{\True $5$}
	{$3$}
	\loigiai{
		$7^{x+1}=\left(\dfrac{1}{7}\right)^{x^2-2x-3}\Leftrightarrow 7^{x+1}=7^{-\left(x^2-2x-3\right)}\Leftrightarrow x+1=-x^2+2x+3\Leftrightarrow x^2-x-2=0\Leftrightarrow\hoac{&x_1=-1\\&x_2=2.}$ \\
		Vậy $x_1^2+x_2^2=5$.
	}
\end{ex}
\begin{ex}
	Tổng bình phương các nghiệm của phương trình $5^{3x-2}=\left(\dfrac{1}{5}\right)^{-x^2}$ bằng
	\choice
	{$2$}
	{\True $5$}
	{$0$}
	{$3$}
	\loigiai{
		Ta có $5^{3x-2}=\left(\dfrac{1}{5}\right)^{-x^2}\Leftrightarrow 5^{3x-2}=5^{x^2}\Leftrightarrow x^2-3x+2=0\Leftrightarrow\hoac{&x=1\\&x=2.}$ \\
		Vậy tổng bình phương các nghiệm của phương trình $5^{3x-2}=\left(\dfrac{1}{5}\right)^{-x^2}$ bằng $5$.
	}
\end{ex}
\begin{ex}
	Nghiệm của phương trình $2^{7x-1}=8^{2x-1}$ là
	\choice
	{$x=2$}
	{$x=-3$}
	{\True $x=-2$}
	{$x=1$}
	\loigiai{
		$2^{7x-1}=8^{2x-1}\Leftrightarrow 2^{7x-1}=2^{3\cdot (2x-1)}\Leftrightarrow 2^{7x-1}=2^{6x-3}\Leftrightarrow 7x-1=6x-3\Leftrightarrow x=-2$.
	}
\end{ex}
\begin{ex}
	[THPT Lương Văn Tụy - Ninh Bình - 2018]%Câu 106.
	Giải phương trình $(2,5)^{5x-7}=\left(\dfrac{2}{5}\right)^{x+1}$. 
	\choice
	{$x\geq 1$}
	{\True $x=1$}
	{$x<1$}
	{$x=2$}
	\loigiai{
		Ta có $(2,5)^{5x-7}=\left(\dfrac{2}{5}\right)^{x+1}\Leftrightarrow\left(\dfrac{5}{2}\right)^{5x-7}=\left(\dfrac{5}{2}\right)^{-x-1}\Leftrightarrow 5x-7=-x-1\Leftrightarrow x=1$.
	}
\end{ex}
\begin{ex}
	[THPT Nguyễn Thị Minh Khai - Hà Tĩnh - 2018]%Câu 107.
	Phương trình $3^{x^2-4}=\left(\dfrac{1}{9}\right)^{3x-1}$ có hai nghiệm $x_1$, $x_2$. Tính $x_1x_2$. 
	\choice
	{\True $-6$}
	{$-5$}
	{$6$}
	{$-2$}
	\loigiai{
		Ta có $3^{x^2-4}=\left(\dfrac{1}{9}\right)^{3x-1}\Leftrightarrow x^2-4=2-6x\Leftrightarrow x^2+6x-6=0$.\\
		Áp dụng Vi-ét suy ra phương trình đã cho có hai nghiệm $x_1$, $x_2$ thì $x_1x_2=-6$.
	}
\end{ex}
\begin{ex}
	[Sở Quảng Nam - 2018]%Câu 108.
	Tổng các nghiệm của phương trình $2^{x^2+2x}=8^{2-x}$ bằng
	\choice
	{$5$}
	{\True $-5$}
	{$6$}
	{$-6$}
	\loigiai{
		Phương trình đã cho tương đương: $2^{x^2+2x}=2^{3(2-x)}\Leftrightarrow x^2+2x=6-3x\Leftrightarrow x^2+5x-6=0$.\\
		Do đó tổng các nghiệm của phương trình là $S=-\dfrac{b}{a}=-5$.
	}
\end{ex}
\begin{ex}
	[THPT Thăng Long - Hà Nội - 2018]%Câu 109.
	Tập nghiệm của phương trình $4^{x-x^2}=\left(\dfrac{1}{2}\right)^x$ là
	\choice
	{$\left\{0;\dfrac{2}{3}\right\}$}
	{$\left\{0;\dfrac{1}{2}\right\}$}
	{$\{0; 2\}$}
	{\True $\left\{0;\dfrac{3}{2}\right\}$}
	\loigiai{
		Ta có $4^{x-x^2}=\left(\dfrac{1}{2}\right)^x\Leftrightarrow 2^{2x-2x^2}=2^{-x}\Leftrightarrow-2x^2+2x=-x\Leftrightarrow-2x^2+3x=0\Leftrightarrow\hoac{&x=0\\&x=\dfrac{3}{2}}$.
	}
\end{ex}
\begin{ex}
	[THPT Hải An - Hải Phòng - 2018]%Câu 110.
	Tìm nghiệm của phương trình $\left(7+4\sqrt{3}\right)^{2x+1}=2-\sqrt{3}$. 
	\choice
	{$x=\dfrac{1}{4}$}
	{$x=-1+\log_{7+4\sqrt{3}}(2-\sqrt{3})$}
	{\True $x=-\dfrac{3}{4}$}
	{$x=\dfrac{25-15\sqrt{3}}{2}$}
	\loigiai{
		Ta có\\
		$\left(7+4\sqrt{3}\right)^{2x+1}=2-\sqrt{3}\Leftrightarrow\left(2+2\sqrt{3}\right)^{4x+2}=(2+\sqrt{3})^{-1}\Leftrightarrow 4x+2=-1\Leftrightarrow 4x=-3\Leftrightarrow x=-\dfrac{3}{4}$.
	}
\end{ex}
\begin{ex}
	[THPT Kim Liên - Hà Nội - 2018]%Câu 111.
	Tính tổng $S=x_1+x_2$ biết $x_1$, $x_2$ là các giá trị thực thỏa mãn đẳng thức $2^{x^2-6x+1}=\left(\dfrac{1}{4}\right)^{x-3}$. 
	\choice
	{$S=-5$}
	{$S=8$}
	{\True $S=4$}
	{$S=2$}
	\loigiai{
		Ta có $2^{x^2-6x+1}=\left(\dfrac{1}{4}\right)^{x-3}\Leftrightarrow 2^{x^2-6x+1}=(2)^{-2(x-3)}\Leftrightarrow x^2-6x+1=-2x+6$ \\
		$ \Leftrightarrow x^2-4x-5=0\Leftrightarrow\hoac{&x_1=-1\\&x_2=5}\Rightarrow S=x_1+x_2=4$.
	}
\end{ex}
\begin{ex}
	[Chuyên Hùng Vương - Bình Dương - 2018]%Câu 112.
	Tập nghiệm $S$ của phương trình $\left(\dfrac{4}{7}\right)^x\left(\dfrac{7}{4}\right)^{3x-1}-\dfrac{16}{49}=0$ là
	\choice
	{\True $S=\left\{\dfrac{-1}{2}\right\}$}
	{$S=\{2\}$}
	{$S=\left\{\dfrac{1}{2};\dfrac{-1}{2}\right\}$}
	{$S=\left\{\dfrac{-1}{2};2\right\}$}
	\loigiai{
		Ta có\\
		$\begin{aligned}&\left(\dfrac{4}{7}\right)^x\left(\dfrac{7}{4}\right)^{3x-1}-\dfrac{16}{49}=\left(\dfrac{4}{7}\right)^x\left(\dfrac{7}{4}\right)^x\left(\dfrac{7}{4}\right)^{2x-1}-\dfrac{16}{49}=\left(\dfrac{7}{4}\right)^{2x-1}-\dfrac{16}{49}=0\\&\Leftrightarrow\left(\dfrac{7}{4}\right)^{2x-1}=\dfrac{16}{49}=\left(\dfrac{7}{4}\right)^{-2}\Leftrightarrow 2x-1=-2\Leftrightarrow x=\dfrac{-1}{2}.\end{aligned}$.
	}
\end{ex}
\begin{ex}
	[THPT Nguyễn Thị Minh Khai - Hà Nội - 2018]%Câu 113.
	Tích các nghiệm của phương trình $(\sqrt{5}+2)^{x-1}=(\sqrt{5}-2)^{\frac{x-1}{x+1}}$ là
	\choice
	{\True $-2$}
	{$-4$}
	{$4$}
	{$2$}
	\loigiai{
		Điều kiện xác định: $x\neq-1$.\\
		Vì $(\sqrt{5}-2)(\sqrt{5}+2)=1$ nên $(\sqrt{5}-2)=(\sqrt{5}+2)^{-1}$.\\
		Khi đó phương trình đã cho tương đương $(\sqrt{5}+2)^{x-1}=(\sqrt{5}+2)^{\frac{-x+1}{x+1}}$ \\
		$\Leftrightarrow x-1=\dfrac{-x+1}{x+1}$ \\
		$\Leftrightarrow\hoac{&x=1\\&x=-2}$. (thỏa điều kiện).\\
		Suy ra tích hai nghiệm là $-2$.
	}
\end{ex}      
\begin{ex}       
	[THCS \& THPT Nguyễn Khuyến - Bình Dương - 2018]%Câu 114.
	Giải phương trình $4^{2x+3}=8^{4-x}$. 
	\choice
	{\True $x=\dfrac{6}{7}$}
	{$x=\dfrac{2}{3}$}
	{$x=2$}
	{$x=\dfrac{4}{5}$}
	\loigiai{
		$4^{2x+3}=8^{4-x}\Leftrightarrow 2^{4x+6}=2^{12-3x}\Leftrightarrow 4x+6=12-3x\Leftrightarrow x=\dfrac{6}{7}$.
	}
\end{ex}                            
\Closesolutionfile{ans}
\indapan{10}{ans/CD19/Muc_5_6}