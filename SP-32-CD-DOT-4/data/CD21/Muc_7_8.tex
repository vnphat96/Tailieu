\Opensolutionfile{ans}[ans/CD21/Muc_7_8]
\setcounter{ex}{0}
\setcounter{dang}{0}
\section{Mức độ 7,8 điểm}
\begin{dang}
	{Diện tích xung quanh, diện tích toàn phần, chiều cao, bán kính đáy, thiết diện}
\end{dang}
\begin{ex}
	[Đề Tham Khảo 2020 Lần 2]%Câu 1.
	Trong không gian, cho tam giác $ABC$ vuông tại $A$, $AB=a$ và $AC=2a$. Khi quay tam giác $ABC$ quanh cạnh góc vuông $AB$ thì đường gấp khúc $ACB$ tạo thành một hình nón. Diện tích xung quanh hình nón đó bằng
	\choice
	{$5\pi a^2$}
	{$\sqrt{5}\pi a^2$}
	{\True $2\sqrt{5}\pi a^2$}
	{$10\pi a^2$}
	\loigiai{
		{\color{red} HINH O DAY}\\
		$BC=\sqrt{AB^2+AC^2}=a\sqrt{5}$.\\
		Diện tích xung quanh hình nón cần tìm là $S=\pi\cdot AC\cdot BC=\pi\cdot 2a\cdot a\sqrt{5}=2\sqrt{5}\pi a^2$.
	}
\end{ex}
\begin{ex}
	[Mã 101 - 2020 Lần 1]%Câu 2.
	Cho hình nón có bán kính đáy bằng $2$ và góc ở đỉnh bằng $60^{\circ}$. Diện tích xung quanh của hình nón đã cho bằng
	\choice
	{\True $8\pi$}
	{$\dfrac{16\sqrt{3}\pi}{3}$}
	{$\dfrac{8\sqrt{3}\pi}{3}$}
	{$16\pi$}
	\loigiai{
		{\color{red} HINH O DAY}\\
		Gọi $S$ là đỉnh của hình nón và $AB$ là một đường kính của đáy.\\
		Theo bài ra, ta có tam giác $SAB$ là tam giác đều $\Rightarrow l=SA=AB=2r=4$.\\
		Vậy diện tích xung quanh của hình nón đã cho là $S_{\mathrm{xq}}=\pi rl=8\pi$.
	}
\end{ex}
\begin{ex}
	[Mã 102 - 2020 Lần 1]%Câu 3
	Cho hình nón có bán kính bằng 5 và góc ở đỉnh bằng $60^{\circ}$. Diện tích xung quanh của hình nón đã cho bằng
	\choice
	{\True $50 \pi$}
	{$\dfrac{100 \sqrt{3}\pi}{3}$}
	{$\dfrac{50 \sqrt{3}\pi}{3}$}
	{$100 \pi$}
	\loigiai{
		Ta có độ dài đường sinh là $l=\dfrac{r}{\sin \dfrac{\alpha}{2}}=\dfrac{5}{\sin 30^{\circ}}=10$.
		Diện tích xung quanh $S_{x q}=\pi r l=50 \pi$.
	}
\end{ex}
\begin{ex}
	[Mã 103 - 2020 Lần 1]%Câu 4.
	Cho hình nón có bán kính bằng 3 và góc ở đỉnh bằng $60^{\circ}$. Diện tích xung quanh của hình nón đã cho bằng
	\choice
	{\True $18\pi$}
	{$36\pi$}
	{$6\sqrt{3}\pi$}
	{$12\sqrt{3}\pi$}
	\loigiai{
		Gọi $l$ là đường sinh, $r$ là bán kính đáy ta có $r=3$.\\
		Gọi $\alpha$ là góc ở đỉnh. Ta có $\sin\alpha=\dfrac{r}{l}\Rightarrow l=\dfrac{r}{\sin\alpha}=\dfrac{3}{\sin{30}^{\circ}}=6$.\\
		Vậy diện tích xung quanh $S=\pi rl=\pi\cdot 3\cdot 6=18\pi$.
	}
\end{ex}
\begin{ex}
	[Mã 104 - 2020 Lần 1]%Câu 5.
	Cho hình nón có bán kính đáy bằng 4 và góc ở đỉnh bằng $60^{\circ}$. Diện tích xung quanh của hình nón đã cho bằng
	\choice
	{$\dfrac{64\sqrt{3}\pi}{3}$}
	{\True $32\pi$}
	{$64\pi$}
	{$\dfrac{32\sqrt{3}\pi}{3}$}
	\loigiai{
		{\color{red} HINH O DAY}\\
		Ta có Góc ở đỉnh bằng $60^{\circ}\Rightarrow\widehat{OSB}=30^{\circ}$.\\
		Độ dài đường sinh: $l=\dfrac{r}{\sin{30}^{\circ}}=\dfrac{4}{\frac{1}{2}}=8$.\\
		Diện tích xung quanh hình nón: $S_{\mathrm{xq}}=\pi rl=\pi\cdot 4\cdot 8=32\pi$.
	}
\end{ex}
\begin{ex}
	[Mã 123 2017]%Câu 6.
	Cho một hình nón có chiều cao $h=a$ và bán kính đáy $r=2a$. Mặt phẳng $(P)$ đi qua $S$ cắt đường tròn đáy tại $A$ và $B$ sao cho $AB=2\sqrt{3}a$. Tính khoảng cách $d$ từ tâm của đường tròn đáy đến $(P)$. 
	\choice
	{$d=\dfrac{\sqrt{3}a}{2}$}
	{$d=\dfrac{\sqrt{5}a}{5}$}
	{\True $d=\dfrac{\sqrt{2}a}{2}$}
	{$d=a$}
	\loigiai{
		{\color{red} HINH O DAY}\\
		Có $(P)\equiv(SAB)$.\\
		Ta có $SO=a=h, OA=OB=r=2a,AB=2a\sqrt{3}$, gọi $M$ là hình chiếu của $O$ lên $AB$ suy ra $M$ là trung điểm $AB$, gọi $K$ là hình chiếu của $O$ lên $SM$ suy ra $\mathrm{d}\left(O;(SAB)\right)=OK$.\\
		Ta tính được $OM=\sqrt{OA^2-MA^2}=a$ suy ra $SOM$ là tam giác vuông cân tại $O$, suy ra $K$ là trung điểm của $SM$ nên $OK=\dfrac{SM}{2}=\dfrac{a\sqrt{2}}{2}$.
	}
\end{ex}
\begin{ex}
	[KSCL THPT Nguyễn Khuyến 2019]%Câu 7.
	Cho hình nón đỉnh $S$, đường cao SO, $A$ và $B$ là hai điểm thuộc đường tròn đáy sao cho khoảng cách từ $O$ đến $(SAB)$ bằng $\dfrac{a\sqrt{3}}{3}$ và $\widehat{SAO}=30^{\circ},\widehat{SAB}=60^{\circ}$. Độ dài đường sinh của hình nón theo $a$ bằng
	\choice
	{\True $a\sqrt{2}$}
	{$a\sqrt{3}$}
	{$2a\sqrt{3}$}
	{$a\sqrt{5}$}
	\loigiai{
		{\color{red} HINH O DAY}\\
		Gọi $K$ là trung điểm của $AB$ ta có $OK\perp AB$ vì tam giác $OAB$ cân tại $O$.\\
		Mà $SO\perp AB$ nên $AB\perp(SOK)\Rightarrow(SOK)\perp(SAB)$ mà $\Rightarrow(SOK)\cap(SAB)=SK$ nên từ $O$ dựng $OH\perp SK$ thì $OH\perp(SAB)\Rightarrow OH=\mathrm{d}\left(O,(SAB)\right)$.\\
		Xét tam giác $SAO$ ta có $\sin\widehat{SAO}=\dfrac{SO}{SA}\Rightarrow SO=\dfrac{SA}{2}$.\\
		Xét tam giác $SAB$ ta có $\sin\widehat{SAB}=\dfrac{SK}{SA}\Rightarrow SK=\dfrac{SA\sqrt{3}}{2}$.\\
		Xét tam giác $SOK$ ta có $\dfrac{1}{OH^2}=\dfrac{1}{OK^2}+\dfrac{1}{OS^2}=\dfrac{1}{SK^2-SO^2}+\dfrac{1}{SO^2}$ \\
		$ \Rightarrow\dfrac{1}{OH^2}=\dfrac{1}{\dfrac{SA^2}{4}}+\dfrac{1}{\dfrac{3SA^2}{4}-\dfrac{SA^2}{4}}=\dfrac{4}{SA^2}+\dfrac{2}{SA^2}\Rightarrow\dfrac{6}{SA^2}=\dfrac{3}{a^2}\Rightarrow SA=2a^2\Rightarrow SA=a\sqrt{2} $.
	}
\end{ex}
\begin{ex}
	[THPT Cẩm Giàng 2 2019]%Câu 8.
	Cho một hình nón có bán kính đáy bằng $a$ và góc ở đỉnh bằng $60^{\circ}$. Tính diện tích xung quanh của hình nón đó. 
	\choice
	{$S_{\mathrm{xq}}=4\pi a^2$}
	{$S_{\mathrm{xq}}=\dfrac{2\sqrt{3}\pi a^2}{3}$}
	{$S_{\mathrm{xq}}=\dfrac{4\sqrt{3}\pi a^2}{3}$}
	{\True $S_{\mathrm{xq}}=2\pi a^2$}
	\loigiai{
		{\color{red} HINH O DAY}\\
		Giả sử hình nón có đỉnh là $S$, $O$ là tâm của đường tròn đáy và $AB$ là một đường kính của đáy.\\
		$r=OA=a$, $\widehat{ASB}=60^{\circ}\Rightarrow\widehat{ASO}=30^{\circ}$.\\
		Độ dài đường sinh là $l=SA=\dfrac{OA}{\sin 30^{\circ}}=2a$.\\
		Vậy diện tích xung quanh của hình nón là $S_{\mathrm{xq}}=\pi rl=\pi\cdot a\cdot 2a=2\pi a^2$.
	}
\end{ex}
\begin{ex}
	[THPT Cẩm Giàng 2 2019]%Câu 9.
	Cho đoạn thẳng $AB$ có độ dài bằng $2a$, vẽ tia $Ax$ về phía điểm $B$ sao cho điểm $B$ luôn cách tia $Ax$ một đoạn bằng $a$. Gọi $H$ là hình chiếu của $B$ lên tia $Ax$, khi tam giác $AHB$ quay quanh trục $AB$ thì đường gấp khúc $AHB$ vẽ thành mặt tròn xoay có diện tích xung quanh bằng 
	\choice
	{$\dfrac{3\sqrt{2}\pi a^2}{2}$}
	{\True $\dfrac{(3+\sqrt{3})\pi a^2}{2}$}
	{$\dfrac{(1+\sqrt{3})\pi a^2}{2}$}
	{$\dfrac{(2+\sqrt{2})\pi a^2}{2}$}
	\loigiai{
		{\color{red} HINH O DAY}\\.\\
		Xét tam giác $AHB$ vuông tại $H$. Ta có $AH=\sqrt{AB^2-HB^2}=a\sqrt{3}$.\\
		Xét tam giác $AHB$ vuông tại $H$, $HI\perp AB$ tại $I$ ta có $HI=\dfrac{AH\cdot HB}{AB}=\dfrac{a\sqrt{3}\cdot a}{2a}=\dfrac{a\sqrt{3}}{2}$.\\
		Khi tam giác $AHB$ quay quanh trục $AB$ thì đường gấp khúc $AHB$ vẽ thành mặt tròn xoay (có diện tích xung quanh là $S$) là hợp của hai mặt xung quanh của hình nón ($N_1$) và ($N_2$).\\
		Trong đó:\\
		($N_1$) là hình nón có được do quay tam giác $AHI$ quanh trục $AI$ có diện tích xung quanh là $S_1=\pi\cdot HI\cdot AH=\pi\cdot\dfrac{a\sqrt{3}}{2}\cdot a\sqrt{3}=\dfrac{3\pi a^2}{2}$.\\
		($N_2$) là hình nón có được do quay tam giác $BHI$ quanh trục $BI$ có diện tích xung quanh là $S_2=\pi\cdot HI\cdot BH=\pi\cdot\dfrac{a\sqrt{3}}{2}\cdot a=\dfrac{\sqrt{3}\pi a^2}{2}$ \\
		$ \Rightarrow S=S_1+S_2=\dfrac{3\pi a^2}{2}+\dfrac{\sqrt{3}\pi a^2}{2}=\dfrac{(3+\sqrt{3})\pi a^2}{2} $.
	}
\end{ex}
\begin{ex}
	[HSG Bắc Ninh 2019]%Câu 10.
	Cho hình nón có chiều cao $h=20$, bán kính đáy $r=25$. Một thiết diện đi qua đỉnh của hình nón có khoảng cách từ tâm của đáy đến mặt phẳng chứa thiết diện là $12$. Tính diện tích $S$ của thiết diện đó. 
	\choice
	{\True $S=500$}
	{$S=400$}
	{$S=300$}
	{$S=406$}
	\loigiai{
		Giả sử hình nón đỉnh $S$, tâm đáy $O$ và có thiết diện qua đỉnh thỏa mãn yêu cầu bài toán là $\triangle SAB$ (hình vẽ).\\
		{\color{red} HINH O DAY}\\
		Ta có $SO$ là đường cao của hình nón. Gọi $I$ là trung điểm của $AB\Rightarrow OI\perp AB$.\\
		Gọi $H$ là hình chiếu của $O$ lên $SI\Rightarrow OH\perp SI$.\\
		Ta chứng minh được $OH\perp(SAB)\Rightarrow OH=12$.\\
		Xét tam giác vuông $SOI$ có $\dfrac{1}{OH^2}=\dfrac{1}{OS^2}+\dfrac{1}{OI^2}\Rightarrow\dfrac{1}{OI^2}=\dfrac{1}{OH^2}-\dfrac{1}{OS^2} =\dfrac{1}{12^2}-\dfrac{1}{20^2} =\dfrac{1}{225}$ \\
		$ \Rightarrow OI^2=225\Rightarrow OI=15 $.\\
		Xét tam giác vuông $SOI$ có $SI=\sqrt{OS^2+OI^2} =\sqrt{20^2+15^2} =25$.\\
		Xét tam giác vuông $OIA$ có $IA=\sqrt{OA^2-OI^2} =\sqrt{25^2-15^2} =20\Rightarrow AB=40$.\\
		Ta có $S=S_{\triangle ABC} =\dfrac{1}{2}AB\cdot SI =\dfrac{1}{2}\cdot 40\cdot 25 =500$.
	}
\end{ex}
\begin{ex}
	[Liên Trường THPT TP Vinh Nghệ An 2019]%Câu 11.
	Cắt hình nón $(N)$ đỉnh $S$ cho trước bởi mặt phẳng qua trục của nó, ta được một tam giác vuông cân có cạnh huyền bằng $2a\sqrt{2}$. Biết $BC$ là một dây cung đường tròn của đáy hình nón sao cho mặt phẳng $(SBC)$ tạo với mặt phẳng đáy của hình nón một góc $60^{\circ}$. Tính diện tích tam giác $SBC$. 
	\choice
	{\True $\dfrac{4a^2\sqrt{2}}{3}$}
	{$\dfrac{4a^2\sqrt{2}}{9}$}
	{$\dfrac{2a^2\sqrt{2}}{3}$}
	{$\dfrac{2a^2\sqrt{2}}{9}$}
	\loigiai{
		{\color{red} HINH O DAY}\\
		Thiết diện qua trục của hình nón là tam giác vuông cân, suy ra $r=SO=a\sqrt{2}$.\\
		Ta có góc giữa mặt phẳng $(SBC)$ tạo với đáy bằng góc $\widehat{SIO}=60^{\circ}$.\\
		Trong tam giác $SIO$ vuông tại $O$ có $SI=\dfrac{SO}{\sin\widehat{SIO}}=\dfrac{2\sqrt{6}}{3}a$ và $OI=SI\cdot\cos\widehat{SIO}=\dfrac{\sqrt{6}}{3}a$.\\
		Mà $BC=2\sqrt{r^2-OI^2}=\dfrac{4\sqrt{3}}{3}a$.\\
		Diện tích tam giác $SBC$ là $S=\dfrac{1}{2}SI\cdot BC=\dfrac{4a^2\sqrt{2}}{3}$.
	}
\end{ex}
\begin{ex}
	[Sở Hà Nội 2019]%Câu 12.
	Cho hình nón tròn xoay có chiều cao bằng $4$ và bán kính bằng 3. Mặt phẳng $(P)$ đi qua đỉnh của hình nón và cắt hình nón theo thiết diện là một tam giác có độ dài cạnh đáy bằng $2$. Diện tích của thiết diện bằng 
	\choice
	{$\sqrt{6}$}
	{$\sqrt{19}$}
	{\True $2\sqrt{6}$}
	{$2\sqrt{3}$}
	\loigiai{
		{\color{red} HINH O DAY}\\
		Ta có $h=OI=4,R=IA=IB=3,AB=2$.\\
		Gọi $M$ là trung điểm AB $\Rightarrow MI\perp AB\Rightarrow AB\perp(SMI)\Rightarrow AB\perp SM$.\\
		Lại có: $SB=\sqrt{OI^2+IB^2}=\sqrt{4^2+3^2}=5$; $SM=\sqrt{SB^2-MB^2}=\sqrt{5^2-1^2}=2\sqrt{6}$.\\
		Vậy: $S_{\triangle SAB}=\dfrac{1}{2}\cdot SM\cdot AB=\dfrac{1}{2}\cdot 2\sqrt{6}\cdot 2=2\sqrt{6}$.
	}
\end{ex}
\begin{ex}
	[Chuyên Hạ Long 2019]%Câu 13.
	Cắt hình nón bằng một mặt phẳng qua trục của nó, ta được một thiết diện là một tam giác vuông cân cạnh bên $a\sqrt{2}$. Tính diện tích toàn phần của hình nón. 
	\choice
	{$4a^2\pi$ (đvdt)}
	{$4\sqrt{2}a^2\pi$ (đvdt)}
	{\True $a^2\pi(\sqrt{2}+1)$ (đvdt)}
	{$2\sqrt{2}a^2\pi$ (đvdt)}
	\loigiai{
		{\color{red} HINH O DAY}\\
		Giả sử hình nón đã cho có độ dài đường sinh $l$, bán kính đáy là $R$.\\
		Thiết diện của hình nón qua trục là tam giác $OAB$ vuông cân tại O và $OA=a\sqrt{2}$.\\
		Áp dụng định lý Pitago trong tam giác vuông cân $OAB$ ta có\\
		$AB^2=OA^2+OB^2=4a^2\Rightarrow AB=2a$.\\
		Vậy: $l=a\sqrt{2}, R=a$.\\
		Diện tích toàn phần của hình nón là\\
		$\pi Rl+\pi R^2=\pi a^2(\sqrt{2}+1)$ (đvdt).
	}
\end{ex}
\begin{ex}
	[Chuyên KHTN 2019]%Câu 14.
	Cho hình lập phương $ABCD.A'B'C'D'$ cạnh $a$. Tính diện tích toàn phần của vật tròn xoay thu được khi quay tam giác $AA'C$ quanh trục $AA'$. 
	\choice
	{$\pi(\sqrt{3}+2)a^2$}
	{$2\pi(\sqrt{2}+1)a^2$}
	{$2\pi(\sqrt{6}+1)a^2$}
	{\True $\pi(\sqrt{6}+2)a^2$}
	\loigiai{
		{\color{red} HINH O DAY}\\		
		Quay tam giác $AA'C$ một vòng quanh trục $AA'$ tạo thành hình nón có chiều cao $AA'=a$, bán kính đáy $r=AC=a\sqrt{2}$, đường sinh $l=A'C=\sqrt{AA'^2+AC^2}=a\sqrt{3}$.\\
		Diện tích toàn phần của hình nón: $S=\pi r(r+l)=\pi a\sqrt{2}\left(a\sqrt{2}+a\sqrt{3}\right)=\pi(\sqrt{6}+2)a^2$.
	}
\end{ex}
\begin{ex}%Câu 15.
	Cho hình nón có chiều cao và bán kính đáy đều bằng $1$. Mặt phẳng $(P)$ qua đỉnh của hình nón và cắt đáy theo dây cung có độ dài bằng $1$. Khoảng cách từ tâm của đáy tới mặt phẳng $(P)$ bằng
	\choice
	{$\dfrac{\sqrt{7}}{7}$}
	{$\dfrac{\sqrt{2}}{2}$}
	{$\dfrac{\sqrt{3}}{3}$}
	{\True $\dfrac{\sqrt{21}}{7}$}
	\loigiai{
		{\color{red} HINH O DAY}\\
		Ta có $l=h=1$.\\
		Mặt phẳng $(P)$ qua đỉnh của hình nón và cắt đáy theo dây cung $AB$ có độ dài bằng $1$. $I$, $K$ là hình chiếu $O$ lên $AB$; $SI$. Ta có $AB\perp(SIO)\Rightarrow OK\perp(SAB)$.\\
		ta có $IO=\sqrt{R^2-OA^2}=\sqrt{1^2-\left(\dfrac{1}{2}\right)^2}=\dfrac{\sqrt{3}}{2}$.\\
		$\dfrac{1}{OK^2}=\dfrac{1}{OI^2}+\dfrac{1}{OS^2}\Rightarrow OK=\dfrac{OI\cdot SO}{\sqrt{OI^2+OS^2}}=\dfrac{\sqrt{21}}{7}$.
	}
\end{ex}
\begin{ex}
	Cho hình nón đỉnh $S$, đáy là đường tròn $(O; 5)$. Một mặt phẳng đi qua đỉnh của hình nón cắt đường tròn đáy tại hai điểm $A$ và $B$ sao cho $SA=AB=8$. Tính khoảng cách từ $O$ đến $(SAB)$. 
	\choice
	{$2\sqrt{2}$}
	{\True $\dfrac{3\sqrt{3}}{4}$}
	{$\dfrac{3\sqrt{2}}{7}$}
	{$\dfrac{\sqrt{13}}{2}$}
	\loigiai{
		{\color{red} HINH O DAY}\\
		Gọi $I$ là trung điểm $AB$.\\
		Ta có $\heva{&AB\perp SO\\&AB\perp OI}\Rightarrow AB\perp(SOI)\Rightarrow(SAB)\perp(SOI)$.\\
		Trong $(SOI)$, kẻ $OH\perp SI$ thì $OH\perp(SAB)$ \\
		$ \Rightarrow\mathrm{d}\left(O;(SAB)\right)=OH $.\\
		Ta có: $SO=\sqrt{SA^2-OA^2}=\sqrt{\left(\dfrac{8\cdot 5}{5}\right)^2-5^2}=\sqrt{39}$.\\
		Ta có: $OI=\sqrt{OA^2-AI^2}=\sqrt{5^2-\left(\dfrac{4\cdot 5}{5}\right)^2}=3$.\\
		Tam giác vuông $SOI$ có: $\dfrac{1}{OH^2}=\dfrac{1}{OI^2}+\dfrac{1}{SO^2}\Rightarrow OH=\dfrac{3\sqrt{13}}{4}$.\\
		Vậy $\mathrm{d}\left(O;(SAB)\right)=OH=\dfrac{3\sqrt{13}}{4}$.
	}
\end{ex}
\begin{ex}
	[Chuyên ĐHSPHN - 2018]%Câu 17.
	Cho hình nón đỉnh $S$, đáy là hình tròn tâm $O$, bán kính, $R=3$ $\mathrm{cm}$, góc ở đỉnh hình nón là $\varphi=120^{\circ}$. Cắt hình nón bởi mặt phẳng qua đỉnh $S$ tạo thành tam giác đều $SAB$, trong đó $A$, $B$ thuộc đường tròn đáy. Diện tích tam giác $SAB$ bằng
	\choice
	{\True $3\sqrt{3}$ $\mathrm{cm}^2$}
	{$6\sqrt{3}$ $\mathrm{cm}^2$}
	{$6$ $\mathrm{cm}^2$}
	{$3$ $\mathrm{cm}^2$}
	\loigiai{
		{\color{red} HINH O DAY}\\
		Theo đề bài ta có góc ở đỉnh hình nón là $\varphi=120^{\circ}$ và khi cắt hình nón bởi mặt phẳng qua đỉnh $S$ tạo thành tam giác đều $SAB$ nên mặt phẳng không chứa trục của hình nón.\\
		Do góc ở đỉnh hình nón là $\varphi=120^{\circ}$ nên $\widehat{OSC}=60^{\circ}$.\\
		Xét tam giác vuông $SOC$ ta có $\tan\widehat{OSC}=\dfrac{OC}{SO}\Rightarrow SO=\dfrac{OC}{\tan\widehat{OSC}} =\dfrac{3}{\tan 60^{\circ}} =\sqrt{3}$.\\
		Xét tam giác vuông $SOA$ ta có $SA=\sqrt{SO^2+OA^2} =2\sqrt{3}$.\\
		Do tam giác $SAB$ đều nên $S_{\triangle SAB}=\dfrac{1}{2}(2\sqrt{3})^2\cdot\sin 60^{\circ} =3\sqrt{3}$ $\left(\mathrm{cm}^2\right)$.
	}
\end{ex}
\begin{ex}
	[Chuyên Nguyễn Quang Diêu - Đồng Tháp - 2018]%Câu 18.
	Cho hình nón có thiết diện qua trục là tam giác vuông có cạnh huyền bằng $a\sqrt{2}$. Tính diện tích xung quanh $S_{\mathrm{xq}}$ của hình nón đó. 
	\choice
	{$S_{\mathrm{xq}}=\dfrac{\pi a^2\sqrt{3}}{3}$}
	{\True $S_{\mathrm{xq}}=\dfrac{\pi a^2\sqrt{2}}{2}$}
	{$S_{\mathrm{xq}}=\dfrac{\pi a^2\sqrt{2}}{6}$}
	{$S_{\mathrm{xq}}=\dfrac{\pi a^2\sqrt{2}}{3}$}
	\loigiai{
		{\color{red} HINH O DAY}\\
		Gọi $S$ là đỉnh hình nón, thiết diện qua trục là tam giác $SAB$.\\
		Ta có $AB=a\sqrt{2}\Rightarrow SA=a$, suy ra $l=SA=a$; $r=\dfrac{AB}{2}=\dfrac{a\sqrt{2}}{2}$.\\
		Vậy $S_{\mathrm{xq}}=\pi rl=\pi\cdot\dfrac{a\sqrt{2}}{2}\cdot a=\dfrac{\pi a^2\sqrt{2}}{2}$.
	}
\end{ex}
\begin{ex}
	[Chuyên Nguyễn Bỉnh Khiêm - Quảng Nam - 2020]%Câu 19.
	Cho hình nón đỉnh $S$ có đáy là hình tròn tâm $O,$ bán kính $R$. Dựng hai đường sinh $SA$ và $SB,$ biết $AB$ chắn trên đường tròn đáy một cung có số đo bằng $60^{\circ},$ khoảng cách từ tâm $O$ đến mặt phẳng $(SAB)$ bằng $\dfrac{R}{2}$. Đường cao $h$ của hình nón bằng
	\choice
	{$h=R\sqrt{3}$}
	{$h=R\sqrt{2}$}
	{$h=\dfrac{R\sqrt{3}}{2}$}
	{\True $h=\dfrac{R\sqrt{6}}{4}$}
	\loigiai{
		{\color{red} HINH O DAY}\\
		Gọi $I$ là trung điểm $AB$.\\
		Kẻ $OH$ vuông góc với $SI$.\\
		$\mathrm{d}\left(O,(SAB)\right)=OH=\dfrac{R}{2}$.\\
		Ta có cung $AB$ bằng $60^{\circ}$ nên $\widehat{AOB}=60^{\circ}$.\\
		Tam giác $AOI$ vuông tại $I,$ ta có $\cos\widehat{IOA}=\dfrac{OI}{OA}\Leftrightarrow OI=OA\cdot\cos 30^{\circ}=\dfrac{\sqrt{3}R}{2}$.\\
		Tam giác $SOI$ vuông tại $O,$ ta có\\
		$\dfrac{1}{OH^2}=\dfrac{1}{SO^2}+\dfrac{1}{OI^2}\Leftrightarrow\dfrac{1}{SO^2}=\dfrac{1}{OH^2}-\dfrac{1}{OI^2}=\dfrac{1}{\left(\dfrac{R}{2}\right)^2}-\dfrac{1}{\left(\dfrac{\sqrt{3}R}{2}\right)^2}=\dfrac{8}{3R^2}\Rightarrow SO=\dfrac{\sqrt{6}R}{4}$.
	}
\end{ex}
\begin{ex}
	[Chuyên Bắc Ninh - 2020]%Câu 20.
	Cho hình nón tròn xoay có chiều cao bằng $2a$, bán kính đáy bằng $3a$. Một thiết diện đi qua đỉnh của hình nón có khoảng cách từ tâm của đáy đến mặt phẳng chứa thiết diện bằng $\dfrac{3a}{2}$. Diện tích của thiết diện đó bằng
	\choice
	{$\dfrac{2a^2\sqrt{3}}{7}$}
	{$12a^2\sqrt{3}$}
	{$\dfrac{12a^2}{7}$}
	{\True $\dfrac{24a^2\sqrt{3}}{7}$}
	\loigiai{
		Xét hình nón đỉnh $S$ có chiều cao $SO=2a$, bán kính đáy $OA=3a$.\\
		Thiết diện đi qua đỉnh của hình nón là tam giác $SAB$ cân tại $S$.\\
		Gọi $I$ là trung điểm của đoạn thẳng $AB$. Trong tam giác $SOI$, kẻ $OH\perp SI$, $H\in SI$.\\
		$\heva{&AB\perp OI\\&AB\perp SO}\Rightarrow AB\perp(SOI)\Rightarrow AB\perp OH$.\\
		$\heva{&OH\perp SI\\&OH\perp AB}\Rightarrow OH\perp(SAB)\Rightarrow\mathrm{d}\left(O,(SAB)\right)=OH=\dfrac{3a}{2.}$ \\
		Xét tam giác $SOI$ vuông tại $O$, ta có $\dfrac{1}{OI^2}=\dfrac{1}{OH^2}-\dfrac{1}{SO^2} =\dfrac{4}{9a^2}-\dfrac{1}{4a^2}=\dfrac{7}{36a^2}\Rightarrow OI=\dfrac{6a}{\sqrt{7}}$.\\
		$SI=\sqrt{SO^2+OI^2}=\sqrt{4a^2+\dfrac{36a^2}{7}}=\dfrac{8a}{\sqrt{7}}$.\\
		Xét tam giác $AOI$ vuông tại $I$, $AI=\sqrt{AO^2-OI^2}=\sqrt{9a^2-\dfrac{36a^2}{7}}=\dfrac{3\sqrt{3}a}{\sqrt{7}}$ \\
		$ \Rightarrow AB=2AI=\dfrac{6\sqrt{3}a}{\sqrt{7}} $.\\
		Vậy diện tích của thiết diện là $S_{\triangle SAB}=\dfrac{1}{2}\cdot SI\cdot AB=\dfrac{1}{2}\cdot\dfrac{8a}{\sqrt{7}}\cdot\dfrac{6\sqrt{3}a}{\sqrt{7}}=\dfrac{24a^2\sqrt{3}}{7}$.
	}
\end{ex}
\begin{ex}
	[Sở Phú Thọ - 2020]%Câu 21.
	Cho hình nón đỉnh $S$ có đáy là hình tròn tâm $O$. Một mặt phẳng đi qua đỉnh của hình nón và cắt hình nón theo thiết diện là một tam giác vuông $SAB$ có diện tích bằng $4a^2$. Góc giữa trục $SO$ và mặt phẳng $(SAB)$ bằng $30^{\circ}$. Diện tích xung quanh của hình nón đã cho bằng
	\choice
	{$4\sqrt{10}\pi a^2$}
	{\True $2\sqrt{10}\pi a^2$}
	{$\sqrt{10}\pi a^2$}
	{$8\sqrt{10}\pi a^2$}
	\loigiai{
		{\color{red} HINH O DAY}\\
		Gọi $M$ là trung điểm của $AB$, tam giác $OAB$ cân đỉnh $O$ nên $OM\perp AB$ và $SO\perp AB$ suy ra $AB\perp(SOM)$.\\
		Dựng $OK\perp SM$.\\
		Theo trên có $OK\perp AB$ nên $OK\perp(SAB)$.\\
		Vậy góc tạo bởi giữa trục $SO$ và mặt phẳng $(SAB)$ là $\widehat{OSM}=30^{\circ}$.\\
		Tam giác vuông cân $SAB$ có diện tích bằng $4a^2$ suy ra $\dfrac{1}{2}SA^2=4a^2\Rightarrow SA=2a\sqrt{2}\Rightarrow AB=4a\Rightarrow SM=2a$.\\
		Xét tam giác vuông $SOM$ có $\cos\widehat{OSM}=\dfrac{SO}{SM}\Rightarrow SO=\dfrac{\sqrt{3}}{2}\cdot 2a=\sqrt{3}a$.\\
		Cuối cùng $OB=\sqrt{SB^2-SO^2}=a\sqrt{5}$.\\
		Vậy diện tích xung quanh của hình nón bằng $S_{\mathrm{xq}}=\pi rl=\pi\cdot a\sqrt{5}\cdot 2a\sqrt{2}=2a^2\sqrt{10}\pi$.
	}
\end{ex}
\begin{ex}
	[Bỉm Sơn - Thanh Hóa - 2020]%Câu 22
	Thiết diện qua trục của một hình nón là một tam giác vuông cân có cạnh huyền bằng $a \sqrt{2}$. Một thiết diện qua đỉnh tạo với đáy một góc $60^{\circ}$. Diện tích của thiết diện này bằng
	\choice
	{\True $\dfrac{a^2 \sqrt{2}}{3}$}
	{$\dfrac{a^2 \sqrt{2}}{2}$}
	{$2 a^2$}
	{$\dfrac{a^2 \sqrt{2}}{4}$}
	\loigiai{
		\textbf{(VẼ HÌNH)}\\
		Giả sử hình nón có đỉnh $S$, tâm đường tròn đáy là $O$. Thiết diện qua trục là $\triangle S A B$, thiết diện qua đỉnh là $\triangle S C D$; gọi $I$ là trung điểm của $C D$.\\
		Theo giả thiết ta có $\triangle S A B$ vuông cân tại $S$, cạnh huyền $A B=a \sqrt{2}\Rightarrow r=O A=\dfrac{a \sqrt{2}}{2}$ $S A=S B=l=a \Rightarrow h=S O=\sqrt{S A^2-O A^2}=\sqrt{a^2-\dfrac{2 a^2}{4}}=\dfrac{a \sqrt{2}}{2}$.\\
		Ta lại có $S I O=60^{\circ}\Rightarrow \sin 60^{\circ}=\dfrac{S O}{S I}\Rightarrow S I=\dfrac{S O}{\sin 60^{\circ}}=\dfrac{\frac{a \sqrt{2}}{2}}{\frac{\sqrt{3}}{2}}=\dfrac{a \sqrt{6}}{3}$; $I D=\sqrt{S D^2-S I^2}=\sqrt{a^2-\dfrac{6 a^2}{9}}=\dfrac{a \sqrt{3}}{3}\Rightarrow C D=\dfrac{2 a \sqrt{3}}{3}$.\\
		Diện tích thiết diện cần tìm là $S_{\triangle S C D}=\dfrac{1}{2}\cdot C D \cdot S I=\dfrac{1}{2}\cdot \dfrac{2 a \sqrt{3}}{3}\cdot \dfrac{a \sqrt{6}}{3}=\dfrac{a^2 \sqrt{2}}{3}$.
	}
\end{ex}
\begin{dang}
	{Thể tích}
\end{dang}
\begin{ex}
	[Đề Minh Họa 2020 Lần 1]%Câu 1.
	Cho hình nón có chiều cao bằng $2\sqrt{5}$. Một mặt phẳng đi qua đỉnh hình nón và cắt hình nón theo một thiết diện là tam giác đều có diện tích bằng $9\sqrt{3}$. Thể tích của khối nón được giới hạn bởi hình nón đã cho bằng
	\choice
	{\True $\dfrac{32\sqrt{5}\pi}{3}$}
	{$32\pi$}
	{$32\sqrt{5}\pi$}
	{$96\pi$}
	\loigiai{
		{\color{red} HINH O DAY}\\
		Theo giả thiết tam giác $SAB$ đều, $S_{\triangle SAB}=9\sqrt{3}$ và $SO=2\sqrt{5}$.\\
		$S_{\triangle SAB}=9\sqrt{3}\Leftrightarrow\dfrac{AB^2\sqrt{3}}{4}=9\sqrt{3}\Leftrightarrow AB=6$.\\
		$\triangle SAB$ đều $SA=AB=6$.\\
		Xét $\triangle SOA$ vuông tại $O$, theo định lý Pytago ta có: $OA=\sqrt{SA^2-SO^2}=\sqrt{6^2-(2\sqrt{5})^2}=4$.\\
		Thể tích hình nón bằng $V=\dfrac{1}{3}\pi r^2h=\dfrac{1}{3}\pi\cdot OA^2\cdot SO=\dfrac{1}{3}\pi 4^2\cdot 2\sqrt{5}=\dfrac{32\sqrt{5}}{3}\pi$.
	}
\end{ex}
\begin{ex}
	[KSCL THPT Nguyễn Khuyến 2019]%Câu 2.
	Tính thể tích của hình nón có góc ở đỉnh bằng $60^{\circ}$ và diện tích xung quanh bằng $6\pi a^2$. 
	\choice
	{$V=\dfrac{3\pi a^3\sqrt{2}}{4}$}
	{\True $V=3\pi a^3$}
	{$V=\dfrac{3\pi a^3\sqrt{2}}{4}$}
	{$V=\pi a^3$}
	\loigiai{
		Khối nón có góc ở đỉnh bằng $60^{\circ}$ nên góc tạo bởi đường sinh và đáy bằng $60^{\circ}$.\\
		Vậy $R=\dfrac{l}{2}$; lại có $S_{\mathrm{xq}}=\pi Rl=\pi R\cdot 2R=6\pi a^2$ nên $R=a\sqrt{3}$; vậy $h=\sqrt{l^2-R^2}=R\sqrt{3}=3a$.\\
		Vậy $V=\dfrac{1}{3}\pi R^2h=3\pi a^3$.
	}
\end{ex}
\begin{ex}
	[Chuyên Thái Nguyên 2019]%Câu 3.
	Cho tam giác $ABC$ vuông tại $A$, cạnh $AB=6$, $AC=8$ và $M$ là trung điểm của cạnh $AC$. Khi đó thể tích của khối tròn xoay do tam giác $BMC$ quanh quanh $AB$ là
	\choice
	{$86\pi$}
	{$106\pi$}
	{\True $96\pi$}
	{$98\pi$}
	\loigiai{
		{\color{red} HINH O DAY}\\
		Khi tam giác $BMC$ quanh quanh trục $AB$ thì thể tích khối tròn xoay tạo thành là hiệu của thể tích khối nón có đường cao $AB$, đường sinh $BC$ và khối nón có đường cao $AB$, đường sinh $BM$. Nên $V=\dfrac{1}{3}AB\cdot\pi\cdot AC^2-\dfrac{1}{3}AB\cdot\pi\cdot AM^2=\dfrac{1}{4}AB\cdot\pi\cdot AC^2=96\pi$.
	}
\end{ex}
\begin{ex}
	[Chuyên Lê Quý Đôn Điện Biên 2019]%Câu 4.
	Cho hình nón có bán kính đáy bằng $2$ $\mathrm{cm}$, góc ở đỉnh bằng $60^{\circ}$. Tính thể tích của khối nón đó. 
	\choice
	{$\dfrac{8\sqrt{3}\pi}{9}$ $\mathrm{cm}^3$}
	{$8\sqrt{3}\pi$ $\mathrm{cm}^3$}
	{\True $\dfrac{8\sqrt{3}\pi}{3}$ $\mathrm{cm}^3$}
	{$\dfrac{8\pi}{3}$ $\mathrm{cm}^3$}
	\loigiai{
		{\color{red} HINH O DAY}\\.\\
		Cắt hình nón bởi một mặt phẳng đi qua trục, ta được thiết diện là tam giác $ABC$ cân tại đỉnh $A$ của hình nón.\\
		Do góc ở đỉnh của hình nón là $\widehat{BAC}=60^{\circ}$, suy ra $\widehat{HAC}=30^{\circ}$. Bán kính đáy $R=HC=2$ cm.\\
		Xét $\triangle AHC$ vuông tại $H$, ta có $AH=\dfrac{HC}{\tan 30^{\circ}} =\dfrac{2}{\dfrac{1}{\sqrt{3}}} =2\sqrt{3}$ cm.\\
		Thể tích của khối nón: $V=\dfrac{1}{3}\pi R^2\cdot AH =\dfrac{8\sqrt{3}\pi}{3}$ $\mathrm{cm}^3$.
	}
\end{ex}
\begin{ex}
	[Việt Đức Hà Nội 2019]%Câu 5.
	Cho tam giác $ABC$ vuông tại $A$, $AB=6cm, AC=8$ $\mathrm{cm}$. Gọi $V_1$ là thể tích khối nón tạo thành khi quay tam giác $ABC$ quanh cạnh $AB$ và $V_2$ là thể tích khối nón tạo thành khi quay tam giác $ABC$ quanh cạnh $AC$. Khi đó, tỷ số $\dfrac{V_1}{V_2}$ bằng 
	\choice
	{$\dfrac{3}{4}$}
	{\True $\dfrac{4}{3}$}
	{$\dfrac{16}{9}$}
	{$\dfrac{9}{16}$}
	\loigiai{
		{\color{red} HINH O DAY}\\{\color{red} HINH O DAY}\\.\\
		Ta có công thức tính thể tích khối nón có chiều cao $h$ và bán kính $r$ là $V=\dfrac{1}{3}\pi r^2h$.\\
		Khi quay tam giác $ABC$ quanh cạnh $AB$ thì:\\
		$h=AB=6$ $\mathrm{cm}$ và $r=AC=8$ $\mathrm{cm}$ thì $V_1=\dfrac{1}{3}\pi\cdot 8^2\cdot 6=128\pi$.\\
		Khi quay tam giác $ABC$ quanh cạnh $AC$ thì:\\
		$h=AC=8$ $\mathrm{cm}$ và $r=AB=6$ $\mathrm{cm}$ thì $V_2=\dfrac{1}{3}\pi\cdot 6^2\cdot 8=96\pi$.\\
		Vậy: $\dfrac{V_1}{V_2}=\dfrac{4}{3}$.
	}
\end{ex}
\begin{ex}
	[Việt Đức Hà Nội 2019]%Câu 6.
	Cho hình nón $N_1$ đỉnh $S$ đáy là đường tròn $C(O; R)$, đường cao $SO=40$ $\mathrm{cm}$. Người ta cắt nón bằng mặt phẳng vuông góc với trục để được nón nhỏ $N_2$ có đỉnh $S$ và đáy là đường tròn $C'(O'; R')$. Biết rằng tỷ số thể tích $\dfrac{V_{N_2}}{V_{N_1}}=\dfrac{1}{8}$. Tính độ dài đường cao nón $N_2$. 
	\choice
	{\True $20$ $\mathrm{cm}$}
	{$5$ $\mathrm{cm}$}
	{$10$ $\mathrm{cm}$}
	{$49$ $\mathrm{cm}$}
	\loigiai{
		{\color{red} HINH O DAY}\\
		Ta có $V_{N_1}=\dfrac{1}{3}\pi R^2\cdot SO$, $V_{N_2}=\dfrac{1}{3}\pi R'^2\cdot SO'$.\\
		Mặt khác, $\triangle SO'A$ và $\triangle SOB$ đồng dạng nên $\dfrac{R'}{R}=\dfrac{SO'}{SO}$.\\
		Suy ra: $\dfrac{V_{N_2}}{V_{N_1}}==\dfrac{R'^2\cdot SO'}{R^2\cdot SO}=\left(\dfrac{SO'}{SO}\right)^3=\dfrac{1}{8}$.
	}
\end{ex}
\begin{ex}
	Suy ra $\dfrac{SO'}{SO}=\dfrac{1}{2}\Rightarrow SO'=\dfrac{1}{2}\cdot 40=20$ $\mathrm{cm}$. Do đó (THPT Lê Quy Đôn Điện Biên 2019) Cho một đồng hồ cát như bên dưới (gồm hai hình nón chung đỉnh ghép lại), trong đó đường sinh bất kỳ của hình nón tạo với đáy một góc $60^{\circ}$. Biết rằng chiều cao của đồng hồ là $30$ $\mathrm{cm}$ và tổng thể tích của đồng hồ là $1000\pi$ $\mathrm{cm}^3$. Hỏi nếu cho đầy lượng cát vào phần bên trên thì khi chảy hết xuống dưới, tỷ số thể tích lượng cát chiếm chỗ và thể tích phần phía dưới là bao nhiêu?
	{\color{red} HINH O DAY}
	\choice
	{$\dfrac{1}{64}$}
	{\True $\dfrac{1}{8}$}
	{$\dfrac{1}{27}$}
	{$\dfrac{1}{3\sqrt{3}}$}
	\loigiai{
		Gọi $r_1,h_1,r_2,h_2$ lần lượt là bán kính, đường cao của hình nón trên và hình nón dưới.\\
		Do đường sinh bất kỳ của hình nón tạo với đáy một góc $60^{\circ}$.\\
		Suy ra: $\widehat{OAI'}=\widehat{OBI}=60^{\circ}$, khi đó ta có mối liên hệ: $h_1=\sqrt{3}r_1, h_2=\sqrt{3}r_2$.\\
		Theo đề ta có: $V=V_1+V_2=\dfrac{1}{3}\pi\left(h_1r_1^2+h_2r_2^2\right)=\dfrac{1}{9}\pi\left(h_1^3+h_2^3\right)=1000\pi$.\\
		Mà: $\left(h_1^3+h_2^3\right)=(h_1+h_2)^3-3(h_1+h_2)\cdot h_1h_2\Rightarrow h_1\cdot h_2=200$.\\
		Kết hợp giả thiết: $h_1+h_2=30$ ta được $\heva{&h_1=10\\&h_2=20.}$ \\
		Từ đó tỉ lệ cần tìm là $\dfrac{V_1}{V_2}=\dfrac{(10\sqrt{3})^2\cdot h_1}{(20\sqrt{3})^2\cdot h_2}=\dfrac{1}{4}\cdot\dfrac{1}{2}=\dfrac{1}{8}$.
	}
\end{ex}
\begin{ex}
	Cho hình chữ nhật $ABCD$ có $AB=2$, $AD=2\sqrt{3}$ và nằm trong măt phẳng $(P)$. Quay $(P)$ một vòng quanh đường thẳng $BD$. Khối tròn xoay được tạo thành có thể tích bằng
	\choice
	{$\dfrac{28\pi}{9}$}
	{$\dfrac{28\pi}{3}$}
	{\True $\dfrac{56\pi}{9}$}
	{$\dfrac{56\pi}{3}$}
	\loigiai{
		{\color{red} HINH O DAY}\\
		Khối nón đỉnh $D$, tâm đáy $I$ có thể tích $V_1$.\\
		Ta có $BD=4$ mà $IC'\cdot BD=BC'\cdot C'D\Rightarrow IC'=\sqrt{3}$.\\
		$ID=\dfrac{DC'^2}{BD}=1$ nên $V_1=\dfrac{1}{3}\pi\cdot IC'^2\cdot ID=\pi$.\\
		Khối nón cụt có tâm đáy $J,I$ có thể tích $V_2$.\\
		Ta có $DI=3,DJ=2$, $\dfrac{JE}{IC'}=\dfrac{DJ}{DI}=\dfrac{2}{3}\Rightarrow JE=\dfrac{2\sqrt{3}}{3}$.\\
		$V_2=\dfrac{1}{3}\pi\left(IC'^2\cdot DI-JE^2\cdot DJ\right)=\dfrac{19\pi}{9}$.\\
		Vậy thể tích cần tìm là $V=2(V_1+V_2)=\dfrac{56}{9}\pi$.
	}
\end{ex}
\begin{ex}
	[Chuyên Nguyễn Trãi Hải Dương 2019]%Câu 9.
	Cho hình chữ nhật $ABCD$ có $AB=2$, $AD=2\sqrt{3}$ và nằm trong mặt phẳng $(P)$. Quay $(P)$ một vòng quanh đường thẳng $BD$. Khối tròn xoay được tạo thành có thể tích bằng
	\choice
	{$\dfrac{28\pi}{9}$}    
	{$\dfrac{28\pi}{3}$}
	{\True $\dfrac{56\pi}{9}$}
	{$\dfrac{56\pi}{3}$}
	\loigiai{
		Gọi điểm như hình vẽ
		{\color{red} HINH O DAY}\\
		$V_1,V_2$ lần lượt là thể tích khói nón, nón cụt nhận được khi quay tam giác $ABH$ và tứ giác $AHLT$ quay $BD$.\\
		Ta có: $AH=\sqrt{3},IL=\dfrac{2}{\sqrt{3}},BH=HL=1$.\\
		Ta có: $V=2(V_1+V_2) =2\left[\dfrac{1}{3}BH\cdot\pi\cdot AH^2+\dfrac{1}{3}HL\cdot\pi\cdot\left(IL^2+IL\cdot AH+AH^2\right)\right] =2\left[\dfrac{1}{3`}\cdot 1\cdot\pi\cdot 3+\dfrac{1}{3}\cdot 1\cdot\pi\cdot\left(\dfrac{4}{3}+2+3\right)\right]=\dfrac{56\pi}{9}$.
	}
\end{ex}
\begin{ex}
	[Cụm 8 Trường Chuyên 2019]%Câu 10.
	Cho hình thang $ABCD$ có $\widehat{A}=\widehat{B}=90^{\circ}$, $AB=BC=a$, $AD=2a$. Tính thể tích khối tròn xoay sinh ra khi quay hình thang $ABCD$ xung quanh trục $CD$. \\
	{\color{red} HINH O DAY}
	\choice
	{\True $\dfrac{7\sqrt{2}\pi a^3}{6}$}
	{$\dfrac{7\sqrt{2}\pi a^3}{12}$}
	{$\dfrac{7\pi a^3}{6}$}
	{$\dfrac{7\pi a^3}{12}$}
	\loigiai{
		{\color{red} HINH O DAY}\\
		{\color{red} HINH O DAY}\\
		Gọi $E$ là giao điểm của $AB$ và $CD$. Gọi $F$ là hình chiếu vuông góc của $B$ trên $CE$.\\
		Ta có: $\triangle BCF=\triangle BEF$ nên tam giác $\triangle BCF$ và $\triangle BEF$ quay quanh trục $CD$ tạo thành hai khối nón bằng nhau có thể tích $V_1$.\\
		$\triangle ADC=\triangle AEC$ nên tam giác $\triangle ADC$ và $\triangle AEC$ quay quanh trục $CD$ tạo thành hai khối nón bằng nhau có thể tích $V$.\\
		Nên thể tích khối tròn xoay sinh ra khi quay hình thang $ABCD$ xung quanh trục $CD$ bằng $2V-2V_1=2\cdot\dfrac{1}{3}\pi\left(CD\cdot AC^2-CF\cdot BF^2\right) =\dfrac{2}{3}\pi\left[(a\sqrt{2})^3-\left(\dfrac{a}{\sqrt{2}}\right)^3\right]=\dfrac{7\sqrt{2}\pi a^3}{6}$.
	}
\end{ex}
\begin{ex}
	[KTNL GV Thpt Lý Thái Tổ 2019]%Câu 11.
	Cho hình tứ diện $ABCD$ có $AD\perp(ABC)$, $ABC$ là tam giác vuông tại $B$. Biết $BC=2(cm),AB=2\sqrt{3}(cm),AD=6(cm)$. Quay các tam giác $ABC$ và $ABD$ (bao gồm cả điểm bên trong $2$ tam giác) xung quanh đường thẳng $AB$ ta được $2$ khối tròn xoay. Thể tích phần chung của $2$ khối tròn xoay đó bằng
	\choice
	{$\sqrt{3}\pi (cm^3)$}
	{$\dfrac{5\sqrt{3}}{2}\pi (cm^3)$}
	{\True $\dfrac{3\sqrt{3}}{2}\pi (cm^3)$}
	{$\dfrac{64\sqrt{3}}{3}\pi (cm^3)$}
	\loigiai{
		{\color{red} HINH O DAY}\\
		Dễ thấy $AD\perp(ABC)\Rightarrow AD=R_1$.\\
		Gọi $\{M\}=BD\cap AC$ và N là hình chiếu của M trên AB. Dễ dàng chứng minh được tỉ lệ:\\
		$\dfrac{MN}{BC}=\dfrac{AN}{AB} (1)$; và $\dfrac{MN}{AD}=\dfrac{BN}{AB} (2)\Rightarrow\dfrac{(1)}{(2)}=\dfrac{AD}{BC}=\dfrac{AN}{BN}=3\Rightarrow\dfrac{AN}{AB}=\dfrac{3}{4};\dfrac{BN}{AB}=\dfrac{1}{4}$ \\
		$ \Rightarrow AN=\dfrac{3\sqrt{3}}{2}; BN=\dfrac{\sqrt{3}}{2}; MN=\dfrac{3}{2} $.\\
		Phần thể tích chung của 2 khối tròn xoay là phần thể tích khi quay tam giác xung quanh trục AB. Gọi $V_1$ là thể tích khối tròn xoay khi quay tam giác xung quanh AB.\\
		Và $V_2$ là thể tích khối tròn xoay khi quay tam giác xung quanh AB.
	}
\end{ex}
\begin{ex}
	Dễ tính được $V_1=\dfrac{3\sqrt{3}\pi}{8}(\mathrm{\,d}vtt)$ và $V_2=\dfrac{9\sqrt{3}\pi}{8}(\mathrm{\,d}vtt)\Rightarrow V_1+V_2=\dfrac{3\sqrt{3}\pi}{2}(\mathrm{\,d}vtt)$. (Chuyên Thái Bình - 2018) Cho hình nón có góc ở đỉnh bằng $60^{\circ},$ diện tích xung quanh bằng $6\pi a^2$. Tính thể tích $V$ của khối nón đã cho. 
	\choice
	{$V=\dfrac{3\pi a^3\sqrt{2}}{4}$}
	{$V=\dfrac{\pi a^3\sqrt{2}}{4}$}
	{\True $V=3\pi a^3$}
	{$V=\pi a^3$}
	\loigiai{
		{\color{red} HINH O DAY}\\.\\
		Thể tích $V=\dfrac{1}{3}\pi R^2h=\dfrac{1}{3}\pi\cdot OA^2\cdot SO$.\\
		Ta có $\widehat{ASB}=60^{\circ}\Rightarrow\widehat{ASO}=30^{\circ}\Rightarrow\tan 30^{\circ}=\dfrac{OA}{SO}=\dfrac{1}{\sqrt{3}}\Rightarrow SO=OA\sqrt{3}$.\\
		Lại có $S_{\mathrm{xq}}=\pi Rl=\pi\cdot OA\cdot SA=\pi\cdot OA\sqrt{OA^2+SO^2}=6\pi a^2$ \\
		$ \Rightarrow OA\sqrt{OA^2+3OA^2}=6a^2\Rightarrow 2OA^2=6a^2\Rightarrow OA=a\sqrt{3}\Rightarrow SO=3a\Rightarrow V=\dfrac{1}{3}\pi\cdot 3a^2\cdot 3a=3\pi a^3 $.
	}
\end{ex}
\begin{ex}
	[Xuân Trường - Nam Định - 2018]%Câu 13.
	Cho hình nón tròn xoay có đỉnh là $S$, $O$ là tâm của đường tròn đáy, đường sinh bằng $a\sqrt{2}$ và góc giữa đường sinh và mặt phẳng đáy bằng $60^{\circ}$. Diện tích xung quanh $S_{\mathrm{xq}}$ của hình nón và thể tích $V$ của khối nón tương ứng là
	\choice
	{\True $S_{\mathrm{xq}}=\pi a^2$, $V=\dfrac{\pi a^3\sqrt{6}}{12}$}
	{$S_{\mathrm{xq}}=\dfrac{\pi a^2}{2}$, $V=\dfrac{\pi a^3\sqrt{3}}{12}$}
	{$S_{\mathrm{xq}}=\pi a^2\sqrt{2}$, $V=\dfrac{\pi a^3\sqrt{6}}{4}$}
	{$S_{\mathrm{xq}}=\pi a^2$, $V=\dfrac{\pi a^3\sqrt{6}}{4}$}
	\loigiai{
		{\color{red} HINH O DAY}\\
		Dựa vào hình vẽ ta có góc giữa đường sinh và mặt đáy là $\widehat{SAO}=60^{\circ}$.\\
		Tam giác $SAO$ vuông tại $O$:\\
		$R=OA=SA\cdot\cos\widehat{SAO}=a\sqrt{2}\cdot\cos 60^{\circ}=\dfrac{a\sqrt{2}}{2}$.\\
		$h=SO=SA\cdot\sin\widehat{SAO}=a\sqrt{2}\cdot\sin 60^{\circ}=\dfrac{a\sqrt{6}}{2}$.\\
		Vậy $S_{\mathrm{xq}}=\pi Rl=\pi a^2$ và $V=\dfrac{1}{3}\pi R^2h=\dfrac{\pi a^3\sqrt{6}}{12}$.
	}
\end{ex}
\begin{ex}
	[Nguyễn Huệ - Phú Yên - 2020]%Câu 14.
	Cho hình nón có chiều cao $6a$. Một mặt phẳng $(P)$ đi qua đỉnh của hình nón và có khoảng cách đến tâm là $3a$, thiết diện thu được là một tam giác vuông cân. Thể tích của khối nón được giới hạn bởi hình nón đã cho bằng
	\choice
	{$150\pi a^3$}
	{$96\pi a^3$}
	{$108\pi a^3$}
	{\True $120\pi a^3$}
	\loigiai{
		{\color{red} HINH O DAY}\\
		Mặt phẳng $(P)$ cắt hình nón theo thiết diện là tam giác $SDE$. Theo giả thiết, tam giác $SDE$ vuông cân tại đỉnh $S$. Gọi $G$ là trung điểm $DE$, kẻ $OH\perp SG\Rightarrow OH=3a$.\\
		Ta có $\dfrac{1}{OH^2}=\dfrac{1}{SO^2}+\dfrac{1}{OG^2}\Rightarrow\dfrac{1}{OG^2}=\dfrac{1}{OH^2}-\dfrac{1}{SO^2}\Rightarrow OG=2a\sqrt{3}$.\\
		Do $SO\cdot OG=OH\cdot SG\Rightarrow SG=\dfrac{SO\cdot OG}{SG}=\dfrac{6a\cdot 2a\sqrt{3}}{3a}=4a\sqrt{3}\Rightarrow DE=8a\sqrt{3}$.\\
		$OD=\sqrt{OG^2+DG^2}=\sqrt{12a^2+48a^2}=2\sqrt{15}a$.\\
		Vậy $V=\dfrac{1}{3}\cdot\pi\cdot\left(2\sqrt{15}a\right)^2\cdot 6a=120\pi a^3$.
	}
\end{ex}
\begin{ex}
	[Tiên Du - Bắc Ninh - 2020]%Câu 15.
	Cho hình nón có bán kính đáy bằng 3 và chiều cao bằng 10. Mặt phẳng $(\alpha)$ vuông góc với trục và cách đỉnh của hình nón một khoảng bằng 4, chia hình nón thành hai phần. Gọi $V_1$ là thể tích của phần chứa đỉnh của hình nón đã cho, $V_2$ là thể tích của phần còn lại. Tính tỉ số $\dfrac{V_1}{V_2}$?
	\choice
	{$\dfrac{4}{25}$}
	{$\dfrac{21}{25}$}
	{\True $\dfrac{8}{117}$}
	{$\dfrac{4}{21}$}
	\loigiai{
		{\color{red} HINH O DAY}\\
		Ta có $IB\parallel OA\Rightarrow\dfrac{IB}{OA}=\dfrac{SI}{SO}=\dfrac{4}{10}=\dfrac{2}{5}$.\\
		Khi đó, $\dfrac{V_1}{V}=\dfrac{\dfrac{1}{3}\pi\cdot IB^2\cdot SI}{\dfrac{1}{3}\pi\cdot OA^2\cdot SO}=\left(\dfrac{IB}{OA}\right)^2\cdot\left(\dfrac{SI}{SO}\right)=\left(\dfrac{2}{5}\right)^3=\dfrac{8}{125}$.\\
		Suy ra: $\dfrac{V_2}{V}=1-\dfrac{8}{125}=\dfrac{117}{125}$.\\
		Vậy $\dfrac{V_1}{V_2}=\dfrac{V_1}{V}\colon\dfrac{V_2}{V}=\dfrac{8}{125}\colon\dfrac{117}{125}=\dfrac{8}{117}$.
	}
\end{ex}
\begin{ex}
	[Thanh Chương 1 - Nghệ An - 2020]%Câu 16.
	Cho một hình nón có bán kính đáy bằng $2a$. Mặt phẳng $(P)$ đi qua đỉnh $(S)$ của hình nón, cắt đường tròn đáy tại $A$ và $B$ sao cho $AB=2a\sqrt{3}$, khoảng cách từ tâm đường tròn đáy đến mặt phẳng $(P)$ bằng $\dfrac{a\sqrt{2}}{2}$. Thể tích khối nón đã cho bằng
	\choice
	{$\dfrac{8\pi a^3}{3}$}
	{\True $\dfrac{4\pi a^3}{3}$}
	{$\dfrac{2\pi a^3}{3}$}
	{$\dfrac{\pi a^3}{3}$}
	\loigiai{
		{\color{red} HINH O DAY}\\
		Gọi $C$ là trung điểm của $AB$, $O$ là tâm của đáy. Khi đó $\heva{&SO\perp AB\\&OC\perp AB}\Rightarrow(SOC)\perp AB$. Gọi $H$ là hình chiếu của $O$ lên $SC$ thì $OH\perp(SAB)$ nên $OH=a\dfrac{\sqrt{2}}{2}$.\\
		$OB=2a,BC=a\sqrt{3}\Rightarrow OC=a$. Xét tam giác vuông $SOC\colon\dfrac{1}{SO^2}=\dfrac{1}{OH^2}-\dfrac{1}{OC^2}=\dfrac{1}{a^2}\Rightarrow SO=a$.\\
		Vậy thể tích khối nón giới hạn bởi hình nón đã cho là $\dfrac{1}{3}\pi\cdot (2a)^2\cdot a=\dfrac{4\pi a^3}{3}$.
	}
\end{ex}
\begin{ex}
	[Mã 103 - 2021 - Lần 1]%Câu 17.
	Cắt hình nón $(N)$ bởi mặt phẳng đi qua đỉnh và tạo với mặt phẳng chứa đáy một góc bằng $30^{\circ}$, ta được thiết diện là tam giác đều cạnh $4a$. Diện tích xung quanh của $(N)$ bằng
	\choice
	{$4\sqrt{7}\pi a^2$}
	{$8\sqrt{7}\pi a^2$}
	{$8\sqrt{13}\pi a^2$}
	{\True $4\sqrt{13}\pi a^2$}
	\loigiai{
		{\color{red} HINH O DAY}\\
		Giả sử mặt phẳng $(P)$ cắt đáy của hình nón theo dây $AB$. Suy ra tam giác $SAB$ đều $\Rightarrow AB=4a$.\\
		Gọi $M$ là trung điểm của $AB\Rightarrow\widehat{SMO}=30^{\circ}$.\\
		Vì $SM$ là đường cao của tam giác $SAB$ nên $SM=\dfrac{4a\sqrt{3}}{2}=2a\sqrt{3}$.\\
		Tam giác $SMO$ vuông tại $O$ nên $\sin\widehat{SMO}=\dfrac{SO}{SM}\Rightarrow SO=SM\cdot\sin 30^{\circ}=2a\sqrt{3}\cdot\dfrac{1}{2}=a\sqrt{3}$.\\
		Suy ra $OM=\sqrt{SM^2-SO^2}=\sqrt{12a^2-3a^2}=3a$; $OA=\sqrt{OM^2+MA^2}=\sqrt{9a^2+4a^2}=a\sqrt{13}$.\\
		Vậy $S_{\mathrm{xq}}=\pi Rl=\pi\cdot OA\cdot SA=\pi\cdot a\sqrt{13}\cdot 4a=4\sqrt{13}\pi a^2$.
	}
\end{ex}
\begin{ex}
	[Mã 102 - 2021 Lần 1]%Câu 18.
	Cắt hình nón $(N)$ bởi mặt phẳng đi qua đỉnh và tạo với mặt phẳng chứa đáy một góc $60^{\circ}$ ta được thiết diện là tam giác đều có cạnh $2a$. Diện tích xung quanh của $(N)$ bằng
	\choice
	{\True $\sqrt{7}\pi a^2$}
	{$\sqrt{13}\pi a^2$}
	{$2\sqrt{7}\pi a^2$}
	{$2\sqrt{13}\pi a^2$}
	\loigiai{
		{\color{red} HINH O DAY}\\
		Giả sử hình nón $(N)$ có $S$ là đỉnh và $O$ là tâm đường tròn đáy.\\
		Giả sử mặt phẳng đề cho cắt nón theo thiết diện là tam giác đều $SAB$, khi đó ta có $l=SA=2a$.\\
		Gọi $H$ là trung điểm $AB\Rightarrow SH=2a\cdot\dfrac{\sqrt{3}}{2}=a\sqrt{3}$.\\
		Ta có góc giữa $(SAB)$ và mặt phẳng chứa đáy là góc $\widehat{SHO}=60^{\circ}$.\\
		Xét $\triangle SHO$ vuông tại $O$ có $OH=SH\cdot\cos 60^{\circ}=a\sqrt{3}\cdot\dfrac{1}{2}=\dfrac{a\sqrt{3}}{2}$.\\
		Xét $\triangle OAH$ vuông tại $H$ có bán kính đường tròn đáy là $R=OA=\sqrt{AH^2+OH^2}=\sqrt{a^2+\dfrac{3a^2}{4}}=\dfrac{a\sqrt{7}}{2}$.\\
		Vậy diện tích xung quanh của hình nón $(N)$ là $S_{\mathrm{xq}}=\pi Rl=\pi\cdot\dfrac{a\sqrt{7}}{2}\cdot 2a=\sqrt{7}\pi a^2$.
	}
\end{ex}
\begin{ex}
	[Mã 104 - 2021 Lần 1]%Câu 19.
	Cắt hình nón $(N)$ bởi mặt phẳng đi qua đỉnh và tạo với mặt phẳng chứa đáy một góc bằng $30^{\circ}$, ta được thiết diện là tam giác đều cạnh $2a$. Diện tích xung quanh của $(N)$ bằng
	\choice
	{$\sqrt{7}\pi a^2$}
	{\True $\sqrt{13}\pi a^2$}
	{$2\sqrt{13}\pi a^2$}
	{$2\sqrt{7}\pi a^2$}
	\loigiai{
		{\color{red} HINH O DAY}\\
		Xét hình nón $(N)$ và mặt phẳng $(SAB)$ đi qua đỉnh cắt $(O)$ tại $A$, $B$.\\
		Gọi $H$ là trung điểm của đoạn thẳng $AB$.\\
		Tam giác $SAB$ đều nên $SH=\dfrac{AB\sqrt{3}}{2}=\dfrac{2a\cdot\sqrt{3}}{2}=a\sqrt{3}$.\\
		Ta có $\heva{&(SAB)\cap(OAB)=AB\\&SH\perp AB\\&OH\perp AB}\Rightarrow\left(\widehat{(SAB),(OAB)}\right)=\left(\widehat{SH,OH}\right)=\widehat{SHO}=30^{\circ.}$ \\
		$\sin\widehat{SHO}=\dfrac{SO}{SH}\Rightarrow SO=SH\cdot\sin 30^{\circ}=\dfrac{a\sqrt{3}}{2}$.\\
		$OB=\sqrt{SB^2-SO^2}=\sqrt{(2a)^2-\left(\dfrac{a\sqrt{3}}{2}\right)^2}=\dfrac{\sqrt{13}}{2}$.\\
		Vậy $S_{\text{xq}}=\pi\cdot SB\cdot OB=\pi\cdot 2a\cdot\dfrac{a\sqrt{13}}{2}=\sqrt{13}\pi a^2$.
	}
\end{ex}
\begin{ex}
	[Mã 101-2021-Lần 1]%Câu 20.
	Cắt hình nón $(N)$ bởi mặt phẳng đi qua đỉnh và tạo với mặt phẳng chứa đáy một góc $60^{\circ}$, ta được thiết diện là tam giác đều cạnh $4a$. Diện tích xung quanh của $(N)$ bằng
	\choice
	{$8\sqrt{7}\pi a^2$}
	{$4\sqrt{13}\pi a^2$}
	{$8\sqrt{13}\pi a^2$}
	{\True $4\sqrt{7}\pi a^2$}
	\loigiai{
		{\color{red} HINH O DAY}\\
		Gọi hình nón $(N)$ có đỉnh $S$, đường tròn đáy có tâm $O$, bán kính $r$. Thiết diện đã cho là tam giác $SAB$ cạnh $4a$ và $I$ là trung điểm của $AB$. Khi đó\\
		$OI\perp AB, SI\perp AB$ nên góc giữa $(SAB)$ và mặt phẳng đáy là $\widehat{SIO}=60^{\circ}$.\\
		$SI=2a\sqrt{3}$ nên $OI=SI\cdot\cos 60^{\circ}=a\sqrt{3}$.\\
		Tam giác $OIA$ vuông tại $I$ có $r=OA=\sqrt{OI^2+AI^2}=a\sqrt{7}$.\\
		Vậy hình nón $(N)$ có diện tích xung quanh bằng $S_{\text{xq}}=\pi rl=4\sqrt{7}\pi a^2$.
	}
\end{ex}
\begin{ex}
	[Đề minh họa 2022]%Câu 21.
	Cho hình nón $S$ có bán kính đáy bằng $2\sqrt{3}a$. Gọi $A$ và $B$ là hai điểm thuộc đường tròn đáy sao cho $AB=4a$. Biết khoảng cách từ tâm của đáy đến mặt phẳng $(SAB)$ bằng $2a$, thể tích của khối nón đã cho bằng 
	\choice
	{\True $\dfrac{8\sqrt{3}}{3}\pi a^3$}
	{$4\sqrt{6}\pi a^3$}
	{$\dfrac{16\sqrt{3}}{3}\pi a^3$}
	{$8\sqrt{2}\pi a^3$}
	\loigiai{
		{\color{red} HINH O DAY}\\
		Ta có: $V=\dfrac{1}{3}S_d\cdot h=\dfrac{1}{2}\pi r^2h$.\\
		Gọi $I$ là trung điểm của $AB$. Khi đó:\\
		$\heva{&SI\perp AB\\&OI\perp AB}\Rightarrow AB\perp(SOI)$ mà $AB\subset(SAB)\Rightarrow(SAB)\perp(SOI)$.\\
		Kẻ $OH\perp SI$, suy ra: $\heva{&(SAB)\perp(SOI)\\&(SAB)\cap(SOI)=SI\\&OH\perp SI}\Rightarrow OH\perp(SAB)\Rightarrow\mathrm{d}\left(O;(SAB)\right)=OH=2a$.\\
		Xét tam giác $AOI$ vuông tại $I$ ta có: $OI=\sqrt{OA^2-AI^2}=\sqrt{OA^2-\left(\dfrac{AB}{2}\right)^2}=\sqrt{(2\sqrt{3}a)^2-\left(\dfrac{4a}{2}\right)^2}=2\sqrt{2}a$.\\
		Xét tam giác $SOI$ vuông tại $S$ ta có:\\
		$\dfrac{1}{OH^2}=\dfrac{1}{SO^2}+\dfrac{1}{OI^2}\Rightarrow\dfrac{1}{SO^2}=\dfrac{1}{OH^2}-\dfrac{1}{OI^2}$ \\
		$ \Rightarrow\dfrac{1}{SO^2}=\dfrac{OI^2-OH^2}{OH^2\cdot OI^2} =\dfrac{(2\sqrt{2}a)^2-4a^2}{4a^2\cdot (2\sqrt{2}a)^2}=\dfrac{1}{8a^2}\Rightarrow SO=2\sqrt{2}a $.\\
		Vậy $V=\dfrac{1}{3}S_d\cdot h=\dfrac{1}{3}\pi r^2h=\dfrac{1}{3}\pi(OA)^2\cdot SO=\dfrac{1}{3}\pi(2\sqrt{3}a)^2\cdot 2\sqrt{2}a=8\sqrt{2}a^3$.
	}
\end{ex}
\begin{dang}
	{Khối tròn xoay nội, ngoại tiếp khối đa diện}
\end{dang}    
\begin{ex}
	[Mã 123 2017]%Câu 1.
	Trong hình chóp tứ giác đều $S.ABCD$ có cạnh đều bằng $a\sqrt{2}$. Tính thể tích $V$ của khối nón đỉnh $S$ và đường tròn đáy là đường tròn nội tiếp tứ giác $ABCD$ 
	\choice
	{$V=\dfrac{\sqrt{2}\pi a^3}{2}$}
	{$V=\dfrac{\pi a^3}{2}$}
	{\True $V=\dfrac{\pi a^3}{6}$}
	{$V=\dfrac{\sqrt{2}\pi a^3}{6}$}
	\loigiai{
		\textbf{(VẼ HÌNH)}\\
		Gọi $O=AC\cap BD\Rightarrow SO\perp(ABCD)$. Lại có $OC=\dfrac{AC}{2}=a\Rightarrow SO=\sqrt{SA^2-OC^2}=a$.\\
		Bán kính $r=\dfrac{AB}{2}=\dfrac{a}{\sqrt{2}}$. Suy thể tích khối nón là $V=\dfrac{1}{3}\pi\left(\dfrac{a}{\sqrt{2}}\right)^2\cdot a=\dfrac{\pi a^3}{6}$.
	}
\end{ex}
\begin{ex}
	[Mã 110 2017]%Câu 2.
	Cho tứ diện đều $ABCD$ có cạnh bằng $3a$. Hình nón $(N)$ có đỉnh $A$ có đáy là đường tròn ngoại tiếp tam giác $BCD$. Tính diện tích xung quanh $S_{\mathrm{xq}}$ của $(N)$. 
	\choice
	{$S_{\mathrm{xq}}=12\pi a^2$}
	{$S_{\mathrm{xq}}=6\pi a^2$}
	{\True $S_{\mathrm{xq}}=3\sqrt{3}\pi a^2$}
	{$S_{\mathrm{xq}}=6\sqrt{3}\pi a^2$}
	\loigiai{
		{\color{red} HINH O DAY}\\
		Gọi $r$ là bán kính đường tròn ngoại tiếp tam giác $BCD$.\\
		Ta có $BM=\dfrac{3a\sqrt{3}}{2}$; $r=\dfrac{2}{3}BM=\dfrac{2}{3}\cdot\dfrac{3a\sqrt{3}}{2}=a\sqrt{3}$.\\
		$S_{\mathrm{xq}}=\pi\cdot r\cdot l=\pi r\cdot AB=\pi a\sqrt{3}\cdot 3a=3\sqrt{3}\cdot\pi a^2$.
	}
\end{ex}
\begin{ex}
	[Chuyên ĐHSPHN - 2018]%Câu 3.
	Cho hình chóp tam giác đều $S.ABC$. Hình nón có đỉnh $S$ và có đường tròn đáy là đường tròn nội tiếp tam giác $ABC$ gọi là hình nón nội tiếp hình chóp $S.ABC$, hình nón có đỉnh $S$ và có đường tròn đáy là đường tròn ngoại tiếp tam giác $ABC$ gọi là hình nón ngoại tiếp hình chóp $S.ABC$. Tỉ số thể tích của hình nón nội tiếp và hình nón ngoại tiếp hình chóp đã cho là
	\choice
	{$\dfrac{1}{2}$}
	{\True $\dfrac{1}{4}$}
	{$\dfrac{2}{3}$}
	{$\dfrac{1}{3}$}
	\loigiai{
		{\color{red} HINH O DAY}\\
		Gọi $M$ là trung điểm của $BC$.\\
		Gọi $O$ là trọng tâm của tam giác $ABC$.\\
		Ta có: $SO\perp(ABC)$ tại $O$.\\
		Suy ra, $O$ là tâm đường tròn nội tiếp và cũng là tâm của đường tròn ngoại tiếp tam giác $ABC$.\\
		Gọi $a$ là độ dài cạnh của tam giác $ABC$.\\
		Gọi $V_1$, $V_2$ lần lượt là thể tích của hình nón nội tiếp và hình nón ngoại tiếp hình chóp $S.ABC$.\\
		Do $OM=\dfrac{1}{2}OA$ nên ta có:\\
		$\dfrac{V_1}{V_2}=\dfrac{\dfrac{1}{3}\cdot\pi\cdot OM^2\cdot SO}{\dfrac{1}{3}\cdot\pi\cdot OA^2\cdot SO} =\dfrac{OM^2}{OA^2}=\left(\dfrac{OM}{OA}\right)^2=\left(\dfrac{1}{2}\right)^2=\dfrac{1}{4}$.
	}
\end{ex}
\begin{ex}
	[Hồng Bàng - Hải Phòng - 2018]%Câu 4.
	Cho hình chóp tam giác đều $S.ABC$ có cạnh đáy bằng $a$, góc giữa mặt bên và đáy bằng $60^{\circ}$. Diện tích xung quanh của hình nón đỉnh $S$, có đáy là hình tròn ngoại tiếp tam giác $ABC$ bằng
	\choice
	{$\dfrac{\pi a^2\sqrt{10}}{8}$}
	{$\dfrac{\pi a^2\sqrt{3}}{3}$}
	{$\dfrac{\pi a^2\sqrt{7}}{4}$}
	{\True $\dfrac{\pi a^2\sqrt{7}}{6}$}
	\loigiai{
		{\color{red} HINH O DAY}\\
		Gọi $I$ là tâm đường tròn $(ABC)\Rightarrow IA=r=\dfrac{a\sqrt{3}}{3}$.\\
		Gọi $M$ là trung điểm của $AB\Rightarrow AB\perp(SMC)$ \\
		$ \Rightarrow $ Góc giữa mặt bên và mặt đáy là góc $\widehat{SMC}=60^{\circ}\Rightarrow SM=2IM=\dfrac{2a\sqrt{3}}{6} =\dfrac{a\sqrt{3}}{3}$, $\Rightarrow SA=\sqrt{SM^2+MA^2} =\sqrt{\dfrac{a^2}{3}+\dfrac{a^2}{4}} =\dfrac{a\sqrt{21}}{6}$.\\
		Diện tích xung quanh hình nón $S_{\text{xq}}=\pi rl =\pi\cdot\dfrac{a\sqrt{3}}{3}\cdot\dfrac{a\sqrt{21}}{6} =\dfrac{\pi a^2\sqrt{7}}{6}$.
	}
\end{ex}
\begin{ex}
	[Chuyên Lê Hồng Phong Nam Định 2019]%Câu 5.
	Cho hình lập phương $ABCD.A'B'C'D'$ có cạnh $a$. Một khối nón có đỉnh là tâm của hình vuông $ABCD$ và đáy là hình tròn nội tiếp hình vuông $A'B'C'D'$. Diện tích toàn phần của khối nón đó là
	\choice
	{$S_{\mathrm{tp}}=\dfrac{\pi a^2}{2}(\sqrt{3}+2)$}
	{\True $S_{\mathrm{tp}}=\dfrac{\pi a^2}{4}(\sqrt{5}+1)$}
	{$S_{\mathrm{tp}}=\dfrac{\pi a^2}{4}(\sqrt{5}+2)$}
	{$S_{\mathrm{tp}}=\dfrac{\pi a^2}{2}(\sqrt{3}+1)$}
	\loigiai{
		{\color{red} HINH O DAY}\\
		Bán kính của đường tròn đáy là $r=\dfrac{a}{2}$.\\
		Diện tích đáy nón là $S_1=\pi r^2=\dfrac{\pi a^2}{4}$.\\
		Độ dài đường sinh là $l=\sqrt{a^2+r^2}=\dfrac{a\sqrt{5}}{2}$.\\
		Diện tích xung quanh của khối nón là $S_2=\pi rl=\dfrac{\pi a^2\sqrt{5}}{4}$.\\
		Vây, diện tích toàn phần của khối nón đó là $S_{\mathrm{tp}}=S_1+S_2=\dfrac{\pi a^2}{4}(\sqrt{5}+1)$.
	}
\end{ex}
\begin{ex}
	[Chuyên Vĩnh Phúc 2019]%Câu 6.
	Cho hình chóp tam giác đều $S.ABC$ có cạnh đáy bằng $a$, góc giữa mặt bên và mặt đáy bằng $60^{\circ}$. Tính diện tích xung quanh của hình nón đỉnh $S$, đáy là hình tròn ngoại tiếp tam giác $ABC$. 
	\choice
	{$\dfrac{\pi a^2\sqrt{3}}{3}$}
	{\True $\dfrac{\pi a^2\sqrt{7}}{6}$}
	{$\dfrac{\pi a^2\sqrt{7}}{4}$}
	{$\dfrac{\pi a^2\sqrt{10}}{8}$}
	\loigiai{
		{\color{red} HINH O DAY}\\
		Gọi $O$ là tâm đường tròn ngoại tiếp tam giác $ABC,M$ là trung điêmt cạnh $BC$, ta có $OM=\dfrac{a\sqrt{3}}{6}$, $OA=\dfrac{a\sqrt{3}}{3}$ và $\widehat{SMO}=60^{\circ}$.\\
		Trong tam giác vuông $SMO$: $SO=OM\cdot\tan 60^{\circ}=\dfrac{a\sqrt{3}}{6}\cdot\sqrt{3}=\dfrac{a}{2}\Rightarrow SA=\sqrt{\dfrac{a^2}{4}+\dfrac{a^2}{3}}=\dfrac{a\sqrt{7}}{2\sqrt{3}}$.\\
		Vậy $S_{\text{xq}}=\pi\cdot OA\cdot SA=\pi\cdot\dfrac{a\sqrt{3}}{3}\cdot\dfrac{a\sqrt{7}}{2\sqrt{3}}=\dfrac{\pi a^2\sqrt{7}}{6}$.
	}
\end{ex}
\begin{ex}
	[Mã 105 2017]%Câu 7.
	Cho hình nón $(N)$ có đường sinh tạo với đáy một góc $60^{\circ}$. Mặt phẳng qua trục của $(N)$ cắt $(N)$ được thiết diện là một tam giác có bán kính đường tròn nội tiếp bằng $1$. Tính thể tích $V$ của khối nón giới hạn bởi $(N)$. 
	\choice
	{$V=9\pi$}
	{$V=3\sqrt{3}\pi$}
	{$V=9\sqrt{3}\pi$}
	{\True $V=3\pi$}
	\loigiai{
		{\color{red} HINH O DAY}\\
		Hình nón $(N)$ có đường sinh tạo với đáy một góc $60^{\circ}$ nên $\widehat{SAH}=60^{\circ}$.\\
		Ta có $\triangle SAB$ cân tại $S$ có $\widehat{A}=60^{\circ}$ nên $\triangle SAB$ đều. Do đó tâm $I$ của đường tròn nội tiếp $\triangle SAB$ cũng là trọng tâm của $\triangle SAB$.\\
		Suy ra $S H=3 I H=3$. Mặt khác $S H=\dfrac{A B \sqrt{3}}{2} \Rightarrow A B=2 \sqrt{3} \Rightarrow R=\sqrt{3} \Rightarrow S_{\text {Đáy}}=\pi R^2=3 \pi$.\\
		Do đó $V=\dfrac{1}{3} S H \cdot S_{\text {Đáy}}=\dfrac{1}{3} 3.3 \pi=3 \pi$.
	}
\end{ex}
\begin{ex}
	[Chuyên Vĩnh Phúc 2019]%Câu 8.
	Cho hình chóp tam giác đều $S.ABC$ có cạnh đáy bằng $a$, góc giữa mặt bên và mặt đáy bằng $60^{\circ}$. Tính diện tích xung quanh của hình nón đỉnh $S$, đáy là hình tròn ngoại tiếp tam giác $ABC$. 
	\choice
	{$\dfrac{\pi a^2\sqrt{3}}{3}$}
	{\True $\dfrac{\pi a^2\sqrt{7}}{6}$}
	{$\dfrac{\pi a^2\sqrt{7}}{4}$}
	{$\dfrac{\pi a^2\sqrt{10}}{8}$}
	\loigiai{
		{\color{red} HINH O DAY}\\
		Gọi $E$ là trung điểm $BC$. Theo giả thiết $\widehat{SEA}=60^{\circ}$.\\
		Suy ra: $SA=\dfrac{a\sqrt{7}}{2\sqrt{3}}=l$.\\
		$S_{\text{xq}}=\pi Rl=\pi\cdot\dfrac{a\sqrt{3}}{3}\cdot\dfrac{a\sqrt{7}}{2\sqrt{3}}=\dfrac{\pi a^2\sqrt{7}}{6}$.
	}
\end{ex}
\begin{ex}
	[THCS - THPT Nguyễn Khuyến 2019]%Câu 9.
	Cho hình chóp tứ giác đều $S.ABCD$ có độ dài cạnh đáy là $a$ và $(N)$ là hình nón có đỉnh là $S$ với đáy là đường tròn ngoại tiếp tứ giác $ABCD$. Tỉ số thể tích của khối chóp $S.ABCD$ và khối nón $(N)$ là
	\choice
	{$\dfrac{\pi}{4}$}
	{$\dfrac{\pi\sqrt{2}}{2}$}
	{\True $\dfrac{2}{\pi}$}
	{$\dfrac{2\sqrt{2}}{\pi}$}
	\loigiai{
		Gọi $h$ là chiều cao của khối chóp và đồng thời là đường cao của khối nón.\\
		Thể tích của khối chóp là $V_1=\dfrac{1}{3}a^2h$.\\
		Bán kính của đường tròn ngoại tiếp đáy $ABCD$ là $r=\dfrac{AC}{2}=\dfrac{a\sqrt{2}}{2}$.\\
		Thể tích của khối nón là $V_2=\dfrac{1}{3}\pi\cdot\dfrac{a^2}{2}\cdot h$.\\
		Tỉ số thể tích của khối chóp $S.ABCD$ và khối nón $(N)$ là $\dfrac{V_1}{V_2}=\dfrac{2}{\pi}$.
	}
\end{ex}
\begin{ex}
	[THPT Ngô Sĩ Liên Bắc Giang 2019]%Câu 10.
	Cho hình chóp đều $S.ABCD$ có đáy là hình vuông cạnh $2a$, cạnh bên tạo với đáy góc $45^{\circ}$. Thể tích khối nón ngoại tiếp hình chóp trên là 
	\choice
	{$\dfrac{8}{3}\pi a^3\sqrt{3}$}
	{$\dfrac{2}{3}\pi a^3\sqrt{3}$}
	{$2\pi a^3\sqrt{2}$}
	{\True $\dfrac{2}{3}\pi a^3\sqrt{2}$}
	\loigiai{
		{\color{red} HINH O DAY}\\.\\
		Ta có $S.ABCD$ là hình chóp đều, gọi $O=AC\cap BD$ \\
		$ \Rightarrow $ Góc giữa cạnh bên với mặt đáy là $\widehat{SBO}=45^{\circ}$.\\
		$ABCD$ là hình vuông cạnh $2a\Rightarrow BD=2\sqrt{2a}$.\\
		Khối nón ngoại tiếp hình chóp $S.ABCD$ có bán kính đường tròn đáy $R=\dfrac{BD}{2}=a\sqrt{2}$.\\
		$\triangle SOB$ vuông cân tại $O$ \\
		$ \Rightarrow $ Chiều cao khối nón $h=SO=OB=\sqrt{2}a$ \\
		$ \Rightarrow $ Thể tích khối nón là $V=\dfrac{1}{3}\pi R^2h=\dfrac{1}{3}\pi(a\sqrt{2})^2\cdot a\sqrt{2}=\dfrac{2}{3}\pi a^3\sqrt{2}$.
	}
\end{ex}
\begin{ex}
	[THPT Lương Thế Vinh - HN - 2018]%Câu 11.
	Cho hình chóp tứ giác đều $S.ABCD$ có cạnh đáy bằng $a$. Tam giác $SAB$ có diện tích bằng $2a^2$. Thể tích của khối nón có đỉnh $S$ và đường tròn đáy nội tiếp tứ giác $ABCD$. 
	\choice
	{\True $\dfrac{\pi a^3\sqrt{7}}{8}$}
	{$\dfrac{\pi a^3\sqrt{7}}{7}$}
	{$\dfrac{\pi a^3\sqrt{7}}{4}$}
	{$\dfrac{\pi a^3\sqrt{15}}{24}$}
	\loigiai{
		\textbf{(VẼ HÌNH)}\\
		Gọi $O=AC\cap BD$ và $M$ là trung điểm $AB$. Hình nón có đỉnh $S$ và đường tròn đáy nội tiếp tứ giác $ABCD$ có bán kính đáy là $R=OM=\dfrac{a}{2}$ và có chiều cao là $h=SO$.\\
		Thể tích khối nón $V=\dfrac{1}{3}Bh$ trong đó $B=\pi R^2=\dfrac{\pi a^2}{4}$.\\
		Diện tích tam giác $SAB$ là $2a^2$ nên $\dfrac{1}{2}SM\cdot AB=2a^2\Leftrightarrow SM=4a$.\\
		Trong tam giác vuông $SOM$ ta có $SO=\sqrt{SM^2-OM^2}=\sqrt{16a^2-\dfrac{a^2}{4}}=\dfrac{3a\sqrt{7}}{2}$ hay $h=\dfrac{3a\sqrt{7}}{2}$.\\
		Vậy thể tích của khối nón $V=\dfrac{\pi a^3\sqrt{7}}{8}$.
	}
\end{ex}
\begin{ex}
	[Toán Học Tuổi Trẻ 2018]%Câu 12.
	Cho hình lập phương $ABCD.A'B'C'D'$ có cạnh $a$. Một khối nón có đỉnh là tâm của hình vuông $ABCD$ và đáy là hình tròn nội tiếp hình vuông $A'B'C'D'$. Kết quả tính diện tích toàn phần $S_{\mathrm{tp}}$ của khối nón đó có dạng bằng $\dfrac{\pi a^2}{4}(\sqrt{b}+c)$ với $b$ và $c$ là hai số nguyên dương và $b>1$. Tính $bc$. 
	\choice
	{\True $bc=5$}
	{$bc=8$}
	{$bc=15$}
	{$bc=7$}
	\loigiai{
		{\color{red} HINH O DAY}\\
		Ta có bán kính hình nón $r=\dfrac{a}{2}$, đường cao $h=a$, đường sinh $l=\dfrac{a\sqrt{5}}{2}$.\\
		Diện tích toàn phần $S_{\text{tp}} =\pi rl+\pi r^2 =\pi\dfrac{a^2\sqrt{5}}{4}+\pi\dfrac{a^2}{4} =\dfrac{\pi a^2}{4}(\sqrt{5}+1)\Rightarrow b=5,c=1$.\\
		Vậy $bc=5$.
	}
\end{ex}
\begin{ex}
	[Chuyên Đh Vinh - 2018]%Câu 13.
	Cho hình chóp tam giác đều $S.ABC$ có cạnh $AB=a$, góc tạo bởi $(SAB)$ và $(ABC)$ bằng $60^{\circ}$. Diện tích xung quanh của hình nón đỉnh $S$ và có đường tròn đáy ngoại tiếp tam giác $ABC$ bằng
	{\color{red} HINH O DAY}
	\choice
	{$\dfrac{\sqrt{7}\pi a^2}{3}$}
	{\True $\dfrac{\sqrt{7}\pi a^2}{6}$}
	{$\dfrac{\sqrt{3}\pi a^2}{2}$}
	{$\dfrac{\sqrt{3}\pi a^2}{6}$}
	\loigiai{
		{\color{red} HINH O DAY}\\
		Gọi $M$ là trung điểm $AB$ và gọi $O$ là tâm của tam giác $ABC$ ta có:\\
		$\heva{&AB\perp CM\\&AB\perp SO}\Rightarrow AB\perp(SCM)\Rightarrow AB\perp SM$ và $AB\perp CM$.\\
		Do đó góc giữa $(SAB)$ và $(ABC)$ là $\widehat{SMO}=60^{\circ}$.\\
		Mặt khác tam giác $ABC$ đều cạnh $a$ nên $CM=\dfrac{a\sqrt{3}}{2}$. Suy ra $OM=\dfrac{1}{3}CM=\dfrac{a\sqrt{3}}{6}$.\\
		$SO=OM\cdot\tan 60^{\circ} =\dfrac{a\sqrt{3}}{6}\cdot\sqrt{3} =\dfrac{a}{2}$.\\
		Hình nón đã cho có chiều cao $h=SO=\dfrac{a}{2}$, bán kính đáy $R=OA=\dfrac{a\sqrt{3}}{3}$, độ dài đường sinh.\\
		$l=\sqrt{h^2+R^2}=\dfrac{a\sqrt{21}}{6}$.\\
		Diện tích xung quanh hình nón là $S_{\text{xq}}=\pi\cdot R\cdot l=\pi\cdot\dfrac{a\sqrt{3}}{3}\cdot\dfrac{a\sqrt{21}}{6}=\dfrac{\sqrt{7}\pi a^2}{6}$.
	}
\end{ex}
\begin{ex}
	[Nam Định - 2018]%Câu 14.
	Cho hình nón đỉnh $S,$ đáy là hình tròn nội tiếp tam giác $ABC$. Biết rằng $AB=BC=10a$, $AC=12a$, góc tạo bởi hai mặt phẳng $(SAB)$ và $(ABC)$ bằng $45^{\circ}$. Tính thể tích $V$ của khối nón đã cho. 
	\choice
	{$V=3\pi a^3$}
	{\True $V=9\pi a^3$}
	{$V=27\pi a^3$}
	{$V=12\pi a^3$}
	\loigiai{
		{\color{red} HINH O DAY}\\
		Dựng $IK\perp AB$ suy ra góc giữa $(SAB)$ và $(ABC)$ là góc $\widehat{SKI}=45^{\circ}$.\\
		Xét $\triangle ABC$ có:\\
		$p=\dfrac{AB+BC+AC}{2}=\dfrac{10a+10a+12a}{2}=16a$.\\
		Suy ra.\\
		$S_{\triangle ABC}=\sqrt{p(p-a)(p-b)(p-c)}$.\\
		$=\sqrt{16a\cdot 6a\cdot 6a\cdot 4a}=48a^2$.\\
		Bán kính đường tròn nội tiếp $r=\dfrac{S}{p}=\dfrac{48a^2}{16a}=3a$.\\
		Xét $\triangle SIK$ có $SI=IK=r=3a$.\\
		Thể tích khối nón là\\
		$V=\dfrac{1}{3}h\cdot\pi r^2 =\dfrac{1}{3}\cdot 3a\cdot\pi\cdot (3a)^2=9\pi a^3$.
	}
\end{ex}   
\begin{ex}
	[Chuyên Trần Phú - Hải Phòng 2018]%Câu 15.
	Cho hình hộp chữ nhật $ABCD.A'B'C'D'$ có đáy là hình vuông cạnh $a$ và cạnh bên bằng $2a$. Tính diện tích xung quanh $S_{\mathrm{xq}}$ của hình nón có đỉnh là tâm $O$ của hình vuông $A'B'C'D'$ và đáy là hình tròn nội tiếp hình vuông $ABCD$. 
	\choice
	{$S_{\mathrm{xq}}=\pi a^2\sqrt{17}$}
	{$S_{\mathrm{xq}}=\dfrac{\pi a^2\sqrt{17}}{2}$}
	{\True $S_{\mathrm{xq}}=\dfrac{\pi a^2\sqrt{17}}{4}$}
	{$S_{\mathrm{xq}}=2\pi a^2\sqrt{17}$}
	\loigiai{
		{\color{red} HINH O DAY}\\
		Bán kính đáy của hình nón: $R=\dfrac{a}{2}$.\\
		Đường sinh của hình nón: $l=OM\Leftrightarrow l=\sqrt{MI^2+OI^2}\Leftrightarrow l=\sqrt{\left(\dfrac{a}{2}\right)^2+4a^2}\Leftrightarrow l=a\dfrac{\sqrt{17}}{2}$.\\
		Diện tích xung quanh của hình nón là $S=\pi\cdot R\cdot l\Leftrightarrow S=\pi\cdot\dfrac{a}{2}\cdot a\dfrac{\sqrt{17}}{2}\Leftrightarrow S=\dfrac{\pi a^2\sqrt{17}}{4}$.
	}
\end{ex}
\Closesolutionfile{ans}
\indapan{10}{ans/CD21/Muc_7_8}