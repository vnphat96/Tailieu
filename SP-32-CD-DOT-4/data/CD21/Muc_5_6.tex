\Opensolutionfile{ans}[ans/CD21/Muc_5_6]
\setcounter{ex}{0}
\setcounter{dang}{0}
\section{Mức độ 5,6 điểm}
\begin{dang}
	{Diện tích xung quanh, diện tích toàn phần, chiều cao, bán kính đáy, thiết diện}
\end{dang}
\begin{ex}
	[Đề Minh Họa 2020 Lần 1]%Câu 1.
	Diện tích xung quanh của hình nón có độ dài đường sinh $l$ và bán kính đáy $r$ bằng
	\choice
	{$4\pi rl$}      
	{$2\pi rl$}
	{\True $\pi rl$}
	{$\dfrac{1}{3}\pi rl$}
	\loigiai{
		Áp dụng công thức diện tích xung quanh hình nón.
	}
\end{ex}
\begin{ex}
	[Mã 102 - 2020 Lần 2]%Câu 2.
	Cho hình nón có bán kính đáy $r=2$ và độ dài đường sinh $l=7$. Diện tích xung quanh của hình nón đã cho bằng
	\choice
	{$28\pi$}
	{\True $14\pi$}
	{$\dfrac{14\pi}{3}$}
	{$\dfrac{98\pi}{3}$}
	\loigiai{
		Có $S_{\mathrm{xq}}=\pi rl=\pi\cdot 7\cdot 12=14\pi$.
	}
\end{ex}
\begin{ex}
	[Mã 101 - 2020 Lần 2]%Câu 3.
	Cho hình nón có bán kính đáy $r=2$ và độ dài đường sinh $l=5$. Diện tích xung quanh của hình nón đã cho bằng
	\choice
	{$20\pi$}
	{$\dfrac{20\pi}{3}$}
	{\True $10\pi$}
	{$\dfrac{10\pi}{3}$}
	\loigiai{
		Ta có diện tích xung quanh của hình nón đã cho là $S_{\mathrm{xq}}=\pi rl =\pi\cdot 2\cdot 5=10\pi$.
	}
\end{ex}
\begin{ex}
	[Mã 104 - 2020 Lần 2]%Câu 4.
	Cho hình nón có bán kính đáy $r=2$ và độ dài đường sinh $l=7$. Diện tích xung quanh của hình nón đã cho bằng
	\choice
	{$\dfrac{28\pi}{3}$}
	{\True $14\pi$}
	{$28\pi$}
	{$\dfrac{14\pi}{3}$}
	\loigiai{
		$S_{\mathrm{xq}}=\pi rl=2\cdot 7\cdot\pi=14\pi$.
	}
\end{ex}
\begin{ex}
	[KTNL GV Thuận Thành 2 Bắc Ninh 2019]%Câu 5.
	Gọi $l$, $h$, $r$ lần lượt là độ dài đường sinh, chiều cao và bán kính mặt đáy của hình nón. Diện tích xung quanh $S_{\mathrm{xq}}$ của hình nón là 
	\choice
	{$S_{\mathrm{xq}}=\dfrac{1}{3}\pi r^2h$}
	{\True $S_{\mathrm{xq}}=\pi rl$}
	{$S_{\mathrm{xq}}=\pi rh$}
	{$S_{\mathrm{xq}}=2\pi rl$}
	\loigiai{
		Diện tích xung quanh của hình nón là $S_{\mathrm{xq}}=\pi rl$.
	}
\end{ex}
\begin{ex}
	[Chuyên Thái Bình 2019]%Câu 6.
	Cho hình nón có bán kính đáy bằng $a$, đường cao là $2a$. Tính diện tích xung quanh hình nón?
	\choice
	{$2\sqrt{5}\pi a^2$}
	{\True $\sqrt{5}\pi a^2$}
	{$2a^2$}
	{$5a^2$}
	\loigiai{
		{\color{red} HINH O DAY}\\
		Ta có $S_{\text{xq}}=\pi Rl=\pi a\sqrt{a^2+4a^2}=\sqrt{5}\pi a^2$ (đvdt).
	}
\end{ex}
\begin{ex}
	[Mã 104 2017]%Câu 7.
	Cho hình nón có bán kính đáy $r=\sqrt{3}$ và độ dài đường sinh $l=4$. Tính diện tích xung quanh của hình nón đã cho. 
	\choice
	{$S_{\mathrm{xq}}=8\sqrt{3}\pi$}
	{$S_{\mathrm{xq}}=12\pi$}
	{\True $S_{\mathrm{xq}}=4\sqrt{3}\pi$}
	{$S_{\mathrm{xq}}=\sqrt{39}\pi$}
	\loigiai{
		Diện tích xung quanh của hình nón là $S_{\mathrm{xq}}=\pi rl=4\sqrt{3}\pi$.}
\end{ex}
\begin{ex}
	[Đề Tham Khảo 2017]%Câu 8.
	Cho hình nón có diện tích xung quanh bằng $3\pi a^2$ và bán kính đáy bằng $a$. Tính độ dài đường sinh $l$ của hình nón đã cho. 
	\choice
	{\True $l=3a$}
	{$l=2\sqrt{2}a$}
	{$l=\dfrac{3a}{2}$}
	{$l=\dfrac{\sqrt{5}a}{2}$}
	\loigiai{
		Diện tích xung quanh của hình nón là $S_{\mathrm{xq}}=\pi rl=\pi al=3\pi a^2\Rightarrow l=3a$.
	}
\end{ex}
\begin{ex}
	[Đề Tham Khảo 2018]%Câu 9.
	Cho hình nón có diện tích xung quanh bằng $3\pi a^2$ và có bán kính đáy bằng $a$. Độ dài đường sinh của hình nón đã cho bằng 
	\choice
	{\True $3a$}
	{$2a$}
	{$\dfrac{3a}{2}$}
	{$2\sqrt{2}a$}
	\loigiai{
		Diện tích xung quanh hình nón: $S_{\mathrm{xq}}=\pi rl$ với $r=a\Rightarrow\pi\cdot a\cdot l=3\pi a^2\Rightarrow l=3a$.
	}
\end{ex}
\begin{ex}
	[Đề Minh Họa 2017]%Câu 10.
	Trong không gian, cho tam giác vuông $ABC$ tại $A$, $AB=a$ và $AC=a\sqrt{3}$. Tính độ dài đường sinh $l$ của hình nón, nhận được khi quay tam giác $ABC$ xung quanh trục $l=2a$. 
	\choice
	{$l=a\sqrt{3}$}
	{\True $l=2a$}
	{$l=a$}
	{$l=a\sqrt{2}$}
	\loigiai{
		{\color{red} HINH O DAY}\\
		Xét tam giác $ABC$ vuông tại $A$ ta có $BC^2=AC^2+AB^2=4a^2\Leftrightarrow BC=2a$.\\
		Đường sinh của hình nón cũng chính là cạnh huyền của tam giác $\Leftrightarrow l=BC=2a$.
	}
\end{ex}
\begin{ex}
	[THPT Lê Quy Đôn Điện Biên 2019]%Câu 11.
	Một hình nón có thiết diện qua trục là một tam giác vuông cân có cạnh góc vuông bằng $a$. Tính diện tích xung quanh của hình nón. 
	\choice
	{$\dfrac{2\pi a^2\sqrt{2}}{3}$}
	{$\dfrac{\pi a^2\sqrt{2}}{4}$}
	{$\pi a^2\sqrt{2}$}
	{\True $\dfrac{\pi a^2\sqrt{2}}{2}$}
	\loigiai{
		{\color{red} HINH O DAY}\\
		Ta có tam giác $SAB$ vuông cân tại $S$ có $SA=a$.\\
		Khi đó: $R=OA=\dfrac{a\sqrt{2}}{2}, l=SA=a$. Nên $S_{\mathrm{xq}}=\pi Rl=\pi\cdot\dfrac{a\sqrt{2}}{2}\cdot a=\dfrac{\pi a^2\sqrt{2}}{2}$.
	}
\end{ex}
\begin{ex}
	[THPT Lương Thế Vinh Hà Nội 2019]%Câu 12.
	Cho hình nón có bán kính đáy bằng $a$ và độ dài đường sinh bằng $2a$. Diện tích xung quanh của hình nón đó bằng
	\choice
	{$4\pi a^2$}
	{$3\pi a^2$}
	{\True $2\pi a^2$}
	{$2a^2$}
	\loigiai{
		{\color{red} HINH O DAY}\\.\\
		Ta có $S_{\mathrm{xq}}=\pi rl=\pi\cdot a\cdot 2a=2\pi a^2$.
	}
\end{ex}
\begin{ex}
	[Sở Vĩnh Phúc 2019]%Câu 13.
	Cho hình nón có diện tích xung quanh bằng $3\pi a^2$, bán kính đáy bằng $a$. Tính độ dài đường sinh của hình nón đó
	\choice
	{$2a\sqrt{2}$}
	{$\dfrac{3a}{2}$}
	{$2a$}
	{\True $3a$}
	\loigiai{
		$S_{\text{xq}}=\pi Rl\Rightarrow l=\dfrac{S_{\mathrm{xq}}}{\pi R}=\dfrac{3\pi a^2}{\pi a}=3a$.
	}
\end{ex}
\begin{ex}
	[THPT - Yên Định Thanh Hóa 2019]%Câu 14.
	Cho khối nón $(N)$ có thể tích bằng $4\pi$ và chiều cao là $3$. Tính bán kính đường tròn đáy của khối nón $(N)$. 
	\choice
	{\True $2$}
	{$\dfrac{2\sqrt{3}}{3}$}
	{$1$}
	{$\dfrac{4}{3}$}
	\loigiai{
		Thể tích của khối nón được tính bởi công thức $V=\dfrac{1}{3}\pi R^2h$ ($R$ là bán kính đáy, $h$ là độ dài đường cao của khối chóp).\\
		Theo bài ra: $V=4\pi,h=3$ nên ta có $4\pi=\dfrac{1}{3}\pi R^2\cdot 3\Leftrightarrow R^2=4\Leftrightarrow R=2$.\\
		Vậy $R=2$.
	}
\end{ex}
\begin{ex}
	[THPT Trần Nhân Tông - QN -2018]%Câu 15.
	Trong không gian, cho tam giác $ABC$ vuông tại cân $A$, gọi $I$ là trung điểm của $BC$, $BC=2$. Tính diện tích xung quanh của hình nón, nhận được khi quay tam giác $ABC$ xung quanh trục $AI$. 
	\choice
	{\True $S_{\mathrm{xq}}=\sqrt{2}\pi$}
	{$S_{\mathrm{xq}}=2\pi$}
	{$S_{\mathrm{xq}}=2\sqrt{2}\pi$}
	{$S_{\mathrm{xq}}=4\pi$}
	\loigiai{
		{\color{red} HINH O DAY}\\
		$R=\dfrac{BC}{2}=1$, $l=AB=AC=\dfrac{2}{\sqrt{2}}=\sqrt{2}$.\\
		$S_{\mathrm{xq}}=\pi R=\sqrt{2}\pi$.
	}
\end{ex}
\begin{ex}
	[Đồng Tháp - 2018]%Câu 16.
	Một hình nón có thiết diện qua trục là một tam giác vuông cân có cạnh góc vuông bằng $a$. Diện tích xung quanh của hình nón bằng
	\choice
	{$\dfrac{\pi a^2\sqrt{2}}{4}$}
	{$\dfrac{2\pi a^2\sqrt{2}}{3}$}
	{\True $\dfrac{\pi a^2\sqrt{2}}{2}$}
	{$\pi a^2\sqrt{2}$}
	\loigiai{
		{\color{red} HINH O DAY}\\
		Ta có $l=AB=a$, $r=\dfrac{BC}{2}=\dfrac{a\sqrt{2}}{2}$, $S_{\text{xq}}=\pi rl =\pi\cdot\dfrac{a\sqrt{2}}{2}\cdot a =\dfrac{\pi a^2\sqrt{2}}{2}$.
	}
\end{ex}
\begin{ex}
	[THPT Hoàng Hoa Thám - Hưng Yên - 2018]%Câu 17.
	Cho hình hình nón có độ dài đường sinh bằng $4$, diện tích xung quanh bằng $8\pi$. Khi đó hình nón có bán kính hình tròn đáy bằng
	\choice
	{$8$}
	{$4$}
	{\True $2$}
	{$1$}
	\loigiai{
		Ta có diện tích xung quanh của hình nón là\\
		$S_{\text{xq}}=\pi Rl=\pi\cdot R\cdot 4=8\pi\Rightarrow R=2$.\\
		Vậy bán kính hình tròn đáy là $R=2$.
	}
\end{ex}
\begin{ex}
	[Chuyên Quốc Học Huế - 2018]%Câu 18.
	Cho hình nón có bán kính đáy bằng $3$ và chiều cao bằng $4$. Tính diện tích xung quanh của hình nón. 
	\choice
	{$12\pi$}
	{$9\pi$}
	{$30\pi$}
	{\True $15\pi$}
	\loigiai{
		Ta có $l=\sqrt{r^2+h^2} =\sqrt{3^2+4^2}=5$.\\
		Diện tích xung quanh của hình nón đã cho là $S_{\mathrm{xq}}=\pi rl=\pi\cdot 3\cdot 5=15\pi$.
	}
\end{ex}
\begin{ex}
	[THPT Hậu Lộc 2 - TH - 2018]%Câu 19.
	Cho hình nón có đường sinh $l=5$, bán kính đáy $r=3$. Diện tích toàn phần của hình nón đó là 
	\choice
	{$S_{\mathrm{tp}}=15\pi$}
	{$S_{\mathrm{tp}}=20\pi$}
	{$S_{\mathrm{tp}}=22\pi$}
	{\True $S_{\mathrm{tp}}=24\pi$}
	\loigiai{
		Áp dụng công thức tính diện tích toàn phàn của hình nón ta có $S_{\mathrm{tp}}=\pi rl+\pi r^2 =15\pi+9\pi =24\pi$.
	}
\end{ex}
\begin{ex}
	[Chuyên Lương Thế Vinh - Đồng Nai - 2018]%Câu 20.
	Cho hình nón $(N)$ có đường kính đáy bằng $4a$, đường sinh bằng $5a$. Tính diện tích xung quanh $S$ của hình nón $(N)$. 
	\choice
	{\True $S=10\pi a^2$}
	{$S=14\pi a^2$}
	{$S=36\pi a^2$}
	{$S=20\pi a^2$}
	\loigiai{
		{\color{red} HINH O DAY}\\
		Diện tích xung quanh của hình nón $(N)$ là $S=\pi rl=\pi\cdot 2a\cdot 5a =10\pi a^2$.
	}
\end{ex}
\begin{ex}
	[Chuyên Vĩnh Phúc - 2018]%Câu 21.
	Cho hình nón có diện tích xung quanh bằng $5\pi a^2$ và bán kính đáy bằng $a$. Tính độ dài đường sinh của hình nón đã cho?
	\choice
	{$a\sqrt{5}$}
	{$3a\sqrt{2}$}
	{$3a$}
	{\True $5a$}
	\loigiai{
		Áp dụng công thức tính diện tích xung quanh của hình nón $S_{\mathrm{xq}}=\pi Rl$, nên ta có:\\
		$l=\dfrac{S_{\mathrm{xq}}}{\pi R} =\dfrac{5\pi a^2}{\pi a} =5a$.
	}
\end{ex}
\begin{ex}
	[Thanh Hóa - 2018]%Câu 22.
	Mặt phẳng chứa trục của một hình nón cắt hình nón theo thiết diện là 
	\choice
	{một hình chữ nhật}
	{\True một tam giác cân}
	{một đường elip}
	{một đường tròn}
	\loigiai{
		{\color{red} HINH O DAY}\\
		Mặt phẳng chứa trục của một hình nón cắt hình nón theo thiết diện là một tam giác cân.
	}
\end{ex}
\begin{ex}
	[Chuyên Bắc Ninh - 2018]%Câu 23.
	Cho hình nón có bán kính đáy $r=\sqrt{3}$ và độ dài đường sinh $l=4$. Tính diện tích xung quanh $S$ của hình nón đã cho. 
	\choice
	{$S=8\sqrt{3}\pi$}
	{$S=24\pi$}
	{$S=16\sqrt{3}\pi$}
	{\True $S=4\sqrt{3}\pi$}
	\loigiai{
		Ta có $S=\pi rl=4\sqrt{3}\pi$.
	}
\end{ex}
\begin{ex}
	[Mã 101 - 2022]%Câu 24.
	Cho tam giác $OIM$ vuông tại $I$ có $OI=3$ và $IM=4$. Khi quay tam giác $OIM$ quanh cạnh góc vuông $OI$ thì đường gấp khúc $OMI$ tạo thành hình nón có độ dài đường sinh bằng
	\choice
	{$7$}
	{$3$}
	{\True $5$}
	{$4$}
	\loigiai{
		{\color{red} HINH O DAY}\\.\\
		Ta có chiều cao hình nón $h=OI=3$, bán kính đáy $r=IM=4$ thì độ dài đường sinh là\\
		$l=OM=\sqrt{IM^2+OI^2}=\sqrt{3^2+4^2}=5$.
	}
\end{ex}    
\begin{dang}
	{Thể tích}
\end{dang}
\begin{ex}
	[Mã 103 - 2019]%Câu 1.
	Thể tích của khối nón có chiều cao $h$ và có bán kính đáy $r$ là
	\choice
	{$2\pi r^2h$}
	{\True $\dfrac{1}{3}\pi r^2h$}
	{$\pi r^2h$}
	{$\dfrac{4}{3}\pi r^2h$}
	\loigiai{
		Thể tích của khối nón có chiều cao $h$ và có bán kính đáy $r$ là $V=\dfrac{1}{3}\pi r^2h$.
	}
\end{ex}
\begin{ex}
	[Đề Tham Khảo 2020 Lần 2]%Câu 2.
	Cho khối nón có chiều cao $h=3$ và bán kính đáy $r=4$. Thể tích của khối nón đã cho bằng
	\choice
	{\True $16\pi$}
	{$48\pi$}
	{$36\pi$}
	{$4\pi$}
	\loigiai{
		Ta có công thức thể tích khối nón $V=\dfrac{1}{3}\cdot\pi\cdot r^2\cdot h=\dfrac{1}{3}\cdot\pi\cdot 16\cdot 3=16\pi$.
	}
\end{ex}
\begin{ex}
	[Mã 101 - 2020 Lần 1]%Câu 3.
	Cho khối nón có bán kính đáy $r=5$ và chiều cao $h=2$. Thể tích khối nón đã cho bằng 
	\choice
	{$\dfrac{10\pi}{3}$}
	{$10\pi$}
	{\True $\dfrac{50\pi}{3}$}
	{$50\pi$}
	\loigiai{
		Thể tích khối nón $V=\dfrac{1}{3}\pi r^2h=\dfrac{50\pi}{3}$.
	}
\end{ex}
\begin{ex}
	[Mã 102 - 2020 Lần 1]%Câu 4
	Cho khối nón có bán kính đáy $r=4$ và chiều cao $h=2$. Thể tích của khối nón đã cho bằng
	\choice
	{$\dfrac{8 \pi}{3}$}
	{$8 \pi$}
	{\True $\dfrac{32 \pi}{3}$}
	{$32 \pi$}
	\loigiai{
		Thể tích của khối nón đã cho là $V=\dfrac{1}{3}\pi r^2 h=\dfrac{1}{3}\pi \cdot 4^2 \cdot 2=\dfrac{32 \pi}{3}$.
	}
\end{ex}
\begin{ex}
	[Mã 103 - 2020 Lần 1]%Câu 5.
	Cho khối nón có bán kính $r=2$ chiều cao $h=5$. Thể tích của khối nón đã cho bằng
	\choice
	{\True $\dfrac{20\pi}{3}$}
	{$20\pi$}
	{$\dfrac{10\pi}{3}$}
	{$10\pi$}
	\loigiai{
		Áp dụng công thức thể tích khối nón ta được: $V=\dfrac{\pi r^2h}{3}=\dfrac{\pi{\cdot 2}^2\cdot 5}{3}=\dfrac{20\pi}{3}$.
	}
\end{ex}
\begin{ex}
	[Mã 104 - 2020 Lần 1]%Câu 6.
	Cho khối nón có bán kính đáy $r=2$ và chiều cao $h=4$. Thể tích của khối nón đã cho bằng
	\choice
	{$8\pi$}
	{$\dfrac{8\pi}{3}$}
	{\True $\dfrac{16\pi}{3}$}
	{$16\pi$}
	\loigiai{
		Ta có $V=\dfrac{1}{3}\cdot r^2\cdot\pi\cdot h=\dfrac{1}{3}\cdot 2^2\cdot\pi\cdot 4=\dfrac{16\pi}{3}$.
	}
\end{ex}
\begin{ex}
	[Mã 110 2017]%Câu 7.
	Cho khối nón có bán kính đáy $r=\sqrt{3}$ và chiều cao $h=4$. Tính thể tích $V$ của khối nón đã cho. 
	\choice
	{$V=12\pi$}
	{\True $V=4\pi$}
	{$V=16\pi\sqrt{3}$}
	{$V=\dfrac{16\pi\sqrt{3}}{3}$}
	\loigiai{
		Ta có $V=\dfrac{1}{3}\pi\cdot r^2\cdot h=\dfrac{1}{3}\pi(\sqrt{3})^2\cdot 4=4\pi$.
	}
\end{ex}
\begin{ex}
	[Mã 101 - 2019]%Câu 8.
	Thể tích của khối nón có chiều cao $h$ và bán kính đáy $r$ là
	\choice
	{$\dfrac{4}{3}\pi r^2h$}
	{$2\pi r^2h$}
	{\True $\dfrac{1}{3}\pi r^2h$}
	{$\pi r^2h$}
	\loigiai{
		Thể tích của khối nón có chiều cao $h$ và bán kính đáy $r$ là $V=\dfrac{1}{3}\pi r^2h$.
	}
\end{ex}
\begin{ex}
	[Mã 104 2019]%Câu 9.
	Thể tích khối nón có chiều cao $h$ và bán kính đáy $r$ là
	\choice
	{\True $\dfrac{1}{3}\pi r^2h$}
	{$\dfrac{4}{3}\pi r^2h$}
	{$2\pi r^2h$}
	{$\pi r^2h$}
	\loigiai{
		Lý thuyết thể tích khối nón.
	}
\end{ex}
\begin{ex}
	[Mã 102 - 2019]%Câu 10.
	Thể tích của khối nón có chiều cao $h$ và bán kính đáy $r$ là
	\choice
	{$\dfrac{4}{3}\pi r^2h$}
	{$\pi r^2h$}
	{$2\pi r^2h$}
	{\True $\dfrac{1}{3}\pi r^2h$}
	\loigiai{
		Thể tích của khối nón có chiều cao $h$ và bán kính đáy $r$ là $V=\dfrac{1}{3}\pi r^2h$.
	}
\end{ex}
\begin{ex}
	[Chuyên Quốc Học Huế 2019]%Câu 11.
	Cho khối nón có bán kính đáy $r=3$, chiều cao $h=\sqrt{2}$. Tính thể tích $V$ của khối nón. 
	\choice
	{$V=\dfrac{3\pi\sqrt{2}}{3}$}
	{$V=3\pi\sqrt{11}$}
	{\True $V=\dfrac{9\pi\sqrt{2}}{3}$}
	{$V=9\pi\sqrt{2}$}
	\loigiai{
		{\color{red} HINH O DAY}\\.\\
		Thể tích khối nón: $V=\dfrac{1}{3}\pi\cdot r^2\cdot h=\dfrac{9\pi\sqrt{2}}{3}$.
	}
\end{ex}
\begin{ex}
	[Chuyên ĐHSP Hà Nội 2019]%Câu 12.
	Cho tam giác $ABC$ vuông tại $A,AB=c,AC=b$. Quay tam giác $ABC$ xung quanh đường thẳng chứa cạnh $AB$ ta được một hình nón có thể tích bằng
	\choice
	{$\dfrac{1}{3}\pi bc^2$}
	{$\dfrac{1}{3}bc^2$}
	{$\dfrac{1}{3}b^2c$}
	{\True $\dfrac{1}{3}\pi b^2c$}
	\loigiai{
		{\color{red} HINH O DAY}\\
		$V=\dfrac{1}{3}\pi r^2h=\dfrac{1}{3}\pi b^2c$.
	}
\end{ex}
\begin{ex}
	[Chuyên Lương Thế Vinh Đồng Nai 2019]%Câu 13.
	Cho hình nón có độ dài đường sinh bằng 25 và bán kính đường tròn đáy bằng 15. Tính thể tích của khối nón đó. 
	\choice
	{\True $1500\pi$}
	{$4500\pi$}
	{$375\pi$}
	{$1875\pi$}
	\loigiai{
		{\color{red} HINH O DAY}\\
		Gọi $h$ là chiều cao khối nón $\Rightarrow h=\sqrt{l^2-r^2}=\sqrt{25^2-15^2}=20$ \\
		$ \Rightarrow V=\dfrac{1}{3}\pi r^2h=\dfrac{1}{3}\cdot\pi\cdot 15^2\cdot 20=1500\pi $.
	}
\end{ex}
\begin{ex}
	[Mã 105 2017]%Câu 14.
	Trong không gian cho tam giác $ABC$ vuông tại $A$, $AB=a$ và $\widehat{ACB}=30^{\circ}$. Tính thể tích $V$ của khối nón nhận được khi quay tam giác $ABC$ quanh cạnh $AC$. 
	\choice
	{$V=\pi a^3$}
	{$V=\sqrt{3}\pi a^3$}
	{$V=\dfrac{\sqrt{3}\pi a^3}{9}$}
	{\True $V=\dfrac{\sqrt{3}\pi a^3}{3}$}
	\loigiai{
		Ta có $AC=AB\cdot\cot 30^{\circ}=a\sqrt{3}$. Vậy thể tích khối nón là $V=\dfrac{1}{3}\pi a^2\cdot a\sqrt{3}=\dfrac{\pi a^3\sqrt{3}}{3}$.
	}
\end{ex}
\begin{ex}
	[Đề Tham Khảo 2019]%Câu 15.
	Cho khối nón có độ dài đường sinh bằng $2a$ và bán kính đáy bằng $a$. Thể tích của khối nón đã cho bằng
	\choice
	{\True $\dfrac{\sqrt{3}\pi a^3}{3}$}
	{$\dfrac{\sqrt{3}\pi a^3}{2}$}
	{$\dfrac{2\pi a^3}{3}$}
	{$\dfrac{\pi a^3}{3}$}
	\loigiai{
		Chiều cao khối nón đã cho là $h=\sqrt{l^2-r^2}=a\sqrt{3}$.\\
		Thể tích khối nón đã cho là $V=\dfrac{1}{3}\pi r^2h=\dfrac{1}{3}\pi a^2\cdot a\sqrt{3}=\dfrac{\sqrt{3}\pi a^3}{3}$.
	}
\end{ex}
\begin{ex}
	[Chuyên Bắc Giang 2019]%Câu 16.
	Cho khối nón có bán kính đáy $r=2,$ chiều cao $h=\sqrt{3}$. Thể tích của khối nón là
	\choice
	{\True $\dfrac{4\pi\sqrt{3}}{3}$}
	{$\dfrac{4\pi}{3}$}
	{$\dfrac{2\pi\sqrt{3}}{3}$}
	{$4\pi\sqrt{3}$}
	\loigiai{
		Khối nón có thể tích là $V=\dfrac{1}{3}\pi r^2h=\dfrac{4\pi\sqrt{3}}{3}$.
	}
\end{ex}
\begin{ex}
	[KTNL Gia Bình 2019]%Câu 17.
	Cho khối nón tròn xoay có chiều cao và bán kính đáy cùng bằng $a$. Khi đó thể tích khối nón là
	\choice
	{$\dfrac{4}{3}\pi a^3$}
	{$\dfrac{2}{3}\pi a^3$}
	{$\pi a^3$}
	{\True $\dfrac{1}{3}\pi a^3$}
	\loigiai{
		Khối nón có bán kính đáy $R=a$. Diện tích đáy $S=\pi a^2$. Thể tích khối nón là $V=\dfrac{1}{3}\pi a^3$.
	}
\end{ex}
\begin{ex}
	[Chuyên Vĩnh Phúc 2019]%Câu 18.
	Cho khối nón có bán kính đáy $r=\sqrt{3}$ và chiều cao $h=4$. Tính thể tích $V$ của khối nón đã cho. 
	\choice
	{$V=16\pi\sqrt{3}$}
	{$V=\dfrac{16\pi\sqrt{3}}{3}$}
	{$V=12\pi$}
	{\True $V=4\pi$}
	\loigiai{
		$V=\dfrac{1}{3}\pi r^2h=\dfrac{1}{3}\pi\cdot 3\cdot 4=4\pi$.
	}
\end{ex}
\begin{ex}
	[THPT Đông Sơn 1 - Thanh Hóa - 2019]%Câu 19.
	Cho khối nón có độ dài đường sinh bằng $2a$ và đường cao bằng $a\sqrt{3}$. Thể tích của khối nón đã cho bằng
	\choice
	{$\dfrac{2\pi a^3}{3}$}
	{$\dfrac{\sqrt{3}\pi a^3}{2}$}
	{\True $\dfrac{\sqrt{3}\pi a^3}{3}$}
	{$\dfrac{\pi a^3}{3}$}
	\loigiai{
		{\color{red} HINH O DAY}\\
		Ta có $l=2a, h=a\sqrt{3}$.\\
		$r^2=l^2-h^2=4a^2-3a^2=a^2\Rightarrow r=a$.\\
		Thể tích khối nón là $V=\dfrac{1}{3}\pi r^2h=\dfrac{1}{3}\pi a^2a\sqrt{3}=\dfrac{\sqrt{3}\pi a^3}{3}$.
	}
\end{ex}
\begin{ex}
	[Chuyên Hà Tĩnh 2019]%Câu 20.
	Cho khối nón có thiết diện qua trục là một tam giác cân có một góc $120^{\circ}$ và cạnh bên bằng $a$. Tính thể tích khối nón. 
	\choice
	{\True $\dfrac{\pi a^3}{8}$}
	{$\dfrac{3\pi a^3}{8}$}
	{$\dfrac{\pi a^3\sqrt{3}}{24}$}
	{$\dfrac{\pi a^3}{4}$}
	\loigiai{
		{\color{red} HINH O DAY}\\
		Gọi thiết diện qua trục là tam giác $ABC$ (Hình vẽ) có $\widehat{BAC}=120^{\circ}$ và $AB=AC=a$. Gọi $O$ là trung điểm của đường kính $BC$ của đường tròn đáy khi đó ta có $r=BO=AB\sin 60^{\circ}=\dfrac{a\sqrt{3}}{2}$ và $h=AO=AB\cos 60^{\circ}=\dfrac{a}{2}$. Vậy thể tích khối nón là $V=\dfrac{1}{3}\pi r^2h=\dfrac{1}{3}\pi\left(\dfrac{a\sqrt{3}}{2}\right)^2\dfrac{a}{2}=\dfrac{\pi a^3}{8}$.
	}
\end{ex}
\begin{ex}
	Nếu giữ nguyên bán kính đáy của một khối nón và giảm chiều cao của nó $2$ lần thì thể tích của khối nón này thay đổi như thế nào?
	\choice
	{Giảm $4$ lần}
	{\True Giảm $2$ lần}
	{Tăng $2$ lần}
	{Không đổi}
	\loigiai{
		Gọi $R,h$ lần lượt là bán kính đường tròn đáy và chiều cao của hình nón ban đầu.\\
		Thể tích khối nón ban đầu là $V_1=\dfrac{1}{3}\pi R^2h$. Giữ nguyên bán kính đáy của khối nón và giảm chiều cao của nó $2$ lần thì thể tích của khối nón này là $V_2=\dfrac{1}{3}\pi\cdot R^2\cdot\dfrac{h}{2}=\dfrac{1}{2}V_1$.
	}
\end{ex}
\begin{ex}
	[THPT Mai Anh Tuấn - Thanh Hóa - 2019]%Câu 22.
	Cho khối nón có độ dài đường sinh bằng đường kính đáy bằng $a$. Thể tích khối nón là 
	\choice
	{$\dfrac{\pi a^3\sqrt{3}}{16}$}
	{$\dfrac{\pi a^3\sqrt{3}}{48}$}
	{\True $\dfrac{\pi a^3\sqrt{3}}{24}$}
	{$\dfrac{\pi a^3\sqrt{3}}{8}$}
	\loigiai{
		{\color{red} HINH O DAY}\\
		Khối nón có độ dài đường sinh bằng đường kính đáy bằng a\\
		$ \Rightarrow\triangle SAB $ đều cạnh $a\Rightarrow SO=\dfrac{a\sqrt{3}}{2}$.\\
		$V_{kn}=\dfrac{1}{3}\cdot SO\cdot S_d=\dfrac{1}{3}\cdot\dfrac{a\sqrt{3}}{2}\cdot\pi\cdot\dfrac{a^2}{4}=\dfrac{\pi a^3\sqrt{3}}{24}$.
	}
\end{ex}
\begin{ex}
	[Chuyên An Giang - 2018]%Câu 23.
	Cho khối nón có bán kính $r=\sqrt{5}$ và chiều cao $h=3$. Tính thể tích $V$ của khối nón. 
	\choice
	{$V=9\pi\sqrt{5}$}
	{$V=3\pi\sqrt{5}$}
	{$V=\pi\sqrt{5}$}
	{\True $V=5\pi$}
	\loigiai{
		Thể tích $V$ của khối nón là $V=\dfrac{1}{3}\pi r^2h=\dfrac{1}{3}\pi 5\cdot 3=5\pi$.
	}
\end{ex}
\begin{ex}
	[Chuyên Lam Sơn - Thanh Hóa - 2018]%Câu 24.
	Cho khối nón có bán kính đáy $r=2$, chiều cao $h=\sqrt{3}$ (hình vẽ). Thể tích của khối nón là 
	{\color{red} HINH O DAY}
	\choice
	{$\dfrac{4\pi}{3}$}
	{$\dfrac{2\pi\sqrt{3}}{3}$}
	{$4\pi\sqrt{3}$}
	{\True $\dfrac{4\pi\sqrt{3}}{3}$}
	\loigiai{
		Ta có $V=\dfrac{1}{3}\pi r^2h =\dfrac{1}{3}\pi\cdot 2^2\cdot\sqrt{3} =\dfrac{4\pi\sqrt{3}}{3}$.
	}
\end{ex}
\begin{ex}
	[THPT Lê Xoay - 2018]%Câu 25.
	Cho hình nón có bán kính đáy bằng $2$ (cm), góc ở đỉnh bằng $60^{\circ}$. Thể tích khối nón là
	\choice
	{$V=\dfrac{8\pi\sqrt{3}}{9}\left(cm^3\right)$}
	{$V=\dfrac{8\pi\sqrt{3}}{2}\left(cm^3\right)$}
	{$V=8\pi\sqrt{3}\left(cm^3\right)$}
	{\True $V=\dfrac{8\pi\sqrt{3}}{3}\left(cm^3\right)$}
	\loigiai{
		{\color{red} HINH O DAY}\\
		Ta có bán kính đáy $r=2$, đường cao $h=\dfrac{r}{\tan{30}^{\circ}}\Rightarrow h=2\sqrt{3}$.\\
		Vậy thể tích khối nón $V=\dfrac{1}{3}\pi r^2h =\dfrac{1}{3}\pi\cdot 4\cdot 2\sqrt{3} =\dfrac{8\pi\sqrt{3}}{3}\left(cm^3\right)$.
	}
\end{ex}
\begin{ex}
	[Cụm 5 Trường Chuyên - ĐBSH - 2018]%Câu 26.
	Cắt hình nón bởi một mặt phẳng đi qua trục ta được thiết diện là một tam giác vuông cân có cạnh huyền bằng $a\sqrt{6}$. Tính thể tích $V$ của khối nón đó. 
	\choice
	{\True $V=\dfrac{\pi a^3\sqrt{6}}{4}$}
	{$V=\dfrac{\pi a^3\sqrt{6}}{2}$}
	{$V=\dfrac{\pi a^3\sqrt{6}}{6}$}
	{$V=\dfrac{\pi a^3\sqrt{6}}{3}$}
	\loigiai{
		{\color{red} HINH O DAY}\\.\\
		Khối nón có $2r=a\sqrt{6}\Leftrightarrow r=\dfrac{a\sqrt{6}}{2}$ và $h=r$ suy ra thể tích $V=\dfrac{1}{3}\pi r^2h=\dfrac{\pi a^3\sqrt{6}}{4}$.
	}
\end{ex}
\begin{ex}
	[THPT Cầu Giấy - 2018]%Câu 27.
	Cho khối nón tròn xoay có đường cao $h=15 cm$ và đường sinh $l=25 cm$. Thể tích $V$ của khối nón là 
	\choice
	{$V=1500\pi\left(cm^3\right)$}
	{$V=500\pi\left(cm^3\right)$}
	{$V=240\pi\left(cm^3\right)$}
	{\True $V=2000\pi\left(cm^3\right)$}
	\loigiai{
		Ta có $V=\dfrac{1}{3}\pi r^2h=\pi\cdot\left(l^2-h^2\right)\cdot h=2000\pi$.\\
		Vậy: $V=2000\pi (cm^3)$.
	}
\end{ex}      
\begin{ex}
	[Mã 103 - 2022]%Câu 28.
	Cho khối nón có diện tích đáy bằng $3a^2$ và chiều cao $2a$. Thể tích của khối nón đã cho bằng
	\choice
	{$3a^3$}
	{$6a^3$}
	{\True $2a^3$}
	{$\dfrac{2}{3}a^3$}
	\loigiai{
		Thể tích của khối nón đã cho bằng $V=\dfrac{1}{3}\cdot 3a^2\cdot 2a=2a^3$.
	}
\end{ex}
\Closesolutionfile{ans}
\indapan{10}{ans/CD21/Muc_5_6}