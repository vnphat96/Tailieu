\Opensolutionfile{ans}[ans/CD21/Muc_9_10]
\setcounter{ex}{0}
\setcounter{dang}{0}
\section{Mức độ 9,10 điểm}
\begin{dang}   
	{Một số bài toán VD - VDC liên quan đến khối nón (các bài toán thực tế - cực trị)}
\end{dang}
\begin{ex}
	[Sở Ninh Bình 2020]%Câu 1.      
	Cho hai khối nón có chung trục $SS’=3r$. Khối nón thứ nhất có đỉnh $S$, đáy là hình tròn tâm $S’$ bán kính $2r$. Khối nón thứ hai có đỉnh $S’$, đáy là hình tròn tâm $S$ bán kính $r$. Thể tích phần chung của hai khối nón đã cho bằng
	\choice
	{$\dfrac{4\pi r^3}{27}$}
	{$\dfrac{\pi r^3}{9}$}
	{\True $\dfrac{4\pi r^3}{9}$}
	{$\dfrac{4\pi r^3}{3}$}
	\loigiai{
		{\color{red} HINH O DAY}\\
		Gọi $(P)$ là mặt phẳng đi qua trục của hai khối nón và lần lượt cắt hai đường tròn $(S,r)$ và $(S’,2r)$ theo đường kính $AB,CD$. Gọi $M=SC\cap S’B,N=SD\cap S’A$. Phần chung của 2 khối nón đã cho gồm 2 khối nón chung đáy là hình tròn đường kính MN và đỉnh lần lượt là $S,S’$.\\
		Ta có $\dfrac{MN}{CD}=\dfrac{SN}{SD}=\dfrac{SN}{SN+ND}=\dfrac{SA}{SA+S’D}=\dfrac{r}{3r}=\dfrac{1}{3}\Rightarrow MN=\dfrac{1}{3}CD=\dfrac{4r}{3}$.\\
		Gọi $I$ là giao điểm của MN và $SS’$. Ta có $SI=\dfrac{1}{3}SS’=r,S’I=\dfrac{2}{3}SS’=2r$.\\
		Do đó thể tích phần chung là\\
		$V=\dfrac{1}{3}\pi SI\cdot\left(\dfrac{MN}{2}\right)^2+\dfrac{1}{3}\pi S’I\cdot\left(\dfrac{MN}{2}\right)^2=\dfrac{1}{3}\pi\cdot r\cdot\dfrac{4r^2}{9}+\dfrac{1}{3}\pi\cdot 2r\cdot\dfrac{4r^2}{9}=\dfrac{4\pi r^3}{9}$.
	}
\end{ex}
\begin{ex}
	[Đặng Thúc Hứa - Nghệ An - 2020]%Câu 2.
	Tính thể tích của vật thể tròn xoay khi quay mô hình (như hình vẽ bên) quanh trục $DB$. \\
	{\color{red} HINH O DAY}
	\choice
	{$\dfrac{9\pi a^3\sqrt{3}}{8}$}
	{\True $\dfrac{3\pi a^3\sqrt{3}}{8}$}
	{$\dfrac{2\pi a^3\sqrt{3}}{3}$}
	{$\dfrac{\pi a^3\sqrt{3}}{12}$}
	\loigiai{
		Thể tích của vật thể tròn xoay gồm hai phần bao gồm thể tích $V_1$ của hình nón tạo bởi tam giác vuông $ABC$ khi quay quanh cạnh $AB$ và thể tích $V_2$ của hình nón tạo bởi tam giác vuông $ADE$ khi quay quanh cạnh $AD$.\\
		Xét tam giác vuông $ABC$ vuông tại $B$ ta có:\\
		$r_1=BC=AC\cdot\sin 30^{\circ}=a$; $h_1=AB=AC\cdot\sin 60^{\circ}=a\sqrt{3}$.\\
		Vậy ta có $V_1=\dfrac{1}{3}\pi\cdot r_1^2\cdot h_1=\dfrac{1}{3}\pi\cdot a^2\cdot a\sqrt{3}=\dfrac{\pi\sqrt{3}a^3}{3}$.\\
		Xét tam giác vuông $ADE$ vuông tại $D$ ta có:\\
		$r_2=DE=AE\cdot\sin 30^{\circ}=\dfrac{a}{2}$; $h_2=AD=AE\cdot\sin 60^{\circ}=\dfrac{a\sqrt{3}}{2}$.\\
		Vậy ta có $V_2=\dfrac{1}{3}\pi\cdot r_2^2\cdot h_2=\dfrac{1}{3}\pi\cdot\left(\dfrac{a}{2}\right)^2\cdot\dfrac{a\sqrt{3}}{2}=\dfrac{\pi\sqrt{3}a^3}{24}$.\\
		Vậy thể tích của vật thể tròn xoay là $V=V_1+V_2=\dfrac{\pi\sqrt{3}a^3}{3}+\dfrac{\pi\sqrt{3}a^3}{24}=\dfrac{3\pi\sqrt{3}a^3}{8}$.
	}
\end{ex}
\begin{ex}
	[Đô Lương 4 - Nghệ An - 2020]%Câu 3.
	Cho tam giác $ABC$ vuông tại $A$, $BC=a$, $AC=b$, $AB=c$, $b<c$. Khi quay tam giác vuông $ABC$ một vòng quanh cạnh $BC$, quay cạnh $AC$, quanh cạnh $AB$, ta thu được các hình có diện tích toàn phần theo thứ tự bằng $S_a,S_b,S_c$. Khẳng định nào sau đây đúng?
	\choice
	{\True $S_b>S_c>S_a$}
	{$S_b>S_a>S_c$}
	{$S_c>S_a>S_b$}
	{$S_a>S_c>S_b$}
	\loigiai{
		{\color{red} HINH O DAY}\\
		Gọi $H$ là hình chiếu của $A$ lên cạnh $BC,AH=h$.\\
		Khi quay tam giác vuông $ABC$ một vòng quanh cạnh $BC$ ta thu được hình hợp bởi hai hình nón tròn xoay có chung đáy bán kính bằng $h$, đường sinh lần lượt là $b,c$. Do đó $S_a=\pi bh+\pi ch$.\\
		Khi quay tam giác vuông $ABC$ một vòng quanh cạnh $AC$ ta thu được hình nón tròn xoay có bán kính đáy bằng $c$, đường sinh bằng $a$, $S_b=\pi ac+\pi c^2=\pi c(a+c)$.\\
		Khi quay tam giác vuông $ABC$ một vòng quanh cạnh $AB$ ta thu được hình nón tròn xoay có bán kính đáy bằng $b$, đường sinh bằng $a$, $S_c=\pi ab+\pi b^2=\pi b(a+b)$.\\
		Do $b<c$ nên $\heva{&ab<ac\\&b^2<c^2}\Rightarrow S_c<S_b$.\\
		Ta có $h=\dfrac{bc}{a}\Rightarrow S_a=\pi b^2\cdot\dfrac{c}{a}+\pi c^2\cdot\dfrac{b}{a}$.\\
		Tam giác $ABC$ vuông nên $\dfrac{c}{a}<1\Rightarrow\pi b^2\dfrac{c}{a}<\pi b^2$; $\dfrac{c^2}{a^2}<1\Rightarrow\pi c^2\dfrac{b}{a}<\pi ab$ \\
		$ \Rightarrow S_a<\pi b^2+\pi ab=\pi b(a+b)=S_c $. Do đó $S_a<S_c$.\\
		Vậy $S_b>S_c>S_a$.
	}
\end{ex}
\begin{ex}
	Cho tam giác $ABC$ cân tại $A$, góc $\widehat{BAC}=120^{\circ}$ và $AB=4cm$. Tính thể tích khối tròn xoay lớn nhất có thể khi ta quay tam giác $ABC$ quanh đường thẳng chứa một cạnh của tam giác $ABC$. 
	\choice
	{$16\sqrt{3}\pi\left(cm^3\right)$}
	{\True $16\pi\left(cm^3\right)$}
	{$\dfrac{16\pi}{\sqrt{3}}\left(cm^3\right)$}
	{$\dfrac{16\pi}{3}\left(cm^3\right)$}
	\loigiai{
		{\color{red} HINH O DAY}\\
		Trường hợp 1: Khối tròn xoay khi quay $\triangle ABC$ quanh đường thẳng chứa $AB$ (hoặc $AC$) có thể tích bằng hiệu thể tích của hai khối nón $(N_1)$ và $(N_2)$.\\
		Dựng $CK\perp BA$ tại $K\Rightarrow\heva{&AK=AC\cdot\cos CAK=4\cdot\cos 60^{\circ}=2cm\\&BK=BA+AK=4+2=6cm\\&CK=AC\cdot sinCAK=4\cdot sin60^{\circ}=2\sqrt{3}cm.}$ \\
		$(N_1)$ có $h_1=BK=6cm$, $r_1=CK=2\sqrt{3}cm$.\\
		$(N_2)$ có $h_2=AK=2cm$, $r_2=CK=2\sqrt{3}cm$.\\
		Do đó $V=\dfrac{1}{3}\pi\cdot CK^2\cdot (BK-AK)=\dfrac{1}{3}\pi\cdot (2\sqrt{3})^2\cdot (6-2)=16\pi\left(cm^3\right)$.\\
		Trường hợp 2: Khối tròn xoay khi quay $\triangle ABC$ quanh đường thẳng chứa $BC$ có thể tích bằng tổng thể tích của hai khối nón $(N_3)$ và $(N_4)$.\\
		Kẻ đường cao $AH (H\in BC)\Rightarrow\heva{&AH=AB\cdot\cos BAH=4\cdot\cos 60^{\circ}=2cm\\&BH=CH=AB\cdot\sin BAH=4\cdot\sin 60^{\circ}=2\sqrt{3}cm.}$ \\
		$(N_3)$ và $(N_4)$ có $h_3=h_4=BH=CH=2\sqrt{3}cm$, $r_3=r_4=HA=2cm$.\\
		Do đó $V=2\cdot\dfrac{1}{3}\pi\cdot AH^2\cdot BH=2\cdot\dfrac{1}{3}\pi\cdot 2^2\cdot 2\sqrt{3}=\dfrac{16\pi}{\sqrt{3}}\left(cm^3\right)$.\\
		Vậy $V_{\max} =16\pi\left(cm^3\right)$.
	}
\end{ex}
\begin{ex}
	[Cụm liên trường Hải Phòng - 2019]%Câu 5.
	Huyền có một tấm bìa hình tròn như hình vẽ, Huyền muốn biến hình tròn đó thành một cái phễu hình nón. Khi đó Huyền phải cắt bỏ hình quạt tròn $AOB$ rồi dán hai bán kính $OA$ và $OB$ lại với nhau. Gọi $x$ là góc ở tâm hình quạt tròn dùng làm phễu. Tìm $x$ để thể tích phễu là lớn nhất?
	{\color{red} HINH O DAY}\\{\color{red} HINH O DAY}
	\choice
	{\True $\dfrac{2\sqrt{6}}{3}\pi$}
	{$\dfrac{\pi}{3}$}
	{$\dfrac{\pi}{2}$}
	{$\dfrac{\pi}{4}$}
	\loigiai{
		Góc $x$ chắn cung $\overset\frown{AB}$ có độ dài $l=R\cdot x$.\\
		Từ giả thiết suy ra bán kính của phễu là $r=\dfrac{Rx}{2\pi}$ và chiều cao của phễu là $h=\sqrt{R^2-\left(\dfrac{Rx}{2\pi}\right)^2}=\dfrac{R}{2\pi}\sqrt{4{\pi}^2-x^2}$.\\
		Khi đó thể tích của phễu là $V=\dfrac{1}{3}\pi r^2h=\dfrac{1}{3}\pi\cdot\dfrac{R^2x^2}{4{\pi}^2}\cdot\dfrac{R}{2\pi}\sqrt{4{\pi}^2-x^2}=\dfrac{R^3}{24{\pi}^2}x^2\sqrt{4{\pi}^2-x^2}$.\\
		Xét hàm số $f(x)=x^2\sqrt{4{\pi}^2-x^2}$, $x\in(0;2\pi)$.\\
		$f’(x)=2x\sqrt{4{\pi}^2-x^2}-\dfrac{x^3}{\sqrt{4{\pi}^2-x^2}}=\dfrac{2x\left(4{\pi}^2-x^2\right)-x^3}{\sqrt{4{\pi}^2-x^2}}=\dfrac{x\left(8{\pi}^2-3x^2\right)}{\sqrt{4{\pi}^2-x^2}}$.\\
		Cho $f’(x)=0\Rightarrow x=\dfrac{2\sqrt{6}}{3}\pi$.\\
		Lập bảng biến thiên, ta có\\
		{\color{red} HINH O DAY}\\
		Vậy thể tích phễu lớn nhất khi $x=\dfrac{2\sqrt{6}}{3}\pi$.
	}
\end{ex}
\begin{ex}
	Một khối nón có thể tích bằng $9a^3\pi\sqrt{2}$. Tính bán kính $R$ đáy khối nón khi diện tích xung quanh nhỏ nhất. 
	\choice
	{\True $R=3a$}
	{$R=\dfrac{3a}{\sqrt[6]{2}}$}
	{$R=\sqrt[3]{9}a$}
	{$R=\dfrac{3a}{\sqrt[3]{2}}$}
	\loigiai{
		Gọi $h, l$ lần lượt là chiều cao và độ dài đường sinh của khối nón.\\
		$210\Rightarrow l=\sqrt{R^2+h^2}=\sqrt{R^2+2\cdot\dfrac{729a^6}{R^4}}$.\\
		$S_{\mathrm{xq}}=\pi\cdot R\cdot l=\pi\sqrt{R^4+\dfrac{729a^6}{R^2}+\dfrac{729a^6}{R^2}}\geq\pi\sqrt{\sqrt[3]{R^4\cdot\dfrac{729a^6}{R^2}\cdot\dfrac{729a^6}{R^2}}}$ \\
		$ \Rightarrow S_{\mathrm{xq}}=9\pi a^2 $. Nên $\min S_{\mathrm{xq}}=9\pi a^2$ khi $R^4=\dfrac{729a^6}{R^2}\Leftrightarrow R=3a$.
	}
\end{ex}
\begin{ex}
	[HSG Sở Nam Định 2019]%Câu 7.
	Cho hai mặt phẳng $(P),(Q)$ song song với nhau và cùng cắt khối cầu tâm $O$, bán kính $R$ thành hai hình tròn cùng bán kính. Xét hình nón có đỉnh trùng với tâm của một trong hai hình tròn này và có đáy là hình tròn còn lại. Tính khoảng cách $h$ giữa hai mặt phẳng $(P),(Q)$ để diện tích xung quanh của hình nón là lớn nhất. 
	\choice
	{$h=R$}
	{$h=R\sqrt{2}$}
	{\True $h=\dfrac{2R\sqrt{3}}{3}$}
	{$2R\sqrt{3}$}
	\loigiai{
		Cắt khối cầu tâm $O$, bán kính $R$ bằng mặt phẳng $(\alpha)$ đi qua tâm $O$ và vuông góc với hai mặt phẳng $(P),(Q)$ ta được hình như hình vẽ bên dưới. 
		{\color{red} HINH O DAY}\\
		Trong đó, $AB=(\alpha)\cap(P), CD=(\alpha)\cap(Q)$ với $AB=CD$, $h=SH=AC=BD$, $R=OB$.\\
		Đường sinh $l=SC=SD$.\\
		Bán kính của mỗi hình tròn giao tuyến là $r=\dfrac{AB}{2}$.\\
		Ta có $l^2=SC^2=AC^2+AS^2=h^2+r^2$ và $r^2=SB^2=OB^2-SO^2=R^2-\dfrac{h^2}{4}$.\\
		Suy ra $l^2=R^2+\dfrac{3h^2}{4}$.\\
		Mà diện tích xung quanh của khối nón được xét là $S_{\mathrm{xq}}=\pi rl$.\\
		Ta có $S_{\mathrm{xq}}$ đạt giá trị lớn nhất $\Leftrightarrow rl$ đạt giá trị lớn nhất.\\
		Áp dụng bất đẳng thức Cauchy cho 2 số $r\sqrt{3}$ và $l$ ta có\\
		$rl=\dfrac{1}{2\sqrt{3}}\cdot 2\cdot (r\sqrt{3})l\leq\dfrac{\sqrt{3}}{6}\left(3r^2+l^2\right)=\dfrac{\sqrt{3}}{6}\cdot 4R^2=\dfrac{2R^2\sqrt{3}}{3}$.\\
		$rl$ lớn nhất là $\dfrac{2R^2\sqrt{3}}{3}$ khi và chỉ khi $3r^2=l^2\Leftrightarrow h^2=\dfrac{4}{3}R^2\Rightarrow h=\dfrac{2R\sqrt{3}}{3}$.
	}
\end{ex}
\begin{ex}
	[Bạc Liêu – Ninh Bình 2019]%Câu 8.
	Cho tam giác $OAB$ vuông cân tại $O$, có $OA=4$. Lấy điểm $M$ thuộc cạnh $AB$ ($M$ không trùng với $A$, $B$) và gọi $H$ là hình chiếu của $M$ trên $OA$. Tìm giá trị lớn nhất của thể tích khối tròn xoay được tạo thành khi quay tam giác $OMH$ quanh $OA$. 
	\choice
	{$\dfrac{128\pi}{81}$}
	{$\dfrac{81\pi}{256}$}
	{\True $\dfrac{256\pi}{81}$}
	{$\dfrac{64\pi}{81}$}
	\loigiai{
		{\color{red} HINH O DAY}\\
		Đặt $h=OH$, $0<h<4$.\\
		Khi quay tam giác $OMH$ quanh $OA$, ta được hình nón đỉnh $O$ chiều cao $h$ bán kính đáy $r=HM$.\\
		Ta có $HM\parallel OB$ nên $\dfrac{AH}{AO}=\dfrac{HM}{OB}\Rightarrow\dfrac{4-h}{4}=\dfrac{r}{4}\Rightarrow r=4-h$.\\
		$V=\dfrac{1}{3}\pi r^2h =\dfrac{1}{3}\pi(4-h)^2\cdot h =\dfrac{1}{6}\pi(4-h)(4-h)\cdot 2h\leq\dfrac{1}{6}\pi\left(\dfrac{4-h+4-h+2h}{3}\right)^3 =\dfrac{256\pi}{81}$.\\
		Vậy $V_{\max} =\dfrac{1}{3}\pi\cdot\dfrac{256}{27} =\dfrac{256\pi}{81}$.
	}
\end{ex}
\begin{ex}
	[THPT Thăng Long-Hà Nội - 2019]%Câu 9.
	Lượng nguyên liệu cần dùng để làm ra một chiếc nón lá được ước lượng qua phép tính diện tích xung quanh của mặt nón. Cứ $1 \mathrm{kg}$ lá dùng để làm nón có thể làm ra số nón có tổng diện tích xung quanh là $6,13 \mathrm{m}^2$. Hỏi nếu muốn làm ra 1000 chiếc nón lá giống nhau có đường trình vành nón $50 \mathrm{cm}$, chiều cao $30 \mathrm{cm}$ thì cần khối lượng lá gần nhất với con số nào dưới đây? (coi mỗi chiếc nón có hình dạng là một hình nón)
	\choice
	{\True $50 \mathrm{kg}$}
	{$76 \mathrm{kg}$}
	{$48 \mathrm{kg}$}
	{$38 \mathrm{kg}$}
	\loigiai{
		Theo giả thiết mỗi chiếc nón lá là một hình nón có bán kính đáy $R=\dfrac{50}{2}=25(cm)=0,25(\mathrm{m})$ và đường cao $h=30(cm)=0,3(\mathrm{m})$. \\
		{\color{red} HINH O DAY}\\
		Gọi $l$ là chiều cao của hình nón $\Rightarrow l=\sqrt{R^2+h^2}=\dfrac{\sqrt{61}}{20}(\mathrm{m})$.\\
		Diện tích xung quanh của 1 chiếc nón lá là $S_{\text{xq}}=\pi Rl=\pi\cdot 0,25\cdot\dfrac{\sqrt{61}}{20}=\dfrac{\pi\sqrt{61}}{80}(\mathrm{m}^2)$.\\
		Tổng diện tích xung quanh của 1000 chiếc nón là $S=1000\cdot\dfrac{\pi\sqrt{61}}{80}=\dfrac{25\pi\sqrt{61}}{2}(\mathrm{m}^2)$.\\
		Do đó khối lượng lá cần dùng là $\dfrac{S}{6,13}\approx 50,03(\mathrm{kg})$.
	}
\end{ex}
\begin{ex}
	Hai chiếc ly đựng chất lỏng giống hệt nhau, mỗi chiếc có phần chứa chất lỏng là một khối nón có chiều cao $2$ $\mathrm{dm}$ (mô tả như hình vẽ). Ban đầu chiếc ly thứ nhất chứa đầy chất lỏng, chiếc ly thứ hai để rỗng. Người ta chuyển chất lỏng từ ly thứ nhất sang ly thứ hai sao cho độ cao của cột chất lỏng trong ly thứ nhất còn $1$ $\mathrm{dm}$. Tính chiều cao $h$ của cột chất lỏng trong ly thứ hai sau khi chuyển (độ cao của cột chất lỏng tính từ đỉnh của khối nón đến mặt phẳng của chất lỏng – lượng chất lỏng coi như không hao hụt khi chuyển. Tính gần đúng $h$ với sai số không quá $0,01$ $\mathrm{dm}$). 
	{\color{red} HINH O DAY}
	\choice
	{$h\approx 1,41$ $\mathrm{dm}$}
	{$h\approx 1,89$ $\mathrm{dm}$}
	{\True $h\approx 1,91$ $\mathrm{dm}$}
	{$h\approx 1,73$ $\mathrm{dm}$}
	\loigiai{
		Gọi bán kính đáy, thể tích (phần chứa chất lỏng là một khối nón có chiều cao $2 dm$) của khối nón lần lượt là $r$; $V$.\\
		Gọi bán kính đáy, thể tích (tính từ đỉnh của khối nón đến mặt phẳng của chất lỏng của ly thứ nhất sau khi rót sang ly thứ hai) của khối nón lần lượt là $r_1$; $V_1$.\\
		Gọi bán kính đáy, chiều cao, thể tích (tính từ đỉnh của khối nón đến mặt phẳng của chất lỏng của ly thứ hai) của khối nón lần lượt là $r_2$; $h; V_2$.\\
		Ta có Thể tích chất lỏng ban đầu là $V=\dfrac{2}{3}\pi r^2$.\\
		Thể tích chất lỏng còn lại sau khi rót sang ly thứ hai là $V_1=\dfrac{1}{3}\pi r_1^2$.\\
		mà $\dfrac{r_1}{r}=\dfrac{1}{2}\Leftrightarrow r_1=\dfrac{r}{2}\Rightarrow V_1=\dfrac{1}{12}\pi r^2$.\\
		Thể tích chất lỏng ly thứ hai là $V_2=\dfrac{1}{3}\pi r_2^2h=V-V_1\Leftrightarrow\dfrac{1}{3}\pi r_2^2h=\dfrac{7}{12}\pi r^2\Leftrightarrow r_2^2h=\dfrac{7}{4}r^2$.\\
		mà $\dfrac{r_2}{r}=\dfrac{h}{2}\Leftrightarrow r_2=\dfrac{hr}{2}\Rightarrow h^3=7\Rightarrow h\approx 1,91 \mathrm{dm}$.\\
		Kết luận: $h\approx 1,91 \mathrm{dm}$.
	}
\end{ex}
\begin{ex}
	Cho một miếng tôn hình tròn có bán kính $50 \mathrm{cm}$. Biết hình nón có thể tích lớn nhất khi diện tích toàn phần của hình nón bằng diện tích miếng tôn ở trên. Khi đó hình nón có bán kính đáy là 
	\choice
	{$10\sqrt{2}(\mathrm{cm})$}
	{$50\sqrt{2}(\mathrm{cm})$}
	{$20(\mathrm{cm})$}
	{\True $25(\mathrm{cm})$}
	\loigiai{
		Ta có diện tích miếng tôn là $S=\pi\cdot 2500\left(cm^2\right)$.\\
		Diện tích toàn phần của hình nón là $S_{\mathrm{tp}}=\pi R^2+\pi\cdot R\cdot l$.\\
		Thỏa mãn yêu cầu bài toán ta có $\pi R^2+\pi\cdot R\cdot l=2500\pi\Leftrightarrow R^2+R\cdot l=2500=A\Leftrightarrow l=\dfrac{A}{R}-R$.\\
		Thể tích khối nón là\\
		$V=\dfrac{1}{3}\pi R^2\cdot h\Leftrightarrow V=\dfrac{1}{3}\pi R^2\cdot\sqrt{l^2-R^2}\Leftrightarrow V=\dfrac{1}{3}\pi R^2\cdot\sqrt{\left(\dfrac{A}{R}-R\right)^2-R^2}$ \\
		$ \Leftrightarrow V=\dfrac{1}{3}\pi R^2\cdot\sqrt{\dfrac{A^2}{R^2}-2A}\Leftrightarrow V=\dfrac{1}{3}\pi\cdot\sqrt{A^2\cdot R^2-2A\cdot R^4}\Leftrightarrow V=\dfrac{1}{3}\pi\cdot\sqrt{\dfrac{A^3}{8}-2A\left(R^2-\dfrac{A}{4}\right)^2} $ \\
		$ \Leftrightarrow V\leq\dfrac{1}{3}\pi\cdot\dfrac{A}{2}\sqrt{\dfrac{A}{2}} $. Dấu bằng xảy ra khi $R=\sqrt{\dfrac{A}{4}}=25$, vậy $V$ đạt GTLN khi $R=25$.
	}
\end{ex}
\begin{ex}
	[Phan Dăng Lưu - Huế - 2018]%Câu 12.
	Cho hình nón $(N)$ có đường cao $SO=h$ và bán kính đáy bằng $R$, gọi $M$ là điểm trên đoạn $SO$, đặt $OM=x$, $0<x<h$. $(C)$ là thiết diện của mặt phẳng $(P)$ vuông góc với trục $SO$ tại $M$, với hình nón $(N)$. Tìm $x$ để thể tích khối nón đỉnh $O$ đáy là $(C)$ lớn nhất. 
	\choice
	{$\dfrac{h}{2}$}
	{$\dfrac{h\sqrt{2}}{2}$}
	{$\dfrac{h\sqrt{3}}{2}$}
	{\True $\dfrac{h}{3}$}
	\loigiai{
		{\color{red} HINH O DAY}\\
		Ta có $BM$ là bán kính đường tròn $(C)$.\\
		Do tam giác nên $\dfrac{BM}{AO}=\dfrac{SM}{SO}\Leftrightarrow BM=\dfrac{AO\cdot SM}{SO}\Leftrightarrow BM=\dfrac{R(h-x)}{h}$.\\
		Thể tích của khối nón đỉnh $O$ đáy là $(C)$ là\\
		$V=\dfrac{1}{3}\pi BM^2\cdot OM =\dfrac{1}{3}\pi\left[\dfrac{R(h-x)}{h}\right]^2x =\dfrac{1}{3}\pi\dfrac{R^2}{h^2}(h-x)^2x$.\\
		Xét hàm số $f(x)=\dfrac{1}{3}\pi\dfrac{R^2}{h^2}(h-x)^2x$, $(0<x<h)$ ta có\\
		Ta có $f’(x)=\dfrac{1}{3}\pi\dfrac{R^2}{h^2}(h-x)(h-3x)$; $f’(x)=0\Leftrightarrow\dfrac{1}{3}\pi\dfrac{R^2}{h^2}(h-x)(h-3x)\Leftrightarrow x=\dfrac{h}{3}$.\\
		Lập bảng biến thiên ta có\\
		{\color{red} HINH O DAY}\\
		Từ bảng biến ta có thể tích khối nón đỉnh $O$ đáy là $(C)$ lớn nhất khi $x=\dfrac{h}{3}$.
	}
\end{ex}
\begin{ex}
	[THPT Lương Văn Tụy - Ninh Bình - 2018]%Câu 13.
	Cho hình tứ diện $ABCD$ có $AD\perp(ABC)$, $ABC$ là tam giác vuông tại $B$. Biết $BC=a$, $AB=a\sqrt{3}$, $AD=3a$. Quay các tam giác $ABC$ và $ABD$ (Bao gồm cả điểm bên trong $2$ tam giác) xung quanh đường thẳng $AB$ ta được $2$ khối tròn xoay. Thể tích phần chung của $2$ khối tròn xoay đó bằng
	\choice
	{\True $\dfrac{3\sqrt{3}\pi a^3}{16}$}
	{$\dfrac{8\sqrt{3}\pi a^3}{3}$}
	{$\dfrac{5\sqrt{3}\pi a^3}{16}$}
	{$\dfrac{4\sqrt{3}\pi a^3}{16}$}
	\loigiai{
		{\color{red} HINH O DAY}\\
		Khi quay tam giác $ABD$ quanh $AB$ ta được khối nón đỉnh $B$ có đường cao $BA$, đáy là đường tròn bán kính $AE=3$ cm. Gọi $I=AC\cap BE$, $IH\perp AB$ tại $H$. Phần chung của $2$ khối nón khi quay tam giác $ABC$ và tam giác $ABD$ quanh $AB$ là $2$ khối nón đỉnh $A$ và đỉnh $B$ có đáy là đường tròn bán kính $IH$.\\
		Ta có $\triangle IBC$ đồng dạng với $\triangle IEA\Rightarrow\dfrac{IC}{IA}=\dfrac{BC}{AE}=\dfrac{1}{3}\Rightarrow IA=3IC$.\\
		Mặt khác $IH\parallel BC\Rightarrow\dfrac{AH}{AB}=\dfrac{IH}{BC}=\dfrac{AI}{AC}=\dfrac{3}{4}\Rightarrow IH=\dfrac{3}{4}BC=\dfrac{3a}{4}$.\\
		Gọi $V_1$, $V_2$ lần lượt là thể tích của khối nón đỉnh $A$ và $B$ có đáy là hình tròn tâm $H$.\\
		$V_1=\dfrac{1}{3}\pi\cdot IH^2\cdot AH$.\\
		$V_2=\dfrac{1}{3}\pi\cdot IH^2\cdot BH$ \\
		$ \Rightarrow V=V_1+V_2\Rightarrow V=\dfrac{\pi}{3}\cdot IH^2\cdot AB\Rightarrow V=\dfrac{\pi}{3}\cdot\dfrac{9a^2}{16}\cdot a\sqrt{3}\Rightarrow V=\dfrac{3a^3\sqrt{3}}{16} $.
	}
\end{ex}
\begin{ex}[THPT Can Lộc - Hà Tĩnh - 2018]%Câu 14.
	Cho tam giác nhọn $ABC$, biết rằng khi quay tam giác này quanh các cạnh $AB$, $BC$, $CA$ ta lần lượt được các hình tròn xoay có thể tích là $672\pi$, $\dfrac{3136\pi}{5}$, $\dfrac{9408\pi}{13}$. Tính diện tích tam giác $ABC$. 
	\choice
	{$S=1979$}
	{$S=364$}
	{\True $S=84$}
	{$S=96$}
	\loigiai{
		Vì tam giác $ABC$ nhọn nên các chân đường cao nằm trong tam giác.\\
		Gọi $h_a$, $h_b$, $h_c$ lần lượt là đường cao từ đỉnh $A$, $B$, $C$ của tam giác $ABC$, và $a$, $b$, $c$ lần lượt là độ dài các cạnh $BC$, $CA$, $AB$.\\
		Khi đó\\
		Thể tích khối tròn xoay khi quay tam giác quanh $AB$ là $\dfrac{1}{3}\cdot\pi\cdot h_c^2\cdot c=672\pi$.\\
		Thể tích khối tròn xoay khi quay tam giác quanh $BC$ là $\dfrac{1}{3}\cdot\pi\cdot h_a^2\cdot a=\dfrac{3136\pi}{5}$.\\
		Thể tích khối tròn xoay khi quay tam giác quanh $CA$ là $\dfrac{1}{3}\cdot\pi\cdot h_b^2\cdot b=\dfrac{9408\pi}{13}$.\\
		Do đó $\heva{&\dfrac{1}{3}c\cdot h_c^2=672\\&\dfrac{1}{3}a\cdot h_a^2=\dfrac{3136}{5}\\&\dfrac{1}{3}b\cdot h_b^2=\dfrac{9408}{13}}\Leftrightarrow\heva{&\dfrac{4}{3}\dfrac{S^2}{c}=672\\&\dfrac{4}{3}\dfrac{S^2}{a}=\dfrac{3136}{5}\\&\dfrac{4}{3}\dfrac{S^2}{b}=\dfrac{9408}{13}}\Leftrightarrow\heva{&c=\dfrac{4S^2}{3\cdot 672}\\&a=\dfrac{20S^2}{3\cdot 3136}\\&b=\dfrac{52S^2}{3\cdot9408}}\Rightarrow(a+b+c)(a+b-c)(b+c-a)(c+a-b)=S^8\cdot\dfrac{1}{3^4}\cdot\dfrac{1}{9408}\cdot\dfrac{1}{28812}\Leftrightarrow 16S^2=S^8\cdot\dfrac{1}{3^4}\cdot\dfrac{1}{9408}\cdot\dfrac{1}{28812}\Leftrightarrow S^6=16\cdot 81\cdot 9408\cdot 28812\Leftrightarrow S=84$.
	}
\end{ex}
\begin{ex}
	[THPT Nam Trực - Nam Định - 2018]%Câu 15.
	Một chiếc ly dạng hình nón (như hình vẽ với chiều cao ly là $h$). Người ta đổ một lượng nước vào ly sao cho chiều cao của lượng nước trong ly bằng $\dfrac{1}{4}$ chiều cao của ly. Hỏi nếu bịt kín miệng ly rồi úp ngược ly lại thì tỷ lệ chiều cao của mực nước và chiều cao của ly nước bây giờ bằng bao nhiêu?
	{\color{red} HINH O DAY}
	\choice
	{\True $\dfrac{4-\sqrt[3]{63}}{4}$}
	{$\dfrac{\sqrt[3]{63}}{4}$}
	{$\dfrac{4-\sqrt{63}}{4}$}
	{$\dfrac{3}{4}$}
	\loigiai{
		{\color{red} HINH O DAY}\\
		Giả sử ly có chiều cao $h$ và đáy là đường tròn có bán kính $r$, nên có thể tích $V=\dfrac{1}{3}\pi hr^2$.\\
		Khối nước trong ly có chiều cao bằng $\dfrac{1}{4}$ chiều cao của ly nên khối nước tạo thành khối nón có chiều cao bằng $\dfrac{h}{4}$ và bán kính đáy $\dfrac{r}{4}$ thể tích nước bằng $\dfrac{1}{3}\cdot\dfrac{h}{4}\pi\left(\dfrac{r}{4}\right)^2=\dfrac{1}{64}\cdot\left(\dfrac{1}{3}\pi hr^2\right)=\dfrac{1}{64}V$.\\
		Do đó thể tích khoảng không bằng $V-\dfrac{1}{64}V=\dfrac{63}{64}V$.\\
		Nên khi úp ngược ly lại thì ta có các tỉ lệ: $\dfrac{x}{r}=\dfrac{h’}{h}\Rightarrow x=\dfrac{r\cdot h’}{h}$.\\
		Suy ra: thể tích khoảng không bằng $\dfrac{1}{3}h’\cdot\pi x^2=\dfrac{1}{3}\cdot h’\cdot\pi\cdot\left(\dfrac{r\cdot h’}{h}\right)^2=\dfrac{1}{3}\pi hr^2\cdot\left(\dfrac{h’}{h}\right)^3=\left(\dfrac{h’}{h}\right)^3\cdot V$ \\
		$\Rightarrow\dfrac{63}{64}V=\left(\dfrac{h’}{h}\right)^3V\Rightarrow\left(\dfrac{h’}{h}\right)^3=\dfrac{63}{64}\Rightarrow\dfrac{h’}{h}=\sqrt[3]{\dfrac{63}{64}}=\dfrac{\sqrt[3]{63}}{4}\Rightarrow h’=\dfrac{\sqrt[3]{63}}{4}h $.\\
		Nên chiều cao mực nước bằng $h-h’=h-\dfrac{\sqrt[3]{63}}{4}h=\dfrac{4-\sqrt[3]{63}}{4}h$.\\
		Vậy tỷ lệ chiều cao của mực nước và chiều cao của ly nước bây giờ bằng $\dfrac{4-\sqrt[3]{63}}{4}$.
	}
\end{ex}
\begin{ex}
	[Nam Định - 2018]%Câu 16.
	Cho tam giác $ABC$ có $\widehat{A}=120^{\circ}, AB=AC=a$. Quay tam giác $ABC$ (bao gồm cả điểm trong tam giác) quanh đường thẳng $AB$ ta được một khối tròn xoay. Thể tích khối tròn xoay đó bằng 
	\choice
	{$\dfrac{\pi a^3}{3}$}
	{\True $\dfrac{\pi a^3}{4}$}
	{$\dfrac{\pi a^3\sqrt{3}}{2}$}
	{$\dfrac{\pi a^3\sqrt{3}}{4}$}
	\loigiai{
		{\color{red} HINH O DAY}\\
		Theo định lý cosin ta có: $BC=\sqrt{AB^2+AC^2-2AB\cdot AC\cdot\cos A}=a\sqrt{3}$.\\
		Quay tam giác $ABC$ (bao gồm cả điểm trong tam giác) quanh đường thẳng $AB$ ta được một.\\
		khối tròn xoay có thể tích $V=V_1-V_2$ với $V_1,V_2$ là thể tích khối tròn xoay khi quay tam giác.\\
		vuông $BCH$ và tam giác $ACH$ quay xung quanh với $HB$ ($H$ là hình chiếu vuông góc của $t=x^2\geq 0$ lên $AB$).\\
		Ta tính được $CH=\dfrac{a\sqrt{3}}{2}; AH=\dfrac{a}{2}$. Khi đó, ta có:\\
		$V=\dfrac{1}{3}\pi\cdot CH^2\cdot BH-\dfrac{1}{3}\pi\cdot CH^2\cdot AH=\dfrac{1}{3}\pi\cdot CH^2\cdot AB=\dfrac{1}{3}\pi\cdot\left(\dfrac{a\sqrt{3}}{2}\right)^2\cdot a=\dfrac{\pi a^3}{4}$.
	}
\end{ex}
\begin{ex}
	[Chuyên Bắc Giang 2019]%Câu 17.
	Một vật $N_1$ có dạng hình nón có chiều cao bằng $40cm$. Người ta cắt vật $N_1$ bằng một mặt cắt song song với mặt đáy của nó để được một hình nón nhỏ $N_2$ có thể tích bằng $\dfrac{1}{8}$ 
	thể tích $N_1$. Tính chiều cao $h$ của hình nón $N_2$?\\
	{\color{red} HINH O DAY}\\
	\choice
	{$10cm$}
	{\True $20cm$}
	{$40cm$}
	{$5cm$}
	\loigiai{
		{\color{red} HINH O DAY}\\{\color{red} HINH O DAY}\\
		Gọi $r_1=BE$, $h_1=AB$ lần lượt là bán kính đáy và chiều cao của hình nón $N_1$.\\
		Gọi $r_2=CD$, $h=AC$ lần lượt là bán kính đáy và chiều cao của hình nón $N_2$.\\
		Khi đó thể tích của hai khối nón lần lượt là\\
		$V_1=\dfrac{1}{3}\pi r_1^2h_1$.\\
		$V_2=\dfrac{1}{3}\pi r_2^2h$.\\
		Theo đề bài ta có\\
		$\dfrac{V_2}{V_1}=\dfrac{\frac{1}{3}\pi r_2^2h}{\frac{1}{3}\pi r_1^2h_1}=\left(\dfrac{r_2}{r_1}\right)^2\cdot\dfrac{h}{h_1}=\dfrac{1}{8} (1)$.\\
		Xét hai tam giác đồng dạng $ACD,ABE$ có:\\
		$\dfrac{AC}{AB}=\dfrac{CD}{BE}\Leftrightarrow\dfrac{r_2}{r_1}=\dfrac{h}{h_1} (2)$.\\
		Từ $(1)$ và $(2)$ suy ra $\left(\dfrac{h}{h_1}\right)^3=\dfrac{1}{8}\Leftrightarrow\dfrac{h}{h_1}=\dfrac{1}{2}\Leftrightarrow h=\dfrac{1}{2}h_1=20$.
	}
\end{ex}
\begin{ex}[Toán Học Tuổi Trẻ 2019]%Câu 18.
	Cho một tấm bìa hình dạng tam giác vuông, biết b và c là độ dài cạnh tam giác vuông của tấm một khối tròn xoay. Hỏi thể tích $V$ của khối tròn xoay sinh ra bởi tấm bìa bằng bao nhiêu?
	\choice
	{$V=\dfrac{b^2c^2}{3\sqrt{b^2+c^2}}$}
	{\True $V=\dfrac{\pi b^2c^2}{3\sqrt{b^2+c^2}}$}
	{$V=\dfrac{2\pi b^2c^2}{3\sqrt{b^2+c^2}}$}
	{$V=\dfrac{\pi b^2c^2}{3\sqrt{2(b^2+c^2)}}$}
	\loigiai{
		{\color{red} HINH O DAY}\\
		Gọi tam giác vuông là $ABC$, kẻ $AH\perp BC$, $H$ là chân đường cao.\\
		Khi đó $\dfrac{1}{AH^2}=\dfrac{1}{AB^2}+\dfrac{1}{AC^2}\Rightarrow AH=\dfrac{bc}{\sqrt{b^2+c^2}}$.\\
		Thể tích khối tròn xoay cần tính bằng tổng thể tích 2 khối nón tạo bởi hai tam giác vuông $ACH$ và $ABH$ khi quay quanh trục $BC$.\\
		Khối nón tạo bởi tam giác vuông $ACH$ khi quay quanh trục $BC$ có thể tích $V_1=\dfrac{1}{3}\pi CH\cdot AH^2$.\\
		Khối nón tạo bởi tam giác vuông $ABH$ khi quay quanh trục $BC$ có thể tích $V_2=\dfrac{1}{3}\pi BH\cdot AH^2$.\\
		Thể tích khối tròn xoay cần tính là\\
		$\begin{aligned}&V=V_1+V_2=\dfrac{1}{3}\pi CH\cdot AH^2+\dfrac{1}{3}\pi BH\cdot AH^2\\&=\dfrac{1}{3}\pi BC\cdot AH^2=\dfrac{1}{3}\pi\sqrt{b^2+c^2}\cdot (\dfrac{bc}{\sqrt{b^2+c^2}})^2=\dfrac{\pi b^2c^2}{3\sqrt{b^2+c^2}}.\end{aligned}$.
	}
\end{ex}
\begin{ex}
	Một chiếc thùng chứa đầy nước có hình một khối lập phương. Đặt vào trong thùng đó một khối nón sao cho đỉnh khối nón trùng với tâm một mặt của khối lập phương, đáy khối nón tiếp xúc với các cạnh của mặt đối diện. Tính tỉ số thể tích của lượng nước trào ra ngoài và lượng nước còn lại ở trong thùng. \\
	{\color{red} HINH O DAY}
	\choice
	{\True $\dfrac{\pi}{12-\pi}$}
	{$\dfrac{1}{11}$}
	{$\dfrac{\pi}{12}$}
	{$\dfrac{11}{12}$}
	\loigiai{
		Coi khối lập phương có cạnh $1$. Thể tích khối lập phường là $V=1$.\\
		Từ giả thiết ta suy ra khối nón có chiều cao $h=1$, bán kính đáy $r=\dfrac{1}{2}$.\\
		Thể tích lượng nước trào ra ngoài là thể tích $V_1$ của khối nón.\\
		Ta có: $V_1=\dfrac{1}{3}\pi r^2h=\dfrac{1}{3}\pi\cdot\dfrac{1}{4}\cdot 1=\dfrac{\pi}{12}$.\\
		Thể tích lượng nước còn lại trong thùng là $V_2=V-V_1=1-\dfrac{\pi}{12}=\dfrac{12-\pi}{12}$.\\
		Do đó: $\dfrac{V_1}{V_2}=\dfrac{\pi}{12-\pi}$.
	}
\end{ex}
\begin{ex}
	[THPT Bạch Đằng Quảng Ninh 2019]%Câu 20.
	Một cái phễu có dạng hình nón. Người ta đổ một lượng nước vào phễu sao cho chiều cao của lượng nước trong phễu bằng $\dfrac{1}{3}$ chiều cao của phễu. Hỏi nếu bịt kín miệng phễu rồi lộn ngược phễu lên thì chiều cao của mực nước xấp xỉ bằng bao nhiêu? Biết rằng chiều cao của phễu là $15\mathrm{cm}$.\\ 
	{\color{red} HINH O DAY}
	\choice
	{$0,501(\mathrm{cm})$}
	{$0,302(\mathrm{cm})$}
	{$0,216(\mathrm{cm})$}
	{\True $0,188(\mathrm{cm})$}
	\loigiai{
		Gọi $h_1$ là chiều cao của nước ta có $h_1=\dfrac{1}{3}h$. Từ hình vẽ ta có: $\dfrac{h_1}{h}=\dfrac{r_1}{r}\Rightarrow r_1=\dfrac{1}{3}r$; $\dfrac{h_2}{h}=\dfrac{r_2}{r}\Leftrightarrow\dfrac{h_2}{r_2}=\dfrac{h}{r}\Leftrightarrow\dfrac{r}{h}h_2=r_2$.\\
		Ta có thể tích của nước trước và sau khi lôn ngược là như nhau:\\
		$h_1\cdot\pi r_1^2=h\cdot\pi r^2-h_2\cdot\pi r_2^2\Leftrightarrow h_2=\dfrac{h\pi r^2-h_1\pi r_1^2}{\pi r_2^2}\Leftrightarrow h_2=\dfrac{hr^2-h_1\cdot r_1^2}{r_2^2}\Leftrightarrow h_2=\dfrac{hr^2}{r_2^2}-\dfrac{h_1\cdot r_1^2}{r_2^2}\Leftrightarrow h_2=\dfrac{h^3}{h_2^2}-\dfrac{h_1\cdot\frac{1}{9}r^2}{\frac{r^2}{h^2}h_2^2}\Leftrightarrow h_2=\dfrac{h^3}{h_2^2}-\dfrac{h_1\cdot\frac{1}{9}}{\frac{1}{h^2}h_2^2}\Leftrightarrow h_2=\dfrac{15^3}{h_2^2}-\dfrac{5\cdot\frac{1}{9}{\cdot 15}^2}{h_2^2}\Leftrightarrow h_2^3=15^3-5\cdot\dfrac{1}{9}\cdot 15^2\Leftrightarrow h_2^3=3250\Leftrightarrow h_2=\sqrt[3]{3250}$ Vậy bịt kín miệng phễu rồi lộn ngược phễu lên thì chiều cao của mực nước xấp xỉ bằng $0,188(\mathrm{cm})$.
	}
\end{ex}
\begin{ex}
	[Chuyên Hùng Vương Gia Lai 2019]%Câu 21.
	Hai hình nón bằng nhau có chiều cao bằng $2$ $\mathrm{dm}$ được đặt như hình vẽ bên (mỗi hình đều đặt thẳng đứng với đỉnh nằm phía dưới). Lúc đầu, hình nón trên chứa đầy nước và hình nón dưới không chứa nước. Sau đó, nước được chảy xuống hình nón dưới thông qua lỗ trống ở đỉnh của hình nón trên. Hãy tính chiều cao của nước trong hình nón dưới tại thời điểm khi mà chiều cao của nước trong hình nón trên bằng $1$ $\mathrm{dm}$.\\
	{\color{red} HINH O DAY}\\
	\choice
	{\True $\sqrt[3]{7}$}
	{$\dfrac{1}{3}$}
	{$\sqrt[3]{5}$}
	{$\dfrac{1}{2}$}
	\loigiai{
		{\color{red} HINH O DAY}\\
		Gọi $a$ là bán kính đáy hình nón;\\
		$V_1,V_2$ lần lượt là thể tích của hình nón trên lúc chứa đầy nước và khi chiều cao của nước bằng $1$ $\mathrm{dm}$;\\
		$h$, $V_3$ lần lượt là chiều cao của nước, thể tích của hình nón dưới khi chiều cao của nước trong hình nón trên bằng $1$ $\mathrm{dm}$;\\
		$R$, $r$ lần lượt là bán kính của hình nón trên của nước, bán kính của hình nón dưới của nước khi chiều cao của nước trong hình nón trên bằng $1$ $\mathrm{dm}$.\\
		Ta có: $\dfrac{R}{a}=\dfrac{1}{2}\Rightarrow R=\dfrac{a}{2}$.\\
		Thể tích nước của hình nón trên khi chiều cao bằng 1 là $V_2={style{{1}\over{3}}}\cdot 1\cdot\pi\left({style{{1}\over{2}}}a\right)^2=\dfrac{\pi a^2}{12}$.\\
		Mặt khác: $\dfrac{r}{a}=\dfrac{h}{2}\Rightarrow r=\dfrac{ah}{2}$.\\
		Do đó thể tích nước hình nón dưới $V_3={style{{1}\over{3}}}\cdot h\cdot\pi\left({style{{h}\over{2}}}a\right)^2=\dfrac{\pi a^2h^3}{12}$.\\
		Thể tích nước của hình nón trên khi đầy nước $V_1={style{{1}\over{3}}}\cdot 2\cdot\pi a^2$.\\
		Lại có: $V_3=V_1-V_2\Rightarrow\dfrac{\pi a^2h^3}{12}={style{{1}\over{3}}}\cdot 2\cdot\pi a^2-\dfrac{\pi a^2}{12}\Leftrightarrow 1+h^3=8\Leftrightarrow h=\sqrt[3]{7}$.
	}
\end{ex}
\begin{ex}
	[Chuyen Phan Bội Châu Nghệ An 2019]%Câu 22.
	Tại trung tâm thành phố người ta tạo điểm nhấn bằng cột trang trí hình nón có kích thước như sau: chiều dài đường sinh $l=10 \mathrm{m}$, bán kính đáy $R=5 \mathrm{m}$. Biết rằng tam giác SAB là thiết diện qua trục của hình nón và C là trung điểm của $SB$. Trang trí một hệ thống đèn điện tử chạy từ A đến C trên mặt nón. Xác định giá trị ngắn nhất của chiều dài dây đèn điện tử. 
	\choice
	{$15 \mathrm{m}$}
	{$10 \mathrm{m}$}
	{$5\sqrt{3} \mathrm{m}$}
	{\True $5\sqrt{5} \mathrm{m}$}
	\loigiai{
		Cắt hình nón theo hai đường sinh $SA$, $SB$ rồi trải ra ta được hình (H2) như sau: 
		{\color{red} HINH O DAY}\\{\color{red} HINH O DAY}\\
		Khi đó, chiều dài dây đèn ngắn nhất là độ dài đoạn thẳng AC trên hình H2.
		Chu vi cung tròn $\overset\frown{AB}$: $C=\dfrac{1}{2}\cdot 2\pi\cdot 5=5\pi$ \\
		$ \Rightarrow\triangle SAC $ vuông tại $S$\\
		$ \Rightarrow AC=\sqrt{SA^2+SC^2}=\sqrt{10^2+5^5}=5\sqrt{5} \mathrm{m} $.
	}
\end{ex}
\begin{ex}
	Một cái phểu có dạng hình nón, chiều cao của phểu là $20\mathrm{cm}$. Người ta đổ một lượng nước vào phểu sao cho chiều cao của cột nước trong phểu là $10\mathrm{cm}$. Nếu bịt kím miêng phểu rồi lật ngược lên chiều cao của cột nước trong phểu gần nhất với giá trị nào sau đây. \\
	{\color{red} HINH O DAY}
	\choice
	{$1,07\mathrm{cm}$}
	{$0,97\mathrm{cm}$}
	{$0,67\mathrm{cm}$}
	{\True $0,87\mathrm{cm}$}
	\loigiai{
		Gọi $R$ là bán kính đáy của cái phểu ta có $\dfrac{R}{2}$ là bán kính của đáy chứa cột nước.\\
		Ta có thể tích phần nón không chứa nước là $V=\dfrac{1}{3}\pi(R)^2\cdot 20-\dfrac{1}{3}\pi\left(\dfrac{R}{2}\right)^2\cdot 10=\dfrac{35}{6}\pi R^2$.\\
		Khi lật ngược phểu Gọi $h$ chiều cao của cột nước trong phểu. phần thể tích phần nón không chứa nước là $V=\dfrac{1}{3}\pi(20-h)\left(\dfrac{R(20-h)}{20}\right)^2=\dfrac{1}{1200}\pi(20-h)^3R^2$.\\
		$\dfrac{1}{1200}\pi(20-h)^3R^2=\dfrac{35}{6}\pi R^2\Rightarrow(20-h)^3=7000\Rightarrow h\approx 0,87$.
	}
\end{ex}
\begin{ex}
	Giả sử đồ thị hàm số $y=\left(m^2+1\right)x^4-2mx^2+m^2+1$ có 3 điểm cực trị là $A,B,C$ mà $x_A<x_B<x_C$. Khi quay tam giác $ABC$ quanh cạnh $AC$ ta được một khối tròn xoay. Giá trị của $m$ để thể tích của khối tròn xoay đó lớn nhất thuộc khoảng nào trong các khoảng dưới đây: 
	\choice
	{$(4;6)$}
	{\True $(2;4)$}
	{$(-2;0)$}
	{$(0;2)$}
	\loigiai{
		$y’=4(m^2+1)x^3-4mx=4x\left[(m^2+1)x^2-m\right]$.\\
		$y’=0\Leftrightarrow 4x\left[(m^2+1)x^2-m\right]=0\Leftrightarrow\hoac{&x=0\\&x=\pm\sqrt{\dfrac{m}{m^2+1}} (m>0).}$ \\
		Với $m>0$ thì đồ thị hàm số có 3 điểm cực trị (với $x_A<x_B<x_C$) là\\
		$A(-\sqrt{\dfrac{m}{m^2+1}};-\dfrac{m^2}{m^2+1}+m^2+1)$; $B(0;m^2+1)$; $C(\sqrt{\dfrac{m}{m^2+1}};-\dfrac{m^2}{m^2+1}+m^2+1)$.\\
		Quay $\triangle ABC$ quanh $AC$ thì được khối tròn xoay có thể tích là 
		{\color{red} HINH O DAY}\\
		$V=2\cdot\dfrac{1}{3}\cdot\pi r^2h=\dfrac{2}{3}\pi BI^2\cdot IC =\dfrac{2}{3}\pi\left(\dfrac{m^2}{m^2+1}\right)^2\cdot\sqrt{\dfrac{m}{m^2+1}}=\dfrac{2}{3}\pi\sqrt{\dfrac{m^9}{\left(m^2+1\right)^5}}$.\\
		Xét hàm số $f(x)=\dfrac{m^9}{\left(m^2+1\right)^5}$.\\
		Có: $f’(x)=\dfrac{m^8(9-m^2)}{\left(m^2+1\right)^6}$; $f’(x)=0\Leftrightarrow m=3 (m>0)$.\\
		Ta có BBT:\\
		{\color{red} HINH O DAY}\\
		Vậy thể tích cần tìm lớn nhất khi $m=3$.
	}
\end{ex}
\begin{ex}
	Khi cắt hình nón có chiều cao $16 \mathrm{cm}$ và đường kính đáy $24 \mathrm{cm}$ bởi một mặt phẳng song song với đường sinh của hình nón ta thu được thiết diện có diện tích lớn nhất gần với giá trị nào sau đây?
	\choice
	{$170$}
	{$260$}
	{$294$}
	{\True $208$}
	\loigiai{
		{\color{red} HINH O DAY}\\
		Cắt hình nón bởi một mặt phẳng song song với đường sinh của hình nón ta thu được thiết diện là một parabol.\\
		Xét dây cung bất kỳ chứa đoạn $KH$ như hình vẽ, suy ra tồn tại đường kính $AB\perp KH$, trong tam giác $SAB$, $KE\parallel SA, E\in SB$, Suy ra Parabol nhận $KE$ làm trục như hình vẽ chính là một thiết diện thỏa yêu cầu bài toán. (Thiết diện này song song với đường sinh $SA$).\\
		Đặt $BK=x$ (với $0<x<24$).\\
		Trong tam giác $ABH$ có: $HK^2=BK\cdot AK=x(24-x)$.\\
		Trong tam giác $SAB$ có: $\dfrac{KE}{SA}=\dfrac{BK}{BA}\Leftrightarrow KE=\dfrac{BK}{BA}\cdot SA\Leftrightarrow KE=\dfrac{5x}{6}$.\\
		Thiết diện thu được là một parabol có diện tích: $S=\dfrac{4}{3}KH\cdot KE$.\\
		Ta có $S^2=\dfrac{16}{9}KH^2\cdot KE^2=\dfrac{16}{9}\cdot x(24-x)\cdot\dfrac{25x^2}{36}=\dfrac{100}{81}\cdot\left(24x^3-x^4\right)\Rightarrow S=\dfrac{10}{9}\cdot\sqrt{24x^3-x^4}$.\\
		Đặt $f(x)=24x^3-x^4$, với $0<x<24$.\\
		Ta có $f’(x)=72x^2-4x^3$. Suy ra $f’(x)=0\Leftrightarrow 72x^2-4x^3=0\Leftrightarrow\hoac{&x=0\\&x=18.}$ \\
		Bảng biến thiên: \\
		{\color{red} HINH O DAY}\\
		Vậy thiết diện có diện tích lớn nhất là $\dfrac{10}{9}\sqrt{34992}\approx 207,8 \mathrm{cm}^2$.
	}
\end{ex}
\begin{ex}
	Một hình nón tròn xoay có đường sinh $2a$. Thể tích lớn nhất của khối nón đó là
	\choice
	{$\dfrac{16\pi a^3}{3\sqrt{3}}$}
	{\True $\dfrac{16\pi a^3}{9\sqrt{3}}$}
	{$\dfrac{4\pi a^3}{3\sqrt{3}}$}
	{$\dfrac{8\pi a^3}{3\sqrt{3}}$}
	\loigiai{
		Gọi hình nón tròn xoay có đường sinh $l=2a$ có bán kính đáy là $R$ và đường cao là $h$.\\
		Thể tích khối nón: $V=\dfrac{1}{3}\pi R^2h$. Ta có: $R^2+h^2=4a^2$.\\
		Áp dụng bất đẳng thức Cô si: $4a^2=R^2+h^2=\dfrac{R^2}{2}+\dfrac{R^2}{2}+h^2\geq 3\sqrt[3]{\dfrac{R^4h^2}{4}}$ \\
		$ \Rightarrow\dfrac{R^4h^2}{4}\leq\dfrac{64}{27}a^6\Rightarrow\dfrac{1}{3}\pi R^2h\leq\dfrac{16\pi\sqrt{3}}{27}a^3 $.\\
		Đẳng thức xảy ra khi và chỉ khi $\heva{&\dfrac{R^2}{2}=h^2\\&h^2+R^2=4a^2}\Leftrightarrow\heva{&h=\dfrac{2\sqrt{3}}{3}a\\&R=\dfrac{2\sqrt{6}}{3}a.}$ \\
		Khi đó $V_{\max} =\dfrac{16\pi\sqrt{3}}{27}a^3$.
	}
\end{ex}
\begin{ex}
	[Cụm Liên Trường Hải Phòng 2019]%Câu 27.
	Huyền có một tấm bìa như hình vẽ, Huyền muốn biến đường tròn đó thành một cái phễu hình nón. Khi đó Huyền phải cắt bỏ hình quạt tròn $AOB$ rồi dán $OA$, $OB$ lại với nhau. Gọi $x$ là góc ở tâm hình quạt tròn dùng làm phễu. Tìm $x$ để thể tích phểu lớn nhất?
	\choice
	{\True $\dfrac{2\sqrt{6}}{3}\pi$}
	{$\dfrac{\pi}{3}$}
	{$\dfrac{\pi}{2}$}
	{$\dfrac{\pi}{4}$}
	\loigiai{
		{\color{red} HINH O DAY}\\{\color{red} HINH O DAY}\\
		Ta có diện tích của hình phểu $S_{\text{xq}}=\dfrac{R^2x}{2}\Rightarrow r=\dfrac{xR}{2\pi}$ là bán kính của đáy phểu; $\Rightarrow x=\dfrac{2\pi r}{R}$.\\
		$V=\dfrac{1}{3}\pi r^2h=\dfrac{1}{3}\pi r^2\sqrt{R^2-r^2}=\dfrac{1}{3}\pi\sqrt{r^4\cdot R^2-r^6}$ là thể tích của phểu.\\
		Xét hàm số phụ $y=r^4\cdot R^2-r^6\Rightarrow y’=4r^3\cdot R^2-6r^5$.\\
		$y’=0\Leftrightarrow 2\cdot R^2-3r^2=0\Leftrightarrow r=\dfrac{\sqrt{6}}{3}R$ 
		{\color{red} HINH O DAY}\\
		Vậy $y$ max thì $V$ và $V$ max khi $r=\dfrac{R\sqrt{6}}{3}\Leftrightarrow x=\dfrac{2\pi r}{R}\Leftrightarrow x=\dfrac{2\pi R\sqrt{6}}{3R}\Leftrightarrow x=\dfrac{2\pi\sqrt{6}}{3}$.
	}
\end{ex}
\begin{ex}
	[Chuyên Phan Bội Châu 2019]%Câu 28.
	Tại trung tâm một thành phố người ta tạo điểm nhấn bằng cột trang trí hình nón có kích thước như sau: đường sinh $l=10\mathrm{m}$, bán kính đáy $R=5\mathrm{m}$. Biết rằng tam giác $SAB$ là thiết diện qua trục của hình nón và $C$ là trung điểm của $SB$. Trang trí một hệ thống đèn điện tử chạy từ $A$ đến $C$ trên mặt nón. Định giá trị ngắn nhất của chiều dài dây đèn điện tử. 
	\choice
	{$15 \mathrm{m}$}
	{$10 \mathrm{m}$}
	{$5\sqrt{3} \mathrm{m}$}
	{\True $5\sqrt{5} \mathrm{m}$}
	\loigiai{
		{\color{red} HINH O DAY}\\
		Ta có $\triangle SAB$ cân và $SB=AB\Rightarrow\triangle SAB$ đều.\\
		Diện tích xung quanh hình nón là $S_{\text{xq}}=\pi Rl=50\pi(m^2)$.\\
		Vẽ $(P)$ đi qua $C$ và vuông góc với $AB$. Mặt phẳng $(P)$ cắt hình nón theo thiết diện là một Elip.\\
		Khi đó, chiều dài dây đèn điện tử ngắn nhất chính là chiều dài dây cung $AC$ trên Elip.\\
		Ta dùng phương pháp trải hình ra sẽ thấy ngay như sau
		{\color{red} HINH O DAY}\\
		Hình trải dài là một hình quạt với $AB$ là độ dài nửa đường tròn và $AB=R\cdot\pi=5\pi(\mathrm{m})$.\\
		$S_{ABS}=\dfrac{1}{2}S=25\pi\Leftrightarrow\dfrac{\widehat{ASB}\cdot\pi R_1^2}{360}=25\pi\Leftrightarrow\widehat{ASB}=\dfrac{360\cdot 25\pi}{\pi{\cdot 10}^2}=90^{\circ}$.\\
		Vậy $\triangle SAC$ vuông tại $S$ và $AC=\sqrt{SA^2+SC^2}=5\sqrt{5}$.
	}
\end{ex}
\begin{ex}
	[Sở Tuyên Quang - 2021]%Câu 29.
	Cho hình nón có chiều cao bằng $3$. Một mặt phẳng $(\alpha)$ đi qua đỉnh hình nón và cắt hình nón theo một thiết diện là tam giác đều, góc giữa trục của hình nón và mặt phẳng $(\alpha)$ là $45^{\circ}$. Thể.\\
	tích của hình nón đã cho bằng
	\choice
	{$5\sqrt{24}\pi$}
	{\True $15\pi$}
	{$45\pi$}
	{$15\sqrt{25}\pi$}
	\loigiai{
		{\color{red} HINH O DAY}\\
		Giả sử mặt phẳng $(\alpha)$ cắt hình nón theo thiết diện là tam giác $SAB$. Theo giả thiết thì tam.\\
		giác $SAB$ đều. Gọi $O$ là tâm của đường tròn đáy; $h,r$ lần lượt là đường cao và bán kính của.\\
		hình nón.\\
		Gọi $M$ là trung điểm của $AB$, tam giác $OAB$ cân đỉnh $O$ nên $OM\perp AB$ và $SO\perp AB$ suy ra $AB\perp(SOM)$.\\
		Dựng $OK\perp SM$ ($K\in SM$).\\
		Theo trên ta có $AB\perp(SOM)\Rightarrow AB\perp OK\Rightarrow OK\perp(SAB)$.\\
		Vậy góc tạo bởi giữa trục $SO$ và mặt phẳng $(SAB)$ là $\widehat{OSM}=45^{\circ}$.\\
		Xét tam giác vuông $SOM$ có $\cos\widehat{OSM}=\dfrac{SO}{SM}\Rightarrow SM=\dfrac{3}{\frac{\sqrt{2}}{2}}=3\sqrt{2},OM=SO\tan\cos\widehat{OSM}=3$.\\
		Do tam giác $SAB$ đều nên $SM=\dfrac{AB\sqrt{3}}{2}\Rightarrow AB=\dfrac{2SM}{\sqrt{3}}=\dfrac{2\cdot 3\sqrt{2}}{\sqrt{3}}=2\sqrt{6}\Rightarrow AM=\sqrt{6}$.\\
		Xét tam giác vuông $OAM$ có $r=OA=\sqrt{OM^2+AM^2}=\sqrt{15}$. Suy ra thể tích của hình nón đã.\\
		cho là $V=\dfrac{1}{3}\pi r^2h=\dfrac{1}{3}\pi\cdot 15\cdot 3=15\pi$.
	}
\end{ex}
\begin{ex}
	Cổ động viên bóng đá của đội tuyển Indonesia muốn làm một chiếc mũ có dạng hình nón sơn hai màu Trắng và Đỏ như trên quốc kỳ. Biết thiết diện qua trục của hình nón là tam giác vuông cân. Cổ động viên muốn sơn màu Đỏ ở bề mặt phần hình nón có đáy là cung nhỏ $\wideparen{MBN}$, phần còn là của hình nón sơn màu Trắng. Tính tỉ số phần diện tích hình nón được sơn màu Đỏ với phần diện tích sơn màu Trắng. \\
	{\color{red} HINH O DAY}
	\choice
	{$\dfrac{2}{7}$}
	{$\dfrac{2}{5}$}
	{$\dfrac{1}{4}$}
	{\True $\dfrac{1}{3}$}
	\loigiai{
		Ta có $SO=OA=OB=r\Rightarrow SM=r\sqrt{2}=MN$.\\
		Do dó tam giác $OMN$ vuông cân tại $O$.\\
		Gọi $S$ là diện tích xung quanh của hình nón, $S_d$ là diện tích xung quanh của phần hình nón được sơn màu đỏ, ứng với góc $\widehat{MON}=90^{\circ}$ nên $\dfrac{S_1}{S}=\dfrac{90^{\circ}}{360^{\circ}}=\dfrac{1}{4}\Rightarrow\dfrac{S_d}{S_t}=\dfrac{1}{3}$.
	}
\end{ex}
\begin{ex}
	[Chuyên ĐHSP - 2021]%Câu 31.
	Cho hình nón $(T)$ có đỉnh là $S$, có đáy là đường tròn $(C_1)$ tâm $O$, bán kính bằng 2. Chiều cao hình nón $(T)$ bằng 2. Khi cắt hình nón $(T)$ bới mặt phẳng đi qua trung điểm của đoạn $SO$ và song song với đáy của hình nón, ta được đường tròn $(C_2)$ tâm $I$. Lấy hai điểm $A,B$ lần lượt trên hai đường tròn $(C_2)$ và $(C_1)$ sao cho góc giữa $\overrightarrow{IA}$ và $\overrightarrow{OB}$ bằng $60^{\circ}$. Thể tích khối tứ diện $IAOB$ bằng
	\choice
	{\True $\dfrac{\sqrt{3}}{6}$}
	{$\dfrac{\sqrt{3}}{12}$}
	{$\dfrac{\sqrt{3}}{4}$}
	{$\dfrac{\sqrt{3}}{24}$}
	\loigiai{
		{\color{red} HINH O DAY}\\
		Kẻ $OA’\parallel IA$, $\widehat{\left(\overrightarrow{IA},\overrightarrow{OB}\right)}=\widehat{\left(\overrightarrow{OA’},\overrightarrow{OB}\right)} =\widehat{A’OB}=60^{\circ}$.\\
		Goi $K$ là trung điểm của $IA’$, khi đó $\heva{&BI\perp OA’\\&BI\perp IO}\Rightarrow BI\perp(SOA’)$ \\
		$ \Rightarrow\mathrm{d}\left(B,(SOA’)\right)=BK=\dfrac{2\sqrt{3}}{2}=\sqrt{3} $.\\
		Ta có $IO\perp IA$, $IO=\dfrac{SO}{2}=1$, $IA=\dfrac{OA’}{2}=1$.\\
		Nên $S_{IAO}=\dfrac{1}{2}IA\cdot IO=\dfrac{1}{2}\cdot 1\cdot 1=\dfrac{1}{2}$.\\
		Vậy $V_{B.AIO}=\dfrac{1}{3}\cdot S_{OIA}\cdot BK =\dfrac{1}{3}\dfrac{1}{2}\cdot\sqrt{3} =\dfrac{\sqrt{3}}{6}$.
	}
\end{ex}
\begin{ex}
	[Chuyên Tuyên Quang - 2021]%Câu 32.
	Cho hình chữ nhật $ABCD$ có $AB=6$, $AD=8$. Thể tích của vật thể tròn xoay thu được khi quay hình chữ nhật $ABCD$ quanh trục $AC$ bằng
	\choice
	{$\dfrac{4271\pi}{80}$}
	{\True $\dfrac{4269\pi}{40}$}
	{$\dfrac{4271\pi}{40}$}
	{$\dfrac{4269\pi}{80}$}
	\loigiai{
		{\color{red} HINH O DAY}\\
		Gọi $J$ là hình chiếu vuông góc của $B$ lên cạnh $AC$ và $B’,D’$ lần lượt là điểm đối xứng của $B,D$ qua $AC$. Gọi $E=B’C\cap AD$; $F=BC\cap AD’$ và $EF\cap AC=H$.\\
		Ta có $AC=\sqrt{AB^2+AC^2}=10$; $BJ=\dfrac{AB\cdot BC}{AC}=\dfrac{24}{5}$;\\
		$CJ=\sqrt{8^2-\left(\dfrac{24}{5}\right)^2}=\dfrac{32}{5}$; $HF=\dfrac{CH}{CJ}\cdot JB=\dfrac{25}{32}\cdot\dfrac{24}{5}=\dfrac{15}{4}$.\\
		Thể tích khối tròn xoay cần tìm: $V=2\cdot\dfrac{1}{3}\pi JB^2\cdot AC-\dfrac{1}{3}\pi HF^2\cdot AC=\dfrac{4269\pi}{40}$.
	}
\end{ex}
\begin{ex}
	[Cụm Trường Nghệ An - 2022]%Câu 33.
	Cắt hình nón $(N)$ bởi mặt phẳng đi qua đỉnh $S$ và tạo với trục của $(N)$ một góc bằng $30^{\circ}$, ta được thiết diện là tam giác $SAB$ vuông và có diện tích bằng $4a^2$. Chiều cao của hình nón bằng
	\choice
	{\True $a\sqrt{3}$}
	{$2a\sqrt{3}$}
	{$2a\sqrt{2}$}
	{$a\sqrt{2}$}
	\loigiai{
		{\color{red} HINH O DAY}\\
		Gọi $M$ là trung điểm của $AB$ và $O$ là tâm của đường tròn đáy của hình nón, tam giác $OAB$ cân đỉnh $O$ nên $OM\perp AB$ và $SO\perp AB$ suy ra $AB\perp(SOM)$.\\
		Dựng $OK\perp SM$ tại $M$.\\
		Theo trên ta có: $\heva{&OK\perp AB\\&OK\perp SM}\Rightarrow OK\perp(SAB)$.\\
		Suy ra góc tạo bởi giữa trục $SO$ và mặt phẳng $(SAB)$ là $\widehat{OSM}=30^{\circ}$.\\
		Tam giác vuông cân $SAB$ có diện tích bằng $4a^2$ suy ra $\dfrac{1}{2}SA^2=4a^2\Rightarrow SA=2a\sqrt{2}\Rightarrow AB=4a\Rightarrow SM=2a$.\\
		Xét tam giác vuông $SOM$ có $\cos\widehat{OSM}=\dfrac{SO}{SM}\Rightarrow SO=\dfrac{\sqrt{3}}{2}\cdot 2a=\sqrt{3}a$.\\
		Vậy chiều cao của hình chóp bằng $a\sqrt{3}$.
	}
\end{ex}
\begin{ex}
	[Đại học Hồng Đức – 2022]%Câu 34.
	Cho hình nón đỉnh $S$ có độ dài đường cao là $R$ và đáy là đường tròn tâm $O$ bán kính $R$. Gọi $(d)$ là tiếp tuyến của đường tròn đáy tại $A$ và $(P)$ là mặt phẳng chứa $SA$ và $(d)$. Mặt phẳng $(Q)$ thay đổi qua $S$ cắt đường tròn $O$ tại hai điểm $C,D$ sao cho $CD=\sqrt{3}R$. Gọi $\alpha$ là góc tạo bởi $(P)$ và $(Q)$. Tính giá trị lớn nhất của $\cos\alpha$. 
	\choice
	{\True $\dfrac{3\sqrt{10}}{10}$}
	{$\dfrac{\sqrt{10}}{5}$}
	{$\dfrac{2\sqrt{6}}{5}$}
	{$\dfrac{\sqrt{10}}{10}$}
	\loigiai{
		Gọi $I$ là trung điểm $CD$, khi đó $OI\perp CD$, hạ $OK\perp SI$ tại $K\Rightarrow OK\perp (Q)$.\\
		$HaOH\perp SA\Rightarrow OH\perp (P)\Rightarrow\alpha=(OH,OK)$ 
		{\color{red} HINH O DAY}\\
		$\Rightarrow\cos\alpha=\left|\dfrac{OK^2+OH^2-HK^2}{2OH\cdot OK}\right|$.\\
		Ta có $OI=\sqrt{OD^2-ID^2}=\dfrac{R}{2},OK=\dfrac{OI\cdot OS}{\sqrt{OI^2+SO^2}}=\dfrac{R}{\sqrt{5}};OH=\dfrac{R}{\sqrt{2}} HK^2=SK^2+SH^2-2SK\cdot SH\cos\widehat{ASI} =SK^2+SH^2-2SK\cdot SH\cdot\dfrac{SI^2+SA^2-AI^2}{2SI\cdot SA}$. $SA=\sqrt{2}R,SI=\dfrac{\sqrt{5}}{2}R,SH=\dfrac{SO^2}{SA}=\dfrac{R}{\sqrt{2}},SK=\dfrac{SO^2}{SI}=\dfrac{2}{\sqrt{5}}R$.\\
		Gọi $M$ và $N$ lần lượt là trung điểm $OA$ và $OB$ khi đó\\
		$AM\leq AI\leq AN$.\\
		Suy ra.\\
		$\begin{aligned}&SK^2+SH^2-2SK\cdot SH\cdot\dfrac{SI^2+SA^2-AM^2}{2SI\cdot SA}\leq HK^2\leq SK^2+SH^2-2SK\cdot SH\cdot\dfrac{SI^2+SA^2-AN^2}{2SI\cdot SA}\\&\Rightarrow\dfrac{1}{10}R^2\leq HK^2\leq\dfrac{9}{10}R^2\Rightarrow-\dfrac{\sqrt{10}}{10}\leq\dfrac{OK^2+OH^2-HK^2}{2OH\cdot OK}\leq\dfrac{3\sqrt{10}}{10}\\&\Rightarrow\cos\alpha\leq\max\left\{\dfrac{\sqrt{10}}{10};\dfrac{3\sqrt{10}}{10}\right\}=\dfrac{3\sqrt{10}}{10}.\end{aligned}$.
	}
\end{ex}
\begin{ex}
	[THPT Hương Sơn - Hà Tĩnh - 2022]%Câu 35.
	Một chiếc kem Ốc quế gồm $2$ phần, phần dưới là một khối nón có chiều cao bằng ba lần đường kính đáy, phần trên là nửa khối cầu có đường kính bằng đường kính khối nón bên dưới (như hình vẽ). Thể tích phần kem phía trên bằng $50 \mathrm{cm}^3$. Thể tích của cả chiếc kem bằng
	{\color{red} HINH O DAY}
	\choice
	{\True $200 \mathrm{cm}^3$}
	{$150 \mathrm{cm}^3$}
	{$125 \mathrm{cm}^3$}
	{$500 \mathrm{cm}^3$}
	\loigiai{
		Gọi bán kính của khối cầu là $R (R>0)$. Theo bài ra ta có\\
		$V_1=\dfrac{1}{2}V_C=50\Leftrightarrow\dfrac{4}{3}\pi R^3=100\Leftrightarrow R^3=\dfrac{75}{\pi}$.\\
		Do đó, khối nón phía dưới có bán kính $R$; $h=3\cdot 2R=6R$.\\
		Thể tích của khối nón bằng $V_2=\dfrac{1}{3}\pi R^2\cdot h=\dfrac{1}{3}\pi R^2\cdot 6R=2\pi\cdot R^3=2\pi\cdot\dfrac{75}{\pi}=150\left(\mathrm{cm}^3\right)$.\\
		Vậy thể tích của cả chiếc kem bằng $V=V_1+V_2=50+150=200\left(\mathrm{cm}^3\right)$.
	}
\end{ex}
\begin{ex}
	[Liên trường Hà Tĩnh – 2022]%Câu 36.
	Cho hình nón có chiều cao bằng $2\sqrt{5}$. Một mặt phẳng đi qua đỉnh hình nón và cắt hình nón theo một thiết diện là tam giác đều có diện tích bằng $9\sqrt{3}$. Thể tích của khối nón được giới hạn bởi hình nón đã cho bằng
	\choice
	{\True $\dfrac{32\sqrt{5}\pi}{3}$}
	{$32\pi$}
	{$32\sqrt{5}\pi$}
	{$\dfrac{18\sqrt{5}\pi}{3}$}
	\loigiai{
		Theo giả thiết tam giác $SAB$ đều, $S_{\triangle SAB}=9\sqrt{3}$ và $SO=2\sqrt{5}$. 
		{\color{red} HINH O DAY}\\
		$S_{\triangle SAB}=9\sqrt{3}\Leftrightarrow\dfrac{AB^2\sqrt{3}}{4}=9\sqrt{3}\Leftrightarrow AB=6$.\\
		$\triangle SAB$ đều $SA=AB=6$.\\
		Xét $\quad\triangle SOA\quad$ vuông tại $O$, $OA=\sqrt{SA^2-SO^2}=\sqrt{6^2-(2\sqrt{5})^2}=4$.\\
		Thể tích hình nón bằng $V=\dfrac{1}{3}\pi r^2h=\dfrac{1}{3}\pi\cdot OA^2\cdot SO=\dfrac{1}{3}\pi 4^2\cdot 2\sqrt{5}=\dfrac{32\sqrt{5}}{3}\pi$.}
\end{ex}
\begin{ex}
	[THPT Nho Quan A – Ninh Bình – 2022]%Câu 37.
	Cho hình nón đỉnh $S$ có đường cao $h=a\sqrt{3}$. Một mặt phẳng $(\alpha)$ đi qua đỉnh $S$, cắt đường tròn đáy tại hai điểm $A,B$ sao cho $AB=8a$ và tạo với mặt đáy một góc $30^{\circ}$. Tính diện tích xung quanh của hình nón. 
	\choice
	{$\dfrac{10\sqrt{7}\pi}{3}a^2$}
	{$20\sqrt{7}\pi a^2$}
	{\True $10\sqrt{7}\pi a^2$}
	{$5\sqrt{7}\pi a^2$}
	\loigiai{
		{\color{red} HINH O DAY}\\
		Gọi $O$ là tâm đường tròn đáy, $I$ là trung điểm $AB$. Khi đó, góc giữa mặt phẳng $(\alpha)$ và mặt đáy là $\widehat{SIO}=30^{\circ}$.\\
		Trong tam giác SOI, ta có $OI=\dfrac{SO}{\tan\widehat{SIO}}=3a$.\\
		Trong tam giác $AIO$, ta có $OA^2=OI^2+AI^2=9a^2+16a^2=5a$ \\
		$ \Rightarrow SA=\sqrt{SO^2+AO^2}=\sqrt{3a^2+25a^2}=2\sqrt{7}a $.\\
		Vậy $S_{\mathrm{xq}}=\pi\cdot OASA=10\sqrt{7}\pi a^2$.
	}
\end{ex}
\begin{ex}
	[THPT Phù Cừ - Hưng Yên - 2022]%Câu 38.
	Một tấm tôn hình tam giác $ABC$ có độ dài cạnh $AB=3;AC=2;BC=\sqrt{19}$. Điểm $H$ là chân đường cao kẻ từ đình $A$ của tam giác $ABC$. Người ta dùng compa có tâm là $A$, bán kính $AH$ vạch một cung tròn $MN$. Lấy phần hình quạt gỏ thành hình nón không có mặt đáy với đỉnh là $A$, cung $MN$ thành đường tròn đáy của hình nón (nhuc hình vẽ). Tính thể tich khối nón trên. 
	\choice
	{\True $\dfrac{2\pi\sqrt{114}}{361}$}
	{$\dfrac{2\pi\sqrt{3}}{19}$}
	{$\dfrac{\pi\sqrt{57}}{361}$}
	{$\dfrac{2\pi\sqrt{19}}{361}$}
	\loigiai{
		{\color{red} HINH O DAY}\\
		Theo định lý côsin trong tam giác $ABC$ ta có $BC^2=AB^2+AC^2-2\cdot AB\cdot AC\cdot\cos\widehat{BAC}\Rightarrow\cos\widehat{BAC}=\dfrac{AB^2+AC^2-BC^2}{2\cdot AB\cdot AC}=-\dfrac{1}{2}\Rightarrow\widehat{BAC}=120^{\circ}$ hay $\widehat{BAC}=\dfrac{2\pi}{3}$.\\
		Suy ra diện tích tam giác $ABC$ là $S_{ABC}=\dfrac{1}{2}AB\cdot AC\cdot\sin\widehat{BAC}=\dfrac{3\sqrt{3}}{2}$.\\
		Mà $S_{ABC}=\dfrac{1}{2}AH\cdot BC\Rightarrow AH=\dfrac{2S_{ABC}}{BC}=\dfrac{3\sqrt{57}}{19}$.\\
		Gọi $r$ là bán kinh đáy của hình nón. Suy ra $2\pi r=\dfrac{2\pi}{3}AH\Rightarrow r=\dfrac{AH}{3}=\dfrac{\sqrt{57}}{19}$.\\
		Chiều cao của khối nón bằng $h=\sqrt{AH^2-r^2}=\dfrac{2\sqrt{114}}{19}$.\\
		Thể tích bằng $V=\dfrac{1}{3}\pi r^2h=\dfrac{1}{3}\pi\cdot\left(\dfrac{\sqrt{57}}{19}\right)^2\cdot\dfrac{2\sqrt{114}}{19}=\dfrac{2\pi\sqrt{114}}{361}$.
	}
\end{ex}
\begin{ex}
	[Sở Hà Tĩnh 2022]%Câu 39.
	Cắt hình nón $(N)$ bởi mặt phẳng đi qua đỉnh $(S)$ và tạo bởi với trục của $(N)$ một góc bằng $30^{\circ}$ ta được thiết diện là tam giác $SAB$ vuông và có diện tích $4a^2$. Chiều cao của hình nón bằng 
	\choice
	{\True $a\sqrt{3}$}
	{$2a\sqrt{3}$}
	{$2a\sqrt{2}$}
	{$a\sqrt{2}$}
	\loigiai{
		{\color{red} HINH O DAY}\\
		Gọi đường sinh $SB=x(x>0)$. Vì tam giác $SAB$ là tam giác vuông nên $AB=x\sqrt{2}$ \\
		$ \Rightarrow SI=\dfrac{x\sqrt{2}}{2} $. Theo đề bài tam giác $SAB$ vuông và có diện tích $4a^2$ nên:\\
		$S_{\triangle SAB}=\dfrac{1}{2}SI\cdot AB=\dfrac{1}{2}\cdot\dfrac{x\sqrt{2}}{2}\cdot x\sqrt{2}=4a^2\Leftrightarrow x=2\sqrt{2}a\Rightarrow SI=2a$.\\
		Ta có $\widehat{IOS}=30^{\circ}\Rightarrow cos\widehat{IOS}=\dfrac{SO}{SI}\Rightarrow SO=SI\cdot cos\widehat{IOS}=2a\dfrac{\sqrt{3}}{2}=a\sqrt{3}$.
	}
\end{ex}
\begin{ex}
	[Sở Phú Thọ 2022]%Câu 40.
	Cho hình nón $(N)$ có chiều cao bằng $2a$. Cắt $(N)$ bởi một mặt phẳng đi qua đỉnh và cách tâm của đáy một khoảng bằng $a$ ta được thiết diện có diện tích bằng $\dfrac{4a^2\sqrt{11}}{3}$. Thể tích của khối nón đã cho bằng
	\choice
	{\True $\dfrac{10\pi a^3}{3}$}
	{$10\pi a^3$}
	{$\dfrac{4\pi a^3\sqrt{5}}{3}$}
	{$\dfrac{4\pi a^3\sqrt{5}}{9}$}
	\loigiai{
		{\color{red} HINH O DAY}\\
		Dựng mặt phẳng qua đỉnh $(SBC)$ của hình nón, gọi $I$ là trung điểm của $BC$.\\
		Theo giả thiết: $SO=2a; S_{\triangle SBC}=\dfrac{4a^2\sqrt{11}}{3}; OH=\mathrm{d}\left(O,(SBC)\right)=a$.\\
		Trong $\triangle SOI$ vuông tại $O$ có: $\dfrac{1}{OH^2}=\dfrac{1}{SO^2}+\dfrac{1}{OI^2}\Rightarrow OI=\dfrac{2a\sqrt{3}}{3}$; $SI=\sqrt{SO^2+OI^2}=\dfrac{4a\sqrt{3}}{3}$.\\
		Ta có $S_{\triangle SBC}=\dfrac{1}{2}SI\cdot BC\Rightarrow BC=\dfrac{2S}{SI}=\dfrac{2a\sqrt{33}}{3}\Rightarrow IC=\dfrac{BC}{2}=\dfrac{a\sqrt{33}}{3}$.\\
		Trong $\triangle OIC$ vuông tại $I$ có $OC=\sqrt{OI^2+IC^2}=a\sqrt{5}$.\\
		Vậy thể tích của khối nón đã cho là $V=\dfrac{1}{3}\pi SO\cdot OC^2=\dfrac{10\pi a^3}{3}$.
	}
\end{ex}  
\begin{ex}
	[Sở Vĩnh Phúc 2022]%Câu 41.
	Cho một hình nón đỉnh $S$ có đáy là đường tròn tâm $O$, bán kính $R=\sqrt{5}$ và góc ở đỉnh là $2\alpha$ với $\sin\alpha=\dfrac{2}{3}$. Một mặt phẳng $(P)$ vuông góc với $SO$ tại $H$ và cắt hình nón theo một đường tròn tâm $H$. Gọi $V$ là thể tích của khối nón đỉnh $O$ và đáy là đường tròn tâm $H$. Biết $V=\dfrac{50\pi}{81}$ khi $SH=\dfrac{a}{b}$ với $a,b\in\mathbb{N}^*$ và $\dfrac{a}{b}$ là phân số tối giản. Tính giá trị của biểu thức $T=3a^2-2b^3$. 
	\choice
	{$12$}
	{$23$}
	{\True $21$}
	{$32$}
	\loigiai{
		{\color{red} HINH O DAY}\\
		Trong $\triangle SOB$ vuông tại $O$ ta có $\widehat{OSB}=\alpha$ với $\sin\alpha=\dfrac{2}{3}\Rightarrow\cos\alpha=\sqrt{1-\sin^2\alpha}=\sqrt{1-\dfrac{4}{9}}=\dfrac{\sqrt{5}}{3}$.\\
		Suy ra $\tan\alpha=\dfrac{2}{\sqrt{5}}\Leftrightarrow\dfrac{HN}{SH}=\dfrac{2}{\sqrt{5}}\Leftrightarrow HN=\dfrac{2\cdot SH}{\sqrt{5}}$.\\
		Mặt khác $\sin\alpha=\dfrac{HN}{SN}=\dfrac{OB}{SB}\Rightarrow SB=\dfrac{OB}{\sin\alpha}=\dfrac{3\sqrt{5}}{2}$;\\
		Trong $\triangle SOB$ vuông tại $O$ ta có $SO=\sqrt{SB^2-OB^2}=\sqrt{\dfrac{45}{4}-5}=\dfrac{5}{2}$.\\
		Theo bài ta có thể tích khối nón đỉnh $O$ và đáy là đường tròn tâm $H$ là\\
		$V=\dfrac{50\pi}{81}\Leftrightarrow\dfrac{1}{3}\cdot OH\cdot\pi\cdot HN^2=\dfrac{50\pi}{81}\Leftrightarrow OH\cdot HN^2=\dfrac{50}{27}\Leftrightarrow(SO-SH)\cdot\dfrac{4SH^2}{5}=\dfrac{50}{27}$ \\
		$ \Leftrightarrow(5-2\cdot SH)SH^2=\dfrac{125}{27}\Leftrightarrow 54SH^3-135SH^2+125=0\Leftrightarrow\hoac{&SH=\dfrac{5}{3}\\&SH=-\dfrac{5}{6}(loai).} $ \\
		Suy ra $a=5; b=3$. Vậy $T=3a^2-2b^3=3\cdot 25-2\cdot 27=21$.
	}
\end{ex}
\begin{ex}
	[Sở Vĩnh Phúc 2022]%Câu 42.
	Cho một hình nón có bán kính đáy bằng $a$. Mặt phẳng $(P)$ đi qua đỉnh $S$ của hình nón cắt đường tròn đáy tại $A$ và $B$ sao cho $AB=a\sqrt{3}$, khoảng cách từ tâm đường tròn đáy đến mặt phẳng $(P)$ bằng $\dfrac{a\sqrt{2}}{4}$. Thể tích khối nón đã cho bằng
	\choice
	{$\dfrac{\pi a^3}{12}$}
	{\True $\dfrac{\pi a^3}{6}$}
	{$\dfrac{\pi a^3}{3}$}
	{$\dfrac{\pi a^3}{24}$}
	\loigiai{
		{\color{red} HINH O DAY}\\
		Gọi $I$ là trung điểm đoạn $AB$ dễ thấy $OI$ là đường trung tuyến của tam giác $OAB$.\\
		Dựng $OK\perp SI$ tại $K$.\\
		Ta có $\heva{&AB\perp OI\\&AB\perp SO}\Rightarrow AB\perp(SOI)\Rightarrow OK\perp AB\Rightarrow OK\perp(SAB)$.\\
		Ta có $BI=\dfrac{a\sqrt{3}}{2}\Rightarrow OI=\sqrt{a^2-\dfrac{3a^2}{4}}=\dfrac{a}{2}$.\\
		Khi đó $\mathrm{d}\left(O;(P)\right)=OK\Rightarrow\dfrac{1}{SO^2}=\dfrac{1}{OK^2}-\dfrac{1}{OI^2}=\dfrac{16}{2a^2}-\dfrac{4}{a^2}=\dfrac{4}{a^2}\Rightarrow SO=\dfrac{a}{2}$.\\
		Thể tích khối nón: $V=\dfrac{1}{3}\pi r^2h=\dfrac{1}{3}\pi\cdot a^2\cdot\dfrac{a}{2}=\dfrac{\pi a^3}{6}$.
	}
\end{ex}
\begin{ex}
	[Chuyên Lam Sơn 2022]%Câu 43.
	Một cái bình thủy tinh có phần không gian bên trong là một hình nón có đỉnh hướng xuống dưới theo chiều thẳng đứng. Rót nước vào bình cho đến khi phần không gian trống trong bình có chiều cao $2\mathrm{cm}$. Sau đó đậy kín miệng bình bởi một cái nắp phẳng và lật ngược bình để đỉnh hướng lên trên theo chiều thẳng đứng, khi đó mực nước cao cách đỉnh của nón $8\mathrm{cm}$ (hình vẽ minh họa bên dướí). \\
	{\color{red} HINH O DAY}\\
	Biết chiều cao của nón là $h=a+\sqrt{b}\mathrm{cm}$. Tính $T=a+b$. 
	\choice
	{$22$}
	{$58$}
	{\True $86$}
	{$72$}
	\loigiai{
		Để ý rằng có 3 hình nón đồng dạng: Phần không gian bên trong bình thủy tinh (có thể tích $V$), phần không chứa nước khi đặt bình có đỉnh hướng lên (có thể tích $V_1$), phần chứa nước khi đặt bình có đỉnh hướng xuống (có thể tích $V_2$). Do tỷ số đồng dạng bằng với tỷ số của chiều cao và tỷ số thể tích là lập phương tỷ số đồng dạng nên ta có $\dfrac{V}{V_1}=\dfrac{h^3}{8^3};\dfrac{V}{V_2}=\dfrac{h^3}{(h-2)^3}\Rightarrow V_1=\dfrac{512V}{h^3};V_2=\dfrac{(h-2)^3V}{h^3}$.\\
		Mà $V_1+V_2=V$ nên ta có: $\dfrac{512V}{h^3}+\dfrac{(h-2)^3V}{h^3}=V\Rightarrow 512+h^3-6h^2+12h-8=h^3\Leftrightarrow h^2-2h-84=0\Rightarrow h=1+\sqrt{85}$.\\
		Vậy $T=86$.
	}
\end{ex}
\begin{ex}
	[Chuyên Nguyễn Trãi – Hải Dương – 2022]%Câu 44.
	Nhân dịp năm mới để trang trí một cây thông Noel, ở sân trung tâm có hình nón $(N)$ như hình vẽ sau. Người ta cuộn quanh cây bằng một sợi dây đèn LED nhấp nháy, bóng đèn hình hoa tuyết từ điểm $A$ đến điểm $M$ sao cho sợi dây luôn tựa trên mặt nón. Biết rằng bán kính đáy hình nón bằng $8\mathrm{m}$, độ dài đường sinh bằng $24\mathrm{m}$ và $M$ là điểm sao cho $2\overrightarrow{MS}+\overrightarrow{MA}=\overrightarrow{0}$. Hãy tính chiều dài nhỏ nhất của sợi dây đèn cần có. \\
	{\color{red} HINH O DAY}
	\choice
	{$8\sqrt{19}(\mathrm{m})$}
	{\True $8\sqrt{13}(\mathrm{m})$}
	{$8\sqrt{7}(\mathrm{m})$}
	{$9\sqrt{12}(\mathrm{m})$}
	\loigiai{
		Ta có $2\overrightarrow{MS}+\overrightarrow{MA}=\overrightarrow{0}\Leftrightarrow\overrightarrow{SM}=\dfrac{1}{3}\overrightarrow{SA}\Rightarrow SM=\dfrac{1}{3}SA=8(m)$.\\
		Trải hình nón ra như hình bên dưới\\
		{\color{red} HINH O DAY}\\
		Khi đó chu vi đáy của hình nón cũng là độ dài cung $AA’$ suy ra $2\pi R=16\pi (m)=l_{AA’}$.\\
		Góc $\alpha=ASA’=\dfrac{l_{AA’}}{SA}=\dfrac{16\pi}{24}=\dfrac{2\pi}{3}$.\\
		Chiều dài nhỏ nhất của sợi dây đèn cần có là đoạn thẳng.\\
		$AM=\sqrt{SA^2+SM^2-2SA\cdot SM\cdot\cos\alpha}=\sqrt{24^2+8^2-2\cdot 24\cdot 8\cdot\cos\dfrac{2\pi}{3}}=8\sqrt{13}(\mathrm{m})$.
	}
\end{ex}
\begin{ex}
	[THPT Yên Lạc - Vĩnh Phúc - 2022]%Câu 45.
	Một khối nón có bán kính đáy bằng $2 \mathrm{cm}$, chiều cao bằng $\sqrt{3} \mathrm{cm}$. Một mặt phẳng đi qua đỉnh và tạo với đáy một góc $60^{\circ}$ chia khối nón làm $2$ phần. Tính thể tích phần nhỏ hơn (kết quả làm tròn đến hàng phần trăm). 
	\choice
	{$2,47 \mathrm{cm}^3$}
	{$2,36 \mathrm{cm}^3$}
	{\True $1,42 \mathrm{cm}^3$}
	{$1,53 \mathrm{cm}^3$}
	\loigiai{
		{\color{red} HINH O DAY}\\
		Mặt phẳng $(SMN)$ qua đỉnh và tạo với đáy một góc $60^{\circ}$.\\
		Ta có $\heva{&SI\perp MN\\&OI\perp MN}\Rightarrow\widehat{SIO}=60^{\circ}$ là góc giữa $(SMN)$ và đáy.\\
		$OI=\dfrac{SO}{\tan 60^{\circ}}=\dfrac{\sqrt{3}}{\sqrt{3}}=1\Rightarrow OI=\dfrac{1}{2}OA\Rightarrow OMAN$ là hình bình hành, có 2 đường chéo $OA\perp MN$.\\
		Nên tứ giác $OMAN$ là hình thoi, có $NA=NO=OA=2\Rightarrow\widehat{ANO}=60^{\circ}\Rightarrow\widehat{MON}=120^{\circ}$.\\
		$S_{\triangle OMN}=\dfrac{1}{2}OM\cdot ON\cdot\sin 120^{\circ}=\dfrac{1}{2}2\cdot 2\cdot\dfrac{\sqrt{3}}{2}=\sqrt{3}$.\\
		Nên.\\
		Thể tích khối nhỏ cần tính: $V=\dfrac{1}{3}\cdot SO\cdot S_{\wideparen{MAN}}=\dfrac{1}{3}\cdot\sqrt{3}\cdot\left(\dfrac{4\pi}{3}-\sqrt{3}\right)\simeq 1,42 \mathrm{cm}^3$.
	}
\end{ex}
\begin{ex}
	[THPT Yên Phong 1 - Bắc Ninh - 2022]%Câu 46.
	Cho hình nón đỉnh $S$ tâm $O$ có độ dài đường sinh bằng $SA=a$, đường kính đáy $AB$. Thiết diện qua đỉnh tạo với đáy một góc $60^{\circ}$ cắt đường tròn đáy theo dây cung $MN=\dfrac{2a\sqrt{3}}{3}$. Biết rằng khoảng cách từ $A$ đến $MN$ bằng $a$. Thể tích khối nón bằng 
	\choice
	{\True $\dfrac{a^3\sqrt{2}\pi}{12}$}
	{$\dfrac{a^3\sqrt{6}\pi}{18}$}
	{$\dfrac{a^3\sqrt{6}\pi}{9}$}
	{$\dfrac{a^3\sqrt{6}\pi}{3}$}
	\loigiai{
		Biết rằng khoảng cách từ $A$ đến $MN$ bằng $a\Rightarrow MN$ không vuông góc với $AB$.\\
		Ta có hình vẽ sau: 
		{\color{red} HINH O DAY}\\
		Gọi $I$ là trung điểm $MN$ thì $SI\perp MN, OI\perp MN\Rightarrow\widehat{SIO}=60^{\circ}$ là góc giữa $(SMN)$ và mặt phẳng đáy.\\
		$MN=\dfrac{2a\sqrt{3}}{3}\Rightarrow IM=IN=\dfrac{a\sqrt{3}}{3}$.\\
		Xét $\triangle OIN$ vuông ở $I$ có: $OI^2=ON^2-IN^2=R^2-\left(\dfrac{a\sqrt{3}}{3}\right)^2=R^2-\dfrac{a}3^2\Rightarrow OI=\sqrt{R^2-\dfrac{a}{3}^2}$.\\
		Xét $\triangle SIM$ vuông ở $I$ có: $SI^2=SM^2-MI^2=a^2-\left(\dfrac{a\sqrt{3}}{3}\right)^2=\dfrac{2a}3^2\Rightarrow SI=\dfrac{a\sqrt{2}}{\sqrt{3}}$.\\
		Xét $\triangle SIO$ vuông ở $O$ có: $\cos 60^{\circ}=\dfrac{OI}{SI}=\dfrac{\sqrt{R^2-\dfrac{a^2}{3}}}{\dfrac{a\sqrt{2}}{\sqrt{3}}}=\dfrac{\sqrt{3R^2-a^2}}{a\sqrt{2}}=\dfrac{1}{2}$ \\
		$ \Leftrightarrow 4\left(3R^2-a^2\right)=2a^2\Leftrightarrow R^2=\dfrac{a^2}{2}\Leftrightarrow R=\dfrac{a\sqrt{2}}{2} $.\\
		$sin60^{\circ}=\dfrac{SO}{SI}\Rightarrow SO=SI\cdot\sin 60^{\circ}=\dfrac{\sqrt{2}a}{\sqrt{3}}\cdot\dfrac{\sqrt{3}}{2}=\dfrac{a\sqrt{2}}{2}$.\\
		$V=\dfrac{1}{3}\cdot SO\cdot\left(\pi R^2\right)=\dfrac{1}{3}\cdot\dfrac{\sqrt{2}a}{2}\cdot\pi\dfrac{a^2}{2}=\dfrac{\pi\sqrt{2}a^3}{12}$.
	}
\end{ex}
\begin{ex}
	[Chuyên Sơn La 2022]%Câu 47.
	Bà Hương nhận làm 100 chiếc nón lá giống nhau có độ dài đường sinh là $30$ $\mathrm{cm}$. Ở phần mặt trước của mỗi chiếc nón (từ $A$ đến $B$ như hình vẽ) bà Hương thuê người sơn và vẽ hình trang trí. Biết $AB=20\sqrt{2}cm$ và giá tiền công để sơn trang trí $1\mathrm{m}^2$ là 50000 đồng. Tính số tiền (làm tròn đến hàng nghìn) mà bà Hương phải thuê sơn trang trí cho cả đợt làm nón
	{\color{red} HINH O DAY}
	\choice
	{$128\cdot 000$ đồng}
	{\True $257\cdot 000$ đồng}
	{$384\cdot 000$ đồng}
	{$209\cdot 000$ đồng}
	\loigiai{
		Đầu tiên theo tính chất góc ở tâm bằng hai lần góc nội tiếp chắn cung tương ứng nên ta suy ra: $\widehat{AOB}=2\widehat{AIB}=120^{\circ}$. Sử dụng đính li Cosin ta có $AB^2=OA^2+OB^2-2OA\cdot OB\cos 120^{\circ}\Leftrightarrow 3R^2=(20\sqrt{2})^2$. Từ đó suy ra $R=\dfrac{20\sqrt{6}}{3}(cm)$. Tiếp theo, ta gọi đỉnh của hình nón là $S$, sau đó ta trải phẳng mặt xung quanh của nón ra, khi ấy diện tích mặt cần sơn và trang trí chính là phần hình quạt $SAB$. Ta có độ dài cung $\overset\frown{AB}$ là $l=\dfrac{2\pi}{3}\cdot\dfrac{20\sqrt{6}}{3}=\dfrac{40\pi\sqrt{6}}{9}(cm)$. Từ đó ta tính được diện tích hình quạt $SAB$ là $S=\dfrac{l\cdot SA}{2}=\dfrac{1}{2}\cdot 30\cdot\dfrac{40\pi\sqrt{6}}{9}=\dfrac{1200\pi\sqrt{6}}{18}\left(cm^2\right)=\dfrac{1200\pi\sqrt{6}}{18}\cdot 10^{-4}\left(m^2\right)$. Mà giá tiền công để sơn trang trí $1m^2$ là 50000 đồng nên giá tiền công sơn 100 cái nón là $\dfrac{1200\pi\sqrt{6}}{18}\cdot 10^{-4}\cdot 50\cdot 000\cdot 100=256\cdot 509$ (đồng). Như vậy tổng tiền này gần với đáp án $B$ nhất.
	}
\end{ex}
\begin{ex}
	[Sở Hải Dương 2022]%Câu 48.
	Một cốc thủy tinh hình nón có chiều cao $20\mathrm{cm}$. Người ta đổ vào cốc thủy tinh một lượng nước, sao cho chiều cao của lượng nước trong cốc bằng $\dfrac{3}{4}$ chiều cao cốc thủy tinh, sau đó người ta bịt kín miệng cốc, rồi lật úp cốc xuống như hình vẽ thì chiều cao của nước lúc này là bao nhiêu (làm tròn đến chữ số thập phân thứ 2)?
	{\color{red} HINH O DAY}
	\choice
	{\True $3,34 \mathrm{cm}$}
	{$2,21 \mathrm{cm}$}
	{$5,09 \mathrm{cm}$}
	{$4,27 \mathrm{cm}$}
	\loigiai{
		Gọi $R$ là bán kính đáy của cái phểu ta có $\dfrac{3R}{4}$ là bán kính của đáy chứa cột nước.\\
		Ta có thể tích phần nón không chứa nước là $V=\dfrac{1}{3}\pi(R)^2\cdot 20-\dfrac{1}{3}\pi\left(\dfrac{3R}{4}\right)^2\cdot\dfrac{3}{4}\cdot 20=\dfrac{185}{48}\pi R^2$.\\
		Khi lật ngược phểu Gọi $h$ chiều cao của cột nước trong phểu. phần thể tích phần nón không chứa nước là $V=\dfrac{1}{3}\pi(20-h)\left(\dfrac{R(20-h)}{20}\right)^2=\dfrac{1}{1200}\pi(20-h)^3R^2$.\\
		Mà: $\dfrac{1}{1200}\pi(20-h)^3R^2=\dfrac{185}{48}\pi R^2\Rightarrow(20-h)^3=4625\Rightarrow h\approx 3,34$.
	}
\end{ex}
\begin{ex}
	[Chuyên Hà Tĩnh 2022]%Câu 49.
	Cho khối nón đỉnh $S$ có đường cao bằng $3a$. $SA, SB$ là hai đường sinh của khối nón. Khoảng cách từ tâm đường tròn đáy đến mặt phẳng $(SAB)$ bằng $a$ và diện tích tam giác $SAB$ bằng $3a^2$. Tính thể tích khối nón. 
	\choice
	{$\dfrac{145\pi a^3}{48}$}
	{\True $\dfrac{145\pi a^3}{72}$}
	{$\dfrac{145\pi a^3}{54}$}
	{$\dfrac{145\pi a^3}{36}$}
	\loigiai{
		{\color{red} HINH O DAY}\\
		Gọi $K$ là trung điểm của $AB$ và $H$ là hình chiếu của $O$ lên $SK$.\\
		Ta có $\heva{&OK\perp AB\\&SO\perp AB}$ \\
		$ \Rightarrow AB\perp (SOK) $.\\
		Mặt khác $\heva{&OH\perp SK\\&OH\perp AB (do AB\perp (SOK))}$ \\
		$ \Rightarrow\quad OH\perp (SAB) $ tại $H$ \\
		$ \Rightarrow\mathrm{d}\left(O,(SAB)\right)=OH=a $.\\
		Xét tam giác $SOK$ vuông tại $O$, ta có\\
		$\dfrac{1}{OH^2}=\dfrac{1}{SO^2}+\dfrac{1}{OK^2}\Rightarrow\dfrac{1}{a^2}=\dfrac{1}{9a^2}+\dfrac{1}{OK^2}$ \\
		$ \Rightarrow\dfrac{1}{OK^2}=\dfrac{8}{9a^2}\Rightarrow OK=\dfrac{3a\sqrt{2}}{4} $.\\
		$SK=\sqrt{SO^2+OK^2}=\sqrt{9a^2+\dfrac{9a^2}{8}}=\dfrac{9a\sqrt{2}}{4}$.\\
		$S_{\triangle SAB}=\dfrac{1}{2}SK\cdot AB=3a^2$ \\
		$ \Rightarrow AB=\dfrac{6a^2}{SK}=\dfrac{6a^2}{\dfrac{9a\sqrt{2}}{4}}=\dfrac{4a\sqrt{2}}{3} $.\\
		$AK=\dfrac{1}{2}AB=\dfrac{2a\sqrt{2}}{3}$ \\
		$ \Rightarrow OA=\sqrt{OK^2+KA^2} =\sqrt{\dfrac{8a^2}{9}+\dfrac{9a^2}{8}} =\dfrac{a\sqrt{290}}{12} $ \\
		$ \Rightarrow V=\dfrac{1}{3}\pi R^2h =\dfrac{1}{3}\pi\cdot\left(\dfrac{\sqrt{290}a}{12}\right)^2\cdot 3a =\dfrac{145}{72}\pi a^3 $.
	}
\end{ex}
\begin{ex}
	[Chuyên Ngoại Ngữ - Hà Nội 2022]%Câu 50.
	Cho khối nón có bán kính đáy bằng $\sqrt{3}a$. Gọi $M,N$ là hai điểm thuộc đường tròn đáy sao cho $MN=2a$. Biết thể tích của khối nón là $\sqrt{2}\pi a^3$, khoảng cách từ tâm của đường tròn đáy đến mặt phẳng $(SMN)$ là
	\choice
	{$\dfrac{a}{\sqrt{2}}$}
	{$2a$}
	{\True $a$}
	{$\sqrt{3}a$}
	\loigiai{
		{\color{red} HINH O DAY}\\
		Gọi $r, h$ lần lượt là bán kính đường tròn đáy và đường cao của khối nón.\\
		Theo giả thiết ta có $V=\dfrac{1}{3}\pi r^2h\Rightarrow\pi a^2h=\sqrt{2}\pi a^3\Rightarrow SO=h=\sqrt{2}a$.\\
		Gọi $I$ là trung điểm của $MN$. $O$ là tâm của đường tròn đáy.\\
		$\triangle OMN$ cân tại $O$, $I$ là trung điểm của $MN$ nên $OI\perp MN\Rightarrow OI=\sqrt{OM^2-IM^2}=a\sqrt{2}$.\\
		Khi đó, ta có $IO\perp MN, SO\perp MN\Rightarrow MN\perp(SIO)$.\\
		Kẻ $OH\perp SI$ tại $H$, có $MN\perp(SIO)\Rightarrow MN\perp OH$ mà $OH\perp SI\Rightarrow OH\perp(SMN)$ tại $H$ \\
		$\Rightarrow\mathrm{d}\left(O,(SMN)\right)=OH=\dfrac{SO\cdot OI}{\sqrt{SO^2+OI^2}}=a $.
	}
\end{ex}
\begin{ex}
	[Chuyên Quốc Học Huế 2022]%Câu 51.
	Cho hình nón đỉnh $S$ có góc ở đỉnh bằng $60^{\circ}$ và có độ dài đường sinh $l=12$ cm. Gọi $AB$ là một đường kính cố định của đáy hình nón, $MN$ là một dây cung thay đổi của đường tròn đáy và luôn vuông góc với $AB$. Biết rằng tâm đường tròn ngoại tiếp của tam giác $SMN$ luôn thuộc một đường tròn $(C)$ cố định. Tính bán kính của đường tròn $(C)$. 
	\choice
	{$6\sqrt{2}$ $\mathrm{cm}$}
	{\True $2\sqrt{3}$ $\mathrm{cm}$}
	{$\dfrac{\sqrt{3}}{2}$ $\mathrm{cm}$}
	{$\dfrac{3\sqrt{2}}{2}$ $\mathrm{cm}$}
	\loigiai{
		{\color{red} HINH O DAY}\\
		Gọi $O$ là tâm đường tròn đáy của hình nón.\\
		Góc ở đỉnh của hình nón bằng $60^{\circ}$ nên $\widehat{ASB}=60^{\circ}$.\\
		Suy ra, tam giác $SAB$ đều có cạnh bằng $l=12$ cm.\\
		Gọi $K$, $I$ lần lượt là tâm đường tròn ngoại tiếp các tam giác $SAB$ và $SMN$.\\
		Khi đó, $K$ là trọng tâm của tam giác $SAB\Rightarrow KS=\dfrac{2}{3}SO =\dfrac{2}{3}\cdot\dfrac{12\sqrt{3}}{2} =4\sqrt{3}$ $\mathrm{cm}$.\\
		Mặt khác, $\triangle KOA=\triangle KOM=\triangle KON\Rightarrow KA=KM=KN$. Mà $KA=KS$ nên.\\
		$KM=KN=KS\Rightarrow KI$ là trục của đường tròn ngoại tiếp tam giác $SMN$ \\
		$ \Rightarrow KI\perp(SMN)\Rightarrow KI\perp IS\Rightarrow I $ thuộc mặt cầu $(S)$ đường kính $KS$ cố định. (1).\\
		Gọi $H$ là giao điểm của $MN$ và $AB$. Dễ thấy, $I\in SH$ nên $I\in(SAB)$ cố định. (2).\\
		Từ (1) và (2), suy ra $I$ thuộc đường tròn $(C)$ là giao tuyến của $(S)$ và mặt phẳng $(SAB)$.\\
		Bán kính của đường tròn $(C)$ là $R=\dfrac{1}{2}KS =2\sqrt{3}$ $\mathrm{cm}$.
	}
\end{ex}
\begin{ex}
	[THPT Hoàng Hoa Thám - Đà Nẵng 2022]%Câu 52.
	Cho hình nón có thiết diện qua đỉnh $S$ là một tam giác đều tạo với đường cao một góc $30^{\circ}$. Khối nón có thể tích bằng $7\pi$. Diện tích xung quanh của khối nón là
	\choice
	{\True $S=4\sqrt{7}\pi$}
	{$S=2\sqrt{7}\pi$}
	{$S=14\pi$}
	{$S=4\sqrt{13}\pi$}
	\loigiai{
		{\color{red} HINH O DAY}\\
		Giả sử thiết diện qua đỉnh là $\triangle SAB$. Suy ra $\left(\widehat{SO,(SAB)}\right)=\widehat{OSM}=30^{\circ}$.\\
		Đặt $SA=x,(x>0)$. Mà $\triangle SAB$ đều $\Rightarrow SM=\dfrac{x\sqrt{3}}{2}$.\\
		Xét $\triangle SOA$ vuông tại $O$ có $SO=\sqrt{SA^2-OA^2} =\sqrt{x^2-R^2}\Rightarrow h=\sqrt{x^2-R^2}$ (với $SO=h>0$).\\
		Xét $\triangle SOM$ vuông tại $O$ có $\cos\widehat{SMO}=\dfrac{SO}{SM}\Leftrightarrow\cos 30^{\circ}=\dfrac{\sqrt{x^2-R^2}}{\dfrac{x\sqrt{3}}{2}}\Leftrightarrow 3x=4\sqrt{x^2-R^2}$ \\
		$ \Leftrightarrow 9x^2=16\left(x^2-R^2\right)\Leftrightarrow 7x^2=16R^2\Leftrightarrow x^2=\dfrac{16R^2}{7}\Leftrightarrow x=\dfrac{4\sqrt{7}R}{7}\Rightarrow h=\dfrac{3\sqrt{7}R}{7} $.\\
		Có $V_N=7\pi\Leftrightarrow\dfrac{1}{3}\pi R^2h=7\pi\Leftrightarrow R^2h=21\Leftrightarrow R^2\cdot\dfrac{3\sqrt{7}R}{7}=21\Leftrightarrow R=\sqrt{7}\Rightarrow SA=x=4$.\\
		Vậy diện tích xung quanh của hình nón là $S_{\text{xq}}=\pi Rl =\pi\cdot\sqrt{7}\cdot 4=4\sqrt{7}\pi$.
	}
\end{ex}
\begin{ex}
	[Liên trường Quảng Nam 2022]%Câu 53.
	Cho hình nón có chiều cao $6a$. Một mặt phẳng $(P)$ đi qua đỉnh của hình nón cắt hình nón theo thiết diện là một tam giác vuông cân và khoảng cách từ tâm đường tròn đáy đến mặt phẳng $(P)$ là $3a$. Thể tích của khối nón được giới hạn bởi hình nón đã cho bằng
	\choice
	{$96\pi a^3$}
	{$108\pi a^3$}
	{\True $120\pi a^3$}
	{$150\pi a^3$}
	\loigiai{
		{\color{red} HINH O DAY}\\.\\
		Giả sử mặt phẳng $(P)$ đi qua đỉnh của hình nón cắt hình nón theo thiết diện là một tam giác vuông cân $SAB$.\\
		Gọi $I$ là trung điểm của $AB$, $O$ là tâm của đường tròn đáy của hình nón.\\
		Kẻ $OK\perp SI$ tại $K$.\\
		Ta có $\mathrm{d}\left(O,(SAB)\right)=OK=3a$; $SO=6a$, suy ra $IO=\sqrt{\dfrac{OK^2\cdot SO^2}{SO^2-OK^2}}=2\sqrt{3}a$.\\
		$SI=\sqrt{SO^2+OI^2}=4\sqrt{3}a$.\\
		$IA=\dfrac{AB}{2}=SI=4\sqrt{3}a$ (Do tam giác $SAB$ vuông tại $S$).\\
		$R=\sqrt{IA^2+IO^2}=2\sqrt{15}a$.\\
		Thể tích của khối nón cần tìm là\\
		$V=\dfrac{1}{3}\pi R^2h=\dfrac{1}{3}\cdot\pi\cdot\left(2\sqrt{15}a\right)^2\cdot 6a=120\pi a^3$.
	}
\end{ex}
\begin{ex}
	[Sở Hà Nam 2022]%Câu 54.
	Cho hình nón đỉnh $S$, đường tròn đáy tâm $O$ và góc ở đỉnh bằng $120^{\circ}$. Một mặt phẳng đi qua $S$ cắt hình nón theo thiết diện là tam giác vuông $SAB$. Biết khoảng cách giữa hai đường thẳng $AB$ và $SO$ bằng $3$, diện tích xung quanh của hình nón đã cho bằng
	\choice
	{$2\pi\sqrt{3}$}
	{$27\pi\sqrt{3}$}
	{$9\pi\sqrt{3}$}
	{\True $18\pi\sqrt{3}$}
	\loigiai{
		{\color{red} HINH O DAY}\\
		Gọi $I$ là trung điểm của $AB$ khi đó $OI\perp AB$.\\
		Mà $SO$ vuông góc với đáy $\Rightarrow SO\perp OI$ nên $\mathrm{d}(SO,AB)=OI=3$.\\
		Gọi bán kính của đường tròn đáy là $r\Rightarrow OB=r$.\\
		Vì góc ở đỉnh bằng $120^{\circ}\Rightarrow\widehat{OSB}=60^{\circ}\Rightarrow\sin\widehat{OSB}=\dfrac{OB}{SB}\Rightarrow SB=\dfrac{r}{\sin 60^{\circ}}=\dfrac{2r}{\sqrt{3}}$.\\
		Xét $\triangle OIB$ vuông tại $I$: $IB^2=OI^2+OB^2=3^2+r^2\Rightarrow IB=\sqrt{3^2+r^2}\Rightarrow AB=2\sqrt{3^2+r^2}$.\\
		Xét $\triangle SAB$ vuông cận tại $S$:\\
		$AB^2=SA^2+SB^2\Leftrightarrow\left(2\sqrt{3^2+r^2}\right)^2=\left(\dfrac{2r}{\sqrt{3}}\right)^2+\left(\dfrac{2r}{\sqrt{3}}\right)^2\Leftrightarrow r^2=27\Leftrightarrow r=3\sqrt{3}$.\\
		$l=SB=\dfrac{2r}{\sqrt{3}}=6$.\\
		Diện tích xung quanh của hình nón: $S=\pi rl=\pi 3\sqrt{3}\cdot 6=18\pi\sqrt{3}$.}
\end{ex}
\begin{ex}
	[Sở Hưng Yên 2022]%Câu 55.
	Cắt hình nón đỉnh $I$ bởi một mặt phẳng đi qua trục hình nón ta được một tam giác vuông cân có cạnh huyền bằng $a\sqrt{2}$; $BC$ là dây cung của đường tròn đáy sao cho mặt phẳng $(IBC)$ tạo với mặt phẳng chứa đáy hình nón một góc $60^{\circ}$. Tính theo $a$ diện tích $S$ của tam giác $IBC$. 
	\choice
	{$S=\dfrac{\sqrt{2}a^2}{6}$}
	{$S=\dfrac{a^2}{3}$}
	{\True $S=\dfrac{\sqrt{2}a^2}{3}$}
	{$S=\dfrac{2a^2}{3}$}
	\loigiai{
		{\color{red} HINH O DAY}\\
		Gọi $r, h, l$ lần lượt là bán kính đáy, chiều cao và đường sinh của hình nón đã cho.\\
		Vì cắt hình nón đỉnh $I$ bởi một mặt phẳng đi qua trục hình nón ta được một tam giác vuông cân có cạnh huyền bằng $a\sqrt{2}$ nên $\heva{&2r=a\sqrt{2}\Rightarrow r=\dfrac{a\sqrt{2}}{2}\\&l=\dfrac{a\sqrt{2}}{\sqrt{2}}=a}\Rightarrow h=\sqrt{l^2-r^2}=\dfrac{a\sqrt{2}}{2.}$ \\
		Gọi $H$ là tâm của đường tròn đáy và $J$ là trung điểm của $BC$.\\
		Ta có $\heva{&BC\perp IH\\&BC\perp HJ}\Rightarrow BC\perp(IHJ)$.\\
		Suy ra góc giữa mặt phẳng $(IBC)$ với mặt phẳng chứa đáy hình nón là góc $\widehat{IJH}=60^{\circ}$.\\
		Ta có $JI=\dfrac{IH}{\sin 60^{\circ}}=\dfrac{h}{\sin 60^{\circ}}=\dfrac{a\sqrt{6}}{3}\Rightarrow BJ=\sqrt{l^2-JI^2}=\dfrac{a\sqrt{3}}{3}\Rightarrow BC=2BJ=\dfrac{2a\sqrt{3}}{3}$.\\
		Vậy $S=\dfrac{1}{2}\cdot JI\cdot BC=\dfrac{\sqrt{2}a^2}{3}$.
	}
\end{ex}
\begin{ex}
	[THPT Ninh Bình - Bạc Liêu 2022]%Câu 56.
	Hình nón $(N)$ có đỉnh $S$, tâm đường tròn đáy là $O$, góc ở đỉnh bằng $120\circ$. Một mặt phẳng qua $S$ cắt hình nón $(N)$ theo thiết diện là tam giác vuông $SAB$. Biết rằng khoảng cách giữa hai đường thẳng $AB$ và $SO$ bằng 3. Tính diện tích xung quanh $S_{\mathrm{xq}}$ của hình nón $(N)$. 
	\choice
	{$S_{\mathrm{xq}}=36\sqrt{3}\pi$}
	{\True $S_{\mathrm{xq}}=18\sqrt{3}\pi$}
	{$S_{\mathrm{xq}}=27\sqrt{3}\pi$}
	{$S_{\mathrm{xq}}=9\sqrt{3}\pi$}
	\loigiai{
		{\color{red} HINH O DAY}\\
		Gọi bán kính hình nón là $OA=OB=r (r>0)$.\\
		Gọi $I$ là trung điểm của $AB$ thì khoảng cách giữa $AB$ và $SO$ là $OI=3$.\\
		Tam giác $OIA$ vuông tại $I$ nên $AI=\sqrt{OA^2-OI^2}=\sqrt{r^2-9}$. Suy ra $AB=2AI=2\sqrt{r^2-9}$.\\
		Góc ở đỉnh hình nón bằng $120^{\circ}$ nên $\widehat{ASO}=\dfrac{120^{\circ}}{2}=60^{\circ}$.\\
		Tam giác $SAO$ vuông tại $O$ nên $\sin\widehat{ASO}=\dfrac{OA}{SA}\Rightarrow SA=\dfrac{OA}{\sin\widehat{ASO}}=\dfrac{2r}{\sqrt{3}}$.\\
		Hai đường sinh có độ dài bằng nhau nên $SA=SB=\dfrac{2r}{\sqrt{3}}$.\\
		Tam giác $SAB$ vuông nên $SA^2+SB^2=AB^2\Rightarrow\dfrac{4r^2}{3}+\dfrac{4r^2}{3}=4\left(r^2-9\right)\Rightarrow r=3\sqrt{3}$.\\
		Suy ra độ dài đường sinh $l=SA=\dfrac{2r}{\sqrt{3}}=6$.\\
		Diện tích xung quanh hình nón là $S_{\mathrm{xq}}=\pi rl=\pi\cdot 3\sqrt{3}\cdot 6=18\pi\sqrt{3}$.
	}
\end{ex}
\begin{ex}
	[THPT Trần Quốc Tuấn - Quảng Ngãi - 2022]%Câu 57.
	Cho một hình nón đỉnh $S$ có chiều cao bằng $4a$, bán kính đáy bằng $2a$. Cắt hình nón đã cho bởi một mặt phẳng vuông góc với trục ta được một hình nón $(N)$ đỉnh $S$ có đường sinh bằng $a$. Tính thể tích của khối nón $(N)$. 
	\choice
	{\True $\dfrac{2\sqrt{5}\pi a^3}{75}$}
	{$\dfrac{13\pi a^3}{125}$}
	{$\dfrac{13\pi a^3}{375}$}
	{$\dfrac{\sqrt{5}\pi a^3}{125}$}
	\loigiai{
		{\color{red} HINH O DAY}\\
		Ta có bán kính hình nón ban đầu bằng $OB=2a$, $SO=4a$ do đó độ dài đường sinh.\\
		$SB=\sqrt{SO^2+OB^2}=\sqrt{(4a)^2+(2a)^2}=2a\sqrt{5}$.\\
		Ta có độ dài đường sinh hình nón $(N)$ là $SN=a$.\\
		Các tam giác $SIN$ và tam giác $SOB$ đồng dạng nên ta có $\dfrac{SI}{SO}=\dfrac{IN}{OB}=\dfrac{SN}{SB}=\dfrac{a}{2a\sqrt{5}}=\dfrac{1}{2\sqrt{5}}$.\\
		Suy ra $SI=\dfrac{2a}{\sqrt{5}}$, $IN=\dfrac{a}{\sqrt{5}}$.\\
		Thể tích khối nón $(N)$ bằng $V=\dfrac{1}{3}\pi r^2h=\dfrac{1}{3}\pi\left(\dfrac{a}{\sqrt{5}}\right)^2\dfrac{2a}{\sqrt{5}}=\dfrac{2\pi a^3\sqrt{5}}{75}$.
	}
\end{ex}
\begin{ex}
	[THPT Ngũ Hành Sơn - Đà Nẵng 2022]%Câu 58.
	Cho hình nón đỉnh $S$ có đáy là hình tròn tâm $O$. Dựng hai đường sinh $SA,SB$, biết tam giác $SAB$ vuông và có diện tích là $4a^2$. Góc tạo bởi giữa trục $SO$ và mặt phẳng $(SAB)$ bằng $30^{\circ}$. Đường cao $h$ của hình nón bằng
	\choice
	{$h=\dfrac{a\sqrt{3}}{2}$}
	{\True $h=a\sqrt{3}$}
	{$h=\dfrac{a\sqrt{6}}{4}$}
	{$h=a\sqrt{2}$}
	\loigiai{
		{\color{red} HINH O DAY}\\
		Dựng $OH\perp AB (H\in AB)$ suy ra $H$ là trung điểm của $AB$.\\
		Kẻ $OD\perp SH (D\in SH)$. Suy ra $OD\perp(SAB)$.\\
		Khi đó $\left(\widehat{OS,(SAB)}\right)=\left(\widehat{OS, DS}\right)=\widehat{OSD}$.\\
		Góc tạo bởi giữa trục $SO$ và mặt phẳng $(SAB)$ là $\widehat{DSO}=30^{\circ}$.\\
		Ta có tam giác $SAB$ vuông cân và có diện tích là $4a^2$ nên $SA=SB=2a\sqrt{2}$.\\
		Suy ra $AB=\sqrt{SA^2+SB^2}=4a$.\\
		Ta có $SH=\dfrac{SA\cdot SB}{AB}=2a$ nên $SO=SH\cdot cos 30^{\circ}=2a\cdot\dfrac{\sqrt{3}}{2}=a\sqrt{3}$.
	}
\end{ex}
\begin{ex}
	[THPT Nguyễn Cảnh Quân - Nghệ An 2022]%Câu 59.
	Cắt hình nón $(N)$ bởi mặt phẳng đi qua đỉnh và tạo với mặt phẳng chứa đáy một góc $30^{\circ}$, ta được thiết diện là tam giác đều cạnh $4a$. Diện tích xung quanh của $(N)$ bằng
	\choice
	{$4\sqrt{11}\pi a^2$}
	{$4\sqrt{7}\pi a^2$}
	{\True $4\sqrt{13}\pi a^2$}
	{$8\sqrt{7}\pi a^2$}
	\loigiai{
		{\color{red} HINH O DAY}\\
		Gọi hình nón đã cho có đỉnh $S$ và tâm hình tròn đáy là $O$, thiết diện qua đỉnh $S$ là $\triangle SAB$ đều cạnh $4a$, $M$ là trung điểm của $AB$ suy ra góc giữa $(SAB)$ và đáy là $\widehat{SMO}=30^{\circ}$.\\
		Ta có $l=SA=SB=AB=4a$ và $SM=\dfrac{4a\sqrt{3}}{2}=2a\sqrt{3}$.\\
		Xét $\triangle SOM$ vuông tại $O$ nên $\sin\widehat{SMO}=\dfrac{SO}{SM}\Rightarrow h=SO=SM\cdot\sin 30^{\circ}=\sqrt{3}a$ suy ra $r=\sqrt{l^2-h^2} =\sqrt{13}a$.\\
		Vậy diện tích xung quanh của $(N)$ bằng $\pi rl=\pi\cdot\sqrt{13}a\cdot 4a=4\sqrt{13}\pi a^2$.
	}
\end{ex}
\begin{ex}
	[Sở Hòa Bình 2022]%Câu 60.
	Cho tam giác $ABC$ vuông tại $A$ có $AB=4,AC=2$. Thể tích khối tròn xoay sinh bởi hình tam giác khi quanh quanh cạnh $BC$ bằng
	\choice
	{\True $\dfrac{32\pi\sqrt{5}}{15}$}
	{$\dfrac{\pi\sqrt{5}}{5}$}
	{$\dfrac{2\pi\sqrt{5}}{3}$}
	{$\dfrac{\pi\sqrt{5}}{15}$}
	\loigiai{
		{\color{red} HINH O DAY}\\
		Xét tam giác $ABC$ vuông tại $A$ ta có $BC=\sqrt{4^2+2^2}=2\sqrt{5}$.\\
		Kẻ $AI\perp BC(I\in BC)\Rightarrow\dfrac{1}{AI^2}=\dfrac{1}{AB^2}+\dfrac{1}{AC^2}\Rightarrow AI=\dfrac{4\sqrt{5}}{5}$.\\
		Thể tích khối tròn xoay sinh bởi hình tam giác khi quanh quanh cạnh $BC$ bằng\\
		$V=\dfrac{1}{3}\pi AI^2(BI+CI)=\dfrac{1}{3}\pi AI^2BC=\dfrac{1}{3}\pi\cdot\left(\dfrac{4\sqrt{5}}{5}\right)^2\cdot 2\sqrt{5}=\dfrac{32\pi\sqrt{5}}{15}$.
	}
\end{ex}
\begin{ex}
	[Sở Cà Mau 2022]%Câu 61.
	Cho hình nón có thiết diện qua đỉnh $S$ là tam giác đều có cạnh bằng $16$ và tạo với mặt đáy một góc $60^{\circ}$. Thể tích của khối nón đó bằng
	\choice
	{$16\sqrt{7}\pi$}
	{\True $448\pi$}
	{$1344\pi$}
	{$192\sqrt{7}\pi$}
	\loigiai{
		{\color{red} HINH O DAY}\\.\\
		Gọi thiết diện qua đỉnh $S$ là tam giác đều $SAB$ cạnh bằng $16$, tạo với mặt đáy một góc $\widehat{SIO}=60^{\circ}$.\\
		Ta có đường cao của tam giác $SI=\dfrac{16\sqrt{3}}{2}=8\sqrt{3}$.\\
		$\sin\widehat{SIO}=\dfrac{SO}{SI}\Rightarrow SO=SI\sin 60^{\circ}=8\sqrt{3}\cdot\dfrac{\sqrt{3}}{2}=12$.\\
		$IO=\sqrt{SI^2-SO^2}=\sqrt{(8\sqrt{3})^2-12^2}=4\sqrt{3}$, suy ra $r=OB=\sqrt{IO^2+IB^2}=\sqrt{(4\sqrt{3})^2+8^2}=4\sqrt{7}$.\\
		Thể tích khối nón bằng $V=\dfrac{1}{3}\pi r^2\cdot h=\dfrac{1}{3}\pi(4\sqrt{7})^2\cdot 12=448\pi$.
	}
\end{ex}
\begin{ex}
	[Sở Thái Bình 2022]%Câu 62.
	Cắt hình nón $(N)$ bởi mặt phẳng đi qua đỉnh và tạo với mặt phẳng chứa đáy một góc $30^{\circ}$, ta được thiết diện là tam giác đều cạnh $4a$. Diện tích xung quanh của $(N)$ bằng
	\choice
	{$8\sqrt{7}\pi a^2$}
	{\True $4\sqrt{13}\pi a^2$}
	{$4\sqrt{7}\pi a^2$}
	{$8\sqrt{13}\pi a^2$}
	\loigiai{
		{\color{red} HINH O DAY}\\
		Giả sử thiết diện là tam giác $SAB$; gọi $I$ là trung điểm của $AB$.\\
		Ta có $AB\perp SI$ và $OI\perp AB$.\\
		Suy ra: góc giữa mặt phẳng $(SAB)$ và mặt đáy là góc $\widehat{SIO}$.\\
		Xét $\triangle SOI$ vuông tại $O$ ta có $OI=\cos 30^{\circ}\cdot SI=\dfrac{\sqrt{3}}{2}\cdot 2a\sqrt{3}=3a$.\\
		Xét $\triangle OAI$ vuông tại $I$ ta có $OA=\sqrt{OI^2+IA^2}=a\sqrt{13}$.\\
		Diện tích xung quanh của hình nón: $S_{\text{xq}}=\pi\cdot r\cdot l=\pi\cdot OA\cdot SA=\pi\cdot a\sqrt{13}\cdot 4a=4\sqrt{13}\pi a^2$.
	}
\end{ex}
\begin{ex}
	[Sở Kiên Giang 2022]%Câu 63.
	Cho hình nón $S$ có bán kính đáy bằng $2\sqrt{3}a$. Gọi $A$ và $B$ là hai điểm thuộc đường tròn đáy sao cho góc giữa mặt phẳng $(SAB)$ với mặt phẳng chứa đường tròn đáy bằng $60^{\circ}$. Biết khoảng cách từ tâm đáy đến mặt phẳng $(SAB)$ bằng $\dfrac{3\sqrt{2}}{2}a$, thể tích của khối nón đã cho bằng
	\choice
	{$V=8\sqrt{3}\pi a^3$}
	{$V=36\sqrt{2}\pi a^3$}
	{$V=24\sqrt{3}\pi a^3$}
	{\True $V=12\sqrt{2}\pi a^3$}
	\loigiai{
		{\color{red} HINH O DAY}\\
		Ta có $SO\perp(ABCD)$. Kẻ $OH\perp AB$ suy ra $SH\perp AB$ (định lý ba đường vuông góc).\\
		Ta có: $(SAB)\cap(ABCD)=AB$. Mà $OH\perp AB$, $SH\perp AB$.\\
		Suy ra góc $\widehat{\left[(SAB),(ABCD)\right]}=\widehat{(OH, SH)}=\widehat{SHO}\Rightarrow\widehat{SHO}=60^{\circ}$.\\
		Kẻ $OK\perp SH$, mà $AB\perp SO$, $AB\perp OH\Rightarrow AB\perp(SOH)\Rightarrow AB\perp OK$.\\
		Suy ra $OK\perp(SAB)\Rightarrow OK=\mathrm{d}\left(O,(SAB)\right)$. Do đó $OK=\dfrac{3\sqrt{2}}{2}a$.\\
		Xét tam giác vuông $OKH$ ta có $sin KHO=\dfrac{OK}{OH}\Rightarrow OH=\dfrac{OK}{\sin 60^{\circ}}=\sqrt{6}a$.\\
		Xét tam giác vuông $SOH$ ta có $\tan SHO=\dfrac{SO}{OH}\Rightarrow SO=OH\cdot\tan 60^{\circ}=3\sqrt{2}a$.\\
		Vậy thể tích khối nón là $V=\dfrac{1}{3}\pi R^2\cdot h=\dfrac{1}{3}\pi\cdot (2\sqrt{3}a)^2\cdot 3\sqrt{2}a=12\sqrt{2}\pi a^3$.
	}
\end{ex}
\begin{ex}
	[Sở Nam Định 2022]%Câu 64.
	Cho hình nón đỉnh $S$ có đáy là hình tròn tâm $O$. Gọi $A$, $B$ là hai điểm thuộc đường tròn đáy sao cho tam giác $SAB$ vuông và có diện tích bằng $16$. Góc tạo bởi trục $SO$ và mặt phẳng $(SAB)$ bằng $30^{\circ}$. Thể tích của khối nón đã cho bằng
	\choice
	{\True $\dfrac{40\sqrt{3}}{3}\pi$}
	{$\dfrac{10\sqrt{6}}{3}\pi$}
	{$\dfrac{20\sqrt{3}}{3}\pi$}
	{$\dfrac{40\sqrt{2}}{3}\pi$}
	\loigiai{
		{\color{red} HINH O DAY}\\
		Gọi $M$ là trung điểm $AB\Rightarrow OM\perp AB\Rightarrow SM\perp AB$.\\
		Gọi $H$ là hình chiếu của $O$ lên $SM\Rightarrow SH$ là hình chiếu của $SO$ lên mặt phẳng $(SAB)$ \\
		$ \Rightarrow\left(SO,(SAB)\right)=(SO, SM)=\widehat{OSM}=30^{\circ} $.\\
		Từ giả thiết suy ra tam giác $SAB$ vuông cân tại $S$.\\
		$S_{\triangle SAB}=16\Leftrightarrow\dfrac{1}{2}SA^2=16\Leftrightarrow SA=4\sqrt{2}\Rightarrow AB=8\Rightarrow MS=MA=MB=4$.\\
		Gọi $r$ là bán kính đáy của hình nón.\\
		Xét tam giác vuông $SOM$: $\cos 30^{\circ}=\dfrac{SO}{SM}\Rightarrow SO=2\sqrt{3}$ \\
		$ \Rightarrow r=OA=\sqrt{SA^2-SO^2}=\sqrt{32-12}=\sqrt{20} $.\\
		Suy ra thể tích của khối nón đã cho là $V=\dfrac{1}{3}\pi r^2\cdot SO=\dfrac{1}{3}\pi\cdot 20\cdot 2\sqrt{3}=\dfrac{40\pi\sqrt{3}}{3}$.
	}
\end{ex} 
\Closesolutionfile{ans}
\indapan{10}{ans/CD21/Muc_9_10}