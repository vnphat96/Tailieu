\Opensolutionfile{ans}[ans/CD25/Muc_5_6]
\setcounter{ex}{0}
\setcounter{dang}{0}
\section{Mức độ 5,6 điểm}
\begin{dang}
	{Nguyên hàm cơ bản}
\end{dang}
\begin{ex}
	[Đề Tham Khảo 2020 Lần 2]%Câu 1
	Hàm số $F(x)$ là một nguyên hàm của hàm số $f(x)$ trên khoảng $K$ nếu
	\choice
	{$ F'(x)=-f(x),\forall x\in \mathbb{K}$}     
	{$ f'(x)=F(x),\forall x\in \mathbb{K}$}
	{\True $ F'(x)=f(x),\forall x\in \mathbb{K}$}
	{$ f'(x)=-F(x),\forall x\in \mathbb{K}$}
	\loigiai{
		Theo định nghĩa thì hàm số $F(x)$ là một nguyên hàm của hàm số $f(x)$ trên khoảng $K$ nếu $ F'(x)=f(x),\forall x\in \mathbb{K}$.
	}
\end{ex}
\begin{ex}
	[Mã 101 - 2020 Lần 1]%Câu 2
	$\displaystyle\int{x^2\mathrm{d}x}$ bằng
	\choice
	{$ 2x+C$}
	{\True $\dfrac{1}{3}{x^3}+C$}
	{$x^3+C$}
	{$ 3x^3+C$}
	\loigiai{}
\end{ex}
\begin{ex}
	[Mã 102 - 2020 Lần 1]%Câu 3
	Họ nguyên hàm của hàm số $f(x)=x^3$ là
	\choice
	{$4 x^4+C$}
	{$3 x^2+C$}
	{$x^4+C$}
	{\True $\dfrac{1}{4}x^4+C$}
	\loigiai{
		Ta có $\displaystyle\int\limits x^3 \mathrm{d}x=\dfrac{x^4}{4}+C$.
	}
\end{ex}
\begin{ex}
	[Mã 103 - 2020 Lần 1]%Câu 4
	$\displaystyle\int{x^4\mathrm{d}x}$ bằng
	\choice
	{\True $\dfrac{1}{5}{x^5}+C$}
	{$4x^3+C$}
	{$x^5+C$}
	{$5x^5+C$}
	\loigiai{
		$\displaystyle\int{x^4\mathrm{d}}x=\dfrac{1}{5}{x^5}+C$.
	}
\end{ex}
\begin{ex}
	[Mã 104 - 2020 Lần 1]%Câu 5
	$\displaystyle\int{x^5\mathrm{d}x}$ bằng
	\choice
	{$ 5x^4+C$}
	{\True $\dfrac{1}{6}{x^6}+C$}
	{$x^6+C$}
	{$ 6x^6+C$}
	\loigiai{}
\end{ex}
\begin{ex}
	[Mã 101 - 2020 Lần 2]%Câu 6
	$\displaystyle\int{5x^4\mathrm{d}x}$ bằng
	\choice
	{$\dfrac{1}{5}{x^5}+C$}
	{\True $x^5+C$}
	{$ 5x^5+C$}
	{$ 20x^3+C$}
	\loigiai{
		Ta có $\displaystyle\int{5x^4\mathrm{d}x}=x^5+C$.
	}
\end{ex}
\begin{ex}
	[Mã 102 - 2020 Lần 2]%Câu 7
	$\displaystyle\int{6x^5\mathrm{d}x}$ bằng
	\choice
	{$ 6x^6+C$}
	{\True $x^6+C$}
	{$\dfrac{1}{6}{x^6}+C$}
	{$ 30x^4+C$}
	\loigiai{
		Ta có $\displaystyle\int{6x^5\mathrm{d}x}=x^6+C$.
	}
\end{ex}
\begin{ex}
	[Mã 103 - 2020 Lần 2]%Câu 8
	$\displaystyle\int{3x^2}\mathrm{d}x$$ $bằng
	\choice
	{$ 3x^3+C$}
	{$ 6x+C$}
	{$ $$\dfrac{1}{3}{x^3}+C$}
	{\True $x^3+C$}
	\loigiai{
		Ta có $\displaystyle\int{3x^2}\mathrm{d}x=3.\dfrac{x^3}{3}+C=x^3+C$.
	}
\end{ex}
\begin{ex}
	[Mã 104 - 2020 Lần 2]%Câu 9
	$\displaystyle\int{4x^3\mathrm{d}x}$ bằng
	\choice
	{$4x^4+C$}
	{$\dfrac{1}{4}{x^4}+C$}
	{$12x^2+C$}
	{\True $x^4+C$}
	\loigiai{
		Ta có $\displaystyle\int{4x^3\mathrm{d}x}$ $=x^4+C$.
	}
\end{ex}
\begin{ex}
	[Mã 103 2018]%Câu 10
	Nguyên hàm của hàm số $f(x)=x^4+x^2$ là
	\choice
	{\True $\dfrac{1}{5}{x^5}+\dfrac{1}{3}{x^3}+C$}
	{$x^4+x^2+C$}
	{$x^5+x^3+C$}
	{$4x^3+2x+C$}
	\loigiai{
		$\displaystyle\int{f(x)\mathrm{d}x=}$ $\displaystyle\int{\left(x^4+x^2\right)\mathrm{d}x}$ $=\dfrac{1}{5}{x^5}+\dfrac{1}{3}{x^3}+C$.
	}
\end{ex}
\begin{ex}
	[Mã 104 - 2019]%Câu 11
	Họ tất cả nguyên hàm của hàm số $ f(x)=2x+4$ là
	\choice
	{$x^2+C$}
	{$ 2x^2+C$}
	{$ 2x^2+4x+C$}
	{\True $x^2+4x+C$}
	\loigiai{
		Ta có $\displaystyle\int{f(x)\mathrm{d}x=\displaystyle\int{\left(2x+4\right)\mathrm{d}x=x^2+4x+C}}$.
	}
\end{ex}
\begin{ex}
	[Mã 102 - 2019]%Câu 12
	Họ tất cả các nguyên hàm của hàm số $ f(x)=2x+6$ là
	\choice
	{$x^2+C$}
	{\True $x^2+6x+C$}
	{$2x^2+C$}
	{$2x^2+6x+C$}
	\loigiai{
		$\displaystyle\int{\left(2x+6\right)\mathrm{d}x=x^2+6x+C}$.
	}
\end{ex}
\begin{ex}
	[Đề Minh Họa 2020 Lần 1]%Câu 13
	Họ nguyên hàm của hàm số $ f(x)=\cos x+6x$ là
	\choice
	{\True $\sin x+3x^2+C$}
	{$-\sin x+3x^2+C$}
	{$\sin x+6x^2+C$}
	{$-\sin x+C$}
	\loigiai{
		Ta có $\displaystyle\int{f(x)\mathrm{d}x=\displaystyle\int{\left(\cos x+6x\right)\mathrm{d}x=\sin x+3x^2+C}}$.
	}
\end{ex}
\begin{ex}
	[Mã 101 - 2021 - Lần 1]%Câu 14
	Cho hàm số $ f(x)=x^2+4$. Khẳng định nào dưới đây đúng?
	\choice
	{$\displaystyle\int{f(x)\mathrm{d}x=2x+C}$}
	{$\displaystyle\int{f(x)\mathrm{d}x=x^2+4x+C}$}
	{\True $\displaystyle\int{f(x)\mathrm{d}x=\dfrac{x^3}{3}+4x+C}$}
	{$\displaystyle\int{f(x)\mathrm{d}x=x^3+4x+C}$}
	\loigiai{
		Ta có $ f(x)=x^2+4$$\Rightarrow $ $\displaystyle\int{f(x)\mathrm{d}x=\dfrac{x^3}{3}+4x+C}$.
	}
\end{ex}
\begin{ex}
	[Mã 101 - 2021 - Lần 2]%Câu 15
	Cho hàm số $ f(x)=4+\cos x$. Khẳng định nào dưới đây đúng?
	\choice
	{$\displaystyle\int{f(x)\mathrm{d}x=-\sin x+C}$}
	{\True $\displaystyle\int{f(x)\mathrm{d}x=4x+\sin x+C}$}
	{$\displaystyle\int{f(x)\mathrm{d}x=4x-\sin x+C}$}
	{$\displaystyle\int{f(x)\mathrm{d}x=4x+\cos x+C}$}
	\loigiai{
		Ta có $\displaystyle\int{f(x)\mathrm{d}x=4x+\sin x+C}$.
	}
\end{ex}
\begin{ex}
	[Mã 101 - 2021 - Lần 1]%Câu 16
	Cho hàm số $ f(x)=\mathrm{e}^x+2$. Khẳng định nào dưới đây là đúng?
	\choice
	{$\displaystyle\int{f(x)}\mathrm{d}x=\mathrm{e}^{x-2}+C$}
	{\True $\displaystyle\int{f(x)}\mathrm{d}x=\mathrm{e}^x+2x+C$}
	{$\displaystyle\int{f(x)}\mathrm{d}x=\mathrm{e}^x+C$}
	{$\displaystyle\int{f(x)}\mathrm{d}x=\mathrm{e}^x-2x+C$}
	\loigiai{
		Ta có $\displaystyle\int{f(x)}\mathrm{d}x=\displaystyle\int{\left(\mathrm{e}^x+2\right)}\mathrm{d}x=\mathrm{e}^x+2x+C$.
	}
\end{ex}
\begin{ex}
	[Mã 105 2017]%Câu 17
	Tìm nguyên hàm của hàm số $ f(x)=2\sin x$.
	\choice
	{\True $\displaystyle\int{2\sin x\mathrm{d}x=-2\cos x+C}$}
	{$\displaystyle\int{2\sin x\mathrm{d}x=2\cos x+C}$}
	{$\displaystyle\int{2\sin x\mathrm{d}x=\sin^2x+C}$}
	{$\displaystyle\int{2\sin x\mathrm{d}x=\sin 2x+C}$}
	\loigiai{}
\end{ex}
\begin{ex}
	[Mã 101 2018]%Câu 18
	Nguyên hàm của hàm số $ f(x)=x^3+x$ là
	\choice
	{\True $\dfrac{1}{4}{x^4}+\dfrac{1}{2}{x^2}+C$}
	{$ 3x^2+1+C$}
	{$x^3+x+C$}
	{$x^4+x^2+C$}
	\loigiai{
		$\displaystyle\int{\left(x^3+x^2\right)\mathrm{d}x}=\dfrac{1}{4}{x^4}+\dfrac{1}{2}{x^2}+C$.
	}
\end{ex}
\begin{ex}
	[Mã 103 - 2019]%Câu 19
	Họ tất cả các nguyên hàm của hàm số $ f(x)=2x+3$ là
	\choice
	{\True $x^2+3x+C$}
	{$ 2x^2+3x+C$}
	{$x^2+C$}
	{$ 2x^2+C$}
	\loigiai{
		Ta có $\displaystyle\int{\left(2x+3\right)\mathrm{d}x}=x^2+3x+C$.
	}
\end{ex}
\begin{ex}
	[Đề Minh Họa 2017]%Câu 20
	Tìm nguyên hàm của hàm số $ f(x)=\sqrt{2x-1}$
	\choice
	{$\displaystyle\int{f(x)\mathrm{d}x=\dfrac{2}{3}\left(2x-1\right)\sqrt{2x-1}+C}$}
	{\True $\displaystyle\int{f(x)\mathrm{d}x=\dfrac{1}{3}\left(2x-1\right)\sqrt{2x-1}+C}$}
	{$\displaystyle\int{f(x)\mathrm{d}x=-\dfrac{1}{3}\sqrt{2x-1}+C}$}
	{$\displaystyle\int{f(x)\mathrm{d}x=\dfrac{1}{2}\sqrt{2x-1}+C}$}
	\loigiai{
		$\begin{aligned}
			\displaystyle\int f(x)\mathrm{d}x&=\int \sqrt{2 x-1}\mathrm{d}x =\dfrac{1}{2} \int(2 x-1)^{\frac{1}{2}}\mathrm{d}(2 x-1) \\
			& =\dfrac{1}{3}(2 x-1) \sqrt{2 x-1}+C
		\end{aligned}$.
	}
\end{ex}
\begin{ex}
	[Đề Tham Khảo 2017]%Câu 21
	Tìm nguyên hàm của hàm số $ f(x)=x^2+\dfrac{2}{x^2}$.
	\choice
	{\True $\displaystyle\int{f(x)\mathrm{d}x}=\dfrac{x^3}{3}+\dfrac{1}{x}+C$}
	{$\displaystyle\int{f(x)\mathrm{d}x}=\dfrac{x^3}{3}-\dfrac{2}{x}+C$}
	{$\displaystyle\int{f(x)\mathrm{d}x}=\dfrac{x^3}{3}-\dfrac{1}{x}+C$}
	{$\displaystyle\int{f(x)\mathrm{d}x}=\dfrac{x^3}{3}+\dfrac{2}{x}+C$}
	\loigiai{
		Ta có $\displaystyle\int{\left(x^2+\dfrac{2}{x^2}\right)\mathrm{d}x}=\dfrac{x^3}{3}-\dfrac{2}{x}+C$.
	}
\end{ex}
\begin{ex}
	[Mã 110 2017]%Câu 22
	Tìm nguyên hàm của hàm số $ f(x)=\dfrac{1}{5x-2}$.
	\choice
	{\True $\displaystyle\int{\dfrac{\mathrm{d}x}{5x-2}=\dfrac{1}{5}\ln \left| 5x-2\right|+C}$}
	{$\displaystyle\int{\dfrac{\mathrm{d}x}{5x-2}=\ln \left| 5x-2\right|+C}$}
	{$\displaystyle\int{\dfrac{\mathrm{d}x}{5x-2}=-\dfrac{1}{2}\ln \left| 5x-2\right|+C}$}
	{$\displaystyle\int{\dfrac{\mathrm{d}x}{5x-2}=5\ln \left| 5x-2\right|+C}$}
	\loigiai{
		Áp dụng công thức $\displaystyle\int{\dfrac{\mathrm{d}x}{ax+b}=\dfrac{1}{a}\ln \left|ax+b\right|+C}\left(a\ne 0\right)$ ta được $\displaystyle\int{\dfrac{\mathrm{d}x}{5x-2}=\dfrac{1}{5}\ln \left|5x-2\right|+C}$.
	}
\end{ex}
\begin{ex}
	[Mã 123 2017]%Câu 23
	Tìm nguyên hàm của hàm số $ f(x)=\cos 3x$
	\choice
	{$\displaystyle\int{\cos 3x\mathrm{d}x=3\sin 3x+C}$}
	{\True $\displaystyle\int{\cos 3x\mathrm{d}x=\dfrac{\sin 3x}{3}+C}$}
	{$\displaystyle\int{\cos 3x\mathrm{d}x=\sin 3x+C}$}
	{$\displaystyle\int{\cos 3x\mathrm{d}x=-\dfrac{\sin 3x}{3}+C}$}
	\loigiai{
		Ta có $\displaystyle\int{\cos 3x\mathrm{d}x=\dfrac{\sin 3x}{3}+C}$.
	}
\end{ex}
\begin{ex}
	[Mã 104 2018]%Câu 24
	Nguyên hàm của hàm số $f(x)=x^3+x^2$ là
	\choice
	{\True $\dfrac{1}{4}{x^4}+\dfrac{1}{3}{x^3}+C$}
	{$3x^2+2x+C$}
	{$x^3+x^2+C$}
	{$x^4+x^3+C$}
	\loigiai{}
\end{ex}
\begin{ex}
	[Đề Tham Khảo 2019]%Câu 25
	Họ nguyên hàm của hàm số $ f(x)=\mathrm{e}^x+x$ là
	\choice
	{$\mathrm{e}^x+1+C$}
	{$\mathrm{e}^x+x^2+C$}
	{\True $\mathrm{e}^x+\dfrac{1}{2}{x^2}+C$}
	{$\dfrac{1}{x+1}{\mathrm{e}^x}+\dfrac{1}{2}{x^2}+C$}
	\loigiai{}
\end{ex}
\begin{ex}
	[Mã 101 - 2019]%Câu 26
	Họ tất cả các nguyên hàm của hàm số $ f(x)=2x+5$ là
	\choice
	{$x^2+C$}
	{\True $x^2+5x+C$}
	{$ 2x^2+5x+C$}
	{$ 2x^2+C$}
	\loigiai{
		Họ tất cả các nguyên hàm của hàm số $ f(x)=2x+5$ là $ F(x)=x^2+5x+C$.
	}
\end{ex}
\begin{ex}
	[Mã 104 2017]%Câu 27
	Tìm nguyên hàm của hàm số $ f(x)=7^x$.
	\choice
	{\True $\displaystyle\int{7^x\mathrm{d}x}=\dfrac{7^x}{\ln 7}+C$}
	{$\displaystyle\int{7^x\mathrm{d}x}=7^{x+1}+C$}
	{$\displaystyle\int{7^x\mathrm{d}x}=\dfrac{7^{x+1}}{x+1}+C$}
	{$\displaystyle\int{7^x\mathrm{d}x}=7^x\ln 7+C$}
	\loigiai{
		Áp dụng công thức $\displaystyle\int{a^x\mathrm{d}x}=\dfrac{a^x}{\ln a}+C,\left(0<a\ne 1\right)$.
	}
\end{ex}
\begin{ex}
	[Mã 102 2018]%Câu 28
	Nguyên hàm của hàm số $ f(x)=x^4+x$ là
	\choice
	{$ 4x^3+1+C$}
	{$x^5+x^2+C$}
	{\True $\dfrac{1}{5}{x^5}+\dfrac{1}{2}{x^2}+C$}
	{$x^4+x+C$}
	\loigiai{
		Ta có $\displaystyle\int{\left(x^4+x\right)\mathrm{d}x}=\dfrac{1}{5}{x^5}+\dfrac{1}{2}{x^2}+C$.
	}
\end{ex}
\begin{ex}
	[Đề Tham Khảo 2018]%Câu 29
	Họ nguyên hàm của hàm số $ f(x)=3x^2+1$ là
	\choice
	{$x^3+C$}
	{$\dfrac{x^3}{3}+x+C$}
	{$ 6x+C$}
	{\True $x^3+x+C$}
	\loigiai{
		$\displaystyle\int\left(3x^2+1\right)\mathrm{d}x=x^3+x+C$.
	}
\end{ex}
\begin{ex}
	[THPT An Lão Hải Phòng 2019]%Câu 30
	Tìm nguyên hàm $\displaystyle\int x\left(x^2+7\right)^{15}\mathrm{d}x$.
	\choice
	{$\dfrac{1}{2}{\left(x^2+7\right)^{16}}+C$}
	{$-\dfrac{1}{32}{\left(x^2+7\right)^{16}}+C$}
	{$\dfrac{1}{16}{\left(x^2+7\right)^{16}}+C$}
	{\True $\dfrac{1}{32}{\left(x^2+7\right)^{16}}+C$}
	\loigiai{
		$\displaystyle\int{x{\left(x^2+7\right)^{15}}{\mathrm{d}x}}=\dfrac{1}{2}\displaystyle\int{\left(x^2+7\right)^{15}\mathrm{d}\left(x^2+7\right)}=\dfrac{1}{32}{\left(x^2+7\right)^{16}}+C$.
	}
\end{ex}
\begin{ex}
	[THPT Ba Đình - 2019]%Câu 31
	Họ nguyên hàm của hàm số $f(x)=\mathrm{e}^{3x}$ là hàm số nào sau đây?
	\choice
	{$3\mathrm{e}^x+C$}
	{\True $\dfrac{1}{3}{\mathrm{e}^{3x}}+C$}
	{$\dfrac{1}{3}{\mathrm{e}^x}+C$}
	{$3\mathrm{e}^{3x}+C$}
	\loigiai{
		Ta có $\displaystyle\int{\mathrm{e}^{3x}\mathrm{d}x=\dfrac{1}{3}{\mathrm{e}^{3x}}+C,}$ với $C$ là hằng số bất kì.
	}
\end{ex}
\begin{ex}
	[THPT Cẩm Giàng 2 2019]%Câu 32
	Tính$\displaystyle\int{\left(x-\sin 2x\right)}\mathrm{d}x$.
	\choice
	{$\dfrac{x^2}{2}+\sin x+C$}
	{$\dfrac{x^2}{2}+\cos 2x+C$}
	{$x^2+\dfrac{\cos 2x}{2}+C$}
	{\True $\dfrac{x^2}{2}+\dfrac{\cos 2x}{2}+C$}
	\loigiai{
		Ta có $\displaystyle\int{\left(x-\sin 2x\right)\mathrm{d}x\text{=}\displaystyle\int{x\mathrm{d}x}}-\displaystyle\int{\sin 2x\mathrm{d}x}$$=\dfrac{x^2}{2}+\dfrac{\cos 2x}{2}+C$.
	}
\end{ex}
\begin{ex}
	[THPT Hoàng Hoa Thám Hưng Yên 2019]%Câu 33
	Nguyên hàm của hàm số $ y=\mathrm{e}^{2x-1}$ là
	\choice
	{${2}{{\mathrm{e}}^{2x-1}}+C$}
	{${\mathrm{e}}^{2x-1}+C$}
	{\True $\dfrac{1}{2}{{\mathrm{e}}^{2x-1}}+C$}
	{$\dfrac{1}{2}{{\mathrm{e}}^x}+C$}
	\loigiai{
		Ta có $\displaystyle\int{{\mathrm{e}}^{2x-1}\mathrm{d}x}=\dfrac{1}{2}\displaystyle\int{{\mathrm{e}}^{2x-1}\mathrm{d}\left(2x-1\right)=}\dfrac{1}{2}{{\mathrm{e}}^{2x-1}}+C$.
	}
\end{ex}
\begin{ex}
	[THPT Hùng Vương Bình Phước 2019]%Câu 34
	Tìm họ nguyên hàm của hàm số $ f(x)=\dfrac{1}{2x+3}$
	\choice
	{$\ln \left| 2x+3\right|+C$}
	{\True $\dfrac{1}{2}\ln \left| 2x+3\right|+C$}
	{$\dfrac{1}{\ln 2}\ln \left| 2x+3\right|+C$}
	{$\dfrac{1}{2}\lg\left(2x+3\right)+C$}
	\loigiai{}
\end{ex}
\begin{ex}
	[THPT Hùng Vương Bình Phước 2019]%Câu 35
	Tìm họ nguyên hàm của hàm số $ y=x^2-3^x+\dfrac{1}{x}$.
	\choice
	{$\dfrac{x^3}{3}-\dfrac{3^x}{\ln 3}-\dfrac{1}{x^2}+C,C\in\mathbb{R}$}
	{$\dfrac{x^3}{3}-3^x+\dfrac{1}{x^2}+C,C\in\mathbb{R}$}
	{\True $\dfrac{x^3}{3}-\dfrac{3^x}{\ln 3}+\ln \left|x\right|+C,C\in\mathbb{R}$}
	{$\dfrac{x^3}{3}-\dfrac{3^x}{\ln 3}-\ln \left|x\right|+C,C\in\mathbb{R}$}
	\loigiai{
		Ta có $\displaystyle\int{\left(x^2-3^x+\dfrac{1}{x}\right)}\mathrm{d}x=\dfrac{x^3}{3}-\dfrac{3^x}{\ln 3}+\ln \left| x\right|+C,C\in\mathbb{R}$.
	}
\end{ex}
\begin{ex}
	[THPT Hùng Vương Bình Phước 2019]%Câu 36
	Tìm họ nguyên hàm của hàm số $ f(x)=\sin 3x$
	\choice
	{$-3{cos}3x+C$}
	{${3cos}3x+C$}
	{$\dfrac{1}{3}{cos}3x+C$}
	{\True $-\dfrac{1}{3}{cos}3x+C$}
	\loigiai{
		$\displaystyle\int{\sin 3x}{\mathrm{d}x}=-\dfrac{\cos 3x}{3}+C$.
	}
\end{ex}
\begin{ex}
	[Chuyên KHTN 2019]%Câu 37
	Họ nguyên hàm của hàm số $f(x)=3x^2+\sin x$ là
	\choice
	{$x^3+\cos x+C$}
	{$6x+\cos x+C$}
	{\True $x^3-\cos x+C$}
	{$6x-\cos x+C$}
	\loigiai{
		Ta có $\displaystyle\int{\left(3x^2+\sin x\right)\mathrm{d}x}=x^3-\cos x+C$.
	}
\end{ex}
\begin{ex}
	[Chuyên Bắc Ninh - 2019]%Câu 38
	Công thức nào sau đây là sai?
	\choice
	{\True $\displaystyle\int{\ln x}\mathrm{d}x=\dfrac{1}{x}+C$}
	{$\displaystyle\int{\dfrac{1}{\cos^2x}}\mathrm{d}x=\tan x+C$}
	{$\displaystyle\int{\sin x}\mathrm{d}x=-\cos x+C$}
	{$\displaystyle\int{{e}^x}\mathrm{d}x={e}^x+C$}
	\loigiai{
		Ta có $\displaystyle\int{\ln x}\mathrm{d}x=\dfrac{1}{x}+C$ sai.
	}
\end{ex}
\begin{ex}
	[Chuyên Bắc Ninh 2019]%Câu 39
	Nếu $\displaystyle\int{f(x)\mathrm{d}x=4x^3+x^2+C}$ thì hàm số $ f(x)$ bằng
	\choice
	{$ f(x)=x^4+\dfrac{x^3}{3}+Cx$}
	{$ f(x)=12x^2+2x+C$}
	{\True $ f(x)=12x^2+2x$}
	{$ f(x)=x^4+\dfrac{x^3}{3}$}
	\loigiai{
		Có $ f(x)=\left(4x^3+x^2+C\right)'=12x^2+2x$.
	}
\end{ex}
\begin{ex}
	[THPT Lương Thế Vinh Hà Nội 2019]%Câu 40
	Trong các khẳng định sau, khẳng định nào sai?
	\choice
	{$\displaystyle\int{\cos 2x\mathrm{d}x=\dfrac{1}{2}\sin 2x+C}$}
	{$\displaystyle\int{x^{{e}}}\mathrm{d}x=\dfrac{x^{{\mathrm{e}}+1}}{{e}+1}+C$}
	{$\displaystyle\int{\dfrac{1}{x}\mathrm{d}x=\ln \left| x\right|}+C$}
	{\True $\displaystyle\int{{e}^x\mathrm{d}x=\dfrac{{\mathrm{e}}^{x+1}}{x+1}+C}$}
	\loigiai{
		Ta có $\displaystyle\int{{\mathrm{e}}^x\mathrm{d}x=\dfrac{{\mathrm{e}}^{x+1}}{x+1}+C}$ sai vì $\displaystyle\int{{\mathrm{e}}^x\mathrm{d}x={\mathrm{e}}^x+C}$.
	}
\end{ex}
\begin{ex}
	[THPT Lương Thế Vinh Hà Nội 2019]%Câu 41
	Nguyên hàm của hàm số $y=2^x$ là
	\choice
	{$\displaystyle\int{2^x\mathrm{d}x=\ln {2.2^x}+C}$}
	{$\displaystyle\int{2^x\mathrm{d}x=2^x+C}$}
	{\True $\displaystyle\int{2^x\mathrm{d}x=\dfrac{2^x}{\ln 2}+C}$}
	{$\displaystyle\int{2^x\mathrm{d}x=\dfrac{2^x}{x+1}+C}$}
	\loigiai{
		Do theo bảng nguyên hàm: $\displaystyle\int{a^x\mathrm{d}x=\dfrac{a^x}{\ln a}+C}$.
	}
\end{ex}
\begin{ex}
	[Liên Trường THPT Tp Vinh Nghệ An 2019]%Câu 42
	Tìm họ nguyên hàm của hàm số $ f(x)=3x-\sin x$.
	\choice
	{$\displaystyle\int{f(x)}\mathrm{d}x=3x^2+\cos x+C$}
	{$\displaystyle\int{f(x)}\mathrm{d}x=\dfrac{3x^2}{2}-\cos x+C$}
	{\True $\displaystyle\int{f(x)}\mathrm{d}x=\dfrac{3x^2}{2}+\cos x+C$}
	{$\displaystyle\int{f(x)}\mathrm{d}x=3+\cos x+C$}
	\loigiai{
		Ta có $\displaystyle\int{f(x)}\mathrm{d}x=\displaystyle\int{\left(3x-\sin x\right)}\mathrm{d}x=\dfrac{3x^2}{2}+\cos x+C$.
	}
\end{ex}
\begin{ex}
	[Sở Bình Phước 2019]%Câu 43
	Họ nguyên hàm của hàm số $ f(x)=x+\operatorname{s}{inx}$là
	\choice
	{$x^2+\cos{x+C}$}
	{$x^2-\cos{x+C}$}
	{\True $\dfrac{x^2}{2}-\cos{x+C}$}
	{$\dfrac{x^2}{2}+\cos{x+C}$}
	\loigiai{
		Theo bảng nguyên hàm cơ bản.
	}
\end{ex}
\begin{ex}
	[THPT Minh Khai Hà Tĩnh 2019]%Câu 44
	Họ nguyên hàm của hàm số $ f(x)=\cos x$ là:
	\choice
	{$\cos x+C$}
	{$-\cos x+C$}
	{$-\sin x+C$}
	{\True $\sin x+C$}
	\loigiai{
		Ta có $\displaystyle\int{\cos x\mathrm{d}x}=\sin x+C$.
	}
\end{ex}
\begin{ex}
	[THPT Đoàn Thượng - Hải Dương - 2019]%Câu 45
	Họ các nguyên hàm của hàm số $ f(x)=x^4+x^2$ là
	\choice
	{$ 4x^3+2x+C$}
	{$x^4+x^2+C$}
	{\True $\dfrac{1}{5}{x^5}+\dfrac{1}{3}{x^3}+C$}
	{$x^5+x^3+C$}
	\loigiai{
		Ta có $\displaystyle\int{f(x)\mathrm{d}x}=\displaystyle\int{\left(x^4+x^2\right)\mathrm{d}x}=\dfrac{1}{5}{x^5}+\dfrac{1}{3}{x^3}+C$.
	}
\end{ex}
\begin{ex}
	[THPT Cù Huy Cận 2019]%Câu 46
	Họ nguyên hàm của hàm số $ f(x)=\mathrm{e}^x-2x$ là.
	\choice
	{$\mathrm{e}^x+x^2+C$}
	{\True $\mathrm{e}^x-x^2+C$}
	{$\dfrac{1}{x+1}{\mathrm{e}^x}-x^2+C$}
	{$\mathrm{e}^x-2+C$}
	\loigiai{
		Ta có $\displaystyle\int{\left(\mathrm{e}^x-2x\right)\mathrm{d}x=\mathrm{e}^x-x^2+C}$.
	}
\end{ex}
\begin{ex}
	[Chuyên Hùng Vương Gia Lai 2019]%Câu 47
	Họ các nguyên hàm của hàm số $y=\cos x+x$ là
	\choice
	{\True $\sin x+\dfrac{1}{2}{x^2}+C$}
	{$\sin x+x^2+C$}
	{$-\sin x+\dfrac{1}{2}{x^2}+C$}
	{$-\sin x+x^2+C$}
	\loigiai{
		$\displaystyle\int{\left(\cos x+x\right)\mathrm{d}x}=\sin x+\dfrac{1}{2}{x^2}+C$.
	}
\end{ex}
\begin{ex}
	[Chuyên Lê Quý Đôn Điện Biên 2019]%Câu 48
	Họ nguyên hàm của hàm số $y=x^2-3x+\dfrac{1}{x}$ là
	\choice
	{$\dfrac{x^3}{3}-\dfrac{3x^2}{2}-\ln \left| x\right|+C$}
	{$\dfrac{x^3}{3}-\dfrac{3x^2}{2}+\ln x+C$}
	{\True $\dfrac{x^3}{3}-\dfrac{3x^2}{2}+\ln \left| x\right|+C$}
	{$\dfrac{x^3}{3}-\dfrac{3x^2}{2}+\dfrac{1}{x^2}+C$}
	\loigiai{
		Ta có
		$\displaystyle\int(x^2-3x+\dfrac{1}{x})\mathrm{d}x=\dfrac{x^3}{3}-\dfrac{3x^2}{2}+\ln \left| x\right|+C$.
	}
\end{ex}
\begin{ex}
	[Chuyên Phan Bội Châu Nghệ An 2019]%Câu 49
	Họ nguyên hàm của hàm số $ f(x)=\dfrac{1}{x}+\sin x$ là
	\choice
	{$\ln x-\cos x+C$}
	{$-\dfrac{1}{x^2}-\cos x+C$}
	{$\ln \left| x\right|+\cos x+C$}
	{\True $\ln \left| x\right|-\cos x+C$}
	\loigiai{
		Ta có $\displaystyle\int{f(x)\mathrm{d}x}=\displaystyle\int{\left(\dfrac{1}{x}+\sin x\right)\mathrm{d}x}=\displaystyle\int{\dfrac{1}{x}\mathrm{d}x}+\displaystyle\int{\sin x\mathrm{d}x}=\ln \left| x\right|-\cos x+C$.
	}
\end{ex}
\begin{ex}
	[THPT Yên Phong 1 Bắc Ninh 2019]%Câu 50
	Hàm số $ F(x)=\dfrac{1}{3}{x^3}$ là một nguyên hàm của hàm số nào sau đây trên $\left(-\infty ;+\infty\right)$?
	\choice
	{$ f(x)=3x^2$}
	{$ f(x)=x^3$}
	{\True $ f(x)=x^2$}
	{$ f(x)=\dfrac{1}{4}{x^4}$}
	\loigiai{
		Gọi $ F(x)=\dfrac{1}{3}{x^3}$ là một nguyên hàm của hàm số $ f(x)$.\\
		Suy ra $ F'(x)=f(x)\Rightarrow f(x)=x^2$.
	}
\end{ex}
\begin{ex}
	[THPT Yên Phong 1 Bắc Ninh 2019]%Câu 51
	Tìm họ nguyên hàm của hàm số $ f(x)=2^x$.
	\choice
	{$\displaystyle\int{f(x)}\mathrm{d}x=2^x+C$}
	{\True $\displaystyle\int{f(x)}\mathrm{d}x=\dfrac{2^x}{\ln 2}+C$}
	{$\displaystyle\int{f(x)}\mathrm{d}x=2^x\ln 2+C$}
	{$\displaystyle\int{f(x)}\mathrm{d}x=\dfrac{2^{x+1}}{x+1}+C$}
	\loigiai{
		Ta có: $\displaystyle\int{f(x)}\mathrm{d}x=\displaystyle\int{2^x}\mathrm{d}x=\dfrac{2^x}{\ln 2}+C$.
	}
\end{ex}
\begin{ex}
	[THPT - Yên Định Thanh Hóa 2019]%Câu 52
	Tìm nguyên hàm của hàm số $ f(x)=\dfrac{x^4+2}{x^2}$.
	\choice
	{$\displaystyle\int{f(x)\mathrm{d}x=}\dfrac{x^3}{3}-\dfrac{1}{x}+C$}
	{$\displaystyle\int{f(x)\mathrm{d}x=}\dfrac{x^3}{3}+\dfrac{2}{x}+C$}
	{$\displaystyle\int{f(x)\mathrm{d}x=}\dfrac{x^3}{3}+\dfrac{1}{x}+C$}
	{\True $\displaystyle\int{f(x)\mathrm{d}x=}\dfrac{x^3}{3}-\dfrac{2}{x}+C$}
	\loigiai{
		Ta có $\displaystyle\int{f(x)\mathrm{d}x=}\displaystyle\int{\dfrac{x^4+2}{x^2}}\mathrm{d}x=\displaystyle\int{\left(x^2+\dfrac{2}{x^2}\right)}\mathrm{d}x=\dfrac{x^3}{3}-\dfrac{2}{x}+C$.
	}
\end{ex}
\begin{ex}
	[Sở Hà Nội 2019]%Câu 53
	Hàm số nào trong các hàm số sau đây là một nguyên hàm của hàm số $ y=\mathrm{e}^x$?
	\choice
	{$ y=\dfrac{1}{x}$}
	{\True $ y=\mathrm{e}^x$}
	{$ y=\mathrm{e}^{-x}$}
	{$ y=\ln x$}
	\loigiai{
		Ta có $\left(\mathrm{e}^x\right)'=\mathrm{e}^x$$\Rightarrow y=\mathrm{e}^x$ là một nguyên hàm của hàm số $ y=\mathrm{e}^x$.
	}
\end{ex}
\begin{ex}
	[Chuyên Lương Thế Vinh Đồng Nai 2019]%Câu 54
	Tính $ F(x)=\displaystyle\int{\mathrm{e}^2}\mathrm{d}x$, trong đó $ \mathrm{e}$ là hằng số và $ \mathrm{e}\approx 2,718$.
	\choice
	{$ F(x)=\dfrac{\mathrm{e}^2x^2}{2}+C$}
	{$ F(x)=\dfrac{\mathrm{e}^3}{3}+C$}
	{\True $ F(x)=\mathrm{e}^2x+C$}
	{$F(x)=2\mathrm{e}x+C$}
	\loigiai{
		Ta có $ F(x)=\displaystyle\int{\mathrm{e}^2}\mathrm{d}x=\mathrm{e}^2x+C$.
	}
\end{ex}
\begin{ex}
	[Chuyên Lê Quý Đôn Quảng Trị 2019]%Câu 55
	Tìm nguyên hàm của hàm số $ f(x)=\dfrac{1}{1-2x}$ trên $\left(-\infty;\dfrac{1}{2}\right)$.
	\choice
	{$\dfrac{1}{2}\ln \left| 2x-1\right|+C$}
	{$\dfrac{1}{2}\ln \left(1-2x\right)+C$}
	{\True $-\dfrac{1}{2}\ln \left| 2x-1\right|+C$}
	{$\ln \left| 2x-1\right|+C$}
	\loigiai{
		Trên khoảng $\left(-\infty;\dfrac{1}{2}\right)$, ta có $\displaystyle\int{f(x)}\mathrm{d}x$ $=\displaystyle\int{\dfrac{1}{1-2x}}\mathrm{d}x$ $=-\dfrac{1}{2}\displaystyle\int{\dfrac{1}{1-2x}}\mathrm{d}\left(1-2x\right)$ $=-\dfrac{1}{2}\ln \left| 2x-1\right|+C$.
	}
\end{ex}
\begin{ex}
	[Chuyên Hưng Yên 2019]%Câu 56
	Nguyên hàm của hàm số $ f(x)=2^x+x$ là
	\choice
	{\True $\dfrac{2^x}{\ln 2}+\dfrac{x^2}{2}+C$}
	{$2^x+x^2+C$}
	{$\dfrac{2^x}{\ln 2}+x^2+C$}
	{$2^x+\dfrac{x^2}{2}+C$}
	\loigiai{
		Ta có $\displaystyle\int{\left(2^x+x\right)}\mathrm{d}x=\dfrac{2^x}{\ln 2}+\dfrac{1}{2}{x^2}+C$.
	}
\end{ex}
\begin{ex}
	[Chuyên Sơn La 2019]%Câu 57
	Họ nguyên hàm của hàm số $f(x)=1+\sin x$ 
	\choice
	{$ 1+\cos x+C$}
	{$ 1-\cos x+C$}
	{$ x+\cos x+C$}
	{\True $ x-\cos x+C$}
	\loigiai{
		Ta có $\displaystyle\int{f(x)}\mathrm{d}x=\displaystyle\int{\left(1+\sin x\right)}\mathrm{d}x=x-\cos x+C$.
	}
\end{ex}
\begin{ex}
	[THPT Đông Sơn Thanh Hóa 2019]%Câu 58
	Nguyên hàm của hàm số $f(x)=$ $\dfrac{1}{3}{x^3}-2x^2+x-2019$ là
	\choice
	{$\dfrac{1}{12}{x^4}-\dfrac{2}{3}{x^3}+\dfrac{x^2}{2}+C$}
	{$\dfrac{1}{9}{x^4}-\dfrac{2}{3}{x^3}+\dfrac{x^2}{2}-2019x+C$}
	{\True $\dfrac{1}{12}{x^4}-\dfrac{2}{3}{x^3}+\dfrac{x^2}{2}-2019x+C$}
	{$\dfrac{1}{9}{x^4}+\dfrac{2}{3}{x^3}-\dfrac{x^2}{2}-2019x+C$}
	\loigiai{
		Sử dụng công thức $\displaystyle\int{x^n\mathrm{d}x=\dfrac{x^{n+1}}{n+1}+C}$ ta được:\\
		$\displaystyle\int{\left(\dfrac{1}{3}{x^3}-2x^2+x-2019\right)\mathrm{d}x=}\dfrac{1}{3}.\dfrac{x^4}{4}-2.\dfrac{x^3}{3}+\dfrac{x^2}{2}-2019x+C=\dfrac{1}{12}{x^4}-\dfrac{2}{3}{x^3}+\dfrac{1}{2}{x^2}-2019x+C$.
	}
\end{ex}
\begin{ex}
	[THPT Yên Khánh - Ninh Bình - 2019]%Câu 59
	Họ nguyên hàm của hàm số $ f(x)=\dfrac{1}{3x-1}$ trên khoảng $\left(-\infty ;\dfrac{1}{3}\right)$ là
	\choice
	{$\dfrac{1}{3}\ln (3x-1)+C$}
	{$\ln (1-3x)+C$}
	{\True $\dfrac{1}{3}\ln (1-3x)+C$}
	{$\ln (3x-1)+C$}
	\loigiai{
		Ta có $\displaystyle\int{\dfrac{1}{3x-1}}\mathrm{d}x=\dfrac{1}{3}\displaystyle\int{\dfrac{d(3x-1)}{3x-1}}=\dfrac{1}{3}\ln \left| 3x-1\right|+C=\dfrac{1}{3}\ln (1-3x)+C$ (do $ x\in\left(-\infty ;\dfrac{1}{3}\right)$).
	}
\end{ex}
\begin{ex}
	[Chuyên Lê Hồng Phong Nam Định 2019]%Câu 60
	Trong các khẳng định sau, khẳng định nào sai?
	\choice
	{\True $\displaystyle\int{2^x\mathrm{d}x}=2^x\ln 2+C$}
	{$\displaystyle\int{{e}^{2x}\mathrm{d}x}=\dfrac{{\mathrm{e}}^{2x}}{2}+C$}
	{$\displaystyle\int{\cos 2x\mathrm{d}x}=\dfrac{1}{2}\sin 2x+C$}
	{$\displaystyle\int{\dfrac{1}{x+1}\mathrm{d}x}=\ln \left| x+1\right|+C$ $\left(\forall x\ne-1\right)$}
	\loigiai{
		Ta có $\displaystyle\int{2^x\mathrm{d}x}=\dfrac{2^x}{\ln 2}+C$.
	}
\end{ex}
\begin{ex}
	[Chuyên Lê Hồng Phong Nam Định 2019]%Câu 61
	Cho hàm số $ f(x)=\dfrac{2x^4+3}{x^2}$. Khẳng định nào sau đây là đúng?
	\choice
	{$\displaystyle\int{f(x)\mathrm{d}x=\dfrac{2x^3}{3}+\dfrac{3}{2x}+C}$}
	{\True $\displaystyle\int{f(x)\mathrm{d}x=\dfrac{2x^3}{3}-\dfrac{3}{x}+C}$}
	{$\displaystyle\int{f(x)\mathrm{d}x=\dfrac{2x^3}{3}+\dfrac{3}{x}+C}$}
	{$\displaystyle\int{f(x)\mathrm{d}x=2x^3-\dfrac{3}{x}+C}$}
	\loigiai{
		Ta có $\displaystyle\int{f(x)\mathrm{d}x=\displaystyle\int{\dfrac{2x^4+3}{x^2}}\mathrm{d}x=}\displaystyle\int{\left(2x^2+\dfrac{3}{x^2}\right)\mathrm{d}x=\dfrac{2x^3}{3}-\dfrac{3}{x}+C}$.
	}
\end{ex}
\begin{ex}
	[Sở Thanh Hóa 2019]%Câu 62
	Cho hàm số $f(x)=2^x+x+1$. Tìm $\displaystyle\int{f(x)\mathrm{d}x}$.
	\choice
	{$\displaystyle\int{f(x)\mathrm{d}x}=2^x+x^2+x+C$}
	{\True $\displaystyle\int{f(x)\mathrm{d}x}=\dfrac{1}{\ln 2}{2^x}+\dfrac{1}{2}{x^2}+x+C$}
	{$\displaystyle\int{f(x)\mathrm{d}x}=2^x+\dfrac{1}{2}{x^2}+x+C$}
	{$\displaystyle\int{f(x)\mathrm{d}x}=\dfrac{1}{x+1}{2^x}+\dfrac{1}{2}{x^2}+x+C$}
	\loigiai{
		Ta có $\displaystyle\int{\left(2^x+x+1\right)\mathrm{d}x}=\dfrac{1}{\ln 2}{2^x}+\dfrac{1}{2}{x^2}+x+C$.
	}
\end{ex}
\begin{ex}
	[Liên Trường THPT Tp Vinh Nghệ An 2019]%Câu 63
	Tìm họ nguyên hàm của hàm số $ f(x)=3x-\sin x$.
	\choice
	{$\displaystyle\int{f(x)}\mathrm{d}x=3x^2+\cos x+C$}
	{$\displaystyle\int{f(x)}\mathrm{d}x=\dfrac{3x^2}{2}-\cos x+C$}
	{\True $\displaystyle\int{f(x)}\mathrm{d}x=\dfrac{3x^2}{2}+\cos x+C$}
	{$\displaystyle\int{f(x)}\mathrm{d}x=3+\cos x+C$}
	\loigiai{
		Ta có $\displaystyle\int{f(x)}\mathrm{d}x=\displaystyle\int{\left(3x-\sin x\right)}\mathrm{d}x=\dfrac{3x^2}{2}+\cos x+C$.
	}
\end{ex}
\begin{ex}
	[Chuyên Bắc Giang 2019]%Câu 64
	Hàm số $ F(x)=\mathrm{e}^{x^2}$ là nguyên hàm của hàm số nào trong các hàm số sau
	\choice
	{\True $ f(x)=2x{\mathrm{e}^{x^2}}$}
	{$ f(x)=x^2\mathrm{e}^{x^2}-1$}
	{$ f(x)=\mathrm{e}^{2x}$}
	{$ f(x)=\dfrac{\mathrm{e}^{x^2}}{2x}$}
	\loigiai{
		Ta có $ f(x)=F'(x)$ $\Rightarrow f(x)=\left(\mathrm{e}^{x^2}\right)'=2x{\mathrm{e}^{x^2}}$.
	}
\end{ex}
\begin{ex}
	[Chuyên Đại Học Vinh 2019]%Câu 65
	Tất cả các nguyên hàm của hàm số $ f(x)=3^{-x}$ là
	\choice
	{\True $-\dfrac{3^{-x}}{\ln 3}+C$}
	{$-3^{-x}+C$}
	{$3^{-x}\ln 3+C$}
	{$\dfrac{3^{-x}}{\ln 3}+C$}
	\loigiai{
		Ta có $\displaystyle\int{f(x)\mathrm{d}x}=\displaystyle\int{3^{-x}\mathrm{d}x}=-\displaystyle\int{3^{-x}\mathrm{d}(-x)}=-\dfrac{3^{-x}}{\ln 3}+C$.
	}
\end{ex}
\begin{ex}
	[Sở Phú Thọ 2019]%Câu 66
	Họ nguyên hàm của hàm số $ f(x)=x^3+x^2$ là
	\choice
	{\True $\dfrac{x^4}{4}+\dfrac{x^3}{3}+C$}
	{$x^4+x^3+C$}
	{$ 3x^2+2x+C$}
	{$\dfrac{x^4}{3}+\dfrac{x^3}{4}+C$}
	\loigiai{
		$\displaystyle\int{f(x)\mathrm{d}x}=\displaystyle\int{\left(x^3+x^2\right)\mathrm{d}x=\dfrac{x^4}{4}+\dfrac{x^3}{3}+C}$.
	}
\end{ex}
\begin{ex}
	[Chuyên ĐHSP Hà Nội 2019]%Câu 67
	Hàm số nào trong các hàm số sau đây không là nguyên hàm của hàm số $ y=x^{2019}$?
	\choice
	{$\dfrac{x^{2020}}{2020}+1$}
	{$\dfrac{x^{2020}}{2020}$}
	{\True $ y=2019x^{2018}$}
	{$\dfrac{x^{2020}}{2020}-1$}
	\loigiai{
		Ta có $\displaystyle\int{x^{2019}\mathrm{d}}x=\dfrac{x^{2020}}{2020}+C$, $C$là hằng số.
	}
\end{ex}
\begin{ex}
	[Chuyên Quốc Học Huế 2019]%Câu 68
	Tìm họ nguyên hàm của hàm số $ y=x^2-3^x+\dfrac{1}{x}$.
	\choice
	{$\dfrac{x^3}{3}-\dfrac{3^x}{\ln 3}-\ln \left| x\right|+C,C\in R$}
	{\True $\dfrac{x^3}{3}-\dfrac{3^x}{\ln 3}+\ln \left| x\right|+C,C\in R$}
	{$\dfrac{x^3}{3}-3^x+\dfrac{1}{x^2}+C,C\in R$}
	{$\dfrac{x^3}{3}-\dfrac{3^x}{\ln 3}-\dfrac{1}{x^2}+C,C\in R$}
	\loigiai{
		Ta có $\displaystyle\int{\left(x^2-3^x+\dfrac{1}{x}\right)}\mathrm{d}x=\dfrac{x^3}{3}-\dfrac{3^x}{\ln 3}+\ln \left| x\right|+C,C\in \mathbb{R}$.
	}
\end{ex}
\begin{ex}
	[Quảng Ninh 2019]%Câu 69
	Tìm nguyên hàm của hàm số $ f(x)=\mathrm{e}^x\left(2017-\dfrac{2018\mathrm{e}^{-x}}{x^5}\right)$.
	\choice
	{$\displaystyle\int{f(x)\mathrm{d}x}=2017\mathrm{e}^x-\dfrac{2018}{x^4}+C$}
	{$\displaystyle\int{f(x)\mathrm{d}x}=2017\mathrm{e}^x+\dfrac{2018}{x^4}+C$}
	{\True $\displaystyle\int{f(x)\mathrm{d}x}=2017\mathrm{e}^x+\dfrac{504,5}{x^4}+C$}
	{$\displaystyle\int{f(x)\mathrm{d}x}=2017\mathrm{e}^x-\dfrac{504,5}{x^4}+C$}
	\loigiai{
		$\displaystyle\int{f(x)}\mathrm{d}x=\displaystyle\int{\mathrm{e}^x\left(2017-\dfrac{2018\mathrm{e}^{-x}}{x^5}\right)}\mathrm{d}x=\displaystyle\int{\left(2017\mathrm{e}^x-\dfrac{2018}{x^5}\right)}\mathrm{d}x=2017\mathrm{e}^x+\dfrac{504,5}{x^4}+C$.
	}
\end{ex}
\begin{ex}
	[HSG Bắc Ninh 2019]%Câu 70
	Họ nguyên hàm của hàm số $y=\mathrm{e}^x\left(2+\dfrac{\mathrm{e}^{-x}}{\cos^2x}\right)$ là
	\choice
	{\True $2\mathrm{e}^x+\tan x+C$}
	{$2\mathrm{e}^x-\tan x+C$}
	{$2\mathrm{e}^x-\dfrac{1}{\cos x}+C$}
	{$2\mathrm{e}^x+\dfrac{1}{\cos x}+C$}
	\loigiai{
		Ta có $y=\mathrm{e}^x\left(2+\dfrac{\mathrm{e}^{-x}}{\cos^2x}\right)=2\mathrm{e}^x+\dfrac{1}{\cos^2x}$\\
		$\displaystyle\int{y\mathrm{d}x=}\displaystyle\int{\left(2\mathrm{e}^x+\dfrac{1}{\cos^2x}\right)\mathrm{d}x=2\mathrm{e}^x+\tan x+C}$.
	}
\end{ex}
\begin{ex}
	[Chuyên Hạ Long 2019]%Câu 71
	Tìm nguyên $F(x)$ của hàm số $f(x)=\left(x+1\right)\left(x+2\right)\left(x+3\right)?$ 
	\choice
	{$F(x)=\dfrac{x^4}{4}-6x^3+\dfrac{11}{2}{x^2}-6x+C$}
	{$F(x)=x^4+6x^3+11x^2+6x+C$}
	{\True $F(x)=\dfrac{x^4}{4}+2x^3+\dfrac{11}{2}{x^2}+6x+C$}
	{$F(x)=x^3+6x^2+11x^2+6x+C$}
	\loigiai{
		Ta có $f(x)=x^3+6x^2+11x+6.$\\
		 Suy ra $F(x)=\displaystyle\int{\left(x^3+6x^2+11x+6\right)}\mathrm{d}x=\dfrac{x^4}{4}+2x^3+\dfrac{11}{2}{x^2}+6x+C$.
	}
\end{ex}
\begin{ex}
	[Sở Bắc Ninh 2019]%Câu 72
	họ nguyên hàm của hàm số $ f(x)=\dfrac{1}{5x+4}$ là
	\choice
	{$\dfrac{1}{5}\ln \left(5x+4\right)+C$}
	{$\ln \left| 5x+4\right|+C$}
	{$\dfrac{1}{\ln 5}\ln \left| 5x+4\right|+C$}
	{\True $\dfrac{1}{5}\ln \left| 5x+4\right|+C$}
	\loigiai{
		Ta có $\displaystyle\int{\dfrac{1}{5x+4}\mathrm{d}x}=\dfrac{1}{5}\displaystyle\int{\dfrac{1}{5x+4}\mathrm{d}\left(5x+4\right)}=\dfrac{1}{5}\ln \left| 5x+4\right|+C$.
	}
\end{ex}
\begin{ex}
	[Đề minh họa 2022]%Câu 73
	Trên khoảng $\left(0;+\infty\right)$, họ nguyên hàm của hàm số $f(x)=x^{\frac{3}{2}}$ là
	\choice
	{$\displaystyle\int{f(x)\mathrm{d}x=\dfrac{3}{2}{x^{\frac{1}{2}}}}+C$}
	{$\displaystyle\int{f(x)\mathrm{d}x=\dfrac{5}{2}{x^{\frac{2}{5}}}}+C$}
	{\True $\displaystyle\int{f(x)\mathrm{d}x=\dfrac{2}{5}{x^{\frac{5}{2}}}}+C$}
	{$\displaystyle\int{f(x)\mathrm{d}x=\dfrac{2}{3}{x^{\frac{1}{2}}}}+C$}
	\loigiai{
		Áp dụng công thức: $\displaystyle\int{x^{\alpha}}\mathrm{d}x=\dfrac{x^{\alpha+1}}{\alpha+1}+C$. Ta có\\
		$\displaystyle\int{f(x)\mathrm{d}x=\displaystyle\int{x^{\frac{3}{2}}}\mathrm{d}x=\dfrac{x^{\frac{3}{2}+1}}{{\frac{3}{2}+1}}+C=\dfrac{2}{5}{x^{\frac{5}{2}}}}+C$.
	}
\end{ex}
\begin{ex}
	[Đề minh họa 2022]%Câu 74
	Cho hàm số $ f(x)=1+\sin x$. Khẳng định nào dưới đây đúng?
	\choice
	{\True $\displaystyle\int{f(x)\mathrm{d}x}=x-\cos x+C$} {$\displaystyle\int{f(x)\mathrm{d}x}=x+\sin x+C$} {$\displaystyle\int{f(x)\mathrm{d}x}=x+\cos x+C$} {$\displaystyle\int{f(x)\mathrm{d}x}=\cos x+C$}
	\loigiai{
		Ta có $\displaystyle\int{f(x)\mathrm{d}x}=\displaystyle\int{\left(1+\sin x\right)\mathrm{d}x}=x-\cos x+C$.
	}
\end{ex}
\begin{ex}
	[Mã 101 - 2022]%Câu 75
	Cho hàm số $ f(x)=\mathrm{e}^x+2x.$ Khẳng định nào dưới đây đúng?
	\choice
	{\True $\displaystyle\int{f(x)}\mathrm{d}x=\mathrm{e}^x+x^2+C$}
	{$\displaystyle\int{f(x)}\mathrm{d}x=\mathrm{e}^x+C$}
	{$\displaystyle\int{f(x)}\mathrm{d}x=\mathrm{e}^x-x^2+C$}
	{$\displaystyle\int{f(x)}\mathrm{d}x=\mathrm{e}^x+2x^2+C$}
	\loigiai{
		Ta có $\displaystyle\int{f(x)}\mathrm{d}x=\displaystyle\int{\left(\mathrm{e}^x+2x\right)\mathrm{d}}x=\mathrm{e}^x+x^2+C$.
	}
\end{ex}
\begin{ex}
	[Mã 101 - 2022]%Câu 76
	Cho hàm số $ f(x)=1-\dfrac{1}{\cos^22x}$. Khẳng định nào dưới đây đúng?
	\choice
	{$\displaystyle\int{f(x)}\mathrm{d}x=x+\tan 2x+C$}
	{$\displaystyle\int{f(x)}\mathrm{d}x=x+\dfrac{1}{2}\cot 2x+C$}
	{\True $\displaystyle\int{f(x)}\mathrm{d}x=x-\dfrac{1}{2}\tan 2x+C$}
	{$\displaystyle\int{f(x)}\mathrm{d}x=x+\dfrac{1}{2}\tan 2x+C$}
	\loigiai{
		$\displaystyle\int{f(x)}\mathrm{d}x=\displaystyle\int{\left(1-\dfrac{1}{\cos^22x}\right)}\mathrm{d}x=\displaystyle\int{\mathrm{d}x}-\dfrac{1}{2}\displaystyle\int{\dfrac{\mathrm{d}\left({2}x\right)}{\cos^22x}}=x-\dfrac{1}{2}\tan 2x+C$.
	}
\end{ex}
\begin{ex}
	[Mã 102 - 2022]%Câu 77
	Cho $\displaystyle\int{f(x)}\mathrm{d}x=-\cos x+C$. Khẳng định nào dưới đây đúng?
	\choice
	{$ f(x)=-\sin x$}
	{$ f(x)=\cos x$}
	{\True $ f(x)=\sin x$}
	{$ f(x)=-\cos x$}
	\loigiai{
		Ta có $ f(x)=\left(-\cos x+C\right)'=\sin x$.
	}
\end{ex}
\begin{ex}
	[Mã 103 - 2022]%Câu 78
	Khẳng định nào dưới đây đúng?
	\choice
	{$\displaystyle\int{\mathrm{e}^x\mathrm{d}x=x{\mathrm{e}^x}+C}$}
	{$\displaystyle\int{\mathrm{e}^x\mathrm{d}x=\mathrm{e}^{x+1}+C}$}
	{$\displaystyle\int{\mathrm{e}^x\mathrm{d}x=-\mathrm{e}^{x+1}+C}$}
	{\True $\displaystyle\int{\mathrm{e}^x\mathrm{d}x=\mathrm{e}^x+C}$}
	\loigiai{
		Ta có $\displaystyle\int{\mathrm{e}^x\mathrm{d}x=\mathrm{e}^x+C}$.
	}
\end{ex}
\begin{ex}
	[Mã 103 - 2022]%Câu 79
	Hàm số $ F(x)=\cot x$là một nguyên hàm của hàm số nào dưới đây trên khoảng $\left(0;\dfrac{\pi}{2}\right)$
	\choice
	{$f_2(x)=\dfrac{1}{\sin^2x}$}
	{$f_1(x)=-\dfrac{1}{\cos^2x}$}
	{$f_4(x)=\dfrac{1}{\cos^2x}$}
	{\True $f_3(x)=-\dfrac{1}{\sin^2x}$}
	\loigiai{
		Có $\displaystyle\int{\dfrac{1}{\sin^2x}\mathrm{d}x}=-\cot x+C$ suy ra $ F(x)=\cot x$ trên khoảng $\left(0;\dfrac{\pi}{2}\right)$ là một nguyên hàm của hàm số $f_3(x)=-\dfrac{1}{\sin^2x}$.
	}
\end{ex}
\begin{ex}
	[Mã 104 - 2022]%Câu 80
	Cho hàm số $ f(x)=1+\mathrm{e}^{2x}$. Khẳng định nào dưới đây đúng?
	\choice
	{$\displaystyle\int{f(x)\mathrm{d}x=x+\dfrac{1}{2}{\mathrm{e}^x}+C}$}
	{$\displaystyle\int{f(x)\mathrm{d}x=x+2\mathrm{e}^{2x}+C}$}
	{$\displaystyle\int{f(x)\mathrm{d}x=x+\mathrm{e}^{2x}+C}$}
	{\True $\displaystyle\int{f(x)\mathrm{d}x=x+\dfrac{1}{2}{\mathrm{e}^{2x}}+C}$}
	\loigiai{
		Ta có $\displaystyle\int{f(x)\mathrm{d}x}=\displaystyle\int{\left(1+\mathrm{e}^{2x}\right)\mathrm{d}x=x+\dfrac{1}{2}{\mathrm{e}^{2x}}+C}$.
	}
\end{ex}         
\Closesolutionfile{ans}
\indapan{10}{ans/CD25/Muc_5_6}