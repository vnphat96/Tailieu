\Opensolutionfile{ans}[ans/CD25/Muc_9_10]
\setcounter{ex}{0}
\setcounter{dang}{0}
\section{Mức độ 9,10 điểm}
\begin{dang}
	{Nguyên hàm của hàm ẩn hoặc liên quan đến phương trình $f(x)$, $f'(x)$, $f''(x)$}
\end{dang}
\begin{ex}
	[Mã 103 2018]%Câu 1
	Cho hàm số $f(x)$ thỏa mãn $f(2)=-\dfrac{1}{25}$ và $f'(x)=4x^3\left[f(x)\right]^2$ với mọi $x\in\mathbb{R}$. Giá trị của $f(1)$ bằng
	\choice
	{$-\dfrac{391}{400}$}
	{$-\dfrac{1}{40}$}
	{$-\dfrac{41}{400}$}
	{\True $-\dfrac{1}{10}$}
	\loigiai{
		Ta có $f'(x)=4x^3\left[f(x)\right]^2$ $\Rightarrow-\dfrac{f'(x)}{\left[f(x)\right]^2}=-4x^3$ $\Rightarrow{\left[\dfrac{1}{f(x)}\right]'}=-4x^3$ $\Rightarrow\dfrac{1}{f(x)}=-x^4+C$\\
		Do $f(2)=-\dfrac{1}{25}$, nên ta có $C=-9$. Do đó $f(x)=-\dfrac{1}{x^4+9}$ $\Rightarrow f(1)=-\dfrac{1}{10}$.
	}
\end{ex}
\begin{ex}
	[Chuyên Phan Bội Châu - Nghệ An - 2020]%Câu 2
	Cho hàm số $ y=f(x)$ đồng biến và có đạo hàm liên tục trên $\mathbb{R}$ thỏa mãn $\left(f'(x)\right)^2=f(x).\mathrm{e}^x,\forall x\in\mathbb{R}$ và $ f(0)=2$. Khi đó $ f(2)$ thuộc khoảng nào sau đây?
	\choice
	{$\left(12;13\right)$}
	{\True $\left(9;10\right)$}
	{$\left(11;12\right)$}
	{$\left(13;14\right)$}
	\loigiai{
		Vì hàm số $ y=f(x)$ đồng biến và có đạo hàm liên tục trên $\mathbb{R}$ đồng thời $ f(0)=2$ nên $f'(x)\ge 0$ và $ f(x)>0$ với mọi $ x\in\left[0;+\infty\right)$.\\
		Từ giả thiết $\left(f'(x)\right)^2=f(x).\mathrm{e}^x,\forall x\in\mathbb{R}$ suy ra $f'(x)=\sqrt{f(x)}.\mathrm{e}^{\frac{x}{2}},\forall x\in\left[0;+\infty\right).$\\
		Do đó, $\dfrac{f'(x)}{2\sqrt{f(x)}}=\dfrac{1}{2}{\mathrm{e}^{\frac{x}{2}}},\forall x\in\left[0;+\infty\right).$\\
		Lấy nguyên hàm hai vế, ta được $\sqrt{f(x)}=\mathrm{e}^{\frac{x}{2}}+C,\forall x\in\left[0;+\infty\right)$ với $ C$ là hằng số nào đó.\\
		Kết hợp với $ f(0)=2$, ta được $ C=\sqrt{2}-1$.\\
		Từ đó, tính được $ f(2)=\left(\mathrm{e}+\sqrt{2}-1\right)^2\approx 9,81$.
	}
\end{ex}
\begin{ex}
	[Chuyên Thái Bình - 2020]%Câu 3
	Cho hàm số $ y=f(x)$ thỏa mãn $ f(2)=-\dfrac{4}{19}$ và $f'(x)=x^3f^2(x)$ $\forall x\in\mathbb{R}$. Giá trị của $ f(1)$ bằng
	\choice
	{$-\dfrac{2}{3}$}
	{$-\dfrac{1}{2}$}
	{\True $-1$}
	{$-\dfrac{3}{4}$}
	\loigiai{
		Ta có $f'(x)=x^3f^2(x)\Leftrightarrow\dfrac{f'(x)}{f^2(x)}=x^3$$\Rightarrow\displaystyle\int{\dfrac{f'(x)}{f^2(x)}dx}=\displaystyle\int{x^3dx}\Leftrightarrow-\dfrac{1}{f(x)}=\dfrac{x^4}{4}+C$.\\
		Mà $ f(2)=-\dfrac{4}{19}$$\Rightarrow\dfrac{19}{4}=\dfrac{16}{4}+C\Rightarrow C=\dfrac{3}{4}$. Suy ra $ f(x)=-\dfrac{4}{x^4+3}$.\\
		Vậy $ f(1)=-1$.
	}
\end{ex}
\begin{ex}
	[Lý Nhân Tông - Bắc Ninh - 2020]%Câu 4
	Cho hàm số $ y=f(x)$ liên tục trên $\mathbb{R}\setminus\left\{-1;0\right\}$ thỏa mãn điều kiện: $ f(1)=-2\ln 2$ và $ x.\left(x+1\right).f'(x)+f(x)=x^2+x$. Biết $ f(2)=a+b.\ln 3$ ($ a$, $ b\in\mathbb{Q}$). Giá trị $ 2\left(a^2+b^2\right)$ là
	\choice
	{$\dfrac{27}{4}$}
	{\True $ 9$}
	{$\dfrac{3}{4}$}
	{$\dfrac{9}{2}$}
	\loigiai{
		Chia cả hai vế của biểu thức $ x.\left(x+1\right).f'(x)+f(x)=x^2+x$ cho $\left(x+1\right)^2$ ta có\\
		$\dfrac{x}{x+1}.f'(x)+\dfrac{1}{\left(x+1\right)^2}f(x)=\dfrac{x}{x+1}\Leftrightarrow{\left[\dfrac{x}{x+1}.f(x)\right]^{\prime}}=\dfrac{x}{x+1}$.\\
		Vậy $\dfrac{x}{x+1}.f(x)=\displaystyle\int{\left[\dfrac{x}{x+1}.f(x)\right]^{\prime}}\mathrm{d}x=\displaystyle\int{\dfrac{x}{x+1}}\mathrm{d}x=\displaystyle\int{\left(1-\dfrac{1}{x+1}\right)}\mathrm{d}x=x-\ln\left| x+1\right|+C$.\\
		Do $ f(1)=-2\ln 2$ nên ta có $\dfrac{1}{2}.f(1)=1-\ln 2+C\Leftrightarrow-\ln 2=1-\ln 2+C\Leftrightarrow C=-1$.\\
		Khi đó $f(x)=\dfrac{x+1}{x}\left(x-\ln\left| x+1\right|-1\right)$.\\
		Vậy ta có $f(2)=\dfrac{3}{2}\left(2-\ln 3-1\right)=\dfrac{3}{2}\left(1-\ln 3\right)=\dfrac{3}{2}-\dfrac{3}{2}\ln 3\Rightarrow a=\dfrac{3}{2},b=-\dfrac{3}{2}$.\\
		Suy ra $ 2\left(a^2+b^2\right)=2\left[\left(\dfrac{3}{2}\right)^2+\left(-\dfrac{3}{2}\right)^2\right]=9$.
	}
\end{ex}
\begin{ex}
	[Hải Hậu - Nam Định - 2020]%Câu 5
	Cho hàm số $ y=f(x)$ thỏa mãn $ f(x)<0,\forall x>0$ và có đạo hàm $f'(x)$ liên tục trên khoảng $\left(0;+\infty\right)$ thỏa mãn $f'(x)=\left(2x+1\right){f^2}(x),\forall x>0$ và $ f(1)=-\dfrac{1}{2}$. Giá trị của biểu thức $ f(1)+f(2)+\ldots+f\left(2020\right)$ bằng
	\choice
	{\True $-\dfrac{2020}{2021}$}
	{$-\dfrac{2015}{2019}$}
	{$-\dfrac{2019}{2020}$}
	{$-\dfrac{2016}{2021}$}
	\loigiai{
		Ta có\\
		$f'(x)=\left(2x+1\right){f^2}(x)\Leftrightarrow\dfrac{f'(x)}{f^2(x)}=2x+1$$\Rightarrow\displaystyle\int{\dfrac{f'(x)}{f^2(x)}\mathrm{d}x=\displaystyle\int{\left(2x+1\right)\mathrm{d}x}}$$\Rightarrow-\dfrac{1}{f(x)}=x^2+x+C$.\\
		Mà $ f(1)=-\dfrac{1}{2}\Rightarrow C=0\Rightarrow f(x)=\dfrac{-1}{x^2+x}=\dfrac{1}{x+1}-\dfrac{1}{x}$.\\
		$\left\{\begin{aligned}
			& f(1)=\dfrac{1}{2}-1\\ 
			& f(2)=\dfrac{1}{3}-\dfrac{1}{2}\\ 
			& f(3)=\dfrac{1}{4}-\dfrac{1}{3}\\ 
			&\ldots\\ 
			& f\left(2020\right)=\dfrac{1}{2021}-\dfrac{1}{2020}\\ 
		\end{aligned}\right.$ $\Rightarrow f(1)+f(2)+\ldots+f\left(2020\right)=-1+\dfrac{1}{2021}=-\dfrac{2020}{2021}$.
	}
\end{ex}
\begin{ex}
	[Bắc Ninh 2019]%Câu 6
	Cho hàm số $y=f(x)$ liên tục trên $\mathbb{R}\setminus\left\{-1;0\right\}$ thỏa mãn $f(1)=2\ln 2+1$, $x\left(x+1\right){f}'(x)+\left(x+2\right)f(x)=x\left(x+1\right)$, $\forall x\in\mathbb{R}\setminus\left\{-1;0\right\}$. Biết $f(2)=a+b\ln 3$, với $a$, $b$ là hai số hữu tỉ. Tính $T=a^2-b$.
	\choice
	{\True $ T=\dfrac{-3}{16}$}
	{$ T=\dfrac{21}{16}$}
	{$ T=\dfrac{3}{2}$}
	{$ T=0$}
	\loigiai{
		Ta có $x\left(x+1\right){f}'(x)+\left(x+2\right)f(x)=x\left(x+1\right)$\\
		$\Leftrightarrow{f}'(x)+\dfrac{x+2}{x\left(x+1\right)}f(x)=1$ $\Leftrightarrow\dfrac{x^2}{x+1}{f}'(x)+\dfrac{x\left(x+2\right)}{\left(x+1\right)^2}f(x)=\dfrac{x^2}{x+1}$\\
		$\Leftrightarrow{\left[\dfrac{x^2}{x+1}f(x)\right]'}=\dfrac{x^2}{x+1}$ $\Leftrightarrow\dfrac{x^2}{x+1}f(x)=\displaystyle\int{\dfrac{x^2}{x+1}}\mathrm{d}x$ $\Leftrightarrow\dfrac{x^2}{x+1}f(x)=\dfrac{x^2}{2}-x+\ln\left| x+1\right|+c$\\
		$\Leftrightarrow f(x)=\dfrac{x+1}{x^2}\left(\dfrac{x^2}{2}-x+\ln\left| x+1\right|+c\right).$\\
		Ta có $f(1)=2\ln 2+1$ $\Leftrightarrow c=1.$\\
		Từ đó $f(x)=\dfrac{x+1}{x^2}\left(\dfrac{x^2}{2}-x+\ln\left| x+1\right|+1\right)$, $f(2)=\dfrac{3}{4}+\dfrac{3}{4}\ln 3$ Nên $\left\{\begin{aligned}
			&a=\dfrac{3}{4}\\
			&b=\dfrac{3}{4}\\
		\end{aligned}\right.$.\\
		Vậy $T=a^2-b=-\dfrac{3}{16}$.
	}
\end{ex}
\begin{ex}
	[THPT Nguyễn Trãi - Đà Nẵng - 2018]%Câu 7
	Cho hs $ y=f(x)$ thỏa mãn $y'=x{y^2}$ và $ f\left(-1\right)=1$ thì giá trị $ f(2)$ là
	\choice
	{$\mathrm{e}^2$}
	{$ 2\mathrm{e}$}
	{$ \mathrm{e}+1$}
	{\True $\mathrm{e}^3$}
	\loigiai{
		Ta có $y'=x{y^2}\Rightarrow\dfrac{y'}{y}=x^2\Rightarrow\displaystyle\int{\dfrac{y'}{y}\mathrm{d}x}=\displaystyle\int{x^2\mathrm{d}x}\Leftrightarrow\ln y=\dfrac{x^3}{3}+C$$\Leftrightarrow y=\mathrm{e}^{\frac{x^3}{3}+C}$.\\
		Theo giả thiết $ f\left(-1\right)=1$ nên $\text{e}^{-\frac{1}{3}+C}=1\Leftrightarrow C=\dfrac{1}{3}$.\\
		Vậy $ y=f(x)={\mathrm{e}^{\frac{x^3}{3}+\frac{1}{3}}}$. Do đó $ f(2)=\text{e}^3$.
	}
\end{ex}
\begin{ex}
	[Sở Hà Nội Năm 2019]%Câu 8
	Cho hàm số $ f(x)$ liên tục trên $\mathbb{R}$, $ f(x)\ne 0$ với mọi $ x$ và thỏa mãn $ f(1)=-\dfrac{1}{2}$,$f'(x)=\left(2x+1\right){f^2}(x)$.Biết $ f(1)+f(2)+\ldots+f\left(2019\right)=\dfrac{a}{b}-1$ với $ a,b\in\mathbb{N},\left(a,b\right)=1$.Khẳng định nào sau đây sai?
	\choice
	{\True $ a-b=2019$}
	{$ ab>2019$}
	{$ 2a+b=2022$}
	{$ b\le 2020$}
	\loigiai{
		$f'(x)=\left(2x+1\right){f^2}(x)$$\Leftrightarrow\dfrac{f'(x)}{f^2(x)}=2x+1$$\Rightarrow\displaystyle\int{\dfrac{f'(x)}{f^2(x)}}\mathrm{d}x=\displaystyle\int{\left(2x+1\right)}\mathrm{d}x$\\
		$\Rightarrow\displaystyle\int{\dfrac{d\left(f(x)\right)}{f^2(x)}}=\displaystyle\int{\left(2x+1\right)\mathrm{d}x}$\\
		$\Rightarrow-\dfrac{1}{f(x)}=x^2+x+C$ $(1)$(Với $ C$ là hằng số thực).\\
		Thay $ x=1$ vào $(1)$được $ 2+C=-\dfrac{1}{-\dfrac{1}{2}}$ $\Leftrightarrow C=0$.Vậy $ f(x)=\dfrac{1}{x+1}-\dfrac{1}{x}$.\\
		$T=f(1)+f(2)+\ldots+f(2019)=\left(\dfrac{1}{2}-\dfrac{1}{1}\right)+\left(\dfrac{1}{3}-\dfrac{1}{2}\right)+\ldots+\left(\dfrac{1}{2020}-\dfrac{1}{2019}\right)=-1+\dfrac{1}{2020}$.\\
		Suy ra: $\left\{\begin{aligned}
			& a=1\\ 
			& b=2020\\ 
		\end{aligned}\right.\Rightarrow a-b=-2019$ (Chọn đáp số sai).
	}
\end{ex}
\begin{ex}
	[THPT Chuyên Lê Hồng Phong Nam Định 2019]%Câu 9
	Cho hàm số $y=f(x)$ liên tục trên $\left(0;+\infty\right)$ thỏa mãn $2xf'(x)+f(x)=3x^2\sqrt{x}$. Biết $f(1)=\dfrac{1}{2}$. Tính $f(4)$?
	\choice
	{$ 24$}
	{$ 14$}
	{$ 4$}
	{\True $ 16$}
	\loigiai{
		Trên khoảng $\left(0;+\infty\right)$ ta có $ 2xf'(x)+f(x)=3x^2\sqrt{x}\Leftrightarrow\sqrt{x}f'(x)+\dfrac{1}{2\sqrt{x}}=\dfrac{3}{2}{x^2}$.\\
		$\Rightarrow{\left(\sqrt{x}.f(x)\right)'}=\dfrac{3}{2}{x^2}\Rightarrow\displaystyle\int{\left(\sqrt{x}.f(x)\right)'\mathrm{d}x=\displaystyle\int{\dfrac{3}{2}{x^2}\mathrm{d}x}}$.\\
		$\Rightarrow\sqrt{x}.f(x)=\dfrac{1}{2}{x^3}+C$. $(*)$\\
		Mà $ f(1)=\dfrac{1}{2}$ nên từ $(*)$ có $\sqrt{1}.f(1)=\dfrac{1}{2}{1^3}+C\Leftrightarrow\dfrac{1}{2}=\dfrac{1}{2}+C\Leftrightarrow C=0$$\Rightarrow f(x)=\dfrac{x^2\sqrt{x}}{2}$.\\
		Vậy $ f(4)=\dfrac{4^2\sqrt{4}}{2}=16$.
	}
\end{ex}
\begin{ex}
	[Chuyên Thái Nguyên 2019]%Câu 10
	Cho hàm số $f(x)>0$ với mọi $x\in\mathbb{R}$, $f(0)=1$ và $f(x)=\sqrt{x+1}.f'(x)$ với mọi $x\in\mathbb{R}$. Mệnh đề nào dưới đây đúng?
	\choice
	{$f(x)<2$}
	{$2<f(x)<4$}
	{\True $f(x)>6$}
	{$4<f(x)<6$}
	\loigiai{
		Ta có $\dfrac{f'(x)}{f(x)}=\dfrac{1}{\sqrt{x+1}}\Rightarrow\displaystyle\int{\dfrac{f'(x)}{f(x)}\mathrm{d}x}=\displaystyle\int{\dfrac{1}{\sqrt{x+1}}\mathrm{d}x}\Leftrightarrow\ln\left(f(x)\right)=2\sqrt{x+1}+C$\\
		Mà $f(0)=1$ nên $C=-2\Rightarrow f(x)=\mathrm{e}^{2\sqrt{x+1}-2}\Rightarrow f(3)=e^2>6$.
	}
\end{ex}
\begin{ex}
	[Chuyên Lê Hồng Phong Nam Định 2019]%Câu 11
	Cho hàm số $y=f(x)$ có đạo hàm liên tục trên $\left[2;4\right]$ và $f'(x)>0,\forall x\in\left[2;4\right]$. Biết $4x^3f(x)=\left[f'(x)\right]^3-x^3,\forall x\in\left[2;4\right],\ f(2)=\dfrac{7}{4}$. Giá trị của $f(4)$ bằng
	\choice
	{$\dfrac{40\sqrt{5}-1}{2}$}
	{$\dfrac{20\sqrt{5}-1}{4}$}
	{$\dfrac{20\sqrt{5}-1}{2}$}
	{\True $\dfrac{40\sqrt{5}-1}{4}$}
	\loigiai{
		Ta có $f'(x)>0,\forall x\in\left[2;4\right]$ nên hàm số $ y=f(x)$ đồng biến trên $\left[2;4\right]$ $\Rightarrow\ f(x)\ge f(2)$ mà $ f(2)=\dfrac{7}{4}$. Do đó: $ f(x)>0,\forall x\in\left[2;4\right]$.\\
		Từ giả thiết ta có $ 4x^3f(x)=\left[f'(x)\right]^3-x^3\Leftrightarrow{x^3}\left[4f(x)+1\right]=\left[f'(x)\right]^3$\\
		$\Leftrightarrow x.\sqrt[3]{4f(x)+1}=f'(x)\Leftrightarrow\dfrac{f'(x)}{\sqrt[3]{4f(x)+1}}=x$.\\
		Suy ra $\displaystyle\int{\dfrac{f'(x)}{\sqrt[3]{4f(x)+1}}\mathrm{d}x}=\displaystyle\int{x}\mathrm{d}x\Leftrightarrow\dfrac{1}{4}\displaystyle\int{\dfrac{\mathrm{d}\left[4f(x)+1\right]}{\sqrt[3]{4f(x)+1}}}=\dfrac{x^2}{2}+C$ $\Leftrightarrow\dfrac{3}{8}\sqrt[3]{\left[4f(x)+1\right]^2}=\dfrac{x^2}{2}+C$.\\
		$ f(2)=\dfrac{7}{4}\Leftrightarrow\dfrac{3}{2}=2+C\Leftrightarrow C=-\dfrac{1}{2}$.\\
		Vậy $ f(x)=\dfrac{\sqrt{\left[\dfrac{4}{3}\left(x^2-1\right)\right]^3}-1}{4}$ $\Rightarrow f(4)=\dfrac{40\sqrt{5}-1}{4}$.
	}
\end{ex}
\begin{ex}
	[Chuyên Thái Bình 2019]%Câu 12
	Cho $f(x)$ là hàm số liên tục trên $\mathbb{R}$ thỏa mãn $f(x)+f'(x)=x,\forall x\in\mathbb{R}$ và $f(0)=1$. Tính $f(1)$.
	\choice
	{\True $\dfrac{2}{\mathrm{e}}$}
	{$\dfrac{1}{\mathrm{e}}$}
	{$\mathrm{e}$}
	{$\dfrac{\mathrm{e}}{2}$}
	\loigiai{
		$ f(x)+f'(x)=x(1)$.\\
		Nhân 2 vế của $ (1)$ với $e^x$ ta được $\mathrm{e}^x.f(x)+\mathrm{e}^x.f'(x)=x.\mathrm{e}^x$.\\
		Hay $\left[\mathrm{e}^x.f(x)\right]'=x.\mathrm{e}^x\Rightarrow{\mathrm{e}^x}.f(x)=\displaystyle\int{x.\mathrm{e}^x}\mathrm{d}x$.\\
		Xét $ I=\displaystyle\int{x.\mathrm{e}^x}\mathrm{d}x$.\\
		Đặt $\left\{\begin{aligned}
			&u=x\Rightarrow\text{d}u=\mathrm{d}x\\
			&{\mathrm{e}^x}\mathrm{d}x=\text{d}v\Rightarrow v=\mathrm{e}^x\\
		\end{aligned}\right.$.\\
		$I=\displaystyle\int{x.\mathrm{e}^x}\mathrm{d}x=x.\mathrm{e}^x-\displaystyle\int{\mathrm{e}^x}\mathrm{d}x=x.\mathrm{e}^x-\mathrm{e}^x+C$. Suy ra $\mathrm{e}^xf(x)=x.\mathrm{e}^x-\mathrm{e}^x+C$.\\
		Theo giả thiết $ f(0)=1$ nên $ C=2$ $\Rightarrow f(x)=\dfrac{x.e^x-\mathrm{e}^x+2}{\mathrm{e}^x}\Rightarrow f(1)=\dfrac{2}{\mathrm{e}}$.
	}
\end{ex}
\begin{ex}
	[THPT NGHĨA HƯNG NĐ - GK2 - 2018 - 2019]%Câu 13
	Cho hàm số $f(x)$ thỏa mãn $\left[x{f}'(x)\right]^2+1=x^2\left[1-f(x).f''(x)\right]$ với mọi $x$ dương. Biết $f(1)=f'(1)=1$. Giá trị $f^2(2)$ bằng
	\choice
	{$f^2(2)=\sqrt{2\ln 2+2}$}
	{\True $f^2(2)=2\ln 2+2$}
	{$f^2(2)=\ln 2+1$}
	{$f^2(2)=\sqrt{\ln 2+1}$}
	\loigiai{
		Ta có: $\left[x{f}'(x)\right]^2+1=x^2\left[1-f(x).f''(x)\right];x>0$\\
		$\Leftrightarrow{x^2}.\left[f'(x)\right]^2+1=x^2\left[1-f(x).f''(x)\right]$\\
		$\begin{aligned}
			&\Leftrightarrow{\left[f'(x)\right]^2}+\dfrac{1}{x^2}=1-f(x).f''(x)\\ 
			&\Leftrightarrow{\left[f'(x)\right]^2}+f(x).f''(x)=1-\dfrac{1}{x^2}\\ 
			&\Leftrightarrow{\left[f(x).f'(x)\right]'}=1-\dfrac{1}{x^2}\\ 
		\end{aligned}$\\
		Do đó: $\displaystyle\int{\left[f(x).f'(x)\right]'.\mathrm{d}x=\displaystyle\int{\left(1-\dfrac{1}{x^2}\right)}}.\mathrm{d}x\Rightarrow f(x).f'(x)=x+\dfrac{1}{x}+c_1.$\\
		Vì $ f(1)=f'(1)=1\Rightarrow 1=2+c_1\Leftrightarrow{c_1}=-1$.\\
		Nên $\displaystyle\int{f(x).f'(x).\mathrm{d}x=\displaystyle\int{\left(x+\dfrac{1}{x}-1\right)}}.\mathrm{d}x$ $\Leftrightarrow\displaystyle\int{f(x).\text{d}\left(f(x)\right)=\displaystyle\int{\left(x+\dfrac{1}{x}-1\right)}}.\mathrm{d}x$\\
		$\Rightarrow\dfrac{f^2(x)}{2}=\dfrac{x^2}{2}+\ln x-x+c_2$. Vì $ f(1)=1\Rightarrow\dfrac{1}{2}=\dfrac{1}{2}-1+c_2\Leftrightarrow{c_2}=1.$\\
		Vậy $\dfrac{f^2(x)}{2}=\dfrac{x^2}{2}+\ln x-x+1\Rightarrow{f^2}(2)=2\ln 2+2$.
	}
\end{ex}
\begin{ex}
	[Chuyên Bắc Ninh 2019]%Câu 14
	Cho hàm số $ f(x)$ thỏa mãn $(f'(x))^2+f(x).f''(x)=x^3-2x,\quad\forall x\in R$ và $ f(0)=f'(0)=1$. Tính giá trị của $ T=f^2(2)$
	\choice
	{$\dfrac{43}{30}$}
	{$\dfrac{16}{15}$}
	{\True $\dfrac{43}{15}$}
	{$\dfrac{26}{15}$}
	\loigiai{
		Có $(f'(x))^2+f(x).f''(x)=x^3-2x\Leftrightarrow (f(x).f'(x))'=x^3-2x$\\
		$\Leftrightarrow f(x).f'(x)=\displaystyle\int{(x^3-2x})\mathrm{d}x=\dfrac{1}{4}{x^4}-x^2+C$\\
		Từ $ f(0)=f'(0)=1$. Suy ra $ C=1$. Vậy $ f(x).f'(x)=\dfrac{1}{4}{x^4}-x^2+1$\\
		Tiếp, có $ 2f(x).f'(x)=\dfrac{1}{2}{x^4}-2x^2+2\Leftrightarrow (f^2(x))'=\dfrac{1}{2}{x^4}-2x^2+2$\\
		$\Leftrightarrow{f^2}(x)=\displaystyle\int{(\dfrac{1}{2}{x^4}-2x^2+2})\mathrm{d}x=\dfrac{1}{10}{x^5}-\dfrac{2}{3}{x^3}+2x+C$\\
		Từ $ f(0)=1$. Suy ra $ C=1$. Vậy $f^2(x)=\dfrac{1}{10}{x^5}-\dfrac{2}{3}{x^3}+2x+1$.\\
		Do đó $ T=\dfrac{43}{15}$.
	}
\end{ex}
\begin{ex}
	[Sở Bình Phước 2019]%Câu 15
	Cho hàm số $ f(x)$ liên tục và có đạo hàm trên $\left(0;\dfrac{\pi}{2}\right)$, thỏa mãn $ f(x)+\tan x.f'(x)=\dfrac{x}{\cos^3x}$. Biết rằng $\sqrt{3}f\left(\dfrac{\pi}{3}\right)-f\left(\dfrac{\pi}{6}\right)=a\pi\sqrt{3}+b\ln 3$ trong đó $ a,b\in\mathbb{Q}$. Giá trị của biểu thức $ P=a+b$ bằng
	\choice
	{$\dfrac{14}{9}$}
	{$-\dfrac{2}{9}$}
	{$\dfrac{7}{9}$}
	{\True $-\dfrac{4}{9}$}
	\loigiai{
		$ f(x)+\tan x.f'(x)=\dfrac{x}{\cos^3x}$ $\Leftrightarrow\cos x.f(x)+\sin x.f'(x)=\dfrac{x}{\cos^2x}$.\\
		$\Leftrightarrow{\left[\sin x.f(x)\right]'}=\dfrac{x}{\cos^2x}$.\\
		Do đó $\displaystyle\int{\left[\sin x.f(x)\right]'\mathrm{d}x}=\displaystyle\int{\dfrac{x}{\cos^2x}\mathrm{d}x}$ $\Rightarrow\sin x.f(x)=\displaystyle\int{\dfrac{x}{\cos^2x}\mathrm{d}x}$\\
		Tính $ I=\displaystyle\int{\dfrac{x}{\cos^2x}\mathrm{d}x}$.\\
		Đặt $\left\{\begin{aligned}
			& u=x\\ 
			&\mathrm{d}v=\dfrac{\mathrm{d}x}{\cos^2x}\\ 
		\end{aligned}\right.\Rightarrow\left\{\begin{aligned}
			&\text{d}u=\mathrm{d}x\\ 
			& v=\tan x\\ 
		\end{aligned}\right.$. Khi đó\\
		$ I=\displaystyle\int{\dfrac{x}{\cos^2x}\mathrm{d}x}=x\tan x-\displaystyle\int{\tan x\mathrm{d}x}=x\tan x+\displaystyle\int{\dfrac{\text{d}\left(\cos x\right)}{\cos x}\mathrm{d}x}=x\tan x+\ln\left|\cos x\right|$.\\
		Suy ra $ f(x)=\dfrac{x.\tan x+\ln\left|\cos x\right|}{\sin x}=\dfrac{x}{\cos x}+\dfrac{\ln\left|\cos x\right|}{\sin x}$.\\
		$ a\pi\sqrt{3}+b\ln 3=\sqrt{3}f\left(\dfrac{\pi}{3}\right)-f\left(\dfrac{\pi}{6}\right)=\sqrt{3}\left(\dfrac{2\pi}{3}-\dfrac{2\ln 2}{\sqrt{3}}\right)-\left(\dfrac{\pi\sqrt{3}}{9}+2\ln\dfrac{\sqrt{3}}{2}\right)$\\
		$=\dfrac{5\pi\sqrt{3}}{9}-\ln 3$. Suy ra $\left\{\begin{aligned}
			& a=\dfrac{5}{9}\\ 
			& b=-1\\ 
		\end{aligned}\right.$.\\
		Vậy $ P=a+b=-\dfrac{4}{9}$.
	}
\end{ex}
\begin{ex}
	[THPT Yên Phong Số 1 Bắc Ninh 2019]%Câu 16
	Cho hàm số $ y=f(x)$ đồng biến trên $\left(0;+\infty\right)$; $ y=f(x)$ liên tục, nhận giá trị dương trên $\left(0;+\infty\right)$ và thỏa mãn $ f(3)=\dfrac{4}{9}$ và $\left[f'(x)\right]^2=\left(x+1\right).f(x)$. Tính $ f(8)$.
	\choice
	{\True $ f(8)=49$}
	{$ f(8)=256$}
	{$ f(8)=\dfrac{1}{16}$}
	{$ f(8)=\dfrac{49}{64}$}
	\loigiai{
		Ta có với $\forall x\in\left(0;+\infty\right)$ thì $ y=f(x)>0$; $ x+1>0$.\\
		Hàm số $ y=f(x)$ đồng biến trên $\left(0;+\infty\right)$ nên $f'(x)\ge 0,\forall x\in\left(0;+\infty\right)$.\\
		Do đó $\left[f'(x)\right]^2=\left(x+1\right)f(x)\Leftrightarrow{f}'(x)=\sqrt{\left(x+1\right)f(x)}$$\Leftrightarrow\dfrac{f'(x)}{\sqrt{f(x)}}=\sqrt{\left(x+1\right)}$.\\
		Suy ra$\displaystyle\int{\dfrac{f'(x)}{\sqrt{f(x)}}\mathrm{d}x}=\displaystyle\int{\sqrt{\left(x+1\right)}\mathrm{d}x}$ $\Rightarrow\sqrt{f(x)}=\dfrac{1}{3}\sqrt{\left(x+1\right)^3}+C$.\\
		Vì $ f(3)=\dfrac{4}{9}$ nên $ C=\dfrac{2}{3}-\dfrac{8}{3}=-2$.\\
		Suy ra $ f(x)=\left(\dfrac{1}{3}\sqrt{\left(x+1\right)^3}-2\right)^2$, suy ra $ f(8)=49$.
	}
\end{ex}
\begin{ex}
	Cho hàm số $f(x)$ thỏa mãn $f(1)=2$ và $\left(x^2+1\right)^2f'(x)=\left[f(x)\right]^2\left(x^2-1\right)$ với mọi $x\in\mathbb{R}$. Giá trị của $f(2)$ bằng
	\choice
	{$\dfrac{2}{5}$}
	{$-\dfrac{2}{5}$}
	{$-\dfrac{5}{2}$}
	{\True $\dfrac{5}{2}$}
	\loigiai{
		Từ giả thiết ta có $f'(x)=\left[f(x)\right]^2.\dfrac{x^2-1}{\left(x^2+1\right)^2}>0$ với mọi $ x\in\left(1;2\right]$.\\
		Do đó $ f(x)\ge f(1)=1>0$ với mọi $ x\in\left[1;2\right]$.\\
		Xét với mọi $ x\in\left[1;2\right]$ ta có\\
		$\left(x^2+1\right){f}'(x)=\left[f(x)\right]^2\left(x^2-1\right)\Leftrightarrow\dfrac{f'(x)}{f^2(x)}=\dfrac{x^2-1}{\left(x^2+1\right)^2}\Rightarrow\displaystyle\int{\dfrac{f'(x)}{f^2(x)}\mathrm{d}x}=\displaystyle\int{\dfrac{x^2-1}{\left(x^2+1\right)^2}\mathrm{d}x}$.\\
		$\Rightarrow\displaystyle\int{\dfrac{f'(x)}{f^2(x)}\mathrm{d}x=\displaystyle\int{\dfrac{1-\dfrac{1}{x^2}}{\left(x+\dfrac{1}{x}\right)^2}\mathrm{d}x}}$$\Rightarrow\displaystyle\int{\dfrac{f'(x)}{f^2(x)}\mathrm{d}x=\displaystyle\int{\dfrac{\text{d}\left(x+\dfrac{1}{x}\right)}{\left(x+\dfrac{1}{x}\right)^2}}}$$\Rightarrow-\dfrac{1}{f(x)}=-\dfrac{1}{x+\dfrac{1}{x}}+C$.\\
		Mà $ f(1)=1\Rightarrow 1=1+C\Leftrightarrow C=0$. Vậy $ f(x)=\dfrac{x^2+1}{x}$$\Rightarrow f(2)=\dfrac{5}{2}$.
	}
\end{ex}
\begin{ex}
	[Chuyên Nguyễn Tất Thành Yên Bái 2019]%Câu 18
	Cho hàm số $ y=f(x)$ có đạo hàm liên tục trên khoảng $\left(0;+\infty\right)$, biết $f'(x)+\left(2x+1\right){f^2}(x)=0$, $ f(x)>0,\forall x>0$ và $ f(2)=\dfrac{1}{6}$. Tính giá trị của $ P=f(1)+f(2)+\ldots+f\left(2019\right)$.
	\choice
	{$\dfrac{2021}{2020}$}
	{$\dfrac{2020}{2019}$}
	{\True $\dfrac{2019}{2020}$}
	{$\dfrac{2018}{2019}$}
	\loigiai{
		TH1: $ f(x)=0$$\Rightarrow{f}'(x)=0$ trái giả thiết.\\
		TH2: $ f(x)\ne 0$ $\Rightarrow{f}'(x)=-\left(2x+1\right).f^2(x)$ $\Rightarrow\dfrac{f'(x)}{f^2(x)}=-\left(2x+1\right)$. $\Rightarrow\displaystyle\int{\dfrac{f'(x)}{f^2(x)}}\mathrm{d}x=-\displaystyle\int{\left(2x+1\right)}\mathrm{d}x$ $\Rightarrow\dfrac{-1}{f(x)}=-\left(x^2+x+C\right)$ .\\
		Ta có $ f(2)=\dfrac{1}{6}$ $\Rightarrow C=0$ $\Rightarrow f(x)=\dfrac{1}{x^2+x}=\dfrac{1}{x}-\dfrac{1}{x+1}$.\\
		$\Rightarrow P=\dfrac{1}{1}-\dfrac{1}{2}+\dfrac{1}{2}-\dfrac{1}{3}+\ldots-\dfrac{1}{2020}=\dfrac{2019}{2020}$.
	}
\end{ex}
\begin{ex}
	Cho hàm số $y=f(x)$ có đạo hàm liên tục trên đoạn $\left[-2;1\right]$ thỏa mãn $f(0)=3$ và $\left(f(x)\right)^2.f'(x)=3x^2+4x+2$. Giá trị lớn nhất của hàm số $y=f(x)$ trên đoạn $\left[-2;1\right]$ là
	\choice
	{$ 2\sqrt[3]{42}$}
	{$ 2\sqrt[3]{15}$}
	{\True $\sqrt[3]{42}$}
	{$\sqrt[3]{15}$}
	\loigiai{
		Ta có:$\left(f(x)\right)^2.f'(x)=3x^2+4x+2$ (*)\\
		Lấy nguyên hàm 2 vế của phương trình trên ta được\\
		$\displaystyle\int{\left(f(x)\right)^2.f'(x)}\mathrm{d}x=\displaystyle\int{\left(3x^2+4x+2\right)}\mathrm{d}x\Leftrightarrow\displaystyle\int{\left(f(x)\right)^2d\left(f(x)\right)=}x^3+2x^2+2x+C$\\
		$\Leftrightarrow\dfrac{\left(f(x)\right)^3}{3}=x^3+2x^2+2x+C\Leftrightarrow{\left(f(x)\right)^3}=3\left(x^3+2x^2+2x+C\right)\text(1)$\\
		Theo đề bài $ f(0)=3$ nên từ (1) ta có $\left(f(0)\right)^3=3\left(0^3+2.0^2+2.0+C\right)\Leftrightarrow 27=3C\Leftrightarrow C=9$\\
		$\Rightarrow{\left(f(x)\right)^3}=3\left(x^3+2x^2+2x+9\right)\Rightarrow f(x)=\sqrt[3]{3\left(x^3+2x^2+2x+9\right)}.$\\
		Tiếp theo chúng ta tìm giá trị lớn nhất của hàm số $ y=f(x)$ trên đoạn $\left[-2;1\right].$\\
		CÁCH 1:\\
		Vì $x^3+2x^2+2x+9=x^2\left(x+2\right)+2\left(x+2\right)+5>0,\forall x\in\left[-2;1\right]$ nên $ f(x)$ có đạo hàm trên $\left[-2;1\right]$ và $f'(x)=\dfrac{3\left(3x^2+4x+2\right)}{3\sqrt[3]{\left[3\left(x^3+2x^2+2x+9\right)\right]^2}}=\dfrac{3x^2+4x+2}{\sqrt[3]{\left[3\left(x^3+2x^2+2x+9\right)\right]^2}}>0,$ $\forall x\in\left[-2;1\right].$\\
		$\Rightarrow$Hàm số $ y=f(x)$đồng biến trên $\left[-2;1\right]\Rightarrow\underset{\left[-2;1\right]}{\max}f(x)=f(1)=\sqrt[3]{42}.$\\
		Vậy $\underset{\left[-2;1\right]}{\max}f(x)=f(1)=\sqrt[3]{42}$.\\
		CÁCH 2:\\
		$f(x)=\sqrt[3]{3\left(x^3+2x^2+2x+9\right)}=\sqrt[3]{3\left(x+\dfrac{2}{3}\right)^3+2\left(x+\dfrac{2}{3}\right)+\dfrac{223}{9}}.$\\
		Vì các hàm số $ y=3\left(x+\dfrac{2}{3}\right)^3,y=2\left(x+\dfrac{2}{3}\right)+\dfrac{223}{9}$ đồng biến trên $\mathbb{R}$ nên hàm số\\
		$y=\sqrt[3]{3\left(x+\dfrac{2}{3}\right)^3+2\left(x+\dfrac{2}{3}\right)+\dfrac{223}{9}}$ cũng đồng biến trên $\mathbb{R}.$ Do đó, hàm số $ y=f(x)$đồng biến trên $\left[-2;1\right].$\\
		Vậy $\underset{\left[-2;1\right]}{\max}f(x)=f(1)=\sqrt[3]{42}$.
	}
\end{ex}
\begin{ex}
	[Đề Thi Công Bằng KHTN 2019]%Câu 20
	Cho hàm số $ f(x)$ thỏa mãn $ f(1)=4$ và $ f(x)=x{f}'(x)-2x^3-3x^2$ với mọi $ x>0$. Giá trị của $ f(2)$ bằng
	\choice
	{$ 5$}
	{$ 10$}
	{\True $ 20$}
	{$ 15$}
	\loigiai{
		$f(x)-x{f}'(x)=-2x^3-3x^2\Leftrightarrow\dfrac{1.f(x)-x.f'(x)}{x^2}=\dfrac{-2x^3-3x^2}{x^2}\Leftrightarrow{\left[\dfrac{f(x)}{x}\right]'}=2x+3$\\
		Suy ra, $\dfrac{f(x)}{x}$ là một nguyên hàm của hàm số $ g(x)=2x+3$.\\
		Ta có $\displaystyle\int{\left(2x+3\right)}\mathrm{d}x=x^2+3x+C,$$ C\in\mathbb{R}$.\\
		Do đó, $\dfrac{f(x)}{x}=x^2+3x+C_1,$ (1) với $C_1\in\mathbb{R}$ nào đó.\\
		Vì $ f(1)=4$ theo giả thiết, nên thay $ x=1$ vào hai vế của (1) ta thu được $C_1=0$, từ đó $ f(x)=x^3+3x^2$. Vậy $ f(2)=20$.
	}
\end{ex}
\begin{ex}
	[Sở Bắc Ninh 2019]%Câu 21
	Cho hàm số $ f(x)$ liên tục trên $ R$ thỏa mãn các điều kiện: $ f(0)=2\sqrt{2},$ $ f(x)>0,$ $\forall x\in\mathbb{R}$ và $ f(x).f'(x)=\left(2x+1\right)\sqrt{1+f^2(x)},$ $\forall x\in\mathbb{R}$. Khi đó giá trị $ f(1)$ bằng
	\choice
	{$\sqrt{26}$}
	{\True $\sqrt{24}$}
	{$\sqrt{15}$}
	{$\sqrt{23}$}
	\loigiai{
		Ta có $ f(x).f'(x)=\left(2x+1\right)\sqrt{1+f^2(x)}\Leftrightarrow\dfrac{f(x).f'(x)}{\sqrt{1+f^2(x)}}=\left(2x+1\right)$.\\
		Suy ra $\displaystyle\int{\dfrac{f(x).f'(x)}{\sqrt{1+f^2(x)}}\mathrm{d}x}=\displaystyle\int{\left(2x+1\right)}\mathrm{d}x\Leftrightarrow\displaystyle\int{\dfrac{\text{d}\left(1+f^2(x)\right)}{2\sqrt{1+f^2(x)}}}=\displaystyle\int{\left(2x+1\right)}\mathrm{d}x\Leftrightarrow\sqrt{1+f^2(x)}=x^2+x+C$.\\
		Theo giả thiết $ f(0)=2\sqrt{2}$, suy ra $\sqrt{1+\left(2\sqrt{2}\right)^2}=C\Leftrightarrow C=3$.\\
		Với $ C=3$ thì $\sqrt{1+f^2(x)}=x^2+x+3\Rightarrow f(x)=\sqrt{\left(x^2+x+3\right)^2-1}$. Vậy $ f(1)=\sqrt{24}$.
	}
\end{ex}
\begin{ex}
	[Cần Thơ 2018]%Câu 22
	Cho hàm số $ f(x)$ thỏa mãn $\left[f'(x)\right]^2+f(x).f''(x)=2x^2-x+1$, $\forall x\in\mathbb{R}$ và $ f(0)=f'(0)=3$. Giá trị của $\left[f(1)\right]^2$ bằng
	\choice
	{\True $ 28$}
	{$ 22$}
	{$\dfrac{19}{2}$}
	{$ 10$}
	\loigiai{
		Ta có $\left[f(x){f}'(x)\right]'=\left[f'(x)\right]^2+f(x){f'}'(x)$.\\
		Do đó theo giả thiết ta được $\left[f(x){f}'(x)\right]'=2x^2-x+1$.\\
		Suy ra $f(x){f}'(x)=\dfrac{2}{3}{x^3}-\dfrac{x^2}{2}+x+C$. Hơn nữa $ f(0)=f'(0)=3$ suy ra $ C=9$.\\
		Tương tự vì $\left[f^2(x)\right]'=2f(x){f}'(x)$ nên $\left[f^2(x)\right]'=2\left(\dfrac{2}{3}{x^3}-\dfrac{x^2}{2}+x+9\right)$. Suy ra\\
		$f^2(x)=\displaystyle\int{2\left(\dfrac{2}{3}{x^3}-\dfrac{x^2}{2}+x+9\right)\mathrm{d}x}=\dfrac{1}{3}{x^4}-\dfrac{x^3}{3}+x^2+18x+C$, cũng vì $ f(0)=3$ suy ra\\
		$f^2(x)=\dfrac{1}{3}{x^4}-\dfrac{x^3}{3}+x^2+18x+9$. Do đó $\left[f(1)\right]^2=28$.
	}
\end{ex}
\begin{ex}
	[Chuyên Lê Hồng Phong - 2018]%Câu 23
	Cho hàm số $ f(x)$ có đạo hàm trên $\mathbb{R}$ thỏa mãn $\left(x+2\right)f(x)+\left(x+1\right){f}'(x)=\mathrm{e}^x$ và $ f(0)=\dfrac{1}{2}$. Tính $ f(2)$.
	\choice
	{$ f(2)=\dfrac{\mathrm{e}}{3}$}
	{$ f(2)=\dfrac{\mathrm{e}}{6}$}
	{$ f(2)=\dfrac{\mathrm{e}^2}{3}$}
	{\True $ f(2)=\dfrac{\mathrm{e}^2}{6}$}
	\loigiai{
		Ta có\\
		$\left(x+2\right)f(x)+\left(x+1\right){f}'(x)=\mathrm{e}^x\Leftrightarrow\left(x+1\right)f(x)+f(x)+\left(x+1\right){f}'(x)=\mathrm{e}^x\Leftrightarrow\left[\left(x+1\right)f(x)\right]+\left[\left(x+1\right)f(x)\right]'=\mathrm{e}^x\Leftrightarrow{\mathrm{e}^x}\left[\left(x+1\right)f(x)\right]+\mathrm{e}^x{\left[\left(x+1\right)f(x)\right]'}=\mathrm{e}^{2x}\Leftrightarrow{\left[\mathrm{e}^x\left(x+1\right)f(x)\right]'}=\mathrm{e}^{2x}\Rightarrow\displaystyle\int{\left[\mathrm{e}^x\left(x+1\right)f(x)\right]'}\mathrm{d}x=\displaystyle\int{\mathrm{e}^{2x}\mathrm{d}x}\Leftrightarrow{\mathrm{e}^x}\left(x+1\right)f(x)=\dfrac{1}{2}{\mathrm{e}^{2x}}+C$\\
		Mà $ f(0)=\dfrac{1}{2}$ $\Rightarrow C=0$. Vậy $ f(x)=\dfrac{1}{2}.\dfrac{\mathrm{e}^x}{x+1}$\\
		Khi đó $ f(2)=\dfrac{\mathrm{e}^2}{6}$.
	}
\end{ex}
\begin{ex}
	[Liên Trường - Nghệ An - 2018]%Câu 24
	Cho hàm số $ y=f(x)$ liên tục trên $\mathbb{R}\setminus\left\{ 0;-1\right\}$ thỏa mãn điều kiện $ f(1)=-2\ln 2$ và $ x\left(x+1\right).f'(x)+f(x)=x^2+x$. Giá trị $ f(2)=a+b\ln 3$, với$ a,b\in\mathbb{Q}$. Tính $a^2+b^2$.
	\choice
	{$\dfrac{25}{4}$}
	{\True $\dfrac{9}{2}$}
	{$\dfrac{5}{2}$}
	{$\dfrac{13}{4}$}
	\loigiai{
		Từ giả thiết, ta có $ x\left(x+1\right).f'(x)+f(x)=x^2+x\Leftrightarrow $ $\dfrac{x}{x+1}.f'(x)+\dfrac{1}{\left(x+1\right)^2}f(x)=\dfrac{x}{x+1}$\\
		$\Leftrightarrow{\left[\dfrac{x}{x+1}.f(x)\right]'}=\dfrac{x}{x+1}$, với $\forall x\in\mathbb{R}\setminus\left\{ 0;-1\right\}$.\\
		Suy ra $\dfrac{x}{x+1}.f(x)=\displaystyle\int{\dfrac{x}{x+1}}\mathrm{d}x$ hay $\dfrac{x}{x+1}.f(x)=x-\ln\left| x+1\right|+C$.\\
		Mặt khác, ta có $ f(1)=-2\ln 2$ nên $ C=-1$. Do đó $\dfrac{x}{x+1}.f(x)=x-\ln\left| x+1\right|-1$.\\
		Với $ x=2$ thì $\dfrac{2}{3}.f(2)=1-\ln 3\Leftrightarrow  f(2)=\dfrac{3}{2}-\dfrac{3}{2}\ln 3$. Suy ra $ a=\dfrac{3}{2}$ và $ b=-\dfrac{3}{2}$.\\
		Vậy $a^2+b^2=\dfrac{9}{2}$.
	}
\end{ex}
\begin{ex}
	[THPT Lê Xoay - 2018]%Câu 25
	Giả sử hàm số $ y=f(x)$ liên tục, nhận giá trị dương trên $\left(0;+\infty\right)$ và thỏa mãn $ f(1)=1$, $ f(x)=f'(x).\sqrt{3x+1}$, với mọi $ x>0$. Mệnh đề nào sau đây đúng?
	\choice
	{$ 2<f(5)<3$}
	{$ 1<f(5)<2$}
	{$ 4<f(5)<5$}
	{\True $ 3<f(5)<4$}
	\loigiai{
		Ta có $f(x)=f'(x).\sqrt{3x+1}$ $\Leftrightarrow\dfrac{f'(x)}{f(x)}=\dfrac{1}{\sqrt{3x+1}}$ $\Rightarrow\displaystyle\int{\dfrac{f'(x)}{f(x)}\mathrm{d}x}=\displaystyle\int{\dfrac{1}{\sqrt{3x+1}}\mathrm{d}x}$ $\Leftrightarrow\displaystyle\int{\dfrac{\text{d}\left(f(x)\right)}{f(x)}}=\displaystyle\int{\dfrac{1}{\sqrt{3x+1}}\mathrm{d}x}$ $\Leftrightarrow\ln f(x)=\dfrac{2}{3}\sqrt{3x+1}+C$ $\Leftrightarrow f(x)=e^{\frac{2}{3}\sqrt{3x+1}+C}$\\
		Mà $ f(1)=1$ nên $e^{\frac{4}{3}+C}=1$ $\Leftrightarrow C=-\dfrac{4}{3}$. Suy ra $ f(5)=e^{\frac{4}{3}}\approx 3,794$.
	}
\end{ex}
\begin{ex}
	[THPT Quỳnh Lưu - Nghệ An - 2018]%Câu 26
	Cho hàm số $f(x)\ne 0$ thỏa mãn điều kiện $f'(x)=\left(2x+3\right){f^2}(x)$ và $ f(0)=-\dfrac{1}{2}$. Biết rằng tổng $ f(1)+f(2)+f(3)+\ldots+f\left(2017\right)+f\left(2018\right)=\dfrac{a}{b}$ với $\left(a\in\mathbb{Z},b\in{\mathbb{N}^*}\right)$ và $\dfrac{a}{b}$ là phân số tối giản. Mệnh đề nào sau đây đúng?
	\choice
	{$\dfrac{a}{b}<-1$}
	{$\dfrac{a}{b}>1$}
	{$a+b=1010$}
	{\True $b-a=3029$}
	\loigiai{
		Ta có $f'(x)=\left(2x+3\right){f^2}(x)$ $\Leftrightarrow\dfrac{f'(x)}{f^2(x)}=2x+3$\\
		$\Leftrightarrow\displaystyle\int{\dfrac{f'(x)}{f(x)}}\mathrm{d}x=\displaystyle\int{\left(2x+3\right)\mathrm{d}x}\Leftrightarrow-\dfrac{1}{f(x)}=x^2+3x+C$.\\
		Vì $ f(0)=-\dfrac{1}{2}\Rightarrow C=2$.\\
		Vậy $ f(x)=-\dfrac{1}{\left(x+1\right)\left(x+2\right)}=\dfrac{1}{x+2}-\dfrac{1}{x+1}$.\\
		Do đó $ f(1)+f(2)+f(3)+\ldots+f\left(2017\right)+f\left(2018\right)=\dfrac{1}{2020}-\dfrac{1}{2}=-\dfrac{1009}{2020}$.\\
		Vậy $ a=-1009$; $ b=2020$. Do đó $ b-a=3029$.
	}
\end{ex}
\begin{ex}
	[THPT Nam Trực - Nam Định - 2018]%Câu 27
	Cho hàm số $f(x)\ne 0$, $f'(x)=\dfrac{3x^4+x^2-1}{x^2}{f^2}(x)$ và $f(1)=-\dfrac{1}{3}$. Tính $f(1)+f(2)+...+f\left(80\right)$.
	\choice
	{\True $-\dfrac{3240}{6481}$}
	{$\dfrac{6480}{6481}$}
	{$-\dfrac{6480}{6481}$}
	{$\dfrac{3240}{6481}$}
	\loigiai{
		$f'(x)=\dfrac{3x^4+x^2-1}{x^2}{f^2}(x)$ $\Leftrightarrow $ $\dfrac{f'(x)}{f^2(x)}=\dfrac{3x^4+x^2-1}{x^2}$.\\
		$\displaystyle\int{\dfrac{f'(x)}{f^2(x)}\mathrm{d}x=\displaystyle\int{\dfrac{3x^4+x^2-1}{x^2}\mathrm{d}x}}$ $\Leftrightarrow $ $\displaystyle\int{\dfrac{\text{d}\left(f(x)\right)}{f^2(x)}=\displaystyle\int{\dfrac{3x^4+x^2-1}{x^2}\mathrm{d}x}}$. $\Leftrightarrow $ $\displaystyle\int{\dfrac{\text{d}\left(f(x)\right)}{f^2(x)}=\displaystyle\int{\left(3x^2+1-\dfrac{1}{x^2}\right)\mathrm{d}x}}$ $\Leftrightarrow $ $\dfrac{-1}{f(x)}=x^3+x+\dfrac{1}{x}+C$ $\Leftrightarrow $ $f(x)=\dfrac{-1}{x^3+x+\dfrac{1}{x}}+C$.\\
		Do $f(1)=-\dfrac{1}{3}$ $\Rightarrow C=0$ $\Rightarrow f(x)=\dfrac{-x}{x^4+x^2+1}$=$\dfrac{1}{2}\left(\dfrac{1}{x^2+x+1}-\dfrac{1}{x^2-x+1}\right)$.\\
		$f(1)=\dfrac{1}{2}\left(\dfrac{1}{3}-\dfrac{1}{1}\right)$; $f(2)=\dfrac{1}{2}\left(\dfrac{1}{7}-\dfrac{1}{3}\right)$; $f(3)=\dfrac{1}{2}\left(\dfrac{1}{13}-\dfrac{1}{7}\right);\ldots;f\left(80\right)=\dfrac{1}{2}\left(\dfrac{1}{6481}-\dfrac{1}{6321}\right)$.\\
		$f(1)+f(2)+\ldots+f\left(80\right)=$ $-\dfrac{1}{2}+\dfrac{1}{2}.\dfrac{1}{6481}$=$-\dfrac{3240}{6481}$.
	}
\end{ex}
\begin{ex}
	[Sở Hà Tĩnh - 2018]%Câu 28
	Cho hàm số $ f(x)$ đồng biến có đạo hàm đến cấp hai trên đoạn $\left[0;2\right]$ và thỏa mãn $\left[f(x)\right]^2-f(x).f''(x)+\left[f'(x)\right]^2=0$. Biết $ f(0)=1$,$ f(2)=\mathrm{e}^6$. Khi đó $ f(1)$ bằng
	\choice
	{$\mathrm{e}^{\frac{3}{2}}$}
	{$\mathrm{e}^3$}
	{\True $\mathrm{e}^{\frac{5}{2}}$}
	{$\mathrm{e}^2$}
	\loigiai{
		Theo đề bài, ta có $\left[f(x)\right]^2-f(x).f''(x)+\left[f'(x)\right]^2=0\Rightarrow\dfrac{f(x).f''(x)-\left[f'(x)\right]^2}{\left[f(x)\right]^2}=1$\\
		$\Rightarrow{\left[\dfrac{f'(x)}{f(x)}\right]'}=1\Rightarrow\dfrac{f'(x)}{f(x)}=x+C\Rightarrow\ln f(x)=\dfrac{x^2}{2}+C.x+D$\\
		Mà $\left\{\begin{aligned}
			& f(0)=1\\ 
			& f(2)=\mathrm{e}^6\\ 
		\end{aligned}\right.\Leftrightarrow\left\{\begin{aligned}
			& C=2\\ 
			& D=0\\ 
		\end{aligned}\right.$. Suy ra $f(x)=\mathrm{e}^{\frac{x^2}{2}+2x}\Rightarrow f(1)=\mathrm{e}^{\frac{5}{2}}$.
	}
\end{ex}
\begin{ex}
	Cho hàm số $ y=f(x)$ liên tục trên $\mathbb{R}$ thỏa mãn $f'(x)+2x.f(x)=\mathrm{e}^{-x^2}$, $\forall x\in\mathbb{R}$ và $ f(0)=0$. Tính $ f(1)$.
	\choice
	{$ f(1)=\mathrm{e}^2$}
	{$ f(1)=-\dfrac{1}{\mathrm{e}}$}
	{$ f(1)=\dfrac{1}{\mathrm{e}^2}$}
	{\True $ f(1)=\dfrac{1}{\mathrm{e}}$}
	\loigiai{
		Ta có\\
		$f'(x)+2x.f(x)=\mathrm{e}^{-x^2}\Leftrightarrow{\mathrm{e}^{x^2}}{f}'(x)+2x.\mathrm{e}^{x^2}.f(x)=1\Leftrightarrow{\left(\mathrm{e}^{x^2}.f(x)\right)'}=1$.\\
		Suy ra $\displaystyle\int{\left(\mathrm{e}^{x^2}.f(x)\right)'\mathrm{d}x=\displaystyle\int{\mathrm{d}x}\Leftrightarrow}\mathrm{e}^{x^2}.f(x)=x+C\Rightarrow f(x)=\dfrac{x+C}{\mathrm{e}^{x^2}}$.\\
		Vì $ f(0)=0\Rightarrow C=0$.\\
		Do đó $ f(x)=\dfrac{x}{\mathrm{e}^{x^2}}$. Vậy $ f(1)=\dfrac{1}{\mathrm{e}}$.
	}
\end{ex}
\begin{ex}
	Cho hàm số $ y=f(x)$ thỏa mãn $ f'(x).f(x)=x^4+x^2$. Biết $ f(0)=2$. Tính $f^2(2)$.
	\choice
	{$f^2(2)=\dfrac{313}{15}$}
	{\True $f^2(2)=\dfrac{332}{15}$}
	{$f^2(2)=\dfrac{324}{15}$}
	{$f^2(2)=\dfrac{323}{15}$}
	\loigiai{
		Ta có $\displaystyle\int{f'(x).f(x)\mathrm{d}x}=\displaystyle\int{\left(x^4+x^2\right)\mathrm{d}x}+C\Rightarrow\dfrac{f^2(x)}{2}=\dfrac{x^5}{5}+\dfrac{x^3}{3}+C$.\\
		Do $ f(0)=2$ nên suy ra $ C=2$.\\
		Vậy $f^2(2)=2\left(\dfrac{32}{5}+\dfrac{8}{3}+2\right)=\dfrac{332}{15}$.
	}
\end{ex}
\begin{ex}
	[Chuyên Đại học Vinh - 2019]%Câu 31
	Cho hàm số $f(x)$ thỏa mãn $f(x)+f'(x)=\mathrm{e}^{-x},\forall x\in\mathbb{R}$ và $f(0)=2$. Tất cả các nguyên hàm của $f(x){\mathrm{e}^{2x}}$ là
	\choice
	{$\left(x-2\right){\mathrm{e}^x}+\mathrm{e}^x+C$}
	{$\left(x+2\right){\mathrm{e}^{2x}}+\mathrm{e}^x+C$}
	{$\left(x-1\right){\mathrm{e}^x}+C$}
	{\True $\left(x+1\right){\mathrm{e}^x}+C$}
	\loigiai{
		$f(x)+f'(x)=\mathrm{e}^{-x}\Leftrightarrow f(x){\mathrm{e}^x}+f'(x){\mathrm{e}^x}=1\Leftrightarrow{\left(f(x){\mathrm{e}^x}\right)'}=1\Leftrightarrow f(x){\mathrm{e}^x}=x+C'$.\\
		Vì $f(0)=2$ nên $C'=2$. Do đó $f(x){\mathrm{e}^{2x}}=\left(x+2\right){\mathrm{e}^x}$. Vậy\\
		$\displaystyle\int{f(x){\mathrm{e}^{2x}}}\mathrm{d}x=\displaystyle\int{\left(x+2\right){\mathrm{e}^x}\mathrm{d}x}=\displaystyle\int{\left(x+2\right)\text{d}\left(\mathrm{e}^x\right)}=\left(x+2\right){\mathrm{e}^x}-\displaystyle\int{\mathrm{e}^x\text{d}\left(x+2\right)}=\left(x+2\right){\mathrm{e}^x}-\displaystyle\int{\mathrm{e}^x\mathrm{d}x}=$\\
		$=\left(x+2\right){\mathrm{e}^x}-\mathrm{e}^x+C=\left(x+1\right){\mathrm{e}^x}+C$.
	}
\end{ex}
\begin{ex}
	Cho hàm số $ y=f(x)$ có đạo hàm trên $\left(0;+\infty\right)$ thỏa mãn $ 2x{f}'(x)+f(x)=2x$ $\forall x\in\left(0;+\infty\right)$, $ f(1)=1$. Giá trị của biểu thức $ f(4)$ là:
	\choice
	{$\dfrac{25}{6}$}
	{$\dfrac{25}{3}$}
	{\True $\dfrac{17}{6}$}
	{$\dfrac{17}{3}$}
	\loigiai{
		Xét phương trình $ 2x{f}'(x)+f(x)=2x$ $(1)$ trên $\left(0;+\infty\right)$: $(1)\Leftrightarrow{f}'(x)+\dfrac{1}{2x}\cdot f(x)=1$ $(2)$.\\
		Đặt $ g(x)=\dfrac{1}{2x}$, ta tìm một nguyên hàm $ G(x)$ của $ g(x)$.\\
		Ta có $\displaystyle\int{g(x)\mathrm{d}x}=\displaystyle\int{\dfrac{1}{2x}\mathrm{d}x}=\dfrac{1}{2}\ln x+C=\ln\sqrt{x}+C$. Ta chọn $G(x)=\ln\sqrt{x}$ .\\
		Nhân cả 2 vế của $(2)$ cho $\mathrm{e}^{G(x)}=\sqrt{x}$, ta được $\sqrt{x}\cdot{f}'(x)+\dfrac{1}{2\sqrt{x}}\cdot f(x)=\sqrt{x}$\\
		$\Leftrightarrow{\left(\sqrt{x}.f(x)\right)'}=\sqrt{x}$ $(3)$.\\
		Lấy tích phân 2 vế của $(3)$ từ $ 1$ đến 4, ta được: $\displaystyle\int\limits_1^4\left(\sqrt{x}.f(x)\right)'\mathrm{d}x=\displaystyle\int\limits_1^4\sqrt{x}\mathrm{d}x$\\
		$\Rightarrow\left.\left(\sqrt{x}.f(x)\right)\right|_1^4=\left.\left(\dfrac{2}{3}\sqrt{x^3}\right)\right|_1^4\Rightarrow 2f(4)-f(1)=\dfrac{14}{3}\Rightarrow f(4)=\dfrac{1}{2}\left(\dfrac{14}{3}+1\right)=\dfrac{17}{6}$ (vì $ f(1)=1$).\\
		Vậy $ f(4)=\dfrac{17}{6}$.
	}
\end{ex}
\begin{ex}
	[Chu Văn An - Hà Nội - 2019]%Câu 33
	Cho hàm số $ y=f(x)$ có đạo hàm liên tục trên $\mathbb{R}$ và thỏa mãn điều kiện $x^6\left[f'(x)\right]^3+27\left[f(x)-1\right]^4=0,\forall x\in\mathbb{R}$ và $ f(1)=0$. Giá trị của $ f(2)$ bằng
	\choice
	{$-1$}
	{$ 1$}
	{$ 7$}
	{\True $-7$}
	\loigiai{
		Ta có $x^6\left[f'(x)\right]^3+27\left[f(x)-1\right]^4=0\Leftrightarrow\dfrac{f'(x)}{-3\left(f(x)-1\right)\sqrt[3]{f(x)-1}}=\dfrac{1}{x^2}\Leftrightarrow{\left[\dfrac{1}{\sqrt[3]{f(x)-1}}\right]'}=\dfrac{1}{x^2}.$\\
		Do đó $\displaystyle\int{\left[\dfrac{1}{\sqrt[3]{f(x)-1}}\right]'}\mathrm{d}x=\displaystyle\int{\dfrac{1}{x^2}\mathrm{d}x}=-\dfrac{1}{x}+C.$ Suy ra $\dfrac{1}{\sqrt[3]{f(x)-1}}=-\dfrac{1}{x}+C$.\\
		Có $ f(1)=0\Rightarrow C=0$. Do đó $ f(x)=1-x^3$.\\
		Khi đó $ f(2)=-7$.
	}
\end{ex}
\begin{ex}[Bến Tre 2019]%Câu 34
	Cho hàm số $ f(x)$ thỏa mãn $\left(f'(x)\right)^2+f(x).f''(x)=15x^4+12x$, $\forall x\in\mathbb{R}$ và $ f(0)=f'(0)=1$. Giá trị của $f^2(1)$ bằng
	\choice
	{$\dfrac{5}{2}$}
	{\True $8$}
	{$10$}
	{$4$}
	\loigiai{
		Theo giả thiết, $\forall x\in\mathbb{R}$: $\left(f'(x)\right)^2+f(x).f''(x)=15x^4+12x$\\
		$\Leftrightarrow{f}'(x).f'(x)+f(x).f''(x)=15x^4+12x$\\
		$\Leftrightarrow{\left[f(x).f'(x)\right]'}=15x^4+12x$\\
		$\Leftrightarrow f(x).f'(x)=\displaystyle\int{\left(15x^4+12x\right)\mathrm{d}x}=3x^5+6x^2+C$ $(1)$.\\
		Thay $ x=0$ vào $(1)$, ta được $ f(0).f'(0)=C\Leftrightarrow C=1$.\\
		Khi đó, $(1)$ trở thành $f(x).f'(x)=3x^5+6x^2+1$\\
		$\Rightarrow\displaystyle\int\limits_0^1f(x).f'(x)\mathrm{d}x=\displaystyle\int\limits_0^1\left(3x^5+6x^2+1\right)\mathrm{d}x\Leftrightarrow\left[\dfrac{1}{2}{f^2}(x)\right]\Biggr|^{4}_{1}=\left(\dfrac{1}{2}{x^6}+2x^3+x\right)\Biggr|^{4}_{1}$\\
		$\Leftrightarrow\dfrac{1}{2}\left[f^2(1)-f^2(0)\right]=\dfrac{7}{2}\Leftrightarrow{f^2}(1)-1=7\Leftrightarrow{f^2}(1)=8$.\\
		Vậy $f^2(1)=8$.
	}
\end{ex}
\begin{ex}
	Cho hàm số $ y=f(x)$ có đạo hàm liên tục trên $\left(1;+\infty\right)$ và thỏa mãn $\left(x{f}'(x)-2f(x)\right).\ln x=x^3-f(x)$, $\forall x\in\left(1;+\infty\right)$; biết $ f\left(\sqrt[3]{\mathrm{e}}\right)=3\mathrm{e}$. Giá trị $ f(2)$ thuộc khoảng nào dưới đây?
	\choice
	{$\left(12;\dfrac{25}{2}\right)$}
	{$\left(13;\dfrac{27}{2}\right)$}
	{\True $\left(\dfrac{23}{2};12\right)$}
	{$\left(14;\dfrac{29}{2}\right)$}
	\loigiai{
		Xét phương trình $\left(x{f}'(x)-2f(x)\right).\ln x=x^3-f(x)$ $(1)$ trên khoảng $\left(1;+\infty\right)$:\\
		$(1)\Leftrightarrow x\ln x.f'(x)+\left(1-2\ln x\right).f(x)=x^3\Leftrightarrow{f}'(x)+\dfrac{1-2\ln x}{x\ln x}\cdot f(x)=\dfrac{x^2}{\ln x}$ $(2)$.\\
		Đặt $ g(x)=\dfrac{1-2\ln x}{x\ln x}$. Ta tìm một nguyên hàm $ G(x)$ của $ g(x)$.\\
		Ta có $\displaystyle\int{g(x)\mathrm{d}x}=\displaystyle\int{\dfrac{1-2\ln x}{x\ln x}\mathrm{d}x}=\displaystyle\int{\dfrac{1-2\ln x}{\ln x}\text{d}\left(\ln x\right)}=\displaystyle\int{\left(\dfrac{1}{\ln x}-2\right)\mathrm{d}\left(\ln x\right)}$\\
		$=\ln\left(\ln x\right)-2\ln x+C=\ln\left(\dfrac{\ln x}{x^2}\right)+C$.\\
		Ta chọn $ G(x)=\ln\left(\dfrac{\ln x}{x^2}\right)$.\\
		Nhân cả 2 vế của $(2)$ cho $\mathrm{e}^{G(x)}=\dfrac{\ln x}{x^2}$, ta được $\dfrac{\ln x}{x^2}\cdot{f}'(x)+\dfrac{1-2\ln x}{x^3}\cdot f(x)=1$\\
		$\Leftrightarrow{\left(\dfrac{\ln x}{x^2}\cdot f(x)\right)'}=1\Leftrightarrow\dfrac{\ln x}{x^2}\cdot f(x)=x+C$ $(3)$.\\
		Theo giả thiết, $ f\left(\sqrt[3]{\mathrm{e}}\right)=3\mathrm{e}$ nên thay $ x=\sqrt[3]{\mathrm{e}}$ vào $(3)$, ta được\\
		$\dfrac{\ln\left(\sqrt[3]{\mathrm{e}}\right)}{\sqrt[3]{\mathrm{e}^2}}.f\left(\sqrt[3]{\mathrm{e}}\right)=\sqrt[3]{\mathrm{e}}+C\Leftrightarrow C=\dfrac{1}{3\sqrt[3]{\mathrm{e}^2}}\cdot 3\mathrm{e}-\sqrt[3]{\mathrm{e}}=0$.\\
		Từ đây, ta tìm được $ f(x)=\dfrac{x^3}{\ln x}\Rightarrow f(2)=\dfrac{2^3}{\ln 2}$.Vậy $ f(2)\in\left(\dfrac{23}{2};12\right)$.
	}
\end{ex}
\begin{ex}
	[Chuyên Nguyễn Du - ĐăkLăk 2019]%Câu 36
	Cho hàm số $ f(x)$ có đạo hàm trên $\mathbb{R}$ thỏa mãn $ 3f'(x).\mathrm{e}^{f^3(x)-x^2-1}-\dfrac{2x}{f^2(x)}=0$ với $\forall x\in\mathbb{R}$. Biết $ f(0)=1$, tính tích phân $\displaystyle\int\limits_0^{\sqrt{7}}{x.f(x)\mathrm{d}x}$.
	\choice
	{$\dfrac{11}{2}$}
	{$\dfrac{15}{4}$}
	{\True $\dfrac{45}{8}$}
	{$\dfrac{9}{2}$}
	\loigiai{
		Ta có $ 3f'(x).\mathrm{e}^{f^3(x)-x^2-1}-\dfrac{2x}{f^2(x)}=0\Leftrightarrow  3f^2(x).f'(x).\mathrm{e}^{f^3(x)}=2x.\mathrm{e}^{x^2+1}$\\
		$\Rightarrow\displaystyle\int{3f^2(x).f'(x).\mathrm{e}^{f^3(x)}\mathrm{d}x=\displaystyle\int{2x.\mathrm{e}^{x^2+1}}}\mathrm{d}x$ $\Rightarrow\displaystyle\int{\mathrm{e}^{f^3(x)}\mathrm{d}\left(f^3(x)\right)=\displaystyle\int{\mathrm{e}^{x^2+1}}}\mathrm{d}\left(x^2+1\right)$ $\Rightarrow{\mathrm{e}^{f^3(x)}}=\mathrm{e}^{x^2+1}+C$. Mặt khác, vì $ f(0)=1$ nên $ C=0$.\\
		Do đó $\mathrm{e}^{f^3(x)}=\mathrm{e}^{x^2+1}\Leftrightarrow{f^3}(x)=x^2+1\Leftrightarrow f(x)=\sqrt[3]{x^2+1}$.\\
		Vậy $\displaystyle\int\limits_0^{\sqrt{7}}{x.f(x)\mathrm{d}x}=\displaystyle\int\limits_0^{\sqrt{7}}{x.\sqrt[3]{x^2+1}\mathrm{d}x}=\dfrac{1}{2}\displaystyle\int\limits_0^{\sqrt{7}}{\sqrt[3]{x^2+1}\mathrm{d}\left(x^2+1\right)}=\dfrac{3}{8}\left.\left[\left(x^2+1\right)\sqrt[3]{x^2+1}\right]\right|_0^{\sqrt{7}}=\dfrac{45}{8}$.
	}
\end{ex}
\begin{ex}
	[SP Đồng Nai - 2019]%Câu 37
	Cho hàm số $ y=f(x)$ liên tục và không âm trên $\mathbb{R}$ thỏa mãn $ f(x).f'(x)=2x\sqrt{f^2(x)+1}$ và $ f(0)=0$. Gọi $ M,m$ lần lượt là giá trị lớn nhất và giá trị nhỏ nhất của hàm số $ y=f(x)$ trên đoạn $\left[1;3\right]$. Biết rằng giá trị của biểu thức $ P=2M-m$ có dạng $ a\sqrt{11}-b\sqrt{3}+c,\left(a,b,c\in\mathbb{Z}\right)$. Tính $ a+b+c$
	\choice
	{\True $ a+b+c=7$}
	{$ a+b+c=4$}
	{$ a+b+c=6$}
	{$ a+b+c=5$}
	\loigiai{
		Ta có $ f(x).f'(x)=2x\sqrt{f^2(x)+1}\Leftrightarrow\dfrac{f(x).f'(x)}{\sqrt{f^2(x)+1}}=2x\Rightarrow\displaystyle\int{\dfrac{f(x).f'(x)}{\sqrt{f^2(x)+1}}\mathrm{d}x}=\displaystyle\int{2x\mathrm{d}x}$\\
		$\Leftrightarrow\sqrt{f^2(x)+1}=x^2+C$.\\
		Mà $ f(0)=0\Leftrightarrow C=1\Rightarrow\sqrt{f^2(x)+1}=x^2+1\Leftrightarrow{f^2}(x)=\left(x^2+1\right)^2-1=x^4+2x^2$\\
		$\Leftrightarrow f(x)=\sqrt{x^4+2x^2}$(do $ f(x)\ge 0,\forall x\in\mathbb{R}$).\\
		Ta có $f'(x)=\dfrac{2x^3+2x}{\sqrt{x^4+2x^2}}>0,\forall x\in\left[1;3\right]\Rightarrow\underset{\left[1;3\right]}{\max}f(x)=f(3)=3\sqrt{11};\underset{\left[1;3\right]}{\min}f(x)=f(1)=\sqrt{3}$.\\
		Ta có $ P=2M-m=6\sqrt{11}-\sqrt{3}\Rightarrow a=6;b=1;c=0\Rightarrow a+b+c=7$.
	}
\end{ex}
\begin{ex}
	Cho hàm số $y=f(x)$ liên tục trên $\mathbb{R}\setminus\left\{-1;0\right\}$ thỏa mãn $f(1)=2\ln 2+1$, $x\left(x+1\right){f}'(x)+\left(x+2\right)f(x)=x\left(x+1\right)$, $\forall x\in\mathbb{R}\setminus\left\{-1;0\right\}$. Biết $f(2)=a+b\ln 3$, với $a,b$ là hai số hữu tỉ. Tính $T=a^2-b$.
	\choice
	{$ T=\dfrac{21}{16}$}
	{$ T=\dfrac{3}{2}$}
	{$ T=0$}
	{\True $ T=-\dfrac{3}{16}$}
	\loigiai{
		Ta có $x\left(x+1\right){f}'(x)+\left(x+2\right)f(x)=x\left(x+1\right)$, $\forall x\in\mathbb{R}\setminus\left\{-1;0\right\}$.\\
		$\Rightarrow\dfrac{x^2}{\left(x+1\right)}{f}'(x)+\dfrac{x^2+2x}{\left(x+1\right)^2}f(x)=\dfrac{x^2}{x+1}$, $\forall x\in\mathbb{R}\setminus\left\{-1;0\right\}$.\\
		$\Rightarrow{\left[\dfrac{x^2}{x+1}f(x)\right]'}=\dfrac{x^2}{x+1}$, $\forall x\in\mathbb{R}\setminus\left\{-1;0\right\}$.\\
		$\Rightarrow\displaystyle\int_{}^{}{\dfrac{x^2}{x+1}\mathrm{d}x}=\dfrac{x^2}{x+1}f(x)+C'$, $\forall x\in\mathbb{R}\setminus\left\{-1;0\right\}$.\\
		$\Rightarrow\displaystyle\int_{}^{}{\left(x-1+\dfrac{1}{x+1}\right)\mathrm{d}x}=\dfrac{x^2}{x+1}f(x)+C'$, $\forall x\in\mathbb{R}\setminus\left\{-1;0\right\}$.\\
		$\Rightarrow\dfrac{x^2}{2}-x+\ln\left| x+1\right|+C''=\dfrac{x^2}{x+1}f(x)+C'$.\\
		$\Rightarrow\dfrac{x^2}{2}-x+\ln\left| x+1\right|+C=\dfrac{x^2}{x+1}f(x)$, $\forall x\in\mathbb{R}\setminus\left\{-1;0\right\}$.\\
		Ta có: $f(1)=2\ln 2+1$ và $f(1)=-1+2\ln 2+2C\Rightarrow C=1$.\\
		$\Rightarrow\dfrac{x^2}{2}-x+\ln\left| x+1\right|+1=\dfrac{x^2}{x+1}f(x)$.\\
		$\Rightarrow f(2)=\dfrac{3}{4}+\dfrac{3}{4}.\ln\left| 3\right|$ và $f(2)=a+b\ln 3\Rightarrow a=\dfrac{3}{4},b=\dfrac{3}{4}\Rightarrow T=a^2-b=\dfrac{9}{16}-\dfrac{3}{4}=\dfrac{-3}{16}$.
	}
\end{ex}
\begin{ex}
	Cho hàm số $y=f(x)$ liên tục trên $\left(0;+\infty\right)$ thỏa mãn $3x.f(x)-x^2.f'(x)=2f^2(x)$, với $f(x)\ne 0,\forall x\in\left(0;+\infty\right)$ và $f(1)=\dfrac{1}{3}$. Gọi $M,m$ lần lượt là giá trị lớn nhất, giá trị nhỏ nhất của hàm số $y=f(x)$ trên đoạn $\left[1;2\right]$. Tính $M+m$.
	\choice
	{$\dfrac{9}{10}$}
	{$\dfrac{21}{10}$}
	{\True $\dfrac{5}{3}$}
	{$\dfrac{7}{3}$}
	\loigiai{
		Ta có $3x.f(x)-x^2.f'(x)=2f^2(x)\Rightarrow 3x^2.f(x)-x^3.f'(x)=2x.f^2(x)$\\
		$\Rightarrow\dfrac{3x^2.f(x)-x^3.f'(x)}{f^2(x)}=2x$ vì $f(x)\ne 0,\forall x\in\left(0;+\infty\right)$.\\
		$\Rightarrow{\left(\dfrac{x^3}{f(x)}\right)'}=2x\Rightarrow\dfrac{x^3}{f(x)}=\displaystyle\int_{}^{}{2x\mathrm{d}x}=x^2+C$.\\
		Mà $f(1)=\dfrac{1}{3}\Rightarrow C=2\Rightarrow f(x)=\dfrac{x^3}{x^2+2}$.\\
		Ta có $f(x)=\dfrac{x^3}{x^2+2}\Rightarrow{f}'(x)=\dfrac{x^4+6x^2}{\left(x^2+2\right)^2}>0,\forall x\in\left(0;+\infty\right)$.\\
		Vậy, hàm số $f(x)=\dfrac{x^3}{x^2+2}$ đồng biến trên khoảng $\left(0;+\infty\right)$.\\
		Mà $\left[1;2\right]\subset\left(0;+\infty\right)$ nên hàm số $f(x)=\dfrac{x^3}{x^2+2}$ đồng biến trên đoạn $\left[1;2\right]$.\\
		Suy ra, $M=f(2)=\dfrac{4}{3};m=f(1)=\dfrac{1}{3}\Rightarrow M+m=\dfrac{5}{3}$.
	}
\end{ex}
\begin{ex}
	[Chuyên KHTN - 2021]%Câu 40
	Cho hàm số $ f(x)$ liên tục trên $\mathbb{R}$ và thỏa mãn $ x{f}'(x)+\left(x+1\right)f(x)=\mathrm{e}^{-x}$ với mọi $ x$. Tính $f'(0)$.
	\choice
	{$ 1$}
	{\True $-1$}
	{$ \mathrm{e}$}
	{$\dfrac{1}{\mathrm{e}}$}
	\loigiai{
		Ta có $x{f}'(x)+\left(x+1\right)f(x)=\mathrm{e}^{-x}\Leftrightarrow x{f}'(x)+f(x)+xf(x)=\mathrm{e}^{-x}$\\
		$\Leftrightarrow x{\mathrm{e}^x}{f}'(x)+\left(x+1\right){\mathrm{e}^x}f(x)=1$\\
		$\Leftrightarrow x{\mathrm{e}^x}{f}'(x)+\left(x{\mathrm{e}^x}\right)'f(x)=1$\\
		$\Leftrightarrow{\left(x{\mathrm{e}^x}f(x)\right)'}=1$\\
		$\Leftrightarrow x{\mathrm{e}^x}f(x)=\displaystyle\int{\mathrm{d}x}=x+C$ $(*)$\\
		Với $ x=0$\\
		Thay vào biểu thức ban đầu ta có $ 0.f'(0)+\left(0+1\right)f(0)=e^{-0}=1\Leftrightarrow f(0)=1$.\\
		Thay vào $(*)$, ta có $C=0$.\\
		Khi đó $x{e^x}f(x)=x\Leftrightarrow f(x)=\left\{\begin{aligned}
			&{\mathrm{e}^{-x}}&\text{khi}&~x\ne 0\\ 
			& 1&\text{khi}&~x=0\\ 
		\end{aligned}\right.$\\
		Suy ra $f'(0)=\underset{x\to 0}{\lim}\dfrac{f(x)-f(0)}{x}=\underset{x\to 0}{\lim}\dfrac{\mathrm{e}^{-x}-1}{x}=-\underset{x\to 0}{\lim}\dfrac{\mathrm{e}^{-x}-1}{-x}=-1$.
	}
\end{ex}
\begin{ex}
	[Chuyên Hạ Long - Quảng Ninh - 2021]%Câu 41
	Cho hàm số $ y=f(x)$ liên tục và có đạo hàm trên $\left(-\sqrt{2};\sqrt{2}\right)\setminus\left\{ 0\right\}$, thỏa mãn $ f(1)=0$ và $f'(x)+x\left(\mathrm{e}^{f(x)}+2\right)+\dfrac{x}{\mathrm{e}^{f(x)}}=0$. Giá trị của $ f\left(\dfrac{1}{2}\right)$ bằng
	\choice
	{\True $\ln 7$}
	{$\ln 5$}
	{$\ln 6$}
	{$\ln 3$}
	\loigiai{
		$\begin{aligned}
			& f'(x)+x+\dfrac{x}{\mathrm{e}^{f(x)}}=0\Leftrightarrow f'(x){\mathrm{e}^{f(x)}}+x{\mathrm{e}^{f(x)}}\left(\mathrm{e}^{f(x)}+2\right)+x=0\\ 
			&\Leftrightarrow f'(x){\mathrm{e}^{f(x)}}+x{\left(\mathrm{e}^{f(x)}+1\right)^2}=0\Leftrightarrow\dfrac{-f'(x){\mathrm{e}^{f(x)}}}{\left(\mathrm{e}^{f(x)}+1\right)^2}=x\\ 
			&\Leftrightarrow\displaystyle\int{\dfrac{-f'(x){\mathrm{e}^{f(x)}}}{\left(\mathrm{e}^{f(x)}+1\right)^2}\mathrm{d}x=\displaystyle\int{x\mathrm{d}x\Leftrightarrow}}\displaystyle\int{\dfrac{-d\left[\mathrm{e}^{f(x)}\right]}{\left(\mathrm{e}^{f(x)}+1\right)^2}=\displaystyle\int{x\mathrm{d}x\Leftrightarrow}}\dfrac{1}{\mathrm{e}^{f(x)}+1}=\dfrac{x^2}{2}+C,(1)\\ 
			&\text{Trong (1) cho x=1}\Rightarrow\dfrac{1}{\mathrm{e}^{f(1)}+1}=\dfrac{1}{2}+C\Rightarrow C=0.\text{Suy ra}\dfrac{1}{\mathrm{e}^{f(x)}+1}=\dfrac{x^2}{2},(2)\\ 
			&\text{Trong (2) cho x=}\dfrac{1}{2}\Rightarrow\dfrac{1}{\mathrm{e}^{f\left(\dfrac{1}{2}\right)}+1}=\dfrac{1}{8}\Rightarrow f\left(\dfrac{1}{2}\right)=\ln 7.\\ 
		\end{aligned}$
	}
\end{ex}
\begin{ex}
	[Chuyên ĐH Vinh - Nghệ An - 2021]%Câu 42
	Giả sử $ f(x)$ là hàm có đạo hàm liên tục trên $\left(0;\pi\right)$ và $f'(x)\sin x=x+f(x)\cos x,\forall x\in\left(0;\pi\right)$. Biết $ f(\dfrac{\pi}{2})=1,f(\dfrac{\pi}{6})=\dfrac{1}{12}(a+b\ln 2+c\pi\sqrt{3})$, với $ a,b,c$ là các số nguyên. Giá trị của $ a+b+c$ bằng
	\choice
	{\True $-1$}
	{$ 1$}
	{$ 11$}
	{$-11$}
	\loigiai{
		Ta có $\begin{aligned}
			&{f}'(x)\sin x=x+f(x)cosx\Leftrightarrow{f}'(x)\sin x-f(x)\cos x=x\\ 
			&\Leftrightarrow\dfrac{f'(x)\sin x-f(x)cosx}{\sin^2x}=\dfrac{x}{\sin^2x}\Leftrightarrow{\left[\dfrac{f(x)}{\sin x}\right]'}=\dfrac{x}{\sin^2x}\\ 
		\end{aligned}$\\
		$\Rightarrow\dfrac{f(x)}{\sin x}=\displaystyle\int{\dfrac{x}{\sin^2x}\mathrm{d}x=-x\cot x+\displaystyle\int{\cot x\mathrm{d}x=-x\cot x+\ln\left|\sin x\right|+C}}$.\\
		Hay $\dfrac{f(x)}{\sin x}=-x\cot x+\ln\left|\sin x\right|+C$.\\
		$f\left(\dfrac{\pi}{2}\right)=1\Rightarrow\dfrac{1}{\sin\dfrac{\pi}{2}}=-\dfrac{\pi}{2}\cot\dfrac{\pi}{2}+\ln\left|\sin\dfrac{\pi}{2}\right|+C\Leftrightarrow C=1$.\\
		$\Rightarrow\dfrac{f(x)}{\sin x}=-x\cot x+\ln\left|\sin x\right|+1$.\\
		Do đó $\Rightarrow\dfrac{f(\dfrac{\pi}{6})}{\sin\dfrac{\pi}{6}}=-\dfrac{\pi}{6}\cot\dfrac{\pi}{6}+\ln\left|\sin\dfrac{\pi}{6}\right|+1\Rightarrow f(\dfrac{\pi}{6})=\dfrac{1}{12}\left(6-6\ln 2-\pi\sqrt{3}\right)$.\\
		$\Rightarrow a=6,b=-6,c=-1$. $ a+b+c=-1$.
	}
\end{ex}
\begin{ex}
	[THPT Đặng Thúc Hứa - Nghệ An - 2021]%Câu 43
	Cho hàm số $ y=f(x)$ có đạo hàm liên tục trên khoảng $\left(0;+\infty\right)$ thỏa mãn $x{f}'(x)=f(x)+x^3\ln x,\forall x>0$ và $ f(1)=\dfrac{3}{4}$. Tính $ f(2)$
	\choice
	{$ 2\ln 2+1$}
	{$ 4\ln 2+1$}
	{$ 2\ln 2$}
	{\True $ 4\ln 2$}
	\loigiai{
		$x{f}'(x)=f(x)+x^3\ln x\Leftrightarrow\dfrac{x{f}'(x)-f(x)}{x^2}=x\ln x\Leftrightarrow{\left(\dfrac{f(x)}{x}\right)'}=x\ln x\Leftrightarrow\dfrac{f(x)}{x}=\displaystyle\int{x\ln x\mathrm{d}x}$\\
		Mà $\displaystyle\int{x\ln x\mathrm{d}x=\displaystyle\int{\ln x.d\left(\dfrac{1}{2}{x^2}\right)}}=\dfrac{1}{2}{x^2}\ln x-\displaystyle\int{\dfrac{1}{2}{x^2}.\dfrac{1}{x}\mathrm{d}x=}\dfrac{1}{2}{x^2}\ln x-\dfrac{1}{4}{x^2}+C$\\
		Suy ra $ f(x)=\dfrac{1}{2}{x^3}\ln x-\dfrac{1}{4}{x^3}+Cx$, mà $ f(1)=\dfrac{3}{4}=C-\dfrac{1}{4}\Rightarrow C=1$.\\
		Vậy $ f(x)=\dfrac{1}{2}{x^3}\ln x-\dfrac{1}{4}{x^3}+x$. Khi đó $ f(2)=4\ln 2$.
	}
\end{ex}
\begin{ex}
	[Liên Trường Nghệ An – 2021]%Câu 44
	Cho hàm số $ f(x)$ liên tục và luôn nhận giá trị dương trên $\mathbb{R}$, thỏa mãn $ f(0)=\mathrm{e}^2$ và $ 2\sin 2x\left[f(x)+\mathrm{e}^{\cos 2x}.\sqrt{f(x)}\right]+f'(x)=0,\mathrm{e} x\in\mathbb{R}$. Khi đó $ f\left(\dfrac{2\pi}{3}\right)$ thuộc khoảng
	\choice
	{$\left(1;2\right)$}
	{$\left(2;3\right)$}
	{$\left(3;4\right)$}
	{\True $\left(0;1\right)$}
	\loigiai{
		Từ giả thiết ta có $ 2\sin 2x.f(x)+f'(x)=-\mathrm{e}^{\cos^2x-\sin^2x}\sin 2x.\sqrt{f(x)}$\\
		$\Leftrightarrow\sin 2x.\mathrm{e}^{\sin^2x}.\sqrt{f(x)}+\mathrm{e}^{\sin^2x}.\dfrac{f'(x)}{2\sqrt{f(x)}}=-\mathrm{e}^{\cos^2x}\sin 2x\Rightarrow{\displaystyle\int{\left(\mathrm{e}^{\sin^2x}.\sqrt{f(x)}\right)}'}\mathrm{d}x=\displaystyle\int{\mathrm{e}^{\cos^2x}}\mathrm{d}\left(\cos^2x\right)$\\
		$\Rightarrow{\mathrm{e}^{\sin^2x}}.\sqrt{f(x)}=\mathrm{e}^{\cos^2x}+C$\\
		Mà $ f(0)=\mathrm{e}^2\Rightarrow C=0\Rightarrow\sqrt{f(x)}=\mathrm{e}^{\cos^2x-\sin^2x}\Rightarrow f(x)=\mathrm{e}^{2\cos 2x}$.\\
		Vậy $ f\left(\dfrac{2\pi}{3}\right)=\dfrac{1}{\mathrm{e}}\approx 0,37$.
	}
\end{ex}
\begin{ex}
	[Chuyên Thái Nguyên 2019]%Câu 45
	Cho $F(x)$ là một nguyên hàm của hàm số $f(x)=\mathrm{e}^{x^2}\left(x^3-4x\right)$. Hàm số $F\left(x^2+x\right)$ có bao nhiêu điểm cực trị?
	\choice
	{$6$}
	{\True $5$}
	{$3$}
	{$4$}
	\loigiai{
		Ta có $F'(x)=f(x)$\\
		$\Rightarrow{F}'\left(x^2+x\right)=f\left(x^2+x\right).\left(x^2+x\right)'=\left(2x+1\right)\left(x^2+x\right){\mathrm{e}^{\left(x^2+x\right)^2}}\left(\left(x^2+x\right)^2-4\right)$\\
		$=\left(2x+1\right)x\left(x+1\right){\mathrm{e}^{\left(x^2+x\right)^2}}\left(x^2+x-2\right)\left(x^2+x+2\right)$\\
		$=\left(2x+1\right)x\left(x+1\right)\left(x+2\right)\left(x-1\right)\left(x^2+x+2\right){\mathrm{e}^{\left(x^2+x\right)^2}}=0\Leftrightarrow x\in\left\{-2;-1;\dfrac{-1}{2};0;1\right\}$\\
		$F'\left(x^2+x\right)=0$ có 5 nghiệm đơn nên $F\left(x^2+x\right)$ có 5 điểm cực trị.
	}
\end{ex}
\begin{ex}
	[THCS - THPT Nguyễn Khuyến 2019]%Câu 46
	Cho $F(x)=\displaystyle\int{\dfrac{\left(1+\cos^2x\right)\left(\sin x+\cot x\right)}{\sin^4x}\mathrm{d}x}$ và $S$ là tổng tất cả các nghiệm của phương trình $F(x)=F\left(\dfrac{\pi}{2}\right)$ trên khoảng $\left(0;4\pi\right)$. Tổng $S$ thuộc khoảng
	\choice
	{\True $\left(6\pi ;9\pi\right)$}
	{$\left(2\pi ;4\pi\right)$}
	{$\left(4\pi ;6\pi\right)$}
	{$\left(0;2\pi\right)$}
	\loigiai{
		Ta có $F(x)=\displaystyle\int{\dfrac{\left(1+\cos^2x\right)\left(\sin x+\cot x\right)}{\sin^4x}\mathrm{d}x}=\displaystyle\int{\dfrac{\left(1+\cos^2x\right)\sin x}{\sin^4x}\mathrm{d}x}+\displaystyle\int{\dfrac{\left(1+\cos^2x\right)\cot x}{\sin^4x}\mathrm{d}x}$\\
		Gọi $A=\displaystyle\int{\dfrac{\left(1+\cos^2x\right)\cot x}{\sin^4x}\mathrm{d}x}$ và $B=\displaystyle\int{\dfrac{\left(1+\cos^2x\right)\sin x}{\sin^4x}\mathrm{d}x}$\\
		Ta có\\
		$\begin{aligned}
			& A=\displaystyle\int{\dfrac{\left(1+\cos^2x\right)\cot x}{\sin^4x}\mathrm{d}x}=\displaystyle\int{\dfrac{\left(1+2\cot^2x\right)\cot x}{\sin^2x}\mathrm{d}x}=-\displaystyle\int{\left(\cot x+2\cot^3x\right).d\left(\cot x\right)}\\ 
			&=-\left(\dfrac{\cot^2x}{2}+\dfrac{\cot^4x}{2}\right)+C_1. 
		\end{aligned}$\\
		$B=\displaystyle\int{\dfrac{\left(1+\cos^2x\right)\sin x}{\sin^4x}\mathrm{d}x}=\displaystyle\int{\dfrac{\left(1+\cos^2x\right)\sin x}{\left(1-\cos^2x\right)^2}\mathrm{d}x}$\\
		Đặt $t=\cos x$ , suy ra $dt=-\sin x.\mathrm{d}x$ . Khi đó:\\
		$\begin{aligned}
			&B=-\displaystyle\int{\dfrac{1+t^2}{\left(t^2-1\right)^2}dt}=-\displaystyle\int{\dfrac{1+t^2}{\left(t-1\right)^2.\left(t+1\right)^2}dt}=-\dfrac{1}{2}\displaystyle\int{\left[\dfrac{1}{\left(t-1\right)^2}+\dfrac{1}{\left(t+1\right)^2}\right]dt=\dfrac{1}{2}\left(\dfrac{1}{t-1}+\dfrac{1}{t+1}\right)}+C_2\\ 
			&=\dfrac{1}{2}\left(\dfrac{1}{\cos x-1}+\dfrac{1}{\cos x+1}\right)+C_2
		\end{aligned}$\\
		Do đó\\
		$F(x)=A+B=\dfrac{1}{2}\left(\dfrac{1}{\cos x-1}+\dfrac{1}{\cos x+1}\right)-\left(\dfrac{\cot^2x}{2}+\dfrac{\cot^4x}{2}\right)+C$\\
		Suy ra\\
		$F(x)=F\left(\dfrac{\pi}{2}\right)\Leftrightarrow\dfrac{1}{2}\left(\dfrac{1}{\cos x-1}+\dfrac{1}{\cos x+1}\right)-\left(\dfrac{\cot^2x}{2}+\dfrac{\cot^4x}{2}\right)+C=C$\\
		$\Leftrightarrow\dfrac{1}{\cos x-1}+\dfrac{1}{\cos x+1}-\cot^2x-\cot^4x=0$\\
		$\Leftrightarrow\dfrac{2\cos x}{\sin^2x}+\dfrac{\cos^2x}{\sin^2x}+\dfrac{\cos^4x}{\sin^4x}=0$\\
		Với điều kiện $\sin x\ne 0$ ,\\
		$\begin{aligned}
			&(*)\Leftrightarrow\left[\begin{aligned}
				&\cos x=0\\ 
				& 2+\cos x+\dfrac{\cos^3x}{\sin^2x}=0\\ 
			\end{aligned}\right.\Leftrightarrow\left[\begin{aligned}
				&\cos x=0\\ 
				& 2\left(1-\cos^2x\right)+\cos x\left(1-\cos^2x\right)+\cos^3x=0\\ 
			\end{aligned}\right.\\ 
			&\Leftrightarrow\left[\begin{aligned}
				&\cos x=0\\ 
				&-2\cos^2x+\cos x+2=0\\ 
			\end{aligned}\right.\Leftrightarrow\left[\begin{aligned}
				&\cos x=0\\ 
				&\cos x=\dfrac{1-\sqrt{17}}{4}\\ 
			\end{aligned}\right. 
		\end{aligned}$\\
		Theo giả thiết $x\in\left(0;4\pi\right)$ nên $x=\dfrac{\pi}{2};x=\dfrac{3\pi}{2};x=\dfrac{\pi}{2}+2\pi ;x=\dfrac{3\pi}{2}+2\pi $.\\
		$x=\alpha ;x=\alpha+2\pi $.\\
		$x=\beta ;x=\beta+2\pi $.\\
		Khi đó tổng các nghiệm này sẽ lớn hơn $9\pi $.
	}
\end{ex}
\begin{ex}
	[Chuyên Quốc Học Huế 2019]%Câu 47
	Cho hàm số $ F(x)$ là một nguyên hàm của hàm số $ f(x)=\dfrac{2\cos x-1}{\sin^2x}$ trên khoảng $\left(0;\pi\right)$. Biết rằng giá trị lớn nhất của $ F(x)$ trên khoảng $\left(0;\pi\right)$ là $\sqrt{3}$. Chọn mệnh đề đúng trong các mệnh đề sau.
	\choice
	{\True $ F\left(\dfrac{\pi}{6}\right)=3\sqrt{3}-4$}
	{$ F\left(\dfrac{2\pi}{3}\right)=\dfrac{\sqrt{3}}{2}$}
	{$ F\left(\dfrac{\pi}{3}\right)=-\sqrt{3}$}
	{$ F\left(\dfrac{5\pi}{6}\right)=3-\sqrt{3}$}
	\loigiai{
		Ta có\\
		$\displaystyle\int{f(x)\mathrm{d}x}=\displaystyle\int{\dfrac{2\cos x-1}{\sin^2x}\mathrm{d}x}=2\displaystyle\int{\dfrac{\cos x}{\sin^2x}\mathrm{d}x-\displaystyle\int{\dfrac{1}{\sin^2x}\mathrm{d}x}}$\\
		$=2\displaystyle\int{\dfrac{d\left(sinx\right)}{\sin^2x}-\displaystyle\int{\dfrac{1}{\sin^2x}\mathrm{d}x}}=-\dfrac{2}{\sin x}+\cot x+C$\\
		Do $ F(x)$ là một nguyên hàm của hàm số $ f(x)=\dfrac{2\cos x-1}{\sin^2x}$ trên khoảng $\left(0;\pi\right)$ nên hàm số $ F(x)$ có công thức dạng $ F(x)=-\dfrac{2}{\sin x}+\cot x+C$ với mọi $ x\in\left(0;\pi\right)$.\\
		Xét hàm số $ F(x)=-\dfrac{2}{\sin x}+\cot x+C$ xác định và liên tục trên $\left(0;\pi\right)$.\\
		$ F'(x)=f(x)=\dfrac{2\cos x-1}{\sin^2x}$\\
		Xét $ F'(x)=0\Leftrightarrow\dfrac{2\cos x-1}{\sin^2x}=0\Leftrightarrow\cos x=\dfrac{1}{2}\Leftrightarrow x=\pm\dfrac{\pi}{3}+k2\pi\left(k\in\mathbb{Z}\right)$.\\
		Trên khoảng $\left(0;\pi\right)$, phương trình $ F'(x)=0$ có một nghiệm $ x=\dfrac{\pi}{3}$\\
		Bảng biến thiên:\\
		{\color{red}HÌNH Ở ĐÂY}\\
		$\underset{\left(0;\pi\right)}{\max}F(x)=F\left(\dfrac{\pi}{3}\right)=-\sqrt{3}+C$\\
		Theo đề bài ta có, $-\sqrt{3}+C=\sqrt{3}\Leftrightarrow C=2\sqrt{3}$.\\
		Do đó, $ F(x)=-\dfrac{2}{\sin x}+\cot x+2\sqrt{3}$.
	}
\end{ex}
\begin{ex}
	Biết $ F(x)$ là nguyên hàm của hàm số $ f(x)=\dfrac{x\cos x-\sin x}{x^2}$. Hỏi đồ thị của hàm số $ y=F(x)$ có bao nhiêu điểm cực trị trên khoảng $\left(0;4\pi\right)$?
	\choice
	{$ 2$}
	{$ 1$}
	{\True $ 3$}
	{$ 0$}
	\loigiai{
		Ta có $ F'(x)=f(x)=\dfrac{x\cos x-\sin x}{x^2}$ trên $\left(0;4\pi\right)$.\\
		$ F'(x)=f(x)=\dfrac{x\cos x-\sin x}{x^2}=0\Leftrightarrow x\cos x-\sin x=0$ trên $\left(0;4\pi\right)$.\\
		Đặt $ g(x)=x\cos x-\sin x$ trên $\left(0;4\pi\right)$.\\
		Ta có $ g'(x)=-x.\sin x=0\Leftrightarrow\left[\begin{aligned}
			& x=\pi\\ 
			& x=2\pi\\ 
			& x=3\pi\\ 
		\end{aligned}\right.$ trên $\left(0;4\pi\right)$.\\
		Từ đó có bảng biến thiên của $ g(x)$:\\
		{\color{red}HÌNH Ở ĐÂY}\\
		Vì $ g(x)$ liên tục và đồng biến trên $\left[\pi ;2\pi\right]$ và $ g\left(\pi\right).g\left(2\pi\right)<0$ nên tồn tại duy nhất $x_1\in\left(\pi ;2\pi\right)$ sao cho $ g\left(x_1\right)=0$.\\
		Tương tự ta có $ g\left(x_2\right)=0$, $ g\left(x_3\right)=0$ với $x_2\in\left(2\pi ;3\pi\right)$, $x_3\in\left(3\pi ;4\pi\right)$.\\
		Từ bảng biến thiên của $ g(x)$ ta thấy $ g(x)<0$ khi $ x\in\left(0;{x_1}\right)$ và $ x\in\left(x_2;{x_3}\right)$; $ g(x)>0$ khi $ x\in\left(x_1;{x_2}\right)$ và $ x\in\left(x_3;4\pi\right)$. Dấu của $ f(x)$ là dấu của $ g(x)$ trên $\left(0;4\pi\right)$.\\
		Do đó ta có bảng biến thiên của $ F(x)$:\\
		{\color{red}HÌNH Ở ĐÂY}\\
		Vậy hàm số $ y=F(x)$ có ba cực trị.}
\end{ex}
\begin{ex}
	[Chuyên - Vĩnh Phúc - 2019]%Câu 49
	Biết $ F(x)$ là nguyên hàm của hàm số $ f(x)=\dfrac{x-\cos x}{x^2}$. Hỏi đồ thị của hàm số $ y=F(x)$ có bao nhiêu điểm cực trị?
	\choice
	{\True $ 1$}
	{$2$}
	{vô số điểm}
	{$0$}
	\loigiai{
		Vì $\left(F(x)\right)'=f(x)$nên ta xét sự đổi dấu của hàm số $ f(x)$ để tìm cực trị hàm số đã cho.\\
		Ta xét hàm số $ g(x)=x-\cos x$, ta có $g'(x)=1+\sin x\ge 0\forall x$.\\
		Vì vậy $ g(x)$ là hàm số đồng biến trên toàn trục số.\\
		Hơn nữa ta có $\left\{\begin{aligned}
			& g\left(\dfrac{\pi}{2}\right)=\dfrac{\pi}{2}>0\\ 
			& g\left(-\dfrac{\pi}{2}\right)=-\dfrac{\pi}{2}<0\\ 
		\end{aligned}\right.$, do đó $ g(x)=0$ có duy nhất nghiệm $\alpha\in\left(-\dfrac{\pi}{2};\dfrac{\pi}{2}\right)$.\\
		Ta có bảng xét dấu\\
		{\color{red}HÌNH Ở ĐÂY}\\
		{\color{red}HÌNH Ở ĐÂY}\\
		Kết luận hàm số đã cho có một cực trị.
	}
\end{ex}
\begin{ex}
	[Chuyên Lê Quý Đôn – Điện Biên 2019]%Câu 50
	Cho hàm số $ y=f(x)$. Đồ thị của hàm số $ y=f'(x)$trên $\left[-5;3\right]$ như hình vẽ (phần cong của đồ thị là một phần của parabol $ y=a{x^2}+bx+c$).\\
	{\color{red}HÌNH Ở ĐÂY}\\
	Biết $ f(0)=0$, giá trị của $ 2f\left(-5\right)+3f(2)$ bằng
	\choice
	{$33$}
	{$\dfrac{109}{3}$}
	{\True $\dfrac{35}{3}$}
	{$11$}
	\loigiai{
		Parabol $ y=a{x^2}+bx+c$ qua các điểm $\left(2;3\right),\left(1;4\right),\left(0;3\right),\left(-1;0\right),\left(3;0\right)$ nên xác định được $ y=-x^2+2x+3,\forall x\ge-1$ suy ra $ f(x)=-\dfrac{x^3}{3}+x^2+3x+C_1$. Mà $ f(0)=0\Rightarrow{C_1}=0,f(x)=-\dfrac{x^3}{3}+x^2+3x$.\\
		Có $ f\left(-1\right)=-\dfrac{5}{3}$; $ f(2)=\dfrac{22}{3}$ (1)\\
		Đồ thị $ f'(x)$ trên đoạn $\left[-4;-1\right]$ qua các điểm $\left(-4;2\right),\left(-1;0\right)$ nên $ f'(x)=\dfrac{-2}{3}\left(x+1\right)\Rightarrow f(x)=\dfrac{-2}{3}\left(\dfrac{x^2}{2}+x\right)+C_2$.\\
		Mà $ f\left(-1\right)=-\dfrac{5}{3}\Leftrightarrow{C_2}=-\dfrac{5}{3}+\dfrac{2}{3}\left(-\dfrac{1}{2}\right)=-2\Rightarrow f(x)=\dfrac{-2}{3}\left(\dfrac{x^2}{2}+x\right)-2$, hay $ f\left(-4\right)=\dfrac{-14}{3}$.\\
		Đồ thị $ f'(x)$ trên đoạn $\left[-5;-4\right]$ qua các điểm $\left(-4;2\right),\left(-5;-1\right)$ nên $ f'(x)=3x+14\Rightarrow f(x)=\dfrac{3x^2}{2}+14x+C_3$.\\
		Mà $ f\left(-4\right)=\dfrac{-14}{3}\Leftrightarrow\dfrac{3.\left(-4\right)^2}{2}+14.\left(-4\right)+C_3=\dfrac{-14}{3}$ suy ra $C_3=\dfrac{82}{3}$.\\
		Ta có $ f(x)=\dfrac{3x^2}{2}+14x+\dfrac{82}{3}\Rightarrow f\left(-5\right)=-\dfrac{31}{6}$ (2).\\
		Từ (1) và (2) ta được $ 2f\left(-5\right)+3f(2)=-\dfrac{31}{3}+22=\dfrac{35}{3}$.
	}
\end{ex}
\begin{ex}
	Cho hàm số $y=f(x)$ có đạo hàm liên tục trên $\left(0;+\infty\right)$ thỏa mãn $f'(x)+\dfrac{f(x)}{x}=4x^2+3x$ và $f(1)=2$. Phương trình tiếp tuyến của đồ thị hàm số $y=f(x)$ tại điểm có hoành độ $x=2$ là
	\choice
	{$y=-16x-20$}
	{\True $y=16x-20$}
	{$y=16x+20$}
	{$y=-16x+20$}
	\loigiai{
		$f'(x)+\dfrac{f(x)}{x}=4x^2+3x\Leftrightarrow x{f}'(x)+f(x)=4x^3+3x^2$.\\
		Lấy nguyên hàm hai vế ta được $ xf(x)=\displaystyle\int{\left(4x^3+3x^2\right)}\mathrm{d}x=x^4+x^3+C$.\\
		Với $ x=1$ ta có: $ f(1)=2+C$.\\
		Theo bài ra $ f(1)=2\Leftrightarrow 2+C=2\Leftrightarrow C=0$.\\
		Vậy $ xf(x)=x^4+x^3\Leftrightarrow f(x)=x^3+x^2$.\\
		Ta có: $f'(x)=3x^2+2x$; $f'(2)=16$; $ f(2)=12$.\\
		Phương trình tiếp tuyến của đồ thị hàm số $ y=f(x)$ tại điểm có hoành độ $ x=2$ là\\
		$ y=16\left(x-2\right)+12\Leftrightarrow y=16x-20$.
	}
\end{ex}
\begin{ex}
	[Chuyên Thoại Ngọc Hầu - An Giang - 2021]%Câu 52
	ho hàm số $f(x)$ liên tục, không âm trên đoạn $\left[0;\dfrac{\pi}{2}\right]$, thỏa mãn $f(0)=\sqrt{3}$ và $f(x).f'(x)=\cos x\sqrt{1+f^2(x)},\forall x\in\left[0;\dfrac{\pi}{2}\right]$. Tìm giá trị nhỏ nhất $m$ và giá trị lớn nhất $M$ của hàm số $f(x)$ trên đoạn $\left[\dfrac{\pi}{6};\dfrac{\pi}{2}\right]$.
	\choice
	{$m=\dfrac{\sqrt{5}}{2},M=\sqrt{3}$}
	{$m=\dfrac{5}{2},M=3$}
	{$m=\sqrt{3},M=2\sqrt{2}$}
	{\True $m=\dfrac{\sqrt{21}}{2},M=2\sqrt{2}$}
	\loigiai{
		$\begin{aligned}
			& f(x).f'(x)=\cos x\sqrt{1+f^2(x)}\Leftrightarrow\dfrac{f(x).f'(x)}{\sqrt{1+f^2(x)}}=\cos x\\ 
			&\Rightarrow\displaystyle\int{\dfrac{f(x).f'(x)}{\sqrt{1+f^2(x)}}\mathrm{d}x=\displaystyle\int{\cos x\mathrm{d}x=\sin x+C_1}}(1)\\ 
		\end{aligned}$\\
		Đặt $t=\sqrt{1+f^2(x)}\Rightarrow{t^2}=1+f^2(x)\Rightarrow t\mathrm{d}t=f(x).f'(x)\mathrm{d}x$.\\
		Suy ra $\displaystyle\int{\dfrac{f(x).f'(x)}{\sqrt{1+f^2(x)}}}\mathrm{d}x=\displaystyle\int{\dfrac{t\mathrm{d}t}{t}}=\displaystyle\int{\mathrm{d}t}=t+C_2=\sqrt{1+f^2(x)}+C_2(2)$\\
		Từ $(1)$ và $(2)$ suy ra $\sqrt{1+f^2(x)}=\sin x+C$. Thay $x=0$ vào ta có: $\sqrt{1+3}=C\Rightarrow C=2$\\
		Hay $\sqrt{1+f^2(x)}=\sin x+2$.\\
		$\begin{aligned}
			&\Leftrightarrow{f^2}(x)=\left(\sin x+2\right)^2-1=\sin^2x+4\sin x+3\\ 
			&\Rightarrow f(x)=\sqrt{\sin^2x+4\sin x+3}\\ 
		\end{aligned}$\\
		Đặt $t=\sin x$. Với $x\in\left[\dfrac{\pi}{6};\dfrac{\pi}{2}\right]\Rightarrow t\in\left[\dfrac{1}{2};1\right]$.\\
		Ta đi xét hàm số $g(t)=\sqrt{t^2+4t+3},t\in\left[\dfrac{1}{2};1\right]$.\\
		$g'(t)=\dfrac{t+2}{\sqrt{t^2+4t+3}}>0,\forall t\in\left[\dfrac{1}{2};1\right]$ do đó $g(t)$ đồng biến trên $\left[\dfrac{1}{2};1\right]$.\\
		$\begin{aligned}
			&\underset{\left[\dfrac{1}{2};1\right]}{\min}g(t)=g\left(\dfrac{1}{2}\right)=\dfrac{\sqrt{21}}{2}=\underset{\left[\dfrac{\pi}{6};\dfrac{\pi}{2}\right]}{\min}f(x)=m\\ 
			&\underset{\left[\dfrac{1}{2};1\right]}{\max}g(t)=g(1)=2\sqrt{2}=\underset{\left[\dfrac{\pi}{6};\dfrac{\pi}{2}\right]}{\max}f(x)=M\\ 
		\end{aligned}$
	}
\end{ex}
\begin{ex}
	[Chuyên Quốc Học Huế - 2021]%Câu 53
	Cho hàm số $ f(x)$có đạo hàm liên tục trên $ R$thỏa mãn: $ f'(x)=f(x)+\mathrm{e}^x.\cos 2021x$ và $ f(0)=0$ Đồ thi hàm số $ y=f(x)$cắt trục hoành tại bao nhiêu điểm có hoành độ thuộc đoạn $\left[-1;1\right]$?
	\choice
	{$3$}
	{$1$}
	{\True $1287$}
	{$4043$}
	\loigiai{
		Ta có phương trình trên tương đương với\\
		$\begin{aligned}
			&\Rightarrow f'(x)=f(x)+\mathrm{e}^x.\cos 2021x\Leftrightarrow f'(x)-f(x)=\mathrm{e}^x.\cos 2021x\\ 
			&\Leftrightarrow{\mathrm{e}^{-x}}f'(x)+\left(-\mathrm{e}^{-x}\right)f(x)=\cos 2021x\\ 
		\end{aligned}$\\
		Đến đây ta nguyên hàm hai vế thu được:\\
		$\Rightarrow{\left(\mathrm{e}^{-x}f(x)\right)'}=\cos 2021x\Leftrightarrow{\mathrm{e}^{-x}}f(x)=\displaystyle\int{\cos 2021x\mathrm{d}x}=\dfrac{\sin 2021x}{2021}+C$\\
		Mà $ f(0)=0$nên $ C=0$suy ra $\mathrm{e}^{-x}f(x)=\dfrac{\sin 2021x}{2021}\Rightarrow f(x)=\dfrac{\mathrm{e}^x.\sin 2021x}{2021}$\\
		Phương trình hoành độ giao điểm của đồ thị hàm số $ y=f(x)$ và trục hoành là\\
		$f(x)=0\Leftrightarrow\dfrac{\mathrm{e}^x.\sin 2021x}{2021}=0\Leftrightarrow\sin 2021x=0\Leftrightarrow 2021x=k\pi ,k\in Z\Rightarrow x=\dfrac{k\pi}{2021},\left(k\in Z\right)$\\
		Vì $ x\in\left[-1;1\right]$ nên $-1\le\dfrac{k\pi}{2021}\le 1\Rightarrow\dfrac{-2021}{\pi}\le k\le\dfrac{2021}{\pi}$\\
		Mà do $ k\in Z$nên suy ra $ k\in\left\{-643;-642;\ldots;643\right\}$ như vậy ta kết luận đồ thi hàm số $ y=f(x)$cắt trục hoành tại 1287 điểm có hoành độ thuộc đoạn $\left[-1;1\right]$.
	}
\end{ex}
\begin{ex}
	[Chuyên Lê Quý Đôn - Điện Biên - 2022]%Câu 54
	Cho hàm số $ f(x)$ thoả mãn $ f(2)=-\dfrac{1}{25}$ và $f'(x)=4x^3\left[f(x)\right]^2$ vói mọi $ x\in\mathbb{R}$. Giá trị của $ f(1)-f(0)$ bằng
	\choice
	{\True $\dfrac{1}{90}$}
	{$-\dfrac{1}{90}$}
	{$-\dfrac{1}{72}$}
	{$\dfrac{1}{72}$}
	\loigiai{
		Ta có $f'(x)=4x^3\left[f(x)\right]^2\Rightarrow\dfrac{f'(x)}{\left[f(x)\right]^2}=4x^3$\\
		$\Leftrightarrow\displaystyle\int{\left(\dfrac{f'(x)}{\left[f(x)\right]^2}\right)\mathrm{d}x}=\displaystyle\int{4x^3\mathrm{d}x}\Leftrightarrow-\dfrac{1}{f(x)}=x^4+C$\\
		$\Leftrightarrow-\dfrac{1}{f(x)}=x^4+C\Rightarrow f(x)=-\dfrac{1}{x^4+C}$\\
		Với $f(2)=-\dfrac{1}{25}\Leftrightarrow f(2)=-\dfrac{1}{16+C}=-\dfrac{1}{25}\Rightarrow C=9$\\
		$\Rightarrow f(x)=-\dfrac{1}{x^4+9}\Rightarrow f(1)-f(0)=-\dfrac{1}{10}+\dfrac{1}{9}=\dfrac{1}{90}$.
	}
\end{ex}
\begin{ex}
	[Cụm Trường Nghệ An - 2022]%Câu 55
	Cho hàm số $y=f(x)$ có đạo hàm liên tục trên $\mathbb{R}$ và thoả mãn $f'(x)-2f(x)=\left(x^2+1\right){\mathrm{e}^{\frac{x^2+4x-1}{2}}},\forall x\in\mathbb{R}$ và $f(1)=\mathrm{e}^2$ . Biết $f(3)=a.\mathrm{e}^b+c$ với $a,b,c\in\mathbb{N}$ . Tính $2a+3b+4c.$
	\choice
	{\True $ 36$}
	{$ 30$}
	{$ 24$}
	{$ 32$}
	\loigiai{
		Ta có $f'(x)-2f(x)=\left(x^2+1\right){\mathrm{e}^{\frac{x^2+4x-1}{2}}},\forall x\in\mathbb{R}$.\\
		Nhân 2 vế cho $\mathrm{e}^{-2x}$ ta được $\mathrm{e}^{-2x}.f'(x)-2.\mathrm{e}^{-2x}.f(x)=\left(x^2+1\right){\mathrm{e}^{\frac{x^2-1}{2}}}$.\\
		$\Rightarrow{\mathrm{e}^{-2x}}.f'(x)+\left(\mathrm{e}^{-2x}\right)'.f(x)=\left(x^2+1\right){\mathrm{e}^{\frac{x^2-1}{2}}}\Rightarrow{\left(\left(\mathrm{e}^{-2x}\right).f(x)\right)'}=\left(x^2+1\right){\mathrm{e}^{\frac{x^2-1}{2}}}$\\
		Lấy nguyên hàm 2 vế ta được\\
		$\Rightarrow\left(\mathrm{e}^{-2x}\right).f(x)=\displaystyle\int{\left(x^2+1\right){\mathrm{e}^{\frac{x^2-1}{2}}}\mathrm{d}x+C}$\\
		$\Rightarrow\left(\mathrm{e}^{-2x}\right).f(x)=\displaystyle\int{\left(x^2+1\right){\mathrm{e}^{\frac{x^2-1}{2}}}\mathrm{d}x+C}$\\
		Đặt $H=\displaystyle\int{\left(x^2+1\right){\mathrm{e}^{\frac{x^2-1}{2}}}\mathrm{d}x}$\\
		$H=\displaystyle\int{\left(x^2+1\right){\mathrm{e}^{\frac{x^2-1}{2}}}\mathrm{d}x}=\displaystyle\int{x^2\mathrm{e}^{\frac{x^2-1}{2}}\mathrm{d}x}+\displaystyle\int{\mathrm{e}^{\frac{x^2-1}{2}}\mathrm{d}x}$ $(*)$\\
		Ta tìm $\displaystyle\int{\mathrm{e}^{\frac{x^2-1}{2}}\mathrm{d}x}$ bằng phương pháp nguyên hàm từng phần.\\
		Đặt $\left\{\begin{aligned}
			& u=\mathrm{e}^{\frac{x^2-1}{2}}\\ 
			&\text{dv}=\mathrm{d}x\\ 
		\end{aligned}\right.$ ta có: $\left\{\begin{aligned}
			&\mathrm{d}u=x.\mathrm{e}^{\frac{x^2-1}{2}}\mathrm{d}x\\ 
			&\text{v}=x\\ 
		\end{aligned}\right.$\\
		Suy ra: $\displaystyle\int{\mathrm{e}^{\frac{x^2-1}{2}}\mathrm{d}x}=x{\mathrm{e}^{\frac{x^2-1}{2}}}-\displaystyle\int{x^2\mathrm{e}^{\frac{x^2-1}{2}}\mathrm{d}x}$\\
		Thế vào $(*)$ ta được: $H=\displaystyle\int{\left(x^2+1\right){\mathrm{e}^{\frac{x^2-1}{2}}}\mathrm{d}x}=\displaystyle\int{x^2\mathrm{e}^{\frac{x^2-1}{2}}\mathrm{d}x}+x{\mathrm{e}^{\frac{x^2-1}{2}}}-\displaystyle\int{x^2\mathrm{e}^{\frac{x^2-1}{2}}\mathrm{d}x}=x{\mathrm{e}^{\frac{x^2-1}{2}}}+C$\\
		$\Rightarrow\left(\mathrm{e}^{-2x}\right).f(x)=x{\mathrm{e}^{\frac{x^2-1}{2}}}+C$\\
		Mà $f(1)=\mathrm{e}^2$\\
		Ta cho $ x=1$ $\Rightarrow\left(\mathrm{e}^{-2}\right).f(1)=1.\mathrm{e}^0+C$ $\Rightarrow\left(\mathrm{e}^{-2}\right).\mathrm{e}^2=1.\mathrm{e}^0+C\Rightarrow C=0$ \\
		$\Rightarrow\left(\mathrm{e}^{-2x}\right).f(x)=x{\mathrm{e}^{\frac{x^2-1}{2}}}$\\
		Để tính $ f(3)$ ta chọn $ x=3$ $\Rightarrow\left(\mathrm{e}^{-6}\right).f(3)=3\mathrm{e}^4\Rightarrow f(3)=3\mathrm{e}^{10}=a.\mathrm{e}^b+c$\\
		Suy ra $\left\{\begin{aligned}
			& a=3\\ 
			& b=10\\ 
			& c=0\\ 
		\end{aligned}\right.$ $\Rightarrow 2a+3b+4c=36$.
	}
\end{ex}
\begin{ex}
	[THPT Hương Sơn - Hà Tĩnh - 2022]%Câu 56
	Cho hàm số $ y=f(x)$ liên tục, nhận giá trị dương trên $\left(0;+\infty\right)$ và thỏa mãn $ f(1)=2$; $f'(x)=\dfrac{x^2}{\left[f(x)\right]^2}$ với mọi $ x\in\left(0;+\infty\right)$. Giá trị $ f(3)$ bằng
	\choice
	{\True $\sqrt[3]{34}$}
	{$ 34$}
	{$ 3$}
	{$\sqrt[3]{20}$}
	\loigiai{
		Ta có $f'(x)=\dfrac{x^2}{\left[f(x)\right]^2}\Leftrightarrow{f}'(x).\left[f(x)\right]^2=x^2$ với mọi $ x\in\left(0;+\infty\right)$.\\
		Lấy nguyên hàm hai vế ta được:\\
		$\displaystyle\int{f'(x).\left[f(x)\right]^2}\mathrm{d}x=\displaystyle\int{x^2}\mathrm{d}x\Leftrightarrow\dfrac{f^3(x)}{3}=\dfrac{x^3}{3}+C$.\\
		Theo đề bài $ f(1)=2$ nên ta có: $\dfrac{8}{3}=\dfrac{1}{3}+C\Leftrightarrow C=\dfrac{8}{3}-\dfrac{1}{3}=\dfrac{7}{3}.$\\
		Khi đó $ f(x)=\sqrt[3]{x^3+7}\Rightarrow f(3)=\sqrt[3]{34}$.
	}
\end{ex}
\begin{ex}
	[Sở Thanh Hóa 2022]%Câu 57
	Cho hàm số $f(x)\neq 0,\forall x>0$ và có đạo hàm $f'(x)$ liên tục trên khoảng $(0 ;+\infty)$ thoả mãn $f'(x)=(2 x+1) f^2(x)$, $\forall x>0$ và $f(1)=-\dfrac{1}{2}$. Giá trị của biểu thức $f(1)+f(2)+f(3)+\ldots+f(2022)$ bằng
	\choice
	{$\dfrac{2022}{2023}$}
	{$\dfrac{2021}{2022}$}
	{$-\dfrac{2021}{2022}$}
	{\True $-\dfrac{2022}{2023}$}
	\loigiai{
		Có $\dfrac{f'(x)}{f^2(x)}=2 x+1\Rightarrow\displaystyle\int\dfrac{f'(x)}{f^2(x)}\mathrm{d}x=\displaystyle\int(2 x+1)\mathrm{d}x\Leftrightarrow-\dfrac{1}{f(x)}=x^2+x+C$.\\
		Có $f(1)=-\dfrac{1}{2}\Rightarrow-\dfrac{1}{-\dfrac{1}{2}}=1+1+C\Leftrightarrow C=0\Rightarrow f(x)=-\dfrac{1}{x^2+x}$.\\
		Vì vậy $f(x)=-\left(\dfrac{1}{x}-\dfrac{1}{x+1}\right)\Rightarrow\sum_{k=1}^{20122}f(k)=-\sum_{k=1}^{2022}\left(\dfrac{1}{k}-\dfrac{1}{k+1}\right)=-\left(\dfrac{1}{1}-\dfrac{1}{2023}\right)=-\dfrac{2022}{2023}$.
	}
\end{ex}
\begin{ex}
	[Sở Lạng Sơn 2022]%Câu 58
	Cho hàm số $ y=f(x)$ có đạo hàm liên tục trên $\mathbb{R}$ thỏa mãn $ f(1)=e$ và $f'(x)+f(x)=x,x\in\mathbb{R}$. Giá trị $ f(2)$ bằng 
	\choice
	{$\dfrac{2}{e}$}
	{$ 1-\dfrac{1}{e}$}
	{$ 1+\dfrac{1}{e}$}
	{\True $ 2$}
	\loigiai{
		Ta có\\
		$f'(x)+f(x)=x\Leftrightarrow{f}'(x).\mathrm{e}^x+f(x).\mathrm{e}^x=x{\mathrm{e}^x}\Leftrightarrow{\left(\mathrm{e}^x.f(x)\right)'}=x{\mathrm{e}^x}$.\\
		Nên $\displaystyle\int{\left(\mathrm{e}^x.f(x)\right)'\mathrm{d}x}=\displaystyle\int{x{\mathrm{e}^x}\mathrm{d}x}\Leftrightarrow{\mathrm{e}^x}.f(x)=x{\mathrm{e}^x}-\mathrm{e}^x+C\Rightarrow f(x)=\dfrac{x{\mathrm{e}^x}-\mathrm{e}^x+C}{\mathrm{e}^x}$.\\
		Do$ f(1)=e\Rightarrow f(1)=\dfrac{\mathrm{e}^1-\mathrm{e}^1+C}{\mathrm{e}^1}\Rightarrow C=\mathrm{e}^2$.\\
		Suy ra $ f(x)=\dfrac{x{\mathrm{e}^x}-\mathrm{e}^x+\mathrm{e}^2}{\mathrm{e}^x}$\\
		$ f(2)=\dfrac{2.\mathrm{e}^2-\mathrm{e}^2+\mathrm{e}^2}{\mathrm{e}^2}=2$.
	}
\end{ex}
\begin{ex}
	[Sở Lạng Sơn 2022]%Câu 59
	Giả sử hàm số $y=f(x)$ liên tục, nhận giá trị dương trên $\left(0;+\infty\right)$ và thỏa mãn $f(1)=\mathrm{e}$, $f(x)=f'(x).\sqrt{3x+1}$, với mọi $x>0$. Mệnh đề nào sau đây đúng?
	\choice
	{$ 3<f(5)<4$}
	{$ 11<f(5)<12$}
	{\True $ 10<f(5)<11$}
	{$ 4<f(5)<5$}
	\loigiai{
		Do hàm số $y=f(x)$ liên tục, nhận giá trị dương trên $\left(0;+\infty\right)$ nên $f(x)=f'(x).\sqrt{3x+1}$\\
		$\Leftrightarrow\dfrac{f'(x)}{f(x)}=\dfrac{1}{\sqrt{3x+1}}\Leftrightarrow\displaystyle\int{\dfrac{f'(x)}{f(x)}\mathrm{d}x}=\displaystyle\int{\dfrac{1}{\sqrt{3x+1}}\mathrm{d}x\Leftrightarrow\ln f(x)}=\dfrac{2}{3}\sqrt{3x+1}+C(*)$\\
		Ta có $f(1)=e$ nên $(*)\Leftrightarrow 1=\dfrac{4}{3}+C\Leftrightarrow C=-\dfrac{1}{3}\left(**\right)$\\
		Từ $(*)$ và $\left(**\right)$ suy ra $f(x)=\mathrm{e}^{\frac{2}{3}\sqrt{3x+1}-\dfrac{1}{3}}\Rightarrow f(5)=\mathrm{e}^{\frac{7}{3}}\approx 10,31$.
	}
\end{ex}
\begin{ex}
	[Sở Phú Thọ 2022]%Câu 60
	Cho hàm số $ y=f(x)$ liên tục trên $ \mathbb{R}\setminus\left\{-2;0\right\}$ thỏa mãn $ x\left(x+2\right).f'(x)+2f(x)=x^2+2x$ và $ f(1)=-6\ln 3$. Biết $ f(3)=a+b\ln 5\left(a,b\in\mathbb{Q}\right)$. Giá trị $ a-b$ bằng?
	\choice
	{$ 20$}
	{$ 10$}
	{$\dfrac{10}{3}$}
	{\True $\dfrac{20}{3}$}
	\loigiai{
		Xét $ x\left(x+2\right).f'(x)+2f(x)=x^2+2x$ chia hai vế cho $\left(x+2\right)^2$ ta được\\
		$\dfrac{x{f}'(x)}{x+2}+\dfrac{2f(x)}{\left(x+2\right)^2}=\dfrac{x}{x+2}\Leftrightarrow{\left[\dfrac{x}{x+2}f(x)\right]'}=\dfrac{x}{x+2}$.\\
		Lấy nguyên hàm hai vế ta được\\
		$\dfrac{x}{x+2}f(x)=\displaystyle\int{\dfrac{x}{x+2}\mathrm{d}x\Leftrightarrow}\dfrac{x}{x+2}f(x)=x-2\ln\left| x+2\right|+C$.\\
		Mà $ f(1)=-6\ln 3$ nên ta có $\dfrac{-6\ln 3}{3}=1-2\ln 3+C\Leftrightarrow C=-1$.\\
		Khi đó $\dfrac{x}{x+2}f(x)=x-\ln\left| x+2\right|-1\Rightarrow\dfrac{3}{5}f(3)=2-2\ln 5\Rightarrow f(3)=\dfrac{10}{3}-\dfrac{10}{3}\ln 5$.\\
		Vậy $ a-b=\dfrac{10}{3}+\dfrac{10}{3}=\dfrac{20}{3}$.
	}
\end{ex}
\begin{ex}
	[Sở Vĩnh Phúc 2022]%Câu 61
	Cho hàm số $f(x)$ có đạo hàm trên $\mathbb{R}$, thoả mãn $ f(x)>-1$ và $f'(x)\sqrt{x^2+1}=2x\sqrt{f(x)+1},\forall x\in\mathbb{R}$ . Biết rằng $ f(0)=0$, khi đó $ f(2)$có giá trị bằng
	\choice
	{$0$}
	{\True $4$}
	{$8$}
	{$6$}
	\loigiai{
		Ta có $f'(x)\sqrt{x^2+1}=2x\sqrt{f(x)+1}\Leftrightarrow\dfrac{f'(x)}{\sqrt{f(x)+1}}=\dfrac{2x}{\sqrt{x^2+1}}$\\
		Nên $\displaystyle\int{\dfrac{f'(x)}{\sqrt{f(x)+1}}\mathrm{d}x}=\displaystyle\int{\dfrac{2x}{\sqrt{x^2+1}}\mathrm{d}x}$ $\Leftrightarrow\displaystyle\int{\dfrac{\mathrm{d}\left(f(x)\right)}{\sqrt{f(x)+1}}}=\displaystyle\int{\dfrac{1}{\sqrt{x^2+1}}\mathrm{d}\left(x^2+1\right)}$\\
		$\Leftrightarrow 2\sqrt{f(x)+1}=2\sqrt{x^2+1}+C$.\\
		Mà $ f(0)=0\Rightarrow C=0$ $\Rightarrow\sqrt{f(x)+1}=\sqrt{x^2+1}$ nên $ f(x)=x^2\Rightarrow f(2)=4$.
	}
\end{ex}
\begin{ex}
	[Chuyên Nguyễn Trãi – Hải Dương – 2022]%Câu 62
	Cho hàm số $y=f(x)$ có đồ thị $(C), f(x)$ có đạo hàm xác định và liên tục trên khoảng $(0 ;+\infty)$ thỏa mãn điều kiện $f'(x)=\ln x\cdot f^2(x),\forall x\in(0 ;+\infty)$. Biết $f(x)\neq 0,\forall x\in(0 ;+\infty)$ và $f(e)=2$. Viết phương trình tiếp tuyến với đồ thị $(C)$ tại điểm có hoành độ $x=1$.
	\choice
	{$y=-\dfrac{2}{3}x+2$}
	{$y=-\dfrac{2}{3}$}
	{$y=\dfrac{2}{3}x+1$}
	{\True $y=\dfrac{2}{3}$}
	\loigiai{
		Ta có $f'(x)=\ln x\cdot f^2(x)\Leftrightarrow\dfrac{f'(x)}{f^2(x)}=\ln x\Leftrightarrow\left(\dfrac{-1}{f(x)}\right)^2=\ln x$\\
		$\Rightarrow\dfrac{-1}{f(x)}=\displaystyle\int{\ln}x\mathrm{d}x=x\ln x-x+C$\\
		Với $x=\mathrm{e}$ ta có $\dfrac{-1}{f(e)}=e\ln e-e+C$ mà $f(e)=2\Rightarrow\dfrac{-1}{2}=C$\\
		Suy ra $f(x)=\dfrac{-1}{x\ln x-x-\dfrac{1}{2}}$\\
		Khi đó $\left\{\begin{array}{*{35}{l}}
			\begin{aligned}
				& f(1)=\dfrac{2}{3}\\ 
				& f'(1)=\ln 1\cdot{f^2}(1)=0\\ 
			\end{aligned}\\
		\end{array}\right.$\\
		Phương trình tiếp tuyến với đồ thị $(C)$ tại điểm có hoành độ $x=1$ là\\
		$ y=f'(x)(x-1)+f(1)=\dfrac{2}{3}$.
	}
\end{ex}
\begin{ex}
	[THPT Trần Phú – Hà Tĩnh – 2022]%Câu 63
	Cho hàm số $y=f(x)>0$ liên tục trên $\mathbb{R}$ và $f(1)=\mathrm{e}^3$. Biết $f'(x)=(2 x-3) f(x),\forall x\in\mathbb{R}$. Hỏi phương trình $f(x)=\mathrm{e}^{2 x^4-3 x+4}$ có bao nhiêu nghiệm?
	\choice
	{$4$}
	{$3$}
	{\True $2$}
	{$0$}
	\loigiai{
		$\begin{aligned}
			& f'(x)=(2x-3)f(x)\Rightarrow\dfrac{f'(x)}{f(x)}=2x-3\\ 
			&\Rightarrow\displaystyle\int{\dfrac{f'(x)}{f(x)}}\mathrm{d}x=\displaystyle\int{(2x-3)}\mathrm{d}x\Rightarrow\ln f(x)=x^2-3x+C,(f(x)>0);f(1)=\mathrm{e}^3\\ 
			&\Rightarrow 3=C-2\Leftrightarrow C=5\Rightarrow\ln f(x)=x^2-3x+5\\ 
			&\Leftrightarrow f(x)=\mathrm{e}^{x^2-3x+5}\Rightarrow f(x)=\mathrm{e}^{2x^4-3x+4}\\ 
			&\Leftrightarrow{\mathrm{e}^{x^2-3x+5}}=\mathrm{e}^{2x^4-3x+4}\Leftrightarrow 2x^4-3x+4=x^2-3x+5\Leftrightarrow x=\pm 1.\\ 
		\end{aligned}$
	}
\end{ex}
\begin{ex}
	[THPT Kim Liên - Hà Nội - 2022]%Câu 64
	Cho hàm số $ y=f(x)$ thỏa mãn $ f(x)>0,\forall x>\dfrac{1}{2}$ và có đạo hàm $f'(x)$ liên tục trên khoảng $\left(\dfrac{1}{2};+\infty\right)$ thỏa mãn $f'(x)+8x{f^2}(x)=0,\forall x>\dfrac{1}{2}$ và $ f(1)=\dfrac{1}{3}$. Tính $ f(1)+f(2)+\ldots+f\left(1011\right)$.
	\choice
	{\True $\dfrac{1}{2}.\dfrac{2022}{2023}$}
	{$\dfrac{2021}{2043}$}
	{$\dfrac{2022}{4045}$}
	{$\dfrac{1}{2}.\dfrac{2021}{2022}$}
	\loigiai{
		Ta có $f'(x)+8x{f^2}(x)=0\Leftrightarrow\dfrac{f'(x)}{f^2(x)}=-8x$.\\
		Lấy nguyên hàm hai vế ta được\\
		$\displaystyle\int{\dfrac{f'(x)}{f^2(x)}}\mathrm{d}x=\displaystyle\int{-8x}\mathrm{d}x\Leftrightarrow\displaystyle\int{\dfrac{1}{f^2(x)}}d\left(f(x)\right)=\displaystyle\int{-8x}\mathrm{d}x\Leftrightarrow-\dfrac{1}{f(x)}=-4x^2+C$.\\
		Mà $ f(1)=\dfrac{1}{3}\Rightarrow-3=-4+C\Leftrightarrow C=1$.\\
		Khi đó $ f(x)=\dfrac{1}{4x^2-1}=\dfrac{1}{2}\left[\dfrac{1}{2x-1}-\dfrac{1}{2x+1}\right]$.\\
		$f(1)=\dfrac{1}{2}\left[1-\dfrac{1}{3}\right],f(2)=\dfrac{1}{2}\left[\dfrac{1}{3}-\dfrac{1}{5}\right],f(3)=\dfrac{1}{2}\left[\dfrac{1}{5}-\dfrac{1}{7}\right]\ldots f\left(1011\right)=\dfrac{1}{2}\left[\dfrac{1}{2001}-\dfrac{1}{2023}\right].$\\
		Vậy $ f(1)+f(2)+\ldots+f\left(1011\right)=\dfrac{1}{2}\left(1-\dfrac{1}{2023}\right)=\dfrac{1}{2}.\dfrac{2022}{2023}$.
	}
\end{ex}
\begin{ex}
	[THPT Yên Lạc - Vĩnh Phúc - 2022]%Câu 65
	Cho hàm số $ y=f(x)$ liên tục trên $\left(0;+\infty\right)$ thỏa mãn $ 2x.f'(x)+f(x)=3x^2\sqrt{x},\forall x\in\left(0;+\infty\right)$. Biết $ f(1)=\dfrac{1}{2}$, tính $ f(4)$.
	\choice
	{$ 14$}
	{$4$}
	{$24$}
	{\True $16$}
	\loigiai{
		$2x.f'(x)+f(x)=3x^2\sqrt{x},\forall x\in\left(0;+\infty\right)$\\
		$\Leftrightarrow\sqrt{x}.f'(x)+\dfrac{1}{2\sqrt{x}}f(x)=\dfrac{3x^2}{2},\forall x\in\left(0;+\infty\right)$\\
		$\Leftrightarrow\sqrt{x}.f'(x)+\left(\sqrt{x}\right)'.f(x)=\dfrac{3x^2}{2},\forall x\in\left(0;+\infty\right)$\\
		$\Leftrightarrow{\left[\sqrt{x}.f(x)\right]'}=\dfrac{3x^2}{2},\forall x\in\left(0;+\infty\right)$\\
		$\Leftrightarrow\sqrt{x}.f(x)=\displaystyle\int{\dfrac{3x^2}{2}\mathrm{d}x=\dfrac{x^3}{2}}+C\Leftrightarrow\sqrt{x}.f(x)=\dfrac{x^3}{2}+C(*)$\\
		Thay $ x=1$ vào $(*)$ ta được $ f(1)=\dfrac{1}{2}+C\Rightarrow C=f(1)-\dfrac{1}{2}=\dfrac{1}{2}-\dfrac{1}{2}=0\Rightarrow f(x)=\dfrac{x^3}{2\sqrt{x}}$\\
		Vậy $ f(4)=16$.
	}
\end{ex}
\begin{ex}
	[Chuyên Hà Tĩnh 2022]%Câu 66
	Cho hàm số $y=f(x)$ liên tục trên $\left(0;+\infty\right)$ thỏa mãn $2x.f'(x)+f(x)=4x\sqrt{x}$. Biết $f(1)=2$. Giá trị của $f(4)$ bằng
	\choice
	{$\dfrac{15}{4}$}
	{$\dfrac{17}{4}$}
	{$\dfrac{15}{2}$}
	{\True $\dfrac{17}{2}$}
	\loigiai{
		Ta có $ 2x{f}'(x)+f(x)=4x\sqrt{x}\Leftrightarrow \sqrt{2x}{f}'(x)+\dfrac{f(x)}{\sqrt{2x}}=2\sqrt{2}x$\\
		$\Leftrightarrow{\left(\sqrt{2x}.f(x)\right)'}=2\sqrt{2}x$\\
		Lấy nguyên hàm hai vế ta được $\sqrt{2x}.f(x)=\sqrt{2}{x^2}+C$.\\
		Với $ f(1)=2\Rightarrow 2\sqrt{2}=\sqrt{2}+C\Rightarrow C=\sqrt{2}\Rightarrow f(x)=\dfrac{\sqrt{2}{x^2}+\sqrt{2}}{\sqrt{2x}}=\dfrac{x^2+1}{\sqrt{x}}$.\\
		Vậy $ f(4)=\dfrac{17}{2}$.
	}
\end{ex}
\begin{ex}
	[Chuyên Lê Khiết - Quảng Ngãi 2022]%Câu 67
	Cho hàm số $ f(x)>0$ có đạo hàm liên tục trên $\mathbb{R}$, thỏa mãn $\left(x+1\right){f}'(x)=\dfrac{\sqrt{f(x)}}{x+2}$ và $ f(0)=\left(\dfrac{\ln 2}{2}\right)^2$. Giá trị $ f(3)$ bằng
	\choice
	{$ 4\left(4\ln 2-\ln 5\right)^2$}
	{$ 2\left(4\ln 2-\ln 5\right)^2$}
	{$\dfrac{1}{2}{\left(4\ln 2-\ln 5\right)^2}$}
	{\True $\dfrac{1}{4}{\left(4\ln 2-\ln 5\right)^2}$}
	\loigiai{
		Ta có\\
		$\left(x+1\right){f}'(x)=\dfrac{\sqrt{f(x)}}{x+2}\Leftrightarrow\dfrac{f'(x)}{\sqrt{f(x)}}=\dfrac{1}{\left(x+2\right)\left(x+1\right)}$.\\
		$\begin{aligned}
			&\Rightarrow\displaystyle\int{\dfrac{f'(x)}{\sqrt{f(x)}}\mathrm{d}x=\displaystyle\int{\dfrac{1}{\left(x+2\right)\left(x+1\right)}\mathrm{d}x}}\\ 
			&\Leftrightarrow\displaystyle\int{\dfrac{d\left(f(x)\right)}{\sqrt{f(x)}}=\ln\left|\dfrac{x+1}{x+2}\right|}+C\\ 
			&\Leftrightarrow 2\sqrt{f(x)}=\ln\left|\dfrac{x+1}{x+2}\right|+C\\ 
		\end{aligned}$\\
		Mà $ f(0)=\left(\dfrac{\ln 2}{2}\right)^2\Rightarrow C=2\ln 2$\\
		$\Rightarrow 2\sqrt{f(3)}=\ln 4-ln5+2ln2\Leftrightarrow 2\sqrt{f(3)}=4\ln 2-\ln 5\Leftrightarrow f(3)=\dfrac{1}{4}{\left(4\ln 2-\ln 5\right)^2}$.
	}
\end{ex}
\begin{ex}
	[Sở Hà Nam 2022]%Câu 68
	Cho hàm số $ f(x)$ liên tục và thỏa mãn $ f(x)>0,\forall x\in\left(1;3\right)$. Biết rằng $\mathrm{e}^{2x}.f^{3}(x)+1=3\mathrm{e}^x.f'(x).\sqrt{f(x)}$, $\forall x\in\left(1;3\right)$ và $ f(2)=\mathrm{e}^{-\dfrac{4}{3}}$, khi đó giá trị của $ f\left(\dfrac{3}{2}\right)$ thuộc khoảng nào dưới đây?
	\choice
	{$\left(\dfrac{1}{3};\dfrac{1}{2}\right)$}
	{\True $\left(0;\dfrac{1}{3}\right)$}
	{$\left(\dfrac{1}{2};\dfrac{2}{3}\right)$}
	{$\left(\dfrac{2}{3};1\right)$}
	\loigiai{
		Ta có $\mathrm{e}^{2x}.f^3(x)+1=3\mathrm{e}^x.f'(x).\sqrt{f(x)}\Leftrightarrow{\mathrm{e}^{2x}}.f^3(x)+1=2\mathrm{e}^x.\left(\sqrt{f^3(x)}\right)'$\\
		$\Leftrightarrow{\mathrm{e}^{2x}}.f^3(x)+1=2\left[\left(\mathrm{e}^x.\sqrt{f^3(x)}\right)'-\mathrm{e}^x.\sqrt{f^3(x)}\right]\Leftrightarrow{\left(\mathrm{e}^x.\sqrt{f^3(x)}+1\right)^2}=2\left(\mathrm{e}^x.\sqrt{f^3(x)}\right)'$ $\Leftrightarrow\dfrac{\left(\mathrm{e}^x.\sqrt{f^3(x)}+1\right)'}{\left(\mathrm{e}^x.\sqrt{f^3(x)}+1\right)^2}=\dfrac{1}{2}\Rightarrow\displaystyle\int{\dfrac{\left(\mathrm{e}^x.\sqrt{f^3(x)}+1\right)'}{\left(\mathrm{e}^x.\sqrt{f^3(x)}+1\right)^2}\mathrm{d}x}=\dfrac{1}{2}\displaystyle\int{\mathrm{d}x}$ $\Leftrightarrow\dfrac{-1}{\mathrm{e}^x.\sqrt{f^3(x)}+1}=\dfrac{1}{2}x+C(*)$ .\\
		Vì $ f(2)=\mathrm{e}^{\frac{-4}{3}}$ nên $(*)\Leftrightarrow\dfrac{-1}{2}=1+C\Leftrightarrow C=\dfrac{-3}{2}$\\
		+Do đó: $\dfrac{-1}{\mathrm{e}^x.\sqrt{f^3(x)}+1}=\dfrac{1}{2}x-\dfrac{3}{2}\Rightarrow f(x)=\sqrt[3]{\left(\dfrac{1-x}{\left(x-3\right).\mathrm{e}^x}\right)^2}$. Suy ra $ f\left(\dfrac{3}{2}\right)\approx 0,18\in\left(0;\dfrac{1}{3}\right)$.
	}
\end{ex}
\begin{ex}
	[Sở Hà Nam 2022]%Câu 69
	Cho hàm số $ f(x)$ thoả mãn $ f\left(\dfrac{\pi}{2}\right)=1$ và $f'(x)=\cos x\left(6\sin^2x-1\right),\forall x\in\mathbb{R}$. Biết $ F(x)$ là nguyên hàm của $ f(x)$ thoả mãn $ F(0)=\dfrac{2}{3}$, khi đó $ F\left(\dfrac{\pi}{2}\right)$ bằng 
	\choice
	{$\dfrac{1}{3}$}
	{$-\dfrac{2}{3}$}
	{\True $ 1$}
	{$ 0$}
	\loigiai{
		Ta có $f(x)=\displaystyle\int{f'(x)\mathrm{d}x}=\displaystyle\int{\cos x\left(6\sin^2x-1\right)\mathrm{d}x}=\displaystyle\int{\left(6\sin^2x\cos x-\cos x\right)\mathrm{d}x}$\\
		$=6\displaystyle\int{\sin^2x\cos x\mathrm{d}x}-\sin x+C$\\
		Đặt $t=\sin x\Rightarrow\mathrm{d}t=\cos x\mathrm{d}x$\\
		Suy ra $f(x)=6\displaystyle\int{t^2\mathrm{d}t}-\sin x+C=2t^3-\sin x+C=2\sin^3x-\sin x+C$\\
		Mà $f\left(\dfrac{\pi}{2}\right)=1\Leftrightarrow 2\sin^3\left(\dfrac{\pi}{2}\right)-\sin\left(\dfrac{\pi}{2}\right)+C=1\Leftrightarrow C=0\Rightarrow f(x)=2\sin^3x-\sin x$\\
		Ta có $F(x)=\displaystyle\int{f(x)\mathrm{d}x}=\displaystyle\int{\left(2\sin^3x-\sin x\right)\mathrm{d}x}=2\displaystyle\int{\left(1-\cos^2x\right)\sin x\mathrm{d}x}+\cos x+C'$\\
		Đặt $u=\cos x\Rightarrow\mathrm{d}u=-\sin x\mathrm{d}x$\\
		Suy ra $F(x)=-2\displaystyle\int{\left(1-u^2\right)\mathrm{d}u}+\cos x+C'=-2\left(u-\dfrac{u^3}{3}\right)+\cos x+C'$\\
		$=-2\cos x+\dfrac{2}{3}{\cos^3}x-\cos x+C'=\dfrac{2}{3}{\cos^3}x-\cos x+C'$\\
		Mà $F(0)=\dfrac{2}{3}\Leftrightarrow\dfrac{2}{3}{\cos^3}0-\cos 0+C'=\dfrac{2}{3}\Leftrightarrow{C}'=1\Rightarrow F(x)=\dfrac{2}{3}{\cos^3}x-\cos x+1$\\
		Vậy $F\left(\dfrac{\pi}{2}\right)=\dfrac{2}{3}{\cos^3}\left(\dfrac{\pi}{2}\right)-\cos\left(\dfrac{\pi}{2}\right)+1=1$.
	}
\end{ex}
\begin{ex}
	[Sở KonTum 2022]%Câu 70
	Cho hàm số $f(x)$ thỏa mãn $f(x)+f'(x)=\mathrm{e}^{-x},\forall x\in\mathbb{R}$ và $ f(0)=2$. Họ nguyên hàm của hàm số $ f(x){\mathrm{e}^{2x}}$ là
	\choice
	{$ x{\mathrm{e}^x}+x+C$}
	{\True $\left(x+1\right){\mathrm{e}^x}+C$}
	{$ x{\mathrm{e}^{-x}}+x+C$}
	{$\left(x-1\right){\mathrm{e}^x}+C$}
	\loigiai{
		Ta có $ f(x)+f'(x)=\mathrm{e}^{-x}\Leftrightarrow{\mathrm{e}^x}f(x)+\mathrm{e}^xf'(x)=1\Leftrightarrow\left(\mathrm{e}^xf(x)\right)'=1\Leftrightarrow\displaystyle\int{\left(\mathrm{e}^xf(x)\right)'\mathrm{d}x=}\displaystyle\int{\mathrm{d}x}$\\
		$\Leftrightarrow{\mathrm{e}^x}f(x)=x+C_1\Leftrightarrow f(x)=\dfrac{x+C_1}{\mathrm{e}^x}$.\\
		Theo giả thiết $ f(0)=2$, ta có $ f(0)=\dfrac{0+C_1}{\mathrm{e}^0}=2\Leftrightarrow{C_1}=2$.\\
		Vậy $ f(x)=\dfrac{x+2}{\mathrm{e}^x}$.\\
		Khi đó $\displaystyle\int{f(x){\mathrm{e}^{2x}}\mathrm{d}x}=\displaystyle\int{\dfrac{x+2}{\mathrm{e}^x}{\mathrm{e}^{2x}}\mathrm{d}x}=\displaystyle\int{\left(x+2\right){\mathrm{e}^x}\mathrm{d}x}=\left(x+2\right){\mathrm{e}^x}-\displaystyle\int{\mathrm{e}^x\mathrm{d}x}=\left(x+1\right){\mathrm{e}^x}+C$.
	}
\end{ex}
\begin{ex}
	[Sở Hậu Giang 2022]%Câu 71
	Cho hàm số $ f(x)$ thỏa mãn $ f(1)=\dfrac{1}{2}$ và $f'(x)-\dfrac{f(x)}{x^2+x}=\dfrac{x}{x+1},\forall x\in\left(0;+\infty\right)$. Giá trị $ f(7)$ bằng
	\choice
	{$\dfrac{7}{8}$}
	{\True $\dfrac{49}{8}$}
	{$\dfrac{1}{8}$}
	{$\dfrac{48}{49}$}
	\loigiai{
		Ta có $f'(x)-\dfrac{f(x)}{x^2+x}=\dfrac{x}{x+1},\forall x\in\left(0;+\infty\right)\Rightarrow \left(x^2+x\right){f}'(x)-f(x)=x^2$.\\
		$\Rightarrow f'(x)+\dfrac{x{f}'(x)-f(x)}{x^2}=1\Rightarrow \displaystyle\int{f(x)\mathrm{d}x}+\displaystyle\int{\left[\dfrac{x{f}'(x)-f(x)}{x^2}\right]}\mathrm{d}x=\displaystyle\int{\mathrm{d}x}$\\
		$\Rightarrow  f(x)+\dfrac{f(x)}{x}=x+C$.\\
		Với $x=1$: $\dfrac{1}{2}+\dfrac{1}{2}=1+C\Rightarrow  C=0$.\\
		$\Rightarrow  f(x)+\dfrac{f(x)}{x}=x\Rightarrow  f(x)\left[1+\dfrac{1}{x}\right]=x\Rightarrow  f(x)=\dfrac{x^2}{x+1}$\\
		Với $ x=7\Rightarrow f(7)=\dfrac{49}{8}$.
	}
\end{ex}
\begin{ex}
	[Cụm trường Bắc Ninh 2022]%Câu 72
	Cho hàm số $ y=f(x)$ có đạo hàm liên tục trên $\mathbb{R}$; thỏa mãn $ f(0)=-1$. Biết $ F(x)=\dfrac{1}{4}\left(2x-1\right).\mathrm{e}^{2x}$ là một nguyên hàm của hàm số $f'(x)-f(x)$. Họ tất cả các nguyên hàm của hàm số $ f(x).\mathrm{e}^{-2x}$ là
	\choice
	{$\displaystyle\int{f(x).\mathrm{e}^{-2x}\mathrm{d}x}=x.\mathrm{e}^x+\dfrac{1}{2}{\mathrm{e}^x}+C$}
	{$\displaystyle\int{f(x).\mathrm{e}^{-2x}\mathrm{d}x}=x^2-x+C$}
	{$\displaystyle\int{f(x).\mathrm{e}^{-2x}\mathrm{d}x}=x.\mathrm{e}^x-\dfrac{1}{2}{\mathrm{e}^x}+C$}
	{\True $\displaystyle\int{f(x).\mathrm{e}^{-2x}\mathrm{d}x}=\dfrac{x^2}{2}-x+C$}
	\loigiai{
		$ F(x)=\dfrac{1}{4}\left(2x-1\right).\mathrm{e}^{2x}$ là một nguyên hàm của hàm số $f'(x)-f(x)$\\
		$\Rightarrow{f}'(x)-f(x)=F'(x)=\dfrac{1}{4}\left[2\mathrm{e}^{2x}+2\left(2x-1\right){\mathrm{e}^{2x}}\right]=x.\mathrm{e}^{2x}$.\\
		$\begin{aligned}
			&\Rightarrow\dfrac{\mathrm{e}^x.f'(x)-\left(\mathrm{e}^x\right)'.f(x)}{\mathrm{e}^{2x}}=x.\mathrm{e}^x\\ 
			&\Rightarrow{\left(\dfrac{f(x)}{\mathrm{e}^x}\right)'}=x.\mathrm{e}^x\\ 
			&\Rightarrow\dfrac{f(x)}{\mathrm{e}^x}=\displaystyle\int{x.\mathrm{e}^x\mathrm{d}x}=\displaystyle\int{x\mathrm{d}\left(\mathrm{e}^x\right)}=x.\mathrm{e}^x-\displaystyle\int{\mathrm{e}^x\mathrm{d}x}=\left(x-1\right).\mathrm{e}^x+C.\\ 
		\end{aligned}$\\
		Với $ x=0$ thì $\dfrac{f(0)}{\mathrm{e}^0}=\left(0-1\right).\mathrm{e}^0+C\Rightarrow-1=-1+C\Rightarrow C=0$.\\
		Từ đó $\dfrac{f(x)}{\mathrm{e}^x}=\left(x-1\right).\mathrm{e}^x\Rightarrow f(x)=\left(x-1\right).\mathrm{e}^{2x}$.\\
		Vậy $\displaystyle\int{f(x).\mathrm{e}^{-2x}}\mathrm{d}x=\displaystyle\int{\left(x-1\right)}\mathrm{d}x=\dfrac{x^2}{2}-x+C$.
	}
\end{ex}
\begin{ex}
	[Sở Nam Định 2022]%Câu 73
	Cho hàm số $y=f(x)$ có đạo hàm $f'(x)$ thỏa mãn $\left(1+x^2\right){f}'(x)-1=3x^4+4x^2$, $\forall x\in\mathbb{R}$ và $ f(1)=0$. Biết $ F(x)$ là một nguyên hàm của hàm số $ 21.f\left(x^2\right)$ và $ F(0)=10$, hãy tính $ F(2)$. 
	\choice
	{$ F(2)=566$}
	{$ F(2)=\dfrac{566}{21}$}
	{\True $ F(2)=366$}
	{$ F(2)=52$}
	\loigiai{
		Ta có       $\left(1+x^2\right){f}'(x)-1=3x^4+4x^2\Leftrightarrow{f}'(x)=\dfrac{3x^4+4x^2+1}{1+x^2}$.\\
		Suy ra\\
		$f(x)=\displaystyle\int{f'(x)}\mathrm{d}x=\displaystyle\int{\dfrac{3x^4+4x^2+1}{1+x^2}}\mathrm{d}x=\displaystyle\int{\dfrac{3x^2\left(x^2+1\right)+\left(x^2+1\right)}{1+x^2}}\mathrm{d}x=\displaystyle\int{\left(3x^2+1\right)}\mathrm{d}x=x^3+x+C$.\\
		Do $ f(1)=0\Leftrightarrow 2+C=0\Leftrightarrow C=-2$.\\
		Vậy $ f(x)=x^3+x-2$. Suy ra $ f\left(x^2\right)=\left(x^2\right)^3+x^2-2=x^6+x^2-2$.\\
		Do $F(x)=\displaystyle\int{21.f\left(x^2\right)\mathrm{d}x}=21\displaystyle\int{f\left(x^2\right)\mathrm{d}x}=21\displaystyle\int{\left(x^6+x^2-2\right)\mathrm{d}x}=21\left(\dfrac{x^7}{7}+\dfrac{x^3}{3}-2x\right)+D$.\\
		Mặt khác $ F(0)=10\Leftrightarrow D=10$. Suy ra $F(x)=3x^7+7x^3-42x+10$.\\
		Vậy ta có $F(2)=3.2^7+7.2^3-42.2+10=366$.
	}
\end{ex}     
\Closesolutionfile{ans}
\indapan{10}{ans/CD25/Muc_9_10}