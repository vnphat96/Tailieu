\Opensolutionfile{ans}[ans/CD25/Muc_7_8]
\setcounter{ex}{0}
\setcounter{dang}{0}
\section{Mức độ 7,8 điểm}
\begin{dang}
	{Nguyên hàm cơ bản có điều kiện}
\end{dang}      
\begin{ex}
	[Đề Tham Khảo 2018]%Câu 1
	Cho hàm số $ f(x)$ xác định trên $\mathbb{R}\setminus\left\{\dfrac{1}{2}\right\}$ thỏa mãn $f'(x)=\dfrac{2}{2x-1},f(0)=1,f(1)=2$. Giá trị của biểu thức $ f\left(-1\right)+f(3)$ bằng
	\choice
	{$ 2+\ln 15$}
	{$ 3+\ln 15$}
	{\True $\ln 15$}
	{$ 4+\ln 15$}
	\loigiai{
		Ta có $\displaystyle\int{\dfrac{2}{2x-1}\mathrm{d}x=\ln\left| 2x-1\right|}+C=f(x)$\\
		Với $ x<\dfrac{1}{2}$, suy ra $ f(0)=1$$\Rightarrow C=1$ nên $ f\left(-1\right)=1+\ln 3$\\
		Với $ x>\dfrac{1}{2}$, suy ra $f(1)=2\Rightarrow C=2$ nên $ f(3)=2+\ln 5$\\
		Nên $ f\left(-1\right)+f(3)=3+\ln 15$.
	}
\end{ex}
\begin{ex}
	[Sở Phú Thọ 2019]%Câu 2
	Cho $ F(x)$ là một nguyên hàm của $ f(x)=\dfrac{1}{x-1}$ trên khoảng $\left(1;+\infty\right)$ thỏa mãn $ F\left(e+1\right)=4$Tìm $ F(x)$.
	\choice
	{$ 2\ln\left(x-1\right)+2$}
	{\True $\ln\left(x-1\right)+3$}
	{$ 4\ln\left(x-1\right)$}
	{$\ln\left(x-1\right)-3$}
	\loigiai{
		$F(x)=\displaystyle\int{\dfrac{1}{x-1}}\mathrm{d}x+C=\ln\left|x-1\right|+C$\\
		Ta có $F\left(\mathrm{e}+1\right)=4$. Suy ra $ 1+C=4\Rightarrow C=3$.
	}
\end{ex}
\begin{ex}
	[THPT Minh Khai Hà Tĩnh 2019]%Câu 3
	Cho $F(x)$ là một nguyên hàm của hàm số $f(x)=\dfrac{1}{x-2},$ biết $F(1)=2.$ Giá trị của $F(0)$ bằng
	\choice
	{\True $ 2+\ln 2$}
	{$\ln 2$}
	{$ 2+\ln\left(-2\right)$}
	{$\ln\left(-2\right)$}
	\loigiai{
		Ta có $\displaystyle\int{f(x)}\mathrm{d}x=\displaystyle\int{\dfrac{1}{x-2}}\mathrm{d}x=\ln\left| x-2\right|+C,C\in\mathbb{R}$.\\
		Giả sử $ F(x)=\ln\left| x-2\right|+C_0$ là một nguyên hàm của hàm số đã cho thỏa mãn $ F(1)=2$.\\
		Do $ F(1)=2\Rightarrow{C_0}=2\Rightarrow F(x)=\ln\left| x-2\right|+2$. Vậy $ F(0)=2+\ln 2$.
	}
\end{ex}
\begin{ex}
	[KTNL GV Thuận Thành 2 Bắc Ninh 2019]%Câu 4
	Cho $F(x)$ là một nguyên hàm của hàm $f(x)=\dfrac{1}{2x+1}$; biết $F(0)=2$. Tính $F(1)$.
	\choice
	{$F(1)=\dfrac{1}{2}ln3-2$}
	{$F(1)=ln3+2$}
	{$F(1)=2ln3-2$}
	{\True $F(1)=\dfrac{1}{2}ln3+2$}
	\loigiai{
		Ta có $F(x)=\displaystyle\int{\dfrac{1}{2x+1}}\mathrm{d}x=\dfrac{1}{2}\ln\left| 2x+1\right|+C$\\
		Do $F(0)=2\Rightarrow\dfrac{1}{2}\ln\left| 2.0+1\right|+C=2\Rightarrow C=2$\\
		Vậy $F(x)=\dfrac{1}{2}\ln\left| 2x+1\right|+2\Rightarrow F(1)=\dfrac{1}{2}\ln 3+2$.
	}
\end{ex}
\begin{ex}
	[Chuyên ĐHSP Hà Nội 2019]%Câu 5
	Hàm số $F(x)$ là một nguyên hàm của hàm số $y=\dfrac{1}{x}$ trên $(-\infty ; 0)$ thỏa mãn $F(-2)=0$. Khẳng định nào sau đây đúng?
	\choice
	{\True $F(x)=\ln \left(\dfrac{-x}{2}\right), \forall x \in(-\infty ; 0)$}
	{$F(x)=\ln |x|+C, \forall x \in(-\infty ; 0)$ với $C$ là một số thực bất kì}
	{$F(x)=\ln |x|+\ln 2, \forall x \in(-\infty ; 0)$}
	{$F(x)=\ln (-x)+C, \forall x \in(-\infty ; 0)$ với $C$ là một số thực bất kì}
	\loigiai{
		Ta có $F(x)=\displaystyle\int\limits \dfrac{1}{x}\mathrm{~d}x=\ln |x|+C=\ln (-x)+C$ với $\forall x \in(-\infty ; 0)$.\\
		Lại có $F(-2)=0 \Leftrightarrow \ln 2+C=0 \Leftrightarrow C=-\ln 2$.\\ 
		Do đó $F(x)=\ln (-x)-\ln 2=\ln \left(\dfrac{-x}{2}\right)$.\\
		Vậy $F(x)=\ln \left(\dfrac{-x}{2}\right) \forall x \in(-\infty ; 0)$.
	}
\end{ex}
\begin{ex}
	[THPT Minh Khai Hà Tĩnh 2019]%Câu 6
	Cho hàm số $f(x)$ xác định trên $R\setminus\left\{ 1\right\}$ thỏa mãn $f'(x)=\dfrac{1}{x-1}$, $f(0)=2017$, $f(2)=2018$. Tính $S=f(3)-f\left(-1\right)$.
	\choice
	{$ S=\ln 4035$}
	{$ S=4$}
	{$ S=\ln 2$}
	{\True $ S=1$}
	\loigiai{
		Trên khoảng $\left(1;+\infty\right)$ ta có $\displaystyle\int{f'(x)\mathrm{d}x=}\displaystyle\int{\dfrac{1}{x-1}}\mathrm{d}x$$=\ln\left(x-1\right)+C_1$$\Rightarrow f(x)=\ln\left(x-1\right)+C_1$.\\
		Mà $ f(2)=2018\Rightarrow{C_1}=2018$.\\
		Trên khoảng$\left(-\infty ;1\right)$ ta có $\displaystyle\int{f'(x)\mathrm{d}x=}\displaystyle\int{\dfrac{1}{x-1}}\mathrm{d}x$ $=\ln\left(1-x\right)+C_2$ $\Rightarrow f(x)=\ln\left(1-x\right)+C_2$ .\\
		Mà $ f(0)=2017$$\Rightarrow{C_2}=2017$.\\
		Vậy $f(x)=\left\{\begin{aligned}
			&\ln (x-1)+2018~\text{khi}~x>1\\ 
			&\ln (1-x)+2017~\text{khi}~x<1\\ 
		\end{aligned}\right.$.\\ 
		Suy ra $ f(3)-f\left(-1\right)=1$.
	}
\end{ex}
\begin{ex}
	[Mã 105 2017]%Câu 7
	Cho $ F(x)$ là một nguyên hàm của hàm số $ f(x)=\mathrm{e}^x+2x$ thỏa mãn $ F(0)=\dfrac{3}{2}$. Tìm $ F(x)$.
	\choice
	{\True $ F(x)=\mathrm{e}^x+x^2+\dfrac{1}{2}$}
	{$ F(x)=\mathrm{e}^x+x^2+\dfrac{5}{2}$}
	{$ F(x)=\mathrm{e}^x+x^2+\dfrac{3}{2}$}
	{$ F(x)=2\mathrm{e}^x+x^2-\dfrac{1}{2}$}
	\loigiai{
		Ta có $ F(x)=\displaystyle\int{\left(\mathrm{e}^x+2x\right)\mathrm{d}x}=\mathrm{e}^x+x^2+C$\\
		Theo bài ra ta có $ F(0)=1+C=\dfrac{3}{2}\Rightarrow C=\dfrac{1}{2}$.
	}
\end{ex}
\begin{ex}
	[THCS - THPT Nguyễn Khuyến 2019]%Câu 8
	Biết $ F(x)$ là một nguyên hàm của hàm số $ f(x)=\mathrm{e}^{2x}$ và $ F(0)=0$. Giá trị của $ F\left(\ln 3\right)$ bằng
	\choice
	{$2$}
	{$6$}
	{$8$}
	{\True $4$}
	\loigiai{
		$F(x)=\displaystyle\int{\mathrm{e}^{2x}\mathrm{d}x}=\dfrac{1}{2}{\mathrm{e}^{2x}}+C;F(0)=0\Rightarrow C=-\dfrac{1}{2}\Rightarrow F(x)=\dfrac{1}{2}{\mathrm{e}^{2x}}-\dfrac{1}{2}$.\\
		Khi đó $ F\left(\ln 3\right)=\dfrac{1}{2}{\mathrm{e}^{2\ln 3}}-\dfrac{1}{2}=4$.
	}
\end{ex}
\begin{ex}
	[Sở Bình Phước 2019]%Câu 9
	Biết $ F(x)$ là một nguyên hàm của hàm số $e^{2x}$ và $ F(0)=\dfrac{201}{2}$. Giá trị $ F\left(\dfrac{1}{2}\right)$ là
	\choice
	{$\dfrac{1}{2}\mathrm{e}+200$}
	{$ 2\mathrm{e}+100$}
	{$\dfrac{1}{2}\mathrm{e}+50$}
	{\True $\dfrac{1}{2}\mathrm{e}+100$}
	\loigiai{
		Ta có $\displaystyle\int{\mathrm{e}^{2x}\mathrm{d}x}=\dfrac{1}{2}\cdot{\mathrm{e}^{2x}}+C$.\\
		Theo đề ra ta được $F(0)=\dfrac{201}{2}\Leftrightarrow\dfrac{1}{2}\cdot{\mathrm{e}^0}+C=\dfrac{201}{2}\Leftrightarrow C=100$.\\
		Vậy $ F(x)=\dfrac{1}{2}{\mathrm{e}^{2x}}+100\Rightarrow F\left(\dfrac{1}{2}\right)=\dfrac{1}{2}{\mathrm{e}^{2\cdot\dfrac{1}{2}}}+100=\dfrac{1}{2}e+100$.
	}
\end{ex}
\begin{ex}
	[Chuyên Nguyễn Trãi Hải Dương 2019]%Câu 10
	Hàm số $f(x)$ có đạo hàm liên tục trên $\mathbb{R}$ và $f'(x)=2\mathrm{e}^{2x}+1,$ $\forall x,f(0)=2$. Hàm $f(x)$ là
	\choice
	{$y=2\mathrm{e}^x+2x$}
	{$y=2\mathrm{e}^x+2$}
	{$y=\mathrm{e}^{2x}+x+2$}
	{\True $y=\mathrm{e}^{2x}+x+1$}
	\loigiai{
		Ta có $\displaystyle\int{f'(x)\mathrm{d}}x$ $=\displaystyle\int{\left(\text{2}{\mathrm{e}^{2x}}+1\right)\mathrm{d}}x$ $=\mathrm{e}^{2x}+x+C$.\\
		Suy ra $f(x)=\mathrm{e}^{2x}+x+C$.\\
		Theo bài ra ta có $f(0)=2$ $\Rightarrow 1+C=2$ $\Leftrightarrow C=1$.\\
		Vậy $f(x)=\mathrm{e}^{2x}+x+1$.
	}
\end{ex}
\begin{ex}
	[Sở Bắc Ninh 2019]%Câu 11
	Cho hàm số $f(x)=2x+\mathrm{e}^x$. Tìm một nguyên hàm $F(x)$ của hàm số $f(x)$ thỏa mãn $F(0)=2019$.
	\choice
	{\True $F(x)=x^2+\mathrm{e}^x+2018$}
	{$F(x)=x^2+\mathrm{e}^x-2018$}
	{$F(x)=x^2+\mathrm{e}^x+2017$}
	{$F(x)=\mathrm{e}^x-2019$}
	\loigiai{
		Ta có $\displaystyle\int{f(x)\mathrm{d}x=\displaystyle\int{\left(2x+\mathrm{e}^x\right)}\mathrm{d}x}=x^2+\mathrm{e}^x+C$.\\
		Có $F(x)$ là một nguyên hàm của $f(x)$ và $F(0)=2019$.\\
		Suy ra $\left\{\begin{aligned}
			& F(x)=x^2+\mathrm{e}^x+C\\ 
			& F(0)=2019\\ 
		\end{aligned}\right.$ $\Rightarrow 1+C=2019\Leftrightarrow C=2018$.\\
		Vậy $F(x)=x^2+\mathrm{e}^x+2018$.
	}
\end{ex}
\begin{ex}
	Gọi $F(x)$ là một nguyên hàm của hàm số $f(x)=2^x$, thỏa mãn $F(0)=\dfrac{1}{\ln 2}$. Tính giá trị biểu thức $T=F(0)+F(1)+\ldots+F\left(2018\right)+F\left(2019\right)$.
	\choice
	{$T=1009.\dfrac{2^{2019}+1}{\ln 2}$}
	{$T=2^{2019.2020}$}
	{$T=\dfrac{2^{2019}-1}{\ln 2}$}
	{\True $T=\dfrac{2^{2020}-1}{\ln 2}$}
	\loigiai{
		Ta có $\displaystyle\int{f(x)}\mathrm{d}x=\displaystyle\int{2^x}\mathrm{d}x=\dfrac{2^x}{\ln 2}+C$\\
		$F(x)$ là một nguyên hàm của hàm số $f(x)=2^x$, ta có $F(x)=\dfrac{2^x}{\ln 2}+C$ mà $F(0)=\dfrac{1}{\ln 2}$\\
		Suy ra $C=0\Rightarrow F(x)=\dfrac{2^x}{\ln 2}$.\\
		$\begin{aligned}
			T&=F(0)+F(1)+\ldots+F\left(2018\right)+F\left(2019\right)\\
			&=\dfrac{1}{\ln 2}\left(1+2+2^2+\ldots+2^{2018}+2^{2019}\right)\\
			&=\dfrac{1}{\ln 2}.\dfrac{2^{2020}-1}{2-1}=\dfrac{2^{2020}-1}{\ln 2}.\\
		\end{aligned}$
	}
\end{ex}
\begin{ex}
	[Mã 104 2017]%Câu 13
	Tìm nguyên hàm $ F(x)$ của hàm số $ f(x)=\sin x+\cos x$ thoả mãn $ F\left(\dfrac{\pi}{2}\right)=2$.
	\choice
	{$ F(x)=-\cos x+\sin x+3$}
	{$ F(x)=-\cos x+\sin x-1$}
	{\True $ F(x)=-\cos x+\sin x+1$}
	{$ F(x)=\cos x-\sin x+3$}
	\loigiai{
		Có $ F(x)=\displaystyle\int{f(x)\mathrm{d}x}=\displaystyle\int{\left(\sin x+\cos x\right)\mathrm{d}x}=-\cos x+\sin x+C$\\
		Do $ F\left(\dfrac{\pi}{2}\right)=-\cos\dfrac{\pi}{2}+\sin\dfrac{\pi}{2}+C=2\Leftrightarrow 1+C=2\Leftrightarrow C=1$.\\
		Suy ra $F(x)=-\cos x+\sin x+1$.
	}
\end{ex}
\begin{ex}
	[Mã 123 2017]%Câu 14
	Cho hàm số $ f(x)$ thỏa mãn $ f'(x)=3-5\sin x$ và $ f(0)=10$. Mệnh đề nào dưới đây đúng?
	\choice
	{$ f(x)=3x-5\cos x+15$}
	{$ f(x)=3x-5\cos x+2$}
	{\True $ f(x)=3x+5\cos x+5$}
	{$ f(x)=3x+5\cos x+2$}
	\loigiai{
		Ta có $ f(x)=\displaystyle\int{\left(3-5\sin x\right)\mathrm{d}x}=3x+5\cos x+C$\\
		Theo giả thiết $ f(0)=10$ nên $ 5+C=10\Rightarrow C=5$.\\
		Vậy $ f(x)=3x+5\cos x+5$.
	}
\end{ex}
\begin{ex}
	[Việt Đức Hà Nội 2019]%Câu 15
	Cho hàm số $f(x)$ thỏa mãn $f'(x)=2-5\sin x$ và $f(0)=10$. Mệnh đề nào dưới đây đúng?
	\choice
	{$ f(x)=2x+5\cos x+3$}
	{$ f(x)=2x-5\cos x+15$}
	{\True $ f(x)=2x+5\cos x+5$}
	{$ f(x)=2x-5\cos x+10$}
	\loigiai{
		Ta có $ f(x)=\displaystyle\int{f'(x)\mathrm{d}x=}\displaystyle\int{\left(2-5\sin x\right)\mathrm{d}x=2x+5\cos x+C}$.\\
		Mà $ f(0)=10$ nên $ 5+C=10\Rightarrow C=5$.\\
		Vậy $ f(x)=2x+5\cos x+5$.
	}
\end{ex}
\begin{ex}
	[Liên Trường THPT Tp Vinh Nghệ An 2019]%Câu 16
	Biết $ F(x)$ là một nguyên hàm của hàm $ f(x)=\cos 3x$ và $ F\left(\dfrac{\pi}{2}\right)=\dfrac{2}{3}$. Tính $ F\left(\dfrac{\pi}{9}\right)$.
	\choice
	{$ F\left(\dfrac{\pi}{9}\right)=\dfrac{\sqrt{3}+2}{6}$}
	{$ F\left(\dfrac{\pi}{9}\right)=\dfrac{\sqrt{3}-2}{6}$}
	{\True $ F\left(\dfrac{\pi}{9}\right)=\dfrac{\sqrt{3}+6}{6}$}
	{$ F\left(\dfrac{\pi}{9}\right)=\dfrac{\sqrt{3}-6}{6}$}
	\loigiai{
		$ F(x)=\displaystyle\int{\cos 3x\mathrm{d}x}=\dfrac{\sin 3x}{3}+C$\\
		$ F\left(\dfrac{\pi}{2}\right)=\dfrac{2}{3}$$\Rightarrow C=1$ $\Rightarrow F(x)=\dfrac{\sin 3x}{3}+1$ $\Rightarrow F\left(\dfrac{\pi}{9}\right)=\dfrac{\sin\dfrac{\pi}{3}}{3}+1=\dfrac{\sqrt{3}+6}{6}$.
	}
\end{ex}
\begin{ex}
	[Chuyên Lê Quý Đôn Quảng Trị 2019]%Câu 17
	Cho $ F(x)$ là một nguyên hàm của hàm số $ f(x)=\dfrac{1}{\cos^2x}$. Biết $ F\left(\dfrac{\pi}{4}+k\pi\right)=k$ với mọi $ k\in\mathbb{Z}$. Tính $ F(0)+F\left(\pi\right)+F\left(2\pi\right)+\ldots+F\left(10\pi\right)$.
	\choice
	{$55$}
	{\True $44$}
	{$45$}
	{$0$}
	\loigiai{
		Ta có $\displaystyle\int{f(x)\mathrm{d}x=}\displaystyle\int{\dfrac{\mathrm{d}x}{\cos^2x}=\tan x+C}$.\\
		Suy ra $ F(x)=\left\{\begin{aligned}
			&\tan x+C_0,x\in\left(-\dfrac{\pi}{2};\dfrac{\pi}{2}\right)\\ 
			&\tan x+C_1,x\in\left(\dfrac{\pi}{2};\dfrac{3\pi}{2}\right)\\ 
			&\tan x+C_2,x\in\left(\dfrac{3\pi}{2};\dfrac{5\pi}{2}\right)\\ 
			&\ldots\\ 
			&\tan x+C_9,x\in\left(\dfrac{17\pi}{2};\dfrac{19\pi}{2}\right)\\ 
			&\tan x+C_{10},x\in\left(\dfrac{19\pi}{2};\dfrac{21\pi}{2}\right)\\ 
		\end{aligned}\right.\Rightarrow\left\{\begin{aligned}
			&F\left(\dfrac{\pi}{4}+0\pi\right)=1+C_0=0\Rightarrow{C_0}=-1\\ 
			&F\left(\dfrac{\pi}{4}+\pi\right)=1+C_1=1\Rightarrow{C_1}=0\\ 
			&F\left(\dfrac{\pi}{4}+2\pi\right)=1+C_2=2\Rightarrow{C_0}=1\\ 
			&\ldots\\ 
			&F\left(\dfrac{\pi}{4}+9\pi\right)=1+C_9=9\Rightarrow{C_9}=8\\ 
			&F\left(\dfrac{\pi}{4}+10\pi\right)=1+C_{10}=10\Rightarrow{C_{10}}=9.\\ 
		\end{aligned}\right.$\\
		Vậy $ F(0)+F\left(\pi\right)+F\left(2\pi\right)+\ldots+F\left(10\pi\right)=\tan 0-1+\tan\pi+\tan 2\pi+1+\ldots+\tan 10\pi+9=44$.
	}
\end{ex}
\begin{ex}
	[Yên Lạc 2 - Vĩnh Phúc - 2020]%Câu 18
	Gọi $ F(x)$ là một nguyên hàm của hàm số $ f(x)=2^x$, thỏa mãn $ F(0)=\dfrac{1}{\ln 2}$. Tính giá trị biểu thức $ T=F(0)+F(1)+F(2)+\ldots+F\left(2019\right)$.
	\choice
	{\True $ T=\dfrac{2^{2020}-1}{\ln 2}$}
	{$ T=1009.\dfrac{2^{2019}-1}{2}$}
	{$ T=2^{2019.2020}$}
	{$ T=\dfrac{2^{2019}-1}{\ln 2}$}
	\loigiai{
		Ta có $ F(x)=\displaystyle\int{2^x\mathrm{d}x}=\dfrac{2^x}{\ln 2}+C$.\\
		Theo giả thiết $ F(0)=\dfrac{1}{\ln 2}\Leftrightarrow\dfrac{2^0}{\ln 2}+C=\dfrac{1}{\ln 2}\Leftrightarrow C=0$. \\
		Suy ra: $ F(x)=\dfrac{2^x}{\ln 2}$\\
		Vậy $\begin{aligned}[t]
			T=F(0)+F(1)+F(2)+\ldots+F\left(2019\right)&=\dfrac{2^0}{\ln 2}+\dfrac{2^1}{\ln 2}+\dfrac{2^2}{\ln 2}+\ldots+\dfrac{2^{2019}}{\ln 2}\\
			&=\dfrac{1}{\ln 2}\left(2^0+2^1+2^2+\ldots+2^{2019}\right)\\
			&=\dfrac{1}{\ln 2}.1.\dfrac{1-2^{2020}}{1-2}=\dfrac{2^{2020}-1}{\ln 2}.
		\end{aligned}$
	}
\end{ex}
\begin{ex}
	[Đề minh họa 2022]%Câu 19
	Cho hàm số $ y=f(x)$ có đạo hàm là $f'(x)=12x^2+2,\forall x\in\mathbb{R}$ và $ f(1)=3$. Biết $ F(x)$ là nguyên hàm của $ f(x)$ thỏa mãn $ F(0)=2$, khi đó $ F(1)$ bằng
	\choice
	{$-3$}
	{\True $ 1$}
	{$ 2$}
	{$ 7$}
	\loigiai{
		Ta có $f(x)=\displaystyle\int{f'(x)\mathrm{d}x}=\displaystyle\int{\left(12x^2+2\right)\mathrm{d}x}=4x^3+2x+C_1$ .\\
		Mà $ f(1)=3$ nên $4.1^3+2.1+C_1=3\Leftrightarrow{C_1}=-3$.\\
		$\Rightarrow f(x)=4x^3+2x-3$.\\
		Lại có $ F(x)=\displaystyle\int{f(x)\mathrm{d}x}=\displaystyle\int{\left(4x^3+2x-3\right)\mathrm{d}x}=x^4+x^2-3x+C_2$.\\
		Hơn nữa, $ F(0)=2\Leftrightarrow{0^4}+0^2-3.0+C_2=2\Leftrightarrow{C_2}=2$.\\
		$\Rightarrow F(x)=x^4+x^2-3x+2$.\\
		Suy ra $ F(1)=1^4+1^2-3.1+2=1$.
	}
\end{ex}
\begin{ex}
	[Sở Hà Tĩnh 2022]%Câu 20
	Cho $ F(x)$ là nguyên hàm của $ f(x)=\sin^2x$ trên $\mathbb{R}$ thoả mãn $ F\left(\dfrac{\pi}{4}\right)=0$. Giá trị biểu thức $ S=F\left(-\pi\right)+2F\left(\dfrac{\pi}{2}\right)$ bằng
	\choice
	{$ S=\dfrac{3}{4}-\dfrac{\pi}{4}$}
	{\True $ S=\dfrac{3}{4}-\dfrac{3\pi}{4}$}
	{$ S=\dfrac{1}{4}+\dfrac{3\pi}{8}$}
	{$ S=\dfrac{3}{2}-\dfrac{3\pi}{8}$}
	\loigiai{
		Ta có $\displaystyle\int{\sin^2x}\mathrm{d}x=\displaystyle\int{\dfrac{1-\cos 2x}{2}}\mathrm{d}x=\dfrac{1}{2}x-\dfrac{1}{4}\sin 2x+C$.\\
		$\Rightarrow F(x)=\dfrac{1}{2}x-\dfrac{1}{4}\sin 2x+C$\\
		Mà $F\left(\dfrac{\pi}{4}\right)=0\Leftrightarrow\dfrac{\pi}{8}-\dfrac{1}{4}\sin\dfrac{\pi}{2}+C=0$ $\Leftrightarrow C=-\dfrac{\pi}{8}+\dfrac{1}{4}$.\\
		Vậy $F(x)=\dfrac{1}{2}x-\dfrac{1}{4}\sin 2x-\dfrac{\pi}{8}+\dfrac{1}{4}$.\\
		$S=F\left(-\pi\right)+2F\left(\dfrac{\pi}{2}\right)$ $=-\dfrac{\pi}{2}-\dfrac{1}{4}\sin\left(-2\pi\right)-\dfrac{\pi}{8}+\dfrac{1}{4}+2\left[\dfrac{1}{2}.\dfrac{\pi}{2}-\dfrac{1}{4}\sin 2.\dfrac{\pi}{2}-\dfrac{\pi}{8}+\dfrac{1}{4}\right]$ $=-\dfrac{\pi}{2}-\dfrac{\pi}{8}+\dfrac{\pi}{4}+\dfrac{1}{4}+\dfrac{1}{2}=\dfrac{3}{4}-\dfrac{3\pi}{8}$.
	}
\end{ex}
\begin{ex}
	[Sở Nam Định 2022]%Câu 21
	Cho hàm số $ y=f(x)$ có đạo hàm là $f'(x)=8x^3+\sin x,\forall x\in\mathbb{R}$ và $ f(0)=3$. Biết $ F(x)$ là nguyên hàm của $f(x)$ thỏa mãn $ F(0)=2$, khi đó $ F(1)$ bằng
	\choice
	{$\dfrac{32}{5}+\cos 1$}
	{$\dfrac{32}{5}-\cos 1$}
	{\True $\dfrac{32}{5}-\sin 1$}
	{$\dfrac{32}{5}+\sin 1$}
	\loigiai{
		Ta có $f'(x)=8x^3+\sin x,\forall x\in\mathbb{R}\Rightarrow f(x)=2x^4-\cos x+C_1$.\\
		Mà $ f(0)=3\Leftrightarrow-1+C_1=3\Rightarrow{C_1}=4$. Vậy $ f(x)=2x^4-\cos x+4$.\\
		Ta có $\displaystyle\int{f(x)\mathrm{d}x}=\displaystyle\int{\left(2x^4-\cos x+4\right)\mathrm{d}}x=\dfrac{2}{5}{x^5}-\sin x+4x+C$.\\
		Do đó $ F(x)=\dfrac{2}{5}{x^5}-\sin x+4x+C_2$.\\
		Mà $ F(0)=2\Leftrightarrow{C_2}=2$. Suy ra $ F(x)=\dfrac{2}{5}{x^5}-\sin x+4x+2$.\\
		Khi đó $ F(1)=\dfrac{32}{5}-\sin 1$.
	}
\end{ex}
\begin{ex}
	[Chuyên Hùng Vương – Gia Lai 2022]%Câu 22
	Cho hàm số $f(x)$ xác định trên $\mathbb{R}\backslash\{2\}$ thỏa mãn $f'(x)=\dfrac{1}{x-2}, f(1)=2021$, $f(3)=2022$. Tính $P=\dfrac{f(2023)}{f(-2019)}$.
	\choice
	{$P=\ln 4042$}
	{$P=\dfrac{\ln 2021}{\ln 2022}$}
	{$P=\ln\dfrac{2021}{2022}$}
	{\True $ P=\dfrac{2022+\ln 2021}{2021+\ln 2021}$}
	\loigiai{
		Trên khoảng $(2 ;+\infty)$: $\displaystyle\int f'(x) \mathrm{d}x=\displaystyle\int\dfrac{1}{x-2}\mathrm{d}x=\ln (x-2)+C_1\Rightarrow f(x)=\ln (x-2)+C_1$.\\
		Mà $f(3)=2022\Rightarrow C_1=2022$.\\
		Trên khoảng $(-\infty ; 2)$: $\displaystyle\int f'(x) \mathrm{d}x=\displaystyle\int\dfrac{1}{x-2}\mathrm{d}x=\ln (2-x)+C_2\Rightarrow f(x)=\ln (2-x)+C_2$.\\
		Mà $f(1)=2021\Rightarrow C_2=2021$.\\
		Vậy $f(x)=\left\{\begin{aligned}
			&\ln (x-2)+2022 &\text{khi}& x>2\\
			&\ln (2-x)+2021 &\text{khi}& x<2\\
		\end{aligned}\right.$.\\
		Suy ra $P=\dfrac{f(2023)}{f(-2019)}=\dfrac{2022+\ln 2021}{2021+\ln 2021}$.
	}
\end{ex}
\begin{ex}
	[THPT Hoàng Hoa Thám - Quảng Ninh - 2022]%Câu 23
	Cho hàm số $ y=f(x)$ có đạo hàm $f'(x)=\dfrac{1}{x-1}+6x,\forall x\in\left(1;+\infty\right)$ và $ f(2)=12$. Biết $ F(x)$ là nguyên hàm của $ f(x)$ thỏa $ F(2)=6$, khi đó giá trị biểu thức $ P=F(5)-4F(3)$ bằng
	\choice
	{$ 25$}
	{$ 10$}
	{$ 20$}
	{\True $ 24$}
	\loigiai{
		Với $\forall x\in\left(1;+\infty\right)$ ta có $ f(x)=\displaystyle\int{\left(\dfrac{1}{x-1}+6x\right)\mathrm{d}x}=\ln\left(x-1\right)+3x^2+C$.\\
		Vì $ f(2)=12\Rightarrow C=0\Rightarrow f(x)=\ln\left(x-1\right)+3x^2$.\\
		$\begin{aligned}
			&F(x)=\displaystyle\int{\left(\ln\left(x-1\right)+3x^2\right)\mathrm{d}x}=x\ln\left(x-1\right)-\displaystyle\int{x\mathrm{d}\left(\text{ln}\left(x-1\right)\right)}+x^3=x\ln\left(x-1\right)-\displaystyle\int{x.\dfrac{1}{x-1}\mathrm{d}x}+x^3\\ 
			&=x\ln\left(x-1\right)-x-\ln\left(x-1\right)+x^3+C'\\ 
		\end{aligned}$\\
		$ F(2)=6$ nên $C'=0$. Suy ra $ F(x)=x\ln\left(x-1\right)-x-\ln\left(x-1\right)+x^3$.\\
		$ P=F(5)-4F(3)=5\ln 4-5-\ln 4+125-4\left(3\ln 2-3-\ln 2+27\right)=120-96=24$.
	}
\end{ex}
\begin{ex}
	[THPT Trần Quốc Tuấn - Quảng Ngãi - 2022]%Câu 24
	Cho hàm số $ f(x)$ có $ f\left(\dfrac{\pi}{4}\right)=-\dfrac{4}{3}$ và $ f'(x)=16\cos 4x.\sin^2x,\forall x\in\mathbb{R}$. Biết $ F(x)$ là nguyên hàm của $ f(x)$ thỏa mãn $ F(0)=\dfrac{15}{26}$. Tính $ F\left(\pi\right)$.
	\choice
	{$\dfrac{64}{27}$}
	{\True $\dfrac{15}{26}$}
	{$\dfrac{31}{18}$}
	{$ 0$}
	\loigiai{
		Ta có\\
		$\begin{aligned}
			& f(x)=\displaystyle\int{f'(x)\mathrm{d}x=\displaystyle\int{16\cos 4x.\sin^2x\mathrm{d}x}}\\ 
			&=\displaystyle\int{16\cos 4x.\dfrac{1-\cos 2x}{2}\mathrm{d}x}\\ 
			&=8\displaystyle\int{\left(\cos 4x-\cos 4x.\cos 2x\right)\mathrm{d}x}\\ 
			&=8\left[\displaystyle\int{\cos 4x\mathrm{d}x-\displaystyle\int{\cos 4x.\cos 2x\mathrm{d}x}}\right]\\ 
			&=8\left[\dfrac{\sin 4x}{4}-\dfrac{1}{2}\displaystyle\int{\left(\cos 6x+\cos 2x\right)\mathrm{d}x}\right]+c\\ 
			&=8\left[\dfrac{\sin 4x}{4}-\dfrac{1}{2}\left(\dfrac{\sin 6x}{6}+\dfrac{\sin 2x}{2}\right)\right]+c\\ 
			&=2\sin 4x-4\left(\dfrac{\sin 6x}{6}+\dfrac{\sin 2x}{2}\right)+c\\ 
			&=2\sin 4x-\dfrac{2}{3}\sin 6x-2\sin 2x+c\\ 
		\end{aligned}$\\
		$ f\left(\dfrac{\pi}{4}\right)=-\dfrac{4}{3}\Leftrightarrow 2\sin\pi-\dfrac{2}{3}\sin\dfrac{3\pi}{2}-2\sin\dfrac{\pi}{2}+c=-\dfrac{4}{3}\Leftrightarrow-\dfrac{4}{3}+c=-\dfrac{4}{3}\Leftrightarrow c=0$\\
		$ f(x)=2\sin 4x-\dfrac{2}{3}\sin 6x-2\sin 2x$\\
		$\begin{aligned}
			&F(x)=\displaystyle\int{f(x)\mathrm{d}x=\displaystyle\int{\left(2\sin 4x-\dfrac{2}{3}\sin 6x-2\sin 2x\right)\mathrm{d}x}}\\ 
			&=-\dfrac{\cos 4x}{2}+\dfrac{\cos 6x}{9}+\cos 2x+c'\\ 
		\end{aligned}$\\
		$F(0)=\dfrac{15}{26}\Leftrightarrow-\dfrac{1}{2}+\dfrac{1}{9}+1+c'=\dfrac{15}{26}\Leftrightarrow\dfrac{11}{18}+c'=\dfrac{15}{26}\Leftrightarrow c'=\dfrac{15}{26}-\dfrac{11}{18}\Leftrightarrow c'=-\dfrac{4}{117}$\\
		$ F(x)=-\dfrac{\cos 4x}{2}+\dfrac{\cos 6x}{9}+\cos 2x-\dfrac{4}{117}$\\
		$\Rightarrow F\left(\pi\right)=-\dfrac{\cos 4\pi}{2}+\dfrac{\cos 6\pi}{9}+\cos 2\pi-\dfrac{4}{117}=-\dfrac{1}{2}+\dfrac{1}{9}+1-\dfrac{4}{117}=\dfrac{15}{26}$.
	}
\end{ex}
\begin{dang}
	{Tìm nguyên hàm bằng phương pháp đổi biến số}
\end{dang}        
\begin{ex}
	[Mã 101 – 2020 Lần 2]%Câu 25
	Biết $ F(x)=\mathrm{e}^x+x^2$ là một nguyên hàm của hàm số $ f(x)$ trên $\mathbb{R}$. Khi đó $\displaystyle\int{f\left(2x\right)}\mathrm{d}x$ bằng
	\choice
	{$ 2\mathrm{e}^x+2x^2+C$}
	{$\dfrac{1}{2}{\mathrm{e}^{2x}}+x^2+C$}
	{\True $\dfrac{1}{2}{\mathrm{e}^{2x}}+2x^2+C$}
	{$\mathrm{e}^{2x}+4x^2+C$}
	\loigiai{
		Ta có $ F(x)=\mathrm{e}^x+x^2$ là một nguyên hàm của hàm số $ f(x)$ trên $\mathbb{R}$\\
		$\Rightarrow\displaystyle\int{f\left(2x\right)\mathrm{d}x=\dfrac{1}{2}\displaystyle\int{f\left(2x\right)}}\mathrm{d}2x=\dfrac{1}{2}F\left(2x\right)+C=\dfrac{1}{2}{\mathrm{e}^{2x}}+2x^2+C$
	}
\end{ex}
\begin{ex}
	[Mã 102 - 2020 Lần 2]%Câu 26
	Biết $ F(x)=\mathrm{e}^x-2x^2$ là một nguyên hàm của hàm số $ f(x)$ trên $\mathbb{R}$. Khi đó $\displaystyle\int{f\left(2x\right)}\mathrm{d}x$ bằng
	\choice
	{$ 2\mathrm{e}^x-4x^2+C$}
	{\True $\dfrac{1}{2}{\mathrm{e}^{2x}}-4x^2+C$}
	{$\mathrm{e}^{2x}-8x^2+C$}
	{$\dfrac{1}{2}{\mathrm{e}^{2x}}-2x^2+C$}
	\loigiai{
		Ta có $ F(x)=\mathrm{e}^x-2x^2$ là một nguyên hàm của hàm số $ f(x)$ trên $\mathbb{R}$\\
		Suy ra\\
		$ f(x)=F'(x)=\left(\mathrm{e}^x-2x^2\right)^{\prime}=\mathrm{e}^x-4x\Rightarrow f\left(2x\right)=\mathrm{e}^{2x}-8x$\\
		$\Rightarrow\displaystyle\int{f\left(2x\right)\mathrm{d}x=\displaystyle\int{\left(\mathrm{e}^{2x}-8x\right)}}\mathrm{d}x=\dfrac{1}{2}{\mathrm{e}^{2x}}-4x^2+C$.
	}
\end{ex}
\begin{ex}
	[Mã 103 - 2020 Lần 2]%Câu 27
	Biết $ F(x)=\mathrm{e}^x-x^2$ là một nguyên hàm của hàm số $ f(x)$ trên $\mathbb{R}$. Khi đó $\displaystyle\int{f\left(2x\right)\mathrm{d}x}$ bằng
	\choice
	{\True $\dfrac{1}{2}{\mathrm{e}^{2x}}-2x^2+C$}
	{$\mathrm{e}^{2x}-4x^2+C$}
	{$ 2\mathrm{e}^x-2x^2+C$}
	{$\dfrac{1}{2}{\mathrm{e}^{2x}}-x^2+C$}
	\loigiai{
		Ta có $\displaystyle\int{f\left(2x\right)\mathrm{d}x}=\dfrac{1}{2}\displaystyle\int{f\left(2x\right)\mathrm{d}\left(2x\right)}$ $=\dfrac{1}{2}F\left(2x\right)+C=\dfrac{1}{2}{\mathrm{e}^{2x}}-2x^2+C$.
	}
\end{ex}
\begin{ex}
	[Mã 104 - 2020 Lần 2]%Câu 28
	Biết $ F(x)=\mathrm{e}^x+2x^2$ là một nguyên hàm của hàm số $ f(x)$ trên $\mathbb{R}$. Khi đó $\displaystyle\int{f\left(2x\right)\mathrm{d}x}$ bằng
	\choice
	{$\mathrm{e}^{2x}+8x^2+C$}
	{$ 2\mathrm{e}^x+4x^2+C$}
	{$\dfrac{1}{2}{\mathrm{e}^{2x}}+2x^2+C$}
	{\True $\dfrac{1}{2}{\mathrm{e}^{2x}}+4x^2+C$}
	\loigiai{
		Đặt $ t=2x\Rightarrow\mathrm{d}t=2\mathrm{d}x\Rightarrow\mathrm{d}x=\dfrac{\mathrm{d}t}{\text{2}}$\\
		$\displaystyle\int{f\left(2x\right)\mathrm{d}x}=\dfrac{1}{2}\displaystyle\int{f(t)\mathrm{d}t}=\dfrac{1}{2}F(t)+C=\dfrac{1}{2}\left[\mathrm{e}^t+2t^2\right]+C=\dfrac{1}{2}{\mathrm{e}^{2x}}+\left(2x\right)^2+C=\dfrac{1}{2}{\mathrm{e}^{2x}}+4x^2+C$.
	}
\end{ex}
\begin{ex}
	[Thi thử Lômônôxốp-Hà Nội lần V 2019]%Câu 29
	Biết $\displaystyle\int{f\left(2x\right)\mathrm{d}x}=\sin^2x+\ln x+C$. Tìm nguyên hàm $\displaystyle\int{f(x)\mathrm{d}x}$?
	\choice
	{$\displaystyle\int{f(x)\mathrm{d}x}=\sin^2\dfrac{x}{2}+\ln x+C$}
	{$\displaystyle\int{f(x)\mathrm{d}x}=2\sin^22x+2\ln x+C$}
	{\True $\displaystyle\int{f(x)\mathrm{d}x}=2\sin^2\dfrac{x}{2}+2\ln x+C$}
	{$\displaystyle\int{f(x)\mathrm{d}x}=2\sin^2x+2\ln x+C$}
	\loigiai{
		Ta có $\displaystyle\int{f\left(2x\right)\mathrm{d}x}=\sin^2x+\ln x+C\Leftrightarrow\dfrac{1}{2}\displaystyle\int{f\left(2x\right)\mathrm{d}\left(2x\right)}=\dfrac{1-\cos 2x}{2}+\ln\left(2x\right)-\ln 2+C$\\
		$\Leftrightarrow $ $\displaystyle\int{f\left(2x\right)\mathrm{d}\left(2x\right)}=1-\cos 2x+2\ln\left(2x\right)-2\ln 2+2C$\\
		$\Leftrightarrow\displaystyle\int{f(x)\mathrm{d}x}=1-\cos x+2\ln x-2\ln 2+2C\Leftrightarrow\displaystyle\int{f(x)\mathrm{d}x}=2\sin^2\dfrac{x}{2}+2\ln x+C'$.
	}
\end{ex}
\begin{ex}
	Cho $\displaystyle\int{f(4x)}\mathrm{d}x=x^2+3x+C$. Mệnh đề nào dưới đây đúng?
	\choice
	{$\displaystyle\int{f(x+2)}\mathrm{d}x=\dfrac{x^2}{4}+2x+C$}
	{$\displaystyle\int{f(x+2)}\mathrm{d}x=x^2+7x+C$}
	{\True $\displaystyle\int{f(x+2)}\mathrm{d}x=\dfrac{x^2}{4}+4x+C$}
	{$\displaystyle\int{f(x+2)}\mathrm{d}x=\dfrac{x^2}{2}+4x+C$}
	\loigiai{
		Từ giả thiết bài toán $\displaystyle\int{f(4x)}\mathrm{d}x=x^2+3x+c$.\\
		Đặt $t=4x\Rightarrow\mathrm{d}t=4\mathrm{d}x$ từ đó ta có $\dfrac{1}{4}\displaystyle\int{f(t)}\mathrm{d}t=\left(\dfrac{t}{4}\right)^2+3\left(\dfrac{t}{4}\right)+C\Rightarrow\displaystyle\int{f(t)}\mathrm{d}t=\dfrac{t^2}{4}+3t+C$.\\
		Xét $\displaystyle\int{f(x+2)}\mathrm{d}x=\displaystyle\int{f(x+2)}\text{d(}x+2)=\dfrac{(x+2)^2}{4}+3(x+2)+c=\dfrac{x^2}{4}+4x+C$.\\
		Vậy mệnh đề đúng là $\displaystyle\int{f(x+2)}\mathrm{d}x=\dfrac{x^2}{4}+4x+C$.
	}
\end{ex}
\begin{ex}
	Cho $\displaystyle\int{f(x)\mathrm{d}x}=4x^3+2x+C_0$. Tính $I=\displaystyle\int{xf\left(x^2\right)\mathrm{d}x}$ .
	\choice
	{\True $ I=2x^6+x^2+C$}
	{$ I=\dfrac{x^{10}}{10}+\dfrac{x^6}{6}+C$}
	{$ I=4x^6+2x^2+C$}
	{$ I=12x^2+2$}
	\loigiai{
		Ta có $I=\displaystyle\int{xf\left(x^2\right)\mathrm{d}x}=\dfrac{1}{2}\displaystyle\int{f\left(x^2\right)\mathrm{d}{x^2}}=\dfrac{1}{2}\left(4\left(x^2\right)^3+2\left(x^2\right)\right)+C=2x^6+x^2+C$.
	}
\end{ex}
\begin{ex}
	[Sở Bắc Ninh 2019]%Câu 32
	Tìm họ nguyên hàm của hàm số $ f(x)=x^2.\mathrm{e}^{x^3+1}$.
	\choice
	{$\displaystyle\int{f(x)\mathrm{d}x=}\dfrac{x^3}{3}.\mathrm{e}^{x^3+1}+C$}
	{$\displaystyle\int{f(x)\mathrm{d}x=}3\mathrm{e}^{x^3+1}+C$}
	{$\displaystyle\int{f(x)\mathrm{d}x=}{\mathrm{e}^{x^3+1}}+C$}
	{\True $\displaystyle\int{f(x)\mathrm{d}x=}\dfrac{1}{3}{\mathrm{e}^{x^3+1}}+C$}
	\loigiai{
		$\displaystyle\int{f(x)\mathrm{d}x}=\displaystyle\int{x^2\mathrm{e}^{x^3+1}\mathrm{d}x}$$=\dfrac{1}{3}\displaystyle\int{\mathrm{e}^{x^3+1}\mathrm{d}\left(x^3+1\right)}=\dfrac{1}{3}{\mathrm{e}^{x^3+1}}+C$.}
\end{ex}
\begin{ex}
	[THPT Hà Huy Tập - 2018]%Câu 33
	Nguyên hàm của $ f(x)=\sin 2x.\mathrm{e}^{\sin^2x}$ là
	\choice
	{$\sin^2x.\mathrm{e}^{\sin^2x-1}+C$}
	{$\dfrac{\mathrm{e}^{\sin^2x+1}}{\sin^2x+1}+C$}
	{\True $\mathrm{e}^{\sin^2x}+C$}
	{$\dfrac{\mathrm{e}^{\sin^2x-1}}{\sin^2x-1}+C$}
	\loigiai{
		Ta có $\displaystyle\int{\sin 2x.\mathrm{e}^{\sin^2x}}\mathrm{d}x=\displaystyle\int{\mathrm{e}^{\sin^2x}}\mathrm{d}\left(\sin^2x\right)=\mathrm{e}^{\sin^2x}+C$.
	}
\end{ex}
\begin{ex}
	Tìm tất cả các họ nguyên hàm của hàm số $ f(x)=\dfrac{1}{x^9+3x^5}$
	\choice
	{$\displaystyle\int{f(x)\mathrm{d}x=-\dfrac{1}{3x^4}+\dfrac{1}{36}\ln\left|\dfrac{x^4}{x^4+3}\right|+C}$}
	{\True $\displaystyle\int{f(x)\mathrm{d}x=-\dfrac{1}{{12}{x^4}}-\dfrac{1}{36}\ln\left|\dfrac{x^4}{x^4+3}\right|+C}$}
	{$\displaystyle\int{f(x)\mathrm{d}x=-\dfrac{1}{3x^4}-\dfrac{1}{36}\ln\left|\dfrac{x^4}{x^4+3}\right|+C}$}
	{$\displaystyle\int{f(x)\mathrm{d}x=-\dfrac{1}{{12}{x^4}}+\dfrac{1}{36}\ln\left|\dfrac{x^4}{x^4+3}\right|+C}$}
	\loigiai{
		$\displaystyle\int{f(x)\mathrm{d}x}=\displaystyle\int{\dfrac{1}{x^9+3x^5}\mathrm{d}x}=\displaystyle\int{\dfrac{x^3}{\left(x^4\right)^2\left(x^4+3\right)}\mathrm{d}x}=\dfrac{1}{4}\displaystyle\int{\dfrac{\mathrm{d}{x^4}}{\left(x^4\right)^2\left(x^4+3\right)}}=\dfrac{1}{12}\displaystyle\int{\dfrac{\left(x^4+3\right)-x^4}{\left(x^4\right)^2\left(x^4+3\right)}\mathrm{d}{x^4}}$\\
		$=\dfrac{1}{12}\displaystyle\int{\dfrac{\mathrm{d}{x^4}}{\left(x^4\right)^2}-\dfrac{1}{12}\displaystyle\int{\dfrac{\mathrm{d}{x^4}}{x^4\left(x^4+3\right)}=}}-\dfrac{1}{12x^4}-\dfrac{1}{36}\ln\left(\dfrac{x^4}{x^4+3}\right)+C$.
	}
\end{ex}
\begin{ex}
	[Chuyên Lê Hồng Phong Nam Định 2019]%Câu 35
	Tìm hàm số $F(x)$ biết $F(x)=\displaystyle\int{\dfrac{x^3}{x^4+1}\mathrm{d}x}$ và $F(0)=1$.
	\choice
	{$ F(x)=\ln\left(x^4+1\right)+1$}
	{$ F(x)=\dfrac{1}{4}\ln\left(x^4+1\right)+\dfrac{3}{4}$}
	{\True $ F(x)=\dfrac{1}{4}\ln\left(x^4+1\right)+1$}
	{$ F(x)=4\ln\left(x^4+1\right)+1$}
	\loigiai{
		Ta có $ F(x)=\dfrac{1}{4}\displaystyle\int{\dfrac{1}{x^4+1}\mathrm{d}\left(x^4+1\right)}=\dfrac{1}{4}\ln\left(x^4+1\right)+C$.\\
		Do $ F(0)=1$ nên $\dfrac{1}{4}\ln\left(0+1\right)+C=1\Leftrightarrow C=1$.\\
		Vậy $ F(x)=\dfrac{1}{4}\ln\left(x^4+1\right)+1$.
	}
\end{ex}
\begin{ex}
	Biết $\displaystyle\int{\dfrac{\left(x-1\right)^{2017}}{\left(x+1\right)^{2019}}}\mathrm{d}x=\dfrac{1}{a}.\left(\dfrac{x-1}{x+1}\right)^b+C,x\ne-1$ với $ a,b\in{\mathbb{N}^*}$. Mệnh đề nào sau đây đúng?
	\choice
	{\True $a=2b$}
	{$b=2a$}
	{$a=2018b$}
	{$b=2018a$}
	\loigiai{
		Ta có\\
		$\displaystyle\int{\dfrac{\left(x-1\right)^{2017}}{\left(x+1\right)^{2019}}}\mathrm{d}x=\displaystyle\int{\left(\dfrac{x-1}{x+1}\right)^{2017}.\dfrac{1}{\left(x+1\right)^2}}\mathrm{d}x=\dfrac{1}{2}\displaystyle\int{\left(\dfrac{x-1}{x+1}\right)^{2017}}d\left(\dfrac{x-1}{x+1}\right)=\dfrac{1}{4036}.\left(\dfrac{x-1}{x+1}\right)^{2018}+C$.\\
		$\Rightarrow a=4036,b=2018$\\
		Do đó $ a=2b$.
	}
\end{ex}
\begin{ex}
	[Chuyên Quốc Học Huế - 2018]%Câu 37
	Biết rằng $ F(x)$ là một nguyên hàm trên $\mathbb{R}$ của hàm số $ f(x)=\dfrac{2017x}{\left(x^2+1\right)^{2018}}$ thỏa mãn $ F(1)=0$. Tìm giá trị nhỏ nhất $ m$ của $ F(x)$.
	\choice
	{$ m=-\dfrac{1}{2}$}
	{\True $ m=\dfrac{1-2^{2017}}{2^{2018}}$}
	{$ m=\dfrac{1+2^{2017}}{2^{2018}}$}
	{$ m=\dfrac{1}{2}$}
	\loigiai{
		Ta có $\displaystyle\int{f(x)\mathrm{d}x=\displaystyle\int{\dfrac{2017x}{\left(x^2+1\right)^{2018}}\mathrm{d}x}}$ $=\dfrac{2017}{2}\displaystyle\int{\left(x^2+1\right)^{-2018}d\left(x^2+1\right)}$ $=\dfrac{2017}{2}.\dfrac{\left(x^2+1\right)^{-2017}}{-2017}+C$ $=-\dfrac{1}{2\left(x^2+1\right)^{2017}}+C$$=F(x)$\\
		Mà $ F(1)=0$$\Rightarrow-\dfrac{1}{2.2^{2017}}+C=0\Rightarrow C=\dfrac{1}{2^{2018}}$\\
		Do đó $ F(x)=-\dfrac{1}{2.\left(x^2+1\right)^{2017}}+\dfrac{1}{2^{2018}}$ suy ra\\
		$ F(x)$ đạt giá trị nhỏ nhất khi và chỉ khi $\dfrac{1}{2\left(x^2+1\right)^{2017}}$ lớn nhất $\Leftrightarrow\left(x^2+1\right)$ nhỏ nhất$\Leftrightarrow x=0$\\
		Vậy $ m=-\dfrac{1}{2}+\dfrac{1}{2^{2018}}=\dfrac{1-2^{2017}}{2^{2018}}$.
	}
\end{ex}
\begin{ex}
	Cho $ F(x)$ là nguyên hàm của hàm số $ f(x)=\dfrac{1}{\mathrm{e}^x+1}$ và $ F(0)=-\ln 2e$. Tập nghiệm $ S$ của phương trình $ F(x)+\ln\left(\mathrm{e}^x+1\right)=2$ là:
	\choice
	{\True $ S=\left\{ 3\right\}$}
	{$ S=\left\{ 2;3\right\}$}
	{$ S=\left\{-2;3\right\}$}
	{$ S=\left\{-3;3\right\}$}
	\loigiai{
		Ta có $ F(x)=\displaystyle\int{f(x)\mathrm{d}x=\displaystyle\int{\dfrac{\mathrm{d}x}{\mathrm{e}^x+1}=}}\displaystyle\int{\left(1-\dfrac{\mathrm{e}^x}{\mathrm{e}^x+1}\right)\mathrm{d}x}=x-\ln\left(\mathrm{e}^x+1\right)+C$\\
		$ F(0)=-\ln 2+C=-\ln 2\mathrm{e}\Rightarrow C=-1$.\\  
		Phương trình: $F(x)+\ln\left(\mathrm{e}^x+1\right)=2\Leftrightarrow x-\ln\left(\mathrm{e}^x+1\right)-1+\ln\left(\mathrm{e}^x+1\right)=2\Leftrightarrow x=3$.
	}
\end{ex}
\begin{ex}
	[THPT Lê Quý Đôn Đà Nẵng 2019]%Câu 39
	Họ nguyên hàm của hàm số $ f(x)=x^3\left(x^2+1\right)^{2019}$là
	\choice
	{$\dfrac{1}{2}\left[\dfrac{\left(x^2+1\right)^{2021}}{2021}-\dfrac{\left(x^2+1\right)^{2020}}{2020}\right]$}
	{$\dfrac{\left(x^2+1\right)^{2021}}{2021}-\dfrac{\left(x^2+1\right)^{2020}}{2020}$}
	{$\dfrac{\left(x^2+1\right)^{2021}}{2021}-\dfrac{\left(x^2+1\right)^{2020}}{2020}+C$}
	{\True $\dfrac{1}{2}\left[\dfrac{\left(x^2+1\right)^{2021}}{2021}-\dfrac{\left(x^2+1\right)^{2020}}{2020}\right]+C$}
	\loigiai{
		Xét $\displaystyle\int{f(x)\mathrm{d}}x=\displaystyle\int{x^3\left(x^2+1\right)^{2019}\mathrm{d}}x=\displaystyle\int{x^2\left(x^2+1\right)^{2019}x\mathrm{d}}x$.\\
		Đổi biến $ t=x^2+1\Rightarrow\text{dt}=2x\mathrm{d}x$, ta có\\
		$\displaystyle\int{f(x)\mathrm{d}}x=\dfrac{1}{2}\displaystyle\int{\left(t-1\right)t^{2019}\text{dt}}=\dfrac{1}{2}\displaystyle\int{\left(t^{2020}-t^{2019}\right)\text{dt}}=$\\
		$=\dfrac{1}{2}\left[\dfrac{t^{2021}}{2021}-\dfrac{t^{2020}}{2020}\right]+C=\dfrac{1}{2}\left[\dfrac{\left(x^2+1\right)^{2021}}{2021}-\dfrac{\left(x^2+1\right)^{2020}}{2020}\right]+C$.
	}
\end{ex}
\begin{ex}
	[THPT Hà Huy Tập - 2018]%Câu 40
	Nguyên hàm của$ f(x)=\dfrac{1+\ln x}{x.\ln x}$ là
	\choice
	{$\displaystyle\int{\dfrac{1+\ln x}{x.\ln x}}\mathrm{d}x=\ln\left|\ln x\right|+C$}
	{$\displaystyle\int{\dfrac{1+\ln x}{x.\ln x}}\mathrm{d}x=\ln\left|x^2.\ln x\right|+C$}
	{$\displaystyle\int{\dfrac{1+\ln x}{x.\ln x}}\mathrm{d}x=\ln\left| x+\ln x\right|+C$}
	{\True $\displaystyle\int{\dfrac{1+\ln x}{x.\ln x}}\mathrm{d}x=\ln\left| x.\ln x\right|+C$}
	\loigiai{
		Ta có $I=\displaystyle\int{f(x)\mathrm{d}x=}\displaystyle\int{\dfrac{1+\ln x}{x.\ln x}\mathrm{d}x}$.\\
		Đặt $ x\ln x=t$$\Rightarrow\left(\ln x+1\right)\mathrm{d}x=\mathrm{d}t$. Khi đó ta có $I=\displaystyle\int{\dfrac{1+\ln x}{x.\ln x}\mathrm{d}x}$ $=\displaystyle\int{\dfrac{1}{t}\text{dt}}$ $=\ln\left| t\right|+C$ $=\ln\left| x.\ln x\right|+C$.
	}
\end{ex}
\begin{ex}
	[Chuyên Hạ Long - 2018]%Câu 41
	Tìm họ nguyên hàm của hàm số $ f(x)=x^2\mathrm{e}^{x^3+1}$
	\choice
	{$\displaystyle\int{\left(-t^{-5}+2t^{-3}-\dfrac{1}{t}\right)dt=}\dfrac{1}{4}{t^{-4}}-t^{-2}-\ln\left| t\right|+C$}
	{$\displaystyle\int{f(x)\mathrm{d}x=3\mathrm{e}^{x^3+1}+C}$}
	{\True $\displaystyle\int{f(x)\mathrm{d}x=\dfrac{1}{3}{\mathrm{e}^{x^3+1}}+C}$}
	{$\displaystyle\int{f(x)\mathrm{d}x=\dfrac{x^3}{3}{\mathrm{e}^{x^3+1}}+C}$}
	\loigiai{
		Đặt $t=x^3+1\Rightarrow\mathrm{d}t=3x^2\mathrm{d}x$\\
		Do đó, ta có $\displaystyle\int{f(x)\mathrm{d}x=\displaystyle\int{x^2\mathrm{e}^{x^3+1}\mathrm{d}x=}\displaystyle\int{\mathrm{e}^t.\dfrac{1}{3}\mathrm{d}t}=\dfrac{1}{3}{\mathrm{e}^t}+C=\dfrac{1}{3}{\mathrm{e}^{x^3+1}}+C}$.\\
		Vậy $\displaystyle\int{f(x)\mathrm{d}x=\dfrac{1}{3}{\mathrm{e}^{x^3+1}}+C}$.
	}
\end{ex}
\begin{ex}
	[Chuyên Lương Văn Chánh Phú Yên 2019]%Câu 42
	Nguyên hàm của hàm số $ f(x)=\sqrt[3]{3x+1}$ là
	\choice
	{$\displaystyle\int{f(x)\mathrm{d}x}=\left(3x+1\right)\sqrt[3]{3x+1}+C$}
	{$\displaystyle\int{f(x)\mathrm{d}x}=\sqrt[3]{3x+1}+C$}
	{$\displaystyle\int{f(x)\mathrm{d}x}=\dfrac{1}{3}\sqrt[3]{3x+1}+C$}
	{\True $\displaystyle\int{f(x)\mathrm{d}x}=\dfrac{1}{4}\left(3x+1\right)\sqrt[3]{3x+1}+C$}
	\loigiai{
		Ta có $\displaystyle\int{f(x)\mathrm{d}x}=\dfrac{1}{3}\displaystyle\int{\left(3x+1\right)^{\dfrac{1}{3}}\mathrm{d}\left(3x+1\right)}$ $=\dfrac{1}{4}\left(3x+1\right)\sqrt[3]{3x+1}+C$.
	}
\end{ex}
\begin{ex}
	Nguyên hàm của hàm số $ f(x)=\sqrt{3x+2}$ là
	\choice
	{$\dfrac{2}{3}(3x+2)\sqrt{3x+2}+C$}
	{$\dfrac{1}{3}(3x+2)\sqrt{3x+2}+C$}
	{\True $\dfrac{2}{9}(3x+2)\sqrt{3x+2}+C$}
	{$\dfrac{3}{2}\dfrac{1}{\sqrt{3x+2}}+C$}
	\loigiai{
		Do $\displaystyle\int{\sqrt{3x+2}\mathrm{d}x=}\dfrac{1}{3}\displaystyle\int{\left(3x+2\right)^{\dfrac{1}{2}}\mathrm{d}\left(3x+2\right)=\dfrac{1}{3}\dfrac{\left(3x+2\right)^{\dfrac{1}{2}+1}}{^{\dfrac{1}{2}+1}}}+C=\dfrac{2}{9}(3x+2)\sqrt{3x+2}+C$.
	}
\end{ex}
\begin{ex}
	[HSG Bắc Ninh 2019]%Câu 44
	Họ nguyên hàm của hàm số $ f(x)=\sqrt{2x+1}$ là
	\choice
	{$-\dfrac{1}{3}\left(2x+1\right)\sqrt{2x+1}+C$}
	{$\dfrac{1}{2}\sqrt{2x+1}+C$}
	{$\dfrac{2}{3}\left(2x+1\right)\sqrt{2x+1}+C$}
	{\True $\dfrac{1}{3}\left(2x+1\right)\sqrt{2x+1}+C$}
	\loigiai{
		Đặt $ t=\sqrt{2x+1}\Rightarrow\mathrm{d}t=\dfrac{1}{\sqrt{2x+1}}\mathrm{d}x\Rightarrow t\mathrm{d}t=\mathrm{d}x$\\
		$\Rightarrow\displaystyle\int{f(x)\mathrm{d}x=}\displaystyle\int{\sqrt{2x+1}}\mathrm{d}x=\displaystyle\int{t^2}\mathrm{d}x=\dfrac{t^3}{3}+C=\dfrac{1}{3}\left(2x+1\right)\sqrt{2x+1}+C$.
	}
\end{ex}
\begin{ex}
	[THPT An Lão Hải Phòng 2019]%Câu 45
	Cho hàm số $f(x)=2^{\sqrt{x}}.\dfrac{\ln 2}{\sqrt{x}}$. Hàm số nào dưới đây không là nguyên hàm của hàm số $f(x)$?
	\choice
	{\True $ F(x)=2^{\sqrt{x}}+C$}
	{$F(x)=2\left(2^{\sqrt{x}}-1\right)+C$}
	{$F(x)=2\left(2^{\sqrt{x}}+1\right)+C$}
	{$ F(x)=2^{\sqrt{x}+1}+C$}
	\loigiai{
		Ta có $F(x)=\displaystyle\int{f(x)\mathrm{d}x}$ $=\displaystyle\int{2^{\sqrt{x}}.\dfrac{\ln 2}{\sqrt{x}}\mathrm{d}x}$ $=\displaystyle\int{2^{\sqrt{x}}.\dfrac{\ln 2}{\sqrt{x}}\mathrm{d}x}$.\\
		Đặt $u=\sqrt{x}\Rightarrow\mathrm{d}u=\dfrac{1}{2\sqrt{x}}\mathrm{d}x$.\\
		Vậy $F(x)=2\ln 2.\displaystyle\int{2^u.\mathrm{d}u}$ $=2\ln 2.\dfrac{2^u}{\ln 2}+C$ $=2^{\sqrt{x}+1}+C$.\\
		Phương án B: $F(x)=2^{\sqrt{x}+1}-2+C$ thỏa.\\
		Phương án C: $F(x)=2^{\sqrt{x}+1}+2+C$ thỏa.
	}
\end{ex}
\begin{ex}
	[THPT Yên Phong Số 1 Bắc Ninh 2019]%Câu 46
	Khi tính nguyên hàm $\displaystyle\int{\dfrac{x-3}{\sqrt{x+1}}\mathrm{d}x}$, bằng cách đặt $ u=\sqrt{x+1}$ ta được nguyên hàm nào?
	\choice
	{\True $\displaystyle\int{2\left(u^2-4\right)\mathrm{d}u}$}
	{$\displaystyle\int{\left(u^2-4\right)\mathrm{d}u}$}
	{$\displaystyle\int{\left(u^2-3\right)\mathrm{d}u}$}
	{$\displaystyle\int{2u\left(u^2-4\right)\mathrm{d}u}$}
	\loigiai{
		Đặt $ u=\sqrt{x+1}$$\Rightarrow x=u^2-1\Rightarrow \mathrm{d}x=2u\mathrm{d}u$.\\
		Khi đó $\displaystyle\int{\dfrac{x-3}{\sqrt{x+1}}\mathrm{d}x}$ trở thành $\displaystyle\int{\dfrac{u^2-4}{u}.2u\mathrm{d}u=\displaystyle\int{2\left(u^2-4\right)\mathrm{d}u}}$.
	}
\end{ex}
\begin{ex}
	[Chuyên Hạ Long - 2018]%Câu 47
	Tìm họ nguyên hàm của hàm số $ f(x)=\dfrac{1}{2\sqrt{2x+1}}$.
	\choice
	{\True $\displaystyle\int{f(x)}\mathrm{d}x=\dfrac{1}{2}\sqrt{2x+1}+C$}
	{$\displaystyle\int{f(x)}\mathrm{d}x=\sqrt{2x+1}+C$}
	{$\displaystyle\int{f(x)}\mathrm{d}x=2\sqrt{2x+1}+C$}
	{$\displaystyle\int{f(x)}\mathrm{d}x=\dfrac{1}{\left(2x+1\right)\sqrt{2x+1}}+C$}
	\loigiai{
		Đặt $\sqrt{2x+1}=t$$\Rightarrow 2x+1=t^2$$\Rightarrow\mathrm{d}x=t\mathrm{d}t$.\\
		Khi đó ta có $\displaystyle\int{\dfrac{1}{2}\sqrt{2x+1}}\mathrm{d}x$ $=\dfrac{1}{2}\displaystyle\int{\dfrac{t\mathrm{d}t}{t}}$ $=$ $=\dfrac{1}{2}\displaystyle\int{\mathrm{d}t}$ $=\dfrac{1}{2}t+C$ $=\dfrac{1}{2}\sqrt{2x+1}+C$.
	}
\end{ex}
\begin{ex}
	[THCS - THPT Nguyễn Khuyến - 2018]%Câu 48
	Nguyên hàm của hàm số $f(x)=\ln\left(x+\sqrt{x^2+1}\right)$ là
	\choice
	{$F(x)=x\ln\left(x+\sqrt{x^2+1}\right)+\sqrt{x^2+1}+C$}
	{\True $F(x)=x\ln\left(x+\sqrt{x^2+1}\right)-\sqrt{x^2+1}+C$}
	{$F(x)=x\ln\left(x+\sqrt{x^2+1}\right)+C$}
	{$F(x)=x^2\ln\left(x+\sqrt{x^2+1}\right)+C$}
	\loigiai{
		Đặt $t=x+\sqrt{x^2+1}$ $\Leftrightarrow $ $t=\dfrac{\left(x+\sqrt{x^2+1}\right)\left(x-\sqrt{x^2+1}\right)}{x-\sqrt{x^2+1}}$=$\dfrac{-1}{x-\sqrt{x^2+1}}$ $\Rightarrow $ $\dfrac{1}{t}=\sqrt{x^2+1}-x$.\\
		$t-\dfrac{1}{t}=2x$ $\Rightarrow $ $\mathrm{d}x=\dfrac{1}{2}\left(1+\dfrac{1}{t^2}\right)$; $t+\dfrac{1}{t}=2\sqrt{x^2+1}$\\
		$\displaystyle\int{f(x)\mathrm{d}x}=\displaystyle\int{\ln\left(x+\sqrt{x^2+1}\right)}\mathrm{d}x$=$\dfrac{1}{2}\displaystyle\int{\left(1+\dfrac{1}{t^2}\right)\ln t}\mathrm{d}t$=$\dfrac{1}{2}\displaystyle\int{\left(1+\dfrac{1}{t^2}\right)\operatorname{lnt}.}\mathrm{d}t=I$.\\
		Đặt $u=\ln t$ $\to $ $\mathrm{d}u=\dfrac{1}{t}\mathrm{d}t$\\
		$\mathrm{d}v=\left(1+\dfrac{1}{t^2}\right)\mathrm{d}t$ $\to $ $v=t-\dfrac{1}{t}$;\\
		$I=\dfrac{1}{2}\left(t-\dfrac{1}{t}\right)\ln t-\dfrac{1}{2}\displaystyle\int{\dfrac{1}{t}\left(t-\dfrac{1}{t}\right)\mathrm{d}t}$=$\dfrac{1}{2}\left(t-\dfrac{1}{t}\right)\ln t-\dfrac{1}{2}\displaystyle\int{\left(1-\dfrac{1}{t^2}\right)\mathrm{d}t}$=$\dfrac{1}{2}\left(t-\dfrac{1}{t}\right)\ln t-\dfrac{1}{2}\left(t+\dfrac{1}{t}\right)+C$\\
		=$x\ln\left(x+\sqrt{x^2+1}\right)-\sqrt{x^2+1}+C$.
	}
\end{ex}
\begin{ex}
	[Chuyên Hạ Long - 2018]%Câu 49
	Biết rằng trên khoảng $\left(\dfrac{3}{2};+\infty\right)$, hàm số $ f(x)=\dfrac{20x^2-30x+7}{\sqrt{2x-3}}$ có một nguyên hàm $ F(x)=\left(a{x^2}+bx+c\right)\sqrt{2x-3}$ ($ a,b,c$ là các số nguyên). Tổng $ S=a+b+c$ bằng
	\choice
	{$ 4$}
	{\True $ 3$}
	{$ 5$}
	{$ 6$}
	\loigiai{
		Đặt $ t=\sqrt{2x-3}\Rightarrow{t^2}=2x-3\Rightarrow\mathrm{d}x=t\mathrm{d}t$\\
		Khi đó $\displaystyle\int{\dfrac{20x^2-30x+7}{\sqrt{2x-3}}\mathrm{d}x}$$=\displaystyle\int{\dfrac{20\left(\dfrac{t^2+3}{2}\right)^2-30\left(\dfrac{t^2+3}{2}\right)+7}{t}t\mathrm{d}t}$$=\displaystyle\int{\left(5t^4+15t^2+7\right)\mathrm{d}t}$$=t^5+5t^3+7t+C$$=\sqrt{\left(2x-3\right)^5}+5\sqrt{\left(2x-3\right)^3}+7\sqrt{2x-3}+C$$=\left(2x-3\right)^2\sqrt{2x-3}+5\left(2x-3\right)\sqrt{2x-3}+7\sqrt{2x-3}+C$$=\left(4x^2-2x+1\right)\sqrt{2x-3}+C$\\
		Vậy $ F(x)=\left(4x^2-2x+1\right)\sqrt{2x-3}$. Suy ra $ S=a+b+c=3$.
	}
\end{ex}
\begin{ex}
	[Chuyên Bắc Ninh 2019]%Câu 50
	Tìm nguyên hàm của hàm số $ f(x)=\dfrac{\sin x}{1+3\cos x}$.
	\choice
	{$\displaystyle\int{f(x)}\mathrm{d}x=\dfrac{1}{3}\ln\left| 1+3\cos x\right|+C$}
	{$\displaystyle\int{f(x)}\mathrm{d}x=\ln\left| 1+3\cos x\right|+C$}
	{$\displaystyle\int{f(x)}\mathrm{d}x=3\ln\left| 1+3\cos x\right|+C$}
	{\True $\displaystyle\int{f(x)}\mathrm{d}x=-\dfrac{1}{3}\ln\left| 1+3\cos x\right|+C$}
	\loigiai{
		Ta có $\displaystyle\int{\dfrac{\sin x}{1+3\cos x}}\mathrm{d}x=-\dfrac{1}{3}\displaystyle\int{\dfrac{1}{1+3\cos x}}\mathrm{d}\left(1+3\cos x\right)=-\dfrac{1}{3}\ln\left| 1+3\cos x\right|+C$.
	}
\end{ex}
\begin{ex}
	[Sở Thanh Hóa 2019]%Câu 51
	Tìm các hàm số $ f(x)$ biết $f'(x)=\dfrac{\cos x}{(2+\sin x)^2}$.
	\choice
	{$ f(x)=\dfrac{\sin x}{(2+\sin x)^2}+C$}
	{$ f(x)=\dfrac{1}{(2+\cos x)}+C$}
	{\True $ f(x)=-\dfrac{1}{2+\sin x}+C$}
	{$ f(x)=\dfrac{\sin x}{2+\sin x}+C$}
	\loigiai{
		Ta có $ f(x)=\displaystyle\int{f'(x)\mathrm{d}x}=\displaystyle\int{\dfrac{\cos x}{(2+\sin x)^2}}\mathrm{d}x=\displaystyle\int{\dfrac{\mathrm{d}(2+\sin x)}{(2+\sin x)^2}}=-\dfrac{1}{2+\sin x}+C$.
	}
\end{ex}
\begin{ex}
	[THPT Quang Trung Đống Đa Hà Nội 2019]%Câu 52
	Biết $F(x)$ là một nguyên hàm của hàm số $ f(x)=\dfrac{\sin x}{1+3\cos x}$ và $F\left(\dfrac{\pi}{2}\right)=2$. Tính $F(0).$ 
	\choice
	{$ F(0)=-\dfrac{1}{3}\ln 2+2$}
	{\True $ F(0)=-\dfrac{2}{3}\ln 2+2$}
	{$ F(0)=-\dfrac{2}{3}\ln 2-2$}
	{$ F(0=-\dfrac{1}{3}\ln 2-2$}
	\loigiai{
		Ta có $ F(x)=\displaystyle\int{\dfrac{\sin x\mathrm{d}x}{1+3\cos x}}=-\displaystyle\int{\dfrac{\mathrm{d}(\cos x)}{3\cos x+1}}$ $=-\dfrac{1}{3}ln\left| 3\cos x+1\right|+C$.\\
		mà $F\left(\dfrac{\pi}{2}\right)=-\dfrac{1}{3}ln\left|3cos\left(\dfrac{\pi}{2}\right)+1\right|+C=2$ $\Rightarrow C\text=2$.\\
		Do đó, $F(0)=-\dfrac{1}{3}ln\left|3cos(0)+1\right|+2=-\dfrac{1}{3}ln4+2\text=-\dfrac{2}{3}ln2+2$.\\
		Vậy $F(0)=-\dfrac{2}{3}ln2+2$.
	}
\end{ex}
\begin{ex}
	[Liên Trường THPT Tp Vinh Nghệ An 2019]%Câu 53
	Biết $\displaystyle\int{f(x)\mathrm{d}x=3x\cos\left(2x-5\right)+C}$. Tìm khẳng định đúng trong các khẳng định sau.
	\choice
	{\True $\displaystyle\int{f\left(3x\right)\mathrm{d}x=3x\cos\left(6x-5\right)+C}$}
	{$\displaystyle\int{f\left(3x\right)\mathrm{d}x=9x\cos\left(6x-5\right)+C}$}
	{$\displaystyle\int{f\left(3x\right)\mathrm{d}x=9x\cos\left(2x-5\right)+C}$}
	{$\displaystyle\int{f\left(3x\right)\mathrm{d}x=3x\cos\left(2x-5\right)+C}$}
	\loigiai{
		Đặt $ x=3t$$\Rightarrow\mathrm{d}x=3\mathrm{d}t$.\\
		Khi đó $\displaystyle\int{f(x)\mathrm{d}x=3x\cos\left(2x-5\right)+C}$ $\Leftrightarrow 3\displaystyle\int{f\left(3t\right)\mathrm{d}t=3.\left(3t\right)\cos\left(2.3t-5\right)+C}$ $\Leftrightarrow\displaystyle\int{f\left(3t\right)\mathrm{d}t=3t\cos\left(6t-5\right)+C}$$\Leftrightarrow\displaystyle\int{f\left(3x\right)\mathrm{d}x=3x\cos\left(6x-5\right)+C}$.
	}
\end{ex}
\begin{ex}
	[Chuyên Hạ Long - 2018]%Câu 54
	Tìm họ nguyên hàm của hàm số $f(x)=\tan^5x$.
	\choice
	{$\displaystyle\int{f(x)\mathrm{d}x=\dfrac{1}{4}{\tan^4}x-\dfrac{1}{2}{\tan^2}x}+\ln\left|\cos x\right|+C$}
	{$\displaystyle\int{f(x)\mathrm{d}x=\dfrac{1}{4}{\tan^4}x+\dfrac{1}{2}{\tan^2}x}-\ln\left|\cos x\right|+C$}
	{$\displaystyle\int{f(x)\mathrm{d}x=\dfrac{1}{4}{\tan^4}x+\dfrac{1}{2}{\tan^2}x}+\ln\left|\cos x\right|+C$}
	{\True $\displaystyle\int{f(x)\mathrm{d}x=\dfrac{1}{4}{\tan^4}x-\dfrac{1}{2}{\tan^2}x}-\ln\left|\cos x\right|+C$}
	\loigiai{
		$\begin{aligned}
			& I=\displaystyle\int f(x) \mathrm{d}x=\displaystyle\int \tan ^5 x \mathrm{~d}x=\displaystyle\int \dfrac{\sin ^5 x}{\cos ^5 x}\mathrm{~d}x \\
			& =\displaystyle\int \dfrac{\sin ^2 x \cdot \sin ^2 \cdot \sin \mathrm{x}}{\cos ^5 x}\mathrm{d}x=\displaystyle\int \dfrac{\left(1-\cos ^2 x\right) \cdot\left(1-\cos ^2 x\right) \cdot \sin \mathrm{x}}{\cos ^5 x}\mathrm{d}x \\
			& \text{Đặt}~t=\cos x \Rightarrow \mathrm{d}t=-\sin x \mathrm{d}x \quad I=\displaystyle\int \dfrac{\left(1-t^2\right) \cdot\left(1-t^2\right)}{t^5}(-\mathrm{d}t)=\displaystyle\int \dfrac{1-2 t^2+t^4}{t^5}(-\mathrm{d}t) \\
			&=\displaystyle\int\left(-\dfrac{1}{t^5}+\dfrac{2}{t^3}-\dfrac{1}{t}\right) \mathrm{d}t=\displaystyle\int\left(-t^{-5}+2 t^{-3}-\dfrac{1}{t}\right) \mathrm{d}t=\dfrac{1}{4}t^{-4}-t^{-2}-\ln |t|+C \\
			& =\dfrac{1}{4}\cos x^{-4}-\cos x^{-2}-\ln |\cos x|+C=\dfrac{1}{4}\cdot \dfrac{1}{\cos x^4}-\dfrac{1}{\cos x^2}-\ln |\cos x|+C \\
			& =\dfrac{1}{4}\cdot\left(\tan ^2 x+1\right)^2-\left(\tan ^2 x+1\right)-\ln |\cos x|+C \\
			& =\dfrac{1}{4}\left(\tan ^4 x+2 \tan ^2 x+1\right)-(\tan 2 x+1)-\ln |\cos x|+C \\
			& =\dfrac{1}{4}\tan ^4 x-\dfrac{1}{2}\tan ^2 x-\ln |\cos x|+\dfrac{1}{4}+C \\
			& =\dfrac{1}{4}\tan ^4 x-\dfrac{1}{2}\tan ^2 x-\ln |\cos x|+C .
		\end{aligned}$
	}
\end{ex}
\begin{ex}
	[Hồng Bàng - Hải Phòng - 2018]%Câu 55
	Biết $F(x)$ là một nguyên hàm của hàm số $f(x)=\sin^3x.\cos x$ và $F(0)=\pi $. Tính $F\left(\dfrac{\pi}{2}\right)$.
	\choice
	{$F\left(\dfrac{\pi}{2}\right)=-\pi $}
	{$F\left(\dfrac{\pi}{2}\right)=\pi $}
	{$F\left(\dfrac{\pi}{2}\right)=-\dfrac{1}{4}+\pi $}
	{\True $F\left(\dfrac{\pi}{2}\right)=\dfrac{1}{4}+\pi $}
	\loigiai{
		Đặt $t=\sin x\Rightarrow\mathrm{d}t=\cos x\mathrm{d}x$.\\
		$F(x)=\displaystyle\int{f(x)\mathrm{d}x}$ $=\displaystyle\int{\sin^3x\cos x\mathrm{d}x}$ $=\displaystyle\int{t^3\mathrm{d}t}$ $=\dfrac{t^4}{4}+C$ $=\dfrac{\sin^4x}{4}+C$.\\
		$F(0)=\pi $ $\Rightarrow\dfrac{\sin^4\pi}{4}+C=\pi $ $\Leftrightarrow C=\pi $ $\Rightarrow F(x)=\dfrac{\sin^4x}{4}+\pi $.\\
		$F\left(\dfrac{\pi}{2}\right)=\dfrac{\sin^4\dfrac{\pi}{2}}{4}$ $=\dfrac{1}{4}+\pi $.
	}
\end{ex}
\begin{ex}
	Cho $F(x)$ là một nguyên hàm của hàm số $f(x)=\dfrac{1}{x\ln x}$ thỏa mãn $F\left(\dfrac{1}{\mathrm{e}}\right)=2$ và $F\left(\mathrm{e}\right)=\ln 2.$ Giá trị của biểu thức $F\left(\dfrac{1}{\mathrm{e}^2}\right)+F\left(\mathrm{e}^2\right)$ bằng
	\choice
	{\True $3\ln 2+2$}
	{$\ln 2+2$}
	{$\ln 2+1$}
	{$2\ln 2+1$}
	\loigiai{
		Ta có $\displaystyle\int{\dfrac{1}{x\ln x}}\mathrm{d}x=\displaystyle\int{\dfrac{\mathrm{d}\left(\ln x\right)}{\ln x}}=\ln\left|\ln x\right|+C$, $ x>0$, $ x\ne 1$.\\
		Nên $ F(x)=\left\{\begin{aligned}
			&\ln\left(\ln x\right)+C_1\text{khi}x>1\\ 
			&\ln\left(-\ln x\right)+C_2\text{khi}0<x<1\\ 
		\end{aligned}\right.$.\\
		Mà $ F\left(\dfrac{1}{\mathrm{e}}\right)=2$ nên $\ln\left(-\ln\dfrac{1}{\mathrm{e}}\right)+C_2=2\Leftrightarrow{C_2}=2$; $ F\left(\mathrm{e}\right)=\ln 2$ nên $\ln\left(\ln\mathrm{e}\right)+C_1=\ln 2\Leftrightarrow{C_1}=\ln 2$.\\
		Suy ra $ F(x)=\left\{\begin{aligned}
			&\ln\left(\ln x\right)+\ln 2&\text{khi}~x>1\\ 
			&\ln\left(-\ln x\right)+2&\text{khi}~0<x<1\\ 
		\end{aligned}\right.$.\\
		Vậy $ F\left(\dfrac{1}{\mathrm{e}^2}\right)+F\left(\mathrm{e}^2\right)=\ln\left(-\ln\dfrac{1}{\mathrm{e}^2}\right)+2+\ln\left(\ln{\mathrm{e}^2}\right)+\ln 2=3\ln 2+2$.
	}
\end{ex}
\begin{ex}
	[Chuyên Nguyễn Huệ - HN 2019]%Câu 57
	Gọi $F(x)$ là nguyên hàm của hàm số $f(x)=\dfrac{x}{\sqrt{8-x^2}}$ thỏa mãn $F(2)=0$ . Khi đó phương trình $F(x)=x$ có nghiệm là:
	\choice
	{$x=0$}
	{$x=1$}
	{$x=-1$}
	{\True $x=1-\sqrt{3}$}
	\loigiai{
		Ta có $\displaystyle\int{\dfrac{x}{\sqrt{8-x^2}}\mathrm{d}x}=-\dfrac{1}{2}\displaystyle\int{\left(8-x^2\right)^{-\dfrac{1}{2}}\mathrm{d}\left(8-x^2\right)}=-\sqrt{8-x^2}+C.$\\
		Mặt khác $F(2)=0\Leftrightarrow-\sqrt{8-2^2}+C=0\Leftrightarrow C=2$.\\
		Nên $F(x)=-\sqrt{8-x^2}+2.$\\
		$\begin{aligned}
			&F(x)=x\Leftrightarrow-\sqrt{8-x^2}+2=x\Leftrightarrow\sqrt{8-x^2}=2-x\\ 
			&\Leftrightarrow\left\{\begin{aligned}
				& 2-x\ge 0\\ 
				& 8-x^2=\left(2-x\right)^2\\ 
			\end{aligned}\right.\Leftrightarrow\left\{\begin{aligned}
				& x\le 2\\ 
				&-2x^2+4x+4=0\\ 
			\end{aligned}\right.\Leftrightarrow\left\{\begin{aligned}
				& x\le 2\\ 
				&\left[\begin{aligned}
					& x=1+\sqrt{3}\\ 
					& x=1-\sqrt{3}\\ 
				\end{aligned}\right.\\ 
			\end{aligned}\right.\Leftrightarrow x=1-\sqrt{3}.\\ 
		\end{aligned}$
	}
\end{ex}
\begin{ex}
	Gọi $ F(x)$ là nguyên hàm của hàm số $ f(x)=\dfrac{2x}{\sqrt{x+1}}-\dfrac{1}{x^2}$. Biết $ F(3)=6$, giá trị của $ F(8)$ là
	\choice
	{\True $\dfrac{217}{8}$}
	{$27$}
	{$\dfrac{215}{24}$}
	{$\dfrac{215}{8}$}
	\loigiai{
		Ta có $\displaystyle\int{f(x)\mathrm{d}x}=\displaystyle\int{\left(\dfrac{2x}{\sqrt{x+1}}-\dfrac{1}{x^2}\right)\mathrm{d}x}=\displaystyle\int{\left(\dfrac{2\left(x+1\right)-2}{\sqrt{x+1}}-\dfrac{1}{x^2}\right)\mathrm{d}x}$\\
		$=2\displaystyle\int{\sqrt{x+1}\mathrm{d}x}-2\displaystyle\int{\left(\dfrac{1}{\sqrt{x+1}}\right)\mathrm{d}x}-\displaystyle\int{\dfrac{1}{x^2}\mathrm{d}x}$\\
		$=2\displaystyle\int{\left(x+1\right)^{\dfrac{1}{2}}d\left(x+1\right)}-2\displaystyle\int{\left(x+1\right)^{-\dfrac{1}{2}}d\left(x+1\right)}-\displaystyle\int{x^{-2}\mathrm{d}x}$\\
		$=\dfrac{4\left(x+1\right)^{\dfrac{3}{2}}}{3}-4\sqrt{x+1}+\dfrac{1}{x}+C$.\\
		Suy ra $ F(x)=\dfrac{4\left(x+1\right)^{\frac{3}{2}}}{3}-4\sqrt{x+1}+\dfrac{1}{x}+C$.\\
		Mặt khác $ F(3)=6\Leftrightarrow 6=\dfrac{4\left(3+1\right)^{\frac{3}{2}}}{3}-4\sqrt{3+1}+\dfrac{1}{3}+C\Leftrightarrow C=3$.\\
		Vậy $ F(8)=\dfrac{4\left(8+1\right)^{\frac{3}{2}}}{3}-4\sqrt{8+1}+\dfrac{1}{8}+3=\dfrac{217}{8}$.
	}
\end{ex}
\begin{ex}
	Họ nguyên hàm của hàm số $f(x)=\dfrac{20x^2-30x+7}{\sqrt{2x-3}}$ trên khoảng $\left(\dfrac{3}{2};+\infty\right)$ là
	\choice
	{$\left(4x^2+2x+1\right)\sqrt{2x-3}+C$}
	{$\left(4x^2-2x+1\right)\sqrt{2x-3}$}
	{$\left(3x^2-2x+1\right)\sqrt{2x-3}$}
	{\True $\left(4x^2-2x+1\right)\sqrt{2x-3}+C$}
	\loigiai{
		Xét trên khoảng $\left(\dfrac{3}{2};+\infty\right)$, ta có\\
		$\displaystyle\int{f(x)\mathrm{d}x}=\displaystyle\int{\dfrac{20x^2-30x+7}{\sqrt{2x-3}}}\mathrm{d}x=\displaystyle\int{\dfrac{10x\left(2x-3\right)+7}{\sqrt{2x-3}}\mathrm{d}x}$ .\\
		Đặt $u=\sqrt{2x-3}\Rightarrow{u^2}=2x-3\Rightarrow 2u\mathrm{d}u=2\mathrm{d}x\Rightarrow u\mathrm{d}u=\mathrm{d}x$ .\\
		Khi đó\\
		$\displaystyle\int{\dfrac{10x\left(2x-3\right)+7}{\sqrt{2x-3}}\mathrm{d}x}=\displaystyle\int{\dfrac{5\left(u^2+3\right){u^2}+7}{u}}u\mathrm{d}u=\displaystyle\int{\left[5\left(u^2+3\right){u^2}+7\right]}\mathrm{d}u=\displaystyle\int{\left[5u^4+15u^2+7\right]}\mathrm{d}u$\\
		$\begin{aligned}
			&=u^5+5u^3+7u+C=\left(u^4+5u^2+7\right)u+C=\left[\left(2x-3\right)^2+5\left(2x-3\right)+7\right]\sqrt{2x-3}+C\\ 
			&=\left(4x^2-2x+1\right)\sqrt{2x-3}+C.\\ 
		\end{aligned}$
	}
\end{ex}
\begin{dang}
	{Nguyên hàm của hàm số hữu tỉ}
\end{dang}
\begin{ex}
	[Đề Minh Họa 2020 Lần 1]%Câu 60
	Họ tất cả các nguyên hàm của hàm số $ f(x)=\dfrac{x+2}{x-1}$ trên khoảng $\left(1;+\infty\right)$ là
	\choice
	{\True $ x+3\ln\left(x-1\right)+C$}
	{$ x-3\ln\left(x-1\right)+C$}
	{$ x-\dfrac{3}{\left(x-1\right)^2}+C$}
	{$ x+\dfrac{3}{\left(x-1\right)^2}+C$}
	\loigiai{
		Trên khoảng $\left(1;+\infty\right)$ thì $ x-1>0$ nên\\
		$\displaystyle\int{f(x)\mathrm{d}x}=\displaystyle\int{\dfrac{x+2}{x-1}}\mathrm{d}x=\displaystyle\int{\left(1+\dfrac{3}{x-1}\right)}\mathrm{d}x=x+3\ln\left| x-1\right|+C=x+3\ln\left(x-1\right)+C$.
	}
\end{ex}
\begin{ex}
	[Mã đề 104 - BGD - 2019]%Câu 61
	Họ tất cả các nguyên hàm của hàm số $ f(x)=\dfrac{3x-2}{\left(x-2\right)^2}$ trên khoảng $\left(2;+\infty\right)$ là
	\choice
	{$ 3\ln\left(x-2\right)+\dfrac{2}{x-2}+C$}
	{$ 3\ln\left(x-2\right)-\dfrac{2}{x-2}+C$}
	{\True $ 3\ln\left(x-2\right)-\dfrac{4}{x-2}+C$}
	{$ 3\ln\left(x-2\right)+\dfrac{4}{x-2}+C$}
	\loigiai{
		Ta có $ f(x)=\dfrac{3x-2}{\left(x-2\right)^2}=\dfrac{3\left(x-2\right)+4}{\left(x-2\right)^2}=\dfrac{3}{x-2}+\dfrac{4}{\left(x-2\right)^2}$.\\ 
		Do đó $\displaystyle\int{\dfrac{3x-2}{\left(x-2\right)^2}}\mathrm{d}x=\displaystyle\int{\left(\dfrac{3}{x-2}+\dfrac{4}{\left(x-2\right)^2}\right)\mathrm{d}x}=3\ln\left(x-2\right)-\dfrac{4}{x-2}+C$.
	}
\end{ex}
\begin{ex}
	[Mã đề 101 - BGD - 2019]%Câu 62
	Họ tất cả các nguyên hàm của hàm số $ f(x)=\dfrac{2x-1}{\left(x+1\right)^2}$ trên khoảng$\left(-1;+\infty\right)$ là
	\choice
	{$ 2\ln\left(x+1\right)+\dfrac{2}{x+1}+C$}
	{\True $ 2\ln\left(x+1\right)+\dfrac{3}{x+1}+C$}
	{$ 2\ln\left(x+1\right)-\dfrac{2}{x+1}+C$}
	{$ 2\ln\left(x+1\right)-\dfrac{3}{x+1}+C$}
	\loigiai{
		Ta có\\ 
		$\displaystyle\int f(x) \mathrm{d}x=\displaystyle\int \dfrac{2 x-1}{(x+1)^2}\mathrm{~d}x=\int \dfrac{2(x+1)-3}{(x+1)^2}\mathrm{d}x=\displaystyle\int\left[\dfrac{2}{x+1}-\dfrac{3}{(x+1)^2}\right] \mathrm{d}x=2 \ln (x+1)+\dfrac{3}{x+1}+C$.
	}
\end{ex}
\begin{ex}
	Họ nguyên hàm của hàm số $ f(x)=\dfrac{x+3}{x^2+3x+2}$ là
	\choice
	{$\ln\left| x+1\right|+2\ln\left| x+2\right|+C$}
	{$ 2\ln\left| x+1\right|+\ln\left| x+2\right|+C$}
	{\True $ 2\ln\left| x+1\right|-\ln\left| x+2\right|+C$}
	{$-\ln\left| x+1\right|+2\ln\left| x+2\right|+C$}
	\loigiai{
		Ta có $ f(x)=\dfrac{x+3}{x^2+3x+2}=\dfrac{x+3}{\left(x+1\right)\left(x+2\right)}=\dfrac{2}{x+1}-\dfrac{1}{x+2}$.\\
		Suy ra họ nguyên hàm của hàm số $ f(x)=\dfrac{x+3}{x^2+3x+2}$ là $ 2\ln\left| x+1\right|-\ln\left| x+2\right|+C$.
	}
\end{ex}
\begin{ex}
	[Chuyên Lê Quý Dôn Diện Biên 2019]%Câu 64
	Tìm một nguyên hàm $ F(x)$của hàm số $ f(x)=ax+\dfrac{b}{x^2}\left(x\ne 0\right),$biết rằng $ F\left(-1\right)=1,F(1)=4,f(1)=0$
	\choice
	{$F(x)=\dfrac{3}{2}{x^2}+\dfrac{3}{4x}-\dfrac{7}{4}$}
	{$F(x)=\dfrac{3}{4}{x^2}-\dfrac{3}{2x}-\dfrac{7}{4}$}
	{\True $F(x)=\dfrac{3}{4}{x^2}+\dfrac{3}{2x}+\dfrac{7}{4}$}
	{$F(x)=\dfrac{3}{2}{x^2}-\dfrac{3}{2x}-\dfrac{1}{2}$}
	\loigiai{
		Ta có $F(x)=\displaystyle\int{f(x){\mathrm{d}x}=\displaystyle\int{\left(ax+\dfrac{b}{x^2}\right){\mathrm{d}x}}=}\dfrac{1}{2}a{x^2}-\dfrac{b}{x}+C$.\\
		Theo bài ra $\left\{\begin{aligned}
			& F\left(-1\right)=1\\ 
			& F(1)=4\\ 
			& f(1)=0\\ 
		\end{aligned}\right.\Leftrightarrow\left\{\begin{aligned}
			&\dfrac{1}{2}a+b+C=1\\ 
			&\dfrac{1}{2}a-b+C=4\\ 
			& a+b=0\\ 
		\end{aligned}\right.\Leftrightarrow\left\{\begin{aligned}
			& b=-\dfrac{3}{2}\\ 
			& a=\dfrac{3}{2}\\ 
			& C=\dfrac{7}{4}\\ 
		\end{aligned}\right.$.\\
		Vậy $F(x)=\dfrac{3}{4}{x^2}+\dfrac{3}{2x}+\dfrac{7}{4}$.
	}
\end{ex}
\begin{ex}
	Cho biết $\displaystyle\int{\dfrac{2x-13}{\left(x+1\right)\left(x-2\right)}}\mathrm{d}x=a\ln\left| x+1\right|+b\ln\left| x-2\right|+C$.\\
	Mệnh đề nào sau đây đúng?
	\choice
	{$ a+2b=8$}
	{$ a+b=8$}
	{$ 2a-b=8$}
	{\True $ a-b=8$}
	\loigiai{
		Ta có $\dfrac{2x-13}{(x+1)(x-2)}=\dfrac{A}{x+1}+\dfrac{B}{x-2}=\dfrac{A(x-2)+B(x+1)}{(x+1)(x-2)}=\dfrac{(A+B) x+(-2 A+B)}{(x+1)(x-2)}$.\\ 
		$\Rightarrow\left\{\begin{aligned}
			&A+B=2 \\ 
			&-2 A+B=-13\\
		\end{aligned}\right.\Leftrightarrow\left\{\begin{aligned}
			&A=5 \\ 
			&B=-3\\
		\end{aligned}\right.$.\\
		Khi đó $\displaystyle\int \dfrac{2 x-13}{(x+1)(x-2)}\mathrm{d}x=\int\left(\dfrac{5}{x+1}-\dfrac{3}{x-2}\right) \mathrm{d}x=5 \ln |x+1|-3 \ln |x-2|+C$.\\
		Suy ra $a=5 ; b=-3$ nên $a-b=8$.
	}
\end{ex}
\begin{ex}
	Cho biết $\displaystyle\int{\dfrac{1}{x^3-x}}{\mathrm{d}x}=a\ln\left|\left(x-1\right)\left(x+1\right)\right|+b\ln\left| x\right|+C$ . Tính giá trị biểu thức $ P=2a+b$.
	\choice
	{\True $0$}
	{$-1$}
	{$\dfrac{1}{2}$}
	{$1$}
	\loigiai{
		Ta có\\ $\dfrac{1}{x^3-x}=\dfrac{A}{x}+\dfrac{B}{x-1}+\dfrac{D}{x+1}$\\ $=\dfrac{A\left(x^2-1\right)+Bx\left(x+1\right)+\mathrm{d}x\left(x-1\right)}{x^3-x}$ $=\dfrac{\left(A+B+D\right){x^2}+\left(B-D\right)x-A}{x^3-x}$\\
		$\Rightarrow\left\{\begin{aligned}
			& A+B+D=0\\ 
			& B-D=0\\ 
			&-A=1\\ 
		\end{aligned}\right.\Leftrightarrow\left\{\begin{aligned}
			& A=-1\\ 
			& B=\dfrac{1}{2}\\ 
			& D=\dfrac{1}{2}\\ 
		\end{aligned}\right.$.\\
		Khi đó $\displaystyle\int{\dfrac{1}{x^3-x}}{\mathrm{d}x}=\displaystyle\int{\left(-\dfrac{1}{x}+\dfrac{1}{2\left(x-1\right)}+\dfrac{1}{2\left(x+1\right)}\right)}{\mathrm{d}x}$ $=\dfrac{1}{2}\ln\left|\left(x-1\right)\left(x+1\right)\right|-\ln\left| x\right|+C$.\\
		Suy ra $ a=\dfrac{1}{2};b=-1$ nên $ P=2a+b=0$.
	}
\end{ex}
\begin{ex}
	Cho biết $\displaystyle\int{\dfrac{4x+11}{x^2+5x+6}}{\mathrm{d}x}=a\ln\left| x+2\right|+b\ln\left| x+3\right|+C$. Tính giá trị biểu thức $ P=a^2+ab+b^2$.
	\choice
	{$12$}
	{\True $13$}
	{$14$}
	{$15$}
	\loigiai{
		Ta có $\dfrac{4x+11}{x^2+5x+6}=\dfrac{A}{x+2}+\dfrac{B}{x+3}$ $=\dfrac{A\left(x+3\right)+B\left(x+2\right)}{\left(x+2\right)\left(x+3\right)}$ $=\dfrac{\left(A+B\right)x+\left(3A+2B\right)}{\left(x+2\right)\left(x+3\right)}$\\
		$\Rightarrow\left\{\begin{aligned}
			& A+B=4\\ 
			& 3A+2B=11\\ 
		\end{aligned}\right.\Leftrightarrow\left\{\begin{aligned}
			& A=3\\ 
			& B=1\\ 
		\end{aligned}\right.$.\\
		Khi đó $\displaystyle\int{\dfrac{4x+11}{x^2+5x+6}}{\mathrm{d}x}=\displaystyle\int{\left(\dfrac{3}{x+2}+\dfrac{1}{x+3}\right)}{\mathrm{d}x}$ $=3\ln\left| x+2\right|+\ln\left| x+3\right|+C$.\\
		Suy ra $ a=3;b=1$ nên $ P=a^2+ab+b^2=13$.
	}
\end{ex}
\begin{ex}
	Cho hàm số $f(x)$ thỏa mãn $f'(x)=a{x^2}+\dfrac{b}{x^3}$, $f'(1)=3$, $f(1)=2$, $f\left(\dfrac{1}{2}\right)=-\dfrac{1}{12}$. Khi đó $2a+b$ bằng
	\choice
	{$-\dfrac{3}{2}$}
	{$ 0$}
	{\True $5$}
	{$\dfrac{3}{2}$}
	\loigiai{
		Ta có $f'(1)=3$ $\Rightarrow a+b=3(1)$.\\
		Hàm số có đạo hàm liên tục trên khoảng $\left(0;+\infty\right)$, các điểm $ x=1$, $ x=\dfrac{1}{2}$ đều thuộc $\left(0;+\infty\right)$ nên\\
		$ f(x)=\displaystyle\int{f'(x)\mathrm{d}x}=\displaystyle\int{\left(a{x^2}+\dfrac{b}{x^3}\right)\mathrm{d}x}=\dfrac{a{x^3}}{3}-\dfrac{b}{2x^2}+C$.\\
		$f(1)=2$ $\Rightarrow $ $\dfrac{a}{3}-\dfrac{b}{2}+C=2(2)$.\\
		$f\left(\dfrac{1}{2}\right)=-\dfrac{1}{12}$ $\Rightarrow $ $\dfrac{a}{24}-2b+C=-\dfrac{1}{12}(3)$.\\
		Từ $(1)$, $(2)$ và $(3)$ ta được hệ phương trình $\left\{\begin{aligned}
			& a+b=3\\ 
			&\dfrac{a}{3}-\dfrac{b}{2}+C=2\\ 
			&\dfrac{a}{24}-2b+C=-\dfrac{1}{12}\\ 
		\end{aligned}\right.$ $\Leftrightarrow $ $\left\{\begin{aligned}
			& a=2\\ 
			& b=1\\ 
			& C=\dfrac{11}{6}\\ 
		\end{aligned}\right.$ $\Rightarrow $ $ 2a+b=2.2+1=5$.
	}
\end{ex}
\begin{ex}
	[Mã 102 2019]%Câu 69
	Họ tất cả các nguyên hàm của hàm số $f(x)=\dfrac{3x-1}{(x-1)^2}$ trên khoảng $(1;+\infty)$ là
	\choice
	{$ 3\ln (x-1)-\dfrac{1}{x-1}+c$}
	{$ 3\ln (x-1)+\dfrac{2}{x-1}+c$}
	{\True $ 3\ln (x-1)-\dfrac{2}{x-1}+c$}
	{$ 3\ln (x-1)+\dfrac{1}{x-1}+c$}
	\loigiai{
		Ta có $f(x)=\dfrac{3x-3+2}{(x-1)^2}=\dfrac{3(x-1)+2}{(x-1)^2}=\dfrac{3}{x-1}+\dfrac{2}{(x-1)^2}$\\
		Vậy $\displaystyle\int{f(x)}\mathrm{d}x=\displaystyle\int(\dfrac{3}{x-1}+\dfrac{2}{(x-1)^2})\mathrm{d}x$ $=3\displaystyle\int{\dfrac{\mathrm{d}(x-1)}{x-1}}+2\displaystyle\int{\dfrac{\mathrm{d}(x-1)}{(x-1)^2}}$\\
		$=3\ln\left| x-1\right|+2\displaystyle\int{(x-1)^{-2}\mathrm{d}(x-1)}$ $=3\ln (x-1)-\dfrac{2}{x-1}+C$ vì $ x>1$.
	}
\end{ex}
\begin{ex}
	[Mã 103 - 2019]%Câu 70
	Họ tất cả các nguyên hàm của hàm số $ f(x)=\dfrac{2x+1}{\left(x+2\right)^2}$ trên khoảng $\left(-2;+\infty\right)$ là
	\choice
	{$2\ln\left(x+2\right)+\dfrac{3}{x+2}+C$}
	{\True $2\ln\left(x+2\right)+\dfrac{1}{x+2}+C$}
	{$2\ln\left(x+2\right)-\dfrac{1}{x+2}+C$}
	{$2\ln\left(x+2\right)-\dfrac{3}{x+2}+C$}
	\loigiai{
		Đặt $ x+2=t\Rightarrow x=t-1\Rightarrow \mathrm{d}x=\mathrm{d}t$ với $ t>0$\\
		Ta có $\displaystyle\int{f(x)\mathrm{d}x=\displaystyle\int{\dfrac{2t-1}{t^2}}}\mathrm{d}t\text{=}\displaystyle\int{\left(\dfrac{2}{t}-\dfrac{1}{t^2}\right)\mathrm{d}t=2\ln t+\dfrac{1}{t}}+C$\\
		Hay $\displaystyle\int{f(x)}\mathrm{d}x=2\ln\left(x+2\right)+\dfrac{1}{x+2}+C$.
	}
\end{ex}
\begin{ex}
	[THPT Yên Khánh - Ninh Bình - 2019]%Câu 71
	Cho $F(x)$ là một nguyên hàm của hàm số $f(x)=\dfrac{2x+1}{x^4+2x^3+x^2}$ trên khoảng $\left(0;+\infty\right)$ thỏa mãn $ F(1)=\dfrac{1}{2}$. Giá trị của biểu thức $ S=F(1)+F(2)+F(3)+\ldots+F\left(2019\right)$ bằng
	\choice
	{$\dfrac{2019}{2020}$}
	{$\dfrac{2019.2021}{2020}$}
	{\True $ 2018\dfrac{1}{2020}$}
	{$-\dfrac{2019}{2020}$}
	\loigiai{
		Ta có $f(x)=\dfrac{2x+1}{x^4+2x^3+x^2}=\dfrac{2x+1}{x^2\left(x+1\right)^2}$.\\
		Đặt $ t=x\left(x+1\right)=x^2+x$ $\Rightarrow\mathrm{d}t=\left(2x+1\right)\mathrm{d}x$.\\
		Khi đó $ F(x)=\displaystyle\int{f(x)\mathrm{d}x=\displaystyle\int{\dfrac{1}{t^2}\mathrm{d}t}}=-\dfrac{1}{t}+C=-\dfrac{1}{x\left(x+1\right)}+C$.\\
		Mặt khác, $ F(1)=\dfrac{1}{2}$ $\Rightarrow-\dfrac{1}{2}+C=\dfrac{1}{2}$ $\Rightarrow C=1$.\\
		Vậy $ F(x)=-\dfrac{1}{x\left(x+1\right)}+1$.\\
		Suy ra\\
		$\begin{aligned}
			&S=F(1)+F(2)+F(3)+\ldots+F\left(2019\right)=-\left(\dfrac{1}{1.2}+\dfrac{1}{2.3}+\dfrac{1}{3.4}+\ldots+\dfrac{1}{2019.2020}\right)+2019\\ 
			&=-\left(1-\dfrac{1}{2}+\dfrac{1}{2}-\dfrac{1}{3}+\dfrac{1}{3}-\dfrac{1}{4}+\ldots+\dfrac{1}{2019}-\dfrac{1}{2020}\right)+2019=-\left(1-\dfrac{1}{2020}\right)+2019\\ 
			&=2018+\dfrac{1}{2020}=2018\dfrac{1}{2020}. 
		\end{aligned}$
	}
\end{ex}
\begin{ex}
	Giả sử $\displaystyle\int{\dfrac{\left(2x+3\right)\mathrm{d}x}{x\left(x+1\right)\left(x+2\right)\left(x+3\right)+1}=-\dfrac{1}{g(x)}+C}$ ($ C$ là hằng số). Tính tổng các nghiệm của phương trình $ g(x)=0$.
	\choice
	{$-1$}
	{$ 1$}
	{$ 3$}
	{\True $-3$}
	\loigiai{
		Ta có $ x\left(x+1\right)\left(x+2\right)\left(x+3\right)+1=\left(x^2+3x\right)\left(x^2+3x+2\right)+1=\left[\left(x^2+3x\right)+1\right]^2$.\\
		Đặt $ t=x^2+3x$, khi đó $\mathrm{d}t=\left(2x+3\right)\mathrm{d}x$.\\
		Tích phân ban đầu trở thành $\displaystyle\int{\dfrac{\mathrm{d}t}{\left(t+1\right)^2}=-\dfrac{1}{t+1}+C}$.\\
		Trở lại biến $ x$, ta có $\displaystyle\int{\dfrac{\left(2x+3\right)\mathrm{d}x}{x\left(x+1\right)\left(x+2\right)\left(x+3\right)+1}=-\dfrac{1}{x^2+3x+1}+C}$.\\
		Vậy $ g(x)=x^2+3x+1$.\\
		$ g(x)=0\Leftrightarrow{x^2}+3x+1=0\Leftrightarrow\left[\begin{aligned}
			& x=\dfrac{-3+\sqrt{5}}{2}\\ 
			& x=\dfrac{-3-\sqrt{5}}{2}\\ 
		\end{aligned}\right.$.\\
		Vậy tổng tất cả các nghiệm của phương trình bằng $-3$.
	}
\end{ex}
\begin{ex}
	[Nam Trực - Nam Định - 2018]%Câu 73
	Cho $ I=\displaystyle\int{\dfrac{1}{x^3\left(1+x^2\right)}\mathrm{d}x}$ $=\dfrac{-a}{x^2}-b\ln\left| x\right|+2c\ln\left(1+x^2\right)+C$. Khi đó $ S=a+b+c$ bằng
	\choice
	{$\dfrac{-1}{4}$}
	{$\dfrac{3}{4}$}
	{\True $\dfrac{7}{4}$}
	{$ 2$}
	\loigiai{
		$ I=\displaystyle\int{\dfrac{x}{x^4\left(1+x^2\right)}\mathrm{d}x}$\\
		$ t=1+x^2$ $\Rightarrow\mathrm{d}t=2x\mathrm{d}x$\\
		$\Rightarrow I=\dfrac{1}{2}\displaystyle\int{\dfrac{1}{\left(t-1\right)^2.t}\mathrm{d}t}=\dfrac{1}{2}\displaystyle\int{\left(\dfrac{-1}{t-1}+\dfrac{1}{\left(t-1\right)^2}+\dfrac{1}{t}\right)\mathrm{d}t}=\dfrac{1}{2}\left(-\ln\left|t-1\right|-\dfrac{1}{t-1}+\ln\left|t\right|\right)+C$\\
		$=\dfrac{1}{2}\left(-\ln\left|x^2\right|-\dfrac{1}{x^2}+\ln\left|1+x^2\right|\right)+C$ $=-\dfrac{1}{2x^2}-\ln\left|x\right|+\dfrac{1}{2}\ln\left(1+x^2\right)+C$\\
		$\Rightarrow\left\{\begin{aligned}
			& a=\dfrac{1}{2}\\ 
			& b=1\\ 
			& c=\dfrac{1}{4}\\ 
		\end{aligned}\right.$ $\Rightarrow S=a+b+c=\dfrac{7}{4}$.
	}
\end{ex}
\begin{ex}
	[Trường VINSCHOOL - 2020]%Câu 74
	Cho hàm số $f(x)$ xác định trên $R\setminus\left\{-1;1\right\}$ thỏa mãn $f'(x)=\dfrac{1}{x^2-1}$ . Biết $f(3)+f\left(-3\right)=4$ và $f\left(\dfrac{1}{3}\right)+f\left(\dfrac{-1}{3}\right)=2$ . Giá trị của biểu thức $f\left(-5\right)+f(0)+f(2)$ bằng
	\choice
	{\True $5-\dfrac{1}{2}\ln 2$}
	{$6-\dfrac{1}{2}\ln 2$}
	{$5+\dfrac{1}{2}\ln 2$}
	{$6+\dfrac{1}{2}\ln 2$}
	\loigiai{
		Ta có $f'(x)=\dfrac{1}{x^2-1}$ $\Rightarrow f(x)=\displaystyle\int{f'(x)\mathrm{d}x=\displaystyle\int{\dfrac{1}{x^2-1}\mathrm{d}x=\dfrac{1}{2}\ln\left|\dfrac{x-1}{x+1}\right|+C}}$ với $x\in $ $R\setminus\left\{-1;1\right\}$.\\
		Khi đó $f(x)=\left\{\begin{aligned}
			&\dfrac{1}{2}\ln\left|\dfrac{x-1}{x+1}\right|+C_1&\text{khi}&x>1\\ 
			&\dfrac{1}{2}\ln\left|\dfrac{x-1}{x+1}\right|+C_2&\text{khi}&-1<x<1\\ 
			&\dfrac{1}{2}\ln\left|\dfrac{x-1}{x+1}\right|+C_3&\text{khi}&x<-1\\ 
		\end{aligned}\right.$ $\Rightarrow\left\{\begin{aligned}
			& f(3)+f\left(-3\right)=C_1+C_3=4\\ 
			& f\left(\dfrac{1}{3}\right)+f\left(\dfrac{-1}{3}\right)=2C_2=2\\ 
		\end{aligned}\right.$ $\Rightarrow\left\{\begin{aligned}
			&{C_1}+C_3=4\\ 
			&{C_2}=1\\ 
		\end{aligned}\right.$\\
		Vậy $f\left(-5\right)+f(0)+f(2)$ $=\dfrac{1}{2}\ln\dfrac{3}{2}+C_3+C_2+\dfrac{1}{2}\ln\dfrac{1}{3}+C_1=\dfrac{1}{2}\ln\dfrac{1}{2}+5=5-\dfrac{1}{2}\ln 2$.
	}
\end{ex}
\begin{ex}
	[Quảng Xương - Thanh Hóa - 2018]%Câu 75
	Cho hàm số $ f(x)$ xác định trên $\mathbb{R}\setminus\left\{-2;1\right\}$ thỏa mãn$f'(x)=\dfrac{1}{x^2+x-2}$, $ f\left(-3\right)-f(3)=0$ và $ f(0)=\dfrac{1}{3}$. Giá trị của biểu thức $ f\left(-4\right)+f\left(-1\right)-f(4)$ bằng
	\choice
	{\True $\dfrac{1}{3}\ln 2+\dfrac{1}{3}$}
	{$\ln 80+1$}
	{$\dfrac{1}{3}\ln\dfrac{4}{5}+\ln 2+1$}
	{$\dfrac{1}{3}\ln\dfrac{8}{5}+1$}
	\loigiai{
		$f(x)=\displaystyle\int{\dfrac{1}{x^2+x-2}\mathrm{d}x}=\left\{\begin{aligned}
			&\dfrac{1}{3}\ln\left|\dfrac{x-1}{x+2}\right|+C_1,\forall x\in\left(-\infty ;-2\right)\\ 
			&\dfrac{1}{3}\ln\left|\dfrac{x-1}{x+2}\right|+C_2,\forall x\in\left(-2;1\right)\\ 
			&\dfrac{1}{3}\ln\left|\dfrac{x-1}{x+2}\right|+C_3,\forall x\in\left(1;+\infty\right)\\ 
		\end{aligned}\right.$.\\
		Ta có $ f\left(-3\right)=\dfrac{1}{3}\ln 4+C_1,\forall x\in\left(-\infty ;2\right)$, $ f(0)=\dfrac{1}{3}\ln\dfrac{1}{2}+C_1,\forall x\in\left(-2;1\right)$,\\
		$ f(3)=\dfrac{1}{3}\ln\dfrac{2}{5}+C_3,\forall x\in\left(1;+\infty\right)$,\\
		Theo giả thiết ta có $f(0)=\dfrac{1}{3}$ $\Leftrightarrow{C_2}=\dfrac{1}{3}\left(1+\ln 2\right)$ .\\
		$\Rightarrow f\left(-1\right)=\dfrac{2}{3}\ln 2+\dfrac{1}{3}$ .\\
		Và $ f\left(-3\right)-f(3)=0\Leftrightarrow{C_1}-C_3=\dfrac{1}{3}\ln\dfrac{1}{10}$.\\
		Vậy $ f\left(-4\right)+f\left(-1\right)-f(4)=\dfrac{1}{3}\ln\dfrac{5}{2}+C_1+\dfrac{1}{3}\ln 2+\dfrac{1}{3}+\dfrac{1}{3}\ln 2+\dfrac{1}{3}\ln 2-C_2=\dfrac{1}{3}\ln 2+\dfrac{1}{3}$.
	}
\end{ex}
\begin{ex}
	[Chuyên Nguyễn Quang Diêu - Dồng Tháp - 2018]%Câu 76
	Cho hàm số $f(x)$ xác định trên $\mathbb{R}\setminus\left\{ 1\right\}$ thỏa mãn $f'(x)=\dfrac{1}{x-1}$, $f(0)=2017$, $f(2)=2018$. Tính $S=\left(f(3)-2018\right)\left(f\left(-1\right)-2017\right)$.
	\choice
	{$S=1$}
	{$S=1+\ln^22$}
	{$S=2\ln 2$}
	{\True $S=\ln^22$}
	\loigiai{
		Ta có $f(x)=\displaystyle\int{\dfrac{1}{x-1}\mathrm{d}x}$ $=\ln\left| x-1\right|+C$ $=\left\{\begin{aligned}
			&\ln\left(x-1\right)+C_1\text{khi}x>1\\ 
			&\ln\left(1-x\right)+C_2\text{khi}x<1\\ 
		\end{aligned}\right.$.\\
		Lại có $f(0)=2017$ $\Rightarrow\ln\left(1-0\right)+C_2=2017$ $\Rightarrow{C_2}=2017$.\\
		$f(2)=2018$ $\ln\left(2-1\right)+C_1=2018$ $\Rightarrow{C_1}=2018$.\\
		Do đó $S=\left[\ln\left(3-1\right)+2018-2018\right]\left[\ln\left(1-\left(-1\right)\right)+2017-2017\right]$ $=\ln^22$.
	}
\end{ex}
\begin{ex}
	[Sở Phú Thọ - 2018]%Câu 77
	Cho hàm số $ f(x)$ xác định trên $\mathbb{R}\setminus\left\{-1;1\right\}$ thỏa mãn $f'(x)=\dfrac{2}{x^2-1}$, $ f\left(-2\right)+f(2)=0$ và $ f\left(-\dfrac{1}{2}\right)+f\left(\dfrac{1}{2}\right)=2$. Tính $ f\left(-3\right)+f(0)+f(4)$ được kết quả
	\choice
	{\True $\ln\dfrac{6}{5}+1$}
	{$\ln\dfrac{6}{5}-1$}
	{$\ln\dfrac{4}{5}+1$}
	{$\ln\dfrac{4}{5}-1$}
	\loigiai{
		Ta có $ f(x)=\displaystyle\int{f'(x)\mathrm{d}x}=\displaystyle\int{\dfrac{2}{x^2-1}\mathrm{d}x}=\displaystyle\int{\left(\dfrac{1}{x-1}-\dfrac{1}{x+1}\right)\mathrm{d}x}$ $=\left\{\begin{aligned}
			&\ln\left|\dfrac{x-1}{x+1}\right|+C_1&\text{khi}~x<-1\\ 
			&\ln\left|\dfrac{x-1}{x+1}\right|+C_2&\text{khi}~-1<x<1\\ 
			&\ln\left|\dfrac{x-1}{x+1}\right|+C_3&\text{khi}~x>1\\ 
		\end{aligned}\right.$ .\\
		Khi đó $\left\{\begin{aligned}
			& f\left(-2\right)+f(2)=0\\ 
			& f\left(-\dfrac{1}{2}\right)+f\left(\dfrac{1}{2}\right)=2\\ 
		\end{aligned}\right.\Rightarrow\left\{\begin{aligned}
			&\ln 3+C_1+\ln\dfrac{1}{3}+C_3=0\\ 
			&\ln 3+C_2+\ln\dfrac{1}{3}+C_2=2\\ 
		\end{aligned}\right.\Rightarrow\left\{\begin{aligned}
			&{C_1}+C_3=0\\ 
			&{C_2}=1\\ 
		\end{aligned}\right.$\\
		Do đó $ f\left(-3\right)+f(0)+f(4)=\ln 2+C_1+C_2+\ln\dfrac{3}{5}+C_3=\ln\dfrac{6}{5}+1$.
	}
\end{ex}
\begin{ex}
	[Liên trường Hà Tĩnh – 2022]%Câu 78
	Cho hàm số $f(x)$ xác định trên $\mathbb{R}\backslash\{-1 ; 2\}$ thỏa mãn $f'(x)=\dfrac{1}{x^2-x-2}$; $f(-3)-f(3)=0$ và $f(0)=\dfrac{1}{3}$. Giá trị của biểu thức $f(-4)+f(1)-f(4)$ bằng
	\choice
	{$\dfrac{1}{3}+\dfrac{1}{3}\ln 2$}
	{$1+\ln{S}0$}
	{\True $\dfrac{1}{3}-\ln 2$}
	{$1+\dfrac{1}{3}\ln\dfrac{8}{5}$}
	\loigiai{
		$f(x)=\displaystyle\int{\dfrac{\mathrm{d}x}{x^2+x-2}}=\dfrac{1}{3}\displaystyle\int{\left(\dfrac{1}{x-2}-\dfrac{1}{x+1}\right)}\mathrm{d}x=\dfrac{1}{3}\ln\left|\dfrac{x-2}{x+1}\right|+C$\\
		$\Rightarrow f(x)=\dfrac{1}{3}\ln\left|\dfrac{x-2}{x+1}\right|+C=\left\{\begin{aligned}
			&\dfrac{1}{3}\ln\dfrac{x-2}{x+1}+C_1&\text{khi}&~x>2\\ 
			&\dfrac{1}{3}\ln\left(\dfrac{2-x}{x+1}\right)+C_2&\text{khi}&-1<x<2\\ 
			&\dfrac{1}{3}\ln\dfrac{x-2}{x+1}+C_3&\text{khi}&~x<-1\\ 
		\end{aligned}\right.$\\
		Khi đó $f(-3)-f(-4)=\dfrac{1}{3}\ln\dfrac{5}{4}; f(4)-f(3)=\dfrac{1}{3}\ln\dfrac{8}{5}$\\
		$ f(-3)-f(-4)+f(4)-f(3)=\dfrac{1}{3}\ln 2\Rightarrow f(-4)-f(4)=-\dfrac{1}{3}\ln 2$\\
		Mặt khác $f(1)-f(0)=\dfrac{1}{3}\ln\dfrac{1}{4}\Rightarrow f(1)=\dfrac{1}{3}+\dfrac{1}{3}\ln\dfrac{1}{4}$\\
		Do đó $f(-4)+f(1)-f(4)=\dfrac{1}{3}-\ln 2$.
	}
\end{ex}
\begin{dang}
	{Nguyên hàm từng phần}
\end{dang}
\begin{ex}
	[Mã 101 - 2020 Lần 1]%Câu 79
	Cho hàm số $ f(x)=\dfrac{x}{\sqrt{x^2+2}}$. Họ tất cả các nguyên hàm của hàm số $ g(x)=\left(x+1\right).f'(x)$ là
	\choice
	{$\dfrac{x^2+2x-2}{2\sqrt{x^2+2}}+C$}
	{\True $\dfrac{x-2}{\sqrt{x^2+2}}+C$}
	{$\dfrac{x^2+x+2}{\sqrt{x^2+2}}+C$}
	{$\dfrac{x+2}{2\sqrt{x^2+2}}+C$}
	\loigiai{
		Tính $ g(x)=\displaystyle\int{\left(x+1\right)}{f}'(x)\mathrm{d}x=\left(x+1\right)f(x)-\displaystyle\int{\left(x+1\right)^{\prime}}f(x)\mathrm{d}x=\dfrac{x^2+x}{\sqrt{x^2+2}}-\displaystyle\int{f(x)\mathrm{d}x}$\\
		$=\dfrac{x^2+x}{\sqrt{x^2+2}}-\displaystyle\int{\dfrac{x}{\sqrt{x^2+2}}\mathrm{d}x}=\dfrac{x^2+x}{\sqrt{x^2+2}}-\sqrt{x^2+2}+C=\dfrac{x-2}{\sqrt{x^2+2}}+C$.
	}
\end{ex}
\begin{ex}
	[Mã 102 - 2020 Lần 1]%Câu 80
	Cho hàm số $ f(x)=\dfrac{x}{\sqrt{x^2+3}}$. Họ tất cả các nguyên hàm của hàm số $ g(x)=\left(x+1\right){f}'(x)$ là
	\choice
	{$\dfrac{x^2+2x-3}{2\sqrt{x^2+3}}+C$}
	{$\dfrac{x+3}{2\sqrt{x^2+3}}+C$}
	{$\dfrac{2x^2+x+3}{\sqrt{x^2+3}}+C$}
	{\True $\dfrac{x-3}{\sqrt{x^2+3}}+C$}
	\loigiai{
		Ta có $\displaystyle\int{\left(x+1\right){f}'(x)\mathrm{d}x}=\left(x+1\right)f(x)-\displaystyle\int{\dfrac{x}{\sqrt{x^2+3}}\mathrm{d}x=\dfrac{x-3}{\sqrt{x^2+3}}+C}$.
	}
\end{ex}
\begin{ex}
	[Mã 103 - 2020 Lần 1]%Câu 81
	Cho hàm số $ f(x)=\dfrac{x}{\sqrt{x^2+1}}$. Họ tất cả các nguyên hàm của hàm số $ g(x)=(x+1)f'(x)$
	\choice
	{$\dfrac{x^2+2x-1}{2\sqrt{x^2+1}}+C$}
	{$\dfrac{x+1}{\sqrt{x^2+1}}+C$}
	{$\dfrac{2x^2+x+1}{\sqrt{x^2+1}}+C$}
	{\True $\dfrac{x-1}{\sqrt{x^2+1}}+C$}
	\loigiai{
		Xét $\displaystyle\int{g(x)\mathrm{d}x=\displaystyle\int{(x+1)f'(x)\mathrm{d}x}}$. Đặt $\left\{\begin{aligned}
			&u=x+1\\
			&dv=f'(x)\mathrm{d}x\\
		\end{aligned}\right.\Leftrightarrow\left\{\begin{aligned}
			&\mathrm{d}u=\mathrm{d}x\\
			&v=f(x)\\
		\end{aligned}\right.$\\
		Vậy $\displaystyle\int{g(x)\mathrm{d}x=(x+1)f(x)-\displaystyle\int{f(x)\mathrm{d}x}}\Rightarrow\displaystyle\int{g(x)\mathrm{d}x=\dfrac{(x+1)x}{\sqrt{x^2+1}}-\displaystyle\int{\dfrac{x}{\sqrt{x^2+1}}\mathrm{d}x}}$\\
		$\Rightarrow\displaystyle\int{g(x)\mathrm{d}x=\dfrac{(x+1)x}{\sqrt{x^2+1}}-\sqrt{x^2+1}+C}$ $\Rightarrow\displaystyle\int{g(x)\mathrm{d}x=\dfrac{x^2+x-x^2-1}{\sqrt{x^2+1}}+C}$\\
		$\Rightarrow\displaystyle\int{g(x)\mathrm{d}x=\dfrac{x-1}{\sqrt{x^2+1}}+C}$.
	}
\end{ex}
\begin{ex}
	[Mã 104 - 2020 Lần 1]%Câu 82
	Cho hàm số $ f(x)=\dfrac{x}{\sqrt{x^2+4}}$. Họ tất cả các nguyên hàm của hàm số $ g(x)=\left(x+1\right){f}'(x)$ là
	\choice
	{$\dfrac{x+4}{2\sqrt{x^2+4}}+C$}
	{\True $\dfrac{x-4}{\sqrt{x^2+4}}+C$}
	{$\dfrac{x^2+2x-4}{2\sqrt{x^2+4}}+C$}
	{$\dfrac{2x^2+x+4}{\sqrt{x^2+4}}+C$}
	\loigiai{
		Ta có $ f(x)=\dfrac{x}{\sqrt{x^2+4}}\Rightarrow{f}'(x)=\dfrac{x'.\sqrt{x^2+4}-\left(\sqrt{x^2+4}\right)'.x}{x^2+4}$\\
		$\Rightarrow{f}'(x)=\dfrac{\sqrt{x^2+4}-\dfrac{x}{\sqrt{x^2+4}}.x}{x^2+4}=\dfrac{\dfrac{x^2+4-x^2}{\sqrt{x^2+4}}}{x^2+4}=\dfrac{4}{\left(\sqrt{x^2+4}\right)^3}$\\
		Suy ra $ g(x)=\left(x+1\right){f}'(x)=x.f'(x)+f'(x)$\\
		$\displaystyle\int{g(x)\mathrm{d}x}=\displaystyle\int{\left[x.f'(x)+f'(x)\right]\mathrm{d}x}=\displaystyle\int{x.f'(x)}\mathrm{d}x+\displaystyle\int{f'(x)\mathrm{d}x}$\\
		$=\displaystyle\int{\dfrac{4x}{\left(\sqrt{x^2+4}\right)^3}}\mathrm{d}x+\displaystyle\int{f'(x)\mathrm{d}x}$\\
		Xét $I=\displaystyle\int{\dfrac{4x}{\left(\sqrt{x^2+4}\right)^3}}\mathrm{d}x$\\
		Đặt $t=x^2+4\Rightarrow \mathrm{d}t=2x\mathrm{d}x$\\
		Suy ra $I=\displaystyle\int{\dfrac{2\mathrm{d}t}{\left(\sqrt{t}\right)^3}}=\displaystyle\int{\dfrac{2\mathrm{d}t}{t^{\dfrac{3}{2}}}}=2\displaystyle\int{t^{-\dfrac{3}{2}}\mathrm{d}t}=2\dfrac{t^{-\dfrac{1}{2}}}{-\dfrac{1}{2}}+C_1=\dfrac{-4}{\sqrt{t}}+C_1=\dfrac{-4}{\sqrt{x^2+4}}+C_1$\\
		và $J=\displaystyle\int{f'(x)}\mathrm{d}x=f(x)+C_2$\\
		Vậy $\displaystyle\int{g(x)\mathrm{d}x}=\dfrac{-4}{\sqrt{x^2+4}}+\dfrac{x}{\sqrt{x^2+4}}+C=\dfrac{x-4}{\sqrt{x^2+4}}+C$ .\\
		Cách 2: $ g(x)=\left(x+1\right){f}'(x)$\\
		$\Rightarrow\displaystyle\int{g(x)\mathrm{d}x}=\displaystyle\int{\left(x+1\right){f}'(x)\mathrm{d}x}$\\
		Đặt: $\left\{\begin{aligned}
			& u=x+1\\ 
			& dv=f'(x)\mathrm{d}x\\ 
		\end{aligned}\right.\Rightarrow\left\{\begin{aligned}
			& \mathrm{d}u=\mathrm{d}x\\ 
			& v=f(x)\\ 
		\end{aligned}\right.$\\
		Suy ra $\displaystyle\int{g(x)\mathrm{d}x}=\left(x+1\right)f(x)-\displaystyle\int{f(x)\mathrm{d}x}=\dfrac{\left(x+1\right)x}{\sqrt{x^2+4}}-\displaystyle\int{\dfrac{x}{\sqrt{x^2+4}}\mathrm{d}x}$\\
		$=\dfrac{x^2+x}{\sqrt{x^2+4}}-\displaystyle\int{\dfrac{d\left(x^2+4\right)}{2\sqrt{x^2+4}}}$ $=\dfrac{x^2+x}{\sqrt{x^2+4}}-\sqrt{x^2+4}+C$ $=\dfrac{x-4}{\sqrt{x^2+4}}+C$.
	}
\end{ex}
\begin{ex}
	[Đề Minh Họa 2020 Lần 1]%Câu 83
	Cho hàm số $ f(x)$ liên tục trên $\mathbb{R}$. Biết $\cos 2x$ là một nguyên hàm của hàm số $f(x){\mathrm{e}^x}$, họ tất cả các nguyên hàm của hàm số $f'(x){\mathrm{e}^x}$ là:
	\choice
	{$-\sin 2x+\cos 2x+C$}
	{$-2\sin 2x+\cos 2x+C$}
	{\True $-2\sin 2x-\cos 2x+C$}
	{$2\sin 2x-\cos 2x+C$}
	\loigiai{
		Do $\cos 2x$ là một nguyên hàm của hàm số $f(x){\mathrm{e}^x}$\\
		nên $f(x){\mathrm{e}^x}=\left(\cos 2x\right)'\Leftrightarrow f(x){\mathrm{e}^x}=-2\sin 2x$.\\
		Khi đó ta có $\displaystyle\int{f(x){\mathrm{e}^x}\mathrm{d}}x=\cos 2x+C$.\\
		Đặt $\left\{\begin{aligned}
			& u=f(x)\\ 
			&\mathrm{d}v=\mathrm{e}^x\mathrm{d}x\\ 
		\end{aligned}\right.\Rightarrow\left\{\begin{aligned}
			&\mathrm{d}u=f'(x)\mathrm{d}x\\ 
			& v=\mathrm{e}^x\\ 
		\end{aligned}\right.$. \\
		Khi đó $\displaystyle\int{f(x){\mathrm{e}^x}\mathrm{d}}x=\cos 2x+C$ $\Leftrightarrow\displaystyle\int{f(x)\mathrm{d}\left(\mathrm{e}^x\right)}=\cos 2x+C$ $\Leftrightarrow f(x){\mathrm{e}^x}-\displaystyle\int{f'(x){\mathrm{e}^x}\mathrm{d}}x=\cos 2x+C$ $\Leftrightarrow\displaystyle\int{f'(x){\mathrm{e}^x}\mathrm{d}}x=-2\sin 2x-\cos 2x+C$.\\
		Vậy tất cả các nguyên hàm của hàm số $f'(x){\mathrm{e}^x}$ là $-2\sin 2x-\cos 2x+C$.
	}
\end{ex}
\begin{ex}
	[Đề Tham Khảo 2019]%Câu 84
	Họ nguyên hàm của hàm số $ f(x)=4x\left(1+\ln x\right)$ là:
	\choice
	{$ 2x^2\ln x+3x^2$}
	{$ 2x^2\ln x+x^2$}
	{$ 2x^2\ln x+3x^2+C$}
	{\True $ 2x^2\ln x+x^2+C$}
	\loigiai{
		Ta có $ f(x)=4x\left(1+\ln x\right)\Rightarrow F(x)=\displaystyle\int{\left(4x\left(1+\ln x\right)\right)\mathrm{d}x}$\\
		đặt $\left\{\begin{aligned}
			& u=1+\ln x\Rightarrow \mathrm{d}u=\dfrac{1}{x}\\ 
			& dv=4x\Rightarrow v=2x^2\\ 
		\end{aligned}\right.$\\ 
		$\Rightarrow F(x)=2x^2\left(1+\ln x\right)-\displaystyle\int{2x\mathrm{d}x=2x^2\left(1+\ln x\right)-x^2+C=2x^2\ln x+x^2+C}$.
	}
\end{ex}
\begin{ex}
	Họ các nguyên hàm của hàm số $ f(x)=x\sin x$ là
	\choice
	{$ F(x)=x\cos x+\sin x+C$}
	{$ F(x)=x\cos x-\sin x+C$}
	{$ F(x)=-x\cos x-\sin x+C$}
	{\True $ F(x)=-x\cos x+\sin x+C$}
	\loigiai{
		Đặt $\left\{\begin{aligned}
			& u=x\\ 
			&\mathrm{d}v=\sin x{\mathrm{d}x}\\ 
		\end{aligned}\right.\Rightarrow\left\{\begin{aligned}
			&{\mathrm{d}u}={\mathrm{d}x}\\ 
			& v=-\cos x\\ 
		\end{aligned}\right.$.\\
		Suy ra $\displaystyle\int{x\sin x{\mathrm{d}x}}=-x\cos x+\displaystyle\int{\cos x{\mathrm{d}x}=-x\cos x+\sin x+C}$.
	}
\end{ex}
\begin{ex}
	[Chuyên Phan Bội Châu 2019]%Câu 86
	Họ nguyên hàm của hàm số $ f(x)=x.\mathrm{e}^{2x}$ là
	\choice
	{\True $ F(x)=\dfrac{1}{2}{\mathrm{e}^{2x}}\left(x-\dfrac{1}{2}\right)+C$}
	{$ F(x)=\dfrac{1}{2}{\mathrm{e}^{2x}}\left(x-2\right)+C$}
	{$ F(x)=2\mathrm{e}^{2x}\left(x-2\right)+C$}
	{$ F(x)=2\mathrm{e}^{2x}\left(x-\dfrac{1}{2}\right)+C$}
	\loigiai{
		Đặt $\left\{\begin{aligned}
			& u=x\\ 
			& \mathrm{d}v=\mathrm{e}^{2x}\\ 
		\end{aligned}\right.\Rightarrow\left\{\begin{aligned}
			& \mathrm{d}u=\mathrm{d}x\\ 
			& v=\dfrac{1}{2}{e^{2x}}\\ 
		\end{aligned}\right.$\\
		$\Rightarrow\displaystyle\int{x.\mathrm{e}^{2x}\mathrm{d}x}=\dfrac{1}{2}x.e^{2x}-\dfrac{1}{2}\displaystyle\int{\mathrm{e}^{2x}\mathrm{d}x}$\\
		$\Rightarrow\displaystyle\int{x.\mathrm{e}^{2x}\mathrm{d}x}=\dfrac{1}{2}x.\mathrm{e}^{2x}-\dfrac{1}{4}{\mathrm{e}^{2x}}+C=\dfrac{1}{2}{\mathrm{e}^{2x}}\left(x-\dfrac{1}{2}\right)+C$.
	}
\end{ex}
\begin{ex}
	[THPT Gia Lộc Hải Dương 2019]%Câu 87
	Họ nguyên hàm của hàm số $f(x)=\left(2x-1\right){\mathrm{e}^x}$ là
	\choice
	{\True $\left(2x-3\right)\mathrm{e}^x+C$}
	{$\left(2x+3\right)\mathrm{e}^x+C$}
	{$\left(2x+1\right)\mathrm{e}^x+C$}
	{$\left(2x-1\right)\mathrm{e}^x+C$}
	\loigiai{
		Gọi $ I=\displaystyle\int{\left(2x-1\right){\mathrm{e}^x}\mathrm{d}x}$.\\
		Đặt $\left\{\begin{aligned}
			& u=2x-1\\ 
			&\mathrm{d}v=\mathrm{e}^x\mathrm{d}x\\ 
		\end{aligned}\right.\Rightarrow\left\{\begin{aligned}
			&{\mathrm{d}u}=2\mathrm{d}x\\ 
			& v=\mathrm{e}^x\\ 
		\end{aligned}\right.$.\\
		$\Rightarrow I=\left(2x-1\right)\mathrm{e}^x-2\displaystyle\int\mathrm{e}^x\mathrm{d}x=\left(2x-1\right)\mathrm{e}^x-2\mathrm{e}^x+C=\left(2x-3\right)\mathrm{e}^x+C$.
	}
\end{ex}
\begin{ex}
	[Chuyen Phan Bội Châu Nghệ An 2019]%Câu 88
	Tìm họ nguyên hàm của hàm số $f(x)=x{e^{2x}}$?
	\choice
	{\True $F(x)=\dfrac{1}{2}{\mathrm{e}^{2x}}\left(x-\dfrac{1}{2}\right)+C$}
	{$F(x)=\dfrac{1}{2}{\mathrm{e}^{2x}}\left(x-2\right)+C$}
	{$F(x)=2\mathrm{e}^{2x}\left(x-2\right)+C$}
	{$F(x)=2\mathrm{e}^{2x}\left(x-\dfrac{1}{2}\right)+C$}
	\loigiai{
		Ta có $F(x)=\displaystyle\int{x{\mathrm{e}^{2x}}\mathrm{d}x}$\\
		Đặt $\left\{\begin{aligned}
			& u=x\\ 
			& dv=e^{2x}\mathrm{d}x\\ 
		\end{aligned}\right.\Rightarrow\left\{\begin{aligned}
			& \mathrm{d}u=\mathrm{d}x\\ 
			& v=\dfrac{1}{2}{\mathrm{e}^{2x}}\\ 
		\end{aligned}\right.$\\
		Suy ra $F(x)=\dfrac{1}{2}x{\mathrm{e}^{2x}}-\dfrac{1}{2}\displaystyle\int{\mathrm{e}^{2x}}\mathrm{d}x$ $=\dfrac{1}{2}x{\mathrm{e}^{2x}}-\dfrac{1}{4}{\mathrm{e}^{2x}}+C=\dfrac{1}{2}{\mathrm{e}^{2x}}\left(x-\dfrac{1}{2}\right)+C$.
	}
\end{ex}
\begin{ex}
	[Chuyên Sơn La 2019]%Câu 89
	Họ nguyên hàm của hàm số $ f(x)=x\left(1+\sin x\right)$ là
	\choice
	{$\dfrac{x^2}{2}-x\sin x+\cos x+C$}
	{\True $\dfrac{x^2}{2}-x\cos x+\sin x+C$}
	{$\dfrac{x^2}{2}-x\cos x-\sin x+C$}
	{$\dfrac{x^2}{2}-x\sin x-\cos x+C$}
	\loigiai{
		Ta có $\displaystyle\int{f(x)\mathrm{d}x}=\displaystyle\int{x\left(1+\sin x\right)\mathrm{d}x=\displaystyle\int{x}}\mathrm{d}x+\displaystyle\int{x.\sin x\mathrm{d}x}=\displaystyle\int{x\mathrm{d}x}-\displaystyle\int{x\mathrm{d}\left(\cos x\right)}$\\
		$=\dfrac{x^2}{2}-\left(x\cos x-\displaystyle\int{\cos x\mathrm{d}x}\right)\text{=}\dfrac{x^2}{2}-x\cos x+\sin x+C$.
	}
\end{ex}
\begin{ex}
	[Chuyên Thái Bình - Lần 3 - 2020]%Câu 90
	Giả sử $ F(x)=\left(a{x^2}+bx+c\right){\mathrm{e}^x}$ là một nguyên hàm của hàm số $ f(x)=x^2\mathrm{e}^x$.Tính tích $ P=abc$.
	\choice
	{\True $-4$}
	{$ 1$}
	{$-5$}
	{$-3$}
	\loigiai{
		Ta đặt $\left\{\begin{aligned}
			& u=x^2\\ 
			& \mathrm{d}v=\mathrm{e}^x\mathrm{d}x\\ 
		\end{aligned}\right.\Rightarrow\left\{\begin{aligned}
			& \mathrm{d}u=2x\mathrm{d}x\\ 
			& v=\mathrm{e}^x\\ 
		\end{aligned}\right.\Rightarrow\displaystyle\int{x^2\mathrm{e}^x\mathrm{d}x}=x^2\mathrm{e}^x-2\displaystyle\int{x{\mathrm{e}^x}\mathrm{d}x}$.\\
		Ta đặt $\left\{\begin{aligned}
			& u=x\\ 
			& \mathrm{d}v=\mathrm{e}^x\mathrm{d}x\\ 
		\end{aligned}\right.\Rightarrow\left\{\begin{aligned}
			& \mathrm{d}u=\mathrm{d}x\\ 
			& v=\mathrm{e}^x\\ 
		\end{aligned}\right.\Rightarrow\displaystyle\int{x^2\mathrm{e}^x\mathrm{d}x}=x^2\mathrm{e}^x-2\left(x{\mathrm{e}^x}-\displaystyle\int{\mathrm{e}^x\mathrm{d}x}\right)=\left(x^2-2x+2\right){\mathrm{e}^x}$.\\
		Vậy $ a=1,b=-2,c=2\Rightarrow P=abc=-4$.
	}
\end{ex}
\begin{ex}
	Họ nguyên hàm của hàm số $ f(x)=2x(1+\mathrm{e}^x)$là
	\choice
	{$\left(2x-1\right){\mathrm{e}^x}+x^2$}
	{$\left(2x+1\right){\mathrm{e}^x}+x^2$}
	{$\left(2x+2\right){\mathrm{e}^x}+x^2$}
	{\True $\left(2x-2\right){\mathrm{e}^x}+x^2$}
	\loigiai{
		Ta có $\displaystyle\int{2x(1+\mathrm{e}^x)\mathrm{d}x=2\displaystyle\int{x\mathrm{d}x+2\displaystyle\int{x{\mathrm{e}^x}\mathrm{d}x}}}$.\\
		Gọi $ I=2\displaystyle\int{x\ln x\mathrm{d}x}$. Đặt $\left\{\begin{aligned}
			&u=x\\
			&\mathrm{d}v=\mathrm{e}^x\mathrm{d}x\\
		\end{aligned}\Rightarrow\left\{\begin{aligned}
			&\mathrm{d}u=\mathrm{d}x\\
			&v=\mathrm{e}^x\\
		\end{aligned}\right.\right.$.\\
		Khi đó $ I=2x{\mathrm{e}^x}-2\displaystyle\int{\mathrm{e}^x\mathrm{d}x}$.\\
		Vậy $\displaystyle\int{2x(1+\mathrm{e}^x)\mathrm{d}x=2\displaystyle\int{x\mathrm{d}x+x{\mathrm{e}^x}-2\displaystyle\int{\mathrm{e}^x\mathrm{d}x}}}=x^2+x{\mathrm{e}^x}-2x+C$\\
		$=\left(2x-2\right){\mathrm{e}^x}+x^2+C$.
	}
\end{ex}
\begin{ex}
	Họ nguyên hàm của $f(x)=x\ln x$ là kết quả nào sau đây?
	\choice
	{$F(x)=\dfrac{1}{2}{x^2}\ln x+\dfrac{1}{2}{x^2}+C$}
	{$F(x)=\dfrac{1}{2}{x^2}\ln x+\dfrac{1}{4}{x^2}+C$}
	{\True $F(x)=\dfrac{1}{2}{x^2}\ln x-\dfrac{1}{4}{x^2}+C$}
	{$F(x)=\dfrac{1}{2}{x^2}\ln x+\dfrac{1}{4}x+C$}
	\loigiai{
		Ta có $F(x)=\displaystyle\int{f(x)\mathrm{d}x}=\displaystyle\int{x\ln x\mathrm{d}x}$. Đặt $\left\{\begin{aligned}
			& u=\ln x\\ 
			& dv=x\mathrm{d}x\\ 
		\end{aligned}\right.\Rightarrow\left\{\begin{aligned}
			& \mathrm{d}u=\dfrac{\mathrm{d}x}{x}\\ 
			& v=\dfrac{x^2}{2}\\ 
		\end{aligned}\right.$.\\
		Theo công thức tính nguyên hàm từng phần, ta có\\
		$F(x)=\dfrac{1}{2}{x^2}\ln x-\dfrac{1}{2}\displaystyle\int{x\mathrm{d}x}=\dfrac{1}{2}{x^2}\ln x-\dfrac{1}{4}{x^2}+C$.
	}
\end{ex}
\begin{ex}
	[Chuyên Lê Hồng Phong Nam Định 2019]%Câu 93
	Tìm tất cả các nguyên hàm của hàm số $ f(x)=\left(3x^2+1\right).\ln x$.
	\choice
	{$\displaystyle\int{f(x)\mathrm{d}x=x\left(x^2+1\right)\ln x-\dfrac{x^3}{3}+C}$}
	{$\displaystyle\int{f(x)\mathrm{d}x=x^3\ln x-\dfrac{x^3}{3}+C}$}
	{\True $\displaystyle\int{f(x)\mathrm{d}x=x\left(x^2+1\right)\ln x-\dfrac{x^3}{3}-x+C}$}
	{$\displaystyle\int{f(x)\mathrm{d}x=x^3\ln x-\dfrac{x^3}{3}-x+C}$}
	\loigiai{
		Ta có $ I=\displaystyle\int{\left(3x^2+1\right)\ln x\mathrm{d}x}$\\
		Đặt $\left\{\begin{aligned}
			& u=\ln x\\ 
			& dv=\left(3x^2+1\right)\mathrm{d}x\\ 
		\end{aligned}\right.\Rightarrow\left\{\begin{aligned}
			& \mathrm{d}u=\dfrac{1}{x}\mathrm{d}x\\ 
			&v=\displaystyle\int{\left(3x^2+1\right)\mathrm{d}x=x^3+x}\\ 
		\end{aligned}\right.$.\\
		$\Rightarrow I=\left(x^3+x\right)\ln x-\displaystyle\int{\left(x^3+x\right)\dfrac{1}{x}\mathrm{d}x=x\left(x^2+1\right)\ln x-\displaystyle\int{\left(x^2+1\right)\mathrm{d}x=}}x\left(x^2+1\right)\ln x-\dfrac{x^3}{3}-x+C$.
	}
\end{ex}
\begin{ex}
	[Chuyên Đại Học Vinh 2019]%Câu 94
	Tất cả các nguyên hàm của hàm số $ f(x)=\dfrac{x}{\sin^2x}$ trên khoảng $\left(0;\pi\right)$ là
	\choice
	{\True $-x\cot x+\ln\left(\sin x\right)+C$}
	{$ x\cot x-\ln\left|\sin x\right|+C$}
	{$ x\cot x+\ln\left|\sin x\right|+C$}
	{$-x\cot x-\ln\left(\sin x\right)+C$}
	\loigiai{
		$F(x)=\displaystyle\int f(x) \mathrm{d}x=\displaystyle\int \dfrac{x}{\sin ^2 x}\mathrm{d}x$.\\
		Đặt $\left\{\begin{aligned}
			&u=x \\ 
			&\mathrm{d}v=\dfrac{1}{\sin ^2 x}\mathrm{~d}x
		\end{aligned}\Rightarrow\left\{\begin{aligned}
			&\mathrm{d}u=\mathrm{d}x \\ 
			&v=-\cot x\\
		\end{aligned}\right.\right.$.\\
		Khi đó $F(x)=\displaystyle\int \dfrac{x}{\sin ^2 x}\mathrm{d}x=-x \cdot \cot x+\int \cot x \mathrm{d}x=-x \cdot \cot x+\displaystyle\int \dfrac{\cos x}{\sin x}\mathrm{d}x=-x \cdot \cot x+\int \dfrac{\mathrm{d}(\sin x)}{\sin x}=-x \cdot \cot x+\ln |\sin x|+C$.\\ 
		Với $x \in(0 ; \pi) \Rightarrow \sin x>0 \Rightarrow \ln |\sin x|=\ln (\sin x)$.\\
		Vậy $F(x)=-x \cot x+\ln (\sin x)+C$.
	}
\end{ex}
\begin{ex}
	[Sở Phú Thọ 2019]%Câu 95
	Họ nguyên hàm của hàm số $y=3x\left(x+\cos x\right)$ là
	\choice
	{\True $x^3+3\left(x\sin x+\cos x\right)+C$}
	{$x^3-3\left(x\sin x+\cos x\right)+C$}
	{$x^3+3\left(x\sin x-\cos x\right)+C$}
	{$x^3-3\left(x\sin x-\cos x\right)+C$}
	\loigiai{
		Ta có $\displaystyle\int{3x\left(x+\cos x\right)\mathrm{d}x}=\displaystyle\int{3x^2}\mathrm{d}x+\displaystyle\int{3x\cos x\mathrm{d}x}$\\
		$\displaystyle\int{3x^2}\mathrm{d}x=x^3+C_1$\\
		$\displaystyle\int{3x\cos x\mathrm{d}x}=\displaystyle\int{3x.\mathrm{d}\left(\sin x\right)}=3x.\sin x-\displaystyle\int{3\sin x\mathrm{d}x}=3x.\sin x+3\cos x+C_2$\\
		Vậy $\displaystyle\int{3x\left(x+\cos x\right)\mathrm{d}x}=x^3+3\left(x\sin x+\cos x\right)+C$.
	}
\end{ex}
\begin{ex}
	[Chuyên Lê Hồng Phong Nam Định 2019]%Câu 96
	Họ nguyên hàm của hàm số $f(x)=x^4+x{\mathrm{e}^x}$ là
	\choice
	{$\dfrac{1}{5}{x^5}+\left(x+1\right){\mathrm{e}^x}+C$}
	{\True $\dfrac{1}{5}{x^5}+\left(x-1\right){\mathrm{e}^x}+C$}
	{$\dfrac{1}{5}{x^5}+x{\mathrm{e}^x}+C$}
	{$4x^3+\left(x+1\right){\mathrm{e}^x}+C$}
	\loigiai{
		Ta có: $\displaystyle\int{\left(x^4+x{\mathrm{e}^x}\right){\mathrm{d}x}=\displaystyle\int{x^4{\mathrm{d}x}+\displaystyle\int{x{\mathrm{e}^x}{\mathrm{d}x}}}}$ .\\
		$\displaystyle\int{x^4{\mathrm{d}x=}\dfrac{1}{5}}{x^5}+C_1$.\\
		Đặt $\left\{\begin{aligned}
			& u=x\\ 
			&\mathrm{d}v=\mathrm{e}^x{\mathrm{d}x}\\ 
		\end{aligned}\right.\Rightarrow\left\{\begin{aligned}
			&{\mathrm{d}u}={\mathrm{d}x}\\ 
			& v=\mathrm{e}^x\\ 
		\end{aligned}\right.$.\\
		Suy ra $\displaystyle\int{x{\mathrm{e}^x}{\mathrm{d}x}}$ $=x{\mathrm{e}^x}-\displaystyle\int{\mathrm{e}^x{\mathrm{d}x}}=x{\mathrm{e}^x}-\mathrm{e}^x+C_2$ $=\left(x-1\right){\mathrm{e}^x}+C_2$ .\\
		Vậy $\displaystyle\int{\left(x^4+x{\mathrm{e}^x}\right){\mathrm{d}x}=\dfrac{1}{5}{x^5}+\left(x-1\right){\mathrm{e}^x}+C}$.
	}
\end{ex}
\begin{ex}
	Cho hai hàm số $ F(x),G(x)$ xác định và có đạo hàm lần lượt là $ f(x),g(x)$ trên $\mathbb{R}$. Biết rằng $ F(x).G(x)=x^2\ln\left(x^2+1\right)$ và $ F(x).g(x)=\dfrac{2x^3}{x^2+1}.$ Họ nguyên hàm của $ f(x).G(x)$ là
	\choice
	{$\left(x^2+1\right)\ln\left(x^2+1\right)+2x^2+C$}
	{$\left(x^2+1\right)\ln\left(x^2+1\right)-2x^2+C$}
	{\True $\left(x^2+1\right)\ln\left(x^2+1\right)-x^2+C$}
	{$\left(x^2+1\right)\ln\left(x^2+1\right)+x^2+C$}
	\loigiai{
		$F(x).G(x)=\displaystyle\int{\left(F(x).G(x)\right)'\mathrm{d}x=}\displaystyle\int{\left(F'(x).G(x)+F(x).G'(x)\right)\mathrm{d}x}$.\\
		$\Rightarrow\displaystyle\int{\left(F'(x).G(x)\right)\mathrm{d}x}=F(x).G(x)-\displaystyle\int{\left(F(x).G'(x)\right)\mathrm{d}x}$\\
		$=x^2\ln\left(x^2+1\right)-\displaystyle\int{\left(\dfrac{2x^3}{x^2+1}\right)\mathrm{d}x}=x^2\ln\left(x^2+1\right)-\left(x^2+1\right)+\ln\left(x^2+1\right)+C$\\
		$=\left(x^2+1\right)\ln\left(x^2+1\right)-x^2+C$.
	}
\end{ex}
\begin{ex}%Câu 98
	Họ nguyên hàm của hàm số $ f(x)=x.\mathrm{e}^{2x}$ là
	\choice
	{\True $ F(x)=\dfrac{1}{2}{\mathrm{e}^{2x}}\left(x-\dfrac{1}{2}\right)+C$}
	{$ F(x)=\dfrac{1}{2}{\mathrm{e}^{2x}}\left(x-2\right)+C$}
	{$ F(x)=2\mathrm{e}^{2x}\left(x-2\right)+C$}
	{$ F(x)=2\mathrm{e}^{2x}\left(x-\dfrac{1}{2}\right)+C$}
	\loigiai{
		Đặt $\left\{\begin{aligned}
			&u=x\\
			&\mathrm{d}v=\mathrm{e}^{2x}\mathrm{d}x\\
		\end{aligned}\right.\Rightarrow\left\{\begin{aligned}
			&\mathrm{d}u=\mathrm{d}x\\
			&v=\dfrac{1}{2}{\mathrm{e}^{2x}}\\
		\end{aligned}\right.$.\\
		$F(x)=x.\mathrm{e}^{2x}-\dfrac{1}{2}\displaystyle\int{\mathrm{e}^{2x}}\mathrm{d}x=\dfrac{1}{2}x.\mathrm{e}^{2x}-\dfrac{1}{4}{\mathrm{e}^{2x}}+C=\dfrac{1}{2}{\mathrm{e}^{2x}}\left(x-\dfrac{1}{2}\right)+C$.
	}
\end{ex}
\begin{ex}
	[Sở Bắc Ninh 2019]%Câu 99
	Mệnh đề nào sau đây là đúng?
	\choice
	{$\displaystyle\int{x{\mathrm{e}^x}}\mathrm{d}x=\mathrm{e}^x+x{\mathrm{e}^x}+C$}
	{$\displaystyle\int{x{\mathrm{e}^x}}\mathrm{d}x=\dfrac{x^2}{2}{\mathrm{e}^x}+\mathrm{e}^x+C$}
	{\True $\displaystyle\int{x{\mathrm{e}^x}\mathrm{d}x}=x{\mathrm{e}^x}-\mathrm{e}^x+C$}
	{$\displaystyle\int{x{\mathrm{e}^x}}\mathrm{d}x=\dfrac{x^2}{2}{\mathrm{e}^x}+C$}
	\loigiai{
		Sử dụng công thức: $\displaystyle\int{u\mathrm{d}v=u.v-\displaystyle\int{v\mathrm{d}u}}$.\\
		Ta có $\displaystyle\int{x{\mathrm{e}^x}\mathrm{d}x}=\displaystyle\int{x\mathrm{d}\left(\mathrm{e}^x\right)=x{\mathrm{e}^x}-\displaystyle\int{\mathrm{e}^x\mathrm{d}x}}=x{\mathrm{e}^x}-\mathrm{e}^x+C$.
	}
\end{ex}
\begin{ex}
	[Sở Bắc Giang 2019]%Câu 100
	Cho hai hàm số $ F(x)$, $ G(x)$ xác đinh và có đạo hàm lần lượt là $ f(x)$, $ g(x)$ trên $\mathbb{R}$. Biết $F(x).G(x)=x^2\ln\left(x^2+1\right)$ và $F(x)g(x)=\dfrac{2x^3}{x^2+1}$. Tìm họ nguyên hàm của $ f(x)G(x)$.
	\choice
	{$\left(x^2+1\right)\ln\left(x^2+1\right)+2x^2+C$}
	{$\left(x^2+1\right)\ln\left(x^2+1\right)-2x^2+C$}
	{\True $\left(x^2+1\right)\ln\left(x^2+1\right)-x^2+C$}
	{$\left(x^2+1\right)\ln\left(x^2+1\right)+x^2+C$}
	\loigiai{
		Ta có\\
		$\displaystyle\int{f(x)G(x)\mathrm{d}x}=\displaystyle\int{G(x)\mathrm{d}\left(F(x)\right)}$\\
		$=G(x).F(x)-\displaystyle\int{F(x)\mathrm{d}\left(G(x)\right)}=G(x).F(x)-\displaystyle\int{F(x){g}(x)\mathrm{d}x}$.\\
		$\Leftrightarrow\displaystyle\int{f(x)G(x)\mathrm{d}x}=x^2\ln\left(x^2+1\right)-\displaystyle\int{\dfrac{2x^3}{x^2+1}\mathrm{d}x=}$\\ $x^2\ln\left(x^2+1\right)-\displaystyle\int{\left(2x-\dfrac{2x}{x^2+1}\right)\mathrm{d}x=}x^2\ln\left(x^2+1\right)-x^2+\displaystyle\int{\dfrac{1}{x^2+1}\mathrm{d}\left(x^2+1\right)}=x^2\ln\left(x^2+1\right)-x^2+\ln\left(x^2+1\right)+C=\left(x^2+1\right)\ln\left(x^2+1\right)-x^2+C$.
	}
\end{ex}
\begin{ex}
	Cho biết $ F(x)=\dfrac{1}{3}{x^3}+2x-\dfrac{1}{x}$ là một nguyên hàm của $ f(x)=\dfrac{\left(x^2+a\right)^2}{x^2}$. Tìm nguyên hàm của $ g(x)=x\cos ax$.
	\choice
	{$ x\sin x-\cos x+C$}
	{$\dfrac{1}{2}x\sin 2x-\dfrac{1}{4}\cos 2x+C$}
	{\True $ x\sin x+\cos+C$}
	{$\dfrac{1}{2}x\sin 2x+\dfrac{1}{4}\cos 2x+C$}
	\loigiai{
		Ta có $F'(x)=x^2+2+\dfrac{1}{x^2}=\dfrac{\left(x^2+1\right)^2}{x^2}$.\\
		Do $ F(x)$ là một nguyên hàm của $ f(x)=\dfrac{\left(x^2+a\right)^2}{x^2}$ nên $ a=1$.\\
		$\displaystyle\int{g(x)}\mathrm{d}x=\displaystyle\int{x\cos x}\mathrm{d}x$\\
		Đặt $\left\{\begin{aligned}
			& u=x\\ 
			&\mathrm{d}v=\cos x\mathrm{d}x\\ 
		\end{aligned}\right.\Rightarrow\left\{\begin{aligned}
			&\mathrm{d}u=\mathrm{d}x\\ 
			& v=\sin x\\ 
		\end{aligned}\right.$\\
		$\displaystyle\int{g(x)}\mathrm{d}x=\displaystyle\int{x\cos x}\mathrm{d}x=x\sin x-\displaystyle\int{\sin x}\mathrm{d}x=x\sin x+\cos x+C$.
	}
\end{ex}
\begin{ex}
	Họ nguyên hàm của hàm số $y=\dfrac{\left(2x^2+x\right)\ln x+1}{x}$ là
	\choice
	{$\left(x^2+x+1\right)\ln x-\dfrac{x^2}{2}+x+C$}
	{$\left(x^2+x-1\right)\ln x+\dfrac{x^2}{2}-x+C$}
	{\True $\left(x^2+x+1\right)\ln x-\dfrac{x^2}{2}-x+C$}
	{$\left(x^2+x-1\right)\ln x-\dfrac{x^2}{2}+x+C$}
	\loigiai{
		Ta có $\displaystyle\int{\dfrac{\left(2x^2+x\right)\ln x+1}{x}\mathrm{d}x=\displaystyle\int{\left(2x+1\right)\ln x\mathrm{d}x+\displaystyle\int{\dfrac{1}{x}\mathrm{d}x=I_1+I_2}}}$.\\
		$I_1=\displaystyle\int{\left(2x+1\right)\ln x\mathrm{d}x}$. Đặt $\left\{\begin{aligned}
			& u=\ln x\\ 
			&\mathrm{d}v=\left(2x+1\right)\mathrm{d}x\\ 
		\end{aligned}\right.\Rightarrow\left\{\begin{aligned}
			&\mathrm{d}u=\dfrac{1}{x}\mathrm{d}x\\ 
			& v=x^2+x\\ 
		\end{aligned}\right.$.\\
		$\begin{aligned}
			&{I_1}=\left(x^2+x\right)\ln x-\displaystyle\int{\left(x^2+x\right)\dfrac{1}{x}\mathrm{d}x=}\left(x^2+x\right)\ln x-\displaystyle\int{\left(x+1\right)\mathrm{d}x}\\ 
			&=\left(x^2+x\right)\ln x-\dfrac{x^2}{2}-x+C_1.\\ 
		\end{aligned}$\\
		$I_2=\displaystyle\int{\dfrac{1}{x}\mathrm{d}x=\ln x+C_2}$ .\\
		$\begin{aligned}
			&\displaystyle\int{\dfrac{\left(2x^2+x\right)\ln x+1}{x}\mathrm{d}x=I_1+I_2}\\ 
			&=\left(x^2+x\right)\ln x-\dfrac{x^2}{2}-x+C_1+\ln x+C_2=\left(x^2+x+1\right)\ln x-\dfrac{x^2}{2}-x+C.\\ 
		\end{aligned}$
	}
\end{ex}
\begin{ex}
	[Mã 104 2017]%Câu 103
	Cho $F(x)=\dfrac{1}{2x^2}$ là một nguyên hàm của hàm số $\dfrac{f(x)}{x}$. Tìm nguyên hàm của hàm số $f'(x)\ln x$ .
	\choice
	{$\displaystyle\int{f'(x)\ln x\mathrm{d}x=-\left(\dfrac{\ln x}{x^2}+\dfrac{1}{x^2}\right)}+C$}
	{$\displaystyle\int{f'(x)\ln x\mathrm{d}x=\dfrac{\ln x}{x^2}+\dfrac{1}{2x^2}}+C$}
	{\True $\displaystyle\int{f'(x)\ln x\mathrm{d}x=-\left(\dfrac{\ln x}{x^2}+\dfrac{1}{2x^2}\right)}+C$}
	{$\displaystyle\int{f'(x)\ln x\mathrm{d}x=\dfrac{\ln x}{x^2}+\dfrac{1}{x^2}}+C$}
	\loigiai{
		Ta có $\displaystyle\int{\dfrac{f(x)}{x}\mathrm{d}x=\dfrac{1}{2x^2}}$. Chọn $f(x)=\dfrac{-1}{x^2}$.\\
		Suy ra $\displaystyle\int{f'(x)\ln x\mathrm{d}x=\displaystyle\int{\dfrac{2}{x^3}\ln x\mathrm{d}x}}$. Đặt $\left\{\begin{aligned}
			& u=\ln x\\ 
			&\mathrm{d}v=\dfrac{2}{x^3}\mathrm{d}x\\ 
		\end{aligned}\right.\Rightarrow\left\{\begin{aligned}
			&\mathrm{d}u=\dfrac{\mathrm{d}x}{x}\\ 
			& v=\dfrac{-1}{x^2}\\ 
		\end{aligned}\right.$.\\
		Khi đó: $\displaystyle\int{f'(x)\ln x\mathrm{d}x=\displaystyle\int{\dfrac{\ln x}{x^3}\mathrm{d}x}=-\dfrac{\ln x}{x^2}+\displaystyle\int{\dfrac{1}{x^3}\mathrm{d}x=}-\left(\dfrac{\ln x}{x^2}+\dfrac{1}{2x^2}\right)}+C$.
	}
\end{ex}
\begin{ex}
	[Mã 105 2017]%Câu 104
	Cho $ F(x)=-\dfrac{1}{3x^3}$ là một nguyên hàm của hàm số $\dfrac{f(x)}{x}$. Tìm nguyên hàm của hàm số $f'(x)\ln x$
	\choice
	{$\displaystyle\int{f'(x)\ln x}\mathrm{d}x=\dfrac{\ln x}{x^3}+\dfrac{1}{5x^5}+C$}
	{$\displaystyle\int{f'(x)\ln x}\mathrm{d}x=\dfrac{\ln x}{x^3}-\dfrac{1}{5x^5}+C$}
	{\True $\displaystyle\int{f'(x)\ln x}\mathrm{d}x=-\dfrac{\ln x}{x^3}+\dfrac{1}{3x^3}+C$}
	{$\displaystyle\int{f'(x)\ln x}\mathrm{d}x=\dfrac{\ln x}{x^3}+\dfrac{1}{3x^3}+C$}
	\loigiai{
		Ta có $F'(x)=\dfrac{f(x)}{x}\Rightarrow f(x)=x.F'(x)=x.\left(-\dfrac{1}{3}.x^{-3}\right)'=\dfrac{1}{x^3}=x^{-3}$\\
		$\Rightarrow{f}'(x)=-3x^{-4}\Rightarrow{f}'(x)\ln x=-3x^{-4}\ln x$\\
		Vậy $\displaystyle\int{f'(x)\ln x}\mathrm{d}x=\displaystyle\int{\left(-3x^{-4}\ln x\right)}\mathrm{d}x=-3\displaystyle\int{\ln x.x^{-4}}\mathrm{d}x$\\
		Đặt $ u=\ln x;dv=x^{-4}\mathrm{d}x\Rightarrow\mathrm{d}u=\dfrac{\mathrm{d}x}{x};v=\dfrac{x^{-3}}{-3}$\\
		Nên $\displaystyle\int{f'(x)\ln x}\mathrm{d}x=-3\displaystyle\int{\ln x.x^{-4}}\mathrm{d}x=-3\left(\dfrac{\ln x}{-3x^3}+\displaystyle\int{\dfrac{x^{-4}}{3}}\mathrm{d}x\right)=\dfrac{\ln x}{x^3}-\displaystyle\int{x^{-4}}\mathrm{d}x=\dfrac{\ln x}{x^3}+\dfrac{1}{3x^3}+C$.
	}
\end{ex}
\begin{ex}
	[Mã 110 2017]%Câu 105
	Cho $F(x)=\left(x-1\right){\mathrm{e}^x}$ là một nguyên hàm của hàm số $f(x){\mathrm{e}^{2x}}$. Tìm nguyên hàm của hàm số $f'(x){e^{2x}}$.
	\choice
	{$\displaystyle\int{f'(x){\mathrm{e}^{2x}}}\mathrm{d}x=\left(4-2x\right){\mathrm{e}^x}+C$}
	{$\displaystyle\int{f'(x){\mathrm{e}^{2x}}}\mathrm{d}x=\left(x-2\right){\mathrm{e}^x}+C$}
	{$\displaystyle\int{f'(x){\mathrm{e}^{2x}}}\mathrm{d}x=\dfrac{2-x}{2}{\mathrm{e}^x}+C$}
	{\True $\displaystyle\int{f'(x){\mathrm{e}^{2x}}}\mathrm{d}x=\left(2-x\right){\mathrm{e}^x}+C$}
	\loigiai{
		Theo đề bài ta có $\displaystyle\int{f(x).\mathrm{e}^{2x}\mathrm{d}x}=\left(x-1\right){\mathrm{e}^x}+C$, suy ra $ f(x).\mathrm{e}^{2x}=\left[\left(x-1\right){\mathrm{e}^x}\right]'=\mathrm{e}^x+\left(x-1\right).\mathrm{e}^x$\\
		$\Rightarrow f(x)=\mathrm{e}^{-x}+\left(x-1\right).e^{-x}=x.\mathrm{e}^{-x}\Rightarrow{f}'(x)=\left(1-x\right).\mathrm{e}^{-x}$\\
		Suy ra $K=\displaystyle\int{f'(x){\mathrm{e}^{2x}}}\mathrm{d}x=\displaystyle\int{\left(1-x\right){\mathrm{e}^x}}\mathrm{d}x=\displaystyle\int{\left(1-x\right)}\mathrm{d}\left(\mathrm{e}^x\right)=\mathrm{e}^x\left(1-x\right)+\displaystyle\int{\mathrm{e}^x\mathrm{d}x}=\left(2-x\right){\mathrm{e}^x}+C$.
	}
\end{ex}
\begin{ex}
	Cho hàm số $f(x)$thỏa mãn $f'(x)=x{\mathrm{e}^x}$và $f(0)=2$.Tính $f(1)$.
	\choice
	{\True $f(1)=3$}
	{$f(1)=\mathrm{e}$}
	{$f(1)=5-\mathrm{e}$}
	{$f(1)=8-2\mathrm{e}$}
	\loigiai{
		Ta có\\
		$f(x)=\displaystyle\int{f'(x)}\mathrm{d}x=\displaystyle\int{x.\mathrm{e}^x}\mathrm{d}x$\\
		Đặt $\left\{\begin{aligned}
			& u=x\\ 
			& \mathrm{d}v=\mathrm{e}^x\mathrm{d}x\\ 
		\end{aligned}\right.\to\left\{\begin{aligned}
			& \mathrm{d}u=\mathrm{d}x\\ 
			& v=\mathrm{e}^x\\ 
		\end{aligned}\right.$ $\to f(x)=x.\mathrm{e}^x-\displaystyle\int{\mathrm{e}^x\mathrm{d}x=x.\mathrm{e}^x-\mathrm{e}^x+C}$\\
		Theo đề $f(0)=2\Leftrightarrow 2=-1+C\Leftrightarrow C=3$\\
		$\Rightarrow f(x)=x.\mathrm{e}^x-\mathrm{e}^x+3$\\
		$\Rightarrow f(1)=3$.
	}
\end{ex}
\begin{ex}
	[Chuyên Đại Học Vinh 2019]%Câu 107
	Cho hàm số $f(x)$ thỏa mãn $f(x)+f'(x)=\mathrm{e}^{-x},\forall x\in\mathbb{R}$ và $f(0)=2$. Tất cả các nguyên hàm của $f(x){\mathrm{e}^{2x}}$ là
	\choice
	{$\left(x-2\right){\mathrm{e}^x}+\mathrm{e}^x+C$}
	{$\left(x+2\right){\mathrm{e}^{2x}}+\mathrm{e}^x+C$} 
	{$\left(x-1\right){\mathrm{e}^x}+C$}
	{\True $\left(x+1\right){\mathrm{e}^x}+C$}
	\loigiai{
		Ta có $f(x)+f'(x)=\mathrm{e}^{-x}\Leftrightarrow f(x){\mathrm{e}^x}+f'(x){\mathrm{e}^x}=1$ $\Leftrightarrow{\left(f(x){\mathrm{e}^x}\right)'}=1\Leftrightarrow f(x){\mathrm{e}^x}=x+C_1$.\\
		Vì $f(0)=2\Rightarrow{C_1}=2\Rightarrow f(x){\mathrm{e}^{2x}}=\left(x+2\right){\mathrm{e}^x}$ $\Rightarrow\displaystyle\int{f(x){\mathrm{e}^{2x}}}\mathrm{d}x=\displaystyle\int{\left(x+2\right){\mathrm{e}^x}\mathrm{d}x}$ .\\
		Đặt $\left\{\begin{aligned}
			& u=x+2\\ 
			&\mathrm{d}v=\mathrm{e}^x\mathrm{d}x\\ 
		\end{aligned}\right.\Rightarrow\left\{\begin{aligned}
			&\mathrm{d}u=\mathrm{d}x\\ 
			& v=\mathrm{e}^x\\ 
		\end{aligned}\right.$\\
		$\Rightarrow\displaystyle\int{f(x){\mathrm{e}^{2x}}}\mathrm{d}x=\displaystyle\int{\left(x+2\right){\mathrm{e}^x}\mathrm{d}x}$ $=\left(x+2\right){\mathrm{e}^x}-\displaystyle\int{\mathrm{e}^x\mathrm{d}x}$ $=\left(x+2\right){\mathrm{e}^x}-\mathrm{e}^x+C=\left(x+1\right){\mathrm{e}^x}+C$.
	}
\end{ex}
\begin{ex}
	[Việt Đức Hà Nội 2019]%Câu 108
	Cho hàm số $ y=f(x)$ thỏa mãn $ f'(x)=\left(x+1\right){\mathrm{e}^x},f(0)=0$ và $\displaystyle\int{f(x)}\mathrm{d}x=\left(ax+b\right){\mathrm{e}^x}+c$ với $ a,b,c$ là các hằng số. Khi đó:
	\choice
	{$ a+b=2$}
	{$ a+b=3$}
	{$ a+b=1$}
	{\True $ a+b=0$}
	\loigiai{
		Theo đề $ f'(x)=\left(x+1\right){\mathrm{e}^x}$. Nguyên hàm 2 vế ta được\\ 
		$\begin{aligned}
			&\displaystyle\int{f'(x)\mathrm{d}x}=\displaystyle\int{\left(x+1\right){\mathrm{e}^x}\mathrm{d}x\Leftrightarrow f(x)}=\left(x+1\right){\mathrm{e}^x}-\displaystyle\int{\mathrm{e}^x}\mathrm{d}x\\ 
			&\Rightarrow f(x)=\left(x+1\right){\mathrm{e}^x}-\mathrm{e}^x+C=x{\mathrm{e}^x}+C\\ 
		\end{aligned}$\\
		Mà $f(0)=0\Rightarrow 0.\mathrm{e}^0+C=0\Leftrightarrow C=0\Rightarrow f(x)=x{\mathrm{e}^x}$.\\
		$\Rightarrow\displaystyle\int{f(x)}\mathrm{d}x=\displaystyle\int{x{\mathrm{e}^x}}\mathrm{d}x=x{\mathrm{e}^x}-\displaystyle\int{\mathrm{e}^x}\mathrm{d}x=x{\mathrm{e}^x}-\mathrm{e}^x+C=\left(x-1\right){\mathrm{e}^x}+C$.\\
		Suy ra $ a=1;b=-1\Rightarrow a+b=0$.
	}
\end{ex}
\begin{ex}
	[THPT Nguyễn Thị Minh Khai - Hà Tĩnh - 2018]%Câu 109
	Gọi $F(x)$ là một nguyên hàm của hàm số $f(x)=x{\mathrm{e}^{-x}}$. Tính $F(x)$ biết $F(0)=1$.
	\choice
	{\True $F(x)=-\left(x+1\right){\mathrm{e}^{-x}}+2$}
	{$F(x)=\left(x+1\right){\mathrm{e}^{-x}}+1$}
	{$F(x)=\left(x+1\right){\mathrm{e}^{-x}}+2$}
	{$F(x)=-\left(x+1\right){\mathrm{e}^{-x}}+1$}
	\loigiai{
		Đặt $\left\{\begin{aligned}
			& u=x\\ 
			&\mathrm{d}v=\mathrm{e}^{-x}\mathrm{d}x\\ 
		\end{aligned}\right.\Rightarrow\left\{\begin{aligned}
			&\mathrm{d}u=\mathrm{d}x\\ 
			& v=-\mathrm{e}^{-x}\\ 
		\end{aligned}\right.$.\\
		Do đó $\displaystyle\int{x{\mathrm{e}^{-x}}\mathrm{d}x}=-x{\mathrm{e}^{-x}}+\displaystyle\int{\mathrm{e}^{-x}\mathrm{d}x}$ $=-x{\mathrm{e}^{-x}}-\mathrm{e}^{-x}+C=F\left(x;C\right)$.\\
		$F(0)=1$ $\Leftrightarrow-\mathrm{e}^{-0}+C=1\Leftrightarrow C=2$. Vậy $F(x)=-\left(x+1\right){\mathrm{e}^{-x}}+2$.
	}
\end{ex}
\begin{ex}
	[Sở Quảng Nam - 2018]%Câu 110
	Biết $\displaystyle\int{x\cos 2x\mathrm{d}x}=ax\sin 2x+b\cos 2x+C$ với $ a$, $ b$ là các số hữu tỉ. Tính tích $ ab$?
	\choice
	{\True $ ab=\dfrac{1}{8}$}
	{$ ab=\dfrac{1}{4}$}
	{$ ab=-\dfrac{1}{8}$}
	{$ ab=-\dfrac{1}{4}$}
	\loigiai{
		Đặt $\left\{\begin{aligned}
			& u=x\\ 
			& \mathrm{d}v=\cos 2x\mathrm{d}x\\ 
		\end{aligned}\right.\Rightarrow\left\{\begin{aligned}
			&\mathrm{d}u=\mathrm{d}x\\ 
			& v=\dfrac{1}{2}\sin 2x\\ 
		\end{aligned}\right.$\\
		Khi đó $\displaystyle\int{x\cos 2x\mathrm{d}x}=\dfrac{1}{2}x\sin 2x-\dfrac{1}{2}\displaystyle\int{\sin 2x\mathrm{d}x}=\dfrac{1}{2}x\sin 2x+\dfrac{1}{4}\cos 2x+C$\\
		$\Rightarrow a=\dfrac{1}{2}$, $ b=\dfrac{1}{4}$.\\
		Vậy $ ab=\dfrac{1}{8}$.
	}
\end{ex}
\begin{ex}
	[Chuyên Đh Vinh - 2018]%Câu 111
	Giả sử $F(x)$ là một nguyên hàm của $f(x)=\dfrac{\ln\left(x+3\right)}{x^2}$ sao cho $F\left(-2\right)+F(1)=0$. Giá trị của $F\left(-1\right)+F(2)$ bằng
	\choice
	{\True $\dfrac{10}{3}\ln 2-\dfrac{5}{6}\ln 5$}
	{$0$}
	{$\dfrac{7}{3}\ln 2$}
	{$\dfrac{2}{3}\ln 2+\dfrac{3}{6}\ln 5$}
	\loigiai{
		Tính $\displaystyle\int{\dfrac{\ln\left(x+3\right)}{x^2}\mathrm{d}x}$.\\
		Đặt $\left\{\begin{aligned}
			& u=\ln\left(x+3\right)\\ 
			&\mathrm{d}v=\dfrac{\mathrm{d}x}{x^2}\\ 
		\end{aligned}\right.\Rightarrow\left\{\begin{aligned}
			&\mathrm{d}u=\dfrac{\mathrm{d}x}{x+3}\\ 
			& v=-\dfrac{1}{x}\\ 
		\end{aligned}\right.$\\
		Ta có $\displaystyle\int{\dfrac{\ln\left(x+3\right)}{x^2}\mathrm{d}x}=-\dfrac{1}{x}\ln\left(x+3\right)+\displaystyle\int{\dfrac{\mathrm{d}x}{x\left(x+3\right)}}$ $=-\dfrac{1}{x}\ln\left(x+3\right)+\dfrac{1}{3}\ln\left|\dfrac{x}{x+3}\right|+C=F\left(x,C\right)$ .\\
		Lại có $F\left(-2\right)+F(1)=0$ $\Leftrightarrow\left(\dfrac{1}{3}\ln 2+C\right)+\left(-\ln 4+\dfrac{1}{3}\ln\dfrac{1}{4}+C\right)=0$ $\Leftrightarrow 2C=\dfrac{7}{3}\ln 2$ .\\
		Suy ra $F\left(-1\right)+F(2)=\ln 2+\dfrac{1}{3}\ln 2-\dfrac{1}{2}\ln 5+\dfrac{1}{3}\ln\dfrac{2}{5}+2C$ $=\dfrac{10}{3}\ln 2-\dfrac{5}{6}\ln 5$.
	}
\end{ex}
\begin{ex}
	[THCS \& THPT Nguyễn Khuyến - Bình Dương - 2018]%Câu 112
	Gọi $ g(x)$ là một nguyên hàm của hàm số $ f(x)=\ln\left(x-1\right)$. Cho biết $ g(2)=1$ và $ g(3)=a\ln b$ trong đó $ a,b$ là các số nguyên dương phân biệt. Hãy tính giá trị của $ T=3a^2-b^2$
	\choice
	{$ T=8$}
	{$ T=-17$}
	{$ T=2$}
	{\True $ T=-13$}
	\loigiai{
		Đặt $\left\{\begin{aligned}
			& u=\ln\left(x-1\right)\\ 
			& dv=\mathrm{d}x\\ 
		\end{aligned}\right.\Rightarrow\left\{\begin{aligned}
			& \mathrm{d}u=\dfrac{1}{x-1}\\ 
			& v=x-1\\ 
		\end{aligned}\right.$\\
		$g(x)=\displaystyle\int{\ln\left(x-1\right)\mathrm{d}x=\left(x-1\right)\ln\left(x+1\right)-\displaystyle\int{\dfrac{x-1}{x-1}\mathrm{d}x}}=\left(x-1\right)\ln\left(x-1\right)-x+C$\\
		Do $ g(2)=1\Leftrightarrow 1\ln 1-2+C=1\Leftrightarrow C=3\Rightarrow g(x)=\left(x-1\right)\ln\left(x-1\right)-x+3$\\
		Suy ra $ g(3)=2\ln 2-3+3=2\ln 2=\ln 4$ $\Rightarrow a=1,b=4\Rightarrow 3a^2-b^2=-13$.
	}
\end{ex}
\begin{ex}
	[Sở Quảng Nam - 2018]%Câu 113
	Biết $\displaystyle\int{x\cos 2x\mathrm{d}x}=ax\sin 2x+b\cos 2x+C$ với $ a$, $ b$ là các số hữu tỉ. Tính tích $ ab$?
	\choice
	{\True $ ab=\dfrac{1}{8}$}
	{$ ab=\dfrac{1}{4}$}
	{$ ab=-\dfrac{1}{8}$}
	{$ ab=-\dfrac{1}{4}$}
	\loigiai{
		Đặt $\left\{\begin{aligned}
			& u=x\\ 
			& \mathrm{d}v=\cos 2x\mathrm{d}x\\ 
		\end{aligned}\right.\Rightarrow\left\{\begin{aligned}
			&\mathrm{d}u=\mathrm{d}x\\ 
			& v=\dfrac{1}{2}\sin 2x\\ 
		\end{aligned}\right.$\\
		Khi đó $\displaystyle\int{x\cos 2x\mathrm{d}x}=\dfrac{1}{2}x\sin 2x-\dfrac{1}{2}\displaystyle\int{\sin 2x\mathrm{d}x}=\dfrac{1}{2}x\sin 2x+\dfrac{1}{4}\cos 2x+C$\\
		$\Rightarrow a=\dfrac{1}{2}$, $ b=\dfrac{1}{4}$.\\
		Vậy $ ab=\dfrac{1}{8}$.
	}
\end{ex}
\begin{ex}
	[Sở Hậu Giang 2022]%Câu 114
	Biết $\displaystyle\int{\left(a{x^2}+bx+5\right){\mathrm{e}^x}\mathrm{d}x}=\left(3x^2-8x+13\right){\mathrm{e}^x}+C$, với $ a,b$ là các số nguyên. Tìm $ S=a+b$.
	\choice
	{\True $ S=1$}
	{$ S=4$}
	{$ S=5$}
	{$ S=9$}
	\loigiai{
		Xét $ I=\displaystyle\int{\left(a{x^2}+bx+5\right){\mathrm{e}^x}\mathrm{d}x}$\\
		Đặt $\left\{\begin{aligned}
			&u=a{x^2}+bx+5\Rightarrow\mathrm{d}u=\left(2ax+b\right)\mathrm{d}x\\ 
			&\mathrm{d}v=\mathrm{e}^x\mathrm{d}x\Rightarrow v=\mathrm{e}^x\\ 
		\end{aligned}\right.$\\
		Khi đó\\
		$I=\displaystyle\int{\left(a{x^2}+bx+5\right){\mathrm{e}^x}\mathrm{d}x}=\left(a{x^2}+bx+5\right){\mathrm{e}^x}-\displaystyle\int{\left(2ax+b\right){\mathrm{e}^x}\mathrm{d}x}$\\
		$\Leftrightarrow I=\left(a{x^2}+bx+5\right){\mathrm{e}^x}-\left(2ax+b\right){\mathrm{e}^x}+\displaystyle\int{2a.\mathrm{e}^x\mathrm{d}x}$\\
		$\Leftrightarrow I=\left(a{x^2}+bx+5\right){\mathrm{e}^x}-\left(2ax+b\right){\mathrm{e}^x}+2a{\mathrm{e}^x}+C$\\
		$\Leftrightarrow I=\mathrm{e}^x\left[a{x^2}+\left(b-2a\right)+\left(5-b+2a\right)\right]+C$.\\
		Vậy       $\displaystyle\int{\left(a{x^2}+bx+5\right){\mathrm{e}^x}\mathrm{d}x}=\mathrm{e}^x\left[a{x^2}+\left(b-2a\right)+\left(5-b+2a\right)\right]+C=\left(3x^2-8x+13\right){\mathrm{e}^x}+C$\\
		Ta có: $\left\{\begin{aligned}
			& a=3\\ 
			& b-2a=-8\\ 
			& 5-b+2a=13\\ 
		\end{aligned}\right.\Leftrightarrow\left\{\begin{aligned}
			& a=3\\ 
			& b=-2\\ 
		\end{aligned}\right.\left(\text{thỏa mãn}\right)$\\
		Vậy $ S=a+b=1$.
	}
\end{ex}    
\Closesolutionfile{ans}
\indapan{10}{ans/CD25/Muc_7_8}