\Opensolutionfile{ans}[ans/CD23/Muc_9_10]
\setcounter{ex}{0}
\setcounter{dang}{0}
\section{Mức độ 9,10 điểm}
\begin{dang}
	{Một số bài toán thực tế - cực trị liên quan đến mặt cầu - khối cầu}
\end{dang}
\begin{ex}
	Cho một bán cầu đựng đầy nước với bán kính $ R=2$. Người ta bỏ vào đó một quả cầu có bán kính bằng $ 2R$. Tính lượng nước còn lại trong bán cầu ban đầu.\\
	{\color{red}HÌNH Ở ĐÂY}
	\choice
	{\True $ V=\left(24\sqrt{3}-\dfrac{112}{3}\right)\pi $}
	{$ V=\dfrac{16\pi}{3}$}
	{$ V=\dfrac{8}{3}\pi $}
	{$ V=\left(24\sqrt{3}-40\right)\pi $}
	\loigiai{
		{\color{red}HÌNH Ở ĐÂY}\\
		Khi đặt khối cầu có bán kính $R'=2R$ vào khối cầu có bán kính $ R$ ta được phần chung của hai khối cầu. phần chung đó gọi là chỏm cầu. Gọi $h$ là chiều cao chỏm cầu. Thể tích khối chỏm cầu là $V_c=\pi{h^2}\left(R'-\dfrac{h}{3}\right)$.\\
		với $ h=R'-\sqrt{R'^2-R^2}=4-\sqrt{4^2-2^2}=4-2\sqrt{3}$.\\
		$\Rightarrow{V_c}=\pi{\left(4-2\sqrt{3}\right)^2}\left(4-\dfrac{4-2\sqrt{3}}{3}\right)=\dfrac{2\pi}{3}\left(64-36\sqrt{3}\right)$.\\
		Thể tích một nửa khối cầu $ V=\dfrac{1}{2}.\dfrac{4}{3}\pi{R^3}=\dfrac{16\pi}{3}$.\\
		Thể tích khối nước còn lại trong nửa khối cầu:\\
		$V_n=V-V_c=\dfrac{16\pi}{3}-\dfrac{2\pi}{3}\left(64-36\sqrt{3}\right)=\left(24\sqrt{3}-\dfrac{112}{3}\right)\pi $.
	}
\end{ex}
\begin{ex}%Câu 2
	Cho khối cầu $(S)$ tâm $ I$, bán kính $ R$ không đổi. Một khối trụ thay đổi có chiều cao $ h$ và bán kính đáy $ r$ nội tiếp khối cầu. Tính chiều cao $ h$ theo $ R$ sao cho thể tích khối trụ lớn nhất.\\
	{\color{red}HÌNH Ở ĐÂY}
	\choice
	{$ h=\dfrac{R\sqrt{2}}{2}$}
	{\True $ h=\dfrac{2R\sqrt{3}}{3}$}
	{$ h=R\sqrt{2}$}
	{$ h=\dfrac{R\sqrt{3}}{3}$}
	\loigiai{
		Ta có $ r=\sqrt{R^2-\dfrac{h^2}{4}}$.\\
		Thể tích khối trụ là $ V=\pi{r^2}h=\pi\left(R^2-\dfrac{h^2}{4}\right)h$, $ 0<h<2R$\\
		$V'_{(h)}=\pi\left(R^2-\dfrac{3h^2}{4}\right)$; $V'_{(h)}=0\Leftrightarrow h=\pm\dfrac{2R\sqrt{3}}{3}$.\\
		Bảng biến thiên\\
		{\color{red}HÌNH Ở ĐÂY}\\
		Vậy thể tích khối trụ lớn nhất khi $ h=\dfrac{2R\sqrt{3}}{3}$.
	}
\end{ex}
\begin{ex}
	[HSG Bắc Ninh 2019]%Câu 3
	Một cơ sở sản suất đồ gia dụng được đặt hàng làm các chiếc hộp kín hình trụ bằng nhôm đề đựng rượu có thể tích là $ V=28\pi{a^3}$ $\left(a>0\right)$. Để tiết kiệm sản suất và mang lại lợi nhuận cao nhất thì cơ sở sẽ sản suất những chiếc hộp hình trụ có bán kính là $ R$ sao cho diện tích nhôm cần dùng là ít nhất. Tìm $ R$
	\choice
	{$ R=a\sqrt[3]{7}$}
	{$ R=2a\sqrt[3]{7}$}
	{$ R=2a\sqrt[3]{14}$}
	{\True $ R=a\sqrt[3]{14}$}
	\loigiai{
		Diện tích nhôm cần dùng đề sản suất là diện tích toàn phần $ S$\\
		Ta có $ l=h$; mà $ V=28\pi{a^3}\Leftrightarrow\pi{R^2}h=28\pi{a^3}\Leftrightarrow h=\dfrac{28a^3}{R^2}$\\
		$ S=2\pi Rl+2\pi{R^2}=2\pi\dfrac{28a^3}{R}+2\pi{R^2}$ với $ R>0$\\
		$S'=2\pi\left(-\dfrac{28a^3}{R^2}+2R\right)=0\Leftrightarrow R=a\sqrt[3]{14}$\\
		Bảng biến thiên\\
		{\color{red}HÌNH Ở ĐÂY}\\
		Vậy $S_{\min}\Leftrightarrow R=a\sqrt[3]{14}$.
	}
\end{ex}
\begin{ex}
	[Mã 104 2017]%Câu 4
	Trong tất cả các hình chóp tứ giác đều nội tiếp mặt cầu có bán kính bằng $ 9$, tính thể tích $ V$của khối chóp có thể tích lớn nhất.
	\choice
	{$ V=576\sqrt{2}$}
	{$ V=144\sqrt{6}$}
	{$ V=144$}
	{\True $ V=576$}
	\loigiai{
		{\color{red}HÌNH Ở ĐÂY}\\
		Xét hình chóp tứ giác đều $ S.ABCD$ nội tiếp mặt cầu có tâm $ I$ và bán kính $ R=9$.\\
		Gọi $ H=AC\cap BD$, $ K$ là trung điểm $ SC$.\\
		Đặt $ AB=x;SH=h$, $\left(x,h>0\right)$.\\
		Ta có $ HC=\dfrac{x}{\sqrt{2}}$$\Rightarrow l=SC=\sqrt{h^2+\dfrac{x^2}{2}}$.\\
		Do $\Delta SHI\backsim\Delta SHC\Rightarrow\dfrac{SK}{SH}=\dfrac{SI}{SC}\Rightarrow{l^2}=2h.R$ $\Rightarrow{x^2}=36h-2h^2$.\\
		Diện tích đáy của hình chóp $S_{ABCD}=x^2$ nên $ V=\dfrac{1}{3}h.x^2=\dfrac{1}{3}h\left(36h-2h^2\right)$.\\
		Ta có $\dfrac{1}{3}h.\left(36h-2h^2\right)=\dfrac{1}{3}.h.h\left(36-2h\right)\le\dfrac{1}{3}.\left(\dfrac{h+h+36-2h}{3}\right)^3=576\Rightarrow V\le 576$, dấu bằng xảy ra khi $ h=h=36-2h\Leftrightarrow h=12,x=12$. Vậy $V_{\max}=576$.
	}
\end{ex}
\begin{ex}
	[Sở Vĩnh Phúc 2019]%Câu 5
	Trong tất cả các hình chóp tứ giác đều nội tiếp mặt cầu có bán kính bằng $ 9$, khối chóp có thể tích lớn nhất bằng bao nhiêu ?
	\choice
	{$ 576\sqrt{2}$}
	{$ 144$}
	{\True $ 576$}
	{$ 144\sqrt{6}$}
	\loigiai{
		{\color{red}HÌNH Ở ĐÂY}\\
		Giả sử khối chóp $ S.ABCD$ là khối chóp tứ giác đều nội tiếp mặt cầu có bán kính bằng $ 9$.\\
		Gọi $ O$ là tâm hình vuông $ ABCD$ thì $ SO\perp\left(ABCD\right)$. $ M$ là trung điểm của $ SA$, kẻ $ MI$ vuông góc với $ SA$ và cắt $ SO$ tại $ I$ thì $ I$ là tâm mặt cầu ngoại tiếp hình chóp $ S.ABCD$, bán kính của mặt cầu là $ IA=IS=9$.\\
		Đặt $ IO=x$, $ 0\le x\le 9$, do $\Delta IAO$ vuông tại $ O$ nên $ AO=\sqrt{A{I^2}-I{O^2}}$$=\sqrt{81-x^2}$, suy ra $ AC=2\sqrt{81-x^2}$.\\
		Do tứ giác $ ABCD$ là hình vuông nên $ AB=\dfrac{AC}{\sqrt{2}}$$=\sqrt{2}.\sqrt{81-x^2}$, suy ra $S_{ABCD}=A{B^2}$$=2\left(81-x^2\right)$.\\
		Vậy $V_{S.ABCD}=\dfrac{1}{3}{S_{\square ABCD}}.SO$$=\dfrac{2}{3}\left(81-x^2\right).\left(9+x\right)$$=\dfrac{2}{3}\left(-x^3-9x^2+81x+729\right)$.\\
		Xét hàm số $ f(x)=$$\dfrac{2}{3}\left(-x^3-9x^2+81x+729\right)$ với $ x\in\left[0;9\right]$.\\
		$f'(x)=2\left(-x^2-6x+27\right)$; $f'(x)=0$$\Leftrightarrow $$\left[\begin{aligned}
			& x=3\\ 
			& x=-9(l)\\ 
		\end{aligned}\right.$\\
		Bảng biến thiên:\\
		{\color{red}HÌNH Ở ĐÂY}\\
		Dựa vào bảng biến thiên ta thấy: $\underset{x\in\left[0;9\right]}{\max}f(x)=f(3)$$=576$.\\
		Vậy khối chóp có thể tích lớn nhất bằng $ 576$.
	}
\end{ex}  
\begin{ex}
	[Chuyên Vĩnh Phúc 2019]%Câu 6
	Trong không gian $ Oxyz$, lấy điểm $C$ trên tia $ Oz$ sao cho $ OC=1$. Trên hai tia $Ox,Oy$ lần lượt lấy hai điểm $A,B$ thay đổi sao cho $ OA+OB=OC$. Tìm giá trị nhỏ nhất của bán kính mặt cầu ngoại tiếp tứ diện $ O.ABC$?
	\choice
	{\True $\dfrac{\sqrt{6}}{4}$}
	{$\sqrt{6}$}
	{$\dfrac{\sqrt{6}}{3}$}
	{$\dfrac{\sqrt{6}}{2}$}
	\loigiai{
		{\color{red}HÌNH Ở ĐÂY}{\color{red}HÌNH Ở ĐÂY}\\
		Bốn điểm $O,A,B,C$ tạo thành 1 tam diện vuông.\\
		Bán kính mặt cầu ngoại tiếp tứ diện $ O.ABC$ là $ R=\dfrac{\sqrt{O{A^2}+O{B^2}+O{C^2}}}{2}$.\\
		Đặt $ OA=a;OB=b,a,b>0.$ Ta có $ a+b=1\Leftrightarrow b=1-a$.\\
		Vậy $ R=\dfrac{\sqrt{O{A^2}+O{B^2}+O{C^2}}}{2}$ $=\dfrac{\sqrt{a^2+b^2+1^2}}{2}$ $=\dfrac{\sqrt{a^2+\left(1-a\right)^2+1^2}}{2}$\\
		$=\dfrac{\sqrt{2\left(\left(a-\frac{1}{2}\right)^2+\frac{3}{4}\right)}}{2}\ge\dfrac{\sqrt{6}}{4}$.\\
		Vậy $R_{\min}=\dfrac{\sqrt{6}}{4}$, tại $ a=b=\dfrac{1}{2}.$.
	}
\end{ex}
\begin{ex}
	[KTNL GV THPT Lý Thái Tổ 2019]%Câu 7
	Cho hình chóp $ S.ABCD$ có đáy $ ABCD$ là hình bình hành, các cạnh bên của hình chóp bằng $\sqrt{6}\mathrm{cm}$, $ AB=4\mathrm{cm}$. Khi thể tích khối chóp $ S.ABCD$ đạt giá trị lớn nhất, tính diện tích mặt cầu ngoại tiếp $ S.ABCD$.
	\choice
	{$ 12\pi\mathrm{cm}^2$}
	{$ 4\pi\mathrm{cm}^2$}
	{$ 9\pi\mathrm{cm}^2$}
	{\True $ 36\pi\mathrm{cm}^2$}
	\loigiai{
		{\color{red}HÌNH Ở ĐÂY}\\
		Gọi $ O$ là giao điểm của $ AC$ và $ BD$.\\
		Ta có $\Delta SAC$ cân tại $ S$ nên $ SO\perp AC$ và $\Delta SBD$ cân tại S nên $ SO\perp BD$.\\
		Khi đó $ SO\perp\left(ABCD\right).$\\
		Ta có: $\Delta SAO=\Delta SBO=\Delta SCO=\Delta SDO\Rightarrow OA=OB=OC=OD$\\
		Vậy hình bình hành $ ABCD$ là hình chữ nhật.\\
		Đặt $ BC=x\Rightarrow AC=\sqrt{4^2+x^2}\Rightarrow AO=\dfrac{AC}{2}=\dfrac{\sqrt{16+x^2}}{2}.$\\
		Xét $\Delta SAO$ vuông tại $ O$, ta có $ SO=\sqrt{S{A^2}-A{O^2}}=\sqrt{6-\dfrac{16+x^2}{4}}=\dfrac{\sqrt{8-x^2}}{2}$\\
		Thể tích khối chóp $ S.ABCD$ là: $V_{S.ABCD}=\dfrac{1}{3}SO.S_{ABCD}=\dfrac{1}{3}.\dfrac{\sqrt{8-x^2}}{2}.4x=\dfrac{2}{3}.\sqrt{8-x^2}.x$\\
		Áp dụng bất đẳng thức :$ ab\le\dfrac{a^2+b^2}{2}$ ta có $ V=\dfrac{2}{3}.\sqrt{8-x^2}.x\le\dfrac{2}{3}.\dfrac{8-x^2+x^2}{2}=\dfrac{8}{3}.$\\
		Dấu \lq\lq$=$\rq\rq xảy ra $\Leftrightarrow\sqrt{8-x^2}=x\Leftrightarrow x=2.$ Do đó: $ BC=2,SO=1.$\\
		Gọi $ M$ là trung điểm của $ SA$, trong $\left(SAO\right)$ kẻ đường trung trực của $ SA$ cắt $ SO$ tại $ I$.\\
		Khi đó mặt cầu ngoại tiếp khối chóp $ S.ABCD$ có tâm $ I$và bán kính $ R=IS.$\\
		Vì $\Delta SMI\backsim\Delta SOA(g.g)$ nên $\dfrac{SI}{SA}=\dfrac{SM}{SO}\Rightarrow SI=\dfrac{S{A^2}}{2.SO}=\dfrac{6}{2.1}=3\Rightarrow R=3(cm).$\\
		Diện tích mặt cầu ngoại tiếp khối chóp $ S.ABCD$ là: $ 4\pi{R^2}=4\pi{3^2}=36\pi (\mathrm{cm}^2)$.
	}
\end{ex}
\begin{ex}
	Cho mặt cầu $ (S)$ có bán kính $ R=5$. Khối tứ diện $ ABCD$ có tất cả các đỉnh thay đổi và cùng thuộc mặt cầu $ (S)$ sao cho tam giác $ABC$ vuông cân tại $ B$ và$ DA=DB=DC$. Biết thể tích lớn nhất của khối tứ diện $ ABCD$ là $\dfrac{a}{b}$ ($ a$,$ b$ là các số nguyên dương và $\dfrac{a}{b}$ là phân số tối giản), tính$ a+b$.
	\choice
	{$a+b=1173$}
	{\True $a+b=4081$}
	{$a+b=128$}
	{$a+b=5035$}
	\loigiai{
		{\color{red}HÌNH Ở ĐÂY}\\
		Gọi $ H$ là trung điểm của $ AC$, Vì tam giác $ ABC$ vuông\\
		cân tại $ B$ và $ DA=DB=DC$ nên $ DH\perp (ABC)$ và tâm\\
		$ I$ của mặt cầu $ (S)$ thuộc tia$ DH$. Đặt $ DH=x$ và $ AH=a$\\
		($ 0<a\le 5,0<x<10$).\\
		Có $ ID=IA=5$ và $ IH=\left| x-5\right|$.\\
		Xét tam giác vuông $ AIH$ có $a^2=A{H^2}=A{I^2}-I{H^2}=25-(x-5)^2=10x-x^2$.\\
		Diện tích tam giác $ ABC$ là $ S=\dfrac{1}{2}AC.BH=a^2=10x-x^2$.\\
		Thể tích khối chóp $ ABCD$ là $ V=\dfrac{1}{3}S_{ABC}^{}.DH=\dfrac{1}{3}(10x-x^2)x$.\\
		Xét $ f(x)=\dfrac{1}{3}(10x-x^2)x=\dfrac{1}{3}(10x^2-x^3)$ với $ 0<x<10$.\\
		Lập bảng biến thiên cho hàm số $ f(x)$ ta được giá trị lớn nhất của hàm số $ f(x)$ trên nửa\\
		khoảng $\left(0;10\right)$ ta có kết quả là $\dfrac{4000}{81}$ tại $ x=\dfrac{20}{3}$.\\
		Vậy $ a=4000,b=81$ nên $ a+b=4081$.
	}
\end{ex}
\begin{ex}
	Trong không gian cho tam giác $ ABC$ có $ AB=2R,AC=R,\widehat{CAB}=120^\circ$. Gọi $ M$ là điểm thay đổi thuộc mặt cầu tâm $ B$, bán kính $ R$. Giá trị nhỏ nhất của $ MA+2MC$ là
	\choice
	{$ 4R$}
	{$ 6R$}
	{\True $ R\sqrt{19}$}
	{$ 2R\sqrt{7}$}
	\loigiai{
		{\color{red}HÌNH Ở ĐÂY}\\
		Ta có $M{A^2}=\left(\overrightarrow{MB}+\overrightarrow{BA}\right)^2=\left(\overrightarrow{MB}^2+2\overrightarrow{MB}.\overrightarrow{BA}+\overrightarrow{BA}^2\right)=\left(\dfrac{BA}{MB}\overrightarrow{MB}+\dfrac{MB}{BA}\overrightarrow{BA}\right)^2=\left(2\overrightarrow{MB}+\dfrac{1}{2}\overrightarrow{BA}\right)^2$.\\
		$\Rightarrow M{A^2}=\left| 2\overrightarrow{MB}+\dfrac{1}{2}\overrightarrow{BA}\right|^2$ $\Rightarrow MA=2\left|\overrightarrow{MB}+\dfrac{\overrightarrow{BA}}{4}\right|$.\\
		Gọi $ D$ là điểm thỏa mãn $\overrightarrow{BD}=\dfrac{\overrightarrow{BA}}{4}$, khi đó $ MA=2\left|\overrightarrow{MB}+\overrightarrow{BD}\right|=2\left|\overrightarrow{MD}\right|=2MD$.\\
		Do đó $ MA+2MC=2\left(MC+MD\right)\ge 2CD$.\\
		Lại có $ C{D^2}=A{C^2}+A{D^2}-2AC.AD\cos 120^\circ=\dfrac{19}{4}{R^2}\Rightarrow CD=R\dfrac{\sqrt{19}}{2}$.\\
		Dấu bằng xảy ra khi $ M$ là giao điểm của đoạn $ CD$ với mặt cầu tâm $ B$ bán kính $ R$.\\
		Vậy giá trị nhỏ nhất của $ MA+2MC$ là $ R\sqrt{19}$.
	}
\end{ex}
\begin{ex}
	Cho mặt cầu $(S)$ có bán kính bằng $ 3(\mathrm{m})$, đường kính $ AB$. Qua $ A$ và $ B$ dựng các tia $ A{t_1}$, $B{t_2}$ tiếp xúc với mặt cầu và vuông góc với nhau. $ M$ và $ N$ là hai điểm lần lượt di chuyển trên $ A{t_1}$, $B{t_2}$ sao cho $ MN$ cũng tiếp xúc với $(S)$. Biết rằng khối tứ diện $ ABMN$ có thể tích $ V\left(\mathrm{m}^3\right)$ không đổi. $ V$ thuộc khoảng nào sau đây?
	\choice
	{\True $\left(17;21\right)$}
	{$\left(15;17\right)$}
	{$\left(25;28\right)$}
	{$\left(23;25\right)$}
	\loigiai{
		{\color{red}HÌNH Ở ĐÂY}\\
		Giả sử $ MN$ tiếp xúc $(S)$ tại $ H$.\\
		Đặt $ MA=MH=x$, $ NB=NH=y$. Khi đó $ V=\dfrac{1}{6}.x.2R.y=\dfrac{1}{3}Rxy$.\\
		Ta có tam giác $ AMN$ vuông tại $ A$ (Vì $ MA\perp AB,\ MA\perp BN$).\\
		$\Rightarrow A{N^2}=\left(x+y\right)^2-x^2$.\\
		Lại có tam giác $ ABN$vuông tại $ B$ $\Rightarrow A{N^2}=4R^2+y^2$.\\
		Suy ra $\left(x+y\right)^2-x^2=4R^2+y^2\Leftrightarrow xy=2R^2$.\\
		Vậy $ V=\dfrac{1}{3}.R.2R^2=\dfrac{2R^3}{3}=18\in\left(17;21\right)$.
	}
\end{ex}
\begin{ex}
	Trên mặt phẳng $(P)$ cho góc $\widehat{xOy}=60^\circ $. Đoạn $ SO=a$ và vuông góc với mặt phẳng $\left(\alpha\right)$. Các điểm $ M;N$ chuyển động trên $Ox,Oy$ sao cho ta luôn có: $ OM+ON=a$. Tính diện tích của mặt cầu $(S)$ có bán kính nhỏ nhất ngoại tiếp tứ diện $ SOMN$.
	\choice
	{\True $\dfrac{4\pi{a^2}}{3}$}
	{$\dfrac{\pi{a^2}}{3}$}
	{$\dfrac{8\pi{a^2}}{3}$}
	{$\dfrac{16\pi{a^2}}{3}$}
	\loigiai{
		{\color{red}HÌNH Ở ĐÂY}\\
		Gọi $ H$, $ I$ lần lượt là tâm đường tròn ngoại tiếp tam giác $ OMN$và tâm bán mặt cầu ngoại tiếp tứ diện $ SOMN$$\Rightarrow{R^2}=O{H^2}+I{H^2}=\dfrac{a^2}{4}+O{H^2}$.\\
		Áp dụng định lý hàm số sin trong tam giác $ OMN$ ta có $\dfrac{MN}{{sin}60^\circ}=2OH$ $\Leftrightarrow OH=\dfrac{MN}{\sqrt{3}}$.\\
		Áp dụng định lý hàm số cosin trong tam giác $ OMN$ ta có $ M{N^2}=O{M^2}+O{N^2}-2.OM.ON{cos}\widehat{MON}$ $=O{M^2}+O{N^2}-OM.ON$$=\left(OM+ON\right)^2-3OM.ON$$\ge{a^2}-3\dfrac{\left(OM+ON\right)^2}{4}=\dfrac{a^2}{4}$\\
		$\Rightarrow M{N^2}\ge\dfrac{a^2}{4}$$\Leftrightarrow 3O{H^2}\ge\dfrac{a^2}{4}$$\Rightarrow{R^2}=\dfrac{a^2}{4}+O{H^2}\ge\dfrac{a^2}{4}+\dfrac{a^2}{3.4}=\dfrac{a^2}{3}$\\
		Bán kính nhỏ nhất của mặt cầu ngoại tiếp tứ diện $ SOMN$bằng $\dfrac{a}{\sqrt{3}}$.\\
		Tính diện tích của mặt cầu $(S)$ có bán kính nhỏ nhất ngoại tiếp tứ diện $ SOMN$là $ 4\pi{R^2}$ $=\dfrac{4\pi{a^2}}{3}$.
	}
\end{ex}
\begin{ex}
	Cho tứ diện $ABCD$ có hình chiếu của $A$ lên mặt phẳng $\left({BCD}\right)$ là $H$ nằm trong tam giác $BCD$. Biết rằng $H$ cũng là tâm của một mặt cầu bán kính $\sqrt{3}$ và tiếp xúc các cạnh $AB,AC,AD$. Dựng hình bình hành $AHBS$. Tính giá trị nhỏ nhất của bán kính mặt cầu ngoại tiếp hình chóp $S.BCD$ 
	\choice
	{$ 3$}
	{$ 3\sqrt{3}$}
	{$\dfrac{3}{2}$}
	{\True $\dfrac{3\sqrt{3}}{2}$}
	\loigiai{
		{\color{red}HÌNH Ở ĐÂY}\\
		Gọi ${M,N,P}$lần lượt là hình chếu của ${H}$lên ${AB,AC,AD}$ta có ${HM=HN=HP=}\sqrt{{3}}\Rightarrow{AM=AN=AP}\Rightarrow{AH}\perp\left({MNP}\right)\Rightarrow\left({MNP}\right)\parallel\left({BCD}\right)\Rightarrow AB=AC=AD$\\
		(${AH}$là trục đường tròn $\Delta{MNP}$)\\
		Vậy ${A}$ thuộc trục đường tròn ngoại tiếp $\!\!\Delta\!\!{BCD}$\\
		${AH}$ là trục đường tròn ngoại tiếp $\!\!\Delta\!\!{BCD}$.\\
		Gọi ${I=AH}\bigcap{BS}\Rightarrow{IB=IC=ID=IS}$. Vậy $I$ là tâm mặt cầu ngoại tiếp S.BCD\\
		$ IH=x\Rightarrow\dfrac{1}{H{M^2}}=\dfrac{1}{H{B^2}}+\dfrac{1}{H{A^2}}\Rightarrow H{B^2}=\dfrac{12x^2}{4x^2-3}$\\
		$\Delta HBI\perp H:B{I^2}=H{B^2}+H{I^2}=\dfrac{4x^4+9x^2}{4x^2-3}$\\
		$ t=x^2\Rightarrow f(t)=\dfrac{4t^2+9t}{4t-3}(t>\dfrac{3}{4})\Rightarrow f'(t)=\dfrac{16t^2-24t-27}{\left(4t-3\right)^2}$\\
		$ f'(t)=0\Rightarrow t=\dfrac{9}{4}(n)\vee t=-\dfrac{3}{4}(l)$\\
		{\color{red}HÌNH Ở ĐÂY}\\
		Vẽ bảng biến thiên $R_{\min}=\dfrac{3\sqrt{3}}{2}$.
	}
\end{ex}
\begin{ex}
	[SGD Điện Biên - 2019]%Câu 13
	Một vật thể đựng đầy nước hình lập phương không có nắp. Khi thả một khối cầu kim loại đặc vào trong hình lập phương thì thấy khối cầu tiếp xúc với tất cả các mặt của hình lập phương đó. Tính bán kính của khối cầu, biết thể tích nước còn lại trong hình lập phương là 10. Giả sử các mặt của hình lập phương có độ dày không đáng kể
	\choice
	{\True $\sqrt[3]{\dfrac{15}{12-2\pi}}$}
	{$\sqrt[3]{\dfrac{9}{24-4\pi}}$}
	{$\sqrt[3]{\dfrac{15}{24-4\pi}}$}
	{$\sqrt[3]{\dfrac{9}{12-2\pi}}$}
	\loigiai{
		Giả sử hình lập phương có cạnh $x$. Khi đó thể tích khối lập phương là $x^3$.\\
		Bán kính khối cầu tiếp xúc với các mặt của khối lập phương là $\dfrac{x}{2}$. Do đó thể tích khối cầu tiếp xúc với các mặt của hình lập phương là $\dfrac{4}{3}\pi{\left(\dfrac{x}{2}\right)^3}=\dfrac{\pi{x^3}}{6}$.\\
		Theo đề ra ta có $x^3-\dfrac{\pi{x^3}}{6}=10\Leftrightarrow x=\sqrt[3]{\dfrac{60}{6-\pi}}$.\\
		Do đó bán kính của khối cầu là $R=\dfrac{x}{2}=\sqrt[3]{\dfrac{15}{12-2\pi}}$.
	}
\end{ex}
\begin{ex}
	[THPT Hoàng Hoa Thám - Hưng Yên 2019]%Câu 14
	Một cái thùng đựng đầy nước được tạo thành từ việc cắt mặt xung quanh của một hình nón bởi một mặt phẳng vuông góc với trục của hình nón. Miệng thùng là đường tròn có bán kính bằng ba lần bán kính mặt đáy của thùng. Người ta thả vào đó một khối cầu có đường kính bằng $\dfrac{3}{2}$ chiều cao của thùng nước và đo được thể tích nước tràn ra ngoài là $ 54\sqrt{3}\pi\left(\mathrm{dm}^3\right)$. Biết rằng khối cầu tiếp xúc với mặt trong của thùng và đúng một nửa của khối cầu đã chìm trong nước (hình vẽ). Thể tích nước còn lại trong thùng có giá trị nào sau đây?\\
	{\color{red}HÌNH Ở ĐÂY}
	\choice
	{$\dfrac{46}{5}\sqrt{3}\pi\left(\mathrm{dm}^3\right)$}
	{$ 18\sqrt{3}\pi\left(\mathrm{dm}^3\right)$}
	{\True $\dfrac{46}{3}\sqrt{3}\pi\left(\mathrm{dm}^3\right)$}
	{$ 18\pi\left(\mathrm{dm}^3\right)$}
	\loigiai{
		Xét một thiết diện qua trục của hình nón như hình vẽ. Hình thang cân $ ABCD$ ($ IJ$ là trục đối xứng) là thiết diện của cái thùng nước, hình tròn tâm $ I$ bán kính $ IH$ là thiết diện của khối cầu. Các đường thẳng $ AD$, $ BC$, $ IJ$ đồng qui tại $ E$.\\
		{\color{red}HÌNH Ở ĐÂY}\\
		Đặt bán kính của khối cầu là $ IH=R$, bán kính mặt đáy của thùng là $ JD=r$, chiều cao của thùng là $ IJ=h$. Ta có\\
		$\dfrac{2}{3}\pi{R^3}=54\sqrt{3}\pi\Leftrightarrow R=3\sqrt{3}$, $\dfrac{3}{2}h=2R=6\sqrt{3}\Leftrightarrow h=4\sqrt{3}$.\\
		$\dfrac{EJ}{EI}=\dfrac{JC}{IB}=\dfrac{r}{3r}=\dfrac{1}{3}\Rightarrow EJ=2\sqrt{3}$, $\dfrac{1}{I{H^2}}=\dfrac{1}{I{A^2}}+\dfrac{1}{I{E^2}}\Leftrightarrow\dfrac{1}{27}=\dfrac{1}{9r^2}+\dfrac{1}{108}\Leftrightarrow r=2$.\\
		Suy ra thể tích của thùng nước là $V_1=\dfrac{1}{3}\pi I{A^2}.IE-\dfrac{1}{3}\pi J{D^2}.JE=\dfrac{208\sqrt{3}\pi}{3}$.\\
		Vậy thể tích nước còn lại trong thùng là $ V=\dfrac{208\sqrt{3}\pi}{3}-54\sqrt{3}\pi=\dfrac{46\sqrt{3}\pi}{3}\left(\mathrm{dm}^3\right)$.
	}
\end{ex}
\begin{ex}
	[THPT Mai Anh Tuấn - Thanh Hóa - 2019]%Câu 15
	Cho tứ diện $ OABC$ có $ OA=a,OB=b,OC=c$ và đôi một vuông góc với nhau. Gọi $ r$ là bán kính mặt cầu tiếp xúc với cả bốn mặt của tứ diện. Giả sử $ a\ge b,a\ge c$. Giá trị nhỏ nhất của $\dfrac{a}{r}$ là
	\choice
	{$ 1+\sqrt{3}$}
	{$ 2+\sqrt{3}$}
	{$\sqrt{3}$}
	{\True $ 3+\sqrt{3}$}
	\loigiai{
		{\color{red}HÌNH Ở ĐÂY}\\
		Kẻ đường cao $ AH$ của tam giác $ ABC$.\\
		Dễ thấy $ OH\perp BC$ nên $\dfrac{1}{O{H^2}}=\dfrac{1}{O{B^2}}+\dfrac{1}{O{C^2}}\Rightarrow OH=\dfrac{bc}{\sqrt{b^2+c^2}}$.\\
		Tam giác $ AOH$ vuông tại $ O$ có $ A{H^2}=O{A^2}+O{H^2}\Rightarrow AH=\dfrac{\sqrt{a^2b^2+b^2c^2+c^2a^2}}{\sqrt{b^2+c^2}}$.\\
		Tam giác $ OBC$ có $ BC=\sqrt{b^2+c^2}$ nên $S_{ABC}=\dfrac{1}{2}AH.BC=\sqrt{a^2b^2+b^2c^2+c^2a^2}$.\\
		Vậy diện tích toàn phần của hình chóp $ O.ABC$l à $S_{\mathrm{tp}}=S_{OAB}+S_{OBC}+S_{OCA}+S_{ABC}=\dfrac{1}{2}\left(ab+bc+ca+\sqrt{a^2b^2+b^2c^2+c^2a^2}\right)$.\\
		Dễ thấy thể tích khối chóp $ O.ABC$ là $ V=\dfrac{1}{6}abc=\dfrac{1}{3}{S_{tp}}.r$.\\
		Suy ra\\
		$\dfrac{1}{6}abc=\dfrac{1}{3}{S_\mathrm{tp}}.r$ $\Rightarrow\dfrac{a}{r}=\dfrac{2S_{tp}}{bc}=\dfrac{ab+bc+ca+\sqrt{a^2b^2+b^2c^2+c^2a^2}}{bc}$\\
		$=\dfrac{a}{c}+1+\dfrac{a}{b}+\sqrt{\dfrac{a^2}{c^2}+1+\dfrac{a^2}{b^2}}\ge 1+1+1+\sqrt{1+1+1}=3+\sqrt{3}$.\\
		Dấu \lq\lq$=$\rq\rq xảy ra khi và chỉ khi $ a=b=c$.
	}
\end{ex}
\begin{ex}
	Cho hai mặt cầu $\left(S_1\right)$ và $\left(S_2\right)$ đồng tâm $ O$, có bán kính lần lượt là $R_1=2$ và $R_2=\sqrt{10}$. Xét tứ diện $ ABCD$ có hai đỉnh $ A,B$ nằm trên $\left(S_1\right)$ và hai đỉnh $ C,D$ nằm trên $\left(S_2\right)$. Thể tích lớn nhất của khối tứ diện $ ABCD$ bằng
	\choice
	{$ 3\sqrt{2}$}
	{$ 7\sqrt{2}$}
	{$ 4\sqrt{2}$}
	{\True $ 6\sqrt{2}$}
	\loigiai{
		{\color{red}HÌNH Ở ĐÂY}{\color{red}HÌNH Ở ĐÂY}\\
		Dựng mặt phẳng $(P)$ chứa $ AB$ và song song với $ CD$, cắt $\left(O;{R_1}\right)$ theo giao tuyến là đường tròn tâm $ I$.\\
		Dựng mặt phẳng $(Q)$ chứa $ CD$ và song song với $ AB$, cắt $\left(O;{R_2}\right)$ theo giao tuyến là đường tròn tâm $ J$.\\
		Dựng hai đường kính $A'{B}',C'{D}'$ lần lượt của hai đườn tròn sao cho $A'{B}'\bot{C}'{D}'$\\
		Khi đó $ IJ=d\left(AB;CD\right)=d\left(A'{B}';{C}'{D}'\right)$.\\
		Xét tất cả các tứ diện có cạnh $ AB$ nằm trên $(P)$ và $ CD$ nằm trên $(Q)$ thì ta có:\\
		$V_{ABCD}=\dfrac{1}{6}AB.CD.IJ.\sin\left(\widehat{AB,CD}\right)\le\dfrac{1}{6}{A}'{B}'.C'{D}'.IJ=V_{A'{B}'{C}'{D}'}$.\\
		Do đó ta chỉ cần xét các tứ diện có cặp cạnh đối $ AB\perp CD$ và chúng có trung điểm $ I,J$ thẳng hàng với $ O$.\\
		Đặt $ IA=x,\left(0<x\le\sqrt{10}\right),JC=y,\left(0<y\le 2\right)$, ta có: $ OI=\sqrt{10-x^2},OJ=\sqrt{4-y^2}$.\\
		Khi đó $ d\left(AB,CD\right)=IJ=OI+OJ=\sqrt{10-x^2}+\sqrt{4-y^2}$.\\
		Thể tích khối tứ diện $ ABCD$ là:\\
		$V_{ABCD}=\dfrac{1}{6}AB.CD.IJ=\dfrac{1}{6}.2x.2y.\left(\sqrt{10-x^2}+\sqrt{4-y^2}\right)=\dfrac{2}{3}xy\left(\sqrt{10-x^2}+\sqrt{4-y^2}\right)$\\
		Có $\sqrt{10-x^2}=\dfrac{1}{2}.2.\sqrt{10-x^2}\le\dfrac{14-x^2}{4};\sqrt{4-y^2}\le\dfrac{5-y^2}{2}$\\
		Suy ra $\sqrt{10-x^2}+\sqrt{4-y^2}\le\dfrac{24-x^2-2y^2}{4}\le\dfrac{24-2\sqrt{2}xy}{4}=\dfrac{12-\sqrt{2}xy}{2}$.\\
		Ta được $V_{ABCD}\le\dfrac{2}{3}xy.\dfrac{12-\sqrt{2}xy}{2}=\dfrac{1}{3\sqrt{2}}\left(\sqrt{2}xy\right)\left(12-\sqrt{2}xy\right)\le\dfrac{1}{3\sqrt{2}}{\left(\dfrac{\sqrt{2}xy+12-\sqrt{2}xy}{2}\right)^2}=6\sqrt{2}$.\\
		Đẳng thức xảy ra khi: $\left\{\begin{aligned}
			& 0<x\le\sqrt{10},0<y\le 2\\ 
			&\sqrt{10-x^2}=2\\ 
			&\sqrt{4-y^2}=1\\ 
			&{x^2}=2y^2\\ 
			&\sqrt{2}xy=12-\sqrt{2}xy\\ 
		\end{aligned}\right.\Leftrightarrow\left\{\begin{aligned}
			& x=\sqrt{6}\\ 
			& y=\sqrt{3}\\ 
		\end{aligned}\right.$\\
		Vậy $\max{V_{ABCD}}=6\sqrt{2}$.
	}
\end{ex}
\begin{ex}
	Cho tứ diện đều $ ABCD$ có mặt cầu nội tiếp là $\left(S_1\right)$ và mặt cầu ngoại tiếp là $\left(S_2\right)$, hình lập phương ngoại tiếp $\left(S_2\right)$ và nội tiếp trong mặt cầu $\left(S_3\right)$. Gọi $r_1$, $r_2$, $r_3$ lần lượt là bán kính các mặt cầu $\left(S_1\right)$, $\left(S_2\right)$, $\left(S_3\right)$. Khẳng định nào sau đây đúng?\\
	(Mặt cầu nội tiếp tứ diện là mặt cầu tiếp xúc với tất cả các mặt của tứ diện, mặt cầu nội tiếp hình lập phương là mặt cầu tiếp xúc với tất cả các mặt của hình lập phương).
	\choice
	{$\dfrac{r_1}{r_2}=\dfrac{1}{3}$ và $\dfrac{r_2}{r_3}=\dfrac{1}{3\sqrt{3}}$}
	{$\dfrac{r_1}{r_2}=\dfrac{2}{3}$ và $\dfrac{r_2}{r_3}=\dfrac{1}{\sqrt{3}}$}
	{\True $\dfrac{r_1}{r_2}=\dfrac{1}{3}$ và $\dfrac{r_2}{r_3}=\dfrac{1}{\sqrt{3}}$}
	{$\dfrac{r_1}{r_2}=\dfrac{2}{3}$ và $\dfrac{r_2}{r_3}=\dfrac{1}{\sqrt{2}}$}
	\loigiai{
		Giả sử tứ diện đều $ ABCD$ có cạnh bằng $ 1$. Khi đó, diện tích của mỗi mặt tứ diện đều là $\dfrac{\sqrt{3}}{4}$.\\
		Gọi $ H$ là tâm của tam giác đều $ BCD$ thì $ AH$ là đường cao của hình chóp $ A.BCD$ và $ BH=\dfrac{2}{3}.\dfrac{1\sqrt{3}}{2}=\dfrac{1}{\sqrt{3}}$.\\
		Do đó chiều cao của hình chóp là $ h=AH=\sqrt{A{B^2}-B{H^2}}=\sqrt{1^2-\left(\dfrac{1}{\sqrt{3}}\right)^2}=\dfrac{\sqrt{2}}{\sqrt{3}}$.\\
		Suy ra thể tích khối tứ diện $ ABCD$ là $ V=\dfrac{1}{3}{S_{BCD}}.h=\dfrac{1}{3}.\dfrac{\sqrt{3}}{4}.\dfrac{\sqrt{2}}{\sqrt{3}}=\dfrac{\sqrt{2}}{12}$.\\
		Bán kính mặt cầu $\left(S_1\right)$ nội tiếp diện đều $ ABCD$ là $r_1=\dfrac{3V}{4S_{BCD}}=\dfrac{3.\dfrac{\sqrt{2}}{12}}{4.\dfrac{\sqrt{3}}{4}}=\dfrac{\sqrt{2}}{4\sqrt{3}}$.\\
		Trong mặt phẳng $ ABH$, đường thẳng trung trực của $ AB$ cắt $ AH$ tại $ I$ thì $ I$ là tâm mặt cầu $\left(S_2\right)$ ngoại tiếp tứ diện đều $ ABCD$.\\
		Gọi $ M$ là trung điểm $ AB$, ta có $\dfrac{AI}{AB}=\dfrac{AM}{AH}$$\Rightarrow $$ AI=\dfrac{A{B^2}}{2AH}=\dfrac{1^2}{2.\dfrac{\sqrt{2}}{\sqrt{3}}}=\dfrac{\sqrt{3}}{2\sqrt{2}}$$\Rightarrow $$r_2=\dfrac{\sqrt{3}}{2\sqrt{2}}$.\\
		Độ dài cạnh hình lập phương ngoại tiếp $\left(S_2\right)$ bằng $ a=2r_2=\dfrac{\sqrt{6}}{2}$.\\
		Bán kính mặt cầu $\left(S_3\right)$ ngoại tiếp hình lập phương đó là $r_3=\dfrac{a\sqrt{3}}{2}=\dfrac{\sqrt{6}}{2}.\dfrac{\sqrt{3}}{2}=\dfrac{3\sqrt{2}}{4}$.\\
		Từ đó ta được $\dfrac{r_1}{r_2}=\dfrac{1}{3}$ và $\dfrac{r_2}{r_3}=\dfrac{1}{\sqrt{3}}$.
	}
\end{ex}
\begin{ex}
	[THPT Lương Văn Tụy - Ninh Bình - 2018]%Câu 18
	Cho hình chóp $S.ABCD$ có $\widehat{ABC}=\widehat{ADC}=90^\circ $, cạnh bên $SA$ vuông góc với $\left(ABCD\right)$, góc tạo bởi $SC$ và đáy $ABCD$ bằng $60^\circ $, $CD=a$ và tam giác $ADC$ có diện tích bằng $\dfrac{a^2\sqrt{3}}{2}$. Diện tích mặt cầu $S_{\mathrm{mc}}$ ngoại tiếp hình chóp $S.ABCD$ là
	\choice
	{\True $S_{\mathrm{mc}}=16\pi{a^2}$}
	{$S_{\mathrm{mc}}=4\pi{a^2}$}
	{$S_{\mathrm{mc}}=32\pi{a^2}$}
	{$S_{\mathrm{mc}}=8\pi{a^2}$}
	\loigiai{
		{\color{red}HÌNH Ở ĐÂY}\\
		Giả thiết $SA\perp\left(ABCD\right)$ $\Rightarrow $ $AC$ là hình chiếu của $SC$ lên $\left(ABCD\right)$.\\
		Do đó $\left(\widehat{SC,\left(ABCD\right)}\right)=\left(\widehat{SC,AC}\right)=\widehat{SCA}=60^\circ $.\\
		Xét tam giác $ADC$ vuông tại $D$, diện tích $S_{\Delta ADC}=\dfrac{1}{2}AD.DC=\dfrac{a^2\sqrt{3}}{2}$ $\Leftrightarrow AD=a\sqrt{3}$.\\
		Khi đó: $AC=\sqrt{A{D^2}+D{C^2}}$ $=\sqrt{\left(a\sqrt{3}\right)^2+a^2}=2a$.\\
		$\Delta SAC$ vuông tại $A$, ta có $\tan\widehat{SAC}=\dfrac{SA}{AC}$ $\Rightarrow SA=AC.\tan 60^\circ=2a\sqrt{3}$.\\
		Gọi $I$ là trung điểm $SC$ $(1)$, $H$ là trung điểm $AC$.\\
		Khi đó $IH\parallel SA$ $\Rightarrow IH\perp\left(ABCD\right)$.\\
		Tứ giác $ABCD$ có $\widehat{D}=\widehat{B}=90^\circ $, $H$ là trung điểm $AC$ nên $H$ là tâm đường tròn ngoại tiếp tứ giác $ABCD$ . Suy ra $IA=IB=IC=ID(2)$.\\
		Từ $(1)$ và $(2)$ suy ra $I$ là tâm mặt cầu ngoại tiếp hình chóp $S.ABCD$.\\
		Bán kính mặt cầu $R=\dfrac{1}{2}SC=\dfrac{1}{2}\sqrt{4a^2+12a^2}=2a$.\\
		Diện tích mặt cầu $S=4\pi{R^2}=16\pi{a^2}$.
	}
\end{ex}
\begin{ex}
	[Yên Phong 1 - 2018]%Câu 19
	Cho mặt cầu tâm $O$ bán kính $2a$, mặt phẳng $(\alpha)$ cố định cách $O$ một đoạn là $a$, $(\alpha)$ cắt mặt cầu theo đường tròn $(T)$. Trên $(T)$ lấy điểm $A$ cố định, một đường thẳng qua $A$ vuông góc với $(\alpha)$ cắt mặt cầu tại điểm $B$ khác $ A$. Trong $(\alpha)$ một góc vuông xAy quay quanh $A$ và cắt $(T)$ tại 2 điểm phân biệt $C$, $D$ không trùng với $A$. Khi đó chọn khẳng định đúng:
	\choice
	{Diện tích tam giác $BCD$ đạt giá trị nhỏ nhất là $a^2\sqrt{21}$}
	{\True Diện tích tam giác $BCD$ đạt giá trị lớn nhất là $a^2\sqrt{21}$}
	{Diện tích tam giác $BCD$ đạt giá trị nhỏ nhất là $ 2a^2\sqrt{21}$}
	{Do $(\alpha)$ không đi qua $O$ nên không tồn tại giá trị lớn nhất hay nhỏ nhất của diện tích tam giác $BCD$}
	\loigiai{
		{\color{red}HÌNH Ở ĐÂY}\\
		Gọi $I$ là tâm đường tròn thiết diện. Ta có $OI=a$, $OI\perp (\alpha)$, $IA=a\sqrt{3}$\\
		Do góc CAD vuông nên $CD$ là đường kính của đường tròn tâm $I$, $ CD=2a\sqrt{3}$\\
		Đặt $AD=x$, $AC=y$. Ta có $x^2+y^2=12a^2$ ($ 0<x,y<2a\sqrt{3}$)\\
		Gọi $H$ là hình chiếu của $A$ lên $CD$. Ta có $BH\perp CD$.\\ $S_{BCD}=\dfrac{1}{2}CD.BH=BH.a\sqrt{3}=a\sqrt{3}.\sqrt{A{B^2}+A{H^2}}$\\
		Ta có $OI$ và $AB$ đồng phẳng, gọi $E$ là trung điểm của $AB$, ta có $OE\perp AB$, tứ giác $OIAE$ là hình chữ nhật, $AB=2OI=2a$\\
		$S_{BCD}=a\sqrt{3}.\sqrt{4a^2+A{H^2}}$\\
		Ta có $\dfrac{1}{A{H^2}}=\dfrac{1}{x^2}+\dfrac{1}{y^2}\ge\dfrac{4}{x^2+y^2}=\dfrac{4}{12a^2}\Rightarrow A{H^2}\le 3a^2$ $\Rightarrow{S_{BCD}}\le a\sqrt{3}.\sqrt{4a^2+3a^2}=a^2\sqrt{21}$.\\
		Dấu bằng xảy ra khi $x=y$.
	}
\end{ex}
\begin{ex}
	[THPT Hải An - Hải Phòng - 2018]%Câu 20
	Trong tất cả các hình chóp tứ giác đều nội tiếp mặt cầu có bán kính bằng $ 9$, tính thể tích $ V$ của khối chóp có thể tích lớn nhất.
	\choice
	{$V=144$}
	{$V=576\sqrt{2}$}
	{\True $V=576$}
	{$V=144\sqrt{6}$}
	\loigiai{
		{\color{red}HÌNH Ở ĐÂY}{\color{red}HÌNH Ở ĐÂY}\\
		Gọi $ I$ là tâm mặt cầu và $ S.ABCD$ là hình chóp nội tiếp mặt cầu.\\
		Gọi $ x$ là độ dài cạnh $ SO$.\\
		Gọi $ M$ là trung điểm của $ SD$.\\
		Ta có $ SI.SO=SM.SD=\dfrac{1}{2}S{D^2}$$\Rightarrow S{D^2}=2SI.SO=18x$.\\
		Suy ra $ O{D^2}=18x-x^2$.\\
		Thể tích khối chóp $ S.ABCD$ bằng $ V=\dfrac{1}{3}SO.S_{ABCD}$$=\dfrac{1}{3}x.2.O{D^2}$$=\dfrac{2}{3}x\left(18x-x^2\right)$$=\dfrac{2}{3}{x^2}\left(18-x\right)$.\\
		Ta có $x^2\left(18-x\right)=$$ 4\dfrac{x}{2}.\dfrac{x}{2}.\left(18-x\right)$$\le 4\left(\dfrac{18}{3}\right)^3=864$.\\
		Vậy thể tích của khối chóp cần tìm là $ V=576$.
	}
\end{ex}
\begin{ex}
	[THPT Yên Khánh A - 2018]%Câu 21
	Cho hình chóp tứ giác đều chiều cao là $ h$ nội tiếp trong một mặt cầu bán kính $ R$. Tìm $ h$ theo $ R$ để thể tích khối chóp là lớn nhất.
	\choice
	{$ h=\sqrt{3}R$}
	{$ h=\sqrt{2}R$}
	{\True $ V=\dfrac{4R}{3}$}
	{$ V=\dfrac{3R}{2}$}
	\loigiai{
		{\color{red}HÌNH Ở ĐÂY}\\
		Gọi $ a$ là độ dài cạnh đáy của hình chóp tứ giác đều $S.ABCD$. Gọi $ O,I$ lần lượt là tâm đáy và tâm cầu ngoai tiếp hình chóp.\\
		Tam giác $ IBO$ có $\left(h-R\right)^2+\dfrac{a^2}{2}=R^2\Rightarrow\dfrac{a^2}{2}=R^2-\left(h-R\right)^2=2Rh-h^2$.\\
		Thể tích của khối chóp là: $V_{}=\dfrac{1}{3}{a^2}h=\dfrac{1}{3}2\left(2Rh-h^2\right).h$.\\
		Xét hàm số $ y=\left(2Rh-h^2\right).h$ với $ 0<h<2R$, $y'=4Rh-3h^2\Rightarrow{y}'=0\Rightarrow h=\dfrac{4R}{3}$.\\
		Trên $\left(0;2R\right)$, $y'$ đổi dấu từ \lq\lq$+$\rq\rq sang \lq\lq$-$\rq\rq qua $ h=\dfrac{4R}{3}$ nên thể tích hình chóp đạt lớn nhất tại $ h=\dfrac{4R}{3}$.
	}
\end{ex}
\begin{ex}
	[Sở Vĩnh Phúc - 2021]%Câu 22
	Cho mặt cầu $(S)$có bán kính không đổi là $ R$. Một hình chóp lục giác đều $ S.ABCDEF$ nội tiếp mặt cầu $(S)$. Tìm giá trị lớn nhất $V_{\max}$ của thể tích khối chóp $ S.ABCDEF$.
	\choice
	{$V_{\max}=\dfrac{3\sqrt{3}{R^3}}{8}$}
	{$V_{\max}=\dfrac{8\sqrt{3}{R^3}}{9}$}
	{\True $V_{\max}=\dfrac{16\sqrt{3}{R^3}}{27}$}
	{$V_{\max}=\dfrac{8\sqrt{3}{R^3}}{27}$}
	\loigiai{
		Ta có $V_{S.ABCDEF}=\dfrac{1}{3}d\left(S;\left(ABCDEF\right)\right).S_{ABCDEF}$ và mặt cầu có tính đối xứng nên để tìm $V_{\max}$ ta xét hình chóp $ S.ABCDEF$ như hình vẽ sau:\\
		{\color{red}HÌNH Ở ĐÂY}\\
		Đáy $ ABCDEF$ nội tiếp trong đường tròn tâm $ H$ bán kính $ r$ và tam giác $ HAB$ đều cạnh $ r=\sqrt{R^2-x^2}$. Đặt $ OH=x\left(0\le x<R\right)$\\
		Ta có $S_{ABCDEF}=6S_{HAB}=6.\dfrac{r^2\sqrt{3}}{4}=\dfrac{3\sqrt{3}\left(R^2-x^2\right)}{2}$.\\
		Khi đó\\
		$V_{S.ABCDEF}=\dfrac{1}{3}d\left(S;\left(ABCDEF\right)\right).S_{ABCDEF}=\dfrac{\sqrt{3}}{2}.\left(R+x\right)\left(R^2-x^2\right)=\dfrac{\sqrt{3}}{2}\left(-x^3-R{x^2}+R^2x+R^3\right)$\\
		Xét hàm số $ f(x)=-x^3-R{x^2}+R^2x+R^3$ với $ x\in\left[0;R\right)$\\
		$ f'(x)=-3x^2-2Rx+R^2;f'(x)=0\Leftrightarrow\left[\begin{aligned}
			& x=\dfrac{R}{3}\\ 
			& x=-R(l)\\ 
		\end{aligned}\right.$\\
		Ta có bẳng biến thiên\\
		{\color{red}HÌNH Ở ĐÂY}\\
		Vậy $V_{\max}=\dfrac{\sqrt{3}}{2}f\left(\dfrac{R}{3}\right)=\dfrac{16\sqrt{3}{R^3}}{27}$.
	}
\end{ex}
\begin{ex}
	[Liên trường Quỳnh Lưu - Hoàng Mai - Nghệ An - 2021]%Câu 23
	Cho hình chóp $S.ABC$ có đáy $ABC$ là tam giác đều cạnh bằng 1, mặt bên $SAB$ là tam giác cân tại $S$ và nằm trong mặt phẳng vuông góc với mặt phẳng đáy. Tính thể tích $V$ của khối cầu ngoại tiếp hình chóp đã cho biết $\widehat{ASB}=120^\circ$?
	\choice
	{$\dfrac{13\sqrt{78}\pi}{27}$}
	{\True $\dfrac{5\sqrt{15}\pi}{54}$}
	{$\dfrac{5\pi}{3}$}
	{$\dfrac{4\sqrt{3}\pi}{27}$}
	\loigiai{
		\textbf{VẼ HÌNH}\\
		Gọi $H$ là trung điểm của $A B: S H \perp A B$ (vì $\triangle S A B$ cân tại $S) \Rightarrow S H \perp(A B C)$
		$S H \perp C H$ và $C H \perp A B$ (vì $\triangle A B C$ đều) $\Rightarrow C H \perp(S A B)$.\\
		Gọi $I$ và $J$ lần lượt là tâm đường tròn ngoại tiếp $\triangle A B C$ và $\triangle S A B$.\\
		Qua $I$ và $J$ lần lượt kẻ đường thẳng $d$ song song với $S H$ và $d'$ song song với $C H$.\\
		$\Rightarrow d$ là trục của $\triangle A B C$ và $d'$ là trục của $\triangle S A B \Rightarrow$ Giao điểm của $d$ và $d'$ là tâm $O$ và $O S$ là bán kính của khối cầu ngoại tiếp hình chóp $S . A B C$.\\
		Áp dụng đính lí sin trong $\triangle S A B: \dfrac{A B}{\sin A S B}=2 R_{\triangle S A B}=2 J S \Leftrightarrow J S=\dfrac{1}{2 \sin 120^{\circ}}=\dfrac{\sqrt{3}}{3}$\\
		Xét $\triangle A B C$ đều: $C H=\dfrac{\sqrt{3}}{2}; I H=\dfrac{1}{3}C H=\dfrac{\sqrt{3}}{6}$\\
		Xét $\triangle O J S$ vuông tại $J: O S=\sqrt{O J^2+J S^2}=\sqrt{I H^2+J S^2}=\sqrt{\left(\dfrac{\sqrt{3}}{6}\right)^2+\left(\dfrac{\sqrt{3}}{3}\right)^2}=\dfrac{\sqrt{15}}{6}$
		$V=\dfrac{4}{3}\pi \cdot O S^3=\dfrac{4}{3}\pi \cdot\left(\dfrac{\sqrt{15}}{6}\right)^3=\dfrac{5 \sqrt{15}\pi}{54}$.
	}
\end{ex}            
\begin{ex}
	[THPT Mai Anh Tuấn - Thanh Hóa - 2021]%Câu 24
	Cho hình lăng trụ đều $ABC.A'B'C'$, biết góc giữa hai mặt phẳng $\left(A'BC\right)$ và $\left(ABC\right)$ bằng $45^0$, diện tích tam giác $A'BC$ bằng $a^2\sqrt{6}$ . Tính diện tích xung quanh của hình trụ ngoại tiếp hình lăng trụ $ABC.A'B'C'$.
	\choice
	{$ 2\pi{a^2}$}
	{$\dfrac{8\pi{a^2}\sqrt{3}}{3}$}
	{\True $ 4\pi{a^2}$}
	{$\dfrac{4\pi{a^2}\sqrt{3}}{3}$}
	\loigiai{
		{\color{red}HÌNH Ở ĐÂY}\\
		Gọi $I$ là trung điểm của $BC\Rightarrow AI\perp BC,A'I\perp BC.$\\
		Ta có $\left\{\begin{aligned}
			&\left(A'BC\right)\cap\left(ABC\right)=BC\\
			&A'I\subset\left(A'BC\right),A'I\perp BC\\
			&AI\subset\left(ABC\right),AI\perp BC\\
		\end{aligned}\right.\Rightarrow\widehat{\left(\left(A'BC\right),\left(ABC\right)\right)}=\widehat{\left(A'I,AI\right)}=\widehat{A'IA}=45^0$.\\
		Do $A'A\perp\left(ABC\right)$ tại $A$, suy ra tam giác $A'BC$ có hình chiếu vuông góc lên mặt đáy $\left(ABC\right)$ là tam giác $ABC$.\\
		Áp dụng công thức tính diện tích hình chiếu của một đa giác, ta được\\
		$S_{ABC}=S_{A'BC}.{cos}\widehat{\left(\left(A'BC\right),\left(ABC\right)\right)}=a^2\sqrt{6}.{cos}{45^0}=a^2\sqrt{3}.$\\
		Tam giác $ABC$ đều $\Rightarrow{S_{ABC}}=\dfrac{A{B^2}\sqrt{3}}{4}\Leftrightarrow\dfrac{A{B^2}\sqrt{3}}{4}=a^2\sqrt{3}\Leftrightarrow A{B^2}=4a^2\Leftrightarrow AB=2a.$\\
		Suy ra bán kính đường tròn ngoại tiếp tam giác $ABC$ là\\
		$r=\dfrac{2}{3}AI=\dfrac{2}{3}\left(\dfrac{AB\sqrt{3}}{2}\right)=\dfrac{2}{3}\left(\dfrac{2a\sqrt{3}}{2}\right)=\dfrac{2a\sqrt{3}}{3}.$\\
		Xét tam giác vuông cân $A'IA\Rightarrow A'A=AI=a\sqrt{3}.$\\
		Vậy diện tích xung quanh của hình trụ ngoại tiếp hình lăng trụ $ABC.A'B'C'$ bằng\\
		$S_{\mathrm{xq}}=2\pi rl=2\pi .\dfrac{2a\sqrt{3}}{3}.a\sqrt{3}=4\pi{a^2}$.
	}
\end{ex}
\begin{ex}
	Ông An cần làm một đồ trang trí như hình vẽ. Phần dưới là một phần của khối cầu bán kính $ 20\mathrm{cm}$ làm bằng gỗ đặc, bán kính của đường tròn phần chỏm cầu bằng $ 10\mathrm{cm}$. Phần phía trên làm bằng lớp vỏ kính trong suốt. Biết giá tiền của $ 1\mathrm{m}^2$ kính như trên là 1.500.000 đồng, giá triền của $ 1\mathrm{m}^3$ gỗ là 100.000.000 đồng. Hỏi số tiền (làm tròn đến hàng nghìn) mà ông An mua vật liệu để làm đồ trang trí là bao nhiêu.\\
	{\color{red}HÌNH Ở ĐÂY}
	\choice
	{$ 1.000.000$}
	{$ 1.100.000$}
	{$ 1.010.000$}
	{\True $ 1.005.000$}
	\loigiai{
		Bán kính mặt cầu là $ R=20\mathrm{cm}$; bán kính đường tròn phần chỏm cầu là $ r=10\mathrm{cm}$.\\
		Theo hình vẽ ta có $\sin\alpha=\dfrac{10}{20}=\dfrac{1}{2}\Rightarrow\alpha=30^0$.\\
		Diện tích phần làm kính là $ S=\dfrac{360-2.30}{360}.4\pi{20^2}=\dfrac{4000\pi}{3}\left(\mathrm{cm}^2\right)$.\\
		Xét hình nón đỉnh là tâm mặt cầu, hình tròn đáy có bán kính bằng $ r=10\mathrm{cm}$; $l=R=20\mathrm{cm}\Rightarrow h=\sqrt{20^2-10^2}=10\sqrt{3}\mathrm{cm}$\\
		Thể tích phần chỏm cầu bằng\\
		$V_\text{chỏm cầu}=\dfrac{2.30}{360}.\dfrac{4}{3}\pi{R^3}-\dfrac{1}{3}\pi{r^2}.h$=$\dfrac{16000\pi}{9}-\dfrac{1000\pi\sqrt{3}}{3}\left(\mathrm{cm}^3\right)$\\
		Vậy số tiền ông An cần mua vật liệu là $\dfrac{4000\pi}{3}.150+\left(\dfrac{16000\pi}{9}-\dfrac{1000\pi\sqrt{3}}{3}\right).100\approx 1.005.000$.
	}
\end{ex}
\begin{ex}
	[Sở Bắc Ninh - 2021]%Câu 26
	Cho hình chóp tam giác đều $S.ABC$ có cạnh đáy bằng $\sqrt{3}$ và cạnh bên bằng $x$, với $x>1$. Gọi $V$ là thể tích khối cầu xác định bởi mặt cầu ngoại tiếp hình chóp $S.ABC$. Giá trị nhỏ nhất của $V$ thuộc khoảng nào sau đây?
	\choice
	{$\left(7;3\pi\right)$}
	{$\left(0;1\right)$}
	{\True $\left(1;5\right)$}
	{$\left(5;7\right)$}
	\loigiai{
		{\color{red}HÌNH Ở ĐÂY}\\     
		Gọi $M$ là trung điểm của $BC$, $G$ là trong tâm tam giác $ABC\Rightarrow SG\perp\left(ABC\right)$.\\
		Trong mặt phẳng $\left(SAM\right)$, đường trung trực của đoạn $SA$ cắt $SG$ tại $I\Rightarrow IA=IB=IC=IS$ là tâm mặt cầu ngoại tiếp khối chóp $S.ABC$.\\
		$AM=\dfrac{3}{2};AG=\dfrac{2}{3}AM=1\Rightarrow SG=\sqrt{x^2-1}$\\
		Hai tam giác $SKI$ và $SGA$ đông dạng $\Rightarrow\dfrac{SI}{SA}=\dfrac{SK}{SG}\Rightarrow SI=x.\dfrac{\frac{x}{2}}{\sqrt{x^2-1}}=\dfrac{x^2}{2\sqrt{x^2-1}}\Rightarrow R=\dfrac{x^2}{2\sqrt{x^2-1}}$.\\
		Thể tích khối cầu: $V=\dfrac{4}{3}\pi{R^3}=\dfrac{1}{6}\pi\dfrac{x^6}{\left(x^2-1\right)\sqrt{x^2-1}}$\\
		Đặt $t=\sqrt{x^2-1}\Rightarrow{x^2}=t^2+1\Rightarrow{x^6}=\left(t^2+1\right)^3=t^6+3t^4+3t^2+1$\\
		$\dfrac{x^6}{\left(x^2-1\right)\sqrt{x^2-1}}=\dfrac{t^6+3t^4+3t^2+1}{t^3}=t^3+3t+\dfrac{3}{t}+\dfrac{1}{t^3}=\left(t+\dfrac{1}{t}\right)^3\ge{\left(2.\sqrt{t.\dfrac{1}{t}}\right)^3}=8$.\\
		Dấu \lq\lq$=$\rq\rq xảy ra khi $t=1\Rightarrow x=\sqrt{2}$.\\
		Do đó, giá trị nhỏ nhất của thể tích khối chóp $S.ABC$ là $V=\dfrac{4}{3}\pi $.
	}
\end{ex}
\Closesolutionfile{ans}
\indapan{10}{ans/CD23/Muc_9_10}