\Opensolutionfile{ans}[ans/CD23/Muc_5_6]
\setcounter{ex}{0}
\setcounter{dang}{0}
\section{Mức độ 5,6 điểm}
\begin{dang}
	{Diện tích xung quanh, bán kính}
\end{dang}
\begin{ex}
	[Đề Tham Khảo 2020 Lần 2]%Câu 1
	Cho mặt cầu có bán kính $R=2$. Diện tích của mặt cầu đã cho bằng
	\choice
	{$\dfrac{32\pi}{3}$}
	{$8\pi $}
	{\True $16\pi $}
	{$4\pi $}
	\loigiai{
		$S=4\pi{R^2}=16\pi $.
	}
\end{ex}
\begin{ex}
	[Mã 102 - 2020 Lần 2]%Câu 2
	Cho mặt cầu có bán kính $r=5$. Diện tích mặt cầu đã cho bằng
	\choice
	{$25\pi $}
	{$\dfrac{500\pi}{3}$}
	{\True $100\pi $}
	{$\dfrac{100\pi}{3}$}
	\loigiai{}{
		Diện tích mặt cầu $S=4\pi{r^2}=4\pi{5^2}=100\pi$.
	}
\end{ex}
\begin{ex}
	[Mã 103 - 2020 Lần 2]%Câu 3
	Cho mặt cầu có bán kính $r=4$. Diện tích của mặt cầu đã cho bằng
	\choice
	{$16\pi $}
	{\True $64\pi $}
	{$\dfrac{64\pi}{3}$}
	{$\dfrac{256\pi}{3}$}
	\loigiai{
		Diện tích của mặt cầu bằng $4\pi{r^2}=4.\pi{4^2}=64\pi$.
	}
\end{ex}
\begin{ex}
	[Mã 104 - 2020 Lần 2]%Câu 4
	Cho mặt cầu bán kính $r=5$. Diện tích của mặt cầu đã cho bằng
	\choice
	{$\dfrac{500\pi}{3}$}
	{$25\pi $}
	{$\dfrac{100\pi}{3}$}
	{\True $100\pi $}
	\loigiai{
		Diện tích của mặt cầu có bán kính $r=5$ là $S=4\pi{r^2}=4\pi{5^2}=100\pi $.
	}
\end{ex}
\begin{ex}
	[Mã 101 2018]%Câu 5
	Diện tích của mặt cầu bán kính $R$ bằng:
	\choice
	{$\pi{R^2}$}
	{$\dfrac{4}{3}\pi{R^2}$}
	{$2\pi{R^2}$}
	{\True $4\pi{R^2}$}
	\loigiai{}
\end{ex}
\begin{ex}
	[THPT Thiệu Hóa – Thanh Hóa 2019]%Câu 6
	Cho mặt cầu có diện tích bằng $16\pi{a^2}$. Khi đó, bán kính mặt cầu bằng
	\choice
	{$2\sqrt{2}a$}
	{$\sqrt{2}a$}
	{\True $2a$}
	{$\dfrac{a\sqrt{2}}{2}$}
	\loigiai{
		Ta có $S=4\pi{R^2}=16\pi{a^2}$ $\Rightarrow R=2a$.
	}
\end{ex}
\begin{ex}
	[Chuyên Đhsp Hà Nội 2019]%Câu 7
	Diện tích mặt cầu bán kính $2a$ là
	\choice
	{$4\pi{a^2}$}
	{\True $16\pi{a^2}$}
	{$16a^2$}
	{$\dfrac{4\pi{a^2}}{3}$}
	\loigiai{
		Ta có $S=4\pi{R^2}=4\pi{\left(2a\right)^2}=16\pi{a^2}$.
	}
\end{ex}
\begin{ex}
	[THPT Nghĩa Hưng Nđ - 2019]%Câu 8
	Diện tích của một mặt cầu bằng $16\pi\left(\mathrm{cm}^2\right)$. Bán kính của mặt cầu đó là.
	\choice
	{$8cm$}
	{\True $2cm$}
	{$4cm$}
	{$6cm$}
	\loigiai{
		Ta có $4\pi{R^2}=16\pi\Leftrightarrow{R^2}=4\Rightarrow R=2(\mathrm{cm})$.
	}
\end{ex}
\begin{ex}
	[Bình Phước 2019]%Câu 9
	Tính diện tích mặt cầu $(S)$ khi biết chu vi đường tròn lớn của nó bằng $4\pi $ 
	\choice
	{$S=32\pi $}
	{\True $S=16\pi $}
	{$S=64\pi $}
	{$S=8\pi $}
	\loigiai{
		Nhận xét: Đường tròn lớn của mặt cầu $(S)$ là đường tròn đi qua tâm của mặt cầu $(S)$ nên bán kính của đường tròn lớn cũng là bán kính của mặt cầu $(S)$.\\
		Chu vi đường tròn lớn của mặt cầu $(S)$ bằng $4\pi \Rightarrow 2\pi R=4\pi\Leftrightarrow R=2$.\\
		Vậy diện tích mặt cầu $(S)$ là $S=4\pi{R^2}=16\pi $.
	}
\end{ex}
\begin{ex}
	[Trường THPT Thăng Long 2019]%Câu 10
	Một mặt cầu có diện tích xung quanh là $\pi $ thì có bán kính bằng
	\choice
	{$\dfrac{\sqrt{3}}{2}$}
	{$\sqrt{3}$}
	{\True $\dfrac{1}{2}$}
	{$1$}
	\loigiai{
		$S_{\mathrm{mc}}=4\pi{R^2}=\pi\Rightarrow R=\dfrac{1}{2}$.
	}
\end{ex}
\begin{ex}
	[THPT Cẩm Bình 2019]%Câu 11
	Diện tích mặt cầu có đường kính bằng $2a$ là
	\choice
	{$16\pi{a^2}$}
	{$\pi{a^2}$}
	{$\dfrac{4\pi{a^3}}{3}$}
	{\True $4\pi{a^2}$}
	\loigiai{
		Bán kính mặt cầu là $R=a\Rightarrow $ Diện tích mặt cầu là $S=4\pi{R^2}=4\pi{a^2}$.
	}
\end{ex}
\begin{ex}
	[Chuyên Lê Hồng Phong - Nam Định - 2019]%Câu 12
	Cho mặt cầu có diện tích bằng $\dfrac{8\pi{a^2}}{3}$ . Bán kính mặt cầu bằng
	\choice
	{\True $\dfrac{a\sqrt{6}}{3}$}
	{$\dfrac{a\sqrt{3}}{3}$}
	{$\dfrac{a\sqrt{2}}{3}$}
	{$\dfrac{a\sqrt{6}}{2}$}
	\loigiai{
		Ta có diện tích mặt cầu $S=4\pi{r^2}\Rightarrow r=\sqrt{\dfrac{S}{4\pi}}=\sqrt{\dfrac{8\pi{a^2}}{3.4\pi}}=\dfrac{a\sqrt{6}}{3}$.
	}
\end{ex}
\begin{ex}
	[Chuyên Lê Quý Đôn Quảng Trị 2019]%Câu 13
	Quả bóng rổ size 7 có đường kính $24,5$ $\mathrm{cm}$. Tính diện tích bề mặt quả bóng rổ đó (làm tròn kết quả đến chữ số hàng đơn vị)
	\choice
	{$629$ $\mathrm{cm}^2$}
	{\True $1886$ $\mathrm{cm}^2$}
	{$8171$ $\mathrm{cm}^2$}
	{$7700$ $\mathrm{cm}^2$}
	\loigiai{
		Ta có bán kính quả bóng rổ là $r=\dfrac{24.5}{2}=12.25(\mathrm{cm)}$.\\
		Vậy diện tích bề mặt quả bóng rổ đó là $S=4\pi{r^2}=4\pi .(12.25)^2\approx 1886(\mathrm{cm}^2)$.
	}
\end{ex}
\begin{ex}
	[SGD Bình Phước - 2019]%Câu 14
	Tính diện tích mặt cầu $(S)$ khi biết chu vi đường tròn lớn của nó bằng $4\pi $ 
	\choice
	{$S=32\pi $}
	{\True $S=16\pi $}
	{$S=64\pi $}
	{$S=8\pi $}
	\loigiai{
		Nhận xét: Đường tròn lớn của mặt cầu $(S)$ là đường tròn đi qua tâm của mặt cầu $(S)$ nên bán kính của đường tròn lớn cũng là bán kính của mặt cầu $(S)$.\\
		Chu vi đường tròn lớn của mặt cầu $(S)$ bằng $4\pi $ $\Rightarrow 2\pi R=4\pi\Leftrightarrow R=2$ .\\
		Vậy diện tích mặt cầu $(S)$ là $S=4\pi{R^2}=16\pi $.
	}
\end{ex}
\begin{ex}
	[Mã 103 - 2022]%Câu 15
	Cho điểm $M$ nằm ngoài mặt cầu $S\left(O;R\right)$. Khẳng định nào dưới đây đúng?
	\choice
	{$OM\le R$}
	{\True $OM>R$}
	{$OM=R$}
	{$OM<R$}
	\loigiai{}
\end{ex}
\begin{dang}
	{Thể tích}
\end{dang}
\begin{ex}
	[Mã 101 - 2020 Lần 1]%Câu 16
	Cho khối cầu có bán kính $r=4$. Thể tích của khối cầu đã cho bằng
	\choice
	{\True $\dfrac{256\pi}{3}$}
	{$64\pi $}
	{$\dfrac{64\pi}{3}$}
	{$256\pi $}
	\loigiai{
		Thể tích của khối cầu $V=\dfrac{4}{3}\pi{r^3}=\dfrac{256\pi}{3}$.
	}
\end{ex}
\begin{ex}
	[Mã 102 - 2020 Lần 1]%Câu 17
	Cho khối cầu có bán kính $r=4$. Thể tích của khối cầu đã cho bằng
	\choice
	{$64 \pi$}
	{$\dfrac{64 \pi}{3}$}
	{$256 \pi$}
	{\True $\dfrac{256 \pi}{3}$}
	\loigiai{
		Thể tích của khối cầu đã cho bằng $V=\dfrac{4}{2}\pi R^3=\dfrac{4}{2}\pi \cdot 4^3=\dfrac{256 \pi}{2}$.
	}
\end{ex}
\begin{ex}%Câu 18
	(Mã 103 - 2020 Lần 1) Cho khối cầu có bán kính $r=2$. Thể tích của khối cầu đã cho bằng
	\choice
	{$16\pi $}
	{\True $\dfrac{32\pi}{3}$}
	{$32\pi $}
	{$\dfrac{8\pi}{3}$}
	\loigiai{
		Thể tích của khối cầu đã cho: $V=\dfrac{4}{3}\pi{r^3}=\dfrac{4}{3}\pi{2^3}=\dfrac{32}{3}\pi $.
	}
\end{ex}
\begin{ex}
	[Mã 104 - 2020 Lần 1]%Câu 19
	Cho khối cầu có bán kính $r=2$. Thể tích của khối cầu bằng
	\choice
	{\True $\dfrac{32\pi}{3}$}
	{$16\pi $}
	{$32\pi $}
	{$\dfrac{8\pi}{3}$}
	\loigiai{
		Ta có $V=\dfrac{4}{3}\pi{r^3}=\dfrac{4}{3}\pi{2^3}=\dfrac{32}{3}\pi $.
	}
\end{ex}
\begin{ex}
	[Mã 102 2018]%Câu 20
	Thể tích của khối cầu bán kính $R$ bằng
	\choice
	{$\dfrac{3}{4}\pi{R^3}$}
	{\True $\dfrac{4}{3}\pi{R^3}$}
	{$4\pi{R^3}$}
	{$2\pi{R^3}$}
	\loigiai{}
\end{ex}
\begin{ex}
	[Đề Tham Khảo 2019]%Câu 21
	Thể tích khối cầu bán kính $a$ bằng
	\choice
	{$\dfrac{\pi{a^3}}{3}$}
	{$2\pi{a^3}$}
	{\True $\dfrac{4\pi{a^3}}{3}$}
	{$4\pi{a^3}$}
	\loigiai{}
\end{ex}
\begin{ex}
	[Lômônôxốp - Hà Nội 2019]%Câu 22
	Thể tích của khối cầu có bán kính là 1 bằng
	\choice
	{$2\pi $}
	{$\dfrac{\pi}{3}$}
	{\True $\dfrac{4\pi}{3}$}
	{$4\pi $}
	\loigiai{
		Thể tích của khối cầu: $V=\dfrac{4}{3}\pi{R^3}=\dfrac{4}{3}\pi $.
	}
\end{ex}
\begin{ex}
	[SP Đồng Nai - 2019]%Câu 23
	Thể tích khối cầu có đường kính $2a$ bằng
	\choice
	{\True $\dfrac{4\pi{a^3}}{3}$}
	{$4\pi{a^3}$}
	{$\dfrac{\pi{a^3}}{3}$}
	{$2\pi{a^3}$}
	\loigiai{
		Đường kính của khối cầu là $2a$, nên bán kính của nó là $a$, thể tích khối cầu là $\dfrac{4\pi{a^3}}{3}$.
	}
\end{ex}
\begin{ex}
	[THPT Đông Sơn Thanh Hóa 2019]%Câu 24
	Thể tích khối cầu bán kính $3\mathrm{cm}$ bằng
	\choice
	{\True $36\pi\left(\mathrm{cm}^3\right)$}
	{$108\pi\left(\mathrm{cm}^3\right)$}
	{$9\pi\left(\mathrm{cm}^3\right)$}
	{$54\pi\left(\mathrm{cm}^3\right)$}
	\loigiai{
		Thể tích khối cầu là $V=\dfrac{4}{3}.\pi .R^3=\dfrac{4}{3}.\pi{3^3}=36\pi\left(\mathrm{cm}^3\right)$.
	}
\end{ex}
\begin{ex}
	[THPT Lê Xoay Vĩnh Phúc 2019]%Câu 25
	Cho mặt cầu $(S)$ có diện tích $4\pi{a^2}\left(\mathrm{cm}^2\right)$. Khi đó, thể tích khối cầu $(S)$ là
	\choice
	{\True $\dfrac{4\pi{a^3}}{3}\left(\mathrm{cm}^3\right)$}
	{$\dfrac{\pi{a^3}}{3}\left(\mathrm{cm}^3\right)$}
	{$\dfrac{64\pi{a^3}}{3}\left(\mathrm{cm}^3\right)$}
	{$\dfrac{16\pi{a^3}}{3}\left(\mathrm{cm}^3\right)$}
	\loigiai{
		Gọi mặt cầu có bán kính $R$. Theo đề ta có $4\pi{R^2}=4\pi{a^2}$. Vậy $R=a(\mathrm{cm})$.\\
		Khi đó, thể tích khối cầu $(S)$ là $V=\dfrac{4\pi{R^3}}{3}=\dfrac{4\pi{a^3}}{3}\left(\mathrm{cm}^3\right)$.
	}
\end{ex}
\begin{ex}
	[Chuyên Phan Bội Châu Nghệ An 2019]%Câu 26
	Cho mặt cầu có diện tích bằng $36\pi{a^2}$. Thể tich khối cầu là
	\choice
	{$18\pi{a^3}$}
	{$12\pi{a^3}$}
	{\True $36\pi{a^3}$}
	{$9\pi{a^3}$}
	\loigiai{
		Gọi $R$ là bán kính mặt cầu.\\
		Mặt cầu có diện tích bằng $36\pi{a^2}$ nên $4\pi{R^2}=36\pi{a^2}\Leftrightarrow{R^2}=9a^2\Rightarrow R=3a$\\
		Thể tích khối cầu là $V=\dfrac{4}{3}\pi{R^3}=\dfrac{4}{3}\pi{(3a)^3}=36\pi{a^3}$.
	}
\end{ex}
\begin{ex}
	[THPT Đoàn Thượng – Hải Dương 2019]%Câu 27
	Tính diện tích $S$ của mặt cầu và thể tích $V$ của khối cầu có bán kính bằng $3\mathrm{cm}$.
	\choice
	{\True $S=36\pi $ $\left(\mathrm{cm}^2\right)$ và $V=36\pi $ $\left(\mathrm{cm}^3\right)$}
	{$S=18\pi $ $\left(\mathrm{cm}^2\right)$ và $V=108\pi $ $\left(\mathrm{cm}^3\right)$}
	{$S=36\pi $ $\left(\mathrm{cm}^2\right)$ và $V=108\pi $ $\left(\mathrm{cm}^3\right)$}
	{$S=18\pi $ $\left(\mathrm{cm}^2\right)$ và $V=36\pi $ $\left(\mathrm{cm}^3\right)$}
	\loigiai{
		Mặt cầu bán kính $r$ có diện tích là $S=4\pi{r^2}=4\pi{3^2}=36\pi $ $\left(\mathrm{cm}^2\right)$.\\
		Khối cầu bán kính $r$ có thể tích là $V=\dfrac{4}{3}\pi{r^3}=\dfrac{4}{3}\pi{3^3}=36\pi $ $\left(\mathrm{cm}^3\right)$.
	}
\end{ex}
\begin{ex}%Câu 28
	(KSCL Sở Hà Nam - 2019) Thể tích của khối cầu bán kính $3a$ là
	\choice
	{$4\pi{a^3}$}
	{$12\pi{a^3}$}
	{$36\pi{a^2}$}
	{\True $36\pi{a^3}$}
	\loigiai{
		Bán kính khối cầu: $R=3a$.\\
		Thể tích của khối cầu: $V=\dfrac{4\pi{R^3}}{3}=\dfrac{4\pi{\left(3a\right)^3}}{3}=36\pi{a^3}$.
	}
\end{ex}
\begin{ex}
	[THPT Phan Bội Châu - Nghệ An - 2019]%Câu 29
	Cho mặt cầu có diện tích bằng $36\pi{a^2}$. Thể tich khối cầu là
	\choice
	{$18\pi{a^3}$}
	{$12\pi{a^3}$}
	{\True $36\pi{a^3}$}
	{$9\pi{a^3}$}
	\loigiai{
		Gọi $R$ là bán kính mặt cầu.\\
		Mặt cầu có diện tích bằng $36\pi{a^2}$ nên $4\pi{R^2}=36\pi{a^2}\Leftrightarrow{R^2}=9a^2\Rightarrow R=3a$.\\
		Thể tích khối cầu là $V=\dfrac{4}{3}\pi{R^3}=\dfrac{4}{3}\pi{(3a)^3}=36\pi{a^3}$.
	}
\end{ex}
\begin{ex}
	[Mã 111 - 2021 - Lần 2]%Câu 30
	Thể tích của khối cầu bán kính $4a$ bằng:
	\choice
	{\True $\dfrac{256}{3}\pi{a^3}$}
	{$64\pi{a^3}$}
	{$\dfrac{4}{3}\pi{a^3}$}
	{$\dfrac{64}{3}\pi{a^3}$}
	\loigiai{
		Ta có $V=\dfrac{4}{3}\pi{R^3}=\dfrac{4}{3}\pi{\left(4a\right)^3}=\dfrac{256}{3}\pi{a^3}$.
	}
\end{ex}
\begin{ex}
	[Mã 120 - 2021 - Lần 2]%Câu 31
	Thể tích của khối cầu bán kính $2a$ bằng
	\choice
	{\True $\dfrac{32}{3}\pi{a^3}$}
	{$8\pi{a^3}$}
	{$\dfrac{4}{3}\pi{a^3}$}
	{$\dfrac{8}{3}\pi{a^3}$}
	\loigiai{
		Theo công thức tính thể tích khối cầu ta có $V=\dfrac{4}{3}\pi{R^3}=\dfrac{4}{3}\pi{\left(2a\right)^3}=\dfrac{32}{3}\pi{a^3}$.
	}
\end{ex}
\begin{ex}[Đề minh họa 2022]%Câu 32
	Thể tích $V$ thị của khối cầu bán kính $r$ được tính theo công thức nào dưới đây?
	\choice
	{$V=\dfrac{1}{3}\pi{r^3}$}
	{$V=2\pi{r^3}$}
	{$V=4\pi{r^3}$}
	{\True $V=\dfrac{4}{3}\pi{r^3}$}
	\loigiai{
		Xét đáp án A: là công thức thể tích của hình chóp không phải thể tích khối cầu, nên loại.\\
		Xét đáp án B: không phải thể tích khối cầu, nên loại.\\
		Xét đáp án B: không phải thể tích khối cầu, nên loại.\\
		Xét đáp án B: đúng là thể tích khối cầu, nên nhận.
	}
\end{ex}
\begin{dang}
	{Khối cầu nội tiếp, ngoại tiếp khối lăng trụ}
\end{dang}
\begin{ex}
	[Mã 123 2017]%Câu 33
	Tìm bán kính$R$ mặt cầu ngoại tiếp một hình lập phương có cạnh bằng $2a$. 
	\choice
	{\True $R=\sqrt{3}a$}
	{$R=a$}
	{$100$}
	{$R=2\sqrt{3}a$}
	\loigiai{
		{\color{red}HÌNH Ở ĐÂY}\\
		Đường chéo của hình lập phương: $A{C}'=2\sqrt{3}a$. Bán kính $R=\dfrac{A{C}'}{2}=a\sqrt{3}$.
	}
\end{ex}
\begin{ex}
	[Mã 110 2017]%Câu 34
	Cho mặt cầu bán kính $R$ ngoại tiếp một hình lập phương cạnh $a$. Mệnh đề nào dưới đây đúng?
	\choice
	{$a=\dfrac{\sqrt{3}R}{3}$}
	{\True $a=\dfrac{2\sqrt{3}R}{3}$}
	{$a=2R$}
	{$a=2\sqrt{3}R$}
	\loigiai{
		{\color{red}HÌNH Ở ĐÂY}\\
		Gọi $O=A{C}'\cap{A}'C$ $\Rightarrow O$ là tâm mặt cầu ngoại tiếp hình lập phương.\\
		Bán kính mặt cầu: $R=OA=\dfrac{1}{2}A{C}'=\dfrac{a\sqrt{3}}{2}\Rightarrow a=\dfrac{2R}{\sqrt{3}}=\dfrac{2\sqrt{3}R}{3}$.
	}
\end{ex}
\begin{ex}
	[Chuyên Đại Học Vinh 2019]%Câu 35
	Cho hình hộp chữ nhật $ABCD.A'B'C'D'$ có $AB=a$, $AD=AA'=2a$. Diện tích của mặt cầu ngoại tiếp của hình hộp chữ nhật đã cho bằng
	\choice
	{\True $9\pi{a^2}$}
	{$\dfrac{3\pi{a^2}}{4}$}
	{$\dfrac{9\pi{a^2}}{4}$}
	{$3\pi{a^2}$}
	\loigiai{
		{\color{red}HÌNH Ở ĐÂY}\\
		Bán kính khối cầu là một nửa đường chéo của hình hộp chữ nhật:\\
		$R=\dfrac{1}{2}\sqrt{A{B^2}+A{D^2}+BB{'^2}}=\dfrac{1}{2}\sqrt{a^2+(2a)^2+(2a)^2}=\dfrac{3}{2}a$.\\
		Diện tích mặt cầu ngoại tiếp hình hộp chữ nhật là\\
		$S=4\pi{R^2}=4\pi{\left(\dfrac{3a}{2}\right)^2}=9\pi{a^2}$.
	}
\end{ex}
\begin{ex}
	[Chuyên Lê Hồng Phong Nam Định 2019]%Câu 36
	Thể tích khối cầu ngoại tiếp hình hộp chữ nhật có ba kích thước $1$, $2$, $3$ là
	\choice
	{$36\pi $}
	{$\dfrac{9\pi}{2}$}
	{\True $\dfrac{7\pi\sqrt{14}}{3}$}
	{$\dfrac{9\pi}{8}$}
	\loigiai{
		{\color{red}HÌNH Ở ĐÂY}\\
		Gọi $R$ là bán kính khối cầu ngoại tiếp hình hộp chữ nhật.\\
		Ta có $R=\dfrac{1}{2}B{D}'$ $=\dfrac{1}{2}\sqrt{1^2+2^2+3^2}$ $=\dfrac{\sqrt{14}}{2}$.\\
		Vậy thể tích khối cầu là: $V=\dfrac{4}{3}\pi{R^3}$ $=\dfrac{4}{3}\pi{\left(\dfrac{\sqrt{14}}{2}\right)^3}=$ $\dfrac{7\pi\sqrt{14}}{3}$.
	}
\end{ex}
\begin{ex}
	[THPT Hoàng Hoa Thám Hưng Yên 2019]%Câu 37
	Thể tích khối cầu ngoại tiếp hình lập phương cạnh $3\mathrm{cm}$ là
	\choice
	{\True $\dfrac{27\pi\sqrt{3}}{2}$ $\mathrm{cm}^3$}
	{$\dfrac{9\pi\sqrt{3}}{2}$ $\mathrm{cm}^3$}
	{$9\pi\sqrt{3}$ $\mathrm{cm}^3$}
	{$\dfrac{27\pi\sqrt{3}}{8}$ $\mathrm{cm}^3$}
	\loigiai{
		{\color{red}HÌNH Ở ĐÂY}\\
		Gọi $R$ là bán kính khối cầu ngoại tiếp hình lập phương $ABCD.EFGH$.\\
		Ta có $CE=AB.\sqrt{3}=3\sqrt{3}$ $\mathrm{cm}$. Suy ra $R=\dfrac{1}{2}CE=\dfrac{3\sqrt{3}}{2}$ $\mathrm{cm}$.\\
		Thể tích khối cầu là: $V=\dfrac{4}{3}\pi{R^3}=\dfrac{4}{3}\pi{\left(\dfrac{3\sqrt{3}}{2}\right)^3}=\dfrac{27\sqrt{3}}{2}\pi $ $\mathrm{cm}^3$.
	}
\end{ex}
\begin{ex}
	[Chuyên Nguyễn Trãi Hải Dương 2019]%Câu 38
	Diện tích mặt cầu ngoại tiếp khối hộp chữ nhật có kích thước $a$, $a\sqrt{3}$, $2a$ là
	\choice
	{$8a^2$}
	{$4\pi{a^2}$}
	{$16\pi{a^2}$}
	{\True $8\pi{a^2}$}
	\loigiai{
		{\color{red}HÌNH Ở ĐÂY}\\
		Xét hình hộp chữ nhật là $ABCD.A'{B}'{C}'{D}'$ có $AB=a$, $AD=a\sqrt{3}$, $A{A}'=2a$.\\
		Gọi $I$ là trung điểm $A'C$, suy ra $I$ là tâm mặt cầu ngoại tiếp hình hộp chữ nhật $ABCD.A'{B}'{C}'{D}'$.\\
		Ta có bán kính mặt cầu ngoại tiếp hình hộp $ABCD.A'{B}'{C}'{D}'$ là\\
		$R=\dfrac{1}{2}A{C}'=\dfrac{1}{2}\sqrt{A{B^2}+A{D^2}+A{A'^2}}=a\sqrt{2}$.\\
		Vậy diện tích mặt cầu là $S=4\pi{R^2}=8\pi{a^2}$.
	}
\end{ex}
\begin{ex}
	[THPT Hoàng Hoa Thám - Hưng Yên 2019]%Câu 39
	Thể tích khối cầu ngoại tiếp hình lập phương cạnh bằng $3\mathrm{cm}$ là
	\choice
	{\True $\dfrac{27\sqrt{3}}{2}\pi$ $\mathrm{cm}^3$}
	{$\dfrac{9\pi\sqrt{3}}{2}$ $\mathrm{cm}^3$}
	{$9\pi\sqrt{3}$ $\mathrm{cm}^3$}
	{$\dfrac{27\sqrt{3}}{8}\pi$ $\mathrm{cm}^3$}
	\loigiai{
		Nhận xét: Khối cầu ngoại tiếp hình lập phương có tâm chính là tâm của hình lập phương và bán kính bằng nửa độ dài đường chéo.\\
		Ta có Độ dài đường chéo $d=3\sqrt{3}$ nên bán kính của khối cầu $R=\dfrac{3\sqrt{3}}{2}$.\\
		Vậy $V=\dfrac{4}{3}\pi{R^3}=\dfrac{27\sqrt{3}}{2}\pi$ $\left(\mathrm{cm}^3\right)$.
	}
\end{ex}
\begin{ex}
	[Chuyên Nguyễn Huệ - 2019]%Câu 40
	Tính đường kính mặt cầu ngoại tiếp hình lập phương có cạnh bằng $a\sqrt{3}$.
	\choice
	{\True $3a$}
	{$a\sqrt{3}$}
	{$6a$}
	{$\dfrac{3a}{2}$}
	\loigiai{
		Đường kính mặt cầu ngoại tiếp hình lập phương bằng độ dài đường chéo của hình lập phương đó.\\
		Do đó, đường kính của mặt cầu cần tìm là $d=a\sqrt{3}.\sqrt{3}=3a$.
	}
\end{ex}
\begin{ex}%Câu 41
	Tính thể tích $V$ cầu khối cầu nội tiếp hình lập phương cạnh $a$.
	\choice
	{\True $V=\dfrac{\pi{a^3}}{6}$}
	{$V=\dfrac{4\pi{a^3}}{3}$}
	{$V=\dfrac{\pi{a^3}}{3}$}
	{$V=\dfrac{\pi{a^3}}{2}$}
	\loigiai{
		{\color{red}HÌNH Ở ĐÂY}\\
		Nhìn vào hình vẽ dễ nhận thấy bán kính mặt cầu nội tiếp hình lập phương là tâm $I$, bán kính $r=IO=\dfrac{a}{2}$. Thể tích của mặt cầu nội tiếp hình lập phương là:\\
		$V=\dfrac{4}{3}\pi{r^3}=\dfrac{4}{3}\pi .\left(\dfrac{a}{2}\right)^3=\dfrac{\pi{a^3}}{6}$ (đvtt). 
	}
\end{ex}
\begin{ex}
	Cho khối cầu tiếp xúc với tất cả các mặt của một hình lập phương. Gọi $V_1$; $V_2$ lần lượt là thể tích của khối cầu và khối lập phương đó. Tính $k=\dfrac{V_1}{V_2}$.
	\choice
	{$k=\dfrac{2\pi}{3}$}
	{\True $k=\dfrac{\pi}{6}$}
	{$k=\dfrac{\pi}{3}$}
	{$k=\dfrac{\pi\sqrt{2}}{3}$}
	\loigiai{
		{\color{red}HÌNH Ở ĐÂY}\\
		Gọi $a$ là cạnh của hình lập phương đã cho.\\
		Bán kính của khối cầu là $R=\dfrac{a}{2}$, nên thể tích của nó là $V_1=\dfrac{4}{3}\pi{R^3}$ $=\dfrac{4}{3}\pi .\left(\dfrac{a}{2}\right)^3$ $=\dfrac{\pi{a^3}}{6}$.\\
		Thể tích khối lập phương là $V_2=a^3$.\\
		Vậy $k=\dfrac{V_1}{V_2}=\dfrac{\pi}{6}$.
	}
\end{ex}
\begin{ex}%Câu 43         
	Tính thể tích của khối cầu nội tiếp hình lập phương có cạnh bằng 1.
	\choice
	{$\dfrac{\pi}{12}$}
	{$\dfrac{\pi}{3}$}
	{\True $\dfrac{\pi}{6}$}
	{$\dfrac{2\pi}{3}$}
	\loigiai{
		{\color{red}HÌNH Ở ĐÂY}\\
		Bán kính của khối cầu $r=\dfrac{1}{2}$.\\
		Thể tích khối cầu $V=\dfrac{4}{3}\pi{r^3}=\dfrac{4}{3}\pi .\left(\dfrac{1}{2}\right)^3=\dfrac{\pi}{6}$.
	}
\end{ex}       
\Closesolutionfile{ans}
\indapan{10}{ans/CD23/Muc_5_6}