\Opensolutionfile{ans}[ans/CD23/Muc_7_8]
\setcounter{ex}{0}
\setcounter{dang}{0}
\section{Mức độ 7,8 điểm}
\begin{dang}
	{Khối cầu ngoại tiếp khối lăng trụ}
\end{dang}
\begin{ex}
	[THPT Ninh Bình - Bạc Liêu - 2019]%Câu 1
	Một hình hộp chữ nhật có ba kích thước $a$, $b$, $c$ nội tiếp một mặt cầu. Tính diện tích $S$ của mặt cầu đó
	\choice
	{$S=16\left(a^2+b^2+c^2\right)\pi $}
	{\True $S=\left(a^2+b^2+c^2\right)\pi $}
	{$S=4\left(a^2+b^2+c^2\right)\pi $}
	{$S=8\left(a^2+b^2+c^2\right)\pi $}
	\loigiai{
		{\color{red}HÌNH Ở ĐÂY}\\
		Bán kính mặt cầu ngoại tiếp hình hộp chữ nhật là $r=OA=\dfrac{A{C}'}{2}=\dfrac{\sqrt{a^2+b^2+c^2}}{2}$.\\
		Diện tích mặt cầu ngoại tiếp hình hộp chữ nhật là $S=4\pi{r^2}=4\left(\dfrac{\sqrt{a^2+b^2+c^2}}{2}\right)^2\pi=\left(a^2+b^2+c^2\right)\pi$.
	}
\end{ex}
\begin{ex}
	[Chuyên Thái Bình - 2018]%Câu 2
	Cho lăng trụ tam giác đều có cạnh đáy bằng $a$ cạnh bên bằng $b$. Tính thể tích của khối cầu đi qua các đỉnh của lăng trụ.
	\choice
	{$\dfrac{1}{18\sqrt{3}}\sqrt{\left(4a^2+3b^2\right)^3}$}
	{\True $\dfrac{\pi}{18\sqrt{3}}\sqrt{\left(4a^2+3b^2\right)^3}$}
	{$\dfrac{\pi}{18\sqrt{3}}\sqrt{\left(4a^2+b^2\right)^3}$}
	{$\dfrac{\pi}{18\sqrt{2}}\sqrt{\left(4a^2+3b^2\right)^3}$}
	\loigiai{
		{\color{red}HÌNH Ở ĐÂY}\\
		Gọi $I,I'$ lần lượt là tâm hai đáy, $O$ là trung điểm của $I{I}'$. Khi đó ta có $O$ là tâm mặt cầu ngoại tiếp lăng trụ.\\
		Ta có: $AI=\dfrac{a\sqrt{3}}{3},IO=\dfrac{b}{2}$ suy ra bán kính mặt cầu ngoại tiếp lăng trụ là $R=\sqrt{\dfrac{a^2}{3}+\dfrac{b^2}{4}}=\dfrac{1}{2\sqrt{3}}\sqrt{4a^2+3b^2}$\\
		Vậy $V_{\left(O;R\right)}=\dfrac{4}{3}\pi{R^3}=\dfrac{\pi}{18\sqrt{3}}\sqrt{\left(4a^2+3b^2\right)^3}$.
	}
\end{ex}
\begin{ex}%Câu 3
	Một mặt cầu ngoại tiếp hình hộp chữ nhật $A B C D \cdot A'B'C'D'$ có kích thước $A B=4 a$, $A D=5 a$, $A A'=3 a$. Mặt cầu trên có bán kính bằng bao nhiêu?
	\choice
	{\True $\dfrac{5 \sqrt{2}a}{2}$}
	{$6 a$}
	{$2 \sqrt{3}a$}
	{$\dfrac{3 \sqrt{2}a}{2}$}
	\loigiai{
		{\color{red}HÌNH Ở ĐÂY}\\
		Gọi $I$ là tâm của hình hộp chữ nhật $A B C D \cdot A'B'C'D'$ khi đó bán kính mặt cầu ngoại tiếp hình hộp này là $R=I A=\dfrac{1}{2}A C=\dfrac{1}{2}\sqrt{A B^2+A D^2+A'A^2}=\dfrac{5 \sqrt{2}a}{2}$.
	}
\end{ex}
\begin{ex}
	[Chuyên Lê Hồng Phong Nam Định 2019]%Câu 4
	Thể tích khối cầu ngoại tiếp hình chữ nhật có ba kích thước $1$, $2$, $3$ là
	\choice
	{$\dfrac{9\pi}{8}$}
	{$\dfrac{9\pi}{2}$}
	{$36\pi $}
	{\True $\dfrac{7\sqrt{14}\pi}{3}$}
	\loigiai{
		{\color{red}HÌNH Ở ĐÂY}\\
		Ta có $A{C}'=\sqrt{A{A'^2}+A{B^2}+A{D^2}}=\sqrt{14}$.\\
		Mặt cầu ngoại tiếp hình hộp chữ nhật nhận đường chéo $A{C}'$ là đường kính, do đó bán kính mặt cầu là $R=\dfrac{1}{2}A{C}'=\dfrac{\sqrt{14}}{2}$. Vậy thể tích khối cầu là $V=\dfrac{4}{3}\pi{R^3}=\dfrac{4}{3}\pi\dfrac{14\sqrt{14}}{8}=\dfrac{7\sqrt{14}\pi}{3}$.
	}
\end{ex}
\begin{ex}
	[THPT Vĩnh Lộc - Thanh Hóa 2019]%Câu 5
	Tính bán kính $R$ của mặt cầu ngoại tiếp của một hình lập phương có cạnh bằng $2a$ 
	\choice
	{$R=\dfrac{a\sqrt{3}}{3}$}
	{$R=a$}
	{$R=2a\sqrt{3}$}
	{\True $R=a\sqrt{3}$}
	\loigiai{
		{\color{red}HÌNH Ở ĐÂY}\\
		Hình lập phương $ABCD.A'{B}'{C}'{D}'$ như hình vẽ. $I$ là tâm của hình lập phương. Khi đó $I$ là tâm mặt cầu ngoại tiếp của hình lập phương.\\
		Ta có $R=\dfrac{A'C}{2}=\dfrac{\sqrt{A{A'^2}+A{C^2}}}{2}=\dfrac{\sqrt{A{A'^2}+A{B^2}+A{D^2}}}{2}=a\sqrt{3}$.
	}
\end{ex}
\begin{ex}
	[Chuyên Nguyễn Trãi Hải Dương 2019]%Câu 6
	Diện tích mặt cầu ngoại tiếp khối hộp chữ nhật có kích thước $a$, $a\sqrt{3}$ và $2a$.
	\choice
	{$8a^2$}
	{$4\pi{a^2}$}
	{$16\pi{a^2}$}
	{\True $8\pi{a^2}$}
	\loigiai{
		{\color{red}HÌNH Ở ĐÂY}\\
		Xét khối hộp chữ nhật $ABCD.A'{B}'{C}'{D}'$ tâm $O$, với $AB=a$, $AD=a\sqrt{3}$ và $A{A}'=2a$. Dễ thấy $O$ cách đều các đỉnh của khối hộp này nên mặt cầu ngoại tiếp khối hộp có tâm $O$, bán kính $R=\dfrac{A{C}'}{2}$.\\
		Ta có\\
		$AC=\sqrt{A{B^2}+A{D^2}}=2a$, $A{C}'=\sqrt{A{C^2}+C{C'^2}}=2a\sqrt{2}$ $\Rightarrow R=\dfrac{A{C}'}{2}=a\sqrt{2}$.\\
		Vậy diện tích mặt cầu ngoại tiếp khối hộp này là $S=4\pi{R^2}=8\pi{a^2}$.
	}
\end{ex}
\begin{ex}
	[Chuyên Đại học Vinh - 2019]%Câu 7
	Cho hình hộp chữ nhật $ABCD.A'{B}'{C}'{D}'$ có $AB=a$, $AD=A{A}'=2a$. Diện tích của mặt cầu ngoại tiếp hình hộp đã cho bằng
	\choice
	{\True $9\pi{a^2}$}
	{$\dfrac{3\pi{a^2}}{4}$}
	{$\dfrac{9\pi{a^2}}{4}$}
	{$3\pi{a^2}$}
	\loigiai{
		{\color{red}HÌNH Ở ĐÂY}\\
		Ta có tâm mặt cầu ngoại tiếp hình hộp $ABCD.A'{B}'{C}'{D}'$ cũng là trung điểm của một đường chéo $A'C$ (giao các đường chéo) của hình hộp.\\
		Hình hộp chữ nhật có độ dài 3 cạnh dài, rộng, cao là: $AD=2a$, $AB=a$, $A{A}'=2a$.\\
		$\Rightarrow $ Bán kính mặt cầu ngoại tiếp hình hộp là: $R=\dfrac{A'C}{2}=\dfrac{\sqrt{A{D^2}+A{B^2}+A{A'^2}}}{2}=\dfrac{3a}{2}$.\\
		$\Rightarrow{S_{\mathrm{mc}}}=4\pi{R^2}=4\pi .\left(\dfrac{3a}{2}\right)^2=9\pi{a^2}$.
	}
\end{ex}
\begin{ex}
	Cho hình lập phương có cạnh bằng $a$. Thể tích khối cầu ngoại tiếp hình lập phương đó bằng
	\choice
	{$V=\dfrac{4\sqrt{3}}{3}\pi{a^3}$}
	{$V=4\sqrt{3}\pi{a^3}$}
	{$V=\dfrac{\pi{a^3}\sqrt{3}}{3}$}
	{\True $V=\dfrac{\pi{a^3}\sqrt{3}}{2}$}
	\loigiai{
		{\color{red}HÌNH Ở ĐÂY}\\
		Tâm $I$ của mặt cầu ngoại tiếp lập phương $ABCD.A'{B}'{C}'{D}'$ là trung điểm của đường chéo $A{C}'$ và $R=IA=\dfrac{A{C}'}{2}$\\
		Khối lập phương cạnh $a$ nên:\\
		$A{A}'=a,{A}'{C}'=a\sqrt{2}$\\
		$\Rightarrow A{C}'=\sqrt{A{A'^2}+A'{C'^2}}=\sqrt{a^2+\left(a\sqrt{2}\right)^2}=a\sqrt{3}\Rightarrow R=\dfrac{A{C}'}{2}=\dfrac{a\sqrt{3}}{2}$.\\
		Vậy thể tích khối cầu cần tính là:\\
		$V=\dfrac{4}{3}.\pi .R^3=\dfrac{4}{3}.\pi .\left(\dfrac{a\sqrt{3}}{2}\right)^3=\dfrac{4}{3}\pi .a^3\dfrac{3\sqrt{3}}{8}=\dfrac{\pi .a^3\sqrt{3}}{2}$ (đvtt).
	}
\end{ex}
\begin{ex}
	[Nho Quan A - Ninh Bình - 2019]%Câu 9
	Cho hình lập phương $ABCD.A'{B}'{C}'{D}'$ cạnh $a$. Tính diện tích $S$ của mặt cầu ngoại tiếp hình lập phương $ABCD.A'{B}'{C}'{D}'$.
	\choice
	{\True $3\pi{a^2}$}
	{$\pi{a^2}$}
	{$\dfrac{4\pi{a^2}}{3}$}
	{$\dfrac{\pi{a^2}\sqrt{3}}{2}$}
	\loigiai{
		{\color{red}HÌNH Ở ĐÂY}\\
		Gọi $O$ là tâm của hình lập phương $ABCD.A'{B}'{C}'{D}'$ khi đó bán kính mặt cầu ngoại tiếp hình lập\\
		phương $ABCD.A'{B}'{C}'{D}'$ là $R=OA=\dfrac{a\sqrt{3}}{2}$. Do đó diện tích $S$ của mặt cầu ngoại tiếp hình lập\\
		phương $ABCD.A'{B}'{C}'{D}'$ là $S=4\pi{R^2}=4\pi{\left(\dfrac{a\sqrt{3}}{2}\right)^2}=3\pi{a^{2.}}$.
	}
\end{ex}
\begin{ex}
	[Đại học Hồng Đức – Thanh Hóa 2019]%Câu 10
	Cho hình lập phương $ABCD.A'{B}'{C}'{D}'$ có cạnh bằng $a$. Tính bán kính $R$ của mặt cầu ngoại tiếp tứ diện $AB{B}'{C}'$. 
	\choice
	{$R=a\sqrt{3}$}
	{$R=\dfrac{a\sqrt{3}}{4}$}
	{\True $R=\dfrac{a\sqrt{3}}{2}$}
	{$R=2a$}
	\loigiai{
		{\color{red}HÌNH Ở ĐÂY}\\
		Gọi $I$ là trung điểm của $AC'$.\\
		Ta có $\Delta AB{C}'$ vuông tại $B$ (vì $AB\perp (BB'C'C)$) và $\Delta A{B}'{C}'$ vuông tại $B'$ (vì $B'{C}'\perp (AB{B}'{A}')$).\\
		Khi đó $IA=IB=I{B}'=I{C}'$, suy ra $I$ là tâm mặt cầu ngoại tiếp tứ diện $AB{B}'{C}'$.\\
		$A{C}'=\sqrt{AB{'^2}+B'{C'^2}}=\sqrt{A{B^2}+B{B'^2}+B'{C'^2}}=a\sqrt{3.}$ Vậy $R=\dfrac{a\sqrt{3}}{2}$.\\
		Cách khác: Mặt cầu ngoại tiếp tứ diện $AB{B}'{C}'$ cũng là mặt cầu ngoại tiếp tứ diện hình lập phương $ABCD.A'{B}'{C}'{D}'$. Bán kính mặt cầu là nửa đường chéo hình lập phương cạnh $a$, tức là bằng $\dfrac{a\sqrt{3}}{2}$.
	}
\end{ex}
\begin{ex}
	[Chuyên Quốc Học Huế 2019]%Câu 11
	Cho lăng trụ đứng $ABC.A'{B}'{C}'$ có đáy là tam giác $ABC$ vuông cân tại $A$, $AB=a$, $A{A}'=a\sqrt{3}$. Tính bán kính $R$ của mặt cầu đi qua tất cả các đỉnh của hình lăng trụ theo $a$.
	\choice
	{\True $R=\dfrac{a\sqrt{5}}{2}$}
	{$R=\dfrac{a}{2}$}
	{$R=2a$}
	{$R=\dfrac{a\sqrt{2}}{2}$}
	\loigiai{
		Hình vẽ.\\
		{\color{red}HÌNH Ở ĐÂY}\\
		Gọi $M$ là trung điểm $BC$, suy ra $M$ là tâm đường tròn ngoại tiếp tam giác $ABC$.\\
		Gọi $M'$ là trung điểm $B'{C}'$, suy ra $M'$ là tâm đường tròn ngoại tiếp tam giác $A'{B}'{C}'$.\\
		Gọi $I$ là trung điểm $M{M}'$, khi đó $I$ chính là tâm đường tròn ngoại tiếp lăng trụ.\\
		Theo đề ta có $MB=\dfrac{BC}{2}=\dfrac{a\sqrt{2}}{2}$ và $IM=\dfrac{M{M}'}{2}=\dfrac{A{A}'}{2}=\dfrac{a\sqrt{3}}{2}$.\\
		Tam giác $MIB$ vuông tại $M$ nên ta tính được $R=IB=\sqrt{I{M^2}+M{B^2}}=\dfrac{a\sqrt{5}}{2}$.
	}
\end{ex}
\begin{ex}
	Tính diện tích $S$ của mặt cầu ngoại tiếp hình lăng trụ tam giác đều có tất cả các cạnh bằng $a$.
	\choice
	{\True $\dfrac{7\pi{a^2}}{3}$}
	{$\dfrac{\pi{a^3}}{8}$}
	{$\pi{a^2}$}
	{$\dfrac{7\pi{a^2}}{9}$}
	\loigiai{
		{\color{red}HÌNH Ở ĐÂY}\\
		Gọi $O$, $O'$ lần lượt là tâm đường tròn ngoại tiếp hai tam giác $ABC$, $A'B'C'$.\\
		Trên $OO’$ lấy trung điểm $I$. Suy ra $IA=IB=IC=IA’=IB’=IC’$.\\
		Vậy $I$ là tâm mặt cầu ngoại tiếp lăng trụ.\\
		Suy ra bán kính mặt cầu $R=IA=\sqrt{O{I^2}+O{A^2}}=\sqrt{O{I^2}+O{A^2}}=\sqrt{\left(\dfrac{a}{2}\right)^2+\left(\dfrac{a\sqrt{3}}{3}\right)^2}=\dfrac{a\sqrt{21}}{6}$.\\
		Diện tích mặt cầu $S=4\pi{R^2}=4\pi\dfrac{7a^2}{12}=\dfrac{7\pi{a^2}}{3}$.
	}
\end{ex}
\begin{ex}
	[Chuyên Bắc Giang 2019]%Câu 13
	Cho hình lập phương có cạnh bằng $1$. Thể tích mặt cầu đi qua các đỉnh của hình lập phương là
	\choice
	{$\dfrac{2\pi}{3}$}
	{\True $\dfrac{\sqrt{3}\pi}{2}$}
	{$\dfrac{3\pi}{2}$}
	{$\dfrac{3\sqrt{3}\pi}{2}$}
	\loigiai{
		{\color{red}HÌNH Ở ĐÂY}\\
		Mặt cầu qua các đỉnh của hình lập phương có đường kính là $A'C$.\\
		Bán kính mặt cầu là $R=\dfrac{A'C}{2}=\dfrac{\sqrt{3}}{2}$.\\
		Thể tích khối cầu là $v=\dfrac{4}{3}\pi{R^3}=\dfrac{\sqrt{3}\pi}{2}$.
	}
\end{ex}
\begin{ex}
	Cho hình lập phương $ABCD.A'B'C'D'$ có cạnh bằng $a$. Đường kính của mặt cầu ngoại tiếp hình lập phương là
	\choice
	{\True $a\sqrt{3}$}
	{$a\sqrt{2}$}
	{$\dfrac{a\sqrt{3}}{2}$}
	{$\dfrac{a\sqrt{2}}{2}$}
	\loigiai{
		{\color{red}HÌNH Ở ĐÂY}\\
		Độ dài đường kính của mặt cầu ngoại tiếp hình lập phương bằng độ dài đường chéo của hình lập phương bằng $AC'$. Ta có $ABCD$ là hình vuông cạnh $a\Rightarrow AC=a\sqrt{2}$. Xét tam giác $A'AC$ vuông tại $A\Rightarrow AC'=\sqrt{AA{'^2}+A{C^2}}=\sqrt{a^2+\left(a\sqrt{2}\right)^2}=a\sqrt{3}$.
	}
\end{ex}
\begin{ex}
	Tỉ số thể tích giữa khối lập phương và khối cầu ngoại tiếp khối lập phương đó bằng
	\choice
	{$\dfrac{\pi\sqrt{3}}{2}$}
	{\True $\dfrac{2\sqrt{3}}{3\pi}$}
	{$\dfrac{3\sqrt{2}}{2\pi}$}
	{$\dfrac{\pi\sqrt{2}}{3}$}
	\loigiai{
		{\color{red}HÌNH Ở ĐÂY}\\
		Xét hình lập phương $ABCD.A'{B}'{C}'{D}'$ cạnh $2a$ nội tiếp trong mặt cầu $(S)$.\\
		Khi ấy, khối lập phương có thể tích $V_1=\left(2a\right)^3$ $=8a^3$ và bán kính mặt cầu $(S)$ là $R=\dfrac{2a\sqrt{3}}{2}$ $=a\sqrt{3}$.\\
		Thể tích khối cầu$(S)$: $V_2=\dfrac{4}{3}\pi{R^3}$ $=\dfrac{4}{3}\pi{\left(a\sqrt{3}\right)^3}$ $=4\pi{a^3}\sqrt{3}$.\\
		Vậy tỉ số thể tích giữa khối lập phương và khối cầu ngoại tiếp khối lập phương bằng\\
		$\dfrac{V_1}{V_2}=\dfrac{8a^3}{4\pi{a^3}\sqrt{3}}$ $=\dfrac{2}{\pi\sqrt{3}}$ $=\dfrac{2\sqrt{3}}{3\pi}$.
	}
\end{ex}
\begin{ex}
	Cho hình hộp chữ nhật $ABCD.A'B'C'D'$ có $AB=a$, $AD=2a$, $AA'=3a$. Thể tích khối cầu ngoại tiếp hình hộp chữ nhật $ABCD.A'B'C'D'$ là
	\choice
	{$\dfrac{28\sqrt{14}\pi{a^3}}{3}$}
	{$\sqrt{6}\pi{a^3}$}
	{\True $\dfrac{7\sqrt{14}\pi{a^3}}{3}$}
	{$4\sqrt{6}\pi{a^3}$}
	\loigiai{
		{\color{red}HÌNH Ở ĐÂY}\\
		Gọi $O$ là tâm của hình hộp $ABCD.A'B'C'D'$.\\
		Tứ giác $ABC'D'$ là hình chữ nhật có tâm $O$ nên $OA=OB=OC'=OD'$ (1).\\
		Tương tự ta có các tứ giác $CDB'A'$, $BDD'B'$ là các hình chữ nhật tâm $O$ nên $OC=OD=OA'=OB'$, $OB=OD=OB'=OD'$ (2).\\
		Từ (1) và (2) ta có điểm $O$ cách đều các đỉnh của hình hộp nên $O$ là tâm mặt cầu ngoại tiếp hình hộp.\\
		Bán kính mặt cầu là: $R=OA=\dfrac{AC'}{2}=\dfrac{\sqrt{AA{'^2}+A'C{'^2}}}{2}=\dfrac{\sqrt{AA{'^2}+A'B{'^2}+A'D{'^2}}}{2}$\\
		$=\dfrac{\sqrt{9a^2+a^2+4a^2}}{2}=\dfrac{a\sqrt{14}}{2}$.\\
		Thể tích khối cầu là $V=\dfrac{4}{3}\pi{\left(\dfrac{a\sqrt{14}}{2}\right)^3}=\dfrac{7\sqrt{14}\pi{a^3}}{3}$.
	}
\end{ex}
\begin{ex}
	Cho hình lăng trụ đứng $ABC.A'{B}'{C}'$ có đáy $ABC$ là tam giác vuông tại $A$, $AB=a\sqrt{3}$, $BC=2a$, đường thẳng $A{C}'$ tạo với mặt phẳng $\left(BC{C}'{B}'\right)$ một góc $30^\circ $ (tham khảo hình vẽ bên dưới). Tính diện tích $S$ của mặt cầu ngoại tiếp hình lăng trụ đã cho?\\
	{\color{red}HÌNH Ở ĐÂY}
	\choice
	{$S=24\pi{a^2}$}
	{\True $S=6\pi{a^2}$}
	{$S=4\pi{a^2}$}
	{$S=3\pi{a^2}$}
	\loigiai{
		{\color{red}HÌNH Ở ĐÂY}\\
		Kẻ $AH\perp BC$ $\left(H\in BC\right)$ thì $AH\perp (BC{C}'{B}')$ (vì $(ABC)$ và $(BC{C}'{B}')$ vuông góc với nhau theo giao tuyến $BC$). Suy ra $\widehat{A{C}'H}=30^\circ $.\\
		$\Delta ABC$ vuông tại $A$ có đường cao $AH$ nên $AC=\sqrt{B{C^2}-A{B^2}}=a$ và$AH=\dfrac{AB.AC}{BC}=\dfrac{a\sqrt{3}}{2}$.\\
		$\Delta AH{C}'$ vuông tại $H$ $\Rightarrow A{C}'=\dfrac{AH}{\sin 30^\circ}=a\sqrt{3}$. Suy ra $A{A}'=\sqrt{A{C'^2}-A{B^2}}=a\sqrt{2}$.\\
		Ta có thể xem hình lăng trụ đã cho là một phần của hình hộp chữ nhật có các kích thước lần lượt là $AB=a\sqrt{3}$, $AC=a$ và $A'A=a\sqrt{2}$.\\
		Do đó bán kính mặt cầu ngoại tiếp hình lăng trụ là $R=\dfrac{1}{2}\sqrt{\left(a\sqrt{3}\right)^2+a^2+\left(a\sqrt{2}\right)^2}=\dfrac{a\sqrt{6}}{2}$.\\
		Diện tích mặt cầu cần tìm: $S=4\pi{R^2}=6\pi{a^2}$.
	}
\end{ex}
\begin{ex}
	[Chuyên ĐH Vinh - Nghệ An - 2020]%Câu 18
	Cho hình lăng trụ tam giác đều $ABC.A'{B}'{C}'$ có $A{A}'=2a$, $BC=a$. Gọi $M$ là trung điểm của $B{B}'$. Bán kính mặt cầu ngoại tiếp khối chóp $M.A'{B}'{C}'$ bằng
	\choice
	{$\dfrac{3\sqrt{3}a}{8}$}
	{$\dfrac{\sqrt{13}a}{2}$}
	{\True $\dfrac{\sqrt{21}a}{6}$}
	{$\dfrac{2\sqrt{3}a}{3}$}
	\loigiai{
		{\color{red}HÌNH Ở ĐÂY}\\
		Gọi $O$; $O'$ lần lượt là trọng tâm của các tam giác $ABC$ và $A'{B}'{C}'$.\\
		Vì $ABC.A'{B}'{C}'$ là lăng trụ tam giác đều $\Rightarrow\left\{\begin{aligned}
			& O{O}'=A{A}'=B{B}'=2a\\ 
			& O{O}'\perp\left(ABC\right);O{O}'\perp\left(A'{B}'{C}'\right)\\ 
			& BC=B'{C}'=a\\ 
		\end{aligned}\right.$.\\
		Như vậy $O{O}'$ là trục đường tròn ngoại tiếp 2 mặt đáy.\\
		$\Rightarrow $ tâm mặt cầu ngoại tiếp khối chóp $M.A'{B}'{C}'$ nằm trên $O{O}'$.\\
		Trong mặt phẳng $\left(OB{B}'{O}'\right)$, từ trung điểm $H$ của $M{B}'$, kẻ đường thẳng vuông góc với $M{B}'$ cắt $O{O}'$ tại $I$.\\
		Suy ra $I{A}'=I{C}'=I{B}'=IM\Rightarrow $ khối chóp $M.A'{B}'{C}'$ nội tiếp mặt cầu tâm $I$, bán kính $R=I{B}'$.\\
		Gọi $N$ là trung điểm của $A'{C}'$.\\
		Dễ dàng chứng minh được $HI{O}'{B}'$ là hình chữ nhật.\\
		Suy ra $I{B}'=\sqrt{I{O'^2}+B'{O'^2}}=\sqrt{H{B'^2}+\left(\dfrac{2}{3}{B}'N\right)^2}=\sqrt{\left(\dfrac{B{B}'}{4}\right)^2+\left(\dfrac{2}{3}.\dfrac{BC\sqrt{3}}{2}\right)^2}$\\
		$\Rightarrow I{B}'=\sqrt{\left(\dfrac{a}{2}\right)^2+\left(\dfrac{a\sqrt{3}}{3}\right)^2}=\dfrac{a\sqrt{21}}{6}$.
	}
\end{ex}
\begin{ex}
	[Chuyên Thái Bình - 2020]%Câu 19
	Cho lăng trụ đứng $ABC.A'{B}'{C}'$ có chiều cao bằng 4, đáy $ABC$ là tam giác cân tại $A$ với $AB=AC=2$; $\widehat{BAC}=120^\circ $. Tính diện tích mặt cầu ngoại tiếp lăng trụ trên
	\choice
	{$\dfrac{64\pi\sqrt{2}}{3}$}
	{$16\pi $}
	{\True $32\pi $}
	{$\dfrac{32\pi\sqrt{2}}{3}$}
	\loigiai{
		{\color{red}HÌNH Ở ĐÂY}\\
		Gọi $M,M'$ lần lượt là trung điểm của $BC$ và $B'{C}'$. Gọi $I,I'$ lần lượt là tâm đường tròn ngoại tiếp tam giác $ABC$ và tam giác $A'{B}'{C}'$. Khi đó, $I{I}'$ là trục đường tròn ngọai tiếp các tam giác $ABC$ và tam giác $A'{B}'{C}'$, suy ra tâm mặt cầu là trung điểm $O$ của $I{I}'$.\\
		Ta có $BM=AB.\sin 60^\circ=\sqrt{3}\Rightarrow BC=2\sqrt{3}$.\\
		$\dfrac{BC}{\sin\widehat{BAC}}=2.IA\Rightarrow IA=\dfrac{2\sqrt{3}}{2.\sin 120^\circ}=2$; $OI=2\Rightarrow OA=\sqrt{O{I^2}+I{A^2}}=2\sqrt{2}$.\\
		Bán kính mặt cầu $R=OA=2\sqrt{2}$. Diện tích mặt cầu là $S=4\pi{R^2}=4\pi{\left(2\sqrt{2}\right)^2}=32\pi $.
	}
\end{ex}
\begin{ex}
	[Chuyên Sơn La - 2020]%Câu 20
	Cho hình lăng trụ tam giác đều $ABC.A'{B}'{C}'$ có các cạnh đều bằng $a$. Tính diện tích $S$ của mặt cầu đi qua $6$ đỉnh của hình lăng trụ đó.
	\choice
	{\True $S=\dfrac{7\pi{a^2}}{3}$}
	{$S=\dfrac{7a^2}{3}$}
	{$S=\dfrac{49\pi{a^2}}{144}$}
	{$S=\dfrac{49a^2}{114}$}
	\loigiai{
		{\color{red}HÌNH Ở ĐÂY}\\
		Gọi $I,I'$ lần lượt là trọng tâm tam giác $ABC,A'{B}'{C}'$, $O$ là trung điểm của $I{I}'$. Khi đó $O$ là tâm mặt cầu ngoại tiếp hình lăng trụ.\\
		Ta có $AI=\dfrac{2}{3}AM=\dfrac{a\sqrt{3}}{3}$, $OI=\dfrac{a}{2}$.\\
		Bán kính mặt cầu ngoại tiếp hình lăng trụ $R=OA=\sqrt{O{I^2}+A{I^2}}=\sqrt{\left(\dfrac{a}{2}\right)^2+\left(\dfrac{a}{\sqrt{3}}\right)^2}=\dfrac{a\sqrt{7}}{\sqrt{12}}$.\\
		Diện tích mặt cầu $S=4\pi{R^2}=4\pi .\dfrac{7a^2}{12}=\dfrac{7\pi{a^2}}{3}$.\\
	}
\end{ex}
\begin{dang}
	{Khối cầu ngoại tiếp khối chóp}
\end{dang}
\begin{ex}
	[Mã 101 - 2020 Lần 1]%Câu 21
	Cho hình chóp $S.ABC$ có đáy là tam giác đều cạnh $4a$, $SA$ vuông góc với mặt phẳng đáy, góc giữa mặt phẳng $\left(SBC\right)$ và mặt phẳng đáy bằng $60^\circ $. Diện tích của mặt cầu ngoại tiếp hình chóp $S.ABC$ bằng
	\choice
	{\True $\dfrac{172\pi{a^2}}{3}$}
	{$\dfrac{76\pi{a^2}}{3}$}
	{$84\pi{a^2}$}
	{$\dfrac{172\pi{a^2}}{9}$}
	\loigiai{
		{\color{red}HÌNH Ở ĐÂY}\\
		Ta có tâm của đáy cũng là giao điểm ba đường cao (ba đường trung tuyến) của tam giác đều $ABC$ nên bán kính đường tròn ngoại tiếp đáy là $r=4a.\dfrac{\sqrt{3}}{3}=\dfrac{4\sqrt{3}a}{3}$.\\
		Đường cao $AH$ của tam giác đều $ABC$ là $AH=\dfrac{4a.\sqrt{3}}{2}=2\sqrt{3}a$.\\
		Góc giữa mặt phẳng $\left(SBC\right)$ và mặt phẳng đáy bằng $60^\circ $ suy ra $\widehat{SHA}=60^\circ $.\\
		Suy ra $\tan SHA=\dfrac{SA}{AH}=\dfrac{SA}{2\sqrt{3}a}=\sqrt{3}\Rightarrow SA=6a$.\\
		Bán kính mặt cầu ngoại tiếp $R_{mc}=\sqrt{\left(\dfrac{SA}{2}\right)^2+r^2}=\sqrt{9a^2+\dfrac{16}{3}{a^2}}=\dfrac{\sqrt{129}}{3}a$.\\
		Diện tích mặt cầu ngoại tiếp của hình chóp $S.ABC$ là $S_{mc}=4\pi{R^2}=4\pi{\left(\dfrac{\sqrt{129}}{3}a\right)^2}=\dfrac{172\pi{a^2}}{3}$.
	}
\end{ex}
\begin{ex}
	[Mã 102 - 2020 Lần 1]%Câu 22
	Cho hình chóp $S . A B C$ có đáy là tam giác đều cạnh $4 a, S A$ vuông góc với mặt phẳng đáy, góc giữa mặt phẳng $(S B C)$ và mặt phẳng đáy bằng $30^{\circ}$. Diện tích mặt cầu ngoại tiếp hình chóp $S . A B C$ bằng
	\choice
	{$52 \pi a^2$}
	{$\dfrac{172 \pi a^2}{3}$}
	{$\dfrac{76 \pi a^2}{9}$}
	{\True $\dfrac{76 \pi a^2}{3}$}
	\loigiai{
		\textbf{(VẼ HÌNH)}
		Gọi $M, N, P$ lần lượt là trung điểm của $B C, A B, S A$
		Gọi $G$ là trọng tâm tam giác đồng thời là tâm đường tròn ngoại tiếp tam giác $A B C$.\\
		Qua $G$ ta dựng đường thẳng $d$ vuông góc mặt đáy.\\
		Kẻ đường trung trực $S A$ cắt đường thẳng $d$ tại $I$, khi đó $I$ là tâm mặt cầu ngoại tiếp khối chóp $S . A B C$.\\
		$\begin{aligned}
			& \text{Ta có}~((S B C),(A B C))=S M A=30^{\circ}, \\
			& \Rightarrow S A=A M \cdot \tan 30^{\circ}=4 a \cdot \dfrac{\sqrt{3}}{2}\cdot \dfrac{\sqrt{3}}{3}=2 a \Rightarrow A P=\dfrac{S A}{2}=a \\
			& A G=\dfrac{2}{3}A M=\dfrac{2}{3}\cdot 4 a \cdot \dfrac{\sqrt{3}}{2}=\dfrac{4 a \sqrt{3}}{3}\Rightarrow P I=A G=\dfrac{4 a \sqrt{3}}{3}
		\end{aligned}$\\
		Xét tam giác $A P I$ vuông tại $P$ có $A I=\sqrt{A P^2+P I^2}=\sqrt{a^2+\left(\dfrac{4 a \sqrt{3}}{3}\right)^2}=\dfrac{a \sqrt{57}}{3}$.\\
		Bán kính $R=A I=\dfrac{a \sqrt{57}}{3}$.\\
		Diện tích mặt cầu $S=4 \pi R^2=\dfrac{76 \pi a^2}{3}$.
	}
\end{ex}
\begin{ex}
	[Mã 103 - 2020 Lần 1]%Câu 23
	Cho hình chóp $S.ABC$ có đáy là tam giác đều cạnh $2a$, $SA$ vuông góc với mặt phẳng đáy, góc giữa mặt $(SBC)$ và mặt phẳng đáy là $60^\circ$  Diện tích của mặt cầu ngoại tiếp hình chóp $S.ABC$ bằng
	\choice
	{\True $\dfrac{43\pi{a^2}}{3}$}
	{$\dfrac{19\pi{a^2}}{3}$}
	{$\dfrac{43\pi{a^2}}{9}$}
	{$21\pi{a^2}$}
	\loigiai{
		Gọi $I,J$ lần lượt là trung điểm của $BC,SA$. Ta có $\left(\widehat{\left(SBC\right),\left(ABC\right)}\right)=\widehat{SIA}=60^\circ .$.\\
		$\Rightarrow SA=AI.\tan 60^\circ=3a$ $\Rightarrow KG=\dfrac{SA}{2}=\dfrac{3a}{2}$\\
		Gọi $G$ trọng tâm tam giác đồng thời là tâm đường tròn ngoại tiếp tam giác $ABC$.\\
		Qua $G$ ta dựng đường thẳng $\Delta\perp\left(ABC\right)$.\\
		Dựng trung trực $SA$ cắt đường thẳng $\Delta $ tại $K$, khi đó $KS=KA=KB=KC$ nên $K$ là tâm mặt cầu ngoại tiếp khối chóp $S.ABC$.\\
		Ta có $R=KA=\sqrt{K{G^2}+A{G^2}}=a.\sqrt{\dfrac{43}{12}}$. Diện tích mặt cầu $S=4\pi{R^2}=\dfrac{43\pi{a^2}}{3}$.
	}
\end{ex}
\begin{ex}
	[Mã 104 - 2020 Lần 1]%Câu 24
	Cho hình chóp $S.ABC$ có đáy là tam giác đều cạnh $2a$, $SA$ vuông góc với mặt phẳng đáy, góc giữa mặt phẳng $\left(SBC\right)$ và mặt phẳng đáy bằng $30^\circ$. Diện tích của mặt cầu ngoại tiếp hình chóp $S.ABC$ bằng
	\choice
	{$\dfrac{43\pi{a^2}}{3}$}
	{\True $\dfrac{19\pi{a^2}}{3}$}
	{$\dfrac{19\pi{a^2}}{9}$}
	{$13\pi{a^2}$}
	\loigiai{
		{\color{red}HÌNH Ở ĐÂY}\\
		Gọi $M$ là trung điểm của đoạn $BC$.\\
		$N$ là trung điểm của đoạn $SA$.\\
		$G$ là trọng tâm $\Delta ABC$.\\
		Gọi $d'$ là đường thẳng đi qua trọng tâm $G$ của $\Delta ABC$ và vuông góc với mặt phẳng đáy.\\
		$d$ là đường trung trực của đoạn thẳng $SA$.\\
		Từ đó suy ra tâm $I$ của mặt cầu ngoại tiếp hình chóp $S.ABC$ là giao điểm của hai đường thẳng $d$ và $d'$.\\
		Suy ra: bán kính mặt cầu $R=AI$.\\
		Ta có: $\Delta ABC$ đều cạnh $2a$ $\Rightarrow AM=2a.\dfrac{\sqrt{3}}{2}=a\sqrt{3}$ và $AG=\dfrac{2a\sqrt{3}}{3}$.\\
		Góc giữa mặt phẳng $\left(SBC\right)$ và mặt phẳng đáy là góc $\widehat{SMA}=30^\circ$\\
		$\tan\widehat{SMA}=\dfrac{SA}{AM}\Rightarrow SA=AM.\tan{30^\circ}=a\sqrt{3}.\dfrac{\sqrt{3}}{3}=a$.\\
		Suy ra $AN=\dfrac{a}{2}$.\\
		Do đó $R=AI=\sqrt{A{N^2}+N{I^2}}=\sqrt{A{N^2}+A{G^2}}=\sqrt{\left(\dfrac{a}{2}\right)^2+\left(\dfrac{2a\sqrt{3}}{3}\right)^2}=\dfrac{\sqrt{57}}{6}$\\
		Vậy diện tích của mặt cầu ngoại tiếp hình chóp $S.ABC$ là $S=4\pi .R^2=4\pi .\left(\dfrac{\sqrt{57}}{6}\right)^2=\dfrac{19\pi{a^2}}{3}$.
	}
\end{ex}    
\begin{ex}
	[Sở Bắc Ninh - 2020]%Câu 25
	Cho hình chóp $ABCD$ có đáy là hình thang vuông tại $A$ và $D$. Biết $SA$ vuông góc với $ABCD$, $AB=BC=a$ ,$AD=2a$, $SA=a\sqrt{2}$. Gọi $E$ là trung điểm của $AD$. Bán kính mặt cầu đi qua các điểm $S$, $A$, $B$, $C$, $E$ bằng
	\choice
	{$\dfrac{a\sqrt{3}}{2}$}
	{$\dfrac{a\sqrt{30}}{6}$}
	{$\dfrac{a\sqrt{6}}{3}$}
	{\True $a$}
	\loigiai{
		{\color{red}HÌNH Ở ĐÂY}\\
		Ta thấy các tam giác $\Delta SAC$; $\Delta SBC$; $\Delta SEC$ vuông tại $A$, $C$, $E$. Vậy các điểm $S$, $A$, $B$, $C$, $E$ nằm trên mặt cầu đường kính $SC\Rightarrow R=\dfrac{SC}{2}=\dfrac{\sqrt{S{A^2}+A{C^2}}}{2}=a$.
	}
\end{ex}
\begin{ex}
	[Sở Yên Bái - 2020]%Câu 26
	Cho hình chóp $S.ABCD$ có đáy $ABCD$ là hình chữ nhật có đường chéo bằng $a\sqrt{2}$, cạnh $SA$ có độ dài bằng $2a$ và vuông góc với mặt phẳng đáy. Tính bán kính mặt cầu ngoại tiếp hình chóp $S.ABCD$.
	\choice
	{\True $\dfrac{a\sqrt{6}}{2}$}
	{$\dfrac{a\sqrt{6}}{12}$}
	{$\dfrac{a\sqrt{6}}{4}$}
	{$\dfrac{2a\sqrt{6}}{3}$}
	\loigiai{
		{\color{red}HÌNH Ở ĐÂY}\\
		Theo giả thiết, $SA\perp\left(ABCD\right)\Rightarrow SA\perp AC$ nên $\Delta SAC$ vuông ta $A$.\\
		Mặt khác\\
		$\left\{\begin{aligned}
			& BC\perp AB\\ 
			& BC\perp SA\\ 
		\end{aligned}\right.\Rightarrow BC\perp SB$. Suy ra $\Delta SBC$ vuông ta $B$.\\
		Tương tự, ta cũng có $\Delta SCD$ vuông ta $D$.\\
		Gọi $I$ là trung điểm của $SC$. Suy ra $IS=IA=IB=IC=ID$.\\
		Do đó, $I$ là tâm của mặt cầu goại tiếp hình chóp $S.ABCD$ và bán kính $R=\dfrac{SC}{2}$.\\
		Ta có $SC=\sqrt{S{A^2}+A{C^2}}=\sqrt{\left(2a\right)^2+\left(a\sqrt{2}\right)^2}=a\sqrt{6}\Rightarrow R=\dfrac{a\sqrt{6}}{2}$.
	}
\end{ex}
\begin{ex}
	[Bỉm Sơn - Thanh Hóa - 2020]%Câu 27
	Cho hình chóp $S . A B C D$, có đáy là hình vuông cạnh bằng $x$. Cạnh bên $S A=x \sqrt{6}$ và vuông góc với mặt phẳng $A B C D$. Tính theo $x$ diện tích mặt cầu ngoại tiếp khối chóp $S . A B C D$.
	\choice
	{\True $8 \pi x^2$}
	{$x^2 \sqrt{2}$}
	{$2 \pi x^2$}
	{$2 x^2$}
	\loigiai{
		$\begin{aligned}
			&\text{Ta có}~S A \perp(A B C D) \Rightarrow S A \perp A C, S A \perp B C, S A \perp C D . \\
			& \left\{\begin{aligned}
				&B C \perp S A \\
				&B C \perp A B\\
			\end{aligned}\Rightarrow B C \perp S B,\left\{\begin{aligned}
				&C D \perp S A \\
				&C D \perp A D\\
			\end{aligned}\Rightarrow C D \perp S D .\right.\right.
		\end{aligned}$\\
		Vậy $S A C=S B C=S D C=90^{\circ}$ do đó $A, B, D, S, C$ thuộc mặt cầu đường kính $S C$.\\
		Ta có $A C=\sqrt{2}x, S C=\sqrt{S A^2+A C^2}=2 \sqrt{2}x . R$ là bán kính mặt cầu ngoại tiếp khối chóp $S . A B C D$ khi đó $R=\dfrac{S C}{2}=\sqrt{2}x$.\\ 
		Diện tích mặt cầu ngoại tiếp khối chóp $S . A B C D$ bằng $S=4 \pi R^2=4 \pi \sqrt{2}x^2=8 \pi x^2$.
	}
\end{ex}
\begin{ex}
	[Chuyên Nguyễn Tất Thành Yên Bái 2019]%Câu 28
	Cho hình chóp $S.ABCD$ có đáy là hình vuông cạnh $a$. Cạnh bên $SA=a\sqrt{6}$ và vuông góc với đáy $\left(ABCD\right)$. Tính theo $a$ diện tích mặt cầu ngoại tiếp khối chóp $S.ABCD$.
	\choice
	{\True $8\pi{a^2}$}
	{$a^2\sqrt{2}$}
	{$2\pi{a^2}$}
	{$2a^2$}
	\loigiai{
		{\color{red}HÌNH Ở ĐÂY}\\
		Gọi $O=AC\cap BD$, đường chéo $AC=a\sqrt{2}$.\\
		Gọi $I$ là trung điểm của $SC$.\\
		Suy ra $OI$ là đường trung bình của tam giác $SAC$. Suy ra $OI\parallel SA$ $\Rightarrow OI\perp\left(ABCD\right)$.\\
		Hay $OI$ là trục đường tròn ngoại tiếp đáy $ABCD$.\\
		Mà $IS=IC$ $\Rightarrow $ $IA=IB=IC=ID=IS$. Suy ra $I$ là tâm mặt cầu ngoại tiếp chóp $S.ABCD$.\\
		Bán kính mặt cầu ngoại tiếp chóp $S.ABCD$: $R=SI=\dfrac{SC}{2}=\dfrac{\sqrt{S{A^2}+A{C^2}}}{2}=a\sqrt{2}$.\\
		Diện tích mặt cầu: $S=4\pi{R^2}=8\pi{a^2}$.
	}
\end{ex}
\begin{ex}
	[Chuyên Thái Nguyên 2019]%Câu 29
	Trong không gian, cho hình chóp $S.ABC$ có $SA$, $AB$, $BC$ đôi một vuông góc với nhau và $SA=a$, $AB=b$, $BC=c$. Mặt cầu đi qua $S$, $A$, $B$, $C$ có bán kính bằng
	\choice
	{$\dfrac{2(a+b+c)}{3}$}
	{$\sqrt{a^2+b^2+c^2}$}
	{$2\sqrt{a^2+b^2+c^2}$}
	{\True $\dfrac{1}{2}\sqrt{a^2+b^2+c^2}$}
	\loigiai{
		{\color{red}HÌNH Ở ĐÂY}\\
		Ta có $\left\{\begin{aligned}
			& SA\perp AB\\ 
			& SA\perp BC\\ 
		\end{aligned}\right.\Rightarrow SA\perp\left(ABC\right)\Rightarrow SA\perp AC.$\\
		Ta có $\left\{\begin{aligned}
			& BC\perp SA\\ 
			& BC\perp AB\\ 
		\end{aligned}\right.\Rightarrow BC\perp\left(SAB\right)\Rightarrow BC\perp SB$.\\
		Gọi $O$ là trung điểm $SC$, ta có tam giác $SAC,SBC$ vuông lần lượt tại $A$ và $B$ nên\\
		$OA=OB=OC=OS=\dfrac{SC}{2}$. Do đó mặt cầu đi qua $S,A,B,C$ có tâm $O$ và bán kính $R=\dfrac{SC}{2}$.\\
		Ta có $S{C^2}=S{B^2}+B{C^2}=S{A^2}+A{B^2}+B{C^2}=a^2+b^2+c^2.$ suy ra $R=\dfrac{1}{2}\sqrt{a^2+b^2+c^2}$.
	}
\end{ex}
\begin{ex}
	[Mã 105 2017]%Câu 30
	Cho tứ diện $ABCD$ có tam giác $BCD$ vuông tại $C$, $AB$ vuông góc với mặt phẳng $\left(BCD\right)$, $AB=5a$, $BC=3a$ và $CD=4a$. Tính bán kính $R$ của mặt cầu ngoại tiếp tứ diện $ABCD$.
	\choice
	{$R=\dfrac{5a\sqrt{2}}{3}$}
	{$R=\dfrac{5a\sqrt{3}}{3}$}
	{\True $R=\dfrac{5a\sqrt{2}}{2}$}
	{$R=\dfrac{5a\sqrt{3}}{2}$}
	\loigiai{
		{\color{red}HÌNH Ở ĐÂY}\\
		Tam giác $BCD$ vuông tại $C$ nên áp dụng định lí Pitago, ta được $BD=5a$.\\
		Tam giác $ABD$ vuông tại $B$ nên áp dụng định lí Pitago, ta được $AD=5a\sqrt{2}$.\\
		Vì $B$ và $C$ cùng nhìn $AD$ dưới một góc vuông nên tâm mặt cầu ngoại tiếp tứ diện $ABCD$ là trung điểm $I$ của $AD$. Bán kính mặt cầu này là $R=\dfrac{AD}{2}=\dfrac{5a\sqrt{2}}{2}$.
	}
\end{ex}
\begin{ex}
	[Mã 104 2017]%Câu 31
	Cho hình chóp $S.ABCD$ có đáy là hình chữ nhật với $AB=3a$, $BC=4a$, $SA=12a$ và $SA$ vuông góc với đáy. Tính bán kính $R$ của mặt cầu ngoại tiếp hình chóp $S.ABCD$.
	\choice
	{\True $R=\dfrac{13a}{2}$}
	{$R=6a$}
	{$R=\dfrac{5a}{2}$}
	{$R=\dfrac{17a}{2}$}
	\loigiai{
		{\color{red}HÌNH Ở ĐÂY}\\
		Ta có $AC=\sqrt{A{B^2}+B{C^2}}=5a$\\
		Vì $SA\perp AC$ nên $SC=\sqrt{S{A^2}+A{C^2}}=13a$\\
		Nhận thấy:$\left\{\begin{aligned}
			& BC\perp AB\\ 
			& BC\perp SA\\ 
		\end{aligned}\right.\Rightarrow BC\perp SB$. Tương tự:$CD\perp SD$\\
		Do các điểm $A,$ $B,$ $D$ đều nhìn đoạn thẳng $SC$ dưới một góc vuông nên gọi $I$ là trung điểm của đoạn thẳng $SC$ thì $I$ là tâm mặt cầu ngoại tiếp hình chóp $S.ABCD$.\\
		Vậy $R=\dfrac{SC}{2}=\dfrac{13a}{2}$.
	}
\end{ex}
\begin{ex}
	[KTNL GV Thuận Thành 2 Bắc Ninh 2019]%Câu 32
	Cho hình chóp $S.ABC$ có tam giác $ABC$ vuông tại $B$, $SA$ vuông góc với mặt phẳng $(ABC)$. $SA=5,AB=3,BC=4$. Tính bán kính mặt cầu ngoại tiếp hình chóp $S.ABC$ 
	\choice
	{\True $R=\dfrac{5\sqrt{2}}{2}$}
	{$R=5$}
	{$R=\dfrac{5}{2}$}
	{$R=5\sqrt{2}$}
	\loigiai{
		{\color{red}HÌNH Ở ĐÂY}\\
		\centerline{\textbf{LỜI GIẢI 1}}\\
		Gọi $K$ là trung điểm $AC$. Gọi $M$ là trung điểm $SA$.\\
		Vì tam giác $ABC$ vuông tại $B$ nên $K$ là tâm đường tròn ngoại tiếp tam giác $ABC$.\\
		Từ K dựng đường thẳng d vuông góc với $mp\left(ABC\right)$.\\
		Trong $mp\left(SAC\right)$ dựng $MI$ là đường trung trực đoạn $SA$ cắt d tại $I$.\\
		Khi đó điểm $I$ là tâm mặt cầu ngoại tiếp hình chóp $S.ABC$ và bán kính mặt cầu là $R=AI$.\\
		Ta có $AC=\sqrt{A{B^2}+B{C^2}}=5\Rightarrow AK=\dfrac{5}{2}$. Có $IK=MA=\dfrac{SA}{2}=\dfrac{5}{2}$.\\
		Vậy $R=AI=\sqrt{A{K^2}+I{K^2}}=\sqrt{\dfrac{25}{4}+\dfrac{25}{4}}=\dfrac{5\sqrt{2}}{2}$.\\
		\centerline{\textbf{LỜI GIẢI 2}}\\
		Gọi $I$ là trung điểm của $SC.$ Tam giác $SAC$ vuông tại $A$ nên $IS=IC=IA$ (1)\\
		Ta có $BC\perp AB;BC\perp SA\Rightarrow BC\perp\left(SAB\right)$ $\Rightarrow BC\perp SB\Rightarrow\Delta SBC$ vuông tại $B$.\\
		Nên $IS=IC=IB$ (2)\\
		Từ (1) và (2) ta có $I$ là tâm mặt cầu ngoại tiếp hình chóp $S.ABC$ bán kính $R=\dfrac{1}{2}SC$.\\
		$AC=\sqrt{A{B^2}+B{C^2}}=5$; $SC=\sqrt{A{S^2}+A{C^2}}=5\sqrt{2}$\\
		Vậy $R=\dfrac{5\sqrt{2}}{2}$.
	}
\end{ex}
\begin{ex}
	[KTNL Gia Bình 2019]%Câu 33
	Cho hình chóp $SABC$ có đáy $ABC$ là tam giác vuông tại $B$, $AB=8$, $BC=6$. Biết $SA=6$ và $SA\perp (ABC)$. Tính thể tích khối cầu có tâm thuộc phần không gian bên trong của hình chóp và tiếp xúc với tất cả các mặt phẳng của hình chóp $SABC$.
	\choice
	{$\dfrac{16\pi}{9}$}
	{$\dfrac{625\pi}{81}$}
	{\True $\dfrac{256\pi}{81}$}
	{$\dfrac{25\pi}{9}$}
	\loigiai{
		{\color{red}HÌNH Ở ĐÂY}\\
		Gọi r là bán kính khối cầu nội tiếp chóp $S.ABC$, ta có $V_{S.ABC}=\dfrac{1}{3}{S_\mathrm{tp}}.r\Rightarrow r=\dfrac{3V_{S.ABC}}{S_\mathrm{tp}}$\\
		$V_{S.ABC}=\dfrac{1}{3}SA.S_{ABC}=48$\\
		Ta dễ dàng có $\Delta SAB$, $\Delta SAC$ vuông tại $S$\\
		Tính được $AC=\sqrt{A{B^2}+B{C^2}}=10$\\
		$S_{tp}=S_{SAB}+S_{SAC}+S_{ABC}=108$ (đvdt) $\Rightarrow r=\dfrac{3V_{S.ABC}}{S_\mathrm{tp}}=\dfrac{4}{3}$\\
		Vậy thể tích khối cầu nội tiếp chóp $S.ABC$ là $V=\dfrac{4}{3}\pi .r^3=\dfrac{256\pi}{81}$.
	}
\end{ex}
\begin{ex}
	[THPT An Lão Hải Phòng 2019]%Câu 34
	Cho hình chóp $S.ABC$ có đường cao $SA$, đáy $ABC$ là tam giác vuông tại $A$. Biết $SA=6a,AB=2a,AC=4a$. Tính bán kính mặt cầu ngoại tiếp hình chóp $S.ABC$.
	\choice
	{$R=2a\sqrt{7}$}
	{\True $R=a\sqrt{14}$}
	{$R=2a\sqrt{3}$}
	{$r=2a\sqrt{5}$}
	\loigiai{
		Ta có\\
		$BC=\sqrt{A{B^2}+A{C^2}}=\sqrt{4a^2+16a^2}=2a\sqrt{5}$\\
		$R_d=a\sqrt{5}$\\
		$R=\sqrt{R_d^2+\dfrac{S{A^2}}{4}}=\sqrt{5a^2+9a^2}=a\sqrt{14}$.
	}
\end{ex}
\begin{ex}
	[THPT Gia Lộc Hải Dương 2019]%Câu 35
	Cho hình chóp $S . A B C D$ có đáy $A B C D$ là hình chữ nhật có đường chéo bằng $\sqrt{2}a$, cạnh $S A$ có độ dài bằng $2 a$ và vuông góc với mặt phẳng đáy. Tính bán kính mặt cầu ngoại tiếp hình chóp $S . A B C D$.
	\choice
	{\True $\dfrac{a \sqrt{6}}{2}$}
	{$\dfrac{a \sqrt{6}}{4}$}
	{$\dfrac{2 a \sqrt{6}}{3}$}
	{$\dfrac{a \sqrt{6}}{12}$}
	\loigiai{
		\textbf{(VẼ HÌNH)}\\
		Ta có $\triangle S A C$ vuông tại $A(1)$.\\
		Chứng minh $\triangle S D C$ vuông tại $D$. Ta có\\
		$A D \perp C D$ ( vì $A B C D$ là hình chữ nhật).\\
		$S A \perp C D$ (vì cạnh $S A$ vuông góc với mặt phẳng đáy).\\
		Ta suy ra $C D \perp(S A D) \Rightarrow C D \perp S D \Rightarrow \triangle S D C$ vuông tại $D(2)$.\\
		Chứng minh tương tự, ta được $\triangle S B C$ vuông tại $B$ (3).\\
		Từ $(1),(2),(3)$: Ta suy ra mặt cầu $(S)$ ngoại tiếp hình chóp $S . A B C D$ có đường kính $S C$.\\
		Ta có $S C=\sqrt{S A^2+A C^2}=\sqrt{4 a^2+2 a^2}=a \sqrt{6}$.\\
		Vậy mặt cầu $(S)$ ngoại tiếp hình chóp $S \cdot A B C D$ có bán kính bằng $R=\dfrac{S C}{2}=\dfrac{a \sqrt{6}}{2}$.
	}
\end{ex}
\begin{ex}
	[HSG Bắc Ninh 2019]%Câu 36
	Cho hình chóp $S.ABC$ có $\widehat{BAC}=60^\circ $, $BC=a$, $SA\perp\left(ABC\right)$. Gọi $M$, $N$ lần lượt là hình chiếu vuông góc của $A$ lên $SB$ và $SC$. Bán kính mặt cầu đi qua các điểm $A$, $B$, $C$, $M$, $N$ bằng
	\choice
	{\True $\dfrac{a\sqrt{3}}{3}$}
	{$\dfrac{2a\sqrt{3}}{3}$}
	{$a$}
	{$2a$}
	\loigiai{
		{\color{red}HÌNH Ở ĐÂY}\\
		Gọi $I$ là tâm đường tròn ngoại tiếp $\Delta ABC$\\
		$\Rightarrow IA=IB=IC(1)$.\\
		Kẻ $IH$ là trung trực của $AC$.\\
		$\left.\begin{aligned}
			&IH\perp AC\\
			&IH\perp SA\\
		\end{aligned}\right\}\Leftrightarrow IH\perp\left(SAC\right)\Leftrightarrow IH\perp\left(ANC\right)$.\\
		Mà $\Delta ANC$ vuông tại $N$ có $AC$ là cạnh huyền và $H$ là trung điểm $AC$ $\Rightarrow IH$ là trục của $\Delta ANC\Rightarrow IA=IC=IN(2)$.\\
		Tương tự kẻ $IK$ là trung trực của $AB\Rightarrow IK$ là trục của $\Delta AMB\Rightarrow IA=IB=IM(3)$.\\
		$(1),(2),(3)\Rightarrow IA=IB=IC=IM=IN\Rightarrow I$ là tâm đường tròn ngoại tiếp chóp $A.BCMN$.\\
		Định lí hàm sin trong $\Delta ABC$: $IA=\dfrac{BC}{2\sin\widehat{BAC}}=\dfrac{a}{2\sin 60^\circ}=\dfrac{a\sqrt{3}}{3}$.
	}
\end{ex}
\begin{ex}
	Hình chóp $S.ABCD$ có đáy là hình chữ nhật, $AB=a$, $SA\perp\left(ABCD\right)$, $SC$ tạo với mặt đáy một góc $45^\circ$. Mặt cầu ngoại tiếp hình chóp $S.ABCD$ có bán kính bằng $a\sqrt{2}$. Thể tích của khối chóp $S.ABCD$ bằng
	\choice
	{$2a^3$}
	{$2a^3\sqrt{3}$}
	{$\dfrac{a^3\sqrt{3}}{3}$}
	{\True $\dfrac{2a^3\sqrt{3}}{3}$}
	\loigiai{
		{\color{red}HÌNH Ở ĐÂY}\\
		Gọi $O$ là tâm của hình chữ nhật $ABCD$; $I$ là trung điểm đoạn $SC$.\\
		$\left\{\begin{aligned}
			& BC\perp SA\\ 
			& BC\perp AB\\ 
		\end{aligned}\right.\Rightarrow BC\perp\left(SAB\right)\Rightarrow BC\perp SB$.\\
		$\left\{\begin{aligned}
			& CD\perp SA\\ 
			& CD\perp AD\\ 
		\end{aligned}\right.\Rightarrow CD\perp\left(SAD\right)\Rightarrow CD\perp SD$\\
		Các điểm $A$, $B$, $D$ cùng nhìn $SC$ dưới một góc vuông nên $I$ chính là tâm mặt cầu ngoại tiếp hình chóp $S.ABCD$.\\
		Mặt khác $AC$ là hình chiếu của $SC$ trên mặt phẳng đáy nên góc giữa $SC$ và mặt phẳng đáy là góc $ACS$ bằng $45^\circ$. Do đó tam giác $SAC$ vuông cân tại $A\Rightarrow SA=AC=2a$.\\
		$V_{S.ABCD}=\dfrac{1}{3}SA.S_{ABCD}=\dfrac{1}{3}.2a.a.a\sqrt{3}=\dfrac{2a^3\sqrt{3}}{3}$.
	}
\end{ex}
\begin{ex}
	[Chuyên Hạ Long 2019]%Câu 38
	Cho hình chóp $S.ABCD$ có ABCD là hình vuông cạnh bằng $a$. $SA\perp (ABCD)$, $SA=a\sqrt{3}$. Tính bán kính mặt cầu ngoại tiếp hình chóp?
	\choice
	{\True $\dfrac{a\sqrt{5}}{2}$}
	{$2a$}
	{$a\sqrt{5}$}
	{$a\sqrt{7}$}
	\loigiai{
		{\color{red}HÌNH Ở ĐÂY}\\
		Gọi $O=AC\cap BD$. Dựng ($d$) đi qua $O$ và vuông góc với $mp\left(ABCD\right)$.\\
		Dựng $\Delta $ là đường trung trực của cạnh $SA$ cắt $SA$ tại $E$.\\
		$I=d\cap\Delta\Rightarrow I$ là tâm của mặt cầu ngoại tiếp hình chóp $S.ABCD\Rightarrow$ Bán kính là $IA$.\\
		Ta có $AO=\dfrac{a\sqrt{2}}{2}$, $AE=\dfrac{a\sqrt{3}}{2}$.\\ $AI=\sqrt{A{O^2}+A{E^2}}=\sqrt{\left(\dfrac{a\sqrt{2}}{2}\right)^2+\left(\dfrac{a\sqrt{3}}{2}\right)^2}=\dfrac{a\sqrt{5}}{2}$.
	}
\end{ex}
\begin{ex}
	[THPT Gang Thép Thái Nguyên 2019]%Câu 39
	Cho hình chóp $S.ABCD$ có đáy $ABC$ là tam giác vuông cân tại $B$, $BC=2a$, cạnh bên $SA$ vuông góc với đáy. Gọi $H$, $K$ lần lượt là hình chiếu của $A$ lên $SB$ và $SC$, khi đó thể tích của khối cầu ngoại tiếp hình chóp $AHKCB$ là
	\choice
	{$\sqrt{2}\pi{a^3}$}
	{$\dfrac{\pi{a^3}}{3}$}
	{$\dfrac{\sqrt{2}\pi{a^3}}{2}$}
	{\True $\dfrac{8\sqrt{2}\pi{a^3}}{3}$}
	\loigiai{
		{\color{red}HÌNH Ở ĐÂY}\\
		Gọi $M$ là trung điểm $BC$.\\
		$\Delta ABC$ vuông cân tại $B$ $\Rightarrow $ $MB=MA=MC=\dfrac{1}{2}AC$. (1)\\
		$\Delta KAC$ vuông tại $K$ $\Rightarrow MK=\dfrac{1}{2}AC$. (2)\\
		$\left.\begin{aligned}
			&\left.\begin{aligned}
				& BC\perp AB\\ 
				& BC\perp SA\\ 
			\end{aligned}\right\}\Rightarrow BC\perp\left(SAB\right)\Rightarrow BC\perp AH\\ 
			& AH\perp SB\\ 
		\end{aligned}\right\}\Rightarrow AH\perp\left(SBC\right)\Rightarrow AH\perp HC$.\\
		$\Rightarrow\Delta AHC$ vuông tại $H$ $\Rightarrow MH=\dfrac{1}{2}AC$. (3)\\
		Từ $(1)\to(3)$ $\Rightarrow $ $M$ là tâm khối cầu ngoại tiếp hình chóp $AHKCB$.\\
		Bán kính khối cầu cần tìm $R=\dfrac{1}{2}AC=\dfrac{1}{2}\sqrt{A{B^2}+B{C^2}}=a\sqrt{2}$.\\
		Thể tích khối cầu $V=\dfrac{4}{3}\pi{R^3}=\dfrac{8\sqrt{2}\pi{a^3}}{3}$.
	}
\end{ex}
\begin{ex}
	[THPT Yên Khánh - Ninh Bình - 2019]%Câu 40
	Cho hình chóp $SABC$, đáy $ABC$ là tam giác đều cạnh $a$; $SA\perp\left(ABC\right)$. Gọi $H$, $K$ lần lượt là hình chiếu vuông góc của $A$ trên $SB$; $SC$. Diện tích mặt cầu đi qua $5$ điểm $A$, $B$, $C$, $K$, $H$ là
	\choice
	{$\dfrac{4\pi{a^2}}{9}$}
	{$3\pi{a^2}$}
	{\True $\dfrac{4\pi{a^2}}{3}$}
	{$\dfrac{\pi{a^2}}{3}$}
	\loigiai{
		{\color{red}HÌNH Ở ĐÂY}\\
		Gọi $I$ và $R$ lần lượt là tâm và bán kính đường tròn ngoại tiếp tam giác $ABC$.\\
		Vì $ABC$ là tam giác đều cạnh nên ta có $IA=IB=IC=R=\dfrac{a\sqrt{3}}{3}$.\\
		Gọi $M$, $N$ lần lượt là trung điểm của $AB$ và $AC$.\\
		Ta có $IM\perp AB$ và $IM\perp SA$ (do $SA\perp\left(ABC\right)$) suy ra $IM\perp\left(SAB\right)$; Mà $AH\perp HB$ nên $~M$ là tâm đường tròn ngoại tiếp tam giác $AHB$; Do đó $IM$ là trục của đường tròn ngoại tiếp tam giác $AHB$ $\Rightarrow IA=IH=IB$ $(1)$\\
		Lại có: $IN\perp AC$ và $IN\perp SA$ (do $SA\perp\left(ABC\right)$) suy ra $IN\perp\left(SAC\right)$; Mà $AK\perp KC$ nên $N$ là tâm đường tròn ngoại tiếp tam giác $AKC$; Do đó $IN$ là trục của đường tròn ngoại tiếp tam giác $AKC$ $\Rightarrow IA=IK=IC$ $(2)$\\
		Từ $(1)$ và $(2)$ suy ra $I$ là tâm mặt cầu đi qua $5$ điểm $A$, $B$, $C$, $K$, $H$ và bán kính mặt cầu đó là $R=\dfrac{a\sqrt{3}}{3}\Rightarrow{S_\mathrm{mc}}=4\pi{R^2}=\dfrac{4\pi{a^2}}{3}$.
	}
\end{ex}
\begin{ex}
	[Lương Thế Vinh Hà Nội 2019]%Câu 41
	Cho hình chóp $SABC$ có đáy $ABC$ là tam giác vuông cân tại $B$ và $AB=a$. Cạnh bên $SA$ vuông góc với mặt phẳng đáy. Đường thẳng $SC$ tạo với đáy một góc $60^\circ$. Tính diện tích mặt cầu đi qua bốn đỉnh của hình chóp $SABC$
	\choice
	{\True $8a^2\pi $}
	{$\dfrac{32a^2}{3}\pi $}
	{$\dfrac{8a^2\pi}{3}$}
	{$4a^2\pi $}
	\loigiai{
		{\color{red}HÌNH Ở ĐÂY}\\
		Gọi $K,M$ lần lượt là trung điểm của $AC$, $AS$\\
		Tam giác $ABC$ là tam giác vuông cân tại $B$ nên $K$ là tâm đường tròn ngoại tiếp\\
		Từ $K$ dựng đường thẳng $d$ vuông góc mặt phẳng $(ABC)$.\\
		Trong $(SAC)$, dựng đường trung trực của $SA$ cắt $d$ tại $I$\\
		Khi đó $I$ là tâm mặt cầu ngoại tiếp hình chóp SABC và bán kính mặt cầu là $R=IA$\\
		Ta có $AC=\sqrt{A{B^2}+B{C^2}}=a\sqrt{2}\Rightarrow AK=\dfrac{AC}{2}=\dfrac{a\sqrt{2}}{2}$\\
		$SA=AC.\tan\widehat{SCA}=a\sqrt{6}\Rightarrow MA=\dfrac{SA}{2}=\dfrac{a\sqrt{6}}{2}$\\
		$\Rightarrow R=IA=\sqrt{M{A^2}+A{K^2}}=a\sqrt{2}$. Diện tích mặt cầu là $S=4\pi{R^2}=8a^2\pi $.
	}
\end{ex}  
\begin{ex}
	[THPT Yên Phong Số 1 Bắc Ninh 2019]%Câu 42
	Cho hình chóp $S.ABC$ có $SA$ vuông góc với mặt phẳng $\left(ABC\right)$, tam giác $ABC$ vuông tại $B$. Biết $SA=2a,AB=a,BC=a\sqrt{3}$. Tính bán kính $R$ của mặt cầu ngoại tiếp hình chóp.
	\choice
	{$a$}
	{$2a\sqrt{2}$}
	{\True $a\sqrt{2}$}
	{$x=3;y=\dfrac{1}{2}$}
	\loigiai{
		{\color{red}HÌNH Ở ĐÂY}\\
		Ta có $\left.\begin{aligned}
			& BC\perp AB\\ 
			& BC\perp SA\\ 
		\end{aligned}\right\}\Rightarrow BC\perp\left(SAB\right)\Rightarrow BC\perp SB$, lại có $CA\perp SA$ .\\
		Do đó 2 điểm $A$, $B$ nhìn đoạn SC dưới một góc vuông. Suy ra mặt cầu ngoại tiếp hình chóp $S. ABC$ là mặt cầu đường kính $SC$.\\
		Xét tam giac $ABC$ có $AC=\sqrt{B{C^2}+B{A^2}}=2a$ suy ra $SC=\sqrt{S{A^2}+A{C^2}}=2a\sqrt{2}$.\\
		Vậy $R=a\sqrt{2}$.
	}
\end{ex}
\begin{ex}
	[THPT Yên Phong Số 1 Bắc Ninh 2019]%Câu 43
	Cho hình chóp tam giác đều $S.ABC$ có các cạnh bên $SA$, $SB$, $SC$ vuông góc với nhau từng đôi một. Biết thể tích của khối chóp bằng $\dfrac{a^3}{6}$. Tính bán kính $r$ của mặt cầu nội tiếp của hình chóp $S.ABC$.
	\choice
	{\True $r=\dfrac{a}{3+\sqrt{3}}$}
	{$r=2a$}
	{$r=\dfrac{a}{3\left(3+2\sqrt{3}\right)}$}
	{$r=\dfrac{2a}{3\left(3+2\sqrt{3}\right)}$}
	\loigiai{
		{\color{red}HÌNH Ở ĐÂY}\\
		Cách 1. Áp dụng công thức: $r=\dfrac{3V}{S_{tp}}^{}(*)$ và tam giác đều cạnh $x$ có diện tích $S=\dfrac{x^2\sqrt{3}}{4}$.\\
		Từ giả thiết $S.ABC$ đều có $SA=SB=SC$. Lại có $SA$, $SB$, $SC$ đôi một vuông góc và thể tích khối chóp $S.ABC$ bằng $\dfrac{a^3}{6}$ nên ta có $SA=SB=SC=a$.\\
		Suy ra $AB=BC=CA=a\sqrt{2}$ và tam giác $ABC$ đều cạnh có độ dài $a\sqrt{2}$. Do đó diện tích toàn phần của khối chóp $S.ABC$ là\\
		$S_{tp}=S_{SAB}+S_{SBC}+S_{SCA}+S_{ABC}=3\dfrac{a^2}{2}+\dfrac{\left(a\sqrt{2}\right)^2\sqrt{3}}{4}$ $=\dfrac{a^2\left(3+\sqrt{3}\right)}{2}$.\\
		Thay vào (*) ta được\\
		$r=\dfrac{3V}{S_{tp}}=\dfrac{3.\dfrac{a^3}{6}}{\dfrac{a^2\left(3+\sqrt{3}\right)}{2}}=\dfrac{a}{3+\sqrt{3}}$.\\
		Cách 2. Xác định tâm và tính bán kính\\
		Từ giả thiết suy ra $SA=SB=SC=a$.\\
		Kẻ $SH\perp (ABC)$, ta có $H$ là trực tâm của tam giác $ABC$.\\
		Gọi $M=AH\cap BC$, dựng tia phân giác trong của góc $\widehat{AMB}$ cắt $SH$ tại $I$, kẻ $IE\perp\left(SBC\right)$ tại $E$. Dễ thấy $E\in SM$. Khi đó ta có $IH=IE$ hay $d(I,ABC)=d(I,SBC)$ do $S.ABC$ là chóp tam giác đều nên hoàn toàn có $d(I,ABC)=d(I,SAB)=d(I,SAC)$ tức là $I$ là tâm mặt cầu nội tiếp khối chóp $S.ABC$.\\
		Ta có $r=IH=IE$.\\
		Xét $\Delta SAM$ vuông tại $S$, đường cao $SH$, tính được $SM=\dfrac{BC}{2}=\dfrac{a\sqrt{2}}{2}=\dfrac{a}{\sqrt{2}}$.\\
		$AM=\sqrt{S{A^2}+S{M^2}}=\sqrt{a^2+\dfrac{a^2}{2}}=\dfrac{a\sqrt{6}}{2}$; $MH=\dfrac{S{M^2}}{AM}=\dfrac{a^2}{2}:\dfrac{a\sqrt{6}}{2}=\dfrac{a}{\sqrt{6}}$.\\
		$\dfrac{1}{S{H^2}}=\dfrac{1}{S{A^2}}+\dfrac{1}{S{B^2}}+\dfrac{1}{S{C^2}}=\dfrac{3}{a^2}\Rightarrow SH=\dfrac{a}{\sqrt{3}}$.\\
		Áp dụng tính chất đường phân giác ta có\\
		$\begin{aligned}
			&\dfrac{IH}{IS}=\dfrac{MH}{MS}\Rightarrow\dfrac{IH}{IH+IS}=\dfrac{MH}{MH+MS}\Leftrightarrow\dfrac{IH}{SH}=\dfrac{MH}{MH+MS}\\ 
			&\Rightarrow IH=\dfrac{MH.SH}{MH+MS}=\dfrac{a}{\sqrt{6}}.\dfrac{a}{\sqrt{3}}:(\dfrac{a}{\sqrt{6}}+\dfrac{a}{\sqrt{2}})=\dfrac{a}{3+\sqrt{3}}\\ 
		\end{aligned}$\\
		Vậy $r=IH=\dfrac{a}{3+\sqrt{3}}$.
	}
\end{ex}
\begin{ex}
	[Cụm Liên Trường Hải Phòng 2019]%Câu 44
	Cho hình chóp $S.ABCD$ có đáy $ABCD$ là hình vuông cạnh bằng $a$. Đường thẳng $SA=a\sqrt{2}$ vuông góc với đáy $\left(ABCD\right)$. Gọi $M$ là trung điểm $SC$, mặt phẳng $\left(\alpha\right)$ đi qua hai điểm $A$ và $M$ đồng thời song song với $BD$ cắt $SB,SD$ lần lượt tại $E$, $F$. Bán kính mặt cầu đi qua năm điểm $S$, $A$, $E$, $M$, $F$ nhận giá trị nào sau đây?
	\choice
	{$a$}
	{$\dfrac{a}{2}$}
	{\True $\dfrac{a\sqrt{2}}{2}$}
	{$a\sqrt{2}$}
	\loigiai{
		{\color{red}HÌNH Ở ĐÂY}\\
		Ta có $\left\{\begin{aligned}
			&\left(\alpha\right)\parallel BD\\ 
			&\left(SBD\right)\cap\left(\alpha\right)=FE\\ 
		\end{aligned}\right.\Rightarrow BD\parallel {EF}$. Gọi $I$ là giao điểm của $AM$ và $SO$\\
		Dễ thấy $I$ là trong tâm tam giác $SAC$\\
		$\dfrac{SF}{SD}=\dfrac{SI}{SO}=\dfrac{2}{3}\Rightarrow SF=\dfrac{2}{3}SD\Rightarrow SF.SD=\dfrac{2}{3}S{D^2}=\dfrac{2}{3}\left(S{A^2}+A{D^2}\right)=2a^2\Rightarrow SF.SD=S{A^2}$\\
		Xét tam giác vuông $SAD$ và $SF.SD=S{A^2}\Rightarrow AF$ là đường cao của tam giác $\Rightarrow AF\perp SF$, chứng minh tương tự ta có $\Rightarrow AE\perp SB$\\
		Tam giác $SA=AC=a\sqrt{2}$ nên $AM$ vừa là trung tuyến vừa là đường cao của tam giác $SAC\Rightarrow AM\perp SM$\\
		Ta có $\left\{\begin{aligned}
			& AF\perp SF\\ 
			& AE\perp SE\\ 
			& AM\perp SM\\ 
		\end{aligned}\right.$ nên mặt cầu đi qua năm điểm $S$, $A$, $E$, $M$, $F$ có tâm là trung điểm của\\
		$SA$ và bán kính bằng $\dfrac{SA}{2}=\dfrac{a\sqrt{2}}{2}$.
	}
\end{ex}
\begin{ex}
	[Việt Đức Hà Nội 2019]%Câu 45
	Trong không gian cho hình chóp $S.ABCD$ có đáy $ABCD$ là hình thang vuông tại $A$ và $B$ với $AB=BC=1$, $AD=2$, cạnh bên $SA=1$ và $SA$ vuông góc với đáy. Gọi $E$ là trung điểm $AD$. Tính diện tích $S_{\mathrm{mc}}$ của mặt cầu ngoại tiếp hình chóp $S.CDE$.
	\choice
	{\True $S_{\mathrm{mc}}=11\pi $}
	{$S_{\mathrm{mc}}=5\pi $}
	{$S_{\mathrm{mc}}=2\pi $}
	{$S_{\mathrm{mc}}=3\pi $}
	\loigiai{
		{\color{red}HÌNH Ở ĐÂY}\\
		Gọi $H$, $G$, $F$ lần lượt là trung điểm $AB$, $SC$, $SE$; $M=AC\cap BD$.\\
		Dễ thấy $AFGH$ là hình bình hành.\\
		Ta có $\left\{\begin{aligned}
			&{AF}\perp SE(SA=AE)\\ 
			& GF\perp SE(GF\parallel AB\parallel CE,AB\perp SE)\\ 
		\end{aligned}\right.$\\
		Khi đó, $(AFGH)$ là mặt phẳng trung trực của $SE$.\\
		Theo giả thiết: tứ giác $ABCE$ là hình vuông $\Rightarrow CE\perp AD\Rightarrow\Delta CED$ vuông tại $E$.\\
		Gọi $I$ là trung điểm của $CD$, ta có $I$ là tâm đường tròn ngoại tiếp tam giác $CDE$.\\
		Đường thẳng $d$ đi qua $I$ và song song $SA$ là trục đường tròn ngoại tiếp tam giác $CDE$.\\
		$GH$ cắt $d$ tại $O$, ta có $O$ là tâm mặt cầu ngoại tiếp hình chóp $S.CDE$, bán kính: $R=OC$\\
		Vì $\left\{\begin{aligned}
			& O\in d\Rightarrow OE=OC=OD\\ 
			& O\in GH\subset ({AF}GH)\Rightarrow{OS=OE}\\ 
		\end{aligned}\right.\Rightarrow OS=OC=OD=OE$\\
		$IC=\dfrac{1}{2}CD=\dfrac{\sqrt{2}}{2}$, $\Delta OIH$ đồng dạng $\Delta GMH$ nên $\dfrac{GM}{MH}=\dfrac{OI}{IH}$ $\Rightarrow OI=\dfrac{3}{2}$ .\\
		Áp dụng định lý Pitago vào tam giác $OIC$, suy ra $R=OC=\dfrac{\sqrt{11}}{2}$.\\
		Diện tích mặt cầu ngoại tiếp hình chóp $S.CDE$ là $S_{\mathrm{mc}}=4\pi{R^2}=11\pi $.
	}
\end{ex}
\begin{ex}
	[Sở Bắc Ninh 2019]%Câu 46
	Cho hình chóp $S.ABC$ có đáy $ABC$ là tam giác vuông tại $A$, $SA$ vuông góc với mặt phẳng $\left(ABC\right)$ và $AB=2$, $AC=4$, $SA=\sqrt{5}$. Mặt cầu đi qua các đỉnh của hình chóp $S.ABC$ có bán kính là:
	\choice
	{$R=\dfrac{25}{2}$}
	{\True $R=\dfrac{5}{2}$}
	{$R=5$}
	{$R=\dfrac{10}{3}$}
	\loigiai{
		{\color{red}HÌNH Ở ĐÂY}\\
		Cách 1.\\
		Gọi $M$, $H$ lần lượt là trung điểm $BC$, $SA$.\\
		Ta có tam giác $ABC$ vuông tại $A$ suy ra $M$ là tâm đường tròn ngoại tiếp tam giác $ABC$.\\
		Qua $M$ kẻ đường thẳng $d$ sao cho $d\perp\left(ABC\right)$ $\Rightarrow d$ là trục đường tròn ngoại tiếp tam giác $ABC$.\\
		Trong mặt phẳng $\left(SAM\right)$ kẻ đường trung trực $\Delta $ của đoạn $SA$, cắt $d$ tại $I$\\
		$\Rightarrow\left\{\begin{aligned}
			& IA=IB=IC\\ 
			& IA=IS\\ 
		\end{aligned}\right.\Rightarrow IA=IB=IC=IS$ $\Rightarrow I$ là tâm mặt cầu ngoại tiếp hình chóp $S.ABC$.\\
		$\left\{\begin{aligned}
			& HA\perp\left(ABC\right)\\ 
			& IM\perp\left(ABC\right)\\ 
		\end{aligned}\right.$ $\Rightarrow\left\{\begin{aligned}
			& HA\perp AM\\ 
			& HA\parallel IM\\ 
		\end{aligned}\right.$.\\
		$\left\{\begin{aligned}
			& HI\perp SA\\ 
			& AM\perp SA\\ 
			& HI,SA,AM\subset\left(SAM\right)\\ 
		\end{aligned}\right.$ $\Rightarrow HI\parallel AM$.\\
		Suy ra tứ giác $HAMI$ là hình chữ nhật.\\
		Ta có $AM=\dfrac{1}{2}BC=\dfrac{1}{2}\sqrt{2^2+4^2}=\sqrt{5}$, $IM=\dfrac{1}{2}SA=\dfrac{\sqrt{5}}{2}$.\\
		Bán kính mặt cầu ngoại tiếp hình chóp $S.ABC$ là $R=AI=\sqrt{A{M^2}+I{M^2}}=\sqrt{5+\dfrac{5}{4}}=\dfrac{5}{2}$ .\\
		Cách 2. Sử dụng kết quả: Nếu $SABC$ là một tứ diện vuông đỉnh $A$ thì bán kính mặt cầu ngoại tiếp tứ diện $SABC$ được tính bởi công thức: $R=\dfrac{1}{2}\sqrt{A{S^2}+A{B^2}+A{C^2}}$\\
		Áp dụng công thức trên, ta có $R=\dfrac{1}{2}\sqrt{\left(\sqrt{5}\right)^2+2^2+4^2}=\dfrac{5}{2}$.
	}
\end{ex}  
\begin{ex}
	[THPT Nguyễn Tất Thành - Đh - SP - HN - 2022]%Câu 47
	Cho hình chóp $S.ABCD$ có đáy $ABCD$ là hình chữ nhật, $AB=2$, $AD=1$. Gọi $M$ là trung điểm của $DC$. Biết $SA\perp\left(ABCD\right)$ và $SA=2$. Tính bán kính mặt cầu ngoại tiếp hình chóp $S.BCM$.\\
	{\color{red}HÌNH Ở ĐÂY}
	\choice
	{$R=\sqrt{3}$}
	{$R=\dfrac{3\sqrt{3}}{2}$}
	{\True $R=\dfrac{\sqrt{11}}{2}$}
	{$R=\dfrac{\sqrt{13}}{2}$}
	\loigiai{
		{\color{red}HÌNH Ở ĐÂY}\\
		Tam giác $MBC$ vuông cân tại $C$. Gọi $H$ là trung điểm của $MB$ suy ra $H$ là tâm đường tròn ngoại tiếp tam giác $MBC$.\\
		Từ $H$ dựng đường thẳng $d$ song song với $SA$.\\
		Mà $SA\perp\left(ABCD\right)\Rightarrow SA\perp\left(MBC\right)\Rightarrow d\perp\left(MBC\right)$.\\
		Suy ra $d$ là trục của đường tròn ngoại tiếp tam giác $MBC$.\\
		Gọi $K$ là trung điểm của $SA$.\\
		Do $AH\perp SA$ từ $K$ dựng đường thẳng song song với $AH$ cắt $d$ tại $I$ suy ra $I$ là tâm mặt cầu ngoại tiếp hình chóp $S.BCM$ .\\
		Tam giác $MBC$ vuông cân tại $C$ $\Rightarrow\widehat{MBC}=45^\circ\Rightarrow\widehat{ABH}=45^\circ $.\\
		Áp dụng định lý hàm số $\cos$ cho tam giác $ABH$ ta có $\Rightarrow\cos\widehat{ABH}=\dfrac{B{A^2}+B{H^2}-A{H^2}}{2BA.BH}$ $\Leftrightarrow\cos 45^\circ=\dfrac{4+\dfrac{1}{2}-A{H^2}}{2\sqrt{2}}\Leftrightarrow A{H^2}=\dfrac{5}{2}$.\\
		Đặt $IH=x$ suy ra $S{I^2}=S{K^2}+K{I^2}=\left(2-x\right)^2+\dfrac{5}{2}$ và.$I{B^2}=x^2+\dfrac{1}{2}$.\\
		Ta có $SI=IB\Rightarrow S{I^2}=I{B^2}\Leftrightarrow{\left(2-x\right)^2}+\dfrac{5}{2}=x^2+\dfrac{1}{2}\Leftrightarrow x=\dfrac{3}{2}$.\\
		Vậy $R=SI=\dfrac{\sqrt{11}}{2}$.
	}
\end{ex}
\begin{ex}
	[Sở Vĩnh Phúc 2022]%Câu 48
	Cho hình chóp $S.ABC$ có đáy $ABC$ là tam giác đều cạnh $2a$, cạnh bên $SA$ vuông góc với mặt phẳng đáy, góc giữa hai mặt phẳng $\left(SBC\right)$ và $\left(ABC\right)$ bằng $45^\circ $. Diện tích mặt cầu ngoại tiếp hình chóp $S.ABC$ bằng
	\choice
	{$\dfrac{25\pi{a^2}}{12}$}
	{\True $\dfrac{25\pi{a^2}}{3}$}
	{$\dfrac{25\pi{a^2}}{9}$}
	{$\dfrac{25\pi{a^2}}{6}$}
	\loigiai{
		{\color{red}HÌNH Ở ĐÂY}\\
		Gọi $M$ là trung điểm của $BC$, $G$ là trọng tâm tam giác $ABC$.\\
		Dựng đường thẳng $d$ qua $G$ và song song với $SA\Rightarrow d\perp\left(ABC\right)$, $d$ là trục đường tròn ngoại tiếp tam giác đều $ABC$.\\
		Dựng đường trung trực cạnh $SA$, cắt $d$ tại $I$ thì $I$ là tâm mặt cầu ngoại tiếp của hình chóp $S.ABC$ và bán kính $R=IA$.\\
		Ta có\\
		$\widehat{\left(\left(SBC\right),\left(ABC\right)\right)}=\widehat{SMA}=45^\circ $\\
		$\Rightarrow SA=AM.\tan 45^\circ=\sqrt{3}a$.\\
		$AG=\dfrac{2}{3}AM=\dfrac{2\sqrt{3}a}{3}$.\\
		Bán kính mặt cầu $IA=\sqrt{A{G^2}+I{G^2}}=\sqrt{A{G^2}+\dfrac{S{A^2}}{4}}=\dfrac{5\sqrt{3}a}{6}$.\\
		Diện tích mặt cầu: $S=4\pi{R^2}=\dfrac{25\pi{a^2}}{3}$.
	}
\end{ex}
\begin{ex}
	[Chuyên Hạ Long 2022]%Câu 49
	Cho hình chóp $S\cdot A B C D$ có đáy $A B C D$ là hình vuông cạnh $a$, $S A=a\sqrt{7}$ và vuông góc với đáy. Lấy điểm $M$ trên cạnh $S C$ sao cho $C M<a$. Gọi $(C)$ là hình nón có đỉnh $C$, các điểm $B$, $M$, $D$ thuộc mặt xung quanh, điểm $A$ thuộc mặt đáy của hình nón. Tính diện tích xung quanh của $(C)$.
	\choice
	{$\dfrac{16\sqrt{7}}{15}\pi a^2$}
	{\True $\dfrac{8\sqrt{30}}{15}\pi a^2$}
	{$\dfrac{32\sqrt{2}}{15}\pi a^2$}
	{$\dfrac{16\sqrt{3}}{9}\pi a^2$}
	\loigiai{
		{\color{red}HÌNH Ở ĐÂY}\\
		Lấy điểm $E$ thuộc đoạn thẳng $S C$ sao cho $C E=a$.\\
		Gọi hình nón $\left(C_1\right)$ ngoại tiếp hình chóp $C . B D E$ có đỉnh $C$.\\
		Gọi $O=A C\cap B D$. $O\in B D$ nên thuộc mặt đáy của hình nón $\left(C_1\right)$ và $C A=2 C O$, điểm $A$ thuộc mặt đáy của hình nón $(C)$. (1)\\
		Hơn nữa $C B=C D=C E=a$ suy ra $(B D E)$ vuông góc với trục của hình nón $(C)$ và thiết diện của $(B D E)$ với mặt xung quanh của hình nón $(C)$ là đường tròn, đồng thời $(B D E)$ song song với mặt chứa đáy của hình nón $(C)$. (2)\\
		Từ $(1)$ và $(2)$ suy ra hình nón $\left(C_1\right)$ đồng dạng với hình nón $(C)$ với tỷ số $\dfrac{1}{2}$.\\
		$SC=3a,\cos SCB=\dfrac{1}{3},ED=EB=\sqrt{2a^2-\dfrac{2}{3}{a^2}}=\dfrac{2\sqrt{3}}{3}a,EO=\sqrt{\dfrac{4}{3}{a^2}-\dfrac{1}{2}{a^2}}=\dfrac{\sqrt{30}}{6}a$ \\
		$S_{EBD}=\dfrac{1}{2}\cdot a\sqrt{2}\cdot\dfrac{a\sqrt{30}}{6}=\dfrac{\sqrt{15}}{6}{a^2}$ \\
		$R_{BDE}=\dfrac{\frac{4a^2}{3}\cdot a\sqrt{2}}{4\cdot\frac{a^2\sqrt{15}}{6}}=\dfrac{2\sqrt{30}}{15}a$. \\
		Diện tích xung quanh của hình nón $(C): S_{\mathrm{xq}}=\pi\cdot\dfrac{4 a\sqrt{30}}{15}\cdot 2 a=\dfrac{8\sqrt{30}}{15}\pi a^2$.
	}
\end{ex}    
\begin{ex}
	[Sở Lạng Sơn 2022]%Câu 50
	Cho hình chóp $S.ABC$ có cạnh đáy $ABC$ là tam giác vuông cân, $AB=AC=a$, $SA\perp\left(ABC\right)$ và $SA=2a$. Thể tích khối cầu ngoại tiếp hình chóp $S.ABC$ bằng 
	\choice
	{$\dfrac{9\pi{a^3}}{2}$}
	{$\dfrac{3\pi{a^3}}{2}$}
	{\True $\sqrt{6}\pi{a^3}$}
	{$3\sqrt{6}\pi{a^3}$}
	\loigiai{
		{\color{red}HÌNH Ở ĐÂY}\\
		Gọi $M$ là trung điểm $BC$ suy ra $M$ là tâm đường tròn ngoại tiếp tam giác $ABC$.\\
		Gọi $\Delta $ là trục đường tròn ngoại tiếp tam giác $ABC$. $d$ là đường trung trực của cạnh $SA$ nằm trong mặt phẳng $\left(SAM\right)$, $d$ cắt $\Delta $ tại $I$ là tâm mặt cầu ngoại tiếp hình chóp $S.ABC$.\\
		Ta có $AM=\dfrac{BC}{2}=\dfrac{a\sqrt{2}}{2}$, $IM=\dfrac{SA}{2}=a$ nên bán kính mặt cầu ngoại tiếp hình chóp $S.ABC$ là $R=\sqrt{A{M^2}+M{I^2}}=\sqrt{\left(\dfrac{a\sqrt{2}}{2}\right)^2+a^2}=\dfrac{a\sqrt{6}}{2}$.\\
		Thể tích khối cầu bằng $V=\dfrac{4}{3}\pi{R^3}=\dfrac{4}{3}\pi{\left(\dfrac{a\sqrt{6}}{2}\right)^3}=\pi{a^3}\sqrt{6}$.
	}
\end{ex}
\begin{ex}       
	[Chuyên ĐHSP Hà Nội 2022]%Câu 51
	Cho hình chóp $S.ABC$, có $SA$ vuông góc với đáy, $AB=3$, $AC=2$, $\widehat{BAC}=60^\circ$. Gọi $M$, $N$ lần lượt là hình chiếu của $A$ lên $SB$, $SC$. Tính bán kính $R$ của mặt cầu ngoại tiếp hình chóp $A.BCNM$ .
	\choice
	{$R=\dfrac{4}{\sqrt{3}}$}
	{\True $R=\dfrac{\sqrt{21}}{3}$}
	{$R=1$}
	{$R=\sqrt{2}$}
	\loigiai{
		{\color{red}HÌNH Ở ĐÂY}\\
		Gọi $I$ là tâm đường tròn ngoại tiếp $\Delta ABC$ nên $IA=IB=IC$.\\
		Gọi $E$ là trung điểm của $AC$, thì $IE$ là trung trực của $AC\Rightarrow IE\perp AC$, mà $IE\perp SA$ $\Rightarrow IE\perp\left(SAC\right)\Rightarrow IE\perp\left(NAC\right)$\\
		$\Delta NAC$ vuông tại $N\Rightarrow E$ là tâm đường tròn ngoại tiếp $\Delta NAC$.\\
		Nên $IE$ là trục của $\Delta NAC\Rightarrow IA=IC=IN$.\\
		Tương tự, gọi $F$ là trung điểm của $AB$, thì $IF$ là trung trực của $AB\Rightarrow IF\perp AB$, mà $IF\perp SA$ $\Rightarrow IF\perp\left(SAB\right)\Rightarrow IF\perp\left(MAB\right)$\\
		$\Delta MAB$ vuông tại $M\Rightarrow F$ là tâm đường tròn ngoại tiếp $\Delta MAB$.\\
		Nên $IF$ là trục của $\Delta MAB\Rightarrow IA=IB=IM$ .\\
		Vậy nên $IA=IB=IC=IM=IN=R$. $I$ là tâm mặt cầu ngoại tiếp khối chóp $A.BCNM$.\\
		Bán kính mặt mặt cầu ngoại tiếp này chính là bán kính mặt cầu ngoại tiếp $\Delta ABC$.\\
		Ta có: $S_{\Delta ABC}=\dfrac{1}{2}AB.AC.\sin\widehat{BAC}=\dfrac{1}{2}.3.2\sin 60^\circ=\dfrac{3\sqrt{3}}{2}$.\\
		Áp dụng định lí $\cos$in: $BC=\sqrt{A{B^2}+A{C^2}-2AB.AC\cos\widehat{BAC}}=\sqrt{3^2+2^2-2.3.2\cos 60^\circ}=\sqrt{7}$.\\
		Vậy nên $R=\dfrac{AB.BC.CA}{4S_{\Delta ABC}}=\dfrac{3.\sqrt{7}.2}{4.\frac{3\sqrt{3}}{2}}=\dfrac{\sqrt{21}}{3}$.
	}
\end{ex}             
\Closesolutionfile{ans}
\indapan{10}{ans/CD23/Muc_7_8}