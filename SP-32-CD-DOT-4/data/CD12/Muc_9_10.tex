\Opensolutionfile{ans}[ans/CD12/Muc_9_10]
\setcounter{ex}{0}
\setcounter{dang}{0}
\section{Mức độ 9,10 điểm}
\begin{ex}[trích dẫn]%[2H1G3-2]
	(Mã 101 2018) Cho khối lăng trụ $ABC.A’B’C’$, khoảng cách từ $C$ đến đường thẳng $BB’$ bằng $2$, khoảng cách từ $A$ đến các đường thẳng $BB’$ và $CC’$ lần lượt bằng $1$ và $\sqrt{3}$, hình chiếu vuông góc của $A$ lên mặt phẳng $(A’B’C’)$ là trung điểm $M$ của $B’C’$ và $A’M=\dfrac{2\sqrt{3}}{3}$. Thể tích của khối lăng trụ đã cho bằng
	\choice
	{$2$}
	{$1$}
	{$\sqrt{3}$}
	{$\dfrac{2\sqrt{3}}{3}$}
	\loigiai{
		\immini{
			Cắt lăng trụ bởi một mặt phẳng qua $A’$ và vuông góc với $AA’$ ta được thiết diện là tam giác $A’B_1C_1$ có các cạnh $A’B_1=1$; $A’C_1=\sqrt{3}$; $B_1C_1=2$.\\
			Suy ra tam giác $A’B_1C_1$ vuông tại $A’$ và trung tuyến $A’H$ của tam giác đó bằng $1$.\\
			Gọi giao điểm của $AM$ và $A’H$ là $T$.\\
			Ta có $A’M=\dfrac{2\sqrt{3}}{3}$; $A’H=1\Rightarrow MH=\dfrac{1}{\sqrt{3}}$.\\ Suy ra $\widehat{MA’H}=30^{\circ}$.\\
			Do đó $\widehat{MA’A}=60^{\circ}\Rightarrow AA’=\dfrac{A’M}{\cos\widehat{MA’A}}=\dfrac{4}{\sqrt{3}}$.\\
			Thể tích khối lăng trụ $ABC.A’B’C’$ bằng thể tích khối lăng trụ $A’B_1C_1\cdot AB_2C_2$ và bằng $$V=AA’\cdot S_{A’B_1C_1}=\dfrac{4}{\sqrt{3}}\cdot\dfrac{\sqrt{3}}{2}=2.$$}{\begin{tikzpicture}[>=stealth,line join=round,line cap=round,font=\footnotesize,scale=0.8]
				\def \a{2.5} \def \b{-2} \def \c{5} \def \h{5} 
				\path (0,0)coordinate(A') 
				+(\a,\b)coordinate(B_1)
				+(\c,0)coordinate(C_1)
				+(0,\h)coordinate(A)
				($(B_1)+(A)-(A')$)coordinate(B_2)
				($(C_1)+(A)-(A')$)coordinate(C_2);
				\coordinate (H) at ($(C_1)!0.5!(B_1)$);
				\coordinate (B') at ($(B_1)!0.3!(B_2)$);
				\coordinate (C') at ($(C_1)!0.5!(C_2)$);
				\coordinate (B) at ($(B_2)+(B')-(B_1)$);
				\coordinate (C) at ($(C_2)+(C')-(C_1)$);
				\coordinate (M) at ($(B')!0.5!(C')$);
				\coordinate (M') at ($(A)!1.5!(M)$);
				\coordinate (H') at ($(A')!1.5!(H)$);
				
				\coordinate (T) at (intersection cs:first line={(A)--(M')}, second line={(A')--(H')});
				\coordinate (E) at (intersection cs:first line={(M)--(T)}, second line={(B_1)--(C_1)});
				\draw[thick] (A')--(B_1)--(H) (A)--(B_2)--(C_2) (C)--(A) (A')--(A) (B_1)--(B_2) (C_1)--(C_2) (A)--(B_1) (A')--(B')--(C') (B_2)--(B)--(C)--(C_2) (A)--(B) (H)--(T) (M)--(T) (E)--(C_1);
				
				\draw [dashed] (A')--(C_1) (A')--(C') (A)--(C_2) (A')--(H) (A)--(M) (H)--(E);	
				\foreach \x/\g in{A'/180,B_1/-90, C_1/0,A/90, B_2/-25, C_2/40, M/90, T/-90, H/-90,C/0,B'/0,C'/0,B/0}
				\fill[black](\x) circle (1pt)($(\x)+(\g:3.5mm)$) node{\small $\x$};
	\end{tikzpicture}}}
\end{ex}
\begin{ex}%[2H1G3-2]
	(Mã 103 -2018) Cho khối lăng trụ $ABC.A’B’C’$, khoảng cách từ $C$ đến đường thẳng $BB’$ bằng 2, khoảng cách từ $A$ đến các đường thẳng $BB’$ và $CC’$ lần lượt bằng 1 và $\sqrt{3}$, hình chiếu vuông góc của $A$ lên mặt phẳng $(A’B’C’)$ là trung điểm $M$ của $B’C’$ và $A’M=2$. Thể tích của khối lăng trụ đã cho bằng
	\choice
	{$\dfrac{2\sqrt{3}}{3}$}
	{$1$}
	{$\sqrt{3}$}
	{$2$}
	\loigiai{
		\immini{
			Gọi $A_1,A_2$ lần lượt là hình chiếu của $A$ trên $BB’$, $CC’$. Theo đề ra $AA_1=1; AA_2=\sqrt{3}; A_1A_2=2$.\\
			Do $A\mathrm{A}_1^2+A\mathrm{A}_2^2=\mathrm{A}_1A_2^2$ nên tam giác $AA_1A_2$ vuông tại $A$.\\
			Gọi $H$ là trung điểm $A_1A_2$ thì\\ $AH=\dfrac{A_1A_2}{2}=1$.
		}{{\begin{tikzpicture}[>=stealth,line join=round,line cap=round,font=\footnotesize,scale=0.8]
					\def \a{2.5} \def \b{-1.5} \def \c{6} \def \h{5} 
					\path (0,0)coordinate(A') 
					+(\a,\b)coordinate(B')
					+(\c,0)coordinate(C');
					\coordinate (M) at ($(C')!0.5!(B')$);
					\path (M)+(0,\h)coordinate(A);
					\path (0,0)coordinate(A')
					($(B')+(A)-(A')$)coordinate(B)
					($(C')+(A)-(A')$)coordinate(C);
					\coordinate (A_2) at ($(C')!0.6!(C)$);
					\coordinate (A_1) at ($(B')!0.8!(B)$);
					\coordinate (H) at ($(A_1)!0.5!(A_2)$);
					\coordinate (K) at ($(A_1)!0.4!(H)$);
					\coordinate (N) at ($(A)!0.4!(A')$);
					\draw[thick] (A')--(B')--(C') (A)--(B)--(C)--(A) (A')--(A) (B')--(B) (C')--(C) (A)--(A_1)--(A_2)
					(M)--(H);
					\draw [dashed] (A')--(C') (A)--(M)--(A')
					(A)--(A_2) (H)--(A)--(K) (M)--(N);	
					\foreach \x/\g in{A'/180,B'/-90, C'/0,A/90, B/-45, C/40, M/-90,N/180, A_1/160,K/-120, H/-90, A_2/-90}
					\fill[black](\x) circle (1pt)($(\x)+(\g:3.8mm)$) node{\small $\x$};
					\foreach \x/\o/\y/\r in {A/K/H/2,M/N/A'/2,A/M/A'/2} \draw ($(\o)!\r mm!(\x)$)--($($(\o)!\r mm!(\x)$)+($(\o)!\r mm!(\y)$)-(\o)$)--($(\o)!\r mm!(\y)$);
		\end{tikzpicture}}}
		\noindent Lại có $MH\parallel BB’\Rightarrow MH\perp (AA_1A_2)\Rightarrow MH\perp AH$ suy ra $MH=\sqrt{AM^2-AH^2}=\sqrt{3}$.\\
		nên $\cos ((ABC),(AA_1A_2))=\cos (MH,AM)=\cos HMA=\dfrac{MH}{AM}=\dfrac{\sqrt{3}}{2}$.\\
		Suy ra $S_{ABC}=\dfrac{S_{AA_1A_2}}{\cos ((ABC),(AA_1A_2))}=1$. Thể tích lăng trụ là $V=AM\cdot S_{ABC}=2$.\\
		Nhận xét. Ý tưởng câu này là dùng diện tích hình chiếu $S’=S\cos\alpha$.}
\end{ex}
\begin{ex}%[2H1G3-2]
	(Mã 102 2018) Cho khối lăng trụ $ABC.A’B’C’$, khoảng cách từ $C$ đến $BB’$ là $\sqrt{5}$, khoảng cách từ $A$ đến $BB’$ và $CC’$ lần lượt là $1; 2$. Hình chiếu vuông góc của $A$ lên mặt phẳng $A’B’C’$ là trung điểm $M$ của $B’C’$, $A’M=\dfrac{\sqrt{15}}{3}$. Thể tích của khối lăng trụ đã cho bằng
	\choice
	{$\dfrac{2\sqrt{5}}{3}$}
	{$\sqrt{5}$}
	{$\dfrac{2\sqrt{15}}{3}$}
	{$\dfrac{\sqrt{15}}{3}$}
	\loigiai{
		\immini{Kẻ $AI\perp BB’$, $AK\perp CC’$ (hình vẽ).\\
			Khoảng cách từ $A$ đến $BB’$ và $CC’$ lần lượt là $1; 2\Rightarrow AI=1$, $AK=2$.\\
			Gọi $F$ là trung điểm của $BC$ thì \\ $A’M=\dfrac{\sqrt{15}}{3}\Rightarrow AF=\dfrac{\sqrt{15}}{3}$.
		}{\begin{tikzpicture}[>=stealth,line join=round,line cap=round,font=\footnotesize,scale=0.8]
				\def \a{2.5} \def \b{-1.5} \def \c{6} \def \h{5} 
				\path (0,0)coordinate(A') 
				+(\a,\b)coordinate(B')
				+(\c,0)coordinate(C');
				\coordinate (M) at ($(C')!0.5!(B')$);
				\path (M)+(0,\h)coordinate(A);
				\path (0,0)coordinate(A')
				($(B')+(A)-(A')$)coordinate(B)
				($(C')+(A)-(A')$)coordinate(C);
				\coordinate (K) at ($(C')!0.6!(C)$);
				\coordinate (I) at ($(B')!0.8!(B)$);
				\coordinate (E) at ($(K)!0.5!(I)$);
				\coordinate (F) at ($(B)!0.5!(C)$);
				\draw[thick] (A')--(B')--(C') (A)--(B)--(C)--(A) (A')--(A) (B')--(B) (C')--(C) (A)--(I)--(K)
				(M)--(F)--(A)(A)--(B');
				\draw [dashed] (A')--(C') (A)--(M)--(A')
				(A)--(K) (A)--(E);	
				\foreach \x/\g in{A'/180,B'/-90, C'/0,A/90, B/90, C/40, M/-90, K/-10,F/90,E/-90, I/-90}
				\fill[black](\x) circle (1pt)($(\x)+(\g:3.5mm)$) node{\small $\x$};
				\foreach \x/\o/\y/\r in {A/M/A'/2} \draw ($(\o)!\r mm!(\x)$)--($($(\o)!\r mm!(\x)$)+($(\o)!\r mm!(\y)$)-(\o)$)--($(\o)!\r mm!(\y)$);
		\end{tikzpicture}}\noindent	Ta có $\heva{&AI\perp BB’\\&BB’\perp AK}\Rightarrow BB’\perp(AIK)\Rightarrow BB’\perp IK$.\\
		Vì $CC’\parallel BB’\Rightarrow\mathrm{d}(C,BB’) =\mathrm{d}(K,BB’) =IK =\sqrt{5}\Rightarrow\triangle AIK$ vuông tại $A$.\\
		Gọi $E$ là trung điểm của $IK\Rightarrow EF\parallel BB’\Rightarrow EF\perp(AIK)\Rightarrow EF\perp AE$.\\
		Lại có $AM\perp(ABC)$. Do đó góc giữa hai mặt phẳng $(ABC)$ và $(AIK)$ là góc giữa $EF$ và $AM$ bằng góc $\widehat{AME}=\widehat{FAE}$. Ta có $\cos\widehat{FAE}=\dfrac{AE}{AF} =\dfrac{\dfrac{\sqrt{5}}{2}}{\dfrac{\sqrt{15}}{3}} =\dfrac{\sqrt{3}}{2}\Rightarrow\widehat{FAE}=30^{\circ}$.\\
		Hình chiếu vuông góc của tam giác $ABC$ lên mặt phẳng $(AIK)$ là $\triangle AIK$ nên ta có $$S_{AIK}=S_{ABC}\cos\widehat{EAF}\Rightarrow 1=S_{ABC}\dfrac{\sqrt{3}}{2}\Rightarrow\dfrac{2}{\sqrt{3}}=S_{ABC}.$$
		Xét $\triangle AMF$ vuông tại $A$: $\tan\widehat{AMF}=\dfrac{AF}{AM}\Rightarrow AM=\dfrac{\dfrac{\sqrt{15}}{3}}{\dfrac{\sqrt{3}}{3}}\Rightarrow AM=\sqrt{5}$.\\
		Vậy $V_{ABC.A’B’C’}=\sqrt{5}\cdot\dfrac{2}{\sqrt{3}} =\dfrac{2\sqrt{15}}{3}$.}
\end{ex}
\begin{ex}%[2H1G3-2]
	(Mã 104 2018) Cho khối lăng trụ $ABC.A’B’C’$. Khoảng cách từ $C$ đến đường thẳng $BB’$ bằng $\sqrt{5}$, khoảng cách từ $A$ đến các đường thẳng $BB’$ và $CC’$ lần lượt bằng $1$ và $2$, hình chiếu vuông góc của $A$ lên mặt phẳng $(A’B’C’)$ là trung điểm $M$ của $B’C’$ và $A’M=\sqrt{5}$. Thể tích của khối lăng trụ đã cho bằng
	\choice
	{$\sqrt{5}$}
	{$\dfrac{\sqrt{15}}{3}$}
	{$\dfrac{2\sqrt{5}}{3}$}
	{$\dfrac{2\sqrt{15}}{3}$}
	\loigiai{
		\immini{Gọi $J$, $K$ lần lượt là hình chiếu vuông góc của $A$ lên $BB’$ và $CC’$, $H$ là hình chiếu vuông góc của $C$ lên $BB’$.\\
			Ta có $AJ\perp BB’(1)$.\\
			$AK\perp CC’\Rightarrow AK\perp BB’(2)$.\\
		}{\begin{tikzpicture}[>=stealth,line join=round,line cap=round,font=\footnotesize,scale=0.8]
				\def \a{2.5} \def \b{-1.5} \def \c{6} \def \h{5} 
				\path (0,0)coordinate(A') 
				+(\a,\b)coordinate(B')
				+(\c,0)coordinate(C');
				\coordinate (M) at ($(C')!0.5!(B')$);
				\path (M)+(0,\h)coordinate(A);
				\path (0,0)coordinate(A')
				($(B')+(A)-(A')$)coordinate(B)
				($(C')+(A)-(A')$)coordinate(C);
				\coordinate (K) at ($(C')!0.6!(C)$);
				\coordinate (J) at ($(B')!0.8!(B)$);
				\coordinate (F) at ($(K)!0.5!(J)$);
				\coordinate (N) at ($(B)!0.5!(C)$);
				\coordinate (H) at ($(B')!1.25!(B)$);
				\coordinate (W) at (intersection cs:first line={(H)--(B)}, second line={(A)--(N)});
				\draw(A')--(B')--(C') (A)--(B)--(C)--(A) (A')--(A) (B')--(B) (C')--(C) (A)--(J)--(K)
				(M)--(N) (W)--(A)(A)--(B') (B)--(H)--(C);
				\draw [dashed] (A')--(C') (A)--(M)--(A')
				(A)--(K) (A)--(F)
				(W)--(N);	
				\foreach \x/\g in{A'/180,B'/-90, C'/0,A/90, B/90, C/40, M/-90,H/140, K/-10,N/-20,F/-90, J/-90}
				\fill[black](\x) circle (1pt)($(\x)+(\g:3.0mm)$) node{\small $\x$};
				\foreach \x/\o/\y/\r in {A/M/A'/2,C/H/B/2} \draw ($(\o)!\r mm!(\x)$)--($($(\o)!\r mm!(\x)$)+($(\o)!\r mm!(\y)$)-(\o)$)--($(\o)!\r mm!(\y)$);
		\end{tikzpicture}}\noindent Từ $(1)$ và $(2)$ suy ra $BB’\perp(AJK)\Rightarrow BB’\perp JK\Rightarrow JK\parallel CH\Rightarrow JK=CH=\sqrt{5}$.\\
		Xét $\triangle AJK$ có $JK^2=AJ^2+AK^2=5$ suy ra $\triangle AJK$ vuông tại $A$.\\
		Gọi $F$ là trung điểm $JK$ khi đó ta có $AF=JF=FK=\dfrac{\sqrt{5}}{2}$.\\
		Gọi $N$ là trung điểm $BC$, xét tam giác vuông $ANF$ ta có:\\
		$\cos\widehat{NAF}=\dfrac{AF}{AN} =\dfrac{\dfrac{\sqrt{5}}{2}}{\sqrt{5}} =\dfrac{1}{2}\Rightarrow\widehat{NAF}=60^{\circ}$. ($AN=AM=\sqrt{5}$ vì $AN\parallel AM$ và $AN=AM$).\\
		Vậy ta có $$S_{\triangle AJK}=\dfrac{1}{2}AJ\cdot AK =\dfrac{1}{2}\cdot 1\cdot 2=1.$$
		Suy ra $$S_{\triangle AJK}=S_{\triangle ABC}\cdot\cos 60^{\circ}\Rightarrow S_{\triangle ABC}=\dfrac{S_{\triangle AJK}}{\cos{60}^{\circ}}=\dfrac{1}{\dfrac{1}{2}}=2.$$
		Xét tam giác $AMA’$ vuông tại $M$ ta có $\widehat{MAA’}=\widehat{AMF}=30^{\circ}$ hay $AM=A’M\cdot\tan 30^{\circ} =\dfrac{\sqrt{15}}{3}$.\\
		Vậy thể tích khối lăng trụ là $V=AM\cdot S_{\triangle ABC} =\dfrac{\sqrt{15}}{3}\cdot 2=\dfrac{2\sqrt{15}}{3}$.}
\end{ex}
\begin{ex}%[2H1G3-2]
	\immini{
		(Chuyên Hưng Yên - 2020) Cho hình lăng trụ tam giác $ABC.A’B’C’$ có đáy là tam giác vuông tại $A$, $AB=2$, $AC=\sqrt{3}$. Góc $\widehat{CAA’}=90^{\circ}$, $\widehat{BAA’}=120^{\circ}$. Gọi $M$ là trung điểm cạnh $BB’$ (tham khảo hình vẽ). Biết $CM$ vuông góc với $A’B$, tính thể tích khối lăng trụ đã cho. 
		\choice
		{$V=\dfrac{3\left(1+\sqrt{33}\right)}{8}$}
		{$V=\dfrac{1+\sqrt{33}}{8}$}
		{\True $V=\dfrac{3\left(1+\sqrt{33}\right)}{4}$}
		{$V=\dfrac{1+\sqrt{33}}{4}$}}{\begin{tikzpicture}[line cap=round,line join=round, >=stealth,font=\footnotesize, scale=0.6]
			\def \a{1.5} \def \b{-1} \def \c{5} \def \h{4.5} 
			\path (.5,.5)coordinate(A) 
			+(\a,\b)coordinate(B)
			+(\c,0)coordinate(C)
			+(0.5,\h)coordinate(A')
			($(B)+(A')-(A)$)coordinate(B')
			($(C)+(A')-(A)$)coordinate(C');
			\coordinate (M) at ($(B)!1/2!(B')$);
			\coordinate (P) at ($(B')!1/2!(A)$);
			%\draw[ultra thin,color=gray] (-.5,-1.5) grid (8.5,5.5);
			\draw (A)--(B)--(C) (B)--(A')--(B')--(C')--(A') (A)--(A') (B)--(B') (M)--(C)--(C') ;
			\draw [dashed] (A)--(C) ;
			\foreach \x/\g in{A/180,B/-90, C/0,A'/90, B'/80, C'/40, M/30}
			\fill[black](\x) circle (1pt)($(\x)+(\g:3.5mm)$) node{\small $\x$};
	\end{tikzpicture}}
	\loigiai{
		\immini{
			Do $AC\perp AB$, $AC\perp AA’$ nên $AC\perp(ABB’A’)$.\\ Mà $A’B\subset(ABB’A’)$ nên $AC\perp A’B$.\\
			$A’B\perp AC$, $A’B\perp CM$ nên $A’B\perp(AMC)\Rightarrow A’B\perp AM$.\\
			Đặt $AA’=x (x>0)$. Ta có $\overrightarrow{A’B}=\overrightarrow{AB}-\overrightarrow{AA’}$ và\\ $\overrightarrow{AM}=\overrightarrow{AB}+\overrightarrow{BM}=\overrightarrow{AB}+\dfrac{1}{2}\overrightarrow{AA’}$.}{\begin{tikzpicture}[line cap=round,line join=round, >=stealth,font=\footnotesize, scale=0.6]
				\def \a{1.5} \def \b{-1} \def \c{5} \def \h{4.5} 
				\path (.5,.5)coordinate(A) 
				+(\a,\b)coordinate(B)
				+(\c,0)coordinate(C)
				+(0.5,\h)coordinate(A')
				($(B)+(A')-(A)$)coordinate(B')
				($(C)+(A')-(A)$)coordinate(C');
				\coordinate (M) at ($(B)!1/2!(B')$);
				\coordinate (P) at ($(B')!1/2!(A)$);
				%\draw[ultra thin,color=gray] (-.5,-1.5) grid (8.5,5.5);
				\draw (A)--(B)--(C) (B)--(A')--(B')--(C')--(A') (A)--(A') (B)--(B') (M)--(C)--(C')(A)--(M) ;
				\draw [dashed] (A)--(C)--(A') ;
				\foreach \x/\g in{A/180,B/-90, C/0,A'/90, B'/90, C'/40, M/30}
				\fill[black](\x) circle (1pt)($(\x)+(\g:3.7mm)$) node{\small $\x$};
		\end{tikzpicture}}
		\noindent	Suy ra
		\allowdisplaybreaks
		\begin{eqnarray*}
			\overrightarrow{A’B}\cdot\overrightarrow{AM} &=&\left(\overrightarrow{AB}-\overrightarrow{AA’}\right)\cdot\left(\overrightarrow{AB}+\dfrac{1}{2}\overrightarrow{AA’}\right)\\ &=&AB^2-\dfrac{1}{2}AA’^2-\dfrac{1}{2}\overrightarrow{AB}\cdot\overrightarrow{AA’}.\\
			&=&AB^2-\dfrac{1}{2}AA’^2-\dfrac{1}{2}AB\cdot AA’\cdot\cos\widehat{BAA’}\\ &=&2^2-\dfrac{1}{2}x^2-\dfrac{1}{2}\cdot 2\cdot x\cdot\cos 120^{\circ}\\ &=&-\dfrac{1}{2}x^2+\dfrac{1}{2}x+4.
		\end{eqnarray*}
		Do $A’B\perp AM$ nên $\overrightarrow{A’B}\cdot\overrightarrow{AM}=0\Leftrightarrow-\dfrac{1}{2}x^2+\dfrac{1}{2}x+4=0\Rightarrow x=\dfrac{1+\sqrt{33}}{2}$.\\
		Lại có $S_{ABB’A’}=AB\cdot AA’\cdot\sin\widehat{BAA’}=2\cdot\dfrac{1+\sqrt{33}}{2}\cdot\sin 120^{\circ} =\dfrac{\sqrt{3}\left(1+\sqrt{33}\right)}{2}$ (đvdt).\\
		Do $AC\perp(ABB’A’)$ nên $V_{C.ABB’A’}=\dfrac{1}{3}\cdot AC\cdot S_{ABB’A’}=\dfrac{1}{3}\cdot\sqrt{3}\cdot\dfrac{\sqrt{3}\left(1+\sqrt{33}\right)}{2}=\dfrac{1+\sqrt{33}}{2}$ (đvtt).\\
		Mà $V_{C.A’B’C’}=\dfrac{1}{3}V_{ABC.A’B’C’}\Rightarrow V_{C.ABB’A’}=V_{ABC.A’B’C’}-V_{C.A’B’C’}=\dfrac{2}{3}V_{ABC.A’B’C’}$.\\
		Vậy $V_{ABC.A’B’C’}=\dfrac{3}{2}V_{C.ABB’A’}=\dfrac{3}{2}\cdot\dfrac{1+\sqrt{33}}{2}=\dfrac{3\left(1+\sqrt{33}\right)}{4}$ (đvtt).}
\end{ex}
\begin{ex}%[2H1G3-2]
	(Chuyên KHTN - 2020) Cho khối lăng trụ đứng $ABC.A’B’C’$ có đáy $ABC$ là tam giác vuông cân tại $C$, $AB=2a$ và góc tạo bởi hai mặt phẳng $(ABC’)$ và $(ABC)$ bằng $60^{\circ}$. Gọi $M,N$ lần lượt là trung điểm của $A’C’$ và $BC$. Mặt phẳng $(AMN)$ chia khối lăng trụ thành hai phần. Thể tích của phần nhỏ bằng
	\choice
	{\True $\dfrac{7\sqrt{3}a^3}{24}$}
	{$\dfrac{\sqrt{6}a^3}{6}$}
	{$\dfrac{7\sqrt{6}a^3}{24}$}
	{$\dfrac{\sqrt{3}a^3}{3}$}
	\loigiai{
		\immini{
			Gọi $I$ là trung điểm $AB$, suy ra $AB\perp(CIC’)$ nên góc giữa $(C’AB)$ và $(ABC)$ là góc $(CI,C’I)$, suy ra $\widehat{C’IC}=60^{\circ}$.\\
			Tam giác $C’IC$ vuông tại $C$ nên  $$C’C=CI\cdot\tan\widehat{C’IC}=\dfrac{AB}{2}\cdot\tan 60^{\circ}=a\sqrt{3}.$$
			Diện tích tam giác $ABC$ là $S_{ABC}=\dfrac{1}{2}\cdot AB\cdot CI=a^2$.\\
			Thể tích khối lăng trụ là $V=CC’\cdot S_{ABC}=a\sqrt{3}\cdot a^2=a^3\sqrt{3}$.\\
			Trong $(ACC’A’)$, kéo dài $AM$ cắt $CC’$ tại $O$, $C’M$ là đường trung bình của $\triangle OAC$ nên $OC=2CC’=2a\sqrt{3}$.\\
			Thể tích khối chóp $O.ACN$ là $$V_{O.ACN}=\dfrac{1}{3}\cdot S_{ACN}\cdot OC=\dfrac{1}{3}\cdot\dfrac{1}{2}\cdot S_{ABC}\cdot 2CC’=\dfrac{1}{3}V.$$
			Thể tích khối chóp $O.C’ME$ là  $$V_{O.C’ME}=\dfrac{1}{3}\cdot S_{C’ME}\cdot OC’=\dfrac{1}{3}\cdot\dfrac{1}{8}S_{A’B’C’}\cdot OC’=\dfrac{1}{24}V.$$
		}{\begin{tikzpicture}[line cap=round,line join=round,font=\footnotesize,  >=stealth, scale=0.8]
				\def \a{1} \def \b{-1} \def \c{6} \def \h{4.5} 
				\path (.5,.5)coordinate(A) 
				+(\a,\b)coordinate(B)
				+(\c,0)coordinate(C)
				+(0,\h)coordinate(A')
				($(B)+(A')-(A)$)coordinate(B')
				($(C)+(A')-(A)$)coordinate(C');
				\coordinate (M) at ($(A')!1/2!(C')$);
				\coordinate (N) at ($(B)!1/2!(C)$);
				\coordinate (I) at ($(B)!1/2!(A)$);
				\coordinate (O) at ($(C)!2!(C')$);
				\coordinate (E) at ($(O)!1/2!(N)$);
				%\draw[ultra thin,color=gray] (-.5,-1.5) grid (8.5,5.5);
				\draw (A)--(B)--(C) (B)--(A')--(B')--(C')(A')--(M) (A)--(A') (B)--(B')(C)--(C') (M)--(O)--(N)(O)--(C')--(B) (M)--(E);
				\draw [dashed] (A)--(C) (A)--(M)--(C')--(A) (C')--(I)--(C);
				\foreach \x/\g in{A/180,B/-90, C/0,A'/90, B'/80, C'/40, M/130,N/-90,O/0,I/180,E/-135}
				\fill[black](\x) circle (1pt)($(\x)+(\g:3.5mm)$) node{\small $\x$};
		\end{tikzpicture}}\noindent	Do đó $$V_{C’EM\cdot CAN}=V_{O.ACN}-V_{O.C’ME}=\dfrac{1}{3}V-\dfrac{1}{24}V=\dfrac{7}{24}V=\dfrac{7}{24}\cdot a^3\sqrt{3}=\dfrac{7\sqrt{3}a^3}{24}.$$
		Vậy phần thể tích nhỏ hơn là $V_{C’EM\cdot CAN}=\dfrac{7\sqrt{3}a^3}{24}$.}
\end{ex}
\begin{ex}%[2H1G3-2]
	(Chuyên Bắc Ninh - 2020) Cho hình chóp tam giác đều $S.ABC$ có $SA=2$. Gọi $D$, $E$ lần lượt là trung điểm của cạnh $SA$, $SC$. Thể tích khối chóp $S.ABC$ biết $BD\perp AE$. 
	\choice
	{$\dfrac{4\sqrt{21}}{7}$}
	{$\dfrac{4\sqrt{21}}{3}$}
	{$\dfrac{4\sqrt{21}}{9}$}
	{\True $\dfrac{4\sqrt{21}}{27}$}
	\loigiai{
		\immini{Gọi $O$ là tâm tam giác đều $ABC$. Do $S.ABC$ là hình chóp đều nên ta có $SO\perp(ABC)$.\\
			Ta có
			\allowdisplaybreaks
			\begin{eqnarray*}
				&&\overrightarrow{AE}=\overrightarrow{SE}-\overrightarrow{SA}
				=\dfrac{1}{2}\overrightarrow{SC}-\overrightarrow{SA};\\
				&&  \overrightarrow{BD}=\overrightarrow{SD}-\overrightarrow{SB}=\dfrac{1}{2}\overrightarrow{SA}-\overrightarrow{SB}.
		\end{eqnarray*}}
		{\begin{tikzpicture}[>=stealth,line join=round,line cap=round,font=\footnotesize,scale=1]
				\def \a{1} \def \b{-1} \def \c{4} \def \h{3.5}
				\path (.5,.5)coordinate(A) 
				+(\a,\b)coordinate(B)
				+(\c,0)coordinate(C)
				($(A)!2/3!($(B)!1/2!(C)$)$)coordinate(O) %tọa độ G là trọng tâm ABC
				+(0,\h)coordinate(S);
				\path ($(C)!1/2!(B)$)coordinate(I)($(S)!1/2!(A)$)coordinate(D)($(C)!1/2!(S)$)coordinate(E);	
				\draw [dashed] (C)--(O)--(A)--(C)(S)--(O)--(I) (A)--(E);
				\draw(S)--(A)--(B)--(S)--(C)--(B) (S)--(I) (B)--(D);
				\foreach \i/\j in {A/150, B/-135,C/-90,S/90,O/240,I/-90,D/180,E/0}\fill[black] (\i) circle (1pt) ($(\i)+(\j:2mm)$)node{$\i$};
				\foreach \x/\o/\y/\r in {S/O/A/2} \draw ($(\o)!\r mm!(\x)$)--($($(\o)!\r mm!(\x)$)+($(\o)!\r mm!(\y)$)-(\o)$)--($(\o)!\r mm!(\y)$);	
		\end{tikzpicture}}
		\noindent	Đặt $\widehat{ASC}=\widehat{BSC}=\widehat{ASB}=\alpha$.
		\allowdisplaybreaks
		\begin{eqnarray*}
			BD\perp AE&\Leftrightarrow&\overrightarrow{BD}\cdot\overrightarrow{AE}=0\\
			&\Leftrightarrow&\left(\dfrac{1}{2}\overrightarrow{SA}-\overrightarrow{SB}\right)\cdot \left(\dfrac{1}{2}\overrightarrow{SC}-\overrightarrow{SA}\right)=0\\
			&\Leftrightarrow&\dfrac{1}{4}\overrightarrow{SA}\cdot\overrightarrow{SC}-\dfrac{1}{2}\overrightarrow{SA}^2-\dfrac{1}{2}\overrightarrow{SB}\cdot\overrightarrow{SC}+\overrightarrow{SA}\cdot\overrightarrow{SB}=0\\
			&\Leftrightarrow&\cos\alpha-2-2\cos\alpha+4\cos\alpha=0\Leftrightarrow\cos\alpha=\dfrac{2}{3}.
		\end{eqnarray*}
		Áp dụng định lý hàm số côsin trong tam giác $SAC$, ta có
		$$AC^2=SA^2+SC^2-2SA\cdot SC\cdot\cos\alpha=\dfrac{8}{3}\Rightarrow AC=\dfrac{2\sqrt{6}}{3}.$$
		Ta tính được $$S_{ABC}=\dfrac{2\sqrt{3}}{3};\;
		AO=\dfrac{2}{3}\cdot\dfrac{2\sqrt{6}}{3}\cdot\dfrac{\sqrt{3}}{2}=\dfrac{2\sqrt{2}}{3};\; SO=\sqrt{SA^2-AO^2}=\dfrac{2\sqrt{7}}{3}.$$
		Thể tích khối chóp $S.ABC$ là $$V=\dfrac{1}{3}SO\cdot S_{ABC}=\dfrac{1}{3}\dfrac{2\sqrt{3}}{3}\cdot\dfrac{2\sqrt{7}}{3}=\dfrac{4\sqrt{21}}{27}.$$}
\end{ex}
\begin{ex}%[2H1G3-2]
	(Chuyên Thái Bình - 2020) Cho hình lăng trụ $ABC.A’B’C’$ có đáy $ABC$ là tam giác vuông tại $A$, cạnh $BC=2a$ và $\widehat{ABC}=60^{\circ}$. Biết tứ giác $BCC’B’$ là hình thoi có $\widehat{B’BC}$ nhọn. Mặt phẳng $(BCC’B’)$ vuông góc với $(ABC)$ và mặt phẳng $(ABB’A’)$ tạo với $(ABC)$ góc $45^{\circ}$. Thể tích khối lăng trụ $ABC.A’B’C’$ bằng
	\choice
	{$\dfrac{\sqrt{7}a^3}{7}$}
	{\True $\dfrac{3\sqrt{7}a^3}{7}$}
	{$\dfrac{6\sqrt{7}a^3}{7}$}
	{$\dfrac{\sqrt{7}a^3}{21}$}
	\loigiai{
		\immini{
			Có $\heva{&(BCC’B’)\perp(ABC)\\&(BCC’B’)\cap(ABC)=BC}$.\\ Do đó trong $(BCC’B’)$ kẻ $B’H$ vuông góc với $BC$ tại $H$ thì $B’H\perp(ABC)$ hay $B’H$ là chiều cao của hình lăng trụ.\\
			Trong $(ABC)$ kẻ $HK$ vuông góc với $AB$ tại $K$.\\ Khi đó $AB\perp(B’HK)$.\\
			Ta có $\heva{&(ABB’A’)\cap(ABC)=AB\\&(B’HK)\perp AB\\&(B’HK)\cap(ABB’A’)=B’K\\&(B’HK)\cap(ABC)=KH}$
		}{\begin{tikzpicture}[line cap=round,line join=round, >=stealth,font=\footnotesize, scale=0.75]
				\def \a{2} \def \b{-1} \def \c{5} \def \h{4.5} 
				\path (.5,.5)coordinate(B) 
				+(\a,\b)coordinate(A)
				+(\c,0)coordinate(C)
				($(B)!1/2!(A)$)coordinate(K)
				($(B)!1/3!(C)$)coordinate(H)
				+(0,\h)coordinate(B')
				($(A)+(B')-(B)$)coordinate(A')
				($(C)+(B')-(B)$)coordinate(C');
				%\draw[ultra thin,color=gray] (-.5,-1.5) grid (8.5,5.5);
				\draw[thick] (B)--(A)--(C) (B')--(A')--(C')--(B') (B)--(B') (A)--(A') (C)--(C') (B')--(K);
				\draw [dashed] (B)--(C) (B) -- (H) (B')--(H) (A)--(K)--(H);
				\foreach \x/\g in{A/-90,B/-90, C/0,A'/90, B'/80, C'/40, K/220, H/-90}
				\fill[black](\x) circle (1pt)($(\x)+(\g:3.5mm)$) node{\small $\x$};
				\foreach \x/\o/\y/\r in {B'/H/B/2} \draw ($(\o)!\r mm!(\x)$)--($($(\o)!\r mm!(\x)$)+($(\o)!\r mm!(\y)$)-(\o)$)--($(\o)!\r mm!(\y)$);	
		\end{tikzpicture}}
		\noindent nên góc giữa $(ABB’A’)$ và $(ABC)$ chính là góc giữa $B’K$ và $KH$.\\
		$\triangle B’HK$ vuông tại $H$ nên $\widehat{B’KH}$ là góc nhọn. Do đó $\widehat{B’KH}=45^{\circ}$.\\
		$\triangle B’HK$ vuông tại $H$ có $\widehat{B’KH}=45^{\circ}\Rightarrow\triangle B’HK$ vuông cân tại $H\Rightarrow B’H=KH$.\\
		Xét hai tam giác vuông $B’BH$ và $BKH$, ta có $$\tan\widehat{B’BH}=\dfrac{B’H}{BH}=\dfrac{KH}{BH}=\sin\widehat{ABC}=\sin 60^{\circ}=\dfrac{\sqrt{3}}{2}$$
		Do đó
		$$\dfrac{B’H}{B’B}=\sin\widehat{B’BH}=\sqrt{1-\cos^2\widehat{B’BH}}=\sqrt{1-\dfrac{1}{\tan^2\widehat{B’BH}+1}}=\sqrt{1-\dfrac{1}{\dfrac{3}{4}+1}}=\dfrac{\sqrt{21}}{7},$$
		và 
		$$ B’H=B’B\cdot\dfrac{\sqrt{21}}{7}=\dfrac{2a\sqrt{21}}{7}\;(\text {vì}\; BCC’B’ \;\text{là hình thoi có cạnh}\; BC=2a).$$
		Ta có $$S_{ABC}=\dfrac{1}{2}AB\cdot AC=\dfrac{1}{2}\left(BC\cdot\cos{60}^{\circ}\right)\cdot \left(BC\cdot\sin{60}^{\circ}\right)=\dfrac{1}{2}\cdot 2a\cdot\dfrac{1}{2}\cdot 2a\cdot\dfrac{\sqrt{3}}{2}=\dfrac{a^2\sqrt{3}}{2}.$$
		Vậy $V_{ABC.A’B’C’}=B’H\cdot S_{ABC}=\dfrac{2a\sqrt{21}}{7}\cdot\dfrac{a^2\sqrt{3}}{2}=\dfrac{3\sqrt{7}a^3}{7}$.}
\end{ex}
\begin{ex}%[2H1G3-2]
	(Chuyên Vĩnh Phúc - 2020) Cho khối lăng trụ đứng $ABC.A’B’C’$ có đáy là tam giác đều. Mặt phẳng $(A’BC)$ tạo với đáy góc $30^{\circ}$ và tam giác $A’BC$ có diện tích bằng $8$. Tính thể tích $V$ của khối lăng trụ đã cho. 
	\choice
	{$64\sqrt{3}$}
	{$2\sqrt{3}$}
	{$16\sqrt{3}$}
	{\True $8\sqrt{3}$}
	\loigiai{
		\immini{
			Gọi $I$ là trung điểm cạnh $BC$.\\
			Vì $ABC.A’B’C’$ là lăng trụ đứng có đáy là tam giác đều nên $ABC.A’B’C’$ là khối lăng trụ đều.\\
			Do đó ta có: $A’B=A’C$. Suy ra tam giác $A’BC$ cân tại $A’\Rightarrow A’I\perp BC$.\\
			Mặt khác: tam giác $ABC$ đều $\Rightarrow AI\perp BC$.\\
			Suy ra $BC\perp(A’IA)$.\\
			Vậy góc giữa mặt phẳng $(A’BC)$ và mặt đáy bằng góc $\widehat{A’IA}=30^{\circ}$.\\
		}{\begin{tikzpicture}[line cap=round,line join=round, >=stealth,font=\footnotesize, scale=0.8]
				\def \a{1} \def \b{-1} \def \c{6} \def \h{4.5} 
				\path (.5,.5)coordinate(A) 
				+(\a,\b)coordinate(B)
				+(\c,0)coordinate(C)
				+(0,\h)coordinate(A')
				($(B)+(A')-(A)$)coordinate(B')
				($(C)+(A')-(A)$)coordinate(C');
				\coordinate (I) at ($(B)!1/2!(C)$);
				\coordinate (P) at ($(B')!1/2!(A)$);
				%\draw[ultra thin,color=gray] (-.5,-1.5) grid (8.5,5.5);
				\draw (A)--(B)--(C) (B)--(A')--(B')--(C')--(A') (A)--(A') (B)--(B') (C)--(C') ;
				\draw [dashed] (A')--(I)--(A)--(C) ;
				\foreach \x/\g in{A/180,B/-90, C/0,A'/90, B'/80, C'/40, I/-90}
				\fill[black](\x) circle (1pt)($(\x)+(\g:3.5mm)$) node{\small $\x$};
		\end{tikzpicture}}
		\noindent
		Tam giác $ABC$ là hình chiếu của tam giác $A’BC$ trên mặt đáy nên $$S_{ABC}=S_{A’BC}\cdot\cos\alpha=8\cdot\cos 30^{\circ}=4\sqrt{3}.$$
		Đặt $AB=x\Rightarrow S_{ABC}=\dfrac{x^2\sqrt{3}}{4}=4\sqrt{3}\Rightarrow x=4$.\\
		Ta có $AI=\dfrac{x\sqrt{3}}{2}=2\sqrt{3}\Rightarrow AA’=AI\cdot\tan\widehat{AIA’}=2$.\\
		Vậy $V_{ABC.A’B’C’}=AA’\cdot S_{ABC}=2\cdot 4\sqrt{3}=8\sqrt{3}$.}
\end{ex}
\begin{ex}%[2H1G3-2]
	(Sở Phú Thọ - 2020) Cho khối lăng trụ $ABC.A’B’C’$ có đáy $ABC$ là tam giác vuông tại $A$, $AB=a$, $BC=2a$. Hình chiếu vuông góc của đỉnh lên mặt phẳng $(ABC)$ là trung điểm của cạnh $H$ của cạnh $AC$. Góc giữa hai mặt phẳng $(BCB’C’)$ và $(ABC)$ bằng $60^{\circ}$. Thể tích khối lăng trụ đã cho bằng 
	\choice
	{$\dfrac{3\sqrt{3}a^3}{4}$}
	{$\dfrac{\sqrt{3}a^3}{8}$}
	{\True $\dfrac{3\sqrt{3}a^3}{8}$}
	{$\dfrac{a^3\sqrt{3}}{16}$}
	\loigiai{
		\immini{
			Tam giác $ABC$ vuông tại $A$, $AB=a$, $BC=2a$ nên $BC=a\sqrt{3}$.\\ Gọi $I$ là hình chiếu của $H$ trên $BC$.\\
			Ta có $\triangle HIC\backsim\triangle BAC$ nên $\dfrac{HI}{AB}=\dfrac{HC}{BC}$.\\
			Suy ra $HI=\dfrac{AB\cdot HC}{BC}=\dfrac{a\sqrt{3}}{4}$.\\
			Gọi $K$ là trung điểm của. Từ $K$ kẻ $KM$ vuông góc với.\\
			Tứ giác $KMIH$ là hình bình hành nên $$KM=IH=\dfrac{a\sqrt{3}}{4}.$$
		}{\begin{tikzpicture}[line cap=round,line join=round, >=stealth,font=\footnotesize, scale=0.8]
				\def \a{1.25} \def \b{-1.3} \def \c{6} \def \h{4.5} 
				\path (.5,.5)coordinate(A) 
				+(\a,\b)coordinate(B)
				+(\c,0)coordinate(C)
				+(0.5,\h)coordinate(A')
				($(B)+(A')-(A)$)coordinate(B')
				($(C)+(A')-(A)$)coordinate(C');
				\coordinate (K) at ($(A')!1/2!(C')$);
				\coordinate (H) at ($(A)!1/2!(C)$);
				\coordinate (M) at ($(C')!1/3!(B')$);
				\coordinate (I) at ($(C)!1/3!(B)$);
				\coordinate (N) at ($(C')!2!(M)$);
				\coordinate (H') at ($(H)+(N)-(I)$);
				%\draw[ultra thin,color=gray] (-.5,-1.5) grid (8.5,5.5);
				\draw (A)--(B)--(C) (B)--(A')--(B')--(C')--(A') (A)--(A') (B)--(B') (C)--(C') (K)--(M)--(I) (A')--(N)--(I);
				\draw [dashed] (I)--(H)--(K)(A)--(C) (H)--(A') (I)--(H');
				\foreach \x/\g in{A/180,B/-90, C/0,A'/90, B'/-20, C'/40,N/-130,H'/20, K/90,H/-90,M/-30,I/-90}
				\fill[black](\x) circle (1pt)($(\x)+(\g:3.5mm)$) node{\small $\x$};
		\end{tikzpicture}}\noindent	Gọi $N$ là điểm trên sao cho $M$ là trung điểm của $\Rightarrow A’N=2KM=\dfrac{a\sqrt{3}}{2}$.\\
		Do $A’H\perp(ABC)$ nên $(A’NIH)\perp(ABC)$. Mà $A’N>HI$ nên $\overset\frown{HIN}$ là góc tù. Suy ra $\overset\frown{HIN}=120^{\circ}\Rightarrow\overset\frown{A’NI}=60^{\circ}$.\\
		Gọi là hình chiếu của $I$ lên suy ra là trung điểm của thì 
		$$ A’H=IH’=NH’\cdot\tan 60^{\circ}=\dfrac{3a}{4} \Rightarrow V=A’H\cdot S_{ABC}=\dfrac{3a}{4}\cdot\dfrac{a^2\sqrt{3}}{2}=\dfrac{3\sqrt{3}a^3}{8} .$$}
\end{ex}
\begin{ex}%[2H1G3-2]
	(Sở Phú Thọ - 2020) Cho khối chóp $S.ABCD$ có đáy $ABCD$ là hình chữ nhật, $AB=a$, $SA$ vuông góc với mặt phẳng đáy và $SA=a$. Góc giữa hai mặt phẳng $(SBC)$ và $(SCD)$ bằng $\varphi$, với $cos\varphi=\dfrac{1}{\sqrt{3}}$. Thể tích của khối chóp đã cho bằng
	\choice
	{\True $\dfrac{a^3\sqrt{2}}{3}$}
	{$a^3\sqrt{2}$}
	{$\dfrac{2\sqrt{2}a^3}{3}$}
	{$\dfrac{2a^3}{3}$}
	\loigiai{
		\immini{Đặt $AD=m$, $m>0$.\\
			Chọn hệ trục tọa độ $Oxyz$ như hình vẽ, gốc tọa độ trùng với $A$, tia $Ox, Oy, Oz$ lần lượt trùng với các tia $AB, AD, AS$. Khi đó tọa độ của các điểm là
			$$B(a; 0; 0); D(0; m; 0); C(a; m; 0); S(0; 0; a).$$
			và 
			$\heva{&\overrightarrow{SB}=(a; 0;-a)\\&\overrightarrow{BC}=(0; m; 0)}\Rightarrow\left[\overrightarrow{SB},\overrightarrow{BC}\right]=(ma; 0; ma)$,\\
			$\heva{&\overrightarrow{SD}=(0; m;-a)\\&\overrightarrow{DC}=(a; 0; 0)}\Rightarrow\left[\overrightarrow{SD},\overrightarrow{DC}\right]=(0;-a;-ma)$.
		}{\begin{tikzpicture}[line cap=round,line join=round, >=stealth,font=\footnotesize,scale=0.75]
				\def \a{-2} \def \b{-1} \def \c{4} \def \h{4} 
				\path (.5,.5)coordinate(A) 
				+(0,\h)coordinate(S) 
				+(\a,\b)coordinate(B)
				+(\c,0)coordinate(D)
				($(B)+(D)-(A)$)coordinate(C); %tọa độ C là điểm chéo của hbh ABCD
				%\draw[ultra thin,color=gray] (-2.5,-1.5) grid (5.5,4.5);
				\draw [dashed] (B)--(A)--(D) (A)--(S)
				(A)--(C);
				\draw(S)--(B)--(C)--(D)--(S)--(C);
				\foreach \d/\g in {S/0,A/-90,B/160,C/-90,D/-90}
				\fill[black](\d) circle (1pt)+(\g:.38)node{$\d$};
				\draw[-stealth](B)--($(A)!1.5!(B)$)node[above]{$x$};
				\draw[-stealth](D)--($(A)!1.2!(D)$)node[above]{$y$};
				\draw[-stealth](S)--($(A)!1.3!(S)$)node[right]{$z$};
				
				\foreach \x/\o/\y/\r in {S/A/B/2} \draw ($(\o)!\r mm!(\x)$)--($($(\o)!\r mm!(\x)$)+($(\o)!\r mm!(\y)$)-(\o)$)--($(\o)!\r mm!(\y)$);	
		\end{tikzpicture}} \noindent	Véc-tơ pháp tuyến của mặt phẳng $(SBC)$ là $\left[\overrightarrow{SB},\overrightarrow{BC}\right]=(ma; 0; ma)$.\\
		Véc-tơ pháp tuyến của mặt phẳng $(SCD)$ là $\left[\overrightarrow{SD},\overrightarrow{DC}\right]=\left(0;-a^2;-ma\right)$.\\
		Theo giả thiết $$\cos\varphi=\dfrac{1}{\sqrt{3}}\Rightarrow\dfrac{m^2a^2}{a\sqrt{a^2+m^2}\cdot ma\cdot\sqrt{2}}=\dfrac{1}{\sqrt{3}}\Leftrightarrow 3m^2=2\left(a^2+m^2\right)\Rightarrow m=a\sqrt{2}.$$
		Vậy thể tích khối chóp $S.ABCD$ bằng $V=\dfrac{1}{3}\cdot SA\cdot S_{ABCD}=\dfrac{1}{3}\cdot a\cdot a\cdot a\sqrt{2}=\dfrac{a^3\sqrt{2}}{3}$.}
\end{ex}
\begin{ex}%[2H1G3-2]
	(Sở Ninh Bình) Cho lăng trụ $ABCD.A’B’C’D’$ có đáy $ABCD$ là hình chữ nhật với $AB=\sqrt{6}$, $AD=\sqrt{3}$, $A’C=3$ và mặt phẳng $(AA’C’C)$ vuông góc với mặt đáy. Biết hai mặt phẳng $(AA’C’C)$, $(AA’B’B)$ tạo với nhau góc $\alpha$ có $\tan\alpha=\dfrac{3}{4}$. Thể tích của khối lăng trụ $ABCD.A’B’C’D’$ là
	\choice
	{$V=12$}
	{$V=6$}
	{\True $V=8$}
	{$V=10$}
	\loigiai{
		\immini{
			Gọi $M$ là trung điểm của $AA’$. Kẻ $A’H$ vuông góc với $AC$ tại $H$, $BK$ vuông góc với $AC$ tại $K$, $KN$ vuông góc với $AA’$ tại $N$.\\
			Do $(AA’C’C)\perp(ABCD)$ suy ra $A’H\perp(ABCD)$ và $BK\perp(AA’C’C)\Rightarrow BK\perp AA’$ \\
			$ \Rightarrow AA’\perp(BKN)\Rightarrow AA’\perp NB $ suy ra $\left(\widehat{(AA’C’C),(AA’B’B)}\right)=\widehat{KNB}=\alpha$.\\
			Ta có: $ABCD$ là hình chữ nhật với $AB=\sqrt{6}$, $AD=\sqrt{3}$ suy ra $BD=3=AC$. Suy ra $\triangle ACA’$ cân tại $C$. }{\begin{tikzpicture}[line cap=round,line join=round, font=\footnotesize, >=stealth,scale=0.8]
				\def \a{-1.5} \def \b{-1}\def \c{4} \def \h{4.5} 
				\path (.5,.5)coordinate(A) 
				+(\a,\b)coordinate(B)
				+(\c,0)coordinate(D)
				($(B)+(D)-(A)$)coordinate(C)
				($(A)!1/3!(C)$)coordinate(H)
				($(A)!2/3!(C)$)coordinate(K)
				(H)+(0,\h)coordinate(A'); 
				\path 
				($(A)!1/3!(A')$)coordinate(N)
				($(A)!1/2!(A')$)coordinate(M)
				($(B)+(A')-(A)$)coordinate(B')
				($(C)+(B')-(B)$)coordinate(C')
				($(D)+(B')-(B)$)coordinate(D');	
				%	\draw[ultra thin,color=gray] (-2.5,-1.5) grid (7.5,5.5);
				\draw [dashed] (C)--(A)--(B)--(D)--(A) (A')--(H) (A) -- (A') (B)--(N)--(K) (C)--(M);
				\draw[thick] (B')--(B)--(C)(B')--(C')--(C)--(D)--(D')--(A')--(B')(C')--(D') (A')--(C');
				\foreach \d/\g in {K/0,A'/90,B'/90,C'/0,D'/0,A/-90,B/160,C/-90,D/-90,H/10,N/180,M/180}
				\fill[black](\d) circle (1pt)+(\g:.30)node{$\d$};
				\foreach \x/\o/\y/\r in {B/K/C/2,A'/H/A/2,C/M/A'/2,B/K/C/2,K/N/A/2,B/N/A/2} \draw ($(\o)!\r mm!(\x)$)--($($(\o)!\r mm!(\x)$)+($(\o)!\r mm!(\y)$)-(\o)$)--($(\o)!\r mm!(\y)$);	
		\end{tikzpicture}}
		\noindent Do đó  $CM\perp AA’\Rightarrow KN\parallel CM$
		$ \Rightarrow\dfrac{AK}{AC}=\dfrac{AN}{AM}=\dfrac{NK}{MC} $.\\
		Xét $\triangle ABC$ vuông tại $B$ có $BK$ là đường cao suy ra $BK=\dfrac{BA\cdot BC}{AC}=\sqrt{2}$ và \\
		$AB^2=AK\cdot AC\Rightarrow AK=\dfrac{AB^2}{AC}=2$.\\
		Xét $\triangle NKB$ vuông tại $K$ có $\tan\alpha=\tan\widehat{KNB}=\dfrac{3}{4}\Rightarrow\dfrac{KB}{KN}=\dfrac{3}{4}\Leftrightarrow KN=\dfrac{4\sqrt{2}}{3}$.\\
		Xét $\triangle ANK$ vuông tại $N$ có $KN=\dfrac{4\sqrt{2}}{3}$, $AK=2$ suy ra $AN=\dfrac{2}{3}$.\\
		Khi đó 
		$$ \dfrac{2}{3}=\dfrac{\dfrac{2}{3}}{AM}=\dfrac{\dfrac{4\sqrt{2}}{3}}{MC}\Rightarrow\heva{&AM=1\Rightarrow AA’=2\\&CM=2\sqrt{2}.} $$
		Ta lại có $A’H\cdot AC=CM\cdot AA’\Rightarrow A’H=\dfrac{CM\cdot AA’}{AC}=\dfrac{2\sqrt{2}\cdot 2}{3}=\dfrac{4\sqrt{2}}{3}$.\\
		Suy ra thể tích khối lăng trụ cần tìm là $V=A’H\cdot AB\cdot AD=\dfrac{4\sqrt{2}}{3}\cdot\sqrt{6}\cdot\sqrt{3}=8$.}
\end{ex}
\begin{ex}%[2H1G3-2]
	(Đô Lương 4 - Nghệ An - 2020) Cho hình lăng trụ $ABC.A’B’C’$ có đáy $ABC$ là tam giác vuông tại $A$, cạnh $BC=2a$ và $\widehat{ABC}=60^{\circ}$. Biết tứ giác $BCC’B’$ là hình thoi có $\widehat{B’BC}$ nhọn. Biết $(BCC’B’)$ vuông góc với $(ABC)$ và $(ABB’A’)$ tạo với $(ABC)$ góc $45^{\circ}$. Thể tích của khối lăng trụ $ABC.A’B’C’$ bằng
	\choice
	{$\dfrac{a^3}{\sqrt{7}}$}
	{\True $\dfrac{3a^3}{\sqrt{7}}$}
	{$\dfrac{6a^3}{\sqrt{7}}$}
	{$\dfrac{a^3}{3\sqrt{7}}$}
	\loigiai{
		\immini{ Gọi $H$ là chân đường cao hạ từ $B’$ của tam giác $B’BC$. Do góc $\widehat{B’BC}$ là góc nhọn nên $H$ thuộc cạnh $BC$.\\ $(BCC’B’)$ vuông góc với $(ABC)$ suy ra $B’H$ là đường cao của lăng trụ $ABC.A’B’C’$.\\
			$BCC’B’$ là hình thoi suy ra $BB’=BC=2a$.\\ Tam giác $ABC$ vuông tại $A$, cạnh $BC=2a$ và $\widehat{ABC}=60^{\circ}$ suy ra $AB=a$, $AC=a\sqrt{3}$.}{\begin{tikzpicture}[line cap=round,line join=round, >=stealth,font=\footnotesize, scale=0.75]
				\def \a{2} \def \b{-1} \def \c{5} \def \h{4.5} 
				\path (.5,.5)coordinate(B) 
				+(\a,\b)coordinate(A)
				+(\c,0)coordinate(C)
				($(B)!1/2!(A)$)coordinate(K)
				($(B)!1/3!(C)$)coordinate(H)
				+(0,\h)coordinate(B')
				($(A)+(B')-(B)$)coordinate(A')
				($(C)+(B')-(B)$)coordinate(C');
				%\draw[ultra thin,color=gray] (-.5,-1.5) grid (8.5,5.5);
				\draw[thick] (B)--(A)--(C) (B')--(A')--(C')--(B') (B)--(B') (A)--(A') (C)--(C') (B')--(K);
				\draw [dashed] (B)--(C) (B) -- (H) (B')--(H) (A)--(K)--(H);
				\foreach \x/\g in{A/-90,B/-90, C/0,A'/90, B'/80, C'/40, K/220, H/-90}
				\fill[black](\x) circle (1pt)($(\x)+(\g:3.5mm)$) node{\small $\x$};
				\foreach \x/\o/\y/\r in {B'/H/B/2} \draw ($(\o)!\r mm!(\x)$)--($($(\o)!\r mm!(\x)$)+($(\o)!\r mm!(\y)$)-(\o)$)--($(\o)!\r mm!(\y)$);	
		\end{tikzpicture}}
		\noindent	Gọi $K$ là hình chiếu của $H$ lên $AB$, do tam giác $ABC$ là tam giác vuông tại $A$ nên $HK\parallel AC\Rightarrow\dfrac{BK}{BA}=\dfrac{BH}{BC}\Rightarrow BH=2BK$.\\
		Khi đó mặt phẳng $(B’HK)$ vuông góc với $AB$ nên góc giữa hai mặt phẳng $(ABB’A’)$ và $(ABC)$ là góc $\widehat{B’KH}$. Theo giả thiết, $\widehat{B’KH}=45^{\circ}\Rightarrow B’K=h\sqrt{2}$, với $B’H=h$.\\
		Xét tam giác vuông $B’BH$ có $B’H^2+BH^2=B’B^2$ hay $h^2+4BK^2=4a^2(1)$.\\
		Xét tam giác vuông $B’BK\colon B’K^2+BK^2=B’B^2$ hay $2h^2+BK^2=4a^2(2)$.\\
		Từ $(1)$ và $(2)$ ta có $h=\dfrac{2\sqrt{3}a}{\sqrt{7}}$.\\
		Vậy thể tích khối lăng trụ $ABC.A’B’C’$ bằng $V=S_{ABC}\cdot h=\dfrac{1}{2}AB\cdot BC\cdot h=\dfrac{3a^3}{\sqrt{7}}$.}
\end{ex}
\begin{ex}%[2H1G3-2]
	(Chuyên Lê Quý Đôn – Điện Biên 2019) Cho lăng trụ $ABC.A’B’C’$ có đáy là tam giác đều cạnh $a$, hình chiếu vuông góc của điểm $A’$ lên mặt phẳng $(ABC)$ trùng với trọng tâm tam giác $ABC$. Biết khoảng cách giữa hai đường thẳng $AA’$ và $BC$ bằng $\dfrac{a\sqrt{3}}{4}$. Tính theo $a$ thể tích khối lăng trụ đó. 
	\choice
	{\True $\dfrac{a^3\sqrt{3}}{12}$}
	{$\dfrac{a^3\sqrt{3}}{6}$}
	{$\dfrac{a^3\sqrt{3}}{3}$}
	{$\dfrac{a^3\sqrt{3}}{24}$}
	\loigiai{
		\immini{
			\begin{itemize}
				\item Gọi $M$ là trung điểm $BC$, $H$ là trọng tâm tam giác $ABC\Rightarrow A’H\perp(ABC)$.
				\item $\heva{&AM\perp BC\\&	AH\perp BC}
				\Rightarrow BC\perp(AA’M) $.
				\item Gọi $N$ là hình chiếu của $M$ trên $AA'$ thì		
				$MN $ là đoạn vuông góc chung của $AA’$ và $BC$.
			\end{itemize}
		}{\begin{tikzpicture}[line cap=round,line join=round, >=stealth,font=\footnotesize, scale=0.75]
				\def \a{1} \def \b{-1} \def \c{6} \def \h{4.5} 
				\path (.5,.5)coordinate(A) 
				+(\a,\b)coordinate(B)
				+(\c,0)coordinate(C);
				\coordinate (M) at ($(B)!1/2!(C)$);
				\coordinate (H) at ($(A)!2/3!(M)$);
				\coordinate (N) at ($(A)!1/3!(A')$);
				\path (H)+(0,\h)coordinate(A')
				($(B)+(A')-(A)$)coordinate(B')
				($(C)+(A')-(A)$)coordinate(C');
				%\draw[ultra thin,color=gray] (-.5,-1.5) grid (8.5,5.5);
				\draw (A)--(B)--(C) (B)--(A')--(B')--(C')--(A') (A)--(A') (B)--(B') (C)--(C') ;
				\draw [dashed] (A')--(M)--(A)--(C) (A')--(H) (M)--(N);
				\foreach \x/\g in{A/180,B/-90, C/0,A'/90, B'/80, C'/40, M/-90,H/-140,N/180}
				\fill[black](\x) circle (1pt)($(\x)+(\g:3.5mm)$) node{\small $\x$};
				\foreach \x/\o/\y/\r in {A'/H/M/2,M/N/A/2} \draw ($(\o)!\r mm!(\x)$)--($($(\o)!\r mm!(\x)$)+($(\o)!\r mm!(\y)$)-(\o)$)--($(\o)!\r mm!(\y)$);	
		\end{tikzpicture}}
		\begin{itemize}
			\item  Khoảng cách giữa hai đường thẳng $AA’$ và $BC$ bằng $\dfrac{a\sqrt{3}}{4}$ nên
			$MN=\dfrac{a\sqrt{3}}{4}$.
			\item Tam giác $AA’M$ có $S_{\triangle AA’M}=\dfrac{1}{2}A’H\cdot AM=\dfrac{1}{2}MN\cdot A A’$ \\
			$ \Rightarrow A’H\cdot AM=MN\cdot A A’\Leftrightarrow A’H\cdot AM=MN\cdot\sqrt{A’H^2+AH^2} $ \\
			$ \Rightarrow A’H=\dfrac{MN\cdot\sqrt{A’H^2+AH^2}}{AM}=\dfrac{\dfrac{a\sqrt{3}}{4}\sqrt{A’H^2+\left(\dfrac{2}{3}\dfrac{a\sqrt{3}}{2}\right)^2}}{\dfrac{a\sqrt{3}}{2}}=\dfrac{\sqrt{A’H^2+\left(\dfrac{a\sqrt{3}}{3}\right)^2}}{2} $ \\
			$ \Rightarrow 4A’H^2=A’H^2+\left(\dfrac{a\sqrt{3}}{3}\right)^2\Rightarrow A’H=\dfrac{a}{3} $.
		\end{itemize}
		Vậy thể tích khối lăng trụ $V_{ABC.A’B’C’}=A’H\cdot S_{\triangle ABC}=\dfrac{a}{3}\cdot\dfrac{a^2\sqrt{3}}{4}=\dfrac{a^3\sqrt{3}}{12}$.}
\end{ex}
\begin{ex}%[2H1G3-2]
	(Bỉm Sơn - Thanh Hóa - 2019) Cho hình chóp $S.ABC$ có $SA$ vuông góc với mặt phẳng $(ABC)$ và tam giác $ABC$ cân tại $A$. Cạnh bên $SB$ lần lượt tạo với mặt phẳng đáy, mặt phẳng trung trực của $BC$ các góc bằng $30^{\circ}$ và $45^{\circ}$, khoảng cách từ $S$ đến cạnh $BC$ bằng $a$. Thể tích khối chóp $S.ABC$ bằng 
	\choice
	{$V_{S.ABC}=\dfrac{a^3}{2}$}
	{$V_{S.ABC}=\dfrac{a^3}{3}$}
	{\True $V_{S.ABC}=\dfrac{a^3}{6}$}
	{$V_{S.ABC}=a^3$}
	\loigiai{
		\immini{
			\begin{itemize}
				\item Lấy $M$ là trung điểm của $BC$, tam giác $ABC$ cân tại $A$
				nên $AM\perp BC$ kết hợp với 
				$SA\perp BC$ suy ra
				$BC\perp(SAM)$. Do đó $(SAM)$ là mặt phẳng trung trực cạnh $BC$.
				\item Góc giữa $SB$ và mặt phẳng $(SAM)$ bằng  góc giữa $SB$ và $SM$. Suy ra $\widehat{BSM}=45^{\circ}$.\\
				\item Góc giữa $SB$ và mặt phẳng $(ABC)$ bằng góc giữa $SB$ và $AB$ nên $\widehat{SBA}=30^{\circ}$.
				\item
				$BC\perp(SAM)\Rightarrow BC\perp SM$. Suy ra khoảng cách từ $S$ đến cạnh $BC$ bằng $SM=a$.	
		\end{itemize}}{	\begin{tikzpicture}[line cap=round,line join=round,>=stealth, font=\footnotesize,scale=0.8]
				\def\k{0.5}
				\coordinate (A) at (0,0);
				\coordinate (B) at (8,0);
				\coordinate (C) at (5,-2.5);
				\coordinate(M) at ($(B)!\k!(C)$);		
				\coordinate(S) at ($(A)+(0,6)$);
				\draw (S)--(A)--(C)--(B)--(S)--(C) (S)--(M);
				\draw[dashed] (M)--(A)--(B);
				\foreach \d/\g in{A/-90,B/-90,S/90,M/0, C/-90}
				\draw[fill=black](\d)circle(1.0pt)node[shift={(\g:0.45)}]{$\d$};	
				\foreach \x/\o/\y/\r in {S/A/B/3,S/A/C/2,C/M/A/2,C/M/S/2} \draw ($(\o)!\r mm!(\x)$)--($($(\o)!\r mm!(\x)$)+($(\o)!\r mm!(\y)$)-(\o)$)--($(\o)!\r mm!(\y)$);
		\end{tikzpicture}}
		\begin{itemize}
			\item Tam giác vuông cân $SBM$ có $BM=a,SB=a\sqrt{2}$ 
			$ \Rightarrow BC=2BM=2a $.
			\item Tam giác vuông $SAB$ có $sin 30^{\circ}=\dfrac{SA}{SB}\Rightarrow SA=a\sqrt{2}\cdot\dfrac{1}{2}=\dfrac{a\sqrt{2}}{2}$; $AB=\dfrac{a\sqrt{6}}{2}$.
			\item Tam giác vuông $ABM$ có $AM=\sqrt{AB^2-BM^2}=\sqrt{\left(\dfrac{a\sqrt{6}}{2}\right)^2-a^2}=\dfrac{a\sqrt{2}}{2}$.
		\end{itemize}
		Vậy thể tích khối chóp $S.ABC$ là $V_{S.ABC}=\dfrac{1}{3}SA\cdot S_{\triangle ABC}=\dfrac{1}{3}\cdot\dfrac{a\sqrt{2}}{2}\cdot\dfrac{1}{2}\cdot 2a\cdot\dfrac{a\sqrt{2}}{2}=\dfrac{a^3}{6}$.}
\end{ex}
\begin{ex}%[2H1G3-2]
	[Chu Văn An - Hà Nội - 2019] Cho tứ diện $ABCD$ có $BC=BD=AC=AD=1,(ACD)\perp(BCD)$ và $(ABD)\perp(ABC)$. Thể tích của tứ diện $ABCD$ bằng
	\choice
	{$\dfrac{2\sqrt{3}}{9}$}
	{\True $\dfrac{\sqrt{3}}{27}$}
	{$\dfrac{2\sqrt{3}}{27}$}
	{$\dfrac{2\sqrt{2}}{27}$}
	\loigiai{
		\begin{center}
			\begin{tikzpicture}[scale=1, font=\footnotesize, line join=round, line cap=round, >=stealth]
				\def\ac{4} % cạnh AC
				\def\ab{2} % cạnh AB
				\def\as{4} % cạnh AS
				\def\gocA{50} % góc A của đáy
				\coordinate[label=left:$A$] (A) at (0,0);
				\coordinate[label=right:$C$] (C) at (\ac,0);
				\coordinate[label=below left:$B$] (B) at (-\gocA:\ab);
				\coordinate[label=above:$D$] (D) at (70:\as);
				\coordinate[label=below left:$K$] (K) at ($(A)!0.5!(B)$) ;
				\coordinate[label=above right:$H$] (H) at ($(C)!0.5!(D)$) ;
				\draw (A)--(B)--(C)--(D)--cycle (D)--(B)--(H);
				\draw[dashed] (A)--(C) (K)--(H) ;
				\foreach \diem in {A,B,C,D,K,H}\fill (\diem)circle(1.5pt);
			\end{tikzpicture}
			
		\end{center}
		Gọi $H,K$ lần lượt là trung điểm cạnh $CD, AB$.\\
		Đặt $AH=x,(x>0)$.\\
		• $\triangle ACD$ và $\triangle BCD$ lần lượt cân tại $A$ và $D$ nên $AH$ và $BH$ là hai đường cao tương ứng.\\
		$\heva{&(ACD)\perp(BCD)\\&(ACD)\cap(BCD)=CD\\&(ACD)\supset AH\perp CD}\Rightarrow AH\perp(BCD)$.\\
		Do đó $AH\perp BH(1)$.\\
		$\triangle ACD=\triangle BCD(c\cdot c\cdot c)$ do đó $AH=BH$ (2 đường cao tương ứng) (2).\\
		Từ (1), (2) suy ra $\triangle AHB$ vuông cân tại $H$ \\
		$ \Rightarrow AB=AH\sqrt{2}=x\sqrt{2} $. (3).\\
		• Chứng minh tương tự ta được $\triangle CKD$ vuông cân tại $K$ \\
		$ \Rightarrow CK=\dfrac{CD}{\sqrt{2}}=\dfrac{2\cdot HD}{\sqrt{2}}=\sqrt{2}\cdot\sqrt{AD^2-AH^2}=\sqrt{2}\cdot\sqrt{1-x^2} $.\\
		Mặt khác, $\triangle ACD$ cân tại A có $CK$ là đường cao nên:\\
		$AB=2AK=2\sqrt{AC^2-CK^2}=2\sqrt{1-2\left(1-x^2\right)}$ (4).\\
		Từ (3), (4) ta có:\\
		$\begin{aligned}&x\sqrt{2}=2\sqrt{1-2\left(1-x^2\right)}\\&\Leftrightarrow 2x^2=4\left(2x^2-1\right)\\&\Leftrightarrow x^2=\dfrac{2}{3}\Leftrightarrow x=\dfrac{\sqrt{6}}{3}(x>0).\end{aligned}$\\
		$CD=2\cdot HD=2\sqrt{1-AH^2}=\dfrac{2\sqrt{3}}{3}$.\\
		$V_{ABCD}=\dfrac{1}{3}AH\cdot S_{\triangle BCD}=\dfrac{1}{3}\cdot\dfrac{\sqrt{6}}{3}\cdot\dfrac{1}{2}\cdot\dfrac{\sqrt{6}}{3}\cdot\dfrac{2\sqrt{3}}{3}=\dfrac{\sqrt{3}}{27}$.}
\end{ex}
\begin{ex}%[2H1G3-2]
	[Chuyên Đại học Vinh - 2019] Cho hình chóp tứ giác đều $S.ABCD$ có $SA=a\sqrt{11}$, cosin góc hợp bởi hai mặt phẳng $(SBC)$ và $(SCD)$ bằng $\dfrac{1}{10}$. Thể tích của khối chóp $S.ABCD$ bằng
	\choice
	{$3a^3$}
	{$9a^3$}
	{\True $4a^3$}
	{$12a^3$}
	\loigiai{
		\begin{center}
			\begin{tikzpicture}[scale=1, font=\footnotesize, line join=round, line cap=round, >=stealth]
				\def\bc{4} % cạnh BC
				\def\ba{2} % cạnh BA
				\def\h{4} % đường cao
				\def\gocB{30} % góc B của đáy
				\coordinate[label=below left:$B$] (B) at (0,0);
				\coordinate[label=above right:$A$] (A) at (\gocB:\ba);
				\coordinate[label=below:$C$] (C) at (\bc,0);
				\coordinate[label=right:$D$] (D) at ($(C)-(B)+(A)$);
				\coordinate[label=below:$H$] (O) at ($(A)!.5!(C)$);
				\coordinate[label=above:$S$] (S) at ($(O)+(90:\h)$);
				\draw (B)--(C)--(D)--(S)--cycle (S)--(C);
				\draw[dashed] (C)--(A)--(D)--(B) (O)--(S)--(A)--(B);
				\foreach \diem in {A,B,C,D,S,O}	\fill (\diem)circle(1.5pt);
			\end{tikzpicture}
		\end{center}		
		Gọi $H$ là tâm của hình vuông $ABCD$ nên $SH\perp (ABCD)$. Đặt $m=HA$, $n=SH$. Do tam giác $SAH$ vuông tại H nên $m^2+n^2=11a^2$.\\
		Xây dựng hệ trục tọa độ như sau: $H(0;0;0)$, $B(m;0;0)$, $D(-m;0;0)$, $C(0;m;0)$, $S(0;0;n)$.\\
		Khi đó phương trình mặt phẳng $(SBC)$ là $\dfrac{x}{m}+\dfrac{y}{m}+\dfrac{z}{n}=1$ hay véc-tơ pháp tuyến của mặt phẳng $(SBC)$ là $\overrightarrow{n_1}=(n;n;m)$.\\
		Khi đó phương trình mặt phẳng $(SCD)$ là $\dfrac{x}{-m}+\dfrac{y}{m}+\dfrac{z}{n}=1$ hay véc-tơ pháp tuyến của mặt phẳng $(SBC)$ là $\overrightarrow{n_2}=(n;-n;-m)$.\\
		Do cosin góc hợp bởi hai mặt phẳng $(SBC)$ và $(SCD)$ bằng $\dfrac{1}{10}$ nên $\dfrac{1}{10}=\dfrac{|\overrightarrow{n_1}\cdot\overrightarrow{n_2}|}{|\overrightarrow{n_1}|\cdot|\overrightarrow{n_2}|}$ hay $\dfrac{m^2}{2n^2+m^2}=\dfrac{1}{10}$ mà $n^2=11a^2-m^2$.\\
		Vậy $\dfrac{m^2}{2n^2+m^2}=\dfrac{1}{10}\Leftrightarrow\dfrac{m^2}{22a^2-m^2}=\dfrac{1}{10}\Leftrightarrow m^2=2a^2\Rightarrow m=a\sqrt{2}\Rightarrow SH=3a$.\\
		$m=HA=a\sqrt{2}$ nên $AB=2a$,\\
		Chiều cao của hình chóp là $SH=3a$.\\
		Diện tích của hình vuông là $S_{ABCD}=4a^2$.\\
		Thể tích của khối chóp $S.ABCD$ là $V=\dfrac{1}{3}S_{ABCD}\cdot SH=\dfrac{1}{3}\cdot 4a^2\cdot 3a=4a^3$.}
\end{ex}
\begin{ex}%[2H1G3-2]
	[THPT Lương Thế Vinh Hà Nội 2019] Cho hình chóp $S.ABC$ có đáy $ABC$ là tam giác đều cạnh $1$, biết khoảng cách từ $A$ đến $(SBC)$ là $\dfrac{\sqrt{6}}{4}$, từ $B$ đến $(SCA)$ là $\dfrac{\sqrt{15}}{10}$, từ $C$ đến $(SAB)$ là $\dfrac{\sqrt{30}}{20}$ và hình chiếu vuông góc của $S$ xuống đáy nằm trong tam giác $ABC$. Tính thể tích khối chóp $V_{S.ABC}$. 
	\choice
	{$\dfrac{1}{36}$}
	{\True $\dfrac{1}{48}$}
	{$\dfrac{1}{12}$}
	{$\dfrac{1}{24}$}
	\loigiai{
		\begin{center}
			\begin{tikzpicture}[scale=1, font=\footnotesize, line join=round, line cap=round, >=stealth]
				\def\ac{5} % cạnh AC
				\def\ab{3} % cạnh AB
				\def\as{4} % cạnh AS
				\def\gocA{70} % góc A của đáy
				\coordinate[label=left:$A$] (A) at (0,0);
				\coordinate[label=right:$C$] (C) at (\ac,0);
				\coordinate[label=below left:$B$] (B) at (-\gocA:\ab);
				\coordinate[label=above:$S$] (S) at (60:\as);
				\coordinate [label=below:$H$] (H) at ($(S)-(0,4.5)$);
				\coordinate[label=above:$M$] (M) at ($(A)!(H)!(C)$);
				\coordinate[label=right:$N$] (N) at ($(B)!(H)!(C)$);
				\coordinate[label=left:$P$] (P) at ($(B)!(H)!(A)$);	
				
				\draw (A)--(B)--(C)--(S)--cycle (S)--(B) (P)--(S)--(N);
				\draw[dashed] (A)--(C) (S)--(M)--(H)--(N) (H)--(P);
				\foreach \diem in {A,B,C,S,H,M,N,P}\fill (\diem)circle(1.5pt);
			\end{tikzpicture}
			
		\end{center}
		Gọi $M,N,P$ lần lượt là hình chiếu của $H$ lên các cạnh $AC,BC,AB$.\\
		Đặt $SH=h\Rightarrow V_{S.ABC}=\dfrac{1}{3}\cdot h\cdot\dfrac{\sqrt{3}}{4}=\dfrac{h\sqrt{3}}{12}$.\\
		Ta có $AP=\dfrac{2S_{SAB}}{AB}=2S_{SAB}=\dfrac{6V_{S.ABC}}{\mathrm{d}\left(C;(SAB)\right)}=\dfrac{h\sqrt{3}}{2}\colon\dfrac{\sqrt{30}}{20}=h\sqrt{10}$.\\
		Tương tự, tính được $HM=2h,HN=h$ \\
		$ \Rightarrow PH=\sqrt{SP^2-SH^2}=3h $.\\
		Ta có $S_{ABC}=S_{HAB}+S_{HAC}+S_{HBC}=\dfrac{1}{2}(HP+HM+HN)\Leftrightarrow 3h=\dfrac{\sqrt{3}}{4}\Leftrightarrow h=\dfrac{\sqrt{3}}{12}$.\\
		Vậy $V_{S.ABC}=\dfrac{\sqrt{3}}{12}\cdot\dfrac{\sqrt{3}}{12}=\dfrac{1}{48}$.}
\end{ex}
\begin{ex}%[2H1G3-2]
	[Cụm Liên Trường Hải Phòng 2019] Cho hình chóp $S.ABC$ có đáy là tam giác đều cạnh $a$. $\widehat{SAB}=\widehat{SCB}=90^{\circ}$. Gọi $M$ là trung điểm của $SA$. Khoảng cách từ $A$ đến mặt phẳng $(MBC)$ bằng $\dfrac{6a}{7}$. Tính thể tích $V$ của khối chóp $S.ABC$. 
	\choice
	{$V=\dfrac{5\sqrt{3}a^3}{12}$}
	{\True $V=\dfrac{5\sqrt{3}a^3}{6}$}
	{$V=\dfrac{4\sqrt{3}a^3}{3}$}
	{$V=\dfrac{7\sqrt{3}a^3}{12}$}
	\loigiai{
		\begin{center}
			\begin{tikzpicture}[scale=1.2, font=\footnotesize, line join=round, line cap=round, >=stealth]
				\coordinate [label=below:$B$] (B) at (2,0);
				\coordinate [label= right: $C$] (C) at (4,2);
				\coordinate [label=left:$A$] (A) at (-2,2);
				\coordinate [label= above:$S$] (S) at (1,5);
				\coordinate [label=above right:$I$] (I) at ($(S)!0.5!(B)$);
				\coordinate [label= below:$O$] (O) at ($(I)-(0,1.2)$);
				\coordinate [label= above left:$H$] (H) at ($(B)!2!(O)$);
				\coordinate [label= above left:$M$] (M) at ($(S)!0.5!(A)$);
				\tkzInterLL(A,I)(M,B), \tkzGetPoint{J};
				\coordinate [label=above:$J$] (J) at (J);
				\coordinate [label=right:$D$] (D) at ($(B)!0.5!(C)$);
				\coordinate [label= below:$N$] (N) at ($(J)-(0,0.8)$);
				\coordinate [label= above right:$E$] (E) at ($(J)!(N)!(D)$);
				\draw[dashed] (A)--(C) (S)--(H)--(B) (O)--(I) (A)--(D) (D)--(J)--(N) (M)--(C) (N)--(E);
				\draw (A)--(B)--(C)--(S)--(A) (S)--(B) (A)--(I) (M)--(B) (I)--(D);
				\foreach \diem in {A,B,C,I,O,H,M,J,D,N,E}\fill (\diem)circle(1.5pt);
			\end{tikzpicture}
			
		\end{center}
		Vì $\widehat{SAB}=\widehat{SCB}=90^{\circ}\Rightarrow S,A,B,C$ cùng thuộc mặt cầu đường kính $SB$.\\
		Gọi $D$ là trung điểm $BC$, $I$ là trung điểm $SB$ và $O$ là tâm đường tròn ngoại tiếp $\triangle ABC$, ta có $OI\perp(ABC)$.\\
		Gọi $H$ là điểm đối xứng với $B$ qua $O\Rightarrow SH\perp(ABC)$ (vì $OI$ là đường trung bình $\triangle SHB$).\\
		Gọi $BM\cap AI=J$, ta có $J$ trọng tâm $\triangle SAB$.\\
		Trong $\triangle AID$, kẻ $JN\parallel IO$. Khi đó, vì $BC\perp(JND)$ nên $(JND)\perp(MBC)$.\\
		Kẻ $NE\perp JD$, ta có $NE\perp(MBC)$. Do đó $\mathrm{d}\left(N;(MBC)\right)=NE$.\\
		Ta có $\dfrac{\mathrm{d}\left(A,(MBC)\right)}{\mathrm{d}\left(N,(MBC)\right)}=\dfrac{AD}{ND}=\dfrac{AD}{AD-AN} =\dfrac{AD}{AD-\dfrac{2}{3}AO}=\dfrac{AD}{AD-\dfrac{4}{9}AD}=\dfrac{9}{5}$.\\
		Suy ra, $\mathrm{d}\left(N,(MBC)\right)=\dfrac{5}{9}\mathrm{d}\left(A,(MBC)\right)=\dfrac{10a}{21}$.\\
		Xét $\triangle JND$ có $\dfrac{1}{NE^2}=\dfrac{1}{ND^2}+\dfrac{1}{NJ^2}$ nên $NJ=\dfrac{10a}{3}\Rightarrow OI=\dfrac{3}{2}NJ=5a\Rightarrow SH=10a$.\\
		Vậy $V_{SABC}=\dfrac{1}{3}SH\cdot S_{ABC}=\dfrac{1}{3}\cdot 10a\cdot\dfrac{a^2\sqrt{3}}{4}=\dfrac{5\sqrt{3}a^3}{6}$.}
\end{ex}
\begin{ex}%[2H1G3-2]
	[Chuyên Vĩnh Phúc 2019] Cho hình chóp $S.ABC$ có các cạnh $SA=BC=3$; $SB=AC=4$; $SC=AB=2\sqrt{5}$. Tính thể tích khối chóp $S.ABC$. 
	\choice
	{$\dfrac{\sqrt{390}}{12}$}
	{\True $\dfrac{\sqrt{390}}{4}$}
	{$\dfrac{\sqrt{390}}{6}$}
	{$\dfrac{\sqrt{390}}{8}$}
	\loigiai{
		\begin{center}
			\begin{tikzpicture}[scale=1.2, font=\footnotesize, line join=round, line cap=round, >=stealth]
				\path  (-1,4) coordinate [label=above:$S$] (S)
				(0,0) coordinate [label=below:$A$] (A)
				(4,0) coordinate [label=right:$C$] (C)
				(2,-2) coordinate [label=below:$B$] (B)
				(6,-2) coordinate [label=below right:$A'$] (A')
				(-2,-2) coordinate [label=below left:$C'$] (C')
				($(A')!2!(C)$) coordinate [label=above:$B'$] (B')
				;
				\draw (A')--(C')--(S)--(C) (S)--(A') (S)--(B') (A')--(B');
				\draw [dashed] (C')-- (A)--(C) (A)--(B) (S)--(A) (A)--(B)--(C) (A)--(B');
				\foreach \diem in {S,A,B,C,A',C',B'}\fill (\diem)circle(1.5pt);
			\end{tikzpicture}
			
		\end{center}
		+ Dựng hình chóp $S.A’B’C’$ sao cho A là trung điểm $B’C’$, $B$ là trung điểm $A’C’$, $C$ là trung điểm $A’B’$.\\
		+ Khi đó $SB=AC=BA’=BC’=4$ nên $\triangle SA’C’$ vuông tại $S$ và $SA’^2+SC’^2=(2\cdot SB)^2=64 (1)$.\\
		+ Tương tự $\triangle SB’C’$, $\triangle SA’B’$ vuông tại $S$ và $\heva{&SA’^2+SB’^2=80 (2)\\&SB’^2+SC’^2=36 (3).}$ \\
		+ Từ $(1);(2);(3)$ ta suy ra $SC’=\sqrt{10}$; $SB’=\sqrt{26}$; $SA’=\sqrt{54}$.\\
		+ Ta tính được $V_{S.A’B’C’}=\dfrac{1}{3}SC’\cdot\dfrac{1}{2}\cdot SA’\cdot SB’=\sqrt{390}$ và $V_{S.ABC}=\dfrac{1}{4}V_{S.A’B’C’}=\dfrac{\sqrt{390}}{4}$ (đvtt).}
\end{ex}
\begin{ex}%[2H1G3-2]
	Cho hình chóp $S.ABC$ có $\widehat{ASB}=\widehat{CSB}=60^{\circ}$, $\widehat{ASC}=90^{\circ}$, $SA=SB=a$, $SC=3a$. Tính thể tích của khối chóp $S.ABC$. 
	\choice
	{\True $\dfrac{a^3\sqrt{2}}{4}$}
	{$\dfrac{a^3\sqrt{6}}{18}$}
	{$\dfrac{a^3\sqrt{2}}{12}$}
	{$\dfrac{a^3\sqrt{6}}{6}$}
	\loigiai{
		\begin{center}
			\begin{tikzpicture}[scale=1, font=\footnotesize, line join=round, line cap=round, >=stealth]
				\path  (0,0) coordinate [label=left:$A$] (A) 
				(1,-1.5) coordinate [label=below:$B$] (B) 
				(6,-4) coordinate [label=right:$C$] (C) 
				(2,4) coordinate [label=above:$S$] (S) 
				
				($(S)!1/3!(C)$) coordinate [label=right:$M$] (M)
				($(A)!0.5!(M)$) coordinate [label=below:$I$] (I) 
				;	
				\draw (S)--(A)--(B)--(C)--(S) (B)--(M);
				\draw [dashed] (A)--(I)--(B) (S)--(I)--(C) (I)--(M);
				\foreach \diem in {A,B,C,S,M,I}\fill (\diem)circle(1.5pt);	
			\end{tikzpicture}	
		\end{center}
		Cách 1.
		Gọi $M$ là điểm nằm trên $SC$ sao cho $SM=\dfrac{1}{3}SC=a$.\\
		Ta có:\\
		Tam giác $SAM$ vuông tại $S\Rightarrow AM=\sqrt{SA^2+SM^2}=a\sqrt{2}$.\\
		Tam giác $SBM$ là tam giác đều có độ dài cạnh $SM=SB=BM=a$.\\
		Tam giác $SAB$ là tam giác đều có độ dài cạnh $SA=SB=AB=a$.\\
		Vậy $AB^2+BM^2=AM^2\Rightarrow$ Tam giác $ABM$ là tam giác vuông tại $B$ \\
		$ \Rightarrow\triangle ABM=\triangle ASM\Rightarrow SI=IB=\dfrac{a\sqrt{2}}{2}\Rightarrow IB^2+SI^2=SB^2\Rightarrow $ Tam giác $SIB$ vuông tại $I$ \\
		$ \Rightarrow\heva{&SI\perp IB\\&SI\perp AM}\Rightarrow SI\perp(ABM)\Rightarrow SI $ là đường cao của khối chóp $SABM$.\\
		Thể tích của khối chóp $S.ABM$ là $V_{S.ABM}=\dfrac{1}{3}\cdot S_{\triangle ABM}\cdot SI=\dfrac{1}{6}\cdot AB\cdot BM\cdot SI=\dfrac{a^3\sqrt{2}}{12}$ (đvtt).\\
		Mà $\dfrac{V_{S.ABM}}{V_{S.ABC}}=\dfrac{SM}{SC}=\dfrac{1}{3}\Rightarrow V_{S.ABC}=3\cdot V_{S.ABM}=\dfrac{a^3\sqrt{2}}{4}$.\\
		Cách 2: Ta có $V_{S.ABC}=\dfrac{abc}{6}\cdot\sqrt{1-\cos^2\alpha-\cos^2\beta-\cos^2\delta-2\cos\alpha\cdot\cos\beta\cdot\cos\delta}$.\\
		Trong đó $a=SA$; $b=SB$; $c=SC$; $\alpha=\widehat{ASB}$; $\beta=\widehat{ASC}$; $\delta=\widehat{BSC}$ \\
		$ \Rightarrow V_{S.ABC}=\dfrac{a\cdot a\cdot 3a}{6}\cdot\sqrt{1-\cos^2{60}^{\circ}-\cos^2{60}^{\circ}-\cos^2{90}^{\circ}-2\cos{60}^{\circ}\cdot\cos{60}^{\circ}\cdot\cos{90}^{\circ}}=\dfrac{a^3\sqrt{2}}{4} $ (đvtt).}
\end{ex}
\begin{ex}%[2H1G3-2]
	Cho hình chóp $S.ABC$ có đáy $ABC$ là tam giác đều cạnh $2a$. Gọi $M$ là trung điểm cạnh $SA$, $\widehat{SAB}=\widehat{SCB}=90^{\circ}$, biết khoảng cách từ $A$ đến $(MBC)$ bằng $\dfrac{6a}{\sqrt{21}}$. Thể tích của khối chóp $S.ABC$ bằng
	\choice
	{\True $\dfrac{10a^3\sqrt{3}}{9}$}
	{$\dfrac{8a^3\sqrt{39}}{3}$}
	{$\dfrac{4a^3\sqrt{13}}{3}$}
	{$2a^3\sqrt{3}$}
	\loigiai{
		\begin{center}
			\begin{tikzpicture}[scale=1.2, font=\footnotesize, line join=round, line cap=round, >=stealth]
				\coordinate [label=below:$B$] (B) at (2,0);
				\coordinate [label= right: $C$] (C) at (4,2);
				\coordinate [label=left:$A$] (A) at (-2,2);
				\coordinate [label= above:$S$] (S) at (1,5);
				\coordinate [label=above right:$I$] (I) at ($(S)!0.5!(B)$);
				\coordinate [label= below:$O$] (O) at ($(I)-(0,1.2)$);
				\coordinate [label= above left:$H$] (H) at ($(B)!2!(O)$);
				\coordinate [label= above left:$M$] (M) at ($(S)!0.5!(A)$);
				\tkzInterLL(A,I)(M,B), \tkzGetPoint{J};
				\coordinate [label=above:$J$] (J) at (J);
				\coordinate [label=right:$D$] (D) at ($(B)!0.5!(C)$);
				\coordinate [label= below:$N$] (N) at ($(J)-(0,0.8)$);
				\coordinate [label= above right:$E$] (E) at ($(J)!(N)!(D)$);
				\draw[dashed] (A)--(C) (S)--(H)--(B) (O)--(I) (A)--(D) (D)--(J)--(N) (M)--(C) (N)--(E);
				\draw (A)--(B)--(C)--(S)--(A) (S)--(B) (A)--(I) (M)--(B) (I)--(D);
				\foreach \diem in {A,B,C,I,O,H,M,J,D,N,E}\fill (\diem)circle(1.5pt);
			\end{tikzpicture}
			
		\end{center}
		Vì $\widehat{SAB}=\widehat{SCB}=90^{\circ}\Rightarrow S,A,B,C$ cùng thuộc mặt cầu đường kính $SB$.\\
		Gọi $D$ là trung điểm $BC$, $I$ là trung điểm $SB$ và $O$ là tâm đường tròn ngoại tiếp $\triangle ABC$, ta có $OI\perp(ABC)$.\\
		Gọi $H$ là điểm đối xứng với $B$ qua $O\Rightarrow SH\perp(ABC)$ (vì $OI$ là đường trung bình $\triangle SHB$).\\
		Gọi $BM\cap AI=J$, ta có $J$ trọng tâm $\triangle SAB$.\\
		Trong $\triangle AID$, kẻ $JN\parallel IO$. Khi đó, vì $BC\perp(JND)$ nên $(JND)\perp(MBC)$.\\
		Kẻ $NE\perp JD$, ta có $NE\perp(MBC)$. Do đó $\mathrm{d}\left(N;(MBC)\right)=NE$.\\
		Ta có $\dfrac{\mathrm{d}\left(A,(MBC)\right)}{\mathrm{d}\left(N,(MBC)\right)}=\dfrac{AD}{ND}=\dfrac{AD}{AD-AN} =\dfrac{AD}{AD-\dfrac{2}{3}AO}=\dfrac{AD}{AD-\dfrac{4}{9}AD}=\dfrac{9}{5}$.\\
		Suy ra, $\mathrm{d}\left(N,(MBC)\right)=\dfrac{5}{9}\mathrm{d}\left(A,(MBC)\right)=\dfrac{10a}{3\sqrt{21}}$.\\
		Xét $\triangle JND$ có $\dfrac{1}{NE^2}=\dfrac{1}{ND^2}+\dfrac{1}{NJ^2}$ nên $NJ=\dfrac{10a}{9}\Rightarrow OI=\dfrac{3}{2}NJ=\dfrac{5a}{3}\Rightarrow SH=\dfrac{10a}{3}$.\\
		Vậy $V_{SABC}=\dfrac{1}{3}SH\cdot S_{ABC}=\dfrac{1}{3}\cdot\dfrac{10a}{3}\cdot\dfrac{(2a)^2\sqrt{3}}{4}=\dfrac{10\sqrt{3}a^3}{9}$.}
\end{ex}
\begin{ex}%[2H1G3-2]
	[Cụm liên trường Hải Phòng 2019] Cho hình chóp $S.ABC$ có đáy là tam giác đều cạnh $a$. $\widehat{SAB}=\widehat{SCB}=90^{\circ}$. Gọi $M$ là trung điểm của $SA$. Khoảng cách từ $A$ đến mặt phẳng $(MBC)$ bằng $\dfrac{6a}{7}$. Tính thể tích $V$ của khối chóp $S.ABC$. 
	\choice
	{$V=\dfrac{5\sqrt{3}a^3}{12}$}
	{\True $V=\dfrac{5\sqrt{3}a^3}{6}$}
	{$V=\dfrac{4\sqrt{3}a^3}{3}$}
	{$V=\dfrac{7\sqrt{3}a^3}{12}$}
	\loigiai{
		\begin{center}
			\begin{tikzpicture}[scale=1, font=\footnotesize, line join=round, line cap=round, >=stealth]
				\path 
				(0,0) coordinate [label=left:$A$] (A)
				(1,-0.7) coordinate [label=below:$H$] (H)
				(3,-1.5) coordinate [label=below:$C$] (C)
				(6,0) coordinate [label=right:$B$] (B)
				($(B)!0.5!(C)$) coordinate [label=right:$N$] (N)
				(-0.7,4) coordinate [label=above:$S$] (S)
				($(S)!0.5!(A)$) coordinate [label=left:$M$] (M)
				($(S)!0.5!(B)$) coordinate [label=right:$I$] (I)
				
				;
				
				\tkzInterLL(A,I)(M,B), \tkzGetPoint{G}
				\coordinate [label=above:$G$] (G) at (G);
				\tkzInterLL(A,N)(H,B), \tkzGetPoint{O}
				\coordinate [label=below:$O$] (O) at (O);
				\coordinate [label=below:$K$] (K) at ($(A)!2/5!(N)$);	
				\coordinate [label=above right:$E$] (E) at ($(G)!(K)!(N)$);
				\draw (S)--(A)--(H)--(C)--(B)--(S)--(H) (S)--(C) (I)--(N);
				\draw[dashed] (A)--(B) (A)--(I) (M)--(B) (A)--(N) (G)--(K) (I)--(O) (B)--(H) (G)--(N) (K)--(E);
				
				\foreach \diem in {A,B,C,H,S,M,N,G,K,O,I,E}\fill (\diem)circle(1.5pt);
			\end{tikzpicture}
			
		\end{center}
		Gọi $I$ là trung điểm của $SB$.\\
		Do $\widehat{SAB}=\widehat{SCB}=90^{\circ}$ nên $I$ là tâm mặt cầu ngoại tiếp hình chóp $S.ABC$.\\
		Gọi $O$ là tâm của đáy $ABC\Rightarrow OI\perp (ABC)$.\\
		Gọi $H$ là hình chiếu của $S$ lên mặt phẳng $(ABC)$. Ta có $AB\perp (SAH)\Rightarrow AB\perp AH$. Tương tự, $BC\perp CH$. Suy ra $H$ thuộc đường tròn ngoại tiếp tam giác $ABC,$ có tâm là $O$ nên $O$ là trung điểm của $BH$. Do đó, $SH=2OI$.\\
		Gọi $N$ là trung điểm của $BC\Rightarrow IN\parallel SC$ nên $BC\perp IN\Rightarrow BC\perp(AIN)(*)$.\\
		Gọi $G$ là trọng tâm của tam giác $SAB$ và $K$ là hình chiếu của $G$ lên mặt phẳng $(ABC)\Rightarrow K\in AO$ và $GK\parallel OI\Rightarrow AK=\dfrac{2}{3}AO=\dfrac{4}{9}AN\Rightarrow KN=\dfrac{5}{9}AN$ \\
		$\Rightarrow d[K,(MBC)]=\dfrac{5}{9}d[A,(MBC)]=\dfrac{10a}{21}$.\\
		Kẻ $KE\perp GN\overset{(*)}{\Rightarrow} KE\perp BC\Rightarrow KE\perp(MBC)\Rightarrow d[K,(MBC)]=KE=\dfrac{10a}{21}$.\\
		Tam giác $GKN$ vuông tại $K$ có $\dfrac{1}{KE^2}=\dfrac{1}{GK^2}+\dfrac{1}{KN^2}\Rightarrow GK=\dfrac{10a}{3}\Rightarrow SH=2OI=3GK=10a$.\\
		Vậy thể tích khối chóp $S.ABC$ là $V=\dfrac{1}{3}\cdot\dfrac{a^2\sqrt{3}}{4}\cdot 10a=\dfrac{5a^3\sqrt{3}}{6}$.}
\end{ex}
\begin{ex}%[2H1G3-2]
	[Chuyên Lê Quý Đôn Điện Biên 2019] Cho tứ diện $ABCD$ có các cạnh $AD=BC=3$, $AC=BD=4$, $AB=CD=2\sqrt{3}$. Tính thể tích khối tứ diện $ABCD$. 
	\choice
	{$\dfrac{\sqrt{2740}}{12}$}
	{$\dfrac{\sqrt{2474}}{12}$}
	{$\dfrac{\sqrt{2047}}{12}$}
	{\True $\dfrac{\sqrt{2470}}{12}$}
	\loigiai{
		\begin{center}
			\begin{tikzpicture}[scale=1, font=\footnotesize, line join=round, line cap=round, >=stealth]
				\path 
				(0,0) coordinate [label=left:$A'$] (A')
				(5,0) coordinate [label=right:$C'$] (C')
				(2,-2) coordinate [label=left:$B'$] (B')
				($(A')!0.5!(C')$) coordinate [label=below:$B$] (B)
				($(B')!0.5!(C')$) coordinate [label=right:$A$] (A)
				($(A')!0.5!(B')$) coordinate [label=left:$C$] (C)
				(2.5,3) coordinate [label=above:$D$] (D)		
				;
				\draw (A')--(B')--(C') (D)--(A') (D)--(C') (D)--(B') (D)--(C) (D)--(A) ;
				\draw [dashed] (A)--(B)--(C)--(A) (D)--(B) (A')--(C');
				
				\foreach \diem in {A,B,C,A',B',C',D}\fill (\diem)circle(1.5pt);
			\end{tikzpicture}			
		\end{center}
		Dựng tứ diện $D.A’B’C’$ sao cho $A$, $B$, $C$ lần lượt là trung điểm của $B’C’$, $A’C’$, $A’B’$.\\
		Theo cách dựng và theo bài ra có: $AC=BC’=BD$.\\
		Xét tam giác $DA’C’$ có: $BD$ là đường trung tuyến và $A’B=BC’=BD\Rightarrow\triangle DA’C’$ vuông tại $D$.\\
		Chứng minh tương tự ta cũng có: $\triangle DB’C’$, $\triangle DA’B’$ vuông tại $D$.\\
		Khi đó tứ diện $D.A’B’C’$ có các cạnh $DA’$, $DB’$, $DC’$ đôi một vuông góc với nhau.\\
		Ta có: $V_{ABCD}=\dfrac{1}{4}V_{D.A’B’C’}=\dfrac{1}{24}DA’\cdot DB’\cdot DC’$.\\
		Theo bài ra ta có: $\heva{&DA’^2+DB’^2=48\\&DA’^2+DC’^2=64\\&DB’^2+DC’^2=36}\Leftrightarrow\heva{&DA’^2=38\\&DB’^2=10\\&DC’^2=26}\Leftrightarrow\heva{&DA’=\sqrt{38}\\&DB’=\sqrt{10}\\&DC’=\sqrt{26}.}$ \\
		Vậy $V_{ABCD}=\dfrac{1}{24}DA’\cdot DB’\cdot DC’=\dfrac{1}{24}\cdot\sqrt{38}\cdot\sqrt{10}\cdot\sqrt{26}=\dfrac{\sqrt{2470}}{12}$.}
\end{ex}
\begin{ex}%[2H1G3-2]
	Cho tứ diện $ABCD$ có $\widehat{DAB}=\widehat{CBD}=90^{\circ}; AB=a; AC=a\sqrt{5};\widehat{ABC}=135^{\circ}$. Biết góc giữa hai mặt phẳng $(ABD)$, $(BCD)$ bằng $30^{\circ}$. Thể tích của tứ diện $ABCD$ là
	\choice
	{$\dfrac{a^3}{\sqrt{2}}$}
	{$\dfrac{a^3}{3\sqrt{2}}$}
	{\True $\dfrac{a^3}{6}$}
	{$\dfrac{a^3}{2\sqrt{3}}$}
	\loigiai{
		\begin{center}
			\begin{tikzpicture}[scale=1, font=\footnotesize, line join=round, line cap=round, >=stealth]
				\path 
				(0,0) coordinate [label=below:$H$] (H)
				(-1,-2)	 coordinate [label=below left:$A$] (A)
				(3,-2)  coordinate [label=below right:$B$] (B)	
				(5,0)   coordinate [label=above right:$C$] (C)	
				($(H) +(0,4)$)   coordinate [label=above :$D$] (D)
				($(A)!(H)!(D)$)   coordinate [label=left:$E$] (E)
				($(B)!(H)!(D)$)   coordinate [label=right:$F$] (F)		
				;
				\draw (A)--(B)--(C)  (D)--(C) (D)--(A) (D)--(C) (D)--(B) (E)--(F);
				\draw [dashed] (A)-- (H)--(C) (H)--(D) (H)--(E) (H)--(F) (A)--(C);
				\foreach \diem in {A,B,C,H}\fill (\diem)circle(1.5pt);
			\end{tikzpicture}
			
		\end{center}
		Gọi $H$ thuộc mặt phẳng $(ABC)$ và $DH\perp(ABC)$.\\
		Ta có $\heva{&BA\perp DA\\&BA\perp DH}\Rightarrow BA\perp AH$. Tương tự $\heva{&BC\perp BD\\&BC\perp DH}\Rightarrow BC\perp BH$.\\
		Tam giác $ABH$ có $AB=a;\widehat{ABC}=135^{\circ};\widehat{CBH}=90^{\circ}\Rightarrow\widehat{ABH}=45^{\circ}$ suy ra $\triangle ABH$ vuông cân tại $A\Rightarrow AH=AB=a$.\\
		Áp dụng định lý côsin ta có $BC=a\sqrt{2}$.\\
		Diện tích tam giác $ABC$: $S_{ABC}=\dfrac{1}{2}\cdot BA\cdot BC\cdot sin\widehat{ABC}=\dfrac{1}{2}\cdot a\cdot a\sqrt{2}\cdot\dfrac{\sqrt{2}}{2}=\dfrac{a^2}{2}$.\\
		Kẻ $HE$, $HF$ lần lượt vuông góc với $DA$, $DB$.\\
		Suy ra $HE\perp(ABD)$, $HF\perp(BCD)$ nên góc giữa hai mặt phẳng $(ABD)$, $(BCD)$ bằng góc $\widehat{EHF}$.\\
		Tam giác $EHF$ vuông tại $E$, ta có $HE=\dfrac{a\cdot DH}{\sqrt{a^2+DH^2}}$, $HF=\dfrac{DH\cdot a\cdot\sqrt{2}}{\sqrt{2a^2+DH^2}}$.\\
		Mặt khác: $\cos\widehat{EHF}=\dfrac{HE}{HF}=\sqrt{\dfrac{3}{4}}=\dfrac{\sqrt{DH^2+2a^2}}{\sqrt{2\cdot DH^2+2a^2}}\Rightarrow DH=a$.\\
		Thể tích tứ diện $ABCD$ là $V_{ABCD}=\dfrac{1}{3}\cdot DH\cdot S_{\triangle ABC}=\dfrac{a^3}{6}$.}
\end{ex}
\begin{ex}%[2H1G3-2]
	Cho hình lăng trụ đều $ABC.A’B’C’$. Biết khoảng cách từ điểm $C$ đến mặt phẳng $(ABC’)$ bằng $a$, góc giữa hai mặt phẳng $(ABC’)$ và $(BCC’B’)$ bằng $\alpha$ với $\cos\alpha=\dfrac{1}{2\sqrt{3}}$. Tính thể tích khối lăng trụ $ABC.A’B’C’$. 
	\choice
	{$V=\dfrac{3a^3\sqrt{2}}{4}$}
	{\True $V=\dfrac{3a^3\sqrt{2}}{2}$}
	{$V=\dfrac{a^3\sqrt{2}}{2}$}
	{$V=\dfrac{3a^3\sqrt{2}}{8}$}
	\loigiai{
		\begin{center}
			\begin{tikzpicture}[scale=1, font=\footnotesize, line join=round, line cap=round, >=stealth]
				\def\ac{4} % cạnh AC
				\def\ab{2} % cạnh AB
				\def\h{4} % chiều cao
				\def\gocA{50} % góc A của đáy
				\coordinate[label=left:$A$] (A) at (0,0);
				\coordinate[label=right:$C$] (C) at (\ac,0);
				\coordinate[label=below left:$B$] (B) at (-\gocA:\ab);
				\coordinate[label=left:$A'$] (A') at ($(A)+(90:\h)$);
				\coordinate[label=below left:$B'$] (B') at ($(B)-(A)+(A')$);
				\coordinate[label=right:$C'$] (C') at ($(C)-(A)+(A')$);
				\path ($(A)!0.5!(B)$) coordinate [label=left:$M$] (M)
				($(C')!0.5!(M)$) coordinate [label=above:$K$] (K)
				($(C')!0.4!(B)$) coordinate [label= below:$E$] (E)
				;
				\draw (A')--(A)--(B)--(C)--(C')--(A')--(B')--(C') (B)--(B') (C')--(B) (C)--(E);
				\draw[dashed] (A)--(C) (C')--(M)  (C')--(K)--(E) (C)--(K) (C)--(M);
				\foreach \diem in {A,B,C,A',B',C',M,K,E} \fill (\diem)circle(1.5pt);
			\end{tikzpicture}
			
			
		\end{center}
		Gọi $M,N$ lần lượt là trung điểm của $AB$ và $BC$.\\
		Do $\heva{&AB\perp CC’\\&AB\perp CM}\Rightarrow AB\perp(MCC’)\Rightarrow(ABC’)\perp(MCC’)$.\\
		Kẻ $CK$ vuông góc với $CM$ tại $K$ thì ta được $CK\perp(ABC’)$, do đó $CK=\mathrm{d}\left(C;(ABC’)\right)=a$.\\
		Đặt $BC=x,CC’=y,(x>0,y>0)$, ta được: $CM=\dfrac{x\sqrt{3}}{2}$.\\
		$\dfrac{1}{CM^2}+\dfrac{1}{CC’^2}=\dfrac{1}{CK^2}\Leftrightarrow\dfrac{4}{3x^2}+\dfrac{1}{y^2}=\dfrac{1}{a^2}(1)$.\\
		Kẻ $CE\perp BC’$ tại $E$, ta được $\widehat{KEC}=\alpha$, $EC=\dfrac{KC}{\sin\alpha}=\dfrac{a}{\sqrt{1-\dfrac{1}{12}}}=a\sqrt{\dfrac{12}{11}}$.\\
		Lại có $\dfrac{1}{x^2}+\dfrac{1}{y^2}=\dfrac{1}{CE^2}=\dfrac{11}{12a^2}(2)$.\\
		Giải $(1),(2)$ ta được $x=2a,y=\dfrac{a\sqrt{6}}{2}$.\\
		Thể tích khối lăng trụ $ABC.A’B’C’$ là\\
		$V=y\cdot\dfrac{x^2\sqrt{3}}{4}=\dfrac{a\sqrt{6}}{2}\cdot\dfrac{4a^2\sqrt{3}}{4}=\dfrac{3\sqrt{2}a^3}{2}$.}
\end{ex}
\begin{ex}%[2H1G3-2]
	[Chuyên Nguyễn Trãi Hải Dương 2019] Cho hình hộp $ABCD.A’B’C’D’$ có $A’B$ vuông góc với mặt phẳng đáy $(ABCD)$. Góc giữa $AA’$ với mặt phẳng $(ABCD)$ bằng $45^{\circ}$. Khoảng cách từ $A$ đến các đường thẳng $BB’$ và $DD’$ bằng $1$. Góc giữa mặt phẳng $(BB’C’C)$ và mặt phẳng $(CC’D’D)$ bằng $60^{\circ}$, Tính thể tích khối hộp đã cho. 
	\choice
	{\True $2\sqrt{3}$}
	{$2$}
	{$\sqrt{3}$}
	{$3\sqrt{3}$}
	\loigiai{
		\begin{center}
			\begin{tikzpicture}[scale=1, font=\footnotesize, line join=round, line cap=round, >=stealth]
				\path 
				(0,0) coordinate [label=below:$A'$] (A')
				(4,0) coordinate [label=right:$B'$] (B')
				(-1,-2) coordinate [label=below left:$D'$] (D')
				(3,-2) coordinate [label=below right:$C'$] (C')
				($(A')+ (0,3)$) coordinate [label=above:$B$] (B)
				($(A')+ (-4,3)$) coordinate [label=above:$A$] (A)
				($(C')+ (-4,3)$) coordinate [label=above:$C$] (C)
				($(D')+ (-4,3)$) coordinate [label=above:$D$] (D)
				($(B)!0.4!(B')$) coordinate [label=above right:$H$] (H)
				;
				\draw (A)--(B)--(C)--(D)--(A) (B')--(C')--(D') (B)--(B') (C)--(C') (D')--(D)
				;
				\draw [dashed] (A)--(A') (B')--(A') -- (D') (A')--(H) (A')--(B);
				\foreach \diem in {A,B,C,D,A',B',C',D',H} \fill (\diem)circle(1.5pt);
			\end{tikzpicture}			
		\end{center}
		Ta có $A’B\perp(ABCD)\Rightarrow(AA’,ABCD)=\widehat{AA’B}=\widehat{B’BA}=45^{\circ}$.\\
		Vì $\mathrm{d}(A,BB’)=\mathrm{d}(A’,BB’)=A’H=1$ ($H$ là hình chiếu của $A$ lên $BB’$). Suy ra ta có $A’B’=\dfrac{A’H}{\sin(BB’A)}=\sqrt{2}$ và $A’B=A’B’\cdot\tan(BB’A’)=\sqrt{2}$.\\
		Gán hệ trục tọa độ gốc $A’$ với điểm $B\in Oz,B’\in Oy$ và mặt phẳng $(A’B’C’D’)\equiv(Oxy)$. Ta có tọa độ các điểm $A’(0,0,0),B\left(0,0,\sqrt{2}\right),B’\left(0,\sqrt{2},0\right)$.\\
		Ta có $D\in(Oxy)$, giả sử $D’(a,b,0);a\geq 0\Rightarrow C’\left(a,b+\sqrt{2},0\right)$.\\
		Chọn $\overrightarrow{n}_{_{_{(BB’C’C)}}}=(-b,a,a)$ và $\overrightarrow{n}_{_{_{(DD’C’C)}}}=(1,0,0)$.\\
		Vì góc giữa mặt phẳng $(BB’C’C)$ và mặt phẳng $(CC’D’D)$ bằng $60^{\circ}$. Ta có $\cos\left({60}^{\circ}\right)=\dfrac{|-b|}{\sqrt{b^2+2a^2}}\Leftrightarrow b=\pm\dfrac{\sqrt{6}}{3}a$.\\
		Mặt khác ta có đường thằng $DD’$ có phương trình $\heva{&x=a\\&y=b-t\\&z=t}$ 4. Vì khoảng cách từ $A$ đến đường thẳng $DD’$ bằng 1. Ta có:\\
		$\mathrm{d}(A,DD’0)=\mathrm{d}(A’,DD’)=\dfrac{\left|\left[\overrightarrow{A’D’},{\overrightarrow{u}}_{_{DD’}}\right]\right|}{\left|{\overrightarrow{u}}_{_{DD’}}\right|}\Leftrightarrow\dfrac{\sqrt{b^2+2a^2}}{\sqrt{2}}=1\Rightarrow\sqrt{b^2+2a^2}=\sqrt{2}\Rightarrow b=\pm\sqrt{2}$.\\
		Trường hợp 1: $D\left(\sqrt{3},\sqrt{2},0\right)\Rightarrow V_{ABCD.A’B’C’D’}=A’B\cdot S_{A’B’C’D’}=\sqrt{2}\cdot\left|\left[\overrightarrow{A’B’},\overrightarrow{A’D’}\right]\right|=2\sqrt{3}$.\\
		Trường hợp 2. $D\left(\sqrt{3},-\sqrt{2},0\right)\Rightarrow V_{ABCD.A’B’C’D’}=A’B\cdot S_{A’B’C’D’}=\sqrt{2}\cdot\left|\left[\overrightarrow{A’B’},\overrightarrow{A’D’}\right]\right|=2\sqrt{3}$.}
\end{ex}
\begin{ex}%[2H1G3-2]
	[Chuyên Thoại Ngọc Hầu - 2018] Cho lăng trụ $ABCD.A’B’C’D’$ có đáy $ABCD$ là hình chữ nhật với $AB=\sqrt{6},AD=\sqrt{3}$, $A’C=3$ và mặt phẳng $(AA’C’C)$ vuông góc với mặt đáy. Biết hai mặt phẳng $(AA’C’C),(AA’B’B)$ tạo với nhau góc $\alpha$ thỏa mãn $\tan\alpha=\dfrac{3}{4}$. Thể tích khối lăng trụ $ABCD.A’B’C’D’$ bằng
	\choice
	{\True $V=8$}
	{$V=12$}
	{$V=10$}
	{$V=6$}
	\loigiai{
		\begin{center}
			\begin{tikzpicture}[scale=1, font=\footnotesize, line join=round, line cap=round, >=stealth]
				\def\ac{4} % cạnh AC
				\def\ab{2} % cạnh AB
				\def\h{4} % chiều cao
				\def\gocA{50} % góc A của đáy
				\coordinate[label=left:$A$] (A) at (0,0);
				\coordinate[label=right:$C$] (C) at (\ac,0);
				\coordinate[label=below left:$B$] (B) at (-\gocA:\ab);
				\coordinate[label=left:$A'$] (A') at ($(A)+(90:\h)$);
				\coordinate[label=above:$B'$] (B') at ($(B)-(A)+(A')$);
				\coordinate[label=right:$C'$] (C') at ($(C)-(A)+(A')$);
				\coordinate[label=left:$M$] (M) at ($(A')!0.3!(A)$);	
				\coordinate[label=above:$H$] (H) at ($(A)!0.7!(C)$);
				\coordinate[label=left:$K$] (K) at ($(A')!1.3!(A)$);
				\draw (A')--(A)--(B)--(C)--(C')--(A')--(B')--(C') (B)--(B') (A)--(K) --(B);
				\draw[dashed] (A)--(C) (C')--(M) (H)--(K);
				\foreach \diem in {A,B,C,A',B',C',M,H,K} \fill (\diem)circle(1.5pt);
			\end{tikzpicture}			
		\end{center}
		Gọi $H$ là hình chiếu của $B$ lên $(ACC’A’)$, vậy $BH\perp(ACC’A’)$.\\
		$AC=\sqrt{AB^2+BC^2}=3$; $BH=\dfrac{AB\cdot BC}{AC}$ = $\dfrac{\sqrt{6}\cdot\sqrt{3}}{3}=\sqrt{2}$; $HC=\sqrt{BC^2-BH^2}=1$; $AH=AC-HC=2$.\\
		Kẻ $HK\perp AA’,(K\in AA’)$, $AA’\perp BH$ vì $BH\perp(ACC’A’)$ nên $AA’\perp BK$.\\
		$\left(\widehat{(ABB’A’);(ACC’A’)}\right)=\widehat{BKH}$; $\triangle BKH$ vuông tại $H$.\\
		$\tan\widehat{BKH}=\dfrac{BH}{KH}\Leftrightarrow\dfrac{3}{4}=\dfrac{\sqrt{2}}{KH}\Rightarrow KH=\dfrac{4\sqrt{2}}{3}$; $AK=\sqrt{AH^2-AK^2}=\dfrac{2}{3}$.\\
		Gọi $M$ là trung điểm $AA’$. Tam giác $A’C’A$ cân tại $C’$, $\left(AC=A’C’=AC’=3\right)\Rightarrow C’M\perp AA’\Rightarrow KH\parallel C’M$ \\
		$ \Rightarrow A’M=\dfrac{AK\cdot A’C’}{AH}=1\Rightarrow AA’=2 $; $C’M=\dfrac{A’C’\cdot KH}{AH}=2\sqrt{2}$.\\
		$S_{ACC’A’}=C’M\cdot AA’=\mathrm{d}(A’;AC)\cdot AC=4\sqrt{2}\Rightarrow\mathrm{d}(A’;AC)=\dfrac{4\sqrt{2}}{3}$.\\
		$V_{ABCD.A’B’C’D’}=\mathrm{d}(A’;AC)\cdot S_{ABCD}$ = $\dfrac{4\sqrt{2}}{3}\cdot\sqrt{6}\cdot\sqrt{3}=8$.}
\end{ex}
\begin{ex}%[2H1G3-2]
	[Cụm 5 Trường Chuyên - Đbsh - 2018] Cho hình lăng trụ đứng $ABC.A’B’C’$ có đáy là tam giác $ABC$ vuông cân tại $A$, cạnh $BC=a\sqrt{6}$. Góc giữa mặt phẳng $(AB’C)$ và mặt phẳng $(BCC’B’)$ bằng $60^{\circ}$. Tính thể tích $V$ của khối đa diện $AB’CA’C’$. 
	\choice
	{\True $a^3\sqrt{3}$}
	{$\dfrac{3a^3\sqrt{3}}{2}$}
	{$\dfrac{a^3\sqrt{3}}{2}$}
	{$\dfrac{a^3\sqrt{3}}{3}$}
	\loigiai{
		\begin{center}
			\begin{tikzpicture}[scale=1, font=\footnotesize, line join=round, line cap=round, >=stealth]
				\def\ac{4} % cạnh AC
				\def\ab{2} % cạnh AB
				\def\h{4} % chiều cao
				\def\gocA{50} % góc A của đáy
				\coordinate[label=left:$A$] (A) at (0,0);
				\coordinate[label=right:$C$] (C) at (\ac,0);
				\coordinate[label=below left:$B$] (B) at (-\gocA:\ab);
				\coordinate[label=left:$A'$] (A') at ($(A)+(90:\h)$);
				\coordinate[label=below left:$B'$] (B') at ($(B)-(A)+(A')$);
				\coordinate[label=right:$C'$] (C') at ($(C)-(A)+(A')$);
				\coordinate[label=below:$M$] (M) at ($(B)!0.5!(C)$);
				\coordinate[label=above right:$H$] (H) at ($(B')!0.75!(C)$);
				\draw (A')--(A)--(B)--(C)--(C')--(A')--(B')--(C') (B)--(B') (C)--(B')--(A) (M)--(H);
				\draw[dashed] (A)--(C) (M)-- (A)--(H) ;
				\foreach \diem in {A,B,C,A',B',C',M,H} \fill (\diem)circle(1.5pt);
			\end{tikzpicture}			
		\end{center}
		Khối đa diện $AB’CA’C’$ là hình chóp $B’\cdot ACC’A’$ có $A’B’\perp(ACC’A’)$.\\
		Từ giả thiết tam giác $ABC$ vuông cân tại $A$, cạnh $BC=a\sqrt{6}$ ta suy ra $AB=AC=a\sqrt{3}$.\\
		Gọi $M$ là trung điểm của $BC$, suy ra $AM\perp BC$ và $AM=\dfrac{a\sqrt{6}}{2}$.\\
		Ta có $\heva{&AM\perp BC\\&AM\perp BB’}\Rightarrow AM\perp(BCC’B’)\Rightarrow AM\perp B’C$ (1).\\
		Gọi $H$ là hình chiếu vuông góc của $M$ lên $B’C$, suy ra $MH\perp B’C$ (2).\\
		Từ (1) và (2) ta suy ra $B’C\perp(AMH)$. Từ đó suy ra góc giữa mặt phẳng $(AB’C)$ và mặt phẳng $(BCC’B’)$ là góc giữa $AH$ và $MH$. Mà tam giác $AMH$ vuông tại $H$ nên $\Rightarrow\widehat{AHM}=60^{\circ}$ \\
		$ \Rightarrow MH=AM\cdot\cot 60^{\circ}=\dfrac{a\sqrt{6}}{2}\cdot\dfrac{1}{\sqrt{3}}=\dfrac{a\sqrt{2}}{2} $.\\
		Tam giác $B’BC$ đồng dạng với tam giác $MHC$ nên suy ra $\sin\widehat{HCM}=\dfrac{MH}{MC}=\dfrac{\dfrac{a\sqrt{2}}{2}}{\dfrac{a\sqrt{6}}{2}}=\dfrac{1}{\sqrt{3}}$ \\
		$ \Rightarrow 1+\tan^2\widehat{MCH}=\dfrac{1}{1-\sin^2\widehat{MCH}}=\dfrac{1}{1-\dfrac{1}{3}}=\dfrac{3}{2}\Rightarrow\tan\widehat{MCH}=\dfrac{\sqrt{2}}{2} $ \\
		$ \Rightarrow BB’=BC\cdot\tan\widehat{MCH}=a\sqrt{6}\cdot\dfrac{\sqrt{2}}{2}=a\sqrt{3} $ \\
		$ \Rightarrow V_{AB’CA’C’}=V_{B’\cdot ACC’A’}=\dfrac{1}{3}B’A’\cdot AC\cdot AA’=\dfrac{1}{3}\cdot a\sqrt{3}\cdot a\sqrt{3}\cdot a\sqrt{3}=a^3\sqrt{3} $.}
\end{ex}
\begin{ex}%[2H1G3-2]
	[Sở Lào Cai - 2021] Cho hình chóp $S.ABCD$ có đáy $ABCD$ là hình chữ nhật. $E$ là điểm trên cạnh $AD$ sao cho $BE$ vuông góc với $AC$ tại $H$ và $AB>AE$, cạnh $SH$ vuông góc với mặt phẳng đáy, góc $\widehat{BSH}=45^{\circ}$. Biết $AH=\dfrac{2a}{\sqrt{5}}$, $BE=a\sqrt{5}$. Thể tích khối chóp $S.ABCD$ bằng
	\choice
	{$\dfrac{a^3\sqrt{5}}{15}$}
	{$\dfrac{16a^3}{3\sqrt{5}}$}
	{$\dfrac{32a^3}{\sqrt{5}}$}
	{$\dfrac{8a^3\sqrt{5}}{5}$}
	\loigiai{
		\begin{center}
			\begin{tikzpicture}[scale=1, font=\footnotesize, line join=round, line cap=round, >=stealth]
				\def\bc{4} % cạnh BC
				\def\ba{3} % cạnh BA
				\def\h{4} % đường cao
				\def\gocB{30} % góc B của đáy
				\coordinate[label=below left:$B$] (B) at (0,0);
				\coordinate[label=above right:$A$] (A) at (\gocB:\ba);
				\coordinate[label=below:$C$] (C) at (\bc,0);
				\coordinate[label=right:$D$] (D) at ($(C)-(B)+(A)$);
				\coordinate  (O) at ($(A)!.5!(C)$);
				\coordinate[label=above:$E$] (E) at ($(A)!.5!(D)$);				
				\tkzInterLL(B,E)(A,C);\tkzGetPoint{H}
				\coordinate [label=below:$H$] (H) at (H) ;
				\coordinate[label=above:$S$] (S) at ($(H)+(90:\h)$);
				\draw (B)--(C)--(D)--(S)--cycle (S)--(C);
				\draw[dashed] (C)--(A)--(D)(B)--(E) (H)--(S)--(A)--(B);
				\foreach \diem in {A,B,C,D,S,H}	\fill (\diem)circle(1.5pt);
			\end{tikzpicture}			
		\end{center}
		Đặt $AE=x$, $AB=y (y>x)$.\\
		Tam giác $ABE$ vuông tại $A$, có đường cao $AH$. Áp dụng các hệ thức lượng trong tam giác vuông ta có:\\
		+) $\heva{&BE^2=AE^2+AB^2\\&\dfrac{1}{AH^2}=\dfrac{1}{AE^2}+\dfrac{1}{AB^2}}\Leftrightarrow\heva{&5a^2=x^2+y^2\\&\dfrac{5}{4a^2}=\dfrac{1}{x^2}+\dfrac{1}{y^2}}\Leftrightarrow\heva{&x^2+y^2=5a^2\\&xy=2a^2}\Leftrightarrow\heva{&x+y=3a\\&xy=2a^2}\Leftrightarrow\heva{&x=a\\&y=2a.}$ \\
		+) $BH=\dfrac{AB^2}{BE}=\dfrac{4a^2}{a\sqrt{5}}=\dfrac{4\sqrt{5}a}{5}\Rightarrow EH=BE-BH=a\sqrt{5}-\dfrac{4\sqrt{5}a}{5}=\dfrac{\sqrt{5}a}{5}$.\\
		Tam giác $SHB$ vuông cân tại $H$ (có $\widehat{BSH}=45^{\circ}$), suy ra: $SH=\dfrac{4\sqrt{5}a}{5}$.\\
		+) $\dfrac{BC}{EA}=\dfrac{BH}{EH}=4\Rightarrow BC=4a$.\\
		Vậy $V_{S.ABCD}=\dfrac{1}{3}\cdot SH\cdot S_{ABCD}=\dfrac{1}{6}\cdot SH\cdot AB\cdot BC=\dfrac{1}{6}\cdot\dfrac{4\sqrt{5}a}{5}\cdot 2a\cdot 4a=\dfrac{16\sqrt{5}a^3}{15}$.}
\end{ex}
\begin{ex}%[2H1G3-3]
	[Liên trường Quỳnh Lưu - Hoàng Mai - Nghệ An - 2021] Cho lăng trụ $ABC.A'B'C'$. Gọi $M$, $N$, $Q$, $R$ lần lượt là trung điểm của các cạnh $AB$, $A'B'$, $BC$, $B'C'$ và $P$, $S$ lần lượt là trọng tâm của các tam giác $AA'B$, $CC'B$. Tỉ số thể tích khối đa diện $MNRQPS$ và khối lăng trụ $ABC.A'B'C'$ là
	\choice
	{$\dfrac{1}{9}$}
	{\True $\dfrac{5}{54}$}
	{$\dfrac{1}{10}$}
	{$\dfrac{2}{27}$}
	\loigiai{
		(*) Cách 1: 
		\begin{center}
			\begin{tikzpicture}[line join=round, line cap = round, >=stealth, scale=1,font=\footnotesize]
				\path
				(-1,1) coordinate (A')
				(2.5,-1) coordinate (B')
				(4.5,1) coordinate (C')
				($(A')!0.5!(B')$) coordinate (N)
				($(B')!0.5!(C')$) coordinate (R)
				($(A')+(1,5)$) coordinate (A)
				($(B')+(1,5)$) coordinate (B)
				($(C')+(1,5)$) coordinate (C)
				($(N)+(1,5)$) coordinate (M)
				($(R)+(1,5)$) coordinate (Q)
				(intersection of A'--M and A--B') coordinate (P)
				(intersection of B'--C and Q--C') coordinate (S)
				;
				\draw[dashed] (A')--(C');
				\draw[dashed,blue] (N)--(R)--(M);
				\draw (A')--(B')--(C')--(C)--(A)--(B)--(B')--(A)--(A')--(B)--(C) (A')--(M) (Q)--(C');
				\draw[blue] (N)--(M)--(Q)--(R);
				\foreach \i/\g in {A/180,B/-30,C/0,A'/180,B'/-30,C'/0,M/90,P/180,N/-150,R/-30,Q/90,S/30}\fill[black] (\i) circle (2pt) +(\g:.4) node{$\i$};
			\end{tikzpicture}	
		\end{center}
		Đặt: $V=V_{ABC.A'B'C'}$; $V_{B'.AA'C'C}=\dfrac{1}{3}S_{AA'C'C}\cdot\mathrm{d}\left(B',(AA'C'C)\right)=\dfrac{2}{3}V$.\\
		$\begin{aligned}
			V_{B'\cdot MNRQ}&=\dfrac{1}{3}\cdot S_{MNRQ}\cdot\mathrm{d}\left(B',(MNRQ)\right)=\dfrac{1}{3}\left(\dfrac{1}{2}S_{AA'C'C}\right)\cdot\left(\dfrac{1}{2}\mathrm{d}\left(B',(AA'C'C)\right)\right)\\
			&=\left(\dfrac{1}{3}\cdot S_{AA'C'C}\cdot\mathrm{d}\left(B',(AA'C'C)\right)\right)\cdot\dfrac{1}{4}=\dfrac{1}{4}\cdot\dfrac{2}{3}V=\dfrac{1}{6}V.
		\end{aligned}$\\
		$V_{P.MNRQ}=\dfrac{1}{3}\cdot V_{A'\cdot MNRQ}=\dfrac{1}{3}\cdot V_{B'\cdot MNRQ}=\dfrac{1}{3}\cdot\dfrac{1}{6}V=\dfrac{1}{18}V$.\\
		$V_{A.BB'C'C}=\dfrac{1}{3}S_{BB'C'C}\cdot\mathrm{d}\left(A,(BB'C'C)\right)=\dfrac{2}{3}V$.\\
		$ S_{\triangle Q R C^{\prime}}=\dfrac{1}{2} S_{Q R C^{\prime} C}=\dfrac{1}{4} S_{B B^{\prime} C^{\prime} C} ; S_{\triangle Q R S}=\dfrac{1}{3} S_{Q R C^{\prime}}=\dfrac{1}{3} \cdot \dfrac{1}{4} S_{B B^{\prime} C^{\prime} C}=\dfrac{1}{12} S_{B B^{\prime} C^{\prime} C}$\\
		$\begin{aligned}
			V_{A \cdot Q R S}&=\dfrac{1}{3} S_{\triangle Q R S} \cdot \mathrm{d}(A,(Q R S))=\dfrac{1}{3}\left(\dfrac{1}{12} S_{B B^{\prime} C^{\prime} C}\right) \cdot\left(\mathrm{d}\left(A,\left(B B^{\prime} C^{\prime} C\right)\right)\right)\\
			&=\left(\dfrac{1}{3}\cdot S_{BB'C'C}\cdot\mathrm{d}\left(A,(BB'C'C)\right)\right)\cdot\dfrac{1}{12}=\dfrac{1}{12}\cdot\dfrac{2}{3}V=\dfrac{1}{18}V.
		\end{aligned}$\\
		$V_{P.QRS}=\dfrac{PB'}{AB'}\cdot V_{A.QRS}=\dfrac{2}{3}\cdot\dfrac{1}{18}V=\dfrac{1}{27}V$.\\
		$V_{MNRQPS}=V_{P.MNRQ}+V_{P.QRS}=\dfrac{1}{18}V+\dfrac{1}{27}V=\dfrac{5}{54}V$.\\
		Vậy: $\dfrac{V_{MNRQPS}}{V_{ABC.A'B'C'}}=\dfrac{5}{54}$.\\
		(*) Cách 2: 
		\begin{center}
			\begin{tikzpicture}[line join=round, line cap = round, >=stealth, scale=1,font=\footnotesize]
				\path
				(-1,1) coordinate (A')
				(2.5,-1) coordinate (B')
				(4.5,1) coordinate (C')
				($(A')!0.5!(B')$) coordinate (N)
				($(B')!0.5!(C')$) coordinate (R)
				($(A')+(0,5)$) coordinate (A)
				($(B')+(0,5)$) coordinate (B)
				($(C')+(0,5)$) coordinate (C)
				($(N)+(0,5)$) coordinate (M)
				($(R)+(0,5)$) coordinate (Q)
				(intersection of A'--M and A--B') coordinate (P)
				(intersection of B'--C and Q--C') coordinate (S)
				;
				\draw
				($(A')!1.2!(A)$) coordinate (z) node[right]{$z$}
				($(A')!1.2!(B')$) coordinate (x) node[right]{$x$}
				($(A')!1.2!(C')$) coordinate (y) node[above]{$y$}
				($(A')!0.75!(B')$) coordinate (c2) node[below left]{$2$}
				;
				\draw[dashed] (A')--(C')node[midway,above]{$2$};
				\draw[dashed,blue] (N)--(R)--(M);
				\draw (A')--(B')--(C')--(C)--(A)--(B)--(B')--(A)--(A')node[midway,left]{$2$}--(B)--(C) (A')--(M) (Q)--(C');
				\draw[blue] (N)--(M)--(Q)--(R);
				\draw[->] (A')--(x);
				\draw[->] (C')--(y);
				\draw[->] (A')--(z) ;
				\foreach \i/\g in {A/180,B/-30,C/0,A'/180,B'/-90,C'/-45,M/90,P/180,N/-150,R/-30,Q/90,S/30}\fill[black] (\i) circle (2pt) +(\g:.4) node{$\i$};
			\end{tikzpicture}	
		\end{center}
		Chuẩn hóa lăng trụ $ABC.A'B'C'$ là lăng trụ đứng có đáy là $\triangle ABC$ vuông tại $A$ và các cạnh $AB=AC=AA'=2$. Khi đó: $V_{ABC.A'B'C'}=\left(\dfrac{1}{2}\cdot 2\cdot 2\right)\cdot 2=4$.\\
		Đặt khối lăng trụ $ABC.A'B'C'$ vào hệ trục tọa độ $Oxyz$ sao cho: $A'\equiv O$ và $B',C',A$ lần lượt nằm trên chiều dương của các trục $Ox,Oy,Oz$ (như hình vẽ).\\
		Khi đó: $A'(0;0;0)$, $B'(2;0;0)$, $C'(0;2;0)$, $A(0;0;2)$, $B(2;0;2)$, $C(0;2;2)$.\\
		$M(1;0;2)$, $N(1;0;0)$, $R(1;1;0)$, $Q(1;1;2)$, $\mathrm{P}\left(\dfrac{2}{3};0;\dfrac{4}{3}\right)$, $\overrightarrow{SC'}=-2\overrightarrow{SQ}\Rightarrow$  $S=\left(\dfrac{2}{3};\dfrac{4}{3};\dfrac{4}{3}\right)$.\\
		$\overrightarrow{PM}=\left(\dfrac{1}{3};0;\dfrac{2}{3}\right)$, $\overrightarrow{PR}=\left(\dfrac{1}{3};1;-\dfrac{4}{3}\right)$, $\overrightarrow{PQ}=\left(\dfrac{1}{3};1;\dfrac{2}{3}\right)$, $\overrightarrow{PS}=\left(0;\dfrac{4}{3};0\right)$.\\
		$V_{P.MQR}=\dfrac{1}{6}\cdot\left|\left[\overrightarrow{PM};\overrightarrow{PQ}\right]\right|\cdot\overrightarrow{PR}=\dfrac{1}{6}\cdot\dfrac{2}{3}=\dfrac{1}{9}$; $V_{P.MQRN}=2\cdot V_{P.MQR}=\dfrac{2}{9}$.\\
		$V_{P.QRS}=\dfrac{1}{6}\cdot\left|\left[\overrightarrow{PR};\overrightarrow{PQ}\right]\right|\cdot\overrightarrow{PS}=\dfrac{1}{6}\cdot\dfrac{8}{9}=\dfrac{4}{27}$.\\
		$V_{MNRQPS}=V_{P.MQRN}+V_{P.QRS}=\dfrac{2}{9}+\dfrac{4}{27}=\dfrac{10}{27}$.\\
		Vậy: $\dfrac{V_{MNRQPS}}{V_{ABC.A'B'C'}}=\dfrac{\dfrac{10}{27}}{4}=\dfrac{5}{54}$.}
\end{ex}
\begin{ex}%[2H1K3-2]
	[Chuyên KHTN - 2021] Cho hình chóp $S.ABC$ có $AB=3a$, $BC=4a$, $CA=5a$, các mặt bên tạo với đáy góc $60^{\circ}$, hình chiếu vuông góc của $S$ lên mặt phẳng $(ABC)$ thuộc miền trong của tam giác $ABC$. Tính thể tích hình chóp $S.ABC$. 
	\choice
	{\True $2a^3\sqrt{3}$}
	{$6a^3\sqrt{3}$}
	{$12a^3\sqrt{3}$}
	{$2a^3\sqrt{2}$}
	\loigiai{
		\begin{center}
			\begin{tikzpicture}[line join=round, line cap = round, >=stealth, scale=1,font=\footnotesize]
				\path
				(0,0) coordinate (A)
				(5,0) coordinate (C)
				(1.5,-2) coordinate (B)
				($(B)!0.5!(C)$) coordinate (I)
				($(B)!0.3!(C)$) coordinate (M)
				($(A)!0.6!(I)$) coordinate (H)
				($(H)+(0,4)$) coordinate (S)
				;
				\draw[dashed] (S)--(H)--(M) (A)--(C);
				\draw (B)--(S)--(A)--(B)--(C)--(S)--(M);
				\foreach \i/\g in {A/180,B/-120,C/0,S/90,H/-90,M/-30}\fill[black] (\i) circle (2pt) +(\g:.4) node{$\i$};
				\newcommand{\gocv}[4][black]{\draw[#1] ($(#3)!6pt!(#2)$)--($(#3)!2!($($(#3)!6pt!(#2)$)!.5!($(#3)!6pt!(#4)$)$)$)--($(#3)!6pt!(#4)$);}
				\gocv{S}{H}{M}
				\gocv{A}{B}{C}
			\end{tikzpicture}
		\end{center}
		Ta có $AC^2=25a^2=9a^2+16a^2=AB^2+BC^2$, vậy tam giác $ABC$ vuông tại $B$.\\
		Gọi $H$ là hình chiếu của $S$ trên mặt phẳng $(ABC)$. Vì các mặt bên tạo với đáy góc $60^{\circ}$ suy ra: $\mathrm{d}(H;AC)=\mathrm{d}(H;BC)=\mathrm{d}(H;AB)$ và $H$ thuộc miền trong của tam giác $ABC$ nên $H$ là tâm đường tròn nội tiếp tam giác $ABC$.\\
		Từ $H$ kẻ đường thẳng vuông góc với $BC$ tại $M$, suy ra:\\ $\heva{&BC\perp HM\\&BC\perp SH}\Rightarrow BC\perp(SHM)\Rightarrow BC\perp SM$.\\
		Suy ra: $\widehat{SMH}=\left((SBC);(ABC)\right)=60^{\circ}$.\\
		Đoạn $HM$ là bán kính đường tròn nội tiếp tam giác $ABC$, suy ra:\\
		$HM=\dfrac{S_{\triangle ABC}}{p}=\dfrac{AB\cdot BC}{AB+BC+CA}=\dfrac{3a\cdot 4a}{3a+4a+5a}=\dfrac{12a^3}{12a}=a$.\\
		$SH=HM\cdot\tan 60^{\circ}=a\sqrt{3}$.\\
		Vậy $V_{S.ABC}=\dfrac{1}{6}AB\cdot BC\cdot SH=\dfrac{1}{6}\cdot 3a\cdot 4a\cdot a\sqrt{3}=2a^3\sqrt{3}$.}
\end{ex}
\begin{ex}%[2H1G3-2]
	[Chuyên Quang Trung - Bình Phước - 2021] Cho hình lăng trụ $ABCD. A'B'C'D'$ đáy là hình bình hành. Với $AC=BC=a, CD=a\sqrt{2}, AC'=a\sqrt{3},\widehat{CA'B'}=\widehat{A'D'C}=90^{\circ}$. Thể tích khối tứ diện $BCDA'$ là
	\begin{center}
		\begin{tikzpicture}[line join=round, line cap = round, >=stealth, scale=1,font=\footnotesize]
			\path
			(0,0) coordinate (B)
			(3,0) coordinate (C)
			($(B)+(5,2)$) coordinate (A)
			($(C)+(5,2)$) coordinate (D)
			($(B)+(-3,4)$) coordinate (B')
			($(C)+(-3,4)$) coordinate (C')
			($(A)+(-3,4)$) coordinate (A')
			($(D)+(-3,4)$) coordinate (D')
			;
			\draw[dashed] (B)--(A)--(D) (A)--(A');
			\draw (D')--(A')--(B')--(C')--(D')--(D)--(C)--(C')--(B')--(B)--(C);
			\foreach \i/\g in {A/30,B/-90,C/-90,D/0,A'/90,B'/90,C'/90,D'/90}\fill[black] (\i) circle (2pt) +(\g:.4) node{$\i$};
		\end{tikzpicture}	
	\end{center}
	\choice
	{\True $\dfrac{a^3}{6}$}
	{$a^3$}
	{$\dfrac{2a^3}{3}$}
	{$\sqrt{6} a^3$}
	\loigiai{
		\begin{center}
			\begin{tikzpicture}[line join=round, line cap = round, >=stealth, scale=1,font=\footnotesize]
				\path
				(0,0) coordinate (D')
				(7,0) coordinate (C')
				($(D')+(3,1.5)$) coordinate (A')
				($(C')+(3,1.5)$) coordinate (B')
				($(D')+(-3,5)$) coordinate (D)
				($(C')+(-3,5)$) coordinate (C)
				($(A')+(-3,5)$) coordinate (A)
				($(B')+(-3,5)$) coordinate (B)
				($(B')!2!(C')$) coordinate (H)
				($(A')!0.5!(C)$) coordinate (O)
				;
				\draw[dashed] (C')--(A)--(A')--(D')--(C')--(A')--(B') (C)--(A')--(H);
				\draw (H)--(B')--(B)--(A)--(D)--(D')--(H)--(C)--(C') (D)--(C)--(B);
				\foreach \i/\g in {A/90,B/90,C/90,D/90,A'/180,B'/0,C'/0,D'/180,H/-90,O/180}\fill[black] (\i) circle (2pt) +(\g:.4) node{$\i$};
				\newcommand{\gocv}[4][black]{\draw[#1] ($(#3)!6pt!(#2)$)--($(#3)!2!($($(#3)!6pt!(#2)$)!.5!($(#3)!6pt!(#4)$)$)$)--($(#3)!6pt!(#4)$);}
				\gocv{C}{H}{C'}
				\gocv{C}{H}{D'}
			\end{tikzpicture}	
		\end{center}
		Ta có tam giác $ABC$ vuông cân tại $C$.\\
		Gọi $O$ là trung điểm của $AC'\Rightarrow OC'=OA=\dfrac{a\sqrt{3}}{2}$.\\
		Gọi $H$ là chân đường cao hạ từ $C'$ xuống mặt $(A'B'C'D')$.\\
		Ta có: $\heva{&A'D'\perp CH\\&A'D'\perp D'C}\Rightarrow A'D'\perp HD'$.\\
		Lại có: $\heva{&A'B'\perp A'C\\&A'B'\perp CH}\Rightarrow A'B'\perp A'H$.\\
		Ta có: $A'H\perp A'B'\Rightarrow\widehat{HA'B'}=90^{\circ};\widehat{A'D'H}=90^{\circ}$. Tam giác $A'D'H$ vuông cân tại $D'$.\\
		Giả sử $CH=x\Rightarrow CA'=\sqrt{x^2+2a^2}$.\\
		$CC'^2=x^2+a^2$.\\
		$C'O=\dfrac{CC'^2+C'A'^2}{2}-\dfrac{CA'^2}{4}\Leftrightarrow\dfrac{3a^2}{4}=\dfrac{x^2+a^2+a^2}{2}-\dfrac{x^2+2a^2}{4}=\dfrac{x^2+2a^2}{4}$.\\
		$x^2+2a^2=3a^2\Rightarrow x=a=CH$.\\
		$V_{BCDA'}=\dfrac{1}{6}V_{ABCD.A'B'C'D}=\dfrac{1}{6}\cdot CH\cdot S_{ABCD}=\dfrac{a^3}{6}$.}
\end{ex}
\begin{ex}%[2H1G3-2]
	[Chuyên ĐHSP Hà Nội - 2021] Cho hình lăng trụ $ABC.A'B'C'$ có đáy là tam giác đều. Hình chiếu vuông góc của $A'$ trên $(ABC)$ là trung điểm của $BC$. Mặt phẳng $(P)$ vuông góc với các cạnh bên và cắt các cạnh bên của hình lăng trụ lần lượt tại $D$, $E$, $F$. Biết mặt phẳng $(ABBA')$ vuông góc với mặt phẳng $(ACC'A')$ và chu vi của tam giác $DEF$ bằng 4, thể tích khối lăng trụ $ABC.A'B'C'$ bằng
	\choice
	{\True $12\left(10-7\sqrt{2}\right)$}
	{$4\left(10+7\sqrt{2}\right)$}
	{$6\left(10-7\sqrt{2}\right)$}
	{$12\left(10+7\sqrt{2}\right)$}
	\loigiai{
		\begin{center}
			\begin{tikzpicture}[line join=round, line cap = round, >=stealth, scale=0.8,font=\footnotesize]
				\path
				(0,0) coordinate (B)
				(2,-2.5) coordinate (A)
				(8,0) coordinate (C)
				($(B)!0.5!(C)$) coordinate (H)
				($(A)+(2,7)$) coordinate (A')
				($(B)+(2,7)$) coordinate (B')
				($(C)+(2,7)$) coordinate (C')
				($(H)+(2,7)$) coordinate (H')
				($(A)!0.35!(A')$) coordinate (D)
				($(B)!0.35!(B')$) coordinate (E)
				($(C)!0.35!(C')$) coordinate (F)
				($(H)!0.35!(H')$) coordinate (I)
				;
				\draw[dashed] (A)--(H)--(H') (B)--(C) (E)--(F) (D)--(I) (A')--(H);
				\draw (A)--(B)--(B')--(C')--(A')--(H')--(B')--(A')--(A)--(C)--(C')--(F)--(D)--(E);
				\foreach \i/\g in {A/-90,B/180,C/0,D/-150,A'/-150,B'/180,C'/0,H/-45,H'/90,I/-30,F/0,E/180}\fill[black] (\i) circle (2pt) +(\g:.4) node{$\i$};
				\newcommand{\gocv}[4][black]{\draw[#1] ($(#3)!6pt!(#2)$)--($(#3)!2!($($(#3)!6pt!(#2)$)!.5!($(#3)!6pt!(#4)$)$)$)--($(#3)!6pt!(#4)$);}
				\gocv{A}{H}{A'}
				\gocv{E}{D}{F}
				\gocv{B'}{E}{D}
				\gocv{D}{F}{C'}
				\gocv{E}{I}{D}
			\end{tikzpicture}\\
			\begin{tikzpicture}[line join=round, line cap = round, >=stealth, scale=1,font=\footnotesize]
				\path
				(0,0) coordinate (A')
				(7,0) coordinate (A)
				($(A)!0.5!(A')$) coordinate (D)
				($(A)+(-2,-3)$) coordinate (H)
				($(A')+(-2,-3)$) coordinate (H')
				($(A')+(0,-3)$) coordinate (K)
				($(D)+(0,-3)$) coordinate (I)
				;
				\draw (A)--(H)--(H')--(A')--(A)--(H)--(A')--(K)--(I)--(D);
				\foreach \i/\g in {A/0,A'/180,H/0,D/90,I/-90,K/-90,H'/180}\fill[black] (\i) circle (2pt) +(\g:.4) node{$\i$};
				\newcommand{\gocv}[4][black]{\draw[#1] ($(#3)!6pt!(#2)$)--($(#3)!2!($($(#3)!6pt!(#2)$)!.5!($(#3)!6pt!(#4)$)$)$)--($(#3)!6pt!(#4)$);}
				\gocv{A'}{D}{I}
				\gocv{A'}{H}{A}
				\gocv{A'}{K}{H}
			\end{tikzpicture}
		\end{center}
		Gọi $H$ và $H'$ lần lượt là trung điểm của $BC$ và $B'C'$.\\ Khi đó ta có $\heva{&BC\perp A'H\\&BC\perp AH}\Rightarrow BC\perp AA'\Rightarrow BC\perp BB', BC\perp CC'$, suy ra $BB'C'C$ là hình chữ nhật.\\
		Vì $E\in BB'$, $F\in CC'$, và $EF\perp BB'$, $EF\perp CC'$ (do $EF\subset(P)$ vuông góc với các cạnh bên của lăng trụ), suy ra $EF\parallel BC$ và $EF=BC =a$ (giả sử cạnh đáy của lăng trụ là $a$).\\
		Gọi $I$ là trung điểm của $HH'\Rightarrow I$ cũng là trung điểm của $EF$.\\
		Kẻ $ED\perp AA'$, $D\in AA'$, suy ra $DF\perp AA'$.\\
		Do $(ABB'A')\perp(ACC'A')$ nên suy ra $ED\perp DF$. Hơn nữa dễ thấy $DE=DF$, nên $\triangle DEF$ vuông cân tại $D$. Suy ra $2ED^2=EF^2=a^2\Rightarrow ED=\dfrac{a\sqrt{2}}{2}$.\\
		Chu vi $\triangle DEF$ bằng $DE+DF+EF=a\sqrt{2}+a=4\Rightarrow a=4(\sqrt{2}-1)$.\\
		Xét hình bình hành $AA'H'H$, kẻ $A'K\perp HH'$.\\ Ta thấy, $ID\perp AA'\Rightarrow ID\perp HH'$, suy ra $A'K\parallel ID\Rightarrow A'K=ID=\dfrac{EF}{2}=\dfrac{a}{2}$ (do $\triangle DEF$ vuông cân tại $D$).\\
		Khi đó, ta có diện tích hình bình hành $AA'H'H$ bằng $A'K\cdot AA'=A'H\cdot AH$ \\
		$ \Rightarrow\dfrac{a}{2}\cdot AA'=\dfrac{a\sqrt{3}}{2}\cdot A'H\Rightarrow AA'=\sqrt{3}A'H $.\\
		Mà $AA'^2=A'H^2+AH^2\Rightarrow 2A'H^2=AH^2=\dfrac{3a^2}{4}\Rightarrow A'H=\dfrac{a\sqrt{3}}{2\sqrt{2}}$.\\
		$S_{ABC}=\dfrac{a^2\sqrt{3}}{4}$.\\
		Suy ra $V_{ABC.A'B'C'}=A'H\cdot S_{ABC}=\dfrac{a\sqrt{3}}{2\sqrt{2}}\cdot\dfrac{a^2\sqrt{3}}{4}$.\\
		Với $a=4(\sqrt{2}-1)$ thì $V_{ABC.A'B'C'}=\dfrac{4(\sqrt{2}-1)}{2\sqrt{2}}\cdot\dfrac{16(\sqrt{2}-1)^2\cdot\sqrt{3}}{4}=12\left(10-7\sqrt{2}\right)$.}
\end{ex}
\begin{ex}%[2H1K3-2]
	[THPT Đào Duy Từ - Hà Nội - 2021] Cho khối chóp $S.ABC$ có $AB=2,AC=3$ và $\widehat{BAC}=120^{\circ},SA$ vuông góc với mặt phẳng đáy. Gọi $M,N$ lần lượt là hình chiếu của $A$ trên $SB$ và $SC$. Biết góc giữa mặt phẳng $(ABC)$ và $(AMN)$ bằng $60^{\circ}$. Thể tích của khối chóp đã cho bằng
	\choice
	{$\sqrt{57}$}
	{$3\sqrt{57}$}
	{\True $\dfrac{\sqrt{57}}{3}$}
	{$\dfrac{3\sqrt{57}}{2}$}
	\loigiai{
		\begin{center}
			\begin{tikzpicture}[line join=round, line cap = round, >=stealth, scale=1,font=\footnotesize]
				\path
				(0,0) coordinate (B)
				(2.5,1) coordinate (A)
				(7,0) coordinate (C)
				($(A)+(0,5)$) coordinate (S)
				(3.5,0) coordinate (D)
				($(A)!2.5!(D)$) coordinate (H)
				($(S)!0.55!(C)$) coordinate (N)
				($(S)!0.65!(B)$) coordinate (M)
				($(S)!0.35!(H)$) coordinate (K)
				($(B)!0.35!(C)$) coordinate (I)
				;
				\draw[dashed] (H)--(A)--(B)--(C)--(A)--(S) (M)--(A)--(N) (A)--(I) (A)--(K);
				\draw (C)--(S)--(B)--(H)--(S)--(C)--(H) (M)--(K)--(N);
				\foreach \i/\g in {A/180,B/180,C/0,H/-90,S/90,M/180, N/0, K/30,I/-90}\fill[black] (\i) circle (2pt) +(\g:.4) node{$\i$};
			\end{tikzpicture}	
		\end{center}
		Ttrong mặt phẳng $(ABC)$: Kẻ $HC\perp AC, HB\perp AB$ \\
		$ \Rightarrow HB\perp(SAB), HC\perp(SAC) $ \\
		$ \Rightarrow AM\perp(SBH), AN\perp(SCH)\Rightarrow SH\perp(AMN) $.\\
		Mà $SA\perp(ABC),\widehat{ASH}<90^{\circ}$ \\
		$ \Rightarrow\left(\widehat{(AMN),(ABC)}\right)=\left(\widehat{SA,SH}\right)=\widehat{ASH} $ \\
		$ \Rightarrow\widehat{ASH}=60^{\circ}; BC=\sqrt{AB^2+AC^2-2\cdot AB\cdot AC\cdot\cos 120^{\circ}}=\sqrt{19} $.\\
		$\begin{aligned}&S_{\triangle ABC}=\dfrac{1}{2}\cdot AB\cdot AC\cdot\sin 120^{\circ}=\dfrac{1}{2}\cdot 2\cdot 3\cdot\dfrac{\sqrt{3}}{2}=\dfrac{3\sqrt{3}}{2}\Rightarrow AI=\dfrac{2S_{\triangle ABC}}{BC}=\dfrac{3\sqrt{3}}{\sqrt{19}}\\&AH=\dfrac{AB}{\sin\widehat{BCA}}=\dfrac{AB}{\dfrac{AI}{AC}}=\dfrac{AB\cdot AC}{AI}=\dfrac{2\cdot 3}{\dfrac{3\sqrt{3}}{\sqrt{19}}}=\dfrac{2\sqrt{19}}{\sqrt{3}}\\&SA=\dfrac{AH}{\tan 60^{\circ}}=\dfrac{\dfrac{2\sqrt{19}}{\sqrt{3}}}{\sqrt{3}}=\dfrac{2\sqrt{19}}{3}\Rightarrow V_{S.ABC}=\dfrac{1}{3}\cdot SA\cdot S_{\triangle ABC}=\dfrac{1}{3}\cdot\dfrac{2\sqrt{19}}{3}\cdot\dfrac{3\sqrt{3}}{2}=\dfrac{\sqrt{57}}{3}.\end{aligned}$}
\end{ex}
\begin{ex}%[2H1G3-2]
	[THPT Đặng Thúc Hứa - Nghệ An - 2021] Cho khối lăng trụ tứ giác đều $ABCD.A'B'C'D'$ có đáy là hình vuông; khoảng cách và góc giữa hai đường thẳng $AC$ và $DC'$ lần lượt bằng $\dfrac{3\sqrt{7}a}{7}$ và $\varphi$ với $\cos\varphi=\dfrac{\sqrt{2}}{4}$. Thể tích khối lăng trụ đã cho bằng
	\choice
	{$3a^3$}
	{\True $9a^3$}
	{$3\sqrt{3}a^3$}
	{$\sqrt{3}a^3$}
	\loigiai{
		\begin{center}
			\begin{tikzpicture}[line join=round, line cap = round, >=stealth, scale=0.8,font=\footnotesize]
				\path
				(0,0) coordinate (A')
				(6,0) coordinate (B')
				($(A')+(3,2)$) coordinate (D')
				($(B')+(3,2)$) coordinate (C')
				($(A')+(0,4)$) coordinate (A)
				($(B')+(0,4)$) coordinate (B)
				($(C')+(0,4)$) coordinate (C)
				($(D')+(0,4)$) coordinate (D)
				($(A')!.5!(C')$) coordinate (O)
				($(D)!0.66!(O)$) coordinate (E)
				;
				\draw[dashed] (D)--(A')--(C')--(D')--(D)--(C') (A')--(D')--(B')--(O)--(D) (E)--(D');
				\draw (A)--(B)--(C)--(D)--(A)--(A')--(B')--(C')--(C)--(A)--(B)--(B');
				\foreach \i/\g in {A/180,B/0,C/0,D/180,A'/180,B'/0, C'/0, D'/180,O/-90,E/30}\fill[black] (\i) circle (2pt) +(\g:.4) node{$\i$};
			\end{tikzpicture}	
		\end{center}
		$\mathrm{d}(AC,DC')=\mathrm{d}\left(AC,(A'C'D)\right)=\mathrm{d}\left(A,(A'C'D)\right) =\mathrm{d}\left(D',(A'C'D)\right)=\dfrac{3a}{\sqrt{7}}$.\\
		$\varphi=\widehat{A'C'D}$ với $\cos\varphi=\dfrac{\sqrt{2}}{4}$.\\
		Đặt $DD'=x,D'E=\dfrac{3a}{\sqrt{7}}$, ta có $\dfrac{1}{DD'^2}+\dfrac{1}{D'O^2}=\dfrac{1}{D'E^2}=\dfrac{7}{9a^2}\Leftrightarrow\dfrac{1}{x^2}+\dfrac{1}{D'O^2}=\dfrac{7}{9a^2}$ \\
		$ \Rightarrow D'O=\dfrac{3ax}{\sqrt{7x^2-9a^2}}\Rightarrow DO=\sqrt{\dfrac{9a^2x^2}{7x^2-9a^2}+x^2}=\dfrac{x^2\cdot\sqrt{7}}{\sqrt{7x^2-9a^2}} $.\\
		và $\tan\varphi=\sqrt{\dfrac{1}{\cos^2\varphi}-1}=\sqrt{7}$.\\
		Khi đó $\tan\varphi=\dfrac{DO}{OC'}=\dfrac{x\sqrt{7}}{3a}=\sqrt{7}\Rightarrow x=3a$.\\
		Vì $AA'=3a$ và $AB=\dfrac{3ax\sqrt{2}}{\sqrt{7x^2-9a^2}}=a\sqrt{3}$ nên $V_{ABCD.A'B'C'D'}=AA'\cdot S_{ABCD}=9a^3$.}
\end{ex}
\begin{ex}%[2H1K3-2]
	[THPT Hậu Lộc 4 - Thanh Hóa - 2021] Cho hình chóp $S.ABC$ có $AB=BC=a$, góc $\widehat{ABC}=120^{\circ}$, $\widehat{SAB}=\widehat{SCB}=90^{\circ}$ và khoảng cách từ $B$ đến mặt phẳng $(SAC)$ bằng $\dfrac{2a}{\sqrt{21}}$. Tính thể tích khối $S.ABC$. 
	\choice
	{$V=\dfrac{a^3\sqrt{5}}{10}$}
	{\True $V=\dfrac{a^3\sqrt{15}}{10}$}
	{$V=\dfrac{a^3\sqrt{15}}{5}$}
	{$V=\dfrac{a^3\sqrt{5}}{2}$}
	\loigiai{
		\begin{center}
			\begin{tikzpicture}[line join=round, line cap = round, >=stealth, scale=0.8,font=\footnotesize]
				\path
				(0,0) coordinate (A)
				(5,0) coordinate (D)
				(-1,-2.5) coordinate (B)
				(6,-2) coordinate (C)
				(intersection of A--C and B--D) coordinate (I)
				($(D)+(0,5)$) coordinate (S)
				($(S)!.75!(I)$) coordinate (H)
				;
				\draw[dashed] (S)--(D)--(A)--(C)--(D)--(B)--(I)--(S) (H)--(D);
				\draw (S)--(B)--(A)--(S)--(C)--(B);
				\foreach \i/\g in {A/180,B/180,C/0,D/0,I/-90,H/180, S/90}\fill[black] (\i) circle (2pt) +(\g:.4) node{$\i$};
			\end{tikzpicture}	
		\end{center}
		Gọi $D$ là hình chiếu vuông góc của $S$ lên mặt phẳng $(ABC)$.\\
		Có $\heva{&BC\perp SC\\&BC\perp SD}\Rightarrow BC\perp CD$.\\
		Có $\heva{&AB\perp SD\\&AB\perp SA}\Rightarrow AB\perp AD$.\\
		Gọi $I$ là giao điểm của $BD$ và $AC$ ($BD$ là đường phân giác của góc $\widehat{ABC}$).\\
		$BD=\dfrac{BC}{\cos 60^{\circ}}=2a$; $BI=BC\cdot\cos 60^{\circ}=\dfrac{a}{2}$.\\
		Gọi $H$ là hình chiếu vuông góc của $D$ lên $SI$.\\
		$\heva{&(SAC)\perp(SBC)\\&(SAC)\cap(SBC)=SI\\&DH\perp SI}\Rightarrow DH\perp(SAC)$ hay $DH=\mathrm{d}\left(D;(SAC)\right)$.\\
		Ta có: $\mathrm{d}\left(D;(SAC)\right)=\dfrac{DI}{BI}\cdot\mathrm{d}\left(B;(SAC)\right)=3\cdot\dfrac{2a}{\sqrt{21}}=\dfrac{6a}{\sqrt{21}}=DH$.\\
		Suy ra: $SD=\dfrac{DI\cdot DH}{\sqrt{DI^2-DH^2}}=\dfrac{\dfrac{3a}{2}\cdot\dfrac{6a}{\sqrt{21}}}{\sqrt{\dfrac{9a^2}{4}-\dfrac{12a^2}{7}}}=\dfrac{6a\sqrt{5}}{5}$.\\
		Vậy $V_{S.ABC}=\dfrac{1}{3}SD\cdot S_{\triangle ABC}=\dfrac{1}{3}\cdot\dfrac{6a\sqrt{5}}{5}\cdot\dfrac{1}{2}a\cdot a\cdot\dfrac{\sqrt{3}}{2}=\dfrac{a^3\sqrt{15}}{10}$.}
\end{ex}
\begin{ex}%[2H1K3-2]
	[THPT Triệu Sơn - Thanh Hóa - 2021] Cho hình chóp $S.ABC$ có đáy $ABC$ là tam giác đều cạnh bằng 1, biết khoảng cách từ $A$ đến mặt phẳng $(SBC)$ bằng $\dfrac{\sqrt{6}}{4}$, khoảng cách từ $B$ đến mặt phẳng $(SCA)$ bằng $\dfrac{\sqrt{15}}{10}$, khoảng cách từ $C$ đến mặt phẳng $(SAB)$ bằng $\dfrac{\sqrt{30}}{20}$ và hình chiếu vuông góc của $S$ xuống đáy nằm trong tam giác $ABC$. Tính thể tích khối chóp $V_{S.ABC}$. 
	\choice
	{$\dfrac{1}{24}$}
	{$\dfrac{1}{12}$}
	{$\dfrac{1}{36}$}
	{\True $\dfrac{1}{48}$}
	\loigiai{
		\begin{center}
			\begin{tikzpicture}[line join=round, line cap = round, >=stealth, scale=0.8,font=\footnotesize]
				\path
				(0,0) coordinate (A)
				(7,0) coordinate (C)
				(2.5,-3) coordinate (B)
				(3.5,-1) coordinate (H)
				($(H)+(0,5)$) coordinate (S)
				($(A)!.6!(C)$) coordinate (K)
				($(A)!.45!(B)$) coordinate (F)
				($(B)!.55!(C)$) coordinate (G)
				;
				\draw[dashed] (A)--(C) (G)--(H)--(S)--(K)--(H)--(F);
				\draw (S)--(A)--(B)--(C)--(S)--(B) (F)--(S)--(G);
				\foreach \i/\g in {A/180,B/-60,C/0,S/90,F/-150,H/-90, G/-60, K/-45}\fill[black] (\i) circle (2pt) +(\g:.4) node{$\i$};
			\end{tikzpicture}	
		\end{center}
		Gọi $H$ là hình chiếu của $S$ trên $(ABC)$. $F,G,K$ lần lượt là hình chiếu của $H$ trên $AB,BC,CA$.\\
		Đặt $V=V_{S.ABC};h=SH$.\\
		Ta có $3V=h\cdot S_{\triangle ABC}=\mathrm{d}(A,(SBC))\cdot S_{\triangle SBC}=\mathrm{d}(B,(SAC))\cdot S_{\triangle SAC}=\mathrm{d}(C,(SAB))\cdot S_{\triangle SAB}$ \\
		$ \Rightarrow\dfrac{\sqrt{3}}{4}h=\dfrac{\sqrt{6}}{4}\cdot\dfrac{1}{2}\cdot SF=\dfrac{\sqrt{15}}{10}\cdot\dfrac{1}{2}\cdot SG=\dfrac{\sqrt{30}}{20}\cdot\dfrac{1}{2}\cdot SK $ \\
		$ \Rightarrow SF=h\sqrt{2};SG=h\sqrt{5};SK=h\sqrt{10}\Rightarrow HF=h;HG=2h;HK=3h $.\\
		Mặt khác $S_{\triangle ABC}=S_{\triangle HAB}+S_{\triangle HBC}+S_{\triangle HCA}\Leftrightarrow\dfrac{\sqrt{3}}{4}=\dfrac{1}{2}HF+\dfrac{1}{2}HG+\dfrac{1}{2}HK\Leftrightarrow h=\dfrac{\sqrt{3}}{12}$.\\
		Vậy $V_{S.ABC}=\dfrac{1}{3}\cdot\dfrac{\sqrt{3}}{12}\cdot\dfrac{\sqrt{3}}{4}=\dfrac{1}{48}$.}
\end{ex}
\begin{ex}%[2H1K3-2]
	[THPT Ngô Quyền - Quảng Ninh - 2021] Tứ diện $ABCD$ có $AB=AC=AD=a$, $\widehat{BAC}=120^{\circ},\widehat{BAD}=60^{\circ}$ và tam giác $BCD$ là tam giác vuông tại $D$. Tính thể tích khối tứ diện $ABCD$. 
	\choice
	{$\dfrac{a^3\sqrt{2}}{4}$}
	{$\dfrac{a^3\sqrt{2}}{3}$}
	{$\dfrac{a^3\sqrt{2}}{6}$}
	{\True $\dfrac{a^3\sqrt{2}}{12}$}
	\loigiai{
		\begin{center}
			\begin{tikzpicture}[line join=round, line cap = round, >=stealth, scale=0.8,font=\footnotesize]
				\path
				(0,0) coordinate (B)
				(7,0) coordinate (D)
				(2.5,-3) coordinate (C)
				($(B)!.5!(C)$) coordinate (H)
				($(H)+(0,6)$) coordinate (A)
				;
				\draw[dashed] (B)--(D)--(H);
				\draw (A)--(B)node[midway, left]{$a$}--(C)--(D)--(A)node[midway, above]{$a$}--(H)--(C)--(A)node[midway, right]{$a$};
				\foreach \i/\g in {A/90,B/180,C/-90,D/0,H/180}\fill[black] (\i) circle (2pt) +(\g:.4) node{$\i$};
			\end{tikzpicture}	
		\end{center}
		Gọi $H$ là hình chiếu của $A$ lên $(BCD)$.\\
		Dễ thấy, $\triangle AHB=\triangle AHC=\triangle AHD\Rightarrow HB=HC=HD$.\\
		Do đó, $H$ là tâm đường tròn ngoại tiếp $\triangle BCD\Rightarrow H$ là trung điểm của $BC$.\\
		Xét tam giác $ABC$, có $BC^2=AB^2+AC^2-2AB\cdot AC\cdot\cos\widehat{BAC}=a^2+a^2-2a\cdot a\cdot\cos 120^{\circ}=3a^2$ \\
		$ \Rightarrow BC=a\sqrt{3}\Rightarrow BH=\dfrac{a\sqrt{3}}{2} $.\\
		Xét $\triangle AHB$ vuông tại $H$, có $AH=\sqrt{AB^2-BH^2}=\sqrt{a^2-\left(\dfrac{a\sqrt{3}}{2}\right)^2}=\dfrac{a}{2}$.\\
		Xét $\triangle ABD,$ có $AB=AD=a$ và $\widehat{BAD}=60^{\circ}\Rightarrow\triangle ABD$ là tam giác đều cạnh $a\Rightarrow BD=a$.\\
		Xét $\triangle BDC$ vuông tại $D$, có $CD=\sqrt{BC^2-BD^2}=\sqrt{3a^2-a^2}=a\sqrt{2}$ \\
		$ \Rightarrow S_{\triangle BDC}=\dfrac{1}{2}\cdot a\cdot a\sqrt{2}=\dfrac{a^2\sqrt{2}}{2} $ (đvdt).\\
		Vậy $V_{ABCD}=\dfrac{1}{3}AH\cdot S_{\triangle BCD}=\dfrac{1}{3}\cdot\dfrac{a}{2}\cdot\dfrac{a^2\sqrt{2}}{2}=\dfrac{a^3\sqrt{2}}{12}$ (đvtt).}
\end{ex}
\begin{ex}%[2H1G3-2]
	[Chuyên Bắc Giang - 2021] Cho hình chóp $S.ABC$ có đáy $ABC$ là tam giác vuông tại $B$, mặt bên $SAC$ là tam giác cân tại $S$ và nằm trong mặt phẳng vuông góc với mặt phẳng đáy. Hai mặt phẳng $(SAB)$ và $(SBC)$ lần lượt tạo với đáy các góc $60^{\circ}$ và $45^{\circ}$, khoảng cách giữa hai đường thẳng $SA$ và $BC$ bằng $a$. Tính thể tích khối chóp $S.ABC$ theo $a$. 
	\choice
	{\True $\dfrac{\sqrt{6}a^3}{18}$}
	{$\dfrac{\sqrt{2}a^3}{12}$}
	{$\dfrac{\sqrt{2}a^3}{6}$}
	{$\dfrac{\sqrt{6}a^3}{12}$}
	\loigiai{
		\begin{center}
			\begin{tikzpicture}[line join=round, line cap = round, >=stealth, scale=0.8,font=\footnotesize]
				\path
				(0,0) coordinate (C)
				(7,0) coordinate (A)
				(5.5,-3) coordinate (B)
				($(A)!.5!(C)$) coordinate (H)
				($(H)+(0,6)$) coordinate (S)
				($(A)!.5!(B)$) coordinate (Q)
				($(C)!.5!(B)$) coordinate (P)
				($(P)!2!(H)$) coordinate (K)
				($(S)!.6!(K)$) coordinate (I)
				;
				\draw[dashed] (S)--(K)--(A)--(H)--(S) (I)--(H) (K)--(P) (C)--(H)--(Q);
				\draw (S)--(C)--(B)--(A)--(S)--(Q)--(B)--(S)--(P);
				\foreach \i/\g in {A/90,B/180,C/-90,D/0,H/130, K/-90,I/150,S/90,P/-150,Q/-60}\fill[black] (\i) circle (2pt) +(\g:.4) node{$\i$};
				\begin{scope}
					\clip (S)--(P)--(H);
					\draw (P) circle(5mm);
				\end{scope}
				\path ($(P)!7mm!($(S)!.65!(H)$)$) node[scale=.8]{$45^{\circ}$};
				\begin{scope}
					\clip (S)--(Q)--(H);
					\draw (Q) circle(5mm);
				\end{scope}
				\path ($(Q)!7mm!($(S)!.65!(H)$)$) node[scale=.8]{$60^{\circ}$};
			\end{tikzpicture}	
		\end{center}
		Gọi $H$ là trung điểm cạnh $AC$, có $\triangle SAC$ cân tại $S$ nên $SH\perp AC$.\\
		Lại có: $(SAC)\perp(ABC)$.\\
		$(SAC)\cap(ABC)=AC$.\\
		Suy ra: $SH\perp(ABC)$.\\
		Kẻ $HP\perp BC, HQ\perp AB$.\\
		Ta có: $\heva{&BC\perp HP\\&BC\perp SH\left(\text{do } SH\perp(ABC)\right)}\Rightarrow BC\perp SP$.\\
		Vậy có: $\heva{&(SBC)\cap(ABC)=BC\\&SP\subset(SBC), SP\perp BC\\&HP\subset(ABC), HP\perp BC}\Rightarrow\left(\widehat{(SBC),(ABC)}\right)=\left(\widehat{SP,HP}\right)=\widehat{SPH}=45^{\circ.}$ \\
		Tương tự, $\left(\widehat{(SAB),(ABC)}\right)=\left(\widehat{SQ,HQ}\right)=\widehat{SQH}=60^{\circ}$.\\
		Từ $A$, kẻ đường thẳng $d$ // $BC$, kẻ $HK\perp d$, nối $SK$, kẻ $HI\perp HK$.\\
		Có $\heva{&AK\perp HK(\rm{cd})\\&AK\perp SH\left(\text{do } SH\perp(ABC), AK\subset(ABC)\right)\\&HK\cap SH=H\\&HK,SH\subset(SHK)}\Rightarrow AK\perp(SHK)\Rightarrow AK\perp HI$.\\
		Mà $HI\perp SK; AK\cap SK=K; AK,SK\subset(SAK)$ \\
		$ \Rightarrow HI\perp(SAK)\Rightarrow\mathrm{d}\left(H,(SAK)\right)=HI $.\\
		Ta có: $\heva{&BC\parallel AK\\&AK\subset(SAK)\\&BC\not\subset(SAK)}\Rightarrow BC\parallel (SAK)$ mà $SA\subset(SAK)$ \\
		$ \Rightarrow\mathrm{d}(SA,BC)=\mathrm{d}\left(BC,(SAK)\right)=\mathrm{d}\left(B,(SAK)\right)=2\mathrm{d}\left(H,(SAK)\right)=2HI=a $ \\
		$ \Rightarrow HI=\dfrac{a}{2} $.\\
		Lại có: $\heva{&BC\parallel AK\\&HK\perp AK, HP\perp BC}\Rightarrow H,K,P$ thẳng hàng và $\dfrac{HP}{HK}=\dfrac{HC}{HA}=1\Rightarrow HK=HP$. Đặt: $SH=x$ $(x>0)$.\\
		Tam giác $SHP$ vuông tại $H$, $\widehat{SPH}=45^{\circ}\Rightarrow HP=x\Rightarrow HK=x$.\\
		$\triangle SHK$ vuông tại $H, HI\perp SK\Rightarrow HI=\dfrac{SH\cdot HK}{\sqrt{SH^2+HK^2}}\Rightarrow\dfrac{a}{2}=\dfrac{x^2}{x\sqrt{2}}\Rightarrow x=\dfrac{a}{\sqrt{2}}$.\\
		Tam giác $SHQ$ vuông tại $H$, $\widehat{SPQ}=60^{\circ}\Rightarrow HQ=\dfrac{SH}{\tan{60}^{\circ}}=\dfrac{x}{\sqrt{3}}$.\\
		Mặt khác, $\triangle ABC$ vuông tại B nên $HP \parallel AB$, $HQ \parallel BC$ mà $H$ là trung điểm của $AC$ nên $HP, HQ$ là các đường trung bình của $\triangle ABC\Rightarrow AB=2x=a\sqrt{2}, BC=\dfrac{2x}{\sqrt{3}}=\dfrac{a\sqrt{2}}{\sqrt{3}}$.\\
		Vậy $V_{S.ABC}=\dfrac{1}{3}\cdot SH\cdot S_{ABC}=\dfrac{1}{3}\cdot\dfrac{a}{\sqrt{2}}\cdot\dfrac{1}{2}\cdot a\sqrt{2}\cdot\dfrac{a\sqrt{2}}{\sqrt{3}}=\dfrac{a^3\sqrt{6}}{18}$.}
\end{ex}
\begin{ex}%[2H1G3-2]
	[Chuyên Biên Hòa - 2021] Cho hình chóp $S.ABC$ có đáy $ABC$ là tam giác vuông cân tại $B$ với $BA=BC=3a$; $\widehat{SAB}=\widehat{SCB}=90^{\circ}$. Biết góc giữa hai mặt phẳng $(SBC)$ và $(SBA)$ bằng $\alpha$ với $\cos\alpha=\dfrac{1}{3}$. Thể tích của khối chóp $S.ABC$ bằng 
	\choice
	{$\dfrac{3\sqrt{2}a^3}{2}$}
	{\True $\dfrac{27\sqrt{2}a^3}{2}$}
	{$\dfrac{9\sqrt{2}a^3}{2}$}
	{$9\sqrt{2}a^3$}
	\loigiai{
		\begin{center}
			\begin{tikzpicture}[line join=round, line cap = round, >=stealth, scale=0.8,font=\footnotesize]
				\path
				(0,0) coordinate (A)
				(7,0) coordinate (B)
				($(A)+(2,3)$) coordinate (D)
				($(B)+(2,3)$) coordinate (C)
				($(D)+(0,5)$) coordinate (S)
				($(A)!.5!(C)$) coordinate (O)
				($(S)!.68!(B)$) coordinate (I)
				;
				\draw[dashed] (S)--(D)--(B) (A)--(C)--(D)--(A) (O)--(I);
				\draw (S)--(A)--(B)--(C)--(S)--(B) (C)--(I)--(A);
				\foreach \i/\g in {A/-150,B/-60,C/0,D/180,S/90,O/-90,I/30}\fill[black] (\i) circle (2pt) +(\g:.4) node{$\i$};
				\newcommand{\gocv}[4][black]{\draw[#1] ($(#3)!6pt!(#2)$)--($(#3)!2!($($(#3)!6pt!(#2)$)!.5!($(#3)!6pt!(#4)$)$)$)--($(#3)!6pt!(#4)$);}
				\gocv{A}{I}{B}
			\end{tikzpicture}	
		\end{center}
		Dựng hình bình hành $ABCD$.\\
		Từ gt $\Rightarrow ABCD$ là hình vuông; $SA\perp AB$; $SC\perp BC\Rightarrow SA\perp CD$; $SC\perp AD$.\\
		Do đó $CD\perp(SAD)$; $AD\perp(SCD)\Rightarrow CD\perp SD$; $AD\perp SD\Rightarrow SD\perp(ABCD)$.\\
		Trong $(ABCD)$: Gọi $O=AC\cap BD$.\\
		Trong $(SAB)$: Kẻ $AI\perp SB$ tại $I$.\\
		Mà $AC\perp BD$; $SD\perp AC\Rightarrow AC\perp(SBD)\Rightarrow AC\perp SB\Rightarrow SB\perp(CIA)$ \\
		$ \Rightarrow\widehat{\left((SBC);(SBA)\right)}=\widehat{(IA; IC)} $.\\
		Dễ thấy $\triangle SBD\sim\triangle OBI\Rightarrow\dfrac{OI}{SD}=\dfrac{OB}{SB}\Rightarrow OI=\dfrac{OB\cdot SD}{SB}=\dfrac{3\sqrt{2}SD}{2\sqrt{SD^2+18}}$ \\
		$ \Rightarrow IC=\sqrt{OI^2+OC^2}=\sqrt{\dfrac{9SD^2}{2\left(SD^2+18\right)}+\dfrac{9}{2}} $.\\
		Lại có: $\triangle SAB=\triangle SCB\Rightarrow IA=IC=\sqrt{\dfrac{9SD^2}{2\left(SD^2+18\right)}+\dfrac{9}{2}}$.\\
		Ta có:\\ $\begin{aligned}
			\cos\alpha=\left|\cos\widehat{CIA}\right|&\Leftrightarrow\left|\dfrac{IA^2+IC^2-AC^2}{2IA\cdot IC}\right|=\dfrac{1}{3}\\
			&\Leftrightarrow\left|\dfrac{\dfrac{9SD^2}{SD^2+18}+9-18}{\dfrac{9SD^2}{SD^2+18}+9}\right|=\dfrac{1}{3}\\
			&\Leftrightarrow SD=3\sqrt{2}.
		\end{aligned}$\\
		Vậy thể tích của khối chóp $S.ABC$ là $V_{S.ABC}=\dfrac{1}{3}SD\cdot S_{\triangle ABC}=\dfrac{1}{3}SD\cdot\dfrac{1}{2}AB\cdot BC=\dfrac{9\sqrt{2}a^3}{2}$.}
\end{ex}
\begin{ex}%[2H1G3-2]
	[Chuyên ĐHSP - 2021] Cho hình chóp $S.ABC$ có đáy $ABC$ là tam giác vuông tại $C$, $H$ là điểm thỏa mãn $\overrightarrow{HB}=-2\cdot\overrightarrow{HA}$ và $SH\perp(ABC)$, các mặt bên $(SAC)$ và $(SBC)$ cùng tạo với đáy góc $45^{\circ}$. Biết $SB=a\sqrt{6}$, thể tích khối chóp $S.ABC$ bằng
	\choice
	{$\dfrac{3a^3}{4}$}
	{\True $\dfrac{9a^3}{4}$}
	{$\dfrac{3\sqrt{2}a^3}{4}$}
	{$\dfrac{3a^3}{2}$}
	\loigiai{
		\begin{center}
			\begin{tikzpicture}[line join=round, line cap = round, >=stealth, scale=0.8,font=\footnotesize]
				\path
				(0,0) coordinate (B)
				(7,0) coordinate (C)
				(6,-3) coordinate (A)
				($(B)!.67!(A)$) coordinate (H)
				($(H)+(0,6)$) coordinate (S)
				($(C)!.67!(A)$) coordinate (M)
				($(B)!.67!(C)$) coordinate (N)
				;
				\draw[dashed] (S)--(N)--(H)--(M) (B)--(C);
				\draw (S)--(B)--(H)node[midway,below]{$2x$}--(A)node[midway,below]{$x$}--(C)--(S)--(M) (A)--(S)--(H);
				\foreach \i/\g in {A/-60,B/180,C/0,S/90,H/-120,M/-30,N/150}\fill[black] (\i) circle (2pt) +(\g:.4) node{$\i$};
				\newcommand{\gocv}[4][black]{\draw[#1] ($(#3)!6pt!(#2)$)--($(#3)!2!($($(#3)!6pt!(#2)$)!.5!($(#3)!6pt!(#4)$)$)$)--($(#3)!6pt!(#4)$);}
				\gocv{B}{C}{A}
			\end{tikzpicture}	
		\end{center}
		Gọi $M,N$ lần lượt là hình chiếu của $H$ trên các cạnh $CA,CB$. Ta có: $\widehat{SMH}=\widehat{SNH}=45^{\circ}$.\\
		Suy ra: $HM=SH=HN$ hay $MCNH$ là hình vuông.\\
		Đặt $HA=x$, suy ra $HB=2x$.\\
		$\heva{&\dfrac{AC}{MC}=\dfrac{AB}{HB}=\dfrac{3}{2}\Rightarrow AC=\dfrac{3}{2}MC\\&\dfrac{BC}{NC}=\dfrac{BA}{HA}=3\Rightarrow BC=3NC\\&MC=NC}\Rightarrow BC=2AC$.\\
		Tam giác $ABC$ vuông tại $C$, suy ra $CB^2+CA^2=BC^2\Leftrightarrow 5CA^2=9x^2\Leftrightarrow CA=\dfrac{3x\sqrt{5}}{5}$.\\
		Suy ra: $SH=HM=CM=\dfrac{2}{3}AC=\dfrac{2x\sqrt{5}}{5}$.\\
		Mặt khác tam giác $SHB$ vuông tại $H$ nên $SB^2=BH^2+HS^2\Leftrightarrow 6a^2=4x^2+\dfrac{20}{25}x^2\Leftrightarrow x=\dfrac{a\sqrt{5}}{2}$.\\
		Vậy thể tích khối chóp $S.ABC$ bằng\\
		$V_{S.ABC}=\dfrac{1}{6}\cdot SH\cdot CA\cdot CB=\dfrac{1}{6}\cdot\dfrac{2x\sqrt{5}}{5}\cdot\dfrac{3x\sqrt{5}}{5}\cdot 2\dfrac{3x\sqrt{5}}{5}=\dfrac{6x^3\sqrt{5}}{25}=\dfrac{3a^3}{4}$.}
\end{ex}
\begin{ex}%[2H1G3-2]
	[Sở Bình Phước - 2021] Cho hình chóp tam giác đều $S.ABC$, cạnh đáy bằng $a$. Các điểm $M,N$ lần lượt là trung điểm của $SA,SC$. Biết rằng $BM$ vuông góc với $AN$. Thể tích của khối chóp bằng
	\choice
	{$\dfrac{\sqrt{7}}{24}a^3$}
	{$\dfrac{\sqrt{7}}{8}a^3$}
	{$\dfrac{\sqrt{14}}{8}a^3$}
	{\True $\dfrac{\sqrt{14}}{24}a^3$}
	\loigiai{
		\begin{center}
			\begin{tikzpicture}[line join=round, line cap = round, >=stealth, scale=0.8,font=\footnotesize]
				\path
				(0,0) coordinate (A)
				(8,0) coordinate (C)
				(5,-2.5) coordinate (B)
				($(A)!.5!(C)$) coordinate (x)
				($(B)!.67!(x)$) coordinate (H)
				($(H)+(0,6)$) coordinate (S)
				($(S)!.5!(A)$) coordinate (M)
				($(S)!.5!(C)$) coordinate (N)
				($(C)!1.35!(A)$) coordinate (D)
				;
				\draw[dashed] (S)--(H) (B)--(x) (M)--(N)--(A)--(C);
				\draw (S)--(A)--(B)--(C)--(S)--(B)--(M)--(D)--(A);
				\foreach \i/\g in {A/-120,B/-90,C/0,S/90,H/180,M/120,N/30,D/180}\fill[black] (\i) circle (2pt) +(\g:.4) node{$\i$};
			\end{tikzpicture}	
		\end{center}
		Gọi $D$ sao cho $MNAD$ là hình bình hành. $BM$ vuông góc với $AN$ nên tam giác $DMB$ vuông cân tại $M$. Suy ra: $BM=\dfrac{BD}{\sqrt{2}}=\dfrac{\sqrt{a^2+\left(\dfrac{a\sqrt{3}}{2}\right)^2}}{\sqrt{2}}=\dfrac{a\sqrt{14}}{4}$.\\
		Gọi cạnh $SA=x, x>0$. $BM$ là đường trung tuyến tam giác $SAB$ nên ta có:\\
		$BM^2=\dfrac{2\left(BA^2+BS^2\right)-SA^2}{4}\Leftrightarrow\left(\dfrac{a\sqrt{14}}{4}\right)^2=\dfrac{2\left(a^2+x^2\right)-x^2}{4}\Leftrightarrow x=\dfrac{a\sqrt{6}}{2}$.\\
		$SH=\sqrt{SA^2-AH^2}=\dfrac{a\sqrt{42}}{6}$. Vậy $V_{S.ABC}=\dfrac{1}{3}\cdot\dfrac{a\sqrt{42}}{6}\cdot\dfrac{a^2\sqrt{3}}{4}=\dfrac{a^3\sqrt{14}}{24}$.}
\end{ex}
\begin{ex}%[2H1K3-2]
	[Sở Quảng Bình - 2021] Cho hình hộp $ABCD.A'B'C'D'$ có đáy là hình thoi cạnh $a$, $\widehat{BCD}=120^{{O}}$ và $AA'=\dfrac{7a}{2}$, hình chiếu vuông góc của $A'$ lên mặt phẳng $(ABCD)$ trùng với giao điểm của $AC$ và $BD$. Gọi $M$, $N$, $P$, $R$ lần lượt là trung điểm của $AB',B'D',AD',DC'$ và $Q$ là trung điểm của $BR$. Thể tích của khối chóp $MNPQ$ bằng
	\choice
	{$\dfrac{a^3}{16}$}
	{$\dfrac{a^3}{24}$}
	{\True $\dfrac{a^3}{8}$}
	{$\dfrac{a^3\sqrt{3}}{27}$}
	\loigiai{
		\begin{center}
			\begin{tikzpicture}[line join=round, line cap = round, >=stealth, scale=0.8,font=\footnotesize]
				\path
				(0,0) coordinate (B)
				(8,0) coordinate (A)
				($(A)+(3,2)$) coordinate (D)
				($(B)+(3,2)$) coordinate (C)
				($(A)!.5!(C)$) coordinate (O)
				($(O)+(0,7)$) coordinate (A')
				($(A')+(3,2)$) coordinate (D')
				($(A')+(-8,0)$) coordinate (B')
				($(D')+(-8,0)$) coordinate (C')
				($(A)!.5!(B')$) coordinate (M)
				($(B')!.5!(D')$) coordinate (N)
				($(A)!.5!(D')$) coordinate (P)
				($(D)!.5!(C')$) coordinate (R)
				($(B)!.5!(R)$) coordinate (Q)
				;
				\draw[dashed] (A)--(C)--(C')--(D)--(C)--(B)--(R)--(C)--(M)--(N)--(P)--(M) (O)--(A');
				\draw (B')--(B)--(A)--(D)--(D')--(A)--(A')--(D')--(C')--(B')--(D') (A')--(B')--(A) (M)--(B);
				\foreach \i/\g in {A/-120,B/-150,C/-90,D/0,A'/90,B'/180,C'/90,D'/30,M/40,N/130,P/0,Q/-15,R/90,O/-90}\fill[black] (\i) circle (2pt) +(\g:.4) node{$\i$};
			\end{tikzpicture}	
		\end{center}
		$A'O=\sqrt{A'A^2-AO^2}=2a\sqrt{3}; S_{ABCD}=2S_{\triangle ABC}=\dfrac{a^2\sqrt{3}}{2}\Rightarrow V_{ABCD.A'B'C'D'}=3a^3$.\\
		Ta có: $\mathrm{d}\left(Q,(MNP)\right)=\dfrac{1}{2}\mathrm{d}\left(C;(MNP)\right)=\dfrac{1}{2}\mathrm{d}\left(C;(AB'D')\right), S_{\triangle MNP}=\dfrac{1}{4}S_{\triangle AB'D'}$.\\
		$\begin{aligned}V_{Q.MNP}&=\dfrac{1}{3}\mathrm{d}\left(Q;(MNP)\right)\cdot S_{\triangle MNP}=\dfrac{1}{3}\cdot\dfrac{1}{2}\mathrm{d}\left(C;(AB'D')\right)\cdot\dfrac{1}{4}S_{\triangle AB'D'}\\&=\dfrac{1}{8}V_{CAB'D'}=\dfrac{1}{8}\cdot\dfrac{V_{ABCD.A'B'C'D'}}{3}=\dfrac{a^3}{8}.\end{aligned}$}
\end{ex}
\begin{ex}%[2H1G3-2]
	[Sở Hưng Yên - 2021] Cho khối hộp $ABCD.A'B'C'D'$ có $A'B$ vuông góc với mặt phẳng đáy $(ABCD)$; góc giữa $AA'$ với $(ABCD)$ bằng $45^{\circ}$. Khoảng cách từ $A$ đến các đường thẳng $BB',DD'$ cùng bằng 1. Góc giữa hai mặt phẳng $(BB'C'C)$ và $(C'CDD')$ bằng $60^{\circ}$. Tính thể tích khối hộp $ABCD.A'B'C'D'$ 
	\choice
	{\True $\sqrt{3}$}
	{$2$}
	{$2\sqrt{3}$}
	{$3\sqrt{3}$}
	\loigiai{
		\begin{center}	
			\begin{tikzpicture}[line join=round, line cap = round, >=stealth, scale=1,font=\footnotesize]
				\path
				(0,0) coordinate (A)
				(5,0) coordinate (D)
				($(A)+(2,-1.5)$) coordinate (B)
				($(D)+(2,-1.5)$) coordinate (C)
				($(A)+(1.5,5)$) coordinate (A')
				($(B)+(1.5,5)$) coordinate (B')
				($(C)+(1.5,5)$) coordinate (C')
				($(D)+(1.5,5)$) coordinate (D')
				($(B)!.55!(B')$) coordinate (H)
				($(D)!.75!(D')$) coordinate (K)
				;
				\draw[dashed] (A')--(H)--(K)--(A') (D')--(D)--(C) (A)--(D);
				\draw (A)--(B)--(C)--(C')--(B')--(B)--(A')--(A) (B')--(D')--(C')--(B')--(A')--(D');
				\foreach \i/\g in {A/180,B/-150,C/-30,D/0,A'/120,B'/180,C'/0,D'/30,H/-15,K/0}\fill[black] (\i) circle (2pt) +(\g:.4) node{$\i$};
				\begin{scope}
					\clip (A')--(A)--(B);
					\draw (A) circle(5mm);
				\end{scope}
				\path ($(A)!7mm!($(A')!.65!(B)$)$) node[scale=.8]{$45^{\circ}$};
				\newcommand{\gocv}[4][black]{\draw[#1] ($(#3)!5pt!(#2)$)--($(#3)!2!($($(#3)!5pt!(#2)$)!.5!($(#3)!5pt!(#4)$)$)$)--($(#3)!5pt!(#4)$);}
				\gocv{A'}{H}{B'}
				\gocv{A'}{K}{D'}
			\end{tikzpicture}	
		\end{center}
		Ta có: $A'B\perp(ABCD)\Rightarrow\left(AA';(ABCD)\right)=\widehat{A'AB}=45^{\circ}$.\\
		Gọi $H,K$ lần lượt là hình chiếu vuông góc của $A'$ lên $BB'$ và $DD'$ \\
		$ \Rightarrow A'H=A'K=1 $ và $\heva{&AA'\perp A'H\\&AA'\perp A'K}\Rightarrow AA'\perp(A'HK)$.\\
		Xét hình bình hành $ABB'A'$ có $\heva{&A'B\perp AB\\&\widehat{A'AB}=45^{\circ}}\Rightarrow\triangle A'AB,\triangle A'B'B$ vuông cân tại $B$ và $A'$.\\
		Do đó $H$ là trung điểm $BB'\Rightarrow A'H=\dfrac{1}{2}BB'\Rightarrow BB'=2A'H=2$.\\
		Xét $\triangle AA'B$ vuông cân tại $B\Rightarrow A'B=\dfrac{AA'}{\sqrt{2}}=\sqrt{2}$.\\
		Do $ABCD.A'B'C'D'$ là hình hộp nên $\left((BB'C'C);(C'CDD')\right)=\left((ABB'A');(ADD'A')\right)$.\\
		Mà $\left((ABB'A');(ADD'A')\right)=(A'H;A'K)=60^{\circ}$.\\
		Do đó $\widehat{HA'K}=60^{\circ}$ hoặc $\widehat{HA'K}=120^{\circ}$.\\
		Ta có: $S_{\triangle A'HK}=\dfrac{1}{2}A'H\cdot A'K\cdot\sin\widehat{HA'K}=\dfrac{\sqrt{3}}{4}$.\\
		Mặt khác: $\heva{&A'A\perp(A'HK)\\&A'B\perp(A'B'C'D')}\Rightarrow\left((A'HK);(A'B'C'D')\right)=(A'A;A'B)=45^{\circ.}$ \\
		Lại có: $\triangle A'HK$ là hình chiếu vuông góc của $\triangle A'B'D'$ nên:\\
		$S_{\triangle A'HK}=S_{\triangle A'B'D'}\cdot\cos 45^{\circ}\Rightarrow S_{\triangle A'B'D'}=\dfrac{\sqrt{6}}{4}$.\\
		Suy ra: $V_{ABCD.A'B'C'D'}=2V_{ABD\cdot A'B'D'}=2\cdot A'B\cdot S_{\triangle A'B'D'}=2\cdot\sqrt{2}\cdot\dfrac{\sqrt{6}}{4}=\sqrt{3}$ (đvtt).}
\end{ex}
\begin{ex}%[2H1K3-2]
	(Cụm Trường Nghệ An - 2022) Cho tứ diện $ABCD$ có $AC=2CD=DB=2a$. Gọi $H$ và $K$ lần lượt là hình chiếu vuông góc của $A$ và $B$ lên đường thẳng $CD$ sao cho $H,C,D,K$ theo thứ tự cách đều nhau. Biết góc tạo bởi $AH$ và $BK$ bằng $60^{\circ}$. Thể tích khối tứ diện $ABCD$ bằng
	\choice
	{$\dfrac{a^3\sqrt{3}}{6}$}
	{$\dfrac{a^3\sqrt{3}}{8}$}
	{$\dfrac{a^3\sqrt{3}}{3}$}
	{\True $\dfrac{a^3\sqrt{3}}{4}$}
	\loigiai{
		\immini{
			Ta có $HC=CD=DK=a; AC=2a; BD=2a$.\\
			Tam giác $AHC$ vuông tại $H$ nên
			$$AH=\sqrt{AC^2-HC^2}=\sqrt{4a^2-a^2}=a\sqrt{3}.$$
			Tam giác $BKD$ vuông tại $K$ nên $$BK=\sqrt{BD^2-HK^2}=\sqrt{4a^2-a^2}=a\sqrt{3}.$$
			Tứ diện $ABKH$ có cặp cạnh đối $AH=BK=a\sqrt{3}$, $(AH;BK)=60^{\circ}$, và $\mathrm{d}(AH;BK)=HK=3a$.
		}{
			\begin{tikzpicture}
				\coordinate (A) at (0,0);
				\coordinate (H) at (3,1);
				\coordinate (C) at (3,2);
				\coordinate (D) at (3,3);
				\coordinate (K) at (3,4);
				\coordinate (B) at (2,5);
				\draw(B)--(A)--(H)--(K)--(B)--(H)(B)--(C)(B)--(D);
				\draw[dashed](A)--(C)(A)--(D)(A)--(K);
				\foreach \p/\g in {A/-90,H/0,C/0,D/0,K/0,B/90}\draw[fill=black] (\p) circle (.1pt)node[shift={(\g:.3)},scale=1]{$\p$};
				\foreach \i/\j/\k/\t in {H/A/C/6,K/D/B/6}{
					\def\dgiua{\i}\def\dmot{\j}\def\dhai{\k}\def\tyso{\t pt}
					\draw ($(\dgiua)!\tyso!(\dmot)$)--($(\dgiua)!2!($($(\dgiua)!\tyso!(\dmot)$)!.5!($(\dgiua)!\tyso!(\dhai)$)$)$)--($(\dgiua)!\tyso!(\dhai)$);}
			\end{tikzpicture}
		}
		Suy ra $V_{ABKH}=\dfrac{1}{6}AH\cdot BK\cdot\mathrm{d}(AH,BK)\cdot\sin (AH,BK)=\dfrac{1}{6}\cdot a\sqrt{3}\cdot a\sqrt{3}\cdot 3a\cdot\dfrac{\sqrt{3}}{2}=\dfrac{3\sqrt{3}}{4}a^3$.\\
		Dễ thấy $V_{ABCD}=\dfrac{1}{3}V_{ABKH}=\dfrac{1}{3}\cdot\dfrac{3a^3\sqrt{3}}{4}=\dfrac{a^3\sqrt{3}}{4}$.}
\end{ex}
\begin{ex}%[2H1K3-2]
	(THPT Hồ Nghinh – Quảng Nam – 2022) Cho khối hộp chữ nhật $ABCD.A'B'C'D'$. Khoảng cách giữa 2 đường thẳng $AB,CB'$ bằng $\dfrac{2}{\sqrt{5}}a$, khoảng cách giữa 2 đường thẳng $A'D',B'A$ bằng $\dfrac{2}{\sqrt{5}}a$. Khoảng cách giữa 2 đường thẳng $BD'$, $AC$ bằng $\dfrac{\sqrt{2}}{\sqrt{6}}a$. Tính thể tích khối hộp chữ nhật đã cho. 
	\choice
	{$a^3$}
	{$\dfrac{a^3}{2}$}
	{\True $2a^3$}
	{$\sqrt{2}a^3$}
	\loigiai{
		\immini{
			Giả sử các kích thước của hình hộp chữ nhật là $AB=x$, $AD=y$, $AA'=z$ với $x,y,z>0$.\\
			Khoảng cảch giữa hai đường thẳng $AB$ và $B'C$ bằng $\dfrac{2a\sqrt{5}}{5}$.\\
			Ta có $\heva{&AB\parallel CD\\&CD\subset(A'B'CD)\\&AB\not\subset(A'B'CD)}\Rightarrow AB\parallel(A'B'CD)$. Do đó
			\begin{align*}
				\mathrm{d}(AB;B'C)&=\mathrm{d}\left(AB;(A'B'CD)\right)=\mathrm{d}\left(A,(A'B'CD)\right)\cr
				&=AH=\dfrac{2a\sqrt{5}}{5},
			\end{align*}
			với $H$ là hình chiếu của $A$ trên $A'D$.\\
			Từ $\dfrac{1}{AH^2}=\dfrac{1}{AA'}^2+\dfrac{1}{AD^2}\Rightarrow\dfrac{1}{y^2}+\dfrac{1}{z^2}=\dfrac{5}{4a^2}$.\hfill(1)\\
		}{
			\begin{tikzpicture}
				\coordinate (A) at (0,0);
				\coordinate (B) at (-1,-1);
				\coordinate (C) at (3,-1);
				\coordinate (D) at ($(A)+(C)-(B)$);
				\coordinate (A') at ($(A)+(0,3)$);
				\coordinate (B') at ($(B)+(0,3)$);
				\coordinate (C') at ($(C)+(0,3)$);
				\coordinate (D') at ($(D)+(0,3)$);
				\coordinate (H) at ($(A')!1/4!(D)$);
				\coordinate (K) at ($(A)!1/4!(B')$);
				\coordinate (O) at ($(A)!1/2!(C)$);
				\coordinate (I) at ($(D)!1/2!(D')$);
				\draw(B)--(C)--(C')--(B')--(B)(B')--(A')--(D')--(C')--(B')(C)--(D)--(D');
				\draw[dashed](B)--(A)--(A')--(D)(A)--(H)(B')--(A)--(D)--(B)--(D')(A')--(K)(A)--(C)--(I)--(A);
				\foreach \p/\g in {A'/90,B'/180,C'/0,D'/90,A/-90,B/-90,C/-90,D/-90, H/45,O/-90,K/-135, I/0}\draw[fill=black] (\p) circle (.1pt)node[shift={(\g:.3)},scale=1]{$\p$};	
				\foreach \i/\j/\k/\t in {H/A/D/6,K/A/A'/6}{
					\def\dgiua{\i}\def\dmot{\j}\def\dhai{\k}\def\tyso{\t pt}
					\draw ($(\dgiua)!\tyso!(\dmot)$)--($(\dgiua)!2!($($(\dgiua)!\tyso!(\dmot)$)!.5!($(\dgiua)!\tyso!(\dhai)$)$)$)--($(\dgiua)!\tyso!(\dhai)$);}
			\end{tikzpicture}
		}
		Khoảng cách giữa hai đường thẳng $A'D'$ và $AB'$ bằng $\dfrac{2a\sqrt{5}}{5}$.\\
		Tương tự, ta có $A'D'\parallel(AB'C'D)\Rightarrow\mathrm{d}(A'D';AB')=\mathrm{d}\left(A'D',(AB'C'D)\right) =A'K=\dfrac{2a\sqrt{5}}{5}$, với $K$ là hình chiếu của $A'$ trên $AB'$.\\
		Từ $\dfrac{1}{A'K^2}=\dfrac{1}{A'A^2}+\dfrac{1}{A'B'^{2}}\Rightarrow\dfrac{1}{x^2}+\dfrac{1}{z^2}=\dfrac{5}{4a^2}$.\hfill(2)\\
		Khoảng cách giữa hai đường thẳng $AC$ và $BD'$ bằng $\dfrac{a\sqrt{3}}{3}$.\\
		Gọi $\{O\}=AC\cap BD\Rightarrow O$ là trung điểm của $BD$.\\
		Gọi $I$ là trung điểm của $DD'$ thì $OI$ là đường trung bình của 	$\triangle BDD'\Rightarrow OI\parallel BD'\Rightarrow BD'\parallel~(ACI)$. Do đó
		$$\mathrm{d}(BD';AC)=\mathrm{d}(BD';(ACI))=\mathrm{d}(D';(ACI))=\mathrm{d}(D;(ACI)).$$
		Ta thấy $DI,DA,DC$ đôi một vuông góc với nhau nên
		$$\dfrac{1}{\mathrm{d}^2(D,(ACI))}=\dfrac{1}{DA^2}+\dfrac{1}{DC^2}+\dfrac{1}{DI^2}=\dfrac{1}{DA^2}+\dfrac{1}{DC^2}+\dfrac{4}{DD'^{2}}\Rightarrow\dfrac{1}{x^2}+\dfrac{1}{y^2}+\dfrac{4}{z^2}=\dfrac{3}{a^2}.\eqno(3)$$
		Từ (1),(2),(3) ta có hệ $\heva{&\dfrac{1}{y^2}+\dfrac{1}{z^2}=\dfrac{5}{4a^2}\\&\dfrac{1}{x^2}+\dfrac{1}{z^2}=\dfrac{5}{4a^2}\\&\dfrac{1}{x^2}+\dfrac{1}{y^2}+\dfrac{4}{z^2}=\dfrac{3}{a^2}}\Leftrightarrow\heva{&\dfrac{1}{x^2}=\dfrac{1}{a^2}\\&\dfrac{1}{y^2}=\dfrac{1}{a^2}\\&\dfrac{1}{z^2}=\dfrac{1}{4a^2}}\Leftrightarrow\heva{&x=a\\&y=a\\&z=2a.}$\\
		Vậy thể tích khối hộp là $V=xyz=a\cdot a\cdot 2a=2a^3$.}
\end{ex}
\begin{ex}%[2H1K3-2]
	(Liên trường Hà Tĩnh - 2022) Cho hình chóp tứ giác $S.ABCD$ có đáy là hình vuông; mặt bên $(SAB)$ là tam giác vuông cân tại $S$ và nằm trong mặt phẳng vuông góc với đáy. Biết khoảng cách giữa hai đường thẳng $AB$ và $SD$ bằng $\dfrac{3\sqrt{5}a}{5}$. Tính thể tích $V$ của khối chóp $S.ABCD$. 
	\choice
	{$V=\dfrac{3}{2}a^3$}
	{$V=\dfrac{6\sqrt{3}}{2}a^3$}
	{$V=\dfrac{27}{2}a^3$}
	{\True $V=\dfrac{9}{2}a^3$}
	\loigiai{
		\immini{
			Gọi $I,J$ lần lượt là trung điểm của $AB,CD$; $K$ là hình chiếu của $I$ lên $SJ$.\\
			Đặt cạnh đáy bằng $AB=x$ khi đó $SI=\dfrac{x}{2},IJ=x$.\\
			Vì $AB\parallel CD$ nên. Suy ra.\\
			$\mathrm{d}(I;(SCD))=IK=\dfrac{IS\cdot IJ}{\sqrt{IS^2+IJ^2}}$ \\
			$ \Leftrightarrow\dfrac{3a\sqrt{5}}{5}=\dfrac{x\cdot\dfrac{x}{2}}{\sqrt{x^2+\dfrac{x^2}{4}}}\Leftrightarrow x=3a $.\\
			Từ đó suy ra $V=\dfrac{1}{3}\cdot\dfrac{x}{2}\cdot x^2=\dfrac{9a^3}{2}$.
		}{
			\begin{tikzpicture}
				\coordinate (A) at (0,0);
				\coordinate (B) at (-1,-1);
				\coordinate (C) at (3,-1);
				\coordinate (D) at ($(A)+(C)-(B)$);
				\coordinate (I) at ($(A)!1/2!(B)$);
				\coordinate (J) at ($(C)!1/2!(D)$);
				\coordinate (S) at ($(I)+(0,4)$);
				\coordinate (K) at ($(S)!2/3!(J)$);	
				\draw(S)--(B)--(C)--(S)--(J)(S)--(D)--(C);
				\draw[dashed](B)--(A)--(S)--(I)--(J)(I)--(K)(A)--(D);
				\foreach \p/\g in {A/45,B/180,C/0,D/0,J/0,I/-90,K/-90,S/90}\draw[fill=black] (\p) circle (1pt)node[shift={(\g:.3)},scale=1]{$\p$};
			\end{tikzpicture}
	}}
\end{ex}
\begin{ex}%[2H1K3-2]
	(Sở Hà Tĩnh 2022) Cho lăng trụ $ABCD.A'B'C'D$ có đáy là hình chữ nhật với $AB=\sqrt{6}$, $AD=\sqrt{3}$, $A'C=3$ và mặt phẳng $(AA'C'C)$ vuông góc với mặt đáy. Biết hai mặt phẳng $(AA'C'C)$ và $(AA'B'B)$ tạo với nhau góc $\alpha$ có $\tan\alpha=\dfrac{3}{4}$. Thể tích $V$ của khối lăng trụ $ABCD.A'B'C'D'$ là
	\choice
	{$12$}
	{$6$}
	{\True $8$}
	{$10$}
	\loigiai{
		\immini{
			Dễ thấy $A'C'=AC=BD=3$ nên tam giác $A'CC'$ cân tại $A'$, do đó $A'F\perp CC'$, với $F$ là trung điểm của $CC'$. Gọi $E$ là điểm thỏa mãn $\overrightarrow{C'E}=\dfrac{3}{2}\overrightarrow{C'D'}$.
			Khi đó $C'E=\dfrac{3\sqrt{6}}{2}$ và $D'E=\dfrac{\sqrt{6}}{2}$, suy ra
			$$A'E^2+A'C'^2=A'D'^2+D'E^2+A'C'^2=\dfrac{27}{2}=C'E^2.$$
			Suy ra $A'E\perp A'C'$,  từ đó $A'E\perp (ACC'A')$, suy ra $EA'\perp A'F$ và $CC'\perp(EA'F)$.\\
			Do đó
			\begin{align*}
				\widehat{EFA'}&=\widehat{(A'F,EF)}=\widehat{\left((AA'C'A),(CDD'C')\right)}\cr
				&=\widehat{\left((AA'C'C),(AA'B'B)\right)}=\alpha.
			\end{align*}
		}{
			\begin{tikzpicture}
				\coordinate (A) at (0,0);
				\coordinate (B) at (-1,-1);
				\coordinate (C) at (3,-1);
				\coordinate (D) at ($(A)+(C)-(B)$);
				\coordinate (A') at ($(A)+(1,4)$);
				\coordinate (B') at ($(B)+(1,4)$);
				\coordinate (C') at ($(C)+(1,4)$);
				\coordinate (D') at ($(D)+(1,4)$);
				\coordinate (F) at ($(C)!1/2!(C')$);
				\coordinate (E) at ($(C')!3/2!(D')$);
				\draw(B)--(C)--(C')--(B')--(B)(B')--(A')--(D')--(C')--(B')(C)--(D)--(D')--(E)--(A')--(C');
				\draw[dashed](B)--(A)--(A')(A)--(D)(A)--(C)--(A')--(F);
				\foreach \p/\g in {A'/90,B'/180,C'/0,D'/0, E/0,A/-90,B/-90,C/-90,D/-90, F/0}\draw[fill=black] (\p) circle (.1pt)node[shift={(\g:.3)},scale=1]{$\p$};
			\end{tikzpicture}
		}
		Ta có $EA'=\sqrt{D'E^2+A'D'^2}=\dfrac{3\sqrt{2}}{2}$.\\
		Suy ra $A'F=A'E\cot\alpha=2\sqrt{2}$ và $CC'=2\sqrt{A'C'^2-A'F^2}~=2$. Do đó chiều cao của khối lăng trụ là
		$$h=\mathrm{d}\left(C,\left(A'B'C'D'\right)\right)=\mathrm{d}(C,A'C')=\dfrac{A'F\cdot CC'}{A'C'}=\dfrac{4\sqrt{2}}{3}.$$
		Vậy $V=AB\cdot AD\cdot h=8$.
	}
\end{ex}
\begin{ex}%[2H1K3-2]
	(Sở Phú Thọ 2022) Cho hình lăng trụ đứng $ABCD.A'B'C'D'$ có đáy là hình vuông cạnh $a$, góc giữa $AC$ và mặt phẳng $(A'CD)$ bằng $30^{\circ}$. Gọi $M$ là điểm sao cho $\overrightarrow{A'M}=\dfrac{1}{3}\overrightarrow{A'B}$. Thể tích khối tứ diện $A'CDM$ bằng
	\choice
	{\True $\dfrac{a^3}{18}$}
	{$\dfrac{a^3}{3}$}
	{$\dfrac{a^3\sqrt{3}}{12}$}
	{$\dfrac{a^3\sqrt{3}}{3}$}
	\loigiai{
		\immini{
			Kẻ $AE\perp A'D$.\\
			Ta có $\heva{&CD\perp AD\\&CD\perp DD'\\&AD,DD'\subset(ADD'A')\\&AD\cap DD'=D}\Rightarrow CD\perp(ADD'A'),$ mà $AE\subset(ADD'A')\Rightarrow AE\perp CD$.\\
			Suy ra $\heva{&AE\perp CD\\&AE\perp A'D\\&CD,A'D\subset(A'CD)\\&CD\cap A'D=D}\Rightarrow AE\perp(A'CD)$.}{
			\begin{tikzpicture}
				\coordinate (A) at (0,0);
				\coordinate (B) at (-1,-1);
				\coordinate (C) at (3,-1);
				\coordinate (D) at ($(A)+(C)-(B)$);
				\coordinate (A') at ($(A)+(0,3)$);
				\coordinate (B') at ($(B)+(0,3)$);
				\coordinate (C') at ($(C)+(0,3)$);
				\coordinate (D') at ($(D)+(0,3)$);
				\coordinate (E) at ($(A')!1/4!(D)$);
				\coordinate (M) at ($(A')!1/3!(B)$);
				\draw(B)--(C)--(C')--(B')--(B)(B')--(A')--(D')--(C')--(B')(C)--(D)--(D');
				\draw[dashed](B)--(A)--(A')--(D)(A)--(E)(A)--(D)(A')(A')--(C)--(E)(A')--(B)(C)--(M)--(D);
				\foreach \p/\g in {A'/90,B'/180,C'/0,D'/90,A/-90,B/-90,C/-90,D/-90, E/45, M/180}\draw[fill=black] (\p) circle (1pt)node[shift={(\g:.3)},scale=1]{$\p$};	
				\foreach \i/\j/\k/\t in {H/A/D/6}{
					\def\dgiua{\i}\def\dmot{\j}\def\dhai{\k}\def\tyso{\t pt}
					\draw ($(\dgiua)!\tyso!(\dmot)$)--($(\dgiua)!2!($($(\dgiua)!\tyso!(\dmot)$)!.5!($(\dgiua)!\tyso!(\dhai)$)$)$)--($(\dgiua)!\tyso!(\dhai)$);}
			\end{tikzpicture}
		}
		Hình chiếu vuông góc của $AC$ lên mặt phẳng $(A'CD)$ là $EC$ nên
		$$\widehat{\left(AC,(A'CD)\right)}=\widehat{(AC,EC)}=\widehat{ACE}=30^{\circ}.$$
		Xét tam giác $ACE$ vuông ở $E\Rightarrow AE=AC\cdot\sin 30^{\circ}=\dfrac{a\sqrt{2}}{2}$.\\
		Ta có chiều cao của hình chóp $A'CDM$ hạ từ đỉnh $M$ là 
		$$h=\mathrm{d}\left(M,(A'CD)\right)=\dfrac{1}{3}\mathrm{d}\left(B,(A'CD)\right)=\dfrac{1}{3}\mathrm{d}\left(A,(A'CD)\right)=\dfrac{AE}{3}=\dfrac{a\sqrt{2}}{6}.$$
		Xét tam giác $AA'D$ vuông ở $A$ có.\\
		$AE\perp A'D\Rightarrow\dfrac{1}{AE^2}=\dfrac{1}{AA'^2}+\dfrac{1}{AD^2}\Rightarrow\dfrac{1}{AA'^2}=\dfrac{1}{AE^2}-\dfrac{1}{AD^2}=\dfrac{1}{\left(\dfrac{a\sqrt{2}}{2}\right)^2}-\dfrac{1}{a^2}\Rightarrow AA'=a$.\\
		Ta có diện tích tam giác $A'CD$ bằng $S_{A'CD}=\dfrac{1}{2}\cdot A'D\cdot DC=\dfrac{1}{2}a\sqrt{2}\cdot a=\dfrac{a^2\sqrt{2}}{2}$.\\
		Thể tích khối tứ diện $A'CDM$ bằng $V_{A'CDM}=\dfrac{1}{3}\cdot h\cdot S_{A'CD}=\dfrac{1}{3}\cdot\dfrac{a\sqrt{2}}{6}\cdot\dfrac{a^2\sqrt{2}}{2}=\dfrac{a^3}{18}$.}
\end{ex}
\begin{ex}%[2H1K3-2]
	(THPT Võ Nguyên Giáp - Quảng Bình - 2022) Cho hình hộp đứng $ABCD.A'B'C'D'$ có đáy $ABCD$ là hình vuông. Gọi $S$ là tâm hình vuông $A'B'C'D'$. Gọi $M$ và $N$ lần lượt là trung điểm của $SA$ và $BC$. Biết rằng, nếu $MN$ tạo với mặt phẳng $(ABCD)$ một góc $60^{\circ}$ và $AB=a$ thì thể tích $S.ABC$ bằng
	\choice
	{\True $\dfrac{a^3\sqrt{30}}{12}$}
	{$\dfrac{a^3\sqrt{30}}{3}$}
	{$a^3\sqrt{30}$}
	{$\dfrac{a^3\sqrt{3}}{2}$}
	\loigiai{
		\immini{
			Gọi $I$ là tâm hình vuông $ABCD$; $H$ là trung điểm của $AI$. Ta có $SI\perp(ABCD)\Rightarrow MH\perp(ABCD)\Rightarrow$ hình chiếu của $MN$ lên $(ABCD)$ là $HN\Rightarrow\widehat{HNM}$ là góc giữa $MN$ và mặt phẳng $(ABCD)\Rightarrow\widehat{HNM}=60^{\circ}$.\\
			Xét $\triangle HNC$ có $NC=\dfrac{a}{2}$; $CH=\dfrac{3}{4}AC=\dfrac{3a\sqrt{2}}{4}$;
			\begin{align*}
				HN^2&=NC^2+HC^2-2NC\cdot HC\cos 45^{\circ}\cr
				&=\left(\dfrac{a}{2}\right)^2+\left(\dfrac{3a\sqrt{2}}{4}\right)^2-2\dfrac{a}{2}\cdot\dfrac{3a\sqrt{2}}{4}\dfrac{\sqrt{2}}{2}=\dfrac{5a}8^2\cr
				\Rightarrow NH&=\dfrac{a\sqrt{5}}{2\sqrt{2}}.
			\end{align*}
		}{
			\begin{tikzpicture}
				\coordinate (A) at (0,0);
				\coordinate (D) at (4,0);
				\coordinate (C) at (5,1);
				\coordinate (B) at ($(A)+(C)-(D)$);
				\coordinate (A') at ($(A)+(0,-3)$);
				\coordinate (B') at ($(B)+(0,-3)$);
				\coordinate (C') at ($(C)+(0,-3)$);
				\coordinate (D') at ($(D)+(0,-3)$);
				\coordinate (S) at ($(A')!1/2!(C')$);
				\coordinate (I) at ($(A)!1/2!(C)$);
				\coordinate (M) at ($(A)!1/2!(S)$);
				\coordinate (N) at ($(B)!1/2!(C)$);
				\coordinate (H) at ($(A)!1/2!(I)$);
				\draw(A)--(B)--(C)--(D)--(A)(A')--(D')--(C')(A)--(C)(B)--(D)(H)--(N)(A)--(A')(C)--(C')(D)--(D');
				\draw[dashed](A')--(B')--(C')--(A')(B')--(D')(S)--(I)(H)--(M)--(N)(A)--(S)(B)--(B');
				\foreach \p/\g in {A/180,B/90,C/90,D/0,A'/180,D'/-90,C'/0,B'/180, H/-45, N/-45,M/-90, S/-90,  I/-135}\draw[fill=black] (\p) circle (1pt)node[shift={(\g:.3)},scale=1]{$\p$};
			\end{tikzpicture}
		}
		Xét $\triangle HMN$ vuông ở $H$ có: $\tan\widehat{HNM}=\dfrac{MH}{HN}\Rightarrow MH=NH\cdot\tan 60^{\circ} =\dfrac{a\sqrt{5}}{2\sqrt{2}}\cdot\sqrt{3}=\dfrac{a\sqrt{15}}{2\sqrt{2}}$.\\
		Do đó, $SI=2MH=\dfrac{a\sqrt{15}}{\sqrt{2}}$.\\
		Thể tích khối chóp $S.ABC$ là $V_{S.ABC}=\dfrac{1}{3}SI\cdot S_{\triangle ABC} =\dfrac{1}{3}\cdot\dfrac{a\sqrt{15}}{\sqrt{2}}\cdot\dfrac{a^2}{2}=\dfrac{a^3\sqrt{30}}{12}$.
	}
\end{ex}
\begin{ex}%[2H1K3-2]
	(Sở Ninh Bình 2022) Cho khối hộp $ABCD.A'B'C'D'$ có $AC'=\sqrt{3}$. Biết rằng các khoảng cách từ các điểm $A',B,D$ đến đường thẳng $AC'$ là độ dài ba cạnh của một tam giác có diện tích $S=\dfrac{\sqrt{6}}{12}$, thể tích của khối hộp đã cho là
	\choice
	{$\dfrac{\sqrt{2}}{12}$}
	{1}
	{\True $\dfrac{\sqrt{2}}{2}$}
	{$\dfrac{3\sqrt{2}}{4}$}
	\loigiai{
		\immini{
			Gọi $G=AC'\cap(A'BD)$, khi đó dễ thấy $G$ là trọng tâm $\triangle A'BD$ và $\dfrac{AG}{AC'}=\dfrac{1}{3}$ nên $AG=\dfrac{\sqrt{3}}{3}$.\\
			Lấy điểm $K$ đối xứng với $B$ qua $G$ và dựng hình lăng trụ $GDK.APN$.\\
			Nhận thấy rằng khoảng cách giữa các mặt bên của lăng trụ $GDK.APN$ bằng với khoảng cách từ đỉnh $A',B,D$ đến $AG$.
		}{
			\begin{tikzpicture}
				\coordinate (G) at (0,0);
				\coordinate (K) at (4,0);
				\coordinate (D) at (3,1);
				\coordinate (A) at ($(G)+(1,3)$);
				\coordinate (P) at ($(A)-(G)+(D)$);
				\coordinate (N) at ($(A)-(G)+(K)$);
				\coordinate (M) at ($(G)!1/2!(K)$);
				\coordinate (B) at ($(M)!3!(G)$);
				\coordinate (A') at ($(D)!2!(M)$);
				\coordinate (m) at ($(N)!1/4!(K)$);
				\coordinate (m') at ($(G)-(A)+(m)$);
				\coordinate (n) at ($(P)!1/4!(D)$);
				\coordinate (n') at ($(G)-(A)+(n)$);
				\draw(A)--(P)--(N)--(A)(M)--(K)(B)--(A')(A)--(B)(A')--(M)(N)--(K)(A)--(A')(A)--(m)(G)--(m')--(K);
				\draw[dashed](G)--(D)--(K)(A)--(G)(P)--(D)(M)--(B)--(D)--(M)(A)--(n)--(m)(G)--(n')--(m')(n')--(D);
				\foreach \p/\g in {G/-90, K/0,D/0, A/180, P/90, N/0, B/180, A'/-90}\draw[fill=black] (\p) circle (1pt)node[shift={(\g:.3)},scale=1]{$\p$};
			\end{tikzpicture}
		}
		Từ đó, qua cách dựng các mặt phẳng đi qua $A,G$ và vuông góc với các cạnh bên của lăng trụ $GDK.APN$, đồng thời cắt các mặt phẳng chứa các mặt bên của lăng trụ này, ta lại thu được một lăng trụ mới (như hình vẽ) là một lăng trụ đứng có chiều cao là $AG$, tam giác đáy có kích thước lần lượt bằng độ dài khoảng cách từ các đỉnh $A',B,D$ đến đường thẳng $AC'$.\\
		Khối lăng trụ mới và lăng trụ $GDK.APN$ có cùng thể tích nên $V_m=V_{GDK\cdot APN}=S_m\cdot AG=\dfrac{\sqrt{2}}{12}$ với $S_m=\dfrac{\sqrt{6}}{12}$.\\
		Suy ra $V_{ABCD.A'B'C'D'}=6V_{A'\cdot ABD}=6V_{GDK\cdot APN}=6\cdot\dfrac{\sqrt{2}}{12}=\dfrac{\sqrt{2}}{2}$.
	}
\end{ex}
\begin{ex}%[2H1K3-2]
	(THPT Quảng Xương 1 – Thanh Hóa 2022) Cho hình chóp $S.ABCD$ có đáy là hình thoi tâm $H$, $SH\perp (ABCD)$. Hai đường chéo $AC=2a$, $BD=a\sqrt{2}$. Gọi $M,N$ lần lượt là trung điểm các cạnh $SA,SB$ và điểm $P$ thuộc cạnh $CD$. Biết rằng khoảng cách từ $A$ đến mặt phẳng $(MNP)$ bằng $a$, thể tích khối đa diện $AMNP$ bằng
	\choice
	{\True $\dfrac{a^3\sqrt{2}}{8}$}
	{$\dfrac{a^3\sqrt{3}}{4}$}
	{$\dfrac{a^3\sqrt{2}}{4}$}
	{$\dfrac{a^3\sqrt{3}}{8}$}
	\loigiai{
		\immini{
			Đầu tiên ta chuẩn hóa $a=1$. Gọi $O=CM\cap DN$ khi đó $O$ là trọng tâm $\triangle SAC$.\hfill(*)\\
			Ta nhận thấy do mặt phẳng $(MNP)$ luôn trùng với mặt phẳng $(MNCD)$ nên ta đặc biệt hóa điểm $P$ nằm tại chân đường vuông góc hạ từ $H$ xuống $CD$. Khi đó ta có: $CD\perp (OHP)$ tức ta suy ra
			\begin{align*}
				\mathrm{d}(A;(MNP))&=2\mathrm{d}(H;(MNP))=2\mathrm{d}(H;OP)=1\cr
				\Rightarrow\mathrm{d}(H;OP)&=\dfrac{1}{2}.
			\end{align*}
			Đặt $SH=x$, khi đó $OH=\dfrac{1}{3}SO=\dfrac{x}{3}$.\hfill(**)
		}{
			{\begin{tikzpicture}[scale=1]
					\def\a{4} % Dài
					\def\c{4} % Cao
					\coordinate (A) at (0,0);
					\coordinate (B) at (\a,0);
					\coordinate (D) at (-1.5,-1.5);
					\coordinate (C) at ($(B)+(D)-(A)$);
					\coordinate (H) at ($(A)!1/2!(C)$);
					\coordinate (S) at ($(H)+(0,\c)$);
					\coordinate (M) at ($(S)!1/2!(A)$);
					\coordinate (N) at ($(S)!1/2!(B)$);
					\coordinate (P) at ($(C)!2/3!(D)$);
					\coordinate (O) at ($(S)!2/3!(H)$);
					\draw(S)--(D)--(C)--(B)--(S)--(C);
					\draw[dashed](S)--(A)--(B)(C)--(A)--(D)(S)--(H)(A)--(C)(B)--(D)(M)--(N)--(P)--(M)--(C)(D)--(N)(H)--(P)--(O);
					\foreach \p/\g in {S/90,A/-90,B/0,D/180,C/-90, H/-90, M/180, N/0, P/-90, O/180}\draw[fill=white] (\p) circle (1pt)node[shift={(\g:.2)},scale=.7]{$\p$};
			\end{tikzpicture}}
		}
		Do $HP\perp CD$ nên xét trong tam giác vuông $CHD$ ta có $HP=\dfrac{HC\cdot HD}{\sqrt{HC^2+HD^2}}=\dfrac{1}{\sqrt{3}}$.\\
		Suy ra $\mathrm{d}(H;OP)=\dfrac{OH\cdot HP}{\sqrt{OH^2+HP^2}}=\dfrac{x}{\sqrt{3\left(x^2+3\right)}}=\dfrac{1}{2}\Rightarrow x=SH=3$.\\
		Ta giả sử tiếp $P$ di động đến trùng điểm $D$, khi đó ta có
		$$V_{AMNP}=V_{AMND}=V_{SMND}=\dfrac{1}{4}V_{SABD}=\dfrac{1}{8}V_{S.ABCD}.$$
		Vậy $V_{AMNP}=\dfrac{1}{8}V_{S.ABCD}=\dfrac{1}{24}SH\cdot S_{ABCD}=\dfrac{1}{24}\cdot 3\cdot\dfrac{1}{2}\cdot AC\cdot BD=\dfrac{a^3\sqrt{2}}{8}$.}
\end{ex}
\begin{ex}%[2H1K3-2]
	(Cụm trường Nam Định 2022) Cho hình chóp $S.ABC$ có đáy $ABC$ là tam giác đều, hình chiếu vuông góc của đỉnh $S$ trên mặt đáy là trung điểm $H$ của cạnh $AB$. Biết $SH=\dfrac{a\sqrt{3}}{2}$ và mặt phẳng $(SAC)$ vuông góc với mặt phẳng $(SBC)$. Thể tích khối chóp $S.ABC$ bằng 
	\choice
	{\True $\dfrac{a^3}{4}$}
	{$\dfrac{a^3}{16}$}
	{$\dfrac{a^3}{2}$}
	{$\dfrac{3a^3}{8}$}
	\loigiai{
		\immini{
			Từ $H$ kẻ $HK\perp SC$. Ta có $\heva{&AB\perp HC\\&AB\perp SH}\Rightarrow AB\perp(SHC)$\\
			$\Rightarrow AB\perp SC$ mà $HK\perp SC\Rightarrow SC\perp(AKB)$.\\
			Suy ra $\heva{&SC\perp AK\\&SC\perp BK}\Rightarrow$ góc giữa $(SAC)$ và $(SBC)$ là $\widehat{AKB}$.\\
			Suy ra $AK\perp BK$  và $AK\perp(SBC), BK\perp(SAC)$.\\
			Gọi $AB=x$. Xét $\triangle AKB$ vuông tại $K, KH$ là trung tuyến $HK=\dfrac{1}{2}AB=\dfrac{x}{2}$.\\
			Mà $\triangle ABC$ đều $\Rightarrow HC=\dfrac{x\sqrt{3}}{2}$.
		}{
			\begin{tikzpicture}
				\coordinate (A) at (0,0);
				\coordinate (B) at (4,0);
				\coordinate (C) at (1,-1);
				\coordinate (H) at ($(A)!1/2!(B)$);
				\coordinate (S) at ($(H)+(0,3)$);
				\coordinate (K) at ($(S)!1/2!(C)$);
				\draw(B)--(K)--(A)--(S)--(C)--(A)(C)--(B)--(S);
				\draw[dashed](C)--(H)--(S)(A)--(B)(H)--(K);
				\foreach \p/\g in {S/90,C/-90,A/180,B/0,H/-45,K/135}\draw[fill=black] (\p) circle (1pt)node[shift={(\g:.2)},scale=1]{$\p$};
				\foreach \i/\j/\k/\t in {K/H/C/6}{
					\def\dgiua{\i}\def\dmot{\j}\def\dhai{\k}\def\tyso{\t pt}
					\draw ($(\dgiua)!\tyso!(\dmot)$)--($(\dgiua)!2!($($(\dgiua)!\tyso!(\dmot)$)!.5!($(\dgiua)!\tyso!(\dhai)$)$)$)--($(\dgiua)!\tyso!(\dhai)$);}
			\end{tikzpicture}
		}
		Xét $\triangle SHC$ vuông tại $H$, có $HK$ là đường cao $\dfrac{1}{HK^2}=\dfrac{1}{SH^2}+\dfrac{1}{HC^2}$ \\
		$ \Leftrightarrow\left(\dfrac{2}{x}\right)^2=\left(\dfrac{2}{a\sqrt{3}}\right)^2+\left(\dfrac{2}{x\sqrt{3}}\right)^2\Leftrightarrow\dfrac{4}{a^2}=\dfrac{4}{3a^2}+\dfrac{4}{3x^2}\Leftrightarrow 3a^2=x^2+a^2\Leftrightarrow x=a\sqrt{2} $.\\
		Thể tích khối chóp: 
		$$V_{S.ABC}=\dfrac{1}{3}\cdot SH\cdot S_{\triangle ABC}=\dfrac{1}{3}\cdot SH\dfrac{1}{2}\cdot AB\cdot CH=\dfrac{1}{3}\cdot\dfrac{a\sqrt{3}}{2}\dfrac{1}{2}\cdot a\sqrt{2}\cdot\dfrac{a\sqrt{2}\cdot\sqrt{3}}{2}=\dfrac{a^3}{4}.$$
	}
\end{ex}
\begin{ex}%[2H1K3-2]
	(Liên trường Quảng Nam 2022) Cho hình chóp $S.ABCD$ có đáy $ABCD$ là hình vuông cạnh $a$, $SA$ vuông góc với mặt phẳng đáy $ABCD$, góc giữa hai mặt phẳng $(SBD)$ và $(ABCD)$ bằng $60^{\circ}$. Gọi $M, N$ lần lượt là trung điểm của cạnh $SB, SC$. Tính thể tích khối chóp $S.ADNM$. 
	\choice
	{$V=\dfrac{a^3\sqrt{6}}{24}$}
	{$V=\dfrac{3a^3\sqrt{6}}{16}$}
	{$V=\dfrac{a^3\sqrt{6}}{8}$}
	{\True $V=\dfrac{a^3\sqrt{6}}{16}$}
	\loigiai{
		\immini
		{
			Ta có $\heva{&BD\perp SA\\&BD\perp AC}\Rightarrow BD\perp(SAC)\Rightarrow BD\perp SO$.\\
			Lại có $\heva{&OA\subset(ABCD), OA\perp BD\\&SO\subset(SBD), SO\perp BD\\&(SBD)\cap(ABCD)=BD}$  nên
			$$\widehat{\left((SBD),(ABCD)\right)}=\widehat{SOA}=60^{\circ}.$$
			$OA=\dfrac{AC}{2}=\dfrac{a\sqrt{2}}{2}$.\\
			Tam giác $SOA$ vuông tại $A$ nên 
			$$SA=OA\cdot\tan\widehat{SOA}=\dfrac{a\sqrt{2}}{2}\cdot\tan 60^{\circ}=\dfrac{a\sqrt{6}}{2}.$$
		}
		{
			\begin{tikzpicture}[line join=round,line cap=round,line width=.6pt,font=\footnotesize,scale=0.5]
				\coordinate[label=below left:$B$] (B) at (0,0);
				\coordinate[label=left:$A$] (A) at (3,2);
				\coordinate[label=below right:$C$] (C) at (7,0);
				\coordinate[label=above right:$D$] (D) at ($(C)-(B)+(A)$);
				\coordinate[label=above left:$S$] (S) at ($(A)+(90:7)$);
				\coordinate (O) at ($(A)!1/2!(C)$);
				\coordinate (M) at ($(S)!1/2!(B)$);
				\coordinate (N) at ($(S)!1/2!(C)$);
				\draw (B)--(C)--(D)--(S)--cycle (S)--(C)(D)--(N)--(M);
				\draw[dashed] (A)--(D) (S)--(A)--(B)--(D)(S)--(O)(A)--(C)(B)--(D)(N)--(A)--(M);
				\fill (A)circle(1.5pt) (B)circle(1.5pt) (C)circle(1.5pt) (D)circle(1.5pt) (S)circle(1.5pt) (M)circle(1.5pt);	
				\foreach \p/\g in {O/-90, M/180, N/45}\draw[fill=black] (\p) circle (1pt)node[shift={(\g:.3)},scale=1]{$\p$};
			\end{tikzpicture}
		}
		Thể tích khối chóp $S.ABCD$ là $V=\dfrac{1}{3}SA\cdot S_{ABCD}=\dfrac{1}{3}\cdot\dfrac{a\sqrt{6}}{2}\cdot a^2=\dfrac{a^3\sqrt{6}}{6}$.\\
		Ta có $\dfrac{V_{S.AMN}}{V_{S.ABC}}=\dfrac{SM}{SB}\cdot\dfrac{SN}{SC}=\dfrac{1}{2}\cdot\dfrac{1}{2}=\dfrac{1}{4}\Rightarrow V_{S.AMN}=\dfrac{1}{4}V_{S.ABC}$.\\
		$\dfrac{V_{S.ADN}}{V_{S.ADC}}=\dfrac{SN}{SC}=\dfrac{1}{2}\Rightarrow V_{S.ADN}=\dfrac{1}{2}V_{S.ADC}$.\\
		$V_{S.ABC}=V_{S.ADC}=\dfrac{1}{2}V_{S.ABCD}$. Suy ra.\\
		$V_{S.ADNM}=V_{S.AMN}+V_{S.ADN}=\dfrac{1}{4}V_{S.ABC}+\dfrac{1}{2}V_{S.ADC} =\dfrac{1}{4}\cdot\dfrac{1}{2}V+\dfrac{1}{2}\cdot\dfrac{1}{2}V=\dfrac{3}{8}V =\dfrac{3}{8}\cdot\dfrac{a^3\sqrt{6}}{6}=\dfrac{a^3\sqrt{6}}{16}$.
	}
\end{ex}
\begin{ex}%[2H1K3-2]
	(Sở Hưng Yên 2022) Cho hình chóp $S.ABCD$ có đáy $ABCD$ là hình chữ nhật, $AB=1, AD=\sqrt{10}, SA=SB, SC=SD$. Biết mặt phẳng $(SAB)$ và $(SCD)$ vuông góc với nhau đồng thời tổng diện tích của hai tam giác $\triangle SAB$ và $\triangle SCD$ bằng 2. Thể tích khối chóp $S.ABCD$ bằng
	\choice
	{$2$}
	{$\dfrac{3}{2}$}
	{\True $1$}
	{$\dfrac{1}{2}$}
	\loigiai{
		\immini{
			Vì $\heva{&S\in(SAB)\cap(SCD)\\&AB\subset(SAB)\\&CD\subset(SCD)\\&AB\parallel CD}$ nên giao tuyến của hai mặt phẳng $(SAB)$ và $(SCD)$ là đường thẳng $d$ đi qua $S$ và song song với $AB, CD$.\\
			Gọi $M, N$ lần lượt là trung điểm của $AB, CD$.\\
			Vì $SA=SB, SC=SD$ nên $SM\perp AB, SN\perp CD\Rightarrow SM\perp d, SN\perp d\Rightarrow d\perp(SMN)$.\\
			Mà mặt phẳng $(SAB)$ và $(SCD)$ vuông góc với nhau nên $SM\perp SN$.\\
			Kẻ $SH\perp MN$.\hfill(1)\\
			Vì $d\perp(SMN)\Rightarrow d\perp SH\Rightarrow SH\perp AB$.\hfill(2)\\
		}{
			\begin{tikzpicture}[line join=round,line cap=round,line width=.6pt,font=\footnotesize,scale=0.5]
				\coordinate[label=below left:$B$] (B) at (0,0);
				\coordinate[label=left:$A$] (A) at (3,2);
				\coordinate[label=below right:$C$] (C) at (7,0);
				\coordinate[label=above right:$D$] (D) at ($(C)-(B)+(A)$);
				\coordinate[label=above left:$S$] (S) at ($(A)+(90:7)$);
				\coordinate (O) at ($(A)!1/2!(C)$);
				\coordinate (M) at ($(A)!1/2!(B)$);
				\coordinate (N) at ($(D)!1/2!(C)$);
				\coordinate (s) at ($(S)+(B)-(A)$);
				\draw (B)--(C)--(D)--(S)--cycle (S)--(C)(S)--(s) node[above]{$d$}(S)--(N);
				\draw[dashed] (A)--(D) (S)--(A)--(B)--(D)(S)--(O)(A)--(C)(B)--(D)(S)--(M)--(N)--(A)--(M);
				\fill (A)circle(1.5pt) (B)circle(1.5pt) (C)circle(1.5pt) (D)circle(1.5pt) (S)circle(1.5pt) (M)circle(1.5pt);	
				\foreach \p/\g in {O/-90, M/135, N/0}\draw[fill=black] (\p) circle (1pt)node[shift={(\g:.3)},scale=1]{$\p$};
			\end{tikzpicture}
		}
		Từ (1), (2) suy ra $SH\perp(ABCD)\Rightarrow V_{S.ABCD}=\dfrac{1}{3}\cdot SH\cdot S_{ABCD}=\dfrac{1}{3}\cdot SH\cdot AB\cdot AD$.\\
		Đặt $SM=x, SN=y\Rightarrow SH=\dfrac{xy}{\sqrt{x^2+y^2}}$.\\
		Ta có $SM^2+SN^2=MN^2\Leftrightarrow x^2+y^2=10$.\\
		Mặt khác $S_{SAB}+S_{SCD}=2\Leftrightarrow\dfrac{1}{2}\cdot x\cdot 1+\dfrac{1}{2}\cdot y\cdot 1=2\Leftrightarrow x+y=4$.\\
		Suy ra $xy=\dfrac{(x+y)^2-\left(x^2+y^2\right)}{2}=3\Rightarrow SH=\dfrac{xy}{\sqrt{x^2+y^2}}=\dfrac{3}{\sqrt{10}}\Rightarrow V_{S.ABCD}=1$.\\
		Vậy thể tích khối chóp $S.ABCD$ bằng 1.
	}
\end{ex}
\begin{ex}%[2H1K3-2]
	(Sở Vĩnh Phúc 2022) Cho hình lăng trụ $ABC.A'B'C'$ có tam giác đáy $ABC$ vuông đỉnh $A$, $AB=a,AC=\sqrt{3}a$, $A'A=A'B=A'C$ và mặt phẳng $(ABB'A')$ tạo với mặt đáy $(ABC)$ một góc $60^{\circ}$. Tính thể tích $V$ của lăng trụ đã cho. 
	\choice
	{$V=\dfrac{3\sqrt{3}a^3}{4}$}
	{\True $V=\dfrac{\sqrt{3}a^3}{4}$}
	{$V=\dfrac{3a^3}{4}$}
	{$V=\dfrac{3\sqrt{3}a^3}{2}$}
	\loigiai{
		\immini{
			Gọi $H$ là trung điểm của $BC$.\\
			Xét ba tam giác $A'HB, A'HA, A'HC$ có $A'H$ chung, $A'A=A'B=A'C$ và $HA=HB=HC$ (vì $AH$ là đường trung tuyến của tam giác vuông $ABC$)\\
			$ \Rightarrow\triangle A'HA=\triangle A'HB=\triangle A'HC $ mà $\triangle A'HB$ vuông tại $H\Rightarrow\widehat{A'HA}=\widehat{A'HB}=\widehat{A'HC}=90^{\circ}$ \\
			$ \Rightarrow A'H\perp(ABC) $.\\
			Tam giác $A'AB$ cân tại $A'$ có $I$ là trung điểm của $AB$ nên $A'I\perp AB$.\\
			Ta có $\heva{&A'I\perp AB\\&A'H\perp AB\left(do A'H\perp(ABC)\right)}$\\
			$\Rightarrow AB\perp(A'HI)\Rightarrow HI\perp AB$.\\
			Do đó, $\widehat{\left((ABB'A'),(ABC)\right)}=\widehat{A'IH}=60^{\circ}$.\\
			Tam giác $ABC$ có $H, I$ lần lượt là trung điểm của $BC, AB$ nên $HI=\dfrac{1}{2}AC=\dfrac{a\sqrt{3}}{2}$.
		}{
			\begin{tikzpicture}
				\coordinate (B) at (0,0);
				\coordinate (C) at (4,0);
				\coordinate (A) at (1,-1);
				\coordinate (H) at ($(B)!1/2!(C)$);
				\coordinate (A') at ($(H)+(0,3)$);
				\coordinate (B') at ($(B)+(A')-(A)$);
				\coordinate (C') at ($(C)+(A')-(A)$);
				\coordinate (I) at ($(A)!1/2!(B)$);
				\draw(B)--(A)--(C)--(A')--(C')--(B')--(A')(C')--(C)(B)--(A')--(I)(B)--(B')(A)--(A');
				\draw[dashed](I)--(H)--(A')(B)--(C);
				\foreach \p/\g in {B/180,C/0,A/-90,B'/180,C'/0,A'/90, H/-90, I/180}\draw[fill=black] (\p) circle (1pt)node[shift={(\g:.3)},scale=1]{$\p$};
			\end{tikzpicture}
		}
		Tam giác $A'HI$ vuông tại $H$ có
		$$\tan\widehat{A'IH}=\dfrac{A'H}{IH}\Rightarrow\tan 60^{\circ}=\dfrac{A'H}{\dfrac{a\sqrt{3}}{2}}\Leftrightarrow A'H=\dfrac{a\sqrt{3}}{2}\cdot\sqrt{3}=\dfrac{3a}{2}.$$
		Thể tích lăng trụ là $V=\dfrac{1}{3}\cdot A'H\cdot S_{ABC}=\dfrac{1}{6}\cdot A'H\cdot AB\cdot AC=\dfrac{1}{6}\cdot\dfrac{3a}{2}\cdot a\cdot a\sqrt{3}=\dfrac{\sqrt{3}a^3}{4}$.
	}
\end{ex}
\begin{ex}%[2H1K3-2]
	(THPT Trần Nhân Tông – Quảng Ninh 2022) Cho hình chóp $S.ABCD$ có $ABCD$ là hình bình hành biết rằng $\widehat{SAD}=\widehat{BAC}=90^{\circ}$, cạnh $SA=2\sqrt{2}a,BC=2a,SB=\sqrt{6}a$. Gọi $H$ là hình chiếu vuông góc của $A$ trên $SD$ biết khoảng cách giữa $CH$ và $SB$ bằng $\sqrt{2}a$. Thể tích khối chóp $S.ABCD$ bằng
	\choice
	{$\sqrt{2}a^3$}
	{$\dfrac{\sqrt{2}a^3}{3}$}
	{$\sqrt{10}a^3$}
	{\True $\dfrac{\sqrt{10}}{3}a^3$}
	\loigiai{
		\immini{Ta gọi $I$ là trung điểm $AD$ và $G$ là trọng tâm $\triangle ACD$, kẻ, khi đó ta có $\dfrac{DG}{DB}=\dfrac{DH}{DS}=\dfrac{1}{3}$.\\
			Kẻ $BK\parallel IC$ với $K\in AD$ ta suy ra và $\mathrm{d}(CH,SB)=\mathrm{d}((HCI);(SBK))=\mathrm{d}(I;(SBK))$. Tiếp đến ta dễ thấy $IKBC$ là hình bình hành nên suy ra $A$ là trung điểm $IK$, từ đó ta có $SK=3a$ và $CI=BK=\dfrac{AD}{2}=a$. Mà $SB=\sqrt{6}a$ nên theo Công thức He-ron ta suy ra $S_{\triangle SBK}=\dfrac{\sqrt{5}}{2}a^2$.
		}{
			\begin{tikzpicture}
				\coordinate (A) at (0,0);
				\coordinate (B) at (-1,-1);
				\coordinate (C) at (4,-1);
				\coordinate (D) at ($(A)+(C)-(B)$);
				\coordinate (O) at ($(A)!1/2!(C)$);
				\coordinate (I) at ($(A)!1/2!(D)$);
				\coordinate (K) at ($(I)!2!(A)$);
				\coordinate (G) at ($(C)!2/3!(I)$);
				\coordinate (S) at ($(A)+(1,3)$);
				\coordinate (H) at ($(D)!1/3!(S)$);
				\draw(S)--(K)--(B)--(C)--(D)--(S)--(C)--(H);
				\draw[dashed](K)--(D)--(B)--(A)--(S)(H)--(A)--(C)--(I)(H)--(G);
				\foreach \p/\g in {S/90,K/180,A/135,I/90,D/0,H/0,G/-90,O/-90,B/-90, C/-90}\draw[fill=black] (\p) circle (1pt)node[shift={(\g:.3)},scale=1]{$\p$};	
			\end{tikzpicture}
		}
		Mặt khác ta có
		$$\mathrm{d}(CH,SB)=\mathrm{d}(I;(SBK))=2\mathrm{d}(A;(SBK))=\dfrac{6V_{SABK}}{S_{\triangle SBK}}=\dfrac{6V_{SABK}}{\dfrac{\sqrt{5}}{2}}a^2=\sqrt{2}a$$
		nên ta suy ra $V_{S.ABK}=\dfrac{1}{6}\dfrac{\sqrt{5}}{2}a^2\cdot\sqrt{2}a=\dfrac{\sqrt{10}}{12}a^3$, mà $S_{\triangle ABK}=S_{\triangle ABI}=\dfrac{S_{ABCD}}{4}\Rightarrow V_{SABK}=\dfrac{V_{S.ABCD}}{4}$ nên ta suy ra $V_{S.ABCD}=4V_{S.ABK}=4\dfrac{\sqrt{10}}{12}a^3=\dfrac{\sqrt{10}}{3}a^3$.}
\end{ex}
\begin{ex}%[2H1K3-2]
	(Sở Nghệ An 2022) Cho hình chóp có đáy $ABCD$ là hình chữ nhật, biết $AB=4a,AD=2a$ và $SA$ vuông góc với đáy. Gọi $M$ là trung điểm của cạnh $SC$, khoảng cách từ $M$ đến mặt phẳng $(SBD)$ bằng $\dfrac{a}{4}$. Thể tích khối chóp $S.ABM$ là
	\choice
	{$\dfrac{16a^3\sqrt{59}}{177}$}
	{\True $\dfrac{8a^3\sqrt{59}}{177}$}
	{$\dfrac{12a^3\sqrt{59}}{177}$}
	{$\dfrac{24a^3\sqrt{59}}{177}$}
	\loigiai{
		\immini{Trong $(ABCD)$, gọi $O$ là giao điểm của $AC$ và $BD$.\\
			Ta có
			\begin{align*}
				\mathrm{d}(M,(SBD))&=\dfrac{MS}{CS}\cdot\mathrm{d}(C,(SBD))\cr
				&=\dfrac{1}{2}\cdot\left[\dfrac{CO}{AO}\cdot\mathrm{d}(A,(SBD))\right]=\dfrac{1}{2}\mathrm{d}(A,(SBD)).
			\end{align*}
			Trong $(ABCD)$, dựng $AH\perp BD$. Trong $(SAH)$, dựng $AK\perp SH$.\\
			Ta có $\heva{&BD\perp AH\\&BD\perp SA}\Rightarrow BD\perp (SAH)\Rightarrow BD\perp AK$. Lại có $\heva{&AK\perp BD\\&AK\perp SH}\Rightarrow AK\perp (SBD)$.}
		{
			\begin{tikzpicture}
				\coordinate (A) at (0,0);
				\coordinate (D) at (4,0);
				\coordinate (B) at (-2,-1);
				\coordinate (C) at ($(B)+(D)-(A)$);
				\coordinate (O) at ($(A)!1/2!(C)$);
				\coordinate (H) at ($(B)!4/5!(O)$);
				\coordinate (S) at ($(A)+(0,5)$);
				\coordinate (K) at ($(S)!1/2!(H)$);
				\coordinate (M) at ($(S)!1/2!(C)$);
				\draw(S)--(B)--(C)--(D)--(S)--(C)(B)--(M);
				\draw[dashed](S)--(A)--(B)--(D)--(A)--(C)(A)--(H)(A)--(K)(A)--(M)(S)--(H);
				\foreach \i/\j/\k/\t in {H/A/B/6,K/A/S/6}{
					\def\dgiua{\i}\def\dmot{\j}\def\dhai{\k}\def\tyso{\t pt}
					\draw ($(\dgiua)!\tyso!(\dmot)$)--($(\dgiua)!2!($($(\dgiua)!\tyso!(\dmot)$)!.5!($(\dgiua)!\tyso!(\dhai)$)$)$)--($(\dgiua)!\tyso!(\dhai)$);}
				\foreach \p/\g in {S/90, A/180, K/0, M/0, D/0, O/90, H/-90, B/-90, C/-90}\draw[fill=black] (\p) circle (1pt)node[shift={(\g:.3)},scale=1]{$\p$};
			\end{tikzpicture}
		}
	
		Do đó $\mathrm{d}(M,(SBD))=\dfrac{1}{2}AK$. Theo giả thiết: $\mathrm{d}(M,(SBD))=\dfrac{a}{4}\Rightarrow AK=\dfrac{a}{2}$.\\
		Xét tam giác $ABD$ vuông tại $A$ có: $\dfrac{1}{AH^2}=\dfrac{1}{AB^2}+\dfrac{1}{AD^2}\Rightarrow AH=\dfrac{4a\sqrt{5}}{5}$.\\
		Xét tam giác $SAH$ vuông tại $A$ có: $\dfrac{1}{AK^2}=\dfrac{1}{SA^2}+\dfrac{1}{AH^2}\Rightarrow SA=\dfrac{4a\sqrt{59}}{59}$.\\
		Ta có: $\dfrac{V_{S.ABM}}{V_{S.ABC}}=\dfrac{SM}{SC}=\dfrac{1}{2}\Rightarrow V_{S.ABM}=\dfrac{1}{2}\cdot V_{S.ABC}$. Lại có: $V_{S.ABC}=\dfrac{1}{3}SA\cdot S_{ABC}=\dfrac{16a^3\sqrt{59}}{177}$.\\
		Do đó: $V_{S.ABM}=\dfrac{8a^3\sqrt{59}}{177}$. Vậy thể tích của khối chóp $S.ABM$ là $\dfrac{8a^3\sqrt{59}}{177}$.
	}
\end{ex}
\begin{ex}%[2H1K3-2]
	(Sở Nam Định 2022) Cho khối chóp $S.ABCD$ có $SA\perp(ABCD)$. Đáy $ABCD$ là hình chữ nhật $AB=a\sqrt{3}$, $AD=a$. Biết góc giữa hai mặt phẳng $(SAB)$ và $(SBD)$ bằng $45^{\circ}$, hãy tính theo $a$ thể tích $V$ của khối chóp $S.ABCD$. 
	\choice
	{$V=\dfrac{\sqrt{6}}{6}a^3$}
	{\True $V=\dfrac{\sqrt{2}}{2}a^3$}
	{$V=\dfrac{\sqrt{2}}{6}a^3$}
	{$V=3\sqrt{2}a^3$}
	\loigiai{
		\immini{
			Đặt $a=1$. Trong không gian $Oxyz$, chọn hệ trục tọa độ như hình vẽ bên, ta có $A\equiv O(0; 0; 0)$, $B\left(0;\sqrt{3}; 0\right)$, $D(1; 0; 0)$, $C\left(1;\sqrt{3}; 0\right)$, $S(0; 0; b)$ với $b>0$.\\
			Ta có $\overrightarrow{AS}=(0; 0; b)$; $\overrightarrow{AB}=\left(0;\sqrt{3}; 0\right)$; $\overrightarrow{DS}=(-1; 0; b)$; $\overrightarrow{DB}=\left(-1;\sqrt{3}; 0\right)$.\\
			Suy ra VTPT của $(SAB)$ là $$\overrightarrow{n_1}=\left[\overrightarrow{AS},\overrightarrow{AB}\right]=\left(-b\sqrt{3}; 0; 0\right).$$
			Suy ra VTPT của $(SBD)$ là $$\overrightarrow{n_2}=\left[\overrightarrow{DS},\overrightarrow{DB}\right]=\left(-b\sqrt{3};-b;-\sqrt{3}\right).$$
			Do góc giữa hai mặt phẳng $(SAB)$ và $(SBD)$ bằng $45^{\circ}$ nên ta có\\
		}
		{
			\begin{tikzpicture}
				\coordinate (A) at (0,0);	
				\coordinate (B) at (3,0);
				\coordinate (y) at (5,0);
				\coordinate (D) at (-1,-2);
				\coordinate (C) at ($(D)+(B)-(A)$);
				\coordinate (x) at ($(A)!3/2!(D)$);
				\coordinate (S) at (0,3);
				\coordinate (z) at ($(A)!3/2!(S)$);
				\draw[dashed](S)--(A)--(D)--(B)--(A);
				\draw(S)--(B)--(C)--(D)--(S);
				\draw[-stealth](B)--(y);
				\draw[-stealth](D)--(x);
				\draw[-stealth](S)--(z);
				\foreach \p/\g in {A/180,B/-90, y/-90, D/180, x/180,C/-90, S/0, z/0}\draw[fill=black] (\p) circle (.1pt)node[shift={(\g:.3)},scale=1]{$\p$};
				\foreach \i/\j/\k/\t in {A/D/S/6,A/D/B/6, A/B/S/6}{
					\def\dgiua{\i}\def\dmot{\j}\def\dhai{\k}\def\tyso{\t pt}
					\draw ($(\dgiua)!\tyso!(\dmot)$)--($(\dgiua)!2!($($(\dgiua)!\tyso!(\dmot)$)!.5!($(\dgiua)!\tyso!(\dhai)$)$)$)--($(\dgiua)!\tyso!(\dhai)$);}
			\end{tikzpicture}
		}
		\begin{align*}
			\cos 45^{\circ}=\dfrac{\left|\overrightarrow{n_1}\cdot\overrightarrow{n_2}\right|}{\left|\overrightarrow{n_1}\right|\cdot\left|\overrightarrow{n_2}\right|}
			&\Leftrightarrow\dfrac{\sqrt{2}}{2}=\dfrac{3b^2}{\sqrt{3b^2}\cdot\sqrt{4b^2+3}}\Leftrightarrow\dfrac{\sqrt{2}}{2}=\dfrac{\sqrt{3}b}{\sqrt{4b^2+3}}\cr
			&\Leftrightarrow\sqrt{8b^2+6}=2\sqrt{3}b\Leftrightarrow 8b^2+6=12b^2\cr
			&\Leftrightarrow 6=4b^2\Leftrightarrow b^2=\dfrac{3}{2}\Rightarrow b=\dfrac{\sqrt{6}}{2}.
		\end{align*}
		Suy ra $S\left(0; 0;\dfrac{\sqrt{6}}{2}\right)$.\\
		Ta có $\overrightarrow{AS}=\left(0; 0;\dfrac{\sqrt{6}}{2}\right)\Rightarrow SA=\dfrac{\sqrt{6}}{2}=\dfrac{\sqrt{6}}{2}a$; $S_{ABCD}=AB\cdot AD=a^2\sqrt{3}$.\\
		Vậy thể tích khối chóp $S.ABCD$ là $V=\dfrac{1}{3}\cdot SA\cdot S_{ABCD}=\dfrac{1}{3}\cdot\dfrac{a\sqrt{6}}{2}\cdot a^2\sqrt{3}=\dfrac{a^3\sqrt{2}}{2}$.
	}
\end{ex}