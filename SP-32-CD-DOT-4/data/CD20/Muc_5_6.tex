\Opensolutionfile{ans}[ans/CD20/Muc_5_6]
\setcounter{ex}{0}
\setcounter{dang}{0}
\section{Mức độ 5,6 điểm}
\begin{dang}
	{Bất phương trình logarit}
 \begin{itemize}
 	\item Nếu $a>1$ thì $\log_a f(x)>\log_a g(x) \Leftrightarrow f(x)>g(x)$ )(cùng chiều).
 \item Nếu $0<a<1$ thì $\log_a f(x)>\log_a g(x) \Leftrightarrow f(x)<g(x)$ )(ngược chiều).
 \item Nếu $a$ chứa ẩn thì $\hoac{&\log_a B>0 \Leftrightarrow (a-1)(B-1)>0\\& \dfrac{\log_a A}{\log_a B}>0 \Leftrightarrow (A-1)(B-1)>0}$.
 \end{itemize}
\end{dang}
%%==========Câu 1
\begin{ex}%[2D2B6-1]
	[Đề Tham Khảo 2020 Lần 2]%Câu 1.
	Tập nghiệm của bất phương trình $\log x\geq 1$ là
	\choice
	{$(10;+\infty)$}
	{$(0;+\infty)$}
	{\True $[10;+\infty)$}
	{$(-\infty;10)$}
	\loigiai{
		$\log x\geq 1\Leftrightarrow\heva{&x>0\\&x\geq 10}\Leftrightarrow x\geq 10$.\\
		Vậy bất phương trình đã cho có tập nghiệm là $[10;+\infty)$.
	}
\end{ex}
%%==========Câu 2
\begin{ex}%[2D2B6-1]
	[Mã 102 - 2020 Lần 2]%Câu 2.
	Tập nghiệm của bất phương trình $\log_3\left(13-x^2\right)\geq 2$ là
	\choice
	{$(-\infty;-2]\cup[2\colon +\infty)$}
	{$(-\infty;2]$}
	{$(0;2]$}
	{\True $[-2;2]$}
	\loigiai{
		Bất phương trình $\log_3\left(13-x^2\right)\geq 2\Leftrightarrow\heva{&13-x^2>0\\&13-x^2\geq 9}\Leftrightarrow\heva{&x^2<13\\&x^2\leq 4}$ \\
		$ \Leftrightarrow\heva{&-\sqrt{13}<x<\sqrt{13}\\&-2\leq x\leq 2}\Leftrightarrow-2\leq x\leq 2 $.\\
		Vậy, tập nghiệm của bất phương trình $\log_3\left(13-x^2\right)\geq 2$ là $[-2;2]$.
	}
\end{ex}
%%==========Câu 3
\begin{ex}%[2D2B6-1]
	[Mã 103 - 2020 Lần 2]%Câu 3.
	Tập nghiệm của bất phương trình $\log_3\left(36-x^2\right)\geq 3$ là
	\choice
	{$(-\infty;-3]\cup[3;+\infty)$}
	{$(-\infty;3]$}
	{\True $[-3;3]$}
	{$(0;3]$}
	\loigiai{
		Ta có: $\log_3\left(36-x^2\right)\geq 3\Leftrightarrow 36-x^2\geq 27\Leftrightarrow 9-x^2\geq 0\Leftrightarrow-3\leq x\leq 3$.
	}
\end{ex}
%%==========Câu 4
\begin{ex}%[2D2B6-1][Mã 101 - 2020 Lần 2]%Câu 4.
	Tập nghiệm của bất phương trình $\log_3\left(18-x^2\right)\geq 2$ là
	\choice
	{$(-\infty; 3]$}
	{$(0; 3]$}
	{\True $[-3; 3]$}
	{$(-\infty;-3]\cup[3;+\infty)$}
	\loigiai{
		Điều kiện: $18-x^2>0\Leftrightarrow x\in\left(-3\sqrt{2}; 3\sqrt{2}\right)$ (*).\\
		Khi đó ta có: $\log_3\left(18-x^2\right)\geq 2\Leftrightarrow 18-x^2\geq 9\Leftrightarrow-3\leq x\leq 3$.\\
		Kết hợp với điều kiện (*) ta được tập ngiệm của bất phương trình đã cho là $[-3; 3]$.
	}
\end{ex}
%%==========Câu 5
\begin{ex}%[2D2B6-1]
	[Mã 104 - 2020 Lần 2]%Câu 5.
	Tập nghiệm của bất phương trình $\log_3\left(31-x^2\right)\geq 3$ là
	\choice
	{$(-\infty;2]$}
	{\True $[-2;2]$}
	{$(-\infty;-2]\cup[2;+\infty)$}
	{$(0;2]$}
	\loigiai{
		$\log_3\left(31-x^2\right)\geq 3\Leftrightarrow 31-x^2\geq 27\Leftrightarrow x^2-4\leq 0\Leftrightarrow x\in[-2;2]$.
	}
\end{ex}
%%==========Câu 6
\begin{ex}%[2D2B6-1]
	[Đề Minh Họa 2017]%Câu 6.
	Giải bất phương trình $\log_2(3x-1)>3$.
	\choice
	{\True $x>3$}
	{$\dfrac{1}{3}<x<3$}
	{$x<3$}
	{$x>\dfrac{10}{3}$}
	\loigiai{
		Điều kiện xác định: $3x-1>0\Leftrightarrow x>\dfrac{1}{3}$.\\
		Bất phương trình $\Leftrightarrow 3x-1>2^3\Leftrightarrow 3x>9\Leftrightarrow x>3$ (thỏa mãn điều kiện).\\
		Vậy bpt có nghiệm $x>3$.
	}
\end{ex}
%%==========Câu 7
\begin{ex}%[2D2B6-1]
	[THPT Bạch Đằng Quảng Ninh 2019]%Câu 7.
	Tìm tập nghiệm $S$ của bất phương trình $\ln x^2<0$.
	\choice
	{$S=(-1;1)$}
	{$S=(-1;0)$}
	{\True $S=(-1;1)\setminus\{0\}$}
	{$S=(0;1)$}
	\loigiai{
		Ta có $\ln x^2<0\Leftrightarrow 0<x^2<1\Leftrightarrow\heva{&x\neq 0\\&-1<x<1}$.\\
		Vậy $S=(-1;1)\setminus\{0\}$.
	}
\end{ex}
%%==========Câu 8
\begin{ex}%[2D2B6-1]
	[THPT Minh Khai Hà Tĩnh 2019]%Câu 8.
	Tìm tập nghiệm $S$ của bất phương trình $\log_{\frac{1}{2}}(x+1)<\log_{\frac{1}{2}}(2x-1)$.
	\choice
	{$S=(2;+\infty)$}
	{$S=(-1;2)$}
	{$S=(-\infty;2)$}
	{\True $S=\left(\dfrac{1}{2};2\right)$}
	\loigiai{
		Ta có $\log_{\frac{1}{2}}(x+1)<\log_{\frac{1}{2}}(2x-1)\Leftrightarrow\heva{&x+1>2x-1\\&2x-1>0}\Leftrightarrow\dfrac{1}{2}<x<2$.
	}
\end{ex}
%%==========Câu 9
\begin{ex}%[2D2B6-1]
	[THPT - Yên Định Thanh Hóa 2019]%Câu 9.
	Tập nghiệm $S$ của bất phương trình $\log_2(2x+3)\geq 0$ là
	\choice
	{$S=(-\infty;-1]$}
	{\True $S=[-1;+\infty)$}
	{$S=(-\infty;-1)$}
	{$S=(-\infty;0]$}
	\loigiai{
		Ta có $\log_2(2x+3)\geq 0\Leftrightarrow 2x+3\geq 1\Leftrightarrow x\geq-1$.\\
		Vậy tập nghiệm bất phương trình $S=[-1;+\infty)$.
	}
\end{ex}
%%==========Câu 10
\begin{ex}%[2D2B6-1]
	[THPT Đông Sơn Thanh Hóa 2019]%Câu 10.
	Tập nghiệm của bất phương trình $\log_{0.3}(5-2x)>\log_{\frac{3}{10}}9$ là
	\choice
	{$\left(0;\dfrac{5}{2}\right)$}
	{$(-\infty;-2)$}
	{\True $\left(-2;\dfrac{5}{2}\right)$}
	{$(-2;+\infty)$}
	\loigiai{
		$\log_{0.3}(5-2x)>\log_{\frac{3}{10}}9\Leftrightarrow\heva{&5-2x>0\\&5-2x<9}\Leftrightarrow\heva{&x<\dfrac{5}{2}\\&x >-2}\Leftrightarrow-2<x<\dfrac{5}{2.}$ \\
		Vậy bất phương trình có tập nghiệm là $S=\left(-2;\dfrac{5}{2}\right)$.
	}
\end{ex}
%%==========Câu 11
\begin{ex}%[2D2B6-1]
	[Chuyên ĐHSP Hà Nội 2019]%Câu 11.
	Tập nghiệm của bất phương trình $\log_{0,5}(x-1)>1$ là
	\choice
	{$\left(-\infty;-\dfrac{3}{2}\right)$}
	{\True $\left(1;\dfrac{3}{2}\right)$}
	{$\left(\dfrac{3}{2};+\infty\right)$}
	{$\left[1;\dfrac{3}{2}\right)$}
	\loigiai{
		Bất phương trình $\Leftrightarrow 0<x-1<0,5\Leftrightarrow 1<x<\dfrac{3}{2}$.\\
		Vậy tập nghiệm bất phương trình đã cho là $S=\left(1;\dfrac{3}{2}\right)$.
	}
\end{ex}
%%==========Câu 12
\begin{ex}%[2D2B6-1]
	[HSG Bắc Ninh 2019]%Câu 12.
	Tập nghiệm của bất phương trình $\log_{\frac{\pi}{4}}(x+1)>\log_{\frac{\pi}{4}}(2x-5)$ là
	\choice
	{$(-1;6)$}
	{$\left(\dfrac{5}{2};6\right)$}
	{\True $(6;+\infty)$}
	{$(-\infty;6)$}
	\loigiai{
		Do $\dfrac{\pi}{4}<1$ nên $\log_{\frac{\pi}{4}}(x+1)>\log_{\frac{\pi}{4}}(2x-5)\Leftrightarrow\heva{&x+1>0\\&x+1<2x-5}\Leftrightarrow x>6$.
	}
\end{ex}
%%==========Câu 13
\begin{ex}%[2D2B6-1]
	[THPT An Lão Hải Phòng 2019]%Câu 13.
	Tìm tập nghiệm $S$ của bất phương trình $\log_3(2x+3)<\log_3(1-x)$
	\choice
	{$\left(-\dfrac{2}{3};+\infty\right)$}
	{\True $\left(-\dfrac{3}{2};-\dfrac{2}{3}\right)$}
	{$\left(-\dfrac{3}{2};1\right)$}
	{$\left(-\infty;-\dfrac{2}{3}\right)$}
	\loigiai{
		Điều kiện: $\heva{&2x+3>0\\&1-x>0}\Leftrightarrow-\dfrac{3}{2}<x<1$.\\
		$\log_3(2x+3)<\log_3(1-x)\Leftrightarrow 2x+3<1-x\Leftrightarrow x <-\dfrac{2}{3}$.\\
		So với điều kiện, ta được tập nghiệm của bất phương trình là $S=\left(-\dfrac{3}{2};-\dfrac{2}{3}\right)$.
	}
\end{ex}
%%==========Câu 14
\begin{ex}%[2D2B6-1]
	[THPT Cẩm Giàng 2 2019]%Câu 14.
	Tập nghiệm của bất phương trình $\log_3\left(\log_{\frac{1}{2}}x\right)<1$ là
	\choice
	{$(0;1)$}
	{$\left(\dfrac{1}{8};3\right)$}
	{\True $\left(\dfrac{1}{8};1\right)$}
	{$\left(\dfrac{1}{8};+\infty\right)$}
	\loigiai{
		Ta có $\log_3\left(\log_{\frac{1}{2}}x\right)<1\Leftrightarrow 0<\log_{\frac{1}{2}}x<3^1\Leftrightarrow\left(\dfrac{1}{2}\right)^{\circ}>x>\left(\dfrac{1}{2}\right)^3\Leftrightarrow 1>x>\dfrac{1}{8}$.\\
		Vậy tập nghiệm của bất phương trình là $S=\left(\dfrac{1}{8};1\right)$.
	}
\end{ex}
%%==========Câu 15
\begin{ex}%[2D2B6-1]
	[Liên Trường THPT Tp Vinh Nghệ An 2019]%Câu 15.
	Số nghiệm nguyên của bất phương trình $\log_{0,8}(15x+2)>\log_{0,8}(13x+8)$ là
	\choice
	{Vô số}
	{$4$}
	{$2$}
	{\True $3$}
	\loigiai{
		Điều kiện $x >-\dfrac{2}{15}$.\\
		Khi đó, $\log_{0,8}(15x+2)>\log_{0,8}(13x+8)\Leftrightarrow 15x+2<13x+8\Leftrightarrow 2x<6\Leftrightarrow x<3$.\\
		Tập nghiệm bất phương trình là $T=\left(-\dfrac{2}{15};3\right)\Rightarrow x\in\{0;1;2\}$.
	}
\end{ex}
%%==========Câu 16
\begin{ex}%[2D2B6-1]
	[Sở Vĩnh Phúc 2019]%Câu 16.
	Tập xác định của hàm số $y=\sqrt{\log_2(4-x)-1}$ là
	\choice
	{$(-\infty;4)$}
	{$[2;4)$}
	{\True $(-\infty;2]$}
	{$(-\infty;2)$}
	\loigiai{
		Hàm số xác định $\Leftrightarrow\log_2(4-x)-1\geq 0\Leftrightarrow\heva{&\log_2(4-x)\geq 1\\&4-x>0}\Leftrightarrow\heva{&4-x\geq 2\\&4-x>0}\Leftrightarrow\heva{&x\leq 2\\&x<4}\Leftrightarrow x\leq 2$.\\
		Vậy tập xác định của hàm số là $\mathscr{D}=(-\infty;2]$.
	}
\end{ex}
%%==========Câu 17
\begin{ex}%[2D2B6-1]
	[Sở Bình Phước 2019]%Câu 17.
	Tập nghiệm của bất phương trình $\log_2(3x+1)<2$ là
	\choice
	{$\left[-\dfrac{1}{3};1\right)$}
	{$\left(-\dfrac{1}{3};\dfrac{1}{3}\right)$}
	{\True $\left(-\dfrac{1}{3};1\right)$}
	{$(-\infty;1)$}
	\loigiai{
		Điều kiện: $x >-\dfrac{1}{3}$.\\
		$\log_2(3x+1)<2\Leftrightarrow 3x+1<4\Leftrightarrow x<1$.\\
		Kết hợp với điều kiện ta được nghiệm của bất phương trình là $-\dfrac{1}{3}<x<1$.\\
		Vậy tập nghiệm của bất phương trình $\left(-\dfrac{1}{3};1\right)$.
	}
\end{ex}
%%==========Câu 18
\begin{ex}%[2D2B6-1]
	[Chuyên Phan Bội Châu Nghệ An 2019]%Câu 18.
	Tập nghiệm của bất phương trình \break$\log_2\left(x^2-1\right)\geq 3$ là
	\choice
	{$[-2;2]$}
	{\True $(-\infty;-3]\cup[3;+\infty)$}
	{$(-\infty;-2]\cup[2;+\infty)$}
	{$[-3;3]$}
	\loigiai{
		$\log_2\left(x^2-1\right)\geq 3\Leftrightarrow x^2-1\geq 8\Leftrightarrow x^2\geq 9\Leftrightarrow\hoac{&x\geq 3\\&x\leq-3}$.
	}
\end{ex}
%%==========Câu 19
\begin{ex}%[2D2B6-1]
	[Sở Bắc Giang 2019]%Câu 19.
	Tập nghiệm $S$ của bất phương trình $\log_{0,8}(2x-1)<0$ là
	\choice
	{$S=\left(-\infty;\dfrac{1}{2}\right)$}
	{\True $S=(1;+\infty)$}
	{$S=\left(\dfrac{1}{2};+\infty\right)$}
	{$S=(-\infty;1)$}
	\loigiai{
		Bất phương trình $\log_{0,8}(2x-1)<0\Leftrightarrow 2x-1>(0,8)^{0}\Leftrightarrow 2x>2\Leftrightarrow x>1$.\\
		Tập nghiệm $S$ của bất phương trình $\log_{0,8}(2x-1)<0$ là $S=(1;+\infty)$.
	}
\end{ex}
%%==========Câu 20
\begin{ex}%[2D2B6-1]
	[Sở Bắc Giang 2019]%Câu 20.
	Tập nghiệm của bất phương trình $\log_{0,5}(5x+14)\leq\log_{0,5}\left(x^2+6x+8\right)$ là
	\choice
	{\True $(-2;2]$}
	{$(-\infty;2]$}
	{$\mathbb{R}\setminus\left[-\dfrac{3}{2};0\right]$}
	{$[-3;2]$}
	\loigiai{
		Điều kiện: $\heva{&5x+14>0\\&x^2+6x+8>0}\Leftrightarrow x >-2\quad(*)$.\\
		Ta có $\log_{0,5}(5x+14)\leq\log_{0,5}\left(x^2+6x+8\right)\Leftrightarrow 5x+14\geq x^2+6x+8\Leftrightarrow-3\leq x\leq 2$.\\
		Kết hợp với điều kiện $(*)$ ta được $-2<x\leq 2$.\\
		Vậy tập nghiệm của bất phương trình là $(-2;2]$.
	}
\end{ex}
%%==========Câu 21
\begin{ex}%[2D2B6-1]
	[Chuyên Trần Phú Hải Phòng 2019]%Câu 21.
	Bất phương trình $\log_2(3x-2)>\log_2(6-5x)$ có tập nghiệm là
	\choice
	{$(0;+\infty)$}
	{$\left(\dfrac{1}{2};3\right)$}
	{$(-3;1)$}
	{\True $\left(1;\dfrac{6}{5}\right)$}
	\loigiai{
		Vì $2>1$ nên.\\
		$\log_2(3x-2)>\log_2(6-5x)\heva{&3x-2>6-5x\\&6-5x>0}\Leftrightarrow\heva{&x>1\\&x<\dfrac{6}{5}}\Leftrightarrow 1<x<\dfrac{6}{5}$.
	}
\end{ex}
%%==========Câu 22
\begin{ex}%[2D2B6-1]
	[KTNL GV THPT Lý Thái Tổ 2019]%Câu 22.
	Tập hợp nghiệm của bất phương trình $\log_2(x+1)<3$ là
	\choice
	{$S=(-1; 8)$}
	{$S=(-\infty; 7)$}
	{$S=(-\infty; 8)$}
	{\True $S=(-1; 7)$}
	\loigiai{
		Ta có $\log_2(x+1)<3\Leftrightarrow\heva{&x+1>0\\&x+1<2^3}\Leftrightarrow\heva{&x >-1\\&x<7}\Leftrightarrow-1<x<7$.\\
		Vậy tập nghiệm của bất phương trình là $S=(-1; 7)$.
	}
\end{ex}
%%==========Câu 23
\begin{ex}%[2D2B6-1]
	[Sở Thanh Hóa 2019]%Câu 23.
	Tìm tập nghiệm $S$ của bất phương trình $\ln x^2>\ln (4x-4)$.
	\choice
	{$S=(2;+\infty)$}
	{$S=(1;+\infty)$}
	{$S=\mathbb{R}\setminus\{2\}$}
	{\True $S=(1;+\infty)\setminus\{2\}$}
	\loigiai{
		$\ln x^2>\ln (4x-4)\Leftrightarrow\heva{&x^2>4x-4\\&4x-4>0}$ \\
		$\Leftrightarrow\heva{&x^2-4x+4>0\\&x>1}\Leftrightarrow\heva{&x\neq 2\\&x>1.} $ \\
		Vậy tập nghiệm của bất phương trình là $S=(1;+\infty)\setminus\{2\}$.
	}
\end{ex}
%%==========Câu 24
\begin{ex}%[2D2B6-1]
	[Chuyên Phan Bội Châu 2019]%Câu 24.
	Tập nghiệm của bất phương trình $\log_2\left[x^2-1\right]\geq 3$ là
	\choice
	{$[-2;2]$}
	{\True $(-\infty;-3]\cup[3;+\infty)$}
	{$(-\infty;-2]\cup[2;+\infty)$}
	{$[-3;3]$}
	\loigiai{
		Ta có $\log_2\left[x^2-1\right]\geq 3\Leftrightarrow x^2-9\geq 0\Leftrightarrow x\in(-\infty;-3]\cup[3;+\infty)$.
	}
\end{ex}
%%==========Câu 25
\begin{ex}%[2D2B6-1][Chuyên KHTN 2019]%Câu 25.
	Tập nghiệm của bất phương trình $\dfrac{\log\left(x^2-9\right)}{\log(3-x)}\leq 1$ là
	\choice
	{$(-4;-3)$}
	{\True $[-4;-3)$}
	{$(3; 4]$}
	{$\phi$}
	\loigiai{
		Điều kiện: $\heva{&x^2-9>0\\&3-x>0\\&3-x\neq 1}\Leftrightarrow\heva{&x>3\vee x <-3\\&x<3\\&x\neq 2}$. $\Leftrightarrow x <-3$.\\
		Với $x <-3$ suy ra $\log (3-x)>0$ nên bất phương trình đã cho tương đương với $\log\left(x^2-9\right)\leq\log(3-x)\Leftrightarrow x^2+x-12\leq 0\Leftrightarrow x\in[-4;3]$.\\
		Kết hợp điều kiện suy ra tập nghiệm của bất phương trình là $[-4;-3)$.
	}
\end{ex}
%%==========Câu 26
\begin{ex}%[2D2B6-1][Chuyên Thái Bình 2019]%Câu 26.
	Có tất cả bao nhiêu giá trị của tham số $m$ để bất phương trình $\log_2\left(x^2+mx+m+2\right)\geq\log_2\left(x^2+2\right)$ nghiệm đúng $\forall x\in\mathbb{R}$?
	\choice
	{$2$}
	{$4$}
	{$3$}
	{\True $1$}
	\loigiai{
		Ta có $\log_2(x^2+mx+m+2)\geq\log_2(x^2+2)$ nghiệm đúng $\forall x\in\mathbb{R}$ \\
		$ \Leftrightarrow x^2+mx+m+2\geq x^2+2,\forall x\in\mathbb{R}\Leftrightarrow mx+m\geq 0,\forall x\in\mathbb{R}\Leftrightarrow m=0 $.\\
		Suy ra có 1 giá trị m thỏa mãn.
	}
\end{ex}
%%==========Câu 27
\begin{ex}%[2D2B6-2]
	[Việt Đức Hà Nội 2019]%Câu 27.
	Giải bất phương trình $\log_2(3x-2)>\log_2(6-5x)$ được tập nghiệm là $(a;b)$. Hãy tính tổng $S=a+b$.
	\choice
	{$S=\dfrac{26}{5}$}
	{\True $S=\dfrac{11}{5}$}
	{$S=\dfrac{28}{15}$}
	{$S=\dfrac{8}{3}$}
	\loigiai{
		Điều kiện $\heva{&3x-2>0\\&6-5x>0}\Leftrightarrow\heva{&x>\dfrac{2}{3}\\&x<\dfrac{6}{5}}\Leftrightarrow\dfrac{2}{3}<x<\dfrac{6}{5.}$ \\
		Ta có\\
		$\log_2(3x-2)>\log_2(6-5x)\Leftrightarrow 3x-2>6-5x\Leftrightarrow 8x>8\Leftrightarrow x>1$.\\
		Kết hợp với điều kiện, ta được $1<x<\dfrac{6}{5}$.\\
		Vậy, tập nghiệm của bất phương trình là $\left(1;\dfrac{6}{5}\right)$.\\
		Từ đó, $S=a+b=1+\dfrac{6}{5}=\dfrac{11}{5}$.\\
		\textit{lời giải ngắn gọn như sau}:\\
		$\log_2(3x-2)>\log_2(6-5x)\Leftrightarrow\heva{&3x-2>6-5x\\&6-5x>0}\Leftrightarrow\heva{&x>1\\&x<\dfrac{6}{5}}\Leftrightarrow 1<x<\dfrac{6}{5}$.
	}
\end{ex}
%%==========Câu 28
\begin{ex}%[2D2B6-1]
	[Sở Ninh Bình 2019]%Câu 28.
	Bất phương trình $\log_3\left(x^2-2x\right)>1$ có tập nghiệm là
	\choice
	{\True $S=(-\infty;-1)\cup(3;+\infty)$}
	{$S=(-1;3)$}
	{$S=(3;+\infty)$}
	{$S=(-\infty;-1)$}
	\loigiai{
		$\log_3\left(x^2-2x\right)>1\Leftrightarrow x^2-2x>3\Leftrightarrow x^2-2x-3>0\Leftrightarrow\hoac{&x>3\\&x <-1.}$ \\
		Vậy tập nghiệm của bất phương trình $S=(-\infty;-1)\cup(3;+\infty)$.
	}
\end{ex}
%%==========Câu 29
\begin{ex}%[2D2B6-2]
	[Hậu Lộc 2 - Thanh Hóa 2019]%Câu 29.
	Tập nghiệm của bất phương trình $\ln 3x<\ln (2x+6)$ là
	\choice
	{$[0; 6)$}
	{\True $(0; 6)$}
	{$(6;+\infty)$}
	{$(-\infty; 6)$}
	\loigiai{
		Bất phương trình $\ln 3x<\ln (2x+6)\Leftrightarrow\heva{&3x>0\\&3x<2x+6}\Leftrightarrow 0<x<6$.
	}
\end{ex}
%%==========Câu 30
\begin{ex}%[2D2B6-1]
	[Hội 8 trường chuyên ĐBSH - 2019]%Câu 30.
	Tập nghiệm $S$ của bất phương trình $\log_2(x-1)<3$ là
	\choice
	{\True $S=(1;9)$}
	{$S=(1;10)$}
	{$S=(-\infty;9)$}
	{$S=(-\infty;10)$}
	\loigiai{
		$\log_2(x-1)<3\Leftrightarrow 0<x-1<2^3\Leftrightarrow 1<x<9$.
	}
\end{ex}
%%==========Câu 31
\begin{ex}%[2D2B6-1]
	[THPT Phan Bội Châu - Nghệ An -2019]%Câu 31.
	Tập nghiệm của bất phương trình $\log_2\left(x^2-1\right)\geq 3$ là
	\choice
	{$[-2;2]$}
	{\True $(-\infty;-3]\cup[3;+\infty)$}
	{$(-\infty;-2]\cup[2;+\infty)$}
	{$[-3;3]$}
	\loigiai{
		$\log_2\left(x^2-1\right)\geq 3\Leftrightarrow x^2-1\geq 8\Leftrightarrow x^2\geq 9\Leftrightarrow\hoac{&x\geq 3\\&x\leq-3}$.
	}
\end{ex}
%%==========Câu 32
\begin{ex}%[2D2B6-2]
	[Bắc Ninh 2019]%Câu 32.
	Bất phương trình $\log_2(3x-2)>\log_2(6-5x)$ có tập nghiệm là $(a; b)$. Tổng $a+b$ bằng
	\choice
	{$\dfrac{8}{3}$}
	{$\dfrac{28}{15}$}
	{$\dfrac{26}{5}$}
	{\True $\dfrac{11}{5}$}
	\loigiai{
		Ta có $\log_2(3x-2)>\log_2(6-5x)\Leftrightarrow\heva{&3x-2>6-5x\\&6-5x>0}\Leftrightarrow\heva{&x>1\\&x<\dfrac{6}{5}}\Leftrightarrow 1<x<\dfrac{6}{5.}$ \\
		Tập nghiệm của bất phương trình là $(1;\dfrac{6}{5})$.\\
		Vậy $a+b=1+\dfrac{6}{5}=\dfrac{11}{5}$.
	}
\end{ex}
%%==========Câu 33
\begin{ex}%[2D2B6-1]
	[THPT Hai Bà Trưng - Huế - 2019]%Câu 33.
	Có tất cả bao nhiêu số nguyên $x$ thỏa mãn bất phương trình $\log_{\frac{1}{2}}\left[\log_2\left(2-x^2\right)\right]>0$?
	\choice
	{Vô số}
	{$1$}
	{\True $0$}
	{$2$}
	\loigiai{
		$\log_{\frac{1}{2}}\left[\log_2\left(2-x^2\right)\right]>0$ \\
		$ \Leftrightarrow 0<\log_2\left(2-x^2\right)<1 $ \\
		$ \Leftrightarrow 1<2-x^2<2 $ \\
		$\Leftrightarrow\heva{&2-x^2<2\\&2-x^2>1}\Leftrightarrow\heva{&x^2>0\\&x^2<1}\Leftrightarrow\heva{&x\neq 0\\&-1<x<1.} $ \\
		Kết hợp với giả thiết $x$ là số nguyên ta thấy không có số nguyên $x$ nào thỏa mãn bất phương trình $\log_{\frac{1}{2}}\left[\log_2\left(2-x^2\right)\right]>0$
	}
\end{ex}
%%==========Câu 34
\begin{ex}%[2D2B6-2]
	[THPT Cẩm Bình 2019]%Câu 34.
	Nghiệm của bất phương trình $\log_{2-\sqrt{3}}(2x-5)\geq\log_{2-\sqrt{3}}(x-1)$ là
	\choice
	{\True $\dfrac{5}{2}<x\leq 4$}
	{$1<x\leq 4$}
	{$\dfrac{5}{2}\leq x\leq 4 1$}
	{$x\geq 4$}
	\loigiai{
		$\log_{2-\sqrt{3}}(2x-5)\geq\log_{2-\sqrt{3}}(x-1)\Leftrightarrow\heva{&2x-5\leq x-1\\&2x-5>0}\Leftrightarrow\heva{&x\leq 4\\&x>\dfrac{5}{2}.}$ \\
		Vậy nghiệm của bất phương trình là $\dfrac{5}{2}<x\leq 4$.
	}
\end{ex}
%%==========Câu 35
\begin{ex}%[2D2B6-2]
	[THPT Hàm Rồng 2019]%Câu 35.
	Bất phương trình $\log_4(x+7)>\log_2(x+1)$ có bao nhiêu nghiệm nguyên
	\choice
	{$3$}
	{$1$}
	{$4$}
	{\True $2$}
	\loigiai{
		Điều kiện xác định của bất phương trình là $\heva{&x+7>0\\&x+1>0}\Leftrightarrow\heva{&x >-7\\&x >-1}\Leftrightarrow x >-1$.\\
		Ta có $\log_4(x+7)>\log_2(x+1)\Leftrightarrow\dfrac{1}{2}\log_2(x+7)>\log_2(x+1)\Leftrightarrow\log_2(x+7)>\log_2(x+1)^2$ \\
		$ \Leftrightarrow x^2+x-6<0\Leftrightarrow-3<x<2 $.\\
		Kết hợp điều kiện ta được $-1<x<2$.\\
		Vì $x\in\mathbb{Z}$ nên tìm được $x=0,x=1$.
	}
\end{ex}
%%==========Câu 36
\begin{ex}%[2D2B6-1]
	[Thi thử cụm Vũng Tàu - 2019]%Câu 36.
	Tập nghiệm của bất phương trình $\log_{\frac{3}{5}}\left(2x^2-x+1\right)<0$ là
	\choice
	{$\left(-1;\dfrac{3}{2}\right)$}
	{$(-\infty;1)\cup\left(\dfrac{3}{2};+\infty\right)$}
	{\True $(-\infty;0)\cup\left(\dfrac{1}{2};+\infty\right)$}
	{$\left(0;\dfrac{1}{2}\right)$}
	\loigiai{
		Ta có: $2x^2-x+1>0$, $\forall x\in\mathbb{R}$.\\
		Do đó $\log_{\frac{3}{5}}\left(2x^2-x+1\right)<0\Leftrightarrow 2x^2-x+1>1\Leftrightarrow 2x^2-x>0\Leftrightarrow\hoac{&x<0\\&x>\dfrac{1}{2}.}$ \\
		Vậy tập nghiệm của bất phương trình là $S=(-\infty;0)\cup\left(\dfrac{1}{2}\colon +\infty\right)$.
	}
\end{ex}
%%==========Câu 37
\begin{ex}%[2D2B6-1]
	[Bình Phước - 2019]%Câu 37.
	Tập nghiệm của bất phương trình $\log_2(3x+1)<2$ là
	\choice
	{$\left[-\dfrac{1}{3};1\right)$}
	{$\left(-\dfrac{1}{3};\dfrac{1}{3}\right)$}
	{\True $\left(-\dfrac{1}{3};1\right)$}
	{$(-\infty;1)$}
	\loigiai{
		Điều kiện: $x >-\dfrac{1}{3}$.\\
		$\log_2(3x+1)<2\Leftrightarrow 3x+1<4\Leftrightarrow x<1$.\\
		Kết hợp với điều kiện ta được nghiệm của bất phương trình là $-\dfrac{1}{3}<x<1$.\\
		Vậy tập nghiệm của bất phương trình $\left(-\dfrac{1}{3};1\right)$.
	}
\end{ex}
%%==========Câu 38
\begin{ex}%[2D2B6-1]
	[Ngô Quyền - Hải Phòng -2019]%Câu 38.
	Số nghiệm nguyên của bất phương trình \break $\log_{\frac{1}{2}}\left(x^2+2x-8\right)\geq-4$ là
	\choice
	{$6$}
	{Vô số}
	{\True $4$}
	{$5$}
	\loigiai{
		Ta có\\
		$\log_{\frac{1}{2}}\left(x^2+2x-8\right)\geq-4\Leftrightarrow\heva{&x^2+2 x-8>0\\&x^2+2 x-8\leq\left(\dfrac{1}{2}\right)^{-4}}\Leftrightarrow\heva{&\hoac{&x>2\\&x <-4}\\&x^2+2 x-24\leq 0}$ \\
		$ \Leftrightarrow\heva{&\hoac{&x>2\\&x <-4}\\&-6\leq x\leq 4}\Leftrightarrow\hoac{&-6\leq x <-4\\&2<x\leq 4.} $ \\
		Do đó các nghiệm nguyên của bất phương trình đã cho là $-6;-5; 3; 4$.
	}
\end{ex}
%%==========Câu 39
\begin{ex}%[2D2B6-2]
	[THPT Thuận Thành 3 - Bắc Ninh 2019]%Câu 39.
	Tập nghiệm $S$ của bất phương trình $\log_6x^2<\log_6(x+6)$ là
	\choice
	{$S=(-\infty;-2)\cup(3;+\infty)$}
	{$S=(-2;3)$}
	{$S=(-3;2)\setminus\{0\}$}
	{\True $S=(-2;3)\setminus\{0\}$}
	\loigiai{
		Điều kiện: $\heva{&x\neq 0\\&x >-6.}$ \\
		$\log_6x^2<\log_6(x+6)\Leftrightarrow x^2<x+6\Leftrightarrow x^2-x-6<0\Leftrightarrow-2<x<3$.\\
		Kết hợp với điều kiện, suy ra tập nghiệm $S=(-2;3)\setminus\{0\}$.
	}
\end{ex}
%%==========Câu 40
\begin{ex}%[2D2B6-1]
	[Nho Quan A - Ninh Bình - 2019]%Câu 40.
	Bất phương trình $\log_2(x-2)<2$ có bao nhiêu nghiệm nguyên?
	\choice
	{$4$}
	{$2$}
	{$5$}
	{\True $3$}
	\loigiai{
		$\log_2(x-2)<2\Leftrightarrow\heva{&x-2>0\\&x-2<4}\Leftrightarrow\heva{&x>2\\&x<6}\Leftrightarrow 2<x<6$.\\
		Vậy bất phương trình đã cho có 3 nghiệm nguyên.
	}
\end{ex}
%%==========Câu 41
\begin{ex}%[2D2B6-1]
	[Cần Thơ 2019]%Câu 41.
	Tập nghiệm của bất phương trình $\log_{0,2}(x-4)+1>0$ là
	\choice
	{$(4;+\infty)$}
	{\True $(4;9)$}
	{$(-\infty;9)$}
	{$(9;+\infty)$}
	\loigiai{
		Ta có $\log_{0,2}(x-4)+1>0\Leftrightarrow\log_{0,2}(x-4) >-1\Leftrightarrow\log_{0,2}(x-4)>\log_{0,2}\left[(0,2)^{-1}\right]$ \\
		$\Leftrightarrow\heva{&x-4>0\\&x-4<5}\Leftrightarrow\heva{&x>4\\&x<9.} $ \\
		Vậy bất phương trình có tập nghiệm là $(4;9)$.
	}
\end{ex}
%%==========Câu 42
\begin{ex}%[2D2B6-2]
	[THPT Cẩm Bình Hà Tỉnh 2019]%Câu 42.
	Tập nghiệm của bất phương trình $\log_2(7-x)+\log_{\frac{1}{2}}(x-1)\leq 0$ là
	\choice
	{$S=(1; 4]$}
	{$S=(-\infty; 4]$}
	{$S=[4;+\infty)$}
	{\True $S=[4; 7)$}
	\loigiai{
		Điều kiện: $1<x<7$.\\
		Ta có:\\
		$\log_2(7-x)+\log_{\frac{1}{2}}(x-1)\leq 0\Leftrightarrow\log_2(7-x)-\log_2(x-1)\leq 0$ \\
		$ \Leftrightarrow\log_2\dfrac{7-x}{x-1}\leq 0\Leftrightarrow\dfrac{7-x}{x-1}\leq 1\Leftrightarrow\dfrac{-2x+8}{x-1}\leq 0\Leftrightarrow\hoac{&x<1\\&x\geq 4.} $ \\
		Kết hợp với điều kiện ta có tập nghiệm là $[4;7)$.
	}
\end{ex}
%%==========Câu 43
\begin{ex}%[2D2B6-2]
	[NK HCM - 2019]%Câu 43.
	Bất phương trình $1+\log_2(x-2)>\log_2\left(x^2-3x+2\right)$ có các nghiệm là
	\choice
	{$S=(3;+\infty)$}
	{$S=(1; 3)$}
	{$S=(2;+\infty)$}
	{\True $S=(2; 3)$}
	\loigiai{
		Điều kiện: $x>2$.\\
		$1+\log_2(x-2)>\log_2\left(x^2-3x+2\right)\Leftrightarrow\log_2\left(x^2-3x+2\right)-\log_2(x-2)<1\Leftrightarrow\log_2(x-1)<1\Leftrightarrow x<3$.\\
		Đối chiếu điều kiện, ta có tập nghiệm là $S=(2; 3)$.
	}
\end{ex}
%%==========Câu 44
\begin{ex}%[2D2B6-1]
	[Mã 101 - 2022]%Câu 44.
	Tập nghiệm của bất phương trình $\log_5(x+1)>2$ là
	\choice
	{$(9;+\infty)$}
	{$(25;+\infty)$}
	{$(31;+\infty)$}
	{\True $(24;+\infty)$}
	\loigiai{
		Điều kiện xác định $x >-1$.\\
		$\log_5(x+1)>2\Leftrightarrow\log_5(x+1)>\log_525\Leftrightarrow x+1>25\Leftrightarrow x>24$.
	}
\end{ex}
\begin{dang}
	{Bất phương trình mũ}
	\begin{itemize}
		\item Nếu $a>1$ thì $a^{f(x)}>a^{g(x)} \Leftrightarrow f(x)>g(x)$ \qquad (cùng chiều).
		\item  Nếu $0<a<1$ thì $a^{f(x)}>a^{g(x)} \Leftrightarrow f(x)<g(x)$ \qquad (ngược chiều).
		\item Nếu $a$ chưa ẩn thì $a^{f(x)}>a^{g(x)} \Leftrightarrow (a-1)\left[f(x)-g(x)\right]>0$.
	\end{itemize}
\end{dang}
\begin{ex}%[2D2Y6-1]
	[Mã 102 - 2021 Lần 1]
	Tập nghiệm của bất phương trình $2^x<5$ là
	\choice
	{\True $\left(-\infty;\log_25\right)$}
	{$\left(\log_25;+\infty\right)$}
	{$\left(-\infty;\log_52\right)$}
	{$\left(\log_52;+\infty\right)$}
	\loigiai{
		Ta có $2^x<5\Leftrightarrow x<\log_25$.\\
		Vậy tập nghiệm $S=\left(-\infty;\log_25\right)$.
	}
\end{ex}
\begin{ex}%[2D2Y6-1]
	[Đề Minh Họa 2021]
	Tập nghiệm của bất phương trình $3^{4-x^2}\geq 27$ là
	\choice
	{\True $[-1;1]$}
	{$(-\infty;1]$}
	{$\left[-\sqrt{7};\sqrt{7}\right]$}
	{$[1;+\infty)$}
	\loigiai{
		Ta có $3^{4-x^2}\geq 27\Leftrightarrow 4-x^2\geq 3\Leftrightarrow-1\leq x\leq 1$.
	}
\end{ex}
\begin{ex}%[2D2Y6-1]
	[Mã 101 - 2021 Lần 1]
	Tập nghiệm của bất phương trình $3^x<2$ là
	\choice
	{\True $\left(-\infty;\log_32\right)$}
	{$\left(\log_32;+\infty\right)$}
	{$\left(-\infty;\log_23\right)$}
	{$\left(\log_23;+\infty\right)$}
	\loigiai{
		Ta có $3^x<2\Leftrightarrow x<\log_32$.\\
		Vậy $S=\left(-\infty;\log_32\right)$.
	}
\end{ex}
\begin{ex}%[2D2Y6-1]
	[Mã 104 - 2021 Lần 1]
	Tập nghiệm của bất phương trình $2^x>5$ là
	\choice
	{$(-\infty;\log_25)$}
	{$(\log_52;+\infty)$}
	{$(-\infty;\log_52)$}
	{\True $(\log_25;+\infty)$}
	\loigiai{
		Ta có $2^x>5\Leftrightarrow x>\log_25$.\\
		Tập nghiệm của bất phương trình là $(\log_25;+\infty)$.
	}
\end{ex}
\begin{ex}%[2D2Y6-1]
	[Mã 103 - 2021 - Lần 1]
	Tập nghiệm của bất phương trình $2^x>3$ là
	\choice
	{$\left(\log_32;+\infty\right)$}
	{$\left(-\infty;\log_23\right)$}
	{$\left(-\infty;\log_32\right)$}
	{\True $\left(\log_23;+\infty\right)$}
	\loigiai{
		Ta có $2^x>3\Leftrightarrow x>\log_23$.\\
		Tập nghiệm của bất phương trình là $\left(\log_23;+\infty\right)$.
	}
\end{ex}
\begin{ex}%[2D2B6-2]
	[Đề Minh Họa 2020 Lần 1]
	Tập nghiệm của bất phương trình $5^{x-1}\geq 5^{x^2-x-9}$ là
	\choice
	{\True $[-2;4]$}
	{$[-4;2]$}
	{$(-\infty;-2]\cup[4;+\infty)$}
	{$(-\infty;-4]\cup[2;+\infty)$}
	\loigiai{
		$5^{x-1}\geq 5^{x^2-x-9}\Leftrightarrow x-1\geq x^2-x-9\Leftrightarrow x^2-2x-8\leq 0\Leftrightarrow-2\leq x\leq 4$.\\
		Vậy Tập nghiệm của bất phương trình là $[-2;4]$.
	}
\end{ex}
\begin{ex}%[2D2Y6-3]
	[Đề Tham Khảo 2020 Lần 2]
	Tập nghiệm của bất phương trình $9^x+2\cdot 3^x-3>0$ là
	\choice
	{$[0;+\infty)$}
	{\True $(0;+\infty)$}
	{$(1;+\infty)$}
	{$[1;+\infty)$}
	\loigiai{
		$9^x+2\cdot 3^x-3>0\Leftrightarrow\left(3^x-1\right)\left(3^x+3\right)>0\Leftrightarrow 3^x>1$ (vì $3^x>0,\forall x\in\mathbb{R}$) $\Leftrightarrow x>0$.\\
		Vậy tập nghiệm của bất phương trình đã cho là $(0;+\infty)$.
	}
\end{ex}
\begin{ex}%[2D2B6-2]
	[Mã 101 - 2020 Lần 1]
	Tập nghiệm của bất phương trình $3^{x^2-13}<27$ là
	\choice
	{$(4;+\infty)$}
	{\True $(-4; 4)$}
	{$(-\infty; 4)$}
	{$(0; 4)$}
	\loigiai{
		Ta có: $3^{x^2-13}<27\Leftrightarrow 3^{x^2-13}<3^3\Leftrightarrow x^2-13<3\Leftrightarrow x^2<16\Leftrightarrow|x|<4\Leftrightarrow-4<x<4$.\\
		Vậy tập nghiệm của bất phương trình đã cho là $S=(-4; 4)$.
	}
\end{ex}
\begin{ex}%[2D2B6-2]
	[Mã 102 - 2020 Lần 1]
	Tập nghiệm của bất phương trình $3^{x^2-23}<9$ là
	\choice
	{\True $(-5 ; 5)$}
	{$(-\infty ; 5)$}
	{$(5 ;+\infty)$}
	{$(0 ; 5)$}
	\loigiai{
		Ta có $3^{x^2-23}<9 \Leftrightarrow x^2-23<2 \Leftrightarrow x^2<25 \Leftrightarrow-5<x<5$.\\
		Vậy nghiệm của bất phương trình $3^{x^2-23}<9$ là $(-5 ; 5)$.
	}
\end{ex}
\begin{ex}%[2D2B6-2]
	[Mã 103 - 2020 Lần 1]
	Tập nghiệm của bất phương trình $2^{x^2-7}<4$ là
	\choice
	{\True $(-3;3)$}
	{$(0;3)$}
	{$(-\infty;3)$}
	{$(3;+\infty)$}
	\loigiai{
		Ta có $2^{x^2-7}<4\Leftrightarrow 2^{x^2-7}<2^2\Rightarrow x^2-7<2\Leftrightarrow x^2<9\Rightarrow x\in(-3;3)$.
	}
\end{ex}
\begin{ex}%[2D2B6-2]
	[Mã 104 - 2020 Lần 1]
	Tập nghiệm của bất phương trình $2^{x^2-1}<8$ là
	\choice
	{$(0;2)$}
	{$(-\infty;2)$}
	{\True $(-2;2)$}
	{$(2;+\infty)$}
	\loigiai{
		Từ phương trình ta có $x^2-1<3\Leftrightarrow-2<x<2$.
	}
\end{ex}
\begin{ex}%[2D2B6-2]
	[Đề Tham Khảo 2018]
	Tập nghiệm của bất phương trình $2^{2x}<2^{x+6}$ là
	\choice
	{\True $(-\infty;6)$}
	{$(0;64)$}
	{$(6;+\infty)$}
	{$(0;6)$}
	\loigiai{
		Cách 1: $2^{2x}<2^{x+6}\Leftrightarrow 2x<x+6\Leftrightarrow x<6$.\\
		Cách 2.
		Đặt $t=2^x$, $t>0$.\\
		Bất phương trình trở thành: $t^2-64t<0\Leftrightarrow 0<t<64\Leftrightarrow 0<2^x<64\Leftrightarrow x<6$.
	}
\end{ex}
\begin{ex}%[2D2B6-2]
	[Đề Tham Khảo 2019]
	Tập nghiệm của bất phương trình $3^{x^2-2x}<27$ là
	\choice
	{$(3;+\infty)$}
	{\True $(-1;3)$}
	{$(-\infty;-1)\cup(3;+\infty)$}
	{$(-\infty;-1)$}
	\loigiai{
		Ta có $3^{x^2-2x}<27\Leftrightarrow x^2-2x<3\Leftrightarrow x^2-2x-3<0\Leftrightarrow-1<x<3$.
	}
\end{ex}
\begin{ex}%[2D2Y6-4]
	[Đề Minh Họa 2017]
	Cho hàm số $f(x)=2^x\cdot 7^{x^2}$. Khẳng định nào sau đây là khẳng định sai?
	\choice
	{$f(x)<1\Leftrightarrow x+x^2\log_27<0$}
	{$f(x)<1\Leftrightarrow x\ln 2+x^2\ln 7<0$}
	{$f(x)<1\Leftrightarrow x\log_72+x^2<0$}
	{\True $f(x)<1\Leftrightarrow 1+x\log_27<0$}
	\loigiai{
		Đáp án A đúng vì $f(x)<1\Leftrightarrow\log_2f(x)<\log_21\Leftrightarrow\log_2\left(2^x{\cdot 7}^{x^2}\right)<0\Leftrightarrow\log_22^x+\log_27^{x^2}<0$ \\
		$ \Leftrightarrow x+x^2\cdot\log_27<0 $.\\
		Đáp án B đúng vì $f(x)<1\Leftrightarrow\ln f(x)<\ln 1\Leftrightarrow\ln \left(2^x{\cdot 7}^{x^2}\right)<0\Leftrightarrow\ln 2^x+\ln 7^{x^2}<0$ \\
		$ \Leftrightarrow x\cdot\ln 2+x^2\cdot\ln 7<0 $.\\
		Đáp án C đúng vì $f(x)<1\Leftrightarrow\log_7f(x)<\log_71\Leftrightarrow\log_7\left(2^x{\cdot 7}^{x^2}\right)<0\Leftrightarrow\log_72^x+\log_77^{x^2}<0$ \\
		$ \Leftrightarrow x\cdot\log_72+x^2<0 $.\\
		Vậy D sai vì $f(x)<1\Leftrightarrow\log_2f(x)<\log_21\Leftrightarrow\log_2\left(2^x{\cdot 7}^{x^2}\right)<0\Leftrightarrow\log_22^x+\log_27^{x^2}<0$ \\
		$ \Leftrightarrow x+x^2\log_27<0 $.
	}
\end{ex}
\begin{ex}%[2D2B6-2]
	[Đề Tham Khảo 2017]
	Tìm tập nghiệm $S$ của bất phương trình $5^{x+1}-\dfrac{1}{5}>0$.
	\choice
	{$S=(-\infty;-2)$}
	{$S=(1;+\infty)$}
	{$S=(-1;+\infty)$}
	{\True $S=(-2;+\infty)$}
	\loigiai{
		Bất phương trình tương đương $5^{x+1}>5^{-1}\Leftrightarrow x+1 >-1\Leftrightarrow x >-2$.\\
		Vậy tập nghiệm của bất phương trình là $S=(-2;+\infty)$.
	}
\end{ex}
\begin{ex}%[2D2B6-2]
	[THPT Hoàng Hoa Thám Hưng Yên 2019]
	Cho hàm số $y=\mathrm{e}^{x^2+2x-3}-1$. Tập nghiệm của bất phương trình $y’\geq 0$ là
	\choice
	{$(-\infty;-1]$}
	{$(-\infty;-3]\cup[1;+\infty)$}
	{$[-3; 1]$}
	{\True $[-1;+\infty)$}
	\loigiai{
		Ta có $y’=(2x+2)\mathrm{e}^{x^2+2x-3}$.\\
		$y’\geq 0\Leftrightarrow(2x+2)\mathrm{e}^{x^2+2x-3}\geq 0\Leftrightarrow 2x+2\geq 0\Leftrightarrow x\geq-1$.\\
		Vậy tập nghiệm của bất phương trình $y’\geq 0$ là $[-1;+\infty)$.
	}
\end{ex}
\begin{ex}%[2D2B6-2]
	[THPT Hùng Vương Bình Phước 2019]
	Tập nghiệm của bất phương trình $\left(\dfrac{1}{3}\right)^x>9$ trên tập số thực là
	\choice
	{$(2;+\infty)$}
	{\True $(-\infty;-2)$}
	{$(-\infty;2)$}
	{$(-2;+\infty)$}
	\loigiai{
		$\left(\dfrac{1}{3}\right)^x>9\Leftrightarrow 3^{-x}>3^2\Leftrightarrow-x>2\Leftrightarrow x <-2$.\\
		Vậy tập nghiệm là $S=(-\infty;-2)$.
	}
\end{ex}
\begin{ex}%[2D2B6-2]
	[THPT Bạch Đằng Quảng Ninh 2019]
	Tập nghiệm của bất phương trình $4^{x+1}\leq 8^{x-2}$ là
	\choice
	{\True $[8;+\infty)$}
	{$\varnothing$}
	{$(0;8)$}
	{$(-\infty;8]$}
	\loigiai{
		Ta có $4^{x+1}\leq 8^{x-2}\Leftrightarrow 2^{2x+2}\leq 2^{3x-6}\Leftrightarrow 2x+2\leq 3x-6\Leftrightarrow x\geq 8$.\\
		Vậy tập nghiệm của bất phương trình là $S=[8;+\infty)$.
	}
\end{ex}
\begin{ex}%[2D2B6-2]
	[THPT Cù Huy Cận 2019]%Câu 19.
	Tập nghiệm của bất phương trình $2^{x^2+2x}\leq 8$ là
	\choice
	{$(-\infty;-3]$}
	{\True $[-3; 1]$}
	{$(-3; 1)$}
	{$(-3; 1]$}
	\loigiai{
		Ta có $2^{x^2+2x}\leq 8\Leftrightarrow 2^{x^2+2x}\leq 2^3\Leftrightarrow x^2+2x-3\leq 0\Leftrightarrow-3\leq x\leq 1$.
	}
\end{ex}
\begin{ex}%[2D2B6-2]
	[Chuyên Vĩnh Phúc 2019]%Câu 20.
	Tập nghiệm $S$ của bất phương trình $5^{x+2}<\left(\dfrac{1}{25}\right)^{-x}$ là
	\choice
	{$S=(-\infty;2)$}
	{$S=(-\infty;1)$}
	{$S=(1;+\infty)$}
	{\True $S=(2;+\infty)$}
	\loigiai{
		$5^{x+2}<\left(\dfrac{1}{25}\right)^{-x}\Leftrightarrow 5^{x+2}<5^{2x}\Leftrightarrow x+2<2x\Leftrightarrow x>2$.
	}
\end{ex}
\begin{ex}%[2D2B6-2]
	[THPT Gang Thép Thái Nguyên 2019]%Câu 21.
	Tập nghiệm bất phương trình $2^{x^2-3x}<16$ là
	\choice
	{$(-\infty;-1)$}
	{$(4;+\infty)$}
	{\True $(-1;4)$}
	{$(-\infty;-1)\cup(4;+\infty)$}
	\loigiai{
		$2^{x^2-3x}<16\Leftrightarrow 2^{x^2-3x}<2^4\Leftrightarrow x^2-3x<4\Leftrightarrow-1<x<4$.
	}
\end{ex}
\begin{ex}%[2D2B6-2]
	[THPT Gang Thép Thái Nguyên 2019]%Câu 22.
	Tập nghiệm bất phương trình: $2^x>8$ là
	\choice
	{$(-\infty; 3)$}
	{$[3;+\infty)$}
	{\True $(3;+\infty)$}
	{$(-\infty; 3]$}
	\loigiai{
		Ta có $2^x>8\Leftrightarrow 2^x>2^3\Leftrightarrow x>3$.\\
		Vậy tập nghiệm bất phương trình là $(3;+\infty)$.
	}
\end{ex}
\begin{ex}%[2D2B6-2]
	[Chuyên Quốc Học Huế 2019]%Câu 23.
	Tìm tập nghiệm $S$ của bất phương trình $\left(\dfrac{1}{2}\right)^{-x^2+3x}<\dfrac{1}{4}$.
	\choice
	{$S=[1; 2]$}
	{$S=(-\infty; 1)$}
	{\True $S=(1; 2)$}
	{$S=(2;+\infty)$}
	\loigiai{
		$\left(\dfrac{1}{2}\right)^{-x^2+3x}<\dfrac{1}{4}\Leftrightarrow\left(\dfrac{1}{2}\right)^{-x^2+3x}<\left(\dfrac{1}{2}\right)^2\Leftrightarrow-x^2+3x>2\Leftrightarrow x^2-3x+2<0\Leftrightarrow 1<x<2$.\\
		Vậy tập nghiệm của bất phương trìnhđã cho là $S=(1; 2)$.
	}
\end{ex}
\begin{ex}%[2D2B6-2]
	[Đề Tham Khảo 2019]%Câu 24.
	Tập nghiệm của bất phương trình $3^{x^2-2x}<27$ là
	\choice
	{$(-\infty;-1)$}
	{$(3;+\infty)$}
	{\True $(-1;3)$}
	{$(-\infty;-1)\cup(3;+\infty)$}
	\loigiai{
		Ta có $3^{x^2-2x}<27\Leftrightarrow x^2-2x<3\Leftrightarrow x^2-2x-3<0\Leftrightarrow-1<x<3$.
	}
\end{ex}
\begin{ex}%[2D2B6-2]
	[Chuyên Vĩnh Phúc 2019]%Câu 25.
	Cho $f(x)=x\cdot\mathrm{e}^{-3x}$. Tập nghiệm của bất phương trình $f’(x)>0$ là
	\choice
	{\True $\left(-\infty;\dfrac{1}{3}\right)$}
	{$\left(0;\dfrac{1}{3}\right)$}
	{$\left(\dfrac{1}{3};+\infty\right)$}
	{$(0; 1)$}
	\loigiai{
		Ta có $f’(x)=\mathrm{e}^{-3x}-3x\cdot\mathrm{e}^{-3x}=\mathrm{e}^{-3x}(1-3x)$.\\
		$f’(x)>0\Leftrightarrow\mathrm{e}^{-3x}(1-3x)>0\Leftrightarrow 1-3x>0\Leftrightarrow x<\dfrac{1}{3}$.
	}
\end{ex}
\begin{ex}%[2D2B6-2]
	[THPT Ba Đình 2019]%Câu 26.
	Số nghiệm nguyên của bất phương trình $\left(\dfrac{1}{3}\right)^{2x^2-3x-7}>3^{2x-21}$ là
	\choice
	{\True $7$}
	{$6$}
	{vô số}
	{$8$}
	\loigiai{
		Ta có $\left(\dfrac{1}{3}\right)^{2x^2-3x-7}>3^{2x-21}\Leftrightarrow 3^{-\left(2x^2-3x-7\right)}>3^{2x-21}$ \\
		$\Leftrightarrow-\left(2x^2-3x-7\right)>2x-21\Leftrightarrow-2x^2+3x+7>2x-21 $ \\
		$\Leftrightarrow-2x^2+x+28>0\Leftrightarrow-\dfrac{7}{2}<x<4 $.\\
		Do $x\in\mathbb{Z}$ nên $x\in\left\{-3;-2;-1; 0; 1; 2; 3\right\}$.\\
		Vậy bất phương trình đã cho có $7$ nghiệm nguyên.
	}
\end{ex}
\begin{ex}%[2D2B6-2]
	[THPT Lương Thế Vinh Hà Nội 2019]%Câu 27.
	Tập nghiệm của bất phương trình $2^{3x}<\left(\dfrac{1}{2}\right)^{-2x-6}$ là
	\choice
	{$(0;6)$}
	{\True $(-\infty;6)$}
	{$(0;64)$}
	{$(6;+\infty)$}
	\loigiai{
		Ta có $2^{3x}<\left(\dfrac{1}{2}\right)^{-2x-6}\Leftrightarrow 2^{3x}<2^{2x+6}\Leftrightarrow 3x<2x+6\Leftrightarrow x<6$.\\
		Vậy tập nghiệm của bất phương trình là $S=(-\infty;6)$.
	}
\end{ex}
\begin{ex}%[2D2B6-2]
	[Chuyên Hùng Vương Gia Lai 2019]%Câu 28.
	Bất phương trình $\left(\dfrac{1}{2}\right)^{x^2-2x}\geq\dfrac{1}{8}$ có tập nghiệm là
	\choice
	{$[3;+\infty)$}
	{$(-\infty;-1]$}
	{\True $[-1;3]$}
	{$(-1;3)$}
	\loigiai{
		Bất phương trình đã cho tương đương với.\\
		$\left(\dfrac{1}{2}\right)^{x^2-2x}\geq\left(\dfrac{1}{2}\right)^3\Leftrightarrow x^2-2x\leq 3\Leftrightarrow x^2-2x-3\leq 0\Leftrightarrow-1\leq x\leq 3$.\\
		Vậy, tập nghiệm của bất phương trình đã cho là $S=[-1;3]$.
	}
\end{ex}
\begin{ex}%[2D2B6-2]
	[THPT Yên Phong 1 Bắc Ninh 2019]%Câu 29.
	Nghiệm nguyên lớn nhất của bất phương trình $4^{x^2-2x}<64$ là
	\choice
	{\True $2$}
	{$-1$}
	{$3$}
	{$0$}
	\loigiai{
		Ta có $4^{x^2-2x}<64\Leftrightarrow 4^{x^2-2x}<4^3\Leftrightarrow x^2-2x-3<0\Leftrightarrow-1<x<3$. Vậy nghiệm nguyên lớn nhất là $x=2$.
	}
\end{ex}
\begin{ex}%[2D2B6-2]
	[Sở Hà Nội 2019]%Câu 30.
	Tập nghiệm của bất phương trình $\left(\dfrac{3}{4}\right)^{-x^2}>\dfrac{81}{256}$ là
	\choice
	{$(-\infty;-2)$}
	{$(-\infty;-2)\cup(2;+\infty)$}
	{\True $\mathbb{R}$}
	{$(-2;2)$}
	\loigiai{
		Ta có $\left(\dfrac{3}{4}\right)^{-x^2}>\dfrac{81}{256}\Leftrightarrow\left(\dfrac{3}{4}\right)^{-x^2}>\left(\dfrac{3}{4}\right)^4\Leftrightarrow-x^2<4\Leftrightarrow-x^2-4<0\Leftrightarrow x\in \mathbb{R}$.
	}
\end{ex}
\begin{ex}%[2D2B6-2]
	[Chuyên Sơn La 2019]%Câu 31.
	Tập nghiệm của bất phương trình $2^{x^2-2x}>8$ là
	\choice
	{$(-\infty;-1)$}
	{$(-1; 3)$}
	{$(3;+\infty)$}
	{\True $(-\infty;-1)\cup(3;+\infty)$}
	\loigiai{
		Bất phương trình $2^{x^2-2x}>8\Leftrightarrow 2^{x^2-2x}>2^3\Leftrightarrow x^2-2x>3\Leftrightarrow x^2-2x-3>0\Leftrightarrow\hoac{&x>3\\&x <-1.}$ \\
		Vậy tập nghiệm của bất phương trình là $S=(-\infty;-1)\cup(3;+\infty)$.
	}
\end{ex}
\begin{ex}%[2D2B6-2]
	[Chuyên ĐHSP Hà Nội 2019]%Câu 32.
	Tập nghiệm của bất phương trình $\left(\dfrac{e}{\pi}\right)^x>1$ là
	\choice
	{$\mathbb{R}$}
	{\True $(-\infty; 0)$}
	{$(0;+\infty)$}
	{$[0;+\infty)$}
	\loigiai{
		Vì $\dfrac{e}{\pi}<1$ nên $\left(\dfrac{e}{\pi}\right)^x>1\Leftrightarrow\log_{\frac{e}{\pi}}\left(\dfrac{e}{\pi}\right)^x<\log_{\frac{e}{\pi}}1\Leftrightarrow x<0$.\\
		Vậy tập nghiệm của bất phương trình là $S=(-\infty; 0)$.
	}
\end{ex}
\begin{ex}%[2D2B6-2]
	[Chuyên Lam Sơn Thanh Hóa 2019]%Câu 33.
	Số nghiệm nguyên của bất phương trình $2^{x^2+3x}\leq 16$ là số nào sau đây?
	\choice
	{$5$}
	{\True $6$}
	{$4$}
	{$3$}
	\loigiai{
		$\quad 2^{x^2+3x}\leq 16\Leftrightarrow 2^{x^2+3x}\leq 2^4\Leftrightarrow x^2+3x\leq 4\Leftrightarrow x\in[-4; 1]$.\\
		Các nghiệm nguyên của bất phương trình là $-4;-3;-2;-1;0;1$.
	}
\end{ex}
\begin{ex}%[2D2B6-2]
	[Chuyên Vĩnh Phúc 2019]%Câu 34.
	Tập nghiệm của bất phương trình $\left(\dfrac{1}{1+a^2}\right)^{2x+1}>1$ (với $a$ là tham số, $a\neq 0$) là
	\choice
	{$(-\infty; 0)$}
	{\True $\left(-\infty;-\dfrac{1}{2}\right)$}
	{$(0;+\infty)$}
	{$\left(-\dfrac{1}{2};+\infty\right)$}
	\loigiai{
		Ta có $\left(\dfrac{1}{1+a^2}\right)^{2x+1}>1\Leftrightarrow\left(\dfrac{1}{1+a^2}\right)^{2x+1}>\left(\dfrac{1}{1+a^2}\right)^{\circ} (1)$.\\
		Nhận thấy $1+a^2>1,\forall a\neq 0$ nên: $\dfrac{1}{1+a^2}<1$.\\
		Khi đó bất phương trình $(1)$ tương đương $2x+1<0\Leftrightarrow x <-\dfrac{1}{2}$.\\
		Vậy tập nghiệm của bất phương trình đã cho: $S=\left(-\infty;-\dfrac{1}{2}\right)$.
	}
\end{ex}
\begin{ex}%[2D2B6-2]
	[Cụm 8 Trường Chuyên 2019]%Câu 35.
	Tập nghiệm $S$ của bất phương trình $3^x<\mathrm{e}^x$ là
	\choice
	{$S=\mathbb{R}\setminus\{0\}$}
	{$S=(0;+\infty)$}
	{$S=\mathbb{R}$}
	{\True $S=(-\infty; 0)$}
	\loigiai{
		$3^x<\mathrm{e}^x\Leftrightarrow\left(\dfrac{3}{e}\right)^x<1\Leftrightarrow x<0$. Tập nghiệm của bất phương trình là $S=(-\infty; 0)$.
	}
\end{ex}
\begin{ex}%[2D2B6-2]
	[Nguyễn Huệ - Ninh Bình - 2019]%Câu 36.
	Bất phương trình $2^{x+1}\leq 4$ có tập nghiệm là
	\choice
	{$[1;+\infty)$}
	{$(-\infty; 1)$}
	{$(1;+\infty)$}
	{\True $(-\infty; 1]$}
	\loigiai{
		Ta có $2^{x+1}\leq 4\Leftrightarrow x+1\leq 2\Leftrightarrow x\leq 1$. Tập nghiệm của bất phương trình là $(-\infty; 1]$.
	}
\end{ex}
\begin{ex}%[2D2Y6-1]
	[THPT Minh Khai - 2019]%Câu 37.
	Tìm tập nghiệm $S$ của bất phương trình $3^x<9$
	\choice
	{$S=(-\infty;2]$}
	{$S=(2;+\infty)$}
	{\True $S=(-\infty;2)$}
	{$S=\{2\}$}
	\loigiai{
		$3^x<9\Leftrightarrow 3^x<3^2\Leftrightarrow x<2$.\\
		Tập nghiệm của bất phương trình là: $S=(-\infty;2)$.
	}
\end{ex}
\begin{ex}%[2D2B6-2]
	[Lômônôxốp - Hà Nội 2019]%Câu 38.
	Tập nghiệm của bất phương trình $\left(\dfrac{1}{\sqrt{3}}\right)^{\frac{1}{x}}\leq\left(\dfrac{1}{\sqrt{3}}\right)^2$ là
	\choice
	{$\left(0;\dfrac{1}{2}\right)$}
	{$\left[\dfrac{1}{2};+\infty\right)$}
	{\True $\left(0;\dfrac{1}{2}\right]$}
	{$\left(-\infty;\dfrac{1}{2}\right)$}
	\loigiai{
		Cơ số $a=\dfrac{1}{\sqrt{3}}<1$ nên bất phương trình: $\left(\dfrac{1}{\sqrt{3}}\right)^{\frac{1}{x}}\leq\left(\dfrac{1}{\sqrt{3}}\right)^2\Leftrightarrow\dfrac{1}{x}\geq 2\Leftrightarrow\dfrac{1-2x}{x}\geq 0\Leftrightarrow 0<x\leq\dfrac{1}{2}$.
	}
\end{ex}
\begin{ex}%[2D2B6-2]
	[Đồng Nai - 2019]%Câu 39.
	Tập nghiệm của bất phương trình $3^{x+2}\geq\dfrac{1}{9}$ là
	\choice
	{\True $[-4;+\infty)$}
	{$(-\infty;4)$}
	{$(-\infty;0)$}
	{$[0;+\infty)$}
	\loigiai{
		Bất phương trình $3^{x+2}\geq\dfrac{1}{9}\Leftrightarrow 3^{x+2}\geq 3^{-2}\Leftrightarrow x+2\geq-2\Leftrightarrow x\geq-4$.
	}
\end{ex}
\begin{ex}%[2D2B6-2]
	[Chuyên Long An - 2019]%Câu 40.
	Tìm nghiệm của bất phương trình $\left(\dfrac{1}{2}\right)^{x-1}\geq\dfrac{1}{4}$.
	\choice
	{\True $x\leq 3$}
	{$x>3$}
	{$x\geq 3$}
	{$1<x\leq 3$}
	\loigiai{
		$\left(\dfrac{1}{2}\right)^{x-1}\geq\dfrac{1}{4}\Leftrightarrow\left(\dfrac{1}{2}\right)^{x-1}\geq\left(\dfrac{1}{2}\right)^2\Leftrightarrow x-1\leq 2\Leftrightarrow x\leq 3$.
	}
\end{ex}
\begin{ex}%[2D2B6-2]
	[Chuyên - Vĩnh Phúc - 2019]%Câu 41.
	Tập nghiệm của bất phương trình $\left(\dfrac{1}{1+a^2}\right)^{2x+1}>1$ (với $a$ là tham số, $a\neq 0$) là
	\choice
	{\True $\left(-\infty;-\dfrac{1}{2}\right)$}
	{$(0;+\infty)$}
	{$(-\infty;0)$}
	{$\left(-\dfrac{1}{2};+\infty\right)$}
	\loigiai{
		Ta có $0<\dfrac{1}{1+a^2}<1,\forall a\neq 0$, nếu $\left(\dfrac{1}{1+a^2}\right)^{2x+1}>1\Leftrightarrow 2x+1<0\Leftrightarrow x <-\dfrac{1}{2}\Leftrightarrow x\in\left(-\infty;-\dfrac{1}{2}\right)$.
	}
\end{ex}
\begin{ex}%[2D2B6-2]
	[Chuyên Lam Sơn - 2019]%Câu 42.
	Số nghiệm nguyên của bất phương trình $2^{x^2+3x}\leq 16$ là
	\choice
	{$5$}
	{\True $6$}
	{$4$}
	{$3$}
	\loigiai{
		Ta có $2^{x^2+3x}\leq 16\Leftrightarrow 2^{x^2+3x}\leq 2^4\Leftrightarrow x^2+3x\leq 4\Leftrightarrow x^2+3x-4\leq 0\Leftrightarrow-4\leq x\leq 1$.\\
		Do đó số nghiệm nguyên của bất phương trình đã cho là 6.
	}
\end{ex}
\begin{ex}%[2D2B6-2]
	[Chuyên Hùng Vương Gia Lai 2019]%Câu 43.
	Bất phương trình $\left(\dfrac{1}{2}\right)^{x^2-2x}\geq\dfrac{1}{8}$ có tập nghiệm là
	\choice
	{$[3;+\infty)$}
	{$(-\infty;-1]$}
	{\True $[-1;3]$}
	{$(-1;3)$}
	\loigiai{
		Bất phương trình đã cho tương đương với.\\
		$\left(\dfrac{1}{2}\right)^{x^2-2x}\geq\left(\dfrac{1}{2}\right)^3\Leftrightarrow x^2-2x\leq 3\Leftrightarrow x^2-2x-3\leq 0\Leftrightarrow-1\leq x\leq 3$.\\
		Vậy, tập nghiệm của bất phương trình đã cho là $S=[-1;3]$.
	}
\end{ex}
\begin{ex}%[2D2B6-2]
	[Chuyên Hoàng Văn Thụ-Hòa Bình 2019]%Câu 44.
	Cho bất phương trình $\left(\dfrac{2}{3}\right)^{x^2-x+1}>\left(\dfrac{2}{3}\right)^{2x+1}$ có tập nghiệm $S=(a;b)$. Giá trị của $b-a$ bằng
	\choice
	{\True $3$}
	{$4$}
	{$2$}
	{$1$}
	\loigiai{
		Ta có $\left(\dfrac{2}{3}\right)^{x^2-x+1}>\left(\dfrac{2}{3}\right)^{2x+1}\Leftrightarrow x^2-x+1<2x+1\Leftrightarrow x^2-3x<0\Leftrightarrow 0<x<3$.\\
		Vậy tập nghiệm $S=(0;3)$, suy ra $b-a=3-0=3$.
	}
\end{ex}
\begin{ex}%[2D2B6-2]
	[SGD Hưng Yên 2019]%Câu 45.
	Tập nghiệm của bất phương trình $\left(\dfrac{2}{3}\right)^{2x+1}>1$ là
	\choice
	{$(-\infty;0)$}
	{$(0;+\infty)$}
	{\True $\left(-\infty;-\dfrac{1}{2}\right)$}
	{$\left(-\dfrac{1}{2};+\infty\right)$}
	\loigiai{
		Ta có $\left(\dfrac{2}{3}\right)^{2x+1}>1\Leftrightarrow 2x+1<0\Leftrightarrow x <-\dfrac{1}{2}$.\\
		Vây: Tập nghiệm của bất phương trình là $\left(-\infty;-\dfrac{1}{2}\right)$.
	}
\end{ex}
\begin{ex}%[2D2B6-2]
	[SGD Điện Biên - 2019]%Câu 46
	Tập nghiệm của bất phương trình $2^{\sqrt{x}}<2$ là
	\choice
	{\True $[0 ; 1)$}
	{$(-\infty ; 1)$}
	{$\mathbb{R}$}
	{$(1 ;+\infty)$}
	\loigiai{
		$2^{\sqrt{x}}<2 \Leftrightarrow\left\{\begin{aligned}
			&{x \geq 0}\\
			&{\sqrt{x}< 1}\\
		\end{aligned}\Leftrightarrow \left\{\begin{aligned}
			&x \geq 0 \\
			&x<1\\
		\end{aligned}\Leftrightarrow x \in[0 ; 1)\right.\right.$.
	}
\end{ex}
\begin{ex}%[2D2B6-2]
	[Ngô Quyền - Ba Vì - Hải Phòng 2019]%Câu 47.
	Tập nghiệm $S$ của bất phương trình $\left(\dfrac{1}{2}\right)^{x^2-4x}<8$ là
	\choice
	{$S=(-\infty; 3)$}
	{$S=(1;+\infty)$}
	{\True $S=(-\infty; 1)\cup(3;+\infty)$}
	{$S=(1; 3)$}
	\loigiai{
		Bất phương trình $\left(\dfrac{1}{2}\right)^{x^2-4x}<8\Leftrightarrow\left(\dfrac{1}{2}\right)^{x^2-4x}<\left(\dfrac{1}{2}\right)^{-3}\Leftrightarrow x^2-4x >-3\Leftrightarrow x^2-4x+3>0\Leftrightarrow\hoac{&x>3\\&x<1.}$ \\
		Nên tập nghiệm của bất phương trình $\left(\dfrac{1}{2}\right)^{x^2-4x}<8$ là $S=(-\infty; 1)\cup(3;+\infty)$.
	}
\end{ex}
\begin{ex}%[2D2B6-2]
	[Cần Thơ - 2019]%Câu 48.
	Nghiệm của bất phương trình $2^{x^2-x}\leq 4$ là
	\choice
	{\True $-1\leq x\leq 2$}
	{$x\leq 1$}
	{$x\leq 2$}
	{$-2\leq x\leq 1$}
	\loigiai{
		$2^{x^2-x}\leq 4\Leftrightarrow 2^{x^2-x}\leq 2^2\Leftrightarrow x^2-x-2\leq 0\Leftrightarrow-1\leq x\leq 2$.
	}
\end{ex}
\begin{ex}%[2D2B6-2]
	[Chuyên Lê Quý Đôn Điện Biên 2019]%Câu 49.
	Tìm tập nghiệm của bất phương trình $2^x+2^{x+1}\leq 3^x+3^{x-1}$.
	\choice
	{$(2;+\infty)$}
	{$(-\infty;2)$}
	{$(-\infty;2]$}
	{\True $[2;+\infty)$}
	\loigiai{
		Ta có $2^x+2^{x+1}\leq 3^x+3^{x-1}\Leftrightarrow 3\cdot 2^x\leq 4\cdot 3^{x-1}\Leftrightarrow 2^{x-2}\leq 3^{x-2}$ \\
		$ \Leftrightarrow\left(\dfrac{2}{3}\right)^{x-2}\leq 1\Leftrightarrow x-2\geq 0\Leftrightarrow x\geq 2 $.
	}
\end{ex}
\begin{ex}%[2D2B6-3]
	[Chuyên Lê Hồng Phong Nam Định 2019]%Câu 50.
	Cho bất phương trình $4^x-5\cdot 2^{x+1}+16\leq 0$ có tập nghiệm là đoạn $[a;b]$. Tính $\log\left(a^2+b^2\right)$
	\choice
	{$2$}
	{\True $1$}
	{$0$}
	{$10$}
	\loigiai{
		Đặt $t=2^x, t>0(*)$.\\
		Khi đó bất phương trình đã cho trở thành: $t^2-10t+16\leq 0\Leftrightarrow 2\leq t\leq 8$ (thỏa mãn (*))\\
		$ \Rightarrow 2\leq 2^x\leq 2^3\Leftrightarrow 1\leq x\leq 3\Rightarrow\heva{&a=1\\&b=3}\Rightarrow\log\left(a^2+b^2\right)=1 $.
	}
\end{ex}
\begin{ex}%[2D2B6-2]
	[Thi thử cụm Vũng Tàu - 2019]%Câu 51.
	Cho bất phương trình $\left(\dfrac{2}{3}\right)^{x^2-x+1}>\left(\dfrac{2}{3}\right)^{2x-1}$ có tập nghiệm $S=(a; b)$. Giá trị của $b-a$ bằng
	\choice
	{$-2$}
	{$-1$}
	{\True $1$}
	{$2$}
	\loigiai{
		$\left(\dfrac{2}{3}\right)^{x^2-x+1}>\left(\dfrac{2}{3}\right)^{2x-1}\Leftrightarrow x^2-x+1<2x-1\Leftrightarrow x^2-3x+2<0\Leftrightarrow 1<x<2\Rightarrow S=(1; 2)$.\\
		Vậy $a=1;b=2\Rightarrow b-a=1$.
	}
\end{ex}
\begin{ex}%[2D2B6-2]
	[Chuyên Quốc Học Huế 2019]%Câu 52.
	Xác định tập nghiệm $S$ của bất phương trình $\left(\dfrac{1}{3}\right)^{2x-3}\geq 3$.
	\choice
	{\True $S=(-\infty;1]$}
	{$S=(1;+\infty)$}
	{$S=[1;+\infty)$}
	{$S=(-\infty;1)$}
	\loigiai{
		Ta có $\left(\dfrac{1}{3}\right)^{2x-3}\geq 3\Leftrightarrow 3^{3-2x}\geq 3^1\Leftrightarrow 3-2x\geq 1\Leftrightarrow x\leq 1$.\\
		Vậy tập nghiệm của bất phương trình là $S=(-\infty;1]$.
	}
\end{ex}
\begin{ex}%[2D2B6-2]
	[Sở Hà Nam - 2019]%Câu 53.
	Tập nghiệm của bất phương trình $(5)^{4+x^2}<\left(\dfrac{1}{5}\right)^{x^2-6x}$ là
	\choice
	{$(-\infty; 1)\cup(2;+\infty)$}
	{$(2;+\infty)$}
	{$(-\infty; 1)$}
	{\True $(1; 2)$}
	\loigiai{
		Ta có $(5)^{4+x^2}<\left(\dfrac{1}{5}\right)^{x^2-6x}\Leftrightarrow 5^{4+x^2}<5^{-x^2+6x}\Leftrightarrow-x^2+6x>4+x^2\Leftrightarrow 2x^2-6x+4<0$ \\
		$ \Leftrightarrow 1<x<2 $.
	}
\end{ex}
\begin{ex}%[2D2B6-2]
	[Chu Văn An - Hà Nội - 2019]%Câu 54.
	Bất phương trình $\left(\dfrac{\pi}{2}\right)^{x-1}\leq\left(\dfrac{\pi}{2}\right)^{2x+3}$ có nghiệm là
	\choice
	{$x\leq-4$}
	{$x >-4$}
	{$x <-4$}
	{\True $x\geq-4$}
	\loigiai{
		Ta có\\
		$\begin{aligned}&\left(\dfrac{\pi}{2}\right)^{x-1}\leq\left(\dfrac{\pi}{2}\right)^{2x+3}\\&\Leftrightarrow x-1\leq 2x+3\\&\Leftrightarrow x\geq-4\end{aligned}$ (vì $\dfrac{\pi}{2}>1$).
	}
\end{ex}
\begin{ex}%[2D2Y6-1]
	[Đề minh họa 2022]%Câu 55.
	Tập nghiệm của bất phương trình $2^x>6$ là
	\choice
	{\True $\left(\log_26;+\infty\right)$}
	{$(-\infty;3)$}
	{$(3;+\infty)$}
	{$\left(-\infty;\log_26\right)$}
	\loigiai{
		$2^x>6\Leftrightarrow x>\log_26$.\\
		Tập nghiệm của bất phương trình là $S=\left(\log_26;+\infty\right)$.
	}
\end{ex}

\Closesolutionfile{ans}
\indapan{10}{ans/CD20/Muc_5_6}

