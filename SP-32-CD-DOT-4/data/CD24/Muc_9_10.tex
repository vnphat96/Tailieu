\Opensolutionfile{ans}[ans/CD24/Muc_9_10]
\setcounter{ex}{0}
\setcounter{dang}{0}
\section{Mức độ 9,10 điểm}
\begin{dang}
	{Tài liệu dành cho đối tượng học sinh - khá giỏi}
\end{dang}
\begin{ex}
	[Mã 102 - 2020 Lần 2]%Câu 1
	Cho hình nón $(N)$ có đỉnh $S$, bán kính đáy bằng $\sqrt{3}a$ và độ dài đường sinh bằng $4a$. Gọi $(T)$ là mặt cầu đi qua $S$ và đường tròn đáy của $(N)$. Bán kính của $(T)$ bằng
	\choice
	{$\dfrac{2\sqrt{10}a}{3}$}
	{$\dfrac{16\sqrt{13}a}{13}$}
	{\True $\dfrac{8\sqrt{13}a}{13}$}
	{$\sqrt{13}a$}
	\loigiai{}{
		Cách 1.\\
		{\color{red}HÌNH Ở ĐÂY}\\
		Nếu cắt mặt cầu ngoại tiếp khối nón $(N)$ bởi mặt phẳng $\left(SAB\right)$, ta được mộ hình tròn ngoại tiếp tam giác $SAB$. Khi đó bán kính mặt cầu $(T)$ bằng bán kính đường tròn ngoại tiếp tam giác $SAB$.\\
		Gọi $M$ là trung điểm của $SB$. Kẻ đường vuông góc với $SB$ tại $M$, cắt $SO$ tại $I$.\\
		Khi đó $I$ là tâm đường tròn ngoại tiếp $\Delta SAB$ và $r=SI$ là bán kính đường tròn ngoại tiếp $\Delta SAB$.\\
		Ta có $\Delta SIM\backsim\Delta SBO\Rightarrow\dfrac{SI}{SB}=\dfrac{SM}{SO}\Rightarrow SI=\dfrac{SM}{SO}.SB$.\\
		Trong đó $\left\{\begin{aligned}
			& SM=2a\\ 
			& SB=4a\\ 
			& SO=\sqrt{S{B^2}-O{B^2}}=a\sqrt{13}\\ 
		\end{aligned}\right.\Rightarrow r=SI=\dfrac{8a\sqrt{13}}{13}$.\\
		Cách 2.\\
		Gọi $O$ là tâm của mặt cầu $(T)$, $H$ là tâm đường tròn đáy của $(N)$, $M$ là một điểm trên đường tròn đáy của $(N)$ và $R$ là bán kính của $(T)$.\\
		Ta có $SO=OM=R$; $O{M^2}=O{H^2}+H{M^2}$; $SH=\sqrt{S{M^2}-H{M^2}}=\sqrt{13}a$.\\
		Do $SH\ne HM$ nên chỉ xảy ra hai trường hợp sau\\
		Trường hợp 1: $SH=SO+OH$\\
		{\color{red}HÌNH Ở ĐÂY}\\
		Ta có hệ phương trình\\
		$\left\{\begin{aligned}
			& R+OH=\sqrt{13}a\\ 
			&{R^2}=O{H^2}+3a^2\\ 
		\end{aligned}\right.\Leftrightarrow\left\{\begin{aligned}
			& OH=\sqrt{13}a-R\\ 
			&{R^2}=13a^2-2\sqrt{3}aR+R^2+3a^2(*)\\ 
		\end{aligned}\right.$.\\
		Giải $(*)$ ta có $R=\dfrac{8\sqrt{13}a}{13}$.\\
		Trường hợp 2: $SH=SO-OH$.\\
		{\color{red}HÌNH Ở ĐÂY}\\
		Ta có hệ phương trình $\left\{\begin{aligned}
			& R=OH+\sqrt{13}a\\ 
			&{R^2}=O{H^2}+3a^2\\ 
		\end{aligned}\right.\Leftrightarrow\left\{\begin{aligned}
			& OH=R-\sqrt{13}a\\ 
			&{R^2}=13a^2-2\sqrt{13}aR+R^2+3a^2(*)\\ 
		\end{aligned}\right.$.\\
		Giải $(*)$ ta có $R=\dfrac{8\sqrt{13}a}{13}$.
	}
\end{ex}
\begin{ex}[Mã 103 - 2020 Lần 2]%Câu 2
	Cho hình nón $(N)$ có đỉnh $S$, bán kính đáy bằng $a$ và độ dài đường sinh bằng $4a$. Gọi $(T)$ là mặt cầu đi qua $S$ và đường tròn đáy của $(N)$. Bán kính của $(T)$ bằng
	\choice
	{$\dfrac{2\sqrt{6}a}{3}$}
	{$\dfrac{16\sqrt{15}a}{15}$}
	{\True $\dfrac{8\sqrt{15}a}{15}$}
	{$\sqrt{15}a$}
	\loigiai{
		{\color{red}HÌNH Ở ĐÂY}\\
		Gọi $I$ là tâm của $(T)$ thì $I\in SO$ và $IS=IA$. Gọi $M$ là trung điểm của $SA$ thì $IM\perp SA$.\\
		Ta có $SO=\sqrt{S{A^2}-O{A^2}}=\sqrt{\left(4a\right)^2-a^2}=a\sqrt{15}$.\\
		Lại có $SM.SA=SI.SO\Rightarrow SI=\dfrac{SM.SA}{SO}=\dfrac{2a.4a}{a\sqrt{15}}=\dfrac{8\sqrt{15}a}{15}$.
	}
\end{ex}
\begin{ex}
	[Mã 101 - 2020 Lần 2]%Câu 3
	Cho hình nón $(N)$ có đỉnh $S$, bán kính đáy bằng $\sqrt{2}a$ và độ dài đường sinh bằng $4a$. Gọi $(T)$ là mặt cầu đi qua S và đường tròn đáy của $(N)$. Bán kính của $(T)$ bằng
	\choice
	{$\dfrac{4\sqrt{2}}{3}a$}
	{$\sqrt{14}a$}
	{\True $\dfrac{4\sqrt{14}}{7}a$}
	{$\dfrac{8\sqrt{14}}{7}a$}
	\loigiai{
		{\color{red}HÌNH Ở ĐÂY}\\
		Gọi $R$ là bán kính mặt cầu $(T)$, $SH$ là đường cao của hình nón $\Rightarrow SH=\sqrt{\left(4a\right)^2-\left(a\sqrt{2}\right)^2}=a\sqrt{14}$\\
		Gọi $~I$ là tâm mặt cầu $\Rightarrow{R^2}=\left(a\sqrt{2}\right)^2+\left(R-a\sqrt{14}\right)^2$ $\Rightarrow R=\dfrac{4\sqrt{14}}{7}a$.
	}
\end{ex}
\begin{ex}
	[Mã 104 - 2020 Lần 2]%Câu 4
	Cho hình nón $(N)$ có đỉnh $S$, bán kính đáy bằng $a$ và độ dài đường sinh bằng $2\sqrt{2}a$. Gọi $(T)$ là mặt cầu đi qua $S$ và đường tròn đáy của $(N)$. Bán kính của $(T)$ bằng
	\choice
	{\True $\dfrac{4\sqrt{7}a}{7}$}
	{$\dfrac{4a}{3}$}
	{$\dfrac{8\sqrt{7}a}{7}$}
	{$\sqrt{7}a$}
	\loigiai{
		Giả sử thiết diện qua trục của hình nón là tam giác $SAB$ cân tại $S$.\\
		Khi đó ta có $S_{SAB}=\dfrac{1}{2}SH.AB=\dfrac{1}{2}\left(a\sqrt{7}\right).2a=\sqrt{7}{a^2}$.\\
		Ta có $S_{SAB}=\dfrac{SA.SB.AB}{4R}\Rightarrow R=\dfrac{SA.SB.SC}{4S_{SAB}}=\dfrac{2\sqrt{2}a.2\sqrt{2}a.2a}{4.a^2\sqrt{7}}=\dfrac{4a\sqrt{7}}{7}$.
	}
\end{ex}
\begin{ex}
	[THPT Bạch Đằng Quảng Ninh 2019]%Câu 5
	Cho hình thang $ABCD$ vuông tại $A$ và $B$ với $AB=BC=\dfrac{AD}{2}=a$. Quay hình thang và miền trong của nó quanh đường thẳng chứa cạnh $BC$. Tính thể tích $V$ của khối tròn xoay được tạo thành.\\
	{\color{red}HÌNH Ở ĐÂY}
	\choice
	{$V=\dfrac{4\pi{a^3}}{3}$}
	{\True $V=\dfrac{5\pi{a^3}}{3}$}
	{$V=\pi{a^3}$}
	{$V=\dfrac{7\pi{a^3}}{3}$}
	\loigiai{
		{\color{red}HÌNH Ở ĐÂY}\\
		Thể tích của khối trụ sinh bởi hình chữ nhật $ABID$ khi quay cạnh $BI$ là\\
		$V_1=\pi .A{B^2}.AD=2\pi{a^3}$.\\
		Thể tích của khối nón sinh bởi tam giác $CID$ khi quay cạnh $CI$ là:\\
		$V_2=\dfrac{1}{3}\pi .I{D^2}.CI=\dfrac{\pi{a^3}}{3}$.\\
		Vậy $V=V_1-V_2=\dfrac{5\pi{a^3}}{3}$.
	}
\end{ex}
\begin{ex}
	[THPT Gia Lộc Hải Dương 2019]%Câu 6
	Một hình nón có chiều cao $9\left(cm\right)$ nội tiếp trong một hình cầu có bán kính $5\left(cm\right)$. Gọi $V_1$, $V_2$ lần lượt là thể tích của khối nón và khối cầu. Tính tỉ số $\dfrac{V_1}{V_2}$.
	\choice
	{$\dfrac{81}{125}$}
	{\True $\dfrac{81}{500}$}
	{$\dfrac{27}{125}$}
	{$\dfrac{27}{500}$}
	\loigiai{
		{\color{red}HÌNH Ở ĐÂY}\\
		Gọi hình cầu có tâm $O$ bán kính $R$.\\
		Gọi hình nón có đỉnh $S$, tâm đáy là $H$, bán kính đáy $r=HA$.\\
		Vì hình nón nội tiếp hình cầu nên đỉnh S thuộc hình cầu, chiều cao SH của hình nón đi qua tâm O của hình cầu, đồng thời cắt hình cầu tại điểm $S'$.\\
		Theo đề chiều cao hình nón $SH=9$, bán kính hình cầu $OS=5\Rightarrow OH=4$, từ đó ta có $HA=\sqrt{O{A^2}-O{H^2}}=\sqrt{5^2-4^2}=3$.\\
		Thể tích khối nón $V_1=\dfrac{1}{3}h\pi{r^2}=\dfrac{1}{3}SH.\pi .H{A^2}=\dfrac{1}{3}.9\pi{3^2}=27\pi $.\\
		Thể tích khối cầu $V_2=\dfrac{4}{3}\pi{R^3}=\dfrac{4}{3}\pi{5^3}=\dfrac{500\pi}{3}$.\\
		Tỉ số $\dfrac{V_1}{V_2}=\dfrac{27\pi}{\dfrac{500\pi}{3}}=\dfrac{81}{500}$.
	}
\end{ex}
\begin{ex}
	[Sở Ninh Bình 2019]%Câu 7
	Một khối gỗ hình trụ tròn xoay có bán kính đáy bằng $1$, chiều cao bằng $2$. Người ta khoét từ hai đầu khối gỗ hai nửa khối cầu mà đường tròn đáy của khối gỗ là đường tròn lớn của mỗi nửa khối cầu. Tỉ số thể tích phần còn lại của khối gỗ và cả khối gỗ ban đầu là
	\choice
	{$\dfrac{2}{3}$}
	{$\dfrac{1}{4}$}
	{\True $\dfrac{1}{3}$}
	{$\dfrac{1}{2}$}
	\loigiai{
		Theo bài toán ta có hình vẽ\\
		{\color{red}HÌNH Ở ĐÂY}\\
		Thể tích của khối trụ là $V=\pi{1^2}.2=2\pi $.\\
		Vì đường tròn đáy của khối trụ là đường tròn lớn của mỗi nửa khối cầu nên bán kính của mỗi nửa khối cầu là $R=1$.\\
		Thể tích của hai nửa khối cầu bị khoét đi là $V_1=2\cdot\dfrac{1}{2}\cdot\dfrac{4\pi{1^3}}{3}=\dfrac{4\pi}{3}$.\\
		Thể tích của phần còn lại của khối gỗ là $V_2=V-V_1=2\pi-\dfrac{4\pi}{3}=\dfrac{2\pi}{3}$.\\
		Vậy tỉ số thể tích cần tìm là $\dfrac{V_2}{V}=\dfrac{\dfrac{2\pi}{3}}{2\pi}=\dfrac{1}{3}$.
	}
\end{ex}
\begin{ex}
	[Chuyên Lam Sơn Thanh Hóa 2019]%Câu 8
	Một khối trụ bán kính đáy là $a\sqrt{3}$, chiều cao là $2a\sqrt{3}$. Tính thể tích khối cầu ngoại tiếp khối trụ.\\
	{\color{red}HÌNH Ở ĐÂY}
	\choice
	{\True $8\sqrt{6}\pi{a^3}$}
	{$6\sqrt{6}\pi{a^3}$}
	{$4\sqrt{3}\pi{a^3}$}
	{$\dfrac{4\sqrt{6}}{3}\pi{a^3}$}
	\loigiai{
		{\color{red}HÌNH Ở ĐÂY}\\
		Xét hình hình chữ nhật $OABO'$ như hình vẽ, với $O$, $O'$ lần lượt là tâm hai đáy của khối trụ. Gọi $I$ là trung điểm đoạn thẳng $OO'$. Khi đó $IA$ là bán kính khối cầu ngoại tiếp khối trụ.\\
		Ta có $I{A^2}=O{A^2}+O{I^2}=3a^2+3a^2=6a^2\Rightarrow IA=\sqrt{6}a$.\\
		Thể tích khối cầu ngoại tiếp khối trụ là $V=\dfrac{4}{3}\pi{\left(\sqrt{6}a\right)^3}=8\sqrt{6}\pi{a^3}$.
	}
\end{ex}
\begin{ex}
	[THPT Chuyên Thái Nguyên 2019]%Câu 9
	Một khối cầu pha lê gồm một hình cầu $\left(H_1\right)$ bán kính $R$ và một hình nón $\left(H_2\right)$ có bán kính đáy và đường sinh lần lượt là $r$, $l$ thỏa mãn $r=\dfrac{1}{2}l$ và $l=\dfrac{3}{2}R$ xếp chồng lên nhau (hình vẽ). Biết tổng diện tích mặt cầu $\left(H_1\right)$ và diện tích toàn phần của hình nón $\left(H_2\right)$ là $91\mathrm{cm}^2$. Tính diện tích của mặt cầu $\left(H_1\right)$\\
	{\color{red}HÌNH Ở ĐÂY}
	\choice
	{$\dfrac{104}{5}\mathrm{cm}^2$}
	{$16\mathrm{cm}^2$}
	{\True $64\mathrm{cm}^2$}
	{$\dfrac{26}{5}\mathrm{cm}^2$}
	\loigiai{
		$r=\dfrac{1}{2}l=\dfrac{1}{2}.\dfrac{3}{2}R=\dfrac{3}{4}R$. Diện tích mặt cầu $S_1=4\pi{R^2}$\\
		Diện tích toàn phần của hình nón $S_2=\pi rl+\pi{r^2}=\pi .\dfrac{3}{4}R.\dfrac{3}{2}R+\pi .\dfrac{9}{16}{R^2}=\dfrac{27\pi{R^2}}{16}$\\
		Theo giả thiết: $4\pi{R^2}+\dfrac{27\pi{R^2}}{16}=91\Leftrightarrow\dfrac{91\pi{R^2}}{16}=91\Leftrightarrow\pi{R^2}=16$\\
		Vậy $S_1=4\pi{R^2}=64\mathrm{cm}^2$.
	}
\end{ex}
\begin{ex}
	[KTNL GV Thuận Thành 2 Bắc Ninh 2019]%Câu 10
	Cho hình thang cân $ABCD$ có đáy nhỏ $AB=1$, đáy lớn $CD=3$, cạnh bên $BC=DA=\sqrt{2}$. Cho hình thang đó quay quanh $AB$ thì được vật tròn xoay có thể tích bằng
	\choice
	{$\dfrac{5}{3}\pi $}
	{$\dfrac{4}{3}\pi $}
	{\True $\dfrac{7}{3}\pi $}
	{$\dfrac{2}{3}\pi $}
	\loigiai{
		{\color{red}HÌNH Ở ĐÂY}\\
		Thể tích của khối tròn xoay bằng thể tích của hình trụ đường cao $DC$ và bán kính đường tròn đáy $AH$. $AH=DH=1$\\
		Trừ đi thể tích hai khối nòn tròn xoay chiều cao $DH$ bán kính đường tròn đáy $AH$\\
		Ta có thể tích khối tròn xoay cần tìm là:\\
		$V=3.\pi{1^2}-2.\dfrac{1}{3}.1.\pi{1^2}=\dfrac{7}{3}\pi $.
	}
\end{ex}
\begin{ex}
	[Sở Thanh Hóa 2019]%Câu 11
	Một hộp đựng mỹ phẩm được thiết kế (tham khảo hình vẽ) có thân hộp là hình trụ có bán kính hình tròn đáy $r=5\mathrm{cm}$, chiều cao $h=6\mathrm{cm}$ và nắp hộp là một nửa hình cầu. Người ta cần sơn mặt ngoài của cái hộp đó (không sơn đáy) thì diện tích $S$ cần sơn là\\
	{\color{red}HÌNH Ở ĐÂY}
	\choice
	{\True $S=110\pi $ $\mathrm{cm}^2$}
	{$S=130\pi $ $\mathrm{cm}^2$}
	{$S=160\pi $ $\mathrm{cm}^2$}
	{$S=80\pi $ $\mathrm{cm}^2$}
	\loigiai{
		Diện tích nắp hộp cần sơn là $S_1=\dfrac{4\pi{r^2}}{2}=50\pi $ $\mathrm{cm}^2$.\\
		Diện tích than hộp cần sơn là $S_2=2\pi rh=60\pi $ $\mathrm{cm}^2$.\\
		Diện tích $S$ cần sơn là $S=S_1+S_2=50\pi+60\pi=110\pi $ $\mathrm{cm}^2$.
	}
\end{ex}
\begin{ex}
	[Sở Bình Phước 2019]%Câu 12
	Một đồ vật được thiết kế bởi một nửa khối cầu và một khối nón úp vào nhau sao cho đáy của khối nón và thiết diện của nửa mặt cầu chồng khít lên nhau như hình vẽ bên. Biết khối nón có đường cao gấp đôi bán kính đáy, thể tích của toàn bộ khối đồ vật bằng $36\pi\mathrm{cm}^3.$ Diện tích bề mặt của toàn bộ đồ vật đó bằng\\
	{\color{red}HÌNH Ở ĐÂY}
	\choice
	{$\pi\left(\sqrt{5}+3\right)\mathrm{cm}^2$}
	{\True $9\pi\left(\sqrt{5}+2\right)\mathrm{cm}^2$}
	{$9\pi\left(\sqrt{5}+3\right)\mathrm{cm}^2$}
	{$\pi\left(\sqrt{5}+2\right)\mathrm{cm}^2$}
	\loigiai{
		Thể tích khối nón là $V_1=\dfrac{1}{3}\pi .R^2.2R=\dfrac{2}{3}\pi .R^3$\\
		Thể tích nửa khối cầu là $V_2=\dfrac{1}{2}.\dfrac{4}{3}\pi .R^3=\dfrac{2}{3}\pi .R^3$\\
		Thể tích của toàn bộ khối đồ vật là $V_1+V_2=36\pi $ $\Leftrightarrow\dfrac{4}{3}\pi .R^3=36\pi\Leftrightarrow R=3$\\
		Diện tích xung quanh của mặt nón là $S_1=\pi R.\sqrt{4R^2+R^2}=\pi{R^2}\sqrt{5}=9\sqrt{5}\pi $\\
		Diện tích của nửa mặt cầu là $S_2=\dfrac{1}{2}.4\pi{R^2}=18\pi $\\
		Diện tích bề mặt của toàn bộ đồ vật bằng $S_1+S_2=9\pi\left(\sqrt{5}+2\right)\mathrm{cm}^2$.
	}
\end{ex}
\begin{ex}
	[Sở Hà Nội 2019]%Câu 13
	Cho khối cầu $(S)$ có bán kính $R$. Một khối trụ có thể tích bằng $\dfrac{4\pi\sqrt{3}}{9}{R^3}$ và nội tiếp khối cầu $(S)$. Chiều cao của khối trụ bằng
	\choice
	{$\dfrac{\sqrt{3}}{3}R$}
	{$R\sqrt{2}$}
	{$\dfrac{\sqrt{2}}{2}R$}
	{\True $\dfrac{2\sqrt{3}}{3}R$}
	\loigiai{
		{\color{red}HÌNH Ở ĐÂY}\\
		Gọi $r$ là bán kính của khối trụ và $h$ là chiều cao của khối tru, khi đó ta có $r^2=R^2-\left(\dfrac{h}{2}\right)^2=R^2-\dfrac{h^2}{4}$.\\
		Thể tích của khối trụ là $V=\pi{r^2}h=\pi\left(R^2-\dfrac{h^2}{4}\right)h$.\\
		Theo đề bài thể tích khối trụ bằng $\dfrac{4\pi\sqrt{3}}{9}{R^3}$ nên ta có phương trình\\
		$\dfrac{4\pi\sqrt{3}}{9}{R^3}=\pi\left(R^2-\dfrac{h^2}{4}\right)h$ $\Leftrightarrow 9h^3-36R^2h+16\sqrt{3}{R^3}=0$ $\Leftrightarrow 9\left(\dfrac{h}{R}\right)^3-36\left(\dfrac{h}{R}\right)+16\sqrt{3}=0$ $\Rightarrow\dfrac{h}{R}=\dfrac{2\sqrt{3}}{3}\Leftrightarrow h=\dfrac{2\sqrt{3}}{3}R$.\\
		Vậy chiều cao khối trụ là $h=\dfrac{2\sqrt{3}}{3}R$.
	}
\end{ex}
\begin{ex}
	Tính thể tích của vật thể tròn xoay khi quay mô hình (như hình vẽ) quanh trục $DF$.\\
	{\color{red}HÌNH Ở ĐÂY}
	\choice
	{$\dfrac{10\pi}{7}{a^3}$}
	{$\dfrac{\pi}{3}{a^3}$}
	{$\dfrac{5\pi}{2}{a^3}$}
	{\True $\dfrac{10\pi}{9}{a^3}$}
	\loigiai{
		Khi quay mô hình trên quanh trục $DF$. Tam giác $AFE$ tạo ra khối nón tròn xoay $(N)$ và hình vuông $ABCD$ tạo ra khối trụ tròn xoay $(T)$.\\
		$(N)$ có chiều cao $AF=a,$ bán kính đáy $EF=AF.\tan{30^0}=\dfrac{a}{\sqrt{3}}\Rightarrow{V_{(N)}}=\dfrac{1}{3}\pi{\left(\dfrac{a}{\sqrt{3}}\right)^2}a=\dfrac{\pi{a^3}}{9}$.\\
		$(T)$ có chiều cao $AD=a,$ bán kính đáy $AB=a\Rightarrow{V_{(T)}}=\pi{a^2}.a=\pi{a^3}$.\\
		Vậy thể tích cần tính là: $V=V_{(N)}+V_{(T)}=\pi{a^3}+\dfrac{\pi{a^3}}{9}=\dfrac{10\pi{a^3}}{9}$.
	}
\end{ex}
\begin{ex}
	[Sở Ninh Bình 2019]%Câu 15
	Cho mặt cầu $(S)$ tâm $O$, bán kính bằng 2. $(P)$ là mặt phẳng cách $O$ một khoảng bằng 1 và cắt $(S)$ theo một đường tròn $(C)$. Hình nón $(N)$ có đáy là $(C)$, đỉnh thuộc $(S)$, đỉnh cách $(P)$ một khoảng lớn hơn $2$. Kí hiệu $V_1$, $V_2$ lần lượt là thể tích của khối cầu $(S)$ và khối nón $(N)$. Tỉ số $\dfrac{V_1}{V_2}$ là
	\choice
	{$\dfrac{1}{3}$}
	{$\dfrac{2}{3}$}
	{$\dfrac{16}{9}$}
	{\True $\dfrac{32}{9}$}
	\loigiai{
		{\color{red}HÌNH Ở ĐÂY}\\
		Thể tích khối cầu $(S)$ là $V_1=\dfrac{4}{3}\pi .R^3=\dfrac{4}{3}\pi{2^3}=\dfrac{32}{3}\pi $\\
		Khối nón $(N)$ có bán kính đáy $r=\sqrt{2^2-1^2}=\sqrt{3}$, chiều cao $h=3$\\
		Thể tích khối nón $(N)$ là $V_2=\dfrac{1}{3}\pi{r^2}h=\dfrac{1}{3}\pi .\left(\sqrt{3}\right)^2.3=3\pi $. Do đó $\dfrac{V_1}{V_2}=\dfrac{32}{9}$.
	}
\end{ex}
\begin{ex}
	[Mã 104 2017]%Câu 16
	Cho mặt cầu $(S)$ tâm $O$, bán kính $R=3$. Mặt phẳng $(P)$ cách $O$ một khoảng bằng $1$ và cắt $(S)$ theo giao tuyến là đường tròn $(C)$ có tâm $H$. Gọi $T$ là giao điểm của tia $HO$ với $(S)$, tính thể tích $V$ của khối nón có đỉnh $T$ và đáy là hình tròn $(C)$.
	\choice
	{\True $V=\dfrac{32\pi}{3}$}
	{$V=16\pi $}
	{$V=\dfrac{16\pi}{3}$}
	{$V=32\pi $}
	\loigiai{
		{\color{red}HÌNH Ở ĐÂY}\\
		Gọi $r$ là bán kính đường tròn $(C)$ thì $r$ là bán kính đáy của hình nón ta có $r^2=R^2-O{H^2}=8$; $HT=HO+OT=1+3=4=h$ là chiều cao của hình nón.\\
		Suy ra: $V_n=\dfrac{1}{3}.h.S_{(C)}=\dfrac{1}{3}.4.\pi .8=\dfrac{32\pi}{3}$.
	}
\end{ex}
\begin{ex}
	Một hình trụ có hai đường tròn đáy nằm trên một mặt cầu bán kính $R$ và có đường cao bằng bán kính mặt cầu. Diện tích toàn phần của hình trụ đó bằng
	\choice
	{$\dfrac{\left(3+2\sqrt{3}\right)\pi{R^2}}{3}$}
	{\True $\dfrac{\left(3+2\sqrt{3}\right)\pi{R^2}}{2}$}
	{$\dfrac{\left(3+2\sqrt{2}\right)\pi{R^2}}{2}$}
	{$\dfrac{\left(3+2\sqrt{2}\right)\pi{R^2}}{3}$}
	\loigiai{
		{\color{red}HÌNH Ở ĐÂY}\\
		Đường cao hình trụ $h=R$ nên ta có bán kính của đáy hình trụ $r=\sqrt{R^2-\dfrac{R^2}{4}}=\dfrac{R\sqrt{3}}{2}$.\\
		$S_{\mathrm{xq}}=2\pi rh=2\pi\dfrac{R\sqrt{3}}{2}R=\pi{R^2}\sqrt{3}$.\\
		Vậy $S_{\mathrm{tp}}=S_{\mathrm{xq}}+2S_{\acute{a}y}=\pi{R^2}\sqrt{3}+2\pi{\left(\dfrac{R\sqrt{3}}{2}\right)^2}=\dfrac{\left(3+2\sqrt{3}\right)\pi{R^2}}{2}$.
	}
\end{ex}
\begin{ex}
	[Mã 110 2017]%Câu 18
	Cho mặt cầu $(S)$ có bán kính bằng $4$, hình trụ $(H)$ có chiều cao bằng $4$ và hai đường tròn đáy nằm trên $(S)$. Gọi $V_1$ là thể tích của khối trụ $(H)$ và $V_2$ là thể tích của khối cầu $(S)$. Tính tỉ số $\dfrac{V_1}{V_2}$ 
	\choice
	{\True $\dfrac{V_1}{V_2}=\dfrac{9}{16}$}
	{$\dfrac{V_1}{V_2}=\dfrac{2}{3}$}
	{$\dfrac{V_1}{V_2}=\dfrac{1}{3}$}
	{$\dfrac{V_1}{V_2}=\dfrac{3}{16}$}
	\loigiai{
		{\color{red}HÌNH Ở ĐÂY}\\
		Ta có $r=\sqrt{4^2-2^2}=2\sqrt{3}$. Thể tích của khối trụ $(H)$ là $V_1=\pi{r^2}h=\pi .12.4=48\pi $.\\
		Thể tích của khối cầu $(S)$ là $V_2=\dfrac{4}{3}\pi{R^3}=\dfrac{4}{3}\pi{4^3}=\dfrac{256\pi}{3}$. Vậy $\dfrac{V_1}{V_2}=\dfrac{9}{16}$.
	}
\end{ex}
\begin{ex}
	[Chuyên Lam Sơn Thanh Hóa 2019]%Câu 19
	Cho tam giác đều $ABC$ nội tiếp đường tròn tâm $I$ đường kính $A{A}'$, $M$ là trung điểm $BC$. Khi quay tam, giác $ABM$ với nữa hình tròn đường kính $A{A}'$ xung quanh đường thẳng $AM$ (như hình vẽ minh hoạ), ta được khối nón và khối cầu có thể tích lần lượt $V_1,V_2$.Tỉ số $\dfrac{V_1}{V_2}$ bằng:\\
	{\color{red}HÌNH Ở ĐÂY}
	\choice
	{$\dfrac{9}{4}$}
	{$\dfrac{4}{9}$}
	{$\dfrac{27}{32}$}
	{\True $\dfrac{9}{32}$}
	\loigiai{
		{\color{red}HÌNH Ở ĐÂY}\\
		Gọi tam giác đều cạnh $a$. Ta có\\
		${r}=\dfrac{a}{2}$ là bán kính đường tròn đáy của khối nón.\\
		${R}=\dfrac{2}{3}.\left(\dfrac{a\sqrt{3}}{2}\right)=\dfrac{a\sqrt{3}}{3}$ là bán kính khối cầu.\\
		$\begin{aligned}
			&{V_1}=\dfrac{1}{3}\pi{r^2}h=\dfrac{1}{3}\pi{\left(\dfrac{a}{2}\right)^2}.\left(\dfrac{a\sqrt{3}}{2}\right)=\dfrac{\sqrt{3}\pi{a^3}}{24}\\ 
			&{V_2}=\dfrac{4}{3}\pi{R^3}=\dfrac{4}{3}\pi{\left(\dfrac{a\sqrt{3}}{3}\right)^3}=\dfrac{4\sqrt{3}\pi{a^3}}{27}\\ 
			&\Rightarrow\dfrac{V_1}{V_2}=\dfrac{9}{32}\\ 
		\end{aligned}$
	}
\end{ex}
\begin{ex}
	[Chuyên ĐHSP Hà Nội 2019]%Câu 20
	Một bình đựng nước dạng hình nón (không có đáy), đựng đầy nước. Người ta thả vào đó một khối cầu có đường kính bằng chiều cao của bình nước và đo được thể tích nước tràn ra ngoài là $18\pi\mathrm{dm}^3$. Biết rằng khối cầu tiếp xúc với tất cả các đường sinh của hình nón và đúng một nửa của khối cầu chìm trong nước (hình bên). Thể tích $V$ của nước còn lại trong bình bằng\\
	{\color{red}HÌNH Ở ĐÂY}
	\choice
	{$24\pi\mathrm{dm}^3$}
	{\True $6\pi\mathrm{dm}^3$}
	{$54\pi\mathrm{dm}^3$}
	{$12\pi\mathrm{dm}^3$}
	\loigiai{
		{\color{red}HÌNH Ở ĐÂY}\\
		Đường kính của khối cầu bằng chiều cao của bình nước nên $OS=2OH$.\\
		Ta có thể tích nước tràn ra ngoài là thể tích của nửa quả cầu chìm trong bình nước:\\
		$18\pi=\dfrac{V_C}{2}=\dfrac{2\pi O{H^3}}{3}\Leftrightarrow OH=3.$\\
		Lại có $\dfrac{1}{O{H^2}}=\dfrac{1}{O{S^2}}+\dfrac{1}{O{B^2}}\Leftrightarrow O{B^2}=12.$\\
		Thể tích bình nước (thể tích nước ban đầu): $V_n=\dfrac{\pi .OS.O{B^2}}{3}=24\pi $ $\left(\mathrm{dm}^3\right)$.\\
		Thể tích nước còn lại là $24\pi-18\pi=6\pi $ $\left(\mathrm{dm}^3\right)$.
	}
\end{ex}
\begin{ex}
	[Chuyên Phan Bội Châu Nghệ An 2019]%Câu 21
	Cho tam giác đều $ABC$ có đường tròn nội tiếp $\left(O;r\right)$, cắt bỏ phần hình tròn và cho hình phẳng thu được quay quanh $AO$. Tính thể tích khối tròn xoay thu được theo $r$.
	\choice
	{$\dfrac{5}{3}\pi{r^3}$}
	{$\dfrac{4}{3}\pi{r^3}$}
	{$\pi{r^3}\sqrt{3}$}
	{\True $\pi{r^3}$}
	\loigiai{
		{\color{red}HÌNH Ở ĐÂY}\\
		Gọi $H$ là chân đường cao $AH$ của tam giác $ABC$\\
		Vì tam giác $ABC$ đều nên ta có: $AH=3OH=3r$, $AH=BC\dfrac{\sqrt{3}}{2}\Leftrightarrow BC=\dfrac{2}{\sqrt{3}}AH=r2\sqrt{3}$\\
		Khi quay tam giác $ABC$ quanh trục$AO$ ta được hình nón có thể tích là: $V_N$, có đáy là đường tròn đường kính $BC$ khi đó: $S_N=\pi H{C^2}=\pi{r^2}3$, chiều cao của hình nón là: $AH=3r$, khi đó thể tích hình nón là: $V_N=\dfrac{1}{3}AH.S_N=\dfrac{1}{3}3r.\pi{r^2}3=3\pi{r^3}$ (đvtt)\\
		Thể tích khối cầu khi quay hình tròn $\left(O;r\right)$ quanh trục $AO$ là: $V_C=\dfrac{4}{3}\pi{r^3}$\\
		Vậy thể tích $V$ của khối tròn xoay thu được khi quay tam giác $ABC$ đã cắt bỏ phần hình tròn quanh trục$AO$ là: $V=V_N-V_C=3\pi{r^3}-\dfrac{4}{3}\pi{r^3}=\dfrac{5}{3}\pi{r^3}$.
	}
\end{ex}
\begin{ex}
	[THCS - THPT Nguyễn Khuyến 2019]%Câu 22
	Cho một cái bình hình trụ có bán kính đáy bằng $R$ và có $4$ quả cam hình cầu, trong đó có $3$ quả cam có cùng bán kính và một quả cam cùng bán kính với đáy bình. Lần lượt bỏ vào bình $3$ quả cam cùng bán kính sao cho chúng đôi một tiếp xúc với nhau, mỗi quả cam đều tiếp xúc với với đáy bình và tiếp xúc với một đường sinh của bình; Bỏ tiếp quả cam thứ tư còn lại vào bình và tiếp xúc với mặt nắp của bình. Chiều cao của bình bằng
	\choice
	{\True $R{\left(\sqrt{2\sqrt{3}-3}+1\right)^2}$}
	{$R{\left(\sqrt{2\sqrt{3}-3}-1\right)^2}$}
	{$R{\left(\sqrt{2\sqrt{3}+3}+1\right)^2}$}
	{$R{\left(\sqrt{2\sqrt{3}+3}-1\right)^2}$}
	\loigiai{
		Gọi $A,B,C$ là tâm ba quả cam có cùng bán kính $r$. $K$ là tâm quả cam có bán kính $R$. $IJ$ là chiều cao của hình trụ.\\
		Khi đó $OA=\dfrac{2}{3}.\dfrac{2r\sqrt{3}}{2}=\dfrac{2r\sqrt{3}}{3}$. Do ba quả cam tiếp xúc với ba đường sinh của hình trụ nên ta có\\
		$R=OA+r=\dfrac{2r\sqrt{3}}{3}+r\Rightarrow r=R\left(2\sqrt{3}-3\right)$ và $OA=2R\left(2-\sqrt{3}\right)$.\\
		Do quả cam có bán kính $R$ tiếp xúc với ba quả cam có bán kính $r$ nên khoảng cách từ tâm $K$ đến mặt phẳng $\left(ABC\right)$ là $OK=\sqrt{K{A^2}-O{A^2}}=\sqrt{\left(R+r\right)^2-\left(2R\left(2-\sqrt{3}\right)\right)^2}=2R\sqrt{2\sqrt{3}-3}$.\\
		Vậy chiều cao của hình trụ là\\
		$IJ=IO+OK+KJ=R\left(2\sqrt{3}-3\right)+2R\sqrt{2\sqrt{3}-3}+R=R{\left(\sqrt{2\sqrt{3}-3}+1\right)^2}$.
	}
\end{ex}
\begin{ex}
	[Liên Trường THPT Tp Vinh Nghệ An 2019]%Câu 23
	Cho hình cầu tâm $O$ bán kính $R=5$, tiếp xúc với mặt phẳng $(P)$. Một hình nón tròn xoay có đáy nằm trên $(P)$, có chiều cao $h=15$, có bán kính đáy bằng $R$. Hình cầu và hình nón nằm về một phía đối với mặt phẳng $(P)$. Người ta cắt hai hình đó bởi mặt phẳng $(Q)$ song song với $(P)$ và thu được hai thiết diện có tổng diện tích là $S$. Gọi $x$ là khoảng cách giữa $(P)$ và $(Q)$, $(0<x\le 5)$. Biết rằng $S$ đạt giá trị lớn nhất khi $x=\dfrac{a}{b}$ (phân số $\dfrac{a}{b}$ tối giản). Tính giá trị $T=a+b$.\\
	{\color{red}HÌNH Ở ĐÂY}
	\choice
	{$T=17$}
	{\True $T=19$}
	{$T=18$}
	{$T=23$}
	\loigiai{
		{\color{red}HÌNH Ở ĐÂY}\\
		Gọi $G$ là tâm của thiết diện cắt bởi mặt phẳng $(Q)$ và mặt cầu.\\
		Theo giả thiết ta có $OA=OB=OH=R=5$ và $HG=x$. $GF$ là bán kính của đường tròn thiết diện. Khi đó $GF=\sqrt{5^2-\left(5-x\right)^2}=\sqrt{10x-x^2}$.\\
		Gọi $S_1$ là tâm của thiết diện cắt bởi mặt phẳng $(Q)$ và mặt cầu.\\
		Gọi $M$ là tâm của thiết diện cắt bởi $(Q)$ và hình nón. Theo giả thiết ta có $MI=x$ và $\dfrac{SM}{SI}=\dfrac{ML}{ID}\Rightarrow ML=\dfrac{SM.ID}{SI}=\dfrac{\left(15-x\right)5}{15}=5-\dfrac{x}{3}$.\\
		Gọi $S_2$ là diện tích thiết diện của mặt phẳng $(Q)$ và hình nón.\\
		Ta có $S_2=\pi{\left(5-\dfrac{x}{3}\right)^2}$\\
		Vậy $S=S_1+S_2=\pi\left[10x-x^2+\left(5-\dfrac{x}{3}\right)^2\right]=\pi\left(-\dfrac{8}{9}{x^2}+\dfrac{20}{3}x+25\right)$\\
		$S$ đạt giá trị lớn nhất khi $f(x)=-\dfrac{8}{9}{x^2}+\dfrac{20}{3}x+25$ đạt giá khi lớn nhất $\Leftrightarrow x=\dfrac{15}{4}$.\\
		Theo đề ra ta có $x=\dfrac{a}{b}=\dfrac{15}{4}\Rightarrow T=a+b=19$.
	}
\end{ex}
\begin{ex}
	[Liên Trường THPT Tp Vinh Nghệ An 2019]%Câu 24
	Một khối đồ chơi gồm một khối hình trụ $(T)$ gắn chồng lên một khối hình nón $(N)$, lần lượt có bán kính đáy và chiều cao tương ứng là $r_1$, $h_1$, $r_2$, $h_2$ thỏa mãn $r_2=2r_1$, $h_1=2h_2$ (hình vẽ). Biết rằng thể tích của khối nón $(N)$ bằng $20\mathrm{cm}^3$. Thể tích của toàn bộ khối đồ chơi bằng\\
	{\color{red}HÌNH Ở ĐÂY}
	\choice
	{$140\mathrm{cm}^3$}
	{$120\mathrm{cm}^3$}
	{$30\mathrm{cm}^3$}
	{\True $50\mathrm{cm}^3$}
	\loigiai{
		Ta có thể tích khối trụ là $V_1=\pi .r_1^2.h_1$, mà $r_2=2r_1,h_1=2h_2$\\
		$V_1=\pi .\left(\dfrac{r_2}{2}\right)^2.2h_2=\dfrac{1}{2}\pi .r_2^2h_2$.\\
		Mặt khác thể tích khối nón là $V_2=\dfrac{1}{3}\pi .r_2^2h_2=20\Rightarrow\pi .r_2^2h_2=60$ $\left(\mathrm{cm}^3\right)$.\\
		Suy ra $V_1=\dfrac{1}{2}.60=30\mathrm{cm}^3$.\\
		Vậy thể tích toàn bộ khối đồ chơi bằng $V_1+V_2=30+20=50\mathrm{cm}^3$.
	}
\end{ex}
\begin{ex}
	[THPT Lê Quý Đôn Đà Nẵng - 2019]%Câu 25
	Thả một quả cầu đặc có bán kính 3 $\left(\mathrm{cm}\right)$ vào một vật hình nón (có đáy nón không kín) (như hình vẽ bên). Cho biết khoảng cách từ tâm quả cầu đến đỉnh nón là 5 $\left(\mathrm{cm}\right)$. Tính thể tích (theo đơn vị cm3) phần không gian kín giới hạn bởi bề mặt quả cầu và bề mặt trong của vật hình nón.\\
	{\color{red}HÌNH Ở ĐÂY}
	\choice
	{\True $\dfrac{12\pi}{5}$}
	{$\dfrac{14\pi}{5}$}
	{$\dfrac{16\pi}{5}$}
	{$\dfrac{18\pi}{5}$}
	\loigiai{
		Xét hình nón và quả cầu như hình vẽ bên dưới.\\
		{\color{red}HÌNH Ở ĐÂY}\\
		$OI=\dfrac{I{K^2}}{SI}=\dfrac{3^2}{5}=\dfrac{9}{5}\left(\text{cm}\right).$\\
		Thể tích chỏm cầu tâm $I$ có bán kính $OK$ là $V_2=\pi .\left(IK-OI\right)^2.\left(IK-\dfrac{IK-OI}{3}\right)=\pi .\left(3-\dfrac{9}{5}\right)^2.\left(3-\dfrac{3-\tfrac{9}{5}}{3}\right)=\dfrac{468\pi}{125}\left(\mathrm{cm}^3\right).$\\
		Thể tích hình nón có đỉnh $S$, đáy hình tròn tâm $O$, bán kính đáy $OK$ là\\
		$V_1=\dfrac{1}{3}.SO.S_{(O;OK)}$ $\dfrac{1}{3}.\dfrac{16}{5}.\pi{\left(\dfrac{12}{5}\right)^2}=\dfrac{768\pi}{125}\left(\mathrm{cm}^3\right).$\\
		Thể tích phần không gian kín giới hạn bởi bề mặt quả cầu và bề mặt trong của vật hình nón là: $V_1-V_2=\dfrac{768\pi}{125}-\dfrac{468\pi}{125}=\dfrac{12\pi}{5}\left(\mathrm{cm}^3\right)$.
	}
\end{ex}
\begin{ex}
	[Sở Hà Nội 2019]%Câu 26
	Cho hình nón có chiều cao $2R$ và bán kính đáy là $R$. Xét hình trụ nội tiếp hình nón sao cho thể tích trụ lớn nhất. Khi đó bán kính đáy của trụ là\\
	{\color{red}HÌNH Ở ĐÂY}
	\choice
	{\True $\dfrac{2R}{3}$}
	{$\dfrac{R}{3}$}
	{$\dfrac{3R}{4}$}
	{$\dfrac{R}{2}$}
	\loigiai{
		{\color{red}HÌNH Ở ĐÂY}\\
		Gọi $D,E$ lần lượt là tâm đáy nhưu hình vẽ. Đặt bán kính đáy là $r=x\in\left(0;R\right)$.\\
		Ta có $\dfrac{GC}{CE}=\dfrac{FG}{AE}$ $\Rightarrow\dfrac{R-x}{R}=\dfrac{FG}{2R}$ $\Rightarrow FG=2\left(R-x\right)=h$\\
		Ta có thể tích trụ là:\\
		$V=\pi{r^2}h=2\pi{x^2}\left(R-x\right)=$ $8\pi .\dfrac{x}{2}.\dfrac{x}{2}\left(R-x\right)\le $ $8\pi{\left(\dfrac{\dfrac{x}{2}+\dfrac{x}{2}+R-x}{3}\right)^3}=\dfrac{8\pi{R^3}}{27}$.\\
		Dấu \lq\lq$=$\rq\rq xảy ra khi $\dfrac{x}{2}=R-x\Leftrightarrow x=\dfrac{2R}{3}$.
	}
\end{ex}
\begin{ex}
	[Thanh Tường Nghệ An - 2019]%Câu 27
	Một con xoay được thiết kế gồm hai khối trụ $(T_1)$, $(T_2)$ chồng lên khối nón $(N)$ (Tham khảo mặt cắt ngang qua trục như hình vẽ). Khối trụ $(T_1)$ có bán kính đáy $r(\mathrm{cm})$, chiều cao $h_1(\mathrm{cm})$. Khối trụ $(T_2)$ có bán kính đáy $2r(\mathrm{cm})$, chiều cao $h_2=2h_1(\mathrm{cm})$. Khối nón $(N)$ có bán kính đáy $r(\mathrm{cm})$, chiều cao $h_n=4h_1(\mathrm{cm})$. Biết rằng thể tích toàn bộ con xoay bằng $31(\mathrm{cm}^3)$. Thể tích khối nón $(N)$ bằng\\
	{\color{red}HÌNH Ở ĐÂY}
	\choice
	{$5(\mathrm{cm}^3)$}
	{$3(\mathrm{cm}^3)$}
	{\True $4(\mathrm{cm}^3)$}
	{$6(\mathrm{cm}^3)$}
	\loigiai{
		Theo bài ta có $h_n=4h_1\Rightarrow{h_1}=\dfrac{1}{4}{h_n};h_2=2h_1=\dfrac{1}{2}{h_n}$.\\
		Thể tích toàn bộ con xoay là\\
		$V=V_{(T_1)}+V_{(T_2)}+V_{(N)}=\pi .r^2.h_1+\pi .(2r)^2.h_2+\dfrac{1}{3}\pi .r^2.h_n$\\
		$\Leftrightarrow 31=\pi .r^2.\dfrac{1}{4}{h_n}+\pi .4r^2.\dfrac{1}{2}{h_n}+\dfrac{1}{3}\pi .r^2.h_n$\\
		$\Leftrightarrow 31=\dfrac{3}{4}\left(\dfrac{1}{3}\pi .r^2.h_n\right)+6\left(\dfrac{1}{3}\pi .r^2.h_n\right)+\dfrac{1}{3}\pi .r^2.h_n\Leftrightarrow 31=\dfrac{31}{4}\left(\dfrac{1}{3}\pi .r^2.h_n\right)$ $\Leftrightarrow\dfrac{1}{3}\pi .r^2.h_n=4$\\
		Vậy thể tích khối nón $(N)$ là: $V_{(N)}=4(\mathrm{cm}^3)$.
	}
\end{ex}
\begin{ex}
	Cho tam giác đều $ABC$ có đỉnh $A\left(5;5\right)$ nội tiếp đường tròn tâm $I$ đường kính $A{A}'$, $M$ là trung điểm $BC$. Khi quay tam giác $ABM$ cùng với nửa hình tròn đường kính $A{A}'$ xung quanh đường thẳng $AM$ (như hình vẽ minh họa), ta được khối nón và khối cầu có thể tích lần lượt là $V_1$ và $V_2$.\\
	{\color{red}HÌNH Ở ĐÂY}\\
	Tỷ số $\dfrac{V_1}{V_2}$ bằng
	\choice
	{\True $\dfrac{9}{32}$}
	{$\dfrac{9}{4}$}
	{$\dfrac{27}{32}$}
	{$\dfrac{4}{9}$}
	\loigiai{
		Gọi độ dài cạnh của tam giác $ABC$ là $a$.\\
		Khi đó khối nón tạo thành có bán kính đáy là: $r=BM=\dfrac{a}{2}$; chiều cao $h=AM=\dfrac{a\sqrt{3}}{2}$\\
		Thể tích khối nón là $V_1=\dfrac{1}{3}\pi{r^2}h=\dfrac{1}{3}.\pi .\left(\dfrac{a}{2}\right)^2.\dfrac{a\sqrt{3}}{2}=\dfrac{\pi{a^3}\sqrt{3}}{24}$\\
		Khối cầu tạo thành có bán kính là $R=\dfrac{2}{3}AM=\dfrac{a\sqrt{3}}{3}$\\
		Thể tích khối cầu là: $V_2=\dfrac{4}{3}\pi{R^3}=\dfrac{4}{3}.\pi .\left(\dfrac{a\sqrt{3}}{3}\right)^3=\dfrac{4\pi{a^3}\sqrt{3}}{27}$\\
		Suy ra $\dfrac{V_1}{V_2}=\dfrac{\pi{a^3}\sqrt{3}}{24}:\dfrac{4\pi{a^3}\sqrt{3}}{27}=\dfrac{9}{32}$.
	}
\end{ex}
\begin{ex}
	[Đề Tham Khảo 2017]%Câu 29
	Cho mặt cầu tâm $O$ bán kính $R$. Xét mặt phẳng $(P)$ thay đổi cắt mặt cầu theo giao tuyến là đường tròn $(C).$ Hình nón $(N)$ có đỉnh $S$ nằm trên mặt cầu, có đáy là đường tròn $(C)$ và có chiều cao $h\left(h>R\right)$. Tính $h$ để thể tích khối nón được tạo nên bởi $(N)$ có giá trị lớn nhất.
	\choice
	{$h=\sqrt{2}R$}
	{\True $h=\dfrac{4R}{3}$}
	{$h=\dfrac{3R}{2}$}
	{$h=\sqrt{3}R$}
	\loigiai{
		{\color{red}HÌNH Ở ĐÂY}\\
		Cách 1:\\
		Gọi $I$ là tâm mặt cầu và $H$, $r$ là tâm và bán kính của $(C)$.\\
		Ta có $IH=h-R$ và $r^2=R^2-I{H^2}=R^2-\left(h-R\right)^2=2Rh-h^2$.\\
		Thể tích khối nón $V=\dfrac{1}{3}h\pi{r^2}=\dfrac{\pi}{3}h\left(2Rh-h^2\right)$.\\
		Ta có $h\cdot h\cdot\left(4R-2h\right)\le{\left(\dfrac{h+h+4R-2h}{3}\right)^3}=\left(\dfrac{4R}{3}\right)^3\Rightarrow{h^2}\left(2R-h\right)\le\dfrac{1}{2}{\left(\dfrac{4R}{3}\right)^3}$.\\
		Do đó $V$ lớn nhất khi $h=4R-2h\Leftrightarrow h=\dfrac{4R}{3}.$\\
		Cách 2:\\
		Gọi $I$ là tâm mặt cầu và $H$, $r$ là tâm và bán kính của $(C)$.\\
		Ta có $IH=h-R$ và $r^2=R^2-I{H^2}=R^2-\left(h-R\right)^2=2Rh-h^2$.\\
		Thể tích khối nón $V=\dfrac{1}{3}h\pi{r^2}=\dfrac{\pi}{3}h\left(2Rh-h^2\right)=\dfrac{\pi}{3}.\left(2h^2R-h^3\right)$\\
		Xét hàm $f(h)=-h^3+2h^2R,h\in\left(R,2R\right)$, có $f'(h)=-3h^2+4hR$.\\
		$f'(h)=0\Leftrightarrow-3h^2+4hR=0\Leftrightarrow h=0$ hoặc $h=\dfrac{4R}{3}$.\\
		Bảng biến thiên\\
		{\color{red}HÌNH Ở ĐÂY}\\
		$\max f(h)=\dfrac{32}{27}{R^3}$, tại $h=\dfrac{4R}{3}$. Vậy thể tích khối nón được tạo nên bởi $(N)$ có giá trị lớn nhất là $V=\dfrac{1}{3}\pi\dfrac{32}{27}{R^3}=\dfrac{32}{81}\pi{R^3}$ khi $h=\dfrac{4R}{3}$.
	}
\end{ex}
\begin{ex}
	[THPT Hoàng Hoa Thám Hưng Yên 2019]%Câu 30
	Một cái thùng đựng đầy nước được tạo thành từ việc cắt mặt xung quanh của một hình nón bởi một mặt phẳng vuông góc với trục của hình nón. Miệng thùng là đường tròn có bán kính bằng ba lần bán kính mặt đáy của thùng. Người ta thả vào đó một khối cầu có đường kính bằng $\dfrac{3}{2}$ chiều cao của thùng nước và đo được thể tích nước tràn ra ngoài là $54\sqrt{3}\pi $ ($\mathrm{dm}^3$). Biết rằng khối cầu tiếp xúc với mặt trong của thùng và đúng một nửa của khối cầu đã chìm trong nước (hình vẽ). Thể tích nước còn lại trong thùng có giá trị nào sau đây?\\
	{\color{red}HÌNH Ở ĐÂY}
	\choice
	{$\dfrac{46}{5}\sqrt{3}\pi $ ($\mathrm{dm}^3$)}
	{$18\sqrt{3}\pi $ ($\mathrm{dm}^3$)}
	{\True $\dfrac{46}{3}\sqrt{3}\pi $ ($\mathrm{dm}^3$)}
	{$18\pi $ ($\mathrm{dm}^3$)}
	\loigiai{
		{\color{red}HÌNH Ở ĐÂY}\\
		Gọi $R$ là bán kính của khối cầu. Khi đó thể tích nước tràn ra ngoài là thể tích của một nửa khối cầu nên $\dfrac{1}{2}.\dfrac{4}{3}\pi{R^3}=54\sqrt{3}\pi\Leftrightarrow R=3\sqrt{3}$.\\
		Do đó chiều cao của thùng nước là $h=\dfrac{2}{3}.2R=4\sqrt{3}$.\\
		Cắt thùng nước bởi thiết diện qua trục ta được hình thang cân $ABCD$ với $AB=3CD$. Gọi $O$ là giao điểm của $AD$ và $BC$ thì tam giác $OAB$ cân tại $O$.\\
		Gọi $H$ là trung điểm của đoạn thẳng $AB$ và $I$ là giao điểm của $OH$ và $CD$ $\to I$ là trung điểm của $DC$ nên $DI=\dfrac{1}{3}AH$.\\
		Ta có $\dfrac{OI}{OH}=\dfrac{DI}{AH}=\dfrac{1}{3}$ $\to OH=\dfrac{3}{2}HI=6\sqrt{3}$\\
		Gọi $K$ là hình chiếu của $H$ trên $OA$ thì $HK=R=3\sqrt{3}$\\
		Tam giác $OHA$ vuông tại $H$ có đường cao $HK$ nên\\
		$\dfrac{1}{H{K^2}}=\dfrac{1}{H{O^2}}+\dfrac{1}{A{H^2}}\to\dfrac{1}{A{H^2}}=\dfrac{1}{H{K^2}}-\dfrac{1}{H{O^2}}=\dfrac{1}{36}$ $\to AH=6\to DI=2$\\
		Thể tích thùng đầy nước là $\dfrac{h\pi\left(A{H^2}+D{I^2}+AH.DI\right)}{3}=\dfrac{4\sqrt{3}\pi\left(6^2+2^2+6.2\right)}{3}=\dfrac{208\sqrt{3}\pi}{3}$\\
		Do đó thể tích nước còn lại là$\dfrac{208\sqrt{3}\pi}{3}-54\sqrt{3}\pi=\dfrac{46\sqrt{3}\pi}{3}\left(\mathrm{dm}^3\right)$.
	}
\end{ex}
\begin{ex}
	[THPT Đoàn Thượng - Hải Dương - 2019]%Câu 31
	Chiều cao của khối trụ có thể tích lớn nhất nội tiếp trong hình cầu có bán kính $R$ là
	\choice
	{$\dfrac{4R\sqrt{3}}{3}$}
	{$R\sqrt{3}$}
	{$\dfrac{R\sqrt{3}}{3}$}
	{\True $\dfrac{2R\sqrt{3}}{3}$}
	\loigiai{
		Gọi $O$ là tâm hình cầu bán kính $R$ và $I,I'$ lần lượt là tâm hai hình tròn đáy của khối trụ với $AB$ là một đường cao của khối trụ như hình vẽ.\\
		{\color{red}HÌNH Ở ĐÂY}\\
		Dễ thấy $O$ là trung điểm $I{I}'$.\\
		Đặt $x$ là chiều cao của khối trụ ta có $0<x<2R$ và $AB=I{I}'=x$\\
		Tam giác $OAI$ có $AI=\sqrt{A{O^2}-O{I^2}}=\sqrt{R^2-\left(\dfrac{x}{2}\right)^2}=\sqrt{R^2-\dfrac{x^2}{4}}$.\\
		Thể tích khối trụ là $f(x)=\pi I{A^2}.AB=\pi\left(R^2-\dfrac{x^2}{4}\right)x=\pi\left(R^2x-\dfrac{x^3}{4}\right)$.\\
		$f'(x)=\pi\left(R^2-\dfrac{3}{4}{x^2}\right)$, $f'(x)=0\Leftrightarrow\left[\begin{aligned}
			& x=\dfrac{2R\sqrt{3}}{3}\\ 
			& x=-\dfrac{2R\sqrt{3}}{3}\\ 
		\end{aligned}\right.$ với $x>0$ nên $x=\dfrac{2R\sqrt{3}}{3}\in\left(0;2R\right)$\\
		Ta có bảng biến thiên:\\
		{\color{red}HÌNH Ở ĐÂY}\\
		Từ bảng biến thiên ta thấy thể tích khối trụ lớn nhất khi chiều cao $x=\dfrac{2R\sqrt{3}}{3}$.
	}
\end{ex}
\begin{ex}
	Một bình đựng nước dạng hình nón (không có đáy) đựng đầy nước. Người ta thả vào đó một khối cầu có đường kính bằng chiều cao của bình nước và đo được thể tích nước tràn ra ngoài là $18\pi\mathrm{dm}^3$ .Biết khối cầu tiếp xúc với tất cả các đường sinh của hình nón và đúng một nửa khối cầu chìm trong nước. Tính thể tích nước còn lại trong bình.
	\choice
	{$27\pi\mathrm{dm}^3$}
	{\True $6\pi\mathrm{dm}^3$}
	{$9\pi\mathrm{dm}^3$}
	{$24\pi\mathrm{dm}^3$}
	\loigiai{
		{\color{red}HÌNH Ở ĐÂY}{\color{red}HÌNH Ở ĐÂY}\\
		Vì đúng một nửa khối cầu chìm trong nước nên thể tích khối cầu gấp 2 lần thể tích nước tràn ra ngoài.\\
		Gọi bán kính khối cầu là $R$, lúc đó: $\dfrac{4}{3}\pi{R^3}\text{=36}\pi\Leftrightarrow{R^3}=27$.\\
		Xét tam giác $ABC$ có $AC$ là chiều cao bình nước nên $AC=2R$ (Vì khối cầu có đường kính bằng chiều cao của bình nước)\\
		Trong tam giác $ABC$ có: $\dfrac{1}{C{H^2}}=\dfrac{1}{C{A^2}}+\dfrac{1}{C{B^2}}\Leftrightarrow\dfrac{1}{R^2}=\dfrac{1}{4R^2}+\dfrac{1}{C{B^2}}\Leftrightarrow C{B^2}=\dfrac{4R^2}{3}$.\\
		Thể tích khối nón: $V_n=\dfrac{1}{3}\pi .C{B^2}.AC=\dfrac{1}{3}\pi .\dfrac{4R^2}{3}.2R=\dfrac{8\pi}{9}.R^3=24\pi\mathrm{dm}^3$.\\
		Vậy thể tích nước còn lại trong bình: $24\pi-18\pi=6\pi\mathrm{dm}^3$.
	}
\end{ex}
\begin{ex}
	[Chuyên Thái Nguyên 2019]%Câu 33
	Cho khối nón có độ lớn góc ở đỉnh là $\dfrac{\pi}{3}$. Một khối cầu $\left(S_1\right)$ nội tiếp trong khối nối nón. Gọi $\left(S_2\right)$ là khối cầu tiếp xúc với tất cả các đường sinh của nón và với $S_1$; $S_3$ là khối cầu tiếp xúc với tất cả các đường sinh của khối nón và với $S_2$;$\ldots$;$S_n$ là khối cầu tiếp xúc với tất cả các đường sinh của nón và với $S_{n-1}$ . Gọi $V_1,V_2$ ,$\ldots$$,V_{n-1},V_n$ lần lượt là thể tích của khối cầu $S_1,S_2,S_3,\ldots,S_n$ và $V$ là thể tích của khối nón. Tính giá trị của biểu thức $T=\underset{n\to+\infty}{\lim}\dfrac{V_1+V_2+\ldots+V_n}{V}$ 
	\choice
	{$T=\dfrac{3}{5}$}
	{\True $T=\dfrac{6}{13}$}
	{$T=\dfrac{7}{9}$}
	{$T=\dfrac{1}{2}$}
	\loigiai{
		{\color{red}HÌNH Ở ĐÂY}\\
		Thiết diện qua trục của hình nón là một tam giác đều cạnh $l$ . Do đó bán kính đường tròn nội tiếp tam giác cũng chính là bán kính mặt cầu nội tiếp chọp là $r_1=\dfrac{1}{3}\dfrac{l\sqrt{3}}{2}=\dfrac{l\sqrt{3}}{6}$\\
		Áp dụng định lí Ta-Let ta có\\
		$\dfrac{A{A}'}{AB}=\dfrac{A{H}'}{AH}=\dfrac{AH-H{H}'}{AH}=\dfrac{\frac{l\sqrt{3}}{2}-\frac{l\sqrt{3}}{3}}{\frac{l\sqrt{3}}{3}}=\dfrac{1}{3}$ $\Rightarrow A{A}'=\dfrac{1}{3}$\\
		Tương tự ta tìm được $r_2=\dfrac{l}{3}.\dfrac{\sqrt{3}}{6}=\dfrac{l\sqrt{3}}{18}=\dfrac{r_1}{3}$ .\\
		Tiếp tục như vậy ta có $r_3=\dfrac{1}{3}{r_2},r_4=\dfrac{1}{3}{r_3},\ldots,r_n=\dfrac{1}{3}{r_{n-1}}$\\
		Ta có $V_1=\dfrac{4}{3}\pi r_1^3,V_2=\dfrac{4}{3}\pi r_2^3=\dfrac{4}{3}\pi{\left(\dfrac{r_1}{3}\right)^3}=\dfrac{1}{3^3}{V_1},V_3=\dfrac{1}{\left(3^3\right)^2}{V_1},\ldots,V_n=\dfrac{4}{3}\pi r_{n-1}^3=\dfrac{1}{\left(3^3\right)^{n-1}}{V_1}$\\
		Do đó $T=\underset{n\to+\infty}{\lim}\dfrac{V_1+V_2+\ldots+V_n}{V}=\underset{n\to+\infty}{\lim}\dfrac{V_1\left[1+\frac{1}{3^3}+\frac{1}{\left(3^3\right)^2}+...+\frac{1}{\left(3^3\right)^{n-1}}\right]}{V}=\underset{n\to+\infty}{\lim}\dfrac{V_1.S}{V}$\\
		Đặt $S=1+\dfrac{1}{3^3}+\dfrac{1}{\left(3^3\right)^2}+\ldots+\dfrac{1}{\left(3^3\right)^{n-1}}$\\
		Đây là tổng của CSN lùi vô hạn với công bội $q=\dfrac{1}{3^3}$ $\Rightarrow\underset{n\to+\infty}{\lim}S=\dfrac{1}{1-\frac{1}{3^3}}=\dfrac{27}{26}$\\
		$\Rightarrow{V_1}+V_2+\ldots+V_n=\dfrac{27}{26}{V_1}=\dfrac{27}{26}\dfrac{4}{3}\pi{\left(\dfrac{l\sqrt{3}}{6}\right)^3}=\dfrac{\sqrt{3}}{52}\pi{l^3}$\\
		$V=\dfrac{1}{3}\pi{r^2}h=\dfrac{1}{3}\pi{\left(\dfrac{l}{2}\right)^2}\dfrac{l\sqrt{3}}{2}=\dfrac{\sqrt{3}\pi{l^3}}{24}$\\
		Vậy $T=\dfrac{\dfrac{\sqrt{3}}{52}\pi{l^3}}{\dfrac{\sqrt{3}}{24}\pi{l^3}}=\dfrac{6}{13}$.
	}
\end{ex}
\begin{ex}
	[Chuyên Bắc Ninh - 2020]%Câu 34
	Cho hình trụ có hai đáy là hai hình tròn $(O)$ và $\left(O'\right)$, bán kính bằng $a$. Một hình nón có đỉnh là $O'$ và có đáy là hình tròn $(O)$. Biết góc giữa đường sinh của hình nón với mặt đáy bằng $60^0$, tỉ số diện tích xung quanh của hình trụ và hình nón bằng
	\choice
	{$2$}
	{$\sqrt{2}$}
	{\True $\sqrt{3}$}
	{$\dfrac{1}{\sqrt{3}}$}
	\loigiai{
		{\color{red}HÌNH Ở ĐÂY}\\
		Gọi $A$ là điểm thuộc đường tròn $(O)$.\\
		Góc giữa $O'A$ và mặt phẳng đáy là góc $\widehat{O'AO}$. Theo giả thiết ta có $\widehat{O'AO}=\text{60}^\circ$.\\
		Xét tam giác $O'OA$ vuông tại $O$, ta có\\
		$\tan\widehat{O'AO}=\dfrac{O'O}{OA}\Rightarrow{O}'O=a.\tan\text{60}^\circ=a\sqrt{3}$.\\
		$\cos\widehat{O'AO}=\dfrac{OA}{O'A}\Rightarrow{O}'A=\dfrac{a}{\cos60^\circ}=2a$.\\
		Diện tích xung quanh của hình trụ là $S_{xq(T)}=2\pi .OA.O'O=2\pi .a.a\sqrt{3}=2\pi{a^2}\sqrt{3}$.\\
		Diện tích xung quanh của hình nón là: $S_{xq(N)}=\pi .OA.O'A=\pi .a.2a=2\pi{a^2}$\\
		$\Rightarrow\dfrac{S_{xq(T)}}{S_{xq(N)}}=\dfrac{2\pi{a^2}\sqrt{3}}{2\pi{a^2}}=\sqrt{3}$.
	}
\end{ex}
\begin{ex}
	[Chuyên Bắc Ninh - 2020]%Câu 35
	Cho một chiếc cốc có dạng hình nón cụt và một viên bi có đường kính bằng chiều cao của cốc. Đổ đầy nước rồi thả viên bi vào, ta thấy lượng nước tràn ra bằng một phần ba lượng nước đổ vào cốc lúc ban đầu. Biết viên bi tiếp xúc với đáy cốc và thành cốc. Tìm tỉ số bán kính của miệng cốc và đáy cốc (bỏ qua độ dày của cốc).\\
	{\color{red}HÌNH Ở ĐÂY}
	\choice
	{\True $\dfrac{5+\sqrt{21}}{2}$}
	{$\dfrac{5}{2}$}
	{$\sqrt{21}$}
	{$\dfrac{21+\sqrt{5}}{2}$}
	\loigiai{
		Gọi bán kính viên bi là $r$ ; bán kính đáy cốc, miệng cốc lần lượt là $r_1,r_2$, $\left(r_1<r_2\right)$. Theo giả thiết thì chiều cao của cốc là $h=2r$.\\
		Thể tích viên bi là $V_B=\dfrac{4}{3}\pi{r^3}$.\\
		Thể tích cốc là $V_C=\dfrac{1}{3}\pi h\left(r_1^2+r_2^2+r_1r_2\right)=\dfrac{2}{3}\pi r\left(r_1^2+r_2^2+r_1r_2\right)$ .\\
		Theo giả thiết thì $V_B=\dfrac{1}{3}{V_C}\Leftrightarrow 6r^2=r_1^2+r_2^2+r_1r_2$ (1).\\
		Mặt cắt chứa trục của cốc là hình thang cân $AB{B}'{A}'$. Đường tròn tâm $\left(O;r\right)$ là đường tròn lớn của viên bi, đồng thời là đường tròn nội tiếp hình thang $AB{B}'{A}'$, tiếp xúc với $A'{B}',AB$ lần lượt tại $H_1,H_2$ và tiếp xúc với $B{B}'$ tại $M$.\\
		{\color{red}HÌNH Ở ĐÂY}\\
		Dễ thấy tam giác $BO{B}'$ vuông tại $O$.\\
		Ta có $O{M^2}=MB.M{B}'\Leftrightarrow{r^2}=r_1r_2$ (2).\\
		Thay (2) vào (1) ta được $6r_1r_2=r_1^2+r_2^2+r_1r_2\Leftrightarrow{\left(\dfrac{r_2}{r_1}\right)^2}-5\dfrac{r_2}{r_1}+1=0$.\\
		Giải phương trình với điều kiện $\dfrac{r_2}{r_1}>1$ ta được $\dfrac{r_2}{r_1}=\dfrac{5+\sqrt{21}}{2}$.\\
		Chú ý Chứng minh công thức thể tích hình nón cụt.\\
		{\color{red}HÌNH Ở ĐÂY}\\
		Ta có $\dfrac{r_1}{r_2}=\dfrac{h_1}{h_1+h}\Leftrightarrow{h_1}=\dfrac{r_1h}{r_2-r_1}$.\\
		$V_1=\dfrac{1}{3}\pi{r_1}^2.h_1=\dfrac{1}{3}\pi h\dfrac{r_1^3}{r_2-r_1}$.\\
		$V_2=\dfrac{1}{3}\pi{r_2}^2.\left(h_1+h\right)=\dfrac{1}{3}\pi h\dfrac{r_2^3}{r_2-r_1}$.\\
		$V=V_2-V_1=\dfrac{1}{3}\pi h\dfrac{r_2^3-r_1^3}{r_2-r_1}=\dfrac{1}{3}\pi h\left(r_1^2+r_2^2+r_1r_2\right)$.
	}
\end{ex}
\begin{ex}
	[Đại Học Hà Tĩnh - 2020]%Câu 36
	Trên bàn có một cốc nước hình trụ chứa đầy nước có chiều cao bằng $3$ lần đường kính của đáy; một viên bi và một khối nón đều bằng thủy tinh. Biết viên bi là một khối cầu có đường kính bằng của cốc nước. Người ta từ từ thả vào cốc nước viên bi và khối nón đó (như hình vẽ) thì thấy nước trong cốc tràn ra ngoài. Tính tỉ số thể tích của lượng nước còn lại trong cốc và lượng nước ban đầu(bỏ qua bề dày của lớp vỏ thủy tinh)\\
	{\color{red}HÌNH Ở ĐÂY}
	\choice
	{\True $\dfrac{5}{9}$}
	{$\dfrac{2}{3}$}
	{$\dfrac{4}{9}$}
	{$\dfrac{1}{2}$}
	\loigiai{
		Gọi $R,h$ lần lượt là bán kính đáy và là chiều cao của khối trụ\\
		$h=6R$\\
		Thể tích của khối trụ $V_T=\pi 6R^3$.\\
		Khối cầu bên trong khối trụ có bán kính $R$ nên khối cầu có thể tích $V_C=\dfrac{4}{3}\pi{R^3}$.\\
		Khối nón bên trong khối trụ có bán kính$R$ và chiều cao $h=4R$ nên khối nón có thể tích $V_N=\dfrac{4}{3}\pi{R^3}$\\
		Thể tích lượng nước còn lại bên trong khối trụ\\
		$V=V_T-\left(V_C+V_N\right)=6\pi{R^3}-\dfrac{8}{3}\pi{R^3}=\dfrac{10}{3}\pi{R^3}$.\\
		Vậy $\dfrac{V}{V_T}=\dfrac{5}{9}$.
	}
\end{ex}
\begin{ex}
	[Sở Ninh Bình 2020]%Câu 37
	Cho tam giác vuông cân $ABC$ có $AB=BC=a\sqrt{2}$. Khi quay tam giác $ABC$ quanh đường thẳng đi qua $B$ và song song với $AC$ ta thu được một khối tròn xoay có thể tích bằng
	\choice
	{$2\pi{a^3}$}
	{$\dfrac{2\pi{a^3}}{3}$}
	{\True $\dfrac{4\pi{a^3}}{3}$}
	{$\pi{a^3}$}
	\loigiai{
		{\color{red}HÌNH Ở ĐÂY}\\
		Gọi $d$ là đường thẳng đi qua B và song song vói $AC$; $H,K$ lần lượt là hình chiếu của $A,C$ trên $d$. Ta có $AC=2a,HA=KC=a$.\\
		Khối tròn xoay cần nhận được khi quay tam giác $ABC$ quanh $d$ chính là khối tròn xoay có được bằng cách từ khối trụ với hai đáy là hình tròn $\left(H,HA\right)$ và $\left(K,KC\right)$ bỏ đi 2 khối nón chung đỉnh $B$ với đáy lần lượt là $\left(H,HA\right)$ và $\left(K,KC\right)$.\\
		Do đó $V=\pi .H{A^2}.AC-2.\dfrac{1}{3}\pi .H{A^2}.\dfrac{AC}{2}=2\pi{a^3}-\dfrac{2}{3}\pi{a^3}=\dfrac{4}{3}\pi{a^3}$.
	}
\end{ex}
\begin{ex}
	[Sở Yên Bái - 2020]%Câu 38
	Một khối đồ chơi gồm một khối trụ và một khối nón có cùng bán kính được chồng lên nhau, độ dài đường sinh khối trụ bằng độ dài đường sinh khối nón và bằng đường kính khối trụ, khối nón (tham khảo hình vẽ). Biết thể tích toàn bộ khối đồ chơi là $50\mathrm{cm}^3$, thể tích khối trụ gần với số nào nhất trong các số sau\\
	{\color{red}HÌNH Ở ĐÂY}
	\choice
	{\True $38,8\mathrm{cm}^3$}
	{$38,2\mathrm{cm}^3$}
	{$36,5\mathrm{cm}^3$}
	{$40,5\mathrm{cm}^3$}
	\loigiai{
		Gọi $l$; $r$ lần lượt là độ dài đường sinh và bán kính đáy khối trụ.\\
		Khi đó ta có $l=2r$.\\
		Suy ra thể tích khối trụ là $V_t=\pi r^2l=2\pi r^3$.\\
		Gọi $h_n;{l_n}$ lần lượt là chiều cao và đường sinh của khối nón.\\
		Theo giả thiết ta có $\left\{\begin{aligned}
			&{l_n}=l\\ 
			&{h_n}=\sqrt{l_{}^2-r_{}^2}=\sqrt{3}r\\ 
		\end{aligned}\right.$.\\
		Khi đó thể tích khối nón là $V_n=\dfrac{1}{3}\pi r^2h_n=\dfrac{\sqrt{3}}{3}\pi r^3.$\\
		Do thể tích toàn bộ khối đồ chơi là $50\mathrm{cm}^3$ nên\\
		$V_t+V_n=2\pi r^3+\dfrac{\sqrt{3}}{3}\pi r^3=\left(2+\dfrac{\sqrt{3}}{3}\right)\pi r^3=50\Rightarrow\pi r^3=\dfrac{150}{6+\sqrt{3}}.$\\
		Khi đó thể tích khối trụ là $V_t=\pi r^2l=2\pi r^3\approx 38,8\mathrm{cm}^3$.
	}
\end{ex}
\begin{ex}
	[Nguyễn Trãi - Thái Bình - 2020]%Câu 39
	Trong tất cả các hình nón nội tiếp trong hình cầu có thể tích bằng $36\pi $, bán kính $r$ của hình nón có diện tích xung quanh lớn nhất là
	\choice
	{$r=\dfrac{3\sqrt{2}}{2}$}
	{$r=\dfrac{3}{2}$}
	{\True $r=2\sqrt{2}$}
	{$r=3$}
	\loigiai{
		{\color{red}HÌNH Ở ĐÂY}\\
		Vì hình cầu có thể tích là $36\pi $ nên bán kính hình cầu là $R=3$.\\
		Ta có diện tích xung quanh của hình nón là $S=\pi rl$.\\
		Để hình nón có diện tích xung quanh lớn nhất thì đỉnh của hình nón và đáy của hình nón phải ở hai phía so với đường tròn kính của hình cầu.\\
		Đặt bán kính đáy hình nón là $r=x$ với $0<x\le 3$ và tâm của đáy hình nón là $I$.\\
		Ta có tam giác $OIB$ vuông tại $I$ nên $OI=\sqrt{9-x^2}$.\\
		Chiều cao của hình nón là $h=3+\sqrt{9-x^2}$.\\
		Độ dài đường sinh của hình nón là $l=\sqrt{\left(3+\sqrt{9-x^2}\right)^2+x^2}=\sqrt{18+6\sqrt{9-x^2}}$.\\
		Suy ra diện tích xung quanh của hình nón là $S=\pi x\sqrt{18+6\sqrt{9-x^2}}$.\\
		Đặt $P=x\sqrt{18+6\sqrt{9-x^2}}$ nên $P^2=x^2\left(18+6\sqrt{9-x^2}\right)$ và đặt $\sqrt{9-x^2}=t$, $\left(0\le t<3\right)$.\\
		Khi đó $P^2=\left(9-t^2\right)\left(18+6t\right)$ với $0\le t<3$.\\
		Xét hàm số $y=\left(9-t^2\right)\left(18+6t\right)\Leftrightarrow y=-6t^3-18t^2+54t+162$ có\\
		$y'=-18t^2-36t+54=0\Leftrightarrow\left[\begin{aligned}
			& t=1\\ 
			& t=-3(L)\\ 
		\end{aligned}\right.$.\\
		Bảng biến thiên của hàm số $y=\left(9-t^2\right)\left(18+6t\right)$ trên $0\le t<3$.\\
		{\color{red}HÌNH Ở ĐÂY}\\
		Từ bảng biến thiên, $P^2$ lớn nhất khi và chỉ khi $t=1$ suy ra $P$ lớn nhất khi và chỉ khi $t=1$.\\
		Khi đó $S=\pi x\sqrt{18+6\sqrt{9-x^2}}$ lớn nhất khi $\sqrt{9-x^2}=1\Leftrightarrow x=2\sqrt{2}$ và diện tích xung quanh của mặt cầu khi đó là $S=8\sqrt{3}\pi $.
	}
\end{ex}
\begin{ex}
	[Sở Ninh Bình 2020]%Câu 40
	Có một bể hình hộp chữ nhật chứa đầy nước. Người ta cho ba khối nón giống nhau có thiết diện qua trục là một tam giác vuông cân vào bể sao cho ba đường tròn đáy của ba khối nón đôi một tiếp xúc với nhau, một khối nón có đường tròn đáy chỉ tiếp xúc với một cạnh của đáy bể và hai khối nón còn lại có đường tròn đáy tiếp xúc với hai cạnh của đáy bể. Sau đó người ta đặt lên đỉnh của ba khối nón một khối cầu có bán kính bằng $\dfrac{4}{3}$ lần bán kính đáy của khối nón. Biết khối cầu vừa đủ ngập trong nước và tổng lượng nước trào ra là $\dfrac{337\pi}{24}$ (lít). Thể tích nước ban đầu ở trong bể thuộc khoảng nào dưới đây (đơn vị tính: lít)?
	\choice
	{$(150;151)$}
	{\True $(151;152)$}
	{$(139;140)$}
	{$(138;139)$}
	\loigiai{
		{\color{red}HÌNH Ở ĐÂY}\\
		Gọi $r$ là bán kính đáy của hình nón suy ra chiều cao nón là $h=r$ (do thiết diện là tam giác vuông cân).\\
		Chiều dài của khối hộp là $b=4 r ;$ bán kính của khối cầu là $R=\dfrac{4}{3}r$.\\
		Thể tích nước bị tràn là $3.\dfrac{1}{3}\pi{r^2}h+\dfrac{4}{3}\pi{R^3}=\dfrac{337\pi}{24}\Leftrightarrow\pi{r^3}+\dfrac{4}{3}\pi\dfrac{64}{27}{r^3}=\dfrac{337\pi}{24}\Rightarrow r=\dfrac{3}{2}(\mathrm{dm})$.\\
		Gọi $A,B,C$ là tâm của 3 đáy của khối nón suy ra $\Delta A B C$ đều cạnh $2 r\Rightarrow R_{A B C}=\dfrac{2 r}{\sqrt{3}}$.\\
		Chiều rộng khối hộp là $a=2 r+\dfrac{2 r\sqrt{3}}{2}=r(2+\sqrt{3})$ ($\mathrm{dm}$).\\
		Ba đỉnh nón chạm mặt cầu tại các điếm $M, N, P\Rightarrow\Delta M N P=\Delta A B C$\\
		$d(I;(MNP))=\sqrt{R^2-R_{(ABC)}^2}$ (với $I$ là tâm mặt cầu), do đó $d(I;(MNP))=\dfrac{2}{3}r$. Suy ra chiều cao của khối trụ là $c=R+\dfrac{2}{3}r+r=3 r$.\\
		Thể tích nước ban đầu là $abc=12(2+\sqrt{3}){r^3}=12(2+\sqrt{3}).\dfrac{27}{8}\approx 151,1\left(\mathrm{dm}^3\right)$ $=151,1$ (lít).
	}
\end{ex}
\begin{ex}
	Cho hình vuông $ABCD$ cạnh $a$. Gọi $N$ là điểm thuộc cạnh $AD$ sao cho $AN=2DN$. Đường thẳng qua $N$ vuông góc với $BN$ cắt $BC$ tại $K$. Thể tích $V$ của khối tròn xoay tạo thành khi quay tứ giác $ANKB$ quanh trục $BK$ là\\
	{\color{red}HÌNH Ở ĐÂY}
	\choice
	{\True $V=\dfrac{7}{6}\pi{a^3}$}
	{$V=\dfrac{9}{14}\pi{a^3}$}
	{$V=\dfrac{6}{7}\pi{a^3}$}
	{$V=\dfrac{14}{9}\pi{a^3}$}
	\loigiai{
		{\color{red}HÌNH Ở ĐÂY}\\
		Dựng đường thẳng qua $N$ vuông góc với $BC$ cắt $BC$ tại $M$.\\
		Lấy $A'$ đối xứng với $A$ qua $B$, $N'$ đối xứng với $N$ qua $M$.\\
		Khối tròn xoay tạo thành khi quay tứ giác $ANKB$ quanh trục $BK$ gồm hai khối là khối\\
		nón có đường cao là $MK$, bán kính đáy là $MN$ và khối trụ có đường cao là $MB$, bán kính đáy là $MN$.\\
		Ta có $AN=\dfrac{2a}{3}$, $AB=a$.\\
		Xét tam giác $ABN$ vuông tại $A$: $NB=\sqrt{A{N^2}+A{B^2}}=\dfrac{a\sqrt{13}}{3}$\\
		Xét tam giác $BNK$ vuông tại $N$\\
		$\dfrac{1}{N{K^2}}=\dfrac{1}{M{N^2}}-\dfrac{1}{N{B^2}}=\dfrac{4}{13}\Rightarrow NK=\dfrac{a\sqrt{13}}{2}$\\
		$BK=\sqrt{N{K^2}+B{N^2}}=\dfrac{13a}{6}$\\
		Lại có $MB=AN=\dfrac{2a}{3}$, $MK=BK-MB=\dfrac{3a}{2}$\\
		$V=V_1+V_2=\dfrac{1}{3}\pi .M{N^2}.MK+\pi .M{N^2}.MB=\dfrac{7\pi{a^3}}{6}$.
	}
\end{ex}
\begin{ex}
	Cho một khối tròn xoay $(H)$, một mặt phẳng chứa trục của $(H)$ cắt $(H)$ theo một thiết diện như trong hình vẽ sau. Tính thể tích của $(H)$ (đơn vị $\mathrm{cm}^3$).\\
	{\color{red}HÌNH Ở ĐÂY}
	\choice
	{$V_{(H)}=13\pi$}
	{$V_{(H)}=23\pi$}
	{\True $V_{(H)}=\dfrac{41\pi}{3}$}
	{$V_{(H)}=17\pi$}
	\loigiai{
		{\color{red}HÌNH Ở ĐÂY}\\
		Ta có\\
		Thể tích của hình nón lớn là $V_1=\dfrac{1}{3}\pi{2^2}.4=\dfrac{16}{3}\pi $\\
		Thể tích của hình trụ là $V_2=\pi .\left(\dfrac{3}{2}\right)^2.4=9\pi $\\
		Thể tích của hình nón nhỏ là $V_3=\dfrac{1}{3}\pi{1^2}.2=\dfrac{2}{3}\pi $\\
		Thể tich của khối $(H)$ là $V=V_1+V_2-V_3=\dfrac{16}{3}\pi+9\pi-\dfrac{2}{3}\pi=\dfrac{41}{3}\pi $.
	}
\end{ex}
\begin{ex}
	Cho hình thang cân $ABCD$, $AB\parallel CD$, $AB=6\mathrm{cm}$, $CD=2\mathrm{cm}$, $AD=BC=\sqrt{13}\mathrm{cm}$, Quay hình thang $ABCD$ xung quanh đường thẳng $AB$ ta được một khối tròn xoay có thể tích là\\
	{\color{red}HÌNH Ở ĐÂY}
	\choice
	{$18\pi\left(\mathrm{cm}^3\right)$}
	{\True $30\pi\left(\mathrm{cm}^3\right)$}
	{$24\pi\left(\mathrm{cm}^3\right)$}
	{$12\pi\left(\mathrm{cm}^3\right)$}
	\loigiai{
		Khi quay hình thang $ABCD$ xung quanh đường thẳng $AB$ ta được một khối tròn xoay được ghép bởi 2 khối nón có thể tích bằng nhau và 1 khối trụ.\\
		{\color{red}HÌNH Ở ĐÂY}{\color{red}HÌNH Ở ĐÂY}\\
		Khối nón có được do $\Delta ADM$ và $\Delta CBN$ quay quanh đường thẳng $AB$ có cùng thể tích, với chiều cao $AM=2\left(\mathrm{cm}\right)$, bán kính đáy $DM=3\left(cm\right)$ $\Rightarrow{V_{non}}=\dfrac{1}{3}.2.\pi{3^2}\left(\mathrm{cm}^3\right)$.\\
		Khối trụ có được do hình chữ nhật $DCMN$ quay quanh đường thẳng $AB$, với chiều cao $MN=2\left(cm\right)$, bán kính đáy $DM=3\left(cm\right)$ $\Rightarrow{V_\text{trụ}}=2.\pi{3^2}\left(\mathrm{cm}^3\right)$ .\\
		Suy ra thể tích khối tròn xoay cần tính là: $V=2V_{non}+V_{tru}$ $=2.\dfrac{1}{3}.2.\pi{3^2}+2.\pi{3^2}$ $=30\pi\left(\mathrm{cm}^3\right)$.
	}
\end{ex}
\begin{ex}
	[Chuyên Long An - 2019]%Câu 44
	Cho một dụng cụ đựng chất lỏng được tạo bởi một hình trụ và hình nón được lắp đặt như hình bên. Bán kính đáy hình nón bằng bán kính đáy hình trụ. Chiều cao hình trụ bằng chiều cao hình nón và bằng $h$. Trong bình, lượng chất lỏng có chiều cao bằng $\dfrac{1}{24}$ chiều cao hình trụ. Lật ngược dụng cụ theo phương vuông góc với mặt đất. Tính độ cao phần chất lỏng trong hình nón theo $h$.
	\choice
	{$\dfrac{h}{8}$}
	{$\dfrac{3h}{8}$}
	{\True $\dfrac{h}{2}$}
	{$\dfrac{h}{4}$}
	\loigiai{
		{\color{red}HÌNH Ở ĐÂY}\\
		Thể tích chất lỏng $V=\pi{r^2}.\dfrac{1}{24}h=\dfrac{1}{24}\pi{r^2}h$.\\
		Khi lật ngược bình, thể tích phần hình nón chứa chất lỏng là $V'=\dfrac{1}{3}\pi{r'^2}{h}'$.\\
		Mà $\dfrac{r'}{r}=\dfrac{h'}{h}\Rightarrow{r}'=\dfrac{h'}{h}.r$. Do đó $V'=\dfrac{1}{3}\pi{\left(\dfrac{h'}{h}.r\right)^2}{h}'=\dfrac{1}{3}\pi{r^2}.\dfrac{h'^3}{h^2}$.\\
		Theo bài ra, $V'=V\Leftrightarrow\dfrac{1}{3}\pi{r^2}.\dfrac{h'^3}{h^2}=\dfrac{1}{24}\pi{r^2}h\Leftrightarrow{h'^3}=\dfrac{1}{8}{h^3}\Leftrightarrow{h}'=\dfrac{h}{2}$.
	}
\end{ex} 
\begin{ex}
	Có một hình chữ nhật $ABCD$ với $AB=2a$, $AD=4a$. Người ta đánh dấu $E$ là trung điểm $BC$ và $F\in AD$ sao cho $AF=a$. Sau đó người ta cuốn mảnh bìa lại sao cho cạnh $DC$ trùng cạnh $AB$ tạo thành một hình trụ. Tính thể tích tứ diện $ABEF$ với các đỉnh $A$, $B$, $E$, $F$ nằm trên hình trụ vừa tạo thành.\\
	{\color{red}HÌNH Ở ĐÂY}
	\choice
	{$\dfrac{16a^3}{3\pi^2}$}
	{\True $\dfrac{8a^3}{3\pi^2}$}
	{$\dfrac{a^3}{3\pi}$}
	{$\dfrac{8a^3}{\pi^2}$}
	\loigiai{
		{\color{red}HÌNH Ở ĐÂY}\\
		Gọi $M$ là trung điểm của cạnh $AD$, $K$ là trung điểm của $BE$. Khi cuốn tấm bìa theo yêu cầu bài toán, ta được một hình trụ có đường kính đáy là $AM$; chiều cao là $AB$; $F$, $K$ lần lượt là các điểm chính giữa các cung $AM$ và $BE$ và khối $AMF.BKE$ là khối lăng trụ đứng (minh họa ở hình trên).\\
		Đường tròn đáy có chu vi bằng $AD=4a$, suy ra bán kính đáy $r=\dfrac{2a}{\pi}$.\\
		Ta có $V_{AFM.BKE}=AB.S_{\Delta BKE}=AB.\dfrac{1}{2}r.2r=2a.\left(\dfrac{2a}{\pi}\right)^2=\dfrac{8a^3}{\pi^2}$.\\
		$V_{ABEF}=V_{EBFK}=\dfrac{1}{3}{V_{AFM.BKE}}=\dfrac{8a^3}{3\pi^2}$.
	}
\end{ex}
\begin{ex}
	[Chuyên Nguyễn Huệ - 2019]%Câu 46
	Cho hình thang $ABCD$ vuông tại $A$ và $B$ với $AB=BC=\dfrac{AD}{2}=a$. Quay hình thang và miền trong của nó quanh đường thẳng chứa cạnh $BC$. Tình thể tích V của khối tròn xoay được tạo thành.
	\choice
	{$V=\pi{a^3}$}
	{$V=\dfrac{4\pi{a^3}}{3}$}
	{\True $V=\dfrac{5\pi{a^3}}{3}$}
	{$V=\dfrac{7\pi{a^3}}{3}$}
	\loigiai{
		{\color{red}HÌNH Ở ĐÂY}\\
		Kẻ $CE\parallel AD$ và $CE=AB=BC=a$\\
		$\Rightarrow ABED$ là hình chữ nhật.\\
		Khi quay hình chữ nhật $ABED$ quanh trục $BC$ ta được hình trụ\\
		$V_t=\pi A{B^2}.AD=\pi .a^2.2a=2\pi{a^3}$.\\
		Khi quay $\Delta CED$ quanh trục $EC$ $(BC)$ ta được hình nón có\\
		$V_n=\dfrac{1}{3}\pi D{E^2}.CE=\dfrac{1}{3}\pi .a^2.a=\dfrac{1}{3}\pi{a^3}.$\\
		Thể tích của khối tròn xoay được tạo ra khi quay ABCD quanh trục BC là\\
		$V=V_t-V_n=2\pi{a^3}-\dfrac{1}{3}\pi{a^3}=\dfrac{5}{3}\pi{a^3}$.\\
		Vậy thể tích khối tròn xoay được tạo thành là $V=\dfrac{5\pi{a^3}}{3}$.
	}
\end{ex}
\begin{ex}
	Để định vị một trụ điện, người ta cần đúc một khối bê tông có chiều cao $h=1,5\mathrm{m}$ gồm:\\
	Phần dưới có dạng hình trụ bán kính đáy $R=1\text{m}$ và có chiều cao bằng $\dfrac{1}{3}h$.\\
	Phần trên có dạng hình nón bán kính đáy bằng $R$ đã bị cắt bỏ bớt một phần hình nón có bán kính đáy bằng $\dfrac{1}{2}R$ ở phía trên (người ta thường gọi hình đó là hình nón cụt).\\
	Phần ở giữa rỗng có dạng hình trụ bán kính đáy bằng $\dfrac{1}{4}R$ (tham khảo hình vẽ bên dưới).\\
	{\color{red}HÌNH Ở ĐÂY}\\
	Thể tích của khối bê tông (làm tròn đến chữ số thập phân thứ ba) bằng
	\choice
	{$2,815\mathrm{m}^3$}
	{$2,814\mathrm{m}^3$}
	{$3,403\mathrm{m}^3$}
	{\True $3,109\mathrm{m}^3$}
	\loigiai{
		Thể tích hình trụ bán kính đáy $R$ và có chiều cao bằng $\dfrac{h}{3}$:\\
		$V_1=\pi{R^2}.\dfrac{h}{3}=\dfrac{1}{3}\pi{R^2}h$.\\
		Thể tích hình nón cụt bán kính đáy lớn $R$, bán kính đáy bé $\dfrac{R}{2}$ và có chiều cao bằng $\dfrac{2h}{3}$:\\
		$V_2=\dfrac{1}{3}\pi{R^2}.\dfrac{4h}{3}-\dfrac{1}{3}\pi\dfrac{R^2}{4}.\dfrac{2h}{3}=\dfrac{7}{18}\pi{R^2}h$.\\
		Thể tích hình trụ bán kính đáy $\dfrac{R}{4}$ và có chiều cao bằng $h$ (phần rỗng ở giữa):\\
		$V_3=\pi\dfrac{R^2}{16}.h=\dfrac{1}{16}\pi{R^2}h$.\\
		Thể tích của khối bê tông bằng\\
		$V=V_1+V_2-V_3$ $=\pi{R^2}h.\left(\dfrac{1}{3}+\dfrac{7}{18}-\dfrac{1}{16}\right)$ $=\dfrac{95}{144}\pi{R^2}.h\approx 3,109\mathrm{m}^3$.
	}
\end{ex}       
\begin{ex}%Câu 48
	Cho hình thang $ABCD$ vuông tại $A$ và $B$ có $AB=a$, $AD=3a$ và $BC=x$ với $0<x<3a$. Gọi $V_1$, $V_2$, lần lượt là thể tích các khối tròn xoay tạo thành khi quay hình thang $ABCD$ (kể cả các điểm trong) quanh đường thẳng $BC$ và $AD$. Tìm $x$ để $\dfrac{V_1}{V_2}=\dfrac{7}{5}$.
	\choice
	{\True $x=a$}
	{$x=2a$}
	{$x=3a$}
	{$x=4a$}
	\loigiai{
		{\color{red}HÌNH Ở ĐÂY}\\
		Dựng các điểm $E$, $F$ để có các hình chữ nhật $ABED$ và $ABCF$ như hình vẽ.\\
		Khi quay hình thang $ABCD$ (kể các điểm trong) quanh đường thẳng $BC$ ta được khối tròn xoay có thể tích là\\
		$V_1=V_3-V_4=3\pi{a^3}-\dfrac{1}{3}\pi\left(3a-x\right){a^2}$ $=2\pi{a^3}+\dfrac{1}{3}\pi x{a^2}$ $=\dfrac{1}{3}\pi{a^2}\left(6a+x\right)$.\\
		Trong đó, $V_3$ là thể tích khối trụ tròn xoay có bán kính đáy bằng $a$ , chiều cao bằng $3a$; $V_4$ là thể tích khối nón tròn xoay có bán kính đáy bằng $a$, chiều cao bằng $3a-x$.\\
		Khi quay hình thang $ABCD$ (kể các điểm trong) quanh đường thẳng $AD$ ta được khối tròn xoay có thể tích là\\
		$V_2=V_5+V_4=\pi{a^2}x+\dfrac{1}{3}\pi\left(3a-x\right){a^2}$ $=\pi{a^3}+\dfrac{2}{3}\pi x{a^2}$ $=\dfrac{1}{3}\pi{a^2}\left(3a+2x\right)$.\\
		Trong đó, $V_5$ là thể tích khối trụ tròn xoay có bán kính đáy bằng $a$, chiều cao bằng $x$.\\
		Theo giả thiết ta có $\dfrac{V_1}{V_2}=\dfrac{7}{5}$ $\Leftrightarrow\dfrac{6a+x}{3a+2x}=\dfrac{7}{5}$ $\Leftrightarrow x=a$.
	}
\end{ex}
\begin{ex}
	[Đề thử nghiệm 2017]%Câu 49
	Cho hai hình vuông có cùng cạnh bằng $5$ được xếp chồng lên nhau sao cho đỉnh $X$ của một hình vuông là tâm của hình vuông còn lại (như hình vẽ). Tính thể tích $V$ của vật thể tròn xoay khi quay mô hình trên xung quanh trục $XY$.\\
	{\color{red}HÌNH Ở ĐÂY}
	\choice
	{$V=\dfrac{125\left(1+\sqrt{2}\right)\pi}{6}$}
	{$V=\dfrac{125\left(5+2\sqrt{2}\right)\pi}{12}$}
	{\True $V=\dfrac{125\left(5+4\sqrt{2}\right)\pi}{24}$}
	{$V=\dfrac{125\left(2+\sqrt{2}\right)\pi}{4}$}
	\loigiai{
		Cách 1:\\
		{\color{red}HÌNH Ở ĐÂY}\\
		Khối tròn xoay gồm 3 phần:\\
		Phần 1: khối trụ có chiều cao bằng $5$, bán kính đáy bằng $\dfrac{5}{2}$ có thể tích $V_1=\pi\times{\left(\dfrac{5}{2}\right)^2}\times 5=\dfrac{125\pi}{4}$\\
		Phần 2: khối nón có chiều cao và bán kính đáy bằng $\dfrac{5\sqrt{2}}{2}$ có thể tích\\
		$V_2=\dfrac{1}{3}\times\pi\times{\left(\dfrac{5\sqrt{2}}{2}\right)^2}\times\dfrac{5\sqrt{2}}{2}=\dfrac{125\pi\sqrt{2}}{12}$\\
		Phần 3: khối nón cụt có thể tích là\\
		$V_3=\dfrac{1}{3}\pi\times\dfrac{5\left(\sqrt{2}-1\right)}{2}\times\left(\left(\dfrac{5\sqrt{2}}{2}\right)^2+\left(\dfrac{5}{2}\right)^2+\dfrac{5\sqrt{2}}{2}\times\dfrac{5}{2}\right)=\dfrac{125\left(2\sqrt{2}-1\right)\pi}{24}$ .\\
		Vậy thể tích khối tròn xoay là\\
		$V=V_1+V_2+V_3=\dfrac{125\pi}{4}+\dfrac{125\pi\sqrt{2}}{12}+\dfrac{125\left(2\sqrt{2}-1\right)\pi}{24}=\dfrac{125\left(5+4\sqrt{2}\right)\pi}{24}$ .\\
		Cách 2:\\
		{\color{red}HÌNH Ở ĐÂY}\\
		Thể tích hình trụ được tạo thành từ hình vuông $ABCD$ là $V_T=\pi{R^2}h=\dfrac{125\pi}{4}$\\
		Thể tích khối tròn xoay được tạo thành từ hình vuông $XEYF$ là $V_{2N}=\dfrac{2}{3}\pi{R^2}h=\dfrac{125\pi\sqrt{2}}{6}$\\
		Thể tích khối tròn xoay được tạo thành từ tam giác $XDC$ là $V_{N'}=\dfrac{1}{3}\pi{R^2}h=\dfrac{125\pi}{24}$\\
		Thể tích cần tìm $V=V_T+V_{2N}-V_{N'}=125\pi\dfrac{5+4\sqrt{2}}{24}$.
	}
\end{ex}
\begin{ex}
	Người ta chế tạo một món đồ chơi cho tre em theo các công đoạn như sau: Trước hết chế tạo ra hình nón tròn xoay có góc ở đỉnh là $2\alpha=60^\circ $ bằng thủy tinh trong suốt. Sau đó đặt hai quả cầu nhỏ bằng thủy tinh có bán kính lớn, nhỏ khác nhau sao cho hai mặt cầu tiếp xúc với nhau và tiếp xúc với mặt nón, quả cầu lớn tiếp xúc với mặt đáy của hình nón (hình vẽ). Biết rằng chiều cao của hình nón bằng $9\mathrm{cm}$.\\
	{\color{red}HÌNH Ở ĐÂY}\\
	Bỏ qua bề dày của các lớp vỏ thủy tinh, tổng thể tích của hai khối cầu bằng
	\choice
	{\True $\dfrac{112\pi}{3}$}
	{$\dfrac{40\pi}{3}$}
	{$\dfrac{38\pi}{3}$}
	{$\dfrac{100\pi}{3}$}
	\loigiai{
		{\color{red}HÌNH Ở ĐÂY}\\
		Gọi $N,r_1$ là tâm và bán kính của mặt câu nhỏ. $M$, $r_2$ là tâm và bán kính của mặt cầu lớn.\\
		Do các mặt cầu tiếp xúc với nhau và tiếp xúc với mặt nón nên tam giác $SNC$ vuông tại $C$, tam giác $SMB$ vuông tại $B$.\\
		Hình nón tròn xoay có góc ở đỉnh là $2\alpha=60^\circ $ nên $\widehat{ASO}=30^\circ $.\\
		Ta có $r_2=SM.\sin 30^\circ\Leftrightarrow{r_2}=\dfrac{1}{2}\left(SO-r_2\right)\Leftrightarrow\dfrac{3}{2}{r_2}=\dfrac{1}{2}SO\Leftrightarrow{r_2}=\dfrac{1}{3}SO=3$.\\
		$r_1=SN.\sin 30^\circ\Leftrightarrow{r_1}=\dfrac{1}{2}\left(SO-NO\right)\Leftrightarrow{r_1}=\dfrac{1}{2}\left(SO-r_1-2r_2\right)\Leftrightarrow{r_1}=\dfrac{SO-2r_2}{3}=1$\\
		Vậy tổng thể tích hai khối cầu là $V=\dfrac{4}{3}\left(r_1^3+r_2^3\right)\pi=\dfrac{112\pi}{3}$.
	}
\end{ex}
\begin{ex}
	Người ta thả một viên billiards snooker có dạng hình cầu với bán kính nhỏ hơn $4,5$ $\mathrm{cm}$ vào một chiếc cốc hình trụ đang chứa nước thì viên billiards đó tiếp xúc với đáy cốc và tiếp xúc với mặt nước sau khi dâng (tham khảo hình vẽ bên). Biết rằng bán kính của phần trong đáy cốc bằng $5,4$ $\mathrm{cm}$ và chiều cao của mực nước ban đầu trong cốc bằng $4,5$ $\mathrm{cm}$.\\
	{\color{red}HÌNH Ở ĐÂY}\\
	Bán kính của viên billiards đó bằng
	\choice
	{$2,6$ $\mathrm{cm}$}
	{\True $2,7$ $\mathrm{cm}$}
	{$4,2$ $\mathrm{cm}$}
	{$3,6$ $\mathrm{cm}$}
	\loigiai{
		Gọi bán kính của viên billiards là $r$ $\mathrm{cm}$.\\
		Thể tích của nước trong cốc là $V_1=h_1.\pi{R^2}=4,5.\pi .(5,4)^2=\dfrac{6561}{50}\pi $.\\
		Thể tích khối trụ tạo bởi đáy cốc và mặt nước sau khi dâng là $V_2=h_2.\pi{R^2}=2r.\pi .(5,4)^2=\dfrac{1458}{25}\pi r$\\
		Thể tích khối cầu là $V_3=\dfrac{4}{3}.\pi{r^3}$\\
		Ta có $V_2-V_1=V_3\Leftrightarrow\dfrac{1458}{25}\pi r-\dfrac{6561}{50}\pi=\dfrac{4}{3}.\pi{r^3}\Leftrightarrow\dfrac{4}{3}{r^3}-\dfrac{1458}{25}r+\dfrac{6561}{50}=0\Rightarrow r=-7,5\vee r=4,8\vee r=2,7.$ Kết hợp với điều kiện suy ra bán kính của viên billiards là $2,7$ $\mathrm{cm}$.
	}
\end{ex}
\begin{ex}
	[THPT Cẩm Bình - 2019]%Câu 52
	Người ta thả một viên bi có dạng hình cầu có bán kính $2,7$ $\mathrm{cm}$ vào một chiếc cốc hình trụ đang chứa nước (tham khảo hình vẽ dưới). Biết rằng bán kính của phần trong đáy cốc bằng $5,4$ $\mathrm{cm}$ và chiều cao của mực nước ban đầu trong cốc bằng $4,5$ $\mathrm{cm}$. Khi đó chiều cao của mực nước trong cốc là?\\
	{\color{red}HÌNH Ở ĐÂY}
	\choice
	{\True $5,4$ $\mathrm{cm}$}
	{$5,7$ $\mathrm{cm}$}
	{$5,6$ $\mathrm{cm}$}
	{$5,5$ $\mathrm{cm}$}
	\loigiai{
		{\color{red}HÌNH Ở ĐÂY}\\
		Gọi $R=2,7$ $\mathrm{cm}$ là bán kính của viên bi. Ta có bán kính phần trong đáy cốc là $2R$.\\
		Thể tích nước ban đầu là $V_1=\pi{\left(2R\right)^2}.4,5=18\pi{R^2}$.\\
		Thể tích viên bi là: $V_2=\dfrac{4}{3}\pi{R^3}$.\\
		Thể tích nước sau khi thả viên bi là $V=V_1+V_2=18\pi{R^2}+\dfrac{4}{3}\pi{R^3}=2\pi{R^2}\left(9+\dfrac{2}{3}R\right)$ .\\
		Gọi $h$ là chiều cao mực nước sau khi thả viên bi vào.\\
		Ta có $V=2\pi{R^2}\left(9+\dfrac{2}{3}R\right)=\pi{\left(2R\right)^2}.h\Rightarrow h=\dfrac{2\pi{R^2}\left(9+\frac{2}{3}R\right)}{\pi{\left(2R\right)^2}}=\dfrac{\left(9+\frac{2}{3}R\right)}{2}=5.4\left(\mathrm{cm}\right)$.
	}
\end{ex}  
\begin{ex}
	Cho tam giác đều $ABC$ nội tiếp đường tròn tâm $I$ đường kính $AA'$, $M$ là trung điểm $BC$. Khi quay tam giác $ABM$ cùng với nửa đường tròn đường kính $AA'$ xung quanh đường thẳng $AM$ (như hình vẽ minh họa), ta được khối nón và khối cầu có thể tích lần lượt là $V_1$ và $V_2$. Tỷ số $\dfrac{V_1}{V_2}$ bằng\\
	{\color{red}HÌNH Ở ĐÂY}
	\choice
	{\True $\dfrac{9}{32}$}
	{$\dfrac{9}{4}$}
	{$\dfrac{27}{32}$}
	{$\dfrac{4}{9}$}
	\loigiai{
		Gọi $a$ là độ dài các cạnh của tam giác $ABC$\\
		Ta có chiều cao, bán kính của hình nón lần lượt là $h_1=AM=a\dfrac{\sqrt{3}}{2},{r_1}=\dfrac{a}{2}$. Do đó thể tích khối nón là $V_1=\dfrac{1}{3}\pi r_1^2h_1=\dfrac{\sqrt{3}{a^3}}{24}\pi $.\\
		Bán kính của hình cầu là ${r}_2=AI=\dfrac{2}{3}AM=\dfrac{\sqrt{3}}{3}a$ nên thể tích khối cầu là\\
		$V_2=\dfrac{4}{3}\pi r_2^3=\dfrac{4\sqrt{3}}{27}\pi{a^3}$.\\
		Từ đó suy ra $\dfrac{V_1}{V_2}=\dfrac{9}{32}$.
	}
\end{ex}
\begin{ex}
	Cho mặt cầu $(S)$ có bán kính bằng $2$ và có một đường tròn lớn là $(C)$. Khối nón $(N)$ có đường tròn đáy là $(C)$ và thiết diện qua trục là tam giác đều. Biết rằng phần khối nón $(N)$ chứa trong mặt cầu $(S)$ có thể tích bằng $\left(a+b\sqrt{3}\right)\pi $, với $a,b$ là các số hữu tỉ. Tính $a+b$.
	\choice
	{\True $a+b=\dfrac{14}{3}$}
	{$a+b=\dfrac{13}{3}$}
	{$a+b=\dfrac{11}{3}$}
	{$a+b=\dfrac{7}{3}$}
	\loigiai{
		{\color{red}HÌNH Ở ĐÂY}\\
		Gọi thể tích khối nón có bán kính đáy $OC$ và đường cao $OA$ là $V_1$\\
		Thể tích khối nón có bán kính đáy $IM$ và đường cao $IA$ là $V_2$\\
		Do $ABC$ là tam giác đều nên $M$ là trung điểm của $AC$ và $OA=2\sqrt{3}$, $IM=1$\\
		suy ra $IA=IO=\sqrt{3},IH=2-\sqrt{3}$\\
		Ta có $V_1^{}=\dfrac{1}{3}.\pi .O{C^2}.OA=\dfrac{1}{3}.\pi{2^2}.2\sqrt{3}=\dfrac{8\sqrt{3}}{3}\pi ,V_2^{}=\dfrac{1}{3}.\pi .I{M^2}.IA=\dfrac{1}{3}.\pi{1^2}.\sqrt{3}=\dfrac{\sqrt{3}}{3}\pi $\\
		Thể tích chỏm cầu có chiều cao $IH$ và bán kính $IM$ là\\
		$V_{C\hom}^{}=\pi .h^2(R-\dfrac{h}{3})=\pi .I{H^2}(R-\dfrac{IH}{3})=\pi .\left(2-\sqrt{3}\right)^2(2-\dfrac{2-\sqrt{3}}{3})=\left(\dfrac{16-9\sqrt{3}}{3}\right)\pi $\\
		Suy ra thể tích phần khối nón $(N)$ chứa trong mặt cầu $(S)$ là: $V=V_1^{}-V_2^{}+V_{C\hom}^{}=\dfrac{8\sqrt{3}}{3}\pi-\dfrac{\sqrt{3}}{3}\pi+\left(\dfrac{16-9\sqrt{3}}{3}\right)\pi=\left(\dfrac{16}{3}-\dfrac{2\sqrt{3}}{3}\right)\pi\Rightarrow a=\dfrac{16}{3},b=-\dfrac{2}{3}$\\
		Suy ra $a+b=\dfrac{14}{3}$.
	}
\end{ex}
\begin{ex}
	[Bình Phước - 2019]%Câu 55
	Một đồ vật được thiết kế bởi một nửa khối cầu và một khối nón úp vào nhau sao cho đáy của khối nón và thiết diện của nửa mặt cầu chồng khít lên nhau như hình vẽ bên.\\
	Biết khối nón có đường cao gấp đôi bán kính đáy, thể tích của toàn bộ khối đồ vật bằng $36\pi(\mathrm{cm}^3)$. Diện tích bề mặt của toàn bộ đồ vật đó bằng\\
	{\color{red}HÌNH Ở ĐÂY}
	\choice
	{$\pi\left(\sqrt{5}+3\right)\pi(\mathrm{cm}^2)$}
	{\True $9\pi\left(\sqrt{5}+2\right)\pi(\mathrm{cm}^2)$}
	{$9\pi\left(\sqrt{5}+3\right)\pi(\mathrm{cm}^2)$}
	{$\pi\left(\sqrt{5}+2\right)\pi(\mathrm{cm}^2)$}
	\loigiai{
		Thể tích khối nón là $V_1=\dfrac{1}{3}\pi .R^2.2R=\dfrac{2}{3}\pi .R^3$\\
		Thể tích nửa khối cầu là $V_2=\dfrac{1}{2}.\dfrac{4}{3}\pi .R^3=\dfrac{2}{3}\pi .R^3$\\
		Thể tích của toàn bộ khối đồ vật là $V_1+V_2=36\pi $ $\Leftrightarrow\dfrac{4}{3}\pi .R^3=36\pi\Leftrightarrow R=3$\\
		Diện tích xung quanh của mặt nón là $S_1=\pi R.\sqrt{4R^2+R^2}=\pi{R^2}\sqrt{5}=9\sqrt{5}\pi $\\
		Diện tích của nửa mặt cầu là $S_2=\dfrac{1}{2}.4\pi{R^2}=18\pi $\\
		Diện tích bề mặt của toàn bộ đồ vật bằng $S_1+S_2=9\pi\left(\sqrt{5}+2\right)\pi(\mathrm{cm}^2)$.
	}
\end{ex}
\begin{ex}%
	Cho một hình cầu nội tiếp hình nón tròn xoay có góc ở đỉnh là $2\alpha $, bán kính đấy là $R$ và chiều cao là $h$. Một hình trụ ngoại tiếp hình cầu đó có đáy dưới nằm trong mặt phẳng đáy của hình nón. Gọi $V_1,V_2$ lần lượt là thể tích của hình nón và hình trụ, biết rằng $V_1\ne{V_2}$. Gọi $M$ là giá trị lớn nhất của tỉ số $\dfrac{V_2}{V_1}$. Giá trị của biểu thức $P=48M+25$ thuộc khoảng nào dưới đây? (tham khảo hình vẽ)\\
	{\color{red}HÌNH Ở ĐÂY}
	\choice
	{$\left(40;60\right)$}
	{\True $\left(60;80\right)$}
	{$\left(20;40\right)$}
	{$\left(0;20\right)$}
	\loigiai{
		Gọi $r$ bán kính kính hình cầu nội tiếp hình nón.\\
		Ta có $Rh=r\left(l+R\right)\Rightarrow r=\dfrac{Rh}{R+\sqrt{R^2+h^2}}$ .\\
		Hình trụ ngoại tiếp hình cầu nên có đường kính đáy và chiều cao bằng đường kính hình cầu. Do đó nó có thể tích là $V_2=\pi{r^2}.2r=2\pi{\left(\dfrac{Rh}{R+\sqrt{R^2+h^2}}\right)^3}$.\\
		Khi đó $\dfrac{V_2}{V_1}=\dfrac{2\pi{\left(\frac{Rh}{R+\sqrt{R^2+h^2}}\right)^3}}{\frac{1}{3}\pi{R^2}h}=6\dfrac{R{h^2}}{\left(R+\sqrt{R^2+h^2}\right)^3}=\dfrac{6\left(\frac{R}{h}\right)}{\left(\frac{R}{h}+\sqrt{\left(\frac{R}{h}\right)^2+1}\right)^3}$ $=\dfrac{6t}{\left(t+\sqrt{t^2+1}\right)^3}$\\
		Với $t=\dfrac{R}{h}>0$, xét hàm số $y=\dfrac{t}{\left(t+\sqrt{t^2+1}\right)^3}$ với $t>0$, ta có\\
		$y'=\dfrac{\sqrt{t^2+1}-3t}{\left(t+\sqrt{t^2+1}\right)^3\sqrt{t^2+1}}$; $y'=0\Leftrightarrow t=\dfrac{1}{2\sqrt{2}}$.\\
		Ta có bảng biến thiên\\
		{\color{red}HÌNH Ở ĐÂY}\\
		Dựa vào bảng biến thiên suy ra $M=\max\left\{\dfrac{V_2}{V_1}\right\}=6\dfrac{1}{8}=\dfrac{3}{4}$.\\
		Do đó $P=48M+25=61$.
	}
\end{ex}
\begin{ex}
	[Hà Nội - 2018]%Câu 57
	Cho khối cầu $(S)$ tâm $I$, bán kính $R$ không đổi. Một khối trụ thay đổi có chiều cao $h$ và bán kính đáy $r$ nội tiếp khối cầu. Tính chiều cao $h$ theo $R$ sao cho thể tích của khối trụ lớn nhất.\\
	{\color{red}HÌNH Ở ĐÂY}
	\choice
	{\True $h=\dfrac{2R\sqrt{3}}{3}$}
	{$h=\dfrac{R\sqrt{2}}{2}$}
	{$h=\dfrac{R\sqrt{3}}{2}$}
	{$h=R\sqrt{2}$}
	\loigiai{
		Ta có $r^2=R^2-\dfrac{h^2}{4}$.\\
		Thể tích của khối trụ: $V=\pi\left(R^2-\dfrac{h^2}{4}\right)h$ $\Leftrightarrow V=\pi{R^2}h-\dfrac{\pi{h^3}}{4}$.\\
		Ta có $V'=\pi{R^2}-\dfrac{3\pi}{4}{h^2}$, $\Rightarrow{V}'=0\Leftrightarrow h=\dfrac{2R\sqrt{3}}{3}$.\\
		Bảng biến thiên:\\
		{\color{red}HÌNH Ở ĐÂY}\\
		Vậy thể tích khối trụ lớn nhất khi $h=\dfrac{2R\sqrt{3}}{3}$.
	}
\end{ex} 
\begin{ex}
	[Toán Học Và Tuổi Trẻ 2018]%Câu 58
	Cho tam giác $SAB$ vuông tại $A$, $\widehat{ABS}=60^\circ $, đường phân giác trong của $\widehat{ABS}$ cắt $SA$ tại điểm $I$. Vẽ nửa đường tròn tâm $I$ bán kính $IA$ (như hình vẽ). Cho $\Delta SAB$ và nửa đường tròn trên cùng quay quanh $SA$ tạo nên các khối cầu và khối nón có thể tích tương ứng $V_1$, $V_2$. Khẳng định nào dưới đây đúng?\\
	{\color{red}HÌNH Ở ĐÂY}
	\choice
	{$4V_1=9V_2$}
	{\True $9V_1=4V_2$}
	{$V_1=3V_2$}
	{$2V_1=3V_2$}
	\loigiai{
		Đặt $AB=x$ $\Rightarrow\left\{\begin{aligned}
			& IA=x\tan 30^\circ\\ 
			& SA=x\tan 60^\circ\\ 
		\end{aligned}\right.$. Chỗ này hình như cô Liên bôi xanh này:D\\
		Khối cầu: $V_1=\dfrac{4}{3}\pi{R^3}=\dfrac{4}{3}\pi I{A^3}=\dfrac{4}{3}\pi{\left(x\tan 30^\circ\right)^3}$.\\
		Khối nón $V_2=\dfrac{1}{3}\pi A{B^2}SA=\dfrac{1}{3}\pi{x^2}.\left(x\tan 60^\circ\right)$.\\
		Vậy $\dfrac{V_1}{V_2}=\dfrac{4}{9}$ hay $9V_1=4V_2$.
	}
\end{ex}
\begin{ex}
	[THPT Hoàng Hoa Thám - Hưng Yên - 2018]%Câu 59
	Người ta đặt được vào trong một hình nón hai khối cầu có bán kính lần lượt là $a$ và $2a$ sao cho các khối cầu đều tiếp xúc với mặt xung quanh của hình nón, hai khối cầu tiếp xúc với nhau và khối cầu lớn tiếp xúc với đáy của hình nón. Bán kính đáy của hình nón đã cho là
	\choice
	{$\sqrt{5}a$}
	{$3a$}
	{\True $2\sqrt{2}a$}
	{$\dfrac{8a}{3}$}
	\loigiai{
		{\color{red}HÌNH Ở ĐÂY}\\
		Gọi thiết diện qua trục của hình nón là tam giác $ABC$ với $A$ là đỉnh của hình nón và $BC$ là đường kính đáy của hình nón có tâm đáy là $I$.\\
		Gọi $M$ và $N$ lần lượt là tâm của hai khối cầu có bán kính $2a$ và $a$. $H$ và $K$ lần lượt là điểm tiếp xúc của $AC$ với hai đường tròn tâm $M$ và $N$.\\
		Ta có $NK$ là đường trung bình trong tam giác $AMH$ suy ra $N$ là trung điểm của $AM$.\\
		$AM=2MN$ $=2.3a$ $=6a$ $\Rightarrow AI=8a$.\\
		Ta lại có hai tam giác vuông $AIC$ và $AHM$ đồng dạng\\
		suy ra $\dfrac{IC}{HM}=\dfrac{AI}{AH}$ $\Leftrightarrow IC=\dfrac{8a.2a}{\sqrt{36a^2-4a^2}}$ $=2a\sqrt{2}$.\\
		Vậy bán kính hình nón là $R=2a\sqrt{2}$.
	}
\end{ex}
\begin{ex}
	[THPT Hậu Lộc 2 - TH - 2018]%Câu 60
	Cho hình nón $(N)$ có bán kính đáy $r=20(\pi\mathrm{cm})$, chiều cao $h=60(\pi\mathrm{cm})$ và một hình trụ $(T)$ nội tiếp hình nón $(N)$ (hình trụ $(T)$ có một đáy thuộc đáy hình nón và một đáy nằm trên mặt xung quanh của hình nón). Tính thể tích $V$ của hình trụ $(T)$ có diện tích xung quanh lớn nhất?
	\choice
	{\True $V=3000\pi (\mathrm{cm}^3)$}
	{$V=\dfrac{32000}{9}\pi (\mathrm{cm}^3)$}
	{$V=3600\pi (\mathrm{cm}^3)$}
	{$V=4000\pi (\mathrm{cm}^3)$}
	\loigiai{
		{\color{red}HÌNH Ở ĐÂY}\\
		Gọi độ dài bán kính hình trụ là $x$ $\mathrm{cm}$ $\left(0<x<20\right)$, chiều cao của hình trụ là $h'$.\\
		Ta có $\dfrac{h'}{h}=\dfrac{S{I}'}{SI}=\dfrac{I'{K}'}{AI}$ $\Leftrightarrow\dfrac{SI-I{I}'}{SI}=\dfrac{I'{K}'}{AI}$ $\Leftrightarrow\dfrac{h-h'}{h}=\dfrac{x}{r}$ $\Leftrightarrow\dfrac{60-h'}{60}=\dfrac{x}{20}$.\\
		$\Leftrightarrow 60-h'=3x$ $\Leftrightarrow{h}'=60-3x$.\\
		Diện tích xung quanh của hình trụ là\\
		$S=2\pi x.h'=$ $2\pi x\left(60-3x\right)$ $2\pi\left(60x-3x^2\right)$ $=2\pi\left[100-3\left(x-10\right)^2\right]$ $\le 200\pi $.\\
		Diện tích xung quanh của hình trụ lớn nhất khi $x=10$.\\
		Khi đó thể tích khối trụ là $V=\pi{x^2}.h'$ $=\pi{10^2}.30$ $=3000\pi $.
	}
\end{ex} 
\begin{ex}
	[Phan Dăng Lưu - Huế - 2018]%Câu 61
	Trong tất cả các hình chóp tứ giác đều nội tiếp hình cầu có bán kính bằng $9$. Tính thể tích $V$ của khối chóp có thể tích lớn nhất.
	\choice
	{$576\sqrt{2}$}
	{\True $576$}
	{$144\sqrt{2}$}
	{$144$}
	\loigiai{
		{\color{red}HÌNH Ở ĐÂY}\\
		Gọi $(S)$ là mặt cầu có tâm $I$ và bán kính $R=9$.\\
		Xét hình chóp tứ giác đều $S.ABCD$ có đáy $ABCD$ là hình vuông tâm $O$, cạnh $3!.2!=12$, $\left(0<a\le 9\sqrt{2}\right)$\\
		Ta có $OA=\dfrac{AC}{2}$ $=\dfrac{a\sqrt{2}}{2}$ $\Rightarrow $ $OI=\sqrt{I{A^2}-O{A^2}}$ $=\sqrt{81-\dfrac{a^2}{2}}$.\\
		Mặt khác ta lại có $SO=SI+IO$ $=9+\sqrt{81-\dfrac{a^2}{2}}$.\\
		Thể tích của khối chóp $S.ABCD$ là $V=\dfrac{1}{3}{a^2}\left(9+\sqrt{81-\dfrac{a^2}{2}}\right)$ $=3a^2+\dfrac{1}{3}{a^2}\sqrt{81-\dfrac{a^2}{2}}$ .\\
		Đặt $a^2=t$, do $0<a\le 9\sqrt{2}$ nên $0<t\le 162$\\
		Xét hàm số $f(t)=3t+\dfrac{1}{3}t\left(9+\sqrt{81-\dfrac{t}{2}}\right)$, với $0<t\le 162$ ta có $f'(t)=3+\dfrac{324-3t}{12\sqrt{81-\dfrac{t}{2}}}$ ; $f'(t)=0$ $\Leftrightarrow\sqrt{81-\dfrac{t}{2}}=\dfrac{t}{12}-9$ $\Leftrightarrow\left\{\begin{aligned}
			& t\ge 108\\ 
			& 81-\dfrac{t}{2}=\left(\dfrac{t}{12}-9\right)^2\\ 
		\end{aligned}\right.$ $\Leftrightarrow\left\{\begin{aligned}
			& t\ge 108\\ 
			&\left[\begin{aligned}
				& t=0\\ 
				& t=144\\ 
			\end{aligned}\right.\\ 
		\end{aligned}\right.$ $\Leftrightarrow t=144$.\\
		Ta có bảng biến thiên\\
		{\color{red}HÌNH Ở ĐÂY}\\
		Từ bảng biến thiên ta có $V_{\max}=576$ khi $t=144$ hay $a=12$.
	}
\end{ex}
\begin{ex}
	[Toán Học Tuổi Trẻ - 2018]%Câu 62
	Cho tam giác $ABC$ vuông ở $A$ có $AB=2AC$. $M$ là một điểm thay đổi trên cạnh $BC$. Gọi $H$, $K$ lần lượt là hình chiếu vuông góc của $M$ trên $AB$, $AC$. Gọi $V$ và $V'$ tương ứng là thể tích của vật thể tròn xoay tạo bởi tam giác $ABC$ và hình chữ nhật $MHAK$ khi quay quanh trục $AB$. Tỉ số $\dfrac{V'}{V}$ lớn nhất bằng
	\choice
	{$\dfrac{1}{2}$}
	{\True $\dfrac{4}{9}$}
	{$\dfrac{2}{3}$}
	{$\dfrac{3}{4}$}
	\loigiai{
		{\color{red}HÌNH Ở ĐÂY}\\
		Giả sử $AC=a$, $AB=2a$, $BM=x$. Ta có\\
		$BC=a\sqrt{5}$, $\sin\alpha=\dfrac{AC}{BC}=\dfrac{1}{\sqrt{5}}$, $\cos\alpha=\dfrac{2}{\sqrt{5}}$.\\
		$MH=x\sin\alpha=\dfrac{x}{\sqrt{5}}$, $HB=x\cos\alpha=\dfrac{2x}{\sqrt{5}}$, $AH=2a-\dfrac{2x}{\sqrt{5}}$.\\
		Khi quay tam giác $ABC$ quanh trục $AB$ ta được một khối nón có thể tích là\\
		$V=\dfrac{1}{3}\pi A{C^2}.AB$ $=\dfrac{2a^3\pi}{3}$.\\
		Khi quay hình chữ nhật $MHAK$ quanh trục $AB$ ta được một khối trụ có thể tích là\\
		$V'=\pi .M{H^2}.AH=\pi\dfrac{x^2}{5}\left(2a-\dfrac{2x}{\sqrt{5}}\right)$.\\
		Do đó, $\dfrac{V'}{V}=\dfrac{3}{5a^2}{x^2}-\dfrac{3}{5\sqrt{5}{a^3}}{x^3}$.\\
		Xét hàm sô $f(x)=\dfrac{3}{5a^2}{x^2}-\dfrac{3}{5\sqrt{5}{a^3}}{x^3}$ trên đoạn $\left[0;a\sqrt{5}\right]$ .\\
		Ta có $f'(x)=\dfrac{6}{5a^2}x-\dfrac{9}{5\sqrt{5}{a^3}}{x^2}$, $f'(x)=0\Leftrightarrow\left[\begin{aligned}
			& x=0\\ 
			& x=\dfrac{2\sqrt{5}a}{3}\in\left[0;\sqrt{5}\right]\\ 
		\end{aligned}\right.$.\\
		$f(0)=0$ , $f\left(a\sqrt{5}\right)=0$, $f\left(\dfrac{2a\sqrt{5}}{3}\right)=\dfrac{4}{9}$.\\
		Suy ra $\underset{\left[0;\sqrt{5}\right]}{\max}f(x)=f\left(\dfrac{2a\sqrt{5}}{3}\right)=\dfrac{4}{9}$.\\
		Vậy giá trị lớn nhất của tỉ số $\dfrac{V'}{V}$ bằng $\dfrac{4}{9}$.
	}
\end{ex}
\begin{ex}
	[THPT Nguyễn Trãi - Đà Nẵng - 2018]%Câu 63
	Xét hình trụ $(T)$ nội tiếp một mặt cầu bán kính $R$ và $S$ là diện tích thiết diện qua trục của $(T)$. Tính diện tích xung quanh của hình trụ $(T)$ biết $S$ đạt giá trị lớn nhất
	\choice
	{$S_{\mathrm{xq}}=\dfrac{2\pi{R^2}}{3}$}
	{\True $S_{\mathrm{xq}}=\dfrac{\pi{R^2}}{3}$}
	{$S_{\mathrm{xq}}=2\pi{R^2}$}
	{$S_{\mathrm{xq}}=\pi{R^2}$}
	\loigiai{
		{\color{red}HÌNH Ở ĐÂY}\\
		Gọi $x$ là bán kính của hình trụ $0<x<R$. Diện tich thiết diện là\\
		$S=2x.2\sqrt{R^2-x^2}=4x\sqrt{R^2-x^2}$.\\
		Vì $4x\sqrt{R^2-x^2}\le 2.\left(x^2+R^2-x^2\right)$ nên $S\le 2R$. Vậy $S_{\max}=2R$ khi $x=\sqrt{R^2-x}\Leftrightarrow x=\dfrac{R\sqrt{2}}{2}$.\\
		Vậy diện tích xung quanh của hình trụ là $S_{\mathrm{xq}}=2\pi\dfrac{R\sqrt{2}}{2}.2\dfrac{R\sqrt{2}}{2}=2\pi{R^2}$.
	}
\end{ex}
\begin{ex}
	[THPT Quỳnh Lưu - Nghệ An - 2018]%Câu 64
	Một khúc gỗ có dạng hình khối nón có bán kính đáy bằng $r=2{\mathrm{m}}$, chiều cao $h=6{\mathrm{m}}$. Bác thợ mộc chế tác từ khúc gỗ đó thành một khúc gỗ có dạng hình khối trụ như hình vẽ. Gọi $V$ là thể tích lớn nhất của khúc gỗ hình trụ sau khi chế tác. Tính $V$.\\
	{\color{red}HÌNH Ở ĐÂY}
	\choice
	{$V=\dfrac{32\pi}{9}\left({\mathrm{m}^2}\right)$}
	{$V=\dfrac{32}{9}\left({\mathrm{m}^2}\right)$}
	{$V=\dfrac{32\pi}{3}\left({\mathrm{m}^2}\right)$}
	{\True $V=\dfrac{32\pi}{9}\left({\mathrm{m}^2}\right)$}
	\loigiai{
		Gọi $r_t$, $h_t$ lần lượt là bán kính và chiều cao của khối trụ.\\
		Ta có $\dfrac{r_t}{2}=\dfrac{6-h_t}{6}$ $\Rightarrow{h_t}=6=3r_t$.\\
		Ta lại có $V=\pi r_t^2.h_t=\pi\left(6r_t^2-3r_t^3\right)$.\\
		Xét hàm số $f\left(r_t\right)=6r_t^2-3r_t^3$, với $r_t\in\left(0;2\right)$\\
		có $f'\left(r_t\right)=12r_t-9r_t^2$; $f'\left(r_t\right)=0\Leftrightarrow{r_t}=\dfrac{4}{3}$ (vì $r_t>0$).\\
		Bảng biến thiên\\
		{\color{red}HÌNH Ở ĐÂY}\\
		Dựa vào BBT ta có $f{\left(r_t\right)_{\max}}=\dfrac{32}{9}$ đạt tại $r_t=\dfrac{4}{3}$.\\
		Vậy $V=\dfrac{32}{9}\pi $.
	}
\end{ex}
\begin{ex}
	[THPT Thanh Miện I - Hải Dương - 2018]%Câu 65
	Cho mặt cầu $(S)$ có bán kính $R$ không đổi, hình nón $(H)$ bất kì nội tiếp mặt cầu $(S)$. Thể tích khối nón $(H)$ là $V_1$; và thể tích phần còn lại của khối cầu là $V_2$. Giá trị lớn nhất của $\dfrac{V_1}{V_2}$ bằng:
	\choice
	{$\dfrac{81}{32}$}
	{$\dfrac{76}{32}$}
	{$\dfrac{32}{81}$}
	{\True $\dfrac{32}{76}$}
	\loigiai{
		{\color{red}HÌNH Ở ĐÂY}\\
		Gọi $I$, $S$ là tâm mặt cầu và đỉnh hình nón.\\
		Gọi $H$ là tâm đường tròn đáy của hình nón và $AB$ là một đường kính của đáy.\\
		Ta có $\dfrac{V_1}{V_2}+1=\dfrac{V}{V-V_1}$. Do đó để $\dfrac{V_1}{V_2}$ đạt GTLN thì $V_1$ đạt GTLN.\\
		Trường hợp 1: Xét trường hợp $SI\le R$\\
		Khi đó thể tích của hình nón đạt GTLN khi $SI=R$ Lúc đó $V_1=\dfrac{\pi{R^3}}{3}$.\\
		Trường hợp 2: $\left(SI>R\right)I$ nằm trong tam giác $SAB$ như hình vẽ.\\
		Đặt $IH=x\left(x>0\right)$. Ta có\\
		$V_1=\dfrac{1}{3}\pi H{A^2}.SH$ $=\dfrac{1}{3}\pi\left(R^2-x^2\right)\left(R+x\right)$ $=\dfrac{\pi}{6}\left(2R-2x\right)\left(R+x\right)\left(R+x\right)$ $\le\dfrac{\pi}{6}{\left(\dfrac{4R}{3}\right)^3}=\dfrac{32\pi}{81}{R^3}$.\\
		Dấu bằng xảy ra khi $x=\dfrac{R}{3}$.\\
		Khi đó $\dfrac{V_1}{V_2}=\dfrac{V}{V-V_1}-1$ $=\dfrac{\frac{4}{3}\pi{R^3}}{\frac{4}{3}\pi{R^3}-\frac{32}{81}\pi{R^3}}-1=\dfrac{8}{19}$.
	}
\end{ex}
\begin{ex}
	[THPT Nguyễn Thị Minh Khai - Hà Tĩnh - 2018]%Câu 66
	Cho tam giác đều và hình vuông cùng có cạnh bằng 8 được xếp chồng lên nhau sao cho một đỉnh của tam giác đều trùng với tâm của hình vuông, trục của tam giác đều trùng với trục của hình vuông (như hình vẽ). Tính thể tích của vật thể tròn xoay sinh bởi hình đã cho khi quay quanh trục.\\
	{\color{red}HÌNH Ở ĐÂY}
	\choice
	{$\dfrac{16\pi\left(23+4\sqrt{3}\right)}{3}$}
	{$\dfrac{64\pi\left(17+\sqrt{3}\right)}{3}$}
	{$\dfrac{16\pi\left(17+3\sqrt{3}\right)}{9}$}
	{\True $\dfrac{64\pi\left(17+3\sqrt{3}\right)}{9}$}
	\loigiai{
		{\color{red}HÌNH Ở ĐÂY}\\
		Ta cần tìm $HM$\\
		Ta có $\dfrac{HM}{KL}=\dfrac{AH}{AK}\Leftrightarrow\dfrac{HM}{4}=\dfrac{4}{4\sqrt{3}}\Rightarrow{R}'=HM=\dfrac{4}{\sqrt{3}}$\\
		Thể tích được tính bằng thể tích trụ cộng với thể tích nón lớn trừ đi thể tích nón nhỏ phía trong.\\
		$V_\text{tru}=\pi{4^2}.8=128\pi .$\\
		$V_\text{nonlon}=\dfrac{1}{3}\pi{4^2}.4\sqrt{3}=\dfrac{64\sqrt{3}\pi}{3}$\\
		$V_\text{nonnho}=\dfrac{1}{3}\pi .\left(\dfrac{4}{\sqrt{3}}\right)^2.4=\dfrac{64\pi}{9}.$\\
		$V=V_\text{tru}+V_\text{nonlon}-V_\text{nonnho}=128\pi+\dfrac{64\sqrt{3}\pi}{3}-\dfrac{64\pi}{9}=64\left(\dfrac{17\pi+3\sqrt{3}\pi}{9}\right)$.
	}
\end{ex}
\begin{ex}
	[Chuyên Lê Hồng Phong - Nam Định - 2018]%Câu 67
	Ban đầu ta có một tam giác đều cạnh bằng $3$ (hình $1$). Tiếp đó ta chia mỗi cạnh của tam giác thành $3$ đoạn bằng nhau và thay mỗi đoạn ở giữa bằng hai đoạn bằng nó sao cho chúng tạo với đoạn bỏ đi một tam giác đều về phía bên ngoài ta được hình $2$ . Khi quay hình $2$ xung quanh trục $d$ ta được một khối tròn xoay. Tính thể tích khối tròn xoay đó.\\
	{\color{red}HÌNH Ở ĐÂY}
	\choice
	{\True $\dfrac{5\pi\sqrt{3}}{3}$}
	{$\dfrac{9\pi\sqrt{3}}{8}$}
	{$\dfrac{5\pi\sqrt{3}}{6}$}
	{$\dfrac{5\pi\sqrt{3}}{2}$}
	\loigiai{
		{\color{red}HÌNH Ở ĐÂY}\\
		Ta có thể tích khối tròn xoay tạo thành bằng 2 lần thể tích nửa trên khi cho hình $SIABK$ quay quanh trục $SK$.\\
		Tam giác $SIH$ quay quanh trục $SK$ tạo thành khối nón có $r_1=IH=\dfrac{1}{2}$; $h_1=SH=\dfrac{\sqrt{3}}{2}$.\\
		Thể tích khối nón này bằng $V_1=\dfrac{1}{3}\pi{r_1}^2h_1=\dfrac{1}{3}\pi .\dfrac{1}{4}.\dfrac{\sqrt{3}}{2}=\dfrac{\sqrt{3}\pi}{24}$\\
		Hình thang vuông $HABK$ quay quanh trục $HK$ tạo thành hình nón cụt có $R=AH=\dfrac{3}{2}$; $r=BK=1$; $h=HK=SH=\dfrac{\sqrt{3}}{2}$.\\
		Thể tích khối nón cụt này bằng $V_2=\dfrac{\pi h}{3}.\left(R^2+r^2+R.r\right)=\dfrac{\pi}{3}.\dfrac{\sqrt{3}}{2}\left(\dfrac{9}{4}+1+\dfrac{3}{2}\right)=\dfrac{19\pi\sqrt{3}}{24}$.\\
		Suy ra thể tích khối tròn xoay đã cho bằng $V=2\left(V_1+V_2\right)=\dfrac{5\sqrt{3}\pi}{3}$.
	}
\end{ex}
\begin{ex}
	[THPT Nguyễn Tất Thành - Yên Bái - 2018]%Câu 68
	Bên trong hình vuông cạnh $a$, dựng hình sao bốn cạnh đều như hình vẽ bên (các kích thước cần thiết cho như trong hình). Tính thể tích $V$ của khối tròn xoay sinh ra khi quay hình sao đó quanh trục $xy$.\\
	{\color{red}HÌNH Ở ĐÂY}
	\choice
	{\True $V=\dfrac{5\pi}{48}{a^3}$}
	{$V=\dfrac{5\pi}{16}{a^3}$}
	{$V=\dfrac{\pi}{6}{a^3}$}
	{$V=\dfrac{\pi}{8}{a^3}$}
	\loigiai{
		{\color{red}HÌNH Ở ĐÂY}\\
		Xét phần gạch chéo quay xung quanh trục $xy$.\\
		Thể tich khối nón cụt tạo thành khi cho hình thang $EDCG$ quay xung quanh trục $xy$ là: $V_1=\dfrac{\pi h}{3}\left(R^2+R.r+r^2\right)=\dfrac{\pi a}{6}\left(\dfrac{a^2}{4}+\dfrac{a^2}{8}+\dfrac{a^2}{16}\right)=\dfrac{7\pi{a^3}}{96}$.\\
		Thể tích khối nón tạo thành khi cho tam giác $FCG$ quay xung quanh trục $xy$ là: $V_2=\dfrac{1}{3}.FG.\pi .C{G^2}=\dfrac{\pi{a^3}}{48}$.\\
		Thể tích khối tòn xoay sinh ra khi cho hình gạch chéo quay xung quanh trục $xy$ là $V_3=V_1-V_2=\dfrac{5\pi{a^3}}{96}$.\\
		Vậy thể tích $V$ của khối tròn xoay sinh ra khi quay hình sao đó quanh trục $xy$ là $V=2.V_3=\dfrac{5\pi{a^3}}{48}$.
	}
\end{ex}
\begin{ex}
	[THPT Chu Văn An - Hà Nội - 2018]%Câu 69
	Bạn An có một cốc giấy hình nón có đường kính đáy là $10$ $\mathrm{cm}$ và độ dài đường sinh là $8$ $\mathrm{cm}$. Bạn dự định đựng một viên kẹo hình cầu sao cho toàn bộ viên kẹo nằm trong cốc (không phần nào của viên kẹo cao hơn miệng cốc).\\
	{\color{red}HÌNH Ở ĐÂY}\\
	Hỏi bạn An có thể đựng được viên kẹo có đường kính lớn nhất bằng bao nhiêu?
	\choice
	{$\dfrac{64}{\sqrt{39}}$ $\mathrm{cm}$}
	{$\dfrac{5\sqrt{39}}{13}$ $\mathrm{cm}$}
	{\True $\dfrac{10\sqrt{39}}{13}$ $\mathrm{cm}$}
	{$\dfrac{32}{\sqrt{39}}$ $\mathrm{cm}$}
	\loigiai{
		{\color{red}HÌNH Ở ĐÂY}\\
		Xét một tiết diện qua trục của hình nón, gồm một tam giác$ACE$ và đường tròn bán kính $r$ tiếp xúc với hai cạnh $AC,AE$ sao cho toàn bộ hình tròn nằm trong tam giác.\\
		Dễ thấy viên bi lớn nhất là viên bi có bán kính bằng bán kính đường tròn nội tiếp tam giác $ACE$.\\
		Tức là bằng $r_0=\dfrac{S_{ACE}}{\frac{1}{2}\left(AC+AE+CE\right)}=\dfrac{10.\sqrt{8^2-5^2}}{8+8+10}=\dfrac{10\sqrt{39}}{26}=\dfrac{5\sqrt{39}}{13}$.\\
		Đường kính $2r_0=\dfrac{10\sqrt{39}}{13}$.
	}
\end{ex}
\begin{ex}
	[Chuyên Nguyễn Đình Triểu - Đồng Tháp - 2018]%Câu 70
	Một trái banh và một chiếc chén hình trụ có cùng chiều cao. Người ta đặt trái banh lên hình trụ thấy phần ở bên ngoài của quả bóng có chiều cao bằng $\dfrac{3}{4}$ chiều cao của nó. Gọi $V_1$, $V_2$ lần lượt là thể tích của quả bóng và chiếc chén, khi đó:\\
	{\color{red}HÌNH Ở ĐÂY}
	\choice
	{\True $9V_1=8V_2$}
	{$3V_1=2V_2$}
	{$16V_1=9V_2$}
	{$27V_1=8V_2$}
	\loigiai{
		{\color{red}HÌNH Ở ĐÂY}\\
		Gọi $R$ là bán kính mặt cầu, $r,h$ lần lượt là bán kính đáy và chiều cao hình trụ.\\
		Theo bài ra ta có $h=2R$ và $r=\sqrt{R^2-\left(\dfrac{R}{2}\right)^2}=\dfrac{R\sqrt{3}}{2}$.\\
		$V_1=\dfrac{4}{3}\pi{R^3}$, $V_2=\pi{r^2}h=\pi\dfrac{3R^2}{4}.2R=\dfrac{3\pi{R^3}}{2}$ $\Rightarrow\dfrac{V_1}{V_2}=\dfrac{8}{9}$ hay $9V_1=8V_2$.
	}
\end{ex}
\begin{ex}
	[THPT Cụm Vũng Tàu - 2021]%Câu 71
	Cho khối nón có góc ở đỉnh của thiết diện qua trục là $\dfrac{\pi}{3}$. Một khối cầu $\left(S_1\right)$ nội tiếp trong khối nón. Gọi $\left(S_2\right)$ là khối cầu tiếp xúc với tất cả các đường sinh của nón và với $\left(S_1\right)$; $\left(S_3\right)$ là khối cầu tiếp xúc với tất cả các đường sinh của nón và với $\left(S_2\right);\ldots;\left(S_n\right)$ là khối cầu tiếp xúc với tất cả các đường sinh của nón và với $\left(S_{n-1}\right)$. Gọi $V_1;V_2;\ldots;V_n$ lần lượt là thể tích của khối cầu $\left(S_1\right)$; $\left(S_2\right);\ldots;\left(S_n\right)$ và $V$ là thể tích của khối nón. Tính giá trị của biểu thức $T=\underset{n\to+\infty}{\lim}\dfrac{V_1+V_2+\ldots+V_n}{V}$:
	\choice
	{$T=\dfrac{7}{9}$}
	{$T=\dfrac{3}{5}$}
	{\True $T=\dfrac{6}{13}$}
	{$T=\dfrac{1}{2}$}
	\loigiai{
		{\color{red}HÌNH Ở ĐÂY}\\
		Xét mặt phẳng $(P)$ chứa trục của hình nón Mặt phẳng $(P)$ cắt hình nón theo thiết diện là tam giác $SAB$ cân tại $S$ , cắt các mặt cầu $\left(S_1\right)$; $\left(S_2\right);\ldots;\left(S_n\right)$ theo thiết diện là các đường tròn tâm $I_1$ ; $I_2;\ldots;I_n$ như hình vẽ.\\
		Từ gt$\Rightarrow $ Tam giác $SAB$ đều\\
		Giả sử $R=OA=1$ $\Rightarrow $ Tam giác $SAB$ đều cạnh bằng $2$ $\Rightarrow $ $SO=\sqrt{3}$\\
		$\Rightarrow $ $R_1=I_1O=\dfrac{S_{\Delta SAB}}{\dfrac{1}{2}\left(SA+SB+AB\right)}=\dfrac{\sqrt{3}}{3}$ $\Rightarrow $ $S{O_1}=O_1I_1=I_1O=\dfrac{\sqrt{3}}{3}$\\
		Dễ thấy $\Delta S{I_1}{C_1}\sim\Delta S{I_2}{C_2}$ $\Rightarrow $ $R_2=I_2C_2=\dfrac{1}{3}{I_1}{C_1}=\dfrac{1}{3}{R_1}$\\
		Tương tự suy ra $R_n=R_1.\left(\dfrac{1}{3}\right)^{n-1}$ $\Rightarrow $ $V_n=\dfrac{4}{3}\pi R_n^3=\dfrac{4}{3}\pi R_1^3.\left(\dfrac{1}{27}\right)^{n-1}=V_1.\left(\dfrac{1}{27}\right)^{n-1}$\\
		Vậy $T=\underset{n\to+\infty}{\lim}\dfrac{V_1+V_2+\ldots+V_n}{V}=\dfrac{\frac{V_1}{1-\frac{1}{27}}}{V}=\dfrac{\frac{4}{3}\pi R_1^3.\frac{27}{26}}{\frac{1}{3}\pi{R^2}.SO}=\dfrac{6}{13}$.
	}
\end{ex}
\begin{ex}
	[Chuyên Hạ Long - Quảng Ninh - 2021]%Câu 72
	Một mặt cầu tâm $O$ nằm trên mặt phẳng đáy của hình chóp tam giác đều $S.ABC$ có tất cả các cạnh bằng nhau, các đỉnh $A,B,C$ thuộc mặt cầu. Biết bán kính mặt cầu là 1. Tính tổng độ dài $l,$ các giao tuyến của mặt cầu với các mặt bên của hình chóp thỏa mãn?
	\choice
	{$l\in\left(1;\sqrt{2}\right)$}
	{$l\in\left(2;3\sqrt{2}\right)$}
	{\True $l\in\left(\sqrt{3};2\right)$}
	{$l\in\left(\dfrac{\sqrt{3}}{2};1\right)$}
	\loigiai{
		{\color{red}HÌNH Ở ĐÂY}\\
		Gọi $D$ là trung điểm của $AB$. Kẻ $OI\perp SD$, khi đó $OI\perp\left(SAB\right)$. Suy ra $I$ là tâm đường tròn $(C)$ giao tuyến của mặt cầu tâm $O$ với mặt phẳng $\left(SAB\right)$.\\
		Gọi $M,N$ lần lượt là giao điểm của đường tròn $(C)$ với $SB,SA$; $K$ là trung điểm của $MB.$\\
		Giả sử $AB=a$, theo giả thiết $OC=1\Leftrightarrow\dfrac{a\sqrt{3}}{3}=1\Leftrightarrow a=\sqrt{3}$.\\
		Ta có $SD=CD=\dfrac{3}{2};OD=\dfrac{1}{2};SO=\sqrt{S{C^2}-O{C^2}}=\sqrt{2};$ $OI=\dfrac{SO.OD}{SD}=\dfrac{\sqrt{2}}{3}$; $ID=\dfrac{O{D^2}}{SD}=\dfrac{1}{6};$ $SI=\dfrac{4}{3}$.\\
		Gọi $r$ là bán kính của đường tròn $(C)$, ta có $r=\sqrt{1-O{I^2}}=\dfrac{\sqrt{7}}{3}$.\\
		Tam giác $SIK$ vuông tại $K$ và có $\widehat{ISK}=30^\circ $, suy ra $IK=\dfrac{1}{2}IS=\dfrac{2}{3}$.\\
		Xét tam giác $MIK$ có $\cos I=\dfrac{IK}{IM}=\dfrac{2}{\sqrt{7}}\Rightarrow\widehat{I}\approx 41^\circ\Rightarrow\widehat{MIN}\approx 38^\circ $. Khi đó chiều dài $\overset\frown{MN}$ bằng $\dfrac{38.\pi}{180}.\dfrac{\sqrt{7}}{3}=\dfrac{19\pi\sqrt{7}}{270}$.\\
		Vậy tổng độ dài $l$, các giao tuyến của mặt cầu với các mặt bên của hình chóp là $l=\dfrac{19\pi\sqrt{7}}{90}\in (\sqrt{3};2)$.
	}
\end{ex}
\begin{ex}
	[Chuyên Nguyễn Bỉnh Khiêm - Quảng Nam - 2021]%Câu 73
	Cho ba hình cầu tiếp xúc ngoài nhau từng đôi một và cùng tiếp xúc với một mặt phẳng. Các tiếp điểm của các hình cầu trên mặt phẳng lập thành một tam giác có các cạnh bằng $6,4$ và $5$. Tích bán kính của ba mặt cầu trên là
	\choice
	{$120$}
	{$225$}
	{\True $15$}
	{$40$}
	\loigiai{
		Gọi ba mặt cầu $\left(S_1\right)$, $\left(S_2\right)$ và $\left(S_3\right)$ đang xét có tâm lần lượt là $I_1$, $I_2$ và $I_3$, và bán kính tương ứng của ba hình cầu lần lượt là $R_1$, $R_2$ và $R_3$.\\
		Ba mặt cầu $\left(S_1\right),\left(S_2\right)$ và $\left(S_3\right)$ tiếp xúc với cùng một mặt phẳng lần lượt tại $A_1,A_2$ và $A_3$.\\
		Theo giả thiết, không mất tổng quát ta giả sử $A_1A_2=4,{A_1}{A_3}=5,{A_2}{A_3}=6$. Khi đó $R_1<R_2<R_3.$\\
		{\color{red}HÌNH Ở ĐÂY}\\
		Xét hình thang vuông $A_1I_1I_2A_2$, kẻ $I_1H_1\parallel {A_1}{A_2}$ thì $I_1H_1=A_1A_2=4$ và $I_1H_1\perp{I_2}{H_1}$.\\
		Trong tam giác vuông $I_1I_2H_1$ : $I_1I_2^2=I_1H_1^2+I_2H_1^2$\\
		$\Leftrightarrow{\left(R_1+R_2\right)^2}=A_1A_2^2+\left(R_2-R_1\right)^2\Leftrightarrow{R_1}{R_2}=\left(\dfrac{A_1A_2}{2}\right)^2$\\
		Tương tự, ta có $R_2R_3=\left(\dfrac{A_2A_3}{2}\right)^2$; ${R_3}{R_1}=\left(\dfrac{A_1A_3}{2}\right)^2$.\\
		Do đó $R_1R_2R_2=\dfrac{A_1A_2.A_2A_3.A_3A_1}{2.2.2}=\dfrac{4.6.5}{8}=15.$.
	}
\end{ex}
\begin{ex}
	[Mã 101 - 2022]%Câu 74
	Cho hình nón có góc ở đỉnh là $120^\circ$ và chiều cao bằng 4. Gọi $(S)$ là mặt cầu đi qua đỉnh và chứa đường tròn đáy của hình nón đã cho. Tính diện tích của $(S)$ bằng:
	\choice
	{$64\pi $}
	{\True $256\pi $}
	{$192\pi $}
	{$96\pi $}
	\loigiai{
		{\color{red}HÌNH Ở ĐÂY}{\color{red}HÌNH Ở ĐÂY}\\
		Ta có $SH=4$\\
		$AB=2AH=2.SH.\tan\widehat{ASH}=2.4.\tan{60^\circ}=8\sqrt{3}$\\
		Có $OS$ là bán kính mặt cầu cũng là bán kính đường tròn ngoại tiếp $\Delta SAB$\\
		Suy ra $2OS=\dfrac{AB}{\sin ASB}\Rightarrow OS=\dfrac{8\sqrt{3}}{2.\sin{120^\circ}}=8$\\
		Vậy diện tích mặt cầu $S=4\pi{8^2}=256\pi $.
	}
\end{ex}
\begin{ex}
	[Mã 102 - 2022]%Câu 75
	Cho hình nón có góc ở đỉnh bằng $120^\circ $ và chiều cao bằng $1$. Gọi $(S)$ là mặt cầu đi qua đỉnh và chứa đường tròn đáy của hình nón đã cho. Diện tích của $(S)$ bằng
	\choice
	{\True $16\pi $}
	{$12\pi $}
	{$4\pi $}
	{$48\pi $}
	\loigiai{
		{\color{red}HÌNH Ở ĐÂY}\\
		Xét tam giác vuông $SMO$ có $\tan\widehat{MSO}=\dfrac{OM}{OS}\Rightarrow\tan 60=\dfrac{OM}{1}\Rightarrow OM=\sqrt{3}$\\
		Kẻ đường kính $S{S}'$ của mặt cầu ngoại tiếp hình nón.\\
		Tam giác $SM{S}'$ vuông tại $M$ có $MO\perp S{S}'$\\
		$\Rightarrow M{O^2}=OS.O{S}'\Rightarrow{\left(\sqrt{3}\right)^2}=1.O{S}'\Rightarrow O{S}'=3$\\
		Vậy bán kính mặt cầu ngoại tiếp hình nón là $R=\dfrac{OS+O{S}'}{2}=\dfrac{1+3}{2}=2$\\
		Diện tích $(S)$ là $S=4\pi{R^2}=4\pi{2^2}=16\pi $.
	}
\end{ex}
\begin{ex}
	[Mã 103 - 2022]%Câu 76
	Cho hình nón có góc ở đỉnh bằng $120^\circ$ và chiều cao bằng $3$. Gọi $(S)$ là mặt cầu đi qua đỉnh và chứa đường tròn đáy của hình nón đã cho. Diện tích của $(S)$ bằng
	\choice
	{\True $144\pi $}
	{$108\pi $}
	{$48\pi $}
	{$96\pi $}
	\loigiai{
		{\color{red}HÌNH Ở ĐÂY}\\
		Gọi $H$ là tâm đáy, $AB$ là đường kính của đáy hình nón và $SC$ là đường kính của mặt cầu $(S)$. Khi đó $SH=3$ và $\widehat{ASC}=60^\circ$.\\
		$SA=\dfrac{SH}{\cos{60^\circ}}=6$ (đvdd)\\
		$S{A^2}=SH.SC\Leftrightarrow{6^2}=3.SC\Leftrightarrow SC=12$\\
		Bán kính của mặt cầu $(S)$ là $R=6$ nên diện tích của $(S)$ là $S=4\pi{6^2}=144\pi $ (đvdt).
	}
\end{ex}
\begin{ex}
	[Mã 104 - 2022]%Câu 77
	Cho hình nón có góc ở đỉnh bằng $120^\circ $ và chiều cao bằng $2$. Gọi $(S)$ là mặt cầu đi qua đỉnh và chứa đường tròn đáy của hình nón đã cho. Diện tích của $(S)$ bằng
	\choice
	{$\dfrac{16\pi}{3}$}
	{$\dfrac{64\pi}{3}$}
	{\True $64\pi $}
	{$48\pi $}
	\loigiai{
		Gọi hình nón đỉnh $A$, đường kính đáy hình nón là $BC$.\\
		Gọi $I$ là tâm mặt cầu $(S)$.\\
		{\color{red}HÌNH Ở ĐÂY}\\
		Ta có $\Delta ABC$ cân tại $A$ có $\widehat{BAC}=120^\circ $ và $AI\perp BC$ tại $O$ nên $\widehat{BAI}=60^\circ $ suy ra $\Delta IAB$ đều.\\
		Tam giác $IAB$ đều và $OB\perp IA$ tại $O$ suy ra $OB$ là đường trung tuyến của $\Delta IAB$.\\
		Mà $OA=2$ suy ra $AI=2OA=4$.\\
		Vậy diện tích mặt cầu $(S)$ là: $S=4\pi A{I^2}=64\pi $.
	}
\end{ex}
\begin{ex}
	[THPT Hương Sơn - Hà Tĩnh - 2022]%Câu 78
	Một chiếc kem Ốc quế gồm $2$ phần, phần dưới là một khối nón có chiều cao bằng ba lần đường kính đáy, phần trên là nửa khối cầu có đường kính bằng đường kính khối nón bên dưới (như hình vẽ). Thể tích phần kem phía trên bằng $50\mathrm{cm}^3$. Thể tích của cả chiếc kem bằng\\
	{\color{red}HÌNH Ở ĐÂY}
	\choice
	{\True $200\mathrm{cm}^3$}
	{$150\mathrm{cm}^3$}
	{$125\mathrm{cm}^3$}
	{$500\mathrm{cm}^3$}
	\loigiai{
		Gọi bán kính của khối cầu là $R$ $\left(R>0\right)$. Theo bài ra ta có\\
		$V_1=\dfrac{1}{2}{V_C}=50\Leftrightarrow\dfrac{4}{3}\pi{R^3}=100$ $\Leftrightarrow{R^3}=\dfrac{75}{\pi}$ .\\
		Do đó, khối nón phía dưới có bán kính $R$; $h=3.2R=6R$.\\
		Thể tích của khối nón bằng $V_2=\dfrac{1}{3}\pi{R^2}.h=\dfrac{1}{3}\pi{R^2}.6R=2\pi .R^3=2\pi .\dfrac{75}{\pi}=150\left(\mathrm{cm}^3\right)$.\\
		Vậy thể tích của cả chiếc kem bằng: $V=V_1+V_2=50+150=200\left(\mathrm{cm}^3\right)$.
	}
\end{ex}
\begin{ex}
	[Sở Bạc Liêu 2022]%Câu 79
	Trên bàn có một cốc nước hình trụ chứa đầy nước, có chiều cao bằng 3 lần đường kính của đáy; một viên bi và một khối nón đều bằng thủy tinh. Biết viên bi là một khối cầu có đường kính bằng đường kính của đường tròn đáy cốc nước. Người ta từ từ thả vào cốc nước viên bi và khối nón sao cho đỉnh khối nốn nằm trên mặt cầu (như hình vẽ) thì thấy nước trong cốc tràn ra ngoài. Tính tỉ số của lượng nước còn lại trong cốc và lượng nước ban đầu.\\
	{\color{red}HÌNH Ở ĐÂY}
	\choice
	{$\dfrac{4}{9}$}
	{\True $\dfrac{5}{9}$}
	{$\dfrac{2}{3}$}
	{$\dfrac{1}{2}$}
	\loigiai{
		Gọi $R$ là bán kính đáy của hình trụ, $h$ là chiều cao của hình trụ $\Rightarrow h=6R$.\\
		Thể tích nước ban đầu trong cốc là $V_1=\pi{R^2}.h=6\pi{R^3}$.\\
		Thể tích nước tràn ra ngoài bằng thể tích của viên bi và thể tích của khối nón $V_2=\dfrac{4}{3}\pi{R^3}+\dfrac{1}{3}\pi{R^2}.4R=\dfrac{8}{3}\pi{R^3}$.\\
		Thể tích nước còn lại trong cốc là $V_3=V_1-V_2=\dfrac{10}{3}\pi{R^3}$.\\
		Tỉ số thể tích nước còn lại và nước ban đầu là $\dfrac{V_3}{V_1}=\dfrac{5}{9}$.
	}
\end{ex}
\begin{ex}
	[Sở Lạng Sơn 2022]%Câu 80
	Một cái cột có hình dạng như hình bên (gồm một khối nón và một khối trụ ghép lại). Chiều cao đo được ghi trên hình, chu vi đáy là $20$ $\mathrm{cm}$.\\
	{\color{red}HÌNH Ở ĐÂY}\\
	Thể tích của cột bằng
	\choice
	{$\dfrac{52000}{3\pi}\left(\mathrm{cm}^3\right)$}
	{$\dfrac{5000}{3\pi}\left(\mathrm{cm}^3\right)$}
	{$\dfrac{5000}{\pi}\left(\mathrm{cm}^3\right)$}
	{\True $\dfrac{13000}{3\pi}\left(\mathrm{cm}^3\right)$}
	\loigiai{
		Gọi $r, h_1, h_2$ lần lượt là bán kính đường tròn đáy, chiều cao khối nón và chiều cao khối trụ.\\ 
		Ta có $h_1=10(\mathrm{cm}) ; h_2=40(\mathrm{cm})$ và chu vi đáy là $20 \mathrm{cm}$ suy ra $2 \pi r=20 \Leftrightarrow r=\dfrac{10}{\pi}(\mathrm{cm})$.\\ 
		Vậy, thể tích của cột là $V=\dfrac{1}{3}\pi r^2 h_1+\pi r^2 h_2=\pi\left(\dfrac{10}{\pi}\right)^2\left(\dfrac{10}{3}+40\right)=\dfrac{13000}{3 \pi}\left(\mathrm{cm}^3\right)$.\\
		Vậy thể tích của khối nón đã cho là $V=\dfrac{1}{3}\pi S O \cdot O C^2=\dfrac{10 \pi a^3}{3}$.
	}
\end{ex}
\begin{ex}
	[Sở Thái Nguyên 2022]%Câu 81
	Cho một dụng cụ đựng chất lỏng như hình 1 có phần trên là mặt xung quanh và đáy của hình trụ, phần dưới là mặt xung quanh của hình nón. Biết hình trụ có cùng bán kính đáy $R$ và cùng chiều cao $h=24(\mathrm{cm})$ với hình nón. Trong hình 1, lượng chất lỏng có chiều cao bằng $12(\mathrm{cm})$. Lật ngược dụng cụ theo phương vuông góc với mặt đất như hình 2. Khi đó chiều cao của chất lỏng trong hình 2 là\\
	{\color{red}HÌNH Ở ĐÂY}
	\choice
	{$3\mathrm{cm}$}
	{$2\mathrm{cm}$}
	{\True $1\mathrm{cm}$}
	{$4\mathrm{cm}$}
	\loigiai{
		Trong hình 1 ta gọi hình nón đỉnh $S$ có chiều cao là $h'=12$ và đường tròn đáy tâm $N$ có bán kính là $R'$\\
		Áp dụng định lí Ta-lét ta có $\dfrac{R'}{R}=\dfrac{h'}{h}\Rightarrow{R}'=\dfrac{h'}{h}R=\dfrac{1}{2}R$\\
		Thể tích của lượng chất lỏng trong hình 1 là $V=\dfrac{1}{3}\pi{R'^2}{h}'=\dfrac{1}{3}\pi\cdot\dfrac{1}{4}{R^2}12=\pi\cdot{R^2}$\\
		Trong hình 2 ta gọi hình trụ có hai đường tròn đáy có tâm lần lượt là $O$, $M$; chiều cao là $x$ có bán kính là $R$.\\
		Thể tích của lượng chất lỏng trong hình 2 là $V'=\pi{R^2}x$\\
		Theo bài ra ta có $V'=V\Rightarrow\pi{R^2}x=\pi{R^2}\Rightarrow x=1$.
	}
\end{ex}
\begin{ex}
	[Chuyên Phan Bội Châu – Nghệ An 2022]%Câu 82
	Từ một tấm tôn hình tam giác đều cạnh bằng $6\mathrm{m}$, ông $A$ cắt thành một tấm tôn hình chữ nhật và cuộn lại được một cái thùng hình trụ(như hình vẽ).\\
	{\color{red}HÌNH Ở ĐÂY}\\
	Ông $A$ làm được cái thùng có thể tích tối đa là $V$ (Vật liệu làm nắp thùng coi như không liên quan). Giá trị của $V$ thỏa mãn
	\choice
	{$V\leq 1\mathrm{m}^3$}
	{$V>3\mathrm{m}^3$}
	{\True $2\mathrm{m}^3<V\leq 3\mathrm{m}^3$}
	{$1\mathrm{m}^3<V\leq 2\mathrm{m}^3$}
	\loigiai{
		{\color{red}HÌNH Ở ĐÂY}\\
		Gọi $h$ là chiều cao và $r$ là bán kính đáy của cái thùng. Khi đó\\
		$\dfrac{3\sqrt{3}-h}{3\sqrt{3}}=\dfrac{2\pi r}{6}\Leftrightarrow r=\dfrac{3\sqrt{3}-h}{\pi\sqrt{3}}$.\\
		Vậy $V=\pi r^2h=\dfrac{1}{6\pi}(3\sqrt{3}-h)^22 h\leq\dfrac{1}{6\pi}\left(\dfrac{3\sqrt{3}-h+3\sqrt{3}-h+2 h}{3}\right)^3=\dfrac{1}{6\pi}(2\sqrt{3})^3=\dfrac{4\sqrt{3}}{\pi}\mathrm{m}^3$. $\Rightarrow 2\mathrm{m}^3<V\leq 3\mathrm{m}^3$.
	}
\end{ex}
\begin{ex}
	[THPT Cò Nòi - Sơn La 2022]%Câu 83
	Cho hình nón có độ dài đường kính đáy là $2R$, độ dài đường sinh là $R\sqrt{10}$ và hình trụ có chiều cao và đường kính đáy đều bằng $2R$, lồng vào nhau như hình vẽ.\\
	{\color{red}HÌNH Ở ĐÂY}\\
	Tỉ số thể tích phần khối nón nằm ngoài khối trụ và phần khối trụ không giao với khối nón là
	\choice
	{$\dfrac{1}{56}$}
	{$\dfrac{1}{27}$}
	{$\dfrac{1}{54}$}
	{\True $\dfrac{1}{28}$}
	\loigiai{
		{\color{red}HÌNH Ở ĐÂY}\\
		Ta có $SI=\sqrt{S{A^2}-I{A^2}}=\sqrt{10R^2-R^2}=3R\Rightarrow SE=SI-EI=R$.\\
		Mặt khác $\dfrac{SE}{SI}=\dfrac{EF}{I{A_1}}=\dfrac{1}{3}\Rightarrow EF=\dfrac{I{A_1}}{3}=\dfrac{R}{3}$\\
		Thể tích khối nón lớn (có đường cao $SI$) là $V_1=\dfrac{1}{3}\pi{R^2}.3R=\pi{R^3}$.\\
		Thể tích khối nón nhỏ (có đường cao $SE$) là $V_2=\dfrac{1}{3}\pi{\left(\dfrac{R}{3}\right)^2}.R=\dfrac{\pi{R^3}}{27}$.\\
		Thể tích phần khối giao nhau giữa khối nón và khối trụ là $V_3=V_1-V_2=\dfrac{26}{27}\pi{R^3}$.\\
		Thể tích khối trụ là là $V_4=\pi{R^2}.2R=2\pi{R^3}$. Suy ra thể tích phần khối trụ không giao với khối nón là $V=V_4-V_3=\dfrac{28}{27}\pi{R^3}$. Vậy tỉ số thể tích cần tìm là $\dfrac{V_2}{V}=\dfrac{1}{28}$.
	}
\end{ex}
\begin{ex}
	[THPT Hoàng Hoa Thám - Quảng Ninh - 2022]%Câu 84
	Cho mặt cầu $(S)$ bán kính $R$. Hình nón $(N)$ thay đổi có đỉnh và đường tròn đáy thuộc mặt cầu $(S)$. Thể tích lớn nhất của khối nón $(N)$ là:
	\choice
	{\True $\dfrac{32\pi{R^3}}{81}$}
	{$\dfrac{32R^3}{81}$}
	{$\dfrac{32R^3}{27}$}
	{$\dfrac{32\pi{R^3}}{27}$}
	\loigiai{
		{\color{red}HÌNH Ở ĐÂY}\\
		Đặt $SI=h\left(h>0\right)$.\\
		Ta có $OI=SI-SO=h-R$.\\
		Trong tam giác $OIA$ vuông tại $I$ có $IA=\sqrt{O{A^2}-O{I^2}}=\sqrt{R^2-\left(h-R\right)^2}$.\\
		Thể tích khối nón đã cho là: $V=\dfrac{1}{3}\pi I{A^2}.SI=\dfrac{1}{3}\pi h\left[R^2-\left(h-R\right)^2\right]=\dfrac{1}{3}\pi\left(2R{h^2}-h^3\right)$.\\
		Xét hàm số $f(h)=2R{h^2}-h^3$ trên $\left(0;+\infty\right)$, có $f'(h)=4Rh-3h^2$.\\
		Xét phương trình $f'(h)=0\Leftrightarrow 4Rh-3h^2=0\Leftrightarrow h=\dfrac{4R}{3}$.\\
		Bảng biến thiên\\
		{\color{red}HÌNH Ở ĐÂY}\\
		Vậy $V_{\max}=\dfrac{1}{3}\pi\dfrac{32R^3}{27}=\dfrac{32\pi{R^3}}{81}$.
	}      
\end{ex}
\begin{ex}
	[Chuyên ĐHSP Hà Nội 2022]%Câu 85
	Đặt ba viên bi bằng nhau bán kính bằng 1 vào cái lọ hình trụ. Nhận thấy các viên bi đôi một tiếp xúc nhau, đồng thời tiếp xúc với hai đáy và các đường sinh của hình trụ. Diện tích xung quanh lọ hình trụ gần nhất với giá trị nào sau đây?
	\choice
	{\True $27,08$}
	{$13,54$}
	{$18,61$}
	{$6,77$}
	\loigiai{
		{\color{red}HÌNH Ở ĐÂY}\\
		Gọi $O_1,O_2,O_3$ lần lượt là tâm của ba viên bi và $r_1=r_2=r_3=1$ là bán kính của ba viên bi đó.\\
		Theo giả thiết thì ba đường tròn lớn của ba viên bi đôi một tiếp xúc với nhau, khi đó ba điểm $O_1,O_2,O_3$ tạo thành một tam giác đều cạnh $2$. Gọi $O$ là trọng tâm của tam giác $O_1O_2O_3$ thì $O{O_1}=O{O_2}=O{O_3}=\dfrac{2}{3}.2.\dfrac{\sqrt{3}}{2}=\dfrac{2\sqrt{3}}{3}$.\\
		Cũng theo giả thiết thì ba viên bi tiếp xúc với các đường sinh của lọ hình trụ tại 3 điểm nằm trên một đường tròn bằng đường tròn đáy của lọ hình trụ. Vậy bán kính đáy của lọ hình trụ là $OM=O{O_3}+O_3M=\dfrac{2\sqrt{3}}{3}+1=\dfrac{2\sqrt{3}+3}{3}$.\\
		$S_{\mathrm{xq}}=2\pi rl=2\pi\left(\dfrac{2\sqrt{3}+3}{3}\right).2\approx 27,08$.
	}
\end{ex}
\Closesolutionfile{ans}
\indapan{10}{ans/CD24/Muc_9_10}