\Opensolutionfile{ans}[ans/CD4/Muc_9_10]
\section{Mức độ 9,10 điểm}
\setcounter{dang}{0}
\setcounter{ex}{0}
\begin{dang}{Xác định tiệm cận của hàm số $g$ khi biết bảng biến thiên hàm số $f\left(x\right)$}
\end{dang}

\begin{ex}%[2D1K4-1]%Cau 1%
	(THPT Lương Văn Can - 2018) Cho đồ thị hàm số $y=f(x)=\dfrac{3 x-1}{x-1}$. Khi đó đường thẳng nào sau đây là đường tiệm cận đứng của đồ thị hàm số $y=\dfrac{1}{f(x)-2}$ ?
	\choice
	{$x=1$}
	{$x=-2$}
	{\True $x=-1$}
	{$x=2$}
	\loigiai{
		$f(x)=2 \Leftrightarrow \dfrac{3 x-1}{x-1}=2 \Rightarrow 3 x-1=2 x-2 \Leftrightarrow x=-1$\\
		Với $y=\dfrac{1}{f(x)-2}$ ta có $\displaystyle \lim_{x \to \left(-1\right)^{-}}y=+\infty$
		
		Vậy đồ thị hàm số $y=\dfrac{1}{f(x)-2}$ có đường tiệm cận đứng $x=-1$.}
\end{ex}

\begin{ex}%[2D1K4-1]%Cau 2%
	Cho hàm số $y=f(x)$ có đồ thị như hình vẽ
	\begin{center}
		\begin{tikzpicture}[line cap=butt,line join=miter,>=stealth]
			\tikzset{declare function={xmin=-3.5;xmax=1.5;ymin=-3.5;ymax=3.5;
					f(\x)=(\x)^3+3*(\x)^2-2;
				},
				smooth,samples=450
			}
			\draw[->] (xmin,0)--(xmax,0) node[shift={(-100:7pt)},font=\normalsize]{$ x $};
			\draw[->] (0,ymin)--(0,ymax) node[shift={(190:7pt)},font=\normalsize]{$ y $};
			\fill (0,0) node[shift={(225:12pt)},font=\normalsize]{$ O $};
			\clip (xmin,ymin) rectangle (xmax,ymax);
			\draw  plot[domain=xmin:xmax] (\x, {f(\x)});
			\draw[dashed,thin](-2,0) node[below] {$-2$}--(-2,2)--(0,2) node[right] {$2$};
		\end{tikzpicture}
	\end{center}
	Số tiệm cận đứng của đồ thị hàm số $y=\dfrac{2019}{f(x)-1}$ là
	\choice
	{1}
	{2}
	{\True 3}
	{4}
	\loigiai{
		Từ đồ thị của hàm số $y=f(x)$ suy ra tập xác định của hàm số $y=f(x)$ là $D=\mathbb{R}$\\
		Do đó số đường tiệm cận đứng của đồ thị hàm số $y=\dfrac{2019}{f(x)-1}$ chính là số nghiệm của phương trình $f(x)=1$.\\
		\begin{center}
			\begin{tikzpicture}[line cap=butt,line join=miter,>=stealth]
				\tikzset{declare function={xmin=-3.5;xmax=1.5;ymin=-3.5;ymax=3.5;
						f(\x)=(\x)^3+3*(\x)^2-2;
					},
					smooth,samples=450
				}
				\draw[->] (xmin,0)--(xmax,0) node[shift={(-100:7pt)},font=\normalsize]{$ x $};
				\draw[->] (0,ymin)--(0,ymax) node[shift={(190:7pt)},font=\normalsize]{$ y $};
				\fill (0,0) node[shift={(225:12pt)},font=\normalsize]{$ O $};
				\clip (xmin,ymin) rectangle (xmax,ymax);
				\draw  plot[domain=xmin:xmax] (\x, {f(\x)});
				\draw[dashed,thin](-2,0) node[below] {$-2$}--(-2,2)--(0,2) node[right] {$2$};
				\draw (-3.5,1)--(0,1) node[above left] {$1$}--(3.5,1);
			\end{tikzpicture}
		\end{center}
		Qua đồ thị ta có: đường thẳng $y=1$ cắt đồ thị hàm số $y=f(x)$ tại 3 điểm phân biệt nên phương trình $f(x)=1$ có 3 nghiệm phân biệt.\\
		Vậy đồ thị hàm số $y=\dfrac{2019}{f(x)-1}$ có 3 đường tiệm cận đứng.}
\end{ex}

\begin{ex}%[2D1K4-1]%Cau 3%
	(Chuyên Thái Bình - 2020) Cho hàm số $f(x)$ xác định và liên tục trên $\mathbb{R} \backslash\{-1\}$ có bảng biến thiên như sau:
	\begin{center}
		\begin{tikzpicture}[scale=1,line cap=round,line join=round,>=stealth]
			\tkzTabInit[lgt=1.5,espcl=3]
			{$x$/1,$y'$/.7,$y$/3}
			{$-\infty$,$-1$,$+\infty$}
			\tkzTabLine{,-,d,-}
			\tkzTabVar{+/$2$,-D+/$-\infty$/$+\infty$,-/$-2$}
		\end{tikzpicture}
	\end{center}
	Hỏi đồ thị hàm số $y=\dfrac{1}{f(x)}$ có tất cả bao nhiêu đường tiệm cận đứng và tiệm cận ngang?
	\choice
	{\True 4}
	{3}
	{2}
	{1}
	\loigiai{
		Ta có $\displaystyle \lim_{x \to -\infty}f\left(x\right)=2 \Rightarrow \lim_{x \to -\infty}\dfrac{1}{f\left(x\right)}=\dfrac{1}{2} ; \lim_{x \to +\infty}f\left(x\right)=-2 \Rightarrow \lim_{x \to +\infty}\dfrac{1}{f\left(x\right)}=-\dfrac{1}{2}$.\\
		Suy ra đồ thị hàm số $y=\dfrac{1}{f(x)}$ có hai đường tiệm cận ngang là $y=\dfrac{1}{2}$ và $y=-\dfrac{1}{2}$.\\
		Dựa vào bảng biến thiên của hàm số $y=f(x)$ ta thấy: phương trình $f(x)=0$ có hai nghiệm phân biệt $x_{1}<-1<x_{2}$.\\
		Khi đó: $f\left(x_{1}\right)=f\left(x_{2}\right)=0$.\\
		Ta có $\heva{&\displaystyle \lim\limits_{x \to x_1^-}f\left(x\right)=0\\&f\left(x\right)>0 \text{ khi } x \to x_1^-} \Rightarrow \displaystyle \lim\limits_{x \to x_1^-}\dfrac{1}{f\left(x\right)}=+\infty$ và $\heva{&\displaystyle \lim\limits_{x \to x_2^-}f\left(x\right)=0\\&f\left(x\right)>0 \text{ khi } x \to x_2^-} \Rightarrow \displaystyle \lim\limits_{x \to x_2^-}\dfrac{1}{f\left(x\right)}=+\infty$
		Vậy đồ thị hàm số $y=\dfrac{1}{f\left(x\right)}$ có hai tiệm cận đứng là đường thẳng $x=x_1$ và $x=x_2$.
	}
\end{ex}

\begin{ex}%[2D1K4-1]%Cau 4%
	(Chuyên Vĩnh Phúc- 2020) Cho hàm số $y=f(x)$ thỏa mãn $\displaystyle \lim\limits_{x \to -\infty}f\left(x\right)=-1$ và $\displaystyle \lim\limits_{x \to +\infty}f\left(x\right)=m$.
	Có bao nhiêu giá trị thực của tham số $m$ để hàm số $y=\dfrac{1}{f(x)+2}$ có duy nhất một tiệm cận ngang. 
	\choice
	{1}
	{0}
	{\True 2}
	{Vô số}
	\loigiai{
		Ta có $\displaystyle \lim\limits_{x \to -\infty}y=\displaystyle \lim\limits_{x \to -\infty}\dfrac{1}{f(x)+2}=1 \Rightarrow$ đồ thị hàm số có tiệm cận ngang $y=1$.\\
		\textbf{TH 1:} Nếu $m=-1$ thì $\displaystyle \lim\limits_{x \to -\infty}\dfrac{1}{f(x)+2}=1$ và $\displaystyle \lim\limits_{x \to +\infty}\dfrac{1}{f(x)+2}=1$ thì đồ thị hàm số có một tiệm cận.\\
		\textbf{TH 2:} Nếu $m \neq-1$\\
		Để đồ thị hàm số có một tiệm cận ngang $\Leftrightarrow \displaystyle \lim\limits_{x \to +\infty}\dfrac{1}{f(x)+2}$ không có giá trị hữu hạn\\
		$\Leftrightarrow m+2=0 \Leftrightarrow m=-2$\\
		Vậy khi $m \in\{-2 ;-1\}$ thì đồ thị hàm số có duy nhất một tiệm cận ngang.}
\end{ex}

\begin{ex}%[2D1K4-1]%Cau 5%
	(Kim Liên - Hà Nội 2019) Cho hàm số $y=f(x)$ thỏa mãn $f(\tan x)=\cos ^{4} x$. Tìm tất cả các giá trị thực của $m$ để đồ thị hàm số $g(x)=\dfrac{2019}{f(x)-m}$ có hai tiệm cận đứng.
	\choice
	{$m<0$}
	{\True $0<m<1$}
	{$m>0$}
	{$m<1$}
	\loigiai{
		$f(\tan x)=\cos ^{4} x \Leftrightarrow f(\tan x)=\dfrac{1}{\left(1+\tan ^{2} x\right)^{2}} \Rightarrow f(t)=\dfrac{1}{\left(1+t^{2}\right)^{2}}$\\
		Hàm số $g(x)=\dfrac{2019}{f(x)-m} \Rightarrow g(x)=\dfrac{2019}{\dfrac{1}{\left(1+x^{2}\right)^{2}}-m}$\\
		Hàm số $g(x)$ có hai tiện cận đứng khi và chỉ khi phương trình $\dfrac{1}{\left(1+x^{2}\right)^{2}}-m=0$ có hai nghiệm phân biệt $\Leftrightarrow\left(1+x^{2}\right)^{2}=\dfrac{1}{m}>1 \Leftrightarrow 0<m<1$}
\end{ex}

\begin{ex}%[2D1K4-1]%Cau 6%
	(THPT Quỳnh Lưu 3 Nghệ An 2019) Cho hàm số $y=f(x)$ xác định, liên tục trên $\mathbb{R}$ và có bảng biến thiên như hình bên dưới:
	\begin{center}
		\begin{tikzpicture}[scale=1,line cap=round,line join=round,>=stealth]
			\tkzTabInit[lgt=1.5,espcl=3]
			{$x$ /1,$f(x)$/3}
			{$-\infty$, $1$, $2$, $+\infty$}
			\tkzTabVar{-/$-\infty$,+/$3$,-/$0$,+/$+\infty$}
		\end{tikzpicture}
	\end{center}
	Tổng số tiệm cận ngang và tiệm cận đứng của đồ thị hàm số $y=\dfrac{1}{2 f(x)-1}$ là:
	\choice
	{\True 4}
	{3}
	{1}
	{2}
	\loigiai{
		Đặt $h(x)=\dfrac{1}{2 f(x)-1}$.\\
		\textbf{Tiệm cận ngang:}
		Ta có $\displaystyle \lim\limits_{x \to +\infty} h(x)=\displaystyle \lim\limits_{x \to +\infty} \dfrac{1}{2 f(x)-1}=0$; $\displaystyle \lim\limits_{x \to -\infty} h(x)=\displaystyle \lim\limits_{x \to -\infty} \dfrac{1}{2 f(x)-1}=0$.\\
		Suy ra đồ thị hàm số có một đường tiệm cận ngang $y=0$.\\
		\textbf{Tiệm cận đứng:}
		Xét phương trình: $2 f(x)-1=0 \Leftrightarrow f(x)=\dfrac{1}{2}$.\\
		Dựa vào bảng biến thiên ta thấy phương trình $f(x)=\dfrac{1}{2}$ có ba nghiệm phân biệt $a, b, c$ thỏa mãn $a<1<b<2<c$.\\
		Đồng thời $\displaystyle \lim\limits_{x \to a^+} h(x)=\displaystyle \lim\limits_{x \to b^-} h(x)=\displaystyle \lim\limits_{x \to c^+} h(x)=+\infty$ nên đồ thị hàm số $y=h(x)$ có ba đường tiệm cận đứng là $x=a, x=b$ và $x=c$.\\
		Vậy tổng số tiệm cận ngang và tiệm cận đứng của đồ thị hàm số $y=h(x)$ là 4.}
\end{ex}
\begin{ex}%Câu 7%(Bình Giang-Hải Dương -2019)%
	Cho hàm số $y=f\left(x\right)$ liên tục trên $\mathbb{R}\backslash \left\{1\right\}$ và có bảng biến thiên như sau:
	\begin{center}
		\begin{tikzpicture}
			\tkzTabInit[nocadre=false,lgt=1.2,espcl=3.5,deltacl=0.6] {$x$/.7,$f'(x)$/.7,$f(x)$/1.7}
			{$-\infty$,$0$,$1$,$+\infty$}
			\tkzTabLine{,+,0,-,d,-,}
			\tkzTabVar{-/  $-2$, +/$-1$,-D+/ $-\infty$/$+\infty$, -/$0$ } 
		\end{tikzpicture}
	\end{center}
	Đồ thị $y=\dfrac{1}{2f\left(x\right)+3}$ có bao nhiêu đường tiệm cận đứng?
	\choice
	{\True $2$}{$0$}{$1$}{$3$}
	\loigiai{
		Đặt $g\left(x\right)=\dfrac{1}{2f\left(x \right)+3}$ có tử số là $1\ne 0,\forall x\in \mathbb{R}.$\\
		Ta có $2f\left(x \right)+3=0 \Leftrightarrow f\left(x \right)=-\dfrac{3}{2}$ \quad (1).
		\begin{center}
			\begin{tikzpicture}
				\tkzTabInit[nocadre=false,lgt=1.2,espcl=3.5,deltacl=0.6] {$x$/.7,$f'(x)$/.7,$f(x)$/3.5}
				{$-\infty$,$0$,$1$,$+\infty$}
				\tkzTabLine{,+,0,-,d,-,}
				\draw[double](N31) --(N33);
				
				\path 
				(N13) node[above=15pt] (A){$ -2$}
				(N23) node[above=40pt] (B){$ -1$}
				(N33) node[above left=1pt] (C){$-\infty$}
				(N32) node[below right] (D){$+\infty$}
				(N43) node[above=50pt] (E){$0$};
				\foreach \x/\y in {A/B,B/C,D/E}{
					\draw[-stealth] (\x)--(\y);
				}
				
				\draw[thick,color=blue] ($(T12)!2/3!(T13)$)--($(T22)!2/3!(T23)$);
			\end{tikzpicture}
		\end{center}
		Từ bảng biến thiên có phương trình (1) có 2 nghiệm phân biệt: ${{x}_1}\in (-\infty;0),{{x}_2}\in (0;1).$\\
		Do đó đồ thị hàm số $y=\dfrac{1}{2f\left(x\right)+3}$ có 2 đường tiệm cận đứng.}
\end{ex}
\begin{ex}%Câu 8% (Chuyên Thoại Ngọc Hầu 2018)%
	Cho hàm số $y=f\left(x\right)$ liên tục trên $\mathbb{R}\backslash \left\{1\right\}$ và có bảng biến thiên như sau:
	\begin{center}
		\begin{tikzpicture}[>=stealth]
			\tkzTabInit[nocadre=false,lgt=1.2,espcl=2.5,deltacl=0.6]
			{$x$ /0.6,$y'$ /0.6, $y$ /2}
			{$-\infty$, $-2$, $1$, $2$, $+\infty$}
			\tkzTabLine{,-,$0$,+,d,+,$0$,-,}
			\tkzTabVar{+/$+\infty$,-/$2$,+D-/$+\infty$/$-\infty$,+/$3$,-/$-\infty$}
		\end{tikzpicture}
	\end{center}
	Đồ thị hàm số $y=\dfrac{1}{2f\left(x \right)-5}$ có bao nhiêu đường tiệm cận đứng?
	\choice
	{$0$}{\True $4$}{$2$}{$1$}
	\loigiai{
		Ta có: $2f\left(x \right)-5=0\Leftrightarrow f\left(x \right)=\dfrac{5}{2}$ \quad (1).\\
		Phương trình (1) có $4$ nghiệm phân biệt ${{x}_1},{{x}_2},{{x}_3},{{x}_4}\ne 1$ và giới hạn của hàm số $y=\dfrac{1}{2f\left(x\right)-5}$ tại các điểm ${{x}_1},{{x}_2},{{x}_3},{{x}_4}$ đều bằng $\pm \infty.$\\
		Mặt khác $\underset{x\to {1^{\pm }}}{\mathop{\lim \limits{n \to +\infty}}}\dfrac{1}{2f\left(x \right)-5}=0$ nên $x=1$ không phải tiệm cận đứng.\\
		Vậy đồ thị hàm số $y=\dfrac{1}{2f\left(x \right)-5}$ có $4$ đường tiệm cận đứng.}
\end{ex}
\begin{ex}%Câu 9% (Chuyên Hưng Yên 2019)%
	Cho hàm số $y=f\left(x\right)$ có bảng biến thiên như hình dưới đây:
	\begin{center}
		\begin{tikzpicture}[font=\normalsize,t style/.style={style=solid}]
			\tkzTabInit[nocadre=true,lgt=1.2,espcl=2.5,deltacl=0.5]
			{$x$ /1, $y'$/0.75, $y$/2.5}
			{$ -\infty $,$-\dfrac{1}{2}$,$ +\infty $}
			\tkzTabLine{  , -,$0$ , +,  } 
			\path ($(N12)!0.2!(N13)$) node (A1){$ 1 $}
			($(N22)!0.9!(N23)$) node (A2){$-3$}
			($(N32)!0.2!(N33)$) node (A3){$1$};
			\foreach \x/\y in {A1/A2,A2/A3}{
				\draw[-stealth] (\x)--(\y);
			}
		\end{tikzpicture}
	\end{center}
	Tổng số tiệm cận ngang và tiệm cận đứng của đồ thị hàm số $y=\dfrac{1}{2f\left(x\right)-1}$ là
	\choice
	{$0$}{$1$}{$2$}{$\True 3$}
	\loigiai{
		Số tiệm cận đứng của đồ thị hàm số $y=\dfrac{1}{2f\left(x\right)-1}$ đúng bằng số nghiệm thực của phương trình $2f\left(x\right)-1=0 \Leftrightarrow f\left(x\right)=\dfrac{1}{2}.$\\
		Mà số nghiệm thực của phương trình $f\left(x\right)=\dfrac{1}{2}$ bằng số giao điểm của đồ thị hàm số $y=f\left(x \right)$ với đường thẳng $y=\dfrac{1}{2}.$\\
		Dựa vào bảng biến thiên ta thấy đường thẳng $y=\dfrac{1}{2}$ cắt đồ thị hàm số $y=f(x)$ tại 2 điểm phân biệt.\\
		Vậy đồ thị hàm số $y=\dfrac{1}{2f\left( x\right)-1}$ có 2 tiệm cận đứng.\\
		Lại có $\underset{x\to \pm \infty }{\mathop{\lim}}\dfrac{1}{2f\left(x \right)-1}=1 \Rightarrow$ đồ thị hàm số có một tiệm cận ngang là $y=1.$\\
		Vậy tổng số tiệm cận ngang và tiệm cận đứng của đồ thị hàm số $y=\dfrac{1}{2f\left(x\right)-1}$ là $3$.}
\end{ex}
\begin{ex}%Câu 10% (THPT Bạch Dằng Quảng Ninh 2019)%
	Cho hàm bậc ba $y=f\left(x\right)$ có đồ thị như hình vẽ bên. 
	\begin{center}
		\begin{tikzpicture}[line cap=butt,line join=miter,>=stealth]
			\tikzset{declare function={xmin=-5;xmax=1.5;ymin=-1;ymax=3.5;
					f(\x)=-1*((\x)+0.5)*((\x)+3)^2;
				},
				smooth,samples=450
			}
			\draw[->] (xmin,0)--(xmax,0) node[shift={(-100:7pt)},font=\normalsize]{$ x $};
			\draw[->] (0,ymin)--(0,ymax) node[shift={(190:7pt)},font=\normalsize]{$ y $};
			\draw(-3,0)node[below,font=\scriptsize]{$-3$};
			\draw[dashed] (-1,0)node[below,font=\scriptsize]{$-1$}--(-1,2)--(0,2)node[right,font=\scriptsize]{$2$} ;
			\fill (0,0) node[shift={(-45:10pt)},font=\normalsize]{$ O $};
			\clip (xmin,ymin) rectangle (xmax,ymax);
			\draw  plot[domain=xmin:xmax] (\x, {f(\x)});
		\end{tikzpicture}
	\end{center}
	Hỏi đồ thị hàm số $y=\dfrac{\left(x^2+4x+3 \right)\sqrt{x^2+x}}{x\left[f^2\left( x \right)-2f\left(x\right) \right]}$ có bao nhiêu đường tiệm cận đứng?
	\choice
	{$2$}{$3$}{$\True 4$}{$6$}
	\loigiai{
		\begin{center}
			\begin{tikzpicture}[line cap=butt,line join=miter,>=stealth]
				\tikzset{declare function={xmin=-5;xmax=1.5;ymin=-1;ymax=3.5;
						f(\x)=-1*((\x)+0.5)*((\x)+3)^2;
					},
					smooth,samples=450
				}
				\draw[->] (xmin,0)--(xmax,0) node[shift={(-100:7pt)},font=\normalsize]{$ x $};
				\draw[->] (0,ymin)--(0,ymax) node[shift={(190:7pt)},font=\normalsize]{$ y $};
				\draw (-3,0)node[below,font=\scriptsize]{$-3$} (-0.5,0)node[above right,font=\scriptsize]{$a$};
				
				\draw[dashed] (-1,0)node[below,font=\scriptsize]{$-1$}--(-1,2)--(0,2)node[right,font=\scriptsize]{$2$} ;
				\fill (0,0) node[shift={(-45:10pt)},font=\normalsize]{$ O $};
				\fill [gray] ($(0,0)$) circle (1.25pt) node at (0,0)[below right] {\color{gray}};
				\clip (xmin,ymin) rectangle (xmax,ymax);
				\draw[name path=one,smooth]  plot[domain=xmin:xmax] (\x, {f(\x)});
				\draw[name path=two,dashed]  plot[domain=-3.781:-0.99] (\x, 2);
				\path[name intersections={of= one and two}];
				\coordinate (A) at (intersection-1);
				\coordinate (B) at (intersection-2);
				\coordinate (C) at (intersection-3);
				\path ($(xmin,0)!(A)!(xmax,0)$) coordinate (M)
				($(xmin,0)!(B)!(xmax,0)$) coordinate (N);
				\draw[dashed] (A)--(M)node[below,font=\scriptsize]{$c$} (B)--(N)node[below,font=\scriptsize]{$b$};
			\end{tikzpicture}
		\end{center}
		Xét hàm số: $y=\dfrac{\left(x^2+4x+3\right)\sqrt{x^2+x}}{x\left[f^2\left(x \right)-2f\left(x\right) \right]}=\dfrac{\left(x+1\right)\left(x+3 \right)\sqrt{x\left(x+1\right)}}{x.f\left( x \right).\left[f\left(x\right)-2\right]}.$\\
		Điều kiện tồn tại căn $\sqrt{x^2+x}$:$\left[ \begin{aligned}
			& x\ge 0 \\ 
			& x\le -1 \\ 
		\end{aligned} \right.$\\
		Xét phương trình$x\left[f^2\left(x\right)-2f\left(x\right) \right]=0\Leftrightarrow \left[ \begin{aligned}
			& x=0 \\ 
			& f\left( x \right)=0 \\ 
			& f\left( x \right)=2 \\ 
		\end{aligned} \right.$
		\begin{itemize}
			\item Với $x=0$ ta có $\underset{x\to 0^+}{\mathop{\lim}}\,\dfrac{\left(x+1 \right)\left(x+3\right)\sqrt{x\left(x+1 \right)}}{x.f\left(x\right).\left[f\left(x \right)-2\right]}=\underset{x\to 0^+}{\mathop{\lim \limits{n \to +\infty}}}\dfrac{\left(x+1\right)\left(x+3\right)\sqrt{x+1}}{\sqrt{x}.f\left(x \right).\left[ f\left(x\right)-2\right]}=+\infty$.\\
			Suy ra $x=0$ là tiệm cận đứng.
			\item Với $f\left(x\right)=0$ $\Rightarrow x=-3$ (nghiệm bội 2) hoặc $x=a$ (loại vì $-1<a<0$).\\
			Ta có: $\underset{x\to -3^+}{\mathop{\lim \limits{n \to +\infty}}}\,\dfrac{\left(x+1\right)\left(x+3\right)\sqrt{x\left(x+1\right)}}{x.f\left(x\right).\left[f\left(x\right)-2\right]}=-\infty$ nên $x=-3$ là tiệm cận đứng.
			\item Với $f\left(x \right)=2\Rightarrow\left[\begin{aligned}
				& x=-1 \\ 
				& x=b\left( -3<b<-1 \right) \\ 
				& x=c\left( c<-3 \right) \\ 
			\end{aligned}\right.$(nghiệm bội 1).\\
			Ta có:\\
			+) $\underset{x\to b^+}{\mathop{\lim \limits{n \to +\infty}}}\,\dfrac{\left(x+1\right)\left(x+3 \right)\sqrt{x\left(x+1\right)}}{x.f\left(x\right).\left[f\left(x\right)-2\right]}=0$ $\left\{\begin{aligned}
				& \underset{x\to -1^+}{\mathop{\lim \limits{n \to +\infty}}}\,\dfrac{\left(x+1\right)\left(x+3 \right)\sqrt{x\left(x+1 \right)}}{x.f\left(x \right).\left[ f\left(x \right)-2\right]}=0 \\ 
				& \underset{x\to -1^-}{\mathop{\lim \limits{n \to +\infty}}}\,\dfrac{\left(x+1\right)\left(x+3\right)\sqrt{x\left(x+1\right)}}{x.f\left(x\right).\left[f\left(x \right)-2 \right]}=0 \\ 
			\end{aligned} \right.$ \\
			nên $x=-1$ không là tiệm cận đứng.\\
			+) $\underset{x\to b^+}{\mathop{\lim \limits{n \to +\infty}}}\,\dfrac{\left(x+1 \right)\left(x+3 \right)\sqrt{x\left(x+1\right)}}{x.f\left(x \right).\left[f\left(x \right)-2 \right]}=+\infty $ (do $x\to b^+$ thì $f\left(x\right)\to 2^+$) \\
			nên $x=b$ là tiệm cận đứng.\\
			+) $\underset{x\to c^+}{\mathop{\lim \limits{n \to +\infty}}}\,\dfrac{\left(x+1 \right)\left(x+3 \right)\sqrt{x\left(x+1 \right)}}{x.f\left( x \right).\left[f\left(x \right)-2 \right]}=+\infty $ (do $x\to c^+$ thì $f\left(x\right)\to 2^-$) \\
			nên $x=c$ là tiệm cận đứng.
		\end{itemize}
		Vậy đồ thị hàm số có $4$ tiệm cận đứng.}
\end{ex}
\begin{ex}%Câu 11% (Lý Nhân Tông - Bắc Ninh 2019)%
	Cho hàm số $f\left(x\right)=ax^3+bx^2+cx+d$ có đồ thị như hình vẽ bên. 
	\begin{center}
		\begin{tikzpicture}[line cap=butt,line join=miter,>=stealth]
			\tikzset{declare function={xmin=-1;xmax=3.5;ymin=-1.5;ymax=2.5;
					f(\x)=(4*(\x)-3)*((\x)-2)^2;
				},
				smooth,samples=450
			}
			\draw[->] (xmin,0)--(xmax,0) node[shift={(-100:7pt)},font=\normalsize]{$ x $};
			\draw[->] (0,ymin)--(0,ymax) node[shift={(190:7pt)},font=\normalsize]{$ y $};
			\fill (0,0) node[shift={(225:12pt)},font=\normalsize]{$ O $};
			\draw[dashed] (1,0)node[below,font=\scriptsize]{$1$}--(1,1)--(0,1)node[left,font=\scriptsize]{$1$}
			(2,0)node[below,font=\scriptsize]{$2$};
			\clip (xmin,ymin) rectangle (xmax,ymax);
			\draw  plot[domain=xmin:xmax] (\x, {f(\x)});
		\end{tikzpicture}
	\end{center}
	Hỏi đồ thị hàm số $g\left(x\right)=\dfrac{\left(x^2-3x+2 \right)\sqrt{x-1}}{x\left[f^2\left(x \right)-f\left(x\right) \right]}$ có bao nhiêu tiệm cận đứng?
	\choice
	{$2$}{$4$}{\True $3$}{$5$}
	\loigiai{
		\begin{itemize}
			\item \textbf{Nhận xét 1}: Với $x_0^{}\ge 1$ và $\underset{x\to x_0^+}{\mathop{\lim \limits{n \to +\infty}}}\,g\left(x\right)$ hoặc $\underset{x\to x_0^-}{\mathop{\lim}}\,g\left(x \right)$ có kết quả là $+\infty$ hoặc $-\infty$ thì $x=x_0^{}$ là tiệm cận đứng của của đồ thị hàm số $g\left(x\right)$
			\item \textbf{Nhận xét 2}: Dựa vào đồ thị hàm số $f\left(x\right)$ ta có: $f\left(x\right)=a\left(x-x_1^{} \right){{\left(x-2\right)}^2}$
		\end{itemize}
		Ta có:\quad  $x\left[f^2\left(x\right)-f\left(x \right)\right]=0\Leftrightarrow \left[ \begin{aligned}
			& x=0 \\ 
			& f\left(x\right)=0 \\ 
			& f\left(x\right)=1 \\ 
		\end{aligned} \right.$
		\begin{itemize}
			\item $f\left(x\right)=0\Leftrightarrow \left[ \begin{aligned}
				& x=x_1^{},\,0<x_1^{}<1 \\ 
				& x=2 \\ 
			\end{aligned}\right.$
			\item $f\left(x\right)=1\Leftrightarrow \left[ \begin{aligned}
				& x=1 \\ 
				& x=x_2^{}\,,\,1<x_2^{}<2 \\ 
				& x=x_3^{}\,,\,x_3^{}>2 \\ 
			\end{aligned} \right.$ suy ra $f\left(x \right)-1=a\left(x-1\right)\left(x-x_2^{} \right)\left(x-x_3^{}\right)$
		\end{itemize}
		Khi đó ta có $g\left(x \right)=\dfrac{\left(x^2-3x+2 \right)\sqrt{x-1}}{x\left[f^2\left(x \right)-f\left(x\right) \right]}=\dfrac{\left(x-1\right)\left(x-2 \right)\sqrt{x-1}}{x.f\left(x\right)\left[ f\left(x\right)-1\right]}.$\\
		$g\left(x\right)=\dfrac{\left(x-1\right)\left(x-2\right)\sqrt{x-1}}{x.a\left(x-x_1^{} \right){{\left(x-2\right)}^2}.a\left(x-1 \right)\left(x-x_2^{}\right)\left(x-x_3^{} \right)}=\dfrac{\sqrt{x-1}}{a^2x\left( x-x_1^{}\right)\left(x-2\right)\left( x-x_2^{}\right)\left(x-x_3^{}\right)}$\\
		$x=0\,,\,x=x_1^{}$ không phải tiệm cận đứng của đồ thị hàm số $y=g\left( x \right)$ không thỏa mãn điều kiện $x_0^{}\ge 1$.\\
		Đồ thị hàm số $g\left( x \right)$ có $3$ đường tiệm cận đứng là: $x=2,\,x=x_2^{},\,x=x_3^{}$}
\end{ex}
\begin{ex}%Câu 12% (THPT Quỳnh Lưu- Nghệ An- 2019) %
	Cho hàm số bậc ba $f\left(x\right)=ax^3+bx^2+cx+d$ có đồ thị như hình vẽ sau.
	\begin{center}
		\begin{tikzpicture}[line cap=butt,line join=miter,>=stealth]
			\tikzset{declare function={xmin=-1;xmax=3.5;ymin=-1.5;ymax=2.5;
					f(\x)=(4*(\x)-3)*((\x)-2)^2;
				},
				smooth,samples=450
			}
			\draw[->] (xmin,0)--(xmax,0) node[shift={(-100:7pt)},font=\normalsize]{$ x $};
			\draw[->] (0,ymin)--(0,ymax) node[shift={(190:7pt)},font=\normalsize]{$ y $};
			\fill (0,0) node[shift={(225:12pt)},font=\normalsize]{$ O $};
			\draw[dashed] (1,0)node[below,font=\scriptsize]{$1$}--(1,1)--(0,1)node[left,font=\scriptsize]{$1$}
			(2,0)node[below,font=\scriptsize]{$2$};
			\clip (xmin,ymin) rectangle (xmax,ymax);
			\draw  plot[domain=xmin:xmax] (\x, {f(\x)});
		\end{tikzpicture}
	\end{center}
	Hỏi đồ thị hàm số $g\left(x\right)=\dfrac{\left(x^2-3x+2 \right)\sqrt{x-1}}{\left(x+1\right)\left[ f^2\left(x\right)-f\left(x\right)\right]}$ có bao nhiêu tiệm cận đứng?
	\choice
	{$5$}{$4$}{$6$}{\True $3$}
	\loigiai{
		Ta có $g\left(x\right)=\dfrac{\left(x-1 \right)\left(x-2\right)\sqrt{x-1}}{\left( x+1\right)f\left(x\right)\left[f\left(x \right)-1\right]}.$
		ĐKXĐ: $\left\{\begin{aligned}
			& x\ge 1 \\ 
			& f\left( x \right)\ne 0 \\ 
			& f\left( x \right)\ne 1 \\ 
		\end{aligned} \right.$\\
		Dựa vào đồ thị hàm số $y=f\left(x \right)$, ta có:
		\begin{itemize}
			\item 
			$f\left(x \right)=0\Leftrightarrow \left[ \begin{aligned}
				& x=2 \\ 
				& x={{x}_1} \\ 
			\end{aligned}\right.$. Với $x=2$ là nghiệm kép, ${{x}_1}\in \left(0;\,1\right)$
			\item $f\left(x\right)=1\Leftrightarrow \left[ \begin{aligned}
				& x=1 \\ 
				& x={{x}_2} \\ 
				& x={{x}_3} \\ 
			\end{aligned}\right.$. Với ${{x}_2}\in \left(1;\,2\right);\,{{x}_3}>2$
		\end{itemize}
		Vậy 
		\begin{eqnarray*}
			g(x)&=& \dfrac{(x-1 )(x-2)\sqrt{x-1}}{a^2(x+1){{(x-2)}^2}(x-{x}_1 )(x-1)(x-{x}_2 )(x-{x}_3)}\\
			&=&\dfrac{\sqrt{x-1}}{a^2\left(x+1\right)\left(x-2\right)\left(x-{{x}_1} \right)\left( x-{{x}_2}\right)\left(x-{{x}_3}\right)}			
		\end{eqnarray*}
		Vậy đồ thị hàm số có $3$ TCĐ $x=2;\,x={{x}_2};\,x={{x}_3}$ (do $x\ge 1$ nên ta loại $x=-1;\,x={{x}_1}$).}
\end{ex}
\begin{ex}%Câu 13% (THPT Thuận Thành 3 - Bắc Ninh 2019)%
	Cho hàm số $y=f(x)$ là hàm số đa thức có đồ thì như hình vẽ dưới đây.
	\begin{center}
		\begin{tikzpicture}[line cap=butt,line join=miter,>=stealth]
			\tikzset{declare function={xmin=-2.5;xmax=2.5;ymin=-1;ymax=5;
					f(\x)=(\x)^3-3*(\x)+2;
				},
				smooth,samples=450
			}
			\draw[->] (xmin,0)--(xmax,0) node[shift={(-100:7pt)},font=\normalsize]{$ x $};
			\draw[->] (0,ymin)--(0,ymax) node[shift={(190:7pt)},font=\normalsize]{$ y $};
			\fill (0,0) node[shift={(225:12pt)},font=\normalsize]{$ O $};
			\draw[dashed] (-1,0)node[below,font=\scriptsize]{$-1$}--(-1,4)--(0,4)node[right,font=\scriptsize]{$4$} (0,2)node[right,font=\scriptsize]{$2$} (1,0)node[below,font=\scriptsize]{$1$};
			\clip (xmin,ymin) rectangle (xmax,ymax);
			\draw  plot[domain=xmin:xmax] (\x, {f(\x)});
		\end{tikzpicture}
	\end{center}
	Đặt $g\left(x\right)=\dfrac{x^2-x}{f^2\left(x\right)-2f\left(x\right)}.$ Hỏi đồ thị hàm số $y=g\left(x\right)$ có bao nhiêu tiệm cận đứng?
	\choice
	{$5$}{$3$}{\True $4$}{$2$}
	\loigiai{
		Ta xét phương trình $f^2\left(x\right)-2f\left(x\right)=0\Leftrightarrow \left[\begin{aligned}
			& f\left( x \right)=0 \\ 
			& f\left( x \right)=2 \\ 
		\end{aligned}\right.\Leftrightarrow \left[ \begin{aligned}
			& \left[\begin{aligned}
				& x=1\text{ } \\ 
				& x={{x}_1}<-1 \\ 
			\end{aligned} \right. \\ 
			& \left[ \begin{aligned}
				& x=0 \\ 
				& x={{x}_2}>1 \\ 
				& x={{x}_3}<-1,{{x}_3}\ne {{x}_1} \\ 
			\end{aligned} \right. \\ 
		\end{aligned} \right.$\\
		Khi đó
		$g\left(x\right)=\dfrac{x^2-x}{ax{{\left( x-1\right)}^2}\left(x-{{x}_1}\right)\left( x-{{x}_2}\right)\left(x-{{x}_3} \right)}=\dfrac{1}{a\left(x-1\right)\left( x-{{x}_1}\right)\left(x-{{x}_2} \right)\left(x-{{x}_3}\right)};\text{ }\left(a\ne 0\right)$
		Vậy đồ thị hàm số $y=g\left(x\right)$ có $4$ đường tiệm cận đứng.}
\end{ex}
\begin{ex}%[2D1G4-1]
	[THPT Thuận Thành 3 - Bắc Ninh 2019] Cho hàm số $y=f(x)$ là hàm số đa thức có đồ thì như hình vẽ dưới đây, đặt $g(x)=\dfrac{x^2-x}{f^2(x)-2f(x)}$. Hỏi đồ thị hàm số   có bao nhiêu tiệm cận đứng?
	\begin{center}
		\begin{tikzpicture}[line join=round, line cap=round,>=stealth,thick,scale=0.8,every node/.style={scale=0.8}]
			\def\f(#1){(#1)^3-3*(#1)+2}
			\def\a{-1}
			\pgfmathsetmacro\fa{\f(\a)}	
			;
			
			\draw[->] (-2.5,0)--(2.5,0) node[below] {$x$};
			\draw[->] (0,-0.8)--(0,4.8) node[right] {$y$};
			\draw (0,0) node [below left] {$O$};
			
			\foreach \x/\gx in {-1/-90,1/90} \draw[thin] (\x,1pt)--(\x,-1pt) ($(\x,0)+(\gx:3mm)$)node{$\x$}
			;
			\foreach \x/\gy in {2/180,4/135} \draw[thin] (1pt,\x)--(-1pt,\x) ($(0,\x)+(\gy:3mm)$)node{$\x$}
			;
			
			\draw[dashed,thin] (\a,0)|-(0,\fa) ;
			
			\draw[samples=100,domain=-2.1:2.1,smooth,variable=\x] plot (\x,{\f(\x)}) node[right]{$y=f(x)$};
		\end{tikzpicture}
	\end{center}
	
	\choice
	{$5$}
	{$3$}
	{\True $4$}
	{$2$}
	\loigiai{
		Ta xét phương trình:\\
		$f^2(x)-2f(x)=0\Leftrightarrow \hoac{& f(x)=0 \\ & f(x)=2} \Leftrightarrow \hoac{& \hoac{& x=1 \\ & x=x_1<-1} \\ & \hoac{& x=0 \\ & x=x_2>1 \\ & x=x_3<-1,x_3 \ne x_1}}$. Khi đó:\\
		$g(x)=\dfrac{x^2-x}{ax(x-1)^2(x-x_1)(x-x_2)(x-x_3)}=\dfrac{1}{a(x-1)(x-x_1)(x-x_2)(x-x_3)}$; $(a\ne 0)$.\\
		Vậy đồ thị hàm số $g(x)$ có $4$ đường tiệm cận đứng.
	}
\end{ex}

\begin{ex}%[2D1G4-1]
	[Chuyên Bắc Giang 2019] Cho hàm số $y=f(x)$ xác định, liên tục trên $\mathbb{R}$ và có bảng biến thiên như hình bên dưới
	\begin{center}
		\begin{tikzpicture}
			\tkzTabInit[espcl=2.5,lgt=1.5,nocadre]
			{$x$/0.7,$y'$/0.7,$y$/2.1}
			{$-\infty$,$1$,$2$,$+\infty$}
			\tkzTabLine{,+,0,-,0,+,}
			\tkzTabVar{-/$-\infty$,+/$3$,-/$0$,+/$+\infty$}
		\end{tikzpicture}
	\end{center}
	Tổng số tiệm cận ngang và tiệm cận đứng của đồ thị hàm số $y=\dfrac{1}{f(x^3+x)+3}$ là
	\choice
	{\True $2$}
	{$4$}
	{$3$}
	{$1$}
	\loigiai{
		* Tính tiệm cận ngang.\\
		Ta có: $x^3+x \xrightarrow[]{x\to +\infty} +\infty \Rightarrow \underset{x\to +\infty}{\mathop{\lim}}\dfrac{1}{f(x^3+x)+3}=0$\\
		$x^3+x \xrightarrow[]{x\to -\infty} -\infty \Rightarrow \underset{x\to -\infty}{\mathop{\lim}}\dfrac{1}{f(x^3+x)+3}=0$\\
		Vậy đồ thị hàm số có $1$ tiệm cận ngang $y=0$.\\
		* Tính tiệm cận đứng.\\
		Số đường tiệm cận đứng của đồ thị hàm số là số nghiệm của phương trình $f(x^3+x)+3=0$.\\
		Dựa vào bảng biến thiên ta có $f(x^3+x)+3=0 \Leftrightarrow f(x^3+x)=-3\Leftrightarrow x^3+x=x_0;x_0\in (-\infty;1)$\\
		Vì hàm số $y=x^3+x$ đồng biến trên $\mathbb{R}$ do đó $x^3+x=x_0;x_0\in (-\infty;1)$ có một nghiệm duy nhất.\\
		Vậy đồ thị hàm số $y=\dfrac{1}{f(x^3+x)+3}$ có $1$ tiệm cận đứng.
	}
\end{ex}

\begin{ex}%[2D1G4-1]
	[THPT Minh Khai 2020] Cho hàm số $y=f(x)$ có đồ thị như bên dưới.
	\begin{center}
		\begin{tikzpicture}[line join=round, line cap=round,>=stealth,thick,scale=0.8,every node/.style={scale=0.8}]
			\def\f(#1){(#1)^3-3*(#1)^2+1}
			\def\a{2}
			\pgfmathsetmacro\fa{\f(\a)}
			
			;
			
			\draw[->] (-1.5,0)--(3.5,0) node[below] {$x$};
			\draw[->] (0,-4.2)--(0,1.8) node[right] {$y$};
			\draw (0,0) node [below left] {$O$};
			
			\foreach \x/\gx in {-1/90,1/80,2/90,3/-90} \draw[thin] (\x,1pt)--(\x,-1pt) ($(\x,0)+(\gx:3mm)$)node{$\x$}
			;
			\foreach \x/\gy in {-3/150,-2/180,-1/180,1/180} \draw[thin] (1pt,\x)--(-1pt,\x) ($(0,\x)+(\gy:3mm)$)node{$\x$}
			;
			
			\draw[dashed,thin] (\a,0)|-(0,\fa) ;
			
			\draw[samples=100,domain=-1.1:3.1,smooth,variable=\x] plot (\x,{\f(\x)}) node[right]{$y=f(x)$};
		\end{tikzpicture}
	\end{center}
	Hỏi đồ thị hàm số $y=\dfrac{(x^2-2x)\sqrt{2-x}}{(x-3)\left[ f^2(x)-f(x)\right] }$ có bao nhiêu đường tiệm cận đứng?
	\choice
	{$4$}
	{$6$}
	{\True $3$}
	{$5$}
	\loigiai{
		Ta có $y'=3ax^2+2bx+c$.\\
		Dựa vào đồ thị hàm số, ta thấy hàm số đạt cực trị tại $x=0$, $x=2$. Do đó, ta có hệ:\\
		$\heva{& y(0)=1 \\ & y(2)=-3 \\ & y'(0)=0 \\ & y'(2)=2} \Leftrightarrow \heva{& d=1 \\ & c=0 \\ & 12a+4b=0 \\ & 8a+4b=-4} \Leftrightarrow \heva{& a=1 \\ & b=-3 \\ & c=0 \\ & d=1}$.\\
		Vậy $y=f(x)=x^3-3x^2+1$. Khi đó:\\
		$y=\dfrac{(x^2-2x)\sqrt{2-x}}{(x-3)\left[ f^2(x)-f(x)\right] } = \dfrac{(x^2-2x)\sqrt{2-x}}{(x-3)(x^3-3x^2+1)(x^3-3x^2)} = \dfrac{(x^2-2x)\sqrt{2-x}}{x^2(x-3)^2(x^3-3x^2+1)}$.\\
		Ta có: $x^2(x-3)^2(x^3-3x^2+1)=0\Leftrightarrow \hoac{& x=0 \\ & x=3 \\ & x=x_1\in (-1;0) \\ & x=x_2\in (0;1) \\ & x=x_3\in (2;3)}$.\\
		Hàm số $y=\dfrac{(x^2-2x)\sqrt{2-x}}{(x-3)\left[ f^2(x)-f(x)\right] }$  có tập xác định $\mathscr{D}=(-\infty;2] \setminus \left\lbrace 0;x_1;x_2\right\rbrace $.\\
		$\underset{x\to 0^+}{\mathop{\lim}}\dfrac{(x^2-2x)\sqrt{2-x}}{x^2(x-3)^2(x^3-3x^2+1)}=\underset{x\to 0^+}{\mathop{\lim}}\dfrac{x(x-2)\sqrt{2-x}}{x^2(x-3)^2(x^3-3x^2+1)}=\underset{x\to 0^+}{\mathop{\lim}}\dfrac{(x-2)\sqrt{2-x}}{x(x-3)^2(x^3-3x^2+1)}=-\infty$. Suy ra $x=0$ là đường tiệm cận đứng.\\
		$\underset{x\to x_1^+}{\mathop{\lim}}\dfrac{(x^2-2x)\sqrt{2-x}}{x^2(x-3)^2(x^3-3x^2+1)}=+\infty$, $\underset{x\to x_2^+}{\mathop{\lim}}\dfrac{(x^2-2x)\sqrt{2-x}}{x^2(x-3)^2(x^3-3x^2+1)}=+\infty$.\\
		Suy ra $x=x_1$ và $x=x_2$ cũng là các đường tiệm cận đứng của đồ thị hàm số.
	}
\end{ex}

\begin{ex}%[2D1G4-1]
	[Yên Phong 1 - 2018] Cho hàm số $y=ax^3+bx^2+cx+d$, $a\ne 0$ có đồ thị như hình dưới đây.
	\begin{center}
		\begin{tikzpicture}[line join=round, line cap=round,>=stealth,thick,scale=0.8,every node/.style={scale=0.8}]
			\def\f(#1){(#1)^3-3*(#1)-2}
			\def\a{1}
			\pgfmathsetmacro\fa{\f(\a)}	
			;
			
			\draw[->] (-2.5,0)--(2.5,0) node[below] {$x$};
			\draw[->] (0,-4.8)--(0,1.8) node[right] {$y$};
			\draw (0,0) node [below left] {$O$};
			
			\foreach \x/\gx in {-2/90,-1/80,1/90,2/135} \draw[thin] (\x,1pt)--(\x,-1pt) ($(\x,0)+(\gx:3mm)$)node{$\x$}
			;
			\foreach \x/\gy in {-4/180,-2/180} \draw[thin] (1pt,\x)--(-1pt,\x) ($(0,\x)+(\gy:3mm)$)node{$\x$}
			;
			
			\draw[dashed,thin] (\a,0)|-(0,\fa) ;
			
			\draw[samples=100,domain=-2.1:2.1,smooth,variable=\x] plot (\x,{\f(\x)}) node[right]{$y=f(x)$};
		\end{tikzpicture}
	\end{center}
	Hỏi đồ thị hàm số $g(x)=\dfrac{\sqrt{f(x)}}{(x+1)^2(x^2-4x+3)}$ có bao nhiêu đường tiệm cận đứng?
	\choice
	{$2$}
	{\True $1$}
	{$3$}
	{$4$}
	\loigiai{
		Điều kiện xác định: $\heva{& f(x)\ge 0 \\ & x\ne -1 \\ & x^2-4x+3\ne 0} \Leftrightarrow \heva{& x\ge 2 \\ & x\ne -1 \\ & x\ne 1 \\ & x\ne 3} \Leftrightarrow \heva{& x \ge 2 \\ & x\ne 3}$.\\
		Ta có: $\underset{x\to 3^+}{\mathop{\lim}}g(x)=\underset{x\to 3^+}{\mathop{\lim}}\dfrac{\sqrt{f(x)}}{(x+1)^2(x^2-4x+3)}=+\infty$ và $\underset{x\to 3^-}{\mathop{\lim}}g(x)=\underset{x\to 3^-}{\mathop{\lim}}\dfrac{\sqrt{f(x)}}{(x+1)^2(x^2-4x+3)}=-\infty$.
		Vậy đồ thị hàm số $g(x)=\dfrac{\sqrt{f(x)}}{(x+1)^2(x^2-4x+3)}$ có một đường tiệm cận đứng là: $x=3$.
	}
\end{ex}


\begin{ex}%[2D1G4-1]
	[Chuyên Quang Trung - 2020] Cho hàm số trùng phương $y=ax^4+bx^2+c$ có đồ thị như hình vẽ.
	\begin{center}
		\begin{tikzpicture}[line join=round, line cap=round,>=stealth,thick,scale=0.8,every node/.style={scale=0.8}]
			\def\f(#1){0.24*(#1)^4-1.98*(#1)^2+1}
			\def\a{2}
			\def\b{-2}
			\pgfmathsetmacro\fa{\f(\a)}
			\pgfmathsetmacro\fb{\f(\b)}
			;
			
			\draw[->] (-3.5,0)--(3.5,0) node[below] {$x$};
			\draw[->] (0,-3.5)--(0,2) node[right] {$y$};
			\draw (0,0) node [below left] {$O$};
			
			\foreach \x/\gx in {-2/90,2/90} \draw[thin] (\x,1pt)--(\x,-1pt) ($(\x,0)+(\gx:3mm)$)node{$\x$}
			;
			\foreach \x/\gy in {-3/180,1/135} \draw[thin] (1pt,\x)--(-1pt,\x) ($(0,\x)+(\gy:3mm)$)node{$\x$}
			;
			
			\draw[dashed,thin] (\a,0)|-(0,\fa) (\b,0)|-(0,\fb) ;
			
			\draw[samples=100,domain=-2.9:2.9,smooth,variable=\x] plot (\x,{\f(\x)}) node[right]{$y=f(x)$};
		\end{tikzpicture}
	\end{center}
	Hỏi đồ thị hàm số $y=\dfrac{(x^2-4)(x^2+2x)}{\left[ f(x)\right]^2+2f(x)-3}$ có tổng cộng bao nhiêu tiệm cận đứng?
	\choice
	{$5$}
	{$2$}
	{$3$}
	{\True $4$}
	\loigiai{
		$y=\dfrac{(x^2-4)(x^2+2x)}{\left[ f(x)\right]^2+2f(x)-3}=\dfrac{x(x+2)^2(x-2)}{\left[ f(x)\right]^2+2f(x)-3}$.\\
		Ta có: $\left[ f(x)\right]^2+2f(x)-3=0\Leftrightarrow \hoac{& f(x)=1 \\ & f(x)=-3} \Leftrightarrow \hoac{& x=m(m<-2) \\ & x=0 \\ & x=n(n>2) \\ &x=2 \\ & x=-2}$.\\
		Dựa vào đồ thị ta thấy các nghiệm $x=0,x=\pm 2$ là các nghiệm kép (nghiệm bội 2) và đa thức $\left[ f(x)\right]^2+2f(x)-3$ có bậc là 8 nên $y=\dfrac{x(x+2)^2(x-2)}{a^2x^2(x+2)^2(x-2)^2(x-m)(x-n)}$.\\
		Vậy hàm số có các tiệm cận đứng là $x=0,x=2,x=m,x=n$.
	}
\end{ex}

\begin{ex}%[2D1G4-1]
	[Chuyên Quang Trung - Bình Phước 2021] Cho hàm số $y=f(x)$ có đạo hàm liên tục trên $\mathbb{R}$. Đồ thị $y=f(x)$ như hình vẽ.
	\begin{center}
		\begin{tikzpicture}[line join=round, line cap=round,>=stealth,thick,scale=0.8,every node/.style={scale=0.8}]
			\def\f(#1){(#1)^3-3*(#1)+2}
			\def\a{-1}
			\pgfmathsetmacro\fa{\f(\a)}
			;
			
			\draw[->] (-2.5,0)--(2.5,0) node[below] {$x$};
			\draw[->] (0,-0.8)--(0,4.8) node[right] {$y$};
			\draw (0,0) node [below left] {$O$};
			
			\foreach \x/\gx in {-2/-45,-1/-90,1/90,2/-90} \draw[thin] (\x,1pt)--(\x,-1pt) ($(\x,0)+(\gx:3mm)$)node{$\x$}
			;
			\foreach \x/\gy in {2/180,4/135} \draw[thin] (1pt,\x)--(-1pt,\x) ($(0,\x)+(\gy:3mm)$)node{$\x$}
			;
			
			\draw[dashed,thin] (\a,0)|-(0,\fa) ;
			
			\draw[samples=100,domain=-2.1:2.1,smooth,variable=\x] plot (\x,{\f(\x)}) node[right]{$y=f(x)$};
		\end{tikzpicture}
	\end{center}
	Số đường tiệm cận đứng của đồ thị hàm số $y=\dfrac{x^2+x-2}{f^2(x)-f(x)}$ là
	\choice
	{\True $4$}
	{$3$}
	{$2$}
	{$5$}
	\loigiai{
		Xét hàm số $y=\dfrac{x^2+x-2}{f^2(x)-f(x)}=\dfrac{(x-1)(x+2)}{f(x)\left[ f(x)-1\right] }$.\\
		Xét phương trình $f(x)\left[ f(x)-1\right]=0\Leftrightarrow \hoac{& f(x)=0 \\ & f(x)=1}$.\\
		Với $f(x)=0\Leftrightarrow \hoac{& x=1(\text{nghiệm kép}) \\ & x=-2(\text{nghiệm đơn})}\Rightarrow x=1$ là TCĐ, $x=-2$ không là TCĐ.\\
		Với $f(x)=1\Leftrightarrow \hoac{& x=0 \\ & x=x_1\in (0;1) \\ & x=x_2\in (-2;-1)}\Rightarrow x=0,x=x_1,x=x_2$ đều là các đường TCĐ.\\
		Vậy đồ thị hàm số có 4 đường TCĐ.
	}
\end{ex}

\begin{ex}%[2D1G4-1]
	[Chuyên Lê Quý Đôn - Điện Biên 2021] Cho hàm số $y=f(x)$ xác định trên $\mathbb{R}$, có bảng biến thiên như hình vẽ.
	\begin{center}
		\begin{tikzpicture}
			\tkzTabInit[espcl=2.5,lgt=1.5,nocadre]
			{$x$/0.7,$y'$/0.7,$y$/2.1}
			{$-\infty$,$0$,$+\infty$}
			\tkzTabLine{,+,0,-,}
			\tkzTabVar{-/$0$,+/$1$,-/$0$}
		\end{tikzpicture}
	\end{center}
	Với giá trị nào của $m$ thì đồ thị hàm số $y=\dfrac{1}{f^2(x)-m}$ có tổng số đường tiệm cận ngang và tiệm cận đứng bằng 3.
	\choice
	{$0<m\le 1$}
	{$0\le m \le 1$}
	{\True $0<m<1$}
	{$m =0$}
	\loigiai{
		Ta có: $\underset{x\to \pm \infty}{\mathop{\lim}}y=\underset{x\to \pm \infty}{\mathop{\lim}}\dfrac{1}{f^2(x)-m}=\dfrac{1}{-m}$ vì $\underset{x\to \pm \infty}{\mathop{\lim}}f(x)=0$. Do đó:\\
		Nếu $m=0$ thì đồ thị hàm số $y=\dfrac{1}{f^2(x)-m}$ không có tiệm cận ngang.\\
		Mặt khác phương trình $f^2(x)-m=0\Leftrightarrow f(x)=0$ vô nghiệm nên đồ thị hàm số không có tiệm cận đứng.\\
		Nếu $m\ne 0$ thì đồ thị hàm số $y=\dfrac{1}{f^2(x)-m}$ có một đường tiệm cận ngang là $y=-\dfrac{1}{m}$.\\
		* $m<0$; Phương trình $f^2(x)-m=0$ vô nghiệm nên đồ thị hàm số không có tiệm cận đứng.\\
		* $m>0$: Phương trình $f^2(x)-m=0\Leftrightarrow \hoac{& f(x)=\sqrt{m} \\ & f(x)=-\sqrt{m}}$. Dựa vào bảng biến thiên ta thấy phương trình $f(x)=-\sqrt{m}$ vô nghiệm với $\forall m>0$.\\
		Vậy đồ thị hàm số có 3 đường tiệm cận $\Leftrightarrow$ phương trình $f(x)=\sqrt{m}$ có hai nghiệm phân biệt $\Leftrightarrow 0<m<1$.\\
		Vậy $0<m<1$ thì đồ thị hàm số $y=\dfrac{1}{f^2(x)-m}$ có $3$ tiệm cận
	}
\end{ex}

\begin{ex}%[2D1G4-1]
	[THPT Đồng Quan - Hà Nội 2021] Cho hàm số $y=f(x)$ có đạo hàm liên tục trên $\mathbb{R}$ và có bảng biến thiên như hình vẽ.
	\begin{center}
		\begin{tikzpicture}[yscale=.6,xscale=1.2]
			\begin{scope}[shift={(-0.5,0.5)}]
				\draw
				%(0,0.5) rectangle +(12,-11)
				(0,-1.5)--+(0:12) (0,-3.5)--+(0:12)  (1,0.5)--+(-90:11);
			\end{scope}
			\path
			(0,0) node{$x$} % --- dòng 1
			++(0:1) node{$-\infty$}
			++(0:1) node{$$}
			++(0:1) node{$-1$}
			++(0:1) node{$$}
			++(0:1) node{$1$}
			++(0:1) node{$$}
			++(0:1) node{$2$}
			++(0:1) node{$$}
			++(0:1) node{$4$}
			++(0:2) node{$+\infty$}
			%--------------------------
			(0,-2) node{$f'(x)$} % --- dòng 2
			++(0:1) node{$ $}
			++(0:1) node{$-$}
			++(0:1) node{$0$}
			++(0:1) node{$+$}
			++(0:1) node{$0$}
			++(0:1) node{$-$}
			++(0:1) node{$0$}
			++(0:1) node{$+$}
			++(0:1) node{$0$}
			++(0:1) node{$-$}
			%--------------------------
			(0,-6.5) node{$f(x)$} % --- dòng 3
			++(0:1) +(-90:1) node (I) {$2$}
			++(0:2) +(-90:2) node (A) {$-8$}
			++(0:2) +(90:2) node (B) {$8$}
			++(0:2) +(90:0) node (C) {$4$}
			++(0:2) +(90:3) node (D) {$10$}
			++(0:2) +(-90:2.5) node[above] (E) {$-\infty$} ;
			\foreach \p/\q in {I/A,A/B,B/C,C/D,D/E}
			\draw[-stealth,blue] (\p)--(\q);
			%\draw (1,-6)--(11,-6);
			
		\end{tikzpicture}
	\end{center}
	Số đường tiệm cận (đứng và ngang) của đồ thị hàm số $y=\dfrac{1}{\left[ f(x+1)-4\right] \sqrt{x^2-4}}$ là
	\choice
	{$5$}
	{$2$}
	{$3$}
	{\True $4$}
	\loigiai{
		Điều kiện: $\hoac{& x^2-4>0 \\ & f(x+1)\ne 4}\Leftrightarrow \hoac{& x<-2 \vee x>2 \\ & f(x+1)\ne 4}$.\\
		Xét $f(x+1) = 4\Leftrightarrow \hoac{& x+1=\alpha\in (-1;1) \\ & x+1=2 \\ & x+1=\beta\in (4;+\infty)}\hoac{& x=\alpha -1\in(-2;0) (\text{loại}) \\ & x=1 (\text{loại}) \\ & x=\beta -1\in (3;+\infty) (\text{nhận})}$.\\
		Khi đó:\\
		* $\underset{x\to (-2)^-}{\mathop{\lim}y}=-\infty$, $\underset{x\to 2^+}{\mathop{\lim}y}=+\infty$, $\underset{x\to (\beta -1)^-}{\mathop{\lim}y}=+\infty$, $\underset{x\to (\beta -1)^+}{\mathop{\lim}y}=-\infty$ nên đồ thị hàm số có $3$ tiệm cận đứng.\\
		* $\underset{x\to -\infty}{\mathop{\lim}y}=0$, $\underset{x\to +\infty}{\mathop{\lim}y}=0$ nên đồ thị có $1$ tiệm cận ngang.
	}
\end{ex}

\begin{ex}%[2D1G4-1]
	[Liên trường Nghệ An - 2021] Cho $f(x)$ là hàm đa thức bậc ba và có đồ thị như hình vẽ.
	\begin{center}
		\begin{tikzpicture}[line join=round, line cap=round,>=stealth,thick,scale=0.8,every node/.style={scale=0.8}]
			\def\f(#1){-(#1)^3+3*(#1)^2-2}
			\def\a{2}
			\pgfmathsetmacro\fa{\f(\a)}
			;
			
			\draw[->] (-2,0)--(3.5,0) node[below] {$x$};
			\draw[->] (0,-2.8)--(0,2.8) node[right] {$y$};
			\draw (0,0) node [below left] {$O$};
			
			\foreach \x/\gx in {-1/-90,1/-90,2/-90,3/-90} \draw[thin] (\x,1pt)--(\x,-1pt) ($(\x,0)+(\gx:3mm)$)node{$\x$}
			;
			\foreach \x/\gy in {-2/-150,2/180} \draw[thin] (1pt,\x)--(-1pt,\x) ($(0,\x)+(\gy:3mm)$)node{$\x$}
			;
			
			\draw[dashed,thin] (\a,0)|-(0,\fa) ;
			
			\draw[samples=100,domain=-1.1:3.1,smooth,variable=\x] plot (\x,{\f(\x)}) node[right]{$y=f(x)$};
		\end{tikzpicture}
	\end{center}
	Có bao nhiêu giá trị nguyên của tham số $m$ thuộc đoạn $\left[ -100;100\right] $ để đồ thị hàm số $y=\dfrac{\sqrt{1+mx^2}}{f(x)-m}$ có đúng hai đường tiệm cận.
	\choice
	{$100$}
	{\True $99$}
	{$2$}
	{$196$}
	\loigiai{
		\textbf{Trường hợp 1:} $m=0\Rightarrow y=\dfrac{1}{f(x)}$. Đồ thị hàm số có một TCN $y=0$ và ba tiệm cận đứng nên $m=0$ không thoả mãn.\\
		\textbf{Trường hợp 2:} $m<0$. Đồ thị hàm số không có TCN.\\
		Yêu cầu bài toán $\Leftrightarrow f(x)=m$ có nghiệm, trong đó có đúng hai nghiệm thoả mãn $1+mx^2\ge 0$, mà $m$ là số nguyên nên dựa vào đồ thị ta chỉ cần xét $m\in \left\lbrace -2;-1\right\rbrace $.\\
		* Với $m=-2\Rightarrow y=\dfrac{\sqrt{1-2x^2}}{f(x)+2}$. Khi đó $f(x)=-2$ có hai nghiệm $x_1=0,x_2=a>2$. Nghiệm $x_2$ không thoả mãn điều kiện $1+mx^2\ge 0$ nên $m=-2$ không thoả mãn.\\
		* Với $m=-1\Rightarrow y=\dfrac{\sqrt{1-x^2}}{f(x)+1}$. Khi đó $f(x)=-1$ có hai nghiệm $x_1=b\in (-1;0), x_2=c\in (0;1)$. cả hai nghiệm đều thoả mãn điều kiện $1-x^2\ge 0$ nên $m=-1$ thoả mãn.\\
		\textbf{Trường hợp 3:} $m>0$. Khi đó $1+mx^2\ge 0, \forall x\in \mathbb{R}$.\\
		Đồ thị hàm số có một TCN $y=0$.\\
		Yêu cầu bài toán $\Leftrightarrow f(x)=m$ có đúng một nghiệm $x\in \mathbb{R} \Leftrightarrow m>2$. \\
		Vì $m$ nguyên thuộc đoạn $\left[ -100;100\right] \Leftrightarrow \heva{& m\in \mathbb{Z} \\ & m\in \left[ 3;100\right] \cup \left\lbrace -1\right\rbrace }$ nên có $99$ giá trị.
	}
\end{ex}

\begin{ex}%[2D1G4-1]
	[Chuyên Bắc Ninh- 2022] Cho hàm số bậc ba $f(x)=ax^3+bx^2+cx+d$ có đồ thị như hình vẽ.
	\begin{center}
		\begin{tikzpicture}[line join=round, line cap=round,>=stealth,thick,scale=0.8,every node/.style={scale=0.8}]
			\def\f(#1){(#1)^3-3*(#1)^2+1}
			\def\a{2}
			\pgfmathsetmacro\fa{\f(\a)}
			;
			
			\draw[->] (-2,0)--(3.5,0) node[below] {$x$};
			\draw[->] (0,-3.8)--(0,1.8) node[right] {$y$};
			\draw (0,0) node [below left] {$O$};
			
			\foreach \x/\gx in {-1/90,1/80,2/90} \draw[thin] (\x,1pt)--(\x,-1pt) ($(\x,0)+(\gx:3mm)$)node{$\x$}
			;
			\foreach \x/\gy in {-3/180,-2/180,-1/180,1/180} \draw[thin] (1pt,\x)--(-1pt,\x) ($(0,\x)+(\gy:3mm)$)node{$\x$}
			;
			
			\draw[dashed,thin] (\a,0)|-(0,\fa) ;
			
			\draw[samples=100,domain=-1.1:3.1,smooth,variable=\x] plot (\x,{\f(\x)}) node[right]{$y=f(x)$};
		\end{tikzpicture}
	\end{center}
	Hỏi đồ thị hàm số $g(x)=\dfrac{(x^2-2x)\sqrt{2-x}}{(x-3)\left[ f^2(x)+3f(x)\right] }$ có bao nhiêu tiệm cận đứng?
	\choice
	{$6$}
	{$3$}
	{\True $4$}
	{$5$}
	\loigiai{
		Điều kiện: $\sqrt{2-x}$ là $x\le 2$ (*).\\
		Ta có: $(x-3)\left[ f^2(x)+3f(x)\right]=0\Leftrightarrow \hoac{& x=3 \\ & f(x)=0 \\ & f(x)=-3}$.\\
		* Ta có $x=3$ không thoả mãn (*).\\
		* $f(x)=0\Leftrightarrow \hoac{& x=a<0 \\ & x=b\in (0;2) \\ & x=c>2}$. Ta có $x=c$ không thoả mãn (*).\\
		Ta có $\underset{x\to a^+}{\mathop{\lim}}g(x)=+\infty$, $\underset{x\to b^+}{\mathop{\lim}}g(x)=+\infty$. Vậy $x=a,x=b$ là các đường tiệm cận đứng.\\
		* $f(x)=-3\Leftrightarrow \hoac{& x=d<0 \\ & x=2}$. Ta có $\underset{x\to d^+}{\mathop{\lim}}g(x)=+\infty$, $\underset{x\to 2^-}{\mathop{\lim}}g(x)=+\infty$. vậy $x=d,x=2$ là các đường tiệm cận đứng.
	}
\end{ex}
\Closesolutionfile{ans}
\indapan{10}{ans/CD4/Muc_9_10}
