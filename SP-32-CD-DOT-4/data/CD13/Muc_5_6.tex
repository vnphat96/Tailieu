\section{Mức độ 5 - 6 điểm}
\Opensolutionfile{ans}[ans/CD13/Muc_5_6]
\begin{dang} {Tỉ số thể tích khối chóp tam giác}\end{dang}
\Opensolutionfile{ans}[ans]
\begin{ex}%[2H1B3-3] % Câu 1:
	[THPT Quỳnh Lưu 3 Nghệ An 2019] 
	Cho hình chóp $S.ABC$. Gọi $M,N,P$ lần lượt là trung điểm 
	của $SA,SB,SC$. Tỉ số thể tích $\dfrac{V_{S.ABC}}{V_{S.MNP}}$ bằng
	\choice 
	{ $12$ }
	{ $2$ }
	{\True $8$ }
	{ $3$ }
	\loigiai{
		\immini {
			Ta có $\dfrac{V_{S.ABC}}{V_{S.MNP}}=\dfrac{SA}{SM}\cdot \dfrac{SB}{SN}\cdot
			\dfrac{SC}{SP}=2\cdot 2\cdot 2=8$ }
		{\begin{tikzpicture}[scale=1, font=\footnotesize, line join=round, line cap=round, >=stealth]
				\coordinate[label=left:$A$] (A)at(0,0);
				\coordinate[label=right:$C$] (C)at(5,0);
				\coordinate[label=below:$B$] (B)at(2,-2);
				\coordinate (H) at ($(A)!0.5!(C)$);
				\coordinate[label=left:$S$] (S) at ($(H)!1.5!90:(C)$);
				\coordinate[label=left:$M$] (M) at ($(S)!0.5!(A)$);
				\coordinate[label=left:$N$] (N) at ($(S)!0.5!(B)$);
				\coordinate[label=right:$P$] (P) at ($(S)!0.5!(C)$);
				\draw (S)--(A)--(B)--(C)--cycle (B)--(S) (M)--(N)--(P);
				\draw[dashed](M)--(P) (A)--(C) ;
				\foreach \diem in {A,B,C,S,M,N,P}	\fill (\diem)circle(1.5pt);
	\end{tikzpicture}}}
\end{ex}
\begin{ex}%[2H1B3-3] % Câu 2: 
	[THPT Lê Văn Thịnh Bắc Ninh 2019] Cho tứ diện $MNPQ$ . Gọi $I$, $J$, $K$ lần lượt là trung 
	điểm của các cạnh $MN$, $MP$, $MQ$. Tỉ số thể tích $\dfrac{V_{MIJK}}{V_{MNPQ}}$ bằng
	\choice 
	{ $\dfrac{1}{3}$ }
	{ $\dfrac{1}{4}$ }
	{ $\dfrac{1}{6}$ }
	{\True $\dfrac{1}{8}$ }
	\loigiai{
		\immini {Ta có: $\dfrac{V_{M.IJK}}{V_{M.NPQ}}=\dfrac{MI}{MN}\cdot \dfrac{MJ}{MP}\cdot
			\dfrac{MK}{MQ}=\dfrac{1}{2}\cdot \dfrac{1}{2}\cdot \dfrac{1}{2}=\dfrac{1}{8}\cdot$ }
		{\begin{tikzpicture}[scale=1, font=\footnotesize, line join=round, line cap=round, >=stealth]
				\coordinate[label=left:$N$] (N)at(0,0);
				\coordinate[label=right:$Q$] (Q)at(5,0);
				\coordinate[label=below:$P$] (P)at(2,-2);
				\coordinate (H) at ($(N)!0.5!(Q)$);
				\coordinate[label=left:$M$] (M) at ($(H)!1.5!90:(Q)$);
				\coordinate[label=left:$I$] (I) at ($(M)!0.5!(N)$);
				\coordinate[label=below left:$J$] (J) at ($(M)!0.5!(P)$);
				\coordinate[label=right:$K$] (K) at ($(M)!0.5!(Q)$);
				\draw (M)--(N)--(P)--(Q)--cycle (P)--(M) (I)--(J)--(K);
				\draw[dashed](N)--(Q) (I)--(K) ;
				\foreach \diem in {M,N,P, Q,I,J,K}	\fill (\diem)circle(1.5pt);
	\end{tikzpicture}}}
\end{ex} 
\begin{ex}%[2H1B3-3] % Câu 3: 
	[THPT Lê Văn Thịnh Bắc Ninh 2019] Cho hình chóp $S.ABCD$. Gọi $A'$, $B'$, $C'$, $D'$ theo 
	thứ tự là trung điểm của $SA$, $SB$, $SC$, $SD$ . Tính tỉ số thể tích của hai khối chóp $S.A'B'C'D'$. 
	và $S.ABCD$.
	\choice 
	{ $\dfrac{1}{16}$ }
	{ $\dfrac{1}{4}$ }
	{\True $\dfrac{1}{8}$ }
	{ $\dfrac{1}{2}$ }
	\loigiai{
		\immini{
			Ta có $\dfrac{V_{S.A'B'D'}}{V_{S.ABD}}=\dfrac{SA'}{SA}\cdot \dfrac{SB'}{SB}\cdot
			\dfrac{SD'}{SD}=\dfrac{1}{8}\Rightarrow \dfrac{V_{S.A'B'D'}}{V_{S.ABCD}}=\dfrac{1}{16}\cdot$ \\
			Và $\dfrac{V_{S.B'D'C'}}{V_{S.BDC}}=\dfrac{SB'}{SB}\cdot \dfrac{SD'}{SD}\cdot
			\dfrac{SC'}{SC}=\dfrac{1}{8}\Rightarrow \dfrac{V_{S.B'D'C'}}{V_{S.ABCD}}=\dfrac{1}{16}\cdot$ \\
			Suy ra 
			$\dfrac{V_{S.A'B'D'}}{V_{S.ABCD}}+\dfrac{V_{S.B'D'C'}}{V_{S.ABCD}}=\dfrac{1}{16}+\dfrac{1}{16}=\dfrac{1}{8}\cdot$\\
			Vậy $ \dfrac{V_{S.A'B'C'D'}}{V_{S.ABCD}}=\dfrac{1}{8}\cdot$}
		{
			\begin{tikzpicture}[line cap=round,line join=round,>=stealth,x=1cm,y=0.5cm,scale=0.7,font=\scriptsize]
				\coordinate[label=left:$A$] (A)at(0,0);
				\coordinate[label=below:$C$] (C)at(4,-4);
				\coordinate[label=right:$B$] (B)at(6,0);
				\coordinate[label=below:$D$] (D)at(1,-3);
				\coordinate[label=above:$S$] (S) at($(A)!1.5!70:(B)$);
				\coordinate[label=left:$A'$] (A') at ($(S)!0.5!(A)$);
				\coordinate[label=right:$B'$] (B') at ($(S)!0.5!(B)$);
				\coordinate[label=right:$C'$] (C') at ($(S)!0.5!(C)$);
				\coordinate[label=left:$D'$] (D') at ($(S)!0.5!(D)$);
				\draw [dash pattern=on 1pt off 1pt] (A)-- (B)--(D) (A')--(B')--(D') ;
				\draw (A)--(D)--(C)--(B) (A)--(S)--(D) (C)--(S)--(B) (A')--(D')--(C')--(B');
				\foreach \diem in {S, A, B, C, D, A', B', C', D'}	\fill (\diem)circle(1.5pt);
	\end{tikzpicture}}}
\end{ex} 
\begin{ex}%[2H1B3-3] % Câu 4: 
	Cho hình chóp $S.ABC$. Gọi $M$, $N$, $P$ theo thứ tự là trung điểm của $SA$, $SB$, $SC$. Tính 
	tỉ số thể tích của $2$ khối chóp $S.MNP$ và $S.ABC$ bằng
	\choice 
	{ $\dfrac{1}{4}$}
	{\True $\dfrac{1}{8}$}
	{ $\dfrac{1}{16}$}
	{ $\dfrac{1}{2}$}
	\loigiai{\immini{
			Ta có $\dfrac{V_{S.MNP}}{V_{S.ABC}}=\dfrac{SM}{SA}\cdot \dfrac{SN}{SB}\cdot
			\dfrac{SP}{SC}=\dfrac{1}{8}\cdot$ }
		{\begin{tikzpicture}[scale=1, font=\footnotesize, line join=round, line cap=round, >=stealth]
				\coordinate[label=left:$A$] (A)at(0,0);
				\coordinate[label=right:$C$] (C)at(5,0);
				\coordinate[label=below:$B$] (B)at(2,-2);
				\coordinate (H) at ($(A)!0.5!(C)$);
				\coordinate[label=left:$S$] (S) at ($(H)!1.5!90:(C)$);
				\coordinate[label=left:$M$] (M) at ($(S)!0.5!(A)$);
				\coordinate[label=left:$N$] (N) at ($(S)!0.5!(B)$);
				\coordinate[label=right:$P$] (P) at ($(S)!0.5!(C)$);
				\draw (S)--(A)--(B)--(C)--cycle (B)--(S) (M)--(N)--(P);
				\draw[dashed](M)--(P) (A)--(C) ;
				\foreach \diem in {A,B,C,S,M,N, P}	\fill (\diem)circle(1.5pt);
	\end{tikzpicture}
	}
}
\end{ex} 

\begin{ex}%[2H1B3-3] % Câu 5: 
	[SGD Hưng Yên 2019] Cho khối chóp $S.ABC$ có thể tích $V$. Gọi $B',C'$ lần lượt là trung điểm của
	$AB, AC$. Tính theo $V$ thể tích khối chóp $S.AB'C'$.
	\choice 
	{ $\dfrac{1}{3}V$}
	{ $\dfrac{1}{2}V$}
	{ $\dfrac{1}{12}V$}
	{\True $\dfrac{1}{4}V$}
	\loigiai{\immini{
			Ta có tỷ số thể tích $\dfrac{V_{ASB'C'}}{V_{A.SBC}}=\dfrac{AB'}{AB}\cdot
			\dfrac{AC'}{AC}=\dfrac{1}{2}\cdot \dfrac{1}{2}=\dfrac{1}{4}$ .\\ Do đó
			$V_{A.SB'C'}=\dfrac{1}{4}V_{A.SBC}$ hay $V_{S.AB'C'}=\dfrac{1}{4}V$.} {\begin{tikzpicture}[scale=1, font=\footnotesize, line join=round, line cap=round, >=stealth]
				\coordinate[label=left:$A$] (A)at(0,0);
				\coordinate[label=right:$C$] (C)at(5,0);
				\coordinate[label=below:$B$] (B)at(2,-2);
				\coordinate [label=below:$C'$](C') at ($(A)!0.5!(C)$);
				\coordinate[label=above:$S$] (S) at ($(C')!1.5!90:(C)$);
				\coordinate[label=left:$B'$] (B') at ($(A)!0.5!(B)$);
				\draw (S)--(A)--(B)--(C)--cycle (B)--(S)--(B');
				\draw[dashed] (S)--(C')--(B') (A)--(C);
				\foreach \diem in {A,B,C,S,C',B'}	\fill (\diem)circle(1.5pt);
	\end{tikzpicture}}}
\end{ex} 
\begin{ex}%[2H1B3-3] % Câu 6: 
	[THPT Thăng Long 2019] Cho hình chóp $S.ABCD$, gọi $I$, $J$, $K$, $H$ lần lượt là trung điểm 
	các cạnh $SA$, $SB$, $SC$, $SD$ . Tính thể tích khối chóp $S.ABCD$ biết thể tích khối chóp 
	$S.IJKH$ bằng $1$ .
	\choice 
	{ $16$}
	{\True $8$}
	{$2$}
	{$4$}
	\loigiai{
		\immini{Ta có: $\dfrac{V_{S.ABC}}{V_{S.IJK}}=\dfrac{SA}{SI}\cdot \dfrac{SB}{SJ}\cdot
			\dfrac{SC}{SK}=8\Rightarrow V_{S.ABC}=8V_{S.IJK}$ .\\
			$\dfrac{V_{S.ACD}}{V_{S.IKH}}=\dfrac{SA}{SI}\cdot \dfrac{SC}{SK}\cdot
			\dfrac{SD}{SH}=8\Rightarrow V_{S.ACD}=8V_{S.IKH}$ \\
			Do đó: $V_{S.ABCD}=8V_{S.IJKH}=8$.}
		{	\begin{tikzpicture}[line cap=round,line join=round,>=stealth,x=1cm,y=1cm,scale=0.7]
				\coordinate[label=left:$A$] (A)at(0,0);
				\coordinate[label=right:$C$] (C)at(4,-3);
				\coordinate[label=below:$B$] (B)at(6,0);
				\coordinate[label=left:$D$] (D)at(1,-2);
				\coordinate[label=above:$S$] (S) at($(A)!1.5!70:(B)$);
				\coordinate[label=left:$I$] (A') at ($(S)!0.5!(A)$);
				\coordinate[label=right:$J$] (B') at ($(S)!0.5!(B)$);
				\coordinate[label=right:$K$] (C') at ($(S)!0.5!(C)$);
				\coordinate[label=left:$H$] (D') at ($(S)!0.5!(D)$);
				\draw [dash pattern=on 1pt off 1pt] (A)-- (B)--(D) (C')--(A')--(B') (A)--(C) ;
				\draw (A)--(D)--(C)--(B) (A)--(S)--(D) (C)--(S)--(B) (A')--(D')--(C')--(B');
				\foreach \diem in {S, A, B, C, D, A', B', C', D'}	\fill (\diem)circle(1.5pt);
	\end{tikzpicture}}}
\end{ex} 

\begin{ex}%[2H1B3-3] % Câu 7: 
	Cho hình chóp $S.ABC$, trên các tia $SA$, $SB$, $SC$ lần lượt lấy các điểm $A'$, $B'$, $C'$. Gọi 
	$V_1$ , $V_2$ lần lượt là thể tích khối chóp $S.ABC$ và $S.A'B'C'$. Khẳng định nào sau đây là
	đúng?
	\choice 
	{ $\dfrac{V_1}{V_2}=\dfrac{SA}{SA'}\cdot \dfrac{SB'}{SB}\cdot \dfrac{SC}{SC'}$}
	{ $\dfrac{V_1}{V_2}=\dfrac{1}{2}\cdot \dfrac{SB}{SB'}\cdot \dfrac{SC}{SC'}$}
	{ $\dfrac{V_1}{V_2}=\dfrac{SA}{SA'}\cdot \dfrac{SB}{SB'}$ }
	{\True $\dfrac{V_1}{V_2}=\dfrac{SA}{SA'}\cdot \dfrac{SB}{SB'}\cdot \dfrac{SC}{SC'}$}
	\loigiai{
		\immini{Theo công thức tỉ số thể tích ta có $\dfrac{V_1}{V_2}=\dfrac{SA}{SA'}\cdot
			\dfrac{SB}{SB'}\cdot \dfrac{SC}{SC'}\cdot$}
		{\begin{tikzpicture}[scale=1, font=\footnotesize, line join=round, line cap=round, >=stealth]
				\coordinate[label=left:$A$] (A)at(0,0);
				\coordinate[label=right:$C$] (C)at(5,0);
				\coordinate[label=below:$B$] (B)at(2,-2);
				\coordinate (H) at ($(A)!0.5!(C)$);
				\coordinate[label=above:$S$] (S) at ($(H)!1.5!100:(C)$);
				\coordinate[label=left:$M$] (M) at ($(S)!0.5!(A)$);
				\coordinate[label=left:$N$] (N) at ($(S)!0.5!(B)$);
				\coordinate[label=right:$P$] (P) at ($(S)!0.5!(C)$);
				\draw (S)--(A)--(B)--(C)--cycle (B)--(S) (M)--(N)--(P);
				\draw[dashed](M)--(P) (A)--(C) ;
				\foreach \diem in {A,B,C,S,M,N, P}	\fill (\diem)circle(1.5pt);
	\end{tikzpicture}}}
\end{ex} 

\begin{ex}%[2H1B3-3] % Câu 8: 
	[Gia Lai 2019] Cho khối chóp SABC có thể tích bằng $5a^3$. Trên các cạnh $SB$, $SC$ lần lượt lấy
	các điểm $M$ và $N$ sao cho $SM=3MB$, $SN=4NC$ (tham khảo hình vẽ). Tính thể tích $V$ của 
	khối chóp $AMNCB$.
	\choice 
	{ $V=\dfrac{3}{5}a^3$ }
	{ $V=\dfrac{3}{4}a^3$ }
	{ $V={a^3}$ }
	{\True $V=2a^3$ }
	\loigiai{\immini{
			Gọi $V_1$ là thể tích khối chóp $SAMN$ và $V_0$ là thể tích khối chóp $SABC$.\\
			Theo công thức tỷ lệ thể tích ta có: $\dfrac{V_1}{V_0}=\dfrac{SM}{SB}\cdot
			\dfrac{SN}{SC}=\dfrac{3}{4}\cdot \dfrac{4}{5}=\dfrac{3}{5}\cdot$ \\
			$V$ là thể tích khối chóp $AMNCB$ ta có $V+V_1=V_0$.\\
			Vậy $V=\dfrac{2}{5}V_0=\dfrac{2}{5}\cdot 5{a^3}=2a^3$.}
		{\begin{tikzpicture}[scale=1, font=\footnotesize, line join=round, line cap=round, >=stealth]
				\coordinate[label=left:$A$] (A)at(0,0);
				\coordinate[label=right:$C$] (C)at(5,0);
				\coordinate[label=below:$B$] (B)at(2,-2);
				\coordinate (H) at ($(A)!0.5!(C)$);
				\coordinate[label=left:$S$] (S) at ($(H)!1.5!90:(C)$);
				\coordinate[label=above left:$M$] (M) at ($(S)!3/4!(B)$);
				\coordinate[label=right:$N$] (N) at ($(S)!4/5!(C)$);
				\draw (S)--(A)--(B)--(C)--cycle (B)--(S) (A)--(M)--(N);
				\draw[dashed](N)--(A)--(C) ;
				\foreach \diem in {A,B,C,S,M,N}	\fill (\diem)circle(1.5pt);
	\end{tikzpicture}}}
\end{ex} 

\begin{ex}%[2H1B3-3] % Câu 9: 
	Nếu một hình chóp tứ giác đều có chiều cao và cạnh đáy cùng tăng lên $2$ lần thì thể tích của nó tăng lên 
	bao nhiêu lần?
	\choice 
	{$2$ lần}
	{$4$ lần}
	{$6$ lần}
	{\True $8$ lần}
	\loigiai{\immini{
			Gọi $ h$ , $ a$ lần lượt là chiều cao và cạnh đáy của hình chóp tứ giác đều.\\
			Thể tích của khối chóp tứ giác đều là $V=\dfrac{1}{3}a^2h$.\\
			Khi tăng chiều cao và cạnh đáy lên $2$ lần thì ta được khối chóp tứ giác đều mới có thể tích 
			là\\
			$V'=\dfrac{1}{3}{{( 2a )}^2}( 2h )=8\cdot \dfrac{1}{3}a^2h=8V$.\\
			Vậy thể tích của khối chóp tăng lên $8$ lần.
		}
		{
			\begin{tikzpicture}[scale=1, font=\footnotesize, line join=round, line cap=round, >=stealth]
				\def\bc{4} % cạnh BC
				\def\ba{2} % cạnh BA
				\def\h{4} % đường cao
				\def\gocB{30} % góc B của đáy
				\coordinate (B) at (0,0);
				\coordinate (A) at (\gocB:\ba);
				\coordinate (C) at (\bc,0);
				\coordinate (D) at ($(C)-(B)+(A)$);
				\coordinate (O) at ($(A)!.5!(C)$);
				\coordinate (S) at ($(O)+(90:\h)$);
				\coordinate (H) at ($(C)!0.5!(D)$);
				\draw (B)--(C)--(D)--(S)--cycle (H)--(S)--(C);
				\draw[dashed] (C)--(A)--(D)--(B) (O)--(S)--(A)--(B);
				\draw pic[angle radius=2mm,draw=blue] {right angle = D--C--B};
				\draw pic[angle radius=2mm,draw=blue] {right angle = A--D--B};
				\draw pic[angle radius=2mm,draw=blue] {right angle = S--O--H};	\draw pic[angle radius=2mm,draw=blue] {right angle = B--A--D}; \draw pic[angle radius=2mm,draw=blue] {right angle = B--C--D};
				\draw pic[angle radius=2mm,draw=blue] {right angle = A--B--D};
				\node at (2,-0.2){$a$};
				\coordinate[label=left:$h$] (I) at($(S)!0.5!(O)$);
				\foreach \diem in {A,B,C,D,S,O,H}	\fill (\diem)circle(1.5pt);
			\end{tikzpicture}
	}}
\end{ex} 

\begin{ex}%[2H1B3-3] % Câu 10: 
	Trên ba cạnh $OA,OB,OC$ của khối chóp $O.ABC$ lần lượt lấy các điểm $A',B',C'$ sao cho 
	$2OA'=OA,$ $4OB'=OB$ và $3OC'=OC$. Tỉ số thể tích giữa hai khối chóp $O.A'B'C'$ và $O.ABC$ là
	\choice 
	{ $\dfrac{1}{12}$}
	{\True $\dfrac{1}{24}$}
	{ $\dfrac{1}{32}$}
	{ $\dfrac{1}{16}$}
	\loigiai{\immini{
			$\dfrac{V_{O.A'B'C'}}{V_{O.ABC}}=\dfrac{OA'}{OA}\cdot \dfrac{OB'}{OB}\cdot
			\dfrac{OC'}{OC}=\dfrac{1}{2}\cdot \dfrac{1}{4}\cdot \dfrac{1}{3}=\dfrac{1}{24}\cdot$}{\begin{tikzpicture}[scale=1, font=\footnotesize, line join=round, line cap=round, >=stealth]
				\coordinate[label=left:$A$] (A)at(0,0);
				\coordinate[label=right:$C$] (C)at(5,-1);
				\coordinate[label=below:$B$] (B)at(2,-2);
				\coordinate (H) at ($(A)!0.5!(C)$);
				\coordinate[label=above:$O$] (O) at ($(H)!1.5!90:(C)$);
				\coordinate[label=above left:$B'$] (B') at ($(O)!0.25!(B)$);
				\coordinate[label=right:$C'$] (C') at ($(O)!1/3!(C)$);
				\coordinate[label=left:$A'$] (A') at ($(O)!0.5!(A)$);
				\draw (O)--(A)--(B)--(C)--cycle (B)--(O)--(B') (A')--(B') --(C');
				\draw[dashed] (A')--(C') (A)--(C);
				\foreach \diem in {A,B,C,O,C',B',A'}	\fill (\diem)circle(1.5pt);
		\end{tikzpicture}}.}
\end{ex} 
\begin{ex}%[2H1B3-3] % Câu 11: 
	Cho khối chóp $SABC$, $M$ là trung điểm của $SA$. Tỉ số thể tích
	$\dfrac{V_{M.ABC}}{V_{S.ABC}}$ bằng
	\choice 
	{ $\dfrac{1}{4}$ }
	{\True $\dfrac{1}{2}$ }
	{ $2$ }
	{ $\dfrac{1}{8}$ }
	\loigiai{
		Ta có $\dfrac{V_{S.MBC}}{V_{S.ABC}}=\dfrac{SM}{SA}=\dfrac{1}{2}\Rightarrow
		\dfrac{V_{M.ABC}}{V_{S.ABC}}=\dfrac{1}{2}$ .}
\end{ex} 

\begin{ex}%[2H1B3-3] % Câu 12: 
	[THPT Hoa Lư A - 2018] Cho khối tứ diện $ABCD$ có thể tích $V$ và điểm $E$ trên cạnh $AB$ sao 
	cho $AE=3EB$ . Tính thể tích khối tứ diện $EBCD$ theo $V.$ 
	\choice 
	{\True $\dfrac{V}{4}$ }
	{ $\dfrac{V}{3}$ }
	{ $\dfrac{V}{2}$ }
	{ $\dfrac{V}{5}$ }
	\loigiai{\immini{
			$\dfrac{V_{BECD}}{V_{A.BCD}}=\dfrac{BE}{BA}\cdot \dfrac{AC}{AC}\cdot
			\dfrac{AD}{AD}=\dfrac{1}{4}\\\Rightarrow V_{B.ECD}=V_{E.BCD}=\dfrac{1}{4}V$.}
		{\begin{tikzpicture}[scale=1, font=\footnotesize, line join=round, line cap=round, >=stealth]
				\def\ac{4} % cạnh BD
				\def\ab{2} % cạnh BC
				\def\h{3} % chiều cao
				\def\gocA{50} % góc B của đáy
				\coordinate[label=left:$B$] (B) at (0,0);
				\coordinate[label=right:$D$] (D) at (\ac,0);
				\coordinate[label=below left:$C$] (C) at (-\gocA:\ab);
				\coordinate[label=above:$A$] (A) at ($(B)+(60:\h)$);
				\coordinate[label=left:$E$] (E) at ($(A)!3/4!(B)$);
				
				\draw (B)--(C)--(D)--(A)--cycle (A)--(C)--(E);
				\draw[dashed] (B)--(D)--(E);
				\foreach \diem in {A,B,C,D,E}	\fill (\diem)circle(1.5pt);
			\end{tikzpicture}
	}}
\end{ex} 

\begin{ex}%[2H1B3-3] % Câu 13: 
	(Chuyên Vinh - 2018) Cho khối chóp $S.ABCD$ có thể tích $V$ . Các điểm $A'$ , $B'$ , $C'$ tương 
	ứng là trung điểm các cạnh $SA$ , $SB$ , $SC$ . Thể tích khối chóp $S.A'B'C'$ bằng
	\choice 
	{\True $\dfrac{V}{8}$ }
	{ $\dfrac{V}{4}$ }
	{ $\dfrac{V}{2}$ }
	{ $\dfrac{V}{16}$ }
	\loigiai{
		Ta có $\dfrac{V_{S.A'B'C'}}{V_{S.ABC}}=\dfrac{SA'}{SA}\cdot \dfrac{SB'}{SB}\cdot
		\dfrac{SC'}{SC}=\dfrac{1}{8}\Rightarrow V_{S.A'B'C'}=\dfrac{V}{8}\cdot$ 
	}
\end{ex} 

\begin{ex}%[2H1B3-3] % Câu 14: 
	(THPT Cao Bá Quát - 2018) Cho tứ diện đều $ABCD$ có cạnh $ a$ . Trên các cạnh $AB$ , $AC$ lần 
	lượt lấy các điểm $B',C'$ sao cho $AB'=\dfrac{a}{2},AC'=\dfrac{2a}{3}$ . Tỉ số thể tích của khối
	tứ diện $AB'C'D$ và khối tứ diện $ABCD$ là
	\choice 
	{ $\dfrac{1}{2}$ }
	{\True $\dfrac{1}{3}$ }
	{ $\dfrac{1}{4}$ }
	{ $\dfrac{1}{5}$ }
	\loigiai{\immini{
			Ta có $\dfrac{V_{AB'C'D}}{V_{ABCD}}=\dfrac{AB'}{AB}\cdot\dfrac{AC'}{AC}=\dfrac{1}{3}\cdot$}
		{\begin{tikzpicture}[scale=1, font=\footnotesize, line join=round, line cap=round, >=stealth]
				\def\ac{4} % cạnh BD
				\def\ab{2} % cạnh BC
				\def\h{4.2} % chiều cao
				\def\gocA{50} % góc B của đáy
				\coordinate[label=left:$B$] (B) at (0,0);
				\coordinate[label=right:$D$] (D) at (\ac,0);
				\coordinate[label=below left:$C$] (C) at (-\gocA:\ab);
				\coordinate[label=above:$A$] (A) at ($(B)+(70:\h)$);
				\coordinate[label=left:$B'$] (B') at ($(A)!0.5!(B)$);
				\coordinate[label=left:$C'$] (C') at ($(A)!2/3!(C)$);
				\draw (B)--(C)--(D)--(A)--cycle (A)--(C) (B')--(C')--(D);
				\draw[dashed] (B)--(D)--(B');
				\foreach \diem in {A,B,C,D,B',C'}	\fill (\diem)circle(1.5pt);
			\end{tikzpicture}
	}}
\end{ex} 

\begin{dang}
	{Tỉ số khối lăng trụ}
\end{dang}

\setcounter{ex}{0}
\begin{ex}%[2H1B3-3] % Câu 15: 
	[Sở Nam Định - 2019] Cho khối lăng trụ $ABC.A'B'C'$ có thể tích bằng $V$. Tính thể tích khối đa
	diện $BAA'C'C$
	\choice 
	{$\dfrac{3V}{4}$}
	{\True $\dfrac{2V}{3}$}
	{$\dfrac{V}{2}$}
	{$\dfrac{V}{4}$}
	\loigiai{\immini{
			Mặt phẳng $(BA'C')$ chia khối lăng trụ $ABC.A'B'C'$ thành hai khối: $B.AA'C'C$ và
			$B.A'B'C'$. \\
			$\Rightarrow V_{B.AA'C'C}=V_{ABC.A'B'C'}-V_{B.A'B'C'}$. \\
			Khối chóp $S.BA'B'C'$ và khối lăng trụ có chung đáy và chung chiều cao $\Rightarrow V_{B.A'B'C'}=\dfrac{1}{3}V$ \\
			$\Rightarrow V_{BAA'C'C}=V-\dfrac{1}{3}V=\dfrac{2V}{3}\cdot$
		} {\begin{tikzpicture}
				\def\a{4}
				\def\h{4.5}
				\path 	(0:0) coordinate (A)
				++(0:\a) coordinate (C)
				++(-150:\a/2) coordinate (B)
				($(A)+(70:\h)$) coordinate (A')
				($(A')+(C)-(A)$) coordinate (C')
				($(C')+(B)-(C)$) coordinate (B');
				\draw[dashed,thick] (A)--(C);
				\draw[thick] (C)--(C') 	(B)--(B') 	(A)--(A')--(B)--(C')
				(A)--(B)--(C) (A')--(B')--(C')--cycle;
				\foreach \x/\g in {A/180,B/-45,C/0,A'/180,B'/-45,C'/0}
				\fill[black] 	(\x) circle (1pt)
				($(\g:4mm)+(\x)$) node {$\x$};
			\end{tikzpicture}
	}}
\end{ex} 

\begin{ex}%[2H1B3-3] % Câu 16: 
	[Chuyên Lê Thánh Tông 2019] Cho lăng trụ $ABC.A'B'C'$, $M$ là trung điểm $CC'$. Mặt phẳng $(ABM )$ chia khối lăng trụ
	thành hai khối đa diện. Gọi $V_1$ là thể tích khối lăng trụ chứa đỉnh $C$
	và $V_2$ là thể tích khối đa diện còn lại. Tính tỉ số $\dfrac{V_1}{V_2}\cdot$
	\choice 
	{\True $\dfrac{1}{5}$}
	{$\dfrac{1}{6}$}
	{$\dfrac{1}{2}$}
	{$\dfrac{2}{5}$}
	\loigiai{\immini{
			Ta có $V_1$ là thể tích khối lăng trụ chứa đỉnh $C$ tức là
			$V_1=V_{M.ABC}=\dfrac{1}{3}S_{ABC}\cdot MC$ \\
			$V_2$ là thể tích khối đa diện còn lại \\$\Rightarrow V_2=V_{ABC.A'B'C'}-V_1=S_{ABC}\cdot CC'-\dfrac{1}{6}S_{ABC}\cdot
			CC'\\~~~~~~~~=\dfrac{5}{6}S_{ABC}\cdot CC'$. \\
			Khi đó ta có tỉ số\\
			$\dfrac{V_1}{V_2}=\dfrac{\dfrac{1}{3}S_{ABC}.MC}{\dfrac{5}{6}S_{ABC}\cdot CC'}=\dfrac{\dfrac{1}{6}S_{ABC}\cdot
				CC'}{\dfrac{5}{6}S_{ABC}\cdot CC'}=\dfrac{1}{5}\cdot$
		} {\begin{tikzpicture}
				\def\a{4}
				\def\h{4.5}
				\path 	(0:0) coordinate (A)
				++(0:\a) coordinate (C)
				++(-150:\a/2) coordinate (B)
				($(A)+(70:\h)$) coordinate (A')
				($(A')+(C)-(A)$) coordinate (C')
				($(C')+(B)-(C)$) coordinate (B');
				\coordinate[label=right:$M$] (M) at ($(C)!0.5!(C')$);
				\draw[dashed,thick] (M)--(A)--(C);
				\draw[thick] (C)--(C') 	(M)--(B)--(B') 	(A)--(A')
				(A)--(B)--(C) (A')--(B')--(C')--cycle;
				\foreach \x/\g in {A/180,B/-45,C/0,A'/180,B'/-45,C'/0}
				\fill[black] 	(\x) circle (1pt)
				($(\g:4mm)+(\x)$) node {$\x$};
			\end{tikzpicture}
	}}
\end{ex}

\begin{ex}%[2H1B3-3] % Câu 17: 
	Khối lăng trụ $ABC.A'B'C'$ có thể tích bằng $6$. Mặt phẳng $( A'BC' )$ chia khối lăng trụ thành một
	khối chóp tam giác và một khối chóp tứ giác có thể tích lần lượt là
	\choice 
	{\True $2$ và $4$ }
	{ $3$ và $3$ }
	{ $4$ và $2$ }
	{ $1$ và $5$ }
	\loigiai{\immini{
			\begin{itemize}
				\item Thể tích khối lăng trụ là $V_{ABC.A'B'C'}=\mathrm d( B,( A'B'C' ) )\cdot S_{A'B'C'}=6$.
				\item  Thể tích khối chóp tam giác $.BA'B'C'$ là\\
				$V_{B.A'B'C'}=\dfrac{1}{3}\cdot\mathrm  d( B,( A'B'C' ) )\cdot S_{A'B'C'}=\dfrac{1}{3}\cdot V_{ABC.A'B'C'}=\dfrac{1}{3}\cdot
				6=2$.
			\end{itemize}  
			Vậy thể tích khối chóp tứ giác $BACC'A'$ là:\\ $V_{B.ACC'A'}=V_{ABC.A'B'C'}-V_{B.A'B'C'}=6-2=4$.}{\begin{tikzpicture}
				\def\a{4}
				\def\h{4.5}
				\path 	(0:0) coordinate (A)
				++(0:\a) coordinate (C)
				++(-150:\a/2) coordinate (B)
				($(A)+(70:\h)$) coordinate (A')
				($(A')+(C)-(A)$) coordinate (C')
				($(C')+(B)-(C)$) coordinate (B');
				\draw[dashed,thick] (A)--(C);
				\draw[thick] (C)--(C') 	(B)--(B') 	(A)--(A')--(B)--(C')
				(A)--(B)--(C) (A')--(B')--(C')--cycle;
				\foreach \x/\g in {A/180,B/-45,C/0,A'/180,B'/-45,C'/0}
				\fill[black] 	(\x) circle (1pt)
				($(\g:4mm)+(\x)$) node {$\x$};
			\end{tikzpicture}
	}}
\end{ex} 

\begin{ex}%[2H1B3-3] % Câu 18: 
	Cho khối lăng trụ tam giác $ABC.A'B'C'$ có thể tích $V$. Gọi $M$ là trung điểm của cạnh $CC'$. Mặt
	phẳng $( MAB )$ chia khối lăng trụ thành hai phần có tỉ số $ k\le 1$. Tìm $ k$ ?
	\choice 
	{ $\dfrac{2}{5}$ }
	{ $\dfrac{3}{5}$ }
	{\True $\dfrac{1}{5}$ }
	{ $\dfrac{1}{6}$ }
	\loigiai{\immini{
			Ta có $V=d( C',( ABC ) )\cdot S_{ABC}$.\\
			Khi đó $V_{M.ABC}=\dfrac{1}{3}\mathrm d( M,( ABC ) )\cdot S_{ABC}=\dfrac{1}{6}\mathrm d( C',(ABC ) )\cdot
			S_{ABC}=\dfrac{1}{6}V\Rightarrow
			V_{ABM.A'B'C'}=\dfrac{5}{6}V$.\\
			Vậy $ k=\dfrac{V_{M.ABC}}{V_{ABM.A'B'C'}}=\dfrac{1}{5}$.}
		{\begin{tikzpicture}
				\def\a{4}
				\def\h{4.5}
				\path 	(0:0) coordinate (A)
				++(0:\a) coordinate (B)
				++(-150:\a/2) coordinate (C)
				($(A)+(70:\h)$) coordinate (A')
				($(A')+(B)-(A)$) coordinate (B')
				($(B')+(C)-(B)$) coordinate (C');
				\coordinate[label=right:$M$] (M) at ($(C)!0.5!(C')$);
				\draw[dashed,thick] (A)--(B);
				\draw[thick] (C)--(C') 	(M)--(B)--(B') 	(A)--(A')
				(A)--(C)--(B) (A')--(B')--(C')--cycle (M)--(A)--(C);
				\foreach \x/\g in {A/180,B/-45,C/0,A'/180,B'/-45,C'/0}
				\fill[black] 	(\x) circle (1pt)
				($(\g:4mm)+(\x)$) node {$\x$};
			\end{tikzpicture}
	}}
\end{ex} 

\begin{ex} % Câu 19: 
	(THPT Thăng Long 2019) Một khối lăng trụ tứ giác đều có thể tích là $4$ . Nếu gấp đôi các cạnh đáy 
	đồng thời giảm chiều cao của khối lăng trụ này hai lần thì được khối lăng trụ mới có thể tích là:
	\choice 
	{\True $8$}
	{$4$}
	{$16$}
	{$2$}
	\loigiai{
		Giả sử khối lăng trụ tứ giác đều có độ dài cạnh đáy là $ a$ và chiều cao là $ h$ . Khi đó thể tích 
		khối lăng trụ tứ giác đều được tính bởi công thức $V=Bh=a^2\cdot h=4$.\\
		Nếu gấp đôi các cạnh đáy thì diện tích đáy mới $B'=4a^2$ . Giảm chiều cao hai lần nên
		chiều cao mới $ h'=\dfrac{h}{2}$.\\ Vì vậy thể tích khối lăng trụ mới sẽ là
		$V=B'h'=4a^2\cdot \dfrac{h}{2}=2a^2h=8$.}
\end{ex} 

\begin{ex}%[2H1B3-3] % Câu 20: 
	Biết khối hộp $ABCD.A'B'C'D'$ có thể tích $V$. Nếu tăng mỗi cạnh của hình hộp đó lên gấp hai lần thì
	thể tích khối hộp mới là:
	\choice 
	{\True $8V$}
	{$4V$}
	{$2V$}
	{$16V$}
	\loigiai{
		Ta có nếu tăng mỗi cạnh của khối hộp lên hai lần thì ta được khối hộp mới đồng dạng với khối 
		hộp cũ theo tỉ số 2. Do đó thể tích khối hộp mới bằng $2^3\cdot V=8V$.}
\end{ex} 

\begin{ex}%[2H1B3-3] % Câu 21: 
	Cho hình lăng trụ đứng $ABC.A'B'C'$ có $M$ là trung điểm của $AA'$. Tỉ số thể tích
	$\dfrac{V_{M.ABC}}{V_{ABC.A'B'C'}}$ bằng
	\choice 
	{\True $\dfrac{1}{6}$}
	{$\dfrac{1}{3}$}
	{$\dfrac{1}{12}$}
	{$\dfrac{1}{2}$}
	\loigiai{\immini{
			Ta có:
			$ V_{ABC.A'B'C'}=AA'\cdot S_{\triangle ABC}\quad(1) $; \\
			$  V_{M.ABC}=\dfrac{1}{3}AM.S_{\triangle ABC}=\dfrac{1}{3}\cdot \dfrac{1}{2}AA'\cdot S_{\triangle
				ABC}=\dfrac{1}{6}V_{ABC.A'B'C'}\quad(2). $\\
			Từ (1) và (2) suy ra\\
			$ \dfrac{V_{M.ABC}}{V_{ABC.A'B'C'}}=\dfrac{1}{6}\cdot$}
		{\begin{tikzpicture}
				\def\a{4}
				\def\h{4.5}
				\path 	(0:0) coordinate (A)
				++(0:\a) coordinate (B)
				++(-150:\a/2) coordinate (C)
				($(A)+(90:\h)$) coordinate (A')
				($(A')+(B)-(A)$) coordinate (B')
				($(B')+(C)-(B)$) coordinate (C');
				\coordinate[label=above right:$M$] (M) at ($(C)!0.5!(C')$);
				\draw[dashed,thick] (A)--(B);
				\draw[thick] (C)--(C') 	(M)--(B)--(B') 	(A)--(A')
				(A)--(C)--(B) (A')--(B')--(C')--cycle (M)--(A)--(C);
				\foreach \x/\g in {A/180,B/-45,C/-90,A'/180,B'/-45,C'/90}
				\fill[black] 	(\x) circle (1pt)
				($(\g:4mm)+(\x)$) node {$\x$};
			\end{tikzpicture}
		}
	}
\end{ex} 

\begin{ex}%[2H1B3-3] % Câu 22: 
	[HKI-NK HCM-2019] Cho lăng trụ tam giác $ABC.A'B'C'$ có thể tích là $V$ . Gọi $M$ là trung điểm
	cạnh $AA'$ . Khi đó thể tích khối chóp $M.BCC'B'$ là
	\choice 
	{$\dfrac{V}{2}$}
	{\True $\dfrac{2V}{3}$}
	{$\dfrac{V}{3}$}
	{$\dfrac{V}{6}$}
	\loigiai{\immini{
			Vì $AA'\parallel ( BB'C'C )$ nên\\
			$\mathrm d( M,( BB'C'C ) )=\mathrm d( A,( BB'C'C ) )$
			\\suy ra
			$V_{M.BB'C'C}=V_{ABB'C'C}$. \\
			Mà\\ $V_{A.BB'C'C}=V_{ABC.A'B'C'}-V_{AA'B'C'}=V-\dfrac{1}{3}V=\dfrac{2}{3}V$.
			\\
			Vậy $V_{M.BB'C'C}=\dfrac{2}{3}V$.}{\begin{tikzpicture}
				\def\a{4}
				\def\h{4.5}
				\path 	(0:0) coordinate (A')
				++(0:\a) coordinate (C')
				++(-150:\a/2) coordinate (B')
				($(A')+(60:\h)$) coordinate (A)
				($(A)+(C')-(A')$) coordinate (C)
				($(C)+(B')-(C')$) coordinate (B);
				\coordinate (M) at ($(A)!0.5!(A')$);
				\draw[dashed] (A')--(C')-- (M) -- (C') -- (A);
				\draw[thick] 	(B)--(B') 	(C)--(C') (A)--(C) (A')--(A)
				(A')--(B')--(C') (M)--(B') (A)--(B)--(C)--cycle;
				\foreach \x/\g in {A'/180,B'/-45,C'/0,A/180,B/-45,C/0,M/180}
				\fill[black] 	(\x) circle (1pt)
				($(\g:4mm)+(\x)$) node {$\x$};
			\end{tikzpicture}
		}
	}
\end{ex} 

\begin{ex}%[2H1B3-3] % Câu 23: 
	[THPT Hoàng Hoa Thám - Hưng Yên 2019] Cho lăng trụ $ABC.A'B'C'$. Biết diện tích mặt bên $(ABB'A' )$ bằng 15, khoảng
	cách từ điểm $C$ đến $( ABB'A' )$ bằng 6. Tính thể tích khối lăng trụ $ABC.A'B'C'.$
	\choice 
	{$30$}
	{\True $45$}
	{$60$}
	{$90$}
	\loigiai{\immini{
			Ta có \\$V_{CABB'A'}=\dfrac{1}{3}\mathrm d( C;( ABB'A' ) )\cdot S_{ABB'A'}=\dfrac{1}{3}\cdot 6\cdot 15=30. $ \\
			Mà $V_{CABB'A'}=\dfrac{2}{3}\cdot V_{ABC.A'B'C'}\\\Rightarrow
			V_{ABC.A'B'C'}=\dfrac{3}{2}V_{CABB'A'}=45 $.}{\begin{tikzpicture}
				\def\a{4}
				\def\h{4.5}
				\path 	(0:0) coordinate (A')
				++(0:\a) coordinate (C')
				++(-150:\a/2) coordinate (B')
				($(A')+(70:\h)$) coordinate (A)
				($(A)+(C')-(A')$) coordinate (C)
				($(C)+(B')-(C')$) coordinate (B);
				%	\coordinate (M) at ($(A)!0.5!(A')$);
				\draw[dashed] (C)--(A')--(C');
				\draw[thick] 	(B)--(B')--(C)--(C') (A)--(C) (A')--(A)
				(A')--(B')--(C') (A)--(B)--(C)--cycle;
				\foreach \x/\g in {A'/180,B'/-90,C'/0,A/180,B/90,C/0}
				\fill[black] 	(\x) circle (1pt)
				($(\g:3mm)+(\x)$) node {$\x$};
			\end{tikzpicture}
	}}
\end{ex}

\begin{ex}%[2H1B3-3] % Câu 24: 
	[Chuyên Vĩnh Phúc - 2019] Cho khối lăng trụ $ABC.A'B'C'$ có thể tích bằng $V$. Tính thể tích khối
	đa diện $ABCB'C'$.
	\choice 
	{$\dfrac{V}{4}$}
	{$\dfrac{V}{2}$}
	{$\dfrac{3V}{4}$}
	{\True $\dfrac{2V}{3}$}
	\loigiai{\immini{
			Gọi chiều cao của lăng trụ là $ h$ , $S_{ABC}=S_{A'B'C'}=S$ . Khi đó $V=Sh$.\\
			Ta có $V_{AA'B'C'}=\dfrac{1}{3}Sh=\dfrac{1}{3}V\Rightarrow V_{ABCB'C'}=\dfrac{2}{3}V$.}{\begin{tikzpicture}
				\def\a{4}
				\def\h{4.5}
				\path 	(0:0) coordinate (A)
				++(0:\a) coordinate (C)
				++(-160:\a/2) coordinate (B)
				($(A)+(80:\h)$) coordinate (A')
				($(A')+(C)-(A)$) coordinate (C')
				($(C')+(B)-(C)$) coordinate (B');
				\draw[dashed,thick] (A)--(C) (A)--(C');
				\draw[thick] (C)--(C') 	(B)--(B') 	(A')--(A)--(B') 
				(A)--(B)--(C) (A')--(B')--(C')--cycle;
				\foreach \x/\g in {A/180,B/-45,C/0,A'/180,B'/-45,C'/0}
				\fill[black] 	(\x) circle (1pt)
				($(\g:4mm)+(\x)$) node {$\x$};
			\end{tikzpicture}
		}
	}
\end{ex} 

\begin{ex}%[2H1B3-3] % Câu 25: 
	Cho hình hộp $ABCD.A'B'C'D'$ có $I$ là giao điểm của $AC$ và $BD$. Gọi $V_1$ và $V_2$ lần
	lượt là thể tích của các khối $ABCD.A'B'C'D'$ và $I.A'B'C'$. Tính tỉ số $\dfrac{V_1}{V_2}\cdot$
	\choice 
	{\True $\dfrac{V_1}{V_2}=6$}
	{ $\dfrac{V_1}{V_2}=2$}
	{ $\dfrac{V_1}{V_2}=\dfrac{3}{2}$}
	{ $\dfrac{V_1}{V_2}=3$}
	\loigiai{ \immini{ 
			Ta có: \\
			$\begin{aligned}
				V_1&=AA'.S_{A'B'C'D'},\\
				V_2&=\dfrac{1}{3}\mathrm d( I;( A'B'C' ) )\cdot S_{\Delta A'B'C'}\\
				&=\dfrac{1}{3}\mathrm d( A;( A'B'C' ))\cdot
				\dfrac{1}{2}S_{A'B'C'D'}\\&=\dfrac{1}{6}AA'.S_{A'B'C'D'}\\&=\dfrac{1}{6}V_1.
			\end{aligned}$\\
			$\Rightarrow \dfrac{V_1}{V_2}=6$.}{
			\begin{tikzpicture}[line cap=round,line join=round,font=\footnotesize,>=stealth,scale=1.1]
				\fill (0,0) coordinate [label=left:$A'$] (A') circle(1pt)
				(4,0) coordinate (B') node[shift={(-90:1ex)}]{$B'$} circle(1pt)
				(40:2) coordinate [label=below:$D'$] (D') circle(1pt)
				(80:3) coordinate [label=left:$A$] (A) circle(1pt)
				($(B')+(D')$) coordinate [label=right:$C'$] (C') circle(1pt)
				($(A)+(B')$) coordinate [label=right:$B$] (B) circle(1pt)
				($(A)+(C')$) coordinate [label=right:$C$] (C) circle(1pt)
				($(A)+(D')$) coordinate [label=above:$D$] (D) circle(1pt)
				($(A)!0.5!(C)$) coordinate [label=above:$I$] (I) circle(1pt)
				($(A')!0.5!(C')$) coordinate  (I') circle(1pt);
				\draw (C)--(B)--(A)--(A') (C')--(C)--(D)--(A) (B)--(B')  (A)--(C) (A')--(B')--(C') (B)--(D);
				\draw [dashed] (A')--(D')--(D) (A')--(C')--(D')  (A')--(I)--(C') (I')--(I)--(B') ;
			\end{tikzpicture}
	}}
\end{ex}

\Closesolutionfile{ans}

\indapan{10}{ans/CD13/Muc_5_6}

