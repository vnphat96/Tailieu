\documentclass[12pt,a4paper,oneside]{extbook}
\usepackage{graphicx}
\usepackage{extsizes}
\usepackage[utf8]{vietnam}
\usepackage{amsmath,amssymb,yhmath,mathrsfs}
\usepackage{pifont,fontawesome,pgfornament}
\usepackage{tikz,xcolor}
\usepackage{enumerate}
\usepackage{tkz-euclide}
\usepackage{tikz-3dplot}
\usepackage{tikz,tkz-tab}
\usetikzlibrary{calc,math,fadings,shadings,shapes,shadows}
\usepackage{pgfplots}
\usepgfplotslibrary{fillbetween}
\usepackage[top=1.5cm, bottom=1.5cm, left=2cm, right=1.5cm]{geometry}
\renewcommand{\baselinestretch}{1.4}
\usepackage[hidelinks,unicode]{hyperref}
\usepackage{currfile}
\usepackage[loigiai]{ex_test}
\usepackage{framed}
\usepackage[many]{tcolorbox}
\colorlet{tcbcol@back}{tcbcolback}
\colorlet{tcbcol@frame}{tcbcolframe}
\usepackage{varwidth}
%%=== Thiết kế header, footer
\newtheorem{bode}{Bổ đề}
\newtheorem{menhde}{Mệnh đề}
\newtheorem{bt}{\fontfamily{anttlc}\selectfont\color{teal}\faCalendar~Bài}
\newtheorem{vd}{\color{teal}\fontfamily{anttlc}\selectfont\faArchive~Ví dụ}
\def\beginbox{\begin{tcolorbox}[enhanced,colframe=teal,interior style={top color=teal!20!white,bottom color=teal!5!white},breakable,boxrule=0.75pt,arc=2mm,frame style={drop shadow}]}
	\def\endbox{\end{tcolorbox}}
\AtBeginEnvironment{vd}{
	\beginbox
	\renewcommand{\loigiai}[1]{
		\renewcommand{\immini}[2]{
			\setbox\imbox=\vbox{\hbox{##2}}
			\widthimmini=\wd\imbox
			\IMleftright{##1}{##2}}
		\endbox
		\begin{onlysolution}
			#1
		\end{onlysolution}
		\def\endbox{}
	}
}
\AtEndEnvironment{vd}{\endbox}


\def\beginboxbt{\begin{tcolorbox}[enhanced,width=\linewidth-6pt,
		enlarge top by=3pt,enlarge bottom by=3pt,
		enlarge left by=3pt,enlarge right by=3pt,frame hidden,boxrule=0pt,top=1mm,bottom=1mm,
		colframe=green!30!black,colback=green!25!white,
		borderline={0.5pt}{-0.5pt}{green!75!blue},
		borderline={1pt}{-3pt}{green!50!blue},frame style={drop shadow}]}
	\def\endboxbt{\end{tcolorbox}}
\AtBeginEnvironment{bt}{
	\beginboxbt
	\renewcommand{\loigiai}[1]{
		\renewcommand{\immini}[2]{
			\setbox\imbox=\vbox{\hbox{##2}}
			\widthimmini=\wd\imbox
			\IMleftright{##1}{##2}}
		\endboxbt
		\begin{onlysolution}
			#1
		\end{onlysolution}
		\def\endboxbt{}
	}
}
\AtEndEnvironment{bt}{\endboxbt}
\theoremstyle{plain}
\renewcommand{\nameex}{\fontfamily{pag}\selectfont\color{teal} Câu}
\def\loigiaiEX{\color{teal}\fontfamily{qag}\bfseries\strut%\faFolderOpen\ Lời giải.
	\centerline{\faCommenting\ Lời giải.}\vspace*{-7mm}
}
%\newtheorem{bt}{\fontfamily{pag}\selectfont\color{violet}Bài}
%-----------------------------------Khoanh tròn đáp án khi sd tùy chọn Dethi
\renewcommand{\TrueEX}{\stepcounter{dapan}
	{\circEX{\textbf{\color{teal}\Alph{dapan}}}} \ignorespaces}
\renewcommand{\FalseEX}{\stepcounter{dapan}
	{\circled{\textbf{\color{teal}\Alph{dapan}}}} \ignorespaces}
%------------------ DẠNG TOÁN
\newcounter{dang}\setcounter{dang}{0}
\renewcommand{\thedang}{\arabic{dang}}
\newtcolorbox{dang}[1]{enhanced, before skip=2mm,after skip=2mm, colback=black!5,colframe=black!50,boxrule=0.2mm, attach boxed title to top left={xshift=1cm,yshift*=1mm-\tcboxedtitleheight}, varwidth boxed title*=-3cm, boxed title style={frame code={ \path[fill=tcbcol@back!30!black] ([yshift=-1mm,xshift=-1mm]frame.north west) arc[start angle=0,end angle=180,radius=1mm] ([yshift=-1mm,xshift=1mm]frame.north east) arc[start angle=180,end angle=0,radius=1mm]; \path[left color=teal!60!black, right color=teal!60!black, middle color=teal!80!black] ([xshift=-2mm]frame.north west) -- ([xshift=2mm]frame.north east) [rounded corners=1mm]-- ([xshift=1mm,yshift=-1mm]frame.north east) -- (frame.south east) -- (frame.south west) -- ([xshift=-1mm,yshift=-1mm]frame.north west) [sharp corners]-- cycle; },interior engine=empty, },
	fonttitle=\bfseries, title={{\fontfamily{pbk}\selectfont\faFolderOpen \ Dạng~\stepcounter{dang}\thedang.\ #1}}
	\addcontentsline{toc}{subsection}{\faFolderOpen~Dạng~\thedang.\ #1}
}
\newenvironment{tomtat}{
	\begin{tcolorbox}[boxrule=0.5pt,arc=0mm,breakable,colback=white]
	}{\end{tcolorbox}}
\theorembodyfont{\it}
\newtheorem{dl}{\color{blue}\faToggleOn\ Định lí}[section]
\theoremseparator{.}
\theorembodyfont{\rm}
\newtheorem{cy}{\fontfamily{anttlc}\selectfont Chú ý}
\newtheorem{hq}{\fontfamily{anttlc}\selectfont Hệ quả}[section]
\newtheorem{tc}{\color{violet!80!black}\faGg\ Tính chất}[section]
\newtheorem{dn}{\color{red}\faGg\ Định nghĩa}[section]
\theoremstyle{nonumberplain}
\newtheorem{kn}{\fontfamily{anttlc}\selectfont Khái niệm}

%Định nghĩa các box lý thuyết
\def\beginboxdn{\begin{tcolorbox}
		[enhanced,breakable,pad at break*=1mm,
		opacityback=0, %ko nền
		boxrule=0pt,frame hidden, left=0.5cm, right=0.5cm, bottom=0pt, top=0cm,
		borderline west={1.5mm}{0cm}{green},borderline east={1.5mm}{0cm}{green}]}%0mm lề trái}
	\def\endboxdn{\end{tcolorbox}}
\AtBeginEnvironment{dn}{
	\beginboxdn
	}
\AtEndEnvironment{dn}{\endboxdn}


\def\beginboxdl{\begin{tcolorbox}
		[enhanced,breakable,pad at break*=1mm,
		colback=yellow!5,
%		opacityback=0, %ko nền
		boxrule=0pt,frame hidden, left=0.5cm, right=0.5cm, bottom=0.5cm, top=0.5cm,
		borderline west={0.75mm}{0mm}{red},borderline west={0.75mm}{0.75mm}{red!50!yellow},borderline west={0.75mm}{1.5mm}{yellow},borderline east={0.75mm}{0mm}{red},borderline east={0.75mm}{0.75mm}{red!50!yellow},borderline east={0.75mm}{1.5mm}{yellow}]}%0mm lề trái}
	\def\endboxdl{\end{tcolorbox}}
\AtBeginEnvironment{dl}{
	\beginboxdl
}
\AtEndEnvironment{dl}{\endboxdn}

\def\beginboxtc{\begin{tcolorbox}
	[enhanced,colback=yellow!10!white,boxrule=0pt,frame hidden,
	borderline north={1mm}{-2mm}{red},
	borderline south={1mm}{-2mm}{blue},
	borderline west={1mm}{-2mm}{green},
	borderline east={1mm}{-2mm}{yellow}]}%0mm lề trái}
	\def\endboxtc{\end{tcolorbox}}
\AtBeginEnvironment{tc}{
	\beginboxtc
}
\AtEndEnvironment{tc}{\endboxtc}



\theoremseparator{}
\theorembodyfont{\it}
\usepackage{stackengine}
\usepackage{scalerel}
\newcommand\dangersign[1][2ex]{\renewcommand\stacktype{L}\scaleto{\stackon[1.3pt]{\color{red}$\triangle$}{\fontsize{4}{0}\selectfont !}}{#1}}

%Note
\newtcolorbox{note}[1][]{skin=enhancedlast jigsaw,interior hidden,
	boxsep=0pt,top=0pt,colframe=red,coltitle=red!50!black,
	fonttitle=\bfseries\sffamily,
	attach boxed title to bottom center,
	boxed title style={empty,boxrule=0.5mm},
	varwidth boxed title=0.5\linewidth,
	underlay boxed title={
		\draw[white,line width=0.5mm]
		([xshift=0.3mm-\tcboxedtitleheight*2,yshift=0.3mm]title.north west)
		--([xshift=-0.3mm+\tcboxedtitleheight*2,yshift=0.3mm]title.north east);
		\path[draw=red,top color=white,bottom color=red!50!white,line width=0.5mm]
		([xshift=0.25mm-\tcboxedtitleheight*2,yshift=0.25mm]title.north west)
		cos +(\tcboxedtitleheight,-\tcboxedtitleheight/2)
		sin +(\tcboxedtitleheight,-\tcboxedtitleheight/2)
		-- ([xshift=0.25mm,yshift=0.25mm]title.south west)
		-- ([yshift=0.25mm]title.south east)
		cos +(\tcboxedtitleheight,\tcboxedtitleheight/2)
		sin +(\tcboxedtitleheight,\tcboxedtitleheight/2); },
	title={Lưu ý quan trọng},#1}


\newtheorem{nx}{Nhận xét.}
\newtcolorbox{note}[1][]{enhanced,breakable,
	before skip=2mm,after skip=3mm,
	boxrule=0.4pt,left=5mm,right=2mm,top=1mm,bottom=1mm,
	colback=yellow!50,colframe=yellow!20!black,sharp corners,rounded corners=southeast,arc is angular,arc=3mm,
	underlay={
		\path[fill=tcbcol@back!80!black] ([yshift=3mm]interior.south east)--++(-0.4,-0.1)--++(0.1,-0.2);
		\path[draw=tcbcol@frame,shorten <=-0.05mm,shorten >=-0.05mm] ([yshift=3mm]interior.south east)--++(-0.4,-0.1)--++(0.1,-0.2);
		\path[fill=yellow!50!black,draw=none] (interior.south west) rectangle node[white]{\Huge\bfseries !} ([xshift=4mm]interior.north west);
	},
	drop fuzzy shadow,#1}
\usepackage{fancyhdr}
%Thiết lập đánh số part, chapter, section, subsection, paragraph,subparagraph
\renewcommand{\thepart}{\Roman{part}}
\renewcommand{\thechapter}{\arabic{chapter}}
\renewcommand{\thesection}{\arabic{section}}
\renewcommand{\thesubsection}{\Alph{subsection}}
\renewcommand{\thesubsubsection}{\arabic{subsubsection}}

%Thiết kế part, chapter, section, subsection, paragraph,subparagraph
\setcounter{secnumdepth}{4}
\setcounter{tocdepth}{2}
\usepackage{titletoc}
\usepackage[explicit]{titlesec}
%===================Làm mục lục
\usepackage{ifoddpage}
%\titlecontents{section}
%[0em]{\sffamily}
%{\color{blue} Mức độ \contentslabel{2em}\color{blue}}{}
%{\color{blue}\dotfill\contentspage\hspace*{1.3ex}}
\titlecontents{section}[5pc]
{\addvspace{8pt}\bfseries\color{blue!70!black}}
{\fontsize{12pt}{18pt}\selectfont\sffamily\contentslabel[{Mức độ\,\,\thecontentslabel.}\,\,]{5pc}}
{}
{\dotfill
	\thecontentspage\hspace*{1.3ex}
}
[]
%\titlecontents{subsection}
%[4.5em]{\sffamily}
%{\color{violet}\contentslabel{2em}\color{violet}}{}
%{\color{violet}\dotfill\contentspage\hspace*{1.3ex}}
\titlecontents{subsection}[0pc]
{\addvspace{8pt}\bfseries\color{blue!70!black}}
{\fontsize{12pt}{18pt}\selectfont\sffamily\contentslabel[{}\,\,]{1pc}}
{}
{{\tiny\dotfill}
	\thecontentspage\hspace*{1.3ex}
}
[]
\makeatletter
\renewcommand*\l@part[2]{%
	\ifnum \c@tocdepth >\m@ne
	\addpenalty{-\@highpenalty}%
	\vskip 1.0em \@plus\p@
	\setlength\@tempdima{1.5em}%
	\begingroup
	\parindent \z@ \rightskip \@pnumwidth
	\parfillskip -\@pnumwidth
	\leavevmode
	\advance\leftskip\@tempdima
	\hskip -\leftskip
	\colorbox{violet}{\strut%
		\makebox[\dimexpr\textwidth-1\fboxsep-6pt\relax][l]{%
			%\fontsize{12pt}{1pt}
			\color{white}\bfseries\fontfamily{qag}\selectfont PHẦN #1%
			\nobreak\hfill\nobreak\hb@xt@\@pnumwidth{\hss #2}}}\par\smallskip
	\penalty\@highpenalty
	\endgroup
	\fi}

\renewcommand*\l@chapter[2]{%
	\ifnum \c@tocdepth >\m@ne
	\addpenalty{-\@highpenalty}%
	\vskip 1.0em \@plus\p@
	\setlength\@tempdima{1.5em}%
	\begingroup
	\parindent \z@ \rightskip \@pnumwidth
	\parfillskip -\@pnumwidth
	\leavevmode
	\advance\leftskip\@tempdima
	\hskip -\leftskip
	\colorbox{violet!80}{\strut%
		\makebox[\dimexpr\textwidth-1\fboxsep-6pt\relax][l]{%
			\sffamily%\fontsize{12pt}{1pt}
			\color{white}\bfseries\selectfont CHỦ ĐỀ #1%
			\nobreak\hfill\nobreak\hb@xt@\@pnumwidth{\hss #2}}}\par\smallskip
	\penalty\@highpenalty
	\endgroup
	\fi}




% Thiết kế part
\titlespacing{\part}{0cm}{-0.5cm}{1cm}[0cm]
\titleformat{\part}[display]
{\Large\bfseries\color{blue!60!black}}
{
		\begin{tikzpicture}[remember picture,overlay,line join=round,line cap=round]
		\coordinate (A1) at ($(current page.north west)!0.15!(current page.south west)$);
		\coordinate (B1) at ($(current page.north east)!0.05!(current page.south east)$);
		\coordinate (A1B1) at ($(A1)!0.5!(B1)$);
		\coordinate (tam) at ($(current page.north west)+(3.5,-3)$); % Tâm Elip
		%------------------------viền cong
		\fill[violet!50] % Xám
		(current page.north west)--([yshift=-2.2cm]A1) to[out=48,in=-148]
		([yshift=-0.5cm]tam) to[out=35,in=150] ([yshift=-3cm]B1)--(current page.north east)--cycle; %phủ trên 3
		\fill[red] % Đỏ
		(current page.north west)--([yshift=-1cm]A1) to[out=37,in=-150]
		([yshift=-0.2cm]tam) to[out=33,in=158] ([yshift=-2.1cm]B1)--(current page.north east)--cycle; %phủ trên 2
		\fill[cyan] % Xanh
		(current page.north west)--([yshift=-0.2cm]A1) to[out=30,in=-150]
		([yshift=0.2cm]tam) to[out=30,in=165] ([yshift=-1.3cm]B1)--(current page.north east)--cycle; %phủ trên 1
		%-----------------------VÒNG ELIP
		\begin{scope}[scale=0.65]
			%	\draw [line width=4pt] ([yshift=-3.5cm]tam)--(tam)--([xshift=3.25cm]tam);
			\def\x{0.7}
			\draw[fill=yellow!50,scale=\x] (tam) circle (3.2cm);
			\shade[even odd rule,top color=black,bottom color=black,middle color=cyan,shading angle=30,scale=\x]
			(tam) circle (3.3cm) circle (3.7cm); % Viền ngoài
			\shade[even odd rule,top color=black,bottom color=black,middle color=white,shading angle=-10,scale=\x]
			(tam) circle (3.15cm) circle (3.3cm); % Viền trong
		\end{scope}
		%-----------------------Số chương
		\draw ([yshift=-0.45cm]tam) node[red,yscale=2.5,xscale=2.8]{\fontfamily{qag}\selectfont\bfseries\thepart};
		\draw (tam) node[violet,xscale=1,yscale=1,above=0.15cm]{\fontfamily{qag}\selectfont\bfseries PHẦN};
	\end{tikzpicture}
}
{-0.5em}
{\hspace*{0.5cm}\fontsize{35pt}{1pt}\fontfamily{qag}\selectfont\centering\MakeUppercase {#1}}
[\vspace{-0.25em}\thispagestyle{empty}]% Nội dung văn bản

% Thiết kế chapter
\titleformat
{\chapter}
{\centering\normalfont\LARGE\sffamily\bfseries}
{}
{0em}% Khoảng cách giữa mục với tên tiêu đề
{	\thispagestyle{empty}
	\begin{tikzpicture}[baseline=0cm,overlay,remember picture]
	\node[inner sep=0pt,text width=0.95\textwidth] (chapname) at ([yshift=-4cm]current page.north){\fontsize{20}{0}\selectfont\color{teal}{\MakeUppercase {#1}}};
	\fill[teal] ([xshift=-2cm,yshift=-2.5cm]current page.north east) circle(0.9cm);
	\node[inner sep=0pt] (chapnum) at ([xshift=-2cm,yshift=-2.5cm]current page.north east){\color{white}\fontsize{25}{0}\selectfont{\MakeUppercase\thechapter}};
	\node[inner sep=0pt] (chapnumname) at ([xshift=-5cm,yshift=-2.4cm]current page.north east){\color{teal}\fontsize{20}{0}\selectfont{\MakeUppercase CHỦ ĐỀ}};
	\draw[teal,thick,double] ([yshift=0.05cm]chapname.north west) --([xshift=1cm,yshift=0.05cm]chapname.north east);
	\draw[teal,thick,double] ([yshift=-0.05cm]chapname.south west) --([xshift=1cm,yshift=-0.05cm]chapname.south east);
	\end{tikzpicture}
	\setcounter{ex}{0}
	\setcounter{bt}{0}
	\setcounter{vd}{0}
	\setcounter{dang}{0}
}% Nội dung văn bản
\titlespacing*{\chapter}{0cm}{0cm}{1cm}

% Thiết kế section
\titleformat{\section}
{\fontfamily{put}\selectfont\color{red!80!black}}
{\fontfamily{put}\normalsize\selectfont\bfseries\color{red!80!black}\Large MỨC ĐỘ\, \Huge\textcolor{red!80!black}{\thesection.} }
{0.5em}
{
	\LARGE\bfseries\filcenter\MakeUppercase{#1}
}
\titlespacing*{\section}{2cm}{0cm}{0cm}

%Thiết kế subsection
\titleformat
{\subsection}
{\normalfont\Large\sffamily\bfseries}
{}
{0em}% Khoảng cách giữa mục với tên tiêu đề
{\begin{tikzpicture}[baseline=0cm]
	\node[inner sep=0pt] (subsecname) {\color{teal}#1};
	\draw[path picture={\node at ([xshift=-0.8cm]subsecname.west){\color{red}\thesubsection};},shading=ball,ball color=teal!30]  ([xshift=-0.8cm]subsecname.west) circle (0.6cm);
	\draw[thick,teal] (subsecname.south east)--(subsecname.south west) arc (-30:-150:0.89cm);
	\end{tikzpicture}}% Nội dung văn bản
\titlespacing*{\subsection}{0cm}{0.7cm}{0.5cm}


\newcommand{\nen}[1]{% Định nghĩa hình tròn
	\begin{tikzpicture}[baseline=(A.base),scale=1.2]%
		\node[circle,draw=blue!70!black,fill=white,inner sep=2pt,outer sep=1pt] (A) {\color{white} #1};
		\node[circle,draw=none,fill=blue!70!black,inner sep=1pt,outer sep=1pt] (A) {\color{white} #1};
	\end{tikzpicture}%
}
\titleformat{\subsubsection}{\large\bfseries\fontfamily{qag}\selectfont\color{blue!70!black}}
{\hspace*{-1mm}\fontfamily{qag}\large\selectfont\nen{\thesubsubsection}\
}{0.5em}{\bfseries\fontfamily{qag}\selectfont\color{blue!70!black}\setcounter{de}{0}\MakeUppercase{#1}
}

%-------------Mục lục - Chapter*
%\def\mautheme{blue!50!green}
%\makeatletter
%\renewcommand{\tableofcontents}{
%\thispagestyle{empty}
%\vspace*{-4mm}
%\noindent\tikz{\fill[yellow](0,-1) rectangle}
%}
%-------------Chapter*-Mục lục
\titleformat{name=\chapter,numberless}[display]
{\normalfont\color{blue!80!black}}
{}
{0ex}
{\vspace*{0em}\hspace*{0cm}\huge\fontfamily{qag}\selectfont\bfseries\centering\MakeUppercase{#1}}
[\vspace{-1em}\thispagestyle{empty}]
%=============================





\usepackage{esvect}
\def\vec{\vv}
\def\overrightarrow{\vv}
%Lệnh của gói mathrsfs
\DeclareSymbolFont{rsfs}{U}{rsfs}{m}{n}
\DeclareSymbolFontAlphabet{\mathscr}{rsfs}
%Lệnh cung
\DeclareSymbolFont{largesymbols}{OMX}{yhex}{m}{n}
\DeclareMathAccent{\wideparen}{\mathord}{largesymbols}{"F3}
%Lệnh song song
\DeclareSymbolFont{symbolsC}{U}{txsyc}{m}{n}
\DeclareMathSymbol{\varparallel}{\mathrel}{symbolsC}{9}
\DeclareMathSymbol{\parallel}{\mathrel}{symbolsC}{9}
\newcommand{\hoac}[1]{\left[\begin{aligned}#1\end{aligned}\right.}
\newcommand{\heva}[1]{\left\{\begin{aligned}#1\end{aligned}\right.}
\renewcommand\labelitemi{\faCheckSquareO}
\renewcommand\labelitemii{$\bullet$}
\usepackage{sansmathaccent}
\pdfmapfile{+sansmathaccent.map}

\newcommand{\DAPAN}{\centerline{\fontfamily{qhv}\selectfont\color{teal}\textbf{\faGg\faGg\faGg~BẢNG ĐÁP ÁN\ \faGg\faGg\faGg}}}

\newcommand{\indapan}[2]{
	\begin{center}
		{\fontfamily{qcs}\selectfont\color{red!70!black}\bfseries BẢNG ĐÁP ÁN}\vspace{-0.5cm}
	\end{center}
	\inputansbox{#1}{#2}
}

% Kiểu đánh enum
\renewcommand{\labelenumi}[1][]{\begin{tikzpicture}[baseline=(1.base)]
		\pgfmathsetmacro{\R}{255*rnd}
		\pgfmathsetmacro{\G}{255*rnd}
		\pgfmathsetmacro{\B}{255*rnd}
		\definecolor{mau}{RGB}{\R,\G,\B}
		\node[shape=circle,fill=mau,text=white,font=\bfseries,minimum size=5mm,inner sep=0mm](1){\arabic{enumi}}; \end{tikzpicture}}
\renewcommand{\labelenumii}{\begin{tikzpicture}[baseline=(1.base)]
		\pgfmathsetmacro{\R}{255*rnd}
		\pgfmathsetmacro{\G}{255*rnd}
		\pgfmathsetmacro{\B}{255*rnd}
		\definecolor{mau}{RGB}{\R,\G,\B}
		\node[shape=circle,fill=mau,text=white,font=\bfseries,minimum size=5mm,inner sep=0mm](1){\alph{enumii})}; \end{tikzpicture}}
%\renewcommand\labelitemi{\begin{tikzpicture}[baseline=(1.base)]
%		\pgfmathsetmacro{\R}{255*rnd}
%		\pgfmathsetmacro{\G}{255*rnd}
%		\pgfmathsetmacro{\B}{255*rnd}
%		\definecolor{mau}{RGB}{\R,\G,\B}
%		\node[font=\bfseries,minimum size=5mm,inner sep=0mm](1){\color{mau}\faFirstOrder}; \end{tikzpicture}}
%\renewcommand\labelitemii{\begin{tikzpicture}[baseline=(1.base)]
%		\pgfmathsetmacro{\R}{255*rnd}
%		\pgfmathsetmacro{\G}{255*rnd}
%		\pgfmathsetmacro{\B}{255*rnd}
%		\definecolor{mau}{RGB}{\R,\G,\B}
%		\node[font=\bfseries,minimum size=5mm,inner sep=0mm](1){\color{mau}\faEnvira}; \end{tikzpicture}}
%%---------------
%%Kí hiệu để liệt kê
%\def\sao{\color{black} %\faCheckSquareO
%	\small\faCheckCircleO}



\newcounter{de}
\newcommand{\boxde}[1]{%
	\stepcounter{de}
	\setcounter{ex}{0} % đặt bộ đếm Câu về 1
	\addcontentsline{toc}{subsection}{#1~\thede} % đưa MT vào mục lục
	\begin{center}
		\begin{tikzpicture}[baseline=(A.base),scale=1]%
		%	\node[diamond,draw=none,fill=violet,inneRsep=1pt,outer sep=2pt] (A) {#1};
		%		\draw[thick, violet] (0,0)--++(2,0) (0,0)--++(-2,0);
		\node[thick,scale=1,fill=teal!80,draw=teal,minimum width=2.5cm,minimum height=0.1cm,rounded corners=2mm] (A) {\bf\fontfamily{qag}\selectfont\color{white}#1~\thede};
		\end{tikzpicture}%
	\end{center}
}
% Hộp văn bản
\newtcolorbox{hop}[1]{breakable,enhanced, colback=white,colframe=red!75!black, colbacktitle=red!50!yellow,fonttitle=\bfseries, title=#1, titlerule=1mm, titlerule style=yellow}
%---------------------------
%---------------------------

\newcommand{\page}[1]{% Định nghĩa hình tròn
	\begin{tikzpicture}[baseline=(A.base)]%
		\node[circle,draw=cyan,line width=0.5pt,fill=white,inner sep=1.5pt,outer sep=0.5pt] (A) {\color{white}\small #1};
		\node[circle,draw=none,fill=cyan,inner sep=1pt,outer sep=0.5pt] (A) {\color{white}\small #1};
	\end{tikzpicture}%
}
\pagestyle{fancy}
\fancyhf{}
\renewcommand{\footrule}{{\color{violet}\vskip-\footruleskip\vskip-\footrulewidth \hrule width\headwidth height\footrulewidth\vskip\footruleskip}}
\renewcommand{\footrulewidth}{0.5pt}
\renewcommand{\headrulewidth}{0.5pt}
\lhead{\sf\fontsize{10pt}{15pt}\selectfont \leftmark}
\rhead{\sf\fontsize{10pt}{1pt}\selectfont\color{violet}\rightmark}
\lfoot{\sf\fontsize{13pt}{0pt}\vspace*{-0.5cm}\selectfont\color{blue}{\faFacebookOfficial~ GV. Võ Hoàng Nghĩa}}
\cfoot{}
\rfoot{\page{\small\thepage}}


\usepackage[final]{pdfpages}
\usepackage{lastpage}
\def\canhgiua{\centering\arraybackslash}
\def\vdm{\begin{center}
		\color{teal}\textbf{VÍ DỤ MẪU}
\end{center}}
\def\bttl{\begin{center}
		\color{teal}\textbf{BÀI TẬP TỰ LUYỆN}
\end{center}}
\begin{document}
		\pagenumbering{gobble}%tắt đánh số trang
%		\includepdf{bia/bia.pdf}%chèn bìa pdf
		\pagenumbering{roman}%đánh số trang dạng i,ii,...
		\thispagestyle{empty}
		\tableofcontents %lệnh in mục lục chính
		\cleardoublepage %tạo trang trống
		\clearpage%đặt lại đánh số trang
		\pagenumbering{arabic}%đánh số trang dạng 1,2,...


	\pagenumbering{arabic}
\setcounter{page}{1}
\setcounter{chapter}{0}
%\part{GIẢI TÍCH}
\chapter{TÍNH ĐƠN ĐIỆU CỦA HÀM SỐ}
\begin{Solution}{1}
C
\end{Solution}
\begin{Solution}{3}
B
\end{Solution}
\begin{Solution}{4}
A
\end{Solution}
\begin{Solution}{5}
A
\end{Solution}
\begin{Solution}{6}
A
\end{Solution}
\begin{Solution}{7}
B
\end{Solution}
\begin{Solution}{8}
A
\end{Solution}
\begin{Solution}{9}
C
\end{Solution}
\begin{Solution}{10}
B
\end{Solution}
\begin{Solution}{11}
C
\end{Solution}
\begin{Solution}{12}
D
\end{Solution}
\begin{Solution}{13}
B
\end{Solution}
\begin{Solution}{14}
D
\end{Solution}
\begin{Solution}{15}
A
\end{Solution}
\begin{Solution}{16}
B
\end{Solution}
\begin{Solution}{17}
C
\end{Solution}
\begin{Solution}{18}
C
\end{Solution}
\begin{Solution}{19}
C
\end{Solution}
\begin{Solution}{20}
B
\end{Solution}
\begin{Solution}{21}
C
\end{Solution}
\begin{Solution}{22}
B
\end{Solution}
\begin{Solution}{23}
D
\end{Solution}
\begin{Solution}{24}
B
\end{Solution}
\begin{Solution}{25}
D
\end{Solution}
\begin{Solution}{26}
D
\end{Solution}
\begin{Solution}{27}
B
\end{Solution}
\begin{Solution}{28}
A
\end{Solution}
\begin{Solution}{29}
C
\end{Solution}
\begin{Solution}{30}
B
\end{Solution}
\begin{Solution}{31}
D
\end{Solution}
\begin{Solution}{32}
B
\end{Solution}
\begin{Solution}{33}
B
\end{Solution}
\begin{Solution}{34}
C
\end{Solution}
\begin{Solution}{35}
D
\end{Solution}
\begin{Solution}{36}
B
\end{Solution}
\begin{Solution}{37}
B
\end{Solution}
\begin{Solution}{38}
A
\end{Solution}
\begin{Solution}{39}
A
\end{Solution}
\begin{Solution}{40}
D
\end{Solution}
\begin{Solution}{41}
C
\end{Solution}
\begin{Solution}{42}
B
\end{Solution}
\begin{Solution}{43}
A
\end{Solution}
\begin{Solution}{44}
A
\end{Solution}
\begin{Solution}{45}
D
\end{Solution}
\begin{Solution}{46}
C
\end{Solution}
\begin{Solution}{47}
A
\end{Solution}
\begin{Solution}{48}
B
\end{Solution}
\begin{Solution}{49}
B
\end{Solution}
\begin{Solution}{50}
B
\end{Solution}
\begin{Solution}{51}
A
\end{Solution}
\begin{Solution}{52}
A
\end{Solution}
\begin{Solution}{53}
C
\end{Solution}
\begin{Solution}{54}
C
\end{Solution}
\begin{Solution}{55}
C
\end{Solution}
\begin{Solution}{56}
B
\end{Solution}
\begin{Solution}{57}
C
\end{Solution}
\begin{Solution}{58}
C
\end{Solution}
\begin{Solution}{59}
B
\end{Solution}
\begin{Solution}{60}
C
\end{Solution}
\begin{Solution}{61}
A
\end{Solution}
\begin{Solution}{62}
B
\end{Solution}
\begin{Solution}{63}
B
\end{Solution}
\begin{Solution}{64}
D
\end{Solution}
\begin{Solution}{65}
D
\end{Solution}
\begin{Solution}{66}
B
\end{Solution}
\begin{Solution}{67}
A
\end{Solution}
\begin{Solution}{68}
D
\end{Solution}

\begin{Solution}{1}
D
\end{Solution}
\begin{Solution}{2}
C
\end{Solution}
\begin{Solution}{3}
C
\end{Solution}
\begin{Solution}{4}
A
\end{Solution}
\begin{Solution}{5}
B
\end{Solution}
\begin{Solution}{6}
D
\end{Solution}
\begin{Solution}{7}
C
\end{Solution}
\begin{Solution}{8}
D
\end{Solution}
\begin{Solution}{9}
A
\end{Solution}
\begin{Solution}{10}
B
\end{Solution}
\begin{Solution}{11}
D
\end{Solution}
\begin{Solution}{12}
A
\end{Solution}
\begin{Solution}{13}
D
\end{Solution}
\begin{Solution}{14}
B
\end{Solution}
\begin{Solution}{15}
B
\end{Solution}
\begin{Solution}{16}
C
\end{Solution}
\begin{Solution}{1}
A
\end{Solution}
\begin{Solution}{2}
B
\end{Solution}
\begin{Solution}{3}
D
\end{Solution}
\begin{Solution}{4}
D
\end{Solution}
\begin{Solution}{5}
C
\end{Solution}
\begin{Solution}{6}
A
\end{Solution}
\begin{Solution}{7}
D
\end{Solution}
\begin{Solution}{8}
B
\end{Solution}
\begin{Solution}{9}
C
\end{Solution}
\begin{Solution}{10}
C
\end{Solution}
\begin{Solution}{1}
D
\end{Solution}
\begin{Solution}{2}
D
\end{Solution}
\begin{Solution}{3}
B
\end{Solution}
\begin{Solution}{4}
C
\end{Solution}
\begin{Solution}{5}
D
\end{Solution}
\begin{Solution}{6}
A
\end{Solution}
\begin{Solution}{7}
C
\end{Solution}
\begin{Solution}{8}
B
\end{Solution}
\begin{Solution}{9}
A
\end{Solution}
\begin{Solution}{10}
C
\end{Solution}
\begin{Solution}{11}
D
\end{Solution}
\begin{Solution}{12}
C
\end{Solution}
\begin{Solution}{13}
A
\end{Solution}
\begin{Solution}{14}
D
\end{Solution}
\begin{Solution}{15}
A
\end{Solution}
\begin{Solution}{16}
A
\end{Solution}
\begin{Solution}{17}
B
\end{Solution}
\begin{Solution}{18}
C
\end{Solution}
\begin{Solution}{19}
C
\end{Solution}
\begin{Solution}{20}
A
\end{Solution}
\begin{Solution}{21}
D
\end{Solution}
\begin{Solution}{22}
C
\end{Solution}
\begin{Solution}{23}
A
\end{Solution}
\begin{Solution}{24}
C
\end{Solution}
\begin{Solution}{25}
A
\end{Solution}
\begin{Solution}{26}
B
\end{Solution}
\begin{Solution}{27}
B
\end{Solution}
\begin{Solution}{28}
D
\end{Solution}
\begin{Solution}{29}
B
\end{Solution}
\begin{Solution}{30}
D
\end{Solution}
\begin{Solution}{31}
D
\end{Solution}
\begin{Solution}{32}
C
\end{Solution}
\begin{Solution}{33}
D
\end{Solution}
\begin{Solution}{34}
C
\end{Solution}
\begin{Solution}{35}
D
\end{Solution}
\begin{Solution}{36}
D
\end{Solution}
\begin{Solution}{37}
D
\end{Solution}
\begin{Solution}{38}
D
\end{Solution}
\begin{Solution}{39}
D
\end{Solution}
\begin{Solution}{40}
C
\end{Solution}
\begin{Solution}{41}
A
\end{Solution}
\begin{Solution}{1}
A
\end{Solution}
\begin{Solution}{2}
B
\end{Solution}
\begin{Solution}{3}
C
\end{Solution}
\begin{Solution}{4}
A
\end{Solution}
\begin{Solution}{5}
A
\end{Solution}
\begin{Solution}{6}
C
\end{Solution}
\begin{Solution}{7}
C
\end{Solution}
\begin{Solution}{8}
B
\end{Solution}
\begin{Solution}{9}
C
\end{Solution}
\begin{Solution}{10}
B
\end{Solution}
\begin{Solution}{11}
A
\end{Solution}
\begin{Solution}{12}
B
\end{Solution}
\begin{Solution}{13}
B
\end{Solution}
\begin{Solution}{14}
B
\end{Solution}
\begin{Solution}{15}
A
\end{Solution}
\begin{Solution}{16}
B
\end{Solution}
\begin{Solution}{17}
A
\end{Solution}
\begin{Solution}{18}
D
\end{Solution}
\begin{Solution}{19}
C
\end{Solution}
\begin{Solution}{20}
C
\end{Solution}
\begin{Solution}{21}
A
\end{Solution}
\begin{Solution}{22}
C
\end{Solution}
\begin{Solution}{23}
C
\end{Solution}
\begin{Solution}{24}
A
\end{Solution}
\begin{Solution}{25}
B
\end{Solution}
\begin{Solution}{26}
B
\end{Solution}
\begin{Solution}{27}
A
\end{Solution}
\begin{Solution}{28}
A
\end{Solution}
\begin{Solution}{29}
C
\end{Solution}
\begin{Solution}{30}
B
\end{Solution}
\begin{Solution}{31}
A
\end{Solution}
\begin{Solution}{32}
C
\end{Solution}
\begin{Solution}{33}
B
\end{Solution}
\begin{Solution}{34}
A
\end{Solution}
\begin{Solution}{35}
B
\end{Solution}
\begin{Solution}{36}
B
\end{Solution}
\begin{Solution}{37}
B
\end{Solution}
\begin{Solution}{38}
D
\end{Solution}
\begin{Solution}{39}
B
\end{Solution}
\begin{Solution}{40}
A
\end{Solution}
\begin{Solution}{41}
D
\end{Solution}
\begin{Solution}{42}
D
\end{Solution}
\begin{Solution}{43}
A
\end{Solution}
\begin{Solution}{44}
D
\end{Solution}
\begin{Solution}{45}
C
\end{Solution}
\begin{Solution}{46}
B
\end{Solution}
\begin{Solution}{47}
A
\end{Solution}
\begin{Solution}{48}
D
\end{Solution}
\begin{Solution}{49}
B
\end{Solution}
\begin{Solution}{50}
B
\end{Solution}
\begin{Solution}{51}
D
\end{Solution}
\begin{Solution}{52}
C
\end{Solution}
\begin{Solution}{53}
C
\end{Solution}
\begin{Solution}{54}
B
\end{Solution}
\begin{Solution}{55}
D
\end{Solution}
\begin{Solution}{56}
B
\end{Solution}
\begin{Solution}{57}
C
\end{Solution}
\begin{Solution}{58}
A
\end{Solution}
\begin{Solution}{59}
A
\end{Solution}
\begin{Solution}{60}
B
\end{Solution}
\begin{Solution}{61}
D
\end{Solution}
\begin{Solution}{62}
D
\end{Solution}
\begin{Solution}{63}
B
\end{Solution}
\begin{Solution}{64}
A
\end{Solution}
\begin{Solution}{65}
D
\end{Solution}
\begin{Solution}{66}
C
\end{Solution}
\begin{Solution}{67}
A
\end{Solution}
\begin{Solution}{68}
A
\end{Solution}
\begin{Solution}{69}
D
\end{Solution}
\begin{Solution}{70}
C
\end{Solution}
\begin{Solution}{71}
B
\end{Solution}
\begin{Solution}{72}
A
\end{Solution}
\begin{Solution}{73}
C
\end{Solution}
\begin{Solution}{74}
C
\end{Solution}
\begin{Solution}{75}
C
\end{Solution}
\begin{Solution}{76}
A
\end{Solution}
\begin{Solution}{77}
C
\end{Solution}
\begin{Solution}{78}
B
\end{Solution}
\begin{Solution}{79}
D
\end{Solution}
\begin{Solution}{80}
B
\end{Solution}

\section{Mức 9,10 điểm}
\setcounter{ex}{0}
\setcounter{dang}{0}
\Opensolutionfile{ans}[ans/CD1/Muc_9_10]
\begin{dang}{Tìm m để hàm số đơn điệu trên các khoảng xác định của nó}
	Đang thiếu bài thầy Jf Câu 1 đến 26 
\end{dang}
\begin{dang}
	{Tìm khoảng đơn điệu của hàm số $g(x) = f\left[ u(x)\right] +v(x)$ khi biết đồ thị hoặc bảng biến thiên của hàm số $y = f'(x)$}
\end{dang}
\begin{ex}[Đề tham khảo 2019]%[2D1K1-2]
	Cho hàm số $f(x)$ có bảng xét dấu của đạo hàm như sau
	\begin{center}
		\begin{tikzpicture}
			\tkzTabInit[nocadre,lgt=1.2,espcl=2,deltacl=0.6]
			{$x$ /0.6,$f'(x)$ /0.6}
			{$-\infty$,$1$,$2$,$3$,$4$,$+\infty$}
			\tkzTabLine{,-,$0$,+,$0$,+,$0$,-,$0$,+,}
		\end{tikzpicture}
	\end{center}
	Hàm số $y=3 f(x+2)-x^3+3 x$ đồng biến trên khoảng nào dưới đây?
	\choice
	{$(-\infty ;-1)$}
	{\True $(-1 ; 0)$}
	{$(0 ; 2)$}
	{$(1 ;+\infty)$}
	\loigiai{
		Ta có $y'=3\left[f'(x+2)-\left(x^2-3\right)\right]$.\\
		Với $x \in(-1 ; 0) \Rightarrow x+2 \in(1 ; 2) \Rightarrow f'(x+2)>0$, lại có $x^2-3<0 \Rightarrow y'>0 ;~ \forall x \in(-1 ; 0)$.\\
		Vậy hàm số $y=3 f(x+2)-x^3+3 x$ đồng biến trên khoảng $(-1 ; 0)$.\\
		Chú ý:\\
		+) Ta xét $x \in(1 ; 2) \subset(1 ;+\infty)
		\Rightarrow x+2 \in(3 ; 4)\\
		\Rightarrow f'(x+2)<0 ;~ x^2-3>0$\\
		Suy ra hàm số nghịch biến trên khoảng $(1 ; 2)$ nên loại hai phương án$(0 ; 2)$ và $(1 ;+\infty)$.\\
		+) Tương tự ta xét
		$x \in(-\infty ;-2) \Rightarrow x+2 \in(-\infty ; 0)\\
		\Rightarrow f'(x+2)<0 ; x^2-3>0 \Rightarrow y'<0 ; ~ \forall x \in(-\infty ;-2)$.\\
		Suy ra hàm số nghịch biến trên khoảng $(-\infty ;-2)$ nên loại$(-\infty ;-1)$.\\
		Vậy hàm số đã cho đồng biến trên khoảng $(-1 ; 0)$.
	}
\end{ex}
\begin{ex}[Đề Tham Khảo 2020 - Lần 1]%[2D1G1-2]
	\immini{
		Cho hàm số $f(x)$. Hàm số $y=f'(x)$ có đồ thị như hình bên. Hàm số $g(x)=f(1-2 x)+x^2-x$ nghịch biến trên khoảng nào dưới đây?
		\choice
		{\True $\left(1 ; \dfrac{3}{2}\right)$}
		{$\left(0 ; \dfrac{1}{2}\right)$}
		{$(-2 ;-1)$}
		{$(2 ; 3)$}
	}
	{
		\begin{tikzpicture}[scale=0.7,>=stealth, font=\footnotesize, line join=round, line cap=round]
			%\def\a{1} \def\b{-6} \def\c{9} \def\d{1} % Hệ số
			\def\xmin{-4} \def\xmax{6}
			\def\ymin{-3} \def\ymax{2} 
			%\draw[color=gray!50,dashed] (\xmin,\ymin) grid (\xmax,\ymax); 
			\draw[->] (\xmin,0)--(\xmax,0) node [below]{$x$};
			\draw[->] (0,\ymin)--(0,\ymax) node [left]{$y$};
			\node at (0,0) [below left]{$O$};
			%\node at (1,3) [below left]{$f'(x)$};
			%\node at (-1.3,4) {$f'(x)$};
			\draw[dashed] (-2,0) node[below]{$-2$}--(-2,1)--(0,1) node[below left]{$1$};
			\draw[dashed] (4,0) node[below left]{$4$}--(4,-2)--(0,-2) node[below left]{$-2$};
			%\draw[dashed] (1,0) node[below]{$1$}--(1,1);
			%\draw[dashed] (-0.5,0) node[below left]{$-0{,}5$}--(-0.5,2.125);
			\clip (\xmin+0.1,\ymin+0.1) rectangle (\xmax-0.5,\ymax-0.1);
			\draw[smooth,samples=300][domain=-4:5.5] plot(\x,{0.071*(\x)^3-0.142*(\x)^2-1.07*(\x)});
		\end{tikzpicture}
	}
	
	\loigiai{
		Ta có : $g(x)=f(1-2 x)+x^2-x \Rightarrow g'(x)=-2 f'(1-2 x)+2 x-1$.\\
		\immini{
			Đặt $t=1-2 x \Rightarrow g'(x)=-2 f'(t)-t$.\\
			$g'(x)=0 \Rightarrow f'(t)=-\dfrac{t}{2}$.\\
			Vẽ đường thẳng $y=-\dfrac{x}{2}$ và đồ thị hàm số $f'(x)$ trên cùng một hệ trục
		}	
		{
			\begin{tikzpicture}[scale=0.7,>=stealth, font=\footnotesize, line join=round, line cap=round]
				%\def\a{1} \def\b{-6} \def\c{9} \def\d{1} % Hệ số
				\def\xmin{-4} \def\xmax{6}
				\def\ymin{-3} \def\ymax{2} 
				%	\draw[color=gray!50,dashed] (\xmin,\ymin) grid (\xmax,\ymax); 
				\draw[->] (\xmin,0)--(\xmax,0) node [below]{$x$};
				\draw[->] (0,\ymin)--(0,\ymax) node [left]{$y$};
				\node at (0,0) [below left]{$O$};
				%\node at (1,3) [below left]{$f'(x)$};
				%\node at (-1.3,4) {$f'(x)$};
				\draw[dashed] (-2,0) node[below]{$-2$}--(-2,1)--(0,1) node[below left]{$1$};
				\draw[dashed] (4,0) node[below]{$4$}--(4,-2)--(0,-2) node[below left]{$-2$};
				%\draw[dashed] (1,0) node[below]{$1$}--(1,1);
				%\draw[dashed] (-0.5,0) node[below left]{$-0{,}5$}--(-0.5,2.125);
				\clip (\xmin+0.1,\ymin+0.1) rectangle (\xmax-0.5,\ymax-0.1);
				\draw[smooth,samples=300][domain=-4:5.5] plot(\x,{0.071*(\x)^3-0.142*(\x)^2-1.07*(\x)});
				\draw[smooth,samples=300][domain=-4:5.5] plot(\x,{(-0.5*(\x)});
			\end{tikzpicture}
		}	Hàm số $g(x)$ nghịch biến $\Rightarrow g'(x) \leq 0 \Rightarrow f'(t) \geq-\dfrac{t}{2}\Rightarrow\hoac{&-2 \leq t \leq 0 \\&t \geq 4.}$\\
		Như vậy $f'(1-2 x) \geq \dfrac{1-2 x}{-2}\Rightarrow\hoac{&-2 \leq 1-2 x \leq 0 \\ &4 \leq 1-2 x}\Rightarrow\hoac{&\dfrac{1}{2}\leq x \leq \dfrac{3}{2}\\ &x \leq-\dfrac{3}{2}.}$\\
		Vậy hàm số $g(x)=f(1-2 x)+x^2-x$ nghịch biến trên các khoảng $\left(\dfrac{1}{2}; \dfrac{3}{2}\right)$ và $\left(-\infty ;-\dfrac{3}{2}\right)$.\\
		Mà $\left(1 ; \dfrac{3}{2}\right) \subset \left(\dfrac{1}{2}; \dfrac{3}{2}\right)$ nên hàm số $g(x)=f(1-2 x)+x^2-x$ nghịch biến trên khoảng $\left(1 ; \dfrac{3}{2}\right)$.
	}
\end{ex}
\begin{ex}[Chuyên Lê Quý Đôn Điện Biên 2019]%[2D1G1-2]
	Cho hàm số $f(x)$ có bảng xét dấu của đạo hàm như sau
	\begin{center}
		\begin{tikzpicture}
			\tkzTabInit[nocadre,lgt=1.2,espcl=2,deltacl=0.6]
			{$x$ /0.6,$f'(x)$ /0.6}
			{$-\infty$,$0$,$1$,$2$,$3$,$+\infty$}
			\tkzTabLine{,+,$0$,-,$0$,-,$0$,+,$0$,-,}
		\end{tikzpicture}
	\end{center}
	Hàm số $y=f(x-1)+x^3-12 x+2019$ nghịch biến trên khoảng nào dưới đây?
	\choice
	{$(1 ;+\infty)$}
	{\True $(1 ; 2)$}
	{$(-\infty ; 1)$}
	{$(3 ; 4)$}
	\loigiai{
		$y'=f'(x-1)+3 x^2-12=f'(t)+3 t^2+6 t-9=f'(t)-\left(-3 t^2-6 t+9\right)$, với $t=x-1$.\\
		\immini{
			Nghiệm của phương trình $y'=0$ là hoành độ giao điểm của các đồ thị hàm số $y=f'(t)$ và $y=-3 t^2-6 t+9$.\\
			Vẽ đồ thị hàm số $y=f'(t)$ và $y=-3 t^2-6 t+9$ trên cùng một hệ trục tọa độ như hình vẽ bên.
		}	
		{		\begin{tikzpicture}[scale=0.5,>=stealth, font=\footnotesize, line join=round, line cap=round]
				\def\a{-3} \def\b{-6} \def\c{9} % Hệ số
				\def\xmin{-9} \def\xmax{7}
				\def\ymin{-3} \def\ymax{13}
				
				%\draw[color=gray!50,dashed] (\xmin,\ymin) grid (\xmax,\ymax);
				
				\draw[->] (\xmin,0)--(\xmax,0) node [below]{$x$};
				\draw[->] (0,\ymin)--(0,\ymax) node [left]{$y$};
				\node at (0,0) [below left]{$O$};
				\clip (\xmin+0.1,\ymin+0.1) rectangle (\xmax-0.5,\ymax-0.1);
				\draw[smooth,samples=300] plot(\x,{\a*(\x)^2+\b*(\x)+\c});
				\node at (1,0) [above right]{$1$};
				\node at (2,0) [below right]{$2$};
				\node at (3,0) [below right]{$3$};
				\node at (-3,-2) [left]{$y=-3t^2-6t+9$};
				\node at (4,0) [below right]{$f'(x)$};
				\draw (-2.2,10).. controls (-1,1.9) and (-0.5,0.8) .. (0,0);
				%\draw (-2,0).. controls (-1.5,-2) and (-0.5,-0) .. (0,0);
				\draw (0,0).. controls (0.4,-0.6) and (0.6,-0.6) .. (0.8,-0.2);
				\draw (0.8,-0.2).. controls (1,0.25) and (1.1,-0.1) .. (1.4,-0.8);
				\draw (1.4,-0.8).. controls (1.6,-1.1) and (1.7,-0.9) .. (2,0);
				\draw (2,0).. controls (2.4,1.1) and (2.6,1.1) .. (3.5,-1);
			\end{tikzpicture}
		}
		Dựa vào đồ thị trên, ta có bảng xét dấu của hàm số $y'=f'(t)-\left(-3 t^2-6 t+9\right)$ như sau $
		\left(t_0<-1\right)$
		\begin{center}
			\begin{tikzpicture}
				\tkzTabInit[nocadre,lgt=2,espcl=2,deltacl=0.6]
				{$x$ /0.6,$y'$ /0.6}
				{$-\infty$,$t_0$,$1$,$+\infty$}
				\tkzTabLine{,+,$0$,-,$0$,+,}
			\end{tikzpicture}
		\end{center}
		Hàm số nghịch biến trên khoảng $t \in\left(t_0 ; 1\right)$.\\
		Do đó hàm số nghịch biến trên khoảng $x \in(1 ; 2) \subset \left(t_0+1 ; 1\right)$.
	}
\end{ex}


\begin{ex}[Chuyên Phan Bội Châu Nghệ An 2019]%[2D1G1-2]
	Cho hàm số $f(x)$ có bảng xét dấu đạo hàm như sau:
	\begin{center}
		\begin{tikzpicture}
			\tkzTabInit[nocadre,lgt=2,espcl=2,deltacl=0.6]
			{$x$ /0.6,$f'(x)$ /0.6}
			{$-\infty$,$1$,$2$,$3$,$4$,$+\infty$}
			\tkzTabLine{,-,$0$,+,$0$,+,$0$,-,$0$,+,}
		\end{tikzpicture}
	\end{center}
	Hàm số $y=2 f(1-x)+\sqrt{x^2+1}-x$ nghịch biến trên những khoảng nào dưới đây
	\choice
	{$(-\infty ;-2)$}
	{$(-\infty ; 1)$}
	{\True $(-2 ; 0)$}
	{$(-3 ;-2)$}
	\loigiai{
		$y'=-2 f'(1-x)+\dfrac{x}{\sqrt{x^2+1}}-1$. \\
		Có $\dfrac{x}{\sqrt{x^2+1}}-1<0,~ \forall x \in(-2 ; 0)$.\\
		Bảng xét dấu:
		\begin{center}
			\begin{tikzpicture}
				\tkzTabInit[nocadre,lgt=2,espcl=2,deltacl=0.6]
				{$x$ /0.7,$f'(1-x)$ /0.7}
				{$-\infty$,$-3$,$-2$,$-1$,$0$,$+\infty$}
				\tkzTabLine{,+,$0$,-,$0$,+,$0$,+,$0$,-,}
			\end{tikzpicture}
		\end{center}
		$\Rightarrow-2 f'(1-x)<0, ~ \forall x \in(-2 ; 0) \\
		\Rightarrow-2 f'(1-x)+\dfrac{x}{\sqrt{x^2+1}}-1<0, ~\forall x \in(-2 ; 0)$.
	}
\end{ex}
\begin{ex}[Sở Vĩnh Phúc 2019]%[2D1G1-2]
	\immini{
		Cho hàm số bậc bốn $y=f(x)$ có đồ thị của hàm số $y=f'(x)$ như hình vẽ bên.\\
		Hàm số $y=3 f(x)+x^3-6 x^2+9 x$ đồng biến trên khoảng nào trong các khoảng sau đây?
		\choice
		{$(0 ; 2)$}
		{$(-1 ; 1)$}
		{$(1 ;+\infty)$}
		{\True $(-2 ; 0)$}
	}
	{
		\begin{tikzpicture}[scale=0.7,>=stealth, font=\footnotesize, line join=round, line cap=round]
			\def\a{0.21} \def\b{0.88} \def\c{-0.58} \def\d{-3} % Hệ số
			\def\xmin{-5} \def\xmax{5}
			\def\ymin{-4} \def\ymax{3} 
			%\draw[color=gray!50,dashed] (\xmin,\ymin) grid (\xmax,\ymax); 
			\draw[->] (\xmin,0)--(\xmax,0) node [below]{$x$};
			\draw[->] (0,\ymin)--(0,\ymax) node [left]{$y$};
			\node at (0,0) [above left]{$O$};
			\node at (-4,0) [below left]{$-4$};
			\node at (-2,0) [below left]{$-2$};
			\node at (0,-3) [below right]{$-3$};
			\draw[dashed] (2,0) node[above right]{$2$}--(2,1) --(0,1) node[above right]{$1$};
			\clip (\xmin+0.1,\ymin+0.1) rectangle (\xmax-0.5,\ymax-0.1);
			\draw[smooth,samples=300] plot(\x,{\a*(\x)^3+\b*(\x)^2+\c*(\x)+\d});
		\end{tikzpicture}
	}
	
	\loigiai{
		Hàm số $f(x)=a x^4+b x^3+c x^2+d x+e,(a \neq 0)$.
		Có $f'(x)=4 a x^3+3 b x^2+2 c x+d$.\\
		Đồ thị hàm số $y=f'(x)$ đi qua các điểm $(-4 ; 0),(-2 ; 0),(0 ;-3),(2 ; 1)$ nên ta có
		$$\heva{&- 2 5 6 a + 4 8 b - 8 c + d = 0\\
			&- 3 2 a + 1 2 b - 4 c + d = 0\\
			&d = - 3\\
			&3 2 a + 1 2 b + 4 c + d = 1}\Leftrightarrow \heva{&
			a=\dfrac{5}{96}\\
			&b=\dfrac{7}{24}\\
			&c=-\dfrac{7}{24}\\
			&d=-3.}
		$$
		Xét hàm số
		$
		y=3 f(x)+x^3-6 x^2+9 x$\\
		Ta có $ y'=3\left(f'(x)+x^2-4 x+3\right)=3\left(\frac{5}{24}x^3+\frac{15}{8}x^2-\frac{55}{12}x\right)
		$\\
		Ta có $y'=0 \Leftrightarrow\hoac{&x=-11 \\&x=0 \\&x=2.}$ \\
		Xét dấu $y'$, ta được hàm số đã cho đồng biến trên các khoảng $(-11 ; 0)$ và $(2 ;+\infty)$.
	}
\end{ex}
\begin{ex}[Học Mãi 2019]%[2D1K1-2]
	\immini
	{Cho hàm số $y=f(x)$ có đạo hàm trên $\mathbb{R}$. Đồ thị hàm số $y=f'(x)$ như hình bên. Hỏi đồ thị hàm số $y=f(x)-2 x$ có bao nhiêu điểm cực trị?
		\choice
		{$4$}
		{\True $3$}
		{$2$}
		{$1$}
	}
	{
		\begin{tikzpicture}[font=\footnotesize,line join=round, line cap=round,>=stealth,scale=0.8]
			\draw[->] (-3.5,0)--(4,0) node[above] {$x$};
			\draw[->] (0,-3)--(0,4) node[left] {$y$};
			%\fill[black] (-2,0)node[below left]{$-2$} circle (1.2pt) (0,0)node[above right]{$O$} circle (1.2pt) (3,0)node[above]{$3$} circle (1.2pt);
			\draw[dashed] (-2,-2)-- (0,-2) node[right]{$-2$};
			\draw[dashed] (2,0) node[below]{$2$}-- (2,2)--(0,2) node[below left]{$2$};
			\node at (0,0) [below left]{$O$};
			\node at (3,0) [below right]{$3$};
			\draw (-3,2.5).. controls (-2.2,-3) and (-1.8,-3) .. (-1.1,0);
			\draw (-1.1,0).. controls (-0.6,2.5) and (-0.4,2.5) .. (0,2);
			\draw (0,2).. controls (0.7,0.5) and (1.1,0.5) .. (1.5,1.5);
			\draw (1.5,1.5).. controls (2,2.5) and (2.8,2.5) .. (3.5,-2.5);
			%\draw (3,0).. controls (3.3,-0.1) and (3.5,-0.5) .. (3.5,-2);
		\end{tikzpicture}
	}
	\loigiai{
		\immini{
			Đặt $g(x)=f(x)-2 x$.\\
			$\Rightarrow g'(x)=f'(x)-2 .
			$\\
			Vẽ đường thẳng $y=2$.\\
			$\Rightarrow$ phương trình $g'(x)=0$ có $3$ nghiệm bội lẻ.\\
			$\Rightarrow$ đồ thị hàm số $y=f(x)-2 x$ có $3$ điểm cực trị.
		}
		{
			\begin{tikzpicture}[font=\footnotesize,line join=round, line cap=round,>=stealth,scale=0.8]
				\draw[->] (-3.5,0)--(4,0) node[above] {$x$};
				\draw[->] (0,-3)--(0,4) node[left] {$y$};
				%\fill[black] (-2,0)node[below left]{$-2$} circle (1.2pt) (0,0)node[above right]{$O$} circle (1.2pt) (3,0)node[above]{$3$} circle (1.2pt);
				\draw[dashed] (-2,-2)-- (0,-2) node[right]{$-2$};
				\draw[dashed] (2,0) node[below]{$2$}-- (2,2)--(0,2) node[below left]{$2$};
				\node at (3,0) [below left]{$3$};
				\draw (-3,2.5).. controls (-2.2,-3) and (-1.8,-3) .. (-1.1,0);
				\draw (-1.1,0).. controls (-0.6,2.5) and (-0.4,2.5) .. (0,2);
				\draw (0,2).. controls (0.7,0.5) and (1.1,0.5) .. (1.5,1.5);
				\draw (1.5,1.5).. controls (2,2.5) and (2.8,2.5) .. (3.5,-2.5);
				\draw (-3.5,2)--(4,2) node[above]{$y=2$};
			\end{tikzpicture}
		}
	}
\end{ex}
\begin{ex}[THPT Hoàng Hoa Thám Hưng Yên 2019]%[2D1G1-2]
	\immini{
		Cho hàm số $y=f(x)$ liên tục trên $\mathbb{R}$. Hàm số $y=f'(x)$ có đồ thị như hình vẽ. 
		Hàm số $g(x)=f(x-1)+\dfrac{2019-2018 x}{2018}$ đồng biến trên khoảng nào dưới đây?
		\choice
		{$(2 ; 3)$}
		{$(0 ; 1)$}
		{\True $(-1 ; 0)$}
		{$(1 ; 2)$}
	}
	{
		\begin{tikzpicture}[scale=1, font=\footnotesize, line join=round, line cap=round, >=stealth]
			\tikzset{label style/.style={font=\footnotesize}}
			\draw[->] (-2,0)--(3,0) node[below left] {$x$};
			\draw[->] (0,-2)--(0,3) node[below left] {$y$};
			\draw[fill=black] (0,0) node [above left] {$O$} circle(1pt);
			\fill (1,1) circle(1pt) (-1,1) circle(1pt) (2,1) circle(1pt);
			\foreach \x in {1,2}
			\draw[thin] (\x,1pt)--(\x,-1pt) node [below] {\footnotesize$\x$};
			\foreach \x in {-1}
			\draw[thin] (\x,1pt)--(\x,-1pt) node [below left] {\footnotesize$\x$};
			\foreach \y in {-1}
			\draw[thin] (1pt,\y)--(-1pt,\y) node [right] {\footnotesize$\y$};
			\foreach \y in {1}
			\draw[thin] (1pt,\y)--(-1pt,\y) node [above left] {\footnotesize$\y$};
			\draw[dashed](-1,0)--(-1,1)--(2,1) (1,1)--(1,0) (2,1)--(2,0);
			\begin{scope}
				\clip (-3,-3) rectangle (3,3);
				\draw[name path=(C)] plot[smooth,tension=0.7] coordinates{(-1.15,3)(-0.5,-1.6)(.8,.88)(1.9,0.8)(2.3,3)};
			\end{scope}
		\end{tikzpicture}
	}	\loigiai{
		Ta có $g'(x)=f'(x-1)-1$.\\
		$
		g'(x) \geq 0 \Leftrightarrow f'(x-1)-1 \geq 0 \Leftrightarrow f'(x-1) \geq 1 \Leftrightarrow \hoac{&x - 1 \leq - 1\\
			&x - 1 \geq 2}\Leftrightarrow \hoac{&
			x \leq 0 \\
			&x \geq 3.}
		$\\
		Từ đó suy ra hàm số $g(x)=f(x-1)+\dfrac{2019-2018 x}{2018}$ đồng biến trên khoảng $(-1 ; 0)$.
	}
\end{ex}

\begin{ex}[(Sở Ninh Bình 2019]%[2D1K1-2]
	Cho hàm số $y=f(x)$ có bảng xét dấu của đạo hàm như sau
	\begin{center}
		\begin{tikzpicture}
			\tkzTabInit[nocadre,lgt=1,espcl=2,deltacl=0.6]
			{$x$ /0.7,$f'(x)$ /0.7}
			{$-\infty$,$-2$,$-1$,$2$,$4$,$+\infty$}
			\tkzTabLine{,+,$0$,-,$0$,+,$0$,-,$0$,+,}
		\end{tikzpicture}
	\end{center}
	Hàm số $y=-2 f(x)+2019$ nghịch biến trên khoảng nào trong các khoảng dưới đây?
	\choice
	{$(-4 ; 2)$}
	{\True $(-1 ; 2)$}
	{$(-2 ;-1)$}
	{$(2 ; 4)$}
	\loigiai{
		Xét $y=g(x)=-2 f(x)+2019$.\\
		Ta có $g'(x)=(-2 f(x)+2019)'=-2 f'(x), g'(x)=0 \Leftrightarrow\hoac{&x=-2 \\&x=-1 \\&x=2 \\&x=4.}$.\\
		Ta có bảng xét dấu của $g'(x)$
		\begin{center}
			\begin{tikzpicture}
				\tkzTabInit[nocadre,lgt=1,espcl=2,deltacl=0.6]
				{$x$ /0.6,$f'(x)$ /0.6}
				{$-\infty$,$-2$,$-1$,$2$,$4$,$+\infty$}
				\tkzTabLine{,-,$0$,+,$0$,-,$0$,+,$0$,+,}
			\end{tikzpicture}
		\end{center}
		Dựa vào bảng xét dấu, ta thấy hàm số $y=g(x)$ nghịch biến trên khoảng $(-1 ; 2)$.
	}
\end{ex}
\begin{ex}[THPT Lương Thế Vinh Hà Nội 2019]%[2D1G1-2]
	\immini{
		Cho hàm số $y=f(x)$. Biết đồ thị hàm số $y=f'(x)$ có đồ thị như hình vẽ bên. 
		Hàm số $y=f \left(3-x^2\right)+2018$ đồng biến trên khoảng nào dưới đây?
		\choice
		{\True $(-1 ; 0)$}
		{$(2 ; 3)$}
		{$(-2 ;-1)$}
		{$(0 ; 1)$}
	}
	{
		\begin{tikzpicture}[scale=0.6,>=stealth, font=\footnotesize, line join=round, line cap=round]
			\def\a{0.065} \def\b{0.32} \def\c{-0.53} \def\d{-0.82} % Hệ số
			\def\xmin{-8} \def\xmax{4}
			\def\ymin{-3} \def\ymax{3} 
			%\draw[color=gray!50,dashed] (\xmin,\ymin) grid (\xmax,\ymax); 
			\draw[->] (\xmin,0)--(\xmax,0) node [below]{$x$};
			\draw[->] (0,\ymin)--(0,\ymax) node [left]{$y$};
			\node at (0,0) [below left]{$O$};
			\node at (-6,0) [below left]{$-6$};
			\node at (-1,0) [below left]{$-1$};
			\node at (2,0) [below right]{$2$};
			\clip (\xmin+0.1,\ymin+0.1) rectangle (\xmax-0.5,\ymax-0.1);
			\draw[smooth,samples=300][domain=-6.5:3.5] plot(\x,{\a*(\x)^3+\b*(\x)^2+\c*(\x)+\d});
		\end{tikzpicture}
	}
	
	\loigiai{
		Ta có $\left[f\left( 3-x^2\right)+2018 \right]'=-2 x \cdot f'\left(3-x^2\right) $.\\
		$
		-2 x \cdot f'\left(3-x^2\right)=0 \Leftrightarrow\hoac{&
			x = 0\\
			&3 - x ^{2}= - 6\\
			&3 - x ^{2}= - 1\\
			&3 - x ^{2}= 2}
		\Leftrightarrow \hoac{
			&x=0 \\
			&x=\pm 3 \\
			&x=\pm 2 \\
			&	x=\pm 1.}
		$\\
		Bảng xét dấu của đạo hàm hàm số đã cho
		\begin{center}
			\begin{center}
				\begin{tikzpicture}
					\tkzTabInit[nocadre,lgt=2.9,espcl=1.5,deltacl=0.6]
					{$x$ /0.7,$f'\left( 3-x^2\right) $/0.7,$-2xf'\left( 3-x^2\right)$/0.8}
					{$-\infty$,$-3$,$-2$,$-1$,$0$,$1$,$2$,$3$,$+\infty$}
					\tkzTabLine{,-,$0$,+,$0$,-,$0$,+,$0$,+,$0$,-,$0$,+,$0$,-}
					\tkzTabLine{,-,$0$,+,$0$,-,$0$,+,$0$,-,$0$,+,$0$,-,$0$,+}
				\end{tikzpicture}
			\end{center}
		\end{center}
		Từ bảng xét dấu suy ra hàm số đồng biến trên $(-1 ; 0)$.
	}
\end{ex}
\begin{ex}[Chuyên Biên Hòa - Hà Nam - 2020]%[2D1G1-2]
	\immini{
		Cho hàm số đa thức $f(x)$ có đạo hàm trên $\mathbb{R}$. Biết $f(0)=0$ và đồ thị hàm số $y=f'(x)$ như hình sau.
		Hàm số $g(x)=\left|4 f(x)+x^2\right|$ đồng biến trên khoảng nào dưới đây?
		\choice
		{$(4 ;+\infty)$}
		{\True $(0 ; 4)$}
		{$(-\infty ;-2)$}
		{$(-2 ; 0)$}
	}	
	{
		\begin{tikzpicture}[scale=0.7,>=stealth, font=\footnotesize, line join=round, line cap=round]
			%\def\a{1} \def\b{-6} \def\c{9} \def\d{1} % Hệ số
			\def\xmin{-4} \def\xmax{6}
			\def\ymin{-3} \def\ymax{2} 
			%\draw[color=gray!50,dashed] (\xmin,\ymin) grid (\xmax,\ymax); 
			\draw[->] (\xmin,0)--(\xmax,0) node [below]{$x$};
			\draw[->] (0,\ymin)--(0,\ymax) node [left]{$y$};
			\node at (0,0) [below left]{$O$};
			%\node at (1,3) [below left]{$f'(x)$};
			%\node at (-1.3,4) {$f'(x)$};
			\draw[dashed] (-2,0) node[below]{$-2$}--(-2,1)--(0,1) node[below left]{$1$};
			\draw[dashed] (4,0) node[below]{$4$}--(4,-2)--(0,-2) node[below left]{$-2$};
			%\draw[dashed] (1,0) node[below]{$1$}--(1,1);
			%\draw[dashed] (-0.5,0) node[below left]{$-0{,}5$}--(-0.5,2.125);
			\clip (\xmin+0.1,\ymin+0.1) rectangle (\xmax-0.5,\ymax-0.1);
			\draw[smooth,samples=300][domain=-4:5.5] plot(\x,{0.071*(\x)^3-0.142*(\x)^2-1.07*(\x)});
		\end{tikzpicture}
	}
	\loigiai{
		\immini{
			Xét hàm số $h(x)=4 f(x)+x^2$ trên $\mathbb{R}$.\\
			Vì $f(x)$ là hàm số đa thức nên $h(x)$ cũng là hàm số đa thức và $h(0)=4 f(0)=0$.\\
			Ta có $h'(x)=4 f'(x)+2 x$. Do đó $h'(x)=0 \Leftrightarrow f'(x)=-\dfrac{1}{2}x$.\\
		}
		{
			\begin{tikzpicture}[scale=0.7,>=stealth, font=\footnotesize, line join=round, line cap=round]
				%\def\a{1} \def\b{-6} \def\c{9} \def\d{1} % Hệ số
				\def\xmin{-4} \def\xmax{6}
				\def\ymin{-3} \def\ymax{2} 
				%\draw[color=gray!50,dashed] (\xmin,\ymin) grid (\xmax,\ymax); 
				\draw[->] (\xmin,0)--(\xmax,0) node [below]{$x$};
				\draw[->] (0,\ymin)--(0,\ymax) node [left]{$y$};
				\node at (0,0) [below left]{$O$};
				%\node at (1,3) [below left]{$f'(x)$};
				%\node at (-1.3,4) {$f'(x)$};
				\draw[dashed] (-2,0) node[below]{$-2$}--(-2,1)--(0,1) node[below left]{$1$};
				\draw[dashed] (4,0) node[below]{$4$}--(4,-2)--(0,-2) node[below left]{$-2$};
				%\draw[dashed] (1,0) node[below]{$1$}--(1,1);
				%\draw[dashed] (-0.5,0) node[below left]{$-0{,}5$}--(-0.5,2.125);
				\clip (\xmin+0.1,\ymin+0.1) rectangle (\xmax-0.5,\ymax-0.1);
				\draw[smooth,samples=300][domain=-4:5.5] plot(\x,{0.071*(\x)^3-0.142*(\x)^2-1.07*(\x)});
				\draw[smooth,samples=300][domain=-4:5.5] plot(\x,{-0.5*(\x)});
			\end{tikzpicture}
		}
		Dựa vào sự tương giao của đồ thị hàm số $y=f'(x)$ và đường thẳng $y=-\dfrac{1}{2}x$, ta có
		$
		h'(x)=0 \Leftrightarrow x \in\{-2 ; 0 ; 4\}.\\
		$
		Bảng biến thiên của hàm số $h(x)$ như sau:
		\begin{center}
			\begin{tikzpicture}
				\tkzTabInit[nocadre,lgt=1.2,espcl=2.5,deltacl=0.6]
				{$x$ /0.6,$y'$ /0.6,$y$ /2}
				{$-\infty$,$-2$,$0$,$4$,$+\infty$}
				\tkzTabLine{,-,$0$,+,$0$,-,$0$,+,}
				\tkzTabVar{+/$+\infty$, -/$y_1$,+/$0$,-/$y_3$,+/$+\infty$}
			\end{tikzpicture}
		\end{center}
		Từ đó suy ra bảng biến thiên của hàm số $g(x)=|h(x)|$.\\
		Dựa vào bảng biến thiên trên, ta thấy hàm số $g(x)$ đồng biến trên khoảng $(0 ; 4)$.
	}
\end{ex}
\begin{ex}[Chuyên Thái Bình - 2020]%[2D1G1-2]
	\immini{
		Cho hàm số $f(x)$ liên tục trên $\mathbb{R}$ có đồ thị hàm số $y=f'(x)$ cho như hình vẽ bên.\\
		Hàm số $g(x)=2 f(|x-1|)-x^2+2 x+2020$ đồng biến trên khoảng nào?
		\choice
		{\True $(0 ; 1)$}
		{$(-3 ; 1)$}
		{$(1 ; 3)$}
		{$(-2 ; 0)$}
	}
	{
		\begin{tikzpicture}[scale=0.7,>=stealth, font=\footnotesize, line join=round, line cap=round]
			%\def\a{1} \def\b{-6} \def\c{9} \def\d{1} % Hệ số
			\def\xmin{-4} \def\xmax{5}
			\def\ymin{-3} \def\ymax{5} 
			%\draw[color=gray!50,dashed] (\xmin,\ymin) grid (\xmax,\ymax); 
			\draw[->] (\xmin,0)--(\xmax,0) node [below]{$x$};
			\draw[->] (0,\ymin)--(0,\ymax) node [left]{$y$};
			\node at (0,0) [below left]{$O$};
			%\node at (1,3) [below left]{$f'(x)$};
			\node at (-1.3,4) {$f'(x)$};
			\draw[dashed] (-1,0) node[above]{$-1$}--(-1,-1)--(0,-1) node[below left]{$-1$};
			\draw[dashed] (1,0) node[below]{$1$}--(1,1)--(0,1) node[below left]{$1$};
			\draw[dashed] (3,0) node[below]{$3$}--(3,3)--(0,3) node[below left]{$3$};
			%\draw[dashed] (1,0) node[below]{$1$}--(1,1);
			%\draw[dashed] (-0.5,0) node[below left]{$-0{,}5$}--(-0.5,2.125);
			\clip (\xmin+0.1,\ymin+0.1) rectangle (\xmax-0.5,\ymax-0.1);
			\draw[smooth,samples=300][domain=-2:4] plot(\x,{-0.5*(\x)^3+1.5*(\x)^2+1.5*(\x)-1.5});
			%\draw[smooth,samples=300] plot(\x,{(\x)^3+(\x)^2-2*(\x)+1});
		\end{tikzpicture}
	}
	\loigiai{
		Ta có đường thẳng $y=x$ cắt đồ thị hàm số $y=f'(x)$ tại các điểm $x=-1 ; x=1 ; x=3$ như hình vẽ sau:
		\begin{center}
			\begin{tikzpicture}[scale=0.7,>=stealth, font=\footnotesize, line join=round, line cap=round]
				%\def\a{1} \def\b{-6} \def\c{9} \def\d{1} % Hệ số
				\def\xmin{-4} \def\xmax{5}
				\def\ymin{-3} \def\ymax{5} 
				%\draw[color=gray!50,dashed] (\xmin,\ymin) grid (\xmax,\ymax); 
				\draw[->] (\xmin,0)--(\xmax,0) node [below]{$x$};
				\draw[->] (0,\ymin)--(0,\ymax) node [left]{$y$};
				\node at (0,0) [below left]{$O$};
				%\node at (1,3) [below left]{$f'(x)$};
				\node at (-1.3,4) {$f'(x)$};
				\node at (4,3.2) {$y=x$};
				\draw[dashed] (-1,0) node[above]{$-1$}--(-1,-1)--(0,-1) node[below left]{$-1$};
				\draw[dashed] (1,0) node[below]{$1$}--(1,1)--(0,1) node[below left]{$1$};
				\draw[dashed] (3,0) node[below]{$3$}--(3,3)--(0,3) node[below left]{$3$};
				%\draw[dashed] (1,0) node[below]{$1$}--(1,1);
				%\draw[dashed] (-0.5,0) node[below left]{$-0{,}5$}--(-0.5,2.125);
				\clip (\xmin+0.1,\ymin+0.1) rectangle (\xmax-0.5,\ymax-0.1);
				\draw[smooth,samples=300][domain=-2:4] plot(\x,{-0.5*(\x)^3+1.5*(\x)^2+1.5*(\x)-1.5});
				\draw[smooth,samples=300] plot(\x,{(\x)});
			\end{tikzpicture}
		\end{center}
		Dựa vào đồ thị của hai hàm số trên ta có $f'(x)>x \Leftrightarrow\hoac{&x<-1 \\ &1<x<3}$ và
		$ f'(x)<x \Leftrightarrow\hoac{&
			-1<x<1 \\
			&x>3.}$\\
		+Trường hợp 1: $x-1<0 \Leftrightarrow x<1$.\\
		Khi đó $g(x)=2 f(1-x)-x^2+2 x+2020$.\\
		Ta có $g'(x)=-2 f'(1-x)+2(1-x)$.
		$$
		g'(x)>0 \Leftrightarrow-2 f'(1-x)+2(1-x)>0 \Leftrightarrow f'(1-x)<1-x \Leftrightarrow\hoac{
			&- 1 < 1 - x < 1\\
			&1 - x > 3} \Leftrightarrow \hoac{&
			0<x<2 \\
			&x<-2.}
		$$
		Kết hợp điều kiện, ta có $g'(x)>0 \Leftrightarrow\hoac{&0<x<1 \\ &x<-2.}$\\
		
		+ Trường hợp 2: $x-1>0 \Leftrightarrow x>1$.\\
		Khi đó ta có $g(x)=2 f(x-1)-x^2+2 x+2020$.\\
		$ g'(x)=2 f'(x-1)-2(x-1)$\\
		$g'(x)>0 \Leftrightarrow 2 f'(x-1)-2(x-1)>0 \Leftrightarrow f'(x-1)>x-1 \Leftrightarrow\hoac{&
			x - 1 < - 1\\
			&1 < x - 1 < 3}\Leftrightarrow \hoac{
			&x<0 \\
			&2<x<4.}$
		Kết hợp điều kiện ta có $g'(x)>0 \Leftrightarrow 2<x<4$.\\
		Vậy hàm số $g(x)=2 f(|x-1|)-x^2+2 x+2020$ đồng biến trên khoảng $(0 ; 1)$.
	}
\end{ex}

\begin{ex}[Chuyên Lào Cai - 2020]%[2D1G1-2]
	\immini{
		Cho hàm số $f'(x)$ có đồ thị như hình bên.\\
		Hàm số $g(x)=f(3 x+1)+9 x^3+\dfrac{9}{2}x^2$ đồng biến trên khoảng nào dưới đây?
		\choice
		{$(-1 ; 1)$}
		{$(-2 ; 0)$}
		{$(-\infty ; 0)$}
		{\True $(1 ;+\infty)$}
	}
	{\begin{tikzpicture}[line join=round, line cap=round,>=stealth,thick,scale=.8]
			\tikzset{label style/.style={font=\footnotesize}}
			\draw[->] (-2.1,0)--(5.1,0) node[below left] {$x$};
			\draw[->] (0,-3.1)--(0,4.1) node[below left] {$y$};
			\draw (0,0) node [below left] {$O$};
			\foreach \x in {1,2,3}
			\draw[thin] (\x,1pt)--(\x,-1pt) node [below] {$\x$};
			\draw[thin](-1,1pt)--(1,-1pt)node[above left]{$-1$};
			\foreach \y in {-2,2}
			\draw[thin] (1pt,\y)--(-1pt,\y) node [above right] {$\y$};
			%\begin{scope}
			\clip (-2,-3) rectangle (5,4);
			\draw[samples=200,domain=-2:4,smooth,variable=\x] plot (\x,{(\x)^3-3*(\x)^2+2});
			%\end{scope}
			\draw[dashed] (-1,0)--(-1,-2)--(0,-2);
			\draw[dashed] (3,0)--(3,2)--(0,2);
			%\begin{scope}[on background layer]\path[white]node{MDD-134};\end{scope}
		\end{tikzpicture}
	}
	\loigiai
	{
		\immini{Xét hàm số $g(x)=f(3 x+1)+9 x^3+\dfrac{9}{2}x^2 \\
			\Rightarrow g'(x)=3 f'(3 x+1)+27 x^2+9 x$.\\
			Hàm số đồng biến  $\Leftrightarrow g'(x)>0 \Leftrightarrow 3 f'(3 x+1)+27 x^2+9 x>0$
			\\
			$
			\Leftrightarrow f'(3 x+1)+3 x(3 x+1)>0 \qquad (*)
			$\\
			Đặt $t=3 x+1$, khi đó  $(*) \Leftrightarrow f'(t)+(t-1) t>0$\\ $\Leftrightarrow f'(t)>-t^2+t$.\\
			Vẽ parabol $y=-x^2+x$ và đồ thị hàm số $f'(x)$ trên cùng một hệ trục
		}
		{
			\begin{tikzpicture}[line join=round, line cap=round,>=stealth,thick,scale=.8]
				\tikzset{label style/.style={font=\footnotesize}}
				\draw[->] (-2.1,0)--(5.1,0) node[below left] {$x$};
				\draw[->] (0,-3.1)--(0,4.1) node[below left] {$y$};
				\draw (0,0) node [below left] {$O$};
				\foreach \x in {1,2,3}
				\draw[thin] (\x,1pt)--(\x,-1pt) node [below] {$\x$};
				\draw[thin](-1,1pt)--(1,-1pt);
				\foreach \y in {-2,2}
				\draw[thin] (1pt,\y)--(-1pt,\y) node [above right] {$\y$};
				%\begin{scope}
				\clip (-2,-3) rectangle (5,4);
				\draw[samples=200,domain=-2:4,smooth,variable=\x] plot (\x,{(\x)^3-3*(\x)^2+2});
				\draw[samples=200,domain=-2:4,smooth,variable=\x] plot (\x,{-(\x)^2+(\x)});
				%\end{scope}
				\draw[dashed] (-1,0) node[above left]{$-1$}--(-1,-2)--(0,-2);
				\draw[dashed] (3,0)--(3,2)--(0,2);
				%\begin{scope}[on background layer]\path[white]node{MDD-134};\end{scope}
			\end{tikzpicture}
		}
		Dựa vào đồ thị ta thấy
		$
		f'(t)>-t^2+t \Leftrightarrow\hoac{&- 1 < t < 1\\
			&t > 2}\Rightarrow \hoac{&
			- 1 < 3 x + 1 < 1\\
			&3 x + 1 > 2} \Leftrightarrow \hoac{&
			\dfrac{-2}{3}<x<0\\
			&x>\dfrac{1}{3}.}
		$}
\end{ex}
\begin{ex}[Sở Phú Thọ-2020]%[2D1G1-2]
	\immini{
		Cho hàm số $y=f(x)$ có đồ thị hàm số $y=f'(x)$ như hình vẽ.\\
		Hàm số $g(x)=f\left(\mathrm{e}^x-2\right)-2020$ nghịch biến trên khoảng nào dưới đây?
		\choice
		{\True $\left(-1 ; \dfrac{3}{2}\right)$}
		{$(-1 ; 2)$}
		{$(0 ;+\infty)$}
		{$\left(\dfrac{3}{2}; 2\right)$}
	}
	{
		\begin{tikzpicture}[scale=0.7,>=stealth, font=\footnotesize, line join=round, line cap=round]
			\def\a{1} \def\b{-3} \def\c{0} \def\d{0} % Hệ số
			\def\xmin{-2} \def\xmax{4}
			\def\ymin{-5} \def\ymax{2} 
			%\draw[color=gray!50,dashed] (\xmin,\ymin) grid (\xmax,\ymax); 
			\draw[->] (\xmin,0)--(\xmax,0) node [below]{$x$};
			\draw[->] (0,\ymin)--(0,\ymax) node [left]{$y$};
			\node at (0,0) [above left]{$O$};
			\node at (3,0) [below right]{$3$};
			\draw[dashed] (2,0) node[above]{$2$}--(2,-4) --(0,-4) node[left]{$-4$};
			\clip (\xmin+0.1,\ymin+0.1) rectangle (\xmax-0.5,\ymax-0.1);
			\draw[smooth,samples=300] plot(\x,{\a*(\x)^3+\b*(\x)^2+\c*(\x)+\d});
		\end{tikzpicture}
	}
	
	\loigiai{
		Dựa vào đồ thị hàm số $y=f'(x)$ suy ra $f'(x) \leq 0 ~ \forall x<3$ và $f'(x)>0 ~ \forall x>3$.
		$
		g'(x)=\mathrm{e}^x f'\left(\mathrm{e}^x-2\right) .
		$
		Hàm số $g(x)=f\left(\mathrm{e}^x-2\right)-2020$ nghịch biến \\ $ \Leftrightarrow g'(x)<0 \Leftrightarrow \mathrm{e}^x f'\left(\mathrm{e}^x-2\right)<0$\\
		$
		\Leftrightarrow f'\left(\mathrm{e}^x-2\right)<0 \Leftrightarrow \mathrm{e}^x-2<3 \Leftrightarrow \mathrm{e}^x<5 \Leftrightarrow x<\ln 5 .
		$\\
		Vậy hàm số đã cho nghịch biến trên $\left(-1 ; \dfrac{3}{2}\right)$.
	}
\end{ex}
\begin{ex}[Lý Nhân Tông - Bắc Ninh - 2020]%[2D1G1-2]
	\immini{
		Cho hàm số $f(x)$ có đồ thị hàm số $f'(x)$ như hình vẽ.\\
		Hàm số $y=f(\cos x)+x^2-x$ đồng biến trên khoảng
		\choice
		{$(-2 ; 1)$}
		{$(0 ; 1)$}
		{\True $(1 ; 2)$}
		{$(-1 ; 0)$}
	}
	{
		\begin{tikzpicture}[scale=1,>=stealth, font=\footnotesize, line join=round, line cap=round]
			\def\a{-0.5} \def\b{0} \def\c{1.5} \def\d{0} % Hệ số
			\def\xmin{-3} \def\xmax{4}
			\def\ymin{-2} \def\ymax{2} 
			%\draw[color=gray!50,dashed] (\xmin,\ymin) grid (\xmax,\ymax); 
			\draw[->] (\xmin,0)--(\xmax,0) node [below]{$x$};
			\draw[->] (0,\ymin)--(0,\ymax) node [left]{$y$};
			\node at (0,0) [above left]{$O$};
			\node at (3,0) [below right]{$3$};
			\draw[dashed] (-2,0) node[below]{$-2$}--(-2,1) --(0,1) node[above right]{$1$} --(1,1)--(1,0) node[below]{$1$};
			\draw[dashed] (-1,0) node[below right]{$-1$}--(-1,-1) --(0,-1) node[above right]{$-1$} --(2,-1)--(2,0) node[below right]{$2$};
			\clip (\xmin+0.1,\ymin+0.1) rectangle (\xmax-0.5,\ymax-0.1);
			\draw[smooth,samples=300][domain=-2:2] plot(\x,{\a*(\x)^3+\b*(\x)^2+\c*(\x)+\d});
		\end{tikzpicture}
	}
	\loigiai{
		Đặt  $g(x)=f(\cos x)+x^2-x$.\\
		Ta có $g'(x)=-\sin x \cdot f'(\cos x)+2 x-1$\\
		Vì $\cos x \in[-1 ; 1]$ nên từ đồ thị $f'(x)$ ta suy ra $f'(\cos x) \in[-1 ; 1]$.\\
		Do đó $\left|-\sin x \cdot f'(\cos x)\right| \leq 1, ~\forall x \in \mathbb{R}$.\\
		Ta suy ra $g'(x)=\sin x \cdot f'(\cos x)+2 x-1 \geq-1+2 x-1=2 x-2$
		$\Rightarrow g'(x)>0, ~\forall x>1$.\\
		Vậy hàm số đồng biến trên $(1 ; 2)$.
	}
\end{ex}
\begin{ex}[THPT Nguyễn Viết Xuân - 2020]%[2D1G1-2]
	\immini{
		Cho hàm số $f(x)$. Hàm số $y=f'(x)$ có đồ thị như hình vẽ.\\
		Hàm số $g(x)=f\left(3 x^2-1\right)-\dfrac{9}{2}x^4+3 x^2$ đồng biến trên khoảng nào dưới đây?
		\choice
		{\True $\left(-\dfrac{2 \sqrt{3}}{3}; \dfrac{-\sqrt{3}}{3}\right)$}
		{$\left(0 ; \dfrac{2 \sqrt{3}}{3}\right)$}
		{$(1 ; 2)$}
		{$\left(-\dfrac{\sqrt{3}}{3}; \dfrac{\sqrt{3}}{3}\right)$} 
	}
	{
		\begin{tikzpicture}[scale=0.6,>=stealth, font=\footnotesize, line join=round, line cap=round]
			\def\a{0.25} \def\b{0.25} \def\c{-2} \def\d{0} % Hệ số
			\def\xmin{-5} \def\xmax{4}
			\def\ymin{-5} \def\ymax{5} 
			%\draw[color=gray!50,dashed] (\xmin,\ymin) grid (\xmax,\ymax); 
			\draw[->] (\xmin,0)--(\xmax,0) node [below]{$x$};
			\draw[->] (0,\ymin)--(0,\ymax) node [left]{$y$};
			\node at (0,0) [above left]{$O$};
			%\node at (3,0) [below right]{$3$};
			\draw[dashed] (-4,0) node[below left]{$-4$}--(-4,-4) --(0,-4) node[above right]{$-4$};
			\draw[dashed] (3,0) node[below right]{$3$}--(3,3) --(0,3) node[above right]{$3$};
			\clip (\xmin+0.1,\ymin+0.1) rectangle (\xmax-0.5,\ymax-0.1);
			\draw[smooth,samples=300] plot(\x,{\a*(\x)^3+\b*(\x)^2+\c*(\x)+\d});
		\end{tikzpicture}
	}
	
	\loigiai
	{
		TXĐ: $\mathscr{D}=\mathbb{R}$.\\
		Ta có $g'(x)=6 x f'\left(3 x^2-1\right)-18 x^3+6 x=6 x\left[f'\left(3 x^2-1\right)-3 x^2+1\right]$.\\
		$
		g'(x)=0 \Leftrightarrow\hoac{
			&x = 0\\
			&f '( 3 x ^{2}- 1 ) = 3 x ^{2}- 1}
		\Leftrightarrow \hoac{
			&x = 0\\
			&3 x ^{2}- 1 = - 4 \text{~(vô nghiệm)}\\
			&3 x ^{2}- 1 = 0\\
			&3 x ^{2}- 1 = 3}\Leftrightarrow \hoac{&x=0 \\
			&x=\pm \dfrac{\sqrt{3}}{3}\\
			&x=\pm \dfrac{2 \sqrt{3}}{3}.}
		$\\
		Bảng xét dấu
		\begin{center}
			\begin{tikzpicture}
				\tkzTabInit[nocadre,lgt=1.2,espcl=2.2,deltacl=0.6]
				{$x$ /1.2,$f'(x)$ /0.7}
				{$-\infty$,$-\dfrac{2 \sqrt{3}}{3}$,$-\dfrac{ \sqrt{3}}{3}$,$0$,$\dfrac{\sqrt{3}}{3}$,$\dfrac{2 \sqrt{3}}{3}$,$+\infty$}
				\tkzTabLine{,-,$0$,+,$0$,-,$0$,+,$0$,-,$0$,+,}
			\end{tikzpicture}
		\end{center}
		Vậy hàm số đồng biến trong khoảng $\left(-\dfrac{2 \sqrt{3}}{3}; \dfrac{-\sqrt{3}}{3}\right)$.}
\end{ex}
\begin{ex}[Trần Phú - Quảng Ninh - 2020]%[2D1G1-2]
	Cho hàm số $f(x)$ có bảng xét dấu của đạo hàm như sau
	\begin{center}
		\begin{tikzpicture}
			\tkzTabInit[nocadre,lgt=1.2,espcl=2,deltacl=0.6]
			{$x$ /0.6,$f'(x)$ /0.6}
			{$-\infty$,$-4$,$-1$,$2$,$7$,$+\infty$}
			\tkzTabLine{,+,$0$,-,$0$,+,$0$,-,$0$,+,}
		\end{tikzpicture}
	\end{center}
	Hàm số $y=f(2 x+1)+\dfrac{2}{3}x^3-8 x+5$ nghịch biến trên khoảng nào dưới đây?
	\choice
	{$(-\infty ;-2)$}
	{$(1 ;+\infty)$}
	{$(-1 ; 7)$}
	{\True $\left(-1 ; \dfrac{1}{2}\right)$}
	\loigiai{
		Ta có $y'=2 f'(2 x+1)+2 x^2-8$.\\
		Xét $y'\leq 0 \Leftrightarrow 2 f'(2 x+1)+2 x^2-8 \leq 0 \Leftrightarrow f'(2 x+1) \leq 4-x^2$.\\
		Đặt $t=2x+1$, ta có $f'(t) \leq \dfrac{-t^2+2 t+15}{4}$.\\
		Vì $\dfrac{-t^2+2 t+15}{4}\geq 0, \forall t \in[-3 ; 5]$.\\
		Mà $f'(t) \leq 0, \forall t \in[-3 ; 2]$.\\
		Nên $f'(t) \leq \dfrac{-t^2+2 t+15}{4}\Rightarrow t \in[-3 ; 2]$.\\
		Suy ra $-3 \leq 2 x+1 \leq 2 \Leftrightarrow-2 \leq x \leq \dfrac{1}{2}$.}
\end{ex}

\begin{ex}[Chuyên Thái Bình - Lần 3 - 2020]%[2D1G1-2]
	\immini{
		Cho hàm số $y=f(x)$ liên tục trên $\mathbb{R}$ có đồ thị hàm số $y=f'(x)$ cho như hình vẽ.\\
		Hàm số $g(x)=2 f(|x-1|)-x^2+2 x+2020$ đồng biến trên khoảng nào?
		\choice
		{\True $(0 ; 1)$}
		{$(-3 ; 1)$}
		{$(1 ; 3)$}
		{$(-2 ; 0)$}
	}
	{
		\begin{tikzpicture}[scale=0.7,>=stealth, font=\footnotesize, line join=round, line cap=round]
			\def\a{-0.333} \def\b{1} \def\c{1.333} \def\d{-1} % Hệ số
			\def\xmin{-3} \def\xmax{5}
			\def\ymin{-3} \def\ymax{5} 
			%\draw[color=gray!50,dashed] (\xmin,\ymin) grid (\xmax,\ymax); 
			\draw[->] (\xmin,0)--(\xmax,0) node [below]{$x$};
			\draw[->] (0,\ymin)--(0,\ymax) node [left]{$y$};
			\node at (0,0) [above left]{$O$};
			%\node at (3,0) [below right]{$3$};
			\draw[dashed] (-1,0) node[above]{$-1$}--(-1,-1) --(0,-1) node[above right]{$-1$};
			\draw[dashed] (1,0) node[below right]{$1$}--(1,1) --(0,1) node[above right]{$1$};
			\draw[dashed] (3,0) node[below right]{$3$}--(3,3) --(0,3) node[above right]{$3$};
			\clip (\xmin+0.1,\ymin+0.1) rectangle (\xmax-0.5,\ymax-0.1);
			\draw[smooth,samples=300] plot(\x,{\a*(\x)^3+\b*(\x)^2+\c*(\x)+\d});
			\draw[smooth,samples=300] plot(\x,{(\x)});
		\end{tikzpicture}
	}
	\loigiai{
		Với $x>1$, ta có $g(x)=2 f(x-1)-(x-1)^2+2021 \Rightarrow g'(x)=2 f'(x-1)-2(x-1)$.\\
		Hàm số đồng biến $\Leftrightarrow 2 f'(x-1)-2(x-1)>0 \Leftrightarrow f'(x-1)>x-1 \quad(*)$.\\
		Đặt $t=x-1$, khi đó $(*) \Leftrightarrow f'(t)>t \Leftrightarrow\hoac{&1<t<3 \\ &t<-1}\Rightarrow\hoac{&2<x<4 \\ &x<0 ~(\text{loại}).}$\\
		Với $x<1$, ta có $g(x)=2 f(1-x)-(1-x)^2+2021 \Rightarrow g'(x)=-2 f'(1-x)+2(1-x)$.\\
		Hàm số đồng biến $\Leftrightarrow-2 f'(1-x)+2(1-x)>0 \Leftrightarrow f'(1-x)<1-x \quad(* *)$.\\
		Đặt $t=1-x$, khi đó $(* *) \Leftrightarrow f'(t)<t \Leftrightarrow\hoac{&-1<t<1 \\ &t>3}\Rightarrow\hoac{&0<x<2 \\ &x<-2}\Rightarrow\hoac{&0<x<1 \\ &x<-2.}$\\
		Vậy hàm số $g(x)$ đồng biến trên các khoảng $(-\infty ;-2),(0 ; 1),(2 ; 4)$.
	}
\end{ex}
\begin{ex}[Sở Phú Thọ - 2020]%[2D1G1-2]
	\immini{
		Cho hàm số $y=f(x)$ có đồ thị hàm số $f'(x)$ như hình vẽ.\\
		Hàm số $g(x)=f\left(1+e^x\right)+2020$ nghịch biến trên khoảng nào dưới đây?
		\choice
		{$(0 ;+\infty)$}
		{$\left(\dfrac{1}{2}; 1\right)$}
		{\True $\left(0 ; \dfrac{1}{2}\right)$}
		{$(-1 ; 1)$}
	}{
		\begin{tikzpicture}[scale=0.7,>=stealth, font=\footnotesize, line join=round, line cap=round]
			\def\a{1} \def\b{-3} \def\c{0} \def\d{0} % Hệ số
			\def\xmin{-2} \def\xmax{4}
			\def\ymin{-5} \def\ymax{2} 
			%\draw[color=gray!50,dashed] (\xmin,\ymin) grid (\xmax,\ymax); 
			\draw[->] (\xmin,0)--(\xmax,0) node [below]{$x$};
			\draw[->] (0,\ymin)--(0,\ymax) node [left]{$y$};
			\node at (0,0) [above left]{$O$};
			\node at (3,0) [below right]{$3$};
			\draw[dashed] (2,0) node[above]{$2$}--(2,-4) --(0,-4) node[left]{$-4$};
			\clip (\xmin+0.1,\ymin+0.1) rectangle (\xmax-0.5,\ymax-0.1);
			\draw[smooth,samples=300] plot(\x,{\a*(\x)^3+\b*(\x)^2+\c*(\x)+\d});
		\end{tikzpicture}
	}
	\loigiai{
		$g'(x)=e^x f'\left(1+e^x\right)$.\\
		Do $e^x>0, \forall x$ nên $g'(x) \leq 0 \Leftrightarrow f'\left(1+e^x\right) \leq 0 \Leftrightarrow 1+e^x \leq 3 \Leftrightarrow x \leq \ln 2$, dấu bằng xảy ra tại hữu hạn điểm.\\
		Nên $g(x)$ nghịch biến trên $(-\infty ; \ln 2)$.\\
		Vì $\left(0 ; \dfrac{1}{2}\right) \subset (-\infty ; \ln 2)$ nên hàm số đã cho nghịch biến trên $\left(0 ; \dfrac{1}{2}\right)$.
	}
\end{ex}

\begin{ex}%[2D1K1-2]
	[THPT Anh Sơn - Nghệ An - 2020]
	Cho hàm số $y=f(x)$ có bảng xét dấu của đạo hàm như sau.
	\begin{center}
		\begin{tikzpicture}
			\tkzTabInit[nocadre,lgt=1.2,espcl=2,deltacl=0.6]
			{$x$ /0.6,$f'(x)$ /0.6}
			{$-\infty$,$-2$,$-1$,$2$,$4$,$+\infty$}
			\tkzTabLine{,+,$0$,-,$0$,+,$0$,-,$0$,+,}
		\end{tikzpicture}
	\end{center}
	Hàm số $y=-2 f(x)+2019$ nghịch biến trên khoảng nào trong các khoảng dưới đây?
	\choice
	{$(2 ; 4)$}
	{$(-4 ; 2)$}
	{$(-2 ;-1)$}
	{\True $(-1 ; 2)$}
	\loigiai{
		Ta có $y'=-2 f'(x)$.\\
		$
		y'=0 \Leftrightarrow-2 f'(x)=0 \Leftrightarrow\hoac{&
			x=-2 \\
			&x=-1 \\
			&x=2 \\
			&x=4.}$\\
		Từ bảng xét dấu của $f'(x)$ ta có
		\begin{center}
			\begin{tikzpicture}
				\tkzTabInit[nocadre,lgt=1,espcl=2,deltacl=0.6]
				{$x$ /0.6,$y'$ /0.6}
				{$-\infty$,$-2$,$-1$,$2$,$4$,$+\infty$}
				\tkzTabLine{,-,$0$,+,$0$,-,$0$,+,$0$,-,}
			\end{tikzpicture}
		\end{center}
		Từ bảng xét dấu ta có hàm số nghịch biến trên khoảng $(-\infty ;-2),(-1 ; 2)$ và $(4 ;+\infty)$.}
\end{ex}

\begin{ex}[THPT Anh Sơn - Nghệ An - 2020]%[2D1G1-2]
	Cho hàm số $f(x)$ xác định và liên tục trên $\mathbb{R}$ và có đạo hàm $f'(x)$ thỏa mãn $f'(x)=(1-x)(x+2) g(x)+2019$ với $g(x)<0, ~\forall x \in \mathbb{R}$ . Hàm số $y=f(1-x)+2019 x+2020$ nghịch biến trên khoảng nào?
	\choice
	{$(1 ;+\infty)$}
	{$(0 ; 3)$}
	{$(-\infty ; 3)$}
	{\True $(3 ;+\infty)$}
	\loigiai{
		Đặt $h(x)=f(1-x)+2019 x+2020$.\\
		Vì hàm số $f(x)$ xác định trên $\mathbb{R}$ nên hàm số $h(x)$ cũng xác định trên $\mathbb{R}$.\\
		Ta có $h'(x)=-f'(1-x)+2019$.\\
		Do $h'(x)=0$ tại hữu hạn điểm nên để tìm khoảng nghịch biến của hàm số $h(x)$, ta tìm các giá trị của $x$ sao cho $h'(x)<0 \Leftrightarrow-f'(1-x)+2019<0$\\ 
		$\Leftrightarrow f'(1-x)-2019>0 \\
		\Leftrightarrow x(3-x) g(1-x)>0 \Leftrightarrow x(3-x)<0(\text{~do~}g(x)<0, \forall x \in \mathbb{R})$\\
		$\Leftrightarrow\hoac{&
			x<0 \\
			&x>3.}$\\
		Vậy hàm số $y=f(1-x)+2019 x+2020$ nghịch biến trên các khoảng $(-\infty ; 0)$ và $(3 ;+\infty).$}
\end{ex}

\begin{ex}%[2D1G1-2]
	Cho hàm số $y=f(x)$ xác định trên $\mathbb{R}$ và có bảng xét dấu đạo hàm như sau:
	\begin{center}
		\begin{tikzpicture}
			\tkzTabInit[nocadre,lgt=2,espcl=2,deltacl=0.6]
			{$x$ /0.6,$f'(x)$ /0.6}
			{$-\infty$,$-1$,$1$,$4$,$+\infty$}
			\tkzTabLine{,-,$0$,+,$0$,-,$0$,+,}
		\end{tikzpicture}
	\end{center}
	Biết $f(x)>2,~ \forall x \in \mathbb{R}$. Xét hàm số $g(x)=f(3-2 f(x))-x^3+3 x^2-2020$. Khẳng định nào sau đây đúng?
	\choice
	{Hàm số $g(x)$ đồng biến trên khoảng $(-2 ;-1)$}
	{Hàm số $g(x)$ nghịch biến trên khoảng $(0 ; 1)$}
	{Hàm số $g(x)$ đồng biến trên khoảng $(3 ; 4)$}
	{\True Hàm số $g(x)$ nghịch biến trên khoảng $(2 ; 3)$}
	\loigiai{
		Ta có $g'(x)=-2 f'(x) f'(3-2 f(x))-3 x^2+6 x$.\\
		Vì $f(x)>2, ~\forall x \in \mathbb{R}$ nên $3-2 f(x)<-1 ~\forall x \in \mathbb{R}$.\\
		Từ bảng xét dấu $f'(x)$ suy ra $f'(3-2 f(x))<0, ~\forall x \in \mathbb{R}$.\\
		Từ đó ta có bảng xét dấu sau:
		\begin{center}
			\begin{tikzpicture}
				\tkzTabInit[nocadre,lgt=4,espcl=1.7,deltacl=0.6]
				{$x$ /0.7,$-f'(x)f'\left( 3-2f(x)\right) $/0.8,$-3x^2+6x$/0.7}
				{$-\infty$,$-1$,$0$,$1$,$2$,$4$,$+\infty$}
				\tkzTabLine{,-,$0$,+,|,+,$0$,-,|,-,$0$,+,}
				\tkzTabLine{,-,|,-,$0$,+,|,+,$0$,-,|,-,}
			\end{tikzpicture}
		\end{center}
		Từ bảng xét dấu trên, loại trừ đáp án suy ra hàm số $g(x)$ nghịch biến trên khoảng $(2 ; 3)$.}
\end{ex}

\begin{ex}%[2D1G1-2]
	Cho hàm số $f(x)$ có bảng biến thiên như sau:
	\begin{center}
		\begin{tikzpicture}
			\tkzTabInit[nocadre,lgt=1.2,espcl=2.5,deltacl=0.6]
			{$x$ /0.7, $f'(x)$ /0.7, $f(x)$ /2.5}
			{$-\infty$,$1$,$2$,$3$,$4$,$+\infty$}
			\tkzTabLine{,+,$0$,-,$0$,+,$0$,-,$0$,+,}
			\tkzTabVar{-/$-\infty$,+/$3$,-/$1$,+/$2$,-/$0$,+/$+\infty$}
		\end{tikzpicture}
	\end{center}
	Hàm số $y=(f(x))^3-3 .(f(x))^2$ nghịch biến trên khoảng nào dưới đây?
	\choice
	{$(1 ; 2)$}
	{$(3 ; 4)$}
	{$(-\infty ; 1)$}
	{\True $(2 ; 3)$}
	\loigiai{
		Ta có $y'=3 \cdot(f(x))^2 \cdot f'(x)-6 \cdot f(x) \cdot f'(x)=3 f(x) \cdot f'(x) \cdot[f(x)-2]. \\
		y'=0 \Leftrightarrow \hoac{&f(x)=0 \Leftrightarrow x \in\left\{x_1, 4 \mid x_1<1\right\}\\
			&f(x)=2 \Leftrightarrow x \in\left\{x_2, x_3, 3, x_4 \mid x_1<x_2<1<x_3<2 ; 4<x_4\right\}\\
			&f'(x)=0 \Leftrightarrow x \in\{1,2,3,4\}.}$\\
		Lập bảng xét dấu ta có
		\begin{center}
			\begin{tikzpicture}
				\tkzTabInit[nocadre,lgt=2,espcl=1.5,deltacl=0.6]
				{$x$ /0.7,$f(x)$ /0.7,$f(x)-2$ /0.7,$f'(x)$/0.7,$y'$/0.7}
				{$-\infty$,$x_1$,$x_2$,$1$,$x_3$,$2$,$3$,$4$,$x_4$,$+\infty$}
				\tkzTabLine{,-,$0$,+,|,+,|,+,|,+,|,+,$0$,+,|,+,|,+,}
				\tkzTabLine{,-,|,-,$0$,+,$0$,+,$0$,-,|,-,$0$,-,|,-,$0$,+}
				\tkzTabLine{,+,|,+,|,+,$0$,-,|,-,$0$,+,$0$,-,$0$,+,|,+}
				\tkzTabLine{,+,$0$,-,$0$,+,$0$,-,$0$,+,$0$,-,$0$,+,$0$,-,$0$,+}
			\end{tikzpicture}
		\end{center}
		
		Do đó hàm số nghịch biến trên khoảng $(2 ; 3)$.
	}
\end{ex}
\begin{ex}%[2D1G1-2]
	Cho hàm số $y=f(x)$ có đồ thị nằm trên trục hoành và có đạo hàm trên $\mathbb{R}$, bảng xét dấu của biểu thức $f'(x)$ như bảng dưới đây.
	\begin{center}
		\begin{tikzpicture}
			\tkzTabInit[nocadre,lgt=1.2,espcl=2,deltacl=0.6]
			{$x$ /0.6,$f'(x)$ /0.6}
			{$-\infty$,$-2$,$-1$,$3$,$+\infty$}
			\tkzTabLine{,-,$0$,+,$0$,-,$0$,+,}
		\end{tikzpicture}
	\end{center}
	Hàm số $y=g(x)=\dfrac{f\left(x^2-2 x\right)}{f\left(x^2-2 x\right)+1}$ nghịch biến trên khoảng nào dưới đây?
	\choice
	{$(-\infty ; 1)$}
	{$\left(-2 ; \dfrac{5}{2}\right)$}
	{\True $(1 ; 3)$}
	{$(2 ;+\infty)$}
	\loigiai{
		$ g'(x)=\dfrac{\left(x^2-2 x\right)'\cdot f'\left(x^2-2 x\right)}{\left(f\left(x^2-2 x\right)+1\right)^2}=\dfrac{(2 x-2) \cdot f'\left(x^2-2 x\right)}{\left(f\left(x^2-2 x\right)+1\right)^2}. \\
		g'(x)=0 \Leftrightarrow\hoac{
			&2 x - 2 = 0\\
			&f '( x ^{2}- 2 x ) = 0}
		\Leftrightarrow \hoac{&x = 1\\
			&x ^{2}- 2 x = - 2\\
			&x ^{2}- 2 x = - 1\\
			&x ^{2}- 2 x = 3}
		\Leftrightarrow \hoac{&x=1 \\
			&x=-1 \\
			&x=3.}
		$\\
		Ta có bảng xét dấu của $g'(x)$
		\begin{center}
			\begin{tikzpicture}
				\tkzTabInit[nocadre,lgt=1.2,espcl=2,deltacl=0.6]
				{$x$ /0.6,$g'(x)$ /0.6}
				{$-\infty$,$-1$,$1$,$3$,$+\infty$}
				\tkzTabLine{,-,$0$,+,$0$,-,$0$,+,}
			\end{tikzpicture}
		\end{center}
		Dựa vào bảng xét dấu ta có hàm số $y=g(x)$ nghịch biến trên các khoảng $(-\infty ;-1)$ và $(1 ; 3)$.}
\end{ex}
\begin{ex}[Liên trường huyện Quảng Xương - Thanh Hóa - 2021]%[2D1G1-2]
	\immini{
		Cho các hàm số $y=f(x)$; $y=g(x)$ liên tục trên $\mathbb{R}$ và có đồ thị các đạo hàm $f'(x) ; g'(x)$ (đồ thị hàm số $y=g'(x)$ là đường đậm hơn) như hình vẽ.\\
		Hàm số $h(x)=f(x-1)-g(x-1)$ nghịch biến trên khoảng nào dưới đây?
		\choice
		{$\left(\dfrac{1}{2}; 1\right)$}
		{$(1 ;+\infty)$}
		{$(2 ;+\infty)$}
		{\True $\left(-1 ; \dfrac{1}{2}\right)$}
	}
	{
		\begin{tikzpicture}[scale=1,>=stealth, font=\footnotesize, line join=round, line cap=round]
			%\def\a{1} \def\b{-6} \def\c{9} \def\d{1} % Hệ số
			\def\xmin{-4} \def\xmax{3}
			\def\ymin{-2} \def\ymax{4} 
			%\draw[color=gray!50,dashed] (\xmin,\ymin) grid (\xmax,\ymax); 
			\draw[->] (\xmin,0)--(\xmax,0) node [below]{$x$};
			\draw[->] (0,\ymin)--(0,\ymax) node [left]{$y$};
			\node at (0,0) [above left]{$O$};
			\node at (1,3) [below left]{$f'(x)$};
			\node at (1.5,3) [below right]{$g'(x)$};
			\draw[dashed] (-2,0) node[above right]{$-2$}--(-2,1);
			\draw[dashed] (1,0) node[below]{$1$}--(1,1);
			\draw[dashed] (-0.5,0) node[below]{$-0{,}5$}--(-0.5,2.125);
			\clip (\xmin+0.1,\ymin+0.1) rectangle (\xmax-0.5,\ymax-0.1);
			\draw[smooth,samples=300][domain=-3:2] plot(\x,{2*(\x)^4+4*(\x)^3-2*(\x)^2-4*(\x)+1});
			\draw[smooth,samples=300,line width=1.2pt] plot(\x,{(\x)^3+(\x)^2-2*(\x)+1});
		\end{tikzpicture}
	}
	
	\loigiai{
		Ta có: $h'(x)=f'(x-1)-g'(x-1)$.\\
		Dựa vào hình vẽ ta có hàm số $h(x)$ nghịch biến\\
		$\Leftrightarrow h'(x)<0 \Leftrightarrow f'(x-1)<g'(x-1)$\\
		$
		\Leftrightarrow\hoac{&- 2 < x - 1 < - \dfrac{1}{2}\\
			&0 < x - 1 < 1}
		\Leftrightarrow \hoac{
			&-1<x<\dfrac{1}{2}\\
			&1<x<2.}$\\
		Do đó hàm số $h(x)$ nghịch biến trên các khoảng $\left(-1 ; \dfrac{1}{2}\right)$ và $(1 ; 2)$.
	}
\end{ex}
\begin{ex}[THPT Quế Võ 1 - Bắc Ninh - 2021] %[2D1G1-2]
	\immini{
		Cho ba hàm số $y=f(x), y=g(x), y=h(x)$. Đồ thị của ba hàm số $y=f'(x), y=g'(x), y=h'(x)$ được cho như hình vẽ.\\
		Hàm số $k(x)=f(x+7)+g(5 x+1)-h\left(4 x+\dfrac{3}{2}\right)$ đồng biến trên khoảng nào dưới đây?
		\choice
		{$\left(-\dfrac{5}{8}; 0\right)$}
		{$\left(\dfrac{5}{8};+\infty\right)$}
		{\True $\left(\dfrac{3}{8}; 1\right)$}
		{$\left(-\dfrac{3}{8}; 1\right)$}
	}
	{
		\begin{tikzpicture}[scale=0.25,>=stealth, font=\footnotesize, line join=round, line cap=round]
			\def\a{-.078} \def\b{1.25} \def\c{0} % Hệ số
			\def\xmin{-4} \def\xmax{25}
			\def\ymin{-8} \def\ymax{18}
			
			%\draw[color=gray!50,dashed] (\xmin,\ymin) grid (\xmax,\ymax);
			
			\draw[->] (\xmin,0)--(\xmax,0) node [below]{$x$};
			\draw[->] (0,\ymin)--(0,\ymax) node [left]{$y$};
			\node at (20,14) [below right]{$y=g'(x)$};
			\node at (18,-2) [below left]{$y=h'(x)$};
			\node at (16,5) [below right]{$y=f'(x)$};
			\node at (0,0) [below left]{$O$};
			\draw[dashed] (3,0) node[below]{$3$}--(3,10)--(0,10) node[left]{$10$};
			\draw[dashed] (8,0) node[below]{$8$}--(8,5)--(0,5) node[left]{$5$};
			\draw[dashed] (4,0) node[below]{$4$}--(4,2)--(0,2) node[left]{$2$};
			\clip (\xmin+0.1,\ymin+0.1) rectangle (\xmax-0.5,\ymax-0.1);
			\draw[smooth,samples=300,domain=-2:18] plot(\x,{\a*(\x)^2+\b*(\x)+\c});
			%\draw[smooth,samples=300,domain=-2:25] plot(\x,{0.02*(\x)^3-0.6*(\x)^2+5.16*(\x)});
			\draw[line width=1.2pt] (-2,5)..controls (1.7,1.5) and (4.5,1.6)..(7,2.6);
			\draw[line width=1.2pt] (7,2.6)..controls (9,3.5) and (12,5)..(20,13);
			\draw (-0.5,-2) -- (0,0)--(3,10).. controls +(65:1) and + (-190:1)..(6,15).. controls +(0:1) and + (-180:1)..(14,-1).. controls +(0:1) and + (+80:1)..(19,16);
			
		\end{tikzpicture}
	}
	\loigiai{
		Ta có $k'(x)=f'(x+7)+5 g'(5 x+1)-4 h'\left(4 x+\dfrac{3}{2}\right)$.\\
		Khi $x \in \left( \dfrac{3}{8};1\right)$ thì $\heva{&7{,}375<x+7<8\\&2{,}875<5x+1<6\\&3<4x+\dfrac{4}{3}<5{,}5}\Leftrightarrow \heva{&f'(x+7)>10\\&g'(5x+1)>2 \Rightarrow 5g'(5x+1)>10  \\&h'\left( 4x+\dfrac{3}{2}\right)<5 \Rightarrow -4h'\left( 4x+\dfrac{3}{2}\right) >-20}.$\\
		Do đó $k'(x)=f'(x+7)+5g'(5x+1)-4h'\left( 4x+\dfrac{3}{2}\right)>0$.\\
		Hàm số $k(x)=f(x+7)+g(5 x+1)-h\left(4 x+\dfrac{3}{2}\right)$ đồng biến trên $\left(\dfrac{3}{8}; 1\right)$.
	}
\end{ex}
\begin{ex}[THPT Thanh Chương 1 - Nghệ An- 2021] %[2D1G1-2]
	Cho hàm số $y=f(x)$ liên tục trên $\mathbb{R}$ có bảng xét dấu đạo hàm như sau
	\begin{center}
		\begin{tikzpicture}
			\tkzTabInit[nocadre,lgt=1.2,espcl=2,deltacl=0.6]
			{$x$ /0.6,$f'(x)$ /0.6}
			{$-\infty$,$1$,$2$,$3$,$4$,$+\infty$}
			\tkzTabLine{,-,$0$,+,$0$,+,$0$,-,$0$,+,}
		\end{tikzpicture}
	\end{center}
	Hàm số $y=3f(2x-1)-4x^3+15x^2-18x+1$ đồng biến trên khoảng nào dưới đây?
	\choice
	{$\left(3;+\infty\right)$}
	{\True $\left(1;\dfrac{3}{2}\right)$}
	{$\left(\dfrac{5}{2}; 3\right)$}
	{$\left(2;\dfrac{5}{2}\right)$}
	\loigiai{
		Ta có $y'=6f'(2x-1)-12x^2+30x-18=6\left[f'(2x-1)-2x^2+5x-3\right] $.\\
		Có $f'(2x-1)=0 \Leftrightarrow \hoac{&2x-1=1\\&2x-1=2\\&2x-1=3\\&2x-1=4} \Leftrightarrow \hoac{&x=1\\&x=\dfrac{3}{2}\\&x=2\\&x=\dfrac{5}{2}.}$
		Ta có bảng xét dấu sau
		\begin{center}
			\begin{tikzpicture}
				\tkzTabInit[nocadre,lgt=3.0,espcl=1.5,deltacl=0.6]
				{$x$ /1.0,$f(x)$ /0.6,$f'(2x-1)$ /0.6,$-2x^2+5x-3$/0.6,$g'(x)$/0.6}
				{$-\infty$,$1$,$\dfrac{3}{2}$,$2$,$\dfrac{5}{2}$,$3$,$4$,$+\infty$}
				\tkzTabLine{,-,$0$,+,|,+,$0$,+,|,+,$0$,-,$0$,+,}
				\tkzTabLine{,-,$0$,+,$0$,+,$0$,-,$0$,+,|,+,|,+,}
				\tkzTabLine{,-,$0$,+,$0$,-,|,-,|,-,|,-,|,-,}
				\tkzTabLine{,-,$0$,+,$0$,,?,,|,,?,?,,?,}
			\end{tikzpicture}
		\end{center}
		Dựa vào bảng xét dấu trên, ta kết luận hàm số đã cho đồng biến trên khoảng $\left( 1; \dfrac{3}{2}\right).$
	}
\end{ex}


\begin{ex}%[2D2G4-3] %Câu 27 
	[THPT Hoàng Hoa Thám-Đà Nẵng-2021]
	Cho hàm số $f(x)$ có bảng xét dấu của $f'(x)$ như sau:\\
	\begin{center}
		\begin{tikzpicture}
			\tkzTabInit[lgt=1.2,espcl=2.3]
			{$x$/0.7, $f'(x)$ /.8} % first column
			{$-\infty$,$-3$,$1$, $2$, $+\infty$} % first row
			\tkzTabLine { ,+,0,-,0,+,0,+ }
		\end{tikzpicture}
	\end{center}	
	Hàm số $y=f\left(2-e^x\right)-\dfrac{1}{3}{e^{3x}}+3e^{2x}-5e^x+1$ đồng biến trên khoảng nào dưới đây?
	\choice
	{$\left(0;\dfrac{3}{2}\right)$}
	{$\left(1;3\right)$}
	{\True $\left(-3;0\right)$}
	{$\left(-4;-3\right)$}
	\loigiai{
		Ta có $y'=-e^x.f'\left(2-e^x\right)-e^{3x}+6e^{2x}-5e^x=e^x\left[-f'\left(2-e^x\right)-e^{2x}+6e^x-5\right]$ .\\
		Đặt $t=2-e^x$, ta được\\
		$y'=\left(2-t\right)\left[-f'(t)-\left(2-t\right)^2+6\left(2-t\right)-5\right]=\left(2-t\right)\left[-f'(t)-t^2-2t+3\right]$ .\\
		$y'=0\Leftrightarrow\left(2-t\right)\left[-f'(t)-t^2-2t+3\right]=0\Leftrightarrow
		\hoac{
			& t=2\\ 
			& f'(t)=-t^2-2t+3.}$\\
		Hàm số $g(x)=-x^2-2x+3$ là parabol có trục đối xứng $x=-1$ và cắt trục hoành tại 2 điểm có hoành độ 
		$\hoac{
			& x=1\\ 
			& x=-3
		}$. Suy ra $f'(t)=-t^2-2t+3\Leftrightarrow \hoac{
			& t=1\\ 
			& t=-3. }$\\
		Bảng xét dấu\\
		\begin{center}
			\begin{tikzpicture}
				\tkzTabInit[lgt=3.9,espcl=2,nocadre]
				{$t$/0.7, $2-t$ /0.8, $-f'(t)-t^2-2t+3$ /0.8, $y'$ /0.8} % first column
				{$-\infty$,$-3$,$1$,$2$,$+\infty$} % first row
				\tkzTabLine { ,+,|,+,|,+,z,-, } % second row
				\tkzTabLine {,-,0,+,0,-,|,-,} % third row
				\tkzTabLine {,-,0,+,0,-,0,+,} % last row
			\end{tikzpicture}
		\end{center}
		Dựa vào bảng xét dấu $y'>0,\forall x\in\left(-3;0\right)$.}
\end{ex}


\begin{ex}%[2D1G1-2]%Câu 28 
	[Sở Lạng Sơn 2022] Cho hàm số $f(x)$ có bảng biến thiên như sau:\\
	\begin{center}
		\begin{tikzpicture}
			\tkzTabInit[espcl=2.5,lgt=1,nocadre]
			{$x$/0.7,$y'$/0.7,$y$/3.5}
			{$-\infty$,$1$,$2$,$3$,$4$,$+\infty$}
			\tkzTabLine{,+,0,-,0,+,0,-,0,+,}
			\node (0) at ($(N12)+(0,-3)$) {$-\infty$};
			\node (1) at ($(N22)+(0,-.5)$) {$3$};
			\node (2) at ($(N32)+(0,-1.7)$) {$1$};
			\node (3) at ($(N42)+(0,-0.7)$) {$2$};
			\node (4) at ($(N52)+(0,-2.3)$) {$0$};
			\node (5) at ($(N62)+(0,-.3)$) {$+\infty$};
			%				\node (8) at ($(N42)+(0,-.5)$) {};
			%				\coordinate (9) at ($(N42)!.6!(N53)+ (-0.5,0)$);
			%				\coordinate (6) at ($(T12)!.6!(T13)$);
			%				\coordinate (7) at ($(T22)!.6!(T23)$);
			\draw[-stealth] (0)--(1);
			\draw[-stealth] (1)--(2);
			\draw[-stealth] (2)--(3);
			\draw[-stealth] (1)--(2);
			\draw[-stealth] (3)--(4);
			\draw[-stealth] (4)--(5);
			%				\draw[->,red] (5)--(8);
			%				\draw[->,red] (8)--(9);
			%				\draw[blue,dashed](6)--(7)node[above left]{$y=0$};
		\end{tikzpicture}		
	\end{center}
	Hàm số $y=\left[f(x)\right]^3-3\left[f(x)\right]^2$ đồng biến trên khoảng nào dưới đây?
	\choice
	{$\left(-\infty\,;1\right)$}
	{$\left(1\,;2\right)$}
	{\True $\left(3\,;4\right)$}
	{$\left(2\,;3\right)$}
	\loigiai{
		Ta có $y'=3f'(x)\left[f^2(x)-2f(x)\right]$. 
		Phương trình $y'=0\Leftrightarrow \hoac{
			&{f}'(x)=0\\ 
			& f(x)=0\\ 
			& f(x)=2.
		}$
		\begin{center}
			\begin{tikzpicture}
				\tkzTabInit[espcl=2.5,lgt=1.5]
				{$x$/0.7,$y'$/0.7,$y$/3.5}
				{$-\infty$,$1$,$2$,$3$,$4$,$+\infty$}
				\tkzTabLine{,+,0,-,0,+,0,-,0,+,}
				\node (0) at ($(N12)+(0,-3)$) {$-\infty$};
				\node (1) at ($(N22)+(0,-.3)$) {$3$};
				\node (2) at ($(N32)+(0,-1.7)$) {$1$};
				\node (3) at ($(N42)+(0,-0.8)$) {$2$};
				\node (4) at ($(N52)+(0,-2.3)$) {$0$};
				\node (5) at ($(N62)+(0,-.3)$) {$+\infty$};
				\node (a) at ($(N11)+(0.65,0.35)$) {$a$};
				\node (b) at ($(N11)+(2.0,0.4)$) {$b$};
				\node (c) at ($(N11)+(3.38,0.35)$) {$c$};
				\node (d) at ($(N11)+(11.85,0.4)$) {$d$};
				\node (6) at ($(N12)+(0,-0.8)$) {};
				\node (7) at ($(N62)+(0,-0.8)$) {};
				\node (8) at ($(N12)+(0,-2.3)$) {};
				\node (9) at ($(N62)+(0,-2.3)$) {};
				%				\node (8) at ($(N42)+(0,-.5)$) {};
				%				\coordinate (9) at ($(N42)!.6!(N53)+ (-0.5,0)$);
				\coordinate (A) at ($(0)!.25!(1)$);
				\coordinate (B) at ($(0)!.8!(1)$);
				\coordinate (C) at ($(1)!.35!(2)$);
				\coordinate (D) at ($(4)!.75!(5)$);
				%				\coordinate (7) at ($(T22)!.6!(T23)$);
				\draw[->] (0)--(1);
				\draw[->] (1)--(2);
				\draw[->] (2)--(3);
				\draw[->] (1)--(2);
				\draw[->] (3)--(4);
				\draw[->] (4)--(5);
				%				\draw[->,red] (5)--(8);
				%				\draw[->,red] (8)--(9);
				\draw[blue,dashed](6)--(7)node[below]{$y=2$} (a)--(A) (b)--(B) (c)--(C) (d)--(D);
				\draw[blue,dashed](8)--(9)node[below left]{$y=0$};
			\end{tikzpicture}		
		\end{center}
		Dựa vào bảng biến thiên, ta thấy $f'(x)=0\Leftrightarrow x\in \{ 1\,;2\,;3\,;4 \}$;\\
		$f(x)=0\Leftrightarrow x=a<1$ hoặc $x=4$;\\
		$f(x)=2\Leftrightarrow \hoac{
			& x=b\,\,\left(a<b<1\right)\\ 
			& x=c\in\left(1\,;2\right)\\ 
			& x=3\\ 
			& x=d>4.
		}$ \\
		Ta lập được bảng xét dấu của $y'$ 
		\begin{center}
			\begin{tikzpicture}
				\tkzTabInit[lgt=1.2,espcl=1.5,nocadre]
				{$x$/1, $f(x)$ /.8} % first column
				{$-\infty$,$a$, $b$, $1$,$c$, $2$,$3$, $4$, $d$, $+\infty$} % first row
				\tkzTabLine { ,+,z,-,z,+,z,-,z,+,z,-,z,+,z,-,z,+, } % second row
				%				\tkzTabLine {,-,z,+,t,+,} % third row
				%				\tkzTabLine {,+,d,-,z,+,} % last row
			\end{tikzpicture}
		\end{center}
		Từ bảng xét dấu, ta thấy hàm số đồng biến trên các khoảng \\
		$\left(-\infty;a\right)$, $\left(b;1\right)$, $\left(c;2\right)$, $\left(3;4\right)$ và $(d;+\infty)$.
	}
\end{ex}

\begin{ex}%[2D1G1-2]%Câu 29 
	[THPT Bùi Thị Xuân – Huế-2022] 
	\immini{
		Cho hàm số $y=f(x)$ là hàm đa thức bậc bốn. Đồ thị hàm số $f'(x+2)$ được cho trong hình vẽ bên. Hàm số 
		$$g(x)=4 f\left(x^2\right)-x^6+5 x^4-4 x^2+1$$
		đồng biến trên khoảng nào dưới đây?
		\choice
		{$(-4 ;-3)$}
		{\True $(2 ;+\infty)$}
		{$(-\sqrt{2};\sqrt{2})$}
		{$(-2 ;-1)$}}{
		\begin{tikzpicture}[scale=0.6,font=\footnotesize, line join=round, line cap=round, >=stealth] %Đường cong bậc 3
			\draw[thick, ->] (-5.3,0)--(5,0);
			\draw[thick, ->] (0,-3.5)--(0,7);
			\draw (5.2,0) node[below] {$x$};
			\draw (0,7.1) node[left]{$y$};
			\draw (0,0) node[below left]{$0$};
			\draw[fill] (-2,0) circle (0.5pt)node[below left]{$ -2 $};
			\draw[fill] (2,0) circle (0.5pt)node[below]{$ 2$};
			\draw[fill] (0,3) circle (0.5pt)node[left]{$ 3 $};
			\draw[fill] (0,1) circle (0.5pt)node[right]{$ 1 $};
			\draw[fill] (0,-1) circle (0.5pt)node[right]{$ -1 $};
			\draw[dashed] (-2,0)--(-2,1) --(0,1); 
			\draw[dashed](2,0)--(2,3)--(0,3);
			\draw[line width=1.2pt,smooth,samples=100,domain=-2.8:4.5] plot(\x,{-0.271*(\x)^3+0.75*(\x)^2+1.583*\x-1});
		\end{tikzpicture}		
	}
	\loigiai{
		$\begin{aligned}
			& g(x)=4f\left(x^2\right)-x^6+5x^4-4x^2+1\Rightarrow g' (x)=8xf'\left(x^2\right)-6x^5+20x^3-8x.\\ 
			& g' (x)=0\Leftrightarrow 8xf'\left(x^2\right)-6x^5+20x^3-8x=0 \\
			& \Leftrightarrow 2x\left[4f'\left(x^2\right)-3x^4+10x^2-4\right]=0\\ 
			&\Leftrightarrow 		\hoac{ 			& 2x=0\\ 
				& 4f'(x^2)-3x^4+10x^2-4=0
			}
			\Leftrightarrow \hoac{	& x=0\\ 
				& f'\left(x^2\right)=\dfrac{3}{4}{x^4}-\dfrac{5}{2}{x^2}+1.}
		\end{aligned}$\\ 
		Xét
		$f'\left(x^2\right)=\dfrac{3}{4}x^4-\dfrac{5}{2}x^2+1$. Đặt $x^2=t+2$, ta có\\
		$ f' (t+2)=\dfrac{3}{4}{(t+2)^2}-\dfrac{5}{2}(t+2)+1=\dfrac{3}{4}\left(t^2+4t+4\right)-\dfrac{5}{2}(t+2)-1=\dfrac{3}{4}{t^2}+\dfrac{1}{2}t-1$\\
		Khi đó số nghiệm của phương trình chính là số giao điểm của đồ thị hàm số $y=f' (t+2)$ và\\
		$ y=\dfrac{3}{4}{t^2}+\dfrac{1}{2}t-1$\\
		Ta có đồ thị 
		\begin{center}
			\begin{tikzpicture}[scale=0.6,font=\footnotesize, line join=round, line cap=round, >=stealth] %Đường cong bậc 3
				\draw[thick, ->] (-5.3,0)--(5,0);
				\draw[thick, ->] (0,-3.5)--(0,7);
				\draw (5.2,0) node[below] {$x$};
				\draw (0,7.1) node[left]{$y$};
				\draw (0,0) node[below left]{$0$};
				\draw[fill] (-2,0) circle (0.5pt)node[below left]{$ -2 $};
				\draw[fill] (2,0) circle (0.5pt)node[below]{$ 2$};
				\draw[fill] (0,3) circle (0.5pt)node[left]{$ 3 $};
				\draw[fill] (0,1) circle (0.5pt)node[right]{$ 1 $};
				\draw[fill] (0,-1) circle (0.5pt)node[right]{$ -1 $};
				\draw[dashed] (-2,0)--(-2,1) --(0,1); 
				\draw[dashed](2,0)--(2,3)--(0,3);
				\draw[line width=1.2pt,smooth,samples=100,domain=-2.8:4.5] plot(\x,{-0.271*(\x)^3+0.75*(\x)^2+1.583*\x-1});		
				\draw[line width=1.2pt,smooth,samples=100,domain=-3.3:2.8] plot(\x,{0.75*(\x)^2+0.5*\x-1});
			\end{tikzpicture}
		\end{center}
		Dựa vào đồ thị ta có $f' (t+2)=\dfrac{3}{4}t^2+\dfrac{1}{2}t-1\Leftrightarrow \hoac{& t=-2\\ & t=0\\ & t=2} \Leftrightarrow\hoac{& x+2=-2\\ & x+2=0\\ & x+2=2} \Leftrightarrow \hoac{& x=-4\\ & x=-2\\ & x=0.}$\\
		Ta có bảng xét dấu $g' (x)$ như sau
		\begin{center}
			\begin{tikzpicture}
				\tkzTabInit[lgt=1.2,espcl=2,nocadre]
				{$x$/0.7, $f(x)$ /.7}
				{$-\infty$, $-4$,$-2$, $0$, $+\infty$} % first row
				\tkzTabLine { ,-,z,+,z,-,z,+, }
			\end{tikzpicture}
		\end{center}
		Vậy hàm số $g(x)=4 f\left(x^2\right)-x^6+5 x^4-4 x^2+1$ đồng biến trên khoảng $(2 ;+\infty)$.}
\end{ex}

\begin{ex}%[2D1G1-2]%Câu 30
	[Chuyên Bắc Ninh 2022] 
	\immini{
		Cho hàm số $ y=f(x)$ liên tục trên $\mathbb{R}$ có đồ thị hàm số $ y=f'(x)$ có đồ thị như hình vẽ bên.
		Hàm số $g(x)=2f\left(\left| x-1\right|\right)-x^2+2x+2020$ đồng biến trên khoảng nào
		\choice
		{$\left(-2;0\right)$}
		{$\left(-3;1\right)$}
		{$\left(1\,;3\right)$}
		{\True $\left(0\,;\,1\right)$}}{
		\begin{tikzpicture}[scale=0.6,font=\footnotesize, line join=round, line cap=round, >=stealth] %Đường cong bậc 3
			\draw[thick, ->] (-3.3,0)--(5,0);
			\draw[thick, ->] (0,-3.0)--(0,5.5);
			\draw (5.2,0) node[below] {$x$};
			\draw (0,5.8) node[left]{$y$};
			\draw (0,0) node[below left]{$0$};
			\draw[fill] (-1,0) circle (0.5pt)node[above]{$ -1 $};
			\draw[fill] (1,0) circle (0.5pt)node[below]{$ 1$};
			\draw[fill] (0,1) circle (0.5pt)node[left]{$ 1 $};
			\draw[fill] (0,-1) circle (0.5pt)node[right]{$ -1 $};
			\draw[fill] (0,3) circle (0.5pt)node[left]{$ 3 $};
			\draw[fill] (3,0) circle (0.5pt)node[below]{$ 3 $};
			\draw[dashed] (-1,0)--(-1,-1) --(0,-1); 
			\draw[dashed](1,0)--(1,1)--(0,1);
			\draw[dashed](3,0)--(3,3)--(0,3);
			\draw[line width=1.2pt,smooth,samples=100,domain=-2.2:4.3] plot(\x,{-0.333*(\x)^3+1*(\x)^2+1.333*\x-1});		
			%\draw[line width=1.2pt,smooth,samples=100,domain=-3.3:2.8] plot(\x,{0.75*(\x)^2+0.5*\x-1});
		\end{tikzpicture}	
	}
	\loigiai{
		Ta có $g(x)=2f\left(\left| x-1\right|\right)-x^2+2x+2020\Leftrightarrow g(x)=2f\left(\left| x-1\right|\right)-\left(x-1\right)^2+2021$.\\
		Xét hàm số $ k\left(x-1\right)=2f\left(x-1\right)-\left(x-1\right)^2+2021$.\\
		Đặt $ t=x-1$\\
		Xét hàm số $ h(t)=2f(t)-t^2+2021$ $\Rightarrow{h}'(t)=2f'(t)-2t$.\\
		Kẻ đường $ y=x$ như hình vẽ.
		\begin{center}
			\begin{tikzpicture}[scale=0.6,font=\footnotesize, line join=round, line cap=round, >=stealth] %Đường cong bậc 3
				\draw[thick, ->] (-3.3,0)--(5,0);
				\draw[thick, ->] (0,-3.0)--(0,5.5);
				\draw (5.2,0) node[below] {$x$};
				\draw (0,5.8) node[left]{$y$};
				%	\draw (0,0) node[below left]{$0$};
				\draw[fill] (-1,0) circle (0.5pt)node[above]{$ -1 $};
				\draw[fill] (1,0) circle (0.5pt)node[below]{$ 1$};
				\draw[fill] (0,1) circle (0.5pt)node[left]{$ 1 $};
				\draw[fill] (0,-1) circle (0.5pt)node[right]{$ -1 $};
				\draw[fill] (0,3) circle (0.5pt)node[left]{$ 3 $};
				\draw[fill] (3,0) circle (0.5pt)node[below]{$ 3 $};
				\draw[dashed] (-1,0)--(-1,-1) --(0,-1); 
				\draw[dashed](1,0)--(1,1)--(0,1);
				\draw[dashed](3,0)--(3,3)--(0,3);
				\draw[line width=1.2pt,smooth,samples=100,domain=-2.2:4.3] plot(\x,{-0.333*(\x)^3+1*(\x)^2+1.333*\x-1});		
				%\draw[line width=1.2pt,smooth,samples=100,domain=-3.3:2.8] plot(\x,{0.75*(\x)^2+0.5*\x-1});
				\draw[line width=1.2pt,smooth,samples=100](-2,-2)--(4,4);
			\end{tikzpicture}
		\end{center}
		Khi đó $h'(t)>0\Leftrightarrow{f}'(t)-t>0\Leftrightarrow{f}'(t)>t$$\Leftrightarrow \hoac{
			& t<-1\\ 
			& 1<t<3.
		}$\\
		Do đó $k'\left(x-1\right)>0\Leftrightarrow \hoac{
			& x-1<-1\\ 
			& 1<x-1<3} \Leftrightarrow \hoac{
			& x<0\\ 
			& 2<x<4.}$\\
		Ta có bảng biến thiên của hàm số $ k\left(x-1\right)=2f\left(x-1\right)-\left(x-1\right)^2+2021$.
		\begin{center}
			\begin{tikzpicture}
				\tkzTabInit[lgt=1.8,espcl=2.3]
				{$x$ /1.2, $k'(x-1)$ /1.2,$k(x-1)$ /2}
				{$-\infty$ , $0$,$2$,$4$, $+\infty$}
				\tkzTabLine{,+,0,-,0,+,0,-,}
				\tkzTabVar{-/$ $ ,+/$ $, -/$ $,+/$ $,-/$ $}
			\end{tikzpicture}
		\end{center}
		Khi đó, ta có bảng biến thiên của $g(x)=2f\left(\left| x-1\right|\right)-\left(x-1\right)^2+2021$ bằng cách lấy đối xứng qua đường thẳng $ x=1$ như sau\\
		\begin{center}
			\begin{tikzpicture}
				\tkzTabInit[lgt=1.2,espcl=2.5,nocadre]
				{$x$ /0.7, $g'(x)$ /0.7,$g(x)$ /2.5}
				{$-\infty$ ,$-2$, $0$,$1$,$2$,$4$, $+\infty$}
				\tkzTabLine{,+,0,-,0,+,0,-,0,+,0,-,}
				\tkzTabVar{-/$ $ ,+/$ $, -/$ $,+/$ $,-/$ $,+/ $ $,-/$ $}
			\end{tikzpicture}
		\end{center}
		Vậy hàm số đồng biến trên $\left(0;1\right)$.}
\end{ex}

\begin{ex}%[2D1G1-2]%Câu 31
	[Chuyên Thái Bình 2022] 
	\immini{
		Cho hàm số $f(x)=a{x^4}+b{x^3}+c{x^2}+dx+a$ có đồ thị hàm số $y=f'(x)$ như hình vẽ bên. Hàm số $y=g(x)=f\left(1-2x\right)f\left(2-x\right)$ đồng biến trên khoảng nào dưới đây?
		\choice
		{$\left(\dfrac{1}{2};\dfrac{3}{2}\right)$}
		{$\left(-\infty ;0\right)$}
		{$\left(0;2\right)$}
		{\True $\left(3;+\infty\right)$}}{
		\begin{tikzpicture}[scale=0.9,font=\footnotesize, line join=round, line cap=round, >=stealth] %Đường cong bậc 3
			\draw[thick, ->] (-2.5,0)--(2.5,0);
			\draw[thick, ->] (0,-2.8)--(0,2.8);
			\draw (2.6,0) node[below] {$x$};
			\draw (0,2.9) node[left]{$y$};
			\draw (0,0) node[below left]{$0$};
			\draw[fill] (-1,0) circle (0.5pt)node[below left]{$ -1 $};
			\draw[fill] (1,0) circle (0.5pt)node[below right]{$ 1$};
			%			\draw[dashed] (-1,0)--(-1,-1) --(0,-1); 
			%			\draw[dashed](1,0)--(1,1)--(0,1);
			%			\draw[dashed](3,0)--(3,3)--(0,3);
			\draw[line width=1.2pt,smooth,samples=100,domain=-1.3:1.3] plot(\x,{3*(\x)^3-3*\x});		
			%\draw[line width=1.2pt,smooth,samples=100,domain=-3.3:2.8] plot(\x,{0.75*(\x)^2+0.5*\x-1});
		\end{tikzpicture}	
	}
	\loigiai{
		Ta có $f'(x)=4a{x^3}+3b{x^2}+2cx+d$, theo đồ thị thì đa thức $f'(x)$ có ba nghiệm phân biệt là $-1,0,1$ nên $f'(x)=4ax\left(x+1\right)\left(x-1\right)=4a{x^3}-4ax\Rightarrow f(x)=a{x^4}-2a{x^2}+a=a{\left(x^2-1\right)^2}$.\\
		Dựa vào đồ thị hàm số $y=f'(x)$ ta có $a>0$ nên $f(x)>0,\forall x\in\mathbb{R}\setminus\left\{\pm 1\right\}$.\\
		$g'(x)=\left[f\left(1-2x\right)\right]'f\left(2-x\right)+f\left(1-2x\right)\left[f\left(2-x\right)\right]'=-2f'\left(1-2x\right)f\left(2-x\right)-f\left(1-2x\right)f'\left(2-x\right)$. Xét $x\in\left(\dfrac{1}{2};\dfrac{3}{2}\right)\Rightarrow
		\heva{		
			& 1-2x\in\left(-2;0\right)\\ 
			& 2-x\in\left(\dfrac{1}{2};\dfrac{3}{2}\right)}$, dấu của $f'(x)$ không cố định trên $\left(\dfrac{1}{2};\dfrac{3}{2}\right)$ nên ta không kết luận được tính đơn điệu của hàm số $g(x)$ trên $\left(\dfrac{1}{2};\dfrac{3}{2}\right)$.\\
		Xét $x\in\left(-\infty ;0\right)\Rightarrow
		\heva{
			& 1-2x\in\left(1;+\infty\right)\\ 
			& 2-x\in\left(2;+\infty\right)} 
		\Rightarrow \heva{
			& f'\left(1-2x\right)>0\\ 
			& f'\left(2-x\right)>0} \Rightarrow g'(x)<0$.\\
		Do đó, hàm số $g(x)$ nghịch biến trên $\left(-\infty ;0\right)$.\\
		$x\in\left(0;2\right)\Rightarrow \heva{
			& 1-2x\in\left(-3;1\right)\\ 
			& 2-x\in\left(0;2\right)}$, dấu của $f'(x)$ không cố định trên $\left(-3;1\right)$ và $\left(0;2\right)$ nên ta không kết luận được tính đơn điệu của hàm số $g(x)$ trên $\left(\dfrac{1}{2};\dfrac{3}{2}\right)$.\\
		Xét $x\in\left(3;+\infty\right)\Rightarrow \heva{
			& 1-2x\in\left(-\infty ;-5\right)\\ 
			& 2-x\in\left(-\infty ;-1\right)} \Rightarrow \heva{
			& f'\left(1-2x\right)<0\\ 
			& f'\left(2-x\right)<0} \Rightarrow g'(x)>0$. \\
		Do đó, hàm số $g(x)$ đồng biến trên $\left(3;+\infty\right)$.}
\end{ex}

\begin{dang}{Bài toán hàm ẩn, hàm hợp liên quan đến tham số và một số bài toán khác}
\end{dang}

\begin{ex}%[2D1G1-3]%Câu 1
	[Chuyên Lê Hồng Phong Nam Định 2019]
	\immini{
		Cho hàm số $ y=f(x)$ có đạo hàm liên tục trên $\mathbb{R}$. Biết hàm số $ y=f'(x)$ có đồ thị như hình vẽ. Gọi $ S$ là tập hợp các giá trị nguyên $ m\in\left[-5\,;\,\text{5}\right]$ để hàm số $ g(x)=f\left(x+m\right)$ nghịch biến trên khoảng $\left(1\,;\,2\right)$. Hỏi $S$ có bao nhiêu phần tử?
		\choice
		{$ 4$}
		{$ 3$}
		{$ 6$}
		{\True $ 5$}}{
		\begin{tikzpicture}[scale=0.9,font=\footnotesize, line join=round, line cap=round, >=stealth] %Đường cong bậc 3
			\draw[thick, ->] (-2.5,0)--(4,0);
			\draw[thick, ->] (0,-2.8)--(0,2.8);
			\draw (4.3,0) node[below] {$x$};
			\draw (0,2.9) node[left]{$y$};
			\draw (0,0) node[below left]{$0$};
			\draw[fill] (-1,0) circle (0.5pt)node[below left]{$ -1 $};
			\draw[fill] (1,0) circle (0.5pt)node[below]{$ 1$};
			\draw[fill] (3,0) circle (0.5pt)node[below right]{$ 3$};
			%			\draw[dashed] (-1,0)--(-1,-1) --(0,-1); 
			%			\draw[dashed](1,0)--(1,1)--(0,1);
			%			\draw[dashed](3,0)--(3,3)--(0,3);
			\draw[line width=1.2pt,smooth,samples=100,domain=-1.65:3.5] plot(\x,{0.33*(\x)^3-(\x)^2-0.333*(\x)+1});		
			%\draw[line width=1.2pt,smooth,samples=100,domain=-3.3:2.8] plot(\x,{0.75*(\x)^2+0.5*\x-1});
		\end{tikzpicture}	
	}
	\loigiai{
		Ta có $g'(x)=f'\left(x+m\right)$. Vì $ y=f'(x)$ liên tục trên $\mathbb{R}$ nên $g'(x)=f'\left(x+m\right)$ cũng liên tục trên $\mathbb{R}$. Căn cứ vào đồ thị hàm số $ y=f'(x)$ ta thấy\\
		$g'(x)<0\Leftrightarrow{f}'\left(x+m\right)<0$ $\Leftrightarrow\hoac{
			& x+m<-1\\ 
			& 1<x+m<3} \Leftrightarrow \hoac{
			& x<-1-m\\ 
			& 1-m<x<3-m.}$\\
		Hàm số $ g(x)=f\left(x+m\right)$ nghịch biến trên khoảng $\left(1\,;\,2\right)$
		$\Leftrightarrow \hoac{
			& 2\le-1-m\\ 
			&\hoac{
				& 3-m\ge 2\\ 
				& 1-m\le 1}} \Leftrightarrow \hoac{
			& m\le-3\\ 
			& 0\le m\le 1.}$\\
		Mà $ m$ là số nguyên thuộc đoạn $\left[-5\,;\,5\right]$ nên ta có $ S=\left\{-5;-4;-3;0;1\right\}$.\\
		Vậy $ S$ có $5$ phần tử.}
\end{ex}

\begin{ex}%[2D1G1-3]%Câu 2
	[Chuyên Nguyễn Bỉnh Khiêm-Quảng Nam-2020] Cho hàm số $ y=f(x)$ có đạo hàm trên $\mathbb{R}$ và bảng xét dấu đạo hàm như hình vẽ sau
	\begin{center}
		\begin{tikzpicture}
			\tkzTabInit[lgt=1.2,espcl=2.5,nocadre]
			{$x$/0.7, $f'(x)$ /2.5} % first column
			{$-\infty$, $-10$,$-2$, $3$,$8$, $+\infty$} % first row
			\tkzTabLine { ,+,z,-,z,+,z,-,z,+, } % second row
			%				\tkzTabLine {,-,z,+,t,+,} % third row
			%				\tkzTabLine {,+,d,-,z,+,} % last row
		\end{tikzpicture}
	\end{center}
	Có bao nhiêu số nguyên $ m$ để hàm số $ y=f\left(x^3+4x+m\right)$ nghịch biến trên khoảng $\left(-1;1\right)$?
	\choice
	{$ 3$}
	{$ 0$}
	{\True $ 1$}
	{$ 2$}
	\loigiai
	{
		Đặt $ t=x^3+4x+m\Rightarrow{t}'=3x^2+4$ nên $ t$ đồng biến trên $\left(-1;1\right)$ và $ t\in\left(m-5;m+5\right)$.\\
		Yêu cầu bài toán trở thành tìm $ m$ để hàm số $ f(t)$ nghịch biến trên khoảng $\left(m-5;m+5\right)$.\\
		Dựa vào bảng biến thiên ta được $\heva{
			& m-5\ge-2\\ 
			& m+5\le 8} \Leftrightarrow \heva{
			& m\ge 3\\ 
			& m\le 3} \Leftrightarrow m=3$.}
\end{ex}

\begin{ex}%[2D1G1-3]%Câu 3
	[Chuyên ĐH Vinh-Nghệ An-2020]
	\immini{
		Cho hàm số $ f(x)$ có đạo hàm trên $\mathbb{R}$và $ f(1)=1$. Đồ thị hàm số $ y=f'(x)$ như hình bên. Có bao nhiêu số nguyên dương $ a$ để hàm số $ y=\left| 4f\left(\sin x\right)+\cos 2x-a\right|$ nghịch biến trên $\left(0;\dfrac{\pi}{2}\right)$?
		\choice
		{$ 2$}
		{\True $ 3$}
		{Vô số}
		{$ 5$}}{
		\begin{tikzpicture}[scale=0.9,font=\footnotesize, line join=round, line cap=round, >=stealth] %Đường cong bậc 3
			\draw[thick, ->] (-2.5,0)--(3,0);
			\draw[thick, ->] (0,-2.8)--(0,2.8);
			\draw (3.1,0) node[below] {$x$};
			\draw (0,2.9) node[left]{$y$};
			\draw (0,0) node[below left]{$0$};
			\draw[fill] (-1,0) circle (0.5pt)node[below]{$ -1 $};
			\draw[fill] (1,0) circle (0.5pt)node[above]{$ 1$};
			%	\draw[fill] (3,0) circle (0.5pt)node[below right]{$ 3$};
			\draw[dashed] (-1,0)--(-1,1); 
			\draw[dashed](1,0)--(1,-1);
			%			\draw[dashed](3,0)--(3,3)--(0,3);
			\draw[line width=1.2pt,smooth,samples=100,domain=-2:2] plot(\x,{.8*(\x)^3+0*(\x)^2-1.8*(\x)});		
			%\draw[line width=1.2pt,smooth,samples=100,domain=-3.3:2.8] plot(\x,{0.75*(\x)^2+0.5*\x-1});
			\draw (2.0,2.8) node[left]{$y=f'(x)$};
		\end{tikzpicture}	
	}
	\loigiai
	{		Đặt $g(x)=\left| 4f\left(\sin x\right)+\cos 2x-a\right|\Rightarrow g(x)=\sqrt{\left[4f\left(\sin x\right)+\cos 2x-a\right]^2}$ .\\
		$\Rightarrow{g}'(x)=\dfrac{\left[4\cos x\cdot f'\left(\sin x\right)-2\sin 2x\right]\left[4f\left(\sin x\right)+\cos 2x-a\right]}{\sqrt{\left[4f\left(\sin x\right)+\cos 2x-a\right]^2}}$.\\
		Ta có $ 4\cos x\cdot f'\left(\sin x\right)-2\sin 2x=4\cos x\left[f'\left(\sin x\right)-\sin x\right]$.\\
		Với $ x\in\left(0;\dfrac{\pi}{2}\right)$ thì $\cos x>0,\sin x\in\left(0;1\right)\Rightarrow{f}'\left(\sin x\right)-\sin x<0$.\\
		Hàm số $ g(x)$ nghịch biến trên $\left(0;\dfrac{\pi}{2}\right)$ khi $ 4f\left(\sin x\right)+\cos 2x-a\ge 0,\forall x\in\left(0;\dfrac{\pi}{2}\right)$\\
		$\Leftrightarrow 4f\left(\sin x\right)+1-2\sin^2x\ge a,\forall x\in\left(0;\dfrac{\pi}{2}\right)$.\\
		Đặt $ t=\sin x$ được $ 4f(t)+1-2t^2\ge a,\forall t\in\left(0;1\right)$ (*).\\
		Xét $ h(t)=4f(t)+1-2t^2\Rightarrow{h}'(t)=4f'(t)-4t=4\left[f'(t)-1\right]$.\\
		Với $ t\in\left(0;1\right)$ thì $h'(t)<0\Rightarrow h(t)$ nghịch biến trên $\left(0;1\right)$.\\
		Do đó (*) $\Leftrightarrow a\le h(1)=4f(1)+1-2.1^2=3$.\\
		Vậy có $3$ giá trị nguyên dương của a thỏa mãn.}
\end{ex}


\begin{ex}%[2D1G1-3]%Câu 4
	[Chuyên Quang Trung-2020]
	\immini{
		Cho hàm số $ y=f(x)$ có đạo hàm liên tục trên $\mathbb{R}$ và có đồ thị $ y=f'(x)$ như hình vẽ. Đặt $ g(x)=f\left(x-m\right)-\dfrac{1}{2}{\left(x-m-1\right)^2}+2019$, với $ m$ là tham số thực. Gọi $ S$ là tập hợp các giá trị nguyên dương của $ m$ để hàm số $ y=g(x)$ đồng biến trên khoảng $\left(5;6\right)$. Tổng tất cả các phần tử trong $ S$ bằng
		\choice
		{$ 4$}
		{$ 11$}
		{\True $ 14$}
		{$ 20$}}{
		\begin{tikzpicture}[scale=0.9,font=\footnotesize, line join=round, line cap=round, >=stealth] %Đường cong bậc 3
			\draw[style=help lines,step=1] (-2.5,-3) grid (3,3.5);
			\draw[thick, ->] (-2.5,0)--(3.5,0);
			\draw[thick, ->] (0,-2.8)--(0,2.8);
			\draw (3.6,0) node[below] {$x$};
			\draw (0,3) node[above left]{$y$};
			\draw (0,0) node[below left]{$0$};
			%\draw[fill] (-1,0) circle (0.5pt)node[below]{$ -1 $};
			\draw[fill] (1,0) circle (0.5pt)node[below left]{$ 1$};
			%	\draw[fill] (3,0) circle (0.5pt)node[below right]{$ 3$};
			\draw[dashed] (-1,0)--(-1,-2) --(2,-2)--(2,0); 
			\draw[dashed](3,0)--(3,2) --(0,2);
			\draw (-1,-2) circle (2pt);
			\draw (3,2) circle (2pt);
			%			\draw[dashed](3,0)--(3,3)--(0,3);
			\draw[line width=1.2pt,smooth,samples=100,domain=-1.1:3.1] plot(\x,{1*(\x)^3-3*(\x)^2-0*(\x)+2});		
			%\draw[line width=1.2pt,smooth,samples=100,domain=-3.3:2.8] plot(\x,{0.75*(\x)^2+0.5*\x-1});
			%\draw (2.0,2.8) node[left]{$y=f'(x)$};
		\end{tikzpicture}	
	}
	\loigiai
	{
		Xét hàm số $ g(x)=f\left(x-m\right)-\dfrac{1}{2}{\left(x-m-1\right)^2}+2019$.\\
		$g'(x)=f'\left(x-m\right)-\left(x-m-1\right)$.\\
		Xét phương trình $g'(x)=0. \quad \quad (1)$\\
		Đặt $ x-m=t$, phương trình $(1)$ trở thành $f'(t)-\left(t-1\right)=0\Leftrightarrow{f}'(t)=t-1. \quad (2)$\\
		Nghiệm của phương trình $(2)$ là hoành độ giao điểm của hai đồ thị hàm số $ y=f'(t)$ và $ y=t-1$.\\
		Ta có đồ thị các hàm số $ y=f'(t)$ và $ y=t-1$ như sau
		\begin{center}
			\begin{tikzpicture}[scale=0.9,font=\footnotesize, line join=round, line cap=round, >=stealth] %Đường cong bậc 3
				\draw[style=help lines,step=1] (-2.5,-3) grid (3,3.5);
				\draw[thick, ->] (-2.5,0)--(3.5,0);
				\draw[thick, ->] (0,-2.8)--(0,2.8);
				\draw (3.6,0) node[below] {$x$};
				\draw (0,3) node[above left]{$y$};
				\draw (0,0) node[below left]{$0$};
				%\draw[fill] (-1,0) circle (0.5pt)node[below]{$ -1 $};
				\draw[fill] (1,0) circle (0.5pt)node[below left]{$ 1$};
				%	\draw[fill] (3,0) circle (0.5pt)node[below right]{$ 3$};
				\draw[dashed] (-1,0)--(-1,-2) --(2,-2)--(2,0); 
				\draw[dashed](3,0)--(3,2) --(0,2);
				\draw (-1,-2) circle (2pt);
				\draw (3,2) circle (2pt);
				%			\draw[dashed](3,0)--(3,3)--(0,3);
				\draw[line width=1.2pt,smooth,samples=100,domain=-1.1:3.1] plot(\x,{1*(\x)^3-3*(\x)^2-0*(\x)+2});		
				%\draw[line width=1.2pt,smooth,samples=100,domain=-3.3:2.8] plot(\x,{0.75*(\x)^2+0.5*\x-1});
				%\draw (2.0,2.8) node[left]{$y=f'(x)$};
				\draw (-2,-3)--(4,3);
			\end{tikzpicture}
		\end{center}
		Căn cứ đồ thị các hàm số ta có phương trình $(2)$ có nghiệm là $\hoac{
			& t=-1\\ 
			& t=1\\ 
			& t=3} \Rightarrow \hoac{
			& x=m-1\\ 
			& x=m+1\\ 
			& x=m+3.}$\\
		Ta có bảng biến thiên của $ y=g(x)$
		\begin{center}
			\begin{tikzpicture}
				\tkzTabInit[lgt=1,espcl=2.5,nocadre]
				{$x$ /0.8, $y'$ /0.8,$y$ /2.5}
				{$-\infty$ , $m-1$,$m+1$,$m+3$, $+\infty$}
				\tkzTabLine{,+,0,-,0,+,0,-,}
				\tkzTabVar{-/$ +\infty$ ,+/$ $, -/$ $,+/$ $,-/$+\infty $}
			\end{tikzpicture}
		\end{center}
		Để hàm số $ y=g(x)$ đồng biến trên khoảng $\left(5;6\right)$ cần $\hoac{
			&\heva{
				& m-1\le 5\\ 
				& m+1\ge 6}\\ 
			& m+3\le 5}\Leftrightarrow\hoac{
			& 5\le m\le 6\\ 
			& m\le 2.}$\\
		Vì $ m\in\mathbb{N}^*\Rightarrow m$ nhận các giá trị $ 1;\,2;\,5;\,6\Rightarrow S=14$.}
\end{ex}

\begin{ex}%[2D1G1-3]%Câu 5
	[Sở Hà Nội-Lần 2-2020] 
	\immini{
		Cho hàm số $y=a{x^4}+b{x^3}+c{x^2}+dx+e,\,\,a\ne 0$. Hàm số $y=f'(x)$ có đồ thị như hình vẽ bên. 
		Gọi S là tập hợp tất cả các giá trị nguyên thuộc khoảng $\left(-6;6\right)$ của tham số $m$ để hàm số $g(x)=f\left(3-2x+m\right)+x^2-\left(m+3\right)x+2m^2$ nghịch biến trên $\left(0;1\right)$. Khi đó, tổng giá trị các phần tử của S là
		\choice
		{$12$}
		{\True $9$}
		{$6$}
		{$15$}}{
		\begin{tikzpicture}[scale=0.7,font=\footnotesize, line join=round, line cap=round, >=stealth] %Đường cong bậc 3
			%	\draw[style=help lines,step=1] (-2.5,-3) grid (3,3.5);
			\draw[thick, ->] (-4.5,0)--(6.5,0);
			\draw[thick, ->] (0,-2.8)--(0,2.8);
			\draw (6.6,0) node[below] {$x$};
			\draw (0,3) node[above left]{$y$};
			\draw (0,0) node[below left]{$0$};
			\draw[fill] (-2,0) circle (0.5pt)node[below]{$ -2 $};
			\draw[fill] (4,0) circle (0.5pt)node[above]{$ 4$};
			\draw[fill] (0,1) circle (0.5pt)node[right]{$ 1 $};
			\draw[fill] (0,-2) circle (0.5pt)node[left]{$ -2$};
			%	\draw[fill] (3,0) circle (0.5pt)node[below right]{$ 3$};
			\draw[dashed] (-2,0)--(-2,1) --(0,1); 
			\draw[dashed](4,0)--(4,-2) --(0,-2);
			%			\draw[dashed](3,0)--(3,3)--(0,3);
			\draw[line width=1.2pt,smooth,samples=100,domain=-3.8:5.5] plot(\x,{0.0714*(\x)^3-0.1423*(\x)^2-1.0714*(\x)});		
			%\draw[line width=1.2pt,smooth,samples=100,domain=-3.3:2.8] plot(\x,{0.75*(\x)^2+0.5*\x-1});
			%\draw (2.0,2.8) node[left]{$y=f'(x)$};
		\end{tikzpicture}	
	}
	\loigiai
	{
		Xét $g'(x)=-2f'\left(3-2x+m\right)+2x-\left(m+3\right)$.\\
		Xét phương trình $g'(x)=0$, đặt $t=3-2x+m$ thì phương trình trở thành\\ $-2\cdot \left[f'(t)-\dfrac{-t}{2}\right]=0\Leftrightarrow\hoac{
			& t=-2\\ 
			& t=4\\ 
			& t=0.}$ \\
		Từ đó, $g'(x)=0\Leftrightarrow{x_1}=\dfrac{5+m}{2},\,x_2=\dfrac{m+3}{2},x_3=\dfrac{-1+m}{2}$.\\
		Lập bảng xét dấu, đồng thời lưu ý nếu $x>x_1$ thì $t<t_1$ nên $f(x)>0$. Và các dấu đan xen nhau do các nghiệm đều làm đổi dấu đạo hàm nên suy ra $g'(x)\le 0\Leftrightarrow x\in\left[x_2;{x_1}\right]\cup\left(-\infty ;{x_3}\right]$.\\
		Vì hàm số nghịch biến trên $\left(0;1\right)$ nên \\
		$g'(x)\le 0,\,\forall x\in\left(0;1\right)$ từ đó suy ra $\hoac{
			&\dfrac{3+m}{2}\le 0<1\le\dfrac{5+m}{2}\\ 
			& 1\le\dfrac{-1+m}{2}.}$ \\
		và giải ra các giá trị nguyên thuộc $\left(-6;6\right)$ của $m$ là $-3$; $3$; $4$; $5$. }
\end{ex}

\begin{ex}%[2D1G1-3]%Câu 6
	[Chuyên Quang Trung-Bình Phước-Lần 2-2020]
	\immini{
		Cho hàm số $ y=f(x)$ có đạo hàm liên tục trên $\mathbb{R}$ và có đồ thị $ y=f'(x)$ như hình vẽ bên. Đặt $ g(x)=f\left(x-m\right)-\dfrac{1}{2}{\left(x-m-1\right)^2}+2019$, với $ m$ là tham số thực. Gọi $ S$ là tập hợp các giá trị nguyên dương của $ m$ để hàm số $ y=g(x)$ đồng biến trên khoảng $\left(5;6\right)$. Tổng tất cả các phần tử trong $ S$ bằng
		\choice
		{$ 4$}
		{$ 11$}
		{\True $ 14$}
		{$ 20$}}{
		\begin{tikzpicture}[scale=0.9,font=\footnotesize, line join=round, line cap=round, >=stealth] %Đường cong bậc 3
			\draw[thick, ->] (-2.5,0)--(3.7,0);
			\draw[thick, ->] (0,-2.8)--(0,2.8);
			\draw (3.9,0) node[below] {$x$};
			\draw (0,2.9) node[left]{$y$};
			\draw (0,0) node[below left]{$0$};
			\draw[fill] (-1,0) circle (0.5pt)node[above]{$ -1 $};
			\draw[fill] (1,0) circle (0.5pt)node[below]{$ 1$};
			\draw[fill] (3,0) circle (0.5pt)node[below]{$ 3$};
			\draw[fill] (2,0) circle (0.5pt)node[above]{$ 2$};
			\draw[fill] (0,2) circle (0.5pt)node[above left]{$ 2$};
			\draw[fill] (0,-2) circle (0.5pt)node[below left]{$ -2$};
			\draw[dashed] (-1,0)--(-1,-2)--(2,-2)--(2,0); 
			\draw[dashed](3,0)--(3,2)--(0,2);
			%			\draw[dashed](3,0)--(3,3)--(0,3);
			\draw[line width=1.2pt,smooth,samples=100,domain=-1.1:3.1] plot(\x,{1*(\x)^3-3*(\x)^2-0*(\x)+2});		
			%\draw[line width=1.2pt,smooth,samples=100,domain=-3.3:2.8] plot(\x,{0.75*(\x)^2+0.5*\x-1});
			%	\draw (2.0,2.8) node[left]{$y=f'(x)$};
	\end{tikzpicture}	}
	\loigiai
	{
		Ta có $g'(x)=f'\left(x-m\right)-\left(x-m-1\right)$.\\
		Cho $g'(x)=0\Leftrightarrow{f}'\left(x-m\right)=x-m-1$.\\
		Đặt $ x-m=t\Rightarrow f'(t)=t-1$\\
		Khi đó nghiệm của phương trình là hoành độ giao điểm của đồ thị hàm số $ y=f'(t)$ và và đường thẳng $ y=t-1$.
		\begin{center}
			\begin{tikzpicture}[scale=0.9,font=\footnotesize, line join=round, line cap=round, >=stealth] %Đường cong bậc 3
				\draw[thick, ->] (-2.5,0)--(3.7,0);
				\draw[thick, ->] (0,-2.8)--(0,2.8);
				\draw (3.9,0) node[below] {$x$};
				\draw (0,2.9) node[left]{$y$};
				\draw (0,0) node[below left]{$0$};
				\draw[fill] (-1,0) circle (0.5pt)node[above]{$ -1 $};
				\draw[fill] (1,0) circle (0.5pt)node[below]{$ 1$};
				\draw[fill] (3,0) circle (0.5pt)node[below]{$ 3$};
				\draw[fill] (2,0) circle (0.5pt)node[above]{$ 2$};
				\draw[fill] (0,2) circle (0.5pt)node[above left]{$ 2$};
				\draw[fill] (0,-2) circle (0.5pt)node[below left]{$ -2$};
				\draw[dashed] (-1,0)--(-1,-2)--(2,-2)--(2,0); 
				\draw[dashed](3,0)--(3,2)--(0,2);
				%			\draw[dashed](3,0)--(3,3)--(0,3);
				\draw[line width=1.2pt,smooth,samples=100,domain=-1.1:3.1] plot(\x,{1*(\x)^3-3*(\x)^2-0*(\x)+2});		
				%\draw[line width=1.2pt,smooth,samples=100,domain=-3.3:2.8] plot(\x,{0.75*(\x)^2+0.5*\x-1});
				%	\draw (2.0,2.8) node[left]{$y=f'(x)$};
				\coordinate (a) at ($(-1,-2)!1.2!(3,2)$);
				\coordinate (b) at ($(-1,-2)!-.2!(3,2)$);
				\draw[line width=1.2pt,smooth] (a)--(b);
			\end{tikzpicture}
		\end{center}
		Dựa vào đồ thị hàm số ta có được $f'(t)=t-1\Leftrightarrow\hoac{
			& t=-1\\ 
			& t=1\\ 
			& t=3.} $ \\
		Bảng xét dấu của $g'(t)$
		\begin{center}
			\begin{tikzpicture}
				\tkzTabInit[lgt=1.2,espcl=2.5,nocadre]
				{$t$/1, $g'(x)$ /.8} % first column
				{$-\infty$, $-1$,$1$, $3$, $+\infty$} % first row
				\tkzTabLine { ,-,0,+,0,-,0,+, } % second row
				%				\tkzTabLine {,-,z,+,t,+,} % third row
				%				\tkzTabLine {,+,d,-,z,+,} % last row
			\end{tikzpicture}
		\end{center}
		Từ bảng xét dấu ta thấy hàm số $ g(t)$ đồng biến trên khoảng $\left(-1;1\right)$ và $\left(3;+\infty\right)$.\\
		Hay $\hoac{
			&-1<t<1\\ 
			& t>3}\Leftrightarrow\hoac{
			&-1<x-m<1\\ 
			& x-m>3} \Leftrightarrow\hoac{
			& m-1<x<m+1\\ 
			& x>m+3.}$\\
		Để hàm số $ g(x)$ đồng biến trên khoảng $\left(5;6\right)$ thì $\hoac{
			& m-1\le 5<6\le m+1\\ 
			& m+3\le 5<6} \Leftrightarrow\hoac{
			& 5\le m\le 6\\ 
			& m\le 2.}$\\
		Vì $ m$ là các số nguyên dương nên $ S=\left\{ 1;2;5;6\right\}$.\\
		Vậy tổng tất cả các phần tử của $ S$ là $ 1+2+5+6=14$.}
\end{ex}

\begin{ex}%[2D1G1-3]%Câu 7
	\immini{
		Cho hàm số $ y=f(x)$ liên tục có đạo hàm trên $\mathbb{R}$. Biết hàm số $ f'(x)$ có đồ thị cho như hình vẽ bên. Có bao nhiêu giá trị nguyên của $ m$ thuộc $\left[-2019;2019\right]$ để hàm só $ g(x)=f\left(2019^x\right)-mx+2$ đồng biến trên $\left[0;1\right]$.
		\choice
		{$ 2028$}
		{$ 2019$}
		{$ 2011$}
		{\True $ 2020$}}{
		\begin{tikzpicture}[scale=0.9,font=\footnotesize, line join=round, line cap=round, >=stealth] %Đường cong bậc 3
			\draw[thick, ->] (-3.5,0)--(2.5,0);
			\draw[thick, ->] (0,-2.8)--(0,2.8);
			\draw (2.7,0) node[below] {$x$};
			\draw (0,2.9) node[left]{$y$};
			\draw (0,0) node[below left]{$0$};
			%	\draw[fill] (-1,0) circle (0.5pt)node[above]{$ -1 $};
			\draw[fill] (1,0) circle (0.5pt)node[below right]{$ 1$};
			%		\draw[fill] (3,0) circle (0.5pt)node[below]{$ 3$};
			%		\draw[fill] (2,0) circle (0.5pt)node[above]{$ 2$};
			%		\draw[fill] (0,2) circle (0.5pt)node[above left]{$ 2$};
			%		\draw[fill] (0,-2) circle (0.5pt)node[below left]{$ -2$};
			%		\draw[dashed] (-1,0)--(-1,-2)--(2,-2)--(2,0); 
			%		\draw[dashed](3,0)--(3,2)--(0,2);
			\draw[line width=1.2pt,smooth,samples=100,domain=-3.28:1.32] plot(\x,{0.667*(\x)^3+2*(\x)^2-0.667*(\x)-2});		
			%\draw[line width=1.2pt,smooth,samples=100,domain=-3.3:2.8] plot(\x,{0.75*(\x)^2+0.5*\x-1});
			%	\draw (2.0,2.8) node[left]{$y=f'(x)$};
	\end{tikzpicture}	}
	\loigiai{
		Ta có $ g'(x)=2019^x\ln 2019\cdot f'\left(2019^x\right)-m$.\\
		Ta lại có hàm số $ y=2019^x$ đồng biến trên $\left[0;1\right]$.\\
		Với $ x\in\left[0;1\right]$ thì $2019^x\in\left[1;2019\right]$ mà hàm $ y=f'(x)$ đồng biến trên $\left(1;+\infty\right)$ nên hàm $ y=f'\left(2019^x\right)$ đồng biến trên $\left[0;1\right]$.\\
		Mà $2019^x\ge 1;f'\left(2019^x\right)>0\,\forall\,x\in\left[0;1\right]$ nên hàm $ h(x)=2019^x\ln 2019\cdot f'\left(2019^x\right)$ đồng biến trên $\left[0;1\right]$.\\
		Hay $ h(x)\ge h(0)=0,\forall\,x\in\left[0;1\right]$.\\
		Do vậy hàm số $ g(x)$ đồng biến trên đoạn $\left[0;1\right]$$\Leftrightarrow g'(x)\ge 0,\forall\,x\in\left[0;1\right]$\\
		$\Leftrightarrow m\le{2019^x}\ln 2019.f'\left(2019^x\right),\forall\,x\in\left[0;1\right]$ $\Leftrightarrow m\le\underset{x\in\left[0;1\right]}{\min}\,h(x)=h(0)=0$\\
		Vì $ m$ nguyên và $ m\in\left[-2019;2019\right]\Rightarrow $có $ 2020$ giá trị $ m$ thỏa mãn yêu cầu bài toán.}
\end{ex}

\begin{ex}%[2D1G1-3]%Câu 8
	\immini{
		Cho hàm số $y=f(x)$ có đồ thị $f'(x)\,$ như hình vẽ. Có bao nhiêu giá trị nguyên $m\in\left(-2020\,;\,2020\right)$ để hàm số $g(x)=f\left(2x-3\right)\,-\ln \left(1+x^2\right)-2mx$ đồng biến trên $\left(\dfrac{1}{2};2\right)$?
		\choice
		{$ 2020$}
		{\True $ 2019$}
		{$ 2021$}
		{$ 2018$}}{
		\begin{tikzpicture}[scale=0.9,font=\footnotesize, line join=round, line cap=round, >=stealth] %Đường cong bậc 3
			\draw[thick, ->] (-2.5,0)--(2.5,0);
			\draw[thick, ->] (0,-1.8)--(0,5.8);
			\draw (2.7,0) node[below] {$x$};
			\draw (0,5.9) node[left]{$y$};
			\draw (0,0) node[below left]{$0$};
			\draw[fill] (-2,0) circle (0.5pt)node[below]{$ -2 $};
			\draw[fill] (1,0) circle (0.5pt)node[below]{$ 1$};
			\draw[fill] (-1,0) circle (0.5pt)node[below]{$-1$};
			\draw[fill] (0,4) circle (0.5pt)node[above left]{$ 2$};
			%		\draw[fill] (0,2) circle (0.5pt)node[above left]{$ 2$};
			%		\draw[fill] (0,-2) circle (0.5pt)node[below left]{$ -2$};
			\draw[dashed] (-2,0)--(-2,4)--(1,4)--(1,0); 
			%		\draw[dashed](3,0)--(3,2)--(0,2);
			\draw[line width=1.2pt,smooth,samples=100,domain=-2.1:2.1] plot(\x,{-1*(\x)^3+0*(\x)^2+3*(\x)+2});		
			%\draw[line width=1.2pt,smooth,samples=100,domain=-3.3:2.8] plot(\x,{0.75*(\x)^2+0.5*\x-1});
			%	\draw (2.0,2.8) node[left]{$y=f'(x)$};
	\end{tikzpicture}	}
	\loigiai{
		Ta có $g'(x)=2f'\left(2x-3\right)-\dfrac{2x}{1+x^2}-2m$.\\
		Hàm số $ g(x)$ đồng biến trên $\left(\dfrac{1}{2};2\right)$ khi và chỉ khi \\
		$g'(x)\ge 0,\,\,\forall x\in\left(-1;\,2\right)$\\
		$\Leftrightarrow m\le{f}'\left(2x-3\right)-\dfrac{x}{1+x^2},\,\,\forall x\in\left(\dfrac{1}{2};2\right)$\\
		$\Leftrightarrow m\le\underset{x\in\left[\dfrac{1}{2};2\right]}{\min}\,\left[f'\left(2x-3\right)-\dfrac{x}{1+x^2}\right]$. \, \,  $(1)$\\
		Đặt $ t=2x-3$, khi đó $ x\in\left(\dfrac{1}{2};2\right)\Leftrightarrow t\in\left(-2;\,1\right)$.\\
		Từ đồ thị hàm $f'(x)$ suy ra $f'(t)\ge 0,\,\,\forall t\in\left(-2;1\right)$ và $f'(t)=0$ khi $ t=-1$.\\
		Tức là $f'\left(2x-3\right)\ge 0,\,\,\forall x\in\left(\dfrac{1}{2};\,2\right)$$\Rightarrow\underset{x\in\left[\dfrac{1}{2};2\right]}{\min}\,f'\left(2x-3\right)=0$ khi $ x=1$. $(2)$\\
		Xét hàm số $ h(x)=-\dfrac{x}{1+x^2}$ trên khoảng $\left(\dfrac{1}{2};\,2\right)$.\\
		Ta có $h'(x)=\dfrac{x^2-1}{\left(1+x^2\right)^2}$ và\\
		$h'(x)=0\Leftrightarrow{x^2}-1=0\Leftrightarrow x=\pm 1$.\\
		Bảng biến thiên của hàm số $ h(x)$ trên $\left(\dfrac{1}{2};\,2\right)$ như sau
		\begin{center}
			\begin{tikzpicture}
				\tkzTabInit[lgt=1.2,espcl=2.5,nocadre]
				{$x$ /0.7, $h'(x)$ /0.7,$h(x)$ /2.5}
				{$\dfrac{1}{2}$ , $1$,$2$}
				\tkzTabLine{,-,0,+,}
				\tkzTabVar{+/$  $ ,-/$ \-\dfrac{1}{2} $, +/$ $}
			\end{tikzpicture}
		\end{center}
		Từ bảng biến thiên suy ra $ h(x)\ge-\dfrac{1}{2}$$\Rightarrow\underset{x\in\left[\dfrac{1}{2};2\right]}{\min}\,h(x)=-\dfrac{1}{2}$ khi $ x=1$. \, \,  $(3)$\\
		Từ $(1)$, $(2)$ và $(3)$ suy ra $ m\le-\dfrac{1}{2}$.\\
		Kết hợp với $ m\in\mathbb{Z}$, $ m\in\left(-2020;\,2020\right)$ thì $ m\in\left\{-2019;\,-201;\ldots ;-2;-1\right\}$.\\
		Vậy có tất cả $ 2019$ giá trị $ m$ cần tìm.}
\end{ex}

\begin{ex}%[2D1G1-3]%Câu 9
	Cho hàm số $ f(x)$ liên tục trên $\mathbb{R}$ và có đạo hàm $f'(x)=x^2\left(x-2\right)\left(x^2-6x+m\right)$ với mọi $ x\in\mathbb{R}$. Có bao nhiêu số nguyên $ m$ thuộc đoạn $\left[-2020;2020\right]$ để hàm số $ g(x)=f\left(1-x\right)$ nghịch biến trên khoảng $\left(-\infty ;-1\right)$?
	\choice
	{$ 2016$}
	{$ 2014$}
	{\True $ 2012$}
	{$ 2010$}
	\loigiai{
		Ta có \\
		$g'(x)=f'\left(1-x\right)=-\left(1-x\right)^2\left(-x-1\right)\left[\left(1-x\right)^2-6\left(1-x\right)+m\right]$
		$=\left(x-1\right)^2\left(x+1\right)\left(x^2+4x+m-5\right)$.\\
		Hàm số $ g(x)$ nghịch biến trên khoảng $\left(-\infty ;-1\right)$\\
		$\Leftrightarrow{g}'(x)\le 0,\forall x<-1$ $(*)$, (dấu \lq\lq $=$\rq\rq \, xảy ra tại hữu hạn điểm).\\
		Với $ x<-1$ thì $\left(x-1\right)^2>0$ và $ x+1<0$ nên\\
		$(*)$ $\Leftrightarrow{x^2}+4x+m-5\ge 0,\forall x<-1 \Leftrightarrow m\ge-x^2-4x+5,\forall x<-1$.\\
		Xét hàm số $ y=-x^2-4x+5$ trên khoảng $\left(-\infty ;-1\right)$, ta có bảng biến thiên
		\begin{center}
			\begin{tikzpicture}
				\tkzTabInit[lgt=1.8,espcl=2.3]
				{$x$ /1.2, $y'$ /1.2,$y$ /2}
				{$-\infty$ , $-2$,$-1$}
				\tkzTabLine{,+,0,-,}
				\tkzTabVar{-/$ -\infty $ ,+/$9 $, -/$ 8$}
			\end{tikzpicture}
		\end{center}
		Từ bảng biến thiên suy ra $ m\ge 9$.\\
		Kết hợp với $ m$ thuộc đoạn $\left[-2020;2020\right]$ và $ m$ nguyên nên $ m\in\left\{ 9;10;11;\ldots ;2020\right\}$.\\
		Vậy có $ 2012$ số nguyên $ m$ thỏa mãn đề bài.}
\end{ex}

\begin{ex}%[2D1G1-3]%Câu 10
	\immini{
		Cho hàm số $f(x)$ xác định và liên tục trên $ R$. Hàm số $y=f'(x)$ liên tục trên $\mathbb{R}$ và có đồ thị như hình vẽ bên.
		Xét hàm số $g(x)=f\left(x-2m\right)+\dfrac{1}{2}{\left(2m-x\right)^2}+2020$, với $ m$ là tham số thực. Gọi $ S$ là tập hợp các giá trị nguyên dương của $ m$ để hàm số $ y=g(x)$ nghịch biến trên khoảng $\left(3;4\right)$. Hỏi số phần tử của $ S$ bằng bao nhiêu?
		\choice
		{$4$}
		{\True $2$}
		{$3$}
		{Vô số}}
	{
		\begin{tikzpicture}[scale=0.7,>=stealth, font=\footnotesize, line join=round, line cap=round]
			\def\xmin{-3.5} \def\xmax{4.5}
			\def\ymin{-5.2} \def\ymax{4}
			\clip(\xmin,\ymin) rectangle (\xmax,\ymax);
			\draw[->] (\xmin,0)--(\xmax,0) node [below]{$x$};
			\draw[->] (0,\ymin)--(0,\ymax) node [left]{$y$};
			\node at (0,0) [below left]{$O$};
			\path
			(-3.1,3.7) coordinate (A)
			(-3,3) coordinate (B)
			(0,-2) coordinate (C)
			(0.65,-2) coordinate (D)
			(1,-1) coordinate (E)
			(3,-3) coordinate (F)
			(3.4,-5) coordinate (G);
			\draw[smooth]
			(A)..controls +(-88:0.1) and +(93:.1)..
			(B)..controls +(-87:0.3) and +(-100:8.5)..
			(C)..controls +(75:.8) and +(180:.1)..
			(D)..controls +(0:.1) and +(-105:.3)..
			(E)..controls +(70:2) and +(97:0.4)..
			(F)..controls +(-80:.1) and +(90:0.3)..
			(G);
			\draw[dashed] 
			(-3,0)node[below]{$-3$}|-(0,3)node[right]{$3$}
			(1,0)node[above]{$1$}|-(0,-1)node[left]{$-1$}
			(3,0)node[above]{$3$}|-(0,-3)node[below right]{$-3$};
			\fill 
			(0,-2) circle(1.5pt)
			(-3,3) circle(1.5pt)
			(3,-3) circle(1.5pt)
			(1,-1) circle(1.5pt);
			\node at (2.1,-4) {$y=f'(x)$};
		\end{tikzpicture}
	}
	\loigiai{
		Ta có $g'(x)=f'\left(x-2m\right)-\left(2m-x\right)$.		Đặt $h(x)=f'(x)-\left(-x\right)$.\\
		Từ đồ thị hàm số $y=f'(x)$ và đồ thị hàm số $y=-x$ trên hình vẽ suy ra \\
		$h(x)\le 0\Leftrightarrow f'(x)\le-x\Leftrightarrow\hoac{
			&-3\le x\le 1\\ 
			& x\ge 3.}$ 
		\begin{center}
			\begin{tikzpicture}[scale=0.7,>=stealth, font=\footnotesize, line join=round, line cap=round]
				\def\xmin{-3.5} \def\xmax{4.5}
				\def\ymin{-5.2} \def\ymax{4}
				\clip(\xmin,\ymin) rectangle (\xmax,\ymax);
				\draw[->] (\xmin,0)--(\xmax,0) node [below]{$x$};
				\draw[->] (0,\ymin)--(0,\ymax) node [left]{$y$};
				\node at (0,0) [below left]{$O$};
				\path
				(-3.1,3.7) coordinate (A)
				(-3,3) coordinate (B)
				(0,-2) coordinate (C)
				(0.65,-2) coordinate (D)
				(1,-1) coordinate (E)
				(3,-3) coordinate (F)
				(3.4,-5) coordinate (G);
				\draw[smooth]
				(A)..controls +(-88:0.1) and +(93:.1)..
				(B)..controls +(-87:0.3) and +(-100:8.5)..
				(C)..controls +(75:.8) and +(180:.1)..
				(D)..controls +(0:.1) and +(-105:.3)..
				(E)..controls +(70:2) and +(97:0.4)..
				(F)..controls +(-80:.1) and +(90:0.3)..
				(G);
				\draw[dashed] 
				(-3,0)node[below]{$-3$}|-(0,3)node[right]{$3$}
				(1,0)node[above]{$1$}|-(0,-1)node[left]{$-1$}
				(3,0)node[above]{$3$}|-(0,-3)node[below right]{$-3$};
				\fill 
				(0,-2) circle(1.5pt)
				(-3,3) circle(1.5pt)
				(3,-3) circle(1.5pt)
				(1,-1) circle(1.5pt);
				\draw[smooth,samples=300,domain=-3.2:3.7] plot(\x,{-(\x)});
				\node at (2.1,-4) {$y=f'(x)$};
				\node at (-1,2.1) {$y=h(x)$};
			\end{tikzpicture}
		\end{center}
		Ta có $ g'(x)=h\left(x-2m\right)\le 0\Leftrightarrow\hoac{
			&-3\le x-2m\le 1\\ 
			& x-2m\ge 3}\Leftrightarrow\hoac{
			& 2m-3\le x\le 2m+1\\ 
			& x\ge 2m+3.}$.\\
		Suy ra hàm số $ y=g(x)$ nghịch biến trên các khoảng $\left(2m-3;2m+1\right)$ và $\left(2m+3;+\infty\right)$.\\
		Do đó hàm số $ y=g(x)$ nghịch biến trên khoảng $\left(3;4\right)$ $\Leftrightarrow\hoac{
			&\heva{
				& 2m-3\le 3\\ 
				& 2m+1\ge 4}\\ 
			& 2m+3\le 3}\Leftrightarrow\hoac{
			&\dfrac{3}{2}\le m\le 3\\ 
			& m\le 0.}$ \\
		Mặt khác, do $ m$ nguyên dương nên $ m\in\left\{ 2;3\right\}\Rightarrow S=\left\{ 2;3\right\}$. Vậy số phần tử của $ S$ bằng $2$.\\
	}
	
\end{ex}

\begin{ex}%[2D1G1-3]%Câu 11
	Cho hàm số $f(x)$ có đạo hàm trên $\mathbb{R}$ là $f'(x)=\left(x-1\right)\left(x+3\right)$. Có bao nhiêu giá trị nguyên của tham số $m$ thuộc đoạn $\left[-10;20\right]$ để hàm số $y=f\left(x^2+3x-m\right)$ đồng biến trên khoảng $\left(0;2\right)$?
	\choice
	{\True $ 18$}
	{$ 17$}
	{$ 16$}
	{$ 20$}
	\loigiai{
		Ta có $y'=f'\left(x^2+3x-m\right)=\left(2x+3\right){f}'\left(x^2+3x-m\right)$.\\
		Theo đề bài ta có $f'(x)=\left(x-1\right)\left(x+3\right)$\\
		suy ra $f'(x)>0\Leftrightarrow\hoac{
			& x<-3\\ 
			& x>1}$ và $f'(x)<0\Leftrightarrow-3<x<1$ .\\
		Hàm số đồng biến trên khoảng $\left(0;2\right)$ khi $y'\ge 0,\forall x\in\left(0;2\right)$\\
		$\Leftrightarrow\left(2x+3\right){f}'\left(x^2+3x-m\right)\ge 0,\forall x\in\left(0;2\right)$.\\
		Do $x\in\left(0;2\right)$ nên $2x+3>0,\forall x\in\left(0;2\right)$. Do đó, ta có\\
		$y'\ge 0,\forall x\in\left(0;2\right)\Leftrightarrow f'\left(x^2+3x-m\right)\ge 0$\\
		$\Leftrightarrow\hoac{
			&{x^2}+3x-m\le-3\\ 
			&{x^2}+3x-m\ge 1}\Leftrightarrow\hoac{
			& m\ge{x^2}+3x+3\\ 
			& m\le{x^2}+3x-1}$\\
		$\Leftrightarrow\hoac{
			& m\ge\underset{\left[0;2\right]}{\max}\,\left(x^2+3x+3\right)\\ 
			& m\le\underset{\left[0;2\right]}{\min}\,\left(x^2+3x-1\right)} \Leftrightarrow\hoac{
			& m\ge 13\\ 
			& m\le-1}$.\\
		Do $m\in\left[-10;20\right]$, $ m\in\mathbb{Z}$ nên có $ 18$ giá trị nguyên của $m$ thỏa yêu cầu đề bài.}
\end{ex}

\begin{ex}%[2D1G1-3]%Câu 12
	Cho các hàm số $f(x)=x^3+4x+m$ và $g(x)=\left(x^2+2018\right){\left(x^2+2019\right)^2}{\left(x^2+2020\right)^3}$ . Có bao nhiêu giá trị nguyên của tham số $m\in\left[-2020;2020\right]$ để hàm số $g\left(f(x)\right)$ đồng biến trên $\left(2;+\infty\right)$ ?
	\choice
	{$2005$}
	{\True $2037$}
	{$4016$}
	{$4041$}
	\loigiai{
		Ta có $f(x)=x^3+4x+m$ và \\
		$g(x)=\left(x^2+2018\right){\left(x^2+2019\right)^2}{\left(x^2+2020\right)^3}=a_{12}{x^{12}}+a_{10}{x^{10}}+...+a_2x^2+a_0$.\\
		Suy ra $f'(x)=3x^2+4$ , $g'(x)=12a_{12}{x^{11}}+10a_{10}{x^9}+...+2a_2x$.\\
		Và có 
		\begin{eqnarray*}
			\left[g\left(f(x)\right)\right]' &=& f'(x)\left[12a_{12}{\left(f(x)\right)^{11}}+10a_{10}{\left(f(x)\right)^9}+...+2a_2f(x)\right]\\
			&=& f(x)f'(x)\left(12a_{12}{\left(f(x)\right)^{10}}+10a_{10}{\left(f(x)\right)^8}+...+2a_2\right).
		\end{eqnarray*} 
		Dễ thấy $a_{12};{a_{10}};...;{a_2};{a_0}>0$ và $f'(x)=3x^2+4>0$, $\forall x>2$.\\
		Do đó $f'(x)\left(12a_{12}{\left(f(x)\right)^{10}}+10a_{10}{\left(f(x)\right)^8}+...+2a_2\right)>0$ , $\forall x>2$.\\
		Hàm số $g\left(f(x)\right)$ đồng biến trên $\left(2;+\infty\right)$ khi $\left[g\left(f(x)\right)\right]^{'}\ge 0$, $\forall x>2$\\
		$\Rightarrow  f(x)\ge 0$, $\forall x>2 \Leftrightarrow x^3+4x+m\ge 0$, $\forall x>3 \Leftrightarrow  m\ge-x^3-4x$, $\forall x>2$\\
		$ \Rightarrow  m\ge\underset{\left[2;+\infty\right)}{\max}\,\left(-x^3-4x\right)=-16$.\\
		Vì $m\in\left[-2020;2020\right]$ và $m\in\mathbb{Z}$ nên có $2037$ giá trị thỏa mãn $m$ .}
\end{ex}

\begin{ex}%[2D1G1-3]%Câu 13
	Cho hàm số $y=f(x)$ có đạo hàm $f'(x)=x{\left(x+1\right)^2}\left(x^2+2mx+1\right)$ với mọi $x \in \mathbb{R}$. Có bao nhiêu số nguyên âm $m$ để hàm số $g(x)=f\left(2x+1\right)$ đồng biến trên khoảng $\left(3;5\right)$?
	\choice
	{\True $3$}
	{$2$}
	{$4$}
	{$6$}
	\loigiai{
		Ta có $g'(x)=2f'(2x+1)=2(2x+1)(2x+2)^2[(2x+1)^2+2m(2x+1)+1]$. 	Đặt $t=2x+1$\\
		Để hàm số $g(x)$ đồng biến trên khoảng $\left(3;5\right)$ khi và chỉ khi 
		\begin{eqnarray*}
			& & g'(x)\ge 0,\forall x\in\left(3;5\right) \\
			& \Leftrightarrow & t(t^2+2mt+1)\ge 0,\forall t\in\left(7;11\right)\Leftrightarrow{t^2}+2mt+1\ge 0,\,\,\forall t\in\left(7;11\right) \\
			&\Leftrightarrow & 2m\ge\dfrac{-t^2-1}{t},\,\,\,\forall t\in\left(7;11\right)
		\end{eqnarray*}	
		Xét hàm số $h(t)=\dfrac{-t^2-1}{t}$ trên $\left[7;11\right]$, có $h'(t)=\dfrac{-t^2+1}{t^2}$\\
		Bảng biến thiên
		\begin{center}
			\begin{tikzpicture}
				\tkzTabInit[espcl=3,lgt=1.2,nocadre]
				{$t$/0.7,$h'(t)$/0.7,$h(t)$/2.5}
				{$-\infty$,$1$,$11$,$+\infty$}
				\tkzTabLine{, ,,-,,,}
				%	\node (0) at ($(N12)+(0,-3)$) {$-\infty$};
				\node (1) at ($(N22)+(0,-0.8)$) [right] {$-\dfrac{50}{7}$};
				\node (2) at ($(N32)+(0,-2.5)$) [left] {$-\dfrac{122}{11}$};
				
				
				%				\node (3) at ($(N11+(-0.5,0))$) {};
				%				\node (4) at ($(N23)$) {};
				\fill[pattern=north east lines] (7.0,-0.7) rectangle (10,-4.4);
				\fill[pattern=north east lines] (1.5,-0.7) rectangle (4.5,-4.4);
				\draw[->] (1)--(2);	
				\draw[dashed] (4.5,-0.7)--(4.5,-4.4);
				\draw[dashed] (7.0,-0.7)--(7.0,-4.4);	
			\end{tikzpicture}		
		\end{center}
		Dựa vào BBT ta có $2m\ge\dfrac{-t^2-1}{t},\,\,\,\forall t\in\left(7;11\right)\Leftrightarrow 2m\ge\underset{\left[7;11\right]}{\max}\,h(t)\Leftrightarrow m\ge-\dfrac{50}{14}$\\
		Vì $ m\in{\mathbb{Z}^-}\Rightarrow m \in \{-3;-2;-1\}$ .
	}
\end{ex}

\begin{ex}%[2D1G1-3]%Câu 14
	Cho hàm số $y=f(x)$ có bảng biến thiên như sau\\
	\begin{center}
		\begin{tikzpicture}[>=stealth,scale = 1]
			\tkzTabInit[lgt=1,espcl=2.5,nocadre]
			{$x$ /0.7, $y'$ /0.7,$y$ /2.5}
			{$-\infty$,$0$,$2$,$+\infty$}
			\tkzTabLine{ ,-,0,+,0,-,}
			\tkzTabVar{-/$-\infty$, +/$4$,- /$0$, +/{ $+\infty$}}
		\end{tikzpicture}
	\end{center}
	Có bao nhiêu số nguyên $m<2019$ để hàm số $g(x)=f\left(x^2-2x+m\right)$ đồng biến trên khoảng $\left(1;+\infty\right)$?
	\choice
	{\True $2016$}
	{$2015$}
	{$2017$}
	{$2018$}
	\loigiai{
		Ta có $g'(x)=\left(x^2-2x+m\right)'{f}'\left(x^2-2x+m\right)=2\left(x-1\right){f}'\left(x^2-2x+m\right)$ .\\
		Hàm số $y=g(x)$ đồng biến trên khoảng $\left(1;+\infty\right)$ khi và chỉ khi $g'(x)\ge 0,\forall x\in\left(1;+\infty\right)$ và\\
		$g'(x)=0$ tại hữu hạn điểm \\
		$\Leftrightarrow 2\left(x-1\right){f}'\left(x^2-2x+m\right)\ge 0,\forall x\in\left(1;+\infty\right)$\\
		$\Leftrightarrow{f}'\left(x^2-2x+m\right)\ge 0,\forall x\in\left(1;+\infty\right)$ $\Leftrightarrow\hoac{
			&{x^2}-2x+m\ge 2,\forall x\in\left(1;+\infty\right)\\ 
			&{x^2}-2x+m\le 0,\forall x\in\left(1;+\infty\right).}$\\
		Xét hàm số $y=x^2-2x+m$, ta có bảng biến thiên
		\begin{center}
			\begin{tikzpicture}[>=stealth,scale = 1]
				\tkzTabInit[lgt=1,espcl=2.5,nocadre]
				{$x$ /0.7, $y'$ /0.7,$y$ /2.5}
				{$-\infty$,$2$,$+\infty$}
				\tkzTabLine{ ,-,0,+,}
				\tkzTabVar{+/$+\infty$, -/$m-1$, +/{$+\infty$}}
			\end{tikzpicture}
		\end{center}
		Dựa vào bảng biến thiên ta có\\
		TH1: $x^2-2x+m\ge 2,\forall x\in\left(1;+\infty\right)\Leftrightarrow m-1\ge 2\Leftrightarrow m\ge 3$ .\\
		TH2: $x^2-2x+m\le 0,\forall x\in\left(1;+\infty\right)$. Không có giá trị $m$ thỏa mãn.\\
		Vậy có $2016$ số nguyên $m<2019$ thỏa mãn yêu cầu bài toán.}
\end{ex}

\begin{ex}%[2D1G1-3]%Câu 15
	\immini{
		Cho hàm số $ y=f(x)$ có đạo hàm là hàm số $f'(x)$ trên $\mathbb{R}$. Biết rằng hàm số $ y=f'\left(x-2\right)+2$ có đồ thị như hình vẽ bên dưới. Hàm số $ f(x)$ đồng biến trên khoảng nào?
		\choice
		{$\left(-\infty ;3\right),\,\,\left(5;+\infty\right)$}
		{\True $\left(-\infty ;-1\right),\,\,\left(1;+\infty\right)$}
		{$\left(-1;1\right)$}
		{$\left(3;5\right)$}}{
		\begin{tikzpicture}[scale=0.7,font=\footnotesize, line join=round, line cap=round, >=stealth] %Đường cong bậc 3
			\draw[thick, ->] (-0.5,0)--(3.5,0);
			\draw[thick, ->] (0,-1.8)--(0,5.3);
			\draw (3.7,0) node[below] {$x$};
			\draw (0,5.4) node[left]{$y$};
			\draw (0,0) node[below left]{$0$};
			\draw[fill] (3,0) circle (0.5pt)node[below]{$ 3$};
			\draw[fill] (1,0) circle (0.5pt)node[below]{$ 1$};
			\draw[fill] (2,0) circle (0.5pt)node[above]{$2$};
			\draw[fill] (0,2) circle (0.5pt)node[left]{$ 2$};
			\draw[fill] (0,-1) circle (0.5pt)node[left]{$ -1$};
			%		\draw[fill] (0,2) circle (0.5pt)node[above left]{$ 2$};
			%		\draw[fill] (0,-2) circle (0.5pt)node[below left]{$ -2$};
			\draw[dashed] (3,0)--(3,2)--(0,2)--(1,2)--(1,0); 
			\draw[dashed](0,-1)--(2,-1)--(2,0);
			\draw[line width=1.2pt,smooth,samples=100,domain=0.6:3.4] plot(\x,{3*(\x)^2-12*(\x)+11});		
			%\draw[line width=1.2pt,smooth,samples=100,domain=-3.3:2.8] plot(\x,{0.75*(\x)^2+0.5*\x-1});
			%	\draw (2.0,2.8) node[left]{$y=f'(x)$};
	\end{tikzpicture}	}
	\loigiai{	
		Hàm số $ y=f'\left(x-2\right)+2$ có đồ thị $(C)$ như sau:\\
		\begin{center}
			\begin{tikzpicture}[scale=0.7,font=\footnotesize, line join=round, line cap=round, >=stealth] %Đường cong bậc 3
				\draw[thick, ->] (-0.5,0)--(3.5,0);
				\draw[thick, ->] (0,-1.8)--(0,5.3);
				\draw (3.7,0) node[below] {$x$};
				\draw (0,5.4) node[left]{$y$};
				\draw (0,0) node[below left]{$0$};
				\draw[fill] (3,0) circle (0.5pt)node[below]{$ 3$};
				\draw[fill] (1,0) circle (0.5pt)node[below]{$ 1$};
				\draw[fill] (2,0) circle (0.5pt)node[above]{$2$};
				\draw[fill] (0,2) circle (0.5pt)node[left]{$ 2$};
				\draw[fill] (0,-1) circle (0.5pt)node[left]{$ -1$};
				%		\draw[fill] (0,2) circle (0.5pt)node[above left]{$ 2$};
				%		\draw[fill] (0,-2) circle (0.5pt)node[below left]{$ -2$};
				\draw[dashed] (3,0)--(3,2)--(0,2)--(1,2)--(1,0); 
				\draw[dashed](0,-1)--(2,-1)--(2,0);
				\draw[line width=1.2pt,smooth,samples=100,domain=0.6:3.4] plot(\x,{3*(\x)^2-12*(\x)+11});		
				%\draw[line width=1.2pt,smooth,samples=100,domain=-3.3:2.8] plot(\x,{0.75*(\x)^2+0.5*\x-1});
				%	\draw (2.0,2.8) node[left]{$y=f'(x)$};
			\end{tikzpicture}
		\end{center}
		Dựa vào đồ thị $(C)$ ta có\\ $f'\left(x-2\right)+2>2,\forall x\in\left(-\infty ;1\right)\cup\left(3;+\infty\right)\Leftrightarrow{f}'\left(x-2\right)>0,\forall x\in\left(-\infty ;1\right)\cup\left(3;+\infty\right)$ .\\
		Đặt $ x*=x-2$ suy ra $f'\left(x*\right)>0,\forall x*\in\left(-\infty ;-1\right)\bigcup\left(1;+\infty\right)$.\\
		Vậy hàm số $ f(x)$ đồng biến trên khoảng $\left(-\infty ;-1\right),\,\,\left(1;+\infty\right)$.}
\end{ex}

\begin{ex}%[2D1G1-2]%Câu 16
	\immini{
		Cho hàm số $ y=f(x)$ có đạo hàm là hàm số $f'(x)$ trên $\mathbb{R}$. Biết rằng hàm số $ y=f'\left(x+2\right)-2$ có đồ thị như hình vẽ bên dưới. Hàm số $ f(x)$ nghịch biến trên khoảng nào?
		\choice
		{$\left(-3;-1\right),\,\,\left(1;3\right)$}
		{\True $\left(-1;1\right),\,\,\left(3;5\right)$}
		{$\left(-\infty ;-2\right),\,\,\left(0;2\right)$}
		{$\left(-5;-3\right),\,\,\left(-1;1\right)$}}{
		\begin{tikzpicture}[scale=0.7,font=\footnotesize, line join=round, line cap=round, >=stealth] %Đường cong bậc 3
			\draw[thick, ->] (-3.8,0)--(4.0,0);
			\draw[thick, ->] (0,-4.8)--(0,3.5);
			\draw (4.2,0) node[below] {$x$};
			\draw (0,3.7) node[left]{$y$};
			\draw (0,0) node[below left]{$0$};
			\draw[fill] (-3,0) circle (0.5pt)node[above]{$ -3$};
			\draw[fill] (-1,0) circle (0.5pt)node[above]{$ -1$};
			\draw[fill] (1,0) circle (0.5pt)node[above]{$ 1$};
			\draw[fill] (3,0) circle (0.5pt)node[above]{$3$};
			\draw[fill] (0,2) circle (0.5pt)node[above left]{$ 2$};
			\draw[fill] (0,-1) circle (0.5pt)node[above right]{$ -1$};
			%		\draw[fill] (0,2) circle (0.5pt)node[above left]{$ 2$};
			%		\draw[fill] (0,-2) circle (0.5pt)node[below left]{$ -2$};
			\draw[dashed] (-3,0)--(-3,-2)--(3,-2)--(3,0) (-1,0)--(-1,-2) (1,0)--(1,-2) (-3.494,0)--(-3.494,2)--(3.494,2)--(3.494,0); 
			\draw[line width=1.2pt,smooth,samples=100,domain=-3.6:3.6] plot(\x,{0.11*(\x)^4-1.11*(\x)^2-1});		
			%\draw[line width=1.2pt,smooth,samples=100,domain=-3.3:2.8] plot(\x,{0.75*(\x)^2+0.5*\x-1});
			%	\draw (2.0,2.8) node[left]{$y=f'(x)$};
	\end{tikzpicture}	}
	\loigiai{
		Hàm số $ y=f'\left(x+2\right)-2$ có đồ thị $(C)$ như sau
		\begin{center}
			\begin{tikzpicture}[scale=0.7,font=\footnotesize, line join=round, line cap=round, >=stealth] %Đường cong bậc 3
				\draw[thick, ->] (-3.8,0)--(4.0,0);
				\draw[thick, ->] (0,-4.8)--(0,3.5);
				\draw (4.2,0) node[below] {$x$};
				\draw (0,3.7) node[left]{$y$};
				\draw (0,0) node[below left]{$0$};
				\draw[fill] (-3,0) circle (0.5pt)node[above]{$ -3$};
				\draw[fill] (-1,0) circle (0.5pt)node[above]{$ -1$};
				\draw[fill] (1,0) circle (0.5pt)node[above]{$ 1$};
				\draw[fill] (3,0) circle (0.5pt)node[above]{$3$};
				\draw[fill] (0,2) circle (0.5pt)node[above left]{$ 2$};
				\draw[fill] (0,-1) circle (0.5pt)node[above right]{$ -1$};
				%		\draw[fill] (0,2) circle (0.5pt)node[above left]{$ 2$};
				%		\draw[fill] (0,-2) circle (0.5pt)node[below left]{$ -2$};
				\draw[dashed] (-3,0)--(-3,-2)--(3,-2)--(3,0) (-1,0)--(-1,-2) (1,0)--(1,-2) (-3.494,0)--(-3.494,2)--(3.494,2)--(3.494,0); 
				\draw[line width=1.2pt,smooth,samples=100,domain=-3.6:3.6] plot(\x,{0.11*(\x)^4-1.11*(\x)^2-1});		
				%\draw[line width=1.2pt,smooth,samples=100,domain=-3.3:2.8] plot(\x,{0.75*(\x)^2+0.5*\x-1});
				%	\draw (2.0,2.8) node[left]{$y=f'(x)$};
			\end{tikzpicture}
		\end{center}
		Dựa vào đồ thị $(C)$ ta có\\
		$f'\left(x+2\right)-2<-2,\forall x\in\left(-3;-1\right)\bigcup\left(1;3\right)\Leftrightarrow{f}'\left(x+2\right)<0,\forall x\in\left(-3;-1\right)\bigcup\left(1;3\right)$.\\
		Đặt $ x^*=x+2$ suy ra: $f'\left(x^*\right)<0,\forall x^*\in\left(-1;1\right)\bigcup\left(3;5\right)$.\\
		Vậy: Hàm số $ f(x)$ đồng biến trên khoảng $\left(-1;1\right),\,\,\left(3;5\right)$.}
\end{ex}

\begin{ex}%[2D1G1-2]%Câu 17
	\immini{
		Cho hàm số $ y=f(x)$ có đạo hàm là hàm số $f'(x)$ trên $\mathbb{R}$. Biết rằng hàm số $ y=f'\left(x-2\right)+2$ có đồ thị như hình vẽ bên dưới. Hàm số $ f(x)$ nghịch biến trên khoảng nào?
		\choice
		{$\left(-\infty ;2\right)$}
		{\True $\left(-1;1\right)$}
		{$\left(\dfrac{3}{2};\dfrac{5}{2}\right)$}
		{$\left(2;+\infty\right)$}}{
		\begin{tikzpicture}[scale=0.7,font=\footnotesize, line join=round, line cap=round, >=stealth] %Đường cong bậc 3
			\draw[thick, ->] (-0.5,0)--(3.5,0);
			\draw[thick, ->] (0,-1.8)--(0,5.3);
			\draw (3.7,0) node[below] {$x$};
			\draw (0,5.4) node[left]{$y$};
			\draw (0,0) node[below left]{$0$};
			\draw[fill] (3,0) circle (0.5pt)node[below]{$ 3$};
			\draw[fill] (1,0) circle (0.5pt)node[below]{$ 1$};
			\draw[fill] (2,0) circle (0.5pt)node[above]{$2$};
			\draw[fill] (0,2) circle (0.5pt)node[left]{$ 2$};
			\draw[fill] (0,-1) circle (0.5pt)node[left]{$ -1$};
			%		\draw[fill] (0,2) circle (0.5pt)node[above left]{$ 2$};
			%		\draw[fill] (0,-2) circle (0.5pt)node[below left]{$ -2$};
			\draw[dashed] (3,0)--(3,2)--(0,2)--(1,2)--(1,0); 
			\draw[dashed](0,-1)--(2,-1)--(2,0);
			\draw[line width=1.2pt,smooth,samples=100,domain=0.6:3.4] plot(\x,{3*(\x)^2-12*(\x)+11});		
			%\draw[line width=1.2pt,smooth,samples=100,domain=-3.3:2.8] plot(\x,{0.75*(\x)^2+0.5*\x-1});
			%	\draw (2.0,2.8) node[left]{$y=f'(x)$};
	\end{tikzpicture}	}
	\loigiai{
		Hàm số $ y=f'\left(x-2\right)+2$ có đồ thị $(C)$ như sau
		\begin{center}
			\begin{tikzpicture}[scale=0.7,font=\footnotesize, line join=round, line cap=round, >=stealth] %Đường cong bậc 3
				\draw[thick, ->] (-0.5,0)--(3.5,0);
				\draw[thick, ->] (0,-1.8)--(0,5.3);
				\draw (3.7,0) node[below] {$x$};
				\draw (0,5.4) node[left]{$y$};
				\draw (0,0) node[below left]{$0$};
				\draw[fill] (3,0) circle (0.5pt)node[below]{$ 3$};
				\draw[fill] (1,0) circle (0.5pt)node[below]{$ 1$};
				\draw[fill] (2,0) circle (0.5pt)node[above]{$2$};
				\draw[fill] (0,2) circle (0.5pt)node[left]{$ 2$};
				\draw[fill] (0,-1) circle (0.5pt)node[left]{$ -1$};
				%		\draw[fill] (0,2) circle (0.5pt)node[above left]{$ 2$};
				%		\draw[fill] (0,-2) circle (0.5pt)node[below left]{$ -2$};
				\draw[dashed] (3,0)--(3,2)--(0,2)--(1,2)--(1,0); 
				\draw[dashed](0,-1)--(2,-1)--(2,0);
				\draw[line width=1.2pt,smooth,samples=100,domain=0.6:3.4] plot(\x,{3*(\x)^2-12*(\x)+11});		
				%\draw[line width=1.2pt,smooth,samples=100,domain=-3.3:2.8] plot(\x,{0.75*(\x)^2+0.5*\x-1});
				%	\draw (2.0,2.8) node[left]{$y=f'(x)$};
			\end{tikzpicture}
		\end{center}
		Dựa vào đồ thị $(C)$ ta có\\
		$f'\left(x-2\right)+2<2,\forall x\in\left(1;3\right)\Leftrightarrow{f}'\left(x-2\right)<0,\forall x\in\left(1;3\right)$.\\
		Đặt $ x^*=x-2$ thì $f'\left(x^*\right)<0,\forall x^*\in\left(-1;1\right)$.\\
		Vậy: Hàm số $ f(x)$ nghịch biến trên khoảng $\left(-1;1\right)$.\\
		Cách khác:\\
		Tịnh tiến sang trái hai đơn vị và xuống dưới $2$ đơn vị thì từ đồ thị $(C)$ sẽ thành đồ thị của hàm$ y=f'(x)$. Khi đó $f'(x)<0,\forall x\in\left(-1;1\right)$.\\
		Vậy hàm số $ f(x)$ nghịch biến trên khoảng $\left(-1;1\right)$.}
\end{ex}

\begin{ex}%[2D1G1-2]%Câu 18
	Cho hàm số $y=f(x)$ có đạo hàm cấp $ 3$ liên tục trên $\mathbb{R}$ và thỏa mãn $f(x)\cdot f'''(x)=x{\left(x-1\right)^2}{\left(x+4\right)^3}$ với mọi $x\in\mathbb{R}$ và $g(x)=\left[f'(x)\right]^2-2f(x)\cdot f''(x)$. Hàm số $h(x)=g\left(x^2-2x\right)$ đồng biến trên khoảng nào dưới đây?
	\choice
	{$\left(-\infty ;1\right)$}
	{$\left(2;+\infty\right)$}
	{$\left(0;1\right)$}
	{\True $\left(1;2\right)$}
	\loigiai{		
		Ta có $g'(x)=2f''(x){f}'(x)-2f'(x)\cdot f''(x)-2f(x)\cdot f'''(x)=-2f(x)\cdot f'''(x);$\\
		Khi đó $\left(h(x)\right)'=\left(2x-2\right){g}'\left(x^2-2x\right)=-2\left(2x-2\right)\left(x^2-2x\right){\left(x^2-2x-1\right)^2}{\left(x^2-2x+4\right)^3}$\\
		$h'(x)=0\Leftrightarrow\hoac{
			& x=0\\ 
			& x=1\\ 
			& x=2\\ 
			& x=1\pm\sqrt{2}.}$ 
		Ta có bảng xét dấu của $h'(x)$
		\begin{center}
			\begin{tikzpicture}
				\tkzTabInit[lgt=1.2,espcl=2,nocadre]
				{$t$/0.7, $h'(x)$ /.7} % first column
				{$-\infty$, $1-\sqrt{2}$,$0$, $1$,$2$,$1+\sqrt{2}$, $+\infty$} % first row
				\tkzTabLine { ,+,0,-,0,+,0,-,0,+,0,- } % second row
				%				\tkzTabLine {,-,z,+,t,+,} % third row
				%				\tkzTabLine {,+,d,-,z,+,} % last row
			\end{tikzpicture}
		\end{center}
		Suy ra hàm số $h(x)=g\left(x^2-2x\right)$ đồng biến trên khoảng $\left(1;2\right)$.}
\end{ex}

\begin{ex}%[2D1G1-2]%Câu 19
	Cho hàm số $ y=f(x)$ xác định trên $\mathbb{R}$. Hàm số $ y=g(x)=f'\left(2x+3\right)+2$ có đồ thị là một parabol với tọa độ đỉnh $ I\left(2;-1\right)$ và đi qua điểm $ A\left(1;2\right)$. Hỏi hàm số $ y=f(x)$ nghịch biến trên khoảng nào dưới đây?
	\choice
	{\True $\left(5;9\right)$}
	{$\left(1;2\right)$}
	{$\left(-\infty ;9\right)$}
	{$\left(1;3\right)$}
	\loigiai{	
		Xét hàm số $ g(x)=f'\left(2x+3\right)+2$ có đồ thị là một Parabol nên có phương trình dạng $ y=g(x)=a{x^2}+bx+c\,\,\,\,(P)$.\\
		Vì $(P)$ có đỉnh $ I\left(2;-1\right)$ nên $\heva{
			&\dfrac{-b}{2a}=2\\ 
			& g(2)=-1} \Leftrightarrow\heva{
			&-b=4a\\ 
			& 4a+2b+c=-1} \Leftrightarrow\heva{
			& 4a+b=0\\ 
			& 4a+2b+c=-1}$.\\
		Vì $(P)$ đi qua điểm $ A\left(1;2\right)$ nên $ g(1)=2\Leftrightarrow a+b+c=2$.\\
		Ta có hệ phương trình $\heva{
			& 4a+b=0\\ 
			& 4a+2b+c=-1\\ 
			& a+b+c=2} \Leftrightarrow\heva{
			& a=3\\ 
			& b=-12\\ 
			& c=11}$ nên $ g(x)=3x^2-12x+11$.\\
		Đồ thị của hàm $ y=g(x)$ là
		\begin{center}
			\begin{tikzpicture}[scale=0.7,font=\footnotesize, line join=round, line cap=round, >=stealth] %Đường cong bậc 3
				\draw[thick, ->] (-0.5,0)--(3.5,0);
				\draw[thick, ->] (0,-1.8)--(0,5.3);
				\draw (3.7,0) node[below] {$x$};
				\draw (0,5.4) node[left]{$y$};
				\draw (0,0) node[below left]{$0$};
				\draw[fill] (3,0) circle (0.5pt)node[below]{$ 3$};
				\draw[fill] (1,0) circle (0.5pt)node[below]{$ 1$};
				\draw[fill] (2,0) circle (0.5pt)node[above]{$2$};
				\draw[fill] (0,2) circle (0.5pt)node[left]{$ 2$};
				\draw[fill] (0,-1) circle (0.5pt)node[left]{$ -1$};
				%		\draw[fill] (0,2) circle (0.5pt)node[above left]{$ 2$};
				%		\draw[fill] (0,-2) circle (0.5pt)node[below left]{$ -2$};
				\draw[dashed] (3,0)--(3,2)--(0,2)--(1,2)--(1,0) (3.2,2)--(3,2); 
				\draw[dashed](0,-1)--(2,-1)--(2,0);
				\draw[line width=1.2pt,smooth,samples=100,domain=0.6:3.4] plot(\x,{3*(\x)^2-12*(\x)+11});		
				%\draw[line width=1.2pt,smooth,samples=100,domain=-3.3:2.8] plot(\x,{0.75*(\x)^2+0.5*\x-1});
				%	\draw (2.0,2.8) node[left]{$y=f'(x)$};
			\end{tikzpicture}	
		\end{center}
		Theo đồ thị ta thấy $ f'(2x+3)\le 0\Leftrightarrow f'(2x+3)+2\le 2\Leftrightarrow 1\le x\le 3$.\\
		Đặt $ t=2x+3\Leftrightarrow x=\dfrac{t-3}{2}$ khi đó $ f'(t)\le 0\Leftrightarrow 1\le\dfrac{t-3}{2}\le 3\Leftrightarrow 5\le t\le 9$.\\
		Vậy $ y=f(x)$ nghịch biến trên khoảng $\left(5;9\right)$.}
\end{ex}

\begin{ex}%[2D1G1-2]%Câu 20
	\immini{
		Cho hàm số $ y=f(x)$, hàm số $f'(x)=x^3+a{x^2}+bx+c\left(a,b,c\in\mathbb{R}\right)$ có đồ thị như hình vẽ bên.
		Hàm số $ g(x)=f\left(f'(x)\right)$ nghịch biến trên khoảng nào dưới đây?
		\choice
		{$\left(1;+\infty\right)$}
		{\True $\left(-\infty ;-2\right)$}
		{$\left(-1;0\right)$}
		{$\left(-\dfrac{\sqrt{3}}{3};\dfrac{\sqrt{3}}{3}\right)$}}{
		\begin{tikzpicture}[scale=0.8,font=\footnotesize, line join=round, line cap=round, >=stealth] %Đường cong bậc 3
			\draw[thick, ->] (-1.7,0)--(1.7,0);
			\draw[thick, ->] (0,-2.7)--(0,3.0);
			\draw (1.9,0) node[below] {$x$};
			\draw (0,3.2) node[left]{$y$};
			\draw (0,0) node[below left]{$0$};
			\draw[fill] (-1,0) circle (0.5pt)node[above left]{$ -1 $};
			\draw[fill] (1,0) circle (0.5pt)node[below right]{$ 1$};
			\draw[line width=1.2pt,smooth,samples=100,domain=-1.3:1.3] plot(\x,{2.667*(\x)^3+0*(\x)^2-2.667*\x});		
			%\draw[line width=1.2pt,smooth,samples=100,domain=-3.3:2.8] plot(\x,{0.75*(\x)^2+0.5*\x-1});
		\end{tikzpicture}	
	}
	\loigiai{	
		Vì các điểm $\left(-1;0\right),\left(0;0\right),\left(1;0\right)$ thuộc đồ thị hàm số $ y=f'(x)$ nên ta có hệ\\
		$\heva{
			&-1+a-b+c=0\\ 
			& c=0\\ 
			& 1+a+b+c=0} \Leftrightarrow\heva{
			& a=0\\ 
			& b=-1\\ 
			& c=0} \Rightarrow {f}'(x)=x^3-x\Rightarrow f''(x)=3x^2-1$.\\
		Ta có $ g(x)=f\left(f'(x)\right)\Rightarrow{g}'(x)=f'\left(f'(x)\right)\cdot f''(x)$.\\
		Xét \\
		$g'(x)=0\Leftrightarrow{g}'(x)=f'\left(f'(x)\right)\cdot f''(x)=0$\\
		$\Leftrightarrow {f}'\left(x^3-x\right)\left(3x^2-1\right)=0\Leftrightarrow\hoac{
			&{x^3}-x=0\\ 
			&{x^3}-x=1\\ 
			&{x^3}-x=-1\\ 
			& 3x^2-1=0} \Leftrightarrow \hoac{
			& x=\pm 1\\ 
			& x=0\\ 
			& x=x_1(x_1\approx 1,325)\\ 
			& x=x_2(x_2\approx-1,325)\\ 
			& x=\pm\dfrac{\sqrt{3}}{3}.}$\\
		Bảng biến thiên
		\begin{center}
			\begin{tikzpicture}
				\tkzTabInit[lgt=1.2,espcl=2,nocadre]
				{$t$/0.7, $h'(x)$ /.7} % first column
				{$-\infty$, $-1{,}325$,$-1$, $-\dfrac{\sqrt{3}}{3}$,$0$,$\dfrac{\sqrt{3}}{3}$,$1$,$1{,}325$, $+\infty$} % first row
				\tkzTabLine { ,-,0,+,0,-,0,+,0,-,0,+,0,-,0,+, } % second row
				%				\tkzTabLine {,-,z,+,t,+,} % third row
				%				\tkzTabLine {,+,d,-,z,+,} % last row
			\end{tikzpicture}
		\end{center}
		Dựa vào bảng biến thiên ta có $ g(x)$ nghịch biến trên $\left(-\infty ;-2\right)$}
\end{ex}
\Closesolutionfile{ans}
\indapan{10}{ans/CD1/Muc_9_10}
\chapter{CỰC TRỊ CỦA HÀM SỐ}
\begin{Solution}{1}
D
\end{Solution}
\begin{Solution}{2}
C
\end{Solution}
\begin{Solution}{3}
C
\end{Solution}
\begin{Solution}{4}
A
\end{Solution}
\begin{Solution}{5}
B
\end{Solution}
\begin{Solution}{6}
D
\end{Solution}
\begin{Solution}{7}
C
\end{Solution}
\begin{Solution}{8}
D
\end{Solution}
\begin{Solution}{9}
A
\end{Solution}
\begin{Solution}{10}
B
\end{Solution}
\begin{Solution}{11}
D
\end{Solution}
\begin{Solution}{12}
A
\end{Solution}
\begin{Solution}{13}
D
\end{Solution}
\begin{Solution}{14}
B
\end{Solution}
\begin{Solution}{15}
B
\end{Solution}
\begin{Solution}{16}
C
\end{Solution}
\begin{Solution}{1}
A
\end{Solution}
\begin{Solution}{2}
B
\end{Solution}
\begin{Solution}{3}
D
\end{Solution}
\begin{Solution}{4}
D
\end{Solution}
\begin{Solution}{5}
C
\end{Solution}
\begin{Solution}{6}
A
\end{Solution}
\begin{Solution}{7}
D
\end{Solution}
\begin{Solution}{8}
B
\end{Solution}
\begin{Solution}{9}
C
\end{Solution}
\begin{Solution}{10}
C
\end{Solution}
\begin{Solution}{1}
D
\end{Solution}
\begin{Solution}{2}
D
\end{Solution}
\begin{Solution}{3}
B
\end{Solution}
\begin{Solution}{4}
C
\end{Solution}
\begin{Solution}{5}
D
\end{Solution}
\begin{Solution}{6}
A
\end{Solution}
\begin{Solution}{7}
C
\end{Solution}
\begin{Solution}{8}
B
\end{Solution}
\begin{Solution}{9}
A
\end{Solution}
\begin{Solution}{10}
C
\end{Solution}
\begin{Solution}{11}
D
\end{Solution}
\begin{Solution}{12}
C
\end{Solution}
\begin{Solution}{13}
A
\end{Solution}
\begin{Solution}{14}
D
\end{Solution}
\begin{Solution}{15}
A
\end{Solution}
\begin{Solution}{16}
A
\end{Solution}
\begin{Solution}{17}
B
\end{Solution}
\begin{Solution}{18}
C
\end{Solution}
\begin{Solution}{19}
C
\end{Solution}
\begin{Solution}{20}
A
\end{Solution}
\begin{Solution}{21}
D
\end{Solution}
\begin{Solution}{22}
C
\end{Solution}
\begin{Solution}{23}
A
\end{Solution}
\begin{Solution}{24}
C
\end{Solution}
\begin{Solution}{25}
A
\end{Solution}
\begin{Solution}{26}
B
\end{Solution}
\begin{Solution}{27}
B
\end{Solution}
\begin{Solution}{28}
D
\end{Solution}
\begin{Solution}{29}
B
\end{Solution}
\begin{Solution}{30}
D
\end{Solution}
\begin{Solution}{31}
D
\end{Solution}
\begin{Solution}{32}
C
\end{Solution}
\begin{Solution}{33}
D
\end{Solution}
\begin{Solution}{34}
C
\end{Solution}
\begin{Solution}{35}
D
\end{Solution}
\begin{Solution}{36}
D
\end{Solution}
\begin{Solution}{37}
D
\end{Solution}
\begin{Solution}{38}
D
\end{Solution}
\begin{Solution}{39}
D
\end{Solution}
\begin{Solution}{40}
C
\end{Solution}
\begin{Solution}{41}
A
\end{Solution}
\begin{Solution}{1}
A
\end{Solution}
\begin{Solution}{2}
B
\end{Solution}
\begin{Solution}{3}
C
\end{Solution}
\begin{Solution}{4}
A
\end{Solution}
\begin{Solution}{5}
A
\end{Solution}
\begin{Solution}{6}
C
\end{Solution}
\begin{Solution}{7}
C
\end{Solution}
\begin{Solution}{8}
B
\end{Solution}
\begin{Solution}{9}
C
\end{Solution}
\begin{Solution}{10}
B
\end{Solution}
\begin{Solution}{11}
A
\end{Solution}
\begin{Solution}{12}
B
\end{Solution}
\begin{Solution}{13}
B
\end{Solution}
\begin{Solution}{14}
B
\end{Solution}
\begin{Solution}{15}
A
\end{Solution}
\begin{Solution}{16}
B
\end{Solution}
\begin{Solution}{17}
A
\end{Solution}
\begin{Solution}{18}
D
\end{Solution}
\begin{Solution}{19}
C
\end{Solution}
\begin{Solution}{20}
C
\end{Solution}
\begin{Solution}{21}
A
\end{Solution}
\begin{Solution}{22}
C
\end{Solution}
\begin{Solution}{23}
C
\end{Solution}
\begin{Solution}{24}
A
\end{Solution}
\begin{Solution}{25}
B
\end{Solution}
\begin{Solution}{26}
B
\end{Solution}
\begin{Solution}{27}
A
\end{Solution}
\begin{Solution}{28}
A
\end{Solution}
\begin{Solution}{29}
C
\end{Solution}
\begin{Solution}{30}
B
\end{Solution}
\begin{Solution}{31}
A
\end{Solution}
\begin{Solution}{32}
C
\end{Solution}
\begin{Solution}{33}
B
\end{Solution}
\begin{Solution}{34}
A
\end{Solution}
\begin{Solution}{35}
B
\end{Solution}
\begin{Solution}{36}
B
\end{Solution}
\begin{Solution}{37}
B
\end{Solution}
\begin{Solution}{38}
D
\end{Solution}
\begin{Solution}{39}
B
\end{Solution}
\begin{Solution}{40}
A
\end{Solution}
\begin{Solution}{41}
D
\end{Solution}
\begin{Solution}{42}
D
\end{Solution}
\begin{Solution}{43}
A
\end{Solution}
\begin{Solution}{44}
D
\end{Solution}
\begin{Solution}{45}
C
\end{Solution}
\begin{Solution}{46}
B
\end{Solution}
\begin{Solution}{47}
A
\end{Solution}
\begin{Solution}{48}
D
\end{Solution}
\begin{Solution}{49}
B
\end{Solution}
\begin{Solution}{50}
B
\end{Solution}
\begin{Solution}{51}
D
\end{Solution}
\begin{Solution}{52}
C
\end{Solution}
\begin{Solution}{53}
C
\end{Solution}
\begin{Solution}{54}
B
\end{Solution}
\begin{Solution}{55}
D
\end{Solution}
\begin{Solution}{56}
B
\end{Solution}
\begin{Solution}{57}
C
\end{Solution}
\begin{Solution}{58}
A
\end{Solution}
\begin{Solution}{59}
A
\end{Solution}
\begin{Solution}{60}
B
\end{Solution}
\begin{Solution}{61}
D
\end{Solution}
\begin{Solution}{62}
D
\end{Solution}
\begin{Solution}{63}
B
\end{Solution}
\begin{Solution}{64}
A
\end{Solution}
\begin{Solution}{65}
D
\end{Solution}
\begin{Solution}{66}
C
\end{Solution}
\begin{Solution}{67}
A
\end{Solution}
\begin{Solution}{68}
A
\end{Solution}
\begin{Solution}{69}
D
\end{Solution}
\begin{Solution}{70}
C
\end{Solution}
\begin{Solution}{71}
B
\end{Solution}
\begin{Solution}{72}
A
\end{Solution}
\begin{Solution}{73}
C
\end{Solution}
\begin{Solution}{74}
C
\end{Solution}
\begin{Solution}{75}
C
\end{Solution}
\begin{Solution}{76}
A
\end{Solution}
\begin{Solution}{77}
C
\end{Solution}
\begin{Solution}{78}
B
\end{Solution}
\begin{Solution}{79}
D
\end{Solution}
\begin{Solution}{80}
B
\end{Solution}

\section{Mức 9,10 điểm}
\setcounter{ex}{0}
\setcounter{dang}{0}
\Opensolutionfile{ans}[ans/CD1/Muc_9_10]
\begin{dang}{Tìm m để hàm số đơn điệu trên các khoảng xác định của nó}
	Đang thiếu bài thầy Jf Câu 1 đến 26 
\end{dang}
\begin{dang}
	{Tìm khoảng đơn điệu của hàm số $g(x) = f\left[ u(x)\right] +v(x)$ khi biết đồ thị hoặc bảng biến thiên của hàm số $y = f'(x)$}
\end{dang}
\begin{ex}[Đề tham khảo 2019]%[2D1K1-2]
	Cho hàm số $f(x)$ có bảng xét dấu của đạo hàm như sau
	\begin{center}
		\begin{tikzpicture}
			\tkzTabInit[nocadre,lgt=1.2,espcl=2,deltacl=0.6]
			{$x$ /0.6,$f'(x)$ /0.6}
			{$-\infty$,$1$,$2$,$3$,$4$,$+\infty$}
			\tkzTabLine{,-,$0$,+,$0$,+,$0$,-,$0$,+,}
		\end{tikzpicture}
	\end{center}
	Hàm số $y=3 f(x+2)-x^3+3 x$ đồng biến trên khoảng nào dưới đây?
	\choice
	{$(-\infty ;-1)$}
	{\True $(-1 ; 0)$}
	{$(0 ; 2)$}
	{$(1 ;+\infty)$}
	\loigiai{
		Ta có $y'=3\left[f'(x+2)-\left(x^2-3\right)\right]$.\\
		Với $x \in(-1 ; 0) \Rightarrow x+2 \in(1 ; 2) \Rightarrow f'(x+2)>0$, lại có $x^2-3<0 \Rightarrow y'>0 ;~ \forall x \in(-1 ; 0)$.\\
		Vậy hàm số $y=3 f(x+2)-x^3+3 x$ đồng biến trên khoảng $(-1 ; 0)$.\\
		Chú ý:\\
		+) Ta xét $x \in(1 ; 2) \subset(1 ;+\infty)
		\Rightarrow x+2 \in(3 ; 4)\\
		\Rightarrow f'(x+2)<0 ;~ x^2-3>0$\\
		Suy ra hàm số nghịch biến trên khoảng $(1 ; 2)$ nên loại hai phương án$(0 ; 2)$ và $(1 ;+\infty)$.\\
		+) Tương tự ta xét
		$x \in(-\infty ;-2) \Rightarrow x+2 \in(-\infty ; 0)\\
		\Rightarrow f'(x+2)<0 ; x^2-3>0 \Rightarrow y'<0 ; ~ \forall x \in(-\infty ;-2)$.\\
		Suy ra hàm số nghịch biến trên khoảng $(-\infty ;-2)$ nên loại$(-\infty ;-1)$.\\
		Vậy hàm số đã cho đồng biến trên khoảng $(-1 ; 0)$.
	}
\end{ex}
\begin{ex}[Đề Tham Khảo 2020 - Lần 1]%[2D1G1-2]
	\immini{
		Cho hàm số $f(x)$. Hàm số $y=f'(x)$ có đồ thị như hình bên. Hàm số $g(x)=f(1-2 x)+x^2-x$ nghịch biến trên khoảng nào dưới đây?
		\choice
		{\True $\left(1 ; \dfrac{3}{2}\right)$}
		{$\left(0 ; \dfrac{1}{2}\right)$}
		{$(-2 ;-1)$}
		{$(2 ; 3)$}
	}
	{
		\begin{tikzpicture}[scale=0.7,>=stealth, font=\footnotesize, line join=round, line cap=round]
			%\def\a{1} \def\b{-6} \def\c{9} \def\d{1} % Hệ số
			\def\xmin{-4} \def\xmax{6}
			\def\ymin{-3} \def\ymax{2} 
			%\draw[color=gray!50,dashed] (\xmin,\ymin) grid (\xmax,\ymax); 
			\draw[->] (\xmin,0)--(\xmax,0) node [below]{$x$};
			\draw[->] (0,\ymin)--(0,\ymax) node [left]{$y$};
			\node at (0,0) [below left]{$O$};
			%\node at (1,3) [below left]{$f'(x)$};
			%\node at (-1.3,4) {$f'(x)$};
			\draw[dashed] (-2,0) node[below]{$-2$}--(-2,1)--(0,1) node[below left]{$1$};
			\draw[dashed] (4,0) node[below left]{$4$}--(4,-2)--(0,-2) node[below left]{$-2$};
			%\draw[dashed] (1,0) node[below]{$1$}--(1,1);
			%\draw[dashed] (-0.5,0) node[below left]{$-0{,}5$}--(-0.5,2.125);
			\clip (\xmin+0.1,\ymin+0.1) rectangle (\xmax-0.5,\ymax-0.1);
			\draw[smooth,samples=300][domain=-4:5.5] plot(\x,{0.071*(\x)^3-0.142*(\x)^2-1.07*(\x)});
		\end{tikzpicture}
	}
	
	\loigiai{
		Ta có : $g(x)=f(1-2 x)+x^2-x \Rightarrow g'(x)=-2 f'(1-2 x)+2 x-1$.\\
		\immini{
			Đặt $t=1-2 x \Rightarrow g'(x)=-2 f'(t)-t$.\\
			$g'(x)=0 \Rightarrow f'(t)=-\dfrac{t}{2}$.\\
			Vẽ đường thẳng $y=-\dfrac{x}{2}$ và đồ thị hàm số $f'(x)$ trên cùng một hệ trục
		}	
		{
			\begin{tikzpicture}[scale=0.7,>=stealth, font=\footnotesize, line join=round, line cap=round]
				%\def\a{1} \def\b{-6} \def\c{9} \def\d{1} % Hệ số
				\def\xmin{-4} \def\xmax{6}
				\def\ymin{-3} \def\ymax{2} 
				%	\draw[color=gray!50,dashed] (\xmin,\ymin) grid (\xmax,\ymax); 
				\draw[->] (\xmin,0)--(\xmax,0) node [below]{$x$};
				\draw[->] (0,\ymin)--(0,\ymax) node [left]{$y$};
				\node at (0,0) [below left]{$O$};
				%\node at (1,3) [below left]{$f'(x)$};
				%\node at (-1.3,4) {$f'(x)$};
				\draw[dashed] (-2,0) node[below]{$-2$}--(-2,1)--(0,1) node[below left]{$1$};
				\draw[dashed] (4,0) node[below]{$4$}--(4,-2)--(0,-2) node[below left]{$-2$};
				%\draw[dashed] (1,0) node[below]{$1$}--(1,1);
				%\draw[dashed] (-0.5,0) node[below left]{$-0{,}5$}--(-0.5,2.125);
				\clip (\xmin+0.1,\ymin+0.1) rectangle (\xmax-0.5,\ymax-0.1);
				\draw[smooth,samples=300][domain=-4:5.5] plot(\x,{0.071*(\x)^3-0.142*(\x)^2-1.07*(\x)});
				\draw[smooth,samples=300][domain=-4:5.5] plot(\x,{(-0.5*(\x)});
			\end{tikzpicture}
		}	Hàm số $g(x)$ nghịch biến $\Rightarrow g'(x) \leq 0 \Rightarrow f'(t) \geq-\dfrac{t}{2}\Rightarrow\hoac{&-2 \leq t \leq 0 \\&t \geq 4.}$\\
		Như vậy $f'(1-2 x) \geq \dfrac{1-2 x}{-2}\Rightarrow\hoac{&-2 \leq 1-2 x \leq 0 \\ &4 \leq 1-2 x}\Rightarrow\hoac{&\dfrac{1}{2}\leq x \leq \dfrac{3}{2}\\ &x \leq-\dfrac{3}{2}.}$\\
		Vậy hàm số $g(x)=f(1-2 x)+x^2-x$ nghịch biến trên các khoảng $\left(\dfrac{1}{2}; \dfrac{3}{2}\right)$ và $\left(-\infty ;-\dfrac{3}{2}\right)$.\\
		Mà $\left(1 ; \dfrac{3}{2}\right) \subset \left(\dfrac{1}{2}; \dfrac{3}{2}\right)$ nên hàm số $g(x)=f(1-2 x)+x^2-x$ nghịch biến trên khoảng $\left(1 ; \dfrac{3}{2}\right)$.
	}
\end{ex}
\begin{ex}[Chuyên Lê Quý Đôn Điện Biên 2019]%[2D1G1-2]
	Cho hàm số $f(x)$ có bảng xét dấu của đạo hàm như sau
	\begin{center}
		\begin{tikzpicture}
			\tkzTabInit[nocadre,lgt=1.2,espcl=2,deltacl=0.6]
			{$x$ /0.6,$f'(x)$ /0.6}
			{$-\infty$,$0$,$1$,$2$,$3$,$+\infty$}
			\tkzTabLine{,+,$0$,-,$0$,-,$0$,+,$0$,-,}
		\end{tikzpicture}
	\end{center}
	Hàm số $y=f(x-1)+x^3-12 x+2019$ nghịch biến trên khoảng nào dưới đây?
	\choice
	{$(1 ;+\infty)$}
	{\True $(1 ; 2)$}
	{$(-\infty ; 1)$}
	{$(3 ; 4)$}
	\loigiai{
		$y'=f'(x-1)+3 x^2-12=f'(t)+3 t^2+6 t-9=f'(t)-\left(-3 t^2-6 t+9\right)$, với $t=x-1$.\\
		\immini{
			Nghiệm của phương trình $y'=0$ là hoành độ giao điểm của các đồ thị hàm số $y=f'(t)$ và $y=-3 t^2-6 t+9$.\\
			Vẽ đồ thị hàm số $y=f'(t)$ và $y=-3 t^2-6 t+9$ trên cùng một hệ trục tọa độ như hình vẽ bên.
		}	
		{		\begin{tikzpicture}[scale=0.5,>=stealth, font=\footnotesize, line join=round, line cap=round]
				\def\a{-3} \def\b{-6} \def\c{9} % Hệ số
				\def\xmin{-9} \def\xmax{7}
				\def\ymin{-3} \def\ymax{13}
				
				%\draw[color=gray!50,dashed] (\xmin,\ymin) grid (\xmax,\ymax);
				
				\draw[->] (\xmin,0)--(\xmax,0) node [below]{$x$};
				\draw[->] (0,\ymin)--(0,\ymax) node [left]{$y$};
				\node at (0,0) [below left]{$O$};
				\clip (\xmin+0.1,\ymin+0.1) rectangle (\xmax-0.5,\ymax-0.1);
				\draw[smooth,samples=300] plot(\x,{\a*(\x)^2+\b*(\x)+\c});
				\node at (1,0) [above right]{$1$};
				\node at (2,0) [below right]{$2$};
				\node at (3,0) [below right]{$3$};
				\node at (-3,-2) [left]{$y=-3t^2-6t+9$};
				\node at (4,0) [below right]{$f'(x)$};
				\draw (-2.2,10).. controls (-1,1.9) and (-0.5,0.8) .. (0,0);
				%\draw (-2,0).. controls (-1.5,-2) and (-0.5,-0) .. (0,0);
				\draw (0,0).. controls (0.4,-0.6) and (0.6,-0.6) .. (0.8,-0.2);
				\draw (0.8,-0.2).. controls (1,0.25) and (1.1,-0.1) .. (1.4,-0.8);
				\draw (1.4,-0.8).. controls (1.6,-1.1) and (1.7,-0.9) .. (2,0);
				\draw (2,0).. controls (2.4,1.1) and (2.6,1.1) .. (3.5,-1);
			\end{tikzpicture}
		}
		Dựa vào đồ thị trên, ta có bảng xét dấu của hàm số $y'=f'(t)-\left(-3 t^2-6 t+9\right)$ như sau $
		\left(t_0<-1\right)$
		\begin{center}
			\begin{tikzpicture}
				\tkzTabInit[nocadre,lgt=2,espcl=2,deltacl=0.6]
				{$x$ /0.6,$y'$ /0.6}
				{$-\infty$,$t_0$,$1$,$+\infty$}
				\tkzTabLine{,+,$0$,-,$0$,+,}
			\end{tikzpicture}
		\end{center}
		Hàm số nghịch biến trên khoảng $t \in\left(t_0 ; 1\right)$.\\
		Do đó hàm số nghịch biến trên khoảng $x \in(1 ; 2) \subset \left(t_0+1 ; 1\right)$.
	}
\end{ex}


\begin{ex}[Chuyên Phan Bội Châu Nghệ An 2019]%[2D1G1-2]
	Cho hàm số $f(x)$ có bảng xét dấu đạo hàm như sau:
	\begin{center}
		\begin{tikzpicture}
			\tkzTabInit[nocadre,lgt=2,espcl=2,deltacl=0.6]
			{$x$ /0.6,$f'(x)$ /0.6}
			{$-\infty$,$1$,$2$,$3$,$4$,$+\infty$}
			\tkzTabLine{,-,$0$,+,$0$,+,$0$,-,$0$,+,}
		\end{tikzpicture}
	\end{center}
	Hàm số $y=2 f(1-x)+\sqrt{x^2+1}-x$ nghịch biến trên những khoảng nào dưới đây
	\choice
	{$(-\infty ;-2)$}
	{$(-\infty ; 1)$}
	{\True $(-2 ; 0)$}
	{$(-3 ;-2)$}
	\loigiai{
		$y'=-2 f'(1-x)+\dfrac{x}{\sqrt{x^2+1}}-1$. \\
		Có $\dfrac{x}{\sqrt{x^2+1}}-1<0,~ \forall x \in(-2 ; 0)$.\\
		Bảng xét dấu:
		\begin{center}
			\begin{tikzpicture}
				\tkzTabInit[nocadre,lgt=2,espcl=2,deltacl=0.6]
				{$x$ /0.7,$f'(1-x)$ /0.7}
				{$-\infty$,$-3$,$-2$,$-1$,$0$,$+\infty$}
				\tkzTabLine{,+,$0$,-,$0$,+,$0$,+,$0$,-,}
			\end{tikzpicture}
		\end{center}
		$\Rightarrow-2 f'(1-x)<0, ~ \forall x \in(-2 ; 0) \\
		\Rightarrow-2 f'(1-x)+\dfrac{x}{\sqrt{x^2+1}}-1<0, ~\forall x \in(-2 ; 0)$.
	}
\end{ex}
\begin{ex}[Sở Vĩnh Phúc 2019]%[2D1G1-2]
	\immini{
		Cho hàm số bậc bốn $y=f(x)$ có đồ thị của hàm số $y=f'(x)$ như hình vẽ bên.\\
		Hàm số $y=3 f(x)+x^3-6 x^2+9 x$ đồng biến trên khoảng nào trong các khoảng sau đây?
		\choice
		{$(0 ; 2)$}
		{$(-1 ; 1)$}
		{$(1 ;+\infty)$}
		{\True $(-2 ; 0)$}
	}
	{
		\begin{tikzpicture}[scale=0.7,>=stealth, font=\footnotesize, line join=round, line cap=round]
			\def\a{0.21} \def\b{0.88} \def\c{-0.58} \def\d{-3} % Hệ số
			\def\xmin{-5} \def\xmax{5}
			\def\ymin{-4} \def\ymax{3} 
			%\draw[color=gray!50,dashed] (\xmin,\ymin) grid (\xmax,\ymax); 
			\draw[->] (\xmin,0)--(\xmax,0) node [below]{$x$};
			\draw[->] (0,\ymin)--(0,\ymax) node [left]{$y$};
			\node at (0,0) [above left]{$O$};
			\node at (-4,0) [below left]{$-4$};
			\node at (-2,0) [below left]{$-2$};
			\node at (0,-3) [below right]{$-3$};
			\draw[dashed] (2,0) node[above right]{$2$}--(2,1) --(0,1) node[above right]{$1$};
			\clip (\xmin+0.1,\ymin+0.1) rectangle (\xmax-0.5,\ymax-0.1);
			\draw[smooth,samples=300] plot(\x,{\a*(\x)^3+\b*(\x)^2+\c*(\x)+\d});
		\end{tikzpicture}
	}
	
	\loigiai{
		Hàm số $f(x)=a x^4+b x^3+c x^2+d x+e,(a \neq 0)$.
		Có $f'(x)=4 a x^3+3 b x^2+2 c x+d$.\\
		Đồ thị hàm số $y=f'(x)$ đi qua các điểm $(-4 ; 0),(-2 ; 0),(0 ;-3),(2 ; 1)$ nên ta có
		$$\heva{&- 2 5 6 a + 4 8 b - 8 c + d = 0\\
			&- 3 2 a + 1 2 b - 4 c + d = 0\\
			&d = - 3\\
			&3 2 a + 1 2 b + 4 c + d = 1}\Leftrightarrow \heva{&
			a=\dfrac{5}{96}\\
			&b=\dfrac{7}{24}\\
			&c=-\dfrac{7}{24}\\
			&d=-3.}
		$$
		Xét hàm số
		$
		y=3 f(x)+x^3-6 x^2+9 x$\\
		Ta có $ y'=3\left(f'(x)+x^2-4 x+3\right)=3\left(\frac{5}{24}x^3+\frac{15}{8}x^2-\frac{55}{12}x\right)
		$\\
		Ta có $y'=0 \Leftrightarrow\hoac{&x=-11 \\&x=0 \\&x=2.}$ \\
		Xét dấu $y'$, ta được hàm số đã cho đồng biến trên các khoảng $(-11 ; 0)$ và $(2 ;+\infty)$.
	}
\end{ex}
\begin{ex}[Học Mãi 2019]%[2D1K1-2]
	\immini
	{Cho hàm số $y=f(x)$ có đạo hàm trên $\mathbb{R}$. Đồ thị hàm số $y=f'(x)$ như hình bên. Hỏi đồ thị hàm số $y=f(x)-2 x$ có bao nhiêu điểm cực trị?
		\choice
		{$4$}
		{\True $3$}
		{$2$}
		{$1$}
	}
	{
		\begin{tikzpicture}[font=\footnotesize,line join=round, line cap=round,>=stealth,scale=0.8]
			\draw[->] (-3.5,0)--(4,0) node[above] {$x$};
			\draw[->] (0,-3)--(0,4) node[left] {$y$};
			%\fill[black] (-2,0)node[below left]{$-2$} circle (1.2pt) (0,0)node[above right]{$O$} circle (1.2pt) (3,0)node[above]{$3$} circle (1.2pt);
			\draw[dashed] (-2,-2)-- (0,-2) node[right]{$-2$};
			\draw[dashed] (2,0) node[below]{$2$}-- (2,2)--(0,2) node[below left]{$2$};
			\node at (0,0) [below left]{$O$};
			\node at (3,0) [below right]{$3$};
			\draw (-3,2.5).. controls (-2.2,-3) and (-1.8,-3) .. (-1.1,0);
			\draw (-1.1,0).. controls (-0.6,2.5) and (-0.4,2.5) .. (0,2);
			\draw (0,2).. controls (0.7,0.5) and (1.1,0.5) .. (1.5,1.5);
			\draw (1.5,1.5).. controls (2,2.5) and (2.8,2.5) .. (3.5,-2.5);
			%\draw (3,0).. controls (3.3,-0.1) and (3.5,-0.5) .. (3.5,-2);
		\end{tikzpicture}
	}
	\loigiai{
		\immini{
			Đặt $g(x)=f(x)-2 x$.\\
			$\Rightarrow g'(x)=f'(x)-2 .
			$\\
			Vẽ đường thẳng $y=2$.\\
			$\Rightarrow$ phương trình $g'(x)=0$ có $3$ nghiệm bội lẻ.\\
			$\Rightarrow$ đồ thị hàm số $y=f(x)-2 x$ có $3$ điểm cực trị.
		}
		{
			\begin{tikzpicture}[font=\footnotesize,line join=round, line cap=round,>=stealth,scale=0.8]
				\draw[->] (-3.5,0)--(4,0) node[above] {$x$};
				\draw[->] (0,-3)--(0,4) node[left] {$y$};
				%\fill[black] (-2,0)node[below left]{$-2$} circle (1.2pt) (0,0)node[above right]{$O$} circle (1.2pt) (3,0)node[above]{$3$} circle (1.2pt);
				\draw[dashed] (-2,-2)-- (0,-2) node[right]{$-2$};
				\draw[dashed] (2,0) node[below]{$2$}-- (2,2)--(0,2) node[below left]{$2$};
				\node at (3,0) [below left]{$3$};
				\draw (-3,2.5).. controls (-2.2,-3) and (-1.8,-3) .. (-1.1,0);
				\draw (-1.1,0).. controls (-0.6,2.5) and (-0.4,2.5) .. (0,2);
				\draw (0,2).. controls (0.7,0.5) and (1.1,0.5) .. (1.5,1.5);
				\draw (1.5,1.5).. controls (2,2.5) and (2.8,2.5) .. (3.5,-2.5);
				\draw (-3.5,2)--(4,2) node[above]{$y=2$};
			\end{tikzpicture}
		}
	}
\end{ex}
\begin{ex}[THPT Hoàng Hoa Thám Hưng Yên 2019]%[2D1G1-2]
	\immini{
		Cho hàm số $y=f(x)$ liên tục trên $\mathbb{R}$. Hàm số $y=f'(x)$ có đồ thị như hình vẽ. 
		Hàm số $g(x)=f(x-1)+\dfrac{2019-2018 x}{2018}$ đồng biến trên khoảng nào dưới đây?
		\choice
		{$(2 ; 3)$}
		{$(0 ; 1)$}
		{\True $(-1 ; 0)$}
		{$(1 ; 2)$}
	}
	{
		\begin{tikzpicture}[scale=1, font=\footnotesize, line join=round, line cap=round, >=stealth]
			\tikzset{label style/.style={font=\footnotesize}}
			\draw[->] (-2,0)--(3,0) node[below left] {$x$};
			\draw[->] (0,-2)--(0,3) node[below left] {$y$};
			\draw[fill=black] (0,0) node [above left] {$O$} circle(1pt);
			\fill (1,1) circle(1pt) (-1,1) circle(1pt) (2,1) circle(1pt);
			\foreach \x in {1,2}
			\draw[thin] (\x,1pt)--(\x,-1pt) node [below] {\footnotesize$\x$};
			\foreach \x in {-1}
			\draw[thin] (\x,1pt)--(\x,-1pt) node [below left] {\footnotesize$\x$};
			\foreach \y in {-1}
			\draw[thin] (1pt,\y)--(-1pt,\y) node [right] {\footnotesize$\y$};
			\foreach \y in {1}
			\draw[thin] (1pt,\y)--(-1pt,\y) node [above left] {\footnotesize$\y$};
			\draw[dashed](-1,0)--(-1,1)--(2,1) (1,1)--(1,0) (2,1)--(2,0);
			\begin{scope}
				\clip (-3,-3) rectangle (3,3);
				\draw[name path=(C)] plot[smooth,tension=0.7] coordinates{(-1.15,3)(-0.5,-1.6)(.8,.88)(1.9,0.8)(2.3,3)};
			\end{scope}
		\end{tikzpicture}
	}	\loigiai{
		Ta có $g'(x)=f'(x-1)-1$.\\
		$
		g'(x) \geq 0 \Leftrightarrow f'(x-1)-1 \geq 0 \Leftrightarrow f'(x-1) \geq 1 \Leftrightarrow \hoac{&x - 1 \leq - 1\\
			&x - 1 \geq 2}\Leftrightarrow \hoac{&
			x \leq 0 \\
			&x \geq 3.}
		$\\
		Từ đó suy ra hàm số $g(x)=f(x-1)+\dfrac{2019-2018 x}{2018}$ đồng biến trên khoảng $(-1 ; 0)$.
	}
\end{ex}

\begin{ex}[(Sở Ninh Bình 2019]%[2D1K1-2]
	Cho hàm số $y=f(x)$ có bảng xét dấu của đạo hàm như sau
	\begin{center}
		\begin{tikzpicture}
			\tkzTabInit[nocadre,lgt=1,espcl=2,deltacl=0.6]
			{$x$ /0.7,$f'(x)$ /0.7}
			{$-\infty$,$-2$,$-1$,$2$,$4$,$+\infty$}
			\tkzTabLine{,+,$0$,-,$0$,+,$0$,-,$0$,+,}
		\end{tikzpicture}
	\end{center}
	Hàm số $y=-2 f(x)+2019$ nghịch biến trên khoảng nào trong các khoảng dưới đây?
	\choice
	{$(-4 ; 2)$}
	{\True $(-1 ; 2)$}
	{$(-2 ;-1)$}
	{$(2 ; 4)$}
	\loigiai{
		Xét $y=g(x)=-2 f(x)+2019$.\\
		Ta có $g'(x)=(-2 f(x)+2019)'=-2 f'(x), g'(x)=0 \Leftrightarrow\hoac{&x=-2 \\&x=-1 \\&x=2 \\&x=4.}$.\\
		Ta có bảng xét dấu của $g'(x)$
		\begin{center}
			\begin{tikzpicture}
				\tkzTabInit[nocadre,lgt=1,espcl=2,deltacl=0.6]
				{$x$ /0.6,$f'(x)$ /0.6}
				{$-\infty$,$-2$,$-1$,$2$,$4$,$+\infty$}
				\tkzTabLine{,-,$0$,+,$0$,-,$0$,+,$0$,+,}
			\end{tikzpicture}
		\end{center}
		Dựa vào bảng xét dấu, ta thấy hàm số $y=g(x)$ nghịch biến trên khoảng $(-1 ; 2)$.
	}
\end{ex}
\begin{ex}[THPT Lương Thế Vinh Hà Nội 2019]%[2D1G1-2]
	\immini{
		Cho hàm số $y=f(x)$. Biết đồ thị hàm số $y=f'(x)$ có đồ thị như hình vẽ bên. 
		Hàm số $y=f \left(3-x^2\right)+2018$ đồng biến trên khoảng nào dưới đây?
		\choice
		{\True $(-1 ; 0)$}
		{$(2 ; 3)$}
		{$(-2 ;-1)$}
		{$(0 ; 1)$}
	}
	{
		\begin{tikzpicture}[scale=0.6,>=stealth, font=\footnotesize, line join=round, line cap=round]
			\def\a{0.065} \def\b{0.32} \def\c{-0.53} \def\d{-0.82} % Hệ số
			\def\xmin{-8} \def\xmax{4}
			\def\ymin{-3} \def\ymax{3} 
			%\draw[color=gray!50,dashed] (\xmin,\ymin) grid (\xmax,\ymax); 
			\draw[->] (\xmin,0)--(\xmax,0) node [below]{$x$};
			\draw[->] (0,\ymin)--(0,\ymax) node [left]{$y$};
			\node at (0,0) [below left]{$O$};
			\node at (-6,0) [below left]{$-6$};
			\node at (-1,0) [below left]{$-1$};
			\node at (2,0) [below right]{$2$};
			\clip (\xmin+0.1,\ymin+0.1) rectangle (\xmax-0.5,\ymax-0.1);
			\draw[smooth,samples=300][domain=-6.5:3.5] plot(\x,{\a*(\x)^3+\b*(\x)^2+\c*(\x)+\d});
		\end{tikzpicture}
	}
	
	\loigiai{
		Ta có $\left[f\left( 3-x^2\right)+2018 \right]'=-2 x \cdot f'\left(3-x^2\right) $.\\
		$
		-2 x \cdot f'\left(3-x^2\right)=0 \Leftrightarrow\hoac{&
			x = 0\\
			&3 - x ^{2}= - 6\\
			&3 - x ^{2}= - 1\\
			&3 - x ^{2}= 2}
		\Leftrightarrow \hoac{
			&x=0 \\
			&x=\pm 3 \\
			&x=\pm 2 \\
			&	x=\pm 1.}
		$\\
		Bảng xét dấu của đạo hàm hàm số đã cho
		\begin{center}
			\begin{center}
				\begin{tikzpicture}
					\tkzTabInit[nocadre,lgt=2.9,espcl=1.5,deltacl=0.6]
					{$x$ /0.7,$f'\left( 3-x^2\right) $/0.7,$-2xf'\left( 3-x^2\right)$/0.8}
					{$-\infty$,$-3$,$-2$,$-1$,$0$,$1$,$2$,$3$,$+\infty$}
					\tkzTabLine{,-,$0$,+,$0$,-,$0$,+,$0$,+,$0$,-,$0$,+,$0$,-}
					\tkzTabLine{,-,$0$,+,$0$,-,$0$,+,$0$,-,$0$,+,$0$,-,$0$,+}
				\end{tikzpicture}
			\end{center}
		\end{center}
		Từ bảng xét dấu suy ra hàm số đồng biến trên $(-1 ; 0)$.
	}
\end{ex}
\begin{ex}[Chuyên Biên Hòa - Hà Nam - 2020]%[2D1G1-2]
	\immini{
		Cho hàm số đa thức $f(x)$ có đạo hàm trên $\mathbb{R}$. Biết $f(0)=0$ và đồ thị hàm số $y=f'(x)$ như hình sau.
		Hàm số $g(x)=\left|4 f(x)+x^2\right|$ đồng biến trên khoảng nào dưới đây?
		\choice
		{$(4 ;+\infty)$}
		{\True $(0 ; 4)$}
		{$(-\infty ;-2)$}
		{$(-2 ; 0)$}
	}	
	{
		\begin{tikzpicture}[scale=0.7,>=stealth, font=\footnotesize, line join=round, line cap=round]
			%\def\a{1} \def\b{-6} \def\c{9} \def\d{1} % Hệ số
			\def\xmin{-4} \def\xmax{6}
			\def\ymin{-3} \def\ymax{2} 
			%\draw[color=gray!50,dashed] (\xmin,\ymin) grid (\xmax,\ymax); 
			\draw[->] (\xmin,0)--(\xmax,0) node [below]{$x$};
			\draw[->] (0,\ymin)--(0,\ymax) node [left]{$y$};
			\node at (0,0) [below left]{$O$};
			%\node at (1,3) [below left]{$f'(x)$};
			%\node at (-1.3,4) {$f'(x)$};
			\draw[dashed] (-2,0) node[below]{$-2$}--(-2,1)--(0,1) node[below left]{$1$};
			\draw[dashed] (4,0) node[below]{$4$}--(4,-2)--(0,-2) node[below left]{$-2$};
			%\draw[dashed] (1,0) node[below]{$1$}--(1,1);
			%\draw[dashed] (-0.5,0) node[below left]{$-0{,}5$}--(-0.5,2.125);
			\clip (\xmin+0.1,\ymin+0.1) rectangle (\xmax-0.5,\ymax-0.1);
			\draw[smooth,samples=300][domain=-4:5.5] plot(\x,{0.071*(\x)^3-0.142*(\x)^2-1.07*(\x)});
		\end{tikzpicture}
	}
	\loigiai{
		\immini{
			Xét hàm số $h(x)=4 f(x)+x^2$ trên $\mathbb{R}$.\\
			Vì $f(x)$ là hàm số đa thức nên $h(x)$ cũng là hàm số đa thức và $h(0)=4 f(0)=0$.\\
			Ta có $h'(x)=4 f'(x)+2 x$. Do đó $h'(x)=0 \Leftrightarrow f'(x)=-\dfrac{1}{2}x$.\\
		}
		{
			\begin{tikzpicture}[scale=0.7,>=stealth, font=\footnotesize, line join=round, line cap=round]
				%\def\a{1} \def\b{-6} \def\c{9} \def\d{1} % Hệ số
				\def\xmin{-4} \def\xmax{6}
				\def\ymin{-3} \def\ymax{2} 
				%\draw[color=gray!50,dashed] (\xmin,\ymin) grid (\xmax,\ymax); 
				\draw[->] (\xmin,0)--(\xmax,0) node [below]{$x$};
				\draw[->] (0,\ymin)--(0,\ymax) node [left]{$y$};
				\node at (0,0) [below left]{$O$};
				%\node at (1,3) [below left]{$f'(x)$};
				%\node at (-1.3,4) {$f'(x)$};
				\draw[dashed] (-2,0) node[below]{$-2$}--(-2,1)--(0,1) node[below left]{$1$};
				\draw[dashed] (4,0) node[below]{$4$}--(4,-2)--(0,-2) node[below left]{$-2$};
				%\draw[dashed] (1,0) node[below]{$1$}--(1,1);
				%\draw[dashed] (-0.5,0) node[below left]{$-0{,}5$}--(-0.5,2.125);
				\clip (\xmin+0.1,\ymin+0.1) rectangle (\xmax-0.5,\ymax-0.1);
				\draw[smooth,samples=300][domain=-4:5.5] plot(\x,{0.071*(\x)^3-0.142*(\x)^2-1.07*(\x)});
				\draw[smooth,samples=300][domain=-4:5.5] plot(\x,{-0.5*(\x)});
			\end{tikzpicture}
		}
		Dựa vào sự tương giao của đồ thị hàm số $y=f'(x)$ và đường thẳng $y=-\dfrac{1}{2}x$, ta có
		$
		h'(x)=0 \Leftrightarrow x \in\{-2 ; 0 ; 4\}.\\
		$
		Bảng biến thiên của hàm số $h(x)$ như sau:
		\begin{center}
			\begin{tikzpicture}
				\tkzTabInit[nocadre,lgt=1.2,espcl=2.5,deltacl=0.6]
				{$x$ /0.6,$y'$ /0.6,$y$ /2}
				{$-\infty$,$-2$,$0$,$4$,$+\infty$}
				\tkzTabLine{,-,$0$,+,$0$,-,$0$,+,}
				\tkzTabVar{+/$+\infty$, -/$y_1$,+/$0$,-/$y_3$,+/$+\infty$}
			\end{tikzpicture}
		\end{center}
		Từ đó suy ra bảng biến thiên của hàm số $g(x)=|h(x)|$.\\
		Dựa vào bảng biến thiên trên, ta thấy hàm số $g(x)$ đồng biến trên khoảng $(0 ; 4)$.
	}
\end{ex}
\begin{ex}[Chuyên Thái Bình - 2020]%[2D1G1-2]
	\immini{
		Cho hàm số $f(x)$ liên tục trên $\mathbb{R}$ có đồ thị hàm số $y=f'(x)$ cho như hình vẽ bên.\\
		Hàm số $g(x)=2 f(|x-1|)-x^2+2 x+2020$ đồng biến trên khoảng nào?
		\choice
		{\True $(0 ; 1)$}
		{$(-3 ; 1)$}
		{$(1 ; 3)$}
		{$(-2 ; 0)$}
	}
	{
		\begin{tikzpicture}[scale=0.7,>=stealth, font=\footnotesize, line join=round, line cap=round]
			%\def\a{1} \def\b{-6} \def\c{9} \def\d{1} % Hệ số
			\def\xmin{-4} \def\xmax{5}
			\def\ymin{-3} \def\ymax{5} 
			%\draw[color=gray!50,dashed] (\xmin,\ymin) grid (\xmax,\ymax); 
			\draw[->] (\xmin,0)--(\xmax,0) node [below]{$x$};
			\draw[->] (0,\ymin)--(0,\ymax) node [left]{$y$};
			\node at (0,0) [below left]{$O$};
			%\node at (1,3) [below left]{$f'(x)$};
			\node at (-1.3,4) {$f'(x)$};
			\draw[dashed] (-1,0) node[above]{$-1$}--(-1,-1)--(0,-1) node[below left]{$-1$};
			\draw[dashed] (1,0) node[below]{$1$}--(1,1)--(0,1) node[below left]{$1$};
			\draw[dashed] (3,0) node[below]{$3$}--(3,3)--(0,3) node[below left]{$3$};
			%\draw[dashed] (1,0) node[below]{$1$}--(1,1);
			%\draw[dashed] (-0.5,0) node[below left]{$-0{,}5$}--(-0.5,2.125);
			\clip (\xmin+0.1,\ymin+0.1) rectangle (\xmax-0.5,\ymax-0.1);
			\draw[smooth,samples=300][domain=-2:4] plot(\x,{-0.5*(\x)^3+1.5*(\x)^2+1.5*(\x)-1.5});
			%\draw[smooth,samples=300] plot(\x,{(\x)^3+(\x)^2-2*(\x)+1});
		\end{tikzpicture}
	}
	\loigiai{
		Ta có đường thẳng $y=x$ cắt đồ thị hàm số $y=f'(x)$ tại các điểm $x=-1 ; x=1 ; x=3$ như hình vẽ sau:
		\begin{center}
			\begin{tikzpicture}[scale=0.7,>=stealth, font=\footnotesize, line join=round, line cap=round]
				%\def\a{1} \def\b{-6} \def\c{9} \def\d{1} % Hệ số
				\def\xmin{-4} \def\xmax{5}
				\def\ymin{-3} \def\ymax{5} 
				%\draw[color=gray!50,dashed] (\xmin,\ymin) grid (\xmax,\ymax); 
				\draw[->] (\xmin,0)--(\xmax,0) node [below]{$x$};
				\draw[->] (0,\ymin)--(0,\ymax) node [left]{$y$};
				\node at (0,0) [below left]{$O$};
				%\node at (1,3) [below left]{$f'(x)$};
				\node at (-1.3,4) {$f'(x)$};
				\node at (4,3.2) {$y=x$};
				\draw[dashed] (-1,0) node[above]{$-1$}--(-1,-1)--(0,-1) node[below left]{$-1$};
				\draw[dashed] (1,0) node[below]{$1$}--(1,1)--(0,1) node[below left]{$1$};
				\draw[dashed] (3,0) node[below]{$3$}--(3,3)--(0,3) node[below left]{$3$};
				%\draw[dashed] (1,0) node[below]{$1$}--(1,1);
				%\draw[dashed] (-0.5,0) node[below left]{$-0{,}5$}--(-0.5,2.125);
				\clip (\xmin+0.1,\ymin+0.1) rectangle (\xmax-0.5,\ymax-0.1);
				\draw[smooth,samples=300][domain=-2:4] plot(\x,{-0.5*(\x)^3+1.5*(\x)^2+1.5*(\x)-1.5});
				\draw[smooth,samples=300] plot(\x,{(\x)});
			\end{tikzpicture}
		\end{center}
		Dựa vào đồ thị của hai hàm số trên ta có $f'(x)>x \Leftrightarrow\hoac{&x<-1 \\ &1<x<3}$ và
		$ f'(x)<x \Leftrightarrow\hoac{&
			-1<x<1 \\
			&x>3.}$\\
		+Trường hợp 1: $x-1<0 \Leftrightarrow x<1$.\\
		Khi đó $g(x)=2 f(1-x)-x^2+2 x+2020$.\\
		Ta có $g'(x)=-2 f'(1-x)+2(1-x)$.
		$$
		g'(x)>0 \Leftrightarrow-2 f'(1-x)+2(1-x)>0 \Leftrightarrow f'(1-x)<1-x \Leftrightarrow\hoac{
			&- 1 < 1 - x < 1\\
			&1 - x > 3} \Leftrightarrow \hoac{&
			0<x<2 \\
			&x<-2.}
		$$
		Kết hợp điều kiện, ta có $g'(x)>0 \Leftrightarrow\hoac{&0<x<1 \\ &x<-2.}$\\
		
		+ Trường hợp 2: $x-1>0 \Leftrightarrow x>1$.\\
		Khi đó ta có $g(x)=2 f(x-1)-x^2+2 x+2020$.\\
		$ g'(x)=2 f'(x-1)-2(x-1)$\\
		$g'(x)>0 \Leftrightarrow 2 f'(x-1)-2(x-1)>0 \Leftrightarrow f'(x-1)>x-1 \Leftrightarrow\hoac{&
			x - 1 < - 1\\
			&1 < x - 1 < 3}\Leftrightarrow \hoac{
			&x<0 \\
			&2<x<4.}$
		Kết hợp điều kiện ta có $g'(x)>0 \Leftrightarrow 2<x<4$.\\
		Vậy hàm số $g(x)=2 f(|x-1|)-x^2+2 x+2020$ đồng biến trên khoảng $(0 ; 1)$.
	}
\end{ex}

\begin{ex}[Chuyên Lào Cai - 2020]%[2D1G1-2]
	\immini{
		Cho hàm số $f'(x)$ có đồ thị như hình bên.\\
		Hàm số $g(x)=f(3 x+1)+9 x^3+\dfrac{9}{2}x^2$ đồng biến trên khoảng nào dưới đây?
		\choice
		{$(-1 ; 1)$}
		{$(-2 ; 0)$}
		{$(-\infty ; 0)$}
		{\True $(1 ;+\infty)$}
	}
	{\begin{tikzpicture}[line join=round, line cap=round,>=stealth,thick,scale=.8]
			\tikzset{label style/.style={font=\footnotesize}}
			\draw[->] (-2.1,0)--(5.1,0) node[below left] {$x$};
			\draw[->] (0,-3.1)--(0,4.1) node[below left] {$y$};
			\draw (0,0) node [below left] {$O$};
			\foreach \x in {1,2,3}
			\draw[thin] (\x,1pt)--(\x,-1pt) node [below] {$\x$};
			\draw[thin](-1,1pt)--(1,-1pt)node[above left]{$-1$};
			\foreach \y in {-2,2}
			\draw[thin] (1pt,\y)--(-1pt,\y) node [above right] {$\y$};
			%\begin{scope}
			\clip (-2,-3) rectangle (5,4);
			\draw[samples=200,domain=-2:4,smooth,variable=\x] plot (\x,{(\x)^3-3*(\x)^2+2});
			%\end{scope}
			\draw[dashed] (-1,0)--(-1,-2)--(0,-2);
			\draw[dashed] (3,0)--(3,2)--(0,2);
			%\begin{scope}[on background layer]\path[white]node{MDD-134};\end{scope}
		\end{tikzpicture}
	}
	\loigiai
	{
		\immini{Xét hàm số $g(x)=f(3 x+1)+9 x^3+\dfrac{9}{2}x^2 \\
			\Rightarrow g'(x)=3 f'(3 x+1)+27 x^2+9 x$.\\
			Hàm số đồng biến  $\Leftrightarrow g'(x)>0 \Leftrightarrow 3 f'(3 x+1)+27 x^2+9 x>0$
			\\
			$
			\Leftrightarrow f'(3 x+1)+3 x(3 x+1)>0 \qquad (*)
			$\\
			Đặt $t=3 x+1$, khi đó  $(*) \Leftrightarrow f'(t)+(t-1) t>0$\\ $\Leftrightarrow f'(t)>-t^2+t$.\\
			Vẽ parabol $y=-x^2+x$ và đồ thị hàm số $f'(x)$ trên cùng một hệ trục
		}
		{
			\begin{tikzpicture}[line join=round, line cap=round,>=stealth,thick,scale=.8]
				\tikzset{label style/.style={font=\footnotesize}}
				\draw[->] (-2.1,0)--(5.1,0) node[below left] {$x$};
				\draw[->] (0,-3.1)--(0,4.1) node[below left] {$y$};
				\draw (0,0) node [below left] {$O$};
				\foreach \x in {1,2,3}
				\draw[thin] (\x,1pt)--(\x,-1pt) node [below] {$\x$};
				\draw[thin](-1,1pt)--(1,-1pt);
				\foreach \y in {-2,2}
				\draw[thin] (1pt,\y)--(-1pt,\y) node [above right] {$\y$};
				%\begin{scope}
				\clip (-2,-3) rectangle (5,4);
				\draw[samples=200,domain=-2:4,smooth,variable=\x] plot (\x,{(\x)^3-3*(\x)^2+2});
				\draw[samples=200,domain=-2:4,smooth,variable=\x] plot (\x,{-(\x)^2+(\x)});
				%\end{scope}
				\draw[dashed] (-1,0) node[above left]{$-1$}--(-1,-2)--(0,-2);
				\draw[dashed] (3,0)--(3,2)--(0,2);
				%\begin{scope}[on background layer]\path[white]node{MDD-134};\end{scope}
			\end{tikzpicture}
		}
		Dựa vào đồ thị ta thấy
		$
		f'(t)>-t^2+t \Leftrightarrow\hoac{&- 1 < t < 1\\
			&t > 2}\Rightarrow \hoac{&
			- 1 < 3 x + 1 < 1\\
			&3 x + 1 > 2} \Leftrightarrow \hoac{&
			\dfrac{-2}{3}<x<0\\
			&x>\dfrac{1}{3}.}
		$}
\end{ex}
\begin{ex}[Sở Phú Thọ-2020]%[2D1G1-2]
	\immini{
		Cho hàm số $y=f(x)$ có đồ thị hàm số $y=f'(x)$ như hình vẽ.\\
		Hàm số $g(x)=f\left(\mathrm{e}^x-2\right)-2020$ nghịch biến trên khoảng nào dưới đây?
		\choice
		{\True $\left(-1 ; \dfrac{3}{2}\right)$}
		{$(-1 ; 2)$}
		{$(0 ;+\infty)$}
		{$\left(\dfrac{3}{2}; 2\right)$}
	}
	{
		\begin{tikzpicture}[scale=0.7,>=stealth, font=\footnotesize, line join=round, line cap=round]
			\def\a{1} \def\b{-3} \def\c{0} \def\d{0} % Hệ số
			\def\xmin{-2} \def\xmax{4}
			\def\ymin{-5} \def\ymax{2} 
			%\draw[color=gray!50,dashed] (\xmin,\ymin) grid (\xmax,\ymax); 
			\draw[->] (\xmin,0)--(\xmax,0) node [below]{$x$};
			\draw[->] (0,\ymin)--(0,\ymax) node [left]{$y$};
			\node at (0,0) [above left]{$O$};
			\node at (3,0) [below right]{$3$};
			\draw[dashed] (2,0) node[above]{$2$}--(2,-4) --(0,-4) node[left]{$-4$};
			\clip (\xmin+0.1,\ymin+0.1) rectangle (\xmax-0.5,\ymax-0.1);
			\draw[smooth,samples=300] plot(\x,{\a*(\x)^3+\b*(\x)^2+\c*(\x)+\d});
		\end{tikzpicture}
	}
	
	\loigiai{
		Dựa vào đồ thị hàm số $y=f'(x)$ suy ra $f'(x) \leq 0 ~ \forall x<3$ và $f'(x)>0 ~ \forall x>3$.
		$
		g'(x)=\mathrm{e}^x f'\left(\mathrm{e}^x-2\right) .
		$
		Hàm số $g(x)=f\left(\mathrm{e}^x-2\right)-2020$ nghịch biến \\ $ \Leftrightarrow g'(x)<0 \Leftrightarrow \mathrm{e}^x f'\left(\mathrm{e}^x-2\right)<0$\\
		$
		\Leftrightarrow f'\left(\mathrm{e}^x-2\right)<0 \Leftrightarrow \mathrm{e}^x-2<3 \Leftrightarrow \mathrm{e}^x<5 \Leftrightarrow x<\ln 5 .
		$\\
		Vậy hàm số đã cho nghịch biến trên $\left(-1 ; \dfrac{3}{2}\right)$.
	}
\end{ex}
\begin{ex}[Lý Nhân Tông - Bắc Ninh - 2020]%[2D1G1-2]
	\immini{
		Cho hàm số $f(x)$ có đồ thị hàm số $f'(x)$ như hình vẽ.\\
		Hàm số $y=f(\cos x)+x^2-x$ đồng biến trên khoảng
		\choice
		{$(-2 ; 1)$}
		{$(0 ; 1)$}
		{\True $(1 ; 2)$}
		{$(-1 ; 0)$}
	}
	{
		\begin{tikzpicture}[scale=1,>=stealth, font=\footnotesize, line join=round, line cap=round]
			\def\a{-0.5} \def\b{0} \def\c{1.5} \def\d{0} % Hệ số
			\def\xmin{-3} \def\xmax{4}
			\def\ymin{-2} \def\ymax{2} 
			%\draw[color=gray!50,dashed] (\xmin,\ymin) grid (\xmax,\ymax); 
			\draw[->] (\xmin,0)--(\xmax,0) node [below]{$x$};
			\draw[->] (0,\ymin)--(0,\ymax) node [left]{$y$};
			\node at (0,0) [above left]{$O$};
			\node at (3,0) [below right]{$3$};
			\draw[dashed] (-2,0) node[below]{$-2$}--(-2,1) --(0,1) node[above right]{$1$} --(1,1)--(1,0) node[below]{$1$};
			\draw[dashed] (-1,0) node[below right]{$-1$}--(-1,-1) --(0,-1) node[above right]{$-1$} --(2,-1)--(2,0) node[below right]{$2$};
			\clip (\xmin+0.1,\ymin+0.1) rectangle (\xmax-0.5,\ymax-0.1);
			\draw[smooth,samples=300][domain=-2:2] plot(\x,{\a*(\x)^3+\b*(\x)^2+\c*(\x)+\d});
		\end{tikzpicture}
	}
	\loigiai{
		Đặt  $g(x)=f(\cos x)+x^2-x$.\\
		Ta có $g'(x)=-\sin x \cdot f'(\cos x)+2 x-1$\\
		Vì $\cos x \in[-1 ; 1]$ nên từ đồ thị $f'(x)$ ta suy ra $f'(\cos x) \in[-1 ; 1]$.\\
		Do đó $\left|-\sin x \cdot f'(\cos x)\right| \leq 1, ~\forall x \in \mathbb{R}$.\\
		Ta suy ra $g'(x)=\sin x \cdot f'(\cos x)+2 x-1 \geq-1+2 x-1=2 x-2$
		$\Rightarrow g'(x)>0, ~\forall x>1$.\\
		Vậy hàm số đồng biến trên $(1 ; 2)$.
	}
\end{ex}
\begin{ex}[THPT Nguyễn Viết Xuân - 2020]%[2D1G1-2]
	\immini{
		Cho hàm số $f(x)$. Hàm số $y=f'(x)$ có đồ thị như hình vẽ.\\
		Hàm số $g(x)=f\left(3 x^2-1\right)-\dfrac{9}{2}x^4+3 x^2$ đồng biến trên khoảng nào dưới đây?
		\choice
		{\True $\left(-\dfrac{2 \sqrt{3}}{3}; \dfrac{-\sqrt{3}}{3}\right)$}
		{$\left(0 ; \dfrac{2 \sqrt{3}}{3}\right)$}
		{$(1 ; 2)$}
		{$\left(-\dfrac{\sqrt{3}}{3}; \dfrac{\sqrt{3}}{3}\right)$} 
	}
	{
		\begin{tikzpicture}[scale=0.6,>=stealth, font=\footnotesize, line join=round, line cap=round]
			\def\a{0.25} \def\b{0.25} \def\c{-2} \def\d{0} % Hệ số
			\def\xmin{-5} \def\xmax{4}
			\def\ymin{-5} \def\ymax{5} 
			%\draw[color=gray!50,dashed] (\xmin,\ymin) grid (\xmax,\ymax); 
			\draw[->] (\xmin,0)--(\xmax,0) node [below]{$x$};
			\draw[->] (0,\ymin)--(0,\ymax) node [left]{$y$};
			\node at (0,0) [above left]{$O$};
			%\node at (3,0) [below right]{$3$};
			\draw[dashed] (-4,0) node[below left]{$-4$}--(-4,-4) --(0,-4) node[above right]{$-4$};
			\draw[dashed] (3,0) node[below right]{$3$}--(3,3) --(0,3) node[above right]{$3$};
			\clip (\xmin+0.1,\ymin+0.1) rectangle (\xmax-0.5,\ymax-0.1);
			\draw[smooth,samples=300] plot(\x,{\a*(\x)^3+\b*(\x)^2+\c*(\x)+\d});
		\end{tikzpicture}
	}
	
	\loigiai
	{
		TXĐ: $\mathscr{D}=\mathbb{R}$.\\
		Ta có $g'(x)=6 x f'\left(3 x^2-1\right)-18 x^3+6 x=6 x\left[f'\left(3 x^2-1\right)-3 x^2+1\right]$.\\
		$
		g'(x)=0 \Leftrightarrow\hoac{
			&x = 0\\
			&f '( 3 x ^{2}- 1 ) = 3 x ^{2}- 1}
		\Leftrightarrow \hoac{
			&x = 0\\
			&3 x ^{2}- 1 = - 4 \text{~(vô nghiệm)}\\
			&3 x ^{2}- 1 = 0\\
			&3 x ^{2}- 1 = 3}\Leftrightarrow \hoac{&x=0 \\
			&x=\pm \dfrac{\sqrt{3}}{3}\\
			&x=\pm \dfrac{2 \sqrt{3}}{3}.}
		$\\
		Bảng xét dấu
		\begin{center}
			\begin{tikzpicture}
				\tkzTabInit[nocadre,lgt=1.2,espcl=2.2,deltacl=0.6]
				{$x$ /1.2,$f'(x)$ /0.7}
				{$-\infty$,$-\dfrac{2 \sqrt{3}}{3}$,$-\dfrac{ \sqrt{3}}{3}$,$0$,$\dfrac{\sqrt{3}}{3}$,$\dfrac{2 \sqrt{3}}{3}$,$+\infty$}
				\tkzTabLine{,-,$0$,+,$0$,-,$0$,+,$0$,-,$0$,+,}
			\end{tikzpicture}
		\end{center}
		Vậy hàm số đồng biến trong khoảng $\left(-\dfrac{2 \sqrt{3}}{3}; \dfrac{-\sqrt{3}}{3}\right)$.}
\end{ex}
\begin{ex}[Trần Phú - Quảng Ninh - 2020]%[2D1G1-2]
	Cho hàm số $f(x)$ có bảng xét dấu của đạo hàm như sau
	\begin{center}
		\begin{tikzpicture}
			\tkzTabInit[nocadre,lgt=1.2,espcl=2,deltacl=0.6]
			{$x$ /0.6,$f'(x)$ /0.6}
			{$-\infty$,$-4$,$-1$,$2$,$7$,$+\infty$}
			\tkzTabLine{,+,$0$,-,$0$,+,$0$,-,$0$,+,}
		\end{tikzpicture}
	\end{center}
	Hàm số $y=f(2 x+1)+\dfrac{2}{3}x^3-8 x+5$ nghịch biến trên khoảng nào dưới đây?
	\choice
	{$(-\infty ;-2)$}
	{$(1 ;+\infty)$}
	{$(-1 ; 7)$}
	{\True $\left(-1 ; \dfrac{1}{2}\right)$}
	\loigiai{
		Ta có $y'=2 f'(2 x+1)+2 x^2-8$.\\
		Xét $y'\leq 0 \Leftrightarrow 2 f'(2 x+1)+2 x^2-8 \leq 0 \Leftrightarrow f'(2 x+1) \leq 4-x^2$.\\
		Đặt $t=2x+1$, ta có $f'(t) \leq \dfrac{-t^2+2 t+15}{4}$.\\
		Vì $\dfrac{-t^2+2 t+15}{4}\geq 0, \forall t \in[-3 ; 5]$.\\
		Mà $f'(t) \leq 0, \forall t \in[-3 ; 2]$.\\
		Nên $f'(t) \leq \dfrac{-t^2+2 t+15}{4}\Rightarrow t \in[-3 ; 2]$.\\
		Suy ra $-3 \leq 2 x+1 \leq 2 \Leftrightarrow-2 \leq x \leq \dfrac{1}{2}$.}
\end{ex}

\begin{ex}[Chuyên Thái Bình - Lần 3 - 2020]%[2D1G1-2]
	\immini{
		Cho hàm số $y=f(x)$ liên tục trên $\mathbb{R}$ có đồ thị hàm số $y=f'(x)$ cho như hình vẽ.\\
		Hàm số $g(x)=2 f(|x-1|)-x^2+2 x+2020$ đồng biến trên khoảng nào?
		\choice
		{\True $(0 ; 1)$}
		{$(-3 ; 1)$}
		{$(1 ; 3)$}
		{$(-2 ; 0)$}
	}
	{
		\begin{tikzpicture}[scale=0.7,>=stealth, font=\footnotesize, line join=round, line cap=round]
			\def\a{-0.333} \def\b{1} \def\c{1.333} \def\d{-1} % Hệ số
			\def\xmin{-3} \def\xmax{5}
			\def\ymin{-3} \def\ymax{5} 
			%\draw[color=gray!50,dashed] (\xmin,\ymin) grid (\xmax,\ymax); 
			\draw[->] (\xmin,0)--(\xmax,0) node [below]{$x$};
			\draw[->] (0,\ymin)--(0,\ymax) node [left]{$y$};
			\node at (0,0) [above left]{$O$};
			%\node at (3,0) [below right]{$3$};
			\draw[dashed] (-1,0) node[above]{$-1$}--(-1,-1) --(0,-1) node[above right]{$-1$};
			\draw[dashed] (1,0) node[below right]{$1$}--(1,1) --(0,1) node[above right]{$1$};
			\draw[dashed] (3,0) node[below right]{$3$}--(3,3) --(0,3) node[above right]{$3$};
			\clip (\xmin+0.1,\ymin+0.1) rectangle (\xmax-0.5,\ymax-0.1);
			\draw[smooth,samples=300] plot(\x,{\a*(\x)^3+\b*(\x)^2+\c*(\x)+\d});
			\draw[smooth,samples=300] plot(\x,{(\x)});
		\end{tikzpicture}
	}
	\loigiai{
		Với $x>1$, ta có $g(x)=2 f(x-1)-(x-1)^2+2021 \Rightarrow g'(x)=2 f'(x-1)-2(x-1)$.\\
		Hàm số đồng biến $\Leftrightarrow 2 f'(x-1)-2(x-1)>0 \Leftrightarrow f'(x-1)>x-1 \quad(*)$.\\
		Đặt $t=x-1$, khi đó $(*) \Leftrightarrow f'(t)>t \Leftrightarrow\hoac{&1<t<3 \\ &t<-1}\Rightarrow\hoac{&2<x<4 \\ &x<0 ~(\text{loại}).}$\\
		Với $x<1$, ta có $g(x)=2 f(1-x)-(1-x)^2+2021 \Rightarrow g'(x)=-2 f'(1-x)+2(1-x)$.\\
		Hàm số đồng biến $\Leftrightarrow-2 f'(1-x)+2(1-x)>0 \Leftrightarrow f'(1-x)<1-x \quad(* *)$.\\
		Đặt $t=1-x$, khi đó $(* *) \Leftrightarrow f'(t)<t \Leftrightarrow\hoac{&-1<t<1 \\ &t>3}\Rightarrow\hoac{&0<x<2 \\ &x<-2}\Rightarrow\hoac{&0<x<1 \\ &x<-2.}$\\
		Vậy hàm số $g(x)$ đồng biến trên các khoảng $(-\infty ;-2),(0 ; 1),(2 ; 4)$.
	}
\end{ex}
\begin{ex}[Sở Phú Thọ - 2020]%[2D1G1-2]
	\immini{
		Cho hàm số $y=f(x)$ có đồ thị hàm số $f'(x)$ như hình vẽ.\\
		Hàm số $g(x)=f\left(1+e^x\right)+2020$ nghịch biến trên khoảng nào dưới đây?
		\choice
		{$(0 ;+\infty)$}
		{$\left(\dfrac{1}{2}; 1\right)$}
		{\True $\left(0 ; \dfrac{1}{2}\right)$}
		{$(-1 ; 1)$}
	}{
		\begin{tikzpicture}[scale=0.7,>=stealth, font=\footnotesize, line join=round, line cap=round]
			\def\a{1} \def\b{-3} \def\c{0} \def\d{0} % Hệ số
			\def\xmin{-2} \def\xmax{4}
			\def\ymin{-5} \def\ymax{2} 
			%\draw[color=gray!50,dashed] (\xmin,\ymin) grid (\xmax,\ymax); 
			\draw[->] (\xmin,0)--(\xmax,0) node [below]{$x$};
			\draw[->] (0,\ymin)--(0,\ymax) node [left]{$y$};
			\node at (0,0) [above left]{$O$};
			\node at (3,0) [below right]{$3$};
			\draw[dashed] (2,0) node[above]{$2$}--(2,-4) --(0,-4) node[left]{$-4$};
			\clip (\xmin+0.1,\ymin+0.1) rectangle (\xmax-0.5,\ymax-0.1);
			\draw[smooth,samples=300] plot(\x,{\a*(\x)^3+\b*(\x)^2+\c*(\x)+\d});
		\end{tikzpicture}
	}
	\loigiai{
		$g'(x)=e^x f'\left(1+e^x\right)$.\\
		Do $e^x>0, \forall x$ nên $g'(x) \leq 0 \Leftrightarrow f'\left(1+e^x\right) \leq 0 \Leftrightarrow 1+e^x \leq 3 \Leftrightarrow x \leq \ln 2$, dấu bằng xảy ra tại hữu hạn điểm.\\
		Nên $g(x)$ nghịch biến trên $(-\infty ; \ln 2)$.\\
		Vì $\left(0 ; \dfrac{1}{2}\right) \subset (-\infty ; \ln 2)$ nên hàm số đã cho nghịch biến trên $\left(0 ; \dfrac{1}{2}\right)$.
	}
\end{ex}

\begin{ex}%[2D1K1-2]
	[THPT Anh Sơn - Nghệ An - 2020]
	Cho hàm số $y=f(x)$ có bảng xét dấu của đạo hàm như sau.
	\begin{center}
		\begin{tikzpicture}
			\tkzTabInit[nocadre,lgt=1.2,espcl=2,deltacl=0.6]
			{$x$ /0.6,$f'(x)$ /0.6}
			{$-\infty$,$-2$,$-1$,$2$,$4$,$+\infty$}
			\tkzTabLine{,+,$0$,-,$0$,+,$0$,-,$0$,+,}
		\end{tikzpicture}
	\end{center}
	Hàm số $y=-2 f(x)+2019$ nghịch biến trên khoảng nào trong các khoảng dưới đây?
	\choice
	{$(2 ; 4)$}
	{$(-4 ; 2)$}
	{$(-2 ;-1)$}
	{\True $(-1 ; 2)$}
	\loigiai{
		Ta có $y'=-2 f'(x)$.\\
		$
		y'=0 \Leftrightarrow-2 f'(x)=0 \Leftrightarrow\hoac{&
			x=-2 \\
			&x=-1 \\
			&x=2 \\
			&x=4.}$\\
		Từ bảng xét dấu của $f'(x)$ ta có
		\begin{center}
			\begin{tikzpicture}
				\tkzTabInit[nocadre,lgt=1,espcl=2,deltacl=0.6]
				{$x$ /0.6,$y'$ /0.6}
				{$-\infty$,$-2$,$-1$,$2$,$4$,$+\infty$}
				\tkzTabLine{,-,$0$,+,$0$,-,$0$,+,$0$,-,}
			\end{tikzpicture}
		\end{center}
		Từ bảng xét dấu ta có hàm số nghịch biến trên khoảng $(-\infty ;-2),(-1 ; 2)$ và $(4 ;+\infty)$.}
\end{ex}

\begin{ex}[THPT Anh Sơn - Nghệ An - 2020]%[2D1G1-2]
	Cho hàm số $f(x)$ xác định và liên tục trên $\mathbb{R}$ và có đạo hàm $f'(x)$ thỏa mãn $f'(x)=(1-x)(x+2) g(x)+2019$ với $g(x)<0, ~\forall x \in \mathbb{R}$ . Hàm số $y=f(1-x)+2019 x+2020$ nghịch biến trên khoảng nào?
	\choice
	{$(1 ;+\infty)$}
	{$(0 ; 3)$}
	{$(-\infty ; 3)$}
	{\True $(3 ;+\infty)$}
	\loigiai{
		Đặt $h(x)=f(1-x)+2019 x+2020$.\\
		Vì hàm số $f(x)$ xác định trên $\mathbb{R}$ nên hàm số $h(x)$ cũng xác định trên $\mathbb{R}$.\\
		Ta có $h'(x)=-f'(1-x)+2019$.\\
		Do $h'(x)=0$ tại hữu hạn điểm nên để tìm khoảng nghịch biến của hàm số $h(x)$, ta tìm các giá trị của $x$ sao cho $h'(x)<0 \Leftrightarrow-f'(1-x)+2019<0$\\ 
		$\Leftrightarrow f'(1-x)-2019>0 \\
		\Leftrightarrow x(3-x) g(1-x)>0 \Leftrightarrow x(3-x)<0(\text{~do~}g(x)<0, \forall x \in \mathbb{R})$\\
		$\Leftrightarrow\hoac{&
			x<0 \\
			&x>3.}$\\
		Vậy hàm số $y=f(1-x)+2019 x+2020$ nghịch biến trên các khoảng $(-\infty ; 0)$ và $(3 ;+\infty).$}
\end{ex}

\begin{ex}%[2D1G1-2]
	Cho hàm số $y=f(x)$ xác định trên $\mathbb{R}$ và có bảng xét dấu đạo hàm như sau:
	\begin{center}
		\begin{tikzpicture}
			\tkzTabInit[nocadre,lgt=2,espcl=2,deltacl=0.6]
			{$x$ /0.6,$f'(x)$ /0.6}
			{$-\infty$,$-1$,$1$,$4$,$+\infty$}
			\tkzTabLine{,-,$0$,+,$0$,-,$0$,+,}
		\end{tikzpicture}
	\end{center}
	Biết $f(x)>2,~ \forall x \in \mathbb{R}$. Xét hàm số $g(x)=f(3-2 f(x))-x^3+3 x^2-2020$. Khẳng định nào sau đây đúng?
	\choice
	{Hàm số $g(x)$ đồng biến trên khoảng $(-2 ;-1)$}
	{Hàm số $g(x)$ nghịch biến trên khoảng $(0 ; 1)$}
	{Hàm số $g(x)$ đồng biến trên khoảng $(3 ; 4)$}
	{\True Hàm số $g(x)$ nghịch biến trên khoảng $(2 ; 3)$}
	\loigiai{
		Ta có $g'(x)=-2 f'(x) f'(3-2 f(x))-3 x^2+6 x$.\\
		Vì $f(x)>2, ~\forall x \in \mathbb{R}$ nên $3-2 f(x)<-1 ~\forall x \in \mathbb{R}$.\\
		Từ bảng xét dấu $f'(x)$ suy ra $f'(3-2 f(x))<0, ~\forall x \in \mathbb{R}$.\\
		Từ đó ta có bảng xét dấu sau:
		\begin{center}
			\begin{tikzpicture}
				\tkzTabInit[nocadre,lgt=4,espcl=1.7,deltacl=0.6]
				{$x$ /0.7,$-f'(x)f'\left( 3-2f(x)\right) $/0.8,$-3x^2+6x$/0.7}
				{$-\infty$,$-1$,$0$,$1$,$2$,$4$,$+\infty$}
				\tkzTabLine{,-,$0$,+,|,+,$0$,-,|,-,$0$,+,}
				\tkzTabLine{,-,|,-,$0$,+,|,+,$0$,-,|,-,}
			\end{tikzpicture}
		\end{center}
		Từ bảng xét dấu trên, loại trừ đáp án suy ra hàm số $g(x)$ nghịch biến trên khoảng $(2 ; 3)$.}
\end{ex}

\begin{ex}%[2D1G1-2]
	Cho hàm số $f(x)$ có bảng biến thiên như sau:
	\begin{center}
		\begin{tikzpicture}
			\tkzTabInit[nocadre,lgt=1.2,espcl=2.5,deltacl=0.6]
			{$x$ /0.7, $f'(x)$ /0.7, $f(x)$ /2.5}
			{$-\infty$,$1$,$2$,$3$,$4$,$+\infty$}
			\tkzTabLine{,+,$0$,-,$0$,+,$0$,-,$0$,+,}
			\tkzTabVar{-/$-\infty$,+/$3$,-/$1$,+/$2$,-/$0$,+/$+\infty$}
		\end{tikzpicture}
	\end{center}
	Hàm số $y=(f(x))^3-3 .(f(x))^2$ nghịch biến trên khoảng nào dưới đây?
	\choice
	{$(1 ; 2)$}
	{$(3 ; 4)$}
	{$(-\infty ; 1)$}
	{\True $(2 ; 3)$}
	\loigiai{
		Ta có $y'=3 \cdot(f(x))^2 \cdot f'(x)-6 \cdot f(x) \cdot f'(x)=3 f(x) \cdot f'(x) \cdot[f(x)-2]. \\
		y'=0 \Leftrightarrow \hoac{&f(x)=0 \Leftrightarrow x \in\left\{x_1, 4 \mid x_1<1\right\}\\
			&f(x)=2 \Leftrightarrow x \in\left\{x_2, x_3, 3, x_4 \mid x_1<x_2<1<x_3<2 ; 4<x_4\right\}\\
			&f'(x)=0 \Leftrightarrow x \in\{1,2,3,4\}.}$\\
		Lập bảng xét dấu ta có
		\begin{center}
			\begin{tikzpicture}
				\tkzTabInit[nocadre,lgt=2,espcl=1.5,deltacl=0.6]
				{$x$ /0.7,$f(x)$ /0.7,$f(x)-2$ /0.7,$f'(x)$/0.7,$y'$/0.7}
				{$-\infty$,$x_1$,$x_2$,$1$,$x_3$,$2$,$3$,$4$,$x_4$,$+\infty$}
				\tkzTabLine{,-,$0$,+,|,+,|,+,|,+,|,+,$0$,+,|,+,|,+,}
				\tkzTabLine{,-,|,-,$0$,+,$0$,+,$0$,-,|,-,$0$,-,|,-,$0$,+}
				\tkzTabLine{,+,|,+,|,+,$0$,-,|,-,$0$,+,$0$,-,$0$,+,|,+}
				\tkzTabLine{,+,$0$,-,$0$,+,$0$,-,$0$,+,$0$,-,$0$,+,$0$,-,$0$,+}
			\end{tikzpicture}
		\end{center}
		
		Do đó hàm số nghịch biến trên khoảng $(2 ; 3)$.
	}
\end{ex}
\begin{ex}%[2D1G1-2]
	Cho hàm số $y=f(x)$ có đồ thị nằm trên trục hoành và có đạo hàm trên $\mathbb{R}$, bảng xét dấu của biểu thức $f'(x)$ như bảng dưới đây.
	\begin{center}
		\begin{tikzpicture}
			\tkzTabInit[nocadre,lgt=1.2,espcl=2,deltacl=0.6]
			{$x$ /0.6,$f'(x)$ /0.6}
			{$-\infty$,$-2$,$-1$,$3$,$+\infty$}
			\tkzTabLine{,-,$0$,+,$0$,-,$0$,+,}
		\end{tikzpicture}
	\end{center}
	Hàm số $y=g(x)=\dfrac{f\left(x^2-2 x\right)}{f\left(x^2-2 x\right)+1}$ nghịch biến trên khoảng nào dưới đây?
	\choice
	{$(-\infty ; 1)$}
	{$\left(-2 ; \dfrac{5}{2}\right)$}
	{\True $(1 ; 3)$}
	{$(2 ;+\infty)$}
	\loigiai{
		$ g'(x)=\dfrac{\left(x^2-2 x\right)'\cdot f'\left(x^2-2 x\right)}{\left(f\left(x^2-2 x\right)+1\right)^2}=\dfrac{(2 x-2) \cdot f'\left(x^2-2 x\right)}{\left(f\left(x^2-2 x\right)+1\right)^2}. \\
		g'(x)=0 \Leftrightarrow\hoac{
			&2 x - 2 = 0\\
			&f '( x ^{2}- 2 x ) = 0}
		\Leftrightarrow \hoac{&x = 1\\
			&x ^{2}- 2 x = - 2\\
			&x ^{2}- 2 x = - 1\\
			&x ^{2}- 2 x = 3}
		\Leftrightarrow \hoac{&x=1 \\
			&x=-1 \\
			&x=3.}
		$\\
		Ta có bảng xét dấu của $g'(x)$
		\begin{center}
			\begin{tikzpicture}
				\tkzTabInit[nocadre,lgt=1.2,espcl=2,deltacl=0.6]
				{$x$ /0.6,$g'(x)$ /0.6}
				{$-\infty$,$-1$,$1$,$3$,$+\infty$}
				\tkzTabLine{,-,$0$,+,$0$,-,$0$,+,}
			\end{tikzpicture}
		\end{center}
		Dựa vào bảng xét dấu ta có hàm số $y=g(x)$ nghịch biến trên các khoảng $(-\infty ;-1)$ và $(1 ; 3)$.}
\end{ex}
\begin{ex}[Liên trường huyện Quảng Xương - Thanh Hóa - 2021]%[2D1G1-2]
	\immini{
		Cho các hàm số $y=f(x)$; $y=g(x)$ liên tục trên $\mathbb{R}$ và có đồ thị các đạo hàm $f'(x) ; g'(x)$ (đồ thị hàm số $y=g'(x)$ là đường đậm hơn) như hình vẽ.\\
		Hàm số $h(x)=f(x-1)-g(x-1)$ nghịch biến trên khoảng nào dưới đây?
		\choice
		{$\left(\dfrac{1}{2}; 1\right)$}
		{$(1 ;+\infty)$}
		{$(2 ;+\infty)$}
		{\True $\left(-1 ; \dfrac{1}{2}\right)$}
	}
	{
		\begin{tikzpicture}[scale=1,>=stealth, font=\footnotesize, line join=round, line cap=round]
			%\def\a{1} \def\b{-6} \def\c{9} \def\d{1} % Hệ số
			\def\xmin{-4} \def\xmax{3}
			\def\ymin{-2} \def\ymax{4} 
			%\draw[color=gray!50,dashed] (\xmin,\ymin) grid (\xmax,\ymax); 
			\draw[->] (\xmin,0)--(\xmax,0) node [below]{$x$};
			\draw[->] (0,\ymin)--(0,\ymax) node [left]{$y$};
			\node at (0,0) [above left]{$O$};
			\node at (1,3) [below left]{$f'(x)$};
			\node at (1.5,3) [below right]{$g'(x)$};
			\draw[dashed] (-2,0) node[above right]{$-2$}--(-2,1);
			\draw[dashed] (1,0) node[below]{$1$}--(1,1);
			\draw[dashed] (-0.5,0) node[below]{$-0{,}5$}--(-0.5,2.125);
			\clip (\xmin+0.1,\ymin+0.1) rectangle (\xmax-0.5,\ymax-0.1);
			\draw[smooth,samples=300][domain=-3:2] plot(\x,{2*(\x)^4+4*(\x)^3-2*(\x)^2-4*(\x)+1});
			\draw[smooth,samples=300,line width=1.2pt] plot(\x,{(\x)^3+(\x)^2-2*(\x)+1});
		\end{tikzpicture}
	}
	
	\loigiai{
		Ta có: $h'(x)=f'(x-1)-g'(x-1)$.\\
		Dựa vào hình vẽ ta có hàm số $h(x)$ nghịch biến\\
		$\Leftrightarrow h'(x)<0 \Leftrightarrow f'(x-1)<g'(x-1)$\\
		$
		\Leftrightarrow\hoac{&- 2 < x - 1 < - \dfrac{1}{2}\\
			&0 < x - 1 < 1}
		\Leftrightarrow \hoac{
			&-1<x<\dfrac{1}{2}\\
			&1<x<2.}$\\
		Do đó hàm số $h(x)$ nghịch biến trên các khoảng $\left(-1 ; \dfrac{1}{2}\right)$ và $(1 ; 2)$.
	}
\end{ex}
\begin{ex}[THPT Quế Võ 1 - Bắc Ninh - 2021] %[2D1G1-2]
	\immini{
		Cho ba hàm số $y=f(x), y=g(x), y=h(x)$. Đồ thị của ba hàm số $y=f'(x), y=g'(x), y=h'(x)$ được cho như hình vẽ.\\
		Hàm số $k(x)=f(x+7)+g(5 x+1)-h\left(4 x+\dfrac{3}{2}\right)$ đồng biến trên khoảng nào dưới đây?
		\choice
		{$\left(-\dfrac{5}{8}; 0\right)$}
		{$\left(\dfrac{5}{8};+\infty\right)$}
		{\True $\left(\dfrac{3}{8}; 1\right)$}
		{$\left(-\dfrac{3}{8}; 1\right)$}
	}
	{
		\begin{tikzpicture}[scale=0.25,>=stealth, font=\footnotesize, line join=round, line cap=round]
			\def\a{-.078} \def\b{1.25} \def\c{0} % Hệ số
			\def\xmin{-4} \def\xmax{25}
			\def\ymin{-8} \def\ymax{18}
			
			%\draw[color=gray!50,dashed] (\xmin,\ymin) grid (\xmax,\ymax);
			
			\draw[->] (\xmin,0)--(\xmax,0) node [below]{$x$};
			\draw[->] (0,\ymin)--(0,\ymax) node [left]{$y$};
			\node at (20,14) [below right]{$y=g'(x)$};
			\node at (18,-2) [below left]{$y=h'(x)$};
			\node at (16,5) [below right]{$y=f'(x)$};
			\node at (0,0) [below left]{$O$};
			\draw[dashed] (3,0) node[below]{$3$}--(3,10)--(0,10) node[left]{$10$};
			\draw[dashed] (8,0) node[below]{$8$}--(8,5)--(0,5) node[left]{$5$};
			\draw[dashed] (4,0) node[below]{$4$}--(4,2)--(0,2) node[left]{$2$};
			\clip (\xmin+0.1,\ymin+0.1) rectangle (\xmax-0.5,\ymax-0.1);
			\draw[smooth,samples=300,domain=-2:18] plot(\x,{\a*(\x)^2+\b*(\x)+\c});
			%\draw[smooth,samples=300,domain=-2:25] plot(\x,{0.02*(\x)^3-0.6*(\x)^2+5.16*(\x)});
			\draw[line width=1.2pt] (-2,5)..controls (1.7,1.5) and (4.5,1.6)..(7,2.6);
			\draw[line width=1.2pt] (7,2.6)..controls (9,3.5) and (12,5)..(20,13);
			\draw (-0.5,-2) -- (0,0)--(3,10).. controls +(65:1) and + (-190:1)..(6,15).. controls +(0:1) and + (-180:1)..(14,-1).. controls +(0:1) and + (+80:1)..(19,16);
			
		\end{tikzpicture}
	}
	\loigiai{
		Ta có $k'(x)=f'(x+7)+5 g'(5 x+1)-4 h'\left(4 x+\dfrac{3}{2}\right)$.\\
		Khi $x \in \left( \dfrac{3}{8};1\right)$ thì $\heva{&7{,}375<x+7<8\\&2{,}875<5x+1<6\\&3<4x+\dfrac{4}{3}<5{,}5}\Leftrightarrow \heva{&f'(x+7)>10\\&g'(5x+1)>2 \Rightarrow 5g'(5x+1)>10  \\&h'\left( 4x+\dfrac{3}{2}\right)<5 \Rightarrow -4h'\left( 4x+\dfrac{3}{2}\right) >-20}.$\\
		Do đó $k'(x)=f'(x+7)+5g'(5x+1)-4h'\left( 4x+\dfrac{3}{2}\right)>0$.\\
		Hàm số $k(x)=f(x+7)+g(5 x+1)-h\left(4 x+\dfrac{3}{2}\right)$ đồng biến trên $\left(\dfrac{3}{8}; 1\right)$.
	}
\end{ex}
\begin{ex}[THPT Thanh Chương 1 - Nghệ An- 2021] %[2D1G1-2]
	Cho hàm số $y=f(x)$ liên tục trên $\mathbb{R}$ có bảng xét dấu đạo hàm như sau
	\begin{center}
		\begin{tikzpicture}
			\tkzTabInit[nocadre,lgt=1.2,espcl=2,deltacl=0.6]
			{$x$ /0.6,$f'(x)$ /0.6}
			{$-\infty$,$1$,$2$,$3$,$4$,$+\infty$}
			\tkzTabLine{,-,$0$,+,$0$,+,$0$,-,$0$,+,}
		\end{tikzpicture}
	\end{center}
	Hàm số $y=3f(2x-1)-4x^3+15x^2-18x+1$ đồng biến trên khoảng nào dưới đây?
	\choice
	{$\left(3;+\infty\right)$}
	{\True $\left(1;\dfrac{3}{2}\right)$}
	{$\left(\dfrac{5}{2}; 3\right)$}
	{$\left(2;\dfrac{5}{2}\right)$}
	\loigiai{
		Ta có $y'=6f'(2x-1)-12x^2+30x-18=6\left[f'(2x-1)-2x^2+5x-3\right] $.\\
		Có $f'(2x-1)=0 \Leftrightarrow \hoac{&2x-1=1\\&2x-1=2\\&2x-1=3\\&2x-1=4} \Leftrightarrow \hoac{&x=1\\&x=\dfrac{3}{2}\\&x=2\\&x=\dfrac{5}{2}.}$
		Ta có bảng xét dấu sau
		\begin{center}
			\begin{tikzpicture}
				\tkzTabInit[nocadre,lgt=3.0,espcl=1.5,deltacl=0.6]
				{$x$ /1.0,$f(x)$ /0.6,$f'(2x-1)$ /0.6,$-2x^2+5x-3$/0.6,$g'(x)$/0.6}
				{$-\infty$,$1$,$\dfrac{3}{2}$,$2$,$\dfrac{5}{2}$,$3$,$4$,$+\infty$}
				\tkzTabLine{,-,$0$,+,|,+,$0$,+,|,+,$0$,-,$0$,+,}
				\tkzTabLine{,-,$0$,+,$0$,+,$0$,-,$0$,+,|,+,|,+,}
				\tkzTabLine{,-,$0$,+,$0$,-,|,-,|,-,|,-,|,-,}
				\tkzTabLine{,-,$0$,+,$0$,,?,,|,,?,?,,?,}
			\end{tikzpicture}
		\end{center}
		Dựa vào bảng xét dấu trên, ta kết luận hàm số đã cho đồng biến trên khoảng $\left( 1; \dfrac{3}{2}\right).$
	}
\end{ex}


\begin{ex}%[2D2G4-3] %Câu 27 
	[THPT Hoàng Hoa Thám-Đà Nẵng-2021]
	Cho hàm số $f(x)$ có bảng xét dấu của $f'(x)$ như sau:\\
	\begin{center}
		\begin{tikzpicture}
			\tkzTabInit[lgt=1.2,espcl=2.3]
			{$x$/0.7, $f'(x)$ /.8} % first column
			{$-\infty$,$-3$,$1$, $2$, $+\infty$} % first row
			\tkzTabLine { ,+,0,-,0,+,0,+ }
		\end{tikzpicture}
	\end{center}	
	Hàm số $y=f\left(2-e^x\right)-\dfrac{1}{3}{e^{3x}}+3e^{2x}-5e^x+1$ đồng biến trên khoảng nào dưới đây?
	\choice
	{$\left(0;\dfrac{3}{2}\right)$}
	{$\left(1;3\right)$}
	{\True $\left(-3;0\right)$}
	{$\left(-4;-3\right)$}
	\loigiai{
		Ta có $y'=-e^x.f'\left(2-e^x\right)-e^{3x}+6e^{2x}-5e^x=e^x\left[-f'\left(2-e^x\right)-e^{2x}+6e^x-5\right]$ .\\
		Đặt $t=2-e^x$, ta được\\
		$y'=\left(2-t\right)\left[-f'(t)-\left(2-t\right)^2+6\left(2-t\right)-5\right]=\left(2-t\right)\left[-f'(t)-t^2-2t+3\right]$ .\\
		$y'=0\Leftrightarrow\left(2-t\right)\left[-f'(t)-t^2-2t+3\right]=0\Leftrightarrow
		\hoac{
			& t=2\\ 
			& f'(t)=-t^2-2t+3.}$\\
		Hàm số $g(x)=-x^2-2x+3$ là parabol có trục đối xứng $x=-1$ và cắt trục hoành tại 2 điểm có hoành độ 
		$\hoac{
			& x=1\\ 
			& x=-3
		}$. Suy ra $f'(t)=-t^2-2t+3\Leftrightarrow \hoac{
			& t=1\\ 
			& t=-3. }$\\
		Bảng xét dấu\\
		\begin{center}
			\begin{tikzpicture}
				\tkzTabInit[lgt=3.9,espcl=2,nocadre]
				{$t$/0.7, $2-t$ /0.8, $-f'(t)-t^2-2t+3$ /0.8, $y'$ /0.8} % first column
				{$-\infty$,$-3$,$1$,$2$,$+\infty$} % first row
				\tkzTabLine { ,+,|,+,|,+,z,-, } % second row
				\tkzTabLine {,-,0,+,0,-,|,-,} % third row
				\tkzTabLine {,-,0,+,0,-,0,+,} % last row
			\end{tikzpicture}
		\end{center}
		Dựa vào bảng xét dấu $y'>0,\forall x\in\left(-3;0\right)$.}
\end{ex}


\begin{ex}%[2D1G1-2]%Câu 28 
	[Sở Lạng Sơn 2022] Cho hàm số $f(x)$ có bảng biến thiên như sau:\\
	\begin{center}
		\begin{tikzpicture}
			\tkzTabInit[espcl=2.5,lgt=1,nocadre]
			{$x$/0.7,$y'$/0.7,$y$/3.5}
			{$-\infty$,$1$,$2$,$3$,$4$,$+\infty$}
			\tkzTabLine{,+,0,-,0,+,0,-,0,+,}
			\node (0) at ($(N12)+(0,-3)$) {$-\infty$};
			\node (1) at ($(N22)+(0,-.5)$) {$3$};
			\node (2) at ($(N32)+(0,-1.7)$) {$1$};
			\node (3) at ($(N42)+(0,-0.7)$) {$2$};
			\node (4) at ($(N52)+(0,-2.3)$) {$0$};
			\node (5) at ($(N62)+(0,-.3)$) {$+\infty$};
			%				\node (8) at ($(N42)+(0,-.5)$) {};
			%				\coordinate (9) at ($(N42)!.6!(N53)+ (-0.5,0)$);
			%				\coordinate (6) at ($(T12)!.6!(T13)$);
			%				\coordinate (7) at ($(T22)!.6!(T23)$);
			\draw[-stealth] (0)--(1);
			\draw[-stealth] (1)--(2);
			\draw[-stealth] (2)--(3);
			\draw[-stealth] (1)--(2);
			\draw[-stealth] (3)--(4);
			\draw[-stealth] (4)--(5);
			%				\draw[->,red] (5)--(8);
			%				\draw[->,red] (8)--(9);
			%				\draw[blue,dashed](6)--(7)node[above left]{$y=0$};
		\end{tikzpicture}		
	\end{center}
	Hàm số $y=\left[f(x)\right]^3-3\left[f(x)\right]^2$ đồng biến trên khoảng nào dưới đây?
	\choice
	{$\left(-\infty\,;1\right)$}
	{$\left(1\,;2\right)$}
	{\True $\left(3\,;4\right)$}
	{$\left(2\,;3\right)$}
	\loigiai{
		Ta có $y'=3f'(x)\left[f^2(x)-2f(x)\right]$. 
		Phương trình $y'=0\Leftrightarrow \hoac{
			&{f}'(x)=0\\ 
			& f(x)=0\\ 
			& f(x)=2.
		}$
		\begin{center}
			\begin{tikzpicture}
				\tkzTabInit[espcl=2.5,lgt=1.5]
				{$x$/0.7,$y'$/0.7,$y$/3.5}
				{$-\infty$,$1$,$2$,$3$,$4$,$+\infty$}
				\tkzTabLine{,+,0,-,0,+,0,-,0,+,}
				\node (0) at ($(N12)+(0,-3)$) {$-\infty$};
				\node (1) at ($(N22)+(0,-.3)$) {$3$};
				\node (2) at ($(N32)+(0,-1.7)$) {$1$};
				\node (3) at ($(N42)+(0,-0.8)$) {$2$};
				\node (4) at ($(N52)+(0,-2.3)$) {$0$};
				\node (5) at ($(N62)+(0,-.3)$) {$+\infty$};
				\node (a) at ($(N11)+(0.65,0.35)$) {$a$};
				\node (b) at ($(N11)+(2.0,0.4)$) {$b$};
				\node (c) at ($(N11)+(3.38,0.35)$) {$c$};
				\node (d) at ($(N11)+(11.85,0.4)$) {$d$};
				\node (6) at ($(N12)+(0,-0.8)$) {};
				\node (7) at ($(N62)+(0,-0.8)$) {};
				\node (8) at ($(N12)+(0,-2.3)$) {};
				\node (9) at ($(N62)+(0,-2.3)$) {};
				%				\node (8) at ($(N42)+(0,-.5)$) {};
				%				\coordinate (9) at ($(N42)!.6!(N53)+ (-0.5,0)$);
				\coordinate (A) at ($(0)!.25!(1)$);
				\coordinate (B) at ($(0)!.8!(1)$);
				\coordinate (C) at ($(1)!.35!(2)$);
				\coordinate (D) at ($(4)!.75!(5)$);
				%				\coordinate (7) at ($(T22)!.6!(T23)$);
				\draw[->] (0)--(1);
				\draw[->] (1)--(2);
				\draw[->] (2)--(3);
				\draw[->] (1)--(2);
				\draw[->] (3)--(4);
				\draw[->] (4)--(5);
				%				\draw[->,red] (5)--(8);
				%				\draw[->,red] (8)--(9);
				\draw[blue,dashed](6)--(7)node[below]{$y=2$} (a)--(A) (b)--(B) (c)--(C) (d)--(D);
				\draw[blue,dashed](8)--(9)node[below left]{$y=0$};
			\end{tikzpicture}		
		\end{center}
		Dựa vào bảng biến thiên, ta thấy $f'(x)=0\Leftrightarrow x\in \{ 1\,;2\,;3\,;4 \}$;\\
		$f(x)=0\Leftrightarrow x=a<1$ hoặc $x=4$;\\
		$f(x)=2\Leftrightarrow \hoac{
			& x=b\,\,\left(a<b<1\right)\\ 
			& x=c\in\left(1\,;2\right)\\ 
			& x=3\\ 
			& x=d>4.
		}$ \\
		Ta lập được bảng xét dấu của $y'$ 
		\begin{center}
			\begin{tikzpicture}
				\tkzTabInit[lgt=1.2,espcl=1.5,nocadre]
				{$x$/1, $f(x)$ /.8} % first column
				{$-\infty$,$a$, $b$, $1$,$c$, $2$,$3$, $4$, $d$, $+\infty$} % first row
				\tkzTabLine { ,+,z,-,z,+,z,-,z,+,z,-,z,+,z,-,z,+, } % second row
				%				\tkzTabLine {,-,z,+,t,+,} % third row
				%				\tkzTabLine {,+,d,-,z,+,} % last row
			\end{tikzpicture}
		\end{center}
		Từ bảng xét dấu, ta thấy hàm số đồng biến trên các khoảng \\
		$\left(-\infty;a\right)$, $\left(b;1\right)$, $\left(c;2\right)$, $\left(3;4\right)$ và $(d;+\infty)$.
	}
\end{ex}

\begin{ex}%[2D1G1-2]%Câu 29 
	[THPT Bùi Thị Xuân – Huế-2022] 
	\immini{
		Cho hàm số $y=f(x)$ là hàm đa thức bậc bốn. Đồ thị hàm số $f'(x+2)$ được cho trong hình vẽ bên. Hàm số 
		$$g(x)=4 f\left(x^2\right)-x^6+5 x^4-4 x^2+1$$
		đồng biến trên khoảng nào dưới đây?
		\choice
		{$(-4 ;-3)$}
		{\True $(2 ;+\infty)$}
		{$(-\sqrt{2};\sqrt{2})$}
		{$(-2 ;-1)$}}{
		\begin{tikzpicture}[scale=0.6,font=\footnotesize, line join=round, line cap=round, >=stealth] %Đường cong bậc 3
			\draw[thick, ->] (-5.3,0)--(5,0);
			\draw[thick, ->] (0,-3.5)--(0,7);
			\draw (5.2,0) node[below] {$x$};
			\draw (0,7.1) node[left]{$y$};
			\draw (0,0) node[below left]{$0$};
			\draw[fill] (-2,0) circle (0.5pt)node[below left]{$ -2 $};
			\draw[fill] (2,0) circle (0.5pt)node[below]{$ 2$};
			\draw[fill] (0,3) circle (0.5pt)node[left]{$ 3 $};
			\draw[fill] (0,1) circle (0.5pt)node[right]{$ 1 $};
			\draw[fill] (0,-1) circle (0.5pt)node[right]{$ -1 $};
			\draw[dashed] (-2,0)--(-2,1) --(0,1); 
			\draw[dashed](2,0)--(2,3)--(0,3);
			\draw[line width=1.2pt,smooth,samples=100,domain=-2.8:4.5] plot(\x,{-0.271*(\x)^3+0.75*(\x)^2+1.583*\x-1});
		\end{tikzpicture}		
	}
	\loigiai{
		$\begin{aligned}
			& g(x)=4f\left(x^2\right)-x^6+5x^4-4x^2+1\Rightarrow g' (x)=8xf'\left(x^2\right)-6x^5+20x^3-8x.\\ 
			& g' (x)=0\Leftrightarrow 8xf'\left(x^2\right)-6x^5+20x^3-8x=0 \\
			& \Leftrightarrow 2x\left[4f'\left(x^2\right)-3x^4+10x^2-4\right]=0\\ 
			&\Leftrightarrow 		\hoac{ 			& 2x=0\\ 
				& 4f'(x^2)-3x^4+10x^2-4=0
			}
			\Leftrightarrow \hoac{	& x=0\\ 
				& f'\left(x^2\right)=\dfrac{3}{4}{x^4}-\dfrac{5}{2}{x^2}+1.}
		\end{aligned}$\\ 
		Xét
		$f'\left(x^2\right)=\dfrac{3}{4}x^4-\dfrac{5}{2}x^2+1$. Đặt $x^2=t+2$, ta có\\
		$ f' (t+2)=\dfrac{3}{4}{(t+2)^2}-\dfrac{5}{2}(t+2)+1=\dfrac{3}{4}\left(t^2+4t+4\right)-\dfrac{5}{2}(t+2)-1=\dfrac{3}{4}{t^2}+\dfrac{1}{2}t-1$\\
		Khi đó số nghiệm của phương trình chính là số giao điểm của đồ thị hàm số $y=f' (t+2)$ và\\
		$ y=\dfrac{3}{4}{t^2}+\dfrac{1}{2}t-1$\\
		Ta có đồ thị 
		\begin{center}
			\begin{tikzpicture}[scale=0.6,font=\footnotesize, line join=round, line cap=round, >=stealth] %Đường cong bậc 3
				\draw[thick, ->] (-5.3,0)--(5,0);
				\draw[thick, ->] (0,-3.5)--(0,7);
				\draw (5.2,0) node[below] {$x$};
				\draw (0,7.1) node[left]{$y$};
				\draw (0,0) node[below left]{$0$};
				\draw[fill] (-2,0) circle (0.5pt)node[below left]{$ -2 $};
				\draw[fill] (2,0) circle (0.5pt)node[below]{$ 2$};
				\draw[fill] (0,3) circle (0.5pt)node[left]{$ 3 $};
				\draw[fill] (0,1) circle (0.5pt)node[right]{$ 1 $};
				\draw[fill] (0,-1) circle (0.5pt)node[right]{$ -1 $};
				\draw[dashed] (-2,0)--(-2,1) --(0,1); 
				\draw[dashed](2,0)--(2,3)--(0,3);
				\draw[line width=1.2pt,smooth,samples=100,domain=-2.8:4.5] plot(\x,{-0.271*(\x)^3+0.75*(\x)^2+1.583*\x-1});		
				\draw[line width=1.2pt,smooth,samples=100,domain=-3.3:2.8] plot(\x,{0.75*(\x)^2+0.5*\x-1});
			\end{tikzpicture}
		\end{center}
		Dựa vào đồ thị ta có $f' (t+2)=\dfrac{3}{4}t^2+\dfrac{1}{2}t-1\Leftrightarrow \hoac{& t=-2\\ & t=0\\ & t=2} \Leftrightarrow\hoac{& x+2=-2\\ & x+2=0\\ & x+2=2} \Leftrightarrow \hoac{& x=-4\\ & x=-2\\ & x=0.}$\\
		Ta có bảng xét dấu $g' (x)$ như sau
		\begin{center}
			\begin{tikzpicture}
				\tkzTabInit[lgt=1.2,espcl=2,nocadre]
				{$x$/0.7, $f(x)$ /.7}
				{$-\infty$, $-4$,$-2$, $0$, $+\infty$} % first row
				\tkzTabLine { ,-,z,+,z,-,z,+, }
			\end{tikzpicture}
		\end{center}
		Vậy hàm số $g(x)=4 f\left(x^2\right)-x^6+5 x^4-4 x^2+1$ đồng biến trên khoảng $(2 ;+\infty)$.}
\end{ex}

\begin{ex}%[2D1G1-2]%Câu 30
	[Chuyên Bắc Ninh 2022] 
	\immini{
		Cho hàm số $ y=f(x)$ liên tục trên $\mathbb{R}$ có đồ thị hàm số $ y=f'(x)$ có đồ thị như hình vẽ bên.
		Hàm số $g(x)=2f\left(\left| x-1\right|\right)-x^2+2x+2020$ đồng biến trên khoảng nào
		\choice
		{$\left(-2;0\right)$}
		{$\left(-3;1\right)$}
		{$\left(1\,;3\right)$}
		{\True $\left(0\,;\,1\right)$}}{
		\begin{tikzpicture}[scale=0.6,font=\footnotesize, line join=round, line cap=round, >=stealth] %Đường cong bậc 3
			\draw[thick, ->] (-3.3,0)--(5,0);
			\draw[thick, ->] (0,-3.0)--(0,5.5);
			\draw (5.2,0) node[below] {$x$};
			\draw (0,5.8) node[left]{$y$};
			\draw (0,0) node[below left]{$0$};
			\draw[fill] (-1,0) circle (0.5pt)node[above]{$ -1 $};
			\draw[fill] (1,0) circle (0.5pt)node[below]{$ 1$};
			\draw[fill] (0,1) circle (0.5pt)node[left]{$ 1 $};
			\draw[fill] (0,-1) circle (0.5pt)node[right]{$ -1 $};
			\draw[fill] (0,3) circle (0.5pt)node[left]{$ 3 $};
			\draw[fill] (3,0) circle (0.5pt)node[below]{$ 3 $};
			\draw[dashed] (-1,0)--(-1,-1) --(0,-1); 
			\draw[dashed](1,0)--(1,1)--(0,1);
			\draw[dashed](3,0)--(3,3)--(0,3);
			\draw[line width=1.2pt,smooth,samples=100,domain=-2.2:4.3] plot(\x,{-0.333*(\x)^3+1*(\x)^2+1.333*\x-1});		
			%\draw[line width=1.2pt,smooth,samples=100,domain=-3.3:2.8] plot(\x,{0.75*(\x)^2+0.5*\x-1});
		\end{tikzpicture}	
	}
	\loigiai{
		Ta có $g(x)=2f\left(\left| x-1\right|\right)-x^2+2x+2020\Leftrightarrow g(x)=2f\left(\left| x-1\right|\right)-\left(x-1\right)^2+2021$.\\
		Xét hàm số $ k\left(x-1\right)=2f\left(x-1\right)-\left(x-1\right)^2+2021$.\\
		Đặt $ t=x-1$\\
		Xét hàm số $ h(t)=2f(t)-t^2+2021$ $\Rightarrow{h}'(t)=2f'(t)-2t$.\\
		Kẻ đường $ y=x$ như hình vẽ.
		\begin{center}
			\begin{tikzpicture}[scale=0.6,font=\footnotesize, line join=round, line cap=round, >=stealth] %Đường cong bậc 3
				\draw[thick, ->] (-3.3,0)--(5,0);
				\draw[thick, ->] (0,-3.0)--(0,5.5);
				\draw (5.2,0) node[below] {$x$};
				\draw (0,5.8) node[left]{$y$};
				%	\draw (0,0) node[below left]{$0$};
				\draw[fill] (-1,0) circle (0.5pt)node[above]{$ -1 $};
				\draw[fill] (1,0) circle (0.5pt)node[below]{$ 1$};
				\draw[fill] (0,1) circle (0.5pt)node[left]{$ 1 $};
				\draw[fill] (0,-1) circle (0.5pt)node[right]{$ -1 $};
				\draw[fill] (0,3) circle (0.5pt)node[left]{$ 3 $};
				\draw[fill] (3,0) circle (0.5pt)node[below]{$ 3 $};
				\draw[dashed] (-1,0)--(-1,-1) --(0,-1); 
				\draw[dashed](1,0)--(1,1)--(0,1);
				\draw[dashed](3,0)--(3,3)--(0,3);
				\draw[line width=1.2pt,smooth,samples=100,domain=-2.2:4.3] plot(\x,{-0.333*(\x)^3+1*(\x)^2+1.333*\x-1});		
				%\draw[line width=1.2pt,smooth,samples=100,domain=-3.3:2.8] plot(\x,{0.75*(\x)^2+0.5*\x-1});
				\draw[line width=1.2pt,smooth,samples=100](-2,-2)--(4,4);
			\end{tikzpicture}
		\end{center}
		Khi đó $h'(t)>0\Leftrightarrow{f}'(t)-t>0\Leftrightarrow{f}'(t)>t$$\Leftrightarrow \hoac{
			& t<-1\\ 
			& 1<t<3.
		}$\\
		Do đó $k'\left(x-1\right)>0\Leftrightarrow \hoac{
			& x-1<-1\\ 
			& 1<x-1<3} \Leftrightarrow \hoac{
			& x<0\\ 
			& 2<x<4.}$\\
		Ta có bảng biến thiên của hàm số $ k\left(x-1\right)=2f\left(x-1\right)-\left(x-1\right)^2+2021$.
		\begin{center}
			\begin{tikzpicture}
				\tkzTabInit[lgt=1.8,espcl=2.3]
				{$x$ /1.2, $k'(x-1)$ /1.2,$k(x-1)$ /2}
				{$-\infty$ , $0$,$2$,$4$, $+\infty$}
				\tkzTabLine{,+,0,-,0,+,0,-,}
				\tkzTabVar{-/$ $ ,+/$ $, -/$ $,+/$ $,-/$ $}
			\end{tikzpicture}
		\end{center}
		Khi đó, ta có bảng biến thiên của $g(x)=2f\left(\left| x-1\right|\right)-\left(x-1\right)^2+2021$ bằng cách lấy đối xứng qua đường thẳng $ x=1$ như sau\\
		\begin{center}
			\begin{tikzpicture}
				\tkzTabInit[lgt=1.2,espcl=2.5,nocadre]
				{$x$ /0.7, $g'(x)$ /0.7,$g(x)$ /2.5}
				{$-\infty$ ,$-2$, $0$,$1$,$2$,$4$, $+\infty$}
				\tkzTabLine{,+,0,-,0,+,0,-,0,+,0,-,}
				\tkzTabVar{-/$ $ ,+/$ $, -/$ $,+/$ $,-/$ $,+/ $ $,-/$ $}
			\end{tikzpicture}
		\end{center}
		Vậy hàm số đồng biến trên $\left(0;1\right)$.}
\end{ex}

\begin{ex}%[2D1G1-2]%Câu 31
	[Chuyên Thái Bình 2022] 
	\immini{
		Cho hàm số $f(x)=a{x^4}+b{x^3}+c{x^2}+dx+a$ có đồ thị hàm số $y=f'(x)$ như hình vẽ bên. Hàm số $y=g(x)=f\left(1-2x\right)f\left(2-x\right)$ đồng biến trên khoảng nào dưới đây?
		\choice
		{$\left(\dfrac{1}{2};\dfrac{3}{2}\right)$}
		{$\left(-\infty ;0\right)$}
		{$\left(0;2\right)$}
		{\True $\left(3;+\infty\right)$}}{
		\begin{tikzpicture}[scale=0.9,font=\footnotesize, line join=round, line cap=round, >=stealth] %Đường cong bậc 3
			\draw[thick, ->] (-2.5,0)--(2.5,0);
			\draw[thick, ->] (0,-2.8)--(0,2.8);
			\draw (2.6,0) node[below] {$x$};
			\draw (0,2.9) node[left]{$y$};
			\draw (0,0) node[below left]{$0$};
			\draw[fill] (-1,0) circle (0.5pt)node[below left]{$ -1 $};
			\draw[fill] (1,0) circle (0.5pt)node[below right]{$ 1$};
			%			\draw[dashed] (-1,0)--(-1,-1) --(0,-1); 
			%			\draw[dashed](1,0)--(1,1)--(0,1);
			%			\draw[dashed](3,0)--(3,3)--(0,3);
			\draw[line width=1.2pt,smooth,samples=100,domain=-1.3:1.3] plot(\x,{3*(\x)^3-3*\x});		
			%\draw[line width=1.2pt,smooth,samples=100,domain=-3.3:2.8] plot(\x,{0.75*(\x)^2+0.5*\x-1});
		\end{tikzpicture}	
	}
	\loigiai{
		Ta có $f'(x)=4a{x^3}+3b{x^2}+2cx+d$, theo đồ thị thì đa thức $f'(x)$ có ba nghiệm phân biệt là $-1,0,1$ nên $f'(x)=4ax\left(x+1\right)\left(x-1\right)=4a{x^3}-4ax\Rightarrow f(x)=a{x^4}-2a{x^2}+a=a{\left(x^2-1\right)^2}$.\\
		Dựa vào đồ thị hàm số $y=f'(x)$ ta có $a>0$ nên $f(x)>0,\forall x\in\mathbb{R}\setminus\left\{\pm 1\right\}$.\\
		$g'(x)=\left[f\left(1-2x\right)\right]'f\left(2-x\right)+f\left(1-2x\right)\left[f\left(2-x\right)\right]'=-2f'\left(1-2x\right)f\left(2-x\right)-f\left(1-2x\right)f'\left(2-x\right)$. Xét $x\in\left(\dfrac{1}{2};\dfrac{3}{2}\right)\Rightarrow
		\heva{		
			& 1-2x\in\left(-2;0\right)\\ 
			& 2-x\in\left(\dfrac{1}{2};\dfrac{3}{2}\right)}$, dấu của $f'(x)$ không cố định trên $\left(\dfrac{1}{2};\dfrac{3}{2}\right)$ nên ta không kết luận được tính đơn điệu của hàm số $g(x)$ trên $\left(\dfrac{1}{2};\dfrac{3}{2}\right)$.\\
		Xét $x\in\left(-\infty ;0\right)\Rightarrow
		\heva{
			& 1-2x\in\left(1;+\infty\right)\\ 
			& 2-x\in\left(2;+\infty\right)} 
		\Rightarrow \heva{
			& f'\left(1-2x\right)>0\\ 
			& f'\left(2-x\right)>0} \Rightarrow g'(x)<0$.\\
		Do đó, hàm số $g(x)$ nghịch biến trên $\left(-\infty ;0\right)$.\\
		$x\in\left(0;2\right)\Rightarrow \heva{
			& 1-2x\in\left(-3;1\right)\\ 
			& 2-x\in\left(0;2\right)}$, dấu của $f'(x)$ không cố định trên $\left(-3;1\right)$ và $\left(0;2\right)$ nên ta không kết luận được tính đơn điệu của hàm số $g(x)$ trên $\left(\dfrac{1}{2};\dfrac{3}{2}\right)$.\\
		Xét $x\in\left(3;+\infty\right)\Rightarrow \heva{
			& 1-2x\in\left(-\infty ;-5\right)\\ 
			& 2-x\in\left(-\infty ;-1\right)} \Rightarrow \heva{
			& f'\left(1-2x\right)<0\\ 
			& f'\left(2-x\right)<0} \Rightarrow g'(x)>0$. \\
		Do đó, hàm số $g(x)$ đồng biến trên $\left(3;+\infty\right)$.}
\end{ex}

\begin{dang}{Bài toán hàm ẩn, hàm hợp liên quan đến tham số và một số bài toán khác}
\end{dang}

\begin{ex}%[2D1G1-3]%Câu 1
	[Chuyên Lê Hồng Phong Nam Định 2019]
	\immini{
		Cho hàm số $ y=f(x)$ có đạo hàm liên tục trên $\mathbb{R}$. Biết hàm số $ y=f'(x)$ có đồ thị như hình vẽ. Gọi $ S$ là tập hợp các giá trị nguyên $ m\in\left[-5\,;\,\text{5}\right]$ để hàm số $ g(x)=f\left(x+m\right)$ nghịch biến trên khoảng $\left(1\,;\,2\right)$. Hỏi $S$ có bao nhiêu phần tử?
		\choice
		{$ 4$}
		{$ 3$}
		{$ 6$}
		{\True $ 5$}}{
		\begin{tikzpicture}[scale=0.9,font=\footnotesize, line join=round, line cap=round, >=stealth] %Đường cong bậc 3
			\draw[thick, ->] (-2.5,0)--(4,0);
			\draw[thick, ->] (0,-2.8)--(0,2.8);
			\draw (4.3,0) node[below] {$x$};
			\draw (0,2.9) node[left]{$y$};
			\draw (0,0) node[below left]{$0$};
			\draw[fill] (-1,0) circle (0.5pt)node[below left]{$ -1 $};
			\draw[fill] (1,0) circle (0.5pt)node[below]{$ 1$};
			\draw[fill] (3,0) circle (0.5pt)node[below right]{$ 3$};
			%			\draw[dashed] (-1,0)--(-1,-1) --(0,-1); 
			%			\draw[dashed](1,0)--(1,1)--(0,1);
			%			\draw[dashed](3,0)--(3,3)--(0,3);
			\draw[line width=1.2pt,smooth,samples=100,domain=-1.65:3.5] plot(\x,{0.33*(\x)^3-(\x)^2-0.333*(\x)+1});		
			%\draw[line width=1.2pt,smooth,samples=100,domain=-3.3:2.8] plot(\x,{0.75*(\x)^2+0.5*\x-1});
		\end{tikzpicture}	
	}
	\loigiai{
		Ta có $g'(x)=f'\left(x+m\right)$. Vì $ y=f'(x)$ liên tục trên $\mathbb{R}$ nên $g'(x)=f'\left(x+m\right)$ cũng liên tục trên $\mathbb{R}$. Căn cứ vào đồ thị hàm số $ y=f'(x)$ ta thấy\\
		$g'(x)<0\Leftrightarrow{f}'\left(x+m\right)<0$ $\Leftrightarrow\hoac{
			& x+m<-1\\ 
			& 1<x+m<3} \Leftrightarrow \hoac{
			& x<-1-m\\ 
			& 1-m<x<3-m.}$\\
		Hàm số $ g(x)=f\left(x+m\right)$ nghịch biến trên khoảng $\left(1\,;\,2\right)$
		$\Leftrightarrow \hoac{
			& 2\le-1-m\\ 
			&\hoac{
				& 3-m\ge 2\\ 
				& 1-m\le 1}} \Leftrightarrow \hoac{
			& m\le-3\\ 
			& 0\le m\le 1.}$\\
		Mà $ m$ là số nguyên thuộc đoạn $\left[-5\,;\,5\right]$ nên ta có $ S=\left\{-5;-4;-3;0;1\right\}$.\\
		Vậy $ S$ có $5$ phần tử.}
\end{ex}

\begin{ex}%[2D1G1-3]%Câu 2
	[Chuyên Nguyễn Bỉnh Khiêm-Quảng Nam-2020] Cho hàm số $ y=f(x)$ có đạo hàm trên $\mathbb{R}$ và bảng xét dấu đạo hàm như hình vẽ sau
	\begin{center}
		\begin{tikzpicture}
			\tkzTabInit[lgt=1.2,espcl=2.5,nocadre]
			{$x$/0.7, $f'(x)$ /2.5} % first column
			{$-\infty$, $-10$,$-2$, $3$,$8$, $+\infty$} % first row
			\tkzTabLine { ,+,z,-,z,+,z,-,z,+, } % second row
			%				\tkzTabLine {,-,z,+,t,+,} % third row
			%				\tkzTabLine {,+,d,-,z,+,} % last row
		\end{tikzpicture}
	\end{center}
	Có bao nhiêu số nguyên $ m$ để hàm số $ y=f\left(x^3+4x+m\right)$ nghịch biến trên khoảng $\left(-1;1\right)$?
	\choice
	{$ 3$}
	{$ 0$}
	{\True $ 1$}
	{$ 2$}
	\loigiai
	{
		Đặt $ t=x^3+4x+m\Rightarrow{t}'=3x^2+4$ nên $ t$ đồng biến trên $\left(-1;1\right)$ và $ t\in\left(m-5;m+5\right)$.\\
		Yêu cầu bài toán trở thành tìm $ m$ để hàm số $ f(t)$ nghịch biến trên khoảng $\left(m-5;m+5\right)$.\\
		Dựa vào bảng biến thiên ta được $\heva{
			& m-5\ge-2\\ 
			& m+5\le 8} \Leftrightarrow \heva{
			& m\ge 3\\ 
			& m\le 3} \Leftrightarrow m=3$.}
\end{ex}

\begin{ex}%[2D1G1-3]%Câu 3
	[Chuyên ĐH Vinh-Nghệ An-2020]
	\immini{
		Cho hàm số $ f(x)$ có đạo hàm trên $\mathbb{R}$và $ f(1)=1$. Đồ thị hàm số $ y=f'(x)$ như hình bên. Có bao nhiêu số nguyên dương $ a$ để hàm số $ y=\left| 4f\left(\sin x\right)+\cos 2x-a\right|$ nghịch biến trên $\left(0;\dfrac{\pi}{2}\right)$?
		\choice
		{$ 2$}
		{\True $ 3$}
		{Vô số}
		{$ 5$}}{
		\begin{tikzpicture}[scale=0.9,font=\footnotesize, line join=round, line cap=round, >=stealth] %Đường cong bậc 3
			\draw[thick, ->] (-2.5,0)--(3,0);
			\draw[thick, ->] (0,-2.8)--(0,2.8);
			\draw (3.1,0) node[below] {$x$};
			\draw (0,2.9) node[left]{$y$};
			\draw (0,0) node[below left]{$0$};
			\draw[fill] (-1,0) circle (0.5pt)node[below]{$ -1 $};
			\draw[fill] (1,0) circle (0.5pt)node[above]{$ 1$};
			%	\draw[fill] (3,0) circle (0.5pt)node[below right]{$ 3$};
			\draw[dashed] (-1,0)--(-1,1); 
			\draw[dashed](1,0)--(1,-1);
			%			\draw[dashed](3,0)--(3,3)--(0,3);
			\draw[line width=1.2pt,smooth,samples=100,domain=-2:2] plot(\x,{.8*(\x)^3+0*(\x)^2-1.8*(\x)});		
			%\draw[line width=1.2pt,smooth,samples=100,domain=-3.3:2.8] plot(\x,{0.75*(\x)^2+0.5*\x-1});
			\draw (2.0,2.8) node[left]{$y=f'(x)$};
		\end{tikzpicture}	
	}
	\loigiai
	{		Đặt $g(x)=\left| 4f\left(\sin x\right)+\cos 2x-a\right|\Rightarrow g(x)=\sqrt{\left[4f\left(\sin x\right)+\cos 2x-a\right]^2}$ .\\
		$\Rightarrow{g}'(x)=\dfrac{\left[4\cos x\cdot f'\left(\sin x\right)-2\sin 2x\right]\left[4f\left(\sin x\right)+\cos 2x-a\right]}{\sqrt{\left[4f\left(\sin x\right)+\cos 2x-a\right]^2}}$.\\
		Ta có $ 4\cos x\cdot f'\left(\sin x\right)-2\sin 2x=4\cos x\left[f'\left(\sin x\right)-\sin x\right]$.\\
		Với $ x\in\left(0;\dfrac{\pi}{2}\right)$ thì $\cos x>0,\sin x\in\left(0;1\right)\Rightarrow{f}'\left(\sin x\right)-\sin x<0$.\\
		Hàm số $ g(x)$ nghịch biến trên $\left(0;\dfrac{\pi}{2}\right)$ khi $ 4f\left(\sin x\right)+\cos 2x-a\ge 0,\forall x\in\left(0;\dfrac{\pi}{2}\right)$\\
		$\Leftrightarrow 4f\left(\sin x\right)+1-2\sin^2x\ge a,\forall x\in\left(0;\dfrac{\pi}{2}\right)$.\\
		Đặt $ t=\sin x$ được $ 4f(t)+1-2t^2\ge a,\forall t\in\left(0;1\right)$ (*).\\
		Xét $ h(t)=4f(t)+1-2t^2\Rightarrow{h}'(t)=4f'(t)-4t=4\left[f'(t)-1\right]$.\\
		Với $ t\in\left(0;1\right)$ thì $h'(t)<0\Rightarrow h(t)$ nghịch biến trên $\left(0;1\right)$.\\
		Do đó (*) $\Leftrightarrow a\le h(1)=4f(1)+1-2.1^2=3$.\\
		Vậy có $3$ giá trị nguyên dương của a thỏa mãn.}
\end{ex}


\begin{ex}%[2D1G1-3]%Câu 4
	[Chuyên Quang Trung-2020]
	\immini{
		Cho hàm số $ y=f(x)$ có đạo hàm liên tục trên $\mathbb{R}$ và có đồ thị $ y=f'(x)$ như hình vẽ. Đặt $ g(x)=f\left(x-m\right)-\dfrac{1}{2}{\left(x-m-1\right)^2}+2019$, với $ m$ là tham số thực. Gọi $ S$ là tập hợp các giá trị nguyên dương của $ m$ để hàm số $ y=g(x)$ đồng biến trên khoảng $\left(5;6\right)$. Tổng tất cả các phần tử trong $ S$ bằng
		\choice
		{$ 4$}
		{$ 11$}
		{\True $ 14$}
		{$ 20$}}{
		\begin{tikzpicture}[scale=0.9,font=\footnotesize, line join=round, line cap=round, >=stealth] %Đường cong bậc 3
			\draw[style=help lines,step=1] (-2.5,-3) grid (3,3.5);
			\draw[thick, ->] (-2.5,0)--(3.5,0);
			\draw[thick, ->] (0,-2.8)--(0,2.8);
			\draw (3.6,0) node[below] {$x$};
			\draw (0,3) node[above left]{$y$};
			\draw (0,0) node[below left]{$0$};
			%\draw[fill] (-1,0) circle (0.5pt)node[below]{$ -1 $};
			\draw[fill] (1,0) circle (0.5pt)node[below left]{$ 1$};
			%	\draw[fill] (3,0) circle (0.5pt)node[below right]{$ 3$};
			\draw[dashed] (-1,0)--(-1,-2) --(2,-2)--(2,0); 
			\draw[dashed](3,0)--(3,2) --(0,2);
			\draw (-1,-2) circle (2pt);
			\draw (3,2) circle (2pt);
			%			\draw[dashed](3,0)--(3,3)--(0,3);
			\draw[line width=1.2pt,smooth,samples=100,domain=-1.1:3.1] plot(\x,{1*(\x)^3-3*(\x)^2-0*(\x)+2});		
			%\draw[line width=1.2pt,smooth,samples=100,domain=-3.3:2.8] plot(\x,{0.75*(\x)^2+0.5*\x-1});
			%\draw (2.0,2.8) node[left]{$y=f'(x)$};
		\end{tikzpicture}	
	}
	\loigiai
	{
		Xét hàm số $ g(x)=f\left(x-m\right)-\dfrac{1}{2}{\left(x-m-1\right)^2}+2019$.\\
		$g'(x)=f'\left(x-m\right)-\left(x-m-1\right)$.\\
		Xét phương trình $g'(x)=0. \quad \quad (1)$\\
		Đặt $ x-m=t$, phương trình $(1)$ trở thành $f'(t)-\left(t-1\right)=0\Leftrightarrow{f}'(t)=t-1. \quad (2)$\\
		Nghiệm của phương trình $(2)$ là hoành độ giao điểm của hai đồ thị hàm số $ y=f'(t)$ và $ y=t-1$.\\
		Ta có đồ thị các hàm số $ y=f'(t)$ và $ y=t-1$ như sau
		\begin{center}
			\begin{tikzpicture}[scale=0.9,font=\footnotesize, line join=round, line cap=round, >=stealth] %Đường cong bậc 3
				\draw[style=help lines,step=1] (-2.5,-3) grid (3,3.5);
				\draw[thick, ->] (-2.5,0)--(3.5,0);
				\draw[thick, ->] (0,-2.8)--(0,2.8);
				\draw (3.6,0) node[below] {$x$};
				\draw (0,3) node[above left]{$y$};
				\draw (0,0) node[below left]{$0$};
				%\draw[fill] (-1,0) circle (0.5pt)node[below]{$ -1 $};
				\draw[fill] (1,0) circle (0.5pt)node[below left]{$ 1$};
				%	\draw[fill] (3,0) circle (0.5pt)node[below right]{$ 3$};
				\draw[dashed] (-1,0)--(-1,-2) --(2,-2)--(2,0); 
				\draw[dashed](3,0)--(3,2) --(0,2);
				\draw (-1,-2) circle (2pt);
				\draw (3,2) circle (2pt);
				%			\draw[dashed](3,0)--(3,3)--(0,3);
				\draw[line width=1.2pt,smooth,samples=100,domain=-1.1:3.1] plot(\x,{1*(\x)^3-3*(\x)^2-0*(\x)+2});		
				%\draw[line width=1.2pt,smooth,samples=100,domain=-3.3:2.8] plot(\x,{0.75*(\x)^2+0.5*\x-1});
				%\draw (2.0,2.8) node[left]{$y=f'(x)$};
				\draw (-2,-3)--(4,3);
			\end{tikzpicture}
		\end{center}
		Căn cứ đồ thị các hàm số ta có phương trình $(2)$ có nghiệm là $\hoac{
			& t=-1\\ 
			& t=1\\ 
			& t=3} \Rightarrow \hoac{
			& x=m-1\\ 
			& x=m+1\\ 
			& x=m+3.}$\\
		Ta có bảng biến thiên của $ y=g(x)$
		\begin{center}
			\begin{tikzpicture}
				\tkzTabInit[lgt=1,espcl=2.5,nocadre]
				{$x$ /0.8, $y'$ /0.8,$y$ /2.5}
				{$-\infty$ , $m-1$,$m+1$,$m+3$, $+\infty$}
				\tkzTabLine{,+,0,-,0,+,0,-,}
				\tkzTabVar{-/$ +\infty$ ,+/$ $, -/$ $,+/$ $,-/$+\infty $}
			\end{tikzpicture}
		\end{center}
		Để hàm số $ y=g(x)$ đồng biến trên khoảng $\left(5;6\right)$ cần $\hoac{
			&\heva{
				& m-1\le 5\\ 
				& m+1\ge 6}\\ 
			& m+3\le 5}\Leftrightarrow\hoac{
			& 5\le m\le 6\\ 
			& m\le 2.}$\\
		Vì $ m\in\mathbb{N}^*\Rightarrow m$ nhận các giá trị $ 1;\,2;\,5;\,6\Rightarrow S=14$.}
\end{ex}

\begin{ex}%[2D1G1-3]%Câu 5
	[Sở Hà Nội-Lần 2-2020] 
	\immini{
		Cho hàm số $y=a{x^4}+b{x^3}+c{x^2}+dx+e,\,\,a\ne 0$. Hàm số $y=f'(x)$ có đồ thị như hình vẽ bên. 
		Gọi S là tập hợp tất cả các giá trị nguyên thuộc khoảng $\left(-6;6\right)$ của tham số $m$ để hàm số $g(x)=f\left(3-2x+m\right)+x^2-\left(m+3\right)x+2m^2$ nghịch biến trên $\left(0;1\right)$. Khi đó, tổng giá trị các phần tử của S là
		\choice
		{$12$}
		{\True $9$}
		{$6$}
		{$15$}}{
		\begin{tikzpicture}[scale=0.7,font=\footnotesize, line join=round, line cap=round, >=stealth] %Đường cong bậc 3
			%	\draw[style=help lines,step=1] (-2.5,-3) grid (3,3.5);
			\draw[thick, ->] (-4.5,0)--(6.5,0);
			\draw[thick, ->] (0,-2.8)--(0,2.8);
			\draw (6.6,0) node[below] {$x$};
			\draw (0,3) node[above left]{$y$};
			\draw (0,0) node[below left]{$0$};
			\draw[fill] (-2,0) circle (0.5pt)node[below]{$ -2 $};
			\draw[fill] (4,0) circle (0.5pt)node[above]{$ 4$};
			\draw[fill] (0,1) circle (0.5pt)node[right]{$ 1 $};
			\draw[fill] (0,-2) circle (0.5pt)node[left]{$ -2$};
			%	\draw[fill] (3,0) circle (0.5pt)node[below right]{$ 3$};
			\draw[dashed] (-2,0)--(-2,1) --(0,1); 
			\draw[dashed](4,0)--(4,-2) --(0,-2);
			%			\draw[dashed](3,0)--(3,3)--(0,3);
			\draw[line width=1.2pt,smooth,samples=100,domain=-3.8:5.5] plot(\x,{0.0714*(\x)^3-0.1423*(\x)^2-1.0714*(\x)});		
			%\draw[line width=1.2pt,smooth,samples=100,domain=-3.3:2.8] plot(\x,{0.75*(\x)^2+0.5*\x-1});
			%\draw (2.0,2.8) node[left]{$y=f'(x)$};
		\end{tikzpicture}	
	}
	\loigiai
	{
		Xét $g'(x)=-2f'\left(3-2x+m\right)+2x-\left(m+3\right)$.\\
		Xét phương trình $g'(x)=0$, đặt $t=3-2x+m$ thì phương trình trở thành\\ $-2\cdot \left[f'(t)-\dfrac{-t}{2}\right]=0\Leftrightarrow\hoac{
			& t=-2\\ 
			& t=4\\ 
			& t=0.}$ \\
		Từ đó, $g'(x)=0\Leftrightarrow{x_1}=\dfrac{5+m}{2},\,x_2=\dfrac{m+3}{2},x_3=\dfrac{-1+m}{2}$.\\
		Lập bảng xét dấu, đồng thời lưu ý nếu $x>x_1$ thì $t<t_1$ nên $f(x)>0$. Và các dấu đan xen nhau do các nghiệm đều làm đổi dấu đạo hàm nên suy ra $g'(x)\le 0\Leftrightarrow x\in\left[x_2;{x_1}\right]\cup\left(-\infty ;{x_3}\right]$.\\
		Vì hàm số nghịch biến trên $\left(0;1\right)$ nên \\
		$g'(x)\le 0,\,\forall x\in\left(0;1\right)$ từ đó suy ra $\hoac{
			&\dfrac{3+m}{2}\le 0<1\le\dfrac{5+m}{2}\\ 
			& 1\le\dfrac{-1+m}{2}.}$ \\
		và giải ra các giá trị nguyên thuộc $\left(-6;6\right)$ của $m$ là $-3$; $3$; $4$; $5$. }
\end{ex}

\begin{ex}%[2D1G1-3]%Câu 6
	[Chuyên Quang Trung-Bình Phước-Lần 2-2020]
	\immini{
		Cho hàm số $ y=f(x)$ có đạo hàm liên tục trên $\mathbb{R}$ và có đồ thị $ y=f'(x)$ như hình vẽ bên. Đặt $ g(x)=f\left(x-m\right)-\dfrac{1}{2}{\left(x-m-1\right)^2}+2019$, với $ m$ là tham số thực. Gọi $ S$ là tập hợp các giá trị nguyên dương của $ m$ để hàm số $ y=g(x)$ đồng biến trên khoảng $\left(5;6\right)$. Tổng tất cả các phần tử trong $ S$ bằng
		\choice
		{$ 4$}
		{$ 11$}
		{\True $ 14$}
		{$ 20$}}{
		\begin{tikzpicture}[scale=0.9,font=\footnotesize, line join=round, line cap=round, >=stealth] %Đường cong bậc 3
			\draw[thick, ->] (-2.5,0)--(3.7,0);
			\draw[thick, ->] (0,-2.8)--(0,2.8);
			\draw (3.9,0) node[below] {$x$};
			\draw (0,2.9) node[left]{$y$};
			\draw (0,0) node[below left]{$0$};
			\draw[fill] (-1,0) circle (0.5pt)node[above]{$ -1 $};
			\draw[fill] (1,0) circle (0.5pt)node[below]{$ 1$};
			\draw[fill] (3,0) circle (0.5pt)node[below]{$ 3$};
			\draw[fill] (2,0) circle (0.5pt)node[above]{$ 2$};
			\draw[fill] (0,2) circle (0.5pt)node[above left]{$ 2$};
			\draw[fill] (0,-2) circle (0.5pt)node[below left]{$ -2$};
			\draw[dashed] (-1,0)--(-1,-2)--(2,-2)--(2,0); 
			\draw[dashed](3,0)--(3,2)--(0,2);
			%			\draw[dashed](3,0)--(3,3)--(0,3);
			\draw[line width=1.2pt,smooth,samples=100,domain=-1.1:3.1] plot(\x,{1*(\x)^3-3*(\x)^2-0*(\x)+2});		
			%\draw[line width=1.2pt,smooth,samples=100,domain=-3.3:2.8] plot(\x,{0.75*(\x)^2+0.5*\x-1});
			%	\draw (2.0,2.8) node[left]{$y=f'(x)$};
	\end{tikzpicture}	}
	\loigiai
	{
		Ta có $g'(x)=f'\left(x-m\right)-\left(x-m-1\right)$.\\
		Cho $g'(x)=0\Leftrightarrow{f}'\left(x-m\right)=x-m-1$.\\
		Đặt $ x-m=t\Rightarrow f'(t)=t-1$\\
		Khi đó nghiệm của phương trình là hoành độ giao điểm của đồ thị hàm số $ y=f'(t)$ và và đường thẳng $ y=t-1$.
		\begin{center}
			\begin{tikzpicture}[scale=0.9,font=\footnotesize, line join=round, line cap=round, >=stealth] %Đường cong bậc 3
				\draw[thick, ->] (-2.5,0)--(3.7,0);
				\draw[thick, ->] (0,-2.8)--(0,2.8);
				\draw (3.9,0) node[below] {$x$};
				\draw (0,2.9) node[left]{$y$};
				\draw (0,0) node[below left]{$0$};
				\draw[fill] (-1,0) circle (0.5pt)node[above]{$ -1 $};
				\draw[fill] (1,0) circle (0.5pt)node[below]{$ 1$};
				\draw[fill] (3,0) circle (0.5pt)node[below]{$ 3$};
				\draw[fill] (2,0) circle (0.5pt)node[above]{$ 2$};
				\draw[fill] (0,2) circle (0.5pt)node[above left]{$ 2$};
				\draw[fill] (0,-2) circle (0.5pt)node[below left]{$ -2$};
				\draw[dashed] (-1,0)--(-1,-2)--(2,-2)--(2,0); 
				\draw[dashed](3,0)--(3,2)--(0,2);
				%			\draw[dashed](3,0)--(3,3)--(0,3);
				\draw[line width=1.2pt,smooth,samples=100,domain=-1.1:3.1] plot(\x,{1*(\x)^3-3*(\x)^2-0*(\x)+2});		
				%\draw[line width=1.2pt,smooth,samples=100,domain=-3.3:2.8] plot(\x,{0.75*(\x)^2+0.5*\x-1});
				%	\draw (2.0,2.8) node[left]{$y=f'(x)$};
				\coordinate (a) at ($(-1,-2)!1.2!(3,2)$);
				\coordinate (b) at ($(-1,-2)!-.2!(3,2)$);
				\draw[line width=1.2pt,smooth] (a)--(b);
			\end{tikzpicture}
		\end{center}
		Dựa vào đồ thị hàm số ta có được $f'(t)=t-1\Leftrightarrow\hoac{
			& t=-1\\ 
			& t=1\\ 
			& t=3.} $ \\
		Bảng xét dấu của $g'(t)$
		\begin{center}
			\begin{tikzpicture}
				\tkzTabInit[lgt=1.2,espcl=2.5,nocadre]
				{$t$/1, $g'(x)$ /.8} % first column
				{$-\infty$, $-1$,$1$, $3$, $+\infty$} % first row
				\tkzTabLine { ,-,0,+,0,-,0,+, } % second row
				%				\tkzTabLine {,-,z,+,t,+,} % third row
				%				\tkzTabLine {,+,d,-,z,+,} % last row
			\end{tikzpicture}
		\end{center}
		Từ bảng xét dấu ta thấy hàm số $ g(t)$ đồng biến trên khoảng $\left(-1;1\right)$ và $\left(3;+\infty\right)$.\\
		Hay $\hoac{
			&-1<t<1\\ 
			& t>3}\Leftrightarrow\hoac{
			&-1<x-m<1\\ 
			& x-m>3} \Leftrightarrow\hoac{
			& m-1<x<m+1\\ 
			& x>m+3.}$\\
		Để hàm số $ g(x)$ đồng biến trên khoảng $\left(5;6\right)$ thì $\hoac{
			& m-1\le 5<6\le m+1\\ 
			& m+3\le 5<6} \Leftrightarrow\hoac{
			& 5\le m\le 6\\ 
			& m\le 2.}$\\
		Vì $ m$ là các số nguyên dương nên $ S=\left\{ 1;2;5;6\right\}$.\\
		Vậy tổng tất cả các phần tử của $ S$ là $ 1+2+5+6=14$.}
\end{ex}

\begin{ex}%[2D1G1-3]%Câu 7
	\immini{
		Cho hàm số $ y=f(x)$ liên tục có đạo hàm trên $\mathbb{R}$. Biết hàm số $ f'(x)$ có đồ thị cho như hình vẽ bên. Có bao nhiêu giá trị nguyên của $ m$ thuộc $\left[-2019;2019\right]$ để hàm só $ g(x)=f\left(2019^x\right)-mx+2$ đồng biến trên $\left[0;1\right]$.
		\choice
		{$ 2028$}
		{$ 2019$}
		{$ 2011$}
		{\True $ 2020$}}{
		\begin{tikzpicture}[scale=0.9,font=\footnotesize, line join=round, line cap=round, >=stealth] %Đường cong bậc 3
			\draw[thick, ->] (-3.5,0)--(2.5,0);
			\draw[thick, ->] (0,-2.8)--(0,2.8);
			\draw (2.7,0) node[below] {$x$};
			\draw (0,2.9) node[left]{$y$};
			\draw (0,0) node[below left]{$0$};
			%	\draw[fill] (-1,0) circle (0.5pt)node[above]{$ -1 $};
			\draw[fill] (1,0) circle (0.5pt)node[below right]{$ 1$};
			%		\draw[fill] (3,0) circle (0.5pt)node[below]{$ 3$};
			%		\draw[fill] (2,0) circle (0.5pt)node[above]{$ 2$};
			%		\draw[fill] (0,2) circle (0.5pt)node[above left]{$ 2$};
			%		\draw[fill] (0,-2) circle (0.5pt)node[below left]{$ -2$};
			%		\draw[dashed] (-1,0)--(-1,-2)--(2,-2)--(2,0); 
			%		\draw[dashed](3,0)--(3,2)--(0,2);
			\draw[line width=1.2pt,smooth,samples=100,domain=-3.28:1.32] plot(\x,{0.667*(\x)^3+2*(\x)^2-0.667*(\x)-2});		
			%\draw[line width=1.2pt,smooth,samples=100,domain=-3.3:2.8] plot(\x,{0.75*(\x)^2+0.5*\x-1});
			%	\draw (2.0,2.8) node[left]{$y=f'(x)$};
	\end{tikzpicture}	}
	\loigiai{
		Ta có $ g'(x)=2019^x\ln 2019\cdot f'\left(2019^x\right)-m$.\\
		Ta lại có hàm số $ y=2019^x$ đồng biến trên $\left[0;1\right]$.\\
		Với $ x\in\left[0;1\right]$ thì $2019^x\in\left[1;2019\right]$ mà hàm $ y=f'(x)$ đồng biến trên $\left(1;+\infty\right)$ nên hàm $ y=f'\left(2019^x\right)$ đồng biến trên $\left[0;1\right]$.\\
		Mà $2019^x\ge 1;f'\left(2019^x\right)>0\,\forall\,x\in\left[0;1\right]$ nên hàm $ h(x)=2019^x\ln 2019\cdot f'\left(2019^x\right)$ đồng biến trên $\left[0;1\right]$.\\
		Hay $ h(x)\ge h(0)=0,\forall\,x\in\left[0;1\right]$.\\
		Do vậy hàm số $ g(x)$ đồng biến trên đoạn $\left[0;1\right]$$\Leftrightarrow g'(x)\ge 0,\forall\,x\in\left[0;1\right]$\\
		$\Leftrightarrow m\le{2019^x}\ln 2019.f'\left(2019^x\right),\forall\,x\in\left[0;1\right]$ $\Leftrightarrow m\le\underset{x\in\left[0;1\right]}{\min}\,h(x)=h(0)=0$\\
		Vì $ m$ nguyên và $ m\in\left[-2019;2019\right]\Rightarrow $có $ 2020$ giá trị $ m$ thỏa mãn yêu cầu bài toán.}
\end{ex}

\begin{ex}%[2D1G1-3]%Câu 8
	\immini{
		Cho hàm số $y=f(x)$ có đồ thị $f'(x)\,$ như hình vẽ. Có bao nhiêu giá trị nguyên $m\in\left(-2020\,;\,2020\right)$ để hàm số $g(x)=f\left(2x-3\right)\,-\ln \left(1+x^2\right)-2mx$ đồng biến trên $\left(\dfrac{1}{2};2\right)$?
		\choice
		{$ 2020$}
		{\True $ 2019$}
		{$ 2021$}
		{$ 2018$}}{
		\begin{tikzpicture}[scale=0.9,font=\footnotesize, line join=round, line cap=round, >=stealth] %Đường cong bậc 3
			\draw[thick, ->] (-2.5,0)--(2.5,0);
			\draw[thick, ->] (0,-1.8)--(0,5.8);
			\draw (2.7,0) node[below] {$x$};
			\draw (0,5.9) node[left]{$y$};
			\draw (0,0) node[below left]{$0$};
			\draw[fill] (-2,0) circle (0.5pt)node[below]{$ -2 $};
			\draw[fill] (1,0) circle (0.5pt)node[below]{$ 1$};
			\draw[fill] (-1,0) circle (0.5pt)node[below]{$-1$};
			\draw[fill] (0,4) circle (0.5pt)node[above left]{$ 2$};
			%		\draw[fill] (0,2) circle (0.5pt)node[above left]{$ 2$};
			%		\draw[fill] (0,-2) circle (0.5pt)node[below left]{$ -2$};
			\draw[dashed] (-2,0)--(-2,4)--(1,4)--(1,0); 
			%		\draw[dashed](3,0)--(3,2)--(0,2);
			\draw[line width=1.2pt,smooth,samples=100,domain=-2.1:2.1] plot(\x,{-1*(\x)^3+0*(\x)^2+3*(\x)+2});		
			%\draw[line width=1.2pt,smooth,samples=100,domain=-3.3:2.8] plot(\x,{0.75*(\x)^2+0.5*\x-1});
			%	\draw (2.0,2.8) node[left]{$y=f'(x)$};
	\end{tikzpicture}	}
	\loigiai{
		Ta có $g'(x)=2f'\left(2x-3\right)-\dfrac{2x}{1+x^2}-2m$.\\
		Hàm số $ g(x)$ đồng biến trên $\left(\dfrac{1}{2};2\right)$ khi và chỉ khi \\
		$g'(x)\ge 0,\,\,\forall x\in\left(-1;\,2\right)$\\
		$\Leftrightarrow m\le{f}'\left(2x-3\right)-\dfrac{x}{1+x^2},\,\,\forall x\in\left(\dfrac{1}{2};2\right)$\\
		$\Leftrightarrow m\le\underset{x\in\left[\dfrac{1}{2};2\right]}{\min}\,\left[f'\left(2x-3\right)-\dfrac{x}{1+x^2}\right]$. \, \,  $(1)$\\
		Đặt $ t=2x-3$, khi đó $ x\in\left(\dfrac{1}{2};2\right)\Leftrightarrow t\in\left(-2;\,1\right)$.\\
		Từ đồ thị hàm $f'(x)$ suy ra $f'(t)\ge 0,\,\,\forall t\in\left(-2;1\right)$ và $f'(t)=0$ khi $ t=-1$.\\
		Tức là $f'\left(2x-3\right)\ge 0,\,\,\forall x\in\left(\dfrac{1}{2};\,2\right)$$\Rightarrow\underset{x\in\left[\dfrac{1}{2};2\right]}{\min}\,f'\left(2x-3\right)=0$ khi $ x=1$. $(2)$\\
		Xét hàm số $ h(x)=-\dfrac{x}{1+x^2}$ trên khoảng $\left(\dfrac{1}{2};\,2\right)$.\\
		Ta có $h'(x)=\dfrac{x^2-1}{\left(1+x^2\right)^2}$ và\\
		$h'(x)=0\Leftrightarrow{x^2}-1=0\Leftrightarrow x=\pm 1$.\\
		Bảng biến thiên của hàm số $ h(x)$ trên $\left(\dfrac{1}{2};\,2\right)$ như sau
		\begin{center}
			\begin{tikzpicture}
				\tkzTabInit[lgt=1.2,espcl=2.5,nocadre]
				{$x$ /0.7, $h'(x)$ /0.7,$h(x)$ /2.5}
				{$\dfrac{1}{2}$ , $1$,$2$}
				\tkzTabLine{,-,0,+,}
				\tkzTabVar{+/$  $ ,-/$ \-\dfrac{1}{2} $, +/$ $}
			\end{tikzpicture}
		\end{center}
		Từ bảng biến thiên suy ra $ h(x)\ge-\dfrac{1}{2}$$\Rightarrow\underset{x\in\left[\dfrac{1}{2};2\right]}{\min}\,h(x)=-\dfrac{1}{2}$ khi $ x=1$. \, \,  $(3)$\\
		Từ $(1)$, $(2)$ và $(3)$ suy ra $ m\le-\dfrac{1}{2}$.\\
		Kết hợp với $ m\in\mathbb{Z}$, $ m\in\left(-2020;\,2020\right)$ thì $ m\in\left\{-2019;\,-201;\ldots ;-2;-1\right\}$.\\
		Vậy có tất cả $ 2019$ giá trị $ m$ cần tìm.}
\end{ex}

\begin{ex}%[2D1G1-3]%Câu 9
	Cho hàm số $ f(x)$ liên tục trên $\mathbb{R}$ và có đạo hàm $f'(x)=x^2\left(x-2\right)\left(x^2-6x+m\right)$ với mọi $ x\in\mathbb{R}$. Có bao nhiêu số nguyên $ m$ thuộc đoạn $\left[-2020;2020\right]$ để hàm số $ g(x)=f\left(1-x\right)$ nghịch biến trên khoảng $\left(-\infty ;-1\right)$?
	\choice
	{$ 2016$}
	{$ 2014$}
	{\True $ 2012$}
	{$ 2010$}
	\loigiai{
		Ta có \\
		$g'(x)=f'\left(1-x\right)=-\left(1-x\right)^2\left(-x-1\right)\left[\left(1-x\right)^2-6\left(1-x\right)+m\right]$
		$=\left(x-1\right)^2\left(x+1\right)\left(x^2+4x+m-5\right)$.\\
		Hàm số $ g(x)$ nghịch biến trên khoảng $\left(-\infty ;-1\right)$\\
		$\Leftrightarrow{g}'(x)\le 0,\forall x<-1$ $(*)$, (dấu \lq\lq $=$\rq\rq \, xảy ra tại hữu hạn điểm).\\
		Với $ x<-1$ thì $\left(x-1\right)^2>0$ và $ x+1<0$ nên\\
		$(*)$ $\Leftrightarrow{x^2}+4x+m-5\ge 0,\forall x<-1 \Leftrightarrow m\ge-x^2-4x+5,\forall x<-1$.\\
		Xét hàm số $ y=-x^2-4x+5$ trên khoảng $\left(-\infty ;-1\right)$, ta có bảng biến thiên
		\begin{center}
			\begin{tikzpicture}
				\tkzTabInit[lgt=1.8,espcl=2.3]
				{$x$ /1.2, $y'$ /1.2,$y$ /2}
				{$-\infty$ , $-2$,$-1$}
				\tkzTabLine{,+,0,-,}
				\tkzTabVar{-/$ -\infty $ ,+/$9 $, -/$ 8$}
			\end{tikzpicture}
		\end{center}
		Từ bảng biến thiên suy ra $ m\ge 9$.\\
		Kết hợp với $ m$ thuộc đoạn $\left[-2020;2020\right]$ và $ m$ nguyên nên $ m\in\left\{ 9;10;11;\ldots ;2020\right\}$.\\
		Vậy có $ 2012$ số nguyên $ m$ thỏa mãn đề bài.}
\end{ex}

\begin{ex}%[2D1G1-3]%Câu 10
	\immini{
		Cho hàm số $f(x)$ xác định và liên tục trên $ R$. Hàm số $y=f'(x)$ liên tục trên $\mathbb{R}$ và có đồ thị như hình vẽ bên.
		Xét hàm số $g(x)=f\left(x-2m\right)+\dfrac{1}{2}{\left(2m-x\right)^2}+2020$, với $ m$ là tham số thực. Gọi $ S$ là tập hợp các giá trị nguyên dương của $ m$ để hàm số $ y=g(x)$ nghịch biến trên khoảng $\left(3;4\right)$. Hỏi số phần tử của $ S$ bằng bao nhiêu?
		\choice
		{$4$}
		{\True $2$}
		{$3$}
		{Vô số}}
	{
		\begin{tikzpicture}[scale=0.7,>=stealth, font=\footnotesize, line join=round, line cap=round]
			\def\xmin{-3.5} \def\xmax{4.5}
			\def\ymin{-5.2} \def\ymax{4}
			\clip(\xmin,\ymin) rectangle (\xmax,\ymax);
			\draw[->] (\xmin,0)--(\xmax,0) node [below]{$x$};
			\draw[->] (0,\ymin)--(0,\ymax) node [left]{$y$};
			\node at (0,0) [below left]{$O$};
			\path
			(-3.1,3.7) coordinate (A)
			(-3,3) coordinate (B)
			(0,-2) coordinate (C)
			(0.65,-2) coordinate (D)
			(1,-1) coordinate (E)
			(3,-3) coordinate (F)
			(3.4,-5) coordinate (G);
			\draw[smooth]
			(A)..controls +(-88:0.1) and +(93:.1)..
			(B)..controls +(-87:0.3) and +(-100:8.5)..
			(C)..controls +(75:.8) and +(180:.1)..
			(D)..controls +(0:.1) and +(-105:.3)..
			(E)..controls +(70:2) and +(97:0.4)..
			(F)..controls +(-80:.1) and +(90:0.3)..
			(G);
			\draw[dashed] 
			(-3,0)node[below]{$-3$}|-(0,3)node[right]{$3$}
			(1,0)node[above]{$1$}|-(0,-1)node[left]{$-1$}
			(3,0)node[above]{$3$}|-(0,-3)node[below right]{$-3$};
			\fill 
			(0,-2) circle(1.5pt)
			(-3,3) circle(1.5pt)
			(3,-3) circle(1.5pt)
			(1,-1) circle(1.5pt);
			\node at (2.1,-4) {$y=f'(x)$};
		\end{tikzpicture}
	}
	\loigiai{
		Ta có $g'(x)=f'\left(x-2m\right)-\left(2m-x\right)$.		Đặt $h(x)=f'(x)-\left(-x\right)$.\\
		Từ đồ thị hàm số $y=f'(x)$ và đồ thị hàm số $y=-x$ trên hình vẽ suy ra \\
		$h(x)\le 0\Leftrightarrow f'(x)\le-x\Leftrightarrow\hoac{
			&-3\le x\le 1\\ 
			& x\ge 3.}$ 
		\begin{center}
			\begin{tikzpicture}[scale=0.7,>=stealth, font=\footnotesize, line join=round, line cap=round]
				\def\xmin{-3.5} \def\xmax{4.5}
				\def\ymin{-5.2} \def\ymax{4}
				\clip(\xmin,\ymin) rectangle (\xmax,\ymax);
				\draw[->] (\xmin,0)--(\xmax,0) node [below]{$x$};
				\draw[->] (0,\ymin)--(0,\ymax) node [left]{$y$};
				\node at (0,0) [below left]{$O$};
				\path
				(-3.1,3.7) coordinate (A)
				(-3,3) coordinate (B)
				(0,-2) coordinate (C)
				(0.65,-2) coordinate (D)
				(1,-1) coordinate (E)
				(3,-3) coordinate (F)
				(3.4,-5) coordinate (G);
				\draw[smooth]
				(A)..controls +(-88:0.1) and +(93:.1)..
				(B)..controls +(-87:0.3) and +(-100:8.5)..
				(C)..controls +(75:.8) and +(180:.1)..
				(D)..controls +(0:.1) and +(-105:.3)..
				(E)..controls +(70:2) and +(97:0.4)..
				(F)..controls +(-80:.1) and +(90:0.3)..
				(G);
				\draw[dashed] 
				(-3,0)node[below]{$-3$}|-(0,3)node[right]{$3$}
				(1,0)node[above]{$1$}|-(0,-1)node[left]{$-1$}
				(3,0)node[above]{$3$}|-(0,-3)node[below right]{$-3$};
				\fill 
				(0,-2) circle(1.5pt)
				(-3,3) circle(1.5pt)
				(3,-3) circle(1.5pt)
				(1,-1) circle(1.5pt);
				\draw[smooth,samples=300,domain=-3.2:3.7] plot(\x,{-(\x)});
				\node at (2.1,-4) {$y=f'(x)$};
				\node at (-1,2.1) {$y=h(x)$};
			\end{tikzpicture}
		\end{center}
		Ta có $ g'(x)=h\left(x-2m\right)\le 0\Leftrightarrow\hoac{
			&-3\le x-2m\le 1\\ 
			& x-2m\ge 3}\Leftrightarrow\hoac{
			& 2m-3\le x\le 2m+1\\ 
			& x\ge 2m+3.}$.\\
		Suy ra hàm số $ y=g(x)$ nghịch biến trên các khoảng $\left(2m-3;2m+1\right)$ và $\left(2m+3;+\infty\right)$.\\
		Do đó hàm số $ y=g(x)$ nghịch biến trên khoảng $\left(3;4\right)$ $\Leftrightarrow\hoac{
			&\heva{
				& 2m-3\le 3\\ 
				& 2m+1\ge 4}\\ 
			& 2m+3\le 3}\Leftrightarrow\hoac{
			&\dfrac{3}{2}\le m\le 3\\ 
			& m\le 0.}$ \\
		Mặt khác, do $ m$ nguyên dương nên $ m\in\left\{ 2;3\right\}\Rightarrow S=\left\{ 2;3\right\}$. Vậy số phần tử của $ S$ bằng $2$.\\
	}
	
\end{ex}

\begin{ex}%[2D1G1-3]%Câu 11
	Cho hàm số $f(x)$ có đạo hàm trên $\mathbb{R}$ là $f'(x)=\left(x-1\right)\left(x+3\right)$. Có bao nhiêu giá trị nguyên của tham số $m$ thuộc đoạn $\left[-10;20\right]$ để hàm số $y=f\left(x^2+3x-m\right)$ đồng biến trên khoảng $\left(0;2\right)$?
	\choice
	{\True $ 18$}
	{$ 17$}
	{$ 16$}
	{$ 20$}
	\loigiai{
		Ta có $y'=f'\left(x^2+3x-m\right)=\left(2x+3\right){f}'\left(x^2+3x-m\right)$.\\
		Theo đề bài ta có $f'(x)=\left(x-1\right)\left(x+3\right)$\\
		suy ra $f'(x)>0\Leftrightarrow\hoac{
			& x<-3\\ 
			& x>1}$ và $f'(x)<0\Leftrightarrow-3<x<1$ .\\
		Hàm số đồng biến trên khoảng $\left(0;2\right)$ khi $y'\ge 0,\forall x\in\left(0;2\right)$\\
		$\Leftrightarrow\left(2x+3\right){f}'\left(x^2+3x-m\right)\ge 0,\forall x\in\left(0;2\right)$.\\
		Do $x\in\left(0;2\right)$ nên $2x+3>0,\forall x\in\left(0;2\right)$. Do đó, ta có\\
		$y'\ge 0,\forall x\in\left(0;2\right)\Leftrightarrow f'\left(x^2+3x-m\right)\ge 0$\\
		$\Leftrightarrow\hoac{
			&{x^2}+3x-m\le-3\\ 
			&{x^2}+3x-m\ge 1}\Leftrightarrow\hoac{
			& m\ge{x^2}+3x+3\\ 
			& m\le{x^2}+3x-1}$\\
		$\Leftrightarrow\hoac{
			& m\ge\underset{\left[0;2\right]}{\max}\,\left(x^2+3x+3\right)\\ 
			& m\le\underset{\left[0;2\right]}{\min}\,\left(x^2+3x-1\right)} \Leftrightarrow\hoac{
			& m\ge 13\\ 
			& m\le-1}$.\\
		Do $m\in\left[-10;20\right]$, $ m\in\mathbb{Z}$ nên có $ 18$ giá trị nguyên của $m$ thỏa yêu cầu đề bài.}
\end{ex}

\begin{ex}%[2D1G1-3]%Câu 12
	Cho các hàm số $f(x)=x^3+4x+m$ và $g(x)=\left(x^2+2018\right){\left(x^2+2019\right)^2}{\left(x^2+2020\right)^3}$ . Có bao nhiêu giá trị nguyên của tham số $m\in\left[-2020;2020\right]$ để hàm số $g\left(f(x)\right)$ đồng biến trên $\left(2;+\infty\right)$ ?
	\choice
	{$2005$}
	{\True $2037$}
	{$4016$}
	{$4041$}
	\loigiai{
		Ta có $f(x)=x^3+4x+m$ và \\
		$g(x)=\left(x^2+2018\right){\left(x^2+2019\right)^2}{\left(x^2+2020\right)^3}=a_{12}{x^{12}}+a_{10}{x^{10}}+...+a_2x^2+a_0$.\\
		Suy ra $f'(x)=3x^2+4$ , $g'(x)=12a_{12}{x^{11}}+10a_{10}{x^9}+...+2a_2x$.\\
		Và có 
		\begin{eqnarray*}
			\left[g\left(f(x)\right)\right]' &=& f'(x)\left[12a_{12}{\left(f(x)\right)^{11}}+10a_{10}{\left(f(x)\right)^9}+...+2a_2f(x)\right]\\
			&=& f(x)f'(x)\left(12a_{12}{\left(f(x)\right)^{10}}+10a_{10}{\left(f(x)\right)^8}+...+2a_2\right).
		\end{eqnarray*} 
		Dễ thấy $a_{12};{a_{10}};...;{a_2};{a_0}>0$ và $f'(x)=3x^2+4>0$, $\forall x>2$.\\
		Do đó $f'(x)\left(12a_{12}{\left(f(x)\right)^{10}}+10a_{10}{\left(f(x)\right)^8}+...+2a_2\right)>0$ , $\forall x>2$.\\
		Hàm số $g\left(f(x)\right)$ đồng biến trên $\left(2;+\infty\right)$ khi $\left[g\left(f(x)\right)\right]^{'}\ge 0$, $\forall x>2$\\
		$\Rightarrow  f(x)\ge 0$, $\forall x>2 \Leftrightarrow x^3+4x+m\ge 0$, $\forall x>3 \Leftrightarrow  m\ge-x^3-4x$, $\forall x>2$\\
		$ \Rightarrow  m\ge\underset{\left[2;+\infty\right)}{\max}\,\left(-x^3-4x\right)=-16$.\\
		Vì $m\in\left[-2020;2020\right]$ và $m\in\mathbb{Z}$ nên có $2037$ giá trị thỏa mãn $m$ .}
\end{ex}

\begin{ex}%[2D1G1-3]%Câu 13
	Cho hàm số $y=f(x)$ có đạo hàm $f'(x)=x{\left(x+1\right)^2}\left(x^2+2mx+1\right)$ với mọi $x \in \mathbb{R}$. Có bao nhiêu số nguyên âm $m$ để hàm số $g(x)=f\left(2x+1\right)$ đồng biến trên khoảng $\left(3;5\right)$?
	\choice
	{\True $3$}
	{$2$}
	{$4$}
	{$6$}
	\loigiai{
		Ta có $g'(x)=2f'(2x+1)=2(2x+1)(2x+2)^2[(2x+1)^2+2m(2x+1)+1]$. 	Đặt $t=2x+1$\\
		Để hàm số $g(x)$ đồng biến trên khoảng $\left(3;5\right)$ khi và chỉ khi 
		\begin{eqnarray*}
			& & g'(x)\ge 0,\forall x\in\left(3;5\right) \\
			& \Leftrightarrow & t(t^2+2mt+1)\ge 0,\forall t\in\left(7;11\right)\Leftrightarrow{t^2}+2mt+1\ge 0,\,\,\forall t\in\left(7;11\right) \\
			&\Leftrightarrow & 2m\ge\dfrac{-t^2-1}{t},\,\,\,\forall t\in\left(7;11\right)
		\end{eqnarray*}	
		Xét hàm số $h(t)=\dfrac{-t^2-1}{t}$ trên $\left[7;11\right]$, có $h'(t)=\dfrac{-t^2+1}{t^2}$\\
		Bảng biến thiên
		\begin{center}
			\begin{tikzpicture}
				\tkzTabInit[espcl=3,lgt=1.2,nocadre]
				{$t$/0.7,$h'(t)$/0.7,$h(t)$/2.5}
				{$-\infty$,$1$,$11$,$+\infty$}
				\tkzTabLine{, ,,-,,,}
				%	\node (0) at ($(N12)+(0,-3)$) {$-\infty$};
				\node (1) at ($(N22)+(0,-0.8)$) [right] {$-\dfrac{50}{7}$};
				\node (2) at ($(N32)+(0,-2.5)$) [left] {$-\dfrac{122}{11}$};
				
				
				%				\node (3) at ($(N11+(-0.5,0))$) {};
				%				\node (4) at ($(N23)$) {};
				\fill[pattern=north east lines] (7.0,-0.7) rectangle (10,-4.4);
				\fill[pattern=north east lines] (1.5,-0.7) rectangle (4.5,-4.4);
				\draw[->] (1)--(2);	
				\draw[dashed] (4.5,-0.7)--(4.5,-4.4);
				\draw[dashed] (7.0,-0.7)--(7.0,-4.4);	
			\end{tikzpicture}		
		\end{center}
		Dựa vào BBT ta có $2m\ge\dfrac{-t^2-1}{t},\,\,\,\forall t\in\left(7;11\right)\Leftrightarrow 2m\ge\underset{\left[7;11\right]}{\max}\,h(t)\Leftrightarrow m\ge-\dfrac{50}{14}$\\
		Vì $ m\in{\mathbb{Z}^-}\Rightarrow m \in \{-3;-2;-1\}$ .
	}
\end{ex}

\begin{ex}%[2D1G1-3]%Câu 14
	Cho hàm số $y=f(x)$ có bảng biến thiên như sau\\
	\begin{center}
		\begin{tikzpicture}[>=stealth,scale = 1]
			\tkzTabInit[lgt=1,espcl=2.5,nocadre]
			{$x$ /0.7, $y'$ /0.7,$y$ /2.5}
			{$-\infty$,$0$,$2$,$+\infty$}
			\tkzTabLine{ ,-,0,+,0,-,}
			\tkzTabVar{-/$-\infty$, +/$4$,- /$0$, +/{ $+\infty$}}
		\end{tikzpicture}
	\end{center}
	Có bao nhiêu số nguyên $m<2019$ để hàm số $g(x)=f\left(x^2-2x+m\right)$ đồng biến trên khoảng $\left(1;+\infty\right)$?
	\choice
	{\True $2016$}
	{$2015$}
	{$2017$}
	{$2018$}
	\loigiai{
		Ta có $g'(x)=\left(x^2-2x+m\right)'{f}'\left(x^2-2x+m\right)=2\left(x-1\right){f}'\left(x^2-2x+m\right)$ .\\
		Hàm số $y=g(x)$ đồng biến trên khoảng $\left(1;+\infty\right)$ khi và chỉ khi $g'(x)\ge 0,\forall x\in\left(1;+\infty\right)$ và\\
		$g'(x)=0$ tại hữu hạn điểm \\
		$\Leftrightarrow 2\left(x-1\right){f}'\left(x^2-2x+m\right)\ge 0,\forall x\in\left(1;+\infty\right)$\\
		$\Leftrightarrow{f}'\left(x^2-2x+m\right)\ge 0,\forall x\in\left(1;+\infty\right)$ $\Leftrightarrow\hoac{
			&{x^2}-2x+m\ge 2,\forall x\in\left(1;+\infty\right)\\ 
			&{x^2}-2x+m\le 0,\forall x\in\left(1;+\infty\right).}$\\
		Xét hàm số $y=x^2-2x+m$, ta có bảng biến thiên
		\begin{center}
			\begin{tikzpicture}[>=stealth,scale = 1]
				\tkzTabInit[lgt=1,espcl=2.5,nocadre]
				{$x$ /0.7, $y'$ /0.7,$y$ /2.5}
				{$-\infty$,$2$,$+\infty$}
				\tkzTabLine{ ,-,0,+,}
				\tkzTabVar{+/$+\infty$, -/$m-1$, +/{$+\infty$}}
			\end{tikzpicture}
		\end{center}
		Dựa vào bảng biến thiên ta có\\
		TH1: $x^2-2x+m\ge 2,\forall x\in\left(1;+\infty\right)\Leftrightarrow m-1\ge 2\Leftrightarrow m\ge 3$ .\\
		TH2: $x^2-2x+m\le 0,\forall x\in\left(1;+\infty\right)$. Không có giá trị $m$ thỏa mãn.\\
		Vậy có $2016$ số nguyên $m<2019$ thỏa mãn yêu cầu bài toán.}
\end{ex}

\begin{ex}%[2D1G1-3]%Câu 15
	\immini{
		Cho hàm số $ y=f(x)$ có đạo hàm là hàm số $f'(x)$ trên $\mathbb{R}$. Biết rằng hàm số $ y=f'\left(x-2\right)+2$ có đồ thị như hình vẽ bên dưới. Hàm số $ f(x)$ đồng biến trên khoảng nào?
		\choice
		{$\left(-\infty ;3\right),\,\,\left(5;+\infty\right)$}
		{\True $\left(-\infty ;-1\right),\,\,\left(1;+\infty\right)$}
		{$\left(-1;1\right)$}
		{$\left(3;5\right)$}}{
		\begin{tikzpicture}[scale=0.7,font=\footnotesize, line join=round, line cap=round, >=stealth] %Đường cong bậc 3
			\draw[thick, ->] (-0.5,0)--(3.5,0);
			\draw[thick, ->] (0,-1.8)--(0,5.3);
			\draw (3.7,0) node[below] {$x$};
			\draw (0,5.4) node[left]{$y$};
			\draw (0,0) node[below left]{$0$};
			\draw[fill] (3,0) circle (0.5pt)node[below]{$ 3$};
			\draw[fill] (1,0) circle (0.5pt)node[below]{$ 1$};
			\draw[fill] (2,0) circle (0.5pt)node[above]{$2$};
			\draw[fill] (0,2) circle (0.5pt)node[left]{$ 2$};
			\draw[fill] (0,-1) circle (0.5pt)node[left]{$ -1$};
			%		\draw[fill] (0,2) circle (0.5pt)node[above left]{$ 2$};
			%		\draw[fill] (0,-2) circle (0.5pt)node[below left]{$ -2$};
			\draw[dashed] (3,0)--(3,2)--(0,2)--(1,2)--(1,0); 
			\draw[dashed](0,-1)--(2,-1)--(2,0);
			\draw[line width=1.2pt,smooth,samples=100,domain=0.6:3.4] plot(\x,{3*(\x)^2-12*(\x)+11});		
			%\draw[line width=1.2pt,smooth,samples=100,domain=-3.3:2.8] plot(\x,{0.75*(\x)^2+0.5*\x-1});
			%	\draw (2.0,2.8) node[left]{$y=f'(x)$};
	\end{tikzpicture}	}
	\loigiai{	
		Hàm số $ y=f'\left(x-2\right)+2$ có đồ thị $(C)$ như sau:\\
		\begin{center}
			\begin{tikzpicture}[scale=0.7,font=\footnotesize, line join=round, line cap=round, >=stealth] %Đường cong bậc 3
				\draw[thick, ->] (-0.5,0)--(3.5,0);
				\draw[thick, ->] (0,-1.8)--(0,5.3);
				\draw (3.7,0) node[below] {$x$};
				\draw (0,5.4) node[left]{$y$};
				\draw (0,0) node[below left]{$0$};
				\draw[fill] (3,0) circle (0.5pt)node[below]{$ 3$};
				\draw[fill] (1,0) circle (0.5pt)node[below]{$ 1$};
				\draw[fill] (2,0) circle (0.5pt)node[above]{$2$};
				\draw[fill] (0,2) circle (0.5pt)node[left]{$ 2$};
				\draw[fill] (0,-1) circle (0.5pt)node[left]{$ -1$};
				%		\draw[fill] (0,2) circle (0.5pt)node[above left]{$ 2$};
				%		\draw[fill] (0,-2) circle (0.5pt)node[below left]{$ -2$};
				\draw[dashed] (3,0)--(3,2)--(0,2)--(1,2)--(1,0); 
				\draw[dashed](0,-1)--(2,-1)--(2,0);
				\draw[line width=1.2pt,smooth,samples=100,domain=0.6:3.4] plot(\x,{3*(\x)^2-12*(\x)+11});		
				%\draw[line width=1.2pt,smooth,samples=100,domain=-3.3:2.8] plot(\x,{0.75*(\x)^2+0.5*\x-1});
				%	\draw (2.0,2.8) node[left]{$y=f'(x)$};
			\end{tikzpicture}
		\end{center}
		Dựa vào đồ thị $(C)$ ta có\\ $f'\left(x-2\right)+2>2,\forall x\in\left(-\infty ;1\right)\cup\left(3;+\infty\right)\Leftrightarrow{f}'\left(x-2\right)>0,\forall x\in\left(-\infty ;1\right)\cup\left(3;+\infty\right)$ .\\
		Đặt $ x*=x-2$ suy ra $f'\left(x*\right)>0,\forall x*\in\left(-\infty ;-1\right)\bigcup\left(1;+\infty\right)$.\\
		Vậy hàm số $ f(x)$ đồng biến trên khoảng $\left(-\infty ;-1\right),\,\,\left(1;+\infty\right)$.}
\end{ex}

\begin{ex}%[2D1G1-2]%Câu 16
	\immini{
		Cho hàm số $ y=f(x)$ có đạo hàm là hàm số $f'(x)$ trên $\mathbb{R}$. Biết rằng hàm số $ y=f'\left(x+2\right)-2$ có đồ thị như hình vẽ bên dưới. Hàm số $ f(x)$ nghịch biến trên khoảng nào?
		\choice
		{$\left(-3;-1\right),\,\,\left(1;3\right)$}
		{\True $\left(-1;1\right),\,\,\left(3;5\right)$}
		{$\left(-\infty ;-2\right),\,\,\left(0;2\right)$}
		{$\left(-5;-3\right),\,\,\left(-1;1\right)$}}{
		\begin{tikzpicture}[scale=0.7,font=\footnotesize, line join=round, line cap=round, >=stealth] %Đường cong bậc 3
			\draw[thick, ->] (-3.8,0)--(4.0,0);
			\draw[thick, ->] (0,-4.8)--(0,3.5);
			\draw (4.2,0) node[below] {$x$};
			\draw (0,3.7) node[left]{$y$};
			\draw (0,0) node[below left]{$0$};
			\draw[fill] (-3,0) circle (0.5pt)node[above]{$ -3$};
			\draw[fill] (-1,0) circle (0.5pt)node[above]{$ -1$};
			\draw[fill] (1,0) circle (0.5pt)node[above]{$ 1$};
			\draw[fill] (3,0) circle (0.5pt)node[above]{$3$};
			\draw[fill] (0,2) circle (0.5pt)node[above left]{$ 2$};
			\draw[fill] (0,-1) circle (0.5pt)node[above right]{$ -1$};
			%		\draw[fill] (0,2) circle (0.5pt)node[above left]{$ 2$};
			%		\draw[fill] (0,-2) circle (0.5pt)node[below left]{$ -2$};
			\draw[dashed] (-3,0)--(-3,-2)--(3,-2)--(3,0) (-1,0)--(-1,-2) (1,0)--(1,-2) (-3.494,0)--(-3.494,2)--(3.494,2)--(3.494,0); 
			\draw[line width=1.2pt,smooth,samples=100,domain=-3.6:3.6] plot(\x,{0.11*(\x)^4-1.11*(\x)^2-1});		
			%\draw[line width=1.2pt,smooth,samples=100,domain=-3.3:2.8] plot(\x,{0.75*(\x)^2+0.5*\x-1});
			%	\draw (2.0,2.8) node[left]{$y=f'(x)$};
	\end{tikzpicture}	}
	\loigiai{
		Hàm số $ y=f'\left(x+2\right)-2$ có đồ thị $(C)$ như sau
		\begin{center}
			\begin{tikzpicture}[scale=0.7,font=\footnotesize, line join=round, line cap=round, >=stealth] %Đường cong bậc 3
				\draw[thick, ->] (-3.8,0)--(4.0,0);
				\draw[thick, ->] (0,-4.8)--(0,3.5);
				\draw (4.2,0) node[below] {$x$};
				\draw (0,3.7) node[left]{$y$};
				\draw (0,0) node[below left]{$0$};
				\draw[fill] (-3,0) circle (0.5pt)node[above]{$ -3$};
				\draw[fill] (-1,0) circle (0.5pt)node[above]{$ -1$};
				\draw[fill] (1,0) circle (0.5pt)node[above]{$ 1$};
				\draw[fill] (3,0) circle (0.5pt)node[above]{$3$};
				\draw[fill] (0,2) circle (0.5pt)node[above left]{$ 2$};
				\draw[fill] (0,-1) circle (0.5pt)node[above right]{$ -1$};
				%		\draw[fill] (0,2) circle (0.5pt)node[above left]{$ 2$};
				%		\draw[fill] (0,-2) circle (0.5pt)node[below left]{$ -2$};
				\draw[dashed] (-3,0)--(-3,-2)--(3,-2)--(3,0) (-1,0)--(-1,-2) (1,0)--(1,-2) (-3.494,0)--(-3.494,2)--(3.494,2)--(3.494,0); 
				\draw[line width=1.2pt,smooth,samples=100,domain=-3.6:3.6] plot(\x,{0.11*(\x)^4-1.11*(\x)^2-1});		
				%\draw[line width=1.2pt,smooth,samples=100,domain=-3.3:2.8] plot(\x,{0.75*(\x)^2+0.5*\x-1});
				%	\draw (2.0,2.8) node[left]{$y=f'(x)$};
			\end{tikzpicture}
		\end{center}
		Dựa vào đồ thị $(C)$ ta có\\
		$f'\left(x+2\right)-2<-2,\forall x\in\left(-3;-1\right)\bigcup\left(1;3\right)\Leftrightarrow{f}'\left(x+2\right)<0,\forall x\in\left(-3;-1\right)\bigcup\left(1;3\right)$.\\
		Đặt $ x^*=x+2$ suy ra: $f'\left(x^*\right)<0,\forall x^*\in\left(-1;1\right)\bigcup\left(3;5\right)$.\\
		Vậy: Hàm số $ f(x)$ đồng biến trên khoảng $\left(-1;1\right),\,\,\left(3;5\right)$.}
\end{ex}

\begin{ex}%[2D1G1-2]%Câu 17
	\immini{
		Cho hàm số $ y=f(x)$ có đạo hàm là hàm số $f'(x)$ trên $\mathbb{R}$. Biết rằng hàm số $ y=f'\left(x-2\right)+2$ có đồ thị như hình vẽ bên dưới. Hàm số $ f(x)$ nghịch biến trên khoảng nào?
		\choice
		{$\left(-\infty ;2\right)$}
		{\True $\left(-1;1\right)$}
		{$\left(\dfrac{3}{2};\dfrac{5}{2}\right)$}
		{$\left(2;+\infty\right)$}}{
		\begin{tikzpicture}[scale=0.7,font=\footnotesize, line join=round, line cap=round, >=stealth] %Đường cong bậc 3
			\draw[thick, ->] (-0.5,0)--(3.5,0);
			\draw[thick, ->] (0,-1.8)--(0,5.3);
			\draw (3.7,0) node[below] {$x$};
			\draw (0,5.4) node[left]{$y$};
			\draw (0,0) node[below left]{$0$};
			\draw[fill] (3,0) circle (0.5pt)node[below]{$ 3$};
			\draw[fill] (1,0) circle (0.5pt)node[below]{$ 1$};
			\draw[fill] (2,0) circle (0.5pt)node[above]{$2$};
			\draw[fill] (0,2) circle (0.5pt)node[left]{$ 2$};
			\draw[fill] (0,-1) circle (0.5pt)node[left]{$ -1$};
			%		\draw[fill] (0,2) circle (0.5pt)node[above left]{$ 2$};
			%		\draw[fill] (0,-2) circle (0.5pt)node[below left]{$ -2$};
			\draw[dashed] (3,0)--(3,2)--(0,2)--(1,2)--(1,0); 
			\draw[dashed](0,-1)--(2,-1)--(2,0);
			\draw[line width=1.2pt,smooth,samples=100,domain=0.6:3.4] plot(\x,{3*(\x)^2-12*(\x)+11});		
			%\draw[line width=1.2pt,smooth,samples=100,domain=-3.3:2.8] plot(\x,{0.75*(\x)^2+0.5*\x-1});
			%	\draw (2.0,2.8) node[left]{$y=f'(x)$};
	\end{tikzpicture}	}
	\loigiai{
		Hàm số $ y=f'\left(x-2\right)+2$ có đồ thị $(C)$ như sau
		\begin{center}
			\begin{tikzpicture}[scale=0.7,font=\footnotesize, line join=round, line cap=round, >=stealth] %Đường cong bậc 3
				\draw[thick, ->] (-0.5,0)--(3.5,0);
				\draw[thick, ->] (0,-1.8)--(0,5.3);
				\draw (3.7,0) node[below] {$x$};
				\draw (0,5.4) node[left]{$y$};
				\draw (0,0) node[below left]{$0$};
				\draw[fill] (3,0) circle (0.5pt)node[below]{$ 3$};
				\draw[fill] (1,0) circle (0.5pt)node[below]{$ 1$};
				\draw[fill] (2,0) circle (0.5pt)node[above]{$2$};
				\draw[fill] (0,2) circle (0.5pt)node[left]{$ 2$};
				\draw[fill] (0,-1) circle (0.5pt)node[left]{$ -1$};
				%		\draw[fill] (0,2) circle (0.5pt)node[above left]{$ 2$};
				%		\draw[fill] (0,-2) circle (0.5pt)node[below left]{$ -2$};
				\draw[dashed] (3,0)--(3,2)--(0,2)--(1,2)--(1,0); 
				\draw[dashed](0,-1)--(2,-1)--(2,0);
				\draw[line width=1.2pt,smooth,samples=100,domain=0.6:3.4] plot(\x,{3*(\x)^2-12*(\x)+11});		
				%\draw[line width=1.2pt,smooth,samples=100,domain=-3.3:2.8] plot(\x,{0.75*(\x)^2+0.5*\x-1});
				%	\draw (2.0,2.8) node[left]{$y=f'(x)$};
			\end{tikzpicture}
		\end{center}
		Dựa vào đồ thị $(C)$ ta có\\
		$f'\left(x-2\right)+2<2,\forall x\in\left(1;3\right)\Leftrightarrow{f}'\left(x-2\right)<0,\forall x\in\left(1;3\right)$.\\
		Đặt $ x^*=x-2$ thì $f'\left(x^*\right)<0,\forall x^*\in\left(-1;1\right)$.\\
		Vậy: Hàm số $ f(x)$ nghịch biến trên khoảng $\left(-1;1\right)$.\\
		Cách khác:\\
		Tịnh tiến sang trái hai đơn vị và xuống dưới $2$ đơn vị thì từ đồ thị $(C)$ sẽ thành đồ thị của hàm$ y=f'(x)$. Khi đó $f'(x)<0,\forall x\in\left(-1;1\right)$.\\
		Vậy hàm số $ f(x)$ nghịch biến trên khoảng $\left(-1;1\right)$.}
\end{ex}

\begin{ex}%[2D1G1-2]%Câu 18
	Cho hàm số $y=f(x)$ có đạo hàm cấp $ 3$ liên tục trên $\mathbb{R}$ và thỏa mãn $f(x)\cdot f'''(x)=x{\left(x-1\right)^2}{\left(x+4\right)^3}$ với mọi $x\in\mathbb{R}$ và $g(x)=\left[f'(x)\right]^2-2f(x)\cdot f''(x)$. Hàm số $h(x)=g\left(x^2-2x\right)$ đồng biến trên khoảng nào dưới đây?
	\choice
	{$\left(-\infty ;1\right)$}
	{$\left(2;+\infty\right)$}
	{$\left(0;1\right)$}
	{\True $\left(1;2\right)$}
	\loigiai{		
		Ta có $g'(x)=2f''(x){f}'(x)-2f'(x)\cdot f''(x)-2f(x)\cdot f'''(x)=-2f(x)\cdot f'''(x);$\\
		Khi đó $\left(h(x)\right)'=\left(2x-2\right){g}'\left(x^2-2x\right)=-2\left(2x-2\right)\left(x^2-2x\right){\left(x^2-2x-1\right)^2}{\left(x^2-2x+4\right)^3}$\\
		$h'(x)=0\Leftrightarrow\hoac{
			& x=0\\ 
			& x=1\\ 
			& x=2\\ 
			& x=1\pm\sqrt{2}.}$ 
		Ta có bảng xét dấu của $h'(x)$
		\begin{center}
			\begin{tikzpicture}
				\tkzTabInit[lgt=1.2,espcl=2,nocadre]
				{$t$/0.7, $h'(x)$ /.7} % first column
				{$-\infty$, $1-\sqrt{2}$,$0$, $1$,$2$,$1+\sqrt{2}$, $+\infty$} % first row
				\tkzTabLine { ,+,0,-,0,+,0,-,0,+,0,- } % second row
				%				\tkzTabLine {,-,z,+,t,+,} % third row
				%				\tkzTabLine {,+,d,-,z,+,} % last row
			\end{tikzpicture}
		\end{center}
		Suy ra hàm số $h(x)=g\left(x^2-2x\right)$ đồng biến trên khoảng $\left(1;2\right)$.}
\end{ex}

\begin{ex}%[2D1G1-2]%Câu 19
	Cho hàm số $ y=f(x)$ xác định trên $\mathbb{R}$. Hàm số $ y=g(x)=f'\left(2x+3\right)+2$ có đồ thị là một parabol với tọa độ đỉnh $ I\left(2;-1\right)$ và đi qua điểm $ A\left(1;2\right)$. Hỏi hàm số $ y=f(x)$ nghịch biến trên khoảng nào dưới đây?
	\choice
	{\True $\left(5;9\right)$}
	{$\left(1;2\right)$}
	{$\left(-\infty ;9\right)$}
	{$\left(1;3\right)$}
	\loigiai{	
		Xét hàm số $ g(x)=f'\left(2x+3\right)+2$ có đồ thị là một Parabol nên có phương trình dạng $ y=g(x)=a{x^2}+bx+c\,\,\,\,(P)$.\\
		Vì $(P)$ có đỉnh $ I\left(2;-1\right)$ nên $\heva{
			&\dfrac{-b}{2a}=2\\ 
			& g(2)=-1} \Leftrightarrow\heva{
			&-b=4a\\ 
			& 4a+2b+c=-1} \Leftrightarrow\heva{
			& 4a+b=0\\ 
			& 4a+2b+c=-1}$.\\
		Vì $(P)$ đi qua điểm $ A\left(1;2\right)$ nên $ g(1)=2\Leftrightarrow a+b+c=2$.\\
		Ta có hệ phương trình $\heva{
			& 4a+b=0\\ 
			& 4a+2b+c=-1\\ 
			& a+b+c=2} \Leftrightarrow\heva{
			& a=3\\ 
			& b=-12\\ 
			& c=11}$ nên $ g(x)=3x^2-12x+11$.\\
		Đồ thị của hàm $ y=g(x)$ là
		\begin{center}
			\begin{tikzpicture}[scale=0.7,font=\footnotesize, line join=round, line cap=round, >=stealth] %Đường cong bậc 3
				\draw[thick, ->] (-0.5,0)--(3.5,0);
				\draw[thick, ->] (0,-1.8)--(0,5.3);
				\draw (3.7,0) node[below] {$x$};
				\draw (0,5.4) node[left]{$y$};
				\draw (0,0) node[below left]{$0$};
				\draw[fill] (3,0) circle (0.5pt)node[below]{$ 3$};
				\draw[fill] (1,0) circle (0.5pt)node[below]{$ 1$};
				\draw[fill] (2,0) circle (0.5pt)node[above]{$2$};
				\draw[fill] (0,2) circle (0.5pt)node[left]{$ 2$};
				\draw[fill] (0,-1) circle (0.5pt)node[left]{$ -1$};
				%		\draw[fill] (0,2) circle (0.5pt)node[above left]{$ 2$};
				%		\draw[fill] (0,-2) circle (0.5pt)node[below left]{$ -2$};
				\draw[dashed] (3,0)--(3,2)--(0,2)--(1,2)--(1,0) (3.2,2)--(3,2); 
				\draw[dashed](0,-1)--(2,-1)--(2,0);
				\draw[line width=1.2pt,smooth,samples=100,domain=0.6:3.4] plot(\x,{3*(\x)^2-12*(\x)+11});		
				%\draw[line width=1.2pt,smooth,samples=100,domain=-3.3:2.8] plot(\x,{0.75*(\x)^2+0.5*\x-1});
				%	\draw (2.0,2.8) node[left]{$y=f'(x)$};
			\end{tikzpicture}	
		\end{center}
		Theo đồ thị ta thấy $ f'(2x+3)\le 0\Leftrightarrow f'(2x+3)+2\le 2\Leftrightarrow 1\le x\le 3$.\\
		Đặt $ t=2x+3\Leftrightarrow x=\dfrac{t-3}{2}$ khi đó $ f'(t)\le 0\Leftrightarrow 1\le\dfrac{t-3}{2}\le 3\Leftrightarrow 5\le t\le 9$.\\
		Vậy $ y=f(x)$ nghịch biến trên khoảng $\left(5;9\right)$.}
\end{ex}

\begin{ex}%[2D1G1-2]%Câu 20
	\immini{
		Cho hàm số $ y=f(x)$, hàm số $f'(x)=x^3+a{x^2}+bx+c\left(a,b,c\in\mathbb{R}\right)$ có đồ thị như hình vẽ bên.
		Hàm số $ g(x)=f\left(f'(x)\right)$ nghịch biến trên khoảng nào dưới đây?
		\choice
		{$\left(1;+\infty\right)$}
		{\True $\left(-\infty ;-2\right)$}
		{$\left(-1;0\right)$}
		{$\left(-\dfrac{\sqrt{3}}{3};\dfrac{\sqrt{3}}{3}\right)$}}{
		\begin{tikzpicture}[scale=0.8,font=\footnotesize, line join=round, line cap=round, >=stealth] %Đường cong bậc 3
			\draw[thick, ->] (-1.7,0)--(1.7,0);
			\draw[thick, ->] (0,-2.7)--(0,3.0);
			\draw (1.9,0) node[below] {$x$};
			\draw (0,3.2) node[left]{$y$};
			\draw (0,0) node[below left]{$0$};
			\draw[fill] (-1,0) circle (0.5pt)node[above left]{$ -1 $};
			\draw[fill] (1,0) circle (0.5pt)node[below right]{$ 1$};
			\draw[line width=1.2pt,smooth,samples=100,domain=-1.3:1.3] plot(\x,{2.667*(\x)^3+0*(\x)^2-2.667*\x});		
			%\draw[line width=1.2pt,smooth,samples=100,domain=-3.3:2.8] plot(\x,{0.75*(\x)^2+0.5*\x-1});
		\end{tikzpicture}	
	}
	\loigiai{	
		Vì các điểm $\left(-1;0\right),\left(0;0\right),\left(1;0\right)$ thuộc đồ thị hàm số $ y=f'(x)$ nên ta có hệ\\
		$\heva{
			&-1+a-b+c=0\\ 
			& c=0\\ 
			& 1+a+b+c=0} \Leftrightarrow\heva{
			& a=0\\ 
			& b=-1\\ 
			& c=0} \Rightarrow {f}'(x)=x^3-x\Rightarrow f''(x)=3x^2-1$.\\
		Ta có $ g(x)=f\left(f'(x)\right)\Rightarrow{g}'(x)=f'\left(f'(x)\right)\cdot f''(x)$.\\
		Xét \\
		$g'(x)=0\Leftrightarrow{g}'(x)=f'\left(f'(x)\right)\cdot f''(x)=0$\\
		$\Leftrightarrow {f}'\left(x^3-x\right)\left(3x^2-1\right)=0\Leftrightarrow\hoac{
			&{x^3}-x=0\\ 
			&{x^3}-x=1\\ 
			&{x^3}-x=-1\\ 
			& 3x^2-1=0} \Leftrightarrow \hoac{
			& x=\pm 1\\ 
			& x=0\\ 
			& x=x_1(x_1\approx 1,325)\\ 
			& x=x_2(x_2\approx-1,325)\\ 
			& x=\pm\dfrac{\sqrt{3}}{3}.}$\\
		Bảng biến thiên
		\begin{center}
			\begin{tikzpicture}
				\tkzTabInit[lgt=1.2,espcl=2,nocadre]
				{$t$/0.7, $h'(x)$ /.7} % first column
				{$-\infty$, $-1{,}325$,$-1$, $-\dfrac{\sqrt{3}}{3}$,$0$,$\dfrac{\sqrt{3}}{3}$,$1$,$1{,}325$, $+\infty$} % first row
				\tkzTabLine { ,-,0,+,0,-,0,+,0,-,0,+,0,-,0,+, } % second row
				%				\tkzTabLine {,-,z,+,t,+,} % third row
				%				\tkzTabLine {,+,d,-,z,+,} % last row
			\end{tikzpicture}
		\end{center}
		Dựa vào bảng biến thiên ta có $ g(x)$ nghịch biến trên $\left(-\infty ;-2\right)$}
\end{ex}
\Closesolutionfile{ans}
\indapan{10}{ans/CD1/Muc_9_10}
\chapter{GIÁ TRỊ LỚN NHẤT - GIÁ TRỊ NHỎ NHẤT CỦA HÀM SỐ}
\begin{Solution}{1}
C
\end{Solution}
\begin{Solution}{3}
B
\end{Solution}
\begin{Solution}{4}
A
\end{Solution}
\begin{Solution}{5}
A
\end{Solution}
\begin{Solution}{6}
A
\end{Solution}
\begin{Solution}{7}
B
\end{Solution}
\begin{Solution}{8}
A
\end{Solution}
\begin{Solution}{9}
C
\end{Solution}
\begin{Solution}{10}
B
\end{Solution}
\begin{Solution}{11}
C
\end{Solution}
\begin{Solution}{12}
D
\end{Solution}
\begin{Solution}{13}
B
\end{Solution}
\begin{Solution}{14}
D
\end{Solution}
\begin{Solution}{15}
A
\end{Solution}
\begin{Solution}{16}
B
\end{Solution}
\begin{Solution}{17}
C
\end{Solution}
\begin{Solution}{18}
C
\end{Solution}
\begin{Solution}{19}
C
\end{Solution}
\begin{Solution}{20}
B
\end{Solution}
\begin{Solution}{21}
C
\end{Solution}
\begin{Solution}{22}
B
\end{Solution}
\begin{Solution}{23}
D
\end{Solution}
\begin{Solution}{24}
B
\end{Solution}
\begin{Solution}{25}
D
\end{Solution}
\begin{Solution}{26}
D
\end{Solution}
\begin{Solution}{27}
B
\end{Solution}
\begin{Solution}{28}
A
\end{Solution}
\begin{Solution}{29}
C
\end{Solution}
\begin{Solution}{30}
B
\end{Solution}
\begin{Solution}{31}
D
\end{Solution}
\begin{Solution}{32}
B
\end{Solution}
\begin{Solution}{33}
B
\end{Solution}
\begin{Solution}{34}
C
\end{Solution}
\begin{Solution}{35}
D
\end{Solution}
\begin{Solution}{36}
B
\end{Solution}
\begin{Solution}{37}
B
\end{Solution}
\begin{Solution}{38}
A
\end{Solution}
\begin{Solution}{39}
A
\end{Solution}
\begin{Solution}{40}
D
\end{Solution}
\begin{Solution}{41}
C
\end{Solution}
\begin{Solution}{42}
B
\end{Solution}
\begin{Solution}{43}
A
\end{Solution}
\begin{Solution}{44}
A
\end{Solution}
\begin{Solution}{45}
D
\end{Solution}
\begin{Solution}{46}
C
\end{Solution}
\begin{Solution}{47}
A
\end{Solution}
\begin{Solution}{48}
B
\end{Solution}
\begin{Solution}{49}
B
\end{Solution}
\begin{Solution}{50}
B
\end{Solution}
\begin{Solution}{51}
A
\end{Solution}
\begin{Solution}{52}
A
\end{Solution}
\begin{Solution}{53}
C
\end{Solution}
\begin{Solution}{54}
C
\end{Solution}
\begin{Solution}{55}
C
\end{Solution}
\begin{Solution}{56}
B
\end{Solution}
\begin{Solution}{57}
C
\end{Solution}
\begin{Solution}{58}
C
\end{Solution}
\begin{Solution}{59}
B
\end{Solution}
\begin{Solution}{60}
C
\end{Solution}
\begin{Solution}{61}
A
\end{Solution}
\begin{Solution}{62}
B
\end{Solution}
\begin{Solution}{63}
B
\end{Solution}
\begin{Solution}{64}
D
\end{Solution}
\begin{Solution}{65}
D
\end{Solution}
\begin{Solution}{66}
B
\end{Solution}
\begin{Solution}{67}
A
\end{Solution}
\begin{Solution}{68}
D
\end{Solution}

\begin{Solution}{1}
D
\end{Solution}
\begin{Solution}{2}
C
\end{Solution}
\begin{Solution}{3}
C
\end{Solution}
\begin{Solution}{4}
A
\end{Solution}
\begin{Solution}{5}
B
\end{Solution}
\begin{Solution}{6}
D
\end{Solution}
\begin{Solution}{7}
C
\end{Solution}
\begin{Solution}{8}
D
\end{Solution}
\begin{Solution}{9}
A
\end{Solution}
\begin{Solution}{10}
B
\end{Solution}
\begin{Solution}{11}
D
\end{Solution}
\begin{Solution}{12}
A
\end{Solution}
\begin{Solution}{13}
D
\end{Solution}
\begin{Solution}{14}
B
\end{Solution}
\begin{Solution}{15}
B
\end{Solution}
\begin{Solution}{16}
C
\end{Solution}
\begin{Solution}{1}
A
\end{Solution}
\begin{Solution}{2}
B
\end{Solution}
\begin{Solution}{3}
D
\end{Solution}
\begin{Solution}{4}
D
\end{Solution}
\begin{Solution}{5}
C
\end{Solution}
\begin{Solution}{6}
A
\end{Solution}
\begin{Solution}{7}
D
\end{Solution}
\begin{Solution}{8}
B
\end{Solution}
\begin{Solution}{9}
C
\end{Solution}
\begin{Solution}{10}
C
\end{Solution}
\begin{Solution}{1}
D
\end{Solution}
\begin{Solution}{2}
D
\end{Solution}
\begin{Solution}{3}
B
\end{Solution}
\begin{Solution}{4}
C
\end{Solution}
\begin{Solution}{5}
D
\end{Solution}
\begin{Solution}{6}
A
\end{Solution}
\begin{Solution}{7}
C
\end{Solution}
\begin{Solution}{8}
B
\end{Solution}
\begin{Solution}{9}
A
\end{Solution}
\begin{Solution}{10}
C
\end{Solution}
\begin{Solution}{11}
D
\end{Solution}
\begin{Solution}{12}
C
\end{Solution}
\begin{Solution}{13}
A
\end{Solution}
\begin{Solution}{14}
D
\end{Solution}
\begin{Solution}{15}
A
\end{Solution}
\begin{Solution}{16}
A
\end{Solution}
\begin{Solution}{17}
B
\end{Solution}
\begin{Solution}{18}
C
\end{Solution}
\begin{Solution}{19}
C
\end{Solution}
\begin{Solution}{20}
A
\end{Solution}
\begin{Solution}{21}
D
\end{Solution}
\begin{Solution}{22}
C
\end{Solution}
\begin{Solution}{23}
A
\end{Solution}
\begin{Solution}{24}
C
\end{Solution}
\begin{Solution}{25}
A
\end{Solution}
\begin{Solution}{26}
B
\end{Solution}
\begin{Solution}{27}
B
\end{Solution}
\begin{Solution}{28}
D
\end{Solution}
\begin{Solution}{29}
B
\end{Solution}
\begin{Solution}{30}
D
\end{Solution}
\begin{Solution}{31}
D
\end{Solution}
\begin{Solution}{32}
C
\end{Solution}
\begin{Solution}{33}
D
\end{Solution}
\begin{Solution}{34}
C
\end{Solution}
\begin{Solution}{35}
D
\end{Solution}
\begin{Solution}{36}
D
\end{Solution}
\begin{Solution}{37}
D
\end{Solution}
\begin{Solution}{38}
D
\end{Solution}
\begin{Solution}{39}
D
\end{Solution}
\begin{Solution}{40}
C
\end{Solution}
\begin{Solution}{41}
A
\end{Solution}
\begin{Solution}{1}
A
\end{Solution}
\begin{Solution}{2}
B
\end{Solution}
\begin{Solution}{3}
C
\end{Solution}
\begin{Solution}{4}
A
\end{Solution}
\begin{Solution}{5}
A
\end{Solution}
\begin{Solution}{6}
C
\end{Solution}
\begin{Solution}{7}
C
\end{Solution}
\begin{Solution}{8}
B
\end{Solution}
\begin{Solution}{9}
C
\end{Solution}
\begin{Solution}{10}
B
\end{Solution}
\begin{Solution}{11}
A
\end{Solution}
\begin{Solution}{12}
B
\end{Solution}
\begin{Solution}{13}
B
\end{Solution}
\begin{Solution}{14}
B
\end{Solution}
\begin{Solution}{15}
A
\end{Solution}
\begin{Solution}{16}
B
\end{Solution}
\begin{Solution}{17}
A
\end{Solution}
\begin{Solution}{18}
D
\end{Solution}
\begin{Solution}{19}
C
\end{Solution}
\begin{Solution}{20}
C
\end{Solution}
\begin{Solution}{21}
A
\end{Solution}
\begin{Solution}{22}
C
\end{Solution}
\begin{Solution}{23}
C
\end{Solution}
\begin{Solution}{24}
A
\end{Solution}
\begin{Solution}{25}
B
\end{Solution}
\begin{Solution}{26}
B
\end{Solution}
\begin{Solution}{27}
A
\end{Solution}
\begin{Solution}{28}
A
\end{Solution}
\begin{Solution}{29}
C
\end{Solution}
\begin{Solution}{30}
B
\end{Solution}
\begin{Solution}{31}
A
\end{Solution}
\begin{Solution}{32}
C
\end{Solution}
\begin{Solution}{33}
B
\end{Solution}
\begin{Solution}{34}
A
\end{Solution}
\begin{Solution}{35}
B
\end{Solution}
\begin{Solution}{36}
B
\end{Solution}
\begin{Solution}{37}
B
\end{Solution}
\begin{Solution}{38}
D
\end{Solution}
\begin{Solution}{39}
B
\end{Solution}
\begin{Solution}{40}
A
\end{Solution}
\begin{Solution}{41}
D
\end{Solution}
\begin{Solution}{42}
D
\end{Solution}
\begin{Solution}{43}
A
\end{Solution}
\begin{Solution}{44}
D
\end{Solution}
\begin{Solution}{45}
C
\end{Solution}
\begin{Solution}{46}
B
\end{Solution}
\begin{Solution}{47}
A
\end{Solution}
\begin{Solution}{48}
D
\end{Solution}
\begin{Solution}{49}
B
\end{Solution}
\begin{Solution}{50}
B
\end{Solution}
\begin{Solution}{51}
D
\end{Solution}
\begin{Solution}{52}
C
\end{Solution}
\begin{Solution}{53}
C
\end{Solution}
\begin{Solution}{54}
B
\end{Solution}
\begin{Solution}{55}
D
\end{Solution}
\begin{Solution}{56}
B
\end{Solution}
\begin{Solution}{57}
C
\end{Solution}
\begin{Solution}{58}
A
\end{Solution}
\begin{Solution}{59}
A
\end{Solution}
\begin{Solution}{60}
B
\end{Solution}
\begin{Solution}{61}
D
\end{Solution}
\begin{Solution}{62}
D
\end{Solution}
\begin{Solution}{63}
B
\end{Solution}
\begin{Solution}{64}
A
\end{Solution}
\begin{Solution}{65}
D
\end{Solution}
\begin{Solution}{66}
C
\end{Solution}
\begin{Solution}{67}
A
\end{Solution}
\begin{Solution}{68}
A
\end{Solution}
\begin{Solution}{69}
D
\end{Solution}
\begin{Solution}{70}
C
\end{Solution}
\begin{Solution}{71}
B
\end{Solution}
\begin{Solution}{72}
A
\end{Solution}
\begin{Solution}{73}
C
\end{Solution}
\begin{Solution}{74}
C
\end{Solution}
\begin{Solution}{75}
C
\end{Solution}
\begin{Solution}{76}
A
\end{Solution}
\begin{Solution}{77}
C
\end{Solution}
\begin{Solution}{78}
B
\end{Solution}
\begin{Solution}{79}
D
\end{Solution}
\begin{Solution}{80}
B
\end{Solution}

\section{Mức 9,10 điểm}
\setcounter{ex}{0}
\setcounter{dang}{0}
\Opensolutionfile{ans}[ans/CD1/Muc_9_10]
\begin{dang}{Tìm m để hàm số đơn điệu trên các khoảng xác định của nó}
	Đang thiếu bài thầy Jf Câu 1 đến 26 
\end{dang}
\begin{dang}
	{Tìm khoảng đơn điệu của hàm số $g(x) = f\left[ u(x)\right] +v(x)$ khi biết đồ thị hoặc bảng biến thiên của hàm số $y = f'(x)$}
\end{dang}
\begin{ex}[Đề tham khảo 2019]%[2D1K1-2]
	Cho hàm số $f(x)$ có bảng xét dấu của đạo hàm như sau
	\begin{center}
		\begin{tikzpicture}
			\tkzTabInit[nocadre,lgt=1.2,espcl=2,deltacl=0.6]
			{$x$ /0.6,$f'(x)$ /0.6}
			{$-\infty$,$1$,$2$,$3$,$4$,$+\infty$}
			\tkzTabLine{,-,$0$,+,$0$,+,$0$,-,$0$,+,}
		\end{tikzpicture}
	\end{center}
	Hàm số $y=3 f(x+2)-x^3+3 x$ đồng biến trên khoảng nào dưới đây?
	\choice
	{$(-\infty ;-1)$}
	{\True $(-1 ; 0)$}
	{$(0 ; 2)$}
	{$(1 ;+\infty)$}
	\loigiai{
		Ta có $y'=3\left[f'(x+2)-\left(x^2-3\right)\right]$.\\
		Với $x \in(-1 ; 0) \Rightarrow x+2 \in(1 ; 2) \Rightarrow f'(x+2)>0$, lại có $x^2-3<0 \Rightarrow y'>0 ;~ \forall x \in(-1 ; 0)$.\\
		Vậy hàm số $y=3 f(x+2)-x^3+3 x$ đồng biến trên khoảng $(-1 ; 0)$.\\
		Chú ý:\\
		+) Ta xét $x \in(1 ; 2) \subset(1 ;+\infty)
		\Rightarrow x+2 \in(3 ; 4)\\
		\Rightarrow f'(x+2)<0 ;~ x^2-3>0$\\
		Suy ra hàm số nghịch biến trên khoảng $(1 ; 2)$ nên loại hai phương án$(0 ; 2)$ và $(1 ;+\infty)$.\\
		+) Tương tự ta xét
		$x \in(-\infty ;-2) \Rightarrow x+2 \in(-\infty ; 0)\\
		\Rightarrow f'(x+2)<0 ; x^2-3>0 \Rightarrow y'<0 ; ~ \forall x \in(-\infty ;-2)$.\\
		Suy ra hàm số nghịch biến trên khoảng $(-\infty ;-2)$ nên loại$(-\infty ;-1)$.\\
		Vậy hàm số đã cho đồng biến trên khoảng $(-1 ; 0)$.
	}
\end{ex}
\begin{ex}[Đề Tham Khảo 2020 - Lần 1]%[2D1G1-2]
	\immini{
		Cho hàm số $f(x)$. Hàm số $y=f'(x)$ có đồ thị như hình bên. Hàm số $g(x)=f(1-2 x)+x^2-x$ nghịch biến trên khoảng nào dưới đây?
		\choice
		{\True $\left(1 ; \dfrac{3}{2}\right)$}
		{$\left(0 ; \dfrac{1}{2}\right)$}
		{$(-2 ;-1)$}
		{$(2 ; 3)$}
	}
	{
		\begin{tikzpicture}[scale=0.7,>=stealth, font=\footnotesize, line join=round, line cap=round]
			%\def\a{1} \def\b{-6} \def\c{9} \def\d{1} % Hệ số
			\def\xmin{-4} \def\xmax{6}
			\def\ymin{-3} \def\ymax{2} 
			%\draw[color=gray!50,dashed] (\xmin,\ymin) grid (\xmax,\ymax); 
			\draw[->] (\xmin,0)--(\xmax,0) node [below]{$x$};
			\draw[->] (0,\ymin)--(0,\ymax) node [left]{$y$};
			\node at (0,0) [below left]{$O$};
			%\node at (1,3) [below left]{$f'(x)$};
			%\node at (-1.3,4) {$f'(x)$};
			\draw[dashed] (-2,0) node[below]{$-2$}--(-2,1)--(0,1) node[below left]{$1$};
			\draw[dashed] (4,0) node[below left]{$4$}--(4,-2)--(0,-2) node[below left]{$-2$};
			%\draw[dashed] (1,0) node[below]{$1$}--(1,1);
			%\draw[dashed] (-0.5,0) node[below left]{$-0{,}5$}--(-0.5,2.125);
			\clip (\xmin+0.1,\ymin+0.1) rectangle (\xmax-0.5,\ymax-0.1);
			\draw[smooth,samples=300][domain=-4:5.5] plot(\x,{0.071*(\x)^3-0.142*(\x)^2-1.07*(\x)});
		\end{tikzpicture}
	}
	
	\loigiai{
		Ta có : $g(x)=f(1-2 x)+x^2-x \Rightarrow g'(x)=-2 f'(1-2 x)+2 x-1$.\\
		\immini{
			Đặt $t=1-2 x \Rightarrow g'(x)=-2 f'(t)-t$.\\
			$g'(x)=0 \Rightarrow f'(t)=-\dfrac{t}{2}$.\\
			Vẽ đường thẳng $y=-\dfrac{x}{2}$ và đồ thị hàm số $f'(x)$ trên cùng một hệ trục
		}	
		{
			\begin{tikzpicture}[scale=0.7,>=stealth, font=\footnotesize, line join=round, line cap=round]
				%\def\a{1} \def\b{-6} \def\c{9} \def\d{1} % Hệ số
				\def\xmin{-4} \def\xmax{6}
				\def\ymin{-3} \def\ymax{2} 
				%	\draw[color=gray!50,dashed] (\xmin,\ymin) grid (\xmax,\ymax); 
				\draw[->] (\xmin,0)--(\xmax,0) node [below]{$x$};
				\draw[->] (0,\ymin)--(0,\ymax) node [left]{$y$};
				\node at (0,0) [below left]{$O$};
				%\node at (1,3) [below left]{$f'(x)$};
				%\node at (-1.3,4) {$f'(x)$};
				\draw[dashed] (-2,0) node[below]{$-2$}--(-2,1)--(0,1) node[below left]{$1$};
				\draw[dashed] (4,0) node[below]{$4$}--(4,-2)--(0,-2) node[below left]{$-2$};
				%\draw[dashed] (1,0) node[below]{$1$}--(1,1);
				%\draw[dashed] (-0.5,0) node[below left]{$-0{,}5$}--(-0.5,2.125);
				\clip (\xmin+0.1,\ymin+0.1) rectangle (\xmax-0.5,\ymax-0.1);
				\draw[smooth,samples=300][domain=-4:5.5] plot(\x,{0.071*(\x)^3-0.142*(\x)^2-1.07*(\x)});
				\draw[smooth,samples=300][domain=-4:5.5] plot(\x,{(-0.5*(\x)});
			\end{tikzpicture}
		}	Hàm số $g(x)$ nghịch biến $\Rightarrow g'(x) \leq 0 \Rightarrow f'(t) \geq-\dfrac{t}{2}\Rightarrow\hoac{&-2 \leq t \leq 0 \\&t \geq 4.}$\\
		Như vậy $f'(1-2 x) \geq \dfrac{1-2 x}{-2}\Rightarrow\hoac{&-2 \leq 1-2 x \leq 0 \\ &4 \leq 1-2 x}\Rightarrow\hoac{&\dfrac{1}{2}\leq x \leq \dfrac{3}{2}\\ &x \leq-\dfrac{3}{2}.}$\\
		Vậy hàm số $g(x)=f(1-2 x)+x^2-x$ nghịch biến trên các khoảng $\left(\dfrac{1}{2}; \dfrac{3}{2}\right)$ và $\left(-\infty ;-\dfrac{3}{2}\right)$.\\
		Mà $\left(1 ; \dfrac{3}{2}\right) \subset \left(\dfrac{1}{2}; \dfrac{3}{2}\right)$ nên hàm số $g(x)=f(1-2 x)+x^2-x$ nghịch biến trên khoảng $\left(1 ; \dfrac{3}{2}\right)$.
	}
\end{ex}
\begin{ex}[Chuyên Lê Quý Đôn Điện Biên 2019]%[2D1G1-2]
	Cho hàm số $f(x)$ có bảng xét dấu của đạo hàm như sau
	\begin{center}
		\begin{tikzpicture}
			\tkzTabInit[nocadre,lgt=1.2,espcl=2,deltacl=0.6]
			{$x$ /0.6,$f'(x)$ /0.6}
			{$-\infty$,$0$,$1$,$2$,$3$,$+\infty$}
			\tkzTabLine{,+,$0$,-,$0$,-,$0$,+,$0$,-,}
		\end{tikzpicture}
	\end{center}
	Hàm số $y=f(x-1)+x^3-12 x+2019$ nghịch biến trên khoảng nào dưới đây?
	\choice
	{$(1 ;+\infty)$}
	{\True $(1 ; 2)$}
	{$(-\infty ; 1)$}
	{$(3 ; 4)$}
	\loigiai{
		$y'=f'(x-1)+3 x^2-12=f'(t)+3 t^2+6 t-9=f'(t)-\left(-3 t^2-6 t+9\right)$, với $t=x-1$.\\
		\immini{
			Nghiệm của phương trình $y'=0$ là hoành độ giao điểm của các đồ thị hàm số $y=f'(t)$ và $y=-3 t^2-6 t+9$.\\
			Vẽ đồ thị hàm số $y=f'(t)$ và $y=-3 t^2-6 t+9$ trên cùng một hệ trục tọa độ như hình vẽ bên.
		}	
		{		\begin{tikzpicture}[scale=0.5,>=stealth, font=\footnotesize, line join=round, line cap=round]
				\def\a{-3} \def\b{-6} \def\c{9} % Hệ số
				\def\xmin{-9} \def\xmax{7}
				\def\ymin{-3} \def\ymax{13}
				
				%\draw[color=gray!50,dashed] (\xmin,\ymin) grid (\xmax,\ymax);
				
				\draw[->] (\xmin,0)--(\xmax,0) node [below]{$x$};
				\draw[->] (0,\ymin)--(0,\ymax) node [left]{$y$};
				\node at (0,0) [below left]{$O$};
				\clip (\xmin+0.1,\ymin+0.1) rectangle (\xmax-0.5,\ymax-0.1);
				\draw[smooth,samples=300] plot(\x,{\a*(\x)^2+\b*(\x)+\c});
				\node at (1,0) [above right]{$1$};
				\node at (2,0) [below right]{$2$};
				\node at (3,0) [below right]{$3$};
				\node at (-3,-2) [left]{$y=-3t^2-6t+9$};
				\node at (4,0) [below right]{$f'(x)$};
				\draw (-2.2,10).. controls (-1,1.9) and (-0.5,0.8) .. (0,0);
				%\draw (-2,0).. controls (-1.5,-2) and (-0.5,-0) .. (0,0);
				\draw (0,0).. controls (0.4,-0.6) and (0.6,-0.6) .. (0.8,-0.2);
				\draw (0.8,-0.2).. controls (1,0.25) and (1.1,-0.1) .. (1.4,-0.8);
				\draw (1.4,-0.8).. controls (1.6,-1.1) and (1.7,-0.9) .. (2,0);
				\draw (2,0).. controls (2.4,1.1) and (2.6,1.1) .. (3.5,-1);
			\end{tikzpicture}
		}
		Dựa vào đồ thị trên, ta có bảng xét dấu của hàm số $y'=f'(t)-\left(-3 t^2-6 t+9\right)$ như sau $
		\left(t_0<-1\right)$
		\begin{center}
			\begin{tikzpicture}
				\tkzTabInit[nocadre,lgt=2,espcl=2,deltacl=0.6]
				{$x$ /0.6,$y'$ /0.6}
				{$-\infty$,$t_0$,$1$,$+\infty$}
				\tkzTabLine{,+,$0$,-,$0$,+,}
			\end{tikzpicture}
		\end{center}
		Hàm số nghịch biến trên khoảng $t \in\left(t_0 ; 1\right)$.\\
		Do đó hàm số nghịch biến trên khoảng $x \in(1 ; 2) \subset \left(t_0+1 ; 1\right)$.
	}
\end{ex}


\begin{ex}[Chuyên Phan Bội Châu Nghệ An 2019]%[2D1G1-2]
	Cho hàm số $f(x)$ có bảng xét dấu đạo hàm như sau:
	\begin{center}
		\begin{tikzpicture}
			\tkzTabInit[nocadre,lgt=2,espcl=2,deltacl=0.6]
			{$x$ /0.6,$f'(x)$ /0.6}
			{$-\infty$,$1$,$2$,$3$,$4$,$+\infty$}
			\tkzTabLine{,-,$0$,+,$0$,+,$0$,-,$0$,+,}
		\end{tikzpicture}
	\end{center}
	Hàm số $y=2 f(1-x)+\sqrt{x^2+1}-x$ nghịch biến trên những khoảng nào dưới đây
	\choice
	{$(-\infty ;-2)$}
	{$(-\infty ; 1)$}
	{\True $(-2 ; 0)$}
	{$(-3 ;-2)$}
	\loigiai{
		$y'=-2 f'(1-x)+\dfrac{x}{\sqrt{x^2+1}}-1$. \\
		Có $\dfrac{x}{\sqrt{x^2+1}}-1<0,~ \forall x \in(-2 ; 0)$.\\
		Bảng xét dấu:
		\begin{center}
			\begin{tikzpicture}
				\tkzTabInit[nocadre,lgt=2,espcl=2,deltacl=0.6]
				{$x$ /0.7,$f'(1-x)$ /0.7}
				{$-\infty$,$-3$,$-2$,$-1$,$0$,$+\infty$}
				\tkzTabLine{,+,$0$,-,$0$,+,$0$,+,$0$,-,}
			\end{tikzpicture}
		\end{center}
		$\Rightarrow-2 f'(1-x)<0, ~ \forall x \in(-2 ; 0) \\
		\Rightarrow-2 f'(1-x)+\dfrac{x}{\sqrt{x^2+1}}-1<0, ~\forall x \in(-2 ; 0)$.
	}
\end{ex}
\begin{ex}[Sở Vĩnh Phúc 2019]%[2D1G1-2]
	\immini{
		Cho hàm số bậc bốn $y=f(x)$ có đồ thị của hàm số $y=f'(x)$ như hình vẽ bên.\\
		Hàm số $y=3 f(x)+x^3-6 x^2+9 x$ đồng biến trên khoảng nào trong các khoảng sau đây?
		\choice
		{$(0 ; 2)$}
		{$(-1 ; 1)$}
		{$(1 ;+\infty)$}
		{\True $(-2 ; 0)$}
	}
	{
		\begin{tikzpicture}[scale=0.7,>=stealth, font=\footnotesize, line join=round, line cap=round]
			\def\a{0.21} \def\b{0.88} \def\c{-0.58} \def\d{-3} % Hệ số
			\def\xmin{-5} \def\xmax{5}
			\def\ymin{-4} \def\ymax{3} 
			%\draw[color=gray!50,dashed] (\xmin,\ymin) grid (\xmax,\ymax); 
			\draw[->] (\xmin,0)--(\xmax,0) node [below]{$x$};
			\draw[->] (0,\ymin)--(0,\ymax) node [left]{$y$};
			\node at (0,0) [above left]{$O$};
			\node at (-4,0) [below left]{$-4$};
			\node at (-2,0) [below left]{$-2$};
			\node at (0,-3) [below right]{$-3$};
			\draw[dashed] (2,0) node[above right]{$2$}--(2,1) --(0,1) node[above right]{$1$};
			\clip (\xmin+0.1,\ymin+0.1) rectangle (\xmax-0.5,\ymax-0.1);
			\draw[smooth,samples=300] plot(\x,{\a*(\x)^3+\b*(\x)^2+\c*(\x)+\d});
		\end{tikzpicture}
	}
	
	\loigiai{
		Hàm số $f(x)=a x^4+b x^3+c x^2+d x+e,(a \neq 0)$.
		Có $f'(x)=4 a x^3+3 b x^2+2 c x+d$.\\
		Đồ thị hàm số $y=f'(x)$ đi qua các điểm $(-4 ; 0),(-2 ; 0),(0 ;-3),(2 ; 1)$ nên ta có
		$$\heva{&- 2 5 6 a + 4 8 b - 8 c + d = 0\\
			&- 3 2 a + 1 2 b - 4 c + d = 0\\
			&d = - 3\\
			&3 2 a + 1 2 b + 4 c + d = 1}\Leftrightarrow \heva{&
			a=\dfrac{5}{96}\\
			&b=\dfrac{7}{24}\\
			&c=-\dfrac{7}{24}\\
			&d=-3.}
		$$
		Xét hàm số
		$
		y=3 f(x)+x^3-6 x^2+9 x$\\
		Ta có $ y'=3\left(f'(x)+x^2-4 x+3\right)=3\left(\frac{5}{24}x^3+\frac{15}{8}x^2-\frac{55}{12}x\right)
		$\\
		Ta có $y'=0 \Leftrightarrow\hoac{&x=-11 \\&x=0 \\&x=2.}$ \\
		Xét dấu $y'$, ta được hàm số đã cho đồng biến trên các khoảng $(-11 ; 0)$ và $(2 ;+\infty)$.
	}
\end{ex}
\begin{ex}[Học Mãi 2019]%[2D1K1-2]
	\immini
	{Cho hàm số $y=f(x)$ có đạo hàm trên $\mathbb{R}$. Đồ thị hàm số $y=f'(x)$ như hình bên. Hỏi đồ thị hàm số $y=f(x)-2 x$ có bao nhiêu điểm cực trị?
		\choice
		{$4$}
		{\True $3$}
		{$2$}
		{$1$}
	}
	{
		\begin{tikzpicture}[font=\footnotesize,line join=round, line cap=round,>=stealth,scale=0.8]
			\draw[->] (-3.5,0)--(4,0) node[above] {$x$};
			\draw[->] (0,-3)--(0,4) node[left] {$y$};
			%\fill[black] (-2,0)node[below left]{$-2$} circle (1.2pt) (0,0)node[above right]{$O$} circle (1.2pt) (3,0)node[above]{$3$} circle (1.2pt);
			\draw[dashed] (-2,-2)-- (0,-2) node[right]{$-2$};
			\draw[dashed] (2,0) node[below]{$2$}-- (2,2)--(0,2) node[below left]{$2$};
			\node at (0,0) [below left]{$O$};
			\node at (3,0) [below right]{$3$};
			\draw (-3,2.5).. controls (-2.2,-3) and (-1.8,-3) .. (-1.1,0);
			\draw (-1.1,0).. controls (-0.6,2.5) and (-0.4,2.5) .. (0,2);
			\draw (0,2).. controls (0.7,0.5) and (1.1,0.5) .. (1.5,1.5);
			\draw (1.5,1.5).. controls (2,2.5) and (2.8,2.5) .. (3.5,-2.5);
			%\draw (3,0).. controls (3.3,-0.1) and (3.5,-0.5) .. (3.5,-2);
		\end{tikzpicture}
	}
	\loigiai{
		\immini{
			Đặt $g(x)=f(x)-2 x$.\\
			$\Rightarrow g'(x)=f'(x)-2 .
			$\\
			Vẽ đường thẳng $y=2$.\\
			$\Rightarrow$ phương trình $g'(x)=0$ có $3$ nghiệm bội lẻ.\\
			$\Rightarrow$ đồ thị hàm số $y=f(x)-2 x$ có $3$ điểm cực trị.
		}
		{
			\begin{tikzpicture}[font=\footnotesize,line join=round, line cap=round,>=stealth,scale=0.8]
				\draw[->] (-3.5,0)--(4,0) node[above] {$x$};
				\draw[->] (0,-3)--(0,4) node[left] {$y$};
				%\fill[black] (-2,0)node[below left]{$-2$} circle (1.2pt) (0,0)node[above right]{$O$} circle (1.2pt) (3,0)node[above]{$3$} circle (1.2pt);
				\draw[dashed] (-2,-2)-- (0,-2) node[right]{$-2$};
				\draw[dashed] (2,0) node[below]{$2$}-- (2,2)--(0,2) node[below left]{$2$};
				\node at (3,0) [below left]{$3$};
				\draw (-3,2.5).. controls (-2.2,-3) and (-1.8,-3) .. (-1.1,0);
				\draw (-1.1,0).. controls (-0.6,2.5) and (-0.4,2.5) .. (0,2);
				\draw (0,2).. controls (0.7,0.5) and (1.1,0.5) .. (1.5,1.5);
				\draw (1.5,1.5).. controls (2,2.5) and (2.8,2.5) .. (3.5,-2.5);
				\draw (-3.5,2)--(4,2) node[above]{$y=2$};
			\end{tikzpicture}
		}
	}
\end{ex}
\begin{ex}[THPT Hoàng Hoa Thám Hưng Yên 2019]%[2D1G1-2]
	\immini{
		Cho hàm số $y=f(x)$ liên tục trên $\mathbb{R}$. Hàm số $y=f'(x)$ có đồ thị như hình vẽ. 
		Hàm số $g(x)=f(x-1)+\dfrac{2019-2018 x}{2018}$ đồng biến trên khoảng nào dưới đây?
		\choice
		{$(2 ; 3)$}
		{$(0 ; 1)$}
		{\True $(-1 ; 0)$}
		{$(1 ; 2)$}
	}
	{
		\begin{tikzpicture}[scale=1, font=\footnotesize, line join=round, line cap=round, >=stealth]
			\tikzset{label style/.style={font=\footnotesize}}
			\draw[->] (-2,0)--(3,0) node[below left] {$x$};
			\draw[->] (0,-2)--(0,3) node[below left] {$y$};
			\draw[fill=black] (0,0) node [above left] {$O$} circle(1pt);
			\fill (1,1) circle(1pt) (-1,1) circle(1pt) (2,1) circle(1pt);
			\foreach \x in {1,2}
			\draw[thin] (\x,1pt)--(\x,-1pt) node [below] {\footnotesize$\x$};
			\foreach \x in {-1}
			\draw[thin] (\x,1pt)--(\x,-1pt) node [below left] {\footnotesize$\x$};
			\foreach \y in {-1}
			\draw[thin] (1pt,\y)--(-1pt,\y) node [right] {\footnotesize$\y$};
			\foreach \y in {1}
			\draw[thin] (1pt,\y)--(-1pt,\y) node [above left] {\footnotesize$\y$};
			\draw[dashed](-1,0)--(-1,1)--(2,1) (1,1)--(1,0) (2,1)--(2,0);
			\begin{scope}
				\clip (-3,-3) rectangle (3,3);
				\draw[name path=(C)] plot[smooth,tension=0.7] coordinates{(-1.15,3)(-0.5,-1.6)(.8,.88)(1.9,0.8)(2.3,3)};
			\end{scope}
		\end{tikzpicture}
	}	\loigiai{
		Ta có $g'(x)=f'(x-1)-1$.\\
		$
		g'(x) \geq 0 \Leftrightarrow f'(x-1)-1 \geq 0 \Leftrightarrow f'(x-1) \geq 1 \Leftrightarrow \hoac{&x - 1 \leq - 1\\
			&x - 1 \geq 2}\Leftrightarrow \hoac{&
			x \leq 0 \\
			&x \geq 3.}
		$\\
		Từ đó suy ra hàm số $g(x)=f(x-1)+\dfrac{2019-2018 x}{2018}$ đồng biến trên khoảng $(-1 ; 0)$.
	}
\end{ex}

\begin{ex}[(Sở Ninh Bình 2019]%[2D1K1-2]
	Cho hàm số $y=f(x)$ có bảng xét dấu của đạo hàm như sau
	\begin{center}
		\begin{tikzpicture}
			\tkzTabInit[nocadre,lgt=1,espcl=2,deltacl=0.6]
			{$x$ /0.7,$f'(x)$ /0.7}
			{$-\infty$,$-2$,$-1$,$2$,$4$,$+\infty$}
			\tkzTabLine{,+,$0$,-,$0$,+,$0$,-,$0$,+,}
		\end{tikzpicture}
	\end{center}
	Hàm số $y=-2 f(x)+2019$ nghịch biến trên khoảng nào trong các khoảng dưới đây?
	\choice
	{$(-4 ; 2)$}
	{\True $(-1 ; 2)$}
	{$(-2 ;-1)$}
	{$(2 ; 4)$}
	\loigiai{
		Xét $y=g(x)=-2 f(x)+2019$.\\
		Ta có $g'(x)=(-2 f(x)+2019)'=-2 f'(x), g'(x)=0 \Leftrightarrow\hoac{&x=-2 \\&x=-1 \\&x=2 \\&x=4.}$.\\
		Ta có bảng xét dấu của $g'(x)$
		\begin{center}
			\begin{tikzpicture}
				\tkzTabInit[nocadre,lgt=1,espcl=2,deltacl=0.6]
				{$x$ /0.6,$f'(x)$ /0.6}
				{$-\infty$,$-2$,$-1$,$2$,$4$,$+\infty$}
				\tkzTabLine{,-,$0$,+,$0$,-,$0$,+,$0$,+,}
			\end{tikzpicture}
		\end{center}
		Dựa vào bảng xét dấu, ta thấy hàm số $y=g(x)$ nghịch biến trên khoảng $(-1 ; 2)$.
	}
\end{ex}
\begin{ex}[THPT Lương Thế Vinh Hà Nội 2019]%[2D1G1-2]
	\immini{
		Cho hàm số $y=f(x)$. Biết đồ thị hàm số $y=f'(x)$ có đồ thị như hình vẽ bên. 
		Hàm số $y=f \left(3-x^2\right)+2018$ đồng biến trên khoảng nào dưới đây?
		\choice
		{\True $(-1 ; 0)$}
		{$(2 ; 3)$}
		{$(-2 ;-1)$}
		{$(0 ; 1)$}
	}
	{
		\begin{tikzpicture}[scale=0.6,>=stealth, font=\footnotesize, line join=round, line cap=round]
			\def\a{0.065} \def\b{0.32} \def\c{-0.53} \def\d{-0.82} % Hệ số
			\def\xmin{-8} \def\xmax{4}
			\def\ymin{-3} \def\ymax{3} 
			%\draw[color=gray!50,dashed] (\xmin,\ymin) grid (\xmax,\ymax); 
			\draw[->] (\xmin,0)--(\xmax,0) node [below]{$x$};
			\draw[->] (0,\ymin)--(0,\ymax) node [left]{$y$};
			\node at (0,0) [below left]{$O$};
			\node at (-6,0) [below left]{$-6$};
			\node at (-1,0) [below left]{$-1$};
			\node at (2,0) [below right]{$2$};
			\clip (\xmin+0.1,\ymin+0.1) rectangle (\xmax-0.5,\ymax-0.1);
			\draw[smooth,samples=300][domain=-6.5:3.5] plot(\x,{\a*(\x)^3+\b*(\x)^2+\c*(\x)+\d});
		\end{tikzpicture}
	}
	
	\loigiai{
		Ta có $\left[f\left( 3-x^2\right)+2018 \right]'=-2 x \cdot f'\left(3-x^2\right) $.\\
		$
		-2 x \cdot f'\left(3-x^2\right)=0 \Leftrightarrow\hoac{&
			x = 0\\
			&3 - x ^{2}= - 6\\
			&3 - x ^{2}= - 1\\
			&3 - x ^{2}= 2}
		\Leftrightarrow \hoac{
			&x=0 \\
			&x=\pm 3 \\
			&x=\pm 2 \\
			&	x=\pm 1.}
		$\\
		Bảng xét dấu của đạo hàm hàm số đã cho
		\begin{center}
			\begin{center}
				\begin{tikzpicture}
					\tkzTabInit[nocadre,lgt=2.9,espcl=1.5,deltacl=0.6]
					{$x$ /0.7,$f'\left( 3-x^2\right) $/0.7,$-2xf'\left( 3-x^2\right)$/0.8}
					{$-\infty$,$-3$,$-2$,$-1$,$0$,$1$,$2$,$3$,$+\infty$}
					\tkzTabLine{,-,$0$,+,$0$,-,$0$,+,$0$,+,$0$,-,$0$,+,$0$,-}
					\tkzTabLine{,-,$0$,+,$0$,-,$0$,+,$0$,-,$0$,+,$0$,-,$0$,+}
				\end{tikzpicture}
			\end{center}
		\end{center}
		Từ bảng xét dấu suy ra hàm số đồng biến trên $(-1 ; 0)$.
	}
\end{ex}
\begin{ex}[Chuyên Biên Hòa - Hà Nam - 2020]%[2D1G1-2]
	\immini{
		Cho hàm số đa thức $f(x)$ có đạo hàm trên $\mathbb{R}$. Biết $f(0)=0$ và đồ thị hàm số $y=f'(x)$ như hình sau.
		Hàm số $g(x)=\left|4 f(x)+x^2\right|$ đồng biến trên khoảng nào dưới đây?
		\choice
		{$(4 ;+\infty)$}
		{\True $(0 ; 4)$}
		{$(-\infty ;-2)$}
		{$(-2 ; 0)$}
	}	
	{
		\begin{tikzpicture}[scale=0.7,>=stealth, font=\footnotesize, line join=round, line cap=round]
			%\def\a{1} \def\b{-6} \def\c{9} \def\d{1} % Hệ số
			\def\xmin{-4} \def\xmax{6}
			\def\ymin{-3} \def\ymax{2} 
			%\draw[color=gray!50,dashed] (\xmin,\ymin) grid (\xmax,\ymax); 
			\draw[->] (\xmin,0)--(\xmax,0) node [below]{$x$};
			\draw[->] (0,\ymin)--(0,\ymax) node [left]{$y$};
			\node at (0,0) [below left]{$O$};
			%\node at (1,3) [below left]{$f'(x)$};
			%\node at (-1.3,4) {$f'(x)$};
			\draw[dashed] (-2,0) node[below]{$-2$}--(-2,1)--(0,1) node[below left]{$1$};
			\draw[dashed] (4,0) node[below]{$4$}--(4,-2)--(0,-2) node[below left]{$-2$};
			%\draw[dashed] (1,0) node[below]{$1$}--(1,1);
			%\draw[dashed] (-0.5,0) node[below left]{$-0{,}5$}--(-0.5,2.125);
			\clip (\xmin+0.1,\ymin+0.1) rectangle (\xmax-0.5,\ymax-0.1);
			\draw[smooth,samples=300][domain=-4:5.5] plot(\x,{0.071*(\x)^3-0.142*(\x)^2-1.07*(\x)});
		\end{tikzpicture}
	}
	\loigiai{
		\immini{
			Xét hàm số $h(x)=4 f(x)+x^2$ trên $\mathbb{R}$.\\
			Vì $f(x)$ là hàm số đa thức nên $h(x)$ cũng là hàm số đa thức và $h(0)=4 f(0)=0$.\\
			Ta có $h'(x)=4 f'(x)+2 x$. Do đó $h'(x)=0 \Leftrightarrow f'(x)=-\dfrac{1}{2}x$.\\
		}
		{
			\begin{tikzpicture}[scale=0.7,>=stealth, font=\footnotesize, line join=round, line cap=round]
				%\def\a{1} \def\b{-6} \def\c{9} \def\d{1} % Hệ số
				\def\xmin{-4} \def\xmax{6}
				\def\ymin{-3} \def\ymax{2} 
				%\draw[color=gray!50,dashed] (\xmin,\ymin) grid (\xmax,\ymax); 
				\draw[->] (\xmin,0)--(\xmax,0) node [below]{$x$};
				\draw[->] (0,\ymin)--(0,\ymax) node [left]{$y$};
				\node at (0,0) [below left]{$O$};
				%\node at (1,3) [below left]{$f'(x)$};
				%\node at (-1.3,4) {$f'(x)$};
				\draw[dashed] (-2,0) node[below]{$-2$}--(-2,1)--(0,1) node[below left]{$1$};
				\draw[dashed] (4,0) node[below]{$4$}--(4,-2)--(0,-2) node[below left]{$-2$};
				%\draw[dashed] (1,0) node[below]{$1$}--(1,1);
				%\draw[dashed] (-0.5,0) node[below left]{$-0{,}5$}--(-0.5,2.125);
				\clip (\xmin+0.1,\ymin+0.1) rectangle (\xmax-0.5,\ymax-0.1);
				\draw[smooth,samples=300][domain=-4:5.5] plot(\x,{0.071*(\x)^3-0.142*(\x)^2-1.07*(\x)});
				\draw[smooth,samples=300][domain=-4:5.5] plot(\x,{-0.5*(\x)});
			\end{tikzpicture}
		}
		Dựa vào sự tương giao của đồ thị hàm số $y=f'(x)$ và đường thẳng $y=-\dfrac{1}{2}x$, ta có
		$
		h'(x)=0 \Leftrightarrow x \in\{-2 ; 0 ; 4\}.\\
		$
		Bảng biến thiên của hàm số $h(x)$ như sau:
		\begin{center}
			\begin{tikzpicture}
				\tkzTabInit[nocadre,lgt=1.2,espcl=2.5,deltacl=0.6]
				{$x$ /0.6,$y'$ /0.6,$y$ /2}
				{$-\infty$,$-2$,$0$,$4$,$+\infty$}
				\tkzTabLine{,-,$0$,+,$0$,-,$0$,+,}
				\tkzTabVar{+/$+\infty$, -/$y_1$,+/$0$,-/$y_3$,+/$+\infty$}
			\end{tikzpicture}
		\end{center}
		Từ đó suy ra bảng biến thiên của hàm số $g(x)=|h(x)|$.\\
		Dựa vào bảng biến thiên trên, ta thấy hàm số $g(x)$ đồng biến trên khoảng $(0 ; 4)$.
	}
\end{ex}
\begin{ex}[Chuyên Thái Bình - 2020]%[2D1G1-2]
	\immini{
		Cho hàm số $f(x)$ liên tục trên $\mathbb{R}$ có đồ thị hàm số $y=f'(x)$ cho như hình vẽ bên.\\
		Hàm số $g(x)=2 f(|x-1|)-x^2+2 x+2020$ đồng biến trên khoảng nào?
		\choice
		{\True $(0 ; 1)$}
		{$(-3 ; 1)$}
		{$(1 ; 3)$}
		{$(-2 ; 0)$}
	}
	{
		\begin{tikzpicture}[scale=0.7,>=stealth, font=\footnotesize, line join=round, line cap=round]
			%\def\a{1} \def\b{-6} \def\c{9} \def\d{1} % Hệ số
			\def\xmin{-4} \def\xmax{5}
			\def\ymin{-3} \def\ymax{5} 
			%\draw[color=gray!50,dashed] (\xmin,\ymin) grid (\xmax,\ymax); 
			\draw[->] (\xmin,0)--(\xmax,0) node [below]{$x$};
			\draw[->] (0,\ymin)--(0,\ymax) node [left]{$y$};
			\node at (0,0) [below left]{$O$};
			%\node at (1,3) [below left]{$f'(x)$};
			\node at (-1.3,4) {$f'(x)$};
			\draw[dashed] (-1,0) node[above]{$-1$}--(-1,-1)--(0,-1) node[below left]{$-1$};
			\draw[dashed] (1,0) node[below]{$1$}--(1,1)--(0,1) node[below left]{$1$};
			\draw[dashed] (3,0) node[below]{$3$}--(3,3)--(0,3) node[below left]{$3$};
			%\draw[dashed] (1,0) node[below]{$1$}--(1,1);
			%\draw[dashed] (-0.5,0) node[below left]{$-0{,}5$}--(-0.5,2.125);
			\clip (\xmin+0.1,\ymin+0.1) rectangle (\xmax-0.5,\ymax-0.1);
			\draw[smooth,samples=300][domain=-2:4] plot(\x,{-0.5*(\x)^3+1.5*(\x)^2+1.5*(\x)-1.5});
			%\draw[smooth,samples=300] plot(\x,{(\x)^3+(\x)^2-2*(\x)+1});
		\end{tikzpicture}
	}
	\loigiai{
		Ta có đường thẳng $y=x$ cắt đồ thị hàm số $y=f'(x)$ tại các điểm $x=-1 ; x=1 ; x=3$ như hình vẽ sau:
		\begin{center}
			\begin{tikzpicture}[scale=0.7,>=stealth, font=\footnotesize, line join=round, line cap=round]
				%\def\a{1} \def\b{-6} \def\c{9} \def\d{1} % Hệ số
				\def\xmin{-4} \def\xmax{5}
				\def\ymin{-3} \def\ymax{5} 
				%\draw[color=gray!50,dashed] (\xmin,\ymin) grid (\xmax,\ymax); 
				\draw[->] (\xmin,0)--(\xmax,0) node [below]{$x$};
				\draw[->] (0,\ymin)--(0,\ymax) node [left]{$y$};
				\node at (0,0) [below left]{$O$};
				%\node at (1,3) [below left]{$f'(x)$};
				\node at (-1.3,4) {$f'(x)$};
				\node at (4,3.2) {$y=x$};
				\draw[dashed] (-1,0) node[above]{$-1$}--(-1,-1)--(0,-1) node[below left]{$-1$};
				\draw[dashed] (1,0) node[below]{$1$}--(1,1)--(0,1) node[below left]{$1$};
				\draw[dashed] (3,0) node[below]{$3$}--(3,3)--(0,3) node[below left]{$3$};
				%\draw[dashed] (1,0) node[below]{$1$}--(1,1);
				%\draw[dashed] (-0.5,0) node[below left]{$-0{,}5$}--(-0.5,2.125);
				\clip (\xmin+0.1,\ymin+0.1) rectangle (\xmax-0.5,\ymax-0.1);
				\draw[smooth,samples=300][domain=-2:4] plot(\x,{-0.5*(\x)^3+1.5*(\x)^2+1.5*(\x)-1.5});
				\draw[smooth,samples=300] plot(\x,{(\x)});
			\end{tikzpicture}
		\end{center}
		Dựa vào đồ thị của hai hàm số trên ta có $f'(x)>x \Leftrightarrow\hoac{&x<-1 \\ &1<x<3}$ và
		$ f'(x)<x \Leftrightarrow\hoac{&
			-1<x<1 \\
			&x>3.}$\\
		+Trường hợp 1: $x-1<0 \Leftrightarrow x<1$.\\
		Khi đó $g(x)=2 f(1-x)-x^2+2 x+2020$.\\
		Ta có $g'(x)=-2 f'(1-x)+2(1-x)$.
		$$
		g'(x)>0 \Leftrightarrow-2 f'(1-x)+2(1-x)>0 \Leftrightarrow f'(1-x)<1-x \Leftrightarrow\hoac{
			&- 1 < 1 - x < 1\\
			&1 - x > 3} \Leftrightarrow \hoac{&
			0<x<2 \\
			&x<-2.}
		$$
		Kết hợp điều kiện, ta có $g'(x)>0 \Leftrightarrow\hoac{&0<x<1 \\ &x<-2.}$\\
		
		+ Trường hợp 2: $x-1>0 \Leftrightarrow x>1$.\\
		Khi đó ta có $g(x)=2 f(x-1)-x^2+2 x+2020$.\\
		$ g'(x)=2 f'(x-1)-2(x-1)$\\
		$g'(x)>0 \Leftrightarrow 2 f'(x-1)-2(x-1)>0 \Leftrightarrow f'(x-1)>x-1 \Leftrightarrow\hoac{&
			x - 1 < - 1\\
			&1 < x - 1 < 3}\Leftrightarrow \hoac{
			&x<0 \\
			&2<x<4.}$
		Kết hợp điều kiện ta có $g'(x)>0 \Leftrightarrow 2<x<4$.\\
		Vậy hàm số $g(x)=2 f(|x-1|)-x^2+2 x+2020$ đồng biến trên khoảng $(0 ; 1)$.
	}
\end{ex}

\begin{ex}[Chuyên Lào Cai - 2020]%[2D1G1-2]
	\immini{
		Cho hàm số $f'(x)$ có đồ thị như hình bên.\\
		Hàm số $g(x)=f(3 x+1)+9 x^3+\dfrac{9}{2}x^2$ đồng biến trên khoảng nào dưới đây?
		\choice
		{$(-1 ; 1)$}
		{$(-2 ; 0)$}
		{$(-\infty ; 0)$}
		{\True $(1 ;+\infty)$}
	}
	{\begin{tikzpicture}[line join=round, line cap=round,>=stealth,thick,scale=.8]
			\tikzset{label style/.style={font=\footnotesize}}
			\draw[->] (-2.1,0)--(5.1,0) node[below left] {$x$};
			\draw[->] (0,-3.1)--(0,4.1) node[below left] {$y$};
			\draw (0,0) node [below left] {$O$};
			\foreach \x in {1,2,3}
			\draw[thin] (\x,1pt)--(\x,-1pt) node [below] {$\x$};
			\draw[thin](-1,1pt)--(1,-1pt)node[above left]{$-1$};
			\foreach \y in {-2,2}
			\draw[thin] (1pt,\y)--(-1pt,\y) node [above right] {$\y$};
			%\begin{scope}
			\clip (-2,-3) rectangle (5,4);
			\draw[samples=200,domain=-2:4,smooth,variable=\x] plot (\x,{(\x)^3-3*(\x)^2+2});
			%\end{scope}
			\draw[dashed] (-1,0)--(-1,-2)--(0,-2);
			\draw[dashed] (3,0)--(3,2)--(0,2);
			%\begin{scope}[on background layer]\path[white]node{MDD-134};\end{scope}
		\end{tikzpicture}
	}
	\loigiai
	{
		\immini{Xét hàm số $g(x)=f(3 x+1)+9 x^3+\dfrac{9}{2}x^2 \\
			\Rightarrow g'(x)=3 f'(3 x+1)+27 x^2+9 x$.\\
			Hàm số đồng biến  $\Leftrightarrow g'(x)>0 \Leftrightarrow 3 f'(3 x+1)+27 x^2+9 x>0$
			\\
			$
			\Leftrightarrow f'(3 x+1)+3 x(3 x+1)>0 \qquad (*)
			$\\
			Đặt $t=3 x+1$, khi đó  $(*) \Leftrightarrow f'(t)+(t-1) t>0$\\ $\Leftrightarrow f'(t)>-t^2+t$.\\
			Vẽ parabol $y=-x^2+x$ và đồ thị hàm số $f'(x)$ trên cùng một hệ trục
		}
		{
			\begin{tikzpicture}[line join=round, line cap=round,>=stealth,thick,scale=.8]
				\tikzset{label style/.style={font=\footnotesize}}
				\draw[->] (-2.1,0)--(5.1,0) node[below left] {$x$};
				\draw[->] (0,-3.1)--(0,4.1) node[below left] {$y$};
				\draw (0,0) node [below left] {$O$};
				\foreach \x in {1,2,3}
				\draw[thin] (\x,1pt)--(\x,-1pt) node [below] {$\x$};
				\draw[thin](-1,1pt)--(1,-1pt);
				\foreach \y in {-2,2}
				\draw[thin] (1pt,\y)--(-1pt,\y) node [above right] {$\y$};
				%\begin{scope}
				\clip (-2,-3) rectangle (5,4);
				\draw[samples=200,domain=-2:4,smooth,variable=\x] plot (\x,{(\x)^3-3*(\x)^2+2});
				\draw[samples=200,domain=-2:4,smooth,variable=\x] plot (\x,{-(\x)^2+(\x)});
				%\end{scope}
				\draw[dashed] (-1,0) node[above left]{$-1$}--(-1,-2)--(0,-2);
				\draw[dashed] (3,0)--(3,2)--(0,2);
				%\begin{scope}[on background layer]\path[white]node{MDD-134};\end{scope}
			\end{tikzpicture}
		}
		Dựa vào đồ thị ta thấy
		$
		f'(t)>-t^2+t \Leftrightarrow\hoac{&- 1 < t < 1\\
			&t > 2}\Rightarrow \hoac{&
			- 1 < 3 x + 1 < 1\\
			&3 x + 1 > 2} \Leftrightarrow \hoac{&
			\dfrac{-2}{3}<x<0\\
			&x>\dfrac{1}{3}.}
		$}
\end{ex}
\begin{ex}[Sở Phú Thọ-2020]%[2D1G1-2]
	\immini{
		Cho hàm số $y=f(x)$ có đồ thị hàm số $y=f'(x)$ như hình vẽ.\\
		Hàm số $g(x)=f\left(\mathrm{e}^x-2\right)-2020$ nghịch biến trên khoảng nào dưới đây?
		\choice
		{\True $\left(-1 ; \dfrac{3}{2}\right)$}
		{$(-1 ; 2)$}
		{$(0 ;+\infty)$}
		{$\left(\dfrac{3}{2}; 2\right)$}
	}
	{
		\begin{tikzpicture}[scale=0.7,>=stealth, font=\footnotesize, line join=round, line cap=round]
			\def\a{1} \def\b{-3} \def\c{0} \def\d{0} % Hệ số
			\def\xmin{-2} \def\xmax{4}
			\def\ymin{-5} \def\ymax{2} 
			%\draw[color=gray!50,dashed] (\xmin,\ymin) grid (\xmax,\ymax); 
			\draw[->] (\xmin,0)--(\xmax,0) node [below]{$x$};
			\draw[->] (0,\ymin)--(0,\ymax) node [left]{$y$};
			\node at (0,0) [above left]{$O$};
			\node at (3,0) [below right]{$3$};
			\draw[dashed] (2,0) node[above]{$2$}--(2,-4) --(0,-4) node[left]{$-4$};
			\clip (\xmin+0.1,\ymin+0.1) rectangle (\xmax-0.5,\ymax-0.1);
			\draw[smooth,samples=300] plot(\x,{\a*(\x)^3+\b*(\x)^2+\c*(\x)+\d});
		\end{tikzpicture}
	}
	
	\loigiai{
		Dựa vào đồ thị hàm số $y=f'(x)$ suy ra $f'(x) \leq 0 ~ \forall x<3$ và $f'(x)>0 ~ \forall x>3$.
		$
		g'(x)=\mathrm{e}^x f'\left(\mathrm{e}^x-2\right) .
		$
		Hàm số $g(x)=f\left(\mathrm{e}^x-2\right)-2020$ nghịch biến \\ $ \Leftrightarrow g'(x)<0 \Leftrightarrow \mathrm{e}^x f'\left(\mathrm{e}^x-2\right)<0$\\
		$
		\Leftrightarrow f'\left(\mathrm{e}^x-2\right)<0 \Leftrightarrow \mathrm{e}^x-2<3 \Leftrightarrow \mathrm{e}^x<5 \Leftrightarrow x<\ln 5 .
		$\\
		Vậy hàm số đã cho nghịch biến trên $\left(-1 ; \dfrac{3}{2}\right)$.
	}
\end{ex}
\begin{ex}[Lý Nhân Tông - Bắc Ninh - 2020]%[2D1G1-2]
	\immini{
		Cho hàm số $f(x)$ có đồ thị hàm số $f'(x)$ như hình vẽ.\\
		Hàm số $y=f(\cos x)+x^2-x$ đồng biến trên khoảng
		\choice
		{$(-2 ; 1)$}
		{$(0 ; 1)$}
		{\True $(1 ; 2)$}
		{$(-1 ; 0)$}
	}
	{
		\begin{tikzpicture}[scale=1,>=stealth, font=\footnotesize, line join=round, line cap=round]
			\def\a{-0.5} \def\b{0} \def\c{1.5} \def\d{0} % Hệ số
			\def\xmin{-3} \def\xmax{4}
			\def\ymin{-2} \def\ymax{2} 
			%\draw[color=gray!50,dashed] (\xmin,\ymin) grid (\xmax,\ymax); 
			\draw[->] (\xmin,0)--(\xmax,0) node [below]{$x$};
			\draw[->] (0,\ymin)--(0,\ymax) node [left]{$y$};
			\node at (0,0) [above left]{$O$};
			\node at (3,0) [below right]{$3$};
			\draw[dashed] (-2,0) node[below]{$-2$}--(-2,1) --(0,1) node[above right]{$1$} --(1,1)--(1,0) node[below]{$1$};
			\draw[dashed] (-1,0) node[below right]{$-1$}--(-1,-1) --(0,-1) node[above right]{$-1$} --(2,-1)--(2,0) node[below right]{$2$};
			\clip (\xmin+0.1,\ymin+0.1) rectangle (\xmax-0.5,\ymax-0.1);
			\draw[smooth,samples=300][domain=-2:2] plot(\x,{\a*(\x)^3+\b*(\x)^2+\c*(\x)+\d});
		\end{tikzpicture}
	}
	\loigiai{
		Đặt  $g(x)=f(\cos x)+x^2-x$.\\
		Ta có $g'(x)=-\sin x \cdot f'(\cos x)+2 x-1$\\
		Vì $\cos x \in[-1 ; 1]$ nên từ đồ thị $f'(x)$ ta suy ra $f'(\cos x) \in[-1 ; 1]$.\\
		Do đó $\left|-\sin x \cdot f'(\cos x)\right| \leq 1, ~\forall x \in \mathbb{R}$.\\
		Ta suy ra $g'(x)=\sin x \cdot f'(\cos x)+2 x-1 \geq-1+2 x-1=2 x-2$
		$\Rightarrow g'(x)>0, ~\forall x>1$.\\
		Vậy hàm số đồng biến trên $(1 ; 2)$.
	}
\end{ex}
\begin{ex}[THPT Nguyễn Viết Xuân - 2020]%[2D1G1-2]
	\immini{
		Cho hàm số $f(x)$. Hàm số $y=f'(x)$ có đồ thị như hình vẽ.\\
		Hàm số $g(x)=f\left(3 x^2-1\right)-\dfrac{9}{2}x^4+3 x^2$ đồng biến trên khoảng nào dưới đây?
		\choice
		{\True $\left(-\dfrac{2 \sqrt{3}}{3}; \dfrac{-\sqrt{3}}{3}\right)$}
		{$\left(0 ; \dfrac{2 \sqrt{3}}{3}\right)$}
		{$(1 ; 2)$}
		{$\left(-\dfrac{\sqrt{3}}{3}; \dfrac{\sqrt{3}}{3}\right)$} 
	}
	{
		\begin{tikzpicture}[scale=0.6,>=stealth, font=\footnotesize, line join=round, line cap=round]
			\def\a{0.25} \def\b{0.25} \def\c{-2} \def\d{0} % Hệ số
			\def\xmin{-5} \def\xmax{4}
			\def\ymin{-5} \def\ymax{5} 
			%\draw[color=gray!50,dashed] (\xmin,\ymin) grid (\xmax,\ymax); 
			\draw[->] (\xmin,0)--(\xmax,0) node [below]{$x$};
			\draw[->] (0,\ymin)--(0,\ymax) node [left]{$y$};
			\node at (0,0) [above left]{$O$};
			%\node at (3,0) [below right]{$3$};
			\draw[dashed] (-4,0) node[below left]{$-4$}--(-4,-4) --(0,-4) node[above right]{$-4$};
			\draw[dashed] (3,0) node[below right]{$3$}--(3,3) --(0,3) node[above right]{$3$};
			\clip (\xmin+0.1,\ymin+0.1) rectangle (\xmax-0.5,\ymax-0.1);
			\draw[smooth,samples=300] plot(\x,{\a*(\x)^3+\b*(\x)^2+\c*(\x)+\d});
		\end{tikzpicture}
	}
	
	\loigiai
	{
		TXĐ: $\mathscr{D}=\mathbb{R}$.\\
		Ta có $g'(x)=6 x f'\left(3 x^2-1\right)-18 x^3+6 x=6 x\left[f'\left(3 x^2-1\right)-3 x^2+1\right]$.\\
		$
		g'(x)=0 \Leftrightarrow\hoac{
			&x = 0\\
			&f '( 3 x ^{2}- 1 ) = 3 x ^{2}- 1}
		\Leftrightarrow \hoac{
			&x = 0\\
			&3 x ^{2}- 1 = - 4 \text{~(vô nghiệm)}\\
			&3 x ^{2}- 1 = 0\\
			&3 x ^{2}- 1 = 3}\Leftrightarrow \hoac{&x=0 \\
			&x=\pm \dfrac{\sqrt{3}}{3}\\
			&x=\pm \dfrac{2 \sqrt{3}}{3}.}
		$\\
		Bảng xét dấu
		\begin{center}
			\begin{tikzpicture}
				\tkzTabInit[nocadre,lgt=1.2,espcl=2.2,deltacl=0.6]
				{$x$ /1.2,$f'(x)$ /0.7}
				{$-\infty$,$-\dfrac{2 \sqrt{3}}{3}$,$-\dfrac{ \sqrt{3}}{3}$,$0$,$\dfrac{\sqrt{3}}{3}$,$\dfrac{2 \sqrt{3}}{3}$,$+\infty$}
				\tkzTabLine{,-,$0$,+,$0$,-,$0$,+,$0$,-,$0$,+,}
			\end{tikzpicture}
		\end{center}
		Vậy hàm số đồng biến trong khoảng $\left(-\dfrac{2 \sqrt{3}}{3}; \dfrac{-\sqrt{3}}{3}\right)$.}
\end{ex}
\begin{ex}[Trần Phú - Quảng Ninh - 2020]%[2D1G1-2]
	Cho hàm số $f(x)$ có bảng xét dấu của đạo hàm như sau
	\begin{center}
		\begin{tikzpicture}
			\tkzTabInit[nocadre,lgt=1.2,espcl=2,deltacl=0.6]
			{$x$ /0.6,$f'(x)$ /0.6}
			{$-\infty$,$-4$,$-1$,$2$,$7$,$+\infty$}
			\tkzTabLine{,+,$0$,-,$0$,+,$0$,-,$0$,+,}
		\end{tikzpicture}
	\end{center}
	Hàm số $y=f(2 x+1)+\dfrac{2}{3}x^3-8 x+5$ nghịch biến trên khoảng nào dưới đây?
	\choice
	{$(-\infty ;-2)$}
	{$(1 ;+\infty)$}
	{$(-1 ; 7)$}
	{\True $\left(-1 ; \dfrac{1}{2}\right)$}
	\loigiai{
		Ta có $y'=2 f'(2 x+1)+2 x^2-8$.\\
		Xét $y'\leq 0 \Leftrightarrow 2 f'(2 x+1)+2 x^2-8 \leq 0 \Leftrightarrow f'(2 x+1) \leq 4-x^2$.\\
		Đặt $t=2x+1$, ta có $f'(t) \leq \dfrac{-t^2+2 t+15}{4}$.\\
		Vì $\dfrac{-t^2+2 t+15}{4}\geq 0, \forall t \in[-3 ; 5]$.\\
		Mà $f'(t) \leq 0, \forall t \in[-3 ; 2]$.\\
		Nên $f'(t) \leq \dfrac{-t^2+2 t+15}{4}\Rightarrow t \in[-3 ; 2]$.\\
		Suy ra $-3 \leq 2 x+1 \leq 2 \Leftrightarrow-2 \leq x \leq \dfrac{1}{2}$.}
\end{ex}

\begin{ex}[Chuyên Thái Bình - Lần 3 - 2020]%[2D1G1-2]
	\immini{
		Cho hàm số $y=f(x)$ liên tục trên $\mathbb{R}$ có đồ thị hàm số $y=f'(x)$ cho như hình vẽ.\\
		Hàm số $g(x)=2 f(|x-1|)-x^2+2 x+2020$ đồng biến trên khoảng nào?
		\choice
		{\True $(0 ; 1)$}
		{$(-3 ; 1)$}
		{$(1 ; 3)$}
		{$(-2 ; 0)$}
	}
	{
		\begin{tikzpicture}[scale=0.7,>=stealth, font=\footnotesize, line join=round, line cap=round]
			\def\a{-0.333} \def\b{1} \def\c{1.333} \def\d{-1} % Hệ số
			\def\xmin{-3} \def\xmax{5}
			\def\ymin{-3} \def\ymax{5} 
			%\draw[color=gray!50,dashed] (\xmin,\ymin) grid (\xmax,\ymax); 
			\draw[->] (\xmin,0)--(\xmax,0) node [below]{$x$};
			\draw[->] (0,\ymin)--(0,\ymax) node [left]{$y$};
			\node at (0,0) [above left]{$O$};
			%\node at (3,0) [below right]{$3$};
			\draw[dashed] (-1,0) node[above]{$-1$}--(-1,-1) --(0,-1) node[above right]{$-1$};
			\draw[dashed] (1,0) node[below right]{$1$}--(1,1) --(0,1) node[above right]{$1$};
			\draw[dashed] (3,0) node[below right]{$3$}--(3,3) --(0,3) node[above right]{$3$};
			\clip (\xmin+0.1,\ymin+0.1) rectangle (\xmax-0.5,\ymax-0.1);
			\draw[smooth,samples=300] plot(\x,{\a*(\x)^3+\b*(\x)^2+\c*(\x)+\d});
			\draw[smooth,samples=300] plot(\x,{(\x)});
		\end{tikzpicture}
	}
	\loigiai{
		Với $x>1$, ta có $g(x)=2 f(x-1)-(x-1)^2+2021 \Rightarrow g'(x)=2 f'(x-1)-2(x-1)$.\\
		Hàm số đồng biến $\Leftrightarrow 2 f'(x-1)-2(x-1)>0 \Leftrightarrow f'(x-1)>x-1 \quad(*)$.\\
		Đặt $t=x-1$, khi đó $(*) \Leftrightarrow f'(t)>t \Leftrightarrow\hoac{&1<t<3 \\ &t<-1}\Rightarrow\hoac{&2<x<4 \\ &x<0 ~(\text{loại}).}$\\
		Với $x<1$, ta có $g(x)=2 f(1-x)-(1-x)^2+2021 \Rightarrow g'(x)=-2 f'(1-x)+2(1-x)$.\\
		Hàm số đồng biến $\Leftrightarrow-2 f'(1-x)+2(1-x)>0 \Leftrightarrow f'(1-x)<1-x \quad(* *)$.\\
		Đặt $t=1-x$, khi đó $(* *) \Leftrightarrow f'(t)<t \Leftrightarrow\hoac{&-1<t<1 \\ &t>3}\Rightarrow\hoac{&0<x<2 \\ &x<-2}\Rightarrow\hoac{&0<x<1 \\ &x<-2.}$\\
		Vậy hàm số $g(x)$ đồng biến trên các khoảng $(-\infty ;-2),(0 ; 1),(2 ; 4)$.
	}
\end{ex}
\begin{ex}[Sở Phú Thọ - 2020]%[2D1G1-2]
	\immini{
		Cho hàm số $y=f(x)$ có đồ thị hàm số $f'(x)$ như hình vẽ.\\
		Hàm số $g(x)=f\left(1+e^x\right)+2020$ nghịch biến trên khoảng nào dưới đây?
		\choice
		{$(0 ;+\infty)$}
		{$\left(\dfrac{1}{2}; 1\right)$}
		{\True $\left(0 ; \dfrac{1}{2}\right)$}
		{$(-1 ; 1)$}
	}{
		\begin{tikzpicture}[scale=0.7,>=stealth, font=\footnotesize, line join=round, line cap=round]
			\def\a{1} \def\b{-3} \def\c{0} \def\d{0} % Hệ số
			\def\xmin{-2} \def\xmax{4}
			\def\ymin{-5} \def\ymax{2} 
			%\draw[color=gray!50,dashed] (\xmin,\ymin) grid (\xmax,\ymax); 
			\draw[->] (\xmin,0)--(\xmax,0) node [below]{$x$};
			\draw[->] (0,\ymin)--(0,\ymax) node [left]{$y$};
			\node at (0,0) [above left]{$O$};
			\node at (3,0) [below right]{$3$};
			\draw[dashed] (2,0) node[above]{$2$}--(2,-4) --(0,-4) node[left]{$-4$};
			\clip (\xmin+0.1,\ymin+0.1) rectangle (\xmax-0.5,\ymax-0.1);
			\draw[smooth,samples=300] plot(\x,{\a*(\x)^3+\b*(\x)^2+\c*(\x)+\d});
		\end{tikzpicture}
	}
	\loigiai{
		$g'(x)=e^x f'\left(1+e^x\right)$.\\
		Do $e^x>0, \forall x$ nên $g'(x) \leq 0 \Leftrightarrow f'\left(1+e^x\right) \leq 0 \Leftrightarrow 1+e^x \leq 3 \Leftrightarrow x \leq \ln 2$, dấu bằng xảy ra tại hữu hạn điểm.\\
		Nên $g(x)$ nghịch biến trên $(-\infty ; \ln 2)$.\\
		Vì $\left(0 ; \dfrac{1}{2}\right) \subset (-\infty ; \ln 2)$ nên hàm số đã cho nghịch biến trên $\left(0 ; \dfrac{1}{2}\right)$.
	}
\end{ex}

\begin{ex}%[2D1K1-2]
	[THPT Anh Sơn - Nghệ An - 2020]
	Cho hàm số $y=f(x)$ có bảng xét dấu của đạo hàm như sau.
	\begin{center}
		\begin{tikzpicture}
			\tkzTabInit[nocadre,lgt=1.2,espcl=2,deltacl=0.6]
			{$x$ /0.6,$f'(x)$ /0.6}
			{$-\infty$,$-2$,$-1$,$2$,$4$,$+\infty$}
			\tkzTabLine{,+,$0$,-,$0$,+,$0$,-,$0$,+,}
		\end{tikzpicture}
	\end{center}
	Hàm số $y=-2 f(x)+2019$ nghịch biến trên khoảng nào trong các khoảng dưới đây?
	\choice
	{$(2 ; 4)$}
	{$(-4 ; 2)$}
	{$(-2 ;-1)$}
	{\True $(-1 ; 2)$}
	\loigiai{
		Ta có $y'=-2 f'(x)$.\\
		$
		y'=0 \Leftrightarrow-2 f'(x)=0 \Leftrightarrow\hoac{&
			x=-2 \\
			&x=-1 \\
			&x=2 \\
			&x=4.}$\\
		Từ bảng xét dấu của $f'(x)$ ta có
		\begin{center}
			\begin{tikzpicture}
				\tkzTabInit[nocadre,lgt=1,espcl=2,deltacl=0.6]
				{$x$ /0.6,$y'$ /0.6}
				{$-\infty$,$-2$,$-1$,$2$,$4$,$+\infty$}
				\tkzTabLine{,-,$0$,+,$0$,-,$0$,+,$0$,-,}
			\end{tikzpicture}
		\end{center}
		Từ bảng xét dấu ta có hàm số nghịch biến trên khoảng $(-\infty ;-2),(-1 ; 2)$ và $(4 ;+\infty)$.}
\end{ex}

\begin{ex}[THPT Anh Sơn - Nghệ An - 2020]%[2D1G1-2]
	Cho hàm số $f(x)$ xác định và liên tục trên $\mathbb{R}$ và có đạo hàm $f'(x)$ thỏa mãn $f'(x)=(1-x)(x+2) g(x)+2019$ với $g(x)<0, ~\forall x \in \mathbb{R}$ . Hàm số $y=f(1-x)+2019 x+2020$ nghịch biến trên khoảng nào?
	\choice
	{$(1 ;+\infty)$}
	{$(0 ; 3)$}
	{$(-\infty ; 3)$}
	{\True $(3 ;+\infty)$}
	\loigiai{
		Đặt $h(x)=f(1-x)+2019 x+2020$.\\
		Vì hàm số $f(x)$ xác định trên $\mathbb{R}$ nên hàm số $h(x)$ cũng xác định trên $\mathbb{R}$.\\
		Ta có $h'(x)=-f'(1-x)+2019$.\\
		Do $h'(x)=0$ tại hữu hạn điểm nên để tìm khoảng nghịch biến của hàm số $h(x)$, ta tìm các giá trị của $x$ sao cho $h'(x)<0 \Leftrightarrow-f'(1-x)+2019<0$\\ 
		$\Leftrightarrow f'(1-x)-2019>0 \\
		\Leftrightarrow x(3-x) g(1-x)>0 \Leftrightarrow x(3-x)<0(\text{~do~}g(x)<0, \forall x \in \mathbb{R})$\\
		$\Leftrightarrow\hoac{&
			x<0 \\
			&x>3.}$\\
		Vậy hàm số $y=f(1-x)+2019 x+2020$ nghịch biến trên các khoảng $(-\infty ; 0)$ và $(3 ;+\infty).$}
\end{ex}

\begin{ex}%[2D1G1-2]
	Cho hàm số $y=f(x)$ xác định trên $\mathbb{R}$ và có bảng xét dấu đạo hàm như sau:
	\begin{center}
		\begin{tikzpicture}
			\tkzTabInit[nocadre,lgt=2,espcl=2,deltacl=0.6]
			{$x$ /0.6,$f'(x)$ /0.6}
			{$-\infty$,$-1$,$1$,$4$,$+\infty$}
			\tkzTabLine{,-,$0$,+,$0$,-,$0$,+,}
		\end{tikzpicture}
	\end{center}
	Biết $f(x)>2,~ \forall x \in \mathbb{R}$. Xét hàm số $g(x)=f(3-2 f(x))-x^3+3 x^2-2020$. Khẳng định nào sau đây đúng?
	\choice
	{Hàm số $g(x)$ đồng biến trên khoảng $(-2 ;-1)$}
	{Hàm số $g(x)$ nghịch biến trên khoảng $(0 ; 1)$}
	{Hàm số $g(x)$ đồng biến trên khoảng $(3 ; 4)$}
	{\True Hàm số $g(x)$ nghịch biến trên khoảng $(2 ; 3)$}
	\loigiai{
		Ta có $g'(x)=-2 f'(x) f'(3-2 f(x))-3 x^2+6 x$.\\
		Vì $f(x)>2, ~\forall x \in \mathbb{R}$ nên $3-2 f(x)<-1 ~\forall x \in \mathbb{R}$.\\
		Từ bảng xét dấu $f'(x)$ suy ra $f'(3-2 f(x))<0, ~\forall x \in \mathbb{R}$.\\
		Từ đó ta có bảng xét dấu sau:
		\begin{center}
			\begin{tikzpicture}
				\tkzTabInit[nocadre,lgt=4,espcl=1.7,deltacl=0.6]
				{$x$ /0.7,$-f'(x)f'\left( 3-2f(x)\right) $/0.8,$-3x^2+6x$/0.7}
				{$-\infty$,$-1$,$0$,$1$,$2$,$4$,$+\infty$}
				\tkzTabLine{,-,$0$,+,|,+,$0$,-,|,-,$0$,+,}
				\tkzTabLine{,-,|,-,$0$,+,|,+,$0$,-,|,-,}
			\end{tikzpicture}
		\end{center}
		Từ bảng xét dấu trên, loại trừ đáp án suy ra hàm số $g(x)$ nghịch biến trên khoảng $(2 ; 3)$.}
\end{ex}

\begin{ex}%[2D1G1-2]
	Cho hàm số $f(x)$ có bảng biến thiên như sau:
	\begin{center}
		\begin{tikzpicture}
			\tkzTabInit[nocadre,lgt=1.2,espcl=2.5,deltacl=0.6]
			{$x$ /0.7, $f'(x)$ /0.7, $f(x)$ /2.5}
			{$-\infty$,$1$,$2$,$3$,$4$,$+\infty$}
			\tkzTabLine{,+,$0$,-,$0$,+,$0$,-,$0$,+,}
			\tkzTabVar{-/$-\infty$,+/$3$,-/$1$,+/$2$,-/$0$,+/$+\infty$}
		\end{tikzpicture}
	\end{center}
	Hàm số $y=(f(x))^3-3 .(f(x))^2$ nghịch biến trên khoảng nào dưới đây?
	\choice
	{$(1 ; 2)$}
	{$(3 ; 4)$}
	{$(-\infty ; 1)$}
	{\True $(2 ; 3)$}
	\loigiai{
		Ta có $y'=3 \cdot(f(x))^2 \cdot f'(x)-6 \cdot f(x) \cdot f'(x)=3 f(x) \cdot f'(x) \cdot[f(x)-2]. \\
		y'=0 \Leftrightarrow \hoac{&f(x)=0 \Leftrightarrow x \in\left\{x_1, 4 \mid x_1<1\right\}\\
			&f(x)=2 \Leftrightarrow x \in\left\{x_2, x_3, 3, x_4 \mid x_1<x_2<1<x_3<2 ; 4<x_4\right\}\\
			&f'(x)=0 \Leftrightarrow x \in\{1,2,3,4\}.}$\\
		Lập bảng xét dấu ta có
		\begin{center}
			\begin{tikzpicture}
				\tkzTabInit[nocadre,lgt=2,espcl=1.5,deltacl=0.6]
				{$x$ /0.7,$f(x)$ /0.7,$f(x)-2$ /0.7,$f'(x)$/0.7,$y'$/0.7}
				{$-\infty$,$x_1$,$x_2$,$1$,$x_3$,$2$,$3$,$4$,$x_4$,$+\infty$}
				\tkzTabLine{,-,$0$,+,|,+,|,+,|,+,|,+,$0$,+,|,+,|,+,}
				\tkzTabLine{,-,|,-,$0$,+,$0$,+,$0$,-,|,-,$0$,-,|,-,$0$,+}
				\tkzTabLine{,+,|,+,|,+,$0$,-,|,-,$0$,+,$0$,-,$0$,+,|,+}
				\tkzTabLine{,+,$0$,-,$0$,+,$0$,-,$0$,+,$0$,-,$0$,+,$0$,-,$0$,+}
			\end{tikzpicture}
		\end{center}
		
		Do đó hàm số nghịch biến trên khoảng $(2 ; 3)$.
	}
\end{ex}
\begin{ex}%[2D1G1-2]
	Cho hàm số $y=f(x)$ có đồ thị nằm trên trục hoành và có đạo hàm trên $\mathbb{R}$, bảng xét dấu của biểu thức $f'(x)$ như bảng dưới đây.
	\begin{center}
		\begin{tikzpicture}
			\tkzTabInit[nocadre,lgt=1.2,espcl=2,deltacl=0.6]
			{$x$ /0.6,$f'(x)$ /0.6}
			{$-\infty$,$-2$,$-1$,$3$,$+\infty$}
			\tkzTabLine{,-,$0$,+,$0$,-,$0$,+,}
		\end{tikzpicture}
	\end{center}
	Hàm số $y=g(x)=\dfrac{f\left(x^2-2 x\right)}{f\left(x^2-2 x\right)+1}$ nghịch biến trên khoảng nào dưới đây?
	\choice
	{$(-\infty ; 1)$}
	{$\left(-2 ; \dfrac{5}{2}\right)$}
	{\True $(1 ; 3)$}
	{$(2 ;+\infty)$}
	\loigiai{
		$ g'(x)=\dfrac{\left(x^2-2 x\right)'\cdot f'\left(x^2-2 x\right)}{\left(f\left(x^2-2 x\right)+1\right)^2}=\dfrac{(2 x-2) \cdot f'\left(x^2-2 x\right)}{\left(f\left(x^2-2 x\right)+1\right)^2}. \\
		g'(x)=0 \Leftrightarrow\hoac{
			&2 x - 2 = 0\\
			&f '( x ^{2}- 2 x ) = 0}
		\Leftrightarrow \hoac{&x = 1\\
			&x ^{2}- 2 x = - 2\\
			&x ^{2}- 2 x = - 1\\
			&x ^{2}- 2 x = 3}
		\Leftrightarrow \hoac{&x=1 \\
			&x=-1 \\
			&x=3.}
		$\\
		Ta có bảng xét dấu của $g'(x)$
		\begin{center}
			\begin{tikzpicture}
				\tkzTabInit[nocadre,lgt=1.2,espcl=2,deltacl=0.6]
				{$x$ /0.6,$g'(x)$ /0.6}
				{$-\infty$,$-1$,$1$,$3$,$+\infty$}
				\tkzTabLine{,-,$0$,+,$0$,-,$0$,+,}
			\end{tikzpicture}
		\end{center}
		Dựa vào bảng xét dấu ta có hàm số $y=g(x)$ nghịch biến trên các khoảng $(-\infty ;-1)$ và $(1 ; 3)$.}
\end{ex}
\begin{ex}[Liên trường huyện Quảng Xương - Thanh Hóa - 2021]%[2D1G1-2]
	\immini{
		Cho các hàm số $y=f(x)$; $y=g(x)$ liên tục trên $\mathbb{R}$ và có đồ thị các đạo hàm $f'(x) ; g'(x)$ (đồ thị hàm số $y=g'(x)$ là đường đậm hơn) như hình vẽ.\\
		Hàm số $h(x)=f(x-1)-g(x-1)$ nghịch biến trên khoảng nào dưới đây?
		\choice
		{$\left(\dfrac{1}{2}; 1\right)$}
		{$(1 ;+\infty)$}
		{$(2 ;+\infty)$}
		{\True $\left(-1 ; \dfrac{1}{2}\right)$}
	}
	{
		\begin{tikzpicture}[scale=1,>=stealth, font=\footnotesize, line join=round, line cap=round]
			%\def\a{1} \def\b{-6} \def\c{9} \def\d{1} % Hệ số
			\def\xmin{-4} \def\xmax{3}
			\def\ymin{-2} \def\ymax{4} 
			%\draw[color=gray!50,dashed] (\xmin,\ymin) grid (\xmax,\ymax); 
			\draw[->] (\xmin,0)--(\xmax,0) node [below]{$x$};
			\draw[->] (0,\ymin)--(0,\ymax) node [left]{$y$};
			\node at (0,0) [above left]{$O$};
			\node at (1,3) [below left]{$f'(x)$};
			\node at (1.5,3) [below right]{$g'(x)$};
			\draw[dashed] (-2,0) node[above right]{$-2$}--(-2,1);
			\draw[dashed] (1,0) node[below]{$1$}--(1,1);
			\draw[dashed] (-0.5,0) node[below]{$-0{,}5$}--(-0.5,2.125);
			\clip (\xmin+0.1,\ymin+0.1) rectangle (\xmax-0.5,\ymax-0.1);
			\draw[smooth,samples=300][domain=-3:2] plot(\x,{2*(\x)^4+4*(\x)^3-2*(\x)^2-4*(\x)+1});
			\draw[smooth,samples=300,line width=1.2pt] plot(\x,{(\x)^3+(\x)^2-2*(\x)+1});
		\end{tikzpicture}
	}
	
	\loigiai{
		Ta có: $h'(x)=f'(x-1)-g'(x-1)$.\\
		Dựa vào hình vẽ ta có hàm số $h(x)$ nghịch biến\\
		$\Leftrightarrow h'(x)<0 \Leftrightarrow f'(x-1)<g'(x-1)$\\
		$
		\Leftrightarrow\hoac{&- 2 < x - 1 < - \dfrac{1}{2}\\
			&0 < x - 1 < 1}
		\Leftrightarrow \hoac{
			&-1<x<\dfrac{1}{2}\\
			&1<x<2.}$\\
		Do đó hàm số $h(x)$ nghịch biến trên các khoảng $\left(-1 ; \dfrac{1}{2}\right)$ và $(1 ; 2)$.
	}
\end{ex}
\begin{ex}[THPT Quế Võ 1 - Bắc Ninh - 2021] %[2D1G1-2]
	\immini{
		Cho ba hàm số $y=f(x), y=g(x), y=h(x)$. Đồ thị của ba hàm số $y=f'(x), y=g'(x), y=h'(x)$ được cho như hình vẽ.\\
		Hàm số $k(x)=f(x+7)+g(5 x+1)-h\left(4 x+\dfrac{3}{2}\right)$ đồng biến trên khoảng nào dưới đây?
		\choice
		{$\left(-\dfrac{5}{8}; 0\right)$}
		{$\left(\dfrac{5}{8};+\infty\right)$}
		{\True $\left(\dfrac{3}{8}; 1\right)$}
		{$\left(-\dfrac{3}{8}; 1\right)$}
	}
	{
		\begin{tikzpicture}[scale=0.25,>=stealth, font=\footnotesize, line join=round, line cap=round]
			\def\a{-.078} \def\b{1.25} \def\c{0} % Hệ số
			\def\xmin{-4} \def\xmax{25}
			\def\ymin{-8} \def\ymax{18}
			
			%\draw[color=gray!50,dashed] (\xmin,\ymin) grid (\xmax,\ymax);
			
			\draw[->] (\xmin,0)--(\xmax,0) node [below]{$x$};
			\draw[->] (0,\ymin)--(0,\ymax) node [left]{$y$};
			\node at (20,14) [below right]{$y=g'(x)$};
			\node at (18,-2) [below left]{$y=h'(x)$};
			\node at (16,5) [below right]{$y=f'(x)$};
			\node at (0,0) [below left]{$O$};
			\draw[dashed] (3,0) node[below]{$3$}--(3,10)--(0,10) node[left]{$10$};
			\draw[dashed] (8,0) node[below]{$8$}--(8,5)--(0,5) node[left]{$5$};
			\draw[dashed] (4,0) node[below]{$4$}--(4,2)--(0,2) node[left]{$2$};
			\clip (\xmin+0.1,\ymin+0.1) rectangle (\xmax-0.5,\ymax-0.1);
			\draw[smooth,samples=300,domain=-2:18] plot(\x,{\a*(\x)^2+\b*(\x)+\c});
			%\draw[smooth,samples=300,domain=-2:25] plot(\x,{0.02*(\x)^3-0.6*(\x)^2+5.16*(\x)});
			\draw[line width=1.2pt] (-2,5)..controls (1.7,1.5) and (4.5,1.6)..(7,2.6);
			\draw[line width=1.2pt] (7,2.6)..controls (9,3.5) and (12,5)..(20,13);
			\draw (-0.5,-2) -- (0,0)--(3,10).. controls +(65:1) and + (-190:1)..(6,15).. controls +(0:1) and + (-180:1)..(14,-1).. controls +(0:1) and + (+80:1)..(19,16);
			
		\end{tikzpicture}
	}
	\loigiai{
		Ta có $k'(x)=f'(x+7)+5 g'(5 x+1)-4 h'\left(4 x+\dfrac{3}{2}\right)$.\\
		Khi $x \in \left( \dfrac{3}{8};1\right)$ thì $\heva{&7{,}375<x+7<8\\&2{,}875<5x+1<6\\&3<4x+\dfrac{4}{3}<5{,}5}\Leftrightarrow \heva{&f'(x+7)>10\\&g'(5x+1)>2 \Rightarrow 5g'(5x+1)>10  \\&h'\left( 4x+\dfrac{3}{2}\right)<5 \Rightarrow -4h'\left( 4x+\dfrac{3}{2}\right) >-20}.$\\
		Do đó $k'(x)=f'(x+7)+5g'(5x+1)-4h'\left( 4x+\dfrac{3}{2}\right)>0$.\\
		Hàm số $k(x)=f(x+7)+g(5 x+1)-h\left(4 x+\dfrac{3}{2}\right)$ đồng biến trên $\left(\dfrac{3}{8}; 1\right)$.
	}
\end{ex}
\begin{ex}[THPT Thanh Chương 1 - Nghệ An- 2021] %[2D1G1-2]
	Cho hàm số $y=f(x)$ liên tục trên $\mathbb{R}$ có bảng xét dấu đạo hàm như sau
	\begin{center}
		\begin{tikzpicture}
			\tkzTabInit[nocadre,lgt=1.2,espcl=2,deltacl=0.6]
			{$x$ /0.6,$f'(x)$ /0.6}
			{$-\infty$,$1$,$2$,$3$,$4$,$+\infty$}
			\tkzTabLine{,-,$0$,+,$0$,+,$0$,-,$0$,+,}
		\end{tikzpicture}
	\end{center}
	Hàm số $y=3f(2x-1)-4x^3+15x^2-18x+1$ đồng biến trên khoảng nào dưới đây?
	\choice
	{$\left(3;+\infty\right)$}
	{\True $\left(1;\dfrac{3}{2}\right)$}
	{$\left(\dfrac{5}{2}; 3\right)$}
	{$\left(2;\dfrac{5}{2}\right)$}
	\loigiai{
		Ta có $y'=6f'(2x-1)-12x^2+30x-18=6\left[f'(2x-1)-2x^2+5x-3\right] $.\\
		Có $f'(2x-1)=0 \Leftrightarrow \hoac{&2x-1=1\\&2x-1=2\\&2x-1=3\\&2x-1=4} \Leftrightarrow \hoac{&x=1\\&x=\dfrac{3}{2}\\&x=2\\&x=\dfrac{5}{2}.}$
		Ta có bảng xét dấu sau
		\begin{center}
			\begin{tikzpicture}
				\tkzTabInit[nocadre,lgt=3.0,espcl=1.5,deltacl=0.6]
				{$x$ /1.0,$f(x)$ /0.6,$f'(2x-1)$ /0.6,$-2x^2+5x-3$/0.6,$g'(x)$/0.6}
				{$-\infty$,$1$,$\dfrac{3}{2}$,$2$,$\dfrac{5}{2}$,$3$,$4$,$+\infty$}
				\tkzTabLine{,-,$0$,+,|,+,$0$,+,|,+,$0$,-,$0$,+,}
				\tkzTabLine{,-,$0$,+,$0$,+,$0$,-,$0$,+,|,+,|,+,}
				\tkzTabLine{,-,$0$,+,$0$,-,|,-,|,-,|,-,|,-,}
				\tkzTabLine{,-,$0$,+,$0$,,?,,|,,?,?,,?,}
			\end{tikzpicture}
		\end{center}
		Dựa vào bảng xét dấu trên, ta kết luận hàm số đã cho đồng biến trên khoảng $\left( 1; \dfrac{3}{2}\right).$
	}
\end{ex}


\begin{ex}%[2D2G4-3] %Câu 27 
	[THPT Hoàng Hoa Thám-Đà Nẵng-2021]
	Cho hàm số $f(x)$ có bảng xét dấu của $f'(x)$ như sau:\\
	\begin{center}
		\begin{tikzpicture}
			\tkzTabInit[lgt=1.2,espcl=2.3]
			{$x$/0.7, $f'(x)$ /.8} % first column
			{$-\infty$,$-3$,$1$, $2$, $+\infty$} % first row
			\tkzTabLine { ,+,0,-,0,+,0,+ }
		\end{tikzpicture}
	\end{center}	
	Hàm số $y=f\left(2-e^x\right)-\dfrac{1}{3}{e^{3x}}+3e^{2x}-5e^x+1$ đồng biến trên khoảng nào dưới đây?
	\choice
	{$\left(0;\dfrac{3}{2}\right)$}
	{$\left(1;3\right)$}
	{\True $\left(-3;0\right)$}
	{$\left(-4;-3\right)$}
	\loigiai{
		Ta có $y'=-e^x.f'\left(2-e^x\right)-e^{3x}+6e^{2x}-5e^x=e^x\left[-f'\left(2-e^x\right)-e^{2x}+6e^x-5\right]$ .\\
		Đặt $t=2-e^x$, ta được\\
		$y'=\left(2-t\right)\left[-f'(t)-\left(2-t\right)^2+6\left(2-t\right)-5\right]=\left(2-t\right)\left[-f'(t)-t^2-2t+3\right]$ .\\
		$y'=0\Leftrightarrow\left(2-t\right)\left[-f'(t)-t^2-2t+3\right]=0\Leftrightarrow
		\hoac{
			& t=2\\ 
			& f'(t)=-t^2-2t+3.}$\\
		Hàm số $g(x)=-x^2-2x+3$ là parabol có trục đối xứng $x=-1$ và cắt trục hoành tại 2 điểm có hoành độ 
		$\hoac{
			& x=1\\ 
			& x=-3
		}$. Suy ra $f'(t)=-t^2-2t+3\Leftrightarrow \hoac{
			& t=1\\ 
			& t=-3. }$\\
		Bảng xét dấu\\
		\begin{center}
			\begin{tikzpicture}
				\tkzTabInit[lgt=3.9,espcl=2,nocadre]
				{$t$/0.7, $2-t$ /0.8, $-f'(t)-t^2-2t+3$ /0.8, $y'$ /0.8} % first column
				{$-\infty$,$-3$,$1$,$2$,$+\infty$} % first row
				\tkzTabLine { ,+,|,+,|,+,z,-, } % second row
				\tkzTabLine {,-,0,+,0,-,|,-,} % third row
				\tkzTabLine {,-,0,+,0,-,0,+,} % last row
			\end{tikzpicture}
		\end{center}
		Dựa vào bảng xét dấu $y'>0,\forall x\in\left(-3;0\right)$.}
\end{ex}


\begin{ex}%[2D1G1-2]%Câu 28 
	[Sở Lạng Sơn 2022] Cho hàm số $f(x)$ có bảng biến thiên như sau:\\
	\begin{center}
		\begin{tikzpicture}
			\tkzTabInit[espcl=2.5,lgt=1,nocadre]
			{$x$/0.7,$y'$/0.7,$y$/3.5}
			{$-\infty$,$1$,$2$,$3$,$4$,$+\infty$}
			\tkzTabLine{,+,0,-,0,+,0,-,0,+,}
			\node (0) at ($(N12)+(0,-3)$) {$-\infty$};
			\node (1) at ($(N22)+(0,-.5)$) {$3$};
			\node (2) at ($(N32)+(0,-1.7)$) {$1$};
			\node (3) at ($(N42)+(0,-0.7)$) {$2$};
			\node (4) at ($(N52)+(0,-2.3)$) {$0$};
			\node (5) at ($(N62)+(0,-.3)$) {$+\infty$};
			%				\node (8) at ($(N42)+(0,-.5)$) {};
			%				\coordinate (9) at ($(N42)!.6!(N53)+ (-0.5,0)$);
			%				\coordinate (6) at ($(T12)!.6!(T13)$);
			%				\coordinate (7) at ($(T22)!.6!(T23)$);
			\draw[-stealth] (0)--(1);
			\draw[-stealth] (1)--(2);
			\draw[-stealth] (2)--(3);
			\draw[-stealth] (1)--(2);
			\draw[-stealth] (3)--(4);
			\draw[-stealth] (4)--(5);
			%				\draw[->,red] (5)--(8);
			%				\draw[->,red] (8)--(9);
			%				\draw[blue,dashed](6)--(7)node[above left]{$y=0$};
		\end{tikzpicture}		
	\end{center}
	Hàm số $y=\left[f(x)\right]^3-3\left[f(x)\right]^2$ đồng biến trên khoảng nào dưới đây?
	\choice
	{$\left(-\infty\,;1\right)$}
	{$\left(1\,;2\right)$}
	{\True $\left(3\,;4\right)$}
	{$\left(2\,;3\right)$}
	\loigiai{
		Ta có $y'=3f'(x)\left[f^2(x)-2f(x)\right]$. 
		Phương trình $y'=0\Leftrightarrow \hoac{
			&{f}'(x)=0\\ 
			& f(x)=0\\ 
			& f(x)=2.
		}$
		\begin{center}
			\begin{tikzpicture}
				\tkzTabInit[espcl=2.5,lgt=1.5]
				{$x$/0.7,$y'$/0.7,$y$/3.5}
				{$-\infty$,$1$,$2$,$3$,$4$,$+\infty$}
				\tkzTabLine{,+,0,-,0,+,0,-,0,+,}
				\node (0) at ($(N12)+(0,-3)$) {$-\infty$};
				\node (1) at ($(N22)+(0,-.3)$) {$3$};
				\node (2) at ($(N32)+(0,-1.7)$) {$1$};
				\node (3) at ($(N42)+(0,-0.8)$) {$2$};
				\node (4) at ($(N52)+(0,-2.3)$) {$0$};
				\node (5) at ($(N62)+(0,-.3)$) {$+\infty$};
				\node (a) at ($(N11)+(0.65,0.35)$) {$a$};
				\node (b) at ($(N11)+(2.0,0.4)$) {$b$};
				\node (c) at ($(N11)+(3.38,0.35)$) {$c$};
				\node (d) at ($(N11)+(11.85,0.4)$) {$d$};
				\node (6) at ($(N12)+(0,-0.8)$) {};
				\node (7) at ($(N62)+(0,-0.8)$) {};
				\node (8) at ($(N12)+(0,-2.3)$) {};
				\node (9) at ($(N62)+(0,-2.3)$) {};
				%				\node (8) at ($(N42)+(0,-.5)$) {};
				%				\coordinate (9) at ($(N42)!.6!(N53)+ (-0.5,0)$);
				\coordinate (A) at ($(0)!.25!(1)$);
				\coordinate (B) at ($(0)!.8!(1)$);
				\coordinate (C) at ($(1)!.35!(2)$);
				\coordinate (D) at ($(4)!.75!(5)$);
				%				\coordinate (7) at ($(T22)!.6!(T23)$);
				\draw[->] (0)--(1);
				\draw[->] (1)--(2);
				\draw[->] (2)--(3);
				\draw[->] (1)--(2);
				\draw[->] (3)--(4);
				\draw[->] (4)--(5);
				%				\draw[->,red] (5)--(8);
				%				\draw[->,red] (8)--(9);
				\draw[blue,dashed](6)--(7)node[below]{$y=2$} (a)--(A) (b)--(B) (c)--(C) (d)--(D);
				\draw[blue,dashed](8)--(9)node[below left]{$y=0$};
			\end{tikzpicture}		
		\end{center}
		Dựa vào bảng biến thiên, ta thấy $f'(x)=0\Leftrightarrow x\in \{ 1\,;2\,;3\,;4 \}$;\\
		$f(x)=0\Leftrightarrow x=a<1$ hoặc $x=4$;\\
		$f(x)=2\Leftrightarrow \hoac{
			& x=b\,\,\left(a<b<1\right)\\ 
			& x=c\in\left(1\,;2\right)\\ 
			& x=3\\ 
			& x=d>4.
		}$ \\
		Ta lập được bảng xét dấu của $y'$ 
		\begin{center}
			\begin{tikzpicture}
				\tkzTabInit[lgt=1.2,espcl=1.5,nocadre]
				{$x$/1, $f(x)$ /.8} % first column
				{$-\infty$,$a$, $b$, $1$,$c$, $2$,$3$, $4$, $d$, $+\infty$} % first row
				\tkzTabLine { ,+,z,-,z,+,z,-,z,+,z,-,z,+,z,-,z,+, } % second row
				%				\tkzTabLine {,-,z,+,t,+,} % third row
				%				\tkzTabLine {,+,d,-,z,+,} % last row
			\end{tikzpicture}
		\end{center}
		Từ bảng xét dấu, ta thấy hàm số đồng biến trên các khoảng \\
		$\left(-\infty;a\right)$, $\left(b;1\right)$, $\left(c;2\right)$, $\left(3;4\right)$ và $(d;+\infty)$.
	}
\end{ex}

\begin{ex}%[2D1G1-2]%Câu 29 
	[THPT Bùi Thị Xuân – Huế-2022] 
	\immini{
		Cho hàm số $y=f(x)$ là hàm đa thức bậc bốn. Đồ thị hàm số $f'(x+2)$ được cho trong hình vẽ bên. Hàm số 
		$$g(x)=4 f\left(x^2\right)-x^6+5 x^4-4 x^2+1$$
		đồng biến trên khoảng nào dưới đây?
		\choice
		{$(-4 ;-3)$}
		{\True $(2 ;+\infty)$}
		{$(-\sqrt{2};\sqrt{2})$}
		{$(-2 ;-1)$}}{
		\begin{tikzpicture}[scale=0.6,font=\footnotesize, line join=round, line cap=round, >=stealth] %Đường cong bậc 3
			\draw[thick, ->] (-5.3,0)--(5,0);
			\draw[thick, ->] (0,-3.5)--(0,7);
			\draw (5.2,0) node[below] {$x$};
			\draw (0,7.1) node[left]{$y$};
			\draw (0,0) node[below left]{$0$};
			\draw[fill] (-2,0) circle (0.5pt)node[below left]{$ -2 $};
			\draw[fill] (2,0) circle (0.5pt)node[below]{$ 2$};
			\draw[fill] (0,3) circle (0.5pt)node[left]{$ 3 $};
			\draw[fill] (0,1) circle (0.5pt)node[right]{$ 1 $};
			\draw[fill] (0,-1) circle (0.5pt)node[right]{$ -1 $};
			\draw[dashed] (-2,0)--(-2,1) --(0,1); 
			\draw[dashed](2,0)--(2,3)--(0,3);
			\draw[line width=1.2pt,smooth,samples=100,domain=-2.8:4.5] plot(\x,{-0.271*(\x)^3+0.75*(\x)^2+1.583*\x-1});
		\end{tikzpicture}		
	}
	\loigiai{
		$\begin{aligned}
			& g(x)=4f\left(x^2\right)-x^6+5x^4-4x^2+1\Rightarrow g' (x)=8xf'\left(x^2\right)-6x^5+20x^3-8x.\\ 
			& g' (x)=0\Leftrightarrow 8xf'\left(x^2\right)-6x^5+20x^3-8x=0 \\
			& \Leftrightarrow 2x\left[4f'\left(x^2\right)-3x^4+10x^2-4\right]=0\\ 
			&\Leftrightarrow 		\hoac{ 			& 2x=0\\ 
				& 4f'(x^2)-3x^4+10x^2-4=0
			}
			\Leftrightarrow \hoac{	& x=0\\ 
				& f'\left(x^2\right)=\dfrac{3}{4}{x^4}-\dfrac{5}{2}{x^2}+1.}
		\end{aligned}$\\ 
		Xét
		$f'\left(x^2\right)=\dfrac{3}{4}x^4-\dfrac{5}{2}x^2+1$. Đặt $x^2=t+2$, ta có\\
		$ f' (t+2)=\dfrac{3}{4}{(t+2)^2}-\dfrac{5}{2}(t+2)+1=\dfrac{3}{4}\left(t^2+4t+4\right)-\dfrac{5}{2}(t+2)-1=\dfrac{3}{4}{t^2}+\dfrac{1}{2}t-1$\\
		Khi đó số nghiệm của phương trình chính là số giao điểm của đồ thị hàm số $y=f' (t+2)$ và\\
		$ y=\dfrac{3}{4}{t^2}+\dfrac{1}{2}t-1$\\
		Ta có đồ thị 
		\begin{center}
			\begin{tikzpicture}[scale=0.6,font=\footnotesize, line join=round, line cap=round, >=stealth] %Đường cong bậc 3
				\draw[thick, ->] (-5.3,0)--(5,0);
				\draw[thick, ->] (0,-3.5)--(0,7);
				\draw (5.2,0) node[below] {$x$};
				\draw (0,7.1) node[left]{$y$};
				\draw (0,0) node[below left]{$0$};
				\draw[fill] (-2,0) circle (0.5pt)node[below left]{$ -2 $};
				\draw[fill] (2,0) circle (0.5pt)node[below]{$ 2$};
				\draw[fill] (0,3) circle (0.5pt)node[left]{$ 3 $};
				\draw[fill] (0,1) circle (0.5pt)node[right]{$ 1 $};
				\draw[fill] (0,-1) circle (0.5pt)node[right]{$ -1 $};
				\draw[dashed] (-2,0)--(-2,1) --(0,1); 
				\draw[dashed](2,0)--(2,3)--(0,3);
				\draw[line width=1.2pt,smooth,samples=100,domain=-2.8:4.5] plot(\x,{-0.271*(\x)^3+0.75*(\x)^2+1.583*\x-1});		
				\draw[line width=1.2pt,smooth,samples=100,domain=-3.3:2.8] plot(\x,{0.75*(\x)^2+0.5*\x-1});
			\end{tikzpicture}
		\end{center}
		Dựa vào đồ thị ta có $f' (t+2)=\dfrac{3}{4}t^2+\dfrac{1}{2}t-1\Leftrightarrow \hoac{& t=-2\\ & t=0\\ & t=2} \Leftrightarrow\hoac{& x+2=-2\\ & x+2=0\\ & x+2=2} \Leftrightarrow \hoac{& x=-4\\ & x=-2\\ & x=0.}$\\
		Ta có bảng xét dấu $g' (x)$ như sau
		\begin{center}
			\begin{tikzpicture}
				\tkzTabInit[lgt=1.2,espcl=2,nocadre]
				{$x$/0.7, $f(x)$ /.7}
				{$-\infty$, $-4$,$-2$, $0$, $+\infty$} % first row
				\tkzTabLine { ,-,z,+,z,-,z,+, }
			\end{tikzpicture}
		\end{center}
		Vậy hàm số $g(x)=4 f\left(x^2\right)-x^6+5 x^4-4 x^2+1$ đồng biến trên khoảng $(2 ;+\infty)$.}
\end{ex}

\begin{ex}%[2D1G1-2]%Câu 30
	[Chuyên Bắc Ninh 2022] 
	\immini{
		Cho hàm số $ y=f(x)$ liên tục trên $\mathbb{R}$ có đồ thị hàm số $ y=f'(x)$ có đồ thị như hình vẽ bên.
		Hàm số $g(x)=2f\left(\left| x-1\right|\right)-x^2+2x+2020$ đồng biến trên khoảng nào
		\choice
		{$\left(-2;0\right)$}
		{$\left(-3;1\right)$}
		{$\left(1\,;3\right)$}
		{\True $\left(0\,;\,1\right)$}}{
		\begin{tikzpicture}[scale=0.6,font=\footnotesize, line join=round, line cap=round, >=stealth] %Đường cong bậc 3
			\draw[thick, ->] (-3.3,0)--(5,0);
			\draw[thick, ->] (0,-3.0)--(0,5.5);
			\draw (5.2,0) node[below] {$x$};
			\draw (0,5.8) node[left]{$y$};
			\draw (0,0) node[below left]{$0$};
			\draw[fill] (-1,0) circle (0.5pt)node[above]{$ -1 $};
			\draw[fill] (1,0) circle (0.5pt)node[below]{$ 1$};
			\draw[fill] (0,1) circle (0.5pt)node[left]{$ 1 $};
			\draw[fill] (0,-1) circle (0.5pt)node[right]{$ -1 $};
			\draw[fill] (0,3) circle (0.5pt)node[left]{$ 3 $};
			\draw[fill] (3,0) circle (0.5pt)node[below]{$ 3 $};
			\draw[dashed] (-1,0)--(-1,-1) --(0,-1); 
			\draw[dashed](1,0)--(1,1)--(0,1);
			\draw[dashed](3,0)--(3,3)--(0,3);
			\draw[line width=1.2pt,smooth,samples=100,domain=-2.2:4.3] plot(\x,{-0.333*(\x)^3+1*(\x)^2+1.333*\x-1});		
			%\draw[line width=1.2pt,smooth,samples=100,domain=-3.3:2.8] plot(\x,{0.75*(\x)^2+0.5*\x-1});
		\end{tikzpicture}	
	}
	\loigiai{
		Ta có $g(x)=2f\left(\left| x-1\right|\right)-x^2+2x+2020\Leftrightarrow g(x)=2f\left(\left| x-1\right|\right)-\left(x-1\right)^2+2021$.\\
		Xét hàm số $ k\left(x-1\right)=2f\left(x-1\right)-\left(x-1\right)^2+2021$.\\
		Đặt $ t=x-1$\\
		Xét hàm số $ h(t)=2f(t)-t^2+2021$ $\Rightarrow{h}'(t)=2f'(t)-2t$.\\
		Kẻ đường $ y=x$ như hình vẽ.
		\begin{center}
			\begin{tikzpicture}[scale=0.6,font=\footnotesize, line join=round, line cap=round, >=stealth] %Đường cong bậc 3
				\draw[thick, ->] (-3.3,0)--(5,0);
				\draw[thick, ->] (0,-3.0)--(0,5.5);
				\draw (5.2,0) node[below] {$x$};
				\draw (0,5.8) node[left]{$y$};
				%	\draw (0,0) node[below left]{$0$};
				\draw[fill] (-1,0) circle (0.5pt)node[above]{$ -1 $};
				\draw[fill] (1,0) circle (0.5pt)node[below]{$ 1$};
				\draw[fill] (0,1) circle (0.5pt)node[left]{$ 1 $};
				\draw[fill] (0,-1) circle (0.5pt)node[right]{$ -1 $};
				\draw[fill] (0,3) circle (0.5pt)node[left]{$ 3 $};
				\draw[fill] (3,0) circle (0.5pt)node[below]{$ 3 $};
				\draw[dashed] (-1,0)--(-1,-1) --(0,-1); 
				\draw[dashed](1,0)--(1,1)--(0,1);
				\draw[dashed](3,0)--(3,3)--(0,3);
				\draw[line width=1.2pt,smooth,samples=100,domain=-2.2:4.3] plot(\x,{-0.333*(\x)^3+1*(\x)^2+1.333*\x-1});		
				%\draw[line width=1.2pt,smooth,samples=100,domain=-3.3:2.8] plot(\x,{0.75*(\x)^2+0.5*\x-1});
				\draw[line width=1.2pt,smooth,samples=100](-2,-2)--(4,4);
			\end{tikzpicture}
		\end{center}
		Khi đó $h'(t)>0\Leftrightarrow{f}'(t)-t>0\Leftrightarrow{f}'(t)>t$$\Leftrightarrow \hoac{
			& t<-1\\ 
			& 1<t<3.
		}$\\
		Do đó $k'\left(x-1\right)>0\Leftrightarrow \hoac{
			& x-1<-1\\ 
			& 1<x-1<3} \Leftrightarrow \hoac{
			& x<0\\ 
			& 2<x<4.}$\\
		Ta có bảng biến thiên của hàm số $ k\left(x-1\right)=2f\left(x-1\right)-\left(x-1\right)^2+2021$.
		\begin{center}
			\begin{tikzpicture}
				\tkzTabInit[lgt=1.8,espcl=2.3]
				{$x$ /1.2, $k'(x-1)$ /1.2,$k(x-1)$ /2}
				{$-\infty$ , $0$,$2$,$4$, $+\infty$}
				\tkzTabLine{,+,0,-,0,+,0,-,}
				\tkzTabVar{-/$ $ ,+/$ $, -/$ $,+/$ $,-/$ $}
			\end{tikzpicture}
		\end{center}
		Khi đó, ta có bảng biến thiên của $g(x)=2f\left(\left| x-1\right|\right)-\left(x-1\right)^2+2021$ bằng cách lấy đối xứng qua đường thẳng $ x=1$ như sau\\
		\begin{center}
			\begin{tikzpicture}
				\tkzTabInit[lgt=1.2,espcl=2.5,nocadre]
				{$x$ /0.7, $g'(x)$ /0.7,$g(x)$ /2.5}
				{$-\infty$ ,$-2$, $0$,$1$,$2$,$4$, $+\infty$}
				\tkzTabLine{,+,0,-,0,+,0,-,0,+,0,-,}
				\tkzTabVar{-/$ $ ,+/$ $, -/$ $,+/$ $,-/$ $,+/ $ $,-/$ $}
			\end{tikzpicture}
		\end{center}
		Vậy hàm số đồng biến trên $\left(0;1\right)$.}
\end{ex}

\begin{ex}%[2D1G1-2]%Câu 31
	[Chuyên Thái Bình 2022] 
	\immini{
		Cho hàm số $f(x)=a{x^4}+b{x^3}+c{x^2}+dx+a$ có đồ thị hàm số $y=f'(x)$ như hình vẽ bên. Hàm số $y=g(x)=f\left(1-2x\right)f\left(2-x\right)$ đồng biến trên khoảng nào dưới đây?
		\choice
		{$\left(\dfrac{1}{2};\dfrac{3}{2}\right)$}
		{$\left(-\infty ;0\right)$}
		{$\left(0;2\right)$}
		{\True $\left(3;+\infty\right)$}}{
		\begin{tikzpicture}[scale=0.9,font=\footnotesize, line join=round, line cap=round, >=stealth] %Đường cong bậc 3
			\draw[thick, ->] (-2.5,0)--(2.5,0);
			\draw[thick, ->] (0,-2.8)--(0,2.8);
			\draw (2.6,0) node[below] {$x$};
			\draw (0,2.9) node[left]{$y$};
			\draw (0,0) node[below left]{$0$};
			\draw[fill] (-1,0) circle (0.5pt)node[below left]{$ -1 $};
			\draw[fill] (1,0) circle (0.5pt)node[below right]{$ 1$};
			%			\draw[dashed] (-1,0)--(-1,-1) --(0,-1); 
			%			\draw[dashed](1,0)--(1,1)--(0,1);
			%			\draw[dashed](3,0)--(3,3)--(0,3);
			\draw[line width=1.2pt,smooth,samples=100,domain=-1.3:1.3] plot(\x,{3*(\x)^3-3*\x});		
			%\draw[line width=1.2pt,smooth,samples=100,domain=-3.3:2.8] plot(\x,{0.75*(\x)^2+0.5*\x-1});
		\end{tikzpicture}	
	}
	\loigiai{
		Ta có $f'(x)=4a{x^3}+3b{x^2}+2cx+d$, theo đồ thị thì đa thức $f'(x)$ có ba nghiệm phân biệt là $-1,0,1$ nên $f'(x)=4ax\left(x+1\right)\left(x-1\right)=4a{x^3}-4ax\Rightarrow f(x)=a{x^4}-2a{x^2}+a=a{\left(x^2-1\right)^2}$.\\
		Dựa vào đồ thị hàm số $y=f'(x)$ ta có $a>0$ nên $f(x)>0,\forall x\in\mathbb{R}\setminus\left\{\pm 1\right\}$.\\
		$g'(x)=\left[f\left(1-2x\right)\right]'f\left(2-x\right)+f\left(1-2x\right)\left[f\left(2-x\right)\right]'=-2f'\left(1-2x\right)f\left(2-x\right)-f\left(1-2x\right)f'\left(2-x\right)$. Xét $x\in\left(\dfrac{1}{2};\dfrac{3}{2}\right)\Rightarrow
		\heva{		
			& 1-2x\in\left(-2;0\right)\\ 
			& 2-x\in\left(\dfrac{1}{2};\dfrac{3}{2}\right)}$, dấu của $f'(x)$ không cố định trên $\left(\dfrac{1}{2};\dfrac{3}{2}\right)$ nên ta không kết luận được tính đơn điệu của hàm số $g(x)$ trên $\left(\dfrac{1}{2};\dfrac{3}{2}\right)$.\\
		Xét $x\in\left(-\infty ;0\right)\Rightarrow
		\heva{
			& 1-2x\in\left(1;+\infty\right)\\ 
			& 2-x\in\left(2;+\infty\right)} 
		\Rightarrow \heva{
			& f'\left(1-2x\right)>0\\ 
			& f'\left(2-x\right)>0} \Rightarrow g'(x)<0$.\\
		Do đó, hàm số $g(x)$ nghịch biến trên $\left(-\infty ;0\right)$.\\
		$x\in\left(0;2\right)\Rightarrow \heva{
			& 1-2x\in\left(-3;1\right)\\ 
			& 2-x\in\left(0;2\right)}$, dấu của $f'(x)$ không cố định trên $\left(-3;1\right)$ và $\left(0;2\right)$ nên ta không kết luận được tính đơn điệu của hàm số $g(x)$ trên $\left(\dfrac{1}{2};\dfrac{3}{2}\right)$.\\
		Xét $x\in\left(3;+\infty\right)\Rightarrow \heva{
			& 1-2x\in\left(-\infty ;-5\right)\\ 
			& 2-x\in\left(-\infty ;-1\right)} \Rightarrow \heva{
			& f'\left(1-2x\right)<0\\ 
			& f'\left(2-x\right)<0} \Rightarrow g'(x)>0$. \\
		Do đó, hàm số $g(x)$ đồng biến trên $\left(3;+\infty\right)$.}
\end{ex}

\begin{dang}{Bài toán hàm ẩn, hàm hợp liên quan đến tham số và một số bài toán khác}
\end{dang}

\begin{ex}%[2D1G1-3]%Câu 1
	[Chuyên Lê Hồng Phong Nam Định 2019]
	\immini{
		Cho hàm số $ y=f(x)$ có đạo hàm liên tục trên $\mathbb{R}$. Biết hàm số $ y=f'(x)$ có đồ thị như hình vẽ. Gọi $ S$ là tập hợp các giá trị nguyên $ m\in\left[-5\,;\,\text{5}\right]$ để hàm số $ g(x)=f\left(x+m\right)$ nghịch biến trên khoảng $\left(1\,;\,2\right)$. Hỏi $S$ có bao nhiêu phần tử?
		\choice
		{$ 4$}
		{$ 3$}
		{$ 6$}
		{\True $ 5$}}{
		\begin{tikzpicture}[scale=0.9,font=\footnotesize, line join=round, line cap=round, >=stealth] %Đường cong bậc 3
			\draw[thick, ->] (-2.5,0)--(4,0);
			\draw[thick, ->] (0,-2.8)--(0,2.8);
			\draw (4.3,0) node[below] {$x$};
			\draw (0,2.9) node[left]{$y$};
			\draw (0,0) node[below left]{$0$};
			\draw[fill] (-1,0) circle (0.5pt)node[below left]{$ -1 $};
			\draw[fill] (1,0) circle (0.5pt)node[below]{$ 1$};
			\draw[fill] (3,0) circle (0.5pt)node[below right]{$ 3$};
			%			\draw[dashed] (-1,0)--(-1,-1) --(0,-1); 
			%			\draw[dashed](1,0)--(1,1)--(0,1);
			%			\draw[dashed](3,0)--(3,3)--(0,3);
			\draw[line width=1.2pt,smooth,samples=100,domain=-1.65:3.5] plot(\x,{0.33*(\x)^3-(\x)^2-0.333*(\x)+1});		
			%\draw[line width=1.2pt,smooth,samples=100,domain=-3.3:2.8] plot(\x,{0.75*(\x)^2+0.5*\x-1});
		\end{tikzpicture}	
	}
	\loigiai{
		Ta có $g'(x)=f'\left(x+m\right)$. Vì $ y=f'(x)$ liên tục trên $\mathbb{R}$ nên $g'(x)=f'\left(x+m\right)$ cũng liên tục trên $\mathbb{R}$. Căn cứ vào đồ thị hàm số $ y=f'(x)$ ta thấy\\
		$g'(x)<0\Leftrightarrow{f}'\left(x+m\right)<0$ $\Leftrightarrow\hoac{
			& x+m<-1\\ 
			& 1<x+m<3} \Leftrightarrow \hoac{
			& x<-1-m\\ 
			& 1-m<x<3-m.}$\\
		Hàm số $ g(x)=f\left(x+m\right)$ nghịch biến trên khoảng $\left(1\,;\,2\right)$
		$\Leftrightarrow \hoac{
			& 2\le-1-m\\ 
			&\hoac{
				& 3-m\ge 2\\ 
				& 1-m\le 1}} \Leftrightarrow \hoac{
			& m\le-3\\ 
			& 0\le m\le 1.}$\\
		Mà $ m$ là số nguyên thuộc đoạn $\left[-5\,;\,5\right]$ nên ta có $ S=\left\{-5;-4;-3;0;1\right\}$.\\
		Vậy $ S$ có $5$ phần tử.}
\end{ex}

\begin{ex}%[2D1G1-3]%Câu 2
	[Chuyên Nguyễn Bỉnh Khiêm-Quảng Nam-2020] Cho hàm số $ y=f(x)$ có đạo hàm trên $\mathbb{R}$ và bảng xét dấu đạo hàm như hình vẽ sau
	\begin{center}
		\begin{tikzpicture}
			\tkzTabInit[lgt=1.2,espcl=2.5,nocadre]
			{$x$/0.7, $f'(x)$ /2.5} % first column
			{$-\infty$, $-10$,$-2$, $3$,$8$, $+\infty$} % first row
			\tkzTabLine { ,+,z,-,z,+,z,-,z,+, } % second row
			%				\tkzTabLine {,-,z,+,t,+,} % third row
			%				\tkzTabLine {,+,d,-,z,+,} % last row
		\end{tikzpicture}
	\end{center}
	Có bao nhiêu số nguyên $ m$ để hàm số $ y=f\left(x^3+4x+m\right)$ nghịch biến trên khoảng $\left(-1;1\right)$?
	\choice
	{$ 3$}
	{$ 0$}
	{\True $ 1$}
	{$ 2$}
	\loigiai
	{
		Đặt $ t=x^3+4x+m\Rightarrow{t}'=3x^2+4$ nên $ t$ đồng biến trên $\left(-1;1\right)$ và $ t\in\left(m-5;m+5\right)$.\\
		Yêu cầu bài toán trở thành tìm $ m$ để hàm số $ f(t)$ nghịch biến trên khoảng $\left(m-5;m+5\right)$.\\
		Dựa vào bảng biến thiên ta được $\heva{
			& m-5\ge-2\\ 
			& m+5\le 8} \Leftrightarrow \heva{
			& m\ge 3\\ 
			& m\le 3} \Leftrightarrow m=3$.}
\end{ex}

\begin{ex}%[2D1G1-3]%Câu 3
	[Chuyên ĐH Vinh-Nghệ An-2020]
	\immini{
		Cho hàm số $ f(x)$ có đạo hàm trên $\mathbb{R}$và $ f(1)=1$. Đồ thị hàm số $ y=f'(x)$ như hình bên. Có bao nhiêu số nguyên dương $ a$ để hàm số $ y=\left| 4f\left(\sin x\right)+\cos 2x-a\right|$ nghịch biến trên $\left(0;\dfrac{\pi}{2}\right)$?
		\choice
		{$ 2$}
		{\True $ 3$}
		{Vô số}
		{$ 5$}}{
		\begin{tikzpicture}[scale=0.9,font=\footnotesize, line join=round, line cap=round, >=stealth] %Đường cong bậc 3
			\draw[thick, ->] (-2.5,0)--(3,0);
			\draw[thick, ->] (0,-2.8)--(0,2.8);
			\draw (3.1,0) node[below] {$x$};
			\draw (0,2.9) node[left]{$y$};
			\draw (0,0) node[below left]{$0$};
			\draw[fill] (-1,0) circle (0.5pt)node[below]{$ -1 $};
			\draw[fill] (1,0) circle (0.5pt)node[above]{$ 1$};
			%	\draw[fill] (3,0) circle (0.5pt)node[below right]{$ 3$};
			\draw[dashed] (-1,0)--(-1,1); 
			\draw[dashed](1,0)--(1,-1);
			%			\draw[dashed](3,0)--(3,3)--(0,3);
			\draw[line width=1.2pt,smooth,samples=100,domain=-2:2] plot(\x,{.8*(\x)^3+0*(\x)^2-1.8*(\x)});		
			%\draw[line width=1.2pt,smooth,samples=100,domain=-3.3:2.8] plot(\x,{0.75*(\x)^2+0.5*\x-1});
			\draw (2.0,2.8) node[left]{$y=f'(x)$};
		\end{tikzpicture}	
	}
	\loigiai
	{		Đặt $g(x)=\left| 4f\left(\sin x\right)+\cos 2x-a\right|\Rightarrow g(x)=\sqrt{\left[4f\left(\sin x\right)+\cos 2x-a\right]^2}$ .\\
		$\Rightarrow{g}'(x)=\dfrac{\left[4\cos x\cdot f'\left(\sin x\right)-2\sin 2x\right]\left[4f\left(\sin x\right)+\cos 2x-a\right]}{\sqrt{\left[4f\left(\sin x\right)+\cos 2x-a\right]^2}}$.\\
		Ta có $ 4\cos x\cdot f'\left(\sin x\right)-2\sin 2x=4\cos x\left[f'\left(\sin x\right)-\sin x\right]$.\\
		Với $ x\in\left(0;\dfrac{\pi}{2}\right)$ thì $\cos x>0,\sin x\in\left(0;1\right)\Rightarrow{f}'\left(\sin x\right)-\sin x<0$.\\
		Hàm số $ g(x)$ nghịch biến trên $\left(0;\dfrac{\pi}{2}\right)$ khi $ 4f\left(\sin x\right)+\cos 2x-a\ge 0,\forall x\in\left(0;\dfrac{\pi}{2}\right)$\\
		$\Leftrightarrow 4f\left(\sin x\right)+1-2\sin^2x\ge a,\forall x\in\left(0;\dfrac{\pi}{2}\right)$.\\
		Đặt $ t=\sin x$ được $ 4f(t)+1-2t^2\ge a,\forall t\in\left(0;1\right)$ (*).\\
		Xét $ h(t)=4f(t)+1-2t^2\Rightarrow{h}'(t)=4f'(t)-4t=4\left[f'(t)-1\right]$.\\
		Với $ t\in\left(0;1\right)$ thì $h'(t)<0\Rightarrow h(t)$ nghịch biến trên $\left(0;1\right)$.\\
		Do đó (*) $\Leftrightarrow a\le h(1)=4f(1)+1-2.1^2=3$.\\
		Vậy có $3$ giá trị nguyên dương của a thỏa mãn.}
\end{ex}


\begin{ex}%[2D1G1-3]%Câu 4
	[Chuyên Quang Trung-2020]
	\immini{
		Cho hàm số $ y=f(x)$ có đạo hàm liên tục trên $\mathbb{R}$ và có đồ thị $ y=f'(x)$ như hình vẽ. Đặt $ g(x)=f\left(x-m\right)-\dfrac{1}{2}{\left(x-m-1\right)^2}+2019$, với $ m$ là tham số thực. Gọi $ S$ là tập hợp các giá trị nguyên dương của $ m$ để hàm số $ y=g(x)$ đồng biến trên khoảng $\left(5;6\right)$. Tổng tất cả các phần tử trong $ S$ bằng
		\choice
		{$ 4$}
		{$ 11$}
		{\True $ 14$}
		{$ 20$}}{
		\begin{tikzpicture}[scale=0.9,font=\footnotesize, line join=round, line cap=round, >=stealth] %Đường cong bậc 3
			\draw[style=help lines,step=1] (-2.5,-3) grid (3,3.5);
			\draw[thick, ->] (-2.5,0)--(3.5,0);
			\draw[thick, ->] (0,-2.8)--(0,2.8);
			\draw (3.6,0) node[below] {$x$};
			\draw (0,3) node[above left]{$y$};
			\draw (0,0) node[below left]{$0$};
			%\draw[fill] (-1,0) circle (0.5pt)node[below]{$ -1 $};
			\draw[fill] (1,0) circle (0.5pt)node[below left]{$ 1$};
			%	\draw[fill] (3,0) circle (0.5pt)node[below right]{$ 3$};
			\draw[dashed] (-1,0)--(-1,-2) --(2,-2)--(2,0); 
			\draw[dashed](3,0)--(3,2) --(0,2);
			\draw (-1,-2) circle (2pt);
			\draw (3,2) circle (2pt);
			%			\draw[dashed](3,0)--(3,3)--(0,3);
			\draw[line width=1.2pt,smooth,samples=100,domain=-1.1:3.1] plot(\x,{1*(\x)^3-3*(\x)^2-0*(\x)+2});		
			%\draw[line width=1.2pt,smooth,samples=100,domain=-3.3:2.8] plot(\x,{0.75*(\x)^2+0.5*\x-1});
			%\draw (2.0,2.8) node[left]{$y=f'(x)$};
		\end{tikzpicture}	
	}
	\loigiai
	{
		Xét hàm số $ g(x)=f\left(x-m\right)-\dfrac{1}{2}{\left(x-m-1\right)^2}+2019$.\\
		$g'(x)=f'\left(x-m\right)-\left(x-m-1\right)$.\\
		Xét phương trình $g'(x)=0. \quad \quad (1)$\\
		Đặt $ x-m=t$, phương trình $(1)$ trở thành $f'(t)-\left(t-1\right)=0\Leftrightarrow{f}'(t)=t-1. \quad (2)$\\
		Nghiệm của phương trình $(2)$ là hoành độ giao điểm của hai đồ thị hàm số $ y=f'(t)$ và $ y=t-1$.\\
		Ta có đồ thị các hàm số $ y=f'(t)$ và $ y=t-1$ như sau
		\begin{center}
			\begin{tikzpicture}[scale=0.9,font=\footnotesize, line join=round, line cap=round, >=stealth] %Đường cong bậc 3
				\draw[style=help lines,step=1] (-2.5,-3) grid (3,3.5);
				\draw[thick, ->] (-2.5,0)--(3.5,0);
				\draw[thick, ->] (0,-2.8)--(0,2.8);
				\draw (3.6,0) node[below] {$x$};
				\draw (0,3) node[above left]{$y$};
				\draw (0,0) node[below left]{$0$};
				%\draw[fill] (-1,0) circle (0.5pt)node[below]{$ -1 $};
				\draw[fill] (1,0) circle (0.5pt)node[below left]{$ 1$};
				%	\draw[fill] (3,0) circle (0.5pt)node[below right]{$ 3$};
				\draw[dashed] (-1,0)--(-1,-2) --(2,-2)--(2,0); 
				\draw[dashed](3,0)--(3,2) --(0,2);
				\draw (-1,-2) circle (2pt);
				\draw (3,2) circle (2pt);
				%			\draw[dashed](3,0)--(3,3)--(0,3);
				\draw[line width=1.2pt,smooth,samples=100,domain=-1.1:3.1] plot(\x,{1*(\x)^3-3*(\x)^2-0*(\x)+2});		
				%\draw[line width=1.2pt,smooth,samples=100,domain=-3.3:2.8] plot(\x,{0.75*(\x)^2+0.5*\x-1});
				%\draw (2.0,2.8) node[left]{$y=f'(x)$};
				\draw (-2,-3)--(4,3);
			\end{tikzpicture}
		\end{center}
		Căn cứ đồ thị các hàm số ta có phương trình $(2)$ có nghiệm là $\hoac{
			& t=-1\\ 
			& t=1\\ 
			& t=3} \Rightarrow \hoac{
			& x=m-1\\ 
			& x=m+1\\ 
			& x=m+3.}$\\
		Ta có bảng biến thiên của $ y=g(x)$
		\begin{center}
			\begin{tikzpicture}
				\tkzTabInit[lgt=1,espcl=2.5,nocadre]
				{$x$ /0.8, $y'$ /0.8,$y$ /2.5}
				{$-\infty$ , $m-1$,$m+1$,$m+3$, $+\infty$}
				\tkzTabLine{,+,0,-,0,+,0,-,}
				\tkzTabVar{-/$ +\infty$ ,+/$ $, -/$ $,+/$ $,-/$+\infty $}
			\end{tikzpicture}
		\end{center}
		Để hàm số $ y=g(x)$ đồng biến trên khoảng $\left(5;6\right)$ cần $\hoac{
			&\heva{
				& m-1\le 5\\ 
				& m+1\ge 6}\\ 
			& m+3\le 5}\Leftrightarrow\hoac{
			& 5\le m\le 6\\ 
			& m\le 2.}$\\
		Vì $ m\in\mathbb{N}^*\Rightarrow m$ nhận các giá trị $ 1;\,2;\,5;\,6\Rightarrow S=14$.}
\end{ex}

\begin{ex}%[2D1G1-3]%Câu 5
	[Sở Hà Nội-Lần 2-2020] 
	\immini{
		Cho hàm số $y=a{x^4}+b{x^3}+c{x^2}+dx+e,\,\,a\ne 0$. Hàm số $y=f'(x)$ có đồ thị như hình vẽ bên. 
		Gọi S là tập hợp tất cả các giá trị nguyên thuộc khoảng $\left(-6;6\right)$ của tham số $m$ để hàm số $g(x)=f\left(3-2x+m\right)+x^2-\left(m+3\right)x+2m^2$ nghịch biến trên $\left(0;1\right)$. Khi đó, tổng giá trị các phần tử của S là
		\choice
		{$12$}
		{\True $9$}
		{$6$}
		{$15$}}{
		\begin{tikzpicture}[scale=0.7,font=\footnotesize, line join=round, line cap=round, >=stealth] %Đường cong bậc 3
			%	\draw[style=help lines,step=1] (-2.5,-3) grid (3,3.5);
			\draw[thick, ->] (-4.5,0)--(6.5,0);
			\draw[thick, ->] (0,-2.8)--(0,2.8);
			\draw (6.6,0) node[below] {$x$};
			\draw (0,3) node[above left]{$y$};
			\draw (0,0) node[below left]{$0$};
			\draw[fill] (-2,0) circle (0.5pt)node[below]{$ -2 $};
			\draw[fill] (4,0) circle (0.5pt)node[above]{$ 4$};
			\draw[fill] (0,1) circle (0.5pt)node[right]{$ 1 $};
			\draw[fill] (0,-2) circle (0.5pt)node[left]{$ -2$};
			%	\draw[fill] (3,0) circle (0.5pt)node[below right]{$ 3$};
			\draw[dashed] (-2,0)--(-2,1) --(0,1); 
			\draw[dashed](4,0)--(4,-2) --(0,-2);
			%			\draw[dashed](3,0)--(3,3)--(0,3);
			\draw[line width=1.2pt,smooth,samples=100,domain=-3.8:5.5] plot(\x,{0.0714*(\x)^3-0.1423*(\x)^2-1.0714*(\x)});		
			%\draw[line width=1.2pt,smooth,samples=100,domain=-3.3:2.8] plot(\x,{0.75*(\x)^2+0.5*\x-1});
			%\draw (2.0,2.8) node[left]{$y=f'(x)$};
		\end{tikzpicture}	
	}
	\loigiai
	{
		Xét $g'(x)=-2f'\left(3-2x+m\right)+2x-\left(m+3\right)$.\\
		Xét phương trình $g'(x)=0$, đặt $t=3-2x+m$ thì phương trình trở thành\\ $-2\cdot \left[f'(t)-\dfrac{-t}{2}\right]=0\Leftrightarrow\hoac{
			& t=-2\\ 
			& t=4\\ 
			& t=0.}$ \\
		Từ đó, $g'(x)=0\Leftrightarrow{x_1}=\dfrac{5+m}{2},\,x_2=\dfrac{m+3}{2},x_3=\dfrac{-1+m}{2}$.\\
		Lập bảng xét dấu, đồng thời lưu ý nếu $x>x_1$ thì $t<t_1$ nên $f(x)>0$. Và các dấu đan xen nhau do các nghiệm đều làm đổi dấu đạo hàm nên suy ra $g'(x)\le 0\Leftrightarrow x\in\left[x_2;{x_1}\right]\cup\left(-\infty ;{x_3}\right]$.\\
		Vì hàm số nghịch biến trên $\left(0;1\right)$ nên \\
		$g'(x)\le 0,\,\forall x\in\left(0;1\right)$ từ đó suy ra $\hoac{
			&\dfrac{3+m}{2}\le 0<1\le\dfrac{5+m}{2}\\ 
			& 1\le\dfrac{-1+m}{2}.}$ \\
		và giải ra các giá trị nguyên thuộc $\left(-6;6\right)$ của $m$ là $-3$; $3$; $4$; $5$. }
\end{ex}

\begin{ex}%[2D1G1-3]%Câu 6
	[Chuyên Quang Trung-Bình Phước-Lần 2-2020]
	\immini{
		Cho hàm số $ y=f(x)$ có đạo hàm liên tục trên $\mathbb{R}$ và có đồ thị $ y=f'(x)$ như hình vẽ bên. Đặt $ g(x)=f\left(x-m\right)-\dfrac{1}{2}{\left(x-m-1\right)^2}+2019$, với $ m$ là tham số thực. Gọi $ S$ là tập hợp các giá trị nguyên dương của $ m$ để hàm số $ y=g(x)$ đồng biến trên khoảng $\left(5;6\right)$. Tổng tất cả các phần tử trong $ S$ bằng
		\choice
		{$ 4$}
		{$ 11$}
		{\True $ 14$}
		{$ 20$}}{
		\begin{tikzpicture}[scale=0.9,font=\footnotesize, line join=round, line cap=round, >=stealth] %Đường cong bậc 3
			\draw[thick, ->] (-2.5,0)--(3.7,0);
			\draw[thick, ->] (0,-2.8)--(0,2.8);
			\draw (3.9,0) node[below] {$x$};
			\draw (0,2.9) node[left]{$y$};
			\draw (0,0) node[below left]{$0$};
			\draw[fill] (-1,0) circle (0.5pt)node[above]{$ -1 $};
			\draw[fill] (1,0) circle (0.5pt)node[below]{$ 1$};
			\draw[fill] (3,0) circle (0.5pt)node[below]{$ 3$};
			\draw[fill] (2,0) circle (0.5pt)node[above]{$ 2$};
			\draw[fill] (0,2) circle (0.5pt)node[above left]{$ 2$};
			\draw[fill] (0,-2) circle (0.5pt)node[below left]{$ -2$};
			\draw[dashed] (-1,0)--(-1,-2)--(2,-2)--(2,0); 
			\draw[dashed](3,0)--(3,2)--(0,2);
			%			\draw[dashed](3,0)--(3,3)--(0,3);
			\draw[line width=1.2pt,smooth,samples=100,domain=-1.1:3.1] plot(\x,{1*(\x)^3-3*(\x)^2-0*(\x)+2});		
			%\draw[line width=1.2pt,smooth,samples=100,domain=-3.3:2.8] plot(\x,{0.75*(\x)^2+0.5*\x-1});
			%	\draw (2.0,2.8) node[left]{$y=f'(x)$};
	\end{tikzpicture}	}
	\loigiai
	{
		Ta có $g'(x)=f'\left(x-m\right)-\left(x-m-1\right)$.\\
		Cho $g'(x)=0\Leftrightarrow{f}'\left(x-m\right)=x-m-1$.\\
		Đặt $ x-m=t\Rightarrow f'(t)=t-1$\\
		Khi đó nghiệm của phương trình là hoành độ giao điểm của đồ thị hàm số $ y=f'(t)$ và và đường thẳng $ y=t-1$.
		\begin{center}
			\begin{tikzpicture}[scale=0.9,font=\footnotesize, line join=round, line cap=round, >=stealth] %Đường cong bậc 3
				\draw[thick, ->] (-2.5,0)--(3.7,0);
				\draw[thick, ->] (0,-2.8)--(0,2.8);
				\draw (3.9,0) node[below] {$x$};
				\draw (0,2.9) node[left]{$y$};
				\draw (0,0) node[below left]{$0$};
				\draw[fill] (-1,0) circle (0.5pt)node[above]{$ -1 $};
				\draw[fill] (1,0) circle (0.5pt)node[below]{$ 1$};
				\draw[fill] (3,0) circle (0.5pt)node[below]{$ 3$};
				\draw[fill] (2,0) circle (0.5pt)node[above]{$ 2$};
				\draw[fill] (0,2) circle (0.5pt)node[above left]{$ 2$};
				\draw[fill] (0,-2) circle (0.5pt)node[below left]{$ -2$};
				\draw[dashed] (-1,0)--(-1,-2)--(2,-2)--(2,0); 
				\draw[dashed](3,0)--(3,2)--(0,2);
				%			\draw[dashed](3,0)--(3,3)--(0,3);
				\draw[line width=1.2pt,smooth,samples=100,domain=-1.1:3.1] plot(\x,{1*(\x)^3-3*(\x)^2-0*(\x)+2});		
				%\draw[line width=1.2pt,smooth,samples=100,domain=-3.3:2.8] plot(\x,{0.75*(\x)^2+0.5*\x-1});
				%	\draw (2.0,2.8) node[left]{$y=f'(x)$};
				\coordinate (a) at ($(-1,-2)!1.2!(3,2)$);
				\coordinate (b) at ($(-1,-2)!-.2!(3,2)$);
				\draw[line width=1.2pt,smooth] (a)--(b);
			\end{tikzpicture}
		\end{center}
		Dựa vào đồ thị hàm số ta có được $f'(t)=t-1\Leftrightarrow\hoac{
			& t=-1\\ 
			& t=1\\ 
			& t=3.} $ \\
		Bảng xét dấu của $g'(t)$
		\begin{center}
			\begin{tikzpicture}
				\tkzTabInit[lgt=1.2,espcl=2.5,nocadre]
				{$t$/1, $g'(x)$ /.8} % first column
				{$-\infty$, $-1$,$1$, $3$, $+\infty$} % first row
				\tkzTabLine { ,-,0,+,0,-,0,+, } % second row
				%				\tkzTabLine {,-,z,+,t,+,} % third row
				%				\tkzTabLine {,+,d,-,z,+,} % last row
			\end{tikzpicture}
		\end{center}
		Từ bảng xét dấu ta thấy hàm số $ g(t)$ đồng biến trên khoảng $\left(-1;1\right)$ và $\left(3;+\infty\right)$.\\
		Hay $\hoac{
			&-1<t<1\\ 
			& t>3}\Leftrightarrow\hoac{
			&-1<x-m<1\\ 
			& x-m>3} \Leftrightarrow\hoac{
			& m-1<x<m+1\\ 
			& x>m+3.}$\\
		Để hàm số $ g(x)$ đồng biến trên khoảng $\left(5;6\right)$ thì $\hoac{
			& m-1\le 5<6\le m+1\\ 
			& m+3\le 5<6} \Leftrightarrow\hoac{
			& 5\le m\le 6\\ 
			& m\le 2.}$\\
		Vì $ m$ là các số nguyên dương nên $ S=\left\{ 1;2;5;6\right\}$.\\
		Vậy tổng tất cả các phần tử của $ S$ là $ 1+2+5+6=14$.}
\end{ex}

\begin{ex}%[2D1G1-3]%Câu 7
	\immini{
		Cho hàm số $ y=f(x)$ liên tục có đạo hàm trên $\mathbb{R}$. Biết hàm số $ f'(x)$ có đồ thị cho như hình vẽ bên. Có bao nhiêu giá trị nguyên của $ m$ thuộc $\left[-2019;2019\right]$ để hàm só $ g(x)=f\left(2019^x\right)-mx+2$ đồng biến trên $\left[0;1\right]$.
		\choice
		{$ 2028$}
		{$ 2019$}
		{$ 2011$}
		{\True $ 2020$}}{
		\begin{tikzpicture}[scale=0.9,font=\footnotesize, line join=round, line cap=round, >=stealth] %Đường cong bậc 3
			\draw[thick, ->] (-3.5,0)--(2.5,0);
			\draw[thick, ->] (0,-2.8)--(0,2.8);
			\draw (2.7,0) node[below] {$x$};
			\draw (0,2.9) node[left]{$y$};
			\draw (0,0) node[below left]{$0$};
			%	\draw[fill] (-1,0) circle (0.5pt)node[above]{$ -1 $};
			\draw[fill] (1,0) circle (0.5pt)node[below right]{$ 1$};
			%		\draw[fill] (3,0) circle (0.5pt)node[below]{$ 3$};
			%		\draw[fill] (2,0) circle (0.5pt)node[above]{$ 2$};
			%		\draw[fill] (0,2) circle (0.5pt)node[above left]{$ 2$};
			%		\draw[fill] (0,-2) circle (0.5pt)node[below left]{$ -2$};
			%		\draw[dashed] (-1,0)--(-1,-2)--(2,-2)--(2,0); 
			%		\draw[dashed](3,0)--(3,2)--(0,2);
			\draw[line width=1.2pt,smooth,samples=100,domain=-3.28:1.32] plot(\x,{0.667*(\x)^3+2*(\x)^2-0.667*(\x)-2});		
			%\draw[line width=1.2pt,smooth,samples=100,domain=-3.3:2.8] plot(\x,{0.75*(\x)^2+0.5*\x-1});
			%	\draw (2.0,2.8) node[left]{$y=f'(x)$};
	\end{tikzpicture}	}
	\loigiai{
		Ta có $ g'(x)=2019^x\ln 2019\cdot f'\left(2019^x\right)-m$.\\
		Ta lại có hàm số $ y=2019^x$ đồng biến trên $\left[0;1\right]$.\\
		Với $ x\in\left[0;1\right]$ thì $2019^x\in\left[1;2019\right]$ mà hàm $ y=f'(x)$ đồng biến trên $\left(1;+\infty\right)$ nên hàm $ y=f'\left(2019^x\right)$ đồng biến trên $\left[0;1\right]$.\\
		Mà $2019^x\ge 1;f'\left(2019^x\right)>0\,\forall\,x\in\left[0;1\right]$ nên hàm $ h(x)=2019^x\ln 2019\cdot f'\left(2019^x\right)$ đồng biến trên $\left[0;1\right]$.\\
		Hay $ h(x)\ge h(0)=0,\forall\,x\in\left[0;1\right]$.\\
		Do vậy hàm số $ g(x)$ đồng biến trên đoạn $\left[0;1\right]$$\Leftrightarrow g'(x)\ge 0,\forall\,x\in\left[0;1\right]$\\
		$\Leftrightarrow m\le{2019^x}\ln 2019.f'\left(2019^x\right),\forall\,x\in\left[0;1\right]$ $\Leftrightarrow m\le\underset{x\in\left[0;1\right]}{\min}\,h(x)=h(0)=0$\\
		Vì $ m$ nguyên và $ m\in\left[-2019;2019\right]\Rightarrow $có $ 2020$ giá trị $ m$ thỏa mãn yêu cầu bài toán.}
\end{ex}

\begin{ex}%[2D1G1-3]%Câu 8
	\immini{
		Cho hàm số $y=f(x)$ có đồ thị $f'(x)\,$ như hình vẽ. Có bao nhiêu giá trị nguyên $m\in\left(-2020\,;\,2020\right)$ để hàm số $g(x)=f\left(2x-3\right)\,-\ln \left(1+x^2\right)-2mx$ đồng biến trên $\left(\dfrac{1}{2};2\right)$?
		\choice
		{$ 2020$}
		{\True $ 2019$}
		{$ 2021$}
		{$ 2018$}}{
		\begin{tikzpicture}[scale=0.9,font=\footnotesize, line join=round, line cap=round, >=stealth] %Đường cong bậc 3
			\draw[thick, ->] (-2.5,0)--(2.5,0);
			\draw[thick, ->] (0,-1.8)--(0,5.8);
			\draw (2.7,0) node[below] {$x$};
			\draw (0,5.9) node[left]{$y$};
			\draw (0,0) node[below left]{$0$};
			\draw[fill] (-2,0) circle (0.5pt)node[below]{$ -2 $};
			\draw[fill] (1,0) circle (0.5pt)node[below]{$ 1$};
			\draw[fill] (-1,0) circle (0.5pt)node[below]{$-1$};
			\draw[fill] (0,4) circle (0.5pt)node[above left]{$ 2$};
			%		\draw[fill] (0,2) circle (0.5pt)node[above left]{$ 2$};
			%		\draw[fill] (0,-2) circle (0.5pt)node[below left]{$ -2$};
			\draw[dashed] (-2,0)--(-2,4)--(1,4)--(1,0); 
			%		\draw[dashed](3,0)--(3,2)--(0,2);
			\draw[line width=1.2pt,smooth,samples=100,domain=-2.1:2.1] plot(\x,{-1*(\x)^3+0*(\x)^2+3*(\x)+2});		
			%\draw[line width=1.2pt,smooth,samples=100,domain=-3.3:2.8] plot(\x,{0.75*(\x)^2+0.5*\x-1});
			%	\draw (2.0,2.8) node[left]{$y=f'(x)$};
	\end{tikzpicture}	}
	\loigiai{
		Ta có $g'(x)=2f'\left(2x-3\right)-\dfrac{2x}{1+x^2}-2m$.\\
		Hàm số $ g(x)$ đồng biến trên $\left(\dfrac{1}{2};2\right)$ khi và chỉ khi \\
		$g'(x)\ge 0,\,\,\forall x\in\left(-1;\,2\right)$\\
		$\Leftrightarrow m\le{f}'\left(2x-3\right)-\dfrac{x}{1+x^2},\,\,\forall x\in\left(\dfrac{1}{2};2\right)$\\
		$\Leftrightarrow m\le\underset{x\in\left[\dfrac{1}{2};2\right]}{\min}\,\left[f'\left(2x-3\right)-\dfrac{x}{1+x^2}\right]$. \, \,  $(1)$\\
		Đặt $ t=2x-3$, khi đó $ x\in\left(\dfrac{1}{2};2\right)\Leftrightarrow t\in\left(-2;\,1\right)$.\\
		Từ đồ thị hàm $f'(x)$ suy ra $f'(t)\ge 0,\,\,\forall t\in\left(-2;1\right)$ và $f'(t)=0$ khi $ t=-1$.\\
		Tức là $f'\left(2x-3\right)\ge 0,\,\,\forall x\in\left(\dfrac{1}{2};\,2\right)$$\Rightarrow\underset{x\in\left[\dfrac{1}{2};2\right]}{\min}\,f'\left(2x-3\right)=0$ khi $ x=1$. $(2)$\\
		Xét hàm số $ h(x)=-\dfrac{x}{1+x^2}$ trên khoảng $\left(\dfrac{1}{2};\,2\right)$.\\
		Ta có $h'(x)=\dfrac{x^2-1}{\left(1+x^2\right)^2}$ và\\
		$h'(x)=0\Leftrightarrow{x^2}-1=0\Leftrightarrow x=\pm 1$.\\
		Bảng biến thiên của hàm số $ h(x)$ trên $\left(\dfrac{1}{2};\,2\right)$ như sau
		\begin{center}
			\begin{tikzpicture}
				\tkzTabInit[lgt=1.2,espcl=2.5,nocadre]
				{$x$ /0.7, $h'(x)$ /0.7,$h(x)$ /2.5}
				{$\dfrac{1}{2}$ , $1$,$2$}
				\tkzTabLine{,-,0,+,}
				\tkzTabVar{+/$  $ ,-/$ \-\dfrac{1}{2} $, +/$ $}
			\end{tikzpicture}
		\end{center}
		Từ bảng biến thiên suy ra $ h(x)\ge-\dfrac{1}{2}$$\Rightarrow\underset{x\in\left[\dfrac{1}{2};2\right]}{\min}\,h(x)=-\dfrac{1}{2}$ khi $ x=1$. \, \,  $(3)$\\
		Từ $(1)$, $(2)$ và $(3)$ suy ra $ m\le-\dfrac{1}{2}$.\\
		Kết hợp với $ m\in\mathbb{Z}$, $ m\in\left(-2020;\,2020\right)$ thì $ m\in\left\{-2019;\,-201;\ldots ;-2;-1\right\}$.\\
		Vậy có tất cả $ 2019$ giá trị $ m$ cần tìm.}
\end{ex}

\begin{ex}%[2D1G1-3]%Câu 9
	Cho hàm số $ f(x)$ liên tục trên $\mathbb{R}$ và có đạo hàm $f'(x)=x^2\left(x-2\right)\left(x^2-6x+m\right)$ với mọi $ x\in\mathbb{R}$. Có bao nhiêu số nguyên $ m$ thuộc đoạn $\left[-2020;2020\right]$ để hàm số $ g(x)=f\left(1-x\right)$ nghịch biến trên khoảng $\left(-\infty ;-1\right)$?
	\choice
	{$ 2016$}
	{$ 2014$}
	{\True $ 2012$}
	{$ 2010$}
	\loigiai{
		Ta có \\
		$g'(x)=f'\left(1-x\right)=-\left(1-x\right)^2\left(-x-1\right)\left[\left(1-x\right)^2-6\left(1-x\right)+m\right]$
		$=\left(x-1\right)^2\left(x+1\right)\left(x^2+4x+m-5\right)$.\\
		Hàm số $ g(x)$ nghịch biến trên khoảng $\left(-\infty ;-1\right)$\\
		$\Leftrightarrow{g}'(x)\le 0,\forall x<-1$ $(*)$, (dấu \lq\lq $=$\rq\rq \, xảy ra tại hữu hạn điểm).\\
		Với $ x<-1$ thì $\left(x-1\right)^2>0$ và $ x+1<0$ nên\\
		$(*)$ $\Leftrightarrow{x^2}+4x+m-5\ge 0,\forall x<-1 \Leftrightarrow m\ge-x^2-4x+5,\forall x<-1$.\\
		Xét hàm số $ y=-x^2-4x+5$ trên khoảng $\left(-\infty ;-1\right)$, ta có bảng biến thiên
		\begin{center}
			\begin{tikzpicture}
				\tkzTabInit[lgt=1.8,espcl=2.3]
				{$x$ /1.2, $y'$ /1.2,$y$ /2}
				{$-\infty$ , $-2$,$-1$}
				\tkzTabLine{,+,0,-,}
				\tkzTabVar{-/$ -\infty $ ,+/$9 $, -/$ 8$}
			\end{tikzpicture}
		\end{center}
		Từ bảng biến thiên suy ra $ m\ge 9$.\\
		Kết hợp với $ m$ thuộc đoạn $\left[-2020;2020\right]$ và $ m$ nguyên nên $ m\in\left\{ 9;10;11;\ldots ;2020\right\}$.\\
		Vậy có $ 2012$ số nguyên $ m$ thỏa mãn đề bài.}
\end{ex}

\begin{ex}%[2D1G1-3]%Câu 10
	\immini{
		Cho hàm số $f(x)$ xác định và liên tục trên $ R$. Hàm số $y=f'(x)$ liên tục trên $\mathbb{R}$ và có đồ thị như hình vẽ bên.
		Xét hàm số $g(x)=f\left(x-2m\right)+\dfrac{1}{2}{\left(2m-x\right)^2}+2020$, với $ m$ là tham số thực. Gọi $ S$ là tập hợp các giá trị nguyên dương của $ m$ để hàm số $ y=g(x)$ nghịch biến trên khoảng $\left(3;4\right)$. Hỏi số phần tử của $ S$ bằng bao nhiêu?
		\choice
		{$4$}
		{\True $2$}
		{$3$}
		{Vô số}}
	{
		\begin{tikzpicture}[scale=0.7,>=stealth, font=\footnotesize, line join=round, line cap=round]
			\def\xmin{-3.5} \def\xmax{4.5}
			\def\ymin{-5.2} \def\ymax{4}
			\clip(\xmin,\ymin) rectangle (\xmax,\ymax);
			\draw[->] (\xmin,0)--(\xmax,0) node [below]{$x$};
			\draw[->] (0,\ymin)--(0,\ymax) node [left]{$y$};
			\node at (0,0) [below left]{$O$};
			\path
			(-3.1,3.7) coordinate (A)
			(-3,3) coordinate (B)
			(0,-2) coordinate (C)
			(0.65,-2) coordinate (D)
			(1,-1) coordinate (E)
			(3,-3) coordinate (F)
			(3.4,-5) coordinate (G);
			\draw[smooth]
			(A)..controls +(-88:0.1) and +(93:.1)..
			(B)..controls +(-87:0.3) and +(-100:8.5)..
			(C)..controls +(75:.8) and +(180:.1)..
			(D)..controls +(0:.1) and +(-105:.3)..
			(E)..controls +(70:2) and +(97:0.4)..
			(F)..controls +(-80:.1) and +(90:0.3)..
			(G);
			\draw[dashed] 
			(-3,0)node[below]{$-3$}|-(0,3)node[right]{$3$}
			(1,0)node[above]{$1$}|-(0,-1)node[left]{$-1$}
			(3,0)node[above]{$3$}|-(0,-3)node[below right]{$-3$};
			\fill 
			(0,-2) circle(1.5pt)
			(-3,3) circle(1.5pt)
			(3,-3) circle(1.5pt)
			(1,-1) circle(1.5pt);
			\node at (2.1,-4) {$y=f'(x)$};
		\end{tikzpicture}
	}
	\loigiai{
		Ta có $g'(x)=f'\left(x-2m\right)-\left(2m-x\right)$.		Đặt $h(x)=f'(x)-\left(-x\right)$.\\
		Từ đồ thị hàm số $y=f'(x)$ và đồ thị hàm số $y=-x$ trên hình vẽ suy ra \\
		$h(x)\le 0\Leftrightarrow f'(x)\le-x\Leftrightarrow\hoac{
			&-3\le x\le 1\\ 
			& x\ge 3.}$ 
		\begin{center}
			\begin{tikzpicture}[scale=0.7,>=stealth, font=\footnotesize, line join=round, line cap=round]
				\def\xmin{-3.5} \def\xmax{4.5}
				\def\ymin{-5.2} \def\ymax{4}
				\clip(\xmin,\ymin) rectangle (\xmax,\ymax);
				\draw[->] (\xmin,0)--(\xmax,0) node [below]{$x$};
				\draw[->] (0,\ymin)--(0,\ymax) node [left]{$y$};
				\node at (0,0) [below left]{$O$};
				\path
				(-3.1,3.7) coordinate (A)
				(-3,3) coordinate (B)
				(0,-2) coordinate (C)
				(0.65,-2) coordinate (D)
				(1,-1) coordinate (E)
				(3,-3) coordinate (F)
				(3.4,-5) coordinate (G);
				\draw[smooth]
				(A)..controls +(-88:0.1) and +(93:.1)..
				(B)..controls +(-87:0.3) and +(-100:8.5)..
				(C)..controls +(75:.8) and +(180:.1)..
				(D)..controls +(0:.1) and +(-105:.3)..
				(E)..controls +(70:2) and +(97:0.4)..
				(F)..controls +(-80:.1) and +(90:0.3)..
				(G);
				\draw[dashed] 
				(-3,0)node[below]{$-3$}|-(0,3)node[right]{$3$}
				(1,0)node[above]{$1$}|-(0,-1)node[left]{$-1$}
				(3,0)node[above]{$3$}|-(0,-3)node[below right]{$-3$};
				\fill 
				(0,-2) circle(1.5pt)
				(-3,3) circle(1.5pt)
				(3,-3) circle(1.5pt)
				(1,-1) circle(1.5pt);
				\draw[smooth,samples=300,domain=-3.2:3.7] plot(\x,{-(\x)});
				\node at (2.1,-4) {$y=f'(x)$};
				\node at (-1,2.1) {$y=h(x)$};
			\end{tikzpicture}
		\end{center}
		Ta có $ g'(x)=h\left(x-2m\right)\le 0\Leftrightarrow\hoac{
			&-3\le x-2m\le 1\\ 
			& x-2m\ge 3}\Leftrightarrow\hoac{
			& 2m-3\le x\le 2m+1\\ 
			& x\ge 2m+3.}$.\\
		Suy ra hàm số $ y=g(x)$ nghịch biến trên các khoảng $\left(2m-3;2m+1\right)$ và $\left(2m+3;+\infty\right)$.\\
		Do đó hàm số $ y=g(x)$ nghịch biến trên khoảng $\left(3;4\right)$ $\Leftrightarrow\hoac{
			&\heva{
				& 2m-3\le 3\\ 
				& 2m+1\ge 4}\\ 
			& 2m+3\le 3}\Leftrightarrow\hoac{
			&\dfrac{3}{2}\le m\le 3\\ 
			& m\le 0.}$ \\
		Mặt khác, do $ m$ nguyên dương nên $ m\in\left\{ 2;3\right\}\Rightarrow S=\left\{ 2;3\right\}$. Vậy số phần tử của $ S$ bằng $2$.\\
	}
	
\end{ex}

\begin{ex}%[2D1G1-3]%Câu 11
	Cho hàm số $f(x)$ có đạo hàm trên $\mathbb{R}$ là $f'(x)=\left(x-1\right)\left(x+3\right)$. Có bao nhiêu giá trị nguyên của tham số $m$ thuộc đoạn $\left[-10;20\right]$ để hàm số $y=f\left(x^2+3x-m\right)$ đồng biến trên khoảng $\left(0;2\right)$?
	\choice
	{\True $ 18$}
	{$ 17$}
	{$ 16$}
	{$ 20$}
	\loigiai{
		Ta có $y'=f'\left(x^2+3x-m\right)=\left(2x+3\right){f}'\left(x^2+3x-m\right)$.\\
		Theo đề bài ta có $f'(x)=\left(x-1\right)\left(x+3\right)$\\
		suy ra $f'(x)>0\Leftrightarrow\hoac{
			& x<-3\\ 
			& x>1}$ và $f'(x)<0\Leftrightarrow-3<x<1$ .\\
		Hàm số đồng biến trên khoảng $\left(0;2\right)$ khi $y'\ge 0,\forall x\in\left(0;2\right)$\\
		$\Leftrightarrow\left(2x+3\right){f}'\left(x^2+3x-m\right)\ge 0,\forall x\in\left(0;2\right)$.\\
		Do $x\in\left(0;2\right)$ nên $2x+3>0,\forall x\in\left(0;2\right)$. Do đó, ta có\\
		$y'\ge 0,\forall x\in\left(0;2\right)\Leftrightarrow f'\left(x^2+3x-m\right)\ge 0$\\
		$\Leftrightarrow\hoac{
			&{x^2}+3x-m\le-3\\ 
			&{x^2}+3x-m\ge 1}\Leftrightarrow\hoac{
			& m\ge{x^2}+3x+3\\ 
			& m\le{x^2}+3x-1}$\\
		$\Leftrightarrow\hoac{
			& m\ge\underset{\left[0;2\right]}{\max}\,\left(x^2+3x+3\right)\\ 
			& m\le\underset{\left[0;2\right]}{\min}\,\left(x^2+3x-1\right)} \Leftrightarrow\hoac{
			& m\ge 13\\ 
			& m\le-1}$.\\
		Do $m\in\left[-10;20\right]$, $ m\in\mathbb{Z}$ nên có $ 18$ giá trị nguyên của $m$ thỏa yêu cầu đề bài.}
\end{ex}

\begin{ex}%[2D1G1-3]%Câu 12
	Cho các hàm số $f(x)=x^3+4x+m$ và $g(x)=\left(x^2+2018\right){\left(x^2+2019\right)^2}{\left(x^2+2020\right)^3}$ . Có bao nhiêu giá trị nguyên của tham số $m\in\left[-2020;2020\right]$ để hàm số $g\left(f(x)\right)$ đồng biến trên $\left(2;+\infty\right)$ ?
	\choice
	{$2005$}
	{\True $2037$}
	{$4016$}
	{$4041$}
	\loigiai{
		Ta có $f(x)=x^3+4x+m$ và \\
		$g(x)=\left(x^2+2018\right){\left(x^2+2019\right)^2}{\left(x^2+2020\right)^3}=a_{12}{x^{12}}+a_{10}{x^{10}}+...+a_2x^2+a_0$.\\
		Suy ra $f'(x)=3x^2+4$ , $g'(x)=12a_{12}{x^{11}}+10a_{10}{x^9}+...+2a_2x$.\\
		Và có 
		\begin{eqnarray*}
			\left[g\left(f(x)\right)\right]' &=& f'(x)\left[12a_{12}{\left(f(x)\right)^{11}}+10a_{10}{\left(f(x)\right)^9}+...+2a_2f(x)\right]\\
			&=& f(x)f'(x)\left(12a_{12}{\left(f(x)\right)^{10}}+10a_{10}{\left(f(x)\right)^8}+...+2a_2\right).
		\end{eqnarray*} 
		Dễ thấy $a_{12};{a_{10}};...;{a_2};{a_0}>0$ và $f'(x)=3x^2+4>0$, $\forall x>2$.\\
		Do đó $f'(x)\left(12a_{12}{\left(f(x)\right)^{10}}+10a_{10}{\left(f(x)\right)^8}+...+2a_2\right)>0$ , $\forall x>2$.\\
		Hàm số $g\left(f(x)\right)$ đồng biến trên $\left(2;+\infty\right)$ khi $\left[g\left(f(x)\right)\right]^{'}\ge 0$, $\forall x>2$\\
		$\Rightarrow  f(x)\ge 0$, $\forall x>2 \Leftrightarrow x^3+4x+m\ge 0$, $\forall x>3 \Leftrightarrow  m\ge-x^3-4x$, $\forall x>2$\\
		$ \Rightarrow  m\ge\underset{\left[2;+\infty\right)}{\max}\,\left(-x^3-4x\right)=-16$.\\
		Vì $m\in\left[-2020;2020\right]$ và $m\in\mathbb{Z}$ nên có $2037$ giá trị thỏa mãn $m$ .}
\end{ex}

\begin{ex}%[2D1G1-3]%Câu 13
	Cho hàm số $y=f(x)$ có đạo hàm $f'(x)=x{\left(x+1\right)^2}\left(x^2+2mx+1\right)$ với mọi $x \in \mathbb{R}$. Có bao nhiêu số nguyên âm $m$ để hàm số $g(x)=f\left(2x+1\right)$ đồng biến trên khoảng $\left(3;5\right)$?
	\choice
	{\True $3$}
	{$2$}
	{$4$}
	{$6$}
	\loigiai{
		Ta có $g'(x)=2f'(2x+1)=2(2x+1)(2x+2)^2[(2x+1)^2+2m(2x+1)+1]$. 	Đặt $t=2x+1$\\
		Để hàm số $g(x)$ đồng biến trên khoảng $\left(3;5\right)$ khi và chỉ khi 
		\begin{eqnarray*}
			& & g'(x)\ge 0,\forall x\in\left(3;5\right) \\
			& \Leftrightarrow & t(t^2+2mt+1)\ge 0,\forall t\in\left(7;11\right)\Leftrightarrow{t^2}+2mt+1\ge 0,\,\,\forall t\in\left(7;11\right) \\
			&\Leftrightarrow & 2m\ge\dfrac{-t^2-1}{t},\,\,\,\forall t\in\left(7;11\right)
		\end{eqnarray*}	
		Xét hàm số $h(t)=\dfrac{-t^2-1}{t}$ trên $\left[7;11\right]$, có $h'(t)=\dfrac{-t^2+1}{t^2}$\\
		Bảng biến thiên
		\begin{center}
			\begin{tikzpicture}
				\tkzTabInit[espcl=3,lgt=1.2,nocadre]
				{$t$/0.7,$h'(t)$/0.7,$h(t)$/2.5}
				{$-\infty$,$1$,$11$,$+\infty$}
				\tkzTabLine{, ,,-,,,}
				%	\node (0) at ($(N12)+(0,-3)$) {$-\infty$};
				\node (1) at ($(N22)+(0,-0.8)$) [right] {$-\dfrac{50}{7}$};
				\node (2) at ($(N32)+(0,-2.5)$) [left] {$-\dfrac{122}{11}$};
				
				
				%				\node (3) at ($(N11+(-0.5,0))$) {};
				%				\node (4) at ($(N23)$) {};
				\fill[pattern=north east lines] (7.0,-0.7) rectangle (10,-4.4);
				\fill[pattern=north east lines] (1.5,-0.7) rectangle (4.5,-4.4);
				\draw[->] (1)--(2);	
				\draw[dashed] (4.5,-0.7)--(4.5,-4.4);
				\draw[dashed] (7.0,-0.7)--(7.0,-4.4);	
			\end{tikzpicture}		
		\end{center}
		Dựa vào BBT ta có $2m\ge\dfrac{-t^2-1}{t},\,\,\,\forall t\in\left(7;11\right)\Leftrightarrow 2m\ge\underset{\left[7;11\right]}{\max}\,h(t)\Leftrightarrow m\ge-\dfrac{50}{14}$\\
		Vì $ m\in{\mathbb{Z}^-}\Rightarrow m \in \{-3;-2;-1\}$ .
	}
\end{ex}

\begin{ex}%[2D1G1-3]%Câu 14
	Cho hàm số $y=f(x)$ có bảng biến thiên như sau\\
	\begin{center}
		\begin{tikzpicture}[>=stealth,scale = 1]
			\tkzTabInit[lgt=1,espcl=2.5,nocadre]
			{$x$ /0.7, $y'$ /0.7,$y$ /2.5}
			{$-\infty$,$0$,$2$,$+\infty$}
			\tkzTabLine{ ,-,0,+,0,-,}
			\tkzTabVar{-/$-\infty$, +/$4$,- /$0$, +/{ $+\infty$}}
		\end{tikzpicture}
	\end{center}
	Có bao nhiêu số nguyên $m<2019$ để hàm số $g(x)=f\left(x^2-2x+m\right)$ đồng biến trên khoảng $\left(1;+\infty\right)$?
	\choice
	{\True $2016$}
	{$2015$}
	{$2017$}
	{$2018$}
	\loigiai{
		Ta có $g'(x)=\left(x^2-2x+m\right)'{f}'\left(x^2-2x+m\right)=2\left(x-1\right){f}'\left(x^2-2x+m\right)$ .\\
		Hàm số $y=g(x)$ đồng biến trên khoảng $\left(1;+\infty\right)$ khi và chỉ khi $g'(x)\ge 0,\forall x\in\left(1;+\infty\right)$ và\\
		$g'(x)=0$ tại hữu hạn điểm \\
		$\Leftrightarrow 2\left(x-1\right){f}'\left(x^2-2x+m\right)\ge 0,\forall x\in\left(1;+\infty\right)$\\
		$\Leftrightarrow{f}'\left(x^2-2x+m\right)\ge 0,\forall x\in\left(1;+\infty\right)$ $\Leftrightarrow\hoac{
			&{x^2}-2x+m\ge 2,\forall x\in\left(1;+\infty\right)\\ 
			&{x^2}-2x+m\le 0,\forall x\in\left(1;+\infty\right).}$\\
		Xét hàm số $y=x^2-2x+m$, ta có bảng biến thiên
		\begin{center}
			\begin{tikzpicture}[>=stealth,scale = 1]
				\tkzTabInit[lgt=1,espcl=2.5,nocadre]
				{$x$ /0.7, $y'$ /0.7,$y$ /2.5}
				{$-\infty$,$2$,$+\infty$}
				\tkzTabLine{ ,-,0,+,}
				\tkzTabVar{+/$+\infty$, -/$m-1$, +/{$+\infty$}}
			\end{tikzpicture}
		\end{center}
		Dựa vào bảng biến thiên ta có\\
		TH1: $x^2-2x+m\ge 2,\forall x\in\left(1;+\infty\right)\Leftrightarrow m-1\ge 2\Leftrightarrow m\ge 3$ .\\
		TH2: $x^2-2x+m\le 0,\forall x\in\left(1;+\infty\right)$. Không có giá trị $m$ thỏa mãn.\\
		Vậy có $2016$ số nguyên $m<2019$ thỏa mãn yêu cầu bài toán.}
\end{ex}

\begin{ex}%[2D1G1-3]%Câu 15
	\immini{
		Cho hàm số $ y=f(x)$ có đạo hàm là hàm số $f'(x)$ trên $\mathbb{R}$. Biết rằng hàm số $ y=f'\left(x-2\right)+2$ có đồ thị như hình vẽ bên dưới. Hàm số $ f(x)$ đồng biến trên khoảng nào?
		\choice
		{$\left(-\infty ;3\right),\,\,\left(5;+\infty\right)$}
		{\True $\left(-\infty ;-1\right),\,\,\left(1;+\infty\right)$}
		{$\left(-1;1\right)$}
		{$\left(3;5\right)$}}{
		\begin{tikzpicture}[scale=0.7,font=\footnotesize, line join=round, line cap=round, >=stealth] %Đường cong bậc 3
			\draw[thick, ->] (-0.5,0)--(3.5,0);
			\draw[thick, ->] (0,-1.8)--(0,5.3);
			\draw (3.7,0) node[below] {$x$};
			\draw (0,5.4) node[left]{$y$};
			\draw (0,0) node[below left]{$0$};
			\draw[fill] (3,0) circle (0.5pt)node[below]{$ 3$};
			\draw[fill] (1,0) circle (0.5pt)node[below]{$ 1$};
			\draw[fill] (2,0) circle (0.5pt)node[above]{$2$};
			\draw[fill] (0,2) circle (0.5pt)node[left]{$ 2$};
			\draw[fill] (0,-1) circle (0.5pt)node[left]{$ -1$};
			%		\draw[fill] (0,2) circle (0.5pt)node[above left]{$ 2$};
			%		\draw[fill] (0,-2) circle (0.5pt)node[below left]{$ -2$};
			\draw[dashed] (3,0)--(3,2)--(0,2)--(1,2)--(1,0); 
			\draw[dashed](0,-1)--(2,-1)--(2,0);
			\draw[line width=1.2pt,smooth,samples=100,domain=0.6:3.4] plot(\x,{3*(\x)^2-12*(\x)+11});		
			%\draw[line width=1.2pt,smooth,samples=100,domain=-3.3:2.8] plot(\x,{0.75*(\x)^2+0.5*\x-1});
			%	\draw (2.0,2.8) node[left]{$y=f'(x)$};
	\end{tikzpicture}	}
	\loigiai{	
		Hàm số $ y=f'\left(x-2\right)+2$ có đồ thị $(C)$ như sau:\\
		\begin{center}
			\begin{tikzpicture}[scale=0.7,font=\footnotesize, line join=round, line cap=round, >=stealth] %Đường cong bậc 3
				\draw[thick, ->] (-0.5,0)--(3.5,0);
				\draw[thick, ->] (0,-1.8)--(0,5.3);
				\draw (3.7,0) node[below] {$x$};
				\draw (0,5.4) node[left]{$y$};
				\draw (0,0) node[below left]{$0$};
				\draw[fill] (3,0) circle (0.5pt)node[below]{$ 3$};
				\draw[fill] (1,0) circle (0.5pt)node[below]{$ 1$};
				\draw[fill] (2,0) circle (0.5pt)node[above]{$2$};
				\draw[fill] (0,2) circle (0.5pt)node[left]{$ 2$};
				\draw[fill] (0,-1) circle (0.5pt)node[left]{$ -1$};
				%		\draw[fill] (0,2) circle (0.5pt)node[above left]{$ 2$};
				%		\draw[fill] (0,-2) circle (0.5pt)node[below left]{$ -2$};
				\draw[dashed] (3,0)--(3,2)--(0,2)--(1,2)--(1,0); 
				\draw[dashed](0,-1)--(2,-1)--(2,0);
				\draw[line width=1.2pt,smooth,samples=100,domain=0.6:3.4] plot(\x,{3*(\x)^2-12*(\x)+11});		
				%\draw[line width=1.2pt,smooth,samples=100,domain=-3.3:2.8] plot(\x,{0.75*(\x)^2+0.5*\x-1});
				%	\draw (2.0,2.8) node[left]{$y=f'(x)$};
			\end{tikzpicture}
		\end{center}
		Dựa vào đồ thị $(C)$ ta có\\ $f'\left(x-2\right)+2>2,\forall x\in\left(-\infty ;1\right)\cup\left(3;+\infty\right)\Leftrightarrow{f}'\left(x-2\right)>0,\forall x\in\left(-\infty ;1\right)\cup\left(3;+\infty\right)$ .\\
		Đặt $ x*=x-2$ suy ra $f'\left(x*\right)>0,\forall x*\in\left(-\infty ;-1\right)\bigcup\left(1;+\infty\right)$.\\
		Vậy hàm số $ f(x)$ đồng biến trên khoảng $\left(-\infty ;-1\right),\,\,\left(1;+\infty\right)$.}
\end{ex}

\begin{ex}%[2D1G1-2]%Câu 16
	\immini{
		Cho hàm số $ y=f(x)$ có đạo hàm là hàm số $f'(x)$ trên $\mathbb{R}$. Biết rằng hàm số $ y=f'\left(x+2\right)-2$ có đồ thị như hình vẽ bên dưới. Hàm số $ f(x)$ nghịch biến trên khoảng nào?
		\choice
		{$\left(-3;-1\right),\,\,\left(1;3\right)$}
		{\True $\left(-1;1\right),\,\,\left(3;5\right)$}
		{$\left(-\infty ;-2\right),\,\,\left(0;2\right)$}
		{$\left(-5;-3\right),\,\,\left(-1;1\right)$}}{
		\begin{tikzpicture}[scale=0.7,font=\footnotesize, line join=round, line cap=round, >=stealth] %Đường cong bậc 3
			\draw[thick, ->] (-3.8,0)--(4.0,0);
			\draw[thick, ->] (0,-4.8)--(0,3.5);
			\draw (4.2,0) node[below] {$x$};
			\draw (0,3.7) node[left]{$y$};
			\draw (0,0) node[below left]{$0$};
			\draw[fill] (-3,0) circle (0.5pt)node[above]{$ -3$};
			\draw[fill] (-1,0) circle (0.5pt)node[above]{$ -1$};
			\draw[fill] (1,0) circle (0.5pt)node[above]{$ 1$};
			\draw[fill] (3,0) circle (0.5pt)node[above]{$3$};
			\draw[fill] (0,2) circle (0.5pt)node[above left]{$ 2$};
			\draw[fill] (0,-1) circle (0.5pt)node[above right]{$ -1$};
			%		\draw[fill] (0,2) circle (0.5pt)node[above left]{$ 2$};
			%		\draw[fill] (0,-2) circle (0.5pt)node[below left]{$ -2$};
			\draw[dashed] (-3,0)--(-3,-2)--(3,-2)--(3,0) (-1,0)--(-1,-2) (1,0)--(1,-2) (-3.494,0)--(-3.494,2)--(3.494,2)--(3.494,0); 
			\draw[line width=1.2pt,smooth,samples=100,domain=-3.6:3.6] plot(\x,{0.11*(\x)^4-1.11*(\x)^2-1});		
			%\draw[line width=1.2pt,smooth,samples=100,domain=-3.3:2.8] plot(\x,{0.75*(\x)^2+0.5*\x-1});
			%	\draw (2.0,2.8) node[left]{$y=f'(x)$};
	\end{tikzpicture}	}
	\loigiai{
		Hàm số $ y=f'\left(x+2\right)-2$ có đồ thị $(C)$ như sau
		\begin{center}
			\begin{tikzpicture}[scale=0.7,font=\footnotesize, line join=round, line cap=round, >=stealth] %Đường cong bậc 3
				\draw[thick, ->] (-3.8,0)--(4.0,0);
				\draw[thick, ->] (0,-4.8)--(0,3.5);
				\draw (4.2,0) node[below] {$x$};
				\draw (0,3.7) node[left]{$y$};
				\draw (0,0) node[below left]{$0$};
				\draw[fill] (-3,0) circle (0.5pt)node[above]{$ -3$};
				\draw[fill] (-1,0) circle (0.5pt)node[above]{$ -1$};
				\draw[fill] (1,0) circle (0.5pt)node[above]{$ 1$};
				\draw[fill] (3,0) circle (0.5pt)node[above]{$3$};
				\draw[fill] (0,2) circle (0.5pt)node[above left]{$ 2$};
				\draw[fill] (0,-1) circle (0.5pt)node[above right]{$ -1$};
				%		\draw[fill] (0,2) circle (0.5pt)node[above left]{$ 2$};
				%		\draw[fill] (0,-2) circle (0.5pt)node[below left]{$ -2$};
				\draw[dashed] (-3,0)--(-3,-2)--(3,-2)--(3,0) (-1,0)--(-1,-2) (1,0)--(1,-2) (-3.494,0)--(-3.494,2)--(3.494,2)--(3.494,0); 
				\draw[line width=1.2pt,smooth,samples=100,domain=-3.6:3.6] plot(\x,{0.11*(\x)^4-1.11*(\x)^2-1});		
				%\draw[line width=1.2pt,smooth,samples=100,domain=-3.3:2.8] plot(\x,{0.75*(\x)^2+0.5*\x-1});
				%	\draw (2.0,2.8) node[left]{$y=f'(x)$};
			\end{tikzpicture}
		\end{center}
		Dựa vào đồ thị $(C)$ ta có\\
		$f'\left(x+2\right)-2<-2,\forall x\in\left(-3;-1\right)\bigcup\left(1;3\right)\Leftrightarrow{f}'\left(x+2\right)<0,\forall x\in\left(-3;-1\right)\bigcup\left(1;3\right)$.\\
		Đặt $ x^*=x+2$ suy ra: $f'\left(x^*\right)<0,\forall x^*\in\left(-1;1\right)\bigcup\left(3;5\right)$.\\
		Vậy: Hàm số $ f(x)$ đồng biến trên khoảng $\left(-1;1\right),\,\,\left(3;5\right)$.}
\end{ex}

\begin{ex}%[2D1G1-2]%Câu 17
	\immini{
		Cho hàm số $ y=f(x)$ có đạo hàm là hàm số $f'(x)$ trên $\mathbb{R}$. Biết rằng hàm số $ y=f'\left(x-2\right)+2$ có đồ thị như hình vẽ bên dưới. Hàm số $ f(x)$ nghịch biến trên khoảng nào?
		\choice
		{$\left(-\infty ;2\right)$}
		{\True $\left(-1;1\right)$}
		{$\left(\dfrac{3}{2};\dfrac{5}{2}\right)$}
		{$\left(2;+\infty\right)$}}{
		\begin{tikzpicture}[scale=0.7,font=\footnotesize, line join=round, line cap=round, >=stealth] %Đường cong bậc 3
			\draw[thick, ->] (-0.5,0)--(3.5,0);
			\draw[thick, ->] (0,-1.8)--(0,5.3);
			\draw (3.7,0) node[below] {$x$};
			\draw (0,5.4) node[left]{$y$};
			\draw (0,0) node[below left]{$0$};
			\draw[fill] (3,0) circle (0.5pt)node[below]{$ 3$};
			\draw[fill] (1,0) circle (0.5pt)node[below]{$ 1$};
			\draw[fill] (2,0) circle (0.5pt)node[above]{$2$};
			\draw[fill] (0,2) circle (0.5pt)node[left]{$ 2$};
			\draw[fill] (0,-1) circle (0.5pt)node[left]{$ -1$};
			%		\draw[fill] (0,2) circle (0.5pt)node[above left]{$ 2$};
			%		\draw[fill] (0,-2) circle (0.5pt)node[below left]{$ -2$};
			\draw[dashed] (3,0)--(3,2)--(0,2)--(1,2)--(1,0); 
			\draw[dashed](0,-1)--(2,-1)--(2,0);
			\draw[line width=1.2pt,smooth,samples=100,domain=0.6:3.4] plot(\x,{3*(\x)^2-12*(\x)+11});		
			%\draw[line width=1.2pt,smooth,samples=100,domain=-3.3:2.8] plot(\x,{0.75*(\x)^2+0.5*\x-1});
			%	\draw (2.0,2.8) node[left]{$y=f'(x)$};
	\end{tikzpicture}	}
	\loigiai{
		Hàm số $ y=f'\left(x-2\right)+2$ có đồ thị $(C)$ như sau
		\begin{center}
			\begin{tikzpicture}[scale=0.7,font=\footnotesize, line join=round, line cap=round, >=stealth] %Đường cong bậc 3
				\draw[thick, ->] (-0.5,0)--(3.5,0);
				\draw[thick, ->] (0,-1.8)--(0,5.3);
				\draw (3.7,0) node[below] {$x$};
				\draw (0,5.4) node[left]{$y$};
				\draw (0,0) node[below left]{$0$};
				\draw[fill] (3,0) circle (0.5pt)node[below]{$ 3$};
				\draw[fill] (1,0) circle (0.5pt)node[below]{$ 1$};
				\draw[fill] (2,0) circle (0.5pt)node[above]{$2$};
				\draw[fill] (0,2) circle (0.5pt)node[left]{$ 2$};
				\draw[fill] (0,-1) circle (0.5pt)node[left]{$ -1$};
				%		\draw[fill] (0,2) circle (0.5pt)node[above left]{$ 2$};
				%		\draw[fill] (0,-2) circle (0.5pt)node[below left]{$ -2$};
				\draw[dashed] (3,0)--(3,2)--(0,2)--(1,2)--(1,0); 
				\draw[dashed](0,-1)--(2,-1)--(2,0);
				\draw[line width=1.2pt,smooth,samples=100,domain=0.6:3.4] plot(\x,{3*(\x)^2-12*(\x)+11});		
				%\draw[line width=1.2pt,smooth,samples=100,domain=-3.3:2.8] plot(\x,{0.75*(\x)^2+0.5*\x-1});
				%	\draw (2.0,2.8) node[left]{$y=f'(x)$};
			\end{tikzpicture}
		\end{center}
		Dựa vào đồ thị $(C)$ ta có\\
		$f'\left(x-2\right)+2<2,\forall x\in\left(1;3\right)\Leftrightarrow{f}'\left(x-2\right)<0,\forall x\in\left(1;3\right)$.\\
		Đặt $ x^*=x-2$ thì $f'\left(x^*\right)<0,\forall x^*\in\left(-1;1\right)$.\\
		Vậy: Hàm số $ f(x)$ nghịch biến trên khoảng $\left(-1;1\right)$.\\
		Cách khác:\\
		Tịnh tiến sang trái hai đơn vị và xuống dưới $2$ đơn vị thì từ đồ thị $(C)$ sẽ thành đồ thị của hàm$ y=f'(x)$. Khi đó $f'(x)<0,\forall x\in\left(-1;1\right)$.\\
		Vậy hàm số $ f(x)$ nghịch biến trên khoảng $\left(-1;1\right)$.}
\end{ex}

\begin{ex}%[2D1G1-2]%Câu 18
	Cho hàm số $y=f(x)$ có đạo hàm cấp $ 3$ liên tục trên $\mathbb{R}$ và thỏa mãn $f(x)\cdot f'''(x)=x{\left(x-1\right)^2}{\left(x+4\right)^3}$ với mọi $x\in\mathbb{R}$ và $g(x)=\left[f'(x)\right]^2-2f(x)\cdot f''(x)$. Hàm số $h(x)=g\left(x^2-2x\right)$ đồng biến trên khoảng nào dưới đây?
	\choice
	{$\left(-\infty ;1\right)$}
	{$\left(2;+\infty\right)$}
	{$\left(0;1\right)$}
	{\True $\left(1;2\right)$}
	\loigiai{		
		Ta có $g'(x)=2f''(x){f}'(x)-2f'(x)\cdot f''(x)-2f(x)\cdot f'''(x)=-2f(x)\cdot f'''(x);$\\
		Khi đó $\left(h(x)\right)'=\left(2x-2\right){g}'\left(x^2-2x\right)=-2\left(2x-2\right)\left(x^2-2x\right){\left(x^2-2x-1\right)^2}{\left(x^2-2x+4\right)^3}$\\
		$h'(x)=0\Leftrightarrow\hoac{
			& x=0\\ 
			& x=1\\ 
			& x=2\\ 
			& x=1\pm\sqrt{2}.}$ 
		Ta có bảng xét dấu của $h'(x)$
		\begin{center}
			\begin{tikzpicture}
				\tkzTabInit[lgt=1.2,espcl=2,nocadre]
				{$t$/0.7, $h'(x)$ /.7} % first column
				{$-\infty$, $1-\sqrt{2}$,$0$, $1$,$2$,$1+\sqrt{2}$, $+\infty$} % first row
				\tkzTabLine { ,+,0,-,0,+,0,-,0,+,0,- } % second row
				%				\tkzTabLine {,-,z,+,t,+,} % third row
				%				\tkzTabLine {,+,d,-,z,+,} % last row
			\end{tikzpicture}
		\end{center}
		Suy ra hàm số $h(x)=g\left(x^2-2x\right)$ đồng biến trên khoảng $\left(1;2\right)$.}
\end{ex}

\begin{ex}%[2D1G1-2]%Câu 19
	Cho hàm số $ y=f(x)$ xác định trên $\mathbb{R}$. Hàm số $ y=g(x)=f'\left(2x+3\right)+2$ có đồ thị là một parabol với tọa độ đỉnh $ I\left(2;-1\right)$ và đi qua điểm $ A\left(1;2\right)$. Hỏi hàm số $ y=f(x)$ nghịch biến trên khoảng nào dưới đây?
	\choice
	{\True $\left(5;9\right)$}
	{$\left(1;2\right)$}
	{$\left(-\infty ;9\right)$}
	{$\left(1;3\right)$}
	\loigiai{	
		Xét hàm số $ g(x)=f'\left(2x+3\right)+2$ có đồ thị là một Parabol nên có phương trình dạng $ y=g(x)=a{x^2}+bx+c\,\,\,\,(P)$.\\
		Vì $(P)$ có đỉnh $ I\left(2;-1\right)$ nên $\heva{
			&\dfrac{-b}{2a}=2\\ 
			& g(2)=-1} \Leftrightarrow\heva{
			&-b=4a\\ 
			& 4a+2b+c=-1} \Leftrightarrow\heva{
			& 4a+b=0\\ 
			& 4a+2b+c=-1}$.\\
		Vì $(P)$ đi qua điểm $ A\left(1;2\right)$ nên $ g(1)=2\Leftrightarrow a+b+c=2$.\\
		Ta có hệ phương trình $\heva{
			& 4a+b=0\\ 
			& 4a+2b+c=-1\\ 
			& a+b+c=2} \Leftrightarrow\heva{
			& a=3\\ 
			& b=-12\\ 
			& c=11}$ nên $ g(x)=3x^2-12x+11$.\\
		Đồ thị của hàm $ y=g(x)$ là
		\begin{center}
			\begin{tikzpicture}[scale=0.7,font=\footnotesize, line join=round, line cap=round, >=stealth] %Đường cong bậc 3
				\draw[thick, ->] (-0.5,0)--(3.5,0);
				\draw[thick, ->] (0,-1.8)--(0,5.3);
				\draw (3.7,0) node[below] {$x$};
				\draw (0,5.4) node[left]{$y$};
				\draw (0,0) node[below left]{$0$};
				\draw[fill] (3,0) circle (0.5pt)node[below]{$ 3$};
				\draw[fill] (1,0) circle (0.5pt)node[below]{$ 1$};
				\draw[fill] (2,0) circle (0.5pt)node[above]{$2$};
				\draw[fill] (0,2) circle (0.5pt)node[left]{$ 2$};
				\draw[fill] (0,-1) circle (0.5pt)node[left]{$ -1$};
				%		\draw[fill] (0,2) circle (0.5pt)node[above left]{$ 2$};
				%		\draw[fill] (0,-2) circle (0.5pt)node[below left]{$ -2$};
				\draw[dashed] (3,0)--(3,2)--(0,2)--(1,2)--(1,0) (3.2,2)--(3,2); 
				\draw[dashed](0,-1)--(2,-1)--(2,0);
				\draw[line width=1.2pt,smooth,samples=100,domain=0.6:3.4] plot(\x,{3*(\x)^2-12*(\x)+11});		
				%\draw[line width=1.2pt,smooth,samples=100,domain=-3.3:2.8] plot(\x,{0.75*(\x)^2+0.5*\x-1});
				%	\draw (2.0,2.8) node[left]{$y=f'(x)$};
			\end{tikzpicture}	
		\end{center}
		Theo đồ thị ta thấy $ f'(2x+3)\le 0\Leftrightarrow f'(2x+3)+2\le 2\Leftrightarrow 1\le x\le 3$.\\
		Đặt $ t=2x+3\Leftrightarrow x=\dfrac{t-3}{2}$ khi đó $ f'(t)\le 0\Leftrightarrow 1\le\dfrac{t-3}{2}\le 3\Leftrightarrow 5\le t\le 9$.\\
		Vậy $ y=f(x)$ nghịch biến trên khoảng $\left(5;9\right)$.}
\end{ex}

\begin{ex}%[2D1G1-2]%Câu 20
	\immini{
		Cho hàm số $ y=f(x)$, hàm số $f'(x)=x^3+a{x^2}+bx+c\left(a,b,c\in\mathbb{R}\right)$ có đồ thị như hình vẽ bên.
		Hàm số $ g(x)=f\left(f'(x)\right)$ nghịch biến trên khoảng nào dưới đây?
		\choice
		{$\left(1;+\infty\right)$}
		{\True $\left(-\infty ;-2\right)$}
		{$\left(-1;0\right)$}
		{$\left(-\dfrac{\sqrt{3}}{3};\dfrac{\sqrt{3}}{3}\right)$}}{
		\begin{tikzpicture}[scale=0.8,font=\footnotesize, line join=round, line cap=round, >=stealth] %Đường cong bậc 3
			\draw[thick, ->] (-1.7,0)--(1.7,0);
			\draw[thick, ->] (0,-2.7)--(0,3.0);
			\draw (1.9,0) node[below] {$x$};
			\draw (0,3.2) node[left]{$y$};
			\draw (0,0) node[below left]{$0$};
			\draw[fill] (-1,0) circle (0.5pt)node[above left]{$ -1 $};
			\draw[fill] (1,0) circle (0.5pt)node[below right]{$ 1$};
			\draw[line width=1.2pt,smooth,samples=100,domain=-1.3:1.3] plot(\x,{2.667*(\x)^3+0*(\x)^2-2.667*\x});		
			%\draw[line width=1.2pt,smooth,samples=100,domain=-3.3:2.8] plot(\x,{0.75*(\x)^2+0.5*\x-1});
		\end{tikzpicture}	
	}
	\loigiai{	
		Vì các điểm $\left(-1;0\right),\left(0;0\right),\left(1;0\right)$ thuộc đồ thị hàm số $ y=f'(x)$ nên ta có hệ\\
		$\heva{
			&-1+a-b+c=0\\ 
			& c=0\\ 
			& 1+a+b+c=0} \Leftrightarrow\heva{
			& a=0\\ 
			& b=-1\\ 
			& c=0} \Rightarrow {f}'(x)=x^3-x\Rightarrow f''(x)=3x^2-1$.\\
		Ta có $ g(x)=f\left(f'(x)\right)\Rightarrow{g}'(x)=f'\left(f'(x)\right)\cdot f''(x)$.\\
		Xét \\
		$g'(x)=0\Leftrightarrow{g}'(x)=f'\left(f'(x)\right)\cdot f''(x)=0$\\
		$\Leftrightarrow {f}'\left(x^3-x\right)\left(3x^2-1\right)=0\Leftrightarrow\hoac{
			&{x^3}-x=0\\ 
			&{x^3}-x=1\\ 
			&{x^3}-x=-1\\ 
			& 3x^2-1=0} \Leftrightarrow \hoac{
			& x=\pm 1\\ 
			& x=0\\ 
			& x=x_1(x_1\approx 1,325)\\ 
			& x=x_2(x_2\approx-1,325)\\ 
			& x=\pm\dfrac{\sqrt{3}}{3}.}$\\
		Bảng biến thiên
		\begin{center}
			\begin{tikzpicture}
				\tkzTabInit[lgt=1.2,espcl=2,nocadre]
				{$t$/0.7, $h'(x)$ /.7} % first column
				{$-\infty$, $-1{,}325$,$-1$, $-\dfrac{\sqrt{3}}{3}$,$0$,$\dfrac{\sqrt{3}}{3}$,$1$,$1{,}325$, $+\infty$} % first row
				\tkzTabLine { ,-,0,+,0,-,0,+,0,-,0,+,0,-,0,+, } % second row
				%				\tkzTabLine {,-,z,+,t,+,} % third row
				%				\tkzTabLine {,+,d,-,z,+,} % last row
			\end{tikzpicture}
		\end{center}
		Dựa vào bảng biến thiên ta có $ g(x)$ nghịch biến trên $\left(-\infty ;-2\right)$}
\end{ex}
\Closesolutionfile{ans}
\indapan{10}{ans/CD1/Muc_9_10}
\chapter{TIỆM CẬN CỦA ĐỒ THỊ HÀM SỐ}
\begin{Solution}{1}
C
\end{Solution}
\begin{Solution}{3}
B
\end{Solution}
\begin{Solution}{4}
A
\end{Solution}
\begin{Solution}{5}
A
\end{Solution}
\begin{Solution}{6}
A
\end{Solution}
\begin{Solution}{7}
B
\end{Solution}
\begin{Solution}{8}
A
\end{Solution}
\begin{Solution}{9}
C
\end{Solution}
\begin{Solution}{10}
B
\end{Solution}
\begin{Solution}{11}
C
\end{Solution}
\begin{Solution}{12}
D
\end{Solution}
\begin{Solution}{13}
B
\end{Solution}
\begin{Solution}{14}
D
\end{Solution}
\begin{Solution}{15}
A
\end{Solution}
\begin{Solution}{16}
B
\end{Solution}
\begin{Solution}{17}
C
\end{Solution}
\begin{Solution}{18}
C
\end{Solution}
\begin{Solution}{19}
C
\end{Solution}
\begin{Solution}{20}
B
\end{Solution}
\begin{Solution}{21}
C
\end{Solution}
\begin{Solution}{22}
B
\end{Solution}
\begin{Solution}{23}
D
\end{Solution}
\begin{Solution}{24}
B
\end{Solution}
\begin{Solution}{25}
D
\end{Solution}
\begin{Solution}{26}
D
\end{Solution}
\begin{Solution}{27}
B
\end{Solution}
\begin{Solution}{28}
A
\end{Solution}
\begin{Solution}{29}
C
\end{Solution}
\begin{Solution}{30}
B
\end{Solution}
\begin{Solution}{31}
D
\end{Solution}
\begin{Solution}{32}
B
\end{Solution}
\begin{Solution}{33}
B
\end{Solution}
\begin{Solution}{34}
C
\end{Solution}
\begin{Solution}{35}
D
\end{Solution}
\begin{Solution}{36}
B
\end{Solution}
\begin{Solution}{37}
B
\end{Solution}
\begin{Solution}{38}
A
\end{Solution}
\begin{Solution}{39}
A
\end{Solution}
\begin{Solution}{40}
D
\end{Solution}
\begin{Solution}{41}
C
\end{Solution}
\begin{Solution}{42}
B
\end{Solution}
\begin{Solution}{43}
A
\end{Solution}
\begin{Solution}{44}
A
\end{Solution}
\begin{Solution}{45}
D
\end{Solution}
\begin{Solution}{46}
C
\end{Solution}
\begin{Solution}{47}
A
\end{Solution}
\begin{Solution}{48}
B
\end{Solution}
\begin{Solution}{49}
B
\end{Solution}
\begin{Solution}{50}
B
\end{Solution}
\begin{Solution}{51}
A
\end{Solution}
\begin{Solution}{52}
A
\end{Solution}
\begin{Solution}{53}
C
\end{Solution}
\begin{Solution}{54}
C
\end{Solution}
\begin{Solution}{55}
C
\end{Solution}
\begin{Solution}{56}
B
\end{Solution}
\begin{Solution}{57}
C
\end{Solution}
\begin{Solution}{58}
C
\end{Solution}
\begin{Solution}{59}
B
\end{Solution}
\begin{Solution}{60}
C
\end{Solution}
\begin{Solution}{61}
A
\end{Solution}
\begin{Solution}{62}
B
\end{Solution}
\begin{Solution}{63}
B
\end{Solution}
\begin{Solution}{64}
D
\end{Solution}
\begin{Solution}{65}
D
\end{Solution}
\begin{Solution}{66}
B
\end{Solution}
\begin{Solution}{67}
A
\end{Solution}
\begin{Solution}{68}
D
\end{Solution}

\begin{Solution}{1}
D
\end{Solution}
\begin{Solution}{2}
C
\end{Solution}
\begin{Solution}{3}
C
\end{Solution}
\begin{Solution}{4}
A
\end{Solution}
\begin{Solution}{5}
B
\end{Solution}
\begin{Solution}{6}
D
\end{Solution}
\begin{Solution}{7}
C
\end{Solution}
\begin{Solution}{8}
D
\end{Solution}
\begin{Solution}{9}
A
\end{Solution}
\begin{Solution}{10}
B
\end{Solution}
\begin{Solution}{11}
D
\end{Solution}
\begin{Solution}{12}
A
\end{Solution}
\begin{Solution}{13}
D
\end{Solution}
\begin{Solution}{14}
B
\end{Solution}
\begin{Solution}{15}
B
\end{Solution}
\begin{Solution}{16}
C
\end{Solution}
\begin{Solution}{1}
A
\end{Solution}
\begin{Solution}{2}
B
\end{Solution}
\begin{Solution}{3}
D
\end{Solution}
\begin{Solution}{4}
D
\end{Solution}
\begin{Solution}{5}
C
\end{Solution}
\begin{Solution}{6}
A
\end{Solution}
\begin{Solution}{7}
D
\end{Solution}
\begin{Solution}{8}
B
\end{Solution}
\begin{Solution}{9}
C
\end{Solution}
\begin{Solution}{10}
C
\end{Solution}
\begin{Solution}{1}
D
\end{Solution}
\begin{Solution}{2}
D
\end{Solution}
\begin{Solution}{3}
B
\end{Solution}
\begin{Solution}{4}
C
\end{Solution}
\begin{Solution}{5}
D
\end{Solution}
\begin{Solution}{6}
A
\end{Solution}
\begin{Solution}{7}
C
\end{Solution}
\begin{Solution}{8}
B
\end{Solution}
\begin{Solution}{9}
A
\end{Solution}
\begin{Solution}{10}
C
\end{Solution}
\begin{Solution}{11}
D
\end{Solution}
\begin{Solution}{12}
C
\end{Solution}
\begin{Solution}{13}
A
\end{Solution}
\begin{Solution}{14}
D
\end{Solution}
\begin{Solution}{15}
A
\end{Solution}
\begin{Solution}{16}
A
\end{Solution}
\begin{Solution}{17}
B
\end{Solution}
\begin{Solution}{18}
C
\end{Solution}
\begin{Solution}{19}
C
\end{Solution}
\begin{Solution}{20}
A
\end{Solution}
\begin{Solution}{21}
D
\end{Solution}
\begin{Solution}{22}
C
\end{Solution}
\begin{Solution}{23}
A
\end{Solution}
\begin{Solution}{24}
C
\end{Solution}
\begin{Solution}{25}
A
\end{Solution}
\begin{Solution}{26}
B
\end{Solution}
\begin{Solution}{27}
B
\end{Solution}
\begin{Solution}{28}
D
\end{Solution}
\begin{Solution}{29}
B
\end{Solution}
\begin{Solution}{30}
D
\end{Solution}
\begin{Solution}{31}
D
\end{Solution}
\begin{Solution}{32}
C
\end{Solution}
\begin{Solution}{33}
D
\end{Solution}
\begin{Solution}{34}
C
\end{Solution}
\begin{Solution}{35}
D
\end{Solution}
\begin{Solution}{36}
D
\end{Solution}
\begin{Solution}{37}
D
\end{Solution}
\begin{Solution}{38}
D
\end{Solution}
\begin{Solution}{39}
D
\end{Solution}
\begin{Solution}{40}
C
\end{Solution}
\begin{Solution}{41}
A
\end{Solution}
\begin{Solution}{1}
A
\end{Solution}
\begin{Solution}{2}
B
\end{Solution}
\begin{Solution}{3}
C
\end{Solution}
\begin{Solution}{4}
A
\end{Solution}
\begin{Solution}{5}
A
\end{Solution}
\begin{Solution}{6}
C
\end{Solution}
\begin{Solution}{7}
C
\end{Solution}
\begin{Solution}{8}
B
\end{Solution}
\begin{Solution}{9}
C
\end{Solution}
\begin{Solution}{10}
B
\end{Solution}
\begin{Solution}{11}
A
\end{Solution}
\begin{Solution}{12}
B
\end{Solution}
\begin{Solution}{13}
B
\end{Solution}
\begin{Solution}{14}
B
\end{Solution}
\begin{Solution}{15}
A
\end{Solution}
\begin{Solution}{16}
B
\end{Solution}
\begin{Solution}{17}
A
\end{Solution}
\begin{Solution}{18}
D
\end{Solution}
\begin{Solution}{19}
C
\end{Solution}
\begin{Solution}{20}
C
\end{Solution}
\begin{Solution}{21}
A
\end{Solution}
\begin{Solution}{22}
C
\end{Solution}
\begin{Solution}{23}
C
\end{Solution}
\begin{Solution}{24}
A
\end{Solution}
\begin{Solution}{25}
B
\end{Solution}
\begin{Solution}{26}
B
\end{Solution}
\begin{Solution}{27}
A
\end{Solution}
\begin{Solution}{28}
A
\end{Solution}
\begin{Solution}{29}
C
\end{Solution}
\begin{Solution}{30}
B
\end{Solution}
\begin{Solution}{31}
A
\end{Solution}
\begin{Solution}{32}
C
\end{Solution}
\begin{Solution}{33}
B
\end{Solution}
\begin{Solution}{34}
A
\end{Solution}
\begin{Solution}{35}
B
\end{Solution}
\begin{Solution}{36}
B
\end{Solution}
\begin{Solution}{37}
B
\end{Solution}
\begin{Solution}{38}
D
\end{Solution}
\begin{Solution}{39}
B
\end{Solution}
\begin{Solution}{40}
A
\end{Solution}
\begin{Solution}{41}
D
\end{Solution}
\begin{Solution}{42}
D
\end{Solution}
\begin{Solution}{43}
A
\end{Solution}
\begin{Solution}{44}
D
\end{Solution}
\begin{Solution}{45}
C
\end{Solution}
\begin{Solution}{46}
B
\end{Solution}
\begin{Solution}{47}
A
\end{Solution}
\begin{Solution}{48}
D
\end{Solution}
\begin{Solution}{49}
B
\end{Solution}
\begin{Solution}{50}
B
\end{Solution}
\begin{Solution}{51}
D
\end{Solution}
\begin{Solution}{52}
C
\end{Solution}
\begin{Solution}{53}
C
\end{Solution}
\begin{Solution}{54}
B
\end{Solution}
\begin{Solution}{55}
D
\end{Solution}
\begin{Solution}{56}
B
\end{Solution}
\begin{Solution}{57}
C
\end{Solution}
\begin{Solution}{58}
A
\end{Solution}
\begin{Solution}{59}
A
\end{Solution}
\begin{Solution}{60}
B
\end{Solution}
\begin{Solution}{61}
D
\end{Solution}
\begin{Solution}{62}
D
\end{Solution}
\begin{Solution}{63}
B
\end{Solution}
\begin{Solution}{64}
A
\end{Solution}
\begin{Solution}{65}
D
\end{Solution}
\begin{Solution}{66}
C
\end{Solution}
\begin{Solution}{67}
A
\end{Solution}
\begin{Solution}{68}
A
\end{Solution}
\begin{Solution}{69}
D
\end{Solution}
\begin{Solution}{70}
C
\end{Solution}
\begin{Solution}{71}
B
\end{Solution}
\begin{Solution}{72}
A
\end{Solution}
\begin{Solution}{73}
C
\end{Solution}
\begin{Solution}{74}
C
\end{Solution}
\begin{Solution}{75}
C
\end{Solution}
\begin{Solution}{76}
A
\end{Solution}
\begin{Solution}{77}
C
\end{Solution}
\begin{Solution}{78}
B
\end{Solution}
\begin{Solution}{79}
D
\end{Solution}
\begin{Solution}{80}
B
\end{Solution}

\section{Mức 9,10 điểm}
\setcounter{ex}{0}
\setcounter{dang}{0}
\Opensolutionfile{ans}[ans/CD1/Muc_9_10]
\begin{dang}{Tìm m để hàm số đơn điệu trên các khoảng xác định của nó}
	Đang thiếu bài thầy Jf Câu 1 đến 26 
\end{dang}
\begin{dang}
	{Tìm khoảng đơn điệu của hàm số $g(x) = f\left[ u(x)\right] +v(x)$ khi biết đồ thị hoặc bảng biến thiên của hàm số $y = f'(x)$}
\end{dang}
\begin{ex}[Đề tham khảo 2019]%[2D1K1-2]
	Cho hàm số $f(x)$ có bảng xét dấu của đạo hàm như sau
	\begin{center}
		\begin{tikzpicture}
			\tkzTabInit[nocadre,lgt=1.2,espcl=2,deltacl=0.6]
			{$x$ /0.6,$f'(x)$ /0.6}
			{$-\infty$,$1$,$2$,$3$,$4$,$+\infty$}
			\tkzTabLine{,-,$0$,+,$0$,+,$0$,-,$0$,+,}
		\end{tikzpicture}
	\end{center}
	Hàm số $y=3 f(x+2)-x^3+3 x$ đồng biến trên khoảng nào dưới đây?
	\choice
	{$(-\infty ;-1)$}
	{\True $(-1 ; 0)$}
	{$(0 ; 2)$}
	{$(1 ;+\infty)$}
	\loigiai{
		Ta có $y'=3\left[f'(x+2)-\left(x^2-3\right)\right]$.\\
		Với $x \in(-1 ; 0) \Rightarrow x+2 \in(1 ; 2) \Rightarrow f'(x+2)>0$, lại có $x^2-3<0 \Rightarrow y'>0 ;~ \forall x \in(-1 ; 0)$.\\
		Vậy hàm số $y=3 f(x+2)-x^3+3 x$ đồng biến trên khoảng $(-1 ; 0)$.\\
		Chú ý:\\
		+) Ta xét $x \in(1 ; 2) \subset(1 ;+\infty)
		\Rightarrow x+2 \in(3 ; 4)\\
		\Rightarrow f'(x+2)<0 ;~ x^2-3>0$\\
		Suy ra hàm số nghịch biến trên khoảng $(1 ; 2)$ nên loại hai phương án$(0 ; 2)$ và $(1 ;+\infty)$.\\
		+) Tương tự ta xét
		$x \in(-\infty ;-2) \Rightarrow x+2 \in(-\infty ; 0)\\
		\Rightarrow f'(x+2)<0 ; x^2-3>0 \Rightarrow y'<0 ; ~ \forall x \in(-\infty ;-2)$.\\
		Suy ra hàm số nghịch biến trên khoảng $(-\infty ;-2)$ nên loại$(-\infty ;-1)$.\\
		Vậy hàm số đã cho đồng biến trên khoảng $(-1 ; 0)$.
	}
\end{ex}
\begin{ex}[Đề Tham Khảo 2020 - Lần 1]%[2D1G1-2]
	\immini{
		Cho hàm số $f(x)$. Hàm số $y=f'(x)$ có đồ thị như hình bên. Hàm số $g(x)=f(1-2 x)+x^2-x$ nghịch biến trên khoảng nào dưới đây?
		\choice
		{\True $\left(1 ; \dfrac{3}{2}\right)$}
		{$\left(0 ; \dfrac{1}{2}\right)$}
		{$(-2 ;-1)$}
		{$(2 ; 3)$}
	}
	{
		\begin{tikzpicture}[scale=0.7,>=stealth, font=\footnotesize, line join=round, line cap=round]
			%\def\a{1} \def\b{-6} \def\c{9} \def\d{1} % Hệ số
			\def\xmin{-4} \def\xmax{6}
			\def\ymin{-3} \def\ymax{2} 
			%\draw[color=gray!50,dashed] (\xmin,\ymin) grid (\xmax,\ymax); 
			\draw[->] (\xmin,0)--(\xmax,0) node [below]{$x$};
			\draw[->] (0,\ymin)--(0,\ymax) node [left]{$y$};
			\node at (0,0) [below left]{$O$};
			%\node at (1,3) [below left]{$f'(x)$};
			%\node at (-1.3,4) {$f'(x)$};
			\draw[dashed] (-2,0) node[below]{$-2$}--(-2,1)--(0,1) node[below left]{$1$};
			\draw[dashed] (4,0) node[below left]{$4$}--(4,-2)--(0,-2) node[below left]{$-2$};
			%\draw[dashed] (1,0) node[below]{$1$}--(1,1);
			%\draw[dashed] (-0.5,0) node[below left]{$-0{,}5$}--(-0.5,2.125);
			\clip (\xmin+0.1,\ymin+0.1) rectangle (\xmax-0.5,\ymax-0.1);
			\draw[smooth,samples=300][domain=-4:5.5] plot(\x,{0.071*(\x)^3-0.142*(\x)^2-1.07*(\x)});
		\end{tikzpicture}
	}
	
	\loigiai{
		Ta có : $g(x)=f(1-2 x)+x^2-x \Rightarrow g'(x)=-2 f'(1-2 x)+2 x-1$.\\
		\immini{
			Đặt $t=1-2 x \Rightarrow g'(x)=-2 f'(t)-t$.\\
			$g'(x)=0 \Rightarrow f'(t)=-\dfrac{t}{2}$.\\
			Vẽ đường thẳng $y=-\dfrac{x}{2}$ và đồ thị hàm số $f'(x)$ trên cùng một hệ trục
		}	
		{
			\begin{tikzpicture}[scale=0.7,>=stealth, font=\footnotesize, line join=round, line cap=round]
				%\def\a{1} \def\b{-6} \def\c{9} \def\d{1} % Hệ số
				\def\xmin{-4} \def\xmax{6}
				\def\ymin{-3} \def\ymax{2} 
				%	\draw[color=gray!50,dashed] (\xmin,\ymin) grid (\xmax,\ymax); 
				\draw[->] (\xmin,0)--(\xmax,0) node [below]{$x$};
				\draw[->] (0,\ymin)--(0,\ymax) node [left]{$y$};
				\node at (0,0) [below left]{$O$};
				%\node at (1,3) [below left]{$f'(x)$};
				%\node at (-1.3,4) {$f'(x)$};
				\draw[dashed] (-2,0) node[below]{$-2$}--(-2,1)--(0,1) node[below left]{$1$};
				\draw[dashed] (4,0) node[below]{$4$}--(4,-2)--(0,-2) node[below left]{$-2$};
				%\draw[dashed] (1,0) node[below]{$1$}--(1,1);
				%\draw[dashed] (-0.5,0) node[below left]{$-0{,}5$}--(-0.5,2.125);
				\clip (\xmin+0.1,\ymin+0.1) rectangle (\xmax-0.5,\ymax-0.1);
				\draw[smooth,samples=300][domain=-4:5.5] plot(\x,{0.071*(\x)^3-0.142*(\x)^2-1.07*(\x)});
				\draw[smooth,samples=300][domain=-4:5.5] plot(\x,{(-0.5*(\x)});
			\end{tikzpicture}
		}	Hàm số $g(x)$ nghịch biến $\Rightarrow g'(x) \leq 0 \Rightarrow f'(t) \geq-\dfrac{t}{2}\Rightarrow\hoac{&-2 \leq t \leq 0 \\&t \geq 4.}$\\
		Như vậy $f'(1-2 x) \geq \dfrac{1-2 x}{-2}\Rightarrow\hoac{&-2 \leq 1-2 x \leq 0 \\ &4 \leq 1-2 x}\Rightarrow\hoac{&\dfrac{1}{2}\leq x \leq \dfrac{3}{2}\\ &x \leq-\dfrac{3}{2}.}$\\
		Vậy hàm số $g(x)=f(1-2 x)+x^2-x$ nghịch biến trên các khoảng $\left(\dfrac{1}{2}; \dfrac{3}{2}\right)$ và $\left(-\infty ;-\dfrac{3}{2}\right)$.\\
		Mà $\left(1 ; \dfrac{3}{2}\right) \subset \left(\dfrac{1}{2}; \dfrac{3}{2}\right)$ nên hàm số $g(x)=f(1-2 x)+x^2-x$ nghịch biến trên khoảng $\left(1 ; \dfrac{3}{2}\right)$.
	}
\end{ex}
\begin{ex}[Chuyên Lê Quý Đôn Điện Biên 2019]%[2D1G1-2]
	Cho hàm số $f(x)$ có bảng xét dấu của đạo hàm như sau
	\begin{center}
		\begin{tikzpicture}
			\tkzTabInit[nocadre,lgt=1.2,espcl=2,deltacl=0.6]
			{$x$ /0.6,$f'(x)$ /0.6}
			{$-\infty$,$0$,$1$,$2$,$3$,$+\infty$}
			\tkzTabLine{,+,$0$,-,$0$,-,$0$,+,$0$,-,}
		\end{tikzpicture}
	\end{center}
	Hàm số $y=f(x-1)+x^3-12 x+2019$ nghịch biến trên khoảng nào dưới đây?
	\choice
	{$(1 ;+\infty)$}
	{\True $(1 ; 2)$}
	{$(-\infty ; 1)$}
	{$(3 ; 4)$}
	\loigiai{
		$y'=f'(x-1)+3 x^2-12=f'(t)+3 t^2+6 t-9=f'(t)-\left(-3 t^2-6 t+9\right)$, với $t=x-1$.\\
		\immini{
			Nghiệm của phương trình $y'=0$ là hoành độ giao điểm của các đồ thị hàm số $y=f'(t)$ và $y=-3 t^2-6 t+9$.\\
			Vẽ đồ thị hàm số $y=f'(t)$ và $y=-3 t^2-6 t+9$ trên cùng một hệ trục tọa độ như hình vẽ bên.
		}	
		{		\begin{tikzpicture}[scale=0.5,>=stealth, font=\footnotesize, line join=round, line cap=round]
				\def\a{-3} \def\b{-6} \def\c{9} % Hệ số
				\def\xmin{-9} \def\xmax{7}
				\def\ymin{-3} \def\ymax{13}
				
				%\draw[color=gray!50,dashed] (\xmin,\ymin) grid (\xmax,\ymax);
				
				\draw[->] (\xmin,0)--(\xmax,0) node [below]{$x$};
				\draw[->] (0,\ymin)--(0,\ymax) node [left]{$y$};
				\node at (0,0) [below left]{$O$};
				\clip (\xmin+0.1,\ymin+0.1) rectangle (\xmax-0.5,\ymax-0.1);
				\draw[smooth,samples=300] plot(\x,{\a*(\x)^2+\b*(\x)+\c});
				\node at (1,0) [above right]{$1$};
				\node at (2,0) [below right]{$2$};
				\node at (3,0) [below right]{$3$};
				\node at (-3,-2) [left]{$y=-3t^2-6t+9$};
				\node at (4,0) [below right]{$f'(x)$};
				\draw (-2.2,10).. controls (-1,1.9) and (-0.5,0.8) .. (0,0);
				%\draw (-2,0).. controls (-1.5,-2) and (-0.5,-0) .. (0,0);
				\draw (0,0).. controls (0.4,-0.6) and (0.6,-0.6) .. (0.8,-0.2);
				\draw (0.8,-0.2).. controls (1,0.25) and (1.1,-0.1) .. (1.4,-0.8);
				\draw (1.4,-0.8).. controls (1.6,-1.1) and (1.7,-0.9) .. (2,0);
				\draw (2,0).. controls (2.4,1.1) and (2.6,1.1) .. (3.5,-1);
			\end{tikzpicture}
		}
		Dựa vào đồ thị trên, ta có bảng xét dấu của hàm số $y'=f'(t)-\left(-3 t^2-6 t+9\right)$ như sau $
		\left(t_0<-1\right)$
		\begin{center}
			\begin{tikzpicture}
				\tkzTabInit[nocadre,lgt=2,espcl=2,deltacl=0.6]
				{$x$ /0.6,$y'$ /0.6}
				{$-\infty$,$t_0$,$1$,$+\infty$}
				\tkzTabLine{,+,$0$,-,$0$,+,}
			\end{tikzpicture}
		\end{center}
		Hàm số nghịch biến trên khoảng $t \in\left(t_0 ; 1\right)$.\\
		Do đó hàm số nghịch biến trên khoảng $x \in(1 ; 2) \subset \left(t_0+1 ; 1\right)$.
	}
\end{ex}


\begin{ex}[Chuyên Phan Bội Châu Nghệ An 2019]%[2D1G1-2]
	Cho hàm số $f(x)$ có bảng xét dấu đạo hàm như sau:
	\begin{center}
		\begin{tikzpicture}
			\tkzTabInit[nocadre,lgt=2,espcl=2,deltacl=0.6]
			{$x$ /0.6,$f'(x)$ /0.6}
			{$-\infty$,$1$,$2$,$3$,$4$,$+\infty$}
			\tkzTabLine{,-,$0$,+,$0$,+,$0$,-,$0$,+,}
		\end{tikzpicture}
	\end{center}
	Hàm số $y=2 f(1-x)+\sqrt{x^2+1}-x$ nghịch biến trên những khoảng nào dưới đây
	\choice
	{$(-\infty ;-2)$}
	{$(-\infty ; 1)$}
	{\True $(-2 ; 0)$}
	{$(-3 ;-2)$}
	\loigiai{
		$y'=-2 f'(1-x)+\dfrac{x}{\sqrt{x^2+1}}-1$. \\
		Có $\dfrac{x}{\sqrt{x^2+1}}-1<0,~ \forall x \in(-2 ; 0)$.\\
		Bảng xét dấu:
		\begin{center}
			\begin{tikzpicture}
				\tkzTabInit[nocadre,lgt=2,espcl=2,deltacl=0.6]
				{$x$ /0.7,$f'(1-x)$ /0.7}
				{$-\infty$,$-3$,$-2$,$-1$,$0$,$+\infty$}
				\tkzTabLine{,+,$0$,-,$0$,+,$0$,+,$0$,-,}
			\end{tikzpicture}
		\end{center}
		$\Rightarrow-2 f'(1-x)<0, ~ \forall x \in(-2 ; 0) \\
		\Rightarrow-2 f'(1-x)+\dfrac{x}{\sqrt{x^2+1}}-1<0, ~\forall x \in(-2 ; 0)$.
	}
\end{ex}
\begin{ex}[Sở Vĩnh Phúc 2019]%[2D1G1-2]
	\immini{
		Cho hàm số bậc bốn $y=f(x)$ có đồ thị của hàm số $y=f'(x)$ như hình vẽ bên.\\
		Hàm số $y=3 f(x)+x^3-6 x^2+9 x$ đồng biến trên khoảng nào trong các khoảng sau đây?
		\choice
		{$(0 ; 2)$}
		{$(-1 ; 1)$}
		{$(1 ;+\infty)$}
		{\True $(-2 ; 0)$}
	}
	{
		\begin{tikzpicture}[scale=0.7,>=stealth, font=\footnotesize, line join=round, line cap=round]
			\def\a{0.21} \def\b{0.88} \def\c{-0.58} \def\d{-3} % Hệ số
			\def\xmin{-5} \def\xmax{5}
			\def\ymin{-4} \def\ymax{3} 
			%\draw[color=gray!50,dashed] (\xmin,\ymin) grid (\xmax,\ymax); 
			\draw[->] (\xmin,0)--(\xmax,0) node [below]{$x$};
			\draw[->] (0,\ymin)--(0,\ymax) node [left]{$y$};
			\node at (0,0) [above left]{$O$};
			\node at (-4,0) [below left]{$-4$};
			\node at (-2,0) [below left]{$-2$};
			\node at (0,-3) [below right]{$-3$};
			\draw[dashed] (2,0) node[above right]{$2$}--(2,1) --(0,1) node[above right]{$1$};
			\clip (\xmin+0.1,\ymin+0.1) rectangle (\xmax-0.5,\ymax-0.1);
			\draw[smooth,samples=300] plot(\x,{\a*(\x)^3+\b*(\x)^2+\c*(\x)+\d});
		\end{tikzpicture}
	}
	
	\loigiai{
		Hàm số $f(x)=a x^4+b x^3+c x^2+d x+e,(a \neq 0)$.
		Có $f'(x)=4 a x^3+3 b x^2+2 c x+d$.\\
		Đồ thị hàm số $y=f'(x)$ đi qua các điểm $(-4 ; 0),(-2 ; 0),(0 ;-3),(2 ; 1)$ nên ta có
		$$\heva{&- 2 5 6 a + 4 8 b - 8 c + d = 0\\
			&- 3 2 a + 1 2 b - 4 c + d = 0\\
			&d = - 3\\
			&3 2 a + 1 2 b + 4 c + d = 1}\Leftrightarrow \heva{&
			a=\dfrac{5}{96}\\
			&b=\dfrac{7}{24}\\
			&c=-\dfrac{7}{24}\\
			&d=-3.}
		$$
		Xét hàm số
		$
		y=3 f(x)+x^3-6 x^2+9 x$\\
		Ta có $ y'=3\left(f'(x)+x^2-4 x+3\right)=3\left(\frac{5}{24}x^3+\frac{15}{8}x^2-\frac{55}{12}x\right)
		$\\
		Ta có $y'=0 \Leftrightarrow\hoac{&x=-11 \\&x=0 \\&x=2.}$ \\
		Xét dấu $y'$, ta được hàm số đã cho đồng biến trên các khoảng $(-11 ; 0)$ và $(2 ;+\infty)$.
	}
\end{ex}
\begin{ex}[Học Mãi 2019]%[2D1K1-2]
	\immini
	{Cho hàm số $y=f(x)$ có đạo hàm trên $\mathbb{R}$. Đồ thị hàm số $y=f'(x)$ như hình bên. Hỏi đồ thị hàm số $y=f(x)-2 x$ có bao nhiêu điểm cực trị?
		\choice
		{$4$}
		{\True $3$}
		{$2$}
		{$1$}
	}
	{
		\begin{tikzpicture}[font=\footnotesize,line join=round, line cap=round,>=stealth,scale=0.8]
			\draw[->] (-3.5,0)--(4,0) node[above] {$x$};
			\draw[->] (0,-3)--(0,4) node[left] {$y$};
			%\fill[black] (-2,0)node[below left]{$-2$} circle (1.2pt) (0,0)node[above right]{$O$} circle (1.2pt) (3,0)node[above]{$3$} circle (1.2pt);
			\draw[dashed] (-2,-2)-- (0,-2) node[right]{$-2$};
			\draw[dashed] (2,0) node[below]{$2$}-- (2,2)--(0,2) node[below left]{$2$};
			\node at (0,0) [below left]{$O$};
			\node at (3,0) [below right]{$3$};
			\draw (-3,2.5).. controls (-2.2,-3) and (-1.8,-3) .. (-1.1,0);
			\draw (-1.1,0).. controls (-0.6,2.5) and (-0.4,2.5) .. (0,2);
			\draw (0,2).. controls (0.7,0.5) and (1.1,0.5) .. (1.5,1.5);
			\draw (1.5,1.5).. controls (2,2.5) and (2.8,2.5) .. (3.5,-2.5);
			%\draw (3,0).. controls (3.3,-0.1) and (3.5,-0.5) .. (3.5,-2);
		\end{tikzpicture}
	}
	\loigiai{
		\immini{
			Đặt $g(x)=f(x)-2 x$.\\
			$\Rightarrow g'(x)=f'(x)-2 .
			$\\
			Vẽ đường thẳng $y=2$.\\
			$\Rightarrow$ phương trình $g'(x)=0$ có $3$ nghiệm bội lẻ.\\
			$\Rightarrow$ đồ thị hàm số $y=f(x)-2 x$ có $3$ điểm cực trị.
		}
		{
			\begin{tikzpicture}[font=\footnotesize,line join=round, line cap=round,>=stealth,scale=0.8]
				\draw[->] (-3.5,0)--(4,0) node[above] {$x$};
				\draw[->] (0,-3)--(0,4) node[left] {$y$};
				%\fill[black] (-2,0)node[below left]{$-2$} circle (1.2pt) (0,0)node[above right]{$O$} circle (1.2pt) (3,0)node[above]{$3$} circle (1.2pt);
				\draw[dashed] (-2,-2)-- (0,-2) node[right]{$-2$};
				\draw[dashed] (2,0) node[below]{$2$}-- (2,2)--(0,2) node[below left]{$2$};
				\node at (3,0) [below left]{$3$};
				\draw (-3,2.5).. controls (-2.2,-3) and (-1.8,-3) .. (-1.1,0);
				\draw (-1.1,0).. controls (-0.6,2.5) and (-0.4,2.5) .. (0,2);
				\draw (0,2).. controls (0.7,0.5) and (1.1,0.5) .. (1.5,1.5);
				\draw (1.5,1.5).. controls (2,2.5) and (2.8,2.5) .. (3.5,-2.5);
				\draw (-3.5,2)--(4,2) node[above]{$y=2$};
			\end{tikzpicture}
		}
	}
\end{ex}
\begin{ex}[THPT Hoàng Hoa Thám Hưng Yên 2019]%[2D1G1-2]
	\immini{
		Cho hàm số $y=f(x)$ liên tục trên $\mathbb{R}$. Hàm số $y=f'(x)$ có đồ thị như hình vẽ. 
		Hàm số $g(x)=f(x-1)+\dfrac{2019-2018 x}{2018}$ đồng biến trên khoảng nào dưới đây?
		\choice
		{$(2 ; 3)$}
		{$(0 ; 1)$}
		{\True $(-1 ; 0)$}
		{$(1 ; 2)$}
	}
	{
		\begin{tikzpicture}[scale=1, font=\footnotesize, line join=round, line cap=round, >=stealth]
			\tikzset{label style/.style={font=\footnotesize}}
			\draw[->] (-2,0)--(3,0) node[below left] {$x$};
			\draw[->] (0,-2)--(0,3) node[below left] {$y$};
			\draw[fill=black] (0,0) node [above left] {$O$} circle(1pt);
			\fill (1,1) circle(1pt) (-1,1) circle(1pt) (2,1) circle(1pt);
			\foreach \x in {1,2}
			\draw[thin] (\x,1pt)--(\x,-1pt) node [below] {\footnotesize$\x$};
			\foreach \x in {-1}
			\draw[thin] (\x,1pt)--(\x,-1pt) node [below left] {\footnotesize$\x$};
			\foreach \y in {-1}
			\draw[thin] (1pt,\y)--(-1pt,\y) node [right] {\footnotesize$\y$};
			\foreach \y in {1}
			\draw[thin] (1pt,\y)--(-1pt,\y) node [above left] {\footnotesize$\y$};
			\draw[dashed](-1,0)--(-1,1)--(2,1) (1,1)--(1,0) (2,1)--(2,0);
			\begin{scope}
				\clip (-3,-3) rectangle (3,3);
				\draw[name path=(C)] plot[smooth,tension=0.7] coordinates{(-1.15,3)(-0.5,-1.6)(.8,.88)(1.9,0.8)(2.3,3)};
			\end{scope}
		\end{tikzpicture}
	}	\loigiai{
		Ta có $g'(x)=f'(x-1)-1$.\\
		$
		g'(x) \geq 0 \Leftrightarrow f'(x-1)-1 \geq 0 \Leftrightarrow f'(x-1) \geq 1 \Leftrightarrow \hoac{&x - 1 \leq - 1\\
			&x - 1 \geq 2}\Leftrightarrow \hoac{&
			x \leq 0 \\
			&x \geq 3.}
		$\\
		Từ đó suy ra hàm số $g(x)=f(x-1)+\dfrac{2019-2018 x}{2018}$ đồng biến trên khoảng $(-1 ; 0)$.
	}
\end{ex}

\begin{ex}[(Sở Ninh Bình 2019]%[2D1K1-2]
	Cho hàm số $y=f(x)$ có bảng xét dấu của đạo hàm như sau
	\begin{center}
		\begin{tikzpicture}
			\tkzTabInit[nocadre,lgt=1,espcl=2,deltacl=0.6]
			{$x$ /0.7,$f'(x)$ /0.7}
			{$-\infty$,$-2$,$-1$,$2$,$4$,$+\infty$}
			\tkzTabLine{,+,$0$,-,$0$,+,$0$,-,$0$,+,}
		\end{tikzpicture}
	\end{center}
	Hàm số $y=-2 f(x)+2019$ nghịch biến trên khoảng nào trong các khoảng dưới đây?
	\choice
	{$(-4 ; 2)$}
	{\True $(-1 ; 2)$}
	{$(-2 ;-1)$}
	{$(2 ; 4)$}
	\loigiai{
		Xét $y=g(x)=-2 f(x)+2019$.\\
		Ta có $g'(x)=(-2 f(x)+2019)'=-2 f'(x), g'(x)=0 \Leftrightarrow\hoac{&x=-2 \\&x=-1 \\&x=2 \\&x=4.}$.\\
		Ta có bảng xét dấu của $g'(x)$
		\begin{center}
			\begin{tikzpicture}
				\tkzTabInit[nocadre,lgt=1,espcl=2,deltacl=0.6]
				{$x$ /0.6,$f'(x)$ /0.6}
				{$-\infty$,$-2$,$-1$,$2$,$4$,$+\infty$}
				\tkzTabLine{,-,$0$,+,$0$,-,$0$,+,$0$,+,}
			\end{tikzpicture}
		\end{center}
		Dựa vào bảng xét dấu, ta thấy hàm số $y=g(x)$ nghịch biến trên khoảng $(-1 ; 2)$.
	}
\end{ex}
\begin{ex}[THPT Lương Thế Vinh Hà Nội 2019]%[2D1G1-2]
	\immini{
		Cho hàm số $y=f(x)$. Biết đồ thị hàm số $y=f'(x)$ có đồ thị như hình vẽ bên. 
		Hàm số $y=f \left(3-x^2\right)+2018$ đồng biến trên khoảng nào dưới đây?
		\choice
		{\True $(-1 ; 0)$}
		{$(2 ; 3)$}
		{$(-2 ;-1)$}
		{$(0 ; 1)$}
	}
	{
		\begin{tikzpicture}[scale=0.6,>=stealth, font=\footnotesize, line join=round, line cap=round]
			\def\a{0.065} \def\b{0.32} \def\c{-0.53} \def\d{-0.82} % Hệ số
			\def\xmin{-8} \def\xmax{4}
			\def\ymin{-3} \def\ymax{3} 
			%\draw[color=gray!50,dashed] (\xmin,\ymin) grid (\xmax,\ymax); 
			\draw[->] (\xmin,0)--(\xmax,0) node [below]{$x$};
			\draw[->] (0,\ymin)--(0,\ymax) node [left]{$y$};
			\node at (0,0) [below left]{$O$};
			\node at (-6,0) [below left]{$-6$};
			\node at (-1,0) [below left]{$-1$};
			\node at (2,0) [below right]{$2$};
			\clip (\xmin+0.1,\ymin+0.1) rectangle (\xmax-0.5,\ymax-0.1);
			\draw[smooth,samples=300][domain=-6.5:3.5] plot(\x,{\a*(\x)^3+\b*(\x)^2+\c*(\x)+\d});
		\end{tikzpicture}
	}
	
	\loigiai{
		Ta có $\left[f\left( 3-x^2\right)+2018 \right]'=-2 x \cdot f'\left(3-x^2\right) $.\\
		$
		-2 x \cdot f'\left(3-x^2\right)=0 \Leftrightarrow\hoac{&
			x = 0\\
			&3 - x ^{2}= - 6\\
			&3 - x ^{2}= - 1\\
			&3 - x ^{2}= 2}
		\Leftrightarrow \hoac{
			&x=0 \\
			&x=\pm 3 \\
			&x=\pm 2 \\
			&	x=\pm 1.}
		$\\
		Bảng xét dấu của đạo hàm hàm số đã cho
		\begin{center}
			\begin{center}
				\begin{tikzpicture}
					\tkzTabInit[nocadre,lgt=2.9,espcl=1.5,deltacl=0.6]
					{$x$ /0.7,$f'\left( 3-x^2\right) $/0.7,$-2xf'\left( 3-x^2\right)$/0.8}
					{$-\infty$,$-3$,$-2$,$-1$,$0$,$1$,$2$,$3$,$+\infty$}
					\tkzTabLine{,-,$0$,+,$0$,-,$0$,+,$0$,+,$0$,-,$0$,+,$0$,-}
					\tkzTabLine{,-,$0$,+,$0$,-,$0$,+,$0$,-,$0$,+,$0$,-,$0$,+}
				\end{tikzpicture}
			\end{center}
		\end{center}
		Từ bảng xét dấu suy ra hàm số đồng biến trên $(-1 ; 0)$.
	}
\end{ex}
\begin{ex}[Chuyên Biên Hòa - Hà Nam - 2020]%[2D1G1-2]
	\immini{
		Cho hàm số đa thức $f(x)$ có đạo hàm trên $\mathbb{R}$. Biết $f(0)=0$ và đồ thị hàm số $y=f'(x)$ như hình sau.
		Hàm số $g(x)=\left|4 f(x)+x^2\right|$ đồng biến trên khoảng nào dưới đây?
		\choice
		{$(4 ;+\infty)$}
		{\True $(0 ; 4)$}
		{$(-\infty ;-2)$}
		{$(-2 ; 0)$}
	}	
	{
		\begin{tikzpicture}[scale=0.7,>=stealth, font=\footnotesize, line join=round, line cap=round]
			%\def\a{1} \def\b{-6} \def\c{9} \def\d{1} % Hệ số
			\def\xmin{-4} \def\xmax{6}
			\def\ymin{-3} \def\ymax{2} 
			%\draw[color=gray!50,dashed] (\xmin,\ymin) grid (\xmax,\ymax); 
			\draw[->] (\xmin,0)--(\xmax,0) node [below]{$x$};
			\draw[->] (0,\ymin)--(0,\ymax) node [left]{$y$};
			\node at (0,0) [below left]{$O$};
			%\node at (1,3) [below left]{$f'(x)$};
			%\node at (-1.3,4) {$f'(x)$};
			\draw[dashed] (-2,0) node[below]{$-2$}--(-2,1)--(0,1) node[below left]{$1$};
			\draw[dashed] (4,0) node[below]{$4$}--(4,-2)--(0,-2) node[below left]{$-2$};
			%\draw[dashed] (1,0) node[below]{$1$}--(1,1);
			%\draw[dashed] (-0.5,0) node[below left]{$-0{,}5$}--(-0.5,2.125);
			\clip (\xmin+0.1,\ymin+0.1) rectangle (\xmax-0.5,\ymax-0.1);
			\draw[smooth,samples=300][domain=-4:5.5] plot(\x,{0.071*(\x)^3-0.142*(\x)^2-1.07*(\x)});
		\end{tikzpicture}
	}
	\loigiai{
		\immini{
			Xét hàm số $h(x)=4 f(x)+x^2$ trên $\mathbb{R}$.\\
			Vì $f(x)$ là hàm số đa thức nên $h(x)$ cũng là hàm số đa thức và $h(0)=4 f(0)=0$.\\
			Ta có $h'(x)=4 f'(x)+2 x$. Do đó $h'(x)=0 \Leftrightarrow f'(x)=-\dfrac{1}{2}x$.\\
		}
		{
			\begin{tikzpicture}[scale=0.7,>=stealth, font=\footnotesize, line join=round, line cap=round]
				%\def\a{1} \def\b{-6} \def\c{9} \def\d{1} % Hệ số
				\def\xmin{-4} \def\xmax{6}
				\def\ymin{-3} \def\ymax{2} 
				%\draw[color=gray!50,dashed] (\xmin,\ymin) grid (\xmax,\ymax); 
				\draw[->] (\xmin,0)--(\xmax,0) node [below]{$x$};
				\draw[->] (0,\ymin)--(0,\ymax) node [left]{$y$};
				\node at (0,0) [below left]{$O$};
				%\node at (1,3) [below left]{$f'(x)$};
				%\node at (-1.3,4) {$f'(x)$};
				\draw[dashed] (-2,0) node[below]{$-2$}--(-2,1)--(0,1) node[below left]{$1$};
				\draw[dashed] (4,0) node[below]{$4$}--(4,-2)--(0,-2) node[below left]{$-2$};
				%\draw[dashed] (1,0) node[below]{$1$}--(1,1);
				%\draw[dashed] (-0.5,0) node[below left]{$-0{,}5$}--(-0.5,2.125);
				\clip (\xmin+0.1,\ymin+0.1) rectangle (\xmax-0.5,\ymax-0.1);
				\draw[smooth,samples=300][domain=-4:5.5] plot(\x,{0.071*(\x)^3-0.142*(\x)^2-1.07*(\x)});
				\draw[smooth,samples=300][domain=-4:5.5] plot(\x,{-0.5*(\x)});
			\end{tikzpicture}
		}
		Dựa vào sự tương giao của đồ thị hàm số $y=f'(x)$ và đường thẳng $y=-\dfrac{1}{2}x$, ta có
		$
		h'(x)=0 \Leftrightarrow x \in\{-2 ; 0 ; 4\}.\\
		$
		Bảng biến thiên của hàm số $h(x)$ như sau:
		\begin{center}
			\begin{tikzpicture}
				\tkzTabInit[nocadre,lgt=1.2,espcl=2.5,deltacl=0.6]
				{$x$ /0.6,$y'$ /0.6,$y$ /2}
				{$-\infty$,$-2$,$0$,$4$,$+\infty$}
				\tkzTabLine{,-,$0$,+,$0$,-,$0$,+,}
				\tkzTabVar{+/$+\infty$, -/$y_1$,+/$0$,-/$y_3$,+/$+\infty$}
			\end{tikzpicture}
		\end{center}
		Từ đó suy ra bảng biến thiên của hàm số $g(x)=|h(x)|$.\\
		Dựa vào bảng biến thiên trên, ta thấy hàm số $g(x)$ đồng biến trên khoảng $(0 ; 4)$.
	}
\end{ex}
\begin{ex}[Chuyên Thái Bình - 2020]%[2D1G1-2]
	\immini{
		Cho hàm số $f(x)$ liên tục trên $\mathbb{R}$ có đồ thị hàm số $y=f'(x)$ cho như hình vẽ bên.\\
		Hàm số $g(x)=2 f(|x-1|)-x^2+2 x+2020$ đồng biến trên khoảng nào?
		\choice
		{\True $(0 ; 1)$}
		{$(-3 ; 1)$}
		{$(1 ; 3)$}
		{$(-2 ; 0)$}
	}
	{
		\begin{tikzpicture}[scale=0.7,>=stealth, font=\footnotesize, line join=round, line cap=round]
			%\def\a{1} \def\b{-6} \def\c{9} \def\d{1} % Hệ số
			\def\xmin{-4} \def\xmax{5}
			\def\ymin{-3} \def\ymax{5} 
			%\draw[color=gray!50,dashed] (\xmin,\ymin) grid (\xmax,\ymax); 
			\draw[->] (\xmin,0)--(\xmax,0) node [below]{$x$};
			\draw[->] (0,\ymin)--(0,\ymax) node [left]{$y$};
			\node at (0,0) [below left]{$O$};
			%\node at (1,3) [below left]{$f'(x)$};
			\node at (-1.3,4) {$f'(x)$};
			\draw[dashed] (-1,0) node[above]{$-1$}--(-1,-1)--(0,-1) node[below left]{$-1$};
			\draw[dashed] (1,0) node[below]{$1$}--(1,1)--(0,1) node[below left]{$1$};
			\draw[dashed] (3,0) node[below]{$3$}--(3,3)--(0,3) node[below left]{$3$};
			%\draw[dashed] (1,0) node[below]{$1$}--(1,1);
			%\draw[dashed] (-0.5,0) node[below left]{$-0{,}5$}--(-0.5,2.125);
			\clip (\xmin+0.1,\ymin+0.1) rectangle (\xmax-0.5,\ymax-0.1);
			\draw[smooth,samples=300][domain=-2:4] plot(\x,{-0.5*(\x)^3+1.5*(\x)^2+1.5*(\x)-1.5});
			%\draw[smooth,samples=300] plot(\x,{(\x)^3+(\x)^2-2*(\x)+1});
		\end{tikzpicture}
	}
	\loigiai{
		Ta có đường thẳng $y=x$ cắt đồ thị hàm số $y=f'(x)$ tại các điểm $x=-1 ; x=1 ; x=3$ như hình vẽ sau:
		\begin{center}
			\begin{tikzpicture}[scale=0.7,>=stealth, font=\footnotesize, line join=round, line cap=round]
				%\def\a{1} \def\b{-6} \def\c{9} \def\d{1} % Hệ số
				\def\xmin{-4} \def\xmax{5}
				\def\ymin{-3} \def\ymax{5} 
				%\draw[color=gray!50,dashed] (\xmin,\ymin) grid (\xmax,\ymax); 
				\draw[->] (\xmin,0)--(\xmax,0) node [below]{$x$};
				\draw[->] (0,\ymin)--(0,\ymax) node [left]{$y$};
				\node at (0,0) [below left]{$O$};
				%\node at (1,3) [below left]{$f'(x)$};
				\node at (-1.3,4) {$f'(x)$};
				\node at (4,3.2) {$y=x$};
				\draw[dashed] (-1,0) node[above]{$-1$}--(-1,-1)--(0,-1) node[below left]{$-1$};
				\draw[dashed] (1,0) node[below]{$1$}--(1,1)--(0,1) node[below left]{$1$};
				\draw[dashed] (3,0) node[below]{$3$}--(3,3)--(0,3) node[below left]{$3$};
				%\draw[dashed] (1,0) node[below]{$1$}--(1,1);
				%\draw[dashed] (-0.5,0) node[below left]{$-0{,}5$}--(-0.5,2.125);
				\clip (\xmin+0.1,\ymin+0.1) rectangle (\xmax-0.5,\ymax-0.1);
				\draw[smooth,samples=300][domain=-2:4] plot(\x,{-0.5*(\x)^3+1.5*(\x)^2+1.5*(\x)-1.5});
				\draw[smooth,samples=300] plot(\x,{(\x)});
			\end{tikzpicture}
		\end{center}
		Dựa vào đồ thị của hai hàm số trên ta có $f'(x)>x \Leftrightarrow\hoac{&x<-1 \\ &1<x<3}$ và
		$ f'(x)<x \Leftrightarrow\hoac{&
			-1<x<1 \\
			&x>3.}$\\
		+Trường hợp 1: $x-1<0 \Leftrightarrow x<1$.\\
		Khi đó $g(x)=2 f(1-x)-x^2+2 x+2020$.\\
		Ta có $g'(x)=-2 f'(1-x)+2(1-x)$.
		$$
		g'(x)>0 \Leftrightarrow-2 f'(1-x)+2(1-x)>0 \Leftrightarrow f'(1-x)<1-x \Leftrightarrow\hoac{
			&- 1 < 1 - x < 1\\
			&1 - x > 3} \Leftrightarrow \hoac{&
			0<x<2 \\
			&x<-2.}
		$$
		Kết hợp điều kiện, ta có $g'(x)>0 \Leftrightarrow\hoac{&0<x<1 \\ &x<-2.}$\\
		
		+ Trường hợp 2: $x-1>0 \Leftrightarrow x>1$.\\
		Khi đó ta có $g(x)=2 f(x-1)-x^2+2 x+2020$.\\
		$ g'(x)=2 f'(x-1)-2(x-1)$\\
		$g'(x)>0 \Leftrightarrow 2 f'(x-1)-2(x-1)>0 \Leftrightarrow f'(x-1)>x-1 \Leftrightarrow\hoac{&
			x - 1 < - 1\\
			&1 < x - 1 < 3}\Leftrightarrow \hoac{
			&x<0 \\
			&2<x<4.}$
		Kết hợp điều kiện ta có $g'(x)>0 \Leftrightarrow 2<x<4$.\\
		Vậy hàm số $g(x)=2 f(|x-1|)-x^2+2 x+2020$ đồng biến trên khoảng $(0 ; 1)$.
	}
\end{ex}

\begin{ex}[Chuyên Lào Cai - 2020]%[2D1G1-2]
	\immini{
		Cho hàm số $f'(x)$ có đồ thị như hình bên.\\
		Hàm số $g(x)=f(3 x+1)+9 x^3+\dfrac{9}{2}x^2$ đồng biến trên khoảng nào dưới đây?
		\choice
		{$(-1 ; 1)$}
		{$(-2 ; 0)$}
		{$(-\infty ; 0)$}
		{\True $(1 ;+\infty)$}
	}
	{\begin{tikzpicture}[line join=round, line cap=round,>=stealth,thick,scale=.8]
			\tikzset{label style/.style={font=\footnotesize}}
			\draw[->] (-2.1,0)--(5.1,0) node[below left] {$x$};
			\draw[->] (0,-3.1)--(0,4.1) node[below left] {$y$};
			\draw (0,0) node [below left] {$O$};
			\foreach \x in {1,2,3}
			\draw[thin] (\x,1pt)--(\x,-1pt) node [below] {$\x$};
			\draw[thin](-1,1pt)--(1,-1pt)node[above left]{$-1$};
			\foreach \y in {-2,2}
			\draw[thin] (1pt,\y)--(-1pt,\y) node [above right] {$\y$};
			%\begin{scope}
			\clip (-2,-3) rectangle (5,4);
			\draw[samples=200,domain=-2:4,smooth,variable=\x] plot (\x,{(\x)^3-3*(\x)^2+2});
			%\end{scope}
			\draw[dashed] (-1,0)--(-1,-2)--(0,-2);
			\draw[dashed] (3,0)--(3,2)--(0,2);
			%\begin{scope}[on background layer]\path[white]node{MDD-134};\end{scope}
		\end{tikzpicture}
	}
	\loigiai
	{
		\immini{Xét hàm số $g(x)=f(3 x+1)+9 x^3+\dfrac{9}{2}x^2 \\
			\Rightarrow g'(x)=3 f'(3 x+1)+27 x^2+9 x$.\\
			Hàm số đồng biến  $\Leftrightarrow g'(x)>0 \Leftrightarrow 3 f'(3 x+1)+27 x^2+9 x>0$
			\\
			$
			\Leftrightarrow f'(3 x+1)+3 x(3 x+1)>0 \qquad (*)
			$\\
			Đặt $t=3 x+1$, khi đó  $(*) \Leftrightarrow f'(t)+(t-1) t>0$\\ $\Leftrightarrow f'(t)>-t^2+t$.\\
			Vẽ parabol $y=-x^2+x$ và đồ thị hàm số $f'(x)$ trên cùng một hệ trục
		}
		{
			\begin{tikzpicture}[line join=round, line cap=round,>=stealth,thick,scale=.8]
				\tikzset{label style/.style={font=\footnotesize}}
				\draw[->] (-2.1,0)--(5.1,0) node[below left] {$x$};
				\draw[->] (0,-3.1)--(0,4.1) node[below left] {$y$};
				\draw (0,0) node [below left] {$O$};
				\foreach \x in {1,2,3}
				\draw[thin] (\x,1pt)--(\x,-1pt) node [below] {$\x$};
				\draw[thin](-1,1pt)--(1,-1pt);
				\foreach \y in {-2,2}
				\draw[thin] (1pt,\y)--(-1pt,\y) node [above right] {$\y$};
				%\begin{scope}
				\clip (-2,-3) rectangle (5,4);
				\draw[samples=200,domain=-2:4,smooth,variable=\x] plot (\x,{(\x)^3-3*(\x)^2+2});
				\draw[samples=200,domain=-2:4,smooth,variable=\x] plot (\x,{-(\x)^2+(\x)});
				%\end{scope}
				\draw[dashed] (-1,0) node[above left]{$-1$}--(-1,-2)--(0,-2);
				\draw[dashed] (3,0)--(3,2)--(0,2);
				%\begin{scope}[on background layer]\path[white]node{MDD-134};\end{scope}
			\end{tikzpicture}
		}
		Dựa vào đồ thị ta thấy
		$
		f'(t)>-t^2+t \Leftrightarrow\hoac{&- 1 < t < 1\\
			&t > 2}\Rightarrow \hoac{&
			- 1 < 3 x + 1 < 1\\
			&3 x + 1 > 2} \Leftrightarrow \hoac{&
			\dfrac{-2}{3}<x<0\\
			&x>\dfrac{1}{3}.}
		$}
\end{ex}
\begin{ex}[Sở Phú Thọ-2020]%[2D1G1-2]
	\immini{
		Cho hàm số $y=f(x)$ có đồ thị hàm số $y=f'(x)$ như hình vẽ.\\
		Hàm số $g(x)=f\left(\mathrm{e}^x-2\right)-2020$ nghịch biến trên khoảng nào dưới đây?
		\choice
		{\True $\left(-1 ; \dfrac{3}{2}\right)$}
		{$(-1 ; 2)$}
		{$(0 ;+\infty)$}
		{$\left(\dfrac{3}{2}; 2\right)$}
	}
	{
		\begin{tikzpicture}[scale=0.7,>=stealth, font=\footnotesize, line join=round, line cap=round]
			\def\a{1} \def\b{-3} \def\c{0} \def\d{0} % Hệ số
			\def\xmin{-2} \def\xmax{4}
			\def\ymin{-5} \def\ymax{2} 
			%\draw[color=gray!50,dashed] (\xmin,\ymin) grid (\xmax,\ymax); 
			\draw[->] (\xmin,0)--(\xmax,0) node [below]{$x$};
			\draw[->] (0,\ymin)--(0,\ymax) node [left]{$y$};
			\node at (0,0) [above left]{$O$};
			\node at (3,0) [below right]{$3$};
			\draw[dashed] (2,0) node[above]{$2$}--(2,-4) --(0,-4) node[left]{$-4$};
			\clip (\xmin+0.1,\ymin+0.1) rectangle (\xmax-0.5,\ymax-0.1);
			\draw[smooth,samples=300] plot(\x,{\a*(\x)^3+\b*(\x)^2+\c*(\x)+\d});
		\end{tikzpicture}
	}
	
	\loigiai{
		Dựa vào đồ thị hàm số $y=f'(x)$ suy ra $f'(x) \leq 0 ~ \forall x<3$ và $f'(x)>0 ~ \forall x>3$.
		$
		g'(x)=\mathrm{e}^x f'\left(\mathrm{e}^x-2\right) .
		$
		Hàm số $g(x)=f\left(\mathrm{e}^x-2\right)-2020$ nghịch biến \\ $ \Leftrightarrow g'(x)<0 \Leftrightarrow \mathrm{e}^x f'\left(\mathrm{e}^x-2\right)<0$\\
		$
		\Leftrightarrow f'\left(\mathrm{e}^x-2\right)<0 \Leftrightarrow \mathrm{e}^x-2<3 \Leftrightarrow \mathrm{e}^x<5 \Leftrightarrow x<\ln 5 .
		$\\
		Vậy hàm số đã cho nghịch biến trên $\left(-1 ; \dfrac{3}{2}\right)$.
	}
\end{ex}
\begin{ex}[Lý Nhân Tông - Bắc Ninh - 2020]%[2D1G1-2]
	\immini{
		Cho hàm số $f(x)$ có đồ thị hàm số $f'(x)$ như hình vẽ.\\
		Hàm số $y=f(\cos x)+x^2-x$ đồng biến trên khoảng
		\choice
		{$(-2 ; 1)$}
		{$(0 ; 1)$}
		{\True $(1 ; 2)$}
		{$(-1 ; 0)$}
	}
	{
		\begin{tikzpicture}[scale=1,>=stealth, font=\footnotesize, line join=round, line cap=round]
			\def\a{-0.5} \def\b{0} \def\c{1.5} \def\d{0} % Hệ số
			\def\xmin{-3} \def\xmax{4}
			\def\ymin{-2} \def\ymax{2} 
			%\draw[color=gray!50,dashed] (\xmin,\ymin) grid (\xmax,\ymax); 
			\draw[->] (\xmin,0)--(\xmax,0) node [below]{$x$};
			\draw[->] (0,\ymin)--(0,\ymax) node [left]{$y$};
			\node at (0,0) [above left]{$O$};
			\node at (3,0) [below right]{$3$};
			\draw[dashed] (-2,0) node[below]{$-2$}--(-2,1) --(0,1) node[above right]{$1$} --(1,1)--(1,0) node[below]{$1$};
			\draw[dashed] (-1,0) node[below right]{$-1$}--(-1,-1) --(0,-1) node[above right]{$-1$} --(2,-1)--(2,0) node[below right]{$2$};
			\clip (\xmin+0.1,\ymin+0.1) rectangle (\xmax-0.5,\ymax-0.1);
			\draw[smooth,samples=300][domain=-2:2] plot(\x,{\a*(\x)^3+\b*(\x)^2+\c*(\x)+\d});
		\end{tikzpicture}
	}
	\loigiai{
		Đặt  $g(x)=f(\cos x)+x^2-x$.\\
		Ta có $g'(x)=-\sin x \cdot f'(\cos x)+2 x-1$\\
		Vì $\cos x \in[-1 ; 1]$ nên từ đồ thị $f'(x)$ ta suy ra $f'(\cos x) \in[-1 ; 1]$.\\
		Do đó $\left|-\sin x \cdot f'(\cos x)\right| \leq 1, ~\forall x \in \mathbb{R}$.\\
		Ta suy ra $g'(x)=\sin x \cdot f'(\cos x)+2 x-1 \geq-1+2 x-1=2 x-2$
		$\Rightarrow g'(x)>0, ~\forall x>1$.\\
		Vậy hàm số đồng biến trên $(1 ; 2)$.
	}
\end{ex}
\begin{ex}[THPT Nguyễn Viết Xuân - 2020]%[2D1G1-2]
	\immini{
		Cho hàm số $f(x)$. Hàm số $y=f'(x)$ có đồ thị như hình vẽ.\\
		Hàm số $g(x)=f\left(3 x^2-1\right)-\dfrac{9}{2}x^4+3 x^2$ đồng biến trên khoảng nào dưới đây?
		\choice
		{\True $\left(-\dfrac{2 \sqrt{3}}{3}; \dfrac{-\sqrt{3}}{3}\right)$}
		{$\left(0 ; \dfrac{2 \sqrt{3}}{3}\right)$}
		{$(1 ; 2)$}
		{$\left(-\dfrac{\sqrt{3}}{3}; \dfrac{\sqrt{3}}{3}\right)$} 
	}
	{
		\begin{tikzpicture}[scale=0.6,>=stealth, font=\footnotesize, line join=round, line cap=round]
			\def\a{0.25} \def\b{0.25} \def\c{-2} \def\d{0} % Hệ số
			\def\xmin{-5} \def\xmax{4}
			\def\ymin{-5} \def\ymax{5} 
			%\draw[color=gray!50,dashed] (\xmin,\ymin) grid (\xmax,\ymax); 
			\draw[->] (\xmin,0)--(\xmax,0) node [below]{$x$};
			\draw[->] (0,\ymin)--(0,\ymax) node [left]{$y$};
			\node at (0,0) [above left]{$O$};
			%\node at (3,0) [below right]{$3$};
			\draw[dashed] (-4,0) node[below left]{$-4$}--(-4,-4) --(0,-4) node[above right]{$-4$};
			\draw[dashed] (3,0) node[below right]{$3$}--(3,3) --(0,3) node[above right]{$3$};
			\clip (\xmin+0.1,\ymin+0.1) rectangle (\xmax-0.5,\ymax-0.1);
			\draw[smooth,samples=300] plot(\x,{\a*(\x)^3+\b*(\x)^2+\c*(\x)+\d});
		\end{tikzpicture}
	}
	
	\loigiai
	{
		TXĐ: $\mathscr{D}=\mathbb{R}$.\\
		Ta có $g'(x)=6 x f'\left(3 x^2-1\right)-18 x^3+6 x=6 x\left[f'\left(3 x^2-1\right)-3 x^2+1\right]$.\\
		$
		g'(x)=0 \Leftrightarrow\hoac{
			&x = 0\\
			&f '( 3 x ^{2}- 1 ) = 3 x ^{2}- 1}
		\Leftrightarrow \hoac{
			&x = 0\\
			&3 x ^{2}- 1 = - 4 \text{~(vô nghiệm)}\\
			&3 x ^{2}- 1 = 0\\
			&3 x ^{2}- 1 = 3}\Leftrightarrow \hoac{&x=0 \\
			&x=\pm \dfrac{\sqrt{3}}{3}\\
			&x=\pm \dfrac{2 \sqrt{3}}{3}.}
		$\\
		Bảng xét dấu
		\begin{center}
			\begin{tikzpicture}
				\tkzTabInit[nocadre,lgt=1.2,espcl=2.2,deltacl=0.6]
				{$x$ /1.2,$f'(x)$ /0.7}
				{$-\infty$,$-\dfrac{2 \sqrt{3}}{3}$,$-\dfrac{ \sqrt{3}}{3}$,$0$,$\dfrac{\sqrt{3}}{3}$,$\dfrac{2 \sqrt{3}}{3}$,$+\infty$}
				\tkzTabLine{,-,$0$,+,$0$,-,$0$,+,$0$,-,$0$,+,}
			\end{tikzpicture}
		\end{center}
		Vậy hàm số đồng biến trong khoảng $\left(-\dfrac{2 \sqrt{3}}{3}; \dfrac{-\sqrt{3}}{3}\right)$.}
\end{ex}
\begin{ex}[Trần Phú - Quảng Ninh - 2020]%[2D1G1-2]
	Cho hàm số $f(x)$ có bảng xét dấu của đạo hàm như sau
	\begin{center}
		\begin{tikzpicture}
			\tkzTabInit[nocadre,lgt=1.2,espcl=2,deltacl=0.6]
			{$x$ /0.6,$f'(x)$ /0.6}
			{$-\infty$,$-4$,$-1$,$2$,$7$,$+\infty$}
			\tkzTabLine{,+,$0$,-,$0$,+,$0$,-,$0$,+,}
		\end{tikzpicture}
	\end{center}
	Hàm số $y=f(2 x+1)+\dfrac{2}{3}x^3-8 x+5$ nghịch biến trên khoảng nào dưới đây?
	\choice
	{$(-\infty ;-2)$}
	{$(1 ;+\infty)$}
	{$(-1 ; 7)$}
	{\True $\left(-1 ; \dfrac{1}{2}\right)$}
	\loigiai{
		Ta có $y'=2 f'(2 x+1)+2 x^2-8$.\\
		Xét $y'\leq 0 \Leftrightarrow 2 f'(2 x+1)+2 x^2-8 \leq 0 \Leftrightarrow f'(2 x+1) \leq 4-x^2$.\\
		Đặt $t=2x+1$, ta có $f'(t) \leq \dfrac{-t^2+2 t+15}{4}$.\\
		Vì $\dfrac{-t^2+2 t+15}{4}\geq 0, \forall t \in[-3 ; 5]$.\\
		Mà $f'(t) \leq 0, \forall t \in[-3 ; 2]$.\\
		Nên $f'(t) \leq \dfrac{-t^2+2 t+15}{4}\Rightarrow t \in[-3 ; 2]$.\\
		Suy ra $-3 \leq 2 x+1 \leq 2 \Leftrightarrow-2 \leq x \leq \dfrac{1}{2}$.}
\end{ex}

\begin{ex}[Chuyên Thái Bình - Lần 3 - 2020]%[2D1G1-2]
	\immini{
		Cho hàm số $y=f(x)$ liên tục trên $\mathbb{R}$ có đồ thị hàm số $y=f'(x)$ cho như hình vẽ.\\
		Hàm số $g(x)=2 f(|x-1|)-x^2+2 x+2020$ đồng biến trên khoảng nào?
		\choice
		{\True $(0 ; 1)$}
		{$(-3 ; 1)$}
		{$(1 ; 3)$}
		{$(-2 ; 0)$}
	}
	{
		\begin{tikzpicture}[scale=0.7,>=stealth, font=\footnotesize, line join=round, line cap=round]
			\def\a{-0.333} \def\b{1} \def\c{1.333} \def\d{-1} % Hệ số
			\def\xmin{-3} \def\xmax{5}
			\def\ymin{-3} \def\ymax{5} 
			%\draw[color=gray!50,dashed] (\xmin,\ymin) grid (\xmax,\ymax); 
			\draw[->] (\xmin,0)--(\xmax,0) node [below]{$x$};
			\draw[->] (0,\ymin)--(0,\ymax) node [left]{$y$};
			\node at (0,0) [above left]{$O$};
			%\node at (3,0) [below right]{$3$};
			\draw[dashed] (-1,0) node[above]{$-1$}--(-1,-1) --(0,-1) node[above right]{$-1$};
			\draw[dashed] (1,0) node[below right]{$1$}--(1,1) --(0,1) node[above right]{$1$};
			\draw[dashed] (3,0) node[below right]{$3$}--(3,3) --(0,3) node[above right]{$3$};
			\clip (\xmin+0.1,\ymin+0.1) rectangle (\xmax-0.5,\ymax-0.1);
			\draw[smooth,samples=300] plot(\x,{\a*(\x)^3+\b*(\x)^2+\c*(\x)+\d});
			\draw[smooth,samples=300] plot(\x,{(\x)});
		\end{tikzpicture}
	}
	\loigiai{
		Với $x>1$, ta có $g(x)=2 f(x-1)-(x-1)^2+2021 \Rightarrow g'(x)=2 f'(x-1)-2(x-1)$.\\
		Hàm số đồng biến $\Leftrightarrow 2 f'(x-1)-2(x-1)>0 \Leftrightarrow f'(x-1)>x-1 \quad(*)$.\\
		Đặt $t=x-1$, khi đó $(*) \Leftrightarrow f'(t)>t \Leftrightarrow\hoac{&1<t<3 \\ &t<-1}\Rightarrow\hoac{&2<x<4 \\ &x<0 ~(\text{loại}).}$\\
		Với $x<1$, ta có $g(x)=2 f(1-x)-(1-x)^2+2021 \Rightarrow g'(x)=-2 f'(1-x)+2(1-x)$.\\
		Hàm số đồng biến $\Leftrightarrow-2 f'(1-x)+2(1-x)>0 \Leftrightarrow f'(1-x)<1-x \quad(* *)$.\\
		Đặt $t=1-x$, khi đó $(* *) \Leftrightarrow f'(t)<t \Leftrightarrow\hoac{&-1<t<1 \\ &t>3}\Rightarrow\hoac{&0<x<2 \\ &x<-2}\Rightarrow\hoac{&0<x<1 \\ &x<-2.}$\\
		Vậy hàm số $g(x)$ đồng biến trên các khoảng $(-\infty ;-2),(0 ; 1),(2 ; 4)$.
	}
\end{ex}
\begin{ex}[Sở Phú Thọ - 2020]%[2D1G1-2]
	\immini{
		Cho hàm số $y=f(x)$ có đồ thị hàm số $f'(x)$ như hình vẽ.\\
		Hàm số $g(x)=f\left(1+e^x\right)+2020$ nghịch biến trên khoảng nào dưới đây?
		\choice
		{$(0 ;+\infty)$}
		{$\left(\dfrac{1}{2}; 1\right)$}
		{\True $\left(0 ; \dfrac{1}{2}\right)$}
		{$(-1 ; 1)$}
	}{
		\begin{tikzpicture}[scale=0.7,>=stealth, font=\footnotesize, line join=round, line cap=round]
			\def\a{1} \def\b{-3} \def\c{0} \def\d{0} % Hệ số
			\def\xmin{-2} \def\xmax{4}
			\def\ymin{-5} \def\ymax{2} 
			%\draw[color=gray!50,dashed] (\xmin,\ymin) grid (\xmax,\ymax); 
			\draw[->] (\xmin,0)--(\xmax,0) node [below]{$x$};
			\draw[->] (0,\ymin)--(0,\ymax) node [left]{$y$};
			\node at (0,0) [above left]{$O$};
			\node at (3,0) [below right]{$3$};
			\draw[dashed] (2,0) node[above]{$2$}--(2,-4) --(0,-4) node[left]{$-4$};
			\clip (\xmin+0.1,\ymin+0.1) rectangle (\xmax-0.5,\ymax-0.1);
			\draw[smooth,samples=300] plot(\x,{\a*(\x)^3+\b*(\x)^2+\c*(\x)+\d});
		\end{tikzpicture}
	}
	\loigiai{
		$g'(x)=e^x f'\left(1+e^x\right)$.\\
		Do $e^x>0, \forall x$ nên $g'(x) \leq 0 \Leftrightarrow f'\left(1+e^x\right) \leq 0 \Leftrightarrow 1+e^x \leq 3 \Leftrightarrow x \leq \ln 2$, dấu bằng xảy ra tại hữu hạn điểm.\\
		Nên $g(x)$ nghịch biến trên $(-\infty ; \ln 2)$.\\
		Vì $\left(0 ; \dfrac{1}{2}\right) \subset (-\infty ; \ln 2)$ nên hàm số đã cho nghịch biến trên $\left(0 ; \dfrac{1}{2}\right)$.
	}
\end{ex}

\begin{ex}%[2D1K1-2]
	[THPT Anh Sơn - Nghệ An - 2020]
	Cho hàm số $y=f(x)$ có bảng xét dấu của đạo hàm như sau.
	\begin{center}
		\begin{tikzpicture}
			\tkzTabInit[nocadre,lgt=1.2,espcl=2,deltacl=0.6]
			{$x$ /0.6,$f'(x)$ /0.6}
			{$-\infty$,$-2$,$-1$,$2$,$4$,$+\infty$}
			\tkzTabLine{,+,$0$,-,$0$,+,$0$,-,$0$,+,}
		\end{tikzpicture}
	\end{center}
	Hàm số $y=-2 f(x)+2019$ nghịch biến trên khoảng nào trong các khoảng dưới đây?
	\choice
	{$(2 ; 4)$}
	{$(-4 ; 2)$}
	{$(-2 ;-1)$}
	{\True $(-1 ; 2)$}
	\loigiai{
		Ta có $y'=-2 f'(x)$.\\
		$
		y'=0 \Leftrightarrow-2 f'(x)=0 \Leftrightarrow\hoac{&
			x=-2 \\
			&x=-1 \\
			&x=2 \\
			&x=4.}$\\
		Từ bảng xét dấu của $f'(x)$ ta có
		\begin{center}
			\begin{tikzpicture}
				\tkzTabInit[nocadre,lgt=1,espcl=2,deltacl=0.6]
				{$x$ /0.6,$y'$ /0.6}
				{$-\infty$,$-2$,$-1$,$2$,$4$,$+\infty$}
				\tkzTabLine{,-,$0$,+,$0$,-,$0$,+,$0$,-,}
			\end{tikzpicture}
		\end{center}
		Từ bảng xét dấu ta có hàm số nghịch biến trên khoảng $(-\infty ;-2),(-1 ; 2)$ và $(4 ;+\infty)$.}
\end{ex}

\begin{ex}[THPT Anh Sơn - Nghệ An - 2020]%[2D1G1-2]
	Cho hàm số $f(x)$ xác định và liên tục trên $\mathbb{R}$ và có đạo hàm $f'(x)$ thỏa mãn $f'(x)=(1-x)(x+2) g(x)+2019$ với $g(x)<0, ~\forall x \in \mathbb{R}$ . Hàm số $y=f(1-x)+2019 x+2020$ nghịch biến trên khoảng nào?
	\choice
	{$(1 ;+\infty)$}
	{$(0 ; 3)$}
	{$(-\infty ; 3)$}
	{\True $(3 ;+\infty)$}
	\loigiai{
		Đặt $h(x)=f(1-x)+2019 x+2020$.\\
		Vì hàm số $f(x)$ xác định trên $\mathbb{R}$ nên hàm số $h(x)$ cũng xác định trên $\mathbb{R}$.\\
		Ta có $h'(x)=-f'(1-x)+2019$.\\
		Do $h'(x)=0$ tại hữu hạn điểm nên để tìm khoảng nghịch biến của hàm số $h(x)$, ta tìm các giá trị của $x$ sao cho $h'(x)<0 \Leftrightarrow-f'(1-x)+2019<0$\\ 
		$\Leftrightarrow f'(1-x)-2019>0 \\
		\Leftrightarrow x(3-x) g(1-x)>0 \Leftrightarrow x(3-x)<0(\text{~do~}g(x)<0, \forall x \in \mathbb{R})$\\
		$\Leftrightarrow\hoac{&
			x<0 \\
			&x>3.}$\\
		Vậy hàm số $y=f(1-x)+2019 x+2020$ nghịch biến trên các khoảng $(-\infty ; 0)$ và $(3 ;+\infty).$}
\end{ex}

\begin{ex}%[2D1G1-2]
	Cho hàm số $y=f(x)$ xác định trên $\mathbb{R}$ và có bảng xét dấu đạo hàm như sau:
	\begin{center}
		\begin{tikzpicture}
			\tkzTabInit[nocadre,lgt=2,espcl=2,deltacl=0.6]
			{$x$ /0.6,$f'(x)$ /0.6}
			{$-\infty$,$-1$,$1$,$4$,$+\infty$}
			\tkzTabLine{,-,$0$,+,$0$,-,$0$,+,}
		\end{tikzpicture}
	\end{center}
	Biết $f(x)>2,~ \forall x \in \mathbb{R}$. Xét hàm số $g(x)=f(3-2 f(x))-x^3+3 x^2-2020$. Khẳng định nào sau đây đúng?
	\choice
	{Hàm số $g(x)$ đồng biến trên khoảng $(-2 ;-1)$}
	{Hàm số $g(x)$ nghịch biến trên khoảng $(0 ; 1)$}
	{Hàm số $g(x)$ đồng biến trên khoảng $(3 ; 4)$}
	{\True Hàm số $g(x)$ nghịch biến trên khoảng $(2 ; 3)$}
	\loigiai{
		Ta có $g'(x)=-2 f'(x) f'(3-2 f(x))-3 x^2+6 x$.\\
		Vì $f(x)>2, ~\forall x \in \mathbb{R}$ nên $3-2 f(x)<-1 ~\forall x \in \mathbb{R}$.\\
		Từ bảng xét dấu $f'(x)$ suy ra $f'(3-2 f(x))<0, ~\forall x \in \mathbb{R}$.\\
		Từ đó ta có bảng xét dấu sau:
		\begin{center}
			\begin{tikzpicture}
				\tkzTabInit[nocadre,lgt=4,espcl=1.7,deltacl=0.6]
				{$x$ /0.7,$-f'(x)f'\left( 3-2f(x)\right) $/0.8,$-3x^2+6x$/0.7}
				{$-\infty$,$-1$,$0$,$1$,$2$,$4$,$+\infty$}
				\tkzTabLine{,-,$0$,+,|,+,$0$,-,|,-,$0$,+,}
				\tkzTabLine{,-,|,-,$0$,+,|,+,$0$,-,|,-,}
			\end{tikzpicture}
		\end{center}
		Từ bảng xét dấu trên, loại trừ đáp án suy ra hàm số $g(x)$ nghịch biến trên khoảng $(2 ; 3)$.}
\end{ex}

\begin{ex}%[2D1G1-2]
	Cho hàm số $f(x)$ có bảng biến thiên như sau:
	\begin{center}
		\begin{tikzpicture}
			\tkzTabInit[nocadre,lgt=1.2,espcl=2.5,deltacl=0.6]
			{$x$ /0.7, $f'(x)$ /0.7, $f(x)$ /2.5}
			{$-\infty$,$1$,$2$,$3$,$4$,$+\infty$}
			\tkzTabLine{,+,$0$,-,$0$,+,$0$,-,$0$,+,}
			\tkzTabVar{-/$-\infty$,+/$3$,-/$1$,+/$2$,-/$0$,+/$+\infty$}
		\end{tikzpicture}
	\end{center}
	Hàm số $y=(f(x))^3-3 .(f(x))^2$ nghịch biến trên khoảng nào dưới đây?
	\choice
	{$(1 ; 2)$}
	{$(3 ; 4)$}
	{$(-\infty ; 1)$}
	{\True $(2 ; 3)$}
	\loigiai{
		Ta có $y'=3 \cdot(f(x))^2 \cdot f'(x)-6 \cdot f(x) \cdot f'(x)=3 f(x) \cdot f'(x) \cdot[f(x)-2]. \\
		y'=0 \Leftrightarrow \hoac{&f(x)=0 \Leftrightarrow x \in\left\{x_1, 4 \mid x_1<1\right\}\\
			&f(x)=2 \Leftrightarrow x \in\left\{x_2, x_3, 3, x_4 \mid x_1<x_2<1<x_3<2 ; 4<x_4\right\}\\
			&f'(x)=0 \Leftrightarrow x \in\{1,2,3,4\}.}$\\
		Lập bảng xét dấu ta có
		\begin{center}
			\begin{tikzpicture}
				\tkzTabInit[nocadre,lgt=2,espcl=1.5,deltacl=0.6]
				{$x$ /0.7,$f(x)$ /0.7,$f(x)-2$ /0.7,$f'(x)$/0.7,$y'$/0.7}
				{$-\infty$,$x_1$,$x_2$,$1$,$x_3$,$2$,$3$,$4$,$x_4$,$+\infty$}
				\tkzTabLine{,-,$0$,+,|,+,|,+,|,+,|,+,$0$,+,|,+,|,+,}
				\tkzTabLine{,-,|,-,$0$,+,$0$,+,$0$,-,|,-,$0$,-,|,-,$0$,+}
				\tkzTabLine{,+,|,+,|,+,$0$,-,|,-,$0$,+,$0$,-,$0$,+,|,+}
				\tkzTabLine{,+,$0$,-,$0$,+,$0$,-,$0$,+,$0$,-,$0$,+,$0$,-,$0$,+}
			\end{tikzpicture}
		\end{center}
		
		Do đó hàm số nghịch biến trên khoảng $(2 ; 3)$.
	}
\end{ex}
\begin{ex}%[2D1G1-2]
	Cho hàm số $y=f(x)$ có đồ thị nằm trên trục hoành và có đạo hàm trên $\mathbb{R}$, bảng xét dấu của biểu thức $f'(x)$ như bảng dưới đây.
	\begin{center}
		\begin{tikzpicture}
			\tkzTabInit[nocadre,lgt=1.2,espcl=2,deltacl=0.6]
			{$x$ /0.6,$f'(x)$ /0.6}
			{$-\infty$,$-2$,$-1$,$3$,$+\infty$}
			\tkzTabLine{,-,$0$,+,$0$,-,$0$,+,}
		\end{tikzpicture}
	\end{center}
	Hàm số $y=g(x)=\dfrac{f\left(x^2-2 x\right)}{f\left(x^2-2 x\right)+1}$ nghịch biến trên khoảng nào dưới đây?
	\choice
	{$(-\infty ; 1)$}
	{$\left(-2 ; \dfrac{5}{2}\right)$}
	{\True $(1 ; 3)$}
	{$(2 ;+\infty)$}
	\loigiai{
		$ g'(x)=\dfrac{\left(x^2-2 x\right)'\cdot f'\left(x^2-2 x\right)}{\left(f\left(x^2-2 x\right)+1\right)^2}=\dfrac{(2 x-2) \cdot f'\left(x^2-2 x\right)}{\left(f\left(x^2-2 x\right)+1\right)^2}. \\
		g'(x)=0 \Leftrightarrow\hoac{
			&2 x - 2 = 0\\
			&f '( x ^{2}- 2 x ) = 0}
		\Leftrightarrow \hoac{&x = 1\\
			&x ^{2}- 2 x = - 2\\
			&x ^{2}- 2 x = - 1\\
			&x ^{2}- 2 x = 3}
		\Leftrightarrow \hoac{&x=1 \\
			&x=-1 \\
			&x=3.}
		$\\
		Ta có bảng xét dấu của $g'(x)$
		\begin{center}
			\begin{tikzpicture}
				\tkzTabInit[nocadre,lgt=1.2,espcl=2,deltacl=0.6]
				{$x$ /0.6,$g'(x)$ /0.6}
				{$-\infty$,$-1$,$1$,$3$,$+\infty$}
				\tkzTabLine{,-,$0$,+,$0$,-,$0$,+,}
			\end{tikzpicture}
		\end{center}
		Dựa vào bảng xét dấu ta có hàm số $y=g(x)$ nghịch biến trên các khoảng $(-\infty ;-1)$ và $(1 ; 3)$.}
\end{ex}
\begin{ex}[Liên trường huyện Quảng Xương - Thanh Hóa - 2021]%[2D1G1-2]
	\immini{
		Cho các hàm số $y=f(x)$; $y=g(x)$ liên tục trên $\mathbb{R}$ và có đồ thị các đạo hàm $f'(x) ; g'(x)$ (đồ thị hàm số $y=g'(x)$ là đường đậm hơn) như hình vẽ.\\
		Hàm số $h(x)=f(x-1)-g(x-1)$ nghịch biến trên khoảng nào dưới đây?
		\choice
		{$\left(\dfrac{1}{2}; 1\right)$}
		{$(1 ;+\infty)$}
		{$(2 ;+\infty)$}
		{\True $\left(-1 ; \dfrac{1}{2}\right)$}
	}
	{
		\begin{tikzpicture}[scale=1,>=stealth, font=\footnotesize, line join=round, line cap=round]
			%\def\a{1} \def\b{-6} \def\c{9} \def\d{1} % Hệ số
			\def\xmin{-4} \def\xmax{3}
			\def\ymin{-2} \def\ymax{4} 
			%\draw[color=gray!50,dashed] (\xmin,\ymin) grid (\xmax,\ymax); 
			\draw[->] (\xmin,0)--(\xmax,0) node [below]{$x$};
			\draw[->] (0,\ymin)--(0,\ymax) node [left]{$y$};
			\node at (0,0) [above left]{$O$};
			\node at (1,3) [below left]{$f'(x)$};
			\node at (1.5,3) [below right]{$g'(x)$};
			\draw[dashed] (-2,0) node[above right]{$-2$}--(-2,1);
			\draw[dashed] (1,0) node[below]{$1$}--(1,1);
			\draw[dashed] (-0.5,0) node[below]{$-0{,}5$}--(-0.5,2.125);
			\clip (\xmin+0.1,\ymin+0.1) rectangle (\xmax-0.5,\ymax-0.1);
			\draw[smooth,samples=300][domain=-3:2] plot(\x,{2*(\x)^4+4*(\x)^3-2*(\x)^2-4*(\x)+1});
			\draw[smooth,samples=300,line width=1.2pt] plot(\x,{(\x)^3+(\x)^2-2*(\x)+1});
		\end{tikzpicture}
	}
	
	\loigiai{
		Ta có: $h'(x)=f'(x-1)-g'(x-1)$.\\
		Dựa vào hình vẽ ta có hàm số $h(x)$ nghịch biến\\
		$\Leftrightarrow h'(x)<0 \Leftrightarrow f'(x-1)<g'(x-1)$\\
		$
		\Leftrightarrow\hoac{&- 2 < x - 1 < - \dfrac{1}{2}\\
			&0 < x - 1 < 1}
		\Leftrightarrow \hoac{
			&-1<x<\dfrac{1}{2}\\
			&1<x<2.}$\\
		Do đó hàm số $h(x)$ nghịch biến trên các khoảng $\left(-1 ; \dfrac{1}{2}\right)$ và $(1 ; 2)$.
	}
\end{ex}
\begin{ex}[THPT Quế Võ 1 - Bắc Ninh - 2021] %[2D1G1-2]
	\immini{
		Cho ba hàm số $y=f(x), y=g(x), y=h(x)$. Đồ thị của ba hàm số $y=f'(x), y=g'(x), y=h'(x)$ được cho như hình vẽ.\\
		Hàm số $k(x)=f(x+7)+g(5 x+1)-h\left(4 x+\dfrac{3}{2}\right)$ đồng biến trên khoảng nào dưới đây?
		\choice
		{$\left(-\dfrac{5}{8}; 0\right)$}
		{$\left(\dfrac{5}{8};+\infty\right)$}
		{\True $\left(\dfrac{3}{8}; 1\right)$}
		{$\left(-\dfrac{3}{8}; 1\right)$}
	}
	{
		\begin{tikzpicture}[scale=0.25,>=stealth, font=\footnotesize, line join=round, line cap=round]
			\def\a{-.078} \def\b{1.25} \def\c{0} % Hệ số
			\def\xmin{-4} \def\xmax{25}
			\def\ymin{-8} \def\ymax{18}
			
			%\draw[color=gray!50,dashed] (\xmin,\ymin) grid (\xmax,\ymax);
			
			\draw[->] (\xmin,0)--(\xmax,0) node [below]{$x$};
			\draw[->] (0,\ymin)--(0,\ymax) node [left]{$y$};
			\node at (20,14) [below right]{$y=g'(x)$};
			\node at (18,-2) [below left]{$y=h'(x)$};
			\node at (16,5) [below right]{$y=f'(x)$};
			\node at (0,0) [below left]{$O$};
			\draw[dashed] (3,0) node[below]{$3$}--(3,10)--(0,10) node[left]{$10$};
			\draw[dashed] (8,0) node[below]{$8$}--(8,5)--(0,5) node[left]{$5$};
			\draw[dashed] (4,0) node[below]{$4$}--(4,2)--(0,2) node[left]{$2$};
			\clip (\xmin+0.1,\ymin+0.1) rectangle (\xmax-0.5,\ymax-0.1);
			\draw[smooth,samples=300,domain=-2:18] plot(\x,{\a*(\x)^2+\b*(\x)+\c});
			%\draw[smooth,samples=300,domain=-2:25] plot(\x,{0.02*(\x)^3-0.6*(\x)^2+5.16*(\x)});
			\draw[line width=1.2pt] (-2,5)..controls (1.7,1.5) and (4.5,1.6)..(7,2.6);
			\draw[line width=1.2pt] (7,2.6)..controls (9,3.5) and (12,5)..(20,13);
			\draw (-0.5,-2) -- (0,0)--(3,10).. controls +(65:1) and + (-190:1)..(6,15).. controls +(0:1) and + (-180:1)..(14,-1).. controls +(0:1) and + (+80:1)..(19,16);
			
		\end{tikzpicture}
	}
	\loigiai{
		Ta có $k'(x)=f'(x+7)+5 g'(5 x+1)-4 h'\left(4 x+\dfrac{3}{2}\right)$.\\
		Khi $x \in \left( \dfrac{3}{8};1\right)$ thì $\heva{&7{,}375<x+7<8\\&2{,}875<5x+1<6\\&3<4x+\dfrac{4}{3}<5{,}5}\Leftrightarrow \heva{&f'(x+7)>10\\&g'(5x+1)>2 \Rightarrow 5g'(5x+1)>10  \\&h'\left( 4x+\dfrac{3}{2}\right)<5 \Rightarrow -4h'\left( 4x+\dfrac{3}{2}\right) >-20}.$\\
		Do đó $k'(x)=f'(x+7)+5g'(5x+1)-4h'\left( 4x+\dfrac{3}{2}\right)>0$.\\
		Hàm số $k(x)=f(x+7)+g(5 x+1)-h\left(4 x+\dfrac{3}{2}\right)$ đồng biến trên $\left(\dfrac{3}{8}; 1\right)$.
	}
\end{ex}
\begin{ex}[THPT Thanh Chương 1 - Nghệ An- 2021] %[2D1G1-2]
	Cho hàm số $y=f(x)$ liên tục trên $\mathbb{R}$ có bảng xét dấu đạo hàm như sau
	\begin{center}
		\begin{tikzpicture}
			\tkzTabInit[nocadre,lgt=1.2,espcl=2,deltacl=0.6]
			{$x$ /0.6,$f'(x)$ /0.6}
			{$-\infty$,$1$,$2$,$3$,$4$,$+\infty$}
			\tkzTabLine{,-,$0$,+,$0$,+,$0$,-,$0$,+,}
		\end{tikzpicture}
	\end{center}
	Hàm số $y=3f(2x-1)-4x^3+15x^2-18x+1$ đồng biến trên khoảng nào dưới đây?
	\choice
	{$\left(3;+\infty\right)$}
	{\True $\left(1;\dfrac{3}{2}\right)$}
	{$\left(\dfrac{5}{2}; 3\right)$}
	{$\left(2;\dfrac{5}{2}\right)$}
	\loigiai{
		Ta có $y'=6f'(2x-1)-12x^2+30x-18=6\left[f'(2x-1)-2x^2+5x-3\right] $.\\
		Có $f'(2x-1)=0 \Leftrightarrow \hoac{&2x-1=1\\&2x-1=2\\&2x-1=3\\&2x-1=4} \Leftrightarrow \hoac{&x=1\\&x=\dfrac{3}{2}\\&x=2\\&x=\dfrac{5}{2}.}$
		Ta có bảng xét dấu sau
		\begin{center}
			\begin{tikzpicture}
				\tkzTabInit[nocadre,lgt=3.0,espcl=1.5,deltacl=0.6]
				{$x$ /1.0,$f(x)$ /0.6,$f'(2x-1)$ /0.6,$-2x^2+5x-3$/0.6,$g'(x)$/0.6}
				{$-\infty$,$1$,$\dfrac{3}{2}$,$2$,$\dfrac{5}{2}$,$3$,$4$,$+\infty$}
				\tkzTabLine{,-,$0$,+,|,+,$0$,+,|,+,$0$,-,$0$,+,}
				\tkzTabLine{,-,$0$,+,$0$,+,$0$,-,$0$,+,|,+,|,+,}
				\tkzTabLine{,-,$0$,+,$0$,-,|,-,|,-,|,-,|,-,}
				\tkzTabLine{,-,$0$,+,$0$,,?,,|,,?,?,,?,}
			\end{tikzpicture}
		\end{center}
		Dựa vào bảng xét dấu trên, ta kết luận hàm số đã cho đồng biến trên khoảng $\left( 1; \dfrac{3}{2}\right).$
	}
\end{ex}


\begin{ex}%[2D2G4-3] %Câu 27 
	[THPT Hoàng Hoa Thám-Đà Nẵng-2021]
	Cho hàm số $f(x)$ có bảng xét dấu của $f'(x)$ như sau:\\
	\begin{center}
		\begin{tikzpicture}
			\tkzTabInit[lgt=1.2,espcl=2.3]
			{$x$/0.7, $f'(x)$ /.8} % first column
			{$-\infty$,$-3$,$1$, $2$, $+\infty$} % first row
			\tkzTabLine { ,+,0,-,0,+,0,+ }
		\end{tikzpicture}
	\end{center}	
	Hàm số $y=f\left(2-e^x\right)-\dfrac{1}{3}{e^{3x}}+3e^{2x}-5e^x+1$ đồng biến trên khoảng nào dưới đây?
	\choice
	{$\left(0;\dfrac{3}{2}\right)$}
	{$\left(1;3\right)$}
	{\True $\left(-3;0\right)$}
	{$\left(-4;-3\right)$}
	\loigiai{
		Ta có $y'=-e^x.f'\left(2-e^x\right)-e^{3x}+6e^{2x}-5e^x=e^x\left[-f'\left(2-e^x\right)-e^{2x}+6e^x-5\right]$ .\\
		Đặt $t=2-e^x$, ta được\\
		$y'=\left(2-t\right)\left[-f'(t)-\left(2-t\right)^2+6\left(2-t\right)-5\right]=\left(2-t\right)\left[-f'(t)-t^2-2t+3\right]$ .\\
		$y'=0\Leftrightarrow\left(2-t\right)\left[-f'(t)-t^2-2t+3\right]=0\Leftrightarrow
		\hoac{
			& t=2\\ 
			& f'(t)=-t^2-2t+3.}$\\
		Hàm số $g(x)=-x^2-2x+3$ là parabol có trục đối xứng $x=-1$ và cắt trục hoành tại 2 điểm có hoành độ 
		$\hoac{
			& x=1\\ 
			& x=-3
		}$. Suy ra $f'(t)=-t^2-2t+3\Leftrightarrow \hoac{
			& t=1\\ 
			& t=-3. }$\\
		Bảng xét dấu\\
		\begin{center}
			\begin{tikzpicture}
				\tkzTabInit[lgt=3.9,espcl=2,nocadre]
				{$t$/0.7, $2-t$ /0.8, $-f'(t)-t^2-2t+3$ /0.8, $y'$ /0.8} % first column
				{$-\infty$,$-3$,$1$,$2$,$+\infty$} % first row
				\tkzTabLine { ,+,|,+,|,+,z,-, } % second row
				\tkzTabLine {,-,0,+,0,-,|,-,} % third row
				\tkzTabLine {,-,0,+,0,-,0,+,} % last row
			\end{tikzpicture}
		\end{center}
		Dựa vào bảng xét dấu $y'>0,\forall x\in\left(-3;0\right)$.}
\end{ex}


\begin{ex}%[2D1G1-2]%Câu 28 
	[Sở Lạng Sơn 2022] Cho hàm số $f(x)$ có bảng biến thiên như sau:\\
	\begin{center}
		\begin{tikzpicture}
			\tkzTabInit[espcl=2.5,lgt=1,nocadre]
			{$x$/0.7,$y'$/0.7,$y$/3.5}
			{$-\infty$,$1$,$2$,$3$,$4$,$+\infty$}
			\tkzTabLine{,+,0,-,0,+,0,-,0,+,}
			\node (0) at ($(N12)+(0,-3)$) {$-\infty$};
			\node (1) at ($(N22)+(0,-.5)$) {$3$};
			\node (2) at ($(N32)+(0,-1.7)$) {$1$};
			\node (3) at ($(N42)+(0,-0.7)$) {$2$};
			\node (4) at ($(N52)+(0,-2.3)$) {$0$};
			\node (5) at ($(N62)+(0,-.3)$) {$+\infty$};
			%				\node (8) at ($(N42)+(0,-.5)$) {};
			%				\coordinate (9) at ($(N42)!.6!(N53)+ (-0.5,0)$);
			%				\coordinate (6) at ($(T12)!.6!(T13)$);
			%				\coordinate (7) at ($(T22)!.6!(T23)$);
			\draw[-stealth] (0)--(1);
			\draw[-stealth] (1)--(2);
			\draw[-stealth] (2)--(3);
			\draw[-stealth] (1)--(2);
			\draw[-stealth] (3)--(4);
			\draw[-stealth] (4)--(5);
			%				\draw[->,red] (5)--(8);
			%				\draw[->,red] (8)--(9);
			%				\draw[blue,dashed](6)--(7)node[above left]{$y=0$};
		\end{tikzpicture}		
	\end{center}
	Hàm số $y=\left[f(x)\right]^3-3\left[f(x)\right]^2$ đồng biến trên khoảng nào dưới đây?
	\choice
	{$\left(-\infty\,;1\right)$}
	{$\left(1\,;2\right)$}
	{\True $\left(3\,;4\right)$}
	{$\left(2\,;3\right)$}
	\loigiai{
		Ta có $y'=3f'(x)\left[f^2(x)-2f(x)\right]$. 
		Phương trình $y'=0\Leftrightarrow \hoac{
			&{f}'(x)=0\\ 
			& f(x)=0\\ 
			& f(x)=2.
		}$
		\begin{center}
			\begin{tikzpicture}
				\tkzTabInit[espcl=2.5,lgt=1.5]
				{$x$/0.7,$y'$/0.7,$y$/3.5}
				{$-\infty$,$1$,$2$,$3$,$4$,$+\infty$}
				\tkzTabLine{,+,0,-,0,+,0,-,0,+,}
				\node (0) at ($(N12)+(0,-3)$) {$-\infty$};
				\node (1) at ($(N22)+(0,-.3)$) {$3$};
				\node (2) at ($(N32)+(0,-1.7)$) {$1$};
				\node (3) at ($(N42)+(0,-0.8)$) {$2$};
				\node (4) at ($(N52)+(0,-2.3)$) {$0$};
				\node (5) at ($(N62)+(0,-.3)$) {$+\infty$};
				\node (a) at ($(N11)+(0.65,0.35)$) {$a$};
				\node (b) at ($(N11)+(2.0,0.4)$) {$b$};
				\node (c) at ($(N11)+(3.38,0.35)$) {$c$};
				\node (d) at ($(N11)+(11.85,0.4)$) {$d$};
				\node (6) at ($(N12)+(0,-0.8)$) {};
				\node (7) at ($(N62)+(0,-0.8)$) {};
				\node (8) at ($(N12)+(0,-2.3)$) {};
				\node (9) at ($(N62)+(0,-2.3)$) {};
				%				\node (8) at ($(N42)+(0,-.5)$) {};
				%				\coordinate (9) at ($(N42)!.6!(N53)+ (-0.5,0)$);
				\coordinate (A) at ($(0)!.25!(1)$);
				\coordinate (B) at ($(0)!.8!(1)$);
				\coordinate (C) at ($(1)!.35!(2)$);
				\coordinate (D) at ($(4)!.75!(5)$);
				%				\coordinate (7) at ($(T22)!.6!(T23)$);
				\draw[->] (0)--(1);
				\draw[->] (1)--(2);
				\draw[->] (2)--(3);
				\draw[->] (1)--(2);
				\draw[->] (3)--(4);
				\draw[->] (4)--(5);
				%				\draw[->,red] (5)--(8);
				%				\draw[->,red] (8)--(9);
				\draw[blue,dashed](6)--(7)node[below]{$y=2$} (a)--(A) (b)--(B) (c)--(C) (d)--(D);
				\draw[blue,dashed](8)--(9)node[below left]{$y=0$};
			\end{tikzpicture}		
		\end{center}
		Dựa vào bảng biến thiên, ta thấy $f'(x)=0\Leftrightarrow x\in \{ 1\,;2\,;3\,;4 \}$;\\
		$f(x)=0\Leftrightarrow x=a<1$ hoặc $x=4$;\\
		$f(x)=2\Leftrightarrow \hoac{
			& x=b\,\,\left(a<b<1\right)\\ 
			& x=c\in\left(1\,;2\right)\\ 
			& x=3\\ 
			& x=d>4.
		}$ \\
		Ta lập được bảng xét dấu của $y'$ 
		\begin{center}
			\begin{tikzpicture}
				\tkzTabInit[lgt=1.2,espcl=1.5,nocadre]
				{$x$/1, $f(x)$ /.8} % first column
				{$-\infty$,$a$, $b$, $1$,$c$, $2$,$3$, $4$, $d$, $+\infty$} % first row
				\tkzTabLine { ,+,z,-,z,+,z,-,z,+,z,-,z,+,z,-,z,+, } % second row
				%				\tkzTabLine {,-,z,+,t,+,} % third row
				%				\tkzTabLine {,+,d,-,z,+,} % last row
			\end{tikzpicture}
		\end{center}
		Từ bảng xét dấu, ta thấy hàm số đồng biến trên các khoảng \\
		$\left(-\infty;a\right)$, $\left(b;1\right)$, $\left(c;2\right)$, $\left(3;4\right)$ và $(d;+\infty)$.
	}
\end{ex}

\begin{ex}%[2D1G1-2]%Câu 29 
	[THPT Bùi Thị Xuân – Huế-2022] 
	\immini{
		Cho hàm số $y=f(x)$ là hàm đa thức bậc bốn. Đồ thị hàm số $f'(x+2)$ được cho trong hình vẽ bên. Hàm số 
		$$g(x)=4 f\left(x^2\right)-x^6+5 x^4-4 x^2+1$$
		đồng biến trên khoảng nào dưới đây?
		\choice
		{$(-4 ;-3)$}
		{\True $(2 ;+\infty)$}
		{$(-\sqrt{2};\sqrt{2})$}
		{$(-2 ;-1)$}}{
		\begin{tikzpicture}[scale=0.6,font=\footnotesize, line join=round, line cap=round, >=stealth] %Đường cong bậc 3
			\draw[thick, ->] (-5.3,0)--(5,0);
			\draw[thick, ->] (0,-3.5)--(0,7);
			\draw (5.2,0) node[below] {$x$};
			\draw (0,7.1) node[left]{$y$};
			\draw (0,0) node[below left]{$0$};
			\draw[fill] (-2,0) circle (0.5pt)node[below left]{$ -2 $};
			\draw[fill] (2,0) circle (0.5pt)node[below]{$ 2$};
			\draw[fill] (0,3) circle (0.5pt)node[left]{$ 3 $};
			\draw[fill] (0,1) circle (0.5pt)node[right]{$ 1 $};
			\draw[fill] (0,-1) circle (0.5pt)node[right]{$ -1 $};
			\draw[dashed] (-2,0)--(-2,1) --(0,1); 
			\draw[dashed](2,0)--(2,3)--(0,3);
			\draw[line width=1.2pt,smooth,samples=100,domain=-2.8:4.5] plot(\x,{-0.271*(\x)^3+0.75*(\x)^2+1.583*\x-1});
		\end{tikzpicture}		
	}
	\loigiai{
		$\begin{aligned}
			& g(x)=4f\left(x^2\right)-x^6+5x^4-4x^2+1\Rightarrow g' (x)=8xf'\left(x^2\right)-6x^5+20x^3-8x.\\ 
			& g' (x)=0\Leftrightarrow 8xf'\left(x^2\right)-6x^5+20x^3-8x=0 \\
			& \Leftrightarrow 2x\left[4f'\left(x^2\right)-3x^4+10x^2-4\right]=0\\ 
			&\Leftrightarrow 		\hoac{ 			& 2x=0\\ 
				& 4f'(x^2)-3x^4+10x^2-4=0
			}
			\Leftrightarrow \hoac{	& x=0\\ 
				& f'\left(x^2\right)=\dfrac{3}{4}{x^4}-\dfrac{5}{2}{x^2}+1.}
		\end{aligned}$\\ 
		Xét
		$f'\left(x^2\right)=\dfrac{3}{4}x^4-\dfrac{5}{2}x^2+1$. Đặt $x^2=t+2$, ta có\\
		$ f' (t+2)=\dfrac{3}{4}{(t+2)^2}-\dfrac{5}{2}(t+2)+1=\dfrac{3}{4}\left(t^2+4t+4\right)-\dfrac{5}{2}(t+2)-1=\dfrac{3}{4}{t^2}+\dfrac{1}{2}t-1$\\
		Khi đó số nghiệm của phương trình chính là số giao điểm của đồ thị hàm số $y=f' (t+2)$ và\\
		$ y=\dfrac{3}{4}{t^2}+\dfrac{1}{2}t-1$\\
		Ta có đồ thị 
		\begin{center}
			\begin{tikzpicture}[scale=0.6,font=\footnotesize, line join=round, line cap=round, >=stealth] %Đường cong bậc 3
				\draw[thick, ->] (-5.3,0)--(5,0);
				\draw[thick, ->] (0,-3.5)--(0,7);
				\draw (5.2,0) node[below] {$x$};
				\draw (0,7.1) node[left]{$y$};
				\draw (0,0) node[below left]{$0$};
				\draw[fill] (-2,0) circle (0.5pt)node[below left]{$ -2 $};
				\draw[fill] (2,0) circle (0.5pt)node[below]{$ 2$};
				\draw[fill] (0,3) circle (0.5pt)node[left]{$ 3 $};
				\draw[fill] (0,1) circle (0.5pt)node[right]{$ 1 $};
				\draw[fill] (0,-1) circle (0.5pt)node[right]{$ -1 $};
				\draw[dashed] (-2,0)--(-2,1) --(0,1); 
				\draw[dashed](2,0)--(2,3)--(0,3);
				\draw[line width=1.2pt,smooth,samples=100,domain=-2.8:4.5] plot(\x,{-0.271*(\x)^3+0.75*(\x)^2+1.583*\x-1});		
				\draw[line width=1.2pt,smooth,samples=100,domain=-3.3:2.8] plot(\x,{0.75*(\x)^2+0.5*\x-1});
			\end{tikzpicture}
		\end{center}
		Dựa vào đồ thị ta có $f' (t+2)=\dfrac{3}{4}t^2+\dfrac{1}{2}t-1\Leftrightarrow \hoac{& t=-2\\ & t=0\\ & t=2} \Leftrightarrow\hoac{& x+2=-2\\ & x+2=0\\ & x+2=2} \Leftrightarrow \hoac{& x=-4\\ & x=-2\\ & x=0.}$\\
		Ta có bảng xét dấu $g' (x)$ như sau
		\begin{center}
			\begin{tikzpicture}
				\tkzTabInit[lgt=1.2,espcl=2,nocadre]
				{$x$/0.7, $f(x)$ /.7}
				{$-\infty$, $-4$,$-2$, $0$, $+\infty$} % first row
				\tkzTabLine { ,-,z,+,z,-,z,+, }
			\end{tikzpicture}
		\end{center}
		Vậy hàm số $g(x)=4 f\left(x^2\right)-x^6+5 x^4-4 x^2+1$ đồng biến trên khoảng $(2 ;+\infty)$.}
\end{ex}

\begin{ex}%[2D1G1-2]%Câu 30
	[Chuyên Bắc Ninh 2022] 
	\immini{
		Cho hàm số $ y=f(x)$ liên tục trên $\mathbb{R}$ có đồ thị hàm số $ y=f'(x)$ có đồ thị như hình vẽ bên.
		Hàm số $g(x)=2f\left(\left| x-1\right|\right)-x^2+2x+2020$ đồng biến trên khoảng nào
		\choice
		{$\left(-2;0\right)$}
		{$\left(-3;1\right)$}
		{$\left(1\,;3\right)$}
		{\True $\left(0\,;\,1\right)$}}{
		\begin{tikzpicture}[scale=0.6,font=\footnotesize, line join=round, line cap=round, >=stealth] %Đường cong bậc 3
			\draw[thick, ->] (-3.3,0)--(5,0);
			\draw[thick, ->] (0,-3.0)--(0,5.5);
			\draw (5.2,0) node[below] {$x$};
			\draw (0,5.8) node[left]{$y$};
			\draw (0,0) node[below left]{$0$};
			\draw[fill] (-1,0) circle (0.5pt)node[above]{$ -1 $};
			\draw[fill] (1,0) circle (0.5pt)node[below]{$ 1$};
			\draw[fill] (0,1) circle (0.5pt)node[left]{$ 1 $};
			\draw[fill] (0,-1) circle (0.5pt)node[right]{$ -1 $};
			\draw[fill] (0,3) circle (0.5pt)node[left]{$ 3 $};
			\draw[fill] (3,0) circle (0.5pt)node[below]{$ 3 $};
			\draw[dashed] (-1,0)--(-1,-1) --(0,-1); 
			\draw[dashed](1,0)--(1,1)--(0,1);
			\draw[dashed](3,0)--(3,3)--(0,3);
			\draw[line width=1.2pt,smooth,samples=100,domain=-2.2:4.3] plot(\x,{-0.333*(\x)^3+1*(\x)^2+1.333*\x-1});		
			%\draw[line width=1.2pt,smooth,samples=100,domain=-3.3:2.8] plot(\x,{0.75*(\x)^2+0.5*\x-1});
		\end{tikzpicture}	
	}
	\loigiai{
		Ta có $g(x)=2f\left(\left| x-1\right|\right)-x^2+2x+2020\Leftrightarrow g(x)=2f\left(\left| x-1\right|\right)-\left(x-1\right)^2+2021$.\\
		Xét hàm số $ k\left(x-1\right)=2f\left(x-1\right)-\left(x-1\right)^2+2021$.\\
		Đặt $ t=x-1$\\
		Xét hàm số $ h(t)=2f(t)-t^2+2021$ $\Rightarrow{h}'(t)=2f'(t)-2t$.\\
		Kẻ đường $ y=x$ như hình vẽ.
		\begin{center}
			\begin{tikzpicture}[scale=0.6,font=\footnotesize, line join=round, line cap=round, >=stealth] %Đường cong bậc 3
				\draw[thick, ->] (-3.3,0)--(5,0);
				\draw[thick, ->] (0,-3.0)--(0,5.5);
				\draw (5.2,0) node[below] {$x$};
				\draw (0,5.8) node[left]{$y$};
				%	\draw (0,0) node[below left]{$0$};
				\draw[fill] (-1,0) circle (0.5pt)node[above]{$ -1 $};
				\draw[fill] (1,0) circle (0.5pt)node[below]{$ 1$};
				\draw[fill] (0,1) circle (0.5pt)node[left]{$ 1 $};
				\draw[fill] (0,-1) circle (0.5pt)node[right]{$ -1 $};
				\draw[fill] (0,3) circle (0.5pt)node[left]{$ 3 $};
				\draw[fill] (3,0) circle (0.5pt)node[below]{$ 3 $};
				\draw[dashed] (-1,0)--(-1,-1) --(0,-1); 
				\draw[dashed](1,0)--(1,1)--(0,1);
				\draw[dashed](3,0)--(3,3)--(0,3);
				\draw[line width=1.2pt,smooth,samples=100,domain=-2.2:4.3] plot(\x,{-0.333*(\x)^3+1*(\x)^2+1.333*\x-1});		
				%\draw[line width=1.2pt,smooth,samples=100,domain=-3.3:2.8] plot(\x,{0.75*(\x)^2+0.5*\x-1});
				\draw[line width=1.2pt,smooth,samples=100](-2,-2)--(4,4);
			\end{tikzpicture}
		\end{center}
		Khi đó $h'(t)>0\Leftrightarrow{f}'(t)-t>0\Leftrightarrow{f}'(t)>t$$\Leftrightarrow \hoac{
			& t<-1\\ 
			& 1<t<3.
		}$\\
		Do đó $k'\left(x-1\right)>0\Leftrightarrow \hoac{
			& x-1<-1\\ 
			& 1<x-1<3} \Leftrightarrow \hoac{
			& x<0\\ 
			& 2<x<4.}$\\
		Ta có bảng biến thiên của hàm số $ k\left(x-1\right)=2f\left(x-1\right)-\left(x-1\right)^2+2021$.
		\begin{center}
			\begin{tikzpicture}
				\tkzTabInit[lgt=1.8,espcl=2.3]
				{$x$ /1.2, $k'(x-1)$ /1.2,$k(x-1)$ /2}
				{$-\infty$ , $0$,$2$,$4$, $+\infty$}
				\tkzTabLine{,+,0,-,0,+,0,-,}
				\tkzTabVar{-/$ $ ,+/$ $, -/$ $,+/$ $,-/$ $}
			\end{tikzpicture}
		\end{center}
		Khi đó, ta có bảng biến thiên của $g(x)=2f\left(\left| x-1\right|\right)-\left(x-1\right)^2+2021$ bằng cách lấy đối xứng qua đường thẳng $ x=1$ như sau\\
		\begin{center}
			\begin{tikzpicture}
				\tkzTabInit[lgt=1.2,espcl=2.5,nocadre]
				{$x$ /0.7, $g'(x)$ /0.7,$g(x)$ /2.5}
				{$-\infty$ ,$-2$, $0$,$1$,$2$,$4$, $+\infty$}
				\tkzTabLine{,+,0,-,0,+,0,-,0,+,0,-,}
				\tkzTabVar{-/$ $ ,+/$ $, -/$ $,+/$ $,-/$ $,+/ $ $,-/$ $}
			\end{tikzpicture}
		\end{center}
		Vậy hàm số đồng biến trên $\left(0;1\right)$.}
\end{ex}

\begin{ex}%[2D1G1-2]%Câu 31
	[Chuyên Thái Bình 2022] 
	\immini{
		Cho hàm số $f(x)=a{x^4}+b{x^3}+c{x^2}+dx+a$ có đồ thị hàm số $y=f'(x)$ như hình vẽ bên. Hàm số $y=g(x)=f\left(1-2x\right)f\left(2-x\right)$ đồng biến trên khoảng nào dưới đây?
		\choice
		{$\left(\dfrac{1}{2};\dfrac{3}{2}\right)$}
		{$\left(-\infty ;0\right)$}
		{$\left(0;2\right)$}
		{\True $\left(3;+\infty\right)$}}{
		\begin{tikzpicture}[scale=0.9,font=\footnotesize, line join=round, line cap=round, >=stealth] %Đường cong bậc 3
			\draw[thick, ->] (-2.5,0)--(2.5,0);
			\draw[thick, ->] (0,-2.8)--(0,2.8);
			\draw (2.6,0) node[below] {$x$};
			\draw (0,2.9) node[left]{$y$};
			\draw (0,0) node[below left]{$0$};
			\draw[fill] (-1,0) circle (0.5pt)node[below left]{$ -1 $};
			\draw[fill] (1,0) circle (0.5pt)node[below right]{$ 1$};
			%			\draw[dashed] (-1,0)--(-1,-1) --(0,-1); 
			%			\draw[dashed](1,0)--(1,1)--(0,1);
			%			\draw[dashed](3,0)--(3,3)--(0,3);
			\draw[line width=1.2pt,smooth,samples=100,domain=-1.3:1.3] plot(\x,{3*(\x)^3-3*\x});		
			%\draw[line width=1.2pt,smooth,samples=100,domain=-3.3:2.8] plot(\x,{0.75*(\x)^2+0.5*\x-1});
		\end{tikzpicture}	
	}
	\loigiai{
		Ta có $f'(x)=4a{x^3}+3b{x^2}+2cx+d$, theo đồ thị thì đa thức $f'(x)$ có ba nghiệm phân biệt là $-1,0,1$ nên $f'(x)=4ax\left(x+1\right)\left(x-1\right)=4a{x^3}-4ax\Rightarrow f(x)=a{x^4}-2a{x^2}+a=a{\left(x^2-1\right)^2}$.\\
		Dựa vào đồ thị hàm số $y=f'(x)$ ta có $a>0$ nên $f(x)>0,\forall x\in\mathbb{R}\setminus\left\{\pm 1\right\}$.\\
		$g'(x)=\left[f\left(1-2x\right)\right]'f\left(2-x\right)+f\left(1-2x\right)\left[f\left(2-x\right)\right]'=-2f'\left(1-2x\right)f\left(2-x\right)-f\left(1-2x\right)f'\left(2-x\right)$. Xét $x\in\left(\dfrac{1}{2};\dfrac{3}{2}\right)\Rightarrow
		\heva{		
			& 1-2x\in\left(-2;0\right)\\ 
			& 2-x\in\left(\dfrac{1}{2};\dfrac{3}{2}\right)}$, dấu của $f'(x)$ không cố định trên $\left(\dfrac{1}{2};\dfrac{3}{2}\right)$ nên ta không kết luận được tính đơn điệu của hàm số $g(x)$ trên $\left(\dfrac{1}{2};\dfrac{3}{2}\right)$.\\
		Xét $x\in\left(-\infty ;0\right)\Rightarrow
		\heva{
			& 1-2x\in\left(1;+\infty\right)\\ 
			& 2-x\in\left(2;+\infty\right)} 
		\Rightarrow \heva{
			& f'\left(1-2x\right)>0\\ 
			& f'\left(2-x\right)>0} \Rightarrow g'(x)<0$.\\
		Do đó, hàm số $g(x)$ nghịch biến trên $\left(-\infty ;0\right)$.\\
		$x\in\left(0;2\right)\Rightarrow \heva{
			& 1-2x\in\left(-3;1\right)\\ 
			& 2-x\in\left(0;2\right)}$, dấu của $f'(x)$ không cố định trên $\left(-3;1\right)$ và $\left(0;2\right)$ nên ta không kết luận được tính đơn điệu của hàm số $g(x)$ trên $\left(\dfrac{1}{2};\dfrac{3}{2}\right)$.\\
		Xét $x\in\left(3;+\infty\right)\Rightarrow \heva{
			& 1-2x\in\left(-\infty ;-5\right)\\ 
			& 2-x\in\left(-\infty ;-1\right)} \Rightarrow \heva{
			& f'\left(1-2x\right)<0\\ 
			& f'\left(2-x\right)<0} \Rightarrow g'(x)>0$. \\
		Do đó, hàm số $g(x)$ đồng biến trên $\left(3;+\infty\right)$.}
\end{ex}

\begin{dang}{Bài toán hàm ẩn, hàm hợp liên quan đến tham số và một số bài toán khác}
\end{dang}

\begin{ex}%[2D1G1-3]%Câu 1
	[Chuyên Lê Hồng Phong Nam Định 2019]
	\immini{
		Cho hàm số $ y=f(x)$ có đạo hàm liên tục trên $\mathbb{R}$. Biết hàm số $ y=f'(x)$ có đồ thị như hình vẽ. Gọi $ S$ là tập hợp các giá trị nguyên $ m\in\left[-5\,;\,\text{5}\right]$ để hàm số $ g(x)=f\left(x+m\right)$ nghịch biến trên khoảng $\left(1\,;\,2\right)$. Hỏi $S$ có bao nhiêu phần tử?
		\choice
		{$ 4$}
		{$ 3$}
		{$ 6$}
		{\True $ 5$}}{
		\begin{tikzpicture}[scale=0.9,font=\footnotesize, line join=round, line cap=round, >=stealth] %Đường cong bậc 3
			\draw[thick, ->] (-2.5,0)--(4,0);
			\draw[thick, ->] (0,-2.8)--(0,2.8);
			\draw (4.3,0) node[below] {$x$};
			\draw (0,2.9) node[left]{$y$};
			\draw (0,0) node[below left]{$0$};
			\draw[fill] (-1,0) circle (0.5pt)node[below left]{$ -1 $};
			\draw[fill] (1,0) circle (0.5pt)node[below]{$ 1$};
			\draw[fill] (3,0) circle (0.5pt)node[below right]{$ 3$};
			%			\draw[dashed] (-1,0)--(-1,-1) --(0,-1); 
			%			\draw[dashed](1,0)--(1,1)--(0,1);
			%			\draw[dashed](3,0)--(3,3)--(0,3);
			\draw[line width=1.2pt,smooth,samples=100,domain=-1.65:3.5] plot(\x,{0.33*(\x)^3-(\x)^2-0.333*(\x)+1});		
			%\draw[line width=1.2pt,smooth,samples=100,domain=-3.3:2.8] plot(\x,{0.75*(\x)^2+0.5*\x-1});
		\end{tikzpicture}	
	}
	\loigiai{
		Ta có $g'(x)=f'\left(x+m\right)$. Vì $ y=f'(x)$ liên tục trên $\mathbb{R}$ nên $g'(x)=f'\left(x+m\right)$ cũng liên tục trên $\mathbb{R}$. Căn cứ vào đồ thị hàm số $ y=f'(x)$ ta thấy\\
		$g'(x)<0\Leftrightarrow{f}'\left(x+m\right)<0$ $\Leftrightarrow\hoac{
			& x+m<-1\\ 
			& 1<x+m<3} \Leftrightarrow \hoac{
			& x<-1-m\\ 
			& 1-m<x<3-m.}$\\
		Hàm số $ g(x)=f\left(x+m\right)$ nghịch biến trên khoảng $\left(1\,;\,2\right)$
		$\Leftrightarrow \hoac{
			& 2\le-1-m\\ 
			&\hoac{
				& 3-m\ge 2\\ 
				& 1-m\le 1}} \Leftrightarrow \hoac{
			& m\le-3\\ 
			& 0\le m\le 1.}$\\
		Mà $ m$ là số nguyên thuộc đoạn $\left[-5\,;\,5\right]$ nên ta có $ S=\left\{-5;-4;-3;0;1\right\}$.\\
		Vậy $ S$ có $5$ phần tử.}
\end{ex}

\begin{ex}%[2D1G1-3]%Câu 2
	[Chuyên Nguyễn Bỉnh Khiêm-Quảng Nam-2020] Cho hàm số $ y=f(x)$ có đạo hàm trên $\mathbb{R}$ và bảng xét dấu đạo hàm như hình vẽ sau
	\begin{center}
		\begin{tikzpicture}
			\tkzTabInit[lgt=1.2,espcl=2.5,nocadre]
			{$x$/0.7, $f'(x)$ /2.5} % first column
			{$-\infty$, $-10$,$-2$, $3$,$8$, $+\infty$} % first row
			\tkzTabLine { ,+,z,-,z,+,z,-,z,+, } % second row
			%				\tkzTabLine {,-,z,+,t,+,} % third row
			%				\tkzTabLine {,+,d,-,z,+,} % last row
		\end{tikzpicture}
	\end{center}
	Có bao nhiêu số nguyên $ m$ để hàm số $ y=f\left(x^3+4x+m\right)$ nghịch biến trên khoảng $\left(-1;1\right)$?
	\choice
	{$ 3$}
	{$ 0$}
	{\True $ 1$}
	{$ 2$}
	\loigiai
	{
		Đặt $ t=x^3+4x+m\Rightarrow{t}'=3x^2+4$ nên $ t$ đồng biến trên $\left(-1;1\right)$ và $ t\in\left(m-5;m+5\right)$.\\
		Yêu cầu bài toán trở thành tìm $ m$ để hàm số $ f(t)$ nghịch biến trên khoảng $\left(m-5;m+5\right)$.\\
		Dựa vào bảng biến thiên ta được $\heva{
			& m-5\ge-2\\ 
			& m+5\le 8} \Leftrightarrow \heva{
			& m\ge 3\\ 
			& m\le 3} \Leftrightarrow m=3$.}
\end{ex}

\begin{ex}%[2D1G1-3]%Câu 3
	[Chuyên ĐH Vinh-Nghệ An-2020]
	\immini{
		Cho hàm số $ f(x)$ có đạo hàm trên $\mathbb{R}$và $ f(1)=1$. Đồ thị hàm số $ y=f'(x)$ như hình bên. Có bao nhiêu số nguyên dương $ a$ để hàm số $ y=\left| 4f\left(\sin x\right)+\cos 2x-a\right|$ nghịch biến trên $\left(0;\dfrac{\pi}{2}\right)$?
		\choice
		{$ 2$}
		{\True $ 3$}
		{Vô số}
		{$ 5$}}{
		\begin{tikzpicture}[scale=0.9,font=\footnotesize, line join=round, line cap=round, >=stealth] %Đường cong bậc 3
			\draw[thick, ->] (-2.5,0)--(3,0);
			\draw[thick, ->] (0,-2.8)--(0,2.8);
			\draw (3.1,0) node[below] {$x$};
			\draw (0,2.9) node[left]{$y$};
			\draw (0,0) node[below left]{$0$};
			\draw[fill] (-1,0) circle (0.5pt)node[below]{$ -1 $};
			\draw[fill] (1,0) circle (0.5pt)node[above]{$ 1$};
			%	\draw[fill] (3,0) circle (0.5pt)node[below right]{$ 3$};
			\draw[dashed] (-1,0)--(-1,1); 
			\draw[dashed](1,0)--(1,-1);
			%			\draw[dashed](3,0)--(3,3)--(0,3);
			\draw[line width=1.2pt,smooth,samples=100,domain=-2:2] plot(\x,{.8*(\x)^3+0*(\x)^2-1.8*(\x)});		
			%\draw[line width=1.2pt,smooth,samples=100,domain=-3.3:2.8] plot(\x,{0.75*(\x)^2+0.5*\x-1});
			\draw (2.0,2.8) node[left]{$y=f'(x)$};
		\end{tikzpicture}	
	}
	\loigiai
	{		Đặt $g(x)=\left| 4f\left(\sin x\right)+\cos 2x-a\right|\Rightarrow g(x)=\sqrt{\left[4f\left(\sin x\right)+\cos 2x-a\right]^2}$ .\\
		$\Rightarrow{g}'(x)=\dfrac{\left[4\cos x\cdot f'\left(\sin x\right)-2\sin 2x\right]\left[4f\left(\sin x\right)+\cos 2x-a\right]}{\sqrt{\left[4f\left(\sin x\right)+\cos 2x-a\right]^2}}$.\\
		Ta có $ 4\cos x\cdot f'\left(\sin x\right)-2\sin 2x=4\cos x\left[f'\left(\sin x\right)-\sin x\right]$.\\
		Với $ x\in\left(0;\dfrac{\pi}{2}\right)$ thì $\cos x>0,\sin x\in\left(0;1\right)\Rightarrow{f}'\left(\sin x\right)-\sin x<0$.\\
		Hàm số $ g(x)$ nghịch biến trên $\left(0;\dfrac{\pi}{2}\right)$ khi $ 4f\left(\sin x\right)+\cos 2x-a\ge 0,\forall x\in\left(0;\dfrac{\pi}{2}\right)$\\
		$\Leftrightarrow 4f\left(\sin x\right)+1-2\sin^2x\ge a,\forall x\in\left(0;\dfrac{\pi}{2}\right)$.\\
		Đặt $ t=\sin x$ được $ 4f(t)+1-2t^2\ge a,\forall t\in\left(0;1\right)$ (*).\\
		Xét $ h(t)=4f(t)+1-2t^2\Rightarrow{h}'(t)=4f'(t)-4t=4\left[f'(t)-1\right]$.\\
		Với $ t\in\left(0;1\right)$ thì $h'(t)<0\Rightarrow h(t)$ nghịch biến trên $\left(0;1\right)$.\\
		Do đó (*) $\Leftrightarrow a\le h(1)=4f(1)+1-2.1^2=3$.\\
		Vậy có $3$ giá trị nguyên dương của a thỏa mãn.}
\end{ex}


\begin{ex}%[2D1G1-3]%Câu 4
	[Chuyên Quang Trung-2020]
	\immini{
		Cho hàm số $ y=f(x)$ có đạo hàm liên tục trên $\mathbb{R}$ và có đồ thị $ y=f'(x)$ như hình vẽ. Đặt $ g(x)=f\left(x-m\right)-\dfrac{1}{2}{\left(x-m-1\right)^2}+2019$, với $ m$ là tham số thực. Gọi $ S$ là tập hợp các giá trị nguyên dương của $ m$ để hàm số $ y=g(x)$ đồng biến trên khoảng $\left(5;6\right)$. Tổng tất cả các phần tử trong $ S$ bằng
		\choice
		{$ 4$}
		{$ 11$}
		{\True $ 14$}
		{$ 20$}}{
		\begin{tikzpicture}[scale=0.9,font=\footnotesize, line join=round, line cap=round, >=stealth] %Đường cong bậc 3
			\draw[style=help lines,step=1] (-2.5,-3) grid (3,3.5);
			\draw[thick, ->] (-2.5,0)--(3.5,0);
			\draw[thick, ->] (0,-2.8)--(0,2.8);
			\draw (3.6,0) node[below] {$x$};
			\draw (0,3) node[above left]{$y$};
			\draw (0,0) node[below left]{$0$};
			%\draw[fill] (-1,0) circle (0.5pt)node[below]{$ -1 $};
			\draw[fill] (1,0) circle (0.5pt)node[below left]{$ 1$};
			%	\draw[fill] (3,0) circle (0.5pt)node[below right]{$ 3$};
			\draw[dashed] (-1,0)--(-1,-2) --(2,-2)--(2,0); 
			\draw[dashed](3,0)--(3,2) --(0,2);
			\draw (-1,-2) circle (2pt);
			\draw (3,2) circle (2pt);
			%			\draw[dashed](3,0)--(3,3)--(0,3);
			\draw[line width=1.2pt,smooth,samples=100,domain=-1.1:3.1] plot(\x,{1*(\x)^3-3*(\x)^2-0*(\x)+2});		
			%\draw[line width=1.2pt,smooth,samples=100,domain=-3.3:2.8] plot(\x,{0.75*(\x)^2+0.5*\x-1});
			%\draw (2.0,2.8) node[left]{$y=f'(x)$};
		\end{tikzpicture}	
	}
	\loigiai
	{
		Xét hàm số $ g(x)=f\left(x-m\right)-\dfrac{1}{2}{\left(x-m-1\right)^2}+2019$.\\
		$g'(x)=f'\left(x-m\right)-\left(x-m-1\right)$.\\
		Xét phương trình $g'(x)=0. \quad \quad (1)$\\
		Đặt $ x-m=t$, phương trình $(1)$ trở thành $f'(t)-\left(t-1\right)=0\Leftrightarrow{f}'(t)=t-1. \quad (2)$\\
		Nghiệm của phương trình $(2)$ là hoành độ giao điểm của hai đồ thị hàm số $ y=f'(t)$ và $ y=t-1$.\\
		Ta có đồ thị các hàm số $ y=f'(t)$ và $ y=t-1$ như sau
		\begin{center}
			\begin{tikzpicture}[scale=0.9,font=\footnotesize, line join=round, line cap=round, >=stealth] %Đường cong bậc 3
				\draw[style=help lines,step=1] (-2.5,-3) grid (3,3.5);
				\draw[thick, ->] (-2.5,0)--(3.5,0);
				\draw[thick, ->] (0,-2.8)--(0,2.8);
				\draw (3.6,0) node[below] {$x$};
				\draw (0,3) node[above left]{$y$};
				\draw (0,0) node[below left]{$0$};
				%\draw[fill] (-1,0) circle (0.5pt)node[below]{$ -1 $};
				\draw[fill] (1,0) circle (0.5pt)node[below left]{$ 1$};
				%	\draw[fill] (3,0) circle (0.5pt)node[below right]{$ 3$};
				\draw[dashed] (-1,0)--(-1,-2) --(2,-2)--(2,0); 
				\draw[dashed](3,0)--(3,2) --(0,2);
				\draw (-1,-2) circle (2pt);
				\draw (3,2) circle (2pt);
				%			\draw[dashed](3,0)--(3,3)--(0,3);
				\draw[line width=1.2pt,smooth,samples=100,domain=-1.1:3.1] plot(\x,{1*(\x)^3-3*(\x)^2-0*(\x)+2});		
				%\draw[line width=1.2pt,smooth,samples=100,domain=-3.3:2.8] plot(\x,{0.75*(\x)^2+0.5*\x-1});
				%\draw (2.0,2.8) node[left]{$y=f'(x)$};
				\draw (-2,-3)--(4,3);
			\end{tikzpicture}
		\end{center}
		Căn cứ đồ thị các hàm số ta có phương trình $(2)$ có nghiệm là $\hoac{
			& t=-1\\ 
			& t=1\\ 
			& t=3} \Rightarrow \hoac{
			& x=m-1\\ 
			& x=m+1\\ 
			& x=m+3.}$\\
		Ta có bảng biến thiên của $ y=g(x)$
		\begin{center}
			\begin{tikzpicture}
				\tkzTabInit[lgt=1,espcl=2.5,nocadre]
				{$x$ /0.8, $y'$ /0.8,$y$ /2.5}
				{$-\infty$ , $m-1$,$m+1$,$m+3$, $+\infty$}
				\tkzTabLine{,+,0,-,0,+,0,-,}
				\tkzTabVar{-/$ +\infty$ ,+/$ $, -/$ $,+/$ $,-/$+\infty $}
			\end{tikzpicture}
		\end{center}
		Để hàm số $ y=g(x)$ đồng biến trên khoảng $\left(5;6\right)$ cần $\hoac{
			&\heva{
				& m-1\le 5\\ 
				& m+1\ge 6}\\ 
			& m+3\le 5}\Leftrightarrow\hoac{
			& 5\le m\le 6\\ 
			& m\le 2.}$\\
		Vì $ m\in\mathbb{N}^*\Rightarrow m$ nhận các giá trị $ 1;\,2;\,5;\,6\Rightarrow S=14$.}
\end{ex}

\begin{ex}%[2D1G1-3]%Câu 5
	[Sở Hà Nội-Lần 2-2020] 
	\immini{
		Cho hàm số $y=a{x^4}+b{x^3}+c{x^2}+dx+e,\,\,a\ne 0$. Hàm số $y=f'(x)$ có đồ thị như hình vẽ bên. 
		Gọi S là tập hợp tất cả các giá trị nguyên thuộc khoảng $\left(-6;6\right)$ của tham số $m$ để hàm số $g(x)=f\left(3-2x+m\right)+x^2-\left(m+3\right)x+2m^2$ nghịch biến trên $\left(0;1\right)$. Khi đó, tổng giá trị các phần tử của S là
		\choice
		{$12$}
		{\True $9$}
		{$6$}
		{$15$}}{
		\begin{tikzpicture}[scale=0.7,font=\footnotesize, line join=round, line cap=round, >=stealth] %Đường cong bậc 3
			%	\draw[style=help lines,step=1] (-2.5,-3) grid (3,3.5);
			\draw[thick, ->] (-4.5,0)--(6.5,0);
			\draw[thick, ->] (0,-2.8)--(0,2.8);
			\draw (6.6,0) node[below] {$x$};
			\draw (0,3) node[above left]{$y$};
			\draw (0,0) node[below left]{$0$};
			\draw[fill] (-2,0) circle (0.5pt)node[below]{$ -2 $};
			\draw[fill] (4,0) circle (0.5pt)node[above]{$ 4$};
			\draw[fill] (0,1) circle (0.5pt)node[right]{$ 1 $};
			\draw[fill] (0,-2) circle (0.5pt)node[left]{$ -2$};
			%	\draw[fill] (3,0) circle (0.5pt)node[below right]{$ 3$};
			\draw[dashed] (-2,0)--(-2,1) --(0,1); 
			\draw[dashed](4,0)--(4,-2) --(0,-2);
			%			\draw[dashed](3,0)--(3,3)--(0,3);
			\draw[line width=1.2pt,smooth,samples=100,domain=-3.8:5.5] plot(\x,{0.0714*(\x)^3-0.1423*(\x)^2-1.0714*(\x)});		
			%\draw[line width=1.2pt,smooth,samples=100,domain=-3.3:2.8] plot(\x,{0.75*(\x)^2+0.5*\x-1});
			%\draw (2.0,2.8) node[left]{$y=f'(x)$};
		\end{tikzpicture}	
	}
	\loigiai
	{
		Xét $g'(x)=-2f'\left(3-2x+m\right)+2x-\left(m+3\right)$.\\
		Xét phương trình $g'(x)=0$, đặt $t=3-2x+m$ thì phương trình trở thành\\ $-2\cdot \left[f'(t)-\dfrac{-t}{2}\right]=0\Leftrightarrow\hoac{
			& t=-2\\ 
			& t=4\\ 
			& t=0.}$ \\
		Từ đó, $g'(x)=0\Leftrightarrow{x_1}=\dfrac{5+m}{2},\,x_2=\dfrac{m+3}{2},x_3=\dfrac{-1+m}{2}$.\\
		Lập bảng xét dấu, đồng thời lưu ý nếu $x>x_1$ thì $t<t_1$ nên $f(x)>0$. Và các dấu đan xen nhau do các nghiệm đều làm đổi dấu đạo hàm nên suy ra $g'(x)\le 0\Leftrightarrow x\in\left[x_2;{x_1}\right]\cup\left(-\infty ;{x_3}\right]$.\\
		Vì hàm số nghịch biến trên $\left(0;1\right)$ nên \\
		$g'(x)\le 0,\,\forall x\in\left(0;1\right)$ từ đó suy ra $\hoac{
			&\dfrac{3+m}{2}\le 0<1\le\dfrac{5+m}{2}\\ 
			& 1\le\dfrac{-1+m}{2}.}$ \\
		và giải ra các giá trị nguyên thuộc $\left(-6;6\right)$ của $m$ là $-3$; $3$; $4$; $5$. }
\end{ex}

\begin{ex}%[2D1G1-3]%Câu 6
	[Chuyên Quang Trung-Bình Phước-Lần 2-2020]
	\immini{
		Cho hàm số $ y=f(x)$ có đạo hàm liên tục trên $\mathbb{R}$ và có đồ thị $ y=f'(x)$ như hình vẽ bên. Đặt $ g(x)=f\left(x-m\right)-\dfrac{1}{2}{\left(x-m-1\right)^2}+2019$, với $ m$ là tham số thực. Gọi $ S$ là tập hợp các giá trị nguyên dương của $ m$ để hàm số $ y=g(x)$ đồng biến trên khoảng $\left(5;6\right)$. Tổng tất cả các phần tử trong $ S$ bằng
		\choice
		{$ 4$}
		{$ 11$}
		{\True $ 14$}
		{$ 20$}}{
		\begin{tikzpicture}[scale=0.9,font=\footnotesize, line join=round, line cap=round, >=stealth] %Đường cong bậc 3
			\draw[thick, ->] (-2.5,0)--(3.7,0);
			\draw[thick, ->] (0,-2.8)--(0,2.8);
			\draw (3.9,0) node[below] {$x$};
			\draw (0,2.9) node[left]{$y$};
			\draw (0,0) node[below left]{$0$};
			\draw[fill] (-1,0) circle (0.5pt)node[above]{$ -1 $};
			\draw[fill] (1,0) circle (0.5pt)node[below]{$ 1$};
			\draw[fill] (3,0) circle (0.5pt)node[below]{$ 3$};
			\draw[fill] (2,0) circle (0.5pt)node[above]{$ 2$};
			\draw[fill] (0,2) circle (0.5pt)node[above left]{$ 2$};
			\draw[fill] (0,-2) circle (0.5pt)node[below left]{$ -2$};
			\draw[dashed] (-1,0)--(-1,-2)--(2,-2)--(2,0); 
			\draw[dashed](3,0)--(3,2)--(0,2);
			%			\draw[dashed](3,0)--(3,3)--(0,3);
			\draw[line width=1.2pt,smooth,samples=100,domain=-1.1:3.1] plot(\x,{1*(\x)^3-3*(\x)^2-0*(\x)+2});		
			%\draw[line width=1.2pt,smooth,samples=100,domain=-3.3:2.8] plot(\x,{0.75*(\x)^2+0.5*\x-1});
			%	\draw (2.0,2.8) node[left]{$y=f'(x)$};
	\end{tikzpicture}	}
	\loigiai
	{
		Ta có $g'(x)=f'\left(x-m\right)-\left(x-m-1\right)$.\\
		Cho $g'(x)=0\Leftrightarrow{f}'\left(x-m\right)=x-m-1$.\\
		Đặt $ x-m=t\Rightarrow f'(t)=t-1$\\
		Khi đó nghiệm của phương trình là hoành độ giao điểm của đồ thị hàm số $ y=f'(t)$ và và đường thẳng $ y=t-1$.
		\begin{center}
			\begin{tikzpicture}[scale=0.9,font=\footnotesize, line join=round, line cap=round, >=stealth] %Đường cong bậc 3
				\draw[thick, ->] (-2.5,0)--(3.7,0);
				\draw[thick, ->] (0,-2.8)--(0,2.8);
				\draw (3.9,0) node[below] {$x$};
				\draw (0,2.9) node[left]{$y$};
				\draw (0,0) node[below left]{$0$};
				\draw[fill] (-1,0) circle (0.5pt)node[above]{$ -1 $};
				\draw[fill] (1,0) circle (0.5pt)node[below]{$ 1$};
				\draw[fill] (3,0) circle (0.5pt)node[below]{$ 3$};
				\draw[fill] (2,0) circle (0.5pt)node[above]{$ 2$};
				\draw[fill] (0,2) circle (0.5pt)node[above left]{$ 2$};
				\draw[fill] (0,-2) circle (0.5pt)node[below left]{$ -2$};
				\draw[dashed] (-1,0)--(-1,-2)--(2,-2)--(2,0); 
				\draw[dashed](3,0)--(3,2)--(0,2);
				%			\draw[dashed](3,0)--(3,3)--(0,3);
				\draw[line width=1.2pt,smooth,samples=100,domain=-1.1:3.1] plot(\x,{1*(\x)^3-3*(\x)^2-0*(\x)+2});		
				%\draw[line width=1.2pt,smooth,samples=100,domain=-3.3:2.8] plot(\x,{0.75*(\x)^2+0.5*\x-1});
				%	\draw (2.0,2.8) node[left]{$y=f'(x)$};
				\coordinate (a) at ($(-1,-2)!1.2!(3,2)$);
				\coordinate (b) at ($(-1,-2)!-.2!(3,2)$);
				\draw[line width=1.2pt,smooth] (a)--(b);
			\end{tikzpicture}
		\end{center}
		Dựa vào đồ thị hàm số ta có được $f'(t)=t-1\Leftrightarrow\hoac{
			& t=-1\\ 
			& t=1\\ 
			& t=3.} $ \\
		Bảng xét dấu của $g'(t)$
		\begin{center}
			\begin{tikzpicture}
				\tkzTabInit[lgt=1.2,espcl=2.5,nocadre]
				{$t$/1, $g'(x)$ /.8} % first column
				{$-\infty$, $-1$,$1$, $3$, $+\infty$} % first row
				\tkzTabLine { ,-,0,+,0,-,0,+, } % second row
				%				\tkzTabLine {,-,z,+,t,+,} % third row
				%				\tkzTabLine {,+,d,-,z,+,} % last row
			\end{tikzpicture}
		\end{center}
		Từ bảng xét dấu ta thấy hàm số $ g(t)$ đồng biến trên khoảng $\left(-1;1\right)$ và $\left(3;+\infty\right)$.\\
		Hay $\hoac{
			&-1<t<1\\ 
			& t>3}\Leftrightarrow\hoac{
			&-1<x-m<1\\ 
			& x-m>3} \Leftrightarrow\hoac{
			& m-1<x<m+1\\ 
			& x>m+3.}$\\
		Để hàm số $ g(x)$ đồng biến trên khoảng $\left(5;6\right)$ thì $\hoac{
			& m-1\le 5<6\le m+1\\ 
			& m+3\le 5<6} \Leftrightarrow\hoac{
			& 5\le m\le 6\\ 
			& m\le 2.}$\\
		Vì $ m$ là các số nguyên dương nên $ S=\left\{ 1;2;5;6\right\}$.\\
		Vậy tổng tất cả các phần tử của $ S$ là $ 1+2+5+6=14$.}
\end{ex}

\begin{ex}%[2D1G1-3]%Câu 7
	\immini{
		Cho hàm số $ y=f(x)$ liên tục có đạo hàm trên $\mathbb{R}$. Biết hàm số $ f'(x)$ có đồ thị cho như hình vẽ bên. Có bao nhiêu giá trị nguyên của $ m$ thuộc $\left[-2019;2019\right]$ để hàm só $ g(x)=f\left(2019^x\right)-mx+2$ đồng biến trên $\left[0;1\right]$.
		\choice
		{$ 2028$}
		{$ 2019$}
		{$ 2011$}
		{\True $ 2020$}}{
		\begin{tikzpicture}[scale=0.9,font=\footnotesize, line join=round, line cap=round, >=stealth] %Đường cong bậc 3
			\draw[thick, ->] (-3.5,0)--(2.5,0);
			\draw[thick, ->] (0,-2.8)--(0,2.8);
			\draw (2.7,0) node[below] {$x$};
			\draw (0,2.9) node[left]{$y$};
			\draw (0,0) node[below left]{$0$};
			%	\draw[fill] (-1,0) circle (0.5pt)node[above]{$ -1 $};
			\draw[fill] (1,0) circle (0.5pt)node[below right]{$ 1$};
			%		\draw[fill] (3,0) circle (0.5pt)node[below]{$ 3$};
			%		\draw[fill] (2,0) circle (0.5pt)node[above]{$ 2$};
			%		\draw[fill] (0,2) circle (0.5pt)node[above left]{$ 2$};
			%		\draw[fill] (0,-2) circle (0.5pt)node[below left]{$ -2$};
			%		\draw[dashed] (-1,0)--(-1,-2)--(2,-2)--(2,0); 
			%		\draw[dashed](3,0)--(3,2)--(0,2);
			\draw[line width=1.2pt,smooth,samples=100,domain=-3.28:1.32] plot(\x,{0.667*(\x)^3+2*(\x)^2-0.667*(\x)-2});		
			%\draw[line width=1.2pt,smooth,samples=100,domain=-3.3:2.8] plot(\x,{0.75*(\x)^2+0.5*\x-1});
			%	\draw (2.0,2.8) node[left]{$y=f'(x)$};
	\end{tikzpicture}	}
	\loigiai{
		Ta có $ g'(x)=2019^x\ln 2019\cdot f'\left(2019^x\right)-m$.\\
		Ta lại có hàm số $ y=2019^x$ đồng biến trên $\left[0;1\right]$.\\
		Với $ x\in\left[0;1\right]$ thì $2019^x\in\left[1;2019\right]$ mà hàm $ y=f'(x)$ đồng biến trên $\left(1;+\infty\right)$ nên hàm $ y=f'\left(2019^x\right)$ đồng biến trên $\left[0;1\right]$.\\
		Mà $2019^x\ge 1;f'\left(2019^x\right)>0\,\forall\,x\in\left[0;1\right]$ nên hàm $ h(x)=2019^x\ln 2019\cdot f'\left(2019^x\right)$ đồng biến trên $\left[0;1\right]$.\\
		Hay $ h(x)\ge h(0)=0,\forall\,x\in\left[0;1\right]$.\\
		Do vậy hàm số $ g(x)$ đồng biến trên đoạn $\left[0;1\right]$$\Leftrightarrow g'(x)\ge 0,\forall\,x\in\left[0;1\right]$\\
		$\Leftrightarrow m\le{2019^x}\ln 2019.f'\left(2019^x\right),\forall\,x\in\left[0;1\right]$ $\Leftrightarrow m\le\underset{x\in\left[0;1\right]}{\min}\,h(x)=h(0)=0$\\
		Vì $ m$ nguyên và $ m\in\left[-2019;2019\right]\Rightarrow $có $ 2020$ giá trị $ m$ thỏa mãn yêu cầu bài toán.}
\end{ex}

\begin{ex}%[2D1G1-3]%Câu 8
	\immini{
		Cho hàm số $y=f(x)$ có đồ thị $f'(x)\,$ như hình vẽ. Có bao nhiêu giá trị nguyên $m\in\left(-2020\,;\,2020\right)$ để hàm số $g(x)=f\left(2x-3\right)\,-\ln \left(1+x^2\right)-2mx$ đồng biến trên $\left(\dfrac{1}{2};2\right)$?
		\choice
		{$ 2020$}
		{\True $ 2019$}
		{$ 2021$}
		{$ 2018$}}{
		\begin{tikzpicture}[scale=0.9,font=\footnotesize, line join=round, line cap=round, >=stealth] %Đường cong bậc 3
			\draw[thick, ->] (-2.5,0)--(2.5,0);
			\draw[thick, ->] (0,-1.8)--(0,5.8);
			\draw (2.7,0) node[below] {$x$};
			\draw (0,5.9) node[left]{$y$};
			\draw (0,0) node[below left]{$0$};
			\draw[fill] (-2,0) circle (0.5pt)node[below]{$ -2 $};
			\draw[fill] (1,0) circle (0.5pt)node[below]{$ 1$};
			\draw[fill] (-1,0) circle (0.5pt)node[below]{$-1$};
			\draw[fill] (0,4) circle (0.5pt)node[above left]{$ 2$};
			%		\draw[fill] (0,2) circle (0.5pt)node[above left]{$ 2$};
			%		\draw[fill] (0,-2) circle (0.5pt)node[below left]{$ -2$};
			\draw[dashed] (-2,0)--(-2,4)--(1,4)--(1,0); 
			%		\draw[dashed](3,0)--(3,2)--(0,2);
			\draw[line width=1.2pt,smooth,samples=100,domain=-2.1:2.1] plot(\x,{-1*(\x)^3+0*(\x)^2+3*(\x)+2});		
			%\draw[line width=1.2pt,smooth,samples=100,domain=-3.3:2.8] plot(\x,{0.75*(\x)^2+0.5*\x-1});
			%	\draw (2.0,2.8) node[left]{$y=f'(x)$};
	\end{tikzpicture}	}
	\loigiai{
		Ta có $g'(x)=2f'\left(2x-3\right)-\dfrac{2x}{1+x^2}-2m$.\\
		Hàm số $ g(x)$ đồng biến trên $\left(\dfrac{1}{2};2\right)$ khi và chỉ khi \\
		$g'(x)\ge 0,\,\,\forall x\in\left(-1;\,2\right)$\\
		$\Leftrightarrow m\le{f}'\left(2x-3\right)-\dfrac{x}{1+x^2},\,\,\forall x\in\left(\dfrac{1}{2};2\right)$\\
		$\Leftrightarrow m\le\underset{x\in\left[\dfrac{1}{2};2\right]}{\min}\,\left[f'\left(2x-3\right)-\dfrac{x}{1+x^2}\right]$. \, \,  $(1)$\\
		Đặt $ t=2x-3$, khi đó $ x\in\left(\dfrac{1}{2};2\right)\Leftrightarrow t\in\left(-2;\,1\right)$.\\
		Từ đồ thị hàm $f'(x)$ suy ra $f'(t)\ge 0,\,\,\forall t\in\left(-2;1\right)$ và $f'(t)=0$ khi $ t=-1$.\\
		Tức là $f'\left(2x-3\right)\ge 0,\,\,\forall x\in\left(\dfrac{1}{2};\,2\right)$$\Rightarrow\underset{x\in\left[\dfrac{1}{2};2\right]}{\min}\,f'\left(2x-3\right)=0$ khi $ x=1$. $(2)$\\
		Xét hàm số $ h(x)=-\dfrac{x}{1+x^2}$ trên khoảng $\left(\dfrac{1}{2};\,2\right)$.\\
		Ta có $h'(x)=\dfrac{x^2-1}{\left(1+x^2\right)^2}$ và\\
		$h'(x)=0\Leftrightarrow{x^2}-1=0\Leftrightarrow x=\pm 1$.\\
		Bảng biến thiên của hàm số $ h(x)$ trên $\left(\dfrac{1}{2};\,2\right)$ như sau
		\begin{center}
			\begin{tikzpicture}
				\tkzTabInit[lgt=1.2,espcl=2.5,nocadre]
				{$x$ /0.7, $h'(x)$ /0.7,$h(x)$ /2.5}
				{$\dfrac{1}{2}$ , $1$,$2$}
				\tkzTabLine{,-,0,+,}
				\tkzTabVar{+/$  $ ,-/$ \-\dfrac{1}{2} $, +/$ $}
			\end{tikzpicture}
		\end{center}
		Từ bảng biến thiên suy ra $ h(x)\ge-\dfrac{1}{2}$$\Rightarrow\underset{x\in\left[\dfrac{1}{2};2\right]}{\min}\,h(x)=-\dfrac{1}{2}$ khi $ x=1$. \, \,  $(3)$\\
		Từ $(1)$, $(2)$ và $(3)$ suy ra $ m\le-\dfrac{1}{2}$.\\
		Kết hợp với $ m\in\mathbb{Z}$, $ m\in\left(-2020;\,2020\right)$ thì $ m\in\left\{-2019;\,-201;\ldots ;-2;-1\right\}$.\\
		Vậy có tất cả $ 2019$ giá trị $ m$ cần tìm.}
\end{ex}

\begin{ex}%[2D1G1-3]%Câu 9
	Cho hàm số $ f(x)$ liên tục trên $\mathbb{R}$ và có đạo hàm $f'(x)=x^2\left(x-2\right)\left(x^2-6x+m\right)$ với mọi $ x\in\mathbb{R}$. Có bao nhiêu số nguyên $ m$ thuộc đoạn $\left[-2020;2020\right]$ để hàm số $ g(x)=f\left(1-x\right)$ nghịch biến trên khoảng $\left(-\infty ;-1\right)$?
	\choice
	{$ 2016$}
	{$ 2014$}
	{\True $ 2012$}
	{$ 2010$}
	\loigiai{
		Ta có \\
		$g'(x)=f'\left(1-x\right)=-\left(1-x\right)^2\left(-x-1\right)\left[\left(1-x\right)^2-6\left(1-x\right)+m\right]$
		$=\left(x-1\right)^2\left(x+1\right)\left(x^2+4x+m-5\right)$.\\
		Hàm số $ g(x)$ nghịch biến trên khoảng $\left(-\infty ;-1\right)$\\
		$\Leftrightarrow{g}'(x)\le 0,\forall x<-1$ $(*)$, (dấu \lq\lq $=$\rq\rq \, xảy ra tại hữu hạn điểm).\\
		Với $ x<-1$ thì $\left(x-1\right)^2>0$ và $ x+1<0$ nên\\
		$(*)$ $\Leftrightarrow{x^2}+4x+m-5\ge 0,\forall x<-1 \Leftrightarrow m\ge-x^2-4x+5,\forall x<-1$.\\
		Xét hàm số $ y=-x^2-4x+5$ trên khoảng $\left(-\infty ;-1\right)$, ta có bảng biến thiên
		\begin{center}
			\begin{tikzpicture}
				\tkzTabInit[lgt=1.8,espcl=2.3]
				{$x$ /1.2, $y'$ /1.2,$y$ /2}
				{$-\infty$ , $-2$,$-1$}
				\tkzTabLine{,+,0,-,}
				\tkzTabVar{-/$ -\infty $ ,+/$9 $, -/$ 8$}
			\end{tikzpicture}
		\end{center}
		Từ bảng biến thiên suy ra $ m\ge 9$.\\
		Kết hợp với $ m$ thuộc đoạn $\left[-2020;2020\right]$ và $ m$ nguyên nên $ m\in\left\{ 9;10;11;\ldots ;2020\right\}$.\\
		Vậy có $ 2012$ số nguyên $ m$ thỏa mãn đề bài.}
\end{ex}

\begin{ex}%[2D1G1-3]%Câu 10
	\immini{
		Cho hàm số $f(x)$ xác định và liên tục trên $ R$. Hàm số $y=f'(x)$ liên tục trên $\mathbb{R}$ và có đồ thị như hình vẽ bên.
		Xét hàm số $g(x)=f\left(x-2m\right)+\dfrac{1}{2}{\left(2m-x\right)^2}+2020$, với $ m$ là tham số thực. Gọi $ S$ là tập hợp các giá trị nguyên dương của $ m$ để hàm số $ y=g(x)$ nghịch biến trên khoảng $\left(3;4\right)$. Hỏi số phần tử của $ S$ bằng bao nhiêu?
		\choice
		{$4$}
		{\True $2$}
		{$3$}
		{Vô số}}
	{
		\begin{tikzpicture}[scale=0.7,>=stealth, font=\footnotesize, line join=round, line cap=round]
			\def\xmin{-3.5} \def\xmax{4.5}
			\def\ymin{-5.2} \def\ymax{4}
			\clip(\xmin,\ymin) rectangle (\xmax,\ymax);
			\draw[->] (\xmin,0)--(\xmax,0) node [below]{$x$};
			\draw[->] (0,\ymin)--(0,\ymax) node [left]{$y$};
			\node at (0,0) [below left]{$O$};
			\path
			(-3.1,3.7) coordinate (A)
			(-3,3) coordinate (B)
			(0,-2) coordinate (C)
			(0.65,-2) coordinate (D)
			(1,-1) coordinate (E)
			(3,-3) coordinate (F)
			(3.4,-5) coordinate (G);
			\draw[smooth]
			(A)..controls +(-88:0.1) and +(93:.1)..
			(B)..controls +(-87:0.3) and +(-100:8.5)..
			(C)..controls +(75:.8) and +(180:.1)..
			(D)..controls +(0:.1) and +(-105:.3)..
			(E)..controls +(70:2) and +(97:0.4)..
			(F)..controls +(-80:.1) and +(90:0.3)..
			(G);
			\draw[dashed] 
			(-3,0)node[below]{$-3$}|-(0,3)node[right]{$3$}
			(1,0)node[above]{$1$}|-(0,-1)node[left]{$-1$}
			(3,0)node[above]{$3$}|-(0,-3)node[below right]{$-3$};
			\fill 
			(0,-2) circle(1.5pt)
			(-3,3) circle(1.5pt)
			(3,-3) circle(1.5pt)
			(1,-1) circle(1.5pt);
			\node at (2.1,-4) {$y=f'(x)$};
		\end{tikzpicture}
	}
	\loigiai{
		Ta có $g'(x)=f'\left(x-2m\right)-\left(2m-x\right)$.		Đặt $h(x)=f'(x)-\left(-x\right)$.\\
		Từ đồ thị hàm số $y=f'(x)$ và đồ thị hàm số $y=-x$ trên hình vẽ suy ra \\
		$h(x)\le 0\Leftrightarrow f'(x)\le-x\Leftrightarrow\hoac{
			&-3\le x\le 1\\ 
			& x\ge 3.}$ 
		\begin{center}
			\begin{tikzpicture}[scale=0.7,>=stealth, font=\footnotesize, line join=round, line cap=round]
				\def\xmin{-3.5} \def\xmax{4.5}
				\def\ymin{-5.2} \def\ymax{4}
				\clip(\xmin,\ymin) rectangle (\xmax,\ymax);
				\draw[->] (\xmin,0)--(\xmax,0) node [below]{$x$};
				\draw[->] (0,\ymin)--(0,\ymax) node [left]{$y$};
				\node at (0,0) [below left]{$O$};
				\path
				(-3.1,3.7) coordinate (A)
				(-3,3) coordinate (B)
				(0,-2) coordinate (C)
				(0.65,-2) coordinate (D)
				(1,-1) coordinate (E)
				(3,-3) coordinate (F)
				(3.4,-5) coordinate (G);
				\draw[smooth]
				(A)..controls +(-88:0.1) and +(93:.1)..
				(B)..controls +(-87:0.3) and +(-100:8.5)..
				(C)..controls +(75:.8) and +(180:.1)..
				(D)..controls +(0:.1) and +(-105:.3)..
				(E)..controls +(70:2) and +(97:0.4)..
				(F)..controls +(-80:.1) and +(90:0.3)..
				(G);
				\draw[dashed] 
				(-3,0)node[below]{$-3$}|-(0,3)node[right]{$3$}
				(1,0)node[above]{$1$}|-(0,-1)node[left]{$-1$}
				(3,0)node[above]{$3$}|-(0,-3)node[below right]{$-3$};
				\fill 
				(0,-2) circle(1.5pt)
				(-3,3) circle(1.5pt)
				(3,-3) circle(1.5pt)
				(1,-1) circle(1.5pt);
				\draw[smooth,samples=300,domain=-3.2:3.7] plot(\x,{-(\x)});
				\node at (2.1,-4) {$y=f'(x)$};
				\node at (-1,2.1) {$y=h(x)$};
			\end{tikzpicture}
		\end{center}
		Ta có $ g'(x)=h\left(x-2m\right)\le 0\Leftrightarrow\hoac{
			&-3\le x-2m\le 1\\ 
			& x-2m\ge 3}\Leftrightarrow\hoac{
			& 2m-3\le x\le 2m+1\\ 
			& x\ge 2m+3.}$.\\
		Suy ra hàm số $ y=g(x)$ nghịch biến trên các khoảng $\left(2m-3;2m+1\right)$ và $\left(2m+3;+\infty\right)$.\\
		Do đó hàm số $ y=g(x)$ nghịch biến trên khoảng $\left(3;4\right)$ $\Leftrightarrow\hoac{
			&\heva{
				& 2m-3\le 3\\ 
				& 2m+1\ge 4}\\ 
			& 2m+3\le 3}\Leftrightarrow\hoac{
			&\dfrac{3}{2}\le m\le 3\\ 
			& m\le 0.}$ \\
		Mặt khác, do $ m$ nguyên dương nên $ m\in\left\{ 2;3\right\}\Rightarrow S=\left\{ 2;3\right\}$. Vậy số phần tử của $ S$ bằng $2$.\\
	}
	
\end{ex}

\begin{ex}%[2D1G1-3]%Câu 11
	Cho hàm số $f(x)$ có đạo hàm trên $\mathbb{R}$ là $f'(x)=\left(x-1\right)\left(x+3\right)$. Có bao nhiêu giá trị nguyên của tham số $m$ thuộc đoạn $\left[-10;20\right]$ để hàm số $y=f\left(x^2+3x-m\right)$ đồng biến trên khoảng $\left(0;2\right)$?
	\choice
	{\True $ 18$}
	{$ 17$}
	{$ 16$}
	{$ 20$}
	\loigiai{
		Ta có $y'=f'\left(x^2+3x-m\right)=\left(2x+3\right){f}'\left(x^2+3x-m\right)$.\\
		Theo đề bài ta có $f'(x)=\left(x-1\right)\left(x+3\right)$\\
		suy ra $f'(x)>0\Leftrightarrow\hoac{
			& x<-3\\ 
			& x>1}$ và $f'(x)<0\Leftrightarrow-3<x<1$ .\\
		Hàm số đồng biến trên khoảng $\left(0;2\right)$ khi $y'\ge 0,\forall x\in\left(0;2\right)$\\
		$\Leftrightarrow\left(2x+3\right){f}'\left(x^2+3x-m\right)\ge 0,\forall x\in\left(0;2\right)$.\\
		Do $x\in\left(0;2\right)$ nên $2x+3>0,\forall x\in\left(0;2\right)$. Do đó, ta có\\
		$y'\ge 0,\forall x\in\left(0;2\right)\Leftrightarrow f'\left(x^2+3x-m\right)\ge 0$\\
		$\Leftrightarrow\hoac{
			&{x^2}+3x-m\le-3\\ 
			&{x^2}+3x-m\ge 1}\Leftrightarrow\hoac{
			& m\ge{x^2}+3x+3\\ 
			& m\le{x^2}+3x-1}$\\
		$\Leftrightarrow\hoac{
			& m\ge\underset{\left[0;2\right]}{\max}\,\left(x^2+3x+3\right)\\ 
			& m\le\underset{\left[0;2\right]}{\min}\,\left(x^2+3x-1\right)} \Leftrightarrow\hoac{
			& m\ge 13\\ 
			& m\le-1}$.\\
		Do $m\in\left[-10;20\right]$, $ m\in\mathbb{Z}$ nên có $ 18$ giá trị nguyên của $m$ thỏa yêu cầu đề bài.}
\end{ex}

\begin{ex}%[2D1G1-3]%Câu 12
	Cho các hàm số $f(x)=x^3+4x+m$ và $g(x)=\left(x^2+2018\right){\left(x^2+2019\right)^2}{\left(x^2+2020\right)^3}$ . Có bao nhiêu giá trị nguyên của tham số $m\in\left[-2020;2020\right]$ để hàm số $g\left(f(x)\right)$ đồng biến trên $\left(2;+\infty\right)$ ?
	\choice
	{$2005$}
	{\True $2037$}
	{$4016$}
	{$4041$}
	\loigiai{
		Ta có $f(x)=x^3+4x+m$ và \\
		$g(x)=\left(x^2+2018\right){\left(x^2+2019\right)^2}{\left(x^2+2020\right)^3}=a_{12}{x^{12}}+a_{10}{x^{10}}+...+a_2x^2+a_0$.\\
		Suy ra $f'(x)=3x^2+4$ , $g'(x)=12a_{12}{x^{11}}+10a_{10}{x^9}+...+2a_2x$.\\
		Và có 
		\begin{eqnarray*}
			\left[g\left(f(x)\right)\right]' &=& f'(x)\left[12a_{12}{\left(f(x)\right)^{11}}+10a_{10}{\left(f(x)\right)^9}+...+2a_2f(x)\right]\\
			&=& f(x)f'(x)\left(12a_{12}{\left(f(x)\right)^{10}}+10a_{10}{\left(f(x)\right)^8}+...+2a_2\right).
		\end{eqnarray*} 
		Dễ thấy $a_{12};{a_{10}};...;{a_2};{a_0}>0$ và $f'(x)=3x^2+4>0$, $\forall x>2$.\\
		Do đó $f'(x)\left(12a_{12}{\left(f(x)\right)^{10}}+10a_{10}{\left(f(x)\right)^8}+...+2a_2\right)>0$ , $\forall x>2$.\\
		Hàm số $g\left(f(x)\right)$ đồng biến trên $\left(2;+\infty\right)$ khi $\left[g\left(f(x)\right)\right]^{'}\ge 0$, $\forall x>2$\\
		$\Rightarrow  f(x)\ge 0$, $\forall x>2 \Leftrightarrow x^3+4x+m\ge 0$, $\forall x>3 \Leftrightarrow  m\ge-x^3-4x$, $\forall x>2$\\
		$ \Rightarrow  m\ge\underset{\left[2;+\infty\right)}{\max}\,\left(-x^3-4x\right)=-16$.\\
		Vì $m\in\left[-2020;2020\right]$ và $m\in\mathbb{Z}$ nên có $2037$ giá trị thỏa mãn $m$ .}
\end{ex}

\begin{ex}%[2D1G1-3]%Câu 13
	Cho hàm số $y=f(x)$ có đạo hàm $f'(x)=x{\left(x+1\right)^2}\left(x^2+2mx+1\right)$ với mọi $x \in \mathbb{R}$. Có bao nhiêu số nguyên âm $m$ để hàm số $g(x)=f\left(2x+1\right)$ đồng biến trên khoảng $\left(3;5\right)$?
	\choice
	{\True $3$}
	{$2$}
	{$4$}
	{$6$}
	\loigiai{
		Ta có $g'(x)=2f'(2x+1)=2(2x+1)(2x+2)^2[(2x+1)^2+2m(2x+1)+1]$. 	Đặt $t=2x+1$\\
		Để hàm số $g(x)$ đồng biến trên khoảng $\left(3;5\right)$ khi và chỉ khi 
		\begin{eqnarray*}
			& & g'(x)\ge 0,\forall x\in\left(3;5\right) \\
			& \Leftrightarrow & t(t^2+2mt+1)\ge 0,\forall t\in\left(7;11\right)\Leftrightarrow{t^2}+2mt+1\ge 0,\,\,\forall t\in\left(7;11\right) \\
			&\Leftrightarrow & 2m\ge\dfrac{-t^2-1}{t},\,\,\,\forall t\in\left(7;11\right)
		\end{eqnarray*}	
		Xét hàm số $h(t)=\dfrac{-t^2-1}{t}$ trên $\left[7;11\right]$, có $h'(t)=\dfrac{-t^2+1}{t^2}$\\
		Bảng biến thiên
		\begin{center}
			\begin{tikzpicture}
				\tkzTabInit[espcl=3,lgt=1.2,nocadre]
				{$t$/0.7,$h'(t)$/0.7,$h(t)$/2.5}
				{$-\infty$,$1$,$11$,$+\infty$}
				\tkzTabLine{, ,,-,,,}
				%	\node (0) at ($(N12)+(0,-3)$) {$-\infty$};
				\node (1) at ($(N22)+(0,-0.8)$) [right] {$-\dfrac{50}{7}$};
				\node (2) at ($(N32)+(0,-2.5)$) [left] {$-\dfrac{122}{11}$};
				
				
				%				\node (3) at ($(N11+(-0.5,0))$) {};
				%				\node (4) at ($(N23)$) {};
				\fill[pattern=north east lines] (7.0,-0.7) rectangle (10,-4.4);
				\fill[pattern=north east lines] (1.5,-0.7) rectangle (4.5,-4.4);
				\draw[->] (1)--(2);	
				\draw[dashed] (4.5,-0.7)--(4.5,-4.4);
				\draw[dashed] (7.0,-0.7)--(7.0,-4.4);	
			\end{tikzpicture}		
		\end{center}
		Dựa vào BBT ta có $2m\ge\dfrac{-t^2-1}{t},\,\,\,\forall t\in\left(7;11\right)\Leftrightarrow 2m\ge\underset{\left[7;11\right]}{\max}\,h(t)\Leftrightarrow m\ge-\dfrac{50}{14}$\\
		Vì $ m\in{\mathbb{Z}^-}\Rightarrow m \in \{-3;-2;-1\}$ .
	}
\end{ex}

\begin{ex}%[2D1G1-3]%Câu 14
	Cho hàm số $y=f(x)$ có bảng biến thiên như sau\\
	\begin{center}
		\begin{tikzpicture}[>=stealth,scale = 1]
			\tkzTabInit[lgt=1,espcl=2.5,nocadre]
			{$x$ /0.7, $y'$ /0.7,$y$ /2.5}
			{$-\infty$,$0$,$2$,$+\infty$}
			\tkzTabLine{ ,-,0,+,0,-,}
			\tkzTabVar{-/$-\infty$, +/$4$,- /$0$, +/{ $+\infty$}}
		\end{tikzpicture}
	\end{center}
	Có bao nhiêu số nguyên $m<2019$ để hàm số $g(x)=f\left(x^2-2x+m\right)$ đồng biến trên khoảng $\left(1;+\infty\right)$?
	\choice
	{\True $2016$}
	{$2015$}
	{$2017$}
	{$2018$}
	\loigiai{
		Ta có $g'(x)=\left(x^2-2x+m\right)'{f}'\left(x^2-2x+m\right)=2\left(x-1\right){f}'\left(x^2-2x+m\right)$ .\\
		Hàm số $y=g(x)$ đồng biến trên khoảng $\left(1;+\infty\right)$ khi và chỉ khi $g'(x)\ge 0,\forall x\in\left(1;+\infty\right)$ và\\
		$g'(x)=0$ tại hữu hạn điểm \\
		$\Leftrightarrow 2\left(x-1\right){f}'\left(x^2-2x+m\right)\ge 0,\forall x\in\left(1;+\infty\right)$\\
		$\Leftrightarrow{f}'\left(x^2-2x+m\right)\ge 0,\forall x\in\left(1;+\infty\right)$ $\Leftrightarrow\hoac{
			&{x^2}-2x+m\ge 2,\forall x\in\left(1;+\infty\right)\\ 
			&{x^2}-2x+m\le 0,\forall x\in\left(1;+\infty\right).}$\\
		Xét hàm số $y=x^2-2x+m$, ta có bảng biến thiên
		\begin{center}
			\begin{tikzpicture}[>=stealth,scale = 1]
				\tkzTabInit[lgt=1,espcl=2.5,nocadre]
				{$x$ /0.7, $y'$ /0.7,$y$ /2.5}
				{$-\infty$,$2$,$+\infty$}
				\tkzTabLine{ ,-,0,+,}
				\tkzTabVar{+/$+\infty$, -/$m-1$, +/{$+\infty$}}
			\end{tikzpicture}
		\end{center}
		Dựa vào bảng biến thiên ta có\\
		TH1: $x^2-2x+m\ge 2,\forall x\in\left(1;+\infty\right)\Leftrightarrow m-1\ge 2\Leftrightarrow m\ge 3$ .\\
		TH2: $x^2-2x+m\le 0,\forall x\in\left(1;+\infty\right)$. Không có giá trị $m$ thỏa mãn.\\
		Vậy có $2016$ số nguyên $m<2019$ thỏa mãn yêu cầu bài toán.}
\end{ex}

\begin{ex}%[2D1G1-3]%Câu 15
	\immini{
		Cho hàm số $ y=f(x)$ có đạo hàm là hàm số $f'(x)$ trên $\mathbb{R}$. Biết rằng hàm số $ y=f'\left(x-2\right)+2$ có đồ thị như hình vẽ bên dưới. Hàm số $ f(x)$ đồng biến trên khoảng nào?
		\choice
		{$\left(-\infty ;3\right),\,\,\left(5;+\infty\right)$}
		{\True $\left(-\infty ;-1\right),\,\,\left(1;+\infty\right)$}
		{$\left(-1;1\right)$}
		{$\left(3;5\right)$}}{
		\begin{tikzpicture}[scale=0.7,font=\footnotesize, line join=round, line cap=round, >=stealth] %Đường cong bậc 3
			\draw[thick, ->] (-0.5,0)--(3.5,0);
			\draw[thick, ->] (0,-1.8)--(0,5.3);
			\draw (3.7,0) node[below] {$x$};
			\draw (0,5.4) node[left]{$y$};
			\draw (0,0) node[below left]{$0$};
			\draw[fill] (3,0) circle (0.5pt)node[below]{$ 3$};
			\draw[fill] (1,0) circle (0.5pt)node[below]{$ 1$};
			\draw[fill] (2,0) circle (0.5pt)node[above]{$2$};
			\draw[fill] (0,2) circle (0.5pt)node[left]{$ 2$};
			\draw[fill] (0,-1) circle (0.5pt)node[left]{$ -1$};
			%		\draw[fill] (0,2) circle (0.5pt)node[above left]{$ 2$};
			%		\draw[fill] (0,-2) circle (0.5pt)node[below left]{$ -2$};
			\draw[dashed] (3,0)--(3,2)--(0,2)--(1,2)--(1,0); 
			\draw[dashed](0,-1)--(2,-1)--(2,0);
			\draw[line width=1.2pt,smooth,samples=100,domain=0.6:3.4] plot(\x,{3*(\x)^2-12*(\x)+11});		
			%\draw[line width=1.2pt,smooth,samples=100,domain=-3.3:2.8] plot(\x,{0.75*(\x)^2+0.5*\x-1});
			%	\draw (2.0,2.8) node[left]{$y=f'(x)$};
	\end{tikzpicture}	}
	\loigiai{	
		Hàm số $ y=f'\left(x-2\right)+2$ có đồ thị $(C)$ như sau:\\
		\begin{center}
			\begin{tikzpicture}[scale=0.7,font=\footnotesize, line join=round, line cap=round, >=stealth] %Đường cong bậc 3
				\draw[thick, ->] (-0.5,0)--(3.5,0);
				\draw[thick, ->] (0,-1.8)--(0,5.3);
				\draw (3.7,0) node[below] {$x$};
				\draw (0,5.4) node[left]{$y$};
				\draw (0,0) node[below left]{$0$};
				\draw[fill] (3,0) circle (0.5pt)node[below]{$ 3$};
				\draw[fill] (1,0) circle (0.5pt)node[below]{$ 1$};
				\draw[fill] (2,0) circle (0.5pt)node[above]{$2$};
				\draw[fill] (0,2) circle (0.5pt)node[left]{$ 2$};
				\draw[fill] (0,-1) circle (0.5pt)node[left]{$ -1$};
				%		\draw[fill] (0,2) circle (0.5pt)node[above left]{$ 2$};
				%		\draw[fill] (0,-2) circle (0.5pt)node[below left]{$ -2$};
				\draw[dashed] (3,0)--(3,2)--(0,2)--(1,2)--(1,0); 
				\draw[dashed](0,-1)--(2,-1)--(2,0);
				\draw[line width=1.2pt,smooth,samples=100,domain=0.6:3.4] plot(\x,{3*(\x)^2-12*(\x)+11});		
				%\draw[line width=1.2pt,smooth,samples=100,domain=-3.3:2.8] plot(\x,{0.75*(\x)^2+0.5*\x-1});
				%	\draw (2.0,2.8) node[left]{$y=f'(x)$};
			\end{tikzpicture}
		\end{center}
		Dựa vào đồ thị $(C)$ ta có\\ $f'\left(x-2\right)+2>2,\forall x\in\left(-\infty ;1\right)\cup\left(3;+\infty\right)\Leftrightarrow{f}'\left(x-2\right)>0,\forall x\in\left(-\infty ;1\right)\cup\left(3;+\infty\right)$ .\\
		Đặt $ x*=x-2$ suy ra $f'\left(x*\right)>0,\forall x*\in\left(-\infty ;-1\right)\bigcup\left(1;+\infty\right)$.\\
		Vậy hàm số $ f(x)$ đồng biến trên khoảng $\left(-\infty ;-1\right),\,\,\left(1;+\infty\right)$.}
\end{ex}

\begin{ex}%[2D1G1-2]%Câu 16
	\immini{
		Cho hàm số $ y=f(x)$ có đạo hàm là hàm số $f'(x)$ trên $\mathbb{R}$. Biết rằng hàm số $ y=f'\left(x+2\right)-2$ có đồ thị như hình vẽ bên dưới. Hàm số $ f(x)$ nghịch biến trên khoảng nào?
		\choice
		{$\left(-3;-1\right),\,\,\left(1;3\right)$}
		{\True $\left(-1;1\right),\,\,\left(3;5\right)$}
		{$\left(-\infty ;-2\right),\,\,\left(0;2\right)$}
		{$\left(-5;-3\right),\,\,\left(-1;1\right)$}}{
		\begin{tikzpicture}[scale=0.7,font=\footnotesize, line join=round, line cap=round, >=stealth] %Đường cong bậc 3
			\draw[thick, ->] (-3.8,0)--(4.0,0);
			\draw[thick, ->] (0,-4.8)--(0,3.5);
			\draw (4.2,0) node[below] {$x$};
			\draw (0,3.7) node[left]{$y$};
			\draw (0,0) node[below left]{$0$};
			\draw[fill] (-3,0) circle (0.5pt)node[above]{$ -3$};
			\draw[fill] (-1,0) circle (0.5pt)node[above]{$ -1$};
			\draw[fill] (1,0) circle (0.5pt)node[above]{$ 1$};
			\draw[fill] (3,0) circle (0.5pt)node[above]{$3$};
			\draw[fill] (0,2) circle (0.5pt)node[above left]{$ 2$};
			\draw[fill] (0,-1) circle (0.5pt)node[above right]{$ -1$};
			%		\draw[fill] (0,2) circle (0.5pt)node[above left]{$ 2$};
			%		\draw[fill] (0,-2) circle (0.5pt)node[below left]{$ -2$};
			\draw[dashed] (-3,0)--(-3,-2)--(3,-2)--(3,0) (-1,0)--(-1,-2) (1,0)--(1,-2) (-3.494,0)--(-3.494,2)--(3.494,2)--(3.494,0); 
			\draw[line width=1.2pt,smooth,samples=100,domain=-3.6:3.6] plot(\x,{0.11*(\x)^4-1.11*(\x)^2-1});		
			%\draw[line width=1.2pt,smooth,samples=100,domain=-3.3:2.8] plot(\x,{0.75*(\x)^2+0.5*\x-1});
			%	\draw (2.0,2.8) node[left]{$y=f'(x)$};
	\end{tikzpicture}	}
	\loigiai{
		Hàm số $ y=f'\left(x+2\right)-2$ có đồ thị $(C)$ như sau
		\begin{center}
			\begin{tikzpicture}[scale=0.7,font=\footnotesize, line join=round, line cap=round, >=stealth] %Đường cong bậc 3
				\draw[thick, ->] (-3.8,0)--(4.0,0);
				\draw[thick, ->] (0,-4.8)--(0,3.5);
				\draw (4.2,0) node[below] {$x$};
				\draw (0,3.7) node[left]{$y$};
				\draw (0,0) node[below left]{$0$};
				\draw[fill] (-3,0) circle (0.5pt)node[above]{$ -3$};
				\draw[fill] (-1,0) circle (0.5pt)node[above]{$ -1$};
				\draw[fill] (1,0) circle (0.5pt)node[above]{$ 1$};
				\draw[fill] (3,0) circle (0.5pt)node[above]{$3$};
				\draw[fill] (0,2) circle (0.5pt)node[above left]{$ 2$};
				\draw[fill] (0,-1) circle (0.5pt)node[above right]{$ -1$};
				%		\draw[fill] (0,2) circle (0.5pt)node[above left]{$ 2$};
				%		\draw[fill] (0,-2) circle (0.5pt)node[below left]{$ -2$};
				\draw[dashed] (-3,0)--(-3,-2)--(3,-2)--(3,0) (-1,0)--(-1,-2) (1,0)--(1,-2) (-3.494,0)--(-3.494,2)--(3.494,2)--(3.494,0); 
				\draw[line width=1.2pt,smooth,samples=100,domain=-3.6:3.6] plot(\x,{0.11*(\x)^4-1.11*(\x)^2-1});		
				%\draw[line width=1.2pt,smooth,samples=100,domain=-3.3:2.8] plot(\x,{0.75*(\x)^2+0.5*\x-1});
				%	\draw (2.0,2.8) node[left]{$y=f'(x)$};
			\end{tikzpicture}
		\end{center}
		Dựa vào đồ thị $(C)$ ta có\\
		$f'\left(x+2\right)-2<-2,\forall x\in\left(-3;-1\right)\bigcup\left(1;3\right)\Leftrightarrow{f}'\left(x+2\right)<0,\forall x\in\left(-3;-1\right)\bigcup\left(1;3\right)$.\\
		Đặt $ x^*=x+2$ suy ra: $f'\left(x^*\right)<0,\forall x^*\in\left(-1;1\right)\bigcup\left(3;5\right)$.\\
		Vậy: Hàm số $ f(x)$ đồng biến trên khoảng $\left(-1;1\right),\,\,\left(3;5\right)$.}
\end{ex}

\begin{ex}%[2D1G1-2]%Câu 17
	\immini{
		Cho hàm số $ y=f(x)$ có đạo hàm là hàm số $f'(x)$ trên $\mathbb{R}$. Biết rằng hàm số $ y=f'\left(x-2\right)+2$ có đồ thị như hình vẽ bên dưới. Hàm số $ f(x)$ nghịch biến trên khoảng nào?
		\choice
		{$\left(-\infty ;2\right)$}
		{\True $\left(-1;1\right)$}
		{$\left(\dfrac{3}{2};\dfrac{5}{2}\right)$}
		{$\left(2;+\infty\right)$}}{
		\begin{tikzpicture}[scale=0.7,font=\footnotesize, line join=round, line cap=round, >=stealth] %Đường cong bậc 3
			\draw[thick, ->] (-0.5,0)--(3.5,0);
			\draw[thick, ->] (0,-1.8)--(0,5.3);
			\draw (3.7,0) node[below] {$x$};
			\draw (0,5.4) node[left]{$y$};
			\draw (0,0) node[below left]{$0$};
			\draw[fill] (3,0) circle (0.5pt)node[below]{$ 3$};
			\draw[fill] (1,0) circle (0.5pt)node[below]{$ 1$};
			\draw[fill] (2,0) circle (0.5pt)node[above]{$2$};
			\draw[fill] (0,2) circle (0.5pt)node[left]{$ 2$};
			\draw[fill] (0,-1) circle (0.5pt)node[left]{$ -1$};
			%		\draw[fill] (0,2) circle (0.5pt)node[above left]{$ 2$};
			%		\draw[fill] (0,-2) circle (0.5pt)node[below left]{$ -2$};
			\draw[dashed] (3,0)--(3,2)--(0,2)--(1,2)--(1,0); 
			\draw[dashed](0,-1)--(2,-1)--(2,0);
			\draw[line width=1.2pt,smooth,samples=100,domain=0.6:3.4] plot(\x,{3*(\x)^2-12*(\x)+11});		
			%\draw[line width=1.2pt,smooth,samples=100,domain=-3.3:2.8] plot(\x,{0.75*(\x)^2+0.5*\x-1});
			%	\draw (2.0,2.8) node[left]{$y=f'(x)$};
	\end{tikzpicture}	}
	\loigiai{
		Hàm số $ y=f'\left(x-2\right)+2$ có đồ thị $(C)$ như sau
		\begin{center}
			\begin{tikzpicture}[scale=0.7,font=\footnotesize, line join=round, line cap=round, >=stealth] %Đường cong bậc 3
				\draw[thick, ->] (-0.5,0)--(3.5,0);
				\draw[thick, ->] (0,-1.8)--(0,5.3);
				\draw (3.7,0) node[below] {$x$};
				\draw (0,5.4) node[left]{$y$};
				\draw (0,0) node[below left]{$0$};
				\draw[fill] (3,0) circle (0.5pt)node[below]{$ 3$};
				\draw[fill] (1,0) circle (0.5pt)node[below]{$ 1$};
				\draw[fill] (2,0) circle (0.5pt)node[above]{$2$};
				\draw[fill] (0,2) circle (0.5pt)node[left]{$ 2$};
				\draw[fill] (0,-1) circle (0.5pt)node[left]{$ -1$};
				%		\draw[fill] (0,2) circle (0.5pt)node[above left]{$ 2$};
				%		\draw[fill] (0,-2) circle (0.5pt)node[below left]{$ -2$};
				\draw[dashed] (3,0)--(3,2)--(0,2)--(1,2)--(1,0); 
				\draw[dashed](0,-1)--(2,-1)--(2,0);
				\draw[line width=1.2pt,smooth,samples=100,domain=0.6:3.4] plot(\x,{3*(\x)^2-12*(\x)+11});		
				%\draw[line width=1.2pt,smooth,samples=100,domain=-3.3:2.8] plot(\x,{0.75*(\x)^2+0.5*\x-1});
				%	\draw (2.0,2.8) node[left]{$y=f'(x)$};
			\end{tikzpicture}
		\end{center}
		Dựa vào đồ thị $(C)$ ta có\\
		$f'\left(x-2\right)+2<2,\forall x\in\left(1;3\right)\Leftrightarrow{f}'\left(x-2\right)<0,\forall x\in\left(1;3\right)$.\\
		Đặt $ x^*=x-2$ thì $f'\left(x^*\right)<0,\forall x^*\in\left(-1;1\right)$.\\
		Vậy: Hàm số $ f(x)$ nghịch biến trên khoảng $\left(-1;1\right)$.\\
		Cách khác:\\
		Tịnh tiến sang trái hai đơn vị và xuống dưới $2$ đơn vị thì từ đồ thị $(C)$ sẽ thành đồ thị của hàm$ y=f'(x)$. Khi đó $f'(x)<0,\forall x\in\left(-1;1\right)$.\\
		Vậy hàm số $ f(x)$ nghịch biến trên khoảng $\left(-1;1\right)$.}
\end{ex}

\begin{ex}%[2D1G1-2]%Câu 18
	Cho hàm số $y=f(x)$ có đạo hàm cấp $ 3$ liên tục trên $\mathbb{R}$ và thỏa mãn $f(x)\cdot f'''(x)=x{\left(x-1\right)^2}{\left(x+4\right)^3}$ với mọi $x\in\mathbb{R}$ và $g(x)=\left[f'(x)\right]^2-2f(x)\cdot f''(x)$. Hàm số $h(x)=g\left(x^2-2x\right)$ đồng biến trên khoảng nào dưới đây?
	\choice
	{$\left(-\infty ;1\right)$}
	{$\left(2;+\infty\right)$}
	{$\left(0;1\right)$}
	{\True $\left(1;2\right)$}
	\loigiai{		
		Ta có $g'(x)=2f''(x){f}'(x)-2f'(x)\cdot f''(x)-2f(x)\cdot f'''(x)=-2f(x)\cdot f'''(x);$\\
		Khi đó $\left(h(x)\right)'=\left(2x-2\right){g}'\left(x^2-2x\right)=-2\left(2x-2\right)\left(x^2-2x\right){\left(x^2-2x-1\right)^2}{\left(x^2-2x+4\right)^3}$\\
		$h'(x)=0\Leftrightarrow\hoac{
			& x=0\\ 
			& x=1\\ 
			& x=2\\ 
			& x=1\pm\sqrt{2}.}$ 
		Ta có bảng xét dấu của $h'(x)$
		\begin{center}
			\begin{tikzpicture}
				\tkzTabInit[lgt=1.2,espcl=2,nocadre]
				{$t$/0.7, $h'(x)$ /.7} % first column
				{$-\infty$, $1-\sqrt{2}$,$0$, $1$,$2$,$1+\sqrt{2}$, $+\infty$} % first row
				\tkzTabLine { ,+,0,-,0,+,0,-,0,+,0,- } % second row
				%				\tkzTabLine {,-,z,+,t,+,} % third row
				%				\tkzTabLine {,+,d,-,z,+,} % last row
			\end{tikzpicture}
		\end{center}
		Suy ra hàm số $h(x)=g\left(x^2-2x\right)$ đồng biến trên khoảng $\left(1;2\right)$.}
\end{ex}

\begin{ex}%[2D1G1-2]%Câu 19
	Cho hàm số $ y=f(x)$ xác định trên $\mathbb{R}$. Hàm số $ y=g(x)=f'\left(2x+3\right)+2$ có đồ thị là một parabol với tọa độ đỉnh $ I\left(2;-1\right)$ và đi qua điểm $ A\left(1;2\right)$. Hỏi hàm số $ y=f(x)$ nghịch biến trên khoảng nào dưới đây?
	\choice
	{\True $\left(5;9\right)$}
	{$\left(1;2\right)$}
	{$\left(-\infty ;9\right)$}
	{$\left(1;3\right)$}
	\loigiai{	
		Xét hàm số $ g(x)=f'\left(2x+3\right)+2$ có đồ thị là một Parabol nên có phương trình dạng $ y=g(x)=a{x^2}+bx+c\,\,\,\,(P)$.\\
		Vì $(P)$ có đỉnh $ I\left(2;-1\right)$ nên $\heva{
			&\dfrac{-b}{2a}=2\\ 
			& g(2)=-1} \Leftrightarrow\heva{
			&-b=4a\\ 
			& 4a+2b+c=-1} \Leftrightarrow\heva{
			& 4a+b=0\\ 
			& 4a+2b+c=-1}$.\\
		Vì $(P)$ đi qua điểm $ A\left(1;2\right)$ nên $ g(1)=2\Leftrightarrow a+b+c=2$.\\
		Ta có hệ phương trình $\heva{
			& 4a+b=0\\ 
			& 4a+2b+c=-1\\ 
			& a+b+c=2} \Leftrightarrow\heva{
			& a=3\\ 
			& b=-12\\ 
			& c=11}$ nên $ g(x)=3x^2-12x+11$.\\
		Đồ thị của hàm $ y=g(x)$ là
		\begin{center}
			\begin{tikzpicture}[scale=0.7,font=\footnotesize, line join=round, line cap=round, >=stealth] %Đường cong bậc 3
				\draw[thick, ->] (-0.5,0)--(3.5,0);
				\draw[thick, ->] (0,-1.8)--(0,5.3);
				\draw (3.7,0) node[below] {$x$};
				\draw (0,5.4) node[left]{$y$};
				\draw (0,0) node[below left]{$0$};
				\draw[fill] (3,0) circle (0.5pt)node[below]{$ 3$};
				\draw[fill] (1,0) circle (0.5pt)node[below]{$ 1$};
				\draw[fill] (2,0) circle (0.5pt)node[above]{$2$};
				\draw[fill] (0,2) circle (0.5pt)node[left]{$ 2$};
				\draw[fill] (0,-1) circle (0.5pt)node[left]{$ -1$};
				%		\draw[fill] (0,2) circle (0.5pt)node[above left]{$ 2$};
				%		\draw[fill] (0,-2) circle (0.5pt)node[below left]{$ -2$};
				\draw[dashed] (3,0)--(3,2)--(0,2)--(1,2)--(1,0) (3.2,2)--(3,2); 
				\draw[dashed](0,-1)--(2,-1)--(2,0);
				\draw[line width=1.2pt,smooth,samples=100,domain=0.6:3.4] plot(\x,{3*(\x)^2-12*(\x)+11});		
				%\draw[line width=1.2pt,smooth,samples=100,domain=-3.3:2.8] plot(\x,{0.75*(\x)^2+0.5*\x-1});
				%	\draw (2.0,2.8) node[left]{$y=f'(x)$};
			\end{tikzpicture}	
		\end{center}
		Theo đồ thị ta thấy $ f'(2x+3)\le 0\Leftrightarrow f'(2x+3)+2\le 2\Leftrightarrow 1\le x\le 3$.\\
		Đặt $ t=2x+3\Leftrightarrow x=\dfrac{t-3}{2}$ khi đó $ f'(t)\le 0\Leftrightarrow 1\le\dfrac{t-3}{2}\le 3\Leftrightarrow 5\le t\le 9$.\\
		Vậy $ y=f(x)$ nghịch biến trên khoảng $\left(5;9\right)$.}
\end{ex}

\begin{ex}%[2D1G1-2]%Câu 20
	\immini{
		Cho hàm số $ y=f(x)$, hàm số $f'(x)=x^3+a{x^2}+bx+c\left(a,b,c\in\mathbb{R}\right)$ có đồ thị như hình vẽ bên.
		Hàm số $ g(x)=f\left(f'(x)\right)$ nghịch biến trên khoảng nào dưới đây?
		\choice
		{$\left(1;+\infty\right)$}
		{\True $\left(-\infty ;-2\right)$}
		{$\left(-1;0\right)$}
		{$\left(-\dfrac{\sqrt{3}}{3};\dfrac{\sqrt{3}}{3}\right)$}}{
		\begin{tikzpicture}[scale=0.8,font=\footnotesize, line join=round, line cap=round, >=stealth] %Đường cong bậc 3
			\draw[thick, ->] (-1.7,0)--(1.7,0);
			\draw[thick, ->] (0,-2.7)--(0,3.0);
			\draw (1.9,0) node[below] {$x$};
			\draw (0,3.2) node[left]{$y$};
			\draw (0,0) node[below left]{$0$};
			\draw[fill] (-1,0) circle (0.5pt)node[above left]{$ -1 $};
			\draw[fill] (1,0) circle (0.5pt)node[below right]{$ 1$};
			\draw[line width=1.2pt,smooth,samples=100,domain=-1.3:1.3] plot(\x,{2.667*(\x)^3+0*(\x)^2-2.667*\x});		
			%\draw[line width=1.2pt,smooth,samples=100,domain=-3.3:2.8] plot(\x,{0.75*(\x)^2+0.5*\x-1});
		\end{tikzpicture}	
	}
	\loigiai{	
		Vì các điểm $\left(-1;0\right),\left(0;0\right),\left(1;0\right)$ thuộc đồ thị hàm số $ y=f'(x)$ nên ta có hệ\\
		$\heva{
			&-1+a-b+c=0\\ 
			& c=0\\ 
			& 1+a+b+c=0} \Leftrightarrow\heva{
			& a=0\\ 
			& b=-1\\ 
			& c=0} \Rightarrow {f}'(x)=x^3-x\Rightarrow f''(x)=3x^2-1$.\\
		Ta có $ g(x)=f\left(f'(x)\right)\Rightarrow{g}'(x)=f'\left(f'(x)\right)\cdot f''(x)$.\\
		Xét \\
		$g'(x)=0\Leftrightarrow{g}'(x)=f'\left(f'(x)\right)\cdot f''(x)=0$\\
		$\Leftrightarrow {f}'\left(x^3-x\right)\left(3x^2-1\right)=0\Leftrightarrow\hoac{
			&{x^3}-x=0\\ 
			&{x^3}-x=1\\ 
			&{x^3}-x=-1\\ 
			& 3x^2-1=0} \Leftrightarrow \hoac{
			& x=\pm 1\\ 
			& x=0\\ 
			& x=x_1(x_1\approx 1,325)\\ 
			& x=x_2(x_2\approx-1,325)\\ 
			& x=\pm\dfrac{\sqrt{3}}{3}.}$\\
		Bảng biến thiên
		\begin{center}
			\begin{tikzpicture}
				\tkzTabInit[lgt=1.2,espcl=2,nocadre]
				{$t$/0.7, $h'(x)$ /.7} % first column
				{$-\infty$, $-1{,}325$,$-1$, $-\dfrac{\sqrt{3}}{3}$,$0$,$\dfrac{\sqrt{3}}{3}$,$1$,$1{,}325$, $+\infty$} % first row
				\tkzTabLine { ,-,0,+,0,-,0,+,0,-,0,+,0,-,0,+, } % second row
				%				\tkzTabLine {,-,z,+,t,+,} % third row
				%				\tkzTabLine {,+,d,-,z,+,} % last row
			\end{tikzpicture}
		\end{center}
		Dựa vào bảng biến thiên ta có $ g(x)$ nghịch biến trên $\left(-\infty ;-2\right)$}
\end{ex}
\Closesolutionfile{ans}
\indapan{10}{ans/CD1/Muc_9_10}
\chapter{ĐỌC ĐỒ THỊ CỦA HÀM SỐ}
\begin{Solution}{1}
C
\end{Solution}
\begin{Solution}{3}
B
\end{Solution}
\begin{Solution}{4}
A
\end{Solution}
\begin{Solution}{5}
A
\end{Solution}
\begin{Solution}{6}
A
\end{Solution}
\begin{Solution}{7}
B
\end{Solution}
\begin{Solution}{8}
A
\end{Solution}
\begin{Solution}{9}
C
\end{Solution}
\begin{Solution}{10}
B
\end{Solution}
\begin{Solution}{11}
C
\end{Solution}
\begin{Solution}{12}
D
\end{Solution}
\begin{Solution}{13}
B
\end{Solution}
\begin{Solution}{14}
D
\end{Solution}
\begin{Solution}{15}
A
\end{Solution}
\begin{Solution}{16}
B
\end{Solution}
\begin{Solution}{17}
C
\end{Solution}
\begin{Solution}{18}
C
\end{Solution}
\begin{Solution}{19}
C
\end{Solution}
\begin{Solution}{20}
B
\end{Solution}
\begin{Solution}{21}
C
\end{Solution}
\begin{Solution}{22}
B
\end{Solution}
\begin{Solution}{23}
D
\end{Solution}
\begin{Solution}{24}
B
\end{Solution}
\begin{Solution}{25}
D
\end{Solution}
\begin{Solution}{26}
D
\end{Solution}
\begin{Solution}{27}
B
\end{Solution}
\begin{Solution}{28}
A
\end{Solution}
\begin{Solution}{29}
C
\end{Solution}
\begin{Solution}{30}
B
\end{Solution}
\begin{Solution}{31}
D
\end{Solution}
\begin{Solution}{32}
B
\end{Solution}
\begin{Solution}{33}
B
\end{Solution}
\begin{Solution}{34}
C
\end{Solution}
\begin{Solution}{35}
D
\end{Solution}
\begin{Solution}{36}
B
\end{Solution}
\begin{Solution}{37}
B
\end{Solution}
\begin{Solution}{38}
A
\end{Solution}
\begin{Solution}{39}
A
\end{Solution}
\begin{Solution}{40}
D
\end{Solution}
\begin{Solution}{41}
C
\end{Solution}
\begin{Solution}{42}
B
\end{Solution}
\begin{Solution}{43}
A
\end{Solution}
\begin{Solution}{44}
A
\end{Solution}
\begin{Solution}{45}
D
\end{Solution}
\begin{Solution}{46}
C
\end{Solution}
\begin{Solution}{47}
A
\end{Solution}
\begin{Solution}{48}
B
\end{Solution}
\begin{Solution}{49}
B
\end{Solution}
\begin{Solution}{50}
B
\end{Solution}
\begin{Solution}{51}
A
\end{Solution}
\begin{Solution}{52}
A
\end{Solution}
\begin{Solution}{53}
C
\end{Solution}
\begin{Solution}{54}
C
\end{Solution}
\begin{Solution}{55}
C
\end{Solution}
\begin{Solution}{56}
B
\end{Solution}
\begin{Solution}{57}
C
\end{Solution}
\begin{Solution}{58}
C
\end{Solution}
\begin{Solution}{59}
B
\end{Solution}
\begin{Solution}{60}
C
\end{Solution}
\begin{Solution}{61}
A
\end{Solution}
\begin{Solution}{62}
B
\end{Solution}
\begin{Solution}{63}
B
\end{Solution}
\begin{Solution}{64}
D
\end{Solution}
\begin{Solution}{65}
D
\end{Solution}
\begin{Solution}{66}
B
\end{Solution}
\begin{Solution}{67}
A
\end{Solution}
\begin{Solution}{68}
D
\end{Solution}

\begin{Solution}{1}
D
\end{Solution}
\begin{Solution}{2}
C
\end{Solution}
\begin{Solution}{3}
C
\end{Solution}
\begin{Solution}{4}
A
\end{Solution}
\begin{Solution}{5}
B
\end{Solution}
\begin{Solution}{6}
D
\end{Solution}
\begin{Solution}{7}
C
\end{Solution}
\begin{Solution}{8}
D
\end{Solution}
\begin{Solution}{9}
A
\end{Solution}
\begin{Solution}{10}
B
\end{Solution}
\begin{Solution}{11}
D
\end{Solution}
\begin{Solution}{12}
A
\end{Solution}
\begin{Solution}{13}
D
\end{Solution}
\begin{Solution}{14}
B
\end{Solution}
\begin{Solution}{15}
B
\end{Solution}
\begin{Solution}{16}
C
\end{Solution}
\begin{Solution}{1}
A
\end{Solution}
\begin{Solution}{2}
B
\end{Solution}
\begin{Solution}{3}
D
\end{Solution}
\begin{Solution}{4}
D
\end{Solution}
\begin{Solution}{5}
C
\end{Solution}
\begin{Solution}{6}
A
\end{Solution}
\begin{Solution}{7}
D
\end{Solution}
\begin{Solution}{8}
B
\end{Solution}
\begin{Solution}{9}
C
\end{Solution}
\begin{Solution}{10}
C
\end{Solution}
\begin{Solution}{1}
D
\end{Solution}
\begin{Solution}{2}
D
\end{Solution}
\begin{Solution}{3}
B
\end{Solution}
\begin{Solution}{4}
C
\end{Solution}
\begin{Solution}{5}
D
\end{Solution}
\begin{Solution}{6}
A
\end{Solution}
\begin{Solution}{7}
C
\end{Solution}
\begin{Solution}{8}
B
\end{Solution}
\begin{Solution}{9}
A
\end{Solution}
\begin{Solution}{10}
C
\end{Solution}
\begin{Solution}{11}
D
\end{Solution}
\begin{Solution}{12}
C
\end{Solution}
\begin{Solution}{13}
A
\end{Solution}
\begin{Solution}{14}
D
\end{Solution}
\begin{Solution}{15}
A
\end{Solution}
\begin{Solution}{16}
A
\end{Solution}
\begin{Solution}{17}
B
\end{Solution}
\begin{Solution}{18}
C
\end{Solution}
\begin{Solution}{19}
C
\end{Solution}
\begin{Solution}{20}
A
\end{Solution}
\begin{Solution}{21}
D
\end{Solution}
\begin{Solution}{22}
C
\end{Solution}
\begin{Solution}{23}
A
\end{Solution}
\begin{Solution}{24}
C
\end{Solution}
\begin{Solution}{25}
A
\end{Solution}
\begin{Solution}{26}
B
\end{Solution}
\begin{Solution}{27}
B
\end{Solution}
\begin{Solution}{28}
D
\end{Solution}
\begin{Solution}{29}
B
\end{Solution}
\begin{Solution}{30}
D
\end{Solution}
\begin{Solution}{31}
D
\end{Solution}
\begin{Solution}{32}
C
\end{Solution}
\begin{Solution}{33}
D
\end{Solution}
\begin{Solution}{34}
C
\end{Solution}
\begin{Solution}{35}
D
\end{Solution}
\begin{Solution}{36}
D
\end{Solution}
\begin{Solution}{37}
D
\end{Solution}
\begin{Solution}{38}
D
\end{Solution}
\begin{Solution}{39}
D
\end{Solution}
\begin{Solution}{40}
C
\end{Solution}
\begin{Solution}{41}
A
\end{Solution}
\begin{Solution}{1}
A
\end{Solution}
\begin{Solution}{2}
B
\end{Solution}
\begin{Solution}{3}
C
\end{Solution}
\begin{Solution}{4}
A
\end{Solution}
\begin{Solution}{5}
A
\end{Solution}
\begin{Solution}{6}
C
\end{Solution}
\begin{Solution}{7}
C
\end{Solution}
\begin{Solution}{8}
B
\end{Solution}
\begin{Solution}{9}
C
\end{Solution}
\begin{Solution}{10}
B
\end{Solution}
\begin{Solution}{11}
A
\end{Solution}
\begin{Solution}{12}
B
\end{Solution}
\begin{Solution}{13}
B
\end{Solution}
\begin{Solution}{14}
B
\end{Solution}
\begin{Solution}{15}
A
\end{Solution}
\begin{Solution}{16}
B
\end{Solution}
\begin{Solution}{17}
A
\end{Solution}
\begin{Solution}{18}
D
\end{Solution}
\begin{Solution}{19}
C
\end{Solution}
\begin{Solution}{20}
C
\end{Solution}
\begin{Solution}{21}
A
\end{Solution}
\begin{Solution}{22}
C
\end{Solution}
\begin{Solution}{23}
C
\end{Solution}
\begin{Solution}{24}
A
\end{Solution}
\begin{Solution}{25}
B
\end{Solution}
\begin{Solution}{26}
B
\end{Solution}
\begin{Solution}{27}
A
\end{Solution}
\begin{Solution}{28}
A
\end{Solution}
\begin{Solution}{29}
C
\end{Solution}
\begin{Solution}{30}
B
\end{Solution}
\begin{Solution}{31}
A
\end{Solution}
\begin{Solution}{32}
C
\end{Solution}
\begin{Solution}{33}
B
\end{Solution}
\begin{Solution}{34}
A
\end{Solution}
\begin{Solution}{35}
B
\end{Solution}
\begin{Solution}{36}
B
\end{Solution}
\begin{Solution}{37}
B
\end{Solution}
\begin{Solution}{38}
D
\end{Solution}
\begin{Solution}{39}
B
\end{Solution}
\begin{Solution}{40}
A
\end{Solution}
\begin{Solution}{41}
D
\end{Solution}
\begin{Solution}{42}
D
\end{Solution}
\begin{Solution}{43}
A
\end{Solution}
\begin{Solution}{44}
D
\end{Solution}
\begin{Solution}{45}
C
\end{Solution}
\begin{Solution}{46}
B
\end{Solution}
\begin{Solution}{47}
A
\end{Solution}
\begin{Solution}{48}
D
\end{Solution}
\begin{Solution}{49}
B
\end{Solution}
\begin{Solution}{50}
B
\end{Solution}
\begin{Solution}{51}
D
\end{Solution}
\begin{Solution}{52}
C
\end{Solution}
\begin{Solution}{53}
C
\end{Solution}
\begin{Solution}{54}
B
\end{Solution}
\begin{Solution}{55}
D
\end{Solution}
\begin{Solution}{56}
B
\end{Solution}
\begin{Solution}{57}
C
\end{Solution}
\begin{Solution}{58}
A
\end{Solution}
\begin{Solution}{59}
A
\end{Solution}
\begin{Solution}{60}
B
\end{Solution}
\begin{Solution}{61}
D
\end{Solution}
\begin{Solution}{62}
D
\end{Solution}
\begin{Solution}{63}
B
\end{Solution}
\begin{Solution}{64}
A
\end{Solution}
\begin{Solution}{65}
D
\end{Solution}
\begin{Solution}{66}
C
\end{Solution}
\begin{Solution}{67}
A
\end{Solution}
\begin{Solution}{68}
A
\end{Solution}
\begin{Solution}{69}
D
\end{Solution}
\begin{Solution}{70}
C
\end{Solution}
\begin{Solution}{71}
B
\end{Solution}
\begin{Solution}{72}
A
\end{Solution}
\begin{Solution}{73}
C
\end{Solution}
\begin{Solution}{74}
C
\end{Solution}
\begin{Solution}{75}
C
\end{Solution}
\begin{Solution}{76}
A
\end{Solution}
\begin{Solution}{77}
C
\end{Solution}
\begin{Solution}{78}
B
\end{Solution}
\begin{Solution}{79}
D
\end{Solution}
\begin{Solution}{80}
B
\end{Solution}

\chapter{TƯƠNG GIAO ĐỒ THỊ HÀM SỐ}
\begin{Solution}{1}
C
\end{Solution}
\begin{Solution}{3}
B
\end{Solution}
\begin{Solution}{4}
A
\end{Solution}
\begin{Solution}{5}
A
\end{Solution}
\begin{Solution}{6}
A
\end{Solution}
\begin{Solution}{7}
B
\end{Solution}
\begin{Solution}{8}
A
\end{Solution}
\begin{Solution}{9}
C
\end{Solution}
\begin{Solution}{10}
B
\end{Solution}
\begin{Solution}{11}
C
\end{Solution}
\begin{Solution}{12}
D
\end{Solution}
\begin{Solution}{13}
B
\end{Solution}
\begin{Solution}{14}
D
\end{Solution}
\begin{Solution}{15}
A
\end{Solution}
\begin{Solution}{16}
B
\end{Solution}
\begin{Solution}{17}
C
\end{Solution}
\begin{Solution}{18}
C
\end{Solution}
\begin{Solution}{19}
C
\end{Solution}
\begin{Solution}{20}
B
\end{Solution}
\begin{Solution}{21}
C
\end{Solution}
\begin{Solution}{22}
B
\end{Solution}
\begin{Solution}{23}
D
\end{Solution}
\begin{Solution}{24}
B
\end{Solution}
\begin{Solution}{25}
D
\end{Solution}
\begin{Solution}{26}
D
\end{Solution}
\begin{Solution}{27}
B
\end{Solution}
\begin{Solution}{28}
A
\end{Solution}
\begin{Solution}{29}
C
\end{Solution}
\begin{Solution}{30}
B
\end{Solution}
\begin{Solution}{31}
D
\end{Solution}
\begin{Solution}{32}
B
\end{Solution}
\begin{Solution}{33}
B
\end{Solution}
\begin{Solution}{34}
C
\end{Solution}
\begin{Solution}{35}
D
\end{Solution}
\begin{Solution}{36}
B
\end{Solution}
\begin{Solution}{37}
B
\end{Solution}
\begin{Solution}{38}
A
\end{Solution}
\begin{Solution}{39}
A
\end{Solution}
\begin{Solution}{40}
D
\end{Solution}
\begin{Solution}{41}
C
\end{Solution}
\begin{Solution}{42}
B
\end{Solution}
\begin{Solution}{43}
A
\end{Solution}
\begin{Solution}{44}
A
\end{Solution}
\begin{Solution}{45}
D
\end{Solution}
\begin{Solution}{46}
C
\end{Solution}
\begin{Solution}{47}
A
\end{Solution}
\begin{Solution}{48}
B
\end{Solution}
\begin{Solution}{49}
B
\end{Solution}
\begin{Solution}{50}
B
\end{Solution}
\begin{Solution}{51}
A
\end{Solution}
\begin{Solution}{52}
A
\end{Solution}
\begin{Solution}{53}
C
\end{Solution}
\begin{Solution}{54}
C
\end{Solution}
\begin{Solution}{55}
C
\end{Solution}
\begin{Solution}{56}
B
\end{Solution}
\begin{Solution}{57}
C
\end{Solution}
\begin{Solution}{58}
C
\end{Solution}
\begin{Solution}{59}
B
\end{Solution}
\begin{Solution}{60}
C
\end{Solution}
\begin{Solution}{61}
A
\end{Solution}
\begin{Solution}{62}
B
\end{Solution}
\begin{Solution}{63}
B
\end{Solution}
\begin{Solution}{64}
D
\end{Solution}
\begin{Solution}{65}
D
\end{Solution}
\begin{Solution}{66}
B
\end{Solution}
\begin{Solution}{67}
A
\end{Solution}
\begin{Solution}{68}
D
\end{Solution}

\begin{Solution}{1}
D
\end{Solution}
\begin{Solution}{2}
C
\end{Solution}
\begin{Solution}{3}
C
\end{Solution}
\begin{Solution}{4}
A
\end{Solution}
\begin{Solution}{5}
B
\end{Solution}
\begin{Solution}{6}
D
\end{Solution}
\begin{Solution}{7}
C
\end{Solution}
\begin{Solution}{8}
D
\end{Solution}
\begin{Solution}{9}
A
\end{Solution}
\begin{Solution}{10}
B
\end{Solution}
\begin{Solution}{11}
D
\end{Solution}
\begin{Solution}{12}
A
\end{Solution}
\begin{Solution}{13}
D
\end{Solution}
\begin{Solution}{14}
B
\end{Solution}
\begin{Solution}{15}
B
\end{Solution}
\begin{Solution}{16}
C
\end{Solution}
\begin{Solution}{1}
A
\end{Solution}
\begin{Solution}{2}
B
\end{Solution}
\begin{Solution}{3}
D
\end{Solution}
\begin{Solution}{4}
D
\end{Solution}
\begin{Solution}{5}
C
\end{Solution}
\begin{Solution}{6}
A
\end{Solution}
\begin{Solution}{7}
D
\end{Solution}
\begin{Solution}{8}
B
\end{Solution}
\begin{Solution}{9}
C
\end{Solution}
\begin{Solution}{10}
C
\end{Solution}
\begin{Solution}{1}
D
\end{Solution}
\begin{Solution}{2}
D
\end{Solution}
\begin{Solution}{3}
B
\end{Solution}
\begin{Solution}{4}
C
\end{Solution}
\begin{Solution}{5}
D
\end{Solution}
\begin{Solution}{6}
A
\end{Solution}
\begin{Solution}{7}
C
\end{Solution}
\begin{Solution}{8}
B
\end{Solution}
\begin{Solution}{9}
A
\end{Solution}
\begin{Solution}{10}
C
\end{Solution}
\begin{Solution}{11}
D
\end{Solution}
\begin{Solution}{12}
C
\end{Solution}
\begin{Solution}{13}
A
\end{Solution}
\begin{Solution}{14}
D
\end{Solution}
\begin{Solution}{15}
A
\end{Solution}
\begin{Solution}{16}
A
\end{Solution}
\begin{Solution}{17}
B
\end{Solution}
\begin{Solution}{18}
C
\end{Solution}
\begin{Solution}{19}
C
\end{Solution}
\begin{Solution}{20}
A
\end{Solution}
\begin{Solution}{21}
D
\end{Solution}
\begin{Solution}{22}
C
\end{Solution}
\begin{Solution}{23}
A
\end{Solution}
\begin{Solution}{24}
C
\end{Solution}
\begin{Solution}{25}
A
\end{Solution}
\begin{Solution}{26}
B
\end{Solution}
\begin{Solution}{27}
B
\end{Solution}
\begin{Solution}{28}
D
\end{Solution}
\begin{Solution}{29}
B
\end{Solution}
\begin{Solution}{30}
D
\end{Solution}
\begin{Solution}{31}
D
\end{Solution}
\begin{Solution}{32}
C
\end{Solution}
\begin{Solution}{33}
D
\end{Solution}
\begin{Solution}{34}
C
\end{Solution}
\begin{Solution}{35}
D
\end{Solution}
\begin{Solution}{36}
D
\end{Solution}
\begin{Solution}{37}
D
\end{Solution}
\begin{Solution}{38}
D
\end{Solution}
\begin{Solution}{39}
D
\end{Solution}
\begin{Solution}{40}
C
\end{Solution}
\begin{Solution}{41}
A
\end{Solution}
\begin{Solution}{1}
A
\end{Solution}
\begin{Solution}{2}
B
\end{Solution}
\begin{Solution}{3}
C
\end{Solution}
\begin{Solution}{4}
A
\end{Solution}
\begin{Solution}{5}
A
\end{Solution}
\begin{Solution}{6}
C
\end{Solution}
\begin{Solution}{7}
C
\end{Solution}
\begin{Solution}{8}
B
\end{Solution}
\begin{Solution}{9}
C
\end{Solution}
\begin{Solution}{10}
B
\end{Solution}
\begin{Solution}{11}
A
\end{Solution}
\begin{Solution}{12}
B
\end{Solution}
\begin{Solution}{13}
B
\end{Solution}
\begin{Solution}{14}
B
\end{Solution}
\begin{Solution}{15}
A
\end{Solution}
\begin{Solution}{16}
B
\end{Solution}
\begin{Solution}{17}
A
\end{Solution}
\begin{Solution}{18}
D
\end{Solution}
\begin{Solution}{19}
C
\end{Solution}
\begin{Solution}{20}
C
\end{Solution}
\begin{Solution}{21}
A
\end{Solution}
\begin{Solution}{22}
C
\end{Solution}
\begin{Solution}{23}
C
\end{Solution}
\begin{Solution}{24}
A
\end{Solution}
\begin{Solution}{25}
B
\end{Solution}
\begin{Solution}{26}
B
\end{Solution}
\begin{Solution}{27}
A
\end{Solution}
\begin{Solution}{28}
A
\end{Solution}
\begin{Solution}{29}
C
\end{Solution}
\begin{Solution}{30}
B
\end{Solution}
\begin{Solution}{31}
A
\end{Solution}
\begin{Solution}{32}
C
\end{Solution}
\begin{Solution}{33}
B
\end{Solution}
\begin{Solution}{34}
A
\end{Solution}
\begin{Solution}{35}
B
\end{Solution}
\begin{Solution}{36}
B
\end{Solution}
\begin{Solution}{37}
B
\end{Solution}
\begin{Solution}{38}
D
\end{Solution}
\begin{Solution}{39}
B
\end{Solution}
\begin{Solution}{40}
A
\end{Solution}
\begin{Solution}{41}
D
\end{Solution}
\begin{Solution}{42}
D
\end{Solution}
\begin{Solution}{43}
A
\end{Solution}
\begin{Solution}{44}
D
\end{Solution}
\begin{Solution}{45}
C
\end{Solution}
\begin{Solution}{46}
B
\end{Solution}
\begin{Solution}{47}
A
\end{Solution}
\begin{Solution}{48}
D
\end{Solution}
\begin{Solution}{49}
B
\end{Solution}
\begin{Solution}{50}
B
\end{Solution}
\begin{Solution}{51}
D
\end{Solution}
\begin{Solution}{52}
C
\end{Solution}
\begin{Solution}{53}
C
\end{Solution}
\begin{Solution}{54}
B
\end{Solution}
\begin{Solution}{55}
D
\end{Solution}
\begin{Solution}{56}
B
\end{Solution}
\begin{Solution}{57}
C
\end{Solution}
\begin{Solution}{58}
A
\end{Solution}
\begin{Solution}{59}
A
\end{Solution}
\begin{Solution}{60}
B
\end{Solution}
\begin{Solution}{61}
D
\end{Solution}
\begin{Solution}{62}
D
\end{Solution}
\begin{Solution}{63}
B
\end{Solution}
\begin{Solution}{64}
A
\end{Solution}
\begin{Solution}{65}
D
\end{Solution}
\begin{Solution}{66}
C
\end{Solution}
\begin{Solution}{67}
A
\end{Solution}
\begin{Solution}{68}
A
\end{Solution}
\begin{Solution}{69}
D
\end{Solution}
\begin{Solution}{70}
C
\end{Solution}
\begin{Solution}{71}
B
\end{Solution}
\begin{Solution}{72}
A
\end{Solution}
\begin{Solution}{73}
C
\end{Solution}
\begin{Solution}{74}
C
\end{Solution}
\begin{Solution}{75}
C
\end{Solution}
\begin{Solution}{76}
A
\end{Solution}
\begin{Solution}{77}
C
\end{Solution}
\begin{Solution}{78}
B
\end{Solution}
\begin{Solution}{79}
D
\end{Solution}
\begin{Solution}{80}
B
\end{Solution}

\section{Mức 9,10 điểm}
\setcounter{ex}{0}
\setcounter{dang}{0}
\Opensolutionfile{ans}[ans/CD1/Muc_9_10]
\begin{dang}{Tìm m để hàm số đơn điệu trên các khoảng xác định của nó}
	Đang thiếu bài thầy Jf Câu 1 đến 26 
\end{dang}
\begin{dang}
	{Tìm khoảng đơn điệu của hàm số $g(x) = f\left[ u(x)\right] +v(x)$ khi biết đồ thị hoặc bảng biến thiên của hàm số $y = f'(x)$}
\end{dang}
\begin{ex}[Đề tham khảo 2019]%[2D1K1-2]
	Cho hàm số $f(x)$ có bảng xét dấu của đạo hàm như sau
	\begin{center}
		\begin{tikzpicture}
			\tkzTabInit[nocadre,lgt=1.2,espcl=2,deltacl=0.6]
			{$x$ /0.6,$f'(x)$ /0.6}
			{$-\infty$,$1$,$2$,$3$,$4$,$+\infty$}
			\tkzTabLine{,-,$0$,+,$0$,+,$0$,-,$0$,+,}
		\end{tikzpicture}
	\end{center}
	Hàm số $y=3 f(x+2)-x^3+3 x$ đồng biến trên khoảng nào dưới đây?
	\choice
	{$(-\infty ;-1)$}
	{\True $(-1 ; 0)$}
	{$(0 ; 2)$}
	{$(1 ;+\infty)$}
	\loigiai{
		Ta có $y'=3\left[f'(x+2)-\left(x^2-3\right)\right]$.\\
		Với $x \in(-1 ; 0) \Rightarrow x+2 \in(1 ; 2) \Rightarrow f'(x+2)>0$, lại có $x^2-3<0 \Rightarrow y'>0 ;~ \forall x \in(-1 ; 0)$.\\
		Vậy hàm số $y=3 f(x+2)-x^3+3 x$ đồng biến trên khoảng $(-1 ; 0)$.\\
		Chú ý:\\
		+) Ta xét $x \in(1 ; 2) \subset(1 ;+\infty)
		\Rightarrow x+2 \in(3 ; 4)\\
		\Rightarrow f'(x+2)<0 ;~ x^2-3>0$\\
		Suy ra hàm số nghịch biến trên khoảng $(1 ; 2)$ nên loại hai phương án$(0 ; 2)$ và $(1 ;+\infty)$.\\
		+) Tương tự ta xét
		$x \in(-\infty ;-2) \Rightarrow x+2 \in(-\infty ; 0)\\
		\Rightarrow f'(x+2)<0 ; x^2-3>0 \Rightarrow y'<0 ; ~ \forall x \in(-\infty ;-2)$.\\
		Suy ra hàm số nghịch biến trên khoảng $(-\infty ;-2)$ nên loại$(-\infty ;-1)$.\\
		Vậy hàm số đã cho đồng biến trên khoảng $(-1 ; 0)$.
	}
\end{ex}
\begin{ex}[Đề Tham Khảo 2020 - Lần 1]%[2D1G1-2]
	\immini{
		Cho hàm số $f(x)$. Hàm số $y=f'(x)$ có đồ thị như hình bên. Hàm số $g(x)=f(1-2 x)+x^2-x$ nghịch biến trên khoảng nào dưới đây?
		\choice
		{\True $\left(1 ; \dfrac{3}{2}\right)$}
		{$\left(0 ; \dfrac{1}{2}\right)$}
		{$(-2 ;-1)$}
		{$(2 ; 3)$}
	}
	{
		\begin{tikzpicture}[scale=0.7,>=stealth, font=\footnotesize, line join=round, line cap=round]
			%\def\a{1} \def\b{-6} \def\c{9} \def\d{1} % Hệ số
			\def\xmin{-4} \def\xmax{6}
			\def\ymin{-3} \def\ymax{2} 
			%\draw[color=gray!50,dashed] (\xmin,\ymin) grid (\xmax,\ymax); 
			\draw[->] (\xmin,0)--(\xmax,0) node [below]{$x$};
			\draw[->] (0,\ymin)--(0,\ymax) node [left]{$y$};
			\node at (0,0) [below left]{$O$};
			%\node at (1,3) [below left]{$f'(x)$};
			%\node at (-1.3,4) {$f'(x)$};
			\draw[dashed] (-2,0) node[below]{$-2$}--(-2,1)--(0,1) node[below left]{$1$};
			\draw[dashed] (4,0) node[below left]{$4$}--(4,-2)--(0,-2) node[below left]{$-2$};
			%\draw[dashed] (1,0) node[below]{$1$}--(1,1);
			%\draw[dashed] (-0.5,0) node[below left]{$-0{,}5$}--(-0.5,2.125);
			\clip (\xmin+0.1,\ymin+0.1) rectangle (\xmax-0.5,\ymax-0.1);
			\draw[smooth,samples=300][domain=-4:5.5] plot(\x,{0.071*(\x)^3-0.142*(\x)^2-1.07*(\x)});
		\end{tikzpicture}
	}
	
	\loigiai{
		Ta có : $g(x)=f(1-2 x)+x^2-x \Rightarrow g'(x)=-2 f'(1-2 x)+2 x-1$.\\
		\immini{
			Đặt $t=1-2 x \Rightarrow g'(x)=-2 f'(t)-t$.\\
			$g'(x)=0 \Rightarrow f'(t)=-\dfrac{t}{2}$.\\
			Vẽ đường thẳng $y=-\dfrac{x}{2}$ và đồ thị hàm số $f'(x)$ trên cùng một hệ trục
		}	
		{
			\begin{tikzpicture}[scale=0.7,>=stealth, font=\footnotesize, line join=round, line cap=round]
				%\def\a{1} \def\b{-6} \def\c{9} \def\d{1} % Hệ số
				\def\xmin{-4} \def\xmax{6}
				\def\ymin{-3} \def\ymax{2} 
				%	\draw[color=gray!50,dashed] (\xmin,\ymin) grid (\xmax,\ymax); 
				\draw[->] (\xmin,0)--(\xmax,0) node [below]{$x$};
				\draw[->] (0,\ymin)--(0,\ymax) node [left]{$y$};
				\node at (0,0) [below left]{$O$};
				%\node at (1,3) [below left]{$f'(x)$};
				%\node at (-1.3,4) {$f'(x)$};
				\draw[dashed] (-2,0) node[below]{$-2$}--(-2,1)--(0,1) node[below left]{$1$};
				\draw[dashed] (4,0) node[below]{$4$}--(4,-2)--(0,-2) node[below left]{$-2$};
				%\draw[dashed] (1,0) node[below]{$1$}--(1,1);
				%\draw[dashed] (-0.5,0) node[below left]{$-0{,}5$}--(-0.5,2.125);
				\clip (\xmin+0.1,\ymin+0.1) rectangle (\xmax-0.5,\ymax-0.1);
				\draw[smooth,samples=300][domain=-4:5.5] plot(\x,{0.071*(\x)^3-0.142*(\x)^2-1.07*(\x)});
				\draw[smooth,samples=300][domain=-4:5.5] plot(\x,{(-0.5*(\x)});
			\end{tikzpicture}
		}	Hàm số $g(x)$ nghịch biến $\Rightarrow g'(x) \leq 0 \Rightarrow f'(t) \geq-\dfrac{t}{2}\Rightarrow\hoac{&-2 \leq t \leq 0 \\&t \geq 4.}$\\
		Như vậy $f'(1-2 x) \geq \dfrac{1-2 x}{-2}\Rightarrow\hoac{&-2 \leq 1-2 x \leq 0 \\ &4 \leq 1-2 x}\Rightarrow\hoac{&\dfrac{1}{2}\leq x \leq \dfrac{3}{2}\\ &x \leq-\dfrac{3}{2}.}$\\
		Vậy hàm số $g(x)=f(1-2 x)+x^2-x$ nghịch biến trên các khoảng $\left(\dfrac{1}{2}; \dfrac{3}{2}\right)$ và $\left(-\infty ;-\dfrac{3}{2}\right)$.\\
		Mà $\left(1 ; \dfrac{3}{2}\right) \subset \left(\dfrac{1}{2}; \dfrac{3}{2}\right)$ nên hàm số $g(x)=f(1-2 x)+x^2-x$ nghịch biến trên khoảng $\left(1 ; \dfrac{3}{2}\right)$.
	}
\end{ex}
\begin{ex}[Chuyên Lê Quý Đôn Điện Biên 2019]%[2D1G1-2]
	Cho hàm số $f(x)$ có bảng xét dấu của đạo hàm như sau
	\begin{center}
		\begin{tikzpicture}
			\tkzTabInit[nocadre,lgt=1.2,espcl=2,deltacl=0.6]
			{$x$ /0.6,$f'(x)$ /0.6}
			{$-\infty$,$0$,$1$,$2$,$3$,$+\infty$}
			\tkzTabLine{,+,$0$,-,$0$,-,$0$,+,$0$,-,}
		\end{tikzpicture}
	\end{center}
	Hàm số $y=f(x-1)+x^3-12 x+2019$ nghịch biến trên khoảng nào dưới đây?
	\choice
	{$(1 ;+\infty)$}
	{\True $(1 ; 2)$}
	{$(-\infty ; 1)$}
	{$(3 ; 4)$}
	\loigiai{
		$y'=f'(x-1)+3 x^2-12=f'(t)+3 t^2+6 t-9=f'(t)-\left(-3 t^2-6 t+9\right)$, với $t=x-1$.\\
		\immini{
			Nghiệm của phương trình $y'=0$ là hoành độ giao điểm của các đồ thị hàm số $y=f'(t)$ và $y=-3 t^2-6 t+9$.\\
			Vẽ đồ thị hàm số $y=f'(t)$ và $y=-3 t^2-6 t+9$ trên cùng một hệ trục tọa độ như hình vẽ bên.
		}	
		{		\begin{tikzpicture}[scale=0.5,>=stealth, font=\footnotesize, line join=round, line cap=round]
				\def\a{-3} \def\b{-6} \def\c{9} % Hệ số
				\def\xmin{-9} \def\xmax{7}
				\def\ymin{-3} \def\ymax{13}
				
				%\draw[color=gray!50,dashed] (\xmin,\ymin) grid (\xmax,\ymax);
				
				\draw[->] (\xmin,0)--(\xmax,0) node [below]{$x$};
				\draw[->] (0,\ymin)--(0,\ymax) node [left]{$y$};
				\node at (0,0) [below left]{$O$};
				\clip (\xmin+0.1,\ymin+0.1) rectangle (\xmax-0.5,\ymax-0.1);
				\draw[smooth,samples=300] plot(\x,{\a*(\x)^2+\b*(\x)+\c});
				\node at (1,0) [above right]{$1$};
				\node at (2,0) [below right]{$2$};
				\node at (3,0) [below right]{$3$};
				\node at (-3,-2) [left]{$y=-3t^2-6t+9$};
				\node at (4,0) [below right]{$f'(x)$};
				\draw (-2.2,10).. controls (-1,1.9) and (-0.5,0.8) .. (0,0);
				%\draw (-2,0).. controls (-1.5,-2) and (-0.5,-0) .. (0,0);
				\draw (0,0).. controls (0.4,-0.6) and (0.6,-0.6) .. (0.8,-0.2);
				\draw (0.8,-0.2).. controls (1,0.25) and (1.1,-0.1) .. (1.4,-0.8);
				\draw (1.4,-0.8).. controls (1.6,-1.1) and (1.7,-0.9) .. (2,0);
				\draw (2,0).. controls (2.4,1.1) and (2.6,1.1) .. (3.5,-1);
			\end{tikzpicture}
		}
		Dựa vào đồ thị trên, ta có bảng xét dấu của hàm số $y'=f'(t)-\left(-3 t^2-6 t+9\right)$ như sau $
		\left(t_0<-1\right)$
		\begin{center}
			\begin{tikzpicture}
				\tkzTabInit[nocadre,lgt=2,espcl=2,deltacl=0.6]
				{$x$ /0.6,$y'$ /0.6}
				{$-\infty$,$t_0$,$1$,$+\infty$}
				\tkzTabLine{,+,$0$,-,$0$,+,}
			\end{tikzpicture}
		\end{center}
		Hàm số nghịch biến trên khoảng $t \in\left(t_0 ; 1\right)$.\\
		Do đó hàm số nghịch biến trên khoảng $x \in(1 ; 2) \subset \left(t_0+1 ; 1\right)$.
	}
\end{ex}


\begin{ex}[Chuyên Phan Bội Châu Nghệ An 2019]%[2D1G1-2]
	Cho hàm số $f(x)$ có bảng xét dấu đạo hàm như sau:
	\begin{center}
		\begin{tikzpicture}
			\tkzTabInit[nocadre,lgt=2,espcl=2,deltacl=0.6]
			{$x$ /0.6,$f'(x)$ /0.6}
			{$-\infty$,$1$,$2$,$3$,$4$,$+\infty$}
			\tkzTabLine{,-,$0$,+,$0$,+,$0$,-,$0$,+,}
		\end{tikzpicture}
	\end{center}
	Hàm số $y=2 f(1-x)+\sqrt{x^2+1}-x$ nghịch biến trên những khoảng nào dưới đây
	\choice
	{$(-\infty ;-2)$}
	{$(-\infty ; 1)$}
	{\True $(-2 ; 0)$}
	{$(-3 ;-2)$}
	\loigiai{
		$y'=-2 f'(1-x)+\dfrac{x}{\sqrt{x^2+1}}-1$. \\
		Có $\dfrac{x}{\sqrt{x^2+1}}-1<0,~ \forall x \in(-2 ; 0)$.\\
		Bảng xét dấu:
		\begin{center}
			\begin{tikzpicture}
				\tkzTabInit[nocadre,lgt=2,espcl=2,deltacl=0.6]
				{$x$ /0.7,$f'(1-x)$ /0.7}
				{$-\infty$,$-3$,$-2$,$-1$,$0$,$+\infty$}
				\tkzTabLine{,+,$0$,-,$0$,+,$0$,+,$0$,-,}
			\end{tikzpicture}
		\end{center}
		$\Rightarrow-2 f'(1-x)<0, ~ \forall x \in(-2 ; 0) \\
		\Rightarrow-2 f'(1-x)+\dfrac{x}{\sqrt{x^2+1}}-1<0, ~\forall x \in(-2 ; 0)$.
	}
\end{ex}
\begin{ex}[Sở Vĩnh Phúc 2019]%[2D1G1-2]
	\immini{
		Cho hàm số bậc bốn $y=f(x)$ có đồ thị của hàm số $y=f'(x)$ như hình vẽ bên.\\
		Hàm số $y=3 f(x)+x^3-6 x^2+9 x$ đồng biến trên khoảng nào trong các khoảng sau đây?
		\choice
		{$(0 ; 2)$}
		{$(-1 ; 1)$}
		{$(1 ;+\infty)$}
		{\True $(-2 ; 0)$}
	}
	{
		\begin{tikzpicture}[scale=0.7,>=stealth, font=\footnotesize, line join=round, line cap=round]
			\def\a{0.21} \def\b{0.88} \def\c{-0.58} \def\d{-3} % Hệ số
			\def\xmin{-5} \def\xmax{5}
			\def\ymin{-4} \def\ymax{3} 
			%\draw[color=gray!50,dashed] (\xmin,\ymin) grid (\xmax,\ymax); 
			\draw[->] (\xmin,0)--(\xmax,0) node [below]{$x$};
			\draw[->] (0,\ymin)--(0,\ymax) node [left]{$y$};
			\node at (0,0) [above left]{$O$};
			\node at (-4,0) [below left]{$-4$};
			\node at (-2,0) [below left]{$-2$};
			\node at (0,-3) [below right]{$-3$};
			\draw[dashed] (2,0) node[above right]{$2$}--(2,1) --(0,1) node[above right]{$1$};
			\clip (\xmin+0.1,\ymin+0.1) rectangle (\xmax-0.5,\ymax-0.1);
			\draw[smooth,samples=300] plot(\x,{\a*(\x)^3+\b*(\x)^2+\c*(\x)+\d});
		\end{tikzpicture}
	}
	
	\loigiai{
		Hàm số $f(x)=a x^4+b x^3+c x^2+d x+e,(a \neq 0)$.
		Có $f'(x)=4 a x^3+3 b x^2+2 c x+d$.\\
		Đồ thị hàm số $y=f'(x)$ đi qua các điểm $(-4 ; 0),(-2 ; 0),(0 ;-3),(2 ; 1)$ nên ta có
		$$\heva{&- 2 5 6 a + 4 8 b - 8 c + d = 0\\
			&- 3 2 a + 1 2 b - 4 c + d = 0\\
			&d = - 3\\
			&3 2 a + 1 2 b + 4 c + d = 1}\Leftrightarrow \heva{&
			a=\dfrac{5}{96}\\
			&b=\dfrac{7}{24}\\
			&c=-\dfrac{7}{24}\\
			&d=-3.}
		$$
		Xét hàm số
		$
		y=3 f(x)+x^3-6 x^2+9 x$\\
		Ta có $ y'=3\left(f'(x)+x^2-4 x+3\right)=3\left(\frac{5}{24}x^3+\frac{15}{8}x^2-\frac{55}{12}x\right)
		$\\
		Ta có $y'=0 \Leftrightarrow\hoac{&x=-11 \\&x=0 \\&x=2.}$ \\
		Xét dấu $y'$, ta được hàm số đã cho đồng biến trên các khoảng $(-11 ; 0)$ và $(2 ;+\infty)$.
	}
\end{ex}
\begin{ex}[Học Mãi 2019]%[2D1K1-2]
	\immini
	{Cho hàm số $y=f(x)$ có đạo hàm trên $\mathbb{R}$. Đồ thị hàm số $y=f'(x)$ như hình bên. Hỏi đồ thị hàm số $y=f(x)-2 x$ có bao nhiêu điểm cực trị?
		\choice
		{$4$}
		{\True $3$}
		{$2$}
		{$1$}
	}
	{
		\begin{tikzpicture}[font=\footnotesize,line join=round, line cap=round,>=stealth,scale=0.8]
			\draw[->] (-3.5,0)--(4,0) node[above] {$x$};
			\draw[->] (0,-3)--(0,4) node[left] {$y$};
			%\fill[black] (-2,0)node[below left]{$-2$} circle (1.2pt) (0,0)node[above right]{$O$} circle (1.2pt) (3,0)node[above]{$3$} circle (1.2pt);
			\draw[dashed] (-2,-2)-- (0,-2) node[right]{$-2$};
			\draw[dashed] (2,0) node[below]{$2$}-- (2,2)--(0,2) node[below left]{$2$};
			\node at (0,0) [below left]{$O$};
			\node at (3,0) [below right]{$3$};
			\draw (-3,2.5).. controls (-2.2,-3) and (-1.8,-3) .. (-1.1,0);
			\draw (-1.1,0).. controls (-0.6,2.5) and (-0.4,2.5) .. (0,2);
			\draw (0,2).. controls (0.7,0.5) and (1.1,0.5) .. (1.5,1.5);
			\draw (1.5,1.5).. controls (2,2.5) and (2.8,2.5) .. (3.5,-2.5);
			%\draw (3,0).. controls (3.3,-0.1) and (3.5,-0.5) .. (3.5,-2);
		\end{tikzpicture}
	}
	\loigiai{
		\immini{
			Đặt $g(x)=f(x)-2 x$.\\
			$\Rightarrow g'(x)=f'(x)-2 .
			$\\
			Vẽ đường thẳng $y=2$.\\
			$\Rightarrow$ phương trình $g'(x)=0$ có $3$ nghiệm bội lẻ.\\
			$\Rightarrow$ đồ thị hàm số $y=f(x)-2 x$ có $3$ điểm cực trị.
		}
		{
			\begin{tikzpicture}[font=\footnotesize,line join=round, line cap=round,>=stealth,scale=0.8]
				\draw[->] (-3.5,0)--(4,0) node[above] {$x$};
				\draw[->] (0,-3)--(0,4) node[left] {$y$};
				%\fill[black] (-2,0)node[below left]{$-2$} circle (1.2pt) (0,0)node[above right]{$O$} circle (1.2pt) (3,0)node[above]{$3$} circle (1.2pt);
				\draw[dashed] (-2,-2)-- (0,-2) node[right]{$-2$};
				\draw[dashed] (2,0) node[below]{$2$}-- (2,2)--(0,2) node[below left]{$2$};
				\node at (3,0) [below left]{$3$};
				\draw (-3,2.5).. controls (-2.2,-3) and (-1.8,-3) .. (-1.1,0);
				\draw (-1.1,0).. controls (-0.6,2.5) and (-0.4,2.5) .. (0,2);
				\draw (0,2).. controls (0.7,0.5) and (1.1,0.5) .. (1.5,1.5);
				\draw (1.5,1.5).. controls (2,2.5) and (2.8,2.5) .. (3.5,-2.5);
				\draw (-3.5,2)--(4,2) node[above]{$y=2$};
			\end{tikzpicture}
		}
	}
\end{ex}
\begin{ex}[THPT Hoàng Hoa Thám Hưng Yên 2019]%[2D1G1-2]
	\immini{
		Cho hàm số $y=f(x)$ liên tục trên $\mathbb{R}$. Hàm số $y=f'(x)$ có đồ thị như hình vẽ. 
		Hàm số $g(x)=f(x-1)+\dfrac{2019-2018 x}{2018}$ đồng biến trên khoảng nào dưới đây?
		\choice
		{$(2 ; 3)$}
		{$(0 ; 1)$}
		{\True $(-1 ; 0)$}
		{$(1 ; 2)$}
	}
	{
		\begin{tikzpicture}[scale=1, font=\footnotesize, line join=round, line cap=round, >=stealth]
			\tikzset{label style/.style={font=\footnotesize}}
			\draw[->] (-2,0)--(3,0) node[below left] {$x$};
			\draw[->] (0,-2)--(0,3) node[below left] {$y$};
			\draw[fill=black] (0,0) node [above left] {$O$} circle(1pt);
			\fill (1,1) circle(1pt) (-1,1) circle(1pt) (2,1) circle(1pt);
			\foreach \x in {1,2}
			\draw[thin] (\x,1pt)--(\x,-1pt) node [below] {\footnotesize$\x$};
			\foreach \x in {-1}
			\draw[thin] (\x,1pt)--(\x,-1pt) node [below left] {\footnotesize$\x$};
			\foreach \y in {-1}
			\draw[thin] (1pt,\y)--(-1pt,\y) node [right] {\footnotesize$\y$};
			\foreach \y in {1}
			\draw[thin] (1pt,\y)--(-1pt,\y) node [above left] {\footnotesize$\y$};
			\draw[dashed](-1,0)--(-1,1)--(2,1) (1,1)--(1,0) (2,1)--(2,0);
			\begin{scope}
				\clip (-3,-3) rectangle (3,3);
				\draw[name path=(C)] plot[smooth,tension=0.7] coordinates{(-1.15,3)(-0.5,-1.6)(.8,.88)(1.9,0.8)(2.3,3)};
			\end{scope}
		\end{tikzpicture}
	}	\loigiai{
		Ta có $g'(x)=f'(x-1)-1$.\\
		$
		g'(x) \geq 0 \Leftrightarrow f'(x-1)-1 \geq 0 \Leftrightarrow f'(x-1) \geq 1 \Leftrightarrow \hoac{&x - 1 \leq - 1\\
			&x - 1 \geq 2}\Leftrightarrow \hoac{&
			x \leq 0 \\
			&x \geq 3.}
		$\\
		Từ đó suy ra hàm số $g(x)=f(x-1)+\dfrac{2019-2018 x}{2018}$ đồng biến trên khoảng $(-1 ; 0)$.
	}
\end{ex}

\begin{ex}[(Sở Ninh Bình 2019]%[2D1K1-2]
	Cho hàm số $y=f(x)$ có bảng xét dấu của đạo hàm như sau
	\begin{center}
		\begin{tikzpicture}
			\tkzTabInit[nocadre,lgt=1,espcl=2,deltacl=0.6]
			{$x$ /0.7,$f'(x)$ /0.7}
			{$-\infty$,$-2$,$-1$,$2$,$4$,$+\infty$}
			\tkzTabLine{,+,$0$,-,$0$,+,$0$,-,$0$,+,}
		\end{tikzpicture}
	\end{center}
	Hàm số $y=-2 f(x)+2019$ nghịch biến trên khoảng nào trong các khoảng dưới đây?
	\choice
	{$(-4 ; 2)$}
	{\True $(-1 ; 2)$}
	{$(-2 ;-1)$}
	{$(2 ; 4)$}
	\loigiai{
		Xét $y=g(x)=-2 f(x)+2019$.\\
		Ta có $g'(x)=(-2 f(x)+2019)'=-2 f'(x), g'(x)=0 \Leftrightarrow\hoac{&x=-2 \\&x=-1 \\&x=2 \\&x=4.}$.\\
		Ta có bảng xét dấu của $g'(x)$
		\begin{center}
			\begin{tikzpicture}
				\tkzTabInit[nocadre,lgt=1,espcl=2,deltacl=0.6]
				{$x$ /0.6,$f'(x)$ /0.6}
				{$-\infty$,$-2$,$-1$,$2$,$4$,$+\infty$}
				\tkzTabLine{,-,$0$,+,$0$,-,$0$,+,$0$,+,}
			\end{tikzpicture}
		\end{center}
		Dựa vào bảng xét dấu, ta thấy hàm số $y=g(x)$ nghịch biến trên khoảng $(-1 ; 2)$.
	}
\end{ex}
\begin{ex}[THPT Lương Thế Vinh Hà Nội 2019]%[2D1G1-2]
	\immini{
		Cho hàm số $y=f(x)$. Biết đồ thị hàm số $y=f'(x)$ có đồ thị như hình vẽ bên. 
		Hàm số $y=f \left(3-x^2\right)+2018$ đồng biến trên khoảng nào dưới đây?
		\choice
		{\True $(-1 ; 0)$}
		{$(2 ; 3)$}
		{$(-2 ;-1)$}
		{$(0 ; 1)$}
	}
	{
		\begin{tikzpicture}[scale=0.6,>=stealth, font=\footnotesize, line join=round, line cap=round]
			\def\a{0.065} \def\b{0.32} \def\c{-0.53} \def\d{-0.82} % Hệ số
			\def\xmin{-8} \def\xmax{4}
			\def\ymin{-3} \def\ymax{3} 
			%\draw[color=gray!50,dashed] (\xmin,\ymin) grid (\xmax,\ymax); 
			\draw[->] (\xmin,0)--(\xmax,0) node [below]{$x$};
			\draw[->] (0,\ymin)--(0,\ymax) node [left]{$y$};
			\node at (0,0) [below left]{$O$};
			\node at (-6,0) [below left]{$-6$};
			\node at (-1,0) [below left]{$-1$};
			\node at (2,0) [below right]{$2$};
			\clip (\xmin+0.1,\ymin+0.1) rectangle (\xmax-0.5,\ymax-0.1);
			\draw[smooth,samples=300][domain=-6.5:3.5] plot(\x,{\a*(\x)^3+\b*(\x)^2+\c*(\x)+\d});
		\end{tikzpicture}
	}
	
	\loigiai{
		Ta có $\left[f\left( 3-x^2\right)+2018 \right]'=-2 x \cdot f'\left(3-x^2\right) $.\\
		$
		-2 x \cdot f'\left(3-x^2\right)=0 \Leftrightarrow\hoac{&
			x = 0\\
			&3 - x ^{2}= - 6\\
			&3 - x ^{2}= - 1\\
			&3 - x ^{2}= 2}
		\Leftrightarrow \hoac{
			&x=0 \\
			&x=\pm 3 \\
			&x=\pm 2 \\
			&	x=\pm 1.}
		$\\
		Bảng xét dấu của đạo hàm hàm số đã cho
		\begin{center}
			\begin{center}
				\begin{tikzpicture}
					\tkzTabInit[nocadre,lgt=2.9,espcl=1.5,deltacl=0.6]
					{$x$ /0.7,$f'\left( 3-x^2\right) $/0.7,$-2xf'\left( 3-x^2\right)$/0.8}
					{$-\infty$,$-3$,$-2$,$-1$,$0$,$1$,$2$,$3$,$+\infty$}
					\tkzTabLine{,-,$0$,+,$0$,-,$0$,+,$0$,+,$0$,-,$0$,+,$0$,-}
					\tkzTabLine{,-,$0$,+,$0$,-,$0$,+,$0$,-,$0$,+,$0$,-,$0$,+}
				\end{tikzpicture}
			\end{center}
		\end{center}
		Từ bảng xét dấu suy ra hàm số đồng biến trên $(-1 ; 0)$.
	}
\end{ex}
\begin{ex}[Chuyên Biên Hòa - Hà Nam - 2020]%[2D1G1-2]
	\immini{
		Cho hàm số đa thức $f(x)$ có đạo hàm trên $\mathbb{R}$. Biết $f(0)=0$ và đồ thị hàm số $y=f'(x)$ như hình sau.
		Hàm số $g(x)=\left|4 f(x)+x^2\right|$ đồng biến trên khoảng nào dưới đây?
		\choice
		{$(4 ;+\infty)$}
		{\True $(0 ; 4)$}
		{$(-\infty ;-2)$}
		{$(-2 ; 0)$}
	}	
	{
		\begin{tikzpicture}[scale=0.7,>=stealth, font=\footnotesize, line join=round, line cap=round]
			%\def\a{1} \def\b{-6} \def\c{9} \def\d{1} % Hệ số
			\def\xmin{-4} \def\xmax{6}
			\def\ymin{-3} \def\ymax{2} 
			%\draw[color=gray!50,dashed] (\xmin,\ymin) grid (\xmax,\ymax); 
			\draw[->] (\xmin,0)--(\xmax,0) node [below]{$x$};
			\draw[->] (0,\ymin)--(0,\ymax) node [left]{$y$};
			\node at (0,0) [below left]{$O$};
			%\node at (1,3) [below left]{$f'(x)$};
			%\node at (-1.3,4) {$f'(x)$};
			\draw[dashed] (-2,0) node[below]{$-2$}--(-2,1)--(0,1) node[below left]{$1$};
			\draw[dashed] (4,0) node[below]{$4$}--(4,-2)--(0,-2) node[below left]{$-2$};
			%\draw[dashed] (1,0) node[below]{$1$}--(1,1);
			%\draw[dashed] (-0.5,0) node[below left]{$-0{,}5$}--(-0.5,2.125);
			\clip (\xmin+0.1,\ymin+0.1) rectangle (\xmax-0.5,\ymax-0.1);
			\draw[smooth,samples=300][domain=-4:5.5] plot(\x,{0.071*(\x)^3-0.142*(\x)^2-1.07*(\x)});
		\end{tikzpicture}
	}
	\loigiai{
		\immini{
			Xét hàm số $h(x)=4 f(x)+x^2$ trên $\mathbb{R}$.\\
			Vì $f(x)$ là hàm số đa thức nên $h(x)$ cũng là hàm số đa thức và $h(0)=4 f(0)=0$.\\
			Ta có $h'(x)=4 f'(x)+2 x$. Do đó $h'(x)=0 \Leftrightarrow f'(x)=-\dfrac{1}{2}x$.\\
		}
		{
			\begin{tikzpicture}[scale=0.7,>=stealth, font=\footnotesize, line join=round, line cap=round]
				%\def\a{1} \def\b{-6} \def\c{9} \def\d{1} % Hệ số
				\def\xmin{-4} \def\xmax{6}
				\def\ymin{-3} \def\ymax{2} 
				%\draw[color=gray!50,dashed] (\xmin,\ymin) grid (\xmax,\ymax); 
				\draw[->] (\xmin,0)--(\xmax,0) node [below]{$x$};
				\draw[->] (0,\ymin)--(0,\ymax) node [left]{$y$};
				\node at (0,0) [below left]{$O$};
				%\node at (1,3) [below left]{$f'(x)$};
				%\node at (-1.3,4) {$f'(x)$};
				\draw[dashed] (-2,0) node[below]{$-2$}--(-2,1)--(0,1) node[below left]{$1$};
				\draw[dashed] (4,0) node[below]{$4$}--(4,-2)--(0,-2) node[below left]{$-2$};
				%\draw[dashed] (1,0) node[below]{$1$}--(1,1);
				%\draw[dashed] (-0.5,0) node[below left]{$-0{,}5$}--(-0.5,2.125);
				\clip (\xmin+0.1,\ymin+0.1) rectangle (\xmax-0.5,\ymax-0.1);
				\draw[smooth,samples=300][domain=-4:5.5] plot(\x,{0.071*(\x)^3-0.142*(\x)^2-1.07*(\x)});
				\draw[smooth,samples=300][domain=-4:5.5] plot(\x,{-0.5*(\x)});
			\end{tikzpicture}
		}
		Dựa vào sự tương giao của đồ thị hàm số $y=f'(x)$ và đường thẳng $y=-\dfrac{1}{2}x$, ta có
		$
		h'(x)=0 \Leftrightarrow x \in\{-2 ; 0 ; 4\}.\\
		$
		Bảng biến thiên của hàm số $h(x)$ như sau:
		\begin{center}
			\begin{tikzpicture}
				\tkzTabInit[nocadre,lgt=1.2,espcl=2.5,deltacl=0.6]
				{$x$ /0.6,$y'$ /0.6,$y$ /2}
				{$-\infty$,$-2$,$0$,$4$,$+\infty$}
				\tkzTabLine{,-,$0$,+,$0$,-,$0$,+,}
				\tkzTabVar{+/$+\infty$, -/$y_1$,+/$0$,-/$y_3$,+/$+\infty$}
			\end{tikzpicture}
		\end{center}
		Từ đó suy ra bảng biến thiên của hàm số $g(x)=|h(x)|$.\\
		Dựa vào bảng biến thiên trên, ta thấy hàm số $g(x)$ đồng biến trên khoảng $(0 ; 4)$.
	}
\end{ex}
\begin{ex}[Chuyên Thái Bình - 2020]%[2D1G1-2]
	\immini{
		Cho hàm số $f(x)$ liên tục trên $\mathbb{R}$ có đồ thị hàm số $y=f'(x)$ cho như hình vẽ bên.\\
		Hàm số $g(x)=2 f(|x-1|)-x^2+2 x+2020$ đồng biến trên khoảng nào?
		\choice
		{\True $(0 ; 1)$}
		{$(-3 ; 1)$}
		{$(1 ; 3)$}
		{$(-2 ; 0)$}
	}
	{
		\begin{tikzpicture}[scale=0.7,>=stealth, font=\footnotesize, line join=round, line cap=round]
			%\def\a{1} \def\b{-6} \def\c{9} \def\d{1} % Hệ số
			\def\xmin{-4} \def\xmax{5}
			\def\ymin{-3} \def\ymax{5} 
			%\draw[color=gray!50,dashed] (\xmin,\ymin) grid (\xmax,\ymax); 
			\draw[->] (\xmin,0)--(\xmax,0) node [below]{$x$};
			\draw[->] (0,\ymin)--(0,\ymax) node [left]{$y$};
			\node at (0,0) [below left]{$O$};
			%\node at (1,3) [below left]{$f'(x)$};
			\node at (-1.3,4) {$f'(x)$};
			\draw[dashed] (-1,0) node[above]{$-1$}--(-1,-1)--(0,-1) node[below left]{$-1$};
			\draw[dashed] (1,0) node[below]{$1$}--(1,1)--(0,1) node[below left]{$1$};
			\draw[dashed] (3,0) node[below]{$3$}--(3,3)--(0,3) node[below left]{$3$};
			%\draw[dashed] (1,0) node[below]{$1$}--(1,1);
			%\draw[dashed] (-0.5,0) node[below left]{$-0{,}5$}--(-0.5,2.125);
			\clip (\xmin+0.1,\ymin+0.1) rectangle (\xmax-0.5,\ymax-0.1);
			\draw[smooth,samples=300][domain=-2:4] plot(\x,{-0.5*(\x)^3+1.5*(\x)^2+1.5*(\x)-1.5});
			%\draw[smooth,samples=300] plot(\x,{(\x)^3+(\x)^2-2*(\x)+1});
		\end{tikzpicture}
	}
	\loigiai{
		Ta có đường thẳng $y=x$ cắt đồ thị hàm số $y=f'(x)$ tại các điểm $x=-1 ; x=1 ; x=3$ như hình vẽ sau:
		\begin{center}
			\begin{tikzpicture}[scale=0.7,>=stealth, font=\footnotesize, line join=round, line cap=round]
				%\def\a{1} \def\b{-6} \def\c{9} \def\d{1} % Hệ số
				\def\xmin{-4} \def\xmax{5}
				\def\ymin{-3} \def\ymax{5} 
				%\draw[color=gray!50,dashed] (\xmin,\ymin) grid (\xmax,\ymax); 
				\draw[->] (\xmin,0)--(\xmax,0) node [below]{$x$};
				\draw[->] (0,\ymin)--(0,\ymax) node [left]{$y$};
				\node at (0,0) [below left]{$O$};
				%\node at (1,3) [below left]{$f'(x)$};
				\node at (-1.3,4) {$f'(x)$};
				\node at (4,3.2) {$y=x$};
				\draw[dashed] (-1,0) node[above]{$-1$}--(-1,-1)--(0,-1) node[below left]{$-1$};
				\draw[dashed] (1,0) node[below]{$1$}--(1,1)--(0,1) node[below left]{$1$};
				\draw[dashed] (3,0) node[below]{$3$}--(3,3)--(0,3) node[below left]{$3$};
				%\draw[dashed] (1,0) node[below]{$1$}--(1,1);
				%\draw[dashed] (-0.5,0) node[below left]{$-0{,}5$}--(-0.5,2.125);
				\clip (\xmin+0.1,\ymin+0.1) rectangle (\xmax-0.5,\ymax-0.1);
				\draw[smooth,samples=300][domain=-2:4] plot(\x,{-0.5*(\x)^3+1.5*(\x)^2+1.5*(\x)-1.5});
				\draw[smooth,samples=300] plot(\x,{(\x)});
			\end{tikzpicture}
		\end{center}
		Dựa vào đồ thị của hai hàm số trên ta có $f'(x)>x \Leftrightarrow\hoac{&x<-1 \\ &1<x<3}$ và
		$ f'(x)<x \Leftrightarrow\hoac{&
			-1<x<1 \\
			&x>3.}$\\
		+Trường hợp 1: $x-1<0 \Leftrightarrow x<1$.\\
		Khi đó $g(x)=2 f(1-x)-x^2+2 x+2020$.\\
		Ta có $g'(x)=-2 f'(1-x)+2(1-x)$.
		$$
		g'(x)>0 \Leftrightarrow-2 f'(1-x)+2(1-x)>0 \Leftrightarrow f'(1-x)<1-x \Leftrightarrow\hoac{
			&- 1 < 1 - x < 1\\
			&1 - x > 3} \Leftrightarrow \hoac{&
			0<x<2 \\
			&x<-2.}
		$$
		Kết hợp điều kiện, ta có $g'(x)>0 \Leftrightarrow\hoac{&0<x<1 \\ &x<-2.}$\\
		
		+ Trường hợp 2: $x-1>0 \Leftrightarrow x>1$.\\
		Khi đó ta có $g(x)=2 f(x-1)-x^2+2 x+2020$.\\
		$ g'(x)=2 f'(x-1)-2(x-1)$\\
		$g'(x)>0 \Leftrightarrow 2 f'(x-1)-2(x-1)>0 \Leftrightarrow f'(x-1)>x-1 \Leftrightarrow\hoac{&
			x - 1 < - 1\\
			&1 < x - 1 < 3}\Leftrightarrow \hoac{
			&x<0 \\
			&2<x<4.}$
		Kết hợp điều kiện ta có $g'(x)>0 \Leftrightarrow 2<x<4$.\\
		Vậy hàm số $g(x)=2 f(|x-1|)-x^2+2 x+2020$ đồng biến trên khoảng $(0 ; 1)$.
	}
\end{ex}

\begin{ex}[Chuyên Lào Cai - 2020]%[2D1G1-2]
	\immini{
		Cho hàm số $f'(x)$ có đồ thị như hình bên.\\
		Hàm số $g(x)=f(3 x+1)+9 x^3+\dfrac{9}{2}x^2$ đồng biến trên khoảng nào dưới đây?
		\choice
		{$(-1 ; 1)$}
		{$(-2 ; 0)$}
		{$(-\infty ; 0)$}
		{\True $(1 ;+\infty)$}
	}
	{\begin{tikzpicture}[line join=round, line cap=round,>=stealth,thick,scale=.8]
			\tikzset{label style/.style={font=\footnotesize}}
			\draw[->] (-2.1,0)--(5.1,0) node[below left] {$x$};
			\draw[->] (0,-3.1)--(0,4.1) node[below left] {$y$};
			\draw (0,0) node [below left] {$O$};
			\foreach \x in {1,2,3}
			\draw[thin] (\x,1pt)--(\x,-1pt) node [below] {$\x$};
			\draw[thin](-1,1pt)--(1,-1pt)node[above left]{$-1$};
			\foreach \y in {-2,2}
			\draw[thin] (1pt,\y)--(-1pt,\y) node [above right] {$\y$};
			%\begin{scope}
			\clip (-2,-3) rectangle (5,4);
			\draw[samples=200,domain=-2:4,smooth,variable=\x] plot (\x,{(\x)^3-3*(\x)^2+2});
			%\end{scope}
			\draw[dashed] (-1,0)--(-1,-2)--(0,-2);
			\draw[dashed] (3,0)--(3,2)--(0,2);
			%\begin{scope}[on background layer]\path[white]node{MDD-134};\end{scope}
		\end{tikzpicture}
	}
	\loigiai
	{
		\immini{Xét hàm số $g(x)=f(3 x+1)+9 x^3+\dfrac{9}{2}x^2 \\
			\Rightarrow g'(x)=3 f'(3 x+1)+27 x^2+9 x$.\\
			Hàm số đồng biến  $\Leftrightarrow g'(x)>0 \Leftrightarrow 3 f'(3 x+1)+27 x^2+9 x>0$
			\\
			$
			\Leftrightarrow f'(3 x+1)+3 x(3 x+1)>0 \qquad (*)
			$\\
			Đặt $t=3 x+1$, khi đó  $(*) \Leftrightarrow f'(t)+(t-1) t>0$\\ $\Leftrightarrow f'(t)>-t^2+t$.\\
			Vẽ parabol $y=-x^2+x$ và đồ thị hàm số $f'(x)$ trên cùng một hệ trục
		}
		{
			\begin{tikzpicture}[line join=round, line cap=round,>=stealth,thick,scale=.8]
				\tikzset{label style/.style={font=\footnotesize}}
				\draw[->] (-2.1,0)--(5.1,0) node[below left] {$x$};
				\draw[->] (0,-3.1)--(0,4.1) node[below left] {$y$};
				\draw (0,0) node [below left] {$O$};
				\foreach \x in {1,2,3}
				\draw[thin] (\x,1pt)--(\x,-1pt) node [below] {$\x$};
				\draw[thin](-1,1pt)--(1,-1pt);
				\foreach \y in {-2,2}
				\draw[thin] (1pt,\y)--(-1pt,\y) node [above right] {$\y$};
				%\begin{scope}
				\clip (-2,-3) rectangle (5,4);
				\draw[samples=200,domain=-2:4,smooth,variable=\x] plot (\x,{(\x)^3-3*(\x)^2+2});
				\draw[samples=200,domain=-2:4,smooth,variable=\x] plot (\x,{-(\x)^2+(\x)});
				%\end{scope}
				\draw[dashed] (-1,0) node[above left]{$-1$}--(-1,-2)--(0,-2);
				\draw[dashed] (3,0)--(3,2)--(0,2);
				%\begin{scope}[on background layer]\path[white]node{MDD-134};\end{scope}
			\end{tikzpicture}
		}
		Dựa vào đồ thị ta thấy
		$
		f'(t)>-t^2+t \Leftrightarrow\hoac{&- 1 < t < 1\\
			&t > 2}\Rightarrow \hoac{&
			- 1 < 3 x + 1 < 1\\
			&3 x + 1 > 2} \Leftrightarrow \hoac{&
			\dfrac{-2}{3}<x<0\\
			&x>\dfrac{1}{3}.}
		$}
\end{ex}
\begin{ex}[Sở Phú Thọ-2020]%[2D1G1-2]
	\immini{
		Cho hàm số $y=f(x)$ có đồ thị hàm số $y=f'(x)$ như hình vẽ.\\
		Hàm số $g(x)=f\left(\mathrm{e}^x-2\right)-2020$ nghịch biến trên khoảng nào dưới đây?
		\choice
		{\True $\left(-1 ; \dfrac{3}{2}\right)$}
		{$(-1 ; 2)$}
		{$(0 ;+\infty)$}
		{$\left(\dfrac{3}{2}; 2\right)$}
	}
	{
		\begin{tikzpicture}[scale=0.7,>=stealth, font=\footnotesize, line join=round, line cap=round]
			\def\a{1} \def\b{-3} \def\c{0} \def\d{0} % Hệ số
			\def\xmin{-2} \def\xmax{4}
			\def\ymin{-5} \def\ymax{2} 
			%\draw[color=gray!50,dashed] (\xmin,\ymin) grid (\xmax,\ymax); 
			\draw[->] (\xmin,0)--(\xmax,0) node [below]{$x$};
			\draw[->] (0,\ymin)--(0,\ymax) node [left]{$y$};
			\node at (0,0) [above left]{$O$};
			\node at (3,0) [below right]{$3$};
			\draw[dashed] (2,0) node[above]{$2$}--(2,-4) --(0,-4) node[left]{$-4$};
			\clip (\xmin+0.1,\ymin+0.1) rectangle (\xmax-0.5,\ymax-0.1);
			\draw[smooth,samples=300] plot(\x,{\a*(\x)^3+\b*(\x)^2+\c*(\x)+\d});
		\end{tikzpicture}
	}
	
	\loigiai{
		Dựa vào đồ thị hàm số $y=f'(x)$ suy ra $f'(x) \leq 0 ~ \forall x<3$ và $f'(x)>0 ~ \forall x>3$.
		$
		g'(x)=\mathrm{e}^x f'\left(\mathrm{e}^x-2\right) .
		$
		Hàm số $g(x)=f\left(\mathrm{e}^x-2\right)-2020$ nghịch biến \\ $ \Leftrightarrow g'(x)<0 \Leftrightarrow \mathrm{e}^x f'\left(\mathrm{e}^x-2\right)<0$\\
		$
		\Leftrightarrow f'\left(\mathrm{e}^x-2\right)<0 \Leftrightarrow \mathrm{e}^x-2<3 \Leftrightarrow \mathrm{e}^x<5 \Leftrightarrow x<\ln 5 .
		$\\
		Vậy hàm số đã cho nghịch biến trên $\left(-1 ; \dfrac{3}{2}\right)$.
	}
\end{ex}
\begin{ex}[Lý Nhân Tông - Bắc Ninh - 2020]%[2D1G1-2]
	\immini{
		Cho hàm số $f(x)$ có đồ thị hàm số $f'(x)$ như hình vẽ.\\
		Hàm số $y=f(\cos x)+x^2-x$ đồng biến trên khoảng
		\choice
		{$(-2 ; 1)$}
		{$(0 ; 1)$}
		{\True $(1 ; 2)$}
		{$(-1 ; 0)$}
	}
	{
		\begin{tikzpicture}[scale=1,>=stealth, font=\footnotesize, line join=round, line cap=round]
			\def\a{-0.5} \def\b{0} \def\c{1.5} \def\d{0} % Hệ số
			\def\xmin{-3} \def\xmax{4}
			\def\ymin{-2} \def\ymax{2} 
			%\draw[color=gray!50,dashed] (\xmin,\ymin) grid (\xmax,\ymax); 
			\draw[->] (\xmin,0)--(\xmax,0) node [below]{$x$};
			\draw[->] (0,\ymin)--(0,\ymax) node [left]{$y$};
			\node at (0,0) [above left]{$O$};
			\node at (3,0) [below right]{$3$};
			\draw[dashed] (-2,0) node[below]{$-2$}--(-2,1) --(0,1) node[above right]{$1$} --(1,1)--(1,0) node[below]{$1$};
			\draw[dashed] (-1,0) node[below right]{$-1$}--(-1,-1) --(0,-1) node[above right]{$-1$} --(2,-1)--(2,0) node[below right]{$2$};
			\clip (\xmin+0.1,\ymin+0.1) rectangle (\xmax-0.5,\ymax-0.1);
			\draw[smooth,samples=300][domain=-2:2] plot(\x,{\a*(\x)^3+\b*(\x)^2+\c*(\x)+\d});
		\end{tikzpicture}
	}
	\loigiai{
		Đặt  $g(x)=f(\cos x)+x^2-x$.\\
		Ta có $g'(x)=-\sin x \cdot f'(\cos x)+2 x-1$\\
		Vì $\cos x \in[-1 ; 1]$ nên từ đồ thị $f'(x)$ ta suy ra $f'(\cos x) \in[-1 ; 1]$.\\
		Do đó $\left|-\sin x \cdot f'(\cos x)\right| \leq 1, ~\forall x \in \mathbb{R}$.\\
		Ta suy ra $g'(x)=\sin x \cdot f'(\cos x)+2 x-1 \geq-1+2 x-1=2 x-2$
		$\Rightarrow g'(x)>0, ~\forall x>1$.\\
		Vậy hàm số đồng biến trên $(1 ; 2)$.
	}
\end{ex}
\begin{ex}[THPT Nguyễn Viết Xuân - 2020]%[2D1G1-2]
	\immini{
		Cho hàm số $f(x)$. Hàm số $y=f'(x)$ có đồ thị như hình vẽ.\\
		Hàm số $g(x)=f\left(3 x^2-1\right)-\dfrac{9}{2}x^4+3 x^2$ đồng biến trên khoảng nào dưới đây?
		\choice
		{\True $\left(-\dfrac{2 \sqrt{3}}{3}; \dfrac{-\sqrt{3}}{3}\right)$}
		{$\left(0 ; \dfrac{2 \sqrt{3}}{3}\right)$}
		{$(1 ; 2)$}
		{$\left(-\dfrac{\sqrt{3}}{3}; \dfrac{\sqrt{3}}{3}\right)$} 
	}
	{
		\begin{tikzpicture}[scale=0.6,>=stealth, font=\footnotesize, line join=round, line cap=round]
			\def\a{0.25} \def\b{0.25} \def\c{-2} \def\d{0} % Hệ số
			\def\xmin{-5} \def\xmax{4}
			\def\ymin{-5} \def\ymax{5} 
			%\draw[color=gray!50,dashed] (\xmin,\ymin) grid (\xmax,\ymax); 
			\draw[->] (\xmin,0)--(\xmax,0) node [below]{$x$};
			\draw[->] (0,\ymin)--(0,\ymax) node [left]{$y$};
			\node at (0,0) [above left]{$O$};
			%\node at (3,0) [below right]{$3$};
			\draw[dashed] (-4,0) node[below left]{$-4$}--(-4,-4) --(0,-4) node[above right]{$-4$};
			\draw[dashed] (3,0) node[below right]{$3$}--(3,3) --(0,3) node[above right]{$3$};
			\clip (\xmin+0.1,\ymin+0.1) rectangle (\xmax-0.5,\ymax-0.1);
			\draw[smooth,samples=300] plot(\x,{\a*(\x)^3+\b*(\x)^2+\c*(\x)+\d});
		\end{tikzpicture}
	}
	
	\loigiai
	{
		TXĐ: $\mathscr{D}=\mathbb{R}$.\\
		Ta có $g'(x)=6 x f'\left(3 x^2-1\right)-18 x^3+6 x=6 x\left[f'\left(3 x^2-1\right)-3 x^2+1\right]$.\\
		$
		g'(x)=0 \Leftrightarrow\hoac{
			&x = 0\\
			&f '( 3 x ^{2}- 1 ) = 3 x ^{2}- 1}
		\Leftrightarrow \hoac{
			&x = 0\\
			&3 x ^{2}- 1 = - 4 \text{~(vô nghiệm)}\\
			&3 x ^{2}- 1 = 0\\
			&3 x ^{2}- 1 = 3}\Leftrightarrow \hoac{&x=0 \\
			&x=\pm \dfrac{\sqrt{3}}{3}\\
			&x=\pm \dfrac{2 \sqrt{3}}{3}.}
		$\\
		Bảng xét dấu
		\begin{center}
			\begin{tikzpicture}
				\tkzTabInit[nocadre,lgt=1.2,espcl=2.2,deltacl=0.6]
				{$x$ /1.2,$f'(x)$ /0.7}
				{$-\infty$,$-\dfrac{2 \sqrt{3}}{3}$,$-\dfrac{ \sqrt{3}}{3}$,$0$,$\dfrac{\sqrt{3}}{3}$,$\dfrac{2 \sqrt{3}}{3}$,$+\infty$}
				\tkzTabLine{,-,$0$,+,$0$,-,$0$,+,$0$,-,$0$,+,}
			\end{tikzpicture}
		\end{center}
		Vậy hàm số đồng biến trong khoảng $\left(-\dfrac{2 \sqrt{3}}{3}; \dfrac{-\sqrt{3}}{3}\right)$.}
\end{ex}
\begin{ex}[Trần Phú - Quảng Ninh - 2020]%[2D1G1-2]
	Cho hàm số $f(x)$ có bảng xét dấu của đạo hàm như sau
	\begin{center}
		\begin{tikzpicture}
			\tkzTabInit[nocadre,lgt=1.2,espcl=2,deltacl=0.6]
			{$x$ /0.6,$f'(x)$ /0.6}
			{$-\infty$,$-4$,$-1$,$2$,$7$,$+\infty$}
			\tkzTabLine{,+,$0$,-,$0$,+,$0$,-,$0$,+,}
		\end{tikzpicture}
	\end{center}
	Hàm số $y=f(2 x+1)+\dfrac{2}{3}x^3-8 x+5$ nghịch biến trên khoảng nào dưới đây?
	\choice
	{$(-\infty ;-2)$}
	{$(1 ;+\infty)$}
	{$(-1 ; 7)$}
	{\True $\left(-1 ; \dfrac{1}{2}\right)$}
	\loigiai{
		Ta có $y'=2 f'(2 x+1)+2 x^2-8$.\\
		Xét $y'\leq 0 \Leftrightarrow 2 f'(2 x+1)+2 x^2-8 \leq 0 \Leftrightarrow f'(2 x+1) \leq 4-x^2$.\\
		Đặt $t=2x+1$, ta có $f'(t) \leq \dfrac{-t^2+2 t+15}{4}$.\\
		Vì $\dfrac{-t^2+2 t+15}{4}\geq 0, \forall t \in[-3 ; 5]$.\\
		Mà $f'(t) \leq 0, \forall t \in[-3 ; 2]$.\\
		Nên $f'(t) \leq \dfrac{-t^2+2 t+15}{4}\Rightarrow t \in[-3 ; 2]$.\\
		Suy ra $-3 \leq 2 x+1 \leq 2 \Leftrightarrow-2 \leq x \leq \dfrac{1}{2}$.}
\end{ex}

\begin{ex}[Chuyên Thái Bình - Lần 3 - 2020]%[2D1G1-2]
	\immini{
		Cho hàm số $y=f(x)$ liên tục trên $\mathbb{R}$ có đồ thị hàm số $y=f'(x)$ cho như hình vẽ.\\
		Hàm số $g(x)=2 f(|x-1|)-x^2+2 x+2020$ đồng biến trên khoảng nào?
		\choice
		{\True $(0 ; 1)$}
		{$(-3 ; 1)$}
		{$(1 ; 3)$}
		{$(-2 ; 0)$}
	}
	{
		\begin{tikzpicture}[scale=0.7,>=stealth, font=\footnotesize, line join=round, line cap=round]
			\def\a{-0.333} \def\b{1} \def\c{1.333} \def\d{-1} % Hệ số
			\def\xmin{-3} \def\xmax{5}
			\def\ymin{-3} \def\ymax{5} 
			%\draw[color=gray!50,dashed] (\xmin,\ymin) grid (\xmax,\ymax); 
			\draw[->] (\xmin,0)--(\xmax,0) node [below]{$x$};
			\draw[->] (0,\ymin)--(0,\ymax) node [left]{$y$};
			\node at (0,0) [above left]{$O$};
			%\node at (3,0) [below right]{$3$};
			\draw[dashed] (-1,0) node[above]{$-1$}--(-1,-1) --(0,-1) node[above right]{$-1$};
			\draw[dashed] (1,0) node[below right]{$1$}--(1,1) --(0,1) node[above right]{$1$};
			\draw[dashed] (3,0) node[below right]{$3$}--(3,3) --(0,3) node[above right]{$3$};
			\clip (\xmin+0.1,\ymin+0.1) rectangle (\xmax-0.5,\ymax-0.1);
			\draw[smooth,samples=300] plot(\x,{\a*(\x)^3+\b*(\x)^2+\c*(\x)+\d});
			\draw[smooth,samples=300] plot(\x,{(\x)});
		\end{tikzpicture}
	}
	\loigiai{
		Với $x>1$, ta có $g(x)=2 f(x-1)-(x-1)^2+2021 \Rightarrow g'(x)=2 f'(x-1)-2(x-1)$.\\
		Hàm số đồng biến $\Leftrightarrow 2 f'(x-1)-2(x-1)>0 \Leftrightarrow f'(x-1)>x-1 \quad(*)$.\\
		Đặt $t=x-1$, khi đó $(*) \Leftrightarrow f'(t)>t \Leftrightarrow\hoac{&1<t<3 \\ &t<-1}\Rightarrow\hoac{&2<x<4 \\ &x<0 ~(\text{loại}).}$\\
		Với $x<1$, ta có $g(x)=2 f(1-x)-(1-x)^2+2021 \Rightarrow g'(x)=-2 f'(1-x)+2(1-x)$.\\
		Hàm số đồng biến $\Leftrightarrow-2 f'(1-x)+2(1-x)>0 \Leftrightarrow f'(1-x)<1-x \quad(* *)$.\\
		Đặt $t=1-x$, khi đó $(* *) \Leftrightarrow f'(t)<t \Leftrightarrow\hoac{&-1<t<1 \\ &t>3}\Rightarrow\hoac{&0<x<2 \\ &x<-2}\Rightarrow\hoac{&0<x<1 \\ &x<-2.}$\\
		Vậy hàm số $g(x)$ đồng biến trên các khoảng $(-\infty ;-2),(0 ; 1),(2 ; 4)$.
	}
\end{ex}
\begin{ex}[Sở Phú Thọ - 2020]%[2D1G1-2]
	\immini{
		Cho hàm số $y=f(x)$ có đồ thị hàm số $f'(x)$ như hình vẽ.\\
		Hàm số $g(x)=f\left(1+e^x\right)+2020$ nghịch biến trên khoảng nào dưới đây?
		\choice
		{$(0 ;+\infty)$}
		{$\left(\dfrac{1}{2}; 1\right)$}
		{\True $\left(0 ; \dfrac{1}{2}\right)$}
		{$(-1 ; 1)$}
	}{
		\begin{tikzpicture}[scale=0.7,>=stealth, font=\footnotesize, line join=round, line cap=round]
			\def\a{1} \def\b{-3} \def\c{0} \def\d{0} % Hệ số
			\def\xmin{-2} \def\xmax{4}
			\def\ymin{-5} \def\ymax{2} 
			%\draw[color=gray!50,dashed] (\xmin,\ymin) grid (\xmax,\ymax); 
			\draw[->] (\xmin,0)--(\xmax,0) node [below]{$x$};
			\draw[->] (0,\ymin)--(0,\ymax) node [left]{$y$};
			\node at (0,0) [above left]{$O$};
			\node at (3,0) [below right]{$3$};
			\draw[dashed] (2,0) node[above]{$2$}--(2,-4) --(0,-4) node[left]{$-4$};
			\clip (\xmin+0.1,\ymin+0.1) rectangle (\xmax-0.5,\ymax-0.1);
			\draw[smooth,samples=300] plot(\x,{\a*(\x)^3+\b*(\x)^2+\c*(\x)+\d});
		\end{tikzpicture}
	}
	\loigiai{
		$g'(x)=e^x f'\left(1+e^x\right)$.\\
		Do $e^x>0, \forall x$ nên $g'(x) \leq 0 \Leftrightarrow f'\left(1+e^x\right) \leq 0 \Leftrightarrow 1+e^x \leq 3 \Leftrightarrow x \leq \ln 2$, dấu bằng xảy ra tại hữu hạn điểm.\\
		Nên $g(x)$ nghịch biến trên $(-\infty ; \ln 2)$.\\
		Vì $\left(0 ; \dfrac{1}{2}\right) \subset (-\infty ; \ln 2)$ nên hàm số đã cho nghịch biến trên $\left(0 ; \dfrac{1}{2}\right)$.
	}
\end{ex}

\begin{ex}%[2D1K1-2]
	[THPT Anh Sơn - Nghệ An - 2020]
	Cho hàm số $y=f(x)$ có bảng xét dấu của đạo hàm như sau.
	\begin{center}
		\begin{tikzpicture}
			\tkzTabInit[nocadre,lgt=1.2,espcl=2,deltacl=0.6]
			{$x$ /0.6,$f'(x)$ /0.6}
			{$-\infty$,$-2$,$-1$,$2$,$4$,$+\infty$}
			\tkzTabLine{,+,$0$,-,$0$,+,$0$,-,$0$,+,}
		\end{tikzpicture}
	\end{center}
	Hàm số $y=-2 f(x)+2019$ nghịch biến trên khoảng nào trong các khoảng dưới đây?
	\choice
	{$(2 ; 4)$}
	{$(-4 ; 2)$}
	{$(-2 ;-1)$}
	{\True $(-1 ; 2)$}
	\loigiai{
		Ta có $y'=-2 f'(x)$.\\
		$
		y'=0 \Leftrightarrow-2 f'(x)=0 \Leftrightarrow\hoac{&
			x=-2 \\
			&x=-1 \\
			&x=2 \\
			&x=4.}$\\
		Từ bảng xét dấu của $f'(x)$ ta có
		\begin{center}
			\begin{tikzpicture}
				\tkzTabInit[nocadre,lgt=1,espcl=2,deltacl=0.6]
				{$x$ /0.6,$y'$ /0.6}
				{$-\infty$,$-2$,$-1$,$2$,$4$,$+\infty$}
				\tkzTabLine{,-,$0$,+,$0$,-,$0$,+,$0$,-,}
			\end{tikzpicture}
		\end{center}
		Từ bảng xét dấu ta có hàm số nghịch biến trên khoảng $(-\infty ;-2),(-1 ; 2)$ và $(4 ;+\infty)$.}
\end{ex}

\begin{ex}[THPT Anh Sơn - Nghệ An - 2020]%[2D1G1-2]
	Cho hàm số $f(x)$ xác định và liên tục trên $\mathbb{R}$ và có đạo hàm $f'(x)$ thỏa mãn $f'(x)=(1-x)(x+2) g(x)+2019$ với $g(x)<0, ~\forall x \in \mathbb{R}$ . Hàm số $y=f(1-x)+2019 x+2020$ nghịch biến trên khoảng nào?
	\choice
	{$(1 ;+\infty)$}
	{$(0 ; 3)$}
	{$(-\infty ; 3)$}
	{\True $(3 ;+\infty)$}
	\loigiai{
		Đặt $h(x)=f(1-x)+2019 x+2020$.\\
		Vì hàm số $f(x)$ xác định trên $\mathbb{R}$ nên hàm số $h(x)$ cũng xác định trên $\mathbb{R}$.\\
		Ta có $h'(x)=-f'(1-x)+2019$.\\
		Do $h'(x)=0$ tại hữu hạn điểm nên để tìm khoảng nghịch biến của hàm số $h(x)$, ta tìm các giá trị của $x$ sao cho $h'(x)<0 \Leftrightarrow-f'(1-x)+2019<0$\\ 
		$\Leftrightarrow f'(1-x)-2019>0 \\
		\Leftrightarrow x(3-x) g(1-x)>0 \Leftrightarrow x(3-x)<0(\text{~do~}g(x)<0, \forall x \in \mathbb{R})$\\
		$\Leftrightarrow\hoac{&
			x<0 \\
			&x>3.}$\\
		Vậy hàm số $y=f(1-x)+2019 x+2020$ nghịch biến trên các khoảng $(-\infty ; 0)$ và $(3 ;+\infty).$}
\end{ex}

\begin{ex}%[2D1G1-2]
	Cho hàm số $y=f(x)$ xác định trên $\mathbb{R}$ và có bảng xét dấu đạo hàm như sau:
	\begin{center}
		\begin{tikzpicture}
			\tkzTabInit[nocadre,lgt=2,espcl=2,deltacl=0.6]
			{$x$ /0.6,$f'(x)$ /0.6}
			{$-\infty$,$-1$,$1$,$4$,$+\infty$}
			\tkzTabLine{,-,$0$,+,$0$,-,$0$,+,}
		\end{tikzpicture}
	\end{center}
	Biết $f(x)>2,~ \forall x \in \mathbb{R}$. Xét hàm số $g(x)=f(3-2 f(x))-x^3+3 x^2-2020$. Khẳng định nào sau đây đúng?
	\choice
	{Hàm số $g(x)$ đồng biến trên khoảng $(-2 ;-1)$}
	{Hàm số $g(x)$ nghịch biến trên khoảng $(0 ; 1)$}
	{Hàm số $g(x)$ đồng biến trên khoảng $(3 ; 4)$}
	{\True Hàm số $g(x)$ nghịch biến trên khoảng $(2 ; 3)$}
	\loigiai{
		Ta có $g'(x)=-2 f'(x) f'(3-2 f(x))-3 x^2+6 x$.\\
		Vì $f(x)>2, ~\forall x \in \mathbb{R}$ nên $3-2 f(x)<-1 ~\forall x \in \mathbb{R}$.\\
		Từ bảng xét dấu $f'(x)$ suy ra $f'(3-2 f(x))<0, ~\forall x \in \mathbb{R}$.\\
		Từ đó ta có bảng xét dấu sau:
		\begin{center}
			\begin{tikzpicture}
				\tkzTabInit[nocadre,lgt=4,espcl=1.7,deltacl=0.6]
				{$x$ /0.7,$-f'(x)f'\left( 3-2f(x)\right) $/0.8,$-3x^2+6x$/0.7}
				{$-\infty$,$-1$,$0$,$1$,$2$,$4$,$+\infty$}
				\tkzTabLine{,-,$0$,+,|,+,$0$,-,|,-,$0$,+,}
				\tkzTabLine{,-,|,-,$0$,+,|,+,$0$,-,|,-,}
			\end{tikzpicture}
		\end{center}
		Từ bảng xét dấu trên, loại trừ đáp án suy ra hàm số $g(x)$ nghịch biến trên khoảng $(2 ; 3)$.}
\end{ex}

\begin{ex}%[2D1G1-2]
	Cho hàm số $f(x)$ có bảng biến thiên như sau:
	\begin{center}
		\begin{tikzpicture}
			\tkzTabInit[nocadre,lgt=1.2,espcl=2.5,deltacl=0.6]
			{$x$ /0.7, $f'(x)$ /0.7, $f(x)$ /2.5}
			{$-\infty$,$1$,$2$,$3$,$4$,$+\infty$}
			\tkzTabLine{,+,$0$,-,$0$,+,$0$,-,$0$,+,}
			\tkzTabVar{-/$-\infty$,+/$3$,-/$1$,+/$2$,-/$0$,+/$+\infty$}
		\end{tikzpicture}
	\end{center}
	Hàm số $y=(f(x))^3-3 .(f(x))^2$ nghịch biến trên khoảng nào dưới đây?
	\choice
	{$(1 ; 2)$}
	{$(3 ; 4)$}
	{$(-\infty ; 1)$}
	{\True $(2 ; 3)$}
	\loigiai{
		Ta có $y'=3 \cdot(f(x))^2 \cdot f'(x)-6 \cdot f(x) \cdot f'(x)=3 f(x) \cdot f'(x) \cdot[f(x)-2]. \\
		y'=0 \Leftrightarrow \hoac{&f(x)=0 \Leftrightarrow x \in\left\{x_1, 4 \mid x_1<1\right\}\\
			&f(x)=2 \Leftrightarrow x \in\left\{x_2, x_3, 3, x_4 \mid x_1<x_2<1<x_3<2 ; 4<x_4\right\}\\
			&f'(x)=0 \Leftrightarrow x \in\{1,2,3,4\}.}$\\
		Lập bảng xét dấu ta có
		\begin{center}
			\begin{tikzpicture}
				\tkzTabInit[nocadre,lgt=2,espcl=1.5,deltacl=0.6]
				{$x$ /0.7,$f(x)$ /0.7,$f(x)-2$ /0.7,$f'(x)$/0.7,$y'$/0.7}
				{$-\infty$,$x_1$,$x_2$,$1$,$x_3$,$2$,$3$,$4$,$x_4$,$+\infty$}
				\tkzTabLine{,-,$0$,+,|,+,|,+,|,+,|,+,$0$,+,|,+,|,+,}
				\tkzTabLine{,-,|,-,$0$,+,$0$,+,$0$,-,|,-,$0$,-,|,-,$0$,+}
				\tkzTabLine{,+,|,+,|,+,$0$,-,|,-,$0$,+,$0$,-,$0$,+,|,+}
				\tkzTabLine{,+,$0$,-,$0$,+,$0$,-,$0$,+,$0$,-,$0$,+,$0$,-,$0$,+}
			\end{tikzpicture}
		\end{center}
		
		Do đó hàm số nghịch biến trên khoảng $(2 ; 3)$.
	}
\end{ex}
\begin{ex}%[2D1G1-2]
	Cho hàm số $y=f(x)$ có đồ thị nằm trên trục hoành và có đạo hàm trên $\mathbb{R}$, bảng xét dấu của biểu thức $f'(x)$ như bảng dưới đây.
	\begin{center}
		\begin{tikzpicture}
			\tkzTabInit[nocadre,lgt=1.2,espcl=2,deltacl=0.6]
			{$x$ /0.6,$f'(x)$ /0.6}
			{$-\infty$,$-2$,$-1$,$3$,$+\infty$}
			\tkzTabLine{,-,$0$,+,$0$,-,$0$,+,}
		\end{tikzpicture}
	\end{center}
	Hàm số $y=g(x)=\dfrac{f\left(x^2-2 x\right)}{f\left(x^2-2 x\right)+1}$ nghịch biến trên khoảng nào dưới đây?
	\choice
	{$(-\infty ; 1)$}
	{$\left(-2 ; \dfrac{5}{2}\right)$}
	{\True $(1 ; 3)$}
	{$(2 ;+\infty)$}
	\loigiai{
		$ g'(x)=\dfrac{\left(x^2-2 x\right)'\cdot f'\left(x^2-2 x\right)}{\left(f\left(x^2-2 x\right)+1\right)^2}=\dfrac{(2 x-2) \cdot f'\left(x^2-2 x\right)}{\left(f\left(x^2-2 x\right)+1\right)^2}. \\
		g'(x)=0 \Leftrightarrow\hoac{
			&2 x - 2 = 0\\
			&f '( x ^{2}- 2 x ) = 0}
		\Leftrightarrow \hoac{&x = 1\\
			&x ^{2}- 2 x = - 2\\
			&x ^{2}- 2 x = - 1\\
			&x ^{2}- 2 x = 3}
		\Leftrightarrow \hoac{&x=1 \\
			&x=-1 \\
			&x=3.}
		$\\
		Ta có bảng xét dấu của $g'(x)$
		\begin{center}
			\begin{tikzpicture}
				\tkzTabInit[nocadre,lgt=1.2,espcl=2,deltacl=0.6]
				{$x$ /0.6,$g'(x)$ /0.6}
				{$-\infty$,$-1$,$1$,$3$,$+\infty$}
				\tkzTabLine{,-,$0$,+,$0$,-,$0$,+,}
			\end{tikzpicture}
		\end{center}
		Dựa vào bảng xét dấu ta có hàm số $y=g(x)$ nghịch biến trên các khoảng $(-\infty ;-1)$ và $(1 ; 3)$.}
\end{ex}
\begin{ex}[Liên trường huyện Quảng Xương - Thanh Hóa - 2021]%[2D1G1-2]
	\immini{
		Cho các hàm số $y=f(x)$; $y=g(x)$ liên tục trên $\mathbb{R}$ và có đồ thị các đạo hàm $f'(x) ; g'(x)$ (đồ thị hàm số $y=g'(x)$ là đường đậm hơn) như hình vẽ.\\
		Hàm số $h(x)=f(x-1)-g(x-1)$ nghịch biến trên khoảng nào dưới đây?
		\choice
		{$\left(\dfrac{1}{2}; 1\right)$}
		{$(1 ;+\infty)$}
		{$(2 ;+\infty)$}
		{\True $\left(-1 ; \dfrac{1}{2}\right)$}
	}
	{
		\begin{tikzpicture}[scale=1,>=stealth, font=\footnotesize, line join=round, line cap=round]
			%\def\a{1} \def\b{-6} \def\c{9} \def\d{1} % Hệ số
			\def\xmin{-4} \def\xmax{3}
			\def\ymin{-2} \def\ymax{4} 
			%\draw[color=gray!50,dashed] (\xmin,\ymin) grid (\xmax,\ymax); 
			\draw[->] (\xmin,0)--(\xmax,0) node [below]{$x$};
			\draw[->] (0,\ymin)--(0,\ymax) node [left]{$y$};
			\node at (0,0) [above left]{$O$};
			\node at (1,3) [below left]{$f'(x)$};
			\node at (1.5,3) [below right]{$g'(x)$};
			\draw[dashed] (-2,0) node[above right]{$-2$}--(-2,1);
			\draw[dashed] (1,0) node[below]{$1$}--(1,1);
			\draw[dashed] (-0.5,0) node[below]{$-0{,}5$}--(-0.5,2.125);
			\clip (\xmin+0.1,\ymin+0.1) rectangle (\xmax-0.5,\ymax-0.1);
			\draw[smooth,samples=300][domain=-3:2] plot(\x,{2*(\x)^4+4*(\x)^3-2*(\x)^2-4*(\x)+1});
			\draw[smooth,samples=300,line width=1.2pt] plot(\x,{(\x)^3+(\x)^2-2*(\x)+1});
		\end{tikzpicture}
	}
	
	\loigiai{
		Ta có: $h'(x)=f'(x-1)-g'(x-1)$.\\
		Dựa vào hình vẽ ta có hàm số $h(x)$ nghịch biến\\
		$\Leftrightarrow h'(x)<0 \Leftrightarrow f'(x-1)<g'(x-1)$\\
		$
		\Leftrightarrow\hoac{&- 2 < x - 1 < - \dfrac{1}{2}\\
			&0 < x - 1 < 1}
		\Leftrightarrow \hoac{
			&-1<x<\dfrac{1}{2}\\
			&1<x<2.}$\\
		Do đó hàm số $h(x)$ nghịch biến trên các khoảng $\left(-1 ; \dfrac{1}{2}\right)$ và $(1 ; 2)$.
	}
\end{ex}
\begin{ex}[THPT Quế Võ 1 - Bắc Ninh - 2021] %[2D1G1-2]
	\immini{
		Cho ba hàm số $y=f(x), y=g(x), y=h(x)$. Đồ thị của ba hàm số $y=f'(x), y=g'(x), y=h'(x)$ được cho như hình vẽ.\\
		Hàm số $k(x)=f(x+7)+g(5 x+1)-h\left(4 x+\dfrac{3}{2}\right)$ đồng biến trên khoảng nào dưới đây?
		\choice
		{$\left(-\dfrac{5}{8}; 0\right)$}
		{$\left(\dfrac{5}{8};+\infty\right)$}
		{\True $\left(\dfrac{3}{8}; 1\right)$}
		{$\left(-\dfrac{3}{8}; 1\right)$}
	}
	{
		\begin{tikzpicture}[scale=0.25,>=stealth, font=\footnotesize, line join=round, line cap=round]
			\def\a{-.078} \def\b{1.25} \def\c{0} % Hệ số
			\def\xmin{-4} \def\xmax{25}
			\def\ymin{-8} \def\ymax{18}
			
			%\draw[color=gray!50,dashed] (\xmin,\ymin) grid (\xmax,\ymax);
			
			\draw[->] (\xmin,0)--(\xmax,0) node [below]{$x$};
			\draw[->] (0,\ymin)--(0,\ymax) node [left]{$y$};
			\node at (20,14) [below right]{$y=g'(x)$};
			\node at (18,-2) [below left]{$y=h'(x)$};
			\node at (16,5) [below right]{$y=f'(x)$};
			\node at (0,0) [below left]{$O$};
			\draw[dashed] (3,0) node[below]{$3$}--(3,10)--(0,10) node[left]{$10$};
			\draw[dashed] (8,0) node[below]{$8$}--(8,5)--(0,5) node[left]{$5$};
			\draw[dashed] (4,0) node[below]{$4$}--(4,2)--(0,2) node[left]{$2$};
			\clip (\xmin+0.1,\ymin+0.1) rectangle (\xmax-0.5,\ymax-0.1);
			\draw[smooth,samples=300,domain=-2:18] plot(\x,{\a*(\x)^2+\b*(\x)+\c});
			%\draw[smooth,samples=300,domain=-2:25] plot(\x,{0.02*(\x)^3-0.6*(\x)^2+5.16*(\x)});
			\draw[line width=1.2pt] (-2,5)..controls (1.7,1.5) and (4.5,1.6)..(7,2.6);
			\draw[line width=1.2pt] (7,2.6)..controls (9,3.5) and (12,5)..(20,13);
			\draw (-0.5,-2) -- (0,0)--(3,10).. controls +(65:1) and + (-190:1)..(6,15).. controls +(0:1) and + (-180:1)..(14,-1).. controls +(0:1) and + (+80:1)..(19,16);
			
		\end{tikzpicture}
	}
	\loigiai{
		Ta có $k'(x)=f'(x+7)+5 g'(5 x+1)-4 h'\left(4 x+\dfrac{3}{2}\right)$.\\
		Khi $x \in \left( \dfrac{3}{8};1\right)$ thì $\heva{&7{,}375<x+7<8\\&2{,}875<5x+1<6\\&3<4x+\dfrac{4}{3}<5{,}5}\Leftrightarrow \heva{&f'(x+7)>10\\&g'(5x+1)>2 \Rightarrow 5g'(5x+1)>10  \\&h'\left( 4x+\dfrac{3}{2}\right)<5 \Rightarrow -4h'\left( 4x+\dfrac{3}{2}\right) >-20}.$\\
		Do đó $k'(x)=f'(x+7)+5g'(5x+1)-4h'\left( 4x+\dfrac{3}{2}\right)>0$.\\
		Hàm số $k(x)=f(x+7)+g(5 x+1)-h\left(4 x+\dfrac{3}{2}\right)$ đồng biến trên $\left(\dfrac{3}{8}; 1\right)$.
	}
\end{ex}
\begin{ex}[THPT Thanh Chương 1 - Nghệ An- 2021] %[2D1G1-2]
	Cho hàm số $y=f(x)$ liên tục trên $\mathbb{R}$ có bảng xét dấu đạo hàm như sau
	\begin{center}
		\begin{tikzpicture}
			\tkzTabInit[nocadre,lgt=1.2,espcl=2,deltacl=0.6]
			{$x$ /0.6,$f'(x)$ /0.6}
			{$-\infty$,$1$,$2$,$3$,$4$,$+\infty$}
			\tkzTabLine{,-,$0$,+,$0$,+,$0$,-,$0$,+,}
		\end{tikzpicture}
	\end{center}
	Hàm số $y=3f(2x-1)-4x^3+15x^2-18x+1$ đồng biến trên khoảng nào dưới đây?
	\choice
	{$\left(3;+\infty\right)$}
	{\True $\left(1;\dfrac{3}{2}\right)$}
	{$\left(\dfrac{5}{2}; 3\right)$}
	{$\left(2;\dfrac{5}{2}\right)$}
	\loigiai{
		Ta có $y'=6f'(2x-1)-12x^2+30x-18=6\left[f'(2x-1)-2x^2+5x-3\right] $.\\
		Có $f'(2x-1)=0 \Leftrightarrow \hoac{&2x-1=1\\&2x-1=2\\&2x-1=3\\&2x-1=4} \Leftrightarrow \hoac{&x=1\\&x=\dfrac{3}{2}\\&x=2\\&x=\dfrac{5}{2}.}$
		Ta có bảng xét dấu sau
		\begin{center}
			\begin{tikzpicture}
				\tkzTabInit[nocadre,lgt=3.0,espcl=1.5,deltacl=0.6]
				{$x$ /1.0,$f(x)$ /0.6,$f'(2x-1)$ /0.6,$-2x^2+5x-3$/0.6,$g'(x)$/0.6}
				{$-\infty$,$1$,$\dfrac{3}{2}$,$2$,$\dfrac{5}{2}$,$3$,$4$,$+\infty$}
				\tkzTabLine{,-,$0$,+,|,+,$0$,+,|,+,$0$,-,$0$,+,}
				\tkzTabLine{,-,$0$,+,$0$,+,$0$,-,$0$,+,|,+,|,+,}
				\tkzTabLine{,-,$0$,+,$0$,-,|,-,|,-,|,-,|,-,}
				\tkzTabLine{,-,$0$,+,$0$,,?,,|,,?,?,,?,}
			\end{tikzpicture}
		\end{center}
		Dựa vào bảng xét dấu trên, ta kết luận hàm số đã cho đồng biến trên khoảng $\left( 1; \dfrac{3}{2}\right).$
	}
\end{ex}


\begin{ex}%[2D2G4-3] %Câu 27 
	[THPT Hoàng Hoa Thám-Đà Nẵng-2021]
	Cho hàm số $f(x)$ có bảng xét dấu của $f'(x)$ như sau:\\
	\begin{center}
		\begin{tikzpicture}
			\tkzTabInit[lgt=1.2,espcl=2.3]
			{$x$/0.7, $f'(x)$ /.8} % first column
			{$-\infty$,$-3$,$1$, $2$, $+\infty$} % first row
			\tkzTabLine { ,+,0,-,0,+,0,+ }
		\end{tikzpicture}
	\end{center}	
	Hàm số $y=f\left(2-e^x\right)-\dfrac{1}{3}{e^{3x}}+3e^{2x}-5e^x+1$ đồng biến trên khoảng nào dưới đây?
	\choice
	{$\left(0;\dfrac{3}{2}\right)$}
	{$\left(1;3\right)$}
	{\True $\left(-3;0\right)$}
	{$\left(-4;-3\right)$}
	\loigiai{
		Ta có $y'=-e^x.f'\left(2-e^x\right)-e^{3x}+6e^{2x}-5e^x=e^x\left[-f'\left(2-e^x\right)-e^{2x}+6e^x-5\right]$ .\\
		Đặt $t=2-e^x$, ta được\\
		$y'=\left(2-t\right)\left[-f'(t)-\left(2-t\right)^2+6\left(2-t\right)-5\right]=\left(2-t\right)\left[-f'(t)-t^2-2t+3\right]$ .\\
		$y'=0\Leftrightarrow\left(2-t\right)\left[-f'(t)-t^2-2t+3\right]=0\Leftrightarrow
		\hoac{
			& t=2\\ 
			& f'(t)=-t^2-2t+3.}$\\
		Hàm số $g(x)=-x^2-2x+3$ là parabol có trục đối xứng $x=-1$ và cắt trục hoành tại 2 điểm có hoành độ 
		$\hoac{
			& x=1\\ 
			& x=-3
		}$. Suy ra $f'(t)=-t^2-2t+3\Leftrightarrow \hoac{
			& t=1\\ 
			& t=-3. }$\\
		Bảng xét dấu\\
		\begin{center}
			\begin{tikzpicture}
				\tkzTabInit[lgt=3.9,espcl=2,nocadre]
				{$t$/0.7, $2-t$ /0.8, $-f'(t)-t^2-2t+3$ /0.8, $y'$ /0.8} % first column
				{$-\infty$,$-3$,$1$,$2$,$+\infty$} % first row
				\tkzTabLine { ,+,|,+,|,+,z,-, } % second row
				\tkzTabLine {,-,0,+,0,-,|,-,} % third row
				\tkzTabLine {,-,0,+,0,-,0,+,} % last row
			\end{tikzpicture}
		\end{center}
		Dựa vào bảng xét dấu $y'>0,\forall x\in\left(-3;0\right)$.}
\end{ex}


\begin{ex}%[2D1G1-2]%Câu 28 
	[Sở Lạng Sơn 2022] Cho hàm số $f(x)$ có bảng biến thiên như sau:\\
	\begin{center}
		\begin{tikzpicture}
			\tkzTabInit[espcl=2.5,lgt=1,nocadre]
			{$x$/0.7,$y'$/0.7,$y$/3.5}
			{$-\infty$,$1$,$2$,$3$,$4$,$+\infty$}
			\tkzTabLine{,+,0,-,0,+,0,-,0,+,}
			\node (0) at ($(N12)+(0,-3)$) {$-\infty$};
			\node (1) at ($(N22)+(0,-.5)$) {$3$};
			\node (2) at ($(N32)+(0,-1.7)$) {$1$};
			\node (3) at ($(N42)+(0,-0.7)$) {$2$};
			\node (4) at ($(N52)+(0,-2.3)$) {$0$};
			\node (5) at ($(N62)+(0,-.3)$) {$+\infty$};
			%				\node (8) at ($(N42)+(0,-.5)$) {};
			%				\coordinate (9) at ($(N42)!.6!(N53)+ (-0.5,0)$);
			%				\coordinate (6) at ($(T12)!.6!(T13)$);
			%				\coordinate (7) at ($(T22)!.6!(T23)$);
			\draw[-stealth] (0)--(1);
			\draw[-stealth] (1)--(2);
			\draw[-stealth] (2)--(3);
			\draw[-stealth] (1)--(2);
			\draw[-stealth] (3)--(4);
			\draw[-stealth] (4)--(5);
			%				\draw[->,red] (5)--(8);
			%				\draw[->,red] (8)--(9);
			%				\draw[blue,dashed](6)--(7)node[above left]{$y=0$};
		\end{tikzpicture}		
	\end{center}
	Hàm số $y=\left[f(x)\right]^3-3\left[f(x)\right]^2$ đồng biến trên khoảng nào dưới đây?
	\choice
	{$\left(-\infty\,;1\right)$}
	{$\left(1\,;2\right)$}
	{\True $\left(3\,;4\right)$}
	{$\left(2\,;3\right)$}
	\loigiai{
		Ta có $y'=3f'(x)\left[f^2(x)-2f(x)\right]$. 
		Phương trình $y'=0\Leftrightarrow \hoac{
			&{f}'(x)=0\\ 
			& f(x)=0\\ 
			& f(x)=2.
		}$
		\begin{center}
			\begin{tikzpicture}
				\tkzTabInit[espcl=2.5,lgt=1.5]
				{$x$/0.7,$y'$/0.7,$y$/3.5}
				{$-\infty$,$1$,$2$,$3$,$4$,$+\infty$}
				\tkzTabLine{,+,0,-,0,+,0,-,0,+,}
				\node (0) at ($(N12)+(0,-3)$) {$-\infty$};
				\node (1) at ($(N22)+(0,-.3)$) {$3$};
				\node (2) at ($(N32)+(0,-1.7)$) {$1$};
				\node (3) at ($(N42)+(0,-0.8)$) {$2$};
				\node (4) at ($(N52)+(0,-2.3)$) {$0$};
				\node (5) at ($(N62)+(0,-.3)$) {$+\infty$};
				\node (a) at ($(N11)+(0.65,0.35)$) {$a$};
				\node (b) at ($(N11)+(2.0,0.4)$) {$b$};
				\node (c) at ($(N11)+(3.38,0.35)$) {$c$};
				\node (d) at ($(N11)+(11.85,0.4)$) {$d$};
				\node (6) at ($(N12)+(0,-0.8)$) {};
				\node (7) at ($(N62)+(0,-0.8)$) {};
				\node (8) at ($(N12)+(0,-2.3)$) {};
				\node (9) at ($(N62)+(0,-2.3)$) {};
				%				\node (8) at ($(N42)+(0,-.5)$) {};
				%				\coordinate (9) at ($(N42)!.6!(N53)+ (-0.5,0)$);
				\coordinate (A) at ($(0)!.25!(1)$);
				\coordinate (B) at ($(0)!.8!(1)$);
				\coordinate (C) at ($(1)!.35!(2)$);
				\coordinate (D) at ($(4)!.75!(5)$);
				%				\coordinate (7) at ($(T22)!.6!(T23)$);
				\draw[->] (0)--(1);
				\draw[->] (1)--(2);
				\draw[->] (2)--(3);
				\draw[->] (1)--(2);
				\draw[->] (3)--(4);
				\draw[->] (4)--(5);
				%				\draw[->,red] (5)--(8);
				%				\draw[->,red] (8)--(9);
				\draw[blue,dashed](6)--(7)node[below]{$y=2$} (a)--(A) (b)--(B) (c)--(C) (d)--(D);
				\draw[blue,dashed](8)--(9)node[below left]{$y=0$};
			\end{tikzpicture}		
		\end{center}
		Dựa vào bảng biến thiên, ta thấy $f'(x)=0\Leftrightarrow x\in \{ 1\,;2\,;3\,;4 \}$;\\
		$f(x)=0\Leftrightarrow x=a<1$ hoặc $x=4$;\\
		$f(x)=2\Leftrightarrow \hoac{
			& x=b\,\,\left(a<b<1\right)\\ 
			& x=c\in\left(1\,;2\right)\\ 
			& x=3\\ 
			& x=d>4.
		}$ \\
		Ta lập được bảng xét dấu của $y'$ 
		\begin{center}
			\begin{tikzpicture}
				\tkzTabInit[lgt=1.2,espcl=1.5,nocadre]
				{$x$/1, $f(x)$ /.8} % first column
				{$-\infty$,$a$, $b$, $1$,$c$, $2$,$3$, $4$, $d$, $+\infty$} % first row
				\tkzTabLine { ,+,z,-,z,+,z,-,z,+,z,-,z,+,z,-,z,+, } % second row
				%				\tkzTabLine {,-,z,+,t,+,} % third row
				%				\tkzTabLine {,+,d,-,z,+,} % last row
			\end{tikzpicture}
		\end{center}
		Từ bảng xét dấu, ta thấy hàm số đồng biến trên các khoảng \\
		$\left(-\infty;a\right)$, $\left(b;1\right)$, $\left(c;2\right)$, $\left(3;4\right)$ và $(d;+\infty)$.
	}
\end{ex}

\begin{ex}%[2D1G1-2]%Câu 29 
	[THPT Bùi Thị Xuân – Huế-2022] 
	\immini{
		Cho hàm số $y=f(x)$ là hàm đa thức bậc bốn. Đồ thị hàm số $f'(x+2)$ được cho trong hình vẽ bên. Hàm số 
		$$g(x)=4 f\left(x^2\right)-x^6+5 x^4-4 x^2+1$$
		đồng biến trên khoảng nào dưới đây?
		\choice
		{$(-4 ;-3)$}
		{\True $(2 ;+\infty)$}
		{$(-\sqrt{2};\sqrt{2})$}
		{$(-2 ;-1)$}}{
		\begin{tikzpicture}[scale=0.6,font=\footnotesize, line join=round, line cap=round, >=stealth] %Đường cong bậc 3
			\draw[thick, ->] (-5.3,0)--(5,0);
			\draw[thick, ->] (0,-3.5)--(0,7);
			\draw (5.2,0) node[below] {$x$};
			\draw (0,7.1) node[left]{$y$};
			\draw (0,0) node[below left]{$0$};
			\draw[fill] (-2,0) circle (0.5pt)node[below left]{$ -2 $};
			\draw[fill] (2,0) circle (0.5pt)node[below]{$ 2$};
			\draw[fill] (0,3) circle (0.5pt)node[left]{$ 3 $};
			\draw[fill] (0,1) circle (0.5pt)node[right]{$ 1 $};
			\draw[fill] (0,-1) circle (0.5pt)node[right]{$ -1 $};
			\draw[dashed] (-2,0)--(-2,1) --(0,1); 
			\draw[dashed](2,0)--(2,3)--(0,3);
			\draw[line width=1.2pt,smooth,samples=100,domain=-2.8:4.5] plot(\x,{-0.271*(\x)^3+0.75*(\x)^2+1.583*\x-1});
		\end{tikzpicture}		
	}
	\loigiai{
		$\begin{aligned}
			& g(x)=4f\left(x^2\right)-x^6+5x^4-4x^2+1\Rightarrow g' (x)=8xf'\left(x^2\right)-6x^5+20x^3-8x.\\ 
			& g' (x)=0\Leftrightarrow 8xf'\left(x^2\right)-6x^5+20x^3-8x=0 \\
			& \Leftrightarrow 2x\left[4f'\left(x^2\right)-3x^4+10x^2-4\right]=0\\ 
			&\Leftrightarrow 		\hoac{ 			& 2x=0\\ 
				& 4f'(x^2)-3x^4+10x^2-4=0
			}
			\Leftrightarrow \hoac{	& x=0\\ 
				& f'\left(x^2\right)=\dfrac{3}{4}{x^4}-\dfrac{5}{2}{x^2}+1.}
		\end{aligned}$\\ 
		Xét
		$f'\left(x^2\right)=\dfrac{3}{4}x^4-\dfrac{5}{2}x^2+1$. Đặt $x^2=t+2$, ta có\\
		$ f' (t+2)=\dfrac{3}{4}{(t+2)^2}-\dfrac{5}{2}(t+2)+1=\dfrac{3}{4}\left(t^2+4t+4\right)-\dfrac{5}{2}(t+2)-1=\dfrac{3}{4}{t^2}+\dfrac{1}{2}t-1$\\
		Khi đó số nghiệm của phương trình chính là số giao điểm của đồ thị hàm số $y=f' (t+2)$ và\\
		$ y=\dfrac{3}{4}{t^2}+\dfrac{1}{2}t-1$\\
		Ta có đồ thị 
		\begin{center}
			\begin{tikzpicture}[scale=0.6,font=\footnotesize, line join=round, line cap=round, >=stealth] %Đường cong bậc 3
				\draw[thick, ->] (-5.3,0)--(5,0);
				\draw[thick, ->] (0,-3.5)--(0,7);
				\draw (5.2,0) node[below] {$x$};
				\draw (0,7.1) node[left]{$y$};
				\draw (0,0) node[below left]{$0$};
				\draw[fill] (-2,0) circle (0.5pt)node[below left]{$ -2 $};
				\draw[fill] (2,0) circle (0.5pt)node[below]{$ 2$};
				\draw[fill] (0,3) circle (0.5pt)node[left]{$ 3 $};
				\draw[fill] (0,1) circle (0.5pt)node[right]{$ 1 $};
				\draw[fill] (0,-1) circle (0.5pt)node[right]{$ -1 $};
				\draw[dashed] (-2,0)--(-2,1) --(0,1); 
				\draw[dashed](2,0)--(2,3)--(0,3);
				\draw[line width=1.2pt,smooth,samples=100,domain=-2.8:4.5] plot(\x,{-0.271*(\x)^3+0.75*(\x)^2+1.583*\x-1});		
				\draw[line width=1.2pt,smooth,samples=100,domain=-3.3:2.8] plot(\x,{0.75*(\x)^2+0.5*\x-1});
			\end{tikzpicture}
		\end{center}
		Dựa vào đồ thị ta có $f' (t+2)=\dfrac{3}{4}t^2+\dfrac{1}{2}t-1\Leftrightarrow \hoac{& t=-2\\ & t=0\\ & t=2} \Leftrightarrow\hoac{& x+2=-2\\ & x+2=0\\ & x+2=2} \Leftrightarrow \hoac{& x=-4\\ & x=-2\\ & x=0.}$\\
		Ta có bảng xét dấu $g' (x)$ như sau
		\begin{center}
			\begin{tikzpicture}
				\tkzTabInit[lgt=1.2,espcl=2,nocadre]
				{$x$/0.7, $f(x)$ /.7}
				{$-\infty$, $-4$,$-2$, $0$, $+\infty$} % first row
				\tkzTabLine { ,-,z,+,z,-,z,+, }
			\end{tikzpicture}
		\end{center}
		Vậy hàm số $g(x)=4 f\left(x^2\right)-x^6+5 x^4-4 x^2+1$ đồng biến trên khoảng $(2 ;+\infty)$.}
\end{ex}

\begin{ex}%[2D1G1-2]%Câu 30
	[Chuyên Bắc Ninh 2022] 
	\immini{
		Cho hàm số $ y=f(x)$ liên tục trên $\mathbb{R}$ có đồ thị hàm số $ y=f'(x)$ có đồ thị như hình vẽ bên.
		Hàm số $g(x)=2f\left(\left| x-1\right|\right)-x^2+2x+2020$ đồng biến trên khoảng nào
		\choice
		{$\left(-2;0\right)$}
		{$\left(-3;1\right)$}
		{$\left(1\,;3\right)$}
		{\True $\left(0\,;\,1\right)$}}{
		\begin{tikzpicture}[scale=0.6,font=\footnotesize, line join=round, line cap=round, >=stealth] %Đường cong bậc 3
			\draw[thick, ->] (-3.3,0)--(5,0);
			\draw[thick, ->] (0,-3.0)--(0,5.5);
			\draw (5.2,0) node[below] {$x$};
			\draw (0,5.8) node[left]{$y$};
			\draw (0,0) node[below left]{$0$};
			\draw[fill] (-1,0) circle (0.5pt)node[above]{$ -1 $};
			\draw[fill] (1,0) circle (0.5pt)node[below]{$ 1$};
			\draw[fill] (0,1) circle (0.5pt)node[left]{$ 1 $};
			\draw[fill] (0,-1) circle (0.5pt)node[right]{$ -1 $};
			\draw[fill] (0,3) circle (0.5pt)node[left]{$ 3 $};
			\draw[fill] (3,0) circle (0.5pt)node[below]{$ 3 $};
			\draw[dashed] (-1,0)--(-1,-1) --(0,-1); 
			\draw[dashed](1,0)--(1,1)--(0,1);
			\draw[dashed](3,0)--(3,3)--(0,3);
			\draw[line width=1.2pt,smooth,samples=100,domain=-2.2:4.3] plot(\x,{-0.333*(\x)^3+1*(\x)^2+1.333*\x-1});		
			%\draw[line width=1.2pt,smooth,samples=100,domain=-3.3:2.8] plot(\x,{0.75*(\x)^2+0.5*\x-1});
		\end{tikzpicture}	
	}
	\loigiai{
		Ta có $g(x)=2f\left(\left| x-1\right|\right)-x^2+2x+2020\Leftrightarrow g(x)=2f\left(\left| x-1\right|\right)-\left(x-1\right)^2+2021$.\\
		Xét hàm số $ k\left(x-1\right)=2f\left(x-1\right)-\left(x-1\right)^2+2021$.\\
		Đặt $ t=x-1$\\
		Xét hàm số $ h(t)=2f(t)-t^2+2021$ $\Rightarrow{h}'(t)=2f'(t)-2t$.\\
		Kẻ đường $ y=x$ như hình vẽ.
		\begin{center}
			\begin{tikzpicture}[scale=0.6,font=\footnotesize, line join=round, line cap=round, >=stealth] %Đường cong bậc 3
				\draw[thick, ->] (-3.3,0)--(5,0);
				\draw[thick, ->] (0,-3.0)--(0,5.5);
				\draw (5.2,0) node[below] {$x$};
				\draw (0,5.8) node[left]{$y$};
				%	\draw (0,0) node[below left]{$0$};
				\draw[fill] (-1,0) circle (0.5pt)node[above]{$ -1 $};
				\draw[fill] (1,0) circle (0.5pt)node[below]{$ 1$};
				\draw[fill] (0,1) circle (0.5pt)node[left]{$ 1 $};
				\draw[fill] (0,-1) circle (0.5pt)node[right]{$ -1 $};
				\draw[fill] (0,3) circle (0.5pt)node[left]{$ 3 $};
				\draw[fill] (3,0) circle (0.5pt)node[below]{$ 3 $};
				\draw[dashed] (-1,0)--(-1,-1) --(0,-1); 
				\draw[dashed](1,0)--(1,1)--(0,1);
				\draw[dashed](3,0)--(3,3)--(0,3);
				\draw[line width=1.2pt,smooth,samples=100,domain=-2.2:4.3] plot(\x,{-0.333*(\x)^3+1*(\x)^2+1.333*\x-1});		
				%\draw[line width=1.2pt,smooth,samples=100,domain=-3.3:2.8] plot(\x,{0.75*(\x)^2+0.5*\x-1});
				\draw[line width=1.2pt,smooth,samples=100](-2,-2)--(4,4);
			\end{tikzpicture}
		\end{center}
		Khi đó $h'(t)>0\Leftrightarrow{f}'(t)-t>0\Leftrightarrow{f}'(t)>t$$\Leftrightarrow \hoac{
			& t<-1\\ 
			& 1<t<3.
		}$\\
		Do đó $k'\left(x-1\right)>0\Leftrightarrow \hoac{
			& x-1<-1\\ 
			& 1<x-1<3} \Leftrightarrow \hoac{
			& x<0\\ 
			& 2<x<4.}$\\
		Ta có bảng biến thiên của hàm số $ k\left(x-1\right)=2f\left(x-1\right)-\left(x-1\right)^2+2021$.
		\begin{center}
			\begin{tikzpicture}
				\tkzTabInit[lgt=1.8,espcl=2.3]
				{$x$ /1.2, $k'(x-1)$ /1.2,$k(x-1)$ /2}
				{$-\infty$ , $0$,$2$,$4$, $+\infty$}
				\tkzTabLine{,+,0,-,0,+,0,-,}
				\tkzTabVar{-/$ $ ,+/$ $, -/$ $,+/$ $,-/$ $}
			\end{tikzpicture}
		\end{center}
		Khi đó, ta có bảng biến thiên của $g(x)=2f\left(\left| x-1\right|\right)-\left(x-1\right)^2+2021$ bằng cách lấy đối xứng qua đường thẳng $ x=1$ như sau\\
		\begin{center}
			\begin{tikzpicture}
				\tkzTabInit[lgt=1.2,espcl=2.5,nocadre]
				{$x$ /0.7, $g'(x)$ /0.7,$g(x)$ /2.5}
				{$-\infty$ ,$-2$, $0$,$1$,$2$,$4$, $+\infty$}
				\tkzTabLine{,+,0,-,0,+,0,-,0,+,0,-,}
				\tkzTabVar{-/$ $ ,+/$ $, -/$ $,+/$ $,-/$ $,+/ $ $,-/$ $}
			\end{tikzpicture}
		\end{center}
		Vậy hàm số đồng biến trên $\left(0;1\right)$.}
\end{ex}

\begin{ex}%[2D1G1-2]%Câu 31
	[Chuyên Thái Bình 2022] 
	\immini{
		Cho hàm số $f(x)=a{x^4}+b{x^3}+c{x^2}+dx+a$ có đồ thị hàm số $y=f'(x)$ như hình vẽ bên. Hàm số $y=g(x)=f\left(1-2x\right)f\left(2-x\right)$ đồng biến trên khoảng nào dưới đây?
		\choice
		{$\left(\dfrac{1}{2};\dfrac{3}{2}\right)$}
		{$\left(-\infty ;0\right)$}
		{$\left(0;2\right)$}
		{\True $\left(3;+\infty\right)$}}{
		\begin{tikzpicture}[scale=0.9,font=\footnotesize, line join=round, line cap=round, >=stealth] %Đường cong bậc 3
			\draw[thick, ->] (-2.5,0)--(2.5,0);
			\draw[thick, ->] (0,-2.8)--(0,2.8);
			\draw (2.6,0) node[below] {$x$};
			\draw (0,2.9) node[left]{$y$};
			\draw (0,0) node[below left]{$0$};
			\draw[fill] (-1,0) circle (0.5pt)node[below left]{$ -1 $};
			\draw[fill] (1,0) circle (0.5pt)node[below right]{$ 1$};
			%			\draw[dashed] (-1,0)--(-1,-1) --(0,-1); 
			%			\draw[dashed](1,0)--(1,1)--(0,1);
			%			\draw[dashed](3,0)--(3,3)--(0,3);
			\draw[line width=1.2pt,smooth,samples=100,domain=-1.3:1.3] plot(\x,{3*(\x)^3-3*\x});		
			%\draw[line width=1.2pt,smooth,samples=100,domain=-3.3:2.8] plot(\x,{0.75*(\x)^2+0.5*\x-1});
		\end{tikzpicture}	
	}
	\loigiai{
		Ta có $f'(x)=4a{x^3}+3b{x^2}+2cx+d$, theo đồ thị thì đa thức $f'(x)$ có ba nghiệm phân biệt là $-1,0,1$ nên $f'(x)=4ax\left(x+1\right)\left(x-1\right)=4a{x^3}-4ax\Rightarrow f(x)=a{x^4}-2a{x^2}+a=a{\left(x^2-1\right)^2}$.\\
		Dựa vào đồ thị hàm số $y=f'(x)$ ta có $a>0$ nên $f(x)>0,\forall x\in\mathbb{R}\setminus\left\{\pm 1\right\}$.\\
		$g'(x)=\left[f\left(1-2x\right)\right]'f\left(2-x\right)+f\left(1-2x\right)\left[f\left(2-x\right)\right]'=-2f'\left(1-2x\right)f\left(2-x\right)-f\left(1-2x\right)f'\left(2-x\right)$. Xét $x\in\left(\dfrac{1}{2};\dfrac{3}{2}\right)\Rightarrow
		\heva{		
			& 1-2x\in\left(-2;0\right)\\ 
			& 2-x\in\left(\dfrac{1}{2};\dfrac{3}{2}\right)}$, dấu của $f'(x)$ không cố định trên $\left(\dfrac{1}{2};\dfrac{3}{2}\right)$ nên ta không kết luận được tính đơn điệu của hàm số $g(x)$ trên $\left(\dfrac{1}{2};\dfrac{3}{2}\right)$.\\
		Xét $x\in\left(-\infty ;0\right)\Rightarrow
		\heva{
			& 1-2x\in\left(1;+\infty\right)\\ 
			& 2-x\in\left(2;+\infty\right)} 
		\Rightarrow \heva{
			& f'\left(1-2x\right)>0\\ 
			& f'\left(2-x\right)>0} \Rightarrow g'(x)<0$.\\
		Do đó, hàm số $g(x)$ nghịch biến trên $\left(-\infty ;0\right)$.\\
		$x\in\left(0;2\right)\Rightarrow \heva{
			& 1-2x\in\left(-3;1\right)\\ 
			& 2-x\in\left(0;2\right)}$, dấu của $f'(x)$ không cố định trên $\left(-3;1\right)$ và $\left(0;2\right)$ nên ta không kết luận được tính đơn điệu của hàm số $g(x)$ trên $\left(\dfrac{1}{2};\dfrac{3}{2}\right)$.\\
		Xét $x\in\left(3;+\infty\right)\Rightarrow \heva{
			& 1-2x\in\left(-\infty ;-5\right)\\ 
			& 2-x\in\left(-\infty ;-1\right)} \Rightarrow \heva{
			& f'\left(1-2x\right)<0\\ 
			& f'\left(2-x\right)<0} \Rightarrow g'(x)>0$. \\
		Do đó, hàm số $g(x)$ đồng biến trên $\left(3;+\infty\right)$.}
\end{ex}

\begin{dang}{Bài toán hàm ẩn, hàm hợp liên quan đến tham số và một số bài toán khác}
\end{dang}

\begin{ex}%[2D1G1-3]%Câu 1
	[Chuyên Lê Hồng Phong Nam Định 2019]
	\immini{
		Cho hàm số $ y=f(x)$ có đạo hàm liên tục trên $\mathbb{R}$. Biết hàm số $ y=f'(x)$ có đồ thị như hình vẽ. Gọi $ S$ là tập hợp các giá trị nguyên $ m\in\left[-5\,;\,\text{5}\right]$ để hàm số $ g(x)=f\left(x+m\right)$ nghịch biến trên khoảng $\left(1\,;\,2\right)$. Hỏi $S$ có bao nhiêu phần tử?
		\choice
		{$ 4$}
		{$ 3$}
		{$ 6$}
		{\True $ 5$}}{
		\begin{tikzpicture}[scale=0.9,font=\footnotesize, line join=round, line cap=round, >=stealth] %Đường cong bậc 3
			\draw[thick, ->] (-2.5,0)--(4,0);
			\draw[thick, ->] (0,-2.8)--(0,2.8);
			\draw (4.3,0) node[below] {$x$};
			\draw (0,2.9) node[left]{$y$};
			\draw (0,0) node[below left]{$0$};
			\draw[fill] (-1,0) circle (0.5pt)node[below left]{$ -1 $};
			\draw[fill] (1,0) circle (0.5pt)node[below]{$ 1$};
			\draw[fill] (3,0) circle (0.5pt)node[below right]{$ 3$};
			%			\draw[dashed] (-1,0)--(-1,-1) --(0,-1); 
			%			\draw[dashed](1,0)--(1,1)--(0,1);
			%			\draw[dashed](3,0)--(3,3)--(0,3);
			\draw[line width=1.2pt,smooth,samples=100,domain=-1.65:3.5] plot(\x,{0.33*(\x)^3-(\x)^2-0.333*(\x)+1});		
			%\draw[line width=1.2pt,smooth,samples=100,domain=-3.3:2.8] plot(\x,{0.75*(\x)^2+0.5*\x-1});
		\end{tikzpicture}	
	}
	\loigiai{
		Ta có $g'(x)=f'\left(x+m\right)$. Vì $ y=f'(x)$ liên tục trên $\mathbb{R}$ nên $g'(x)=f'\left(x+m\right)$ cũng liên tục trên $\mathbb{R}$. Căn cứ vào đồ thị hàm số $ y=f'(x)$ ta thấy\\
		$g'(x)<0\Leftrightarrow{f}'\left(x+m\right)<0$ $\Leftrightarrow\hoac{
			& x+m<-1\\ 
			& 1<x+m<3} \Leftrightarrow \hoac{
			& x<-1-m\\ 
			& 1-m<x<3-m.}$\\
		Hàm số $ g(x)=f\left(x+m\right)$ nghịch biến trên khoảng $\left(1\,;\,2\right)$
		$\Leftrightarrow \hoac{
			& 2\le-1-m\\ 
			&\hoac{
				& 3-m\ge 2\\ 
				& 1-m\le 1}} \Leftrightarrow \hoac{
			& m\le-3\\ 
			& 0\le m\le 1.}$\\
		Mà $ m$ là số nguyên thuộc đoạn $\left[-5\,;\,5\right]$ nên ta có $ S=\left\{-5;-4;-3;0;1\right\}$.\\
		Vậy $ S$ có $5$ phần tử.}
\end{ex}

\begin{ex}%[2D1G1-3]%Câu 2
	[Chuyên Nguyễn Bỉnh Khiêm-Quảng Nam-2020] Cho hàm số $ y=f(x)$ có đạo hàm trên $\mathbb{R}$ và bảng xét dấu đạo hàm như hình vẽ sau
	\begin{center}
		\begin{tikzpicture}
			\tkzTabInit[lgt=1.2,espcl=2.5,nocadre]
			{$x$/0.7, $f'(x)$ /2.5} % first column
			{$-\infty$, $-10$,$-2$, $3$,$8$, $+\infty$} % first row
			\tkzTabLine { ,+,z,-,z,+,z,-,z,+, } % second row
			%				\tkzTabLine {,-,z,+,t,+,} % third row
			%				\tkzTabLine {,+,d,-,z,+,} % last row
		\end{tikzpicture}
	\end{center}
	Có bao nhiêu số nguyên $ m$ để hàm số $ y=f\left(x^3+4x+m\right)$ nghịch biến trên khoảng $\left(-1;1\right)$?
	\choice
	{$ 3$}
	{$ 0$}
	{\True $ 1$}
	{$ 2$}
	\loigiai
	{
		Đặt $ t=x^3+4x+m\Rightarrow{t}'=3x^2+4$ nên $ t$ đồng biến trên $\left(-1;1\right)$ và $ t\in\left(m-5;m+5\right)$.\\
		Yêu cầu bài toán trở thành tìm $ m$ để hàm số $ f(t)$ nghịch biến trên khoảng $\left(m-5;m+5\right)$.\\
		Dựa vào bảng biến thiên ta được $\heva{
			& m-5\ge-2\\ 
			& m+5\le 8} \Leftrightarrow \heva{
			& m\ge 3\\ 
			& m\le 3} \Leftrightarrow m=3$.}
\end{ex}

\begin{ex}%[2D1G1-3]%Câu 3
	[Chuyên ĐH Vinh-Nghệ An-2020]
	\immini{
		Cho hàm số $ f(x)$ có đạo hàm trên $\mathbb{R}$và $ f(1)=1$. Đồ thị hàm số $ y=f'(x)$ như hình bên. Có bao nhiêu số nguyên dương $ a$ để hàm số $ y=\left| 4f\left(\sin x\right)+\cos 2x-a\right|$ nghịch biến trên $\left(0;\dfrac{\pi}{2}\right)$?
		\choice
		{$ 2$}
		{\True $ 3$}
		{Vô số}
		{$ 5$}}{
		\begin{tikzpicture}[scale=0.9,font=\footnotesize, line join=round, line cap=round, >=stealth] %Đường cong bậc 3
			\draw[thick, ->] (-2.5,0)--(3,0);
			\draw[thick, ->] (0,-2.8)--(0,2.8);
			\draw (3.1,0) node[below] {$x$};
			\draw (0,2.9) node[left]{$y$};
			\draw (0,0) node[below left]{$0$};
			\draw[fill] (-1,0) circle (0.5pt)node[below]{$ -1 $};
			\draw[fill] (1,0) circle (0.5pt)node[above]{$ 1$};
			%	\draw[fill] (3,0) circle (0.5pt)node[below right]{$ 3$};
			\draw[dashed] (-1,0)--(-1,1); 
			\draw[dashed](1,0)--(1,-1);
			%			\draw[dashed](3,0)--(3,3)--(0,3);
			\draw[line width=1.2pt,smooth,samples=100,domain=-2:2] plot(\x,{.8*(\x)^3+0*(\x)^2-1.8*(\x)});		
			%\draw[line width=1.2pt,smooth,samples=100,domain=-3.3:2.8] plot(\x,{0.75*(\x)^2+0.5*\x-1});
			\draw (2.0,2.8) node[left]{$y=f'(x)$};
		\end{tikzpicture}	
	}
	\loigiai
	{		Đặt $g(x)=\left| 4f\left(\sin x\right)+\cos 2x-a\right|\Rightarrow g(x)=\sqrt{\left[4f\left(\sin x\right)+\cos 2x-a\right]^2}$ .\\
		$\Rightarrow{g}'(x)=\dfrac{\left[4\cos x\cdot f'\left(\sin x\right)-2\sin 2x\right]\left[4f\left(\sin x\right)+\cos 2x-a\right]}{\sqrt{\left[4f\left(\sin x\right)+\cos 2x-a\right]^2}}$.\\
		Ta có $ 4\cos x\cdot f'\left(\sin x\right)-2\sin 2x=4\cos x\left[f'\left(\sin x\right)-\sin x\right]$.\\
		Với $ x\in\left(0;\dfrac{\pi}{2}\right)$ thì $\cos x>0,\sin x\in\left(0;1\right)\Rightarrow{f}'\left(\sin x\right)-\sin x<0$.\\
		Hàm số $ g(x)$ nghịch biến trên $\left(0;\dfrac{\pi}{2}\right)$ khi $ 4f\left(\sin x\right)+\cos 2x-a\ge 0,\forall x\in\left(0;\dfrac{\pi}{2}\right)$\\
		$\Leftrightarrow 4f\left(\sin x\right)+1-2\sin^2x\ge a,\forall x\in\left(0;\dfrac{\pi}{2}\right)$.\\
		Đặt $ t=\sin x$ được $ 4f(t)+1-2t^2\ge a,\forall t\in\left(0;1\right)$ (*).\\
		Xét $ h(t)=4f(t)+1-2t^2\Rightarrow{h}'(t)=4f'(t)-4t=4\left[f'(t)-1\right]$.\\
		Với $ t\in\left(0;1\right)$ thì $h'(t)<0\Rightarrow h(t)$ nghịch biến trên $\left(0;1\right)$.\\
		Do đó (*) $\Leftrightarrow a\le h(1)=4f(1)+1-2.1^2=3$.\\
		Vậy có $3$ giá trị nguyên dương của a thỏa mãn.}
\end{ex}


\begin{ex}%[2D1G1-3]%Câu 4
	[Chuyên Quang Trung-2020]
	\immini{
		Cho hàm số $ y=f(x)$ có đạo hàm liên tục trên $\mathbb{R}$ và có đồ thị $ y=f'(x)$ như hình vẽ. Đặt $ g(x)=f\left(x-m\right)-\dfrac{1}{2}{\left(x-m-1\right)^2}+2019$, với $ m$ là tham số thực. Gọi $ S$ là tập hợp các giá trị nguyên dương của $ m$ để hàm số $ y=g(x)$ đồng biến trên khoảng $\left(5;6\right)$. Tổng tất cả các phần tử trong $ S$ bằng
		\choice
		{$ 4$}
		{$ 11$}
		{\True $ 14$}
		{$ 20$}}{
		\begin{tikzpicture}[scale=0.9,font=\footnotesize, line join=round, line cap=round, >=stealth] %Đường cong bậc 3
			\draw[style=help lines,step=1] (-2.5,-3) grid (3,3.5);
			\draw[thick, ->] (-2.5,0)--(3.5,0);
			\draw[thick, ->] (0,-2.8)--(0,2.8);
			\draw (3.6,0) node[below] {$x$};
			\draw (0,3) node[above left]{$y$};
			\draw (0,0) node[below left]{$0$};
			%\draw[fill] (-1,0) circle (0.5pt)node[below]{$ -1 $};
			\draw[fill] (1,0) circle (0.5pt)node[below left]{$ 1$};
			%	\draw[fill] (3,0) circle (0.5pt)node[below right]{$ 3$};
			\draw[dashed] (-1,0)--(-1,-2) --(2,-2)--(2,0); 
			\draw[dashed](3,0)--(3,2) --(0,2);
			\draw (-1,-2) circle (2pt);
			\draw (3,2) circle (2pt);
			%			\draw[dashed](3,0)--(3,3)--(0,3);
			\draw[line width=1.2pt,smooth,samples=100,domain=-1.1:3.1] plot(\x,{1*(\x)^3-3*(\x)^2-0*(\x)+2});		
			%\draw[line width=1.2pt,smooth,samples=100,domain=-3.3:2.8] plot(\x,{0.75*(\x)^2+0.5*\x-1});
			%\draw (2.0,2.8) node[left]{$y=f'(x)$};
		\end{tikzpicture}	
	}
	\loigiai
	{
		Xét hàm số $ g(x)=f\left(x-m\right)-\dfrac{1}{2}{\left(x-m-1\right)^2}+2019$.\\
		$g'(x)=f'\left(x-m\right)-\left(x-m-1\right)$.\\
		Xét phương trình $g'(x)=0. \quad \quad (1)$\\
		Đặt $ x-m=t$, phương trình $(1)$ trở thành $f'(t)-\left(t-1\right)=0\Leftrightarrow{f}'(t)=t-1. \quad (2)$\\
		Nghiệm của phương trình $(2)$ là hoành độ giao điểm của hai đồ thị hàm số $ y=f'(t)$ và $ y=t-1$.\\
		Ta có đồ thị các hàm số $ y=f'(t)$ và $ y=t-1$ như sau
		\begin{center}
			\begin{tikzpicture}[scale=0.9,font=\footnotesize, line join=round, line cap=round, >=stealth] %Đường cong bậc 3
				\draw[style=help lines,step=1] (-2.5,-3) grid (3,3.5);
				\draw[thick, ->] (-2.5,0)--(3.5,0);
				\draw[thick, ->] (0,-2.8)--(0,2.8);
				\draw (3.6,0) node[below] {$x$};
				\draw (0,3) node[above left]{$y$};
				\draw (0,0) node[below left]{$0$};
				%\draw[fill] (-1,0) circle (0.5pt)node[below]{$ -1 $};
				\draw[fill] (1,0) circle (0.5pt)node[below left]{$ 1$};
				%	\draw[fill] (3,0) circle (0.5pt)node[below right]{$ 3$};
				\draw[dashed] (-1,0)--(-1,-2) --(2,-2)--(2,0); 
				\draw[dashed](3,0)--(3,2) --(0,2);
				\draw (-1,-2) circle (2pt);
				\draw (3,2) circle (2pt);
				%			\draw[dashed](3,0)--(3,3)--(0,3);
				\draw[line width=1.2pt,smooth,samples=100,domain=-1.1:3.1] plot(\x,{1*(\x)^3-3*(\x)^2-0*(\x)+2});		
				%\draw[line width=1.2pt,smooth,samples=100,domain=-3.3:2.8] plot(\x,{0.75*(\x)^2+0.5*\x-1});
				%\draw (2.0,2.8) node[left]{$y=f'(x)$};
				\draw (-2,-3)--(4,3);
			\end{tikzpicture}
		\end{center}
		Căn cứ đồ thị các hàm số ta có phương trình $(2)$ có nghiệm là $\hoac{
			& t=-1\\ 
			& t=1\\ 
			& t=3} \Rightarrow \hoac{
			& x=m-1\\ 
			& x=m+1\\ 
			& x=m+3.}$\\
		Ta có bảng biến thiên của $ y=g(x)$
		\begin{center}
			\begin{tikzpicture}
				\tkzTabInit[lgt=1,espcl=2.5,nocadre]
				{$x$ /0.8, $y'$ /0.8,$y$ /2.5}
				{$-\infty$ , $m-1$,$m+1$,$m+3$, $+\infty$}
				\tkzTabLine{,+,0,-,0,+,0,-,}
				\tkzTabVar{-/$ +\infty$ ,+/$ $, -/$ $,+/$ $,-/$+\infty $}
			\end{tikzpicture}
		\end{center}
		Để hàm số $ y=g(x)$ đồng biến trên khoảng $\left(5;6\right)$ cần $\hoac{
			&\heva{
				& m-1\le 5\\ 
				& m+1\ge 6}\\ 
			& m+3\le 5}\Leftrightarrow\hoac{
			& 5\le m\le 6\\ 
			& m\le 2.}$\\
		Vì $ m\in\mathbb{N}^*\Rightarrow m$ nhận các giá trị $ 1;\,2;\,5;\,6\Rightarrow S=14$.}
\end{ex}

\begin{ex}%[2D1G1-3]%Câu 5
	[Sở Hà Nội-Lần 2-2020] 
	\immini{
		Cho hàm số $y=a{x^4}+b{x^3}+c{x^2}+dx+e,\,\,a\ne 0$. Hàm số $y=f'(x)$ có đồ thị như hình vẽ bên. 
		Gọi S là tập hợp tất cả các giá trị nguyên thuộc khoảng $\left(-6;6\right)$ của tham số $m$ để hàm số $g(x)=f\left(3-2x+m\right)+x^2-\left(m+3\right)x+2m^2$ nghịch biến trên $\left(0;1\right)$. Khi đó, tổng giá trị các phần tử của S là
		\choice
		{$12$}
		{\True $9$}
		{$6$}
		{$15$}}{
		\begin{tikzpicture}[scale=0.7,font=\footnotesize, line join=round, line cap=round, >=stealth] %Đường cong bậc 3
			%	\draw[style=help lines,step=1] (-2.5,-3) grid (3,3.5);
			\draw[thick, ->] (-4.5,0)--(6.5,0);
			\draw[thick, ->] (0,-2.8)--(0,2.8);
			\draw (6.6,0) node[below] {$x$};
			\draw (0,3) node[above left]{$y$};
			\draw (0,0) node[below left]{$0$};
			\draw[fill] (-2,0) circle (0.5pt)node[below]{$ -2 $};
			\draw[fill] (4,0) circle (0.5pt)node[above]{$ 4$};
			\draw[fill] (0,1) circle (0.5pt)node[right]{$ 1 $};
			\draw[fill] (0,-2) circle (0.5pt)node[left]{$ -2$};
			%	\draw[fill] (3,0) circle (0.5pt)node[below right]{$ 3$};
			\draw[dashed] (-2,0)--(-2,1) --(0,1); 
			\draw[dashed](4,0)--(4,-2) --(0,-2);
			%			\draw[dashed](3,0)--(3,3)--(0,3);
			\draw[line width=1.2pt,smooth,samples=100,domain=-3.8:5.5] plot(\x,{0.0714*(\x)^3-0.1423*(\x)^2-1.0714*(\x)});		
			%\draw[line width=1.2pt,smooth,samples=100,domain=-3.3:2.8] plot(\x,{0.75*(\x)^2+0.5*\x-1});
			%\draw (2.0,2.8) node[left]{$y=f'(x)$};
		\end{tikzpicture}	
	}
	\loigiai
	{
		Xét $g'(x)=-2f'\left(3-2x+m\right)+2x-\left(m+3\right)$.\\
		Xét phương trình $g'(x)=0$, đặt $t=3-2x+m$ thì phương trình trở thành\\ $-2\cdot \left[f'(t)-\dfrac{-t}{2}\right]=0\Leftrightarrow\hoac{
			& t=-2\\ 
			& t=4\\ 
			& t=0.}$ \\
		Từ đó, $g'(x)=0\Leftrightarrow{x_1}=\dfrac{5+m}{2},\,x_2=\dfrac{m+3}{2},x_3=\dfrac{-1+m}{2}$.\\
		Lập bảng xét dấu, đồng thời lưu ý nếu $x>x_1$ thì $t<t_1$ nên $f(x)>0$. Và các dấu đan xen nhau do các nghiệm đều làm đổi dấu đạo hàm nên suy ra $g'(x)\le 0\Leftrightarrow x\in\left[x_2;{x_1}\right]\cup\left(-\infty ;{x_3}\right]$.\\
		Vì hàm số nghịch biến trên $\left(0;1\right)$ nên \\
		$g'(x)\le 0,\,\forall x\in\left(0;1\right)$ từ đó suy ra $\hoac{
			&\dfrac{3+m}{2}\le 0<1\le\dfrac{5+m}{2}\\ 
			& 1\le\dfrac{-1+m}{2}.}$ \\
		và giải ra các giá trị nguyên thuộc $\left(-6;6\right)$ của $m$ là $-3$; $3$; $4$; $5$. }
\end{ex}

\begin{ex}%[2D1G1-3]%Câu 6
	[Chuyên Quang Trung-Bình Phước-Lần 2-2020]
	\immini{
		Cho hàm số $ y=f(x)$ có đạo hàm liên tục trên $\mathbb{R}$ và có đồ thị $ y=f'(x)$ như hình vẽ bên. Đặt $ g(x)=f\left(x-m\right)-\dfrac{1}{2}{\left(x-m-1\right)^2}+2019$, với $ m$ là tham số thực. Gọi $ S$ là tập hợp các giá trị nguyên dương của $ m$ để hàm số $ y=g(x)$ đồng biến trên khoảng $\left(5;6\right)$. Tổng tất cả các phần tử trong $ S$ bằng
		\choice
		{$ 4$}
		{$ 11$}
		{\True $ 14$}
		{$ 20$}}{
		\begin{tikzpicture}[scale=0.9,font=\footnotesize, line join=round, line cap=round, >=stealth] %Đường cong bậc 3
			\draw[thick, ->] (-2.5,0)--(3.7,0);
			\draw[thick, ->] (0,-2.8)--(0,2.8);
			\draw (3.9,0) node[below] {$x$};
			\draw (0,2.9) node[left]{$y$};
			\draw (0,0) node[below left]{$0$};
			\draw[fill] (-1,0) circle (0.5pt)node[above]{$ -1 $};
			\draw[fill] (1,0) circle (0.5pt)node[below]{$ 1$};
			\draw[fill] (3,0) circle (0.5pt)node[below]{$ 3$};
			\draw[fill] (2,0) circle (0.5pt)node[above]{$ 2$};
			\draw[fill] (0,2) circle (0.5pt)node[above left]{$ 2$};
			\draw[fill] (0,-2) circle (0.5pt)node[below left]{$ -2$};
			\draw[dashed] (-1,0)--(-1,-2)--(2,-2)--(2,0); 
			\draw[dashed](3,0)--(3,2)--(0,2);
			%			\draw[dashed](3,0)--(3,3)--(0,3);
			\draw[line width=1.2pt,smooth,samples=100,domain=-1.1:3.1] plot(\x,{1*(\x)^3-3*(\x)^2-0*(\x)+2});		
			%\draw[line width=1.2pt,smooth,samples=100,domain=-3.3:2.8] plot(\x,{0.75*(\x)^2+0.5*\x-1});
			%	\draw (2.0,2.8) node[left]{$y=f'(x)$};
	\end{tikzpicture}	}
	\loigiai
	{
		Ta có $g'(x)=f'\left(x-m\right)-\left(x-m-1\right)$.\\
		Cho $g'(x)=0\Leftrightarrow{f}'\left(x-m\right)=x-m-1$.\\
		Đặt $ x-m=t\Rightarrow f'(t)=t-1$\\
		Khi đó nghiệm của phương trình là hoành độ giao điểm của đồ thị hàm số $ y=f'(t)$ và và đường thẳng $ y=t-1$.
		\begin{center}
			\begin{tikzpicture}[scale=0.9,font=\footnotesize, line join=round, line cap=round, >=stealth] %Đường cong bậc 3
				\draw[thick, ->] (-2.5,0)--(3.7,0);
				\draw[thick, ->] (0,-2.8)--(0,2.8);
				\draw (3.9,0) node[below] {$x$};
				\draw (0,2.9) node[left]{$y$};
				\draw (0,0) node[below left]{$0$};
				\draw[fill] (-1,0) circle (0.5pt)node[above]{$ -1 $};
				\draw[fill] (1,0) circle (0.5pt)node[below]{$ 1$};
				\draw[fill] (3,0) circle (0.5pt)node[below]{$ 3$};
				\draw[fill] (2,0) circle (0.5pt)node[above]{$ 2$};
				\draw[fill] (0,2) circle (0.5pt)node[above left]{$ 2$};
				\draw[fill] (0,-2) circle (0.5pt)node[below left]{$ -2$};
				\draw[dashed] (-1,0)--(-1,-2)--(2,-2)--(2,0); 
				\draw[dashed](3,0)--(3,2)--(0,2);
				%			\draw[dashed](3,0)--(3,3)--(0,3);
				\draw[line width=1.2pt,smooth,samples=100,domain=-1.1:3.1] plot(\x,{1*(\x)^3-3*(\x)^2-0*(\x)+2});		
				%\draw[line width=1.2pt,smooth,samples=100,domain=-3.3:2.8] plot(\x,{0.75*(\x)^2+0.5*\x-1});
				%	\draw (2.0,2.8) node[left]{$y=f'(x)$};
				\coordinate (a) at ($(-1,-2)!1.2!(3,2)$);
				\coordinate (b) at ($(-1,-2)!-.2!(3,2)$);
				\draw[line width=1.2pt,smooth] (a)--(b);
			\end{tikzpicture}
		\end{center}
		Dựa vào đồ thị hàm số ta có được $f'(t)=t-1\Leftrightarrow\hoac{
			& t=-1\\ 
			& t=1\\ 
			& t=3.} $ \\
		Bảng xét dấu của $g'(t)$
		\begin{center}
			\begin{tikzpicture}
				\tkzTabInit[lgt=1.2,espcl=2.5,nocadre]
				{$t$/1, $g'(x)$ /.8} % first column
				{$-\infty$, $-1$,$1$, $3$, $+\infty$} % first row
				\tkzTabLine { ,-,0,+,0,-,0,+, } % second row
				%				\tkzTabLine {,-,z,+,t,+,} % third row
				%				\tkzTabLine {,+,d,-,z,+,} % last row
			\end{tikzpicture}
		\end{center}
		Từ bảng xét dấu ta thấy hàm số $ g(t)$ đồng biến trên khoảng $\left(-1;1\right)$ và $\left(3;+\infty\right)$.\\
		Hay $\hoac{
			&-1<t<1\\ 
			& t>3}\Leftrightarrow\hoac{
			&-1<x-m<1\\ 
			& x-m>3} \Leftrightarrow\hoac{
			& m-1<x<m+1\\ 
			& x>m+3.}$\\
		Để hàm số $ g(x)$ đồng biến trên khoảng $\left(5;6\right)$ thì $\hoac{
			& m-1\le 5<6\le m+1\\ 
			& m+3\le 5<6} \Leftrightarrow\hoac{
			& 5\le m\le 6\\ 
			& m\le 2.}$\\
		Vì $ m$ là các số nguyên dương nên $ S=\left\{ 1;2;5;6\right\}$.\\
		Vậy tổng tất cả các phần tử của $ S$ là $ 1+2+5+6=14$.}
\end{ex}

\begin{ex}%[2D1G1-3]%Câu 7
	\immini{
		Cho hàm số $ y=f(x)$ liên tục có đạo hàm trên $\mathbb{R}$. Biết hàm số $ f'(x)$ có đồ thị cho như hình vẽ bên. Có bao nhiêu giá trị nguyên của $ m$ thuộc $\left[-2019;2019\right]$ để hàm só $ g(x)=f\left(2019^x\right)-mx+2$ đồng biến trên $\left[0;1\right]$.
		\choice
		{$ 2028$}
		{$ 2019$}
		{$ 2011$}
		{\True $ 2020$}}{
		\begin{tikzpicture}[scale=0.9,font=\footnotesize, line join=round, line cap=round, >=stealth] %Đường cong bậc 3
			\draw[thick, ->] (-3.5,0)--(2.5,0);
			\draw[thick, ->] (0,-2.8)--(0,2.8);
			\draw (2.7,0) node[below] {$x$};
			\draw (0,2.9) node[left]{$y$};
			\draw (0,0) node[below left]{$0$};
			%	\draw[fill] (-1,0) circle (0.5pt)node[above]{$ -1 $};
			\draw[fill] (1,0) circle (0.5pt)node[below right]{$ 1$};
			%		\draw[fill] (3,0) circle (0.5pt)node[below]{$ 3$};
			%		\draw[fill] (2,0) circle (0.5pt)node[above]{$ 2$};
			%		\draw[fill] (0,2) circle (0.5pt)node[above left]{$ 2$};
			%		\draw[fill] (0,-2) circle (0.5pt)node[below left]{$ -2$};
			%		\draw[dashed] (-1,0)--(-1,-2)--(2,-2)--(2,0); 
			%		\draw[dashed](3,0)--(3,2)--(0,2);
			\draw[line width=1.2pt,smooth,samples=100,domain=-3.28:1.32] plot(\x,{0.667*(\x)^3+2*(\x)^2-0.667*(\x)-2});		
			%\draw[line width=1.2pt,smooth,samples=100,domain=-3.3:2.8] plot(\x,{0.75*(\x)^2+0.5*\x-1});
			%	\draw (2.0,2.8) node[left]{$y=f'(x)$};
	\end{tikzpicture}	}
	\loigiai{
		Ta có $ g'(x)=2019^x\ln 2019\cdot f'\left(2019^x\right)-m$.\\
		Ta lại có hàm số $ y=2019^x$ đồng biến trên $\left[0;1\right]$.\\
		Với $ x\in\left[0;1\right]$ thì $2019^x\in\left[1;2019\right]$ mà hàm $ y=f'(x)$ đồng biến trên $\left(1;+\infty\right)$ nên hàm $ y=f'\left(2019^x\right)$ đồng biến trên $\left[0;1\right]$.\\
		Mà $2019^x\ge 1;f'\left(2019^x\right)>0\,\forall\,x\in\left[0;1\right]$ nên hàm $ h(x)=2019^x\ln 2019\cdot f'\left(2019^x\right)$ đồng biến trên $\left[0;1\right]$.\\
		Hay $ h(x)\ge h(0)=0,\forall\,x\in\left[0;1\right]$.\\
		Do vậy hàm số $ g(x)$ đồng biến trên đoạn $\left[0;1\right]$$\Leftrightarrow g'(x)\ge 0,\forall\,x\in\left[0;1\right]$\\
		$\Leftrightarrow m\le{2019^x}\ln 2019.f'\left(2019^x\right),\forall\,x\in\left[0;1\right]$ $\Leftrightarrow m\le\underset{x\in\left[0;1\right]}{\min}\,h(x)=h(0)=0$\\
		Vì $ m$ nguyên và $ m\in\left[-2019;2019\right]\Rightarrow $có $ 2020$ giá trị $ m$ thỏa mãn yêu cầu bài toán.}
\end{ex}

\begin{ex}%[2D1G1-3]%Câu 8
	\immini{
		Cho hàm số $y=f(x)$ có đồ thị $f'(x)\,$ như hình vẽ. Có bao nhiêu giá trị nguyên $m\in\left(-2020\,;\,2020\right)$ để hàm số $g(x)=f\left(2x-3\right)\,-\ln \left(1+x^2\right)-2mx$ đồng biến trên $\left(\dfrac{1}{2};2\right)$?
		\choice
		{$ 2020$}
		{\True $ 2019$}
		{$ 2021$}
		{$ 2018$}}{
		\begin{tikzpicture}[scale=0.9,font=\footnotesize, line join=round, line cap=round, >=stealth] %Đường cong bậc 3
			\draw[thick, ->] (-2.5,0)--(2.5,0);
			\draw[thick, ->] (0,-1.8)--(0,5.8);
			\draw (2.7,0) node[below] {$x$};
			\draw (0,5.9) node[left]{$y$};
			\draw (0,0) node[below left]{$0$};
			\draw[fill] (-2,0) circle (0.5pt)node[below]{$ -2 $};
			\draw[fill] (1,0) circle (0.5pt)node[below]{$ 1$};
			\draw[fill] (-1,0) circle (0.5pt)node[below]{$-1$};
			\draw[fill] (0,4) circle (0.5pt)node[above left]{$ 2$};
			%		\draw[fill] (0,2) circle (0.5pt)node[above left]{$ 2$};
			%		\draw[fill] (0,-2) circle (0.5pt)node[below left]{$ -2$};
			\draw[dashed] (-2,0)--(-2,4)--(1,4)--(1,0); 
			%		\draw[dashed](3,0)--(3,2)--(0,2);
			\draw[line width=1.2pt,smooth,samples=100,domain=-2.1:2.1] plot(\x,{-1*(\x)^3+0*(\x)^2+3*(\x)+2});		
			%\draw[line width=1.2pt,smooth,samples=100,domain=-3.3:2.8] plot(\x,{0.75*(\x)^2+0.5*\x-1});
			%	\draw (2.0,2.8) node[left]{$y=f'(x)$};
	\end{tikzpicture}	}
	\loigiai{
		Ta có $g'(x)=2f'\left(2x-3\right)-\dfrac{2x}{1+x^2}-2m$.\\
		Hàm số $ g(x)$ đồng biến trên $\left(\dfrac{1}{2};2\right)$ khi và chỉ khi \\
		$g'(x)\ge 0,\,\,\forall x\in\left(-1;\,2\right)$\\
		$\Leftrightarrow m\le{f}'\left(2x-3\right)-\dfrac{x}{1+x^2},\,\,\forall x\in\left(\dfrac{1}{2};2\right)$\\
		$\Leftrightarrow m\le\underset{x\in\left[\dfrac{1}{2};2\right]}{\min}\,\left[f'\left(2x-3\right)-\dfrac{x}{1+x^2}\right]$. \, \,  $(1)$\\
		Đặt $ t=2x-3$, khi đó $ x\in\left(\dfrac{1}{2};2\right)\Leftrightarrow t\in\left(-2;\,1\right)$.\\
		Từ đồ thị hàm $f'(x)$ suy ra $f'(t)\ge 0,\,\,\forall t\in\left(-2;1\right)$ và $f'(t)=0$ khi $ t=-1$.\\
		Tức là $f'\left(2x-3\right)\ge 0,\,\,\forall x\in\left(\dfrac{1}{2};\,2\right)$$\Rightarrow\underset{x\in\left[\dfrac{1}{2};2\right]}{\min}\,f'\left(2x-3\right)=0$ khi $ x=1$. $(2)$\\
		Xét hàm số $ h(x)=-\dfrac{x}{1+x^2}$ trên khoảng $\left(\dfrac{1}{2};\,2\right)$.\\
		Ta có $h'(x)=\dfrac{x^2-1}{\left(1+x^2\right)^2}$ và\\
		$h'(x)=0\Leftrightarrow{x^2}-1=0\Leftrightarrow x=\pm 1$.\\
		Bảng biến thiên của hàm số $ h(x)$ trên $\left(\dfrac{1}{2};\,2\right)$ như sau
		\begin{center}
			\begin{tikzpicture}
				\tkzTabInit[lgt=1.2,espcl=2.5,nocadre]
				{$x$ /0.7, $h'(x)$ /0.7,$h(x)$ /2.5}
				{$\dfrac{1}{2}$ , $1$,$2$}
				\tkzTabLine{,-,0,+,}
				\tkzTabVar{+/$  $ ,-/$ \-\dfrac{1}{2} $, +/$ $}
			\end{tikzpicture}
		\end{center}
		Từ bảng biến thiên suy ra $ h(x)\ge-\dfrac{1}{2}$$\Rightarrow\underset{x\in\left[\dfrac{1}{2};2\right]}{\min}\,h(x)=-\dfrac{1}{2}$ khi $ x=1$. \, \,  $(3)$\\
		Từ $(1)$, $(2)$ và $(3)$ suy ra $ m\le-\dfrac{1}{2}$.\\
		Kết hợp với $ m\in\mathbb{Z}$, $ m\in\left(-2020;\,2020\right)$ thì $ m\in\left\{-2019;\,-201;\ldots ;-2;-1\right\}$.\\
		Vậy có tất cả $ 2019$ giá trị $ m$ cần tìm.}
\end{ex}

\begin{ex}%[2D1G1-3]%Câu 9
	Cho hàm số $ f(x)$ liên tục trên $\mathbb{R}$ và có đạo hàm $f'(x)=x^2\left(x-2\right)\left(x^2-6x+m\right)$ với mọi $ x\in\mathbb{R}$. Có bao nhiêu số nguyên $ m$ thuộc đoạn $\left[-2020;2020\right]$ để hàm số $ g(x)=f\left(1-x\right)$ nghịch biến trên khoảng $\left(-\infty ;-1\right)$?
	\choice
	{$ 2016$}
	{$ 2014$}
	{\True $ 2012$}
	{$ 2010$}
	\loigiai{
		Ta có \\
		$g'(x)=f'\left(1-x\right)=-\left(1-x\right)^2\left(-x-1\right)\left[\left(1-x\right)^2-6\left(1-x\right)+m\right]$
		$=\left(x-1\right)^2\left(x+1\right)\left(x^2+4x+m-5\right)$.\\
		Hàm số $ g(x)$ nghịch biến trên khoảng $\left(-\infty ;-1\right)$\\
		$\Leftrightarrow{g}'(x)\le 0,\forall x<-1$ $(*)$, (dấu \lq\lq $=$\rq\rq \, xảy ra tại hữu hạn điểm).\\
		Với $ x<-1$ thì $\left(x-1\right)^2>0$ và $ x+1<0$ nên\\
		$(*)$ $\Leftrightarrow{x^2}+4x+m-5\ge 0,\forall x<-1 \Leftrightarrow m\ge-x^2-4x+5,\forall x<-1$.\\
		Xét hàm số $ y=-x^2-4x+5$ trên khoảng $\left(-\infty ;-1\right)$, ta có bảng biến thiên
		\begin{center}
			\begin{tikzpicture}
				\tkzTabInit[lgt=1.8,espcl=2.3]
				{$x$ /1.2, $y'$ /1.2,$y$ /2}
				{$-\infty$ , $-2$,$-1$}
				\tkzTabLine{,+,0,-,}
				\tkzTabVar{-/$ -\infty $ ,+/$9 $, -/$ 8$}
			\end{tikzpicture}
		\end{center}
		Từ bảng biến thiên suy ra $ m\ge 9$.\\
		Kết hợp với $ m$ thuộc đoạn $\left[-2020;2020\right]$ và $ m$ nguyên nên $ m\in\left\{ 9;10;11;\ldots ;2020\right\}$.\\
		Vậy có $ 2012$ số nguyên $ m$ thỏa mãn đề bài.}
\end{ex}

\begin{ex}%[2D1G1-3]%Câu 10
	\immini{
		Cho hàm số $f(x)$ xác định và liên tục trên $ R$. Hàm số $y=f'(x)$ liên tục trên $\mathbb{R}$ và có đồ thị như hình vẽ bên.
		Xét hàm số $g(x)=f\left(x-2m\right)+\dfrac{1}{2}{\left(2m-x\right)^2}+2020$, với $ m$ là tham số thực. Gọi $ S$ là tập hợp các giá trị nguyên dương của $ m$ để hàm số $ y=g(x)$ nghịch biến trên khoảng $\left(3;4\right)$. Hỏi số phần tử của $ S$ bằng bao nhiêu?
		\choice
		{$4$}
		{\True $2$}
		{$3$}
		{Vô số}}
	{
		\begin{tikzpicture}[scale=0.7,>=stealth, font=\footnotesize, line join=round, line cap=round]
			\def\xmin{-3.5} \def\xmax{4.5}
			\def\ymin{-5.2} \def\ymax{4}
			\clip(\xmin,\ymin) rectangle (\xmax,\ymax);
			\draw[->] (\xmin,0)--(\xmax,0) node [below]{$x$};
			\draw[->] (0,\ymin)--(0,\ymax) node [left]{$y$};
			\node at (0,0) [below left]{$O$};
			\path
			(-3.1,3.7) coordinate (A)
			(-3,3) coordinate (B)
			(0,-2) coordinate (C)
			(0.65,-2) coordinate (D)
			(1,-1) coordinate (E)
			(3,-3) coordinate (F)
			(3.4,-5) coordinate (G);
			\draw[smooth]
			(A)..controls +(-88:0.1) and +(93:.1)..
			(B)..controls +(-87:0.3) and +(-100:8.5)..
			(C)..controls +(75:.8) and +(180:.1)..
			(D)..controls +(0:.1) and +(-105:.3)..
			(E)..controls +(70:2) and +(97:0.4)..
			(F)..controls +(-80:.1) and +(90:0.3)..
			(G);
			\draw[dashed] 
			(-3,0)node[below]{$-3$}|-(0,3)node[right]{$3$}
			(1,0)node[above]{$1$}|-(0,-1)node[left]{$-1$}
			(3,0)node[above]{$3$}|-(0,-3)node[below right]{$-3$};
			\fill 
			(0,-2) circle(1.5pt)
			(-3,3) circle(1.5pt)
			(3,-3) circle(1.5pt)
			(1,-1) circle(1.5pt);
			\node at (2.1,-4) {$y=f'(x)$};
		\end{tikzpicture}
	}
	\loigiai{
		Ta có $g'(x)=f'\left(x-2m\right)-\left(2m-x\right)$.		Đặt $h(x)=f'(x)-\left(-x\right)$.\\
		Từ đồ thị hàm số $y=f'(x)$ và đồ thị hàm số $y=-x$ trên hình vẽ suy ra \\
		$h(x)\le 0\Leftrightarrow f'(x)\le-x\Leftrightarrow\hoac{
			&-3\le x\le 1\\ 
			& x\ge 3.}$ 
		\begin{center}
			\begin{tikzpicture}[scale=0.7,>=stealth, font=\footnotesize, line join=round, line cap=round]
				\def\xmin{-3.5} \def\xmax{4.5}
				\def\ymin{-5.2} \def\ymax{4}
				\clip(\xmin,\ymin) rectangle (\xmax,\ymax);
				\draw[->] (\xmin,0)--(\xmax,0) node [below]{$x$};
				\draw[->] (0,\ymin)--(0,\ymax) node [left]{$y$};
				\node at (0,0) [below left]{$O$};
				\path
				(-3.1,3.7) coordinate (A)
				(-3,3) coordinate (B)
				(0,-2) coordinate (C)
				(0.65,-2) coordinate (D)
				(1,-1) coordinate (E)
				(3,-3) coordinate (F)
				(3.4,-5) coordinate (G);
				\draw[smooth]
				(A)..controls +(-88:0.1) and +(93:.1)..
				(B)..controls +(-87:0.3) and +(-100:8.5)..
				(C)..controls +(75:.8) and +(180:.1)..
				(D)..controls +(0:.1) and +(-105:.3)..
				(E)..controls +(70:2) and +(97:0.4)..
				(F)..controls +(-80:.1) and +(90:0.3)..
				(G);
				\draw[dashed] 
				(-3,0)node[below]{$-3$}|-(0,3)node[right]{$3$}
				(1,0)node[above]{$1$}|-(0,-1)node[left]{$-1$}
				(3,0)node[above]{$3$}|-(0,-3)node[below right]{$-3$};
				\fill 
				(0,-2) circle(1.5pt)
				(-3,3) circle(1.5pt)
				(3,-3) circle(1.5pt)
				(1,-1) circle(1.5pt);
				\draw[smooth,samples=300,domain=-3.2:3.7] plot(\x,{-(\x)});
				\node at (2.1,-4) {$y=f'(x)$};
				\node at (-1,2.1) {$y=h(x)$};
			\end{tikzpicture}
		\end{center}
		Ta có $ g'(x)=h\left(x-2m\right)\le 0\Leftrightarrow\hoac{
			&-3\le x-2m\le 1\\ 
			& x-2m\ge 3}\Leftrightarrow\hoac{
			& 2m-3\le x\le 2m+1\\ 
			& x\ge 2m+3.}$.\\
		Suy ra hàm số $ y=g(x)$ nghịch biến trên các khoảng $\left(2m-3;2m+1\right)$ và $\left(2m+3;+\infty\right)$.\\
		Do đó hàm số $ y=g(x)$ nghịch biến trên khoảng $\left(3;4\right)$ $\Leftrightarrow\hoac{
			&\heva{
				& 2m-3\le 3\\ 
				& 2m+1\ge 4}\\ 
			& 2m+3\le 3}\Leftrightarrow\hoac{
			&\dfrac{3}{2}\le m\le 3\\ 
			& m\le 0.}$ \\
		Mặt khác, do $ m$ nguyên dương nên $ m\in\left\{ 2;3\right\}\Rightarrow S=\left\{ 2;3\right\}$. Vậy số phần tử của $ S$ bằng $2$.\\
	}
	
\end{ex}

\begin{ex}%[2D1G1-3]%Câu 11
	Cho hàm số $f(x)$ có đạo hàm trên $\mathbb{R}$ là $f'(x)=\left(x-1\right)\left(x+3\right)$. Có bao nhiêu giá trị nguyên của tham số $m$ thuộc đoạn $\left[-10;20\right]$ để hàm số $y=f\left(x^2+3x-m\right)$ đồng biến trên khoảng $\left(0;2\right)$?
	\choice
	{\True $ 18$}
	{$ 17$}
	{$ 16$}
	{$ 20$}
	\loigiai{
		Ta có $y'=f'\left(x^2+3x-m\right)=\left(2x+3\right){f}'\left(x^2+3x-m\right)$.\\
		Theo đề bài ta có $f'(x)=\left(x-1\right)\left(x+3\right)$\\
		suy ra $f'(x)>0\Leftrightarrow\hoac{
			& x<-3\\ 
			& x>1}$ và $f'(x)<0\Leftrightarrow-3<x<1$ .\\
		Hàm số đồng biến trên khoảng $\left(0;2\right)$ khi $y'\ge 0,\forall x\in\left(0;2\right)$\\
		$\Leftrightarrow\left(2x+3\right){f}'\left(x^2+3x-m\right)\ge 0,\forall x\in\left(0;2\right)$.\\
		Do $x\in\left(0;2\right)$ nên $2x+3>0,\forall x\in\left(0;2\right)$. Do đó, ta có\\
		$y'\ge 0,\forall x\in\left(0;2\right)\Leftrightarrow f'\left(x^2+3x-m\right)\ge 0$\\
		$\Leftrightarrow\hoac{
			&{x^2}+3x-m\le-3\\ 
			&{x^2}+3x-m\ge 1}\Leftrightarrow\hoac{
			& m\ge{x^2}+3x+3\\ 
			& m\le{x^2}+3x-1}$\\
		$\Leftrightarrow\hoac{
			& m\ge\underset{\left[0;2\right]}{\max}\,\left(x^2+3x+3\right)\\ 
			& m\le\underset{\left[0;2\right]}{\min}\,\left(x^2+3x-1\right)} \Leftrightarrow\hoac{
			& m\ge 13\\ 
			& m\le-1}$.\\
		Do $m\in\left[-10;20\right]$, $ m\in\mathbb{Z}$ nên có $ 18$ giá trị nguyên của $m$ thỏa yêu cầu đề bài.}
\end{ex}

\begin{ex}%[2D1G1-3]%Câu 12
	Cho các hàm số $f(x)=x^3+4x+m$ và $g(x)=\left(x^2+2018\right){\left(x^2+2019\right)^2}{\left(x^2+2020\right)^3}$ . Có bao nhiêu giá trị nguyên của tham số $m\in\left[-2020;2020\right]$ để hàm số $g\left(f(x)\right)$ đồng biến trên $\left(2;+\infty\right)$ ?
	\choice
	{$2005$}
	{\True $2037$}
	{$4016$}
	{$4041$}
	\loigiai{
		Ta có $f(x)=x^3+4x+m$ và \\
		$g(x)=\left(x^2+2018\right){\left(x^2+2019\right)^2}{\left(x^2+2020\right)^3}=a_{12}{x^{12}}+a_{10}{x^{10}}+...+a_2x^2+a_0$.\\
		Suy ra $f'(x)=3x^2+4$ , $g'(x)=12a_{12}{x^{11}}+10a_{10}{x^9}+...+2a_2x$.\\
		Và có 
		\begin{eqnarray*}
			\left[g\left(f(x)\right)\right]' &=& f'(x)\left[12a_{12}{\left(f(x)\right)^{11}}+10a_{10}{\left(f(x)\right)^9}+...+2a_2f(x)\right]\\
			&=& f(x)f'(x)\left(12a_{12}{\left(f(x)\right)^{10}}+10a_{10}{\left(f(x)\right)^8}+...+2a_2\right).
		\end{eqnarray*} 
		Dễ thấy $a_{12};{a_{10}};...;{a_2};{a_0}>0$ và $f'(x)=3x^2+4>0$, $\forall x>2$.\\
		Do đó $f'(x)\left(12a_{12}{\left(f(x)\right)^{10}}+10a_{10}{\left(f(x)\right)^8}+...+2a_2\right)>0$ , $\forall x>2$.\\
		Hàm số $g\left(f(x)\right)$ đồng biến trên $\left(2;+\infty\right)$ khi $\left[g\left(f(x)\right)\right]^{'}\ge 0$, $\forall x>2$\\
		$\Rightarrow  f(x)\ge 0$, $\forall x>2 \Leftrightarrow x^3+4x+m\ge 0$, $\forall x>3 \Leftrightarrow  m\ge-x^3-4x$, $\forall x>2$\\
		$ \Rightarrow  m\ge\underset{\left[2;+\infty\right)}{\max}\,\left(-x^3-4x\right)=-16$.\\
		Vì $m\in\left[-2020;2020\right]$ và $m\in\mathbb{Z}$ nên có $2037$ giá trị thỏa mãn $m$ .}
\end{ex}

\begin{ex}%[2D1G1-3]%Câu 13
	Cho hàm số $y=f(x)$ có đạo hàm $f'(x)=x{\left(x+1\right)^2}\left(x^2+2mx+1\right)$ với mọi $x \in \mathbb{R}$. Có bao nhiêu số nguyên âm $m$ để hàm số $g(x)=f\left(2x+1\right)$ đồng biến trên khoảng $\left(3;5\right)$?
	\choice
	{\True $3$}
	{$2$}
	{$4$}
	{$6$}
	\loigiai{
		Ta có $g'(x)=2f'(2x+1)=2(2x+1)(2x+2)^2[(2x+1)^2+2m(2x+1)+1]$. 	Đặt $t=2x+1$\\
		Để hàm số $g(x)$ đồng biến trên khoảng $\left(3;5\right)$ khi và chỉ khi 
		\begin{eqnarray*}
			& & g'(x)\ge 0,\forall x\in\left(3;5\right) \\
			& \Leftrightarrow & t(t^2+2mt+1)\ge 0,\forall t\in\left(7;11\right)\Leftrightarrow{t^2}+2mt+1\ge 0,\,\,\forall t\in\left(7;11\right) \\
			&\Leftrightarrow & 2m\ge\dfrac{-t^2-1}{t},\,\,\,\forall t\in\left(7;11\right)
		\end{eqnarray*}	
		Xét hàm số $h(t)=\dfrac{-t^2-1}{t}$ trên $\left[7;11\right]$, có $h'(t)=\dfrac{-t^2+1}{t^2}$\\
		Bảng biến thiên
		\begin{center}
			\begin{tikzpicture}
				\tkzTabInit[espcl=3,lgt=1.2,nocadre]
				{$t$/0.7,$h'(t)$/0.7,$h(t)$/2.5}
				{$-\infty$,$1$,$11$,$+\infty$}
				\tkzTabLine{, ,,-,,,}
				%	\node (0) at ($(N12)+(0,-3)$) {$-\infty$};
				\node (1) at ($(N22)+(0,-0.8)$) [right] {$-\dfrac{50}{7}$};
				\node (2) at ($(N32)+(0,-2.5)$) [left] {$-\dfrac{122}{11}$};
				
				
				%				\node (3) at ($(N11+(-0.5,0))$) {};
				%				\node (4) at ($(N23)$) {};
				\fill[pattern=north east lines] (7.0,-0.7) rectangle (10,-4.4);
				\fill[pattern=north east lines] (1.5,-0.7) rectangle (4.5,-4.4);
				\draw[->] (1)--(2);	
				\draw[dashed] (4.5,-0.7)--(4.5,-4.4);
				\draw[dashed] (7.0,-0.7)--(7.0,-4.4);	
			\end{tikzpicture}		
		\end{center}
		Dựa vào BBT ta có $2m\ge\dfrac{-t^2-1}{t},\,\,\,\forall t\in\left(7;11\right)\Leftrightarrow 2m\ge\underset{\left[7;11\right]}{\max}\,h(t)\Leftrightarrow m\ge-\dfrac{50}{14}$\\
		Vì $ m\in{\mathbb{Z}^-}\Rightarrow m \in \{-3;-2;-1\}$ .
	}
\end{ex}

\begin{ex}%[2D1G1-3]%Câu 14
	Cho hàm số $y=f(x)$ có bảng biến thiên như sau\\
	\begin{center}
		\begin{tikzpicture}[>=stealth,scale = 1]
			\tkzTabInit[lgt=1,espcl=2.5,nocadre]
			{$x$ /0.7, $y'$ /0.7,$y$ /2.5}
			{$-\infty$,$0$,$2$,$+\infty$}
			\tkzTabLine{ ,-,0,+,0,-,}
			\tkzTabVar{-/$-\infty$, +/$4$,- /$0$, +/{ $+\infty$}}
		\end{tikzpicture}
	\end{center}
	Có bao nhiêu số nguyên $m<2019$ để hàm số $g(x)=f\left(x^2-2x+m\right)$ đồng biến trên khoảng $\left(1;+\infty\right)$?
	\choice
	{\True $2016$}
	{$2015$}
	{$2017$}
	{$2018$}
	\loigiai{
		Ta có $g'(x)=\left(x^2-2x+m\right)'{f}'\left(x^2-2x+m\right)=2\left(x-1\right){f}'\left(x^2-2x+m\right)$ .\\
		Hàm số $y=g(x)$ đồng biến trên khoảng $\left(1;+\infty\right)$ khi và chỉ khi $g'(x)\ge 0,\forall x\in\left(1;+\infty\right)$ và\\
		$g'(x)=0$ tại hữu hạn điểm \\
		$\Leftrightarrow 2\left(x-1\right){f}'\left(x^2-2x+m\right)\ge 0,\forall x\in\left(1;+\infty\right)$\\
		$\Leftrightarrow{f}'\left(x^2-2x+m\right)\ge 0,\forall x\in\left(1;+\infty\right)$ $\Leftrightarrow\hoac{
			&{x^2}-2x+m\ge 2,\forall x\in\left(1;+\infty\right)\\ 
			&{x^2}-2x+m\le 0,\forall x\in\left(1;+\infty\right).}$\\
		Xét hàm số $y=x^2-2x+m$, ta có bảng biến thiên
		\begin{center}
			\begin{tikzpicture}[>=stealth,scale = 1]
				\tkzTabInit[lgt=1,espcl=2.5,nocadre]
				{$x$ /0.7, $y'$ /0.7,$y$ /2.5}
				{$-\infty$,$2$,$+\infty$}
				\tkzTabLine{ ,-,0,+,}
				\tkzTabVar{+/$+\infty$, -/$m-1$, +/{$+\infty$}}
			\end{tikzpicture}
		\end{center}
		Dựa vào bảng biến thiên ta có\\
		TH1: $x^2-2x+m\ge 2,\forall x\in\left(1;+\infty\right)\Leftrightarrow m-1\ge 2\Leftrightarrow m\ge 3$ .\\
		TH2: $x^2-2x+m\le 0,\forall x\in\left(1;+\infty\right)$. Không có giá trị $m$ thỏa mãn.\\
		Vậy có $2016$ số nguyên $m<2019$ thỏa mãn yêu cầu bài toán.}
\end{ex}

\begin{ex}%[2D1G1-3]%Câu 15
	\immini{
		Cho hàm số $ y=f(x)$ có đạo hàm là hàm số $f'(x)$ trên $\mathbb{R}$. Biết rằng hàm số $ y=f'\left(x-2\right)+2$ có đồ thị như hình vẽ bên dưới. Hàm số $ f(x)$ đồng biến trên khoảng nào?
		\choice
		{$\left(-\infty ;3\right),\,\,\left(5;+\infty\right)$}
		{\True $\left(-\infty ;-1\right),\,\,\left(1;+\infty\right)$}
		{$\left(-1;1\right)$}
		{$\left(3;5\right)$}}{
		\begin{tikzpicture}[scale=0.7,font=\footnotesize, line join=round, line cap=round, >=stealth] %Đường cong bậc 3
			\draw[thick, ->] (-0.5,0)--(3.5,0);
			\draw[thick, ->] (0,-1.8)--(0,5.3);
			\draw (3.7,0) node[below] {$x$};
			\draw (0,5.4) node[left]{$y$};
			\draw (0,0) node[below left]{$0$};
			\draw[fill] (3,0) circle (0.5pt)node[below]{$ 3$};
			\draw[fill] (1,0) circle (0.5pt)node[below]{$ 1$};
			\draw[fill] (2,0) circle (0.5pt)node[above]{$2$};
			\draw[fill] (0,2) circle (0.5pt)node[left]{$ 2$};
			\draw[fill] (0,-1) circle (0.5pt)node[left]{$ -1$};
			%		\draw[fill] (0,2) circle (0.5pt)node[above left]{$ 2$};
			%		\draw[fill] (0,-2) circle (0.5pt)node[below left]{$ -2$};
			\draw[dashed] (3,0)--(3,2)--(0,2)--(1,2)--(1,0); 
			\draw[dashed](0,-1)--(2,-1)--(2,0);
			\draw[line width=1.2pt,smooth,samples=100,domain=0.6:3.4] plot(\x,{3*(\x)^2-12*(\x)+11});		
			%\draw[line width=1.2pt,smooth,samples=100,domain=-3.3:2.8] plot(\x,{0.75*(\x)^2+0.5*\x-1});
			%	\draw (2.0,2.8) node[left]{$y=f'(x)$};
	\end{tikzpicture}	}
	\loigiai{	
		Hàm số $ y=f'\left(x-2\right)+2$ có đồ thị $(C)$ như sau:\\
		\begin{center}
			\begin{tikzpicture}[scale=0.7,font=\footnotesize, line join=round, line cap=round, >=stealth] %Đường cong bậc 3
				\draw[thick, ->] (-0.5,0)--(3.5,0);
				\draw[thick, ->] (0,-1.8)--(0,5.3);
				\draw (3.7,0) node[below] {$x$};
				\draw (0,5.4) node[left]{$y$};
				\draw (0,0) node[below left]{$0$};
				\draw[fill] (3,0) circle (0.5pt)node[below]{$ 3$};
				\draw[fill] (1,0) circle (0.5pt)node[below]{$ 1$};
				\draw[fill] (2,0) circle (0.5pt)node[above]{$2$};
				\draw[fill] (0,2) circle (0.5pt)node[left]{$ 2$};
				\draw[fill] (0,-1) circle (0.5pt)node[left]{$ -1$};
				%		\draw[fill] (0,2) circle (0.5pt)node[above left]{$ 2$};
				%		\draw[fill] (0,-2) circle (0.5pt)node[below left]{$ -2$};
				\draw[dashed] (3,0)--(3,2)--(0,2)--(1,2)--(1,0); 
				\draw[dashed](0,-1)--(2,-1)--(2,0);
				\draw[line width=1.2pt,smooth,samples=100,domain=0.6:3.4] plot(\x,{3*(\x)^2-12*(\x)+11});		
				%\draw[line width=1.2pt,smooth,samples=100,domain=-3.3:2.8] plot(\x,{0.75*(\x)^2+0.5*\x-1});
				%	\draw (2.0,2.8) node[left]{$y=f'(x)$};
			\end{tikzpicture}
		\end{center}
		Dựa vào đồ thị $(C)$ ta có\\ $f'\left(x-2\right)+2>2,\forall x\in\left(-\infty ;1\right)\cup\left(3;+\infty\right)\Leftrightarrow{f}'\left(x-2\right)>0,\forall x\in\left(-\infty ;1\right)\cup\left(3;+\infty\right)$ .\\
		Đặt $ x*=x-2$ suy ra $f'\left(x*\right)>0,\forall x*\in\left(-\infty ;-1\right)\bigcup\left(1;+\infty\right)$.\\
		Vậy hàm số $ f(x)$ đồng biến trên khoảng $\left(-\infty ;-1\right),\,\,\left(1;+\infty\right)$.}
\end{ex}

\begin{ex}%[2D1G1-2]%Câu 16
	\immini{
		Cho hàm số $ y=f(x)$ có đạo hàm là hàm số $f'(x)$ trên $\mathbb{R}$. Biết rằng hàm số $ y=f'\left(x+2\right)-2$ có đồ thị như hình vẽ bên dưới. Hàm số $ f(x)$ nghịch biến trên khoảng nào?
		\choice
		{$\left(-3;-1\right),\,\,\left(1;3\right)$}
		{\True $\left(-1;1\right),\,\,\left(3;5\right)$}
		{$\left(-\infty ;-2\right),\,\,\left(0;2\right)$}
		{$\left(-5;-3\right),\,\,\left(-1;1\right)$}}{
		\begin{tikzpicture}[scale=0.7,font=\footnotesize, line join=round, line cap=round, >=stealth] %Đường cong bậc 3
			\draw[thick, ->] (-3.8,0)--(4.0,0);
			\draw[thick, ->] (0,-4.8)--(0,3.5);
			\draw (4.2,0) node[below] {$x$};
			\draw (0,3.7) node[left]{$y$};
			\draw (0,0) node[below left]{$0$};
			\draw[fill] (-3,0) circle (0.5pt)node[above]{$ -3$};
			\draw[fill] (-1,0) circle (0.5pt)node[above]{$ -1$};
			\draw[fill] (1,0) circle (0.5pt)node[above]{$ 1$};
			\draw[fill] (3,0) circle (0.5pt)node[above]{$3$};
			\draw[fill] (0,2) circle (0.5pt)node[above left]{$ 2$};
			\draw[fill] (0,-1) circle (0.5pt)node[above right]{$ -1$};
			%		\draw[fill] (0,2) circle (0.5pt)node[above left]{$ 2$};
			%		\draw[fill] (0,-2) circle (0.5pt)node[below left]{$ -2$};
			\draw[dashed] (-3,0)--(-3,-2)--(3,-2)--(3,0) (-1,0)--(-1,-2) (1,0)--(1,-2) (-3.494,0)--(-3.494,2)--(3.494,2)--(3.494,0); 
			\draw[line width=1.2pt,smooth,samples=100,domain=-3.6:3.6] plot(\x,{0.11*(\x)^4-1.11*(\x)^2-1});		
			%\draw[line width=1.2pt,smooth,samples=100,domain=-3.3:2.8] plot(\x,{0.75*(\x)^2+0.5*\x-1});
			%	\draw (2.0,2.8) node[left]{$y=f'(x)$};
	\end{tikzpicture}	}
	\loigiai{
		Hàm số $ y=f'\left(x+2\right)-2$ có đồ thị $(C)$ như sau
		\begin{center}
			\begin{tikzpicture}[scale=0.7,font=\footnotesize, line join=round, line cap=round, >=stealth] %Đường cong bậc 3
				\draw[thick, ->] (-3.8,0)--(4.0,0);
				\draw[thick, ->] (0,-4.8)--(0,3.5);
				\draw (4.2,0) node[below] {$x$};
				\draw (0,3.7) node[left]{$y$};
				\draw (0,0) node[below left]{$0$};
				\draw[fill] (-3,0) circle (0.5pt)node[above]{$ -3$};
				\draw[fill] (-1,0) circle (0.5pt)node[above]{$ -1$};
				\draw[fill] (1,0) circle (0.5pt)node[above]{$ 1$};
				\draw[fill] (3,0) circle (0.5pt)node[above]{$3$};
				\draw[fill] (0,2) circle (0.5pt)node[above left]{$ 2$};
				\draw[fill] (0,-1) circle (0.5pt)node[above right]{$ -1$};
				%		\draw[fill] (0,2) circle (0.5pt)node[above left]{$ 2$};
				%		\draw[fill] (0,-2) circle (0.5pt)node[below left]{$ -2$};
				\draw[dashed] (-3,0)--(-3,-2)--(3,-2)--(3,0) (-1,0)--(-1,-2) (1,0)--(1,-2) (-3.494,0)--(-3.494,2)--(3.494,2)--(3.494,0); 
				\draw[line width=1.2pt,smooth,samples=100,domain=-3.6:3.6] plot(\x,{0.11*(\x)^4-1.11*(\x)^2-1});		
				%\draw[line width=1.2pt,smooth,samples=100,domain=-3.3:2.8] plot(\x,{0.75*(\x)^2+0.5*\x-1});
				%	\draw (2.0,2.8) node[left]{$y=f'(x)$};
			\end{tikzpicture}
		\end{center}
		Dựa vào đồ thị $(C)$ ta có\\
		$f'\left(x+2\right)-2<-2,\forall x\in\left(-3;-1\right)\bigcup\left(1;3\right)\Leftrightarrow{f}'\left(x+2\right)<0,\forall x\in\left(-3;-1\right)\bigcup\left(1;3\right)$.\\
		Đặt $ x^*=x+2$ suy ra: $f'\left(x^*\right)<0,\forall x^*\in\left(-1;1\right)\bigcup\left(3;5\right)$.\\
		Vậy: Hàm số $ f(x)$ đồng biến trên khoảng $\left(-1;1\right),\,\,\left(3;5\right)$.}
\end{ex}

\begin{ex}%[2D1G1-2]%Câu 17
	\immini{
		Cho hàm số $ y=f(x)$ có đạo hàm là hàm số $f'(x)$ trên $\mathbb{R}$. Biết rằng hàm số $ y=f'\left(x-2\right)+2$ có đồ thị như hình vẽ bên dưới. Hàm số $ f(x)$ nghịch biến trên khoảng nào?
		\choice
		{$\left(-\infty ;2\right)$}
		{\True $\left(-1;1\right)$}
		{$\left(\dfrac{3}{2};\dfrac{5}{2}\right)$}
		{$\left(2;+\infty\right)$}}{
		\begin{tikzpicture}[scale=0.7,font=\footnotesize, line join=round, line cap=round, >=stealth] %Đường cong bậc 3
			\draw[thick, ->] (-0.5,0)--(3.5,0);
			\draw[thick, ->] (0,-1.8)--(0,5.3);
			\draw (3.7,0) node[below] {$x$};
			\draw (0,5.4) node[left]{$y$};
			\draw (0,0) node[below left]{$0$};
			\draw[fill] (3,0) circle (0.5pt)node[below]{$ 3$};
			\draw[fill] (1,0) circle (0.5pt)node[below]{$ 1$};
			\draw[fill] (2,0) circle (0.5pt)node[above]{$2$};
			\draw[fill] (0,2) circle (0.5pt)node[left]{$ 2$};
			\draw[fill] (0,-1) circle (0.5pt)node[left]{$ -1$};
			%		\draw[fill] (0,2) circle (0.5pt)node[above left]{$ 2$};
			%		\draw[fill] (0,-2) circle (0.5pt)node[below left]{$ -2$};
			\draw[dashed] (3,0)--(3,2)--(0,2)--(1,2)--(1,0); 
			\draw[dashed](0,-1)--(2,-1)--(2,0);
			\draw[line width=1.2pt,smooth,samples=100,domain=0.6:3.4] plot(\x,{3*(\x)^2-12*(\x)+11});		
			%\draw[line width=1.2pt,smooth,samples=100,domain=-3.3:2.8] plot(\x,{0.75*(\x)^2+0.5*\x-1});
			%	\draw (2.0,2.8) node[left]{$y=f'(x)$};
	\end{tikzpicture}	}
	\loigiai{
		Hàm số $ y=f'\left(x-2\right)+2$ có đồ thị $(C)$ như sau
		\begin{center}
			\begin{tikzpicture}[scale=0.7,font=\footnotesize, line join=round, line cap=round, >=stealth] %Đường cong bậc 3
				\draw[thick, ->] (-0.5,0)--(3.5,0);
				\draw[thick, ->] (0,-1.8)--(0,5.3);
				\draw (3.7,0) node[below] {$x$};
				\draw (0,5.4) node[left]{$y$};
				\draw (0,0) node[below left]{$0$};
				\draw[fill] (3,0) circle (0.5pt)node[below]{$ 3$};
				\draw[fill] (1,0) circle (0.5pt)node[below]{$ 1$};
				\draw[fill] (2,0) circle (0.5pt)node[above]{$2$};
				\draw[fill] (0,2) circle (0.5pt)node[left]{$ 2$};
				\draw[fill] (0,-1) circle (0.5pt)node[left]{$ -1$};
				%		\draw[fill] (0,2) circle (0.5pt)node[above left]{$ 2$};
				%		\draw[fill] (0,-2) circle (0.5pt)node[below left]{$ -2$};
				\draw[dashed] (3,0)--(3,2)--(0,2)--(1,2)--(1,0); 
				\draw[dashed](0,-1)--(2,-1)--(2,0);
				\draw[line width=1.2pt,smooth,samples=100,domain=0.6:3.4] plot(\x,{3*(\x)^2-12*(\x)+11});		
				%\draw[line width=1.2pt,smooth,samples=100,domain=-3.3:2.8] plot(\x,{0.75*(\x)^2+0.5*\x-1});
				%	\draw (2.0,2.8) node[left]{$y=f'(x)$};
			\end{tikzpicture}
		\end{center}
		Dựa vào đồ thị $(C)$ ta có\\
		$f'\left(x-2\right)+2<2,\forall x\in\left(1;3\right)\Leftrightarrow{f}'\left(x-2\right)<0,\forall x\in\left(1;3\right)$.\\
		Đặt $ x^*=x-2$ thì $f'\left(x^*\right)<0,\forall x^*\in\left(-1;1\right)$.\\
		Vậy: Hàm số $ f(x)$ nghịch biến trên khoảng $\left(-1;1\right)$.\\
		Cách khác:\\
		Tịnh tiến sang trái hai đơn vị và xuống dưới $2$ đơn vị thì từ đồ thị $(C)$ sẽ thành đồ thị của hàm$ y=f'(x)$. Khi đó $f'(x)<0,\forall x\in\left(-1;1\right)$.\\
		Vậy hàm số $ f(x)$ nghịch biến trên khoảng $\left(-1;1\right)$.}
\end{ex}

\begin{ex}%[2D1G1-2]%Câu 18
	Cho hàm số $y=f(x)$ có đạo hàm cấp $ 3$ liên tục trên $\mathbb{R}$ và thỏa mãn $f(x)\cdot f'''(x)=x{\left(x-1\right)^2}{\left(x+4\right)^3}$ với mọi $x\in\mathbb{R}$ và $g(x)=\left[f'(x)\right]^2-2f(x)\cdot f''(x)$. Hàm số $h(x)=g\left(x^2-2x\right)$ đồng biến trên khoảng nào dưới đây?
	\choice
	{$\left(-\infty ;1\right)$}
	{$\left(2;+\infty\right)$}
	{$\left(0;1\right)$}
	{\True $\left(1;2\right)$}
	\loigiai{		
		Ta có $g'(x)=2f''(x){f}'(x)-2f'(x)\cdot f''(x)-2f(x)\cdot f'''(x)=-2f(x)\cdot f'''(x);$\\
		Khi đó $\left(h(x)\right)'=\left(2x-2\right){g}'\left(x^2-2x\right)=-2\left(2x-2\right)\left(x^2-2x\right){\left(x^2-2x-1\right)^2}{\left(x^2-2x+4\right)^3}$\\
		$h'(x)=0\Leftrightarrow\hoac{
			& x=0\\ 
			& x=1\\ 
			& x=2\\ 
			& x=1\pm\sqrt{2}.}$ 
		Ta có bảng xét dấu của $h'(x)$
		\begin{center}
			\begin{tikzpicture}
				\tkzTabInit[lgt=1.2,espcl=2,nocadre]
				{$t$/0.7, $h'(x)$ /.7} % first column
				{$-\infty$, $1-\sqrt{2}$,$0$, $1$,$2$,$1+\sqrt{2}$, $+\infty$} % first row
				\tkzTabLine { ,+,0,-,0,+,0,-,0,+,0,- } % second row
				%				\tkzTabLine {,-,z,+,t,+,} % third row
				%				\tkzTabLine {,+,d,-,z,+,} % last row
			\end{tikzpicture}
		\end{center}
		Suy ra hàm số $h(x)=g\left(x^2-2x\right)$ đồng biến trên khoảng $\left(1;2\right)$.}
\end{ex}

\begin{ex}%[2D1G1-2]%Câu 19
	Cho hàm số $ y=f(x)$ xác định trên $\mathbb{R}$. Hàm số $ y=g(x)=f'\left(2x+3\right)+2$ có đồ thị là một parabol với tọa độ đỉnh $ I\left(2;-1\right)$ và đi qua điểm $ A\left(1;2\right)$. Hỏi hàm số $ y=f(x)$ nghịch biến trên khoảng nào dưới đây?
	\choice
	{\True $\left(5;9\right)$}
	{$\left(1;2\right)$}
	{$\left(-\infty ;9\right)$}
	{$\left(1;3\right)$}
	\loigiai{	
		Xét hàm số $ g(x)=f'\left(2x+3\right)+2$ có đồ thị là một Parabol nên có phương trình dạng $ y=g(x)=a{x^2}+bx+c\,\,\,\,(P)$.\\
		Vì $(P)$ có đỉnh $ I\left(2;-1\right)$ nên $\heva{
			&\dfrac{-b}{2a}=2\\ 
			& g(2)=-1} \Leftrightarrow\heva{
			&-b=4a\\ 
			& 4a+2b+c=-1} \Leftrightarrow\heva{
			& 4a+b=0\\ 
			& 4a+2b+c=-1}$.\\
		Vì $(P)$ đi qua điểm $ A\left(1;2\right)$ nên $ g(1)=2\Leftrightarrow a+b+c=2$.\\
		Ta có hệ phương trình $\heva{
			& 4a+b=0\\ 
			& 4a+2b+c=-1\\ 
			& a+b+c=2} \Leftrightarrow\heva{
			& a=3\\ 
			& b=-12\\ 
			& c=11}$ nên $ g(x)=3x^2-12x+11$.\\
		Đồ thị của hàm $ y=g(x)$ là
		\begin{center}
			\begin{tikzpicture}[scale=0.7,font=\footnotesize, line join=round, line cap=round, >=stealth] %Đường cong bậc 3
				\draw[thick, ->] (-0.5,0)--(3.5,0);
				\draw[thick, ->] (0,-1.8)--(0,5.3);
				\draw (3.7,0) node[below] {$x$};
				\draw (0,5.4) node[left]{$y$};
				\draw (0,0) node[below left]{$0$};
				\draw[fill] (3,0) circle (0.5pt)node[below]{$ 3$};
				\draw[fill] (1,0) circle (0.5pt)node[below]{$ 1$};
				\draw[fill] (2,0) circle (0.5pt)node[above]{$2$};
				\draw[fill] (0,2) circle (0.5pt)node[left]{$ 2$};
				\draw[fill] (0,-1) circle (0.5pt)node[left]{$ -1$};
				%		\draw[fill] (0,2) circle (0.5pt)node[above left]{$ 2$};
				%		\draw[fill] (0,-2) circle (0.5pt)node[below left]{$ -2$};
				\draw[dashed] (3,0)--(3,2)--(0,2)--(1,2)--(1,0) (3.2,2)--(3,2); 
				\draw[dashed](0,-1)--(2,-1)--(2,0);
				\draw[line width=1.2pt,smooth,samples=100,domain=0.6:3.4] plot(\x,{3*(\x)^2-12*(\x)+11});		
				%\draw[line width=1.2pt,smooth,samples=100,domain=-3.3:2.8] plot(\x,{0.75*(\x)^2+0.5*\x-1});
				%	\draw (2.0,2.8) node[left]{$y=f'(x)$};
			\end{tikzpicture}	
		\end{center}
		Theo đồ thị ta thấy $ f'(2x+3)\le 0\Leftrightarrow f'(2x+3)+2\le 2\Leftrightarrow 1\le x\le 3$.\\
		Đặt $ t=2x+3\Leftrightarrow x=\dfrac{t-3}{2}$ khi đó $ f'(t)\le 0\Leftrightarrow 1\le\dfrac{t-3}{2}\le 3\Leftrightarrow 5\le t\le 9$.\\
		Vậy $ y=f(x)$ nghịch biến trên khoảng $\left(5;9\right)$.}
\end{ex}

\begin{ex}%[2D1G1-2]%Câu 20
	\immini{
		Cho hàm số $ y=f(x)$, hàm số $f'(x)=x^3+a{x^2}+bx+c\left(a,b,c\in\mathbb{R}\right)$ có đồ thị như hình vẽ bên.
		Hàm số $ g(x)=f\left(f'(x)\right)$ nghịch biến trên khoảng nào dưới đây?
		\choice
		{$\left(1;+\infty\right)$}
		{\True $\left(-\infty ;-2\right)$}
		{$\left(-1;0\right)$}
		{$\left(-\dfrac{\sqrt{3}}{3};\dfrac{\sqrt{3}}{3}\right)$}}{
		\begin{tikzpicture}[scale=0.8,font=\footnotesize, line join=round, line cap=round, >=stealth] %Đường cong bậc 3
			\draw[thick, ->] (-1.7,0)--(1.7,0);
			\draw[thick, ->] (0,-2.7)--(0,3.0);
			\draw (1.9,0) node[below] {$x$};
			\draw (0,3.2) node[left]{$y$};
			\draw (0,0) node[below left]{$0$};
			\draw[fill] (-1,0) circle (0.5pt)node[above left]{$ -1 $};
			\draw[fill] (1,0) circle (0.5pt)node[below right]{$ 1$};
			\draw[line width=1.2pt,smooth,samples=100,domain=-1.3:1.3] plot(\x,{2.667*(\x)^3+0*(\x)^2-2.667*\x});		
			%\draw[line width=1.2pt,smooth,samples=100,domain=-3.3:2.8] plot(\x,{0.75*(\x)^2+0.5*\x-1});
		\end{tikzpicture}	
	}
	\loigiai{	
		Vì các điểm $\left(-1;0\right),\left(0;0\right),\left(1;0\right)$ thuộc đồ thị hàm số $ y=f'(x)$ nên ta có hệ\\
		$\heva{
			&-1+a-b+c=0\\ 
			& c=0\\ 
			& 1+a+b+c=0} \Leftrightarrow\heva{
			& a=0\\ 
			& b=-1\\ 
			& c=0} \Rightarrow {f}'(x)=x^3-x\Rightarrow f''(x)=3x^2-1$.\\
		Ta có $ g(x)=f\left(f'(x)\right)\Rightarrow{g}'(x)=f'\left(f'(x)\right)\cdot f''(x)$.\\
		Xét \\
		$g'(x)=0\Leftrightarrow{g}'(x)=f'\left(f'(x)\right)\cdot f''(x)=0$\\
		$\Leftrightarrow {f}'\left(x^3-x\right)\left(3x^2-1\right)=0\Leftrightarrow\hoac{
			&{x^3}-x=0\\ 
			&{x^3}-x=1\\ 
			&{x^3}-x=-1\\ 
			& 3x^2-1=0} \Leftrightarrow \hoac{
			& x=\pm 1\\ 
			& x=0\\ 
			& x=x_1(x_1\approx 1,325)\\ 
			& x=x_2(x_2\approx-1,325)\\ 
			& x=\pm\dfrac{\sqrt{3}}{3}.}$\\
		Bảng biến thiên
		\begin{center}
			\begin{tikzpicture}
				\tkzTabInit[lgt=1.2,espcl=2,nocadre]
				{$t$/0.7, $h'(x)$ /.7} % first column
				{$-\infty$, $-1{,}325$,$-1$, $-\dfrac{\sqrt{3}}{3}$,$0$,$\dfrac{\sqrt{3}}{3}$,$1$,$1{,}325$, $+\infty$} % first row
				\tkzTabLine { ,-,0,+,0,-,0,+,0,-,0,+,0,-,0,+, } % second row
				%				\tkzTabLine {,-,z,+,t,+,} % third row
				%				\tkzTabLine {,+,d,-,z,+,} % last row
			\end{tikzpicture}
		\end{center}
		Dựa vào bảng biến thiên ta có $ g(x)$ nghịch biến trên $\left(-\infty ;-2\right)$}
\end{ex}
\Closesolutionfile{ans}
\indapan{10}{ans/CD1/Muc_9_10}
\chapter{GÓC - HÌNH HỌC KHÔNG GIAN}
\setcounter{dang}{0}
\setcounter{ex}{0}
\section{Mức độ 7,8,9,10}
\Opensolutionfile{ans}[ans/CD7/DA2]
\begin{dang}
	{Góc giữa đường thẳng với đường thẳng}
\end{dang}
\begin{ex}%[1H3B2-3]
	[Mã 101 - 2021 Lần 1]
    \immini{Cho hình lăng trụ đứng $ABC.A'B'C'$ có tất cả các cạnh bằng Góc giữa đường thẳng $AA'$ và $BC'$ bằng
        \choice
        {$30^{\circ}$}
        {$90^{\circ}$}
        {\True $45^{\circ}$}
        {$60^{\circ}$}}
    {\begin{tikzpicture}[scale=0.7, font=\footnotesize, line join=round, line cap=round, >=stealth]
            \path
            (0,0) coordinate (A)
            (2,-2) coordinate (B)
            (5,0) coordinate (C) ;
            \foreach \x in {A,B,C}
            \path (\x)+(0,-4.5) coordinate(\x');
            \draw[dashed] (A')--(C');
            \draw(A')--(B')--(C') (A')--(A)--(B)--(C)--(C') (B)--(B') (A)--(C);
            \foreach \p/\g in {A/180,B/90,C/0,A'/180,B'/-45,C'/0}\draw[fill=black] (\p) circle (1pt)node[shift={(\g:.4)}]{$\p$};
    \end{tikzpicture}}
    \loigiai{
        \immini{Vì $AA'\parallel BB'$ nên $(AA',BC')=(BB',BC')=\widehat{B'BC}$.\\
            Ta có $\tan\widehat{B'BC}=\dfrac{B'C'}{BB'}=1\Rightarrow\widehat{B'BC}=45^{\circ}$.}
        {\begin{tikzpicture}[scale=0.7, font=\footnotesize, line join=round, line cap=round, >=stealth]
                \path
                (0,0) coordinate (A)
                (2,-2) coordinate (B)
                (5,0) coordinate (C) ;
                \foreach \x in {A,B,C}
                \path (\x)+(0,-4.5) coordinate(\x');
                \draw[dashed] (A')--(C');
                \draw(A')--(B')--(C') (A')--(A)--(B)--(C)--(C')--(B)--(B') (A)--(C);

                \foreach \p/\g in {A/180,B/90,C/0,A'/180,B'/-45,C'/0}\draw[fill=black] (\p) circle (1pt)node[shift={(\g:.4)}]{$\p$};
                \pic[draw,angle radius=3.5mm,angle eccentricity=1.5] {angle = B'--B--C'};
        \end{tikzpicture}}
    }
\end{ex}

\begin{ex}%[Mã 103 - 2021 - Lần 1]%Câu 2.%[1H3B2-3]
    \immini{Cho hình lăng trụ đứng $ABC.A'B'C'$ có tất cả các cạnh bằng nhau (tham khảo hình bên dưới).
        Góc giữa hai đường thẳng $A'B$ và $CC'$ bằng
        \choice
        {\True $45^{\circ}$}
        {$30^{\circ}$}
        {$90^{\circ}$}
        {$60^{\circ}$}}
    {\begin{tikzpicture}[scale=0.7, font=\footnotesize, line join=round, line cap=round, >=stealth]
            \path
            (0,0) coordinate (A)
            (2,-2) coordinate (B)
            (5,0) coordinate (C) ;
            \foreach \x in {A,B,C}
            \path (\x)+(0,-4.5) coordinate(\x');
            \draw[dashed] (A')--(C');
            \draw(A')--(B')--(C') (A')--(A)--(B)--(C)--(C') (B)--(B') (A)--(C);
            \foreach \p/\g in {A/180,B/90,C/0,A'/180,B'/-45,C'/0}\draw[fill=black] (\p) circle (1pt)node[shift={(\g:.4)}]{$\p$};
    \end{tikzpicture}}
    \loigiai{
        \immini{Ta có $CC'\parallel BB'$. Nên $\left(\widehat{A'B; CC'}\right) =\left(\widehat{A'B; BB'}\right)$ = $\widehat{A'BB'}$ ( $\widehat{A'BB'}$ là góc nhọn).\\
            Mặt khác, tam giác $A'BB'$ là tam giác vuông cân ($A'B=BB'$ và $A'B\perp BB'$) suy ra $\widehat{A'BB'}=45^{\circ}$.\\
            Vậy góc giữa hai đường thẳng $A'B$ và $CC'$ bằng $45^{\circ}$.}
        {\begin{tikzpicture}[scale=0.7, font=\footnotesize, line join=round, line cap=round, >=stealth]
                \path
                (0,0) coordinate (A)
                (2,-2) coordinate (B)
                (5,0) coordinate (C) ;
                \foreach \x in {A,B,C}
                \path (\x)+(0,-4.5) coordinate(\x');
                \draw[dashed] (A')--(C');
                \draw(A')--(B')--(C') (B)--(A')--(A)--(B)--(C)--(C') (B)--(B') (A)--(C);
                \foreach \p/\g in {A/180,B/90,C/0,A'/180,B'/-45,C'/0}\draw[fill=black] (\p) circle (1pt)node[shift={(\g:.4)}]{$\p$};
                \pic[draw,angle radius=3.5mm,angle eccentricity=1.5] {angle = A'--B--B'};
        \end{tikzpicture}}
    }
\end{ex}

\begin{ex}%[Mã 102 - 2021 Lần 1]%Câu 3.%[1H3B2-3]
    \immini{Cho hình lăng trụ đứng $ABC.A'B'C'$ có tất cả các cạnh bằng nhau (tham khảo hình bên).
        Góc giữa hai đường thẳng $AA'$ và $B'C$ bằng
        \choice
        {$90^{\circ}$}
        {\True $45^{\circ}$}
        {$30^{\circ}$}
        {$60^{\circ}$}}
    {\begin{tikzpicture}[scale=0.7, font=\footnotesize, line join=round, line cap=round, >=stealth]
            \path
            (0,0) coordinate (A)
            (2,-2) coordinate (B)
            (5,0) coordinate (C) ;
            \foreach \x in {A,B,C}
            \path (\x)+(0,-4.5) coordinate(\x');
            \draw[dashed] (A')--(C');
            \draw(A')--(B')--(C') (A')--(A)--(B)--(C)--(C') (B)--(B') (A)--(C);
            \foreach \p/\g in {A/180,B/90,C/0,A'/180,B'/-45,C'/0}\draw[fill=black] (\p) circle (1pt)node[shift={(\g:.4)}]{$\p$};
    \end{tikzpicture}}
    \loigiai{
        \immini{Ta có $AA'\parallel CC'$ nên
            $(AA',B'C)=(CC',B'C)$.\\
            Mặt khác tam giác $BCC'$ vuông tại $C'$ có $CC'=B'C'$ nên là tam giác vuông cân.\\
            Vậy góc giữa hai đường thẳng $AA'$ và $B'C$ bằng $45^{\circ}$.}
        {\begin{tikzpicture}[scale=0.7, font=\footnotesize, line join=round, line cap=round, >=stealth]
                \path
                (0,0) coordinate (A)
                (2,-2) coordinate (B)
                (5,0) coordinate (C) ;
                \foreach \x in {A,B,C}
                \path (\x)+(0,-4.5) coordinate(\x');
                \draw[dashed] (A')--(C');
                \draw(A')--(B')--(C') (A')--(A)--(B)--(C)--(C') (B)--(B')--(C) (A)--(C);
                \foreach \p/\g in {A/180,B/90,C/0,A'/180,B'/-45,C'/0}\draw[fill=black] (\p) circle (1pt)node[shift={(\g:.4)}]{$\p$};
                \pic[draw,angle radius=3.5mm,angle eccentricity=1.5] {angle = B'--C--C'};
        \end{tikzpicture}}
    }
\end{ex}

\begin{ex}%[Mã 104 - 2021 Lần 1]%Câu 4.%[1H3B2-3]
    \immini{Cho hình lăng trụ đứng $ABC.A'B'C'$ có tất cả các cạnh bằng nhau (tham khảo hình bên).
        Góc giữa hai đường thẳng $AB'$ và $CC'$ bằng
        \choice
        {$30^{\circ}$}
        {$90^{\circ}$}
        {$60^{\circ}$}
        {\True $45^{\circ}$}}
    {\begin{tikzpicture}[scale=0.7, font=\footnotesize, line join=round, line cap=round, >=stealth]
            \path
            (0,0) coordinate (A)
            (2,-2) coordinate (B)
            (5,0) coordinate (C) ;
            \foreach \x in {A,B,C}
            \path (\x)+(0,-4.5) coordinate(\x');
            \draw[dashed] (A')--(C');
            \draw(A')--(B')--(C') (A')--(A)--(B)--(C)--(C') (B)--(B') (A)--(C);
            \foreach \p/\g in {A/180,B/90,C/0,A'/180,B'/-45,C'/0}\draw[fill=black] (\p) circle (1pt)node[shift={(\g:.4)}]{$\p$};
    \end{tikzpicture}}
    \loigiai{
        \immini{Ta có $BB'\parallel CC'$ (do $BB'$ và $CC'$ là cạnh bên của hình lăng trụ).\\
            Suy ra $\left(\widehat{AB',CC'}\right)=\left(\widehat{AB',BB'}\right)$.\\
            Tứ giác $ABB'A'$ là hình vuông (do $ABC.A'B'C'$ là lăng trụ đứng có tất cả các cạnh bằng nhau) nên $\widehat{AB'B}=45^{\circ}$.\\
            Vậy $\left(\widehat{AB',CC'}\right)=\left(\widehat{AB',BB'}\right)=\widehat{AB'B}=45^{\circ}$.}
        {\begin{tikzpicture}[scale=0.7, font=\footnotesize, line join=round, line cap=round, >=stealth]
                \path
                (0,0) coordinate (A)
                (2,-2) coordinate (B)
                (5,0) coordinate (C) ;
                \foreach \x in {A,B,C}
                \path (\x)+(0,-4.5) coordinate(\x');
                \draw[dashed] (A')--(C');
                \draw(A')--(B')--(C') (A')--(A)--(B)--(C)--(C') (B)--(B')--(A)--(C);
                \foreach \p/\g in {A/180,B/90,C/0,A'/180,B'/-45,C'/0}\draw[fill=black] (\p) circle (1pt)node[shift={(\g:.4)}]{$\p$};
                \pic[draw,angle radius=3.5mm,angle eccentricity=1.5] {angle = B--B'--A};
        \end{tikzpicture}}
    }
\end{ex}

\begin{ex}%[Đề Tham Khảo 2018]%Câu 5.%[1H3B2-3]
    \immini{Cho tứ diện $OABC$ có $OA$, $OB$, $OC$ đôi một vuông góc với nhau và $OA=OB=OC$. Gọi $M$ là trung điểm của $BC$ (tham khảo hình vẽ bên dưới). Góc giữa hai đường thẳng $OM$ và $AB$ bằng
        \choice
        {$45^{\circ}$}
        {$90^{\circ}$}
        {$30^{\circ}$}
        {\True $60^{\circ}$}}
    {\begin{tikzpicture}[>=stealth,line join=round,line cap=round,font=\footnotesize,scale=0.8]
            \path (0,0) coordinate (O)
            (3,-2.5) coordinate (C)
            (6,0) coordinate (B)
            ($(O)+(0,4)$)coordinate (A)
            ($(B)!0.5!(C)$)coordinate (M)
            ;
            \draw (C)--(O)--(A)--(B)--(C)--(A);
            \draw[dashed] (M)--(O)--(B);
            \foreach \p / \r in {A/90,B/0,C/-90,O/180,M/-45}
            \fill (\p) circle (1.2pt) node[shift={(\r:3mm)}]{$\p$};
    \end{tikzpicture}}
    \loigiai{
        \immini{Đặt $OA=a$ suy ra $OB=OC=a$ và $AB=BC=AC=a\sqrt{2}$.\\
            Gọi $N$ là trung điểm $AC$ ta có $MN\parallel AB$ và $MN=\dfrac{a\sqrt{2}}{2}$.\\
            Suy ra góc $\widehat{(OM,AB)}=\widehat{(OM,MN)}$. Xét $\widehat{OMN}$.\\
            Trong tam giác $OMN$ có $ON=OM=MN=\dfrac{a\sqrt{2}}{2}$ nên $OMN$ là tam giác đều.\\
            Suy ra $\widehat{OMN}=60^{\circ}$.\\
            Vậy $\widehat{(OM,AB)}=\widehat{(OM,MN)}=60^{\circ}$.}
        {\begin{tikzpicture}[>=stealth,line join=round,line cap=round,font=\footnotesize,scale=0.8]
                \path (0,0) coordinate (O)
                (3,-2.5) coordinate (C)
                (6,0) coordinate (B)
                ($(O)+(0,4)$)coordinate (A)
                ($(B)!0.5!(C)$)coordinate (M)
                ($(A)!0.5!(C)$)coordinate (N)
                ;
                \draw (C)--(O)--(A)--(B)--(C)--(A) (O)--(N)--(M);
                \draw[dashed] (M)--(O)--(B);
                \foreach \p / \r in {A/90,B/0,C/-90,O/180,M/-45,N/70}
                \fill (\p) circle (1.2pt) node[shift={(\r:3mm)}]{$\p$};
        \end{tikzpicture}}
    }
\end{ex}

\begin{ex}%[Chuyên Long An - 2021]%Câu 6.%[1H3B2-3]
    Cho hình lập phương $ABCD.EFGH$. Góc giữa cặp véc-tơ $\overrightarrow{AF}$ và $\overrightarrow{EG}$ bằng
    \choice
    {$30^{\circ}$}
    {$120^{\circ}$}
    {\True $60^{\circ}$}
    {$90^{\circ}$}
    \loigiai{
        \immini{Ta có $\left(\overrightarrow{AF},\overrightarrow{EG}\right)=\left(\overrightarrow{AF},\overrightarrow{AC}\right)=\widehat{CAF}$.\\
            $\triangle CAF$ là tam giác đều, nên $\widehat{CAF}=60^{\circ}$.}
        {\begin{tikzpicture}[scale=.7, font=\footnotesize, line join=round, line cap=round, >=stealth]
                \path
                (0,0) coordinate (D)
                (3.5,0) coordinate (C)
                (1.5,1.5) coordinate (A)
                (A)+(C) coordinate (B);
                \path (A)+(0,-3.5) coordinate(E);
                \path (B)+(0,-3.5) coordinate(F);
                \path (C)+(0,-3.5) coordinate(G);
                \path (D)+(0,-3.5) coordinate(H);
                \draw[dashed] (H)--(E)--(F)--(A)--(E)--(G);
                \draw (B)--(F)--(G)  (A)--(D)--(C)--(A)--(B)--(C)--(G) (C)--(F) (D)--(H)--(G);
                \foreach \p/\g in {E/150,F/0,G/0,H/180,A/180,B/0,C/0,D/160}\draw[fill=black] (\p) circle (1pt)node[shift={(\g:.3)}]{$\p$};
        \end{tikzpicture}}
    }
\end{ex}

\begin{ex}%[THPT Nguyễn Huệ - Phú Yên - 2021]%Câu 7.%[1H3K2-2]
    Hình chóp $S.ABC$ có $SA$, $SB$, $SC$ đôi một vuông góc với nhau và $SA=SB=SC$. Gọi $I$ là trung điểm của $AB$. Góc giữa $SI$ và $BC$ bằng
    \choice
    {$30^{\circ}$}
    {\True $60^{\circ}$}
    {$45^{\circ}$}
    {$90^{\circ}$}
    \loigiai{
        \immini{Ta có
            \begin{eqnarray*}
                \cos\left(\overrightarrow{SI};\overrightarrow{BC}\right)&=& \dfrac{\overrightarrow{SI}\cdot\overrightarrow{BC}}{SI\cdot BC}=\dfrac{\dfrac{1}{2}\left(\overrightarrow{SA}+\overrightarrow{SB}\right)\cdot\overrightarrow{BC}}{\dfrac{BC}{2}\cdot BC}\\
                &= & \dfrac{\overrightarrow{SA}\cdot\overrightarrow{BC}+\overrightarrow{SB}\cdot\overrightarrow{BC}}{BC^2}=\dfrac{\overrightarrow{SB}\cdot\overrightarrow{BC}}{BC^2}\\
                &= &\dfrac{SB\cdot BC\cdot\cos 135^{\circ}}{BC^2}=\dfrac{SB\cdot SB\sqrt{2}\cdot\cos 135^{\circ}}{2SB^2}\\
                &= & \dfrac{\sqrt{2}\cdot\cos 135^{\circ}}{2}=-\dfrac{1}{2}.
            \end{eqnarray*}
            Suy ra $\left(\overrightarrow{SI};\overrightarrow{BC}\right)=120^{\circ}\Rightarrow(SI;BC)=60^{\circ}$.}
        {\begin{tikzpicture}[>=stealth,line join=round,line cap=round,font=\footnotesize,scale=0.8]
                \path (0,0) coordinate (S)
                (3,-2.5) coordinate (C)
                (6,0) coordinate (B)
                ($(S)+(0,4)$)coordinate (A)
                ($(B)!0.5!(A)$)coordinate (I)
                ;
                \draw (C)--(S)--(A)--(B)--(C)--(A);
                \draw[dashed] (I)--(S)--(B);
                \foreach \p / \r in {A/90,B/0,C/-90,S/180,I/45}
                \fill (\p) circle (1.2pt) node[shift={(\r:3mm)}]{$\p$};
        \end{tikzpicture}}
    }
\end{ex}

\begin{ex}%[THPT Nguyễn Đức Cảnh - Thái Bình - 2021]%Câu 8.%[1H3B2-3]
    Cho hình lập phương $ABCD.A_1B_1C_1D_1$ có cạnh $a$. Gọi $I$ là trung điểm $BD$. Góc giữa hai đường thẳng $A_1D$ và $B_1I$ bằng
    \choice
    {$120^{\circ}$}
    {\True $30^{\circ}$}
    {$45^{\circ}$}
    {$60^{\circ}$}
    \loigiai{
        \immini{Ta có $B_1C\parallel A_1D\Rightarrow(A_1D, B_1I)=(B_1C, B_1I)$.\\
            Vì $ABCD.A_1B_1C_1D_1$ là hình lập phương cạnh $a$ nên $B_1C=a\sqrt{2}; IC=\dfrac{a\sqrt{2}}{2}; B_1I=\dfrac{a\sqrt{6}}{2}$.\\
            Xét $\triangle B_1IC$ có $\cos\widehat{IB_1C}=\dfrac{B_1I^2+B_1C^2-IC^2}{2B_1I\cdot B_1C}=\dfrac{\sqrt{3}}{2}$ \\
            $ \Rightarrow\widehat{IB_1C}=30^{\circ} $.\\
            Do đó $\left(A_1D, B_1I\right)=(B_1C, B_1I)=\widehat{IB_1C}=30^{\circ}$.}
        {\begin{tikzpicture}[scale=0.7, font=\footnotesize, line join=round, line cap=round, >=stealth]
                \path
                (0,0) coordinate (A)
                (3.5,0) coordinate (B)
                (1.5,1.5) coordinate (D)
                (D)+(B) coordinate (C)
                ($(B)!0.5!(D)$) coordinate (I)
                ;
                \path (A)+(0,3.5) coordinate(A_1);
                \path (B)+(0,3.5) coordinate(B_1);
                \path (C)+(0,3.5) coordinate(C_1);
                \path (D)+(0,3.5) coordinate(D_1);
                \draw[dashed] (C)--(A)--(D)--(C) (A_1)--(D)--(D_1) (C)--(I)--(B_1);
                \draw (A_1)--(A)--(B)--(C)--(C_1)--(D_1)--(A_1)--(B_1)--(B) (B_1)--(C_1) (B_1)--(C);
                \foreach \p/\g in {A/-90,B/-90,C/0,D/180,A_1/180,B_1/0,C_1/0,D_1/160,I/-110}\draw[fill=black] (\p) circle (1pt)node[shift={(\g:.4)}]{$\p$};
        \end{tikzpicture}}
    }
\end{ex}

\begin{ex}%[THPT Lê Quý Đôn Điện Biên 2019]%Câu 9.%[1H3K2-2]
    Cho tứ diện $ABCD$ với $AC=\dfrac{3}{2}AD$, $\widehat{CAB}=\widehat{DAB}=60^{\circ}$, $CD=AD$. Gọi $\varphi$ là góc giữa hai đường thẳng $AB$ và $CD$. Chọn khẳng định đúng về góc $\varphi$.
    \choice
    {$\cos\varphi=\dfrac{3}{4}$}
    {$30^{\circ}$}
    {$60^{\circ}$}
    {\True $\cos\varphi=\dfrac{1}{4}$}
    \loigiai{
        \immini{Ta có
            \begin{eqnarray*}
                \overrightarrow{AB}\cdot\overrightarrow{CD}&=& \overrightarrow{AB}\cdot\left(\overrightarrow{AD}-\overrightarrow{AC}\right)=\overrightarrow{AB}\cdot\overrightarrow{AD}-\overrightarrow{AB\cdot}\overrightarrow{AC}\\
                &= & AB\cdot AD\cdot \cos 60^{\circ}-AB\cdot AC\cdot \cos 60^{\circ}\\
                &= & AB\cdot AD\cdot \cos 60^{\circ}-AB\cdot\dfrac{3}{2}AD\cdot \cos 60^{\circ}\\
                &= &\dfrac{-1}{4}AB\cdot AD.
            \end{eqnarray*}
            $\cos\left(\overrightarrow{AB},\overrightarrow{CD}\right)=\dfrac{\overrightarrow{AB}\cdot\overrightarrow{CD}}{AB\cdot CD}=\dfrac{-1}{4}\Rightarrow \cos\varphi=\dfrac{1}{4}$.}
        {\begin{tikzpicture}[font=\footnotesize, line join=round, line cap=round, >=stealth]
                \path 	(0:0) coordinate (B)
                ++(0:4) coordinate (D)
                ++(-145:3) coordinate (C)
                ($(B)+(60:3)$) coordinate (A);
                \draw 	(A)--(C)--(B)--(A)--(D)--(C);
                \draw[dashed] 	(D)--(B);
                \draw pic[draw,angle radius=17,angle eccentricity=1.5]{angle=B--A--C};
                \draw pic[draw,angle radius=15,angle eccentricity=1.5]{angle=B--A--D};
                \draw (1.3,2)node[below]{$60^\circ$};
                \draw (1.9,2.1)node[below]{$60^\circ$};
                \foreach \x /\goc in {A/90,B/180,C/-90,D/0}
                \fill[black] (\x) circle (1pt)
                ($(\x)+(\goc:3mm)$) node {$\x$};
        \end{tikzpicture}}
    }
\end{ex}

\begin{ex}%[THPT Hoàng Hoa Thám Hưng Yên 2019]%Câu 10.%[1H3B2-3]
    \immini{Cho hình hộp chữ nhật $ABCD.A'B'C'D'$, biết đáy $ABCD$ là hình vuông. Tính góc giữa $A'C$ và $BD$.
        \choice
        {\True $90^{\circ}$}
        {$30^{\circ}$}
        {$60^{\circ}$}
        {$45^{\circ}$}}
    {\begin{tikzpicture}[scale=0.7, font=\footnotesize, line join=round, line cap=round, >=stealth]
            \path
            (0,0) coordinate (A)
            (4,0) coordinate (D)
            (1.5,1.5) coordinate (B)
            (B)+(4,0) coordinate (C);
            \foreach \x in {A,B,C,D}
            \path (\x)+(0,3.8) coordinate(\x');
            \draw[dashed] (A)--(B)--(C)--(A') (D)--(B)--(B');
            \draw(A')--(D')--(C')--(B')--(A')--(A)--(D)--(C)--(C') (D)--(D');
            \foreach \p/\g in {A/180,D/0,C/0,B/180,A'/180,D'/0,C'/0,B'/160}\draw[fill=black] (\p) circle (1pt)node[shift={(\g:.3)}]{$\p$};
    \end{tikzpicture}}
    \loigiai{
        Vì $ABCD$ là hình vuông nên $BD\perp AC$.\\
        Mặt khác $AA'\perp(ABCD)\Rightarrow BD\perp AA'$.\\
        Ta có $\heva{& BD\perp AC\\& BD\perp AA'}\Rightarrow BD\perp(AA'C)\Rightarrow BD\perp A'C$.\\
        Do đó góc giữa $A'C$ và $BD$ bằng $90^{\circ}$.
    }
\end{ex}

\begin{ex}%[Chuyên KHTN 2019]%Câu 11.%[1H3B2-3]
    Cho tứ diện $ABCD$ có $AB=CD=2a$. Gọi $M$, $N$ lần lượt là trung điểm $AD$ và $BC$. Biết $MN=a\sqrt{3}$, góc giữa hai đường thẳng $AB$ và $CD$ bằng
    \choice
    {$45^{\circ}$}
    {$90^{\circ}$}
    {\True $60^{\circ}$}
    {$30^{\circ}$}
    \loigiai{
        \immini{Gọi $P$ là trung điểm $AC$, ta có $PM\parallel CD$ và $PN\parallel AB$, suy ra $\left(\widehat{AB,CD}\right)=\left(\widehat{PM,PN}\right)$.\\
            Dễ thấy $PM=PN=a$.\\
            Xét $\triangle PMN$ ta có
            \begin{eqnarray*}
                \cos\widehat{MPN}&= & \dfrac{PM^2+PN^2-MN^2}{2PM\cdot PN}\\
                &= & \dfrac{a^2+a^2-3a^2}{2\cdot a\cdot a}=-\dfrac{1}{2}\\
            \end{eqnarray*}
            $ \Rightarrow\widehat{MPN}=120^{\circ}\Rightarrow\left(\widehat{AB,CD}\right)=180^{\circ}-120^{\circ}=60^{\circ}$.}
        {\begin{tikzpicture}[font=\footnotesize, line join=round, line cap=round, >=stealth]
                \path 	(0:0) coordinate (A)
                ++(0:4) coordinate (D)
                ++(-145:3) coordinate (C)
                ($(A)+(60:3)$) coordinate (B)
                ($(A)!.5!(C)$) coordinate (P)
                ($(B)!.5!(C)$) coordinate (N)
                ($(A)!.5!(D)$) coordinate (M)
                ;
                \draw 	(B)--(C)--(A)--(B)--(D)--(C) (P)--(N);
                \draw[dashed] 	(D)--(A) (P)--(M)--(N);
                \foreach \x /\goc in {B/90,A/180,C/-90,D/0,N/160,M/30,P/200}
                \fill[black] (\x) circle (1pt)
                ($(\x)+(\goc:3mm)$) node {$\x$};
        \end{tikzpicture}}
    }
\end{ex}

\begin{ex}%[Chuyên Lương Văn Chánh Phú Yên 2019]%Câu 12.%[1H3B2-3]
    Cho hình lập phương $ABCD.A'B'C'D'$; gọi $M$ là trung điểm của $B'C'$. Góc giữa hai đường thẳng $AM$ và $BC'$ bằng
    \choice
    {\True $45^{\circ}$}
    {$90^{\circ}$}
    {$30^{\circ}$}
    {$60^{\circ}$}
    \loigiai{
        \immini{Giả sử cạnh của hình lập phương là $a>0$.\\
            Gọi $N$ là trung điểm đoạn thẳng $BB'$. Khi đó, $MN\parallel BC'$ nên $(AM,BC')=(AM,MN)$.\\
            Xét tam giác $A'B'M$ vuông tại $B'$ ta có $A'M =\sqrt{A'B'^2+B'M^2} =\sqrt{a^2+\dfrac{a^2}{4}} =\dfrac{a\sqrt{5}}{2}$.\\
            Xét tam giác $AA'M$ vuông tại $A'$ ta có $AM=\sqrt{AA'^2+A'M^2} =\sqrt{a^2+\dfrac{5a^2}{4}} =\dfrac{3a}{2}$.\\
            Có $AN=A'M=\dfrac{a\sqrt{5}}{2}$; $MN=\dfrac{BC'}{2}=\dfrac{a\sqrt{2}}{2}$.\\
            Trong tam giác $AMN$ ta có\\
            $\cos\widehat{AMN} =\dfrac{MA^2+MN^2-AN^2}{2\cdot MA\cdot MN} =\dfrac{\dfrac{9a^2}{4}+\dfrac{2a^2}{4}-\dfrac{5a^2}{4}}{2\cdot\dfrac{3a}{2}\cdot\dfrac{a\sqrt{2}}{2}} =\dfrac{6a^2}{4}\cdot\dfrac{4}{6a^2\sqrt{2}} =\dfrac{1}{\sqrt{2}}$.\\
            Suy ra $\widehat{AMN}=45^{\circ}$.\\
            Vậy $(AM,BC')=(AM,MN) =\widehat{AMN}=45^{\circ}$.}
        {\begin{tikzpicture}[scale=0.7, font=\footnotesize, line join=round, line cap=round, >=stealth]
                \path
                (0,0) coordinate (A)
                (4,0) coordinate (D)
                (1.5,1.5) coordinate (B)
                (B)+(4,0) coordinate (C)
                ;
                \foreach \x in {A,B,C,D}
                \path (\x)+(0,-3.9) coordinate(\x');
                \path ($(B)!.5!(B')$) coordinate (N);
                \path ($(C')!.5!(B')$) coordinate (M);
                \draw[dashed] (M)--(N)--(A)--(M)--(A')--(B')--(C')--(B)--(B');
                \draw (C')--(D')--(A')--(A)--(B)--(C)--(C') (A)--(D)--(C) (D)--(D');
                \foreach \p/\g in {A/180,B/180,C/0,D/0,A'/180,B'/180,C'/0,D'/0,M/-90,N/30}\draw[fill=black] (\p) circle (1pt)node[shift={(\g:.3)}]{$\p$};
        \end{tikzpicture}}
    }
\end{ex}

\begin{ex}%[Chuyên Hạ Long - 2018]%Câu 13.%[1H3K2-2]
    Cho hình chóp $S.ABC$ có độ dài các cạnh $SA=SB=SC=AB=AC=a$ và $BC=a\sqrt{2}$. Góc giữa hai đường thẳng $AB$ và $SC$ là
    \choice
    {$45^{\circ}$}
    {$90^{\circ}$}
    {\True $60^{\circ}$}
    {$30^{\circ}$}
    \loigiai{
        \immini{Ta có $BC=a\sqrt{2}$ nên tam giác $ABC$ vuông tại $A$.\\
            Vì $SA=SB=SC=a$ nên hình chiếu vuông góc của $S$ lên $(ABC)$ trùng với tâm $I$ của đường tròn ngoại tiếp tam giác $ABC$.\\
            Tam giác $ABC$ vuông tại $A$ nên $I$ là trung điểm của $BC$.\\
            Ta có $\cos(AB,SC) =\left|\cos\left(\overrightarrow{AB},\overrightarrow{SC}\right)\right|=\dfrac{\left|\overrightarrow{AB}\cdot\overrightarrow{SC}\right|}{AB\cdot SC}$.\\
            $\overrightarrow{AB}\cdot\overrightarrow{SC}=\overrightarrow{AB}\left(\overrightarrow{SI}+\overrightarrow{IC}\right) =\overrightarrow{AB}\cdot\overrightarrow{SI} =-\dfrac{1}{2}\overrightarrow{BA}\cdot\overrightarrow{BC} =-\dfrac{1}{2}BA\cdot BC\cdot\cos 45^{\circ} =-\dfrac{a^2}{2}$.\\
            $\cos(AB,SC)=\dfrac{\dfrac{a^2}{2}}{a^2} =\dfrac{1}{2}\Rightarrow\widehat{(AB,SC)} =60^{\circ}$.}
        {\begin{tikzpicture}[font=\footnotesize, line join=round, line cap=round, >=stealth]
                \path 	(0:0) coordinate (A)
                ++(0:4) coordinate (B)
                ++(-150:3) coordinate (C)
                ($(B)!0.5!(C)$) coordinate (M)
                ($(M)+(90:3.5)$) coordinate (S)
                ;
                \draw 	(A)--(C)--(B)
                (A)--(S)	(B)--(S)--(M)	(C)--(S)--(B);
                \draw[dashed] 	(A)--(B);
                \foreach \x /\goc in {A/180,B/0,C/-135,S/90,M/-90}
                \fill[black] (\x) circle (1pt)
                ($(\x)+(\goc:3mm)$) node {$\x$};
        \end{tikzpicture}}
        \noindent Cách 2: $\cos(AB,SC) =\left|\cos\left(\overrightarrow{AB},\overrightarrow{SC}\right)\right|=\dfrac{\left|\overrightarrow{AB}\cdot\overrightarrow{SC}\right|}{AB\cdot SC}$.\\
        Ta có $\overrightarrow{AB}\cdot\overrightarrow{SC} =\left(\overrightarrow{SB}-\overrightarrow{SA}\right)\overrightarrow{SC} =\overrightarrow{SB}\cdot\overrightarrow{SC}-\overrightarrow{SA}\cdot\overrightarrow{SC} =SB\cdot SC\cdot\cos 90^{\circ}-SA\cdot SC\cdot\cos 60^{\circ} =-\dfrac{a^2}{2}$.\\
        Khi đó $\cos(AB,SC)=\dfrac{\left|\dfrac{-a^2}{2}\right|}{a^2}=\dfrac{1}{2}$.
    }
\end{ex}

\begin{ex}%[Chuyên Đh Vinh 2018]%Câu 14.%[1H3K2-2]
    \immini{Cho hình lăng trụ tam giác đều $ABC.A'B'C'$ có $AB=a$ và $AA'=\sqrt{2} a$. Góc giữa hai đường thẳng $AB'$ và $BC'$ bằng
        \choice
        {\True $60^{\circ}$}
        {$45^{\circ}$}
        {$90^{\circ}$}
        {$30^{\circ}$}}
    {\begin{tikzpicture}[scale=0.7, font=\footnotesize, line join=round, line cap=round, >=stealth]
            \path
            (0,0) coordinate (A)
            (2,-2) coordinate (B)
            (5,0) coordinate (C) ;
            \foreach \x in {A,B,C}
            \path (\x)+(0,-4.5) coordinate(\x');
            \draw[dashed] (A')--(C');
            \draw (A')--(B')--(C') (A')--(A)--(B)--(C)--(C')--(B)--(B')--(A)--(C);
            \foreach \p/\g in {A/180,B/90,C/0,A'/180,B'/-45,C'/0}\draw[fill=black] (\p) circle (1pt)node[shift={(\g:.4)}]{$\p$};
    \end{tikzpicture}}
    \loigiai{
        Ta có
        \begin{eqnarray*}
            \overrightarrow{AB'}\cdot\overrightarrow{BC'}&= & \left(\overrightarrow{AB}+\overrightarrow{BB'}\right)\left(\overrightarrow{BC}+\overrightarrow{CC'}\right)\\
            &= & \overrightarrow{AB}\cdot\overrightarrow{BC}+\overrightarrow{AB}\cdot\overrightarrow{CC'}+\overrightarrow{BB'}\cdot\overrightarrow{BC}+\overrightarrow{BB'}\cdot\overrightarrow{CC'}\\
            &= & \overrightarrow{AB}\cdot\overrightarrow{BC}+\overrightarrow{AB}\cdot\overrightarrow{CC'}+\overrightarrow{BB'}\cdot\overrightarrow{BC}+\overrightarrow{BB'}\cdot\overrightarrow{CC'}\\
            &= & -\dfrac{a^2}{2}+0+0+2a^2=\dfrac{3a^2}{2}.
        \end{eqnarray*}
        Suy ra $\cos\left(\overrightarrow{AB'},\overrightarrow{BC'}\right)=\dfrac{\overrightarrow{AB'}\cdot\overrightarrow{BC'}}{\left|\overrightarrow{AB'}\right|\cdot\left|\overrightarrow{BC'}\right|} =\dfrac{\dfrac{3a^2}{2}}{a\sqrt{3}\cdot a\sqrt{3}}=\dfrac{1}{2}\Rightarrow\widehat{(AB',BC')}=60^{\circ}$.
    }
\end{ex}
%---------------------
\begin{ex}%%[1H3K2-4]
    [Kim Liên - Hà Nội - 2018]
    Cho tứ diện $ABCD$ có $DA=DB=DC=AC=AB=a$, $\widehat{ABC}=45^{\circ}$. Tính góc giữa hai đường thẳng $AB$ và $DC$.
    \choice
    {\True $60^{\circ}$}
    {$120^{\circ}$}
    {$90^{\circ}$}
    {$30^{\circ}$}
    \loigiai{
        Ta có tam giác $ABC$ vuông cân tại $A$, tam giác $BDC$ vuông cân tại $D$.\\
        Ta có
        \allowdisplaybreaks
        $\begin{aligned}[t]
            \vec{AB}\cdot\vec{CD}=\left(\vec{DB}-\vec{DA}\right)\vec{CD}&=\vec{DB}\cdot\vec{CD}-\vec{DA}\cdot\vec{CD}\\
            &=\left|\vec{DB}\right|\left|\vec{CD}\right|\cos\left(\vec{DB},\vec{CD}\right)-\left|\vec{DA}\right|\left|\vec{CD}\right|\cos\left(\vec{DA},\vec{CD}\right)\\
            &=-\dfrac{1}{2}a^2
        \end{aligned}$\\
        Mặt khác ta lại có $\cos\left(\vec{AB},\vec{CD}\right)=\dfrac{\vec{AB}\cdot\vec{CD}}{\left|\vec{AB}\right|\left|\vec{CD}\right|}=-\dfrac{1}{2}$\\
        Suy ra $\left(\vec{AB},\vec{DC}\right)=120^{\circ}\Rightarrow(AB,CD)=60^{\circ} $.
    }
\end{ex}
\begin{ex}%%[1H3K2-4]
    [Chuyên Trần Phú - Hải Phòng - 2018]
    Cho hình lập phương $ABCD.A'B'C'D'$. Gọi $M$, $N$ lần lượt là trung điểm của $AD$, $BB'$. Cosin của góc hợp bởi $MN$ và $AC'$ bằng
    \choice
    {$\dfrac{\sqrt{3}}{3}$}
    {\True $\dfrac{\sqrt{2}}{3}$}
    {$\dfrac{\sqrt{5}}{3}$}
    {$\dfrac{\sqrt{2}}{4}$}
    \loigiai{
        \begin{center}
            \begin{tikzpicture}[line cap=round,line join=round, >=stealth,scale=1]
                \def \a{-1.5} \def \b{-1}\def \c{4.5} \def \h{4}
                \path (.5,.5)coordinate(A)
                +(\a,\b)coordinate(B)
                +(\c,0)coordinate(D)
                ($(B)+(D)-(A)$)coordinate(C)
                +(0,\h) coordinate(C')
                ($(B)+(C')-(C)$)coordinate(B')
                ($(A)+(C')-(C)$)coordinate(A')
                ($(D)+(C')-(C)$)coordinate(D')
                ($(B)!.5!(B')$) coordinate (N)
                ($(A)!.5!(D)$) coordinate (M);
                \draw [dashed] (C')--(A)--(B)(D)--(A)--(A') (M)--(N);
                \draw (B')--(B)--(C)(B')--(C')--(C)--(D)--(D')--(A')--(B')(C')--(D');
                \foreach \x/\g in {A/135,B/-135,C/-45,D/0,A'/135,B'/180,C'/-20,D'/0,M/60,N/180}\fill[red] (\x) circle (1pt)+(\g:3mm) node[black]{$\x$};
            \end{tikzpicture}
        \end{center}
        Xét hình lập phương $ABCD.A'B'C'D'$ cạnh $a$.\\
        Đặt $\vec{a}=\vec{AB}$, $\vec{b}=\vec{AD}$, $\vec{c}=\vec{AA'}\Rightarrow\left|\vec{a}\right|=\left|\vec{b}\right|=\left|\vec{c}\right|=a$.\\
        Từ đó suy ra  $\vec{a}\cdot\vec{b}=\vec{b}\cdot\vec{c}=\vec{a}\cdot\vec{c}=0$.\\
        Ta có
        \begin{itemize}
            \item $\vec{MN}=\vec{AB}+\vec{BN}-\vec{AM}=\vec{a}-\dfrac{1}{2}\vec{b}+\dfrac{1}{2}\vec{c}\Rightarrow\left|\vec{MN}\right|=\sqrt{a^2+\dfrac{1}{4}a^2+\dfrac{1}{4}a^2}=\dfrac{a\sqrt{3}}{\sqrt{2}}$.
            \item $\vec{AC'}=\vec{AB}+\vec{AD}+\vec{AA'}=\vec{a}+\vec{b}+\vec{c}\Rightarrow\left|\vec{AC'}\right|=\sqrt{a^2+a^2+a^2}=a\sqrt{3}$.
            \item $\vec{AC'}\cdot\vec{MN}=a^2-\dfrac{1}{2}a^2+\dfrac{1}{2}a^2=a^2$.
        \end{itemize}
        Từ đso suy ra
        $\cos(MN;AC')=\left|\cos\left(\vec{MN};\vec{AC'}\right)\right|=\dfrac{\left|\vec{MN}\cdot\vec{AC'}\right|}{\left|\vec{MN}\right|\cdot\left|\vec{AC'}\right|}=\dfrac{\sqrt{2}}{3}$.
    }
\end{ex}
\begin{ex}%%[1H3K2-4]
    [Cụm 5 Trường Chuyên - ĐBSH - 2018]
    Cho hình chóp $S.ABCD$ có đáy là hình chữ nhật, $AB=2a$, $BC=a$. Hình chiếu vuông góc $H$ của đỉnh $S$ trên mặt phẳng đáy là trung điểm của cạnh $AB$, góc giữa đường thẳng $SC$ và mặt phẳng đáy bằng $60^{\circ}$. Tính cosin góc giữa hai đường thẳng $SB$ và $AC$
    \choice
    {$\dfrac{2}{\sqrt{7}}$}
    {\True $\dfrac{2}{\sqrt{35}}$}
    {$\dfrac{2}{\sqrt{5}}$}
    {$\dfrac{\sqrt{2}}{\sqrt{7}}$}
    \loigiai{
        \immini
        {
            Ta có $\left(SC,(ABCD)\right)= (SC,CH)=\widehat{SCH}=60^{\circ}$.\\
            %        $\cos(SB,AC)=\dfrac{\left|\vec{SB}\cdot\vec{AC}\right|}{SB\cdot AC}$.\\
            Ta có $AC=a\sqrt{5}$, $CH=\sqrt{a^2+a^2}=a\sqrt{2}$,\\ $SH=CH\cdot\tan\widehat{SCH}=a\sqrt{6}$
            \allowdisplaybreaks
            \begin{align*}
                \vec{SB}\cdot\vec{AC}&=\left(\vec{SH}+\vec{HB}\right)\left(\vec{AB}+\vec{BC}\right)\\ &=\vec{SH}\cdot\vec{AB}+\vec{SH}\cdot\vec{BC}+\vec{HB}\cdot\vec{AB}+\vec{HB}\cdot\vec{BC}\\
                &=\vec{HB}\cdot\vec{AB}+\vec{HB}\cdot\vec{BC} =\dfrac{1}{2}AB^2=2a^2.
            \end{align*}
            $SB=\sqrt{SH^2+HB^2} =\sqrt{(a\sqrt{6})^2+a^2}=a\sqrt{7}$.
        }
        {
            \begin{tikzpicture}[line cap=round,line join=round, >=stealth,scale=1]
                \def \a{-2} \def \b{-1} \def \c{4} \def \h{4}
                \path (0,0)coordinate(A)
                +(\a,\b)coordinate(B)
                +(\c,0)coordinate(D)
                ($(B)+(D)-(A)$)coordinate(C)
                ($(A)!1/2!(B)$)coordinate(H)
                +(0,\h)coordinate(S)
                ;
                \draw [dashed] (B)--(A)--(D) (A)--(S) (S)--(H)--(C)--(A);
                \draw (S)--(B)--(C)--(D)--(S)--(C)
                pic[draw,angle radius=2mm]{right angle=S--H--A}
                ;
                \foreach \x/\g in {A/160,B/-145,C/-45,D/0,S/90,H/145}\fill[draw,fill=white] (\x) circle (1pt)+(\g:3mm) node[black]{$\x$};
            \end{tikzpicture}
        }
        \noindent
        Vậy $\cos(SB,AC)=\dfrac{\vec{SB}\cdot\vec{AC}}{SB\cdot AC} =\dfrac{2a^2}{a\sqrt{7}\cdot a\sqrt{5}} =\dfrac{2}{\sqrt{35}}$.
    }
\end{ex}
\begin{ex}%[1H3B2-3]
    [Chuyên Thái Bình - 2018]%Câu 18.
    Cho hình chóp tứ giác đều $S.ABCD$ có đáy $ABCD$ là hình vuông, $E$ là điểm đối xứng của $D$ qua trung điểm $SA$. Gọi $M$, $N$ lần lượt là trung điểm của $AE$ và $BC$. Góc giữa hai đường thẳng $MN$ và $BD$ bằng
    \choice
    {\True $90^{\circ}$}
    {$60^{\circ}$}
    {$45^{\circ}$}
    {$75^{\circ}$}
    \loigiai{
        \begin{center}
            \begin{tikzpicture}[scale=1, font=\footnotesize, line join=round, line cap=round, >=stealth]
                \def\bc{4} % cạnh BC
                \def\ba{2} % cạnh BA
                \def\h{4} % đường cao
                \def\gocB{30} % góc B của đáy
                \path
                (0,0) coordinate (B)
                (\gocB:\ba) coordinate (A)
                (\bc,0) coordinate (C)
                ($(C)-(B)+(A)$) coordinate (D)
                ($(A)!.5!(C)$) coordinate (O)
                ($(O)+(90:\h)$) coordinate (S)
                ($(S)!.5!(A)$) coordinate (I)
                ($(I)!-1!(D)$) coordinate (E)
                ($(B)!.5!(C)$) coordinate (N)
                ($(A)!.5!(E)$) coordinate (M)
                \foreach \x/\y/\z/\t/\l in {E/D/S/B/ed,M/I/S/B/mi,M/N/S/B/mn,E/A/S/B/ea}
                {
                    (intersection of {\x}--{\y} and {\z}--{\t}) coordinate (\l)
                };
                \draw (B)--(C)--(D)--(S)--cycle (S)--(C)
                (ed)--(E)--(ea)
                (mi)--(M)--(mn);
                \draw[dashed] (C)--(A)--(D)--(B) (O)--(S)--(A)--(B)
                (D)--(ed)
                (C)--(I)--(mi)
                (N)--(mn)
                (A)--(ea);
                \foreach \x/\g in {A/30,B/-135,C/-45,D/0,O/-120,S/90,I/10,E/90,M/-150,N/-90}\fill[red] (\x) circle (1pt)+(\g:3mm) node[black]{$ \x $};
            \end{tikzpicture}
        \end{center}
        Gọi $I$ là trung điểm $SA$ thì $IMNC$ là hình bình hành nên $MN\parallel IC$.\\
        Ta có $BD\perp(SAC)\Rightarrow BD\perp IC$ mà $MN\parallel IC\Rightarrow BD\perp MN$ nên góc giữa hai đường thẳng $MN$ và $BD$ bằng $90^{\circ}$.\\
        \textbf{Cách khác:} có thể dùng hệ trục tọa độ của lớp 12, tính tích vô hướng $\vec{BD}\cdot\vec{MN}=0$.
    }
\end{ex}
\begin{ex}%[1H3B2-3]
    [Chuyên Thái Bình - 2018]%Câu 19.
    Cho hình chóp đều $S.ABCD$ có tất cả các cạnh đều bằng $a$. Gọi $M$, $N$ lần lượt là trung điểm của $AD$ và $SD$. Số đo của góc giữa hai đường thẳng $MN$ và $SC$ là
    \choice
    {$45^{\circ}$}
    {$60^{\circ}$}
    {$30^{\circ}$}
    {\True $90^{\circ}$}
    \loigiai{
        \immini
        {
            Ta có $MN \parallel SA$ ($MN$ là đường trung bình của $\triangle SAD$).\\
            Suy ra $(MN,SC)=(SA,SC)$.\\
            Trong $\triangle SAC$, có $SA=SC=a$ và $AC=a\sqrt{2}$\\
            Nên $\triangle SAC$ vuông cân tại $S$.\\
            Vậy $(MN,SC)=(SA,SC)=\widehat{ASC}=90^\circ$.
        }
        {
            \begin{tikzpicture}[scale=1, font=\footnotesize, line join=round, line cap=round, >=stealth]
                \def\bc{4} % cạnh BC
                \def\ba{2.2} % cạnh BA
                \def\h{4} % đường cao
                \def\gocB{30} % góc B của đáy
                \path
                (0,0) coordinate (B)
                (\gocB:\ba) coordinate (A)
                (\bc,0) coordinate (C)
                ($(C)-(B)+(A)$) coordinate (D)
                ($(A)!.5!(C)$) coordinate (O)
                ($(O)+(90:\h)$) coordinate (S)
                ($(D)!.5!(A)$) coordinate (M)
                ($(D)!.5!(S)$) coordinate (N);
                \draw (B)--(C)--(D)--(S)--cycle (S)--(C);
                \draw[dashed] (C)--(A)--(D)--(B) (O)--(S)--(A)--(B) (M)--(N);
                \foreach \x/\g in {A/120,B/-135,C/-45,D/0,O/-120,S/90,M/-40,N/60}\fill[red] (\x) circle (1pt)+(\g:3mm) node[black]{$ \x $};
            \end{tikzpicture}
        }
    }
\end{ex}
\begin{ex}%[1H3K2-3]
    [Sở Quảng Nam - 2018]%Câu 20.
    Cho hình lăng trụ $ABC.A'B'C'$ có đáy $ABC$ là tam giác vuông tại $A$, $AB=a$, $AC=a\sqrt{3}$. Hình chiếu vuông góc của $A'$ lên mặt phẳng $(ABC)$ là trung điểm $H$ của $BC$, $A'H=a\sqrt{3}$. Gọi $\varphi$ là góc giữa hai đường thẳng $A'B$ và $B'C$. Tính $\cos\varphi$.
    \choice
    {$\cos\varphi=\dfrac{1}{2}$}
    {\True $\cos\varphi=\dfrac{\sqrt{6}}{8}$}
    {$\cos\varphi=\dfrac{\sqrt{6}}{4}$}
    {$\cos\varphi=\dfrac{\sqrt{3}}{2}$}
    \loigiai{
        \begin{center}
            \begin{tikzpicture}[scale=1, font=\footnotesize, line join=round, line cap=round, >=stealth]
                \def\ac{5} % cạnh AC
                \def\ab{2.2} % cạnh AB
                \def\h{4} % chiều cao
                \def\gocA{30} % góc A của đáy
                \path
                (0,0) coordinate (A)
                (\ac,0) coordinate (C)
                (-\gocA:\ab) coordinate (B)
                ($(B)!.5!(C)$) coordinate (H)
                ($(H)+(90:\h)$) coordinate (A')
                ($(B)-(A)+(A')$) coordinate (B')
                ($(C)-(A)+(A')$) coordinate (C')
                ($(A)!.5!(C)$) coordinate (E)
                ($(B)!-1!(A)$) coordinate (D)
                ($(D)-(B)+(H)$) coordinate (K)
                (intersection of E--K and D--C) coordinate (ek);
                \draw (B)--(A')--(A)--(B) (C)--(C')--(A')--(B')--(C') (D)--(B)--(B')--(D)--(C)--(B')--(K)
                (C)--(K)--(ek);
                \draw[dashed] (A)--(C) (A')--(H)--(A) (B)--(C)
                (E)--(ek);
                \foreach \x/\g in {A/180,B/-90,C/130,A'/180,C'/0,B'/-40,D/-90,K/-40,H/-90,E/50}\fill[red] (\x) circle (1pt)+(\g:3mm) node[black]{$ \x $};
            \end{tikzpicture}
        \end{center}
        Gọi $E$ là trung điểm của $AC$; $D$ và $K$ là các điểm thỏa $\vec{BD}=\vec{HK}=\vec{A'B'}$.\\
        Ta có $B'K\perp(ABC)$ và $B'D\parallel A'B\Rightarrow(A'B,B'C)=(B'D,B'C) =\widehat{DB'C}$.\\
        Ta tính được $BC=2a\Rightarrow BH=a$; $B'D=A'B=\sqrt{(a\sqrt{3})^2+a^2}=2a$.\\
        $CD=\sqrt{AC^2+AD^2}=a\sqrt{7}$; $CK=\sqrt{CE^2+EK^2}=a\sqrt{3}$.\\
        $B'C=\sqrt{B'K^2+CK^2}=\sqrt{3a^2+3a^2}=a\sqrt{6}$.\\
        $\cos\widehat{CB'D}=\dfrac{B'D^2+B'C^2-CD^2}{2\cdot B'D\cdot B'C} =\dfrac{4a^2+6a^2-7a^2}{2\cdot 2a\cdot a\sqrt{6}}=\dfrac{\sqrt{6}}{8}$.
    }
\end{ex}
\begin{ex}%[1H3B2-3]
    [Sở Yên Bái - 2018]%Câu 21.
    Cho tứ diện đều $ABCD$, $M$ là trung điểm của cạnh $BC$. Tính giá trị của $\cos(AB,DM)$.
    \choice
    {$\dfrac{\sqrt{3}}{2}$}
    {\True $\dfrac{\sqrt{3}}{6}$}
    {$\dfrac{1}{2}$}
    {$\dfrac{\sqrt{2}}{2}$}
    \loigiai{
        \immini
        {
            Giả sử cạnh của tứ diện đều bằng $a$.\\
            Gọi $N$ là trung điểm của $AC$.\\
            Khi đó $\left(AB,DM\right)=\left(MN,DM\right)$.\\
            Ta có $MN=\dfrac{a}{2}$, $DM=DN=\dfrac{a\sqrt{3}}{2}$.\\
            $\cos\widehat{NMD}=\dfrac{MN^2+MD^2-ND^2}{2\cdot MN\cdot MD}=\dfrac{\dfrac{a^2}{4}}{2\cdot\dfrac{a}{2}\cdot\dfrac{a\sqrt{3}}{2}}=\dfrac{\sqrt{3}}{6}$.\\
            Vậy $\cos(AB,DM)=\dfrac{\sqrt{3}}{6}$.
        }
        {
            \begin{tikzpicture}[scale=1, font=\footnotesize, line join=round, line cap=round, >=stealth]
                \def\ac{4} % cạnh AC
                \def\ab{2} % cạnh AB
                \def\h{4} % chiều cao
                \def\gocA{50} % góc A của đáy
                \path
                (0,0) coordinate (D)
                (\ac,0) coordinate (C)
                (-\gocA:\ab) coordinate (B)
                ($(B)!.5!(C)$) coordinate (M)
                --(D) coordinate[pos=1/3](G)
                ($(G)+(90:\h)$) coordinate (A)
                ($(A)!.5!(C)$) coordinate (N)
                ;
                \draw (D)--(B)--(C)--(A)--cycle (A)--(B) (M)--(N);
                \draw[dashed] (N)--(D)--(M) (A)--(G) (C)--(D);
                \foreach \x/\g in {A/90,B/-90,C/0,D/180,M/-35,N/20}\fill[draw,fill=white] (\x) circle (1pt)+(\g:3mm) node[black]{$\x$};
            \end{tikzpicture}
        }
    }
\end{ex}
\begin{ex}%[1H3K2-4]
    [Sở Nam Định - 2018]%Câu 22.
    Cho hình lăng trụ $ABC.A'B'C'$ có đáy $ABC$ là tam giác đều cạnh $a$, tam giác $A'BC$ đều nằm trong mặt phẳng vuông góc với $(ABC)$. $M$ là trung điểm cạnh $CC'$. Tính cosin góc $\alpha$ giữa hai đường thẳng $AA'$ và $BM$.
    \choice
    {$\cos\alpha=\dfrac{2\sqrt{22}}{11}$}
    {\True $\cos\alpha=\dfrac{\sqrt{33}}{11}$}
    {$\cos\alpha=\dfrac{\sqrt{11}}{11}$}
    {$\cos\alpha=\dfrac{\sqrt{22}}{11}$}
    \loigiai{
        \immini
        {
            Ta có $AH=A'H=\dfrac{a\sqrt{3}}{2}$ và $AH\perp BC$, $A'H\perp BC$.\\
            Suy ra $BC\perp(AA'H)\Rightarrow BC\perp AA'$ hay $BC\perp BB'$.\\
            Do đó $BCC'B'$ là hình chữ nhật.\\
            Khi đó $CC'=AA'=\dfrac{a\sqrt{3}}{2}\cdot\sqrt{2}=\dfrac{a\sqrt{6}}{2}$.\\
            $ BM=\sqrt{a^2+\dfrac{a^2\cdot 6}{16}}=a\dfrac{\sqrt{22}}{4}$.\\
            Xét $\vec{AA'}\cdot\vec{BM}=\vec{AA'}\cdot\left(\vec{BC}+\vec{CM}\right)=\dfrac{3a^2}{4}$.\\
            Suy ra $\cos(AA',BM)=\dfrac{\left|\dfrac{3a^2}{4}\right|}{\dfrac{a\sqrt{6}}{2}\cdot\dfrac{a\sqrt{22}}{4}} =\dfrac{\sqrt{33}}{11}$.
        }
        {
            \begin{tikzpicture}[scale=1, font=\footnotesize, line join=round, line cap=round, >=stealth]
                \def\ac{3} % cạnh AC
                \def\ab{1.6} % cạnh AB
                \def\h{3.5} % chiều cao
                \def\gocA{60} % góc A của đáy
                \path
                (0,0) coordinate (A)
                (\ac,0) coordinate (C)
                (-\gocA:\ab) coordinate (B)
                ($(B)!.5!(C)$) coordinate (H)
                ($(H)+(90:\h)$) coordinate (A')
                ($(B)-(A)+(A')$) coordinate (B')
                ($(C)-(A)+(A')$) coordinate (C')
                ($(C)!.5!(C')$) coordinate (M);
                \draw (A')--(A)--(B)--(C)--(C')--(A')--(B')--(C') (M)--(B)--(B');
                \draw[dashed] (A)--(C) (A)--(H)--(A')
                (A')--(C);
                \foreach \x/\g in {A/180,B/-90,C/0,A'/180,C'/0,B'/-40,H/-50,M/-10}\fill[red] (\x) circle (1pt)+(\g:3mm) node[black]{$ \x $};
            \end{tikzpicture}
        }
    }
\end{ex}
\begin{ex}[Sở Hà Tĩnh - 2018]%Câu 23.%[1H3B2-3]
    Cho hình lăng trụ tam giác đều $ABC.MNP$ có tất cả các cạnh bằng nhau. Gọi $I$ là trung điểm cạnh $AC$. Côsin của góc giữa hai đường thẳng $NC$ và $BI$ bằng
    \choice
    {\True $\dfrac{\sqrt{6}}{4}$}
    {$\dfrac{\sqrt{15}}{5}$}
    {$\dfrac{\sqrt{6}}{2}$}
    {$\dfrac{\sqrt{10}}{4}$}
    \loigiai{
        \immini
        {
            Giả sử các cạnh của lăng trụ bằng $a$.\\
            Gọi $K$ là trung điểm của $MP$.\\
            Suy ra $ BI\parallel NK\Rightarrow(NC,BI)=(NC,NK)$.\\
            Vì $ABC.MNP$ là lăng trụ tam giác đều $\Rightarrow CP\perp(MNP)$.\\
            $CK=\sqrt{CP^2+PK^2} =\dfrac{a\sqrt{5}}{2}$.\\
            $CN=\sqrt{CP^2+NP^2} =a\sqrt{2}$.\\
            $NK=\sqrt{NP^2-KP^2} =\dfrac{a\sqrt{3}}{2}$.\\
            Vậy $\cos\widehat{CNK}=\dfrac{NC^2+NK^2-CK^2}{2NC\cdot NK} =\dfrac{\sqrt{6}}{4}$.
        }
        {
            \begin{tikzpicture}[scale=1, font=\footnotesize, line join=round, line cap=round, >=stealth]
                \def\ac{5} % cạnh AC
                \def\ab{4} % cạnh AB
                \def\h{5} % chiều cao
                \def\gocA{30} % góc A của đáy
                \path
                (0,0) coordinate (A)
                (\ac,0) coordinate (C)
                (-\gocA:\ab) coordinate (B)
                ($(A)+(-90:\h)$) coordinate (M)
                ($(B)-(A)+(M)$) coordinate (N)
                ($(C)-(A)+(M)$) coordinate (P)
                ($(M)!.5!(P)$) coordinate (K)
                ($(A)!.5!(C)$) coordinate (I);
                \draw (M)--(N)--(P)--(M)--(A)--(B)--(C)--(P) (B)--(N)--(C)--(A) (B)--(I);
                \draw[dashed] (M)--(P) (C)--(K)--(N);
                \foreach \x/\g in {A/180,B/-150,C/0,M/180,N/-90,P/-40,I/90,K/120}\fill[red] (\x) circle (1pt)+(\g:3mm) node[black]{$ \x $};
            \end{tikzpicture}
        }
    }
\end{ex}
\begin{ex}[Chuyên Biên Hòa - Hà Nam - 2020]%Câu 24.%[1H3B2-3]
    Cho tứ diện đều $ABCD$, $M$ là trung điểm của cạnh $BC$. Khi đó $\cos(AB, DM)$ bằng
    \choice
    {$\dfrac{\sqrt{2}}{2}$}
    {\True $\dfrac{\sqrt{3}}{6}$}
    {$\dfrac{1}{2}$}
    {$\dfrac{\sqrt{3}}{2}$}
    \loigiai{
        \immini
        {
            Gọi $N$ là trung điểm của $AC$. Suy ra $MN\parallel AB$.\\
            Do đó $\cos(AB, DM)=\cos(MN, DM)$.\\
            Gọi $a$ là độ dài cạnh của tứ diện đều $ABCD$.\\
            Suy ra $MN=\dfrac{a}{2}$; $ND=MD=\dfrac{a\sqrt{3}}{2}$.\\
            Trong tam giác $MND$ ta có \[\cos\widehat{NMD}=\dfrac{MN^2+MD^2-ND^2}{2\cdot MN\cdot MD}=\dfrac{\sqrt{3}}{6}.\]
            Vậy $\cos(AB, DM)=\cos\widehat{NMD}=\dfrac{\sqrt{3}}{6}$.
        }
        {
            \begin{tikzpicture}[scale=1, font=\footnotesize, line join=round, line cap=round, >=stealth]
                \def\ac{4}
                \def\ab{2}
                \def\h{4}
                \def\gocA{50}
                \path
                (0,0) coordinate (D)
                (\ac,0) coordinate (C)
                (-\gocA:\ab) coordinate (B)
                ($(B)!.5!(C)$) coordinate (M)
                --(D) coordinate[pos=1/3](G)
                ($(G)+(90:\h)$) coordinate (A)
                ($(A)!.5!(C)$) coordinate (N)
                ;
                \draw (D)--(B)--(C)--(A)--cycle (A)--(B) (M)--(N);
                \draw[dashed] (N)--(D)--(M) (C)--(D);
                \foreach \x/\g in {A/90,B/-90,C/0,D/180,M/-35,N/20}\fill[draw,fill=white] (\x) circle (1pt)+(\g:3mm) node[black]{$\x$};
            \end{tikzpicture}
        }
    }
\end{ex}
\begin{ex}[ĐHQG Hà Nội - 2020]%Câu 25.%[2H3B4-1]
    Cho hình chóp $S.ABCD$ có đáy hình vuông. Cho tam giác $SAB$ vuông tại $S$ và góc $SBA$ bằng $30^{\circ}$. Mặt phẳng $(SAB)$ vuông góc mặt phẳng đáy. Gọi $M,N$ là trung điểm $AB,BC$. Tìm cosin góc tạo bởi hai đường thẳng $(SM,DN)$.
    \choice
    {$\dfrac{2}{\sqrt{5}}$}
    {\True $\dfrac{1}{\sqrt{5}}$}
    {$\dfrac{1}{\sqrt{3}}$}
    {$\dfrac{\sqrt{2}}{\sqrt{3}}$}
    \loigiai{
        Trong $(SAB)$, kẻ $SH\perp AB$ tại $H$. Ta có: $\heva{&(SAB)\perp(ABCD)\\&(SAB)\cap(ABCD)=AB\\&SH\subset(SAB),SH\perp AB}\Rightarrow SH\perp(ABCD)$.\\
        Kẻ tia $Az\parallel SH$ và chọn hệ trục tọa độ $Axyz$ như hình vẽ sau đây.
        \begin{center}
            \begin{tikzpicture}[line cap=round,line join=round, >=stealth,scale=1]
                \def \a{-2} \def \b{-1} \def \c{4} \def \h{4}
                \path (0,0)coordinate(A)
                +(\a,\b)coordinate(B)
                +(\c,0)coordinate(D)
                ($(B)+(D)-(A)$)coordinate(C)
                ($(A)!1/2!(B)$)coordinate(H)
                +(0,\h)coordinate(S)
                (A)--(B)--([turn]0:1)coordinate (y)node[below right]{$y$}
                (A)--(D)--([turn]0:1)coordinate (x)node[below]{$x$}
                ($(A)+(90:.1)$) coordinate (zz)
                (intersection of A--zz and S--D) coordinate (zzz)
                ($(B)!.5!(C)$) coordinate (N)
                ;
                \draw[->] (B)--(y);
                \draw[->] (D)--(x);
                \draw[->] (zzz)--++(90:1)node[above left]{$z$};
                \draw [dashed] (B)--(A)--(D)--(N) (A)--(S) (S)--(H) (A)--(zzz);
                \draw (S)--(B)--(C)--(D)--(S)--(C)
                pic[draw,angle radius=2mm]{right angle=S--H--A}
                ;
                \foreach \x/\g in {A/160,B/-65,C/-45,D/-60,S/90,H/145,N/-90}\fill[draw,fill=white] (\x) circle (1pt)+(\g:3mm) node[black]{$\x$};
            \end{tikzpicture}
        \end{center}
        Trong tam giác $SAB$ vuông tại $S$, $SB=AB\cdot\cos\widehat{SBA}=a\cdot\cos 30^{\circ}=\dfrac{a\sqrt{3}}{2}$.\\
        Trong tam giác $SBH$ vuông tại $H$, $BH=SB\cdot\cos\widehat{SBH}=\dfrac{3a}{4}$ và $SH=BH\cdot\sin\widehat{SBA}=\dfrac{a\sqrt{3}}{4}$.\\
        Ta có $AH=AB-BH=a-\dfrac{3a}{4}=\dfrac{a}{4}$. Suy ra $ H\left(0;\dfrac{a}{4};0\right)\Rightarrow S\left(0;\dfrac{a}{4};\dfrac{a\sqrt{3}}{4}\right)$.\\
        $M\left(0;\dfrac{a}{2};0\right)$, $D(a;0;0)$, $N\left(\dfrac{a}{2};a;0\right)$.\\
        Ta có $\vec{SM}=\left(0;\dfrac{a}{4};-\dfrac{a\sqrt{3}}{4}\right)$, $\vec{DN}=\left(-\dfrac{a}{2};a;0\right)$.
        \[\cos(SM,DN)=\dfrac{\left|\vec{SM}\cdot\vec{DN}\right|}{SN\cdot DN}=\dfrac{\dfrac{a^2}{4}}{\dfrac{a}{2}\cdot\dfrac{a\sqrt{5}}{2}}=\dfrac{1}{\sqrt{5}}.\]
    }
\end{ex}
\begin{ex}[Đề minh họa 2022]%Câu 26.%[1H3B2-3]
    \immini
    {
        Cho hình hộp $ABCD.A'B'C'D'$ có tất cả các cạnh bằng nhau (tham khảo hình bên).
        Góc giữa hai đường thẳng $A'C'$ và $BD$ bằng
        \choice
        {\True $90^{\circ}$}
        {$30^{\circ}$}
        {$45^{\circ}$}
        {$60^{\circ}$}
    }
    {
        \begin{tikzpicture}[line cap=round,line join=round, >=stealth,scale=.7]
            \def \a{-1.5} \def \b{-1}\def \c{4.5} \def \h{4}
            \path (.5,.5)coordinate(A)
            +(\a,\b)coordinate(B)
            +(\c,0)coordinate(D)
            ($(B)+(D)-(A)$)coordinate(C)
            +(0,\h) coordinate(C')
            ($(B)+(C')-(C)$)coordinate(B')
            ($(A)+(C')-(C)$)coordinate(A')
            ($(D)+(C')-(C)$)coordinate(D');
            \draw [dashed] (A)--(B)(D)--(A)--(A');
            \draw (B')--(B)--(C)(B')--(C')--(C)--(D)--(D')--(A')--(B')(C')--(D');
            \foreach \x/\g in {A/135,B/-135,C/-45,D/0,A'/135,B'/180,C'/-20,D'/0}\fill[red] (\x) circle (1pt)+(\g:3mm) node[black]{$\x$};
        \end{tikzpicture}
    }
    \loigiai{
        Ta có $A'C'\parallel AC$ nên $(A'C';BD)=(AC;BD)=90^{\circ}$.
    }
\end{ex}
\begin{dang}
	{Góc của đường thẳng với mặt phẳng}
\end{dang}
\begin{ex}[Đề Minh Hoạ 2021]%%[1H3B3-3]
    \immini{
        Cho hình hộp chữ nhật $ABCD.A’B’C’D’$ có $AB=AD=2$, $AA’=2\sqrt{2}$ (tham khảo hình vẽ bên). Góc giữa đường thẳng $CA’$ và mặt phẳng $(ABCD)$ bằng
        \choice
        {$30^{\circ}$}
        {\True $45^{\circ}$}
        {$60^{\circ}$}
        {$90^{\circ}$}
    }{
        \begin{tikzpicture}[scale=1]
            \path
            (0,0) coordinate (A)
            (-1,-0.8) coordinate (B)
            (1.5,-0.8) coordinate (C)
            ;
            \coordinate (D) at ($(A)+(C)-(B)$);
            \foreach \x / \y in {A'/A,B'/B, C'/C, D'/D}
            \coordinate (\x) at ($(\y)+(0,2)$);
            \draw (A') node [above] {$A'$} -- (B') node [left] {$B'$} -- (C') node [above] {$C'$} -- (D') node [above] {$D'$} -- (D) node [right] {$D$} -- (C)  node [below] {$C$}-- (B) node [below] {$B$} -- (B') 		(C) -- (C') (A') -- (D');
            \draw [dashed] (A') -- (A) node [left] {$A$} -- (D) (A) -- (B) (A') -- (C) -- (A);
            \foreach \x in {A,B,C,D,A',B',C',D'}
            \draw [fill=black] (\x) circle (0.4mm);
        \end{tikzpicture}
    }
    \loigiai{
        Vì $AA’\perp (ABCD)$ nên góc giữa đường thẳng $CA’$ và mặt phẳng $(ABCD)$ bằng góc $\widehat{A’CA}$.\\
        Ta có $AC=\sqrt{AB^2+BC^2}=2\sqrt{2}$.\\
        Khi đó ta có $\tan\widehat{A’CA}=\dfrac{AA’}{AC}=\dfrac{2\sqrt{2}}{2\sqrt{2}}=1$.\\
        Vậy số đo góc $\widehat{A’CA}=45^{\circ}$.
    }
\end{ex}
\begin{ex}%[1H3B3-3]
    [Đề Minh Hoạ 2020 lần 1]
    \immini{
        Cho hình chóp $S.ABCD$ có đáy là hình vuông cạnh $a\sqrt{3}$, $SA$ vuông góc với mặt phẳng đáy và $SA=2a$. Góc giữa $SC$ và mặt phẳng $(ABCD)$ bằng
        \choice
        {$45^{\circ}$}
        {$60^{\circ}$}
        {\True $30^{\circ}$}
        {$90^{\circ}$}
    }{
        \begin{tikzpicture}[scale=0.8]
            \path (0,0) coordinate (A) -- (4,0) coordinate (D) -- (-1.5,-1) coordinate (B) -- (0,3) coordinate (S);
            \coordinate (C) at ($(B)+(D)-(A)$);
            \draw (B) -- (C) --(D) -- (S) node [above] {$S$} -- (B) node [below] {$B$} (S) -- (C) node [below] {$C$};
            \draw [dashed] (A) -- (B) (S) -- (A) node [left] {$A$} -- (D) node [right] {$D$};
            \foreach \x in {S,A,B,C,D} \draw [fill=black] (\x) circle (0.4 mm);
        \end{tikzpicture}
    }
    \loigiai{
        Ta có $SA\perp (ABCD)$ nên $\left(\widehat{SC,(ABCD)}\right)=\widehat{SCA}$.\\
        $\tan\widehat{SCA}=\dfrac{SA}{AC}=\dfrac{a \cdot \sqrt{2}}{a \cdot \sqrt{3} \cdot \sqrt{2}}=\dfrac{1}{\sqrt{3}}\Rightarrow\widehat{SCA}=30^{\circ}$.
    }
\end{ex}

\begin{ex}%[1H3B3-3]
    [Đề Tham Khảo 2020 Lần 2]
    \immini{
        Cho hình chóp $S.ABC$ có $SA$ vuông góc với mặt phẳng $(ABCD)$, $SA=a\sqrt{2}$, tam giác $ABC$ vuông cân tại $B$ và $AC=2a$ (minh hoạ hình bên). Góc giữa đường thẳng $SB$ và mặt phẳng $(ABC)$ bằng
        \choice
        {$30^{\circ}$}
        {\True $45^{\circ}$}
        {$60^{\circ}$}
        {$90^{\circ}$}
    }{
        \begin{tikzpicture}[scale=0.8]
            \path (0,0) coordinate (A) -- (4,0) coordinate (C) -- (1.5,-1) coordinate (B) -- (0,3) coordinate (S);
            \draw (A) -- (B) -- (C) -- (S) -- (A) (S) node [above] {$S$} -- (B) node [below] {$B$};
            \draw [dashed] (A) node [left] {$A$} -- (C) node [right] {$C$};
            \foreach \x in {S,A,B,C} \draw [fill=black] (\x) circle (0.4 mm);
        \end{tikzpicture}
    }
    \loigiai{
        Ta có $\heva{&SB \cap (ABC)\\&SA\perp (ABC)} \Rightarrow AB$ là hình chiếu của $SB$ lên mặt phẳng $(ABC)$.\\
        Do đó $\left(\widehat{SB,(ABC)}\right)=\widehat{SBA}$.\\
        Do tam giác $ABC$ vuông cân tại $B$ suy ra $AB=\dfrac{AC}{\sqrt{2}}=a\sqrt{2}$.\\
        Xét tam giác $SAB$ vuông tại $A$, có $SA=AB=a\sqrt{2}$.\\
        Suy ra $\triangle SAB$ vuông cân tại $A$ và $\widehat{SBA}=45^{\circ}$.\\
        Vậy góc giữa đường thẳng $SB$ và mặt phẳng $ABC$ bằng $45^{\circ}$.
    }
\end{ex}

\begin{ex}%[1H3B3-3]
    [Mã 101 – 2020 Lần 1]
    \immini{
        Cho hình chóp $S.ABC$ có đáy $ABC$ là tam giác vuông tại $B$, $AB=a$, $BC=2a$, $SA$ vuông góc với mặt phẳng đáy và $SA=a\sqrt{15}$ (tham khảo hình bên). Góc giữa đường thẳng $SC$ và mặt phẳng đáy bằng
        \choice
        {$45^{\circ}$}
        {$30^{\circ}$}
        {\True $60^{\circ}$}
        {$90^{\circ}$}
    }{
        \begin{tikzpicture}[scale=0.8]
            \path (0,0) coordinate (A) -- (4,0) coordinate (C) -- (1.5,-1) coordinate (B) -- (0,3) coordinate (S);
            \draw (A) -- (B) -- (C) -- (S) -- (A) (S) node [above] {$S$} -- (B) node [below] {$B$};
            \draw [dashed] (A) node [left] {$A$} -- (C) node [right] {$C$};
            \foreach \x in {S,A,B,C} \draw [fill=black] (\x) circle (0.4 mm);
        \end{tikzpicture}
    }
    \loigiai{
        Do $SA$ vuông góc với mặt phẳng đáy nên $AC$ là hình chiếu vuông góc của $SC$ lên mặt phẳng đáy. Từ đó suy ra $\left(\widehat{SC,(ABC)}\right)=\left(\widehat{SC,AC}\right)=\widehat{SCA}$.\\
        Trong tam giác $ABC$ vuông tại $B$ ta có $AC=\sqrt{AB^2+BC^2}=\sqrt{a^2+4a^2}=a\sqrt{5}$.\\
        Trong tam giác $SAC$ vuông tại $A$ ta có $\tan\widehat{SCA}=\dfrac{SA}{AC}=\dfrac{a\sqrt{15}}{a\sqrt{5}}=\sqrt{3}\Rightarrow \widehat{SCA}=60^{\circ}$.\\
        Vậy $\left(\widehat{SC,(ABC)}\right)=60^{\circ}$.
    }
\end{ex}

\begin{ex}%[1H3B3-3]
    [Mã 102 – 2020 Lần 1]
    \immini{
        Cho hình chóp $S.ABC$ có đáy là tam giác vuông tại $B$, $AB=3a$, $BC=a\sqrt{3}$, $SA$ vuông góc với mặt phẳng đáy và $SA=2a$ (tham khảo hình vẽ). Góc giữa đường thẳng $SC$ và mặt phẳng đáy bằng
        \choice
        {$60^{\circ}$}
        {$45^{\circ}$}
        {\True $30^{\circ}$}
        {$90^{\circ}$}
    }{
        \begin{tikzpicture}[scale=0.8]
            \path (0,0) coordinate (A) -- (4,0) coordinate (C) -- (1.5,-1) coordinate (B) -- (0,3) coordinate (S);
            \draw (A) -- (B) -- (C) -- (S) -- (A) (S) node [above] {$S$} -- (B) node [below] {$B$};
            \draw [dashed] (A) node [left] {$A$} -- (C) node [right] {$C$};
            \foreach \x in {S,A,B,C} \draw [fill=black] (\x) circle (0.4 mm);
        \end{tikzpicture}
    }
    \loigiai{
        Vì $SA$ vuông góc với mặt phẳng đáy nên $AC$ là hình chiếu vuông góc của $SC$ lên đáy.\\
        Suy ra $\left(\widehat{SC,(ABC)}\right)=\left(\widehat{SC,AC}\right)=\widehat{SCA}$.
        \[\tan \widehat{SCA}=\dfrac{SA}{AC}=\dfrac{2a}{\sqrt{(3a)^2+(a\sqrt{3})^2}}=\dfrac{\sqrt{3}}{3}\Rightarrow\widehat{SCA}=30^{\circ}.\]
        Vậy $\left(\widehat{SC,(ABC)}\right)=30^{\circ}$.
    }
\end{ex}

\begin{ex}%[1H3B3-3]
    [Mã 103 – 2020 Lần 1]
    \immini{
        Cho hình chóp $S.ABC$ có đáy $ABC$ là tam giác vuông tại $B$, $AB=a$, $BC=3a$, $SA$ vuông góc với mặt phẳng đáy và $SA=a\sqrt{30}$ (tham khảo hình bên). Góc giữa đường thẳng $SC$ và mặt phẳng đáy bằng
        \choice
        {$45^{\circ}$}
        {$90^{\circ}$}
        {\True $60^{\circ}$}
        {$30^{\circ}$}
    }{
        \begin{tikzpicture}[scale=0.8]
            \path (0,0) coordinate (A) -- (4,0) coordinate (C) -- (1.5,-1) coordinate (B) -- (0,3) coordinate (S);
            \draw (A) -- (B) -- (C) -- (S) -- (A) (S) node [above] {$S$} -- (B) node [below] {$B$};
            \draw [dashed] (A) node [left] {$A$} -- (C) node [right] {$C$};
            \foreach \x in {S,A,B,C} \draw [fill=black] (\x) circle (0.4 mm);
        \end{tikzpicture}
    }
    \loigiai{
        Do $AC$ là hình chiếu vuông góc với SC trên mặt phẳng $(ABC)$ nên $\left(\widehat{SC,(ABC)}\right)=\widehat{SCA}$.\\
        Ta có $AC=\sqrt{AB^2+BC^2}=a\sqrt{10}$.\\
        Khi đó $\tan\widehat{SCA}=\dfrac{SA}{AC}=\dfrac{a\sqrt{30}}{a\sqrt{10}}=\sqrt{3}\Rightarrow \widehat{SCA}=60^{\circ}$.\\
        Vậy góc giữa đường thẳng $SC$ và mặt phẳng đáy bằng $60^{\circ}$.
    }
\end{ex}

\begin{ex}%[1H3B3-3]
    [Mã 104 – 2020 Lần 1]
    \immini{
        Cho hình chóp $S.ABC$ có đáy $ABC$ là tam giác vuông tại $B$, $AB=a$, $BC=a\sqrt{2}$, $SA$ vuông góc với mặt phẳng đáy và $SA=a$ (tham khảo hình bên). Góc giữa đường thẳng $SC$ và mặt phẳng đáy bằng
        \choice
        {$90^{\circ}$}
        {$45^{\circ}$}
        {$60^{\circ}$}
        {\True $30^{\circ}$}
    }{
        \begin{tikzpicture}[scale=0.8]
            \path (0,0) coordinate (A) -- (4,0) coordinate (C) -- (1.5,-1) coordinate (B) -- (0,3) coordinate (S);
            \draw (A) -- (B) -- (C) -- (S) -- (A) (S) node [above] {$S$} -- (B) node [below] {$B$};
            \draw [dashed] (A) node [left] {$A$} -- (C) node [right] {$C$};
            \foreach \x in {S,A,B,C} \draw [fill=black] (\x) circle (0.4 mm);
        \end{tikzpicture}
    }
    \loigiai{
        Ta có góc giữa $SC$ và đáy là $\widehat{SCA}$.\\
        Xét tam giác $SCA$ vuông tại $A$ có $AC=\sqrt{AB^2+BC^2}=a\sqrt{3}$.\\
        $\tan\widehat{SCA}=\dfrac{SA}{AC}=\dfrac{a}{a\sqrt{3}}\Rightarrow \widehat{SCA}=30^{\circ}$.
    }
\end{ex}

\begin{ex}%[1H3B3-3]
    [Mã 101 – 2020 Lần 2]
    \immini{
        Cho hình hộp chữ nhật $ABCD.A’B’C’D’$ có $AB=BC=a$, $A’A=a\sqrt{6}$ (tham khảo hình bên). Góc giữa đường thẳng $A’C$ và mặt phẳng $(ABCD)$ bằng
        \choice
        {\True $60^{\circ}$}
        {$90^{\circ}$}
        {$30^{\circ}$}
        {$45^{\circ}$}
    }{
        \begin{tikzpicture}[scale=1]
            \path
            (0,0) coordinate (A)
            (-1,-0.8) coordinate (B)
            (1.5,-0.8) coordinate (C)
            ;
            \coordinate (D) at ($(A)+(C)-(B)$);
            \foreach \x / \y in {A'/A,B'/B, C'/C, D'/D}
            \coordinate (\x) at ($(\y)+(0,2)$);
            \draw (A') node [above] {$A'$} -- (B') node [left] {$B'$} -- (C') node [above] {$C'$} -- (D') node [above] {$D'$} -- (D) node [right] {$D$} -- (C)  node [below] {$C$}-- (B) node [below] {$B$} -- (B') 		(C) -- (C') (A') -- (D');
            \draw [dashed] (A') -- (A) node [left] {$A$} -- (D) (A) -- (B) (A') -- (C) -- (A);
            \foreach \x in {A,B,C,D,A',B',C',D'}
            \draw [fill=black] (\x) circle (0.4mm);
            \node at ($(A)!0.35!(C)$) [below] {\tiny $a\sqrt{2}$};
            \node at ($(A)!0.35!(A')$) [above,rotate=90] {\tiny $a\sqrt{6}$};
        \end{tikzpicture}
    }
    \loigiai{
        \immini{
            Ta có góc giữa đường thẳng $A’C$ và mặt phẳng $(ABCD)$ bằng góc giữa $A’C$ và $AC$ là $\widehat{A’CA}$.\\
            Ta có $AC=\sqrt{AB^2+BC^2}=a\sqrt{2}$.\\
            Xét tam giác $A’CA$ có
            \[\tan\widehat{A’CA}=\dfrac{A’A}{AC}=\dfrac{a\sqrt{6}}{a\sqrt{2}}=\sqrt{3}\Rightarrow \widehat{A’CA}=60^{\circ}.\]
            Vậy góc $A’C$ và mặt phẳng $(ABCD)$ bằng $60^{\circ}$.
        }{
            \begin{tikzpicture}[scale=1]
                \path
                (0,0) coordinate (A)
                (-1,-0.8) coordinate (B)
                (1.5,-0.8) coordinate (C)
                ;
                \coordinate (D) at ($(A)+(C)-(B)$);
                \foreach \x / \y in {A'/A,B'/B, C'/C, D'/D}
                \coordinate (\x) at ($(\y)+(0,2)$);
                \draw (A') node [above] {$A'$} -- (B') node [left] {$B'$} -- (C') node [above] {$C'$} -- (D') node [above] {$D'$} -- (D) node [right] {$D$} -- (C)  node [below] {$C$}-- (B) node [below] {$B$} -- (B') 		(C) -- (C') (A') -- (D');
                \draw [dashed] (A') -- (A) node [left] {$A$} -- (D) (A) -- (B) (A') -- (C) -- (A);
                \foreach \x in {A,B,C,D,A',B',C',D'}
                \draw [fill=black] (\x) circle (0.4mm);
                \node at ($(A)!0.35!(C)$) [below] {\tiny $a\sqrt{2}$};
                \node at ($(A)!0.35!(A')$) [above,rotate=90] {\tiny $a\sqrt{6}$};
            \end{tikzpicture}
        }
    }
\end{ex}

\begin{ex}%[1H3B3-3]
    [Mã 102 – 2020 Lần 2]
    \immini{
        Cho hình hộp chữ nhật $ABCD.A’B’C’D’$ có $AB=a$, $AD=2a\sqrt{2}$, $A’A=a\sqrt{3}$ (tham khảo hình bên). Góc giữa đường thẳng $A’C$ và mặt phẳng $(ABCD)$ bằng
        \choice
        {\True $45^{\circ}$}
        {$90^{\circ}$}
        {$60^{\circ}$}
        {$30^{\circ}$}
    }{
        \begin{tikzpicture}[scale=1]
            \path
            (0,0) coordinate (A)
            (-1,-0.8) coordinate (B)
            (1.5,-0.8) coordinate (C)
            ;
            \coordinate (D) at ($(A)+(C)-(B)$);
            \foreach \x / \y in {A'/A,B'/B, C'/C, D'/D}
            \coordinate (\x) at ($(\y)+(0,2)$);
            \draw (A') node [above] {$A'$} -- (B') node [left] {$B'$} -- (C') node [above] {$C'$} -- (D') node [above] {$D'$} -- (D) node [right] {$D$} -- (C)  node [below] {$C$}-- (B) node [below] {$B$} -- (B') 		(C) -- (C') (A') -- (D');
            \draw [dashed] (A') -- (A) node [left] {$A$} -- (D) (A) -- (B) (A') -- (C);
            \foreach \x in {A,B,C,D,A',B',C',D'}
            \draw [fill=black] (\x) circle (0.4mm);
        \end{tikzpicture}
    }
    \loigiai{
        Ta thấy hình chiếu của $A’C$ xuống $(ABCD)$ là $AC$ do đó $\left(A’C,(ABCD)\right)=(A’C,AC)=\widehat{A’CA}$.\\
        Ta có $AC=\sqrt{AB^2+AD^2}=3a$.\\
        Xét tam giác $A’CA$ vuông tại $C$ ta có $\tan\widehat{A’CA}=\dfrac{A’A}{AC}=\dfrac{a\sqrt{3}}{3a}=\dfrac{\sqrt{3}}{3}$. Suy ra $\widehat{A’CA}=30^{\circ}$.\\
        Vậy $\left(\widehat{A’C,(ABCD)}\right)=\widehat{A’CA}=30^{\circ}$.
    }
\end{ex}

\begin{ex}%[1H3B3-3]
    [Mã 103 – 2020 Lần 2]
    \immini{
        Cho hình hộp chữ nhật $ABCD.A’B’C’D’$ có $AB=AA’=a$, $AD=a\sqrt{2}$ (tham khảo hình bên). Góc giữa đường thẳng $A’C$ và mặt phẳng $(ABCD)$ bằng
        \choice
        {\True $30^{\circ}$}
        {$45^{\circ}$}
        {$90^{\circ}$}
        {$60^{\circ}$}
    }{
        \begin{tikzpicture}[scale=1]
            \path
            (0,0) coordinate (A)
            (-1,-0.8) coordinate (B)
            (1.5,-0.8) coordinate (C)
            ;
            \coordinate (D) at ($(A)+(C)-(B)$);
            \foreach \x / \y in {A'/A,B'/B, C'/C, D'/D}
            \coordinate (\x) at ($(\y)+(0,2)$);
            \draw (A') node [above] {$A'$} -- (B') node [left] {$B'$} -- (C') node [above] {$C'$} -- (D') node [above] {$D'$} -- (D) node [right] {$D$} -- (C)  node [below] {$C$}-- (B) node [below] {$B$} -- (B') 		(C) -- (C') (A') -- (D');
            \draw [dashed] (A') -- (A) node [left] {$A$} -- (D) (A) -- (B) (A') -- (C);
            \foreach \x in {A,B,C,D,A',B',C',D'}
            \draw [fill=black] (\x) circle (0.4mm);
        \end{tikzpicture}
    }
    \loigiai{
        Vì $ABCD$ là hình chữ nhật, có $AB=a$, $AD=a\sqrt{2}$ nên
        \[AC=BD=\sqrt{AB^2+AD^2}=\sqrt{a^2+\left(a\sqrt{2}\right)^2}=a\sqrt{3}.\]
        Ta có $\left(\widehat{A’C,(ABCD)}\right)=\left(\widehat{A’C,CA}\right)=\widehat{A’CA}$.\\
        Do tam giác $A’AC$ vuông tại $A$ nên $\tan\widehat{A’AC}=\dfrac{A’A}{AC}=\dfrac{a}{a\sqrt{3}}=\dfrac{1}{\sqrt{3}}\Rightarrow \widehat{A’AC}=30^{\circ}$.\\
        Vậy góc giữa đường thẳng $A’C$ và mặt phẳng $(ABCD)$ bằng $30^{\circ}$.
    }
\end{ex}

\begin{ex}%[1H3B3-3]
    [Mã 104 – 2020 Lần 2]
    \immini{
        Cho hình hộp chữ nhật $ABCD.A’B’C’D’$ có $AB=a$, $AD=a\sqrt{3}$, $AA’=2a\sqrt{3}$ (tham khảo hình bên). Góc giữa đường thẳng $A’C$ và mặt phẳng $(ABCD)$ bằng
        \choice
        {$45^{\circ}$}
        {$30^{\circ}$}
        {\True $60^{\circ}$}
        {$90^{\circ}$}
    }{
        \begin{tikzpicture}[scale=1]
            \path
            (0,0) coordinate (A)
            (-1,-0.8) coordinate (B)
            (1.5,-0.8) coordinate (C)
            ;
            \coordinate (D) at ($(A)+(C)-(B)$);
            \foreach \x / \y in {A'/A,B'/B, C'/C, D'/D}
            \coordinate (\x) at ($(\y)+(0,2)$);
            \draw (A') node [above] {$A'$} -- (B') node [left] {$B'$} -- (C') node [above] {$C'$} -- (D') node [above] {$D'$} -- (D) node [right] {$D$} -- (C)  node [below] {$C$}-- (B) node [below] {$B$} -- (B') 		(C) -- (C') (A') -- (D');
            \draw [dashed] (A') -- (A) node [left] {$A$} -- (D) (A) -- (B);
            \foreach \x in {A,B,C,D,A',B',C',D'}
            \draw [fill=black] (\x) circle (0.4mm);
        \end{tikzpicture}
    }
    \loigiai{
        \immini{
            Do $AA’\perp (ABCD)$ nên $AC$ là hình chiếu của $A’C$ lên mặt phẳng $(ABCD)$, suy ra góc giữa đường thẳng $A’C$ và mặt phẳng $(ABCD)$ bằng $\widehat{A’CA}$.\\
            $\tan \widehat{A’CA}=\dfrac{A’A}{\sqrt{AB^2+AD^2}}=\dfrac{2a\sqrt{3}}{\sqrt{a^2+\left(a\sqrt{3}\right)^2}}=\sqrt{3}$.\\
            Suy ra $\widehat{A’CA}=60^{\circ}$.\\
            Vậy góc giữa đường thẳng $A’C$ và mặt phẳng $(ABCD)$ bằng $\widehat{A’CA}=60^{\circ}$.
        }{
            \begin{tikzpicture}[scale=1]
                \path
                (0,0) coordinate (A)
                (-1,-0.8) coordinate (B)
                (1.5,-0.8) coordinate (C)
                ;
                \coordinate (D) at ($(A)+(C)-(B)$);
                \foreach \x / \y in {A'/A,B'/B, C'/C, D'/D}
                \coordinate (\x) at ($(\y)+(0,2)$);
                \draw (A') node [above] {$A'$} -- (B') node [left] {$B'$} -- (C') node [above] {$C'$} -- (D') node [above] {$D'$} -- (D) node [right] {$D$} -- (C)  node [below] {$C$}-- (B) node [below] {$B$} -- (B') 		(C) -- (C') (A') -- (D');
                \draw [dashed] (A') -- (A) node [left] {$A$} -- (D) (A) -- (B) (A') -- (C) -- (A);
                \foreach \x in {A,B,C,D,A',B',C',D'}
                \draw [fill=black] (\x) circle (0.4mm);
            \end{tikzpicture}
        }
    }
\end{ex}

\begin{ex}%[1H3B3-3]
    [Mã 103 – 2018]
    \immini{
        Cho hình chóp $S.ABC$ có đáy là tam giác vuông tại $C$, $AC=a$, $BC=a\sqrt{2}$, $SA$ vuông góc với mặt phẳng đáy và $SA=a$. Góc giữa đường thẳng $SB$ và mặt phẳng đáy bằng
        \choice
        {$60^{\circ}$}
        {$90^{\circ}$}
        {\True $30^{\circ}$}
        {$45^{\circ}$}
    }{
        \begin{tikzpicture}[scale=0.8]
            \path (0,0) coordinate (A) -- (4,0) coordinate (C) -- (1.5,-1) coordinate (B) -- (0,3) coordinate (S);
            \draw (A) -- (B) -- (C) -- (S) -- (A) (S) node [above] {$S$} -- (B) node [below] {$B$};
            \draw [dashed] (A) node [left] {$A$} -- (C) node [right] {$C$};
            \foreach \x in {S,A,B,C} \draw [fill=black] (\x) circle (0.4 mm);
            \node at ($(A)!0.5!(B)$) [below] {\tiny $a\sqrt{3}$};
            \node at ($(A)!0.5!(C)$) [above] {\tiny $a$};
            \node at ($(C)!0.5!(B)$) [below] {\tiny $a\sqrt{2}$};
        \end{tikzpicture}
    }
    \loigiai{
        Có $SA\perp (ABC)$ nên $AB$ là hình chiếu $SA$ trên mặt phẳng $(ABC)$.\\
        Suy ra $\left(\widehat{SB,(ABC)}\right)=(\widehat{SB,AB})=\widehat{SBA}$.\\
        Mặt khác có $\triangle ABC$ vuông tại $C$ nên $AB=\sqrt{AC^2+BC^2}=a\sqrt{3}$.\\
        Khi đó $\tan\widehat{SBA}=\dfrac{SA}{AB}=\dfrac{1}{\sqrt{3}}$ nên $\left(\widehat{SB,(ABC)}\right)=30^{\circ}$.
    }
\end{ex}

\begin{ex}%[1H3B3-3]
    [Mã 102 – 2019]
    \immini{
        Cho hình chóp $S.ABC$ có $SA$ vuông góc với mặt phẳng $(ABC)$, $SA=2a$, tam giác $ABC$ vuông tại $B$, $AB=a$ và $BC=a\sqrt{3}$ (minh hoạ như hình bên). Góc giữa đường thẳng $SC$ và mặt phẳng $(ABC)$ bằng
        \choice
        {$30^{\circ}$}
        {$60^{\circ}$}
        {\True $45^{\circ}$}
        {$90^{\circ}$}
    }{
        \begin{tikzpicture}[scale=0.8]
            \path (0,0) coordinate (A) -- (4,0) coordinate (C) -- (1.5,-1) coordinate (B) -- (0,3) coordinate (S);
            \draw (A) -- (B) -- (C) -- (S) -- (A) (S) node [above] {$S$} -- (B) node [below] {$B$};
            \draw [dashed] (A) node [left] {$A$} -- (C) node [right] {$C$};
            \foreach \x in {S,A,B,C} \draw [fill=black] (\x) circle (0.4 mm);
        \end{tikzpicture}
    }
    \loigiai{
        Vì $SA\perp (ABC)$ suy ra góc giữa đường thẳng $SC$ và mặt phẳng $(ABC)$ bằng $\widehat{SCA}$.\\
        Mà $\tan\widehat{SCA}=\dfrac{SA}{AC}=\dfrac{2a}{\sqrt{a^2+3a^2}}=1$.\\
        Vậy $\widehat{SCA}=45^{\circ}$
    }
\end{ex}
% Câu 14
\begin{ex}%[1H3B3-3]
    [THPT Phan Đình Phùng – Quảng Bình - 2021]
    \immini{
        Cho hình chóp $S.ABC$ có $SA$ vuông góc với mặt phẳng $(ABC)$, $SA=a\sqrt{2}$, tam giác $ABC$ vuông tại $A$ và $AC=a$, $\sin B=\dfrac{1}{\sqrt{3}}$ (minh hoạ như hình bên). Góc giữa đường thẳng $SB$ và mặt phẳng $(ABC)$ bằng
        \choice
        {$90^{\circ}$}
        {$30^{\circ}$}
        {\True $45^{\circ}$}
        {$60^{\circ}$}
    }{
        \begin{tikzpicture}[scale=1]
            \path (0,0) coordinate (A) -- (4,0) coordinate (C) -- (1.5,-1) coordinate (B) -- (0,3) coordinate (S);
            \draw (A) -- (B) -- (C) -- (S) -- (A) (S) node [above] {$S$} -- (B) node [below] {$B$};
            \draw [dashed] (A) node [left] {$A$} -- (C) node [right] {$C$};
            \foreach \x in {S,A,B,C} \draw [fill=black] (\x) circle (0.4 mm);
        \end{tikzpicture}
    }
    \loigiai{
        Ta có $SA\perp (ABC) \Rightarrow \left(\widehat{SB,(ABC)}\right)=\widehat{SBA}$.\\
        Do $\sin B=\sin \widehat{SBA}=\dfrac{1}{\sqrt{3}}$ nên ta có
        \[\Rightarrow \cos B=\cos \widehat{SBA}=\dfrac{\sqrt{6}}{3},\tan B=\tan \widehat{SBA}=\dfrac{AC}{AB}=\dfrac{1}{\sqrt{2}} \Rightarrow AB=a\sqrt{2}.\]
        Vậy tam giác $SAB$ vuông cân tại $A$.\\
        Suy ra $\left(\widehat{SB,(ABC)}\right)=\widehat{SBA}=45^{\circ}$.
    }
\end{ex}

\begin{ex}%[1H3B3-3]
    [THPT Hậu Lộc 4 - Thanh Hóa - 2021] Cho hình chóp $S.ABC$ có $SA\perp(ABC)$, $SA=a\sqrt{3}$, tam giác $ABC$ vuông tại $B$ có $AC=2a$, $BC=a$. Góc giữa đường thẳng $SB$ và mặt phẳng $(ABC)$ bằng
    \choice
    {$60^{\circ}$}
    {$90^{\circ}$}
    {$30^{\circ}$}
    {\True $45^{\circ}$}
    \loigiai{
        \immini{
            Vì $SA\perp(ABC)$ nên $AB$ là hình chiếu vuông góc của $SB$ lên $(ABC)$.\\
            Do đó $\left(\widehat{SB,(ABC)}\right)=\left(\widehat{SB,AB}\right)=\widehat{SBA}$.\\
            Tam giác $ABC$ vuông tại $B$ nên $AB=\sqrt{(2a)^2-a^2}=a\sqrt{3}$.\\
            Do đó $\triangle SAB$ vuông cân tại $A$ suy ra $\widehat{SBA}=45^{\circ}$.\\
            Vậy $\left(\widehat{SB,(ABC)}\right)=45^{\circ}$.}
        {
            \begin{tikzpicture}
                \def\a{4}
                \path 	(0:0) coordinate (A)
                ++(0:\a) coordinate (C)
                ++(-120:\a/2) coordinate (B)
                ($(A)+(90:\a)$) coordinate (S);
                \draw[dashed] 	(A)--(C);
                \draw (S)--(A)--(B) (B)--(C) (S)--(B) (S)--(C);
                \foreach \x/\g in {A/180,B/-90,C/0,S/90}
                \fill[black] 	(\x) circle (1pt)
                ($(\g:3mm)+(\x)$) node {$\x$};
                %Hình chóp S.ABC có SA vuông góc đáy
        \end{tikzpicture}}

    }
\end{ex}
\begin{ex}[Sở Lào Cai - 2021]%[1H3K3-3]%Câu 16.
    Cho hình lập phương $ABCD.A'B'C'D'$. Gọi $M$, $N$ lần lượt là trung điểm $AC$ và $B'C'$, $\alpha$ là góc giữa đường thẳng $MN$ và mặt phẳng $(A'B'C'D')$. Tính giá trị $\alpha$.
    \choice
    {$\sin\alpha=\dfrac{\sqrt{2}}{2}$}
    {\True $\sin\alpha=\dfrac{2\sqrt{5}}{5}$}
    {$\sin\alpha=\dfrac{1}{2}$}
    {$\sin\alpha=\dfrac{\sqrt{5}}{5}$}
    \loigiai{
        \begin{center}
            \begin{tikzpicture}[line cap=round,line join=round,>=triangle 45,thick]
                \draw (1,4)-- (4,4) -- (3,3) -- (0,3) -- (1,4);
                \draw [dashed] (0,0) -- (1,1) -- (1,4) (1,1) -- (4,1);
                \draw (0,0)-- (0,3) (3,3)-- (3,0) -- (0,0) (3,0)-- (4,1) -- (4,4) (1,4)-- (3,3);
                \draw [dashed] (2,3.5)-- (2,0.5) -- (3.5,0.5) -- (2,3.5);
                \draw (1,4) node[above] {$A$};
                \draw (4,4) node[above] {$B$};
                \draw (3,3) node[right] {$C$};
                \draw (0,3) node[left] {$D$};
                \draw (1,1) node[left] {$A'$};
                \draw (4,1) node[right] {$B'$};
                \draw (3,0) node [right] {$C'$};
                \draw (0,0) node [left] {$D'$};
                \draw (2,3.5) node[above] {$M$};
                \draw (3.5,0.5) node[right] {$N$};
                \draw (2,0.5) node[left] {$O'$};
            \end{tikzpicture}
        \end{center}
        Giả sử cạnh hình lập phương là $a$.\\
        Gọi $O'$ là tâm của hình vuông $A'B'C'D'$. Suy ra $O'N$ là hình chiếu của $MN$ lên $(A'B'C'D')$. Do đó góc giữa $MN$ và $(A'B'C'D')$ là góc giữa $MN$ và $O'N$.\\
        Tam giác $O'MN$ vuông tại $O$ có $O'N=\dfrac{1}{2}a$, $O'M=a$ nên\\ $\sin\widehat{O'NM}=\dfrac{O'M}{MN}=\dfrac{O'M}{\sqrt{O'N^2+O'M^2}}=\dfrac{a}{\sqrt{\dfrac{a^2}{4}+a^2}}=\dfrac{2\sqrt{5}}{5}$.
    }
\end{ex}
\begin{ex}[Sở Tuyên Quang - 2021]%[1H3K3-3]%Câu 17.
    Cho hình chóp $S.ABCD$ có đáy $ABCD$ là hình vuông cạnh $a$, đường thẳng $SA$ vuông góc với mặt phẳng đáy và $SA=2a$. Góc giữa đường thẳng $SC$ và mặt phẳng $(ABCD)$ là $\alpha$. Khi đó $\tan\alpha$ bằng
    \choice
    {$2\sqrt{2}$}
    {$2$}
    {\True $\sqrt{2}$}
    {$\dfrac{2}{\sqrt{3}}$}
    \loigiai{
        \begin{center}
            \begin{tikzpicture}[line cap=round,line join=round,>=triangle 45,thick]
                \draw (1,4) -- (4,) (1,4) -- (3,0) (1,4) -- (0,0);
                \draw [dashed] (0,0) -- (1,1) -- (1,4) (3,0) -- (1,1) -- (4,1);
                \draw (0,0)-- (3,0) -- (4,1);
                \draw [shift={(3,0)}] (0,0) -- (116.565:0.6) arc (116.565:153.435:0.6);
                \draw (1,4) node[above] {$S$};
                \draw (1,1) node[left] {$A$};
                \draw (4,1) node[right] {$B$};
                \draw (3,0) node [right] {$C$};
                \draw (0,0) node [left] {$D$};
                \draw (2.35,0.7) node {$\alpha$};
            \end{tikzpicture}
        \end{center}
        Ta có $\widehat{\left(SC;(ABCD)\right)}=\widehat{SCA}=\alpha$.\\
        Xét tam giác $SAC$ vuông tại $A$ có: $\tan\widehat{SCA}=\dfrac{SA}{AC} =\dfrac{2a}{a\sqrt{2}} =\sqrt{2}$ \\
        $ \Rightarrow\tan\alpha=\sqrt{2}$.
    }
\end{ex}
\begin{ex}[Chuyên Thoại Ngọc Hầu - An Giang - 2021]%[1H3K3-3]%Câu 18.
    Cho hình chóp $S.ABCD$ có đáy là hình vuông $ABCD$ cạnh bằng $3a$, $SA$ vuông góc với mặt đáy $(ABCD)$, $SB=5a$. Tính $\sin$ của góc giữa cạnh $SC$ và mặt đáy $(ABCD)$.
    \choice
    {$\dfrac{3\sqrt{2}}{4}$}
    {\True $\dfrac{2\sqrt{34}}{17}$}
    {$\dfrac{4}{5}$}
    {$\dfrac{2\sqrt{2}}{3}$}
    \loigiai{
        \begin{center}
            \begin{tikzpicture}[line cap=round,line join=round,>=triangle 45,thick]
                \draw (1,4) -- (4,) (1,4) -- (3,0) (1,4) -- (0,0);
                \draw [dashed] (0,0) -- (1,1) -- (1,4) (3,0) -- (1,1) -- (4,1);
                \draw (0,0)-- (3,0) -- (4,1);
                \draw [shift={(3,0)}] (0,0) -- (116.565:0.6) arc (116.565:153.435:0.6);
                \draw (1,4) node[above] {$S$};
                \draw (1,1) node[left] {$A$};
                \draw (4,1) node[right] {$B$};
                \draw (3,0) node [right] {$C$};
                \draw (0,0) node [left] {$D$};
                \draw (2.35,0.7) node {$\alpha$};
            \end{tikzpicture}
        \end{center}
        Do $SA\perp(ABCD)$ nên $AC$ là hình chiếu của $SC$ lên mặt phẳng $(ABCD)$. Do đó góc giữa cạnh $SC$ và mặt đáy $(ABCD)$ là $\widehat{SCA}$.\\
        Xét tam giác $ABC$ có $AC=\sqrt{AB^2+BC^2}=3a\sqrt{2}$.\\
        Xét tam giác $SAB$ có $SA=\sqrt{SB^2-AB^2}=4a$.\\
        Xét tam giác $SAC$ có $SC=\sqrt{SA^2+AC^2}=a\sqrt{34}$.\\
        Xét tam giác $SAC$ có $\sin\widehat{SCA}=\dfrac{SA}{SC}=\dfrac{4a}{a\sqrt{34}}=\dfrac{2\sqrt{34}}{17}$.\\
        Vậy $\sin$ của góc giữa cạnh $SC$ và mặt đáy $(ABCD)$ bằng $\dfrac{2\sqrt{34}}{17}$.
    }
\end{ex}
\begin{ex}[Chuyên Lê Hồng Phong - TPHCM - 2021]%[1H3K3-3]%Câu 19.
    Cho hình chóp $S.ABC$ có đáy $ABC$ là tam giác đều cạnh $a$, cạnh $SA$ vuông góc với mặt phẳng đáy và $SA=2a$, gọi $M$ là trung điểm của $SC$. Tính cosin của góc $\alpha$ là góc giữa đường thẳng $BM$ và $(ABC)$.
    \choice
    {$\cos\alpha=\dfrac{\sqrt{7}}{14}$}
    {$\cos\alpha=\dfrac{2\sqrt{7}}{7}$}
    {\True $\cos\alpha=\dfrac{\sqrt{21}}{7}$}
    {$\cos\alpha=\dfrac{\sqrt{5}}{7}$}
    \loigiai{
        \begin{center}
            \begin{tikzpicture}[line cap=round,line join=round,>=triangle 45,thick]
                \draw (1,4) -- (4,) (1,4) -- (3,0) (1,4) -- (0,0);
                \draw [dashed] (0,0) -- (1,1) -- (1,4) (3,0) -- (1,1) -- (4,1) -- (2,0.5) -- (2,2);
                \draw (0,0)-- (3,0) -- (4,1) -- (2,2);
                \draw [shift={(4,1)}] (0,0) -- (153.435:0.556) arc (153.435:194.036:0.556);
                \draw (1,4) node[above] {$S$};
                \draw (1,1.1) node[left] {$A$};
                \draw (4,1) node[right] {$B$};
                \draw (3,0) node [right] {$C$};
                \draw (0,0) node [left] {$D$};
                \draw (2,2) node [left] {$M$};
                \draw (2.1,0.6) node [below left] {$H$};
                \draw (3.2,1.2) node {$\alpha$};
            \end{tikzpicture}
        \end{center}
        Trong mặt phẳng $(SAC)$, dựng $MH\perp AC$ tại $H$.\\
        Do $SA\perp(ABC)\Rightarrow SA\perp AC\subset(ABC)\Rightarrow SA\parallel MH$.\\
        Khi đó: $MH\perp(ABC)$.\\
        Suy ra: $\left(\widehat{BM,(ABC)}\right)=\left(\widehat{BM,BH}\right)=\widehat{MBH}$. Khi đó:
        \[\cos\alpha=\dfrac{BH}{BM}=\dfrac{\dfrac{a\sqrt{3}}{2}}{\sqrt{\left(\dfrac{a\sqrt{3}}{2}\right)^2+\left(\dfrac{2a}{2}\right)^2}}=\dfrac{\sqrt{21}}{7}.\]
    }
\end{ex}
\begin{ex}[Chuyên Hoàng Văn Thụ - Hòa Bình - 2021]%[1H3B3-3]%Câu 20.
    Cho hình chóp $S.ABCD$ có đáy là hình vuông cạnh $a$, $SA\perp(ABC),SA=a\sqrt{2}$. Góc giữa đường thẳng $SC$ và mặt phẳng $(ABCD)$ bằng
    \choice
    {$60^{\circ}$}
    {$90^{\circ}$}
    {\True $45^{\circ}$}
    {$30^{\circ}$}
    \loigiai{
        \begin{center}
            \begin{tikzpicture}[line cap=round,line join=round,>=triangle 45,thick]
                \draw (1,4) -- (4,) (1,4) -- (3,0) (1,4) -- (0,0);
                \draw [dashed] (0,0) -- (1,1) -- (1,4) (3,0) -- (1,1) -- (4,1);
                \draw (0,0)-- (3,0) -- (4,1);
                \draw [shift={(3,0)}] (0,0) -- (116.565:0.6) arc (116.565:153.435:0.6);
                \draw (1,4) node[above] {$S$};
                \draw (1,1.1) node[left] {$A$};
                \draw (4,1) node[right] {$B$};
                \draw (3,0) node [right] {$C$};
                \draw (0,0) node [left] {$D$};
            \end{tikzpicture}
        \end{center}
        Ta có $AC=a\sqrt{2}$ suy ra $\triangle SAC$ vuông cân tại $A$.\\
        Góc giữa $SC$ và mp $(ABCD)$ chính là góc $\widehat{SCA}=45^{\circ}$.
    }
\end{ex}
\begin{ex}[Sở Yên Bái - 2021]%[1H3B3-3]%Câu 21.
    Cho hình chóp $S.ABCD$ có đáy là hình vuông canh $a, SA$ vuông góc với mặt phẳng đáy và $SA=a\sqrt{6}$. Góc giữa đường thẳng $SC$ và mặt phẳng $(ABCD)$ bằng
    \choice
    {\True $60^{\circ}$}
    {$45^{\circ}$}
    {$90^{\circ}$}
    {$30^{\circ}$}
    \loigiai{
        \begin{center}
            \begin{tikzpicture}[line cap=round,line join=round,>=triangle 45,thick]
                \draw (1,4) -- (4,) (1,4) -- (3,0) (1,4) -- (0,0);
                \draw [dashed] (0,0) -- (1,1) -- (1,4) (3,0) -- (1,1) -- (4,1);
                \draw (0,0)-- (3,0) -- (4,1);
                \draw [shift={(3,0)}] (0,0) -- (116.565:0.6) arc (116.565:153.435:0.6);
                \draw (1,4) node[above] {$S$};
                \draw (1,1.1) node[left] {$A$};
                \draw (4,1) node[right] {$B$};
                \draw (3,0) node [right] {$C$};
                \draw (0,0) node [left] {$D$};
            \end{tikzpicture}
        \end{center}
        Ta có $AC$ là hình chiếu của $SC$ lên mặt đáy $(ABCD)$ \\
        $\Rightarrow\widehat{\left(SC,(ABCD)\right)}=\widehat{(SC,AC)}=\widehat{SCA}\Rightarrow\tan\widehat{SCA}=\dfrac{SA}{AC}=\sqrt{3}\Rightarrow\widehat{\left(SC,(ABCD)\right)}=60^{\circ}$.
    }
\end{ex}
\begin{ex}[THPT Lương Thế Vinh - 2021]%[1H3K3-3]%Câu 22.
    Cho hình chóp $S.ABCD$ có đáy là hình thoi tâm $O$, $\triangle ABD$ đều cạnh $a\sqrt{2}, SA$ vuông góc với mặt phẳng đáy và $SA=\dfrac{3a\sqrt{2}}{2}$. Góc giữa đường thẳng $SO$ và mặt phẳng $(ABCD)$ bằng
    \choice
    {$45^{\circ}$}
    {$90^{\circ}$}
    {$30^{\circ}$}
    {\True $60^{\circ}$}
    \loigiai{
        \begin{center}
            \begin{tikzpicture}[line cap=round,line join=round,>=triangle 45,thick]
                \draw (1,4) -- (3,0) (4,1) -- (1,4) -- (0,0);
                \draw [dashed] (0,0) -- (1,1) -- (1,4) -- (2,0.5) (3,0) -- (1,1) -- (4,1);
                \draw (0,0)-- (3,0) -- (4,1);
                \draw [shift={(2,0.5)},dashed] (0,0) -- (105.945:0.452) arc (105.945:153.435:0.452) -- cycle;
                \draw (1,4) node[above] {$S$};
                \draw (1,1.1) node[left] {$A$};
                \draw (4,1) node[right] {$B$};
                \draw (3,0) node [right] {$C$};
                \draw (0,0) node [left] {$D$};
                \draw (2.1,0.4) node [left] {$O$};
            \end{tikzpicture}
        \end{center}
        Ta có: $ABCD$ là hình thoi có tâm là $O\Rightarrow O$ là trung điểm của $BD$.\\
        Mà $\triangle ABD$ đều nên $AO\perp BD$.\\
        Lại có $SA\perp(ABCD)\Rightarrow\widehat{\left(SO,(ABCD)\right)}=\widehat{SOA}$.\\
        Xét $\triangle ABO$ có: $AO=\sqrt{AB^2-BO^2}=\sqrt{(a\sqrt{2})^2-\left(\dfrac{a\sqrt{2}}{2}\right)^2}=\dfrac{a\sqrt{6}}{2}$.\\
        Ta có: $\tan\widehat{SAO}=\dfrac{SA}{AO}=\dfrac{\dfrac{3a\sqrt{2}}{2}}{\dfrac{a\sqrt{6}}{2}}=\sqrt{3}\Rightarrow\widehat{SOA}=60^{\circ}$.
    }
\end{ex}
\begin{ex}[THPT Đặng Thúc Hứa - Nghệ An - 2021]%[1H3B3-3]%Câu 23.
    Cho khối lăng trụ đứng $ABC.A'B'C'$ có $AA'=a\sqrt{6}$, đáy $ABC$ là tam giác vuông cân tại $B$ và $BA=BC=a$. Góc giữa đường thẳng $A'C$ và mặt phẳng đáy bằng
    \choice
    {$45^{\circ}$}
    {$90^{\circ}$}
    {\True $60^{\circ}$}
    {$30^{\circ}$}
    \loigiai{
        \begin{center}
            \begin{tikzpicture}[line cap=round,line join=round,>=triangle 45,thick]
                \draw (1,5)-- (4,5) -- (3,4) -- (1,1);
                \draw [dashed, thick] (1,1) -- (4,1);
                \draw (1,1)-- (3,0)-- (4,1) -- (4,5) (1,1) -- (1,5)-- (3,4) -- (3,0);
                \draw (1,5) node[above] {$A$};
                \draw (4,5) node[above] {$B$};
                \draw (3,4) node[right] {$C$};
                \draw (1,1) node[left] {$A'$};
                \draw (4,1) node[right] {$B'$};
                \draw (3,0) node [right] {$C'$};
            \end{tikzpicture}
        \end{center}
        Ta có $AA'\perp(ABC)\Rightarrow AC$ là hình chiếu của $A'C$ lên mặt phẳng $(ABC)$.\\
        Khi đó $\widehat{\left(A'C,(ABC)\right)}=\widehat{(A'C,AC)}=\widehat{A'CA}$.\\
        Ta có $AC=AB\sqrt{2}=a\sqrt{2}$.\\
        $\tan\widehat{A'CA}=\dfrac{AA'}{AC}=\dfrac{a\sqrt{6}}{a\sqrt{2}}=\sqrt{3}\Rightarrow\widehat{A'CA}=60^{\circ}$.
    }
\end{ex}
\begin{ex}[THPT Chu Văn An - Thái Nguyên - 2021]%[1H3B3-3]%Câu 24.
    Cho chóp $S.ABC$ có đáy là tam giác vuông tại $B$. $AB=3a,BC=\sqrt{3}a$. $SA$ vuông góc với đáy và $SA=2a$. Góc giữa $SC$ và đáy là
    \choice
    {$90^{\circ}$}
    {$45^{\circ}$}
    {$60^{\circ}$}
    {\True $30^{\circ}$}
    \loigiai{
        \begin{center}
            \begin{tikzpicture}[line cap=round,line join=round,>=triangle 45,thick]
                \draw (1,1) -- (1,4) -- (3,0) (4,1) -- (1,4);
                \draw [dashed] (1,1) -- (1,4) (3,0) -- (1,1) -- (4,1);
                \draw (1,1) -- (3,0) -- (4,1);
                \draw (1,4) node[above] {$S$};
                \draw (1,1.1) node[left] {$A$};
                \draw (4,1) node[right] {$C$};
                \draw (3,0) node [below] {$B$};
            \end{tikzpicture}
        \end{center}
        Ta có $AC=\sqrt{12}a$.\\
        Xét tam giác $\triangle SAC$ ta có  $\tan\widehat{SCA}=\dfrac{2a}{\sqrt{12}a}=\dfrac{1}{\sqrt{3}}\Rightarrow\widehat{SCA}={30}^{\circ}$.
    }
\end{ex}
\begin{ex}[THPT Quảng Xương 1-Thanh Hóa - 2021]%[1H3K3-3]%Câu 25.
    Cho hình chóp $S.ABCD$ có đáy là hình thoi tâm $O$, $\triangle ABD$ đều cạnh $a\sqrt{2}$, $SA$ vuông góc với mặt phẳng đáy và $SA=\dfrac{3a\sqrt{2}}{2}$. Góc giữa đường thẳng $SO$ và mặt phẳng $(ABCD)$ bằng
    \choice
    {$45^{\circ}$}
    {$30^{\circ}$}
    {\True $60^{\circ}$}
    {$90^{\circ}$}
    \loigiai{
        \begin{center}
            \begin{tikzpicture}[line cap=round,line join=round,>=triangle 45,thick]
                \draw (1,4) -- (3,0) (4,1) -- (1,4) -- (0,0);
                \draw [dashed] (0,0) -- (1,1) -- (1,4) -- (2,0.5) (3,0) -- (1,1) -- (4,1);
                \draw (0,0)-- (3,0) -- (4,1);
                \draw [shift={(2,0.5)},dashed] (0,0) -- (105.945:0.452) arc (105.945:153.435:0.452) -- cycle;
                \draw (1,4) node[above] {$S$};
                \draw (1,1.1) node[left] {$A$};
                \draw (4,1) node[right] {$B$};
                \draw (3,0) node [right] {$C$};
                \draw (0,0) node [left] {$D$};
                \draw (2.1,0.4) node [left] {$O$};
            \end{tikzpicture}
        \end{center}
        Tam giác $ABD$ đều cạnh $a\sqrt{2}$, suy ra $AO=\dfrac{(a\sqrt{2})\sqrt{3}}{2}=\dfrac{a\sqrt{6}}{2}$.\\
        Vì $SA\perp(ABCD)$, suy ra $OA$ là hình chiếu của $OS$ lên mặt phẳng $(ABCD)$, suy ra:\\
        $\left(SO;(ABCD)\right)=\widehat{SOA}$.\\
        Xét tam giác vuông $SAO$ ta có $\tan\widehat{SOA}=\dfrac{SA}{AO}=\dfrac{3a\sqrt{2}}{2}\cdot\dfrac{2}{a\sqrt{6}}=\sqrt{3}\Rightarrow\widehat{SOA}=60^{\circ}$.\\
        Vậy $\left(SO;(ABCD)\right)=60^{\circ}$.
    }
\end{ex}
\begin{ex}[Trung Tâm Thanh Tường - 2021]%[1H3B3-3]%Câu 26.
    Cho hình chóp $S.ABC$ có $SA\perp(ABC)$, $SA=a\sqrt{3}$, tam giác $ABC$ vuông tại $B$ có $AC=2a,BC=a\sqrt{3}$. Góc giữa $SB$ và mặt phẳng $(ABC)$ bằng
    \choice
    {$90^{\circ}$}
    {$45^{\circ}$}
    {$30^{\circ}$}
    {\True $60^{\circ}$}
    \loigiai{
        \begin{center}
            \begin{tikzpicture}[line cap=round,line join=round,>=triangle 45,thick]
                \draw (1,1) -- (1,4) -- (3,0) (4,1) -- (1,4);
                \draw [dashed] (1,1) -- (1,4) (3,0) -- (1,1) -- (4,1);
                \draw (1,1) -- (3,0) -- (4,1);
                \draw (1,4) node[above] {$S$};
                \draw (1,1.1) node[left] {$A$};
                \draw (4,1) node[right] {$C$};
                \draw (3,0) node [below] {$B$};
            \end{tikzpicture}
        \end{center}
        Ta có $AB=\sqrt{AC^2-BC^2}=\sqrt{(2a)^2-(a\sqrt{3})^2}=a$.\\
        Dễ thấy $\widehat{\left(SB;(ABC)\right)}=\widehat{(SB;AB)}=\widehat{SBA}$. Khi đó $\tan\widehat{SBA}=\dfrac{SA}{AB}=\dfrac{a\sqrt{3}}{a}=\sqrt{3}\Rightarrow\widehat{SBA}=60^{\circ}$.\\
        Vậy $\widehat{\left(SB;(ABC)\right)}=60^{\circ}$.
    }
\end{ex}
\begin{ex}[THPT Trần Phú - Đà Nẵng - 2021]%[1H3K3-3]%Câu 27.
    \immini{
        Cho hình chóp tứ giác $S.ABCD$ có đáy $ABCD$ là hình vuông tâm $I$, cạnh $a$. Biết $SA$ vuông góc với mặt đáy $(ABCD)$ và $SA=a\sqrt{3}$ (tham khảo hình vẽ bên). Khi đó $\tan$ của góc giữa đường thẳng $SI$ và mặt phẳng $(ABCD)$ là
        \choice
        {\True $\sqrt{6}$}
        {$\dfrac{\sqrt{6}}{6}$}
        {$\sqrt{3}$}
        {$\dfrac{\sqrt{3}}{3}$}
    }{
        \begin{tikzpicture}[line cap=round,line join=round,>=triangle 45,thick]
            \draw (1,4) -- (3,0) (4,1) -- (1,4) -- (0,0);
            \draw [dashed] (0,0) -- (1,1) -- (1,4) -- (2,0.5) (3,0) -- (1,1) -- (4,1) -- (0,0);
            \draw (0,0)-- (3,0) -- (4,1);
            \draw (1,4) node[above] {$S$};
            \draw (1,1.1) node[left] {$A$};
            \draw (4,1) node[right] {$B$};
            \draw (3,0) node [right] {$C$};
            \draw (0,0) node [left] {$D$};
            \draw (2.25,0.25) node [left] {$I$};
        \end{tikzpicture}
    }
    \loigiai{
        Ta có: $AI=\dfrac{1}{2}AC=\dfrac{a\sqrt{2}}{2}$.\\
        Mà $SA\perp(ABCD)$ nên $AI$ là hình chiếu của $SI$ trên mặt phẳng $(ABCD)$ \\
        $ \Rightarrow\widehat{\left(SI;(ABCD)\right)}=\widehat{(SI; AI)}=\widehat{SIA} $ (do tam giác $SAI$ vuông tại $A$).\\
        Vậy $\tan\widehat{\left(SI;(ABCD)\right)}=\tan\widehat{SIA}=\dfrac{SA}{AI}=\sqrt{6}$.
    }
\end{ex}
\begin{ex}[Chuyên Biên Hòa - 2021]%[1H3K3-3]%Câu 28.
    Cho hình chóp $S.ABCD$ có đáy $ABCD$ là hình thoi cạnh $a$, $O$ là giao điểm của $AC$ và $BD$. $\widehat{ABC}=60^{\circ}$; $SO$ vuông góc với $(ABCD)$ và $SO=a\sqrt{3}$. Góc giữa $SB$ và mặt phẳng $(SAC)$ bằng
    \choice
    {\True $\left(25^{\circ}; 27^{\circ}\right)$}
    {$\left(62^{\circ}; 66^{\circ}\right)$}
    {$\left(53^{\circ}; 61^{\circ}\right)$}
    {$\left(27^{\circ}; 33^{\circ}\right)$}
    \loigiai{
        \begin{center}
            \begin{tikzpicture}[line cap=round,line join=round,>=triangle 45,thick]
                \draw (2,4) -- (3,0) (4,1) -- (2,4) -- (0,0);
                \draw [dashed] (0,0) -- (1,1) -- (2,4) -- (2,0.5) (3,0) -- (1,1) -- (4,1) (0,0) -- (4,1);
                \draw (0,0)-- (3,0) -- (4,1);
                \draw (2,4) node[above] {$S$};
                \draw (1,1.1) node[left] {$A$};
                \draw (4,1) node[right] {$B$};
                \draw (3,0) node [right] {$C$};
                \draw (0,0) node [left] {$D$};
                \draw (2.25,0.25) node [left] {$O$};
            \end{tikzpicture}
        \end{center}
        Ta có $\widehat{ABC}=60^{\circ};\triangle ABC$ cân tại $B$ nên $\triangle ABC$ đều cạnh $a$.\\
        Suy ra: $BO=\dfrac{a\sqrt{3}}{2}$.\\
        Do $OB\perp (SAC)$ nên góc giữa $SB$ và mặt phẳng $(SAC)$ chính là góc $\widehat{BSO}$.\\
        Ta có tam giác $SOB$ vuông tại $O$ nên $\tan\widehat{BSO}=\dfrac{BO}{SO}=\dfrac{1}{2}\Rightarrow\widehat{BSO}\approx 26,56^{\circ}$.
    }
\end{ex}
\begin{ex}[Sở Cần Thơ - 2021]%[1H3B3-3]%Câu 29
    Cho hình chóp $S.ABC$ có tam giác $ABC$ vuông tại $B$, có $AB=a\sqrt{3}$, $BC=a, SA$ vuông góc với mặt phẳng $(ABC)$ và $SA=2a$. Góc giữa đường thẳng $SC$ và mặt phẳng $(ABC)$ bằng
    \choice
    {\True $45^{\circ}$}
    {$30^{\circ}$}
    {$60^{\circ}$}
    {$90^{\circ}$}
    \loigiai{
        \begin{center}
            \begin{tikzpicture}[line cap=round,line join=round,>=triangle 45,thick]
                \draw (1,1) -- (1,4) -- (3,0) (4,1) -- (1,4);
                \draw [dashed] (1,1) -- (1,4) (3,0) -- (1,1) -- (4,1);
                \draw (1,1) -- (3,0) -- (4,1);
                \draw (1,4) node[above] {$S$};
                \draw (1,1.1) node[left] {$A$};
                \draw (4,1) node[right] {$C$};
                \draw (3,0) node [below] {$B$};
            \end{tikzpicture}
        \end{center}
        Hình chiếu của $SC$ lên mặt phẳng $(ABC)$ là $AC$. Khi đó $(SC,(ABC))=(SC, AC)=\widehat{SCA}$.\\
        Xét tam giác $ABC$ vuông tại $B$ ta có $AC=\sqrt{AB^2+BC^2}=\sqrt{(a\sqrt{3})^2+a^2}=2a$.\\
        Xét tam giác $SAC$ vuông tại $A$ ta có $\tan\widehat{SCA}=\dfrac{SA}{AC}=\dfrac{2a}{2a}=1\Rightarrow\widehat{SCA}=45^{\circ}$.
    }
\end{ex}
\begin{ex}[Sở Quảng Bình - 2021]%[1H3B3-3]
    Cho hình chóp $S.ABCD$ có đáy $ABCD$ là hình vuông cạnh $a$, biết $SA\perp(ABCD)$ và $SA=a\sqrt{2}$. Góc giữa đường thẳng $SC$ và mặt phẳng $(ABCD)$ bằng
    \choice
    {$60^{\circ}$}
    {\True $45^{\circ}$}
    {$30^{\circ}$}
    {$90^{\circ}$}
    \loigiai{\immini{
            Ta có $SA\perp(ABCD)$ nên $AC$ là hình chiếu của $SC$ trên mặt phẳng $(ABCD)$.\\
            Do đó: $\widehat{\left(SC,(ABCD)\right)}=\widehat{(SC,AC)}=\widehat{SCA}$.\\
            Xét hình vuông $ABCD$ ta có: $AC=a\sqrt{2}$.\\
            Xét $\triangle ABC$ vuông tại $A$, ta có: $\tan \widehat{SCA}=\dfrac{SA}{AC}=\dfrac{a\sqrt{2}}{a\sqrt{2}}=1\Rightarrow\widehat{SCA}=45^{\circ}$.	}
        {	\begin{tikzpicture}[scale=0.57,>=stealth, font=\footnotesize, line join=round, line cap=round]	%hình chóp có đáy là hình thang.
                \coordinate[label=above left:$A$] (A) at (0,0);
                \coordinate[label=below left:$B$] (B) at (-4,-3);
                \coordinate[label=below left:$C$] (C) at (4,-3);
                \coordinate[label=right:$D$] (D) at (8,0);
                \coordinate[label=above:$S$] (S) at ($(A)+(0,6)$);
                \draw (S)--(B)--(C)--(D)--(S)--(C);
                \draw[dashed] (A)--(B) (S)--(A)--(D) (A)--(C);
                \pic[draw,angle radius=2mm,angle eccentricity=2] {right angle = S--A--D};
                \pic[draw,angle radius=4mm,angle eccentricity=2] {angle = S--C--A};
                \foreach \diem in {A,B,C,D,S}	\fill (\diem)circle(1.5pt);
        \end{tikzpicture}}
    }
\end{ex}
\begin{ex}[Chuyên Tuyên Quang - 2021]%[1H3B3-3]
    Cho hình chóp $S.ABCD$ có đáy là hình thoi tâm $O$, tam giác $ABD$ đều cạnh bằng $a\sqrt{2}$, $SA=\dfrac{3a\sqrt{2}}{2}$ và vuông góc với mặt phẳng đáy. Góc giữa đường thẳng $SO$ và mặt phẳng $(ABCD)$ bằng
    \choice
    {\True $60^{\circ}$}
    {$45^{\circ}$}
    {$30^{\circ}$}
    {$90^{\circ}$}
    \loigiai{\immini{
            Ta có $AO$ là hình chiếu vuông góc của $SO$ trên mp $(ABCD)$ nên góc giữa giữa đường thẳng $SO$ và mặt phẳng $(ABCD)$ bằng góc giữa $SO$ và $AO$.\\
            Xét tam giác $SAO$ vuông tại $A$ có\\ $SA=\dfrac{3a\sqrt{2}}{2}$; $AO=\dfrac{a\sqrt{6}}{2}$.\\
            $\tan\widehat{SOA}=\dfrac{SA}{OA}=\dfrac{\dfrac{3a\sqrt{2}}{2}}{\dfrac{\sqrt{6}a}{2}}=\sqrt{3}\Rightarrow\widehat{SOA}=60^{\circ}$.\\
            Vậy góc giữa giữa đường thẳng $SO$ và mặt phẳng $(ABCD)$ bằng $60^{\circ}$.}
        {\begin{tikzpicture}[scale=0.57,>=stealth, font=\footnotesize, line join=round, line cap=round]	%hình chóp có đáy là hình thang.
                \coordinate[label=above left:$A$] (A) at (0,0);
                \coordinate[label=below left:$B$] (B) at (-4,-3);
                \coordinate[label=below left:$C$] (C) at (4,-3);
                \coordinate[label=right:$D$] (D) at (8,0);
                \coordinate[label=above:$S$] (S) at ($(A)+(0,4)$);
                \coordinate[label=below:$O$] (O) at ($(B)!1/2!(D)$);
                \draw (S)--(B)--(C)--(D)--(S)--(C);
                \draw[dashed] (C)--(A)--(B) (S)--(A)--(D) (B)--(D) (S)--(O);
                \pic[draw,angle radius=2mm,angle eccentricity=2] {right angle = S--A--B};
                \pic[draw,angle radius=2mm,angle eccentricity=2] {right angle = S--A--D};
                \pic[draw,angle radius=4mm,angle eccentricity=2] {angle = S--O--A};
                \foreach \diem in {A,B,C,D,S,O}	\fill (\diem)circle(1.5pt);
        \end{tikzpicture}}
    }
\end{ex}
\begin{ex}[Chuyên Vinh - 2021]%[1H3K3-3]
    Cho hình lăng trụ tam giác đều $ABC.A'B'C'$ có $AB=a$, $AA'=a\sqrt{2}$. Góc giữa đường thẳng $A'C$ và mặt phẳng $(ABB'A')$ bằng
    \choice
    {$45^{\circ}$}
    {$75^{\circ}$}
    {$60^{\circ}$}
    {\True $30^{\circ}$}
    \loigiai{
        \immini{
            Gọi $M$ là trung điểm $AB$. Do tam giác $ABC$ đều nên $CM\perp AB$.\\
            Lại có $CM\perp A'A$ nên suy ra $CM\perp(ABB'A')\\ \Rightarrow\widehat{\left(A'C,(ABB'A')\right)}=\widehat{(A'C, A'M)}=\widehat{MA'C}$.\\
            Ta có $A'C=\sqrt{A'A^2+AC^2}=\sqrt{2a^2+a^2}=a\sqrt{3}$ và $CM=\dfrac{a\sqrt{3}}{2}$.\\
            Trong tam giác vuông $CMA'$, ta có\\ $\sin\widehat{MA'C}=\dfrac{MC}{A'C}=\dfrac{\dfrac{a\sqrt{3}}{2}}{a\sqrt{3}}=\dfrac{1}{2}\Rightarrow\widehat{MA'C}=30^{\circ}$.\\
            Vậy góc giữa đường thẳng $A'C$ và mặt phẳng $(ABB'A')$ bằng $30^{\circ}$.	}
        {\begin{tikzpicture}[scale=0.7, font=\footnotesize, line join=round, line cap=round, >=stealth]
                \coordinate[label=above :$A'$] (A') at (0,4);
                \coordinate[label=above:$B'$] (B') at (4.5,2.5);
                \coordinate[label=right :$C'$] (C') at (6,4);
                \coordinate[label=below :$A$] (A) at (0,0);
                \coordinate[label=below :$B$] (B) at ($(A)+(B')-(A')$);
                \coordinate[label=below :$C$] (C) at ($(B)+(C')-(B')$);
                \coordinate[label=below :$M$] (M) at ($(A)!0.5!(B)$);
                \draw (A)--(B)--(C)--(C')--(A')--(B')--(B) (B')--(C') (A)--(A')--(B);
                \draw[dashed] (A)--(C) (A')--(M)--(C) (A')--(C);
                \pic[draw,angle radius=2mm,angle eccentricity=1.5] {right angle = C--M--B};
                \pic[draw,angle radius=4mm,angle eccentricity=2] {angle = M--A'--C};
                \foreach \diem in {A,B,M,A',B',C'}	\fill (\diem)circle(1.5pt);
        \end{tikzpicture}}
    }
\end{ex}
\begin{ex}[Cụm Liên Trường Hải Phòng 2019]%[1H3K3-3]
    Cho khối chóp $S.ABC$ có $SA\perp(ABC)$, tam giác $ABC$ vuông tại $B$, $AC=2a$, $BC=a$, $SB=2a\sqrt{3}$. Tính góc giữa $SA$ và mặt phẳng $(SBC)$.
    \choice
    {$45^{\circ}$}
    {\True $30^{\circ}$}
    {$60^{\circ}$}
    {$90^{\circ}$}
    \loigiai{
        \immini{
            Trong $(SAB)$ kẻ $AH\perp SB (H\in SB)$.\\
            Vì $\heva{&SA\perp BC\\&AB\perp BC}\Rightarrow BC\perp(SAB)\Rightarrow BC\perp AH$.\\
            Mà $SB\perp AH$ do cách dựng nên $AH\perp(SBC)$, hay $H$ là hình chiếu của $A$ lên $(SBC)$ suy ra góc giữa $SA$ và $(SBC)$ là góc $\widehat{ASH}$ hay góc $\widehat{ASB}$.\\
            Tam giác $ABC$ vuông ở $B\Rightarrow AB=\sqrt{AC^2-BC^2}=a\sqrt{3}$.\\
            Tam giác $SAB$ vuông ở $A\Rightarrow\sin\widehat{ASB}=\dfrac{AB}{SB}=\dfrac{1}{2}\Rightarrow\widehat{ASB}=30^{\circ}$.}
        {\begin{tikzpicture}[scale=0.8,>=stealth, font=\footnotesize, line join=round, line cap=round]
                \coordinate (S) at (0,3);
                \coordinate (A) at (0,0);
                \coordinate (B) at (2.5,-1.5);
                \coordinate (C) at (5,0);
                \coordinate[label=below:$H$] (H) at ($(S)!0.75!(B)$);
                \draw (A) node[left]{$A$}--(B) node[below]{$B$}--(C) node[right]{$C$}--(S) node[above]{$S$}--(A)--(H) (S)--(B);
                \draw[dashed] (A)--(C);
                \pic[draw,angle radius=2mm,angle eccentricity=2] {right angle = S--A--C};
                \pic[draw,angle radius=2mm,angle eccentricity=2] {right angle = S--H--A};
                \pic[draw,angle radius=4mm,angle eccentricity=2] {right angle =A--B--C};
                \pic[draw,angle radius=4mm,angle eccentricity=4] {angle = A--S--B};
                \foreach \diem in {A,B,C,H}	\fill (\diem)circle(1.5pt);
            \end{tikzpicture}
        }
    }
\end{ex}
\begin{ex}[Chuyên Bắc Ninh 2019]%[1H3K3-3]
    Cho hình chóp $SABCD$ có đáy là hình thang vuông tại $A$ và $B$,  $AB=BC=a,AD=2a$. Biết $SA$ vuông góc với đáy $(ABCD)$ và $SA=a$. Gọi $M$, $N$ lần lượt là trung điểm của $SB$ và $CD$. Tính sin góc giữa đường thẳng $MN$ và mặt phẳng $\left(SAC\right)$.
    \choice
    {$\dfrac{\sqrt{5}}{5}$}
    {$\dfrac{\sqrt{55}}{10}$}
    {\True $\dfrac{3\sqrt{5}}{10}$}
    {$\dfrac{2\sqrt{5}}{5}$}
    \loigiai{
        \immini{	Ta gọi $E,~F$ lần lượt là trung điểm của $SC,~AB$.\\
            Ta có $ME\parallel NF$ (do cùng song song với $BC$).\\ Nên tứ giác $MENF$ là hình thang,\\
            và $\heva{&MF\parallel SA\\&SA\perp (ABCD)}\Rightarrow MF\perp (ABCD)$\\ hay tứ giác $MENF$ là hình thang vuông tại $M,~F$.\\
            Gọi $K=NF\cap AC,I=EK\cap M$ thì $I=MN\cap (SAC)$.\\
            Ta có $\heva{&NC\perp AC\\&NC\perp SA}\Rightarrow NC\perp (SAC)$\\ hay $C$ là hình chiếu vuông góc của $N$ lên $(SAC)$.\\
            Từ đó ta có được, góc giữa $MN$ và $(SAC)$ là góc giữa $MN$ và $CI$.\\
            Suy ra, gọi $\alpha$ là góc giữa $MN$ và $(SAC)$ thì $\sin\alpha=\dfrac{CN}{IN}$.
        }
        {\begin{tikzpicture}[scale=0.8,>=stealth, font=\footnotesize, line join=round, line cap=round]	%hình chóp có đáy là hình thang.
                \coordinate[label=left:$A$] (A) at (0,0);
                \coordinate[label=below left:$B$] (B) at (-2,-3);
                \coordinate[label=below left:$C$] (C) at (1,-3);
                \coordinate[label=right:$D$] (D) at (6,0);
                \coordinate[label=above:$S$] (S) at ($(A)+(0.5,4)$);
                \coordinate[label=below:$N$] (N) at ($(C)!.5!(D)$);
                \coordinate[label= left:$M$] (M) at ($(S)!.5!(B)$);
                \coordinate[label= below right:$F$] (F) at ($(A)!.5!(B)$);
                \coordinate[label=right:$E$] (E) at ($(S)!0.5!(C)$);
                \coordinate[label=below left:$K$] (K) at (intersection of N--F and A--C);
                \coordinate[label=below left:$I$] (I) at (intersection of E--K and M--N);
                \draw (S)--(B)--(C)--(D)--(S)--(C) (M)--(E)--(N);
                \draw[dashed] (C)--(A)--(B) (S)--(A)--(D) (F)--(A) (C)--(I) (F)--(N) (M)--(F) (K)--(E) (M)--(N);
                \pic[draw,angle radius=2mm,angle eccentricity=2] {right angle = B--F--A};
                \pic[draw,angle radius=2mm,angle eccentricity=2] {right angle = B--A--D};
                \pic[draw,angle radius=2mm,angle eccentricity=2] {right angle = C--B--A};
                \pic[draw,angle radius=2mm,angle eccentricity=2] {right angle = A--C--D};
                \pic[draw,angle radius=2mm,angle eccentricity=2] {right angle = S--A--D};
                \pic[draw,angle radius=2mm,angle eccentricity=2] {right angle = E--M--F};
                \pic[draw,angle radius=2mm,angle eccentricity=2] {right angle = M--F--N};
                \pic[draw,angle radius=8mm,angle eccentricity=2] {angle = S--C--A};
                \foreach \diem in {A,B,C,D,S,M,N,K,F,I,E}	\fill (\diem)circle(1.5pt);
        \end{tikzpicture}}
        \noindent
        $NC=\dfrac{1}{2}CD=\dfrac{a\sqrt{2}}{2}$; $\dfrac{IN}{IM}=\dfrac{KN}{ME}=2\Rightarrow IN=\dfrac{2}{3}MN =\dfrac{2}{3}\sqrt{MF^2+FN^2}=\dfrac{a\sqrt{10}}{3}$.\\
        Vậy $\sin\alpha=\dfrac{CN}{IN}=\dfrac{3\sqrt{5}}{10}$.
    }
\end{ex}
\begin{ex}[Mã 102 - 2018]%[1H3B3-3]
    Cho hình chóp $S.ABCD$ có đáy là hình vuông cạnh $a$, $SA$ vuông góc với mặt phẳng đáy và $SA=\sqrt{6}a$. Góc giữa đường thẳng $SC$ và mặt phẳng đáy bằng
    \choice
    {$45^{\circ}$}
    {\True $60^{\circ}$}
    {$30^{\circ}$}
    {$90^{\circ}$}
    \loigiai{\immini
        {
            Do $SA\perp(ABCD)$ nên góc giữa đường thẳng $SC$ và mặt phẳng đáy bằng góc $\widehat{SCA}$.\\
            Ta có $SA=\sqrt{6}a$, $AC=\sqrt{2}a\\ \Rightarrow\tan\widehat{SCA}=\dfrac{SA}{AC} =\sqrt{3}\Rightarrow\widehat{SCA}=60^{\circ}$.\\
            Vậy góc giữa đường thẳng $SC$ và và mặt phẳng đáy bằng bằng $60^{\circ}$.}
        {	\begin{tikzpicture}[scale=0.5,>=stealth, font=\footnotesize, line join=round, line cap=round]
                \coordinate[label=above left:$A$] (A) at (0,0);
                \coordinate[label=below left:$B$] (B) at (-4,-3);
                \coordinate[label=below left:$C$] (C) at (4,-3);
                \coordinate[label=right:$D$] (D) at (8,0);
                \coordinate[label=above:$S$] (S) at ($(A)+(0,6)$);
                \draw (S)--(B)--(C)--(D)--(S)--(C);
                \draw[dashed] (A)--(B) (S)--(A)--(D) (A)--(C);
                \pic[draw,angle radius=2mm,angle eccentricity=2] {right angle = S--A--D};
                \pic[draw,angle radius=4mm,angle eccentricity=2] {angle = S--C--A};
                \foreach \diem in {A,B,C,D,S}	\fill (\diem)circle(1.5pt);
        \end{tikzpicture}}
    }
\end{ex}
\begin{ex}[Mã 101 - 2018]%[1H3B3-3]
    Cho hình chóp $S.ABCD$ có đáy là hình vuông cạnh $a$, $SA$ vuông góc với mặt phẳng đáy và $SB=2a$. Góc giữa đường thẳng $SB$ và mặt phẳng đáy bằng
    \choice
    {$45^{\circ}$}
    {\True $60^{\circ}$}
    {$90^{\circ}$}
    {$30^{\circ}$}
    \loigiai{
        \immini{Do $SA\perp(ABCD)$ nên góc giữa đường thẳng $SB$ và mặt phẳng đáy bằng góc $\widehat{SBA}$.\\
            Ta có $\cos\widehat{SBA}=\dfrac{AB}{SB} =\dfrac{1}{2}\Rightarrow\widehat{SBA}=60^{\circ}$.\\
            Vậy góc giữa đường thẳng $SB$ và và mặt phẳng đáy bằng bằng $60^{\circ}$.	}
        {	\begin{tikzpicture}[scale=0.5,>=stealth, font=\footnotesize, line join=round, line cap=round]
                \coordinate[label=above left:$A$] (A) at (0,0);
                \coordinate[label=below left:$B$] (B) at (-4,-3);
                \coordinate[label=below left:$C$] (C) at (4,-3);
                \coordinate[label=right:$D$] (D) at (8,0);
                \coordinate[label=above:$S$] (S) at ($(A)+(0,6)$);
                \draw (S)--(B)--(C)--(D)--(S)--(C);
                \draw[dashed] (A)--(B) (S)--(A)--(D);
                \pic[draw,angle radius=2mm,angle eccentricity=2] {right angle = S--A--D};
                \pic[draw,angle radius=4mm,angle eccentricity=2] {angle = A--B--S};
                \foreach \diem in {A,B,C,D,S}	\fill (\diem)circle(1.5pt);
    \end{tikzpicture}}	}
\end{ex}
\begin{ex}[Mã 101 - 2019]%[1H3B3-3]
    \immini{	Cho hình chóp $S.ABC$ có $SA$ vuông góc với mặt phẳng $(ABC)$,     $SA=2a$, tam giác $ABC$ vuông tại $B, AB=a\sqrt{3}$ và $BC=a$ (minh họa như hình vẽ bên). Góc giữa đường thẳng $SC$ và mặt phẳng $(ABC)$ bằng
        \choice
        {\True $45^{\circ}$}
        {$30^{\circ}$}
        {$60^{\circ}$}
        {$90^{\circ}$}}
    {	{\begin{tikzpicture}[scale=0.8,>=stealth, font=\footnotesize, line join=round, line cap=round]
                \coordinate (S) at (0,3);
                \coordinate (A) at (0,0);
                \coordinate (B) at (2.5,-1.5);
                \coordinate (C) at (5,0);
                \draw (A) node[left]{$A$}--(B) node[below]{$B$}--(C) node[right]{$C$}--(S) node[above]{$S$}--(A) (S)--(B);
                \draw[dashed] (A)--(C);
                \pic[draw,angle radius=2mm,angle eccentricity=2] {right angle = S--A--C};
                \pic[draw,angle radius=4mm,angle eccentricity=2] {right angle =A--B--C};
                \pic[draw,angle radius=4mm,angle eccentricity=4] {angle = S--C--A};
                \foreach \diem in {A,B,C,S}	\fill (\diem)circle(1.5pt);
            \end{tikzpicture}
    }}
    \loigiai{
        Ta có $SA\perp (ABC)$ nên $AC$ là hình chiếu của $SC$ lên mặt phẳng $(ABC)$.\\
        Do đó $\left(SC,(ABC)\right)=(SC,AC)=\widehat{SCA}$.\\
        Tam giác $ABC$ vuông tại $B, AB=a\sqrt{3}$ và $BC=a$ nên $AC=\sqrt{AB^2+BC^2}=\sqrt{4a^2}=2a$.\\
        Do đó tam giác $SAC$ vuông cân tại $A$ nên $\widehat{SCA}=45^{\circ}$.\\
        Vậy $\left(SC,(ABC)\right)=45^{\circ}$.
    }
\end{ex}
\begin{ex}[Đề Tham Khảo 2018]%[1H3K3-3]
    \immini{
        Cho hình chóp tứ giác đều $S.ABCD$ có tất cả các cạnh bằng $a$. Gọi $M$ là trung điểm của $SD$ (tham khảo hình vẽ bên). Tang của góc giữa đường thẳng $BM$ và mặt phẳng $(ABCD)$ bằng
        \choice
        {$\dfrac{\sqrt{2}}{2}$}
        {$\dfrac{\sqrt{3}}{3}$}
        {$\dfrac{2}{3}$}
        {\True $\dfrac{1}{3}$}
    }
    {\begin{tikzpicture}[scale=1,>=stealth, font=\footnotesize, line join=round, line cap=round]
            \coordinate[label=left:$A$] (A) at (0,0);
            \coordinate[label=below:$B$] (B) at (-1.9,-1.6);
            \coordinate[label=below:$C$] (C) at (1.6,-1.6);
            \coordinate[label=right:$D$] (D) at ($(A)+(C)-(B)$);
            \coordinate[label=above:$S$] (S) at ($(O)+(0,3.5)$);
            \coordinate[label=right:$M$] (M) at ($(S)!0.5!(D)$);
            \draw (S)--(B)--(C)--(D)--(S)--(C);
            \draw[dashed] (D)--(A) (S)--(A)--(B)--(M);
            \foreach \diem in {A,B,C,S,M}	\fill (\diem)circle(1.5pt);
    \end{tikzpicture}}
    \loigiai{
        \immini{
            Gọi $O$ là tâm của hình vuông.\\ Ta có $SO\perp(ABCD)$ và $SO=\sqrt{a^2-\dfrac{a^2}{2}}=\dfrac{a\sqrt{2}}{2}$.\\
            Gọi $M$ là trung điểm của $OD$ ta có $MH\parallel SO$ nên $H$ là hình chiếu của $M$ lên mặt phẳng $(ABCD)$ và $MH=\dfrac{1}{2}SO=\dfrac{a\sqrt{2}}{4}$.\\
            Do đó góc giữa đường thẳng $BM$ và mặt phẳng $(ABCD)$ là $\widehat{MBH}$.\\
            Khi đó ta có $\tan\widehat{MBH}=\dfrac{MH}{BH}=\dfrac{\dfrac{a\sqrt{2}}{4}}{\dfrac{3a\sqrt{2}}{4}}=\dfrac{1}{3}$.
        }
        {\begin{tikzpicture}[scale=1,>=stealth, font=\footnotesize, line join=round, line cap=round]
                \coordinate[label=left:$A$] (A) at (0,0);
                \coordinate[label=below:$B$] (B) at (-1.9,-1.6);
                \coordinate[label=below:$C$] (C) at (1.6,-1.6);
                \coordinate[label=right:$D$] (D) at ($(A)+(C)-(B)$);
                \coordinate[label=below:$O$] (O) at ($(A)!1/2!(C)$);
                \coordinate[label=above:$S$] (S) at ($(O)+(0,3.5)$);
                \coordinate[label=right:$M$] (M) at ($(S)!0.5!(D)$);
                \coordinate[label=below:$H$] (H) at ($(O)!0.5!(D)$);
                \draw (S)--(B)--(C)--(D)--(S)--(C);
                \draw[dashed](S)--(O) (D)--(A)--(C) (B)--(D) (S)--(A)--(B)--(M)--(H);
                \pic[draw,angle radius=4mm,angle eccentricity=2] {right angle =S--O--D};
                \pic[draw,angle radius=4mm,angle eccentricity=2] {right angle =M--H--D};
                \pic[draw,angle radius=4mm,angle eccentricity=2] {angle = H--B--M};
                \foreach \diem in {A,B,C,S,M,O,H}	\fill (\diem)circle(1.5pt);
        \end{tikzpicture}}
        Vậy tang của góc giữa đường thẳng $BM$ và mặt phẳng $(ABCD)$ bằng $\dfrac{1}{3}$.
    }
\end{ex}
\begin{ex}[Mã 104 - 2019]%[1H3K3-3]
    \immini{
        Cho hình chóp $S.ABC$ có $SA$ vuông góc với mặt phẳng $(ABC)$, $SA=2a$, tam giác $ABC$ vuông cân tại $B$ và $AB=a\sqrt{2}$ (minh họa như hình vẽ bên). Góc giữa đường thẳng $SC$ và mặt phẳng $(ABC)$ bằng
        \choice
        {$30^{\circ}$}
        {$90^{\circ}$}
        {$60^{\circ}$}
        {\True $45^{\circ}$}
    }
    {\begin{tikzpicture}[scale=0.8,>=stealth, font=\footnotesize, line join=round, line cap=round]
            \coordinate (S) at (0,3);
            \coordinate (A) at (0,0);
            \coordinate (B) at (2.5,-1.5);
            \coordinate (C) at (5,0);
            \draw (A) node[left]{$A$}--(B) node[below]{$B$}--(C) node[right]{$C$}--(S) node[above]{$S$}--(A) (S)--(B);
            \draw[dashed] (A)--(C);
            \pic[draw,angle radius=2mm,angle eccentricity=2] {right angle = S--A--C};
            \pic[draw,angle radius=4mm,angle eccentricity=2] {right angle =A--B--C};
            \pic[draw,angle radius=4mm,angle eccentricity=4] {angle = S--C--A};
            \foreach \diem in {A,B,C,S}	\fill (\diem)circle(1.5pt);
        \end{tikzpicture}
    }
    \loigiai{
        Ta có $SA\perp(ABC)$ nên đường thẳng $AC$ là hình chiếu vuông góc của đường thẳng $SC$ lên mặt phẳng $(ABC)$.\\
        Do đó, $\alpha=\left(\widehat{SC,(ABC)}\right)=\left(\widehat{SC, AC}\right)=\widehat{SCA}$ (tam giác $SAC$ vuông tại $A$).\\
        Tam giác $ABC$ vuông cân tại $B$ nên $AC=AB\sqrt{2}=2a$.\\
        Suy ra $\tan\widehat{SCA}=\dfrac{SA}{AC}=1$ nên $\alpha=45^{\circ}$.
    }
\end{ex}
\begin{ex}[Sở Vĩnh Phúc 2019]%[1H3K3-3]
    Cho hình chóp tứ giác đều $S.ABCD$ có tất cả các cạnh bằng $2a$. Gọi $M$ là trung điểm của $SD$. Tính $\tan$ của góc giữa đường thẳng $BM$ và mặt phẳng $(ABCD)$.
    \choice
    {$\dfrac{\sqrt{2}}{2}$}
    {$\dfrac{\sqrt{3}}{3}$}
    {$\dfrac{2}{3}$}
    {\True $\dfrac{1}{3}$}
    \loigiai{
        \immini{
            Trong tam giác $SOD$ dựng $MH\parallel SO,H\in OD$ ta có $MH\perp(ABCD)$.\\
            Vậy góc tạo bởi $BM$ và mặt phẳng $(ABCD)$ là $\widehat{MBH}$.\\
            Ta có $MH=\dfrac{1}{2}SO=\dfrac{1}{2}\sqrt{SD^2-OD^2}=\dfrac{1}{2}\sqrt{4a^2-2a^2}=\dfrac{a\sqrt{2}}{2}$.\\
            $BH=\dfrac{3}{4}BD=\dfrac{3}{4}2a\sqrt{2}=\dfrac{3a\sqrt{2}}{2}$.\\
            Vậy $\tan\widehat{MBH}=\dfrac{MH}{BH}=\dfrac{1}{3}$.	}
        {\begin{tikzpicture}[scale=1,>=stealth, font=\footnotesize, line join=round, line cap=round]
                \coordinate[label=left:$A$] (A) at (0,0);
                \coordinate[label=below:$B$] (B) at (-1.9,-1.6);
                \coordinate[label=below:$C$] (C) at (1.6,-1.6);
                \coordinate[label=right:$D$] (D) at ($(A)+(C)-(B)$);
                \coordinate[label=below:$O$] (O) at ($(A)!1/2!(C)$);
                \coordinate[label=above:$S$] (S) at ($(O)+(0,3.5)$);
                \coordinate[label=right:$M$] (M) at ($(S)!0.5!(D)$);
                \coordinate[label=below:$H$] (H) at ($(O)!0.5!(D)$);
                \draw (S)--(B)--(C)--(D)--(S)--(C);
                \draw[dashed](S)--(O) (D)--(A)--(C) (B)--(D) (S)--(A)--(B)--(M)--(H);
                \pic[draw,angle radius=4mm,angle eccentricity=2] {right angle =S--O--D};
                \pic[draw,angle radius=4mm,angle eccentricity=2] {right angle =M--H--D};
                \pic[draw,angle radius=4mm,angle eccentricity=2] {angle = H--B--M};
                \foreach \diem in {A,B,C,S,M,O,H}	\fill (\diem)circle(1.5pt);
        \end{tikzpicture}}
    }
\end{ex}
\begin{ex}[Chuyên Bắc Giang 2019]%[1H3K3-3]
    Cho hình chóp $S.ABCD$ có đáy $ABCD$ là hình vuông cạnh $a$ và $SA\perp(ABCD)$. Biết $SA=\dfrac{a\sqrt{6}}{3}$. Tính góc giữa $SC$ và $(ABCD)$.
    \choice
    {\True $30^{\circ}$}
    {$60^{\circ}$}
    {$75^{\circ}$}
    {$45^{\circ}$}
    \loigiai{\immini{
            Ta có $AC=a\sqrt{2}$.\\
            Vì $AC$ là hình chiếu của SC lên $(ABCD)$ nên góc giữa $SC$ và $(ABCD)$ là góc giữa $SC$ và $AC$.\\
            Xét $\triangle SAC$ vuông tại A, ta có\\ $\tan\widehat{SCA}=\dfrac{\dfrac{a\sqrt{6}}{3}}{a\sqrt{2}}=\dfrac{\sqrt{3}}{3}$. Suy ra $\widehat{SCA}=30^{\circ}$.
        }
        {	\begin{tikzpicture}[scale=0.57,>=stealth, font=\footnotesize, line join=round, line cap=round]	%hình chóp có đáy là hình thang.
                \coordinate[label=above left:$A$] (A) at (0,0);
                \coordinate[label=below left:$B$] (B) at (-4,-3);
                \coordinate[label=below left:$C$] (C) at (4,-3);
                \coordinate[label=right:$D$] (D) at (8,0);
                \coordinate[label=above:$S$] (S) at ($(A)+(0,6)$);
                \draw (S)--(B)--(C)--(D)--(S)--(C);
                \draw[dashed] (A)--(B) (S)--(A)--(D) (A)--(C);
                \pic[draw,angle radius=2mm,angle eccentricity=2] {right angle = S--A--D};
                \pic[draw,angle radius=4mm,angle eccentricity=2] {angle = S--C--A};
                \foreach \diem in {A,B,C,D,S}	\fill (\diem)circle(1.5pt);
        \end{tikzpicture}}
    }
\end{ex}
\begin{ex}[Chuyên Hùng Vương Gia Lai 2019]%[1H3K3-3]
    Cho hình chóp $S.ABCD$ có đáy $ABCD$ là hình vuông cạnh $a$, $SA$ vuông góc với đáy và $SA=a\sqrt{3}$. Gọi $\alpha$ là góc giữa $SD$ và $(SAC)$. Giá trị $\sin\alpha$ bằng
    \choice
    {\True $\dfrac{\sqrt{2}}{4}$}
    {$\dfrac{\sqrt{2}}{2}$}
    {$\dfrac{\sqrt{3}}{2}$}
    {$\dfrac{\sqrt{2}}{3}$}
    \loigiai{
        \immini{
            Gọi $O=AC\cap BD$. \\Ta có $\heva{&DO\perp AC\\&DO\perp SA\left(SA\perp(ABCD)\right)}\\ \Rightarrow DO\perp(ABCD)$ \\
            $ \Rightarrow SO $ là hình chiếu của $SD$ lên mặt phẳng $(SAC)\\ \Rightarrow\widehat{\left(SD;(SAC)\right)}=\widehat{(SD; SO)}=\widehat{DSO}=\alpha$.\\
            Xét $\triangle SAD$ vuông tại $A$ có $SD=\sqrt{3a^2+a^2}=2a$.\\
            Xét $\triangle SOD$ vuông tại $O$ có có $SD=2a$, $OD=\dfrac{a\sqrt{2}}{2}\\ \Rightarrow\sin\alpha=\sin\widehat{DSO}=\dfrac{DO}{SD}=\dfrac{\sqrt{2}}{4}$.
        }
        {	\begin{tikzpicture}[scale=0.57,>=stealth, font=\footnotesize, line join=round, line cap=round]	%hình chóp có đáy là hình thang.
                \coordinate[label=above left:$A$] (A) at (0,0);
                \coordinate[label=below left:$B$] (B) at (-4,-3);
                \coordinate[label=below left:$C$] (C) at (4,-3);
                \coordinate[label=right:$D$] (D) at (8,0);
                \coordinate[label=above:$S$] (S) at ($(A)+(0,6)$);
                \coordinate[label=below left:$O$] (O) at ($(A)!0.5!(C)$);
                \draw (S)--(B)--(C)--(D)--(S)--(C);
                \draw[dashed] (A)--(B) (S)--(A)--(D) (A)--(C) (B)--(D) (S)--(O);
                \pic[draw,angle radius=2mm,angle eccentricity=2] {right angle = S--A--D};
                \pic[draw,angle radius=2mm,angle eccentricity=2] {right angle = S--O--D};
                \pic[draw,angle radius=4mm,angle eccentricity=2] {angle = O--S--D};
                \foreach \diem in {A,B,C,D,S}	\fill (\diem)circle(1.5pt);
        \end{tikzpicture}}
    }
\end{ex}
\begin{ex}[Sở Bắc Giang 2019]%[1H3K3-3]
    Cho hình chóp tam giác $S.ABC$ có đáy là tam giác đều cạnh $a$. Tam giác $SAB$ cân tại $S$ và thuộc mặt phẳng vuông góc với đáy. Biết $SC$ tạo với mặt phẳng đáy một góc $60^{\circ}$, gọi $M$ là trung điểm của $BC$. Gọi $\alpha$ là góc giữa đường thẳng $SM$ và mặt phẳng $(ABC)$. Tính $\cos\alpha$.
    \choice
    {$\cos\alpha=\dfrac{\sqrt{6}}{3}$}
    {$\cos\alpha=\dfrac{\sqrt{3}}{3}$}
    {$\cos\alpha=\dfrac{3}{\sqrt{10}}$}
    {\True $\cos\alpha=\dfrac{1}{\sqrt{10}}$}
    \loigiai{
        \immini{
            Gọi $H$ là trung điểm $AB$ dễ thấy $SH\perp(ABC)$.\\
            $SC$ tạo với mặt phẳng đáy một góc $60^{\circ}$ suy ra $\widehat{SCH}=60^{\circ}$.\\
            Có $HC=\dfrac{a\sqrt{3}}{2}\Rightarrow SH=HC\cdot\tan\widehat{SCH}=\dfrac{3a}{2}$.\\
            Dễ thấy $\alpha=\widehat{SMH}$, $HM=\dfrac{1}{2}AC=\dfrac{a}{2}\\
            \Rightarrow SM=\dfrac{a\sqrt{10}}{2}\Rightarrow\cos\alpha=\dfrac{HM}{SM}=\dfrac{1}{\sqrt{10}}$.	}
        {\begin{tikzpicture}[scale=1,>=stealth, font=\footnotesize, line join=round, line cap=round]
                \tkzDefPoints{0/0/A,1.2/-1.5/B,4/0/C}
                \coordinate[label=left:$A$] (A) at (0,0);
                \coordinate[label=below:$B$] (B) at (1.2,-1.5);
                \coordinate[label=right:$C$] (C) at (4,0);
                \coordinate[label=above:$S$] (S) at ($(A)+(1,3)$);
                \coordinate[label=left:$H$](H) at ($(A)!0.5!(B)$);
                \coordinate[label=below:$M$] (M) at ($(C)!0.5!(B)$);
                \draw (A)--(B)  (S)--(A) (B)--(S)--(C)--(B) (S)--(H) (S)--(M);
                \draw[dashed] (A)--(C) (H)--(M);
                \coordinate[label=above:$60^\circ$] () at ($(C)+(-1.0,.15)$);
                \pic[draw,angle radius=6mm,angle eccentricity=3] { angle = S--C--A};
                \pic[draw,angle radius=6mm,angle eccentricity=3] { angle = S--M--H};
                \pic[draw,angle radius=2mm,angle eccentricity=3] { right angle = A--H--S};
                \foreach \diem in {A,B,C,S,M,H}	\fill (\diem)circle(1.5pt);
        \end{tikzpicture}}
    }
\end{ex}

\begin{ex}[THCS - THPT Nguyễn Khuyến 2019]%[1H3K3-3]%Câu 70.
    Cho hình chóp tứ giác đều $S.ABCD$ có $AB=a$, $O$ là trung điểm $AC$ và $SO=b$. Gọi $(\Delta)$ là đường thẳng đi qua $C$, $(\Delta)$ chứa trong mặt phẳng $(ABCD)$ và khoảng cách từ $O$ đến $(\Delta)$ là $\dfrac{a\sqrt{14}}{6}$. Giá trị lượng giác $\cos\left((SA),(\Delta)\right)$ bằng
    \choice
    {$\dfrac{2a}{3\sqrt{4b^2-2a^2}}$}
    {\True $\dfrac{2a}{3\sqrt{2a^2+4b^2}}$}
    {$\dfrac{a}{3\sqrt{2a^2+4b^2}}$}
    {$\dfrac{a}{3\sqrt{4b^2-2a^2}}$}
    \loigiai{
        \immini{
            Gọi $(\Delta')$ là đường thẳng đi qua $A$ và song song với $(\Delta)$. Hạ $OH\perp(\Delta')\left(H\in(\Delta')\right)$. Do $O$ là trung điểm của $AC$ và $(\Delta)\parallel(\Delta')$ nên $\mathrm{d}\left(O,(\Delta')\right)=\mathrm{d}\left(O,(\Delta)\right)$ hay $OH=\dfrac{a\sqrt{14}}{6}$.\\
            Do $S.ABCD$ là hình chóp tứ giác đều nên đáy $ABCD$ là hình vuông và $SO\perp(ABCD)$.\\
            Do $AH\perp OH$ và $AH\perp SO$ nên, suy ra $AH\perp SH$.\\
            Do $ABCD$ là hình vuông cạnh $a$ nên $AC=a\sqrt{2}$, suy ra $OA=\dfrac{a\sqrt{2}}{2}$.
        }{
            \begin{tikzpicture}[line join=round,line cap=round, font=\footnotesize,scale=1,>=stealth]
                \def\r{1.8}
                \path
                (0,0) coordinate (O)
                (-\r,0) coordinate (M)
                (\r,0) coordinate (N)
                (10:1.5*\r) coordinate (B)
                (90:1.9*\r) coordinate (S)
                ($(M)+(B)-(N)$) coordinate (A)
                ($(B)!2!(O)$) coordinate (D)
                ($(A)!2!(O)$) coordinate (C)
                ($(C)!1.2!-15:(B)$) coordinate (B')node[below]{$(\Delta)$}
                ($(B')!1.3!(C)$) coordinate (K)
                ($(A)+(B')-(C)$) coordinate (m)node[above]{$(\Delta')$}
                ($(m)!1.3!(A)$) coordinate (n)
                ($(m)!0.7!(A)$) coordinate (H)
                ;
                \draw (S)--(B)-- (C) --(D)--(S)(S)--(C)(B')--(K)
                pic [draw,angle radius=2mm] {right angle =A--H--O};
                \begin{scope}
                    \clip (H)--(A)--(S);
                    \draw (A) circle(.4);
                \end{scope}
                \draw[dashed] (S)--(A)--(B)(A)--(D)(m)--(n)(A)--(C)(B)--(D)(H)--(S)--(O)--(H);
                \foreach \x/\g in {S/90,A/120,B/0,C/-80,D/180,O/-90,H/120}\fill[black] (\x) circle(1pt)+(\g:0.3)node{$\x$};
            \end{tikzpicture}
        }
        \noindent Áp dụng Định lí Pitago vào tam giác vuông $AHO$ ta có\\
        $AH=\sqrt{OA^2-OH^2}=\sqrt{\left(\dfrac{a\sqrt{2}}{2}\right)^2-\left(\dfrac{a\sqrt{14}}{6}\right)^2}=\dfrac{a}{3}$.\\
        Áp dụng Định lí Pitago vào tam giác vuông $SAO$ ta có\\
        $SA=\sqrt{OA^2+SO^2}=\sqrt{\left(\dfrac{a\sqrt{2}}{2}\right)^2+b^2}=\dfrac{\sqrt{2a^2+4b^2}}{2}$.\\
        Do $(\Delta)\parallel(\Delta')$ nên $\cos\left((SA),(\Delta)\right)=\cos\left((SA),(\Delta')\right)=\cos\widehat{SAH}=\dfrac{AH}{SA}=\dfrac{2a}{3\sqrt{2a^2+4b^2}}$.
    }
\end{ex}

\begin{ex}[HSG Bắc Ninh 2019]%[1H3K3-3]%Câu 71.
    Cho hình chóp $S.ABCD$ có đáy $ABCD$ là hình chữ nhật, $AB=a$, $AD=a\sqrt{3}$. Mặt bên $SAB$ là tam giác đều và nằm trong mặt phẳng vuông góc với mặt đáy. Cosin của góc giữa đường thẳng $SD$ và mặt phẳng $(SBC)$ bằng
    \choice
    {\True $\dfrac{\sqrt{13}}{4}$}
    {$\dfrac{\sqrt{3}}{4}$}
    {$\dfrac{2\sqrt{5}}{5}$}
    {$\dfrac{1}{4}$}
    \loigiai{
        \immini{
            Gọi $H$, $M$ lần lượt là trung điểm của $AB, SB$; $O$ là tâm của hình chữ nhật $ABCD$.\\
            Ta có $MO\parallel SD$.\\
            Dễ thấy $BC\perp(SAB)\Rightarrow BC\perp AM$, mà $SB\perp AM$ nên $AM\perp(SBC)$.\\
            Xét tam giác $AMO$, có:\\
            $AM=\dfrac{a\sqrt{3}}{2}$;\\
            $AO=\dfrac{1}{2}AC=\dfrac{1}{2}\sqrt{a^2+3a^2}=a$;
        }{
            \begin{tikzpicture}[line join=round,line cap=round, font=\footnotesize,scale=1,>=stealth]
                \def\r{1.8}
                \path
                (0,0) coordinate (O)
                (-\r,0) coordinate (H)
                (\r,0) coordinate (N)
                (10:1.5*\r) coordinate (D)
                (90:2*\r) coordinate (S')
                ($(H)+(S')-(O)$) coordinate (S)
                ($(H)+(D)-(N)$) coordinate (A)
                ($(D)!2!(O)$) coordinate (B)
                ($(A)!2!(O)$) coordinate (C)
                ($(S)!0.5!(B)$) coordinate (M)
                ;
                \draw (S)--(B)-- (C) --(D)--(S)(S)--(C)
                pic [draw,angle radius=2mm] {right angle =A--H--S};
                \draw[dashed] (S)--(A)--(B)(A)--(D)(A)--(C)(B)--(D)(H)--(S)--(O)--(H)(O)--(M)(A)--(M);
                \foreach \x/\g in {S/90,A/-100,B/180,C/-80,D/0,O/-90,H/150,M/160}\fill[black] (\x) circle(1pt)+(\g:0.3)node{$\x$};
            \end{tikzpicture}
        }
        \noindent $MO=\dfrac{1}{2}SD=\dfrac{1}{2}\sqrt{SH^2+HD^2}=\dfrac{1}{2}\sqrt{SH^2+HA^2+AD^2}=\dfrac{1}{2}\sqrt{\left(\dfrac{a\sqrt{3}}{2}\right)^2+\left(\dfrac{a}{2}\right)^2+3a^2}=a$ \\
        $ \Rightarrow\triangle AMO$ cân tại $O$. \\
        $\Rightarrow\sin\widehat{AMO}=\dfrac{\mathrm{d}(O;AM)}{OM}=\dfrac{\sqrt{MO^2-\dfrac{AM^2}{4}}}{OM}=\dfrac{\sqrt{a^2-\dfrac{3a^2}{16}}}{a}=\dfrac{\sqrt{13}}{4}$.\\
        $\Rightarrow\cos\left(SD,(SBC)\right)=\sin\widehat{AMO}=\dfrac{\sqrt{13}}{4}$.
    }
\end{ex}

\begin{ex}[Sở Hà Nội 2019]%[1H3K3-3]%Câu 72.
    Cho hình chóp $S.ABC$ có đáy là tam giác vuông tại $C$, $CH$ vuông góc với $AB$ tại $H$, $I$ là trung điểm của đoạn $HC$. Biết $SI$ vuông góc với mặt phẳng đáy, $\widehat{ASB}=90^{\circ}$. Gọi $O$ là trung điểm của đoạn $AB$, $O'$ là tâm mặt cầu ngoại tiếp tứ diện $SABI$. Góc tạo bởi đường thẳng $OO'$ và mặt phẳng $(ABC)$ bằng
    \choice
    {$60^{\circ}$}
    {\True $30^{\circ}$}
    {$90^{\circ}$}
    {$45^{\circ}$}
    \loigiai{
        \immini{Do $\widehat{ASB}=90^{\circ}$ nên tâm $O'$ của mặt cầu ngoại tiếp tứ diện $SABI$ nằm trên đường thẳng $d$ đi qua trung điểm $O$ của đoạn thẳng $AB$ và $d\perp(SAB)$. $(1)$\\
            Trong mặt phẳng $(SCH)$ kẻ $IK\perp SH$ tại $K$.\\
            Theo giả thiết $SI\perp(ABC)$ suy ra $SI\perp AB$.\\
            Từ $SI\perp AB$ và $AB\perp CH$ suy ra $AB\perp(SCH)\Rightarrow AB\perp IK$.\\
            Từ $IK\perp SH$ và $AB\perp IK$ ta có $IK\perp(SAB)$. $(2)$\\
            Từ $(1)$ và $(2)$ ta có $IK\parallel d$. Bởi vậy \\
            $\left(OO';(ABC)\right)=\left(d;(ABC)\right)=\left(IK;(ABC)\right)$.\\
            Vì $(SCH)\perp(ABC)$ nên $IH$ là hình chiếu vuông góc của $IK$ trên mặt phẳng $(ABC)$. Bởi vậy
        }{
            \begin{tikzpicture}[line join=round,line cap=round, font=\footnotesize,scale=1,>=stealth]
                \def\r{3}
                \path
                (0,0) coordinate (C)
                (\r,0) coordinate (A)
                (-40:0.6*\r) coordinate (B)
                (90:1*\r) coordinate (S')
                ($(B)!0.4!(A)$) coordinate (H)
                ($(C)!0.5!(H)$) coordinate (I)
                ($(S')+(I)$) coordinate (S)
                ($(A)!0.5!(B)$) coordinate (O)
                ($(S)!0.6!(H)$) coordinate (K)
                ($(K)+(O)-(I)$) coordinate (d')
                ($(O)!1.5!(d')$) coordinate (d)node[above]{$d$}
                ($(O)!1.3!(d')$) coordinate (O')
                ;
                \draw (S)--(A)-- (B) --(C)--(S)--(A)(S)--(B)(S)--(H)(O)--(d)
                pic [draw,angle radius=2mm] {right angle =B--C--A};
                \draw[dashed] (A)--(C)--(H)(O)--(I)(S)--(I)--(K);
                \foreach \x/\g in {S/90,A/0,B/-90,C/180,H/-60,I/-90,O/-10,K/60,O'/-20}\fill[black] (\x) circle(1pt)+(\g:0.3)node{$\x$};
            \end{tikzpicture}
        }
        \noindent $\left(IK,(ABC)\right)=(IK,IH)=\widehat{HIK}=\widehat{HSI}$.\\
        Do tam giác $ABC$ vuông tại $C$ và $SAB$ vuông tại $S$ nên $CO=SO=\dfrac{AB}{2}$.\\
        Xét hai tam giác vuông $CHO$ và $SHO$ có $CO=SO$, cạnh $OH$ chung nên\\
        $\triangle CHO=\triangle SHO$ (c.g.c), bởi vậy $CH=SH$.\\
        Xét tam giác $SIH$ vuông tại $I$ có $IH=\dfrac{CH}{2}=\dfrac{SH}{2}$, ta có $\sin\widehat{HSI}=\dfrac{IH}{SH}=\dfrac{1}{2}\Rightarrow\widehat{HSI}=30^{\circ}$.\\
        Vậy $\left(OO';(ABC)\right)=30^{\circ}$.
    }
\end{ex}

\begin{ex}[Sở Bắc Ninh 2019]%[1H3K3-3]%Câu 73.
    Cho hình chóp $S.ABCD$ có đáy $ABCD$ là hình thoi cạnh $a$ và $\widehat{ABC}=60^{\circ}$. Hình chiếu vuông góc của điểm $S$ lên mặt phẳng $(ABCD)$ trùng với trọng tâm của tam giác $ABC$, gọi $\varphi$ là góc giữa đường thẳng $SB$ và mặt phẳng $(SCD)$, tính $\sin\varphi$ biết rằng $SB=a$.
    \choice
    {$\sin\varphi=\dfrac{\sqrt{3}}{2}$}
    {$\sin\varphi=\dfrac{1}{4}$}
    {$\sin\varphi=\dfrac{1}{2}$}
    {\True $\sin\varphi=\dfrac{\sqrt{2}}{2}$}
    \loigiai{
        \immini{
            \textbf{Cách 1:}
            Gọi $O$ là trọng tâm của tam giác $ABC$. Dựng đường thẳng $d$ qua $O$ và $d\parallel SB$, $d$ cắt $SD$ tại $K$. Khi đó góc giữa $SB$ và $(SCD)$ chính là góc giữa $OK$ và $(SCD)$.\\
            Vì $SO\perp (ABCD)\Rightarrow SO\perp CD$.\\
            Ta lại có $\triangle ABC$ đều ($\triangle ABC$ cân tại $B$ và $\widehat{BAC}=60^{\circ}$)\\
            $\Rightarrow AB\perp CO\Rightarrow CD\perp CO$ \\
            $\Rightarrow CD\perp (SCO)\Rightarrow (SCD)\perp (SCO)$.\\
            Gọi $H$ là hình chiếu của $O$ trên $SC$, khi đó ta có:\\
            $\heva{&OH\perp SC\\&OH\perp CD}\Rightarrow OH\perp(SCD)$. Do đó góc giữa $SB$ và mặt phẳng $(SCD)$ là $\widehat{OKH}=\varphi$.
        }{
            \begin{tikzpicture}[line join=round,line cap=round, font=\footnotesize,scale=1,>=stealth]
                \def\r{3}
                \path
                (0,0) coordinate (A)
                (\r,0) coordinate (D)
                (235:0.55*\r) coordinate (B)
                ($(B)+(D)$) coordinate (C)
                ($(A)!0.5!(C)$) coordinate (I)
                ($(B)!1/3!(D)$) coordinate (O)
                ($(C)!1/3!(D)$) coordinate (m)
                ($(O)!1.7!90:(m)$) coordinate (S)
                ($(S)!1/3!(D)$) coordinate (K)
                ($(S)!0.5!(C)$) coordinate (H)
                ;
                \draw (S)--(B)-- (C) --(D)--(S)--(C)(H)--(K)
                pic [draw,angle radius=2mm] {right angle =O--H--C};
                \draw[dashed] (D)--(A)--(B)(O)--(S)--(A)(A)--(C)(B)--(D)(O)--(K)(O)--(H)(O)--(C);
                \foreach \x/\g in {S/90,A/160,B/180,C/-90,D/0,O/-90,K/60,H/30}\fill[black] (\x) circle(1pt)+(\g:0.3)node{$\x$};
            \end{tikzpicture}
        }
        \noindent Ta có $\sin\varphi=\sin\widehat{OKH}=\dfrac{OH}{OK}$.\\
        Tứ diện $S.ABC$ là tứ diện đều cạnh $a$ nên ta tính được:\\
        $OC=\dfrac{a\sqrt{3}}{3}$, $SO=\dfrac{a\sqrt{6}}{3}\Rightarrow OH=\dfrac{a\sqrt{2}}{3}$.\\
        Vì $OK\parallel SB\Rightarrow\dfrac{OK}{SB}=\dfrac{DO}{DB}=\dfrac{2}{3}\Rightarrow OK=\dfrac{2}{3}SB=\dfrac{2}{3}a$.\\
        Vậy: $\sin\varphi=\dfrac{OH}{OK}=\dfrac{\sqrt{2}}{2}$.\\
        \textbf{Cách 2:}
        Trước hết ta chứng minh được $\sin (SB;(SCD))=\dfrac{\mathrm{d}(B,(SCD))}{SB}$ (như hình trên).\\
        Gọi $O$ là trọng tâm tam giác $ABC$. Khi đó ta có $CO\perp CD$.\\
        Dựng $OH\perp SC$ suy ra $OH\perp (SCD)$. Ta tính được $OC=\dfrac{a\sqrt{3}}{3},SO=\dfrac{a\sqrt{6}}{3}\Rightarrow OH=\dfrac{a\sqrt{2}}{3}$.\\
        Khi đó $\mathrm{d}(B,(SCD))=\dfrac{3}{2}\mathrm{d}(O,(SCD))=\dfrac{3}{2}OH=\dfrac{3}{2}\dfrac{a\sqrt{2}}{3}=\dfrac{a\sqrt{2}}{2}$.\\
        Vậy $\sin (SB;(SCD))=\dfrac{\dfrac{a\sqrt{2}}{2}}{a}=\dfrac{\sqrt{2}}{2}$.
    }
\end{ex}

\begin{ex}[Sở Bình Phước - 2018]%[1H3K3-3]%Câu 74.
    Cho hình chóp $S.ABCD$ có đáy là hình vuông cạnh $a$, $SA\perp(ABCD)$, $SA=x$. Xác định $x$ để hai mặt phẳng $(SBC)$ và $(SCD)$ hợp với nhau góc $60^{\circ}$.
    \choice
    {$x=2a$}
    {\True $x=a$}
    {$x=\dfrac{3a}{2}$}
    {$x=\dfrac{a}{2}$}
    \loigiai{
        \immini{
            $SB=SD=\sqrt{SA^2+AD^2}$ = $\sqrt{x^2+a^2}$.\\
            $\triangle SDC=\triangle SBC$; $BM\perp SC$; $DM\perp SC$; $BM=DM$; $M\in SC$.\\
            $SC=\sqrt{SA^2+AC^2}$ = $\sqrt{x^2+2a^2}$; $MD=\dfrac{SD\cdot CD}{SC}$ = $\dfrac{a\sqrt{x^2+a^2}}{\sqrt{x^2+2a^2}}$.\\
            $\left(\widehat{(SBC);(SDC)}\right)=\widehat{(BM;BD)}=60^{\circ}$.\\
            TH1: $\widehat{BMD}=60^{\circ}\Rightarrow MD=BD\Leftrightarrow\dfrac{a\sqrt{x^2+a^2}}{\sqrt{x^2+2a^2}}=a\sqrt{2}$ (vô nghiệm).
        }{
            \begin{tikzpicture}[line join=round,line cap=round, font=\footnotesize,scale=0.8,>=stealth]
                \def\r{3}
                \path
                (0,0) coordinate (A)
                (\r,0) coordinate (B)
                (90:0.9*\r) coordinate (S)
                (235:0.55*\r) coordinate (D)
                ($(B)+(D)$) coordinate (C)
                ($(S)!0.55!(C)$) coordinate (M)
                ;
                \draw (S)--(B)-- (C) --(D)--(S)--(C)(D)--(M)--(B)
                pic [draw,angle radius=2mm] {right angle =C--M--D};
                \draw[dashed] (D)--(A)--(B)(S)--(A)(B)--(D);
                \foreach \x/\g in {S/90,A/160,B/0,C/0,D/180,M/40}\fill[black] (\x) circle(1pt)+(\g:0.3)node{$\x$};
            \end{tikzpicture}
        }
        \noindent TH2: $\widehat{BMD}=120^{\circ}\Rightarrow BD=MD\sqrt{3}\Leftrightarrow a\sqrt{2}=\dfrac{a\sqrt{3}\sqrt{x^2+a^2}}{\sqrt{x^2+2a^2}}\Leftrightarrow x=a$.

    }
\end{ex}

\begin{ex}[Sở Lào Cai - 2018]%[1H3K3-3]%Câu 75.
    Cho hình chóp $S.ABC$ có đáy là tam giác vuông tại $B$, cạnh bên $SA$ vuông góc với mặt đáy, $AB=2a$, $\widehat{BAC}=60^{\circ}$ và $SA=a\sqrt{2}$. Góc giữa đường thẳng $SB$ và mặt phẳng $(SAC)$ bằng
    \choice
    {\True $45^{\circ}$}
    {$60^{\circ}$}
    {$30^{\circ}$}
    {$90^{\circ}$}
    \loigiai{
        \immini{
            Kẻ $BH\perp AC$ ($H\in AC$) và theo giả thiết $BH\perp SA$ nên $BH\perp (SAC)$.\\
            Do đó, $SH$ là hình chiếu vuông góc của $SB$ lên mặt phẳng $(SAC)$.\\
            Suy ra, $(SB,(SAC))=(SB,SH)=\widehat{BSH}$.\\
            Mà ta có: $SB=a\sqrt{6}$, $HB=AB\sin 60^{\circ}=a\sqrt{3}$.\\
            $\Rightarrow\sin (\widehat{BSH)}=\dfrac{1}{\sqrt{2}}\Rightarrow\widehat{BSH}=45^{\circ}$.
        }{
            \begin{tikzpicture}[line join=round,line cap=round, font=\footnotesize,scale=1,>=stealth]
                \def\r{3}
                \path
                (0,0) coordinate (A)
                (\r,0) coordinate (B)
                (90:0.9*\r) coordinate (S)
                (-40:0.7*\r) coordinate (C)
                ($(A)!0.6!(C)$) coordinate (H)
                ;
                \draw (A)--(S)-- (B) --(C)--(A)(C)--(S)--(H)
                pic [draw,angle radius=2mm] {right angle =B--H--C};
                \draw[dashed] (A)--(B)--(H);
                \foreach \x/\g in {S/90,A/160,B/0,C/0,H/200}\fill[black] (\x) circle(1pt)+(\g:0.3)node{$\x$};
            \end{tikzpicture}
        }
        \noindent
    }
\end{ex}

\begin{ex}[Chuyên Hạ Long - 2018]%[1H3K3-3]%Câu 76.
    Cho hình chóp $S.ABCD$ có đáy $ABCD$ là hình vuông cạnh $a$, cạnh bên $SA$ vuông góc với mặt phẳng đáy, $SA=a\sqrt{2}$. Gọi $M$, $N$ lần lượt là hình chiếu vuông góc của điểm $A$ trên các cạnh $SB$, $SD$. Góc giữa mặt phẳng $(AMN)$ và đường thẳng $SB$ bằng
    \choice
    {$45^{\circ}$}
    {$90^{\circ}$}
    {$120^{\circ}$}
    {\True $60^{\circ}$}
    \loigiai{
        \immini{
            Ta có $BC\perp(SAB)\Rightarrow BC\perp AM\Rightarrow AM\perp(SBC)\Rightarrow AM\perp SC$. Tương tự ta cũng có $AN\perp SC\Rightarrow(AMN)\perp SC$. Gọi $\varphi$ là góc giữa đường thẳng $SB$ và $(AMN)$.\\
            Chuẩn hóa và chọn hệ trục tọa độ sao cho $A(0;0;0)$, $B(0;1;0)$, $D(1;0;0)$, $S\left(0;0;\sqrt{2}\right)$, $C(1;1;0)$,\\
            $\overrightarrow{SC}=\left(1;1;-\sqrt{2}\right)$, $\overrightarrow{SB}=\left(0;1;-\sqrt{2}\right)$.\\
            Do $(AMN)\perp SC$ nên $(AMN)$ có véc-tơ pháp tuyến $\overrightarrow{SC}$.\\
            $\sin\varphi=\dfrac{|3|}{2\sqrt{3}} =\dfrac{\sqrt{3}}{2}\Rightarrow\varphi=60^{\circ}$.
        }{
            \begin{tikzpicture}[line join=round,line cap=round, font=\footnotesize,scale=1,>=stealth]
                \def\r{3}
                \path
                (0,0) coordinate (A)
                (\r,0) coordinate (D)
                (90:0.9*\r) coordinate (S)
                (235:0.55*\r) coordinate (B)
                ($(B)+(D)$) coordinate (C)
                ($(S)!0.45!(B)$) coordinate (M)
                ($(S)!0.45!(D)$) coordinate (N)
                ($(A)!1.3!(S)$) coordinate (z)
                ($(A)!1.3!(D)$) coordinate (x)
                ($(A)!1.3!(B)$) coordinate (y)
                ;
                \draw (S)--(D)-- (C) --(B)--(S)--(C)
                pic [draw,angle radius=2mm] {right angle =B--M--A}
                pic [draw,angle radius=2mm] {right angle =A--N--D};
                \draw[dashed] (D)--(A)--(B)(M)--(A)--(N)--(M)(S)--(A)(B)--(D);
                \draw[->] (S)--(z)node[right]{$z$};
                \draw[->] (D)--(x)node[above]{$x$};
                \draw[->] (B)--(y)node[right]{$y$};
                \foreach \x/\g in {S/170,A/160,B/160,C/0,D/60,M/150,N/40}\fill[black] (\x) circle(1pt)+(\g:0.3)node{$\x$};
            \end{tikzpicture}
        }
    }
\end{ex}

\begin{ex}[Sở Bắc Giang - 2018]%[1H3K3-3]%Câu 77.
    Cho hình chóp $S.ABCD$ có đáy $ABCD$ là hình chữ nhật, $AB=a$, $BC=a\sqrt{3}$, $SA=a$ và $SA$ vuông góc với đáy $ABCD$. Tính $\sin\alpha$, với $\alpha$ là góc tạo bởi giữa đường thẳng $BD$ và mặt phẳng $(SBC)$.
    \choice
    {$\sin\alpha=\dfrac{\sqrt{7}}{8}$}
    {$\sin\alpha=\dfrac{\sqrt{3}}{2}$}
    {\True $\sin\alpha=\dfrac{\sqrt{2}}{4}$}
    {$\sin\alpha=\dfrac{\sqrt{3}}{5}$}
    \loigiai{
        \immini{
            Đặt hệ trục tọa độ $Oxyz$ như hình vẽ. Khi đó, ta có $A(0;0;0)$, $B(a;0;0)$, $D\left(0;a\sqrt{3};0\right)$, $S(0;0;a)$.\\
            Ta có $\overrightarrow{BD}=\left(-a;a\sqrt{3};0\right)=a\left(-1;\sqrt{3};0\right)$, nên đường thẳng $BD$ có véc-tơ chỉ phương là $\overrightarrow{u}=\left(-1;\sqrt{3};0\right)$.\\
            Ta có $\overrightarrow{SB}=(a;0;-a)$, $\overrightarrow{BC}=\left(0;a\sqrt{3};0\right)\Rightarrow\left[\overrightarrow{SB},\overrightarrow{BC}\right]=\left(a^2\sqrt{3};0;a^2\sqrt{3}\right) =a^2\sqrt{3}(1;0;1)$.\\
            Như vậy, mặt phẳng $(SBC)$ có véc-tơ pháp tuyến là $\overrightarrow{n}=(1;0;1)$.\\
            Do đó, $\alpha$ là góc tạo bởi giữa đường thẳng $BD$ và mặt phẳng $(SBC)$ thì\\
            $\sin\alpha=\dfrac{\left|\overrightarrow{u}\cdot\overrightarrow{n}\right|}{\left|\overrightarrow{u}\right|\cdot\left|\overrightarrow{n}\right|} =\dfrac{\left|(-1)\cdot 1+\sqrt{3}\cdot 0+0\cdot 1\right|}{\sqrt{(-1)^2+\sqrt{3}^2+0^2}\cdot\sqrt{1^2+0^2+1^2}} =\dfrac{\sqrt{2}}{4}$.
        }{
            \begin{tikzpicture}[line join=round,line cap=round, font=\footnotesize,scale=1,>=stealth]
                \def\r{2.5}
                \path
                (0,0) coordinate (A)
                (\r,0) coordinate (B)
                (90:0.9*\r) coordinate (S)
                (235:0.55*\r) coordinate (D)
                ($(B)+(D)$) coordinate (C)
                ($(A)!1.3!(S)$) coordinate (z)
                ($(A)!1.3!(D)$) coordinate (y)
                ($(A)!1.3!(B)$) coordinate (x)
                ;
                \draw (S)--(B)-- (C) --(D)--(S)--(C);
                \draw[dashed] (D)--(A)--(B)(S)--(A)(B)--(D);
                \draw[->] (S)--(z)node[right]{$z$};
                \draw[->] (B)--(x)node[above]{$x$};
                \draw[->] (D)--(y)node[right]{$y$};
                \foreach \x/\g in {S/170,A/160,B/60,C/0,D/160}\fill[black] (\x) circle(1pt)+(\g:0.3)node{$\x$};
            \end{tikzpicture}
        }
        \noindent
    }
\end{ex}

\begin{ex}[Chuyên ĐHSPHN - 2018]%[1H3K3-3]%Câu 78.
    Cho hình chóp $S.ABC$ có đáy là tam giác vuông tại $B$, cạnh bên $SA$ vuông góc với mặt phẳng đáy, $AB=2a$, $\widehat{BAC}=60^{\circ}$ và $SA=a\sqrt{2}$. Góc giữa đường thẳng $SB$ và mặt phẳng $(SAC)$ bằng
    \choice
    {$30^{\circ}$}
    {\True $45^{\circ}$}
    {$60^{\circ}$}
    {$90^{\circ}$}
    \loigiai{
        \immini{
            Trong mặt phẳng $(ABC)$ kẻ $BH\perp AC$.\\
            Mà $BH\perp SA\Rightarrow BH\perp(SAC)$.\\
            Góc giữa đường thẳng $SB$ và mặt phẳng $(SAC)$ bằng $\widehat{BSH}$.\\
            Xét tam giác $ABH$ vuông tại $H$, $BH=AB\cdot\sin 60^{\circ} =2a\cdot\dfrac{\sqrt{3}}{2} =a\sqrt{3}$.\\
            $AH=AB\cdot\cos 60^{\circ} =2a\cdot\dfrac{1}{2} =a$.\\
            Xét tam giác $SAH$ vuông tại $S$ có\\
            $SH=\sqrt{SA^2+AH^2} =\sqrt{(a\sqrt{2})^2+a^2} =a\sqrt{3}$.
        }{
            \begin{tikzpicture}[line join=round,line cap=round, font=\footnotesize,scale=1,>=stealth]
                \def\r{3}
                \path
                (0,0) coordinate (A)
                (\r,0) coordinate (B)
                (90:0.8*\r) coordinate (S)
                (-40:0.7*\r) coordinate (C)
                ($(A)!0.6!(C)$) coordinate (H)
                ;
                \draw (A)--(S)-- (B) --(C)--(A)(C)--(S)--(H)
                pic [draw,angle radius=2mm] {right angle =B--H--C};
                \draw[dashed] (A)--(B)--(H);
                \foreach \x/\g in {S/90,A/160,B/0,C/0,H/200}\fill[black] (\x) circle(1pt)+(\g:0.3)node{$\x$};
            \end{tikzpicture}
        }
        \noindent Xét tam giác $SBH$ vuông tại $H$ có $SH=HB=a\sqrt{3}$ suy ra tam giác $SBH$ vuông tại $H$.\\
        Vậy $\widehat{BSH}=45^{\circ}$.
    }
\end{ex}

\begin{ex}[Chuyên Vĩnh Phúc - 2018]%[1H3K3-3]%Câu 79.
    Cho hình chóp tứ giác đều $S.ABCD$ có cạnh đáy bằng $a$, tâm $O$. Gọi $M$ và $N$ lần lượt là trung điểm của $SA$ và $BC$. Biết rằng góc giữa $MN$ và $(ABCD)$ bằng $60^{\circ}$, cosin góc giữa $MN$ và mặt phẳng $(SBD)$ bằng
    \choice
    {$\dfrac{\sqrt{41}}{41}$}
    {$\dfrac{\sqrt{5}}{5}$}
    {\True $\dfrac{2\sqrt{5}}{5}$}
    {$\dfrac{2\sqrt{41}}{41}$}
    \loigiai{
        \immini{
            Gọi $E$, $F$ lần lượt là trung điểm $SO$, $OB$ thì $EF$ là hình chiếu của $MN$ trên $(SBD)$.\\
            Gọi $P$ là trung điểm $OA$ thì $PN$ là hình chiếu của $MN$ trên $(ABCD)$.\\
            Theo bài ra: $\widehat{MNP}=60^{\circ}$.\\
            Áp dụng định lý cos trong tam giác $CNP$ ta được:\\
            $\begin{aligned}
                NP^2=\,&CP^2+CN^2-2CP\cdot CN\cdot\cos 45^{\circ}\\
                =&\left(\dfrac{3a\sqrt{2}}{4}\right)^2+\dfrac{a^2}{4}-2\cdot\dfrac{3a\sqrt{2}}{4}\cdot\dfrac{a}{2}\cdot\dfrac{\sqrt{2}}{2}=\dfrac{5a^2}{8}.\end{aligned}$
        }{
            \begin{tikzpicture}[line join=round,line cap=round, font=\footnotesize,scale=1,>=stealth]
                \def\r{1.8}
                \path
                (0,0) coordinate (O)
                (-\r,0) coordinate (m)
                (\r,0) coordinate (N)
                (10:1.5*\r) coordinate (B)
                (90:2*\r) coordinate (S)
                ($(m)+(B)-(N)$) coordinate (A)
                ($(S)!0.5!(A)$) coordinate (M)
                ($(B)!2!(O)$) coordinate (D)
                ($(A)!2!(O)$) coordinate (C)
                ($(S)!0.5!(O)$) coordinate (E)
                ($(B)!0.5!(O)$) coordinate (F)
                ($(A)!0.5!(O)$) coordinate (P)
                (intersection of E--F and M--N) coordinate (I)
                ;
                \draw (S)--(B)-- (C) --(D)--(S)(S)--(C);
                \draw[dashed] (S)--(A)--(B)(A)--(D)(A)--(C)(S)--(O)(B)--(D)(E)--(F)(M)--(N)(M)--(P)--(N);
                \foreach \x/\g in {S/90,A/180,B/0,C/-80,D/180,O/-90,E/60,F/160,M/200,N/-70,P/220,I/80}\fill[black] (\x) circle(1pt)+(\g:0.3)node{$\x$};
            \end{tikzpicture}
        }
        \noindent Suy ra: $NP=\dfrac{a\sqrt{10}}{4}$, $MP=NP\cdot\tan 60^{\circ}=\dfrac{a\sqrt{30}}{4}$; $SO=2MP=\dfrac{a\sqrt{30}}{2}$.\\
        $SB=\sqrt{SO^2+OB^2}=2a\sqrt{2}\Rightarrow EF=a\sqrt{2}$.\\
        Ta lại có: $MENF$ là hình bình hành (vì $ME$ và $NF$ song song và cùng bằng $\dfrac{1}{2}OA$).\\
        Gọi $I$ là giao điểm của $MN$ và $EF$, khi đó góc giữa $MN$ và mặt phẳng $(SBD)$ là $\widehat{NIF}$.\\
        $\cos\widehat{NIF}=\dfrac{IK}{IN}=\dfrac{a\sqrt{2}}{2}\cdot\dfrac{4}{a\sqrt{10}}=\dfrac{2\sqrt{5}}{5}$.
    }
\end{ex}

\begin{ex}[Chuyên Vinh -2018]%[1H3K3-3]%Câu 80.
    \immini{Cho hình chóp $S.ABCD$ có đáy $ABCD$ là hình bình hành, $AB=2a$, $BC=a$, $\widehat{ABC}=120^{\circ}$. Cạnh bên $SD=a\sqrt{3}$ và $SD$ vuông góc với mặt phẳng đáy (tham khảo hình vẽ bên). Tính $\sin$ của góc tạo bởi $SB$ và mặt phẳng $(SAC)$
        \choice
        {$\dfrac{3}{4}$}
        {$\dfrac{\sqrt{3}}{4}$}
        {\True $\dfrac{1}{4}$}
        {$\dfrac{\sqrt{3}}{7}$}
    }{
        \begin{tikzpicture}[line join=round,line cap=round, font=\footnotesize,scale=0.75,>=stealth]
            \def\r{2.5}
            \path
            (0,0) coordinate (D)
            (\r,0) coordinate (A)
            (90:0.9*\r) coordinate (S)
            (235:0.55*\r) coordinate (C)
            ($(A)+(C)$) coordinate (B)
            ;
            \draw (S)--(A)--(B)-- (C) --(S)--(B);
            \draw[dashed] (C)--(D)--(A)(S)--(D);
            \foreach \x/\g in {S/90,A/0,B/0,C/180,D/180}\fill[black] (\x) circle(1pt)+(\g:0.3)node{$\x$};
        \end{tikzpicture}
    }
    \loigiai{
        \immini{
            Ta có $\sin\left(SB,(SAC)\right)=\dfrac{\mathrm{d}\left(B,(SAC)\right)}{SB} =\dfrac{\mathrm{d}(D, SAC)}{SB}$.\\
            Xét tam giác $ABC$ ta có\\
            $AC=\sqrt{BA^2+BC^2-2BA\cdot BC\cdot\cos\widehat{BAC}} =a\sqrt{7}$.\\
            $BO=\sqrt{\dfrac{BA^2+BC^2}{2}-\dfrac{AC^2}{4}} =\sqrt{\dfrac{4a^2+a^2}{2}-\dfrac{7a^2}{4}}=\dfrac{a\sqrt{3}}{2}$ \\
            $\Rightarrow BD=a\sqrt{3} $ và $SB=\sqrt{SD^2+BD^2} =\sqrt{3a^2+3a^2} =a\sqrt{6}$.\\
            Xét tam giác $ADC$ ta có
        }{
            \begin{tikzpicture}[line join=round,line cap=round, font=\footnotesize,scale=0.8,>=stealth]
                \def\r{3}
                \path
                (0,0) coordinate (D)
                (\r,0) coordinate (A)
                (90:0.9*\r) coordinate (S)
                (235:0.55*\r) coordinate (C)
                ($(A)+(C)$) coordinate (B)
                ($(C)!0.4!(A)$) coordinate (K)
                ($(S)!0.6!(K)$) coordinate (I)
                ;
                \draw (S)--(A)--(B)-- (C) --(S)--(B)
                pic [draw,angle radius=2mm] {right angle =D--K--C}
                pic [draw,angle radius=2mm] {right angle =D--I--K};
                \draw[dashed] (C)--(D)--(A)(S)--(D)(A)--(C)(B)--(D)--(K)--(S)(D)--(I);
                \foreach \x/\g in {S/90,A/0,B/0,C/180,D/180,I/60,K/-80}\fill[black] (\x) circle(1pt)+(\g:0.3)node{$\x$};
            \end{tikzpicture}
        }
        $\dfrac{AD}{\sin C}=\dfrac{AC}{\sin D}\Rightarrow\sin C=\dfrac{AD\cdot\sin D}{AC} =\dfrac{a\cdot\sin 120^{\circ}}{a\sqrt{7}}=\dfrac{\sqrt{21}}{14}$.\\
        Gọi $K$ là hình chiếu của $D$ lên $AC$, và $I$ là hình chiếu của $D$ lên $SK$.\\
        Ta có $\heva{&AC\perp DK\\&AC\perp SD}\Rightarrow AC\perp DI$. Do đó $\heva{&DI\perp SK\\&DI\perp AC}\Rightarrow\mathrm{d}\left(D,(SAC)\right)=DI$.\\
        Mặt khác $\sin C=\dfrac{DK}{DC}\Rightarrow DK=DC\cdot \sin C =2a\cdot\dfrac{\sqrt{21}}{14} =\dfrac{a\sqrt{21}}{7}$.\\
        Xét tam giác $SDK$ ta có $DI=\dfrac{SD\cdot DK}{\sqrt{SD^2+DK^2}} =\dfrac{a\sqrt{3}\cdot\dfrac{a\sqrt{21}}{7}}{\sqrt{3a^2+\dfrac{21}{49}a^2}} =\dfrac{\sqrt{6}}{4}a$.\\
        Vậy $\sin\left(SB,(SAC)\right)=\dfrac{\mathrm{d}(D, SAC)}{SB} =\dfrac{DI}{SB} =\dfrac{\dfrac{\sqrt{6}}{4}a}{a\sqrt{6}}=\dfrac{1}{4}$.\\
        Trong mặt phẳng $(SDK)$ kẻ $DI\perp SK$ suy ra $\mathrm{d}\left(D,(SAC)\right)=DI$.
    }
\end{ex}

\begin{ex}[Chuyên Lương Văn Chánh - Phú Yên - 2020]%[1H3K3-3]%Câu 81.
    \immini{Cho hình chóp $S.ABCD$ có $SA$ vuông góc với mặt phẳng đáy, $SA=a\sqrt{3}$, tứ giác $ABCD$ là hình vuông, $BD=a\sqrt{2}$ (minh họa như hình bên). Góc giữa đường thẳng $SB$ và mặt phẳng $(SAD)$ bằng
        \choice
        {$0^{\circ}$}
        {\True $30^{\circ}$}
        {$45^{\circ}$}
        {$60^{\circ}$}
    }{
        \begin{tikzpicture}[line join=round,line cap=round, font=\footnotesize,scale=0.75,>=stealth]
            \def\r{2.5}
            \path
            (0,0) coordinate (A)
            (\r,0) coordinate (B)
            (90:0.9*\r) coordinate (S)
            (235:0.55*\r) coordinate (D)
            ($(B)+(D)$) coordinate (C)
            ;
            \draw (S)--(B)-- (C) --(D)--(S)--(C);
            \draw[dashed] (D)--(A)--(B)(S)--(A);
            \foreach \x/\g in {S/170,A/160,B/60,C/0,D/160}\fill[black] (\x) circle(1pt)+(\g:0.3)node{$\x$};
        \end{tikzpicture}
    }
    \loigiai{
        \immini{
            Đáy $ABCD$ là hình vuông có đường chéo $BD=a\sqrt{2}$ nên cạnh $AB=a$.\\
            Ta có $\heva{&AB\perp AD\\&AB\perp SA}\Rightarrow AB\perp(SAD)\Rightarrow SA$ là hình chiếu của $SB$ trên mặt phẳng $(SAD)$ \\
            $\Rightarrow\left(SB,(SAD)\right)=\left(SB, SA\right)=\widehat{BSA}$.\\
            Trong tam giác vuông $BSA$, ta có $\tan\widehat{BSA}=\dfrac{AB}{AS}=\dfrac{a}{a\sqrt{3}}=\dfrac{\sqrt{3}}{3}\Rightarrow\widehat{BSA}=30^{\circ}$.\\
            Vậy $\left(SB,(SAD)\right)=30^{\circ}$.
        }{
            \begin{tikzpicture}[line join=round,line cap=round, font=\footnotesize,scale=1,>=stealth]
                \def\r{3}
                \path
                (0,0) coordinate (A)
                (\r,0) coordinate (B)
                (90:0.9*\r) coordinate (S)
                (235:0.55*\r) coordinate (D)
                ($(B)+(D)$) coordinate (C)
                ;
                \draw (S)--(B)-- (C) --(D)--(S)--(C);
                \draw[dashed] (D)--(A)--(B)(S)--(A)(B)--(D);
                \foreach \x/\g in {S/170,A/160,B/60,C/0,D/160}\fill[black] (\x) circle(1pt)+(\g:0.3)node{$\x$};
            \end{tikzpicture}
        }
    }
\end{ex}

\begin{ex}[Chuyên Thái Bình - 2020]%[1H3K3-3]%Câu 82.
    Cho hình chóp tứ giác đều $S.ABCD$ có đáy là hình vuông tâm $O$, cạnh $a$. Gọi $M,N$ lần lượt là trung điểm của $SA$ và $BC$. Góc giữa đường thẳng $MN$ và mặt phẳng $(ABCD)$ bằng $60^{\circ}$. Tính cos của góc giữa đường thẳng $MN$ và mặt phẳng $(SBD)$.
    \choice
    {$\dfrac{\sqrt{41}}{4}$}
    {$\dfrac{\sqrt{5}}{5}$}
    {\True $\dfrac{2\sqrt{5}}{5}$}
    {$\dfrac{2\sqrt{41}}{4}$}
    \loigiai{
        \immini{
            Từ giả thiết ta có $SO\perp(ABCD)$.\\
            Gọi $I$ là trung điểm $OA$ thì $MI$ là đường trung bình của $\triangle SOA\Rightarrow MI\parallel SO\Rightarrow MI\perp(ABCD)$ \\
            $\Rightarrow I $ là hình chiếu của $M$ trên mặt phẳng $(ABCD)\Rightarrow IN$ là hình chiếu của $MN$ trên mặt phẳng $(ABCD)$. Suy ra $\left(\widehat{MN,(ABCD)}\right)=\left(\widehat{MN,IN}\right)\Rightarrow\widehat{MNI}=60^{\circ}$.\\
            Ta có $NC=\dfrac{1}{2}BC=\dfrac{a}{2}$; $IC=\dfrac{3}{4}AC=\dfrac{3a\sqrt{2}}{4}$.\\
            Áp dụng định lý cosin trong $\triangle INC$ ta có
        }{
            \begin{tikzpicture}[line join=round,line cap=round, font=\footnotesize,scale=1,>=stealth]
                \def\r{1.8}
                \path
                (0,0) coordinate (O)
                (-\r,0) coordinate (m)
                (\r,0) coordinate (N)
                (10:1.5*\r) coordinate (B)
                (90:2*\r) coordinate (S)
                ($(m)+(B)-(N)$) coordinate (A)
                ($(S)!0.5!(A)$) coordinate (M)
                ($(B)!2!(O)$) coordinate (D)
                ($(A)!2!(O)$) coordinate (C)
                ($(S)!0.5!(O)$) coordinate (F)
                ($(B)!0.5!(O)$) coordinate (E)
                ($(A)!0.5!(O)$) coordinate (I)
                ($(M)!0.5!(N)$) coordinate (J)
                ;
                \draw (S)--(B)-- (C) --(D)--(S)(S)--(C);
                \draw[dashed] (S)--(A)--(B)(A)--(D)(A)--(C)(S)--(O)(B)--(D)(E)--(F)(M)--(N)(F)--(M)--(I)--(N);
                \foreach \x/\g in {S/90,A/180,B/0,C/-80,D/180,O/-90,E/160,F/60,M/200,N/-70,I/200,J/80}\fill[black] (\x) circle(1pt)+(\g:0.3)node{$\x$};
            \end{tikzpicture}
        }
        \noindent $IN^2=CI^2+CN^2-2CI\cdot CN\cdot\cos\widehat{NCI}$ \\
        $ \Rightarrow IN^2=\left(\dfrac{3a\sqrt{2}}{4}\right)^2+\left(\dfrac{a}{2}\right)^2-2\cdot\dfrac{3a\sqrt{2}}{4}\cdot\dfrac{a}{2}\cdot\cos 45^{\circ}=\dfrac{5a^2}{8}\Rightarrow IN=\dfrac{a\sqrt{10}}{4} $.\\
        Do $\triangle MIN$ vuông tại $I$ nên $\cos\widehat{MNI}=\dfrac{IN}{MN}\Rightarrow MN=\dfrac{IN}{\cos 60^{\circ}}=\dfrac{a\sqrt{10}}{4}\colon\dfrac{1}{2}=\dfrac{a\sqrt{10}}{2}$.\\
        Lại có $AC\perp BD,AC\perp SO\Rightarrow AC\perp(SBD)$.\\
        Gọi $E$ là trung điểm $OB\Rightarrow EN$ là đường trung bình của $\triangle BOC\Rightarrow EN\parallel OC$ hay $EN\parallel AC$ \\
        $\Rightarrow NE\perp(SBD) $ hay $E$ là hình chiếu của $N$ trên mặt phẳng $(SBD)$.\\
        Gọi $F$ là trung điểm của $SO\Rightarrow MF$ là đường trung bình của $\triangle SAO\Rightarrow MF\parallel AO$ hay $MF\parallel AC$ \\
        $\Rightarrow MF\perp(SBD) $ hay $F$ là hình chiếu của $M$ trên mặt phẳng $(SBD)$.\\
        Ta có $MF\parallel NE$ nên bốn điểm $E,N,F,M$ cùng nằm trên một mặt phẳng.\\
        Trong mặt phẳng $(ENFM)$ gọi $J=MN\cap EF\Rightarrow J=MN\cap(SBD)$ (do $EF\subset(SBD)$).\\
        Suy ra $\left(\widehat{MN,(SBD)}\right)=\left(\widehat{MN,EF}\right)=\widehat{EJN}$ (do $\widehat{EJN}<90^{\circ}$).\\
        Ta có $EN=\dfrac{1}{2}OC=\dfrac{1}{4}AC=\dfrac{a\sqrt{2}}{4}$; $MF=\dfrac{1}{2}AO=\dfrac{1}{4}AC=\dfrac{a\sqrt{2}}{4}\Rightarrow EN=MF$, mà $EN\parallel MF$ \\
        $\Rightarrow $ Tứ giác $ENFM$ là hình bình hành $\Rightarrow J$ là trung điểm $MN\Rightarrow JN=\dfrac{1}{2}MN=\dfrac{a\sqrt{10}}{4}$.\\
        Vậy $\cos\left(MN,(SBD)\right)=\cos\widehat{EJN}=\dfrac{JE}{JN}=\dfrac{\sqrt{\left(\dfrac{a\sqrt{10}}{4}\right)^2-\left(\dfrac{a\sqrt{2}}{4}\right)^2}}{\dfrac{a\sqrt{10}}{4}} =\dfrac{2\sqrt{5}}{5}$.
    }
\end{ex}

\begin{ex}[Đô Lương 4 - Nghệ An - 2020]%[1H3K3-3]%Câu 83.
    Cho hình chóp tứ giác đều $S.ABCD$ có cạnh đáy bằng $a$, tâm $O$. Gọi $M$ và $N$ lần lượt là trung điểm của $SA$ và $BC$. Biết rằng góc giữa $MN$ và $(ABCD)$ bằng $60^{\circ}$, côsin của góc giữa đường thẳng $MN$ và mặt phẳng $(SBD)$ bằng
    \choice
    {$\dfrac{\sqrt{5}}{5}$}
    {$\dfrac{\sqrt{41}}{41}$}
    {\True $\dfrac{2\sqrt{5}}{5}$}
    {$\dfrac{2\sqrt{41}}{41}$}
    \loigiai{
        \immini{
            Chọn hệ trục tọa độ $Oxyz$ như hình vẽ.\\
            Đặt $SO=m,(m>0)$.\\
            $A\left(\dfrac{a\sqrt{2}}{2};0;0\right); S(0;0;m); N\left(-\dfrac{a\sqrt{2}}{4};\dfrac{a\sqrt{2}}{4};0\right)$\\
            $\Rightarrow M\left(\dfrac{a\sqrt{2}}{4}; 0;\dfrac{m}{2}\right)$. $\Rightarrow\overrightarrow{MN}=\left(-\dfrac{a\sqrt{2}}{2};\dfrac{a\sqrt{2}}{4};-\dfrac{m}{2}\right)$.\\
            Mặt phẳng $(ABCD)$ có véc-tơ pháp tuyến $\overrightarrow{k}=(0;0;1)$
        }{
            \begin{tikzpicture}[line join=round,line cap=round, font=\footnotesize,scale=1,>=stealth]
                \def\r{1.5}
                \path
                (0,0) coordinate (O)
                (-\r,0) coordinate (m)
                (\r,0) coordinate (N)
                (10:1.5*\r) coordinate (B)
                (90:1.7*\r) coordinate (S)
                ($(m)+(B)-(N)$) coordinate (A)
                ($(S)!0.5!(A)$) coordinate (M)
                ($(B)!2!(O)$) coordinate (D)
                ($(A)!2!(O)$) coordinate (C)
                ($(O)!1.5!(A)$) coordinate (x)
                ($(O)!1.3!(B)$) coordinate (y)
                ($(O)!1.3!(S)$) coordinate (z)
                ;
                \draw (S)--(B)-- (C) --(D)--(S)(S)--(C);
                \draw[dashed] (S)--(A)--(B)(A)--(D)(A)--(C)(S)--(O)(B)--(D)(M)--(N);
                \draw[dashed,->] (A)--(x)node[above]{$x$};
                \draw[->] (B)--(y)node[above]{$y$};
                \draw[->] (S)--(z)node[right]{$z$};
                \foreach \x/\g in {S/30,A/-120,B/-50,C/-80,D/180,O/-90,M/200,N/-70}\fill[black] (\x) circle(1pt)+(\g:0.3)node{$\x$};
            \end{tikzpicture}
        }
        \noindent Ta có $\sin\left(MN,(ABCD)\right)=\dfrac{\left|\overrightarrow{MN}\cdot\overrightarrow{k}\right|}{\left|\overrightarrow{MN}\right|\left|\overrightarrow{k}\right|}=\dfrac{\dfrac{m}{2}}{\sqrt{\dfrac{5a^2}{8}+\dfrac{m^2}{4}}}=\dfrac{\sqrt{3}}{2}\Leftrightarrow m^2=\dfrac{15a^2}{8}+\dfrac{3m^2}{4}$ \\
        $ \Rightarrow 2m^2=15a^2\Rightarrow m=\dfrac{a\sqrt{30}}{2}$.\\
        Nên $\overrightarrow{MN}=\left(-\dfrac{a\sqrt{2}}{2};\dfrac{a\sqrt{2}}{4};-\dfrac{a\sqrt{30}}{4}\right)$, mặt phẳng $(SBD)$ có véc-tơ pháp tuyến là $\overrightarrow{i}=(1;0;0)$.\\
        $\Rightarrow\sin\left(MN,(SBD)\right)=\dfrac{\left|\overrightarrow{MN}\cdot\overrightarrow{i}\right|}{\left|\overrightarrow{MN}\right|\left|\overrightarrow{i}\right|}=\dfrac{\dfrac{a\sqrt{2}}{2}}{\sqrt{\dfrac{a^2}{2}+\dfrac{a^2}{8}+\dfrac{30a^2}{16}}}=\dfrac{\sqrt{5}}{5}$.\\
        Suy ra $\cos\left(MN,(SBD)\right)=\dfrac{2\sqrt{5}}{5}$.
    }
\end{ex}
%%==========Câu 84
\begin{ex}[THPT Nguyễn Viết Xuân - 2020]%Câu 58.%[1H3G4-3]
    Cho hình lăng trụ đứng $ABC.A'B'C'$ có $AB=AC=a, \widehat{BAC}=120^{\circ}$. Gọi $M, N$ lần lượt là trung điểm của $B'C'$ và $CC'$. Biết thể tích khối lăng trụ $ABC.A'B'C'$ bằng $\dfrac{\sqrt{3}a^3}{4}$. Gọi $\alpha$ là góc giữa mặt phẳng $(AMN)$ và mặt phẳng $(ABC)$. Khi đó
    \choice
    {$\cos\alpha=\dfrac{\sqrt{3}}{2}$}
    {$\cos\alpha=\dfrac{1}{2}$}
    {$\cos\alpha=\dfrac{\sqrt{13}}{4}$}
    {\True $\cos\alpha=\dfrac{\sqrt{3}}{4}$}
    \loigiai{
        \immini{Ta có: $V_{ABC.A'BC'}=CC'\cdot S_{\triangle ABC}=\dfrac{\sqrt{3}a^3}{4}\\
            \Rightarrow CC'=a$ vì $S_{\triangle ABC}=\dfrac{\sqrt{3}a^2}{4}$.\\
            Chọn hệ trục tọa độ $Oxyz$ như hình vẽ. Ta có $M\equiv O$.\\
            $M(0;0;0), A'\left(\dfrac{a}{2};0;0\right), B'\left(0;\dfrac{\sqrt{3}a}{2};0\right), C'\left(0;-\dfrac{\sqrt{3}a}{2};0\right)$; $A\left(\dfrac{a}{2};0;a\right); N\left(0;-\dfrac{\sqrt{3}a}{2};\dfrac{a}{2}\right)$.\\
            Ta có: $(ABC)\perp Oz$ nên $(ABC)$ có một véc-tơ pháp tuyến là $\overrightarrow{k}=(0;0;1)$.\\
            Ta có $\overrightarrow{MA}=\left(\dfrac{a}{2};0;a\right)$, $\overrightarrow{MN}=\left(0;-\dfrac{\sqrt{3}a}{2};\dfrac{a}{2}\right)$.\\
            Gọi $\overrightarrow{v}_1=\dfrac{a}{2}\overrightarrow{MA}\Rightarrow\overrightarrow{v}_1=(1;0;2)$,\\ $\overrightarrow{v}_2=\dfrac{a}{2}\overrightarrow{MN}\Rightarrow\overrightarrow{v}_2=\left(0;-\sqrt{3};1\right)$.
        }{\begin{tikzpicture}[scale=1, font=\footnotesize, line join=round, line cap=round, >=stealth]
                \coordinate[label=above right:$M$] (M)at(0,0);
                \coordinate[label=left:$C'$] (C')at(-2,0);
                \coordinate[label=above right:$B'$] (B')at(2,0);
                \coordinate[label=right:$A'$] (A')at(-1,-2);
                \coordinate[label=left:$C$] (C)at(-2,4);
                \coordinate[label=right:$B$] (B)at(2,4);
                \coordinate[label=above:$A$] (A)at(-1,2);
                \coordinate[label=below:$y$] (y) at ($(M)!1.5!(B')$);
                \coordinate[label=left:$x$] (x) at ($(M)!1.5!(A')$);
                \coordinate (E) at ($(B)!0.5!(C)$);
                \coordinate[label=left:$z$] (z) at ($(M)!1.2!(E)$);
                \coordinate[label=left:$N$] (N) at ($(C)!0.5!(C')$);
                \draw (C')--(A')--(B')--(B)--(A)--(C)--cycle (B)--(C) (A')--(A)--(N);
                \draw[->](A')--(x) ;
                \draw[->](B')--(y);
                \draw[->](E)--(z);
                \draw[dashed](B')--(C') (A')--(M)--(E) (N)--(M)--(A);
                \foreach \diem in {A,B,C,A',B',C',M,N} \fill (\diem)circle(1.5pt);
        \end{tikzpicture}}
        \noindent
        Khi đó mặt phẳng $(AMN)$ song song hoặc chứa giá của hai véc-tơ không cùng phương là $\overrightarrow{v}_1$ và $\overrightarrow{v}_2$ nên có một véc-tơ pháp tuyến là $\overrightarrow{n}=\left[{\overrightarrow{v}}_1,{\overrightarrow{v}}_2\right]=\left(2\sqrt{3};-1;-\sqrt{3}\right)$.\\
        Vậy $\cos\alpha=\left|\cos\left(\overrightarrow{k},\overrightarrow{n}\right)\right|=\dfrac{\left|\overrightarrow{k}\cdot\overrightarrow{n}\right|}{\left|\overrightarrow{k}\right|\cdot\left|\overrightarrow{n}\right|}=\dfrac{\sqrt{3}}{4}$.
    }
\end{ex}
%%==========Câu 85
\begin{ex}[Chuyên Hạ Long - Quảng Ninh - 2020]%Câu 59.%[1H3G4-3]
    Cho hình chóp $S.ABCD$ có đáy $ABCD$ là hình vuông cạnh bằng $2a$. Tam giác $SAB$ cân tại $S$ và $(SAB)\perp(ABCD)$. Biết thể tích của khối chóp $S.ABCD$ là $\dfrac{4a^3}{3}$. Gọi $\alpha$ là góc giữa $SC$ và $(ABCD)$. Tính $\tan\alpha$.
    \choice
    {\True $\tan\alpha=\dfrac{\sqrt{5}}{5}$}
    {$\tan\alpha=\dfrac{2\sqrt{5}}{5}$}
    {$\tan\alpha=\dfrac{\sqrt{3}}{3}$}
    {$\tan\alpha=\dfrac{\sqrt{7}}{7}$}
    \loigiai{
        \immini{Gọi $H$ là trung điểm $AB$.\\
            Vì $\triangle SAB$ cân tại $S$ nên $SH\perp AB$.\\
            Và $\heva{&(SAB)\perp(ABCD)\\&(SAB)\cap(ABCD)=AB}$\\
            nên suy ra $SH\perp(ABCD)$.\\
            Khi đó ta có: $V_{S.ABCD}=\dfrac{1}{3}\cdot SH\cdot S_{ABCD}\\
            \Rightarrow SH=\dfrac{3V_{S.ABCD}}{S_{ABCD}} =\dfrac{3\cdot\dfrac{4a^3}{3}}{(2a)^2} =a$.
        }{
            \begin{tikzpicture}[scale=1, font=\footnotesize, line join=round, line cap=round, >=stealth]
                \def\bc{3.5} % cạnh BC
                \def\ba{2} % cạnh BA
                \def\h{4} % đường cao
                \def\gocB{25} % góc B của đáy
                \coordinate[label=below left:$B$] (B) at (0,0);
                \coordinate[label=above right:$A$] (A) at (\gocB:\ba);
                \coordinate[label=below:$C$] (C) at (\bc,0);
                \coordinate[label=right:$D$] (D) at ($(C)-(B)+(A)$);
                \coordinate[label=above left:$H$] (H) at ($(A)!1/2!(B)$);
                \coordinate[label=above left:$S$] (S) at ($(H)+(90:\h)$);
                \draw (B)--(C)--(D)--(S)--cycle (S)--(C);
                \draw[dashed] (A)--(D) (H)--(S)--(A)--(B) (H)--(C);
                \foreach \diem in {A,B,C,D,S,H}	\fill (\diem)circle(1.5pt);
                \draw pic[angle radius=5mm,draw=gray,fill=gray,opacity=1,"$\alpha$",angle eccentricity=1.5] {angle = S--C--H};
                \newcommand{\gocv}[4][black]{\draw[#1] ($(#3)!5pt!(#2)$)--($(#3)!2!($($(#3)!5pt!(#2)$)!.5!($(#3)!5pt!(#4)$)$)$)--($(#3)!5pt!(#4)$);}
                \gocv{S}{H}{A}
                \gocv{A}{B}{C}
            \end{tikzpicture}
        }
        \noindent
        Lại có $HC$ là cạnh huyền trong tam giác vuông $BHC$ nên $HC=\sqrt{BH^2+BC^2}=a\sqrt{5}$.\\
        Mặt khác, do $SH\perp(ABCD)$, $\left(H\in(ABCD)\right)$ nên $HC$ là hình chiếu của $SC$ lên mặt phẳng $(ABCD)$. Suy ra $\alpha=\widehat{\left(SC,(ABCD)\right)}=\widehat{SCH}$.\\
        Vậy, trong tam giác vuông $SHC$, $\tan\alpha=\tan\widehat{SCH}=\dfrac{SH}{HC} =\dfrac{a}{a\sqrt{5}} =\dfrac{\sqrt{5}}{5}$.
    }
\end{ex}
%%==========Câu 86
\begin{ex}[Chuyên Lê Hồng Phong - Nam Định - 2020]%Câu 60.%[1H3G4-3]
    Cho tứ diện đều $SABC$ cạnh $a$. Gọi $M,N$ lần lượt là trung điểm của các cạnh $AB,SC$. Tính tan của góc giữa đường thẳng $MN$ và mặt phẳng $(ABC)$.
    \choice
    {$\dfrac{\sqrt{3}}{2}$}
    {$\dfrac{1}{2}$}
    {\True $\dfrac{\sqrt{2}}{2}$}
    {$1$}
    \loigiai{
        \immini{Gọi $O$ là tâm đường tròn ngoại tiếp đáy.\\
            Vì $SABC$ là tứ diện đều cạnh $a$ nên $h=\dfrac{\sqrt{6}}{3}a$.\\
            Gọi $H$ là chân đường vuông góc từ $N$ xuống $(ABC)$ \\
            $ \Rightarrow H $ là trung điểm của $OC$ \\
            $ \Rightarrow MH=\dfrac{2}{3}MC=\dfrac{2}{3}\cdot\sqrt{a^2-\left(\dfrac{a}{2}\right)^2}=\dfrac{\sqrt{3}}{3}a $.\\
            Vì $N$ là trung điểm của $SC$ nên $NH=\dfrac{1}{2}h=\dfrac{\sqrt{6}}{6}a$.\\
            Góc giữa đường thẳng $MN$ và mặt phẳng $(ABC)$ là $\widehat{NMH}$.\\
            Vậy $\tan\widehat{NMH}=\dfrac{NH}{MH}=\left(\dfrac{\sqrt{6}}{6}a\right)\colon\left(\dfrac{\sqrt{3}}{3}a\right)=\dfrac{\sqrt{2}}{2}$.
        }{
            \begin{tikzpicture}[scale=1.2, font=\footnotesize, line join=round, line cap=round, >=stealth]
                \def\ac{4} % cạnh AC
                \def\ab{2} % cạnh AB
                \def\h{4} % chiều cao
                \def\gocA{50} % góc A của đáy
                \coordinate[label=left:$A$] (A) at (0,0);
                \coordinate[label=right:$C$] (C) at (\ac,0);
                \coordinate[label=below left:$B$] (B) at (-\gocA:\ab);
                \coordinate[label=below right:$D$] (D) at ($(B)!.5!(C)$);
                \coordinate[label=below:$O$] (O) at ($(A)!2/3!(D)$);
                \coordinate[label=above:$S$] (S) at ($(O)+(90:\h)$);
                \coordinate[label=left:$M$] (M) at ($(A)!0.5!(B)$);
                \coordinate[label=right:$N$] (N) at ($(S)!0.5!(C)$);
                \coordinate[label=below:$H$] (H) at ($(O)!0.5!(C)$);
                \draw (A)--(B)--(C)--(S)--cycle (S)--(B) (M)--(S)--(D);
                \draw[dashed] (D)--(A)--(C) (O)--(S) (N)--(M)--(C) (N)--(H);
                \foreach \diem in {A,B,C,S,O,D,M,H,N}	\fill (\diem)circle(1.5pt);
                \draw pic[angle radius=2mm,draw=black,angle eccentricity=1.5] {right angle = S--O--C};
                \draw pic[angle radius=2mm,draw=black,angle eccentricity=1.5] {right angle = N--H--C};
                \draw pic[angle radius=7mm,draw=blue] {angle = C--M--N};
            \end{tikzpicture}
        }
    }
\end{ex}
%%==========Câu 87
\begin{ex}[Mã 103 - 2022]%Câu 61.%[1H3B3-3]
    \immini{Cho hình lập phương $ABCD.A'B'C'D'$ (tham khảo hình bên). Giá trị sin của góc giữa đường thẳng $AC'$ và mặt phẳng $(ABCD)$ bằng
        \choice
        {\True $\dfrac{\sqrt{3}}{3}$}
        {$\dfrac{\sqrt{6}}{3}$}
        {$\dfrac{\sqrt{3}}{2}$}
        {$\dfrac{\sqrt{2}}{2}$}
    }{
        \begin{tikzpicture}[scale=.7, font=\footnotesize, line join=round, line cap=round, >=stealth]
            \def\bc{4} % cạnh BC
            \def\ba{2} % cạnh BA
            \def\gocB{35} % góc B của đáy
            \coordinate[label=below left:$B$] (B) at (0,0);
            \coordinate[label=above left:$A$] (A) at (\gocB:\ba);
            \coordinate[label=below:$C$] (C) at (\bc,0);
            \coordinate[label=right:$D$] (D) at ($(C)-(B)+(A)$);
            \coordinate[label=above left:$A'$] (A') at ($(A)+(90:\bc)$);
            \coordinate[label=left:$B'$] (B') at ($(B)-(A)+(A')$);
            \coordinate[label=below right:$C'$] (C') at ($(C)-(A)+(A')$);
            \coordinate[label=right:$D'$] (D') at ($(D)-(A)+(A')$);
            \draw (B')--(B)--(C)--(D)--(D')--(A')--(B')--(C')--(D') (C)--(C');
            \draw[dashed] (A')--(A)--(D) (A)--(B) (C)--(A)--(C');
            \foreach \diem in {A,B,C,D,A',B',C',D'}	\fill (\diem)circle(1.5pt);
        \end{tikzpicture}
    }
    \loigiai{
        Ta có $AC=CC'\sqrt{2}$ 	$\Rightarrow AC'=\sqrt{AC^2+CC'^2}=CC'\sqrt{3} $.\\
        Khi đó $\widehat{\left(AC';(ABCD)\right)}=\widehat{(AC';AC)}=\widehat{CAC'}$.\\
        Suy ra $\sin\widehat{CAC'}=\dfrac{CC'}{AC'}=\dfrac{CC'}{CC'\sqrt{3}}=\dfrac{\sqrt{3}}{3}$.
    }
\end{ex}
%%==========Câu 88
\begin{ex}[Mã 104-2022]%Câu 62.%[1H3B3-3]
    \immini{Cho hình lập phương $ABCD.A'B'C'D'$ (tham khảo hình vẽ bên). Giá trị sin của góc giữa đường thẳng $AC'$ và mặt phẳng $(ABCD)$ bằng
        \choice
        {\True $\dfrac{\sqrt{3}}{3}$}
        {$\dfrac{\sqrt{2}}{2}$}
        {$\dfrac{\sqrt{3}}{2}$}
        {$\dfrac{\sqrt{6}}{3}$}
    }{
        \begin{tikzpicture}[scale=.7, font=\footnotesize, line join=round, line cap=round, >=stealth]
            \def\bc{4} % cạnh BC
            \def\ba{2} % cạnh BA
            \def\gocB{35} % góc B của đáy
            \coordinate[label=below left:$B$] (B) at (0,0);
            \coordinate[label=above left:$A$] (A) at (\gocB:\ba);
            \coordinate[label=below:$C$] (C) at (\bc,0);
            \coordinate[label=right:$D$] (D) at ($(C)-(B)+(A)$);
            \coordinate[label=above left:$A'$] (A') at ($(A)+(90:\bc)$);
            \coordinate[label=left:$B'$] (B') at ($(B)-(A)+(A')$);
            \coordinate[label=below right:$C'$] (C') at ($(C)-(A)+(A')$);
            \coordinate[label=right:$D'$] (D') at ($(D)-(A)+(A')$);
            \draw (B')--(B)--(C)--(D)--(D')--(A')--(B')--(C')--(D') (C)--(C');
            \draw[dashed] (A')--(A)--(D) (A)--(B);
            \foreach \diem in {A,B,C,D,A',B',C',D'}	\fill (\diem)circle(1.5pt);
        \end{tikzpicture}
    }
    \loigiai{
        \immini{Hình chiếu của đường thẳng $AC'$ lên mặt phẳng $(ABCD)$ là đường thẳng $AC$ suy ra góc giữa đường thẳng $AC'$ và mặt phẳng $(ABCD)$, suy ra $\widehat{\left(CA',(ACBCD)\right)}=\widehat{(CA,CA')}=\widehat{CAC'}$.\\
            Gọi cạnh hình lập phương bằng 1, suy ra $AC=\sqrt{2}$.\\
            Xét tam giác vuông $CAC'$ vuông tại C ta có:\\
            $AC'=\sqrt{CC'^2+AC'^2}=\sqrt{(\sqrt{2})^2+1}=\sqrt{3}$.\\
            Suy ra: $\sin\widehat{(CA',(ABCD))}=\sin\widehat{CAC'}=\dfrac{CC'}{AC'}=\dfrac{\sqrt{3}}{3}$.
        }{\begin{tikzpicture}[scale=.7, font=\footnotesize, line join=round, line cap=round, >=stealth]
                \def\bc{4} % cạnh BC
                \def\ba{2} % cạnh BA
                \def\gocB{35} % góc B của đáy
                \coordinate[label=below left:$B$] (B) at (0,0);
                \coordinate[label=above left:$A$] (A) at (\gocB:\ba);
                \coordinate[label=below:$C$] (C) at (\bc,0);
                \coordinate[label=right:$D$] (D) at ($(C)-(B)+(A)$);
                \coordinate[label=above left:$A'$] (A') at ($(A)+(90:\bc)$);
                \coordinate[label=left:$B'$] (B') at ($(B)-(A)+(A')$);
                \coordinate[label=below right:$C'$] (C') at ($(C)-(A)+(A')$);
                \coordinate[label=right:$D'$] (D') at ($(D)-(A)+(A')$);
                \draw (B')--(B)--(C)--(D)--(D')--(A')--(B')--(C')--(D') (C)--(C');
                \draw[dashed] (A')--(A)--(D) (A)--(B) (C)--(A)--(C');
                \foreach \diem in {A,B,C,D,A',B',C',D'}	\fill (\diem)circle(1.5pt);
        \end{tikzpicture}}
    }
\end{ex}
%%==========Câu 89
\begin{dang}
	{Góc của mặt phẳng với mặt phẳng}
\end{dang}
\begin{ex}[Chuyên KHTN - 2021]%Câu 1.%[1H3B4-3]
    Cho hình lăng trụ tam giác đều $ABC.A'B'C'$ có cạnh đáy bằng $a$ và cạnh bên bằng $\dfrac{3a}{2}$. Góc giữa hai mặt phẳng $(A'BC)$ và $(ABC)$ bằng
    \choice
    {$30^{\circ}$}
    {\True $60^{\circ}$}
    {$45^{\circ}$}
    {$90^{\circ}$}
    \loigiai{
        \immini{Gọi $M$ là trung điểm của $BC\Rightarrow AM\perp BC$ (vì tam giác $ABC$ đều)\\
            $ \Rightarrow AM=\sqrt{AB^2-BM^2}=\sqrt{a^2-\dfrac{a^2}{4}}=\dfrac{a\sqrt{3}}{2} $.\\
            $\left((A'BC),(ABC)\right)=\widehat{AMA'}$.\\
            Lại có: $\tan\widehat{AMA'}=\dfrac{AA'}{AM}=\dfrac{\dfrac{3a}{2}}{\dfrac{a\sqrt{3}}{2}}=\sqrt{3}$. \\
            $ \Rightarrow\widehat{AMA'}=60^{\circ}\Leftrightarrow\left((A'BC),(ABC)\right)=60^{\circ}$.
        }{\begin{tikzpicture}[scale=1, font=\footnotesize, line join=round, line cap=round, >=stealth]
                \def\ac{4} % cạnh AC
                \def\ab{2} % cạnh AB
                \def\h{3} % chiều cao
                \def\gocA{50} % góc A của đáy
                \coordinate[label=left:$A$] (A) at (0,0);
                \coordinate[label=right:$C$] (C) at (\ac,0);
                \coordinate[label=below left:$B$] (B) at (-\gocA:\ab);
                \coordinate[label=left:$A'$] (A') at ($(A)+(90:\h)$);
                \coordinate[label=below right:$B'$] (B') at ($(B)-(A)+(A')$);
                \coordinate[label=right:$C'$] (C') at ($(C)-(A)+(A')$);
                \coordinate[label=right:$M$] (M) at ($(B)!0.5!(C)$);
                \draw (A')--(A)--(B)--(C)--(C')--(A')--(B')--(C') (A')--(B)--(B');
                \draw[dashed] (A)--(C)--(A')--(M)--cycle;
                \foreach \diem in {A,B,C,A',B',C',M} \fill (\diem)circle(1.5pt);
                \draw pic[angle radius=3mm,draw=blue] {right angle = B--M--A};

            \end{tikzpicture}
        }
    }
\end{ex}
%%==========Câu 90
\begin{ex}[Chuyên Quốc Học Huế - 2021]%Câu 2.%[1H3B4-3]
    Cho hình lập phương $ABCD.A'B'C'D'$ có $O,O'$ lần lượt là tâm của các hình vuông $ABCD$ và $A'B'C'D'$. Góc giữa hai mặt phẳng $(A'BD)$ và $(ABCD)$ bằng
    \choice
    {$\widehat{A'AD}$}
    {$\widehat{A'OC}$}
    {\True $\widehat{A'OA}$}
    {$\widehat{OA'A}$}
    \loigiai{
        \immini{Ta có $ABCD$ là hình vuông nên $AO\perp BD$, đồng thời $BD\perp A'A\Rightarrow BD\perp(A'AO)\Rightarrow BD\perp A'O$.\\
            Ta có $\heva{&(A'BD)\cap(ABCD)=BD\\&AO\perp BD\\&A'O\perp BD}\\
            \Rightarrow\left(\widehat{(A'BD);(ABCD)}\right)=\left(\widehat{A'O;AO}\right)=\widehat{A'OA}$.
        }{\begin{tikzpicture}[scale=1, font=\footnotesize, line join=round, line cap=round, >=stealth]
                \def\bc{4} % cạnh BC
                \def\ba{2} % cạnh BA
                \def\h{3} % đường cao
                \def\gocB{35} % góc B của đáy
                \coordinate[label=below left:$B$] (B) at (0,0);
                \coordinate[label=above left:$A$] (A) at (\gocB:\ba);
                \coordinate[label=below:$C$] (C) at (\bc,0);
                \coordinate[label=right:$D$] (D) at ($(C)-(B)+(A)$);
                \coordinate[label=above left:$A'$] (A') at ($(A)+(90:\h)$);
                \coordinate[label=left:$B'$] (B') at ($(B)-(A)+(A')$);
                \coordinate[label=below right:$C'$] (C') at ($(C)-(A)+(A')$);
                \coordinate[label=right:$D'$] (D') at ($(D)-(A)+(A')$);
                \coordinate[label=below:$O$] (O) at (intersection cs:first line={(A)--(C)}, second line={(B)--(D)});
                \draw (B')--(B)--(C)--(D)--(D')--(A')--(B')--(C')--(D') (C)--(C');
                \draw[dashed] (A')--(A)--(D) (A)--(B)--(D)--(A')--(B) (A')--(O) (A)--(C);
                \draw pic[angle radius=5mm,draw=blue] {angle = A'--O--A};

                \foreach \diem in {A,B,C,D,A',B',C',D',O}	\fill (\diem)circle(1.5pt);
            \end{tikzpicture}
        }
    }
\end{ex}
%%==========Câu 91
\begin{ex}[THPT Thanh Chương 1- Nghệ An - 2021]%Câu 3.%[1H3B4-3]
    Cho hình chóp tứ giác đều $S.ABCD$ có cạnh đáy bằng $2a$ cạnh bên bằng $\sqrt{5}a$. Góc giữa mặt bên và mặt phẳng đáy bằng
    \choice
    {\True $60^{\circ}$}
    {$30^{\circ}$}
    {$70^{\circ}$}
    {$45^{\circ}$}
    \loigiai{
        \immini{Gọi $O$ là tâm hình vuông $ABCD$. Khi đó $SO\perp(ABCD)$.\\
            Gọi $H$ là trung điểm cạnh $CD$. Ta có: $OH\perp CD$ và $HD=OH=\dfrac{CD}{2}=a$.\\
            Do $\triangle SCD$ cân tại $S$ nên $SH\perp CD$.\\
            Vậy góc giữa mặt bên $(SCD)$ và mặt phẳng $(ABCD)$ là góc $\widehat{SHO}$.\\
            Trong $\triangle SHD$ vuông tại $H$ ta có:\\
            $SH=\sqrt{SD^2-HD^2}=\sqrt{5a^2-a^2}=2a$.\\
            Khi đó $\cos\widehat{SHO}=\dfrac{OH}{SH}=\dfrac{a}{2a}=\dfrac{1}{2}\Rightarrow\widehat{SHO}=60^{\circ}$.
        }{
            \begin{tikzpicture}[scale=1, font=\footnotesize, line join=round, line cap=round, >=stealth]
                \def\bc{4} % cạnh BC
                \def\ba{2} % cạnh BA
                \def\h{4} % đường cao
                \def\gocB{30} % góc B của đáy
                \coordinate[label=below left:$B$] (B) at (0,0);
                \coordinate[label=above right:$A$] (A) at (\gocB:\ba);
                \coordinate[label=below:$C$] (C) at (\bc,0);
                \coordinate[label=right:$D$] (D) at ($(C)-(B)+(A)$);
                \coordinate[label=below:$O$] (O) at ($(A)!.5!(C)$);
                \coordinate[label=above:$S$] (S) at ($(O)+(90:\h)$);
                \coordinate[label=right:$H$] (H) at ($(C)!0.5!(D)$);
                \draw (B)--(C)--(D)--(S)--cycle (H)--(S)--(C);
                \draw[dashed] (C)--(A)--(D)--(B) (O)--(S)--(A)--(B) (O)--(H);
                \draw pic[angle radius=2mm,draw=blue] {right angle = D--H--S};
                \draw pic[angle radius=2mm,draw=blue] {right angle = C--H--O};
                \draw pic[angle radius=2mm,draw=blue] {right angle = S--O--H};
                \foreach \diem in {A,B,C,D,S,O,H}	\fill (\diem)circle(1.5pt);
            \end{tikzpicture}
        }
    }
\end{ex}
%%==========Câu 92Câu này trùng với câu 90.
%%==========Câu 93
\begin{ex}[Chuyên Lê Khiết - Quảng Ngãi - 2021]%Câu 5.%[1H3K4-3]
    Cho hình chóp $S.ABC$ có đáy $ABC$ là tam giác vuông cân tại $A$, cạnh $AC=a$, các cạnh bên $SA=SB=SC=\dfrac{a\sqrt{6}}{2}$. Tính góc tạo bởi mặt bên $(SAB)$ và mặt phẳng đáy $(ABC)$.
    \choice
    {$\dfrac{\pi}{6}$}
    {$\dfrac{\pi}{4}$}
    {$\arctan\sqrt{2}$}
    {\True $\arctan 2$}
    \loigiai{
        \immini{Gọi $H$ là trung điểm của $BC\\
            \Rightarrow HA=HB=HC=\dfrac{1}{2}BC=\dfrac{1}{2}a\sqrt{2}$.\\
            mà $SA=SB=SC=\dfrac{a\sqrt{6}}{2}$ nên $SH\perp BC$,\\
            $\triangle SHA=\triangle SHB=\triangle SHC$.\\
            suy ra $SH\perp(ABC)$.\\
            Kẻ $HI\perp AB\Rightarrow\left(\widehat{(SAB),(ABC)}\right)=\left(\widehat{SI,HI}\right)=\widehat{SIH}$.\\
            Ta có $HI=\dfrac{1}{2}AB=\dfrac{1}{2}AC=\dfrac{1}{2}a$ (do tam giác $ABH$ vuông cân tại $H$).\\
            $SH=\sqrt{SC^2-HC^2}=\sqrt{\left(\dfrac{a\sqrt{6}}{2}\right)^2-\left(\dfrac{a\sqrt{2}}{2}\right)^2}=a$.\\
            Xét tam giác $SIH$ vuông tại $H$, ta có\\
            $\tan\widehat{SIH}=\dfrac{SH}{IH}=\dfrac{a}{\dfrac{1}{2}a}=2\Rightarrow\widehat{SIH}=\arctan 2$.
        }{\begin{tikzpicture}[scale=1, font=\footnotesize, line join=round, line cap=round, >=stealth]
                \coordinate[label=left:$B$] (B)at(0,0);
                \coordinate[label=right:$C$] (C)at(5,0);
                \coordinate[label=below:$A$] (A)at(2,-2);
                \coordinate[label=above right:$H$] (H) at ($(B)!0.5!(C)$);
                \coordinate[label=left:$S$] (S) at ($(H)!1.5!90:(C)$);
                \coordinate[label=left:$I$] (I) at ($(A)!0.5!(B)$);

                \draw (S)--(B)--(A)--(C)--cycle (A)--(S)--(I);
                \draw[dashed](S)--(H)--(A) (B)--(C) (H)--(I);
                \foreach \diem in {A,B,C,S,H,I}	\fill (\diem)circle(1.5pt);
            \end{tikzpicture}
        }
    }
\end{ex}
%%==========Câu 94
\begin{ex}[Sở Cần Thơ - 2021]%Câu 6.%[1H3B4-3]
    Cho hình chóp $S.ABCD$ có đáy $ABCD$ là hình vuông cạnh $a$, $SA$ vuông góc với mặt phẳng $(ABCD)$ và $SA=\sqrt{3}a$. Gọi $\varphi$ là góc giữa hai mặt phẳng $(SBC)$ và $(ABCD)$. Giá trị $\tan\varphi$ là
    \choice
    {\True $\sqrt{3}$}
    {$\dfrac{\sqrt{3}}{3}$}
    {$\dfrac{\sqrt{6}}{2}$}
    {$\dfrac{\sqrt{3}}{2}$}
    \loigiai{
        \immini{Ta có\\
            $\heva{&(SBC)\cap(ABCD)=BC\\&SB\subset(SBC),SB\perp BC\\&AB\subset(ABCD),AB\perp BC}\\
            \Rightarrow\widehat{(SBC),(ABCD)}=\widehat{SB,AB}=\widehat{SBA}$.\\
            $\tan\varphi=\tan\widehat{SBA}=\dfrac{SA}{AB}=\dfrac{\sqrt{3}a}{a}=\sqrt{3}$.
        }{\begin{tikzpicture}[scale=1, font=\footnotesize, line join=round, line cap=round, >=stealth]
                \def\bc{4} % cạnh BC
                \def\ba{2} % cạnh BA
                \def\h{3} % đường cao
                \def\gocB{30} % góc B của đáy
                \coordinate[label=below left:$B$] (B) at (0,0);
                \coordinate[label=above left:$A$] (A) at (\gocB:\ba);
                \coordinate[label=below:$C$] (C) at (\bc,0);
                \coordinate[label=right:$D$] (D) at ($(C)-(B)+(A)$);
                \coordinate[label=above:$S$] (S) at ($(A)+(90:\h)$);
                \draw (B)--(C)--(D)--(S)--cycle (S)--(C);
                \draw[dashed] (A)--(D) (S)--(A)--(B);
                \foreach \diem in {A,B,C,D,S}	\fill (\diem)circle(1.5pt);
                \newcommand{\gocv}[4][black]{\draw[#1] ($(#3)!5pt!(#2)$)--($(#3)!2!($($(#3)!5pt!(#2)$)!.5!($(#3)!5pt!(#4)$)$)$)--($(#3)!5pt!(#4)$);}
                \gocv{S}{A}{D}
            \end{tikzpicture}
        }
    }
\end{ex}
%%==========Câu 95
\begin{ex}[Sở Cần Thơ - 2021]%Câu 7.%[1H3B4-3]
    \immini{Cho hình chóp $S.ABCD$ có đáy ABCD là hình vuông, $SA$ vuông góc với mặt phẳng $(ABCD)$. Góc giữa hai mặt phẳng $(SCD)$ và mặt phẳng $(ABCD)$ là
        \choice
        {$\widehat{SDC}$}
        {$\widehat{SCD}$}
        {$\widehat{DSA}$}
        {\True $\widehat{SDA}$}
    }{\begin{tikzpicture}[scale=1, font=\footnotesize, line join=round, line cap=round, >=stealth]
            \def\bc{4} % cạnh BC
            \def\ba{2} % cạnh BA
            \def\h{3} % đường cao
            \def\gocB{30} % góc B của đáy
            \coordinate[label=below left:$B$] (B) at (0,0);
            \coordinate[label=above left:$A$] (A) at (\gocB:\ba);
            \coordinate[label=below:$C$] (C) at (\bc,0);
            \coordinate[label=right:$D$] (D) at ($(C)-(B)+(A)$);
            \coordinate[label=above:$S$] (S) at ($(A)+(90:\h)$);
            \draw (B)--(C)--(D)--(S)--cycle (S)--(C);
            \draw[dashed] (A)--(D) (S)--(A)--(B);
            \foreach \diem in {A,B,C,D,S}	\fill (\diem)circle(1.5pt);
            \newcommand{\gocv}[4][black]{\draw[#1] ($(#3)!5pt!(#2)$)--($(#3)!2!($($(#3)!5pt!(#2)$)!.5!($(#3)!5pt!(#4)$)$)$)--($(#3)!5pt!(#4)$);}
            \gocv{S}{A}{D}
        \end{tikzpicture}
    }
    \loigiai{
        Ta có $(SCD)\cap (ABCD) = CD$.\\
        Mặt khác $CD\perp (SAD)\Rightarrow CD\perp SD$, lại có $AD\perp CD$.\\
        Vậy góc giữa hai mặt phẳng $(SCD)$ và mặt phẳng $(ABCD)$ là $\widehat{SDA}$.
    }
\end{ex}
%%==========Câu 96
\begin{ex}[Sở Sơn La - 2021]%Câu 8.%[1H3B4-3]
    Cho hình chóp $S.ABC$ có đáy tam giác đều cạnh $a$. Cạnh bên $SA=a\sqrt{3}$ vuông góc với mặt đáy $(ABC)$. Gọi $\varphi$ là góc giữa hai mặt phẳng $(SBC)$ và $(ABC)$. Khi đó $\sin\varphi$ bằng
    \choice
    {$\dfrac{\sqrt{3}}{5}$}
    {\True $\dfrac{2\sqrt{5}}{5}$}
    {$\dfrac{2\sqrt{3}}{5}$}
    {$\dfrac{\sqrt{5}}{5}$}
    \loigiai{
        \immini{Ta có $(SBC)\cap(ABC)=BC$; gọi $M$ là trung điểm $BC$ (1), tam giác $ABC$ đều nên $AM\perp BC$ (2).\\
            $\heva{&BC\perp AM\\&BC\perp SA}\Rightarrow BC\perp SM$ (3).\\
            Từ (1), (2) và (3) ta có $\varphi=\widehat{(SM,AM)}=\widehat{SMA}$.\\
            $SM=\sqrt{SA^2+AM^2}=\sqrt{(a\sqrt{3})^2+\left(\dfrac{a\sqrt{3}}{2}\right)^2}=\dfrac{a\sqrt{15}}{2}$.\\
            $\sin\varphi=\sin\widehat{SMA}=\dfrac{SA}{SM}=a\sqrt{3}\colon\dfrac{a\sqrt{15}}{2}=\dfrac{2\sqrt{5}}{5}$.
        }{\begin{tikzpicture}[scale=1, font=\footnotesize, line join=round, line cap=round, >=stealth]
                \def\ac{4} % cạnh AC
                \def\ab{2} % cạnh AB
                \def\h{3} % chiều cao
                \def\gocA{50} % góc A của đáy
                \coordinate[label=left:$A$] (A) at (0,0);
                \coordinate[label=right:$C$] (C) at (\ac,0);
                \coordinate[label=below left:$B$] (B) at (-\gocA:\ab);
                \coordinate[label=above:$S$] (S) at ($(A)+(90:\h)$);
                \coordinate[label=right:$M$] (M) at ($(B)!0.5!(C)$);

                \draw (A)--(B)--(C)--(S)--cycle (M)--(S)--(B);
                \draw[dashed] (M)--(A)--(C);
                \foreach \diem in {A,B,C,S,M}	\fill (\diem)circle(1.5pt);
            \end{tikzpicture}
        }
    }
\end{ex}
%%==========Câu 97
\begin{ex}%[Đề tham khảo - THPT.QG 2018]%[2H3G4-1]
    \immini{Cho hình lăng trụ tam giác đều $ABC.A'B'C'$ có $AB=2\sqrt{3}$ và $AA'=2$. Gọi $M$, $N$, $P$ lần lượt là trung điểm các cạnh $A'B'$, $A'C'$ và $BC$ (tham khảo hình vẽ bên dưới). Côsin của góc tạo bởi hai mặt phẳng $(AB'C')$ và $(MNP)$ bằng
        \choice
        {$\dfrac{6\sqrt{13}}{65}$}
        {\True $\dfrac{\sqrt{13}}{65}$}
        {$\dfrac{17\sqrt{13}}{65}$}
        {$\dfrac{18\sqrt{13}}{65}$}}
    {\begin{tikzpicture}
            \coordinate[label=right:{$A$}] (A) at (4,0);
            \coordinate[label=left:{$B$}] (B) at (0,0);
            \coordinate[label=right:{$C$}] (C) at (3,1.3);
            \coordinate[label=right:{$A'$}] (A') at (4,3);
            \coordinate[label=left:{$B'$}] (B') at ($(B)+(A')-(A)$);
            \coordinate[label=above:{$C'$}] (C') at ($(C)+(A')-(A)$);
            \coordinate[label=above:{$M$}] (M) at ($(B')!1/2!(A')$);
            \coordinate[label=right:{$N$}] (N) at ($(A')!1/2!(C')$);
            \coordinate[label=below:{$P$}] (P) at ($(B)!1/2!(C)$);
            \draw (A')--(C')--(B')--(B)--(A)--(A')--(B')--(A) (M)--(N);
            \draw[dashed] (B)--(C)--(C')--(A)--(C) (M)--(P)--(N);
    \end{tikzpicture}}
    \loigiai{
        \textbf{\underline{Cách 1}: dùng phương pháp cổ điển}\\
        \immini{Gọi $I$, $Q$ lần lượt là trung điểm của $MN$, $B'C'$. Gọi $O=PI\cap AQ$. \\
            Khi đó $\left\{\begin{aligned}
                &O\in ~(AB'C')\cap (MNP) \\
                &B'C'\ \parallel \ MN \\
                &B'C'\subset (AB'C'),~MN\subset (MNP)
            \end{aligned}\right. $ nên giao tuyến của $(AB'C')$ và $(MNP)$ là đường thẳng $d$ qua $O$ và $d \parallel MN \parallel B'C'$. \\
            Tam giác $AB'C'$ cân tại $A$ nên $AQ\perp B'C' \Rightarrow AQ\perp d$. \\
            Tam giác $PMN$ cân tại $P$ nên $PI\perp MN \Rightarrow PI\perp d$. \\
            Vậy $(\widehat{(AB'C'),(MNP)})=(\widehat{AQ,PI})$.}
        {\begin{tikzpicture}
                \coordinate[label=right:{$A$}] (A) at (4,0);
                \coordinate[label=left:{$B$}] (B) at (0,0);
                \coordinate[label=right:{$C$}] (C) at (3,1.3);
                \coordinate[label=right:{$A'$}] (A') at (4,4);
                \coordinate[label=left:{$B'$}] (B') at ($(B)+(A')-(A)$);
                \coordinate[label=above:{$C'$}] (C') at ($(C)+(A')-(A)$);
                \coordinate[label=above:{$M$}] (M) at ($(B')!1/2!(A')$);
                \coordinate[label=right:{$N$}] (N) at ($(A')!1/2!(C')$);
                \coordinate[label=below:{$P$}] (P) at ($(B)!1/2!(C)$);
                \coordinate[label=above:{$Q$}] (Q) at ($(B')!1/2!(C')$);
                \coordinate[label=above:{$I$}] (I) at ($(Q)!1/2!(A')$);
                \coordinate[label=below:{$O$}] (O) at ($(A)!2/3!(Q)$);
                \draw (Q)--(A')--(C')--(B')--(B)--(A)--(A')--(B')--(A) (M)--(N);
                \draw[dashed] (B)--(C)--(C')--(A)--(C) (M)--(P)--(N) (A)--(Q)--(P)--(I) ($(A)!2/3!(B')$)--($(A)!2/3!(C')$);
        \end{tikzpicture}}
        \noindent
        Ta có $AP=3$, $AQ=\sqrt{13}$, $IP=\dfrac{5}{2}$. \\
        Vì $\triangle OAP\backsim \triangle OQI$ và $\dfrac{AP}{IQ}=2$ nên $OA=\dfrac{2}{3}AQ=\dfrac{2\sqrt{13}}{3}$; $OP=\dfrac{2}{3}IP=\dfrac{5}{3}$. \\
        $\cos \left(\widehat{(AB'C'),(MNP)}\right)=\cos (\widehat{AQ,PI})=\left|\cos (\widehat{AOP})\right|=\dfrac{OA^2+OP^2-AP^2}{2OA.OP}=\dfrac{\sqrt{13}}{65}$.\\
        \textbf{\underline{Cách 2}: dùng phương pháp toạ độ hoá một hình không gian}\\
        \immini{Gắn hệ trục toạ độ $Pxyz$ như hình vẽ, lúc đó\\
            $P(0,0,0),\,A(0;3;0),\,B'(\sqrt{3};0;2),\,C'(-\sqrt{3};0;2),\,A'(0;3;2)$\\
            \centerline{$ \Rightarrow M\left(\dfrac{\sqrt{3}}{2};\dfrac{3}{2};2\right),\,N\left(-\dfrac{\sqrt{3}}{2};\dfrac{3}{2};2\right)$.}
            Ta có $\overrightarrow{PM}=\left(\dfrac{\sqrt{3}}{2};\dfrac{3}{2};2\right),\,\overrightarrow{PN}=\left(-\dfrac{\sqrt{3}}{2};\dfrac{3}{2};2\right)$\\
            \centerline{$\Rightarrow \left[\overrightarrow{PM},\overrightarrow{PN}\right]=\left(0;-2\sqrt{3};\dfrac{3\sqrt{3}}{2} \right)$.}
            Và $\left\{\begin{aligned}& \overrightarrow{AB'}=(\sqrt{3};-3;2)\\&\overrightarrow{AC'}=(-\sqrt{3};-3;2)  \end{aligned}\right.\\ \Rightarrow \left[\overrightarrow{AB'},\overrightarrow{AC'}\right]=(0;-4\sqrt{3};-6\sqrt{3})$.
        }
        {\begin{tikzpicture}[line width=0.6pt,>=stealth,scale=0.95]
                \coordinate[label=below:{$A$}] (A) at (4,0);
                \coordinate[label=below:{$B$}] (B) at (0,0);
                \coordinate[label=right:{$C$}] (C) at (3,1.3);
                \coordinate[label=right:{$A'$}] (A') at (4,3);
                \coordinate[label=left:{$B'$}] (B') at ($(B)+(A')-(A)$);
                \coordinate[label=above:{$C'$}] (C') at ($(C)+(A')-(A)$);
                \coordinate[label=above:{$M$}] (M) at ($(B')!1/2!(A')$);
                \coordinate[label=right:{$N$}] (N) at ($(A')!1/2!(C')$);
                \coordinate[label=below:{$P$}] (P) at ($(B)!1/2!(C)$);
                \coordinate[label=below:{$x$}] (x) at ($(P)!3/2!(B)$);
                \coordinate[label=below:{$y$}] (y) at ($(P)!3/2!(A)$);
                \coordinate (P') at ($(B')!1/2!(C')$);
                \coordinate[label=left:{$z$}] (z) at ($(P)!1.2!(P')$);
                \draw (A')--(C')--(B')--(B)--(A)--(A')--(B')--(A) (M)--(N);
                \draw[dashed] (B)--(C)--(C')--(A)--(C) (M)--(P)--(N) (P')--(P)--(A);
                \draw[->] (A)--(y);
                \draw[->] (B)--(x);
                \draw[->] (P')--(z);
        \end{tikzpicture}}
        \noindent
        Như vậy $(PMN)$ có VTPT $\overrightarrow{n_1}=(0;-4;3);~(AB'C')$ có VTPT $\overrightarrow{n_2}=(0;2;3)$.\\
        Cuối cùng $\cos(\widehat{(PMN),(AB'C')})=\dfrac{|\overrightarrow{n_1}.\overrightarrow{n_2}|}{|\overrightarrow{n_1}|.|\overrightarrow{n_2}|}=\dfrac{\sqrt{13}}{65}.$\\
        \textbf{\underline{Cách 3}}\\
        \immini{Gọi $Q$ là trung điểm của $AA'$, khi đó mặt phẳng $(AB'C')$ song song với mặt phẳng $(MNQ)$ nên góc giữa hai mặt phẳng $(AB'C')$ và $(MNP)$ cũng bằng góc giữa hai mặt phẳng $(MNQ)$ và $(MNP)$.\\
            Ta có:\\
            $\heva{&(MNP)\cap(MNQ)=MN\\&PE\subset(MNP); PE\perp MN\\&QE\subset(MNQ); QE\perp MN}\\\Rightarrow\widehat{\left((MNP);(MNQ)\right)}=\widehat{PEQ}$\\
            hoặc $\widehat{\left((MNP);(MNQ)\right)}=180^{\circ}-\widehat{PEQ}$.
        }{\begin{tikzpicture}[scale=1.2]
                \coordinate[label=right:{$A$}] (A) at (4,0);
                \coordinate[label=left:{$B$}] (B) at (0,0);
                \coordinate[label=right:{$C$}] (C) at (3,1.3);
                \coordinate[label=right:{$A'$}] (A') at (4,3);
                \coordinate[label=left:{$B'$}] (B') at ($(B)+(A')-(A)$);
                \coordinate[label=above:{$C'$}] (C') at ($(C)+(A')-(A)$);
                \coordinate[label=above:{$M$}] (M) at ($(B')!1/2!(A')$);
                \coordinate[label=right:{$N$}] (N) at ($(A')!1/2!(C')$);
                \coordinate[label=below:{$P$}] (P) at ($(B)!1/2!(C)$);
                \coordinate[label=above left:$F$] (F) at ($(B')!0.5!(C')$);
                \coordinate[label=above:$E$] (E) at (intersection cs:first line={(M)--(N)}, second line={(A')--(F)});
                \coordinate[label=right:$Q$] (Q) at ($(A)!0.5!(A')$);
                \draw (A')--(C')--(B')--(B)--(A)--(A')--(B')--(A) (M)--(N) (A')--(F) (Q)--(M);
                \draw[dashed] (B)--(C)--(C')--(A)--(C) (M)--(P)--(N) (P)--(F) (N)--(Q)--(E)--(P);
                \draw pic[angle radius=3mm,draw=blue] {angle = P--E--Q};

        \end{tikzpicture}}
        Tam giác $ABC$ đều có cạnh $2\sqrt{3}\Rightarrow AP=3$.\\
        Tam giác $APQ$ vuông tại $A$ nên ta có: $PQ=\sqrt{AP^2+AQ^2}=\sqrt{3^2+1^2}=\sqrt{10}$.\\
        Tam giác $A'QE$ vuông tại $A'$ nên ta có: $QE=\sqrt{A'E^2+A'Q^2}=\sqrt{\left(\dfrac{3}{2}\right)^2+1^2}=\dfrac{\sqrt{13}}{2}$.\\
        Tam giác $PEF$ vuông tại $F$ nên ta có: $PE=\sqrt{FP^2+FE^2}=\sqrt{2^2+\left(\dfrac{3}{2}\right)^2}=\dfrac{5}{2}$.\\
        Áp dụng định lý hàm số côsin vào tam giác $PQE$ ta có:\\
        $\cos\widehat{PEQ}=\dfrac{EP^2+EQ^2-PQ^2}{2\cdot EP\cdot EQ}=\dfrac{\dfrac{25}{4}+\dfrac{13}{4}-10}{2\cdot\dfrac{5}{2}\cdot\dfrac{\sqrt{13}}{2}}=-\dfrac{\sqrt{13}}{65}$.\\
        Do đó: $\cos\widehat{\left((MNP);(AB'C')\right)}=\cos\left({180}^{\circ}-\widehat{PEQ}\right)=-\cos\widehat{PEQ}=\dfrac{\sqrt{13}}{65}$.
    }
\end{ex}
\Closesolutionfile{ans}
\indapan{10}{ans/CD7/DA2}
\chapter{KHOẢNG CÁCH - GÓC TRONG KHÔNG GIAN}
\section{Mức 9,10 điểm}
\setcounter{ex}{0}
\setcounter{dang}{0}
\Opensolutionfile{ans}[ans/CD1/Muc_9_10]
\begin{dang}{Tìm m để hàm số đơn điệu trên các khoảng xác định của nó}
	Đang thiếu bài thầy Jf Câu 1 đến 26 
\end{dang}
\begin{dang}
	{Tìm khoảng đơn điệu của hàm số $g(x) = f\left[ u(x)\right] +v(x)$ khi biết đồ thị hoặc bảng biến thiên của hàm số $y = f'(x)$}
\end{dang}
\begin{ex}[Đề tham khảo 2019]%[2D1K1-2]
	Cho hàm số $f(x)$ có bảng xét dấu của đạo hàm như sau
	\begin{center}
		\begin{tikzpicture}
			\tkzTabInit[nocadre,lgt=1.2,espcl=2,deltacl=0.6]
			{$x$ /0.6,$f'(x)$ /0.6}
			{$-\infty$,$1$,$2$,$3$,$4$,$+\infty$}
			\tkzTabLine{,-,$0$,+,$0$,+,$0$,-,$0$,+,}
		\end{tikzpicture}
	\end{center}
	Hàm số $y=3 f(x+2)-x^3+3 x$ đồng biến trên khoảng nào dưới đây?
	\choice
	{$(-\infty ;-1)$}
	{\True $(-1 ; 0)$}
	{$(0 ; 2)$}
	{$(1 ;+\infty)$}
	\loigiai{
		Ta có $y'=3\left[f'(x+2)-\left(x^2-3\right)\right]$.\\
		Với $x \in(-1 ; 0) \Rightarrow x+2 \in(1 ; 2) \Rightarrow f'(x+2)>0$, lại có $x^2-3<0 \Rightarrow y'>0 ;~ \forall x \in(-1 ; 0)$.\\
		Vậy hàm số $y=3 f(x+2)-x^3+3 x$ đồng biến trên khoảng $(-1 ; 0)$.\\
		Chú ý:\\
		+) Ta xét $x \in(1 ; 2) \subset(1 ;+\infty)
		\Rightarrow x+2 \in(3 ; 4)\\
		\Rightarrow f'(x+2)<0 ;~ x^2-3>0$\\
		Suy ra hàm số nghịch biến trên khoảng $(1 ; 2)$ nên loại hai phương án$(0 ; 2)$ và $(1 ;+\infty)$.\\
		+) Tương tự ta xét
		$x \in(-\infty ;-2) \Rightarrow x+2 \in(-\infty ; 0)\\
		\Rightarrow f'(x+2)<0 ; x^2-3>0 \Rightarrow y'<0 ; ~ \forall x \in(-\infty ;-2)$.\\
		Suy ra hàm số nghịch biến trên khoảng $(-\infty ;-2)$ nên loại$(-\infty ;-1)$.\\
		Vậy hàm số đã cho đồng biến trên khoảng $(-1 ; 0)$.
	}
\end{ex}
\begin{ex}[Đề Tham Khảo 2020 - Lần 1]%[2D1G1-2]
	\immini{
		Cho hàm số $f(x)$. Hàm số $y=f'(x)$ có đồ thị như hình bên. Hàm số $g(x)=f(1-2 x)+x^2-x$ nghịch biến trên khoảng nào dưới đây?
		\choice
		{\True $\left(1 ; \dfrac{3}{2}\right)$}
		{$\left(0 ; \dfrac{1}{2}\right)$}
		{$(-2 ;-1)$}
		{$(2 ; 3)$}
	}
	{
		\begin{tikzpicture}[scale=0.7,>=stealth, font=\footnotesize, line join=round, line cap=round]
			%\def\a{1} \def\b{-6} \def\c{9} \def\d{1} % Hệ số
			\def\xmin{-4} \def\xmax{6}
			\def\ymin{-3} \def\ymax{2} 
			%\draw[color=gray!50,dashed] (\xmin,\ymin) grid (\xmax,\ymax); 
			\draw[->] (\xmin,0)--(\xmax,0) node [below]{$x$};
			\draw[->] (0,\ymin)--(0,\ymax) node [left]{$y$};
			\node at (0,0) [below left]{$O$};
			%\node at (1,3) [below left]{$f'(x)$};
			%\node at (-1.3,4) {$f'(x)$};
			\draw[dashed] (-2,0) node[below]{$-2$}--(-2,1)--(0,1) node[below left]{$1$};
			\draw[dashed] (4,0) node[below left]{$4$}--(4,-2)--(0,-2) node[below left]{$-2$};
			%\draw[dashed] (1,0) node[below]{$1$}--(1,1);
			%\draw[dashed] (-0.5,0) node[below left]{$-0{,}5$}--(-0.5,2.125);
			\clip (\xmin+0.1,\ymin+0.1) rectangle (\xmax-0.5,\ymax-0.1);
			\draw[smooth,samples=300][domain=-4:5.5] plot(\x,{0.071*(\x)^3-0.142*(\x)^2-1.07*(\x)});
		\end{tikzpicture}
	}
	
	\loigiai{
		Ta có : $g(x)=f(1-2 x)+x^2-x \Rightarrow g'(x)=-2 f'(1-2 x)+2 x-1$.\\
		\immini{
			Đặt $t=1-2 x \Rightarrow g'(x)=-2 f'(t)-t$.\\
			$g'(x)=0 \Rightarrow f'(t)=-\dfrac{t}{2}$.\\
			Vẽ đường thẳng $y=-\dfrac{x}{2}$ và đồ thị hàm số $f'(x)$ trên cùng một hệ trục
		}	
		{
			\begin{tikzpicture}[scale=0.7,>=stealth, font=\footnotesize, line join=round, line cap=round]
				%\def\a{1} \def\b{-6} \def\c{9} \def\d{1} % Hệ số
				\def\xmin{-4} \def\xmax{6}
				\def\ymin{-3} \def\ymax{2} 
				%	\draw[color=gray!50,dashed] (\xmin,\ymin) grid (\xmax,\ymax); 
				\draw[->] (\xmin,0)--(\xmax,0) node [below]{$x$};
				\draw[->] (0,\ymin)--(0,\ymax) node [left]{$y$};
				\node at (0,0) [below left]{$O$};
				%\node at (1,3) [below left]{$f'(x)$};
				%\node at (-1.3,4) {$f'(x)$};
				\draw[dashed] (-2,0) node[below]{$-2$}--(-2,1)--(0,1) node[below left]{$1$};
				\draw[dashed] (4,0) node[below]{$4$}--(4,-2)--(0,-2) node[below left]{$-2$};
				%\draw[dashed] (1,0) node[below]{$1$}--(1,1);
				%\draw[dashed] (-0.5,0) node[below left]{$-0{,}5$}--(-0.5,2.125);
				\clip (\xmin+0.1,\ymin+0.1) rectangle (\xmax-0.5,\ymax-0.1);
				\draw[smooth,samples=300][domain=-4:5.5] plot(\x,{0.071*(\x)^3-0.142*(\x)^2-1.07*(\x)});
				\draw[smooth,samples=300][domain=-4:5.5] plot(\x,{(-0.5*(\x)});
			\end{tikzpicture}
		}	Hàm số $g(x)$ nghịch biến $\Rightarrow g'(x) \leq 0 \Rightarrow f'(t) \geq-\dfrac{t}{2}\Rightarrow\hoac{&-2 \leq t \leq 0 \\&t \geq 4.}$\\
		Như vậy $f'(1-2 x) \geq \dfrac{1-2 x}{-2}\Rightarrow\hoac{&-2 \leq 1-2 x \leq 0 \\ &4 \leq 1-2 x}\Rightarrow\hoac{&\dfrac{1}{2}\leq x \leq \dfrac{3}{2}\\ &x \leq-\dfrac{3}{2}.}$\\
		Vậy hàm số $g(x)=f(1-2 x)+x^2-x$ nghịch biến trên các khoảng $\left(\dfrac{1}{2}; \dfrac{3}{2}\right)$ và $\left(-\infty ;-\dfrac{3}{2}\right)$.\\
		Mà $\left(1 ; \dfrac{3}{2}\right) \subset \left(\dfrac{1}{2}; \dfrac{3}{2}\right)$ nên hàm số $g(x)=f(1-2 x)+x^2-x$ nghịch biến trên khoảng $\left(1 ; \dfrac{3}{2}\right)$.
	}
\end{ex}
\begin{ex}[Chuyên Lê Quý Đôn Điện Biên 2019]%[2D1G1-2]
	Cho hàm số $f(x)$ có bảng xét dấu của đạo hàm như sau
	\begin{center}
		\begin{tikzpicture}
			\tkzTabInit[nocadre,lgt=1.2,espcl=2,deltacl=0.6]
			{$x$ /0.6,$f'(x)$ /0.6}
			{$-\infty$,$0$,$1$,$2$,$3$,$+\infty$}
			\tkzTabLine{,+,$0$,-,$0$,-,$0$,+,$0$,-,}
		\end{tikzpicture}
	\end{center}
	Hàm số $y=f(x-1)+x^3-12 x+2019$ nghịch biến trên khoảng nào dưới đây?
	\choice
	{$(1 ;+\infty)$}
	{\True $(1 ; 2)$}
	{$(-\infty ; 1)$}
	{$(3 ; 4)$}
	\loigiai{
		$y'=f'(x-1)+3 x^2-12=f'(t)+3 t^2+6 t-9=f'(t)-\left(-3 t^2-6 t+9\right)$, với $t=x-1$.\\
		\immini{
			Nghiệm của phương trình $y'=0$ là hoành độ giao điểm của các đồ thị hàm số $y=f'(t)$ và $y=-3 t^2-6 t+9$.\\
			Vẽ đồ thị hàm số $y=f'(t)$ và $y=-3 t^2-6 t+9$ trên cùng một hệ trục tọa độ như hình vẽ bên.
		}	
		{		\begin{tikzpicture}[scale=0.5,>=stealth, font=\footnotesize, line join=round, line cap=round]
				\def\a{-3} \def\b{-6} \def\c{9} % Hệ số
				\def\xmin{-9} \def\xmax{7}
				\def\ymin{-3} \def\ymax{13}
				
				%\draw[color=gray!50,dashed] (\xmin,\ymin) grid (\xmax,\ymax);
				
				\draw[->] (\xmin,0)--(\xmax,0) node [below]{$x$};
				\draw[->] (0,\ymin)--(0,\ymax) node [left]{$y$};
				\node at (0,0) [below left]{$O$};
				\clip (\xmin+0.1,\ymin+0.1) rectangle (\xmax-0.5,\ymax-0.1);
				\draw[smooth,samples=300] plot(\x,{\a*(\x)^2+\b*(\x)+\c});
				\node at (1,0) [above right]{$1$};
				\node at (2,0) [below right]{$2$};
				\node at (3,0) [below right]{$3$};
				\node at (-3,-2) [left]{$y=-3t^2-6t+9$};
				\node at (4,0) [below right]{$f'(x)$};
				\draw (-2.2,10).. controls (-1,1.9) and (-0.5,0.8) .. (0,0);
				%\draw (-2,0).. controls (-1.5,-2) and (-0.5,-0) .. (0,0);
				\draw (0,0).. controls (0.4,-0.6) and (0.6,-0.6) .. (0.8,-0.2);
				\draw (0.8,-0.2).. controls (1,0.25) and (1.1,-0.1) .. (1.4,-0.8);
				\draw (1.4,-0.8).. controls (1.6,-1.1) and (1.7,-0.9) .. (2,0);
				\draw (2,0).. controls (2.4,1.1) and (2.6,1.1) .. (3.5,-1);
			\end{tikzpicture}
		}
		Dựa vào đồ thị trên, ta có bảng xét dấu của hàm số $y'=f'(t)-\left(-3 t^2-6 t+9\right)$ như sau $
		\left(t_0<-1\right)$
		\begin{center}
			\begin{tikzpicture}
				\tkzTabInit[nocadre,lgt=2,espcl=2,deltacl=0.6]
				{$x$ /0.6,$y'$ /0.6}
				{$-\infty$,$t_0$,$1$,$+\infty$}
				\tkzTabLine{,+,$0$,-,$0$,+,}
			\end{tikzpicture}
		\end{center}
		Hàm số nghịch biến trên khoảng $t \in\left(t_0 ; 1\right)$.\\
		Do đó hàm số nghịch biến trên khoảng $x \in(1 ; 2) \subset \left(t_0+1 ; 1\right)$.
	}
\end{ex}


\begin{ex}[Chuyên Phan Bội Châu Nghệ An 2019]%[2D1G1-2]
	Cho hàm số $f(x)$ có bảng xét dấu đạo hàm như sau:
	\begin{center}
		\begin{tikzpicture}
			\tkzTabInit[nocadre,lgt=2,espcl=2,deltacl=0.6]
			{$x$ /0.6,$f'(x)$ /0.6}
			{$-\infty$,$1$,$2$,$3$,$4$,$+\infty$}
			\tkzTabLine{,-,$0$,+,$0$,+,$0$,-,$0$,+,}
		\end{tikzpicture}
	\end{center}
	Hàm số $y=2 f(1-x)+\sqrt{x^2+1}-x$ nghịch biến trên những khoảng nào dưới đây
	\choice
	{$(-\infty ;-2)$}
	{$(-\infty ; 1)$}
	{\True $(-2 ; 0)$}
	{$(-3 ;-2)$}
	\loigiai{
		$y'=-2 f'(1-x)+\dfrac{x}{\sqrt{x^2+1}}-1$. \\
		Có $\dfrac{x}{\sqrt{x^2+1}}-1<0,~ \forall x \in(-2 ; 0)$.\\
		Bảng xét dấu:
		\begin{center}
			\begin{tikzpicture}
				\tkzTabInit[nocadre,lgt=2,espcl=2,deltacl=0.6]
				{$x$ /0.7,$f'(1-x)$ /0.7}
				{$-\infty$,$-3$,$-2$,$-1$,$0$,$+\infty$}
				\tkzTabLine{,+,$0$,-,$0$,+,$0$,+,$0$,-,}
			\end{tikzpicture}
		\end{center}
		$\Rightarrow-2 f'(1-x)<0, ~ \forall x \in(-2 ; 0) \\
		\Rightarrow-2 f'(1-x)+\dfrac{x}{\sqrt{x^2+1}}-1<0, ~\forall x \in(-2 ; 0)$.
	}
\end{ex}
\begin{ex}[Sở Vĩnh Phúc 2019]%[2D1G1-2]
	\immini{
		Cho hàm số bậc bốn $y=f(x)$ có đồ thị của hàm số $y=f'(x)$ như hình vẽ bên.\\
		Hàm số $y=3 f(x)+x^3-6 x^2+9 x$ đồng biến trên khoảng nào trong các khoảng sau đây?
		\choice
		{$(0 ; 2)$}
		{$(-1 ; 1)$}
		{$(1 ;+\infty)$}
		{\True $(-2 ; 0)$}
	}
	{
		\begin{tikzpicture}[scale=0.7,>=stealth, font=\footnotesize, line join=round, line cap=round]
			\def\a{0.21} \def\b{0.88} \def\c{-0.58} \def\d{-3} % Hệ số
			\def\xmin{-5} \def\xmax{5}
			\def\ymin{-4} \def\ymax{3} 
			%\draw[color=gray!50,dashed] (\xmin,\ymin) grid (\xmax,\ymax); 
			\draw[->] (\xmin,0)--(\xmax,0) node [below]{$x$};
			\draw[->] (0,\ymin)--(0,\ymax) node [left]{$y$};
			\node at (0,0) [above left]{$O$};
			\node at (-4,0) [below left]{$-4$};
			\node at (-2,0) [below left]{$-2$};
			\node at (0,-3) [below right]{$-3$};
			\draw[dashed] (2,0) node[above right]{$2$}--(2,1) --(0,1) node[above right]{$1$};
			\clip (\xmin+0.1,\ymin+0.1) rectangle (\xmax-0.5,\ymax-0.1);
			\draw[smooth,samples=300] plot(\x,{\a*(\x)^3+\b*(\x)^2+\c*(\x)+\d});
		\end{tikzpicture}
	}
	
	\loigiai{
		Hàm số $f(x)=a x^4+b x^3+c x^2+d x+e,(a \neq 0)$.
		Có $f'(x)=4 a x^3+3 b x^2+2 c x+d$.\\
		Đồ thị hàm số $y=f'(x)$ đi qua các điểm $(-4 ; 0),(-2 ; 0),(0 ;-3),(2 ; 1)$ nên ta có
		$$\heva{&- 2 5 6 a + 4 8 b - 8 c + d = 0\\
			&- 3 2 a + 1 2 b - 4 c + d = 0\\
			&d = - 3\\
			&3 2 a + 1 2 b + 4 c + d = 1}\Leftrightarrow \heva{&
			a=\dfrac{5}{96}\\
			&b=\dfrac{7}{24}\\
			&c=-\dfrac{7}{24}\\
			&d=-3.}
		$$
		Xét hàm số
		$
		y=3 f(x)+x^3-6 x^2+9 x$\\
		Ta có $ y'=3\left(f'(x)+x^2-4 x+3\right)=3\left(\frac{5}{24}x^3+\frac{15}{8}x^2-\frac{55}{12}x\right)
		$\\
		Ta có $y'=0 \Leftrightarrow\hoac{&x=-11 \\&x=0 \\&x=2.}$ \\
		Xét dấu $y'$, ta được hàm số đã cho đồng biến trên các khoảng $(-11 ; 0)$ và $(2 ;+\infty)$.
	}
\end{ex}
\begin{ex}[Học Mãi 2019]%[2D1K1-2]
	\immini
	{Cho hàm số $y=f(x)$ có đạo hàm trên $\mathbb{R}$. Đồ thị hàm số $y=f'(x)$ như hình bên. Hỏi đồ thị hàm số $y=f(x)-2 x$ có bao nhiêu điểm cực trị?
		\choice
		{$4$}
		{\True $3$}
		{$2$}
		{$1$}
	}
	{
		\begin{tikzpicture}[font=\footnotesize,line join=round, line cap=round,>=stealth,scale=0.8]
			\draw[->] (-3.5,0)--(4,0) node[above] {$x$};
			\draw[->] (0,-3)--(0,4) node[left] {$y$};
			%\fill[black] (-2,0)node[below left]{$-2$} circle (1.2pt) (0,0)node[above right]{$O$} circle (1.2pt) (3,0)node[above]{$3$} circle (1.2pt);
			\draw[dashed] (-2,-2)-- (0,-2) node[right]{$-2$};
			\draw[dashed] (2,0) node[below]{$2$}-- (2,2)--(0,2) node[below left]{$2$};
			\node at (0,0) [below left]{$O$};
			\node at (3,0) [below right]{$3$};
			\draw (-3,2.5).. controls (-2.2,-3) and (-1.8,-3) .. (-1.1,0);
			\draw (-1.1,0).. controls (-0.6,2.5) and (-0.4,2.5) .. (0,2);
			\draw (0,2).. controls (0.7,0.5) and (1.1,0.5) .. (1.5,1.5);
			\draw (1.5,1.5).. controls (2,2.5) and (2.8,2.5) .. (3.5,-2.5);
			%\draw (3,0).. controls (3.3,-0.1) and (3.5,-0.5) .. (3.5,-2);
		\end{tikzpicture}
	}
	\loigiai{
		\immini{
			Đặt $g(x)=f(x)-2 x$.\\
			$\Rightarrow g'(x)=f'(x)-2 .
			$\\
			Vẽ đường thẳng $y=2$.\\
			$\Rightarrow$ phương trình $g'(x)=0$ có $3$ nghiệm bội lẻ.\\
			$\Rightarrow$ đồ thị hàm số $y=f(x)-2 x$ có $3$ điểm cực trị.
		}
		{
			\begin{tikzpicture}[font=\footnotesize,line join=round, line cap=round,>=stealth,scale=0.8]
				\draw[->] (-3.5,0)--(4,0) node[above] {$x$};
				\draw[->] (0,-3)--(0,4) node[left] {$y$};
				%\fill[black] (-2,0)node[below left]{$-2$} circle (1.2pt) (0,0)node[above right]{$O$} circle (1.2pt) (3,0)node[above]{$3$} circle (1.2pt);
				\draw[dashed] (-2,-2)-- (0,-2) node[right]{$-2$};
				\draw[dashed] (2,0) node[below]{$2$}-- (2,2)--(0,2) node[below left]{$2$};
				\node at (3,0) [below left]{$3$};
				\draw (-3,2.5).. controls (-2.2,-3) and (-1.8,-3) .. (-1.1,0);
				\draw (-1.1,0).. controls (-0.6,2.5) and (-0.4,2.5) .. (0,2);
				\draw (0,2).. controls (0.7,0.5) and (1.1,0.5) .. (1.5,1.5);
				\draw (1.5,1.5).. controls (2,2.5) and (2.8,2.5) .. (3.5,-2.5);
				\draw (-3.5,2)--(4,2) node[above]{$y=2$};
			\end{tikzpicture}
		}
	}
\end{ex}
\begin{ex}[THPT Hoàng Hoa Thám Hưng Yên 2019]%[2D1G1-2]
	\immini{
		Cho hàm số $y=f(x)$ liên tục trên $\mathbb{R}$. Hàm số $y=f'(x)$ có đồ thị như hình vẽ. 
		Hàm số $g(x)=f(x-1)+\dfrac{2019-2018 x}{2018}$ đồng biến trên khoảng nào dưới đây?
		\choice
		{$(2 ; 3)$}
		{$(0 ; 1)$}
		{\True $(-1 ; 0)$}
		{$(1 ; 2)$}
	}
	{
		\begin{tikzpicture}[scale=1, font=\footnotesize, line join=round, line cap=round, >=stealth]
			\tikzset{label style/.style={font=\footnotesize}}
			\draw[->] (-2,0)--(3,0) node[below left] {$x$};
			\draw[->] (0,-2)--(0,3) node[below left] {$y$};
			\draw[fill=black] (0,0) node [above left] {$O$} circle(1pt);
			\fill (1,1) circle(1pt) (-1,1) circle(1pt) (2,1) circle(1pt);
			\foreach \x in {1,2}
			\draw[thin] (\x,1pt)--(\x,-1pt) node [below] {\footnotesize$\x$};
			\foreach \x in {-1}
			\draw[thin] (\x,1pt)--(\x,-1pt) node [below left] {\footnotesize$\x$};
			\foreach \y in {-1}
			\draw[thin] (1pt,\y)--(-1pt,\y) node [right] {\footnotesize$\y$};
			\foreach \y in {1}
			\draw[thin] (1pt,\y)--(-1pt,\y) node [above left] {\footnotesize$\y$};
			\draw[dashed](-1,0)--(-1,1)--(2,1) (1,1)--(1,0) (2,1)--(2,0);
			\begin{scope}
				\clip (-3,-3) rectangle (3,3);
				\draw[name path=(C)] plot[smooth,tension=0.7] coordinates{(-1.15,3)(-0.5,-1.6)(.8,.88)(1.9,0.8)(2.3,3)};
			\end{scope}
		\end{tikzpicture}
	}	\loigiai{
		Ta có $g'(x)=f'(x-1)-1$.\\
		$
		g'(x) \geq 0 \Leftrightarrow f'(x-1)-1 \geq 0 \Leftrightarrow f'(x-1) \geq 1 \Leftrightarrow \hoac{&x - 1 \leq - 1\\
			&x - 1 \geq 2}\Leftrightarrow \hoac{&
			x \leq 0 \\
			&x \geq 3.}
		$\\
		Từ đó suy ra hàm số $g(x)=f(x-1)+\dfrac{2019-2018 x}{2018}$ đồng biến trên khoảng $(-1 ; 0)$.
	}
\end{ex}

\begin{ex}[(Sở Ninh Bình 2019]%[2D1K1-2]
	Cho hàm số $y=f(x)$ có bảng xét dấu của đạo hàm như sau
	\begin{center}
		\begin{tikzpicture}
			\tkzTabInit[nocadre,lgt=1,espcl=2,deltacl=0.6]
			{$x$ /0.7,$f'(x)$ /0.7}
			{$-\infty$,$-2$,$-1$,$2$,$4$,$+\infty$}
			\tkzTabLine{,+,$0$,-,$0$,+,$0$,-,$0$,+,}
		\end{tikzpicture}
	\end{center}
	Hàm số $y=-2 f(x)+2019$ nghịch biến trên khoảng nào trong các khoảng dưới đây?
	\choice
	{$(-4 ; 2)$}
	{\True $(-1 ; 2)$}
	{$(-2 ;-1)$}
	{$(2 ; 4)$}
	\loigiai{
		Xét $y=g(x)=-2 f(x)+2019$.\\
		Ta có $g'(x)=(-2 f(x)+2019)'=-2 f'(x), g'(x)=0 \Leftrightarrow\hoac{&x=-2 \\&x=-1 \\&x=2 \\&x=4.}$.\\
		Ta có bảng xét dấu của $g'(x)$
		\begin{center}
			\begin{tikzpicture}
				\tkzTabInit[nocadre,lgt=1,espcl=2,deltacl=0.6]
				{$x$ /0.6,$f'(x)$ /0.6}
				{$-\infty$,$-2$,$-1$,$2$,$4$,$+\infty$}
				\tkzTabLine{,-,$0$,+,$0$,-,$0$,+,$0$,+,}
			\end{tikzpicture}
		\end{center}
		Dựa vào bảng xét dấu, ta thấy hàm số $y=g(x)$ nghịch biến trên khoảng $(-1 ; 2)$.
	}
\end{ex}
\begin{ex}[THPT Lương Thế Vinh Hà Nội 2019]%[2D1G1-2]
	\immini{
		Cho hàm số $y=f(x)$. Biết đồ thị hàm số $y=f'(x)$ có đồ thị như hình vẽ bên. 
		Hàm số $y=f \left(3-x^2\right)+2018$ đồng biến trên khoảng nào dưới đây?
		\choice
		{\True $(-1 ; 0)$}
		{$(2 ; 3)$}
		{$(-2 ;-1)$}
		{$(0 ; 1)$}
	}
	{
		\begin{tikzpicture}[scale=0.6,>=stealth, font=\footnotesize, line join=round, line cap=round]
			\def\a{0.065} \def\b{0.32} \def\c{-0.53} \def\d{-0.82} % Hệ số
			\def\xmin{-8} \def\xmax{4}
			\def\ymin{-3} \def\ymax{3} 
			%\draw[color=gray!50,dashed] (\xmin,\ymin) grid (\xmax,\ymax); 
			\draw[->] (\xmin,0)--(\xmax,0) node [below]{$x$};
			\draw[->] (0,\ymin)--(0,\ymax) node [left]{$y$};
			\node at (0,0) [below left]{$O$};
			\node at (-6,0) [below left]{$-6$};
			\node at (-1,0) [below left]{$-1$};
			\node at (2,0) [below right]{$2$};
			\clip (\xmin+0.1,\ymin+0.1) rectangle (\xmax-0.5,\ymax-0.1);
			\draw[smooth,samples=300][domain=-6.5:3.5] plot(\x,{\a*(\x)^3+\b*(\x)^2+\c*(\x)+\d});
		\end{tikzpicture}
	}
	
	\loigiai{
		Ta có $\left[f\left( 3-x^2\right)+2018 \right]'=-2 x \cdot f'\left(3-x^2\right) $.\\
		$
		-2 x \cdot f'\left(3-x^2\right)=0 \Leftrightarrow\hoac{&
			x = 0\\
			&3 - x ^{2}= - 6\\
			&3 - x ^{2}= - 1\\
			&3 - x ^{2}= 2}
		\Leftrightarrow \hoac{
			&x=0 \\
			&x=\pm 3 \\
			&x=\pm 2 \\
			&	x=\pm 1.}
		$\\
		Bảng xét dấu của đạo hàm hàm số đã cho
		\begin{center}
			\begin{center}
				\begin{tikzpicture}
					\tkzTabInit[nocadre,lgt=2.9,espcl=1.5,deltacl=0.6]
					{$x$ /0.7,$f'\left( 3-x^2\right) $/0.7,$-2xf'\left( 3-x^2\right)$/0.8}
					{$-\infty$,$-3$,$-2$,$-1$,$0$,$1$,$2$,$3$,$+\infty$}
					\tkzTabLine{,-,$0$,+,$0$,-,$0$,+,$0$,+,$0$,-,$0$,+,$0$,-}
					\tkzTabLine{,-,$0$,+,$0$,-,$0$,+,$0$,-,$0$,+,$0$,-,$0$,+}
				\end{tikzpicture}
			\end{center}
		\end{center}
		Từ bảng xét dấu suy ra hàm số đồng biến trên $(-1 ; 0)$.
	}
\end{ex}
\begin{ex}[Chuyên Biên Hòa - Hà Nam - 2020]%[2D1G1-2]
	\immini{
		Cho hàm số đa thức $f(x)$ có đạo hàm trên $\mathbb{R}$. Biết $f(0)=0$ và đồ thị hàm số $y=f'(x)$ như hình sau.
		Hàm số $g(x)=\left|4 f(x)+x^2\right|$ đồng biến trên khoảng nào dưới đây?
		\choice
		{$(4 ;+\infty)$}
		{\True $(0 ; 4)$}
		{$(-\infty ;-2)$}
		{$(-2 ; 0)$}
	}	
	{
		\begin{tikzpicture}[scale=0.7,>=stealth, font=\footnotesize, line join=round, line cap=round]
			%\def\a{1} \def\b{-6} \def\c{9} \def\d{1} % Hệ số
			\def\xmin{-4} \def\xmax{6}
			\def\ymin{-3} \def\ymax{2} 
			%\draw[color=gray!50,dashed] (\xmin,\ymin) grid (\xmax,\ymax); 
			\draw[->] (\xmin,0)--(\xmax,0) node [below]{$x$};
			\draw[->] (0,\ymin)--(0,\ymax) node [left]{$y$};
			\node at (0,0) [below left]{$O$};
			%\node at (1,3) [below left]{$f'(x)$};
			%\node at (-1.3,4) {$f'(x)$};
			\draw[dashed] (-2,0) node[below]{$-2$}--(-2,1)--(0,1) node[below left]{$1$};
			\draw[dashed] (4,0) node[below]{$4$}--(4,-2)--(0,-2) node[below left]{$-2$};
			%\draw[dashed] (1,0) node[below]{$1$}--(1,1);
			%\draw[dashed] (-0.5,0) node[below left]{$-0{,}5$}--(-0.5,2.125);
			\clip (\xmin+0.1,\ymin+0.1) rectangle (\xmax-0.5,\ymax-0.1);
			\draw[smooth,samples=300][domain=-4:5.5] plot(\x,{0.071*(\x)^3-0.142*(\x)^2-1.07*(\x)});
		\end{tikzpicture}
	}
	\loigiai{
		\immini{
			Xét hàm số $h(x)=4 f(x)+x^2$ trên $\mathbb{R}$.\\
			Vì $f(x)$ là hàm số đa thức nên $h(x)$ cũng là hàm số đa thức và $h(0)=4 f(0)=0$.\\
			Ta có $h'(x)=4 f'(x)+2 x$. Do đó $h'(x)=0 \Leftrightarrow f'(x)=-\dfrac{1}{2}x$.\\
		}
		{
			\begin{tikzpicture}[scale=0.7,>=stealth, font=\footnotesize, line join=round, line cap=round]
				%\def\a{1} \def\b{-6} \def\c{9} \def\d{1} % Hệ số
				\def\xmin{-4} \def\xmax{6}
				\def\ymin{-3} \def\ymax{2} 
				%\draw[color=gray!50,dashed] (\xmin,\ymin) grid (\xmax,\ymax); 
				\draw[->] (\xmin,0)--(\xmax,0) node [below]{$x$};
				\draw[->] (0,\ymin)--(0,\ymax) node [left]{$y$};
				\node at (0,0) [below left]{$O$};
				%\node at (1,3) [below left]{$f'(x)$};
				%\node at (-1.3,4) {$f'(x)$};
				\draw[dashed] (-2,0) node[below]{$-2$}--(-2,1)--(0,1) node[below left]{$1$};
				\draw[dashed] (4,0) node[below]{$4$}--(4,-2)--(0,-2) node[below left]{$-2$};
				%\draw[dashed] (1,0) node[below]{$1$}--(1,1);
				%\draw[dashed] (-0.5,0) node[below left]{$-0{,}5$}--(-0.5,2.125);
				\clip (\xmin+0.1,\ymin+0.1) rectangle (\xmax-0.5,\ymax-0.1);
				\draw[smooth,samples=300][domain=-4:5.5] plot(\x,{0.071*(\x)^3-0.142*(\x)^2-1.07*(\x)});
				\draw[smooth,samples=300][domain=-4:5.5] plot(\x,{-0.5*(\x)});
			\end{tikzpicture}
		}
		Dựa vào sự tương giao của đồ thị hàm số $y=f'(x)$ và đường thẳng $y=-\dfrac{1}{2}x$, ta có
		$
		h'(x)=0 \Leftrightarrow x \in\{-2 ; 0 ; 4\}.\\
		$
		Bảng biến thiên của hàm số $h(x)$ như sau:
		\begin{center}
			\begin{tikzpicture}
				\tkzTabInit[nocadre,lgt=1.2,espcl=2.5,deltacl=0.6]
				{$x$ /0.6,$y'$ /0.6,$y$ /2}
				{$-\infty$,$-2$,$0$,$4$,$+\infty$}
				\tkzTabLine{,-,$0$,+,$0$,-,$0$,+,}
				\tkzTabVar{+/$+\infty$, -/$y_1$,+/$0$,-/$y_3$,+/$+\infty$}
			\end{tikzpicture}
		\end{center}
		Từ đó suy ra bảng biến thiên của hàm số $g(x)=|h(x)|$.\\
		Dựa vào bảng biến thiên trên, ta thấy hàm số $g(x)$ đồng biến trên khoảng $(0 ; 4)$.
	}
\end{ex}
\begin{ex}[Chuyên Thái Bình - 2020]%[2D1G1-2]
	\immini{
		Cho hàm số $f(x)$ liên tục trên $\mathbb{R}$ có đồ thị hàm số $y=f'(x)$ cho như hình vẽ bên.\\
		Hàm số $g(x)=2 f(|x-1|)-x^2+2 x+2020$ đồng biến trên khoảng nào?
		\choice
		{\True $(0 ; 1)$}
		{$(-3 ; 1)$}
		{$(1 ; 3)$}
		{$(-2 ; 0)$}
	}
	{
		\begin{tikzpicture}[scale=0.7,>=stealth, font=\footnotesize, line join=round, line cap=round]
			%\def\a{1} \def\b{-6} \def\c{9} \def\d{1} % Hệ số
			\def\xmin{-4} \def\xmax{5}
			\def\ymin{-3} \def\ymax{5} 
			%\draw[color=gray!50,dashed] (\xmin,\ymin) grid (\xmax,\ymax); 
			\draw[->] (\xmin,0)--(\xmax,0) node [below]{$x$};
			\draw[->] (0,\ymin)--(0,\ymax) node [left]{$y$};
			\node at (0,0) [below left]{$O$};
			%\node at (1,3) [below left]{$f'(x)$};
			\node at (-1.3,4) {$f'(x)$};
			\draw[dashed] (-1,0) node[above]{$-1$}--(-1,-1)--(0,-1) node[below left]{$-1$};
			\draw[dashed] (1,0) node[below]{$1$}--(1,1)--(0,1) node[below left]{$1$};
			\draw[dashed] (3,0) node[below]{$3$}--(3,3)--(0,3) node[below left]{$3$};
			%\draw[dashed] (1,0) node[below]{$1$}--(1,1);
			%\draw[dashed] (-0.5,0) node[below left]{$-0{,}5$}--(-0.5,2.125);
			\clip (\xmin+0.1,\ymin+0.1) rectangle (\xmax-0.5,\ymax-0.1);
			\draw[smooth,samples=300][domain=-2:4] plot(\x,{-0.5*(\x)^3+1.5*(\x)^2+1.5*(\x)-1.5});
			%\draw[smooth,samples=300] plot(\x,{(\x)^3+(\x)^2-2*(\x)+1});
		\end{tikzpicture}
	}
	\loigiai{
		Ta có đường thẳng $y=x$ cắt đồ thị hàm số $y=f'(x)$ tại các điểm $x=-1 ; x=1 ; x=3$ như hình vẽ sau:
		\begin{center}
			\begin{tikzpicture}[scale=0.7,>=stealth, font=\footnotesize, line join=round, line cap=round]
				%\def\a{1} \def\b{-6} \def\c{9} \def\d{1} % Hệ số
				\def\xmin{-4} \def\xmax{5}
				\def\ymin{-3} \def\ymax{5} 
				%\draw[color=gray!50,dashed] (\xmin,\ymin) grid (\xmax,\ymax); 
				\draw[->] (\xmin,0)--(\xmax,0) node [below]{$x$};
				\draw[->] (0,\ymin)--(0,\ymax) node [left]{$y$};
				\node at (0,0) [below left]{$O$};
				%\node at (1,3) [below left]{$f'(x)$};
				\node at (-1.3,4) {$f'(x)$};
				\node at (4,3.2) {$y=x$};
				\draw[dashed] (-1,0) node[above]{$-1$}--(-1,-1)--(0,-1) node[below left]{$-1$};
				\draw[dashed] (1,0) node[below]{$1$}--(1,1)--(0,1) node[below left]{$1$};
				\draw[dashed] (3,0) node[below]{$3$}--(3,3)--(0,3) node[below left]{$3$};
				%\draw[dashed] (1,0) node[below]{$1$}--(1,1);
				%\draw[dashed] (-0.5,0) node[below left]{$-0{,}5$}--(-0.5,2.125);
				\clip (\xmin+0.1,\ymin+0.1) rectangle (\xmax-0.5,\ymax-0.1);
				\draw[smooth,samples=300][domain=-2:4] plot(\x,{-0.5*(\x)^3+1.5*(\x)^2+1.5*(\x)-1.5});
				\draw[smooth,samples=300] plot(\x,{(\x)});
			\end{tikzpicture}
		\end{center}
		Dựa vào đồ thị của hai hàm số trên ta có $f'(x)>x \Leftrightarrow\hoac{&x<-1 \\ &1<x<3}$ và
		$ f'(x)<x \Leftrightarrow\hoac{&
			-1<x<1 \\
			&x>3.}$\\
		+Trường hợp 1: $x-1<0 \Leftrightarrow x<1$.\\
		Khi đó $g(x)=2 f(1-x)-x^2+2 x+2020$.\\
		Ta có $g'(x)=-2 f'(1-x)+2(1-x)$.
		$$
		g'(x)>0 \Leftrightarrow-2 f'(1-x)+2(1-x)>0 \Leftrightarrow f'(1-x)<1-x \Leftrightarrow\hoac{
			&- 1 < 1 - x < 1\\
			&1 - x > 3} \Leftrightarrow \hoac{&
			0<x<2 \\
			&x<-2.}
		$$
		Kết hợp điều kiện, ta có $g'(x)>0 \Leftrightarrow\hoac{&0<x<1 \\ &x<-2.}$\\
		
		+ Trường hợp 2: $x-1>0 \Leftrightarrow x>1$.\\
		Khi đó ta có $g(x)=2 f(x-1)-x^2+2 x+2020$.\\
		$ g'(x)=2 f'(x-1)-2(x-1)$\\
		$g'(x)>0 \Leftrightarrow 2 f'(x-1)-2(x-1)>0 \Leftrightarrow f'(x-1)>x-1 \Leftrightarrow\hoac{&
			x - 1 < - 1\\
			&1 < x - 1 < 3}\Leftrightarrow \hoac{
			&x<0 \\
			&2<x<4.}$
		Kết hợp điều kiện ta có $g'(x)>0 \Leftrightarrow 2<x<4$.\\
		Vậy hàm số $g(x)=2 f(|x-1|)-x^2+2 x+2020$ đồng biến trên khoảng $(0 ; 1)$.
	}
\end{ex}

\begin{ex}[Chuyên Lào Cai - 2020]%[2D1G1-2]
	\immini{
		Cho hàm số $f'(x)$ có đồ thị như hình bên.\\
		Hàm số $g(x)=f(3 x+1)+9 x^3+\dfrac{9}{2}x^2$ đồng biến trên khoảng nào dưới đây?
		\choice
		{$(-1 ; 1)$}
		{$(-2 ; 0)$}
		{$(-\infty ; 0)$}
		{\True $(1 ;+\infty)$}
	}
	{\begin{tikzpicture}[line join=round, line cap=round,>=stealth,thick,scale=.8]
			\tikzset{label style/.style={font=\footnotesize}}
			\draw[->] (-2.1,0)--(5.1,0) node[below left] {$x$};
			\draw[->] (0,-3.1)--(0,4.1) node[below left] {$y$};
			\draw (0,0) node [below left] {$O$};
			\foreach \x in {1,2,3}
			\draw[thin] (\x,1pt)--(\x,-1pt) node [below] {$\x$};
			\draw[thin](-1,1pt)--(1,-1pt)node[above left]{$-1$};
			\foreach \y in {-2,2}
			\draw[thin] (1pt,\y)--(-1pt,\y) node [above right] {$\y$};
			%\begin{scope}
			\clip (-2,-3) rectangle (5,4);
			\draw[samples=200,domain=-2:4,smooth,variable=\x] plot (\x,{(\x)^3-3*(\x)^2+2});
			%\end{scope}
			\draw[dashed] (-1,0)--(-1,-2)--(0,-2);
			\draw[dashed] (3,0)--(3,2)--(0,2);
			%\begin{scope}[on background layer]\path[white]node{MDD-134};\end{scope}
		\end{tikzpicture}
	}
	\loigiai
	{
		\immini{Xét hàm số $g(x)=f(3 x+1)+9 x^3+\dfrac{9}{2}x^2 \\
			\Rightarrow g'(x)=3 f'(3 x+1)+27 x^2+9 x$.\\
			Hàm số đồng biến  $\Leftrightarrow g'(x)>0 \Leftrightarrow 3 f'(3 x+1)+27 x^2+9 x>0$
			\\
			$
			\Leftrightarrow f'(3 x+1)+3 x(3 x+1)>0 \qquad (*)
			$\\
			Đặt $t=3 x+1$, khi đó  $(*) \Leftrightarrow f'(t)+(t-1) t>0$\\ $\Leftrightarrow f'(t)>-t^2+t$.\\
			Vẽ parabol $y=-x^2+x$ và đồ thị hàm số $f'(x)$ trên cùng một hệ trục
		}
		{
			\begin{tikzpicture}[line join=round, line cap=round,>=stealth,thick,scale=.8]
				\tikzset{label style/.style={font=\footnotesize}}
				\draw[->] (-2.1,0)--(5.1,0) node[below left] {$x$};
				\draw[->] (0,-3.1)--(0,4.1) node[below left] {$y$};
				\draw (0,0) node [below left] {$O$};
				\foreach \x in {1,2,3}
				\draw[thin] (\x,1pt)--(\x,-1pt) node [below] {$\x$};
				\draw[thin](-1,1pt)--(1,-1pt);
				\foreach \y in {-2,2}
				\draw[thin] (1pt,\y)--(-1pt,\y) node [above right] {$\y$};
				%\begin{scope}
				\clip (-2,-3) rectangle (5,4);
				\draw[samples=200,domain=-2:4,smooth,variable=\x] plot (\x,{(\x)^3-3*(\x)^2+2});
				\draw[samples=200,domain=-2:4,smooth,variable=\x] plot (\x,{-(\x)^2+(\x)});
				%\end{scope}
				\draw[dashed] (-1,0) node[above left]{$-1$}--(-1,-2)--(0,-2);
				\draw[dashed] (3,0)--(3,2)--(0,2);
				%\begin{scope}[on background layer]\path[white]node{MDD-134};\end{scope}
			\end{tikzpicture}
		}
		Dựa vào đồ thị ta thấy
		$
		f'(t)>-t^2+t \Leftrightarrow\hoac{&- 1 < t < 1\\
			&t > 2}\Rightarrow \hoac{&
			- 1 < 3 x + 1 < 1\\
			&3 x + 1 > 2} \Leftrightarrow \hoac{&
			\dfrac{-2}{3}<x<0\\
			&x>\dfrac{1}{3}.}
		$}
\end{ex}
\begin{ex}[Sở Phú Thọ-2020]%[2D1G1-2]
	\immini{
		Cho hàm số $y=f(x)$ có đồ thị hàm số $y=f'(x)$ như hình vẽ.\\
		Hàm số $g(x)=f\left(\mathrm{e}^x-2\right)-2020$ nghịch biến trên khoảng nào dưới đây?
		\choice
		{\True $\left(-1 ; \dfrac{3}{2}\right)$}
		{$(-1 ; 2)$}
		{$(0 ;+\infty)$}
		{$\left(\dfrac{3}{2}; 2\right)$}
	}
	{
		\begin{tikzpicture}[scale=0.7,>=stealth, font=\footnotesize, line join=round, line cap=round]
			\def\a{1} \def\b{-3} \def\c{0} \def\d{0} % Hệ số
			\def\xmin{-2} \def\xmax{4}
			\def\ymin{-5} \def\ymax{2} 
			%\draw[color=gray!50,dashed] (\xmin,\ymin) grid (\xmax,\ymax); 
			\draw[->] (\xmin,0)--(\xmax,0) node [below]{$x$};
			\draw[->] (0,\ymin)--(0,\ymax) node [left]{$y$};
			\node at (0,0) [above left]{$O$};
			\node at (3,0) [below right]{$3$};
			\draw[dashed] (2,0) node[above]{$2$}--(2,-4) --(0,-4) node[left]{$-4$};
			\clip (\xmin+0.1,\ymin+0.1) rectangle (\xmax-0.5,\ymax-0.1);
			\draw[smooth,samples=300] plot(\x,{\a*(\x)^3+\b*(\x)^2+\c*(\x)+\d});
		\end{tikzpicture}
	}
	
	\loigiai{
		Dựa vào đồ thị hàm số $y=f'(x)$ suy ra $f'(x) \leq 0 ~ \forall x<3$ và $f'(x)>0 ~ \forall x>3$.
		$
		g'(x)=\mathrm{e}^x f'\left(\mathrm{e}^x-2\right) .
		$
		Hàm số $g(x)=f\left(\mathrm{e}^x-2\right)-2020$ nghịch biến \\ $ \Leftrightarrow g'(x)<0 \Leftrightarrow \mathrm{e}^x f'\left(\mathrm{e}^x-2\right)<0$\\
		$
		\Leftrightarrow f'\left(\mathrm{e}^x-2\right)<0 \Leftrightarrow \mathrm{e}^x-2<3 \Leftrightarrow \mathrm{e}^x<5 \Leftrightarrow x<\ln 5 .
		$\\
		Vậy hàm số đã cho nghịch biến trên $\left(-1 ; \dfrac{3}{2}\right)$.
	}
\end{ex}
\begin{ex}[Lý Nhân Tông - Bắc Ninh - 2020]%[2D1G1-2]
	\immini{
		Cho hàm số $f(x)$ có đồ thị hàm số $f'(x)$ như hình vẽ.\\
		Hàm số $y=f(\cos x)+x^2-x$ đồng biến trên khoảng
		\choice
		{$(-2 ; 1)$}
		{$(0 ; 1)$}
		{\True $(1 ; 2)$}
		{$(-1 ; 0)$}
	}
	{
		\begin{tikzpicture}[scale=1,>=stealth, font=\footnotesize, line join=round, line cap=round]
			\def\a{-0.5} \def\b{0} \def\c{1.5} \def\d{0} % Hệ số
			\def\xmin{-3} \def\xmax{4}
			\def\ymin{-2} \def\ymax{2} 
			%\draw[color=gray!50,dashed] (\xmin,\ymin) grid (\xmax,\ymax); 
			\draw[->] (\xmin,0)--(\xmax,0) node [below]{$x$};
			\draw[->] (0,\ymin)--(0,\ymax) node [left]{$y$};
			\node at (0,0) [above left]{$O$};
			\node at (3,0) [below right]{$3$};
			\draw[dashed] (-2,0) node[below]{$-2$}--(-2,1) --(0,1) node[above right]{$1$} --(1,1)--(1,0) node[below]{$1$};
			\draw[dashed] (-1,0) node[below right]{$-1$}--(-1,-1) --(0,-1) node[above right]{$-1$} --(2,-1)--(2,0) node[below right]{$2$};
			\clip (\xmin+0.1,\ymin+0.1) rectangle (\xmax-0.5,\ymax-0.1);
			\draw[smooth,samples=300][domain=-2:2] plot(\x,{\a*(\x)^3+\b*(\x)^2+\c*(\x)+\d});
		\end{tikzpicture}
	}
	\loigiai{
		Đặt  $g(x)=f(\cos x)+x^2-x$.\\
		Ta có $g'(x)=-\sin x \cdot f'(\cos x)+2 x-1$\\
		Vì $\cos x \in[-1 ; 1]$ nên từ đồ thị $f'(x)$ ta suy ra $f'(\cos x) \in[-1 ; 1]$.\\
		Do đó $\left|-\sin x \cdot f'(\cos x)\right| \leq 1, ~\forall x \in \mathbb{R}$.\\
		Ta suy ra $g'(x)=\sin x \cdot f'(\cos x)+2 x-1 \geq-1+2 x-1=2 x-2$
		$\Rightarrow g'(x)>0, ~\forall x>1$.\\
		Vậy hàm số đồng biến trên $(1 ; 2)$.
	}
\end{ex}
\begin{ex}[THPT Nguyễn Viết Xuân - 2020]%[2D1G1-2]
	\immini{
		Cho hàm số $f(x)$. Hàm số $y=f'(x)$ có đồ thị như hình vẽ.\\
		Hàm số $g(x)=f\left(3 x^2-1\right)-\dfrac{9}{2}x^4+3 x^2$ đồng biến trên khoảng nào dưới đây?
		\choice
		{\True $\left(-\dfrac{2 \sqrt{3}}{3}; \dfrac{-\sqrt{3}}{3}\right)$}
		{$\left(0 ; \dfrac{2 \sqrt{3}}{3}\right)$}
		{$(1 ; 2)$}
		{$\left(-\dfrac{\sqrt{3}}{3}; \dfrac{\sqrt{3}}{3}\right)$} 
	}
	{
		\begin{tikzpicture}[scale=0.6,>=stealth, font=\footnotesize, line join=round, line cap=round]
			\def\a{0.25} \def\b{0.25} \def\c{-2} \def\d{0} % Hệ số
			\def\xmin{-5} \def\xmax{4}
			\def\ymin{-5} \def\ymax{5} 
			%\draw[color=gray!50,dashed] (\xmin,\ymin) grid (\xmax,\ymax); 
			\draw[->] (\xmin,0)--(\xmax,0) node [below]{$x$};
			\draw[->] (0,\ymin)--(0,\ymax) node [left]{$y$};
			\node at (0,0) [above left]{$O$};
			%\node at (3,0) [below right]{$3$};
			\draw[dashed] (-4,0) node[below left]{$-4$}--(-4,-4) --(0,-4) node[above right]{$-4$};
			\draw[dashed] (3,0) node[below right]{$3$}--(3,3) --(0,3) node[above right]{$3$};
			\clip (\xmin+0.1,\ymin+0.1) rectangle (\xmax-0.5,\ymax-0.1);
			\draw[smooth,samples=300] plot(\x,{\a*(\x)^3+\b*(\x)^2+\c*(\x)+\d});
		\end{tikzpicture}
	}
	
	\loigiai
	{
		TXĐ: $\mathscr{D}=\mathbb{R}$.\\
		Ta có $g'(x)=6 x f'\left(3 x^2-1\right)-18 x^3+6 x=6 x\left[f'\left(3 x^2-1\right)-3 x^2+1\right]$.\\
		$
		g'(x)=0 \Leftrightarrow\hoac{
			&x = 0\\
			&f '( 3 x ^{2}- 1 ) = 3 x ^{2}- 1}
		\Leftrightarrow \hoac{
			&x = 0\\
			&3 x ^{2}- 1 = - 4 \text{~(vô nghiệm)}\\
			&3 x ^{2}- 1 = 0\\
			&3 x ^{2}- 1 = 3}\Leftrightarrow \hoac{&x=0 \\
			&x=\pm \dfrac{\sqrt{3}}{3}\\
			&x=\pm \dfrac{2 \sqrt{3}}{3}.}
		$\\
		Bảng xét dấu
		\begin{center}
			\begin{tikzpicture}
				\tkzTabInit[nocadre,lgt=1.2,espcl=2.2,deltacl=0.6]
				{$x$ /1.2,$f'(x)$ /0.7}
				{$-\infty$,$-\dfrac{2 \sqrt{3}}{3}$,$-\dfrac{ \sqrt{3}}{3}$,$0$,$\dfrac{\sqrt{3}}{3}$,$\dfrac{2 \sqrt{3}}{3}$,$+\infty$}
				\tkzTabLine{,-,$0$,+,$0$,-,$0$,+,$0$,-,$0$,+,}
			\end{tikzpicture}
		\end{center}
		Vậy hàm số đồng biến trong khoảng $\left(-\dfrac{2 \sqrt{3}}{3}; \dfrac{-\sqrt{3}}{3}\right)$.}
\end{ex}
\begin{ex}[Trần Phú - Quảng Ninh - 2020]%[2D1G1-2]
	Cho hàm số $f(x)$ có bảng xét dấu của đạo hàm như sau
	\begin{center}
		\begin{tikzpicture}
			\tkzTabInit[nocadre,lgt=1.2,espcl=2,deltacl=0.6]
			{$x$ /0.6,$f'(x)$ /0.6}
			{$-\infty$,$-4$,$-1$,$2$,$7$,$+\infty$}
			\tkzTabLine{,+,$0$,-,$0$,+,$0$,-,$0$,+,}
		\end{tikzpicture}
	\end{center}
	Hàm số $y=f(2 x+1)+\dfrac{2}{3}x^3-8 x+5$ nghịch biến trên khoảng nào dưới đây?
	\choice
	{$(-\infty ;-2)$}
	{$(1 ;+\infty)$}
	{$(-1 ; 7)$}
	{\True $\left(-1 ; \dfrac{1}{2}\right)$}
	\loigiai{
		Ta có $y'=2 f'(2 x+1)+2 x^2-8$.\\
		Xét $y'\leq 0 \Leftrightarrow 2 f'(2 x+1)+2 x^2-8 \leq 0 \Leftrightarrow f'(2 x+1) \leq 4-x^2$.\\
		Đặt $t=2x+1$, ta có $f'(t) \leq \dfrac{-t^2+2 t+15}{4}$.\\
		Vì $\dfrac{-t^2+2 t+15}{4}\geq 0, \forall t \in[-3 ; 5]$.\\
		Mà $f'(t) \leq 0, \forall t \in[-3 ; 2]$.\\
		Nên $f'(t) \leq \dfrac{-t^2+2 t+15}{4}\Rightarrow t \in[-3 ; 2]$.\\
		Suy ra $-3 \leq 2 x+1 \leq 2 \Leftrightarrow-2 \leq x \leq \dfrac{1}{2}$.}
\end{ex}

\begin{ex}[Chuyên Thái Bình - Lần 3 - 2020]%[2D1G1-2]
	\immini{
		Cho hàm số $y=f(x)$ liên tục trên $\mathbb{R}$ có đồ thị hàm số $y=f'(x)$ cho như hình vẽ.\\
		Hàm số $g(x)=2 f(|x-1|)-x^2+2 x+2020$ đồng biến trên khoảng nào?
		\choice
		{\True $(0 ; 1)$}
		{$(-3 ; 1)$}
		{$(1 ; 3)$}
		{$(-2 ; 0)$}
	}
	{
		\begin{tikzpicture}[scale=0.7,>=stealth, font=\footnotesize, line join=round, line cap=round]
			\def\a{-0.333} \def\b{1} \def\c{1.333} \def\d{-1} % Hệ số
			\def\xmin{-3} \def\xmax{5}
			\def\ymin{-3} \def\ymax{5} 
			%\draw[color=gray!50,dashed] (\xmin,\ymin) grid (\xmax,\ymax); 
			\draw[->] (\xmin,0)--(\xmax,0) node [below]{$x$};
			\draw[->] (0,\ymin)--(0,\ymax) node [left]{$y$};
			\node at (0,0) [above left]{$O$};
			%\node at (3,0) [below right]{$3$};
			\draw[dashed] (-1,0) node[above]{$-1$}--(-1,-1) --(0,-1) node[above right]{$-1$};
			\draw[dashed] (1,0) node[below right]{$1$}--(1,1) --(0,1) node[above right]{$1$};
			\draw[dashed] (3,0) node[below right]{$3$}--(3,3) --(0,3) node[above right]{$3$};
			\clip (\xmin+0.1,\ymin+0.1) rectangle (\xmax-0.5,\ymax-0.1);
			\draw[smooth,samples=300] plot(\x,{\a*(\x)^3+\b*(\x)^2+\c*(\x)+\d});
			\draw[smooth,samples=300] plot(\x,{(\x)});
		\end{tikzpicture}
	}
	\loigiai{
		Với $x>1$, ta có $g(x)=2 f(x-1)-(x-1)^2+2021 \Rightarrow g'(x)=2 f'(x-1)-2(x-1)$.\\
		Hàm số đồng biến $\Leftrightarrow 2 f'(x-1)-2(x-1)>0 \Leftrightarrow f'(x-1)>x-1 \quad(*)$.\\
		Đặt $t=x-1$, khi đó $(*) \Leftrightarrow f'(t)>t \Leftrightarrow\hoac{&1<t<3 \\ &t<-1}\Rightarrow\hoac{&2<x<4 \\ &x<0 ~(\text{loại}).}$\\
		Với $x<1$, ta có $g(x)=2 f(1-x)-(1-x)^2+2021 \Rightarrow g'(x)=-2 f'(1-x)+2(1-x)$.\\
		Hàm số đồng biến $\Leftrightarrow-2 f'(1-x)+2(1-x)>0 \Leftrightarrow f'(1-x)<1-x \quad(* *)$.\\
		Đặt $t=1-x$, khi đó $(* *) \Leftrightarrow f'(t)<t \Leftrightarrow\hoac{&-1<t<1 \\ &t>3}\Rightarrow\hoac{&0<x<2 \\ &x<-2}\Rightarrow\hoac{&0<x<1 \\ &x<-2.}$\\
		Vậy hàm số $g(x)$ đồng biến trên các khoảng $(-\infty ;-2),(0 ; 1),(2 ; 4)$.
	}
\end{ex}
\begin{ex}[Sở Phú Thọ - 2020]%[2D1G1-2]
	\immini{
		Cho hàm số $y=f(x)$ có đồ thị hàm số $f'(x)$ như hình vẽ.\\
		Hàm số $g(x)=f\left(1+e^x\right)+2020$ nghịch biến trên khoảng nào dưới đây?
		\choice
		{$(0 ;+\infty)$}
		{$\left(\dfrac{1}{2}; 1\right)$}
		{\True $\left(0 ; \dfrac{1}{2}\right)$}
		{$(-1 ; 1)$}
	}{
		\begin{tikzpicture}[scale=0.7,>=stealth, font=\footnotesize, line join=round, line cap=round]
			\def\a{1} \def\b{-3} \def\c{0} \def\d{0} % Hệ số
			\def\xmin{-2} \def\xmax{4}
			\def\ymin{-5} \def\ymax{2} 
			%\draw[color=gray!50,dashed] (\xmin,\ymin) grid (\xmax,\ymax); 
			\draw[->] (\xmin,0)--(\xmax,0) node [below]{$x$};
			\draw[->] (0,\ymin)--(0,\ymax) node [left]{$y$};
			\node at (0,0) [above left]{$O$};
			\node at (3,0) [below right]{$3$};
			\draw[dashed] (2,0) node[above]{$2$}--(2,-4) --(0,-4) node[left]{$-4$};
			\clip (\xmin+0.1,\ymin+0.1) rectangle (\xmax-0.5,\ymax-0.1);
			\draw[smooth,samples=300] plot(\x,{\a*(\x)^3+\b*(\x)^2+\c*(\x)+\d});
		\end{tikzpicture}
	}
	\loigiai{
		$g'(x)=e^x f'\left(1+e^x\right)$.\\
		Do $e^x>0, \forall x$ nên $g'(x) \leq 0 \Leftrightarrow f'\left(1+e^x\right) \leq 0 \Leftrightarrow 1+e^x \leq 3 \Leftrightarrow x \leq \ln 2$, dấu bằng xảy ra tại hữu hạn điểm.\\
		Nên $g(x)$ nghịch biến trên $(-\infty ; \ln 2)$.\\
		Vì $\left(0 ; \dfrac{1}{2}\right) \subset (-\infty ; \ln 2)$ nên hàm số đã cho nghịch biến trên $\left(0 ; \dfrac{1}{2}\right)$.
	}
\end{ex}

\begin{ex}%[2D1K1-2]
	[THPT Anh Sơn - Nghệ An - 2020]
	Cho hàm số $y=f(x)$ có bảng xét dấu của đạo hàm như sau.
	\begin{center}
		\begin{tikzpicture}
			\tkzTabInit[nocadre,lgt=1.2,espcl=2,deltacl=0.6]
			{$x$ /0.6,$f'(x)$ /0.6}
			{$-\infty$,$-2$,$-1$,$2$,$4$,$+\infty$}
			\tkzTabLine{,+,$0$,-,$0$,+,$0$,-,$0$,+,}
		\end{tikzpicture}
	\end{center}
	Hàm số $y=-2 f(x)+2019$ nghịch biến trên khoảng nào trong các khoảng dưới đây?
	\choice
	{$(2 ; 4)$}
	{$(-4 ; 2)$}
	{$(-2 ;-1)$}
	{\True $(-1 ; 2)$}
	\loigiai{
		Ta có $y'=-2 f'(x)$.\\
		$
		y'=0 \Leftrightarrow-2 f'(x)=0 \Leftrightarrow\hoac{&
			x=-2 \\
			&x=-1 \\
			&x=2 \\
			&x=4.}$\\
		Từ bảng xét dấu của $f'(x)$ ta có
		\begin{center}
			\begin{tikzpicture}
				\tkzTabInit[nocadre,lgt=1,espcl=2,deltacl=0.6]
				{$x$ /0.6,$y'$ /0.6}
				{$-\infty$,$-2$,$-1$,$2$,$4$,$+\infty$}
				\tkzTabLine{,-,$0$,+,$0$,-,$0$,+,$0$,-,}
			\end{tikzpicture}
		\end{center}
		Từ bảng xét dấu ta có hàm số nghịch biến trên khoảng $(-\infty ;-2),(-1 ; 2)$ và $(4 ;+\infty)$.}
\end{ex}

\begin{ex}[THPT Anh Sơn - Nghệ An - 2020]%[2D1G1-2]
	Cho hàm số $f(x)$ xác định và liên tục trên $\mathbb{R}$ và có đạo hàm $f'(x)$ thỏa mãn $f'(x)=(1-x)(x+2) g(x)+2019$ với $g(x)<0, ~\forall x \in \mathbb{R}$ . Hàm số $y=f(1-x)+2019 x+2020$ nghịch biến trên khoảng nào?
	\choice
	{$(1 ;+\infty)$}
	{$(0 ; 3)$}
	{$(-\infty ; 3)$}
	{\True $(3 ;+\infty)$}
	\loigiai{
		Đặt $h(x)=f(1-x)+2019 x+2020$.\\
		Vì hàm số $f(x)$ xác định trên $\mathbb{R}$ nên hàm số $h(x)$ cũng xác định trên $\mathbb{R}$.\\
		Ta có $h'(x)=-f'(1-x)+2019$.\\
		Do $h'(x)=0$ tại hữu hạn điểm nên để tìm khoảng nghịch biến của hàm số $h(x)$, ta tìm các giá trị của $x$ sao cho $h'(x)<0 \Leftrightarrow-f'(1-x)+2019<0$\\ 
		$\Leftrightarrow f'(1-x)-2019>0 \\
		\Leftrightarrow x(3-x) g(1-x)>0 \Leftrightarrow x(3-x)<0(\text{~do~}g(x)<0, \forall x \in \mathbb{R})$\\
		$\Leftrightarrow\hoac{&
			x<0 \\
			&x>3.}$\\
		Vậy hàm số $y=f(1-x)+2019 x+2020$ nghịch biến trên các khoảng $(-\infty ; 0)$ và $(3 ;+\infty).$}
\end{ex}

\begin{ex}%[2D1G1-2]
	Cho hàm số $y=f(x)$ xác định trên $\mathbb{R}$ và có bảng xét dấu đạo hàm như sau:
	\begin{center}
		\begin{tikzpicture}
			\tkzTabInit[nocadre,lgt=2,espcl=2,deltacl=0.6]
			{$x$ /0.6,$f'(x)$ /0.6}
			{$-\infty$,$-1$,$1$,$4$,$+\infty$}
			\tkzTabLine{,-,$0$,+,$0$,-,$0$,+,}
		\end{tikzpicture}
	\end{center}
	Biết $f(x)>2,~ \forall x \in \mathbb{R}$. Xét hàm số $g(x)=f(3-2 f(x))-x^3+3 x^2-2020$. Khẳng định nào sau đây đúng?
	\choice
	{Hàm số $g(x)$ đồng biến trên khoảng $(-2 ;-1)$}
	{Hàm số $g(x)$ nghịch biến trên khoảng $(0 ; 1)$}
	{Hàm số $g(x)$ đồng biến trên khoảng $(3 ; 4)$}
	{\True Hàm số $g(x)$ nghịch biến trên khoảng $(2 ; 3)$}
	\loigiai{
		Ta có $g'(x)=-2 f'(x) f'(3-2 f(x))-3 x^2+6 x$.\\
		Vì $f(x)>2, ~\forall x \in \mathbb{R}$ nên $3-2 f(x)<-1 ~\forall x \in \mathbb{R}$.\\
		Từ bảng xét dấu $f'(x)$ suy ra $f'(3-2 f(x))<0, ~\forall x \in \mathbb{R}$.\\
		Từ đó ta có bảng xét dấu sau:
		\begin{center}
			\begin{tikzpicture}
				\tkzTabInit[nocadre,lgt=4,espcl=1.7,deltacl=0.6]
				{$x$ /0.7,$-f'(x)f'\left( 3-2f(x)\right) $/0.8,$-3x^2+6x$/0.7}
				{$-\infty$,$-1$,$0$,$1$,$2$,$4$,$+\infty$}
				\tkzTabLine{,-,$0$,+,|,+,$0$,-,|,-,$0$,+,}
				\tkzTabLine{,-,|,-,$0$,+,|,+,$0$,-,|,-,}
			\end{tikzpicture}
		\end{center}
		Từ bảng xét dấu trên, loại trừ đáp án suy ra hàm số $g(x)$ nghịch biến trên khoảng $(2 ; 3)$.}
\end{ex}

\begin{ex}%[2D1G1-2]
	Cho hàm số $f(x)$ có bảng biến thiên như sau:
	\begin{center}
		\begin{tikzpicture}
			\tkzTabInit[nocadre,lgt=1.2,espcl=2.5,deltacl=0.6]
			{$x$ /0.7, $f'(x)$ /0.7, $f(x)$ /2.5}
			{$-\infty$,$1$,$2$,$3$,$4$,$+\infty$}
			\tkzTabLine{,+,$0$,-,$0$,+,$0$,-,$0$,+,}
			\tkzTabVar{-/$-\infty$,+/$3$,-/$1$,+/$2$,-/$0$,+/$+\infty$}
		\end{tikzpicture}
	\end{center}
	Hàm số $y=(f(x))^3-3 .(f(x))^2$ nghịch biến trên khoảng nào dưới đây?
	\choice
	{$(1 ; 2)$}
	{$(3 ; 4)$}
	{$(-\infty ; 1)$}
	{\True $(2 ; 3)$}
	\loigiai{
		Ta có $y'=3 \cdot(f(x))^2 \cdot f'(x)-6 \cdot f(x) \cdot f'(x)=3 f(x) \cdot f'(x) \cdot[f(x)-2]. \\
		y'=0 \Leftrightarrow \hoac{&f(x)=0 \Leftrightarrow x \in\left\{x_1, 4 \mid x_1<1\right\}\\
			&f(x)=2 \Leftrightarrow x \in\left\{x_2, x_3, 3, x_4 \mid x_1<x_2<1<x_3<2 ; 4<x_4\right\}\\
			&f'(x)=0 \Leftrightarrow x \in\{1,2,3,4\}.}$\\
		Lập bảng xét dấu ta có
		\begin{center}
			\begin{tikzpicture}
				\tkzTabInit[nocadre,lgt=2,espcl=1.5,deltacl=0.6]
				{$x$ /0.7,$f(x)$ /0.7,$f(x)-2$ /0.7,$f'(x)$/0.7,$y'$/0.7}
				{$-\infty$,$x_1$,$x_2$,$1$,$x_3$,$2$,$3$,$4$,$x_4$,$+\infty$}
				\tkzTabLine{,-,$0$,+,|,+,|,+,|,+,|,+,$0$,+,|,+,|,+,}
				\tkzTabLine{,-,|,-,$0$,+,$0$,+,$0$,-,|,-,$0$,-,|,-,$0$,+}
				\tkzTabLine{,+,|,+,|,+,$0$,-,|,-,$0$,+,$0$,-,$0$,+,|,+}
				\tkzTabLine{,+,$0$,-,$0$,+,$0$,-,$0$,+,$0$,-,$0$,+,$0$,-,$0$,+}
			\end{tikzpicture}
		\end{center}
		
		Do đó hàm số nghịch biến trên khoảng $(2 ; 3)$.
	}
\end{ex}
\begin{ex}%[2D1G1-2]
	Cho hàm số $y=f(x)$ có đồ thị nằm trên trục hoành và có đạo hàm trên $\mathbb{R}$, bảng xét dấu của biểu thức $f'(x)$ như bảng dưới đây.
	\begin{center}
		\begin{tikzpicture}
			\tkzTabInit[nocadre,lgt=1.2,espcl=2,deltacl=0.6]
			{$x$ /0.6,$f'(x)$ /0.6}
			{$-\infty$,$-2$,$-1$,$3$,$+\infty$}
			\tkzTabLine{,-,$0$,+,$0$,-,$0$,+,}
		\end{tikzpicture}
	\end{center}
	Hàm số $y=g(x)=\dfrac{f\left(x^2-2 x\right)}{f\left(x^2-2 x\right)+1}$ nghịch biến trên khoảng nào dưới đây?
	\choice
	{$(-\infty ; 1)$}
	{$\left(-2 ; \dfrac{5}{2}\right)$}
	{\True $(1 ; 3)$}
	{$(2 ;+\infty)$}
	\loigiai{
		$ g'(x)=\dfrac{\left(x^2-2 x\right)'\cdot f'\left(x^2-2 x\right)}{\left(f\left(x^2-2 x\right)+1\right)^2}=\dfrac{(2 x-2) \cdot f'\left(x^2-2 x\right)}{\left(f\left(x^2-2 x\right)+1\right)^2}. \\
		g'(x)=0 \Leftrightarrow\hoac{
			&2 x - 2 = 0\\
			&f '( x ^{2}- 2 x ) = 0}
		\Leftrightarrow \hoac{&x = 1\\
			&x ^{2}- 2 x = - 2\\
			&x ^{2}- 2 x = - 1\\
			&x ^{2}- 2 x = 3}
		\Leftrightarrow \hoac{&x=1 \\
			&x=-1 \\
			&x=3.}
		$\\
		Ta có bảng xét dấu của $g'(x)$
		\begin{center}
			\begin{tikzpicture}
				\tkzTabInit[nocadre,lgt=1.2,espcl=2,deltacl=0.6]
				{$x$ /0.6,$g'(x)$ /0.6}
				{$-\infty$,$-1$,$1$,$3$,$+\infty$}
				\tkzTabLine{,-,$0$,+,$0$,-,$0$,+,}
			\end{tikzpicture}
		\end{center}
		Dựa vào bảng xét dấu ta có hàm số $y=g(x)$ nghịch biến trên các khoảng $(-\infty ;-1)$ và $(1 ; 3)$.}
\end{ex}
\begin{ex}[Liên trường huyện Quảng Xương - Thanh Hóa - 2021]%[2D1G1-2]
	\immini{
		Cho các hàm số $y=f(x)$; $y=g(x)$ liên tục trên $\mathbb{R}$ và có đồ thị các đạo hàm $f'(x) ; g'(x)$ (đồ thị hàm số $y=g'(x)$ là đường đậm hơn) như hình vẽ.\\
		Hàm số $h(x)=f(x-1)-g(x-1)$ nghịch biến trên khoảng nào dưới đây?
		\choice
		{$\left(\dfrac{1}{2}; 1\right)$}
		{$(1 ;+\infty)$}
		{$(2 ;+\infty)$}
		{\True $\left(-1 ; \dfrac{1}{2}\right)$}
	}
	{
		\begin{tikzpicture}[scale=1,>=stealth, font=\footnotesize, line join=round, line cap=round]
			%\def\a{1} \def\b{-6} \def\c{9} \def\d{1} % Hệ số
			\def\xmin{-4} \def\xmax{3}
			\def\ymin{-2} \def\ymax{4} 
			%\draw[color=gray!50,dashed] (\xmin,\ymin) grid (\xmax,\ymax); 
			\draw[->] (\xmin,0)--(\xmax,0) node [below]{$x$};
			\draw[->] (0,\ymin)--(0,\ymax) node [left]{$y$};
			\node at (0,0) [above left]{$O$};
			\node at (1,3) [below left]{$f'(x)$};
			\node at (1.5,3) [below right]{$g'(x)$};
			\draw[dashed] (-2,0) node[above right]{$-2$}--(-2,1);
			\draw[dashed] (1,0) node[below]{$1$}--(1,1);
			\draw[dashed] (-0.5,0) node[below]{$-0{,}5$}--(-0.5,2.125);
			\clip (\xmin+0.1,\ymin+0.1) rectangle (\xmax-0.5,\ymax-0.1);
			\draw[smooth,samples=300][domain=-3:2] plot(\x,{2*(\x)^4+4*(\x)^3-2*(\x)^2-4*(\x)+1});
			\draw[smooth,samples=300,line width=1.2pt] plot(\x,{(\x)^3+(\x)^2-2*(\x)+1});
		\end{tikzpicture}
	}
	
	\loigiai{
		Ta có: $h'(x)=f'(x-1)-g'(x-1)$.\\
		Dựa vào hình vẽ ta có hàm số $h(x)$ nghịch biến\\
		$\Leftrightarrow h'(x)<0 \Leftrightarrow f'(x-1)<g'(x-1)$\\
		$
		\Leftrightarrow\hoac{&- 2 < x - 1 < - \dfrac{1}{2}\\
			&0 < x - 1 < 1}
		\Leftrightarrow \hoac{
			&-1<x<\dfrac{1}{2}\\
			&1<x<2.}$\\
		Do đó hàm số $h(x)$ nghịch biến trên các khoảng $\left(-1 ; \dfrac{1}{2}\right)$ và $(1 ; 2)$.
	}
\end{ex}
\begin{ex}[THPT Quế Võ 1 - Bắc Ninh - 2021] %[2D1G1-2]
	\immini{
		Cho ba hàm số $y=f(x), y=g(x), y=h(x)$. Đồ thị của ba hàm số $y=f'(x), y=g'(x), y=h'(x)$ được cho như hình vẽ.\\
		Hàm số $k(x)=f(x+7)+g(5 x+1)-h\left(4 x+\dfrac{3}{2}\right)$ đồng biến trên khoảng nào dưới đây?
		\choice
		{$\left(-\dfrac{5}{8}; 0\right)$}
		{$\left(\dfrac{5}{8};+\infty\right)$}
		{\True $\left(\dfrac{3}{8}; 1\right)$}
		{$\left(-\dfrac{3}{8}; 1\right)$}
	}
	{
		\begin{tikzpicture}[scale=0.25,>=stealth, font=\footnotesize, line join=round, line cap=round]
			\def\a{-.078} \def\b{1.25} \def\c{0} % Hệ số
			\def\xmin{-4} \def\xmax{25}
			\def\ymin{-8} \def\ymax{18}
			
			%\draw[color=gray!50,dashed] (\xmin,\ymin) grid (\xmax,\ymax);
			
			\draw[->] (\xmin,0)--(\xmax,0) node [below]{$x$};
			\draw[->] (0,\ymin)--(0,\ymax) node [left]{$y$};
			\node at (20,14) [below right]{$y=g'(x)$};
			\node at (18,-2) [below left]{$y=h'(x)$};
			\node at (16,5) [below right]{$y=f'(x)$};
			\node at (0,0) [below left]{$O$};
			\draw[dashed] (3,0) node[below]{$3$}--(3,10)--(0,10) node[left]{$10$};
			\draw[dashed] (8,0) node[below]{$8$}--(8,5)--(0,5) node[left]{$5$};
			\draw[dashed] (4,0) node[below]{$4$}--(4,2)--(0,2) node[left]{$2$};
			\clip (\xmin+0.1,\ymin+0.1) rectangle (\xmax-0.5,\ymax-0.1);
			\draw[smooth,samples=300,domain=-2:18] plot(\x,{\a*(\x)^2+\b*(\x)+\c});
			%\draw[smooth,samples=300,domain=-2:25] plot(\x,{0.02*(\x)^3-0.6*(\x)^2+5.16*(\x)});
			\draw[line width=1.2pt] (-2,5)..controls (1.7,1.5) and (4.5,1.6)..(7,2.6);
			\draw[line width=1.2pt] (7,2.6)..controls (9,3.5) and (12,5)..(20,13);
			\draw (-0.5,-2) -- (0,0)--(3,10).. controls +(65:1) and + (-190:1)..(6,15).. controls +(0:1) and + (-180:1)..(14,-1).. controls +(0:1) and + (+80:1)..(19,16);
			
		\end{tikzpicture}
	}
	\loigiai{
		Ta có $k'(x)=f'(x+7)+5 g'(5 x+1)-4 h'\left(4 x+\dfrac{3}{2}\right)$.\\
		Khi $x \in \left( \dfrac{3}{8};1\right)$ thì $\heva{&7{,}375<x+7<8\\&2{,}875<5x+1<6\\&3<4x+\dfrac{4}{3}<5{,}5}\Leftrightarrow \heva{&f'(x+7)>10\\&g'(5x+1)>2 \Rightarrow 5g'(5x+1)>10  \\&h'\left( 4x+\dfrac{3}{2}\right)<5 \Rightarrow -4h'\left( 4x+\dfrac{3}{2}\right) >-20}.$\\
		Do đó $k'(x)=f'(x+7)+5g'(5x+1)-4h'\left( 4x+\dfrac{3}{2}\right)>0$.\\
		Hàm số $k(x)=f(x+7)+g(5 x+1)-h\left(4 x+\dfrac{3}{2}\right)$ đồng biến trên $\left(\dfrac{3}{8}; 1\right)$.
	}
\end{ex}
\begin{ex}[THPT Thanh Chương 1 - Nghệ An- 2021] %[2D1G1-2]
	Cho hàm số $y=f(x)$ liên tục trên $\mathbb{R}$ có bảng xét dấu đạo hàm như sau
	\begin{center}
		\begin{tikzpicture}
			\tkzTabInit[nocadre,lgt=1.2,espcl=2,deltacl=0.6]
			{$x$ /0.6,$f'(x)$ /0.6}
			{$-\infty$,$1$,$2$,$3$,$4$,$+\infty$}
			\tkzTabLine{,-,$0$,+,$0$,+,$0$,-,$0$,+,}
		\end{tikzpicture}
	\end{center}
	Hàm số $y=3f(2x-1)-4x^3+15x^2-18x+1$ đồng biến trên khoảng nào dưới đây?
	\choice
	{$\left(3;+\infty\right)$}
	{\True $\left(1;\dfrac{3}{2}\right)$}
	{$\left(\dfrac{5}{2}; 3\right)$}
	{$\left(2;\dfrac{5}{2}\right)$}
	\loigiai{
		Ta có $y'=6f'(2x-1)-12x^2+30x-18=6\left[f'(2x-1)-2x^2+5x-3\right] $.\\
		Có $f'(2x-1)=0 \Leftrightarrow \hoac{&2x-1=1\\&2x-1=2\\&2x-1=3\\&2x-1=4} \Leftrightarrow \hoac{&x=1\\&x=\dfrac{3}{2}\\&x=2\\&x=\dfrac{5}{2}.}$
		Ta có bảng xét dấu sau
		\begin{center}
			\begin{tikzpicture}
				\tkzTabInit[nocadre,lgt=3.0,espcl=1.5,deltacl=0.6]
				{$x$ /1.0,$f(x)$ /0.6,$f'(2x-1)$ /0.6,$-2x^2+5x-3$/0.6,$g'(x)$/0.6}
				{$-\infty$,$1$,$\dfrac{3}{2}$,$2$,$\dfrac{5}{2}$,$3$,$4$,$+\infty$}
				\tkzTabLine{,-,$0$,+,|,+,$0$,+,|,+,$0$,-,$0$,+,}
				\tkzTabLine{,-,$0$,+,$0$,+,$0$,-,$0$,+,|,+,|,+,}
				\tkzTabLine{,-,$0$,+,$0$,-,|,-,|,-,|,-,|,-,}
				\tkzTabLine{,-,$0$,+,$0$,,?,,|,,?,?,,?,}
			\end{tikzpicture}
		\end{center}
		Dựa vào bảng xét dấu trên, ta kết luận hàm số đã cho đồng biến trên khoảng $\left( 1; \dfrac{3}{2}\right).$
	}
\end{ex}


\begin{ex}%[2D2G4-3] %Câu 27 
	[THPT Hoàng Hoa Thám-Đà Nẵng-2021]
	Cho hàm số $f(x)$ có bảng xét dấu của $f'(x)$ như sau:\\
	\begin{center}
		\begin{tikzpicture}
			\tkzTabInit[lgt=1.2,espcl=2.3]
			{$x$/0.7, $f'(x)$ /.8} % first column
			{$-\infty$,$-3$,$1$, $2$, $+\infty$} % first row
			\tkzTabLine { ,+,0,-,0,+,0,+ }
		\end{tikzpicture}
	\end{center}	
	Hàm số $y=f\left(2-e^x\right)-\dfrac{1}{3}{e^{3x}}+3e^{2x}-5e^x+1$ đồng biến trên khoảng nào dưới đây?
	\choice
	{$\left(0;\dfrac{3}{2}\right)$}
	{$\left(1;3\right)$}
	{\True $\left(-3;0\right)$}
	{$\left(-4;-3\right)$}
	\loigiai{
		Ta có $y'=-e^x.f'\left(2-e^x\right)-e^{3x}+6e^{2x}-5e^x=e^x\left[-f'\left(2-e^x\right)-e^{2x}+6e^x-5\right]$ .\\
		Đặt $t=2-e^x$, ta được\\
		$y'=\left(2-t\right)\left[-f'(t)-\left(2-t\right)^2+6\left(2-t\right)-5\right]=\left(2-t\right)\left[-f'(t)-t^2-2t+3\right]$ .\\
		$y'=0\Leftrightarrow\left(2-t\right)\left[-f'(t)-t^2-2t+3\right]=0\Leftrightarrow
		\hoac{
			& t=2\\ 
			& f'(t)=-t^2-2t+3.}$\\
		Hàm số $g(x)=-x^2-2x+3$ là parabol có trục đối xứng $x=-1$ và cắt trục hoành tại 2 điểm có hoành độ 
		$\hoac{
			& x=1\\ 
			& x=-3
		}$. Suy ra $f'(t)=-t^2-2t+3\Leftrightarrow \hoac{
			& t=1\\ 
			& t=-3. }$\\
		Bảng xét dấu\\
		\begin{center}
			\begin{tikzpicture}
				\tkzTabInit[lgt=3.9,espcl=2,nocadre]
				{$t$/0.7, $2-t$ /0.8, $-f'(t)-t^2-2t+3$ /0.8, $y'$ /0.8} % first column
				{$-\infty$,$-3$,$1$,$2$,$+\infty$} % first row
				\tkzTabLine { ,+,|,+,|,+,z,-, } % second row
				\tkzTabLine {,-,0,+,0,-,|,-,} % third row
				\tkzTabLine {,-,0,+,0,-,0,+,} % last row
			\end{tikzpicture}
		\end{center}
		Dựa vào bảng xét dấu $y'>0,\forall x\in\left(-3;0\right)$.}
\end{ex}


\begin{ex}%[2D1G1-2]%Câu 28 
	[Sở Lạng Sơn 2022] Cho hàm số $f(x)$ có bảng biến thiên như sau:\\
	\begin{center}
		\begin{tikzpicture}
			\tkzTabInit[espcl=2.5,lgt=1,nocadre]
			{$x$/0.7,$y'$/0.7,$y$/3.5}
			{$-\infty$,$1$,$2$,$3$,$4$,$+\infty$}
			\tkzTabLine{,+,0,-,0,+,0,-,0,+,}
			\node (0) at ($(N12)+(0,-3)$) {$-\infty$};
			\node (1) at ($(N22)+(0,-.5)$) {$3$};
			\node (2) at ($(N32)+(0,-1.7)$) {$1$};
			\node (3) at ($(N42)+(0,-0.7)$) {$2$};
			\node (4) at ($(N52)+(0,-2.3)$) {$0$};
			\node (5) at ($(N62)+(0,-.3)$) {$+\infty$};
			%				\node (8) at ($(N42)+(0,-.5)$) {};
			%				\coordinate (9) at ($(N42)!.6!(N53)+ (-0.5,0)$);
			%				\coordinate (6) at ($(T12)!.6!(T13)$);
			%				\coordinate (7) at ($(T22)!.6!(T23)$);
			\draw[-stealth] (0)--(1);
			\draw[-stealth] (1)--(2);
			\draw[-stealth] (2)--(3);
			\draw[-stealth] (1)--(2);
			\draw[-stealth] (3)--(4);
			\draw[-stealth] (4)--(5);
			%				\draw[->,red] (5)--(8);
			%				\draw[->,red] (8)--(9);
			%				\draw[blue,dashed](6)--(7)node[above left]{$y=0$};
		\end{tikzpicture}		
	\end{center}
	Hàm số $y=\left[f(x)\right]^3-3\left[f(x)\right]^2$ đồng biến trên khoảng nào dưới đây?
	\choice
	{$\left(-\infty\,;1\right)$}
	{$\left(1\,;2\right)$}
	{\True $\left(3\,;4\right)$}
	{$\left(2\,;3\right)$}
	\loigiai{
		Ta có $y'=3f'(x)\left[f^2(x)-2f(x)\right]$. 
		Phương trình $y'=0\Leftrightarrow \hoac{
			&{f}'(x)=0\\ 
			& f(x)=0\\ 
			& f(x)=2.
		}$
		\begin{center}
			\begin{tikzpicture}
				\tkzTabInit[espcl=2.5,lgt=1.5]
				{$x$/0.7,$y'$/0.7,$y$/3.5}
				{$-\infty$,$1$,$2$,$3$,$4$,$+\infty$}
				\tkzTabLine{,+,0,-,0,+,0,-,0,+,}
				\node (0) at ($(N12)+(0,-3)$) {$-\infty$};
				\node (1) at ($(N22)+(0,-.3)$) {$3$};
				\node (2) at ($(N32)+(0,-1.7)$) {$1$};
				\node (3) at ($(N42)+(0,-0.8)$) {$2$};
				\node (4) at ($(N52)+(0,-2.3)$) {$0$};
				\node (5) at ($(N62)+(0,-.3)$) {$+\infty$};
				\node (a) at ($(N11)+(0.65,0.35)$) {$a$};
				\node (b) at ($(N11)+(2.0,0.4)$) {$b$};
				\node (c) at ($(N11)+(3.38,0.35)$) {$c$};
				\node (d) at ($(N11)+(11.85,0.4)$) {$d$};
				\node (6) at ($(N12)+(0,-0.8)$) {};
				\node (7) at ($(N62)+(0,-0.8)$) {};
				\node (8) at ($(N12)+(0,-2.3)$) {};
				\node (9) at ($(N62)+(0,-2.3)$) {};
				%				\node (8) at ($(N42)+(0,-.5)$) {};
				%				\coordinate (9) at ($(N42)!.6!(N53)+ (-0.5,0)$);
				\coordinate (A) at ($(0)!.25!(1)$);
				\coordinate (B) at ($(0)!.8!(1)$);
				\coordinate (C) at ($(1)!.35!(2)$);
				\coordinate (D) at ($(4)!.75!(5)$);
				%				\coordinate (7) at ($(T22)!.6!(T23)$);
				\draw[->] (0)--(1);
				\draw[->] (1)--(2);
				\draw[->] (2)--(3);
				\draw[->] (1)--(2);
				\draw[->] (3)--(4);
				\draw[->] (4)--(5);
				%				\draw[->,red] (5)--(8);
				%				\draw[->,red] (8)--(9);
				\draw[blue,dashed](6)--(7)node[below]{$y=2$} (a)--(A) (b)--(B) (c)--(C) (d)--(D);
				\draw[blue,dashed](8)--(9)node[below left]{$y=0$};
			\end{tikzpicture}		
		\end{center}
		Dựa vào bảng biến thiên, ta thấy $f'(x)=0\Leftrightarrow x\in \{ 1\,;2\,;3\,;4 \}$;\\
		$f(x)=0\Leftrightarrow x=a<1$ hoặc $x=4$;\\
		$f(x)=2\Leftrightarrow \hoac{
			& x=b\,\,\left(a<b<1\right)\\ 
			& x=c\in\left(1\,;2\right)\\ 
			& x=3\\ 
			& x=d>4.
		}$ \\
		Ta lập được bảng xét dấu của $y'$ 
		\begin{center}
			\begin{tikzpicture}
				\tkzTabInit[lgt=1.2,espcl=1.5,nocadre]
				{$x$/1, $f(x)$ /.8} % first column
				{$-\infty$,$a$, $b$, $1$,$c$, $2$,$3$, $4$, $d$, $+\infty$} % first row
				\tkzTabLine { ,+,z,-,z,+,z,-,z,+,z,-,z,+,z,-,z,+, } % second row
				%				\tkzTabLine {,-,z,+,t,+,} % third row
				%				\tkzTabLine {,+,d,-,z,+,} % last row
			\end{tikzpicture}
		\end{center}
		Từ bảng xét dấu, ta thấy hàm số đồng biến trên các khoảng \\
		$\left(-\infty;a\right)$, $\left(b;1\right)$, $\left(c;2\right)$, $\left(3;4\right)$ và $(d;+\infty)$.
	}
\end{ex}

\begin{ex}%[2D1G1-2]%Câu 29 
	[THPT Bùi Thị Xuân – Huế-2022] 
	\immini{
		Cho hàm số $y=f(x)$ là hàm đa thức bậc bốn. Đồ thị hàm số $f'(x+2)$ được cho trong hình vẽ bên. Hàm số 
		$$g(x)=4 f\left(x^2\right)-x^6+5 x^4-4 x^2+1$$
		đồng biến trên khoảng nào dưới đây?
		\choice
		{$(-4 ;-3)$}
		{\True $(2 ;+\infty)$}
		{$(-\sqrt{2};\sqrt{2})$}
		{$(-2 ;-1)$}}{
		\begin{tikzpicture}[scale=0.6,font=\footnotesize, line join=round, line cap=round, >=stealth] %Đường cong bậc 3
			\draw[thick, ->] (-5.3,0)--(5,0);
			\draw[thick, ->] (0,-3.5)--(0,7);
			\draw (5.2,0) node[below] {$x$};
			\draw (0,7.1) node[left]{$y$};
			\draw (0,0) node[below left]{$0$};
			\draw[fill] (-2,0) circle (0.5pt)node[below left]{$ -2 $};
			\draw[fill] (2,0) circle (0.5pt)node[below]{$ 2$};
			\draw[fill] (0,3) circle (0.5pt)node[left]{$ 3 $};
			\draw[fill] (0,1) circle (0.5pt)node[right]{$ 1 $};
			\draw[fill] (0,-1) circle (0.5pt)node[right]{$ -1 $};
			\draw[dashed] (-2,0)--(-2,1) --(0,1); 
			\draw[dashed](2,0)--(2,3)--(0,3);
			\draw[line width=1.2pt,smooth,samples=100,domain=-2.8:4.5] plot(\x,{-0.271*(\x)^3+0.75*(\x)^2+1.583*\x-1});
		\end{tikzpicture}		
	}
	\loigiai{
		$\begin{aligned}
			& g(x)=4f\left(x^2\right)-x^6+5x^4-4x^2+1\Rightarrow g' (x)=8xf'\left(x^2\right)-6x^5+20x^3-8x.\\ 
			& g' (x)=0\Leftrightarrow 8xf'\left(x^2\right)-6x^5+20x^3-8x=0 \\
			& \Leftrightarrow 2x\left[4f'\left(x^2\right)-3x^4+10x^2-4\right]=0\\ 
			&\Leftrightarrow 		\hoac{ 			& 2x=0\\ 
				& 4f'(x^2)-3x^4+10x^2-4=0
			}
			\Leftrightarrow \hoac{	& x=0\\ 
				& f'\left(x^2\right)=\dfrac{3}{4}{x^4}-\dfrac{5}{2}{x^2}+1.}
		\end{aligned}$\\ 
		Xét
		$f'\left(x^2\right)=\dfrac{3}{4}x^4-\dfrac{5}{2}x^2+1$. Đặt $x^2=t+2$, ta có\\
		$ f' (t+2)=\dfrac{3}{4}{(t+2)^2}-\dfrac{5}{2}(t+2)+1=\dfrac{3}{4}\left(t^2+4t+4\right)-\dfrac{5}{2}(t+2)-1=\dfrac{3}{4}{t^2}+\dfrac{1}{2}t-1$\\
		Khi đó số nghiệm của phương trình chính là số giao điểm của đồ thị hàm số $y=f' (t+2)$ và\\
		$ y=\dfrac{3}{4}{t^2}+\dfrac{1}{2}t-1$\\
		Ta có đồ thị 
		\begin{center}
			\begin{tikzpicture}[scale=0.6,font=\footnotesize, line join=round, line cap=round, >=stealth] %Đường cong bậc 3
				\draw[thick, ->] (-5.3,0)--(5,0);
				\draw[thick, ->] (0,-3.5)--(0,7);
				\draw (5.2,0) node[below] {$x$};
				\draw (0,7.1) node[left]{$y$};
				\draw (0,0) node[below left]{$0$};
				\draw[fill] (-2,0) circle (0.5pt)node[below left]{$ -2 $};
				\draw[fill] (2,0) circle (0.5pt)node[below]{$ 2$};
				\draw[fill] (0,3) circle (0.5pt)node[left]{$ 3 $};
				\draw[fill] (0,1) circle (0.5pt)node[right]{$ 1 $};
				\draw[fill] (0,-1) circle (0.5pt)node[right]{$ -1 $};
				\draw[dashed] (-2,0)--(-2,1) --(0,1); 
				\draw[dashed](2,0)--(2,3)--(0,3);
				\draw[line width=1.2pt,smooth,samples=100,domain=-2.8:4.5] plot(\x,{-0.271*(\x)^3+0.75*(\x)^2+1.583*\x-1});		
				\draw[line width=1.2pt,smooth,samples=100,domain=-3.3:2.8] plot(\x,{0.75*(\x)^2+0.5*\x-1});
			\end{tikzpicture}
		\end{center}
		Dựa vào đồ thị ta có $f' (t+2)=\dfrac{3}{4}t^2+\dfrac{1}{2}t-1\Leftrightarrow \hoac{& t=-2\\ & t=0\\ & t=2} \Leftrightarrow\hoac{& x+2=-2\\ & x+2=0\\ & x+2=2} \Leftrightarrow \hoac{& x=-4\\ & x=-2\\ & x=0.}$\\
		Ta có bảng xét dấu $g' (x)$ như sau
		\begin{center}
			\begin{tikzpicture}
				\tkzTabInit[lgt=1.2,espcl=2,nocadre]
				{$x$/0.7, $f(x)$ /.7}
				{$-\infty$, $-4$,$-2$, $0$, $+\infty$} % first row
				\tkzTabLine { ,-,z,+,z,-,z,+, }
			\end{tikzpicture}
		\end{center}
		Vậy hàm số $g(x)=4 f\left(x^2\right)-x^6+5 x^4-4 x^2+1$ đồng biến trên khoảng $(2 ;+\infty)$.}
\end{ex}

\begin{ex}%[2D1G1-2]%Câu 30
	[Chuyên Bắc Ninh 2022] 
	\immini{
		Cho hàm số $ y=f(x)$ liên tục trên $\mathbb{R}$ có đồ thị hàm số $ y=f'(x)$ có đồ thị như hình vẽ bên.
		Hàm số $g(x)=2f\left(\left| x-1\right|\right)-x^2+2x+2020$ đồng biến trên khoảng nào
		\choice
		{$\left(-2;0\right)$}
		{$\left(-3;1\right)$}
		{$\left(1\,;3\right)$}
		{\True $\left(0\,;\,1\right)$}}{
		\begin{tikzpicture}[scale=0.6,font=\footnotesize, line join=round, line cap=round, >=stealth] %Đường cong bậc 3
			\draw[thick, ->] (-3.3,0)--(5,0);
			\draw[thick, ->] (0,-3.0)--(0,5.5);
			\draw (5.2,0) node[below] {$x$};
			\draw (0,5.8) node[left]{$y$};
			\draw (0,0) node[below left]{$0$};
			\draw[fill] (-1,0) circle (0.5pt)node[above]{$ -1 $};
			\draw[fill] (1,0) circle (0.5pt)node[below]{$ 1$};
			\draw[fill] (0,1) circle (0.5pt)node[left]{$ 1 $};
			\draw[fill] (0,-1) circle (0.5pt)node[right]{$ -1 $};
			\draw[fill] (0,3) circle (0.5pt)node[left]{$ 3 $};
			\draw[fill] (3,0) circle (0.5pt)node[below]{$ 3 $};
			\draw[dashed] (-1,0)--(-1,-1) --(0,-1); 
			\draw[dashed](1,0)--(1,1)--(0,1);
			\draw[dashed](3,0)--(3,3)--(0,3);
			\draw[line width=1.2pt,smooth,samples=100,domain=-2.2:4.3] plot(\x,{-0.333*(\x)^3+1*(\x)^2+1.333*\x-1});		
			%\draw[line width=1.2pt,smooth,samples=100,domain=-3.3:2.8] plot(\x,{0.75*(\x)^2+0.5*\x-1});
		\end{tikzpicture}	
	}
	\loigiai{
		Ta có $g(x)=2f\left(\left| x-1\right|\right)-x^2+2x+2020\Leftrightarrow g(x)=2f\left(\left| x-1\right|\right)-\left(x-1\right)^2+2021$.\\
		Xét hàm số $ k\left(x-1\right)=2f\left(x-1\right)-\left(x-1\right)^2+2021$.\\
		Đặt $ t=x-1$\\
		Xét hàm số $ h(t)=2f(t)-t^2+2021$ $\Rightarrow{h}'(t)=2f'(t)-2t$.\\
		Kẻ đường $ y=x$ như hình vẽ.
		\begin{center}
			\begin{tikzpicture}[scale=0.6,font=\footnotesize, line join=round, line cap=round, >=stealth] %Đường cong bậc 3
				\draw[thick, ->] (-3.3,0)--(5,0);
				\draw[thick, ->] (0,-3.0)--(0,5.5);
				\draw (5.2,0) node[below] {$x$};
				\draw (0,5.8) node[left]{$y$};
				%	\draw (0,0) node[below left]{$0$};
				\draw[fill] (-1,0) circle (0.5pt)node[above]{$ -1 $};
				\draw[fill] (1,0) circle (0.5pt)node[below]{$ 1$};
				\draw[fill] (0,1) circle (0.5pt)node[left]{$ 1 $};
				\draw[fill] (0,-1) circle (0.5pt)node[right]{$ -1 $};
				\draw[fill] (0,3) circle (0.5pt)node[left]{$ 3 $};
				\draw[fill] (3,0) circle (0.5pt)node[below]{$ 3 $};
				\draw[dashed] (-1,0)--(-1,-1) --(0,-1); 
				\draw[dashed](1,0)--(1,1)--(0,1);
				\draw[dashed](3,0)--(3,3)--(0,3);
				\draw[line width=1.2pt,smooth,samples=100,domain=-2.2:4.3] plot(\x,{-0.333*(\x)^3+1*(\x)^2+1.333*\x-1});		
				%\draw[line width=1.2pt,smooth,samples=100,domain=-3.3:2.8] plot(\x,{0.75*(\x)^2+0.5*\x-1});
				\draw[line width=1.2pt,smooth,samples=100](-2,-2)--(4,4);
			\end{tikzpicture}
		\end{center}
		Khi đó $h'(t)>0\Leftrightarrow{f}'(t)-t>0\Leftrightarrow{f}'(t)>t$$\Leftrightarrow \hoac{
			& t<-1\\ 
			& 1<t<3.
		}$\\
		Do đó $k'\left(x-1\right)>0\Leftrightarrow \hoac{
			& x-1<-1\\ 
			& 1<x-1<3} \Leftrightarrow \hoac{
			& x<0\\ 
			& 2<x<4.}$\\
		Ta có bảng biến thiên của hàm số $ k\left(x-1\right)=2f\left(x-1\right)-\left(x-1\right)^2+2021$.
		\begin{center}
			\begin{tikzpicture}
				\tkzTabInit[lgt=1.8,espcl=2.3]
				{$x$ /1.2, $k'(x-1)$ /1.2,$k(x-1)$ /2}
				{$-\infty$ , $0$,$2$,$4$, $+\infty$}
				\tkzTabLine{,+,0,-,0,+,0,-,}
				\tkzTabVar{-/$ $ ,+/$ $, -/$ $,+/$ $,-/$ $}
			\end{tikzpicture}
		\end{center}
		Khi đó, ta có bảng biến thiên của $g(x)=2f\left(\left| x-1\right|\right)-\left(x-1\right)^2+2021$ bằng cách lấy đối xứng qua đường thẳng $ x=1$ như sau\\
		\begin{center}
			\begin{tikzpicture}
				\tkzTabInit[lgt=1.2,espcl=2.5,nocadre]
				{$x$ /0.7, $g'(x)$ /0.7,$g(x)$ /2.5}
				{$-\infty$ ,$-2$, $0$,$1$,$2$,$4$, $+\infty$}
				\tkzTabLine{,+,0,-,0,+,0,-,0,+,0,-,}
				\tkzTabVar{-/$ $ ,+/$ $, -/$ $,+/$ $,-/$ $,+/ $ $,-/$ $}
			\end{tikzpicture}
		\end{center}
		Vậy hàm số đồng biến trên $\left(0;1\right)$.}
\end{ex}

\begin{ex}%[2D1G1-2]%Câu 31
	[Chuyên Thái Bình 2022] 
	\immini{
		Cho hàm số $f(x)=a{x^4}+b{x^3}+c{x^2}+dx+a$ có đồ thị hàm số $y=f'(x)$ như hình vẽ bên. Hàm số $y=g(x)=f\left(1-2x\right)f\left(2-x\right)$ đồng biến trên khoảng nào dưới đây?
		\choice
		{$\left(\dfrac{1}{2};\dfrac{3}{2}\right)$}
		{$\left(-\infty ;0\right)$}
		{$\left(0;2\right)$}
		{\True $\left(3;+\infty\right)$}}{
		\begin{tikzpicture}[scale=0.9,font=\footnotesize, line join=round, line cap=round, >=stealth] %Đường cong bậc 3
			\draw[thick, ->] (-2.5,0)--(2.5,0);
			\draw[thick, ->] (0,-2.8)--(0,2.8);
			\draw (2.6,0) node[below] {$x$};
			\draw (0,2.9) node[left]{$y$};
			\draw (0,0) node[below left]{$0$};
			\draw[fill] (-1,0) circle (0.5pt)node[below left]{$ -1 $};
			\draw[fill] (1,0) circle (0.5pt)node[below right]{$ 1$};
			%			\draw[dashed] (-1,0)--(-1,-1) --(0,-1); 
			%			\draw[dashed](1,0)--(1,1)--(0,1);
			%			\draw[dashed](3,0)--(3,3)--(0,3);
			\draw[line width=1.2pt,smooth,samples=100,domain=-1.3:1.3] plot(\x,{3*(\x)^3-3*\x});		
			%\draw[line width=1.2pt,smooth,samples=100,domain=-3.3:2.8] plot(\x,{0.75*(\x)^2+0.5*\x-1});
		\end{tikzpicture}	
	}
	\loigiai{
		Ta có $f'(x)=4a{x^3}+3b{x^2}+2cx+d$, theo đồ thị thì đa thức $f'(x)$ có ba nghiệm phân biệt là $-1,0,1$ nên $f'(x)=4ax\left(x+1\right)\left(x-1\right)=4a{x^3}-4ax\Rightarrow f(x)=a{x^4}-2a{x^2}+a=a{\left(x^2-1\right)^2}$.\\
		Dựa vào đồ thị hàm số $y=f'(x)$ ta có $a>0$ nên $f(x)>0,\forall x\in\mathbb{R}\setminus\left\{\pm 1\right\}$.\\
		$g'(x)=\left[f\left(1-2x\right)\right]'f\left(2-x\right)+f\left(1-2x\right)\left[f\left(2-x\right)\right]'=-2f'\left(1-2x\right)f\left(2-x\right)-f\left(1-2x\right)f'\left(2-x\right)$. Xét $x\in\left(\dfrac{1}{2};\dfrac{3}{2}\right)\Rightarrow
		\heva{		
			& 1-2x\in\left(-2;0\right)\\ 
			& 2-x\in\left(\dfrac{1}{2};\dfrac{3}{2}\right)}$, dấu của $f'(x)$ không cố định trên $\left(\dfrac{1}{2};\dfrac{3}{2}\right)$ nên ta không kết luận được tính đơn điệu của hàm số $g(x)$ trên $\left(\dfrac{1}{2};\dfrac{3}{2}\right)$.\\
		Xét $x\in\left(-\infty ;0\right)\Rightarrow
		\heva{
			& 1-2x\in\left(1;+\infty\right)\\ 
			& 2-x\in\left(2;+\infty\right)} 
		\Rightarrow \heva{
			& f'\left(1-2x\right)>0\\ 
			& f'\left(2-x\right)>0} \Rightarrow g'(x)<0$.\\
		Do đó, hàm số $g(x)$ nghịch biến trên $\left(-\infty ;0\right)$.\\
		$x\in\left(0;2\right)\Rightarrow \heva{
			& 1-2x\in\left(-3;1\right)\\ 
			& 2-x\in\left(0;2\right)}$, dấu của $f'(x)$ không cố định trên $\left(-3;1\right)$ và $\left(0;2\right)$ nên ta không kết luận được tính đơn điệu của hàm số $g(x)$ trên $\left(\dfrac{1}{2};\dfrac{3}{2}\right)$.\\
		Xét $x\in\left(3;+\infty\right)\Rightarrow \heva{
			& 1-2x\in\left(-\infty ;-5\right)\\ 
			& 2-x\in\left(-\infty ;-1\right)} \Rightarrow \heva{
			& f'\left(1-2x\right)<0\\ 
			& f'\left(2-x\right)<0} \Rightarrow g'(x)>0$. \\
		Do đó, hàm số $g(x)$ đồng biến trên $\left(3;+\infty\right)$.}
\end{ex}

\begin{dang}{Bài toán hàm ẩn, hàm hợp liên quan đến tham số và một số bài toán khác}
\end{dang}

\begin{ex}%[2D1G1-3]%Câu 1
	[Chuyên Lê Hồng Phong Nam Định 2019]
	\immini{
		Cho hàm số $ y=f(x)$ có đạo hàm liên tục trên $\mathbb{R}$. Biết hàm số $ y=f'(x)$ có đồ thị như hình vẽ. Gọi $ S$ là tập hợp các giá trị nguyên $ m\in\left[-5\,;\,\text{5}\right]$ để hàm số $ g(x)=f\left(x+m\right)$ nghịch biến trên khoảng $\left(1\,;\,2\right)$. Hỏi $S$ có bao nhiêu phần tử?
		\choice
		{$ 4$}
		{$ 3$}
		{$ 6$}
		{\True $ 5$}}{
		\begin{tikzpicture}[scale=0.9,font=\footnotesize, line join=round, line cap=round, >=stealth] %Đường cong bậc 3
			\draw[thick, ->] (-2.5,0)--(4,0);
			\draw[thick, ->] (0,-2.8)--(0,2.8);
			\draw (4.3,0) node[below] {$x$};
			\draw (0,2.9) node[left]{$y$};
			\draw (0,0) node[below left]{$0$};
			\draw[fill] (-1,0) circle (0.5pt)node[below left]{$ -1 $};
			\draw[fill] (1,0) circle (0.5pt)node[below]{$ 1$};
			\draw[fill] (3,0) circle (0.5pt)node[below right]{$ 3$};
			%			\draw[dashed] (-1,0)--(-1,-1) --(0,-1); 
			%			\draw[dashed](1,0)--(1,1)--(0,1);
			%			\draw[dashed](3,0)--(3,3)--(0,3);
			\draw[line width=1.2pt,smooth,samples=100,domain=-1.65:3.5] plot(\x,{0.33*(\x)^3-(\x)^2-0.333*(\x)+1});		
			%\draw[line width=1.2pt,smooth,samples=100,domain=-3.3:2.8] plot(\x,{0.75*(\x)^2+0.5*\x-1});
		\end{tikzpicture}	
	}
	\loigiai{
		Ta có $g'(x)=f'\left(x+m\right)$. Vì $ y=f'(x)$ liên tục trên $\mathbb{R}$ nên $g'(x)=f'\left(x+m\right)$ cũng liên tục trên $\mathbb{R}$. Căn cứ vào đồ thị hàm số $ y=f'(x)$ ta thấy\\
		$g'(x)<0\Leftrightarrow{f}'\left(x+m\right)<0$ $\Leftrightarrow\hoac{
			& x+m<-1\\ 
			& 1<x+m<3} \Leftrightarrow \hoac{
			& x<-1-m\\ 
			& 1-m<x<3-m.}$\\
		Hàm số $ g(x)=f\left(x+m\right)$ nghịch biến trên khoảng $\left(1\,;\,2\right)$
		$\Leftrightarrow \hoac{
			& 2\le-1-m\\ 
			&\hoac{
				& 3-m\ge 2\\ 
				& 1-m\le 1}} \Leftrightarrow \hoac{
			& m\le-3\\ 
			& 0\le m\le 1.}$\\
		Mà $ m$ là số nguyên thuộc đoạn $\left[-5\,;\,5\right]$ nên ta có $ S=\left\{-5;-4;-3;0;1\right\}$.\\
		Vậy $ S$ có $5$ phần tử.}
\end{ex}

\begin{ex}%[2D1G1-3]%Câu 2
	[Chuyên Nguyễn Bỉnh Khiêm-Quảng Nam-2020] Cho hàm số $ y=f(x)$ có đạo hàm trên $\mathbb{R}$ và bảng xét dấu đạo hàm như hình vẽ sau
	\begin{center}
		\begin{tikzpicture}
			\tkzTabInit[lgt=1.2,espcl=2.5,nocadre]
			{$x$/0.7, $f'(x)$ /2.5} % first column
			{$-\infty$, $-10$,$-2$, $3$,$8$, $+\infty$} % first row
			\tkzTabLine { ,+,z,-,z,+,z,-,z,+, } % second row
			%				\tkzTabLine {,-,z,+,t,+,} % third row
			%				\tkzTabLine {,+,d,-,z,+,} % last row
		\end{tikzpicture}
	\end{center}
	Có bao nhiêu số nguyên $ m$ để hàm số $ y=f\left(x^3+4x+m\right)$ nghịch biến trên khoảng $\left(-1;1\right)$?
	\choice
	{$ 3$}
	{$ 0$}
	{\True $ 1$}
	{$ 2$}
	\loigiai
	{
		Đặt $ t=x^3+4x+m\Rightarrow{t}'=3x^2+4$ nên $ t$ đồng biến trên $\left(-1;1\right)$ và $ t\in\left(m-5;m+5\right)$.\\
		Yêu cầu bài toán trở thành tìm $ m$ để hàm số $ f(t)$ nghịch biến trên khoảng $\left(m-5;m+5\right)$.\\
		Dựa vào bảng biến thiên ta được $\heva{
			& m-5\ge-2\\ 
			& m+5\le 8} \Leftrightarrow \heva{
			& m\ge 3\\ 
			& m\le 3} \Leftrightarrow m=3$.}
\end{ex}

\begin{ex}%[2D1G1-3]%Câu 3
	[Chuyên ĐH Vinh-Nghệ An-2020]
	\immini{
		Cho hàm số $ f(x)$ có đạo hàm trên $\mathbb{R}$và $ f(1)=1$. Đồ thị hàm số $ y=f'(x)$ như hình bên. Có bao nhiêu số nguyên dương $ a$ để hàm số $ y=\left| 4f\left(\sin x\right)+\cos 2x-a\right|$ nghịch biến trên $\left(0;\dfrac{\pi}{2}\right)$?
		\choice
		{$ 2$}
		{\True $ 3$}
		{Vô số}
		{$ 5$}}{
		\begin{tikzpicture}[scale=0.9,font=\footnotesize, line join=round, line cap=round, >=stealth] %Đường cong bậc 3
			\draw[thick, ->] (-2.5,0)--(3,0);
			\draw[thick, ->] (0,-2.8)--(0,2.8);
			\draw (3.1,0) node[below] {$x$};
			\draw (0,2.9) node[left]{$y$};
			\draw (0,0) node[below left]{$0$};
			\draw[fill] (-1,0) circle (0.5pt)node[below]{$ -1 $};
			\draw[fill] (1,0) circle (0.5pt)node[above]{$ 1$};
			%	\draw[fill] (3,0) circle (0.5pt)node[below right]{$ 3$};
			\draw[dashed] (-1,0)--(-1,1); 
			\draw[dashed](1,0)--(1,-1);
			%			\draw[dashed](3,0)--(3,3)--(0,3);
			\draw[line width=1.2pt,smooth,samples=100,domain=-2:2] plot(\x,{.8*(\x)^3+0*(\x)^2-1.8*(\x)});		
			%\draw[line width=1.2pt,smooth,samples=100,domain=-3.3:2.8] plot(\x,{0.75*(\x)^2+0.5*\x-1});
			\draw (2.0,2.8) node[left]{$y=f'(x)$};
		\end{tikzpicture}	
	}
	\loigiai
	{		Đặt $g(x)=\left| 4f\left(\sin x\right)+\cos 2x-a\right|\Rightarrow g(x)=\sqrt{\left[4f\left(\sin x\right)+\cos 2x-a\right]^2}$ .\\
		$\Rightarrow{g}'(x)=\dfrac{\left[4\cos x\cdot f'\left(\sin x\right)-2\sin 2x\right]\left[4f\left(\sin x\right)+\cos 2x-a\right]}{\sqrt{\left[4f\left(\sin x\right)+\cos 2x-a\right]^2}}$.\\
		Ta có $ 4\cos x\cdot f'\left(\sin x\right)-2\sin 2x=4\cos x\left[f'\left(\sin x\right)-\sin x\right]$.\\
		Với $ x\in\left(0;\dfrac{\pi}{2}\right)$ thì $\cos x>0,\sin x\in\left(0;1\right)\Rightarrow{f}'\left(\sin x\right)-\sin x<0$.\\
		Hàm số $ g(x)$ nghịch biến trên $\left(0;\dfrac{\pi}{2}\right)$ khi $ 4f\left(\sin x\right)+\cos 2x-a\ge 0,\forall x\in\left(0;\dfrac{\pi}{2}\right)$\\
		$\Leftrightarrow 4f\left(\sin x\right)+1-2\sin^2x\ge a,\forall x\in\left(0;\dfrac{\pi}{2}\right)$.\\
		Đặt $ t=\sin x$ được $ 4f(t)+1-2t^2\ge a,\forall t\in\left(0;1\right)$ (*).\\
		Xét $ h(t)=4f(t)+1-2t^2\Rightarrow{h}'(t)=4f'(t)-4t=4\left[f'(t)-1\right]$.\\
		Với $ t\in\left(0;1\right)$ thì $h'(t)<0\Rightarrow h(t)$ nghịch biến trên $\left(0;1\right)$.\\
		Do đó (*) $\Leftrightarrow a\le h(1)=4f(1)+1-2.1^2=3$.\\
		Vậy có $3$ giá trị nguyên dương của a thỏa mãn.}
\end{ex}


\begin{ex}%[2D1G1-3]%Câu 4
	[Chuyên Quang Trung-2020]
	\immini{
		Cho hàm số $ y=f(x)$ có đạo hàm liên tục trên $\mathbb{R}$ và có đồ thị $ y=f'(x)$ như hình vẽ. Đặt $ g(x)=f\left(x-m\right)-\dfrac{1}{2}{\left(x-m-1\right)^2}+2019$, với $ m$ là tham số thực. Gọi $ S$ là tập hợp các giá trị nguyên dương của $ m$ để hàm số $ y=g(x)$ đồng biến trên khoảng $\left(5;6\right)$. Tổng tất cả các phần tử trong $ S$ bằng
		\choice
		{$ 4$}
		{$ 11$}
		{\True $ 14$}
		{$ 20$}}{
		\begin{tikzpicture}[scale=0.9,font=\footnotesize, line join=round, line cap=round, >=stealth] %Đường cong bậc 3
			\draw[style=help lines,step=1] (-2.5,-3) grid (3,3.5);
			\draw[thick, ->] (-2.5,0)--(3.5,0);
			\draw[thick, ->] (0,-2.8)--(0,2.8);
			\draw (3.6,0) node[below] {$x$};
			\draw (0,3) node[above left]{$y$};
			\draw (0,0) node[below left]{$0$};
			%\draw[fill] (-1,0) circle (0.5pt)node[below]{$ -1 $};
			\draw[fill] (1,0) circle (0.5pt)node[below left]{$ 1$};
			%	\draw[fill] (3,0) circle (0.5pt)node[below right]{$ 3$};
			\draw[dashed] (-1,0)--(-1,-2) --(2,-2)--(2,0); 
			\draw[dashed](3,0)--(3,2) --(0,2);
			\draw (-1,-2) circle (2pt);
			\draw (3,2) circle (2pt);
			%			\draw[dashed](3,0)--(3,3)--(0,3);
			\draw[line width=1.2pt,smooth,samples=100,domain=-1.1:3.1] plot(\x,{1*(\x)^3-3*(\x)^2-0*(\x)+2});		
			%\draw[line width=1.2pt,smooth,samples=100,domain=-3.3:2.8] plot(\x,{0.75*(\x)^2+0.5*\x-1});
			%\draw (2.0,2.8) node[left]{$y=f'(x)$};
		\end{tikzpicture}	
	}
	\loigiai
	{
		Xét hàm số $ g(x)=f\left(x-m\right)-\dfrac{1}{2}{\left(x-m-1\right)^2}+2019$.\\
		$g'(x)=f'\left(x-m\right)-\left(x-m-1\right)$.\\
		Xét phương trình $g'(x)=0. \quad \quad (1)$\\
		Đặt $ x-m=t$, phương trình $(1)$ trở thành $f'(t)-\left(t-1\right)=0\Leftrightarrow{f}'(t)=t-1. \quad (2)$\\
		Nghiệm của phương trình $(2)$ là hoành độ giao điểm của hai đồ thị hàm số $ y=f'(t)$ và $ y=t-1$.\\
		Ta có đồ thị các hàm số $ y=f'(t)$ và $ y=t-1$ như sau
		\begin{center}
			\begin{tikzpicture}[scale=0.9,font=\footnotesize, line join=round, line cap=round, >=stealth] %Đường cong bậc 3
				\draw[style=help lines,step=1] (-2.5,-3) grid (3,3.5);
				\draw[thick, ->] (-2.5,0)--(3.5,0);
				\draw[thick, ->] (0,-2.8)--(0,2.8);
				\draw (3.6,0) node[below] {$x$};
				\draw (0,3) node[above left]{$y$};
				\draw (0,0) node[below left]{$0$};
				%\draw[fill] (-1,0) circle (0.5pt)node[below]{$ -1 $};
				\draw[fill] (1,0) circle (0.5pt)node[below left]{$ 1$};
				%	\draw[fill] (3,0) circle (0.5pt)node[below right]{$ 3$};
				\draw[dashed] (-1,0)--(-1,-2) --(2,-2)--(2,0); 
				\draw[dashed](3,0)--(3,2) --(0,2);
				\draw (-1,-2) circle (2pt);
				\draw (3,2) circle (2pt);
				%			\draw[dashed](3,0)--(3,3)--(0,3);
				\draw[line width=1.2pt,smooth,samples=100,domain=-1.1:3.1] plot(\x,{1*(\x)^3-3*(\x)^2-0*(\x)+2});		
				%\draw[line width=1.2pt,smooth,samples=100,domain=-3.3:2.8] plot(\x,{0.75*(\x)^2+0.5*\x-1});
				%\draw (2.0,2.8) node[left]{$y=f'(x)$};
				\draw (-2,-3)--(4,3);
			\end{tikzpicture}
		\end{center}
		Căn cứ đồ thị các hàm số ta có phương trình $(2)$ có nghiệm là $\hoac{
			& t=-1\\ 
			& t=1\\ 
			& t=3} \Rightarrow \hoac{
			& x=m-1\\ 
			& x=m+1\\ 
			& x=m+3.}$\\
		Ta có bảng biến thiên của $ y=g(x)$
		\begin{center}
			\begin{tikzpicture}
				\tkzTabInit[lgt=1,espcl=2.5,nocadre]
				{$x$ /0.8, $y'$ /0.8,$y$ /2.5}
				{$-\infty$ , $m-1$,$m+1$,$m+3$, $+\infty$}
				\tkzTabLine{,+,0,-,0,+,0,-,}
				\tkzTabVar{-/$ +\infty$ ,+/$ $, -/$ $,+/$ $,-/$+\infty $}
			\end{tikzpicture}
		\end{center}
		Để hàm số $ y=g(x)$ đồng biến trên khoảng $\left(5;6\right)$ cần $\hoac{
			&\heva{
				& m-1\le 5\\ 
				& m+1\ge 6}\\ 
			& m+3\le 5}\Leftrightarrow\hoac{
			& 5\le m\le 6\\ 
			& m\le 2.}$\\
		Vì $ m\in\mathbb{N}^*\Rightarrow m$ nhận các giá trị $ 1;\,2;\,5;\,6\Rightarrow S=14$.}
\end{ex}

\begin{ex}%[2D1G1-3]%Câu 5
	[Sở Hà Nội-Lần 2-2020] 
	\immini{
		Cho hàm số $y=a{x^4}+b{x^3}+c{x^2}+dx+e,\,\,a\ne 0$. Hàm số $y=f'(x)$ có đồ thị như hình vẽ bên. 
		Gọi S là tập hợp tất cả các giá trị nguyên thuộc khoảng $\left(-6;6\right)$ của tham số $m$ để hàm số $g(x)=f\left(3-2x+m\right)+x^2-\left(m+3\right)x+2m^2$ nghịch biến trên $\left(0;1\right)$. Khi đó, tổng giá trị các phần tử của S là
		\choice
		{$12$}
		{\True $9$}
		{$6$}
		{$15$}}{
		\begin{tikzpicture}[scale=0.7,font=\footnotesize, line join=round, line cap=round, >=stealth] %Đường cong bậc 3
			%	\draw[style=help lines,step=1] (-2.5,-3) grid (3,3.5);
			\draw[thick, ->] (-4.5,0)--(6.5,0);
			\draw[thick, ->] (0,-2.8)--(0,2.8);
			\draw (6.6,0) node[below] {$x$};
			\draw (0,3) node[above left]{$y$};
			\draw (0,0) node[below left]{$0$};
			\draw[fill] (-2,0) circle (0.5pt)node[below]{$ -2 $};
			\draw[fill] (4,0) circle (0.5pt)node[above]{$ 4$};
			\draw[fill] (0,1) circle (0.5pt)node[right]{$ 1 $};
			\draw[fill] (0,-2) circle (0.5pt)node[left]{$ -2$};
			%	\draw[fill] (3,0) circle (0.5pt)node[below right]{$ 3$};
			\draw[dashed] (-2,0)--(-2,1) --(0,1); 
			\draw[dashed](4,0)--(4,-2) --(0,-2);
			%			\draw[dashed](3,0)--(3,3)--(0,3);
			\draw[line width=1.2pt,smooth,samples=100,domain=-3.8:5.5] plot(\x,{0.0714*(\x)^3-0.1423*(\x)^2-1.0714*(\x)});		
			%\draw[line width=1.2pt,smooth,samples=100,domain=-3.3:2.8] plot(\x,{0.75*(\x)^2+0.5*\x-1});
			%\draw (2.0,2.8) node[left]{$y=f'(x)$};
		\end{tikzpicture}	
	}
	\loigiai
	{
		Xét $g'(x)=-2f'\left(3-2x+m\right)+2x-\left(m+3\right)$.\\
		Xét phương trình $g'(x)=0$, đặt $t=3-2x+m$ thì phương trình trở thành\\ $-2\cdot \left[f'(t)-\dfrac{-t}{2}\right]=0\Leftrightarrow\hoac{
			& t=-2\\ 
			& t=4\\ 
			& t=0.}$ \\
		Từ đó, $g'(x)=0\Leftrightarrow{x_1}=\dfrac{5+m}{2},\,x_2=\dfrac{m+3}{2},x_3=\dfrac{-1+m}{2}$.\\
		Lập bảng xét dấu, đồng thời lưu ý nếu $x>x_1$ thì $t<t_1$ nên $f(x)>0$. Và các dấu đan xen nhau do các nghiệm đều làm đổi dấu đạo hàm nên suy ra $g'(x)\le 0\Leftrightarrow x\in\left[x_2;{x_1}\right]\cup\left(-\infty ;{x_3}\right]$.\\
		Vì hàm số nghịch biến trên $\left(0;1\right)$ nên \\
		$g'(x)\le 0,\,\forall x\in\left(0;1\right)$ từ đó suy ra $\hoac{
			&\dfrac{3+m}{2}\le 0<1\le\dfrac{5+m}{2}\\ 
			& 1\le\dfrac{-1+m}{2}.}$ \\
		và giải ra các giá trị nguyên thuộc $\left(-6;6\right)$ của $m$ là $-3$; $3$; $4$; $5$. }
\end{ex}

\begin{ex}%[2D1G1-3]%Câu 6
	[Chuyên Quang Trung-Bình Phước-Lần 2-2020]
	\immini{
		Cho hàm số $ y=f(x)$ có đạo hàm liên tục trên $\mathbb{R}$ và có đồ thị $ y=f'(x)$ như hình vẽ bên. Đặt $ g(x)=f\left(x-m\right)-\dfrac{1}{2}{\left(x-m-1\right)^2}+2019$, với $ m$ là tham số thực. Gọi $ S$ là tập hợp các giá trị nguyên dương của $ m$ để hàm số $ y=g(x)$ đồng biến trên khoảng $\left(5;6\right)$. Tổng tất cả các phần tử trong $ S$ bằng
		\choice
		{$ 4$}
		{$ 11$}
		{\True $ 14$}
		{$ 20$}}{
		\begin{tikzpicture}[scale=0.9,font=\footnotesize, line join=round, line cap=round, >=stealth] %Đường cong bậc 3
			\draw[thick, ->] (-2.5,0)--(3.7,0);
			\draw[thick, ->] (0,-2.8)--(0,2.8);
			\draw (3.9,0) node[below] {$x$};
			\draw (0,2.9) node[left]{$y$};
			\draw (0,0) node[below left]{$0$};
			\draw[fill] (-1,0) circle (0.5pt)node[above]{$ -1 $};
			\draw[fill] (1,0) circle (0.5pt)node[below]{$ 1$};
			\draw[fill] (3,0) circle (0.5pt)node[below]{$ 3$};
			\draw[fill] (2,0) circle (0.5pt)node[above]{$ 2$};
			\draw[fill] (0,2) circle (0.5pt)node[above left]{$ 2$};
			\draw[fill] (0,-2) circle (0.5pt)node[below left]{$ -2$};
			\draw[dashed] (-1,0)--(-1,-2)--(2,-2)--(2,0); 
			\draw[dashed](3,0)--(3,2)--(0,2);
			%			\draw[dashed](3,0)--(3,3)--(0,3);
			\draw[line width=1.2pt,smooth,samples=100,domain=-1.1:3.1] plot(\x,{1*(\x)^3-3*(\x)^2-0*(\x)+2});		
			%\draw[line width=1.2pt,smooth,samples=100,domain=-3.3:2.8] plot(\x,{0.75*(\x)^2+0.5*\x-1});
			%	\draw (2.0,2.8) node[left]{$y=f'(x)$};
	\end{tikzpicture}	}
	\loigiai
	{
		Ta có $g'(x)=f'\left(x-m\right)-\left(x-m-1\right)$.\\
		Cho $g'(x)=0\Leftrightarrow{f}'\left(x-m\right)=x-m-1$.\\
		Đặt $ x-m=t\Rightarrow f'(t)=t-1$\\
		Khi đó nghiệm của phương trình là hoành độ giao điểm của đồ thị hàm số $ y=f'(t)$ và và đường thẳng $ y=t-1$.
		\begin{center}
			\begin{tikzpicture}[scale=0.9,font=\footnotesize, line join=round, line cap=round, >=stealth] %Đường cong bậc 3
				\draw[thick, ->] (-2.5,0)--(3.7,0);
				\draw[thick, ->] (0,-2.8)--(0,2.8);
				\draw (3.9,0) node[below] {$x$};
				\draw (0,2.9) node[left]{$y$};
				\draw (0,0) node[below left]{$0$};
				\draw[fill] (-1,0) circle (0.5pt)node[above]{$ -1 $};
				\draw[fill] (1,0) circle (0.5pt)node[below]{$ 1$};
				\draw[fill] (3,0) circle (0.5pt)node[below]{$ 3$};
				\draw[fill] (2,0) circle (0.5pt)node[above]{$ 2$};
				\draw[fill] (0,2) circle (0.5pt)node[above left]{$ 2$};
				\draw[fill] (0,-2) circle (0.5pt)node[below left]{$ -2$};
				\draw[dashed] (-1,0)--(-1,-2)--(2,-2)--(2,0); 
				\draw[dashed](3,0)--(3,2)--(0,2);
				%			\draw[dashed](3,0)--(3,3)--(0,3);
				\draw[line width=1.2pt,smooth,samples=100,domain=-1.1:3.1] plot(\x,{1*(\x)^3-3*(\x)^2-0*(\x)+2});		
				%\draw[line width=1.2pt,smooth,samples=100,domain=-3.3:2.8] plot(\x,{0.75*(\x)^2+0.5*\x-1});
				%	\draw (2.0,2.8) node[left]{$y=f'(x)$};
				\coordinate (a) at ($(-1,-2)!1.2!(3,2)$);
				\coordinate (b) at ($(-1,-2)!-.2!(3,2)$);
				\draw[line width=1.2pt,smooth] (a)--(b);
			\end{tikzpicture}
		\end{center}
		Dựa vào đồ thị hàm số ta có được $f'(t)=t-1\Leftrightarrow\hoac{
			& t=-1\\ 
			& t=1\\ 
			& t=3.} $ \\
		Bảng xét dấu của $g'(t)$
		\begin{center}
			\begin{tikzpicture}
				\tkzTabInit[lgt=1.2,espcl=2.5,nocadre]
				{$t$/1, $g'(x)$ /.8} % first column
				{$-\infty$, $-1$,$1$, $3$, $+\infty$} % first row
				\tkzTabLine { ,-,0,+,0,-,0,+, } % second row
				%				\tkzTabLine {,-,z,+,t,+,} % third row
				%				\tkzTabLine {,+,d,-,z,+,} % last row
			\end{tikzpicture}
		\end{center}
		Từ bảng xét dấu ta thấy hàm số $ g(t)$ đồng biến trên khoảng $\left(-1;1\right)$ và $\left(3;+\infty\right)$.\\
		Hay $\hoac{
			&-1<t<1\\ 
			& t>3}\Leftrightarrow\hoac{
			&-1<x-m<1\\ 
			& x-m>3} \Leftrightarrow\hoac{
			& m-1<x<m+1\\ 
			& x>m+3.}$\\
		Để hàm số $ g(x)$ đồng biến trên khoảng $\left(5;6\right)$ thì $\hoac{
			& m-1\le 5<6\le m+1\\ 
			& m+3\le 5<6} \Leftrightarrow\hoac{
			& 5\le m\le 6\\ 
			& m\le 2.}$\\
		Vì $ m$ là các số nguyên dương nên $ S=\left\{ 1;2;5;6\right\}$.\\
		Vậy tổng tất cả các phần tử của $ S$ là $ 1+2+5+6=14$.}
\end{ex}

\begin{ex}%[2D1G1-3]%Câu 7
	\immini{
		Cho hàm số $ y=f(x)$ liên tục có đạo hàm trên $\mathbb{R}$. Biết hàm số $ f'(x)$ có đồ thị cho như hình vẽ bên. Có bao nhiêu giá trị nguyên của $ m$ thuộc $\left[-2019;2019\right]$ để hàm só $ g(x)=f\left(2019^x\right)-mx+2$ đồng biến trên $\left[0;1\right]$.
		\choice
		{$ 2028$}
		{$ 2019$}
		{$ 2011$}
		{\True $ 2020$}}{
		\begin{tikzpicture}[scale=0.9,font=\footnotesize, line join=round, line cap=round, >=stealth] %Đường cong bậc 3
			\draw[thick, ->] (-3.5,0)--(2.5,0);
			\draw[thick, ->] (0,-2.8)--(0,2.8);
			\draw (2.7,0) node[below] {$x$};
			\draw (0,2.9) node[left]{$y$};
			\draw (0,0) node[below left]{$0$};
			%	\draw[fill] (-1,0) circle (0.5pt)node[above]{$ -1 $};
			\draw[fill] (1,0) circle (0.5pt)node[below right]{$ 1$};
			%		\draw[fill] (3,0) circle (0.5pt)node[below]{$ 3$};
			%		\draw[fill] (2,0) circle (0.5pt)node[above]{$ 2$};
			%		\draw[fill] (0,2) circle (0.5pt)node[above left]{$ 2$};
			%		\draw[fill] (0,-2) circle (0.5pt)node[below left]{$ -2$};
			%		\draw[dashed] (-1,0)--(-1,-2)--(2,-2)--(2,0); 
			%		\draw[dashed](3,0)--(3,2)--(0,2);
			\draw[line width=1.2pt,smooth,samples=100,domain=-3.28:1.32] plot(\x,{0.667*(\x)^3+2*(\x)^2-0.667*(\x)-2});		
			%\draw[line width=1.2pt,smooth,samples=100,domain=-3.3:2.8] plot(\x,{0.75*(\x)^2+0.5*\x-1});
			%	\draw (2.0,2.8) node[left]{$y=f'(x)$};
	\end{tikzpicture}	}
	\loigiai{
		Ta có $ g'(x)=2019^x\ln 2019\cdot f'\left(2019^x\right)-m$.\\
		Ta lại có hàm số $ y=2019^x$ đồng biến trên $\left[0;1\right]$.\\
		Với $ x\in\left[0;1\right]$ thì $2019^x\in\left[1;2019\right]$ mà hàm $ y=f'(x)$ đồng biến trên $\left(1;+\infty\right)$ nên hàm $ y=f'\left(2019^x\right)$ đồng biến trên $\left[0;1\right]$.\\
		Mà $2019^x\ge 1;f'\left(2019^x\right)>0\,\forall\,x\in\left[0;1\right]$ nên hàm $ h(x)=2019^x\ln 2019\cdot f'\left(2019^x\right)$ đồng biến trên $\left[0;1\right]$.\\
		Hay $ h(x)\ge h(0)=0,\forall\,x\in\left[0;1\right]$.\\
		Do vậy hàm số $ g(x)$ đồng biến trên đoạn $\left[0;1\right]$$\Leftrightarrow g'(x)\ge 0,\forall\,x\in\left[0;1\right]$\\
		$\Leftrightarrow m\le{2019^x}\ln 2019.f'\left(2019^x\right),\forall\,x\in\left[0;1\right]$ $\Leftrightarrow m\le\underset{x\in\left[0;1\right]}{\min}\,h(x)=h(0)=0$\\
		Vì $ m$ nguyên và $ m\in\left[-2019;2019\right]\Rightarrow $có $ 2020$ giá trị $ m$ thỏa mãn yêu cầu bài toán.}
\end{ex}

\begin{ex}%[2D1G1-3]%Câu 8
	\immini{
		Cho hàm số $y=f(x)$ có đồ thị $f'(x)\,$ như hình vẽ. Có bao nhiêu giá trị nguyên $m\in\left(-2020\,;\,2020\right)$ để hàm số $g(x)=f\left(2x-3\right)\,-\ln \left(1+x^2\right)-2mx$ đồng biến trên $\left(\dfrac{1}{2};2\right)$?
		\choice
		{$ 2020$}
		{\True $ 2019$}
		{$ 2021$}
		{$ 2018$}}{
		\begin{tikzpicture}[scale=0.9,font=\footnotesize, line join=round, line cap=round, >=stealth] %Đường cong bậc 3
			\draw[thick, ->] (-2.5,0)--(2.5,0);
			\draw[thick, ->] (0,-1.8)--(0,5.8);
			\draw (2.7,0) node[below] {$x$};
			\draw (0,5.9) node[left]{$y$};
			\draw (0,0) node[below left]{$0$};
			\draw[fill] (-2,0) circle (0.5pt)node[below]{$ -2 $};
			\draw[fill] (1,0) circle (0.5pt)node[below]{$ 1$};
			\draw[fill] (-1,0) circle (0.5pt)node[below]{$-1$};
			\draw[fill] (0,4) circle (0.5pt)node[above left]{$ 2$};
			%		\draw[fill] (0,2) circle (0.5pt)node[above left]{$ 2$};
			%		\draw[fill] (0,-2) circle (0.5pt)node[below left]{$ -2$};
			\draw[dashed] (-2,0)--(-2,4)--(1,4)--(1,0); 
			%		\draw[dashed](3,0)--(3,2)--(0,2);
			\draw[line width=1.2pt,smooth,samples=100,domain=-2.1:2.1] plot(\x,{-1*(\x)^3+0*(\x)^2+3*(\x)+2});		
			%\draw[line width=1.2pt,smooth,samples=100,domain=-3.3:2.8] plot(\x,{0.75*(\x)^2+0.5*\x-1});
			%	\draw (2.0,2.8) node[left]{$y=f'(x)$};
	\end{tikzpicture}	}
	\loigiai{
		Ta có $g'(x)=2f'\left(2x-3\right)-\dfrac{2x}{1+x^2}-2m$.\\
		Hàm số $ g(x)$ đồng biến trên $\left(\dfrac{1}{2};2\right)$ khi và chỉ khi \\
		$g'(x)\ge 0,\,\,\forall x\in\left(-1;\,2\right)$\\
		$\Leftrightarrow m\le{f}'\left(2x-3\right)-\dfrac{x}{1+x^2},\,\,\forall x\in\left(\dfrac{1}{2};2\right)$\\
		$\Leftrightarrow m\le\underset{x\in\left[\dfrac{1}{2};2\right]}{\min}\,\left[f'\left(2x-3\right)-\dfrac{x}{1+x^2}\right]$. \, \,  $(1)$\\
		Đặt $ t=2x-3$, khi đó $ x\in\left(\dfrac{1}{2};2\right)\Leftrightarrow t\in\left(-2;\,1\right)$.\\
		Từ đồ thị hàm $f'(x)$ suy ra $f'(t)\ge 0,\,\,\forall t\in\left(-2;1\right)$ và $f'(t)=0$ khi $ t=-1$.\\
		Tức là $f'\left(2x-3\right)\ge 0,\,\,\forall x\in\left(\dfrac{1}{2};\,2\right)$$\Rightarrow\underset{x\in\left[\dfrac{1}{2};2\right]}{\min}\,f'\left(2x-3\right)=0$ khi $ x=1$. $(2)$\\
		Xét hàm số $ h(x)=-\dfrac{x}{1+x^2}$ trên khoảng $\left(\dfrac{1}{2};\,2\right)$.\\
		Ta có $h'(x)=\dfrac{x^2-1}{\left(1+x^2\right)^2}$ và\\
		$h'(x)=0\Leftrightarrow{x^2}-1=0\Leftrightarrow x=\pm 1$.\\
		Bảng biến thiên của hàm số $ h(x)$ trên $\left(\dfrac{1}{2};\,2\right)$ như sau
		\begin{center}
			\begin{tikzpicture}
				\tkzTabInit[lgt=1.2,espcl=2.5,nocadre]
				{$x$ /0.7, $h'(x)$ /0.7,$h(x)$ /2.5}
				{$\dfrac{1}{2}$ , $1$,$2$}
				\tkzTabLine{,-,0,+,}
				\tkzTabVar{+/$  $ ,-/$ \-\dfrac{1}{2} $, +/$ $}
			\end{tikzpicture}
		\end{center}
		Từ bảng biến thiên suy ra $ h(x)\ge-\dfrac{1}{2}$$\Rightarrow\underset{x\in\left[\dfrac{1}{2};2\right]}{\min}\,h(x)=-\dfrac{1}{2}$ khi $ x=1$. \, \,  $(3)$\\
		Từ $(1)$, $(2)$ và $(3)$ suy ra $ m\le-\dfrac{1}{2}$.\\
		Kết hợp với $ m\in\mathbb{Z}$, $ m\in\left(-2020;\,2020\right)$ thì $ m\in\left\{-2019;\,-201;\ldots ;-2;-1\right\}$.\\
		Vậy có tất cả $ 2019$ giá trị $ m$ cần tìm.}
\end{ex}

\begin{ex}%[2D1G1-3]%Câu 9
	Cho hàm số $ f(x)$ liên tục trên $\mathbb{R}$ và có đạo hàm $f'(x)=x^2\left(x-2\right)\left(x^2-6x+m\right)$ với mọi $ x\in\mathbb{R}$. Có bao nhiêu số nguyên $ m$ thuộc đoạn $\left[-2020;2020\right]$ để hàm số $ g(x)=f\left(1-x\right)$ nghịch biến trên khoảng $\left(-\infty ;-1\right)$?
	\choice
	{$ 2016$}
	{$ 2014$}
	{\True $ 2012$}
	{$ 2010$}
	\loigiai{
		Ta có \\
		$g'(x)=f'\left(1-x\right)=-\left(1-x\right)^2\left(-x-1\right)\left[\left(1-x\right)^2-6\left(1-x\right)+m\right]$
		$=\left(x-1\right)^2\left(x+1\right)\left(x^2+4x+m-5\right)$.\\
		Hàm số $ g(x)$ nghịch biến trên khoảng $\left(-\infty ;-1\right)$\\
		$\Leftrightarrow{g}'(x)\le 0,\forall x<-1$ $(*)$, (dấu \lq\lq $=$\rq\rq \, xảy ra tại hữu hạn điểm).\\
		Với $ x<-1$ thì $\left(x-1\right)^2>0$ và $ x+1<0$ nên\\
		$(*)$ $\Leftrightarrow{x^2}+4x+m-5\ge 0,\forall x<-1 \Leftrightarrow m\ge-x^2-4x+5,\forall x<-1$.\\
		Xét hàm số $ y=-x^2-4x+5$ trên khoảng $\left(-\infty ;-1\right)$, ta có bảng biến thiên
		\begin{center}
			\begin{tikzpicture}
				\tkzTabInit[lgt=1.8,espcl=2.3]
				{$x$ /1.2, $y'$ /1.2,$y$ /2}
				{$-\infty$ , $-2$,$-1$}
				\tkzTabLine{,+,0,-,}
				\tkzTabVar{-/$ -\infty $ ,+/$9 $, -/$ 8$}
			\end{tikzpicture}
		\end{center}
		Từ bảng biến thiên suy ra $ m\ge 9$.\\
		Kết hợp với $ m$ thuộc đoạn $\left[-2020;2020\right]$ và $ m$ nguyên nên $ m\in\left\{ 9;10;11;\ldots ;2020\right\}$.\\
		Vậy có $ 2012$ số nguyên $ m$ thỏa mãn đề bài.}
\end{ex}

\begin{ex}%[2D1G1-3]%Câu 10
	\immini{
		Cho hàm số $f(x)$ xác định và liên tục trên $ R$. Hàm số $y=f'(x)$ liên tục trên $\mathbb{R}$ và có đồ thị như hình vẽ bên.
		Xét hàm số $g(x)=f\left(x-2m\right)+\dfrac{1}{2}{\left(2m-x\right)^2}+2020$, với $ m$ là tham số thực. Gọi $ S$ là tập hợp các giá trị nguyên dương của $ m$ để hàm số $ y=g(x)$ nghịch biến trên khoảng $\left(3;4\right)$. Hỏi số phần tử của $ S$ bằng bao nhiêu?
		\choice
		{$4$}
		{\True $2$}
		{$3$}
		{Vô số}}
	{
		\begin{tikzpicture}[scale=0.7,>=stealth, font=\footnotesize, line join=round, line cap=round]
			\def\xmin{-3.5} \def\xmax{4.5}
			\def\ymin{-5.2} \def\ymax{4}
			\clip(\xmin,\ymin) rectangle (\xmax,\ymax);
			\draw[->] (\xmin,0)--(\xmax,0) node [below]{$x$};
			\draw[->] (0,\ymin)--(0,\ymax) node [left]{$y$};
			\node at (0,0) [below left]{$O$};
			\path
			(-3.1,3.7) coordinate (A)
			(-3,3) coordinate (B)
			(0,-2) coordinate (C)
			(0.65,-2) coordinate (D)
			(1,-1) coordinate (E)
			(3,-3) coordinate (F)
			(3.4,-5) coordinate (G);
			\draw[smooth]
			(A)..controls +(-88:0.1) and +(93:.1)..
			(B)..controls +(-87:0.3) and +(-100:8.5)..
			(C)..controls +(75:.8) and +(180:.1)..
			(D)..controls +(0:.1) and +(-105:.3)..
			(E)..controls +(70:2) and +(97:0.4)..
			(F)..controls +(-80:.1) and +(90:0.3)..
			(G);
			\draw[dashed] 
			(-3,0)node[below]{$-3$}|-(0,3)node[right]{$3$}
			(1,0)node[above]{$1$}|-(0,-1)node[left]{$-1$}
			(3,0)node[above]{$3$}|-(0,-3)node[below right]{$-3$};
			\fill 
			(0,-2) circle(1.5pt)
			(-3,3) circle(1.5pt)
			(3,-3) circle(1.5pt)
			(1,-1) circle(1.5pt);
			\node at (2.1,-4) {$y=f'(x)$};
		\end{tikzpicture}
	}
	\loigiai{
		Ta có $g'(x)=f'\left(x-2m\right)-\left(2m-x\right)$.		Đặt $h(x)=f'(x)-\left(-x\right)$.\\
		Từ đồ thị hàm số $y=f'(x)$ và đồ thị hàm số $y=-x$ trên hình vẽ suy ra \\
		$h(x)\le 0\Leftrightarrow f'(x)\le-x\Leftrightarrow\hoac{
			&-3\le x\le 1\\ 
			& x\ge 3.}$ 
		\begin{center}
			\begin{tikzpicture}[scale=0.7,>=stealth, font=\footnotesize, line join=round, line cap=round]
				\def\xmin{-3.5} \def\xmax{4.5}
				\def\ymin{-5.2} \def\ymax{4}
				\clip(\xmin,\ymin) rectangle (\xmax,\ymax);
				\draw[->] (\xmin,0)--(\xmax,0) node [below]{$x$};
				\draw[->] (0,\ymin)--(0,\ymax) node [left]{$y$};
				\node at (0,0) [below left]{$O$};
				\path
				(-3.1,3.7) coordinate (A)
				(-3,3) coordinate (B)
				(0,-2) coordinate (C)
				(0.65,-2) coordinate (D)
				(1,-1) coordinate (E)
				(3,-3) coordinate (F)
				(3.4,-5) coordinate (G);
				\draw[smooth]
				(A)..controls +(-88:0.1) and +(93:.1)..
				(B)..controls +(-87:0.3) and +(-100:8.5)..
				(C)..controls +(75:.8) and +(180:.1)..
				(D)..controls +(0:.1) and +(-105:.3)..
				(E)..controls +(70:2) and +(97:0.4)..
				(F)..controls +(-80:.1) and +(90:0.3)..
				(G);
				\draw[dashed] 
				(-3,0)node[below]{$-3$}|-(0,3)node[right]{$3$}
				(1,0)node[above]{$1$}|-(0,-1)node[left]{$-1$}
				(3,0)node[above]{$3$}|-(0,-3)node[below right]{$-3$};
				\fill 
				(0,-2) circle(1.5pt)
				(-3,3) circle(1.5pt)
				(3,-3) circle(1.5pt)
				(1,-1) circle(1.5pt);
				\draw[smooth,samples=300,domain=-3.2:3.7] plot(\x,{-(\x)});
				\node at (2.1,-4) {$y=f'(x)$};
				\node at (-1,2.1) {$y=h(x)$};
			\end{tikzpicture}
		\end{center}
		Ta có $ g'(x)=h\left(x-2m\right)\le 0\Leftrightarrow\hoac{
			&-3\le x-2m\le 1\\ 
			& x-2m\ge 3}\Leftrightarrow\hoac{
			& 2m-3\le x\le 2m+1\\ 
			& x\ge 2m+3.}$.\\
		Suy ra hàm số $ y=g(x)$ nghịch biến trên các khoảng $\left(2m-3;2m+1\right)$ và $\left(2m+3;+\infty\right)$.\\
		Do đó hàm số $ y=g(x)$ nghịch biến trên khoảng $\left(3;4\right)$ $\Leftrightarrow\hoac{
			&\heva{
				& 2m-3\le 3\\ 
				& 2m+1\ge 4}\\ 
			& 2m+3\le 3}\Leftrightarrow\hoac{
			&\dfrac{3}{2}\le m\le 3\\ 
			& m\le 0.}$ \\
		Mặt khác, do $ m$ nguyên dương nên $ m\in\left\{ 2;3\right\}\Rightarrow S=\left\{ 2;3\right\}$. Vậy số phần tử của $ S$ bằng $2$.\\
	}
	
\end{ex}

\begin{ex}%[2D1G1-3]%Câu 11
	Cho hàm số $f(x)$ có đạo hàm trên $\mathbb{R}$ là $f'(x)=\left(x-1\right)\left(x+3\right)$. Có bao nhiêu giá trị nguyên của tham số $m$ thuộc đoạn $\left[-10;20\right]$ để hàm số $y=f\left(x^2+3x-m\right)$ đồng biến trên khoảng $\left(0;2\right)$?
	\choice
	{\True $ 18$}
	{$ 17$}
	{$ 16$}
	{$ 20$}
	\loigiai{
		Ta có $y'=f'\left(x^2+3x-m\right)=\left(2x+3\right){f}'\left(x^2+3x-m\right)$.\\
		Theo đề bài ta có $f'(x)=\left(x-1\right)\left(x+3\right)$\\
		suy ra $f'(x)>0\Leftrightarrow\hoac{
			& x<-3\\ 
			& x>1}$ và $f'(x)<0\Leftrightarrow-3<x<1$ .\\
		Hàm số đồng biến trên khoảng $\left(0;2\right)$ khi $y'\ge 0,\forall x\in\left(0;2\right)$\\
		$\Leftrightarrow\left(2x+3\right){f}'\left(x^2+3x-m\right)\ge 0,\forall x\in\left(0;2\right)$.\\
		Do $x\in\left(0;2\right)$ nên $2x+3>0,\forall x\in\left(0;2\right)$. Do đó, ta có\\
		$y'\ge 0,\forall x\in\left(0;2\right)\Leftrightarrow f'\left(x^2+3x-m\right)\ge 0$\\
		$\Leftrightarrow\hoac{
			&{x^2}+3x-m\le-3\\ 
			&{x^2}+3x-m\ge 1}\Leftrightarrow\hoac{
			& m\ge{x^2}+3x+3\\ 
			& m\le{x^2}+3x-1}$\\
		$\Leftrightarrow\hoac{
			& m\ge\underset{\left[0;2\right]}{\max}\,\left(x^2+3x+3\right)\\ 
			& m\le\underset{\left[0;2\right]}{\min}\,\left(x^2+3x-1\right)} \Leftrightarrow\hoac{
			& m\ge 13\\ 
			& m\le-1}$.\\
		Do $m\in\left[-10;20\right]$, $ m\in\mathbb{Z}$ nên có $ 18$ giá trị nguyên của $m$ thỏa yêu cầu đề bài.}
\end{ex}

\begin{ex}%[2D1G1-3]%Câu 12
	Cho các hàm số $f(x)=x^3+4x+m$ và $g(x)=\left(x^2+2018\right){\left(x^2+2019\right)^2}{\left(x^2+2020\right)^3}$ . Có bao nhiêu giá trị nguyên của tham số $m\in\left[-2020;2020\right]$ để hàm số $g\left(f(x)\right)$ đồng biến trên $\left(2;+\infty\right)$ ?
	\choice
	{$2005$}
	{\True $2037$}
	{$4016$}
	{$4041$}
	\loigiai{
		Ta có $f(x)=x^3+4x+m$ và \\
		$g(x)=\left(x^2+2018\right){\left(x^2+2019\right)^2}{\left(x^2+2020\right)^3}=a_{12}{x^{12}}+a_{10}{x^{10}}+...+a_2x^2+a_0$.\\
		Suy ra $f'(x)=3x^2+4$ , $g'(x)=12a_{12}{x^{11}}+10a_{10}{x^9}+...+2a_2x$.\\
		Và có 
		\begin{eqnarray*}
			\left[g\left(f(x)\right)\right]' &=& f'(x)\left[12a_{12}{\left(f(x)\right)^{11}}+10a_{10}{\left(f(x)\right)^9}+...+2a_2f(x)\right]\\
			&=& f(x)f'(x)\left(12a_{12}{\left(f(x)\right)^{10}}+10a_{10}{\left(f(x)\right)^8}+...+2a_2\right).
		\end{eqnarray*} 
		Dễ thấy $a_{12};{a_{10}};...;{a_2};{a_0}>0$ và $f'(x)=3x^2+4>0$, $\forall x>2$.\\
		Do đó $f'(x)\left(12a_{12}{\left(f(x)\right)^{10}}+10a_{10}{\left(f(x)\right)^8}+...+2a_2\right)>0$ , $\forall x>2$.\\
		Hàm số $g\left(f(x)\right)$ đồng biến trên $\left(2;+\infty\right)$ khi $\left[g\left(f(x)\right)\right]^{'}\ge 0$, $\forall x>2$\\
		$\Rightarrow  f(x)\ge 0$, $\forall x>2 \Leftrightarrow x^3+4x+m\ge 0$, $\forall x>3 \Leftrightarrow  m\ge-x^3-4x$, $\forall x>2$\\
		$ \Rightarrow  m\ge\underset{\left[2;+\infty\right)}{\max}\,\left(-x^3-4x\right)=-16$.\\
		Vì $m\in\left[-2020;2020\right]$ và $m\in\mathbb{Z}$ nên có $2037$ giá trị thỏa mãn $m$ .}
\end{ex}

\begin{ex}%[2D1G1-3]%Câu 13
	Cho hàm số $y=f(x)$ có đạo hàm $f'(x)=x{\left(x+1\right)^2}\left(x^2+2mx+1\right)$ với mọi $x \in \mathbb{R}$. Có bao nhiêu số nguyên âm $m$ để hàm số $g(x)=f\left(2x+1\right)$ đồng biến trên khoảng $\left(3;5\right)$?
	\choice
	{\True $3$}
	{$2$}
	{$4$}
	{$6$}
	\loigiai{
		Ta có $g'(x)=2f'(2x+1)=2(2x+1)(2x+2)^2[(2x+1)^2+2m(2x+1)+1]$. 	Đặt $t=2x+1$\\
		Để hàm số $g(x)$ đồng biến trên khoảng $\left(3;5\right)$ khi và chỉ khi 
		\begin{eqnarray*}
			& & g'(x)\ge 0,\forall x\in\left(3;5\right) \\
			& \Leftrightarrow & t(t^2+2mt+1)\ge 0,\forall t\in\left(7;11\right)\Leftrightarrow{t^2}+2mt+1\ge 0,\,\,\forall t\in\left(7;11\right) \\
			&\Leftrightarrow & 2m\ge\dfrac{-t^2-1}{t},\,\,\,\forall t\in\left(7;11\right)
		\end{eqnarray*}	
		Xét hàm số $h(t)=\dfrac{-t^2-1}{t}$ trên $\left[7;11\right]$, có $h'(t)=\dfrac{-t^2+1}{t^2}$\\
		Bảng biến thiên
		\begin{center}
			\begin{tikzpicture}
				\tkzTabInit[espcl=3,lgt=1.2,nocadre]
				{$t$/0.7,$h'(t)$/0.7,$h(t)$/2.5}
				{$-\infty$,$1$,$11$,$+\infty$}
				\tkzTabLine{, ,,-,,,}
				%	\node (0) at ($(N12)+(0,-3)$) {$-\infty$};
				\node (1) at ($(N22)+(0,-0.8)$) [right] {$-\dfrac{50}{7}$};
				\node (2) at ($(N32)+(0,-2.5)$) [left] {$-\dfrac{122}{11}$};
				
				
				%				\node (3) at ($(N11+(-0.5,0))$) {};
				%				\node (4) at ($(N23)$) {};
				\fill[pattern=north east lines] (7.0,-0.7) rectangle (10,-4.4);
				\fill[pattern=north east lines] (1.5,-0.7) rectangle (4.5,-4.4);
				\draw[->] (1)--(2);	
				\draw[dashed] (4.5,-0.7)--(4.5,-4.4);
				\draw[dashed] (7.0,-0.7)--(7.0,-4.4);	
			\end{tikzpicture}		
		\end{center}
		Dựa vào BBT ta có $2m\ge\dfrac{-t^2-1}{t},\,\,\,\forall t\in\left(7;11\right)\Leftrightarrow 2m\ge\underset{\left[7;11\right]}{\max}\,h(t)\Leftrightarrow m\ge-\dfrac{50}{14}$\\
		Vì $ m\in{\mathbb{Z}^-}\Rightarrow m \in \{-3;-2;-1\}$ .
	}
\end{ex}

\begin{ex}%[2D1G1-3]%Câu 14
	Cho hàm số $y=f(x)$ có bảng biến thiên như sau\\
	\begin{center}
		\begin{tikzpicture}[>=stealth,scale = 1]
			\tkzTabInit[lgt=1,espcl=2.5,nocadre]
			{$x$ /0.7, $y'$ /0.7,$y$ /2.5}
			{$-\infty$,$0$,$2$,$+\infty$}
			\tkzTabLine{ ,-,0,+,0,-,}
			\tkzTabVar{-/$-\infty$, +/$4$,- /$0$, +/{ $+\infty$}}
		\end{tikzpicture}
	\end{center}
	Có bao nhiêu số nguyên $m<2019$ để hàm số $g(x)=f\left(x^2-2x+m\right)$ đồng biến trên khoảng $\left(1;+\infty\right)$?
	\choice
	{\True $2016$}
	{$2015$}
	{$2017$}
	{$2018$}
	\loigiai{
		Ta có $g'(x)=\left(x^2-2x+m\right)'{f}'\left(x^2-2x+m\right)=2\left(x-1\right){f}'\left(x^2-2x+m\right)$ .\\
		Hàm số $y=g(x)$ đồng biến trên khoảng $\left(1;+\infty\right)$ khi và chỉ khi $g'(x)\ge 0,\forall x\in\left(1;+\infty\right)$ và\\
		$g'(x)=0$ tại hữu hạn điểm \\
		$\Leftrightarrow 2\left(x-1\right){f}'\left(x^2-2x+m\right)\ge 0,\forall x\in\left(1;+\infty\right)$\\
		$\Leftrightarrow{f}'\left(x^2-2x+m\right)\ge 0,\forall x\in\left(1;+\infty\right)$ $\Leftrightarrow\hoac{
			&{x^2}-2x+m\ge 2,\forall x\in\left(1;+\infty\right)\\ 
			&{x^2}-2x+m\le 0,\forall x\in\left(1;+\infty\right).}$\\
		Xét hàm số $y=x^2-2x+m$, ta có bảng biến thiên
		\begin{center}
			\begin{tikzpicture}[>=stealth,scale = 1]
				\tkzTabInit[lgt=1,espcl=2.5,nocadre]
				{$x$ /0.7, $y'$ /0.7,$y$ /2.5}
				{$-\infty$,$2$,$+\infty$}
				\tkzTabLine{ ,-,0,+,}
				\tkzTabVar{+/$+\infty$, -/$m-1$, +/{$+\infty$}}
			\end{tikzpicture}
		\end{center}
		Dựa vào bảng biến thiên ta có\\
		TH1: $x^2-2x+m\ge 2,\forall x\in\left(1;+\infty\right)\Leftrightarrow m-1\ge 2\Leftrightarrow m\ge 3$ .\\
		TH2: $x^2-2x+m\le 0,\forall x\in\left(1;+\infty\right)$. Không có giá trị $m$ thỏa mãn.\\
		Vậy có $2016$ số nguyên $m<2019$ thỏa mãn yêu cầu bài toán.}
\end{ex}

\begin{ex}%[2D1G1-3]%Câu 15
	\immini{
		Cho hàm số $ y=f(x)$ có đạo hàm là hàm số $f'(x)$ trên $\mathbb{R}$. Biết rằng hàm số $ y=f'\left(x-2\right)+2$ có đồ thị như hình vẽ bên dưới. Hàm số $ f(x)$ đồng biến trên khoảng nào?
		\choice
		{$\left(-\infty ;3\right),\,\,\left(5;+\infty\right)$}
		{\True $\left(-\infty ;-1\right),\,\,\left(1;+\infty\right)$}
		{$\left(-1;1\right)$}
		{$\left(3;5\right)$}}{
		\begin{tikzpicture}[scale=0.7,font=\footnotesize, line join=round, line cap=round, >=stealth] %Đường cong bậc 3
			\draw[thick, ->] (-0.5,0)--(3.5,0);
			\draw[thick, ->] (0,-1.8)--(0,5.3);
			\draw (3.7,0) node[below] {$x$};
			\draw (0,5.4) node[left]{$y$};
			\draw (0,0) node[below left]{$0$};
			\draw[fill] (3,0) circle (0.5pt)node[below]{$ 3$};
			\draw[fill] (1,0) circle (0.5pt)node[below]{$ 1$};
			\draw[fill] (2,0) circle (0.5pt)node[above]{$2$};
			\draw[fill] (0,2) circle (0.5pt)node[left]{$ 2$};
			\draw[fill] (0,-1) circle (0.5pt)node[left]{$ -1$};
			%		\draw[fill] (0,2) circle (0.5pt)node[above left]{$ 2$};
			%		\draw[fill] (0,-2) circle (0.5pt)node[below left]{$ -2$};
			\draw[dashed] (3,0)--(3,2)--(0,2)--(1,2)--(1,0); 
			\draw[dashed](0,-1)--(2,-1)--(2,0);
			\draw[line width=1.2pt,smooth,samples=100,domain=0.6:3.4] plot(\x,{3*(\x)^2-12*(\x)+11});		
			%\draw[line width=1.2pt,smooth,samples=100,domain=-3.3:2.8] plot(\x,{0.75*(\x)^2+0.5*\x-1});
			%	\draw (2.0,2.8) node[left]{$y=f'(x)$};
	\end{tikzpicture}	}
	\loigiai{	
		Hàm số $ y=f'\left(x-2\right)+2$ có đồ thị $(C)$ như sau:\\
		\begin{center}
			\begin{tikzpicture}[scale=0.7,font=\footnotesize, line join=round, line cap=round, >=stealth] %Đường cong bậc 3
				\draw[thick, ->] (-0.5,0)--(3.5,0);
				\draw[thick, ->] (0,-1.8)--(0,5.3);
				\draw (3.7,0) node[below] {$x$};
				\draw (0,5.4) node[left]{$y$};
				\draw (0,0) node[below left]{$0$};
				\draw[fill] (3,0) circle (0.5pt)node[below]{$ 3$};
				\draw[fill] (1,0) circle (0.5pt)node[below]{$ 1$};
				\draw[fill] (2,0) circle (0.5pt)node[above]{$2$};
				\draw[fill] (0,2) circle (0.5pt)node[left]{$ 2$};
				\draw[fill] (0,-1) circle (0.5pt)node[left]{$ -1$};
				%		\draw[fill] (0,2) circle (0.5pt)node[above left]{$ 2$};
				%		\draw[fill] (0,-2) circle (0.5pt)node[below left]{$ -2$};
				\draw[dashed] (3,0)--(3,2)--(0,2)--(1,2)--(1,0); 
				\draw[dashed](0,-1)--(2,-1)--(2,0);
				\draw[line width=1.2pt,smooth,samples=100,domain=0.6:3.4] plot(\x,{3*(\x)^2-12*(\x)+11});		
				%\draw[line width=1.2pt,smooth,samples=100,domain=-3.3:2.8] plot(\x,{0.75*(\x)^2+0.5*\x-1});
				%	\draw (2.0,2.8) node[left]{$y=f'(x)$};
			\end{tikzpicture}
		\end{center}
		Dựa vào đồ thị $(C)$ ta có\\ $f'\left(x-2\right)+2>2,\forall x\in\left(-\infty ;1\right)\cup\left(3;+\infty\right)\Leftrightarrow{f}'\left(x-2\right)>0,\forall x\in\left(-\infty ;1\right)\cup\left(3;+\infty\right)$ .\\
		Đặt $ x*=x-2$ suy ra $f'\left(x*\right)>0,\forall x*\in\left(-\infty ;-1\right)\bigcup\left(1;+\infty\right)$.\\
		Vậy hàm số $ f(x)$ đồng biến trên khoảng $\left(-\infty ;-1\right),\,\,\left(1;+\infty\right)$.}
\end{ex}

\begin{ex}%[2D1G1-2]%Câu 16
	\immini{
		Cho hàm số $ y=f(x)$ có đạo hàm là hàm số $f'(x)$ trên $\mathbb{R}$. Biết rằng hàm số $ y=f'\left(x+2\right)-2$ có đồ thị như hình vẽ bên dưới. Hàm số $ f(x)$ nghịch biến trên khoảng nào?
		\choice
		{$\left(-3;-1\right),\,\,\left(1;3\right)$}
		{\True $\left(-1;1\right),\,\,\left(3;5\right)$}
		{$\left(-\infty ;-2\right),\,\,\left(0;2\right)$}
		{$\left(-5;-3\right),\,\,\left(-1;1\right)$}}{
		\begin{tikzpicture}[scale=0.7,font=\footnotesize, line join=round, line cap=round, >=stealth] %Đường cong bậc 3
			\draw[thick, ->] (-3.8,0)--(4.0,0);
			\draw[thick, ->] (0,-4.8)--(0,3.5);
			\draw (4.2,0) node[below] {$x$};
			\draw (0,3.7) node[left]{$y$};
			\draw (0,0) node[below left]{$0$};
			\draw[fill] (-3,0) circle (0.5pt)node[above]{$ -3$};
			\draw[fill] (-1,0) circle (0.5pt)node[above]{$ -1$};
			\draw[fill] (1,0) circle (0.5pt)node[above]{$ 1$};
			\draw[fill] (3,0) circle (0.5pt)node[above]{$3$};
			\draw[fill] (0,2) circle (0.5pt)node[above left]{$ 2$};
			\draw[fill] (0,-1) circle (0.5pt)node[above right]{$ -1$};
			%		\draw[fill] (0,2) circle (0.5pt)node[above left]{$ 2$};
			%		\draw[fill] (0,-2) circle (0.5pt)node[below left]{$ -2$};
			\draw[dashed] (-3,0)--(-3,-2)--(3,-2)--(3,0) (-1,0)--(-1,-2) (1,0)--(1,-2) (-3.494,0)--(-3.494,2)--(3.494,2)--(3.494,0); 
			\draw[line width=1.2pt,smooth,samples=100,domain=-3.6:3.6] plot(\x,{0.11*(\x)^4-1.11*(\x)^2-1});		
			%\draw[line width=1.2pt,smooth,samples=100,domain=-3.3:2.8] plot(\x,{0.75*(\x)^2+0.5*\x-1});
			%	\draw (2.0,2.8) node[left]{$y=f'(x)$};
	\end{tikzpicture}	}
	\loigiai{
		Hàm số $ y=f'\left(x+2\right)-2$ có đồ thị $(C)$ như sau
		\begin{center}
			\begin{tikzpicture}[scale=0.7,font=\footnotesize, line join=round, line cap=round, >=stealth] %Đường cong bậc 3
				\draw[thick, ->] (-3.8,0)--(4.0,0);
				\draw[thick, ->] (0,-4.8)--(0,3.5);
				\draw (4.2,0) node[below] {$x$};
				\draw (0,3.7) node[left]{$y$};
				\draw (0,0) node[below left]{$0$};
				\draw[fill] (-3,0) circle (0.5pt)node[above]{$ -3$};
				\draw[fill] (-1,0) circle (0.5pt)node[above]{$ -1$};
				\draw[fill] (1,0) circle (0.5pt)node[above]{$ 1$};
				\draw[fill] (3,0) circle (0.5pt)node[above]{$3$};
				\draw[fill] (0,2) circle (0.5pt)node[above left]{$ 2$};
				\draw[fill] (0,-1) circle (0.5pt)node[above right]{$ -1$};
				%		\draw[fill] (0,2) circle (0.5pt)node[above left]{$ 2$};
				%		\draw[fill] (0,-2) circle (0.5pt)node[below left]{$ -2$};
				\draw[dashed] (-3,0)--(-3,-2)--(3,-2)--(3,0) (-1,0)--(-1,-2) (1,0)--(1,-2) (-3.494,0)--(-3.494,2)--(3.494,2)--(3.494,0); 
				\draw[line width=1.2pt,smooth,samples=100,domain=-3.6:3.6] plot(\x,{0.11*(\x)^4-1.11*(\x)^2-1});		
				%\draw[line width=1.2pt,smooth,samples=100,domain=-3.3:2.8] plot(\x,{0.75*(\x)^2+0.5*\x-1});
				%	\draw (2.0,2.8) node[left]{$y=f'(x)$};
			\end{tikzpicture}
		\end{center}
		Dựa vào đồ thị $(C)$ ta có\\
		$f'\left(x+2\right)-2<-2,\forall x\in\left(-3;-1\right)\bigcup\left(1;3\right)\Leftrightarrow{f}'\left(x+2\right)<0,\forall x\in\left(-3;-1\right)\bigcup\left(1;3\right)$.\\
		Đặt $ x^*=x+2$ suy ra: $f'\left(x^*\right)<0,\forall x^*\in\left(-1;1\right)\bigcup\left(3;5\right)$.\\
		Vậy: Hàm số $ f(x)$ đồng biến trên khoảng $\left(-1;1\right),\,\,\left(3;5\right)$.}
\end{ex}

\begin{ex}%[2D1G1-2]%Câu 17
	\immini{
		Cho hàm số $ y=f(x)$ có đạo hàm là hàm số $f'(x)$ trên $\mathbb{R}$. Biết rằng hàm số $ y=f'\left(x-2\right)+2$ có đồ thị như hình vẽ bên dưới. Hàm số $ f(x)$ nghịch biến trên khoảng nào?
		\choice
		{$\left(-\infty ;2\right)$}
		{\True $\left(-1;1\right)$}
		{$\left(\dfrac{3}{2};\dfrac{5}{2}\right)$}
		{$\left(2;+\infty\right)$}}{
		\begin{tikzpicture}[scale=0.7,font=\footnotesize, line join=round, line cap=round, >=stealth] %Đường cong bậc 3
			\draw[thick, ->] (-0.5,0)--(3.5,0);
			\draw[thick, ->] (0,-1.8)--(0,5.3);
			\draw (3.7,0) node[below] {$x$};
			\draw (0,5.4) node[left]{$y$};
			\draw (0,0) node[below left]{$0$};
			\draw[fill] (3,0) circle (0.5pt)node[below]{$ 3$};
			\draw[fill] (1,0) circle (0.5pt)node[below]{$ 1$};
			\draw[fill] (2,0) circle (0.5pt)node[above]{$2$};
			\draw[fill] (0,2) circle (0.5pt)node[left]{$ 2$};
			\draw[fill] (0,-1) circle (0.5pt)node[left]{$ -1$};
			%		\draw[fill] (0,2) circle (0.5pt)node[above left]{$ 2$};
			%		\draw[fill] (0,-2) circle (0.5pt)node[below left]{$ -2$};
			\draw[dashed] (3,0)--(3,2)--(0,2)--(1,2)--(1,0); 
			\draw[dashed](0,-1)--(2,-1)--(2,0);
			\draw[line width=1.2pt,smooth,samples=100,domain=0.6:3.4] plot(\x,{3*(\x)^2-12*(\x)+11});		
			%\draw[line width=1.2pt,smooth,samples=100,domain=-3.3:2.8] plot(\x,{0.75*(\x)^2+0.5*\x-1});
			%	\draw (2.0,2.8) node[left]{$y=f'(x)$};
	\end{tikzpicture}	}
	\loigiai{
		Hàm số $ y=f'\left(x-2\right)+2$ có đồ thị $(C)$ như sau
		\begin{center}
			\begin{tikzpicture}[scale=0.7,font=\footnotesize, line join=round, line cap=round, >=stealth] %Đường cong bậc 3
				\draw[thick, ->] (-0.5,0)--(3.5,0);
				\draw[thick, ->] (0,-1.8)--(0,5.3);
				\draw (3.7,0) node[below] {$x$};
				\draw (0,5.4) node[left]{$y$};
				\draw (0,0) node[below left]{$0$};
				\draw[fill] (3,0) circle (0.5pt)node[below]{$ 3$};
				\draw[fill] (1,0) circle (0.5pt)node[below]{$ 1$};
				\draw[fill] (2,0) circle (0.5pt)node[above]{$2$};
				\draw[fill] (0,2) circle (0.5pt)node[left]{$ 2$};
				\draw[fill] (0,-1) circle (0.5pt)node[left]{$ -1$};
				%		\draw[fill] (0,2) circle (0.5pt)node[above left]{$ 2$};
				%		\draw[fill] (0,-2) circle (0.5pt)node[below left]{$ -2$};
				\draw[dashed] (3,0)--(3,2)--(0,2)--(1,2)--(1,0); 
				\draw[dashed](0,-1)--(2,-1)--(2,0);
				\draw[line width=1.2pt,smooth,samples=100,domain=0.6:3.4] plot(\x,{3*(\x)^2-12*(\x)+11});		
				%\draw[line width=1.2pt,smooth,samples=100,domain=-3.3:2.8] plot(\x,{0.75*(\x)^2+0.5*\x-1});
				%	\draw (2.0,2.8) node[left]{$y=f'(x)$};
			\end{tikzpicture}
		\end{center}
		Dựa vào đồ thị $(C)$ ta có\\
		$f'\left(x-2\right)+2<2,\forall x\in\left(1;3\right)\Leftrightarrow{f}'\left(x-2\right)<0,\forall x\in\left(1;3\right)$.\\
		Đặt $ x^*=x-2$ thì $f'\left(x^*\right)<0,\forall x^*\in\left(-1;1\right)$.\\
		Vậy: Hàm số $ f(x)$ nghịch biến trên khoảng $\left(-1;1\right)$.\\
		Cách khác:\\
		Tịnh tiến sang trái hai đơn vị và xuống dưới $2$ đơn vị thì từ đồ thị $(C)$ sẽ thành đồ thị của hàm$ y=f'(x)$. Khi đó $f'(x)<0,\forall x\in\left(-1;1\right)$.\\
		Vậy hàm số $ f(x)$ nghịch biến trên khoảng $\left(-1;1\right)$.}
\end{ex}

\begin{ex}%[2D1G1-2]%Câu 18
	Cho hàm số $y=f(x)$ có đạo hàm cấp $ 3$ liên tục trên $\mathbb{R}$ và thỏa mãn $f(x)\cdot f'''(x)=x{\left(x-1\right)^2}{\left(x+4\right)^3}$ với mọi $x\in\mathbb{R}$ và $g(x)=\left[f'(x)\right]^2-2f(x)\cdot f''(x)$. Hàm số $h(x)=g\left(x^2-2x\right)$ đồng biến trên khoảng nào dưới đây?
	\choice
	{$\left(-\infty ;1\right)$}
	{$\left(2;+\infty\right)$}
	{$\left(0;1\right)$}
	{\True $\left(1;2\right)$}
	\loigiai{		
		Ta có $g'(x)=2f''(x){f}'(x)-2f'(x)\cdot f''(x)-2f(x)\cdot f'''(x)=-2f(x)\cdot f'''(x);$\\
		Khi đó $\left(h(x)\right)'=\left(2x-2\right){g}'\left(x^2-2x\right)=-2\left(2x-2\right)\left(x^2-2x\right){\left(x^2-2x-1\right)^2}{\left(x^2-2x+4\right)^3}$\\
		$h'(x)=0\Leftrightarrow\hoac{
			& x=0\\ 
			& x=1\\ 
			& x=2\\ 
			& x=1\pm\sqrt{2}.}$ 
		Ta có bảng xét dấu của $h'(x)$
		\begin{center}
			\begin{tikzpicture}
				\tkzTabInit[lgt=1.2,espcl=2,nocadre]
				{$t$/0.7, $h'(x)$ /.7} % first column
				{$-\infty$, $1-\sqrt{2}$,$0$, $1$,$2$,$1+\sqrt{2}$, $+\infty$} % first row
				\tkzTabLine { ,+,0,-,0,+,0,-,0,+,0,- } % second row
				%				\tkzTabLine {,-,z,+,t,+,} % third row
				%				\tkzTabLine {,+,d,-,z,+,} % last row
			\end{tikzpicture}
		\end{center}
		Suy ra hàm số $h(x)=g\left(x^2-2x\right)$ đồng biến trên khoảng $\left(1;2\right)$.}
\end{ex}

\begin{ex}%[2D1G1-2]%Câu 19
	Cho hàm số $ y=f(x)$ xác định trên $\mathbb{R}$. Hàm số $ y=g(x)=f'\left(2x+3\right)+2$ có đồ thị là một parabol với tọa độ đỉnh $ I\left(2;-1\right)$ và đi qua điểm $ A\left(1;2\right)$. Hỏi hàm số $ y=f(x)$ nghịch biến trên khoảng nào dưới đây?
	\choice
	{\True $\left(5;9\right)$}
	{$\left(1;2\right)$}
	{$\left(-\infty ;9\right)$}
	{$\left(1;3\right)$}
	\loigiai{	
		Xét hàm số $ g(x)=f'\left(2x+3\right)+2$ có đồ thị là một Parabol nên có phương trình dạng $ y=g(x)=a{x^2}+bx+c\,\,\,\,(P)$.\\
		Vì $(P)$ có đỉnh $ I\left(2;-1\right)$ nên $\heva{
			&\dfrac{-b}{2a}=2\\ 
			& g(2)=-1} \Leftrightarrow\heva{
			&-b=4a\\ 
			& 4a+2b+c=-1} \Leftrightarrow\heva{
			& 4a+b=0\\ 
			& 4a+2b+c=-1}$.\\
		Vì $(P)$ đi qua điểm $ A\left(1;2\right)$ nên $ g(1)=2\Leftrightarrow a+b+c=2$.\\
		Ta có hệ phương trình $\heva{
			& 4a+b=0\\ 
			& 4a+2b+c=-1\\ 
			& a+b+c=2} \Leftrightarrow\heva{
			& a=3\\ 
			& b=-12\\ 
			& c=11}$ nên $ g(x)=3x^2-12x+11$.\\
		Đồ thị của hàm $ y=g(x)$ là
		\begin{center}
			\begin{tikzpicture}[scale=0.7,font=\footnotesize, line join=round, line cap=round, >=stealth] %Đường cong bậc 3
				\draw[thick, ->] (-0.5,0)--(3.5,0);
				\draw[thick, ->] (0,-1.8)--(0,5.3);
				\draw (3.7,0) node[below] {$x$};
				\draw (0,5.4) node[left]{$y$};
				\draw (0,0) node[below left]{$0$};
				\draw[fill] (3,0) circle (0.5pt)node[below]{$ 3$};
				\draw[fill] (1,0) circle (0.5pt)node[below]{$ 1$};
				\draw[fill] (2,0) circle (0.5pt)node[above]{$2$};
				\draw[fill] (0,2) circle (0.5pt)node[left]{$ 2$};
				\draw[fill] (0,-1) circle (0.5pt)node[left]{$ -1$};
				%		\draw[fill] (0,2) circle (0.5pt)node[above left]{$ 2$};
				%		\draw[fill] (0,-2) circle (0.5pt)node[below left]{$ -2$};
				\draw[dashed] (3,0)--(3,2)--(0,2)--(1,2)--(1,0) (3.2,2)--(3,2); 
				\draw[dashed](0,-1)--(2,-1)--(2,0);
				\draw[line width=1.2pt,smooth,samples=100,domain=0.6:3.4] plot(\x,{3*(\x)^2-12*(\x)+11});		
				%\draw[line width=1.2pt,smooth,samples=100,domain=-3.3:2.8] plot(\x,{0.75*(\x)^2+0.5*\x-1});
				%	\draw (2.0,2.8) node[left]{$y=f'(x)$};
			\end{tikzpicture}	
		\end{center}
		Theo đồ thị ta thấy $ f'(2x+3)\le 0\Leftrightarrow f'(2x+3)+2\le 2\Leftrightarrow 1\le x\le 3$.\\
		Đặt $ t=2x+3\Leftrightarrow x=\dfrac{t-3}{2}$ khi đó $ f'(t)\le 0\Leftrightarrow 1\le\dfrac{t-3}{2}\le 3\Leftrightarrow 5\le t\le 9$.\\
		Vậy $ y=f(x)$ nghịch biến trên khoảng $\left(5;9\right)$.}
\end{ex}

\begin{ex}%[2D1G1-2]%Câu 20
	\immini{
		Cho hàm số $ y=f(x)$, hàm số $f'(x)=x^3+a{x^2}+bx+c\left(a,b,c\in\mathbb{R}\right)$ có đồ thị như hình vẽ bên.
		Hàm số $ g(x)=f\left(f'(x)\right)$ nghịch biến trên khoảng nào dưới đây?
		\choice
		{$\left(1;+\infty\right)$}
		{\True $\left(-\infty ;-2\right)$}
		{$\left(-1;0\right)$}
		{$\left(-\dfrac{\sqrt{3}}{3};\dfrac{\sqrt{3}}{3}\right)$}}{
		\begin{tikzpicture}[scale=0.8,font=\footnotesize, line join=round, line cap=round, >=stealth] %Đường cong bậc 3
			\draw[thick, ->] (-1.7,0)--(1.7,0);
			\draw[thick, ->] (0,-2.7)--(0,3.0);
			\draw (1.9,0) node[below] {$x$};
			\draw (0,3.2) node[left]{$y$};
			\draw (0,0) node[below left]{$0$};
			\draw[fill] (-1,0) circle (0.5pt)node[above left]{$ -1 $};
			\draw[fill] (1,0) circle (0.5pt)node[below right]{$ 1$};
			\draw[line width=1.2pt,smooth,samples=100,domain=-1.3:1.3] plot(\x,{2.667*(\x)^3+0*(\x)^2-2.667*\x});		
			%\draw[line width=1.2pt,smooth,samples=100,domain=-3.3:2.8] plot(\x,{0.75*(\x)^2+0.5*\x-1});
		\end{tikzpicture}	
	}
	\loigiai{	
		Vì các điểm $\left(-1;0\right),\left(0;0\right),\left(1;0\right)$ thuộc đồ thị hàm số $ y=f'(x)$ nên ta có hệ\\
		$\heva{
			&-1+a-b+c=0\\ 
			& c=0\\ 
			& 1+a+b+c=0} \Leftrightarrow\heva{
			& a=0\\ 
			& b=-1\\ 
			& c=0} \Rightarrow {f}'(x)=x^3-x\Rightarrow f''(x)=3x^2-1$.\\
		Ta có $ g(x)=f\left(f'(x)\right)\Rightarrow{g}'(x)=f'\left(f'(x)\right)\cdot f''(x)$.\\
		Xét \\
		$g'(x)=0\Leftrightarrow{g}'(x)=f'\left(f'(x)\right)\cdot f''(x)=0$\\
		$\Leftrightarrow {f}'\left(x^3-x\right)\left(3x^2-1\right)=0\Leftrightarrow\hoac{
			&{x^3}-x=0\\ 
			&{x^3}-x=1\\ 
			&{x^3}-x=-1\\ 
			& 3x^2-1=0} \Leftrightarrow \hoac{
			& x=\pm 1\\ 
			& x=0\\ 
			& x=x_1(x_1\approx 1,325)\\ 
			& x=x_2(x_2\approx-1,325)\\ 
			& x=\pm\dfrac{\sqrt{3}}{3}.}$\\
		Bảng biến thiên
		\begin{center}
			\begin{tikzpicture}
				\tkzTabInit[lgt=1.2,espcl=2,nocadre]
				{$t$/0.7, $h'(x)$ /.7} % first column
				{$-\infty$, $-1{,}325$,$-1$, $-\dfrac{\sqrt{3}}{3}$,$0$,$\dfrac{\sqrt{3}}{3}$,$1$,$1{,}325$, $+\infty$} % first row
				\tkzTabLine { ,-,0,+,0,-,0,+,0,-,0,+,0,-,0,+, } % second row
				%				\tkzTabLine {,-,z,+,t,+,} % third row
				%				\tkzTabLine {,+,d,-,z,+,} % last row
			\end{tikzpicture}
		\end{center}
		Dựa vào bảng biến thiên ta có $ g(x)$ nghịch biến trên $\left(-\infty ;-2\right)$}
\end{ex}
\Closesolutionfile{ans}
\indapan{10}{ans/CD1/Muc_9_10}
\chapter{NHẬN DIỆN KHỐI ĐA DIỆN}
\begin{Solution}{1}
C
\end{Solution}
\begin{Solution}{3}
B
\end{Solution}
\begin{Solution}{4}
A
\end{Solution}
\begin{Solution}{5}
A
\end{Solution}
\begin{Solution}{6}
A
\end{Solution}
\begin{Solution}{7}
B
\end{Solution}
\begin{Solution}{8}
A
\end{Solution}
\begin{Solution}{9}
C
\end{Solution}
\begin{Solution}{10}
B
\end{Solution}
\begin{Solution}{11}
C
\end{Solution}
\begin{Solution}{12}
D
\end{Solution}
\begin{Solution}{13}
B
\end{Solution}
\begin{Solution}{14}
D
\end{Solution}
\begin{Solution}{15}
A
\end{Solution}
\begin{Solution}{16}
B
\end{Solution}
\begin{Solution}{17}
C
\end{Solution}
\begin{Solution}{18}
C
\end{Solution}
\begin{Solution}{19}
C
\end{Solution}
\begin{Solution}{20}
B
\end{Solution}
\begin{Solution}{21}
C
\end{Solution}
\begin{Solution}{22}
B
\end{Solution}
\begin{Solution}{23}
D
\end{Solution}
\begin{Solution}{24}
B
\end{Solution}
\begin{Solution}{25}
D
\end{Solution}
\begin{Solution}{26}
D
\end{Solution}
\begin{Solution}{27}
B
\end{Solution}
\begin{Solution}{28}
A
\end{Solution}
\begin{Solution}{29}
C
\end{Solution}
\begin{Solution}{30}
B
\end{Solution}
\begin{Solution}{31}
D
\end{Solution}
\begin{Solution}{32}
B
\end{Solution}
\begin{Solution}{33}
B
\end{Solution}
\begin{Solution}{34}
C
\end{Solution}
\begin{Solution}{35}
D
\end{Solution}
\begin{Solution}{36}
B
\end{Solution}
\begin{Solution}{37}
B
\end{Solution}
\begin{Solution}{38}
A
\end{Solution}
\begin{Solution}{39}
A
\end{Solution}
\begin{Solution}{40}
D
\end{Solution}
\begin{Solution}{41}
C
\end{Solution}
\begin{Solution}{42}
B
\end{Solution}
\begin{Solution}{43}
A
\end{Solution}
\begin{Solution}{44}
A
\end{Solution}
\begin{Solution}{45}
D
\end{Solution}
\begin{Solution}{46}
C
\end{Solution}
\begin{Solution}{47}
A
\end{Solution}
\begin{Solution}{48}
B
\end{Solution}
\begin{Solution}{49}
B
\end{Solution}
\begin{Solution}{50}
B
\end{Solution}
\begin{Solution}{51}
A
\end{Solution}
\begin{Solution}{52}
A
\end{Solution}
\begin{Solution}{53}
C
\end{Solution}
\begin{Solution}{54}
C
\end{Solution}
\begin{Solution}{55}
C
\end{Solution}
\begin{Solution}{56}
B
\end{Solution}
\begin{Solution}{57}
C
\end{Solution}
\begin{Solution}{58}
C
\end{Solution}
\begin{Solution}{59}
B
\end{Solution}
\begin{Solution}{60}
C
\end{Solution}
\begin{Solution}{61}
A
\end{Solution}
\begin{Solution}{62}
B
\end{Solution}
\begin{Solution}{63}
B
\end{Solution}
\begin{Solution}{64}
D
\end{Solution}
\begin{Solution}{65}
D
\end{Solution}
\begin{Solution}{66}
B
\end{Solution}
\begin{Solution}{67}
A
\end{Solution}
\begin{Solution}{68}
D
\end{Solution}

\chapter{THỂ TÍCH KHỐI CHÓP}
\begin{Solution}{1}
C
\end{Solution}
\begin{Solution}{3}
B
\end{Solution}
\begin{Solution}{4}
A
\end{Solution}
\begin{Solution}{5}
A
\end{Solution}
\begin{Solution}{6}
A
\end{Solution}
\begin{Solution}{7}
B
\end{Solution}
\begin{Solution}{8}
A
\end{Solution}
\begin{Solution}{9}
C
\end{Solution}
\begin{Solution}{10}
B
\end{Solution}
\begin{Solution}{11}
C
\end{Solution}
\begin{Solution}{12}
D
\end{Solution}
\begin{Solution}{13}
B
\end{Solution}
\begin{Solution}{14}
D
\end{Solution}
\begin{Solution}{15}
A
\end{Solution}
\begin{Solution}{16}
B
\end{Solution}
\begin{Solution}{17}
C
\end{Solution}
\begin{Solution}{18}
C
\end{Solution}
\begin{Solution}{19}
C
\end{Solution}
\begin{Solution}{20}
B
\end{Solution}
\begin{Solution}{21}
C
\end{Solution}
\begin{Solution}{22}
B
\end{Solution}
\begin{Solution}{23}
D
\end{Solution}
\begin{Solution}{24}
B
\end{Solution}
\begin{Solution}{25}
D
\end{Solution}
\begin{Solution}{26}
D
\end{Solution}
\begin{Solution}{27}
B
\end{Solution}
\begin{Solution}{28}
A
\end{Solution}
\begin{Solution}{29}
C
\end{Solution}
\begin{Solution}{30}
B
\end{Solution}
\begin{Solution}{31}
D
\end{Solution}
\begin{Solution}{32}
B
\end{Solution}
\begin{Solution}{33}
B
\end{Solution}
\begin{Solution}{34}
C
\end{Solution}
\begin{Solution}{35}
D
\end{Solution}
\begin{Solution}{36}
B
\end{Solution}
\begin{Solution}{37}
B
\end{Solution}
\begin{Solution}{38}
A
\end{Solution}
\begin{Solution}{39}
A
\end{Solution}
\begin{Solution}{40}
D
\end{Solution}
\begin{Solution}{41}
C
\end{Solution}
\begin{Solution}{42}
B
\end{Solution}
\begin{Solution}{43}
A
\end{Solution}
\begin{Solution}{44}
A
\end{Solution}
\begin{Solution}{45}
D
\end{Solution}
\begin{Solution}{46}
C
\end{Solution}
\begin{Solution}{47}
A
\end{Solution}
\begin{Solution}{48}
B
\end{Solution}
\begin{Solution}{49}
B
\end{Solution}
\begin{Solution}{50}
B
\end{Solution}
\begin{Solution}{51}
A
\end{Solution}
\begin{Solution}{52}
A
\end{Solution}
\begin{Solution}{53}
C
\end{Solution}
\begin{Solution}{54}
C
\end{Solution}
\begin{Solution}{55}
C
\end{Solution}
\begin{Solution}{56}
B
\end{Solution}
\begin{Solution}{57}
C
\end{Solution}
\begin{Solution}{58}
C
\end{Solution}
\begin{Solution}{59}
B
\end{Solution}
\begin{Solution}{60}
C
\end{Solution}
\begin{Solution}{61}
A
\end{Solution}
\begin{Solution}{62}
B
\end{Solution}
\begin{Solution}{63}
B
\end{Solution}
\begin{Solution}{64}
D
\end{Solution}
\begin{Solution}{65}
D
\end{Solution}
\begin{Solution}{66}
B
\end{Solution}
\begin{Solution}{67}
A
\end{Solution}
\begin{Solution}{68}
D
\end{Solution}

\begin{Solution}{1}
D
\end{Solution}
\begin{Solution}{2}
C
\end{Solution}
\begin{Solution}{3}
C
\end{Solution}
\begin{Solution}{4}
A
\end{Solution}
\begin{Solution}{5}
B
\end{Solution}
\begin{Solution}{6}
D
\end{Solution}
\begin{Solution}{7}
C
\end{Solution}
\begin{Solution}{8}
D
\end{Solution}
\begin{Solution}{9}
A
\end{Solution}
\begin{Solution}{10}
B
\end{Solution}
\begin{Solution}{11}
D
\end{Solution}
\begin{Solution}{12}
A
\end{Solution}
\begin{Solution}{13}
D
\end{Solution}
\begin{Solution}{14}
B
\end{Solution}
\begin{Solution}{15}
B
\end{Solution}
\begin{Solution}{16}
C
\end{Solution}
\begin{Solution}{1}
A
\end{Solution}
\begin{Solution}{2}
B
\end{Solution}
\begin{Solution}{3}
D
\end{Solution}
\begin{Solution}{4}
D
\end{Solution}
\begin{Solution}{5}
C
\end{Solution}
\begin{Solution}{6}
A
\end{Solution}
\begin{Solution}{7}
D
\end{Solution}
\begin{Solution}{8}
B
\end{Solution}
\begin{Solution}{9}
C
\end{Solution}
\begin{Solution}{10}
C
\end{Solution}
\begin{Solution}{1}
D
\end{Solution}
\begin{Solution}{2}
D
\end{Solution}
\begin{Solution}{3}
B
\end{Solution}
\begin{Solution}{4}
C
\end{Solution}
\begin{Solution}{5}
D
\end{Solution}
\begin{Solution}{6}
A
\end{Solution}
\begin{Solution}{7}
C
\end{Solution}
\begin{Solution}{8}
B
\end{Solution}
\begin{Solution}{9}
A
\end{Solution}
\begin{Solution}{10}
C
\end{Solution}
\begin{Solution}{11}
D
\end{Solution}
\begin{Solution}{12}
C
\end{Solution}
\begin{Solution}{13}
A
\end{Solution}
\begin{Solution}{14}
D
\end{Solution}
\begin{Solution}{15}
A
\end{Solution}
\begin{Solution}{16}
A
\end{Solution}
\begin{Solution}{17}
B
\end{Solution}
\begin{Solution}{18}
C
\end{Solution}
\begin{Solution}{19}
C
\end{Solution}
\begin{Solution}{20}
A
\end{Solution}
\begin{Solution}{21}
D
\end{Solution}
\begin{Solution}{22}
C
\end{Solution}
\begin{Solution}{23}
A
\end{Solution}
\begin{Solution}{24}
C
\end{Solution}
\begin{Solution}{25}
A
\end{Solution}
\begin{Solution}{26}
B
\end{Solution}
\begin{Solution}{27}
B
\end{Solution}
\begin{Solution}{28}
D
\end{Solution}
\begin{Solution}{29}
B
\end{Solution}
\begin{Solution}{30}
D
\end{Solution}
\begin{Solution}{31}
D
\end{Solution}
\begin{Solution}{32}
C
\end{Solution}
\begin{Solution}{33}
D
\end{Solution}
\begin{Solution}{34}
C
\end{Solution}
\begin{Solution}{35}
D
\end{Solution}
\begin{Solution}{36}
D
\end{Solution}
\begin{Solution}{37}
D
\end{Solution}
\begin{Solution}{38}
D
\end{Solution}
\begin{Solution}{39}
D
\end{Solution}
\begin{Solution}{40}
C
\end{Solution}
\begin{Solution}{41}
A
\end{Solution}
\begin{Solution}{1}
A
\end{Solution}
\begin{Solution}{2}
B
\end{Solution}
\begin{Solution}{3}
C
\end{Solution}
\begin{Solution}{4}
A
\end{Solution}
\begin{Solution}{5}
A
\end{Solution}
\begin{Solution}{6}
C
\end{Solution}
\begin{Solution}{7}
C
\end{Solution}
\begin{Solution}{8}
B
\end{Solution}
\begin{Solution}{9}
C
\end{Solution}
\begin{Solution}{10}
B
\end{Solution}
\begin{Solution}{11}
A
\end{Solution}
\begin{Solution}{12}
B
\end{Solution}
\begin{Solution}{13}
B
\end{Solution}
\begin{Solution}{14}
B
\end{Solution}
\begin{Solution}{15}
A
\end{Solution}
\begin{Solution}{16}
B
\end{Solution}
\begin{Solution}{17}
A
\end{Solution}
\begin{Solution}{18}
D
\end{Solution}
\begin{Solution}{19}
C
\end{Solution}
\begin{Solution}{20}
C
\end{Solution}
\begin{Solution}{21}
A
\end{Solution}
\begin{Solution}{22}
C
\end{Solution}
\begin{Solution}{23}
C
\end{Solution}
\begin{Solution}{24}
A
\end{Solution}
\begin{Solution}{25}
B
\end{Solution}
\begin{Solution}{26}
B
\end{Solution}
\begin{Solution}{27}
A
\end{Solution}
\begin{Solution}{28}
A
\end{Solution}
\begin{Solution}{29}
C
\end{Solution}
\begin{Solution}{30}
B
\end{Solution}
\begin{Solution}{31}
A
\end{Solution}
\begin{Solution}{32}
C
\end{Solution}
\begin{Solution}{33}
B
\end{Solution}
\begin{Solution}{34}
A
\end{Solution}
\begin{Solution}{35}
B
\end{Solution}
\begin{Solution}{36}
B
\end{Solution}
\begin{Solution}{37}
B
\end{Solution}
\begin{Solution}{38}
D
\end{Solution}
\begin{Solution}{39}
B
\end{Solution}
\begin{Solution}{40}
A
\end{Solution}
\begin{Solution}{41}
D
\end{Solution}
\begin{Solution}{42}
D
\end{Solution}
\begin{Solution}{43}
A
\end{Solution}
\begin{Solution}{44}
D
\end{Solution}
\begin{Solution}{45}
C
\end{Solution}
\begin{Solution}{46}
B
\end{Solution}
\begin{Solution}{47}
A
\end{Solution}
\begin{Solution}{48}
D
\end{Solution}
\begin{Solution}{49}
B
\end{Solution}
\begin{Solution}{50}
B
\end{Solution}
\begin{Solution}{51}
D
\end{Solution}
\begin{Solution}{52}
C
\end{Solution}
\begin{Solution}{53}
C
\end{Solution}
\begin{Solution}{54}
B
\end{Solution}
\begin{Solution}{55}
D
\end{Solution}
\begin{Solution}{56}
B
\end{Solution}
\begin{Solution}{57}
C
\end{Solution}
\begin{Solution}{58}
A
\end{Solution}
\begin{Solution}{59}
A
\end{Solution}
\begin{Solution}{60}
B
\end{Solution}
\begin{Solution}{61}
D
\end{Solution}
\begin{Solution}{62}
D
\end{Solution}
\begin{Solution}{63}
B
\end{Solution}
\begin{Solution}{64}
A
\end{Solution}
\begin{Solution}{65}
D
\end{Solution}
\begin{Solution}{66}
C
\end{Solution}
\begin{Solution}{67}
A
\end{Solution}
\begin{Solution}{68}
A
\end{Solution}
\begin{Solution}{69}
D
\end{Solution}
\begin{Solution}{70}
C
\end{Solution}
\begin{Solution}{71}
B
\end{Solution}
\begin{Solution}{72}
A
\end{Solution}
\begin{Solution}{73}
C
\end{Solution}
\begin{Solution}{74}
C
\end{Solution}
\begin{Solution}{75}
C
\end{Solution}
\begin{Solution}{76}
A
\end{Solution}
\begin{Solution}{77}
C
\end{Solution}
\begin{Solution}{78}
B
\end{Solution}
\begin{Solution}{79}
D
\end{Solution}
\begin{Solution}{80}
B
\end{Solution}

\chapter{THỂ TÍCH KHỐI LĂNG TRỤ}
\begin{Solution}{1}
C
\end{Solution}
\begin{Solution}{3}
B
\end{Solution}
\begin{Solution}{4}
A
\end{Solution}
\begin{Solution}{5}
A
\end{Solution}
\begin{Solution}{6}
A
\end{Solution}
\begin{Solution}{7}
B
\end{Solution}
\begin{Solution}{8}
A
\end{Solution}
\begin{Solution}{9}
C
\end{Solution}
\begin{Solution}{10}
B
\end{Solution}
\begin{Solution}{11}
C
\end{Solution}
\begin{Solution}{12}
D
\end{Solution}
\begin{Solution}{13}
B
\end{Solution}
\begin{Solution}{14}
D
\end{Solution}
\begin{Solution}{15}
A
\end{Solution}
\begin{Solution}{16}
B
\end{Solution}
\begin{Solution}{17}
C
\end{Solution}
\begin{Solution}{18}
C
\end{Solution}
\begin{Solution}{19}
C
\end{Solution}
\begin{Solution}{20}
B
\end{Solution}
\begin{Solution}{21}
C
\end{Solution}
\begin{Solution}{22}
B
\end{Solution}
\begin{Solution}{23}
D
\end{Solution}
\begin{Solution}{24}
B
\end{Solution}
\begin{Solution}{25}
D
\end{Solution}
\begin{Solution}{26}
D
\end{Solution}
\begin{Solution}{27}
B
\end{Solution}
\begin{Solution}{28}
A
\end{Solution}
\begin{Solution}{29}
C
\end{Solution}
\begin{Solution}{30}
B
\end{Solution}
\begin{Solution}{31}
D
\end{Solution}
\begin{Solution}{32}
B
\end{Solution}
\begin{Solution}{33}
B
\end{Solution}
\begin{Solution}{34}
C
\end{Solution}
\begin{Solution}{35}
D
\end{Solution}
\begin{Solution}{36}
B
\end{Solution}
\begin{Solution}{37}
B
\end{Solution}
\begin{Solution}{38}
A
\end{Solution}
\begin{Solution}{39}
A
\end{Solution}
\begin{Solution}{40}
D
\end{Solution}
\begin{Solution}{41}
C
\end{Solution}
\begin{Solution}{42}
B
\end{Solution}
\begin{Solution}{43}
A
\end{Solution}
\begin{Solution}{44}
A
\end{Solution}
\begin{Solution}{45}
D
\end{Solution}
\begin{Solution}{46}
C
\end{Solution}
\begin{Solution}{47}
A
\end{Solution}
\begin{Solution}{48}
B
\end{Solution}
\begin{Solution}{49}
B
\end{Solution}
\begin{Solution}{50}
B
\end{Solution}
\begin{Solution}{51}
A
\end{Solution}
\begin{Solution}{52}
A
\end{Solution}
\begin{Solution}{53}
C
\end{Solution}
\begin{Solution}{54}
C
\end{Solution}
\begin{Solution}{55}
C
\end{Solution}
\begin{Solution}{56}
B
\end{Solution}
\begin{Solution}{57}
C
\end{Solution}
\begin{Solution}{58}
C
\end{Solution}
\begin{Solution}{59}
B
\end{Solution}
\begin{Solution}{60}
C
\end{Solution}
\begin{Solution}{61}
A
\end{Solution}
\begin{Solution}{62}
B
\end{Solution}
\begin{Solution}{63}
B
\end{Solution}
\begin{Solution}{64}
D
\end{Solution}
\begin{Solution}{65}
D
\end{Solution}
\begin{Solution}{66}
B
\end{Solution}
\begin{Solution}{67}
A
\end{Solution}
\begin{Solution}{68}
D
\end{Solution}

\begin{Solution}{1}
D
\end{Solution}
\begin{Solution}{2}
C
\end{Solution}
\begin{Solution}{3}
C
\end{Solution}
\begin{Solution}{4}
A
\end{Solution}
\begin{Solution}{5}
B
\end{Solution}
\begin{Solution}{6}
D
\end{Solution}
\begin{Solution}{7}
C
\end{Solution}
\begin{Solution}{8}
D
\end{Solution}
\begin{Solution}{9}
A
\end{Solution}
\begin{Solution}{10}
B
\end{Solution}
\begin{Solution}{11}
D
\end{Solution}
\begin{Solution}{12}
A
\end{Solution}
\begin{Solution}{13}
D
\end{Solution}
\begin{Solution}{14}
B
\end{Solution}
\begin{Solution}{15}
B
\end{Solution}
\begin{Solution}{16}
C
\end{Solution}
\begin{Solution}{1}
A
\end{Solution}
\begin{Solution}{2}
B
\end{Solution}
\begin{Solution}{3}
D
\end{Solution}
\begin{Solution}{4}
D
\end{Solution}
\begin{Solution}{5}
C
\end{Solution}
\begin{Solution}{6}
A
\end{Solution}
\begin{Solution}{7}
D
\end{Solution}
\begin{Solution}{8}
B
\end{Solution}
\begin{Solution}{9}
C
\end{Solution}
\begin{Solution}{10}
C
\end{Solution}
\begin{Solution}{1}
D
\end{Solution}
\begin{Solution}{2}
D
\end{Solution}
\begin{Solution}{3}
B
\end{Solution}
\begin{Solution}{4}
C
\end{Solution}
\begin{Solution}{5}
D
\end{Solution}
\begin{Solution}{6}
A
\end{Solution}
\begin{Solution}{7}
C
\end{Solution}
\begin{Solution}{8}
B
\end{Solution}
\begin{Solution}{9}
A
\end{Solution}
\begin{Solution}{10}
C
\end{Solution}
\begin{Solution}{11}
D
\end{Solution}
\begin{Solution}{12}
C
\end{Solution}
\begin{Solution}{13}
A
\end{Solution}
\begin{Solution}{14}
D
\end{Solution}
\begin{Solution}{15}
A
\end{Solution}
\begin{Solution}{16}
A
\end{Solution}
\begin{Solution}{17}
B
\end{Solution}
\begin{Solution}{18}
C
\end{Solution}
\begin{Solution}{19}
C
\end{Solution}
\begin{Solution}{20}
A
\end{Solution}
\begin{Solution}{21}
D
\end{Solution}
\begin{Solution}{22}
C
\end{Solution}
\begin{Solution}{23}
A
\end{Solution}
\begin{Solution}{24}
C
\end{Solution}
\begin{Solution}{25}
A
\end{Solution}
\begin{Solution}{26}
B
\end{Solution}
\begin{Solution}{27}
B
\end{Solution}
\begin{Solution}{28}
D
\end{Solution}
\begin{Solution}{29}
B
\end{Solution}
\begin{Solution}{30}
D
\end{Solution}
\begin{Solution}{31}
D
\end{Solution}
\begin{Solution}{32}
C
\end{Solution}
\begin{Solution}{33}
D
\end{Solution}
\begin{Solution}{34}
C
\end{Solution}
\begin{Solution}{35}
D
\end{Solution}
\begin{Solution}{36}
D
\end{Solution}
\begin{Solution}{37}
D
\end{Solution}
\begin{Solution}{38}
D
\end{Solution}
\begin{Solution}{39}
D
\end{Solution}
\begin{Solution}{40}
C
\end{Solution}
\begin{Solution}{41}
A
\end{Solution}
\begin{Solution}{1}
A
\end{Solution}
\begin{Solution}{2}
B
\end{Solution}
\begin{Solution}{3}
C
\end{Solution}
\begin{Solution}{4}
A
\end{Solution}
\begin{Solution}{5}
A
\end{Solution}
\begin{Solution}{6}
C
\end{Solution}
\begin{Solution}{7}
C
\end{Solution}
\begin{Solution}{8}
B
\end{Solution}
\begin{Solution}{9}
C
\end{Solution}
\begin{Solution}{10}
B
\end{Solution}
\begin{Solution}{11}
A
\end{Solution}
\begin{Solution}{12}
B
\end{Solution}
\begin{Solution}{13}
B
\end{Solution}
\begin{Solution}{14}
B
\end{Solution}
\begin{Solution}{15}
A
\end{Solution}
\begin{Solution}{16}
B
\end{Solution}
\begin{Solution}{17}
A
\end{Solution}
\begin{Solution}{18}
D
\end{Solution}
\begin{Solution}{19}
C
\end{Solution}
\begin{Solution}{20}
C
\end{Solution}
\begin{Solution}{21}
A
\end{Solution}
\begin{Solution}{22}
C
\end{Solution}
\begin{Solution}{23}
C
\end{Solution}
\begin{Solution}{24}
A
\end{Solution}
\begin{Solution}{25}
B
\end{Solution}
\begin{Solution}{26}
B
\end{Solution}
\begin{Solution}{27}
A
\end{Solution}
\begin{Solution}{28}
A
\end{Solution}
\begin{Solution}{29}
C
\end{Solution}
\begin{Solution}{30}
B
\end{Solution}
\begin{Solution}{31}
A
\end{Solution}
\begin{Solution}{32}
C
\end{Solution}
\begin{Solution}{33}
B
\end{Solution}
\begin{Solution}{34}
A
\end{Solution}
\begin{Solution}{35}
B
\end{Solution}
\begin{Solution}{36}
B
\end{Solution}
\begin{Solution}{37}
B
\end{Solution}
\begin{Solution}{38}
D
\end{Solution}
\begin{Solution}{39}
B
\end{Solution}
\begin{Solution}{40}
A
\end{Solution}
\begin{Solution}{41}
D
\end{Solution}
\begin{Solution}{42}
D
\end{Solution}
\begin{Solution}{43}
A
\end{Solution}
\begin{Solution}{44}
D
\end{Solution}
\begin{Solution}{45}
C
\end{Solution}
\begin{Solution}{46}
B
\end{Solution}
\begin{Solution}{47}
A
\end{Solution}
\begin{Solution}{48}
D
\end{Solution}
\begin{Solution}{49}
B
\end{Solution}
\begin{Solution}{50}
B
\end{Solution}
\begin{Solution}{51}
D
\end{Solution}
\begin{Solution}{52}
C
\end{Solution}
\begin{Solution}{53}
C
\end{Solution}
\begin{Solution}{54}
B
\end{Solution}
\begin{Solution}{55}
D
\end{Solution}
\begin{Solution}{56}
B
\end{Solution}
\begin{Solution}{57}
C
\end{Solution}
\begin{Solution}{58}
A
\end{Solution}
\begin{Solution}{59}
A
\end{Solution}
\begin{Solution}{60}
B
\end{Solution}
\begin{Solution}{61}
D
\end{Solution}
\begin{Solution}{62}
D
\end{Solution}
\begin{Solution}{63}
B
\end{Solution}
\begin{Solution}{64}
A
\end{Solution}
\begin{Solution}{65}
D
\end{Solution}
\begin{Solution}{66}
C
\end{Solution}
\begin{Solution}{67}
A
\end{Solution}
\begin{Solution}{68}
A
\end{Solution}
\begin{Solution}{69}
D
\end{Solution}
\begin{Solution}{70}
C
\end{Solution}
\begin{Solution}{71}
B
\end{Solution}
\begin{Solution}{72}
A
\end{Solution}
\begin{Solution}{73}
C
\end{Solution}
\begin{Solution}{74}
C
\end{Solution}
\begin{Solution}{75}
C
\end{Solution}
\begin{Solution}{76}
A
\end{Solution}
\begin{Solution}{77}
C
\end{Solution}
\begin{Solution}{78}
B
\end{Solution}
\begin{Solution}{79}
D
\end{Solution}
\begin{Solution}{80}
B
\end{Solution}

\chapter{MỘT SỐ BÀI TOÁN KHÓ VỀ THỂ TÍCH}
\section{Mức 9,10 điểm}
\setcounter{ex}{0}
\setcounter{dang}{0}
\Opensolutionfile{ans}[ans/CD1/Muc_9_10]
\begin{dang}{Tìm m để hàm số đơn điệu trên các khoảng xác định của nó}
	Đang thiếu bài thầy Jf Câu 1 đến 26 
\end{dang}
\begin{dang}
	{Tìm khoảng đơn điệu của hàm số $g(x) = f\left[ u(x)\right] +v(x)$ khi biết đồ thị hoặc bảng biến thiên của hàm số $y = f'(x)$}
\end{dang}
\begin{ex}[Đề tham khảo 2019]%[2D1K1-2]
	Cho hàm số $f(x)$ có bảng xét dấu của đạo hàm như sau
	\begin{center}
		\begin{tikzpicture}
			\tkzTabInit[nocadre,lgt=1.2,espcl=2,deltacl=0.6]
			{$x$ /0.6,$f'(x)$ /0.6}
			{$-\infty$,$1$,$2$,$3$,$4$,$+\infty$}
			\tkzTabLine{,-,$0$,+,$0$,+,$0$,-,$0$,+,}
		\end{tikzpicture}
	\end{center}
	Hàm số $y=3 f(x+2)-x^3+3 x$ đồng biến trên khoảng nào dưới đây?
	\choice
	{$(-\infty ;-1)$}
	{\True $(-1 ; 0)$}
	{$(0 ; 2)$}
	{$(1 ;+\infty)$}
	\loigiai{
		Ta có $y'=3\left[f'(x+2)-\left(x^2-3\right)\right]$.\\
		Với $x \in(-1 ; 0) \Rightarrow x+2 \in(1 ; 2) \Rightarrow f'(x+2)>0$, lại có $x^2-3<0 \Rightarrow y'>0 ;~ \forall x \in(-1 ; 0)$.\\
		Vậy hàm số $y=3 f(x+2)-x^3+3 x$ đồng biến trên khoảng $(-1 ; 0)$.\\
		Chú ý:\\
		+) Ta xét $x \in(1 ; 2) \subset(1 ;+\infty)
		\Rightarrow x+2 \in(3 ; 4)\\
		\Rightarrow f'(x+2)<0 ;~ x^2-3>0$\\
		Suy ra hàm số nghịch biến trên khoảng $(1 ; 2)$ nên loại hai phương án$(0 ; 2)$ và $(1 ;+\infty)$.\\
		+) Tương tự ta xét
		$x \in(-\infty ;-2) \Rightarrow x+2 \in(-\infty ; 0)\\
		\Rightarrow f'(x+2)<0 ; x^2-3>0 \Rightarrow y'<0 ; ~ \forall x \in(-\infty ;-2)$.\\
		Suy ra hàm số nghịch biến trên khoảng $(-\infty ;-2)$ nên loại$(-\infty ;-1)$.\\
		Vậy hàm số đã cho đồng biến trên khoảng $(-1 ; 0)$.
	}
\end{ex}
\begin{ex}[Đề Tham Khảo 2020 - Lần 1]%[2D1G1-2]
	\immini{
		Cho hàm số $f(x)$. Hàm số $y=f'(x)$ có đồ thị như hình bên. Hàm số $g(x)=f(1-2 x)+x^2-x$ nghịch biến trên khoảng nào dưới đây?
		\choice
		{\True $\left(1 ; \dfrac{3}{2}\right)$}
		{$\left(0 ; \dfrac{1}{2}\right)$}
		{$(-2 ;-1)$}
		{$(2 ; 3)$}
	}
	{
		\begin{tikzpicture}[scale=0.7,>=stealth, font=\footnotesize, line join=round, line cap=round]
			%\def\a{1} \def\b{-6} \def\c{9} \def\d{1} % Hệ số
			\def\xmin{-4} \def\xmax{6}
			\def\ymin{-3} \def\ymax{2} 
			%\draw[color=gray!50,dashed] (\xmin,\ymin) grid (\xmax,\ymax); 
			\draw[->] (\xmin,0)--(\xmax,0) node [below]{$x$};
			\draw[->] (0,\ymin)--(0,\ymax) node [left]{$y$};
			\node at (0,0) [below left]{$O$};
			%\node at (1,3) [below left]{$f'(x)$};
			%\node at (-1.3,4) {$f'(x)$};
			\draw[dashed] (-2,0) node[below]{$-2$}--(-2,1)--(0,1) node[below left]{$1$};
			\draw[dashed] (4,0) node[below left]{$4$}--(4,-2)--(0,-2) node[below left]{$-2$};
			%\draw[dashed] (1,0) node[below]{$1$}--(1,1);
			%\draw[dashed] (-0.5,0) node[below left]{$-0{,}5$}--(-0.5,2.125);
			\clip (\xmin+0.1,\ymin+0.1) rectangle (\xmax-0.5,\ymax-0.1);
			\draw[smooth,samples=300][domain=-4:5.5] plot(\x,{0.071*(\x)^3-0.142*(\x)^2-1.07*(\x)});
		\end{tikzpicture}
	}
	
	\loigiai{
		Ta có : $g(x)=f(1-2 x)+x^2-x \Rightarrow g'(x)=-2 f'(1-2 x)+2 x-1$.\\
		\immini{
			Đặt $t=1-2 x \Rightarrow g'(x)=-2 f'(t)-t$.\\
			$g'(x)=0 \Rightarrow f'(t)=-\dfrac{t}{2}$.\\
			Vẽ đường thẳng $y=-\dfrac{x}{2}$ và đồ thị hàm số $f'(x)$ trên cùng một hệ trục
		}	
		{
			\begin{tikzpicture}[scale=0.7,>=stealth, font=\footnotesize, line join=round, line cap=round]
				%\def\a{1} \def\b{-6} \def\c{9} \def\d{1} % Hệ số
				\def\xmin{-4} \def\xmax{6}
				\def\ymin{-3} \def\ymax{2} 
				%	\draw[color=gray!50,dashed] (\xmin,\ymin) grid (\xmax,\ymax); 
				\draw[->] (\xmin,0)--(\xmax,0) node [below]{$x$};
				\draw[->] (0,\ymin)--(0,\ymax) node [left]{$y$};
				\node at (0,0) [below left]{$O$};
				%\node at (1,3) [below left]{$f'(x)$};
				%\node at (-1.3,4) {$f'(x)$};
				\draw[dashed] (-2,0) node[below]{$-2$}--(-2,1)--(0,1) node[below left]{$1$};
				\draw[dashed] (4,0) node[below]{$4$}--(4,-2)--(0,-2) node[below left]{$-2$};
				%\draw[dashed] (1,0) node[below]{$1$}--(1,1);
				%\draw[dashed] (-0.5,0) node[below left]{$-0{,}5$}--(-0.5,2.125);
				\clip (\xmin+0.1,\ymin+0.1) rectangle (\xmax-0.5,\ymax-0.1);
				\draw[smooth,samples=300][domain=-4:5.5] plot(\x,{0.071*(\x)^3-0.142*(\x)^2-1.07*(\x)});
				\draw[smooth,samples=300][domain=-4:5.5] plot(\x,{(-0.5*(\x)});
			\end{tikzpicture}
		}	Hàm số $g(x)$ nghịch biến $\Rightarrow g'(x) \leq 0 \Rightarrow f'(t) \geq-\dfrac{t}{2}\Rightarrow\hoac{&-2 \leq t \leq 0 \\&t \geq 4.}$\\
		Như vậy $f'(1-2 x) \geq \dfrac{1-2 x}{-2}\Rightarrow\hoac{&-2 \leq 1-2 x \leq 0 \\ &4 \leq 1-2 x}\Rightarrow\hoac{&\dfrac{1}{2}\leq x \leq \dfrac{3}{2}\\ &x \leq-\dfrac{3}{2}.}$\\
		Vậy hàm số $g(x)=f(1-2 x)+x^2-x$ nghịch biến trên các khoảng $\left(\dfrac{1}{2}; \dfrac{3}{2}\right)$ và $\left(-\infty ;-\dfrac{3}{2}\right)$.\\
		Mà $\left(1 ; \dfrac{3}{2}\right) \subset \left(\dfrac{1}{2}; \dfrac{3}{2}\right)$ nên hàm số $g(x)=f(1-2 x)+x^2-x$ nghịch biến trên khoảng $\left(1 ; \dfrac{3}{2}\right)$.
	}
\end{ex}
\begin{ex}[Chuyên Lê Quý Đôn Điện Biên 2019]%[2D1G1-2]
	Cho hàm số $f(x)$ có bảng xét dấu của đạo hàm như sau
	\begin{center}
		\begin{tikzpicture}
			\tkzTabInit[nocadre,lgt=1.2,espcl=2,deltacl=0.6]
			{$x$ /0.6,$f'(x)$ /0.6}
			{$-\infty$,$0$,$1$,$2$,$3$,$+\infty$}
			\tkzTabLine{,+,$0$,-,$0$,-,$0$,+,$0$,-,}
		\end{tikzpicture}
	\end{center}
	Hàm số $y=f(x-1)+x^3-12 x+2019$ nghịch biến trên khoảng nào dưới đây?
	\choice
	{$(1 ;+\infty)$}
	{\True $(1 ; 2)$}
	{$(-\infty ; 1)$}
	{$(3 ; 4)$}
	\loigiai{
		$y'=f'(x-1)+3 x^2-12=f'(t)+3 t^2+6 t-9=f'(t)-\left(-3 t^2-6 t+9\right)$, với $t=x-1$.\\
		\immini{
			Nghiệm của phương trình $y'=0$ là hoành độ giao điểm của các đồ thị hàm số $y=f'(t)$ và $y=-3 t^2-6 t+9$.\\
			Vẽ đồ thị hàm số $y=f'(t)$ và $y=-3 t^2-6 t+9$ trên cùng một hệ trục tọa độ như hình vẽ bên.
		}	
		{		\begin{tikzpicture}[scale=0.5,>=stealth, font=\footnotesize, line join=round, line cap=round]
				\def\a{-3} \def\b{-6} \def\c{9} % Hệ số
				\def\xmin{-9} \def\xmax{7}
				\def\ymin{-3} \def\ymax{13}
				
				%\draw[color=gray!50,dashed] (\xmin,\ymin) grid (\xmax,\ymax);
				
				\draw[->] (\xmin,0)--(\xmax,0) node [below]{$x$};
				\draw[->] (0,\ymin)--(0,\ymax) node [left]{$y$};
				\node at (0,0) [below left]{$O$};
				\clip (\xmin+0.1,\ymin+0.1) rectangle (\xmax-0.5,\ymax-0.1);
				\draw[smooth,samples=300] plot(\x,{\a*(\x)^2+\b*(\x)+\c});
				\node at (1,0) [above right]{$1$};
				\node at (2,0) [below right]{$2$};
				\node at (3,0) [below right]{$3$};
				\node at (-3,-2) [left]{$y=-3t^2-6t+9$};
				\node at (4,0) [below right]{$f'(x)$};
				\draw (-2.2,10).. controls (-1,1.9) and (-0.5,0.8) .. (0,0);
				%\draw (-2,0).. controls (-1.5,-2) and (-0.5,-0) .. (0,0);
				\draw (0,0).. controls (0.4,-0.6) and (0.6,-0.6) .. (0.8,-0.2);
				\draw (0.8,-0.2).. controls (1,0.25) and (1.1,-0.1) .. (1.4,-0.8);
				\draw (1.4,-0.8).. controls (1.6,-1.1) and (1.7,-0.9) .. (2,0);
				\draw (2,0).. controls (2.4,1.1) and (2.6,1.1) .. (3.5,-1);
			\end{tikzpicture}
		}
		Dựa vào đồ thị trên, ta có bảng xét dấu của hàm số $y'=f'(t)-\left(-3 t^2-6 t+9\right)$ như sau $
		\left(t_0<-1\right)$
		\begin{center}
			\begin{tikzpicture}
				\tkzTabInit[nocadre,lgt=2,espcl=2,deltacl=0.6]
				{$x$ /0.6,$y'$ /0.6}
				{$-\infty$,$t_0$,$1$,$+\infty$}
				\tkzTabLine{,+,$0$,-,$0$,+,}
			\end{tikzpicture}
		\end{center}
		Hàm số nghịch biến trên khoảng $t \in\left(t_0 ; 1\right)$.\\
		Do đó hàm số nghịch biến trên khoảng $x \in(1 ; 2) \subset \left(t_0+1 ; 1\right)$.
	}
\end{ex}


\begin{ex}[Chuyên Phan Bội Châu Nghệ An 2019]%[2D1G1-2]
	Cho hàm số $f(x)$ có bảng xét dấu đạo hàm như sau:
	\begin{center}
		\begin{tikzpicture}
			\tkzTabInit[nocadre,lgt=2,espcl=2,deltacl=0.6]
			{$x$ /0.6,$f'(x)$ /0.6}
			{$-\infty$,$1$,$2$,$3$,$4$,$+\infty$}
			\tkzTabLine{,-,$0$,+,$0$,+,$0$,-,$0$,+,}
		\end{tikzpicture}
	\end{center}
	Hàm số $y=2 f(1-x)+\sqrt{x^2+1}-x$ nghịch biến trên những khoảng nào dưới đây
	\choice
	{$(-\infty ;-2)$}
	{$(-\infty ; 1)$}
	{\True $(-2 ; 0)$}
	{$(-3 ;-2)$}
	\loigiai{
		$y'=-2 f'(1-x)+\dfrac{x}{\sqrt{x^2+1}}-1$. \\
		Có $\dfrac{x}{\sqrt{x^2+1}}-1<0,~ \forall x \in(-2 ; 0)$.\\
		Bảng xét dấu:
		\begin{center}
			\begin{tikzpicture}
				\tkzTabInit[nocadre,lgt=2,espcl=2,deltacl=0.6]
				{$x$ /0.7,$f'(1-x)$ /0.7}
				{$-\infty$,$-3$,$-2$,$-1$,$0$,$+\infty$}
				\tkzTabLine{,+,$0$,-,$0$,+,$0$,+,$0$,-,}
			\end{tikzpicture}
		\end{center}
		$\Rightarrow-2 f'(1-x)<0, ~ \forall x \in(-2 ; 0) \\
		\Rightarrow-2 f'(1-x)+\dfrac{x}{\sqrt{x^2+1}}-1<0, ~\forall x \in(-2 ; 0)$.
	}
\end{ex}
\begin{ex}[Sở Vĩnh Phúc 2019]%[2D1G1-2]
	\immini{
		Cho hàm số bậc bốn $y=f(x)$ có đồ thị của hàm số $y=f'(x)$ như hình vẽ bên.\\
		Hàm số $y=3 f(x)+x^3-6 x^2+9 x$ đồng biến trên khoảng nào trong các khoảng sau đây?
		\choice
		{$(0 ; 2)$}
		{$(-1 ; 1)$}
		{$(1 ;+\infty)$}
		{\True $(-2 ; 0)$}
	}
	{
		\begin{tikzpicture}[scale=0.7,>=stealth, font=\footnotesize, line join=round, line cap=round]
			\def\a{0.21} \def\b{0.88} \def\c{-0.58} \def\d{-3} % Hệ số
			\def\xmin{-5} \def\xmax{5}
			\def\ymin{-4} \def\ymax{3} 
			%\draw[color=gray!50,dashed] (\xmin,\ymin) grid (\xmax,\ymax); 
			\draw[->] (\xmin,0)--(\xmax,0) node [below]{$x$};
			\draw[->] (0,\ymin)--(0,\ymax) node [left]{$y$};
			\node at (0,0) [above left]{$O$};
			\node at (-4,0) [below left]{$-4$};
			\node at (-2,0) [below left]{$-2$};
			\node at (0,-3) [below right]{$-3$};
			\draw[dashed] (2,0) node[above right]{$2$}--(2,1) --(0,1) node[above right]{$1$};
			\clip (\xmin+0.1,\ymin+0.1) rectangle (\xmax-0.5,\ymax-0.1);
			\draw[smooth,samples=300] plot(\x,{\a*(\x)^3+\b*(\x)^2+\c*(\x)+\d});
		\end{tikzpicture}
	}
	
	\loigiai{
		Hàm số $f(x)=a x^4+b x^3+c x^2+d x+e,(a \neq 0)$.
		Có $f'(x)=4 a x^3+3 b x^2+2 c x+d$.\\
		Đồ thị hàm số $y=f'(x)$ đi qua các điểm $(-4 ; 0),(-2 ; 0),(0 ;-3),(2 ; 1)$ nên ta có
		$$\heva{&- 2 5 6 a + 4 8 b - 8 c + d = 0\\
			&- 3 2 a + 1 2 b - 4 c + d = 0\\
			&d = - 3\\
			&3 2 a + 1 2 b + 4 c + d = 1}\Leftrightarrow \heva{&
			a=\dfrac{5}{96}\\
			&b=\dfrac{7}{24}\\
			&c=-\dfrac{7}{24}\\
			&d=-3.}
		$$
		Xét hàm số
		$
		y=3 f(x)+x^3-6 x^2+9 x$\\
		Ta có $ y'=3\left(f'(x)+x^2-4 x+3\right)=3\left(\frac{5}{24}x^3+\frac{15}{8}x^2-\frac{55}{12}x\right)
		$\\
		Ta có $y'=0 \Leftrightarrow\hoac{&x=-11 \\&x=0 \\&x=2.}$ \\
		Xét dấu $y'$, ta được hàm số đã cho đồng biến trên các khoảng $(-11 ; 0)$ và $(2 ;+\infty)$.
	}
\end{ex}
\begin{ex}[Học Mãi 2019]%[2D1K1-2]
	\immini
	{Cho hàm số $y=f(x)$ có đạo hàm trên $\mathbb{R}$. Đồ thị hàm số $y=f'(x)$ như hình bên. Hỏi đồ thị hàm số $y=f(x)-2 x$ có bao nhiêu điểm cực trị?
		\choice
		{$4$}
		{\True $3$}
		{$2$}
		{$1$}
	}
	{
		\begin{tikzpicture}[font=\footnotesize,line join=round, line cap=round,>=stealth,scale=0.8]
			\draw[->] (-3.5,0)--(4,0) node[above] {$x$};
			\draw[->] (0,-3)--(0,4) node[left] {$y$};
			%\fill[black] (-2,0)node[below left]{$-2$} circle (1.2pt) (0,0)node[above right]{$O$} circle (1.2pt) (3,0)node[above]{$3$} circle (1.2pt);
			\draw[dashed] (-2,-2)-- (0,-2) node[right]{$-2$};
			\draw[dashed] (2,0) node[below]{$2$}-- (2,2)--(0,2) node[below left]{$2$};
			\node at (0,0) [below left]{$O$};
			\node at (3,0) [below right]{$3$};
			\draw (-3,2.5).. controls (-2.2,-3) and (-1.8,-3) .. (-1.1,0);
			\draw (-1.1,0).. controls (-0.6,2.5) and (-0.4,2.5) .. (0,2);
			\draw (0,2).. controls (0.7,0.5) and (1.1,0.5) .. (1.5,1.5);
			\draw (1.5,1.5).. controls (2,2.5) and (2.8,2.5) .. (3.5,-2.5);
			%\draw (3,0).. controls (3.3,-0.1) and (3.5,-0.5) .. (3.5,-2);
		\end{tikzpicture}
	}
	\loigiai{
		\immini{
			Đặt $g(x)=f(x)-2 x$.\\
			$\Rightarrow g'(x)=f'(x)-2 .
			$\\
			Vẽ đường thẳng $y=2$.\\
			$\Rightarrow$ phương trình $g'(x)=0$ có $3$ nghiệm bội lẻ.\\
			$\Rightarrow$ đồ thị hàm số $y=f(x)-2 x$ có $3$ điểm cực trị.
		}
		{
			\begin{tikzpicture}[font=\footnotesize,line join=round, line cap=round,>=stealth,scale=0.8]
				\draw[->] (-3.5,0)--(4,0) node[above] {$x$};
				\draw[->] (0,-3)--(0,4) node[left] {$y$};
				%\fill[black] (-2,0)node[below left]{$-2$} circle (1.2pt) (0,0)node[above right]{$O$} circle (1.2pt) (3,0)node[above]{$3$} circle (1.2pt);
				\draw[dashed] (-2,-2)-- (0,-2) node[right]{$-2$};
				\draw[dashed] (2,0) node[below]{$2$}-- (2,2)--(0,2) node[below left]{$2$};
				\node at (3,0) [below left]{$3$};
				\draw (-3,2.5).. controls (-2.2,-3) and (-1.8,-3) .. (-1.1,0);
				\draw (-1.1,0).. controls (-0.6,2.5) and (-0.4,2.5) .. (0,2);
				\draw (0,2).. controls (0.7,0.5) and (1.1,0.5) .. (1.5,1.5);
				\draw (1.5,1.5).. controls (2,2.5) and (2.8,2.5) .. (3.5,-2.5);
				\draw (-3.5,2)--(4,2) node[above]{$y=2$};
			\end{tikzpicture}
		}
	}
\end{ex}
\begin{ex}[THPT Hoàng Hoa Thám Hưng Yên 2019]%[2D1G1-2]
	\immini{
		Cho hàm số $y=f(x)$ liên tục trên $\mathbb{R}$. Hàm số $y=f'(x)$ có đồ thị như hình vẽ. 
		Hàm số $g(x)=f(x-1)+\dfrac{2019-2018 x}{2018}$ đồng biến trên khoảng nào dưới đây?
		\choice
		{$(2 ; 3)$}
		{$(0 ; 1)$}
		{\True $(-1 ; 0)$}
		{$(1 ; 2)$}
	}
	{
		\begin{tikzpicture}[scale=1, font=\footnotesize, line join=round, line cap=round, >=stealth]
			\tikzset{label style/.style={font=\footnotesize}}
			\draw[->] (-2,0)--(3,0) node[below left] {$x$};
			\draw[->] (0,-2)--(0,3) node[below left] {$y$};
			\draw[fill=black] (0,0) node [above left] {$O$} circle(1pt);
			\fill (1,1) circle(1pt) (-1,1) circle(1pt) (2,1) circle(1pt);
			\foreach \x in {1,2}
			\draw[thin] (\x,1pt)--(\x,-1pt) node [below] {\footnotesize$\x$};
			\foreach \x in {-1}
			\draw[thin] (\x,1pt)--(\x,-1pt) node [below left] {\footnotesize$\x$};
			\foreach \y in {-1}
			\draw[thin] (1pt,\y)--(-1pt,\y) node [right] {\footnotesize$\y$};
			\foreach \y in {1}
			\draw[thin] (1pt,\y)--(-1pt,\y) node [above left] {\footnotesize$\y$};
			\draw[dashed](-1,0)--(-1,1)--(2,1) (1,1)--(1,0) (2,1)--(2,0);
			\begin{scope}
				\clip (-3,-3) rectangle (3,3);
				\draw[name path=(C)] plot[smooth,tension=0.7] coordinates{(-1.15,3)(-0.5,-1.6)(.8,.88)(1.9,0.8)(2.3,3)};
			\end{scope}
		\end{tikzpicture}
	}	\loigiai{
		Ta có $g'(x)=f'(x-1)-1$.\\
		$
		g'(x) \geq 0 \Leftrightarrow f'(x-1)-1 \geq 0 \Leftrightarrow f'(x-1) \geq 1 \Leftrightarrow \hoac{&x - 1 \leq - 1\\
			&x - 1 \geq 2}\Leftrightarrow \hoac{&
			x \leq 0 \\
			&x \geq 3.}
		$\\
		Từ đó suy ra hàm số $g(x)=f(x-1)+\dfrac{2019-2018 x}{2018}$ đồng biến trên khoảng $(-1 ; 0)$.
	}
\end{ex}

\begin{ex}[(Sở Ninh Bình 2019]%[2D1K1-2]
	Cho hàm số $y=f(x)$ có bảng xét dấu của đạo hàm như sau
	\begin{center}
		\begin{tikzpicture}
			\tkzTabInit[nocadre,lgt=1,espcl=2,deltacl=0.6]
			{$x$ /0.7,$f'(x)$ /0.7}
			{$-\infty$,$-2$,$-1$,$2$,$4$,$+\infty$}
			\tkzTabLine{,+,$0$,-,$0$,+,$0$,-,$0$,+,}
		\end{tikzpicture}
	\end{center}
	Hàm số $y=-2 f(x)+2019$ nghịch biến trên khoảng nào trong các khoảng dưới đây?
	\choice
	{$(-4 ; 2)$}
	{\True $(-1 ; 2)$}
	{$(-2 ;-1)$}
	{$(2 ; 4)$}
	\loigiai{
		Xét $y=g(x)=-2 f(x)+2019$.\\
		Ta có $g'(x)=(-2 f(x)+2019)'=-2 f'(x), g'(x)=0 \Leftrightarrow\hoac{&x=-2 \\&x=-1 \\&x=2 \\&x=4.}$.\\
		Ta có bảng xét dấu của $g'(x)$
		\begin{center}
			\begin{tikzpicture}
				\tkzTabInit[nocadre,lgt=1,espcl=2,deltacl=0.6]
				{$x$ /0.6,$f'(x)$ /0.6}
				{$-\infty$,$-2$,$-1$,$2$,$4$,$+\infty$}
				\tkzTabLine{,-,$0$,+,$0$,-,$0$,+,$0$,+,}
			\end{tikzpicture}
		\end{center}
		Dựa vào bảng xét dấu, ta thấy hàm số $y=g(x)$ nghịch biến trên khoảng $(-1 ; 2)$.
	}
\end{ex}
\begin{ex}[THPT Lương Thế Vinh Hà Nội 2019]%[2D1G1-2]
	\immini{
		Cho hàm số $y=f(x)$. Biết đồ thị hàm số $y=f'(x)$ có đồ thị như hình vẽ bên. 
		Hàm số $y=f \left(3-x^2\right)+2018$ đồng biến trên khoảng nào dưới đây?
		\choice
		{\True $(-1 ; 0)$}
		{$(2 ; 3)$}
		{$(-2 ;-1)$}
		{$(0 ; 1)$}
	}
	{
		\begin{tikzpicture}[scale=0.6,>=stealth, font=\footnotesize, line join=round, line cap=round]
			\def\a{0.065} \def\b{0.32} \def\c{-0.53} \def\d{-0.82} % Hệ số
			\def\xmin{-8} \def\xmax{4}
			\def\ymin{-3} \def\ymax{3} 
			%\draw[color=gray!50,dashed] (\xmin,\ymin) grid (\xmax,\ymax); 
			\draw[->] (\xmin,0)--(\xmax,0) node [below]{$x$};
			\draw[->] (0,\ymin)--(0,\ymax) node [left]{$y$};
			\node at (0,0) [below left]{$O$};
			\node at (-6,0) [below left]{$-6$};
			\node at (-1,0) [below left]{$-1$};
			\node at (2,0) [below right]{$2$};
			\clip (\xmin+0.1,\ymin+0.1) rectangle (\xmax-0.5,\ymax-0.1);
			\draw[smooth,samples=300][domain=-6.5:3.5] plot(\x,{\a*(\x)^3+\b*(\x)^2+\c*(\x)+\d});
		\end{tikzpicture}
	}
	
	\loigiai{
		Ta có $\left[f\left( 3-x^2\right)+2018 \right]'=-2 x \cdot f'\left(3-x^2\right) $.\\
		$
		-2 x \cdot f'\left(3-x^2\right)=0 \Leftrightarrow\hoac{&
			x = 0\\
			&3 - x ^{2}= - 6\\
			&3 - x ^{2}= - 1\\
			&3 - x ^{2}= 2}
		\Leftrightarrow \hoac{
			&x=0 \\
			&x=\pm 3 \\
			&x=\pm 2 \\
			&	x=\pm 1.}
		$\\
		Bảng xét dấu của đạo hàm hàm số đã cho
		\begin{center}
			\begin{center}
				\begin{tikzpicture}
					\tkzTabInit[nocadre,lgt=2.9,espcl=1.5,deltacl=0.6]
					{$x$ /0.7,$f'\left( 3-x^2\right) $/0.7,$-2xf'\left( 3-x^2\right)$/0.8}
					{$-\infty$,$-3$,$-2$,$-1$,$0$,$1$,$2$,$3$,$+\infty$}
					\tkzTabLine{,-,$0$,+,$0$,-,$0$,+,$0$,+,$0$,-,$0$,+,$0$,-}
					\tkzTabLine{,-,$0$,+,$0$,-,$0$,+,$0$,-,$0$,+,$0$,-,$0$,+}
				\end{tikzpicture}
			\end{center}
		\end{center}
		Từ bảng xét dấu suy ra hàm số đồng biến trên $(-1 ; 0)$.
	}
\end{ex}
\begin{ex}[Chuyên Biên Hòa - Hà Nam - 2020]%[2D1G1-2]
	\immini{
		Cho hàm số đa thức $f(x)$ có đạo hàm trên $\mathbb{R}$. Biết $f(0)=0$ và đồ thị hàm số $y=f'(x)$ như hình sau.
		Hàm số $g(x)=\left|4 f(x)+x^2\right|$ đồng biến trên khoảng nào dưới đây?
		\choice
		{$(4 ;+\infty)$}
		{\True $(0 ; 4)$}
		{$(-\infty ;-2)$}
		{$(-2 ; 0)$}
	}	
	{
		\begin{tikzpicture}[scale=0.7,>=stealth, font=\footnotesize, line join=round, line cap=round]
			%\def\a{1} \def\b{-6} \def\c{9} \def\d{1} % Hệ số
			\def\xmin{-4} \def\xmax{6}
			\def\ymin{-3} \def\ymax{2} 
			%\draw[color=gray!50,dashed] (\xmin,\ymin) grid (\xmax,\ymax); 
			\draw[->] (\xmin,0)--(\xmax,0) node [below]{$x$};
			\draw[->] (0,\ymin)--(0,\ymax) node [left]{$y$};
			\node at (0,0) [below left]{$O$};
			%\node at (1,3) [below left]{$f'(x)$};
			%\node at (-1.3,4) {$f'(x)$};
			\draw[dashed] (-2,0) node[below]{$-2$}--(-2,1)--(0,1) node[below left]{$1$};
			\draw[dashed] (4,0) node[below]{$4$}--(4,-2)--(0,-2) node[below left]{$-2$};
			%\draw[dashed] (1,0) node[below]{$1$}--(1,1);
			%\draw[dashed] (-0.5,0) node[below left]{$-0{,}5$}--(-0.5,2.125);
			\clip (\xmin+0.1,\ymin+0.1) rectangle (\xmax-0.5,\ymax-0.1);
			\draw[smooth,samples=300][domain=-4:5.5] plot(\x,{0.071*(\x)^3-0.142*(\x)^2-1.07*(\x)});
		\end{tikzpicture}
	}
	\loigiai{
		\immini{
			Xét hàm số $h(x)=4 f(x)+x^2$ trên $\mathbb{R}$.\\
			Vì $f(x)$ là hàm số đa thức nên $h(x)$ cũng là hàm số đa thức và $h(0)=4 f(0)=0$.\\
			Ta có $h'(x)=4 f'(x)+2 x$. Do đó $h'(x)=0 \Leftrightarrow f'(x)=-\dfrac{1}{2}x$.\\
		}
		{
			\begin{tikzpicture}[scale=0.7,>=stealth, font=\footnotesize, line join=round, line cap=round]
				%\def\a{1} \def\b{-6} \def\c{9} \def\d{1} % Hệ số
				\def\xmin{-4} \def\xmax{6}
				\def\ymin{-3} \def\ymax{2} 
				%\draw[color=gray!50,dashed] (\xmin,\ymin) grid (\xmax,\ymax); 
				\draw[->] (\xmin,0)--(\xmax,0) node [below]{$x$};
				\draw[->] (0,\ymin)--(0,\ymax) node [left]{$y$};
				\node at (0,0) [below left]{$O$};
				%\node at (1,3) [below left]{$f'(x)$};
				%\node at (-1.3,4) {$f'(x)$};
				\draw[dashed] (-2,0) node[below]{$-2$}--(-2,1)--(0,1) node[below left]{$1$};
				\draw[dashed] (4,0) node[below]{$4$}--(4,-2)--(0,-2) node[below left]{$-2$};
				%\draw[dashed] (1,0) node[below]{$1$}--(1,1);
				%\draw[dashed] (-0.5,0) node[below left]{$-0{,}5$}--(-0.5,2.125);
				\clip (\xmin+0.1,\ymin+0.1) rectangle (\xmax-0.5,\ymax-0.1);
				\draw[smooth,samples=300][domain=-4:5.5] plot(\x,{0.071*(\x)^3-0.142*(\x)^2-1.07*(\x)});
				\draw[smooth,samples=300][domain=-4:5.5] plot(\x,{-0.5*(\x)});
			\end{tikzpicture}
		}
		Dựa vào sự tương giao của đồ thị hàm số $y=f'(x)$ và đường thẳng $y=-\dfrac{1}{2}x$, ta có
		$
		h'(x)=0 \Leftrightarrow x \in\{-2 ; 0 ; 4\}.\\
		$
		Bảng biến thiên của hàm số $h(x)$ như sau:
		\begin{center}
			\begin{tikzpicture}
				\tkzTabInit[nocadre,lgt=1.2,espcl=2.5,deltacl=0.6]
				{$x$ /0.6,$y'$ /0.6,$y$ /2}
				{$-\infty$,$-2$,$0$,$4$,$+\infty$}
				\tkzTabLine{,-,$0$,+,$0$,-,$0$,+,}
				\tkzTabVar{+/$+\infty$, -/$y_1$,+/$0$,-/$y_3$,+/$+\infty$}
			\end{tikzpicture}
		\end{center}
		Từ đó suy ra bảng biến thiên của hàm số $g(x)=|h(x)|$.\\
		Dựa vào bảng biến thiên trên, ta thấy hàm số $g(x)$ đồng biến trên khoảng $(0 ; 4)$.
	}
\end{ex}
\begin{ex}[Chuyên Thái Bình - 2020]%[2D1G1-2]
	\immini{
		Cho hàm số $f(x)$ liên tục trên $\mathbb{R}$ có đồ thị hàm số $y=f'(x)$ cho như hình vẽ bên.\\
		Hàm số $g(x)=2 f(|x-1|)-x^2+2 x+2020$ đồng biến trên khoảng nào?
		\choice
		{\True $(0 ; 1)$}
		{$(-3 ; 1)$}
		{$(1 ; 3)$}
		{$(-2 ; 0)$}
	}
	{
		\begin{tikzpicture}[scale=0.7,>=stealth, font=\footnotesize, line join=round, line cap=round]
			%\def\a{1} \def\b{-6} \def\c{9} \def\d{1} % Hệ số
			\def\xmin{-4} \def\xmax{5}
			\def\ymin{-3} \def\ymax{5} 
			%\draw[color=gray!50,dashed] (\xmin,\ymin) grid (\xmax,\ymax); 
			\draw[->] (\xmin,0)--(\xmax,0) node [below]{$x$};
			\draw[->] (0,\ymin)--(0,\ymax) node [left]{$y$};
			\node at (0,0) [below left]{$O$};
			%\node at (1,3) [below left]{$f'(x)$};
			\node at (-1.3,4) {$f'(x)$};
			\draw[dashed] (-1,0) node[above]{$-1$}--(-1,-1)--(0,-1) node[below left]{$-1$};
			\draw[dashed] (1,0) node[below]{$1$}--(1,1)--(0,1) node[below left]{$1$};
			\draw[dashed] (3,0) node[below]{$3$}--(3,3)--(0,3) node[below left]{$3$};
			%\draw[dashed] (1,0) node[below]{$1$}--(1,1);
			%\draw[dashed] (-0.5,0) node[below left]{$-0{,}5$}--(-0.5,2.125);
			\clip (\xmin+0.1,\ymin+0.1) rectangle (\xmax-0.5,\ymax-0.1);
			\draw[smooth,samples=300][domain=-2:4] plot(\x,{-0.5*(\x)^3+1.5*(\x)^2+1.5*(\x)-1.5});
			%\draw[smooth,samples=300] plot(\x,{(\x)^3+(\x)^2-2*(\x)+1});
		\end{tikzpicture}
	}
	\loigiai{
		Ta có đường thẳng $y=x$ cắt đồ thị hàm số $y=f'(x)$ tại các điểm $x=-1 ; x=1 ; x=3$ như hình vẽ sau:
		\begin{center}
			\begin{tikzpicture}[scale=0.7,>=stealth, font=\footnotesize, line join=round, line cap=round]
				%\def\a{1} \def\b{-6} \def\c{9} \def\d{1} % Hệ số
				\def\xmin{-4} \def\xmax{5}
				\def\ymin{-3} \def\ymax{5} 
				%\draw[color=gray!50,dashed] (\xmin,\ymin) grid (\xmax,\ymax); 
				\draw[->] (\xmin,0)--(\xmax,0) node [below]{$x$};
				\draw[->] (0,\ymin)--(0,\ymax) node [left]{$y$};
				\node at (0,0) [below left]{$O$};
				%\node at (1,3) [below left]{$f'(x)$};
				\node at (-1.3,4) {$f'(x)$};
				\node at (4,3.2) {$y=x$};
				\draw[dashed] (-1,0) node[above]{$-1$}--(-1,-1)--(0,-1) node[below left]{$-1$};
				\draw[dashed] (1,0) node[below]{$1$}--(1,1)--(0,1) node[below left]{$1$};
				\draw[dashed] (3,0) node[below]{$3$}--(3,3)--(0,3) node[below left]{$3$};
				%\draw[dashed] (1,0) node[below]{$1$}--(1,1);
				%\draw[dashed] (-0.5,0) node[below left]{$-0{,}5$}--(-0.5,2.125);
				\clip (\xmin+0.1,\ymin+0.1) rectangle (\xmax-0.5,\ymax-0.1);
				\draw[smooth,samples=300][domain=-2:4] plot(\x,{-0.5*(\x)^3+1.5*(\x)^2+1.5*(\x)-1.5});
				\draw[smooth,samples=300] plot(\x,{(\x)});
			\end{tikzpicture}
		\end{center}
		Dựa vào đồ thị của hai hàm số trên ta có $f'(x)>x \Leftrightarrow\hoac{&x<-1 \\ &1<x<3}$ và
		$ f'(x)<x \Leftrightarrow\hoac{&
			-1<x<1 \\
			&x>3.}$\\
		+Trường hợp 1: $x-1<0 \Leftrightarrow x<1$.\\
		Khi đó $g(x)=2 f(1-x)-x^2+2 x+2020$.\\
		Ta có $g'(x)=-2 f'(1-x)+2(1-x)$.
		$$
		g'(x)>0 \Leftrightarrow-2 f'(1-x)+2(1-x)>0 \Leftrightarrow f'(1-x)<1-x \Leftrightarrow\hoac{
			&- 1 < 1 - x < 1\\
			&1 - x > 3} \Leftrightarrow \hoac{&
			0<x<2 \\
			&x<-2.}
		$$
		Kết hợp điều kiện, ta có $g'(x)>0 \Leftrightarrow\hoac{&0<x<1 \\ &x<-2.}$\\
		
		+ Trường hợp 2: $x-1>0 \Leftrightarrow x>1$.\\
		Khi đó ta có $g(x)=2 f(x-1)-x^2+2 x+2020$.\\
		$ g'(x)=2 f'(x-1)-2(x-1)$\\
		$g'(x)>0 \Leftrightarrow 2 f'(x-1)-2(x-1)>0 \Leftrightarrow f'(x-1)>x-1 \Leftrightarrow\hoac{&
			x - 1 < - 1\\
			&1 < x - 1 < 3}\Leftrightarrow \hoac{
			&x<0 \\
			&2<x<4.}$
		Kết hợp điều kiện ta có $g'(x)>0 \Leftrightarrow 2<x<4$.\\
		Vậy hàm số $g(x)=2 f(|x-1|)-x^2+2 x+2020$ đồng biến trên khoảng $(0 ; 1)$.
	}
\end{ex}

\begin{ex}[Chuyên Lào Cai - 2020]%[2D1G1-2]
	\immini{
		Cho hàm số $f'(x)$ có đồ thị như hình bên.\\
		Hàm số $g(x)=f(3 x+1)+9 x^3+\dfrac{9}{2}x^2$ đồng biến trên khoảng nào dưới đây?
		\choice
		{$(-1 ; 1)$}
		{$(-2 ; 0)$}
		{$(-\infty ; 0)$}
		{\True $(1 ;+\infty)$}
	}
	{\begin{tikzpicture}[line join=round, line cap=round,>=stealth,thick,scale=.8]
			\tikzset{label style/.style={font=\footnotesize}}
			\draw[->] (-2.1,0)--(5.1,0) node[below left] {$x$};
			\draw[->] (0,-3.1)--(0,4.1) node[below left] {$y$};
			\draw (0,0) node [below left] {$O$};
			\foreach \x in {1,2,3}
			\draw[thin] (\x,1pt)--(\x,-1pt) node [below] {$\x$};
			\draw[thin](-1,1pt)--(1,-1pt)node[above left]{$-1$};
			\foreach \y in {-2,2}
			\draw[thin] (1pt,\y)--(-1pt,\y) node [above right] {$\y$};
			%\begin{scope}
			\clip (-2,-3) rectangle (5,4);
			\draw[samples=200,domain=-2:4,smooth,variable=\x] plot (\x,{(\x)^3-3*(\x)^2+2});
			%\end{scope}
			\draw[dashed] (-1,0)--(-1,-2)--(0,-2);
			\draw[dashed] (3,0)--(3,2)--(0,2);
			%\begin{scope}[on background layer]\path[white]node{MDD-134};\end{scope}
		\end{tikzpicture}
	}
	\loigiai
	{
		\immini{Xét hàm số $g(x)=f(3 x+1)+9 x^3+\dfrac{9}{2}x^2 \\
			\Rightarrow g'(x)=3 f'(3 x+1)+27 x^2+9 x$.\\
			Hàm số đồng biến  $\Leftrightarrow g'(x)>0 \Leftrightarrow 3 f'(3 x+1)+27 x^2+9 x>0$
			\\
			$
			\Leftrightarrow f'(3 x+1)+3 x(3 x+1)>0 \qquad (*)
			$\\
			Đặt $t=3 x+1$, khi đó  $(*) \Leftrightarrow f'(t)+(t-1) t>0$\\ $\Leftrightarrow f'(t)>-t^2+t$.\\
			Vẽ parabol $y=-x^2+x$ và đồ thị hàm số $f'(x)$ trên cùng một hệ trục
		}
		{
			\begin{tikzpicture}[line join=round, line cap=round,>=stealth,thick,scale=.8]
				\tikzset{label style/.style={font=\footnotesize}}
				\draw[->] (-2.1,0)--(5.1,0) node[below left] {$x$};
				\draw[->] (0,-3.1)--(0,4.1) node[below left] {$y$};
				\draw (0,0) node [below left] {$O$};
				\foreach \x in {1,2,3}
				\draw[thin] (\x,1pt)--(\x,-1pt) node [below] {$\x$};
				\draw[thin](-1,1pt)--(1,-1pt);
				\foreach \y in {-2,2}
				\draw[thin] (1pt,\y)--(-1pt,\y) node [above right] {$\y$};
				%\begin{scope}
				\clip (-2,-3) rectangle (5,4);
				\draw[samples=200,domain=-2:4,smooth,variable=\x] plot (\x,{(\x)^3-3*(\x)^2+2});
				\draw[samples=200,domain=-2:4,smooth,variable=\x] plot (\x,{-(\x)^2+(\x)});
				%\end{scope}
				\draw[dashed] (-1,0) node[above left]{$-1$}--(-1,-2)--(0,-2);
				\draw[dashed] (3,0)--(3,2)--(0,2);
				%\begin{scope}[on background layer]\path[white]node{MDD-134};\end{scope}
			\end{tikzpicture}
		}
		Dựa vào đồ thị ta thấy
		$
		f'(t)>-t^2+t \Leftrightarrow\hoac{&- 1 < t < 1\\
			&t > 2}\Rightarrow \hoac{&
			- 1 < 3 x + 1 < 1\\
			&3 x + 1 > 2} \Leftrightarrow \hoac{&
			\dfrac{-2}{3}<x<0\\
			&x>\dfrac{1}{3}.}
		$}
\end{ex}
\begin{ex}[Sở Phú Thọ-2020]%[2D1G1-2]
	\immini{
		Cho hàm số $y=f(x)$ có đồ thị hàm số $y=f'(x)$ như hình vẽ.\\
		Hàm số $g(x)=f\left(\mathrm{e}^x-2\right)-2020$ nghịch biến trên khoảng nào dưới đây?
		\choice
		{\True $\left(-1 ; \dfrac{3}{2}\right)$}
		{$(-1 ; 2)$}
		{$(0 ;+\infty)$}
		{$\left(\dfrac{3}{2}; 2\right)$}
	}
	{
		\begin{tikzpicture}[scale=0.7,>=stealth, font=\footnotesize, line join=round, line cap=round]
			\def\a{1} \def\b{-3} \def\c{0} \def\d{0} % Hệ số
			\def\xmin{-2} \def\xmax{4}
			\def\ymin{-5} \def\ymax{2} 
			%\draw[color=gray!50,dashed] (\xmin,\ymin) grid (\xmax,\ymax); 
			\draw[->] (\xmin,0)--(\xmax,0) node [below]{$x$};
			\draw[->] (0,\ymin)--(0,\ymax) node [left]{$y$};
			\node at (0,0) [above left]{$O$};
			\node at (3,0) [below right]{$3$};
			\draw[dashed] (2,0) node[above]{$2$}--(2,-4) --(0,-4) node[left]{$-4$};
			\clip (\xmin+0.1,\ymin+0.1) rectangle (\xmax-0.5,\ymax-0.1);
			\draw[smooth,samples=300] plot(\x,{\a*(\x)^3+\b*(\x)^2+\c*(\x)+\d});
		\end{tikzpicture}
	}
	
	\loigiai{
		Dựa vào đồ thị hàm số $y=f'(x)$ suy ra $f'(x) \leq 0 ~ \forall x<3$ và $f'(x)>0 ~ \forall x>3$.
		$
		g'(x)=\mathrm{e}^x f'\left(\mathrm{e}^x-2\right) .
		$
		Hàm số $g(x)=f\left(\mathrm{e}^x-2\right)-2020$ nghịch biến \\ $ \Leftrightarrow g'(x)<0 \Leftrightarrow \mathrm{e}^x f'\left(\mathrm{e}^x-2\right)<0$\\
		$
		\Leftrightarrow f'\left(\mathrm{e}^x-2\right)<0 \Leftrightarrow \mathrm{e}^x-2<3 \Leftrightarrow \mathrm{e}^x<5 \Leftrightarrow x<\ln 5 .
		$\\
		Vậy hàm số đã cho nghịch biến trên $\left(-1 ; \dfrac{3}{2}\right)$.
	}
\end{ex}
\begin{ex}[Lý Nhân Tông - Bắc Ninh - 2020]%[2D1G1-2]
	\immini{
		Cho hàm số $f(x)$ có đồ thị hàm số $f'(x)$ như hình vẽ.\\
		Hàm số $y=f(\cos x)+x^2-x$ đồng biến trên khoảng
		\choice
		{$(-2 ; 1)$}
		{$(0 ; 1)$}
		{\True $(1 ; 2)$}
		{$(-1 ; 0)$}
	}
	{
		\begin{tikzpicture}[scale=1,>=stealth, font=\footnotesize, line join=round, line cap=round]
			\def\a{-0.5} \def\b{0} \def\c{1.5} \def\d{0} % Hệ số
			\def\xmin{-3} \def\xmax{4}
			\def\ymin{-2} \def\ymax{2} 
			%\draw[color=gray!50,dashed] (\xmin,\ymin) grid (\xmax,\ymax); 
			\draw[->] (\xmin,0)--(\xmax,0) node [below]{$x$};
			\draw[->] (0,\ymin)--(0,\ymax) node [left]{$y$};
			\node at (0,0) [above left]{$O$};
			\node at (3,0) [below right]{$3$};
			\draw[dashed] (-2,0) node[below]{$-2$}--(-2,1) --(0,1) node[above right]{$1$} --(1,1)--(1,0) node[below]{$1$};
			\draw[dashed] (-1,0) node[below right]{$-1$}--(-1,-1) --(0,-1) node[above right]{$-1$} --(2,-1)--(2,0) node[below right]{$2$};
			\clip (\xmin+0.1,\ymin+0.1) rectangle (\xmax-0.5,\ymax-0.1);
			\draw[smooth,samples=300][domain=-2:2] plot(\x,{\a*(\x)^3+\b*(\x)^2+\c*(\x)+\d});
		\end{tikzpicture}
	}
	\loigiai{
		Đặt  $g(x)=f(\cos x)+x^2-x$.\\
		Ta có $g'(x)=-\sin x \cdot f'(\cos x)+2 x-1$\\
		Vì $\cos x \in[-1 ; 1]$ nên từ đồ thị $f'(x)$ ta suy ra $f'(\cos x) \in[-1 ; 1]$.\\
		Do đó $\left|-\sin x \cdot f'(\cos x)\right| \leq 1, ~\forall x \in \mathbb{R}$.\\
		Ta suy ra $g'(x)=\sin x \cdot f'(\cos x)+2 x-1 \geq-1+2 x-1=2 x-2$
		$\Rightarrow g'(x)>0, ~\forall x>1$.\\
		Vậy hàm số đồng biến trên $(1 ; 2)$.
	}
\end{ex}
\begin{ex}[THPT Nguyễn Viết Xuân - 2020]%[2D1G1-2]
	\immini{
		Cho hàm số $f(x)$. Hàm số $y=f'(x)$ có đồ thị như hình vẽ.\\
		Hàm số $g(x)=f\left(3 x^2-1\right)-\dfrac{9}{2}x^4+3 x^2$ đồng biến trên khoảng nào dưới đây?
		\choice
		{\True $\left(-\dfrac{2 \sqrt{3}}{3}; \dfrac{-\sqrt{3}}{3}\right)$}
		{$\left(0 ; \dfrac{2 \sqrt{3}}{3}\right)$}
		{$(1 ; 2)$}
		{$\left(-\dfrac{\sqrt{3}}{3}; \dfrac{\sqrt{3}}{3}\right)$} 
	}
	{
		\begin{tikzpicture}[scale=0.6,>=stealth, font=\footnotesize, line join=round, line cap=round]
			\def\a{0.25} \def\b{0.25} \def\c{-2} \def\d{0} % Hệ số
			\def\xmin{-5} \def\xmax{4}
			\def\ymin{-5} \def\ymax{5} 
			%\draw[color=gray!50,dashed] (\xmin,\ymin) grid (\xmax,\ymax); 
			\draw[->] (\xmin,0)--(\xmax,0) node [below]{$x$};
			\draw[->] (0,\ymin)--(0,\ymax) node [left]{$y$};
			\node at (0,0) [above left]{$O$};
			%\node at (3,0) [below right]{$3$};
			\draw[dashed] (-4,0) node[below left]{$-4$}--(-4,-4) --(0,-4) node[above right]{$-4$};
			\draw[dashed] (3,0) node[below right]{$3$}--(3,3) --(0,3) node[above right]{$3$};
			\clip (\xmin+0.1,\ymin+0.1) rectangle (\xmax-0.5,\ymax-0.1);
			\draw[smooth,samples=300] plot(\x,{\a*(\x)^3+\b*(\x)^2+\c*(\x)+\d});
		\end{tikzpicture}
	}
	
	\loigiai
	{
		TXĐ: $\mathscr{D}=\mathbb{R}$.\\
		Ta có $g'(x)=6 x f'\left(3 x^2-1\right)-18 x^3+6 x=6 x\left[f'\left(3 x^2-1\right)-3 x^2+1\right]$.\\
		$
		g'(x)=0 \Leftrightarrow\hoac{
			&x = 0\\
			&f '( 3 x ^{2}- 1 ) = 3 x ^{2}- 1}
		\Leftrightarrow \hoac{
			&x = 0\\
			&3 x ^{2}- 1 = - 4 \text{~(vô nghiệm)}\\
			&3 x ^{2}- 1 = 0\\
			&3 x ^{2}- 1 = 3}\Leftrightarrow \hoac{&x=0 \\
			&x=\pm \dfrac{\sqrt{3}}{3}\\
			&x=\pm \dfrac{2 \sqrt{3}}{3}.}
		$\\
		Bảng xét dấu
		\begin{center}
			\begin{tikzpicture}
				\tkzTabInit[nocadre,lgt=1.2,espcl=2.2,deltacl=0.6]
				{$x$ /1.2,$f'(x)$ /0.7}
				{$-\infty$,$-\dfrac{2 \sqrt{3}}{3}$,$-\dfrac{ \sqrt{3}}{3}$,$0$,$\dfrac{\sqrt{3}}{3}$,$\dfrac{2 \sqrt{3}}{3}$,$+\infty$}
				\tkzTabLine{,-,$0$,+,$0$,-,$0$,+,$0$,-,$0$,+,}
			\end{tikzpicture}
		\end{center}
		Vậy hàm số đồng biến trong khoảng $\left(-\dfrac{2 \sqrt{3}}{3}; \dfrac{-\sqrt{3}}{3}\right)$.}
\end{ex}
\begin{ex}[Trần Phú - Quảng Ninh - 2020]%[2D1G1-2]
	Cho hàm số $f(x)$ có bảng xét dấu của đạo hàm như sau
	\begin{center}
		\begin{tikzpicture}
			\tkzTabInit[nocadre,lgt=1.2,espcl=2,deltacl=0.6]
			{$x$ /0.6,$f'(x)$ /0.6}
			{$-\infty$,$-4$,$-1$,$2$,$7$,$+\infty$}
			\tkzTabLine{,+,$0$,-,$0$,+,$0$,-,$0$,+,}
		\end{tikzpicture}
	\end{center}
	Hàm số $y=f(2 x+1)+\dfrac{2}{3}x^3-8 x+5$ nghịch biến trên khoảng nào dưới đây?
	\choice
	{$(-\infty ;-2)$}
	{$(1 ;+\infty)$}
	{$(-1 ; 7)$}
	{\True $\left(-1 ; \dfrac{1}{2}\right)$}
	\loigiai{
		Ta có $y'=2 f'(2 x+1)+2 x^2-8$.\\
		Xét $y'\leq 0 \Leftrightarrow 2 f'(2 x+1)+2 x^2-8 \leq 0 \Leftrightarrow f'(2 x+1) \leq 4-x^2$.\\
		Đặt $t=2x+1$, ta có $f'(t) \leq \dfrac{-t^2+2 t+15}{4}$.\\
		Vì $\dfrac{-t^2+2 t+15}{4}\geq 0, \forall t \in[-3 ; 5]$.\\
		Mà $f'(t) \leq 0, \forall t \in[-3 ; 2]$.\\
		Nên $f'(t) \leq \dfrac{-t^2+2 t+15}{4}\Rightarrow t \in[-3 ; 2]$.\\
		Suy ra $-3 \leq 2 x+1 \leq 2 \Leftrightarrow-2 \leq x \leq \dfrac{1}{2}$.}
\end{ex}

\begin{ex}[Chuyên Thái Bình - Lần 3 - 2020]%[2D1G1-2]
	\immini{
		Cho hàm số $y=f(x)$ liên tục trên $\mathbb{R}$ có đồ thị hàm số $y=f'(x)$ cho như hình vẽ.\\
		Hàm số $g(x)=2 f(|x-1|)-x^2+2 x+2020$ đồng biến trên khoảng nào?
		\choice
		{\True $(0 ; 1)$}
		{$(-3 ; 1)$}
		{$(1 ; 3)$}
		{$(-2 ; 0)$}
	}
	{
		\begin{tikzpicture}[scale=0.7,>=stealth, font=\footnotesize, line join=round, line cap=round]
			\def\a{-0.333} \def\b{1} \def\c{1.333} \def\d{-1} % Hệ số
			\def\xmin{-3} \def\xmax{5}
			\def\ymin{-3} \def\ymax{5} 
			%\draw[color=gray!50,dashed] (\xmin,\ymin) grid (\xmax,\ymax); 
			\draw[->] (\xmin,0)--(\xmax,0) node [below]{$x$};
			\draw[->] (0,\ymin)--(0,\ymax) node [left]{$y$};
			\node at (0,0) [above left]{$O$};
			%\node at (3,0) [below right]{$3$};
			\draw[dashed] (-1,0) node[above]{$-1$}--(-1,-1) --(0,-1) node[above right]{$-1$};
			\draw[dashed] (1,0) node[below right]{$1$}--(1,1) --(0,1) node[above right]{$1$};
			\draw[dashed] (3,0) node[below right]{$3$}--(3,3) --(0,3) node[above right]{$3$};
			\clip (\xmin+0.1,\ymin+0.1) rectangle (\xmax-0.5,\ymax-0.1);
			\draw[smooth,samples=300] plot(\x,{\a*(\x)^3+\b*(\x)^2+\c*(\x)+\d});
			\draw[smooth,samples=300] plot(\x,{(\x)});
		\end{tikzpicture}
	}
	\loigiai{
		Với $x>1$, ta có $g(x)=2 f(x-1)-(x-1)^2+2021 \Rightarrow g'(x)=2 f'(x-1)-2(x-1)$.\\
		Hàm số đồng biến $\Leftrightarrow 2 f'(x-1)-2(x-1)>0 \Leftrightarrow f'(x-1)>x-1 \quad(*)$.\\
		Đặt $t=x-1$, khi đó $(*) \Leftrightarrow f'(t)>t \Leftrightarrow\hoac{&1<t<3 \\ &t<-1}\Rightarrow\hoac{&2<x<4 \\ &x<0 ~(\text{loại}).}$\\
		Với $x<1$, ta có $g(x)=2 f(1-x)-(1-x)^2+2021 \Rightarrow g'(x)=-2 f'(1-x)+2(1-x)$.\\
		Hàm số đồng biến $\Leftrightarrow-2 f'(1-x)+2(1-x)>0 \Leftrightarrow f'(1-x)<1-x \quad(* *)$.\\
		Đặt $t=1-x$, khi đó $(* *) \Leftrightarrow f'(t)<t \Leftrightarrow\hoac{&-1<t<1 \\ &t>3}\Rightarrow\hoac{&0<x<2 \\ &x<-2}\Rightarrow\hoac{&0<x<1 \\ &x<-2.}$\\
		Vậy hàm số $g(x)$ đồng biến trên các khoảng $(-\infty ;-2),(0 ; 1),(2 ; 4)$.
	}
\end{ex}
\begin{ex}[Sở Phú Thọ - 2020]%[2D1G1-2]
	\immini{
		Cho hàm số $y=f(x)$ có đồ thị hàm số $f'(x)$ như hình vẽ.\\
		Hàm số $g(x)=f\left(1+e^x\right)+2020$ nghịch biến trên khoảng nào dưới đây?
		\choice
		{$(0 ;+\infty)$}
		{$\left(\dfrac{1}{2}; 1\right)$}
		{\True $\left(0 ; \dfrac{1}{2}\right)$}
		{$(-1 ; 1)$}
	}{
		\begin{tikzpicture}[scale=0.7,>=stealth, font=\footnotesize, line join=round, line cap=round]
			\def\a{1} \def\b{-3} \def\c{0} \def\d{0} % Hệ số
			\def\xmin{-2} \def\xmax{4}
			\def\ymin{-5} \def\ymax{2} 
			%\draw[color=gray!50,dashed] (\xmin,\ymin) grid (\xmax,\ymax); 
			\draw[->] (\xmin,0)--(\xmax,0) node [below]{$x$};
			\draw[->] (0,\ymin)--(0,\ymax) node [left]{$y$};
			\node at (0,0) [above left]{$O$};
			\node at (3,0) [below right]{$3$};
			\draw[dashed] (2,0) node[above]{$2$}--(2,-4) --(0,-4) node[left]{$-4$};
			\clip (\xmin+0.1,\ymin+0.1) rectangle (\xmax-0.5,\ymax-0.1);
			\draw[smooth,samples=300] plot(\x,{\a*(\x)^3+\b*(\x)^2+\c*(\x)+\d});
		\end{tikzpicture}
	}
	\loigiai{
		$g'(x)=e^x f'\left(1+e^x\right)$.\\
		Do $e^x>0, \forall x$ nên $g'(x) \leq 0 \Leftrightarrow f'\left(1+e^x\right) \leq 0 \Leftrightarrow 1+e^x \leq 3 \Leftrightarrow x \leq \ln 2$, dấu bằng xảy ra tại hữu hạn điểm.\\
		Nên $g(x)$ nghịch biến trên $(-\infty ; \ln 2)$.\\
		Vì $\left(0 ; \dfrac{1}{2}\right) \subset (-\infty ; \ln 2)$ nên hàm số đã cho nghịch biến trên $\left(0 ; \dfrac{1}{2}\right)$.
	}
\end{ex}

\begin{ex}%[2D1K1-2]
	[THPT Anh Sơn - Nghệ An - 2020]
	Cho hàm số $y=f(x)$ có bảng xét dấu của đạo hàm như sau.
	\begin{center}
		\begin{tikzpicture}
			\tkzTabInit[nocadre,lgt=1.2,espcl=2,deltacl=0.6]
			{$x$ /0.6,$f'(x)$ /0.6}
			{$-\infty$,$-2$,$-1$,$2$,$4$,$+\infty$}
			\tkzTabLine{,+,$0$,-,$0$,+,$0$,-,$0$,+,}
		\end{tikzpicture}
	\end{center}
	Hàm số $y=-2 f(x)+2019$ nghịch biến trên khoảng nào trong các khoảng dưới đây?
	\choice
	{$(2 ; 4)$}
	{$(-4 ; 2)$}
	{$(-2 ;-1)$}
	{\True $(-1 ; 2)$}
	\loigiai{
		Ta có $y'=-2 f'(x)$.\\
		$
		y'=0 \Leftrightarrow-2 f'(x)=0 \Leftrightarrow\hoac{&
			x=-2 \\
			&x=-1 \\
			&x=2 \\
			&x=4.}$\\
		Từ bảng xét dấu của $f'(x)$ ta có
		\begin{center}
			\begin{tikzpicture}
				\tkzTabInit[nocadre,lgt=1,espcl=2,deltacl=0.6]
				{$x$ /0.6,$y'$ /0.6}
				{$-\infty$,$-2$,$-1$,$2$,$4$,$+\infty$}
				\tkzTabLine{,-,$0$,+,$0$,-,$0$,+,$0$,-,}
			\end{tikzpicture}
		\end{center}
		Từ bảng xét dấu ta có hàm số nghịch biến trên khoảng $(-\infty ;-2),(-1 ; 2)$ và $(4 ;+\infty)$.}
\end{ex}

\begin{ex}[THPT Anh Sơn - Nghệ An - 2020]%[2D1G1-2]
	Cho hàm số $f(x)$ xác định và liên tục trên $\mathbb{R}$ và có đạo hàm $f'(x)$ thỏa mãn $f'(x)=(1-x)(x+2) g(x)+2019$ với $g(x)<0, ~\forall x \in \mathbb{R}$ . Hàm số $y=f(1-x)+2019 x+2020$ nghịch biến trên khoảng nào?
	\choice
	{$(1 ;+\infty)$}
	{$(0 ; 3)$}
	{$(-\infty ; 3)$}
	{\True $(3 ;+\infty)$}
	\loigiai{
		Đặt $h(x)=f(1-x)+2019 x+2020$.\\
		Vì hàm số $f(x)$ xác định trên $\mathbb{R}$ nên hàm số $h(x)$ cũng xác định trên $\mathbb{R}$.\\
		Ta có $h'(x)=-f'(1-x)+2019$.\\
		Do $h'(x)=0$ tại hữu hạn điểm nên để tìm khoảng nghịch biến của hàm số $h(x)$, ta tìm các giá trị của $x$ sao cho $h'(x)<0 \Leftrightarrow-f'(1-x)+2019<0$\\ 
		$\Leftrightarrow f'(1-x)-2019>0 \\
		\Leftrightarrow x(3-x) g(1-x)>0 \Leftrightarrow x(3-x)<0(\text{~do~}g(x)<0, \forall x \in \mathbb{R})$\\
		$\Leftrightarrow\hoac{&
			x<0 \\
			&x>3.}$\\
		Vậy hàm số $y=f(1-x)+2019 x+2020$ nghịch biến trên các khoảng $(-\infty ; 0)$ và $(3 ;+\infty).$}
\end{ex}

\begin{ex}%[2D1G1-2]
	Cho hàm số $y=f(x)$ xác định trên $\mathbb{R}$ và có bảng xét dấu đạo hàm như sau:
	\begin{center}
		\begin{tikzpicture}
			\tkzTabInit[nocadre,lgt=2,espcl=2,deltacl=0.6]
			{$x$ /0.6,$f'(x)$ /0.6}
			{$-\infty$,$-1$,$1$,$4$,$+\infty$}
			\tkzTabLine{,-,$0$,+,$0$,-,$0$,+,}
		\end{tikzpicture}
	\end{center}
	Biết $f(x)>2,~ \forall x \in \mathbb{R}$. Xét hàm số $g(x)=f(3-2 f(x))-x^3+3 x^2-2020$. Khẳng định nào sau đây đúng?
	\choice
	{Hàm số $g(x)$ đồng biến trên khoảng $(-2 ;-1)$}
	{Hàm số $g(x)$ nghịch biến trên khoảng $(0 ; 1)$}
	{Hàm số $g(x)$ đồng biến trên khoảng $(3 ; 4)$}
	{\True Hàm số $g(x)$ nghịch biến trên khoảng $(2 ; 3)$}
	\loigiai{
		Ta có $g'(x)=-2 f'(x) f'(3-2 f(x))-3 x^2+6 x$.\\
		Vì $f(x)>2, ~\forall x \in \mathbb{R}$ nên $3-2 f(x)<-1 ~\forall x \in \mathbb{R}$.\\
		Từ bảng xét dấu $f'(x)$ suy ra $f'(3-2 f(x))<0, ~\forall x \in \mathbb{R}$.\\
		Từ đó ta có bảng xét dấu sau:
		\begin{center}
			\begin{tikzpicture}
				\tkzTabInit[nocadre,lgt=4,espcl=1.7,deltacl=0.6]
				{$x$ /0.7,$-f'(x)f'\left( 3-2f(x)\right) $/0.8,$-3x^2+6x$/0.7}
				{$-\infty$,$-1$,$0$,$1$,$2$,$4$,$+\infty$}
				\tkzTabLine{,-,$0$,+,|,+,$0$,-,|,-,$0$,+,}
				\tkzTabLine{,-,|,-,$0$,+,|,+,$0$,-,|,-,}
			\end{tikzpicture}
		\end{center}
		Từ bảng xét dấu trên, loại trừ đáp án suy ra hàm số $g(x)$ nghịch biến trên khoảng $(2 ; 3)$.}
\end{ex}

\begin{ex}%[2D1G1-2]
	Cho hàm số $f(x)$ có bảng biến thiên như sau:
	\begin{center}
		\begin{tikzpicture}
			\tkzTabInit[nocadre,lgt=1.2,espcl=2.5,deltacl=0.6]
			{$x$ /0.7, $f'(x)$ /0.7, $f(x)$ /2.5}
			{$-\infty$,$1$,$2$,$3$,$4$,$+\infty$}
			\tkzTabLine{,+,$0$,-,$0$,+,$0$,-,$0$,+,}
			\tkzTabVar{-/$-\infty$,+/$3$,-/$1$,+/$2$,-/$0$,+/$+\infty$}
		\end{tikzpicture}
	\end{center}
	Hàm số $y=(f(x))^3-3 .(f(x))^2$ nghịch biến trên khoảng nào dưới đây?
	\choice
	{$(1 ; 2)$}
	{$(3 ; 4)$}
	{$(-\infty ; 1)$}
	{\True $(2 ; 3)$}
	\loigiai{
		Ta có $y'=3 \cdot(f(x))^2 \cdot f'(x)-6 \cdot f(x) \cdot f'(x)=3 f(x) \cdot f'(x) \cdot[f(x)-2]. \\
		y'=0 \Leftrightarrow \hoac{&f(x)=0 \Leftrightarrow x \in\left\{x_1, 4 \mid x_1<1\right\}\\
			&f(x)=2 \Leftrightarrow x \in\left\{x_2, x_3, 3, x_4 \mid x_1<x_2<1<x_3<2 ; 4<x_4\right\}\\
			&f'(x)=0 \Leftrightarrow x \in\{1,2,3,4\}.}$\\
		Lập bảng xét dấu ta có
		\begin{center}
			\begin{tikzpicture}
				\tkzTabInit[nocadre,lgt=2,espcl=1.5,deltacl=0.6]
				{$x$ /0.7,$f(x)$ /0.7,$f(x)-2$ /0.7,$f'(x)$/0.7,$y'$/0.7}
				{$-\infty$,$x_1$,$x_2$,$1$,$x_3$,$2$,$3$,$4$,$x_4$,$+\infty$}
				\tkzTabLine{,-,$0$,+,|,+,|,+,|,+,|,+,$0$,+,|,+,|,+,}
				\tkzTabLine{,-,|,-,$0$,+,$0$,+,$0$,-,|,-,$0$,-,|,-,$0$,+}
				\tkzTabLine{,+,|,+,|,+,$0$,-,|,-,$0$,+,$0$,-,$0$,+,|,+}
				\tkzTabLine{,+,$0$,-,$0$,+,$0$,-,$0$,+,$0$,-,$0$,+,$0$,-,$0$,+}
			\end{tikzpicture}
		\end{center}
		
		Do đó hàm số nghịch biến trên khoảng $(2 ; 3)$.
	}
\end{ex}
\begin{ex}%[2D1G1-2]
	Cho hàm số $y=f(x)$ có đồ thị nằm trên trục hoành và có đạo hàm trên $\mathbb{R}$, bảng xét dấu của biểu thức $f'(x)$ như bảng dưới đây.
	\begin{center}
		\begin{tikzpicture}
			\tkzTabInit[nocadre,lgt=1.2,espcl=2,deltacl=0.6]
			{$x$ /0.6,$f'(x)$ /0.6}
			{$-\infty$,$-2$,$-1$,$3$,$+\infty$}
			\tkzTabLine{,-,$0$,+,$0$,-,$0$,+,}
		\end{tikzpicture}
	\end{center}
	Hàm số $y=g(x)=\dfrac{f\left(x^2-2 x\right)}{f\left(x^2-2 x\right)+1}$ nghịch biến trên khoảng nào dưới đây?
	\choice
	{$(-\infty ; 1)$}
	{$\left(-2 ; \dfrac{5}{2}\right)$}
	{\True $(1 ; 3)$}
	{$(2 ;+\infty)$}
	\loigiai{
		$ g'(x)=\dfrac{\left(x^2-2 x\right)'\cdot f'\left(x^2-2 x\right)}{\left(f\left(x^2-2 x\right)+1\right)^2}=\dfrac{(2 x-2) \cdot f'\left(x^2-2 x\right)}{\left(f\left(x^2-2 x\right)+1\right)^2}. \\
		g'(x)=0 \Leftrightarrow\hoac{
			&2 x - 2 = 0\\
			&f '( x ^{2}- 2 x ) = 0}
		\Leftrightarrow \hoac{&x = 1\\
			&x ^{2}- 2 x = - 2\\
			&x ^{2}- 2 x = - 1\\
			&x ^{2}- 2 x = 3}
		\Leftrightarrow \hoac{&x=1 \\
			&x=-1 \\
			&x=3.}
		$\\
		Ta có bảng xét dấu của $g'(x)$
		\begin{center}
			\begin{tikzpicture}
				\tkzTabInit[nocadre,lgt=1.2,espcl=2,deltacl=0.6]
				{$x$ /0.6,$g'(x)$ /0.6}
				{$-\infty$,$-1$,$1$,$3$,$+\infty$}
				\tkzTabLine{,-,$0$,+,$0$,-,$0$,+,}
			\end{tikzpicture}
		\end{center}
		Dựa vào bảng xét dấu ta có hàm số $y=g(x)$ nghịch biến trên các khoảng $(-\infty ;-1)$ và $(1 ; 3)$.}
\end{ex}
\begin{ex}[Liên trường huyện Quảng Xương - Thanh Hóa - 2021]%[2D1G1-2]
	\immini{
		Cho các hàm số $y=f(x)$; $y=g(x)$ liên tục trên $\mathbb{R}$ và có đồ thị các đạo hàm $f'(x) ; g'(x)$ (đồ thị hàm số $y=g'(x)$ là đường đậm hơn) như hình vẽ.\\
		Hàm số $h(x)=f(x-1)-g(x-1)$ nghịch biến trên khoảng nào dưới đây?
		\choice
		{$\left(\dfrac{1}{2}; 1\right)$}
		{$(1 ;+\infty)$}
		{$(2 ;+\infty)$}
		{\True $\left(-1 ; \dfrac{1}{2}\right)$}
	}
	{
		\begin{tikzpicture}[scale=1,>=stealth, font=\footnotesize, line join=round, line cap=round]
			%\def\a{1} \def\b{-6} \def\c{9} \def\d{1} % Hệ số
			\def\xmin{-4} \def\xmax{3}
			\def\ymin{-2} \def\ymax{4} 
			%\draw[color=gray!50,dashed] (\xmin,\ymin) grid (\xmax,\ymax); 
			\draw[->] (\xmin,0)--(\xmax,0) node [below]{$x$};
			\draw[->] (0,\ymin)--(0,\ymax) node [left]{$y$};
			\node at (0,0) [above left]{$O$};
			\node at (1,3) [below left]{$f'(x)$};
			\node at (1.5,3) [below right]{$g'(x)$};
			\draw[dashed] (-2,0) node[above right]{$-2$}--(-2,1);
			\draw[dashed] (1,0) node[below]{$1$}--(1,1);
			\draw[dashed] (-0.5,0) node[below]{$-0{,}5$}--(-0.5,2.125);
			\clip (\xmin+0.1,\ymin+0.1) rectangle (\xmax-0.5,\ymax-0.1);
			\draw[smooth,samples=300][domain=-3:2] plot(\x,{2*(\x)^4+4*(\x)^3-2*(\x)^2-4*(\x)+1});
			\draw[smooth,samples=300,line width=1.2pt] plot(\x,{(\x)^3+(\x)^2-2*(\x)+1});
		\end{tikzpicture}
	}
	
	\loigiai{
		Ta có: $h'(x)=f'(x-1)-g'(x-1)$.\\
		Dựa vào hình vẽ ta có hàm số $h(x)$ nghịch biến\\
		$\Leftrightarrow h'(x)<0 \Leftrightarrow f'(x-1)<g'(x-1)$\\
		$
		\Leftrightarrow\hoac{&- 2 < x - 1 < - \dfrac{1}{2}\\
			&0 < x - 1 < 1}
		\Leftrightarrow \hoac{
			&-1<x<\dfrac{1}{2}\\
			&1<x<2.}$\\
		Do đó hàm số $h(x)$ nghịch biến trên các khoảng $\left(-1 ; \dfrac{1}{2}\right)$ và $(1 ; 2)$.
	}
\end{ex}
\begin{ex}[THPT Quế Võ 1 - Bắc Ninh - 2021] %[2D1G1-2]
	\immini{
		Cho ba hàm số $y=f(x), y=g(x), y=h(x)$. Đồ thị của ba hàm số $y=f'(x), y=g'(x), y=h'(x)$ được cho như hình vẽ.\\
		Hàm số $k(x)=f(x+7)+g(5 x+1)-h\left(4 x+\dfrac{3}{2}\right)$ đồng biến trên khoảng nào dưới đây?
		\choice
		{$\left(-\dfrac{5}{8}; 0\right)$}
		{$\left(\dfrac{5}{8};+\infty\right)$}
		{\True $\left(\dfrac{3}{8}; 1\right)$}
		{$\left(-\dfrac{3}{8}; 1\right)$}
	}
	{
		\begin{tikzpicture}[scale=0.25,>=stealth, font=\footnotesize, line join=round, line cap=round]
			\def\a{-.078} \def\b{1.25} \def\c{0} % Hệ số
			\def\xmin{-4} \def\xmax{25}
			\def\ymin{-8} \def\ymax{18}
			
			%\draw[color=gray!50,dashed] (\xmin,\ymin) grid (\xmax,\ymax);
			
			\draw[->] (\xmin,0)--(\xmax,0) node [below]{$x$};
			\draw[->] (0,\ymin)--(0,\ymax) node [left]{$y$};
			\node at (20,14) [below right]{$y=g'(x)$};
			\node at (18,-2) [below left]{$y=h'(x)$};
			\node at (16,5) [below right]{$y=f'(x)$};
			\node at (0,0) [below left]{$O$};
			\draw[dashed] (3,0) node[below]{$3$}--(3,10)--(0,10) node[left]{$10$};
			\draw[dashed] (8,0) node[below]{$8$}--(8,5)--(0,5) node[left]{$5$};
			\draw[dashed] (4,0) node[below]{$4$}--(4,2)--(0,2) node[left]{$2$};
			\clip (\xmin+0.1,\ymin+0.1) rectangle (\xmax-0.5,\ymax-0.1);
			\draw[smooth,samples=300,domain=-2:18] plot(\x,{\a*(\x)^2+\b*(\x)+\c});
			%\draw[smooth,samples=300,domain=-2:25] plot(\x,{0.02*(\x)^3-0.6*(\x)^2+5.16*(\x)});
			\draw[line width=1.2pt] (-2,5)..controls (1.7,1.5) and (4.5,1.6)..(7,2.6);
			\draw[line width=1.2pt] (7,2.6)..controls (9,3.5) and (12,5)..(20,13);
			\draw (-0.5,-2) -- (0,0)--(3,10).. controls +(65:1) and + (-190:1)..(6,15).. controls +(0:1) and + (-180:1)..(14,-1).. controls +(0:1) and + (+80:1)..(19,16);
			
		\end{tikzpicture}
	}
	\loigiai{
		Ta có $k'(x)=f'(x+7)+5 g'(5 x+1)-4 h'\left(4 x+\dfrac{3}{2}\right)$.\\
		Khi $x \in \left( \dfrac{3}{8};1\right)$ thì $\heva{&7{,}375<x+7<8\\&2{,}875<5x+1<6\\&3<4x+\dfrac{4}{3}<5{,}5}\Leftrightarrow \heva{&f'(x+7)>10\\&g'(5x+1)>2 \Rightarrow 5g'(5x+1)>10  \\&h'\left( 4x+\dfrac{3}{2}\right)<5 \Rightarrow -4h'\left( 4x+\dfrac{3}{2}\right) >-20}.$\\
		Do đó $k'(x)=f'(x+7)+5g'(5x+1)-4h'\left( 4x+\dfrac{3}{2}\right)>0$.\\
		Hàm số $k(x)=f(x+7)+g(5 x+1)-h\left(4 x+\dfrac{3}{2}\right)$ đồng biến trên $\left(\dfrac{3}{8}; 1\right)$.
	}
\end{ex}
\begin{ex}[THPT Thanh Chương 1 - Nghệ An- 2021] %[2D1G1-2]
	Cho hàm số $y=f(x)$ liên tục trên $\mathbb{R}$ có bảng xét dấu đạo hàm như sau
	\begin{center}
		\begin{tikzpicture}
			\tkzTabInit[nocadre,lgt=1.2,espcl=2,deltacl=0.6]
			{$x$ /0.6,$f'(x)$ /0.6}
			{$-\infty$,$1$,$2$,$3$,$4$,$+\infty$}
			\tkzTabLine{,-,$0$,+,$0$,+,$0$,-,$0$,+,}
		\end{tikzpicture}
	\end{center}
	Hàm số $y=3f(2x-1)-4x^3+15x^2-18x+1$ đồng biến trên khoảng nào dưới đây?
	\choice
	{$\left(3;+\infty\right)$}
	{\True $\left(1;\dfrac{3}{2}\right)$}
	{$\left(\dfrac{5}{2}; 3\right)$}
	{$\left(2;\dfrac{5}{2}\right)$}
	\loigiai{
		Ta có $y'=6f'(2x-1)-12x^2+30x-18=6\left[f'(2x-1)-2x^2+5x-3\right] $.\\
		Có $f'(2x-1)=0 \Leftrightarrow \hoac{&2x-1=1\\&2x-1=2\\&2x-1=3\\&2x-1=4} \Leftrightarrow \hoac{&x=1\\&x=\dfrac{3}{2}\\&x=2\\&x=\dfrac{5}{2}.}$
		Ta có bảng xét dấu sau
		\begin{center}
			\begin{tikzpicture}
				\tkzTabInit[nocadre,lgt=3.0,espcl=1.5,deltacl=0.6]
				{$x$ /1.0,$f(x)$ /0.6,$f'(2x-1)$ /0.6,$-2x^2+5x-3$/0.6,$g'(x)$/0.6}
				{$-\infty$,$1$,$\dfrac{3}{2}$,$2$,$\dfrac{5}{2}$,$3$,$4$,$+\infty$}
				\tkzTabLine{,-,$0$,+,|,+,$0$,+,|,+,$0$,-,$0$,+,}
				\tkzTabLine{,-,$0$,+,$0$,+,$0$,-,$0$,+,|,+,|,+,}
				\tkzTabLine{,-,$0$,+,$0$,-,|,-,|,-,|,-,|,-,}
				\tkzTabLine{,-,$0$,+,$0$,,?,,|,,?,?,,?,}
			\end{tikzpicture}
		\end{center}
		Dựa vào bảng xét dấu trên, ta kết luận hàm số đã cho đồng biến trên khoảng $\left( 1; \dfrac{3}{2}\right).$
	}
\end{ex}


\begin{ex}%[2D2G4-3] %Câu 27 
	[THPT Hoàng Hoa Thám-Đà Nẵng-2021]
	Cho hàm số $f(x)$ có bảng xét dấu của $f'(x)$ như sau:\\
	\begin{center}
		\begin{tikzpicture}
			\tkzTabInit[lgt=1.2,espcl=2.3]
			{$x$/0.7, $f'(x)$ /.8} % first column
			{$-\infty$,$-3$,$1$, $2$, $+\infty$} % first row
			\tkzTabLine { ,+,0,-,0,+,0,+ }
		\end{tikzpicture}
	\end{center}	
	Hàm số $y=f\left(2-e^x\right)-\dfrac{1}{3}{e^{3x}}+3e^{2x}-5e^x+1$ đồng biến trên khoảng nào dưới đây?
	\choice
	{$\left(0;\dfrac{3}{2}\right)$}
	{$\left(1;3\right)$}
	{\True $\left(-3;0\right)$}
	{$\left(-4;-3\right)$}
	\loigiai{
		Ta có $y'=-e^x.f'\left(2-e^x\right)-e^{3x}+6e^{2x}-5e^x=e^x\left[-f'\left(2-e^x\right)-e^{2x}+6e^x-5\right]$ .\\
		Đặt $t=2-e^x$, ta được\\
		$y'=\left(2-t\right)\left[-f'(t)-\left(2-t\right)^2+6\left(2-t\right)-5\right]=\left(2-t\right)\left[-f'(t)-t^2-2t+3\right]$ .\\
		$y'=0\Leftrightarrow\left(2-t\right)\left[-f'(t)-t^2-2t+3\right]=0\Leftrightarrow
		\hoac{
			& t=2\\ 
			& f'(t)=-t^2-2t+3.}$\\
		Hàm số $g(x)=-x^2-2x+3$ là parabol có trục đối xứng $x=-1$ và cắt trục hoành tại 2 điểm có hoành độ 
		$\hoac{
			& x=1\\ 
			& x=-3
		}$. Suy ra $f'(t)=-t^2-2t+3\Leftrightarrow \hoac{
			& t=1\\ 
			& t=-3. }$\\
		Bảng xét dấu\\
		\begin{center}
			\begin{tikzpicture}
				\tkzTabInit[lgt=3.9,espcl=2,nocadre]
				{$t$/0.7, $2-t$ /0.8, $-f'(t)-t^2-2t+3$ /0.8, $y'$ /0.8} % first column
				{$-\infty$,$-3$,$1$,$2$,$+\infty$} % first row
				\tkzTabLine { ,+,|,+,|,+,z,-, } % second row
				\tkzTabLine {,-,0,+,0,-,|,-,} % third row
				\tkzTabLine {,-,0,+,0,-,0,+,} % last row
			\end{tikzpicture}
		\end{center}
		Dựa vào bảng xét dấu $y'>0,\forall x\in\left(-3;0\right)$.}
\end{ex}


\begin{ex}%[2D1G1-2]%Câu 28 
	[Sở Lạng Sơn 2022] Cho hàm số $f(x)$ có bảng biến thiên như sau:\\
	\begin{center}
		\begin{tikzpicture}
			\tkzTabInit[espcl=2.5,lgt=1,nocadre]
			{$x$/0.7,$y'$/0.7,$y$/3.5}
			{$-\infty$,$1$,$2$,$3$,$4$,$+\infty$}
			\tkzTabLine{,+,0,-,0,+,0,-,0,+,}
			\node (0) at ($(N12)+(0,-3)$) {$-\infty$};
			\node (1) at ($(N22)+(0,-.5)$) {$3$};
			\node (2) at ($(N32)+(0,-1.7)$) {$1$};
			\node (3) at ($(N42)+(0,-0.7)$) {$2$};
			\node (4) at ($(N52)+(0,-2.3)$) {$0$};
			\node (5) at ($(N62)+(0,-.3)$) {$+\infty$};
			%				\node (8) at ($(N42)+(0,-.5)$) {};
			%				\coordinate (9) at ($(N42)!.6!(N53)+ (-0.5,0)$);
			%				\coordinate (6) at ($(T12)!.6!(T13)$);
			%				\coordinate (7) at ($(T22)!.6!(T23)$);
			\draw[-stealth] (0)--(1);
			\draw[-stealth] (1)--(2);
			\draw[-stealth] (2)--(3);
			\draw[-stealth] (1)--(2);
			\draw[-stealth] (3)--(4);
			\draw[-stealth] (4)--(5);
			%				\draw[->,red] (5)--(8);
			%				\draw[->,red] (8)--(9);
			%				\draw[blue,dashed](6)--(7)node[above left]{$y=0$};
		\end{tikzpicture}		
	\end{center}
	Hàm số $y=\left[f(x)\right]^3-3\left[f(x)\right]^2$ đồng biến trên khoảng nào dưới đây?
	\choice
	{$\left(-\infty\,;1\right)$}
	{$\left(1\,;2\right)$}
	{\True $\left(3\,;4\right)$}
	{$\left(2\,;3\right)$}
	\loigiai{
		Ta có $y'=3f'(x)\left[f^2(x)-2f(x)\right]$. 
		Phương trình $y'=0\Leftrightarrow \hoac{
			&{f}'(x)=0\\ 
			& f(x)=0\\ 
			& f(x)=2.
		}$
		\begin{center}
			\begin{tikzpicture}
				\tkzTabInit[espcl=2.5,lgt=1.5]
				{$x$/0.7,$y'$/0.7,$y$/3.5}
				{$-\infty$,$1$,$2$,$3$,$4$,$+\infty$}
				\tkzTabLine{,+,0,-,0,+,0,-,0,+,}
				\node (0) at ($(N12)+(0,-3)$) {$-\infty$};
				\node (1) at ($(N22)+(0,-.3)$) {$3$};
				\node (2) at ($(N32)+(0,-1.7)$) {$1$};
				\node (3) at ($(N42)+(0,-0.8)$) {$2$};
				\node (4) at ($(N52)+(0,-2.3)$) {$0$};
				\node (5) at ($(N62)+(0,-.3)$) {$+\infty$};
				\node (a) at ($(N11)+(0.65,0.35)$) {$a$};
				\node (b) at ($(N11)+(2.0,0.4)$) {$b$};
				\node (c) at ($(N11)+(3.38,0.35)$) {$c$};
				\node (d) at ($(N11)+(11.85,0.4)$) {$d$};
				\node (6) at ($(N12)+(0,-0.8)$) {};
				\node (7) at ($(N62)+(0,-0.8)$) {};
				\node (8) at ($(N12)+(0,-2.3)$) {};
				\node (9) at ($(N62)+(0,-2.3)$) {};
				%				\node (8) at ($(N42)+(0,-.5)$) {};
				%				\coordinate (9) at ($(N42)!.6!(N53)+ (-0.5,0)$);
				\coordinate (A) at ($(0)!.25!(1)$);
				\coordinate (B) at ($(0)!.8!(1)$);
				\coordinate (C) at ($(1)!.35!(2)$);
				\coordinate (D) at ($(4)!.75!(5)$);
				%				\coordinate (7) at ($(T22)!.6!(T23)$);
				\draw[->] (0)--(1);
				\draw[->] (1)--(2);
				\draw[->] (2)--(3);
				\draw[->] (1)--(2);
				\draw[->] (3)--(4);
				\draw[->] (4)--(5);
				%				\draw[->,red] (5)--(8);
				%				\draw[->,red] (8)--(9);
				\draw[blue,dashed](6)--(7)node[below]{$y=2$} (a)--(A) (b)--(B) (c)--(C) (d)--(D);
				\draw[blue,dashed](8)--(9)node[below left]{$y=0$};
			\end{tikzpicture}		
		\end{center}
		Dựa vào bảng biến thiên, ta thấy $f'(x)=0\Leftrightarrow x\in \{ 1\,;2\,;3\,;4 \}$;\\
		$f(x)=0\Leftrightarrow x=a<1$ hoặc $x=4$;\\
		$f(x)=2\Leftrightarrow \hoac{
			& x=b\,\,\left(a<b<1\right)\\ 
			& x=c\in\left(1\,;2\right)\\ 
			& x=3\\ 
			& x=d>4.
		}$ \\
		Ta lập được bảng xét dấu của $y'$ 
		\begin{center}
			\begin{tikzpicture}
				\tkzTabInit[lgt=1.2,espcl=1.5,nocadre]
				{$x$/1, $f(x)$ /.8} % first column
				{$-\infty$,$a$, $b$, $1$,$c$, $2$,$3$, $4$, $d$, $+\infty$} % first row
				\tkzTabLine { ,+,z,-,z,+,z,-,z,+,z,-,z,+,z,-,z,+, } % second row
				%				\tkzTabLine {,-,z,+,t,+,} % third row
				%				\tkzTabLine {,+,d,-,z,+,} % last row
			\end{tikzpicture}
		\end{center}
		Từ bảng xét dấu, ta thấy hàm số đồng biến trên các khoảng \\
		$\left(-\infty;a\right)$, $\left(b;1\right)$, $\left(c;2\right)$, $\left(3;4\right)$ và $(d;+\infty)$.
	}
\end{ex}

\begin{ex}%[2D1G1-2]%Câu 29 
	[THPT Bùi Thị Xuân – Huế-2022] 
	\immini{
		Cho hàm số $y=f(x)$ là hàm đa thức bậc bốn. Đồ thị hàm số $f'(x+2)$ được cho trong hình vẽ bên. Hàm số 
		$$g(x)=4 f\left(x^2\right)-x^6+5 x^4-4 x^2+1$$
		đồng biến trên khoảng nào dưới đây?
		\choice
		{$(-4 ;-3)$}
		{\True $(2 ;+\infty)$}
		{$(-\sqrt{2};\sqrt{2})$}
		{$(-2 ;-1)$}}{
		\begin{tikzpicture}[scale=0.6,font=\footnotesize, line join=round, line cap=round, >=stealth] %Đường cong bậc 3
			\draw[thick, ->] (-5.3,0)--(5,0);
			\draw[thick, ->] (0,-3.5)--(0,7);
			\draw (5.2,0) node[below] {$x$};
			\draw (0,7.1) node[left]{$y$};
			\draw (0,0) node[below left]{$0$};
			\draw[fill] (-2,0) circle (0.5pt)node[below left]{$ -2 $};
			\draw[fill] (2,0) circle (0.5pt)node[below]{$ 2$};
			\draw[fill] (0,3) circle (0.5pt)node[left]{$ 3 $};
			\draw[fill] (0,1) circle (0.5pt)node[right]{$ 1 $};
			\draw[fill] (0,-1) circle (0.5pt)node[right]{$ -1 $};
			\draw[dashed] (-2,0)--(-2,1) --(0,1); 
			\draw[dashed](2,0)--(2,3)--(0,3);
			\draw[line width=1.2pt,smooth,samples=100,domain=-2.8:4.5] plot(\x,{-0.271*(\x)^3+0.75*(\x)^2+1.583*\x-1});
		\end{tikzpicture}		
	}
	\loigiai{
		$\begin{aligned}
			& g(x)=4f\left(x^2\right)-x^6+5x^4-4x^2+1\Rightarrow g' (x)=8xf'\left(x^2\right)-6x^5+20x^3-8x.\\ 
			& g' (x)=0\Leftrightarrow 8xf'\left(x^2\right)-6x^5+20x^3-8x=0 \\
			& \Leftrightarrow 2x\left[4f'\left(x^2\right)-3x^4+10x^2-4\right]=0\\ 
			&\Leftrightarrow 		\hoac{ 			& 2x=0\\ 
				& 4f'(x^2)-3x^4+10x^2-4=0
			}
			\Leftrightarrow \hoac{	& x=0\\ 
				& f'\left(x^2\right)=\dfrac{3}{4}{x^4}-\dfrac{5}{2}{x^2}+1.}
		\end{aligned}$\\ 
		Xét
		$f'\left(x^2\right)=\dfrac{3}{4}x^4-\dfrac{5}{2}x^2+1$. Đặt $x^2=t+2$, ta có\\
		$ f' (t+2)=\dfrac{3}{4}{(t+2)^2}-\dfrac{5}{2}(t+2)+1=\dfrac{3}{4}\left(t^2+4t+4\right)-\dfrac{5}{2}(t+2)-1=\dfrac{3}{4}{t^2}+\dfrac{1}{2}t-1$\\
		Khi đó số nghiệm của phương trình chính là số giao điểm của đồ thị hàm số $y=f' (t+2)$ và\\
		$ y=\dfrac{3}{4}{t^2}+\dfrac{1}{2}t-1$\\
		Ta có đồ thị 
		\begin{center}
			\begin{tikzpicture}[scale=0.6,font=\footnotesize, line join=round, line cap=round, >=stealth] %Đường cong bậc 3
				\draw[thick, ->] (-5.3,0)--(5,0);
				\draw[thick, ->] (0,-3.5)--(0,7);
				\draw (5.2,0) node[below] {$x$};
				\draw (0,7.1) node[left]{$y$};
				\draw (0,0) node[below left]{$0$};
				\draw[fill] (-2,0) circle (0.5pt)node[below left]{$ -2 $};
				\draw[fill] (2,0) circle (0.5pt)node[below]{$ 2$};
				\draw[fill] (0,3) circle (0.5pt)node[left]{$ 3 $};
				\draw[fill] (0,1) circle (0.5pt)node[right]{$ 1 $};
				\draw[fill] (0,-1) circle (0.5pt)node[right]{$ -1 $};
				\draw[dashed] (-2,0)--(-2,1) --(0,1); 
				\draw[dashed](2,0)--(2,3)--(0,3);
				\draw[line width=1.2pt,smooth,samples=100,domain=-2.8:4.5] plot(\x,{-0.271*(\x)^3+0.75*(\x)^2+1.583*\x-1});		
				\draw[line width=1.2pt,smooth,samples=100,domain=-3.3:2.8] plot(\x,{0.75*(\x)^2+0.5*\x-1});
			\end{tikzpicture}
		\end{center}
		Dựa vào đồ thị ta có $f' (t+2)=\dfrac{3}{4}t^2+\dfrac{1}{2}t-1\Leftrightarrow \hoac{& t=-2\\ & t=0\\ & t=2} \Leftrightarrow\hoac{& x+2=-2\\ & x+2=0\\ & x+2=2} \Leftrightarrow \hoac{& x=-4\\ & x=-2\\ & x=0.}$\\
		Ta có bảng xét dấu $g' (x)$ như sau
		\begin{center}
			\begin{tikzpicture}
				\tkzTabInit[lgt=1.2,espcl=2,nocadre]
				{$x$/0.7, $f(x)$ /.7}
				{$-\infty$, $-4$,$-2$, $0$, $+\infty$} % first row
				\tkzTabLine { ,-,z,+,z,-,z,+, }
			\end{tikzpicture}
		\end{center}
		Vậy hàm số $g(x)=4 f\left(x^2\right)-x^6+5 x^4-4 x^2+1$ đồng biến trên khoảng $(2 ;+\infty)$.}
\end{ex}

\begin{ex}%[2D1G1-2]%Câu 30
	[Chuyên Bắc Ninh 2022] 
	\immini{
		Cho hàm số $ y=f(x)$ liên tục trên $\mathbb{R}$ có đồ thị hàm số $ y=f'(x)$ có đồ thị như hình vẽ bên.
		Hàm số $g(x)=2f\left(\left| x-1\right|\right)-x^2+2x+2020$ đồng biến trên khoảng nào
		\choice
		{$\left(-2;0\right)$}
		{$\left(-3;1\right)$}
		{$\left(1\,;3\right)$}
		{\True $\left(0\,;\,1\right)$}}{
		\begin{tikzpicture}[scale=0.6,font=\footnotesize, line join=round, line cap=round, >=stealth] %Đường cong bậc 3
			\draw[thick, ->] (-3.3,0)--(5,0);
			\draw[thick, ->] (0,-3.0)--(0,5.5);
			\draw (5.2,0) node[below] {$x$};
			\draw (0,5.8) node[left]{$y$};
			\draw (0,0) node[below left]{$0$};
			\draw[fill] (-1,0) circle (0.5pt)node[above]{$ -1 $};
			\draw[fill] (1,0) circle (0.5pt)node[below]{$ 1$};
			\draw[fill] (0,1) circle (0.5pt)node[left]{$ 1 $};
			\draw[fill] (0,-1) circle (0.5pt)node[right]{$ -1 $};
			\draw[fill] (0,3) circle (0.5pt)node[left]{$ 3 $};
			\draw[fill] (3,0) circle (0.5pt)node[below]{$ 3 $};
			\draw[dashed] (-1,0)--(-1,-1) --(0,-1); 
			\draw[dashed](1,0)--(1,1)--(0,1);
			\draw[dashed](3,0)--(3,3)--(0,3);
			\draw[line width=1.2pt,smooth,samples=100,domain=-2.2:4.3] plot(\x,{-0.333*(\x)^3+1*(\x)^2+1.333*\x-1});		
			%\draw[line width=1.2pt,smooth,samples=100,domain=-3.3:2.8] plot(\x,{0.75*(\x)^2+0.5*\x-1});
		\end{tikzpicture}	
	}
	\loigiai{
		Ta có $g(x)=2f\left(\left| x-1\right|\right)-x^2+2x+2020\Leftrightarrow g(x)=2f\left(\left| x-1\right|\right)-\left(x-1\right)^2+2021$.\\
		Xét hàm số $ k\left(x-1\right)=2f\left(x-1\right)-\left(x-1\right)^2+2021$.\\
		Đặt $ t=x-1$\\
		Xét hàm số $ h(t)=2f(t)-t^2+2021$ $\Rightarrow{h}'(t)=2f'(t)-2t$.\\
		Kẻ đường $ y=x$ như hình vẽ.
		\begin{center}
			\begin{tikzpicture}[scale=0.6,font=\footnotesize, line join=round, line cap=round, >=stealth] %Đường cong bậc 3
				\draw[thick, ->] (-3.3,0)--(5,0);
				\draw[thick, ->] (0,-3.0)--(0,5.5);
				\draw (5.2,0) node[below] {$x$};
				\draw (0,5.8) node[left]{$y$};
				%	\draw (0,0) node[below left]{$0$};
				\draw[fill] (-1,0) circle (0.5pt)node[above]{$ -1 $};
				\draw[fill] (1,0) circle (0.5pt)node[below]{$ 1$};
				\draw[fill] (0,1) circle (0.5pt)node[left]{$ 1 $};
				\draw[fill] (0,-1) circle (0.5pt)node[right]{$ -1 $};
				\draw[fill] (0,3) circle (0.5pt)node[left]{$ 3 $};
				\draw[fill] (3,0) circle (0.5pt)node[below]{$ 3 $};
				\draw[dashed] (-1,0)--(-1,-1) --(0,-1); 
				\draw[dashed](1,0)--(1,1)--(0,1);
				\draw[dashed](3,0)--(3,3)--(0,3);
				\draw[line width=1.2pt,smooth,samples=100,domain=-2.2:4.3] plot(\x,{-0.333*(\x)^3+1*(\x)^2+1.333*\x-1});		
				%\draw[line width=1.2pt,smooth,samples=100,domain=-3.3:2.8] plot(\x,{0.75*(\x)^2+0.5*\x-1});
				\draw[line width=1.2pt,smooth,samples=100](-2,-2)--(4,4);
			\end{tikzpicture}
		\end{center}
		Khi đó $h'(t)>0\Leftrightarrow{f}'(t)-t>0\Leftrightarrow{f}'(t)>t$$\Leftrightarrow \hoac{
			& t<-1\\ 
			& 1<t<3.
		}$\\
		Do đó $k'\left(x-1\right)>0\Leftrightarrow \hoac{
			& x-1<-1\\ 
			& 1<x-1<3} \Leftrightarrow \hoac{
			& x<0\\ 
			& 2<x<4.}$\\
		Ta có bảng biến thiên của hàm số $ k\left(x-1\right)=2f\left(x-1\right)-\left(x-1\right)^2+2021$.
		\begin{center}
			\begin{tikzpicture}
				\tkzTabInit[lgt=1.8,espcl=2.3]
				{$x$ /1.2, $k'(x-1)$ /1.2,$k(x-1)$ /2}
				{$-\infty$ , $0$,$2$,$4$, $+\infty$}
				\tkzTabLine{,+,0,-,0,+,0,-,}
				\tkzTabVar{-/$ $ ,+/$ $, -/$ $,+/$ $,-/$ $}
			\end{tikzpicture}
		\end{center}
		Khi đó, ta có bảng biến thiên của $g(x)=2f\left(\left| x-1\right|\right)-\left(x-1\right)^2+2021$ bằng cách lấy đối xứng qua đường thẳng $ x=1$ như sau\\
		\begin{center}
			\begin{tikzpicture}
				\tkzTabInit[lgt=1.2,espcl=2.5,nocadre]
				{$x$ /0.7, $g'(x)$ /0.7,$g(x)$ /2.5}
				{$-\infty$ ,$-2$, $0$,$1$,$2$,$4$, $+\infty$}
				\tkzTabLine{,+,0,-,0,+,0,-,0,+,0,-,}
				\tkzTabVar{-/$ $ ,+/$ $, -/$ $,+/$ $,-/$ $,+/ $ $,-/$ $}
			\end{tikzpicture}
		\end{center}
		Vậy hàm số đồng biến trên $\left(0;1\right)$.}
\end{ex}

\begin{ex}%[2D1G1-2]%Câu 31
	[Chuyên Thái Bình 2022] 
	\immini{
		Cho hàm số $f(x)=a{x^4}+b{x^3}+c{x^2}+dx+a$ có đồ thị hàm số $y=f'(x)$ như hình vẽ bên. Hàm số $y=g(x)=f\left(1-2x\right)f\left(2-x\right)$ đồng biến trên khoảng nào dưới đây?
		\choice
		{$\left(\dfrac{1}{2};\dfrac{3}{2}\right)$}
		{$\left(-\infty ;0\right)$}
		{$\left(0;2\right)$}
		{\True $\left(3;+\infty\right)$}}{
		\begin{tikzpicture}[scale=0.9,font=\footnotesize, line join=round, line cap=round, >=stealth] %Đường cong bậc 3
			\draw[thick, ->] (-2.5,0)--(2.5,0);
			\draw[thick, ->] (0,-2.8)--(0,2.8);
			\draw (2.6,0) node[below] {$x$};
			\draw (0,2.9) node[left]{$y$};
			\draw (0,0) node[below left]{$0$};
			\draw[fill] (-1,0) circle (0.5pt)node[below left]{$ -1 $};
			\draw[fill] (1,0) circle (0.5pt)node[below right]{$ 1$};
			%			\draw[dashed] (-1,0)--(-1,-1) --(0,-1); 
			%			\draw[dashed](1,0)--(1,1)--(0,1);
			%			\draw[dashed](3,0)--(3,3)--(0,3);
			\draw[line width=1.2pt,smooth,samples=100,domain=-1.3:1.3] plot(\x,{3*(\x)^3-3*\x});		
			%\draw[line width=1.2pt,smooth,samples=100,domain=-3.3:2.8] plot(\x,{0.75*(\x)^2+0.5*\x-1});
		\end{tikzpicture}	
	}
	\loigiai{
		Ta có $f'(x)=4a{x^3}+3b{x^2}+2cx+d$, theo đồ thị thì đa thức $f'(x)$ có ba nghiệm phân biệt là $-1,0,1$ nên $f'(x)=4ax\left(x+1\right)\left(x-1\right)=4a{x^3}-4ax\Rightarrow f(x)=a{x^4}-2a{x^2}+a=a{\left(x^2-1\right)^2}$.\\
		Dựa vào đồ thị hàm số $y=f'(x)$ ta có $a>0$ nên $f(x)>0,\forall x\in\mathbb{R}\setminus\left\{\pm 1\right\}$.\\
		$g'(x)=\left[f\left(1-2x\right)\right]'f\left(2-x\right)+f\left(1-2x\right)\left[f\left(2-x\right)\right]'=-2f'\left(1-2x\right)f\left(2-x\right)-f\left(1-2x\right)f'\left(2-x\right)$. Xét $x\in\left(\dfrac{1}{2};\dfrac{3}{2}\right)\Rightarrow
		\heva{		
			& 1-2x\in\left(-2;0\right)\\ 
			& 2-x\in\left(\dfrac{1}{2};\dfrac{3}{2}\right)}$, dấu của $f'(x)$ không cố định trên $\left(\dfrac{1}{2};\dfrac{3}{2}\right)$ nên ta không kết luận được tính đơn điệu của hàm số $g(x)$ trên $\left(\dfrac{1}{2};\dfrac{3}{2}\right)$.\\
		Xét $x\in\left(-\infty ;0\right)\Rightarrow
		\heva{
			& 1-2x\in\left(1;+\infty\right)\\ 
			& 2-x\in\left(2;+\infty\right)} 
		\Rightarrow \heva{
			& f'\left(1-2x\right)>0\\ 
			& f'\left(2-x\right)>0} \Rightarrow g'(x)<0$.\\
		Do đó, hàm số $g(x)$ nghịch biến trên $\left(-\infty ;0\right)$.\\
		$x\in\left(0;2\right)\Rightarrow \heva{
			& 1-2x\in\left(-3;1\right)\\ 
			& 2-x\in\left(0;2\right)}$, dấu của $f'(x)$ không cố định trên $\left(-3;1\right)$ và $\left(0;2\right)$ nên ta không kết luận được tính đơn điệu của hàm số $g(x)$ trên $\left(\dfrac{1}{2};\dfrac{3}{2}\right)$.\\
		Xét $x\in\left(3;+\infty\right)\Rightarrow \heva{
			& 1-2x\in\left(-\infty ;-5\right)\\ 
			& 2-x\in\left(-\infty ;-1\right)} \Rightarrow \heva{
			& f'\left(1-2x\right)<0\\ 
			& f'\left(2-x\right)<0} \Rightarrow g'(x)>0$. \\
		Do đó, hàm số $g(x)$ đồng biến trên $\left(3;+\infty\right)$.}
\end{ex}

\begin{dang}{Bài toán hàm ẩn, hàm hợp liên quan đến tham số và một số bài toán khác}
\end{dang}

\begin{ex}%[2D1G1-3]%Câu 1
	[Chuyên Lê Hồng Phong Nam Định 2019]
	\immini{
		Cho hàm số $ y=f(x)$ có đạo hàm liên tục trên $\mathbb{R}$. Biết hàm số $ y=f'(x)$ có đồ thị như hình vẽ. Gọi $ S$ là tập hợp các giá trị nguyên $ m\in\left[-5\,;\,\text{5}\right]$ để hàm số $ g(x)=f\left(x+m\right)$ nghịch biến trên khoảng $\left(1\,;\,2\right)$. Hỏi $S$ có bao nhiêu phần tử?
		\choice
		{$ 4$}
		{$ 3$}
		{$ 6$}
		{\True $ 5$}}{
		\begin{tikzpicture}[scale=0.9,font=\footnotesize, line join=round, line cap=round, >=stealth] %Đường cong bậc 3
			\draw[thick, ->] (-2.5,0)--(4,0);
			\draw[thick, ->] (0,-2.8)--(0,2.8);
			\draw (4.3,0) node[below] {$x$};
			\draw (0,2.9) node[left]{$y$};
			\draw (0,0) node[below left]{$0$};
			\draw[fill] (-1,0) circle (0.5pt)node[below left]{$ -1 $};
			\draw[fill] (1,0) circle (0.5pt)node[below]{$ 1$};
			\draw[fill] (3,0) circle (0.5pt)node[below right]{$ 3$};
			%			\draw[dashed] (-1,0)--(-1,-1) --(0,-1); 
			%			\draw[dashed](1,0)--(1,1)--(0,1);
			%			\draw[dashed](3,0)--(3,3)--(0,3);
			\draw[line width=1.2pt,smooth,samples=100,domain=-1.65:3.5] plot(\x,{0.33*(\x)^3-(\x)^2-0.333*(\x)+1});		
			%\draw[line width=1.2pt,smooth,samples=100,domain=-3.3:2.8] plot(\x,{0.75*(\x)^2+0.5*\x-1});
		\end{tikzpicture}	
	}
	\loigiai{
		Ta có $g'(x)=f'\left(x+m\right)$. Vì $ y=f'(x)$ liên tục trên $\mathbb{R}$ nên $g'(x)=f'\left(x+m\right)$ cũng liên tục trên $\mathbb{R}$. Căn cứ vào đồ thị hàm số $ y=f'(x)$ ta thấy\\
		$g'(x)<0\Leftrightarrow{f}'\left(x+m\right)<0$ $\Leftrightarrow\hoac{
			& x+m<-1\\ 
			& 1<x+m<3} \Leftrightarrow \hoac{
			& x<-1-m\\ 
			& 1-m<x<3-m.}$\\
		Hàm số $ g(x)=f\left(x+m\right)$ nghịch biến trên khoảng $\left(1\,;\,2\right)$
		$\Leftrightarrow \hoac{
			& 2\le-1-m\\ 
			&\hoac{
				& 3-m\ge 2\\ 
				& 1-m\le 1}} \Leftrightarrow \hoac{
			& m\le-3\\ 
			& 0\le m\le 1.}$\\
		Mà $ m$ là số nguyên thuộc đoạn $\left[-5\,;\,5\right]$ nên ta có $ S=\left\{-5;-4;-3;0;1\right\}$.\\
		Vậy $ S$ có $5$ phần tử.}
\end{ex}

\begin{ex}%[2D1G1-3]%Câu 2
	[Chuyên Nguyễn Bỉnh Khiêm-Quảng Nam-2020] Cho hàm số $ y=f(x)$ có đạo hàm trên $\mathbb{R}$ và bảng xét dấu đạo hàm như hình vẽ sau
	\begin{center}
		\begin{tikzpicture}
			\tkzTabInit[lgt=1.2,espcl=2.5,nocadre]
			{$x$/0.7, $f'(x)$ /2.5} % first column
			{$-\infty$, $-10$,$-2$, $3$,$8$, $+\infty$} % first row
			\tkzTabLine { ,+,z,-,z,+,z,-,z,+, } % second row
			%				\tkzTabLine {,-,z,+,t,+,} % third row
			%				\tkzTabLine {,+,d,-,z,+,} % last row
		\end{tikzpicture}
	\end{center}
	Có bao nhiêu số nguyên $ m$ để hàm số $ y=f\left(x^3+4x+m\right)$ nghịch biến trên khoảng $\left(-1;1\right)$?
	\choice
	{$ 3$}
	{$ 0$}
	{\True $ 1$}
	{$ 2$}
	\loigiai
	{
		Đặt $ t=x^3+4x+m\Rightarrow{t}'=3x^2+4$ nên $ t$ đồng biến trên $\left(-1;1\right)$ và $ t\in\left(m-5;m+5\right)$.\\
		Yêu cầu bài toán trở thành tìm $ m$ để hàm số $ f(t)$ nghịch biến trên khoảng $\left(m-5;m+5\right)$.\\
		Dựa vào bảng biến thiên ta được $\heva{
			& m-5\ge-2\\ 
			& m+5\le 8} \Leftrightarrow \heva{
			& m\ge 3\\ 
			& m\le 3} \Leftrightarrow m=3$.}
\end{ex}

\begin{ex}%[2D1G1-3]%Câu 3
	[Chuyên ĐH Vinh-Nghệ An-2020]
	\immini{
		Cho hàm số $ f(x)$ có đạo hàm trên $\mathbb{R}$và $ f(1)=1$. Đồ thị hàm số $ y=f'(x)$ như hình bên. Có bao nhiêu số nguyên dương $ a$ để hàm số $ y=\left| 4f\left(\sin x\right)+\cos 2x-a\right|$ nghịch biến trên $\left(0;\dfrac{\pi}{2}\right)$?
		\choice
		{$ 2$}
		{\True $ 3$}
		{Vô số}
		{$ 5$}}{
		\begin{tikzpicture}[scale=0.9,font=\footnotesize, line join=round, line cap=round, >=stealth] %Đường cong bậc 3
			\draw[thick, ->] (-2.5,0)--(3,0);
			\draw[thick, ->] (0,-2.8)--(0,2.8);
			\draw (3.1,0) node[below] {$x$};
			\draw (0,2.9) node[left]{$y$};
			\draw (0,0) node[below left]{$0$};
			\draw[fill] (-1,0) circle (0.5pt)node[below]{$ -1 $};
			\draw[fill] (1,0) circle (0.5pt)node[above]{$ 1$};
			%	\draw[fill] (3,0) circle (0.5pt)node[below right]{$ 3$};
			\draw[dashed] (-1,0)--(-1,1); 
			\draw[dashed](1,0)--(1,-1);
			%			\draw[dashed](3,0)--(3,3)--(0,3);
			\draw[line width=1.2pt,smooth,samples=100,domain=-2:2] plot(\x,{.8*(\x)^3+0*(\x)^2-1.8*(\x)});		
			%\draw[line width=1.2pt,smooth,samples=100,domain=-3.3:2.8] plot(\x,{0.75*(\x)^2+0.5*\x-1});
			\draw (2.0,2.8) node[left]{$y=f'(x)$};
		\end{tikzpicture}	
	}
	\loigiai
	{		Đặt $g(x)=\left| 4f\left(\sin x\right)+\cos 2x-a\right|\Rightarrow g(x)=\sqrt{\left[4f\left(\sin x\right)+\cos 2x-a\right]^2}$ .\\
		$\Rightarrow{g}'(x)=\dfrac{\left[4\cos x\cdot f'\left(\sin x\right)-2\sin 2x\right]\left[4f\left(\sin x\right)+\cos 2x-a\right]}{\sqrt{\left[4f\left(\sin x\right)+\cos 2x-a\right]^2}}$.\\
		Ta có $ 4\cos x\cdot f'\left(\sin x\right)-2\sin 2x=4\cos x\left[f'\left(\sin x\right)-\sin x\right]$.\\
		Với $ x\in\left(0;\dfrac{\pi}{2}\right)$ thì $\cos x>0,\sin x\in\left(0;1\right)\Rightarrow{f}'\left(\sin x\right)-\sin x<0$.\\
		Hàm số $ g(x)$ nghịch biến trên $\left(0;\dfrac{\pi}{2}\right)$ khi $ 4f\left(\sin x\right)+\cos 2x-a\ge 0,\forall x\in\left(0;\dfrac{\pi}{2}\right)$\\
		$\Leftrightarrow 4f\left(\sin x\right)+1-2\sin^2x\ge a,\forall x\in\left(0;\dfrac{\pi}{2}\right)$.\\
		Đặt $ t=\sin x$ được $ 4f(t)+1-2t^2\ge a,\forall t\in\left(0;1\right)$ (*).\\
		Xét $ h(t)=4f(t)+1-2t^2\Rightarrow{h}'(t)=4f'(t)-4t=4\left[f'(t)-1\right]$.\\
		Với $ t\in\left(0;1\right)$ thì $h'(t)<0\Rightarrow h(t)$ nghịch biến trên $\left(0;1\right)$.\\
		Do đó (*) $\Leftrightarrow a\le h(1)=4f(1)+1-2.1^2=3$.\\
		Vậy có $3$ giá trị nguyên dương của a thỏa mãn.}
\end{ex}


\begin{ex}%[2D1G1-3]%Câu 4
	[Chuyên Quang Trung-2020]
	\immini{
		Cho hàm số $ y=f(x)$ có đạo hàm liên tục trên $\mathbb{R}$ và có đồ thị $ y=f'(x)$ như hình vẽ. Đặt $ g(x)=f\left(x-m\right)-\dfrac{1}{2}{\left(x-m-1\right)^2}+2019$, với $ m$ là tham số thực. Gọi $ S$ là tập hợp các giá trị nguyên dương của $ m$ để hàm số $ y=g(x)$ đồng biến trên khoảng $\left(5;6\right)$. Tổng tất cả các phần tử trong $ S$ bằng
		\choice
		{$ 4$}
		{$ 11$}
		{\True $ 14$}
		{$ 20$}}{
		\begin{tikzpicture}[scale=0.9,font=\footnotesize, line join=round, line cap=round, >=stealth] %Đường cong bậc 3
			\draw[style=help lines,step=1] (-2.5,-3) grid (3,3.5);
			\draw[thick, ->] (-2.5,0)--(3.5,0);
			\draw[thick, ->] (0,-2.8)--(0,2.8);
			\draw (3.6,0) node[below] {$x$};
			\draw (0,3) node[above left]{$y$};
			\draw (0,0) node[below left]{$0$};
			%\draw[fill] (-1,0) circle (0.5pt)node[below]{$ -1 $};
			\draw[fill] (1,0) circle (0.5pt)node[below left]{$ 1$};
			%	\draw[fill] (3,0) circle (0.5pt)node[below right]{$ 3$};
			\draw[dashed] (-1,0)--(-1,-2) --(2,-2)--(2,0); 
			\draw[dashed](3,0)--(3,2) --(0,2);
			\draw (-1,-2) circle (2pt);
			\draw (3,2) circle (2pt);
			%			\draw[dashed](3,0)--(3,3)--(0,3);
			\draw[line width=1.2pt,smooth,samples=100,domain=-1.1:3.1] plot(\x,{1*(\x)^3-3*(\x)^2-0*(\x)+2});		
			%\draw[line width=1.2pt,smooth,samples=100,domain=-3.3:2.8] plot(\x,{0.75*(\x)^2+0.5*\x-1});
			%\draw (2.0,2.8) node[left]{$y=f'(x)$};
		\end{tikzpicture}	
	}
	\loigiai
	{
		Xét hàm số $ g(x)=f\left(x-m\right)-\dfrac{1}{2}{\left(x-m-1\right)^2}+2019$.\\
		$g'(x)=f'\left(x-m\right)-\left(x-m-1\right)$.\\
		Xét phương trình $g'(x)=0. \quad \quad (1)$\\
		Đặt $ x-m=t$, phương trình $(1)$ trở thành $f'(t)-\left(t-1\right)=0\Leftrightarrow{f}'(t)=t-1. \quad (2)$\\
		Nghiệm của phương trình $(2)$ là hoành độ giao điểm của hai đồ thị hàm số $ y=f'(t)$ và $ y=t-1$.\\
		Ta có đồ thị các hàm số $ y=f'(t)$ và $ y=t-1$ như sau
		\begin{center}
			\begin{tikzpicture}[scale=0.9,font=\footnotesize, line join=round, line cap=round, >=stealth] %Đường cong bậc 3
				\draw[style=help lines,step=1] (-2.5,-3) grid (3,3.5);
				\draw[thick, ->] (-2.5,0)--(3.5,0);
				\draw[thick, ->] (0,-2.8)--(0,2.8);
				\draw (3.6,0) node[below] {$x$};
				\draw (0,3) node[above left]{$y$};
				\draw (0,0) node[below left]{$0$};
				%\draw[fill] (-1,0) circle (0.5pt)node[below]{$ -1 $};
				\draw[fill] (1,0) circle (0.5pt)node[below left]{$ 1$};
				%	\draw[fill] (3,0) circle (0.5pt)node[below right]{$ 3$};
				\draw[dashed] (-1,0)--(-1,-2) --(2,-2)--(2,0); 
				\draw[dashed](3,0)--(3,2) --(0,2);
				\draw (-1,-2) circle (2pt);
				\draw (3,2) circle (2pt);
				%			\draw[dashed](3,0)--(3,3)--(0,3);
				\draw[line width=1.2pt,smooth,samples=100,domain=-1.1:3.1] plot(\x,{1*(\x)^3-3*(\x)^2-0*(\x)+2});		
				%\draw[line width=1.2pt,smooth,samples=100,domain=-3.3:2.8] plot(\x,{0.75*(\x)^2+0.5*\x-1});
				%\draw (2.0,2.8) node[left]{$y=f'(x)$};
				\draw (-2,-3)--(4,3);
			\end{tikzpicture}
		\end{center}
		Căn cứ đồ thị các hàm số ta có phương trình $(2)$ có nghiệm là $\hoac{
			& t=-1\\ 
			& t=1\\ 
			& t=3} \Rightarrow \hoac{
			& x=m-1\\ 
			& x=m+1\\ 
			& x=m+3.}$\\
		Ta có bảng biến thiên của $ y=g(x)$
		\begin{center}
			\begin{tikzpicture}
				\tkzTabInit[lgt=1,espcl=2.5,nocadre]
				{$x$ /0.8, $y'$ /0.8,$y$ /2.5}
				{$-\infty$ , $m-1$,$m+1$,$m+3$, $+\infty$}
				\tkzTabLine{,+,0,-,0,+,0,-,}
				\tkzTabVar{-/$ +\infty$ ,+/$ $, -/$ $,+/$ $,-/$+\infty $}
			\end{tikzpicture}
		\end{center}
		Để hàm số $ y=g(x)$ đồng biến trên khoảng $\left(5;6\right)$ cần $\hoac{
			&\heva{
				& m-1\le 5\\ 
				& m+1\ge 6}\\ 
			& m+3\le 5}\Leftrightarrow\hoac{
			& 5\le m\le 6\\ 
			& m\le 2.}$\\
		Vì $ m\in\mathbb{N}^*\Rightarrow m$ nhận các giá trị $ 1;\,2;\,5;\,6\Rightarrow S=14$.}
\end{ex}

\begin{ex}%[2D1G1-3]%Câu 5
	[Sở Hà Nội-Lần 2-2020] 
	\immini{
		Cho hàm số $y=a{x^4}+b{x^3}+c{x^2}+dx+e,\,\,a\ne 0$. Hàm số $y=f'(x)$ có đồ thị như hình vẽ bên. 
		Gọi S là tập hợp tất cả các giá trị nguyên thuộc khoảng $\left(-6;6\right)$ của tham số $m$ để hàm số $g(x)=f\left(3-2x+m\right)+x^2-\left(m+3\right)x+2m^2$ nghịch biến trên $\left(0;1\right)$. Khi đó, tổng giá trị các phần tử của S là
		\choice
		{$12$}
		{\True $9$}
		{$6$}
		{$15$}}{
		\begin{tikzpicture}[scale=0.7,font=\footnotesize, line join=round, line cap=round, >=stealth] %Đường cong bậc 3
			%	\draw[style=help lines,step=1] (-2.5,-3) grid (3,3.5);
			\draw[thick, ->] (-4.5,0)--(6.5,0);
			\draw[thick, ->] (0,-2.8)--(0,2.8);
			\draw (6.6,0) node[below] {$x$};
			\draw (0,3) node[above left]{$y$};
			\draw (0,0) node[below left]{$0$};
			\draw[fill] (-2,0) circle (0.5pt)node[below]{$ -2 $};
			\draw[fill] (4,0) circle (0.5pt)node[above]{$ 4$};
			\draw[fill] (0,1) circle (0.5pt)node[right]{$ 1 $};
			\draw[fill] (0,-2) circle (0.5pt)node[left]{$ -2$};
			%	\draw[fill] (3,0) circle (0.5pt)node[below right]{$ 3$};
			\draw[dashed] (-2,0)--(-2,1) --(0,1); 
			\draw[dashed](4,0)--(4,-2) --(0,-2);
			%			\draw[dashed](3,0)--(3,3)--(0,3);
			\draw[line width=1.2pt,smooth,samples=100,domain=-3.8:5.5] plot(\x,{0.0714*(\x)^3-0.1423*(\x)^2-1.0714*(\x)});		
			%\draw[line width=1.2pt,smooth,samples=100,domain=-3.3:2.8] plot(\x,{0.75*(\x)^2+0.5*\x-1});
			%\draw (2.0,2.8) node[left]{$y=f'(x)$};
		\end{tikzpicture}	
	}
	\loigiai
	{
		Xét $g'(x)=-2f'\left(3-2x+m\right)+2x-\left(m+3\right)$.\\
		Xét phương trình $g'(x)=0$, đặt $t=3-2x+m$ thì phương trình trở thành\\ $-2\cdot \left[f'(t)-\dfrac{-t}{2}\right]=0\Leftrightarrow\hoac{
			& t=-2\\ 
			& t=4\\ 
			& t=0.}$ \\
		Từ đó, $g'(x)=0\Leftrightarrow{x_1}=\dfrac{5+m}{2},\,x_2=\dfrac{m+3}{2},x_3=\dfrac{-1+m}{2}$.\\
		Lập bảng xét dấu, đồng thời lưu ý nếu $x>x_1$ thì $t<t_1$ nên $f(x)>0$. Và các dấu đan xen nhau do các nghiệm đều làm đổi dấu đạo hàm nên suy ra $g'(x)\le 0\Leftrightarrow x\in\left[x_2;{x_1}\right]\cup\left(-\infty ;{x_3}\right]$.\\
		Vì hàm số nghịch biến trên $\left(0;1\right)$ nên \\
		$g'(x)\le 0,\,\forall x\in\left(0;1\right)$ từ đó suy ra $\hoac{
			&\dfrac{3+m}{2}\le 0<1\le\dfrac{5+m}{2}\\ 
			& 1\le\dfrac{-1+m}{2}.}$ \\
		và giải ra các giá trị nguyên thuộc $\left(-6;6\right)$ của $m$ là $-3$; $3$; $4$; $5$. }
\end{ex}

\begin{ex}%[2D1G1-3]%Câu 6
	[Chuyên Quang Trung-Bình Phước-Lần 2-2020]
	\immini{
		Cho hàm số $ y=f(x)$ có đạo hàm liên tục trên $\mathbb{R}$ và có đồ thị $ y=f'(x)$ như hình vẽ bên. Đặt $ g(x)=f\left(x-m\right)-\dfrac{1}{2}{\left(x-m-1\right)^2}+2019$, với $ m$ là tham số thực. Gọi $ S$ là tập hợp các giá trị nguyên dương của $ m$ để hàm số $ y=g(x)$ đồng biến trên khoảng $\left(5;6\right)$. Tổng tất cả các phần tử trong $ S$ bằng
		\choice
		{$ 4$}
		{$ 11$}
		{\True $ 14$}
		{$ 20$}}{
		\begin{tikzpicture}[scale=0.9,font=\footnotesize, line join=round, line cap=round, >=stealth] %Đường cong bậc 3
			\draw[thick, ->] (-2.5,0)--(3.7,0);
			\draw[thick, ->] (0,-2.8)--(0,2.8);
			\draw (3.9,0) node[below] {$x$};
			\draw (0,2.9) node[left]{$y$};
			\draw (0,0) node[below left]{$0$};
			\draw[fill] (-1,0) circle (0.5pt)node[above]{$ -1 $};
			\draw[fill] (1,0) circle (0.5pt)node[below]{$ 1$};
			\draw[fill] (3,0) circle (0.5pt)node[below]{$ 3$};
			\draw[fill] (2,0) circle (0.5pt)node[above]{$ 2$};
			\draw[fill] (0,2) circle (0.5pt)node[above left]{$ 2$};
			\draw[fill] (0,-2) circle (0.5pt)node[below left]{$ -2$};
			\draw[dashed] (-1,0)--(-1,-2)--(2,-2)--(2,0); 
			\draw[dashed](3,0)--(3,2)--(0,2);
			%			\draw[dashed](3,0)--(3,3)--(0,3);
			\draw[line width=1.2pt,smooth,samples=100,domain=-1.1:3.1] plot(\x,{1*(\x)^3-3*(\x)^2-0*(\x)+2});		
			%\draw[line width=1.2pt,smooth,samples=100,domain=-3.3:2.8] plot(\x,{0.75*(\x)^2+0.5*\x-1});
			%	\draw (2.0,2.8) node[left]{$y=f'(x)$};
	\end{tikzpicture}	}
	\loigiai
	{
		Ta có $g'(x)=f'\left(x-m\right)-\left(x-m-1\right)$.\\
		Cho $g'(x)=0\Leftrightarrow{f}'\left(x-m\right)=x-m-1$.\\
		Đặt $ x-m=t\Rightarrow f'(t)=t-1$\\
		Khi đó nghiệm của phương trình là hoành độ giao điểm của đồ thị hàm số $ y=f'(t)$ và và đường thẳng $ y=t-1$.
		\begin{center}
			\begin{tikzpicture}[scale=0.9,font=\footnotesize, line join=round, line cap=round, >=stealth] %Đường cong bậc 3
				\draw[thick, ->] (-2.5,0)--(3.7,0);
				\draw[thick, ->] (0,-2.8)--(0,2.8);
				\draw (3.9,0) node[below] {$x$};
				\draw (0,2.9) node[left]{$y$};
				\draw (0,0) node[below left]{$0$};
				\draw[fill] (-1,0) circle (0.5pt)node[above]{$ -1 $};
				\draw[fill] (1,0) circle (0.5pt)node[below]{$ 1$};
				\draw[fill] (3,0) circle (0.5pt)node[below]{$ 3$};
				\draw[fill] (2,0) circle (0.5pt)node[above]{$ 2$};
				\draw[fill] (0,2) circle (0.5pt)node[above left]{$ 2$};
				\draw[fill] (0,-2) circle (0.5pt)node[below left]{$ -2$};
				\draw[dashed] (-1,0)--(-1,-2)--(2,-2)--(2,0); 
				\draw[dashed](3,0)--(3,2)--(0,2);
				%			\draw[dashed](3,0)--(3,3)--(0,3);
				\draw[line width=1.2pt,smooth,samples=100,domain=-1.1:3.1] plot(\x,{1*(\x)^3-3*(\x)^2-0*(\x)+2});		
				%\draw[line width=1.2pt,smooth,samples=100,domain=-3.3:2.8] plot(\x,{0.75*(\x)^2+0.5*\x-1});
				%	\draw (2.0,2.8) node[left]{$y=f'(x)$};
				\coordinate (a) at ($(-1,-2)!1.2!(3,2)$);
				\coordinate (b) at ($(-1,-2)!-.2!(3,2)$);
				\draw[line width=1.2pt,smooth] (a)--(b);
			\end{tikzpicture}
		\end{center}
		Dựa vào đồ thị hàm số ta có được $f'(t)=t-1\Leftrightarrow\hoac{
			& t=-1\\ 
			& t=1\\ 
			& t=3.} $ \\
		Bảng xét dấu của $g'(t)$
		\begin{center}
			\begin{tikzpicture}
				\tkzTabInit[lgt=1.2,espcl=2.5,nocadre]
				{$t$/1, $g'(x)$ /.8} % first column
				{$-\infty$, $-1$,$1$, $3$, $+\infty$} % first row
				\tkzTabLine { ,-,0,+,0,-,0,+, } % second row
				%				\tkzTabLine {,-,z,+,t,+,} % third row
				%				\tkzTabLine {,+,d,-,z,+,} % last row
			\end{tikzpicture}
		\end{center}
		Từ bảng xét dấu ta thấy hàm số $ g(t)$ đồng biến trên khoảng $\left(-1;1\right)$ và $\left(3;+\infty\right)$.\\
		Hay $\hoac{
			&-1<t<1\\ 
			& t>3}\Leftrightarrow\hoac{
			&-1<x-m<1\\ 
			& x-m>3} \Leftrightarrow\hoac{
			& m-1<x<m+1\\ 
			& x>m+3.}$\\
		Để hàm số $ g(x)$ đồng biến trên khoảng $\left(5;6\right)$ thì $\hoac{
			& m-1\le 5<6\le m+1\\ 
			& m+3\le 5<6} \Leftrightarrow\hoac{
			& 5\le m\le 6\\ 
			& m\le 2.}$\\
		Vì $ m$ là các số nguyên dương nên $ S=\left\{ 1;2;5;6\right\}$.\\
		Vậy tổng tất cả các phần tử của $ S$ là $ 1+2+5+6=14$.}
\end{ex}

\begin{ex}%[2D1G1-3]%Câu 7
	\immini{
		Cho hàm số $ y=f(x)$ liên tục có đạo hàm trên $\mathbb{R}$. Biết hàm số $ f'(x)$ có đồ thị cho như hình vẽ bên. Có bao nhiêu giá trị nguyên của $ m$ thuộc $\left[-2019;2019\right]$ để hàm só $ g(x)=f\left(2019^x\right)-mx+2$ đồng biến trên $\left[0;1\right]$.
		\choice
		{$ 2028$}
		{$ 2019$}
		{$ 2011$}
		{\True $ 2020$}}{
		\begin{tikzpicture}[scale=0.9,font=\footnotesize, line join=round, line cap=round, >=stealth] %Đường cong bậc 3
			\draw[thick, ->] (-3.5,0)--(2.5,0);
			\draw[thick, ->] (0,-2.8)--(0,2.8);
			\draw (2.7,0) node[below] {$x$};
			\draw (0,2.9) node[left]{$y$};
			\draw (0,0) node[below left]{$0$};
			%	\draw[fill] (-1,0) circle (0.5pt)node[above]{$ -1 $};
			\draw[fill] (1,0) circle (0.5pt)node[below right]{$ 1$};
			%		\draw[fill] (3,0) circle (0.5pt)node[below]{$ 3$};
			%		\draw[fill] (2,0) circle (0.5pt)node[above]{$ 2$};
			%		\draw[fill] (0,2) circle (0.5pt)node[above left]{$ 2$};
			%		\draw[fill] (0,-2) circle (0.5pt)node[below left]{$ -2$};
			%		\draw[dashed] (-1,0)--(-1,-2)--(2,-2)--(2,0); 
			%		\draw[dashed](3,0)--(3,2)--(0,2);
			\draw[line width=1.2pt,smooth,samples=100,domain=-3.28:1.32] plot(\x,{0.667*(\x)^3+2*(\x)^2-0.667*(\x)-2});		
			%\draw[line width=1.2pt,smooth,samples=100,domain=-3.3:2.8] plot(\x,{0.75*(\x)^2+0.5*\x-1});
			%	\draw (2.0,2.8) node[left]{$y=f'(x)$};
	\end{tikzpicture}	}
	\loigiai{
		Ta có $ g'(x)=2019^x\ln 2019\cdot f'\left(2019^x\right)-m$.\\
		Ta lại có hàm số $ y=2019^x$ đồng biến trên $\left[0;1\right]$.\\
		Với $ x\in\left[0;1\right]$ thì $2019^x\in\left[1;2019\right]$ mà hàm $ y=f'(x)$ đồng biến trên $\left(1;+\infty\right)$ nên hàm $ y=f'\left(2019^x\right)$ đồng biến trên $\left[0;1\right]$.\\
		Mà $2019^x\ge 1;f'\left(2019^x\right)>0\,\forall\,x\in\left[0;1\right]$ nên hàm $ h(x)=2019^x\ln 2019\cdot f'\left(2019^x\right)$ đồng biến trên $\left[0;1\right]$.\\
		Hay $ h(x)\ge h(0)=0,\forall\,x\in\left[0;1\right]$.\\
		Do vậy hàm số $ g(x)$ đồng biến trên đoạn $\left[0;1\right]$$\Leftrightarrow g'(x)\ge 0,\forall\,x\in\left[0;1\right]$\\
		$\Leftrightarrow m\le{2019^x}\ln 2019.f'\left(2019^x\right),\forall\,x\in\left[0;1\right]$ $\Leftrightarrow m\le\underset{x\in\left[0;1\right]}{\min}\,h(x)=h(0)=0$\\
		Vì $ m$ nguyên và $ m\in\left[-2019;2019\right]\Rightarrow $có $ 2020$ giá trị $ m$ thỏa mãn yêu cầu bài toán.}
\end{ex}

\begin{ex}%[2D1G1-3]%Câu 8
	\immini{
		Cho hàm số $y=f(x)$ có đồ thị $f'(x)\,$ như hình vẽ. Có bao nhiêu giá trị nguyên $m\in\left(-2020\,;\,2020\right)$ để hàm số $g(x)=f\left(2x-3\right)\,-\ln \left(1+x^2\right)-2mx$ đồng biến trên $\left(\dfrac{1}{2};2\right)$?
		\choice
		{$ 2020$}
		{\True $ 2019$}
		{$ 2021$}
		{$ 2018$}}{
		\begin{tikzpicture}[scale=0.9,font=\footnotesize, line join=round, line cap=round, >=stealth] %Đường cong bậc 3
			\draw[thick, ->] (-2.5,0)--(2.5,0);
			\draw[thick, ->] (0,-1.8)--(0,5.8);
			\draw (2.7,0) node[below] {$x$};
			\draw (0,5.9) node[left]{$y$};
			\draw (0,0) node[below left]{$0$};
			\draw[fill] (-2,0) circle (0.5pt)node[below]{$ -2 $};
			\draw[fill] (1,0) circle (0.5pt)node[below]{$ 1$};
			\draw[fill] (-1,0) circle (0.5pt)node[below]{$-1$};
			\draw[fill] (0,4) circle (0.5pt)node[above left]{$ 2$};
			%		\draw[fill] (0,2) circle (0.5pt)node[above left]{$ 2$};
			%		\draw[fill] (0,-2) circle (0.5pt)node[below left]{$ -2$};
			\draw[dashed] (-2,0)--(-2,4)--(1,4)--(1,0); 
			%		\draw[dashed](3,0)--(3,2)--(0,2);
			\draw[line width=1.2pt,smooth,samples=100,domain=-2.1:2.1] plot(\x,{-1*(\x)^3+0*(\x)^2+3*(\x)+2});		
			%\draw[line width=1.2pt,smooth,samples=100,domain=-3.3:2.8] plot(\x,{0.75*(\x)^2+0.5*\x-1});
			%	\draw (2.0,2.8) node[left]{$y=f'(x)$};
	\end{tikzpicture}	}
	\loigiai{
		Ta có $g'(x)=2f'\left(2x-3\right)-\dfrac{2x}{1+x^2}-2m$.\\
		Hàm số $ g(x)$ đồng biến trên $\left(\dfrac{1}{2};2\right)$ khi và chỉ khi \\
		$g'(x)\ge 0,\,\,\forall x\in\left(-1;\,2\right)$\\
		$\Leftrightarrow m\le{f}'\left(2x-3\right)-\dfrac{x}{1+x^2},\,\,\forall x\in\left(\dfrac{1}{2};2\right)$\\
		$\Leftrightarrow m\le\underset{x\in\left[\dfrac{1}{2};2\right]}{\min}\,\left[f'\left(2x-3\right)-\dfrac{x}{1+x^2}\right]$. \, \,  $(1)$\\
		Đặt $ t=2x-3$, khi đó $ x\in\left(\dfrac{1}{2};2\right)\Leftrightarrow t\in\left(-2;\,1\right)$.\\
		Từ đồ thị hàm $f'(x)$ suy ra $f'(t)\ge 0,\,\,\forall t\in\left(-2;1\right)$ và $f'(t)=0$ khi $ t=-1$.\\
		Tức là $f'\left(2x-3\right)\ge 0,\,\,\forall x\in\left(\dfrac{1}{2};\,2\right)$$\Rightarrow\underset{x\in\left[\dfrac{1}{2};2\right]}{\min}\,f'\left(2x-3\right)=0$ khi $ x=1$. $(2)$\\
		Xét hàm số $ h(x)=-\dfrac{x}{1+x^2}$ trên khoảng $\left(\dfrac{1}{2};\,2\right)$.\\
		Ta có $h'(x)=\dfrac{x^2-1}{\left(1+x^2\right)^2}$ và\\
		$h'(x)=0\Leftrightarrow{x^2}-1=0\Leftrightarrow x=\pm 1$.\\
		Bảng biến thiên của hàm số $ h(x)$ trên $\left(\dfrac{1}{2};\,2\right)$ như sau
		\begin{center}
			\begin{tikzpicture}
				\tkzTabInit[lgt=1.2,espcl=2.5,nocadre]
				{$x$ /0.7, $h'(x)$ /0.7,$h(x)$ /2.5}
				{$\dfrac{1}{2}$ , $1$,$2$}
				\tkzTabLine{,-,0,+,}
				\tkzTabVar{+/$  $ ,-/$ \-\dfrac{1}{2} $, +/$ $}
			\end{tikzpicture}
		\end{center}
		Từ bảng biến thiên suy ra $ h(x)\ge-\dfrac{1}{2}$$\Rightarrow\underset{x\in\left[\dfrac{1}{2};2\right]}{\min}\,h(x)=-\dfrac{1}{2}$ khi $ x=1$. \, \,  $(3)$\\
		Từ $(1)$, $(2)$ và $(3)$ suy ra $ m\le-\dfrac{1}{2}$.\\
		Kết hợp với $ m\in\mathbb{Z}$, $ m\in\left(-2020;\,2020\right)$ thì $ m\in\left\{-2019;\,-201;\ldots ;-2;-1\right\}$.\\
		Vậy có tất cả $ 2019$ giá trị $ m$ cần tìm.}
\end{ex}

\begin{ex}%[2D1G1-3]%Câu 9
	Cho hàm số $ f(x)$ liên tục trên $\mathbb{R}$ và có đạo hàm $f'(x)=x^2\left(x-2\right)\left(x^2-6x+m\right)$ với mọi $ x\in\mathbb{R}$. Có bao nhiêu số nguyên $ m$ thuộc đoạn $\left[-2020;2020\right]$ để hàm số $ g(x)=f\left(1-x\right)$ nghịch biến trên khoảng $\left(-\infty ;-1\right)$?
	\choice
	{$ 2016$}
	{$ 2014$}
	{\True $ 2012$}
	{$ 2010$}
	\loigiai{
		Ta có \\
		$g'(x)=f'\left(1-x\right)=-\left(1-x\right)^2\left(-x-1\right)\left[\left(1-x\right)^2-6\left(1-x\right)+m\right]$
		$=\left(x-1\right)^2\left(x+1\right)\left(x^2+4x+m-5\right)$.\\
		Hàm số $ g(x)$ nghịch biến trên khoảng $\left(-\infty ;-1\right)$\\
		$\Leftrightarrow{g}'(x)\le 0,\forall x<-1$ $(*)$, (dấu \lq\lq $=$\rq\rq \, xảy ra tại hữu hạn điểm).\\
		Với $ x<-1$ thì $\left(x-1\right)^2>0$ và $ x+1<0$ nên\\
		$(*)$ $\Leftrightarrow{x^2}+4x+m-5\ge 0,\forall x<-1 \Leftrightarrow m\ge-x^2-4x+5,\forall x<-1$.\\
		Xét hàm số $ y=-x^2-4x+5$ trên khoảng $\left(-\infty ;-1\right)$, ta có bảng biến thiên
		\begin{center}
			\begin{tikzpicture}
				\tkzTabInit[lgt=1.8,espcl=2.3]
				{$x$ /1.2, $y'$ /1.2,$y$ /2}
				{$-\infty$ , $-2$,$-1$}
				\tkzTabLine{,+,0,-,}
				\tkzTabVar{-/$ -\infty $ ,+/$9 $, -/$ 8$}
			\end{tikzpicture}
		\end{center}
		Từ bảng biến thiên suy ra $ m\ge 9$.\\
		Kết hợp với $ m$ thuộc đoạn $\left[-2020;2020\right]$ và $ m$ nguyên nên $ m\in\left\{ 9;10;11;\ldots ;2020\right\}$.\\
		Vậy có $ 2012$ số nguyên $ m$ thỏa mãn đề bài.}
\end{ex}

\begin{ex}%[2D1G1-3]%Câu 10
	\immini{
		Cho hàm số $f(x)$ xác định và liên tục trên $ R$. Hàm số $y=f'(x)$ liên tục trên $\mathbb{R}$ và có đồ thị như hình vẽ bên.
		Xét hàm số $g(x)=f\left(x-2m\right)+\dfrac{1}{2}{\left(2m-x\right)^2}+2020$, với $ m$ là tham số thực. Gọi $ S$ là tập hợp các giá trị nguyên dương của $ m$ để hàm số $ y=g(x)$ nghịch biến trên khoảng $\left(3;4\right)$. Hỏi số phần tử của $ S$ bằng bao nhiêu?
		\choice
		{$4$}
		{\True $2$}
		{$3$}
		{Vô số}}
	{
		\begin{tikzpicture}[scale=0.7,>=stealth, font=\footnotesize, line join=round, line cap=round]
			\def\xmin{-3.5} \def\xmax{4.5}
			\def\ymin{-5.2} \def\ymax{4}
			\clip(\xmin,\ymin) rectangle (\xmax,\ymax);
			\draw[->] (\xmin,0)--(\xmax,0) node [below]{$x$};
			\draw[->] (0,\ymin)--(0,\ymax) node [left]{$y$};
			\node at (0,0) [below left]{$O$};
			\path
			(-3.1,3.7) coordinate (A)
			(-3,3) coordinate (B)
			(0,-2) coordinate (C)
			(0.65,-2) coordinate (D)
			(1,-1) coordinate (E)
			(3,-3) coordinate (F)
			(3.4,-5) coordinate (G);
			\draw[smooth]
			(A)..controls +(-88:0.1) and +(93:.1)..
			(B)..controls +(-87:0.3) and +(-100:8.5)..
			(C)..controls +(75:.8) and +(180:.1)..
			(D)..controls +(0:.1) and +(-105:.3)..
			(E)..controls +(70:2) and +(97:0.4)..
			(F)..controls +(-80:.1) and +(90:0.3)..
			(G);
			\draw[dashed] 
			(-3,0)node[below]{$-3$}|-(0,3)node[right]{$3$}
			(1,0)node[above]{$1$}|-(0,-1)node[left]{$-1$}
			(3,0)node[above]{$3$}|-(0,-3)node[below right]{$-3$};
			\fill 
			(0,-2) circle(1.5pt)
			(-3,3) circle(1.5pt)
			(3,-3) circle(1.5pt)
			(1,-1) circle(1.5pt);
			\node at (2.1,-4) {$y=f'(x)$};
		\end{tikzpicture}
	}
	\loigiai{
		Ta có $g'(x)=f'\left(x-2m\right)-\left(2m-x\right)$.		Đặt $h(x)=f'(x)-\left(-x\right)$.\\
		Từ đồ thị hàm số $y=f'(x)$ và đồ thị hàm số $y=-x$ trên hình vẽ suy ra \\
		$h(x)\le 0\Leftrightarrow f'(x)\le-x\Leftrightarrow\hoac{
			&-3\le x\le 1\\ 
			& x\ge 3.}$ 
		\begin{center}
			\begin{tikzpicture}[scale=0.7,>=stealth, font=\footnotesize, line join=round, line cap=round]
				\def\xmin{-3.5} \def\xmax{4.5}
				\def\ymin{-5.2} \def\ymax{4}
				\clip(\xmin,\ymin) rectangle (\xmax,\ymax);
				\draw[->] (\xmin,0)--(\xmax,0) node [below]{$x$};
				\draw[->] (0,\ymin)--(0,\ymax) node [left]{$y$};
				\node at (0,0) [below left]{$O$};
				\path
				(-3.1,3.7) coordinate (A)
				(-3,3) coordinate (B)
				(0,-2) coordinate (C)
				(0.65,-2) coordinate (D)
				(1,-1) coordinate (E)
				(3,-3) coordinate (F)
				(3.4,-5) coordinate (G);
				\draw[smooth]
				(A)..controls +(-88:0.1) and +(93:.1)..
				(B)..controls +(-87:0.3) and +(-100:8.5)..
				(C)..controls +(75:.8) and +(180:.1)..
				(D)..controls +(0:.1) and +(-105:.3)..
				(E)..controls +(70:2) and +(97:0.4)..
				(F)..controls +(-80:.1) and +(90:0.3)..
				(G);
				\draw[dashed] 
				(-3,0)node[below]{$-3$}|-(0,3)node[right]{$3$}
				(1,0)node[above]{$1$}|-(0,-1)node[left]{$-1$}
				(3,0)node[above]{$3$}|-(0,-3)node[below right]{$-3$};
				\fill 
				(0,-2) circle(1.5pt)
				(-3,3) circle(1.5pt)
				(3,-3) circle(1.5pt)
				(1,-1) circle(1.5pt);
				\draw[smooth,samples=300,domain=-3.2:3.7] plot(\x,{-(\x)});
				\node at (2.1,-4) {$y=f'(x)$};
				\node at (-1,2.1) {$y=h(x)$};
			\end{tikzpicture}
		\end{center}
		Ta có $ g'(x)=h\left(x-2m\right)\le 0\Leftrightarrow\hoac{
			&-3\le x-2m\le 1\\ 
			& x-2m\ge 3}\Leftrightarrow\hoac{
			& 2m-3\le x\le 2m+1\\ 
			& x\ge 2m+3.}$.\\
		Suy ra hàm số $ y=g(x)$ nghịch biến trên các khoảng $\left(2m-3;2m+1\right)$ và $\left(2m+3;+\infty\right)$.\\
		Do đó hàm số $ y=g(x)$ nghịch biến trên khoảng $\left(3;4\right)$ $\Leftrightarrow\hoac{
			&\heva{
				& 2m-3\le 3\\ 
				& 2m+1\ge 4}\\ 
			& 2m+3\le 3}\Leftrightarrow\hoac{
			&\dfrac{3}{2}\le m\le 3\\ 
			& m\le 0.}$ \\
		Mặt khác, do $ m$ nguyên dương nên $ m\in\left\{ 2;3\right\}\Rightarrow S=\left\{ 2;3\right\}$. Vậy số phần tử của $ S$ bằng $2$.\\
	}
	
\end{ex}

\begin{ex}%[2D1G1-3]%Câu 11
	Cho hàm số $f(x)$ có đạo hàm trên $\mathbb{R}$ là $f'(x)=\left(x-1\right)\left(x+3\right)$. Có bao nhiêu giá trị nguyên của tham số $m$ thuộc đoạn $\left[-10;20\right]$ để hàm số $y=f\left(x^2+3x-m\right)$ đồng biến trên khoảng $\left(0;2\right)$?
	\choice
	{\True $ 18$}
	{$ 17$}
	{$ 16$}
	{$ 20$}
	\loigiai{
		Ta có $y'=f'\left(x^2+3x-m\right)=\left(2x+3\right){f}'\left(x^2+3x-m\right)$.\\
		Theo đề bài ta có $f'(x)=\left(x-1\right)\left(x+3\right)$\\
		suy ra $f'(x)>0\Leftrightarrow\hoac{
			& x<-3\\ 
			& x>1}$ và $f'(x)<0\Leftrightarrow-3<x<1$ .\\
		Hàm số đồng biến trên khoảng $\left(0;2\right)$ khi $y'\ge 0,\forall x\in\left(0;2\right)$\\
		$\Leftrightarrow\left(2x+3\right){f}'\left(x^2+3x-m\right)\ge 0,\forall x\in\left(0;2\right)$.\\
		Do $x\in\left(0;2\right)$ nên $2x+3>0,\forall x\in\left(0;2\right)$. Do đó, ta có\\
		$y'\ge 0,\forall x\in\left(0;2\right)\Leftrightarrow f'\left(x^2+3x-m\right)\ge 0$\\
		$\Leftrightarrow\hoac{
			&{x^2}+3x-m\le-3\\ 
			&{x^2}+3x-m\ge 1}\Leftrightarrow\hoac{
			& m\ge{x^2}+3x+3\\ 
			& m\le{x^2}+3x-1}$\\
		$\Leftrightarrow\hoac{
			& m\ge\underset{\left[0;2\right]}{\max}\,\left(x^2+3x+3\right)\\ 
			& m\le\underset{\left[0;2\right]}{\min}\,\left(x^2+3x-1\right)} \Leftrightarrow\hoac{
			& m\ge 13\\ 
			& m\le-1}$.\\
		Do $m\in\left[-10;20\right]$, $ m\in\mathbb{Z}$ nên có $ 18$ giá trị nguyên của $m$ thỏa yêu cầu đề bài.}
\end{ex}

\begin{ex}%[2D1G1-3]%Câu 12
	Cho các hàm số $f(x)=x^3+4x+m$ và $g(x)=\left(x^2+2018\right){\left(x^2+2019\right)^2}{\left(x^2+2020\right)^3}$ . Có bao nhiêu giá trị nguyên của tham số $m\in\left[-2020;2020\right]$ để hàm số $g\left(f(x)\right)$ đồng biến trên $\left(2;+\infty\right)$ ?
	\choice
	{$2005$}
	{\True $2037$}
	{$4016$}
	{$4041$}
	\loigiai{
		Ta có $f(x)=x^3+4x+m$ và \\
		$g(x)=\left(x^2+2018\right){\left(x^2+2019\right)^2}{\left(x^2+2020\right)^3}=a_{12}{x^{12}}+a_{10}{x^{10}}+...+a_2x^2+a_0$.\\
		Suy ra $f'(x)=3x^2+4$ , $g'(x)=12a_{12}{x^{11}}+10a_{10}{x^9}+...+2a_2x$.\\
		Và có 
		\begin{eqnarray*}
			\left[g\left(f(x)\right)\right]' &=& f'(x)\left[12a_{12}{\left(f(x)\right)^{11}}+10a_{10}{\left(f(x)\right)^9}+...+2a_2f(x)\right]\\
			&=& f(x)f'(x)\left(12a_{12}{\left(f(x)\right)^{10}}+10a_{10}{\left(f(x)\right)^8}+...+2a_2\right).
		\end{eqnarray*} 
		Dễ thấy $a_{12};{a_{10}};...;{a_2};{a_0}>0$ và $f'(x)=3x^2+4>0$, $\forall x>2$.\\
		Do đó $f'(x)\left(12a_{12}{\left(f(x)\right)^{10}}+10a_{10}{\left(f(x)\right)^8}+...+2a_2\right)>0$ , $\forall x>2$.\\
		Hàm số $g\left(f(x)\right)$ đồng biến trên $\left(2;+\infty\right)$ khi $\left[g\left(f(x)\right)\right]^{'}\ge 0$, $\forall x>2$\\
		$\Rightarrow  f(x)\ge 0$, $\forall x>2 \Leftrightarrow x^3+4x+m\ge 0$, $\forall x>3 \Leftrightarrow  m\ge-x^3-4x$, $\forall x>2$\\
		$ \Rightarrow  m\ge\underset{\left[2;+\infty\right)}{\max}\,\left(-x^3-4x\right)=-16$.\\
		Vì $m\in\left[-2020;2020\right]$ và $m\in\mathbb{Z}$ nên có $2037$ giá trị thỏa mãn $m$ .}
\end{ex}

\begin{ex}%[2D1G1-3]%Câu 13
	Cho hàm số $y=f(x)$ có đạo hàm $f'(x)=x{\left(x+1\right)^2}\left(x^2+2mx+1\right)$ với mọi $x \in \mathbb{R}$. Có bao nhiêu số nguyên âm $m$ để hàm số $g(x)=f\left(2x+1\right)$ đồng biến trên khoảng $\left(3;5\right)$?
	\choice
	{\True $3$}
	{$2$}
	{$4$}
	{$6$}
	\loigiai{
		Ta có $g'(x)=2f'(2x+1)=2(2x+1)(2x+2)^2[(2x+1)^2+2m(2x+1)+1]$. 	Đặt $t=2x+1$\\
		Để hàm số $g(x)$ đồng biến trên khoảng $\left(3;5\right)$ khi và chỉ khi 
		\begin{eqnarray*}
			& & g'(x)\ge 0,\forall x\in\left(3;5\right) \\
			& \Leftrightarrow & t(t^2+2mt+1)\ge 0,\forall t\in\left(7;11\right)\Leftrightarrow{t^2}+2mt+1\ge 0,\,\,\forall t\in\left(7;11\right) \\
			&\Leftrightarrow & 2m\ge\dfrac{-t^2-1}{t},\,\,\,\forall t\in\left(7;11\right)
		\end{eqnarray*}	
		Xét hàm số $h(t)=\dfrac{-t^2-1}{t}$ trên $\left[7;11\right]$, có $h'(t)=\dfrac{-t^2+1}{t^2}$\\
		Bảng biến thiên
		\begin{center}
			\begin{tikzpicture}
				\tkzTabInit[espcl=3,lgt=1.2,nocadre]
				{$t$/0.7,$h'(t)$/0.7,$h(t)$/2.5}
				{$-\infty$,$1$,$11$,$+\infty$}
				\tkzTabLine{, ,,-,,,}
				%	\node (0) at ($(N12)+(0,-3)$) {$-\infty$};
				\node (1) at ($(N22)+(0,-0.8)$) [right] {$-\dfrac{50}{7}$};
				\node (2) at ($(N32)+(0,-2.5)$) [left] {$-\dfrac{122}{11}$};
				
				
				%				\node (3) at ($(N11+(-0.5,0))$) {};
				%				\node (4) at ($(N23)$) {};
				\fill[pattern=north east lines] (7.0,-0.7) rectangle (10,-4.4);
				\fill[pattern=north east lines] (1.5,-0.7) rectangle (4.5,-4.4);
				\draw[->] (1)--(2);	
				\draw[dashed] (4.5,-0.7)--(4.5,-4.4);
				\draw[dashed] (7.0,-0.7)--(7.0,-4.4);	
			\end{tikzpicture}		
		\end{center}
		Dựa vào BBT ta có $2m\ge\dfrac{-t^2-1}{t},\,\,\,\forall t\in\left(7;11\right)\Leftrightarrow 2m\ge\underset{\left[7;11\right]}{\max}\,h(t)\Leftrightarrow m\ge-\dfrac{50}{14}$\\
		Vì $ m\in{\mathbb{Z}^-}\Rightarrow m \in \{-3;-2;-1\}$ .
	}
\end{ex}

\begin{ex}%[2D1G1-3]%Câu 14
	Cho hàm số $y=f(x)$ có bảng biến thiên như sau\\
	\begin{center}
		\begin{tikzpicture}[>=stealth,scale = 1]
			\tkzTabInit[lgt=1,espcl=2.5,nocadre]
			{$x$ /0.7, $y'$ /0.7,$y$ /2.5}
			{$-\infty$,$0$,$2$,$+\infty$}
			\tkzTabLine{ ,-,0,+,0,-,}
			\tkzTabVar{-/$-\infty$, +/$4$,- /$0$, +/{ $+\infty$}}
		\end{tikzpicture}
	\end{center}
	Có bao nhiêu số nguyên $m<2019$ để hàm số $g(x)=f\left(x^2-2x+m\right)$ đồng biến trên khoảng $\left(1;+\infty\right)$?
	\choice
	{\True $2016$}
	{$2015$}
	{$2017$}
	{$2018$}
	\loigiai{
		Ta có $g'(x)=\left(x^2-2x+m\right)'{f}'\left(x^2-2x+m\right)=2\left(x-1\right){f}'\left(x^2-2x+m\right)$ .\\
		Hàm số $y=g(x)$ đồng biến trên khoảng $\left(1;+\infty\right)$ khi và chỉ khi $g'(x)\ge 0,\forall x\in\left(1;+\infty\right)$ và\\
		$g'(x)=0$ tại hữu hạn điểm \\
		$\Leftrightarrow 2\left(x-1\right){f}'\left(x^2-2x+m\right)\ge 0,\forall x\in\left(1;+\infty\right)$\\
		$\Leftrightarrow{f}'\left(x^2-2x+m\right)\ge 0,\forall x\in\left(1;+\infty\right)$ $\Leftrightarrow\hoac{
			&{x^2}-2x+m\ge 2,\forall x\in\left(1;+\infty\right)\\ 
			&{x^2}-2x+m\le 0,\forall x\in\left(1;+\infty\right).}$\\
		Xét hàm số $y=x^2-2x+m$, ta có bảng biến thiên
		\begin{center}
			\begin{tikzpicture}[>=stealth,scale = 1]
				\tkzTabInit[lgt=1,espcl=2.5,nocadre]
				{$x$ /0.7, $y'$ /0.7,$y$ /2.5}
				{$-\infty$,$2$,$+\infty$}
				\tkzTabLine{ ,-,0,+,}
				\tkzTabVar{+/$+\infty$, -/$m-1$, +/{$+\infty$}}
			\end{tikzpicture}
		\end{center}
		Dựa vào bảng biến thiên ta có\\
		TH1: $x^2-2x+m\ge 2,\forall x\in\left(1;+\infty\right)\Leftrightarrow m-1\ge 2\Leftrightarrow m\ge 3$ .\\
		TH2: $x^2-2x+m\le 0,\forall x\in\left(1;+\infty\right)$. Không có giá trị $m$ thỏa mãn.\\
		Vậy có $2016$ số nguyên $m<2019$ thỏa mãn yêu cầu bài toán.}
\end{ex}

\begin{ex}%[2D1G1-3]%Câu 15
	\immini{
		Cho hàm số $ y=f(x)$ có đạo hàm là hàm số $f'(x)$ trên $\mathbb{R}$. Biết rằng hàm số $ y=f'\left(x-2\right)+2$ có đồ thị như hình vẽ bên dưới. Hàm số $ f(x)$ đồng biến trên khoảng nào?
		\choice
		{$\left(-\infty ;3\right),\,\,\left(5;+\infty\right)$}
		{\True $\left(-\infty ;-1\right),\,\,\left(1;+\infty\right)$}
		{$\left(-1;1\right)$}
		{$\left(3;5\right)$}}{
		\begin{tikzpicture}[scale=0.7,font=\footnotesize, line join=round, line cap=round, >=stealth] %Đường cong bậc 3
			\draw[thick, ->] (-0.5,0)--(3.5,0);
			\draw[thick, ->] (0,-1.8)--(0,5.3);
			\draw (3.7,0) node[below] {$x$};
			\draw (0,5.4) node[left]{$y$};
			\draw (0,0) node[below left]{$0$};
			\draw[fill] (3,0) circle (0.5pt)node[below]{$ 3$};
			\draw[fill] (1,0) circle (0.5pt)node[below]{$ 1$};
			\draw[fill] (2,0) circle (0.5pt)node[above]{$2$};
			\draw[fill] (0,2) circle (0.5pt)node[left]{$ 2$};
			\draw[fill] (0,-1) circle (0.5pt)node[left]{$ -1$};
			%		\draw[fill] (0,2) circle (0.5pt)node[above left]{$ 2$};
			%		\draw[fill] (0,-2) circle (0.5pt)node[below left]{$ -2$};
			\draw[dashed] (3,0)--(3,2)--(0,2)--(1,2)--(1,0); 
			\draw[dashed](0,-1)--(2,-1)--(2,0);
			\draw[line width=1.2pt,smooth,samples=100,domain=0.6:3.4] plot(\x,{3*(\x)^2-12*(\x)+11});		
			%\draw[line width=1.2pt,smooth,samples=100,domain=-3.3:2.8] plot(\x,{0.75*(\x)^2+0.5*\x-1});
			%	\draw (2.0,2.8) node[left]{$y=f'(x)$};
	\end{tikzpicture}	}
	\loigiai{	
		Hàm số $ y=f'\left(x-2\right)+2$ có đồ thị $(C)$ như sau:\\
		\begin{center}
			\begin{tikzpicture}[scale=0.7,font=\footnotesize, line join=round, line cap=round, >=stealth] %Đường cong bậc 3
				\draw[thick, ->] (-0.5,0)--(3.5,0);
				\draw[thick, ->] (0,-1.8)--(0,5.3);
				\draw (3.7,0) node[below] {$x$};
				\draw (0,5.4) node[left]{$y$};
				\draw (0,0) node[below left]{$0$};
				\draw[fill] (3,0) circle (0.5pt)node[below]{$ 3$};
				\draw[fill] (1,0) circle (0.5pt)node[below]{$ 1$};
				\draw[fill] (2,0) circle (0.5pt)node[above]{$2$};
				\draw[fill] (0,2) circle (0.5pt)node[left]{$ 2$};
				\draw[fill] (0,-1) circle (0.5pt)node[left]{$ -1$};
				%		\draw[fill] (0,2) circle (0.5pt)node[above left]{$ 2$};
				%		\draw[fill] (0,-2) circle (0.5pt)node[below left]{$ -2$};
				\draw[dashed] (3,0)--(3,2)--(0,2)--(1,2)--(1,0); 
				\draw[dashed](0,-1)--(2,-1)--(2,0);
				\draw[line width=1.2pt,smooth,samples=100,domain=0.6:3.4] plot(\x,{3*(\x)^2-12*(\x)+11});		
				%\draw[line width=1.2pt,smooth,samples=100,domain=-3.3:2.8] plot(\x,{0.75*(\x)^2+0.5*\x-1});
				%	\draw (2.0,2.8) node[left]{$y=f'(x)$};
			\end{tikzpicture}
		\end{center}
		Dựa vào đồ thị $(C)$ ta có\\ $f'\left(x-2\right)+2>2,\forall x\in\left(-\infty ;1\right)\cup\left(3;+\infty\right)\Leftrightarrow{f}'\left(x-2\right)>0,\forall x\in\left(-\infty ;1\right)\cup\left(3;+\infty\right)$ .\\
		Đặt $ x*=x-2$ suy ra $f'\left(x*\right)>0,\forall x*\in\left(-\infty ;-1\right)\bigcup\left(1;+\infty\right)$.\\
		Vậy hàm số $ f(x)$ đồng biến trên khoảng $\left(-\infty ;-1\right),\,\,\left(1;+\infty\right)$.}
\end{ex}

\begin{ex}%[2D1G1-2]%Câu 16
	\immini{
		Cho hàm số $ y=f(x)$ có đạo hàm là hàm số $f'(x)$ trên $\mathbb{R}$. Biết rằng hàm số $ y=f'\left(x+2\right)-2$ có đồ thị như hình vẽ bên dưới. Hàm số $ f(x)$ nghịch biến trên khoảng nào?
		\choice
		{$\left(-3;-1\right),\,\,\left(1;3\right)$}
		{\True $\left(-1;1\right),\,\,\left(3;5\right)$}
		{$\left(-\infty ;-2\right),\,\,\left(0;2\right)$}
		{$\left(-5;-3\right),\,\,\left(-1;1\right)$}}{
		\begin{tikzpicture}[scale=0.7,font=\footnotesize, line join=round, line cap=round, >=stealth] %Đường cong bậc 3
			\draw[thick, ->] (-3.8,0)--(4.0,0);
			\draw[thick, ->] (0,-4.8)--(0,3.5);
			\draw (4.2,0) node[below] {$x$};
			\draw (0,3.7) node[left]{$y$};
			\draw (0,0) node[below left]{$0$};
			\draw[fill] (-3,0) circle (0.5pt)node[above]{$ -3$};
			\draw[fill] (-1,0) circle (0.5pt)node[above]{$ -1$};
			\draw[fill] (1,0) circle (0.5pt)node[above]{$ 1$};
			\draw[fill] (3,0) circle (0.5pt)node[above]{$3$};
			\draw[fill] (0,2) circle (0.5pt)node[above left]{$ 2$};
			\draw[fill] (0,-1) circle (0.5pt)node[above right]{$ -1$};
			%		\draw[fill] (0,2) circle (0.5pt)node[above left]{$ 2$};
			%		\draw[fill] (0,-2) circle (0.5pt)node[below left]{$ -2$};
			\draw[dashed] (-3,0)--(-3,-2)--(3,-2)--(3,0) (-1,0)--(-1,-2) (1,0)--(1,-2) (-3.494,0)--(-3.494,2)--(3.494,2)--(3.494,0); 
			\draw[line width=1.2pt,smooth,samples=100,domain=-3.6:3.6] plot(\x,{0.11*(\x)^4-1.11*(\x)^2-1});		
			%\draw[line width=1.2pt,smooth,samples=100,domain=-3.3:2.8] plot(\x,{0.75*(\x)^2+0.5*\x-1});
			%	\draw (2.0,2.8) node[left]{$y=f'(x)$};
	\end{tikzpicture}	}
	\loigiai{
		Hàm số $ y=f'\left(x+2\right)-2$ có đồ thị $(C)$ như sau
		\begin{center}
			\begin{tikzpicture}[scale=0.7,font=\footnotesize, line join=round, line cap=round, >=stealth] %Đường cong bậc 3
				\draw[thick, ->] (-3.8,0)--(4.0,0);
				\draw[thick, ->] (0,-4.8)--(0,3.5);
				\draw (4.2,0) node[below] {$x$};
				\draw (0,3.7) node[left]{$y$};
				\draw (0,0) node[below left]{$0$};
				\draw[fill] (-3,0) circle (0.5pt)node[above]{$ -3$};
				\draw[fill] (-1,0) circle (0.5pt)node[above]{$ -1$};
				\draw[fill] (1,0) circle (0.5pt)node[above]{$ 1$};
				\draw[fill] (3,0) circle (0.5pt)node[above]{$3$};
				\draw[fill] (0,2) circle (0.5pt)node[above left]{$ 2$};
				\draw[fill] (0,-1) circle (0.5pt)node[above right]{$ -1$};
				%		\draw[fill] (0,2) circle (0.5pt)node[above left]{$ 2$};
				%		\draw[fill] (0,-2) circle (0.5pt)node[below left]{$ -2$};
				\draw[dashed] (-3,0)--(-3,-2)--(3,-2)--(3,0) (-1,0)--(-1,-2) (1,0)--(1,-2) (-3.494,0)--(-3.494,2)--(3.494,2)--(3.494,0); 
				\draw[line width=1.2pt,smooth,samples=100,domain=-3.6:3.6] plot(\x,{0.11*(\x)^4-1.11*(\x)^2-1});		
				%\draw[line width=1.2pt,smooth,samples=100,domain=-3.3:2.8] plot(\x,{0.75*(\x)^2+0.5*\x-1});
				%	\draw (2.0,2.8) node[left]{$y=f'(x)$};
			\end{tikzpicture}
		\end{center}
		Dựa vào đồ thị $(C)$ ta có\\
		$f'\left(x+2\right)-2<-2,\forall x\in\left(-3;-1\right)\bigcup\left(1;3\right)\Leftrightarrow{f}'\left(x+2\right)<0,\forall x\in\left(-3;-1\right)\bigcup\left(1;3\right)$.\\
		Đặt $ x^*=x+2$ suy ra: $f'\left(x^*\right)<0,\forall x^*\in\left(-1;1\right)\bigcup\left(3;5\right)$.\\
		Vậy: Hàm số $ f(x)$ đồng biến trên khoảng $\left(-1;1\right),\,\,\left(3;5\right)$.}
\end{ex}

\begin{ex}%[2D1G1-2]%Câu 17
	\immini{
		Cho hàm số $ y=f(x)$ có đạo hàm là hàm số $f'(x)$ trên $\mathbb{R}$. Biết rằng hàm số $ y=f'\left(x-2\right)+2$ có đồ thị như hình vẽ bên dưới. Hàm số $ f(x)$ nghịch biến trên khoảng nào?
		\choice
		{$\left(-\infty ;2\right)$}
		{\True $\left(-1;1\right)$}
		{$\left(\dfrac{3}{2};\dfrac{5}{2}\right)$}
		{$\left(2;+\infty\right)$}}{
		\begin{tikzpicture}[scale=0.7,font=\footnotesize, line join=round, line cap=round, >=stealth] %Đường cong bậc 3
			\draw[thick, ->] (-0.5,0)--(3.5,0);
			\draw[thick, ->] (0,-1.8)--(0,5.3);
			\draw (3.7,0) node[below] {$x$};
			\draw (0,5.4) node[left]{$y$};
			\draw (0,0) node[below left]{$0$};
			\draw[fill] (3,0) circle (0.5pt)node[below]{$ 3$};
			\draw[fill] (1,0) circle (0.5pt)node[below]{$ 1$};
			\draw[fill] (2,0) circle (0.5pt)node[above]{$2$};
			\draw[fill] (0,2) circle (0.5pt)node[left]{$ 2$};
			\draw[fill] (0,-1) circle (0.5pt)node[left]{$ -1$};
			%		\draw[fill] (0,2) circle (0.5pt)node[above left]{$ 2$};
			%		\draw[fill] (0,-2) circle (0.5pt)node[below left]{$ -2$};
			\draw[dashed] (3,0)--(3,2)--(0,2)--(1,2)--(1,0); 
			\draw[dashed](0,-1)--(2,-1)--(2,0);
			\draw[line width=1.2pt,smooth,samples=100,domain=0.6:3.4] plot(\x,{3*(\x)^2-12*(\x)+11});		
			%\draw[line width=1.2pt,smooth,samples=100,domain=-3.3:2.8] plot(\x,{0.75*(\x)^2+0.5*\x-1});
			%	\draw (2.0,2.8) node[left]{$y=f'(x)$};
	\end{tikzpicture}	}
	\loigiai{
		Hàm số $ y=f'\left(x-2\right)+2$ có đồ thị $(C)$ như sau
		\begin{center}
			\begin{tikzpicture}[scale=0.7,font=\footnotesize, line join=round, line cap=round, >=stealth] %Đường cong bậc 3
				\draw[thick, ->] (-0.5,0)--(3.5,0);
				\draw[thick, ->] (0,-1.8)--(0,5.3);
				\draw (3.7,0) node[below] {$x$};
				\draw (0,5.4) node[left]{$y$};
				\draw (0,0) node[below left]{$0$};
				\draw[fill] (3,0) circle (0.5pt)node[below]{$ 3$};
				\draw[fill] (1,0) circle (0.5pt)node[below]{$ 1$};
				\draw[fill] (2,0) circle (0.5pt)node[above]{$2$};
				\draw[fill] (0,2) circle (0.5pt)node[left]{$ 2$};
				\draw[fill] (0,-1) circle (0.5pt)node[left]{$ -1$};
				%		\draw[fill] (0,2) circle (0.5pt)node[above left]{$ 2$};
				%		\draw[fill] (0,-2) circle (0.5pt)node[below left]{$ -2$};
				\draw[dashed] (3,0)--(3,2)--(0,2)--(1,2)--(1,0); 
				\draw[dashed](0,-1)--(2,-1)--(2,0);
				\draw[line width=1.2pt,smooth,samples=100,domain=0.6:3.4] plot(\x,{3*(\x)^2-12*(\x)+11});		
				%\draw[line width=1.2pt,smooth,samples=100,domain=-3.3:2.8] plot(\x,{0.75*(\x)^2+0.5*\x-1});
				%	\draw (2.0,2.8) node[left]{$y=f'(x)$};
			\end{tikzpicture}
		\end{center}
		Dựa vào đồ thị $(C)$ ta có\\
		$f'\left(x-2\right)+2<2,\forall x\in\left(1;3\right)\Leftrightarrow{f}'\left(x-2\right)<0,\forall x\in\left(1;3\right)$.\\
		Đặt $ x^*=x-2$ thì $f'\left(x^*\right)<0,\forall x^*\in\left(-1;1\right)$.\\
		Vậy: Hàm số $ f(x)$ nghịch biến trên khoảng $\left(-1;1\right)$.\\
		Cách khác:\\
		Tịnh tiến sang trái hai đơn vị và xuống dưới $2$ đơn vị thì từ đồ thị $(C)$ sẽ thành đồ thị của hàm$ y=f'(x)$. Khi đó $f'(x)<0,\forall x\in\left(-1;1\right)$.\\
		Vậy hàm số $ f(x)$ nghịch biến trên khoảng $\left(-1;1\right)$.}
\end{ex}

\begin{ex}%[2D1G1-2]%Câu 18
	Cho hàm số $y=f(x)$ có đạo hàm cấp $ 3$ liên tục trên $\mathbb{R}$ và thỏa mãn $f(x)\cdot f'''(x)=x{\left(x-1\right)^2}{\left(x+4\right)^3}$ với mọi $x\in\mathbb{R}$ và $g(x)=\left[f'(x)\right]^2-2f(x)\cdot f''(x)$. Hàm số $h(x)=g\left(x^2-2x\right)$ đồng biến trên khoảng nào dưới đây?
	\choice
	{$\left(-\infty ;1\right)$}
	{$\left(2;+\infty\right)$}
	{$\left(0;1\right)$}
	{\True $\left(1;2\right)$}
	\loigiai{		
		Ta có $g'(x)=2f''(x){f}'(x)-2f'(x)\cdot f''(x)-2f(x)\cdot f'''(x)=-2f(x)\cdot f'''(x);$\\
		Khi đó $\left(h(x)\right)'=\left(2x-2\right){g}'\left(x^2-2x\right)=-2\left(2x-2\right)\left(x^2-2x\right){\left(x^2-2x-1\right)^2}{\left(x^2-2x+4\right)^3}$\\
		$h'(x)=0\Leftrightarrow\hoac{
			& x=0\\ 
			& x=1\\ 
			& x=2\\ 
			& x=1\pm\sqrt{2}.}$ 
		Ta có bảng xét dấu của $h'(x)$
		\begin{center}
			\begin{tikzpicture}
				\tkzTabInit[lgt=1.2,espcl=2,nocadre]
				{$t$/0.7, $h'(x)$ /.7} % first column
				{$-\infty$, $1-\sqrt{2}$,$0$, $1$,$2$,$1+\sqrt{2}$, $+\infty$} % first row
				\tkzTabLine { ,+,0,-,0,+,0,-,0,+,0,- } % second row
				%				\tkzTabLine {,-,z,+,t,+,} % third row
				%				\tkzTabLine {,+,d,-,z,+,} % last row
			\end{tikzpicture}
		\end{center}
		Suy ra hàm số $h(x)=g\left(x^2-2x\right)$ đồng biến trên khoảng $\left(1;2\right)$.}
\end{ex}

\begin{ex}%[2D1G1-2]%Câu 19
	Cho hàm số $ y=f(x)$ xác định trên $\mathbb{R}$. Hàm số $ y=g(x)=f'\left(2x+3\right)+2$ có đồ thị là một parabol với tọa độ đỉnh $ I\left(2;-1\right)$ và đi qua điểm $ A\left(1;2\right)$. Hỏi hàm số $ y=f(x)$ nghịch biến trên khoảng nào dưới đây?
	\choice
	{\True $\left(5;9\right)$}
	{$\left(1;2\right)$}
	{$\left(-\infty ;9\right)$}
	{$\left(1;3\right)$}
	\loigiai{	
		Xét hàm số $ g(x)=f'\left(2x+3\right)+2$ có đồ thị là một Parabol nên có phương trình dạng $ y=g(x)=a{x^2}+bx+c\,\,\,\,(P)$.\\
		Vì $(P)$ có đỉnh $ I\left(2;-1\right)$ nên $\heva{
			&\dfrac{-b}{2a}=2\\ 
			& g(2)=-1} \Leftrightarrow\heva{
			&-b=4a\\ 
			& 4a+2b+c=-1} \Leftrightarrow\heva{
			& 4a+b=0\\ 
			& 4a+2b+c=-1}$.\\
		Vì $(P)$ đi qua điểm $ A\left(1;2\right)$ nên $ g(1)=2\Leftrightarrow a+b+c=2$.\\
		Ta có hệ phương trình $\heva{
			& 4a+b=0\\ 
			& 4a+2b+c=-1\\ 
			& a+b+c=2} \Leftrightarrow\heva{
			& a=3\\ 
			& b=-12\\ 
			& c=11}$ nên $ g(x)=3x^2-12x+11$.\\
		Đồ thị của hàm $ y=g(x)$ là
		\begin{center}
			\begin{tikzpicture}[scale=0.7,font=\footnotesize, line join=round, line cap=round, >=stealth] %Đường cong bậc 3
				\draw[thick, ->] (-0.5,0)--(3.5,0);
				\draw[thick, ->] (0,-1.8)--(0,5.3);
				\draw (3.7,0) node[below] {$x$};
				\draw (0,5.4) node[left]{$y$};
				\draw (0,0) node[below left]{$0$};
				\draw[fill] (3,0) circle (0.5pt)node[below]{$ 3$};
				\draw[fill] (1,0) circle (0.5pt)node[below]{$ 1$};
				\draw[fill] (2,0) circle (0.5pt)node[above]{$2$};
				\draw[fill] (0,2) circle (0.5pt)node[left]{$ 2$};
				\draw[fill] (0,-1) circle (0.5pt)node[left]{$ -1$};
				%		\draw[fill] (0,2) circle (0.5pt)node[above left]{$ 2$};
				%		\draw[fill] (0,-2) circle (0.5pt)node[below left]{$ -2$};
				\draw[dashed] (3,0)--(3,2)--(0,2)--(1,2)--(1,0) (3.2,2)--(3,2); 
				\draw[dashed](0,-1)--(2,-1)--(2,0);
				\draw[line width=1.2pt,smooth,samples=100,domain=0.6:3.4] plot(\x,{3*(\x)^2-12*(\x)+11});		
				%\draw[line width=1.2pt,smooth,samples=100,domain=-3.3:2.8] plot(\x,{0.75*(\x)^2+0.5*\x-1});
				%	\draw (2.0,2.8) node[left]{$y=f'(x)$};
			\end{tikzpicture}	
		\end{center}
		Theo đồ thị ta thấy $ f'(2x+3)\le 0\Leftrightarrow f'(2x+3)+2\le 2\Leftrightarrow 1\le x\le 3$.\\
		Đặt $ t=2x+3\Leftrightarrow x=\dfrac{t-3}{2}$ khi đó $ f'(t)\le 0\Leftrightarrow 1\le\dfrac{t-3}{2}\le 3\Leftrightarrow 5\le t\le 9$.\\
		Vậy $ y=f(x)$ nghịch biến trên khoảng $\left(5;9\right)$.}
\end{ex}

\begin{ex}%[2D1G1-2]%Câu 20
	\immini{
		Cho hàm số $ y=f(x)$, hàm số $f'(x)=x^3+a{x^2}+bx+c\left(a,b,c\in\mathbb{R}\right)$ có đồ thị như hình vẽ bên.
		Hàm số $ g(x)=f\left(f'(x)\right)$ nghịch biến trên khoảng nào dưới đây?
		\choice
		{$\left(1;+\infty\right)$}
		{\True $\left(-\infty ;-2\right)$}
		{$\left(-1;0\right)$}
		{$\left(-\dfrac{\sqrt{3}}{3};\dfrac{\sqrt{3}}{3}\right)$}}{
		\begin{tikzpicture}[scale=0.8,font=\footnotesize, line join=round, line cap=round, >=stealth] %Đường cong bậc 3
			\draw[thick, ->] (-1.7,0)--(1.7,0);
			\draw[thick, ->] (0,-2.7)--(0,3.0);
			\draw (1.9,0) node[below] {$x$};
			\draw (0,3.2) node[left]{$y$};
			\draw (0,0) node[below left]{$0$};
			\draw[fill] (-1,0) circle (0.5pt)node[above left]{$ -1 $};
			\draw[fill] (1,0) circle (0.5pt)node[below right]{$ 1$};
			\draw[line width=1.2pt,smooth,samples=100,domain=-1.3:1.3] plot(\x,{2.667*(\x)^3+0*(\x)^2-2.667*\x});		
			%\draw[line width=1.2pt,smooth,samples=100,domain=-3.3:2.8] plot(\x,{0.75*(\x)^2+0.5*\x-1});
		\end{tikzpicture}	
	}
	\loigiai{	
		Vì các điểm $\left(-1;0\right),\left(0;0\right),\left(1;0\right)$ thuộc đồ thị hàm số $ y=f'(x)$ nên ta có hệ\\
		$\heva{
			&-1+a-b+c=0\\ 
			& c=0\\ 
			& 1+a+b+c=0} \Leftrightarrow\heva{
			& a=0\\ 
			& b=-1\\ 
			& c=0} \Rightarrow {f}'(x)=x^3-x\Rightarrow f''(x)=3x^2-1$.\\
		Ta có $ g(x)=f\left(f'(x)\right)\Rightarrow{g}'(x)=f'\left(f'(x)\right)\cdot f''(x)$.\\
		Xét \\
		$g'(x)=0\Leftrightarrow{g}'(x)=f'\left(f'(x)\right)\cdot f''(x)=0$\\
		$\Leftrightarrow {f}'\left(x^3-x\right)\left(3x^2-1\right)=0\Leftrightarrow\hoac{
			&{x^3}-x=0\\ 
			&{x^3}-x=1\\ 
			&{x^3}-x=-1\\ 
			& 3x^2-1=0} \Leftrightarrow \hoac{
			& x=\pm 1\\ 
			& x=0\\ 
			& x=x_1(x_1\approx 1,325)\\ 
			& x=x_2(x_2\approx-1,325)\\ 
			& x=\pm\dfrac{\sqrt{3}}{3}.}$\\
		Bảng biến thiên
		\begin{center}
			\begin{tikzpicture}
				\tkzTabInit[lgt=1.2,espcl=2,nocadre]
				{$t$/0.7, $h'(x)$ /.7} % first column
				{$-\infty$, $-1{,}325$,$-1$, $-\dfrac{\sqrt{3}}{3}$,$0$,$\dfrac{\sqrt{3}}{3}$,$1$,$1{,}325$, $+\infty$} % first row
				\tkzTabLine { ,-,0,+,0,-,0,+,0,-,0,+,0,-,0,+, } % second row
				%				\tkzTabLine {,-,z,+,t,+,} % third row
				%				\tkzTabLine {,+,d,-,z,+,} % last row
			\end{tikzpicture}
		\end{center}
		Dựa vào bảng biến thiên ta có $ g(x)$ nghịch biến trên $\left(-\infty ;-2\right)$}
\end{ex}
\Closesolutionfile{ans}
\indapan{10}{ans/CD1/Muc_9_10}
\chapter{THỂ TÍCH KHỐI ĐA DIỆN KHÁC}
\begin{Solution}{1}
C
\end{Solution}
\begin{Solution}{3}
B
\end{Solution}
\begin{Solution}{4}
A
\end{Solution}
\begin{Solution}{5}
A
\end{Solution}
\begin{Solution}{6}
A
\end{Solution}
\begin{Solution}{7}
B
\end{Solution}
\begin{Solution}{8}
A
\end{Solution}
\begin{Solution}{9}
C
\end{Solution}
\begin{Solution}{10}
B
\end{Solution}
\begin{Solution}{11}
C
\end{Solution}
\begin{Solution}{12}
D
\end{Solution}
\begin{Solution}{13}
B
\end{Solution}
\begin{Solution}{14}
D
\end{Solution}
\begin{Solution}{15}
A
\end{Solution}
\begin{Solution}{16}
B
\end{Solution}
\begin{Solution}{17}
C
\end{Solution}
\begin{Solution}{18}
C
\end{Solution}
\begin{Solution}{19}
C
\end{Solution}
\begin{Solution}{20}
B
\end{Solution}
\begin{Solution}{21}
C
\end{Solution}
\begin{Solution}{22}
B
\end{Solution}
\begin{Solution}{23}
D
\end{Solution}
\begin{Solution}{24}
B
\end{Solution}
\begin{Solution}{25}
D
\end{Solution}
\begin{Solution}{26}
D
\end{Solution}
\begin{Solution}{27}
B
\end{Solution}
\begin{Solution}{28}
A
\end{Solution}
\begin{Solution}{29}
C
\end{Solution}
\begin{Solution}{30}
B
\end{Solution}
\begin{Solution}{31}
D
\end{Solution}
\begin{Solution}{32}
B
\end{Solution}
\begin{Solution}{33}
B
\end{Solution}
\begin{Solution}{34}
C
\end{Solution}
\begin{Solution}{35}
D
\end{Solution}
\begin{Solution}{36}
B
\end{Solution}
\begin{Solution}{37}
B
\end{Solution}
\begin{Solution}{38}
A
\end{Solution}
\begin{Solution}{39}
A
\end{Solution}
\begin{Solution}{40}
D
\end{Solution}
\begin{Solution}{41}
C
\end{Solution}
\begin{Solution}{42}
B
\end{Solution}
\begin{Solution}{43}
A
\end{Solution}
\begin{Solution}{44}
A
\end{Solution}
\begin{Solution}{45}
D
\end{Solution}
\begin{Solution}{46}
C
\end{Solution}
\begin{Solution}{47}
A
\end{Solution}
\begin{Solution}{48}
B
\end{Solution}
\begin{Solution}{49}
B
\end{Solution}
\begin{Solution}{50}
B
\end{Solution}
\begin{Solution}{51}
A
\end{Solution}
\begin{Solution}{52}
A
\end{Solution}
\begin{Solution}{53}
C
\end{Solution}
\begin{Solution}{54}
C
\end{Solution}
\begin{Solution}{55}
C
\end{Solution}
\begin{Solution}{56}
B
\end{Solution}
\begin{Solution}{57}
C
\end{Solution}
\begin{Solution}{58}
C
\end{Solution}
\begin{Solution}{59}
B
\end{Solution}
\begin{Solution}{60}
C
\end{Solution}
\begin{Solution}{61}
A
\end{Solution}
\begin{Solution}{62}
B
\end{Solution}
\begin{Solution}{63}
B
\end{Solution}
\begin{Solution}{64}
D
\end{Solution}
\begin{Solution}{65}
D
\end{Solution}
\begin{Solution}{66}
B
\end{Solution}
\begin{Solution}{67}
A
\end{Solution}
\begin{Solution}{68}
D
\end{Solution}

\begin{Solution}{1}
D
\end{Solution}
\begin{Solution}{2}
B
\end{Solution}
\begin{Solution}{3}
C
\end{Solution}
\begin{Solution}{4}
B
\end{Solution}
\begin{Solution}{5}
C
\end{Solution}
\begin{Solution}{6}
C
\end{Solution}
\begin{Solution}{7}
A
\end{Solution}
\begin{Solution}{8}
D
\end{Solution}
\begin{Solution}{9}
C
\end{Solution}
\begin{Solution}{10}
D
\end{Solution}
\begin{Solution}{11}
C
\end{Solution}
\begin{Solution}{12}
D
\end{Solution}
\begin{Solution}{13}
A
\end{Solution}
\begin{Solution}{14}
A
\end{Solution}

\setcounter{ex}{0}
\setcounter{dang}{0}
\Opensolutionfile{ans}[ans/CD14/Muc10]
\begin{ex}%[2H1G3-4]
	Cho hình hộp $ABCD.A'B'C'D'$ có chiều cao bằng $8$ và diện tích đáy bằng $9$. Gọi $M$, $N$, $P$ và $Q$ lần lượt là tâm của các mặt bên $ABB'A'$, $BCC'B'$, $CDD'C'$ và $DAA'D'$ . Thể tích của khối đa diện lồi có các đỉnh là các điểm $A$, $B$, $C$, $D$, $M$, $N$, $P$ và $Q$ bằng
	\choice
	{$27$}
	{\True $30$}
	{$18$}
	{$36$}
	\loigiai{
		\immini{Gọi $E$, $F$, $G$, $H$ lần lượt là trung điểm của $AA'$, $BB'$, $CC'$, $DD'$.\\
			Khi đó $V_{ABCD.EFGH}=\dfrac12V_{ABCD.A'B'C'D'}=\dfrac12\cdot9\cdot8=36$\\
			Gọi $V$ là thể tích khối đa diện lồi cần tính, khi đó\\ $V=V_{ABCD.EFGH}-V_{E.AMQ}-V_{F.BMN}-V_{G.CNP}-V_{H.DPQ}$.\\
			Trong đó $V_{E.AMQ}=V_{F.BMN}=V_{G.CNP}=V_{H.DPQ}$\\ $=\dfrac{EQ}{EH}\cdot\dfrac{EM}{EF}\cdot V_{E.AHF}=\dfrac14\cdot\dfrac16\cdot V_{ABCD.EFGH}=\dfrac{36}{24}=\dfrac32$\\
			$\Rightarrow V=36-4\cdot\dfrac32=30$.}
		{
			\begin{tikzpicture}[line join=round,font=\scriptsize,scale=.9]
				\def\a{4}
				\pgfmathsetmacro{\am}{\a *sqrt(2)}
				\pgfmathsetmacro{\bm}{\am/3}
				\pgfmathsetmacro{\g}{20}
				\pgfmathsetmacro{\xa}{\am *cos(\g)/2}
				\pgfmathsetmacro{\ya}{\bm *sin(\g)/2}
				\pgfmathsetmacro{\xb}{\am *cos(\g+90)/2}
				\pgfmathsetmacro{\yb}{\bm *sin(\g+90)/2}
				\path(0,0)coordinate(O)
				(.5,\a)coordinate(O')
				(\xa,\ya)coordinate(A)
				(\xb,\yb)coordinate(B)
				(-\xa,-\ya)coordinate(C)
				(-\xb,-\yb)coordinate(D)
				(A)+(O')coordinate(A')
				(B)+(O')coordinate(B')
				(C)+(O')coordinate(C')
				(D)+(O')coordinate(D')
				(barycentric cs:A=1,B'=1)coordinate(M)
				(barycentric cs:B=1,C'=1)coordinate(N)
				(barycentric cs:C=1,D'=1)coordinate(P)
				(barycentric cs:D=1,A'=1)coordinate(Q)
				(barycentric cs:A=1,A'=1)coordinate(E)
				(barycentric cs:B=1,B'=1)coordinate(F)
				(barycentric cs:C=1,C'=1)coordinate(G)
				(barycentric cs:D=1,D'=1)coordinate(H);
				\draw[dashed] (C)--(B)--(A)(B)--(B');
				\draw[dashed] (M)--(N)--(P)--(Q)--cycle (E)--(F)--(G)
				(A)--(M)--(B)--(N)--(C);
				\draw (A')--(A)--(D)--(D')(C')--(C)--(D) (A')--(B')--(C')--(D')--cycle
				(G)--(H)--(E)(C)--(P)--(D)--(Q)--(A);
				\foreach \p/\q in{A/0,B/-90,C/-90,D/-90,A'/0,B'/90,C'/90,D'/90,M/90,N/90,P/90,Q/90,E/0,F/120,G/180,H/-30}\draw[fill=white](\p)circle(1pt)+(\q:.2)node{$\p$};
		\end{tikzpicture}}
	}
\end{ex}

\begin{ex}%[2H1G3-4]
	Cho hình chóp đều $ S.ABCD$ có cạnh đáy bằng $ a$, cạnh bên bằng $ 2a$ và $ O$ là tâm của đáy. Gọi $ M$,$ N$, $ P$, $ Q$ lần lượt là các điểm đối xứng với $ O$ qua trọng tâm của các tam giác $ SAB$, $ SBC$, $ SCD$, $ SDA$ và $ S'$ là điểm đối xứng với $ S$ qua $O$. Thể tích của khối chóp $ S'.MNPQ$ bằng
	\choice
	{\True $\dfrac{20\sqrt{14}{a^3}}{81}$}
	{$\dfrac{40\sqrt{14}{a^3}}{81}$}
	{$\dfrac{10\sqrt{14}{a^3}}{81}$}
	{$\dfrac{2\sqrt{14}{a^3}}{9}$}
	\loigiai{
		\begin{center}
			\begin{tikzpicture}[>=stealth, scale=1,samples=100,smooth]
				% \clip (-3.5,-2.5) rectangle (4.5,4.5);
				\def\h{7}
				\coordinate (O) at (0,0);
				\coordinate (S) at (0,\h);
				\coordinate (A) at (-2,1);
				\coordinate (B) at (-6,-1.5);
				\coordinate (C) at ($(A)!2!(O)$);
				\coordinate (D) at ($(B)!2!(O)$);
				\coordinate (E) at ($(A)!0.5!(B)$);
				\coordinate (F) at ($(B)!0.5!(C)$);
				\coordinate (G) at ($(C)!0.5!(D)$);
				\coordinate (H) at ($(A)!0.5!(D)$);
				\tkzCentroid(S,A,B) \tkzGetPoint{G1}
				\coordinate (M) at ($(O)!2!(G1)$);
				\tkzCentroid(S,B,C) \tkzGetPoint{G2}
				\coordinate (N) at ($(O)!2!(G2)$);
				\tkzCentroid(S,C,D) \tkzGetPoint{G3}
				\coordinate (P) at ($(O)!2!(G3)$);
				\tkzCentroid(S,A,D) \tkzGetPoint{G4}
				\coordinate (Q) at ($(O)!2!(G4)$);
				\coordinate (S') at ($(S)!2!(O)$);
				\path
				(intersection of S--B and M--G1) coordinate (L1)
				(intersection of S--B and M--Q) coordinate (L2)
				(intersection of S--D and M--Q) coordinate (L3)
				(intersection of S'--O and B--C) coordinate (L4);
				\draw (C)--(D)--(S)--(C)--(B)--(S)--(G) (M)--(N)--(P)--(Q) (L1)--(M)--(L2) (G2)--(N) (G3)--(P) (S)--(F) (L3)--(Q) (L4)--(S');
				\draw[dashed] (C)--(A)--(B)--(D)--(A)--(S)--(O) (S)--(E)--(F)--(G)--(H)--(E) (S)--(H) (G2)--(O)--(L1) (Q)--(O)--(G3)--(G2) (G1)--(G2) (G3)--(G4)--(G1) (L2)--(L3) (L4)--(O);
				\foreach \x/\g in {A/160,B/-90,C/-90,D/0,E/160,F/-90,G/-40,H/60,M/180,N/-90,P/0,Q/90/,O/-130,S/60,S'/0}
				\fill[black](\x) circle (1pt)
				($(\x)+(\g:3mm)$) node{\color{black}$\x$};
				\draw(G1)node[below left,yshift=0.1cm]{$G_1$} (G2) node[right,yshift=0.25cm]{$G_2$} (G3)node[right]{$G_3$} (G4)node[right]{$G_4$};
			\end{tikzpicture}
		\end{center}
		Gọi $G_1,G_2,G_3,G_4$ lần lượt là trọng tâm $\triangle SAB$, $\triangle SBC$, $\triangle SCD$, $\triangle SDA$ .\\
		$E$, $F$, $G$, $H$ lần lượt là trung điểm của các cạnh $AB$, $BC$, $CD$, $DA$ .\\
		Ta có $S_{MNPQ}=4S_{G_1G_2G_3G_4}=4.\dfrac{4}{9}{S_{EFGH}}=4.\dfrac{4}{9}\cdot\dfrac{1}{2}EG\cdot HF=\dfrac{8a^2}{9}$ .\\
		$\begin{array}{l}
			\mathrm{d}\left(S',\left(MNPQ\right)\right)=\mathrm{d}\left(S',\left(ABCD\right)\right)+\mathrm{d}\left(O,\left(MNPQ\right)\right)\\
			=\mathrm{d}\left(S,\left(ABCD\right)\right)+2\mathrm{d}\left(O,\left(G_1G_2G_3G_4\right)\right)\\
			=\mathrm{d}\left(S,\left(ABCD\right)\right)+\dfrac{2}{3}d\left(S,\left(ABCD\right)\right)\\
			=\dfrac{5}{3}\mathrm{d}\left(S,\left(ABCD\right)\right)=\dfrac{5a\sqrt{14}}{6}
		\end{array}$\\
		Vậy $V_{S'.MNPQ}=\dfrac{1}{3}\cdot\dfrac{5a\sqrt{14}}{6}\cdot\dfrac{8a^2}{9}=\dfrac{20a^3\sqrt{14}}{81}$.}
\end{ex}

\begin{ex}%[2H1G3-4]
	Cho hình chóp đều $S.ABCD$ có cạnh đáy bằng $a$, cạnh bên bằng $a\sqrt{3}$ và $O$ là tâm của đáy. Gọi $M$, $N$, $P$, $Q$ lần lượt là các điểm đối xứng với $O$ qua trọng tâm các tam giác $SAB$, $SBC$, $SCD$, $SDA$ và $S'$ là điểm đối xứng  với $S$ qua $O$. Thể tích của khối chóp $S'.MNPQ$ bằng
	\choice
	{$\dfrac{40\sqrt{10}a^{3}}{81}$}
	{$\dfrac{10\sqrt{10}a^{3}}{81}$}
	{\True $\dfrac{20\sqrt{10}a^{3}}{81}$}
	{$\dfrac{2\sqrt{10}a^{3}}{9}$}
	\loigiai{
		\begin{center}
			\begin{tikzpicture}[>=stealth, scale=.8,samples=100,smooth]
				% \clip (-3.5,-2.5) rectangle (4.5,4.5);
				\def\h{7}
				\coordinate (O) at (0,0);
				\coordinate (S) at (0,\h);
				\coordinate (A) at (-2,1);
				\coordinate (B) at (-6,-1.5);
				\coordinate (C) at ($(A)!2!(O)$);
				\coordinate (D) at ($(B)!2!(O)$);
				\coordinate (E) at ($(A)!0.5!(B)$);
				\coordinate (F) at ($(B)!0.5!(C)$);
				\coordinate (G) at ($(C)!0.5!(D)$);
				\coordinate (H) at ($(A)!0.5!(D)$);
				\tkzCentroid(S,A,B) \tkzGetPoint{G1}
				\coordinate (M) at ($(O)!2!(G1)$);
				\tkzCentroid(S,B,C) \tkzGetPoint{G2}
				\coordinate (N) at ($(O)!2!(G2)$);
				\tkzCentroid(S,C,D) \tkzGetPoint{G3}
				\coordinate (P) at ($(O)!2!(G3)$);
				\tkzCentroid(S,A,D) \tkzGetPoint{G4}
				\coordinate (Q) at ($(O)!2!(G4)$);
				\coordinate (S') at ($(S)!2!(O)$);
				\path
				(intersection of S--B and M--G1) coordinate (L1)
				(intersection of S--B and M--Q) coordinate (L2)
				(intersection of S--D and M--Q) coordinate (L3)
				(intersection of S'--O and B--C) coordinate (L4);
				\draw (C)--(D)--(S)--(C)--(B)--(S)--(C) (S)--(G) (M)--(N)--(P)--(Q) (L1)--(M)--(L2) (G2)--(N) (G3)--(P) (S)--(F) (L3)--(Q) (L4)--(S')--(P) (M)--(S')--(N);
				\draw[dashed] (C)--(A)--(B)--(D)--(A)--(S)--(O) (S)--(E)--(F)--(G)--(H)--(E) (S)--(H) (G2)--(O)--(L1) (Q)--(O)--(G3)--(G2) (G1)--(G2) (G3)--(G4)--(G1) (L2)--(L3) (L4)--(O) (Q)--(S');
				\foreach \x/\g in {A/160,B/-90,C/-90,D/0,O/-130,S/60,S'/0}
				\fill[black](\x) circle (1pt)
				($(\x)+(\g:3mm)$) node{\color{black}$\x$};
				\draw(G1)node[below left,yshift=0.1cm]{$G_1$} (G2) node[right,yshift=0.25cm]{$G_2$} (G3)node[right]{$G_3$} (G4)node[right]{$G_4$};
				\path
				(M)node[left]{$Q$}
				(Q)node[above]{$M$}
				(P)node[right]{$N$}
				(N)node[below]{$P$};
			\end{tikzpicture}
		\end{center}
		Ta gọi $G_1$, $G_2$, $G_3$, $G_4$ lần lượt là trọng tâm của tam giác $SAB$, $SBC$, $SCD$, $SDA$ thì
		\begin{eqnarray*}
			\mathrm{d}\left(S',\left(MNPQ\right)\right)&=& \dfrac{5}{2}\mathrm{d}\left(O,\left(MNPQ\right)\right)\\
			&\Rightarrow & {V_{S'.MNPQ}}=\dfrac{5}{2}{V_{O.MNPQ}}=\dfrac{5}{2}\cdot8V_{O.G_1G_2G_3G_4}\\
			&= & 10V_{S.G_1G_2G_3G_4}=10\cdot\dfrac{4}{27}{V_{S.ABCD}}\\
			& =&\dfrac{40}{27}\cdot\dfrac{1}{3}\cdot\dfrac{a\sqrt{10}}{2}\cdot a^2=\dfrac{20\sqrt{10}{a^3}}{81}.
		\end{eqnarray*}
	}
\end{ex}

\begin{ex}%[2H1G3-4]
	Cho hình chóp đều $S.ABCD$ có cạnh đáy bằng $a$, cạnh bên bằng $\sqrt 2a$ và $O$ là tâm của đáy. Gọi $M$, $N$, $P$, $Q$ lần lượt là các điểm đối xứng với $O$ qua trọng tâm của các tam giác $SAB$, $SBC$, $SCD$, $SDA$ và $S'$ là điểm đối xứng với $S$ qua $O$. Thể tích khối chóp $S'.MNPQ$ bằng.
	\choice
	{$\dfrac{2\sqrt 6a^3}{9}$}
	{$\dfrac{40\sqrt 6a^3}{81}$}
	{$\dfrac{10\sqrt 6a^3}{81}$}
	{\True $\dfrac{20\sqrt 6a^3}{81}$}
	\loigiai{
		\begin{center}
			\begin{center}
				\begin{tikzpicture}[>=stealth, scale=.8,samples=100,smooth]
					\def\h{7}
					\coordinate (O) at (0,0);
					\coordinate (S) at (0,\h);
					\coordinate (A) at (-2,1);
					\coordinate (B) at (-6,-1.5);
					\coordinate (C) at ($(A)!2!(O)$);
					\coordinate (D) at ($(B)!2!(O)$);
					\coordinate (E) at ($(A)!0.5!(B)$);
					\coordinate (F) at ($(B)!0.5!(C)$);
					\coordinate (G) at ($(C)!0.5!(D)$);
					\coordinate (H) at ($(A)!0.5!(D)$);
					\tkzCentroid(S,A,B) \tkzGetPoint{G1}
					\coordinate (M) at ($(O)!2!(G1)$);
					\tkzCentroid(S,B,C) \tkzGetPoint{G2}
					\coordinate (N) at ($(O)!2!(G2)$);
					\tkzCentroid(S,C,D) \tkzGetPoint{G3}
					\coordinate (P) at ($(O)!2!(G3)$);
					\tkzCentroid(S,A,D) \tkzGetPoint{G4}
					\coordinate (Q) at ($(O)!2!(G4)$);
					\coordinate (S') at ($(S)!2!(O)$);
					\coordinate (K) at ($(M)!0.5!(P)$);
					\path
					(intersection of S--B and M--G1) coordinate (L1)
					(intersection of S--B and M--Q) coordinate (L2)
					(intersection of S--D and M--Q) coordinate (L3)
					(intersection of S'--O and B--C) coordinate (L4)
					(intersection of S--B and M--P) coordinate (L5)
					(intersection of S--D and M--P) coordinate (L6)
					(intersection of S--D and N--Q) coordinate (L7)
					;
					\draw (C)--(D)--(S)--(C)--(B)--(S) (M)--(N)--(P)--(Q) (L1)--(M)--(L2) (G2)--(N) (G3)--(P) (L3)--(Q) (L4)--(S') (M)--(L5) (P)--(L6) (Q)--(L7);
					\draw[dashed] (C)--(A)--(B)--(D)--(A)--(S)--(O) (G2)--(O)--(L1) (Q)--(O)--(G3)--(G2) (G1)--(G2) (G3)--(G4)--(G1) (L2)--(L3) (L4)--(O) (L5)--(L6) (N)--(L7);
					\foreach \x/\g in {A/160,B/-90,C/-90,D/0,M/180,N/-90,P/0,Q/90/,O/-130,S/60,S'/0,K/120}
					\fill[black](\x) circle (1pt)
					($(\x)+(\g:3mm)$) node{\color{black}$\x$};
				\end{tikzpicture}
			\end{center}
		\end{center}
		Ta có $S'K=S'O+OK=SO+\dfrac{2}{3}SO=\dfrac{5a\sqrt 6}{6}\cdot $\\
		Ta có ${MNPQ}=4\cdot\dfrac{1}{2}\cdot\dfrac{4}{9}{S_{ABCD}}=\dfrac{8}{9}{a^2}.$\\
		Vậy $V_{S'.MNPQ}=\dfrac{20\sqrt 6a^3}{81}\cdot $}
\end{ex}

\begin{ex}%[2H1G3-4]
	Cho hình chóp đều $S.ABCD$ có tất cả các cạnh bằng $a$ và $O$ là tâm của đáy. Gọi $M$, $N$, $P$, $Q$ lần lượt là các điểm đối xứng với $O$ qua trọng tâm của các tam giác $SAB$, $SBC$, $SCD$, $SDA$ và $S'$ là điểm đối xứng với $S$ qua $O$. Thể tích khối chóp $S'.MNPQ$ bằng
	\choice
	{$\dfrac{2\sqrt 2a^3}{9}$}
	{\True $\dfrac{20\sqrt 2a^3}{81}$}
	{$\dfrac{40\sqrt 2a^3}{81}$}
	{$\dfrac{10\sqrt 2a^3}{81}$}
	\loigiai{
		\begin{center}
			\begin{tikzpicture}[>=stealth, scale=1,samples=100,smooth]
				\def\h{7}
				\coordinate (O) at (0,0);
				\coordinate (S) at (0,\h);
				\coordinate (A) at (-2,1);
				\coordinate (B) at (-6,-1.5);
				\coordinate (C) at ($(A)!2!(O)$);
				\coordinate (D) at ($(B)!2!(O)$);
				\coordinate (E) at ($(A)!0.5!(B)$);
				\coordinate (F) at ($(B)!0.5!(C)$);
				\coordinate (G) at ($(C)!0.5!(D)$);
				\coordinate (H) at ($(A)!0.5!(D)$);
				\tkzCentroid(S,A,B) \tkzGetPoint{G1}
				\coordinate (M) at ($(O)!2!(G1)$);
				\tkzCentroid(S,B,C) \tkzGetPoint{G2}
				\coordinate (N) at ($(O)!2!(G2)$);
				\tkzCentroid(S,C,D) \tkzGetPoint{G3}
				\coordinate (P) at ($(O)!2!(G3)$);
				\tkzCentroid(S,A,D) \tkzGetPoint{G4}
				\coordinate (Q) at ($(O)!2!(G4)$);
				\coordinate (S') at ($(S)!2!(O)$);
				\coordinate (I) at ($(M)!0.5!(P)$);
				\path
				(intersection of S--B and M--G1) coordinate (L1)
				(intersection of S--B and M--Q) coordinate (L2)
				(intersection of S--D and M--Q) coordinate (L3)
				(intersection of S'--O and B--C) coordinate (L4)
				(intersection of S--B and M--P) coordinate (L5)
				(intersection of S--D and M--P) coordinate (L6)
				(intersection of S--D and N--Q) coordinate (L7)
				;
				\draw (C)--(D)--(S)--(C)--(B)--(S) (M)--(N)--(P)--(Q) (L1)--(M)--(L2) (G2)--(N) (G3)--(P) (L3)--(Q) (L4)--(S') (M)--(L5) (L6)--(P) (L7)--(Q);
				\draw[dashed] (C)--(A)--(B)--(D)--(A)--(S)--(O)  (G2)--(O)--(L1) (Q)--(O)--(G3)--(G2) (G1)--(G2) (G3)--(G4)--(G1) (L2)--(L3) (L4)--(O) (L5)--(L6) (N)--(L7);
				\foreach \x/\g in {A/160,B/-90,C/-90,D/0,M/180,N/-90,P/0,Q/90/,O/-130,S/60,S'/0,I/120}
				\fill[black](\x) circle (1pt)
				($(\x)+(\g:3mm)$) node{\color{black}$\x$};
				\draw(G1)node[below left,yshift=0.1cm]{$G$} (G3)node[right]{$K$};
			\end{tikzpicture}
		\end{center}
		Ta có $SO=\dfrac{a\sqrt 2}{2}$\\
		Gọi $G,K$ lần lượt là trọng tâm của tam giác $SAB$ và tam giác $SCD$.\\
		Suy ra $MP=2GK=\dfrac{4}{3}a$, tương tự $NQ=\dfrac{4}{3}a$.\\
		$\Rightarrow{S_{MNPQ}}=\dfrac{8}{9}{a^2}$.\\
		Ta có $\left(MNPQ\right)\parallel\left(ABCD\right)$\\
		$\mathrm{d}\left(M,\left(ABCD\right)\right)=2\mathrm{d}\left(G,\left(ABCD\right)\right)=\dfrac{2}{3}SO=\dfrac{a\sqrt 2}{3}$.\\
		$\Rightarrow \mathrm{d}\left(\left(MNPQ\right),\left(ABCD\right)\right)=\dfrac{a\sqrt 2}{3}$\\
		$\Rightarrow \mathrm{d}\left(S',\left(MNPQ\right)\right)=S'O+\dfrac{a\sqrt 2}{3}=\dfrac{5a\sqrt 2}{6}$\\
		$\Rightarrow{V_{S'.MNPQ}}=\dfrac{1}{3}.\dfrac{5a\sqrt 2}{6}.\dfrac{8a^2}{9}=\dfrac{20\sqrt 2a^3}{81}$.}
\end{ex}

\begin{ex}%[2H1G3-4]
	Cho hình chóp đều $S.ABCD$ có cạnh đáy bằng $4a$ , cạnh bên bằng $2\sqrt 3 a$ và $O$ là tâm của đáy. Gọi $M,N,P$ và $Q$ lần lượt là hình chiếu vuông góc của $O$ lên các mặt phẳng $\left(SAB\right)$, $\left(SBC\right)$, $\left(SCD\right)$ và $\left(SDA\right)$. Thể tích khối chóp $O.MNPQ$ bằng
	\choice
	{$\dfrac{4a^3}{3}$}
	{$\dfrac{64a^3}{81}$}
	{$\dfrac{128a^3}{81}$}
	{\True $\dfrac{2a^3}{3}$}
	\loigiai{
		\begin{center}
			\begin{tikzpicture}[>=stealth, scale=1,samples=100,smooth,xscale=1,yscale=1]
				\def\h{7}
				\coordinate (O) at (0,0);
				\coordinate (A) at (-2,1);
				\coordinate (B) at (-6,0);
				\coordinate (C) at ($(A)!2!(O)$);
				\coordinate (D) at ($(B)!2!(O)$);
				\coordinate (S) at (0,\h);
				\coordinate (E) at ($(A)!0.5!(B)$);
				\coordinate (F) at ($(B)!0.5!(C)$);
				\coordinate (G) at ($(C)!0.5!(D)$);
				\coordinate (H) at ($(A)!0.5!(D)$);
				\coordinate (M) at ($(S)!0.5!(E)$);
				\coordinate (N) at ($(S)!0.5!(F)$);
				\coordinate (P) at ($(S)!0.5!(G)$);
				\coordinate (Q) at ($(S)!0.5!(H)$);
				\draw (S)--(B)--(C)--(D)--(S)--(C) (F)--(S)--(G);
				\draw[dashed] (D)--(A)--(B)--(D) (C)--(A)--(S)--(O)--(M) (S)--(H) (S)--(E)--(F)--(G)--(H)--(E) (P)--(O)--(N)--(M)--(Q) (O)--(Q)--(P)--(N);
				\foreach \x/\g in {A/110,B/-90,C/-90,D/0,S/40,E/110,F/-90,G/-30,H/45,M/180,N/-45,P/0,Q/60,O/-90}
				\fill[black](\x) circle (1pt)
				($(\x)+(\g:3mm)$) node{\color{black}$\x$};
			\end{tikzpicture}
		\end{center}
		Gọi $E$ là trung điểm của $AB$, vẽ $OM\perp SE$ suy ra $OM\perp\left(SAB\right)$.\\
		$SO=\sqrt{SB^2-OB^2}=\sqrt{12a^2-8a^2}=2a$ và $SM\cdot SE=SO^2$.\\
		Suy ra $\dfrac{SM}{SE}=\dfrac{SO^2}{SE^2}=\dfrac{4a^2}{8a^2}=\dfrac{1}{2}$ suy ra $M$ là trung điểm của $SE$.\\
		Chứng minh tương tự đối với $N$, $P$, $Q$.\\
		Suy ra $MNPQ$ là hình vuông cạnh $\dfrac{AC}{4}=\sqrt 2 a$\\
		$\mathrm{d}\left(O,\left(MNPQ\right)\right)=\mathrm{d}\left(S,\left(MNPQ\right)\right)=\dfrac{SO}{2}=a$\\
		$\Rightarrow{V_{O.MNPQ}}=\dfrac{1}{3}a\cdot 2a^2=\dfrac{2a^3}{3}$.
	}
\end{ex}

\begin{ex}%[2H1G3-4]
	Cho hình chóp đều $S.ABCD$ có cạnh đáy bằng $a$, cạnh bên bằng $\dfrac{\sqrt{3}a}{2}$ và $O$ là tâm của đáy. Gọi $M$, $N$, $P$ và $Q$ lần lượt là hình chiếu vuông góc của $O$ trên các mặt phẳng $\left( SAB \right)$, $\left( SBC \right)$, $\left( SCD \right)$ và $\left( SDA \right)$. Thể tích của khối chóp $O.MNPQ$ bằng
	\choice
	{$\dfrac{a^{3}}{48}$}
	{$\dfrac{2a^{3}}{81}$}
	{$\dfrac{a^{3}}{81}$}
	{\True $\dfrac{a^{3}}{96}$}
	\loigiai{
		\begin{center}
			\begin{tikzpicture}[>=stealth, scale=1,samples=100,smooth,xscale=1,yscale=1]
				\def\h{7}
				\coordinate (O) at (0,0);
				\coordinate (A) at (-2,1);
				\coordinate (B) at (-6,0);
				\coordinate (C) at ($(A)!2!(O)$);
				\coordinate (D) at ($(B)!2!(O)$);
				\coordinate (S) at (0,\h);
				\coordinate (M') at ($(A)!0.5!(B)$);
				\coordinate (N') at ($(B)!0.5!(C)$);
				\coordinate (P') at ($(C)!0.5!(D)$);
				\coordinate (Q') at ($(A)!0.5!(D)$);
				\coordinate (M) at ($(S)!0.5!(M')$);
				\coordinate (N) at ($(S)!0.5!(N')$);
				\coordinate (P) at ($(S)!0.5!(P')$);
				\coordinate (Q) at ($(S)!0.5!(Q')$);
				\coordinate (I) at ($(S)!0.5!(O)$);
				\draw (S)--(B)--(C)--(D)--(S)--(C) (P')--(S)--(N');
				\draw[dashed] (D)--(A)--(B)--(D) (C)--(A)--(S)--(O)--(M) (S)--(Q') (S)--(M')--(N')--(P')--(Q')--(M') (P)--(O)--(N)--(M)--(Q) (O)--(Q)--(P)--(N);
				\foreach \x/\g in {A/110,B/-90,C/-90,D/0,S/40,M'/110,N'/-90,P'/-30,Q'/45,M/180,N/-45,P/0,Q/60,O/-90,I/30}
				\fill[black](\x) circle (1pt)
				($(\x)+(\g:3mm)$) node{\color{black}$\x$};
			\end{tikzpicture}
		\end{center}
		Gọi $M'$, $N'$, $P'$, $Q'$ lần lượt là trung điểm của các cạnh $AB$, $BC$, $CD$, $DA$.\\
		Ta có $AB\perp OM'$ và $AB\perp SO$ nên $AB\perp \left( SOM'\right)$.\\
		Suy ra $\left( SAB \right)\perp \left( SO{M}' \right)$ theo giao tuyến $SM'$.\\
		Theo giả thiết ta có $OM\perp \left( SAB \right)$ nên $OM\perp SM'$, do đó $M$ là hình chiếu vuông góc của $O$ trên $SM'$.\\
		Tương tự như vậy: $N,P,Q$ là hình chiếu vuông góc của $O$ lần lượt trên $SN'$, $SP'$, $SQ'$.\\
		Ta có $SO=\sqrt{SA^{2}-AO^{2}}=\sqrt{\dfrac{3a^{2}}{4}-\dfrac{2a^{2}}{4}}=\dfrac{a}{2}=OM'$.
		Suy ra tam giác $SOM'$ vuông cân tại $O$ nên $M$ là trung điểm của $SM'$.\\
		Từ đó dễ chứng minh được $MNPQ$ là hình vuông có tâm $I$ thuộc $SO$ và nằm trong mặt phẳng song song với $\left( ABCD \right)$, với $I$ là trung điểm của $SO$.
		Suy ra $OI=\dfrac{1}{2}OS=\dfrac{a}{4}$.\\
		Do đó $MN=\dfrac{1}{2}M'N'=\dfrac{1}{4}AC=\dfrac{\sqrt{2}a}{4}$.\\
		Thể tích khối chóp $O.MNPQ$ bằng $\dfrac{1}{3}{{S}_{MNPQ}}\cdot OI=\dfrac{1}{3}\cdot MN^{2}\cdot OI=\dfrac{1}{3}\cdot \dfrac{a^{2}}{8}\cdot\dfrac{a}{4}=\dfrac{{a^{3}}}{96}$.
	}
\end{ex}

\begin{ex}%[2H1G3-4]
	Cho hình chóp đều $ S.ABCD$ có cạnh đáy bằng $ 3a$, cạnh bên bằng $\dfrac{3a\sqrt 3}{2}$ và $ O$ là tâm của đáy. Gọi $ M$, $ N$, $ P$ và $ Q$ lần lượt là hình chiếu vuông góc của $ O$ trên các mặt phẳng $ (SAB)$, $ (SBC)$, $ (SCD)$ và $ (SAD)$. Thể tích khối chóp $ O.MNPQ$ bằng
	\choice
	{$\dfrac{9a^3}{16}$}
	{$\dfrac{2a^3}{3}$}
	{\True $\dfrac{9a^3}{32}$}
	{$\dfrac{a^3}{3}$}
	\loigiai{
		\begin{center}
			\begin{tikzpicture}[>=stealth, scale=1,samples=100,smooth,xscale=1,yscale=1]
				\def\h{7}
				\coordinate (O) at (0,0);
				\coordinate (A) at (-2,1);
				\coordinate (B) at (-6,0);
				\coordinate (C) at ($(A)!2!(O)$);
				\coordinate (D) at ($(B)!2!(O)$);
				\coordinate (S) at (0,\h);
				\coordinate (E) at ($(A)!0.5!(B)$);
				\coordinate (F) at ($(B)!0.5!(C)$);
				\coordinate (G) at ($(C)!0.5!(D)$);
				\coordinate (H) at ($(A)!0.5!(D)$);
				\coordinate (M) at ($(S)!0.5!(E)$);
				\coordinate (N) at ($(S)!0.5!(F)$);
				\coordinate (P) at ($(S)!0.5!(G)$);
				\coordinate (Q) at ($(S)!0.5!(H)$);
				\draw (S)--(B)--(C)--(D)--(S)--(C) (F)--(S)--(G);
				\draw[dashed] (D)--(A)--(B)--(D) (C)--(A)--(S)--(O)--(M) (S)--(H) (S)--(E)--(F)--(G)--(H)--(E) (P)--(O)--(N)--(M)--(Q) (O)--(Q)--(P)--(N);
				\foreach \x/\g in {A/110,B/-90,C/-90,D/0,S/40,E/110,F/-90,G/-30,H/45,M/180,N/-45,P/0,Q/60,O/-90}
				\fill[black](\x) circle (1pt)
				($(\x)+(\g:3mm)$) node{\color{black}$\x$};
			\end{tikzpicture}
		\end{center}
		Gọi $ E,\,F,\,G,\,H$ lần lượt là giao điểm của $ SM$ với $ AB$, $ SN$ với $ BC$, $ SP$ với $ CD$, $ SQ$ với $ DA$ thì $ E,\,F,\,G,\,H$ là trung điểm của $ AB,\,BC,\,CD,\,DA$ thì\\
		Ta có $\dfrac{SP}{SG}=\dfrac{SP\cdot SG}{S{G^2}}=\dfrac{SO^2}{SG^2}=\dfrac{\dfrac{9a^2}{4}}{\dfrac{9a^2}{2}}=\dfrac{1}{2}\Rightarrow P$ là trung điểm $ SG$.\\
		Chứng minh tương tự ta cũng có $M$, $N$, $Q$ lần lượt là trung điểm $ AB,\,BC,\,DA$.\\
		Khi đó $\mathrm{d}(O,(MNPQ))=\dfrac{1}{2}SO=\dfrac{3a}{4}$.\\
		$S_{MNPQ}=\dfrac{1}{4}{S_{EFGH}}=\dfrac{1}{8}{S_{ABCD}}=\dfrac{9a^2}{8}$.\\
		Vậy $V_{O.MNPQ}=\dfrac{1}{3}\cdot\dfrac{3a}{4}\cdot\dfrac{9a^2}{8}=\dfrac{9a^3}{32}$.
	}
\end{ex}

%---------------------
\begin{ex}%[2H1G3-2]
	[Mã 104 - 2020 đợt 2]
	Cho hình chóp đều $S.ABCD$ có cạnh đáy bằng $2a$, cạnh bên bằng $\sqrt{3}a$ và $O$ là tâm của đáy. Gọi $M$, $N$, $P$ và $Q$ lần lượt là hình chiếu vuông góc của $O$ trên các mặt phẳng $(SAB)$, $(SBC)$, $(SCD)$ và $(SDA)$. Thể tích của khối chóp $O.MNPQ$ bằng
	\choice
	{$\dfrac{8a^3}{81}$}
	{$\dfrac{a^3}{6}$}
	{\True $\dfrac{a^3}{12}$}
	{$\dfrac{16a^3}{81}$}
	\loigiai
	{
		\immini
		{
			Gọi $E$ là trung điểm $AB$. Gọi $M$ là hình chiếu vuông góc của $O$ lên $SE$, suy ra $OM\perp(SAB)$.\\
			Tam giác $SOB$ có
			\[SO=\sqrt{SB^2-OB^2}=\sqrt{3a^2-2a^2}=a.\]
			Tam giác $SOE$ có $SE^2=SO^2+OE^2=a^2+a^2=2a^2$ và
			\[SM\cdot SE=SO^2\Leftrightarrow\dfrac{SM}{SE}=\dfrac{SO^2}{SE^2}=\dfrac{a^2}{2a^2}=\dfrac{1}{2}.\]
			Do đó $M$ là trung điểm $SE$.
		}
		{\begin{tikzpicture}[line cap=round,line join=round,font=\footnotesize,>=stealth,scale=1]
				\fill (0,0) coordinate [label=left:$A$] (A) circle(1pt)
				(3.5,0) coordinate [label=right:$D$] (D) circle(1pt)
				(-140:2) coordinate [label=below:$B$] (B) circle(1pt)
				($(B)+(D)$) coordinate [label=right:$C$] (C) circle(1pt)
				($(A)!0.5!(C)$) coordinate [label=below:$O$] (O) circle(1pt)
				($(C)!0.5!(D)$) coordinate [label=right:$G$] (G) circle(1pt)
				($(B)!0.5!(C)$) coordinate [label=below:$F$] (F) circle(1pt)
				($(A)!0.5!(B)$) coordinate (E) node[shift={(160:1ex)}]{$E$} circle(1pt)
				($(A)!0.5!(D)$) coordinate (H) node[shift={(50:1.5ex)}]{$H$} circle(1pt)
				($(O)!2!90:(G)$) coordinate [label=above:$S$] (S) circle(1pt)
				($(S)!0.5!(E)$) coordinate (M) node[shift={(150:2ex)}]{$M$} circle(1pt)
				($(S)!0.5!(F)$) coordinate (N) node[shift={(50:1.5ex)}]{$N$} circle(1pt)
				($(S)!0.5!(G)$) coordinate [label=right:$P$] (P) circle(1pt)
				($(S)!0.5!(H)$) coordinate (Q) node[shift={(150:1.5ex)}]{$Q$} circle(1pt);
				\draw (S)--(B)--(C)--(D)--(S)--(C) (F)--(S)--(G);
				\draw[dashed] (E)--(F)--(G)--(H)--(E)--(S)--(H)
				(B)--(A)--(D) (C)--(A)--(S)--(O)
				(M)--(N)--(P)--(Q)--(M)--(O) (B)--(D) (P)--(O)--(G)
				(Q)--(O)--(N);
		\end{tikzpicture}}
		\noindent
		Gọi $H$, $G$, $F$ lần lượt là trung điểm của $AD$, $DC$, $CB$.\\ Tương tự ta chứng minh được $N$, $P$, $Q$ lần lượt là trung điểm của $SF$, $SG$, $SH$.\\
		Suy ra $MNPQ$ là hình vuông có độ dài cạnh bằng $\dfrac{AC}{4}=\dfrac{2a\sqrt{2}}{4}=\dfrac{a\sqrt{2}}{2}$.\\
		Ta có
		\[\mathrm{d}(O,(MNPQ))=\mathrm{d}(S,(MNPQ))=\dfrac{SO}{2}=\dfrac{a}{2}.\]
		Vậy $V_{O.MNPQ}=\dfrac{1}{3}\cdot \mathrm{d}(O,(MNPQ))\cdot S_{MNPQ}=\dfrac{1}{3}\cdot\dfrac{a}{2}\cdot\left(\dfrac{a\sqrt{2}}{2}\right)^2=\dfrac{a^3}{12}$.
	}
\end{ex}
\begin{ex}%[2H1G3-4]
	[Đề tham khảo 2018]
	Cho hình vuông $ABCD$ và $ABEF$ có cạnh bằng $1$, lần lượt nằm trên hai mặt phẳng vuông góc với nhau. Gọi $S$ là điểm đối xứng với $B$ qua đường thẳng $DE$. Thể tích của khối đa diện $ABCDSEF$ bằng
	\choice
	{$\dfrac{7}{6}$}
	{$\dfrac{11}{12}$}
	{$\dfrac{2}{3}$}
	{\True $\dfrac{5}{6}$}
	\loigiai{
		\begin{center}
			\begin{tikzpicture}[line cap=round,line join=round, >=stealth,scale=1]
				\def \a{-2} \def \b{-1} \def \c{4} \def \h{4}
				\path
				(0,0)coordinate(A)
				+(\a,\b)coordinate(B)
				+(\c,0)coordinate(D)
				($(B)+(D)-(A)$)coordinate(C)
				($(A)+(90:\c)$) coordinate (F)
				($(B)+(F)-(A)$) coordinate (E)
				($(D)!.5!(F)$) coordinate (M)
				($(D)!.5!(E)$) coordinate (I)
				($(M)!-1!(A)$) coordinate (S)
				;
				\draw[dashed]
				(S)--(B)--(A)--(D)--(F)--(A)--(S)
				(B)--(D)--(E)
				;
				\draw
				(D)--(S)--(F)--(E)--(C)--(S)--(E)
				(E)--(B)--(C)--(D)
				;
				\foreach \x/\g in {A/160,B/-145,C/-45,D/0,S/90,E/145,F/90,M/0,I/90}\fill[draw,fill=white] (\x) circle (1pt)+(\g:3mm) node{$\x$};
			\end{tikzpicture}
		\end{center}
		Gọi $M$, $I$ lần lượt là trung điểm của $DF$, $DE \Rightarrow AM \perp \left( DCEF \right)$. \\
		Vì $S$ là điểm đối xứng với $B$ qua $DE$, suy ra  $M$ là trung điểm của $SA$.\\
		Suy ra $SA \perp \left( DCEF \right)$ và $SM = AM = \dfrac{1}{2}DF = \dfrac{{\sqrt 2 }}{2}$.\\
		Khi đó $V_{ABCDSEF}=V_{ADF.BCE}+V_{S.DCEF}=AB\cdot S_{\triangle \,ADF}+\dfrac{1}{3}\cdot SM\cdot S_{DCEF}$.\\
		Vậy $V_{ABCDSEF}=1\cdot \dfrac{1}{2} + \dfrac{1}{3}\cdot \dfrac{\sqrt{2}}{2}\cdot \sqrt{2}=\dfrac{5}{6}$.
	}
\end{ex}
\begin{ex}%[2H1G3-4]
	[Mã đề 104 - BGD - 2019]
	Cho lăng trụ $ ABC.A'B'C'$ có chiều cao bằng $4$ và đáy là tam giác đều cạnh bằng $4$. Gọi $ M$, $N$, $P$ lần lượt là tâm của các mặt bên $ABB'A'$, $BCC'B'$ và $ ACC'A'$. Thể tích của khối đa diện lồi có các đỉnh là các điểm $A$, $B$, $C$, $M$, $N$, $P$ bằng
	\choice
	{$ 8\sqrt 3 $}
	{\True$ 6\sqrt 3 $}
	{ $\dfrac{20\sqrt{3}}{3}$}
	{$\dfrac{14\sqrt{3}}{3}$}
	\loigiai{
		\immini{
			Mặt phẳng $(MNP)$ cắt các cạnh $ AA'$, $BB'$, $CC'$ lần lượt tại các điểm $A_1$, $B_1$, $C_1$.\\
			Dễ thấy, $(MNP)\parallel (ABC)$ và $(MNP)$ chia khối lăng trụ thành hai phần có thể tích bằng nhau.\\
			Gọi $ V$ là thể tích khối đa diện cần tìm. Khi đó:\\
			$ V=\dfrac{1}{2}{V_{ABC.A'B'C'}}-V_{A{A_1}MP}-V_{C{C_1}PN}-V_{B{B_1}MN}$.\\
			Mặt khác $
			\begin{aligned}[t]
				V_{A{A_1}MP}&=\dfrac{1}{3}\mathrm{d}\left(A,\left(MNP\right)\right)\cdot S_{\Delta{A_1}MP}\\
				&=\dfrac{1}{3}\cdot \dfrac{1}{2}\mathrm{d}\left(A,\left(A'B'C'\right)\right)\cdot\dfrac{1}{4}{S_{A'B'C'}}\\
				&=\dfrac{1}{24}{V_{ABC.A'B'C'}}.
			\end{aligned}
			$\\
			Tương tự: $V_{C{C_1}PN}=V_{B{B_1}MN}=\dfrac{1}{24}{V_{ABC.A'B'C'}}$.
		}{
			\begin{tikzpicture}
				\def\kc{2.5}
				\def\goc{60}
				\path
				(0,0) coordinate (A)
				+(0:2*\kc) coordinate (C)
				+(-0.5*\goc:0.8*\kc) coordinate (B)
				;
				\foreach \p in {A,B,C} \path ([shift={(75:1.8*\kc)}]\p) coordinate (\p');
				\foreach \p/\q/\g in {A/A'/A1,B/B'/B1,C/C'/C1,A/B'/M,A/C'/P,B/C'/N} \path
				($(\p)!1/2!(\q)$) coordinate (\g)
				;
				
				\pgfresetboundingbox
				\draw (A')--(B')--(C')--cycle
				(A')--(A)--(B)--(B')
				(B')--(B)--(C)--(C')
				(A)--(B')--(C)
				(A')--(B)--(C')
				(A1)--(B1)--(C1)
				;
				\draw[densely dash dot] (C')--(A)--(C)--(A')
				(M)--(N)--(P)--cycle
				(A1)--(C1)
				;
				\foreach \p/\g in {A/180,B/-90,A'/90,C/0,B'/90,C'/90,M/-160,N/-20,P/90}\draw[fill=black] (\p) circle (1pt)node[shift={(\g:.3)}]{$\p$};
				\draw[fill=black]
				(A1) circle (1pt)node[left]{$A_{1}$}
				(B1) circle (1pt)node[below right]{$B_{1}$}
				(C1) circle (1pt)node[right]{$C_{1}$}
				;
			\end{tikzpicture}
		}\noindent
		Do đó: $V=\dfrac{1}{2}{V_{ABC.A'B'C'}}-\dfrac{3}{24}{V_{ABC.A'B'C'}}=\dfrac{3}{8}{V_{ABC.A'B'C'}}=\dfrac{3}{8}\cdot\dfrac{4^2\sqrt 3}{4}\cdot 4=6\sqrt 3 $.
	}
\end{ex}
\begin{ex}%[2H1G3-2]
	[Mã đề 103 - BGD - 2019]
	Cho lăng trụ $ABC.A'B'C'$ có chiều cao bằng $6$ và đáy là tam giác đều cạnh bằng $4$. Gọi $M$, $N$ và $P$ lần lượt là tâm của các mặt bên $ABB'A'$, $ACC'A'$ và $BCC'B'$. Thể tích của khối đa diện lồi có các đỉnh là các điểm $A,~B,~C,~M,~N,~P$ bằng
	\choice
	{\True $9\sqrt{3}$}
	{$10\sqrt{3}$}
	{$7\sqrt{3}$}
	{$12\sqrt{3}$}
	\loigiai
	{
		\immini
		{Gọi $h=\mathrm{d}(A',(ABC))=6$, ta có $S_{ABC}=\dfrac{4^2\cdot\sqrt{3}}{4}=4\sqrt{3}$.\\
			Khi đó, $V=V_{ABC.A'B'C'}=S_{ABC}\cdot h=4\sqrt{3}\cdot 6=24\sqrt{3}$.\\
			Ta có $(MNP)\parallel (ABC)$ và $\mathrm{d}(M,(ABC))=\dfrac{1}{2}h=3$.\\
			Suy ra, $V_{MABC}=\dfrac{1}{3}\cdot3\cdot4\sqrt{3}=4\sqrt{3}$.\\
			Tam giác $ABC$ đồng dạng với tam giác $MNP$ theo tỉ số $2$ nên \[S_{MNP}=\dfrac{1}{4}\cdot S_{ABC}=\dfrac{1}{4}\cdot4\sqrt{3}=\sqrt{3}\Rightarrow V_{CMNP}=\dfrac{1}{3}\cdot3\cdot\sqrt{3}=\sqrt{3}.\]
			Ta có,
		}
		{\begin{tikzpicture}[scale=1, font=\footnotesize, line join=round, line cap=round, >=stealth]
				\fill (0,0) coordinate [label=below:$A$] (A) circle(1pt)
				(4,0) coordinate [label=below right:$B$] (B) circle(1pt)
				(-40:2.0) coordinate [label=below right:$C$] (C) circle(1pt)
				(78:4.2) coordinate [label=above left:$A'$] (A') circle(1pt)
				($(A')+(C)$) coordinate [label=above:$C'$] (C') circle(1pt)
				($(A')+(B)$) coordinate [label=above left:$B'$] (B') circle(1pt)
				($(A')!0.5!(C)$) coordinate [label=above left:$N$] (N) circle(1pt)
				($(B')!0.5!(C)$) coordinate [label=right:$P$] (P) circle(1pt)
				($(A')!0.5!(B)$) coordinate [label=right:$M$] (M) circle(1pt);
				\draw (B')--(C)--(A')--(B')--(C')--(A')--(A)--(C)--(B)--(B') (C)--(C');
				\draw[dashed](A)--(B) (B')--(A)--(M)--(N)--(P)--(M)--(C) (M)--(B);
				\draw (N)--(A) (B)--(P);
		\end{tikzpicture}}
		\noindent
		$V_{MNAC}=\dfrac{1}{3}\mathrm{d}(M,(NAC)\cdot S_{NAC}=\dfrac{1}{3}\cdot\dfrac{1}{2}\mathrm{d}(B',(ACC'A'))\cdot\dfrac{1}{4}S_{ACC'A'}=\dfrac{1}{8}V_{B'.ACC'A'}=\dfrac{1}{8}\cdot\dfrac{2}{3}V=2\sqrt{3}$.\\
		Tương tự, $V_{MPBC}=2\sqrt{3}$.\\
		Vậy $V_{MNPACB}=V_{MABC}+V_{CMNP}+V_{MNAC}+V_{MPBC}=4\sqrt{3}+\sqrt{3}+2\sqrt{3}+2\sqrt{3}=9\sqrt{3}$.
	}
\end{ex}

\begin{ex}%[2H1G3-3]
	[Mã đề 102 - BGD - 2019]
	Cho lăng trụ $ABC.A'B'C'$ có chiều cao bằng $8$ và đáy là tam giác đều cạnh bằng $4$. Gọi $M$, $N$ và $P$ lần lượt là tâm của các mặt bên $ABB'A'$, $ACC'A'$ và $BCC'B'$. Thể tích của khối đa  diện lồi có các đỉnh là các điểm $A$, $B$, $C$, $M$, $N$, $P$ bằng
	\choice
	{\True $12\sqrt{3}$}
	{$16\sqrt{3}$}
	{$\dfrac{28\sqrt{3}}{3}$}
	{$\dfrac{40\sqrt{3}}{3}$}
	\loigiai{
		\immini
		{
			Ta có $V_{ABC.A'B'C'}=8\cdot \dfrac{\sqrt{3}}{4}\cdot 4^2=32\sqrt{3}$;\\
			$V_{C'.ABC}=\dfrac{1}{3}V_{ABC.A'B'C'}$; $V_{A.BC'B'}=\dfrac{1}{3}V_{ABC.A'B'C'}$.\\
			Gọi thể tích của khối đa  diện lồi có các đỉnh là các điểm $A$, $B$, $C$, $M$, $N$, $P$ là $V$.\\
			Ta có $V=V_{C.ABPN}+V_{P.AMN}+V_{P.ABM}$.\\
			Mặt khác
			\begin{eqnarray*}
				&&V_{C.ABPN}=\dfrac{3}{4} V_{C'.ABC}=\dfrac{1}{4}V_{ABC.A'B'C'}\\
				&&V_{PAMN}=\dfrac{1}{8}V_{ABC'B'}=\dfrac{1}{24}V_{ABC.A'B'C'}\\
				&&V_{PABM}=\dfrac{1}{4}V_{ABC'B'}=\dfrac{1}{12}V_{ABC.A'B'C'}.
			\end{eqnarray*}
		}
		{
			\begin{tikzpicture}[scale=1,>=stealth, line join=round, line cap=round]
				\path
				foreach \x/\y/\z in {0/0/A,1.2/-1.4/B,5/0/C,0.5/4/A'}
				{(\x,\y) coordinate (\z)}
				($(A')+(B)-(A)$) coordinate (B')
				($(A')+(C)-(A)$) coordinate (C')
				($(A)!.5!(B')$) coordinate (M)
				($(B)!.5!(C')$) coordinate (P)
				($(A)!.5!(C')$) coordinate (N)
				;
				\draw (A)--(B)--(C)--(C')--(A')--(B')--(B)--(M)
				(B')--(A)--(A')
				(B)--(C')--(B')
				;
				\draw[dashed] (C')--(A)--(C)--(N)--(M)--(P)--(N) (A)--(P);
				\foreach \p/\g in {A/180,B/-90,A'/90,C/0,B'/70,C'/90,M/180,N/130,P/-70}\draw[fill=black] (\p) circle (1pt)node[shift={(\g:.3)}]{$\p$};
			\end{tikzpicture}
		}
		\noindent Vậy $V=\dfrac{1}{4}V_{ABC.A'B'C'}+\dfrac{1}{24}V_{ABC.A'B'C'}+\dfrac{1}{12}V_{ABC.A'B'C'}=\dfrac{3}{8}V_{ABC.A'B'C'}=12\sqrt{3}$.
	}
\end{ex}

\begin{ex}%[2H1G3-3]
	[Mã đề 101 - BGD - 2019]
	Cho lăng trụ $ABC.A'B'C'$ có chiều cao bằng $8$ và đáy là tam giác đều cạnh bằng $6$. Gọi $M,N$ và $P$ lần lượt là tâm của các mặt bên $ABB'A'$, $ACC'A'$ và $BCC'B'$. Thể tích của khối đa diện lồi có các đỉnh là các điểm $A,B,C,M,N,P$ bằng
	\choice
	{\True $27\sqrt{3}$}
	{$21\sqrt{3}$}
	{$30\sqrt{3}$}
	{$36\sqrt{3}$}
	\loigiai{
		\immini{
			Thể tích của khối lăng trụ $ABC.A'B'C'$ là
			$$V=8\cdot \dfrac{6^2\sqrt{3}}{4}=72\sqrt{3}.$$
			Gọi $A_1,B_1,C_1$ lần lượt là trung điểm của $AA',BB',CC'$.\\
			Ta có thể tích khối lăng trụ $ABC.A_1B_1C_1$ là $V_{ABC.A_1B_1C_1}=\dfrac{V}{2}$.\\
			Thể tích khối chóp $A.A'B'C'$ là $$V_{A.A'B'C'}=\dfrac{1}{3}V_{ABC.A'B'C'}=\dfrac{V}{3}.$$
			Mặt khác, ta có $$\dfrac{V_{A.A_1MN}}{V_{A.A'B'C'}}=\dfrac{AA_1}{AA'}\cdot \dfrac{AM}{AB'}\cdot \dfrac{AN}{AC'}=\dfrac{1}{2}\cdot \dfrac{1}{2}\cdot \dfrac{1}{2}=\dfrac{1}{8}.$$
			Suy ra $V_{A.A_1MN}=\dfrac{1}{8}V_{A.A'B'C'}=\dfrac{1}{8}\cdot \dfrac{1}{3}V=\dfrac{V}{24}$.\\
			Tương tự ta tính được $V_{B.B_1MP}=V_{C.C_1NP}=\dfrac{V}{24}$.
		}
		{
			\begin{tikzpicture}[line join = round, line cap = round,>=stealth,font=\footnotesize,scale=.9]
				\path
				(0,0) coordinate (A)
				($(A)+(5,0)$) coordinate (C)
				($(A)+(-40:2.5)$) coordinate (B)
				($(A)+(80:5.5)$) coordinate (A')
				($(A')+(B)-(A)$) coordinate (B')
				($(A')+(C)-(A)$) coordinate (C')
				foreach \x/\y/\z in {A/A'/A_1,B/B'/B_1,C/C'/C_1,A'/B/M,B'/C/P,C'/A/N}
				{($(\x)!.5!(\y)$) coordinate (\z)}
				;
				\draw (A)--(B)--(C)--(C')--(B')--(A')--(A)--(B')
				(A')--(C')
				(A_1)--(B_1)--(C_1)
				(M)--(B)--(P)--(C)
				(B)--(B')
				;
				\draw[dashed] (C')--(A)--(C)--(N)--(M)--(P)--(N)
				(A_1)--(C_1)
				;
				\foreach \p/\g in {A/180,B/-90,A'/90,C/0,B'/70,C'/90,M/180,N/110,P/-20,A_1/180,B_1/-150,C_1/0}\draw[fill=black] (\p) circle (1pt)node[shift={(\g:.3)}]{$\p$};
			\end{tikzpicture}
		}
		\noindent Khi đó $$V_{ABCMNP}=V_{ABC.A_1B_1C_1}-V_{A.A_1MN}-V_{B.B_1MP}-V_{C.C_1NP}=\dfrac{V}{2}-\dfrac{V}{24}-\dfrac{V}{24}-\dfrac{V}{24}=\dfrac{3V}{8}=27\sqrt{3}.$$
	}
\end{ex}
\begin{ex}%[2H1G3-3]
	[Chuyên Hạ Long -2019] thể tích của bát diện đều cạnh bằng $a\sqrt{3}$ là.
	\choice
	{$6a^3$}
	{\True $\sqrt{6}a^3$}
	{$\dfrac{4}{3}a^3$}
	{$a^3$}
	\loigiai{
		\immini
		{
			Ta có khối bát diện đều cạnh $a\sqrt{3}$ được tạo từ 2 khối chóp tứ giác đều có cạnh đáy và cạnh bên bằng $a\sqrt{3}$.\\
			Chiều cao của khối chóp là: $h=\sqrt{{{\left(a\sqrt{3}\right)}^2}-{{\left(\dfrac{a\sqrt{6}}{2}\right)}^2}}=\dfrac{a\sqrt{6}}{2}$.\\
			Thể tích của khối chóp: ${{V}_{chop}}=\dfrac{1}{3}{{\left(a\sqrt{3}\right)}^2}\cdot \dfrac{a\sqrt{6}}{2}=\dfrac{a^3\sqrt{6}}{2}$ (đvtt).\\
			Vậy thể tích khối bát diện là: $V=2{{V}_{chop}}=a^3\sqrt{6}$(đvtt).\\
		}
		{
			\begin{tikzpicture}[>=stealth, line join=round, line cap=round, font=\footnotesize, scale=.7]
				\def\a{3}
				\def\b{1}
				\def\gA{20}
				\path
				(0:0) coordinate (O)
				(0:\a) arc (0:\gA:{\a} and {\b}) coordinate (A)
				arc (\gA:\gA+90:{\a} and {\b}) coordinate (B)
				arc (\gA+90:\gA+180:{\a} and {\b}) coordinate (C)
				arc (\gA+180:\gA+270:{\a} and {\b}) coordinate (D)
				(O)+(90:\a) coordinate (E)
				(O)+(-90:\a) coordinate (F)
				;
				\draw[dotted]
				(E)--(F) (A)--(C) (B)--(D)
				(A)--(B)--(C) (E)--(B)--(F)
				;
				\draw
				(E)--(A)--(F) (E)--(C)--(F) (E)--(D)--(F)
				(C)--(D)--(A)
				;
				\foreach \x in {A,B,C,D,O,E,F} \draw[fill=white](\x) circle (.05);
			\end{tikzpicture}
		}
	}
\end{ex}
\begin{ex}%[2H1G3-3]
	(Chuyên Lào Cai-2020) Cho lăng trụ đều $ABC.A'B'C'$ có tất cả các cạnh bằnga. Gọi $S$ là điểm đối xứng của $A$ qua $BC'$. Thể tích khối đa diện $ABCSB'C'$ là
	\choice
	{\True $\dfrac{a^3\sqrt 3}{3}$}
	{$a^3\sqrt 3$}
	{$\dfrac{a^3\sqrt 3}{6}$}
	{$\dfrac{a^3\sqrt 3}{2}$}
	\loigiai{
		\immini{Chia khối đa diện $ABCSB'C'$ thành 2 khối là khối chóp $A.BCC'B'$ và khối chóp $S.BCC'B'$\\
			$V_{ABCSB'C'}=V_{ABCC'B'}+V_{S.BCC'B'}$\\
			Gọi $M$ là trung điểm $BC$.\\
			Ta có: $\left.\begin{array}{l}AM\perp BC\\AM\perp BB'\end{array}\right\}\Rightarrow AM\perp\left(BCC'B'\right)$.\\
			Tam giác $ABC$ đều $\Rightarrow AM=\dfrac{a\sqrt 3}{2}$.\\
			Thể tích khối chóp $ A.BCC'B'$ là:\\
			$V_{A.BCC'B'}=\dfrac{1}{3}AM.S_{BCC'B'}=\dfrac{1}{3}.\dfrac{a\sqrt 3}{2}.a^2=\dfrac{a^3\sqrt 3}{6}$.\\
			Thể tích khối chóp $ S.BCC'B'$ là:
			\allowdisplaybreaks
			\begin{eqnarray*}
				\dfrac{V_{S.BCC'B'}}{V_{A.BCC'B'}}&=&\dfrac{\dfrac{1}{3}d\left(S;\left(BCC'B'\right)\right).S_{BCC'B'}}{\dfrac{1}{3}d\left(A;\left(BCC'B'\right)\right).S_{BCC'B'}}\\
				&=&\dfrac{d\left(S;\left(BCC'B'\right)\right)}{d\left(A;\left(BCC'B'\right)\right)}=\dfrac{SI}{AI}=1.
			\end{eqnarray*}
			$\Rightarrow{V_{S.BCC'B'}}=V_{A.BCC'B'}=\dfrac{a^3\sqrt 3}{6}$\\
			$\Rightarrow{V_{ABCSB'C'}}=V_{A.BCC'B'}+V_{S.BCC'B'}=\dfrac{a^3\sqrt 3}{6}+\dfrac{a^3\sqrt 3}{6}=\dfrac{a^3\sqrt 3}{3}$
		}{
			\begin{tikzpicture}[line join=round, line cap=round,>=stealth,thick,scale=0.7]
				\coordinate (A) at (0,0);
				\coordinate (B) at ($(A) + (3,-3)$);
				\coordinate (C) at ($(A) + (5,0)$);
				\coordinate (A') at ($(A) + (0,4)$);
				\coordinate (B') at ($(B) + (0,4)$);
				\coordinate (C') at ($(C) + (0,4)$);
				\coordinate (Tempt) at ($(B)!(A)!(C')$);
				\coordinate (S) at ($(Tempt)!-1!(A)$);
				\coordinate (I) at ($(A)!0.5!(S)$);
				\coordinate (M) at ($(B)!0.5!(C)$);
				\fill[color=gray!35](B)--(B')--(C')--(C)--cycle;
				\foreach \i in {A,B,C,A',B',C',S,I,M}{\fill[black](\i) circle (1.5pt);}
				\foreach \i in {A,B,A',B'}{\draw(\i) node[scale=0.9, left]{$\i$};}
				\foreach \i in {C,C',S,M}{\draw(\i) node[scale=0.9, right]{$\i$};}
				\foreach \i in {I}{\draw(\i) node[scale=0.9, above right]{$\i$};}
				\draw(A)--(B) (B)--(S) (A)--(A') (A)--(B') (B)--(B') (A)--(A') (A')--(C') (B')--(C') (B')--(S) (C')--(S) (A')--(B');
				\draw[dashed](A)--(C) (A)--(S) (A)--(M) (B)--(C) (C)--(C') (B)--(C') (C)--(S) (A)--(C');
				\draw[dashed] pic[draw,angle radius=0.2cm]{right angle=A--M--C};
				\draw[dashed] pic[draw,angle radius=0.2cm]{right angle=A--I--C'};
			\end{tikzpicture}
		}
	}
\end{ex}

\begin{ex}%[2H1G3-2]
	(Chuyên Lê Hồng Phong-Nam Định-2020) Cho hình hộp $ ABCD.A'B'C'D'$ có đáy $ABCD$ là hình thoi tâm $O$, cạnh bằng $a$ và $\widehat{BAC}=60^\circ$. Gọi $I$, $J$ lần lượt là tâm của các mặt bên $ABB'A',CDD'C'$. Biết $AI=\dfrac{a\sqrt 7}{2}$, $ AA'=2a$ và góc giữa hai mặt phẳng $\left(ABB'A'\right),\left(A'B'C'D'\right)$ bằng $60^\circ$. Tính theo $a$ thể tích khối tứ diện $AOIJ$.
	\choice
	{$\dfrac{3\sqrt 3a^3}{64}$}
	{$\dfrac{\sqrt 3a^3}{48}$}
	{\True $\dfrac{\sqrt 3a^3}{32}$}
	{$\dfrac{\sqrt 3a^3}{192}$}
	\loigiai{
		\immini{
			Ta có $ A{I^2}=\dfrac{A{A'^2}+A{B^2}}{2}-\dfrac{A'{B^2}}{4}$\\
			$\Rightarrow A'{B^2}=2\left(A{A'^2}+A{B^2}\right)-4A{I^2}=3a^2\Rightarrow A'B=a\sqrt 3 $\\
			Do $ A'{B^2}+A{B^2}=A{A'^2}$ nên tam giác $ A'AB$vuông tại B$\Rightarrow{S_{A'AB}}=\dfrac{a^2\sqrt 3}{2}$\\
			Tam giác ABC đều cạnh a nên $S_{ABC}=\dfrac{a^2\sqrt 3}{4}$\\
			Theo đề góc giữa hai mặt phẳng $\left(ABB'A'\right),\left(A'B'C'D'\right)$ bằng $60^\circ$,\\
			nên suy ra $V_{A'ABC}=\dfrac{2S_{A'AB}.S_{ABC}\sin{60^\circ}}{3AB}=\dfrac{a^3\sqrt 3}{8}$
			\allowdisplaybreaks
			\begin{eqnarray*}
				V_{AOIJ}&=&\dfrac{1}{3}d\left(O;\left(IAJ\right)\right).S_{IAJ}\\
				&=&\dfrac{1}{3}.\dfrac{1}{2}d\left(B;\left(B'AD\right)\right).\dfrac{1}{2}{S_{B'AD}}\\
				&=&\dfrac{1}{4}{V_{B'ABD}}=\dfrac{1}{4}{V_{A'ABC}}=\dfrac{a^3\sqrt 3}{32}
			\end{eqnarray*}
		}{
			\begin{tikzpicture}[line join=round, line cap=round,>=stealth,thick,scale=0.65]
				\coordinate (A) at (0,0);
				\coordinate (B) at ($(A) + (-2,-2)$);
				\coordinate (C) at ($(B) + (5,0)$);
				\coordinate (D) at ($(A) + (5,0)$);
				\coordinate (A') at ($(A) + (1.5,4)$);
				\coordinate (B') at ($(B) + (1.5,4)$);
				\coordinate (C') at ($(C) + (1.5,4)$);
				\coordinate (D') at ($(D) + (1.5,4)$);
				\coordinate (O) at ($(A)!0.5!(C)$);
				\coordinate (I) at ($(A)!0.5!(B')$);
				\coordinate (J) at ($(C)!0.5!(D')$);
				\foreach \i in {A,B,C,D,A',B',C',D',O,I,J}{\fill[black](\i) circle (1.5pt);}
				\foreach \i in {A',B',C'}{\draw(\i) node[scale=0.9, above left]{$\i$};}
				\foreach \i in {A,C,D,D'}{\draw(\i) node[scale=0.9,below right]{$\i$};}
				\foreach \i in {I}{\draw(\i) node[scale=0.9, left]{$\i$};}
				\foreach \i in {J}{\draw(\i) node[scale=0.9, right]{$\i$};}
				\foreach \i in {B,O}{\draw(\i) node[scale=0.9,below left]{$\i$};}
				\draw(B')--(B) (B')--(A') (B')--(C') (B)--(B) (A')--(D') (C')--(D') (C')--(C) (C)--(D) (C)--(D') (C)--(D) (C)--(D') (B)--(C) (D)--(D') (C')--(D);
				\draw[dashed](A)--(B) (A)--(B') (A)--(A') (A)--(D) (A)--(C) (B)--(A') (B)--(D) (I)--(O) (I)--(J) (J)--(O);
			\end{tikzpicture}
		}
		\textit{\textbf{Bổ sung:} Công thức tính nhanh thể tích tứ diện theo góc giữa hai mặt phẳng}\\
		Cho tứ diện $ABCD$ có diện tích tam giác $ABC$ bằng $S_1$, diện tích tam giác $BCD$ là $S_2$và góc giữa hai mặt phẳng $(ABC)$ và $(DBC)$ là $\varphi $. Khi đó ta có: $V_{ABCD}=\dfrac{2S_1S_2.\sin\varphi}{3BC}$\\
		\immini{
			\textit{Chứng minh:} Gọi $H$ là hình chiếu của $A$ lên $(BCD)$, kẻ $HI \bot BC$ tại $I$ thì $AI \bot BC$ và $\left(\left(ABC\right);\left(DBC\right)\right)=\left(AI;HI\right)=\widehat{AIH}=\varphi$; $ AH=AI\sin\varphi $
			\allowdisplaybreaks
			\begin{eqnarray*}
				V_{ABCD}&=&\dfrac{1}{3}AH.S_{DBC}=\dfrac{1}{3}AI\sin\varphi \cdot S_2\\&=&\dfrac{1}{3}\dfrac{2S_{ABC}}{BC} \cdot \sin\varphi \cdot S_2=\dfrac{2S_1S_2\sin\varphi}{3BC}.
			\end{eqnarray*}
		}{
			\begin{tikzpicture}[line join=round, line cap=round,>=stealth,thick,scale=0.9]
				\coordinate (D) at (0,0);
				\coordinate (B) at ($(D) + (4,0)$);
				\coordinate (C) at ($(D) + (3,-1.5)$);
				\coordinate (A) at ($(D) + (2,2.5)$);
				\foreach \i in {A,B,C,D}{\fill[black](\i) circle (1.5pt);}
				\foreach \i in {A,D}{\draw(\i) node[scale=0.9,above left]{$\i$};}
				\foreach \i in {B,C}{\draw(\i) node[scale=0.9,below right]{$\i$};}
				\draw[dashed](B)--(D);
				\draw(A)--(D) (A)--(B) (A)--(C) (D)--(C) (B)--(C);
				\coordinate (H) at ($(A) + (0,-3)$);
				\coordinate (I) at ([scale around={0.3:(C)}] B);
				\foreach \i in {H,I}{\fill[black](\i) circle (1.5pt);}
				\foreach \i in {H}{\draw(\i) node[scale=0.9, left]{$\i$};}
				\foreach \i in {I}{\draw(\i) node[scale=0.9,right]{$\i$};}
				\draw[dashed](A)--(H) (H)--(I);
				\draw(A)--(I);
				\draw[dashed] pic[draw,angle radius=0.2cm]{right angle=A--H--I};
				\draw[dashed] pic[draw,angle radius=0.2cm]{right angle=C--I--H};
				\draw pic["$\varphi$",draw,angle eccentricity=1.9,angle radius=0.2cm]{angle=A--I--H};
			\end{tikzpicture}
		}
	}
\end{ex}

\begin{ex}%[2H1G3-2]
	(Chuyên Quang Trung-2020) Cho hình chóp $ S.ABCD$ đáy là hình vuông cạnh $a$, $SA$ vuông góc với mặt phẳng $\left(ABCD\right)$, $SA=a$. $ M,K$ tương ứng là trọng tâm tam giác $SAB,SCD$; $ N$ là trung điểm $ BC$. Thể tích khối tứ diện $SMNK$ bằng $\dfrac{m}{n}.a^3$với . Giá trị $m+n$ bằng:
	\choice
	{\True $28$}
	{$12$}
	{$19$}
	{$32$}
	\loigiai{
		\immini{
			Ta có: $V_{S.ABCD}=\dfrac{1}{3}SA.S_{ABCD}=\dfrac{a^3}{3}$.\\
			Gọi $ I$ là trung điểm của $ AB$, $ J$ là trung điểm của $ CD$.\\
			Ta có: $\Delta SMN$ đồng dạng với $\Delta SIJ$ theo tỉ số $\dfrac{2}{3}$.\\
			Do đó $V_{SMNK}=V_{P.SMN}=\left(\dfrac{2}{3}\right)^2V_{P.SIJ}=\dfrac{4}{9}{V_{P.SIJ}}$.\\
			Mặt khác $S_{\Delta PIJ}=\dfrac{1}{4}{S_{ABCD}}$.\\
			Do đó $V_{P.SIJ}=V_{S.PIJ}=\dfrac{1}{4}{V_{S.ABCD}}=\dfrac{a^3}{12}$\\
			Nên $V_{SMNK}=\dfrac{4}{9}.\dfrac{a^3}{12}=\dfrac{a^3}{27}$.\\
			Vậy $ m=1,n=27\Rightarrow m+n=28$.
		}{
			\begin{tikzpicture}[line join=round, line cap=round,>=stealth,thick,scale=0.7]
				\coordinate (A) at (0,0);
				\coordinate (D) at ($(A) + (6,0)$);
				\coordinate (V) at (-4,-2);
				\coordinate (B) at ($(A) + (V)$);
				\coordinate (C) at ($(D) + (V)$);
				\coordinate (S) at ($(A) + (0,4)$);
				\foreach \i in {A,B,C,D}{\fill[black](\i) circle (1.5pt);}
				\foreach \i in {A,B,C,D}{\draw(\i) node[scale=0.9,below right]{$\i$};}
				\foreach \i in {S}{\fill[black](\i) circle (1.5pt);}
				\foreach \i in {S}{\draw(\i) node[scale=0.9,above]{$\i$};}
				\draw[dashed](A)--(S) (A)--(B) (A)--(D);
				\draw(S)--(B) (S)--(C) (S)--(D) (B)--(C) (C)--(D);
				\coordinate (I) at ($(B)!0.5!(A)$);
				\coordinate (M) at ([scale around={0.66666666:(S)}] I);
				\coordinate (J) at ($(C)!0.5!(D)$);
				\coordinate (K) at ([scale around={0.66666666:(S)}] J);
				\coordinate (N) at ($(C)!0.5!(B)$);
				\foreach \i in {M,N,K,I,J}{\fill[black](\i) circle (1.5pt);}
				\foreach \i in {M,I}{\draw(\i) node[scale=0.9,above left]{$\i$};}
				\foreach \i in {N}{\draw(\i) node[scale=0.9,below]{$\i$};}
				\foreach \i in {K}{\draw(\i) node[scale=0.9,right]{$\i$};}
				\foreach \i in {J}{\draw(\i) node[scale=0.9,below right]{$\i$};}
				\draw[dashed](S)--(I) (M)--(K) (I)--(J) (N)--(K) (M)--(N);
				\draw(S)--(J) (S)--(N);
			\end{tikzpicture}
		}
	}
\end{ex}

\begin{ex}%[2H1G3-2]
	(Chuyên Quang Trung-2020) Cho hình lăng trụ đứng$\,ABCD.A'B'C'D'$có đáy là hình thoi có cạnh $4a$ , $A'A=8a$ , $\widehat{BAD}=120^^\circ$. Gọi $M,N,K$ lần lượt là trung điểm cạnh $AB',B'C,BD'$ . Thể tích khối da diện lồi có các đỉnh là các điểm $A,B,C,M,N,K$ là:
	\choice
	{\True $ 12\sqrt 3\,a^3$}
	{$\dfrac{28\sqrt 3}{3}\,a^3$}
	{$ 16\sqrt 3\,a^3$}
	{$\dfrac{40\sqrt 3}{3}\,a^3$}
	\loigiai{
		\immini{
			$ MN//AC;\,MN=\dfrac{1}{2}AC$, $MNCA$ là hình thang.\\
			$V_{MNKABC}=V_{K.MNCA}+V_{B.MNCA}$\\
			$DK$ cắt $(B’AC)$ tại $B’$, $\dfrac{B'K}{B'D}=\dfrac{1}{2}$\\
			$\Rightarrow\dfrac{d\left(K;(MNCA)\right)}{d\left(D;(MNCA)\right)}=\dfrac{1}{2}$\\
			$\Rightarrow{V_{K.MNCA}}=\dfrac{1}{2}{V_{D.MNCA}}$\\
			Mà: $V_{B.MNCA}=V_{D.MNCA}$ nên ta có\\ $V_{MNKABC}=\dfrac{1}{2}{V_{B.MNCA}}+V_{B.MNCA}=\dfrac{3}{2}{V_{B.MNCA}}$\\
			Mặt khác: $S_{MNCA}=\dfrac{3}{4}{S_{B'AC}}$
			\allowdisplaybreaks
			\begin{eqnarray*}
				\Rightarrow{V_{B.MNCA}}&=&\dfrac{3}{4}{V_{B.B'AC}}=\dfrac{3}{4}{V_{B'.ABC}}\\
				&=&\dfrac{3}{4}\cdot \dfrac{1}{6}{V_{ABCD.A'B'C'D'}}=8\sqrt 3a^3
			\end{eqnarray*}
			$V_{MNKABC}=\dfrac{3}{2}{V_{B.MNCA}}=\dfrac{3}{2}8\sqrt 3\,a^3=12\sqrt 3\,a^3$.
		}{
			\begin{tikzpicture}[line join=round, line cap=round,>=stealth,thick,scale=0.5]
				\coordinate (D) at (0,0);
				\coordinate (V) at (0,5.6);
				\coordinate (C) at ($(D) + (3,-2)$);
				\coordinate (A) at ($(D) + (9,1)$);
				\coordinate (B) at ($(C) + (9,1)$);
				\coordinate (A') at ($(A) + (V)$);
				\coordinate (B') at ($(B) + (V)$);
				\coordinate (C') at ($(C) + (V)$);
				\coordinate (D') at ($(D) + (V)$);
				\coordinate (O) at ($(A)!0.5!(C)$);
				\foreach \i in {A,B,C,D,A',B',C',D'}{\fill[black](\i) circle (1.5pt);}
				\foreach \i in {A,B}{\draw(\i) node[scale=0.9,right]{$\i$};}
				\foreach \i in {C,C',D,D'}{\draw(\i) node[scale=0.9, below left]{$\i$};}
				\foreach \i in {A',B'}{\draw(\i) node[scale=0.9, above right]{$\i$};}
				\draw[dashed](A)--(B) (A)--(C) (A)--(D) (A)--(A');
				\draw(A')--(B') (B')--(C) (A')--(D') (B)--(B') (C)--(C') (D)--(D') (B)--(C) (C)--(D) (B')--(C')--(D');
				\coordinate (M) at ($(A)!0.5!(B')$);
				\coordinate (N) at ($(C)!0.5!(B')$);
				\coordinate (K) at ($(B)!0.5!(D')$);
				\foreach \i in {M,N,K}{\fill[black](\i) circle (1.5pt);}
				\foreach \i in {K}{\draw(\i) node[scale=0.9, above]{$\i$};}
				\foreach \i in {N}{\draw(\i) node[scale=0.9, below]{$\i$};}
				\foreach \i in {M}{\draw(\i) node[scale=0.9, right]{$\i$};}
				\draw[dashed](A)--(B') (B')--(C) (B')--(D);
				\draw[dashed](C)--(K) (B)--(N) (A)--(K) (M)--(K) (M)--(N) (M)--(B) (K)--(N);
			\end{tikzpicture}
		}
	}
\end{ex}

\begin{ex}%[2H1G3-3]
	\immini{
		(Chuyên Sơn La-2020) Cho hình chóp tứ giác đều $S.ABCD$ có cạnh đáy bằng a, cạnh bên hợp với đáy một góc $60^\circ$ . Gọi $M$ là điểm đối xứng của $C$ qua $D$, $N$ là trung điểm $SC$. Mặt phẳng $(BMN)$ chia khối chóp $S.ABCD$ thành hai phần (như hình vẽ bên). Tỉ số thể tích giữa hai phần $\dfrac{V_{SABFEN}}{V_{BFDCNE}}$ bằng\\
	}{\begin{tikzpicture}[line join=round, line cap=round,>=stealth,thick,scale=0.8]
			\coordinate (C) at (0,0);
			\coordinate (D) at ($(C) + (4,0)$);
			\coordinate (V) at (-2,-2);
			\coordinate (B) at ($(C) + (V)$);
			\coordinate (A) at ($(D) + (V)$);
			\coordinate (S) at ($(C) + (1,3)$);
			\foreach \i in {A,B,C,D}{\fill[black](\i) circle (1.5pt);}
			\foreach \i in {C,D}{\draw(\i) node[scale=0.9,above right]{$\i$};}
			\foreach \i in {A,B}{\draw(\i) node[scale=0.9,below left]{$\i$};}
			\foreach \i in {S}{\fill[black](\i) circle (1.5pt);}
			\foreach \i in {S}{\draw(\i) node[scale=0.9,above]{$\i$};}
			\draw(A)--(S)--(B) (A)--(B);
			\draw[dashed](C)--(D) (C)--(B) (C)--(S);
			\coordinate (M) at ($(D)!-1!(C)$);
			\coordinate (N) at ($(C)!0.5!(S)$);
			\coordinate (O) at ($(C)!0.5!(A)$);
			\coordinate (E) at (intersection of M--N and S--D);
			\coordinate (F) at (intersection of B--M and A--D);
			\foreach \i in {M,N,E,F}{\fill[black](\i) circle (1.5pt);}
			\foreach \i in {N,O}{\draw(\i) node[scale=0.9, left]{$\i$};}
			\foreach \i in {M}{\draw(\i) node[scale=0.9, right]{$\i$};}
			\foreach \i in {E}{\draw(\i) node[scale=0.9, above right]{$\i$};}
			\foreach \i in {F}{\draw(\i) node[scale=0.9, below right]{$\i$};}
			\draw[dashed](B)--(N) (N)--(E) (B)--(F) (A)--(C) (B)--(D)--(N) (S)--(O)--(F)--(D) (E)--(D)--(M);
			\draw(S)--(E) (A)--(F) (F)--(M) (M)--(E) (E)--(F);
	\end{tikzpicture}}
	\choice
	{\True $\dfrac{7}{5}$}
	{$\dfrac{7}{6}$}
	{$\dfrac{7}{3}$}
	{$\dfrac{7}{4}$}
	\loigiai{
		Ta có $N$ là trung điểm của $SO$, $D$ là trung điểm của $CM$ nên $E$ là trọng tâm tam giác $SCM$.\\
		Ký hiệu $h,S,V$ tương ứng là chiều cao, diện tích đáy và thể tích khối chóp $S.ABCD$ ta có\\
		$S_{BCM}=S\Rightarrow{V_{N.BCM}}=\dfrac{1}{3}\cdot \dfrac{h}{2}.S=\dfrac{V}{2}$ .\\
		Khi đó $\dfrac{V_{M.EDF}}{V_{M.NCB}}=\dfrac{ME}{MN} \cdot \dfrac{MD}{MC} \cdot \dfrac{MF}{MB}=\dfrac{2}{3} \cdot \dfrac{1}{2}\cdot \dfrac{1}{2}=\dfrac{1}{6}\Leftrightarrow{V_{M.EDF}}=\dfrac{V}{2}\cdot\dfrac{1}{6}=\dfrac{V}{12}$ .\\
		Như vậy $V_{BFDCNE}=\dfrac{V}{2}-\dfrac{V}{12}=\dfrac{5V}{12}\Rightarrow{V_{SABFEN}}=\dfrac{7V}{12}\Rightarrow\dfrac{V_{SABFEN}}{V_{BFDCNE}}=\dfrac{7}{5}$ .}
\end{ex}

\begin{ex}%[2H1G3-2]
	(Chuyên Thái Bình-2020) Cho hình chóp $S.ABCD$ có đáy là hình vuông cạnh $2\sqrt 2 $. Cạnh bên $SA$ vuông góc với đáy và $SA=3$. Mặt phẳng $\left(\alpha\right)$ qua $A$ và vuông góc với $SC$ cắt các cạnh $SB,SC,SD$ tại $M,N,P$. Tính thể tích mặt cầu ngoại tiếp tứ diện $CMNP$
	\choice
	{\True $\dfrac{32\pi}{3}$}
	{$\dfrac{64\sqrt 2\pi}{3}$}
	{$\dfrac{108\pi}{3}$}
	{$\dfrac{125\pi}{6}$}
	\loigiai{
		\immini{
			Ta có: $\heva{&SA\perp BC\\&AB\perp BC}\Rightarrow BC\perp\left(SAB\right)\Rightarrow BC\perp MA$.\\
			Lại có $MA\perp SC\Rightarrow MA\perp\left(SBC\right)\Rightarrow MA\perp MC\,\,\,\,(1)$ .\\
			Tương tự: $AP\perp PC\,\,\,\,(2).$\\
			Mặt khác $AN\perp NC\,\,\,\,(3)$.\\
			Gọi $I$ là trung điểm của $AC$, từ $(1)$ $(2)$ $(3)$ ta có\\
			$IN=IM=IC=IP\left(=IA\right)$.\\
			Suy ra mặt cầu ngoại tiếp $CMNP$ là mặt cầu tâm $I$, bán kính $IA$.\\
			$IA=\dfrac{AC}{2}=\dfrac{\sqrt{\left(2\sqrt 2\right)^2+\left(2\sqrt 2\right)^2}}{2}=2.$\\
			Thể tích khối cầu ngoại tiếp tứ diện $CMNP$ là:\\
			$V=\dfrac{4}{3}\pi{2^3}=\dfrac{32\pi}{3}$.
		}{
			\begin{tikzpicture}[line join=round, line cap=round,>=stealth,thick,scale=0.7]
				\coordinate (A) at (0,0);
				\coordinate (D) at ($(A) + (6,0)$);
				\coordinate (V) at (-1.5,-2);
				\coordinate (B) at ($(A) + (V)$);
				\coordinate (C) at ($(D) + (V)$);
				\coordinate (S) at ($(A) + (0,4)$);
				\foreach \i in {A,B,C,D}{\fill[black](\i) circle (1.5pt);}
				\foreach \i in {A,B,C,D}{\draw(\i) node[scale=0.9,below right]{$\i$};}
				\foreach \i in {S}{\fill[black](\i) circle (1.5pt);}
				\foreach \i in {S}{\draw(\i) node[scale=0.9,above]{$\i$};}
				\draw[dashed](A)--(S) (A)--(B) (A)--(D);
				\draw(S)--(B) (S)--(C) (S)--(D) (B)--(C) (C)--(D);
				\coordinate (M) at ($(B)!0.5!(S)$);
				\coordinate (N) at ([scale around={0.33333333:(S)}] C);
				\coordinate (P) at ($(S)!0.5!(D)$);
				\coordinate (I) at ($(C)!0.5!(A)$);
				\foreach \i in {M,N,I,P}{\fill[black](\i) circle (1.5pt);}
				\foreach \i in {M}{\draw(\i) node[scale=0.9,above left]{$\i$};}
				\foreach \i in {I}{\draw(\i) node[scale=0.9,below]{$\i$};}
				\foreach \i in {N,P}{\draw(\i) node[scale=0.9,above right]{$\i$};}
				\draw[dashed](S)--(I) (M)--(A)--(N) (C)--(A)--(P);
				\draw (C)--(M)--(N)--(P)--(C)--(N);
			\end{tikzpicture}
		}
	}
\end{ex}

\begin{ex}%[2H1G3-2]
	(Chuyên Thái Nguyên-2020) Cho hình chóp $ S.ABC$ có đáy $ ABC$ là tam giác vuông cân đỉnh $ B$, $ AB=4$, $ SA=SB=SC=12$. Gọi $ M,N,E$ lần lượt là trung điểm của $ AC,BC,AB$. Trên cạnh $ SB$ lấy điểm $ F$ sao cho $\dfrac{BF}{BS}=\dfrac{2}{3}$. Thể tích khối tứ diện $ MNEF$ bằng
	\choice
	{$\dfrac{8\sqrt{34}}{3}$}
	{$\dfrac{4\sqrt{34}}{3}$}
	{\True $\dfrac{8\sqrt{34}}{9}$}
	{$\dfrac{16\sqrt{34}}{9}$}
	\loigiai{
		\immini{
			Vì $SA=SB=SC$ nên hình chiếu của $S$ lên $\left(ABC\right)$ là tâm đường tròn ngoại tiếp $ABC$, suy ra $ SM\perp\left(ABC\right)$.\\
			Từ $ AB=4\Rightarrow AC=4\sqrt 2 $.\\
			Tam giác $ SAM$ vuông tại $ M$ nên\\
			$SM=\sqrt{S{A^2}-A{M^2}}=\sqrt{12^2-\left(2\sqrt 2\right)^2}=2\sqrt{34}$. Khi đó\\
			\allowdisplaybreaks
			\begin{eqnarray*}
				V_{S.ABC}&=&\dfrac{1}{3}\cdot S_{ABC} \cdot SM=\dfrac{1}{3}\cdot\dfrac{1}{2}\cdot AB^2\cdot SM\\
				&=&\dfrac{1}{3}\cdot\dfrac{1}{2}\cdot 4^2\cdot 2\sqrt{34}=\dfrac{16\sqrt{34}}{3}
			\end{eqnarray*}
			Suy ra thể tích
			\begin{eqnarray*}
				V_{MNEF}&=&\dfrac{1}{3}\cdot S_{MNE}\cdot d\left(F,\left(MNE\right)\right)\\
				&=&\dfrac{1}{3}\cdot\dfrac{1}{4}\cdot S_{ABC} \cdot\dfrac{2}{3}\cdot SM\\
				&=&\dfrac{1}{12}\cdot V_{S.ABC}=\dfrac{1}{12}\cdot\dfrac{32\sqrt{34}}{3}=\dfrac{8\sqrt{34}}{9}.
			\end{eqnarray*}
		}{
			\begin{tikzpicture}[line join=round, line cap=round,>=stealth,thick,scale=0.9]
				\coordinate (A) at (0,0);
				\coordinate (C) at ($(A) + (6,0)$);
				\coordinate (B) at ($(A) + (2,-1.5)$);
				\coordinate (S) at ($(A) + (3,3)$);
				\foreach \i in {S,B,C,A}{\fill[black](\i) circle (1.5pt);}
				\foreach \i in {S,A}{\draw(\i) node[scale=0.9,above left]{$\i$};}
				\foreach \i in {B,C}{\draw(\i) node[scale=0.9,below right]{$\i$};}
				\draw[dashed](C)--(A);
				\draw(S)--(A) (S)--(B) (S)--(C) (A)--(B)--(C);
				\coordinate (M) at ($(C)!0.5!(A)$);
				\coordinate (N) at ($(C)!0.5!(B)$);
				\coordinate (E) at ($(B)!0.5!(A)$);
				\coordinate (F) at ([scale around={0.66666666:(B)}] S);
				\foreach \i in {M,N,E,F}{\fill[black](\i) circle (1.5pt);}
				\foreach \i in {F}{\draw(\i) node[scale=0.9, left]{$\i$};}
				\foreach \i in {E}{\draw(\i) node[scale=0.9, below left]{$\i$};}
				\foreach \i in {M}{\draw(\i) node[scale=0.9, below]{$\i$};}
				\foreach \i in {N}{\draw(\i) node[scale=0.9, below right]{$\i$};}
				\draw[dashed](S)--(M)--(E)--(N)--(M)--(F);
				\draw(E)--(F)--(N);
			\end{tikzpicture}
		}
	}
\end{ex}
\begin{ex}%[2H1G3-2]
	[Nguyễn Huệ-Phú Yên-2020] Cho hình hộp $ABCD.A'B'C'D'$ có chiều cao $8$ và diện tích đáy bằng $11$. Gọi $M$ là trung điểm của $AA'$, $N$ là điểm trên cạnh $BB'$ sao cho $BN=3B'N$ và $P$ là điểm trên cạnh $CC'$ sao cho $6CP=5C'P$. Mặt phẳng $(MNP)$ cắt cạnh $DD'$ tại $Q$. Thể tích của khối đa diện lồi có các đỉnh là các điểm $A$, $B$, $C$, $D$, $M$, $N$, $P$ và $Q$ bằng
	\choice
	{$\dfrac{88}{3}$}
	{\True $42$}
	{$44$}
	{$\dfrac{220}{3}$}
	\loigiai{
		\immini{Cho hình lăng trụ như hình vẽ\\
			$V_{ABC.MNP}=\dfrac{1}{3}\left(\dfrac{AM}{AA'}+\dfrac{BN}{BB'}+\dfrac{CP}{CC'}\right)\cdot V_{ABC.A'B'C'}$.\\
			Chứng minh:\\
			$V_{ABC.MNP}=V_{N.ACB}+V_{N.ACPM}$\\
			$V_{N.ACB}=\dfrac{BN}{BB'}\cdot V_{B'.ACB}=\dfrac{BN}{BB'}\cdot \dfrac{1}{3}\cdot V_{ABC.A'B'C'}$\\
			$\dfrac{V_{N.ACPM}}{V_{B'.ACC'A'}}=\dfrac{S_{ACPM}}{S_{ACC'A'}}=\dfrac{\dfrac{1}{2}\cdot\left(CP+AM\right)}{AA'}=\dfrac{1}{2}\cdot\left(\dfrac{CP}{CC'}+\dfrac{AM}{AA'}\right)$.
		}{
			\begin{tikzpicture}[scale=0.75, font=\footnotesize, line join=round, line cap=round, >=stealth]
				\def \a{3}
				\path
				(0,0) coordinate (A)
				(\a,0) coordinate (C)
				(-40:0.7*\a) coordinate (B)
				(80:1.2*\a) coordinate (A')
				($(A')+(B)$) coordinate (B')
				($(A')+(C)$) coordinate (C')
				($(A)!0.5!(A')$) coordinate (M)
				($(B)!0.75!(B')$) coordinate (N)
				($(C)!5/11!(C')$) coordinate (P)
				;
				\draw  (A)--(B)--(C)--(C')--(A')--(B')--(C')(A)--(A')(B)--(B')(M)--(N)--(P);
				\draw [dashed] (A)--(C)(M)--(P);
				\foreach \x/\g in {A/180,B/-90,C/0,A'/180,B'/-30,C'/0,M/180,N/-30,P/0}\fill[black] (\x) circle (1pt)+(\g:.3)node{$\x$};
			\end{tikzpicture}
		}
		\noindent $\Rightarrow{V_{N.ACPM}}=\dfrac{1}{2}\cdot\left(\dfrac{CP}{CC'}+\dfrac{AM}{AA'}\right)\cdot\dfrac{2}{3}{V_{ABC.A'B'C'}}$\\
		Từ đó ta suy ra điều phải chứng minh.\\
		Bây giờ ta áp dụng vào giải bài toán.\\
		\immini{
			Ta có $\heva{&\left(ADD'A'\right)\parallel\left(BCC'B'\right)\\&MQ\subset(MNP)\cap\left(ADD'A'\right)\\&NP\subset(MNP)\cap\left(BCC'B'\right)}\Rightarrow NP\parallel MQ$. Tương tự ta cũng có $MN\parallel PQ$. Do đó $MNPQ$ là hình bình hành.\\
			Ta có $OI$ là đường trung bình của hai hình thang $AMPC$ và $BNQD$ suy ra $2OI=MA+PC=DQ+NB$ $\Rightarrow\dfrac{MA}{AA'}+\dfrac{PC}{CC'}=\dfrac{BN}{BB'}+\dfrac{DQ}{DD'}$\\
			Dựa vào hình vẽ ta chia khối lăng trụ làm hai phần khi cắt bởi mặt phẳng $\left(BDD'B'\right)$. Do đó $V_{A'D'B'.ADB}=V_{BD'C'.BDC}=44$.\\
			
		}{
			\begin{tikzpicture}[scale=0.75, font=\footnotesize, line join=round, line cap=round, >=stealth]
				\def \a{4}
				\path
				(0,0) coordinate (D)
				(\a,0) coordinate (C)
				(210:0.7*\a) coordinate (A)
				($(A)+(C)$) coordinate (B)
				(80:1.2*\a) coordinate (D')
				($(D')+(B)$) coordinate (B')
				($(D')+(C)$) coordinate (C')
				($(D')+(A)$) coordinate (A')
				($(A)!0.5!(A')$) coordinate (M)
				($(B)!0.8!(B')$) coordinate (N)
				($(C)!0.4!(C')$) coordinate (P)
				($(M)!0.5!(P)$) coordinate (I)
				($(N)!2!(I)$) coordinate (Q)
				($(D)!.5!(B)$) coordinate (O)
				($(D')!.5!(B')$) coordinate (O')
				;
				\draw  (A)--(B)--(C)--(C')--(D')--(A')--(B')--(C')(A)--(A')(B)--(B')(M)--(N)--(P)(A')--(C')(B')--(D');
				\draw [dashed] (A)--(C)(M)--(P)(D)--(D')(O)--(O')(Q)--(N)(D)--(A')(Q)--(P)(B)--(D)(A)--(D)--(C)(M)--(Q);
				\foreach \x/\g in {A/180,B/-90,C/0,A'/180,B'/-30,C'/0,M/180,N/20,P/0,D/-90,D'/160,O/-90,O'/80,I/-40,Q/-40}\fill[black] (\x) circle (1pt)+(\g:.3)node{$\x$};
			\end{tikzpicture}
		}
		\begin{eqnarray*}
			V_{ABCD.MNPQ}&=&V_{ABD.MNQ}+V_{BCD.NPQ}\\
			&=&\dfrac{1}{3}\left(\dfrac{MA}{AA'}+\dfrac{BN}{BB'}+\dfrac{DQ}{DD'}\right)\cdot V_{ABD.A'B'D'}+\dfrac{1}{3}\left(\dfrac{CP}{CC'}+\dfrac{BN}{BB'}+\dfrac{DQ}{DD'}\right)\cdot V_{BCD.B'C'D'}\\
			&=&\dfrac{1}{3}\left(\dfrac{MA}{AA'}+\dfrac{BN}{BB'}+\dfrac{DQ}{DD'}+\dfrac{CP}{CC'}+\dfrac{BN}{BB'}+\dfrac{DQ}{DD'}\right)\cdot\dfrac{1}{2}{V_{ABC.A'B'C'}}\\
			&=&\dfrac{1}{3\cdot2}\left[3\cdot\left(\dfrac{MA}{AA'}+\dfrac{CP}{CC'}\right)\right]\cdot V_{ABC.A'B'C'}\\
			&=&\dfrac{1}{2}\left(\dfrac{MA}{AA'}+\dfrac{CP}{CC'}\right)\cdot V_{ABC.A'B'C'}\\
			&=&\dfrac{1}{2}\left(\dfrac{1}{2}+\dfrac{5}{11}\right)\cdot88=42.
		\end{eqnarray*}
	}
\end{ex}

\begin{ex}%[2H1G3-2]
	[Nguyễn Trãi-Thái Bình-2020] Cho hình chóp $S.ABCD$ có đáy là hình vuông, mặt bên $(SAB)$ là một tam giác đều nằm trong mặt phẳng vuông góc với mặt đáy $(ABCD)$ và có diện tích bằng $\dfrac{27\sqrt 3}{4}$ (đvdt). Một mặt phẳng đi qua trọng tâm tam giác $SAB$ và song song với mặt đáy $(ABCD)$ chia khối chóp $S.ABCD$ thành hai phần. Tính thể tích $V$ của phần chứa điểm $S$.
	\choice
	{$V=8$}
	{$V=24$}
	{$V=36$}
	{\True $V=12$}
	\loigiai{
		\immini{
			Gọi $H$ là trung điểm $AB$. Do $\triangle SAB$ đều và $(SAB)\perp(ABCD)$ nên $SH\perp(ABCD)$.\\
			Ta có $S_{\triangle SAB}=\dfrac{A{B^2}\sqrt 3}{4}=\dfrac{27\sqrt 3}{4}$ $\Rightarrow AB=3\sqrt 3\Rightarrow SH=\dfrac{AB\sqrt 3}{2}=\dfrac{3\sqrt 3\cdot\sqrt 3}{2}=\dfrac{9}{2}$.
			\begin{eqnarray*}
				\Rightarrow{V_{S.ABCD}}&=&\dfrac{1}{3}\cdot S_{ABCD}\cdot SH=\dfrac{1}{3}.A{B^2}\cdot SH\\
				&=&\dfrac{1}{3}{\left(3\sqrt 3\right)^2}\cdot\dfrac{9}{2}=\dfrac{81}{2}\text{ (đvtt)}.
			\end{eqnarray*}
		}{
			\begin{tikzpicture}[scale=1, font=\footnotesize, line join=round, line cap=round, >=stealth]
				\def \a{3.5}
				\path
				(0,0) coordinate (A)
				(\a,0) coordinate (D)
				(210:0.6*\a) coordinate (B)
				($(B)+(D)$) coordinate (C)
				($(A)!0.5!(B)$) coordinate (H)
				($(C)!0.5!(D)$) coordinate (K)
				($(H)!1!90:(K)$) coordinate (S)
				($(S)!2/3!(H)$) coordinate (G)
				($(S)!2/3!(A)$) coordinate (M)
				($(S)!2/3!(B)$) coordinate (N)
				($(S)!2/3!(C)$) coordinate (P)
				($(S)!2/3!(D)$) coordinate (Q)
				;
				\draw  (S)--(B)--(C)--(D)--(S)--(C) (N)--(P)--(Q);
				\draw [dashed] (B)--(A)--(D)(A)--(C) (M)--(P)(N)--(M)--(Q)(S)--(A)(S)--(H);
				\foreach \x/\g in {A/160,B/180,C/-20,D/0,M/50,N/180,P/-10,Q/20,S/90,G/-30,H/-70}\fill[black] (\x) circle (1pt)+(\g:.3)node{$\x$};
			\end{tikzpicture}
		}
		\noindent Gọi $G$ là trọng tâm tam giác $SAB$, qua $G$ kẻ đường thẳng song song với $AB$, cắt $SA$ và $SB$ lần lượt tại $M$, $N$. Qua $N$ kẻ đường thẳng song song với $BC$ cắt $SC$ tại $P$, qua $M$ kẻ đường thẳng song song với $AD$ cắt $SD$ tại $Q$. Suy ra $(MNPQ)$ là mặt phẳng đi qua $G$ và song song với $(ABCD)$.\\
		Khi đó $\dfrac{SM}{SA}=\dfrac{SN}{SB}=\dfrac{SP}{SC}=\dfrac{SQ}{SD}=\dfrac{SG}{SH}=\dfrac{2}{3}$.\\
		Có $\dfrac{V_{S.MNP}}{V_{S.ABC}}=\dfrac{SM}{SA}\cdot\dfrac{SN}{SB}\cdot\dfrac{SP}{SC}=\left(\dfrac{2}{3}\right)^3=\dfrac{8}{27}$\\
		$\Rightarrow{V_{S.MNP}}=\dfrac{8}{27}{V_{S.ABC}}=\dfrac{8}{27}\cdot\dfrac{1}{2}{V_{S.ABCD}}=\dfrac{4}{27}{V_{S.ABCD}}$.\\
		Có $\dfrac{V_{S.MPQ}}{V_{S.ACD}}=\dfrac{SM}{SA}\cdot\dfrac{SP}{SC}\cdot\dfrac{SQ}{SD}=\left(\dfrac{2}{3}\right)^3=\dfrac{8}{27}$.\\
		$\Rightarrow{V_{S.MPQ}}=\dfrac{8}{27}{V_{S.ACD}}=\dfrac{8}{27}\cdot\dfrac{1}{2}{V_{S.ABCD}}=\dfrac{4}{27}{V_{S.ABCD}}$.\\
		$V_{S.MNPQ}=V_{S.MNP}+V_{S.MPQ}=\dfrac{4}{27}{V_{S.ABCD}}+\dfrac{4}{27}{V_{S.ABCD}}=\dfrac{8}{27}{V_{S.ABCD}}=\dfrac{8}{27}\cdot\dfrac{81}{2}=12$ (đvtt).
	}
\end{ex}

\begin{ex}%[2H1G3-2]
	[Tiên Du-Bắc Ninh-2020] Cho hai hình chóp tam giác đều có cùng chiều cao. Biết đỉnh của hình chóp này trùng với tâm của đáy hình chóp kia, mỗi cạnh bên của hình chóp này đều cắt một cạnh bên của hình chóp kia. Cạnh bên có độ dài bằng $a$ của hình chóp thứ nhất tạo với đường cao một góc $30^\circ$, cạnh bên của hình chóp thứ hai tạo với đường cao một góc $45^\circ$. Tính thể tích phần chung của hai hình chóp đã cho.
	\choice
	{$\dfrac{3\left(2-\sqrt 3\right){a^3}}{64}$}
	{$\dfrac{\left(2-\sqrt 3\right){a^3}}{32}$}
	{\True $\dfrac{9\left(2-\sqrt 3\right){a^3}}{64}$}
	{$\dfrac{27\left(2-\sqrt 3\right){a^3}}{64}$}
	\loigiai{
		\immini{
			Hai hình chóp $A.BCD$ và $A'.B'C'D'$ là hai hình chóp đều, có chung đường cao $AA'$, $A$ là tâm của tam giác $B'C'D'$ và $A'$ là tâm của tam giác $BCD$.\\
			Ta có $(BCD)\parallel\left(B'C'D'\right)$; $AB=AC=AD=a$; $\widehat{BAA'}=\alpha$; $\widehat{AA'B'}=\beta$.\\
			Do $AB$ cắt $A'B'$ tại $M$ nên $AB'\parallel A'B$.\\
			Gọi $N$ là giao điểm của $AC$ và $A'C'$; $P$ là giao điểm của $AD$ và $A'D'$.\\
			Tương tự ta có: $AC'\parallel A'C$, $AD'\parallel A'D$.\\
			Từ đó suy ra các cạnh của $\triangle BCD$ và $\triangle B'C'D'$ song song với nhau từng đôi một.\\
			Ta có\\
			$\heva{&\dfrac{MB}{MA}=\dfrac{A'B}{AB'}\\&\dfrac{NC}{NA}=\dfrac{A'C}{AC'}\\&
				AB'=AC';\,A'B=A'C}\Rightarrow\dfrac{MB}{MA}=\dfrac{NC}{NA}\Rightarrow MN\parallel BC$.
		}{
			\begin{tikzpicture}[scale=1, font=\footnotesize, line join=round, line cap=round, >=stealth]
				\def \a{4}
				\path
				(0,0) coordinate (B)
				(\a,0) coordinate (D)
				(-60:0.5*\a) coordinate (C)
				($(C)!0.5!(D)$) coordinate (b)
				($(B)!0.5!(C)$) coordinate (d)
				($(B)!2/3!(b)$) coordinate (A')
				($(B)!(A')!(D)$) coordinate (X)
				($(X)!1.6!90:(D)$) coordinate (A)
				($(A)!0.5!(A')$) coordinate (H)
				($(A)!0.5!(B)$) coordinate (M)
				($(A)!0.5!(C)$) coordinate (N)
				($(A)!0.5!(D)$) coordinate (P)
				($(A')!2!(M)$) coordinate (B')
				($(A')!2!(N)$) coordinate (C')
				($(A')!2!(P)$) coordinate (D')
				($(B')!0.5!(C')$) coordinate (d')
				($(D')!0.5!(C')$) coordinate (b')
				($(N)!0.5!(P)$) coordinate (m)
				;
				\draw  (M)--(B)--(C)--(D)--(P)--(D')-- (D')--(B')--(C')--(D')(N)--(C)(C')--(N)(M)--(B')(b')--(B')(d')--(D')(N)--(P)(M)--(N);
				\draw [dashed] (A)--(M)--(P)--(A)--(A') (b)--(B)--(D)--(d)(A)--(N)(M)--(A')--(P)(M)--(m)(N)--(A');
				\foreach \x/\g in {A/90,B/180,C/-20,D/0,A'/-90,B'/180,C'/80,D'/0,M/180,N/180,P/0,H/20}\fill[black] (\x) circle (1pt)+(\g:.3)node{$\x$};
			\end{tikzpicture}
		}
		\noindent Tương tự ta có $NP\parallel CD$ và $MP\parallel BD$.\\
		Suy ra $\triangle MNP$ là tam giác đều. Gọi $H$ là giao điểm của $OO'$ và $(MNP)$, $H$ là tâm của tam giác $MNP$.\\
		Trong tam giác $AA'D$ có $AA'=AD\cdot\cos\alpha=a\cdot\cos\alpha $. $(1)$\\
		Đặt $x=MH$. Hai tam giác $AHM$ và tam giác $A'HM$ vuông tại $H$ cho\\
		$\heva{&AH=MH.\cot\alpha=x\cot\alpha\\&A'H=MH\cdot\cot\beta=x\cot\beta
		}\Rightarrow AA'=x\left(\cot\alpha+\cot\beta\right)$. $(2)$\\
		Từ $(1)$ và $(2)$ suy ra $a.\cos\alpha=x\left(\cot\alpha+\cot\beta\right)\Leftrightarrow x=\dfrac{a.\cos\alpha}{\cot\alpha+\cot\beta}$.\\
		Tam giác $MNP$ đều có cạnh $MN=x\sqrt 3$ nên\\
		$S_{\triangle MNP}=\dfrac{M{N^2}\sqrt 3}{4}=\dfrac{3\sqrt 3x^2}{4}=\dfrac{3\sqrt 3}{4}\cdot\dfrac{a^2\cos^2\alpha}{\left(\cot\alpha+\cot\beta\right)^2}$.\\
		Phần chung của hai hình chóp $A.BCD$ và $A'.B'C'D'$ là hai hình chóp đỉnh $A$ và $A'$ có chung nhau mặt đáy là tam giác $MNP$. Do đó thể tích của nó là\\
		$V=\dfrac{1}{3}S_{\triangle MNP}\left(AH+A'H\right)=\dfrac{1}{3}S_{\triangle MNP}\cdot AA'=\dfrac{a^3\sqrt 3\cos^3\alpha}{4\left(\cot\alpha+\cot\beta\right)^2}$.\\
		Với $\alpha=30^\circ $ và $\beta=45^\circ$ thì $V=\dfrac{9a^3}{32\left(\sqrt 3+1\right)^2}=\dfrac{9\left(2-\sqrt 3\right){a^3}}{64}$.
	}
\end{ex}

\begin{ex}%[2H1G3-2]
	[Lương Thế Vinh-Hà Nội-2020] Cho hình chóp $S.ABCD$ có đáy $ABCD$ là hình bình hành có diện tích bằng $12a^2$; khoảng cách từ $S$ tới mặt phẳng $(ABCD)$ bằng $4a$. Gọi $L$ là trọng tâm tam giác $ACD$; gọi $T$ và $V$ lần lượt là trung điểm các cạnh $SB$ và $SC$. Mặt phẳng $(LTV)$ chia hình chóp thành hai khối đa diện, hãy tính thể tích của khối đa diện chứa đỉnh $S$.
	\choice
	{$\dfrac{20a^3}{3}$}
	{$8a^3$}
	{\True $\dfrac{28a^3}{3}$}
	{$\dfrac{32a^3}{3}$}
	\loigiai{
		\immini{
			$V=V_{S.ABCD}=\dfrac{1}{3}12a^2\cdot4a=16a^3$.\\
			Mặt phẳng $(LTV)$ cắt $AB$, $CD$ ở $M$ và $N$ sao cho $MN\parallel BC\parallel TV$.\\
			Đặt $V'=V_{S.ADNMTV}=V_{S.ABMN}+V_{S.TVMN}$.\\
			Ta có $V_{S.ADNM}=\dfrac{1}{3}V$.\\
			Xét khối chóp $S.MNCB$ có đáy là hình bình hành\\
			$a=\dfrac{SM}{SM}=1$; $b=\dfrac{SN}{SN}=1$; $c=\dfrac{SB}{ST}=2$; $d=\dfrac{SC}{SV}=2$.\\
			Khi đó\\
			$\dfrac{V_{S.TVMN}}{V_{S.MNBC}}=\dfrac{a+b+c+d}{4abcd}=\dfrac{3}{8}\Rightarrow{V_{S.TVMN}}=\dfrac{2}{3}V.\dfrac{3}{8}=\dfrac{1}{4}V$.\\
			Do đó $V'=\dfrac{1}{3}V+\dfrac{1}{4}V=\dfrac{7}{12}V$ $=\dfrac{7}{12}\cdot16a^3=\dfrac{28}{3}{a^3}$.
		}{
			\begin{tikzpicture}[scale=0.8, font=\footnotesize, line join=round, line cap=round, >=stealth]
				\def \a{3.5}
				\path
				(0,0) coordinate (A)
				(\a,0) coordinate (D)
				(220:0.7*\a) coordinate (B)
				($(B)+(D)$) coordinate (C)
				(80:0.8*\a) coordinate (S)
				($(S)!0.5!(B)$) coordinate (T)
				($(S)!0.5!(C)$) coordinate (V)
				($(A)!1/3!(B)$) coordinate (M)
				($(D)!1/3!(B)$) coordinate (L)
				($(D)!1/3!(C)$) coordinate (N)
				;
				\draw  (S)--(B)--(C)--(D)--(S)--(C)(T)--(V)--(N);
				\draw [dashed] (B)--(A)--(D)(T)--(M)--(N) (A)--(C)(B)--(D)(S)--(A);
				\foreach \x/\g in {S/90,A/160,B/180,C/-20,D/0,T/150,V/50,M/-90,N/-10,L/-80}\fill[black] (\x) circle (1pt)+(\g:.3)node{$\x$};
			\end{tikzpicture}
		}
	}
\end{ex}

\begin{ex}%[2H1G3-2]
	[Thanh Chương 1-Nghệ An-2020] Cho hình chóp tứ giác đều $S.ABCD$ có thể tích bằng $1$. Gọi $M$ là trung điểm của $SA$ và $N$ là điểm đối xứng của của $A$ qua $D$ . Mặt phẳng $(BMN)$ chia khối chóp thành hai khối đa diện. Gọi $(H)$ là khối đa diện có chứa đỉnh. Thể tích của khối đa diện $(H)$ bằng
	\choice
	{\True $\dfrac{7}{12}$}
	{$\dfrac{4}{7}$}
	{$\dfrac{5}{12}$}
	{$\dfrac{3}{7}$}
	\loigiai{
		\immini{
			Gọi $O$ là tâm của hình vuông $ABCD$ ta có $SO$ là chiều cao của hình chóp.\\
			Trong mặt phẳng $(SAD)$ gọi $I$ là giao điểm của $MN$ và $SD$ ta suy ra $I$ là trọng tâm của tam giác $SAN$.\\
			Do đó $\dfrac{SI}{SD}=\dfrac{NI}{NM}=\dfrac{2}{3}$.\\
			Trong mặt phẳng $(ABCD)$ gọi $J$ là giao điểm của $BN$ và $CD$ ta suy ra $J$ là trung điểm của $CD$ và $BN$.
		}{
			\begin{tikzpicture}[scale=0.8, font=\footnotesize, line join=round, line cap=round, >=stealth]
				\def \a{3}
				\path
				(0,0) coordinate (A)
				(\a,0) coordinate (D)
				(210:0.7*\a) coordinate (B)
				($(B)+(D)$) coordinate (C)
				($(C)!0.5!(A)$) coordinate (O)
				($(A)!(O)!(D)$) coordinate (X)
				($(X)!1.4!90:(D)$) coordinate (S)
				($(S)!0.5!(A)$) coordinate (M)
				($(A)!2!(D)$) coordinate (N)
				($(D)!0.5!(C)$) coordinate (J)
				($(S)!2/3!(D)$) coordinate (I)
				;
				\draw  (S)--(B)--(C)--(J)--(I)--(S)--(C)(I)--(N)--(J);
				\draw [dashed] (B)--(A)--(D)--(J)(B)--(M)--(I) (A)--(C)(B)--(D)(C)--(A)--(S)--(O)(I)--(D)--(N)(B)--(J);
				\foreach \x/\g in {S/90,A/160,B/160,C/-20,D/40,I/60,J/-80,M/150,N/60,O/45}\fill[black] (\x) circle (1pt)+(\g:.3)node{$\x$};
			\end{tikzpicture}
		}
		\noindent Ta có $S_{\triangle ABN}=S_{ABCD}$ và $d(M,(ABCD))=\dfrac{1}{2}SO$ suy ra $V_{MABN}=\dfrac{1}{2}{V_{S.ABCD}}$. $(1)$\\
		Từ giả thiết ta có $V_{(H)}=V_{S.ABCD}-V_{ABM.DJI}$. $(2)$\\
		Xét trong khối chóp $N.ABM$ áp dụng công thức tính tỷ số thể tích ta có\\
		$\dfrac{V_{NDJI}}{V_{NABM}}=\dfrac{NI}{NM}\cdot\dfrac{ND}{NA}\cdot\dfrac{NJ}{NB}=\dfrac{1}{6}\Leftrightarrow{V_{NDJI}}=\dfrac{1}{6}{V_{NABM}}$.\\
		Do vậy $V_{ABM.DJI}=\dfrac{5}{6}{V_{NABM}}=\dfrac{5}{6}{V_{MABN}}$. $(3)$\\
		Từ $(1)$, $(2)$ và $(3)$ ta có thể tích của $(H)$ là\\
		$V_{(H)}=V_{S.ABCD}-\dfrac{5}{6}\cdot\dfrac{1}{2}{V_{S.ABCD}}=\dfrac{7}{12}$.\\
		Vậy thể tích của khối đa diện $(H)$ bằng $\dfrac{7}{12}$.
	}
\end{ex}

\begin{ex}%[2H1G3-2]
	[Tiên Lãng-Hải Phòng-2020]
	\immini{ Cho tứ diện $ABCD$ có thể tích $V$. Gọi $M$, $N$, $P$, $Q$, $R$ lần lượt là trung điểm của các cạnh $AB$, $AD$, $AC$, $DC$, $BD$ và $G$ là trọng tâm tam giác $ABC$ (như hình vẽ). Tính thể tích khối đa diện lồi $MNPQRG$ theo $V$.
		\choice
		{$\dfrac{V}{2}$}
		{$\dfrac{V}{6}$}
		{\True $\dfrac{V}{3}$}
		{$\dfrac{2V}{5}$}
	}{
		\begin{tikzpicture}[scale=1, font=\footnotesize, line join=round, line cap=round, >=stealth]
			\def \a{3.5}
			\path
			(0,0) coordinate (B)
			(\a,0) coordinate (D)
			(-60:0.5*\a) coordinate (C)
			(80:1.0*\a) coordinate (A)
			($(A)!0.5!(B)$) coordinate (M)
			($(A)!0.5!(D)$) coordinate (N)
			($(A)!0.5!(C)$) coordinate (P)
			($(B)!0.5!(D)$) coordinate (R)
			($(D)!0.5!(C)$) coordinate (Q)
			($(B)!0.5!(C)$) coordinate (K)
			($(A)!2/3!(K)$) coordinate (G)
			;
			\draw  (A)--(B)--(C)--(D)--(A)--(C)(M)--(P)--(N)--(Q)--(P)(M)--(G)--(P)(G)--(B);
			\draw [dashed] (B)--(D)(Q)--(G)--(R)--(Q)(M)--(N)--(R)(B)--(Q);
			\foreach \x/\g in {A/160,B/160,C/-20,D/40,M/160,N/60,P/-10,Q/-30,G/-90,R/45}\fill[black] (\x) circle (1pt)+(\g:.3)node{$\x$};
		\end{tikzpicture}
	}
	\loigiai{
		Ta có $V_{MNPQRG}=V_{G.MPQR}+V_{N.MPQR}$\\
		$V_{G.MPQR}=\dfrac{1}{3}{V_{B.MPQR}}=\dfrac{2}{3}{V_{B.PQR}}=\dfrac{2}{3}{V_{P.BQR}}=\dfrac{2}{3}\cdot\dfrac{1}{2}{V_{A.BQR}}=\dfrac{1}{3}\cdot\dfrac{1}{4}{V_{A.BCD}}=\dfrac{1}{12}V$.\\
		$V_{N.MPQR}=2V_{N.MPR}$ $=2\cdot V_{P.MNR}=2\cdot\dfrac{1}{2}{V_{C.MNR}}=\dfrac{1}{4}{V_{C.ABD}}=\dfrac{1}{4}V$.\\
		Vậy $V_{MNPQRG}=\dfrac{1}{12}V+\dfrac{1}{4}V=\dfrac{V}{3}$.
	}
\end{ex}

\begin{ex}%[2H1G3-2]
	[Trần Phú-Quảng Ninh-2020] Cho lăng trụ $ABC.A'B'C'$ có thể tích bằng $6$. Gọi $M$, $N$ và $P$ là các điểm nằm trên cạnh $A'B'$, $B'C'$ và $BC$ sao cho $M$ là trung điểm của $A'B'$, $B'N=\dfrac{3}{4}B'C'$ và $BP=\dfrac{1}{4}BC$. Đường thẳng $NP$ cắt đường thẳng $BB'$ tại $E$ và đường thẳng $EM$ cắt đường thẳng $AB$ tại $Q$. Thể tích của khối đa diện lồi $AQPCA'MNC'$ bằng
	\choice
	{$\dfrac{23}{3}$}
	{$\dfrac{23}{6}$}
	{\True $\dfrac{59}{12}$}
	{$\dfrac{19}{6}$}
	\loigiai{
		\immini{
			Ta có $\dfrac{EB}{EB'}=\dfrac{EQ}{EM}=\dfrac{EP}{EN}=\dfrac{BP}{B'N}=\dfrac{1}{3}$.\\
			Suy ra $\mathrm d\left(E,\left(A'B'C'\right)\right)=\dfrac{3}{2}\mathrm d\left(B,\left(A'B'C'\right)\right)$.\\
			Mà ta lại có $\dfrac{S_{B'MN}}{S_{A'B'C'}}=\dfrac{B'N}{B'C'}\cdot\dfrac{B'M}{B'A'}=\dfrac{3}{8}$.\\
			$V_{E.MB'N}=\dfrac{1}{3}\mathrm d\left(E,\left(MB'N\right)\right)\cdot S_{MB'N}=\dfrac{3}{16}{V_{ABC.A'B'C'}}=\dfrac{9}{8}$.\\
			Ta lại có $\dfrac{V_{E.QPB}}{V_{E.MNB'}}=\dfrac{EQ}{EM}\cdot\dfrac{EP}{EN}.\dfrac{EB}{EB'}=\left(\dfrac{EB}{EB'}\right)^3=\dfrac{1}{27}$.\\
			Suy ra $V_{BQP.B'MN}=V_{E.MB'N}-V_{EBQP}=\dfrac{26}{27}{V_{E.MB'N}}$.\\
			$V_{AQPCA'MNC'}=V_{ABC.A'B'C'}-V_{BQP.B'MN}=6-\dfrac{26}{27}\cdot\dfrac{9}{8}=\dfrac{59}{12}$.
		}{
			\begin{tikzpicture}[scale=1, font=\footnotesize, line join=round, line cap=round, >=stealth]
				\def \a{3.5}
				\path
				(0,0) coordinate (C')
				(\a,0) coordinate (B')
				(-60:0.5*\a) coordinate (A')
				(80:1.0*\a) coordinate (C)
				($(C)+(A')$) coordinate (A)
				($(C)+(B')$) coordinate (B)
				($(C)!0.75!(B)$) coordinate (P)
				($(A')!0.5!(B')$) coordinate (M)
				($(C')!0.25!(B')$) coordinate (N)
				(intersection of N--P and B'--B) coordinate (E)
				(intersection of M--E and A--B) coordinate (Q)
				;
				\draw  (C)--(C')--(A')--(B')--(E)--(M)(P)--(C)--(A)--(B)(E)--(P)(A)--(A')(Q)--(P);
				\draw [dashed] (C)--(N)--(P)--(B)(C')--(B')(A')--(N)--(M);
				\foreach \x/\g in {A/180,B/0,C/180,A'/200,B'/0,C'/180,M/-20,N/210,P/100,Q/-60,E/45}\fill[black] (\x) circle (1pt)+(\g:.3)node{$\x$};
			\end{tikzpicture}
		}
	}
\end{ex}
%%==========Câu 45
\begin{ex}%[2H1G3-3]
	[Sở Hà Tĩnh - 2021] Cho khối hộp $ABCD.A'B'C'D'$ có thể tích bằng $V$. Gọi $M$, $N$, $P$ lần lượt là trung điểm của $AB$, $B'C'$, $DD'$. Gọi thể tích khối tứ diện $CMNP$ là $V'$, khi đó tỉ số $\dfrac{V'}{V}$ bằng
	\choice
	{$\dfrac{1}{16}$}
	{\True $\dfrac{3}{16}$}
	{$\dfrac{1}{64}$}
	{$\dfrac{3}{64}$}
	\loigiai{
		\begin{center}
			\begin{tikzpicture}[scale=1.5, font=\footnotesize, line join=round, line cap=round, >=stealth]
				\def\bc{4} % cạnh BC
				\def\ba{2} % cạnh BA
				\def\h{4} % đường cao
				\def\gocB{35} % góc B của đáy
				\coordinate[label=below left:$B$] (B) at (0,0);
				\coordinate[label=above left:$A$] (A) at (\gocB:\ba);
				\coordinate[label=below:$C$] (C) at (\bc,0);
				\coordinate[label=right:$D$] (D) at ($(C)-(B)+(A)$);
				\coordinate[label=above left:$A'$] (A') at ($(A)+(90:\h)$);
				\coordinate[label=left:$B'$] (B') at ($(B)-(A)+(A')$);
				\coordinate[label=below right:$C'$] (C') at ($(C)-(A)+(A')$);
				\coordinate[label=right:$D'$] (D') at ($(D)-(A)+(A')$);
				\coordinate[label=left:$M$] (M) at ($(A)!0.5!(B)$);
				\coordinate[label=below left:$N$] (N) at ($(B')!0.5!(C')$);
				\coordinate[label=right:$P$] (P) at ($(D)!0.5!(D')$);
				\coordinate[label=below:$Q$] (Q) at ($(B)!0.5!(C)$);
				\coordinate[label=left:$H$] (H) at ($(A')!0.5!(B')$);
				\draw (B')--(B)--(C)--(D)--(D')--(A')--(B')--(C')--(D') (C)--(C') (Q)--(N)--(H)--(D')--(N)--(C)--(P);
				\draw[dashed,color=blue] (A')--(A)--(D) (A)--(B) (M)--(N)--(P)--(M)--(Q) (C)--(M)--(D) (P)--(H)--(M);
				\foreach \diem in {A,B,C,D,A',B',C',D',M,N,P,Q,H}	\fill (\diem)circle(1.5pt);
			\end{tikzpicture}
			
		\end{center}
		Ta có\\ $V=V'+V_{B'HN.BMQ}+V_{A'HD'.AMD}+V_{N.MQC}+V_{P.NCC'}+V_{P.D'C'N}+V_{P.D'HN}+V_{P.HNM}+V_{P.MDC}$.\\
		Gọi $S$ là diện tích đáy và $h$ là chiều cao khối hộp.\\
		Xét: $V_{B'HN.BMQ}=\dfrac{1}{8}Sh$, $V_{A'HD'.AMD}=\dfrac{1}{4}Sh$, $V_{N.MQC}=\dfrac{1}{24}Sh$, $V_{P.NCC'}=\dfrac{1}{12}Sh$,\\ $V_{P.D'C'N}=\dfrac{1}{24}Sh$, $V_{P.D'HN}=\dfrac{1}{16}Sh$, $V_{P.HNM}=V_{D'.HNM}=V_{M.HND'}=\dfrac{1}{8}Sh$, $V_{P.MDC}=\dfrac{1}{12}Sh$.\\
		Suy ra: $V=V'+\dfrac{13}{16}V\Leftrightarrow V'=\dfrac{3}{16}V\Leftrightarrow \dfrac{V'}{V}=\dfrac{3}{16}$.
	}
\end{ex}
%%==========Câu 46
\begin{ex}%[2H1G3-3]
	[Sở Tuyên Quang - 2021] Cho tứ diện $SABC$ và hai điểm $M$, $N$ lần lượt thuộc các cạnh $SA$, $SB$ sao cho $\dfrac{SM}{AM}=\dfrac{1}{2}$, $\dfrac{SN}{BN}=2$. Mặt phẳng $(P)$ đi qua hai điểm $M$, $N$ và song song với cạnh $SC$ cắt $AC$, $BC$ lần lượt tại $L$, $K$. Gọi $V$, $V'$ lần lượt là thể tích các khối đa diện $SCMNKL$, $SABC$. Tỉ số $\dfrac{V}{V'}$ bằng
	\choice
	{$\dfrac{2}{3}$}
	{\True $\dfrac{4}{9}$}
	{$\dfrac{1}{4}$}
	{$\dfrac{1}{3}$}
	\loigiai{
		\begin{center}
			\begin{tikzpicture}[scale=1.5, font=\footnotesize, line join=round, line cap=round, >=stealth]
				\def\ac{4} % cạnh AC
				\def\ab{2} % cạnh AB
				\def\as{4} % cạnh AS
				\def\gocA{50} % góc A của đáy
				\coordinate[label=left:$A$] (A) at (0,0);
				\coordinate[label=right:$C$] (C) at (\ac,0);
				\coordinate[label=below left:$B$] (B) at (-\gocA:\ab);
				\coordinate[label=above:$S$] (S) at (70:\as);
				\coordinate[label=left:$M$] (M) at ($(S)!1/3!(A)$);
				\coordinate[label=left:$N$] (N) at ($(S)!2/3!(B)$);
				\coordinate[label=above:$L$] (L) at ($(C)!1/3!(A)$);
				\coordinate[label=below right:$K$] (K) at ($(C)!2/3!(B)$);
				\coordinate[label=below:$I$] (I) at (intersection cs:first line={(M)--(N)}, second line={(L)--(K)});
				
				\draw (A)--(S)--(C)--(K)--(N)--(M) (A)--(I)--(N)--(B) (S)--(N) (I)--(K);
				\draw[dashed] (A)--(C) (B)--(K)--(L)--(M);
				\foreach \diem in {A,B,C,S,M,N,L,K,I}\fill (\diem)circle(1.5pt);
			\end{tikzpicture}
			
		\end{center}
		Gọi $I$ là giao điểm của $AB$, $MN$, $KL$.\\
		Do $ML$ $\parallel$ $SC$ và $NK$ $\parallel$ $SC$ nên ta có $\dfrac{AM}{AS}=\dfrac{AL}{AC}=\dfrac{2}{3}$ và $\dfrac{BN}{BS}=\dfrac{BK}{BC}=\dfrac{1}{3}$.\\
		Ta có $\dfrac{MA}{MS}\cdot \dfrac{NS}{NB}\cdot \dfrac{IB}{IA}=1$ suy ra $\dfrac{IB}{IA}=\dfrac{1}{4}$.\\
		Ta có $\dfrac{CL}{CA}\cdot \dfrac{BA}{BI}\cdot \dfrac{KI}{KL}=1\Leftrightarrow \dfrac{1}{3}\cdot \dfrac{3}{1}\cdot \dfrac{KI}{KL}=1\Leftrightarrow KL=KI$\\
		suy ra $MN=NI$ hay $\dfrac{IN}{IM}=\dfrac{IK}{IL}=\dfrac{1}{2}$.\\
		Xét hình chóp $IAML$ ta có $\dfrac{V_{I.BNK}}{V_{I.AML}}=\dfrac{IB\cdot IN\cdot IK}{IA\cdot IM\cdot IL}=\dfrac{1}{4}\cdot \dfrac{1}{2}\cdot \dfrac{1}{2}=\dfrac{1}{16}$.\\
		Mặt khác ta có $V_{IAML}=\dfrac{1}{3}d(I;(AML))\cdot{S_{\triangle AML}}=\dfrac{1}{3}\cdot \dfrac{4}{3}d(B;(AML))\cdot \dfrac{2}{3}\cdot \dfrac{2}{3}S_{\triangle SAC}=\dfrac{16}{27}V_{SABC}$.\\
		Suy ra $\dfrac{V_{I.BNK}}{V_{SABC}}=\dfrac{1}{16}\cdot \dfrac{16}{27}=\dfrac{1}{27}$.\\
		Suy ra $V_{I.BNK}=\dfrac{1}{27}\cdot V'\Rightarrow {V_{BNKAML}}=\dfrac{16}{27}V'-\dfrac{1}{27}\cdot V'=\dfrac{5}{9}V'$.\\
		Ta có $V_{SCMNKL}=V'-V_{BNKAML}=V'-\dfrac{5}{9}V'=\dfrac{4}{9}V'$.\\
		Từ đó ta có $\dfrac{V}{V'}=\dfrac{4}{9}$.}
\end{ex}
%%==========Câu 47
\begin{ex}%[2H1G3-3]
	[Liên trường Quỳnh Lưu - Hoàng Mai - Nghệ An - 2021] Cho lăng trụ $ABC.A'B'C'$. Gọi $M$, $N$, $Q$, $R$ lần lượt là trung điểm của các cạnh $AB$, $A'B'$, $BC$, $B'C'$ và $P$, $S$ lần lượt là trọng tâm của các tam giác $AA'B$, $CC'B$. Tỉ số thể tích khối đa diện $MNRQPS$ và khối lăng trụ $ABC.A'B'C'$ là
	\choice
	{$\dfrac{1}{9}$}
	{\True$\dfrac{5}{54}$}
	{$\dfrac{1}{10}$}
	{$\dfrac{2}{27}$}
	\loigiai{
		\textbf{(*) Cách 1:}
		\begin{center}
			\begin{tikzpicture}[scale=1.25, font=\footnotesize, line join=round, line cap=round, >=stealth]
				\def\ac{4} % cạnh AC
				\def\ab{2} % cạnh AB
				\def\ben{4} % cạnh bên
				\def\gocnghieng{75} % góc nghiêng cạnh bên
				\def\gocA{50} % góc A của đáy
				\coordinate[label=left:$A'$] (A') at (0,0);
				\coordinate[label=right:$C'$] (C') at (\ac,0);
				\coordinate[label=below left:$B'$] (B') at (-\gocA:\ab);
				\coordinate[label=left:$A$] (A) at ($(A')+(\gocnghieng:\ben)$);
				\coordinate[label=above:$B$] (B) at ($(B)-(A')+(A)$);
				\coordinate[label=right:$C$] (C) at ($(C)-(A')+(A)$);
				\coordinate[label=above:$M$] (M) at ($(A)!0.5!(B)$);
				\coordinate[label=below:$N$] (N) at ($(A')!0.5!(B')$);
				\coordinate[label=above:$Q$] (Q) at ($(B)!0.5!(C)$);
				\coordinate[label=right:$R$] (R) at ($(B')!0.5!(C')$);
				\coordinate[label=left:$P$] (P) at (intersection cs:first line={(A')--(M)}, second line={(A)--(B')});
				\coordinate[label=right:$S$] (S) at (intersection cs:first line={(Q)--(C')}, second line={(C)--(B')});
				\draw (A')--(A)--(B)--(C)--(C')--(B')--cycle (B)--(B') (N)--(M)--(Q)--(R) (M)--(A')--(B) (B')--(A)--(C) (Q)--(C')--(B);
				\draw[dashed] (A')--(C') (N)--(R)--(M);
				\foreach \diem in {A,B,C,A',B',C',M,N,Q,R,P,S} \fill (\diem)circle(1.5pt);
			\end{tikzpicture}
		\end{center}
		\begin{itemize}
			\item Đặt: $V=V_{ABC.A'B'C'}$; $V_{B'.AA'C'C}=\dfrac{1}{3}S_{AA'C'C}\cdot d\left(B',(AA'C'C)\right)=\dfrac{2}{3}V$.\\
			$\begin{aligned}[t]
				V_{B'.MNRQ}&=\dfrac{1}{3}\cdot{S_{MNRQ}}\cdot d\left(B',(MNRQ)\right)\\
				&=\dfrac{1}{3}\left(\dfrac{1}{2}S_{AA'C'C}\right)\cdot \left(\dfrac{1}{2}d\left(B',(AA'C'C)\right)\right)\\
				&=\left(\dfrac{1}{3}\cdot{S_{AA'C'C}}\cdot d\left(B',(AA'C'C)\right)\right)\cdot \dfrac{1}{4}\\
				&=\dfrac{1}{4}\cdot \dfrac{2}{3}V=\dfrac{1}{6}V.\end{aligned}$\\
			$V_{P.MNRQ}=\dfrac{1}{3}\cdot{V_{A'.MNRQ}}=\dfrac{1}{3}\cdot{V_{B'.MNRQ}}=\dfrac{1}{3}\cdot \dfrac{1}{6}V=\dfrac{1}{18}V$.
			\item $V_{A.BB'C'C}=\dfrac{1}{3}S_{BB'C'C}\cdot d\left(A,(BB'C'C)\right)=\dfrac{2}{3}V$.\\
			$S_{\triangle QRC'}=\dfrac{1}{2}S_{QRC'C}=\dfrac{1}{4}S_{BB'C'C}$;\\ $S_{\triangle QRS}=\dfrac{1}{3}S_{QRC'}=\dfrac{1}{3}\cdot \dfrac{1}{4}S_{BB'C'C}=\dfrac{1}{12}S_{BB'C'C}$.\\
			$\begin{aligned}[t]
				V_{A.QRS}&=\dfrac{1}{3}S_{\triangle QRS}\cdot d\left(A,(QRS)\right)\\
				&=\dfrac{1}{3}\left(\dfrac{1}{12}S_{BB'C'C}\right)\cdot \left(d\left(A,(BB'C'C)\right)\right)\\
				&=\left(\dfrac{1}{3}\cdot{S_{BB'C'C}}\cdot d\left(A,(BB'C'C)\right)\right)\cdot \dfrac{1}{12}\\
				&=\dfrac{1}{12}\cdot \dfrac{2}{3}V=\dfrac{1}{18}V.\end{aligned}$\\
			$V_{P.QRS}=\dfrac{PB'}{AB'}\cdot{V_{A.QRS}}=\dfrac{2}{3}\cdot \dfrac{1}{18}V=\dfrac{1}{27}V$.\\
			$V_{MNRQPS}=V_{P.MNRQ}+V_{P.QRS}=\dfrac{1}{18}V+\dfrac{1}{27}V=\dfrac{5}{54}V$.
		\end{itemize}
		Vậy: $\dfrac{V_{MNRQPS}}{V_{ABC.A'B'C'}}=\dfrac{5}{54}$.\\
		\textbf{(*) Cách 2:}
		\begin{center}
			\begin{tikzpicture}[scale=1.3, font=\footnotesize, line join=round, line cap=round, >=stealth]
				\def\ac{4} % cạnh AC
				\def\ab{2} % cạnh AB
				\def\h{4} % chiều cao
				\def\gocA{50} % góc A của đáy
				\coordinate[label=left:$A'$] (A') at (0,0);
				\coordinate[label=above right:$C'$] (C') at (\ac,0);
				\coordinate[label=below left:$B'$] (B') at (-\gocA:\ab);
				\coordinate[label=left:$A$] (A) at ($(A')+(90:\h)$);
				\coordinate[label=below left:$B$] (B) at ($(B')-(A')+(A)$);
				\coordinate[label=right:$C$] (C) at ($(C')-(A')+(A)$);
				\coordinate[label=above:$M$] (M) at ($(A)!0.5!(B)$);
				\coordinate[label=above:$Q$] (Q) at ($(B)!0.5!(C)$);
				\coordinate[label=left:$N$] (N) at ($(A')!0.5!(B')$);
				\coordinate[label=right:$R$] (R) at ($(B')!0.5!(C')$);
				\coordinate [label=right:$x$] (x) at ($(A')!1.5!(B')$);
				\coordinate [label=above:$y$] (y) at ($(A')!1.5!(C')$);
				\coordinate [label=left:$z$] (z) at ($(A')!1.3!(A)$);
				\draw[->] (B')--(x);
				\draw[->] (C')--(y);
				\draw[->] (A)--(z);
				\draw (A')--(A)--(B)--(C)--(C')--(B')--cycle (A)--(C) (B)--(B') (N)--(M)--(Q)--(R) (A')--(M) (Q)--(C');
				\draw (0,2)node[left]{$2$} (2,0)node[above]{$2$} (1,-1)node[below left]{$2$};
				\draw[dashed] (A')--(C') (N)--(R);
				\foreach \diem in {A,B,C,A',B',C',M,N,Q,R} \fill (\diem)circle(1.5pt);
			\end{tikzpicture}
		\end{center}
		\begin{itemize}
			\item Chuẩn hóa lăng trụ $ABC.A'B'C'$ là lăng trụ đứng có đáy $\triangle ABC$ vuông tại $A$ và các cạnh $AB=AC=AA'=2$.\\
			Khi đó: $V_{ABC.A'B'C'}=\left(\dfrac{1}{2}\cdot 2\cdot 2\right)\cdot 2=4$.\\
			Đặt khối lăng trụ $ABC.A'B'C'$ vào hệ trục tọa độ $Oxyz$ sao cho: $A'\equiv O$ và $B'$, $C'$, $A$ lần lượt nằm trên chiều dương của các trục $Ox$, $Oy$, $Oz$ (như hình vẽ).\\
			Ta có $A'(0;0;0)$, $B'(2;0;0)$, $C'(0;2;0)$, $A(0;0;2)$, $B(2;0;2)$, $C(0;2;2)$
			$M(1;0;2)$, $N(1;0;0)$, $R(1;1;0)$, $Q(1;1;2)$, $P\left(\dfrac{2}{3};0;\dfrac{4}{3}\right)$, $\overrightarrow{SC'}=-2\overrightarrow{SQ}$
			$\Rightarrow S=\left(\dfrac{2}{3};\dfrac{4}{3};\dfrac{4}{3}\right)$.\\
			$\overrightarrow{PM}=\left(\dfrac{1}{3};0;\dfrac{2}{3}\right)$, $\overrightarrow{PR}=\left(\dfrac{1}{3};1;-\dfrac{4}{3}\right)$, $\overrightarrow{PQ}=\left(\dfrac{1}{3};1;\dfrac{2}{3}\right)$, $\overrightarrow{PS}=\left(0;\dfrac{4}{3};0\right)$.\\
			$V_{P.MQR}=\dfrac{1}{6}\cdot \left| \left[\overrightarrow{PM};\overrightarrow{PQ}\right]\right|\cdot \overrightarrow{PR}=\dfrac{1}{6}\cdot \dfrac{2}{3}=\dfrac{1}{9}$;\\ $V_{P.MQRN}=2\cdot{V_{P.MQR}}=\dfrac{2}{9}$.\\
			$V_{P.QRS}=\dfrac{1}{6}\cdot \left| \left[\overrightarrow{PR};\overrightarrow{PQ}\right]\right|\cdot \overrightarrow{PS}=\dfrac{1}{6}\cdot \dfrac{8}{9}=\dfrac{4}{27}$.\\
			\item $V_{MNRQPS}=V_{P.MQRN}+V_{P.QRS}=\dfrac{2}{9}+\dfrac{4}{27}=\dfrac{10}{27}$.
		\end{itemize}
		Vậy: $\dfrac{V_{MNRQPS}}{V_{ABC.A'B'C'}}=\dfrac{\dfrac{10}{27}}{4}=\dfrac{5}{54}.$}
\end{ex}
%%==========Câu 48
\begin{ex}%[2H1G3-3]
	[Chuyên KHTN - 2021] Cho khối chóp tứ giác $S.ABCD$ có cạnh đáy bằng $a$ và cạnh bên bằng $2a$. Gọi $M$ là điểm đối xứng của $C$ qua $D$, $N$ là trung điểm của $SC$. Mặt phẳng $(BMN)$ chia khi chóp đã cho thành $2$ phần. Thể tích của phần chứa đỉnh $S$ bằng
	\choice
	{$\dfrac{3\sqrt{14}a^3}{32}$}
	{$\dfrac{5\sqrt{14}a^3}{72}$}
	{$\dfrac{7\sqrt{14}a^3}{96}$}
	{\True $\dfrac{7\sqrt{14}a^3}{72}$}
	\loigiai{
		\begin{center}
			\begin{tikzpicture}[scale=1.5, font=\footnotesize, line join=round, line cap=round, >=stealth]
				\def\bc{4} % cạnh BC
				\def\ba{2} % cạnh BA
				\def\h{4} % đường cao
				\def\gocB{30} % góc B của đáy
				\coordinate[label=below left:$B$] (B) at (0,0);
				\coordinate[label=above right:$C$] (C) at (\gocB:\ba);
				\coordinate[label=below:$A$] (A) at (\bc,0);
				\coordinate[label=above right:$D$] (D) at ($(C)-(B)+(A)$);
				\coordinate[label=above right:$O$] (O) at ($(A)!.5!(C)$);
				\coordinate[label=above:$S$] (S) at ($(O)+(90:\h)$);
				\coordinate[label=left:$N$] (N) at ($(S)!0.5!(C)$);
				\coordinate[label=above right:$M$] (M) at ($(C)!2!(D)$);
				\coordinate[label=above:$E$] (E) at (intersection cs:first line={(S)--(D)}, second line={(M)--(N)});
				\coordinate[label=below:$F$] (F) at (intersection cs:first line={(M)--(B)}, second line={(A)--(D)});
				\draw (B)--(A)--(F)--(E)--(S)--cycle (S)--(A) (E)--(M)--(F);
				\draw[dashed] (C)--(C)--(D)--(B) (O)--(S)--(C)--(B)--(N)--(E)--(D)--(F) (D)--(M) (B)--(F) (A)--(C);
				\foreach \diem in {A,B,C,D,S,O,M,N,E,F}	\fill (\diem)circle(1.5pt);
			\end{tikzpicture}
		\end{center}
		Giả sử các điểm như hình vẽ. $F=(BMN)\cap AD$; Kẻ $OH\perp SF$.\\
		Gọi $E=SD\cap MN\Rightarrow E$ là trọng tâm $\triangle SCM$, $DF\parallel BC\Rightarrow F$ là trung điểm $BM$.\\
		Ta có $SO=\sqrt{SD^2-DO^2}=\sqrt{(2a)^2-\left(\dfrac{a\sqrt{2}}{2}\right)^2}=\sqrt{4a^2-\dfrac{a^2}{2}}=\dfrac{a\sqrt{14}}{2}$.\\
		$\Rightarrow SF=\sqrt{SO^2+OF^2}=\sqrt{\left(\dfrac{a\sqrt{14}}{2}\right)^2+\left(\dfrac{a}{2}\right)^2}=\dfrac{a\sqrt{15}}{2}$.
		\begin{itemize}
			\item $d(M;(SBC))=4\cdot d(O;(SAD))=4OH=4\dfrac{SO\cdot OF}{SF}=\dfrac{2a\sqrt{210}}{15};\\
			S_{\triangle SAD}=\dfrac{1}{2}\cdot SF\cdot AD=\dfrac{a^2\sqrt{15}}{4}$.\\
			\item $\dfrac{V_{MEFD}}{V_{MNBC}}=\dfrac{ME}{MN}\cdot \dfrac{MF}{MB}\cdot \dfrac{MD}{MC}=\dfrac{2}{3}\cdot \dfrac{1}{2}\cdot \dfrac{1}{2}=\dfrac{1}{6}\Rightarrow {V_{MEFD}}=\dfrac{1}{6}\cdot{V_{MNBC}}$.
		\end{itemize}
		$\begin{aligned}[t]
			\Rightarrow {V_{BFDCNE}}&=\dfrac{5}{6}\cdot{V_{MNBC}}\\
			&=\dfrac{5}{6}\cdot \dfrac{1}{3}\cdot d(M;(SBC))\cdot \dfrac{1}{2}\cdot{S_{\triangle SBC}}\\
			&=\dfrac{5}{6}\cdot \dfrac{1}{3}\cdot 4OH\cdot \dfrac{1}{2}\cdot{S_{\triangle SAD}}=\dfrac{5a^3\sqrt{14}}{72}.\end{aligned}$\\
		Suy ra $V_{S.ABCD}=\dfrac{1}{3}\cdot SO\cdot{S_{ABCD}}=\dfrac{a^3\sqrt{14}}{6}\Rightarrow {V_{SABFEN}}=V_{S.ABCD}-V_{BFDCNE}=\dfrac{7a^3\sqrt{14}}{72}$.}
\end{ex}
%%==========Câu 49
\begin{ex}%[2H1G3-2]
	[Chuyên Quang Trung - Bình Phước - 2021]
	\immini{Cho hình lăng trụ $ABCD.A'B'C'D'$ đáy là hình bình hành. Với $AC=BC=a$, $CD=a\sqrt{2}$, $AC'=a\sqrt{3}$, $\widehat{CA'B'}=\widehat{A'D'C}=90^\circ$. Thể tích khối tứ diện $BCDA'$ là
		\choice
		{\True $\dfrac{a^3}{6}$}
		{$a^3$}
		{$\dfrac{2a^3}{3}$}
		{$\sqrt{6}a^3$}
	}{
		\begin{tikzpicture}[scale=.8, font=\footnotesize, line join=round, line cap=round, >=stealth]
			\def\bc{4} % cạnh BC
			\def\ba{2} % cạnh BA
			\def\h{4} % đường cao
			\def\gocB{45} % góc B của đáy
			\coordinate[label=below left:$D'$] (B') at (0,0);
			\coordinate[label=above left:$A'$] (A') at (\gocB:\ba);
			\coordinate[label=below:$C'$] (C') at (\bc,0);
			\coordinate[label=right:$B'$] (D') at ($(C')-(B')+(A')$);
			\coordinate (H) at ($(B')!0.4!(D')$); % Hình chiếu của A'
			\coordinate[label=above left:$A$] (A) at ($(H)+(122:\h)$);
			\coordinate[label=left:$D$] (B) at ($(B')-(A')+(A)$);
			\coordinate[label=below right:$C$] (C) at ($(C')-(A')+(A)$);
			\coordinate[label=right:$B$] (D) at ($(D')-(A')+(A)$);
			\draw (B)--(B') (C')--(D')--(D)--(A)--(B)--(C)--(D) (B')--(C')--(C) ;
			\draw[dashed] (A')--(D') (A)--(A')--(B') (A)--(A') ;
			\foreach \diem in {A,B,C,D,A',B',C',D'}	\fill (\diem)circle(1.5pt);
		\end{tikzpicture}
	}
	\loigiai{
		\begin{center}
			\begin{tikzpicture}[scale=.8, font=\footnotesize, line join=round, line cap=round, >=stealth]
				\def\bc{4} % cạnh BC
				\def\ba{2} % cạnh BA
				\def\h{4} % đường cao
				\def\gocB{45} % góc B của đáy
				\coordinate[label=below left:$D'$] (B') at (0,0);
				\coordinate[label=above left:$A'$] (A') at (\gocB:\ba);
				\coordinate[label=below:$C'$] (C') at (\bc,0);
				\coordinate[label=right:$B'$] (D') at ($(C')-(B')+(A')$);
				\coordinate (H) at ($(B')!0.4!(D')$); % Hình chiếu của A'
				\coordinate[label=above left:$A$] (A) at ($(H)+(122:\h)$);
				\coordinate[label=left:$D$] (B) at ($(B')-(A')+(A)$);
				\coordinate[label=below right:$C$] (C) at ($(C')-(A')+(A)$);
				\coordinate[label=right:$B$] (D) at ($(D')-(A')+(A)$);
				\coordinate[label=left:$O$] (O) at ($(A')!0.5!(C)$);
				\coordinate[label=below:$H$] (H) at ($(D')!2!(C')$);
				\draw (B)--(B') (C')--(D')--(D)--(A)--(B)--(C)--(D) (C')--(C)--(A) (C)--(H) (B')--(H)--(C');
				\draw[dashed] (A')--(D') (A)--(A')--(B') (A')--(C) (A)--(C')--(A') (B')--(C') (A')--(H);
				\foreach \diem in {A,B,C,D,A',B',C',D',O,H}	\fill (\diem)circle(1.5pt);
				\draw pic[angle radius=3mm,draw=blue,fill=green!50,opacity=.3,angle eccentricity=1.5] {right angle = C--H--C'};
				\draw pic[angle radius=3mm,draw=blue,fill=blue,opacity=.3,angle eccentricity=1.5] {right angle = C--H--B'};
			\end{tikzpicture}
		\end{center}
		Ta có tam giác $ABC$ vuông cân tại $C$.\\
		Gọi $O$ là trung điểm của $AC'\Rightarrow OC'=OA=\dfrac{a\sqrt{3}}{2}$.\\
		Gọi $H$ là chân đường cao hạ từ $C$ xuống mặt $(A'B'C'D')$.\\
		Ta có $\heva{&A'D'\perp CH \\& A'D'\perp D'C}\Rightarrow A'D'\perp HD'$.\\
		Lại có: $\heva{&A'B'\perp A'C \\&A'B'\perp CH}\Rightarrow A'B'\perp A'H$.\\
		Ta có $A'H\perp A'B'\Rightarrow \widehat{HA'B'}=90^\circ;\widehat{A'D'H}=90^\circ$. Tam giác $A'D'H$ vuông cân tại $D'$.\\
		Giả sử $CH=x\Rightarrow CA'=\sqrt{x^2+2a^2}$.\\
		$CC'^2=x^2+a^2$\\
		$C'O=\dfrac{CC'^2+C'A'^2}{2}-\dfrac{CA'^2}{4}\Leftrightarrow \dfrac{3a^2}{4}=\dfrac{x^2+a^2+a^2}{2}-\dfrac{x^2+2a^2}{4}=\dfrac{x^2+2a^2}{4}$.\\
		Suy ra $x^2+2a^2=3a^2\Rightarrow x=a=CH$.\\
		Vậy $V_{BCDA'}=\dfrac{1}{6}V_{ABCD.A'B'C'D}=\dfrac{1}{6}\cdot CH\cdot{S_{ABCD}}=\dfrac{a^3}{6}$.}
\end{ex}
%%==========Câu 50
\begin{ex}%[2H1G3-3]
	[THPT Hoàng Hoa Thám - Đà Nẵng - 2021] Cho hình hộp $ABCD.A'B'C'D'$. Gọi $G$ là trọng tâm của tam giác $ABD$. Mặt phẳng $(P)$ đi qua hai điểm $C'$, $G$ và song song với đường thẳng $BD$, chia khối hộp thành hai phần có thể thể tích $V_1$, $V_2$ $(V_1<V_2)$. Tỉ số $\dfrac{V_1}{V_2}$ bằng
	\choice
	{$\dfrac{V_1}{V_2}=\dfrac{1}{2}$}
	{$\dfrac{V_1}{V_2}=\dfrac{7}{17}$}
	{$\dfrac{V_1}{V_2}=\dfrac{2}{3}$}
	{\True $\dfrac{V_1}{V_2}=\dfrac{31}{77}$}
	\loigiai{
		\begin{center}
			\begin{tikzpicture}[scale=1.2, font=\footnotesize, line join=round, line cap=round, >=stealth]
				\def\bc{5} % cạnh BC
				\def\ba{2} % cạnh BA
				\def\h{4} % đường cao
				\def\gocB{45} % góc B của đáy
				\coordinate[label=below left:$A$] (A) at (0,0);
				\coordinate[label=above left:$B$] (B) at (\gocB:\ba);
				\coordinate[label=below:$D$] (D) at (\bc,0);
				\coordinate[label=right:$C$] (C) at ($(D)-(A)+(B)$);
				\coordinate (H) at ($(A')!0.4!(D)$); % Hình chiếu của A'
				\coordinate[label=above left:$B'$] (B') at ($(H)+(90:\h)$);
				\coordinate[label=left:$A'$] (A') at ($(A)-(B)+(B')$);
				\coordinate[label=right:$D'$] (D') at ($(D)-(B)+(B')$);
				\coordinate[label=right:$C'$] (C') at ($(C)-(B)+(B')$);
				\coordinate[label=below:$G$] (G) at ($(A)!1/3!(C)$);
				\coordinate[label=below:$J$] (J) at ($(B)!1/3!(A)$);
				\coordinate[label=below:$M$] (M) at ($(C)!4/3!(B)$);
				\coordinate[label=below:$N$] (N) at ($(C)!4/3!(D)$);
				\coordinate[label=below:$K$] (K) at (intersection cs:first line={(M)--(N)}, second line={(A)--(D)});
				\coordinate[label=right:$E$] (E) at (intersection cs:first line={(C')--(N)}, second line={(D')--(D)});
				\coordinate[label=above left:$F$] (F) at (intersection cs:first line={(B)--(B')}, second line={(C')--(M)});
				
				\draw (A')--(A)--(K) (D)--(C)--(C')--(B')--(A')--(D')--(C') (D)--(D') (C')--(N)--(K)--(E) (D)--(N);
				\draw[dashed] (C)--(B)--(A)--(C) (B')--(B)--(D) (C')--(M)--(K)--(D) (C')--(G) (M)--(B);
				\foreach \diem in {A,B,C,D,A',B',C',D',G,J,M,N,K,E,F}	\fill (\diem)circle(1.5pt);
			\end{tikzpicture}
		\end{center}
		\begin{itemize}
			\item Gọi $V$ là thể tích khối hộp $ABCD.A'B'C'D'$.\\
			Dựng $\Delta =(P)\cap (ABCD)$, ta có $\Delta \parallel BD$ (do $(P)\parallel BD$).\\
			Gọi $M$, $J$, $K$, $N$ lần lượt là giao điểm của $\Delta$ với $BC$, $AB$, $AD$, $DC$ và $F$, $E$ lần lượt là giao điểm của $MC'$ với $BB'$ và $NC'$ với $DD'$.
			\item Ta có $\dfrac{CM}{CB}=\dfrac{CN}{CD}=\dfrac{4}{3}$. Suy ra $S_{CMN}=\left(\dfrac{4}{3}\right)^2\cdot{S_{CBD}}=\dfrac{16}{9}S_{CBD}$.\\
			Mặt khác $\dfrac{JB}{JA}=\dfrac{JM}{JK}=\dfrac{1}{2}$. Suy ra $S_{JBM}=\left(\dfrac{1}{2}\right)^2\cdot{S_{JAK}}=\dfrac{1}{4}S_{JAK}$.\\
			Mà $\dfrac{AJ}{AB}=\dfrac{AK}{AD}=\dfrac{2}{3}$. Suy ra $S_{AJK}=\left(\dfrac{2}{3}\right)^2\cdot{S_{ABD}}=\dfrac{4}{9}S_{ABD}$.\\ Suy ra $S_{JBM}=\dfrac{1}{4}\cdot \dfrac{4}{9}S_{ABD}=\dfrac{1}{9}S_{ABD}$.\\
			\item Tương tự $S_{NKD}=\dfrac{1}{9}S_{ABD}$.\\
			\item Ta lại có $\dfrac{d\left(C',\left(ABCD\right)\right)}{d\left(F,\left(ABCD\right)\right)}=\dfrac{MC'}{MF}=4\Rightarrow h=d\left(C',\left(ABCD\right)\right)=4d\left(F,\left(ABCD\right)\right)$.\\
			\item Tương tự $h=d\left(C', (ABCD)\right)=4d\left(E, (ABCD)\right)$.\\
			\item Thể tích $\begin{aligned}[t]
				V_1&=V_{C'.CMN}-V_{F.MBJ}-V_{E.KDN}\\
				&=\dfrac{1}{3}\cdot \dfrac{16}{9}S_{BCD}\cdot h-2\cdot \dfrac{1}{3}\cdot \dfrac{1}{9}S_{BCD}\cdot \dfrac{1}{4}h\\
				&=\dfrac{31}{54}S_{BCD}\cdot h\\
				&=\dfrac{31}{108}S_{ABCD}\cdot h\\
				&=\dfrac{31}{108}V\Rightarrow V_2=\dfrac{77}{108}V.\end{aligned}$
		\end{itemize}
		Vậy $\dfrac{V_1}{V_2}=\dfrac{31}{77}$.}
\end{ex}
%%==========Câu 51
\begin{ex}%[2H1G3-3]
	[THPT Chu Văn An - Thái Nguyên - 2021] Cho khối lăng trụ đứng $ABC.A'B'C'$ có đáy tam giác vuông cân tại $C$. $BA=2a$ và góc tạo bởi $(ABC')$ và $(ABC)$ bằng $60^\circ$. Gọi $M$, $N$ lần lượt là trung điểm của $A'C'$ và $BC$. Mặt $(AMN)$ chia khối lăng trụ thành hai phần. Tìm thể tích phần nhỏ.
	\choice
	{\True $\dfrac{7\sqrt{3}a^3}{24}$}
	{$\dfrac{7\sqrt{6}a^3}{24}$}
	{$\dfrac{\sqrt{3}a^3}{3}$}
	{$\dfrac{\sqrt{6}a^3}{6}$}
	\loigiai{
		\begin{center}
			\begin{tikzpicture}[scale=1, font=\footnotesize, line join=round, line cap=round, >=stealth]
				\def\ac{4} % cạnh AC
				\def\ab{2} % cạnh AB
				\def\h{4} % chiều cao
				\def\gocA{50} % góc A của đáy
				\coordinate[label=left:$A$] (A) at (0,0);
				\coordinate[label=right:$C$] (C) at (\ac,0);
				\coordinate[label=below left:$B$] (B) at (-\gocA:\ab);
				\coordinate[label=left:$A'$] (A') at ($(A)+(90:\h)$);
				\coordinate[label=below right:$B'$] (B') at ($(B)-(A)+(A')$);
				\coordinate[label=right:$C'$] (C') at ($(C)-(A)+(A')$);
				\coordinate[label=above:$M$] (M) at ($(A')!0.5!(C')$);
				\coordinate[label=below:$N$] (N) at ($(B)!0.5!(C)$);
				\coordinate[label=left:$J$] (J) at ($(A)!0.5!(B)$);
				\coordinate[label=above:$P$] (P) at ($(C')!1/4!(B')$);
				\draw (A')--(A)--(B)--(C)--(C')--(A')--(B')--(C') (B)--(B') (M)--(P)--(N);
				\draw[dashed] (A)--(C) (A)--(M)--(N)--cycle (A')--(N) (C')--(J)--(C);
				\foreach \diem in {A,B,C,A',B',C',M,N,J,P} \fill (\diem)circle(1.5pt);
			\end{tikzpicture}
		\end{center}
		Kẻ $MP\parallel A'B'$\\
		Góc tạo bởi $(ABC')$ và $(ABC)$ là góc $\widehat{C'JC}=60^\circ$ với $J$ là trung điểm $AB$.\\
		$CC'=CJ\cdot \tan{60^\circ}=a\sqrt{3}$.\\
		$S_{ABC}=\dfrac{1}{2}CJ\cdot AB=a^2$.\\
		$S_1=S_{ACN}=\dfrac{1}{2}S_{ABC}=\dfrac{1}{2}a^2$.\\
		$S_2=S_{C'MP}=\dfrac{1}{4}S_{ABC}=\dfrac{1}{2}C'M\cdot C'P=\dfrac{1}{8}a^2$.\\
		$V=\dfrac{CC'}{3}(S_1+S_2+\sqrt{S_1S_2})=\dfrac{7\sqrt{3}a^3}{24}$.}
\end{ex}


\begin{ex}%[2H1K3-2]
	[THPT Ba Đình - Thanh Hóa - 2021]
	Cho hình lăng trụ $ABC.A'B'C'$ có thể tích bằng $2$. Gọi $M,N$ là các điểm lần lượt nằm trên các cạnh $AA',BB'$ sao cho $M$ là trung điểm của $AA'$ và $BN=\dfrac{1}{2}B'N$. Đường thẳng $CM$ cắt đường thẳng $A'C'$ tại điểm $P$, đường thẳng $CN$ cắt đường thẳng $A'B'$ tại $Q$. Tính thể tích của khối đa diện $A'MPB'NQ$ bằng.
	\choice
	{ $\dfrac{13}{18}$}
	{\True $\dfrac{23}{9}$}
	{ $\dfrac{21}{9}$}
	{ $\dfrac{7}{18}$}
	\loigiai{
		\begin{center}
			\begin{tikzpicture}[scale=1,font=\footnotesize,line join=round,line cap=round,>=stealth]
				\def\a{4}
				\def\h{4.5}
				\path 	(0:0) coordinate (A')
				++(0:\a) coordinate (C')
				++(-160:\a/2) coordinate (B')
				($(B')!0.5!(A')$) coordinate (H)
				($(C')!3/2!(H)$) coordinate (G)
				($(G)+(90:\h)$) coordinate (A)
				($(A)+(C')-(A')$) coordinate (C)
				($(C)+(B')-(C')$) coordinate (B)
				($(A)!0.5!(A')$) coordinate (M)
				($(B')!2/3!(B)$) coordinate (N)
				($(M)!1!180:(C)$) coordinate (P)
				($(N)!2!180:(C)$) coordinate (Q);
				\draw[dashed,thick] (A')--(C')  (A')--(B') (M)--(C) (A')--(P);
				\draw[thick] (C)--(C') 	(B)--(B') 	(A)--(A') (M)--(N) (P)--(M) (P)--(Q)--(B') (Q)--(C)
				(A)--(B)--(C)--cycle (B')--(C');
				\foreach \x/\g in {A/180,B/-145,C/0,A'/-150,B'/-45,C'/0,M/180,N/-45,P/90,Q/-90}
				\fill[black] 	(\x) circle (1pt)
				($(\g:4mm)+(\x)$) node {$\x$};
				%			\draw pic[draw,angle radius=2mm]{right angle=A'--G--C};%Theo chiều dương
			\end{tikzpicture}
		\end{center}
		Đặt $S={S_{\triangle A'B'C'}}$ và $h=d(C,(A'B'C'))$ ta có ${V_{ABC.A'B'C'}}=hS=2$.\\
		Trong mặt phẳng $\left( AA'C'C \right)$ ta có $\heva{
			& A'M=\dfrac{1}{2}CC' \\
			& A'M\subset CC' \\
		}$ nên ta có $A'$ là trung điểm của $PC'$.\\
		Tương tự trong mặt mặt phẳng $\left( BCC'B' \right)$ ta có $C'B'=\dfrac{1}{3}C'Q$.\\
		Từ đây ta có diện tích tam giác $C'PQ$ là ${S_{\triangle C'PQ}}=6S$ do vậy thể tích khối tứ diện $CC'PQ$ là $V_{CC'PQ}=\dfrac{1}{3}h\cdot 6S=2hS=4$.\\
		Trong khối lăng trụ $ABC.A'B'C'$ ta có $$\dfrac{V_{CABMN}}{V_{CAB.C'A'B'}}=\dfrac{1}{3}\left(\dfrac{AM}{A'M}+\dfrac{BN}{B'N}+\dfrac{CC}{CC'}\right)=\dfrac{\dfrac{1}{2}+\dfrac{1}{3}+0}{3}=\dfrac{5}{18}.$$
		Suy ra ${V_{CABMN}}=\dfrac{5}{18}.{V_{CAB.C'A'B'}}=\dfrac{5}{9}$ do đó thể tích khối $A'B'C'MNC$ bằng $2-\dfrac{5}{9}=\dfrac{13}{9}$.\\
		Do vậy thể tích của khối đa diện $A'MPB'NQ$ bằng $4-\dfrac{13}{9}=\dfrac{23}{9}$.}
\end{ex}
\begin{ex}%[2H1G3-2]
	[THPT Quốc Oai - Hà Nội - 2021]
	Cho hình chóp $S.ABC$ có đáy là tam giác đều cạnh $a$, mặt bên $SAB$ là tam giác đều và nằm trong mặt phẳng vuông góc với đáy. Gọi $H,M,O$ lần lượt là trung điểm các cạnh $AB$, $SA$, $AC$ và $G$ là trọng tâm tam giác $SBC$. Thể tích khối tứ diện $GHMO$ bằng
	\choice
	{ $\dfrac{3a^3}{64}$}
	{ $\dfrac{3a^3}{128}$}
	{ $\dfrac{a^3}{128}$}
	{\True $\dfrac{a^3}{64}$}
	\loigiai{
		\begin{center}
			\begin{tikzpicture}[scale=1,font=\footnotesize,line join=round,line cap=round,>=stealth]
				\def\a{5}
				\pgfmathsetmacro\h{\a*sqrt(3)/2}
				\path 	(0:0) coordinate (A)
				++(0:\a) coordinate (B)
				++(-140:4*\a/5) coordinate (C)
				($(A)!1/2!(B)$) coordinate (H)
				($(H)+(90:\h)$) coordinate (S)
				($(A)!1/2!(S)$) coordinate (M)
				($(A)!1/2!(C)$) coordinate (O)
				($(C)!1/2!(B)$) coordinate (N)
				($(S)!2/3!(N)$) coordinate (G)
				($(S)!1/2!(B)$) coordinate (E);
				\draw[thick,blue] (C)--(A) (E)--(C)--(B)
				(A)--(S)--(N)	(B)--(S)--(O)--(M)	(C)--(S);
				\draw[dashed,thick,red] (A)--(B) (S)--(H)--(M)--(G)--(H) (O)--(H)--(N)--(O);
				\foreach \x / \goc in 		{A/180,B/0,C/-135,H/-90,S/90,M/145,O/-135,G/45,N/-45,E/45}
				\fill (\x) circle (1.5pt)
				($(\x)+(\goc:3mm)$) node {$\x$};
				
				\draw pic[draw,angle radius=2mm]{right angle=A--H--S};%Theo chiều dương
			\end{tikzpicture}
		\end{center}
		Gọi $N,E$ lần lượt là trung điểm của $CB$ và $SB$.\\
		Ta có ${V_{S.ABC}}=\dfrac{1}{3}{S_{\triangle ABC}}\cdot SH=\dfrac{1}{3}\cdot\dfrac{a^2\sqrt{3}}{4}\cdot\dfrac{a\sqrt{3}}{2}=\dfrac{a^3}{8}$.\\
		+)${S_{OAHN}}=\dfrac{1}{2}{S_{\triangle ABC}}\Rightarrow {V_{S.OAHN}}=\dfrac{1}{2}{V_{S.ABC}}=\dfrac{a^3}{16}$, ${V_{S.AHN}}={V_{S.OAN}}=\dfrac{1}{2}{V_{S.AHNO}}=\dfrac{a^3}{32}$.\\
		+) $\dfrac{{V_{S.GMH}}}{{V_{S.NAH}}}=\dfrac{SG}{SN}.\dfrac{SM}{SA}\cdot\dfrac{SH}{SH}=\dfrac{2}{3}\cdot\dfrac{1}{2}=\dfrac{1}{3}\Rightarrow {V_{S.GMH}}=\dfrac{1}{3}{V_{S.NAH}}=\dfrac{a^3}{96}$.\\
		+) $\dfrac{{V_{S.GMO}}}{{V_{S.NAO}}}=\dfrac{SG}{SN}\cdot\dfrac{SM}{SA}\cdot\dfrac{SO}{SO}=\dfrac{2}{3}\cdot\dfrac{1}{2}=\dfrac{1}{3}\Rightarrow {V_{S.GMO}}=\dfrac{1}{3}{V_{S.NAO}}=\dfrac{a^3}{96}$.\\
		+) ${V_{G.ONH}}=\dfrac{1}{3}d\left( G,\left( ABC \right) \right)\cdot{S_{\triangle ONH}}=\dfrac{1}{3}\cdot\dfrac{1}{3}SH\cdot\dfrac{1}{4}{S_{\triangle ABC}}=\dfrac{1}{12}\cdot\dfrac{1}{3}SH\cdot{S_{\triangle ABC}}=\dfrac{1}{12}{V_{S.ABC}}=\dfrac{a^3}{96}$.\\
		+) ${V_{M.OAH}}=\dfrac{1}{3}d\left( M,\left( ABC \right) \right)\cdot{S_{\triangle OAH}}=\dfrac{1}{3}\cdot\dfrac{1}{2}SH\cdot\dfrac{1}{4}{S_{\triangle ABC}}=\dfrac{1}{8}\cdot\dfrac{1}{3}SH\cdot{S_{\triangle ABC}}=\dfrac{1}{8}{V_{S.ABC}}=\dfrac{a^3}{64}$.\\
		Vậy ${V_{GMOHN}}={V_{S.OAHN}}-{V_{S.GMH}}-{V_{S.GMO}}-{V_{G.HNO}}-{V_{G.HAO}}=\dfrac{a^3}{16}-3\cdot\dfrac{a^3}{96}-\dfrac{a^3}{64}=\dfrac{a^3}{64}$.}
\end{ex}
\begin{ex}%[2H1G3-2]
	[THPT Đồng Quan - Hà Nội - 2021]
	Cho lăng trụ đứng $ABC.A'B'C'$ có $AB=a$, $AC=2a$, $\widehat{BAC}=120^{\circ} $. Gọi $L$, $K$ lần lượt là tâm của các mặt bên $BCC'B'$, $ABB'A'$ và $E$ là trung điểm của $CC'$ (tham khảo hình vẽ).
	\begin{center}
		\begin{tikzpicture}[scale=1,font=\footnotesize,line join=round,line cap=round,>=stealth]
			\def\a{5}
			\def\h{4}
			\path 	(0:0) coordinate (B)
			++(0:\a) coordinate (C)
			++(-150:3*\a/4) coordinate (A)
			($(A)+(90:\h)$) coordinate (A')
			($(B)+(90:\h)$) coordinate (B')
			($(C)+(90:\h)$) coordinate (C')
			($(C)!1/2!(C')$) coordinate (E)
			($(A)!1/2!(B')$) coordinate (K)
			($(C)!1/2!(B')$) coordinate (L);
			\draw[dashed,thick,red] 	(C')--(B)--(C)--(B') (L)--(K)--(E)--(L);
			\draw[thick,blue]	(C)--(C') 	(B)--(B')--(A)	(A)--(A')--(B) (B)--(A)--(C) (A')--(B')--(C')--cycle;
			\foreach \x/\g in {A/-45,B/180,C/0,A'/-145,B'/-180,C'/0,E/0,K/145,L/90}
			\fill[black] 	(\x) circle (1pt)
			($(\g:4mm)+(\x)$) node {$\x$};
		\end{tikzpicture}
	\end{center}
	Biết hai mặt phẳng $\left( ACB' \right)$, $\left( ABC' \right)$ tạo với nhau một góc $\alpha $ thỏa mãn $\cos \alpha =\dfrac{\sqrt{10}}{5}$. Thể tích khối đa diện lồi có các đỉnh $A,B,C,K,E,I$ là.
	\choice
	{ $\dfrac{a^3}{2}$}
	{ $\dfrac{7a^3}{16}$}
	{ $\dfrac{5a^3}{8}$}
	{\True $\dfrac{9a^3}{16}$}
	\loigiai{
		\begin{center}
			\begin{tikzpicture}[scale=1,font=\footnotesize,line join=round,line cap=round,>=stealth]
				\def\a{5}
				\def\h{4}
				\path 	(0:0) coordinate (B)
				++(0:\a) coordinate (C)
				++(-150:3*\a/4) coordinate (A)
				($(A)+(90:\h)$) coordinate (A')
				($(B)+(90:\h)$) coordinate (B')
				($(C)+(90:\h)$) coordinate (C')
				($(C)!1/2!(C')$) coordinate (E)
				($(A)!1/2!(B')$) coordinate (K)
				($(C)!1/2!(B')$) coordinate (L)
				($(A)!2!-90:(B)$) coordinate (I')
				($(A)!1.5!(B)$) coordinate (Z)
				($ (A)!0.6!(I') $) coordinate (I);
				\draw[dashed,thick,red] 	(C')--(B)--(C)--(B') ;
				\draw[thick,blue]	(C)--(C') 	(B)--(B')--(A)	(A)--(A')--(B) (B)--(A)--(C) (A')--(B')--(C')--cycle;
				\draw[dashed] (C)--(I);
				\draw[->] (A')--($ (A')+(90:1) $)node[right]{$ z $};
				\draw[->] (B)--(Z)node[below left]{$ y $};
				\draw[dashed,->] (A)--(I')node[below right]{$ x $};
				\foreach \x/\g in {A/-45,B/180,C/0,A'/-180,B'/-180,C'/0,K/145,L/90,I/180}
				\fill[black] 	(\x) circle (1pt)
				($(\g:4mm)+(\x)$) node {$\x$};
			\end{tikzpicture}
			\begin{tikzpicture}[scale=1,font=\footnotesize,line join=round,line cap=round,>=stealth]
				\def\a{5}
				\def\h{4}
				\path 	(0:0) coordinate (B)
				++(0:\a) coordinate (C)
				++(-150:3*\a/4) coordinate (A)
				($(A)+(90:\h)$) coordinate (A')
				($(B)+(90:\h)$) coordinate (B')
				($(C)+(90:\h)$) coordinate (C')
				($(C)!1/2!(C')$) coordinate (E)
				($(A)!1/2!(B')$) coordinate (K)
				($(A)!1/2!(A')$) coordinate (H)
				($(B)!1/2!(B')$) coordinate (F)
				($(C)!1/2!(B')$) coordinate (L);
				\draw[dashed,thick,red] 	(C')--(B)--(C)--(B') (L)--(K)--(E)--(L)--(F);
				\draw[thick,blue]	(C)--(C') 	(B)--(B')--(A)	(A)--(A')--(B) (B)--(A)--(C) (A')--(B')--(C')--cycle (F)--(H)--(E)--(A);
				\foreach \x/\g in {A/-45,B/180,C/0,A'/-180,B'/-180,C'/0,E/0,K/182,L/90,H/45,F/180}
				\fill[black] 	(\x) circle (1pt)
				($(\g:4mm)+(\x)$) node {$\x$};
			\end{tikzpicture}
		\end{center}
		Kẻ tia $Ax$ vuông góc với $AB$ trên mặt phẳng $\left( ABC \right)$, chọn hệ trục tọa độ như hình vẽ.\\
		Gọi $I$ là hình chiếu của $C$ trên trục $Ax$, đặt $AA'=h$.\\
		Vì $ \widehat{BAC}=120^{\circ}=\widehat{IAB}+\widehat{IAC}=90^{\circ}+\widehat{IAC}
		\Rightarrow \widehat{IAC}=30^{\circ} $.\\
		Ta có $\widehat{IAC}=30^{\circ} \Rightarrow IC=AC\sin 30^{\circ} =a$, $IA=AC\cos 30^{\circ} =a\sqrt{3}$.\\
		Khi đó, trên hệ trục tọa độ đã chọn Ta có\\
		$A\left( 0;0;0 \right)$, $B\left( 0;a;0 \right)$, $C\left( a\sqrt{3};-a;0 \right)$, $B'\left( 0;a;h \right)$, $C'\left( a\sqrt{3};-a;h \right)$.\\
		+) $\overrightarrow{AC}=\left( a\sqrt{3};-a;0 \right),\overrightarrow{AB'}=\left( 0;a;h \right)\Rightarrow \left[ \overrightarrow{AC};\overrightarrow{AB'} \right]=\left( -ah;-ah\sqrt{3};a^2\sqrt{3} \right)$.\\
		Mặt phẳng $\left( ACB' \right)$ có VTPT $\overrightarrow{n_1}=\left( h;h\sqrt{3};-a\sqrt{3} \right)$.\\
		+) $\overrightarrow{AB}=\left( 0;a;0 \right),\overrightarrow{AC'}=\left( a\sqrt{3};-a;h \right)\Rightarrow \left[ \overrightarrow{AB};\overrightarrow{AC'} \right]=\left( ah;0;-a^2\sqrt{3} \right)$.\\
		Mặt phẳng $\left( ABC' \right)$ có VTPT $\overrightarrow{n_2}=\left( h;0;-a\sqrt{3} \right)$.\\
		Suy ra: $\cos \alpha =\left| \cos \left( \overrightarrow{n_1},\overrightarrow{n_2} \right) \right|\Leftrightarrow \dfrac{\sqrt{10}}{5}=\dfrac{\left| {h^2}+3a^2 \right|}{\sqrt{4h^2+3a^2}\sqrt{h^2+3a^2}}\Leftrightarrow \dfrac{\sqrt{10}}{5}=\dfrac{\sqrt{h^2+3a^2}}{\sqrt{4h^2+3a^2}}$\\
		$\Leftrightarrow \dfrac{2}{5}=\dfrac{h^2+3a^2}{4h^2+3a^2}\Leftrightarrow 8h^2+6a^2=5h^2+15a^2\Leftrightarrow h=a\sqrt{3}$.\\
		Gọi $V$ là thể tích khối đa diện lồi có các đỉnh $A,B,C,K,E,L$. Hai điểm $F,H$ lần lượt là trung điểm của $BB'$ và $AA'$.\\
		Ta có $\dfrac{{V_{B.FKL}}}{{V_{B.B'A'C'}}}=\dfrac{BF}{BB'}\cdot\dfrac{BK}{BA'}\cdot\dfrac{BL}{BC'}=\dfrac{1}{2}\cdot\dfrac{1}{2}\cdot\dfrac{1}{2}=\dfrac{1}{8}$.\\
		$\Rightarrow {V_{B.FKL}}=\dfrac{1}{8}{V_{B.B'A'C'}}=\dfrac{1}{8}\cdot\dfrac{1}{3}{V_{ABC.A'B'C'}}=\dfrac{1}{24}{V_{ABC.A'B'C'}}$.\\
		${V_{A.KHE}}=\dfrac{1}{3}AH\cdot{S_{HKE}}=\dfrac{1}{3}\cdot\dfrac{1}{2}AA'\cdot\dfrac{1}{2}{S_{HFE}}=\dfrac{1}{12}\cdot AA'\cdot{S_{HFE}}=\dfrac{1}{12}\cdot {V_{ABC.A'B'C'}}$.\\
		Vậy
		\begin{eqnarray*}
			V
			&=& {V_{ABC.HEF}}-{V_{A.HKE}}-{V_{B.FKL}}=\dfrac{1}{2}{V_{ABC.A'B'C'}}-\dfrac{1}{24}{V_{ABC.A'B'C'}}-\dfrac{1}{12}{V_{ABC.A'B'C'}}\\
			&=& \dfrac{3}{8}{V_{ABC.A'B'C'}}=\dfrac{3}{8}\cdot AA'\cdot\dfrac{1}{2}\cdot AB\cdot AC\sin 120^{\circ} =\dfrac{3}{8}\cdot h\cdot\dfrac{1}{2}\cdot AB\cdot AC\sin 120^{\circ} =\dfrac{9a^3}{16}.
		\end{eqnarray*}
	}
\end{ex}
\begin{ex}%[2H1K3-2]
	[Trung Tâm Thanh Tường -2021]
	Cho hình lăng trụ $ABC.A'B'C'$ có đáy ABC là tam giác đều cạnh bằng 2, $A'A=A'B=A'C=2$, M là trung điểm của $AA'$. Tính thể tích phần chung của 2 khối đa diện $A'M.BCC'B'$ và $A.A'B'C'$.\\
	\immini{
		\choice
		{\True $\dfrac{17\sqrt{2}}{27}$}
		{ $\dfrac{17\sqrt{3}}{18}$}
		{ $\dfrac{17\sqrt{3}}{27}$}
		{ $\dfrac{5\sqrt{2}}{3}$}}{
		\begin{tikzpicture}[scale=1,font=\footnotesize,line join=round,line cap=round,>=stealth]
			\def\a{4}
			\def\h{4.5}
			\path 	(0:0) coordinate (A)
			++(0:\a) coordinate (C)
			++(-135:\a/2) coordinate (B)
			($(B)!0.5!(C)$) coordinate (I)
			($(A)!2/3!(I)$) coordinate (H)
			($(H)+(90:\h)$) coordinate (A')
			($(A')+(C)-(A)$) coordinate (C')
			($(C')+(B)-(C)$) coordinate (B')
			($(A)!0.5!(A')$) coordinate (M);
			\draw[dashed,thick] (A)--(C) (M)--(C) (A)--(C');
			\draw[thick] (C)--(C') 	(B)--(B') 	(A)--(A') (B)--(M) (A)--(B')
			(A)--(B)--(C) (A')--(B')--(C')--cycle;
			\foreach \x/\g in {A/180,B/-45,C/0,A'/180,B'/-45,C'/0,M/180}
			\fill[black] 	(\x) circle (1pt)
			($(\g:4mm)+(\x)$) node {$\x$};
			%	\draw pic[draw,angle radius=2mm]{right angle=A'--G--C};%Theo chiều dương
	\end{tikzpicture}}
	
	\loigiai{
		\begin{center}
			\begin{tikzpicture}[scale=1,font=\footnotesize,line join=round,line cap=round,>=stealth]
				\def\a{4}
				\def\h{4.5}
				\path 	(0:0) coordinate (A)
				++(0:\a) coordinate (C)
				++(-135:\a/2) coordinate (B)
				($(B)!0.5!(C)$) coordinate (I)
				($(A)!2/3!(I)$) coordinate (H)
				($(H)+(90:\h)$) coordinate (A')
				($(A')+(C)-(A)$) coordinate (C')
				($(C')+(B)-(C)$) coordinate (B')
				($(A)!0.5!(A')$) coordinate (M);
				\path[name path=MB] (M)--(B);
				\path[name path=AB'] (A)--(B');
				\path[name intersections={of=MB and AB'}] (intersection-1) coordinate (P);
				\path[name path=MC] (M)--(C);
				\path[name path=AC'] (A)--(C');
				\path[name intersections={of=MC and AC'}] (intersection-1) coordinate (Q);
				\draw[dashed,thick] (A)--(C) (M)--(C) (I)--(A)--(C') (A')--(H);
				\draw[thick] (C)--(C') 	(B)--(B') 	(A)--(A') (B)--(M) (A)--(B')
				(A)--(B)--(C) (A')--(B')--(C')--cycle;
				\foreach \x/\g in {A/180,B/-45,C/0,A'/180,B'/-45,C'/0,H/45,M/180,I/-45,P/-90,Q/100}
				\fill[black] 	(\x) circle (1pt)
				($(\g:4mm)+(\x)$) node {$\x$};
				\draw pic[draw,angle radius=2mm]{right angle=I--H--A'};%Theo chiều dương
			\end{tikzpicture}
		\end{center}
		Gọi $ H $ là hình chiếu vuông góc của $A'$ lên mặt phẳng $ (ABC) $. Vì $A'A=A'B=A'C$ nên $ H $ trùng với tâm đường tròn ngoại tiếp tam giác $ ABC $ cũng là trọng tâm tam giác $ ABC $.\\
		Gọi $ I $ là trung điểm $ BC $.\\
		Ta có\\
		$AI=\sqrt{3}\Rightarrow AH=\dfrac{2}{3}AI=\dfrac{2\sqrt{3}}{3}$; $A'H=\sqrt{{AA'}^2-AH^2}=\sqrt{4-\dfrac{12}{9}}=\dfrac{2\sqrt{6}}{3}$;\\
		${V_{ABC.A'B'C'}}=A'H\cdot{S_{ABC}}=\dfrac{2\sqrt{6}}{3}\cdot\sqrt{3}=2\sqrt{2}$; ${V_{A.A'B'C'}}=\dfrac{1}{3}{V_{ABC.A'B'C'}}=\dfrac{2\sqrt{2}}{3}$;\\
		Gọi $P=AB'\cap BM$; $Q=AC'\cap CM$.\\
		Khi đó phần chung của 2 khối đa diện $A'M.BCC'B'$ và $A.A'B'C'$ là khối đa diện $MPQ.A'B'C'$.\\
		Ta có $\dfrac{{V_{A.MPQ}}}{{V_{A.A'B'C'}}}=\dfrac{AM}{AA'}\cdot\dfrac{AP}{AB'}\cdot\dfrac{AQ}{AC'}=\dfrac{1}{2}\cdot\dfrac{1}{3}\cdot\dfrac{1}{3}=\dfrac{1}{18}$\\
		$\Rightarrow {V_{MPQ.A'B'C'}}=\dfrac{17}{18}{V_{A.A'B'C'}}=\dfrac{17}{18}\cdot\dfrac{2\sqrt{2}}{3}=\dfrac{17\sqrt{2}}{27}$.}
\end{ex}
\begin{ex}%[2H1K3-2]
	[Trung Tâm Thanh Tường - 2021]
	Cho hình chóp $S.ABCD$ có đáy $ABCD$ là hình thang vuông tại $A$ và $B$, $AB=BC=a$, $AD=2a$, $SA$ vuông góc với đáy, $SA=a\sqrt{2}$. Gọi $B'$ là điểm đối xứng của $B$ qua mặt phẳng $\left( SCD \right)$. Tính thể tích khối đa diện $SB'.ABCD$ bằng
	\begin{center}
		\begin{tikzpicture}[scale=1,font=\footnotesize,line join=round,line cap=round,>=stealth]
			\def\a{4}
			\def\h{3.5}
			\path 	(0:0) coordinate (A)
			++(0:\a) coordinate (D)
			(-130:\a/2) coordinate (B)
			($ (B)+(0:\a/1.5) $) coordinate (C)
			($(A)+(90:\h)$) coordinate (S)
			($ (C)!0.5!(D) $) coordinate (C')
			($ (S)!2/3!(C') $) coordinate (B')
			(intersection of A--C and B--D) coordinate (O);%giao điểm O
			\draw[dashed,thick] 	(B)--(A)--(D)	(A)--(S) (C)--(S) ;
			\draw[thick] 			(B)-- (C)--(D)
			(B)--(S)--(B')--(C)		(B')--(D)--(S) (B)--(B');
			\foreach \x/\g in {A/135,B/-135,C/-45,D/45,S/90,B'/45}
			\fill[black] 	(\x) circle (1.5pt)
			($(\g:3mm)+(\x)$) node {$\x$};
			\draw pic[draw,angle radius=3mm]{right angle=D--A--S};%Theo chiều dương
		\end{tikzpicture}
	\end{center}
	\choice
	{ $\dfrac{5\sqrt{2}{a^3}}{6}$}
	{ $\dfrac{7\sqrt{2}{a^3}}{3}$}
	{ $\sqrt{2}{a^3}$}
	{\True $\dfrac{2\sqrt{2}{a^3}}{3}$}
	\loigiai{
		Ta có
		\begin{eqnarray*}
			{V_{SB'.ABCD}}
			&=& {V_{S.ABCD}}+{V_{B'SCD}}=\dfrac{1}{3}SA\cdot{S_{ABCD}}+\dfrac{1}{3}{S_{SCD}}\cdot d\left( B',\left( SCD \right) \right)\\
			&=& \dfrac{1}{3}SA\cdot{S_{ABCD}}+\dfrac{1}{3}{S_{SCD}}\cdot d\left( B,\left( SCD \right) \right).
		\end{eqnarray*}
		(vì $B'$ là điểm đối xứng của $B$ qua mặt phẳng $\left( SCD \right)$).
		\begin{center}
			\begin{tikzpicture}[scale=1,font=\footnotesize,line join=round,line cap=round,>=stealth]
				\def\a{4}
				\def\h{3.5}
				\path 	(0:0) coordinate (A)
				++(0:\a) coordinate (D)
				(-130:\a/2) coordinate (B)
				($ (B)+(0:\a/2) $) coordinate (C)
				($(A)+(90:\h)$) coordinate (S)
				($ (C)!0.5!(D) $) coordinate (C')
				($ (S)!2/3!(C') $) coordinate (B')
				($ (C)!0.5!(S) $) coordinate (H)
				(intersection of A--B and D--C) coordinate (M)
				(intersection of A--C and B--D) coordinate (O);%giao điểm O
				\draw[dashed,thick] 	(B)--(A)--(D)	(H)--(A)--(S) (A)--(C)--(S) ;
				\draw[thick] 			(B)-- (C)--(D) (B)--(M)--(C)
				(B)--(S)--(B')--(C)		(B')--(D)--(S) (B)--(B');
				\foreach \x/\g in {A/135,B/-180,C/-45,D/45,S/90,B'/45,M/145,H/45}
				\fill[black] 	(\x) circle (1.5pt)
				($(\g:3mm)+(\x)$) node {$\x$};
				\draw pic[draw,angle radius=2mm]{right angle=D--A--S};%Theo chiều dương
				\draw pic[draw,angle radius=3mm]{right angle=A--H--C};%Theo chiều dương
			\end{tikzpicture}
		\end{center}
		+ ${V_{S.ABCD}}=\dfrac{1}{3}SA.{S_{ABCD}}=\dfrac{1}{3}.a\sqrt{2}.\dfrac{\left( a+2a \right)a}{2}=\dfrac{a^3}{\sqrt{2}}$\\
		+ Gọi $M$ là giao điểm của $AB$ và $CD$, dễ dàng chứng minh được $B$ là trung điểm của $MA$.\\
		$\Rightarrow d\left( B,\left( SCD \right) \right)=\dfrac{1}{2}d\left( A,\left( SCD \right) \right)=\dfrac{1}{2}AH$.\\
		Lại có tam giác $SAC$ vuông cân tại $A$ ( vì $SA=AC=a\sqrt{2}$)\\
		$\Rightarrow d\left( B,\left( SCD \right) \right)=\dfrac{1}{2}d\left( A,\left( SCD \right) \right)=\dfrac{1}{2}AH=\dfrac{1}{4}SC=\dfrac{1}{4}\cdot 2a=\dfrac{a}{2}$.\\
		Ta có $ {V_{B'.SCD}}=\dfrac{1}{3}{S_{SCD}}\cdot d\left( B',\left( SCD \right) \right)=\dfrac{1}{3}\cdot\dfrac{1}{2}SC\cdot CD\dfrac{a}{2}=\dfrac{1}{6}\cdot 2a\cdot a\sqrt{2}\cdot\dfrac{a}{2}=\dfrac{a^3\sqrt{2}}{6}=\dfrac{a^3}{3\sqrt{2}}$\\
		${V_{SB'.ABCD}}={V_{S.ABCD}}+{V_{B'SCD}}=\dfrac{a^3}{\sqrt{2}}+\dfrac{a^3}{3\sqrt{2}}=\dfrac{2\sqrt{2}{a^3}}{3}$.}
\end{ex}
\begin{ex}%[2H1K3-2]
	[Chuyên Hà Tĩnh - 2021]
	Cho hình chóp $S.ABC$ có $SA$ vuông góc với mặt đáy, $SA=2BC=2a\sqrt{3}$, $AC=a$ và $\widehat{BAC}={{120}^{\text{o}}}$. Hình chiếu vuông góc của $A$ lên các cạnh $SB$ và $SC$ lần lượt là $M$ và $N$. Thể tích của khối đa diện $AMNCB$ bằng.
	\choice
	{ $\dfrac{24}{169}{a^3}$}
	{\True $\dfrac{25}{338}{a^3}$}
	{ $\dfrac{25}{169}{a^3}$}
	{ $\dfrac{12}{169}{a^3}$}
	\loigiai{
		\begin{center}
			\begin{tikzpicture}[scale=1,font=\footnotesize,line join=round,line cap=round,>=stealth]
				\def\a{4} %Khai báo cạnh
				\def\h{4}
				\path 	(0:0) coordinate (A)
				++(0:\a) coordinate (B)
				++(-140:4*\a/5) coordinate (C)
				($(A)+(90:\h)$) coordinate (S)
				($(S)!(A)!(B)$) coordinate (M)
				($(S)!(A)!(C)$) coordinate (N);;
				\draw[thick] 	(A)--(C)--(B)
				(M)--(N)--(A)--(S)	(B)--(S)	(C)--(S);
				\draw[dashed,thick] 	(M)--(A)--(B);
				\foreach \x /\goc in {A/180,B/0,C/-135,S/90,M/45,N/0}
				\fill[black] (\x) circle (1.5pt)
				($(\x)+(\goc:3mm)$) node {$\x$};
				\draw pic[draw,angle radius=2mm]{right angle=B--A--S};%Theo chiều dương
			\end{tikzpicture}
		\end{center}
		Có $SA=2BC=2a\sqrt{3}\Rightarrow SA=2a\sqrt{3};BC=a\sqrt{3}$.\\
		Xét $\triangle ABC$ có $\cos A=\dfrac{AB^2+AC^2-BC^2}{2AB\cdot AC}\Rightarrow -\dfrac{1}{2}=\dfrac{AB^2+a^2-3a^2}{2AB\cdot a}$.\\
		$\Rightarrow -a.AB=AB^2-2a^2\Leftrightarrow AB^2+a\cdot AB-2a^2=0\Leftrightarrow \left( AB-a \right)\left( AB+2a \right)=0$.\\
		$\Leftrightarrow \hoac{
			& AB=a \\
			& AB=-2a<0\left( KTM \right) \\
		}\Rightarrow AB=a$.\\
		Có ${V_{S.ABC}}=\dfrac{1}{3}SA\cdot{S_{ABC}}=\dfrac{1}{6}SA\cdot AB\cdot AC\sin \widehat{BAC}=\dfrac{a^3}{2}$.\\
		$\triangle SAB$ vuông tại $A$, đường cao $AM$.\\
		$\Rightarrow SA^2=SM.SB\Rightarrow SM=\dfrac{SA^2}{SB}=\dfrac{SA^2}{\sqrt{SA^2+A{B^2}}}=\dfrac{12a^2}{a\sqrt{13}}=\dfrac{12a}{\sqrt{13}}$.\\
		Tương tự, $SN=\dfrac{SA^2}{\sqrt{SA^2+A{C^2}}}=\dfrac{12a}{\sqrt{13}}$.\\
		Và $ SB=\sqrt{AB^2+AS^2}=\sqrt{a^2+(2a\sqrt{3})^2}=a\sqrt{13} $, $ SC=\sqrt{AC^2+AS^2}=\sqrt{a^2+(2a\sqrt{3})^2}=a\sqrt{13} $.\\
		Có $\dfrac{V_{SAMN}}{V_{SABC}}=\dfrac{SM}{SB}\cdot\dfrac{SN}{SC}=\dfrac{\dfrac{12a}{\sqrt{13}}}{a\sqrt{13}}\cdot\dfrac{\dfrac{12a}{\sqrt{13}}}{a\sqrt{13}}=\dfrac{144}{169}\Rightarrow \dfrac{V_{AMNCB}}{V_{SABC}}=\dfrac{25}{169}$.\\
		$\Rightarrow {V_{AMNCB}}=\dfrac{25}{169}{V_{SABC}}=\dfrac{25}{169}\cdot\dfrac{a^3}{2}=\dfrac{25a^3}{338}$.}
\end{ex}
\begin{ex}%[2H1K3-2]
	[Chuyên Long An - 2021]
	Cho hình chóp $S.ABCD$ có đáy là hình vuông cạnh $a$, $SA$ vuông góc với mặt đáy $\left( ABCD \right)$ và $SA=a$. Điểm $M$ thuộc cạnh $SA$ sao cho $\dfrac{SM}{SA}=k$, $0<k<1$. Tìm giá trị của $k$ để mặt phẳng $\left( BMC \right)$ chia khối chóp $S.ABCD$ thành hai phần có thể tích bằng nhau.
	\choice
	{ $k=\dfrac{-1+\sqrt{2}}{2}$}
	{ $k=\dfrac{1+\sqrt{5}}{4}$}
	{\True $k=\dfrac{-1+\sqrt{5}}{2}$}
	{ $k=\dfrac{-1+\sqrt{5}}{4}$}
	\loigiai{
		\begin{center}
			\begin{tikzpicture}[scale=1,font=\footnotesize,line join=round,line cap=round,>=stealth]
				\def\a{4}
				\def\h{4}
				\path 	(0:0) coordinate (A)
				++(0:\a) coordinate (D)
				++(-140:\a/1.5) coordinate (C)
				($(A)+(C)-(D)$) coordinate (B)
				($(A)+(90:\h)$) coordinate (S)
				($ (A)!0.6!(S) $) coordinate (M)
				($ (M)+(A)!0.4!(D) $) coordinate (N)
				(intersection of A--C and B--D) coordinate (O)
				(intersection of S--O and B--N) coordinate (I);
				\draw[dashed,thick] 	(N)--(B)--(A)--(D)	(M)--(C)--(A)--(S)--(O) (N)--(M)--(B)--(D);
				\draw[thick] 			(B)-- (C)--(D)
				(B)--(S)	(N)--(C)--(S)	(D)--(S);
				\foreach \x/\g in {A/180,B/-135,C/-45,D/45,S/90,M/135,N/45,I/180,O/-90}
				\fill[black] 	(\x) circle (1.5pt)
				($(\g:3mm)+(\x)$) node {$\x$};
				\draw pic[draw,angle radius=3mm]{right angle=D--A--S};%Theo chiều dương
			\end{tikzpicture}
		\end{center}
		Ta gọi $N$ là giao điểm của $SD$ và mặt phẳng $\left( BMC \right)$.\\
		Ta có\\
		$\heva{
			& \left( BMC \right)\cap \left( SAD \right)=MN \\
			& BC\subset \left( BMC \right) \\
			& AD\subset \left( SAD \right) \\
			& AD//BC \\
		}\Rightarrow (MNAD)\Rightarrow \dfrac{SM}{SA}=\dfrac{SN}{SD}=k$.\\
		Mặt khác: $\dfrac{{V_{S.BCM}}}{{V_{S.BCA}}}=\dfrac{SB}{SB}\cdot\dfrac{SC}{SC}\cdot\dfrac{SM}{SA}=k\Rightarrow {V_{S.BCM}}=k\cdot{V_{S.BCA}}={V_{S.BCM}}=\dfrac{k}{2}{V_{S.ABCD}}$.\\
		Lại có: $\dfrac{{V_{S.MCN}}}{{V_{S.ACD}}}=\dfrac{SM}{SA}\cdot\dfrac{SC}{SC}\cdot\dfrac{SN}{SD}=k^2\Rightarrow {V_{S.MCN}}=k^2\cdot{V_{S.ACD}}\Rightarrow {V_{S.MCN}}=\dfrac{k^2}{2}{V_{S.ABCD}}.$\\
		Do đó, ${V_{S.BCNM}}={V_{S.BCM}}+{V_{S.MCN}}=\left( \dfrac{k+k^2}{2} \right){V_{S.ABCD}}.$\\
		Theo đề bài, $\left( BMC \right)$ chia đôi khối chóp $S.ABCD$ thành hai phần có thể tích bằng nhau nên:\\
		${V_{S.BCNM}}=\dfrac{1}{2}{V_{S.ABCD}}\Rightarrow \dfrac{k^2+k}{2}=\dfrac{1}{2}\Leftrightarrow {k^2}+k-1=0\Leftrightarrow \hoac{
			& k=\dfrac{-1+\sqrt{5}}{2} \\
			& k=\dfrac{-1-\sqrt{5}}{2} \\
		}$.\\
		Mà $0<k<1$ nên $k=\dfrac{-1+\sqrt{5}}{2}$ thoả mãn yêu cầu.}
\end{ex}
%Câu 66
\begin{ex}%[2H1G3-5]
	[Chuyên Lương Văn Tụy-Ninh Bình 2022]
	Bạn $A$  định làm một cái hộp quà lưu niệm (không nắp) bằng cách cắt từ một tấm bìa hình tròn bán kính  $4$ cm để tạo thành một khối lăng trụ lục giác đều, biết $6$ hình chữ nhật có các kích thước là $1$ cm  và  $x$ cm (tham khảo hình vẽ). Thể tích của hộp quà gần nhất với giá trị nào sau đây?
	\begin{center}
		\begin{tikzpicture}[declare function={r=3;}]
			\path
			(1,0) coordinate (A')
			(1,3) coordinate (B');
			\path ($(B')!1!120:(A')$) coordinate (C')
			($(C')!1!120:(B')$) coordinate (D')
			($(D')!1!120:(C')$) coordinate (E')
			($(E')!1!120:(D')$) coordinate (F');
			
			\path
			(A')--(B')--([turn]90:1)coordinate (A)
			($(A')+(A)-(B')$)coordinate (D)
			(E')--(D')--([turn]-90:1)coordinate (B)
			($(E')+(B)-(D')$)coordinate (C)
			(A')--(F')--([turn]-90:1)coordinate (x)
			($(A')+(x)-(F')$)coordinate (x')
			(F')--(E')--([turn]-90:1)coordinate (y)
			($(F')+(y)-(E')$)coordinate (y')
			(C')--(D')--([turn]90:1)coordinate (z)
			($(C')+(z)-(D')$)coordinate (z')
			(B')--(C')--([turn]90:1)coordinate (t)
			($(B')+(t)-(C')$)coordinate (t')
			;
			\path (D')--(E')--([turn]-90:0.35) coordinate (xt)
			($(D')-(E')+(xt)$) coordinate (yt);
			\draw[dash pattern=on 2pt off 2pt] (D')--(yt) (E')--(xt);
			\draw[>=stealth,|<->|] (xt)--(yt) node[fill=white,inner sep=0pt,font=\scriptsize,midway,sloped]{$x$};
			\path (E')--(C)--([turn]-90:0.35) coordinate (xt)
			($(E')-(C)+(xt)$) coordinate (yt);
			\draw[dash pattern=on 2pt off 2pt] (E')--(yt) (C)--(xt);
			\draw[>=stealth,|<->|] (xt)--(yt) node[fill=white,inner sep=0pt,font=\scriptsize,midway,sloped]{$1$};
			\draw (A')--(B')--(C')--(D')--(E')--(F')--cycle (B')--(A)--(D)--(A')--(x')--(x)--(F')--(y')--(y)--(E')--(C)--(B)--(D')--(z)--(z')--(C')--(t)--(t')--(B')
			;
			\begin{scope}[overlay]
				\path ($ (A)!0.5!(B) $) coordinate (Nt)
				($ (Nt)!1!90:(A) $) coordinate (Xt)
				($ (B)!0.5!(C) $) coordinate (Mt)
				($ (Mt)!1!90:(B) $) coordinate (Yt)
				(intersection of Nt--Xt and Mt--Yt) coordinate (O);
			\end{scope}
			\draw let \p1=($  (O) -  (A) $) in  (O)  circle ({veclen(\x1,\y1)});
			\foreach \t/\g in {O/0}{
				\draw[fill=white] (\t) circle (1pt) node[shift={(\g:7pt)},font=\scriptsize]{$ \t $};
			}
		\end{tikzpicture}
	\end{center}
	
	\choice
	{$24{,}5$ cm$ ^3$}
	{\True $25$ cm$ ^3$}
	{$25{,}5$ cm$ ^3$}
	{$24$ cm$ ^3$}
	\loigiai{
		\begin{center}
			\begin{tikzpicture}[declare function={r=3;}]
				\path
				(1,0) coordinate (A')
				(1,3) coordinate (B');
				\path ($(B')!1!120:(A')$) coordinate (C')
				($(C')!1!120:(B')$) coordinate (D')
				($(D')!1!120:(C')$) coordinate (E')
				($(E')!1!120:(D')$) coordinate (F');
				
				\path
				(A')--(B')--([turn]90:1)coordinate (A)
				($(A')+(A)-(B')$)coordinate (D)
				(E')--(D')--([turn]-90:1)coordinate (B)
				($(E')+(B)-(D')$)coordinate (C)
				(A')--(F')--([turn]-90:1)coordinate (x)
				($(A')+(x)-(F')$)coordinate (x')
				(F')--(E')--([turn]-90:1)coordinate (y)
				($(F')+(y)-(E')$)coordinate (y')
				(C')--(D')--([turn]90:1)coordinate (z)
				($(C')+(z)-(D')$)coordinate (z')
				(B')--(C')--([turn]90:1)coordinate (t)
				($(B')+(t)-(C')$)coordinate (t')
				;
				\path (D)--(A')--([turn]90:0.5) coordinate (xt)
				($(D)-(A')+(xt)$) coordinate (yt);
				\draw[dash pattern=on 2pt off 2pt] (D)--(yt) (A')--(xt);
				\draw[>=stealth,|<->|] (xt)--(yt) node[fill=white,inner sep=0pt,font=\scriptsize,midway,sloped]{$1$};
				\path (A')--(E')--([turn]90:0.5) coordinate (xt)
				($(A')-(E')+(xt)$) coordinate (yt);
				\draw[dash pattern=on 2pt off 2pt] (A')--(yt) (E')--(xt);
				\draw[>=stealth,|<->|] (xt)--(yt) node[fill=white,inner sep=0pt,font=\scriptsize,midway,sloped]{$x\sqrt{3}$};
				\path (E')--(C)--([turn]90:0.5) coordinate (xt)
				($(E')-(C)+(xt)$) coordinate (yt);
				\draw[dash pattern=on 2pt off 2pt] (E')--(yt) (C)--(xt);
				\draw[>=stealth,|<->|] (xt)--(yt) node[fill=white,inner sep=0pt,font=\scriptsize,midway,sloped]{$1$};
				\path (C)--(B)--([turn]-90:0.5) coordinate (xt)
				($(C)-(B)+(xt)$) coordinate (yt);
				\path pic["\scriptsize$120^\circ$", angle eccentricity=2,draw,angle radius=7pt, double]{angle= E'--F'--A'};
				\draw[dash pattern=on 2pt off 2pt] (C)--(yt) (B)--(xt);
				\draw[>=stealth,|<->|] (xt)--(yt) node[fill=white,inner sep=0pt,font=\scriptsize,midway,sloped]{$x$};
				\path (A')--(F')--([turn]-90:0.35) coordinate (xt)
				($(A')-(F')+(xt)$) coordinate (yt);
				\draw[dash pattern=on 2pt off 2pt] (A')--(yt) (F')--(xt);
				\draw[>=stealth,|<->|] (xt)--(yt) node[fill=white,inner sep=0pt,font=\scriptsize,midway,sloped]{$x$};
				\path (F')--(E')--([turn]-90:0.35) coordinate (xt)
				($(F')-(E')+(xt)$) coordinate (yt);
				\draw[dash pattern=on 2pt off 2pt] (F')--(yt) (E')--(xt);
				\draw[>=stealth,|<->|] (xt)--(yt) node[fill=white,inner sep=0pt,font=\scriptsize,midway,sloped]{$x$};
				\draw (A')--(B')--(C')--(D')--(E')--(F')--cycle (B')--(A)--(D)--(A')--(x')--(x)--(F')--(y')--(y)--(E')--(C)--(B)--(D')--(z)--(z')--(C')--(t)--(t')--(B') (A)--(C) (B)--(D) (E')--(A')
				;
				\begin{scope}[overlay]
					\path ($ (A)!0.5!(B) $) coordinate (Nt)
					($ (Nt)!1!90:(A) $) coordinate (Xt)
					($ (B)!0.5!(C) $) coordinate (Mt)
					($ (Mt)!1!90:(B) $) coordinate (Yt)
					(intersection of Nt--Xt and Mt--Yt) coordinate (O);
				\end{scope}
				\draw let \p1=($  (O) -  (A) $) in  (O)  circle ({veclen(\x1,\y1)});
				\foreach \t/\g in {O/-90,D/180,A/180,B/60,C/-60}{
					\draw[fill=white] (\t) circle (1pt) node[shift={(\g:7pt)},font=\scriptsize]{$ \t $};
				}
			\end{tikzpicture}
		\end{center}
		
		Xét hình chữ nhật $ABCD$ nội tiếp $(O)$, do đó, $AC$ là đường kính của $(O)$. Ta có $A C=8~cm$.\\
		Tính được $DC=1+x \sqrt{3}+1=x \sqrt{3}+2$.\\
		Áp dụng định lý Py-ta-go vào tam giác $ADC$ ta có\\
		$x^2+\left(2+x\sqrt{3}\right)^2=8^2\Leftrightarrow 4x^2+4x\sqrt{3}-60=0\Leftrightarrow x=\dfrac{3\sqrt{7}-\sqrt{3}}{2}$. \\
		$ V=h\cdot{S_d}=1\cdot 6\cdot \dfrac{x^2\sqrt{3}}{4}=\dfrac{3}{2}x^2\sqrt{3}=\dfrac{-27\sqrt{7}+99\sqrt{3}}{4}\approx 25{,}0094$ cm$^3$.
	}
\end{ex}
%Câu 67
\begin{ex}%[2H1G3-5]
	[THPT Kinh Môn - Hải Dương - 2022]
	Người ta dùng thuỷ tinh trong suốt để làm một cái chặn giấy hình tứ diện đều. Để trang trí cho nó, người thiết kế đặt trong khối tứ diện $4$ quả cầu nhựa màu xanh có bán kính bằng nhau là  $r=\sqrt{2}$. Biết rằng $4$ quả cầu này đôi một tiếp xúc với nhau và mỗi mặt của tứ diện tiếp xúc với $3$ quả cầu, đồng thời không cắt quả cầu còn lại. Nếu bỏ qua bề dày của các mặt thì người ta cần dùng bao nhiêu thủy tinh để làm chặn giấy trên (làm tròn đến chữ số thập phân thứ $2$).
	\choice
	{$195{,}66$ cm$ ^3$}
	{\True $62{,}09$ cm$ ^3$}
	{$30{,}03$ cm$ ^3$}
	{$65{,}55$ cm$ ^3$}
	\loigiai{
		\begin{center}
			\begin{tikzpicture}[declare function={r=5;}]
				\path (160:{r} and {r*0.35}) coordinate(B)
				(260:{r} and {r*0.5}) coordinate (C)
				(20:{r} and {r*0.35})coordinate (D)
				(90:{r*1.25}) coordinate (A)
				($(B)!.5!(C)$) coordinate (x)
				($(C)!.5!(D)$) coordinate (y)
				($(D)!.5!(B)$) coordinate (z)
				($(B)!2/3!(y)$) coordinate (G)
				($(A)!3/4!(G)$) coordinate (O)
				($(A)!.5!(O)$) coordinate (I_1)
				($(G)+(-1,0) $) coordinate (I_3)
				($(O)!.5!(G)$) coordinate (G')
				($(B)!.5!(G)$) coordinate (x')
				($(D)!.5!(G)$) coordinate (y')
				($(x')+(0,.7) $) coordinate (I_2)
				($(y')+(0,.7) $) coordinate (I_4)
				($(I_3)!.5!(I_4)$) coordinate (t)
				;
				\draw[dash pattern=on 2pt off 2 pt] (B)--(D) (B)--(y) (C)--(z) (D)--(x) (A)--(G) (I_1)--(I_2)--(I_3)--(I_4)--cycle (I_2)--(I_4) (I_2)--(x') (I_4)--(y') (I_2)--(t)
				;
				\draw (B)--(C)--(D)--(A)--cycle (A)--(C);
				\foreach \t/\g in {B/180,D/0,C/-90,A/90,G/-60,O/0,I_1/0,I_3/180,G'/230,I_2/180,I_4/0}{
					\draw[fill=white] (\t) circle (1pt) node[shift={(\g:7pt)},font=\scriptsize]{$ \t $};
				}
			\end{tikzpicture}
		\end{center}
		\begin{itemize}
			\item Goi $I_1$, $I_2$, $I_3$, $I_4$ lần lượt là tâm của $4$ hình cầu đã cho. Khi đó $I_1I_2I_3I_4$ là một tứ diện đều có cạnh bằng $2r=2\sqrt{2}\,(\mathrm{cm})$ và chiều cao $h=\dfrac{2r\sqrt{6}}{3}=\dfrac{4\sqrt{3}}{3}\,(\mathrm{cm})$. Do đó hai tứ diện $ABCD$ và $I_1I_2I_3I_4$ đồng dạng với nhau theo tỉ số $k$.
			\item Gọi $O$ là trọng tâm của tứ diện $ABCD$ thì $O$ cũng là trọng tâm của tứ diện $I_1I_2I_3I_4$. Do đó $O$ là tâm đồng dạng. Giả sử phép đồng dạng tâm $O$ lần lượt biến các đỉnh $A$, $B$, $C$, $D$ thành $I_1$, $I_2$, $I_3$, $I_4$. Khi đó, hai mặt phẳng $(BCD)$ và $(I_2I_3I_4)$ song song nhau.
			\item Gọi $G$, $G'$ lần lượt là trọng tâm các tam giác $BCD$ và $I_2I_3I_4$. Khi đó\\
			$OG'=\dfrac{1}{4}h=\dfrac{\sqrt{3}}{3}$ và $OG=OG'+r=\dfrac{\sqrt{3}}{3}+\sqrt{2}$. Suy ra $k=\dfrac{OG}{OG'}=1+\sqrt{6}$.
			\item Khi đó, thể tích của tứ diện $ABCD$ là\\
			$V_{ABCD}=k^3\cdot{V_{I_1I_2I_3I_4}}=(1+\sqrt{6})^3\cdot \dfrac{1}{3}\cdot \dfrac{(2r)^2\sqrt{3}}{4}\cdot h=\dfrac{8(19+9\sqrt{6})}{3}\,(\mathrm{cm^3})$.
			\item Thể tích của mỗi khối cầu là\\
			$V_c=\dfrac{4}{3}\pi\cdot r^3=\dfrac{8\pi \sqrt{2}}{3}\,(\mathrm{cm^3})$.
			\item Vậy lượng thuỷ tinh cần dùng là\\
			$V=V_{ABCD}-4\cdot{V_c}=\dfrac{8(19+9\sqrt{6})}{3}-4\cdot \dfrac{8\pi \sqrt{2}}{3}\approx 62{,}06\,(\mathrm{cm^3})$.
		\end{itemize}
	}
\end{ex}
%Câu 68
\begin{ex}%[2H1G3-2]
	[Sở Hải Dương 2022]
	\immini{
		Cho hình lập phương $ABCD.A'B'C'D'$  cạnh bằng $2$. Thể tích $V$  của khối bát diện đều có các đỉnh nằm trên các cạnh $BC$, $A'D'$, $A'B'$, $AA'$, $CD$, $CC'$  (như hình vẽ) bằng
		\choice
		{\True $\dfrac{9}{2}$}
		{$\dfrac{6\sqrt{2}}{3}$}
		{$\dfrac{9\sqrt{3}}{2}$}
		{$3$}
	}
	{
		\begin{tikzpicture}[declare function={r=3;}]
			\path (0:0) coordinate (B)
			(0:r) coordinate (C)
			++(37:{0.65*r}) coordinate (D)
			++(180:r) coordinate (A)
			\foreach \x in {A,B,C,D}{(\x)++(90:r) coordinate (\x')}
			($(B)!0.3!(C)$) coordinate (S)
			($(D)!0.3!(C)$) coordinate (P)
			($(B')!0.3!(A')$) coordinate (N)
			($(D')!0.3!(A')$) coordinate (M)
			($(A)!0.3!(A')$) coordinate (R)
			($(C')!0.3!(C)$) coordinate (Q)
			;
			\draw[dash pattern=on 2pt off 2 pt] (A')--(A)--(B) (A)--(D) (N)--(S)--(P)--(M) (S)--(R)--(P) (N)--(R)--(M) (N)--(Q)--(M)
			;
			\draw (B)--(C)--(D) (B)--(B') (C)--(C') (D)--(D') (A')--(B')--(C')--(D')--cycle (N)--(M) (S)--(Q)--(P)
			;
			\foreach \t/\g in {A/-90,B/180,C/0,D/0,A'/180,B'/180,C'/220,D'/0}{
				\draw[fill=white] (\t) circle (1pt) node[shift={(\g:7pt)},font=\scriptsize]{$ \t $};
			}
		\end{tikzpicture}
	}
	\loigiai{
		\begin{center}
			\begin{tikzpicture}[declare function={r=3;}]
				\path (0:0) coordinate (B)
				(0:r) coordinate (C)
				++(37:{0.65*r}) coordinate (D)
				++(180:r) coordinate (A)
				\foreach \x in {A,B,C,D}{(\x)++(90:r) coordinate (\x')}
				($(B)!0.3!(C)$) coordinate (S)
				($(D)!0.3!(C)$) coordinate (P)
				($(B')!0.3!(A')$) coordinate (N)
				($(D')!0.3!(A')$) coordinate (M)
				($(A)!0.3!(A')$) coordinate (R)
				($(C')!0.3!(C)$) coordinate (Q)
				;
				\draw[dash pattern=on 2pt off 2 pt] (A')--(A)--(B) (A)--(D) (N)--(S)--(P)--(M) (S)--(R)--(P) (N)--(R)--(M) (N)--(Q)--(M)
				;
				\draw (B)--(C)--(D) (B)--(B') (C)--(C') (D)--(D') (A')--(B')--(C')--(D')--cycle (N)--(M) (S)--(Q)--(P)
				;
				\foreach \t/\g in {A/-90,B/180,C/0,D/0,A'/180,B'/180,C'/220,D'/0,S/-90,P/0,N/180,M/90,R/180,Q/0}{
					\draw[fill=white] (\t) circle (1pt) node[shift={(\g:7pt)},font=\scriptsize]{$ \t $};
				}
			\end{tikzpicture}
		\end{center}
		Do các mặt của bát diện đều là $1$ tam giác đều nên chắn các góc đỉnh $C$ và đỉnh $A'$ những đoạn bằng nhau bằng $x$, đoạn còn lại bằng $2-x$.\\
		Đặt $A'M=x\,(0<x<2)$. Gọi $M$, $N$, $P$, $Q$, $R$, $S$ lần lượt là các đỉnh của bát diện nằm trên các cạnh $A'D'$, $A'B'$, $CD$, $CC'$, $A'A$, $BC$.\\
		Ta có $MN=x\sqrt{2}$, $MQ=\sqrt{2(2-x)^2+4}$. \\
		Do $MN=MQ\Leftrightarrow 2x^2=2(2-x)^2+4\Leftrightarrow 4x=6\Leftrightarrow x=\dfrac{3}{2}$.\\
		Ta có \\
		$V_{MNPQRS}=2V_{MNPQR}=\dfrac{2}{3}\cdot \mathrm{d}\left(M, (NPQR)\right)\cdot{(x\sqrt{2})^2}=\dfrac{2}{3}\cdot\dfrac{x\sqrt{2}}{2}\cdot \sqrt{2}\cdot 2x^2=\dfrac{4}{3}x^3=\dfrac{4}{3}\left(\dfrac{3}{2}\right)^3=\dfrac{9}{2}$.
	}
\end{ex}
%Câu 69
\begin{ex}%[2H1G3-2]
	[Sở Sơn La 2022]
	Cho hình chóp $S.ABCD$  có đáy $ABCD$  là hình chữ nhật $AB=2$, $AD=4$, $SA$  vuông góc với mặt đáy, $SB$  tạo với đáy góc  $60^\circ$, điểm $E$  thuộc cạnh  $SA$ và  $AE=\dfrac{2\sqrt{3}}{3}$. Mặt phẳng $(BCE)$  cắt  $SD$ tại  $F$. Thể tích khối đa diện $ABCDEF$  bằng
	\choice
	{$\dfrac{64\sqrt{3}}{9}$}
	{\True $\dfrac{64\sqrt{3}}{27}$}
	{$\dfrac{80\sqrt{3}}{27}$}
	{$\dfrac{16\sqrt{3}}{3}$}
	\loigiai{
		\immini{
			Xét $(BEC)$ và $(SAD)$ có điểm $E$ chung và $BC$ song song $AD$ nên giao tuyến là đường thẳng qua $E$ và song song $AD$ cắt $SD$ tại $F$.\\
			Góc giữa $SB$ với đáy bằng $60^\circ \Rightarrow \widehat{SBA}=60^\circ\\
			\Rightarrow S A=AB \cdot \tan 60^\circ=2 \sqrt{3}$.\\
			Mặt khác $AE=\dfrac{2 \sqrt{3}}{3}$ nên $AE=\dfrac{1}{3} SA\\
			\Rightarrow  SE=\dfrac{2}{3} SA$.\\
			Xét $\triangle SAD$ ta có  $\dfrac{SE}{SA}=\dfrac{SF}{SD}=\dfrac{2}{3}$.
		}
		{
			\begin{tikzpicture}[declare function={a=2.65;b=4.5;h=3;},line join=round]
				\path (0,0) coordinate (A)
				(-135:a) coordinate (B)
				(b,0) coordinate (D)
				(0,h) coordinate (S)
				($(D)-(A)+(B)$) coordinate (C)
				($(S)!2/3!(A) $)  coordinate (E)
				($(S)!2/3!(D) $)  coordinate (F)
				;
				\draw[dashed] (S)--(A)--(B) (A)--(D) (B)--(E)--(F);
				\draw (B)--(C)--(D) (B)--(S)  (D)--(S)--(C)--(F);
				\foreach \t/\g in {A/40,B/-90,C/-90,D/0,S/90,E/180,F/30}{
					\draw[fill=white] (\t) circle (1pt) node[shift={(\g:7pt)},font=\scriptsize]{$ \t $};
				}
			\end{tikzpicture}
		}
		\noindent
		Ta có $\dfrac{V_{SBEC}}{V_{SBAC}}=\dfrac{SE}{SA}=\dfrac{2}{3} \Rightarrow V_{SBE C}=\dfrac{2}{3} V_{SBAC} \Rightarrow V_{SBEC}=\dfrac{1}{3} V_{SABCD}$.\\
		$\dfrac{V_{SEFC}}{V_{SADC}}=\dfrac{SE}{SA}\cdot \dfrac{SF}{SD}=\dfrac{2}{3}\cdot \dfrac{2}{3}=\dfrac{4}{9}\Rightarrow {V_{SEFC}}=\dfrac{4}{9}V_{SADC}\Rightarrow {V_{SEFC}}=\dfrac{2}{9}V_{SABCD}$.\\
		Khi đó $V_{SBCFE}=V_{SBEC}+V_{SEFC}=\dfrac{1}{3} V_{SABCD}+\dfrac{2}{9} V_{S ABCD}=\dfrac{5}{9} V_{SABCD}$.\\
		Suy ra $V_{ABCDFE}=\dfrac{4}{9}V_{SABCD}=\dfrac{4}{9}\cdot \dfrac{1}{3}\cdot SA\cdot{S_{ABCD}}=\dfrac{4}{9}\cdot \dfrac{1}{3}\cdot 2\sqrt{3}\cdot 2\cdot 4=\dfrac{64\sqrt{3}}{27}$.
	}
\end{ex}
%Câu 70
\begin{ex}%[2H1G3-3]
	[Chuyên Thái Bình 2022]
	Cho khối lăng trụ tam giác $ABC.A'B'C'$  có thể tích  $V$. Gọi $M$, $N$, $P$  lần lượt là trung điểm của các cạnh  $A'B'$, $BC$, $CC'$. Mặt phẳng $MNP$  chia khối lăng trụ đã cho thành $2$  phần, phần chứa điểm $B$  có thể tích là  $V_1$. Tỉ số $\dfrac{V_1}{V}$  bằng
	\choice
	{$\dfrac{61}{144}$}
	{$\dfrac{37}{144}$}
	{\True $\dfrac{49}{144}$}
	{$\dfrac{25}{144}$}
	\loigiai{
		\begin{center}
			\begin{tikzpicture}[line cap=round,line join=round,>=stealth,scale=0.7]
				\path
				(0,0) coordinate (A)
				(1,-2) coordinate (B)
				(5,0) coordinate (C)
				($(B)!.5!(C)$) coordinate (H)
				($(H)+(0,5)$) coordinate (A')
				($(B)+(A')-(A) $) coordinate (B')
				($(C)+(B')-(B) $) coordinate (C')
				($(A')!.5!(B')$) coordinate (M)
				($(B)!.5!(C)$) coordinate (N)
				($(C)!.5!(C')$) coordinate (P)
				(intersection of N--P and B--B') coordinate (E)
				(intersection of N--P and C'--B') coordinate (F)
				(intersection of M--F and A'--C') coordinate (Q)
				(intersection of M--E and B--A) coordinate (R)
				;
				\draw  (B)--(A)--(A') (A')--(C')--(B')--cycle (B)--(C) (B)--(B') (C)--(C') (F)--(M)--(E)--(B) (C')--(F)--(E);
				\draw[dashed]  (A)--(C) (Q)--(P)--(M)--(N)--(R);
				\foreach \t/\g in {B/-90,A/180,C/0,B'/-20,A'/180,C'/0,M/180,N/-30,P/0,E/-90,F/0,Q/90,R/180}{
					\draw[fill=white] (\t) circle (1pt) node[shift={(\g:7pt)},font=\scriptsize]{$ \t $};
				}
			\end{tikzpicture}
		\end{center}
		Gọi $S$ và $h$ lần lượt là diện tích đáy và chiều cao của lăng trụ $ABC.A'B'C'\Rightarrow V=Sh$.\\
		Gọi $NP\cap BB'=E$, $NP\cap B'C'=F$, $MF\cap A'C'=Q$, $ME\cap AB=R$.\\
		Suy ra mặt phẳng $(MNP)$ cắt khối lăng trụ theo thiết diện là $MRNPQ$.\\
		Ta có $BEPC'$ là hình bình hành $\Rightarrow BE=PC'=\dfrac{1}{2}CC'=\dfrac{1}{2}BB'$, tương tự ta có $BNFC'$ là hình bình hành $\Rightarrow C'F=BN=\dfrac{1}{2}BC=\dfrac{1}{2}B'C'$.\\
		\begin{itemize}
			\item  $S_{MB'F}=\dfrac{1}{2}\cdot B'M\cdot B'F\cdot \sin \widehat{MB'F}=\dfrac{3}{4}\cdot \dfrac{1}{2}\cdot A'B'\cdot B'C'\cdot \sin \widehat{A'B'C'}=\dfrac{3}{4}S$.
			\item  $\mathrm{d}\left(E, (A'B'C')\right)=\dfrac{3}{2}\mathrm{d}\left(B, (A'B'C')\right)=\dfrac{3}{2}h$.
		\end{itemize}
		$\Rightarrow {V_{E.B'MF}}=\dfrac{1}{3}\cdot \mathrm{d}\left(E, (A'B'C')\right)\cdot{S_{B'MF}}=\dfrac{1}{3}\cdot \dfrac{3}{2}h\cdot \dfrac{3}{4}S=\dfrac{3}{8}V$.\\
		Lại có $\dfrac{V_{E.BNR}}{V_{E. B'FM}}=\left(\dfrac{EB}{EB'}\right)^3=\dfrac{1}{27}\Rightarrow {V_{E.BNR}}=\dfrac{1}{27}\cdot \dfrac{3}{8}V=\dfrac{1}{72}V$.\\
		Ta cũng có $\dfrac{V_{F.C'PQ}}{V_{F.B'EM}}=\dfrac{FC'}{FB'}\cdot \dfrac{FP}{FE}\cdot \dfrac{FQ}{FM}=\dfrac{1}{3}\cdot \dfrac{1}{3}\cdot \dfrac{1}{2}=\dfrac{1}{18}\Rightarrow {V_{F.C'PQ}}=\dfrac{1}{18}\cdot \dfrac{3}{8}V=\dfrac{1}{48}V$.\\
		Suy ra $V_1=V_{E.B'MF}-(V_{E.BNR}+V_{F.C'PQ})=\dfrac{49}{144}V$.\\
		Vậy $\dfrac{V_1}{V}=\dfrac{49}{144}$.
	}
\end{ex}
%Câu 71
\begin{ex}%[2H1G3-2]
	[THPT Trần Quốc Tuấn - Quảng Ngãi - 2022]
	Cho hình chóp $S.ABCD$ có đáy $ABCD$  là hình thang vuông tại  $A$ và $B$, $BA=BC=a$, $AD=2a$. Cạnh bên $SA$  vuông góc với mặt phẳng đáy và $SA=a\sqrt{2}$.  Gọi $H$  là hình chiếu vuông góc của điểm $A$  lên $SB$.  Tính thể tích $V$  của khối đa diện  	$SAHCD$.
	\choice
	{$V=\dfrac{4\sqrt{2}a^3}{3}$}
	{\True $V=\dfrac{4\sqrt{2}a^3}{9}$}
	{$V=\dfrac{2\sqrt{2}a^3}{3}$}
	{$V=\dfrac{2\sqrt{2}a^3}{9}$}
	\loigiai{
		\immini{
			Gọi $I$ là trung điểm của $AD\Rightarrow$ tứ giác $ABCI$ là hình vuông cạnh $a\Rightarrow CI=a=\dfrac{AD}{2}$.\\
			Do đó tam giác $ACD$ vuông tại $C$ (tính chất đường trung tuyến ứng với cạnh huyền).\\
			Ta có $S_{ABCD}=\dfrac{(BC+AD)\cdot AB}{2}=\dfrac{(a+2a)\cdot a}{2}=\dfrac{3a^2}{2};\\
			S_{\triangle ABC}=\dfrac{1}{2}AB\cdot BC=\dfrac{1}{2}\cdot a\cdot a=\dfrac{a^2}{2}$.\\
			Suy ra $S_{\triangle ACD}=\dfrac{3a^2}{2}-\dfrac{a^2}{2}=a^2$.\\
		}
		{
			\begin{tikzpicture}[declare function={a=2.65;b=4.5;h=3;},line join=round]
				\path (0,0) coordinate (A)
				(-135:a) coordinate (B)
				(b,0) coordinate (D)
				(0,h) coordinate (S)
				($(D)-(A)+(B)$) coordinate (x)
				($(B)!.5!(x) $)  coordinate (C)
				($(A)!.5!(D) $)  coordinate (I)
				($(S)!.3!(B) $)  coordinate (H)
				;
				\path pic[draw,angle radius=5pt]{right angle= S--H--A};
				\draw[dashed] (S)--(A)--(B) (A)--(D) (A)--(C)--(I) (A)--(H);
				\draw (B)--(C)--(D) (B)--(S)  (D)--(S)--(C);
				\foreach \t/\g in {A/180,B/-90,C/-90,D/0,S/90,I/90,H/180}{
					\draw[fill=white] (\t) circle (1pt) node[shift={(\g:7pt)},font=\scriptsize]{$ \t $};
				}
			\end{tikzpicture}
		}
		\noindent
		Khi đó $V_{S.ACD}=\dfrac{1}{3}\cdot{S_{\triangle ACD}}\cdot SA=\dfrac{1}{3}\cdot a^2\cdot a\sqrt{2}=\dfrac{a^3\sqrt{2}}{3}$ và $V_{S.ABC}=\dfrac{1}{3}\cdot S_{\triangle ABC}\cdot SA=\dfrac{1}{3}\cdot \dfrac{a^2}{2}\cdot a\sqrt{2}=\dfrac{a^3\sqrt{2}}{6}$.
		Xét $\triangle SAB$ vuông tại $A$ có $\heva{&SB=\sqrt{SA^2+AB^2}=\sqrt{(a\sqrt{2})^2+a^2}=a\sqrt{3} \\&AH=\dfrac{SA\cdot AB}{\sqrt{SA^2+AB^2}}=\dfrac{a\sqrt{2}\cdot a}{\sqrt{(a\sqrt{2})^2+a^2}}=\dfrac{a^2\sqrt{2}}{a\sqrt{3}}=\dfrac{a\sqrt{6}}{3}}$.\\
		Xét $\triangle SAH$ vuông tại $H$ có $SH=\sqrt{SA^2-AH^2}=\sqrt{(a\sqrt{2})^2-\left(\dfrac{a\sqrt{6}}{3}\right)^2}=\dfrac{2a\sqrt{3}}{3}\Rightarrow \dfrac{SH}{SB}=\dfrac{2}{3}$.\\
		Mặt khác, ta có $\dfrac{V_{S.AHC}}{V_{S.ABC}}=\dfrac{SH}{SB}\Rightarrow V_{S.AHC}=\dfrac{2}{3}V_{S.ABC}=\dfrac{2}{3}\cdot \dfrac{a^3\sqrt{2}}{6}=\dfrac{a^3\sqrt{2}}{9}$.\\
		Vậy $V_{S.AHCD}=V_{S.AHC}+V_{S.ACD}=\dfrac{a^3\sqrt{2}}{9}+\dfrac{a^3\sqrt{2}}{3}=\dfrac{4a^3\sqrt{2}}{9}$.
	}
\end{ex}
%Câu 72
\begin{ex}%[2H1G3-2]
	[THPT Nguyễn Cảnh Quân - Nghệ An 2022]
	Cho lăng trụ  $ABC.A'B'C'$ có chiều cao bằng $6$ và đáy là tam giác đều cạnh bằng $4$. Gọi $M$, $N$, $P$  lần lượt là tâm của các mặt bên  $ABB'A'$, $ACC'A'$, $BCC'B'$. Thể tích của khối đa diện lồi có các đỉnh là các điểm $A$, $B$, $C$, $M$, $N$, $P$ bằng
	\choice
	{\True $9\sqrt{3}$}
	{$10\sqrt{3}$}
	{$7\sqrt{3}$}
	{$12\sqrt{3}$}
	\loigiai{
		\begin{center}
			\begin{tikzpicture}[line cap=round,line join=round,>=stealth,scale=0.7]
				\path
				(0,0) coordinate (A')
				(1,-2) coordinate (C')
				(5,0) coordinate (B')
				($(B')!.5!(C')$) coordinate (H)
				($(H)+(0,8)$) coordinate (A)
				($(C')+(A)-(A') $) coordinate (C)
				($(B')+(C)-(C') $) coordinate (B)
				($(A)!.5!(A')$) coordinate (D)
				($(B)!.5!(B')$) coordinate (E)
				($(C)!.5!(C')$) coordinate (F)
				($(D)!.5!(E)$) coordinate (M)
				($(D)!.5!(F)$) coordinate (N)
				($(F)!.5!(E)$) coordinate (P)
				;
				\draw  (C')--(A')--(A) (A)--(B)--(C)--cycle (C')--(B') (C)--(C') (B)--(B') (D)--(F)--(E);
				\draw[dashed]  (A')--(B') (D)--(E) (N)--(M)--(B)--(P)--cycle (M)--(P)
				;
				\foreach \t/\g in {C'/-90,A'/180,B'/0,C/-20,A/180,B/0,M/90,N/180,P/-90,D/180,E/0,F/190}{
					\draw[fill=white] (\t) circle (1pt) node[shift={(\g:7pt)},font=\scriptsize]{$ \t $};
				}
			\end{tikzpicture}
		\end{center}
		Gọi $D$, $E$, $F$ lần lượt là trung điểm của các cạnh bên $AA'$, $BB'$, $CC'$.\\
		Khi đó $M$, $N$, $P$ lần lượt là trung điểm của các đoạn thẳng $DE$, $DF$, $FE$.\\
		Ta có $V_{MNP.ABC}=V_{DEF.ABC}-3V_{B.MEP}=\dfrac{1}{2}\cdot{V_{ABC.A'B'C'}}-3\cdot{V_{B.MEP}}$.\\
		Mà $\dfrac{V_{B.MEP}}{V_{ABC.A'B'C'}}=\dfrac{\dfrac{1}{3}\cdot \mathrm{d}\left(B,\left(MEP\right)\right)\cdot{S_{MEP}}}{\mathrm{d}\left(B,\left(A'B'C'\right)\right)\cdot{S_{ABC}}}=\dfrac{1}{3}\cdot \dfrac{\mathrm{d}\left(B,\left(MEP\right)\right)}{\mathrm{d}\left(B,\left(A'B'C'\right)\right)}\cdot \dfrac{S_{MEP}}{S_{DEF}}=\dfrac{1}{3}\cdot \dfrac{1}{2}\cdot \dfrac{1}{4}=\dfrac{1}{24}$.\\
		Suy ra $V_{MNP.ABC}=\dfrac{1}{2}\cdot{V_{ABC.A'B'C'}}-3\cdot \dfrac{1}{24}\cdot{V_{ABC.A'B'C'}}=\dfrac{3}{8}\cdot{V_{ABC.A'B'C'}}=\dfrac{3}{8}\cdot 6\cdot \dfrac{4^2\sqrt{3}}{4}=9\sqrt{3}$.
	}
\end{ex}
\Closesolutionfile{ans}
\indapan{10}{ans/CD14/Muc10}
\chapter{CỰC TRỊ THỂ TÍCH KHỐI ĐA DIỆN}
\section{Mức 9,10 điểm}
\setcounter{ex}{0}
\setcounter{dang}{0}
\Opensolutionfile{ans}[ans/CD1/Muc_9_10]
\begin{dang}{Tìm m để hàm số đơn điệu trên các khoảng xác định của nó}
	Đang thiếu bài thầy Jf Câu 1 đến 26 
\end{dang}
\begin{dang}
	{Tìm khoảng đơn điệu của hàm số $g(x) = f\left[ u(x)\right] +v(x)$ khi biết đồ thị hoặc bảng biến thiên của hàm số $y = f'(x)$}
\end{dang}
\begin{ex}[Đề tham khảo 2019]%[2D1K1-2]
	Cho hàm số $f(x)$ có bảng xét dấu của đạo hàm như sau
	\begin{center}
		\begin{tikzpicture}
			\tkzTabInit[nocadre,lgt=1.2,espcl=2,deltacl=0.6]
			{$x$ /0.6,$f'(x)$ /0.6}
			{$-\infty$,$1$,$2$,$3$,$4$,$+\infty$}
			\tkzTabLine{,-,$0$,+,$0$,+,$0$,-,$0$,+,}
		\end{tikzpicture}
	\end{center}
	Hàm số $y=3 f(x+2)-x^3+3 x$ đồng biến trên khoảng nào dưới đây?
	\choice
	{$(-\infty ;-1)$}
	{\True $(-1 ; 0)$}
	{$(0 ; 2)$}
	{$(1 ;+\infty)$}
	\loigiai{
		Ta có $y'=3\left[f'(x+2)-\left(x^2-3\right)\right]$.\\
		Với $x \in(-1 ; 0) \Rightarrow x+2 \in(1 ; 2) \Rightarrow f'(x+2)>0$, lại có $x^2-3<0 \Rightarrow y'>0 ;~ \forall x \in(-1 ; 0)$.\\
		Vậy hàm số $y=3 f(x+2)-x^3+3 x$ đồng biến trên khoảng $(-1 ; 0)$.\\
		Chú ý:\\
		+) Ta xét $x \in(1 ; 2) \subset(1 ;+\infty)
		\Rightarrow x+2 \in(3 ; 4)\\
		\Rightarrow f'(x+2)<0 ;~ x^2-3>0$\\
		Suy ra hàm số nghịch biến trên khoảng $(1 ; 2)$ nên loại hai phương án$(0 ; 2)$ và $(1 ;+\infty)$.\\
		+) Tương tự ta xét
		$x \in(-\infty ;-2) \Rightarrow x+2 \in(-\infty ; 0)\\
		\Rightarrow f'(x+2)<0 ; x^2-3>0 \Rightarrow y'<0 ; ~ \forall x \in(-\infty ;-2)$.\\
		Suy ra hàm số nghịch biến trên khoảng $(-\infty ;-2)$ nên loại$(-\infty ;-1)$.\\
		Vậy hàm số đã cho đồng biến trên khoảng $(-1 ; 0)$.
	}
\end{ex}
\begin{ex}[Đề Tham Khảo 2020 - Lần 1]%[2D1G1-2]
	\immini{
		Cho hàm số $f(x)$. Hàm số $y=f'(x)$ có đồ thị như hình bên. Hàm số $g(x)=f(1-2 x)+x^2-x$ nghịch biến trên khoảng nào dưới đây?
		\choice
		{\True $\left(1 ; \dfrac{3}{2}\right)$}
		{$\left(0 ; \dfrac{1}{2}\right)$}
		{$(-2 ;-1)$}
		{$(2 ; 3)$}
	}
	{
		\begin{tikzpicture}[scale=0.7,>=stealth, font=\footnotesize, line join=round, line cap=round]
			%\def\a{1} \def\b{-6} \def\c{9} \def\d{1} % Hệ số
			\def\xmin{-4} \def\xmax{6}
			\def\ymin{-3} \def\ymax{2} 
			%\draw[color=gray!50,dashed] (\xmin,\ymin) grid (\xmax,\ymax); 
			\draw[->] (\xmin,0)--(\xmax,0) node [below]{$x$};
			\draw[->] (0,\ymin)--(0,\ymax) node [left]{$y$};
			\node at (0,0) [below left]{$O$};
			%\node at (1,3) [below left]{$f'(x)$};
			%\node at (-1.3,4) {$f'(x)$};
			\draw[dashed] (-2,0) node[below]{$-2$}--(-2,1)--(0,1) node[below left]{$1$};
			\draw[dashed] (4,0) node[below left]{$4$}--(4,-2)--(0,-2) node[below left]{$-2$};
			%\draw[dashed] (1,0) node[below]{$1$}--(1,1);
			%\draw[dashed] (-0.5,0) node[below left]{$-0{,}5$}--(-0.5,2.125);
			\clip (\xmin+0.1,\ymin+0.1) rectangle (\xmax-0.5,\ymax-0.1);
			\draw[smooth,samples=300][domain=-4:5.5] plot(\x,{0.071*(\x)^3-0.142*(\x)^2-1.07*(\x)});
		\end{tikzpicture}
	}
	
	\loigiai{
		Ta có : $g(x)=f(1-2 x)+x^2-x \Rightarrow g'(x)=-2 f'(1-2 x)+2 x-1$.\\
		\immini{
			Đặt $t=1-2 x \Rightarrow g'(x)=-2 f'(t)-t$.\\
			$g'(x)=0 \Rightarrow f'(t)=-\dfrac{t}{2}$.\\
			Vẽ đường thẳng $y=-\dfrac{x}{2}$ và đồ thị hàm số $f'(x)$ trên cùng một hệ trục
		}	
		{
			\begin{tikzpicture}[scale=0.7,>=stealth, font=\footnotesize, line join=round, line cap=round]
				%\def\a{1} \def\b{-6} \def\c{9} \def\d{1} % Hệ số
				\def\xmin{-4} \def\xmax{6}
				\def\ymin{-3} \def\ymax{2} 
				%	\draw[color=gray!50,dashed] (\xmin,\ymin) grid (\xmax,\ymax); 
				\draw[->] (\xmin,0)--(\xmax,0) node [below]{$x$};
				\draw[->] (0,\ymin)--(0,\ymax) node [left]{$y$};
				\node at (0,0) [below left]{$O$};
				%\node at (1,3) [below left]{$f'(x)$};
				%\node at (-1.3,4) {$f'(x)$};
				\draw[dashed] (-2,0) node[below]{$-2$}--(-2,1)--(0,1) node[below left]{$1$};
				\draw[dashed] (4,0) node[below]{$4$}--(4,-2)--(0,-2) node[below left]{$-2$};
				%\draw[dashed] (1,0) node[below]{$1$}--(1,1);
				%\draw[dashed] (-0.5,0) node[below left]{$-0{,}5$}--(-0.5,2.125);
				\clip (\xmin+0.1,\ymin+0.1) rectangle (\xmax-0.5,\ymax-0.1);
				\draw[smooth,samples=300][domain=-4:5.5] plot(\x,{0.071*(\x)^3-0.142*(\x)^2-1.07*(\x)});
				\draw[smooth,samples=300][domain=-4:5.5] plot(\x,{(-0.5*(\x)});
			\end{tikzpicture}
		}	Hàm số $g(x)$ nghịch biến $\Rightarrow g'(x) \leq 0 \Rightarrow f'(t) \geq-\dfrac{t}{2}\Rightarrow\hoac{&-2 \leq t \leq 0 \\&t \geq 4.}$\\
		Như vậy $f'(1-2 x) \geq \dfrac{1-2 x}{-2}\Rightarrow\hoac{&-2 \leq 1-2 x \leq 0 \\ &4 \leq 1-2 x}\Rightarrow\hoac{&\dfrac{1}{2}\leq x \leq \dfrac{3}{2}\\ &x \leq-\dfrac{3}{2}.}$\\
		Vậy hàm số $g(x)=f(1-2 x)+x^2-x$ nghịch biến trên các khoảng $\left(\dfrac{1}{2}; \dfrac{3}{2}\right)$ và $\left(-\infty ;-\dfrac{3}{2}\right)$.\\
		Mà $\left(1 ; \dfrac{3}{2}\right) \subset \left(\dfrac{1}{2}; \dfrac{3}{2}\right)$ nên hàm số $g(x)=f(1-2 x)+x^2-x$ nghịch biến trên khoảng $\left(1 ; \dfrac{3}{2}\right)$.
	}
\end{ex}
\begin{ex}[Chuyên Lê Quý Đôn Điện Biên 2019]%[2D1G1-2]
	Cho hàm số $f(x)$ có bảng xét dấu của đạo hàm như sau
	\begin{center}
		\begin{tikzpicture}
			\tkzTabInit[nocadre,lgt=1.2,espcl=2,deltacl=0.6]
			{$x$ /0.6,$f'(x)$ /0.6}
			{$-\infty$,$0$,$1$,$2$,$3$,$+\infty$}
			\tkzTabLine{,+,$0$,-,$0$,-,$0$,+,$0$,-,}
		\end{tikzpicture}
	\end{center}
	Hàm số $y=f(x-1)+x^3-12 x+2019$ nghịch biến trên khoảng nào dưới đây?
	\choice
	{$(1 ;+\infty)$}
	{\True $(1 ; 2)$}
	{$(-\infty ; 1)$}
	{$(3 ; 4)$}
	\loigiai{
		$y'=f'(x-1)+3 x^2-12=f'(t)+3 t^2+6 t-9=f'(t)-\left(-3 t^2-6 t+9\right)$, với $t=x-1$.\\
		\immini{
			Nghiệm của phương trình $y'=0$ là hoành độ giao điểm của các đồ thị hàm số $y=f'(t)$ và $y=-3 t^2-6 t+9$.\\
			Vẽ đồ thị hàm số $y=f'(t)$ và $y=-3 t^2-6 t+9$ trên cùng một hệ trục tọa độ như hình vẽ bên.
		}	
		{		\begin{tikzpicture}[scale=0.5,>=stealth, font=\footnotesize, line join=round, line cap=round]
				\def\a{-3} \def\b{-6} \def\c{9} % Hệ số
				\def\xmin{-9} \def\xmax{7}
				\def\ymin{-3} \def\ymax{13}
				
				%\draw[color=gray!50,dashed] (\xmin,\ymin) grid (\xmax,\ymax);
				
				\draw[->] (\xmin,0)--(\xmax,0) node [below]{$x$};
				\draw[->] (0,\ymin)--(0,\ymax) node [left]{$y$};
				\node at (0,0) [below left]{$O$};
				\clip (\xmin+0.1,\ymin+0.1) rectangle (\xmax-0.5,\ymax-0.1);
				\draw[smooth,samples=300] plot(\x,{\a*(\x)^2+\b*(\x)+\c});
				\node at (1,0) [above right]{$1$};
				\node at (2,0) [below right]{$2$};
				\node at (3,0) [below right]{$3$};
				\node at (-3,-2) [left]{$y=-3t^2-6t+9$};
				\node at (4,0) [below right]{$f'(x)$};
				\draw (-2.2,10).. controls (-1,1.9) and (-0.5,0.8) .. (0,0);
				%\draw (-2,0).. controls (-1.5,-2) and (-0.5,-0) .. (0,0);
				\draw (0,0).. controls (0.4,-0.6) and (0.6,-0.6) .. (0.8,-0.2);
				\draw (0.8,-0.2).. controls (1,0.25) and (1.1,-0.1) .. (1.4,-0.8);
				\draw (1.4,-0.8).. controls (1.6,-1.1) and (1.7,-0.9) .. (2,0);
				\draw (2,0).. controls (2.4,1.1) and (2.6,1.1) .. (3.5,-1);
			\end{tikzpicture}
		}
		Dựa vào đồ thị trên, ta có bảng xét dấu của hàm số $y'=f'(t)-\left(-3 t^2-6 t+9\right)$ như sau $
		\left(t_0<-1\right)$
		\begin{center}
			\begin{tikzpicture}
				\tkzTabInit[nocadre,lgt=2,espcl=2,deltacl=0.6]
				{$x$ /0.6,$y'$ /0.6}
				{$-\infty$,$t_0$,$1$,$+\infty$}
				\tkzTabLine{,+,$0$,-,$0$,+,}
			\end{tikzpicture}
		\end{center}
		Hàm số nghịch biến trên khoảng $t \in\left(t_0 ; 1\right)$.\\
		Do đó hàm số nghịch biến trên khoảng $x \in(1 ; 2) \subset \left(t_0+1 ; 1\right)$.
	}
\end{ex}


\begin{ex}[Chuyên Phan Bội Châu Nghệ An 2019]%[2D1G1-2]
	Cho hàm số $f(x)$ có bảng xét dấu đạo hàm như sau:
	\begin{center}
		\begin{tikzpicture}
			\tkzTabInit[nocadre,lgt=2,espcl=2,deltacl=0.6]
			{$x$ /0.6,$f'(x)$ /0.6}
			{$-\infty$,$1$,$2$,$3$,$4$,$+\infty$}
			\tkzTabLine{,-,$0$,+,$0$,+,$0$,-,$0$,+,}
		\end{tikzpicture}
	\end{center}
	Hàm số $y=2 f(1-x)+\sqrt{x^2+1}-x$ nghịch biến trên những khoảng nào dưới đây
	\choice
	{$(-\infty ;-2)$}
	{$(-\infty ; 1)$}
	{\True $(-2 ; 0)$}
	{$(-3 ;-2)$}
	\loigiai{
		$y'=-2 f'(1-x)+\dfrac{x}{\sqrt{x^2+1}}-1$. \\
		Có $\dfrac{x}{\sqrt{x^2+1}}-1<0,~ \forall x \in(-2 ; 0)$.\\
		Bảng xét dấu:
		\begin{center}
			\begin{tikzpicture}
				\tkzTabInit[nocadre,lgt=2,espcl=2,deltacl=0.6]
				{$x$ /0.7,$f'(1-x)$ /0.7}
				{$-\infty$,$-3$,$-2$,$-1$,$0$,$+\infty$}
				\tkzTabLine{,+,$0$,-,$0$,+,$0$,+,$0$,-,}
			\end{tikzpicture}
		\end{center}
		$\Rightarrow-2 f'(1-x)<0, ~ \forall x \in(-2 ; 0) \\
		\Rightarrow-2 f'(1-x)+\dfrac{x}{\sqrt{x^2+1}}-1<0, ~\forall x \in(-2 ; 0)$.
	}
\end{ex}
\begin{ex}[Sở Vĩnh Phúc 2019]%[2D1G1-2]
	\immini{
		Cho hàm số bậc bốn $y=f(x)$ có đồ thị của hàm số $y=f'(x)$ như hình vẽ bên.\\
		Hàm số $y=3 f(x)+x^3-6 x^2+9 x$ đồng biến trên khoảng nào trong các khoảng sau đây?
		\choice
		{$(0 ; 2)$}
		{$(-1 ; 1)$}
		{$(1 ;+\infty)$}
		{\True $(-2 ; 0)$}
	}
	{
		\begin{tikzpicture}[scale=0.7,>=stealth, font=\footnotesize, line join=round, line cap=round]
			\def\a{0.21} \def\b{0.88} \def\c{-0.58} \def\d{-3} % Hệ số
			\def\xmin{-5} \def\xmax{5}
			\def\ymin{-4} \def\ymax{3} 
			%\draw[color=gray!50,dashed] (\xmin,\ymin) grid (\xmax,\ymax); 
			\draw[->] (\xmin,0)--(\xmax,0) node [below]{$x$};
			\draw[->] (0,\ymin)--(0,\ymax) node [left]{$y$};
			\node at (0,0) [above left]{$O$};
			\node at (-4,0) [below left]{$-4$};
			\node at (-2,0) [below left]{$-2$};
			\node at (0,-3) [below right]{$-3$};
			\draw[dashed] (2,0) node[above right]{$2$}--(2,1) --(0,1) node[above right]{$1$};
			\clip (\xmin+0.1,\ymin+0.1) rectangle (\xmax-0.5,\ymax-0.1);
			\draw[smooth,samples=300] plot(\x,{\a*(\x)^3+\b*(\x)^2+\c*(\x)+\d});
		\end{tikzpicture}
	}
	
	\loigiai{
		Hàm số $f(x)=a x^4+b x^3+c x^2+d x+e,(a \neq 0)$.
		Có $f'(x)=4 a x^3+3 b x^2+2 c x+d$.\\
		Đồ thị hàm số $y=f'(x)$ đi qua các điểm $(-4 ; 0),(-2 ; 0),(0 ;-3),(2 ; 1)$ nên ta có
		$$\heva{&- 2 5 6 a + 4 8 b - 8 c + d = 0\\
			&- 3 2 a + 1 2 b - 4 c + d = 0\\
			&d = - 3\\
			&3 2 a + 1 2 b + 4 c + d = 1}\Leftrightarrow \heva{&
			a=\dfrac{5}{96}\\
			&b=\dfrac{7}{24}\\
			&c=-\dfrac{7}{24}\\
			&d=-3.}
		$$
		Xét hàm số
		$
		y=3 f(x)+x^3-6 x^2+9 x$\\
		Ta có $ y'=3\left(f'(x)+x^2-4 x+3\right)=3\left(\frac{5}{24}x^3+\frac{15}{8}x^2-\frac{55}{12}x\right)
		$\\
		Ta có $y'=0 \Leftrightarrow\hoac{&x=-11 \\&x=0 \\&x=2.}$ \\
		Xét dấu $y'$, ta được hàm số đã cho đồng biến trên các khoảng $(-11 ; 0)$ và $(2 ;+\infty)$.
	}
\end{ex}
\begin{ex}[Học Mãi 2019]%[2D1K1-2]
	\immini
	{Cho hàm số $y=f(x)$ có đạo hàm trên $\mathbb{R}$. Đồ thị hàm số $y=f'(x)$ như hình bên. Hỏi đồ thị hàm số $y=f(x)-2 x$ có bao nhiêu điểm cực trị?
		\choice
		{$4$}
		{\True $3$}
		{$2$}
		{$1$}
	}
	{
		\begin{tikzpicture}[font=\footnotesize,line join=round, line cap=round,>=stealth,scale=0.8]
			\draw[->] (-3.5,0)--(4,0) node[above] {$x$};
			\draw[->] (0,-3)--(0,4) node[left] {$y$};
			%\fill[black] (-2,0)node[below left]{$-2$} circle (1.2pt) (0,0)node[above right]{$O$} circle (1.2pt) (3,0)node[above]{$3$} circle (1.2pt);
			\draw[dashed] (-2,-2)-- (0,-2) node[right]{$-2$};
			\draw[dashed] (2,0) node[below]{$2$}-- (2,2)--(0,2) node[below left]{$2$};
			\node at (0,0) [below left]{$O$};
			\node at (3,0) [below right]{$3$};
			\draw (-3,2.5).. controls (-2.2,-3) and (-1.8,-3) .. (-1.1,0);
			\draw (-1.1,0).. controls (-0.6,2.5) and (-0.4,2.5) .. (0,2);
			\draw (0,2).. controls (0.7,0.5) and (1.1,0.5) .. (1.5,1.5);
			\draw (1.5,1.5).. controls (2,2.5) and (2.8,2.5) .. (3.5,-2.5);
			%\draw (3,0).. controls (3.3,-0.1) and (3.5,-0.5) .. (3.5,-2);
		\end{tikzpicture}
	}
	\loigiai{
		\immini{
			Đặt $g(x)=f(x)-2 x$.\\
			$\Rightarrow g'(x)=f'(x)-2 .
			$\\
			Vẽ đường thẳng $y=2$.\\
			$\Rightarrow$ phương trình $g'(x)=0$ có $3$ nghiệm bội lẻ.\\
			$\Rightarrow$ đồ thị hàm số $y=f(x)-2 x$ có $3$ điểm cực trị.
		}
		{
			\begin{tikzpicture}[font=\footnotesize,line join=round, line cap=round,>=stealth,scale=0.8]
				\draw[->] (-3.5,0)--(4,0) node[above] {$x$};
				\draw[->] (0,-3)--(0,4) node[left] {$y$};
				%\fill[black] (-2,0)node[below left]{$-2$} circle (1.2pt) (0,0)node[above right]{$O$} circle (1.2pt) (3,0)node[above]{$3$} circle (1.2pt);
				\draw[dashed] (-2,-2)-- (0,-2) node[right]{$-2$};
				\draw[dashed] (2,0) node[below]{$2$}-- (2,2)--(0,2) node[below left]{$2$};
				\node at (3,0) [below left]{$3$};
				\draw (-3,2.5).. controls (-2.2,-3) and (-1.8,-3) .. (-1.1,0);
				\draw (-1.1,0).. controls (-0.6,2.5) and (-0.4,2.5) .. (0,2);
				\draw (0,2).. controls (0.7,0.5) and (1.1,0.5) .. (1.5,1.5);
				\draw (1.5,1.5).. controls (2,2.5) and (2.8,2.5) .. (3.5,-2.5);
				\draw (-3.5,2)--(4,2) node[above]{$y=2$};
			\end{tikzpicture}
		}
	}
\end{ex}
\begin{ex}[THPT Hoàng Hoa Thám Hưng Yên 2019]%[2D1G1-2]
	\immini{
		Cho hàm số $y=f(x)$ liên tục trên $\mathbb{R}$. Hàm số $y=f'(x)$ có đồ thị như hình vẽ. 
		Hàm số $g(x)=f(x-1)+\dfrac{2019-2018 x}{2018}$ đồng biến trên khoảng nào dưới đây?
		\choice
		{$(2 ; 3)$}
		{$(0 ; 1)$}
		{\True $(-1 ; 0)$}
		{$(1 ; 2)$}
	}
	{
		\begin{tikzpicture}[scale=1, font=\footnotesize, line join=round, line cap=round, >=stealth]
			\tikzset{label style/.style={font=\footnotesize}}
			\draw[->] (-2,0)--(3,0) node[below left] {$x$};
			\draw[->] (0,-2)--(0,3) node[below left] {$y$};
			\draw[fill=black] (0,0) node [above left] {$O$} circle(1pt);
			\fill (1,1) circle(1pt) (-1,1) circle(1pt) (2,1) circle(1pt);
			\foreach \x in {1,2}
			\draw[thin] (\x,1pt)--(\x,-1pt) node [below] {\footnotesize$\x$};
			\foreach \x in {-1}
			\draw[thin] (\x,1pt)--(\x,-1pt) node [below left] {\footnotesize$\x$};
			\foreach \y in {-1}
			\draw[thin] (1pt,\y)--(-1pt,\y) node [right] {\footnotesize$\y$};
			\foreach \y in {1}
			\draw[thin] (1pt,\y)--(-1pt,\y) node [above left] {\footnotesize$\y$};
			\draw[dashed](-1,0)--(-1,1)--(2,1) (1,1)--(1,0) (2,1)--(2,0);
			\begin{scope}
				\clip (-3,-3) rectangle (3,3);
				\draw[name path=(C)] plot[smooth,tension=0.7] coordinates{(-1.15,3)(-0.5,-1.6)(.8,.88)(1.9,0.8)(2.3,3)};
			\end{scope}
		\end{tikzpicture}
	}	\loigiai{
		Ta có $g'(x)=f'(x-1)-1$.\\
		$
		g'(x) \geq 0 \Leftrightarrow f'(x-1)-1 \geq 0 \Leftrightarrow f'(x-1) \geq 1 \Leftrightarrow \hoac{&x - 1 \leq - 1\\
			&x - 1 \geq 2}\Leftrightarrow \hoac{&
			x \leq 0 \\
			&x \geq 3.}
		$\\
		Từ đó suy ra hàm số $g(x)=f(x-1)+\dfrac{2019-2018 x}{2018}$ đồng biến trên khoảng $(-1 ; 0)$.
	}
\end{ex}

\begin{ex}[(Sở Ninh Bình 2019]%[2D1K1-2]
	Cho hàm số $y=f(x)$ có bảng xét dấu của đạo hàm như sau
	\begin{center}
		\begin{tikzpicture}
			\tkzTabInit[nocadre,lgt=1,espcl=2,deltacl=0.6]
			{$x$ /0.7,$f'(x)$ /0.7}
			{$-\infty$,$-2$,$-1$,$2$,$4$,$+\infty$}
			\tkzTabLine{,+,$0$,-,$0$,+,$0$,-,$0$,+,}
		\end{tikzpicture}
	\end{center}
	Hàm số $y=-2 f(x)+2019$ nghịch biến trên khoảng nào trong các khoảng dưới đây?
	\choice
	{$(-4 ; 2)$}
	{\True $(-1 ; 2)$}
	{$(-2 ;-1)$}
	{$(2 ; 4)$}
	\loigiai{
		Xét $y=g(x)=-2 f(x)+2019$.\\
		Ta có $g'(x)=(-2 f(x)+2019)'=-2 f'(x), g'(x)=0 \Leftrightarrow\hoac{&x=-2 \\&x=-1 \\&x=2 \\&x=4.}$.\\
		Ta có bảng xét dấu của $g'(x)$
		\begin{center}
			\begin{tikzpicture}
				\tkzTabInit[nocadre,lgt=1,espcl=2,deltacl=0.6]
				{$x$ /0.6,$f'(x)$ /0.6}
				{$-\infty$,$-2$,$-1$,$2$,$4$,$+\infty$}
				\tkzTabLine{,-,$0$,+,$0$,-,$0$,+,$0$,+,}
			\end{tikzpicture}
		\end{center}
		Dựa vào bảng xét dấu, ta thấy hàm số $y=g(x)$ nghịch biến trên khoảng $(-1 ; 2)$.
	}
\end{ex}
\begin{ex}[THPT Lương Thế Vinh Hà Nội 2019]%[2D1G1-2]
	\immini{
		Cho hàm số $y=f(x)$. Biết đồ thị hàm số $y=f'(x)$ có đồ thị như hình vẽ bên. 
		Hàm số $y=f \left(3-x^2\right)+2018$ đồng biến trên khoảng nào dưới đây?
		\choice
		{\True $(-1 ; 0)$}
		{$(2 ; 3)$}
		{$(-2 ;-1)$}
		{$(0 ; 1)$}
	}
	{
		\begin{tikzpicture}[scale=0.6,>=stealth, font=\footnotesize, line join=round, line cap=round]
			\def\a{0.065} \def\b{0.32} \def\c{-0.53} \def\d{-0.82} % Hệ số
			\def\xmin{-8} \def\xmax{4}
			\def\ymin{-3} \def\ymax{3} 
			%\draw[color=gray!50,dashed] (\xmin,\ymin) grid (\xmax,\ymax); 
			\draw[->] (\xmin,0)--(\xmax,0) node [below]{$x$};
			\draw[->] (0,\ymin)--(0,\ymax) node [left]{$y$};
			\node at (0,0) [below left]{$O$};
			\node at (-6,0) [below left]{$-6$};
			\node at (-1,0) [below left]{$-1$};
			\node at (2,0) [below right]{$2$};
			\clip (\xmin+0.1,\ymin+0.1) rectangle (\xmax-0.5,\ymax-0.1);
			\draw[smooth,samples=300][domain=-6.5:3.5] plot(\x,{\a*(\x)^3+\b*(\x)^2+\c*(\x)+\d});
		\end{tikzpicture}
	}
	
	\loigiai{
		Ta có $\left[f\left( 3-x^2\right)+2018 \right]'=-2 x \cdot f'\left(3-x^2\right) $.\\
		$
		-2 x \cdot f'\left(3-x^2\right)=0 \Leftrightarrow\hoac{&
			x = 0\\
			&3 - x ^{2}= - 6\\
			&3 - x ^{2}= - 1\\
			&3 - x ^{2}= 2}
		\Leftrightarrow \hoac{
			&x=0 \\
			&x=\pm 3 \\
			&x=\pm 2 \\
			&	x=\pm 1.}
		$\\
		Bảng xét dấu của đạo hàm hàm số đã cho
		\begin{center}
			\begin{center}
				\begin{tikzpicture}
					\tkzTabInit[nocadre,lgt=2.9,espcl=1.5,deltacl=0.6]
					{$x$ /0.7,$f'\left( 3-x^2\right) $/0.7,$-2xf'\left( 3-x^2\right)$/0.8}
					{$-\infty$,$-3$,$-2$,$-1$,$0$,$1$,$2$,$3$,$+\infty$}
					\tkzTabLine{,-,$0$,+,$0$,-,$0$,+,$0$,+,$0$,-,$0$,+,$0$,-}
					\tkzTabLine{,-,$0$,+,$0$,-,$0$,+,$0$,-,$0$,+,$0$,-,$0$,+}
				\end{tikzpicture}
			\end{center}
		\end{center}
		Từ bảng xét dấu suy ra hàm số đồng biến trên $(-1 ; 0)$.
	}
\end{ex}
\begin{ex}[Chuyên Biên Hòa - Hà Nam - 2020]%[2D1G1-2]
	\immini{
		Cho hàm số đa thức $f(x)$ có đạo hàm trên $\mathbb{R}$. Biết $f(0)=0$ và đồ thị hàm số $y=f'(x)$ như hình sau.
		Hàm số $g(x)=\left|4 f(x)+x^2\right|$ đồng biến trên khoảng nào dưới đây?
		\choice
		{$(4 ;+\infty)$}
		{\True $(0 ; 4)$}
		{$(-\infty ;-2)$}
		{$(-2 ; 0)$}
	}	
	{
		\begin{tikzpicture}[scale=0.7,>=stealth, font=\footnotesize, line join=round, line cap=round]
			%\def\a{1} \def\b{-6} \def\c{9} \def\d{1} % Hệ số
			\def\xmin{-4} \def\xmax{6}
			\def\ymin{-3} \def\ymax{2} 
			%\draw[color=gray!50,dashed] (\xmin,\ymin) grid (\xmax,\ymax); 
			\draw[->] (\xmin,0)--(\xmax,0) node [below]{$x$};
			\draw[->] (0,\ymin)--(0,\ymax) node [left]{$y$};
			\node at (0,0) [below left]{$O$};
			%\node at (1,3) [below left]{$f'(x)$};
			%\node at (-1.3,4) {$f'(x)$};
			\draw[dashed] (-2,0) node[below]{$-2$}--(-2,1)--(0,1) node[below left]{$1$};
			\draw[dashed] (4,0) node[below]{$4$}--(4,-2)--(0,-2) node[below left]{$-2$};
			%\draw[dashed] (1,0) node[below]{$1$}--(1,1);
			%\draw[dashed] (-0.5,0) node[below left]{$-0{,}5$}--(-0.5,2.125);
			\clip (\xmin+0.1,\ymin+0.1) rectangle (\xmax-0.5,\ymax-0.1);
			\draw[smooth,samples=300][domain=-4:5.5] plot(\x,{0.071*(\x)^3-0.142*(\x)^2-1.07*(\x)});
		\end{tikzpicture}
	}
	\loigiai{
		\immini{
			Xét hàm số $h(x)=4 f(x)+x^2$ trên $\mathbb{R}$.\\
			Vì $f(x)$ là hàm số đa thức nên $h(x)$ cũng là hàm số đa thức và $h(0)=4 f(0)=0$.\\
			Ta có $h'(x)=4 f'(x)+2 x$. Do đó $h'(x)=0 \Leftrightarrow f'(x)=-\dfrac{1}{2}x$.\\
		}
		{
			\begin{tikzpicture}[scale=0.7,>=stealth, font=\footnotesize, line join=round, line cap=round]
				%\def\a{1} \def\b{-6} \def\c{9} \def\d{1} % Hệ số
				\def\xmin{-4} \def\xmax{6}
				\def\ymin{-3} \def\ymax{2} 
				%\draw[color=gray!50,dashed] (\xmin,\ymin) grid (\xmax,\ymax); 
				\draw[->] (\xmin,0)--(\xmax,0) node [below]{$x$};
				\draw[->] (0,\ymin)--(0,\ymax) node [left]{$y$};
				\node at (0,0) [below left]{$O$};
				%\node at (1,3) [below left]{$f'(x)$};
				%\node at (-1.3,4) {$f'(x)$};
				\draw[dashed] (-2,0) node[below]{$-2$}--(-2,1)--(0,1) node[below left]{$1$};
				\draw[dashed] (4,0) node[below]{$4$}--(4,-2)--(0,-2) node[below left]{$-2$};
				%\draw[dashed] (1,0) node[below]{$1$}--(1,1);
				%\draw[dashed] (-0.5,0) node[below left]{$-0{,}5$}--(-0.5,2.125);
				\clip (\xmin+0.1,\ymin+0.1) rectangle (\xmax-0.5,\ymax-0.1);
				\draw[smooth,samples=300][domain=-4:5.5] plot(\x,{0.071*(\x)^3-0.142*(\x)^2-1.07*(\x)});
				\draw[smooth,samples=300][domain=-4:5.5] plot(\x,{-0.5*(\x)});
			\end{tikzpicture}
		}
		Dựa vào sự tương giao của đồ thị hàm số $y=f'(x)$ và đường thẳng $y=-\dfrac{1}{2}x$, ta có
		$
		h'(x)=0 \Leftrightarrow x \in\{-2 ; 0 ; 4\}.\\
		$
		Bảng biến thiên của hàm số $h(x)$ như sau:
		\begin{center}
			\begin{tikzpicture}
				\tkzTabInit[nocadre,lgt=1.2,espcl=2.5,deltacl=0.6]
				{$x$ /0.6,$y'$ /0.6,$y$ /2}
				{$-\infty$,$-2$,$0$,$4$,$+\infty$}
				\tkzTabLine{,-,$0$,+,$0$,-,$0$,+,}
				\tkzTabVar{+/$+\infty$, -/$y_1$,+/$0$,-/$y_3$,+/$+\infty$}
			\end{tikzpicture}
		\end{center}
		Từ đó suy ra bảng biến thiên của hàm số $g(x)=|h(x)|$.\\
		Dựa vào bảng biến thiên trên, ta thấy hàm số $g(x)$ đồng biến trên khoảng $(0 ; 4)$.
	}
\end{ex}
\begin{ex}[Chuyên Thái Bình - 2020]%[2D1G1-2]
	\immini{
		Cho hàm số $f(x)$ liên tục trên $\mathbb{R}$ có đồ thị hàm số $y=f'(x)$ cho như hình vẽ bên.\\
		Hàm số $g(x)=2 f(|x-1|)-x^2+2 x+2020$ đồng biến trên khoảng nào?
		\choice
		{\True $(0 ; 1)$}
		{$(-3 ; 1)$}
		{$(1 ; 3)$}
		{$(-2 ; 0)$}
	}
	{
		\begin{tikzpicture}[scale=0.7,>=stealth, font=\footnotesize, line join=round, line cap=round]
			%\def\a{1} \def\b{-6} \def\c{9} \def\d{1} % Hệ số
			\def\xmin{-4} \def\xmax{5}
			\def\ymin{-3} \def\ymax{5} 
			%\draw[color=gray!50,dashed] (\xmin,\ymin) grid (\xmax,\ymax); 
			\draw[->] (\xmin,0)--(\xmax,0) node [below]{$x$};
			\draw[->] (0,\ymin)--(0,\ymax) node [left]{$y$};
			\node at (0,0) [below left]{$O$};
			%\node at (1,3) [below left]{$f'(x)$};
			\node at (-1.3,4) {$f'(x)$};
			\draw[dashed] (-1,0) node[above]{$-1$}--(-1,-1)--(0,-1) node[below left]{$-1$};
			\draw[dashed] (1,0) node[below]{$1$}--(1,1)--(0,1) node[below left]{$1$};
			\draw[dashed] (3,0) node[below]{$3$}--(3,3)--(0,3) node[below left]{$3$};
			%\draw[dashed] (1,0) node[below]{$1$}--(1,1);
			%\draw[dashed] (-0.5,0) node[below left]{$-0{,}5$}--(-0.5,2.125);
			\clip (\xmin+0.1,\ymin+0.1) rectangle (\xmax-0.5,\ymax-0.1);
			\draw[smooth,samples=300][domain=-2:4] plot(\x,{-0.5*(\x)^3+1.5*(\x)^2+1.5*(\x)-1.5});
			%\draw[smooth,samples=300] plot(\x,{(\x)^3+(\x)^2-2*(\x)+1});
		\end{tikzpicture}
	}
	\loigiai{
		Ta có đường thẳng $y=x$ cắt đồ thị hàm số $y=f'(x)$ tại các điểm $x=-1 ; x=1 ; x=3$ như hình vẽ sau:
		\begin{center}
			\begin{tikzpicture}[scale=0.7,>=stealth, font=\footnotesize, line join=round, line cap=round]
				%\def\a{1} \def\b{-6} \def\c{9} \def\d{1} % Hệ số
				\def\xmin{-4} \def\xmax{5}
				\def\ymin{-3} \def\ymax{5} 
				%\draw[color=gray!50,dashed] (\xmin,\ymin) grid (\xmax,\ymax); 
				\draw[->] (\xmin,0)--(\xmax,0) node [below]{$x$};
				\draw[->] (0,\ymin)--(0,\ymax) node [left]{$y$};
				\node at (0,0) [below left]{$O$};
				%\node at (1,3) [below left]{$f'(x)$};
				\node at (-1.3,4) {$f'(x)$};
				\node at (4,3.2) {$y=x$};
				\draw[dashed] (-1,0) node[above]{$-1$}--(-1,-1)--(0,-1) node[below left]{$-1$};
				\draw[dashed] (1,0) node[below]{$1$}--(1,1)--(0,1) node[below left]{$1$};
				\draw[dashed] (3,0) node[below]{$3$}--(3,3)--(0,3) node[below left]{$3$};
				%\draw[dashed] (1,0) node[below]{$1$}--(1,1);
				%\draw[dashed] (-0.5,0) node[below left]{$-0{,}5$}--(-0.5,2.125);
				\clip (\xmin+0.1,\ymin+0.1) rectangle (\xmax-0.5,\ymax-0.1);
				\draw[smooth,samples=300][domain=-2:4] plot(\x,{-0.5*(\x)^3+1.5*(\x)^2+1.5*(\x)-1.5});
				\draw[smooth,samples=300] plot(\x,{(\x)});
			\end{tikzpicture}
		\end{center}
		Dựa vào đồ thị của hai hàm số trên ta có $f'(x)>x \Leftrightarrow\hoac{&x<-1 \\ &1<x<3}$ và
		$ f'(x)<x \Leftrightarrow\hoac{&
			-1<x<1 \\
			&x>3.}$\\
		+Trường hợp 1: $x-1<0 \Leftrightarrow x<1$.\\
		Khi đó $g(x)=2 f(1-x)-x^2+2 x+2020$.\\
		Ta có $g'(x)=-2 f'(1-x)+2(1-x)$.
		$$
		g'(x)>0 \Leftrightarrow-2 f'(1-x)+2(1-x)>0 \Leftrightarrow f'(1-x)<1-x \Leftrightarrow\hoac{
			&- 1 < 1 - x < 1\\
			&1 - x > 3} \Leftrightarrow \hoac{&
			0<x<2 \\
			&x<-2.}
		$$
		Kết hợp điều kiện, ta có $g'(x)>0 \Leftrightarrow\hoac{&0<x<1 \\ &x<-2.}$\\
		
		+ Trường hợp 2: $x-1>0 \Leftrightarrow x>1$.\\
		Khi đó ta có $g(x)=2 f(x-1)-x^2+2 x+2020$.\\
		$ g'(x)=2 f'(x-1)-2(x-1)$\\
		$g'(x)>0 \Leftrightarrow 2 f'(x-1)-2(x-1)>0 \Leftrightarrow f'(x-1)>x-1 \Leftrightarrow\hoac{&
			x - 1 < - 1\\
			&1 < x - 1 < 3}\Leftrightarrow \hoac{
			&x<0 \\
			&2<x<4.}$
		Kết hợp điều kiện ta có $g'(x)>0 \Leftrightarrow 2<x<4$.\\
		Vậy hàm số $g(x)=2 f(|x-1|)-x^2+2 x+2020$ đồng biến trên khoảng $(0 ; 1)$.
	}
\end{ex}

\begin{ex}[Chuyên Lào Cai - 2020]%[2D1G1-2]
	\immini{
		Cho hàm số $f'(x)$ có đồ thị như hình bên.\\
		Hàm số $g(x)=f(3 x+1)+9 x^3+\dfrac{9}{2}x^2$ đồng biến trên khoảng nào dưới đây?
		\choice
		{$(-1 ; 1)$}
		{$(-2 ; 0)$}
		{$(-\infty ; 0)$}
		{\True $(1 ;+\infty)$}
	}
	{\begin{tikzpicture}[line join=round, line cap=round,>=stealth,thick,scale=.8]
			\tikzset{label style/.style={font=\footnotesize}}
			\draw[->] (-2.1,0)--(5.1,0) node[below left] {$x$};
			\draw[->] (0,-3.1)--(0,4.1) node[below left] {$y$};
			\draw (0,0) node [below left] {$O$};
			\foreach \x in {1,2,3}
			\draw[thin] (\x,1pt)--(\x,-1pt) node [below] {$\x$};
			\draw[thin](-1,1pt)--(1,-1pt)node[above left]{$-1$};
			\foreach \y in {-2,2}
			\draw[thin] (1pt,\y)--(-1pt,\y) node [above right] {$\y$};
			%\begin{scope}
			\clip (-2,-3) rectangle (5,4);
			\draw[samples=200,domain=-2:4,smooth,variable=\x] plot (\x,{(\x)^3-3*(\x)^2+2});
			%\end{scope}
			\draw[dashed] (-1,0)--(-1,-2)--(0,-2);
			\draw[dashed] (3,0)--(3,2)--(0,2);
			%\begin{scope}[on background layer]\path[white]node{MDD-134};\end{scope}
		\end{tikzpicture}
	}
	\loigiai
	{
		\immini{Xét hàm số $g(x)=f(3 x+1)+9 x^3+\dfrac{9}{2}x^2 \\
			\Rightarrow g'(x)=3 f'(3 x+1)+27 x^2+9 x$.\\
			Hàm số đồng biến  $\Leftrightarrow g'(x)>0 \Leftrightarrow 3 f'(3 x+1)+27 x^2+9 x>0$
			\\
			$
			\Leftrightarrow f'(3 x+1)+3 x(3 x+1)>0 \qquad (*)
			$\\
			Đặt $t=3 x+1$, khi đó  $(*) \Leftrightarrow f'(t)+(t-1) t>0$\\ $\Leftrightarrow f'(t)>-t^2+t$.\\
			Vẽ parabol $y=-x^2+x$ và đồ thị hàm số $f'(x)$ trên cùng một hệ trục
		}
		{
			\begin{tikzpicture}[line join=round, line cap=round,>=stealth,thick,scale=.8]
				\tikzset{label style/.style={font=\footnotesize}}
				\draw[->] (-2.1,0)--(5.1,0) node[below left] {$x$};
				\draw[->] (0,-3.1)--(0,4.1) node[below left] {$y$};
				\draw (0,0) node [below left] {$O$};
				\foreach \x in {1,2,3}
				\draw[thin] (\x,1pt)--(\x,-1pt) node [below] {$\x$};
				\draw[thin](-1,1pt)--(1,-1pt);
				\foreach \y in {-2,2}
				\draw[thin] (1pt,\y)--(-1pt,\y) node [above right] {$\y$};
				%\begin{scope}
				\clip (-2,-3) rectangle (5,4);
				\draw[samples=200,domain=-2:4,smooth,variable=\x] plot (\x,{(\x)^3-3*(\x)^2+2});
				\draw[samples=200,domain=-2:4,smooth,variable=\x] plot (\x,{-(\x)^2+(\x)});
				%\end{scope}
				\draw[dashed] (-1,0) node[above left]{$-1$}--(-1,-2)--(0,-2);
				\draw[dashed] (3,0)--(3,2)--(0,2);
				%\begin{scope}[on background layer]\path[white]node{MDD-134};\end{scope}
			\end{tikzpicture}
		}
		Dựa vào đồ thị ta thấy
		$
		f'(t)>-t^2+t \Leftrightarrow\hoac{&- 1 < t < 1\\
			&t > 2}\Rightarrow \hoac{&
			- 1 < 3 x + 1 < 1\\
			&3 x + 1 > 2} \Leftrightarrow \hoac{&
			\dfrac{-2}{3}<x<0\\
			&x>\dfrac{1}{3}.}
		$}
\end{ex}
\begin{ex}[Sở Phú Thọ-2020]%[2D1G1-2]
	\immini{
		Cho hàm số $y=f(x)$ có đồ thị hàm số $y=f'(x)$ như hình vẽ.\\
		Hàm số $g(x)=f\left(\mathrm{e}^x-2\right)-2020$ nghịch biến trên khoảng nào dưới đây?
		\choice
		{\True $\left(-1 ; \dfrac{3}{2}\right)$}
		{$(-1 ; 2)$}
		{$(0 ;+\infty)$}
		{$\left(\dfrac{3}{2}; 2\right)$}
	}
	{
		\begin{tikzpicture}[scale=0.7,>=stealth, font=\footnotesize, line join=round, line cap=round]
			\def\a{1} \def\b{-3} \def\c{0} \def\d{0} % Hệ số
			\def\xmin{-2} \def\xmax{4}
			\def\ymin{-5} \def\ymax{2} 
			%\draw[color=gray!50,dashed] (\xmin,\ymin) grid (\xmax,\ymax); 
			\draw[->] (\xmin,0)--(\xmax,0) node [below]{$x$};
			\draw[->] (0,\ymin)--(0,\ymax) node [left]{$y$};
			\node at (0,0) [above left]{$O$};
			\node at (3,0) [below right]{$3$};
			\draw[dashed] (2,0) node[above]{$2$}--(2,-4) --(0,-4) node[left]{$-4$};
			\clip (\xmin+0.1,\ymin+0.1) rectangle (\xmax-0.5,\ymax-0.1);
			\draw[smooth,samples=300] plot(\x,{\a*(\x)^3+\b*(\x)^2+\c*(\x)+\d});
		\end{tikzpicture}
	}
	
	\loigiai{
		Dựa vào đồ thị hàm số $y=f'(x)$ suy ra $f'(x) \leq 0 ~ \forall x<3$ và $f'(x)>0 ~ \forall x>3$.
		$
		g'(x)=\mathrm{e}^x f'\left(\mathrm{e}^x-2\right) .
		$
		Hàm số $g(x)=f\left(\mathrm{e}^x-2\right)-2020$ nghịch biến \\ $ \Leftrightarrow g'(x)<0 \Leftrightarrow \mathrm{e}^x f'\left(\mathrm{e}^x-2\right)<0$\\
		$
		\Leftrightarrow f'\left(\mathrm{e}^x-2\right)<0 \Leftrightarrow \mathrm{e}^x-2<3 \Leftrightarrow \mathrm{e}^x<5 \Leftrightarrow x<\ln 5 .
		$\\
		Vậy hàm số đã cho nghịch biến trên $\left(-1 ; \dfrac{3}{2}\right)$.
	}
\end{ex}
\begin{ex}[Lý Nhân Tông - Bắc Ninh - 2020]%[2D1G1-2]
	\immini{
		Cho hàm số $f(x)$ có đồ thị hàm số $f'(x)$ như hình vẽ.\\
		Hàm số $y=f(\cos x)+x^2-x$ đồng biến trên khoảng
		\choice
		{$(-2 ; 1)$}
		{$(0 ; 1)$}
		{\True $(1 ; 2)$}
		{$(-1 ; 0)$}
	}
	{
		\begin{tikzpicture}[scale=1,>=stealth, font=\footnotesize, line join=round, line cap=round]
			\def\a{-0.5} \def\b{0} \def\c{1.5} \def\d{0} % Hệ số
			\def\xmin{-3} \def\xmax{4}
			\def\ymin{-2} \def\ymax{2} 
			%\draw[color=gray!50,dashed] (\xmin,\ymin) grid (\xmax,\ymax); 
			\draw[->] (\xmin,0)--(\xmax,0) node [below]{$x$};
			\draw[->] (0,\ymin)--(0,\ymax) node [left]{$y$};
			\node at (0,0) [above left]{$O$};
			\node at (3,0) [below right]{$3$};
			\draw[dashed] (-2,0) node[below]{$-2$}--(-2,1) --(0,1) node[above right]{$1$} --(1,1)--(1,0) node[below]{$1$};
			\draw[dashed] (-1,0) node[below right]{$-1$}--(-1,-1) --(0,-1) node[above right]{$-1$} --(2,-1)--(2,0) node[below right]{$2$};
			\clip (\xmin+0.1,\ymin+0.1) rectangle (\xmax-0.5,\ymax-0.1);
			\draw[smooth,samples=300][domain=-2:2] plot(\x,{\a*(\x)^3+\b*(\x)^2+\c*(\x)+\d});
		\end{tikzpicture}
	}
	\loigiai{
		Đặt  $g(x)=f(\cos x)+x^2-x$.\\
		Ta có $g'(x)=-\sin x \cdot f'(\cos x)+2 x-1$\\
		Vì $\cos x \in[-1 ; 1]$ nên từ đồ thị $f'(x)$ ta suy ra $f'(\cos x) \in[-1 ; 1]$.\\
		Do đó $\left|-\sin x \cdot f'(\cos x)\right| \leq 1, ~\forall x \in \mathbb{R}$.\\
		Ta suy ra $g'(x)=\sin x \cdot f'(\cos x)+2 x-1 \geq-1+2 x-1=2 x-2$
		$\Rightarrow g'(x)>0, ~\forall x>1$.\\
		Vậy hàm số đồng biến trên $(1 ; 2)$.
	}
\end{ex}
\begin{ex}[THPT Nguyễn Viết Xuân - 2020]%[2D1G1-2]
	\immini{
		Cho hàm số $f(x)$. Hàm số $y=f'(x)$ có đồ thị như hình vẽ.\\
		Hàm số $g(x)=f\left(3 x^2-1\right)-\dfrac{9}{2}x^4+3 x^2$ đồng biến trên khoảng nào dưới đây?
		\choice
		{\True $\left(-\dfrac{2 \sqrt{3}}{3}; \dfrac{-\sqrt{3}}{3}\right)$}
		{$\left(0 ; \dfrac{2 \sqrt{3}}{3}\right)$}
		{$(1 ; 2)$}
		{$\left(-\dfrac{\sqrt{3}}{3}; \dfrac{\sqrt{3}}{3}\right)$} 
	}
	{
		\begin{tikzpicture}[scale=0.6,>=stealth, font=\footnotesize, line join=round, line cap=round]
			\def\a{0.25} \def\b{0.25} \def\c{-2} \def\d{0} % Hệ số
			\def\xmin{-5} \def\xmax{4}
			\def\ymin{-5} \def\ymax{5} 
			%\draw[color=gray!50,dashed] (\xmin,\ymin) grid (\xmax,\ymax); 
			\draw[->] (\xmin,0)--(\xmax,0) node [below]{$x$};
			\draw[->] (0,\ymin)--(0,\ymax) node [left]{$y$};
			\node at (0,0) [above left]{$O$};
			%\node at (3,0) [below right]{$3$};
			\draw[dashed] (-4,0) node[below left]{$-4$}--(-4,-4) --(0,-4) node[above right]{$-4$};
			\draw[dashed] (3,0) node[below right]{$3$}--(3,3) --(0,3) node[above right]{$3$};
			\clip (\xmin+0.1,\ymin+0.1) rectangle (\xmax-0.5,\ymax-0.1);
			\draw[smooth,samples=300] plot(\x,{\a*(\x)^3+\b*(\x)^2+\c*(\x)+\d});
		\end{tikzpicture}
	}
	
	\loigiai
	{
		TXĐ: $\mathscr{D}=\mathbb{R}$.\\
		Ta có $g'(x)=6 x f'\left(3 x^2-1\right)-18 x^3+6 x=6 x\left[f'\left(3 x^2-1\right)-3 x^2+1\right]$.\\
		$
		g'(x)=0 \Leftrightarrow\hoac{
			&x = 0\\
			&f '( 3 x ^{2}- 1 ) = 3 x ^{2}- 1}
		\Leftrightarrow \hoac{
			&x = 0\\
			&3 x ^{2}- 1 = - 4 \text{~(vô nghiệm)}\\
			&3 x ^{2}- 1 = 0\\
			&3 x ^{2}- 1 = 3}\Leftrightarrow \hoac{&x=0 \\
			&x=\pm \dfrac{\sqrt{3}}{3}\\
			&x=\pm \dfrac{2 \sqrt{3}}{3}.}
		$\\
		Bảng xét dấu
		\begin{center}
			\begin{tikzpicture}
				\tkzTabInit[nocadre,lgt=1.2,espcl=2.2,deltacl=0.6]
				{$x$ /1.2,$f'(x)$ /0.7}
				{$-\infty$,$-\dfrac{2 \sqrt{3}}{3}$,$-\dfrac{ \sqrt{3}}{3}$,$0$,$\dfrac{\sqrt{3}}{3}$,$\dfrac{2 \sqrt{3}}{3}$,$+\infty$}
				\tkzTabLine{,-,$0$,+,$0$,-,$0$,+,$0$,-,$0$,+,}
			\end{tikzpicture}
		\end{center}
		Vậy hàm số đồng biến trong khoảng $\left(-\dfrac{2 \sqrt{3}}{3}; \dfrac{-\sqrt{3}}{3}\right)$.}
\end{ex}
\begin{ex}[Trần Phú - Quảng Ninh - 2020]%[2D1G1-2]
	Cho hàm số $f(x)$ có bảng xét dấu của đạo hàm như sau
	\begin{center}
		\begin{tikzpicture}
			\tkzTabInit[nocadre,lgt=1.2,espcl=2,deltacl=0.6]
			{$x$ /0.6,$f'(x)$ /0.6}
			{$-\infty$,$-4$,$-1$,$2$,$7$,$+\infty$}
			\tkzTabLine{,+,$0$,-,$0$,+,$0$,-,$0$,+,}
		\end{tikzpicture}
	\end{center}
	Hàm số $y=f(2 x+1)+\dfrac{2}{3}x^3-8 x+5$ nghịch biến trên khoảng nào dưới đây?
	\choice
	{$(-\infty ;-2)$}
	{$(1 ;+\infty)$}
	{$(-1 ; 7)$}
	{\True $\left(-1 ; \dfrac{1}{2}\right)$}
	\loigiai{
		Ta có $y'=2 f'(2 x+1)+2 x^2-8$.\\
		Xét $y'\leq 0 \Leftrightarrow 2 f'(2 x+1)+2 x^2-8 \leq 0 \Leftrightarrow f'(2 x+1) \leq 4-x^2$.\\
		Đặt $t=2x+1$, ta có $f'(t) \leq \dfrac{-t^2+2 t+15}{4}$.\\
		Vì $\dfrac{-t^2+2 t+15}{4}\geq 0, \forall t \in[-3 ; 5]$.\\
		Mà $f'(t) \leq 0, \forall t \in[-3 ; 2]$.\\
		Nên $f'(t) \leq \dfrac{-t^2+2 t+15}{4}\Rightarrow t \in[-3 ; 2]$.\\
		Suy ra $-3 \leq 2 x+1 \leq 2 \Leftrightarrow-2 \leq x \leq \dfrac{1}{2}$.}
\end{ex}

\begin{ex}[Chuyên Thái Bình - Lần 3 - 2020]%[2D1G1-2]
	\immini{
		Cho hàm số $y=f(x)$ liên tục trên $\mathbb{R}$ có đồ thị hàm số $y=f'(x)$ cho như hình vẽ.\\
		Hàm số $g(x)=2 f(|x-1|)-x^2+2 x+2020$ đồng biến trên khoảng nào?
		\choice
		{\True $(0 ; 1)$}
		{$(-3 ; 1)$}
		{$(1 ; 3)$}
		{$(-2 ; 0)$}
	}
	{
		\begin{tikzpicture}[scale=0.7,>=stealth, font=\footnotesize, line join=round, line cap=round]
			\def\a{-0.333} \def\b{1} \def\c{1.333} \def\d{-1} % Hệ số
			\def\xmin{-3} \def\xmax{5}
			\def\ymin{-3} \def\ymax{5} 
			%\draw[color=gray!50,dashed] (\xmin,\ymin) grid (\xmax,\ymax); 
			\draw[->] (\xmin,0)--(\xmax,0) node [below]{$x$};
			\draw[->] (0,\ymin)--(0,\ymax) node [left]{$y$};
			\node at (0,0) [above left]{$O$};
			%\node at (3,0) [below right]{$3$};
			\draw[dashed] (-1,0) node[above]{$-1$}--(-1,-1) --(0,-1) node[above right]{$-1$};
			\draw[dashed] (1,0) node[below right]{$1$}--(1,1) --(0,1) node[above right]{$1$};
			\draw[dashed] (3,0) node[below right]{$3$}--(3,3) --(0,3) node[above right]{$3$};
			\clip (\xmin+0.1,\ymin+0.1) rectangle (\xmax-0.5,\ymax-0.1);
			\draw[smooth,samples=300] plot(\x,{\a*(\x)^3+\b*(\x)^2+\c*(\x)+\d});
			\draw[smooth,samples=300] plot(\x,{(\x)});
		\end{tikzpicture}
	}
	\loigiai{
		Với $x>1$, ta có $g(x)=2 f(x-1)-(x-1)^2+2021 \Rightarrow g'(x)=2 f'(x-1)-2(x-1)$.\\
		Hàm số đồng biến $\Leftrightarrow 2 f'(x-1)-2(x-1)>0 \Leftrightarrow f'(x-1)>x-1 \quad(*)$.\\
		Đặt $t=x-1$, khi đó $(*) \Leftrightarrow f'(t)>t \Leftrightarrow\hoac{&1<t<3 \\ &t<-1}\Rightarrow\hoac{&2<x<4 \\ &x<0 ~(\text{loại}).}$\\
		Với $x<1$, ta có $g(x)=2 f(1-x)-(1-x)^2+2021 \Rightarrow g'(x)=-2 f'(1-x)+2(1-x)$.\\
		Hàm số đồng biến $\Leftrightarrow-2 f'(1-x)+2(1-x)>0 \Leftrightarrow f'(1-x)<1-x \quad(* *)$.\\
		Đặt $t=1-x$, khi đó $(* *) \Leftrightarrow f'(t)<t \Leftrightarrow\hoac{&-1<t<1 \\ &t>3}\Rightarrow\hoac{&0<x<2 \\ &x<-2}\Rightarrow\hoac{&0<x<1 \\ &x<-2.}$\\
		Vậy hàm số $g(x)$ đồng biến trên các khoảng $(-\infty ;-2),(0 ; 1),(2 ; 4)$.
	}
\end{ex}
\begin{ex}[Sở Phú Thọ - 2020]%[2D1G1-2]
	\immini{
		Cho hàm số $y=f(x)$ có đồ thị hàm số $f'(x)$ như hình vẽ.\\
		Hàm số $g(x)=f\left(1+e^x\right)+2020$ nghịch biến trên khoảng nào dưới đây?
		\choice
		{$(0 ;+\infty)$}
		{$\left(\dfrac{1}{2}; 1\right)$}
		{\True $\left(0 ; \dfrac{1}{2}\right)$}
		{$(-1 ; 1)$}
	}{
		\begin{tikzpicture}[scale=0.7,>=stealth, font=\footnotesize, line join=round, line cap=round]
			\def\a{1} \def\b{-3} \def\c{0} \def\d{0} % Hệ số
			\def\xmin{-2} \def\xmax{4}
			\def\ymin{-5} \def\ymax{2} 
			%\draw[color=gray!50,dashed] (\xmin,\ymin) grid (\xmax,\ymax); 
			\draw[->] (\xmin,0)--(\xmax,0) node [below]{$x$};
			\draw[->] (0,\ymin)--(0,\ymax) node [left]{$y$};
			\node at (0,0) [above left]{$O$};
			\node at (3,0) [below right]{$3$};
			\draw[dashed] (2,0) node[above]{$2$}--(2,-4) --(0,-4) node[left]{$-4$};
			\clip (\xmin+0.1,\ymin+0.1) rectangle (\xmax-0.5,\ymax-0.1);
			\draw[smooth,samples=300] plot(\x,{\a*(\x)^3+\b*(\x)^2+\c*(\x)+\d});
		\end{tikzpicture}
	}
	\loigiai{
		$g'(x)=e^x f'\left(1+e^x\right)$.\\
		Do $e^x>0, \forall x$ nên $g'(x) \leq 0 \Leftrightarrow f'\left(1+e^x\right) \leq 0 \Leftrightarrow 1+e^x \leq 3 \Leftrightarrow x \leq \ln 2$, dấu bằng xảy ra tại hữu hạn điểm.\\
		Nên $g(x)$ nghịch biến trên $(-\infty ; \ln 2)$.\\
		Vì $\left(0 ; \dfrac{1}{2}\right) \subset (-\infty ; \ln 2)$ nên hàm số đã cho nghịch biến trên $\left(0 ; \dfrac{1}{2}\right)$.
	}
\end{ex}

\begin{ex}%[2D1K1-2]
	[THPT Anh Sơn - Nghệ An - 2020]
	Cho hàm số $y=f(x)$ có bảng xét dấu của đạo hàm như sau.
	\begin{center}
		\begin{tikzpicture}
			\tkzTabInit[nocadre,lgt=1.2,espcl=2,deltacl=0.6]
			{$x$ /0.6,$f'(x)$ /0.6}
			{$-\infty$,$-2$,$-1$,$2$,$4$,$+\infty$}
			\tkzTabLine{,+,$0$,-,$0$,+,$0$,-,$0$,+,}
		\end{tikzpicture}
	\end{center}
	Hàm số $y=-2 f(x)+2019$ nghịch biến trên khoảng nào trong các khoảng dưới đây?
	\choice
	{$(2 ; 4)$}
	{$(-4 ; 2)$}
	{$(-2 ;-1)$}
	{\True $(-1 ; 2)$}
	\loigiai{
		Ta có $y'=-2 f'(x)$.\\
		$
		y'=0 \Leftrightarrow-2 f'(x)=0 \Leftrightarrow\hoac{&
			x=-2 \\
			&x=-1 \\
			&x=2 \\
			&x=4.}$\\
		Từ bảng xét dấu của $f'(x)$ ta có
		\begin{center}
			\begin{tikzpicture}
				\tkzTabInit[nocadre,lgt=1,espcl=2,deltacl=0.6]
				{$x$ /0.6,$y'$ /0.6}
				{$-\infty$,$-2$,$-1$,$2$,$4$,$+\infty$}
				\tkzTabLine{,-,$0$,+,$0$,-,$0$,+,$0$,-,}
			\end{tikzpicture}
		\end{center}
		Từ bảng xét dấu ta có hàm số nghịch biến trên khoảng $(-\infty ;-2),(-1 ; 2)$ và $(4 ;+\infty)$.}
\end{ex}

\begin{ex}[THPT Anh Sơn - Nghệ An - 2020]%[2D1G1-2]
	Cho hàm số $f(x)$ xác định và liên tục trên $\mathbb{R}$ và có đạo hàm $f'(x)$ thỏa mãn $f'(x)=(1-x)(x+2) g(x)+2019$ với $g(x)<0, ~\forall x \in \mathbb{R}$ . Hàm số $y=f(1-x)+2019 x+2020$ nghịch biến trên khoảng nào?
	\choice
	{$(1 ;+\infty)$}
	{$(0 ; 3)$}
	{$(-\infty ; 3)$}
	{\True $(3 ;+\infty)$}
	\loigiai{
		Đặt $h(x)=f(1-x)+2019 x+2020$.\\
		Vì hàm số $f(x)$ xác định trên $\mathbb{R}$ nên hàm số $h(x)$ cũng xác định trên $\mathbb{R}$.\\
		Ta có $h'(x)=-f'(1-x)+2019$.\\
		Do $h'(x)=0$ tại hữu hạn điểm nên để tìm khoảng nghịch biến của hàm số $h(x)$, ta tìm các giá trị của $x$ sao cho $h'(x)<0 \Leftrightarrow-f'(1-x)+2019<0$\\ 
		$\Leftrightarrow f'(1-x)-2019>0 \\
		\Leftrightarrow x(3-x) g(1-x)>0 \Leftrightarrow x(3-x)<0(\text{~do~}g(x)<0, \forall x \in \mathbb{R})$\\
		$\Leftrightarrow\hoac{&
			x<0 \\
			&x>3.}$\\
		Vậy hàm số $y=f(1-x)+2019 x+2020$ nghịch biến trên các khoảng $(-\infty ; 0)$ và $(3 ;+\infty).$}
\end{ex}

\begin{ex}%[2D1G1-2]
	Cho hàm số $y=f(x)$ xác định trên $\mathbb{R}$ và có bảng xét dấu đạo hàm như sau:
	\begin{center}
		\begin{tikzpicture}
			\tkzTabInit[nocadre,lgt=2,espcl=2,deltacl=0.6]
			{$x$ /0.6,$f'(x)$ /0.6}
			{$-\infty$,$-1$,$1$,$4$,$+\infty$}
			\tkzTabLine{,-,$0$,+,$0$,-,$0$,+,}
		\end{tikzpicture}
	\end{center}
	Biết $f(x)>2,~ \forall x \in \mathbb{R}$. Xét hàm số $g(x)=f(3-2 f(x))-x^3+3 x^2-2020$. Khẳng định nào sau đây đúng?
	\choice
	{Hàm số $g(x)$ đồng biến trên khoảng $(-2 ;-1)$}
	{Hàm số $g(x)$ nghịch biến trên khoảng $(0 ; 1)$}
	{Hàm số $g(x)$ đồng biến trên khoảng $(3 ; 4)$}
	{\True Hàm số $g(x)$ nghịch biến trên khoảng $(2 ; 3)$}
	\loigiai{
		Ta có $g'(x)=-2 f'(x) f'(3-2 f(x))-3 x^2+6 x$.\\
		Vì $f(x)>2, ~\forall x \in \mathbb{R}$ nên $3-2 f(x)<-1 ~\forall x \in \mathbb{R}$.\\
		Từ bảng xét dấu $f'(x)$ suy ra $f'(3-2 f(x))<0, ~\forall x \in \mathbb{R}$.\\
		Từ đó ta có bảng xét dấu sau:
		\begin{center}
			\begin{tikzpicture}
				\tkzTabInit[nocadre,lgt=4,espcl=1.7,deltacl=0.6]
				{$x$ /0.7,$-f'(x)f'\left( 3-2f(x)\right) $/0.8,$-3x^2+6x$/0.7}
				{$-\infty$,$-1$,$0$,$1$,$2$,$4$,$+\infty$}
				\tkzTabLine{,-,$0$,+,|,+,$0$,-,|,-,$0$,+,}
				\tkzTabLine{,-,|,-,$0$,+,|,+,$0$,-,|,-,}
			\end{tikzpicture}
		\end{center}
		Từ bảng xét dấu trên, loại trừ đáp án suy ra hàm số $g(x)$ nghịch biến trên khoảng $(2 ; 3)$.}
\end{ex}

\begin{ex}%[2D1G1-2]
	Cho hàm số $f(x)$ có bảng biến thiên như sau:
	\begin{center}
		\begin{tikzpicture}
			\tkzTabInit[nocadre,lgt=1.2,espcl=2.5,deltacl=0.6]
			{$x$ /0.7, $f'(x)$ /0.7, $f(x)$ /2.5}
			{$-\infty$,$1$,$2$,$3$,$4$,$+\infty$}
			\tkzTabLine{,+,$0$,-,$0$,+,$0$,-,$0$,+,}
			\tkzTabVar{-/$-\infty$,+/$3$,-/$1$,+/$2$,-/$0$,+/$+\infty$}
		\end{tikzpicture}
	\end{center}
	Hàm số $y=(f(x))^3-3 .(f(x))^2$ nghịch biến trên khoảng nào dưới đây?
	\choice
	{$(1 ; 2)$}
	{$(3 ; 4)$}
	{$(-\infty ; 1)$}
	{\True $(2 ; 3)$}
	\loigiai{
		Ta có $y'=3 \cdot(f(x))^2 \cdot f'(x)-6 \cdot f(x) \cdot f'(x)=3 f(x) \cdot f'(x) \cdot[f(x)-2]. \\
		y'=0 \Leftrightarrow \hoac{&f(x)=0 \Leftrightarrow x \in\left\{x_1, 4 \mid x_1<1\right\}\\
			&f(x)=2 \Leftrightarrow x \in\left\{x_2, x_3, 3, x_4 \mid x_1<x_2<1<x_3<2 ; 4<x_4\right\}\\
			&f'(x)=0 \Leftrightarrow x \in\{1,2,3,4\}.}$\\
		Lập bảng xét dấu ta có
		\begin{center}
			\begin{tikzpicture}
				\tkzTabInit[nocadre,lgt=2,espcl=1.5,deltacl=0.6]
				{$x$ /0.7,$f(x)$ /0.7,$f(x)-2$ /0.7,$f'(x)$/0.7,$y'$/0.7}
				{$-\infty$,$x_1$,$x_2$,$1$,$x_3$,$2$,$3$,$4$,$x_4$,$+\infty$}
				\tkzTabLine{,-,$0$,+,|,+,|,+,|,+,|,+,$0$,+,|,+,|,+,}
				\tkzTabLine{,-,|,-,$0$,+,$0$,+,$0$,-,|,-,$0$,-,|,-,$0$,+}
				\tkzTabLine{,+,|,+,|,+,$0$,-,|,-,$0$,+,$0$,-,$0$,+,|,+}
				\tkzTabLine{,+,$0$,-,$0$,+,$0$,-,$0$,+,$0$,-,$0$,+,$0$,-,$0$,+}
			\end{tikzpicture}
		\end{center}
		
		Do đó hàm số nghịch biến trên khoảng $(2 ; 3)$.
	}
\end{ex}
\begin{ex}%[2D1G1-2]
	Cho hàm số $y=f(x)$ có đồ thị nằm trên trục hoành và có đạo hàm trên $\mathbb{R}$, bảng xét dấu của biểu thức $f'(x)$ như bảng dưới đây.
	\begin{center}
		\begin{tikzpicture}
			\tkzTabInit[nocadre,lgt=1.2,espcl=2,deltacl=0.6]
			{$x$ /0.6,$f'(x)$ /0.6}
			{$-\infty$,$-2$,$-1$,$3$,$+\infty$}
			\tkzTabLine{,-,$0$,+,$0$,-,$0$,+,}
		\end{tikzpicture}
	\end{center}
	Hàm số $y=g(x)=\dfrac{f\left(x^2-2 x\right)}{f\left(x^2-2 x\right)+1}$ nghịch biến trên khoảng nào dưới đây?
	\choice
	{$(-\infty ; 1)$}
	{$\left(-2 ; \dfrac{5}{2}\right)$}
	{\True $(1 ; 3)$}
	{$(2 ;+\infty)$}
	\loigiai{
		$ g'(x)=\dfrac{\left(x^2-2 x\right)'\cdot f'\left(x^2-2 x\right)}{\left(f\left(x^2-2 x\right)+1\right)^2}=\dfrac{(2 x-2) \cdot f'\left(x^2-2 x\right)}{\left(f\left(x^2-2 x\right)+1\right)^2}. \\
		g'(x)=0 \Leftrightarrow\hoac{
			&2 x - 2 = 0\\
			&f '( x ^{2}- 2 x ) = 0}
		\Leftrightarrow \hoac{&x = 1\\
			&x ^{2}- 2 x = - 2\\
			&x ^{2}- 2 x = - 1\\
			&x ^{2}- 2 x = 3}
		\Leftrightarrow \hoac{&x=1 \\
			&x=-1 \\
			&x=3.}
		$\\
		Ta có bảng xét dấu của $g'(x)$
		\begin{center}
			\begin{tikzpicture}
				\tkzTabInit[nocadre,lgt=1.2,espcl=2,deltacl=0.6]
				{$x$ /0.6,$g'(x)$ /0.6}
				{$-\infty$,$-1$,$1$,$3$,$+\infty$}
				\tkzTabLine{,-,$0$,+,$0$,-,$0$,+,}
			\end{tikzpicture}
		\end{center}
		Dựa vào bảng xét dấu ta có hàm số $y=g(x)$ nghịch biến trên các khoảng $(-\infty ;-1)$ và $(1 ; 3)$.}
\end{ex}
\begin{ex}[Liên trường huyện Quảng Xương - Thanh Hóa - 2021]%[2D1G1-2]
	\immini{
		Cho các hàm số $y=f(x)$; $y=g(x)$ liên tục trên $\mathbb{R}$ và có đồ thị các đạo hàm $f'(x) ; g'(x)$ (đồ thị hàm số $y=g'(x)$ là đường đậm hơn) như hình vẽ.\\
		Hàm số $h(x)=f(x-1)-g(x-1)$ nghịch biến trên khoảng nào dưới đây?
		\choice
		{$\left(\dfrac{1}{2}; 1\right)$}
		{$(1 ;+\infty)$}
		{$(2 ;+\infty)$}
		{\True $\left(-1 ; \dfrac{1}{2}\right)$}
	}
	{
		\begin{tikzpicture}[scale=1,>=stealth, font=\footnotesize, line join=round, line cap=round]
			%\def\a{1} \def\b{-6} \def\c{9} \def\d{1} % Hệ số
			\def\xmin{-4} \def\xmax{3}
			\def\ymin{-2} \def\ymax{4} 
			%\draw[color=gray!50,dashed] (\xmin,\ymin) grid (\xmax,\ymax); 
			\draw[->] (\xmin,0)--(\xmax,0) node [below]{$x$};
			\draw[->] (0,\ymin)--(0,\ymax) node [left]{$y$};
			\node at (0,0) [above left]{$O$};
			\node at (1,3) [below left]{$f'(x)$};
			\node at (1.5,3) [below right]{$g'(x)$};
			\draw[dashed] (-2,0) node[above right]{$-2$}--(-2,1);
			\draw[dashed] (1,0) node[below]{$1$}--(1,1);
			\draw[dashed] (-0.5,0) node[below]{$-0{,}5$}--(-0.5,2.125);
			\clip (\xmin+0.1,\ymin+0.1) rectangle (\xmax-0.5,\ymax-0.1);
			\draw[smooth,samples=300][domain=-3:2] plot(\x,{2*(\x)^4+4*(\x)^3-2*(\x)^2-4*(\x)+1});
			\draw[smooth,samples=300,line width=1.2pt] plot(\x,{(\x)^3+(\x)^2-2*(\x)+1});
		\end{tikzpicture}
	}
	
	\loigiai{
		Ta có: $h'(x)=f'(x-1)-g'(x-1)$.\\
		Dựa vào hình vẽ ta có hàm số $h(x)$ nghịch biến\\
		$\Leftrightarrow h'(x)<0 \Leftrightarrow f'(x-1)<g'(x-1)$\\
		$
		\Leftrightarrow\hoac{&- 2 < x - 1 < - \dfrac{1}{2}\\
			&0 < x - 1 < 1}
		\Leftrightarrow \hoac{
			&-1<x<\dfrac{1}{2}\\
			&1<x<2.}$\\
		Do đó hàm số $h(x)$ nghịch biến trên các khoảng $\left(-1 ; \dfrac{1}{2}\right)$ và $(1 ; 2)$.
	}
\end{ex}
\begin{ex}[THPT Quế Võ 1 - Bắc Ninh - 2021] %[2D1G1-2]
	\immini{
		Cho ba hàm số $y=f(x), y=g(x), y=h(x)$. Đồ thị của ba hàm số $y=f'(x), y=g'(x), y=h'(x)$ được cho như hình vẽ.\\
		Hàm số $k(x)=f(x+7)+g(5 x+1)-h\left(4 x+\dfrac{3}{2}\right)$ đồng biến trên khoảng nào dưới đây?
		\choice
		{$\left(-\dfrac{5}{8}; 0\right)$}
		{$\left(\dfrac{5}{8};+\infty\right)$}
		{\True $\left(\dfrac{3}{8}; 1\right)$}
		{$\left(-\dfrac{3}{8}; 1\right)$}
	}
	{
		\begin{tikzpicture}[scale=0.25,>=stealth, font=\footnotesize, line join=round, line cap=round]
			\def\a{-.078} \def\b{1.25} \def\c{0} % Hệ số
			\def\xmin{-4} \def\xmax{25}
			\def\ymin{-8} \def\ymax{18}
			
			%\draw[color=gray!50,dashed] (\xmin,\ymin) grid (\xmax,\ymax);
			
			\draw[->] (\xmin,0)--(\xmax,0) node [below]{$x$};
			\draw[->] (0,\ymin)--(0,\ymax) node [left]{$y$};
			\node at (20,14) [below right]{$y=g'(x)$};
			\node at (18,-2) [below left]{$y=h'(x)$};
			\node at (16,5) [below right]{$y=f'(x)$};
			\node at (0,0) [below left]{$O$};
			\draw[dashed] (3,0) node[below]{$3$}--(3,10)--(0,10) node[left]{$10$};
			\draw[dashed] (8,0) node[below]{$8$}--(8,5)--(0,5) node[left]{$5$};
			\draw[dashed] (4,0) node[below]{$4$}--(4,2)--(0,2) node[left]{$2$};
			\clip (\xmin+0.1,\ymin+0.1) rectangle (\xmax-0.5,\ymax-0.1);
			\draw[smooth,samples=300,domain=-2:18] plot(\x,{\a*(\x)^2+\b*(\x)+\c});
			%\draw[smooth,samples=300,domain=-2:25] plot(\x,{0.02*(\x)^3-0.6*(\x)^2+5.16*(\x)});
			\draw[line width=1.2pt] (-2,5)..controls (1.7,1.5) and (4.5,1.6)..(7,2.6);
			\draw[line width=1.2pt] (7,2.6)..controls (9,3.5) and (12,5)..(20,13);
			\draw (-0.5,-2) -- (0,0)--(3,10).. controls +(65:1) and + (-190:1)..(6,15).. controls +(0:1) and + (-180:1)..(14,-1).. controls +(0:1) and + (+80:1)..(19,16);
			
		\end{tikzpicture}
	}
	\loigiai{
		Ta có $k'(x)=f'(x+7)+5 g'(5 x+1)-4 h'\left(4 x+\dfrac{3}{2}\right)$.\\
		Khi $x \in \left( \dfrac{3}{8};1\right)$ thì $\heva{&7{,}375<x+7<8\\&2{,}875<5x+1<6\\&3<4x+\dfrac{4}{3}<5{,}5}\Leftrightarrow \heva{&f'(x+7)>10\\&g'(5x+1)>2 \Rightarrow 5g'(5x+1)>10  \\&h'\left( 4x+\dfrac{3}{2}\right)<5 \Rightarrow -4h'\left( 4x+\dfrac{3}{2}\right) >-20}.$\\
		Do đó $k'(x)=f'(x+7)+5g'(5x+1)-4h'\left( 4x+\dfrac{3}{2}\right)>0$.\\
		Hàm số $k(x)=f(x+7)+g(5 x+1)-h\left(4 x+\dfrac{3}{2}\right)$ đồng biến trên $\left(\dfrac{3}{8}; 1\right)$.
	}
\end{ex}
\begin{ex}[THPT Thanh Chương 1 - Nghệ An- 2021] %[2D1G1-2]
	Cho hàm số $y=f(x)$ liên tục trên $\mathbb{R}$ có bảng xét dấu đạo hàm như sau
	\begin{center}
		\begin{tikzpicture}
			\tkzTabInit[nocadre,lgt=1.2,espcl=2,deltacl=0.6]
			{$x$ /0.6,$f'(x)$ /0.6}
			{$-\infty$,$1$,$2$,$3$,$4$,$+\infty$}
			\tkzTabLine{,-,$0$,+,$0$,+,$0$,-,$0$,+,}
		\end{tikzpicture}
	\end{center}
	Hàm số $y=3f(2x-1)-4x^3+15x^2-18x+1$ đồng biến trên khoảng nào dưới đây?
	\choice
	{$\left(3;+\infty\right)$}
	{\True $\left(1;\dfrac{3}{2}\right)$}
	{$\left(\dfrac{5}{2}; 3\right)$}
	{$\left(2;\dfrac{5}{2}\right)$}
	\loigiai{
		Ta có $y'=6f'(2x-1)-12x^2+30x-18=6\left[f'(2x-1)-2x^2+5x-3\right] $.\\
		Có $f'(2x-1)=0 \Leftrightarrow \hoac{&2x-1=1\\&2x-1=2\\&2x-1=3\\&2x-1=4} \Leftrightarrow \hoac{&x=1\\&x=\dfrac{3}{2}\\&x=2\\&x=\dfrac{5}{2}.}$
		Ta có bảng xét dấu sau
		\begin{center}
			\begin{tikzpicture}
				\tkzTabInit[nocadre,lgt=3.0,espcl=1.5,deltacl=0.6]
				{$x$ /1.0,$f(x)$ /0.6,$f'(2x-1)$ /0.6,$-2x^2+5x-3$/0.6,$g'(x)$/0.6}
				{$-\infty$,$1$,$\dfrac{3}{2}$,$2$,$\dfrac{5}{2}$,$3$,$4$,$+\infty$}
				\tkzTabLine{,-,$0$,+,|,+,$0$,+,|,+,$0$,-,$0$,+,}
				\tkzTabLine{,-,$0$,+,$0$,+,$0$,-,$0$,+,|,+,|,+,}
				\tkzTabLine{,-,$0$,+,$0$,-,|,-,|,-,|,-,|,-,}
				\tkzTabLine{,-,$0$,+,$0$,,?,,|,,?,?,,?,}
			\end{tikzpicture}
		\end{center}
		Dựa vào bảng xét dấu trên, ta kết luận hàm số đã cho đồng biến trên khoảng $\left( 1; \dfrac{3}{2}\right).$
	}
\end{ex}


\begin{ex}%[2D2G4-3] %Câu 27 
	[THPT Hoàng Hoa Thám-Đà Nẵng-2021]
	Cho hàm số $f(x)$ có bảng xét dấu của $f'(x)$ như sau:\\
	\begin{center}
		\begin{tikzpicture}
			\tkzTabInit[lgt=1.2,espcl=2.3]
			{$x$/0.7, $f'(x)$ /.8} % first column
			{$-\infty$,$-3$,$1$, $2$, $+\infty$} % first row
			\tkzTabLine { ,+,0,-,0,+,0,+ }
		\end{tikzpicture}
	\end{center}	
	Hàm số $y=f\left(2-e^x\right)-\dfrac{1}{3}{e^{3x}}+3e^{2x}-5e^x+1$ đồng biến trên khoảng nào dưới đây?
	\choice
	{$\left(0;\dfrac{3}{2}\right)$}
	{$\left(1;3\right)$}
	{\True $\left(-3;0\right)$}
	{$\left(-4;-3\right)$}
	\loigiai{
		Ta có $y'=-e^x.f'\left(2-e^x\right)-e^{3x}+6e^{2x}-5e^x=e^x\left[-f'\left(2-e^x\right)-e^{2x}+6e^x-5\right]$ .\\
		Đặt $t=2-e^x$, ta được\\
		$y'=\left(2-t\right)\left[-f'(t)-\left(2-t\right)^2+6\left(2-t\right)-5\right]=\left(2-t\right)\left[-f'(t)-t^2-2t+3\right]$ .\\
		$y'=0\Leftrightarrow\left(2-t\right)\left[-f'(t)-t^2-2t+3\right]=0\Leftrightarrow
		\hoac{
			& t=2\\ 
			& f'(t)=-t^2-2t+3.}$\\
		Hàm số $g(x)=-x^2-2x+3$ là parabol có trục đối xứng $x=-1$ và cắt trục hoành tại 2 điểm có hoành độ 
		$\hoac{
			& x=1\\ 
			& x=-3
		}$. Suy ra $f'(t)=-t^2-2t+3\Leftrightarrow \hoac{
			& t=1\\ 
			& t=-3. }$\\
		Bảng xét dấu\\
		\begin{center}
			\begin{tikzpicture}
				\tkzTabInit[lgt=3.9,espcl=2,nocadre]
				{$t$/0.7, $2-t$ /0.8, $-f'(t)-t^2-2t+3$ /0.8, $y'$ /0.8} % first column
				{$-\infty$,$-3$,$1$,$2$,$+\infty$} % first row
				\tkzTabLine { ,+,|,+,|,+,z,-, } % second row
				\tkzTabLine {,-,0,+,0,-,|,-,} % third row
				\tkzTabLine {,-,0,+,0,-,0,+,} % last row
			\end{tikzpicture}
		\end{center}
		Dựa vào bảng xét dấu $y'>0,\forall x\in\left(-3;0\right)$.}
\end{ex}


\begin{ex}%[2D1G1-2]%Câu 28 
	[Sở Lạng Sơn 2022] Cho hàm số $f(x)$ có bảng biến thiên như sau:\\
	\begin{center}
		\begin{tikzpicture}
			\tkzTabInit[espcl=2.5,lgt=1,nocadre]
			{$x$/0.7,$y'$/0.7,$y$/3.5}
			{$-\infty$,$1$,$2$,$3$,$4$,$+\infty$}
			\tkzTabLine{,+,0,-,0,+,0,-,0,+,}
			\node (0) at ($(N12)+(0,-3)$) {$-\infty$};
			\node (1) at ($(N22)+(0,-.5)$) {$3$};
			\node (2) at ($(N32)+(0,-1.7)$) {$1$};
			\node (3) at ($(N42)+(0,-0.7)$) {$2$};
			\node (4) at ($(N52)+(0,-2.3)$) {$0$};
			\node (5) at ($(N62)+(0,-.3)$) {$+\infty$};
			%				\node (8) at ($(N42)+(0,-.5)$) {};
			%				\coordinate (9) at ($(N42)!.6!(N53)+ (-0.5,0)$);
			%				\coordinate (6) at ($(T12)!.6!(T13)$);
			%				\coordinate (7) at ($(T22)!.6!(T23)$);
			\draw[-stealth] (0)--(1);
			\draw[-stealth] (1)--(2);
			\draw[-stealth] (2)--(3);
			\draw[-stealth] (1)--(2);
			\draw[-stealth] (3)--(4);
			\draw[-stealth] (4)--(5);
			%				\draw[->,red] (5)--(8);
			%				\draw[->,red] (8)--(9);
			%				\draw[blue,dashed](6)--(7)node[above left]{$y=0$};
		\end{tikzpicture}		
	\end{center}
	Hàm số $y=\left[f(x)\right]^3-3\left[f(x)\right]^2$ đồng biến trên khoảng nào dưới đây?
	\choice
	{$\left(-\infty\,;1\right)$}
	{$\left(1\,;2\right)$}
	{\True $\left(3\,;4\right)$}
	{$\left(2\,;3\right)$}
	\loigiai{
		Ta có $y'=3f'(x)\left[f^2(x)-2f(x)\right]$. 
		Phương trình $y'=0\Leftrightarrow \hoac{
			&{f}'(x)=0\\ 
			& f(x)=0\\ 
			& f(x)=2.
		}$
		\begin{center}
			\begin{tikzpicture}
				\tkzTabInit[espcl=2.5,lgt=1.5]
				{$x$/0.7,$y'$/0.7,$y$/3.5}
				{$-\infty$,$1$,$2$,$3$,$4$,$+\infty$}
				\tkzTabLine{,+,0,-,0,+,0,-,0,+,}
				\node (0) at ($(N12)+(0,-3)$) {$-\infty$};
				\node (1) at ($(N22)+(0,-.3)$) {$3$};
				\node (2) at ($(N32)+(0,-1.7)$) {$1$};
				\node (3) at ($(N42)+(0,-0.8)$) {$2$};
				\node (4) at ($(N52)+(0,-2.3)$) {$0$};
				\node (5) at ($(N62)+(0,-.3)$) {$+\infty$};
				\node (a) at ($(N11)+(0.65,0.35)$) {$a$};
				\node (b) at ($(N11)+(2.0,0.4)$) {$b$};
				\node (c) at ($(N11)+(3.38,0.35)$) {$c$};
				\node (d) at ($(N11)+(11.85,0.4)$) {$d$};
				\node (6) at ($(N12)+(0,-0.8)$) {};
				\node (7) at ($(N62)+(0,-0.8)$) {};
				\node (8) at ($(N12)+(0,-2.3)$) {};
				\node (9) at ($(N62)+(0,-2.3)$) {};
				%				\node (8) at ($(N42)+(0,-.5)$) {};
				%				\coordinate (9) at ($(N42)!.6!(N53)+ (-0.5,0)$);
				\coordinate (A) at ($(0)!.25!(1)$);
				\coordinate (B) at ($(0)!.8!(1)$);
				\coordinate (C) at ($(1)!.35!(2)$);
				\coordinate (D) at ($(4)!.75!(5)$);
				%				\coordinate (7) at ($(T22)!.6!(T23)$);
				\draw[->] (0)--(1);
				\draw[->] (1)--(2);
				\draw[->] (2)--(3);
				\draw[->] (1)--(2);
				\draw[->] (3)--(4);
				\draw[->] (4)--(5);
				%				\draw[->,red] (5)--(8);
				%				\draw[->,red] (8)--(9);
				\draw[blue,dashed](6)--(7)node[below]{$y=2$} (a)--(A) (b)--(B) (c)--(C) (d)--(D);
				\draw[blue,dashed](8)--(9)node[below left]{$y=0$};
			\end{tikzpicture}		
		\end{center}
		Dựa vào bảng biến thiên, ta thấy $f'(x)=0\Leftrightarrow x\in \{ 1\,;2\,;3\,;4 \}$;\\
		$f(x)=0\Leftrightarrow x=a<1$ hoặc $x=4$;\\
		$f(x)=2\Leftrightarrow \hoac{
			& x=b\,\,\left(a<b<1\right)\\ 
			& x=c\in\left(1\,;2\right)\\ 
			& x=3\\ 
			& x=d>4.
		}$ \\
		Ta lập được bảng xét dấu của $y'$ 
		\begin{center}
			\begin{tikzpicture}
				\tkzTabInit[lgt=1.2,espcl=1.5,nocadre]
				{$x$/1, $f(x)$ /.8} % first column
				{$-\infty$,$a$, $b$, $1$,$c$, $2$,$3$, $4$, $d$, $+\infty$} % first row
				\tkzTabLine { ,+,z,-,z,+,z,-,z,+,z,-,z,+,z,-,z,+, } % second row
				%				\tkzTabLine {,-,z,+,t,+,} % third row
				%				\tkzTabLine {,+,d,-,z,+,} % last row
			\end{tikzpicture}
		\end{center}
		Từ bảng xét dấu, ta thấy hàm số đồng biến trên các khoảng \\
		$\left(-\infty;a\right)$, $\left(b;1\right)$, $\left(c;2\right)$, $\left(3;4\right)$ và $(d;+\infty)$.
	}
\end{ex}

\begin{ex}%[2D1G1-2]%Câu 29 
	[THPT Bùi Thị Xuân – Huế-2022] 
	\immini{
		Cho hàm số $y=f(x)$ là hàm đa thức bậc bốn. Đồ thị hàm số $f'(x+2)$ được cho trong hình vẽ bên. Hàm số 
		$$g(x)=4 f\left(x^2\right)-x^6+5 x^4-4 x^2+1$$
		đồng biến trên khoảng nào dưới đây?
		\choice
		{$(-4 ;-3)$}
		{\True $(2 ;+\infty)$}
		{$(-\sqrt{2};\sqrt{2})$}
		{$(-2 ;-1)$}}{
		\begin{tikzpicture}[scale=0.6,font=\footnotesize, line join=round, line cap=round, >=stealth] %Đường cong bậc 3
			\draw[thick, ->] (-5.3,0)--(5,0);
			\draw[thick, ->] (0,-3.5)--(0,7);
			\draw (5.2,0) node[below] {$x$};
			\draw (0,7.1) node[left]{$y$};
			\draw (0,0) node[below left]{$0$};
			\draw[fill] (-2,0) circle (0.5pt)node[below left]{$ -2 $};
			\draw[fill] (2,0) circle (0.5pt)node[below]{$ 2$};
			\draw[fill] (0,3) circle (0.5pt)node[left]{$ 3 $};
			\draw[fill] (0,1) circle (0.5pt)node[right]{$ 1 $};
			\draw[fill] (0,-1) circle (0.5pt)node[right]{$ -1 $};
			\draw[dashed] (-2,0)--(-2,1) --(0,1); 
			\draw[dashed](2,0)--(2,3)--(0,3);
			\draw[line width=1.2pt,smooth,samples=100,domain=-2.8:4.5] plot(\x,{-0.271*(\x)^3+0.75*(\x)^2+1.583*\x-1});
		\end{tikzpicture}		
	}
	\loigiai{
		$\begin{aligned}
			& g(x)=4f\left(x^2\right)-x^6+5x^4-4x^2+1\Rightarrow g' (x)=8xf'\left(x^2\right)-6x^5+20x^3-8x.\\ 
			& g' (x)=0\Leftrightarrow 8xf'\left(x^2\right)-6x^5+20x^3-8x=0 \\
			& \Leftrightarrow 2x\left[4f'\left(x^2\right)-3x^4+10x^2-4\right]=0\\ 
			&\Leftrightarrow 		\hoac{ 			& 2x=0\\ 
				& 4f'(x^2)-3x^4+10x^2-4=0
			}
			\Leftrightarrow \hoac{	& x=0\\ 
				& f'\left(x^2\right)=\dfrac{3}{4}{x^4}-\dfrac{5}{2}{x^2}+1.}
		\end{aligned}$\\ 
		Xét
		$f'\left(x^2\right)=\dfrac{3}{4}x^4-\dfrac{5}{2}x^2+1$. Đặt $x^2=t+2$, ta có\\
		$ f' (t+2)=\dfrac{3}{4}{(t+2)^2}-\dfrac{5}{2}(t+2)+1=\dfrac{3}{4}\left(t^2+4t+4\right)-\dfrac{5}{2}(t+2)-1=\dfrac{3}{4}{t^2}+\dfrac{1}{2}t-1$\\
		Khi đó số nghiệm của phương trình chính là số giao điểm của đồ thị hàm số $y=f' (t+2)$ và\\
		$ y=\dfrac{3}{4}{t^2}+\dfrac{1}{2}t-1$\\
		Ta có đồ thị 
		\begin{center}
			\begin{tikzpicture}[scale=0.6,font=\footnotesize, line join=round, line cap=round, >=stealth] %Đường cong bậc 3
				\draw[thick, ->] (-5.3,0)--(5,0);
				\draw[thick, ->] (0,-3.5)--(0,7);
				\draw (5.2,0) node[below] {$x$};
				\draw (0,7.1) node[left]{$y$};
				\draw (0,0) node[below left]{$0$};
				\draw[fill] (-2,0) circle (0.5pt)node[below left]{$ -2 $};
				\draw[fill] (2,0) circle (0.5pt)node[below]{$ 2$};
				\draw[fill] (0,3) circle (0.5pt)node[left]{$ 3 $};
				\draw[fill] (0,1) circle (0.5pt)node[right]{$ 1 $};
				\draw[fill] (0,-1) circle (0.5pt)node[right]{$ -1 $};
				\draw[dashed] (-2,0)--(-2,1) --(0,1); 
				\draw[dashed](2,0)--(2,3)--(0,3);
				\draw[line width=1.2pt,smooth,samples=100,domain=-2.8:4.5] plot(\x,{-0.271*(\x)^3+0.75*(\x)^2+1.583*\x-1});		
				\draw[line width=1.2pt,smooth,samples=100,domain=-3.3:2.8] plot(\x,{0.75*(\x)^2+0.5*\x-1});
			\end{tikzpicture}
		\end{center}
		Dựa vào đồ thị ta có $f' (t+2)=\dfrac{3}{4}t^2+\dfrac{1}{2}t-1\Leftrightarrow \hoac{& t=-2\\ & t=0\\ & t=2} \Leftrightarrow\hoac{& x+2=-2\\ & x+2=0\\ & x+2=2} \Leftrightarrow \hoac{& x=-4\\ & x=-2\\ & x=0.}$\\
		Ta có bảng xét dấu $g' (x)$ như sau
		\begin{center}
			\begin{tikzpicture}
				\tkzTabInit[lgt=1.2,espcl=2,nocadre]
				{$x$/0.7, $f(x)$ /.7}
				{$-\infty$, $-4$,$-2$, $0$, $+\infty$} % first row
				\tkzTabLine { ,-,z,+,z,-,z,+, }
			\end{tikzpicture}
		\end{center}
		Vậy hàm số $g(x)=4 f\left(x^2\right)-x^6+5 x^4-4 x^2+1$ đồng biến trên khoảng $(2 ;+\infty)$.}
\end{ex}

\begin{ex}%[2D1G1-2]%Câu 30
	[Chuyên Bắc Ninh 2022] 
	\immini{
		Cho hàm số $ y=f(x)$ liên tục trên $\mathbb{R}$ có đồ thị hàm số $ y=f'(x)$ có đồ thị như hình vẽ bên.
		Hàm số $g(x)=2f\left(\left| x-1\right|\right)-x^2+2x+2020$ đồng biến trên khoảng nào
		\choice
		{$\left(-2;0\right)$}
		{$\left(-3;1\right)$}
		{$\left(1\,;3\right)$}
		{\True $\left(0\,;\,1\right)$}}{
		\begin{tikzpicture}[scale=0.6,font=\footnotesize, line join=round, line cap=round, >=stealth] %Đường cong bậc 3
			\draw[thick, ->] (-3.3,0)--(5,0);
			\draw[thick, ->] (0,-3.0)--(0,5.5);
			\draw (5.2,0) node[below] {$x$};
			\draw (0,5.8) node[left]{$y$};
			\draw (0,0) node[below left]{$0$};
			\draw[fill] (-1,0) circle (0.5pt)node[above]{$ -1 $};
			\draw[fill] (1,0) circle (0.5pt)node[below]{$ 1$};
			\draw[fill] (0,1) circle (0.5pt)node[left]{$ 1 $};
			\draw[fill] (0,-1) circle (0.5pt)node[right]{$ -1 $};
			\draw[fill] (0,3) circle (0.5pt)node[left]{$ 3 $};
			\draw[fill] (3,0) circle (0.5pt)node[below]{$ 3 $};
			\draw[dashed] (-1,0)--(-1,-1) --(0,-1); 
			\draw[dashed](1,0)--(1,1)--(0,1);
			\draw[dashed](3,0)--(3,3)--(0,3);
			\draw[line width=1.2pt,smooth,samples=100,domain=-2.2:4.3] plot(\x,{-0.333*(\x)^3+1*(\x)^2+1.333*\x-1});		
			%\draw[line width=1.2pt,smooth,samples=100,domain=-3.3:2.8] plot(\x,{0.75*(\x)^2+0.5*\x-1});
		\end{tikzpicture}	
	}
	\loigiai{
		Ta có $g(x)=2f\left(\left| x-1\right|\right)-x^2+2x+2020\Leftrightarrow g(x)=2f\left(\left| x-1\right|\right)-\left(x-1\right)^2+2021$.\\
		Xét hàm số $ k\left(x-1\right)=2f\left(x-1\right)-\left(x-1\right)^2+2021$.\\
		Đặt $ t=x-1$\\
		Xét hàm số $ h(t)=2f(t)-t^2+2021$ $\Rightarrow{h}'(t)=2f'(t)-2t$.\\
		Kẻ đường $ y=x$ như hình vẽ.
		\begin{center}
			\begin{tikzpicture}[scale=0.6,font=\footnotesize, line join=round, line cap=round, >=stealth] %Đường cong bậc 3
				\draw[thick, ->] (-3.3,0)--(5,0);
				\draw[thick, ->] (0,-3.0)--(0,5.5);
				\draw (5.2,0) node[below] {$x$};
				\draw (0,5.8) node[left]{$y$};
				%	\draw (0,0) node[below left]{$0$};
				\draw[fill] (-1,0) circle (0.5pt)node[above]{$ -1 $};
				\draw[fill] (1,0) circle (0.5pt)node[below]{$ 1$};
				\draw[fill] (0,1) circle (0.5pt)node[left]{$ 1 $};
				\draw[fill] (0,-1) circle (0.5pt)node[right]{$ -1 $};
				\draw[fill] (0,3) circle (0.5pt)node[left]{$ 3 $};
				\draw[fill] (3,0) circle (0.5pt)node[below]{$ 3 $};
				\draw[dashed] (-1,0)--(-1,-1) --(0,-1); 
				\draw[dashed](1,0)--(1,1)--(0,1);
				\draw[dashed](3,0)--(3,3)--(0,3);
				\draw[line width=1.2pt,smooth,samples=100,domain=-2.2:4.3] plot(\x,{-0.333*(\x)^3+1*(\x)^2+1.333*\x-1});		
				%\draw[line width=1.2pt,smooth,samples=100,domain=-3.3:2.8] plot(\x,{0.75*(\x)^2+0.5*\x-1});
				\draw[line width=1.2pt,smooth,samples=100](-2,-2)--(4,4);
			\end{tikzpicture}
		\end{center}
		Khi đó $h'(t)>0\Leftrightarrow{f}'(t)-t>0\Leftrightarrow{f}'(t)>t$$\Leftrightarrow \hoac{
			& t<-1\\ 
			& 1<t<3.
		}$\\
		Do đó $k'\left(x-1\right)>0\Leftrightarrow \hoac{
			& x-1<-1\\ 
			& 1<x-1<3} \Leftrightarrow \hoac{
			& x<0\\ 
			& 2<x<4.}$\\
		Ta có bảng biến thiên của hàm số $ k\left(x-1\right)=2f\left(x-1\right)-\left(x-1\right)^2+2021$.
		\begin{center}
			\begin{tikzpicture}
				\tkzTabInit[lgt=1.8,espcl=2.3]
				{$x$ /1.2, $k'(x-1)$ /1.2,$k(x-1)$ /2}
				{$-\infty$ , $0$,$2$,$4$, $+\infty$}
				\tkzTabLine{,+,0,-,0,+,0,-,}
				\tkzTabVar{-/$ $ ,+/$ $, -/$ $,+/$ $,-/$ $}
			\end{tikzpicture}
		\end{center}
		Khi đó, ta có bảng biến thiên của $g(x)=2f\left(\left| x-1\right|\right)-\left(x-1\right)^2+2021$ bằng cách lấy đối xứng qua đường thẳng $ x=1$ như sau\\
		\begin{center}
			\begin{tikzpicture}
				\tkzTabInit[lgt=1.2,espcl=2.5,nocadre]
				{$x$ /0.7, $g'(x)$ /0.7,$g(x)$ /2.5}
				{$-\infty$ ,$-2$, $0$,$1$,$2$,$4$, $+\infty$}
				\tkzTabLine{,+,0,-,0,+,0,-,0,+,0,-,}
				\tkzTabVar{-/$ $ ,+/$ $, -/$ $,+/$ $,-/$ $,+/ $ $,-/$ $}
			\end{tikzpicture}
		\end{center}
		Vậy hàm số đồng biến trên $\left(0;1\right)$.}
\end{ex}

\begin{ex}%[2D1G1-2]%Câu 31
	[Chuyên Thái Bình 2022] 
	\immini{
		Cho hàm số $f(x)=a{x^4}+b{x^3}+c{x^2}+dx+a$ có đồ thị hàm số $y=f'(x)$ như hình vẽ bên. Hàm số $y=g(x)=f\left(1-2x\right)f\left(2-x\right)$ đồng biến trên khoảng nào dưới đây?
		\choice
		{$\left(\dfrac{1}{2};\dfrac{3}{2}\right)$}
		{$\left(-\infty ;0\right)$}
		{$\left(0;2\right)$}
		{\True $\left(3;+\infty\right)$}}{
		\begin{tikzpicture}[scale=0.9,font=\footnotesize, line join=round, line cap=round, >=stealth] %Đường cong bậc 3
			\draw[thick, ->] (-2.5,0)--(2.5,0);
			\draw[thick, ->] (0,-2.8)--(0,2.8);
			\draw (2.6,0) node[below] {$x$};
			\draw (0,2.9) node[left]{$y$};
			\draw (0,0) node[below left]{$0$};
			\draw[fill] (-1,0) circle (0.5pt)node[below left]{$ -1 $};
			\draw[fill] (1,0) circle (0.5pt)node[below right]{$ 1$};
			%			\draw[dashed] (-1,0)--(-1,-1) --(0,-1); 
			%			\draw[dashed](1,0)--(1,1)--(0,1);
			%			\draw[dashed](3,0)--(3,3)--(0,3);
			\draw[line width=1.2pt,smooth,samples=100,domain=-1.3:1.3] plot(\x,{3*(\x)^3-3*\x});		
			%\draw[line width=1.2pt,smooth,samples=100,domain=-3.3:2.8] plot(\x,{0.75*(\x)^2+0.5*\x-1});
		\end{tikzpicture}	
	}
	\loigiai{
		Ta có $f'(x)=4a{x^3}+3b{x^2}+2cx+d$, theo đồ thị thì đa thức $f'(x)$ có ba nghiệm phân biệt là $-1,0,1$ nên $f'(x)=4ax\left(x+1\right)\left(x-1\right)=4a{x^3}-4ax\Rightarrow f(x)=a{x^4}-2a{x^2}+a=a{\left(x^2-1\right)^2}$.\\
		Dựa vào đồ thị hàm số $y=f'(x)$ ta có $a>0$ nên $f(x)>0,\forall x\in\mathbb{R}\setminus\left\{\pm 1\right\}$.\\
		$g'(x)=\left[f\left(1-2x\right)\right]'f\left(2-x\right)+f\left(1-2x\right)\left[f\left(2-x\right)\right]'=-2f'\left(1-2x\right)f\left(2-x\right)-f\left(1-2x\right)f'\left(2-x\right)$. Xét $x\in\left(\dfrac{1}{2};\dfrac{3}{2}\right)\Rightarrow
		\heva{		
			& 1-2x\in\left(-2;0\right)\\ 
			& 2-x\in\left(\dfrac{1}{2};\dfrac{3}{2}\right)}$, dấu của $f'(x)$ không cố định trên $\left(\dfrac{1}{2};\dfrac{3}{2}\right)$ nên ta không kết luận được tính đơn điệu của hàm số $g(x)$ trên $\left(\dfrac{1}{2};\dfrac{3}{2}\right)$.\\
		Xét $x\in\left(-\infty ;0\right)\Rightarrow
		\heva{
			& 1-2x\in\left(1;+\infty\right)\\ 
			& 2-x\in\left(2;+\infty\right)} 
		\Rightarrow \heva{
			& f'\left(1-2x\right)>0\\ 
			& f'\left(2-x\right)>0} \Rightarrow g'(x)<0$.\\
		Do đó, hàm số $g(x)$ nghịch biến trên $\left(-\infty ;0\right)$.\\
		$x\in\left(0;2\right)\Rightarrow \heva{
			& 1-2x\in\left(-3;1\right)\\ 
			& 2-x\in\left(0;2\right)}$, dấu của $f'(x)$ không cố định trên $\left(-3;1\right)$ và $\left(0;2\right)$ nên ta không kết luận được tính đơn điệu của hàm số $g(x)$ trên $\left(\dfrac{1}{2};\dfrac{3}{2}\right)$.\\
		Xét $x\in\left(3;+\infty\right)\Rightarrow \heva{
			& 1-2x\in\left(-\infty ;-5\right)\\ 
			& 2-x\in\left(-\infty ;-1\right)} \Rightarrow \heva{
			& f'\left(1-2x\right)<0\\ 
			& f'\left(2-x\right)<0} \Rightarrow g'(x)>0$. \\
		Do đó, hàm số $g(x)$ đồng biến trên $\left(3;+\infty\right)$.}
\end{ex}

\begin{dang}{Bài toán hàm ẩn, hàm hợp liên quan đến tham số và một số bài toán khác}
\end{dang}

\begin{ex}%[2D1G1-3]%Câu 1
	[Chuyên Lê Hồng Phong Nam Định 2019]
	\immini{
		Cho hàm số $ y=f(x)$ có đạo hàm liên tục trên $\mathbb{R}$. Biết hàm số $ y=f'(x)$ có đồ thị như hình vẽ. Gọi $ S$ là tập hợp các giá trị nguyên $ m\in\left[-5\,;\,\text{5}\right]$ để hàm số $ g(x)=f\left(x+m\right)$ nghịch biến trên khoảng $\left(1\,;\,2\right)$. Hỏi $S$ có bao nhiêu phần tử?
		\choice
		{$ 4$}
		{$ 3$}
		{$ 6$}
		{\True $ 5$}}{
		\begin{tikzpicture}[scale=0.9,font=\footnotesize, line join=round, line cap=round, >=stealth] %Đường cong bậc 3
			\draw[thick, ->] (-2.5,0)--(4,0);
			\draw[thick, ->] (0,-2.8)--(0,2.8);
			\draw (4.3,0) node[below] {$x$};
			\draw (0,2.9) node[left]{$y$};
			\draw (0,0) node[below left]{$0$};
			\draw[fill] (-1,0) circle (0.5pt)node[below left]{$ -1 $};
			\draw[fill] (1,0) circle (0.5pt)node[below]{$ 1$};
			\draw[fill] (3,0) circle (0.5pt)node[below right]{$ 3$};
			%			\draw[dashed] (-1,0)--(-1,-1) --(0,-1); 
			%			\draw[dashed](1,0)--(1,1)--(0,1);
			%			\draw[dashed](3,0)--(3,3)--(0,3);
			\draw[line width=1.2pt,smooth,samples=100,domain=-1.65:3.5] plot(\x,{0.33*(\x)^3-(\x)^2-0.333*(\x)+1});		
			%\draw[line width=1.2pt,smooth,samples=100,domain=-3.3:2.8] plot(\x,{0.75*(\x)^2+0.5*\x-1});
		\end{tikzpicture}	
	}
	\loigiai{
		Ta có $g'(x)=f'\left(x+m\right)$. Vì $ y=f'(x)$ liên tục trên $\mathbb{R}$ nên $g'(x)=f'\left(x+m\right)$ cũng liên tục trên $\mathbb{R}$. Căn cứ vào đồ thị hàm số $ y=f'(x)$ ta thấy\\
		$g'(x)<0\Leftrightarrow{f}'\left(x+m\right)<0$ $\Leftrightarrow\hoac{
			& x+m<-1\\ 
			& 1<x+m<3} \Leftrightarrow \hoac{
			& x<-1-m\\ 
			& 1-m<x<3-m.}$\\
		Hàm số $ g(x)=f\left(x+m\right)$ nghịch biến trên khoảng $\left(1\,;\,2\right)$
		$\Leftrightarrow \hoac{
			& 2\le-1-m\\ 
			&\hoac{
				& 3-m\ge 2\\ 
				& 1-m\le 1}} \Leftrightarrow \hoac{
			& m\le-3\\ 
			& 0\le m\le 1.}$\\
		Mà $ m$ là số nguyên thuộc đoạn $\left[-5\,;\,5\right]$ nên ta có $ S=\left\{-5;-4;-3;0;1\right\}$.\\
		Vậy $ S$ có $5$ phần tử.}
\end{ex}

\begin{ex}%[2D1G1-3]%Câu 2
	[Chuyên Nguyễn Bỉnh Khiêm-Quảng Nam-2020] Cho hàm số $ y=f(x)$ có đạo hàm trên $\mathbb{R}$ và bảng xét dấu đạo hàm như hình vẽ sau
	\begin{center}
		\begin{tikzpicture}
			\tkzTabInit[lgt=1.2,espcl=2.5,nocadre]
			{$x$/0.7, $f'(x)$ /2.5} % first column
			{$-\infty$, $-10$,$-2$, $3$,$8$, $+\infty$} % first row
			\tkzTabLine { ,+,z,-,z,+,z,-,z,+, } % second row
			%				\tkzTabLine {,-,z,+,t,+,} % third row
			%				\tkzTabLine {,+,d,-,z,+,} % last row
		\end{tikzpicture}
	\end{center}
	Có bao nhiêu số nguyên $ m$ để hàm số $ y=f\left(x^3+4x+m\right)$ nghịch biến trên khoảng $\left(-1;1\right)$?
	\choice
	{$ 3$}
	{$ 0$}
	{\True $ 1$}
	{$ 2$}
	\loigiai
	{
		Đặt $ t=x^3+4x+m\Rightarrow{t}'=3x^2+4$ nên $ t$ đồng biến trên $\left(-1;1\right)$ và $ t\in\left(m-5;m+5\right)$.\\
		Yêu cầu bài toán trở thành tìm $ m$ để hàm số $ f(t)$ nghịch biến trên khoảng $\left(m-5;m+5\right)$.\\
		Dựa vào bảng biến thiên ta được $\heva{
			& m-5\ge-2\\ 
			& m+5\le 8} \Leftrightarrow \heva{
			& m\ge 3\\ 
			& m\le 3} \Leftrightarrow m=3$.}
\end{ex}

\begin{ex}%[2D1G1-3]%Câu 3
	[Chuyên ĐH Vinh-Nghệ An-2020]
	\immini{
		Cho hàm số $ f(x)$ có đạo hàm trên $\mathbb{R}$và $ f(1)=1$. Đồ thị hàm số $ y=f'(x)$ như hình bên. Có bao nhiêu số nguyên dương $ a$ để hàm số $ y=\left| 4f\left(\sin x\right)+\cos 2x-a\right|$ nghịch biến trên $\left(0;\dfrac{\pi}{2}\right)$?
		\choice
		{$ 2$}
		{\True $ 3$}
		{Vô số}
		{$ 5$}}{
		\begin{tikzpicture}[scale=0.9,font=\footnotesize, line join=round, line cap=round, >=stealth] %Đường cong bậc 3
			\draw[thick, ->] (-2.5,0)--(3,0);
			\draw[thick, ->] (0,-2.8)--(0,2.8);
			\draw (3.1,0) node[below] {$x$};
			\draw (0,2.9) node[left]{$y$};
			\draw (0,0) node[below left]{$0$};
			\draw[fill] (-1,0) circle (0.5pt)node[below]{$ -1 $};
			\draw[fill] (1,0) circle (0.5pt)node[above]{$ 1$};
			%	\draw[fill] (3,0) circle (0.5pt)node[below right]{$ 3$};
			\draw[dashed] (-1,0)--(-1,1); 
			\draw[dashed](1,0)--(1,-1);
			%			\draw[dashed](3,0)--(3,3)--(0,3);
			\draw[line width=1.2pt,smooth,samples=100,domain=-2:2] plot(\x,{.8*(\x)^3+0*(\x)^2-1.8*(\x)});		
			%\draw[line width=1.2pt,smooth,samples=100,domain=-3.3:2.8] plot(\x,{0.75*(\x)^2+0.5*\x-1});
			\draw (2.0,2.8) node[left]{$y=f'(x)$};
		\end{tikzpicture}	
	}
	\loigiai
	{		Đặt $g(x)=\left| 4f\left(\sin x\right)+\cos 2x-a\right|\Rightarrow g(x)=\sqrt{\left[4f\left(\sin x\right)+\cos 2x-a\right]^2}$ .\\
		$\Rightarrow{g}'(x)=\dfrac{\left[4\cos x\cdot f'\left(\sin x\right)-2\sin 2x\right]\left[4f\left(\sin x\right)+\cos 2x-a\right]}{\sqrt{\left[4f\left(\sin x\right)+\cos 2x-a\right]^2}}$.\\
		Ta có $ 4\cos x\cdot f'\left(\sin x\right)-2\sin 2x=4\cos x\left[f'\left(\sin x\right)-\sin x\right]$.\\
		Với $ x\in\left(0;\dfrac{\pi}{2}\right)$ thì $\cos x>0,\sin x\in\left(0;1\right)\Rightarrow{f}'\left(\sin x\right)-\sin x<0$.\\
		Hàm số $ g(x)$ nghịch biến trên $\left(0;\dfrac{\pi}{2}\right)$ khi $ 4f\left(\sin x\right)+\cos 2x-a\ge 0,\forall x\in\left(0;\dfrac{\pi}{2}\right)$\\
		$\Leftrightarrow 4f\left(\sin x\right)+1-2\sin^2x\ge a,\forall x\in\left(0;\dfrac{\pi}{2}\right)$.\\
		Đặt $ t=\sin x$ được $ 4f(t)+1-2t^2\ge a,\forall t\in\left(0;1\right)$ (*).\\
		Xét $ h(t)=4f(t)+1-2t^2\Rightarrow{h}'(t)=4f'(t)-4t=4\left[f'(t)-1\right]$.\\
		Với $ t\in\left(0;1\right)$ thì $h'(t)<0\Rightarrow h(t)$ nghịch biến trên $\left(0;1\right)$.\\
		Do đó (*) $\Leftrightarrow a\le h(1)=4f(1)+1-2.1^2=3$.\\
		Vậy có $3$ giá trị nguyên dương của a thỏa mãn.}
\end{ex}


\begin{ex}%[2D1G1-3]%Câu 4
	[Chuyên Quang Trung-2020]
	\immini{
		Cho hàm số $ y=f(x)$ có đạo hàm liên tục trên $\mathbb{R}$ và có đồ thị $ y=f'(x)$ như hình vẽ. Đặt $ g(x)=f\left(x-m\right)-\dfrac{1}{2}{\left(x-m-1\right)^2}+2019$, với $ m$ là tham số thực. Gọi $ S$ là tập hợp các giá trị nguyên dương của $ m$ để hàm số $ y=g(x)$ đồng biến trên khoảng $\left(5;6\right)$. Tổng tất cả các phần tử trong $ S$ bằng
		\choice
		{$ 4$}
		{$ 11$}
		{\True $ 14$}
		{$ 20$}}{
		\begin{tikzpicture}[scale=0.9,font=\footnotesize, line join=round, line cap=round, >=stealth] %Đường cong bậc 3
			\draw[style=help lines,step=1] (-2.5,-3) grid (3,3.5);
			\draw[thick, ->] (-2.5,0)--(3.5,0);
			\draw[thick, ->] (0,-2.8)--(0,2.8);
			\draw (3.6,0) node[below] {$x$};
			\draw (0,3) node[above left]{$y$};
			\draw (0,0) node[below left]{$0$};
			%\draw[fill] (-1,0) circle (0.5pt)node[below]{$ -1 $};
			\draw[fill] (1,0) circle (0.5pt)node[below left]{$ 1$};
			%	\draw[fill] (3,0) circle (0.5pt)node[below right]{$ 3$};
			\draw[dashed] (-1,0)--(-1,-2) --(2,-2)--(2,0); 
			\draw[dashed](3,0)--(3,2) --(0,2);
			\draw (-1,-2) circle (2pt);
			\draw (3,2) circle (2pt);
			%			\draw[dashed](3,0)--(3,3)--(0,3);
			\draw[line width=1.2pt,smooth,samples=100,domain=-1.1:3.1] plot(\x,{1*(\x)^3-3*(\x)^2-0*(\x)+2});		
			%\draw[line width=1.2pt,smooth,samples=100,domain=-3.3:2.8] plot(\x,{0.75*(\x)^2+0.5*\x-1});
			%\draw (2.0,2.8) node[left]{$y=f'(x)$};
		\end{tikzpicture}	
	}
	\loigiai
	{
		Xét hàm số $ g(x)=f\left(x-m\right)-\dfrac{1}{2}{\left(x-m-1\right)^2}+2019$.\\
		$g'(x)=f'\left(x-m\right)-\left(x-m-1\right)$.\\
		Xét phương trình $g'(x)=0. \quad \quad (1)$\\
		Đặt $ x-m=t$, phương trình $(1)$ trở thành $f'(t)-\left(t-1\right)=0\Leftrightarrow{f}'(t)=t-1. \quad (2)$\\
		Nghiệm của phương trình $(2)$ là hoành độ giao điểm của hai đồ thị hàm số $ y=f'(t)$ và $ y=t-1$.\\
		Ta có đồ thị các hàm số $ y=f'(t)$ và $ y=t-1$ như sau
		\begin{center}
			\begin{tikzpicture}[scale=0.9,font=\footnotesize, line join=round, line cap=round, >=stealth] %Đường cong bậc 3
				\draw[style=help lines,step=1] (-2.5,-3) grid (3,3.5);
				\draw[thick, ->] (-2.5,0)--(3.5,0);
				\draw[thick, ->] (0,-2.8)--(0,2.8);
				\draw (3.6,0) node[below] {$x$};
				\draw (0,3) node[above left]{$y$};
				\draw (0,0) node[below left]{$0$};
				%\draw[fill] (-1,0) circle (0.5pt)node[below]{$ -1 $};
				\draw[fill] (1,0) circle (0.5pt)node[below left]{$ 1$};
				%	\draw[fill] (3,0) circle (0.5pt)node[below right]{$ 3$};
				\draw[dashed] (-1,0)--(-1,-2) --(2,-2)--(2,0); 
				\draw[dashed](3,0)--(3,2) --(0,2);
				\draw (-1,-2) circle (2pt);
				\draw (3,2) circle (2pt);
				%			\draw[dashed](3,0)--(3,3)--(0,3);
				\draw[line width=1.2pt,smooth,samples=100,domain=-1.1:3.1] plot(\x,{1*(\x)^3-3*(\x)^2-0*(\x)+2});		
				%\draw[line width=1.2pt,smooth,samples=100,domain=-3.3:2.8] plot(\x,{0.75*(\x)^2+0.5*\x-1});
				%\draw (2.0,2.8) node[left]{$y=f'(x)$};
				\draw (-2,-3)--(4,3);
			\end{tikzpicture}
		\end{center}
		Căn cứ đồ thị các hàm số ta có phương trình $(2)$ có nghiệm là $\hoac{
			& t=-1\\ 
			& t=1\\ 
			& t=3} \Rightarrow \hoac{
			& x=m-1\\ 
			& x=m+1\\ 
			& x=m+3.}$\\
		Ta có bảng biến thiên của $ y=g(x)$
		\begin{center}
			\begin{tikzpicture}
				\tkzTabInit[lgt=1,espcl=2.5,nocadre]
				{$x$ /0.8, $y'$ /0.8,$y$ /2.5}
				{$-\infty$ , $m-1$,$m+1$,$m+3$, $+\infty$}
				\tkzTabLine{,+,0,-,0,+,0,-,}
				\tkzTabVar{-/$ +\infty$ ,+/$ $, -/$ $,+/$ $,-/$+\infty $}
			\end{tikzpicture}
		\end{center}
		Để hàm số $ y=g(x)$ đồng biến trên khoảng $\left(5;6\right)$ cần $\hoac{
			&\heva{
				& m-1\le 5\\ 
				& m+1\ge 6}\\ 
			& m+3\le 5}\Leftrightarrow\hoac{
			& 5\le m\le 6\\ 
			& m\le 2.}$\\
		Vì $ m\in\mathbb{N}^*\Rightarrow m$ nhận các giá trị $ 1;\,2;\,5;\,6\Rightarrow S=14$.}
\end{ex}

\begin{ex}%[2D1G1-3]%Câu 5
	[Sở Hà Nội-Lần 2-2020] 
	\immini{
		Cho hàm số $y=a{x^4}+b{x^3}+c{x^2}+dx+e,\,\,a\ne 0$. Hàm số $y=f'(x)$ có đồ thị như hình vẽ bên. 
		Gọi S là tập hợp tất cả các giá trị nguyên thuộc khoảng $\left(-6;6\right)$ của tham số $m$ để hàm số $g(x)=f\left(3-2x+m\right)+x^2-\left(m+3\right)x+2m^2$ nghịch biến trên $\left(0;1\right)$. Khi đó, tổng giá trị các phần tử của S là
		\choice
		{$12$}
		{\True $9$}
		{$6$}
		{$15$}}{
		\begin{tikzpicture}[scale=0.7,font=\footnotesize, line join=round, line cap=round, >=stealth] %Đường cong bậc 3
			%	\draw[style=help lines,step=1] (-2.5,-3) grid (3,3.5);
			\draw[thick, ->] (-4.5,0)--(6.5,0);
			\draw[thick, ->] (0,-2.8)--(0,2.8);
			\draw (6.6,0) node[below] {$x$};
			\draw (0,3) node[above left]{$y$};
			\draw (0,0) node[below left]{$0$};
			\draw[fill] (-2,0) circle (0.5pt)node[below]{$ -2 $};
			\draw[fill] (4,0) circle (0.5pt)node[above]{$ 4$};
			\draw[fill] (0,1) circle (0.5pt)node[right]{$ 1 $};
			\draw[fill] (0,-2) circle (0.5pt)node[left]{$ -2$};
			%	\draw[fill] (3,0) circle (0.5pt)node[below right]{$ 3$};
			\draw[dashed] (-2,0)--(-2,1) --(0,1); 
			\draw[dashed](4,0)--(4,-2) --(0,-2);
			%			\draw[dashed](3,0)--(3,3)--(0,3);
			\draw[line width=1.2pt,smooth,samples=100,domain=-3.8:5.5] plot(\x,{0.0714*(\x)^3-0.1423*(\x)^2-1.0714*(\x)});		
			%\draw[line width=1.2pt,smooth,samples=100,domain=-3.3:2.8] plot(\x,{0.75*(\x)^2+0.5*\x-1});
			%\draw (2.0,2.8) node[left]{$y=f'(x)$};
		\end{tikzpicture}	
	}
	\loigiai
	{
		Xét $g'(x)=-2f'\left(3-2x+m\right)+2x-\left(m+3\right)$.\\
		Xét phương trình $g'(x)=0$, đặt $t=3-2x+m$ thì phương trình trở thành\\ $-2\cdot \left[f'(t)-\dfrac{-t}{2}\right]=0\Leftrightarrow\hoac{
			& t=-2\\ 
			& t=4\\ 
			& t=0.}$ \\
		Từ đó, $g'(x)=0\Leftrightarrow{x_1}=\dfrac{5+m}{2},\,x_2=\dfrac{m+3}{2},x_3=\dfrac{-1+m}{2}$.\\
		Lập bảng xét dấu, đồng thời lưu ý nếu $x>x_1$ thì $t<t_1$ nên $f(x)>0$. Và các dấu đan xen nhau do các nghiệm đều làm đổi dấu đạo hàm nên suy ra $g'(x)\le 0\Leftrightarrow x\in\left[x_2;{x_1}\right]\cup\left(-\infty ;{x_3}\right]$.\\
		Vì hàm số nghịch biến trên $\left(0;1\right)$ nên \\
		$g'(x)\le 0,\,\forall x\in\left(0;1\right)$ từ đó suy ra $\hoac{
			&\dfrac{3+m}{2}\le 0<1\le\dfrac{5+m}{2}\\ 
			& 1\le\dfrac{-1+m}{2}.}$ \\
		và giải ra các giá trị nguyên thuộc $\left(-6;6\right)$ của $m$ là $-3$; $3$; $4$; $5$. }
\end{ex}

\begin{ex}%[2D1G1-3]%Câu 6
	[Chuyên Quang Trung-Bình Phước-Lần 2-2020]
	\immini{
		Cho hàm số $ y=f(x)$ có đạo hàm liên tục trên $\mathbb{R}$ và có đồ thị $ y=f'(x)$ như hình vẽ bên. Đặt $ g(x)=f\left(x-m\right)-\dfrac{1}{2}{\left(x-m-1\right)^2}+2019$, với $ m$ là tham số thực. Gọi $ S$ là tập hợp các giá trị nguyên dương của $ m$ để hàm số $ y=g(x)$ đồng biến trên khoảng $\left(5;6\right)$. Tổng tất cả các phần tử trong $ S$ bằng
		\choice
		{$ 4$}
		{$ 11$}
		{\True $ 14$}
		{$ 20$}}{
		\begin{tikzpicture}[scale=0.9,font=\footnotesize, line join=round, line cap=round, >=stealth] %Đường cong bậc 3
			\draw[thick, ->] (-2.5,0)--(3.7,0);
			\draw[thick, ->] (0,-2.8)--(0,2.8);
			\draw (3.9,0) node[below] {$x$};
			\draw (0,2.9) node[left]{$y$};
			\draw (0,0) node[below left]{$0$};
			\draw[fill] (-1,0) circle (0.5pt)node[above]{$ -1 $};
			\draw[fill] (1,0) circle (0.5pt)node[below]{$ 1$};
			\draw[fill] (3,0) circle (0.5pt)node[below]{$ 3$};
			\draw[fill] (2,0) circle (0.5pt)node[above]{$ 2$};
			\draw[fill] (0,2) circle (0.5pt)node[above left]{$ 2$};
			\draw[fill] (0,-2) circle (0.5pt)node[below left]{$ -2$};
			\draw[dashed] (-1,0)--(-1,-2)--(2,-2)--(2,0); 
			\draw[dashed](3,0)--(3,2)--(0,2);
			%			\draw[dashed](3,0)--(3,3)--(0,3);
			\draw[line width=1.2pt,smooth,samples=100,domain=-1.1:3.1] plot(\x,{1*(\x)^3-3*(\x)^2-0*(\x)+2});		
			%\draw[line width=1.2pt,smooth,samples=100,domain=-3.3:2.8] plot(\x,{0.75*(\x)^2+0.5*\x-1});
			%	\draw (2.0,2.8) node[left]{$y=f'(x)$};
	\end{tikzpicture}	}
	\loigiai
	{
		Ta có $g'(x)=f'\left(x-m\right)-\left(x-m-1\right)$.\\
		Cho $g'(x)=0\Leftrightarrow{f}'\left(x-m\right)=x-m-1$.\\
		Đặt $ x-m=t\Rightarrow f'(t)=t-1$\\
		Khi đó nghiệm của phương trình là hoành độ giao điểm của đồ thị hàm số $ y=f'(t)$ và và đường thẳng $ y=t-1$.
		\begin{center}
			\begin{tikzpicture}[scale=0.9,font=\footnotesize, line join=round, line cap=round, >=stealth] %Đường cong bậc 3
				\draw[thick, ->] (-2.5,0)--(3.7,0);
				\draw[thick, ->] (0,-2.8)--(0,2.8);
				\draw (3.9,0) node[below] {$x$};
				\draw (0,2.9) node[left]{$y$};
				\draw (0,0) node[below left]{$0$};
				\draw[fill] (-1,0) circle (0.5pt)node[above]{$ -1 $};
				\draw[fill] (1,0) circle (0.5pt)node[below]{$ 1$};
				\draw[fill] (3,0) circle (0.5pt)node[below]{$ 3$};
				\draw[fill] (2,0) circle (0.5pt)node[above]{$ 2$};
				\draw[fill] (0,2) circle (0.5pt)node[above left]{$ 2$};
				\draw[fill] (0,-2) circle (0.5pt)node[below left]{$ -2$};
				\draw[dashed] (-1,0)--(-1,-2)--(2,-2)--(2,0); 
				\draw[dashed](3,0)--(3,2)--(0,2);
				%			\draw[dashed](3,0)--(3,3)--(0,3);
				\draw[line width=1.2pt,smooth,samples=100,domain=-1.1:3.1] plot(\x,{1*(\x)^3-3*(\x)^2-0*(\x)+2});		
				%\draw[line width=1.2pt,smooth,samples=100,domain=-3.3:2.8] plot(\x,{0.75*(\x)^2+0.5*\x-1});
				%	\draw (2.0,2.8) node[left]{$y=f'(x)$};
				\coordinate (a) at ($(-1,-2)!1.2!(3,2)$);
				\coordinate (b) at ($(-1,-2)!-.2!(3,2)$);
				\draw[line width=1.2pt,smooth] (a)--(b);
			\end{tikzpicture}
		\end{center}
		Dựa vào đồ thị hàm số ta có được $f'(t)=t-1\Leftrightarrow\hoac{
			& t=-1\\ 
			& t=1\\ 
			& t=3.} $ \\
		Bảng xét dấu của $g'(t)$
		\begin{center}
			\begin{tikzpicture}
				\tkzTabInit[lgt=1.2,espcl=2.5,nocadre]
				{$t$/1, $g'(x)$ /.8} % first column
				{$-\infty$, $-1$,$1$, $3$, $+\infty$} % first row
				\tkzTabLine { ,-,0,+,0,-,0,+, } % second row
				%				\tkzTabLine {,-,z,+,t,+,} % third row
				%				\tkzTabLine {,+,d,-,z,+,} % last row
			\end{tikzpicture}
		\end{center}
		Từ bảng xét dấu ta thấy hàm số $ g(t)$ đồng biến trên khoảng $\left(-1;1\right)$ và $\left(3;+\infty\right)$.\\
		Hay $\hoac{
			&-1<t<1\\ 
			& t>3}\Leftrightarrow\hoac{
			&-1<x-m<1\\ 
			& x-m>3} \Leftrightarrow\hoac{
			& m-1<x<m+1\\ 
			& x>m+3.}$\\
		Để hàm số $ g(x)$ đồng biến trên khoảng $\left(5;6\right)$ thì $\hoac{
			& m-1\le 5<6\le m+1\\ 
			& m+3\le 5<6} \Leftrightarrow\hoac{
			& 5\le m\le 6\\ 
			& m\le 2.}$\\
		Vì $ m$ là các số nguyên dương nên $ S=\left\{ 1;2;5;6\right\}$.\\
		Vậy tổng tất cả các phần tử của $ S$ là $ 1+2+5+6=14$.}
\end{ex}

\begin{ex}%[2D1G1-3]%Câu 7
	\immini{
		Cho hàm số $ y=f(x)$ liên tục có đạo hàm trên $\mathbb{R}$. Biết hàm số $ f'(x)$ có đồ thị cho như hình vẽ bên. Có bao nhiêu giá trị nguyên của $ m$ thuộc $\left[-2019;2019\right]$ để hàm só $ g(x)=f\left(2019^x\right)-mx+2$ đồng biến trên $\left[0;1\right]$.
		\choice
		{$ 2028$}
		{$ 2019$}
		{$ 2011$}
		{\True $ 2020$}}{
		\begin{tikzpicture}[scale=0.9,font=\footnotesize, line join=round, line cap=round, >=stealth] %Đường cong bậc 3
			\draw[thick, ->] (-3.5,0)--(2.5,0);
			\draw[thick, ->] (0,-2.8)--(0,2.8);
			\draw (2.7,0) node[below] {$x$};
			\draw (0,2.9) node[left]{$y$};
			\draw (0,0) node[below left]{$0$};
			%	\draw[fill] (-1,0) circle (0.5pt)node[above]{$ -1 $};
			\draw[fill] (1,0) circle (0.5pt)node[below right]{$ 1$};
			%		\draw[fill] (3,0) circle (0.5pt)node[below]{$ 3$};
			%		\draw[fill] (2,0) circle (0.5pt)node[above]{$ 2$};
			%		\draw[fill] (0,2) circle (0.5pt)node[above left]{$ 2$};
			%		\draw[fill] (0,-2) circle (0.5pt)node[below left]{$ -2$};
			%		\draw[dashed] (-1,0)--(-1,-2)--(2,-2)--(2,0); 
			%		\draw[dashed](3,0)--(3,2)--(0,2);
			\draw[line width=1.2pt,smooth,samples=100,domain=-3.28:1.32] plot(\x,{0.667*(\x)^3+2*(\x)^2-0.667*(\x)-2});		
			%\draw[line width=1.2pt,smooth,samples=100,domain=-3.3:2.8] plot(\x,{0.75*(\x)^2+0.5*\x-1});
			%	\draw (2.0,2.8) node[left]{$y=f'(x)$};
	\end{tikzpicture}	}
	\loigiai{
		Ta có $ g'(x)=2019^x\ln 2019\cdot f'\left(2019^x\right)-m$.\\
		Ta lại có hàm số $ y=2019^x$ đồng biến trên $\left[0;1\right]$.\\
		Với $ x\in\left[0;1\right]$ thì $2019^x\in\left[1;2019\right]$ mà hàm $ y=f'(x)$ đồng biến trên $\left(1;+\infty\right)$ nên hàm $ y=f'\left(2019^x\right)$ đồng biến trên $\left[0;1\right]$.\\
		Mà $2019^x\ge 1;f'\left(2019^x\right)>0\,\forall\,x\in\left[0;1\right]$ nên hàm $ h(x)=2019^x\ln 2019\cdot f'\left(2019^x\right)$ đồng biến trên $\left[0;1\right]$.\\
		Hay $ h(x)\ge h(0)=0,\forall\,x\in\left[0;1\right]$.\\
		Do vậy hàm số $ g(x)$ đồng biến trên đoạn $\left[0;1\right]$$\Leftrightarrow g'(x)\ge 0,\forall\,x\in\left[0;1\right]$\\
		$\Leftrightarrow m\le{2019^x}\ln 2019.f'\left(2019^x\right),\forall\,x\in\left[0;1\right]$ $\Leftrightarrow m\le\underset{x\in\left[0;1\right]}{\min}\,h(x)=h(0)=0$\\
		Vì $ m$ nguyên và $ m\in\left[-2019;2019\right]\Rightarrow $có $ 2020$ giá trị $ m$ thỏa mãn yêu cầu bài toán.}
\end{ex}

\begin{ex}%[2D1G1-3]%Câu 8
	\immini{
		Cho hàm số $y=f(x)$ có đồ thị $f'(x)\,$ như hình vẽ. Có bao nhiêu giá trị nguyên $m\in\left(-2020\,;\,2020\right)$ để hàm số $g(x)=f\left(2x-3\right)\,-\ln \left(1+x^2\right)-2mx$ đồng biến trên $\left(\dfrac{1}{2};2\right)$?
		\choice
		{$ 2020$}
		{\True $ 2019$}
		{$ 2021$}
		{$ 2018$}}{
		\begin{tikzpicture}[scale=0.9,font=\footnotesize, line join=round, line cap=round, >=stealth] %Đường cong bậc 3
			\draw[thick, ->] (-2.5,0)--(2.5,0);
			\draw[thick, ->] (0,-1.8)--(0,5.8);
			\draw (2.7,0) node[below] {$x$};
			\draw (0,5.9) node[left]{$y$};
			\draw (0,0) node[below left]{$0$};
			\draw[fill] (-2,0) circle (0.5pt)node[below]{$ -2 $};
			\draw[fill] (1,0) circle (0.5pt)node[below]{$ 1$};
			\draw[fill] (-1,0) circle (0.5pt)node[below]{$-1$};
			\draw[fill] (0,4) circle (0.5pt)node[above left]{$ 2$};
			%		\draw[fill] (0,2) circle (0.5pt)node[above left]{$ 2$};
			%		\draw[fill] (0,-2) circle (0.5pt)node[below left]{$ -2$};
			\draw[dashed] (-2,0)--(-2,4)--(1,4)--(1,0); 
			%		\draw[dashed](3,0)--(3,2)--(0,2);
			\draw[line width=1.2pt,smooth,samples=100,domain=-2.1:2.1] plot(\x,{-1*(\x)^3+0*(\x)^2+3*(\x)+2});		
			%\draw[line width=1.2pt,smooth,samples=100,domain=-3.3:2.8] plot(\x,{0.75*(\x)^2+0.5*\x-1});
			%	\draw (2.0,2.8) node[left]{$y=f'(x)$};
	\end{tikzpicture}	}
	\loigiai{
		Ta có $g'(x)=2f'\left(2x-3\right)-\dfrac{2x}{1+x^2}-2m$.\\
		Hàm số $ g(x)$ đồng biến trên $\left(\dfrac{1}{2};2\right)$ khi và chỉ khi \\
		$g'(x)\ge 0,\,\,\forall x\in\left(-1;\,2\right)$\\
		$\Leftrightarrow m\le{f}'\left(2x-3\right)-\dfrac{x}{1+x^2},\,\,\forall x\in\left(\dfrac{1}{2};2\right)$\\
		$\Leftrightarrow m\le\underset{x\in\left[\dfrac{1}{2};2\right]}{\min}\,\left[f'\left(2x-3\right)-\dfrac{x}{1+x^2}\right]$. \, \,  $(1)$\\
		Đặt $ t=2x-3$, khi đó $ x\in\left(\dfrac{1}{2};2\right)\Leftrightarrow t\in\left(-2;\,1\right)$.\\
		Từ đồ thị hàm $f'(x)$ suy ra $f'(t)\ge 0,\,\,\forall t\in\left(-2;1\right)$ và $f'(t)=0$ khi $ t=-1$.\\
		Tức là $f'\left(2x-3\right)\ge 0,\,\,\forall x\in\left(\dfrac{1}{2};\,2\right)$$\Rightarrow\underset{x\in\left[\dfrac{1}{2};2\right]}{\min}\,f'\left(2x-3\right)=0$ khi $ x=1$. $(2)$\\
		Xét hàm số $ h(x)=-\dfrac{x}{1+x^2}$ trên khoảng $\left(\dfrac{1}{2};\,2\right)$.\\
		Ta có $h'(x)=\dfrac{x^2-1}{\left(1+x^2\right)^2}$ và\\
		$h'(x)=0\Leftrightarrow{x^2}-1=0\Leftrightarrow x=\pm 1$.\\
		Bảng biến thiên của hàm số $ h(x)$ trên $\left(\dfrac{1}{2};\,2\right)$ như sau
		\begin{center}
			\begin{tikzpicture}
				\tkzTabInit[lgt=1.2,espcl=2.5,nocadre]
				{$x$ /0.7, $h'(x)$ /0.7,$h(x)$ /2.5}
				{$\dfrac{1}{2}$ , $1$,$2$}
				\tkzTabLine{,-,0,+,}
				\tkzTabVar{+/$  $ ,-/$ \-\dfrac{1}{2} $, +/$ $}
			\end{tikzpicture}
		\end{center}
		Từ bảng biến thiên suy ra $ h(x)\ge-\dfrac{1}{2}$$\Rightarrow\underset{x\in\left[\dfrac{1}{2};2\right]}{\min}\,h(x)=-\dfrac{1}{2}$ khi $ x=1$. \, \,  $(3)$\\
		Từ $(1)$, $(2)$ và $(3)$ suy ra $ m\le-\dfrac{1}{2}$.\\
		Kết hợp với $ m\in\mathbb{Z}$, $ m\in\left(-2020;\,2020\right)$ thì $ m\in\left\{-2019;\,-201;\ldots ;-2;-1\right\}$.\\
		Vậy có tất cả $ 2019$ giá trị $ m$ cần tìm.}
\end{ex}

\begin{ex}%[2D1G1-3]%Câu 9
	Cho hàm số $ f(x)$ liên tục trên $\mathbb{R}$ và có đạo hàm $f'(x)=x^2\left(x-2\right)\left(x^2-6x+m\right)$ với mọi $ x\in\mathbb{R}$. Có bao nhiêu số nguyên $ m$ thuộc đoạn $\left[-2020;2020\right]$ để hàm số $ g(x)=f\left(1-x\right)$ nghịch biến trên khoảng $\left(-\infty ;-1\right)$?
	\choice
	{$ 2016$}
	{$ 2014$}
	{\True $ 2012$}
	{$ 2010$}
	\loigiai{
		Ta có \\
		$g'(x)=f'\left(1-x\right)=-\left(1-x\right)^2\left(-x-1\right)\left[\left(1-x\right)^2-6\left(1-x\right)+m\right]$
		$=\left(x-1\right)^2\left(x+1\right)\left(x^2+4x+m-5\right)$.\\
		Hàm số $ g(x)$ nghịch biến trên khoảng $\left(-\infty ;-1\right)$\\
		$\Leftrightarrow{g}'(x)\le 0,\forall x<-1$ $(*)$, (dấu \lq\lq $=$\rq\rq \, xảy ra tại hữu hạn điểm).\\
		Với $ x<-1$ thì $\left(x-1\right)^2>0$ và $ x+1<0$ nên\\
		$(*)$ $\Leftrightarrow{x^2}+4x+m-5\ge 0,\forall x<-1 \Leftrightarrow m\ge-x^2-4x+5,\forall x<-1$.\\
		Xét hàm số $ y=-x^2-4x+5$ trên khoảng $\left(-\infty ;-1\right)$, ta có bảng biến thiên
		\begin{center}
			\begin{tikzpicture}
				\tkzTabInit[lgt=1.8,espcl=2.3]
				{$x$ /1.2, $y'$ /1.2,$y$ /2}
				{$-\infty$ , $-2$,$-1$}
				\tkzTabLine{,+,0,-,}
				\tkzTabVar{-/$ -\infty $ ,+/$9 $, -/$ 8$}
			\end{tikzpicture}
		\end{center}
		Từ bảng biến thiên suy ra $ m\ge 9$.\\
		Kết hợp với $ m$ thuộc đoạn $\left[-2020;2020\right]$ và $ m$ nguyên nên $ m\in\left\{ 9;10;11;\ldots ;2020\right\}$.\\
		Vậy có $ 2012$ số nguyên $ m$ thỏa mãn đề bài.}
\end{ex}

\begin{ex}%[2D1G1-3]%Câu 10
	\immini{
		Cho hàm số $f(x)$ xác định và liên tục trên $ R$. Hàm số $y=f'(x)$ liên tục trên $\mathbb{R}$ và có đồ thị như hình vẽ bên.
		Xét hàm số $g(x)=f\left(x-2m\right)+\dfrac{1}{2}{\left(2m-x\right)^2}+2020$, với $ m$ là tham số thực. Gọi $ S$ là tập hợp các giá trị nguyên dương của $ m$ để hàm số $ y=g(x)$ nghịch biến trên khoảng $\left(3;4\right)$. Hỏi số phần tử của $ S$ bằng bao nhiêu?
		\choice
		{$4$}
		{\True $2$}
		{$3$}
		{Vô số}}
	{
		\begin{tikzpicture}[scale=0.7,>=stealth, font=\footnotesize, line join=round, line cap=round]
			\def\xmin{-3.5} \def\xmax{4.5}
			\def\ymin{-5.2} \def\ymax{4}
			\clip(\xmin,\ymin) rectangle (\xmax,\ymax);
			\draw[->] (\xmin,0)--(\xmax,0) node [below]{$x$};
			\draw[->] (0,\ymin)--(0,\ymax) node [left]{$y$};
			\node at (0,0) [below left]{$O$};
			\path
			(-3.1,3.7) coordinate (A)
			(-3,3) coordinate (B)
			(0,-2) coordinate (C)
			(0.65,-2) coordinate (D)
			(1,-1) coordinate (E)
			(3,-3) coordinate (F)
			(3.4,-5) coordinate (G);
			\draw[smooth]
			(A)..controls +(-88:0.1) and +(93:.1)..
			(B)..controls +(-87:0.3) and +(-100:8.5)..
			(C)..controls +(75:.8) and +(180:.1)..
			(D)..controls +(0:.1) and +(-105:.3)..
			(E)..controls +(70:2) and +(97:0.4)..
			(F)..controls +(-80:.1) and +(90:0.3)..
			(G);
			\draw[dashed] 
			(-3,0)node[below]{$-3$}|-(0,3)node[right]{$3$}
			(1,0)node[above]{$1$}|-(0,-1)node[left]{$-1$}
			(3,0)node[above]{$3$}|-(0,-3)node[below right]{$-3$};
			\fill 
			(0,-2) circle(1.5pt)
			(-3,3) circle(1.5pt)
			(3,-3) circle(1.5pt)
			(1,-1) circle(1.5pt);
			\node at (2.1,-4) {$y=f'(x)$};
		\end{tikzpicture}
	}
	\loigiai{
		Ta có $g'(x)=f'\left(x-2m\right)-\left(2m-x\right)$.		Đặt $h(x)=f'(x)-\left(-x\right)$.\\
		Từ đồ thị hàm số $y=f'(x)$ và đồ thị hàm số $y=-x$ trên hình vẽ suy ra \\
		$h(x)\le 0\Leftrightarrow f'(x)\le-x\Leftrightarrow\hoac{
			&-3\le x\le 1\\ 
			& x\ge 3.}$ 
		\begin{center}
			\begin{tikzpicture}[scale=0.7,>=stealth, font=\footnotesize, line join=round, line cap=round]
				\def\xmin{-3.5} \def\xmax{4.5}
				\def\ymin{-5.2} \def\ymax{4}
				\clip(\xmin,\ymin) rectangle (\xmax,\ymax);
				\draw[->] (\xmin,0)--(\xmax,0) node [below]{$x$};
				\draw[->] (0,\ymin)--(0,\ymax) node [left]{$y$};
				\node at (0,0) [below left]{$O$};
				\path
				(-3.1,3.7) coordinate (A)
				(-3,3) coordinate (B)
				(0,-2) coordinate (C)
				(0.65,-2) coordinate (D)
				(1,-1) coordinate (E)
				(3,-3) coordinate (F)
				(3.4,-5) coordinate (G);
				\draw[smooth]
				(A)..controls +(-88:0.1) and +(93:.1)..
				(B)..controls +(-87:0.3) and +(-100:8.5)..
				(C)..controls +(75:.8) and +(180:.1)..
				(D)..controls +(0:.1) and +(-105:.3)..
				(E)..controls +(70:2) and +(97:0.4)..
				(F)..controls +(-80:.1) and +(90:0.3)..
				(G);
				\draw[dashed] 
				(-3,0)node[below]{$-3$}|-(0,3)node[right]{$3$}
				(1,0)node[above]{$1$}|-(0,-1)node[left]{$-1$}
				(3,0)node[above]{$3$}|-(0,-3)node[below right]{$-3$};
				\fill 
				(0,-2) circle(1.5pt)
				(-3,3) circle(1.5pt)
				(3,-3) circle(1.5pt)
				(1,-1) circle(1.5pt);
				\draw[smooth,samples=300,domain=-3.2:3.7] plot(\x,{-(\x)});
				\node at (2.1,-4) {$y=f'(x)$};
				\node at (-1,2.1) {$y=h(x)$};
			\end{tikzpicture}
		\end{center}
		Ta có $ g'(x)=h\left(x-2m\right)\le 0\Leftrightarrow\hoac{
			&-3\le x-2m\le 1\\ 
			& x-2m\ge 3}\Leftrightarrow\hoac{
			& 2m-3\le x\le 2m+1\\ 
			& x\ge 2m+3.}$.\\
		Suy ra hàm số $ y=g(x)$ nghịch biến trên các khoảng $\left(2m-3;2m+1\right)$ và $\left(2m+3;+\infty\right)$.\\
		Do đó hàm số $ y=g(x)$ nghịch biến trên khoảng $\left(3;4\right)$ $\Leftrightarrow\hoac{
			&\heva{
				& 2m-3\le 3\\ 
				& 2m+1\ge 4}\\ 
			& 2m+3\le 3}\Leftrightarrow\hoac{
			&\dfrac{3}{2}\le m\le 3\\ 
			& m\le 0.}$ \\
		Mặt khác, do $ m$ nguyên dương nên $ m\in\left\{ 2;3\right\}\Rightarrow S=\left\{ 2;3\right\}$. Vậy số phần tử của $ S$ bằng $2$.\\
	}
	
\end{ex}

\begin{ex}%[2D1G1-3]%Câu 11
	Cho hàm số $f(x)$ có đạo hàm trên $\mathbb{R}$ là $f'(x)=\left(x-1\right)\left(x+3\right)$. Có bao nhiêu giá trị nguyên của tham số $m$ thuộc đoạn $\left[-10;20\right]$ để hàm số $y=f\left(x^2+3x-m\right)$ đồng biến trên khoảng $\left(0;2\right)$?
	\choice
	{\True $ 18$}
	{$ 17$}
	{$ 16$}
	{$ 20$}
	\loigiai{
		Ta có $y'=f'\left(x^2+3x-m\right)=\left(2x+3\right){f}'\left(x^2+3x-m\right)$.\\
		Theo đề bài ta có $f'(x)=\left(x-1\right)\left(x+3\right)$\\
		suy ra $f'(x)>0\Leftrightarrow\hoac{
			& x<-3\\ 
			& x>1}$ và $f'(x)<0\Leftrightarrow-3<x<1$ .\\
		Hàm số đồng biến trên khoảng $\left(0;2\right)$ khi $y'\ge 0,\forall x\in\left(0;2\right)$\\
		$\Leftrightarrow\left(2x+3\right){f}'\left(x^2+3x-m\right)\ge 0,\forall x\in\left(0;2\right)$.\\
		Do $x\in\left(0;2\right)$ nên $2x+3>0,\forall x\in\left(0;2\right)$. Do đó, ta có\\
		$y'\ge 0,\forall x\in\left(0;2\right)\Leftrightarrow f'\left(x^2+3x-m\right)\ge 0$\\
		$\Leftrightarrow\hoac{
			&{x^2}+3x-m\le-3\\ 
			&{x^2}+3x-m\ge 1}\Leftrightarrow\hoac{
			& m\ge{x^2}+3x+3\\ 
			& m\le{x^2}+3x-1}$\\
		$\Leftrightarrow\hoac{
			& m\ge\underset{\left[0;2\right]}{\max}\,\left(x^2+3x+3\right)\\ 
			& m\le\underset{\left[0;2\right]}{\min}\,\left(x^2+3x-1\right)} \Leftrightarrow\hoac{
			& m\ge 13\\ 
			& m\le-1}$.\\
		Do $m\in\left[-10;20\right]$, $ m\in\mathbb{Z}$ nên có $ 18$ giá trị nguyên của $m$ thỏa yêu cầu đề bài.}
\end{ex}

\begin{ex}%[2D1G1-3]%Câu 12
	Cho các hàm số $f(x)=x^3+4x+m$ và $g(x)=\left(x^2+2018\right){\left(x^2+2019\right)^2}{\left(x^2+2020\right)^3}$ . Có bao nhiêu giá trị nguyên của tham số $m\in\left[-2020;2020\right]$ để hàm số $g\left(f(x)\right)$ đồng biến trên $\left(2;+\infty\right)$ ?
	\choice
	{$2005$}
	{\True $2037$}
	{$4016$}
	{$4041$}
	\loigiai{
		Ta có $f(x)=x^3+4x+m$ và \\
		$g(x)=\left(x^2+2018\right){\left(x^2+2019\right)^2}{\left(x^2+2020\right)^3}=a_{12}{x^{12}}+a_{10}{x^{10}}+...+a_2x^2+a_0$.\\
		Suy ra $f'(x)=3x^2+4$ , $g'(x)=12a_{12}{x^{11}}+10a_{10}{x^9}+...+2a_2x$.\\
		Và có 
		\begin{eqnarray*}
			\left[g\left(f(x)\right)\right]' &=& f'(x)\left[12a_{12}{\left(f(x)\right)^{11}}+10a_{10}{\left(f(x)\right)^9}+...+2a_2f(x)\right]\\
			&=& f(x)f'(x)\left(12a_{12}{\left(f(x)\right)^{10}}+10a_{10}{\left(f(x)\right)^8}+...+2a_2\right).
		\end{eqnarray*} 
		Dễ thấy $a_{12};{a_{10}};...;{a_2};{a_0}>0$ và $f'(x)=3x^2+4>0$, $\forall x>2$.\\
		Do đó $f'(x)\left(12a_{12}{\left(f(x)\right)^{10}}+10a_{10}{\left(f(x)\right)^8}+...+2a_2\right)>0$ , $\forall x>2$.\\
		Hàm số $g\left(f(x)\right)$ đồng biến trên $\left(2;+\infty\right)$ khi $\left[g\left(f(x)\right)\right]^{'}\ge 0$, $\forall x>2$\\
		$\Rightarrow  f(x)\ge 0$, $\forall x>2 \Leftrightarrow x^3+4x+m\ge 0$, $\forall x>3 \Leftrightarrow  m\ge-x^3-4x$, $\forall x>2$\\
		$ \Rightarrow  m\ge\underset{\left[2;+\infty\right)}{\max}\,\left(-x^3-4x\right)=-16$.\\
		Vì $m\in\left[-2020;2020\right]$ và $m\in\mathbb{Z}$ nên có $2037$ giá trị thỏa mãn $m$ .}
\end{ex}

\begin{ex}%[2D1G1-3]%Câu 13
	Cho hàm số $y=f(x)$ có đạo hàm $f'(x)=x{\left(x+1\right)^2}\left(x^2+2mx+1\right)$ với mọi $x \in \mathbb{R}$. Có bao nhiêu số nguyên âm $m$ để hàm số $g(x)=f\left(2x+1\right)$ đồng biến trên khoảng $\left(3;5\right)$?
	\choice
	{\True $3$}
	{$2$}
	{$4$}
	{$6$}
	\loigiai{
		Ta có $g'(x)=2f'(2x+1)=2(2x+1)(2x+2)^2[(2x+1)^2+2m(2x+1)+1]$. 	Đặt $t=2x+1$\\
		Để hàm số $g(x)$ đồng biến trên khoảng $\left(3;5\right)$ khi và chỉ khi 
		\begin{eqnarray*}
			& & g'(x)\ge 0,\forall x\in\left(3;5\right) \\
			& \Leftrightarrow & t(t^2+2mt+1)\ge 0,\forall t\in\left(7;11\right)\Leftrightarrow{t^2}+2mt+1\ge 0,\,\,\forall t\in\left(7;11\right) \\
			&\Leftrightarrow & 2m\ge\dfrac{-t^2-1}{t},\,\,\,\forall t\in\left(7;11\right)
		\end{eqnarray*}	
		Xét hàm số $h(t)=\dfrac{-t^2-1}{t}$ trên $\left[7;11\right]$, có $h'(t)=\dfrac{-t^2+1}{t^2}$\\
		Bảng biến thiên
		\begin{center}
			\begin{tikzpicture}
				\tkzTabInit[espcl=3,lgt=1.2,nocadre]
				{$t$/0.7,$h'(t)$/0.7,$h(t)$/2.5}
				{$-\infty$,$1$,$11$,$+\infty$}
				\tkzTabLine{, ,,-,,,}
				%	\node (0) at ($(N12)+(0,-3)$) {$-\infty$};
				\node (1) at ($(N22)+(0,-0.8)$) [right] {$-\dfrac{50}{7}$};
				\node (2) at ($(N32)+(0,-2.5)$) [left] {$-\dfrac{122}{11}$};
				
				
				%				\node (3) at ($(N11+(-0.5,0))$) {};
				%				\node (4) at ($(N23)$) {};
				\fill[pattern=north east lines] (7.0,-0.7) rectangle (10,-4.4);
				\fill[pattern=north east lines] (1.5,-0.7) rectangle (4.5,-4.4);
				\draw[->] (1)--(2);	
				\draw[dashed] (4.5,-0.7)--(4.5,-4.4);
				\draw[dashed] (7.0,-0.7)--(7.0,-4.4);	
			\end{tikzpicture}		
		\end{center}
		Dựa vào BBT ta có $2m\ge\dfrac{-t^2-1}{t},\,\,\,\forall t\in\left(7;11\right)\Leftrightarrow 2m\ge\underset{\left[7;11\right]}{\max}\,h(t)\Leftrightarrow m\ge-\dfrac{50}{14}$\\
		Vì $ m\in{\mathbb{Z}^-}\Rightarrow m \in \{-3;-2;-1\}$ .
	}
\end{ex}

\begin{ex}%[2D1G1-3]%Câu 14
	Cho hàm số $y=f(x)$ có bảng biến thiên như sau\\
	\begin{center}
		\begin{tikzpicture}[>=stealth,scale = 1]
			\tkzTabInit[lgt=1,espcl=2.5,nocadre]
			{$x$ /0.7, $y'$ /0.7,$y$ /2.5}
			{$-\infty$,$0$,$2$,$+\infty$}
			\tkzTabLine{ ,-,0,+,0,-,}
			\tkzTabVar{-/$-\infty$, +/$4$,- /$0$, +/{ $+\infty$}}
		\end{tikzpicture}
	\end{center}
	Có bao nhiêu số nguyên $m<2019$ để hàm số $g(x)=f\left(x^2-2x+m\right)$ đồng biến trên khoảng $\left(1;+\infty\right)$?
	\choice
	{\True $2016$}
	{$2015$}
	{$2017$}
	{$2018$}
	\loigiai{
		Ta có $g'(x)=\left(x^2-2x+m\right)'{f}'\left(x^2-2x+m\right)=2\left(x-1\right){f}'\left(x^2-2x+m\right)$ .\\
		Hàm số $y=g(x)$ đồng biến trên khoảng $\left(1;+\infty\right)$ khi và chỉ khi $g'(x)\ge 0,\forall x\in\left(1;+\infty\right)$ và\\
		$g'(x)=0$ tại hữu hạn điểm \\
		$\Leftrightarrow 2\left(x-1\right){f}'\left(x^2-2x+m\right)\ge 0,\forall x\in\left(1;+\infty\right)$\\
		$\Leftrightarrow{f}'\left(x^2-2x+m\right)\ge 0,\forall x\in\left(1;+\infty\right)$ $\Leftrightarrow\hoac{
			&{x^2}-2x+m\ge 2,\forall x\in\left(1;+\infty\right)\\ 
			&{x^2}-2x+m\le 0,\forall x\in\left(1;+\infty\right).}$\\
		Xét hàm số $y=x^2-2x+m$, ta có bảng biến thiên
		\begin{center}
			\begin{tikzpicture}[>=stealth,scale = 1]
				\tkzTabInit[lgt=1,espcl=2.5,nocadre]
				{$x$ /0.7, $y'$ /0.7,$y$ /2.5}
				{$-\infty$,$2$,$+\infty$}
				\tkzTabLine{ ,-,0,+,}
				\tkzTabVar{+/$+\infty$, -/$m-1$, +/{$+\infty$}}
			\end{tikzpicture}
		\end{center}
		Dựa vào bảng biến thiên ta có\\
		TH1: $x^2-2x+m\ge 2,\forall x\in\left(1;+\infty\right)\Leftrightarrow m-1\ge 2\Leftrightarrow m\ge 3$ .\\
		TH2: $x^2-2x+m\le 0,\forall x\in\left(1;+\infty\right)$. Không có giá trị $m$ thỏa mãn.\\
		Vậy có $2016$ số nguyên $m<2019$ thỏa mãn yêu cầu bài toán.}
\end{ex}

\begin{ex}%[2D1G1-3]%Câu 15
	\immini{
		Cho hàm số $ y=f(x)$ có đạo hàm là hàm số $f'(x)$ trên $\mathbb{R}$. Biết rằng hàm số $ y=f'\left(x-2\right)+2$ có đồ thị như hình vẽ bên dưới. Hàm số $ f(x)$ đồng biến trên khoảng nào?
		\choice
		{$\left(-\infty ;3\right),\,\,\left(5;+\infty\right)$}
		{\True $\left(-\infty ;-1\right),\,\,\left(1;+\infty\right)$}
		{$\left(-1;1\right)$}
		{$\left(3;5\right)$}}{
		\begin{tikzpicture}[scale=0.7,font=\footnotesize, line join=round, line cap=round, >=stealth] %Đường cong bậc 3
			\draw[thick, ->] (-0.5,0)--(3.5,0);
			\draw[thick, ->] (0,-1.8)--(0,5.3);
			\draw (3.7,0) node[below] {$x$};
			\draw (0,5.4) node[left]{$y$};
			\draw (0,0) node[below left]{$0$};
			\draw[fill] (3,0) circle (0.5pt)node[below]{$ 3$};
			\draw[fill] (1,0) circle (0.5pt)node[below]{$ 1$};
			\draw[fill] (2,0) circle (0.5pt)node[above]{$2$};
			\draw[fill] (0,2) circle (0.5pt)node[left]{$ 2$};
			\draw[fill] (0,-1) circle (0.5pt)node[left]{$ -1$};
			%		\draw[fill] (0,2) circle (0.5pt)node[above left]{$ 2$};
			%		\draw[fill] (0,-2) circle (0.5pt)node[below left]{$ -2$};
			\draw[dashed] (3,0)--(3,2)--(0,2)--(1,2)--(1,0); 
			\draw[dashed](0,-1)--(2,-1)--(2,0);
			\draw[line width=1.2pt,smooth,samples=100,domain=0.6:3.4] plot(\x,{3*(\x)^2-12*(\x)+11});		
			%\draw[line width=1.2pt,smooth,samples=100,domain=-3.3:2.8] plot(\x,{0.75*(\x)^2+0.5*\x-1});
			%	\draw (2.0,2.8) node[left]{$y=f'(x)$};
	\end{tikzpicture}	}
	\loigiai{	
		Hàm số $ y=f'\left(x-2\right)+2$ có đồ thị $(C)$ như sau:\\
		\begin{center}
			\begin{tikzpicture}[scale=0.7,font=\footnotesize, line join=round, line cap=round, >=stealth] %Đường cong bậc 3
				\draw[thick, ->] (-0.5,0)--(3.5,0);
				\draw[thick, ->] (0,-1.8)--(0,5.3);
				\draw (3.7,0) node[below] {$x$};
				\draw (0,5.4) node[left]{$y$};
				\draw (0,0) node[below left]{$0$};
				\draw[fill] (3,0) circle (0.5pt)node[below]{$ 3$};
				\draw[fill] (1,0) circle (0.5pt)node[below]{$ 1$};
				\draw[fill] (2,0) circle (0.5pt)node[above]{$2$};
				\draw[fill] (0,2) circle (0.5pt)node[left]{$ 2$};
				\draw[fill] (0,-1) circle (0.5pt)node[left]{$ -1$};
				%		\draw[fill] (0,2) circle (0.5pt)node[above left]{$ 2$};
				%		\draw[fill] (0,-2) circle (0.5pt)node[below left]{$ -2$};
				\draw[dashed] (3,0)--(3,2)--(0,2)--(1,2)--(1,0); 
				\draw[dashed](0,-1)--(2,-1)--(2,0);
				\draw[line width=1.2pt,smooth,samples=100,domain=0.6:3.4] plot(\x,{3*(\x)^2-12*(\x)+11});		
				%\draw[line width=1.2pt,smooth,samples=100,domain=-3.3:2.8] plot(\x,{0.75*(\x)^2+0.5*\x-1});
				%	\draw (2.0,2.8) node[left]{$y=f'(x)$};
			\end{tikzpicture}
		\end{center}
		Dựa vào đồ thị $(C)$ ta có\\ $f'\left(x-2\right)+2>2,\forall x\in\left(-\infty ;1\right)\cup\left(3;+\infty\right)\Leftrightarrow{f}'\left(x-2\right)>0,\forall x\in\left(-\infty ;1\right)\cup\left(3;+\infty\right)$ .\\
		Đặt $ x*=x-2$ suy ra $f'\left(x*\right)>0,\forall x*\in\left(-\infty ;-1\right)\bigcup\left(1;+\infty\right)$.\\
		Vậy hàm số $ f(x)$ đồng biến trên khoảng $\left(-\infty ;-1\right),\,\,\left(1;+\infty\right)$.}
\end{ex}

\begin{ex}%[2D1G1-2]%Câu 16
	\immini{
		Cho hàm số $ y=f(x)$ có đạo hàm là hàm số $f'(x)$ trên $\mathbb{R}$. Biết rằng hàm số $ y=f'\left(x+2\right)-2$ có đồ thị như hình vẽ bên dưới. Hàm số $ f(x)$ nghịch biến trên khoảng nào?
		\choice
		{$\left(-3;-1\right),\,\,\left(1;3\right)$}
		{\True $\left(-1;1\right),\,\,\left(3;5\right)$}
		{$\left(-\infty ;-2\right),\,\,\left(0;2\right)$}
		{$\left(-5;-3\right),\,\,\left(-1;1\right)$}}{
		\begin{tikzpicture}[scale=0.7,font=\footnotesize, line join=round, line cap=round, >=stealth] %Đường cong bậc 3
			\draw[thick, ->] (-3.8,0)--(4.0,0);
			\draw[thick, ->] (0,-4.8)--(0,3.5);
			\draw (4.2,0) node[below] {$x$};
			\draw (0,3.7) node[left]{$y$};
			\draw (0,0) node[below left]{$0$};
			\draw[fill] (-3,0) circle (0.5pt)node[above]{$ -3$};
			\draw[fill] (-1,0) circle (0.5pt)node[above]{$ -1$};
			\draw[fill] (1,0) circle (0.5pt)node[above]{$ 1$};
			\draw[fill] (3,0) circle (0.5pt)node[above]{$3$};
			\draw[fill] (0,2) circle (0.5pt)node[above left]{$ 2$};
			\draw[fill] (0,-1) circle (0.5pt)node[above right]{$ -1$};
			%		\draw[fill] (0,2) circle (0.5pt)node[above left]{$ 2$};
			%		\draw[fill] (0,-2) circle (0.5pt)node[below left]{$ -2$};
			\draw[dashed] (-3,0)--(-3,-2)--(3,-2)--(3,0) (-1,0)--(-1,-2) (1,0)--(1,-2) (-3.494,0)--(-3.494,2)--(3.494,2)--(3.494,0); 
			\draw[line width=1.2pt,smooth,samples=100,domain=-3.6:3.6] plot(\x,{0.11*(\x)^4-1.11*(\x)^2-1});		
			%\draw[line width=1.2pt,smooth,samples=100,domain=-3.3:2.8] plot(\x,{0.75*(\x)^2+0.5*\x-1});
			%	\draw (2.0,2.8) node[left]{$y=f'(x)$};
	\end{tikzpicture}	}
	\loigiai{
		Hàm số $ y=f'\left(x+2\right)-2$ có đồ thị $(C)$ như sau
		\begin{center}
			\begin{tikzpicture}[scale=0.7,font=\footnotesize, line join=round, line cap=round, >=stealth] %Đường cong bậc 3
				\draw[thick, ->] (-3.8,0)--(4.0,0);
				\draw[thick, ->] (0,-4.8)--(0,3.5);
				\draw (4.2,0) node[below] {$x$};
				\draw (0,3.7) node[left]{$y$};
				\draw (0,0) node[below left]{$0$};
				\draw[fill] (-3,0) circle (0.5pt)node[above]{$ -3$};
				\draw[fill] (-1,0) circle (0.5pt)node[above]{$ -1$};
				\draw[fill] (1,0) circle (0.5pt)node[above]{$ 1$};
				\draw[fill] (3,0) circle (0.5pt)node[above]{$3$};
				\draw[fill] (0,2) circle (0.5pt)node[above left]{$ 2$};
				\draw[fill] (0,-1) circle (0.5pt)node[above right]{$ -1$};
				%		\draw[fill] (0,2) circle (0.5pt)node[above left]{$ 2$};
				%		\draw[fill] (0,-2) circle (0.5pt)node[below left]{$ -2$};
				\draw[dashed] (-3,0)--(-3,-2)--(3,-2)--(3,0) (-1,0)--(-1,-2) (1,0)--(1,-2) (-3.494,0)--(-3.494,2)--(3.494,2)--(3.494,0); 
				\draw[line width=1.2pt,smooth,samples=100,domain=-3.6:3.6] plot(\x,{0.11*(\x)^4-1.11*(\x)^2-1});		
				%\draw[line width=1.2pt,smooth,samples=100,domain=-3.3:2.8] plot(\x,{0.75*(\x)^2+0.5*\x-1});
				%	\draw (2.0,2.8) node[left]{$y=f'(x)$};
			\end{tikzpicture}
		\end{center}
		Dựa vào đồ thị $(C)$ ta có\\
		$f'\left(x+2\right)-2<-2,\forall x\in\left(-3;-1\right)\bigcup\left(1;3\right)\Leftrightarrow{f}'\left(x+2\right)<0,\forall x\in\left(-3;-1\right)\bigcup\left(1;3\right)$.\\
		Đặt $ x^*=x+2$ suy ra: $f'\left(x^*\right)<0,\forall x^*\in\left(-1;1\right)\bigcup\left(3;5\right)$.\\
		Vậy: Hàm số $ f(x)$ đồng biến trên khoảng $\left(-1;1\right),\,\,\left(3;5\right)$.}
\end{ex}

\begin{ex}%[2D1G1-2]%Câu 17
	\immini{
		Cho hàm số $ y=f(x)$ có đạo hàm là hàm số $f'(x)$ trên $\mathbb{R}$. Biết rằng hàm số $ y=f'\left(x-2\right)+2$ có đồ thị như hình vẽ bên dưới. Hàm số $ f(x)$ nghịch biến trên khoảng nào?
		\choice
		{$\left(-\infty ;2\right)$}
		{\True $\left(-1;1\right)$}
		{$\left(\dfrac{3}{2};\dfrac{5}{2}\right)$}
		{$\left(2;+\infty\right)$}}{
		\begin{tikzpicture}[scale=0.7,font=\footnotesize, line join=round, line cap=round, >=stealth] %Đường cong bậc 3
			\draw[thick, ->] (-0.5,0)--(3.5,0);
			\draw[thick, ->] (0,-1.8)--(0,5.3);
			\draw (3.7,0) node[below] {$x$};
			\draw (0,5.4) node[left]{$y$};
			\draw (0,0) node[below left]{$0$};
			\draw[fill] (3,0) circle (0.5pt)node[below]{$ 3$};
			\draw[fill] (1,0) circle (0.5pt)node[below]{$ 1$};
			\draw[fill] (2,0) circle (0.5pt)node[above]{$2$};
			\draw[fill] (0,2) circle (0.5pt)node[left]{$ 2$};
			\draw[fill] (0,-1) circle (0.5pt)node[left]{$ -1$};
			%		\draw[fill] (0,2) circle (0.5pt)node[above left]{$ 2$};
			%		\draw[fill] (0,-2) circle (0.5pt)node[below left]{$ -2$};
			\draw[dashed] (3,0)--(3,2)--(0,2)--(1,2)--(1,0); 
			\draw[dashed](0,-1)--(2,-1)--(2,0);
			\draw[line width=1.2pt,smooth,samples=100,domain=0.6:3.4] plot(\x,{3*(\x)^2-12*(\x)+11});		
			%\draw[line width=1.2pt,smooth,samples=100,domain=-3.3:2.8] plot(\x,{0.75*(\x)^2+0.5*\x-1});
			%	\draw (2.0,2.8) node[left]{$y=f'(x)$};
	\end{tikzpicture}	}
	\loigiai{
		Hàm số $ y=f'\left(x-2\right)+2$ có đồ thị $(C)$ như sau
		\begin{center}
			\begin{tikzpicture}[scale=0.7,font=\footnotesize, line join=round, line cap=round, >=stealth] %Đường cong bậc 3
				\draw[thick, ->] (-0.5,0)--(3.5,0);
				\draw[thick, ->] (0,-1.8)--(0,5.3);
				\draw (3.7,0) node[below] {$x$};
				\draw (0,5.4) node[left]{$y$};
				\draw (0,0) node[below left]{$0$};
				\draw[fill] (3,0) circle (0.5pt)node[below]{$ 3$};
				\draw[fill] (1,0) circle (0.5pt)node[below]{$ 1$};
				\draw[fill] (2,0) circle (0.5pt)node[above]{$2$};
				\draw[fill] (0,2) circle (0.5pt)node[left]{$ 2$};
				\draw[fill] (0,-1) circle (0.5pt)node[left]{$ -1$};
				%		\draw[fill] (0,2) circle (0.5pt)node[above left]{$ 2$};
				%		\draw[fill] (0,-2) circle (0.5pt)node[below left]{$ -2$};
				\draw[dashed] (3,0)--(3,2)--(0,2)--(1,2)--(1,0); 
				\draw[dashed](0,-1)--(2,-1)--(2,0);
				\draw[line width=1.2pt,smooth,samples=100,domain=0.6:3.4] plot(\x,{3*(\x)^2-12*(\x)+11});		
				%\draw[line width=1.2pt,smooth,samples=100,domain=-3.3:2.8] plot(\x,{0.75*(\x)^2+0.5*\x-1});
				%	\draw (2.0,2.8) node[left]{$y=f'(x)$};
			\end{tikzpicture}
		\end{center}
		Dựa vào đồ thị $(C)$ ta có\\
		$f'\left(x-2\right)+2<2,\forall x\in\left(1;3\right)\Leftrightarrow{f}'\left(x-2\right)<0,\forall x\in\left(1;3\right)$.\\
		Đặt $ x^*=x-2$ thì $f'\left(x^*\right)<0,\forall x^*\in\left(-1;1\right)$.\\
		Vậy: Hàm số $ f(x)$ nghịch biến trên khoảng $\left(-1;1\right)$.\\
		Cách khác:\\
		Tịnh tiến sang trái hai đơn vị và xuống dưới $2$ đơn vị thì từ đồ thị $(C)$ sẽ thành đồ thị của hàm$ y=f'(x)$. Khi đó $f'(x)<0,\forall x\in\left(-1;1\right)$.\\
		Vậy hàm số $ f(x)$ nghịch biến trên khoảng $\left(-1;1\right)$.}
\end{ex}

\begin{ex}%[2D1G1-2]%Câu 18
	Cho hàm số $y=f(x)$ có đạo hàm cấp $ 3$ liên tục trên $\mathbb{R}$ và thỏa mãn $f(x)\cdot f'''(x)=x{\left(x-1\right)^2}{\left(x+4\right)^3}$ với mọi $x\in\mathbb{R}$ và $g(x)=\left[f'(x)\right]^2-2f(x)\cdot f''(x)$. Hàm số $h(x)=g\left(x^2-2x\right)$ đồng biến trên khoảng nào dưới đây?
	\choice
	{$\left(-\infty ;1\right)$}
	{$\left(2;+\infty\right)$}
	{$\left(0;1\right)$}
	{\True $\left(1;2\right)$}
	\loigiai{		
		Ta có $g'(x)=2f''(x){f}'(x)-2f'(x)\cdot f''(x)-2f(x)\cdot f'''(x)=-2f(x)\cdot f'''(x);$\\
		Khi đó $\left(h(x)\right)'=\left(2x-2\right){g}'\left(x^2-2x\right)=-2\left(2x-2\right)\left(x^2-2x\right){\left(x^2-2x-1\right)^2}{\left(x^2-2x+4\right)^3}$\\
		$h'(x)=0\Leftrightarrow\hoac{
			& x=0\\ 
			& x=1\\ 
			& x=2\\ 
			& x=1\pm\sqrt{2}.}$ 
		Ta có bảng xét dấu của $h'(x)$
		\begin{center}
			\begin{tikzpicture}
				\tkzTabInit[lgt=1.2,espcl=2,nocadre]
				{$t$/0.7, $h'(x)$ /.7} % first column
				{$-\infty$, $1-\sqrt{2}$,$0$, $1$,$2$,$1+\sqrt{2}$, $+\infty$} % first row
				\tkzTabLine { ,+,0,-,0,+,0,-,0,+,0,- } % second row
				%				\tkzTabLine {,-,z,+,t,+,} % third row
				%				\tkzTabLine {,+,d,-,z,+,} % last row
			\end{tikzpicture}
		\end{center}
		Suy ra hàm số $h(x)=g\left(x^2-2x\right)$ đồng biến trên khoảng $\left(1;2\right)$.}
\end{ex}

\begin{ex}%[2D1G1-2]%Câu 19
	Cho hàm số $ y=f(x)$ xác định trên $\mathbb{R}$. Hàm số $ y=g(x)=f'\left(2x+3\right)+2$ có đồ thị là một parabol với tọa độ đỉnh $ I\left(2;-1\right)$ và đi qua điểm $ A\left(1;2\right)$. Hỏi hàm số $ y=f(x)$ nghịch biến trên khoảng nào dưới đây?
	\choice
	{\True $\left(5;9\right)$}
	{$\left(1;2\right)$}
	{$\left(-\infty ;9\right)$}
	{$\left(1;3\right)$}
	\loigiai{	
		Xét hàm số $ g(x)=f'\left(2x+3\right)+2$ có đồ thị là một Parabol nên có phương trình dạng $ y=g(x)=a{x^2}+bx+c\,\,\,\,(P)$.\\
		Vì $(P)$ có đỉnh $ I\left(2;-1\right)$ nên $\heva{
			&\dfrac{-b}{2a}=2\\ 
			& g(2)=-1} \Leftrightarrow\heva{
			&-b=4a\\ 
			& 4a+2b+c=-1} \Leftrightarrow\heva{
			& 4a+b=0\\ 
			& 4a+2b+c=-1}$.\\
		Vì $(P)$ đi qua điểm $ A\left(1;2\right)$ nên $ g(1)=2\Leftrightarrow a+b+c=2$.\\
		Ta có hệ phương trình $\heva{
			& 4a+b=0\\ 
			& 4a+2b+c=-1\\ 
			& a+b+c=2} \Leftrightarrow\heva{
			& a=3\\ 
			& b=-12\\ 
			& c=11}$ nên $ g(x)=3x^2-12x+11$.\\
		Đồ thị của hàm $ y=g(x)$ là
		\begin{center}
			\begin{tikzpicture}[scale=0.7,font=\footnotesize, line join=round, line cap=round, >=stealth] %Đường cong bậc 3
				\draw[thick, ->] (-0.5,0)--(3.5,0);
				\draw[thick, ->] (0,-1.8)--(0,5.3);
				\draw (3.7,0) node[below] {$x$};
				\draw (0,5.4) node[left]{$y$};
				\draw (0,0) node[below left]{$0$};
				\draw[fill] (3,0) circle (0.5pt)node[below]{$ 3$};
				\draw[fill] (1,0) circle (0.5pt)node[below]{$ 1$};
				\draw[fill] (2,0) circle (0.5pt)node[above]{$2$};
				\draw[fill] (0,2) circle (0.5pt)node[left]{$ 2$};
				\draw[fill] (0,-1) circle (0.5pt)node[left]{$ -1$};
				%		\draw[fill] (0,2) circle (0.5pt)node[above left]{$ 2$};
				%		\draw[fill] (0,-2) circle (0.5pt)node[below left]{$ -2$};
				\draw[dashed] (3,0)--(3,2)--(0,2)--(1,2)--(1,0) (3.2,2)--(3,2); 
				\draw[dashed](0,-1)--(2,-1)--(2,0);
				\draw[line width=1.2pt,smooth,samples=100,domain=0.6:3.4] plot(\x,{3*(\x)^2-12*(\x)+11});		
				%\draw[line width=1.2pt,smooth,samples=100,domain=-3.3:2.8] plot(\x,{0.75*(\x)^2+0.5*\x-1});
				%	\draw (2.0,2.8) node[left]{$y=f'(x)$};
			\end{tikzpicture}	
		\end{center}
		Theo đồ thị ta thấy $ f'(2x+3)\le 0\Leftrightarrow f'(2x+3)+2\le 2\Leftrightarrow 1\le x\le 3$.\\
		Đặt $ t=2x+3\Leftrightarrow x=\dfrac{t-3}{2}$ khi đó $ f'(t)\le 0\Leftrightarrow 1\le\dfrac{t-3}{2}\le 3\Leftrightarrow 5\le t\le 9$.\\
		Vậy $ y=f(x)$ nghịch biến trên khoảng $\left(5;9\right)$.}
\end{ex}

\begin{ex}%[2D1G1-2]%Câu 20
	\immini{
		Cho hàm số $ y=f(x)$, hàm số $f'(x)=x^3+a{x^2}+bx+c\left(a,b,c\in\mathbb{R}\right)$ có đồ thị như hình vẽ bên.
		Hàm số $ g(x)=f\left(f'(x)\right)$ nghịch biến trên khoảng nào dưới đây?
		\choice
		{$\left(1;+\infty\right)$}
		{\True $\left(-\infty ;-2\right)$}
		{$\left(-1;0\right)$}
		{$\left(-\dfrac{\sqrt{3}}{3};\dfrac{\sqrt{3}}{3}\right)$}}{
		\begin{tikzpicture}[scale=0.8,font=\footnotesize, line join=round, line cap=round, >=stealth] %Đường cong bậc 3
			\draw[thick, ->] (-1.7,0)--(1.7,0);
			\draw[thick, ->] (0,-2.7)--(0,3.0);
			\draw (1.9,0) node[below] {$x$};
			\draw (0,3.2) node[left]{$y$};
			\draw (0,0) node[below left]{$0$};
			\draw[fill] (-1,0) circle (0.5pt)node[above left]{$ -1 $};
			\draw[fill] (1,0) circle (0.5pt)node[below right]{$ 1$};
			\draw[line width=1.2pt,smooth,samples=100,domain=-1.3:1.3] plot(\x,{2.667*(\x)^3+0*(\x)^2-2.667*\x});		
			%\draw[line width=1.2pt,smooth,samples=100,domain=-3.3:2.8] plot(\x,{0.75*(\x)^2+0.5*\x-1});
		\end{tikzpicture}	
	}
	\loigiai{	
		Vì các điểm $\left(-1;0\right),\left(0;0\right),\left(1;0\right)$ thuộc đồ thị hàm số $ y=f'(x)$ nên ta có hệ\\
		$\heva{
			&-1+a-b+c=0\\ 
			& c=0\\ 
			& 1+a+b+c=0} \Leftrightarrow\heva{
			& a=0\\ 
			& b=-1\\ 
			& c=0} \Rightarrow {f}'(x)=x^3-x\Rightarrow f''(x)=3x^2-1$.\\
		Ta có $ g(x)=f\left(f'(x)\right)\Rightarrow{g}'(x)=f'\left(f'(x)\right)\cdot f''(x)$.\\
		Xét \\
		$g'(x)=0\Leftrightarrow{g}'(x)=f'\left(f'(x)\right)\cdot f''(x)=0$\\
		$\Leftrightarrow {f}'\left(x^3-x\right)\left(3x^2-1\right)=0\Leftrightarrow\hoac{
			&{x^3}-x=0\\ 
			&{x^3}-x=1\\ 
			&{x^3}-x=-1\\ 
			& 3x^2-1=0} \Leftrightarrow \hoac{
			& x=\pm 1\\ 
			& x=0\\ 
			& x=x_1(x_1\approx 1,325)\\ 
			& x=x_2(x_2\approx-1,325)\\ 
			& x=\pm\dfrac{\sqrt{3}}{3}.}$\\
		Bảng biến thiên
		\begin{center}
			\begin{tikzpicture}
				\tkzTabInit[lgt=1.2,espcl=2,nocadre]
				{$t$/0.7, $h'(x)$ /.7} % first column
				{$-\infty$, $-1{,}325$,$-1$, $-\dfrac{\sqrt{3}}{3}$,$0$,$\dfrac{\sqrt{3}}{3}$,$1$,$1{,}325$, $+\infty$} % first row
				\tkzTabLine { ,-,0,+,0,-,0,+,0,-,0,+,0,-,0,+, } % second row
				%				\tkzTabLine {,-,z,+,t,+,} % third row
				%				\tkzTabLine {,+,d,-,z,+,} % last row
			\end{tikzpicture}
		\end{center}
		Dựa vào bảng biến thiên ta có $ g(x)$ nghịch biến trên $\left(-\infty ;-2\right)$}
\end{ex}
\Closesolutionfile{ans}
\indapan{10}{ans/CD1/Muc_9_10}
\chapter{LŨY THỪA, HÀM SỐ LŨY THỪA}
\begin{Solution}{1}
C
\end{Solution}
\begin{Solution}{3}
B
\end{Solution}
\begin{Solution}{4}
A
\end{Solution}
\begin{Solution}{5}
A
\end{Solution}
\begin{Solution}{6}
A
\end{Solution}
\begin{Solution}{7}
B
\end{Solution}
\begin{Solution}{8}
A
\end{Solution}
\begin{Solution}{9}
C
\end{Solution}
\begin{Solution}{10}
B
\end{Solution}
\begin{Solution}{11}
C
\end{Solution}
\begin{Solution}{12}
D
\end{Solution}
\begin{Solution}{13}
B
\end{Solution}
\begin{Solution}{14}
D
\end{Solution}
\begin{Solution}{15}
A
\end{Solution}
\begin{Solution}{16}
B
\end{Solution}
\begin{Solution}{17}
C
\end{Solution}
\begin{Solution}{18}
C
\end{Solution}
\begin{Solution}{19}
C
\end{Solution}
\begin{Solution}{20}
B
\end{Solution}
\begin{Solution}{21}
C
\end{Solution}
\begin{Solution}{22}
B
\end{Solution}
\begin{Solution}{23}
D
\end{Solution}
\begin{Solution}{24}
B
\end{Solution}
\begin{Solution}{25}
D
\end{Solution}
\begin{Solution}{26}
D
\end{Solution}
\begin{Solution}{27}
B
\end{Solution}
\begin{Solution}{28}
A
\end{Solution}
\begin{Solution}{29}
C
\end{Solution}
\begin{Solution}{30}
B
\end{Solution}
\begin{Solution}{31}
D
\end{Solution}
\begin{Solution}{32}
B
\end{Solution}
\begin{Solution}{33}
B
\end{Solution}
\begin{Solution}{34}
C
\end{Solution}
\begin{Solution}{35}
D
\end{Solution}
\begin{Solution}{36}
B
\end{Solution}
\begin{Solution}{37}
B
\end{Solution}
\begin{Solution}{38}
A
\end{Solution}
\begin{Solution}{39}
A
\end{Solution}
\begin{Solution}{40}
D
\end{Solution}
\begin{Solution}{41}
C
\end{Solution}
\begin{Solution}{42}
B
\end{Solution}
\begin{Solution}{43}
A
\end{Solution}
\begin{Solution}{44}
A
\end{Solution}
\begin{Solution}{45}
D
\end{Solution}
\begin{Solution}{46}
C
\end{Solution}
\begin{Solution}{47}
A
\end{Solution}
\begin{Solution}{48}
B
\end{Solution}
\begin{Solution}{49}
B
\end{Solution}
\begin{Solution}{50}
B
\end{Solution}
\begin{Solution}{51}
A
\end{Solution}
\begin{Solution}{52}
A
\end{Solution}
\begin{Solution}{53}
C
\end{Solution}
\begin{Solution}{54}
C
\end{Solution}
\begin{Solution}{55}
C
\end{Solution}
\begin{Solution}{56}
B
\end{Solution}
\begin{Solution}{57}
C
\end{Solution}
\begin{Solution}{58}
C
\end{Solution}
\begin{Solution}{59}
B
\end{Solution}
\begin{Solution}{60}
C
\end{Solution}
\begin{Solution}{61}
A
\end{Solution}
\begin{Solution}{62}
B
\end{Solution}
\begin{Solution}{63}
B
\end{Solution}
\begin{Solution}{64}
D
\end{Solution}
\begin{Solution}{65}
D
\end{Solution}
\begin{Solution}{66}
B
\end{Solution}
\begin{Solution}{67}
A
\end{Solution}
\begin{Solution}{68}
D
\end{Solution}

\chapter{CÔNG THỨC, BIẾN ĐỔI LOGARIT}
\begin{Solution}{1}
C
\end{Solution}
\begin{Solution}{3}
B
\end{Solution}
\begin{Solution}{4}
A
\end{Solution}
\begin{Solution}{5}
A
\end{Solution}
\begin{Solution}{6}
A
\end{Solution}
\begin{Solution}{7}
B
\end{Solution}
\begin{Solution}{8}
A
\end{Solution}
\begin{Solution}{9}
C
\end{Solution}
\begin{Solution}{10}
B
\end{Solution}
\begin{Solution}{11}
C
\end{Solution}
\begin{Solution}{12}
D
\end{Solution}
\begin{Solution}{13}
B
\end{Solution}
\begin{Solution}{14}
D
\end{Solution}
\begin{Solution}{15}
A
\end{Solution}
\begin{Solution}{16}
B
\end{Solution}
\begin{Solution}{17}
C
\end{Solution}
\begin{Solution}{18}
C
\end{Solution}
\begin{Solution}{19}
C
\end{Solution}
\begin{Solution}{20}
B
\end{Solution}
\begin{Solution}{21}
C
\end{Solution}
\begin{Solution}{22}
B
\end{Solution}
\begin{Solution}{23}
D
\end{Solution}
\begin{Solution}{24}
B
\end{Solution}
\begin{Solution}{25}
D
\end{Solution}
\begin{Solution}{26}
D
\end{Solution}
\begin{Solution}{27}
B
\end{Solution}
\begin{Solution}{28}
A
\end{Solution}
\begin{Solution}{29}
C
\end{Solution}
\begin{Solution}{30}
B
\end{Solution}
\begin{Solution}{31}
D
\end{Solution}
\begin{Solution}{32}
B
\end{Solution}
\begin{Solution}{33}
B
\end{Solution}
\begin{Solution}{34}
C
\end{Solution}
\begin{Solution}{35}
D
\end{Solution}
\begin{Solution}{36}
B
\end{Solution}
\begin{Solution}{37}
B
\end{Solution}
\begin{Solution}{38}
A
\end{Solution}
\begin{Solution}{39}
A
\end{Solution}
\begin{Solution}{40}
D
\end{Solution}
\begin{Solution}{41}
C
\end{Solution}
\begin{Solution}{42}
B
\end{Solution}
\begin{Solution}{43}
A
\end{Solution}
\begin{Solution}{44}
A
\end{Solution}
\begin{Solution}{45}
D
\end{Solution}
\begin{Solution}{46}
C
\end{Solution}
\begin{Solution}{47}
A
\end{Solution}
\begin{Solution}{48}
B
\end{Solution}
\begin{Solution}{49}
B
\end{Solution}
\begin{Solution}{50}
B
\end{Solution}
\begin{Solution}{51}
A
\end{Solution}
\begin{Solution}{52}
A
\end{Solution}
\begin{Solution}{53}
C
\end{Solution}
\begin{Solution}{54}
C
\end{Solution}
\begin{Solution}{55}
C
\end{Solution}
\begin{Solution}{56}
B
\end{Solution}
\begin{Solution}{57}
C
\end{Solution}
\begin{Solution}{58}
C
\end{Solution}
\begin{Solution}{59}
B
\end{Solution}
\begin{Solution}{60}
C
\end{Solution}
\begin{Solution}{61}
A
\end{Solution}
\begin{Solution}{62}
B
\end{Solution}
\begin{Solution}{63}
B
\end{Solution}
\begin{Solution}{64}
D
\end{Solution}
\begin{Solution}{65}
D
\end{Solution}
\begin{Solution}{66}
B
\end{Solution}
\begin{Solution}{67}
A
\end{Solution}
\begin{Solution}{68}
D
\end{Solution}

\begin{Solution}{1}
D
\end{Solution}
\begin{Solution}{2}
C
\end{Solution}
\begin{Solution}{3}
C
\end{Solution}
\begin{Solution}{4}
A
\end{Solution}
\begin{Solution}{5}
B
\end{Solution}
\begin{Solution}{6}
D
\end{Solution}
\begin{Solution}{7}
C
\end{Solution}
\begin{Solution}{8}
D
\end{Solution}
\begin{Solution}{9}
A
\end{Solution}
\begin{Solution}{10}
B
\end{Solution}
\begin{Solution}{11}
D
\end{Solution}
\begin{Solution}{12}
A
\end{Solution}
\begin{Solution}{13}
D
\end{Solution}
\begin{Solution}{14}
B
\end{Solution}
\begin{Solution}{15}
B
\end{Solution}
\begin{Solution}{16}
C
\end{Solution}
\begin{Solution}{1}
A
\end{Solution}
\begin{Solution}{2}
B
\end{Solution}
\begin{Solution}{3}
D
\end{Solution}
\begin{Solution}{4}
D
\end{Solution}
\begin{Solution}{5}
C
\end{Solution}
\begin{Solution}{6}
A
\end{Solution}
\begin{Solution}{7}
D
\end{Solution}
\begin{Solution}{8}
B
\end{Solution}
\begin{Solution}{9}
C
\end{Solution}
\begin{Solution}{10}
C
\end{Solution}
\begin{Solution}{1}
D
\end{Solution}
\begin{Solution}{2}
D
\end{Solution}
\begin{Solution}{3}
B
\end{Solution}
\begin{Solution}{4}
C
\end{Solution}
\begin{Solution}{5}
D
\end{Solution}
\begin{Solution}{6}
A
\end{Solution}
\begin{Solution}{7}
C
\end{Solution}
\begin{Solution}{8}
B
\end{Solution}
\begin{Solution}{9}
A
\end{Solution}
\begin{Solution}{10}
C
\end{Solution}
\begin{Solution}{11}
D
\end{Solution}
\begin{Solution}{12}
C
\end{Solution}
\begin{Solution}{13}
A
\end{Solution}
\begin{Solution}{14}
D
\end{Solution}
\begin{Solution}{15}
A
\end{Solution}
\begin{Solution}{16}
A
\end{Solution}
\begin{Solution}{17}
B
\end{Solution}
\begin{Solution}{18}
C
\end{Solution}
\begin{Solution}{19}
C
\end{Solution}
\begin{Solution}{20}
A
\end{Solution}
\begin{Solution}{21}
D
\end{Solution}
\begin{Solution}{22}
C
\end{Solution}
\begin{Solution}{23}
A
\end{Solution}
\begin{Solution}{24}
C
\end{Solution}
\begin{Solution}{25}
A
\end{Solution}
\begin{Solution}{26}
B
\end{Solution}
\begin{Solution}{27}
B
\end{Solution}
\begin{Solution}{28}
D
\end{Solution}
\begin{Solution}{29}
B
\end{Solution}
\begin{Solution}{30}
D
\end{Solution}
\begin{Solution}{31}
D
\end{Solution}
\begin{Solution}{32}
C
\end{Solution}
\begin{Solution}{33}
D
\end{Solution}
\begin{Solution}{34}
C
\end{Solution}
\begin{Solution}{35}
D
\end{Solution}
\begin{Solution}{36}
D
\end{Solution}
\begin{Solution}{37}
D
\end{Solution}
\begin{Solution}{38}
D
\end{Solution}
\begin{Solution}{39}
D
\end{Solution}
\begin{Solution}{40}
C
\end{Solution}
\begin{Solution}{41}
A
\end{Solution}
\begin{Solution}{1}
A
\end{Solution}
\begin{Solution}{2}
B
\end{Solution}
\begin{Solution}{3}
C
\end{Solution}
\begin{Solution}{4}
A
\end{Solution}
\begin{Solution}{5}
A
\end{Solution}
\begin{Solution}{6}
C
\end{Solution}
\begin{Solution}{7}
C
\end{Solution}
\begin{Solution}{8}
B
\end{Solution}
\begin{Solution}{9}
C
\end{Solution}
\begin{Solution}{10}
B
\end{Solution}
\begin{Solution}{11}
A
\end{Solution}
\begin{Solution}{12}
B
\end{Solution}
\begin{Solution}{13}
B
\end{Solution}
\begin{Solution}{14}
B
\end{Solution}
\begin{Solution}{15}
A
\end{Solution}
\begin{Solution}{16}
B
\end{Solution}
\begin{Solution}{17}
A
\end{Solution}
\begin{Solution}{18}
D
\end{Solution}
\begin{Solution}{19}
C
\end{Solution}
\begin{Solution}{20}
C
\end{Solution}
\begin{Solution}{21}
A
\end{Solution}
\begin{Solution}{22}
C
\end{Solution}
\begin{Solution}{23}
C
\end{Solution}
\begin{Solution}{24}
A
\end{Solution}
\begin{Solution}{25}
B
\end{Solution}
\begin{Solution}{26}
B
\end{Solution}
\begin{Solution}{27}
A
\end{Solution}
\begin{Solution}{28}
A
\end{Solution}
\begin{Solution}{29}
C
\end{Solution}
\begin{Solution}{30}
B
\end{Solution}
\begin{Solution}{31}
A
\end{Solution}
\begin{Solution}{32}
C
\end{Solution}
\begin{Solution}{33}
B
\end{Solution}
\begin{Solution}{34}
A
\end{Solution}
\begin{Solution}{35}
B
\end{Solution}
\begin{Solution}{36}
B
\end{Solution}
\begin{Solution}{37}
B
\end{Solution}
\begin{Solution}{38}
D
\end{Solution}
\begin{Solution}{39}
B
\end{Solution}
\begin{Solution}{40}
A
\end{Solution}
\begin{Solution}{41}
D
\end{Solution}
\begin{Solution}{42}
D
\end{Solution}
\begin{Solution}{43}
A
\end{Solution}
\begin{Solution}{44}
D
\end{Solution}
\begin{Solution}{45}
C
\end{Solution}
\begin{Solution}{46}
B
\end{Solution}
\begin{Solution}{47}
A
\end{Solution}
\begin{Solution}{48}
D
\end{Solution}
\begin{Solution}{49}
B
\end{Solution}
\begin{Solution}{50}
B
\end{Solution}
\begin{Solution}{51}
D
\end{Solution}
\begin{Solution}{52}
C
\end{Solution}
\begin{Solution}{53}
C
\end{Solution}
\begin{Solution}{54}
B
\end{Solution}
\begin{Solution}{55}
D
\end{Solution}
\begin{Solution}{56}
B
\end{Solution}
\begin{Solution}{57}
C
\end{Solution}
\begin{Solution}{58}
A
\end{Solution}
\begin{Solution}{59}
A
\end{Solution}
\begin{Solution}{60}
B
\end{Solution}
\begin{Solution}{61}
D
\end{Solution}
\begin{Solution}{62}
D
\end{Solution}
\begin{Solution}{63}
B
\end{Solution}
\begin{Solution}{64}
A
\end{Solution}
\begin{Solution}{65}
D
\end{Solution}
\begin{Solution}{66}
C
\end{Solution}
\begin{Solution}{67}
A
\end{Solution}
\begin{Solution}{68}
A
\end{Solution}
\begin{Solution}{69}
D
\end{Solution}
\begin{Solution}{70}
C
\end{Solution}
\begin{Solution}{71}
B
\end{Solution}
\begin{Solution}{72}
A
\end{Solution}
\begin{Solution}{73}
C
\end{Solution}
\begin{Solution}{74}
C
\end{Solution}
\begin{Solution}{75}
C
\end{Solution}
\begin{Solution}{76}
A
\end{Solution}
\begin{Solution}{77}
C
\end{Solution}
\begin{Solution}{78}
B
\end{Solution}
\begin{Solution}{79}
D
\end{Solution}
\begin{Solution}{80}
B
\end{Solution}

\section{Mức 9,10 điểm}
\setcounter{ex}{0}
\setcounter{dang}{0}
\Opensolutionfile{ans}[ans/CD1/Muc_9_10]
\begin{dang}{Tìm m để hàm số đơn điệu trên các khoảng xác định của nó}
	Đang thiếu bài thầy Jf Câu 1 đến 26 
\end{dang}
\begin{dang}
	{Tìm khoảng đơn điệu của hàm số $g(x) = f\left[ u(x)\right] +v(x)$ khi biết đồ thị hoặc bảng biến thiên của hàm số $y = f'(x)$}
\end{dang}
\begin{ex}[Đề tham khảo 2019]%[2D1K1-2]
	Cho hàm số $f(x)$ có bảng xét dấu của đạo hàm như sau
	\begin{center}
		\begin{tikzpicture}
			\tkzTabInit[nocadre,lgt=1.2,espcl=2,deltacl=0.6]
			{$x$ /0.6,$f'(x)$ /0.6}
			{$-\infty$,$1$,$2$,$3$,$4$,$+\infty$}
			\tkzTabLine{,-,$0$,+,$0$,+,$0$,-,$0$,+,}
		\end{tikzpicture}
	\end{center}
	Hàm số $y=3 f(x+2)-x^3+3 x$ đồng biến trên khoảng nào dưới đây?
	\choice
	{$(-\infty ;-1)$}
	{\True $(-1 ; 0)$}
	{$(0 ; 2)$}
	{$(1 ;+\infty)$}
	\loigiai{
		Ta có $y'=3\left[f'(x+2)-\left(x^2-3\right)\right]$.\\
		Với $x \in(-1 ; 0) \Rightarrow x+2 \in(1 ; 2) \Rightarrow f'(x+2)>0$, lại có $x^2-3<0 \Rightarrow y'>0 ;~ \forall x \in(-1 ; 0)$.\\
		Vậy hàm số $y=3 f(x+2)-x^3+3 x$ đồng biến trên khoảng $(-1 ; 0)$.\\
		Chú ý:\\
		+) Ta xét $x \in(1 ; 2) \subset(1 ;+\infty)
		\Rightarrow x+2 \in(3 ; 4)\\
		\Rightarrow f'(x+2)<0 ;~ x^2-3>0$\\
		Suy ra hàm số nghịch biến trên khoảng $(1 ; 2)$ nên loại hai phương án$(0 ; 2)$ và $(1 ;+\infty)$.\\
		+) Tương tự ta xét
		$x \in(-\infty ;-2) \Rightarrow x+2 \in(-\infty ; 0)\\
		\Rightarrow f'(x+2)<0 ; x^2-3>0 \Rightarrow y'<0 ; ~ \forall x \in(-\infty ;-2)$.\\
		Suy ra hàm số nghịch biến trên khoảng $(-\infty ;-2)$ nên loại$(-\infty ;-1)$.\\
		Vậy hàm số đã cho đồng biến trên khoảng $(-1 ; 0)$.
	}
\end{ex}
\begin{ex}[Đề Tham Khảo 2020 - Lần 1]%[2D1G1-2]
	\immini{
		Cho hàm số $f(x)$. Hàm số $y=f'(x)$ có đồ thị như hình bên. Hàm số $g(x)=f(1-2 x)+x^2-x$ nghịch biến trên khoảng nào dưới đây?
		\choice
		{\True $\left(1 ; \dfrac{3}{2}\right)$}
		{$\left(0 ; \dfrac{1}{2}\right)$}
		{$(-2 ;-1)$}
		{$(2 ; 3)$}
	}
	{
		\begin{tikzpicture}[scale=0.7,>=stealth, font=\footnotesize, line join=round, line cap=round]
			%\def\a{1} \def\b{-6} \def\c{9} \def\d{1} % Hệ số
			\def\xmin{-4} \def\xmax{6}
			\def\ymin{-3} \def\ymax{2} 
			%\draw[color=gray!50,dashed] (\xmin,\ymin) grid (\xmax,\ymax); 
			\draw[->] (\xmin,0)--(\xmax,0) node [below]{$x$};
			\draw[->] (0,\ymin)--(0,\ymax) node [left]{$y$};
			\node at (0,0) [below left]{$O$};
			%\node at (1,3) [below left]{$f'(x)$};
			%\node at (-1.3,4) {$f'(x)$};
			\draw[dashed] (-2,0) node[below]{$-2$}--(-2,1)--(0,1) node[below left]{$1$};
			\draw[dashed] (4,0) node[below left]{$4$}--(4,-2)--(0,-2) node[below left]{$-2$};
			%\draw[dashed] (1,0) node[below]{$1$}--(1,1);
			%\draw[dashed] (-0.5,0) node[below left]{$-0{,}5$}--(-0.5,2.125);
			\clip (\xmin+0.1,\ymin+0.1) rectangle (\xmax-0.5,\ymax-0.1);
			\draw[smooth,samples=300][domain=-4:5.5] plot(\x,{0.071*(\x)^3-0.142*(\x)^2-1.07*(\x)});
		\end{tikzpicture}
	}
	
	\loigiai{
		Ta có : $g(x)=f(1-2 x)+x^2-x \Rightarrow g'(x)=-2 f'(1-2 x)+2 x-1$.\\
		\immini{
			Đặt $t=1-2 x \Rightarrow g'(x)=-2 f'(t)-t$.\\
			$g'(x)=0 \Rightarrow f'(t)=-\dfrac{t}{2}$.\\
			Vẽ đường thẳng $y=-\dfrac{x}{2}$ và đồ thị hàm số $f'(x)$ trên cùng một hệ trục
		}	
		{
			\begin{tikzpicture}[scale=0.7,>=stealth, font=\footnotesize, line join=round, line cap=round]
				%\def\a{1} \def\b{-6} \def\c{9} \def\d{1} % Hệ số
				\def\xmin{-4} \def\xmax{6}
				\def\ymin{-3} \def\ymax{2} 
				%	\draw[color=gray!50,dashed] (\xmin,\ymin) grid (\xmax,\ymax); 
				\draw[->] (\xmin,0)--(\xmax,0) node [below]{$x$};
				\draw[->] (0,\ymin)--(0,\ymax) node [left]{$y$};
				\node at (0,0) [below left]{$O$};
				%\node at (1,3) [below left]{$f'(x)$};
				%\node at (-1.3,4) {$f'(x)$};
				\draw[dashed] (-2,0) node[below]{$-2$}--(-2,1)--(0,1) node[below left]{$1$};
				\draw[dashed] (4,0) node[below]{$4$}--(4,-2)--(0,-2) node[below left]{$-2$};
				%\draw[dashed] (1,0) node[below]{$1$}--(1,1);
				%\draw[dashed] (-0.5,0) node[below left]{$-0{,}5$}--(-0.5,2.125);
				\clip (\xmin+0.1,\ymin+0.1) rectangle (\xmax-0.5,\ymax-0.1);
				\draw[smooth,samples=300][domain=-4:5.5] plot(\x,{0.071*(\x)^3-0.142*(\x)^2-1.07*(\x)});
				\draw[smooth,samples=300][domain=-4:5.5] plot(\x,{(-0.5*(\x)});
			\end{tikzpicture}
		}	Hàm số $g(x)$ nghịch biến $\Rightarrow g'(x) \leq 0 \Rightarrow f'(t) \geq-\dfrac{t}{2}\Rightarrow\hoac{&-2 \leq t \leq 0 \\&t \geq 4.}$\\
		Như vậy $f'(1-2 x) \geq \dfrac{1-2 x}{-2}\Rightarrow\hoac{&-2 \leq 1-2 x \leq 0 \\ &4 \leq 1-2 x}\Rightarrow\hoac{&\dfrac{1}{2}\leq x \leq \dfrac{3}{2}\\ &x \leq-\dfrac{3}{2}.}$\\
		Vậy hàm số $g(x)=f(1-2 x)+x^2-x$ nghịch biến trên các khoảng $\left(\dfrac{1}{2}; \dfrac{3}{2}\right)$ và $\left(-\infty ;-\dfrac{3}{2}\right)$.\\
		Mà $\left(1 ; \dfrac{3}{2}\right) \subset \left(\dfrac{1}{2}; \dfrac{3}{2}\right)$ nên hàm số $g(x)=f(1-2 x)+x^2-x$ nghịch biến trên khoảng $\left(1 ; \dfrac{3}{2}\right)$.
	}
\end{ex}
\begin{ex}[Chuyên Lê Quý Đôn Điện Biên 2019]%[2D1G1-2]
	Cho hàm số $f(x)$ có bảng xét dấu của đạo hàm như sau
	\begin{center}
		\begin{tikzpicture}
			\tkzTabInit[nocadre,lgt=1.2,espcl=2,deltacl=0.6]
			{$x$ /0.6,$f'(x)$ /0.6}
			{$-\infty$,$0$,$1$,$2$,$3$,$+\infty$}
			\tkzTabLine{,+,$0$,-,$0$,-,$0$,+,$0$,-,}
		\end{tikzpicture}
	\end{center}
	Hàm số $y=f(x-1)+x^3-12 x+2019$ nghịch biến trên khoảng nào dưới đây?
	\choice
	{$(1 ;+\infty)$}
	{\True $(1 ; 2)$}
	{$(-\infty ; 1)$}
	{$(3 ; 4)$}
	\loigiai{
		$y'=f'(x-1)+3 x^2-12=f'(t)+3 t^2+6 t-9=f'(t)-\left(-3 t^2-6 t+9\right)$, với $t=x-1$.\\
		\immini{
			Nghiệm của phương trình $y'=0$ là hoành độ giao điểm của các đồ thị hàm số $y=f'(t)$ và $y=-3 t^2-6 t+9$.\\
			Vẽ đồ thị hàm số $y=f'(t)$ và $y=-3 t^2-6 t+9$ trên cùng một hệ trục tọa độ như hình vẽ bên.
		}	
		{		\begin{tikzpicture}[scale=0.5,>=stealth, font=\footnotesize, line join=round, line cap=round]
				\def\a{-3} \def\b{-6} \def\c{9} % Hệ số
				\def\xmin{-9} \def\xmax{7}
				\def\ymin{-3} \def\ymax{13}
				
				%\draw[color=gray!50,dashed] (\xmin,\ymin) grid (\xmax,\ymax);
				
				\draw[->] (\xmin,0)--(\xmax,0) node [below]{$x$};
				\draw[->] (0,\ymin)--(0,\ymax) node [left]{$y$};
				\node at (0,0) [below left]{$O$};
				\clip (\xmin+0.1,\ymin+0.1) rectangle (\xmax-0.5,\ymax-0.1);
				\draw[smooth,samples=300] plot(\x,{\a*(\x)^2+\b*(\x)+\c});
				\node at (1,0) [above right]{$1$};
				\node at (2,0) [below right]{$2$};
				\node at (3,0) [below right]{$3$};
				\node at (-3,-2) [left]{$y=-3t^2-6t+9$};
				\node at (4,0) [below right]{$f'(x)$};
				\draw (-2.2,10).. controls (-1,1.9) and (-0.5,0.8) .. (0,0);
				%\draw (-2,0).. controls (-1.5,-2) and (-0.5,-0) .. (0,0);
				\draw (0,0).. controls (0.4,-0.6) and (0.6,-0.6) .. (0.8,-0.2);
				\draw (0.8,-0.2).. controls (1,0.25) and (1.1,-0.1) .. (1.4,-0.8);
				\draw (1.4,-0.8).. controls (1.6,-1.1) and (1.7,-0.9) .. (2,0);
				\draw (2,0).. controls (2.4,1.1) and (2.6,1.1) .. (3.5,-1);
			\end{tikzpicture}
		}
		Dựa vào đồ thị trên, ta có bảng xét dấu của hàm số $y'=f'(t)-\left(-3 t^2-6 t+9\right)$ như sau $
		\left(t_0<-1\right)$
		\begin{center}
			\begin{tikzpicture}
				\tkzTabInit[nocadre,lgt=2,espcl=2,deltacl=0.6]
				{$x$ /0.6,$y'$ /0.6}
				{$-\infty$,$t_0$,$1$,$+\infty$}
				\tkzTabLine{,+,$0$,-,$0$,+,}
			\end{tikzpicture}
		\end{center}
		Hàm số nghịch biến trên khoảng $t \in\left(t_0 ; 1\right)$.\\
		Do đó hàm số nghịch biến trên khoảng $x \in(1 ; 2) \subset \left(t_0+1 ; 1\right)$.
	}
\end{ex}


\begin{ex}[Chuyên Phan Bội Châu Nghệ An 2019]%[2D1G1-2]
	Cho hàm số $f(x)$ có bảng xét dấu đạo hàm như sau:
	\begin{center}
		\begin{tikzpicture}
			\tkzTabInit[nocadre,lgt=2,espcl=2,deltacl=0.6]
			{$x$ /0.6,$f'(x)$ /0.6}
			{$-\infty$,$1$,$2$,$3$,$4$,$+\infty$}
			\tkzTabLine{,-,$0$,+,$0$,+,$0$,-,$0$,+,}
		\end{tikzpicture}
	\end{center}
	Hàm số $y=2 f(1-x)+\sqrt{x^2+1}-x$ nghịch biến trên những khoảng nào dưới đây
	\choice
	{$(-\infty ;-2)$}
	{$(-\infty ; 1)$}
	{\True $(-2 ; 0)$}
	{$(-3 ;-2)$}
	\loigiai{
		$y'=-2 f'(1-x)+\dfrac{x}{\sqrt{x^2+1}}-1$. \\
		Có $\dfrac{x}{\sqrt{x^2+1}}-1<0,~ \forall x \in(-2 ; 0)$.\\
		Bảng xét dấu:
		\begin{center}
			\begin{tikzpicture}
				\tkzTabInit[nocadre,lgt=2,espcl=2,deltacl=0.6]
				{$x$ /0.7,$f'(1-x)$ /0.7}
				{$-\infty$,$-3$,$-2$,$-1$,$0$,$+\infty$}
				\tkzTabLine{,+,$0$,-,$0$,+,$0$,+,$0$,-,}
			\end{tikzpicture}
		\end{center}
		$\Rightarrow-2 f'(1-x)<0, ~ \forall x \in(-2 ; 0) \\
		\Rightarrow-2 f'(1-x)+\dfrac{x}{\sqrt{x^2+1}}-1<0, ~\forall x \in(-2 ; 0)$.
	}
\end{ex}
\begin{ex}[Sở Vĩnh Phúc 2019]%[2D1G1-2]
	\immini{
		Cho hàm số bậc bốn $y=f(x)$ có đồ thị của hàm số $y=f'(x)$ như hình vẽ bên.\\
		Hàm số $y=3 f(x)+x^3-6 x^2+9 x$ đồng biến trên khoảng nào trong các khoảng sau đây?
		\choice
		{$(0 ; 2)$}
		{$(-1 ; 1)$}
		{$(1 ;+\infty)$}
		{\True $(-2 ; 0)$}
	}
	{
		\begin{tikzpicture}[scale=0.7,>=stealth, font=\footnotesize, line join=round, line cap=round]
			\def\a{0.21} \def\b{0.88} \def\c{-0.58} \def\d{-3} % Hệ số
			\def\xmin{-5} \def\xmax{5}
			\def\ymin{-4} \def\ymax{3} 
			%\draw[color=gray!50,dashed] (\xmin,\ymin) grid (\xmax,\ymax); 
			\draw[->] (\xmin,0)--(\xmax,0) node [below]{$x$};
			\draw[->] (0,\ymin)--(0,\ymax) node [left]{$y$};
			\node at (0,0) [above left]{$O$};
			\node at (-4,0) [below left]{$-4$};
			\node at (-2,0) [below left]{$-2$};
			\node at (0,-3) [below right]{$-3$};
			\draw[dashed] (2,0) node[above right]{$2$}--(2,1) --(0,1) node[above right]{$1$};
			\clip (\xmin+0.1,\ymin+0.1) rectangle (\xmax-0.5,\ymax-0.1);
			\draw[smooth,samples=300] plot(\x,{\a*(\x)^3+\b*(\x)^2+\c*(\x)+\d});
		\end{tikzpicture}
	}
	
	\loigiai{
		Hàm số $f(x)=a x^4+b x^3+c x^2+d x+e,(a \neq 0)$.
		Có $f'(x)=4 a x^3+3 b x^2+2 c x+d$.\\
		Đồ thị hàm số $y=f'(x)$ đi qua các điểm $(-4 ; 0),(-2 ; 0),(0 ;-3),(2 ; 1)$ nên ta có
		$$\heva{&- 2 5 6 a + 4 8 b - 8 c + d = 0\\
			&- 3 2 a + 1 2 b - 4 c + d = 0\\
			&d = - 3\\
			&3 2 a + 1 2 b + 4 c + d = 1}\Leftrightarrow \heva{&
			a=\dfrac{5}{96}\\
			&b=\dfrac{7}{24}\\
			&c=-\dfrac{7}{24}\\
			&d=-3.}
		$$
		Xét hàm số
		$
		y=3 f(x)+x^3-6 x^2+9 x$\\
		Ta có $ y'=3\left(f'(x)+x^2-4 x+3\right)=3\left(\frac{5}{24}x^3+\frac{15}{8}x^2-\frac{55}{12}x\right)
		$\\
		Ta có $y'=0 \Leftrightarrow\hoac{&x=-11 \\&x=0 \\&x=2.}$ \\
		Xét dấu $y'$, ta được hàm số đã cho đồng biến trên các khoảng $(-11 ; 0)$ và $(2 ;+\infty)$.
	}
\end{ex}
\begin{ex}[Học Mãi 2019]%[2D1K1-2]
	\immini
	{Cho hàm số $y=f(x)$ có đạo hàm trên $\mathbb{R}$. Đồ thị hàm số $y=f'(x)$ như hình bên. Hỏi đồ thị hàm số $y=f(x)-2 x$ có bao nhiêu điểm cực trị?
		\choice
		{$4$}
		{\True $3$}
		{$2$}
		{$1$}
	}
	{
		\begin{tikzpicture}[font=\footnotesize,line join=round, line cap=round,>=stealth,scale=0.8]
			\draw[->] (-3.5,0)--(4,0) node[above] {$x$};
			\draw[->] (0,-3)--(0,4) node[left] {$y$};
			%\fill[black] (-2,0)node[below left]{$-2$} circle (1.2pt) (0,0)node[above right]{$O$} circle (1.2pt) (3,0)node[above]{$3$} circle (1.2pt);
			\draw[dashed] (-2,-2)-- (0,-2) node[right]{$-2$};
			\draw[dashed] (2,0) node[below]{$2$}-- (2,2)--(0,2) node[below left]{$2$};
			\node at (0,0) [below left]{$O$};
			\node at (3,0) [below right]{$3$};
			\draw (-3,2.5).. controls (-2.2,-3) and (-1.8,-3) .. (-1.1,0);
			\draw (-1.1,0).. controls (-0.6,2.5) and (-0.4,2.5) .. (0,2);
			\draw (0,2).. controls (0.7,0.5) and (1.1,0.5) .. (1.5,1.5);
			\draw (1.5,1.5).. controls (2,2.5) and (2.8,2.5) .. (3.5,-2.5);
			%\draw (3,0).. controls (3.3,-0.1) and (3.5,-0.5) .. (3.5,-2);
		\end{tikzpicture}
	}
	\loigiai{
		\immini{
			Đặt $g(x)=f(x)-2 x$.\\
			$\Rightarrow g'(x)=f'(x)-2 .
			$\\
			Vẽ đường thẳng $y=2$.\\
			$\Rightarrow$ phương trình $g'(x)=0$ có $3$ nghiệm bội lẻ.\\
			$\Rightarrow$ đồ thị hàm số $y=f(x)-2 x$ có $3$ điểm cực trị.
		}
		{
			\begin{tikzpicture}[font=\footnotesize,line join=round, line cap=round,>=stealth,scale=0.8]
				\draw[->] (-3.5,0)--(4,0) node[above] {$x$};
				\draw[->] (0,-3)--(0,4) node[left] {$y$};
				%\fill[black] (-2,0)node[below left]{$-2$} circle (1.2pt) (0,0)node[above right]{$O$} circle (1.2pt) (3,0)node[above]{$3$} circle (1.2pt);
				\draw[dashed] (-2,-2)-- (0,-2) node[right]{$-2$};
				\draw[dashed] (2,0) node[below]{$2$}-- (2,2)--(0,2) node[below left]{$2$};
				\node at (3,0) [below left]{$3$};
				\draw (-3,2.5).. controls (-2.2,-3) and (-1.8,-3) .. (-1.1,0);
				\draw (-1.1,0).. controls (-0.6,2.5) and (-0.4,2.5) .. (0,2);
				\draw (0,2).. controls (0.7,0.5) and (1.1,0.5) .. (1.5,1.5);
				\draw (1.5,1.5).. controls (2,2.5) and (2.8,2.5) .. (3.5,-2.5);
				\draw (-3.5,2)--(4,2) node[above]{$y=2$};
			\end{tikzpicture}
		}
	}
\end{ex}
\begin{ex}[THPT Hoàng Hoa Thám Hưng Yên 2019]%[2D1G1-2]
	\immini{
		Cho hàm số $y=f(x)$ liên tục trên $\mathbb{R}$. Hàm số $y=f'(x)$ có đồ thị như hình vẽ. 
		Hàm số $g(x)=f(x-1)+\dfrac{2019-2018 x}{2018}$ đồng biến trên khoảng nào dưới đây?
		\choice
		{$(2 ; 3)$}
		{$(0 ; 1)$}
		{\True $(-1 ; 0)$}
		{$(1 ; 2)$}
	}
	{
		\begin{tikzpicture}[scale=1, font=\footnotesize, line join=round, line cap=round, >=stealth]
			\tikzset{label style/.style={font=\footnotesize}}
			\draw[->] (-2,0)--(3,0) node[below left] {$x$};
			\draw[->] (0,-2)--(0,3) node[below left] {$y$};
			\draw[fill=black] (0,0) node [above left] {$O$} circle(1pt);
			\fill (1,1) circle(1pt) (-1,1) circle(1pt) (2,1) circle(1pt);
			\foreach \x in {1,2}
			\draw[thin] (\x,1pt)--(\x,-1pt) node [below] {\footnotesize$\x$};
			\foreach \x in {-1}
			\draw[thin] (\x,1pt)--(\x,-1pt) node [below left] {\footnotesize$\x$};
			\foreach \y in {-1}
			\draw[thin] (1pt,\y)--(-1pt,\y) node [right] {\footnotesize$\y$};
			\foreach \y in {1}
			\draw[thin] (1pt,\y)--(-1pt,\y) node [above left] {\footnotesize$\y$};
			\draw[dashed](-1,0)--(-1,1)--(2,1) (1,1)--(1,0) (2,1)--(2,0);
			\begin{scope}
				\clip (-3,-3) rectangle (3,3);
				\draw[name path=(C)] plot[smooth,tension=0.7] coordinates{(-1.15,3)(-0.5,-1.6)(.8,.88)(1.9,0.8)(2.3,3)};
			\end{scope}
		\end{tikzpicture}
	}	\loigiai{
		Ta có $g'(x)=f'(x-1)-1$.\\
		$
		g'(x) \geq 0 \Leftrightarrow f'(x-1)-1 \geq 0 \Leftrightarrow f'(x-1) \geq 1 \Leftrightarrow \hoac{&x - 1 \leq - 1\\
			&x - 1 \geq 2}\Leftrightarrow \hoac{&
			x \leq 0 \\
			&x \geq 3.}
		$\\
		Từ đó suy ra hàm số $g(x)=f(x-1)+\dfrac{2019-2018 x}{2018}$ đồng biến trên khoảng $(-1 ; 0)$.
	}
\end{ex}

\begin{ex}[(Sở Ninh Bình 2019]%[2D1K1-2]
	Cho hàm số $y=f(x)$ có bảng xét dấu của đạo hàm như sau
	\begin{center}
		\begin{tikzpicture}
			\tkzTabInit[nocadre,lgt=1,espcl=2,deltacl=0.6]
			{$x$ /0.7,$f'(x)$ /0.7}
			{$-\infty$,$-2$,$-1$,$2$,$4$,$+\infty$}
			\tkzTabLine{,+,$0$,-,$0$,+,$0$,-,$0$,+,}
		\end{tikzpicture}
	\end{center}
	Hàm số $y=-2 f(x)+2019$ nghịch biến trên khoảng nào trong các khoảng dưới đây?
	\choice
	{$(-4 ; 2)$}
	{\True $(-1 ; 2)$}
	{$(-2 ;-1)$}
	{$(2 ; 4)$}
	\loigiai{
		Xét $y=g(x)=-2 f(x)+2019$.\\
		Ta có $g'(x)=(-2 f(x)+2019)'=-2 f'(x), g'(x)=0 \Leftrightarrow\hoac{&x=-2 \\&x=-1 \\&x=2 \\&x=4.}$.\\
		Ta có bảng xét dấu của $g'(x)$
		\begin{center}
			\begin{tikzpicture}
				\tkzTabInit[nocadre,lgt=1,espcl=2,deltacl=0.6]
				{$x$ /0.6,$f'(x)$ /0.6}
				{$-\infty$,$-2$,$-1$,$2$,$4$,$+\infty$}
				\tkzTabLine{,-,$0$,+,$0$,-,$0$,+,$0$,+,}
			\end{tikzpicture}
		\end{center}
		Dựa vào bảng xét dấu, ta thấy hàm số $y=g(x)$ nghịch biến trên khoảng $(-1 ; 2)$.
	}
\end{ex}
\begin{ex}[THPT Lương Thế Vinh Hà Nội 2019]%[2D1G1-2]
	\immini{
		Cho hàm số $y=f(x)$. Biết đồ thị hàm số $y=f'(x)$ có đồ thị như hình vẽ bên. 
		Hàm số $y=f \left(3-x^2\right)+2018$ đồng biến trên khoảng nào dưới đây?
		\choice
		{\True $(-1 ; 0)$}
		{$(2 ; 3)$}
		{$(-2 ;-1)$}
		{$(0 ; 1)$}
	}
	{
		\begin{tikzpicture}[scale=0.6,>=stealth, font=\footnotesize, line join=round, line cap=round]
			\def\a{0.065} \def\b{0.32} \def\c{-0.53} \def\d{-0.82} % Hệ số
			\def\xmin{-8} \def\xmax{4}
			\def\ymin{-3} \def\ymax{3} 
			%\draw[color=gray!50,dashed] (\xmin,\ymin) grid (\xmax,\ymax); 
			\draw[->] (\xmin,0)--(\xmax,0) node [below]{$x$};
			\draw[->] (0,\ymin)--(0,\ymax) node [left]{$y$};
			\node at (0,0) [below left]{$O$};
			\node at (-6,0) [below left]{$-6$};
			\node at (-1,0) [below left]{$-1$};
			\node at (2,0) [below right]{$2$};
			\clip (\xmin+0.1,\ymin+0.1) rectangle (\xmax-0.5,\ymax-0.1);
			\draw[smooth,samples=300][domain=-6.5:3.5] plot(\x,{\a*(\x)^3+\b*(\x)^2+\c*(\x)+\d});
		\end{tikzpicture}
	}
	
	\loigiai{
		Ta có $\left[f\left( 3-x^2\right)+2018 \right]'=-2 x \cdot f'\left(3-x^2\right) $.\\
		$
		-2 x \cdot f'\left(3-x^2\right)=0 \Leftrightarrow\hoac{&
			x = 0\\
			&3 - x ^{2}= - 6\\
			&3 - x ^{2}= - 1\\
			&3 - x ^{2}= 2}
		\Leftrightarrow \hoac{
			&x=0 \\
			&x=\pm 3 \\
			&x=\pm 2 \\
			&	x=\pm 1.}
		$\\
		Bảng xét dấu của đạo hàm hàm số đã cho
		\begin{center}
			\begin{center}
				\begin{tikzpicture}
					\tkzTabInit[nocadre,lgt=2.9,espcl=1.5,deltacl=0.6]
					{$x$ /0.7,$f'\left( 3-x^2\right) $/0.7,$-2xf'\left( 3-x^2\right)$/0.8}
					{$-\infty$,$-3$,$-2$,$-1$,$0$,$1$,$2$,$3$,$+\infty$}
					\tkzTabLine{,-,$0$,+,$0$,-,$0$,+,$0$,+,$0$,-,$0$,+,$0$,-}
					\tkzTabLine{,-,$0$,+,$0$,-,$0$,+,$0$,-,$0$,+,$0$,-,$0$,+}
				\end{tikzpicture}
			\end{center}
		\end{center}
		Từ bảng xét dấu suy ra hàm số đồng biến trên $(-1 ; 0)$.
	}
\end{ex}
\begin{ex}[Chuyên Biên Hòa - Hà Nam - 2020]%[2D1G1-2]
	\immini{
		Cho hàm số đa thức $f(x)$ có đạo hàm trên $\mathbb{R}$. Biết $f(0)=0$ và đồ thị hàm số $y=f'(x)$ như hình sau.
		Hàm số $g(x)=\left|4 f(x)+x^2\right|$ đồng biến trên khoảng nào dưới đây?
		\choice
		{$(4 ;+\infty)$}
		{\True $(0 ; 4)$}
		{$(-\infty ;-2)$}
		{$(-2 ; 0)$}
	}	
	{
		\begin{tikzpicture}[scale=0.7,>=stealth, font=\footnotesize, line join=round, line cap=round]
			%\def\a{1} \def\b{-6} \def\c{9} \def\d{1} % Hệ số
			\def\xmin{-4} \def\xmax{6}
			\def\ymin{-3} \def\ymax{2} 
			%\draw[color=gray!50,dashed] (\xmin,\ymin) grid (\xmax,\ymax); 
			\draw[->] (\xmin,0)--(\xmax,0) node [below]{$x$};
			\draw[->] (0,\ymin)--(0,\ymax) node [left]{$y$};
			\node at (0,0) [below left]{$O$};
			%\node at (1,3) [below left]{$f'(x)$};
			%\node at (-1.3,4) {$f'(x)$};
			\draw[dashed] (-2,0) node[below]{$-2$}--(-2,1)--(0,1) node[below left]{$1$};
			\draw[dashed] (4,0) node[below]{$4$}--(4,-2)--(0,-2) node[below left]{$-2$};
			%\draw[dashed] (1,0) node[below]{$1$}--(1,1);
			%\draw[dashed] (-0.5,0) node[below left]{$-0{,}5$}--(-0.5,2.125);
			\clip (\xmin+0.1,\ymin+0.1) rectangle (\xmax-0.5,\ymax-0.1);
			\draw[smooth,samples=300][domain=-4:5.5] plot(\x,{0.071*(\x)^3-0.142*(\x)^2-1.07*(\x)});
		\end{tikzpicture}
	}
	\loigiai{
		\immini{
			Xét hàm số $h(x)=4 f(x)+x^2$ trên $\mathbb{R}$.\\
			Vì $f(x)$ là hàm số đa thức nên $h(x)$ cũng là hàm số đa thức và $h(0)=4 f(0)=0$.\\
			Ta có $h'(x)=4 f'(x)+2 x$. Do đó $h'(x)=0 \Leftrightarrow f'(x)=-\dfrac{1}{2}x$.\\
		}
		{
			\begin{tikzpicture}[scale=0.7,>=stealth, font=\footnotesize, line join=round, line cap=round]
				%\def\a{1} \def\b{-6} \def\c{9} \def\d{1} % Hệ số
				\def\xmin{-4} \def\xmax{6}
				\def\ymin{-3} \def\ymax{2} 
				%\draw[color=gray!50,dashed] (\xmin,\ymin) grid (\xmax,\ymax); 
				\draw[->] (\xmin,0)--(\xmax,0) node [below]{$x$};
				\draw[->] (0,\ymin)--(0,\ymax) node [left]{$y$};
				\node at (0,0) [below left]{$O$};
				%\node at (1,3) [below left]{$f'(x)$};
				%\node at (-1.3,4) {$f'(x)$};
				\draw[dashed] (-2,0) node[below]{$-2$}--(-2,1)--(0,1) node[below left]{$1$};
				\draw[dashed] (4,0) node[below]{$4$}--(4,-2)--(0,-2) node[below left]{$-2$};
				%\draw[dashed] (1,0) node[below]{$1$}--(1,1);
				%\draw[dashed] (-0.5,0) node[below left]{$-0{,}5$}--(-0.5,2.125);
				\clip (\xmin+0.1,\ymin+0.1) rectangle (\xmax-0.5,\ymax-0.1);
				\draw[smooth,samples=300][domain=-4:5.5] plot(\x,{0.071*(\x)^3-0.142*(\x)^2-1.07*(\x)});
				\draw[smooth,samples=300][domain=-4:5.5] plot(\x,{-0.5*(\x)});
			\end{tikzpicture}
		}
		Dựa vào sự tương giao của đồ thị hàm số $y=f'(x)$ và đường thẳng $y=-\dfrac{1}{2}x$, ta có
		$
		h'(x)=0 \Leftrightarrow x \in\{-2 ; 0 ; 4\}.\\
		$
		Bảng biến thiên của hàm số $h(x)$ như sau:
		\begin{center}
			\begin{tikzpicture}
				\tkzTabInit[nocadre,lgt=1.2,espcl=2.5,deltacl=0.6]
				{$x$ /0.6,$y'$ /0.6,$y$ /2}
				{$-\infty$,$-2$,$0$,$4$,$+\infty$}
				\tkzTabLine{,-,$0$,+,$0$,-,$0$,+,}
				\tkzTabVar{+/$+\infty$, -/$y_1$,+/$0$,-/$y_3$,+/$+\infty$}
			\end{tikzpicture}
		\end{center}
		Từ đó suy ra bảng biến thiên của hàm số $g(x)=|h(x)|$.\\
		Dựa vào bảng biến thiên trên, ta thấy hàm số $g(x)$ đồng biến trên khoảng $(0 ; 4)$.
	}
\end{ex}
\begin{ex}[Chuyên Thái Bình - 2020]%[2D1G1-2]
	\immini{
		Cho hàm số $f(x)$ liên tục trên $\mathbb{R}$ có đồ thị hàm số $y=f'(x)$ cho như hình vẽ bên.\\
		Hàm số $g(x)=2 f(|x-1|)-x^2+2 x+2020$ đồng biến trên khoảng nào?
		\choice
		{\True $(0 ; 1)$}
		{$(-3 ; 1)$}
		{$(1 ; 3)$}
		{$(-2 ; 0)$}
	}
	{
		\begin{tikzpicture}[scale=0.7,>=stealth, font=\footnotesize, line join=round, line cap=round]
			%\def\a{1} \def\b{-6} \def\c{9} \def\d{1} % Hệ số
			\def\xmin{-4} \def\xmax{5}
			\def\ymin{-3} \def\ymax{5} 
			%\draw[color=gray!50,dashed] (\xmin,\ymin) grid (\xmax,\ymax); 
			\draw[->] (\xmin,0)--(\xmax,0) node [below]{$x$};
			\draw[->] (0,\ymin)--(0,\ymax) node [left]{$y$};
			\node at (0,0) [below left]{$O$};
			%\node at (1,3) [below left]{$f'(x)$};
			\node at (-1.3,4) {$f'(x)$};
			\draw[dashed] (-1,0) node[above]{$-1$}--(-1,-1)--(0,-1) node[below left]{$-1$};
			\draw[dashed] (1,0) node[below]{$1$}--(1,1)--(0,1) node[below left]{$1$};
			\draw[dashed] (3,0) node[below]{$3$}--(3,3)--(0,3) node[below left]{$3$};
			%\draw[dashed] (1,0) node[below]{$1$}--(1,1);
			%\draw[dashed] (-0.5,0) node[below left]{$-0{,}5$}--(-0.5,2.125);
			\clip (\xmin+0.1,\ymin+0.1) rectangle (\xmax-0.5,\ymax-0.1);
			\draw[smooth,samples=300][domain=-2:4] plot(\x,{-0.5*(\x)^3+1.5*(\x)^2+1.5*(\x)-1.5});
			%\draw[smooth,samples=300] plot(\x,{(\x)^3+(\x)^2-2*(\x)+1});
		\end{tikzpicture}
	}
	\loigiai{
		Ta có đường thẳng $y=x$ cắt đồ thị hàm số $y=f'(x)$ tại các điểm $x=-1 ; x=1 ; x=3$ như hình vẽ sau:
		\begin{center}
			\begin{tikzpicture}[scale=0.7,>=stealth, font=\footnotesize, line join=round, line cap=round]
				%\def\a{1} \def\b{-6} \def\c{9} \def\d{1} % Hệ số
				\def\xmin{-4} \def\xmax{5}
				\def\ymin{-3} \def\ymax{5} 
				%\draw[color=gray!50,dashed] (\xmin,\ymin) grid (\xmax,\ymax); 
				\draw[->] (\xmin,0)--(\xmax,0) node [below]{$x$};
				\draw[->] (0,\ymin)--(0,\ymax) node [left]{$y$};
				\node at (0,0) [below left]{$O$};
				%\node at (1,3) [below left]{$f'(x)$};
				\node at (-1.3,4) {$f'(x)$};
				\node at (4,3.2) {$y=x$};
				\draw[dashed] (-1,0) node[above]{$-1$}--(-1,-1)--(0,-1) node[below left]{$-1$};
				\draw[dashed] (1,0) node[below]{$1$}--(1,1)--(0,1) node[below left]{$1$};
				\draw[dashed] (3,0) node[below]{$3$}--(3,3)--(0,3) node[below left]{$3$};
				%\draw[dashed] (1,0) node[below]{$1$}--(1,1);
				%\draw[dashed] (-0.5,0) node[below left]{$-0{,}5$}--(-0.5,2.125);
				\clip (\xmin+0.1,\ymin+0.1) rectangle (\xmax-0.5,\ymax-0.1);
				\draw[smooth,samples=300][domain=-2:4] plot(\x,{-0.5*(\x)^3+1.5*(\x)^2+1.5*(\x)-1.5});
				\draw[smooth,samples=300] plot(\x,{(\x)});
			\end{tikzpicture}
		\end{center}
		Dựa vào đồ thị của hai hàm số trên ta có $f'(x)>x \Leftrightarrow\hoac{&x<-1 \\ &1<x<3}$ và
		$ f'(x)<x \Leftrightarrow\hoac{&
			-1<x<1 \\
			&x>3.}$\\
		+Trường hợp 1: $x-1<0 \Leftrightarrow x<1$.\\
		Khi đó $g(x)=2 f(1-x)-x^2+2 x+2020$.\\
		Ta có $g'(x)=-2 f'(1-x)+2(1-x)$.
		$$
		g'(x)>0 \Leftrightarrow-2 f'(1-x)+2(1-x)>0 \Leftrightarrow f'(1-x)<1-x \Leftrightarrow\hoac{
			&- 1 < 1 - x < 1\\
			&1 - x > 3} \Leftrightarrow \hoac{&
			0<x<2 \\
			&x<-2.}
		$$
		Kết hợp điều kiện, ta có $g'(x)>0 \Leftrightarrow\hoac{&0<x<1 \\ &x<-2.}$\\
		
		+ Trường hợp 2: $x-1>0 \Leftrightarrow x>1$.\\
		Khi đó ta có $g(x)=2 f(x-1)-x^2+2 x+2020$.\\
		$ g'(x)=2 f'(x-1)-2(x-1)$\\
		$g'(x)>0 \Leftrightarrow 2 f'(x-1)-2(x-1)>0 \Leftrightarrow f'(x-1)>x-1 \Leftrightarrow\hoac{&
			x - 1 < - 1\\
			&1 < x - 1 < 3}\Leftrightarrow \hoac{
			&x<0 \\
			&2<x<4.}$
		Kết hợp điều kiện ta có $g'(x)>0 \Leftrightarrow 2<x<4$.\\
		Vậy hàm số $g(x)=2 f(|x-1|)-x^2+2 x+2020$ đồng biến trên khoảng $(0 ; 1)$.
	}
\end{ex}

\begin{ex}[Chuyên Lào Cai - 2020]%[2D1G1-2]
	\immini{
		Cho hàm số $f'(x)$ có đồ thị như hình bên.\\
		Hàm số $g(x)=f(3 x+1)+9 x^3+\dfrac{9}{2}x^2$ đồng biến trên khoảng nào dưới đây?
		\choice
		{$(-1 ; 1)$}
		{$(-2 ; 0)$}
		{$(-\infty ; 0)$}
		{\True $(1 ;+\infty)$}
	}
	{\begin{tikzpicture}[line join=round, line cap=round,>=stealth,thick,scale=.8]
			\tikzset{label style/.style={font=\footnotesize}}
			\draw[->] (-2.1,0)--(5.1,0) node[below left] {$x$};
			\draw[->] (0,-3.1)--(0,4.1) node[below left] {$y$};
			\draw (0,0) node [below left] {$O$};
			\foreach \x in {1,2,3}
			\draw[thin] (\x,1pt)--(\x,-1pt) node [below] {$\x$};
			\draw[thin](-1,1pt)--(1,-1pt)node[above left]{$-1$};
			\foreach \y in {-2,2}
			\draw[thin] (1pt,\y)--(-1pt,\y) node [above right] {$\y$};
			%\begin{scope}
			\clip (-2,-3) rectangle (5,4);
			\draw[samples=200,domain=-2:4,smooth,variable=\x] plot (\x,{(\x)^3-3*(\x)^2+2});
			%\end{scope}
			\draw[dashed] (-1,0)--(-1,-2)--(0,-2);
			\draw[dashed] (3,0)--(3,2)--(0,2);
			%\begin{scope}[on background layer]\path[white]node{MDD-134};\end{scope}
		\end{tikzpicture}
	}
	\loigiai
	{
		\immini{Xét hàm số $g(x)=f(3 x+1)+9 x^3+\dfrac{9}{2}x^2 \\
			\Rightarrow g'(x)=3 f'(3 x+1)+27 x^2+9 x$.\\
			Hàm số đồng biến  $\Leftrightarrow g'(x)>0 \Leftrightarrow 3 f'(3 x+1)+27 x^2+9 x>0$
			\\
			$
			\Leftrightarrow f'(3 x+1)+3 x(3 x+1)>0 \qquad (*)
			$\\
			Đặt $t=3 x+1$, khi đó  $(*) \Leftrightarrow f'(t)+(t-1) t>0$\\ $\Leftrightarrow f'(t)>-t^2+t$.\\
			Vẽ parabol $y=-x^2+x$ và đồ thị hàm số $f'(x)$ trên cùng một hệ trục
		}
		{
			\begin{tikzpicture}[line join=round, line cap=round,>=stealth,thick,scale=.8]
				\tikzset{label style/.style={font=\footnotesize}}
				\draw[->] (-2.1,0)--(5.1,0) node[below left] {$x$};
				\draw[->] (0,-3.1)--(0,4.1) node[below left] {$y$};
				\draw (0,0) node [below left] {$O$};
				\foreach \x in {1,2,3}
				\draw[thin] (\x,1pt)--(\x,-1pt) node [below] {$\x$};
				\draw[thin](-1,1pt)--(1,-1pt);
				\foreach \y in {-2,2}
				\draw[thin] (1pt,\y)--(-1pt,\y) node [above right] {$\y$};
				%\begin{scope}
				\clip (-2,-3) rectangle (5,4);
				\draw[samples=200,domain=-2:4,smooth,variable=\x] plot (\x,{(\x)^3-3*(\x)^2+2});
				\draw[samples=200,domain=-2:4,smooth,variable=\x] plot (\x,{-(\x)^2+(\x)});
				%\end{scope}
				\draw[dashed] (-1,0) node[above left]{$-1$}--(-1,-2)--(0,-2);
				\draw[dashed] (3,0)--(3,2)--(0,2);
				%\begin{scope}[on background layer]\path[white]node{MDD-134};\end{scope}
			\end{tikzpicture}
		}
		Dựa vào đồ thị ta thấy
		$
		f'(t)>-t^2+t \Leftrightarrow\hoac{&- 1 < t < 1\\
			&t > 2}\Rightarrow \hoac{&
			- 1 < 3 x + 1 < 1\\
			&3 x + 1 > 2} \Leftrightarrow \hoac{&
			\dfrac{-2}{3}<x<0\\
			&x>\dfrac{1}{3}.}
		$}
\end{ex}
\begin{ex}[Sở Phú Thọ-2020]%[2D1G1-2]
	\immini{
		Cho hàm số $y=f(x)$ có đồ thị hàm số $y=f'(x)$ như hình vẽ.\\
		Hàm số $g(x)=f\left(\mathrm{e}^x-2\right)-2020$ nghịch biến trên khoảng nào dưới đây?
		\choice
		{\True $\left(-1 ; \dfrac{3}{2}\right)$}
		{$(-1 ; 2)$}
		{$(0 ;+\infty)$}
		{$\left(\dfrac{3}{2}; 2\right)$}
	}
	{
		\begin{tikzpicture}[scale=0.7,>=stealth, font=\footnotesize, line join=round, line cap=round]
			\def\a{1} \def\b{-3} \def\c{0} \def\d{0} % Hệ số
			\def\xmin{-2} \def\xmax{4}
			\def\ymin{-5} \def\ymax{2} 
			%\draw[color=gray!50,dashed] (\xmin,\ymin) grid (\xmax,\ymax); 
			\draw[->] (\xmin,0)--(\xmax,0) node [below]{$x$};
			\draw[->] (0,\ymin)--(0,\ymax) node [left]{$y$};
			\node at (0,0) [above left]{$O$};
			\node at (3,0) [below right]{$3$};
			\draw[dashed] (2,0) node[above]{$2$}--(2,-4) --(0,-4) node[left]{$-4$};
			\clip (\xmin+0.1,\ymin+0.1) rectangle (\xmax-0.5,\ymax-0.1);
			\draw[smooth,samples=300] plot(\x,{\a*(\x)^3+\b*(\x)^2+\c*(\x)+\d});
		\end{tikzpicture}
	}
	
	\loigiai{
		Dựa vào đồ thị hàm số $y=f'(x)$ suy ra $f'(x) \leq 0 ~ \forall x<3$ và $f'(x)>0 ~ \forall x>3$.
		$
		g'(x)=\mathrm{e}^x f'\left(\mathrm{e}^x-2\right) .
		$
		Hàm số $g(x)=f\left(\mathrm{e}^x-2\right)-2020$ nghịch biến \\ $ \Leftrightarrow g'(x)<0 \Leftrightarrow \mathrm{e}^x f'\left(\mathrm{e}^x-2\right)<0$\\
		$
		\Leftrightarrow f'\left(\mathrm{e}^x-2\right)<0 \Leftrightarrow \mathrm{e}^x-2<3 \Leftrightarrow \mathrm{e}^x<5 \Leftrightarrow x<\ln 5 .
		$\\
		Vậy hàm số đã cho nghịch biến trên $\left(-1 ; \dfrac{3}{2}\right)$.
	}
\end{ex}
\begin{ex}[Lý Nhân Tông - Bắc Ninh - 2020]%[2D1G1-2]
	\immini{
		Cho hàm số $f(x)$ có đồ thị hàm số $f'(x)$ như hình vẽ.\\
		Hàm số $y=f(\cos x)+x^2-x$ đồng biến trên khoảng
		\choice
		{$(-2 ; 1)$}
		{$(0 ; 1)$}
		{\True $(1 ; 2)$}
		{$(-1 ; 0)$}
	}
	{
		\begin{tikzpicture}[scale=1,>=stealth, font=\footnotesize, line join=round, line cap=round]
			\def\a{-0.5} \def\b{0} \def\c{1.5} \def\d{0} % Hệ số
			\def\xmin{-3} \def\xmax{4}
			\def\ymin{-2} \def\ymax{2} 
			%\draw[color=gray!50,dashed] (\xmin,\ymin) grid (\xmax,\ymax); 
			\draw[->] (\xmin,0)--(\xmax,0) node [below]{$x$};
			\draw[->] (0,\ymin)--(0,\ymax) node [left]{$y$};
			\node at (0,0) [above left]{$O$};
			\node at (3,0) [below right]{$3$};
			\draw[dashed] (-2,0) node[below]{$-2$}--(-2,1) --(0,1) node[above right]{$1$} --(1,1)--(1,0) node[below]{$1$};
			\draw[dashed] (-1,0) node[below right]{$-1$}--(-1,-1) --(0,-1) node[above right]{$-1$} --(2,-1)--(2,0) node[below right]{$2$};
			\clip (\xmin+0.1,\ymin+0.1) rectangle (\xmax-0.5,\ymax-0.1);
			\draw[smooth,samples=300][domain=-2:2] plot(\x,{\a*(\x)^3+\b*(\x)^2+\c*(\x)+\d});
		\end{tikzpicture}
	}
	\loigiai{
		Đặt  $g(x)=f(\cos x)+x^2-x$.\\
		Ta có $g'(x)=-\sin x \cdot f'(\cos x)+2 x-1$\\
		Vì $\cos x \in[-1 ; 1]$ nên từ đồ thị $f'(x)$ ta suy ra $f'(\cos x) \in[-1 ; 1]$.\\
		Do đó $\left|-\sin x \cdot f'(\cos x)\right| \leq 1, ~\forall x \in \mathbb{R}$.\\
		Ta suy ra $g'(x)=\sin x \cdot f'(\cos x)+2 x-1 \geq-1+2 x-1=2 x-2$
		$\Rightarrow g'(x)>0, ~\forall x>1$.\\
		Vậy hàm số đồng biến trên $(1 ; 2)$.
	}
\end{ex}
\begin{ex}[THPT Nguyễn Viết Xuân - 2020]%[2D1G1-2]
	\immini{
		Cho hàm số $f(x)$. Hàm số $y=f'(x)$ có đồ thị như hình vẽ.\\
		Hàm số $g(x)=f\left(3 x^2-1\right)-\dfrac{9}{2}x^4+3 x^2$ đồng biến trên khoảng nào dưới đây?
		\choice
		{\True $\left(-\dfrac{2 \sqrt{3}}{3}; \dfrac{-\sqrt{3}}{3}\right)$}
		{$\left(0 ; \dfrac{2 \sqrt{3}}{3}\right)$}
		{$(1 ; 2)$}
		{$\left(-\dfrac{\sqrt{3}}{3}; \dfrac{\sqrt{3}}{3}\right)$} 
	}
	{
		\begin{tikzpicture}[scale=0.6,>=stealth, font=\footnotesize, line join=round, line cap=round]
			\def\a{0.25} \def\b{0.25} \def\c{-2} \def\d{0} % Hệ số
			\def\xmin{-5} \def\xmax{4}
			\def\ymin{-5} \def\ymax{5} 
			%\draw[color=gray!50,dashed] (\xmin,\ymin) grid (\xmax,\ymax); 
			\draw[->] (\xmin,0)--(\xmax,0) node [below]{$x$};
			\draw[->] (0,\ymin)--(0,\ymax) node [left]{$y$};
			\node at (0,0) [above left]{$O$};
			%\node at (3,0) [below right]{$3$};
			\draw[dashed] (-4,0) node[below left]{$-4$}--(-4,-4) --(0,-4) node[above right]{$-4$};
			\draw[dashed] (3,0) node[below right]{$3$}--(3,3) --(0,3) node[above right]{$3$};
			\clip (\xmin+0.1,\ymin+0.1) rectangle (\xmax-0.5,\ymax-0.1);
			\draw[smooth,samples=300] plot(\x,{\a*(\x)^3+\b*(\x)^2+\c*(\x)+\d});
		\end{tikzpicture}
	}
	
	\loigiai
	{
		TXĐ: $\mathscr{D}=\mathbb{R}$.\\
		Ta có $g'(x)=6 x f'\left(3 x^2-1\right)-18 x^3+6 x=6 x\left[f'\left(3 x^2-1\right)-3 x^2+1\right]$.\\
		$
		g'(x)=0 \Leftrightarrow\hoac{
			&x = 0\\
			&f '( 3 x ^{2}- 1 ) = 3 x ^{2}- 1}
		\Leftrightarrow \hoac{
			&x = 0\\
			&3 x ^{2}- 1 = - 4 \text{~(vô nghiệm)}\\
			&3 x ^{2}- 1 = 0\\
			&3 x ^{2}- 1 = 3}\Leftrightarrow \hoac{&x=0 \\
			&x=\pm \dfrac{\sqrt{3}}{3}\\
			&x=\pm \dfrac{2 \sqrt{3}}{3}.}
		$\\
		Bảng xét dấu
		\begin{center}
			\begin{tikzpicture}
				\tkzTabInit[nocadre,lgt=1.2,espcl=2.2,deltacl=0.6]
				{$x$ /1.2,$f'(x)$ /0.7}
				{$-\infty$,$-\dfrac{2 \sqrt{3}}{3}$,$-\dfrac{ \sqrt{3}}{3}$,$0$,$\dfrac{\sqrt{3}}{3}$,$\dfrac{2 \sqrt{3}}{3}$,$+\infty$}
				\tkzTabLine{,-,$0$,+,$0$,-,$0$,+,$0$,-,$0$,+,}
			\end{tikzpicture}
		\end{center}
		Vậy hàm số đồng biến trong khoảng $\left(-\dfrac{2 \sqrt{3}}{3}; \dfrac{-\sqrt{3}}{3}\right)$.}
\end{ex}
\begin{ex}[Trần Phú - Quảng Ninh - 2020]%[2D1G1-2]
	Cho hàm số $f(x)$ có bảng xét dấu của đạo hàm như sau
	\begin{center}
		\begin{tikzpicture}
			\tkzTabInit[nocadre,lgt=1.2,espcl=2,deltacl=0.6]
			{$x$ /0.6,$f'(x)$ /0.6}
			{$-\infty$,$-4$,$-1$,$2$,$7$,$+\infty$}
			\tkzTabLine{,+,$0$,-,$0$,+,$0$,-,$0$,+,}
		\end{tikzpicture}
	\end{center}
	Hàm số $y=f(2 x+1)+\dfrac{2}{3}x^3-8 x+5$ nghịch biến trên khoảng nào dưới đây?
	\choice
	{$(-\infty ;-2)$}
	{$(1 ;+\infty)$}
	{$(-1 ; 7)$}
	{\True $\left(-1 ; \dfrac{1}{2}\right)$}
	\loigiai{
		Ta có $y'=2 f'(2 x+1)+2 x^2-8$.\\
		Xét $y'\leq 0 \Leftrightarrow 2 f'(2 x+1)+2 x^2-8 \leq 0 \Leftrightarrow f'(2 x+1) \leq 4-x^2$.\\
		Đặt $t=2x+1$, ta có $f'(t) \leq \dfrac{-t^2+2 t+15}{4}$.\\
		Vì $\dfrac{-t^2+2 t+15}{4}\geq 0, \forall t \in[-3 ; 5]$.\\
		Mà $f'(t) \leq 0, \forall t \in[-3 ; 2]$.\\
		Nên $f'(t) \leq \dfrac{-t^2+2 t+15}{4}\Rightarrow t \in[-3 ; 2]$.\\
		Suy ra $-3 \leq 2 x+1 \leq 2 \Leftrightarrow-2 \leq x \leq \dfrac{1}{2}$.}
\end{ex}

\begin{ex}[Chuyên Thái Bình - Lần 3 - 2020]%[2D1G1-2]
	\immini{
		Cho hàm số $y=f(x)$ liên tục trên $\mathbb{R}$ có đồ thị hàm số $y=f'(x)$ cho như hình vẽ.\\
		Hàm số $g(x)=2 f(|x-1|)-x^2+2 x+2020$ đồng biến trên khoảng nào?
		\choice
		{\True $(0 ; 1)$}
		{$(-3 ; 1)$}
		{$(1 ; 3)$}
		{$(-2 ; 0)$}
	}
	{
		\begin{tikzpicture}[scale=0.7,>=stealth, font=\footnotesize, line join=round, line cap=round]
			\def\a{-0.333} \def\b{1} \def\c{1.333} \def\d{-1} % Hệ số
			\def\xmin{-3} \def\xmax{5}
			\def\ymin{-3} \def\ymax{5} 
			%\draw[color=gray!50,dashed] (\xmin,\ymin) grid (\xmax,\ymax); 
			\draw[->] (\xmin,0)--(\xmax,0) node [below]{$x$};
			\draw[->] (0,\ymin)--(0,\ymax) node [left]{$y$};
			\node at (0,0) [above left]{$O$};
			%\node at (3,0) [below right]{$3$};
			\draw[dashed] (-1,0) node[above]{$-1$}--(-1,-1) --(0,-1) node[above right]{$-1$};
			\draw[dashed] (1,0) node[below right]{$1$}--(1,1) --(0,1) node[above right]{$1$};
			\draw[dashed] (3,0) node[below right]{$3$}--(3,3) --(0,3) node[above right]{$3$};
			\clip (\xmin+0.1,\ymin+0.1) rectangle (\xmax-0.5,\ymax-0.1);
			\draw[smooth,samples=300] plot(\x,{\a*(\x)^3+\b*(\x)^2+\c*(\x)+\d});
			\draw[smooth,samples=300] plot(\x,{(\x)});
		\end{tikzpicture}
	}
	\loigiai{
		Với $x>1$, ta có $g(x)=2 f(x-1)-(x-1)^2+2021 \Rightarrow g'(x)=2 f'(x-1)-2(x-1)$.\\
		Hàm số đồng biến $\Leftrightarrow 2 f'(x-1)-2(x-1)>0 \Leftrightarrow f'(x-1)>x-1 \quad(*)$.\\
		Đặt $t=x-1$, khi đó $(*) \Leftrightarrow f'(t)>t \Leftrightarrow\hoac{&1<t<3 \\ &t<-1}\Rightarrow\hoac{&2<x<4 \\ &x<0 ~(\text{loại}).}$\\
		Với $x<1$, ta có $g(x)=2 f(1-x)-(1-x)^2+2021 \Rightarrow g'(x)=-2 f'(1-x)+2(1-x)$.\\
		Hàm số đồng biến $\Leftrightarrow-2 f'(1-x)+2(1-x)>0 \Leftrightarrow f'(1-x)<1-x \quad(* *)$.\\
		Đặt $t=1-x$, khi đó $(* *) \Leftrightarrow f'(t)<t \Leftrightarrow\hoac{&-1<t<1 \\ &t>3}\Rightarrow\hoac{&0<x<2 \\ &x<-2}\Rightarrow\hoac{&0<x<1 \\ &x<-2.}$\\
		Vậy hàm số $g(x)$ đồng biến trên các khoảng $(-\infty ;-2),(0 ; 1),(2 ; 4)$.
	}
\end{ex}
\begin{ex}[Sở Phú Thọ - 2020]%[2D1G1-2]
	\immini{
		Cho hàm số $y=f(x)$ có đồ thị hàm số $f'(x)$ như hình vẽ.\\
		Hàm số $g(x)=f\left(1+e^x\right)+2020$ nghịch biến trên khoảng nào dưới đây?
		\choice
		{$(0 ;+\infty)$}
		{$\left(\dfrac{1}{2}; 1\right)$}
		{\True $\left(0 ; \dfrac{1}{2}\right)$}
		{$(-1 ; 1)$}
	}{
		\begin{tikzpicture}[scale=0.7,>=stealth, font=\footnotesize, line join=round, line cap=round]
			\def\a{1} \def\b{-3} \def\c{0} \def\d{0} % Hệ số
			\def\xmin{-2} \def\xmax{4}
			\def\ymin{-5} \def\ymax{2} 
			%\draw[color=gray!50,dashed] (\xmin,\ymin) grid (\xmax,\ymax); 
			\draw[->] (\xmin,0)--(\xmax,0) node [below]{$x$};
			\draw[->] (0,\ymin)--(0,\ymax) node [left]{$y$};
			\node at (0,0) [above left]{$O$};
			\node at (3,0) [below right]{$3$};
			\draw[dashed] (2,0) node[above]{$2$}--(2,-4) --(0,-4) node[left]{$-4$};
			\clip (\xmin+0.1,\ymin+0.1) rectangle (\xmax-0.5,\ymax-0.1);
			\draw[smooth,samples=300] plot(\x,{\a*(\x)^3+\b*(\x)^2+\c*(\x)+\d});
		\end{tikzpicture}
	}
	\loigiai{
		$g'(x)=e^x f'\left(1+e^x\right)$.\\
		Do $e^x>0, \forall x$ nên $g'(x) \leq 0 \Leftrightarrow f'\left(1+e^x\right) \leq 0 \Leftrightarrow 1+e^x \leq 3 \Leftrightarrow x \leq \ln 2$, dấu bằng xảy ra tại hữu hạn điểm.\\
		Nên $g(x)$ nghịch biến trên $(-\infty ; \ln 2)$.\\
		Vì $\left(0 ; \dfrac{1}{2}\right) \subset (-\infty ; \ln 2)$ nên hàm số đã cho nghịch biến trên $\left(0 ; \dfrac{1}{2}\right)$.
	}
\end{ex}

\begin{ex}%[2D1K1-2]
	[THPT Anh Sơn - Nghệ An - 2020]
	Cho hàm số $y=f(x)$ có bảng xét dấu của đạo hàm như sau.
	\begin{center}
		\begin{tikzpicture}
			\tkzTabInit[nocadre,lgt=1.2,espcl=2,deltacl=0.6]
			{$x$ /0.6,$f'(x)$ /0.6}
			{$-\infty$,$-2$,$-1$,$2$,$4$,$+\infty$}
			\tkzTabLine{,+,$0$,-,$0$,+,$0$,-,$0$,+,}
		\end{tikzpicture}
	\end{center}
	Hàm số $y=-2 f(x)+2019$ nghịch biến trên khoảng nào trong các khoảng dưới đây?
	\choice
	{$(2 ; 4)$}
	{$(-4 ; 2)$}
	{$(-2 ;-1)$}
	{\True $(-1 ; 2)$}
	\loigiai{
		Ta có $y'=-2 f'(x)$.\\
		$
		y'=0 \Leftrightarrow-2 f'(x)=0 \Leftrightarrow\hoac{&
			x=-2 \\
			&x=-1 \\
			&x=2 \\
			&x=4.}$\\
		Từ bảng xét dấu của $f'(x)$ ta có
		\begin{center}
			\begin{tikzpicture}
				\tkzTabInit[nocadre,lgt=1,espcl=2,deltacl=0.6]
				{$x$ /0.6,$y'$ /0.6}
				{$-\infty$,$-2$,$-1$,$2$,$4$,$+\infty$}
				\tkzTabLine{,-,$0$,+,$0$,-,$0$,+,$0$,-,}
			\end{tikzpicture}
		\end{center}
		Từ bảng xét dấu ta có hàm số nghịch biến trên khoảng $(-\infty ;-2),(-1 ; 2)$ và $(4 ;+\infty)$.}
\end{ex}

\begin{ex}[THPT Anh Sơn - Nghệ An - 2020]%[2D1G1-2]
	Cho hàm số $f(x)$ xác định và liên tục trên $\mathbb{R}$ và có đạo hàm $f'(x)$ thỏa mãn $f'(x)=(1-x)(x+2) g(x)+2019$ với $g(x)<0, ~\forall x \in \mathbb{R}$ . Hàm số $y=f(1-x)+2019 x+2020$ nghịch biến trên khoảng nào?
	\choice
	{$(1 ;+\infty)$}
	{$(0 ; 3)$}
	{$(-\infty ; 3)$}
	{\True $(3 ;+\infty)$}
	\loigiai{
		Đặt $h(x)=f(1-x)+2019 x+2020$.\\
		Vì hàm số $f(x)$ xác định trên $\mathbb{R}$ nên hàm số $h(x)$ cũng xác định trên $\mathbb{R}$.\\
		Ta có $h'(x)=-f'(1-x)+2019$.\\
		Do $h'(x)=0$ tại hữu hạn điểm nên để tìm khoảng nghịch biến của hàm số $h(x)$, ta tìm các giá trị của $x$ sao cho $h'(x)<0 \Leftrightarrow-f'(1-x)+2019<0$\\ 
		$\Leftrightarrow f'(1-x)-2019>0 \\
		\Leftrightarrow x(3-x) g(1-x)>0 \Leftrightarrow x(3-x)<0(\text{~do~}g(x)<0, \forall x \in \mathbb{R})$\\
		$\Leftrightarrow\hoac{&
			x<0 \\
			&x>3.}$\\
		Vậy hàm số $y=f(1-x)+2019 x+2020$ nghịch biến trên các khoảng $(-\infty ; 0)$ và $(3 ;+\infty).$}
\end{ex}

\begin{ex}%[2D1G1-2]
	Cho hàm số $y=f(x)$ xác định trên $\mathbb{R}$ và có bảng xét dấu đạo hàm như sau:
	\begin{center}
		\begin{tikzpicture}
			\tkzTabInit[nocadre,lgt=2,espcl=2,deltacl=0.6]
			{$x$ /0.6,$f'(x)$ /0.6}
			{$-\infty$,$-1$,$1$,$4$,$+\infty$}
			\tkzTabLine{,-,$0$,+,$0$,-,$0$,+,}
		\end{tikzpicture}
	\end{center}
	Biết $f(x)>2,~ \forall x \in \mathbb{R}$. Xét hàm số $g(x)=f(3-2 f(x))-x^3+3 x^2-2020$. Khẳng định nào sau đây đúng?
	\choice
	{Hàm số $g(x)$ đồng biến trên khoảng $(-2 ;-1)$}
	{Hàm số $g(x)$ nghịch biến trên khoảng $(0 ; 1)$}
	{Hàm số $g(x)$ đồng biến trên khoảng $(3 ; 4)$}
	{\True Hàm số $g(x)$ nghịch biến trên khoảng $(2 ; 3)$}
	\loigiai{
		Ta có $g'(x)=-2 f'(x) f'(3-2 f(x))-3 x^2+6 x$.\\
		Vì $f(x)>2, ~\forall x \in \mathbb{R}$ nên $3-2 f(x)<-1 ~\forall x \in \mathbb{R}$.\\
		Từ bảng xét dấu $f'(x)$ suy ra $f'(3-2 f(x))<0, ~\forall x \in \mathbb{R}$.\\
		Từ đó ta có bảng xét dấu sau:
		\begin{center}
			\begin{tikzpicture}
				\tkzTabInit[nocadre,lgt=4,espcl=1.7,deltacl=0.6]
				{$x$ /0.7,$-f'(x)f'\left( 3-2f(x)\right) $/0.8,$-3x^2+6x$/0.7}
				{$-\infty$,$-1$,$0$,$1$,$2$,$4$,$+\infty$}
				\tkzTabLine{,-,$0$,+,|,+,$0$,-,|,-,$0$,+,}
				\tkzTabLine{,-,|,-,$0$,+,|,+,$0$,-,|,-,}
			\end{tikzpicture}
		\end{center}
		Từ bảng xét dấu trên, loại trừ đáp án suy ra hàm số $g(x)$ nghịch biến trên khoảng $(2 ; 3)$.}
\end{ex}

\begin{ex}%[2D1G1-2]
	Cho hàm số $f(x)$ có bảng biến thiên như sau:
	\begin{center}
		\begin{tikzpicture}
			\tkzTabInit[nocadre,lgt=1.2,espcl=2.5,deltacl=0.6]
			{$x$ /0.7, $f'(x)$ /0.7, $f(x)$ /2.5}
			{$-\infty$,$1$,$2$,$3$,$4$,$+\infty$}
			\tkzTabLine{,+,$0$,-,$0$,+,$0$,-,$0$,+,}
			\tkzTabVar{-/$-\infty$,+/$3$,-/$1$,+/$2$,-/$0$,+/$+\infty$}
		\end{tikzpicture}
	\end{center}
	Hàm số $y=(f(x))^3-3 .(f(x))^2$ nghịch biến trên khoảng nào dưới đây?
	\choice
	{$(1 ; 2)$}
	{$(3 ; 4)$}
	{$(-\infty ; 1)$}
	{\True $(2 ; 3)$}
	\loigiai{
		Ta có $y'=3 \cdot(f(x))^2 \cdot f'(x)-6 \cdot f(x) \cdot f'(x)=3 f(x) \cdot f'(x) \cdot[f(x)-2]. \\
		y'=0 \Leftrightarrow \hoac{&f(x)=0 \Leftrightarrow x \in\left\{x_1, 4 \mid x_1<1\right\}\\
			&f(x)=2 \Leftrightarrow x \in\left\{x_2, x_3, 3, x_4 \mid x_1<x_2<1<x_3<2 ; 4<x_4\right\}\\
			&f'(x)=0 \Leftrightarrow x \in\{1,2,3,4\}.}$\\
		Lập bảng xét dấu ta có
		\begin{center}
			\begin{tikzpicture}
				\tkzTabInit[nocadre,lgt=2,espcl=1.5,deltacl=0.6]
				{$x$ /0.7,$f(x)$ /0.7,$f(x)-2$ /0.7,$f'(x)$/0.7,$y'$/0.7}
				{$-\infty$,$x_1$,$x_2$,$1$,$x_3$,$2$,$3$,$4$,$x_4$,$+\infty$}
				\tkzTabLine{,-,$0$,+,|,+,|,+,|,+,|,+,$0$,+,|,+,|,+,}
				\tkzTabLine{,-,|,-,$0$,+,$0$,+,$0$,-,|,-,$0$,-,|,-,$0$,+}
				\tkzTabLine{,+,|,+,|,+,$0$,-,|,-,$0$,+,$0$,-,$0$,+,|,+}
				\tkzTabLine{,+,$0$,-,$0$,+,$0$,-,$0$,+,$0$,-,$0$,+,$0$,-,$0$,+}
			\end{tikzpicture}
		\end{center}
		
		Do đó hàm số nghịch biến trên khoảng $(2 ; 3)$.
	}
\end{ex}
\begin{ex}%[2D1G1-2]
	Cho hàm số $y=f(x)$ có đồ thị nằm trên trục hoành và có đạo hàm trên $\mathbb{R}$, bảng xét dấu của biểu thức $f'(x)$ như bảng dưới đây.
	\begin{center}
		\begin{tikzpicture}
			\tkzTabInit[nocadre,lgt=1.2,espcl=2,deltacl=0.6]
			{$x$ /0.6,$f'(x)$ /0.6}
			{$-\infty$,$-2$,$-1$,$3$,$+\infty$}
			\tkzTabLine{,-,$0$,+,$0$,-,$0$,+,}
		\end{tikzpicture}
	\end{center}
	Hàm số $y=g(x)=\dfrac{f\left(x^2-2 x\right)}{f\left(x^2-2 x\right)+1}$ nghịch biến trên khoảng nào dưới đây?
	\choice
	{$(-\infty ; 1)$}
	{$\left(-2 ; \dfrac{5}{2}\right)$}
	{\True $(1 ; 3)$}
	{$(2 ;+\infty)$}
	\loigiai{
		$ g'(x)=\dfrac{\left(x^2-2 x\right)'\cdot f'\left(x^2-2 x\right)}{\left(f\left(x^2-2 x\right)+1\right)^2}=\dfrac{(2 x-2) \cdot f'\left(x^2-2 x\right)}{\left(f\left(x^2-2 x\right)+1\right)^2}. \\
		g'(x)=0 \Leftrightarrow\hoac{
			&2 x - 2 = 0\\
			&f '( x ^{2}- 2 x ) = 0}
		\Leftrightarrow \hoac{&x = 1\\
			&x ^{2}- 2 x = - 2\\
			&x ^{2}- 2 x = - 1\\
			&x ^{2}- 2 x = 3}
		\Leftrightarrow \hoac{&x=1 \\
			&x=-1 \\
			&x=3.}
		$\\
		Ta có bảng xét dấu của $g'(x)$
		\begin{center}
			\begin{tikzpicture}
				\tkzTabInit[nocadre,lgt=1.2,espcl=2,deltacl=0.6]
				{$x$ /0.6,$g'(x)$ /0.6}
				{$-\infty$,$-1$,$1$,$3$,$+\infty$}
				\tkzTabLine{,-,$0$,+,$0$,-,$0$,+,}
			\end{tikzpicture}
		\end{center}
		Dựa vào bảng xét dấu ta có hàm số $y=g(x)$ nghịch biến trên các khoảng $(-\infty ;-1)$ và $(1 ; 3)$.}
\end{ex}
\begin{ex}[Liên trường huyện Quảng Xương - Thanh Hóa - 2021]%[2D1G1-2]
	\immini{
		Cho các hàm số $y=f(x)$; $y=g(x)$ liên tục trên $\mathbb{R}$ và có đồ thị các đạo hàm $f'(x) ; g'(x)$ (đồ thị hàm số $y=g'(x)$ là đường đậm hơn) như hình vẽ.\\
		Hàm số $h(x)=f(x-1)-g(x-1)$ nghịch biến trên khoảng nào dưới đây?
		\choice
		{$\left(\dfrac{1}{2}; 1\right)$}
		{$(1 ;+\infty)$}
		{$(2 ;+\infty)$}
		{\True $\left(-1 ; \dfrac{1}{2}\right)$}
	}
	{
		\begin{tikzpicture}[scale=1,>=stealth, font=\footnotesize, line join=round, line cap=round]
			%\def\a{1} \def\b{-6} \def\c{9} \def\d{1} % Hệ số
			\def\xmin{-4} \def\xmax{3}
			\def\ymin{-2} \def\ymax{4} 
			%\draw[color=gray!50,dashed] (\xmin,\ymin) grid (\xmax,\ymax); 
			\draw[->] (\xmin,0)--(\xmax,0) node [below]{$x$};
			\draw[->] (0,\ymin)--(0,\ymax) node [left]{$y$};
			\node at (0,0) [above left]{$O$};
			\node at (1,3) [below left]{$f'(x)$};
			\node at (1.5,3) [below right]{$g'(x)$};
			\draw[dashed] (-2,0) node[above right]{$-2$}--(-2,1);
			\draw[dashed] (1,0) node[below]{$1$}--(1,1);
			\draw[dashed] (-0.5,0) node[below]{$-0{,}5$}--(-0.5,2.125);
			\clip (\xmin+0.1,\ymin+0.1) rectangle (\xmax-0.5,\ymax-0.1);
			\draw[smooth,samples=300][domain=-3:2] plot(\x,{2*(\x)^4+4*(\x)^3-2*(\x)^2-4*(\x)+1});
			\draw[smooth,samples=300,line width=1.2pt] plot(\x,{(\x)^3+(\x)^2-2*(\x)+1});
		\end{tikzpicture}
	}
	
	\loigiai{
		Ta có: $h'(x)=f'(x-1)-g'(x-1)$.\\
		Dựa vào hình vẽ ta có hàm số $h(x)$ nghịch biến\\
		$\Leftrightarrow h'(x)<0 \Leftrightarrow f'(x-1)<g'(x-1)$\\
		$
		\Leftrightarrow\hoac{&- 2 < x - 1 < - \dfrac{1}{2}\\
			&0 < x - 1 < 1}
		\Leftrightarrow \hoac{
			&-1<x<\dfrac{1}{2}\\
			&1<x<2.}$\\
		Do đó hàm số $h(x)$ nghịch biến trên các khoảng $\left(-1 ; \dfrac{1}{2}\right)$ và $(1 ; 2)$.
	}
\end{ex}
\begin{ex}[THPT Quế Võ 1 - Bắc Ninh - 2021] %[2D1G1-2]
	\immini{
		Cho ba hàm số $y=f(x), y=g(x), y=h(x)$. Đồ thị của ba hàm số $y=f'(x), y=g'(x), y=h'(x)$ được cho như hình vẽ.\\
		Hàm số $k(x)=f(x+7)+g(5 x+1)-h\left(4 x+\dfrac{3}{2}\right)$ đồng biến trên khoảng nào dưới đây?
		\choice
		{$\left(-\dfrac{5}{8}; 0\right)$}
		{$\left(\dfrac{5}{8};+\infty\right)$}
		{\True $\left(\dfrac{3}{8}; 1\right)$}
		{$\left(-\dfrac{3}{8}; 1\right)$}
	}
	{
		\begin{tikzpicture}[scale=0.25,>=stealth, font=\footnotesize, line join=round, line cap=round]
			\def\a{-.078} \def\b{1.25} \def\c{0} % Hệ số
			\def\xmin{-4} \def\xmax{25}
			\def\ymin{-8} \def\ymax{18}
			
			%\draw[color=gray!50,dashed] (\xmin,\ymin) grid (\xmax,\ymax);
			
			\draw[->] (\xmin,0)--(\xmax,0) node [below]{$x$};
			\draw[->] (0,\ymin)--(0,\ymax) node [left]{$y$};
			\node at (20,14) [below right]{$y=g'(x)$};
			\node at (18,-2) [below left]{$y=h'(x)$};
			\node at (16,5) [below right]{$y=f'(x)$};
			\node at (0,0) [below left]{$O$};
			\draw[dashed] (3,0) node[below]{$3$}--(3,10)--(0,10) node[left]{$10$};
			\draw[dashed] (8,0) node[below]{$8$}--(8,5)--(0,5) node[left]{$5$};
			\draw[dashed] (4,0) node[below]{$4$}--(4,2)--(0,2) node[left]{$2$};
			\clip (\xmin+0.1,\ymin+0.1) rectangle (\xmax-0.5,\ymax-0.1);
			\draw[smooth,samples=300,domain=-2:18] plot(\x,{\a*(\x)^2+\b*(\x)+\c});
			%\draw[smooth,samples=300,domain=-2:25] plot(\x,{0.02*(\x)^3-0.6*(\x)^2+5.16*(\x)});
			\draw[line width=1.2pt] (-2,5)..controls (1.7,1.5) and (4.5,1.6)..(7,2.6);
			\draw[line width=1.2pt] (7,2.6)..controls (9,3.5) and (12,5)..(20,13);
			\draw (-0.5,-2) -- (0,0)--(3,10).. controls +(65:1) and + (-190:1)..(6,15).. controls +(0:1) and + (-180:1)..(14,-1).. controls +(0:1) and + (+80:1)..(19,16);
			
		\end{tikzpicture}
	}
	\loigiai{
		Ta có $k'(x)=f'(x+7)+5 g'(5 x+1)-4 h'\left(4 x+\dfrac{3}{2}\right)$.\\
		Khi $x \in \left( \dfrac{3}{8};1\right)$ thì $\heva{&7{,}375<x+7<8\\&2{,}875<5x+1<6\\&3<4x+\dfrac{4}{3}<5{,}5}\Leftrightarrow \heva{&f'(x+7)>10\\&g'(5x+1)>2 \Rightarrow 5g'(5x+1)>10  \\&h'\left( 4x+\dfrac{3}{2}\right)<5 \Rightarrow -4h'\left( 4x+\dfrac{3}{2}\right) >-20}.$\\
		Do đó $k'(x)=f'(x+7)+5g'(5x+1)-4h'\left( 4x+\dfrac{3}{2}\right)>0$.\\
		Hàm số $k(x)=f(x+7)+g(5 x+1)-h\left(4 x+\dfrac{3}{2}\right)$ đồng biến trên $\left(\dfrac{3}{8}; 1\right)$.
	}
\end{ex}
\begin{ex}[THPT Thanh Chương 1 - Nghệ An- 2021] %[2D1G1-2]
	Cho hàm số $y=f(x)$ liên tục trên $\mathbb{R}$ có bảng xét dấu đạo hàm như sau
	\begin{center}
		\begin{tikzpicture}
			\tkzTabInit[nocadre,lgt=1.2,espcl=2,deltacl=0.6]
			{$x$ /0.6,$f'(x)$ /0.6}
			{$-\infty$,$1$,$2$,$3$,$4$,$+\infty$}
			\tkzTabLine{,-,$0$,+,$0$,+,$0$,-,$0$,+,}
		\end{tikzpicture}
	\end{center}
	Hàm số $y=3f(2x-1)-4x^3+15x^2-18x+1$ đồng biến trên khoảng nào dưới đây?
	\choice
	{$\left(3;+\infty\right)$}
	{\True $\left(1;\dfrac{3}{2}\right)$}
	{$\left(\dfrac{5}{2}; 3\right)$}
	{$\left(2;\dfrac{5}{2}\right)$}
	\loigiai{
		Ta có $y'=6f'(2x-1)-12x^2+30x-18=6\left[f'(2x-1)-2x^2+5x-3\right] $.\\
		Có $f'(2x-1)=0 \Leftrightarrow \hoac{&2x-1=1\\&2x-1=2\\&2x-1=3\\&2x-1=4} \Leftrightarrow \hoac{&x=1\\&x=\dfrac{3}{2}\\&x=2\\&x=\dfrac{5}{2}.}$
		Ta có bảng xét dấu sau
		\begin{center}
			\begin{tikzpicture}
				\tkzTabInit[nocadre,lgt=3.0,espcl=1.5,deltacl=0.6]
				{$x$ /1.0,$f(x)$ /0.6,$f'(2x-1)$ /0.6,$-2x^2+5x-3$/0.6,$g'(x)$/0.6}
				{$-\infty$,$1$,$\dfrac{3}{2}$,$2$,$\dfrac{5}{2}$,$3$,$4$,$+\infty$}
				\tkzTabLine{,-,$0$,+,|,+,$0$,+,|,+,$0$,-,$0$,+,}
				\tkzTabLine{,-,$0$,+,$0$,+,$0$,-,$0$,+,|,+,|,+,}
				\tkzTabLine{,-,$0$,+,$0$,-,|,-,|,-,|,-,|,-,}
				\tkzTabLine{,-,$0$,+,$0$,,?,,|,,?,?,,?,}
			\end{tikzpicture}
		\end{center}
		Dựa vào bảng xét dấu trên, ta kết luận hàm số đã cho đồng biến trên khoảng $\left( 1; \dfrac{3}{2}\right).$
	}
\end{ex}


\begin{ex}%[2D2G4-3] %Câu 27 
	[THPT Hoàng Hoa Thám-Đà Nẵng-2021]
	Cho hàm số $f(x)$ có bảng xét dấu của $f'(x)$ như sau:\\
	\begin{center}
		\begin{tikzpicture}
			\tkzTabInit[lgt=1.2,espcl=2.3]
			{$x$/0.7, $f'(x)$ /.8} % first column
			{$-\infty$,$-3$,$1$, $2$, $+\infty$} % first row
			\tkzTabLine { ,+,0,-,0,+,0,+ }
		\end{tikzpicture}
	\end{center}	
	Hàm số $y=f\left(2-e^x\right)-\dfrac{1}{3}{e^{3x}}+3e^{2x}-5e^x+1$ đồng biến trên khoảng nào dưới đây?
	\choice
	{$\left(0;\dfrac{3}{2}\right)$}
	{$\left(1;3\right)$}
	{\True $\left(-3;0\right)$}
	{$\left(-4;-3\right)$}
	\loigiai{
		Ta có $y'=-e^x.f'\left(2-e^x\right)-e^{3x}+6e^{2x}-5e^x=e^x\left[-f'\left(2-e^x\right)-e^{2x}+6e^x-5\right]$ .\\
		Đặt $t=2-e^x$, ta được\\
		$y'=\left(2-t\right)\left[-f'(t)-\left(2-t\right)^2+6\left(2-t\right)-5\right]=\left(2-t\right)\left[-f'(t)-t^2-2t+3\right]$ .\\
		$y'=0\Leftrightarrow\left(2-t\right)\left[-f'(t)-t^2-2t+3\right]=0\Leftrightarrow
		\hoac{
			& t=2\\ 
			& f'(t)=-t^2-2t+3.}$\\
		Hàm số $g(x)=-x^2-2x+3$ là parabol có trục đối xứng $x=-1$ và cắt trục hoành tại 2 điểm có hoành độ 
		$\hoac{
			& x=1\\ 
			& x=-3
		}$. Suy ra $f'(t)=-t^2-2t+3\Leftrightarrow \hoac{
			& t=1\\ 
			& t=-3. }$\\
		Bảng xét dấu\\
		\begin{center}
			\begin{tikzpicture}
				\tkzTabInit[lgt=3.9,espcl=2,nocadre]
				{$t$/0.7, $2-t$ /0.8, $-f'(t)-t^2-2t+3$ /0.8, $y'$ /0.8} % first column
				{$-\infty$,$-3$,$1$,$2$,$+\infty$} % first row
				\tkzTabLine { ,+,|,+,|,+,z,-, } % second row
				\tkzTabLine {,-,0,+,0,-,|,-,} % third row
				\tkzTabLine {,-,0,+,0,-,0,+,} % last row
			\end{tikzpicture}
		\end{center}
		Dựa vào bảng xét dấu $y'>0,\forall x\in\left(-3;0\right)$.}
\end{ex}


\begin{ex}%[2D1G1-2]%Câu 28 
	[Sở Lạng Sơn 2022] Cho hàm số $f(x)$ có bảng biến thiên như sau:\\
	\begin{center}
		\begin{tikzpicture}
			\tkzTabInit[espcl=2.5,lgt=1,nocadre]
			{$x$/0.7,$y'$/0.7,$y$/3.5}
			{$-\infty$,$1$,$2$,$3$,$4$,$+\infty$}
			\tkzTabLine{,+,0,-,0,+,0,-,0,+,}
			\node (0) at ($(N12)+(0,-3)$) {$-\infty$};
			\node (1) at ($(N22)+(0,-.5)$) {$3$};
			\node (2) at ($(N32)+(0,-1.7)$) {$1$};
			\node (3) at ($(N42)+(0,-0.7)$) {$2$};
			\node (4) at ($(N52)+(0,-2.3)$) {$0$};
			\node (5) at ($(N62)+(0,-.3)$) {$+\infty$};
			%				\node (8) at ($(N42)+(0,-.5)$) {};
			%				\coordinate (9) at ($(N42)!.6!(N53)+ (-0.5,0)$);
			%				\coordinate (6) at ($(T12)!.6!(T13)$);
			%				\coordinate (7) at ($(T22)!.6!(T23)$);
			\draw[-stealth] (0)--(1);
			\draw[-stealth] (1)--(2);
			\draw[-stealth] (2)--(3);
			\draw[-stealth] (1)--(2);
			\draw[-stealth] (3)--(4);
			\draw[-stealth] (4)--(5);
			%				\draw[->,red] (5)--(8);
			%				\draw[->,red] (8)--(9);
			%				\draw[blue,dashed](6)--(7)node[above left]{$y=0$};
		\end{tikzpicture}		
	\end{center}
	Hàm số $y=\left[f(x)\right]^3-3\left[f(x)\right]^2$ đồng biến trên khoảng nào dưới đây?
	\choice
	{$\left(-\infty\,;1\right)$}
	{$\left(1\,;2\right)$}
	{\True $\left(3\,;4\right)$}
	{$\left(2\,;3\right)$}
	\loigiai{
		Ta có $y'=3f'(x)\left[f^2(x)-2f(x)\right]$. 
		Phương trình $y'=0\Leftrightarrow \hoac{
			&{f}'(x)=0\\ 
			& f(x)=0\\ 
			& f(x)=2.
		}$
		\begin{center}
			\begin{tikzpicture}
				\tkzTabInit[espcl=2.5,lgt=1.5]
				{$x$/0.7,$y'$/0.7,$y$/3.5}
				{$-\infty$,$1$,$2$,$3$,$4$,$+\infty$}
				\tkzTabLine{,+,0,-,0,+,0,-,0,+,}
				\node (0) at ($(N12)+(0,-3)$) {$-\infty$};
				\node (1) at ($(N22)+(0,-.3)$) {$3$};
				\node (2) at ($(N32)+(0,-1.7)$) {$1$};
				\node (3) at ($(N42)+(0,-0.8)$) {$2$};
				\node (4) at ($(N52)+(0,-2.3)$) {$0$};
				\node (5) at ($(N62)+(0,-.3)$) {$+\infty$};
				\node (a) at ($(N11)+(0.65,0.35)$) {$a$};
				\node (b) at ($(N11)+(2.0,0.4)$) {$b$};
				\node (c) at ($(N11)+(3.38,0.35)$) {$c$};
				\node (d) at ($(N11)+(11.85,0.4)$) {$d$};
				\node (6) at ($(N12)+(0,-0.8)$) {};
				\node (7) at ($(N62)+(0,-0.8)$) {};
				\node (8) at ($(N12)+(0,-2.3)$) {};
				\node (9) at ($(N62)+(0,-2.3)$) {};
				%				\node (8) at ($(N42)+(0,-.5)$) {};
				%				\coordinate (9) at ($(N42)!.6!(N53)+ (-0.5,0)$);
				\coordinate (A) at ($(0)!.25!(1)$);
				\coordinate (B) at ($(0)!.8!(1)$);
				\coordinate (C) at ($(1)!.35!(2)$);
				\coordinate (D) at ($(4)!.75!(5)$);
				%				\coordinate (7) at ($(T22)!.6!(T23)$);
				\draw[->] (0)--(1);
				\draw[->] (1)--(2);
				\draw[->] (2)--(3);
				\draw[->] (1)--(2);
				\draw[->] (3)--(4);
				\draw[->] (4)--(5);
				%				\draw[->,red] (5)--(8);
				%				\draw[->,red] (8)--(9);
				\draw[blue,dashed](6)--(7)node[below]{$y=2$} (a)--(A) (b)--(B) (c)--(C) (d)--(D);
				\draw[blue,dashed](8)--(9)node[below left]{$y=0$};
			\end{tikzpicture}		
		\end{center}
		Dựa vào bảng biến thiên, ta thấy $f'(x)=0\Leftrightarrow x\in \{ 1\,;2\,;3\,;4 \}$;\\
		$f(x)=0\Leftrightarrow x=a<1$ hoặc $x=4$;\\
		$f(x)=2\Leftrightarrow \hoac{
			& x=b\,\,\left(a<b<1\right)\\ 
			& x=c\in\left(1\,;2\right)\\ 
			& x=3\\ 
			& x=d>4.
		}$ \\
		Ta lập được bảng xét dấu của $y'$ 
		\begin{center}
			\begin{tikzpicture}
				\tkzTabInit[lgt=1.2,espcl=1.5,nocadre]
				{$x$/1, $f(x)$ /.8} % first column
				{$-\infty$,$a$, $b$, $1$,$c$, $2$,$3$, $4$, $d$, $+\infty$} % first row
				\tkzTabLine { ,+,z,-,z,+,z,-,z,+,z,-,z,+,z,-,z,+, } % second row
				%				\tkzTabLine {,-,z,+,t,+,} % third row
				%				\tkzTabLine {,+,d,-,z,+,} % last row
			\end{tikzpicture}
		\end{center}
		Từ bảng xét dấu, ta thấy hàm số đồng biến trên các khoảng \\
		$\left(-\infty;a\right)$, $\left(b;1\right)$, $\left(c;2\right)$, $\left(3;4\right)$ và $(d;+\infty)$.
	}
\end{ex}

\begin{ex}%[2D1G1-2]%Câu 29 
	[THPT Bùi Thị Xuân – Huế-2022] 
	\immini{
		Cho hàm số $y=f(x)$ là hàm đa thức bậc bốn. Đồ thị hàm số $f'(x+2)$ được cho trong hình vẽ bên. Hàm số 
		$$g(x)=4 f\left(x^2\right)-x^6+5 x^4-4 x^2+1$$
		đồng biến trên khoảng nào dưới đây?
		\choice
		{$(-4 ;-3)$}
		{\True $(2 ;+\infty)$}
		{$(-\sqrt{2};\sqrt{2})$}
		{$(-2 ;-1)$}}{
		\begin{tikzpicture}[scale=0.6,font=\footnotesize, line join=round, line cap=round, >=stealth] %Đường cong bậc 3
			\draw[thick, ->] (-5.3,0)--(5,0);
			\draw[thick, ->] (0,-3.5)--(0,7);
			\draw (5.2,0) node[below] {$x$};
			\draw (0,7.1) node[left]{$y$};
			\draw (0,0) node[below left]{$0$};
			\draw[fill] (-2,0) circle (0.5pt)node[below left]{$ -2 $};
			\draw[fill] (2,0) circle (0.5pt)node[below]{$ 2$};
			\draw[fill] (0,3) circle (0.5pt)node[left]{$ 3 $};
			\draw[fill] (0,1) circle (0.5pt)node[right]{$ 1 $};
			\draw[fill] (0,-1) circle (0.5pt)node[right]{$ -1 $};
			\draw[dashed] (-2,0)--(-2,1) --(0,1); 
			\draw[dashed](2,0)--(2,3)--(0,3);
			\draw[line width=1.2pt,smooth,samples=100,domain=-2.8:4.5] plot(\x,{-0.271*(\x)^3+0.75*(\x)^2+1.583*\x-1});
		\end{tikzpicture}		
	}
	\loigiai{
		$\begin{aligned}
			& g(x)=4f\left(x^2\right)-x^6+5x^4-4x^2+1\Rightarrow g' (x)=8xf'\left(x^2\right)-6x^5+20x^3-8x.\\ 
			& g' (x)=0\Leftrightarrow 8xf'\left(x^2\right)-6x^5+20x^3-8x=0 \\
			& \Leftrightarrow 2x\left[4f'\left(x^2\right)-3x^4+10x^2-4\right]=0\\ 
			&\Leftrightarrow 		\hoac{ 			& 2x=0\\ 
				& 4f'(x^2)-3x^4+10x^2-4=0
			}
			\Leftrightarrow \hoac{	& x=0\\ 
				& f'\left(x^2\right)=\dfrac{3}{4}{x^4}-\dfrac{5}{2}{x^2}+1.}
		\end{aligned}$\\ 
		Xét
		$f'\left(x^2\right)=\dfrac{3}{4}x^4-\dfrac{5}{2}x^2+1$. Đặt $x^2=t+2$, ta có\\
		$ f' (t+2)=\dfrac{3}{4}{(t+2)^2}-\dfrac{5}{2}(t+2)+1=\dfrac{3}{4}\left(t^2+4t+4\right)-\dfrac{5}{2}(t+2)-1=\dfrac{3}{4}{t^2}+\dfrac{1}{2}t-1$\\
		Khi đó số nghiệm của phương trình chính là số giao điểm của đồ thị hàm số $y=f' (t+2)$ và\\
		$ y=\dfrac{3}{4}{t^2}+\dfrac{1}{2}t-1$\\
		Ta có đồ thị 
		\begin{center}
			\begin{tikzpicture}[scale=0.6,font=\footnotesize, line join=round, line cap=round, >=stealth] %Đường cong bậc 3
				\draw[thick, ->] (-5.3,0)--(5,0);
				\draw[thick, ->] (0,-3.5)--(0,7);
				\draw (5.2,0) node[below] {$x$};
				\draw (0,7.1) node[left]{$y$};
				\draw (0,0) node[below left]{$0$};
				\draw[fill] (-2,0) circle (0.5pt)node[below left]{$ -2 $};
				\draw[fill] (2,0) circle (0.5pt)node[below]{$ 2$};
				\draw[fill] (0,3) circle (0.5pt)node[left]{$ 3 $};
				\draw[fill] (0,1) circle (0.5pt)node[right]{$ 1 $};
				\draw[fill] (0,-1) circle (0.5pt)node[right]{$ -1 $};
				\draw[dashed] (-2,0)--(-2,1) --(0,1); 
				\draw[dashed](2,0)--(2,3)--(0,3);
				\draw[line width=1.2pt,smooth,samples=100,domain=-2.8:4.5] plot(\x,{-0.271*(\x)^3+0.75*(\x)^2+1.583*\x-1});		
				\draw[line width=1.2pt,smooth,samples=100,domain=-3.3:2.8] plot(\x,{0.75*(\x)^2+0.5*\x-1});
			\end{tikzpicture}
		\end{center}
		Dựa vào đồ thị ta có $f' (t+2)=\dfrac{3}{4}t^2+\dfrac{1}{2}t-1\Leftrightarrow \hoac{& t=-2\\ & t=0\\ & t=2} \Leftrightarrow\hoac{& x+2=-2\\ & x+2=0\\ & x+2=2} \Leftrightarrow \hoac{& x=-4\\ & x=-2\\ & x=0.}$\\
		Ta có bảng xét dấu $g' (x)$ như sau
		\begin{center}
			\begin{tikzpicture}
				\tkzTabInit[lgt=1.2,espcl=2,nocadre]
				{$x$/0.7, $f(x)$ /.7}
				{$-\infty$, $-4$,$-2$, $0$, $+\infty$} % first row
				\tkzTabLine { ,-,z,+,z,-,z,+, }
			\end{tikzpicture}
		\end{center}
		Vậy hàm số $g(x)=4 f\left(x^2\right)-x^6+5 x^4-4 x^2+1$ đồng biến trên khoảng $(2 ;+\infty)$.}
\end{ex}

\begin{ex}%[2D1G1-2]%Câu 30
	[Chuyên Bắc Ninh 2022] 
	\immini{
		Cho hàm số $ y=f(x)$ liên tục trên $\mathbb{R}$ có đồ thị hàm số $ y=f'(x)$ có đồ thị như hình vẽ bên.
		Hàm số $g(x)=2f\left(\left| x-1\right|\right)-x^2+2x+2020$ đồng biến trên khoảng nào
		\choice
		{$\left(-2;0\right)$}
		{$\left(-3;1\right)$}
		{$\left(1\,;3\right)$}
		{\True $\left(0\,;\,1\right)$}}{
		\begin{tikzpicture}[scale=0.6,font=\footnotesize, line join=round, line cap=round, >=stealth] %Đường cong bậc 3
			\draw[thick, ->] (-3.3,0)--(5,0);
			\draw[thick, ->] (0,-3.0)--(0,5.5);
			\draw (5.2,0) node[below] {$x$};
			\draw (0,5.8) node[left]{$y$};
			\draw (0,0) node[below left]{$0$};
			\draw[fill] (-1,0) circle (0.5pt)node[above]{$ -1 $};
			\draw[fill] (1,0) circle (0.5pt)node[below]{$ 1$};
			\draw[fill] (0,1) circle (0.5pt)node[left]{$ 1 $};
			\draw[fill] (0,-1) circle (0.5pt)node[right]{$ -1 $};
			\draw[fill] (0,3) circle (0.5pt)node[left]{$ 3 $};
			\draw[fill] (3,0) circle (0.5pt)node[below]{$ 3 $};
			\draw[dashed] (-1,0)--(-1,-1) --(0,-1); 
			\draw[dashed](1,0)--(1,1)--(0,1);
			\draw[dashed](3,0)--(3,3)--(0,3);
			\draw[line width=1.2pt,smooth,samples=100,domain=-2.2:4.3] plot(\x,{-0.333*(\x)^3+1*(\x)^2+1.333*\x-1});		
			%\draw[line width=1.2pt,smooth,samples=100,domain=-3.3:2.8] plot(\x,{0.75*(\x)^2+0.5*\x-1});
		\end{tikzpicture}	
	}
	\loigiai{
		Ta có $g(x)=2f\left(\left| x-1\right|\right)-x^2+2x+2020\Leftrightarrow g(x)=2f\left(\left| x-1\right|\right)-\left(x-1\right)^2+2021$.\\
		Xét hàm số $ k\left(x-1\right)=2f\left(x-1\right)-\left(x-1\right)^2+2021$.\\
		Đặt $ t=x-1$\\
		Xét hàm số $ h(t)=2f(t)-t^2+2021$ $\Rightarrow{h}'(t)=2f'(t)-2t$.\\
		Kẻ đường $ y=x$ như hình vẽ.
		\begin{center}
			\begin{tikzpicture}[scale=0.6,font=\footnotesize, line join=round, line cap=round, >=stealth] %Đường cong bậc 3
				\draw[thick, ->] (-3.3,0)--(5,0);
				\draw[thick, ->] (0,-3.0)--(0,5.5);
				\draw (5.2,0) node[below] {$x$};
				\draw (0,5.8) node[left]{$y$};
				%	\draw (0,0) node[below left]{$0$};
				\draw[fill] (-1,0) circle (0.5pt)node[above]{$ -1 $};
				\draw[fill] (1,0) circle (0.5pt)node[below]{$ 1$};
				\draw[fill] (0,1) circle (0.5pt)node[left]{$ 1 $};
				\draw[fill] (0,-1) circle (0.5pt)node[right]{$ -1 $};
				\draw[fill] (0,3) circle (0.5pt)node[left]{$ 3 $};
				\draw[fill] (3,0) circle (0.5pt)node[below]{$ 3 $};
				\draw[dashed] (-1,0)--(-1,-1) --(0,-1); 
				\draw[dashed](1,0)--(1,1)--(0,1);
				\draw[dashed](3,0)--(3,3)--(0,3);
				\draw[line width=1.2pt,smooth,samples=100,domain=-2.2:4.3] plot(\x,{-0.333*(\x)^3+1*(\x)^2+1.333*\x-1});		
				%\draw[line width=1.2pt,smooth,samples=100,domain=-3.3:2.8] plot(\x,{0.75*(\x)^2+0.5*\x-1});
				\draw[line width=1.2pt,smooth,samples=100](-2,-2)--(4,4);
			\end{tikzpicture}
		\end{center}
		Khi đó $h'(t)>0\Leftrightarrow{f}'(t)-t>0\Leftrightarrow{f}'(t)>t$$\Leftrightarrow \hoac{
			& t<-1\\ 
			& 1<t<3.
		}$\\
		Do đó $k'\left(x-1\right)>0\Leftrightarrow \hoac{
			& x-1<-1\\ 
			& 1<x-1<3} \Leftrightarrow \hoac{
			& x<0\\ 
			& 2<x<4.}$\\
		Ta có bảng biến thiên của hàm số $ k\left(x-1\right)=2f\left(x-1\right)-\left(x-1\right)^2+2021$.
		\begin{center}
			\begin{tikzpicture}
				\tkzTabInit[lgt=1.8,espcl=2.3]
				{$x$ /1.2, $k'(x-1)$ /1.2,$k(x-1)$ /2}
				{$-\infty$ , $0$,$2$,$4$, $+\infty$}
				\tkzTabLine{,+,0,-,0,+,0,-,}
				\tkzTabVar{-/$ $ ,+/$ $, -/$ $,+/$ $,-/$ $}
			\end{tikzpicture}
		\end{center}
		Khi đó, ta có bảng biến thiên của $g(x)=2f\left(\left| x-1\right|\right)-\left(x-1\right)^2+2021$ bằng cách lấy đối xứng qua đường thẳng $ x=1$ như sau\\
		\begin{center}
			\begin{tikzpicture}
				\tkzTabInit[lgt=1.2,espcl=2.5,nocadre]
				{$x$ /0.7, $g'(x)$ /0.7,$g(x)$ /2.5}
				{$-\infty$ ,$-2$, $0$,$1$,$2$,$4$, $+\infty$}
				\tkzTabLine{,+,0,-,0,+,0,-,0,+,0,-,}
				\tkzTabVar{-/$ $ ,+/$ $, -/$ $,+/$ $,-/$ $,+/ $ $,-/$ $}
			\end{tikzpicture}
		\end{center}
		Vậy hàm số đồng biến trên $\left(0;1\right)$.}
\end{ex}

\begin{ex}%[2D1G1-2]%Câu 31
	[Chuyên Thái Bình 2022] 
	\immini{
		Cho hàm số $f(x)=a{x^4}+b{x^3}+c{x^2}+dx+a$ có đồ thị hàm số $y=f'(x)$ như hình vẽ bên. Hàm số $y=g(x)=f\left(1-2x\right)f\left(2-x\right)$ đồng biến trên khoảng nào dưới đây?
		\choice
		{$\left(\dfrac{1}{2};\dfrac{3}{2}\right)$}
		{$\left(-\infty ;0\right)$}
		{$\left(0;2\right)$}
		{\True $\left(3;+\infty\right)$}}{
		\begin{tikzpicture}[scale=0.9,font=\footnotesize, line join=round, line cap=round, >=stealth] %Đường cong bậc 3
			\draw[thick, ->] (-2.5,0)--(2.5,0);
			\draw[thick, ->] (0,-2.8)--(0,2.8);
			\draw (2.6,0) node[below] {$x$};
			\draw (0,2.9) node[left]{$y$};
			\draw (0,0) node[below left]{$0$};
			\draw[fill] (-1,0) circle (0.5pt)node[below left]{$ -1 $};
			\draw[fill] (1,0) circle (0.5pt)node[below right]{$ 1$};
			%			\draw[dashed] (-1,0)--(-1,-1) --(0,-1); 
			%			\draw[dashed](1,0)--(1,1)--(0,1);
			%			\draw[dashed](3,0)--(3,3)--(0,3);
			\draw[line width=1.2pt,smooth,samples=100,domain=-1.3:1.3] plot(\x,{3*(\x)^3-3*\x});		
			%\draw[line width=1.2pt,smooth,samples=100,domain=-3.3:2.8] plot(\x,{0.75*(\x)^2+0.5*\x-1});
		\end{tikzpicture}	
	}
	\loigiai{
		Ta có $f'(x)=4a{x^3}+3b{x^2}+2cx+d$, theo đồ thị thì đa thức $f'(x)$ có ba nghiệm phân biệt là $-1,0,1$ nên $f'(x)=4ax\left(x+1\right)\left(x-1\right)=4a{x^3}-4ax\Rightarrow f(x)=a{x^4}-2a{x^2}+a=a{\left(x^2-1\right)^2}$.\\
		Dựa vào đồ thị hàm số $y=f'(x)$ ta có $a>0$ nên $f(x)>0,\forall x\in\mathbb{R}\setminus\left\{\pm 1\right\}$.\\
		$g'(x)=\left[f\left(1-2x\right)\right]'f\left(2-x\right)+f\left(1-2x\right)\left[f\left(2-x\right)\right]'=-2f'\left(1-2x\right)f\left(2-x\right)-f\left(1-2x\right)f'\left(2-x\right)$. Xét $x\in\left(\dfrac{1}{2};\dfrac{3}{2}\right)\Rightarrow
		\heva{		
			& 1-2x\in\left(-2;0\right)\\ 
			& 2-x\in\left(\dfrac{1}{2};\dfrac{3}{2}\right)}$, dấu của $f'(x)$ không cố định trên $\left(\dfrac{1}{2};\dfrac{3}{2}\right)$ nên ta không kết luận được tính đơn điệu của hàm số $g(x)$ trên $\left(\dfrac{1}{2};\dfrac{3}{2}\right)$.\\
		Xét $x\in\left(-\infty ;0\right)\Rightarrow
		\heva{
			& 1-2x\in\left(1;+\infty\right)\\ 
			& 2-x\in\left(2;+\infty\right)} 
		\Rightarrow \heva{
			& f'\left(1-2x\right)>0\\ 
			& f'\left(2-x\right)>0} \Rightarrow g'(x)<0$.\\
		Do đó, hàm số $g(x)$ nghịch biến trên $\left(-\infty ;0\right)$.\\
		$x\in\left(0;2\right)\Rightarrow \heva{
			& 1-2x\in\left(-3;1\right)\\ 
			& 2-x\in\left(0;2\right)}$, dấu của $f'(x)$ không cố định trên $\left(-3;1\right)$ và $\left(0;2\right)$ nên ta không kết luận được tính đơn điệu của hàm số $g(x)$ trên $\left(\dfrac{1}{2};\dfrac{3}{2}\right)$.\\
		Xét $x\in\left(3;+\infty\right)\Rightarrow \heva{
			& 1-2x\in\left(-\infty ;-5\right)\\ 
			& 2-x\in\left(-\infty ;-1\right)} \Rightarrow \heva{
			& f'\left(1-2x\right)<0\\ 
			& f'\left(2-x\right)<0} \Rightarrow g'(x)>0$. \\
		Do đó, hàm số $g(x)$ đồng biến trên $\left(3;+\infty\right)$.}
\end{ex}

\begin{dang}{Bài toán hàm ẩn, hàm hợp liên quan đến tham số và một số bài toán khác}
\end{dang}

\begin{ex}%[2D1G1-3]%Câu 1
	[Chuyên Lê Hồng Phong Nam Định 2019]
	\immini{
		Cho hàm số $ y=f(x)$ có đạo hàm liên tục trên $\mathbb{R}$. Biết hàm số $ y=f'(x)$ có đồ thị như hình vẽ. Gọi $ S$ là tập hợp các giá trị nguyên $ m\in\left[-5\,;\,\text{5}\right]$ để hàm số $ g(x)=f\left(x+m\right)$ nghịch biến trên khoảng $\left(1\,;\,2\right)$. Hỏi $S$ có bao nhiêu phần tử?
		\choice
		{$ 4$}
		{$ 3$}
		{$ 6$}
		{\True $ 5$}}{
		\begin{tikzpicture}[scale=0.9,font=\footnotesize, line join=round, line cap=round, >=stealth] %Đường cong bậc 3
			\draw[thick, ->] (-2.5,0)--(4,0);
			\draw[thick, ->] (0,-2.8)--(0,2.8);
			\draw (4.3,0) node[below] {$x$};
			\draw (0,2.9) node[left]{$y$};
			\draw (0,0) node[below left]{$0$};
			\draw[fill] (-1,0) circle (0.5pt)node[below left]{$ -1 $};
			\draw[fill] (1,0) circle (0.5pt)node[below]{$ 1$};
			\draw[fill] (3,0) circle (0.5pt)node[below right]{$ 3$};
			%			\draw[dashed] (-1,0)--(-1,-1) --(0,-1); 
			%			\draw[dashed](1,0)--(1,1)--(0,1);
			%			\draw[dashed](3,0)--(3,3)--(0,3);
			\draw[line width=1.2pt,smooth,samples=100,domain=-1.65:3.5] plot(\x,{0.33*(\x)^3-(\x)^2-0.333*(\x)+1});		
			%\draw[line width=1.2pt,smooth,samples=100,domain=-3.3:2.8] plot(\x,{0.75*(\x)^2+0.5*\x-1});
		\end{tikzpicture}	
	}
	\loigiai{
		Ta có $g'(x)=f'\left(x+m\right)$. Vì $ y=f'(x)$ liên tục trên $\mathbb{R}$ nên $g'(x)=f'\left(x+m\right)$ cũng liên tục trên $\mathbb{R}$. Căn cứ vào đồ thị hàm số $ y=f'(x)$ ta thấy\\
		$g'(x)<0\Leftrightarrow{f}'\left(x+m\right)<0$ $\Leftrightarrow\hoac{
			& x+m<-1\\ 
			& 1<x+m<3} \Leftrightarrow \hoac{
			& x<-1-m\\ 
			& 1-m<x<3-m.}$\\
		Hàm số $ g(x)=f\left(x+m\right)$ nghịch biến trên khoảng $\left(1\,;\,2\right)$
		$\Leftrightarrow \hoac{
			& 2\le-1-m\\ 
			&\hoac{
				& 3-m\ge 2\\ 
				& 1-m\le 1}} \Leftrightarrow \hoac{
			& m\le-3\\ 
			& 0\le m\le 1.}$\\
		Mà $ m$ là số nguyên thuộc đoạn $\left[-5\,;\,5\right]$ nên ta có $ S=\left\{-5;-4;-3;0;1\right\}$.\\
		Vậy $ S$ có $5$ phần tử.}
\end{ex}

\begin{ex}%[2D1G1-3]%Câu 2
	[Chuyên Nguyễn Bỉnh Khiêm-Quảng Nam-2020] Cho hàm số $ y=f(x)$ có đạo hàm trên $\mathbb{R}$ và bảng xét dấu đạo hàm như hình vẽ sau
	\begin{center}
		\begin{tikzpicture}
			\tkzTabInit[lgt=1.2,espcl=2.5,nocadre]
			{$x$/0.7, $f'(x)$ /2.5} % first column
			{$-\infty$, $-10$,$-2$, $3$,$8$, $+\infty$} % first row
			\tkzTabLine { ,+,z,-,z,+,z,-,z,+, } % second row
			%				\tkzTabLine {,-,z,+,t,+,} % third row
			%				\tkzTabLine {,+,d,-,z,+,} % last row
		\end{tikzpicture}
	\end{center}
	Có bao nhiêu số nguyên $ m$ để hàm số $ y=f\left(x^3+4x+m\right)$ nghịch biến trên khoảng $\left(-1;1\right)$?
	\choice
	{$ 3$}
	{$ 0$}
	{\True $ 1$}
	{$ 2$}
	\loigiai
	{
		Đặt $ t=x^3+4x+m\Rightarrow{t}'=3x^2+4$ nên $ t$ đồng biến trên $\left(-1;1\right)$ và $ t\in\left(m-5;m+5\right)$.\\
		Yêu cầu bài toán trở thành tìm $ m$ để hàm số $ f(t)$ nghịch biến trên khoảng $\left(m-5;m+5\right)$.\\
		Dựa vào bảng biến thiên ta được $\heva{
			& m-5\ge-2\\ 
			& m+5\le 8} \Leftrightarrow \heva{
			& m\ge 3\\ 
			& m\le 3} \Leftrightarrow m=3$.}
\end{ex}

\begin{ex}%[2D1G1-3]%Câu 3
	[Chuyên ĐH Vinh-Nghệ An-2020]
	\immini{
		Cho hàm số $ f(x)$ có đạo hàm trên $\mathbb{R}$và $ f(1)=1$. Đồ thị hàm số $ y=f'(x)$ như hình bên. Có bao nhiêu số nguyên dương $ a$ để hàm số $ y=\left| 4f\left(\sin x\right)+\cos 2x-a\right|$ nghịch biến trên $\left(0;\dfrac{\pi}{2}\right)$?
		\choice
		{$ 2$}
		{\True $ 3$}
		{Vô số}
		{$ 5$}}{
		\begin{tikzpicture}[scale=0.9,font=\footnotesize, line join=round, line cap=round, >=stealth] %Đường cong bậc 3
			\draw[thick, ->] (-2.5,0)--(3,0);
			\draw[thick, ->] (0,-2.8)--(0,2.8);
			\draw (3.1,0) node[below] {$x$};
			\draw (0,2.9) node[left]{$y$};
			\draw (0,0) node[below left]{$0$};
			\draw[fill] (-1,0) circle (0.5pt)node[below]{$ -1 $};
			\draw[fill] (1,0) circle (0.5pt)node[above]{$ 1$};
			%	\draw[fill] (3,0) circle (0.5pt)node[below right]{$ 3$};
			\draw[dashed] (-1,0)--(-1,1); 
			\draw[dashed](1,0)--(1,-1);
			%			\draw[dashed](3,0)--(3,3)--(0,3);
			\draw[line width=1.2pt,smooth,samples=100,domain=-2:2] plot(\x,{.8*(\x)^3+0*(\x)^2-1.8*(\x)});		
			%\draw[line width=1.2pt,smooth,samples=100,domain=-3.3:2.8] plot(\x,{0.75*(\x)^2+0.5*\x-1});
			\draw (2.0,2.8) node[left]{$y=f'(x)$};
		\end{tikzpicture}	
	}
	\loigiai
	{		Đặt $g(x)=\left| 4f\left(\sin x\right)+\cos 2x-a\right|\Rightarrow g(x)=\sqrt{\left[4f\left(\sin x\right)+\cos 2x-a\right]^2}$ .\\
		$\Rightarrow{g}'(x)=\dfrac{\left[4\cos x\cdot f'\left(\sin x\right)-2\sin 2x\right]\left[4f\left(\sin x\right)+\cos 2x-a\right]}{\sqrt{\left[4f\left(\sin x\right)+\cos 2x-a\right]^2}}$.\\
		Ta có $ 4\cos x\cdot f'\left(\sin x\right)-2\sin 2x=4\cos x\left[f'\left(\sin x\right)-\sin x\right]$.\\
		Với $ x\in\left(0;\dfrac{\pi}{2}\right)$ thì $\cos x>0,\sin x\in\left(0;1\right)\Rightarrow{f}'\left(\sin x\right)-\sin x<0$.\\
		Hàm số $ g(x)$ nghịch biến trên $\left(0;\dfrac{\pi}{2}\right)$ khi $ 4f\left(\sin x\right)+\cos 2x-a\ge 0,\forall x\in\left(0;\dfrac{\pi}{2}\right)$\\
		$\Leftrightarrow 4f\left(\sin x\right)+1-2\sin^2x\ge a,\forall x\in\left(0;\dfrac{\pi}{2}\right)$.\\
		Đặt $ t=\sin x$ được $ 4f(t)+1-2t^2\ge a,\forall t\in\left(0;1\right)$ (*).\\
		Xét $ h(t)=4f(t)+1-2t^2\Rightarrow{h}'(t)=4f'(t)-4t=4\left[f'(t)-1\right]$.\\
		Với $ t\in\left(0;1\right)$ thì $h'(t)<0\Rightarrow h(t)$ nghịch biến trên $\left(0;1\right)$.\\
		Do đó (*) $\Leftrightarrow a\le h(1)=4f(1)+1-2.1^2=3$.\\
		Vậy có $3$ giá trị nguyên dương của a thỏa mãn.}
\end{ex}


\begin{ex}%[2D1G1-3]%Câu 4
	[Chuyên Quang Trung-2020]
	\immini{
		Cho hàm số $ y=f(x)$ có đạo hàm liên tục trên $\mathbb{R}$ và có đồ thị $ y=f'(x)$ như hình vẽ. Đặt $ g(x)=f\left(x-m\right)-\dfrac{1}{2}{\left(x-m-1\right)^2}+2019$, với $ m$ là tham số thực. Gọi $ S$ là tập hợp các giá trị nguyên dương của $ m$ để hàm số $ y=g(x)$ đồng biến trên khoảng $\left(5;6\right)$. Tổng tất cả các phần tử trong $ S$ bằng
		\choice
		{$ 4$}
		{$ 11$}
		{\True $ 14$}
		{$ 20$}}{
		\begin{tikzpicture}[scale=0.9,font=\footnotesize, line join=round, line cap=round, >=stealth] %Đường cong bậc 3
			\draw[style=help lines,step=1] (-2.5,-3) grid (3,3.5);
			\draw[thick, ->] (-2.5,0)--(3.5,0);
			\draw[thick, ->] (0,-2.8)--(0,2.8);
			\draw (3.6,0) node[below] {$x$};
			\draw (0,3) node[above left]{$y$};
			\draw (0,0) node[below left]{$0$};
			%\draw[fill] (-1,0) circle (0.5pt)node[below]{$ -1 $};
			\draw[fill] (1,0) circle (0.5pt)node[below left]{$ 1$};
			%	\draw[fill] (3,0) circle (0.5pt)node[below right]{$ 3$};
			\draw[dashed] (-1,0)--(-1,-2) --(2,-2)--(2,0); 
			\draw[dashed](3,0)--(3,2) --(0,2);
			\draw (-1,-2) circle (2pt);
			\draw (3,2) circle (2pt);
			%			\draw[dashed](3,0)--(3,3)--(0,3);
			\draw[line width=1.2pt,smooth,samples=100,domain=-1.1:3.1] plot(\x,{1*(\x)^3-3*(\x)^2-0*(\x)+2});		
			%\draw[line width=1.2pt,smooth,samples=100,domain=-3.3:2.8] plot(\x,{0.75*(\x)^2+0.5*\x-1});
			%\draw (2.0,2.8) node[left]{$y=f'(x)$};
		\end{tikzpicture}	
	}
	\loigiai
	{
		Xét hàm số $ g(x)=f\left(x-m\right)-\dfrac{1}{2}{\left(x-m-1\right)^2}+2019$.\\
		$g'(x)=f'\left(x-m\right)-\left(x-m-1\right)$.\\
		Xét phương trình $g'(x)=0. \quad \quad (1)$\\
		Đặt $ x-m=t$, phương trình $(1)$ trở thành $f'(t)-\left(t-1\right)=0\Leftrightarrow{f}'(t)=t-1. \quad (2)$\\
		Nghiệm của phương trình $(2)$ là hoành độ giao điểm của hai đồ thị hàm số $ y=f'(t)$ và $ y=t-1$.\\
		Ta có đồ thị các hàm số $ y=f'(t)$ và $ y=t-1$ như sau
		\begin{center}
			\begin{tikzpicture}[scale=0.9,font=\footnotesize, line join=round, line cap=round, >=stealth] %Đường cong bậc 3
				\draw[style=help lines,step=1] (-2.5,-3) grid (3,3.5);
				\draw[thick, ->] (-2.5,0)--(3.5,0);
				\draw[thick, ->] (0,-2.8)--(0,2.8);
				\draw (3.6,0) node[below] {$x$};
				\draw (0,3) node[above left]{$y$};
				\draw (0,0) node[below left]{$0$};
				%\draw[fill] (-1,0) circle (0.5pt)node[below]{$ -1 $};
				\draw[fill] (1,0) circle (0.5pt)node[below left]{$ 1$};
				%	\draw[fill] (3,0) circle (0.5pt)node[below right]{$ 3$};
				\draw[dashed] (-1,0)--(-1,-2) --(2,-2)--(2,0); 
				\draw[dashed](3,0)--(3,2) --(0,2);
				\draw (-1,-2) circle (2pt);
				\draw (3,2) circle (2pt);
				%			\draw[dashed](3,0)--(3,3)--(0,3);
				\draw[line width=1.2pt,smooth,samples=100,domain=-1.1:3.1] plot(\x,{1*(\x)^3-3*(\x)^2-0*(\x)+2});		
				%\draw[line width=1.2pt,smooth,samples=100,domain=-3.3:2.8] plot(\x,{0.75*(\x)^2+0.5*\x-1});
				%\draw (2.0,2.8) node[left]{$y=f'(x)$};
				\draw (-2,-3)--(4,3);
			\end{tikzpicture}
		\end{center}
		Căn cứ đồ thị các hàm số ta có phương trình $(2)$ có nghiệm là $\hoac{
			& t=-1\\ 
			& t=1\\ 
			& t=3} \Rightarrow \hoac{
			& x=m-1\\ 
			& x=m+1\\ 
			& x=m+3.}$\\
		Ta có bảng biến thiên của $ y=g(x)$
		\begin{center}
			\begin{tikzpicture}
				\tkzTabInit[lgt=1,espcl=2.5,nocadre]
				{$x$ /0.8, $y'$ /0.8,$y$ /2.5}
				{$-\infty$ , $m-1$,$m+1$,$m+3$, $+\infty$}
				\tkzTabLine{,+,0,-,0,+,0,-,}
				\tkzTabVar{-/$ +\infty$ ,+/$ $, -/$ $,+/$ $,-/$+\infty $}
			\end{tikzpicture}
		\end{center}
		Để hàm số $ y=g(x)$ đồng biến trên khoảng $\left(5;6\right)$ cần $\hoac{
			&\heva{
				& m-1\le 5\\ 
				& m+1\ge 6}\\ 
			& m+3\le 5}\Leftrightarrow\hoac{
			& 5\le m\le 6\\ 
			& m\le 2.}$\\
		Vì $ m\in\mathbb{N}^*\Rightarrow m$ nhận các giá trị $ 1;\,2;\,5;\,6\Rightarrow S=14$.}
\end{ex}

\begin{ex}%[2D1G1-3]%Câu 5
	[Sở Hà Nội-Lần 2-2020] 
	\immini{
		Cho hàm số $y=a{x^4}+b{x^3}+c{x^2}+dx+e,\,\,a\ne 0$. Hàm số $y=f'(x)$ có đồ thị như hình vẽ bên. 
		Gọi S là tập hợp tất cả các giá trị nguyên thuộc khoảng $\left(-6;6\right)$ của tham số $m$ để hàm số $g(x)=f\left(3-2x+m\right)+x^2-\left(m+3\right)x+2m^2$ nghịch biến trên $\left(0;1\right)$. Khi đó, tổng giá trị các phần tử của S là
		\choice
		{$12$}
		{\True $9$}
		{$6$}
		{$15$}}{
		\begin{tikzpicture}[scale=0.7,font=\footnotesize, line join=round, line cap=round, >=stealth] %Đường cong bậc 3
			%	\draw[style=help lines,step=1] (-2.5,-3) grid (3,3.5);
			\draw[thick, ->] (-4.5,0)--(6.5,0);
			\draw[thick, ->] (0,-2.8)--(0,2.8);
			\draw (6.6,0) node[below] {$x$};
			\draw (0,3) node[above left]{$y$};
			\draw (0,0) node[below left]{$0$};
			\draw[fill] (-2,0) circle (0.5pt)node[below]{$ -2 $};
			\draw[fill] (4,0) circle (0.5pt)node[above]{$ 4$};
			\draw[fill] (0,1) circle (0.5pt)node[right]{$ 1 $};
			\draw[fill] (0,-2) circle (0.5pt)node[left]{$ -2$};
			%	\draw[fill] (3,0) circle (0.5pt)node[below right]{$ 3$};
			\draw[dashed] (-2,0)--(-2,1) --(0,1); 
			\draw[dashed](4,0)--(4,-2) --(0,-2);
			%			\draw[dashed](3,0)--(3,3)--(0,3);
			\draw[line width=1.2pt,smooth,samples=100,domain=-3.8:5.5] plot(\x,{0.0714*(\x)^3-0.1423*(\x)^2-1.0714*(\x)});		
			%\draw[line width=1.2pt,smooth,samples=100,domain=-3.3:2.8] plot(\x,{0.75*(\x)^2+0.5*\x-1});
			%\draw (2.0,2.8) node[left]{$y=f'(x)$};
		\end{tikzpicture}	
	}
	\loigiai
	{
		Xét $g'(x)=-2f'\left(3-2x+m\right)+2x-\left(m+3\right)$.\\
		Xét phương trình $g'(x)=0$, đặt $t=3-2x+m$ thì phương trình trở thành\\ $-2\cdot \left[f'(t)-\dfrac{-t}{2}\right]=0\Leftrightarrow\hoac{
			& t=-2\\ 
			& t=4\\ 
			& t=0.}$ \\
		Từ đó, $g'(x)=0\Leftrightarrow{x_1}=\dfrac{5+m}{2},\,x_2=\dfrac{m+3}{2},x_3=\dfrac{-1+m}{2}$.\\
		Lập bảng xét dấu, đồng thời lưu ý nếu $x>x_1$ thì $t<t_1$ nên $f(x)>0$. Và các dấu đan xen nhau do các nghiệm đều làm đổi dấu đạo hàm nên suy ra $g'(x)\le 0\Leftrightarrow x\in\left[x_2;{x_1}\right]\cup\left(-\infty ;{x_3}\right]$.\\
		Vì hàm số nghịch biến trên $\left(0;1\right)$ nên \\
		$g'(x)\le 0,\,\forall x\in\left(0;1\right)$ từ đó suy ra $\hoac{
			&\dfrac{3+m}{2}\le 0<1\le\dfrac{5+m}{2}\\ 
			& 1\le\dfrac{-1+m}{2}.}$ \\
		và giải ra các giá trị nguyên thuộc $\left(-6;6\right)$ của $m$ là $-3$; $3$; $4$; $5$. }
\end{ex}

\begin{ex}%[2D1G1-3]%Câu 6
	[Chuyên Quang Trung-Bình Phước-Lần 2-2020]
	\immini{
		Cho hàm số $ y=f(x)$ có đạo hàm liên tục trên $\mathbb{R}$ và có đồ thị $ y=f'(x)$ như hình vẽ bên. Đặt $ g(x)=f\left(x-m\right)-\dfrac{1}{2}{\left(x-m-1\right)^2}+2019$, với $ m$ là tham số thực. Gọi $ S$ là tập hợp các giá trị nguyên dương của $ m$ để hàm số $ y=g(x)$ đồng biến trên khoảng $\left(5;6\right)$. Tổng tất cả các phần tử trong $ S$ bằng
		\choice
		{$ 4$}
		{$ 11$}
		{\True $ 14$}
		{$ 20$}}{
		\begin{tikzpicture}[scale=0.9,font=\footnotesize, line join=round, line cap=round, >=stealth] %Đường cong bậc 3
			\draw[thick, ->] (-2.5,0)--(3.7,0);
			\draw[thick, ->] (0,-2.8)--(0,2.8);
			\draw (3.9,0) node[below] {$x$};
			\draw (0,2.9) node[left]{$y$};
			\draw (0,0) node[below left]{$0$};
			\draw[fill] (-1,0) circle (0.5pt)node[above]{$ -1 $};
			\draw[fill] (1,0) circle (0.5pt)node[below]{$ 1$};
			\draw[fill] (3,0) circle (0.5pt)node[below]{$ 3$};
			\draw[fill] (2,0) circle (0.5pt)node[above]{$ 2$};
			\draw[fill] (0,2) circle (0.5pt)node[above left]{$ 2$};
			\draw[fill] (0,-2) circle (0.5pt)node[below left]{$ -2$};
			\draw[dashed] (-1,0)--(-1,-2)--(2,-2)--(2,0); 
			\draw[dashed](3,0)--(3,2)--(0,2);
			%			\draw[dashed](3,0)--(3,3)--(0,3);
			\draw[line width=1.2pt,smooth,samples=100,domain=-1.1:3.1] plot(\x,{1*(\x)^3-3*(\x)^2-0*(\x)+2});		
			%\draw[line width=1.2pt,smooth,samples=100,domain=-3.3:2.8] plot(\x,{0.75*(\x)^2+0.5*\x-1});
			%	\draw (2.0,2.8) node[left]{$y=f'(x)$};
	\end{tikzpicture}	}
	\loigiai
	{
		Ta có $g'(x)=f'\left(x-m\right)-\left(x-m-1\right)$.\\
		Cho $g'(x)=0\Leftrightarrow{f}'\left(x-m\right)=x-m-1$.\\
		Đặt $ x-m=t\Rightarrow f'(t)=t-1$\\
		Khi đó nghiệm của phương trình là hoành độ giao điểm của đồ thị hàm số $ y=f'(t)$ và và đường thẳng $ y=t-1$.
		\begin{center}
			\begin{tikzpicture}[scale=0.9,font=\footnotesize, line join=round, line cap=round, >=stealth] %Đường cong bậc 3
				\draw[thick, ->] (-2.5,0)--(3.7,0);
				\draw[thick, ->] (0,-2.8)--(0,2.8);
				\draw (3.9,0) node[below] {$x$};
				\draw (0,2.9) node[left]{$y$};
				\draw (0,0) node[below left]{$0$};
				\draw[fill] (-1,0) circle (0.5pt)node[above]{$ -1 $};
				\draw[fill] (1,0) circle (0.5pt)node[below]{$ 1$};
				\draw[fill] (3,0) circle (0.5pt)node[below]{$ 3$};
				\draw[fill] (2,0) circle (0.5pt)node[above]{$ 2$};
				\draw[fill] (0,2) circle (0.5pt)node[above left]{$ 2$};
				\draw[fill] (0,-2) circle (0.5pt)node[below left]{$ -2$};
				\draw[dashed] (-1,0)--(-1,-2)--(2,-2)--(2,0); 
				\draw[dashed](3,0)--(3,2)--(0,2);
				%			\draw[dashed](3,0)--(3,3)--(0,3);
				\draw[line width=1.2pt,smooth,samples=100,domain=-1.1:3.1] plot(\x,{1*(\x)^3-3*(\x)^2-0*(\x)+2});		
				%\draw[line width=1.2pt,smooth,samples=100,domain=-3.3:2.8] plot(\x,{0.75*(\x)^2+0.5*\x-1});
				%	\draw (2.0,2.8) node[left]{$y=f'(x)$};
				\coordinate (a) at ($(-1,-2)!1.2!(3,2)$);
				\coordinate (b) at ($(-1,-2)!-.2!(3,2)$);
				\draw[line width=1.2pt,smooth] (a)--(b);
			\end{tikzpicture}
		\end{center}
		Dựa vào đồ thị hàm số ta có được $f'(t)=t-1\Leftrightarrow\hoac{
			& t=-1\\ 
			& t=1\\ 
			& t=3.} $ \\
		Bảng xét dấu của $g'(t)$
		\begin{center}
			\begin{tikzpicture}
				\tkzTabInit[lgt=1.2,espcl=2.5,nocadre]
				{$t$/1, $g'(x)$ /.8} % first column
				{$-\infty$, $-1$,$1$, $3$, $+\infty$} % first row
				\tkzTabLine { ,-,0,+,0,-,0,+, } % second row
				%				\tkzTabLine {,-,z,+,t,+,} % third row
				%				\tkzTabLine {,+,d,-,z,+,} % last row
			\end{tikzpicture}
		\end{center}
		Từ bảng xét dấu ta thấy hàm số $ g(t)$ đồng biến trên khoảng $\left(-1;1\right)$ và $\left(3;+\infty\right)$.\\
		Hay $\hoac{
			&-1<t<1\\ 
			& t>3}\Leftrightarrow\hoac{
			&-1<x-m<1\\ 
			& x-m>3} \Leftrightarrow\hoac{
			& m-1<x<m+1\\ 
			& x>m+3.}$\\
		Để hàm số $ g(x)$ đồng biến trên khoảng $\left(5;6\right)$ thì $\hoac{
			& m-1\le 5<6\le m+1\\ 
			& m+3\le 5<6} \Leftrightarrow\hoac{
			& 5\le m\le 6\\ 
			& m\le 2.}$\\
		Vì $ m$ là các số nguyên dương nên $ S=\left\{ 1;2;5;6\right\}$.\\
		Vậy tổng tất cả các phần tử của $ S$ là $ 1+2+5+6=14$.}
\end{ex}

\begin{ex}%[2D1G1-3]%Câu 7
	\immini{
		Cho hàm số $ y=f(x)$ liên tục có đạo hàm trên $\mathbb{R}$. Biết hàm số $ f'(x)$ có đồ thị cho như hình vẽ bên. Có bao nhiêu giá trị nguyên của $ m$ thuộc $\left[-2019;2019\right]$ để hàm só $ g(x)=f\left(2019^x\right)-mx+2$ đồng biến trên $\left[0;1\right]$.
		\choice
		{$ 2028$}
		{$ 2019$}
		{$ 2011$}
		{\True $ 2020$}}{
		\begin{tikzpicture}[scale=0.9,font=\footnotesize, line join=round, line cap=round, >=stealth] %Đường cong bậc 3
			\draw[thick, ->] (-3.5,0)--(2.5,0);
			\draw[thick, ->] (0,-2.8)--(0,2.8);
			\draw (2.7,0) node[below] {$x$};
			\draw (0,2.9) node[left]{$y$};
			\draw (0,0) node[below left]{$0$};
			%	\draw[fill] (-1,0) circle (0.5pt)node[above]{$ -1 $};
			\draw[fill] (1,0) circle (0.5pt)node[below right]{$ 1$};
			%		\draw[fill] (3,0) circle (0.5pt)node[below]{$ 3$};
			%		\draw[fill] (2,0) circle (0.5pt)node[above]{$ 2$};
			%		\draw[fill] (0,2) circle (0.5pt)node[above left]{$ 2$};
			%		\draw[fill] (0,-2) circle (0.5pt)node[below left]{$ -2$};
			%		\draw[dashed] (-1,0)--(-1,-2)--(2,-2)--(2,0); 
			%		\draw[dashed](3,0)--(3,2)--(0,2);
			\draw[line width=1.2pt,smooth,samples=100,domain=-3.28:1.32] plot(\x,{0.667*(\x)^3+2*(\x)^2-0.667*(\x)-2});		
			%\draw[line width=1.2pt,smooth,samples=100,domain=-3.3:2.8] plot(\x,{0.75*(\x)^2+0.5*\x-1});
			%	\draw (2.0,2.8) node[left]{$y=f'(x)$};
	\end{tikzpicture}	}
	\loigiai{
		Ta có $ g'(x)=2019^x\ln 2019\cdot f'\left(2019^x\right)-m$.\\
		Ta lại có hàm số $ y=2019^x$ đồng biến trên $\left[0;1\right]$.\\
		Với $ x\in\left[0;1\right]$ thì $2019^x\in\left[1;2019\right]$ mà hàm $ y=f'(x)$ đồng biến trên $\left(1;+\infty\right)$ nên hàm $ y=f'\left(2019^x\right)$ đồng biến trên $\left[0;1\right]$.\\
		Mà $2019^x\ge 1;f'\left(2019^x\right)>0\,\forall\,x\in\left[0;1\right]$ nên hàm $ h(x)=2019^x\ln 2019\cdot f'\left(2019^x\right)$ đồng biến trên $\left[0;1\right]$.\\
		Hay $ h(x)\ge h(0)=0,\forall\,x\in\left[0;1\right]$.\\
		Do vậy hàm số $ g(x)$ đồng biến trên đoạn $\left[0;1\right]$$\Leftrightarrow g'(x)\ge 0,\forall\,x\in\left[0;1\right]$\\
		$\Leftrightarrow m\le{2019^x}\ln 2019.f'\left(2019^x\right),\forall\,x\in\left[0;1\right]$ $\Leftrightarrow m\le\underset{x\in\left[0;1\right]}{\min}\,h(x)=h(0)=0$\\
		Vì $ m$ nguyên và $ m\in\left[-2019;2019\right]\Rightarrow $có $ 2020$ giá trị $ m$ thỏa mãn yêu cầu bài toán.}
\end{ex}

\begin{ex}%[2D1G1-3]%Câu 8
	\immini{
		Cho hàm số $y=f(x)$ có đồ thị $f'(x)\,$ như hình vẽ. Có bao nhiêu giá trị nguyên $m\in\left(-2020\,;\,2020\right)$ để hàm số $g(x)=f\left(2x-3\right)\,-\ln \left(1+x^2\right)-2mx$ đồng biến trên $\left(\dfrac{1}{2};2\right)$?
		\choice
		{$ 2020$}
		{\True $ 2019$}
		{$ 2021$}
		{$ 2018$}}{
		\begin{tikzpicture}[scale=0.9,font=\footnotesize, line join=round, line cap=round, >=stealth] %Đường cong bậc 3
			\draw[thick, ->] (-2.5,0)--(2.5,0);
			\draw[thick, ->] (0,-1.8)--(0,5.8);
			\draw (2.7,0) node[below] {$x$};
			\draw (0,5.9) node[left]{$y$};
			\draw (0,0) node[below left]{$0$};
			\draw[fill] (-2,0) circle (0.5pt)node[below]{$ -2 $};
			\draw[fill] (1,0) circle (0.5pt)node[below]{$ 1$};
			\draw[fill] (-1,0) circle (0.5pt)node[below]{$-1$};
			\draw[fill] (0,4) circle (0.5pt)node[above left]{$ 2$};
			%		\draw[fill] (0,2) circle (0.5pt)node[above left]{$ 2$};
			%		\draw[fill] (0,-2) circle (0.5pt)node[below left]{$ -2$};
			\draw[dashed] (-2,0)--(-2,4)--(1,4)--(1,0); 
			%		\draw[dashed](3,0)--(3,2)--(0,2);
			\draw[line width=1.2pt,smooth,samples=100,domain=-2.1:2.1] plot(\x,{-1*(\x)^3+0*(\x)^2+3*(\x)+2});		
			%\draw[line width=1.2pt,smooth,samples=100,domain=-3.3:2.8] plot(\x,{0.75*(\x)^2+0.5*\x-1});
			%	\draw (2.0,2.8) node[left]{$y=f'(x)$};
	\end{tikzpicture}	}
	\loigiai{
		Ta có $g'(x)=2f'\left(2x-3\right)-\dfrac{2x}{1+x^2}-2m$.\\
		Hàm số $ g(x)$ đồng biến trên $\left(\dfrac{1}{2};2\right)$ khi và chỉ khi \\
		$g'(x)\ge 0,\,\,\forall x\in\left(-1;\,2\right)$\\
		$\Leftrightarrow m\le{f}'\left(2x-3\right)-\dfrac{x}{1+x^2},\,\,\forall x\in\left(\dfrac{1}{2};2\right)$\\
		$\Leftrightarrow m\le\underset{x\in\left[\dfrac{1}{2};2\right]}{\min}\,\left[f'\left(2x-3\right)-\dfrac{x}{1+x^2}\right]$. \, \,  $(1)$\\
		Đặt $ t=2x-3$, khi đó $ x\in\left(\dfrac{1}{2};2\right)\Leftrightarrow t\in\left(-2;\,1\right)$.\\
		Từ đồ thị hàm $f'(x)$ suy ra $f'(t)\ge 0,\,\,\forall t\in\left(-2;1\right)$ và $f'(t)=0$ khi $ t=-1$.\\
		Tức là $f'\left(2x-3\right)\ge 0,\,\,\forall x\in\left(\dfrac{1}{2};\,2\right)$$\Rightarrow\underset{x\in\left[\dfrac{1}{2};2\right]}{\min}\,f'\left(2x-3\right)=0$ khi $ x=1$. $(2)$\\
		Xét hàm số $ h(x)=-\dfrac{x}{1+x^2}$ trên khoảng $\left(\dfrac{1}{2};\,2\right)$.\\
		Ta có $h'(x)=\dfrac{x^2-1}{\left(1+x^2\right)^2}$ và\\
		$h'(x)=0\Leftrightarrow{x^2}-1=0\Leftrightarrow x=\pm 1$.\\
		Bảng biến thiên của hàm số $ h(x)$ trên $\left(\dfrac{1}{2};\,2\right)$ như sau
		\begin{center}
			\begin{tikzpicture}
				\tkzTabInit[lgt=1.2,espcl=2.5,nocadre]
				{$x$ /0.7, $h'(x)$ /0.7,$h(x)$ /2.5}
				{$\dfrac{1}{2}$ , $1$,$2$}
				\tkzTabLine{,-,0,+,}
				\tkzTabVar{+/$  $ ,-/$ \-\dfrac{1}{2} $, +/$ $}
			\end{tikzpicture}
		\end{center}
		Từ bảng biến thiên suy ra $ h(x)\ge-\dfrac{1}{2}$$\Rightarrow\underset{x\in\left[\dfrac{1}{2};2\right]}{\min}\,h(x)=-\dfrac{1}{2}$ khi $ x=1$. \, \,  $(3)$\\
		Từ $(1)$, $(2)$ và $(3)$ suy ra $ m\le-\dfrac{1}{2}$.\\
		Kết hợp với $ m\in\mathbb{Z}$, $ m\in\left(-2020;\,2020\right)$ thì $ m\in\left\{-2019;\,-201;\ldots ;-2;-1\right\}$.\\
		Vậy có tất cả $ 2019$ giá trị $ m$ cần tìm.}
\end{ex}

\begin{ex}%[2D1G1-3]%Câu 9
	Cho hàm số $ f(x)$ liên tục trên $\mathbb{R}$ và có đạo hàm $f'(x)=x^2\left(x-2\right)\left(x^2-6x+m\right)$ với mọi $ x\in\mathbb{R}$. Có bao nhiêu số nguyên $ m$ thuộc đoạn $\left[-2020;2020\right]$ để hàm số $ g(x)=f\left(1-x\right)$ nghịch biến trên khoảng $\left(-\infty ;-1\right)$?
	\choice
	{$ 2016$}
	{$ 2014$}
	{\True $ 2012$}
	{$ 2010$}
	\loigiai{
		Ta có \\
		$g'(x)=f'\left(1-x\right)=-\left(1-x\right)^2\left(-x-1\right)\left[\left(1-x\right)^2-6\left(1-x\right)+m\right]$
		$=\left(x-1\right)^2\left(x+1\right)\left(x^2+4x+m-5\right)$.\\
		Hàm số $ g(x)$ nghịch biến trên khoảng $\left(-\infty ;-1\right)$\\
		$\Leftrightarrow{g}'(x)\le 0,\forall x<-1$ $(*)$, (dấu \lq\lq $=$\rq\rq \, xảy ra tại hữu hạn điểm).\\
		Với $ x<-1$ thì $\left(x-1\right)^2>0$ và $ x+1<0$ nên\\
		$(*)$ $\Leftrightarrow{x^2}+4x+m-5\ge 0,\forall x<-1 \Leftrightarrow m\ge-x^2-4x+5,\forall x<-1$.\\
		Xét hàm số $ y=-x^2-4x+5$ trên khoảng $\left(-\infty ;-1\right)$, ta có bảng biến thiên
		\begin{center}
			\begin{tikzpicture}
				\tkzTabInit[lgt=1.8,espcl=2.3]
				{$x$ /1.2, $y'$ /1.2,$y$ /2}
				{$-\infty$ , $-2$,$-1$}
				\tkzTabLine{,+,0,-,}
				\tkzTabVar{-/$ -\infty $ ,+/$9 $, -/$ 8$}
			\end{tikzpicture}
		\end{center}
		Từ bảng biến thiên suy ra $ m\ge 9$.\\
		Kết hợp với $ m$ thuộc đoạn $\left[-2020;2020\right]$ và $ m$ nguyên nên $ m\in\left\{ 9;10;11;\ldots ;2020\right\}$.\\
		Vậy có $ 2012$ số nguyên $ m$ thỏa mãn đề bài.}
\end{ex}

\begin{ex}%[2D1G1-3]%Câu 10
	\immini{
		Cho hàm số $f(x)$ xác định và liên tục trên $ R$. Hàm số $y=f'(x)$ liên tục trên $\mathbb{R}$ và có đồ thị như hình vẽ bên.
		Xét hàm số $g(x)=f\left(x-2m\right)+\dfrac{1}{2}{\left(2m-x\right)^2}+2020$, với $ m$ là tham số thực. Gọi $ S$ là tập hợp các giá trị nguyên dương của $ m$ để hàm số $ y=g(x)$ nghịch biến trên khoảng $\left(3;4\right)$. Hỏi số phần tử của $ S$ bằng bao nhiêu?
		\choice
		{$4$}
		{\True $2$}
		{$3$}
		{Vô số}}
	{
		\begin{tikzpicture}[scale=0.7,>=stealth, font=\footnotesize, line join=round, line cap=round]
			\def\xmin{-3.5} \def\xmax{4.5}
			\def\ymin{-5.2} \def\ymax{4}
			\clip(\xmin,\ymin) rectangle (\xmax,\ymax);
			\draw[->] (\xmin,0)--(\xmax,0) node [below]{$x$};
			\draw[->] (0,\ymin)--(0,\ymax) node [left]{$y$};
			\node at (0,0) [below left]{$O$};
			\path
			(-3.1,3.7) coordinate (A)
			(-3,3) coordinate (B)
			(0,-2) coordinate (C)
			(0.65,-2) coordinate (D)
			(1,-1) coordinate (E)
			(3,-3) coordinate (F)
			(3.4,-5) coordinate (G);
			\draw[smooth]
			(A)..controls +(-88:0.1) and +(93:.1)..
			(B)..controls +(-87:0.3) and +(-100:8.5)..
			(C)..controls +(75:.8) and +(180:.1)..
			(D)..controls +(0:.1) and +(-105:.3)..
			(E)..controls +(70:2) and +(97:0.4)..
			(F)..controls +(-80:.1) and +(90:0.3)..
			(G);
			\draw[dashed] 
			(-3,0)node[below]{$-3$}|-(0,3)node[right]{$3$}
			(1,0)node[above]{$1$}|-(0,-1)node[left]{$-1$}
			(3,0)node[above]{$3$}|-(0,-3)node[below right]{$-3$};
			\fill 
			(0,-2) circle(1.5pt)
			(-3,3) circle(1.5pt)
			(3,-3) circle(1.5pt)
			(1,-1) circle(1.5pt);
			\node at (2.1,-4) {$y=f'(x)$};
		\end{tikzpicture}
	}
	\loigiai{
		Ta có $g'(x)=f'\left(x-2m\right)-\left(2m-x\right)$.		Đặt $h(x)=f'(x)-\left(-x\right)$.\\
		Từ đồ thị hàm số $y=f'(x)$ và đồ thị hàm số $y=-x$ trên hình vẽ suy ra \\
		$h(x)\le 0\Leftrightarrow f'(x)\le-x\Leftrightarrow\hoac{
			&-3\le x\le 1\\ 
			& x\ge 3.}$ 
		\begin{center}
			\begin{tikzpicture}[scale=0.7,>=stealth, font=\footnotesize, line join=round, line cap=round]
				\def\xmin{-3.5} \def\xmax{4.5}
				\def\ymin{-5.2} \def\ymax{4}
				\clip(\xmin,\ymin) rectangle (\xmax,\ymax);
				\draw[->] (\xmin,0)--(\xmax,0) node [below]{$x$};
				\draw[->] (0,\ymin)--(0,\ymax) node [left]{$y$};
				\node at (0,0) [below left]{$O$};
				\path
				(-3.1,3.7) coordinate (A)
				(-3,3) coordinate (B)
				(0,-2) coordinate (C)
				(0.65,-2) coordinate (D)
				(1,-1) coordinate (E)
				(3,-3) coordinate (F)
				(3.4,-5) coordinate (G);
				\draw[smooth]
				(A)..controls +(-88:0.1) and +(93:.1)..
				(B)..controls +(-87:0.3) and +(-100:8.5)..
				(C)..controls +(75:.8) and +(180:.1)..
				(D)..controls +(0:.1) and +(-105:.3)..
				(E)..controls +(70:2) and +(97:0.4)..
				(F)..controls +(-80:.1) and +(90:0.3)..
				(G);
				\draw[dashed] 
				(-3,0)node[below]{$-3$}|-(0,3)node[right]{$3$}
				(1,0)node[above]{$1$}|-(0,-1)node[left]{$-1$}
				(3,0)node[above]{$3$}|-(0,-3)node[below right]{$-3$};
				\fill 
				(0,-2) circle(1.5pt)
				(-3,3) circle(1.5pt)
				(3,-3) circle(1.5pt)
				(1,-1) circle(1.5pt);
				\draw[smooth,samples=300,domain=-3.2:3.7] plot(\x,{-(\x)});
				\node at (2.1,-4) {$y=f'(x)$};
				\node at (-1,2.1) {$y=h(x)$};
			\end{tikzpicture}
		\end{center}
		Ta có $ g'(x)=h\left(x-2m\right)\le 0\Leftrightarrow\hoac{
			&-3\le x-2m\le 1\\ 
			& x-2m\ge 3}\Leftrightarrow\hoac{
			& 2m-3\le x\le 2m+1\\ 
			& x\ge 2m+3.}$.\\
		Suy ra hàm số $ y=g(x)$ nghịch biến trên các khoảng $\left(2m-3;2m+1\right)$ và $\left(2m+3;+\infty\right)$.\\
		Do đó hàm số $ y=g(x)$ nghịch biến trên khoảng $\left(3;4\right)$ $\Leftrightarrow\hoac{
			&\heva{
				& 2m-3\le 3\\ 
				& 2m+1\ge 4}\\ 
			& 2m+3\le 3}\Leftrightarrow\hoac{
			&\dfrac{3}{2}\le m\le 3\\ 
			& m\le 0.}$ \\
		Mặt khác, do $ m$ nguyên dương nên $ m\in\left\{ 2;3\right\}\Rightarrow S=\left\{ 2;3\right\}$. Vậy số phần tử của $ S$ bằng $2$.\\
	}
	
\end{ex}

\begin{ex}%[2D1G1-3]%Câu 11
	Cho hàm số $f(x)$ có đạo hàm trên $\mathbb{R}$ là $f'(x)=\left(x-1\right)\left(x+3\right)$. Có bao nhiêu giá trị nguyên của tham số $m$ thuộc đoạn $\left[-10;20\right]$ để hàm số $y=f\left(x^2+3x-m\right)$ đồng biến trên khoảng $\left(0;2\right)$?
	\choice
	{\True $ 18$}
	{$ 17$}
	{$ 16$}
	{$ 20$}
	\loigiai{
		Ta có $y'=f'\left(x^2+3x-m\right)=\left(2x+3\right){f}'\left(x^2+3x-m\right)$.\\
		Theo đề bài ta có $f'(x)=\left(x-1\right)\left(x+3\right)$\\
		suy ra $f'(x)>0\Leftrightarrow\hoac{
			& x<-3\\ 
			& x>1}$ và $f'(x)<0\Leftrightarrow-3<x<1$ .\\
		Hàm số đồng biến trên khoảng $\left(0;2\right)$ khi $y'\ge 0,\forall x\in\left(0;2\right)$\\
		$\Leftrightarrow\left(2x+3\right){f}'\left(x^2+3x-m\right)\ge 0,\forall x\in\left(0;2\right)$.\\
		Do $x\in\left(0;2\right)$ nên $2x+3>0,\forall x\in\left(0;2\right)$. Do đó, ta có\\
		$y'\ge 0,\forall x\in\left(0;2\right)\Leftrightarrow f'\left(x^2+3x-m\right)\ge 0$\\
		$\Leftrightarrow\hoac{
			&{x^2}+3x-m\le-3\\ 
			&{x^2}+3x-m\ge 1}\Leftrightarrow\hoac{
			& m\ge{x^2}+3x+3\\ 
			& m\le{x^2}+3x-1}$\\
		$\Leftrightarrow\hoac{
			& m\ge\underset{\left[0;2\right]}{\max}\,\left(x^2+3x+3\right)\\ 
			& m\le\underset{\left[0;2\right]}{\min}\,\left(x^2+3x-1\right)} \Leftrightarrow\hoac{
			& m\ge 13\\ 
			& m\le-1}$.\\
		Do $m\in\left[-10;20\right]$, $ m\in\mathbb{Z}$ nên có $ 18$ giá trị nguyên của $m$ thỏa yêu cầu đề bài.}
\end{ex}

\begin{ex}%[2D1G1-3]%Câu 12
	Cho các hàm số $f(x)=x^3+4x+m$ và $g(x)=\left(x^2+2018\right){\left(x^2+2019\right)^2}{\left(x^2+2020\right)^3}$ . Có bao nhiêu giá trị nguyên của tham số $m\in\left[-2020;2020\right]$ để hàm số $g\left(f(x)\right)$ đồng biến trên $\left(2;+\infty\right)$ ?
	\choice
	{$2005$}
	{\True $2037$}
	{$4016$}
	{$4041$}
	\loigiai{
		Ta có $f(x)=x^3+4x+m$ và \\
		$g(x)=\left(x^2+2018\right){\left(x^2+2019\right)^2}{\left(x^2+2020\right)^3}=a_{12}{x^{12}}+a_{10}{x^{10}}+...+a_2x^2+a_0$.\\
		Suy ra $f'(x)=3x^2+4$ , $g'(x)=12a_{12}{x^{11}}+10a_{10}{x^9}+...+2a_2x$.\\
		Và có 
		\begin{eqnarray*}
			\left[g\left(f(x)\right)\right]' &=& f'(x)\left[12a_{12}{\left(f(x)\right)^{11}}+10a_{10}{\left(f(x)\right)^9}+...+2a_2f(x)\right]\\
			&=& f(x)f'(x)\left(12a_{12}{\left(f(x)\right)^{10}}+10a_{10}{\left(f(x)\right)^8}+...+2a_2\right).
		\end{eqnarray*} 
		Dễ thấy $a_{12};{a_{10}};...;{a_2};{a_0}>0$ và $f'(x)=3x^2+4>0$, $\forall x>2$.\\
		Do đó $f'(x)\left(12a_{12}{\left(f(x)\right)^{10}}+10a_{10}{\left(f(x)\right)^8}+...+2a_2\right)>0$ , $\forall x>2$.\\
		Hàm số $g\left(f(x)\right)$ đồng biến trên $\left(2;+\infty\right)$ khi $\left[g\left(f(x)\right)\right]^{'}\ge 0$, $\forall x>2$\\
		$\Rightarrow  f(x)\ge 0$, $\forall x>2 \Leftrightarrow x^3+4x+m\ge 0$, $\forall x>3 \Leftrightarrow  m\ge-x^3-4x$, $\forall x>2$\\
		$ \Rightarrow  m\ge\underset{\left[2;+\infty\right)}{\max}\,\left(-x^3-4x\right)=-16$.\\
		Vì $m\in\left[-2020;2020\right]$ và $m\in\mathbb{Z}$ nên có $2037$ giá trị thỏa mãn $m$ .}
\end{ex}

\begin{ex}%[2D1G1-3]%Câu 13
	Cho hàm số $y=f(x)$ có đạo hàm $f'(x)=x{\left(x+1\right)^2}\left(x^2+2mx+1\right)$ với mọi $x \in \mathbb{R}$. Có bao nhiêu số nguyên âm $m$ để hàm số $g(x)=f\left(2x+1\right)$ đồng biến trên khoảng $\left(3;5\right)$?
	\choice
	{\True $3$}
	{$2$}
	{$4$}
	{$6$}
	\loigiai{
		Ta có $g'(x)=2f'(2x+1)=2(2x+1)(2x+2)^2[(2x+1)^2+2m(2x+1)+1]$. 	Đặt $t=2x+1$\\
		Để hàm số $g(x)$ đồng biến trên khoảng $\left(3;5\right)$ khi và chỉ khi 
		\begin{eqnarray*}
			& & g'(x)\ge 0,\forall x\in\left(3;5\right) \\
			& \Leftrightarrow & t(t^2+2mt+1)\ge 0,\forall t\in\left(7;11\right)\Leftrightarrow{t^2}+2mt+1\ge 0,\,\,\forall t\in\left(7;11\right) \\
			&\Leftrightarrow & 2m\ge\dfrac{-t^2-1}{t},\,\,\,\forall t\in\left(7;11\right)
		\end{eqnarray*}	
		Xét hàm số $h(t)=\dfrac{-t^2-1}{t}$ trên $\left[7;11\right]$, có $h'(t)=\dfrac{-t^2+1}{t^2}$\\
		Bảng biến thiên
		\begin{center}
			\begin{tikzpicture}
				\tkzTabInit[espcl=3,lgt=1.2,nocadre]
				{$t$/0.7,$h'(t)$/0.7,$h(t)$/2.5}
				{$-\infty$,$1$,$11$,$+\infty$}
				\tkzTabLine{, ,,-,,,}
				%	\node (0) at ($(N12)+(0,-3)$) {$-\infty$};
				\node (1) at ($(N22)+(0,-0.8)$) [right] {$-\dfrac{50}{7}$};
				\node (2) at ($(N32)+(0,-2.5)$) [left] {$-\dfrac{122}{11}$};
				
				
				%				\node (3) at ($(N11+(-0.5,0))$) {};
				%				\node (4) at ($(N23)$) {};
				\fill[pattern=north east lines] (7.0,-0.7) rectangle (10,-4.4);
				\fill[pattern=north east lines] (1.5,-0.7) rectangle (4.5,-4.4);
				\draw[->] (1)--(2);	
				\draw[dashed] (4.5,-0.7)--(4.5,-4.4);
				\draw[dashed] (7.0,-0.7)--(7.0,-4.4);	
			\end{tikzpicture}		
		\end{center}
		Dựa vào BBT ta có $2m\ge\dfrac{-t^2-1}{t},\,\,\,\forall t\in\left(7;11\right)\Leftrightarrow 2m\ge\underset{\left[7;11\right]}{\max}\,h(t)\Leftrightarrow m\ge-\dfrac{50}{14}$\\
		Vì $ m\in{\mathbb{Z}^-}\Rightarrow m \in \{-3;-2;-1\}$ .
	}
\end{ex}

\begin{ex}%[2D1G1-3]%Câu 14
	Cho hàm số $y=f(x)$ có bảng biến thiên như sau\\
	\begin{center}
		\begin{tikzpicture}[>=stealth,scale = 1]
			\tkzTabInit[lgt=1,espcl=2.5,nocadre]
			{$x$ /0.7, $y'$ /0.7,$y$ /2.5}
			{$-\infty$,$0$,$2$,$+\infty$}
			\tkzTabLine{ ,-,0,+,0,-,}
			\tkzTabVar{-/$-\infty$, +/$4$,- /$0$, +/{ $+\infty$}}
		\end{tikzpicture}
	\end{center}
	Có bao nhiêu số nguyên $m<2019$ để hàm số $g(x)=f\left(x^2-2x+m\right)$ đồng biến trên khoảng $\left(1;+\infty\right)$?
	\choice
	{\True $2016$}
	{$2015$}
	{$2017$}
	{$2018$}
	\loigiai{
		Ta có $g'(x)=\left(x^2-2x+m\right)'{f}'\left(x^2-2x+m\right)=2\left(x-1\right){f}'\left(x^2-2x+m\right)$ .\\
		Hàm số $y=g(x)$ đồng biến trên khoảng $\left(1;+\infty\right)$ khi và chỉ khi $g'(x)\ge 0,\forall x\in\left(1;+\infty\right)$ và\\
		$g'(x)=0$ tại hữu hạn điểm \\
		$\Leftrightarrow 2\left(x-1\right){f}'\left(x^2-2x+m\right)\ge 0,\forall x\in\left(1;+\infty\right)$\\
		$\Leftrightarrow{f}'\left(x^2-2x+m\right)\ge 0,\forall x\in\left(1;+\infty\right)$ $\Leftrightarrow\hoac{
			&{x^2}-2x+m\ge 2,\forall x\in\left(1;+\infty\right)\\ 
			&{x^2}-2x+m\le 0,\forall x\in\left(1;+\infty\right).}$\\
		Xét hàm số $y=x^2-2x+m$, ta có bảng biến thiên
		\begin{center}
			\begin{tikzpicture}[>=stealth,scale = 1]
				\tkzTabInit[lgt=1,espcl=2.5,nocadre]
				{$x$ /0.7, $y'$ /0.7,$y$ /2.5}
				{$-\infty$,$2$,$+\infty$}
				\tkzTabLine{ ,-,0,+,}
				\tkzTabVar{+/$+\infty$, -/$m-1$, +/{$+\infty$}}
			\end{tikzpicture}
		\end{center}
		Dựa vào bảng biến thiên ta có\\
		TH1: $x^2-2x+m\ge 2,\forall x\in\left(1;+\infty\right)\Leftrightarrow m-1\ge 2\Leftrightarrow m\ge 3$ .\\
		TH2: $x^2-2x+m\le 0,\forall x\in\left(1;+\infty\right)$. Không có giá trị $m$ thỏa mãn.\\
		Vậy có $2016$ số nguyên $m<2019$ thỏa mãn yêu cầu bài toán.}
\end{ex}

\begin{ex}%[2D1G1-3]%Câu 15
	\immini{
		Cho hàm số $ y=f(x)$ có đạo hàm là hàm số $f'(x)$ trên $\mathbb{R}$. Biết rằng hàm số $ y=f'\left(x-2\right)+2$ có đồ thị như hình vẽ bên dưới. Hàm số $ f(x)$ đồng biến trên khoảng nào?
		\choice
		{$\left(-\infty ;3\right),\,\,\left(5;+\infty\right)$}
		{\True $\left(-\infty ;-1\right),\,\,\left(1;+\infty\right)$}
		{$\left(-1;1\right)$}
		{$\left(3;5\right)$}}{
		\begin{tikzpicture}[scale=0.7,font=\footnotesize, line join=round, line cap=round, >=stealth] %Đường cong bậc 3
			\draw[thick, ->] (-0.5,0)--(3.5,0);
			\draw[thick, ->] (0,-1.8)--(0,5.3);
			\draw (3.7,0) node[below] {$x$};
			\draw (0,5.4) node[left]{$y$};
			\draw (0,0) node[below left]{$0$};
			\draw[fill] (3,0) circle (0.5pt)node[below]{$ 3$};
			\draw[fill] (1,0) circle (0.5pt)node[below]{$ 1$};
			\draw[fill] (2,0) circle (0.5pt)node[above]{$2$};
			\draw[fill] (0,2) circle (0.5pt)node[left]{$ 2$};
			\draw[fill] (0,-1) circle (0.5pt)node[left]{$ -1$};
			%		\draw[fill] (0,2) circle (0.5pt)node[above left]{$ 2$};
			%		\draw[fill] (0,-2) circle (0.5pt)node[below left]{$ -2$};
			\draw[dashed] (3,0)--(3,2)--(0,2)--(1,2)--(1,0); 
			\draw[dashed](0,-1)--(2,-1)--(2,0);
			\draw[line width=1.2pt,smooth,samples=100,domain=0.6:3.4] plot(\x,{3*(\x)^2-12*(\x)+11});		
			%\draw[line width=1.2pt,smooth,samples=100,domain=-3.3:2.8] plot(\x,{0.75*(\x)^2+0.5*\x-1});
			%	\draw (2.0,2.8) node[left]{$y=f'(x)$};
	\end{tikzpicture}	}
	\loigiai{	
		Hàm số $ y=f'\left(x-2\right)+2$ có đồ thị $(C)$ như sau:\\
		\begin{center}
			\begin{tikzpicture}[scale=0.7,font=\footnotesize, line join=round, line cap=round, >=stealth] %Đường cong bậc 3
				\draw[thick, ->] (-0.5,0)--(3.5,0);
				\draw[thick, ->] (0,-1.8)--(0,5.3);
				\draw (3.7,0) node[below] {$x$};
				\draw (0,5.4) node[left]{$y$};
				\draw (0,0) node[below left]{$0$};
				\draw[fill] (3,0) circle (0.5pt)node[below]{$ 3$};
				\draw[fill] (1,0) circle (0.5pt)node[below]{$ 1$};
				\draw[fill] (2,0) circle (0.5pt)node[above]{$2$};
				\draw[fill] (0,2) circle (0.5pt)node[left]{$ 2$};
				\draw[fill] (0,-1) circle (0.5pt)node[left]{$ -1$};
				%		\draw[fill] (0,2) circle (0.5pt)node[above left]{$ 2$};
				%		\draw[fill] (0,-2) circle (0.5pt)node[below left]{$ -2$};
				\draw[dashed] (3,0)--(3,2)--(0,2)--(1,2)--(1,0); 
				\draw[dashed](0,-1)--(2,-1)--(2,0);
				\draw[line width=1.2pt,smooth,samples=100,domain=0.6:3.4] plot(\x,{3*(\x)^2-12*(\x)+11});		
				%\draw[line width=1.2pt,smooth,samples=100,domain=-3.3:2.8] plot(\x,{0.75*(\x)^2+0.5*\x-1});
				%	\draw (2.0,2.8) node[left]{$y=f'(x)$};
			\end{tikzpicture}
		\end{center}
		Dựa vào đồ thị $(C)$ ta có\\ $f'\left(x-2\right)+2>2,\forall x\in\left(-\infty ;1\right)\cup\left(3;+\infty\right)\Leftrightarrow{f}'\left(x-2\right)>0,\forall x\in\left(-\infty ;1\right)\cup\left(3;+\infty\right)$ .\\
		Đặt $ x*=x-2$ suy ra $f'\left(x*\right)>0,\forall x*\in\left(-\infty ;-1\right)\bigcup\left(1;+\infty\right)$.\\
		Vậy hàm số $ f(x)$ đồng biến trên khoảng $\left(-\infty ;-1\right),\,\,\left(1;+\infty\right)$.}
\end{ex}

\begin{ex}%[2D1G1-2]%Câu 16
	\immini{
		Cho hàm số $ y=f(x)$ có đạo hàm là hàm số $f'(x)$ trên $\mathbb{R}$. Biết rằng hàm số $ y=f'\left(x+2\right)-2$ có đồ thị như hình vẽ bên dưới. Hàm số $ f(x)$ nghịch biến trên khoảng nào?
		\choice
		{$\left(-3;-1\right),\,\,\left(1;3\right)$}
		{\True $\left(-1;1\right),\,\,\left(3;5\right)$}
		{$\left(-\infty ;-2\right),\,\,\left(0;2\right)$}
		{$\left(-5;-3\right),\,\,\left(-1;1\right)$}}{
		\begin{tikzpicture}[scale=0.7,font=\footnotesize, line join=round, line cap=round, >=stealth] %Đường cong bậc 3
			\draw[thick, ->] (-3.8,0)--(4.0,0);
			\draw[thick, ->] (0,-4.8)--(0,3.5);
			\draw (4.2,0) node[below] {$x$};
			\draw (0,3.7) node[left]{$y$};
			\draw (0,0) node[below left]{$0$};
			\draw[fill] (-3,0) circle (0.5pt)node[above]{$ -3$};
			\draw[fill] (-1,0) circle (0.5pt)node[above]{$ -1$};
			\draw[fill] (1,0) circle (0.5pt)node[above]{$ 1$};
			\draw[fill] (3,0) circle (0.5pt)node[above]{$3$};
			\draw[fill] (0,2) circle (0.5pt)node[above left]{$ 2$};
			\draw[fill] (0,-1) circle (0.5pt)node[above right]{$ -1$};
			%		\draw[fill] (0,2) circle (0.5pt)node[above left]{$ 2$};
			%		\draw[fill] (0,-2) circle (0.5pt)node[below left]{$ -2$};
			\draw[dashed] (-3,0)--(-3,-2)--(3,-2)--(3,0) (-1,0)--(-1,-2) (1,0)--(1,-2) (-3.494,0)--(-3.494,2)--(3.494,2)--(3.494,0); 
			\draw[line width=1.2pt,smooth,samples=100,domain=-3.6:3.6] plot(\x,{0.11*(\x)^4-1.11*(\x)^2-1});		
			%\draw[line width=1.2pt,smooth,samples=100,domain=-3.3:2.8] plot(\x,{0.75*(\x)^2+0.5*\x-1});
			%	\draw (2.0,2.8) node[left]{$y=f'(x)$};
	\end{tikzpicture}	}
	\loigiai{
		Hàm số $ y=f'\left(x+2\right)-2$ có đồ thị $(C)$ như sau
		\begin{center}
			\begin{tikzpicture}[scale=0.7,font=\footnotesize, line join=round, line cap=round, >=stealth] %Đường cong bậc 3
				\draw[thick, ->] (-3.8,0)--(4.0,0);
				\draw[thick, ->] (0,-4.8)--(0,3.5);
				\draw (4.2,0) node[below] {$x$};
				\draw (0,3.7) node[left]{$y$};
				\draw (0,0) node[below left]{$0$};
				\draw[fill] (-3,0) circle (0.5pt)node[above]{$ -3$};
				\draw[fill] (-1,0) circle (0.5pt)node[above]{$ -1$};
				\draw[fill] (1,0) circle (0.5pt)node[above]{$ 1$};
				\draw[fill] (3,0) circle (0.5pt)node[above]{$3$};
				\draw[fill] (0,2) circle (0.5pt)node[above left]{$ 2$};
				\draw[fill] (0,-1) circle (0.5pt)node[above right]{$ -1$};
				%		\draw[fill] (0,2) circle (0.5pt)node[above left]{$ 2$};
				%		\draw[fill] (0,-2) circle (0.5pt)node[below left]{$ -2$};
				\draw[dashed] (-3,0)--(-3,-2)--(3,-2)--(3,0) (-1,0)--(-1,-2) (1,0)--(1,-2) (-3.494,0)--(-3.494,2)--(3.494,2)--(3.494,0); 
				\draw[line width=1.2pt,smooth,samples=100,domain=-3.6:3.6] plot(\x,{0.11*(\x)^4-1.11*(\x)^2-1});		
				%\draw[line width=1.2pt,smooth,samples=100,domain=-3.3:2.8] plot(\x,{0.75*(\x)^2+0.5*\x-1});
				%	\draw (2.0,2.8) node[left]{$y=f'(x)$};
			\end{tikzpicture}
		\end{center}
		Dựa vào đồ thị $(C)$ ta có\\
		$f'\left(x+2\right)-2<-2,\forall x\in\left(-3;-1\right)\bigcup\left(1;3\right)\Leftrightarrow{f}'\left(x+2\right)<0,\forall x\in\left(-3;-1\right)\bigcup\left(1;3\right)$.\\
		Đặt $ x^*=x+2$ suy ra: $f'\left(x^*\right)<0,\forall x^*\in\left(-1;1\right)\bigcup\left(3;5\right)$.\\
		Vậy: Hàm số $ f(x)$ đồng biến trên khoảng $\left(-1;1\right),\,\,\left(3;5\right)$.}
\end{ex}

\begin{ex}%[2D1G1-2]%Câu 17
	\immini{
		Cho hàm số $ y=f(x)$ có đạo hàm là hàm số $f'(x)$ trên $\mathbb{R}$. Biết rằng hàm số $ y=f'\left(x-2\right)+2$ có đồ thị như hình vẽ bên dưới. Hàm số $ f(x)$ nghịch biến trên khoảng nào?
		\choice
		{$\left(-\infty ;2\right)$}
		{\True $\left(-1;1\right)$}
		{$\left(\dfrac{3}{2};\dfrac{5}{2}\right)$}
		{$\left(2;+\infty\right)$}}{
		\begin{tikzpicture}[scale=0.7,font=\footnotesize, line join=round, line cap=round, >=stealth] %Đường cong bậc 3
			\draw[thick, ->] (-0.5,0)--(3.5,0);
			\draw[thick, ->] (0,-1.8)--(0,5.3);
			\draw (3.7,0) node[below] {$x$};
			\draw (0,5.4) node[left]{$y$};
			\draw (0,0) node[below left]{$0$};
			\draw[fill] (3,0) circle (0.5pt)node[below]{$ 3$};
			\draw[fill] (1,0) circle (0.5pt)node[below]{$ 1$};
			\draw[fill] (2,0) circle (0.5pt)node[above]{$2$};
			\draw[fill] (0,2) circle (0.5pt)node[left]{$ 2$};
			\draw[fill] (0,-1) circle (0.5pt)node[left]{$ -1$};
			%		\draw[fill] (0,2) circle (0.5pt)node[above left]{$ 2$};
			%		\draw[fill] (0,-2) circle (0.5pt)node[below left]{$ -2$};
			\draw[dashed] (3,0)--(3,2)--(0,2)--(1,2)--(1,0); 
			\draw[dashed](0,-1)--(2,-1)--(2,0);
			\draw[line width=1.2pt,smooth,samples=100,domain=0.6:3.4] plot(\x,{3*(\x)^2-12*(\x)+11});		
			%\draw[line width=1.2pt,smooth,samples=100,domain=-3.3:2.8] plot(\x,{0.75*(\x)^2+0.5*\x-1});
			%	\draw (2.0,2.8) node[left]{$y=f'(x)$};
	\end{tikzpicture}	}
	\loigiai{
		Hàm số $ y=f'\left(x-2\right)+2$ có đồ thị $(C)$ như sau
		\begin{center}
			\begin{tikzpicture}[scale=0.7,font=\footnotesize, line join=round, line cap=round, >=stealth] %Đường cong bậc 3
				\draw[thick, ->] (-0.5,0)--(3.5,0);
				\draw[thick, ->] (0,-1.8)--(0,5.3);
				\draw (3.7,0) node[below] {$x$};
				\draw (0,5.4) node[left]{$y$};
				\draw (0,0) node[below left]{$0$};
				\draw[fill] (3,0) circle (0.5pt)node[below]{$ 3$};
				\draw[fill] (1,0) circle (0.5pt)node[below]{$ 1$};
				\draw[fill] (2,0) circle (0.5pt)node[above]{$2$};
				\draw[fill] (0,2) circle (0.5pt)node[left]{$ 2$};
				\draw[fill] (0,-1) circle (0.5pt)node[left]{$ -1$};
				%		\draw[fill] (0,2) circle (0.5pt)node[above left]{$ 2$};
				%		\draw[fill] (0,-2) circle (0.5pt)node[below left]{$ -2$};
				\draw[dashed] (3,0)--(3,2)--(0,2)--(1,2)--(1,0); 
				\draw[dashed](0,-1)--(2,-1)--(2,0);
				\draw[line width=1.2pt,smooth,samples=100,domain=0.6:3.4] plot(\x,{3*(\x)^2-12*(\x)+11});		
				%\draw[line width=1.2pt,smooth,samples=100,domain=-3.3:2.8] plot(\x,{0.75*(\x)^2+0.5*\x-1});
				%	\draw (2.0,2.8) node[left]{$y=f'(x)$};
			\end{tikzpicture}
		\end{center}
		Dựa vào đồ thị $(C)$ ta có\\
		$f'\left(x-2\right)+2<2,\forall x\in\left(1;3\right)\Leftrightarrow{f}'\left(x-2\right)<0,\forall x\in\left(1;3\right)$.\\
		Đặt $ x^*=x-2$ thì $f'\left(x^*\right)<0,\forall x^*\in\left(-1;1\right)$.\\
		Vậy: Hàm số $ f(x)$ nghịch biến trên khoảng $\left(-1;1\right)$.\\
		Cách khác:\\
		Tịnh tiến sang trái hai đơn vị và xuống dưới $2$ đơn vị thì từ đồ thị $(C)$ sẽ thành đồ thị của hàm$ y=f'(x)$. Khi đó $f'(x)<0,\forall x\in\left(-1;1\right)$.\\
		Vậy hàm số $ f(x)$ nghịch biến trên khoảng $\left(-1;1\right)$.}
\end{ex}

\begin{ex}%[2D1G1-2]%Câu 18
	Cho hàm số $y=f(x)$ có đạo hàm cấp $ 3$ liên tục trên $\mathbb{R}$ và thỏa mãn $f(x)\cdot f'''(x)=x{\left(x-1\right)^2}{\left(x+4\right)^3}$ với mọi $x\in\mathbb{R}$ và $g(x)=\left[f'(x)\right]^2-2f(x)\cdot f''(x)$. Hàm số $h(x)=g\left(x^2-2x\right)$ đồng biến trên khoảng nào dưới đây?
	\choice
	{$\left(-\infty ;1\right)$}
	{$\left(2;+\infty\right)$}
	{$\left(0;1\right)$}
	{\True $\left(1;2\right)$}
	\loigiai{		
		Ta có $g'(x)=2f''(x){f}'(x)-2f'(x)\cdot f''(x)-2f(x)\cdot f'''(x)=-2f(x)\cdot f'''(x);$\\
		Khi đó $\left(h(x)\right)'=\left(2x-2\right){g}'\left(x^2-2x\right)=-2\left(2x-2\right)\left(x^2-2x\right){\left(x^2-2x-1\right)^2}{\left(x^2-2x+4\right)^3}$\\
		$h'(x)=0\Leftrightarrow\hoac{
			& x=0\\ 
			& x=1\\ 
			& x=2\\ 
			& x=1\pm\sqrt{2}.}$ 
		Ta có bảng xét dấu của $h'(x)$
		\begin{center}
			\begin{tikzpicture}
				\tkzTabInit[lgt=1.2,espcl=2,nocadre]
				{$t$/0.7, $h'(x)$ /.7} % first column
				{$-\infty$, $1-\sqrt{2}$,$0$, $1$,$2$,$1+\sqrt{2}$, $+\infty$} % first row
				\tkzTabLine { ,+,0,-,0,+,0,-,0,+,0,- } % second row
				%				\tkzTabLine {,-,z,+,t,+,} % third row
				%				\tkzTabLine {,+,d,-,z,+,} % last row
			\end{tikzpicture}
		\end{center}
		Suy ra hàm số $h(x)=g\left(x^2-2x\right)$ đồng biến trên khoảng $\left(1;2\right)$.}
\end{ex}

\begin{ex}%[2D1G1-2]%Câu 19
	Cho hàm số $ y=f(x)$ xác định trên $\mathbb{R}$. Hàm số $ y=g(x)=f'\left(2x+3\right)+2$ có đồ thị là một parabol với tọa độ đỉnh $ I\left(2;-1\right)$ và đi qua điểm $ A\left(1;2\right)$. Hỏi hàm số $ y=f(x)$ nghịch biến trên khoảng nào dưới đây?
	\choice
	{\True $\left(5;9\right)$}
	{$\left(1;2\right)$}
	{$\left(-\infty ;9\right)$}
	{$\left(1;3\right)$}
	\loigiai{	
		Xét hàm số $ g(x)=f'\left(2x+3\right)+2$ có đồ thị là một Parabol nên có phương trình dạng $ y=g(x)=a{x^2}+bx+c\,\,\,\,(P)$.\\
		Vì $(P)$ có đỉnh $ I\left(2;-1\right)$ nên $\heva{
			&\dfrac{-b}{2a}=2\\ 
			& g(2)=-1} \Leftrightarrow\heva{
			&-b=4a\\ 
			& 4a+2b+c=-1} \Leftrightarrow\heva{
			& 4a+b=0\\ 
			& 4a+2b+c=-1}$.\\
		Vì $(P)$ đi qua điểm $ A\left(1;2\right)$ nên $ g(1)=2\Leftrightarrow a+b+c=2$.\\
		Ta có hệ phương trình $\heva{
			& 4a+b=0\\ 
			& 4a+2b+c=-1\\ 
			& a+b+c=2} \Leftrightarrow\heva{
			& a=3\\ 
			& b=-12\\ 
			& c=11}$ nên $ g(x)=3x^2-12x+11$.\\
		Đồ thị của hàm $ y=g(x)$ là
		\begin{center}
			\begin{tikzpicture}[scale=0.7,font=\footnotesize, line join=round, line cap=round, >=stealth] %Đường cong bậc 3
				\draw[thick, ->] (-0.5,0)--(3.5,0);
				\draw[thick, ->] (0,-1.8)--(0,5.3);
				\draw (3.7,0) node[below] {$x$};
				\draw (0,5.4) node[left]{$y$};
				\draw (0,0) node[below left]{$0$};
				\draw[fill] (3,0) circle (0.5pt)node[below]{$ 3$};
				\draw[fill] (1,0) circle (0.5pt)node[below]{$ 1$};
				\draw[fill] (2,0) circle (0.5pt)node[above]{$2$};
				\draw[fill] (0,2) circle (0.5pt)node[left]{$ 2$};
				\draw[fill] (0,-1) circle (0.5pt)node[left]{$ -1$};
				%		\draw[fill] (0,2) circle (0.5pt)node[above left]{$ 2$};
				%		\draw[fill] (0,-2) circle (0.5pt)node[below left]{$ -2$};
				\draw[dashed] (3,0)--(3,2)--(0,2)--(1,2)--(1,0) (3.2,2)--(3,2); 
				\draw[dashed](0,-1)--(2,-1)--(2,0);
				\draw[line width=1.2pt,smooth,samples=100,domain=0.6:3.4] plot(\x,{3*(\x)^2-12*(\x)+11});		
				%\draw[line width=1.2pt,smooth,samples=100,domain=-3.3:2.8] plot(\x,{0.75*(\x)^2+0.5*\x-1});
				%	\draw (2.0,2.8) node[left]{$y=f'(x)$};
			\end{tikzpicture}	
		\end{center}
		Theo đồ thị ta thấy $ f'(2x+3)\le 0\Leftrightarrow f'(2x+3)+2\le 2\Leftrightarrow 1\le x\le 3$.\\
		Đặt $ t=2x+3\Leftrightarrow x=\dfrac{t-3}{2}$ khi đó $ f'(t)\le 0\Leftrightarrow 1\le\dfrac{t-3}{2}\le 3\Leftrightarrow 5\le t\le 9$.\\
		Vậy $ y=f(x)$ nghịch biến trên khoảng $\left(5;9\right)$.}
\end{ex}

\begin{ex}%[2D1G1-2]%Câu 20
	\immini{
		Cho hàm số $ y=f(x)$, hàm số $f'(x)=x^3+a{x^2}+bx+c\left(a,b,c\in\mathbb{R}\right)$ có đồ thị như hình vẽ bên.
		Hàm số $ g(x)=f\left(f'(x)\right)$ nghịch biến trên khoảng nào dưới đây?
		\choice
		{$\left(1;+\infty\right)$}
		{\True $\left(-\infty ;-2\right)$}
		{$\left(-1;0\right)$}
		{$\left(-\dfrac{\sqrt{3}}{3};\dfrac{\sqrt{3}}{3}\right)$}}{
		\begin{tikzpicture}[scale=0.8,font=\footnotesize, line join=round, line cap=round, >=stealth] %Đường cong bậc 3
			\draw[thick, ->] (-1.7,0)--(1.7,0);
			\draw[thick, ->] (0,-2.7)--(0,3.0);
			\draw (1.9,0) node[below] {$x$};
			\draw (0,3.2) node[left]{$y$};
			\draw (0,0) node[below left]{$0$};
			\draw[fill] (-1,0) circle (0.5pt)node[above left]{$ -1 $};
			\draw[fill] (1,0) circle (0.5pt)node[below right]{$ 1$};
			\draw[line width=1.2pt,smooth,samples=100,domain=-1.3:1.3] plot(\x,{2.667*(\x)^3+0*(\x)^2-2.667*\x});		
			%\draw[line width=1.2pt,smooth,samples=100,domain=-3.3:2.8] plot(\x,{0.75*(\x)^2+0.5*\x-1});
		\end{tikzpicture}	
	}
	\loigiai{	
		Vì các điểm $\left(-1;0\right),\left(0;0\right),\left(1;0\right)$ thuộc đồ thị hàm số $ y=f'(x)$ nên ta có hệ\\
		$\heva{
			&-1+a-b+c=0\\ 
			& c=0\\ 
			& 1+a+b+c=0} \Leftrightarrow\heva{
			& a=0\\ 
			& b=-1\\ 
			& c=0} \Rightarrow {f}'(x)=x^3-x\Rightarrow f''(x)=3x^2-1$.\\
		Ta có $ g(x)=f\left(f'(x)\right)\Rightarrow{g}'(x)=f'\left(f'(x)\right)\cdot f''(x)$.\\
		Xét \\
		$g'(x)=0\Leftrightarrow{g}'(x)=f'\left(f'(x)\right)\cdot f''(x)=0$\\
		$\Leftrightarrow {f}'\left(x^3-x\right)\left(3x^2-1\right)=0\Leftrightarrow\hoac{
			&{x^3}-x=0\\ 
			&{x^3}-x=1\\ 
			&{x^3}-x=-1\\ 
			& 3x^2-1=0} \Leftrightarrow \hoac{
			& x=\pm 1\\ 
			& x=0\\ 
			& x=x_1(x_1\approx 1,325)\\ 
			& x=x_2(x_2\approx-1,325)\\ 
			& x=\pm\dfrac{\sqrt{3}}{3}.}$\\
		Bảng biến thiên
		\begin{center}
			\begin{tikzpicture}
				\tkzTabInit[lgt=1.2,espcl=2,nocadre]
				{$t$/0.7, $h'(x)$ /.7} % first column
				{$-\infty$, $-1{,}325$,$-1$, $-\dfrac{\sqrt{3}}{3}$,$0$,$\dfrac{\sqrt{3}}{3}$,$1$,$1{,}325$, $+\infty$} % first row
				\tkzTabLine { ,-,0,+,0,-,0,+,0,-,0,+,0,-,0,+, } % second row
				%				\tkzTabLine {,-,z,+,t,+,} % third row
				%				\tkzTabLine {,+,d,-,z,+,} % last row
			\end{tikzpicture}
		\end{center}
		Dựa vào bảng biến thiên ta có $ g(x)$ nghịch biến trên $\left(-\infty ;-2\right)$}
\end{ex}
\Closesolutionfile{ans}
\indapan{10}{ans/CD1/Muc_9_10}
\chapter{HÁM SỐ MŨ HÀM SỐ LOGARIT}
\begin{Solution}{1}
C
\end{Solution}
\begin{Solution}{3}
B
\end{Solution}
\begin{Solution}{4}
A
\end{Solution}
\begin{Solution}{5}
A
\end{Solution}
\begin{Solution}{6}
A
\end{Solution}
\begin{Solution}{7}
B
\end{Solution}
\begin{Solution}{8}
A
\end{Solution}
\begin{Solution}{9}
C
\end{Solution}
\begin{Solution}{10}
B
\end{Solution}
\begin{Solution}{11}
C
\end{Solution}
\begin{Solution}{12}
D
\end{Solution}
\begin{Solution}{13}
B
\end{Solution}
\begin{Solution}{14}
D
\end{Solution}
\begin{Solution}{15}
A
\end{Solution}
\begin{Solution}{16}
B
\end{Solution}
\begin{Solution}{17}
C
\end{Solution}
\begin{Solution}{18}
C
\end{Solution}
\begin{Solution}{19}
C
\end{Solution}
\begin{Solution}{20}
B
\end{Solution}
\begin{Solution}{21}
C
\end{Solution}
\begin{Solution}{22}
B
\end{Solution}
\begin{Solution}{23}
D
\end{Solution}
\begin{Solution}{24}
B
\end{Solution}
\begin{Solution}{25}
D
\end{Solution}
\begin{Solution}{26}
D
\end{Solution}
\begin{Solution}{27}
B
\end{Solution}
\begin{Solution}{28}
A
\end{Solution}
\begin{Solution}{29}
C
\end{Solution}
\begin{Solution}{30}
B
\end{Solution}
\begin{Solution}{31}
D
\end{Solution}
\begin{Solution}{32}
B
\end{Solution}
\begin{Solution}{33}
B
\end{Solution}
\begin{Solution}{34}
C
\end{Solution}
\begin{Solution}{35}
D
\end{Solution}
\begin{Solution}{36}
B
\end{Solution}
\begin{Solution}{37}
B
\end{Solution}
\begin{Solution}{38}
A
\end{Solution}
\begin{Solution}{39}
A
\end{Solution}
\begin{Solution}{40}
D
\end{Solution}
\begin{Solution}{41}
C
\end{Solution}
\begin{Solution}{42}
B
\end{Solution}
\begin{Solution}{43}
A
\end{Solution}
\begin{Solution}{44}
A
\end{Solution}
\begin{Solution}{45}
D
\end{Solution}
\begin{Solution}{46}
C
\end{Solution}
\begin{Solution}{47}
A
\end{Solution}
\begin{Solution}{48}
B
\end{Solution}
\begin{Solution}{49}
B
\end{Solution}
\begin{Solution}{50}
B
\end{Solution}
\begin{Solution}{51}
A
\end{Solution}
\begin{Solution}{52}
A
\end{Solution}
\begin{Solution}{53}
C
\end{Solution}
\begin{Solution}{54}
C
\end{Solution}
\begin{Solution}{55}
C
\end{Solution}
\begin{Solution}{56}
B
\end{Solution}
\begin{Solution}{57}
C
\end{Solution}
\begin{Solution}{58}
C
\end{Solution}
\begin{Solution}{59}
B
\end{Solution}
\begin{Solution}{60}
C
\end{Solution}
\begin{Solution}{61}
A
\end{Solution}
\begin{Solution}{62}
B
\end{Solution}
\begin{Solution}{63}
B
\end{Solution}
\begin{Solution}{64}
D
\end{Solution}
\begin{Solution}{65}
D
\end{Solution}
\begin{Solution}{66}
B
\end{Solution}
\begin{Solution}{67}
A
\end{Solution}
\begin{Solution}{68}
D
\end{Solution}

\begin{Solution}{1}
D
\end{Solution}
\begin{Solution}{2}
C
\end{Solution}
\begin{Solution}{3}
C
\end{Solution}
\begin{Solution}{4}
A
\end{Solution}
\begin{Solution}{5}
B
\end{Solution}
\begin{Solution}{6}
D
\end{Solution}
\begin{Solution}{7}
C
\end{Solution}
\begin{Solution}{8}
D
\end{Solution}
\begin{Solution}{9}
A
\end{Solution}
\begin{Solution}{10}
B
\end{Solution}
\begin{Solution}{11}
D
\end{Solution}
\begin{Solution}{12}
A
\end{Solution}
\begin{Solution}{13}
D
\end{Solution}
\begin{Solution}{14}
B
\end{Solution}
\begin{Solution}{15}
B
\end{Solution}
\begin{Solution}{16}
C
\end{Solution}
\begin{Solution}{1}
A
\end{Solution}
\begin{Solution}{2}
B
\end{Solution}
\begin{Solution}{3}
D
\end{Solution}
\begin{Solution}{4}
D
\end{Solution}
\begin{Solution}{5}
C
\end{Solution}
\begin{Solution}{6}
A
\end{Solution}
\begin{Solution}{7}
D
\end{Solution}
\begin{Solution}{8}
B
\end{Solution}
\begin{Solution}{9}
C
\end{Solution}
\begin{Solution}{10}
C
\end{Solution}
\begin{Solution}{1}
D
\end{Solution}
\begin{Solution}{2}
D
\end{Solution}
\begin{Solution}{3}
B
\end{Solution}
\begin{Solution}{4}
C
\end{Solution}
\begin{Solution}{5}
D
\end{Solution}
\begin{Solution}{6}
A
\end{Solution}
\begin{Solution}{7}
C
\end{Solution}
\begin{Solution}{8}
B
\end{Solution}
\begin{Solution}{9}
A
\end{Solution}
\begin{Solution}{10}
C
\end{Solution}
\begin{Solution}{11}
D
\end{Solution}
\begin{Solution}{12}
C
\end{Solution}
\begin{Solution}{13}
A
\end{Solution}
\begin{Solution}{14}
D
\end{Solution}
\begin{Solution}{15}
A
\end{Solution}
\begin{Solution}{16}
A
\end{Solution}
\begin{Solution}{17}
B
\end{Solution}
\begin{Solution}{18}
C
\end{Solution}
\begin{Solution}{19}
C
\end{Solution}
\begin{Solution}{20}
A
\end{Solution}
\begin{Solution}{21}
D
\end{Solution}
\begin{Solution}{22}
C
\end{Solution}
\begin{Solution}{23}
A
\end{Solution}
\begin{Solution}{24}
C
\end{Solution}
\begin{Solution}{25}
A
\end{Solution}
\begin{Solution}{26}
B
\end{Solution}
\begin{Solution}{27}
B
\end{Solution}
\begin{Solution}{28}
D
\end{Solution}
\begin{Solution}{29}
B
\end{Solution}
\begin{Solution}{30}
D
\end{Solution}
\begin{Solution}{31}
D
\end{Solution}
\begin{Solution}{32}
C
\end{Solution}
\begin{Solution}{33}
D
\end{Solution}
\begin{Solution}{34}
C
\end{Solution}
\begin{Solution}{35}
D
\end{Solution}
\begin{Solution}{36}
D
\end{Solution}
\begin{Solution}{37}
D
\end{Solution}
\begin{Solution}{38}
D
\end{Solution}
\begin{Solution}{39}
D
\end{Solution}
\begin{Solution}{40}
C
\end{Solution}
\begin{Solution}{41}
A
\end{Solution}
\begin{Solution}{1}
A
\end{Solution}
\begin{Solution}{2}
B
\end{Solution}
\begin{Solution}{3}
C
\end{Solution}
\begin{Solution}{4}
A
\end{Solution}
\begin{Solution}{5}
A
\end{Solution}
\begin{Solution}{6}
C
\end{Solution}
\begin{Solution}{7}
C
\end{Solution}
\begin{Solution}{8}
B
\end{Solution}
\begin{Solution}{9}
C
\end{Solution}
\begin{Solution}{10}
B
\end{Solution}
\begin{Solution}{11}
A
\end{Solution}
\begin{Solution}{12}
B
\end{Solution}
\begin{Solution}{13}
B
\end{Solution}
\begin{Solution}{14}
B
\end{Solution}
\begin{Solution}{15}
A
\end{Solution}
\begin{Solution}{16}
B
\end{Solution}
\begin{Solution}{17}
A
\end{Solution}
\begin{Solution}{18}
D
\end{Solution}
\begin{Solution}{19}
C
\end{Solution}
\begin{Solution}{20}
C
\end{Solution}
\begin{Solution}{21}
A
\end{Solution}
\begin{Solution}{22}
C
\end{Solution}
\begin{Solution}{23}
C
\end{Solution}
\begin{Solution}{24}
A
\end{Solution}
\begin{Solution}{25}
B
\end{Solution}
\begin{Solution}{26}
B
\end{Solution}
\begin{Solution}{27}
A
\end{Solution}
\begin{Solution}{28}
A
\end{Solution}
\begin{Solution}{29}
C
\end{Solution}
\begin{Solution}{30}
B
\end{Solution}
\begin{Solution}{31}
A
\end{Solution}
\begin{Solution}{32}
C
\end{Solution}
\begin{Solution}{33}
B
\end{Solution}
\begin{Solution}{34}
A
\end{Solution}
\begin{Solution}{35}
B
\end{Solution}
\begin{Solution}{36}
B
\end{Solution}
\begin{Solution}{37}
B
\end{Solution}
\begin{Solution}{38}
D
\end{Solution}
\begin{Solution}{39}
B
\end{Solution}
\begin{Solution}{40}
A
\end{Solution}
\begin{Solution}{41}
D
\end{Solution}
\begin{Solution}{42}
D
\end{Solution}
\begin{Solution}{43}
A
\end{Solution}
\begin{Solution}{44}
D
\end{Solution}
\begin{Solution}{45}
C
\end{Solution}
\begin{Solution}{46}
B
\end{Solution}
\begin{Solution}{47}
A
\end{Solution}
\begin{Solution}{48}
D
\end{Solution}
\begin{Solution}{49}
B
\end{Solution}
\begin{Solution}{50}
B
\end{Solution}
\begin{Solution}{51}
D
\end{Solution}
\begin{Solution}{52}
C
\end{Solution}
\begin{Solution}{53}
C
\end{Solution}
\begin{Solution}{54}
B
\end{Solution}
\begin{Solution}{55}
D
\end{Solution}
\begin{Solution}{56}
B
\end{Solution}
\begin{Solution}{57}
C
\end{Solution}
\begin{Solution}{58}
A
\end{Solution}
\begin{Solution}{59}
A
\end{Solution}
\begin{Solution}{60}
B
\end{Solution}
\begin{Solution}{61}
D
\end{Solution}
\begin{Solution}{62}
D
\end{Solution}
\begin{Solution}{63}
B
\end{Solution}
\begin{Solution}{64}
A
\end{Solution}
\begin{Solution}{65}
D
\end{Solution}
\begin{Solution}{66}
C
\end{Solution}
\begin{Solution}{67}
A
\end{Solution}
\begin{Solution}{68}
A
\end{Solution}
\begin{Solution}{69}
D
\end{Solution}
\begin{Solution}{70}
C
\end{Solution}
\begin{Solution}{71}
B
\end{Solution}
\begin{Solution}{72}
A
\end{Solution}
\begin{Solution}{73}
C
\end{Solution}
\begin{Solution}{74}
C
\end{Solution}
\begin{Solution}{75}
C
\end{Solution}
\begin{Solution}{76}
A
\end{Solution}
\begin{Solution}{77}
C
\end{Solution}
\begin{Solution}{78}
B
\end{Solution}
\begin{Solution}{79}
D
\end{Solution}
\begin{Solution}{80}
B
\end{Solution}

\section{Mức 9,10 điểm}
\setcounter{ex}{0}
\setcounter{dang}{0}
\Opensolutionfile{ans}[ans/CD1/Muc_9_10]
\begin{dang}{Tìm m để hàm số đơn điệu trên các khoảng xác định của nó}
	Đang thiếu bài thầy Jf Câu 1 đến 26 
\end{dang}
\begin{dang}
	{Tìm khoảng đơn điệu của hàm số $g(x) = f\left[ u(x)\right] +v(x)$ khi biết đồ thị hoặc bảng biến thiên của hàm số $y = f'(x)$}
\end{dang}
\begin{ex}[Đề tham khảo 2019]%[2D1K1-2]
	Cho hàm số $f(x)$ có bảng xét dấu của đạo hàm như sau
	\begin{center}
		\begin{tikzpicture}
			\tkzTabInit[nocadre,lgt=1.2,espcl=2,deltacl=0.6]
			{$x$ /0.6,$f'(x)$ /0.6}
			{$-\infty$,$1$,$2$,$3$,$4$,$+\infty$}
			\tkzTabLine{,-,$0$,+,$0$,+,$0$,-,$0$,+,}
		\end{tikzpicture}
	\end{center}
	Hàm số $y=3 f(x+2)-x^3+3 x$ đồng biến trên khoảng nào dưới đây?
	\choice
	{$(-\infty ;-1)$}
	{\True $(-1 ; 0)$}
	{$(0 ; 2)$}
	{$(1 ;+\infty)$}
	\loigiai{
		Ta có $y'=3\left[f'(x+2)-\left(x^2-3\right)\right]$.\\
		Với $x \in(-1 ; 0) \Rightarrow x+2 \in(1 ; 2) \Rightarrow f'(x+2)>0$, lại có $x^2-3<0 \Rightarrow y'>0 ;~ \forall x \in(-1 ; 0)$.\\
		Vậy hàm số $y=3 f(x+2)-x^3+3 x$ đồng biến trên khoảng $(-1 ; 0)$.\\
		Chú ý:\\
		+) Ta xét $x \in(1 ; 2) \subset(1 ;+\infty)
		\Rightarrow x+2 \in(3 ; 4)\\
		\Rightarrow f'(x+2)<0 ;~ x^2-3>0$\\
		Suy ra hàm số nghịch biến trên khoảng $(1 ; 2)$ nên loại hai phương án$(0 ; 2)$ và $(1 ;+\infty)$.\\
		+) Tương tự ta xét
		$x \in(-\infty ;-2) \Rightarrow x+2 \in(-\infty ; 0)\\
		\Rightarrow f'(x+2)<0 ; x^2-3>0 \Rightarrow y'<0 ; ~ \forall x \in(-\infty ;-2)$.\\
		Suy ra hàm số nghịch biến trên khoảng $(-\infty ;-2)$ nên loại$(-\infty ;-1)$.\\
		Vậy hàm số đã cho đồng biến trên khoảng $(-1 ; 0)$.
	}
\end{ex}
\begin{ex}[Đề Tham Khảo 2020 - Lần 1]%[2D1G1-2]
	\immini{
		Cho hàm số $f(x)$. Hàm số $y=f'(x)$ có đồ thị như hình bên. Hàm số $g(x)=f(1-2 x)+x^2-x$ nghịch biến trên khoảng nào dưới đây?
		\choice
		{\True $\left(1 ; \dfrac{3}{2}\right)$}
		{$\left(0 ; \dfrac{1}{2}\right)$}
		{$(-2 ;-1)$}
		{$(2 ; 3)$}
	}
	{
		\begin{tikzpicture}[scale=0.7,>=stealth, font=\footnotesize, line join=round, line cap=round]
			%\def\a{1} \def\b{-6} \def\c{9} \def\d{1} % Hệ số
			\def\xmin{-4} \def\xmax{6}
			\def\ymin{-3} \def\ymax{2} 
			%\draw[color=gray!50,dashed] (\xmin,\ymin) grid (\xmax,\ymax); 
			\draw[->] (\xmin,0)--(\xmax,0) node [below]{$x$};
			\draw[->] (0,\ymin)--(0,\ymax) node [left]{$y$};
			\node at (0,0) [below left]{$O$};
			%\node at (1,3) [below left]{$f'(x)$};
			%\node at (-1.3,4) {$f'(x)$};
			\draw[dashed] (-2,0) node[below]{$-2$}--(-2,1)--(0,1) node[below left]{$1$};
			\draw[dashed] (4,0) node[below left]{$4$}--(4,-2)--(0,-2) node[below left]{$-2$};
			%\draw[dashed] (1,0) node[below]{$1$}--(1,1);
			%\draw[dashed] (-0.5,0) node[below left]{$-0{,}5$}--(-0.5,2.125);
			\clip (\xmin+0.1,\ymin+0.1) rectangle (\xmax-0.5,\ymax-0.1);
			\draw[smooth,samples=300][domain=-4:5.5] plot(\x,{0.071*(\x)^3-0.142*(\x)^2-1.07*(\x)});
		\end{tikzpicture}
	}
	
	\loigiai{
		Ta có : $g(x)=f(1-2 x)+x^2-x \Rightarrow g'(x)=-2 f'(1-2 x)+2 x-1$.\\
		\immini{
			Đặt $t=1-2 x \Rightarrow g'(x)=-2 f'(t)-t$.\\
			$g'(x)=0 \Rightarrow f'(t)=-\dfrac{t}{2}$.\\
			Vẽ đường thẳng $y=-\dfrac{x}{2}$ và đồ thị hàm số $f'(x)$ trên cùng một hệ trục
		}	
		{
			\begin{tikzpicture}[scale=0.7,>=stealth, font=\footnotesize, line join=round, line cap=round]
				%\def\a{1} \def\b{-6} \def\c{9} \def\d{1} % Hệ số
				\def\xmin{-4} \def\xmax{6}
				\def\ymin{-3} \def\ymax{2} 
				%	\draw[color=gray!50,dashed] (\xmin,\ymin) grid (\xmax,\ymax); 
				\draw[->] (\xmin,0)--(\xmax,0) node [below]{$x$};
				\draw[->] (0,\ymin)--(0,\ymax) node [left]{$y$};
				\node at (0,0) [below left]{$O$};
				%\node at (1,3) [below left]{$f'(x)$};
				%\node at (-1.3,4) {$f'(x)$};
				\draw[dashed] (-2,0) node[below]{$-2$}--(-2,1)--(0,1) node[below left]{$1$};
				\draw[dashed] (4,0) node[below]{$4$}--(4,-2)--(0,-2) node[below left]{$-2$};
				%\draw[dashed] (1,0) node[below]{$1$}--(1,1);
				%\draw[dashed] (-0.5,0) node[below left]{$-0{,}5$}--(-0.5,2.125);
				\clip (\xmin+0.1,\ymin+0.1) rectangle (\xmax-0.5,\ymax-0.1);
				\draw[smooth,samples=300][domain=-4:5.5] plot(\x,{0.071*(\x)^3-0.142*(\x)^2-1.07*(\x)});
				\draw[smooth,samples=300][domain=-4:5.5] plot(\x,{(-0.5*(\x)});
			\end{tikzpicture}
		}	Hàm số $g(x)$ nghịch biến $\Rightarrow g'(x) \leq 0 \Rightarrow f'(t) \geq-\dfrac{t}{2}\Rightarrow\hoac{&-2 \leq t \leq 0 \\&t \geq 4.}$\\
		Như vậy $f'(1-2 x) \geq \dfrac{1-2 x}{-2}\Rightarrow\hoac{&-2 \leq 1-2 x \leq 0 \\ &4 \leq 1-2 x}\Rightarrow\hoac{&\dfrac{1}{2}\leq x \leq \dfrac{3}{2}\\ &x \leq-\dfrac{3}{2}.}$\\
		Vậy hàm số $g(x)=f(1-2 x)+x^2-x$ nghịch biến trên các khoảng $\left(\dfrac{1}{2}; \dfrac{3}{2}\right)$ và $\left(-\infty ;-\dfrac{3}{2}\right)$.\\
		Mà $\left(1 ; \dfrac{3}{2}\right) \subset \left(\dfrac{1}{2}; \dfrac{3}{2}\right)$ nên hàm số $g(x)=f(1-2 x)+x^2-x$ nghịch biến trên khoảng $\left(1 ; \dfrac{3}{2}\right)$.
	}
\end{ex}
\begin{ex}[Chuyên Lê Quý Đôn Điện Biên 2019]%[2D1G1-2]
	Cho hàm số $f(x)$ có bảng xét dấu của đạo hàm như sau
	\begin{center}
		\begin{tikzpicture}
			\tkzTabInit[nocadre,lgt=1.2,espcl=2,deltacl=0.6]
			{$x$ /0.6,$f'(x)$ /0.6}
			{$-\infty$,$0$,$1$,$2$,$3$,$+\infty$}
			\tkzTabLine{,+,$0$,-,$0$,-,$0$,+,$0$,-,}
		\end{tikzpicture}
	\end{center}
	Hàm số $y=f(x-1)+x^3-12 x+2019$ nghịch biến trên khoảng nào dưới đây?
	\choice
	{$(1 ;+\infty)$}
	{\True $(1 ; 2)$}
	{$(-\infty ; 1)$}
	{$(3 ; 4)$}
	\loigiai{
		$y'=f'(x-1)+3 x^2-12=f'(t)+3 t^2+6 t-9=f'(t)-\left(-3 t^2-6 t+9\right)$, với $t=x-1$.\\
		\immini{
			Nghiệm của phương trình $y'=0$ là hoành độ giao điểm của các đồ thị hàm số $y=f'(t)$ và $y=-3 t^2-6 t+9$.\\
			Vẽ đồ thị hàm số $y=f'(t)$ và $y=-3 t^2-6 t+9$ trên cùng một hệ trục tọa độ như hình vẽ bên.
		}	
		{		\begin{tikzpicture}[scale=0.5,>=stealth, font=\footnotesize, line join=round, line cap=round]
				\def\a{-3} \def\b{-6} \def\c{9} % Hệ số
				\def\xmin{-9} \def\xmax{7}
				\def\ymin{-3} \def\ymax{13}
				
				%\draw[color=gray!50,dashed] (\xmin,\ymin) grid (\xmax,\ymax);
				
				\draw[->] (\xmin,0)--(\xmax,0) node [below]{$x$};
				\draw[->] (0,\ymin)--(0,\ymax) node [left]{$y$};
				\node at (0,0) [below left]{$O$};
				\clip (\xmin+0.1,\ymin+0.1) rectangle (\xmax-0.5,\ymax-0.1);
				\draw[smooth,samples=300] plot(\x,{\a*(\x)^2+\b*(\x)+\c});
				\node at (1,0) [above right]{$1$};
				\node at (2,0) [below right]{$2$};
				\node at (3,0) [below right]{$3$};
				\node at (-3,-2) [left]{$y=-3t^2-6t+9$};
				\node at (4,0) [below right]{$f'(x)$};
				\draw (-2.2,10).. controls (-1,1.9) and (-0.5,0.8) .. (0,0);
				%\draw (-2,0).. controls (-1.5,-2) and (-0.5,-0) .. (0,0);
				\draw (0,0).. controls (0.4,-0.6) and (0.6,-0.6) .. (0.8,-0.2);
				\draw (0.8,-0.2).. controls (1,0.25) and (1.1,-0.1) .. (1.4,-0.8);
				\draw (1.4,-0.8).. controls (1.6,-1.1) and (1.7,-0.9) .. (2,0);
				\draw (2,0).. controls (2.4,1.1) and (2.6,1.1) .. (3.5,-1);
			\end{tikzpicture}
		}
		Dựa vào đồ thị trên, ta có bảng xét dấu của hàm số $y'=f'(t)-\left(-3 t^2-6 t+9\right)$ như sau $
		\left(t_0<-1\right)$
		\begin{center}
			\begin{tikzpicture}
				\tkzTabInit[nocadre,lgt=2,espcl=2,deltacl=0.6]
				{$x$ /0.6,$y'$ /0.6}
				{$-\infty$,$t_0$,$1$,$+\infty$}
				\tkzTabLine{,+,$0$,-,$0$,+,}
			\end{tikzpicture}
		\end{center}
		Hàm số nghịch biến trên khoảng $t \in\left(t_0 ; 1\right)$.\\
		Do đó hàm số nghịch biến trên khoảng $x \in(1 ; 2) \subset \left(t_0+1 ; 1\right)$.
	}
\end{ex}


\begin{ex}[Chuyên Phan Bội Châu Nghệ An 2019]%[2D1G1-2]
	Cho hàm số $f(x)$ có bảng xét dấu đạo hàm như sau:
	\begin{center}
		\begin{tikzpicture}
			\tkzTabInit[nocadre,lgt=2,espcl=2,deltacl=0.6]
			{$x$ /0.6,$f'(x)$ /0.6}
			{$-\infty$,$1$,$2$,$3$,$4$,$+\infty$}
			\tkzTabLine{,-,$0$,+,$0$,+,$0$,-,$0$,+,}
		\end{tikzpicture}
	\end{center}
	Hàm số $y=2 f(1-x)+\sqrt{x^2+1}-x$ nghịch biến trên những khoảng nào dưới đây
	\choice
	{$(-\infty ;-2)$}
	{$(-\infty ; 1)$}
	{\True $(-2 ; 0)$}
	{$(-3 ;-2)$}
	\loigiai{
		$y'=-2 f'(1-x)+\dfrac{x}{\sqrt{x^2+1}}-1$. \\
		Có $\dfrac{x}{\sqrt{x^2+1}}-1<0,~ \forall x \in(-2 ; 0)$.\\
		Bảng xét dấu:
		\begin{center}
			\begin{tikzpicture}
				\tkzTabInit[nocadre,lgt=2,espcl=2,deltacl=0.6]
				{$x$ /0.7,$f'(1-x)$ /0.7}
				{$-\infty$,$-3$,$-2$,$-1$,$0$,$+\infty$}
				\tkzTabLine{,+,$0$,-,$0$,+,$0$,+,$0$,-,}
			\end{tikzpicture}
		\end{center}
		$\Rightarrow-2 f'(1-x)<0, ~ \forall x \in(-2 ; 0) \\
		\Rightarrow-2 f'(1-x)+\dfrac{x}{\sqrt{x^2+1}}-1<0, ~\forall x \in(-2 ; 0)$.
	}
\end{ex}
\begin{ex}[Sở Vĩnh Phúc 2019]%[2D1G1-2]
	\immini{
		Cho hàm số bậc bốn $y=f(x)$ có đồ thị của hàm số $y=f'(x)$ như hình vẽ bên.\\
		Hàm số $y=3 f(x)+x^3-6 x^2+9 x$ đồng biến trên khoảng nào trong các khoảng sau đây?
		\choice
		{$(0 ; 2)$}
		{$(-1 ; 1)$}
		{$(1 ;+\infty)$}
		{\True $(-2 ; 0)$}
	}
	{
		\begin{tikzpicture}[scale=0.7,>=stealth, font=\footnotesize, line join=round, line cap=round]
			\def\a{0.21} \def\b{0.88} \def\c{-0.58} \def\d{-3} % Hệ số
			\def\xmin{-5} \def\xmax{5}
			\def\ymin{-4} \def\ymax{3} 
			%\draw[color=gray!50,dashed] (\xmin,\ymin) grid (\xmax,\ymax); 
			\draw[->] (\xmin,0)--(\xmax,0) node [below]{$x$};
			\draw[->] (0,\ymin)--(0,\ymax) node [left]{$y$};
			\node at (0,0) [above left]{$O$};
			\node at (-4,0) [below left]{$-4$};
			\node at (-2,0) [below left]{$-2$};
			\node at (0,-3) [below right]{$-3$};
			\draw[dashed] (2,0) node[above right]{$2$}--(2,1) --(0,1) node[above right]{$1$};
			\clip (\xmin+0.1,\ymin+0.1) rectangle (\xmax-0.5,\ymax-0.1);
			\draw[smooth,samples=300] plot(\x,{\a*(\x)^3+\b*(\x)^2+\c*(\x)+\d});
		\end{tikzpicture}
	}
	
	\loigiai{
		Hàm số $f(x)=a x^4+b x^3+c x^2+d x+e,(a \neq 0)$.
		Có $f'(x)=4 a x^3+3 b x^2+2 c x+d$.\\
		Đồ thị hàm số $y=f'(x)$ đi qua các điểm $(-4 ; 0),(-2 ; 0),(0 ;-3),(2 ; 1)$ nên ta có
		$$\heva{&- 2 5 6 a + 4 8 b - 8 c + d = 0\\
			&- 3 2 a + 1 2 b - 4 c + d = 0\\
			&d = - 3\\
			&3 2 a + 1 2 b + 4 c + d = 1}\Leftrightarrow \heva{&
			a=\dfrac{5}{96}\\
			&b=\dfrac{7}{24}\\
			&c=-\dfrac{7}{24}\\
			&d=-3.}
		$$
		Xét hàm số
		$
		y=3 f(x)+x^3-6 x^2+9 x$\\
		Ta có $ y'=3\left(f'(x)+x^2-4 x+3\right)=3\left(\frac{5}{24}x^3+\frac{15}{8}x^2-\frac{55}{12}x\right)
		$\\
		Ta có $y'=0 \Leftrightarrow\hoac{&x=-11 \\&x=0 \\&x=2.}$ \\
		Xét dấu $y'$, ta được hàm số đã cho đồng biến trên các khoảng $(-11 ; 0)$ và $(2 ;+\infty)$.
	}
\end{ex}
\begin{ex}[Học Mãi 2019]%[2D1K1-2]
	\immini
	{Cho hàm số $y=f(x)$ có đạo hàm trên $\mathbb{R}$. Đồ thị hàm số $y=f'(x)$ như hình bên. Hỏi đồ thị hàm số $y=f(x)-2 x$ có bao nhiêu điểm cực trị?
		\choice
		{$4$}
		{\True $3$}
		{$2$}
		{$1$}
	}
	{
		\begin{tikzpicture}[font=\footnotesize,line join=round, line cap=round,>=stealth,scale=0.8]
			\draw[->] (-3.5,0)--(4,0) node[above] {$x$};
			\draw[->] (0,-3)--(0,4) node[left] {$y$};
			%\fill[black] (-2,0)node[below left]{$-2$} circle (1.2pt) (0,0)node[above right]{$O$} circle (1.2pt) (3,0)node[above]{$3$} circle (1.2pt);
			\draw[dashed] (-2,-2)-- (0,-2) node[right]{$-2$};
			\draw[dashed] (2,0) node[below]{$2$}-- (2,2)--(0,2) node[below left]{$2$};
			\node at (0,0) [below left]{$O$};
			\node at (3,0) [below right]{$3$};
			\draw (-3,2.5).. controls (-2.2,-3) and (-1.8,-3) .. (-1.1,0);
			\draw (-1.1,0).. controls (-0.6,2.5) and (-0.4,2.5) .. (0,2);
			\draw (0,2).. controls (0.7,0.5) and (1.1,0.5) .. (1.5,1.5);
			\draw (1.5,1.5).. controls (2,2.5) and (2.8,2.5) .. (3.5,-2.5);
			%\draw (3,0).. controls (3.3,-0.1) and (3.5,-0.5) .. (3.5,-2);
		\end{tikzpicture}
	}
	\loigiai{
		\immini{
			Đặt $g(x)=f(x)-2 x$.\\
			$\Rightarrow g'(x)=f'(x)-2 .
			$\\
			Vẽ đường thẳng $y=2$.\\
			$\Rightarrow$ phương trình $g'(x)=0$ có $3$ nghiệm bội lẻ.\\
			$\Rightarrow$ đồ thị hàm số $y=f(x)-2 x$ có $3$ điểm cực trị.
		}
		{
			\begin{tikzpicture}[font=\footnotesize,line join=round, line cap=round,>=stealth,scale=0.8]
				\draw[->] (-3.5,0)--(4,0) node[above] {$x$};
				\draw[->] (0,-3)--(0,4) node[left] {$y$};
				%\fill[black] (-2,0)node[below left]{$-2$} circle (1.2pt) (0,0)node[above right]{$O$} circle (1.2pt) (3,0)node[above]{$3$} circle (1.2pt);
				\draw[dashed] (-2,-2)-- (0,-2) node[right]{$-2$};
				\draw[dashed] (2,0) node[below]{$2$}-- (2,2)--(0,2) node[below left]{$2$};
				\node at (3,0) [below left]{$3$};
				\draw (-3,2.5).. controls (-2.2,-3) and (-1.8,-3) .. (-1.1,0);
				\draw (-1.1,0).. controls (-0.6,2.5) and (-0.4,2.5) .. (0,2);
				\draw (0,2).. controls (0.7,0.5) and (1.1,0.5) .. (1.5,1.5);
				\draw (1.5,1.5).. controls (2,2.5) and (2.8,2.5) .. (3.5,-2.5);
				\draw (-3.5,2)--(4,2) node[above]{$y=2$};
			\end{tikzpicture}
		}
	}
\end{ex}
\begin{ex}[THPT Hoàng Hoa Thám Hưng Yên 2019]%[2D1G1-2]
	\immini{
		Cho hàm số $y=f(x)$ liên tục trên $\mathbb{R}$. Hàm số $y=f'(x)$ có đồ thị như hình vẽ. 
		Hàm số $g(x)=f(x-1)+\dfrac{2019-2018 x}{2018}$ đồng biến trên khoảng nào dưới đây?
		\choice
		{$(2 ; 3)$}
		{$(0 ; 1)$}
		{\True $(-1 ; 0)$}
		{$(1 ; 2)$}
	}
	{
		\begin{tikzpicture}[scale=1, font=\footnotesize, line join=round, line cap=round, >=stealth]
			\tikzset{label style/.style={font=\footnotesize}}
			\draw[->] (-2,0)--(3,0) node[below left] {$x$};
			\draw[->] (0,-2)--(0,3) node[below left] {$y$};
			\draw[fill=black] (0,0) node [above left] {$O$} circle(1pt);
			\fill (1,1) circle(1pt) (-1,1) circle(1pt) (2,1) circle(1pt);
			\foreach \x in {1,2}
			\draw[thin] (\x,1pt)--(\x,-1pt) node [below] {\footnotesize$\x$};
			\foreach \x in {-1}
			\draw[thin] (\x,1pt)--(\x,-1pt) node [below left] {\footnotesize$\x$};
			\foreach \y in {-1}
			\draw[thin] (1pt,\y)--(-1pt,\y) node [right] {\footnotesize$\y$};
			\foreach \y in {1}
			\draw[thin] (1pt,\y)--(-1pt,\y) node [above left] {\footnotesize$\y$};
			\draw[dashed](-1,0)--(-1,1)--(2,1) (1,1)--(1,0) (2,1)--(2,0);
			\begin{scope}
				\clip (-3,-3) rectangle (3,3);
				\draw[name path=(C)] plot[smooth,tension=0.7] coordinates{(-1.15,3)(-0.5,-1.6)(.8,.88)(1.9,0.8)(2.3,3)};
			\end{scope}
		\end{tikzpicture}
	}	\loigiai{
		Ta có $g'(x)=f'(x-1)-1$.\\
		$
		g'(x) \geq 0 \Leftrightarrow f'(x-1)-1 \geq 0 \Leftrightarrow f'(x-1) \geq 1 \Leftrightarrow \hoac{&x - 1 \leq - 1\\
			&x - 1 \geq 2}\Leftrightarrow \hoac{&
			x \leq 0 \\
			&x \geq 3.}
		$\\
		Từ đó suy ra hàm số $g(x)=f(x-1)+\dfrac{2019-2018 x}{2018}$ đồng biến trên khoảng $(-1 ; 0)$.
	}
\end{ex}

\begin{ex}[(Sở Ninh Bình 2019]%[2D1K1-2]
	Cho hàm số $y=f(x)$ có bảng xét dấu của đạo hàm như sau
	\begin{center}
		\begin{tikzpicture}
			\tkzTabInit[nocadre,lgt=1,espcl=2,deltacl=0.6]
			{$x$ /0.7,$f'(x)$ /0.7}
			{$-\infty$,$-2$,$-1$,$2$,$4$,$+\infty$}
			\tkzTabLine{,+,$0$,-,$0$,+,$0$,-,$0$,+,}
		\end{tikzpicture}
	\end{center}
	Hàm số $y=-2 f(x)+2019$ nghịch biến trên khoảng nào trong các khoảng dưới đây?
	\choice
	{$(-4 ; 2)$}
	{\True $(-1 ; 2)$}
	{$(-2 ;-1)$}
	{$(2 ; 4)$}
	\loigiai{
		Xét $y=g(x)=-2 f(x)+2019$.\\
		Ta có $g'(x)=(-2 f(x)+2019)'=-2 f'(x), g'(x)=0 \Leftrightarrow\hoac{&x=-2 \\&x=-1 \\&x=2 \\&x=4.}$.\\
		Ta có bảng xét dấu của $g'(x)$
		\begin{center}
			\begin{tikzpicture}
				\tkzTabInit[nocadre,lgt=1,espcl=2,deltacl=0.6]
				{$x$ /0.6,$f'(x)$ /0.6}
				{$-\infty$,$-2$,$-1$,$2$,$4$,$+\infty$}
				\tkzTabLine{,-,$0$,+,$0$,-,$0$,+,$0$,+,}
			\end{tikzpicture}
		\end{center}
		Dựa vào bảng xét dấu, ta thấy hàm số $y=g(x)$ nghịch biến trên khoảng $(-1 ; 2)$.
	}
\end{ex}
\begin{ex}[THPT Lương Thế Vinh Hà Nội 2019]%[2D1G1-2]
	\immini{
		Cho hàm số $y=f(x)$. Biết đồ thị hàm số $y=f'(x)$ có đồ thị như hình vẽ bên. 
		Hàm số $y=f \left(3-x^2\right)+2018$ đồng biến trên khoảng nào dưới đây?
		\choice
		{\True $(-1 ; 0)$}
		{$(2 ; 3)$}
		{$(-2 ;-1)$}
		{$(0 ; 1)$}
	}
	{
		\begin{tikzpicture}[scale=0.6,>=stealth, font=\footnotesize, line join=round, line cap=round]
			\def\a{0.065} \def\b{0.32} \def\c{-0.53} \def\d{-0.82} % Hệ số
			\def\xmin{-8} \def\xmax{4}
			\def\ymin{-3} \def\ymax{3} 
			%\draw[color=gray!50,dashed] (\xmin,\ymin) grid (\xmax,\ymax); 
			\draw[->] (\xmin,0)--(\xmax,0) node [below]{$x$};
			\draw[->] (0,\ymin)--(0,\ymax) node [left]{$y$};
			\node at (0,0) [below left]{$O$};
			\node at (-6,0) [below left]{$-6$};
			\node at (-1,0) [below left]{$-1$};
			\node at (2,0) [below right]{$2$};
			\clip (\xmin+0.1,\ymin+0.1) rectangle (\xmax-0.5,\ymax-0.1);
			\draw[smooth,samples=300][domain=-6.5:3.5] plot(\x,{\a*(\x)^3+\b*(\x)^2+\c*(\x)+\d});
		\end{tikzpicture}
	}
	
	\loigiai{
		Ta có $\left[f\left( 3-x^2\right)+2018 \right]'=-2 x \cdot f'\left(3-x^2\right) $.\\
		$
		-2 x \cdot f'\left(3-x^2\right)=0 \Leftrightarrow\hoac{&
			x = 0\\
			&3 - x ^{2}= - 6\\
			&3 - x ^{2}= - 1\\
			&3 - x ^{2}= 2}
		\Leftrightarrow \hoac{
			&x=0 \\
			&x=\pm 3 \\
			&x=\pm 2 \\
			&	x=\pm 1.}
		$\\
		Bảng xét dấu của đạo hàm hàm số đã cho
		\begin{center}
			\begin{center}
				\begin{tikzpicture}
					\tkzTabInit[nocadre,lgt=2.9,espcl=1.5,deltacl=0.6]
					{$x$ /0.7,$f'\left( 3-x^2\right) $/0.7,$-2xf'\left( 3-x^2\right)$/0.8}
					{$-\infty$,$-3$,$-2$,$-1$,$0$,$1$,$2$,$3$,$+\infty$}
					\tkzTabLine{,-,$0$,+,$0$,-,$0$,+,$0$,+,$0$,-,$0$,+,$0$,-}
					\tkzTabLine{,-,$0$,+,$0$,-,$0$,+,$0$,-,$0$,+,$0$,-,$0$,+}
				\end{tikzpicture}
			\end{center}
		\end{center}
		Từ bảng xét dấu suy ra hàm số đồng biến trên $(-1 ; 0)$.
	}
\end{ex}
\begin{ex}[Chuyên Biên Hòa - Hà Nam - 2020]%[2D1G1-2]
	\immini{
		Cho hàm số đa thức $f(x)$ có đạo hàm trên $\mathbb{R}$. Biết $f(0)=0$ và đồ thị hàm số $y=f'(x)$ như hình sau.
		Hàm số $g(x)=\left|4 f(x)+x^2\right|$ đồng biến trên khoảng nào dưới đây?
		\choice
		{$(4 ;+\infty)$}
		{\True $(0 ; 4)$}
		{$(-\infty ;-2)$}
		{$(-2 ; 0)$}
	}	
	{
		\begin{tikzpicture}[scale=0.7,>=stealth, font=\footnotesize, line join=round, line cap=round]
			%\def\a{1} \def\b{-6} \def\c{9} \def\d{1} % Hệ số
			\def\xmin{-4} \def\xmax{6}
			\def\ymin{-3} \def\ymax{2} 
			%\draw[color=gray!50,dashed] (\xmin,\ymin) grid (\xmax,\ymax); 
			\draw[->] (\xmin,0)--(\xmax,0) node [below]{$x$};
			\draw[->] (0,\ymin)--(0,\ymax) node [left]{$y$};
			\node at (0,0) [below left]{$O$};
			%\node at (1,3) [below left]{$f'(x)$};
			%\node at (-1.3,4) {$f'(x)$};
			\draw[dashed] (-2,0) node[below]{$-2$}--(-2,1)--(0,1) node[below left]{$1$};
			\draw[dashed] (4,0) node[below]{$4$}--(4,-2)--(0,-2) node[below left]{$-2$};
			%\draw[dashed] (1,0) node[below]{$1$}--(1,1);
			%\draw[dashed] (-0.5,0) node[below left]{$-0{,}5$}--(-0.5,2.125);
			\clip (\xmin+0.1,\ymin+0.1) rectangle (\xmax-0.5,\ymax-0.1);
			\draw[smooth,samples=300][domain=-4:5.5] plot(\x,{0.071*(\x)^3-0.142*(\x)^2-1.07*(\x)});
		\end{tikzpicture}
	}
	\loigiai{
		\immini{
			Xét hàm số $h(x)=4 f(x)+x^2$ trên $\mathbb{R}$.\\
			Vì $f(x)$ là hàm số đa thức nên $h(x)$ cũng là hàm số đa thức và $h(0)=4 f(0)=0$.\\
			Ta có $h'(x)=4 f'(x)+2 x$. Do đó $h'(x)=0 \Leftrightarrow f'(x)=-\dfrac{1}{2}x$.\\
		}
		{
			\begin{tikzpicture}[scale=0.7,>=stealth, font=\footnotesize, line join=round, line cap=round]
				%\def\a{1} \def\b{-6} \def\c{9} \def\d{1} % Hệ số
				\def\xmin{-4} \def\xmax{6}
				\def\ymin{-3} \def\ymax{2} 
				%\draw[color=gray!50,dashed] (\xmin,\ymin) grid (\xmax,\ymax); 
				\draw[->] (\xmin,0)--(\xmax,0) node [below]{$x$};
				\draw[->] (0,\ymin)--(0,\ymax) node [left]{$y$};
				\node at (0,0) [below left]{$O$};
				%\node at (1,3) [below left]{$f'(x)$};
				%\node at (-1.3,4) {$f'(x)$};
				\draw[dashed] (-2,0) node[below]{$-2$}--(-2,1)--(0,1) node[below left]{$1$};
				\draw[dashed] (4,0) node[below]{$4$}--(4,-2)--(0,-2) node[below left]{$-2$};
				%\draw[dashed] (1,0) node[below]{$1$}--(1,1);
				%\draw[dashed] (-0.5,0) node[below left]{$-0{,}5$}--(-0.5,2.125);
				\clip (\xmin+0.1,\ymin+0.1) rectangle (\xmax-0.5,\ymax-0.1);
				\draw[smooth,samples=300][domain=-4:5.5] plot(\x,{0.071*(\x)^3-0.142*(\x)^2-1.07*(\x)});
				\draw[smooth,samples=300][domain=-4:5.5] plot(\x,{-0.5*(\x)});
			\end{tikzpicture}
		}
		Dựa vào sự tương giao của đồ thị hàm số $y=f'(x)$ và đường thẳng $y=-\dfrac{1}{2}x$, ta có
		$
		h'(x)=0 \Leftrightarrow x \in\{-2 ; 0 ; 4\}.\\
		$
		Bảng biến thiên của hàm số $h(x)$ như sau:
		\begin{center}
			\begin{tikzpicture}
				\tkzTabInit[nocadre,lgt=1.2,espcl=2.5,deltacl=0.6]
				{$x$ /0.6,$y'$ /0.6,$y$ /2}
				{$-\infty$,$-2$,$0$,$4$,$+\infty$}
				\tkzTabLine{,-,$0$,+,$0$,-,$0$,+,}
				\tkzTabVar{+/$+\infty$, -/$y_1$,+/$0$,-/$y_3$,+/$+\infty$}
			\end{tikzpicture}
		\end{center}
		Từ đó suy ra bảng biến thiên của hàm số $g(x)=|h(x)|$.\\
		Dựa vào bảng biến thiên trên, ta thấy hàm số $g(x)$ đồng biến trên khoảng $(0 ; 4)$.
	}
\end{ex}
\begin{ex}[Chuyên Thái Bình - 2020]%[2D1G1-2]
	\immini{
		Cho hàm số $f(x)$ liên tục trên $\mathbb{R}$ có đồ thị hàm số $y=f'(x)$ cho như hình vẽ bên.\\
		Hàm số $g(x)=2 f(|x-1|)-x^2+2 x+2020$ đồng biến trên khoảng nào?
		\choice
		{\True $(0 ; 1)$}
		{$(-3 ; 1)$}
		{$(1 ; 3)$}
		{$(-2 ; 0)$}
	}
	{
		\begin{tikzpicture}[scale=0.7,>=stealth, font=\footnotesize, line join=round, line cap=round]
			%\def\a{1} \def\b{-6} \def\c{9} \def\d{1} % Hệ số
			\def\xmin{-4} \def\xmax{5}
			\def\ymin{-3} \def\ymax{5} 
			%\draw[color=gray!50,dashed] (\xmin,\ymin) grid (\xmax,\ymax); 
			\draw[->] (\xmin,0)--(\xmax,0) node [below]{$x$};
			\draw[->] (0,\ymin)--(0,\ymax) node [left]{$y$};
			\node at (0,0) [below left]{$O$};
			%\node at (1,3) [below left]{$f'(x)$};
			\node at (-1.3,4) {$f'(x)$};
			\draw[dashed] (-1,0) node[above]{$-1$}--(-1,-1)--(0,-1) node[below left]{$-1$};
			\draw[dashed] (1,0) node[below]{$1$}--(1,1)--(0,1) node[below left]{$1$};
			\draw[dashed] (3,0) node[below]{$3$}--(3,3)--(0,3) node[below left]{$3$};
			%\draw[dashed] (1,0) node[below]{$1$}--(1,1);
			%\draw[dashed] (-0.5,0) node[below left]{$-0{,}5$}--(-0.5,2.125);
			\clip (\xmin+0.1,\ymin+0.1) rectangle (\xmax-0.5,\ymax-0.1);
			\draw[smooth,samples=300][domain=-2:4] plot(\x,{-0.5*(\x)^3+1.5*(\x)^2+1.5*(\x)-1.5});
			%\draw[smooth,samples=300] plot(\x,{(\x)^3+(\x)^2-2*(\x)+1});
		\end{tikzpicture}
	}
	\loigiai{
		Ta có đường thẳng $y=x$ cắt đồ thị hàm số $y=f'(x)$ tại các điểm $x=-1 ; x=1 ; x=3$ như hình vẽ sau:
		\begin{center}
			\begin{tikzpicture}[scale=0.7,>=stealth, font=\footnotesize, line join=round, line cap=round]
				%\def\a{1} \def\b{-6} \def\c{9} \def\d{1} % Hệ số
				\def\xmin{-4} \def\xmax{5}
				\def\ymin{-3} \def\ymax{5} 
				%\draw[color=gray!50,dashed] (\xmin,\ymin) grid (\xmax,\ymax); 
				\draw[->] (\xmin,0)--(\xmax,0) node [below]{$x$};
				\draw[->] (0,\ymin)--(0,\ymax) node [left]{$y$};
				\node at (0,0) [below left]{$O$};
				%\node at (1,3) [below left]{$f'(x)$};
				\node at (-1.3,4) {$f'(x)$};
				\node at (4,3.2) {$y=x$};
				\draw[dashed] (-1,0) node[above]{$-1$}--(-1,-1)--(0,-1) node[below left]{$-1$};
				\draw[dashed] (1,0) node[below]{$1$}--(1,1)--(0,1) node[below left]{$1$};
				\draw[dashed] (3,0) node[below]{$3$}--(3,3)--(0,3) node[below left]{$3$};
				%\draw[dashed] (1,0) node[below]{$1$}--(1,1);
				%\draw[dashed] (-0.5,0) node[below left]{$-0{,}5$}--(-0.5,2.125);
				\clip (\xmin+0.1,\ymin+0.1) rectangle (\xmax-0.5,\ymax-0.1);
				\draw[smooth,samples=300][domain=-2:4] plot(\x,{-0.5*(\x)^3+1.5*(\x)^2+1.5*(\x)-1.5});
				\draw[smooth,samples=300] plot(\x,{(\x)});
			\end{tikzpicture}
		\end{center}
		Dựa vào đồ thị của hai hàm số trên ta có $f'(x)>x \Leftrightarrow\hoac{&x<-1 \\ &1<x<3}$ và
		$ f'(x)<x \Leftrightarrow\hoac{&
			-1<x<1 \\
			&x>3.}$\\
		+Trường hợp 1: $x-1<0 \Leftrightarrow x<1$.\\
		Khi đó $g(x)=2 f(1-x)-x^2+2 x+2020$.\\
		Ta có $g'(x)=-2 f'(1-x)+2(1-x)$.
		$$
		g'(x)>0 \Leftrightarrow-2 f'(1-x)+2(1-x)>0 \Leftrightarrow f'(1-x)<1-x \Leftrightarrow\hoac{
			&- 1 < 1 - x < 1\\
			&1 - x > 3} \Leftrightarrow \hoac{&
			0<x<2 \\
			&x<-2.}
		$$
		Kết hợp điều kiện, ta có $g'(x)>0 \Leftrightarrow\hoac{&0<x<1 \\ &x<-2.}$\\
		
		+ Trường hợp 2: $x-1>0 \Leftrightarrow x>1$.\\
		Khi đó ta có $g(x)=2 f(x-1)-x^2+2 x+2020$.\\
		$ g'(x)=2 f'(x-1)-2(x-1)$\\
		$g'(x)>0 \Leftrightarrow 2 f'(x-1)-2(x-1)>0 \Leftrightarrow f'(x-1)>x-1 \Leftrightarrow\hoac{&
			x - 1 < - 1\\
			&1 < x - 1 < 3}\Leftrightarrow \hoac{
			&x<0 \\
			&2<x<4.}$
		Kết hợp điều kiện ta có $g'(x)>0 \Leftrightarrow 2<x<4$.\\
		Vậy hàm số $g(x)=2 f(|x-1|)-x^2+2 x+2020$ đồng biến trên khoảng $(0 ; 1)$.
	}
\end{ex}

\begin{ex}[Chuyên Lào Cai - 2020]%[2D1G1-2]
	\immini{
		Cho hàm số $f'(x)$ có đồ thị như hình bên.\\
		Hàm số $g(x)=f(3 x+1)+9 x^3+\dfrac{9}{2}x^2$ đồng biến trên khoảng nào dưới đây?
		\choice
		{$(-1 ; 1)$}
		{$(-2 ; 0)$}
		{$(-\infty ; 0)$}
		{\True $(1 ;+\infty)$}
	}
	{\begin{tikzpicture}[line join=round, line cap=round,>=stealth,thick,scale=.8]
			\tikzset{label style/.style={font=\footnotesize}}
			\draw[->] (-2.1,0)--(5.1,0) node[below left] {$x$};
			\draw[->] (0,-3.1)--(0,4.1) node[below left] {$y$};
			\draw (0,0) node [below left] {$O$};
			\foreach \x in {1,2,3}
			\draw[thin] (\x,1pt)--(\x,-1pt) node [below] {$\x$};
			\draw[thin](-1,1pt)--(1,-1pt)node[above left]{$-1$};
			\foreach \y in {-2,2}
			\draw[thin] (1pt,\y)--(-1pt,\y) node [above right] {$\y$};
			%\begin{scope}
			\clip (-2,-3) rectangle (5,4);
			\draw[samples=200,domain=-2:4,smooth,variable=\x] plot (\x,{(\x)^3-3*(\x)^2+2});
			%\end{scope}
			\draw[dashed] (-1,0)--(-1,-2)--(0,-2);
			\draw[dashed] (3,0)--(3,2)--(0,2);
			%\begin{scope}[on background layer]\path[white]node{MDD-134};\end{scope}
		\end{tikzpicture}
	}
	\loigiai
	{
		\immini{Xét hàm số $g(x)=f(3 x+1)+9 x^3+\dfrac{9}{2}x^2 \\
			\Rightarrow g'(x)=3 f'(3 x+1)+27 x^2+9 x$.\\
			Hàm số đồng biến  $\Leftrightarrow g'(x)>0 \Leftrightarrow 3 f'(3 x+1)+27 x^2+9 x>0$
			\\
			$
			\Leftrightarrow f'(3 x+1)+3 x(3 x+1)>0 \qquad (*)
			$\\
			Đặt $t=3 x+1$, khi đó  $(*) \Leftrightarrow f'(t)+(t-1) t>0$\\ $\Leftrightarrow f'(t)>-t^2+t$.\\
			Vẽ parabol $y=-x^2+x$ và đồ thị hàm số $f'(x)$ trên cùng một hệ trục
		}
		{
			\begin{tikzpicture}[line join=round, line cap=round,>=stealth,thick,scale=.8]
				\tikzset{label style/.style={font=\footnotesize}}
				\draw[->] (-2.1,0)--(5.1,0) node[below left] {$x$};
				\draw[->] (0,-3.1)--(0,4.1) node[below left] {$y$};
				\draw (0,0) node [below left] {$O$};
				\foreach \x in {1,2,3}
				\draw[thin] (\x,1pt)--(\x,-1pt) node [below] {$\x$};
				\draw[thin](-1,1pt)--(1,-1pt);
				\foreach \y in {-2,2}
				\draw[thin] (1pt,\y)--(-1pt,\y) node [above right] {$\y$};
				%\begin{scope}
				\clip (-2,-3) rectangle (5,4);
				\draw[samples=200,domain=-2:4,smooth,variable=\x] plot (\x,{(\x)^3-3*(\x)^2+2});
				\draw[samples=200,domain=-2:4,smooth,variable=\x] plot (\x,{-(\x)^2+(\x)});
				%\end{scope}
				\draw[dashed] (-1,0) node[above left]{$-1$}--(-1,-2)--(0,-2);
				\draw[dashed] (3,0)--(3,2)--(0,2);
				%\begin{scope}[on background layer]\path[white]node{MDD-134};\end{scope}
			\end{tikzpicture}
		}
		Dựa vào đồ thị ta thấy
		$
		f'(t)>-t^2+t \Leftrightarrow\hoac{&- 1 < t < 1\\
			&t > 2}\Rightarrow \hoac{&
			- 1 < 3 x + 1 < 1\\
			&3 x + 1 > 2} \Leftrightarrow \hoac{&
			\dfrac{-2}{3}<x<0\\
			&x>\dfrac{1}{3}.}
		$}
\end{ex}
\begin{ex}[Sở Phú Thọ-2020]%[2D1G1-2]
	\immini{
		Cho hàm số $y=f(x)$ có đồ thị hàm số $y=f'(x)$ như hình vẽ.\\
		Hàm số $g(x)=f\left(\mathrm{e}^x-2\right)-2020$ nghịch biến trên khoảng nào dưới đây?
		\choice
		{\True $\left(-1 ; \dfrac{3}{2}\right)$}
		{$(-1 ; 2)$}
		{$(0 ;+\infty)$}
		{$\left(\dfrac{3}{2}; 2\right)$}
	}
	{
		\begin{tikzpicture}[scale=0.7,>=stealth, font=\footnotesize, line join=round, line cap=round]
			\def\a{1} \def\b{-3} \def\c{0} \def\d{0} % Hệ số
			\def\xmin{-2} \def\xmax{4}
			\def\ymin{-5} \def\ymax{2} 
			%\draw[color=gray!50,dashed] (\xmin,\ymin) grid (\xmax,\ymax); 
			\draw[->] (\xmin,0)--(\xmax,0) node [below]{$x$};
			\draw[->] (0,\ymin)--(0,\ymax) node [left]{$y$};
			\node at (0,0) [above left]{$O$};
			\node at (3,0) [below right]{$3$};
			\draw[dashed] (2,0) node[above]{$2$}--(2,-4) --(0,-4) node[left]{$-4$};
			\clip (\xmin+0.1,\ymin+0.1) rectangle (\xmax-0.5,\ymax-0.1);
			\draw[smooth,samples=300] plot(\x,{\a*(\x)^3+\b*(\x)^2+\c*(\x)+\d});
		\end{tikzpicture}
	}
	
	\loigiai{
		Dựa vào đồ thị hàm số $y=f'(x)$ suy ra $f'(x) \leq 0 ~ \forall x<3$ và $f'(x)>0 ~ \forall x>3$.
		$
		g'(x)=\mathrm{e}^x f'\left(\mathrm{e}^x-2\right) .
		$
		Hàm số $g(x)=f\left(\mathrm{e}^x-2\right)-2020$ nghịch biến \\ $ \Leftrightarrow g'(x)<0 \Leftrightarrow \mathrm{e}^x f'\left(\mathrm{e}^x-2\right)<0$\\
		$
		\Leftrightarrow f'\left(\mathrm{e}^x-2\right)<0 \Leftrightarrow \mathrm{e}^x-2<3 \Leftrightarrow \mathrm{e}^x<5 \Leftrightarrow x<\ln 5 .
		$\\
		Vậy hàm số đã cho nghịch biến trên $\left(-1 ; \dfrac{3}{2}\right)$.
	}
\end{ex}
\begin{ex}[Lý Nhân Tông - Bắc Ninh - 2020]%[2D1G1-2]
	\immini{
		Cho hàm số $f(x)$ có đồ thị hàm số $f'(x)$ như hình vẽ.\\
		Hàm số $y=f(\cos x)+x^2-x$ đồng biến trên khoảng
		\choice
		{$(-2 ; 1)$}
		{$(0 ; 1)$}
		{\True $(1 ; 2)$}
		{$(-1 ; 0)$}
	}
	{
		\begin{tikzpicture}[scale=1,>=stealth, font=\footnotesize, line join=round, line cap=round]
			\def\a{-0.5} \def\b{0} \def\c{1.5} \def\d{0} % Hệ số
			\def\xmin{-3} \def\xmax{4}
			\def\ymin{-2} \def\ymax{2} 
			%\draw[color=gray!50,dashed] (\xmin,\ymin) grid (\xmax,\ymax); 
			\draw[->] (\xmin,0)--(\xmax,0) node [below]{$x$};
			\draw[->] (0,\ymin)--(0,\ymax) node [left]{$y$};
			\node at (0,0) [above left]{$O$};
			\node at (3,0) [below right]{$3$};
			\draw[dashed] (-2,0) node[below]{$-2$}--(-2,1) --(0,1) node[above right]{$1$} --(1,1)--(1,0) node[below]{$1$};
			\draw[dashed] (-1,0) node[below right]{$-1$}--(-1,-1) --(0,-1) node[above right]{$-1$} --(2,-1)--(2,0) node[below right]{$2$};
			\clip (\xmin+0.1,\ymin+0.1) rectangle (\xmax-0.5,\ymax-0.1);
			\draw[smooth,samples=300][domain=-2:2] plot(\x,{\a*(\x)^3+\b*(\x)^2+\c*(\x)+\d});
		\end{tikzpicture}
	}
	\loigiai{
		Đặt  $g(x)=f(\cos x)+x^2-x$.\\
		Ta có $g'(x)=-\sin x \cdot f'(\cos x)+2 x-1$\\
		Vì $\cos x \in[-1 ; 1]$ nên từ đồ thị $f'(x)$ ta suy ra $f'(\cos x) \in[-1 ; 1]$.\\
		Do đó $\left|-\sin x \cdot f'(\cos x)\right| \leq 1, ~\forall x \in \mathbb{R}$.\\
		Ta suy ra $g'(x)=\sin x \cdot f'(\cos x)+2 x-1 \geq-1+2 x-1=2 x-2$
		$\Rightarrow g'(x)>0, ~\forall x>1$.\\
		Vậy hàm số đồng biến trên $(1 ; 2)$.
	}
\end{ex}
\begin{ex}[THPT Nguyễn Viết Xuân - 2020]%[2D1G1-2]
	\immini{
		Cho hàm số $f(x)$. Hàm số $y=f'(x)$ có đồ thị như hình vẽ.\\
		Hàm số $g(x)=f\left(3 x^2-1\right)-\dfrac{9}{2}x^4+3 x^2$ đồng biến trên khoảng nào dưới đây?
		\choice
		{\True $\left(-\dfrac{2 \sqrt{3}}{3}; \dfrac{-\sqrt{3}}{3}\right)$}
		{$\left(0 ; \dfrac{2 \sqrt{3}}{3}\right)$}
		{$(1 ; 2)$}
		{$\left(-\dfrac{\sqrt{3}}{3}; \dfrac{\sqrt{3}}{3}\right)$} 
	}
	{
		\begin{tikzpicture}[scale=0.6,>=stealth, font=\footnotesize, line join=round, line cap=round]
			\def\a{0.25} \def\b{0.25} \def\c{-2} \def\d{0} % Hệ số
			\def\xmin{-5} \def\xmax{4}
			\def\ymin{-5} \def\ymax{5} 
			%\draw[color=gray!50,dashed] (\xmin,\ymin) grid (\xmax,\ymax); 
			\draw[->] (\xmin,0)--(\xmax,0) node [below]{$x$};
			\draw[->] (0,\ymin)--(0,\ymax) node [left]{$y$};
			\node at (0,0) [above left]{$O$};
			%\node at (3,0) [below right]{$3$};
			\draw[dashed] (-4,0) node[below left]{$-4$}--(-4,-4) --(0,-4) node[above right]{$-4$};
			\draw[dashed] (3,0) node[below right]{$3$}--(3,3) --(0,3) node[above right]{$3$};
			\clip (\xmin+0.1,\ymin+0.1) rectangle (\xmax-0.5,\ymax-0.1);
			\draw[smooth,samples=300] plot(\x,{\a*(\x)^3+\b*(\x)^2+\c*(\x)+\d});
		\end{tikzpicture}
	}
	
	\loigiai
	{
		TXĐ: $\mathscr{D}=\mathbb{R}$.\\
		Ta có $g'(x)=6 x f'\left(3 x^2-1\right)-18 x^3+6 x=6 x\left[f'\left(3 x^2-1\right)-3 x^2+1\right]$.\\
		$
		g'(x)=0 \Leftrightarrow\hoac{
			&x = 0\\
			&f '( 3 x ^{2}- 1 ) = 3 x ^{2}- 1}
		\Leftrightarrow \hoac{
			&x = 0\\
			&3 x ^{2}- 1 = - 4 \text{~(vô nghiệm)}\\
			&3 x ^{2}- 1 = 0\\
			&3 x ^{2}- 1 = 3}\Leftrightarrow \hoac{&x=0 \\
			&x=\pm \dfrac{\sqrt{3}}{3}\\
			&x=\pm \dfrac{2 \sqrt{3}}{3}.}
		$\\
		Bảng xét dấu
		\begin{center}
			\begin{tikzpicture}
				\tkzTabInit[nocadre,lgt=1.2,espcl=2.2,deltacl=0.6]
				{$x$ /1.2,$f'(x)$ /0.7}
				{$-\infty$,$-\dfrac{2 \sqrt{3}}{3}$,$-\dfrac{ \sqrt{3}}{3}$,$0$,$\dfrac{\sqrt{3}}{3}$,$\dfrac{2 \sqrt{3}}{3}$,$+\infty$}
				\tkzTabLine{,-,$0$,+,$0$,-,$0$,+,$0$,-,$0$,+,}
			\end{tikzpicture}
		\end{center}
		Vậy hàm số đồng biến trong khoảng $\left(-\dfrac{2 \sqrt{3}}{3}; \dfrac{-\sqrt{3}}{3}\right)$.}
\end{ex}
\begin{ex}[Trần Phú - Quảng Ninh - 2020]%[2D1G1-2]
	Cho hàm số $f(x)$ có bảng xét dấu của đạo hàm như sau
	\begin{center}
		\begin{tikzpicture}
			\tkzTabInit[nocadre,lgt=1.2,espcl=2,deltacl=0.6]
			{$x$ /0.6,$f'(x)$ /0.6}
			{$-\infty$,$-4$,$-1$,$2$,$7$,$+\infty$}
			\tkzTabLine{,+,$0$,-,$0$,+,$0$,-,$0$,+,}
		\end{tikzpicture}
	\end{center}
	Hàm số $y=f(2 x+1)+\dfrac{2}{3}x^3-8 x+5$ nghịch biến trên khoảng nào dưới đây?
	\choice
	{$(-\infty ;-2)$}
	{$(1 ;+\infty)$}
	{$(-1 ; 7)$}
	{\True $\left(-1 ; \dfrac{1}{2}\right)$}
	\loigiai{
		Ta có $y'=2 f'(2 x+1)+2 x^2-8$.\\
		Xét $y'\leq 0 \Leftrightarrow 2 f'(2 x+1)+2 x^2-8 \leq 0 \Leftrightarrow f'(2 x+1) \leq 4-x^2$.\\
		Đặt $t=2x+1$, ta có $f'(t) \leq \dfrac{-t^2+2 t+15}{4}$.\\
		Vì $\dfrac{-t^2+2 t+15}{4}\geq 0, \forall t \in[-3 ; 5]$.\\
		Mà $f'(t) \leq 0, \forall t \in[-3 ; 2]$.\\
		Nên $f'(t) \leq \dfrac{-t^2+2 t+15}{4}\Rightarrow t \in[-3 ; 2]$.\\
		Suy ra $-3 \leq 2 x+1 \leq 2 \Leftrightarrow-2 \leq x \leq \dfrac{1}{2}$.}
\end{ex}

\begin{ex}[Chuyên Thái Bình - Lần 3 - 2020]%[2D1G1-2]
	\immini{
		Cho hàm số $y=f(x)$ liên tục trên $\mathbb{R}$ có đồ thị hàm số $y=f'(x)$ cho như hình vẽ.\\
		Hàm số $g(x)=2 f(|x-1|)-x^2+2 x+2020$ đồng biến trên khoảng nào?
		\choice
		{\True $(0 ; 1)$}
		{$(-3 ; 1)$}
		{$(1 ; 3)$}
		{$(-2 ; 0)$}
	}
	{
		\begin{tikzpicture}[scale=0.7,>=stealth, font=\footnotesize, line join=round, line cap=round]
			\def\a{-0.333} \def\b{1} \def\c{1.333} \def\d{-1} % Hệ số
			\def\xmin{-3} \def\xmax{5}
			\def\ymin{-3} \def\ymax{5} 
			%\draw[color=gray!50,dashed] (\xmin,\ymin) grid (\xmax,\ymax); 
			\draw[->] (\xmin,0)--(\xmax,0) node [below]{$x$};
			\draw[->] (0,\ymin)--(0,\ymax) node [left]{$y$};
			\node at (0,0) [above left]{$O$};
			%\node at (3,0) [below right]{$3$};
			\draw[dashed] (-1,0) node[above]{$-1$}--(-1,-1) --(0,-1) node[above right]{$-1$};
			\draw[dashed] (1,0) node[below right]{$1$}--(1,1) --(0,1) node[above right]{$1$};
			\draw[dashed] (3,0) node[below right]{$3$}--(3,3) --(0,3) node[above right]{$3$};
			\clip (\xmin+0.1,\ymin+0.1) rectangle (\xmax-0.5,\ymax-0.1);
			\draw[smooth,samples=300] plot(\x,{\a*(\x)^3+\b*(\x)^2+\c*(\x)+\d});
			\draw[smooth,samples=300] plot(\x,{(\x)});
		\end{tikzpicture}
	}
	\loigiai{
		Với $x>1$, ta có $g(x)=2 f(x-1)-(x-1)^2+2021 \Rightarrow g'(x)=2 f'(x-1)-2(x-1)$.\\
		Hàm số đồng biến $\Leftrightarrow 2 f'(x-1)-2(x-1)>0 \Leftrightarrow f'(x-1)>x-1 \quad(*)$.\\
		Đặt $t=x-1$, khi đó $(*) \Leftrightarrow f'(t)>t \Leftrightarrow\hoac{&1<t<3 \\ &t<-1}\Rightarrow\hoac{&2<x<4 \\ &x<0 ~(\text{loại}).}$\\
		Với $x<1$, ta có $g(x)=2 f(1-x)-(1-x)^2+2021 \Rightarrow g'(x)=-2 f'(1-x)+2(1-x)$.\\
		Hàm số đồng biến $\Leftrightarrow-2 f'(1-x)+2(1-x)>0 \Leftrightarrow f'(1-x)<1-x \quad(* *)$.\\
		Đặt $t=1-x$, khi đó $(* *) \Leftrightarrow f'(t)<t \Leftrightarrow\hoac{&-1<t<1 \\ &t>3}\Rightarrow\hoac{&0<x<2 \\ &x<-2}\Rightarrow\hoac{&0<x<1 \\ &x<-2.}$\\
		Vậy hàm số $g(x)$ đồng biến trên các khoảng $(-\infty ;-2),(0 ; 1),(2 ; 4)$.
	}
\end{ex}
\begin{ex}[Sở Phú Thọ - 2020]%[2D1G1-2]
	\immini{
		Cho hàm số $y=f(x)$ có đồ thị hàm số $f'(x)$ như hình vẽ.\\
		Hàm số $g(x)=f\left(1+e^x\right)+2020$ nghịch biến trên khoảng nào dưới đây?
		\choice
		{$(0 ;+\infty)$}
		{$\left(\dfrac{1}{2}; 1\right)$}
		{\True $\left(0 ; \dfrac{1}{2}\right)$}
		{$(-1 ; 1)$}
	}{
		\begin{tikzpicture}[scale=0.7,>=stealth, font=\footnotesize, line join=round, line cap=round]
			\def\a{1} \def\b{-3} \def\c{0} \def\d{0} % Hệ số
			\def\xmin{-2} \def\xmax{4}
			\def\ymin{-5} \def\ymax{2} 
			%\draw[color=gray!50,dashed] (\xmin,\ymin) grid (\xmax,\ymax); 
			\draw[->] (\xmin,0)--(\xmax,0) node [below]{$x$};
			\draw[->] (0,\ymin)--(0,\ymax) node [left]{$y$};
			\node at (0,0) [above left]{$O$};
			\node at (3,0) [below right]{$3$};
			\draw[dashed] (2,0) node[above]{$2$}--(2,-4) --(0,-4) node[left]{$-4$};
			\clip (\xmin+0.1,\ymin+0.1) rectangle (\xmax-0.5,\ymax-0.1);
			\draw[smooth,samples=300] plot(\x,{\a*(\x)^3+\b*(\x)^2+\c*(\x)+\d});
		\end{tikzpicture}
	}
	\loigiai{
		$g'(x)=e^x f'\left(1+e^x\right)$.\\
		Do $e^x>0, \forall x$ nên $g'(x) \leq 0 \Leftrightarrow f'\left(1+e^x\right) \leq 0 \Leftrightarrow 1+e^x \leq 3 \Leftrightarrow x \leq \ln 2$, dấu bằng xảy ra tại hữu hạn điểm.\\
		Nên $g(x)$ nghịch biến trên $(-\infty ; \ln 2)$.\\
		Vì $\left(0 ; \dfrac{1}{2}\right) \subset (-\infty ; \ln 2)$ nên hàm số đã cho nghịch biến trên $\left(0 ; \dfrac{1}{2}\right)$.
	}
\end{ex}

\begin{ex}%[2D1K1-2]
	[THPT Anh Sơn - Nghệ An - 2020]
	Cho hàm số $y=f(x)$ có bảng xét dấu của đạo hàm như sau.
	\begin{center}
		\begin{tikzpicture}
			\tkzTabInit[nocadre,lgt=1.2,espcl=2,deltacl=0.6]
			{$x$ /0.6,$f'(x)$ /0.6}
			{$-\infty$,$-2$,$-1$,$2$,$4$,$+\infty$}
			\tkzTabLine{,+,$0$,-,$0$,+,$0$,-,$0$,+,}
		\end{tikzpicture}
	\end{center}
	Hàm số $y=-2 f(x)+2019$ nghịch biến trên khoảng nào trong các khoảng dưới đây?
	\choice
	{$(2 ; 4)$}
	{$(-4 ; 2)$}
	{$(-2 ;-1)$}
	{\True $(-1 ; 2)$}
	\loigiai{
		Ta có $y'=-2 f'(x)$.\\
		$
		y'=0 \Leftrightarrow-2 f'(x)=0 \Leftrightarrow\hoac{&
			x=-2 \\
			&x=-1 \\
			&x=2 \\
			&x=4.}$\\
		Từ bảng xét dấu của $f'(x)$ ta có
		\begin{center}
			\begin{tikzpicture}
				\tkzTabInit[nocadre,lgt=1,espcl=2,deltacl=0.6]
				{$x$ /0.6,$y'$ /0.6}
				{$-\infty$,$-2$,$-1$,$2$,$4$,$+\infty$}
				\tkzTabLine{,-,$0$,+,$0$,-,$0$,+,$0$,-,}
			\end{tikzpicture}
		\end{center}
		Từ bảng xét dấu ta có hàm số nghịch biến trên khoảng $(-\infty ;-2),(-1 ; 2)$ và $(4 ;+\infty)$.}
\end{ex}

\begin{ex}[THPT Anh Sơn - Nghệ An - 2020]%[2D1G1-2]
	Cho hàm số $f(x)$ xác định và liên tục trên $\mathbb{R}$ và có đạo hàm $f'(x)$ thỏa mãn $f'(x)=(1-x)(x+2) g(x)+2019$ với $g(x)<0, ~\forall x \in \mathbb{R}$ . Hàm số $y=f(1-x)+2019 x+2020$ nghịch biến trên khoảng nào?
	\choice
	{$(1 ;+\infty)$}
	{$(0 ; 3)$}
	{$(-\infty ; 3)$}
	{\True $(3 ;+\infty)$}
	\loigiai{
		Đặt $h(x)=f(1-x)+2019 x+2020$.\\
		Vì hàm số $f(x)$ xác định trên $\mathbb{R}$ nên hàm số $h(x)$ cũng xác định trên $\mathbb{R}$.\\
		Ta có $h'(x)=-f'(1-x)+2019$.\\
		Do $h'(x)=0$ tại hữu hạn điểm nên để tìm khoảng nghịch biến của hàm số $h(x)$, ta tìm các giá trị của $x$ sao cho $h'(x)<0 \Leftrightarrow-f'(1-x)+2019<0$\\ 
		$\Leftrightarrow f'(1-x)-2019>0 \\
		\Leftrightarrow x(3-x) g(1-x)>0 \Leftrightarrow x(3-x)<0(\text{~do~}g(x)<0, \forall x \in \mathbb{R})$\\
		$\Leftrightarrow\hoac{&
			x<0 \\
			&x>3.}$\\
		Vậy hàm số $y=f(1-x)+2019 x+2020$ nghịch biến trên các khoảng $(-\infty ; 0)$ và $(3 ;+\infty).$}
\end{ex}

\begin{ex}%[2D1G1-2]
	Cho hàm số $y=f(x)$ xác định trên $\mathbb{R}$ và có bảng xét dấu đạo hàm như sau:
	\begin{center}
		\begin{tikzpicture}
			\tkzTabInit[nocadre,lgt=2,espcl=2,deltacl=0.6]
			{$x$ /0.6,$f'(x)$ /0.6}
			{$-\infty$,$-1$,$1$,$4$,$+\infty$}
			\tkzTabLine{,-,$0$,+,$0$,-,$0$,+,}
		\end{tikzpicture}
	\end{center}
	Biết $f(x)>2,~ \forall x \in \mathbb{R}$. Xét hàm số $g(x)=f(3-2 f(x))-x^3+3 x^2-2020$. Khẳng định nào sau đây đúng?
	\choice
	{Hàm số $g(x)$ đồng biến trên khoảng $(-2 ;-1)$}
	{Hàm số $g(x)$ nghịch biến trên khoảng $(0 ; 1)$}
	{Hàm số $g(x)$ đồng biến trên khoảng $(3 ; 4)$}
	{\True Hàm số $g(x)$ nghịch biến trên khoảng $(2 ; 3)$}
	\loigiai{
		Ta có $g'(x)=-2 f'(x) f'(3-2 f(x))-3 x^2+6 x$.\\
		Vì $f(x)>2, ~\forall x \in \mathbb{R}$ nên $3-2 f(x)<-1 ~\forall x \in \mathbb{R}$.\\
		Từ bảng xét dấu $f'(x)$ suy ra $f'(3-2 f(x))<0, ~\forall x \in \mathbb{R}$.\\
		Từ đó ta có bảng xét dấu sau:
		\begin{center}
			\begin{tikzpicture}
				\tkzTabInit[nocadre,lgt=4,espcl=1.7,deltacl=0.6]
				{$x$ /0.7,$-f'(x)f'\left( 3-2f(x)\right) $/0.8,$-3x^2+6x$/0.7}
				{$-\infty$,$-1$,$0$,$1$,$2$,$4$,$+\infty$}
				\tkzTabLine{,-,$0$,+,|,+,$0$,-,|,-,$0$,+,}
				\tkzTabLine{,-,|,-,$0$,+,|,+,$0$,-,|,-,}
			\end{tikzpicture}
		\end{center}
		Từ bảng xét dấu trên, loại trừ đáp án suy ra hàm số $g(x)$ nghịch biến trên khoảng $(2 ; 3)$.}
\end{ex}

\begin{ex}%[2D1G1-2]
	Cho hàm số $f(x)$ có bảng biến thiên như sau:
	\begin{center}
		\begin{tikzpicture}
			\tkzTabInit[nocadre,lgt=1.2,espcl=2.5,deltacl=0.6]
			{$x$ /0.7, $f'(x)$ /0.7, $f(x)$ /2.5}
			{$-\infty$,$1$,$2$,$3$,$4$,$+\infty$}
			\tkzTabLine{,+,$0$,-,$0$,+,$0$,-,$0$,+,}
			\tkzTabVar{-/$-\infty$,+/$3$,-/$1$,+/$2$,-/$0$,+/$+\infty$}
		\end{tikzpicture}
	\end{center}
	Hàm số $y=(f(x))^3-3 .(f(x))^2$ nghịch biến trên khoảng nào dưới đây?
	\choice
	{$(1 ; 2)$}
	{$(3 ; 4)$}
	{$(-\infty ; 1)$}
	{\True $(2 ; 3)$}
	\loigiai{
		Ta có $y'=3 \cdot(f(x))^2 \cdot f'(x)-6 \cdot f(x) \cdot f'(x)=3 f(x) \cdot f'(x) \cdot[f(x)-2]. \\
		y'=0 \Leftrightarrow \hoac{&f(x)=0 \Leftrightarrow x \in\left\{x_1, 4 \mid x_1<1\right\}\\
			&f(x)=2 \Leftrightarrow x \in\left\{x_2, x_3, 3, x_4 \mid x_1<x_2<1<x_3<2 ; 4<x_4\right\}\\
			&f'(x)=0 \Leftrightarrow x \in\{1,2,3,4\}.}$\\
		Lập bảng xét dấu ta có
		\begin{center}
			\begin{tikzpicture}
				\tkzTabInit[nocadre,lgt=2,espcl=1.5,deltacl=0.6]
				{$x$ /0.7,$f(x)$ /0.7,$f(x)-2$ /0.7,$f'(x)$/0.7,$y'$/0.7}
				{$-\infty$,$x_1$,$x_2$,$1$,$x_3$,$2$,$3$,$4$,$x_4$,$+\infty$}
				\tkzTabLine{,-,$0$,+,|,+,|,+,|,+,|,+,$0$,+,|,+,|,+,}
				\tkzTabLine{,-,|,-,$0$,+,$0$,+,$0$,-,|,-,$0$,-,|,-,$0$,+}
				\tkzTabLine{,+,|,+,|,+,$0$,-,|,-,$0$,+,$0$,-,$0$,+,|,+}
				\tkzTabLine{,+,$0$,-,$0$,+,$0$,-,$0$,+,$0$,-,$0$,+,$0$,-,$0$,+}
			\end{tikzpicture}
		\end{center}
		
		Do đó hàm số nghịch biến trên khoảng $(2 ; 3)$.
	}
\end{ex}
\begin{ex}%[2D1G1-2]
	Cho hàm số $y=f(x)$ có đồ thị nằm trên trục hoành và có đạo hàm trên $\mathbb{R}$, bảng xét dấu của biểu thức $f'(x)$ như bảng dưới đây.
	\begin{center}
		\begin{tikzpicture}
			\tkzTabInit[nocadre,lgt=1.2,espcl=2,deltacl=0.6]
			{$x$ /0.6,$f'(x)$ /0.6}
			{$-\infty$,$-2$,$-1$,$3$,$+\infty$}
			\tkzTabLine{,-,$0$,+,$0$,-,$0$,+,}
		\end{tikzpicture}
	\end{center}
	Hàm số $y=g(x)=\dfrac{f\left(x^2-2 x\right)}{f\left(x^2-2 x\right)+1}$ nghịch biến trên khoảng nào dưới đây?
	\choice
	{$(-\infty ; 1)$}
	{$\left(-2 ; \dfrac{5}{2}\right)$}
	{\True $(1 ; 3)$}
	{$(2 ;+\infty)$}
	\loigiai{
		$ g'(x)=\dfrac{\left(x^2-2 x\right)'\cdot f'\left(x^2-2 x\right)}{\left(f\left(x^2-2 x\right)+1\right)^2}=\dfrac{(2 x-2) \cdot f'\left(x^2-2 x\right)}{\left(f\left(x^2-2 x\right)+1\right)^2}. \\
		g'(x)=0 \Leftrightarrow\hoac{
			&2 x - 2 = 0\\
			&f '( x ^{2}- 2 x ) = 0}
		\Leftrightarrow \hoac{&x = 1\\
			&x ^{2}- 2 x = - 2\\
			&x ^{2}- 2 x = - 1\\
			&x ^{2}- 2 x = 3}
		\Leftrightarrow \hoac{&x=1 \\
			&x=-1 \\
			&x=3.}
		$\\
		Ta có bảng xét dấu của $g'(x)$
		\begin{center}
			\begin{tikzpicture}
				\tkzTabInit[nocadre,lgt=1.2,espcl=2,deltacl=0.6]
				{$x$ /0.6,$g'(x)$ /0.6}
				{$-\infty$,$-1$,$1$,$3$,$+\infty$}
				\tkzTabLine{,-,$0$,+,$0$,-,$0$,+,}
			\end{tikzpicture}
		\end{center}
		Dựa vào bảng xét dấu ta có hàm số $y=g(x)$ nghịch biến trên các khoảng $(-\infty ;-1)$ và $(1 ; 3)$.}
\end{ex}
\begin{ex}[Liên trường huyện Quảng Xương - Thanh Hóa - 2021]%[2D1G1-2]
	\immini{
		Cho các hàm số $y=f(x)$; $y=g(x)$ liên tục trên $\mathbb{R}$ và có đồ thị các đạo hàm $f'(x) ; g'(x)$ (đồ thị hàm số $y=g'(x)$ là đường đậm hơn) như hình vẽ.\\
		Hàm số $h(x)=f(x-1)-g(x-1)$ nghịch biến trên khoảng nào dưới đây?
		\choice
		{$\left(\dfrac{1}{2}; 1\right)$}
		{$(1 ;+\infty)$}
		{$(2 ;+\infty)$}
		{\True $\left(-1 ; \dfrac{1}{2}\right)$}
	}
	{
		\begin{tikzpicture}[scale=1,>=stealth, font=\footnotesize, line join=round, line cap=round]
			%\def\a{1} \def\b{-6} \def\c{9} \def\d{1} % Hệ số
			\def\xmin{-4} \def\xmax{3}
			\def\ymin{-2} \def\ymax{4} 
			%\draw[color=gray!50,dashed] (\xmin,\ymin) grid (\xmax,\ymax); 
			\draw[->] (\xmin,0)--(\xmax,0) node [below]{$x$};
			\draw[->] (0,\ymin)--(0,\ymax) node [left]{$y$};
			\node at (0,0) [above left]{$O$};
			\node at (1,3) [below left]{$f'(x)$};
			\node at (1.5,3) [below right]{$g'(x)$};
			\draw[dashed] (-2,0) node[above right]{$-2$}--(-2,1);
			\draw[dashed] (1,0) node[below]{$1$}--(1,1);
			\draw[dashed] (-0.5,0) node[below]{$-0{,}5$}--(-0.5,2.125);
			\clip (\xmin+0.1,\ymin+0.1) rectangle (\xmax-0.5,\ymax-0.1);
			\draw[smooth,samples=300][domain=-3:2] plot(\x,{2*(\x)^4+4*(\x)^3-2*(\x)^2-4*(\x)+1});
			\draw[smooth,samples=300,line width=1.2pt] plot(\x,{(\x)^3+(\x)^2-2*(\x)+1});
		\end{tikzpicture}
	}
	
	\loigiai{
		Ta có: $h'(x)=f'(x-1)-g'(x-1)$.\\
		Dựa vào hình vẽ ta có hàm số $h(x)$ nghịch biến\\
		$\Leftrightarrow h'(x)<0 \Leftrightarrow f'(x-1)<g'(x-1)$\\
		$
		\Leftrightarrow\hoac{&- 2 < x - 1 < - \dfrac{1}{2}\\
			&0 < x - 1 < 1}
		\Leftrightarrow \hoac{
			&-1<x<\dfrac{1}{2}\\
			&1<x<2.}$\\
		Do đó hàm số $h(x)$ nghịch biến trên các khoảng $\left(-1 ; \dfrac{1}{2}\right)$ và $(1 ; 2)$.
	}
\end{ex}
\begin{ex}[THPT Quế Võ 1 - Bắc Ninh - 2021] %[2D1G1-2]
	\immini{
		Cho ba hàm số $y=f(x), y=g(x), y=h(x)$. Đồ thị của ba hàm số $y=f'(x), y=g'(x), y=h'(x)$ được cho như hình vẽ.\\
		Hàm số $k(x)=f(x+7)+g(5 x+1)-h\left(4 x+\dfrac{3}{2}\right)$ đồng biến trên khoảng nào dưới đây?
		\choice
		{$\left(-\dfrac{5}{8}; 0\right)$}
		{$\left(\dfrac{5}{8};+\infty\right)$}
		{\True $\left(\dfrac{3}{8}; 1\right)$}
		{$\left(-\dfrac{3}{8}; 1\right)$}
	}
	{
		\begin{tikzpicture}[scale=0.25,>=stealth, font=\footnotesize, line join=round, line cap=round]
			\def\a{-.078} \def\b{1.25} \def\c{0} % Hệ số
			\def\xmin{-4} \def\xmax{25}
			\def\ymin{-8} \def\ymax{18}
			
			%\draw[color=gray!50,dashed] (\xmin,\ymin) grid (\xmax,\ymax);
			
			\draw[->] (\xmin,0)--(\xmax,0) node [below]{$x$};
			\draw[->] (0,\ymin)--(0,\ymax) node [left]{$y$};
			\node at (20,14) [below right]{$y=g'(x)$};
			\node at (18,-2) [below left]{$y=h'(x)$};
			\node at (16,5) [below right]{$y=f'(x)$};
			\node at (0,0) [below left]{$O$};
			\draw[dashed] (3,0) node[below]{$3$}--(3,10)--(0,10) node[left]{$10$};
			\draw[dashed] (8,0) node[below]{$8$}--(8,5)--(0,5) node[left]{$5$};
			\draw[dashed] (4,0) node[below]{$4$}--(4,2)--(0,2) node[left]{$2$};
			\clip (\xmin+0.1,\ymin+0.1) rectangle (\xmax-0.5,\ymax-0.1);
			\draw[smooth,samples=300,domain=-2:18] plot(\x,{\a*(\x)^2+\b*(\x)+\c});
			%\draw[smooth,samples=300,domain=-2:25] plot(\x,{0.02*(\x)^3-0.6*(\x)^2+5.16*(\x)});
			\draw[line width=1.2pt] (-2,5)..controls (1.7,1.5) and (4.5,1.6)..(7,2.6);
			\draw[line width=1.2pt] (7,2.6)..controls (9,3.5) and (12,5)..(20,13);
			\draw (-0.5,-2) -- (0,0)--(3,10).. controls +(65:1) and + (-190:1)..(6,15).. controls +(0:1) and + (-180:1)..(14,-1).. controls +(0:1) and + (+80:1)..(19,16);
			
		\end{tikzpicture}
	}
	\loigiai{
		Ta có $k'(x)=f'(x+7)+5 g'(5 x+1)-4 h'\left(4 x+\dfrac{3}{2}\right)$.\\
		Khi $x \in \left( \dfrac{3}{8};1\right)$ thì $\heva{&7{,}375<x+7<8\\&2{,}875<5x+1<6\\&3<4x+\dfrac{4}{3}<5{,}5}\Leftrightarrow \heva{&f'(x+7)>10\\&g'(5x+1)>2 \Rightarrow 5g'(5x+1)>10  \\&h'\left( 4x+\dfrac{3}{2}\right)<5 \Rightarrow -4h'\left( 4x+\dfrac{3}{2}\right) >-20}.$\\
		Do đó $k'(x)=f'(x+7)+5g'(5x+1)-4h'\left( 4x+\dfrac{3}{2}\right)>0$.\\
		Hàm số $k(x)=f(x+7)+g(5 x+1)-h\left(4 x+\dfrac{3}{2}\right)$ đồng biến trên $\left(\dfrac{3}{8}; 1\right)$.
	}
\end{ex}
\begin{ex}[THPT Thanh Chương 1 - Nghệ An- 2021] %[2D1G1-2]
	Cho hàm số $y=f(x)$ liên tục trên $\mathbb{R}$ có bảng xét dấu đạo hàm như sau
	\begin{center}
		\begin{tikzpicture}
			\tkzTabInit[nocadre,lgt=1.2,espcl=2,deltacl=0.6]
			{$x$ /0.6,$f'(x)$ /0.6}
			{$-\infty$,$1$,$2$,$3$,$4$,$+\infty$}
			\tkzTabLine{,-,$0$,+,$0$,+,$0$,-,$0$,+,}
		\end{tikzpicture}
	\end{center}
	Hàm số $y=3f(2x-1)-4x^3+15x^2-18x+1$ đồng biến trên khoảng nào dưới đây?
	\choice
	{$\left(3;+\infty\right)$}
	{\True $\left(1;\dfrac{3}{2}\right)$}
	{$\left(\dfrac{5}{2}; 3\right)$}
	{$\left(2;\dfrac{5}{2}\right)$}
	\loigiai{
		Ta có $y'=6f'(2x-1)-12x^2+30x-18=6\left[f'(2x-1)-2x^2+5x-3\right] $.\\
		Có $f'(2x-1)=0 \Leftrightarrow \hoac{&2x-1=1\\&2x-1=2\\&2x-1=3\\&2x-1=4} \Leftrightarrow \hoac{&x=1\\&x=\dfrac{3}{2}\\&x=2\\&x=\dfrac{5}{2}.}$
		Ta có bảng xét dấu sau
		\begin{center}
			\begin{tikzpicture}
				\tkzTabInit[nocadre,lgt=3.0,espcl=1.5,deltacl=0.6]
				{$x$ /1.0,$f(x)$ /0.6,$f'(2x-1)$ /0.6,$-2x^2+5x-3$/0.6,$g'(x)$/0.6}
				{$-\infty$,$1$,$\dfrac{3}{2}$,$2$,$\dfrac{5}{2}$,$3$,$4$,$+\infty$}
				\tkzTabLine{,-,$0$,+,|,+,$0$,+,|,+,$0$,-,$0$,+,}
				\tkzTabLine{,-,$0$,+,$0$,+,$0$,-,$0$,+,|,+,|,+,}
				\tkzTabLine{,-,$0$,+,$0$,-,|,-,|,-,|,-,|,-,}
				\tkzTabLine{,-,$0$,+,$0$,,?,,|,,?,?,,?,}
			\end{tikzpicture}
		\end{center}
		Dựa vào bảng xét dấu trên, ta kết luận hàm số đã cho đồng biến trên khoảng $\left( 1; \dfrac{3}{2}\right).$
	}
\end{ex}


\begin{ex}%[2D2G4-3] %Câu 27 
	[THPT Hoàng Hoa Thám-Đà Nẵng-2021]
	Cho hàm số $f(x)$ có bảng xét dấu của $f'(x)$ như sau:\\
	\begin{center}
		\begin{tikzpicture}
			\tkzTabInit[lgt=1.2,espcl=2.3]
			{$x$/0.7, $f'(x)$ /.8} % first column
			{$-\infty$,$-3$,$1$, $2$, $+\infty$} % first row
			\tkzTabLine { ,+,0,-,0,+,0,+ }
		\end{tikzpicture}
	\end{center}	
	Hàm số $y=f\left(2-e^x\right)-\dfrac{1}{3}{e^{3x}}+3e^{2x}-5e^x+1$ đồng biến trên khoảng nào dưới đây?
	\choice
	{$\left(0;\dfrac{3}{2}\right)$}
	{$\left(1;3\right)$}
	{\True $\left(-3;0\right)$}
	{$\left(-4;-3\right)$}
	\loigiai{
		Ta có $y'=-e^x.f'\left(2-e^x\right)-e^{3x}+6e^{2x}-5e^x=e^x\left[-f'\left(2-e^x\right)-e^{2x}+6e^x-5\right]$ .\\
		Đặt $t=2-e^x$, ta được\\
		$y'=\left(2-t\right)\left[-f'(t)-\left(2-t\right)^2+6\left(2-t\right)-5\right]=\left(2-t\right)\left[-f'(t)-t^2-2t+3\right]$ .\\
		$y'=0\Leftrightarrow\left(2-t\right)\left[-f'(t)-t^2-2t+3\right]=0\Leftrightarrow
		\hoac{
			& t=2\\ 
			& f'(t)=-t^2-2t+3.}$\\
		Hàm số $g(x)=-x^2-2x+3$ là parabol có trục đối xứng $x=-1$ và cắt trục hoành tại 2 điểm có hoành độ 
		$\hoac{
			& x=1\\ 
			& x=-3
		}$. Suy ra $f'(t)=-t^2-2t+3\Leftrightarrow \hoac{
			& t=1\\ 
			& t=-3. }$\\
		Bảng xét dấu\\
		\begin{center}
			\begin{tikzpicture}
				\tkzTabInit[lgt=3.9,espcl=2,nocadre]
				{$t$/0.7, $2-t$ /0.8, $-f'(t)-t^2-2t+3$ /0.8, $y'$ /0.8} % first column
				{$-\infty$,$-3$,$1$,$2$,$+\infty$} % first row
				\tkzTabLine { ,+,|,+,|,+,z,-, } % second row
				\tkzTabLine {,-,0,+,0,-,|,-,} % third row
				\tkzTabLine {,-,0,+,0,-,0,+,} % last row
			\end{tikzpicture}
		\end{center}
		Dựa vào bảng xét dấu $y'>0,\forall x\in\left(-3;0\right)$.}
\end{ex}


\begin{ex}%[2D1G1-2]%Câu 28 
	[Sở Lạng Sơn 2022] Cho hàm số $f(x)$ có bảng biến thiên như sau:\\
	\begin{center}
		\begin{tikzpicture}
			\tkzTabInit[espcl=2.5,lgt=1,nocadre]
			{$x$/0.7,$y'$/0.7,$y$/3.5}
			{$-\infty$,$1$,$2$,$3$,$4$,$+\infty$}
			\tkzTabLine{,+,0,-,0,+,0,-,0,+,}
			\node (0) at ($(N12)+(0,-3)$) {$-\infty$};
			\node (1) at ($(N22)+(0,-.5)$) {$3$};
			\node (2) at ($(N32)+(0,-1.7)$) {$1$};
			\node (3) at ($(N42)+(0,-0.7)$) {$2$};
			\node (4) at ($(N52)+(0,-2.3)$) {$0$};
			\node (5) at ($(N62)+(0,-.3)$) {$+\infty$};
			%				\node (8) at ($(N42)+(0,-.5)$) {};
			%				\coordinate (9) at ($(N42)!.6!(N53)+ (-0.5,0)$);
			%				\coordinate (6) at ($(T12)!.6!(T13)$);
			%				\coordinate (7) at ($(T22)!.6!(T23)$);
			\draw[-stealth] (0)--(1);
			\draw[-stealth] (1)--(2);
			\draw[-stealth] (2)--(3);
			\draw[-stealth] (1)--(2);
			\draw[-stealth] (3)--(4);
			\draw[-stealth] (4)--(5);
			%				\draw[->,red] (5)--(8);
			%				\draw[->,red] (8)--(9);
			%				\draw[blue,dashed](6)--(7)node[above left]{$y=0$};
		\end{tikzpicture}		
	\end{center}
	Hàm số $y=\left[f(x)\right]^3-3\left[f(x)\right]^2$ đồng biến trên khoảng nào dưới đây?
	\choice
	{$\left(-\infty\,;1\right)$}
	{$\left(1\,;2\right)$}
	{\True $\left(3\,;4\right)$}
	{$\left(2\,;3\right)$}
	\loigiai{
		Ta có $y'=3f'(x)\left[f^2(x)-2f(x)\right]$. 
		Phương trình $y'=0\Leftrightarrow \hoac{
			&{f}'(x)=0\\ 
			& f(x)=0\\ 
			& f(x)=2.
		}$
		\begin{center}
			\begin{tikzpicture}
				\tkzTabInit[espcl=2.5,lgt=1.5]
				{$x$/0.7,$y'$/0.7,$y$/3.5}
				{$-\infty$,$1$,$2$,$3$,$4$,$+\infty$}
				\tkzTabLine{,+,0,-,0,+,0,-,0,+,}
				\node (0) at ($(N12)+(0,-3)$) {$-\infty$};
				\node (1) at ($(N22)+(0,-.3)$) {$3$};
				\node (2) at ($(N32)+(0,-1.7)$) {$1$};
				\node (3) at ($(N42)+(0,-0.8)$) {$2$};
				\node (4) at ($(N52)+(0,-2.3)$) {$0$};
				\node (5) at ($(N62)+(0,-.3)$) {$+\infty$};
				\node (a) at ($(N11)+(0.65,0.35)$) {$a$};
				\node (b) at ($(N11)+(2.0,0.4)$) {$b$};
				\node (c) at ($(N11)+(3.38,0.35)$) {$c$};
				\node (d) at ($(N11)+(11.85,0.4)$) {$d$};
				\node (6) at ($(N12)+(0,-0.8)$) {};
				\node (7) at ($(N62)+(0,-0.8)$) {};
				\node (8) at ($(N12)+(0,-2.3)$) {};
				\node (9) at ($(N62)+(0,-2.3)$) {};
				%				\node (8) at ($(N42)+(0,-.5)$) {};
				%				\coordinate (9) at ($(N42)!.6!(N53)+ (-0.5,0)$);
				\coordinate (A) at ($(0)!.25!(1)$);
				\coordinate (B) at ($(0)!.8!(1)$);
				\coordinate (C) at ($(1)!.35!(2)$);
				\coordinate (D) at ($(4)!.75!(5)$);
				%				\coordinate (7) at ($(T22)!.6!(T23)$);
				\draw[->] (0)--(1);
				\draw[->] (1)--(2);
				\draw[->] (2)--(3);
				\draw[->] (1)--(2);
				\draw[->] (3)--(4);
				\draw[->] (4)--(5);
				%				\draw[->,red] (5)--(8);
				%				\draw[->,red] (8)--(9);
				\draw[blue,dashed](6)--(7)node[below]{$y=2$} (a)--(A) (b)--(B) (c)--(C) (d)--(D);
				\draw[blue,dashed](8)--(9)node[below left]{$y=0$};
			\end{tikzpicture}		
		\end{center}
		Dựa vào bảng biến thiên, ta thấy $f'(x)=0\Leftrightarrow x\in \{ 1\,;2\,;3\,;4 \}$;\\
		$f(x)=0\Leftrightarrow x=a<1$ hoặc $x=4$;\\
		$f(x)=2\Leftrightarrow \hoac{
			& x=b\,\,\left(a<b<1\right)\\ 
			& x=c\in\left(1\,;2\right)\\ 
			& x=3\\ 
			& x=d>4.
		}$ \\
		Ta lập được bảng xét dấu của $y'$ 
		\begin{center}
			\begin{tikzpicture}
				\tkzTabInit[lgt=1.2,espcl=1.5,nocadre]
				{$x$/1, $f(x)$ /.8} % first column
				{$-\infty$,$a$, $b$, $1$,$c$, $2$,$3$, $4$, $d$, $+\infty$} % first row
				\tkzTabLine { ,+,z,-,z,+,z,-,z,+,z,-,z,+,z,-,z,+, } % second row
				%				\tkzTabLine {,-,z,+,t,+,} % third row
				%				\tkzTabLine {,+,d,-,z,+,} % last row
			\end{tikzpicture}
		\end{center}
		Từ bảng xét dấu, ta thấy hàm số đồng biến trên các khoảng \\
		$\left(-\infty;a\right)$, $\left(b;1\right)$, $\left(c;2\right)$, $\left(3;4\right)$ và $(d;+\infty)$.
	}
\end{ex}

\begin{ex}%[2D1G1-2]%Câu 29 
	[THPT Bùi Thị Xuân – Huế-2022] 
	\immini{
		Cho hàm số $y=f(x)$ là hàm đa thức bậc bốn. Đồ thị hàm số $f'(x+2)$ được cho trong hình vẽ bên. Hàm số 
		$$g(x)=4 f\left(x^2\right)-x^6+5 x^4-4 x^2+1$$
		đồng biến trên khoảng nào dưới đây?
		\choice
		{$(-4 ;-3)$}
		{\True $(2 ;+\infty)$}
		{$(-\sqrt{2};\sqrt{2})$}
		{$(-2 ;-1)$}}{
		\begin{tikzpicture}[scale=0.6,font=\footnotesize, line join=round, line cap=round, >=stealth] %Đường cong bậc 3
			\draw[thick, ->] (-5.3,0)--(5,0);
			\draw[thick, ->] (0,-3.5)--(0,7);
			\draw (5.2,0) node[below] {$x$};
			\draw (0,7.1) node[left]{$y$};
			\draw (0,0) node[below left]{$0$};
			\draw[fill] (-2,0) circle (0.5pt)node[below left]{$ -2 $};
			\draw[fill] (2,0) circle (0.5pt)node[below]{$ 2$};
			\draw[fill] (0,3) circle (0.5pt)node[left]{$ 3 $};
			\draw[fill] (0,1) circle (0.5pt)node[right]{$ 1 $};
			\draw[fill] (0,-1) circle (0.5pt)node[right]{$ -1 $};
			\draw[dashed] (-2,0)--(-2,1) --(0,1); 
			\draw[dashed](2,0)--(2,3)--(0,3);
			\draw[line width=1.2pt,smooth,samples=100,domain=-2.8:4.5] plot(\x,{-0.271*(\x)^3+0.75*(\x)^2+1.583*\x-1});
		\end{tikzpicture}		
	}
	\loigiai{
		$\begin{aligned}
			& g(x)=4f\left(x^2\right)-x^6+5x^4-4x^2+1\Rightarrow g' (x)=8xf'\left(x^2\right)-6x^5+20x^3-8x.\\ 
			& g' (x)=0\Leftrightarrow 8xf'\left(x^2\right)-6x^5+20x^3-8x=0 \\
			& \Leftrightarrow 2x\left[4f'\left(x^2\right)-3x^4+10x^2-4\right]=0\\ 
			&\Leftrightarrow 		\hoac{ 			& 2x=0\\ 
				& 4f'(x^2)-3x^4+10x^2-4=0
			}
			\Leftrightarrow \hoac{	& x=0\\ 
				& f'\left(x^2\right)=\dfrac{3}{4}{x^4}-\dfrac{5}{2}{x^2}+1.}
		\end{aligned}$\\ 
		Xét
		$f'\left(x^2\right)=\dfrac{3}{4}x^4-\dfrac{5}{2}x^2+1$. Đặt $x^2=t+2$, ta có\\
		$ f' (t+2)=\dfrac{3}{4}{(t+2)^2}-\dfrac{5}{2}(t+2)+1=\dfrac{3}{4}\left(t^2+4t+4\right)-\dfrac{5}{2}(t+2)-1=\dfrac{3}{4}{t^2}+\dfrac{1}{2}t-1$\\
		Khi đó số nghiệm của phương trình chính là số giao điểm của đồ thị hàm số $y=f' (t+2)$ và\\
		$ y=\dfrac{3}{4}{t^2}+\dfrac{1}{2}t-1$\\
		Ta có đồ thị 
		\begin{center}
			\begin{tikzpicture}[scale=0.6,font=\footnotesize, line join=round, line cap=round, >=stealth] %Đường cong bậc 3
				\draw[thick, ->] (-5.3,0)--(5,0);
				\draw[thick, ->] (0,-3.5)--(0,7);
				\draw (5.2,0) node[below] {$x$};
				\draw (0,7.1) node[left]{$y$};
				\draw (0,0) node[below left]{$0$};
				\draw[fill] (-2,0) circle (0.5pt)node[below left]{$ -2 $};
				\draw[fill] (2,0) circle (0.5pt)node[below]{$ 2$};
				\draw[fill] (0,3) circle (0.5pt)node[left]{$ 3 $};
				\draw[fill] (0,1) circle (0.5pt)node[right]{$ 1 $};
				\draw[fill] (0,-1) circle (0.5pt)node[right]{$ -1 $};
				\draw[dashed] (-2,0)--(-2,1) --(0,1); 
				\draw[dashed](2,0)--(2,3)--(0,3);
				\draw[line width=1.2pt,smooth,samples=100,domain=-2.8:4.5] plot(\x,{-0.271*(\x)^3+0.75*(\x)^2+1.583*\x-1});		
				\draw[line width=1.2pt,smooth,samples=100,domain=-3.3:2.8] plot(\x,{0.75*(\x)^2+0.5*\x-1});
			\end{tikzpicture}
		\end{center}
		Dựa vào đồ thị ta có $f' (t+2)=\dfrac{3}{4}t^2+\dfrac{1}{2}t-1\Leftrightarrow \hoac{& t=-2\\ & t=0\\ & t=2} \Leftrightarrow\hoac{& x+2=-2\\ & x+2=0\\ & x+2=2} \Leftrightarrow \hoac{& x=-4\\ & x=-2\\ & x=0.}$\\
		Ta có bảng xét dấu $g' (x)$ như sau
		\begin{center}
			\begin{tikzpicture}
				\tkzTabInit[lgt=1.2,espcl=2,nocadre]
				{$x$/0.7, $f(x)$ /.7}
				{$-\infty$, $-4$,$-2$, $0$, $+\infty$} % first row
				\tkzTabLine { ,-,z,+,z,-,z,+, }
			\end{tikzpicture}
		\end{center}
		Vậy hàm số $g(x)=4 f\left(x^2\right)-x^6+5 x^4-4 x^2+1$ đồng biến trên khoảng $(2 ;+\infty)$.}
\end{ex}

\begin{ex}%[2D1G1-2]%Câu 30
	[Chuyên Bắc Ninh 2022] 
	\immini{
		Cho hàm số $ y=f(x)$ liên tục trên $\mathbb{R}$ có đồ thị hàm số $ y=f'(x)$ có đồ thị như hình vẽ bên.
		Hàm số $g(x)=2f\left(\left| x-1\right|\right)-x^2+2x+2020$ đồng biến trên khoảng nào
		\choice
		{$\left(-2;0\right)$}
		{$\left(-3;1\right)$}
		{$\left(1\,;3\right)$}
		{\True $\left(0\,;\,1\right)$}}{
		\begin{tikzpicture}[scale=0.6,font=\footnotesize, line join=round, line cap=round, >=stealth] %Đường cong bậc 3
			\draw[thick, ->] (-3.3,0)--(5,0);
			\draw[thick, ->] (0,-3.0)--(0,5.5);
			\draw (5.2,0) node[below] {$x$};
			\draw (0,5.8) node[left]{$y$};
			\draw (0,0) node[below left]{$0$};
			\draw[fill] (-1,0) circle (0.5pt)node[above]{$ -1 $};
			\draw[fill] (1,0) circle (0.5pt)node[below]{$ 1$};
			\draw[fill] (0,1) circle (0.5pt)node[left]{$ 1 $};
			\draw[fill] (0,-1) circle (0.5pt)node[right]{$ -1 $};
			\draw[fill] (0,3) circle (0.5pt)node[left]{$ 3 $};
			\draw[fill] (3,0) circle (0.5pt)node[below]{$ 3 $};
			\draw[dashed] (-1,0)--(-1,-1) --(0,-1); 
			\draw[dashed](1,0)--(1,1)--(0,1);
			\draw[dashed](3,0)--(3,3)--(0,3);
			\draw[line width=1.2pt,smooth,samples=100,domain=-2.2:4.3] plot(\x,{-0.333*(\x)^3+1*(\x)^2+1.333*\x-1});		
			%\draw[line width=1.2pt,smooth,samples=100,domain=-3.3:2.8] plot(\x,{0.75*(\x)^2+0.5*\x-1});
		\end{tikzpicture}	
	}
	\loigiai{
		Ta có $g(x)=2f\left(\left| x-1\right|\right)-x^2+2x+2020\Leftrightarrow g(x)=2f\left(\left| x-1\right|\right)-\left(x-1\right)^2+2021$.\\
		Xét hàm số $ k\left(x-1\right)=2f\left(x-1\right)-\left(x-1\right)^2+2021$.\\
		Đặt $ t=x-1$\\
		Xét hàm số $ h(t)=2f(t)-t^2+2021$ $\Rightarrow{h}'(t)=2f'(t)-2t$.\\
		Kẻ đường $ y=x$ như hình vẽ.
		\begin{center}
			\begin{tikzpicture}[scale=0.6,font=\footnotesize, line join=round, line cap=round, >=stealth] %Đường cong bậc 3
				\draw[thick, ->] (-3.3,0)--(5,0);
				\draw[thick, ->] (0,-3.0)--(0,5.5);
				\draw (5.2,0) node[below] {$x$};
				\draw (0,5.8) node[left]{$y$};
				%	\draw (0,0) node[below left]{$0$};
				\draw[fill] (-1,0) circle (0.5pt)node[above]{$ -1 $};
				\draw[fill] (1,0) circle (0.5pt)node[below]{$ 1$};
				\draw[fill] (0,1) circle (0.5pt)node[left]{$ 1 $};
				\draw[fill] (0,-1) circle (0.5pt)node[right]{$ -1 $};
				\draw[fill] (0,3) circle (0.5pt)node[left]{$ 3 $};
				\draw[fill] (3,0) circle (0.5pt)node[below]{$ 3 $};
				\draw[dashed] (-1,0)--(-1,-1) --(0,-1); 
				\draw[dashed](1,0)--(1,1)--(0,1);
				\draw[dashed](3,0)--(3,3)--(0,3);
				\draw[line width=1.2pt,smooth,samples=100,domain=-2.2:4.3] plot(\x,{-0.333*(\x)^3+1*(\x)^2+1.333*\x-1});		
				%\draw[line width=1.2pt,smooth,samples=100,domain=-3.3:2.8] plot(\x,{0.75*(\x)^2+0.5*\x-1});
				\draw[line width=1.2pt,smooth,samples=100](-2,-2)--(4,4);
			\end{tikzpicture}
		\end{center}
		Khi đó $h'(t)>0\Leftrightarrow{f}'(t)-t>0\Leftrightarrow{f}'(t)>t$$\Leftrightarrow \hoac{
			& t<-1\\ 
			& 1<t<3.
		}$\\
		Do đó $k'\left(x-1\right)>0\Leftrightarrow \hoac{
			& x-1<-1\\ 
			& 1<x-1<3} \Leftrightarrow \hoac{
			& x<0\\ 
			& 2<x<4.}$\\
		Ta có bảng biến thiên của hàm số $ k\left(x-1\right)=2f\left(x-1\right)-\left(x-1\right)^2+2021$.
		\begin{center}
			\begin{tikzpicture}
				\tkzTabInit[lgt=1.8,espcl=2.3]
				{$x$ /1.2, $k'(x-1)$ /1.2,$k(x-1)$ /2}
				{$-\infty$ , $0$,$2$,$4$, $+\infty$}
				\tkzTabLine{,+,0,-,0,+,0,-,}
				\tkzTabVar{-/$ $ ,+/$ $, -/$ $,+/$ $,-/$ $}
			\end{tikzpicture}
		\end{center}
		Khi đó, ta có bảng biến thiên của $g(x)=2f\left(\left| x-1\right|\right)-\left(x-1\right)^2+2021$ bằng cách lấy đối xứng qua đường thẳng $ x=1$ như sau\\
		\begin{center}
			\begin{tikzpicture}
				\tkzTabInit[lgt=1.2,espcl=2.5,nocadre]
				{$x$ /0.7, $g'(x)$ /0.7,$g(x)$ /2.5}
				{$-\infty$ ,$-2$, $0$,$1$,$2$,$4$, $+\infty$}
				\tkzTabLine{,+,0,-,0,+,0,-,0,+,0,-,}
				\tkzTabVar{-/$ $ ,+/$ $, -/$ $,+/$ $,-/$ $,+/ $ $,-/$ $}
			\end{tikzpicture}
		\end{center}
		Vậy hàm số đồng biến trên $\left(0;1\right)$.}
\end{ex}

\begin{ex}%[2D1G1-2]%Câu 31
	[Chuyên Thái Bình 2022] 
	\immini{
		Cho hàm số $f(x)=a{x^4}+b{x^3}+c{x^2}+dx+a$ có đồ thị hàm số $y=f'(x)$ như hình vẽ bên. Hàm số $y=g(x)=f\left(1-2x\right)f\left(2-x\right)$ đồng biến trên khoảng nào dưới đây?
		\choice
		{$\left(\dfrac{1}{2};\dfrac{3}{2}\right)$}
		{$\left(-\infty ;0\right)$}
		{$\left(0;2\right)$}
		{\True $\left(3;+\infty\right)$}}{
		\begin{tikzpicture}[scale=0.9,font=\footnotesize, line join=round, line cap=round, >=stealth] %Đường cong bậc 3
			\draw[thick, ->] (-2.5,0)--(2.5,0);
			\draw[thick, ->] (0,-2.8)--(0,2.8);
			\draw (2.6,0) node[below] {$x$};
			\draw (0,2.9) node[left]{$y$};
			\draw (0,0) node[below left]{$0$};
			\draw[fill] (-1,0) circle (0.5pt)node[below left]{$ -1 $};
			\draw[fill] (1,0) circle (0.5pt)node[below right]{$ 1$};
			%			\draw[dashed] (-1,0)--(-1,-1) --(0,-1); 
			%			\draw[dashed](1,0)--(1,1)--(0,1);
			%			\draw[dashed](3,0)--(3,3)--(0,3);
			\draw[line width=1.2pt,smooth,samples=100,domain=-1.3:1.3] plot(\x,{3*(\x)^3-3*\x});		
			%\draw[line width=1.2pt,smooth,samples=100,domain=-3.3:2.8] plot(\x,{0.75*(\x)^2+0.5*\x-1});
		\end{tikzpicture}	
	}
	\loigiai{
		Ta có $f'(x)=4a{x^3}+3b{x^2}+2cx+d$, theo đồ thị thì đa thức $f'(x)$ có ba nghiệm phân biệt là $-1,0,1$ nên $f'(x)=4ax\left(x+1\right)\left(x-1\right)=4a{x^3}-4ax\Rightarrow f(x)=a{x^4}-2a{x^2}+a=a{\left(x^2-1\right)^2}$.\\
		Dựa vào đồ thị hàm số $y=f'(x)$ ta có $a>0$ nên $f(x)>0,\forall x\in\mathbb{R}\setminus\left\{\pm 1\right\}$.\\
		$g'(x)=\left[f\left(1-2x\right)\right]'f\left(2-x\right)+f\left(1-2x\right)\left[f\left(2-x\right)\right]'=-2f'\left(1-2x\right)f\left(2-x\right)-f\left(1-2x\right)f'\left(2-x\right)$. Xét $x\in\left(\dfrac{1}{2};\dfrac{3}{2}\right)\Rightarrow
		\heva{		
			& 1-2x\in\left(-2;0\right)\\ 
			& 2-x\in\left(\dfrac{1}{2};\dfrac{3}{2}\right)}$, dấu của $f'(x)$ không cố định trên $\left(\dfrac{1}{2};\dfrac{3}{2}\right)$ nên ta không kết luận được tính đơn điệu của hàm số $g(x)$ trên $\left(\dfrac{1}{2};\dfrac{3}{2}\right)$.\\
		Xét $x\in\left(-\infty ;0\right)\Rightarrow
		\heva{
			& 1-2x\in\left(1;+\infty\right)\\ 
			& 2-x\in\left(2;+\infty\right)} 
		\Rightarrow \heva{
			& f'\left(1-2x\right)>0\\ 
			& f'\left(2-x\right)>0} \Rightarrow g'(x)<0$.\\
		Do đó, hàm số $g(x)$ nghịch biến trên $\left(-\infty ;0\right)$.\\
		$x\in\left(0;2\right)\Rightarrow \heva{
			& 1-2x\in\left(-3;1\right)\\ 
			& 2-x\in\left(0;2\right)}$, dấu của $f'(x)$ không cố định trên $\left(-3;1\right)$ và $\left(0;2\right)$ nên ta không kết luận được tính đơn điệu của hàm số $g(x)$ trên $\left(\dfrac{1}{2};\dfrac{3}{2}\right)$.\\
		Xét $x\in\left(3;+\infty\right)\Rightarrow \heva{
			& 1-2x\in\left(-\infty ;-5\right)\\ 
			& 2-x\in\left(-\infty ;-1\right)} \Rightarrow \heva{
			& f'\left(1-2x\right)<0\\ 
			& f'\left(2-x\right)<0} \Rightarrow g'(x)>0$. \\
		Do đó, hàm số $g(x)$ đồng biến trên $\left(3;+\infty\right)$.}
\end{ex}

\begin{dang}{Bài toán hàm ẩn, hàm hợp liên quan đến tham số và một số bài toán khác}
\end{dang}

\begin{ex}%[2D1G1-3]%Câu 1
	[Chuyên Lê Hồng Phong Nam Định 2019]
	\immini{
		Cho hàm số $ y=f(x)$ có đạo hàm liên tục trên $\mathbb{R}$. Biết hàm số $ y=f'(x)$ có đồ thị như hình vẽ. Gọi $ S$ là tập hợp các giá trị nguyên $ m\in\left[-5\,;\,\text{5}\right]$ để hàm số $ g(x)=f\left(x+m\right)$ nghịch biến trên khoảng $\left(1\,;\,2\right)$. Hỏi $S$ có bao nhiêu phần tử?
		\choice
		{$ 4$}
		{$ 3$}
		{$ 6$}
		{\True $ 5$}}{
		\begin{tikzpicture}[scale=0.9,font=\footnotesize, line join=round, line cap=round, >=stealth] %Đường cong bậc 3
			\draw[thick, ->] (-2.5,0)--(4,0);
			\draw[thick, ->] (0,-2.8)--(0,2.8);
			\draw (4.3,0) node[below] {$x$};
			\draw (0,2.9) node[left]{$y$};
			\draw (0,0) node[below left]{$0$};
			\draw[fill] (-1,0) circle (0.5pt)node[below left]{$ -1 $};
			\draw[fill] (1,0) circle (0.5pt)node[below]{$ 1$};
			\draw[fill] (3,0) circle (0.5pt)node[below right]{$ 3$};
			%			\draw[dashed] (-1,0)--(-1,-1) --(0,-1); 
			%			\draw[dashed](1,0)--(1,1)--(0,1);
			%			\draw[dashed](3,0)--(3,3)--(0,3);
			\draw[line width=1.2pt,smooth,samples=100,domain=-1.65:3.5] plot(\x,{0.33*(\x)^3-(\x)^2-0.333*(\x)+1});		
			%\draw[line width=1.2pt,smooth,samples=100,domain=-3.3:2.8] plot(\x,{0.75*(\x)^2+0.5*\x-1});
		\end{tikzpicture}	
	}
	\loigiai{
		Ta có $g'(x)=f'\left(x+m\right)$. Vì $ y=f'(x)$ liên tục trên $\mathbb{R}$ nên $g'(x)=f'\left(x+m\right)$ cũng liên tục trên $\mathbb{R}$. Căn cứ vào đồ thị hàm số $ y=f'(x)$ ta thấy\\
		$g'(x)<0\Leftrightarrow{f}'\left(x+m\right)<0$ $\Leftrightarrow\hoac{
			& x+m<-1\\ 
			& 1<x+m<3} \Leftrightarrow \hoac{
			& x<-1-m\\ 
			& 1-m<x<3-m.}$\\
		Hàm số $ g(x)=f\left(x+m\right)$ nghịch biến trên khoảng $\left(1\,;\,2\right)$
		$\Leftrightarrow \hoac{
			& 2\le-1-m\\ 
			&\hoac{
				& 3-m\ge 2\\ 
				& 1-m\le 1}} \Leftrightarrow \hoac{
			& m\le-3\\ 
			& 0\le m\le 1.}$\\
		Mà $ m$ là số nguyên thuộc đoạn $\left[-5\,;\,5\right]$ nên ta có $ S=\left\{-5;-4;-3;0;1\right\}$.\\
		Vậy $ S$ có $5$ phần tử.}
\end{ex}

\begin{ex}%[2D1G1-3]%Câu 2
	[Chuyên Nguyễn Bỉnh Khiêm-Quảng Nam-2020] Cho hàm số $ y=f(x)$ có đạo hàm trên $\mathbb{R}$ và bảng xét dấu đạo hàm như hình vẽ sau
	\begin{center}
		\begin{tikzpicture}
			\tkzTabInit[lgt=1.2,espcl=2.5,nocadre]
			{$x$/0.7, $f'(x)$ /2.5} % first column
			{$-\infty$, $-10$,$-2$, $3$,$8$, $+\infty$} % first row
			\tkzTabLine { ,+,z,-,z,+,z,-,z,+, } % second row
			%				\tkzTabLine {,-,z,+,t,+,} % third row
			%				\tkzTabLine {,+,d,-,z,+,} % last row
		\end{tikzpicture}
	\end{center}
	Có bao nhiêu số nguyên $ m$ để hàm số $ y=f\left(x^3+4x+m\right)$ nghịch biến trên khoảng $\left(-1;1\right)$?
	\choice
	{$ 3$}
	{$ 0$}
	{\True $ 1$}
	{$ 2$}
	\loigiai
	{
		Đặt $ t=x^3+4x+m\Rightarrow{t}'=3x^2+4$ nên $ t$ đồng biến trên $\left(-1;1\right)$ và $ t\in\left(m-5;m+5\right)$.\\
		Yêu cầu bài toán trở thành tìm $ m$ để hàm số $ f(t)$ nghịch biến trên khoảng $\left(m-5;m+5\right)$.\\
		Dựa vào bảng biến thiên ta được $\heva{
			& m-5\ge-2\\ 
			& m+5\le 8} \Leftrightarrow \heva{
			& m\ge 3\\ 
			& m\le 3} \Leftrightarrow m=3$.}
\end{ex}

\begin{ex}%[2D1G1-3]%Câu 3
	[Chuyên ĐH Vinh-Nghệ An-2020]
	\immini{
		Cho hàm số $ f(x)$ có đạo hàm trên $\mathbb{R}$và $ f(1)=1$. Đồ thị hàm số $ y=f'(x)$ như hình bên. Có bao nhiêu số nguyên dương $ a$ để hàm số $ y=\left| 4f\left(\sin x\right)+\cos 2x-a\right|$ nghịch biến trên $\left(0;\dfrac{\pi}{2}\right)$?
		\choice
		{$ 2$}
		{\True $ 3$}
		{Vô số}
		{$ 5$}}{
		\begin{tikzpicture}[scale=0.9,font=\footnotesize, line join=round, line cap=round, >=stealth] %Đường cong bậc 3
			\draw[thick, ->] (-2.5,0)--(3,0);
			\draw[thick, ->] (0,-2.8)--(0,2.8);
			\draw (3.1,0) node[below] {$x$};
			\draw (0,2.9) node[left]{$y$};
			\draw (0,0) node[below left]{$0$};
			\draw[fill] (-1,0) circle (0.5pt)node[below]{$ -1 $};
			\draw[fill] (1,0) circle (0.5pt)node[above]{$ 1$};
			%	\draw[fill] (3,0) circle (0.5pt)node[below right]{$ 3$};
			\draw[dashed] (-1,0)--(-1,1); 
			\draw[dashed](1,0)--(1,-1);
			%			\draw[dashed](3,0)--(3,3)--(0,3);
			\draw[line width=1.2pt,smooth,samples=100,domain=-2:2] plot(\x,{.8*(\x)^3+0*(\x)^2-1.8*(\x)});		
			%\draw[line width=1.2pt,smooth,samples=100,domain=-3.3:2.8] plot(\x,{0.75*(\x)^2+0.5*\x-1});
			\draw (2.0,2.8) node[left]{$y=f'(x)$};
		\end{tikzpicture}	
	}
	\loigiai
	{		Đặt $g(x)=\left| 4f\left(\sin x\right)+\cos 2x-a\right|\Rightarrow g(x)=\sqrt{\left[4f\left(\sin x\right)+\cos 2x-a\right]^2}$ .\\
		$\Rightarrow{g}'(x)=\dfrac{\left[4\cos x\cdot f'\left(\sin x\right)-2\sin 2x\right]\left[4f\left(\sin x\right)+\cos 2x-a\right]}{\sqrt{\left[4f\left(\sin x\right)+\cos 2x-a\right]^2}}$.\\
		Ta có $ 4\cos x\cdot f'\left(\sin x\right)-2\sin 2x=4\cos x\left[f'\left(\sin x\right)-\sin x\right]$.\\
		Với $ x\in\left(0;\dfrac{\pi}{2}\right)$ thì $\cos x>0,\sin x\in\left(0;1\right)\Rightarrow{f}'\left(\sin x\right)-\sin x<0$.\\
		Hàm số $ g(x)$ nghịch biến trên $\left(0;\dfrac{\pi}{2}\right)$ khi $ 4f\left(\sin x\right)+\cos 2x-a\ge 0,\forall x\in\left(0;\dfrac{\pi}{2}\right)$\\
		$\Leftrightarrow 4f\left(\sin x\right)+1-2\sin^2x\ge a,\forall x\in\left(0;\dfrac{\pi}{2}\right)$.\\
		Đặt $ t=\sin x$ được $ 4f(t)+1-2t^2\ge a,\forall t\in\left(0;1\right)$ (*).\\
		Xét $ h(t)=4f(t)+1-2t^2\Rightarrow{h}'(t)=4f'(t)-4t=4\left[f'(t)-1\right]$.\\
		Với $ t\in\left(0;1\right)$ thì $h'(t)<0\Rightarrow h(t)$ nghịch biến trên $\left(0;1\right)$.\\
		Do đó (*) $\Leftrightarrow a\le h(1)=4f(1)+1-2.1^2=3$.\\
		Vậy có $3$ giá trị nguyên dương của a thỏa mãn.}
\end{ex}


\begin{ex}%[2D1G1-3]%Câu 4
	[Chuyên Quang Trung-2020]
	\immini{
		Cho hàm số $ y=f(x)$ có đạo hàm liên tục trên $\mathbb{R}$ và có đồ thị $ y=f'(x)$ như hình vẽ. Đặt $ g(x)=f\left(x-m\right)-\dfrac{1}{2}{\left(x-m-1\right)^2}+2019$, với $ m$ là tham số thực. Gọi $ S$ là tập hợp các giá trị nguyên dương của $ m$ để hàm số $ y=g(x)$ đồng biến trên khoảng $\left(5;6\right)$. Tổng tất cả các phần tử trong $ S$ bằng
		\choice
		{$ 4$}
		{$ 11$}
		{\True $ 14$}
		{$ 20$}}{
		\begin{tikzpicture}[scale=0.9,font=\footnotesize, line join=round, line cap=round, >=stealth] %Đường cong bậc 3
			\draw[style=help lines,step=1] (-2.5,-3) grid (3,3.5);
			\draw[thick, ->] (-2.5,0)--(3.5,0);
			\draw[thick, ->] (0,-2.8)--(0,2.8);
			\draw (3.6,0) node[below] {$x$};
			\draw (0,3) node[above left]{$y$};
			\draw (0,0) node[below left]{$0$};
			%\draw[fill] (-1,0) circle (0.5pt)node[below]{$ -1 $};
			\draw[fill] (1,0) circle (0.5pt)node[below left]{$ 1$};
			%	\draw[fill] (3,0) circle (0.5pt)node[below right]{$ 3$};
			\draw[dashed] (-1,0)--(-1,-2) --(2,-2)--(2,0); 
			\draw[dashed](3,0)--(3,2) --(0,2);
			\draw (-1,-2) circle (2pt);
			\draw (3,2) circle (2pt);
			%			\draw[dashed](3,0)--(3,3)--(0,3);
			\draw[line width=1.2pt,smooth,samples=100,domain=-1.1:3.1] plot(\x,{1*(\x)^3-3*(\x)^2-0*(\x)+2});		
			%\draw[line width=1.2pt,smooth,samples=100,domain=-3.3:2.8] plot(\x,{0.75*(\x)^2+0.5*\x-1});
			%\draw (2.0,2.8) node[left]{$y=f'(x)$};
		\end{tikzpicture}	
	}
	\loigiai
	{
		Xét hàm số $ g(x)=f\left(x-m\right)-\dfrac{1}{2}{\left(x-m-1\right)^2}+2019$.\\
		$g'(x)=f'\left(x-m\right)-\left(x-m-1\right)$.\\
		Xét phương trình $g'(x)=0. \quad \quad (1)$\\
		Đặt $ x-m=t$, phương trình $(1)$ trở thành $f'(t)-\left(t-1\right)=0\Leftrightarrow{f}'(t)=t-1. \quad (2)$\\
		Nghiệm của phương trình $(2)$ là hoành độ giao điểm của hai đồ thị hàm số $ y=f'(t)$ và $ y=t-1$.\\
		Ta có đồ thị các hàm số $ y=f'(t)$ và $ y=t-1$ như sau
		\begin{center}
			\begin{tikzpicture}[scale=0.9,font=\footnotesize, line join=round, line cap=round, >=stealth] %Đường cong bậc 3
				\draw[style=help lines,step=1] (-2.5,-3) grid (3,3.5);
				\draw[thick, ->] (-2.5,0)--(3.5,0);
				\draw[thick, ->] (0,-2.8)--(0,2.8);
				\draw (3.6,0) node[below] {$x$};
				\draw (0,3) node[above left]{$y$};
				\draw (0,0) node[below left]{$0$};
				%\draw[fill] (-1,0) circle (0.5pt)node[below]{$ -1 $};
				\draw[fill] (1,0) circle (0.5pt)node[below left]{$ 1$};
				%	\draw[fill] (3,0) circle (0.5pt)node[below right]{$ 3$};
				\draw[dashed] (-1,0)--(-1,-2) --(2,-2)--(2,0); 
				\draw[dashed](3,0)--(3,2) --(0,2);
				\draw (-1,-2) circle (2pt);
				\draw (3,2) circle (2pt);
				%			\draw[dashed](3,0)--(3,3)--(0,3);
				\draw[line width=1.2pt,smooth,samples=100,domain=-1.1:3.1] plot(\x,{1*(\x)^3-3*(\x)^2-0*(\x)+2});		
				%\draw[line width=1.2pt,smooth,samples=100,domain=-3.3:2.8] plot(\x,{0.75*(\x)^2+0.5*\x-1});
				%\draw (2.0,2.8) node[left]{$y=f'(x)$};
				\draw (-2,-3)--(4,3);
			\end{tikzpicture}
		\end{center}
		Căn cứ đồ thị các hàm số ta có phương trình $(2)$ có nghiệm là $\hoac{
			& t=-1\\ 
			& t=1\\ 
			& t=3} \Rightarrow \hoac{
			& x=m-1\\ 
			& x=m+1\\ 
			& x=m+3.}$\\
		Ta có bảng biến thiên của $ y=g(x)$
		\begin{center}
			\begin{tikzpicture}
				\tkzTabInit[lgt=1,espcl=2.5,nocadre]
				{$x$ /0.8, $y'$ /0.8,$y$ /2.5}
				{$-\infty$ , $m-1$,$m+1$,$m+3$, $+\infty$}
				\tkzTabLine{,+,0,-,0,+,0,-,}
				\tkzTabVar{-/$ +\infty$ ,+/$ $, -/$ $,+/$ $,-/$+\infty $}
			\end{tikzpicture}
		\end{center}
		Để hàm số $ y=g(x)$ đồng biến trên khoảng $\left(5;6\right)$ cần $\hoac{
			&\heva{
				& m-1\le 5\\ 
				& m+1\ge 6}\\ 
			& m+3\le 5}\Leftrightarrow\hoac{
			& 5\le m\le 6\\ 
			& m\le 2.}$\\
		Vì $ m\in\mathbb{N}^*\Rightarrow m$ nhận các giá trị $ 1;\,2;\,5;\,6\Rightarrow S=14$.}
\end{ex}

\begin{ex}%[2D1G1-3]%Câu 5
	[Sở Hà Nội-Lần 2-2020] 
	\immini{
		Cho hàm số $y=a{x^4}+b{x^3}+c{x^2}+dx+e,\,\,a\ne 0$. Hàm số $y=f'(x)$ có đồ thị như hình vẽ bên. 
		Gọi S là tập hợp tất cả các giá trị nguyên thuộc khoảng $\left(-6;6\right)$ của tham số $m$ để hàm số $g(x)=f\left(3-2x+m\right)+x^2-\left(m+3\right)x+2m^2$ nghịch biến trên $\left(0;1\right)$. Khi đó, tổng giá trị các phần tử của S là
		\choice
		{$12$}
		{\True $9$}
		{$6$}
		{$15$}}{
		\begin{tikzpicture}[scale=0.7,font=\footnotesize, line join=round, line cap=round, >=stealth] %Đường cong bậc 3
			%	\draw[style=help lines,step=1] (-2.5,-3) grid (3,3.5);
			\draw[thick, ->] (-4.5,0)--(6.5,0);
			\draw[thick, ->] (0,-2.8)--(0,2.8);
			\draw (6.6,0) node[below] {$x$};
			\draw (0,3) node[above left]{$y$};
			\draw (0,0) node[below left]{$0$};
			\draw[fill] (-2,0) circle (0.5pt)node[below]{$ -2 $};
			\draw[fill] (4,0) circle (0.5pt)node[above]{$ 4$};
			\draw[fill] (0,1) circle (0.5pt)node[right]{$ 1 $};
			\draw[fill] (0,-2) circle (0.5pt)node[left]{$ -2$};
			%	\draw[fill] (3,0) circle (0.5pt)node[below right]{$ 3$};
			\draw[dashed] (-2,0)--(-2,1) --(0,1); 
			\draw[dashed](4,0)--(4,-2) --(0,-2);
			%			\draw[dashed](3,0)--(3,3)--(0,3);
			\draw[line width=1.2pt,smooth,samples=100,domain=-3.8:5.5] plot(\x,{0.0714*(\x)^3-0.1423*(\x)^2-1.0714*(\x)});		
			%\draw[line width=1.2pt,smooth,samples=100,domain=-3.3:2.8] plot(\x,{0.75*(\x)^2+0.5*\x-1});
			%\draw (2.0,2.8) node[left]{$y=f'(x)$};
		\end{tikzpicture}	
	}
	\loigiai
	{
		Xét $g'(x)=-2f'\left(3-2x+m\right)+2x-\left(m+3\right)$.\\
		Xét phương trình $g'(x)=0$, đặt $t=3-2x+m$ thì phương trình trở thành\\ $-2\cdot \left[f'(t)-\dfrac{-t}{2}\right]=0\Leftrightarrow\hoac{
			& t=-2\\ 
			& t=4\\ 
			& t=0.}$ \\
		Từ đó, $g'(x)=0\Leftrightarrow{x_1}=\dfrac{5+m}{2},\,x_2=\dfrac{m+3}{2},x_3=\dfrac{-1+m}{2}$.\\
		Lập bảng xét dấu, đồng thời lưu ý nếu $x>x_1$ thì $t<t_1$ nên $f(x)>0$. Và các dấu đan xen nhau do các nghiệm đều làm đổi dấu đạo hàm nên suy ra $g'(x)\le 0\Leftrightarrow x\in\left[x_2;{x_1}\right]\cup\left(-\infty ;{x_3}\right]$.\\
		Vì hàm số nghịch biến trên $\left(0;1\right)$ nên \\
		$g'(x)\le 0,\,\forall x\in\left(0;1\right)$ từ đó suy ra $\hoac{
			&\dfrac{3+m}{2}\le 0<1\le\dfrac{5+m}{2}\\ 
			& 1\le\dfrac{-1+m}{2}.}$ \\
		và giải ra các giá trị nguyên thuộc $\left(-6;6\right)$ của $m$ là $-3$; $3$; $4$; $5$. }
\end{ex}

\begin{ex}%[2D1G1-3]%Câu 6
	[Chuyên Quang Trung-Bình Phước-Lần 2-2020]
	\immini{
		Cho hàm số $ y=f(x)$ có đạo hàm liên tục trên $\mathbb{R}$ và có đồ thị $ y=f'(x)$ như hình vẽ bên. Đặt $ g(x)=f\left(x-m\right)-\dfrac{1}{2}{\left(x-m-1\right)^2}+2019$, với $ m$ là tham số thực. Gọi $ S$ là tập hợp các giá trị nguyên dương của $ m$ để hàm số $ y=g(x)$ đồng biến trên khoảng $\left(5;6\right)$. Tổng tất cả các phần tử trong $ S$ bằng
		\choice
		{$ 4$}
		{$ 11$}
		{\True $ 14$}
		{$ 20$}}{
		\begin{tikzpicture}[scale=0.9,font=\footnotesize, line join=round, line cap=round, >=stealth] %Đường cong bậc 3
			\draw[thick, ->] (-2.5,0)--(3.7,0);
			\draw[thick, ->] (0,-2.8)--(0,2.8);
			\draw (3.9,0) node[below] {$x$};
			\draw (0,2.9) node[left]{$y$};
			\draw (0,0) node[below left]{$0$};
			\draw[fill] (-1,0) circle (0.5pt)node[above]{$ -1 $};
			\draw[fill] (1,0) circle (0.5pt)node[below]{$ 1$};
			\draw[fill] (3,0) circle (0.5pt)node[below]{$ 3$};
			\draw[fill] (2,0) circle (0.5pt)node[above]{$ 2$};
			\draw[fill] (0,2) circle (0.5pt)node[above left]{$ 2$};
			\draw[fill] (0,-2) circle (0.5pt)node[below left]{$ -2$};
			\draw[dashed] (-1,0)--(-1,-2)--(2,-2)--(2,0); 
			\draw[dashed](3,0)--(3,2)--(0,2);
			%			\draw[dashed](3,0)--(3,3)--(0,3);
			\draw[line width=1.2pt,smooth,samples=100,domain=-1.1:3.1] plot(\x,{1*(\x)^3-3*(\x)^2-0*(\x)+2});		
			%\draw[line width=1.2pt,smooth,samples=100,domain=-3.3:2.8] plot(\x,{0.75*(\x)^2+0.5*\x-1});
			%	\draw (2.0,2.8) node[left]{$y=f'(x)$};
	\end{tikzpicture}	}
	\loigiai
	{
		Ta có $g'(x)=f'\left(x-m\right)-\left(x-m-1\right)$.\\
		Cho $g'(x)=0\Leftrightarrow{f}'\left(x-m\right)=x-m-1$.\\
		Đặt $ x-m=t\Rightarrow f'(t)=t-1$\\
		Khi đó nghiệm của phương trình là hoành độ giao điểm của đồ thị hàm số $ y=f'(t)$ và và đường thẳng $ y=t-1$.
		\begin{center}
			\begin{tikzpicture}[scale=0.9,font=\footnotesize, line join=round, line cap=round, >=stealth] %Đường cong bậc 3
				\draw[thick, ->] (-2.5,0)--(3.7,0);
				\draw[thick, ->] (0,-2.8)--(0,2.8);
				\draw (3.9,0) node[below] {$x$};
				\draw (0,2.9) node[left]{$y$};
				\draw (0,0) node[below left]{$0$};
				\draw[fill] (-1,0) circle (0.5pt)node[above]{$ -1 $};
				\draw[fill] (1,0) circle (0.5pt)node[below]{$ 1$};
				\draw[fill] (3,0) circle (0.5pt)node[below]{$ 3$};
				\draw[fill] (2,0) circle (0.5pt)node[above]{$ 2$};
				\draw[fill] (0,2) circle (0.5pt)node[above left]{$ 2$};
				\draw[fill] (0,-2) circle (0.5pt)node[below left]{$ -2$};
				\draw[dashed] (-1,0)--(-1,-2)--(2,-2)--(2,0); 
				\draw[dashed](3,0)--(3,2)--(0,2);
				%			\draw[dashed](3,0)--(3,3)--(0,3);
				\draw[line width=1.2pt,smooth,samples=100,domain=-1.1:3.1] plot(\x,{1*(\x)^3-3*(\x)^2-0*(\x)+2});		
				%\draw[line width=1.2pt,smooth,samples=100,domain=-3.3:2.8] plot(\x,{0.75*(\x)^2+0.5*\x-1});
				%	\draw (2.0,2.8) node[left]{$y=f'(x)$};
				\coordinate (a) at ($(-1,-2)!1.2!(3,2)$);
				\coordinate (b) at ($(-1,-2)!-.2!(3,2)$);
				\draw[line width=1.2pt,smooth] (a)--(b);
			\end{tikzpicture}
		\end{center}
		Dựa vào đồ thị hàm số ta có được $f'(t)=t-1\Leftrightarrow\hoac{
			& t=-1\\ 
			& t=1\\ 
			& t=3.} $ \\
		Bảng xét dấu của $g'(t)$
		\begin{center}
			\begin{tikzpicture}
				\tkzTabInit[lgt=1.2,espcl=2.5,nocadre]
				{$t$/1, $g'(x)$ /.8} % first column
				{$-\infty$, $-1$,$1$, $3$, $+\infty$} % first row
				\tkzTabLine { ,-,0,+,0,-,0,+, } % second row
				%				\tkzTabLine {,-,z,+,t,+,} % third row
				%				\tkzTabLine {,+,d,-,z,+,} % last row
			\end{tikzpicture}
		\end{center}
		Từ bảng xét dấu ta thấy hàm số $ g(t)$ đồng biến trên khoảng $\left(-1;1\right)$ và $\left(3;+\infty\right)$.\\
		Hay $\hoac{
			&-1<t<1\\ 
			& t>3}\Leftrightarrow\hoac{
			&-1<x-m<1\\ 
			& x-m>3} \Leftrightarrow\hoac{
			& m-1<x<m+1\\ 
			& x>m+3.}$\\
		Để hàm số $ g(x)$ đồng biến trên khoảng $\left(5;6\right)$ thì $\hoac{
			& m-1\le 5<6\le m+1\\ 
			& m+3\le 5<6} \Leftrightarrow\hoac{
			& 5\le m\le 6\\ 
			& m\le 2.}$\\
		Vì $ m$ là các số nguyên dương nên $ S=\left\{ 1;2;5;6\right\}$.\\
		Vậy tổng tất cả các phần tử của $ S$ là $ 1+2+5+6=14$.}
\end{ex}

\begin{ex}%[2D1G1-3]%Câu 7
	\immini{
		Cho hàm số $ y=f(x)$ liên tục có đạo hàm trên $\mathbb{R}$. Biết hàm số $ f'(x)$ có đồ thị cho như hình vẽ bên. Có bao nhiêu giá trị nguyên của $ m$ thuộc $\left[-2019;2019\right]$ để hàm só $ g(x)=f\left(2019^x\right)-mx+2$ đồng biến trên $\left[0;1\right]$.
		\choice
		{$ 2028$}
		{$ 2019$}
		{$ 2011$}
		{\True $ 2020$}}{
		\begin{tikzpicture}[scale=0.9,font=\footnotesize, line join=round, line cap=round, >=stealth] %Đường cong bậc 3
			\draw[thick, ->] (-3.5,0)--(2.5,0);
			\draw[thick, ->] (0,-2.8)--(0,2.8);
			\draw (2.7,0) node[below] {$x$};
			\draw (0,2.9) node[left]{$y$};
			\draw (0,0) node[below left]{$0$};
			%	\draw[fill] (-1,0) circle (0.5pt)node[above]{$ -1 $};
			\draw[fill] (1,0) circle (0.5pt)node[below right]{$ 1$};
			%		\draw[fill] (3,0) circle (0.5pt)node[below]{$ 3$};
			%		\draw[fill] (2,0) circle (0.5pt)node[above]{$ 2$};
			%		\draw[fill] (0,2) circle (0.5pt)node[above left]{$ 2$};
			%		\draw[fill] (0,-2) circle (0.5pt)node[below left]{$ -2$};
			%		\draw[dashed] (-1,0)--(-1,-2)--(2,-2)--(2,0); 
			%		\draw[dashed](3,0)--(3,2)--(0,2);
			\draw[line width=1.2pt,smooth,samples=100,domain=-3.28:1.32] plot(\x,{0.667*(\x)^3+2*(\x)^2-0.667*(\x)-2});		
			%\draw[line width=1.2pt,smooth,samples=100,domain=-3.3:2.8] plot(\x,{0.75*(\x)^2+0.5*\x-1});
			%	\draw (2.0,2.8) node[left]{$y=f'(x)$};
	\end{tikzpicture}	}
	\loigiai{
		Ta có $ g'(x)=2019^x\ln 2019\cdot f'\left(2019^x\right)-m$.\\
		Ta lại có hàm số $ y=2019^x$ đồng biến trên $\left[0;1\right]$.\\
		Với $ x\in\left[0;1\right]$ thì $2019^x\in\left[1;2019\right]$ mà hàm $ y=f'(x)$ đồng biến trên $\left(1;+\infty\right)$ nên hàm $ y=f'\left(2019^x\right)$ đồng biến trên $\left[0;1\right]$.\\
		Mà $2019^x\ge 1;f'\left(2019^x\right)>0\,\forall\,x\in\left[0;1\right]$ nên hàm $ h(x)=2019^x\ln 2019\cdot f'\left(2019^x\right)$ đồng biến trên $\left[0;1\right]$.\\
		Hay $ h(x)\ge h(0)=0,\forall\,x\in\left[0;1\right]$.\\
		Do vậy hàm số $ g(x)$ đồng biến trên đoạn $\left[0;1\right]$$\Leftrightarrow g'(x)\ge 0,\forall\,x\in\left[0;1\right]$\\
		$\Leftrightarrow m\le{2019^x}\ln 2019.f'\left(2019^x\right),\forall\,x\in\left[0;1\right]$ $\Leftrightarrow m\le\underset{x\in\left[0;1\right]}{\min}\,h(x)=h(0)=0$\\
		Vì $ m$ nguyên và $ m\in\left[-2019;2019\right]\Rightarrow $có $ 2020$ giá trị $ m$ thỏa mãn yêu cầu bài toán.}
\end{ex}

\begin{ex}%[2D1G1-3]%Câu 8
	\immini{
		Cho hàm số $y=f(x)$ có đồ thị $f'(x)\,$ như hình vẽ. Có bao nhiêu giá trị nguyên $m\in\left(-2020\,;\,2020\right)$ để hàm số $g(x)=f\left(2x-3\right)\,-\ln \left(1+x^2\right)-2mx$ đồng biến trên $\left(\dfrac{1}{2};2\right)$?
		\choice
		{$ 2020$}
		{\True $ 2019$}
		{$ 2021$}
		{$ 2018$}}{
		\begin{tikzpicture}[scale=0.9,font=\footnotesize, line join=round, line cap=round, >=stealth] %Đường cong bậc 3
			\draw[thick, ->] (-2.5,0)--(2.5,0);
			\draw[thick, ->] (0,-1.8)--(0,5.8);
			\draw (2.7,0) node[below] {$x$};
			\draw (0,5.9) node[left]{$y$};
			\draw (0,0) node[below left]{$0$};
			\draw[fill] (-2,0) circle (0.5pt)node[below]{$ -2 $};
			\draw[fill] (1,0) circle (0.5pt)node[below]{$ 1$};
			\draw[fill] (-1,0) circle (0.5pt)node[below]{$-1$};
			\draw[fill] (0,4) circle (0.5pt)node[above left]{$ 2$};
			%		\draw[fill] (0,2) circle (0.5pt)node[above left]{$ 2$};
			%		\draw[fill] (0,-2) circle (0.5pt)node[below left]{$ -2$};
			\draw[dashed] (-2,0)--(-2,4)--(1,4)--(1,0); 
			%		\draw[dashed](3,0)--(3,2)--(0,2);
			\draw[line width=1.2pt,smooth,samples=100,domain=-2.1:2.1] plot(\x,{-1*(\x)^3+0*(\x)^2+3*(\x)+2});		
			%\draw[line width=1.2pt,smooth,samples=100,domain=-3.3:2.8] plot(\x,{0.75*(\x)^2+0.5*\x-1});
			%	\draw (2.0,2.8) node[left]{$y=f'(x)$};
	\end{tikzpicture}	}
	\loigiai{
		Ta có $g'(x)=2f'\left(2x-3\right)-\dfrac{2x}{1+x^2}-2m$.\\
		Hàm số $ g(x)$ đồng biến trên $\left(\dfrac{1}{2};2\right)$ khi và chỉ khi \\
		$g'(x)\ge 0,\,\,\forall x\in\left(-1;\,2\right)$\\
		$\Leftrightarrow m\le{f}'\left(2x-3\right)-\dfrac{x}{1+x^2},\,\,\forall x\in\left(\dfrac{1}{2};2\right)$\\
		$\Leftrightarrow m\le\underset{x\in\left[\dfrac{1}{2};2\right]}{\min}\,\left[f'\left(2x-3\right)-\dfrac{x}{1+x^2}\right]$. \, \,  $(1)$\\
		Đặt $ t=2x-3$, khi đó $ x\in\left(\dfrac{1}{2};2\right)\Leftrightarrow t\in\left(-2;\,1\right)$.\\
		Từ đồ thị hàm $f'(x)$ suy ra $f'(t)\ge 0,\,\,\forall t\in\left(-2;1\right)$ và $f'(t)=0$ khi $ t=-1$.\\
		Tức là $f'\left(2x-3\right)\ge 0,\,\,\forall x\in\left(\dfrac{1}{2};\,2\right)$$\Rightarrow\underset{x\in\left[\dfrac{1}{2};2\right]}{\min}\,f'\left(2x-3\right)=0$ khi $ x=1$. $(2)$\\
		Xét hàm số $ h(x)=-\dfrac{x}{1+x^2}$ trên khoảng $\left(\dfrac{1}{2};\,2\right)$.\\
		Ta có $h'(x)=\dfrac{x^2-1}{\left(1+x^2\right)^2}$ và\\
		$h'(x)=0\Leftrightarrow{x^2}-1=0\Leftrightarrow x=\pm 1$.\\
		Bảng biến thiên của hàm số $ h(x)$ trên $\left(\dfrac{1}{2};\,2\right)$ như sau
		\begin{center}
			\begin{tikzpicture}
				\tkzTabInit[lgt=1.2,espcl=2.5,nocadre]
				{$x$ /0.7, $h'(x)$ /0.7,$h(x)$ /2.5}
				{$\dfrac{1}{2}$ , $1$,$2$}
				\tkzTabLine{,-,0,+,}
				\tkzTabVar{+/$  $ ,-/$ \-\dfrac{1}{2} $, +/$ $}
			\end{tikzpicture}
		\end{center}
		Từ bảng biến thiên suy ra $ h(x)\ge-\dfrac{1}{2}$$\Rightarrow\underset{x\in\left[\dfrac{1}{2};2\right]}{\min}\,h(x)=-\dfrac{1}{2}$ khi $ x=1$. \, \,  $(3)$\\
		Từ $(1)$, $(2)$ và $(3)$ suy ra $ m\le-\dfrac{1}{2}$.\\
		Kết hợp với $ m\in\mathbb{Z}$, $ m\in\left(-2020;\,2020\right)$ thì $ m\in\left\{-2019;\,-201;\ldots ;-2;-1\right\}$.\\
		Vậy có tất cả $ 2019$ giá trị $ m$ cần tìm.}
\end{ex}

\begin{ex}%[2D1G1-3]%Câu 9
	Cho hàm số $ f(x)$ liên tục trên $\mathbb{R}$ và có đạo hàm $f'(x)=x^2\left(x-2\right)\left(x^2-6x+m\right)$ với mọi $ x\in\mathbb{R}$. Có bao nhiêu số nguyên $ m$ thuộc đoạn $\left[-2020;2020\right]$ để hàm số $ g(x)=f\left(1-x\right)$ nghịch biến trên khoảng $\left(-\infty ;-1\right)$?
	\choice
	{$ 2016$}
	{$ 2014$}
	{\True $ 2012$}
	{$ 2010$}
	\loigiai{
		Ta có \\
		$g'(x)=f'\left(1-x\right)=-\left(1-x\right)^2\left(-x-1\right)\left[\left(1-x\right)^2-6\left(1-x\right)+m\right]$
		$=\left(x-1\right)^2\left(x+1\right)\left(x^2+4x+m-5\right)$.\\
		Hàm số $ g(x)$ nghịch biến trên khoảng $\left(-\infty ;-1\right)$\\
		$\Leftrightarrow{g}'(x)\le 0,\forall x<-1$ $(*)$, (dấu \lq\lq $=$\rq\rq \, xảy ra tại hữu hạn điểm).\\
		Với $ x<-1$ thì $\left(x-1\right)^2>0$ và $ x+1<0$ nên\\
		$(*)$ $\Leftrightarrow{x^2}+4x+m-5\ge 0,\forall x<-1 \Leftrightarrow m\ge-x^2-4x+5,\forall x<-1$.\\
		Xét hàm số $ y=-x^2-4x+5$ trên khoảng $\left(-\infty ;-1\right)$, ta có bảng biến thiên
		\begin{center}
			\begin{tikzpicture}
				\tkzTabInit[lgt=1.8,espcl=2.3]
				{$x$ /1.2, $y'$ /1.2,$y$ /2}
				{$-\infty$ , $-2$,$-1$}
				\tkzTabLine{,+,0,-,}
				\tkzTabVar{-/$ -\infty $ ,+/$9 $, -/$ 8$}
			\end{tikzpicture}
		\end{center}
		Từ bảng biến thiên suy ra $ m\ge 9$.\\
		Kết hợp với $ m$ thuộc đoạn $\left[-2020;2020\right]$ và $ m$ nguyên nên $ m\in\left\{ 9;10;11;\ldots ;2020\right\}$.\\
		Vậy có $ 2012$ số nguyên $ m$ thỏa mãn đề bài.}
\end{ex}

\begin{ex}%[2D1G1-3]%Câu 10
	\immini{
		Cho hàm số $f(x)$ xác định và liên tục trên $ R$. Hàm số $y=f'(x)$ liên tục trên $\mathbb{R}$ và có đồ thị như hình vẽ bên.
		Xét hàm số $g(x)=f\left(x-2m\right)+\dfrac{1}{2}{\left(2m-x\right)^2}+2020$, với $ m$ là tham số thực. Gọi $ S$ là tập hợp các giá trị nguyên dương của $ m$ để hàm số $ y=g(x)$ nghịch biến trên khoảng $\left(3;4\right)$. Hỏi số phần tử của $ S$ bằng bao nhiêu?
		\choice
		{$4$}
		{\True $2$}
		{$3$}
		{Vô số}}
	{
		\begin{tikzpicture}[scale=0.7,>=stealth, font=\footnotesize, line join=round, line cap=round]
			\def\xmin{-3.5} \def\xmax{4.5}
			\def\ymin{-5.2} \def\ymax{4}
			\clip(\xmin,\ymin) rectangle (\xmax,\ymax);
			\draw[->] (\xmin,0)--(\xmax,0) node [below]{$x$};
			\draw[->] (0,\ymin)--(0,\ymax) node [left]{$y$};
			\node at (0,0) [below left]{$O$};
			\path
			(-3.1,3.7) coordinate (A)
			(-3,3) coordinate (B)
			(0,-2) coordinate (C)
			(0.65,-2) coordinate (D)
			(1,-1) coordinate (E)
			(3,-3) coordinate (F)
			(3.4,-5) coordinate (G);
			\draw[smooth]
			(A)..controls +(-88:0.1) and +(93:.1)..
			(B)..controls +(-87:0.3) and +(-100:8.5)..
			(C)..controls +(75:.8) and +(180:.1)..
			(D)..controls +(0:.1) and +(-105:.3)..
			(E)..controls +(70:2) and +(97:0.4)..
			(F)..controls +(-80:.1) and +(90:0.3)..
			(G);
			\draw[dashed] 
			(-3,0)node[below]{$-3$}|-(0,3)node[right]{$3$}
			(1,0)node[above]{$1$}|-(0,-1)node[left]{$-1$}
			(3,0)node[above]{$3$}|-(0,-3)node[below right]{$-3$};
			\fill 
			(0,-2) circle(1.5pt)
			(-3,3) circle(1.5pt)
			(3,-3) circle(1.5pt)
			(1,-1) circle(1.5pt);
			\node at (2.1,-4) {$y=f'(x)$};
		\end{tikzpicture}
	}
	\loigiai{
		Ta có $g'(x)=f'\left(x-2m\right)-\left(2m-x\right)$.		Đặt $h(x)=f'(x)-\left(-x\right)$.\\
		Từ đồ thị hàm số $y=f'(x)$ và đồ thị hàm số $y=-x$ trên hình vẽ suy ra \\
		$h(x)\le 0\Leftrightarrow f'(x)\le-x\Leftrightarrow\hoac{
			&-3\le x\le 1\\ 
			& x\ge 3.}$ 
		\begin{center}
			\begin{tikzpicture}[scale=0.7,>=stealth, font=\footnotesize, line join=round, line cap=round]
				\def\xmin{-3.5} \def\xmax{4.5}
				\def\ymin{-5.2} \def\ymax{4}
				\clip(\xmin,\ymin) rectangle (\xmax,\ymax);
				\draw[->] (\xmin,0)--(\xmax,0) node [below]{$x$};
				\draw[->] (0,\ymin)--(0,\ymax) node [left]{$y$};
				\node at (0,0) [below left]{$O$};
				\path
				(-3.1,3.7) coordinate (A)
				(-3,3) coordinate (B)
				(0,-2) coordinate (C)
				(0.65,-2) coordinate (D)
				(1,-1) coordinate (E)
				(3,-3) coordinate (F)
				(3.4,-5) coordinate (G);
				\draw[smooth]
				(A)..controls +(-88:0.1) and +(93:.1)..
				(B)..controls +(-87:0.3) and +(-100:8.5)..
				(C)..controls +(75:.8) and +(180:.1)..
				(D)..controls +(0:.1) and +(-105:.3)..
				(E)..controls +(70:2) and +(97:0.4)..
				(F)..controls +(-80:.1) and +(90:0.3)..
				(G);
				\draw[dashed] 
				(-3,0)node[below]{$-3$}|-(0,3)node[right]{$3$}
				(1,0)node[above]{$1$}|-(0,-1)node[left]{$-1$}
				(3,0)node[above]{$3$}|-(0,-3)node[below right]{$-3$};
				\fill 
				(0,-2) circle(1.5pt)
				(-3,3) circle(1.5pt)
				(3,-3) circle(1.5pt)
				(1,-1) circle(1.5pt);
				\draw[smooth,samples=300,domain=-3.2:3.7] plot(\x,{-(\x)});
				\node at (2.1,-4) {$y=f'(x)$};
				\node at (-1,2.1) {$y=h(x)$};
			\end{tikzpicture}
		\end{center}
		Ta có $ g'(x)=h\left(x-2m\right)\le 0\Leftrightarrow\hoac{
			&-3\le x-2m\le 1\\ 
			& x-2m\ge 3}\Leftrightarrow\hoac{
			& 2m-3\le x\le 2m+1\\ 
			& x\ge 2m+3.}$.\\
		Suy ra hàm số $ y=g(x)$ nghịch biến trên các khoảng $\left(2m-3;2m+1\right)$ và $\left(2m+3;+\infty\right)$.\\
		Do đó hàm số $ y=g(x)$ nghịch biến trên khoảng $\left(3;4\right)$ $\Leftrightarrow\hoac{
			&\heva{
				& 2m-3\le 3\\ 
				& 2m+1\ge 4}\\ 
			& 2m+3\le 3}\Leftrightarrow\hoac{
			&\dfrac{3}{2}\le m\le 3\\ 
			& m\le 0.}$ \\
		Mặt khác, do $ m$ nguyên dương nên $ m\in\left\{ 2;3\right\}\Rightarrow S=\left\{ 2;3\right\}$. Vậy số phần tử của $ S$ bằng $2$.\\
	}
	
\end{ex}

\begin{ex}%[2D1G1-3]%Câu 11
	Cho hàm số $f(x)$ có đạo hàm trên $\mathbb{R}$ là $f'(x)=\left(x-1\right)\left(x+3\right)$. Có bao nhiêu giá trị nguyên của tham số $m$ thuộc đoạn $\left[-10;20\right]$ để hàm số $y=f\left(x^2+3x-m\right)$ đồng biến trên khoảng $\left(0;2\right)$?
	\choice
	{\True $ 18$}
	{$ 17$}
	{$ 16$}
	{$ 20$}
	\loigiai{
		Ta có $y'=f'\left(x^2+3x-m\right)=\left(2x+3\right){f}'\left(x^2+3x-m\right)$.\\
		Theo đề bài ta có $f'(x)=\left(x-1\right)\left(x+3\right)$\\
		suy ra $f'(x)>0\Leftrightarrow\hoac{
			& x<-3\\ 
			& x>1}$ và $f'(x)<0\Leftrightarrow-3<x<1$ .\\
		Hàm số đồng biến trên khoảng $\left(0;2\right)$ khi $y'\ge 0,\forall x\in\left(0;2\right)$\\
		$\Leftrightarrow\left(2x+3\right){f}'\left(x^2+3x-m\right)\ge 0,\forall x\in\left(0;2\right)$.\\
		Do $x\in\left(0;2\right)$ nên $2x+3>0,\forall x\in\left(0;2\right)$. Do đó, ta có\\
		$y'\ge 0,\forall x\in\left(0;2\right)\Leftrightarrow f'\left(x^2+3x-m\right)\ge 0$\\
		$\Leftrightarrow\hoac{
			&{x^2}+3x-m\le-3\\ 
			&{x^2}+3x-m\ge 1}\Leftrightarrow\hoac{
			& m\ge{x^2}+3x+3\\ 
			& m\le{x^2}+3x-1}$\\
		$\Leftrightarrow\hoac{
			& m\ge\underset{\left[0;2\right]}{\max}\,\left(x^2+3x+3\right)\\ 
			& m\le\underset{\left[0;2\right]}{\min}\,\left(x^2+3x-1\right)} \Leftrightarrow\hoac{
			& m\ge 13\\ 
			& m\le-1}$.\\
		Do $m\in\left[-10;20\right]$, $ m\in\mathbb{Z}$ nên có $ 18$ giá trị nguyên của $m$ thỏa yêu cầu đề bài.}
\end{ex}

\begin{ex}%[2D1G1-3]%Câu 12
	Cho các hàm số $f(x)=x^3+4x+m$ và $g(x)=\left(x^2+2018\right){\left(x^2+2019\right)^2}{\left(x^2+2020\right)^3}$ . Có bao nhiêu giá trị nguyên của tham số $m\in\left[-2020;2020\right]$ để hàm số $g\left(f(x)\right)$ đồng biến trên $\left(2;+\infty\right)$ ?
	\choice
	{$2005$}
	{\True $2037$}
	{$4016$}
	{$4041$}
	\loigiai{
		Ta có $f(x)=x^3+4x+m$ và \\
		$g(x)=\left(x^2+2018\right){\left(x^2+2019\right)^2}{\left(x^2+2020\right)^3}=a_{12}{x^{12}}+a_{10}{x^{10}}+...+a_2x^2+a_0$.\\
		Suy ra $f'(x)=3x^2+4$ , $g'(x)=12a_{12}{x^{11}}+10a_{10}{x^9}+...+2a_2x$.\\
		Và có 
		\begin{eqnarray*}
			\left[g\left(f(x)\right)\right]' &=& f'(x)\left[12a_{12}{\left(f(x)\right)^{11}}+10a_{10}{\left(f(x)\right)^9}+...+2a_2f(x)\right]\\
			&=& f(x)f'(x)\left(12a_{12}{\left(f(x)\right)^{10}}+10a_{10}{\left(f(x)\right)^8}+...+2a_2\right).
		\end{eqnarray*} 
		Dễ thấy $a_{12};{a_{10}};...;{a_2};{a_0}>0$ và $f'(x)=3x^2+4>0$, $\forall x>2$.\\
		Do đó $f'(x)\left(12a_{12}{\left(f(x)\right)^{10}}+10a_{10}{\left(f(x)\right)^8}+...+2a_2\right)>0$ , $\forall x>2$.\\
		Hàm số $g\left(f(x)\right)$ đồng biến trên $\left(2;+\infty\right)$ khi $\left[g\left(f(x)\right)\right]^{'}\ge 0$, $\forall x>2$\\
		$\Rightarrow  f(x)\ge 0$, $\forall x>2 \Leftrightarrow x^3+4x+m\ge 0$, $\forall x>3 \Leftrightarrow  m\ge-x^3-4x$, $\forall x>2$\\
		$ \Rightarrow  m\ge\underset{\left[2;+\infty\right)}{\max}\,\left(-x^3-4x\right)=-16$.\\
		Vì $m\in\left[-2020;2020\right]$ và $m\in\mathbb{Z}$ nên có $2037$ giá trị thỏa mãn $m$ .}
\end{ex}

\begin{ex}%[2D1G1-3]%Câu 13
	Cho hàm số $y=f(x)$ có đạo hàm $f'(x)=x{\left(x+1\right)^2}\left(x^2+2mx+1\right)$ với mọi $x \in \mathbb{R}$. Có bao nhiêu số nguyên âm $m$ để hàm số $g(x)=f\left(2x+1\right)$ đồng biến trên khoảng $\left(3;5\right)$?
	\choice
	{\True $3$}
	{$2$}
	{$4$}
	{$6$}
	\loigiai{
		Ta có $g'(x)=2f'(2x+1)=2(2x+1)(2x+2)^2[(2x+1)^2+2m(2x+1)+1]$. 	Đặt $t=2x+1$\\
		Để hàm số $g(x)$ đồng biến trên khoảng $\left(3;5\right)$ khi và chỉ khi 
		\begin{eqnarray*}
			& & g'(x)\ge 0,\forall x\in\left(3;5\right) \\
			& \Leftrightarrow & t(t^2+2mt+1)\ge 0,\forall t\in\left(7;11\right)\Leftrightarrow{t^2}+2mt+1\ge 0,\,\,\forall t\in\left(7;11\right) \\
			&\Leftrightarrow & 2m\ge\dfrac{-t^2-1}{t},\,\,\,\forall t\in\left(7;11\right)
		\end{eqnarray*}	
		Xét hàm số $h(t)=\dfrac{-t^2-1}{t}$ trên $\left[7;11\right]$, có $h'(t)=\dfrac{-t^2+1}{t^2}$\\
		Bảng biến thiên
		\begin{center}
			\begin{tikzpicture}
				\tkzTabInit[espcl=3,lgt=1.2,nocadre]
				{$t$/0.7,$h'(t)$/0.7,$h(t)$/2.5}
				{$-\infty$,$1$,$11$,$+\infty$}
				\tkzTabLine{, ,,-,,,}
				%	\node (0) at ($(N12)+(0,-3)$) {$-\infty$};
				\node (1) at ($(N22)+(0,-0.8)$) [right] {$-\dfrac{50}{7}$};
				\node (2) at ($(N32)+(0,-2.5)$) [left] {$-\dfrac{122}{11}$};
				
				
				%				\node (3) at ($(N11+(-0.5,0))$) {};
				%				\node (4) at ($(N23)$) {};
				\fill[pattern=north east lines] (7.0,-0.7) rectangle (10,-4.4);
				\fill[pattern=north east lines] (1.5,-0.7) rectangle (4.5,-4.4);
				\draw[->] (1)--(2);	
				\draw[dashed] (4.5,-0.7)--(4.5,-4.4);
				\draw[dashed] (7.0,-0.7)--(7.0,-4.4);	
			\end{tikzpicture}		
		\end{center}
		Dựa vào BBT ta có $2m\ge\dfrac{-t^2-1}{t},\,\,\,\forall t\in\left(7;11\right)\Leftrightarrow 2m\ge\underset{\left[7;11\right]}{\max}\,h(t)\Leftrightarrow m\ge-\dfrac{50}{14}$\\
		Vì $ m\in{\mathbb{Z}^-}\Rightarrow m \in \{-3;-2;-1\}$ .
	}
\end{ex}

\begin{ex}%[2D1G1-3]%Câu 14
	Cho hàm số $y=f(x)$ có bảng biến thiên như sau\\
	\begin{center}
		\begin{tikzpicture}[>=stealth,scale = 1]
			\tkzTabInit[lgt=1,espcl=2.5,nocadre]
			{$x$ /0.7, $y'$ /0.7,$y$ /2.5}
			{$-\infty$,$0$,$2$,$+\infty$}
			\tkzTabLine{ ,-,0,+,0,-,}
			\tkzTabVar{-/$-\infty$, +/$4$,- /$0$, +/{ $+\infty$}}
		\end{tikzpicture}
	\end{center}
	Có bao nhiêu số nguyên $m<2019$ để hàm số $g(x)=f\left(x^2-2x+m\right)$ đồng biến trên khoảng $\left(1;+\infty\right)$?
	\choice
	{\True $2016$}
	{$2015$}
	{$2017$}
	{$2018$}
	\loigiai{
		Ta có $g'(x)=\left(x^2-2x+m\right)'{f}'\left(x^2-2x+m\right)=2\left(x-1\right){f}'\left(x^2-2x+m\right)$ .\\
		Hàm số $y=g(x)$ đồng biến trên khoảng $\left(1;+\infty\right)$ khi và chỉ khi $g'(x)\ge 0,\forall x\in\left(1;+\infty\right)$ và\\
		$g'(x)=0$ tại hữu hạn điểm \\
		$\Leftrightarrow 2\left(x-1\right){f}'\left(x^2-2x+m\right)\ge 0,\forall x\in\left(1;+\infty\right)$\\
		$\Leftrightarrow{f}'\left(x^2-2x+m\right)\ge 0,\forall x\in\left(1;+\infty\right)$ $\Leftrightarrow\hoac{
			&{x^2}-2x+m\ge 2,\forall x\in\left(1;+\infty\right)\\ 
			&{x^2}-2x+m\le 0,\forall x\in\left(1;+\infty\right).}$\\
		Xét hàm số $y=x^2-2x+m$, ta có bảng biến thiên
		\begin{center}
			\begin{tikzpicture}[>=stealth,scale = 1]
				\tkzTabInit[lgt=1,espcl=2.5,nocadre]
				{$x$ /0.7, $y'$ /0.7,$y$ /2.5}
				{$-\infty$,$2$,$+\infty$}
				\tkzTabLine{ ,-,0,+,}
				\tkzTabVar{+/$+\infty$, -/$m-1$, +/{$+\infty$}}
			\end{tikzpicture}
		\end{center}
		Dựa vào bảng biến thiên ta có\\
		TH1: $x^2-2x+m\ge 2,\forall x\in\left(1;+\infty\right)\Leftrightarrow m-1\ge 2\Leftrightarrow m\ge 3$ .\\
		TH2: $x^2-2x+m\le 0,\forall x\in\left(1;+\infty\right)$. Không có giá trị $m$ thỏa mãn.\\
		Vậy có $2016$ số nguyên $m<2019$ thỏa mãn yêu cầu bài toán.}
\end{ex}

\begin{ex}%[2D1G1-3]%Câu 15
	\immini{
		Cho hàm số $ y=f(x)$ có đạo hàm là hàm số $f'(x)$ trên $\mathbb{R}$. Biết rằng hàm số $ y=f'\left(x-2\right)+2$ có đồ thị như hình vẽ bên dưới. Hàm số $ f(x)$ đồng biến trên khoảng nào?
		\choice
		{$\left(-\infty ;3\right),\,\,\left(5;+\infty\right)$}
		{\True $\left(-\infty ;-1\right),\,\,\left(1;+\infty\right)$}
		{$\left(-1;1\right)$}
		{$\left(3;5\right)$}}{
		\begin{tikzpicture}[scale=0.7,font=\footnotesize, line join=round, line cap=round, >=stealth] %Đường cong bậc 3
			\draw[thick, ->] (-0.5,0)--(3.5,0);
			\draw[thick, ->] (0,-1.8)--(0,5.3);
			\draw (3.7,0) node[below] {$x$};
			\draw (0,5.4) node[left]{$y$};
			\draw (0,0) node[below left]{$0$};
			\draw[fill] (3,0) circle (0.5pt)node[below]{$ 3$};
			\draw[fill] (1,0) circle (0.5pt)node[below]{$ 1$};
			\draw[fill] (2,0) circle (0.5pt)node[above]{$2$};
			\draw[fill] (0,2) circle (0.5pt)node[left]{$ 2$};
			\draw[fill] (0,-1) circle (0.5pt)node[left]{$ -1$};
			%		\draw[fill] (0,2) circle (0.5pt)node[above left]{$ 2$};
			%		\draw[fill] (0,-2) circle (0.5pt)node[below left]{$ -2$};
			\draw[dashed] (3,0)--(3,2)--(0,2)--(1,2)--(1,0); 
			\draw[dashed](0,-1)--(2,-1)--(2,0);
			\draw[line width=1.2pt,smooth,samples=100,domain=0.6:3.4] plot(\x,{3*(\x)^2-12*(\x)+11});		
			%\draw[line width=1.2pt,smooth,samples=100,domain=-3.3:2.8] plot(\x,{0.75*(\x)^2+0.5*\x-1});
			%	\draw (2.0,2.8) node[left]{$y=f'(x)$};
	\end{tikzpicture}	}
	\loigiai{	
		Hàm số $ y=f'\left(x-2\right)+2$ có đồ thị $(C)$ như sau:\\
		\begin{center}
			\begin{tikzpicture}[scale=0.7,font=\footnotesize, line join=round, line cap=round, >=stealth] %Đường cong bậc 3
				\draw[thick, ->] (-0.5,0)--(3.5,0);
				\draw[thick, ->] (0,-1.8)--(0,5.3);
				\draw (3.7,0) node[below] {$x$};
				\draw (0,5.4) node[left]{$y$};
				\draw (0,0) node[below left]{$0$};
				\draw[fill] (3,0) circle (0.5pt)node[below]{$ 3$};
				\draw[fill] (1,0) circle (0.5pt)node[below]{$ 1$};
				\draw[fill] (2,0) circle (0.5pt)node[above]{$2$};
				\draw[fill] (0,2) circle (0.5pt)node[left]{$ 2$};
				\draw[fill] (0,-1) circle (0.5pt)node[left]{$ -1$};
				%		\draw[fill] (0,2) circle (0.5pt)node[above left]{$ 2$};
				%		\draw[fill] (0,-2) circle (0.5pt)node[below left]{$ -2$};
				\draw[dashed] (3,0)--(3,2)--(0,2)--(1,2)--(1,0); 
				\draw[dashed](0,-1)--(2,-1)--(2,0);
				\draw[line width=1.2pt,smooth,samples=100,domain=0.6:3.4] plot(\x,{3*(\x)^2-12*(\x)+11});		
				%\draw[line width=1.2pt,smooth,samples=100,domain=-3.3:2.8] plot(\x,{0.75*(\x)^2+0.5*\x-1});
				%	\draw (2.0,2.8) node[left]{$y=f'(x)$};
			\end{tikzpicture}
		\end{center}
		Dựa vào đồ thị $(C)$ ta có\\ $f'\left(x-2\right)+2>2,\forall x\in\left(-\infty ;1\right)\cup\left(3;+\infty\right)\Leftrightarrow{f}'\left(x-2\right)>0,\forall x\in\left(-\infty ;1\right)\cup\left(3;+\infty\right)$ .\\
		Đặt $ x*=x-2$ suy ra $f'\left(x*\right)>0,\forall x*\in\left(-\infty ;-1\right)\bigcup\left(1;+\infty\right)$.\\
		Vậy hàm số $ f(x)$ đồng biến trên khoảng $\left(-\infty ;-1\right),\,\,\left(1;+\infty\right)$.}
\end{ex}

\begin{ex}%[2D1G1-2]%Câu 16
	\immini{
		Cho hàm số $ y=f(x)$ có đạo hàm là hàm số $f'(x)$ trên $\mathbb{R}$. Biết rằng hàm số $ y=f'\left(x+2\right)-2$ có đồ thị như hình vẽ bên dưới. Hàm số $ f(x)$ nghịch biến trên khoảng nào?
		\choice
		{$\left(-3;-1\right),\,\,\left(1;3\right)$}
		{\True $\left(-1;1\right),\,\,\left(3;5\right)$}
		{$\left(-\infty ;-2\right),\,\,\left(0;2\right)$}
		{$\left(-5;-3\right),\,\,\left(-1;1\right)$}}{
		\begin{tikzpicture}[scale=0.7,font=\footnotesize, line join=round, line cap=round, >=stealth] %Đường cong bậc 3
			\draw[thick, ->] (-3.8,0)--(4.0,0);
			\draw[thick, ->] (0,-4.8)--(0,3.5);
			\draw (4.2,0) node[below] {$x$};
			\draw (0,3.7) node[left]{$y$};
			\draw (0,0) node[below left]{$0$};
			\draw[fill] (-3,0) circle (0.5pt)node[above]{$ -3$};
			\draw[fill] (-1,0) circle (0.5pt)node[above]{$ -1$};
			\draw[fill] (1,0) circle (0.5pt)node[above]{$ 1$};
			\draw[fill] (3,0) circle (0.5pt)node[above]{$3$};
			\draw[fill] (0,2) circle (0.5pt)node[above left]{$ 2$};
			\draw[fill] (0,-1) circle (0.5pt)node[above right]{$ -1$};
			%		\draw[fill] (0,2) circle (0.5pt)node[above left]{$ 2$};
			%		\draw[fill] (0,-2) circle (0.5pt)node[below left]{$ -2$};
			\draw[dashed] (-3,0)--(-3,-2)--(3,-2)--(3,0) (-1,0)--(-1,-2) (1,0)--(1,-2) (-3.494,0)--(-3.494,2)--(3.494,2)--(3.494,0); 
			\draw[line width=1.2pt,smooth,samples=100,domain=-3.6:3.6] plot(\x,{0.11*(\x)^4-1.11*(\x)^2-1});		
			%\draw[line width=1.2pt,smooth,samples=100,domain=-3.3:2.8] plot(\x,{0.75*(\x)^2+0.5*\x-1});
			%	\draw (2.0,2.8) node[left]{$y=f'(x)$};
	\end{tikzpicture}	}
	\loigiai{
		Hàm số $ y=f'\left(x+2\right)-2$ có đồ thị $(C)$ như sau
		\begin{center}
			\begin{tikzpicture}[scale=0.7,font=\footnotesize, line join=round, line cap=round, >=stealth] %Đường cong bậc 3
				\draw[thick, ->] (-3.8,0)--(4.0,0);
				\draw[thick, ->] (0,-4.8)--(0,3.5);
				\draw (4.2,0) node[below] {$x$};
				\draw (0,3.7) node[left]{$y$};
				\draw (0,0) node[below left]{$0$};
				\draw[fill] (-3,0) circle (0.5pt)node[above]{$ -3$};
				\draw[fill] (-1,0) circle (0.5pt)node[above]{$ -1$};
				\draw[fill] (1,0) circle (0.5pt)node[above]{$ 1$};
				\draw[fill] (3,0) circle (0.5pt)node[above]{$3$};
				\draw[fill] (0,2) circle (0.5pt)node[above left]{$ 2$};
				\draw[fill] (0,-1) circle (0.5pt)node[above right]{$ -1$};
				%		\draw[fill] (0,2) circle (0.5pt)node[above left]{$ 2$};
				%		\draw[fill] (0,-2) circle (0.5pt)node[below left]{$ -2$};
				\draw[dashed] (-3,0)--(-3,-2)--(3,-2)--(3,0) (-1,0)--(-1,-2) (1,0)--(1,-2) (-3.494,0)--(-3.494,2)--(3.494,2)--(3.494,0); 
				\draw[line width=1.2pt,smooth,samples=100,domain=-3.6:3.6] plot(\x,{0.11*(\x)^4-1.11*(\x)^2-1});		
				%\draw[line width=1.2pt,smooth,samples=100,domain=-3.3:2.8] plot(\x,{0.75*(\x)^2+0.5*\x-1});
				%	\draw (2.0,2.8) node[left]{$y=f'(x)$};
			\end{tikzpicture}
		\end{center}
		Dựa vào đồ thị $(C)$ ta có\\
		$f'\left(x+2\right)-2<-2,\forall x\in\left(-3;-1\right)\bigcup\left(1;3\right)\Leftrightarrow{f}'\left(x+2\right)<0,\forall x\in\left(-3;-1\right)\bigcup\left(1;3\right)$.\\
		Đặt $ x^*=x+2$ suy ra: $f'\left(x^*\right)<0,\forall x^*\in\left(-1;1\right)\bigcup\left(3;5\right)$.\\
		Vậy: Hàm số $ f(x)$ đồng biến trên khoảng $\left(-1;1\right),\,\,\left(3;5\right)$.}
\end{ex}

\begin{ex}%[2D1G1-2]%Câu 17
	\immini{
		Cho hàm số $ y=f(x)$ có đạo hàm là hàm số $f'(x)$ trên $\mathbb{R}$. Biết rằng hàm số $ y=f'\left(x-2\right)+2$ có đồ thị như hình vẽ bên dưới. Hàm số $ f(x)$ nghịch biến trên khoảng nào?
		\choice
		{$\left(-\infty ;2\right)$}
		{\True $\left(-1;1\right)$}
		{$\left(\dfrac{3}{2};\dfrac{5}{2}\right)$}
		{$\left(2;+\infty\right)$}}{
		\begin{tikzpicture}[scale=0.7,font=\footnotesize, line join=round, line cap=round, >=stealth] %Đường cong bậc 3
			\draw[thick, ->] (-0.5,0)--(3.5,0);
			\draw[thick, ->] (0,-1.8)--(0,5.3);
			\draw (3.7,0) node[below] {$x$};
			\draw (0,5.4) node[left]{$y$};
			\draw (0,0) node[below left]{$0$};
			\draw[fill] (3,0) circle (0.5pt)node[below]{$ 3$};
			\draw[fill] (1,0) circle (0.5pt)node[below]{$ 1$};
			\draw[fill] (2,0) circle (0.5pt)node[above]{$2$};
			\draw[fill] (0,2) circle (0.5pt)node[left]{$ 2$};
			\draw[fill] (0,-1) circle (0.5pt)node[left]{$ -1$};
			%		\draw[fill] (0,2) circle (0.5pt)node[above left]{$ 2$};
			%		\draw[fill] (0,-2) circle (0.5pt)node[below left]{$ -2$};
			\draw[dashed] (3,0)--(3,2)--(0,2)--(1,2)--(1,0); 
			\draw[dashed](0,-1)--(2,-1)--(2,0);
			\draw[line width=1.2pt,smooth,samples=100,domain=0.6:3.4] plot(\x,{3*(\x)^2-12*(\x)+11});		
			%\draw[line width=1.2pt,smooth,samples=100,domain=-3.3:2.8] plot(\x,{0.75*(\x)^2+0.5*\x-1});
			%	\draw (2.0,2.8) node[left]{$y=f'(x)$};
	\end{tikzpicture}	}
	\loigiai{
		Hàm số $ y=f'\left(x-2\right)+2$ có đồ thị $(C)$ như sau
		\begin{center}
			\begin{tikzpicture}[scale=0.7,font=\footnotesize, line join=round, line cap=round, >=stealth] %Đường cong bậc 3
				\draw[thick, ->] (-0.5,0)--(3.5,0);
				\draw[thick, ->] (0,-1.8)--(0,5.3);
				\draw (3.7,0) node[below] {$x$};
				\draw (0,5.4) node[left]{$y$};
				\draw (0,0) node[below left]{$0$};
				\draw[fill] (3,0) circle (0.5pt)node[below]{$ 3$};
				\draw[fill] (1,0) circle (0.5pt)node[below]{$ 1$};
				\draw[fill] (2,0) circle (0.5pt)node[above]{$2$};
				\draw[fill] (0,2) circle (0.5pt)node[left]{$ 2$};
				\draw[fill] (0,-1) circle (0.5pt)node[left]{$ -1$};
				%		\draw[fill] (0,2) circle (0.5pt)node[above left]{$ 2$};
				%		\draw[fill] (0,-2) circle (0.5pt)node[below left]{$ -2$};
				\draw[dashed] (3,0)--(3,2)--(0,2)--(1,2)--(1,0); 
				\draw[dashed](0,-1)--(2,-1)--(2,0);
				\draw[line width=1.2pt,smooth,samples=100,domain=0.6:3.4] plot(\x,{3*(\x)^2-12*(\x)+11});		
				%\draw[line width=1.2pt,smooth,samples=100,domain=-3.3:2.8] plot(\x,{0.75*(\x)^2+0.5*\x-1});
				%	\draw (2.0,2.8) node[left]{$y=f'(x)$};
			\end{tikzpicture}
		\end{center}
		Dựa vào đồ thị $(C)$ ta có\\
		$f'\left(x-2\right)+2<2,\forall x\in\left(1;3\right)\Leftrightarrow{f}'\left(x-2\right)<0,\forall x\in\left(1;3\right)$.\\
		Đặt $ x^*=x-2$ thì $f'\left(x^*\right)<0,\forall x^*\in\left(-1;1\right)$.\\
		Vậy: Hàm số $ f(x)$ nghịch biến trên khoảng $\left(-1;1\right)$.\\
		Cách khác:\\
		Tịnh tiến sang trái hai đơn vị và xuống dưới $2$ đơn vị thì từ đồ thị $(C)$ sẽ thành đồ thị của hàm$ y=f'(x)$. Khi đó $f'(x)<0,\forall x\in\left(-1;1\right)$.\\
		Vậy hàm số $ f(x)$ nghịch biến trên khoảng $\left(-1;1\right)$.}
\end{ex}

\begin{ex}%[2D1G1-2]%Câu 18
	Cho hàm số $y=f(x)$ có đạo hàm cấp $ 3$ liên tục trên $\mathbb{R}$ và thỏa mãn $f(x)\cdot f'''(x)=x{\left(x-1\right)^2}{\left(x+4\right)^3}$ với mọi $x\in\mathbb{R}$ và $g(x)=\left[f'(x)\right]^2-2f(x)\cdot f''(x)$. Hàm số $h(x)=g\left(x^2-2x\right)$ đồng biến trên khoảng nào dưới đây?
	\choice
	{$\left(-\infty ;1\right)$}
	{$\left(2;+\infty\right)$}
	{$\left(0;1\right)$}
	{\True $\left(1;2\right)$}
	\loigiai{		
		Ta có $g'(x)=2f''(x){f}'(x)-2f'(x)\cdot f''(x)-2f(x)\cdot f'''(x)=-2f(x)\cdot f'''(x);$\\
		Khi đó $\left(h(x)\right)'=\left(2x-2\right){g}'\left(x^2-2x\right)=-2\left(2x-2\right)\left(x^2-2x\right){\left(x^2-2x-1\right)^2}{\left(x^2-2x+4\right)^3}$\\
		$h'(x)=0\Leftrightarrow\hoac{
			& x=0\\ 
			& x=1\\ 
			& x=2\\ 
			& x=1\pm\sqrt{2}.}$ 
		Ta có bảng xét dấu của $h'(x)$
		\begin{center}
			\begin{tikzpicture}
				\tkzTabInit[lgt=1.2,espcl=2,nocadre]
				{$t$/0.7, $h'(x)$ /.7} % first column
				{$-\infty$, $1-\sqrt{2}$,$0$, $1$,$2$,$1+\sqrt{2}$, $+\infty$} % first row
				\tkzTabLine { ,+,0,-,0,+,0,-,0,+,0,- } % second row
				%				\tkzTabLine {,-,z,+,t,+,} % third row
				%				\tkzTabLine {,+,d,-,z,+,} % last row
			\end{tikzpicture}
		\end{center}
		Suy ra hàm số $h(x)=g\left(x^2-2x\right)$ đồng biến trên khoảng $\left(1;2\right)$.}
\end{ex}

\begin{ex}%[2D1G1-2]%Câu 19
	Cho hàm số $ y=f(x)$ xác định trên $\mathbb{R}$. Hàm số $ y=g(x)=f'\left(2x+3\right)+2$ có đồ thị là một parabol với tọa độ đỉnh $ I\left(2;-1\right)$ và đi qua điểm $ A\left(1;2\right)$. Hỏi hàm số $ y=f(x)$ nghịch biến trên khoảng nào dưới đây?
	\choice
	{\True $\left(5;9\right)$}
	{$\left(1;2\right)$}
	{$\left(-\infty ;9\right)$}
	{$\left(1;3\right)$}
	\loigiai{	
		Xét hàm số $ g(x)=f'\left(2x+3\right)+2$ có đồ thị là một Parabol nên có phương trình dạng $ y=g(x)=a{x^2}+bx+c\,\,\,\,(P)$.\\
		Vì $(P)$ có đỉnh $ I\left(2;-1\right)$ nên $\heva{
			&\dfrac{-b}{2a}=2\\ 
			& g(2)=-1} \Leftrightarrow\heva{
			&-b=4a\\ 
			& 4a+2b+c=-1} \Leftrightarrow\heva{
			& 4a+b=0\\ 
			& 4a+2b+c=-1}$.\\
		Vì $(P)$ đi qua điểm $ A\left(1;2\right)$ nên $ g(1)=2\Leftrightarrow a+b+c=2$.\\
		Ta có hệ phương trình $\heva{
			& 4a+b=0\\ 
			& 4a+2b+c=-1\\ 
			& a+b+c=2} \Leftrightarrow\heva{
			& a=3\\ 
			& b=-12\\ 
			& c=11}$ nên $ g(x)=3x^2-12x+11$.\\
		Đồ thị của hàm $ y=g(x)$ là
		\begin{center}
			\begin{tikzpicture}[scale=0.7,font=\footnotesize, line join=round, line cap=round, >=stealth] %Đường cong bậc 3
				\draw[thick, ->] (-0.5,0)--(3.5,0);
				\draw[thick, ->] (0,-1.8)--(0,5.3);
				\draw (3.7,0) node[below] {$x$};
				\draw (0,5.4) node[left]{$y$};
				\draw (0,0) node[below left]{$0$};
				\draw[fill] (3,0) circle (0.5pt)node[below]{$ 3$};
				\draw[fill] (1,0) circle (0.5pt)node[below]{$ 1$};
				\draw[fill] (2,0) circle (0.5pt)node[above]{$2$};
				\draw[fill] (0,2) circle (0.5pt)node[left]{$ 2$};
				\draw[fill] (0,-1) circle (0.5pt)node[left]{$ -1$};
				%		\draw[fill] (0,2) circle (0.5pt)node[above left]{$ 2$};
				%		\draw[fill] (0,-2) circle (0.5pt)node[below left]{$ -2$};
				\draw[dashed] (3,0)--(3,2)--(0,2)--(1,2)--(1,0) (3.2,2)--(3,2); 
				\draw[dashed](0,-1)--(2,-1)--(2,0);
				\draw[line width=1.2pt,smooth,samples=100,domain=0.6:3.4] plot(\x,{3*(\x)^2-12*(\x)+11});		
				%\draw[line width=1.2pt,smooth,samples=100,domain=-3.3:2.8] plot(\x,{0.75*(\x)^2+0.5*\x-1});
				%	\draw (2.0,2.8) node[left]{$y=f'(x)$};
			\end{tikzpicture}	
		\end{center}
		Theo đồ thị ta thấy $ f'(2x+3)\le 0\Leftrightarrow f'(2x+3)+2\le 2\Leftrightarrow 1\le x\le 3$.\\
		Đặt $ t=2x+3\Leftrightarrow x=\dfrac{t-3}{2}$ khi đó $ f'(t)\le 0\Leftrightarrow 1\le\dfrac{t-3}{2}\le 3\Leftrightarrow 5\le t\le 9$.\\
		Vậy $ y=f(x)$ nghịch biến trên khoảng $\left(5;9\right)$.}
\end{ex}

\begin{ex}%[2D1G1-2]%Câu 20
	\immini{
		Cho hàm số $ y=f(x)$, hàm số $f'(x)=x^3+a{x^2}+bx+c\left(a,b,c\in\mathbb{R}\right)$ có đồ thị như hình vẽ bên.
		Hàm số $ g(x)=f\left(f'(x)\right)$ nghịch biến trên khoảng nào dưới đây?
		\choice
		{$\left(1;+\infty\right)$}
		{\True $\left(-\infty ;-2\right)$}
		{$\left(-1;0\right)$}
		{$\left(-\dfrac{\sqrt{3}}{3};\dfrac{\sqrt{3}}{3}\right)$}}{
		\begin{tikzpicture}[scale=0.8,font=\footnotesize, line join=round, line cap=round, >=stealth] %Đường cong bậc 3
			\draw[thick, ->] (-1.7,0)--(1.7,0);
			\draw[thick, ->] (0,-2.7)--(0,3.0);
			\draw (1.9,0) node[below] {$x$};
			\draw (0,3.2) node[left]{$y$};
			\draw (0,0) node[below left]{$0$};
			\draw[fill] (-1,0) circle (0.5pt)node[above left]{$ -1 $};
			\draw[fill] (1,0) circle (0.5pt)node[below right]{$ 1$};
			\draw[line width=1.2pt,smooth,samples=100,domain=-1.3:1.3] plot(\x,{2.667*(\x)^3+0*(\x)^2-2.667*\x});		
			%\draw[line width=1.2pt,smooth,samples=100,domain=-3.3:2.8] plot(\x,{0.75*(\x)^2+0.5*\x-1});
		\end{tikzpicture}	
	}
	\loigiai{	
		Vì các điểm $\left(-1;0\right),\left(0;0\right),\left(1;0\right)$ thuộc đồ thị hàm số $ y=f'(x)$ nên ta có hệ\\
		$\heva{
			&-1+a-b+c=0\\ 
			& c=0\\ 
			& 1+a+b+c=0} \Leftrightarrow\heva{
			& a=0\\ 
			& b=-1\\ 
			& c=0} \Rightarrow {f}'(x)=x^3-x\Rightarrow f''(x)=3x^2-1$.\\
		Ta có $ g(x)=f\left(f'(x)\right)\Rightarrow{g}'(x)=f'\left(f'(x)\right)\cdot f''(x)$.\\
		Xét \\
		$g'(x)=0\Leftrightarrow{g}'(x)=f'\left(f'(x)\right)\cdot f''(x)=0$\\
		$\Leftrightarrow {f}'\left(x^3-x\right)\left(3x^2-1\right)=0\Leftrightarrow\hoac{
			&{x^3}-x=0\\ 
			&{x^3}-x=1\\ 
			&{x^3}-x=-1\\ 
			& 3x^2-1=0} \Leftrightarrow \hoac{
			& x=\pm 1\\ 
			& x=0\\ 
			& x=x_1(x_1\approx 1,325)\\ 
			& x=x_2(x_2\approx-1,325)\\ 
			& x=\pm\dfrac{\sqrt{3}}{3}.}$\\
		Bảng biến thiên
		\begin{center}
			\begin{tikzpicture}
				\tkzTabInit[lgt=1.2,espcl=2,nocadre]
				{$t$/0.7, $h'(x)$ /.7} % first column
				{$-\infty$, $-1{,}325$,$-1$, $-\dfrac{\sqrt{3}}{3}$,$0$,$\dfrac{\sqrt{3}}{3}$,$1$,$1{,}325$, $+\infty$} % first row
				\tkzTabLine { ,-,0,+,0,-,0,+,0,-,0,+,0,-,0,+, } % second row
				%				\tkzTabLine {,-,z,+,t,+,} % third row
				%				\tkzTabLine {,+,d,-,z,+,} % last row
			\end{tikzpicture}
		\end{center}
		Dựa vào bảng biến thiên ta có $ g(x)$ nghịch biến trên $\left(-\infty ;-2\right)$}
\end{ex}
\Closesolutionfile{ans}
\indapan{10}{ans/CD1/Muc_9_10}
\chapter{CD20}
\begin{Solution}{1}
C
\end{Solution}
\begin{Solution}{3}
B
\end{Solution}
\begin{Solution}{4}
A
\end{Solution}
\begin{Solution}{5}
A
\end{Solution}
\begin{Solution}{6}
A
\end{Solution}
\begin{Solution}{7}
B
\end{Solution}
\begin{Solution}{8}
A
\end{Solution}
\begin{Solution}{9}
C
\end{Solution}
\begin{Solution}{10}
B
\end{Solution}
\begin{Solution}{11}
C
\end{Solution}
\begin{Solution}{12}
D
\end{Solution}
\begin{Solution}{13}
B
\end{Solution}
\begin{Solution}{14}
D
\end{Solution}
\begin{Solution}{15}
A
\end{Solution}
\begin{Solution}{16}
B
\end{Solution}
\begin{Solution}{17}
C
\end{Solution}
\begin{Solution}{18}
C
\end{Solution}
\begin{Solution}{19}
C
\end{Solution}
\begin{Solution}{20}
B
\end{Solution}
\begin{Solution}{21}
C
\end{Solution}
\begin{Solution}{22}
B
\end{Solution}
\begin{Solution}{23}
D
\end{Solution}
\begin{Solution}{24}
B
\end{Solution}
\begin{Solution}{25}
D
\end{Solution}
\begin{Solution}{26}
D
\end{Solution}
\begin{Solution}{27}
B
\end{Solution}
\begin{Solution}{28}
A
\end{Solution}
\begin{Solution}{29}
C
\end{Solution}
\begin{Solution}{30}
B
\end{Solution}
\begin{Solution}{31}
D
\end{Solution}
\begin{Solution}{32}
B
\end{Solution}
\begin{Solution}{33}
B
\end{Solution}
\begin{Solution}{34}
C
\end{Solution}
\begin{Solution}{35}
D
\end{Solution}
\begin{Solution}{36}
B
\end{Solution}
\begin{Solution}{37}
B
\end{Solution}
\begin{Solution}{38}
A
\end{Solution}
\begin{Solution}{39}
A
\end{Solution}
\begin{Solution}{40}
D
\end{Solution}
\begin{Solution}{41}
C
\end{Solution}
\begin{Solution}{42}
B
\end{Solution}
\begin{Solution}{43}
A
\end{Solution}
\begin{Solution}{44}
A
\end{Solution}
\begin{Solution}{45}
D
\end{Solution}
\begin{Solution}{46}
C
\end{Solution}
\begin{Solution}{47}
A
\end{Solution}
\begin{Solution}{48}
B
\end{Solution}
\begin{Solution}{49}
B
\end{Solution}
\begin{Solution}{50}
B
\end{Solution}
\begin{Solution}{51}
A
\end{Solution}
\begin{Solution}{52}
A
\end{Solution}
\begin{Solution}{53}
C
\end{Solution}
\begin{Solution}{54}
C
\end{Solution}
\begin{Solution}{55}
C
\end{Solution}
\begin{Solution}{56}
B
\end{Solution}
\begin{Solution}{57}
C
\end{Solution}
\begin{Solution}{58}
C
\end{Solution}
\begin{Solution}{59}
B
\end{Solution}
\begin{Solution}{60}
C
\end{Solution}
\begin{Solution}{61}
A
\end{Solution}
\begin{Solution}{62}
B
\end{Solution}
\begin{Solution}{63}
B
\end{Solution}
\begin{Solution}{64}
D
\end{Solution}
\begin{Solution}{65}
D
\end{Solution}
\begin{Solution}{66}
B
\end{Solution}
\begin{Solution}{67}
A
\end{Solution}
\begin{Solution}{68}
D
\end{Solution}

\begin{Solution}{1}
D
\end{Solution}
\begin{Solution}{2}
C
\end{Solution}
\begin{Solution}{3}
C
\end{Solution}
\begin{Solution}{4}
A
\end{Solution}
\begin{Solution}{5}
B
\end{Solution}
\begin{Solution}{6}
D
\end{Solution}
\begin{Solution}{7}
C
\end{Solution}
\begin{Solution}{8}
D
\end{Solution}
\begin{Solution}{9}
A
\end{Solution}
\begin{Solution}{10}
B
\end{Solution}
\begin{Solution}{11}
D
\end{Solution}
\begin{Solution}{12}
A
\end{Solution}
\begin{Solution}{13}
D
\end{Solution}
\begin{Solution}{14}
B
\end{Solution}
\begin{Solution}{15}
B
\end{Solution}
\begin{Solution}{16}
C
\end{Solution}
\begin{Solution}{1}
A
\end{Solution}
\begin{Solution}{2}
B
\end{Solution}
\begin{Solution}{3}
D
\end{Solution}
\begin{Solution}{4}
D
\end{Solution}
\begin{Solution}{5}
C
\end{Solution}
\begin{Solution}{6}
A
\end{Solution}
\begin{Solution}{7}
D
\end{Solution}
\begin{Solution}{8}
B
\end{Solution}
\begin{Solution}{9}
C
\end{Solution}
\begin{Solution}{10}
C
\end{Solution}
\begin{Solution}{1}
D
\end{Solution}
\begin{Solution}{2}
D
\end{Solution}
\begin{Solution}{3}
B
\end{Solution}
\begin{Solution}{4}
C
\end{Solution}
\begin{Solution}{5}
D
\end{Solution}
\begin{Solution}{6}
A
\end{Solution}
\begin{Solution}{7}
C
\end{Solution}
\begin{Solution}{8}
B
\end{Solution}
\begin{Solution}{9}
A
\end{Solution}
\begin{Solution}{10}
C
\end{Solution}
\begin{Solution}{11}
D
\end{Solution}
\begin{Solution}{12}
C
\end{Solution}
\begin{Solution}{13}
A
\end{Solution}
\begin{Solution}{14}
D
\end{Solution}
\begin{Solution}{15}
A
\end{Solution}
\begin{Solution}{16}
A
\end{Solution}
\begin{Solution}{17}
B
\end{Solution}
\begin{Solution}{18}
C
\end{Solution}
\begin{Solution}{19}
C
\end{Solution}
\begin{Solution}{20}
A
\end{Solution}
\begin{Solution}{21}
D
\end{Solution}
\begin{Solution}{22}
C
\end{Solution}
\begin{Solution}{23}
A
\end{Solution}
\begin{Solution}{24}
C
\end{Solution}
\begin{Solution}{25}
A
\end{Solution}
\begin{Solution}{26}
B
\end{Solution}
\begin{Solution}{27}
B
\end{Solution}
\begin{Solution}{28}
D
\end{Solution}
\begin{Solution}{29}
B
\end{Solution}
\begin{Solution}{30}
D
\end{Solution}
\begin{Solution}{31}
D
\end{Solution}
\begin{Solution}{32}
C
\end{Solution}
\begin{Solution}{33}
D
\end{Solution}
\begin{Solution}{34}
C
\end{Solution}
\begin{Solution}{35}
D
\end{Solution}
\begin{Solution}{36}
D
\end{Solution}
\begin{Solution}{37}
D
\end{Solution}
\begin{Solution}{38}
D
\end{Solution}
\begin{Solution}{39}
D
\end{Solution}
\begin{Solution}{40}
C
\end{Solution}
\begin{Solution}{41}
A
\end{Solution}
\begin{Solution}{1}
A
\end{Solution}
\begin{Solution}{2}
B
\end{Solution}
\begin{Solution}{3}
C
\end{Solution}
\begin{Solution}{4}
A
\end{Solution}
\begin{Solution}{5}
A
\end{Solution}
\begin{Solution}{6}
C
\end{Solution}
\begin{Solution}{7}
C
\end{Solution}
\begin{Solution}{8}
B
\end{Solution}
\begin{Solution}{9}
C
\end{Solution}
\begin{Solution}{10}
B
\end{Solution}
\begin{Solution}{11}
A
\end{Solution}
\begin{Solution}{12}
B
\end{Solution}
\begin{Solution}{13}
B
\end{Solution}
\begin{Solution}{14}
B
\end{Solution}
\begin{Solution}{15}
A
\end{Solution}
\begin{Solution}{16}
B
\end{Solution}
\begin{Solution}{17}
A
\end{Solution}
\begin{Solution}{18}
D
\end{Solution}
\begin{Solution}{19}
C
\end{Solution}
\begin{Solution}{20}
C
\end{Solution}
\begin{Solution}{21}
A
\end{Solution}
\begin{Solution}{22}
C
\end{Solution}
\begin{Solution}{23}
C
\end{Solution}
\begin{Solution}{24}
A
\end{Solution}
\begin{Solution}{25}
B
\end{Solution}
\begin{Solution}{26}
B
\end{Solution}
\begin{Solution}{27}
A
\end{Solution}
\begin{Solution}{28}
A
\end{Solution}
\begin{Solution}{29}
C
\end{Solution}
\begin{Solution}{30}
B
\end{Solution}
\begin{Solution}{31}
A
\end{Solution}
\begin{Solution}{32}
C
\end{Solution}
\begin{Solution}{33}
B
\end{Solution}
\begin{Solution}{34}
A
\end{Solution}
\begin{Solution}{35}
B
\end{Solution}
\begin{Solution}{36}
B
\end{Solution}
\begin{Solution}{37}
B
\end{Solution}
\begin{Solution}{38}
D
\end{Solution}
\begin{Solution}{39}
B
\end{Solution}
\begin{Solution}{40}
A
\end{Solution}
\begin{Solution}{41}
D
\end{Solution}
\begin{Solution}{42}
D
\end{Solution}
\begin{Solution}{43}
A
\end{Solution}
\begin{Solution}{44}
D
\end{Solution}
\begin{Solution}{45}
C
\end{Solution}
\begin{Solution}{46}
B
\end{Solution}
\begin{Solution}{47}
A
\end{Solution}
\begin{Solution}{48}
D
\end{Solution}
\begin{Solution}{49}
B
\end{Solution}
\begin{Solution}{50}
B
\end{Solution}
\begin{Solution}{51}
D
\end{Solution}
\begin{Solution}{52}
C
\end{Solution}
\begin{Solution}{53}
C
\end{Solution}
\begin{Solution}{54}
B
\end{Solution}
\begin{Solution}{55}
D
\end{Solution}
\begin{Solution}{56}
B
\end{Solution}
\begin{Solution}{57}
C
\end{Solution}
\begin{Solution}{58}
A
\end{Solution}
\begin{Solution}{59}
A
\end{Solution}
\begin{Solution}{60}
B
\end{Solution}
\begin{Solution}{61}
D
\end{Solution}
\begin{Solution}{62}
D
\end{Solution}
\begin{Solution}{63}
B
\end{Solution}
\begin{Solution}{64}
A
\end{Solution}
\begin{Solution}{65}
D
\end{Solution}
\begin{Solution}{66}
C
\end{Solution}
\begin{Solution}{67}
A
\end{Solution}
\begin{Solution}{68}
A
\end{Solution}
\begin{Solution}{69}
D
\end{Solution}
\begin{Solution}{70}
C
\end{Solution}
\begin{Solution}{71}
B
\end{Solution}
\begin{Solution}{72}
A
\end{Solution}
\begin{Solution}{73}
C
\end{Solution}
\begin{Solution}{74}
C
\end{Solution}
\begin{Solution}{75}
C
\end{Solution}
\begin{Solution}{76}
A
\end{Solution}
\begin{Solution}{77}
C
\end{Solution}
\begin{Solution}{78}
B
\end{Solution}
\begin{Solution}{79}
D
\end{Solution}
\begin{Solution}{80}
B
\end{Solution}

\section{Mức 9,10 điểm}
\setcounter{ex}{0}
\setcounter{dang}{0}
\Opensolutionfile{ans}[ans/CD1/Muc_9_10]
\begin{dang}{Tìm m để hàm số đơn điệu trên các khoảng xác định của nó}
	Đang thiếu bài thầy Jf Câu 1 đến 26 
\end{dang}
\begin{dang}
	{Tìm khoảng đơn điệu của hàm số $g(x) = f\left[ u(x)\right] +v(x)$ khi biết đồ thị hoặc bảng biến thiên của hàm số $y = f'(x)$}
\end{dang}
\begin{ex}[Đề tham khảo 2019]%[2D1K1-2]
	Cho hàm số $f(x)$ có bảng xét dấu của đạo hàm như sau
	\begin{center}
		\begin{tikzpicture}
			\tkzTabInit[nocadre,lgt=1.2,espcl=2,deltacl=0.6]
			{$x$ /0.6,$f'(x)$ /0.6}
			{$-\infty$,$1$,$2$,$3$,$4$,$+\infty$}
			\tkzTabLine{,-,$0$,+,$0$,+,$0$,-,$0$,+,}
		\end{tikzpicture}
	\end{center}
	Hàm số $y=3 f(x+2)-x^3+3 x$ đồng biến trên khoảng nào dưới đây?
	\choice
	{$(-\infty ;-1)$}
	{\True $(-1 ; 0)$}
	{$(0 ; 2)$}
	{$(1 ;+\infty)$}
	\loigiai{
		Ta có $y'=3\left[f'(x+2)-\left(x^2-3\right)\right]$.\\
		Với $x \in(-1 ; 0) \Rightarrow x+2 \in(1 ; 2) \Rightarrow f'(x+2)>0$, lại có $x^2-3<0 \Rightarrow y'>0 ;~ \forall x \in(-1 ; 0)$.\\
		Vậy hàm số $y=3 f(x+2)-x^3+3 x$ đồng biến trên khoảng $(-1 ; 0)$.\\
		Chú ý:\\
		+) Ta xét $x \in(1 ; 2) \subset(1 ;+\infty)
		\Rightarrow x+2 \in(3 ; 4)\\
		\Rightarrow f'(x+2)<0 ;~ x^2-3>0$\\
		Suy ra hàm số nghịch biến trên khoảng $(1 ; 2)$ nên loại hai phương án$(0 ; 2)$ và $(1 ;+\infty)$.\\
		+) Tương tự ta xét
		$x \in(-\infty ;-2) \Rightarrow x+2 \in(-\infty ; 0)\\
		\Rightarrow f'(x+2)<0 ; x^2-3>0 \Rightarrow y'<0 ; ~ \forall x \in(-\infty ;-2)$.\\
		Suy ra hàm số nghịch biến trên khoảng $(-\infty ;-2)$ nên loại$(-\infty ;-1)$.\\
		Vậy hàm số đã cho đồng biến trên khoảng $(-1 ; 0)$.
	}
\end{ex}
\begin{ex}[Đề Tham Khảo 2020 - Lần 1]%[2D1G1-2]
	\immini{
		Cho hàm số $f(x)$. Hàm số $y=f'(x)$ có đồ thị như hình bên. Hàm số $g(x)=f(1-2 x)+x^2-x$ nghịch biến trên khoảng nào dưới đây?
		\choice
		{\True $\left(1 ; \dfrac{3}{2}\right)$}
		{$\left(0 ; \dfrac{1}{2}\right)$}
		{$(-2 ;-1)$}
		{$(2 ; 3)$}
	}
	{
		\begin{tikzpicture}[scale=0.7,>=stealth, font=\footnotesize, line join=round, line cap=round]
			%\def\a{1} \def\b{-6} \def\c{9} \def\d{1} % Hệ số
			\def\xmin{-4} \def\xmax{6}
			\def\ymin{-3} \def\ymax{2} 
			%\draw[color=gray!50,dashed] (\xmin,\ymin) grid (\xmax,\ymax); 
			\draw[->] (\xmin,0)--(\xmax,0) node [below]{$x$};
			\draw[->] (0,\ymin)--(0,\ymax) node [left]{$y$};
			\node at (0,0) [below left]{$O$};
			%\node at (1,3) [below left]{$f'(x)$};
			%\node at (-1.3,4) {$f'(x)$};
			\draw[dashed] (-2,0) node[below]{$-2$}--(-2,1)--(0,1) node[below left]{$1$};
			\draw[dashed] (4,0) node[below left]{$4$}--(4,-2)--(0,-2) node[below left]{$-2$};
			%\draw[dashed] (1,0) node[below]{$1$}--(1,1);
			%\draw[dashed] (-0.5,0) node[below left]{$-0{,}5$}--(-0.5,2.125);
			\clip (\xmin+0.1,\ymin+0.1) rectangle (\xmax-0.5,\ymax-0.1);
			\draw[smooth,samples=300][domain=-4:5.5] plot(\x,{0.071*(\x)^3-0.142*(\x)^2-1.07*(\x)});
		\end{tikzpicture}
	}
	
	\loigiai{
		Ta có : $g(x)=f(1-2 x)+x^2-x \Rightarrow g'(x)=-2 f'(1-2 x)+2 x-1$.\\
		\immini{
			Đặt $t=1-2 x \Rightarrow g'(x)=-2 f'(t)-t$.\\
			$g'(x)=0 \Rightarrow f'(t)=-\dfrac{t}{2}$.\\
			Vẽ đường thẳng $y=-\dfrac{x}{2}$ và đồ thị hàm số $f'(x)$ trên cùng một hệ trục
		}	
		{
			\begin{tikzpicture}[scale=0.7,>=stealth, font=\footnotesize, line join=round, line cap=round]
				%\def\a{1} \def\b{-6} \def\c{9} \def\d{1} % Hệ số
				\def\xmin{-4} \def\xmax{6}
				\def\ymin{-3} \def\ymax{2} 
				%	\draw[color=gray!50,dashed] (\xmin,\ymin) grid (\xmax,\ymax); 
				\draw[->] (\xmin,0)--(\xmax,0) node [below]{$x$};
				\draw[->] (0,\ymin)--(0,\ymax) node [left]{$y$};
				\node at (0,0) [below left]{$O$};
				%\node at (1,3) [below left]{$f'(x)$};
				%\node at (-1.3,4) {$f'(x)$};
				\draw[dashed] (-2,0) node[below]{$-2$}--(-2,1)--(0,1) node[below left]{$1$};
				\draw[dashed] (4,0) node[below]{$4$}--(4,-2)--(0,-2) node[below left]{$-2$};
				%\draw[dashed] (1,0) node[below]{$1$}--(1,1);
				%\draw[dashed] (-0.5,0) node[below left]{$-0{,}5$}--(-0.5,2.125);
				\clip (\xmin+0.1,\ymin+0.1) rectangle (\xmax-0.5,\ymax-0.1);
				\draw[smooth,samples=300][domain=-4:5.5] plot(\x,{0.071*(\x)^3-0.142*(\x)^2-1.07*(\x)});
				\draw[smooth,samples=300][domain=-4:5.5] plot(\x,{(-0.5*(\x)});
			\end{tikzpicture}
		}	Hàm số $g(x)$ nghịch biến $\Rightarrow g'(x) \leq 0 \Rightarrow f'(t) \geq-\dfrac{t}{2}\Rightarrow\hoac{&-2 \leq t \leq 0 \\&t \geq 4.}$\\
		Như vậy $f'(1-2 x) \geq \dfrac{1-2 x}{-2}\Rightarrow\hoac{&-2 \leq 1-2 x \leq 0 \\ &4 \leq 1-2 x}\Rightarrow\hoac{&\dfrac{1}{2}\leq x \leq \dfrac{3}{2}\\ &x \leq-\dfrac{3}{2}.}$\\
		Vậy hàm số $g(x)=f(1-2 x)+x^2-x$ nghịch biến trên các khoảng $\left(\dfrac{1}{2}; \dfrac{3}{2}\right)$ và $\left(-\infty ;-\dfrac{3}{2}\right)$.\\
		Mà $\left(1 ; \dfrac{3}{2}\right) \subset \left(\dfrac{1}{2}; \dfrac{3}{2}\right)$ nên hàm số $g(x)=f(1-2 x)+x^2-x$ nghịch biến trên khoảng $\left(1 ; \dfrac{3}{2}\right)$.
	}
\end{ex}
\begin{ex}[Chuyên Lê Quý Đôn Điện Biên 2019]%[2D1G1-2]
	Cho hàm số $f(x)$ có bảng xét dấu của đạo hàm như sau
	\begin{center}
		\begin{tikzpicture}
			\tkzTabInit[nocadre,lgt=1.2,espcl=2,deltacl=0.6]
			{$x$ /0.6,$f'(x)$ /0.6}
			{$-\infty$,$0$,$1$,$2$,$3$,$+\infty$}
			\tkzTabLine{,+,$0$,-,$0$,-,$0$,+,$0$,-,}
		\end{tikzpicture}
	\end{center}
	Hàm số $y=f(x-1)+x^3-12 x+2019$ nghịch biến trên khoảng nào dưới đây?
	\choice
	{$(1 ;+\infty)$}
	{\True $(1 ; 2)$}
	{$(-\infty ; 1)$}
	{$(3 ; 4)$}
	\loigiai{
		$y'=f'(x-1)+3 x^2-12=f'(t)+3 t^2+6 t-9=f'(t)-\left(-3 t^2-6 t+9\right)$, với $t=x-1$.\\
		\immini{
			Nghiệm của phương trình $y'=0$ là hoành độ giao điểm của các đồ thị hàm số $y=f'(t)$ và $y=-3 t^2-6 t+9$.\\
			Vẽ đồ thị hàm số $y=f'(t)$ và $y=-3 t^2-6 t+9$ trên cùng một hệ trục tọa độ như hình vẽ bên.
		}	
		{		\begin{tikzpicture}[scale=0.5,>=stealth, font=\footnotesize, line join=round, line cap=round]
				\def\a{-3} \def\b{-6} \def\c{9} % Hệ số
				\def\xmin{-9} \def\xmax{7}
				\def\ymin{-3} \def\ymax{13}
				
				%\draw[color=gray!50,dashed] (\xmin,\ymin) grid (\xmax,\ymax);
				
				\draw[->] (\xmin,0)--(\xmax,0) node [below]{$x$};
				\draw[->] (0,\ymin)--(0,\ymax) node [left]{$y$};
				\node at (0,0) [below left]{$O$};
				\clip (\xmin+0.1,\ymin+0.1) rectangle (\xmax-0.5,\ymax-0.1);
				\draw[smooth,samples=300] plot(\x,{\a*(\x)^2+\b*(\x)+\c});
				\node at (1,0) [above right]{$1$};
				\node at (2,0) [below right]{$2$};
				\node at (3,0) [below right]{$3$};
				\node at (-3,-2) [left]{$y=-3t^2-6t+9$};
				\node at (4,0) [below right]{$f'(x)$};
				\draw (-2.2,10).. controls (-1,1.9) and (-0.5,0.8) .. (0,0);
				%\draw (-2,0).. controls (-1.5,-2) and (-0.5,-0) .. (0,0);
				\draw (0,0).. controls (0.4,-0.6) and (0.6,-0.6) .. (0.8,-0.2);
				\draw (0.8,-0.2).. controls (1,0.25) and (1.1,-0.1) .. (1.4,-0.8);
				\draw (1.4,-0.8).. controls (1.6,-1.1) and (1.7,-0.9) .. (2,0);
				\draw (2,0).. controls (2.4,1.1) and (2.6,1.1) .. (3.5,-1);
			\end{tikzpicture}
		}
		Dựa vào đồ thị trên, ta có bảng xét dấu của hàm số $y'=f'(t)-\left(-3 t^2-6 t+9\right)$ như sau $
		\left(t_0<-1\right)$
		\begin{center}
			\begin{tikzpicture}
				\tkzTabInit[nocadre,lgt=2,espcl=2,deltacl=0.6]
				{$x$ /0.6,$y'$ /0.6}
				{$-\infty$,$t_0$,$1$,$+\infty$}
				\tkzTabLine{,+,$0$,-,$0$,+,}
			\end{tikzpicture}
		\end{center}
		Hàm số nghịch biến trên khoảng $t \in\left(t_0 ; 1\right)$.\\
		Do đó hàm số nghịch biến trên khoảng $x \in(1 ; 2) \subset \left(t_0+1 ; 1\right)$.
	}
\end{ex}


\begin{ex}[Chuyên Phan Bội Châu Nghệ An 2019]%[2D1G1-2]
	Cho hàm số $f(x)$ có bảng xét dấu đạo hàm như sau:
	\begin{center}
		\begin{tikzpicture}
			\tkzTabInit[nocadre,lgt=2,espcl=2,deltacl=0.6]
			{$x$ /0.6,$f'(x)$ /0.6}
			{$-\infty$,$1$,$2$,$3$,$4$,$+\infty$}
			\tkzTabLine{,-,$0$,+,$0$,+,$0$,-,$0$,+,}
		\end{tikzpicture}
	\end{center}
	Hàm số $y=2 f(1-x)+\sqrt{x^2+1}-x$ nghịch biến trên những khoảng nào dưới đây
	\choice
	{$(-\infty ;-2)$}
	{$(-\infty ; 1)$}
	{\True $(-2 ; 0)$}
	{$(-3 ;-2)$}
	\loigiai{
		$y'=-2 f'(1-x)+\dfrac{x}{\sqrt{x^2+1}}-1$. \\
		Có $\dfrac{x}{\sqrt{x^2+1}}-1<0,~ \forall x \in(-2 ; 0)$.\\
		Bảng xét dấu:
		\begin{center}
			\begin{tikzpicture}
				\tkzTabInit[nocadre,lgt=2,espcl=2,deltacl=0.6]
				{$x$ /0.7,$f'(1-x)$ /0.7}
				{$-\infty$,$-3$,$-2$,$-1$,$0$,$+\infty$}
				\tkzTabLine{,+,$0$,-,$0$,+,$0$,+,$0$,-,}
			\end{tikzpicture}
		\end{center}
		$\Rightarrow-2 f'(1-x)<0, ~ \forall x \in(-2 ; 0) \\
		\Rightarrow-2 f'(1-x)+\dfrac{x}{\sqrt{x^2+1}}-1<0, ~\forall x \in(-2 ; 0)$.
	}
\end{ex}
\begin{ex}[Sở Vĩnh Phúc 2019]%[2D1G1-2]
	\immini{
		Cho hàm số bậc bốn $y=f(x)$ có đồ thị của hàm số $y=f'(x)$ như hình vẽ bên.\\
		Hàm số $y=3 f(x)+x^3-6 x^2+9 x$ đồng biến trên khoảng nào trong các khoảng sau đây?
		\choice
		{$(0 ; 2)$}
		{$(-1 ; 1)$}
		{$(1 ;+\infty)$}
		{\True $(-2 ; 0)$}
	}
	{
		\begin{tikzpicture}[scale=0.7,>=stealth, font=\footnotesize, line join=round, line cap=round]
			\def\a{0.21} \def\b{0.88} \def\c{-0.58} \def\d{-3} % Hệ số
			\def\xmin{-5} \def\xmax{5}
			\def\ymin{-4} \def\ymax{3} 
			%\draw[color=gray!50,dashed] (\xmin,\ymin) grid (\xmax,\ymax); 
			\draw[->] (\xmin,0)--(\xmax,0) node [below]{$x$};
			\draw[->] (0,\ymin)--(0,\ymax) node [left]{$y$};
			\node at (0,0) [above left]{$O$};
			\node at (-4,0) [below left]{$-4$};
			\node at (-2,0) [below left]{$-2$};
			\node at (0,-3) [below right]{$-3$};
			\draw[dashed] (2,0) node[above right]{$2$}--(2,1) --(0,1) node[above right]{$1$};
			\clip (\xmin+0.1,\ymin+0.1) rectangle (\xmax-0.5,\ymax-0.1);
			\draw[smooth,samples=300] plot(\x,{\a*(\x)^3+\b*(\x)^2+\c*(\x)+\d});
		\end{tikzpicture}
	}
	
	\loigiai{
		Hàm số $f(x)=a x^4+b x^3+c x^2+d x+e,(a \neq 0)$.
		Có $f'(x)=4 a x^3+3 b x^2+2 c x+d$.\\
		Đồ thị hàm số $y=f'(x)$ đi qua các điểm $(-4 ; 0),(-2 ; 0),(0 ;-3),(2 ; 1)$ nên ta có
		$$\heva{&- 2 5 6 a + 4 8 b - 8 c + d = 0\\
			&- 3 2 a + 1 2 b - 4 c + d = 0\\
			&d = - 3\\
			&3 2 a + 1 2 b + 4 c + d = 1}\Leftrightarrow \heva{&
			a=\dfrac{5}{96}\\
			&b=\dfrac{7}{24}\\
			&c=-\dfrac{7}{24}\\
			&d=-3.}
		$$
		Xét hàm số
		$
		y=3 f(x)+x^3-6 x^2+9 x$\\
		Ta có $ y'=3\left(f'(x)+x^2-4 x+3\right)=3\left(\frac{5}{24}x^3+\frac{15}{8}x^2-\frac{55}{12}x\right)
		$\\
		Ta có $y'=0 \Leftrightarrow\hoac{&x=-11 \\&x=0 \\&x=2.}$ \\
		Xét dấu $y'$, ta được hàm số đã cho đồng biến trên các khoảng $(-11 ; 0)$ và $(2 ;+\infty)$.
	}
\end{ex}
\begin{ex}[Học Mãi 2019]%[2D1K1-2]
	\immini
	{Cho hàm số $y=f(x)$ có đạo hàm trên $\mathbb{R}$. Đồ thị hàm số $y=f'(x)$ như hình bên. Hỏi đồ thị hàm số $y=f(x)-2 x$ có bao nhiêu điểm cực trị?
		\choice
		{$4$}
		{\True $3$}
		{$2$}
		{$1$}
	}
	{
		\begin{tikzpicture}[font=\footnotesize,line join=round, line cap=round,>=stealth,scale=0.8]
			\draw[->] (-3.5,0)--(4,0) node[above] {$x$};
			\draw[->] (0,-3)--(0,4) node[left] {$y$};
			%\fill[black] (-2,0)node[below left]{$-2$} circle (1.2pt) (0,0)node[above right]{$O$} circle (1.2pt) (3,0)node[above]{$3$} circle (1.2pt);
			\draw[dashed] (-2,-2)-- (0,-2) node[right]{$-2$};
			\draw[dashed] (2,0) node[below]{$2$}-- (2,2)--(0,2) node[below left]{$2$};
			\node at (0,0) [below left]{$O$};
			\node at (3,0) [below right]{$3$};
			\draw (-3,2.5).. controls (-2.2,-3) and (-1.8,-3) .. (-1.1,0);
			\draw (-1.1,0).. controls (-0.6,2.5) and (-0.4,2.5) .. (0,2);
			\draw (0,2).. controls (0.7,0.5) and (1.1,0.5) .. (1.5,1.5);
			\draw (1.5,1.5).. controls (2,2.5) and (2.8,2.5) .. (3.5,-2.5);
			%\draw (3,0).. controls (3.3,-0.1) and (3.5,-0.5) .. (3.5,-2);
		\end{tikzpicture}
	}
	\loigiai{
		\immini{
			Đặt $g(x)=f(x)-2 x$.\\
			$\Rightarrow g'(x)=f'(x)-2 .
			$\\
			Vẽ đường thẳng $y=2$.\\
			$\Rightarrow$ phương trình $g'(x)=0$ có $3$ nghiệm bội lẻ.\\
			$\Rightarrow$ đồ thị hàm số $y=f(x)-2 x$ có $3$ điểm cực trị.
		}
		{
			\begin{tikzpicture}[font=\footnotesize,line join=round, line cap=round,>=stealth,scale=0.8]
				\draw[->] (-3.5,0)--(4,0) node[above] {$x$};
				\draw[->] (0,-3)--(0,4) node[left] {$y$};
				%\fill[black] (-2,0)node[below left]{$-2$} circle (1.2pt) (0,0)node[above right]{$O$} circle (1.2pt) (3,0)node[above]{$3$} circle (1.2pt);
				\draw[dashed] (-2,-2)-- (0,-2) node[right]{$-2$};
				\draw[dashed] (2,0) node[below]{$2$}-- (2,2)--(0,2) node[below left]{$2$};
				\node at (3,0) [below left]{$3$};
				\draw (-3,2.5).. controls (-2.2,-3) and (-1.8,-3) .. (-1.1,0);
				\draw (-1.1,0).. controls (-0.6,2.5) and (-0.4,2.5) .. (0,2);
				\draw (0,2).. controls (0.7,0.5) and (1.1,0.5) .. (1.5,1.5);
				\draw (1.5,1.5).. controls (2,2.5) and (2.8,2.5) .. (3.5,-2.5);
				\draw (-3.5,2)--(4,2) node[above]{$y=2$};
			\end{tikzpicture}
		}
	}
\end{ex}
\begin{ex}[THPT Hoàng Hoa Thám Hưng Yên 2019]%[2D1G1-2]
	\immini{
		Cho hàm số $y=f(x)$ liên tục trên $\mathbb{R}$. Hàm số $y=f'(x)$ có đồ thị như hình vẽ. 
		Hàm số $g(x)=f(x-1)+\dfrac{2019-2018 x}{2018}$ đồng biến trên khoảng nào dưới đây?
		\choice
		{$(2 ; 3)$}
		{$(0 ; 1)$}
		{\True $(-1 ; 0)$}
		{$(1 ; 2)$}
	}
	{
		\begin{tikzpicture}[scale=1, font=\footnotesize, line join=round, line cap=round, >=stealth]
			\tikzset{label style/.style={font=\footnotesize}}
			\draw[->] (-2,0)--(3,0) node[below left] {$x$};
			\draw[->] (0,-2)--(0,3) node[below left] {$y$};
			\draw[fill=black] (0,0) node [above left] {$O$} circle(1pt);
			\fill (1,1) circle(1pt) (-1,1) circle(1pt) (2,1) circle(1pt);
			\foreach \x in {1,2}
			\draw[thin] (\x,1pt)--(\x,-1pt) node [below] {\footnotesize$\x$};
			\foreach \x in {-1}
			\draw[thin] (\x,1pt)--(\x,-1pt) node [below left] {\footnotesize$\x$};
			\foreach \y in {-1}
			\draw[thin] (1pt,\y)--(-1pt,\y) node [right] {\footnotesize$\y$};
			\foreach \y in {1}
			\draw[thin] (1pt,\y)--(-1pt,\y) node [above left] {\footnotesize$\y$};
			\draw[dashed](-1,0)--(-1,1)--(2,1) (1,1)--(1,0) (2,1)--(2,0);
			\begin{scope}
				\clip (-3,-3) rectangle (3,3);
				\draw[name path=(C)] plot[smooth,tension=0.7] coordinates{(-1.15,3)(-0.5,-1.6)(.8,.88)(1.9,0.8)(2.3,3)};
			\end{scope}
		\end{tikzpicture}
	}	\loigiai{
		Ta có $g'(x)=f'(x-1)-1$.\\
		$
		g'(x) \geq 0 \Leftrightarrow f'(x-1)-1 \geq 0 \Leftrightarrow f'(x-1) \geq 1 \Leftrightarrow \hoac{&x - 1 \leq - 1\\
			&x - 1 \geq 2}\Leftrightarrow \hoac{&
			x \leq 0 \\
			&x \geq 3.}
		$\\
		Từ đó suy ra hàm số $g(x)=f(x-1)+\dfrac{2019-2018 x}{2018}$ đồng biến trên khoảng $(-1 ; 0)$.
	}
\end{ex}

\begin{ex}[(Sở Ninh Bình 2019]%[2D1K1-2]
	Cho hàm số $y=f(x)$ có bảng xét dấu của đạo hàm như sau
	\begin{center}
		\begin{tikzpicture}
			\tkzTabInit[nocadre,lgt=1,espcl=2,deltacl=0.6]
			{$x$ /0.7,$f'(x)$ /0.7}
			{$-\infty$,$-2$,$-1$,$2$,$4$,$+\infty$}
			\tkzTabLine{,+,$0$,-,$0$,+,$0$,-,$0$,+,}
		\end{tikzpicture}
	\end{center}
	Hàm số $y=-2 f(x)+2019$ nghịch biến trên khoảng nào trong các khoảng dưới đây?
	\choice
	{$(-4 ; 2)$}
	{\True $(-1 ; 2)$}
	{$(-2 ;-1)$}
	{$(2 ; 4)$}
	\loigiai{
		Xét $y=g(x)=-2 f(x)+2019$.\\
		Ta có $g'(x)=(-2 f(x)+2019)'=-2 f'(x), g'(x)=0 \Leftrightarrow\hoac{&x=-2 \\&x=-1 \\&x=2 \\&x=4.}$.\\
		Ta có bảng xét dấu của $g'(x)$
		\begin{center}
			\begin{tikzpicture}
				\tkzTabInit[nocadre,lgt=1,espcl=2,deltacl=0.6]
				{$x$ /0.6,$f'(x)$ /0.6}
				{$-\infty$,$-2$,$-1$,$2$,$4$,$+\infty$}
				\tkzTabLine{,-,$0$,+,$0$,-,$0$,+,$0$,+,}
			\end{tikzpicture}
		\end{center}
		Dựa vào bảng xét dấu, ta thấy hàm số $y=g(x)$ nghịch biến trên khoảng $(-1 ; 2)$.
	}
\end{ex}
\begin{ex}[THPT Lương Thế Vinh Hà Nội 2019]%[2D1G1-2]
	\immini{
		Cho hàm số $y=f(x)$. Biết đồ thị hàm số $y=f'(x)$ có đồ thị như hình vẽ bên. 
		Hàm số $y=f \left(3-x^2\right)+2018$ đồng biến trên khoảng nào dưới đây?
		\choice
		{\True $(-1 ; 0)$}
		{$(2 ; 3)$}
		{$(-2 ;-1)$}
		{$(0 ; 1)$}
	}
	{
		\begin{tikzpicture}[scale=0.6,>=stealth, font=\footnotesize, line join=round, line cap=round]
			\def\a{0.065} \def\b{0.32} \def\c{-0.53} \def\d{-0.82} % Hệ số
			\def\xmin{-8} \def\xmax{4}
			\def\ymin{-3} \def\ymax{3} 
			%\draw[color=gray!50,dashed] (\xmin,\ymin) grid (\xmax,\ymax); 
			\draw[->] (\xmin,0)--(\xmax,0) node [below]{$x$};
			\draw[->] (0,\ymin)--(0,\ymax) node [left]{$y$};
			\node at (0,0) [below left]{$O$};
			\node at (-6,0) [below left]{$-6$};
			\node at (-1,0) [below left]{$-1$};
			\node at (2,0) [below right]{$2$};
			\clip (\xmin+0.1,\ymin+0.1) rectangle (\xmax-0.5,\ymax-0.1);
			\draw[smooth,samples=300][domain=-6.5:3.5] plot(\x,{\a*(\x)^3+\b*(\x)^2+\c*(\x)+\d});
		\end{tikzpicture}
	}
	
	\loigiai{
		Ta có $\left[f\left( 3-x^2\right)+2018 \right]'=-2 x \cdot f'\left(3-x^2\right) $.\\
		$
		-2 x \cdot f'\left(3-x^2\right)=0 \Leftrightarrow\hoac{&
			x = 0\\
			&3 - x ^{2}= - 6\\
			&3 - x ^{2}= - 1\\
			&3 - x ^{2}= 2}
		\Leftrightarrow \hoac{
			&x=0 \\
			&x=\pm 3 \\
			&x=\pm 2 \\
			&	x=\pm 1.}
		$\\
		Bảng xét dấu của đạo hàm hàm số đã cho
		\begin{center}
			\begin{center}
				\begin{tikzpicture}
					\tkzTabInit[nocadre,lgt=2.9,espcl=1.5,deltacl=0.6]
					{$x$ /0.7,$f'\left( 3-x^2\right) $/0.7,$-2xf'\left( 3-x^2\right)$/0.8}
					{$-\infty$,$-3$,$-2$,$-1$,$0$,$1$,$2$,$3$,$+\infty$}
					\tkzTabLine{,-,$0$,+,$0$,-,$0$,+,$0$,+,$0$,-,$0$,+,$0$,-}
					\tkzTabLine{,-,$0$,+,$0$,-,$0$,+,$0$,-,$0$,+,$0$,-,$0$,+}
				\end{tikzpicture}
			\end{center}
		\end{center}
		Từ bảng xét dấu suy ra hàm số đồng biến trên $(-1 ; 0)$.
	}
\end{ex}
\begin{ex}[Chuyên Biên Hòa - Hà Nam - 2020]%[2D1G1-2]
	\immini{
		Cho hàm số đa thức $f(x)$ có đạo hàm trên $\mathbb{R}$. Biết $f(0)=0$ và đồ thị hàm số $y=f'(x)$ như hình sau.
		Hàm số $g(x)=\left|4 f(x)+x^2\right|$ đồng biến trên khoảng nào dưới đây?
		\choice
		{$(4 ;+\infty)$}
		{\True $(0 ; 4)$}
		{$(-\infty ;-2)$}
		{$(-2 ; 0)$}
	}	
	{
		\begin{tikzpicture}[scale=0.7,>=stealth, font=\footnotesize, line join=round, line cap=round]
			%\def\a{1} \def\b{-6} \def\c{9} \def\d{1} % Hệ số
			\def\xmin{-4} \def\xmax{6}
			\def\ymin{-3} \def\ymax{2} 
			%\draw[color=gray!50,dashed] (\xmin,\ymin) grid (\xmax,\ymax); 
			\draw[->] (\xmin,0)--(\xmax,0) node [below]{$x$};
			\draw[->] (0,\ymin)--(0,\ymax) node [left]{$y$};
			\node at (0,0) [below left]{$O$};
			%\node at (1,3) [below left]{$f'(x)$};
			%\node at (-1.3,4) {$f'(x)$};
			\draw[dashed] (-2,0) node[below]{$-2$}--(-2,1)--(0,1) node[below left]{$1$};
			\draw[dashed] (4,0) node[below]{$4$}--(4,-2)--(0,-2) node[below left]{$-2$};
			%\draw[dashed] (1,0) node[below]{$1$}--(1,1);
			%\draw[dashed] (-0.5,0) node[below left]{$-0{,}5$}--(-0.5,2.125);
			\clip (\xmin+0.1,\ymin+0.1) rectangle (\xmax-0.5,\ymax-0.1);
			\draw[smooth,samples=300][domain=-4:5.5] plot(\x,{0.071*(\x)^3-0.142*(\x)^2-1.07*(\x)});
		\end{tikzpicture}
	}
	\loigiai{
		\immini{
			Xét hàm số $h(x)=4 f(x)+x^2$ trên $\mathbb{R}$.\\
			Vì $f(x)$ là hàm số đa thức nên $h(x)$ cũng là hàm số đa thức và $h(0)=4 f(0)=0$.\\
			Ta có $h'(x)=4 f'(x)+2 x$. Do đó $h'(x)=0 \Leftrightarrow f'(x)=-\dfrac{1}{2}x$.\\
		}
		{
			\begin{tikzpicture}[scale=0.7,>=stealth, font=\footnotesize, line join=round, line cap=round]
				%\def\a{1} \def\b{-6} \def\c{9} \def\d{1} % Hệ số
				\def\xmin{-4} \def\xmax{6}
				\def\ymin{-3} \def\ymax{2} 
				%\draw[color=gray!50,dashed] (\xmin,\ymin) grid (\xmax,\ymax); 
				\draw[->] (\xmin,0)--(\xmax,0) node [below]{$x$};
				\draw[->] (0,\ymin)--(0,\ymax) node [left]{$y$};
				\node at (0,0) [below left]{$O$};
				%\node at (1,3) [below left]{$f'(x)$};
				%\node at (-1.3,4) {$f'(x)$};
				\draw[dashed] (-2,0) node[below]{$-2$}--(-2,1)--(0,1) node[below left]{$1$};
				\draw[dashed] (4,0) node[below]{$4$}--(4,-2)--(0,-2) node[below left]{$-2$};
				%\draw[dashed] (1,0) node[below]{$1$}--(1,1);
				%\draw[dashed] (-0.5,0) node[below left]{$-0{,}5$}--(-0.5,2.125);
				\clip (\xmin+0.1,\ymin+0.1) rectangle (\xmax-0.5,\ymax-0.1);
				\draw[smooth,samples=300][domain=-4:5.5] plot(\x,{0.071*(\x)^3-0.142*(\x)^2-1.07*(\x)});
				\draw[smooth,samples=300][domain=-4:5.5] plot(\x,{-0.5*(\x)});
			\end{tikzpicture}
		}
		Dựa vào sự tương giao của đồ thị hàm số $y=f'(x)$ và đường thẳng $y=-\dfrac{1}{2}x$, ta có
		$
		h'(x)=0 \Leftrightarrow x \in\{-2 ; 0 ; 4\}.\\
		$
		Bảng biến thiên của hàm số $h(x)$ như sau:
		\begin{center}
			\begin{tikzpicture}
				\tkzTabInit[nocadre,lgt=1.2,espcl=2.5,deltacl=0.6]
				{$x$ /0.6,$y'$ /0.6,$y$ /2}
				{$-\infty$,$-2$,$0$,$4$,$+\infty$}
				\tkzTabLine{,-,$0$,+,$0$,-,$0$,+,}
				\tkzTabVar{+/$+\infty$, -/$y_1$,+/$0$,-/$y_3$,+/$+\infty$}
			\end{tikzpicture}
		\end{center}
		Từ đó suy ra bảng biến thiên của hàm số $g(x)=|h(x)|$.\\
		Dựa vào bảng biến thiên trên, ta thấy hàm số $g(x)$ đồng biến trên khoảng $(0 ; 4)$.
	}
\end{ex}
\begin{ex}[Chuyên Thái Bình - 2020]%[2D1G1-2]
	\immini{
		Cho hàm số $f(x)$ liên tục trên $\mathbb{R}$ có đồ thị hàm số $y=f'(x)$ cho như hình vẽ bên.\\
		Hàm số $g(x)=2 f(|x-1|)-x^2+2 x+2020$ đồng biến trên khoảng nào?
		\choice
		{\True $(0 ; 1)$}
		{$(-3 ; 1)$}
		{$(1 ; 3)$}
		{$(-2 ; 0)$}
	}
	{
		\begin{tikzpicture}[scale=0.7,>=stealth, font=\footnotesize, line join=round, line cap=round]
			%\def\a{1} \def\b{-6} \def\c{9} \def\d{1} % Hệ số
			\def\xmin{-4} \def\xmax{5}
			\def\ymin{-3} \def\ymax{5} 
			%\draw[color=gray!50,dashed] (\xmin,\ymin) grid (\xmax,\ymax); 
			\draw[->] (\xmin,0)--(\xmax,0) node [below]{$x$};
			\draw[->] (0,\ymin)--(0,\ymax) node [left]{$y$};
			\node at (0,0) [below left]{$O$};
			%\node at (1,3) [below left]{$f'(x)$};
			\node at (-1.3,4) {$f'(x)$};
			\draw[dashed] (-1,0) node[above]{$-1$}--(-1,-1)--(0,-1) node[below left]{$-1$};
			\draw[dashed] (1,0) node[below]{$1$}--(1,1)--(0,1) node[below left]{$1$};
			\draw[dashed] (3,0) node[below]{$3$}--(3,3)--(0,3) node[below left]{$3$};
			%\draw[dashed] (1,0) node[below]{$1$}--(1,1);
			%\draw[dashed] (-0.5,0) node[below left]{$-0{,}5$}--(-0.5,2.125);
			\clip (\xmin+0.1,\ymin+0.1) rectangle (\xmax-0.5,\ymax-0.1);
			\draw[smooth,samples=300][domain=-2:4] plot(\x,{-0.5*(\x)^3+1.5*(\x)^2+1.5*(\x)-1.5});
			%\draw[smooth,samples=300] plot(\x,{(\x)^3+(\x)^2-2*(\x)+1});
		\end{tikzpicture}
	}
	\loigiai{
		Ta có đường thẳng $y=x$ cắt đồ thị hàm số $y=f'(x)$ tại các điểm $x=-1 ; x=1 ; x=3$ như hình vẽ sau:
		\begin{center}
			\begin{tikzpicture}[scale=0.7,>=stealth, font=\footnotesize, line join=round, line cap=round]
				%\def\a{1} \def\b{-6} \def\c{9} \def\d{1} % Hệ số
				\def\xmin{-4} \def\xmax{5}
				\def\ymin{-3} \def\ymax{5} 
				%\draw[color=gray!50,dashed] (\xmin,\ymin) grid (\xmax,\ymax); 
				\draw[->] (\xmin,0)--(\xmax,0) node [below]{$x$};
				\draw[->] (0,\ymin)--(0,\ymax) node [left]{$y$};
				\node at (0,0) [below left]{$O$};
				%\node at (1,3) [below left]{$f'(x)$};
				\node at (-1.3,4) {$f'(x)$};
				\node at (4,3.2) {$y=x$};
				\draw[dashed] (-1,0) node[above]{$-1$}--(-1,-1)--(0,-1) node[below left]{$-1$};
				\draw[dashed] (1,0) node[below]{$1$}--(1,1)--(0,1) node[below left]{$1$};
				\draw[dashed] (3,0) node[below]{$3$}--(3,3)--(0,3) node[below left]{$3$};
				%\draw[dashed] (1,0) node[below]{$1$}--(1,1);
				%\draw[dashed] (-0.5,0) node[below left]{$-0{,}5$}--(-0.5,2.125);
				\clip (\xmin+0.1,\ymin+0.1) rectangle (\xmax-0.5,\ymax-0.1);
				\draw[smooth,samples=300][domain=-2:4] plot(\x,{-0.5*(\x)^3+1.5*(\x)^2+1.5*(\x)-1.5});
				\draw[smooth,samples=300] plot(\x,{(\x)});
			\end{tikzpicture}
		\end{center}
		Dựa vào đồ thị của hai hàm số trên ta có $f'(x)>x \Leftrightarrow\hoac{&x<-1 \\ &1<x<3}$ và
		$ f'(x)<x \Leftrightarrow\hoac{&
			-1<x<1 \\
			&x>3.}$\\
		+Trường hợp 1: $x-1<0 \Leftrightarrow x<1$.\\
		Khi đó $g(x)=2 f(1-x)-x^2+2 x+2020$.\\
		Ta có $g'(x)=-2 f'(1-x)+2(1-x)$.
		$$
		g'(x)>0 \Leftrightarrow-2 f'(1-x)+2(1-x)>0 \Leftrightarrow f'(1-x)<1-x \Leftrightarrow\hoac{
			&- 1 < 1 - x < 1\\
			&1 - x > 3} \Leftrightarrow \hoac{&
			0<x<2 \\
			&x<-2.}
		$$
		Kết hợp điều kiện, ta có $g'(x)>0 \Leftrightarrow\hoac{&0<x<1 \\ &x<-2.}$\\
		
		+ Trường hợp 2: $x-1>0 \Leftrightarrow x>1$.\\
		Khi đó ta có $g(x)=2 f(x-1)-x^2+2 x+2020$.\\
		$ g'(x)=2 f'(x-1)-2(x-1)$\\
		$g'(x)>0 \Leftrightarrow 2 f'(x-1)-2(x-1)>0 \Leftrightarrow f'(x-1)>x-1 \Leftrightarrow\hoac{&
			x - 1 < - 1\\
			&1 < x - 1 < 3}\Leftrightarrow \hoac{
			&x<0 \\
			&2<x<4.}$
		Kết hợp điều kiện ta có $g'(x)>0 \Leftrightarrow 2<x<4$.\\
		Vậy hàm số $g(x)=2 f(|x-1|)-x^2+2 x+2020$ đồng biến trên khoảng $(0 ; 1)$.
	}
\end{ex}

\begin{ex}[Chuyên Lào Cai - 2020]%[2D1G1-2]
	\immini{
		Cho hàm số $f'(x)$ có đồ thị như hình bên.\\
		Hàm số $g(x)=f(3 x+1)+9 x^3+\dfrac{9}{2}x^2$ đồng biến trên khoảng nào dưới đây?
		\choice
		{$(-1 ; 1)$}
		{$(-2 ; 0)$}
		{$(-\infty ; 0)$}
		{\True $(1 ;+\infty)$}
	}
	{\begin{tikzpicture}[line join=round, line cap=round,>=stealth,thick,scale=.8]
			\tikzset{label style/.style={font=\footnotesize}}
			\draw[->] (-2.1,0)--(5.1,0) node[below left] {$x$};
			\draw[->] (0,-3.1)--(0,4.1) node[below left] {$y$};
			\draw (0,0) node [below left] {$O$};
			\foreach \x in {1,2,3}
			\draw[thin] (\x,1pt)--(\x,-1pt) node [below] {$\x$};
			\draw[thin](-1,1pt)--(1,-1pt)node[above left]{$-1$};
			\foreach \y in {-2,2}
			\draw[thin] (1pt,\y)--(-1pt,\y) node [above right] {$\y$};
			%\begin{scope}
			\clip (-2,-3) rectangle (5,4);
			\draw[samples=200,domain=-2:4,smooth,variable=\x] plot (\x,{(\x)^3-3*(\x)^2+2});
			%\end{scope}
			\draw[dashed] (-1,0)--(-1,-2)--(0,-2);
			\draw[dashed] (3,0)--(3,2)--(0,2);
			%\begin{scope}[on background layer]\path[white]node{MDD-134};\end{scope}
		\end{tikzpicture}
	}
	\loigiai
	{
		\immini{Xét hàm số $g(x)=f(3 x+1)+9 x^3+\dfrac{9}{2}x^2 \\
			\Rightarrow g'(x)=3 f'(3 x+1)+27 x^2+9 x$.\\
			Hàm số đồng biến  $\Leftrightarrow g'(x)>0 \Leftrightarrow 3 f'(3 x+1)+27 x^2+9 x>0$
			\\
			$
			\Leftrightarrow f'(3 x+1)+3 x(3 x+1)>0 \qquad (*)
			$\\
			Đặt $t=3 x+1$, khi đó  $(*) \Leftrightarrow f'(t)+(t-1) t>0$\\ $\Leftrightarrow f'(t)>-t^2+t$.\\
			Vẽ parabol $y=-x^2+x$ và đồ thị hàm số $f'(x)$ trên cùng một hệ trục
		}
		{
			\begin{tikzpicture}[line join=round, line cap=round,>=stealth,thick,scale=.8]
				\tikzset{label style/.style={font=\footnotesize}}
				\draw[->] (-2.1,0)--(5.1,0) node[below left] {$x$};
				\draw[->] (0,-3.1)--(0,4.1) node[below left] {$y$};
				\draw (0,0) node [below left] {$O$};
				\foreach \x in {1,2,3}
				\draw[thin] (\x,1pt)--(\x,-1pt) node [below] {$\x$};
				\draw[thin](-1,1pt)--(1,-1pt);
				\foreach \y in {-2,2}
				\draw[thin] (1pt,\y)--(-1pt,\y) node [above right] {$\y$};
				%\begin{scope}
				\clip (-2,-3) rectangle (5,4);
				\draw[samples=200,domain=-2:4,smooth,variable=\x] plot (\x,{(\x)^3-3*(\x)^2+2});
				\draw[samples=200,domain=-2:4,smooth,variable=\x] plot (\x,{-(\x)^2+(\x)});
				%\end{scope}
				\draw[dashed] (-1,0) node[above left]{$-1$}--(-1,-2)--(0,-2);
				\draw[dashed] (3,0)--(3,2)--(0,2);
				%\begin{scope}[on background layer]\path[white]node{MDD-134};\end{scope}
			\end{tikzpicture}
		}
		Dựa vào đồ thị ta thấy
		$
		f'(t)>-t^2+t \Leftrightarrow\hoac{&- 1 < t < 1\\
			&t > 2}\Rightarrow \hoac{&
			- 1 < 3 x + 1 < 1\\
			&3 x + 1 > 2} \Leftrightarrow \hoac{&
			\dfrac{-2}{3}<x<0\\
			&x>\dfrac{1}{3}.}
		$}
\end{ex}
\begin{ex}[Sở Phú Thọ-2020]%[2D1G1-2]
	\immini{
		Cho hàm số $y=f(x)$ có đồ thị hàm số $y=f'(x)$ như hình vẽ.\\
		Hàm số $g(x)=f\left(\mathrm{e}^x-2\right)-2020$ nghịch biến trên khoảng nào dưới đây?
		\choice
		{\True $\left(-1 ; \dfrac{3}{2}\right)$}
		{$(-1 ; 2)$}
		{$(0 ;+\infty)$}
		{$\left(\dfrac{3}{2}; 2\right)$}
	}
	{
		\begin{tikzpicture}[scale=0.7,>=stealth, font=\footnotesize, line join=round, line cap=round]
			\def\a{1} \def\b{-3} \def\c{0} \def\d{0} % Hệ số
			\def\xmin{-2} \def\xmax{4}
			\def\ymin{-5} \def\ymax{2} 
			%\draw[color=gray!50,dashed] (\xmin,\ymin) grid (\xmax,\ymax); 
			\draw[->] (\xmin,0)--(\xmax,0) node [below]{$x$};
			\draw[->] (0,\ymin)--(0,\ymax) node [left]{$y$};
			\node at (0,0) [above left]{$O$};
			\node at (3,0) [below right]{$3$};
			\draw[dashed] (2,0) node[above]{$2$}--(2,-4) --(0,-4) node[left]{$-4$};
			\clip (\xmin+0.1,\ymin+0.1) rectangle (\xmax-0.5,\ymax-0.1);
			\draw[smooth,samples=300] plot(\x,{\a*(\x)^3+\b*(\x)^2+\c*(\x)+\d});
		\end{tikzpicture}
	}
	
	\loigiai{
		Dựa vào đồ thị hàm số $y=f'(x)$ suy ra $f'(x) \leq 0 ~ \forall x<3$ và $f'(x)>0 ~ \forall x>3$.
		$
		g'(x)=\mathrm{e}^x f'\left(\mathrm{e}^x-2\right) .
		$
		Hàm số $g(x)=f\left(\mathrm{e}^x-2\right)-2020$ nghịch biến \\ $ \Leftrightarrow g'(x)<0 \Leftrightarrow \mathrm{e}^x f'\left(\mathrm{e}^x-2\right)<0$\\
		$
		\Leftrightarrow f'\left(\mathrm{e}^x-2\right)<0 \Leftrightarrow \mathrm{e}^x-2<3 \Leftrightarrow \mathrm{e}^x<5 \Leftrightarrow x<\ln 5 .
		$\\
		Vậy hàm số đã cho nghịch biến trên $\left(-1 ; \dfrac{3}{2}\right)$.
	}
\end{ex}
\begin{ex}[Lý Nhân Tông - Bắc Ninh - 2020]%[2D1G1-2]
	\immini{
		Cho hàm số $f(x)$ có đồ thị hàm số $f'(x)$ như hình vẽ.\\
		Hàm số $y=f(\cos x)+x^2-x$ đồng biến trên khoảng
		\choice
		{$(-2 ; 1)$}
		{$(0 ; 1)$}
		{\True $(1 ; 2)$}
		{$(-1 ; 0)$}
	}
	{
		\begin{tikzpicture}[scale=1,>=stealth, font=\footnotesize, line join=round, line cap=round]
			\def\a{-0.5} \def\b{0} \def\c{1.5} \def\d{0} % Hệ số
			\def\xmin{-3} \def\xmax{4}
			\def\ymin{-2} \def\ymax{2} 
			%\draw[color=gray!50,dashed] (\xmin,\ymin) grid (\xmax,\ymax); 
			\draw[->] (\xmin,0)--(\xmax,0) node [below]{$x$};
			\draw[->] (0,\ymin)--(0,\ymax) node [left]{$y$};
			\node at (0,0) [above left]{$O$};
			\node at (3,0) [below right]{$3$};
			\draw[dashed] (-2,0) node[below]{$-2$}--(-2,1) --(0,1) node[above right]{$1$} --(1,1)--(1,0) node[below]{$1$};
			\draw[dashed] (-1,0) node[below right]{$-1$}--(-1,-1) --(0,-1) node[above right]{$-1$} --(2,-1)--(2,0) node[below right]{$2$};
			\clip (\xmin+0.1,\ymin+0.1) rectangle (\xmax-0.5,\ymax-0.1);
			\draw[smooth,samples=300][domain=-2:2] plot(\x,{\a*(\x)^3+\b*(\x)^2+\c*(\x)+\d});
		\end{tikzpicture}
	}
	\loigiai{
		Đặt  $g(x)=f(\cos x)+x^2-x$.\\
		Ta có $g'(x)=-\sin x \cdot f'(\cos x)+2 x-1$\\
		Vì $\cos x \in[-1 ; 1]$ nên từ đồ thị $f'(x)$ ta suy ra $f'(\cos x) \in[-1 ; 1]$.\\
		Do đó $\left|-\sin x \cdot f'(\cos x)\right| \leq 1, ~\forall x \in \mathbb{R}$.\\
		Ta suy ra $g'(x)=\sin x \cdot f'(\cos x)+2 x-1 \geq-1+2 x-1=2 x-2$
		$\Rightarrow g'(x)>0, ~\forall x>1$.\\
		Vậy hàm số đồng biến trên $(1 ; 2)$.
	}
\end{ex}
\begin{ex}[THPT Nguyễn Viết Xuân - 2020]%[2D1G1-2]
	\immini{
		Cho hàm số $f(x)$. Hàm số $y=f'(x)$ có đồ thị như hình vẽ.\\
		Hàm số $g(x)=f\left(3 x^2-1\right)-\dfrac{9}{2}x^4+3 x^2$ đồng biến trên khoảng nào dưới đây?
		\choice
		{\True $\left(-\dfrac{2 \sqrt{3}}{3}; \dfrac{-\sqrt{3}}{3}\right)$}
		{$\left(0 ; \dfrac{2 \sqrt{3}}{3}\right)$}
		{$(1 ; 2)$}
		{$\left(-\dfrac{\sqrt{3}}{3}; \dfrac{\sqrt{3}}{3}\right)$} 
	}
	{
		\begin{tikzpicture}[scale=0.6,>=stealth, font=\footnotesize, line join=round, line cap=round]
			\def\a{0.25} \def\b{0.25} \def\c{-2} \def\d{0} % Hệ số
			\def\xmin{-5} \def\xmax{4}
			\def\ymin{-5} \def\ymax{5} 
			%\draw[color=gray!50,dashed] (\xmin,\ymin) grid (\xmax,\ymax); 
			\draw[->] (\xmin,0)--(\xmax,0) node [below]{$x$};
			\draw[->] (0,\ymin)--(0,\ymax) node [left]{$y$};
			\node at (0,0) [above left]{$O$};
			%\node at (3,0) [below right]{$3$};
			\draw[dashed] (-4,0) node[below left]{$-4$}--(-4,-4) --(0,-4) node[above right]{$-4$};
			\draw[dashed] (3,0) node[below right]{$3$}--(3,3) --(0,3) node[above right]{$3$};
			\clip (\xmin+0.1,\ymin+0.1) rectangle (\xmax-0.5,\ymax-0.1);
			\draw[smooth,samples=300] plot(\x,{\a*(\x)^3+\b*(\x)^2+\c*(\x)+\d});
		\end{tikzpicture}
	}
	
	\loigiai
	{
		TXĐ: $\mathscr{D}=\mathbb{R}$.\\
		Ta có $g'(x)=6 x f'\left(3 x^2-1\right)-18 x^3+6 x=6 x\left[f'\left(3 x^2-1\right)-3 x^2+1\right]$.\\
		$
		g'(x)=0 \Leftrightarrow\hoac{
			&x = 0\\
			&f '( 3 x ^{2}- 1 ) = 3 x ^{2}- 1}
		\Leftrightarrow \hoac{
			&x = 0\\
			&3 x ^{2}- 1 = - 4 \text{~(vô nghiệm)}\\
			&3 x ^{2}- 1 = 0\\
			&3 x ^{2}- 1 = 3}\Leftrightarrow \hoac{&x=0 \\
			&x=\pm \dfrac{\sqrt{3}}{3}\\
			&x=\pm \dfrac{2 \sqrt{3}}{3}.}
		$\\
		Bảng xét dấu
		\begin{center}
			\begin{tikzpicture}
				\tkzTabInit[nocadre,lgt=1.2,espcl=2.2,deltacl=0.6]
				{$x$ /1.2,$f'(x)$ /0.7}
				{$-\infty$,$-\dfrac{2 \sqrt{3}}{3}$,$-\dfrac{ \sqrt{3}}{3}$,$0$,$\dfrac{\sqrt{3}}{3}$,$\dfrac{2 \sqrt{3}}{3}$,$+\infty$}
				\tkzTabLine{,-,$0$,+,$0$,-,$0$,+,$0$,-,$0$,+,}
			\end{tikzpicture}
		\end{center}
		Vậy hàm số đồng biến trong khoảng $\left(-\dfrac{2 \sqrt{3}}{3}; \dfrac{-\sqrt{3}}{3}\right)$.}
\end{ex}
\begin{ex}[Trần Phú - Quảng Ninh - 2020]%[2D1G1-2]
	Cho hàm số $f(x)$ có bảng xét dấu của đạo hàm như sau
	\begin{center}
		\begin{tikzpicture}
			\tkzTabInit[nocadre,lgt=1.2,espcl=2,deltacl=0.6]
			{$x$ /0.6,$f'(x)$ /0.6}
			{$-\infty$,$-4$,$-1$,$2$,$7$,$+\infty$}
			\tkzTabLine{,+,$0$,-,$0$,+,$0$,-,$0$,+,}
		\end{tikzpicture}
	\end{center}
	Hàm số $y=f(2 x+1)+\dfrac{2}{3}x^3-8 x+5$ nghịch biến trên khoảng nào dưới đây?
	\choice
	{$(-\infty ;-2)$}
	{$(1 ;+\infty)$}
	{$(-1 ; 7)$}
	{\True $\left(-1 ; \dfrac{1}{2}\right)$}
	\loigiai{
		Ta có $y'=2 f'(2 x+1)+2 x^2-8$.\\
		Xét $y'\leq 0 \Leftrightarrow 2 f'(2 x+1)+2 x^2-8 \leq 0 \Leftrightarrow f'(2 x+1) \leq 4-x^2$.\\
		Đặt $t=2x+1$, ta có $f'(t) \leq \dfrac{-t^2+2 t+15}{4}$.\\
		Vì $\dfrac{-t^2+2 t+15}{4}\geq 0, \forall t \in[-3 ; 5]$.\\
		Mà $f'(t) \leq 0, \forall t \in[-3 ; 2]$.\\
		Nên $f'(t) \leq \dfrac{-t^2+2 t+15}{4}\Rightarrow t \in[-3 ; 2]$.\\
		Suy ra $-3 \leq 2 x+1 \leq 2 \Leftrightarrow-2 \leq x \leq \dfrac{1}{2}$.}
\end{ex}

\begin{ex}[Chuyên Thái Bình - Lần 3 - 2020]%[2D1G1-2]
	\immini{
		Cho hàm số $y=f(x)$ liên tục trên $\mathbb{R}$ có đồ thị hàm số $y=f'(x)$ cho như hình vẽ.\\
		Hàm số $g(x)=2 f(|x-1|)-x^2+2 x+2020$ đồng biến trên khoảng nào?
		\choice
		{\True $(0 ; 1)$}
		{$(-3 ; 1)$}
		{$(1 ; 3)$}
		{$(-2 ; 0)$}
	}
	{
		\begin{tikzpicture}[scale=0.7,>=stealth, font=\footnotesize, line join=round, line cap=round]
			\def\a{-0.333} \def\b{1} \def\c{1.333} \def\d{-1} % Hệ số
			\def\xmin{-3} \def\xmax{5}
			\def\ymin{-3} \def\ymax{5} 
			%\draw[color=gray!50,dashed] (\xmin,\ymin) grid (\xmax,\ymax); 
			\draw[->] (\xmin,0)--(\xmax,0) node [below]{$x$};
			\draw[->] (0,\ymin)--(0,\ymax) node [left]{$y$};
			\node at (0,0) [above left]{$O$};
			%\node at (3,0) [below right]{$3$};
			\draw[dashed] (-1,0) node[above]{$-1$}--(-1,-1) --(0,-1) node[above right]{$-1$};
			\draw[dashed] (1,0) node[below right]{$1$}--(1,1) --(0,1) node[above right]{$1$};
			\draw[dashed] (3,0) node[below right]{$3$}--(3,3) --(0,3) node[above right]{$3$};
			\clip (\xmin+0.1,\ymin+0.1) rectangle (\xmax-0.5,\ymax-0.1);
			\draw[smooth,samples=300] plot(\x,{\a*(\x)^3+\b*(\x)^2+\c*(\x)+\d});
			\draw[smooth,samples=300] plot(\x,{(\x)});
		\end{tikzpicture}
	}
	\loigiai{
		Với $x>1$, ta có $g(x)=2 f(x-1)-(x-1)^2+2021 \Rightarrow g'(x)=2 f'(x-1)-2(x-1)$.\\
		Hàm số đồng biến $\Leftrightarrow 2 f'(x-1)-2(x-1)>0 \Leftrightarrow f'(x-1)>x-1 \quad(*)$.\\
		Đặt $t=x-1$, khi đó $(*) \Leftrightarrow f'(t)>t \Leftrightarrow\hoac{&1<t<3 \\ &t<-1}\Rightarrow\hoac{&2<x<4 \\ &x<0 ~(\text{loại}).}$\\
		Với $x<1$, ta có $g(x)=2 f(1-x)-(1-x)^2+2021 \Rightarrow g'(x)=-2 f'(1-x)+2(1-x)$.\\
		Hàm số đồng biến $\Leftrightarrow-2 f'(1-x)+2(1-x)>0 \Leftrightarrow f'(1-x)<1-x \quad(* *)$.\\
		Đặt $t=1-x$, khi đó $(* *) \Leftrightarrow f'(t)<t \Leftrightarrow\hoac{&-1<t<1 \\ &t>3}\Rightarrow\hoac{&0<x<2 \\ &x<-2}\Rightarrow\hoac{&0<x<1 \\ &x<-2.}$\\
		Vậy hàm số $g(x)$ đồng biến trên các khoảng $(-\infty ;-2),(0 ; 1),(2 ; 4)$.
	}
\end{ex}
\begin{ex}[Sở Phú Thọ - 2020]%[2D1G1-2]
	\immini{
		Cho hàm số $y=f(x)$ có đồ thị hàm số $f'(x)$ như hình vẽ.\\
		Hàm số $g(x)=f\left(1+e^x\right)+2020$ nghịch biến trên khoảng nào dưới đây?
		\choice
		{$(0 ;+\infty)$}
		{$\left(\dfrac{1}{2}; 1\right)$}
		{\True $\left(0 ; \dfrac{1}{2}\right)$}
		{$(-1 ; 1)$}
	}{
		\begin{tikzpicture}[scale=0.7,>=stealth, font=\footnotesize, line join=round, line cap=round]
			\def\a{1} \def\b{-3} \def\c{0} \def\d{0} % Hệ số
			\def\xmin{-2} \def\xmax{4}
			\def\ymin{-5} \def\ymax{2} 
			%\draw[color=gray!50,dashed] (\xmin,\ymin) grid (\xmax,\ymax); 
			\draw[->] (\xmin,0)--(\xmax,0) node [below]{$x$};
			\draw[->] (0,\ymin)--(0,\ymax) node [left]{$y$};
			\node at (0,0) [above left]{$O$};
			\node at (3,0) [below right]{$3$};
			\draw[dashed] (2,0) node[above]{$2$}--(2,-4) --(0,-4) node[left]{$-4$};
			\clip (\xmin+0.1,\ymin+0.1) rectangle (\xmax-0.5,\ymax-0.1);
			\draw[smooth,samples=300] plot(\x,{\a*(\x)^3+\b*(\x)^2+\c*(\x)+\d});
		\end{tikzpicture}
	}
	\loigiai{
		$g'(x)=e^x f'\left(1+e^x\right)$.\\
		Do $e^x>0, \forall x$ nên $g'(x) \leq 0 \Leftrightarrow f'\left(1+e^x\right) \leq 0 \Leftrightarrow 1+e^x \leq 3 \Leftrightarrow x \leq \ln 2$, dấu bằng xảy ra tại hữu hạn điểm.\\
		Nên $g(x)$ nghịch biến trên $(-\infty ; \ln 2)$.\\
		Vì $\left(0 ; \dfrac{1}{2}\right) \subset (-\infty ; \ln 2)$ nên hàm số đã cho nghịch biến trên $\left(0 ; \dfrac{1}{2}\right)$.
	}
\end{ex}

\begin{ex}%[2D1K1-2]
	[THPT Anh Sơn - Nghệ An - 2020]
	Cho hàm số $y=f(x)$ có bảng xét dấu của đạo hàm như sau.
	\begin{center}
		\begin{tikzpicture}
			\tkzTabInit[nocadre,lgt=1.2,espcl=2,deltacl=0.6]
			{$x$ /0.6,$f'(x)$ /0.6}
			{$-\infty$,$-2$,$-1$,$2$,$4$,$+\infty$}
			\tkzTabLine{,+,$0$,-,$0$,+,$0$,-,$0$,+,}
		\end{tikzpicture}
	\end{center}
	Hàm số $y=-2 f(x)+2019$ nghịch biến trên khoảng nào trong các khoảng dưới đây?
	\choice
	{$(2 ; 4)$}
	{$(-4 ; 2)$}
	{$(-2 ;-1)$}
	{\True $(-1 ; 2)$}
	\loigiai{
		Ta có $y'=-2 f'(x)$.\\
		$
		y'=0 \Leftrightarrow-2 f'(x)=0 \Leftrightarrow\hoac{&
			x=-2 \\
			&x=-1 \\
			&x=2 \\
			&x=4.}$\\
		Từ bảng xét dấu của $f'(x)$ ta có
		\begin{center}
			\begin{tikzpicture}
				\tkzTabInit[nocadre,lgt=1,espcl=2,deltacl=0.6]
				{$x$ /0.6,$y'$ /0.6}
				{$-\infty$,$-2$,$-1$,$2$,$4$,$+\infty$}
				\tkzTabLine{,-,$0$,+,$0$,-,$0$,+,$0$,-,}
			\end{tikzpicture}
		\end{center}
		Từ bảng xét dấu ta có hàm số nghịch biến trên khoảng $(-\infty ;-2),(-1 ; 2)$ và $(4 ;+\infty)$.}
\end{ex}

\begin{ex}[THPT Anh Sơn - Nghệ An - 2020]%[2D1G1-2]
	Cho hàm số $f(x)$ xác định và liên tục trên $\mathbb{R}$ và có đạo hàm $f'(x)$ thỏa mãn $f'(x)=(1-x)(x+2) g(x)+2019$ với $g(x)<0, ~\forall x \in \mathbb{R}$ . Hàm số $y=f(1-x)+2019 x+2020$ nghịch biến trên khoảng nào?
	\choice
	{$(1 ;+\infty)$}
	{$(0 ; 3)$}
	{$(-\infty ; 3)$}
	{\True $(3 ;+\infty)$}
	\loigiai{
		Đặt $h(x)=f(1-x)+2019 x+2020$.\\
		Vì hàm số $f(x)$ xác định trên $\mathbb{R}$ nên hàm số $h(x)$ cũng xác định trên $\mathbb{R}$.\\
		Ta có $h'(x)=-f'(1-x)+2019$.\\
		Do $h'(x)=0$ tại hữu hạn điểm nên để tìm khoảng nghịch biến của hàm số $h(x)$, ta tìm các giá trị của $x$ sao cho $h'(x)<0 \Leftrightarrow-f'(1-x)+2019<0$\\ 
		$\Leftrightarrow f'(1-x)-2019>0 \\
		\Leftrightarrow x(3-x) g(1-x)>0 \Leftrightarrow x(3-x)<0(\text{~do~}g(x)<0, \forall x \in \mathbb{R})$\\
		$\Leftrightarrow\hoac{&
			x<0 \\
			&x>3.}$\\
		Vậy hàm số $y=f(1-x)+2019 x+2020$ nghịch biến trên các khoảng $(-\infty ; 0)$ và $(3 ;+\infty).$}
\end{ex}

\begin{ex}%[2D1G1-2]
	Cho hàm số $y=f(x)$ xác định trên $\mathbb{R}$ và có bảng xét dấu đạo hàm như sau:
	\begin{center}
		\begin{tikzpicture}
			\tkzTabInit[nocadre,lgt=2,espcl=2,deltacl=0.6]
			{$x$ /0.6,$f'(x)$ /0.6}
			{$-\infty$,$-1$,$1$,$4$,$+\infty$}
			\tkzTabLine{,-,$0$,+,$0$,-,$0$,+,}
		\end{tikzpicture}
	\end{center}
	Biết $f(x)>2,~ \forall x \in \mathbb{R}$. Xét hàm số $g(x)=f(3-2 f(x))-x^3+3 x^2-2020$. Khẳng định nào sau đây đúng?
	\choice
	{Hàm số $g(x)$ đồng biến trên khoảng $(-2 ;-1)$}
	{Hàm số $g(x)$ nghịch biến trên khoảng $(0 ; 1)$}
	{Hàm số $g(x)$ đồng biến trên khoảng $(3 ; 4)$}
	{\True Hàm số $g(x)$ nghịch biến trên khoảng $(2 ; 3)$}
	\loigiai{
		Ta có $g'(x)=-2 f'(x) f'(3-2 f(x))-3 x^2+6 x$.\\
		Vì $f(x)>2, ~\forall x \in \mathbb{R}$ nên $3-2 f(x)<-1 ~\forall x \in \mathbb{R}$.\\
		Từ bảng xét dấu $f'(x)$ suy ra $f'(3-2 f(x))<0, ~\forall x \in \mathbb{R}$.\\
		Từ đó ta có bảng xét dấu sau:
		\begin{center}
			\begin{tikzpicture}
				\tkzTabInit[nocadre,lgt=4,espcl=1.7,deltacl=0.6]
				{$x$ /0.7,$-f'(x)f'\left( 3-2f(x)\right) $/0.8,$-3x^2+6x$/0.7}
				{$-\infty$,$-1$,$0$,$1$,$2$,$4$,$+\infty$}
				\tkzTabLine{,-,$0$,+,|,+,$0$,-,|,-,$0$,+,}
				\tkzTabLine{,-,|,-,$0$,+,|,+,$0$,-,|,-,}
			\end{tikzpicture}
		\end{center}
		Từ bảng xét dấu trên, loại trừ đáp án suy ra hàm số $g(x)$ nghịch biến trên khoảng $(2 ; 3)$.}
\end{ex}

\begin{ex}%[2D1G1-2]
	Cho hàm số $f(x)$ có bảng biến thiên như sau:
	\begin{center}
		\begin{tikzpicture}
			\tkzTabInit[nocadre,lgt=1.2,espcl=2.5,deltacl=0.6]
			{$x$ /0.7, $f'(x)$ /0.7, $f(x)$ /2.5}
			{$-\infty$,$1$,$2$,$3$,$4$,$+\infty$}
			\tkzTabLine{,+,$0$,-,$0$,+,$0$,-,$0$,+,}
			\tkzTabVar{-/$-\infty$,+/$3$,-/$1$,+/$2$,-/$0$,+/$+\infty$}
		\end{tikzpicture}
	\end{center}
	Hàm số $y=(f(x))^3-3 .(f(x))^2$ nghịch biến trên khoảng nào dưới đây?
	\choice
	{$(1 ; 2)$}
	{$(3 ; 4)$}
	{$(-\infty ; 1)$}
	{\True $(2 ; 3)$}
	\loigiai{
		Ta có $y'=3 \cdot(f(x))^2 \cdot f'(x)-6 \cdot f(x) \cdot f'(x)=3 f(x) \cdot f'(x) \cdot[f(x)-2]. \\
		y'=0 \Leftrightarrow \hoac{&f(x)=0 \Leftrightarrow x \in\left\{x_1, 4 \mid x_1<1\right\}\\
			&f(x)=2 \Leftrightarrow x \in\left\{x_2, x_3, 3, x_4 \mid x_1<x_2<1<x_3<2 ; 4<x_4\right\}\\
			&f'(x)=0 \Leftrightarrow x \in\{1,2,3,4\}.}$\\
		Lập bảng xét dấu ta có
		\begin{center}
			\begin{tikzpicture}
				\tkzTabInit[nocadre,lgt=2,espcl=1.5,deltacl=0.6]
				{$x$ /0.7,$f(x)$ /0.7,$f(x)-2$ /0.7,$f'(x)$/0.7,$y'$/0.7}
				{$-\infty$,$x_1$,$x_2$,$1$,$x_3$,$2$,$3$,$4$,$x_4$,$+\infty$}
				\tkzTabLine{,-,$0$,+,|,+,|,+,|,+,|,+,$0$,+,|,+,|,+,}
				\tkzTabLine{,-,|,-,$0$,+,$0$,+,$0$,-,|,-,$0$,-,|,-,$0$,+}
				\tkzTabLine{,+,|,+,|,+,$0$,-,|,-,$0$,+,$0$,-,$0$,+,|,+}
				\tkzTabLine{,+,$0$,-,$0$,+,$0$,-,$0$,+,$0$,-,$0$,+,$0$,-,$0$,+}
			\end{tikzpicture}
		\end{center}
		
		Do đó hàm số nghịch biến trên khoảng $(2 ; 3)$.
	}
\end{ex}
\begin{ex}%[2D1G1-2]
	Cho hàm số $y=f(x)$ có đồ thị nằm trên trục hoành và có đạo hàm trên $\mathbb{R}$, bảng xét dấu của biểu thức $f'(x)$ như bảng dưới đây.
	\begin{center}
		\begin{tikzpicture}
			\tkzTabInit[nocadre,lgt=1.2,espcl=2,deltacl=0.6]
			{$x$ /0.6,$f'(x)$ /0.6}
			{$-\infty$,$-2$,$-1$,$3$,$+\infty$}
			\tkzTabLine{,-,$0$,+,$0$,-,$0$,+,}
		\end{tikzpicture}
	\end{center}
	Hàm số $y=g(x)=\dfrac{f\left(x^2-2 x\right)}{f\left(x^2-2 x\right)+1}$ nghịch biến trên khoảng nào dưới đây?
	\choice
	{$(-\infty ; 1)$}
	{$\left(-2 ; \dfrac{5}{2}\right)$}
	{\True $(1 ; 3)$}
	{$(2 ;+\infty)$}
	\loigiai{
		$ g'(x)=\dfrac{\left(x^2-2 x\right)'\cdot f'\left(x^2-2 x\right)}{\left(f\left(x^2-2 x\right)+1\right)^2}=\dfrac{(2 x-2) \cdot f'\left(x^2-2 x\right)}{\left(f\left(x^2-2 x\right)+1\right)^2}. \\
		g'(x)=0 \Leftrightarrow\hoac{
			&2 x - 2 = 0\\
			&f '( x ^{2}- 2 x ) = 0}
		\Leftrightarrow \hoac{&x = 1\\
			&x ^{2}- 2 x = - 2\\
			&x ^{2}- 2 x = - 1\\
			&x ^{2}- 2 x = 3}
		\Leftrightarrow \hoac{&x=1 \\
			&x=-1 \\
			&x=3.}
		$\\
		Ta có bảng xét dấu của $g'(x)$
		\begin{center}
			\begin{tikzpicture}
				\tkzTabInit[nocadre,lgt=1.2,espcl=2,deltacl=0.6]
				{$x$ /0.6,$g'(x)$ /0.6}
				{$-\infty$,$-1$,$1$,$3$,$+\infty$}
				\tkzTabLine{,-,$0$,+,$0$,-,$0$,+,}
			\end{tikzpicture}
		\end{center}
		Dựa vào bảng xét dấu ta có hàm số $y=g(x)$ nghịch biến trên các khoảng $(-\infty ;-1)$ và $(1 ; 3)$.}
\end{ex}
\begin{ex}[Liên trường huyện Quảng Xương - Thanh Hóa - 2021]%[2D1G1-2]
	\immini{
		Cho các hàm số $y=f(x)$; $y=g(x)$ liên tục trên $\mathbb{R}$ và có đồ thị các đạo hàm $f'(x) ; g'(x)$ (đồ thị hàm số $y=g'(x)$ là đường đậm hơn) như hình vẽ.\\
		Hàm số $h(x)=f(x-1)-g(x-1)$ nghịch biến trên khoảng nào dưới đây?
		\choice
		{$\left(\dfrac{1}{2}; 1\right)$}
		{$(1 ;+\infty)$}
		{$(2 ;+\infty)$}
		{\True $\left(-1 ; \dfrac{1}{2}\right)$}
	}
	{
		\begin{tikzpicture}[scale=1,>=stealth, font=\footnotesize, line join=round, line cap=round]
			%\def\a{1} \def\b{-6} \def\c{9} \def\d{1} % Hệ số
			\def\xmin{-4} \def\xmax{3}
			\def\ymin{-2} \def\ymax{4} 
			%\draw[color=gray!50,dashed] (\xmin,\ymin) grid (\xmax,\ymax); 
			\draw[->] (\xmin,0)--(\xmax,0) node [below]{$x$};
			\draw[->] (0,\ymin)--(0,\ymax) node [left]{$y$};
			\node at (0,0) [above left]{$O$};
			\node at (1,3) [below left]{$f'(x)$};
			\node at (1.5,3) [below right]{$g'(x)$};
			\draw[dashed] (-2,0) node[above right]{$-2$}--(-2,1);
			\draw[dashed] (1,0) node[below]{$1$}--(1,1);
			\draw[dashed] (-0.5,0) node[below]{$-0{,}5$}--(-0.5,2.125);
			\clip (\xmin+0.1,\ymin+0.1) rectangle (\xmax-0.5,\ymax-0.1);
			\draw[smooth,samples=300][domain=-3:2] plot(\x,{2*(\x)^4+4*(\x)^3-2*(\x)^2-4*(\x)+1});
			\draw[smooth,samples=300,line width=1.2pt] plot(\x,{(\x)^3+(\x)^2-2*(\x)+1});
		\end{tikzpicture}
	}
	
	\loigiai{
		Ta có: $h'(x)=f'(x-1)-g'(x-1)$.\\
		Dựa vào hình vẽ ta có hàm số $h(x)$ nghịch biến\\
		$\Leftrightarrow h'(x)<0 \Leftrightarrow f'(x-1)<g'(x-1)$\\
		$
		\Leftrightarrow\hoac{&- 2 < x - 1 < - \dfrac{1}{2}\\
			&0 < x - 1 < 1}
		\Leftrightarrow \hoac{
			&-1<x<\dfrac{1}{2}\\
			&1<x<2.}$\\
		Do đó hàm số $h(x)$ nghịch biến trên các khoảng $\left(-1 ; \dfrac{1}{2}\right)$ và $(1 ; 2)$.
	}
\end{ex}
\begin{ex}[THPT Quế Võ 1 - Bắc Ninh - 2021] %[2D1G1-2]
	\immini{
		Cho ba hàm số $y=f(x), y=g(x), y=h(x)$. Đồ thị của ba hàm số $y=f'(x), y=g'(x), y=h'(x)$ được cho như hình vẽ.\\
		Hàm số $k(x)=f(x+7)+g(5 x+1)-h\left(4 x+\dfrac{3}{2}\right)$ đồng biến trên khoảng nào dưới đây?
		\choice
		{$\left(-\dfrac{5}{8}; 0\right)$}
		{$\left(\dfrac{5}{8};+\infty\right)$}
		{\True $\left(\dfrac{3}{8}; 1\right)$}
		{$\left(-\dfrac{3}{8}; 1\right)$}
	}
	{
		\begin{tikzpicture}[scale=0.25,>=stealth, font=\footnotesize, line join=round, line cap=round]
			\def\a{-.078} \def\b{1.25} \def\c{0} % Hệ số
			\def\xmin{-4} \def\xmax{25}
			\def\ymin{-8} \def\ymax{18}
			
			%\draw[color=gray!50,dashed] (\xmin,\ymin) grid (\xmax,\ymax);
			
			\draw[->] (\xmin,0)--(\xmax,0) node [below]{$x$};
			\draw[->] (0,\ymin)--(0,\ymax) node [left]{$y$};
			\node at (20,14) [below right]{$y=g'(x)$};
			\node at (18,-2) [below left]{$y=h'(x)$};
			\node at (16,5) [below right]{$y=f'(x)$};
			\node at (0,0) [below left]{$O$};
			\draw[dashed] (3,0) node[below]{$3$}--(3,10)--(0,10) node[left]{$10$};
			\draw[dashed] (8,0) node[below]{$8$}--(8,5)--(0,5) node[left]{$5$};
			\draw[dashed] (4,0) node[below]{$4$}--(4,2)--(0,2) node[left]{$2$};
			\clip (\xmin+0.1,\ymin+0.1) rectangle (\xmax-0.5,\ymax-0.1);
			\draw[smooth,samples=300,domain=-2:18] plot(\x,{\a*(\x)^2+\b*(\x)+\c});
			%\draw[smooth,samples=300,domain=-2:25] plot(\x,{0.02*(\x)^3-0.6*(\x)^2+5.16*(\x)});
			\draw[line width=1.2pt] (-2,5)..controls (1.7,1.5) and (4.5,1.6)..(7,2.6);
			\draw[line width=1.2pt] (7,2.6)..controls (9,3.5) and (12,5)..(20,13);
			\draw (-0.5,-2) -- (0,0)--(3,10).. controls +(65:1) and + (-190:1)..(6,15).. controls +(0:1) and + (-180:1)..(14,-1).. controls +(0:1) and + (+80:1)..(19,16);
			
		\end{tikzpicture}
	}
	\loigiai{
		Ta có $k'(x)=f'(x+7)+5 g'(5 x+1)-4 h'\left(4 x+\dfrac{3}{2}\right)$.\\
		Khi $x \in \left( \dfrac{3}{8};1\right)$ thì $\heva{&7{,}375<x+7<8\\&2{,}875<5x+1<6\\&3<4x+\dfrac{4}{3}<5{,}5}\Leftrightarrow \heva{&f'(x+7)>10\\&g'(5x+1)>2 \Rightarrow 5g'(5x+1)>10  \\&h'\left( 4x+\dfrac{3}{2}\right)<5 \Rightarrow -4h'\left( 4x+\dfrac{3}{2}\right) >-20}.$\\
		Do đó $k'(x)=f'(x+7)+5g'(5x+1)-4h'\left( 4x+\dfrac{3}{2}\right)>0$.\\
		Hàm số $k(x)=f(x+7)+g(5 x+1)-h\left(4 x+\dfrac{3}{2}\right)$ đồng biến trên $\left(\dfrac{3}{8}; 1\right)$.
	}
\end{ex}
\begin{ex}[THPT Thanh Chương 1 - Nghệ An- 2021] %[2D1G1-2]
	Cho hàm số $y=f(x)$ liên tục trên $\mathbb{R}$ có bảng xét dấu đạo hàm như sau
	\begin{center}
		\begin{tikzpicture}
			\tkzTabInit[nocadre,lgt=1.2,espcl=2,deltacl=0.6]
			{$x$ /0.6,$f'(x)$ /0.6}
			{$-\infty$,$1$,$2$,$3$,$4$,$+\infty$}
			\tkzTabLine{,-,$0$,+,$0$,+,$0$,-,$0$,+,}
		\end{tikzpicture}
	\end{center}
	Hàm số $y=3f(2x-1)-4x^3+15x^2-18x+1$ đồng biến trên khoảng nào dưới đây?
	\choice
	{$\left(3;+\infty\right)$}
	{\True $\left(1;\dfrac{3}{2}\right)$}
	{$\left(\dfrac{5}{2}; 3\right)$}
	{$\left(2;\dfrac{5}{2}\right)$}
	\loigiai{
		Ta có $y'=6f'(2x-1)-12x^2+30x-18=6\left[f'(2x-1)-2x^2+5x-3\right] $.\\
		Có $f'(2x-1)=0 \Leftrightarrow \hoac{&2x-1=1\\&2x-1=2\\&2x-1=3\\&2x-1=4} \Leftrightarrow \hoac{&x=1\\&x=\dfrac{3}{2}\\&x=2\\&x=\dfrac{5}{2}.}$
		Ta có bảng xét dấu sau
		\begin{center}
			\begin{tikzpicture}
				\tkzTabInit[nocadre,lgt=3.0,espcl=1.5,deltacl=0.6]
				{$x$ /1.0,$f(x)$ /0.6,$f'(2x-1)$ /0.6,$-2x^2+5x-3$/0.6,$g'(x)$/0.6}
				{$-\infty$,$1$,$\dfrac{3}{2}$,$2$,$\dfrac{5}{2}$,$3$,$4$,$+\infty$}
				\tkzTabLine{,-,$0$,+,|,+,$0$,+,|,+,$0$,-,$0$,+,}
				\tkzTabLine{,-,$0$,+,$0$,+,$0$,-,$0$,+,|,+,|,+,}
				\tkzTabLine{,-,$0$,+,$0$,-,|,-,|,-,|,-,|,-,}
				\tkzTabLine{,-,$0$,+,$0$,,?,,|,,?,?,,?,}
			\end{tikzpicture}
		\end{center}
		Dựa vào bảng xét dấu trên, ta kết luận hàm số đã cho đồng biến trên khoảng $\left( 1; \dfrac{3}{2}\right).$
	}
\end{ex}


\begin{ex}%[2D2G4-3] %Câu 27 
	[THPT Hoàng Hoa Thám-Đà Nẵng-2021]
	Cho hàm số $f(x)$ có bảng xét dấu của $f'(x)$ như sau:\\
	\begin{center}
		\begin{tikzpicture}
			\tkzTabInit[lgt=1.2,espcl=2.3]
			{$x$/0.7, $f'(x)$ /.8} % first column
			{$-\infty$,$-3$,$1$, $2$, $+\infty$} % first row
			\tkzTabLine { ,+,0,-,0,+,0,+ }
		\end{tikzpicture}
	\end{center}	
	Hàm số $y=f\left(2-e^x\right)-\dfrac{1}{3}{e^{3x}}+3e^{2x}-5e^x+1$ đồng biến trên khoảng nào dưới đây?
	\choice
	{$\left(0;\dfrac{3}{2}\right)$}
	{$\left(1;3\right)$}
	{\True $\left(-3;0\right)$}
	{$\left(-4;-3\right)$}
	\loigiai{
		Ta có $y'=-e^x.f'\left(2-e^x\right)-e^{3x}+6e^{2x}-5e^x=e^x\left[-f'\left(2-e^x\right)-e^{2x}+6e^x-5\right]$ .\\
		Đặt $t=2-e^x$, ta được\\
		$y'=\left(2-t\right)\left[-f'(t)-\left(2-t\right)^2+6\left(2-t\right)-5\right]=\left(2-t\right)\left[-f'(t)-t^2-2t+3\right]$ .\\
		$y'=0\Leftrightarrow\left(2-t\right)\left[-f'(t)-t^2-2t+3\right]=0\Leftrightarrow
		\hoac{
			& t=2\\ 
			& f'(t)=-t^2-2t+3.}$\\
		Hàm số $g(x)=-x^2-2x+3$ là parabol có trục đối xứng $x=-1$ và cắt trục hoành tại 2 điểm có hoành độ 
		$\hoac{
			& x=1\\ 
			& x=-3
		}$. Suy ra $f'(t)=-t^2-2t+3\Leftrightarrow \hoac{
			& t=1\\ 
			& t=-3. }$\\
		Bảng xét dấu\\
		\begin{center}
			\begin{tikzpicture}
				\tkzTabInit[lgt=3.9,espcl=2,nocadre]
				{$t$/0.7, $2-t$ /0.8, $-f'(t)-t^2-2t+3$ /0.8, $y'$ /0.8} % first column
				{$-\infty$,$-3$,$1$,$2$,$+\infty$} % first row
				\tkzTabLine { ,+,|,+,|,+,z,-, } % second row
				\tkzTabLine {,-,0,+,0,-,|,-,} % third row
				\tkzTabLine {,-,0,+,0,-,0,+,} % last row
			\end{tikzpicture}
		\end{center}
		Dựa vào bảng xét dấu $y'>0,\forall x\in\left(-3;0\right)$.}
\end{ex}


\begin{ex}%[2D1G1-2]%Câu 28 
	[Sở Lạng Sơn 2022] Cho hàm số $f(x)$ có bảng biến thiên như sau:\\
	\begin{center}
		\begin{tikzpicture}
			\tkzTabInit[espcl=2.5,lgt=1,nocadre]
			{$x$/0.7,$y'$/0.7,$y$/3.5}
			{$-\infty$,$1$,$2$,$3$,$4$,$+\infty$}
			\tkzTabLine{,+,0,-,0,+,0,-,0,+,}
			\node (0) at ($(N12)+(0,-3)$) {$-\infty$};
			\node (1) at ($(N22)+(0,-.5)$) {$3$};
			\node (2) at ($(N32)+(0,-1.7)$) {$1$};
			\node (3) at ($(N42)+(0,-0.7)$) {$2$};
			\node (4) at ($(N52)+(0,-2.3)$) {$0$};
			\node (5) at ($(N62)+(0,-.3)$) {$+\infty$};
			%				\node (8) at ($(N42)+(0,-.5)$) {};
			%				\coordinate (9) at ($(N42)!.6!(N53)+ (-0.5,0)$);
			%				\coordinate (6) at ($(T12)!.6!(T13)$);
			%				\coordinate (7) at ($(T22)!.6!(T23)$);
			\draw[-stealth] (0)--(1);
			\draw[-stealth] (1)--(2);
			\draw[-stealth] (2)--(3);
			\draw[-stealth] (1)--(2);
			\draw[-stealth] (3)--(4);
			\draw[-stealth] (4)--(5);
			%				\draw[->,red] (5)--(8);
			%				\draw[->,red] (8)--(9);
			%				\draw[blue,dashed](6)--(7)node[above left]{$y=0$};
		\end{tikzpicture}		
	\end{center}
	Hàm số $y=\left[f(x)\right]^3-3\left[f(x)\right]^2$ đồng biến trên khoảng nào dưới đây?
	\choice
	{$\left(-\infty\,;1\right)$}
	{$\left(1\,;2\right)$}
	{\True $\left(3\,;4\right)$}
	{$\left(2\,;3\right)$}
	\loigiai{
		Ta có $y'=3f'(x)\left[f^2(x)-2f(x)\right]$. 
		Phương trình $y'=0\Leftrightarrow \hoac{
			&{f}'(x)=0\\ 
			& f(x)=0\\ 
			& f(x)=2.
		}$
		\begin{center}
			\begin{tikzpicture}
				\tkzTabInit[espcl=2.5,lgt=1.5]
				{$x$/0.7,$y'$/0.7,$y$/3.5}
				{$-\infty$,$1$,$2$,$3$,$4$,$+\infty$}
				\tkzTabLine{,+,0,-,0,+,0,-,0,+,}
				\node (0) at ($(N12)+(0,-3)$) {$-\infty$};
				\node (1) at ($(N22)+(0,-.3)$) {$3$};
				\node (2) at ($(N32)+(0,-1.7)$) {$1$};
				\node (3) at ($(N42)+(0,-0.8)$) {$2$};
				\node (4) at ($(N52)+(0,-2.3)$) {$0$};
				\node (5) at ($(N62)+(0,-.3)$) {$+\infty$};
				\node (a) at ($(N11)+(0.65,0.35)$) {$a$};
				\node (b) at ($(N11)+(2.0,0.4)$) {$b$};
				\node (c) at ($(N11)+(3.38,0.35)$) {$c$};
				\node (d) at ($(N11)+(11.85,0.4)$) {$d$};
				\node (6) at ($(N12)+(0,-0.8)$) {};
				\node (7) at ($(N62)+(0,-0.8)$) {};
				\node (8) at ($(N12)+(0,-2.3)$) {};
				\node (9) at ($(N62)+(0,-2.3)$) {};
				%				\node (8) at ($(N42)+(0,-.5)$) {};
				%				\coordinate (9) at ($(N42)!.6!(N53)+ (-0.5,0)$);
				\coordinate (A) at ($(0)!.25!(1)$);
				\coordinate (B) at ($(0)!.8!(1)$);
				\coordinate (C) at ($(1)!.35!(2)$);
				\coordinate (D) at ($(4)!.75!(5)$);
				%				\coordinate (7) at ($(T22)!.6!(T23)$);
				\draw[->] (0)--(1);
				\draw[->] (1)--(2);
				\draw[->] (2)--(3);
				\draw[->] (1)--(2);
				\draw[->] (3)--(4);
				\draw[->] (4)--(5);
				%				\draw[->,red] (5)--(8);
				%				\draw[->,red] (8)--(9);
				\draw[blue,dashed](6)--(7)node[below]{$y=2$} (a)--(A) (b)--(B) (c)--(C) (d)--(D);
				\draw[blue,dashed](8)--(9)node[below left]{$y=0$};
			\end{tikzpicture}		
		\end{center}
		Dựa vào bảng biến thiên, ta thấy $f'(x)=0\Leftrightarrow x\in \{ 1\,;2\,;3\,;4 \}$;\\
		$f(x)=0\Leftrightarrow x=a<1$ hoặc $x=4$;\\
		$f(x)=2\Leftrightarrow \hoac{
			& x=b\,\,\left(a<b<1\right)\\ 
			& x=c\in\left(1\,;2\right)\\ 
			& x=3\\ 
			& x=d>4.
		}$ \\
		Ta lập được bảng xét dấu của $y'$ 
		\begin{center}
			\begin{tikzpicture}
				\tkzTabInit[lgt=1.2,espcl=1.5,nocadre]
				{$x$/1, $f(x)$ /.8} % first column
				{$-\infty$,$a$, $b$, $1$,$c$, $2$,$3$, $4$, $d$, $+\infty$} % first row
				\tkzTabLine { ,+,z,-,z,+,z,-,z,+,z,-,z,+,z,-,z,+, } % second row
				%				\tkzTabLine {,-,z,+,t,+,} % third row
				%				\tkzTabLine {,+,d,-,z,+,} % last row
			\end{tikzpicture}
		\end{center}
		Từ bảng xét dấu, ta thấy hàm số đồng biến trên các khoảng \\
		$\left(-\infty;a\right)$, $\left(b;1\right)$, $\left(c;2\right)$, $\left(3;4\right)$ và $(d;+\infty)$.
	}
\end{ex}

\begin{ex}%[2D1G1-2]%Câu 29 
	[THPT Bùi Thị Xuân – Huế-2022] 
	\immini{
		Cho hàm số $y=f(x)$ là hàm đa thức bậc bốn. Đồ thị hàm số $f'(x+2)$ được cho trong hình vẽ bên. Hàm số 
		$$g(x)=4 f\left(x^2\right)-x^6+5 x^4-4 x^2+1$$
		đồng biến trên khoảng nào dưới đây?
		\choice
		{$(-4 ;-3)$}
		{\True $(2 ;+\infty)$}
		{$(-\sqrt{2};\sqrt{2})$}
		{$(-2 ;-1)$}}{
		\begin{tikzpicture}[scale=0.6,font=\footnotesize, line join=round, line cap=round, >=stealth] %Đường cong bậc 3
			\draw[thick, ->] (-5.3,0)--(5,0);
			\draw[thick, ->] (0,-3.5)--(0,7);
			\draw (5.2,0) node[below] {$x$};
			\draw (0,7.1) node[left]{$y$};
			\draw (0,0) node[below left]{$0$};
			\draw[fill] (-2,0) circle (0.5pt)node[below left]{$ -2 $};
			\draw[fill] (2,0) circle (0.5pt)node[below]{$ 2$};
			\draw[fill] (0,3) circle (0.5pt)node[left]{$ 3 $};
			\draw[fill] (0,1) circle (0.5pt)node[right]{$ 1 $};
			\draw[fill] (0,-1) circle (0.5pt)node[right]{$ -1 $};
			\draw[dashed] (-2,0)--(-2,1) --(0,1); 
			\draw[dashed](2,0)--(2,3)--(0,3);
			\draw[line width=1.2pt,smooth,samples=100,domain=-2.8:4.5] plot(\x,{-0.271*(\x)^3+0.75*(\x)^2+1.583*\x-1});
		\end{tikzpicture}		
	}
	\loigiai{
		$\begin{aligned}
			& g(x)=4f\left(x^2\right)-x^6+5x^4-4x^2+1\Rightarrow g' (x)=8xf'\left(x^2\right)-6x^5+20x^3-8x.\\ 
			& g' (x)=0\Leftrightarrow 8xf'\left(x^2\right)-6x^5+20x^3-8x=0 \\
			& \Leftrightarrow 2x\left[4f'\left(x^2\right)-3x^4+10x^2-4\right]=0\\ 
			&\Leftrightarrow 		\hoac{ 			& 2x=0\\ 
				& 4f'(x^2)-3x^4+10x^2-4=0
			}
			\Leftrightarrow \hoac{	& x=0\\ 
				& f'\left(x^2\right)=\dfrac{3}{4}{x^4}-\dfrac{5}{2}{x^2}+1.}
		\end{aligned}$\\ 
		Xét
		$f'\left(x^2\right)=\dfrac{3}{4}x^4-\dfrac{5}{2}x^2+1$. Đặt $x^2=t+2$, ta có\\
		$ f' (t+2)=\dfrac{3}{4}{(t+2)^2}-\dfrac{5}{2}(t+2)+1=\dfrac{3}{4}\left(t^2+4t+4\right)-\dfrac{5}{2}(t+2)-1=\dfrac{3}{4}{t^2}+\dfrac{1}{2}t-1$\\
		Khi đó số nghiệm của phương trình chính là số giao điểm của đồ thị hàm số $y=f' (t+2)$ và\\
		$ y=\dfrac{3}{4}{t^2}+\dfrac{1}{2}t-1$\\
		Ta có đồ thị 
		\begin{center}
			\begin{tikzpicture}[scale=0.6,font=\footnotesize, line join=round, line cap=round, >=stealth] %Đường cong bậc 3
				\draw[thick, ->] (-5.3,0)--(5,0);
				\draw[thick, ->] (0,-3.5)--(0,7);
				\draw (5.2,0) node[below] {$x$};
				\draw (0,7.1) node[left]{$y$};
				\draw (0,0) node[below left]{$0$};
				\draw[fill] (-2,0) circle (0.5pt)node[below left]{$ -2 $};
				\draw[fill] (2,0) circle (0.5pt)node[below]{$ 2$};
				\draw[fill] (0,3) circle (0.5pt)node[left]{$ 3 $};
				\draw[fill] (0,1) circle (0.5pt)node[right]{$ 1 $};
				\draw[fill] (0,-1) circle (0.5pt)node[right]{$ -1 $};
				\draw[dashed] (-2,0)--(-2,1) --(0,1); 
				\draw[dashed](2,0)--(2,3)--(0,3);
				\draw[line width=1.2pt,smooth,samples=100,domain=-2.8:4.5] plot(\x,{-0.271*(\x)^3+0.75*(\x)^2+1.583*\x-1});		
				\draw[line width=1.2pt,smooth,samples=100,domain=-3.3:2.8] plot(\x,{0.75*(\x)^2+0.5*\x-1});
			\end{tikzpicture}
		\end{center}
		Dựa vào đồ thị ta có $f' (t+2)=\dfrac{3}{4}t^2+\dfrac{1}{2}t-1\Leftrightarrow \hoac{& t=-2\\ & t=0\\ & t=2} \Leftrightarrow\hoac{& x+2=-2\\ & x+2=0\\ & x+2=2} \Leftrightarrow \hoac{& x=-4\\ & x=-2\\ & x=0.}$\\
		Ta có bảng xét dấu $g' (x)$ như sau
		\begin{center}
			\begin{tikzpicture}
				\tkzTabInit[lgt=1.2,espcl=2,nocadre]
				{$x$/0.7, $f(x)$ /.7}
				{$-\infty$, $-4$,$-2$, $0$, $+\infty$} % first row
				\tkzTabLine { ,-,z,+,z,-,z,+, }
			\end{tikzpicture}
		\end{center}
		Vậy hàm số $g(x)=4 f\left(x^2\right)-x^6+5 x^4-4 x^2+1$ đồng biến trên khoảng $(2 ;+\infty)$.}
\end{ex}

\begin{ex}%[2D1G1-2]%Câu 30
	[Chuyên Bắc Ninh 2022] 
	\immini{
		Cho hàm số $ y=f(x)$ liên tục trên $\mathbb{R}$ có đồ thị hàm số $ y=f'(x)$ có đồ thị như hình vẽ bên.
		Hàm số $g(x)=2f\left(\left| x-1\right|\right)-x^2+2x+2020$ đồng biến trên khoảng nào
		\choice
		{$\left(-2;0\right)$}
		{$\left(-3;1\right)$}
		{$\left(1\,;3\right)$}
		{\True $\left(0\,;\,1\right)$}}{
		\begin{tikzpicture}[scale=0.6,font=\footnotesize, line join=round, line cap=round, >=stealth] %Đường cong bậc 3
			\draw[thick, ->] (-3.3,0)--(5,0);
			\draw[thick, ->] (0,-3.0)--(0,5.5);
			\draw (5.2,0) node[below] {$x$};
			\draw (0,5.8) node[left]{$y$};
			\draw (0,0) node[below left]{$0$};
			\draw[fill] (-1,0) circle (0.5pt)node[above]{$ -1 $};
			\draw[fill] (1,0) circle (0.5pt)node[below]{$ 1$};
			\draw[fill] (0,1) circle (0.5pt)node[left]{$ 1 $};
			\draw[fill] (0,-1) circle (0.5pt)node[right]{$ -1 $};
			\draw[fill] (0,3) circle (0.5pt)node[left]{$ 3 $};
			\draw[fill] (3,0) circle (0.5pt)node[below]{$ 3 $};
			\draw[dashed] (-1,0)--(-1,-1) --(0,-1); 
			\draw[dashed](1,0)--(1,1)--(0,1);
			\draw[dashed](3,0)--(3,3)--(0,3);
			\draw[line width=1.2pt,smooth,samples=100,domain=-2.2:4.3] plot(\x,{-0.333*(\x)^3+1*(\x)^2+1.333*\x-1});		
			%\draw[line width=1.2pt,smooth,samples=100,domain=-3.3:2.8] plot(\x,{0.75*(\x)^2+0.5*\x-1});
		\end{tikzpicture}	
	}
	\loigiai{
		Ta có $g(x)=2f\left(\left| x-1\right|\right)-x^2+2x+2020\Leftrightarrow g(x)=2f\left(\left| x-1\right|\right)-\left(x-1\right)^2+2021$.\\
		Xét hàm số $ k\left(x-1\right)=2f\left(x-1\right)-\left(x-1\right)^2+2021$.\\
		Đặt $ t=x-1$\\
		Xét hàm số $ h(t)=2f(t)-t^2+2021$ $\Rightarrow{h}'(t)=2f'(t)-2t$.\\
		Kẻ đường $ y=x$ như hình vẽ.
		\begin{center}
			\begin{tikzpicture}[scale=0.6,font=\footnotesize, line join=round, line cap=round, >=stealth] %Đường cong bậc 3
				\draw[thick, ->] (-3.3,0)--(5,0);
				\draw[thick, ->] (0,-3.0)--(0,5.5);
				\draw (5.2,0) node[below] {$x$};
				\draw (0,5.8) node[left]{$y$};
				%	\draw (0,0) node[below left]{$0$};
				\draw[fill] (-1,0) circle (0.5pt)node[above]{$ -1 $};
				\draw[fill] (1,0) circle (0.5pt)node[below]{$ 1$};
				\draw[fill] (0,1) circle (0.5pt)node[left]{$ 1 $};
				\draw[fill] (0,-1) circle (0.5pt)node[right]{$ -1 $};
				\draw[fill] (0,3) circle (0.5pt)node[left]{$ 3 $};
				\draw[fill] (3,0) circle (0.5pt)node[below]{$ 3 $};
				\draw[dashed] (-1,0)--(-1,-1) --(0,-1); 
				\draw[dashed](1,0)--(1,1)--(0,1);
				\draw[dashed](3,0)--(3,3)--(0,3);
				\draw[line width=1.2pt,smooth,samples=100,domain=-2.2:4.3] plot(\x,{-0.333*(\x)^3+1*(\x)^2+1.333*\x-1});		
				%\draw[line width=1.2pt,smooth,samples=100,domain=-3.3:2.8] plot(\x,{0.75*(\x)^2+0.5*\x-1});
				\draw[line width=1.2pt,smooth,samples=100](-2,-2)--(4,4);
			\end{tikzpicture}
		\end{center}
		Khi đó $h'(t)>0\Leftrightarrow{f}'(t)-t>0\Leftrightarrow{f}'(t)>t$$\Leftrightarrow \hoac{
			& t<-1\\ 
			& 1<t<3.
		}$\\
		Do đó $k'\left(x-1\right)>0\Leftrightarrow \hoac{
			& x-1<-1\\ 
			& 1<x-1<3} \Leftrightarrow \hoac{
			& x<0\\ 
			& 2<x<4.}$\\
		Ta có bảng biến thiên của hàm số $ k\left(x-1\right)=2f\left(x-1\right)-\left(x-1\right)^2+2021$.
		\begin{center}
			\begin{tikzpicture}
				\tkzTabInit[lgt=1.8,espcl=2.3]
				{$x$ /1.2, $k'(x-1)$ /1.2,$k(x-1)$ /2}
				{$-\infty$ , $0$,$2$,$4$, $+\infty$}
				\tkzTabLine{,+,0,-,0,+,0,-,}
				\tkzTabVar{-/$ $ ,+/$ $, -/$ $,+/$ $,-/$ $}
			\end{tikzpicture}
		\end{center}
		Khi đó, ta có bảng biến thiên của $g(x)=2f\left(\left| x-1\right|\right)-\left(x-1\right)^2+2021$ bằng cách lấy đối xứng qua đường thẳng $ x=1$ như sau\\
		\begin{center}
			\begin{tikzpicture}
				\tkzTabInit[lgt=1.2,espcl=2.5,nocadre]
				{$x$ /0.7, $g'(x)$ /0.7,$g(x)$ /2.5}
				{$-\infty$ ,$-2$, $0$,$1$,$2$,$4$, $+\infty$}
				\tkzTabLine{,+,0,-,0,+,0,-,0,+,0,-,}
				\tkzTabVar{-/$ $ ,+/$ $, -/$ $,+/$ $,-/$ $,+/ $ $,-/$ $}
			\end{tikzpicture}
		\end{center}
		Vậy hàm số đồng biến trên $\left(0;1\right)$.}
\end{ex}

\begin{ex}%[2D1G1-2]%Câu 31
	[Chuyên Thái Bình 2022] 
	\immini{
		Cho hàm số $f(x)=a{x^4}+b{x^3}+c{x^2}+dx+a$ có đồ thị hàm số $y=f'(x)$ như hình vẽ bên. Hàm số $y=g(x)=f\left(1-2x\right)f\left(2-x\right)$ đồng biến trên khoảng nào dưới đây?
		\choice
		{$\left(\dfrac{1}{2};\dfrac{3}{2}\right)$}
		{$\left(-\infty ;0\right)$}
		{$\left(0;2\right)$}
		{\True $\left(3;+\infty\right)$}}{
		\begin{tikzpicture}[scale=0.9,font=\footnotesize, line join=round, line cap=round, >=stealth] %Đường cong bậc 3
			\draw[thick, ->] (-2.5,0)--(2.5,0);
			\draw[thick, ->] (0,-2.8)--(0,2.8);
			\draw (2.6,0) node[below] {$x$};
			\draw (0,2.9) node[left]{$y$};
			\draw (0,0) node[below left]{$0$};
			\draw[fill] (-1,0) circle (0.5pt)node[below left]{$ -1 $};
			\draw[fill] (1,0) circle (0.5pt)node[below right]{$ 1$};
			%			\draw[dashed] (-1,0)--(-1,-1) --(0,-1); 
			%			\draw[dashed](1,0)--(1,1)--(0,1);
			%			\draw[dashed](3,0)--(3,3)--(0,3);
			\draw[line width=1.2pt,smooth,samples=100,domain=-1.3:1.3] plot(\x,{3*(\x)^3-3*\x});		
			%\draw[line width=1.2pt,smooth,samples=100,domain=-3.3:2.8] plot(\x,{0.75*(\x)^2+0.5*\x-1});
		\end{tikzpicture}	
	}
	\loigiai{
		Ta có $f'(x)=4a{x^3}+3b{x^2}+2cx+d$, theo đồ thị thì đa thức $f'(x)$ có ba nghiệm phân biệt là $-1,0,1$ nên $f'(x)=4ax\left(x+1\right)\left(x-1\right)=4a{x^3}-4ax\Rightarrow f(x)=a{x^4}-2a{x^2}+a=a{\left(x^2-1\right)^2}$.\\
		Dựa vào đồ thị hàm số $y=f'(x)$ ta có $a>0$ nên $f(x)>0,\forall x\in\mathbb{R}\setminus\left\{\pm 1\right\}$.\\
		$g'(x)=\left[f\left(1-2x\right)\right]'f\left(2-x\right)+f\left(1-2x\right)\left[f\left(2-x\right)\right]'=-2f'\left(1-2x\right)f\left(2-x\right)-f\left(1-2x\right)f'\left(2-x\right)$. Xét $x\in\left(\dfrac{1}{2};\dfrac{3}{2}\right)\Rightarrow
		\heva{		
			& 1-2x\in\left(-2;0\right)\\ 
			& 2-x\in\left(\dfrac{1}{2};\dfrac{3}{2}\right)}$, dấu của $f'(x)$ không cố định trên $\left(\dfrac{1}{2};\dfrac{3}{2}\right)$ nên ta không kết luận được tính đơn điệu của hàm số $g(x)$ trên $\left(\dfrac{1}{2};\dfrac{3}{2}\right)$.\\
		Xét $x\in\left(-\infty ;0\right)\Rightarrow
		\heva{
			& 1-2x\in\left(1;+\infty\right)\\ 
			& 2-x\in\left(2;+\infty\right)} 
		\Rightarrow \heva{
			& f'\left(1-2x\right)>0\\ 
			& f'\left(2-x\right)>0} \Rightarrow g'(x)<0$.\\
		Do đó, hàm số $g(x)$ nghịch biến trên $\left(-\infty ;0\right)$.\\
		$x\in\left(0;2\right)\Rightarrow \heva{
			& 1-2x\in\left(-3;1\right)\\ 
			& 2-x\in\left(0;2\right)}$, dấu của $f'(x)$ không cố định trên $\left(-3;1\right)$ và $\left(0;2\right)$ nên ta không kết luận được tính đơn điệu của hàm số $g(x)$ trên $\left(\dfrac{1}{2};\dfrac{3}{2}\right)$.\\
		Xét $x\in\left(3;+\infty\right)\Rightarrow \heva{
			& 1-2x\in\left(-\infty ;-5\right)\\ 
			& 2-x\in\left(-\infty ;-1\right)} \Rightarrow \heva{
			& f'\left(1-2x\right)<0\\ 
			& f'\left(2-x\right)<0} \Rightarrow g'(x)>0$. \\
		Do đó, hàm số $g(x)$ đồng biến trên $\left(3;+\infty\right)$.}
\end{ex}

\begin{dang}{Bài toán hàm ẩn, hàm hợp liên quan đến tham số và một số bài toán khác}
\end{dang}

\begin{ex}%[2D1G1-3]%Câu 1
	[Chuyên Lê Hồng Phong Nam Định 2019]
	\immini{
		Cho hàm số $ y=f(x)$ có đạo hàm liên tục trên $\mathbb{R}$. Biết hàm số $ y=f'(x)$ có đồ thị như hình vẽ. Gọi $ S$ là tập hợp các giá trị nguyên $ m\in\left[-5\,;\,\text{5}\right]$ để hàm số $ g(x)=f\left(x+m\right)$ nghịch biến trên khoảng $\left(1\,;\,2\right)$. Hỏi $S$ có bao nhiêu phần tử?
		\choice
		{$ 4$}
		{$ 3$}
		{$ 6$}
		{\True $ 5$}}{
		\begin{tikzpicture}[scale=0.9,font=\footnotesize, line join=round, line cap=round, >=stealth] %Đường cong bậc 3
			\draw[thick, ->] (-2.5,0)--(4,0);
			\draw[thick, ->] (0,-2.8)--(0,2.8);
			\draw (4.3,0) node[below] {$x$};
			\draw (0,2.9) node[left]{$y$};
			\draw (0,0) node[below left]{$0$};
			\draw[fill] (-1,0) circle (0.5pt)node[below left]{$ -1 $};
			\draw[fill] (1,0) circle (0.5pt)node[below]{$ 1$};
			\draw[fill] (3,0) circle (0.5pt)node[below right]{$ 3$};
			%			\draw[dashed] (-1,0)--(-1,-1) --(0,-1); 
			%			\draw[dashed](1,0)--(1,1)--(0,1);
			%			\draw[dashed](3,0)--(3,3)--(0,3);
			\draw[line width=1.2pt,smooth,samples=100,domain=-1.65:3.5] plot(\x,{0.33*(\x)^3-(\x)^2-0.333*(\x)+1});		
			%\draw[line width=1.2pt,smooth,samples=100,domain=-3.3:2.8] plot(\x,{0.75*(\x)^2+0.5*\x-1});
		\end{tikzpicture}	
	}
	\loigiai{
		Ta có $g'(x)=f'\left(x+m\right)$. Vì $ y=f'(x)$ liên tục trên $\mathbb{R}$ nên $g'(x)=f'\left(x+m\right)$ cũng liên tục trên $\mathbb{R}$. Căn cứ vào đồ thị hàm số $ y=f'(x)$ ta thấy\\
		$g'(x)<0\Leftrightarrow{f}'\left(x+m\right)<0$ $\Leftrightarrow\hoac{
			& x+m<-1\\ 
			& 1<x+m<3} \Leftrightarrow \hoac{
			& x<-1-m\\ 
			& 1-m<x<3-m.}$\\
		Hàm số $ g(x)=f\left(x+m\right)$ nghịch biến trên khoảng $\left(1\,;\,2\right)$
		$\Leftrightarrow \hoac{
			& 2\le-1-m\\ 
			&\hoac{
				& 3-m\ge 2\\ 
				& 1-m\le 1}} \Leftrightarrow \hoac{
			& m\le-3\\ 
			& 0\le m\le 1.}$\\
		Mà $ m$ là số nguyên thuộc đoạn $\left[-5\,;\,5\right]$ nên ta có $ S=\left\{-5;-4;-3;0;1\right\}$.\\
		Vậy $ S$ có $5$ phần tử.}
\end{ex}

\begin{ex}%[2D1G1-3]%Câu 2
	[Chuyên Nguyễn Bỉnh Khiêm-Quảng Nam-2020] Cho hàm số $ y=f(x)$ có đạo hàm trên $\mathbb{R}$ và bảng xét dấu đạo hàm như hình vẽ sau
	\begin{center}
		\begin{tikzpicture}
			\tkzTabInit[lgt=1.2,espcl=2.5,nocadre]
			{$x$/0.7, $f'(x)$ /2.5} % first column
			{$-\infty$, $-10$,$-2$, $3$,$8$, $+\infty$} % first row
			\tkzTabLine { ,+,z,-,z,+,z,-,z,+, } % second row
			%				\tkzTabLine {,-,z,+,t,+,} % third row
			%				\tkzTabLine {,+,d,-,z,+,} % last row
		\end{tikzpicture}
	\end{center}
	Có bao nhiêu số nguyên $ m$ để hàm số $ y=f\left(x^3+4x+m\right)$ nghịch biến trên khoảng $\left(-1;1\right)$?
	\choice
	{$ 3$}
	{$ 0$}
	{\True $ 1$}
	{$ 2$}
	\loigiai
	{
		Đặt $ t=x^3+4x+m\Rightarrow{t}'=3x^2+4$ nên $ t$ đồng biến trên $\left(-1;1\right)$ và $ t\in\left(m-5;m+5\right)$.\\
		Yêu cầu bài toán trở thành tìm $ m$ để hàm số $ f(t)$ nghịch biến trên khoảng $\left(m-5;m+5\right)$.\\
		Dựa vào bảng biến thiên ta được $\heva{
			& m-5\ge-2\\ 
			& m+5\le 8} \Leftrightarrow \heva{
			& m\ge 3\\ 
			& m\le 3} \Leftrightarrow m=3$.}
\end{ex}

\begin{ex}%[2D1G1-3]%Câu 3
	[Chuyên ĐH Vinh-Nghệ An-2020]
	\immini{
		Cho hàm số $ f(x)$ có đạo hàm trên $\mathbb{R}$và $ f(1)=1$. Đồ thị hàm số $ y=f'(x)$ như hình bên. Có bao nhiêu số nguyên dương $ a$ để hàm số $ y=\left| 4f\left(\sin x\right)+\cos 2x-a\right|$ nghịch biến trên $\left(0;\dfrac{\pi}{2}\right)$?
		\choice
		{$ 2$}
		{\True $ 3$}
		{Vô số}
		{$ 5$}}{
		\begin{tikzpicture}[scale=0.9,font=\footnotesize, line join=round, line cap=round, >=stealth] %Đường cong bậc 3
			\draw[thick, ->] (-2.5,0)--(3,0);
			\draw[thick, ->] (0,-2.8)--(0,2.8);
			\draw (3.1,0) node[below] {$x$};
			\draw (0,2.9) node[left]{$y$};
			\draw (0,0) node[below left]{$0$};
			\draw[fill] (-1,0) circle (0.5pt)node[below]{$ -1 $};
			\draw[fill] (1,0) circle (0.5pt)node[above]{$ 1$};
			%	\draw[fill] (3,0) circle (0.5pt)node[below right]{$ 3$};
			\draw[dashed] (-1,0)--(-1,1); 
			\draw[dashed](1,0)--(1,-1);
			%			\draw[dashed](3,0)--(3,3)--(0,3);
			\draw[line width=1.2pt,smooth,samples=100,domain=-2:2] plot(\x,{.8*(\x)^3+0*(\x)^2-1.8*(\x)});		
			%\draw[line width=1.2pt,smooth,samples=100,domain=-3.3:2.8] plot(\x,{0.75*(\x)^2+0.5*\x-1});
			\draw (2.0,2.8) node[left]{$y=f'(x)$};
		\end{tikzpicture}	
	}
	\loigiai
	{		Đặt $g(x)=\left| 4f\left(\sin x\right)+\cos 2x-a\right|\Rightarrow g(x)=\sqrt{\left[4f\left(\sin x\right)+\cos 2x-a\right]^2}$ .\\
		$\Rightarrow{g}'(x)=\dfrac{\left[4\cos x\cdot f'\left(\sin x\right)-2\sin 2x\right]\left[4f\left(\sin x\right)+\cos 2x-a\right]}{\sqrt{\left[4f\left(\sin x\right)+\cos 2x-a\right]^2}}$.\\
		Ta có $ 4\cos x\cdot f'\left(\sin x\right)-2\sin 2x=4\cos x\left[f'\left(\sin x\right)-\sin x\right]$.\\
		Với $ x\in\left(0;\dfrac{\pi}{2}\right)$ thì $\cos x>0,\sin x\in\left(0;1\right)\Rightarrow{f}'\left(\sin x\right)-\sin x<0$.\\
		Hàm số $ g(x)$ nghịch biến trên $\left(0;\dfrac{\pi}{2}\right)$ khi $ 4f\left(\sin x\right)+\cos 2x-a\ge 0,\forall x\in\left(0;\dfrac{\pi}{2}\right)$\\
		$\Leftrightarrow 4f\left(\sin x\right)+1-2\sin^2x\ge a,\forall x\in\left(0;\dfrac{\pi}{2}\right)$.\\
		Đặt $ t=\sin x$ được $ 4f(t)+1-2t^2\ge a,\forall t\in\left(0;1\right)$ (*).\\
		Xét $ h(t)=4f(t)+1-2t^2\Rightarrow{h}'(t)=4f'(t)-4t=4\left[f'(t)-1\right]$.\\
		Với $ t\in\left(0;1\right)$ thì $h'(t)<0\Rightarrow h(t)$ nghịch biến trên $\left(0;1\right)$.\\
		Do đó (*) $\Leftrightarrow a\le h(1)=4f(1)+1-2.1^2=3$.\\
		Vậy có $3$ giá trị nguyên dương của a thỏa mãn.}
\end{ex}


\begin{ex}%[2D1G1-3]%Câu 4
	[Chuyên Quang Trung-2020]
	\immini{
		Cho hàm số $ y=f(x)$ có đạo hàm liên tục trên $\mathbb{R}$ và có đồ thị $ y=f'(x)$ như hình vẽ. Đặt $ g(x)=f\left(x-m\right)-\dfrac{1}{2}{\left(x-m-1\right)^2}+2019$, với $ m$ là tham số thực. Gọi $ S$ là tập hợp các giá trị nguyên dương của $ m$ để hàm số $ y=g(x)$ đồng biến trên khoảng $\left(5;6\right)$. Tổng tất cả các phần tử trong $ S$ bằng
		\choice
		{$ 4$}
		{$ 11$}
		{\True $ 14$}
		{$ 20$}}{
		\begin{tikzpicture}[scale=0.9,font=\footnotesize, line join=round, line cap=round, >=stealth] %Đường cong bậc 3
			\draw[style=help lines,step=1] (-2.5,-3) grid (3,3.5);
			\draw[thick, ->] (-2.5,0)--(3.5,0);
			\draw[thick, ->] (0,-2.8)--(0,2.8);
			\draw (3.6,0) node[below] {$x$};
			\draw (0,3) node[above left]{$y$};
			\draw (0,0) node[below left]{$0$};
			%\draw[fill] (-1,0) circle (0.5pt)node[below]{$ -1 $};
			\draw[fill] (1,0) circle (0.5pt)node[below left]{$ 1$};
			%	\draw[fill] (3,0) circle (0.5pt)node[below right]{$ 3$};
			\draw[dashed] (-1,0)--(-1,-2) --(2,-2)--(2,0); 
			\draw[dashed](3,0)--(3,2) --(0,2);
			\draw (-1,-2) circle (2pt);
			\draw (3,2) circle (2pt);
			%			\draw[dashed](3,0)--(3,3)--(0,3);
			\draw[line width=1.2pt,smooth,samples=100,domain=-1.1:3.1] plot(\x,{1*(\x)^3-3*(\x)^2-0*(\x)+2});		
			%\draw[line width=1.2pt,smooth,samples=100,domain=-3.3:2.8] plot(\x,{0.75*(\x)^2+0.5*\x-1});
			%\draw (2.0,2.8) node[left]{$y=f'(x)$};
		\end{tikzpicture}	
	}
	\loigiai
	{
		Xét hàm số $ g(x)=f\left(x-m\right)-\dfrac{1}{2}{\left(x-m-1\right)^2}+2019$.\\
		$g'(x)=f'\left(x-m\right)-\left(x-m-1\right)$.\\
		Xét phương trình $g'(x)=0. \quad \quad (1)$\\
		Đặt $ x-m=t$, phương trình $(1)$ trở thành $f'(t)-\left(t-1\right)=0\Leftrightarrow{f}'(t)=t-1. \quad (2)$\\
		Nghiệm của phương trình $(2)$ là hoành độ giao điểm của hai đồ thị hàm số $ y=f'(t)$ và $ y=t-1$.\\
		Ta có đồ thị các hàm số $ y=f'(t)$ và $ y=t-1$ như sau
		\begin{center}
			\begin{tikzpicture}[scale=0.9,font=\footnotesize, line join=round, line cap=round, >=stealth] %Đường cong bậc 3
				\draw[style=help lines,step=1] (-2.5,-3) grid (3,3.5);
				\draw[thick, ->] (-2.5,0)--(3.5,0);
				\draw[thick, ->] (0,-2.8)--(0,2.8);
				\draw (3.6,0) node[below] {$x$};
				\draw (0,3) node[above left]{$y$};
				\draw (0,0) node[below left]{$0$};
				%\draw[fill] (-1,0) circle (0.5pt)node[below]{$ -1 $};
				\draw[fill] (1,0) circle (0.5pt)node[below left]{$ 1$};
				%	\draw[fill] (3,0) circle (0.5pt)node[below right]{$ 3$};
				\draw[dashed] (-1,0)--(-1,-2) --(2,-2)--(2,0); 
				\draw[dashed](3,0)--(3,2) --(0,2);
				\draw (-1,-2) circle (2pt);
				\draw (3,2) circle (2pt);
				%			\draw[dashed](3,0)--(3,3)--(0,3);
				\draw[line width=1.2pt,smooth,samples=100,domain=-1.1:3.1] plot(\x,{1*(\x)^3-3*(\x)^2-0*(\x)+2});		
				%\draw[line width=1.2pt,smooth,samples=100,domain=-3.3:2.8] plot(\x,{0.75*(\x)^2+0.5*\x-1});
				%\draw (2.0,2.8) node[left]{$y=f'(x)$};
				\draw (-2,-3)--(4,3);
			\end{tikzpicture}
		\end{center}
		Căn cứ đồ thị các hàm số ta có phương trình $(2)$ có nghiệm là $\hoac{
			& t=-1\\ 
			& t=1\\ 
			& t=3} \Rightarrow \hoac{
			& x=m-1\\ 
			& x=m+1\\ 
			& x=m+3.}$\\
		Ta có bảng biến thiên của $ y=g(x)$
		\begin{center}
			\begin{tikzpicture}
				\tkzTabInit[lgt=1,espcl=2.5,nocadre]
				{$x$ /0.8, $y'$ /0.8,$y$ /2.5}
				{$-\infty$ , $m-1$,$m+1$,$m+3$, $+\infty$}
				\tkzTabLine{,+,0,-,0,+,0,-,}
				\tkzTabVar{-/$ +\infty$ ,+/$ $, -/$ $,+/$ $,-/$+\infty $}
			\end{tikzpicture}
		\end{center}
		Để hàm số $ y=g(x)$ đồng biến trên khoảng $\left(5;6\right)$ cần $\hoac{
			&\heva{
				& m-1\le 5\\ 
				& m+1\ge 6}\\ 
			& m+3\le 5}\Leftrightarrow\hoac{
			& 5\le m\le 6\\ 
			& m\le 2.}$\\
		Vì $ m\in\mathbb{N}^*\Rightarrow m$ nhận các giá trị $ 1;\,2;\,5;\,6\Rightarrow S=14$.}
\end{ex}

\begin{ex}%[2D1G1-3]%Câu 5
	[Sở Hà Nội-Lần 2-2020] 
	\immini{
		Cho hàm số $y=a{x^4}+b{x^3}+c{x^2}+dx+e,\,\,a\ne 0$. Hàm số $y=f'(x)$ có đồ thị như hình vẽ bên. 
		Gọi S là tập hợp tất cả các giá trị nguyên thuộc khoảng $\left(-6;6\right)$ của tham số $m$ để hàm số $g(x)=f\left(3-2x+m\right)+x^2-\left(m+3\right)x+2m^2$ nghịch biến trên $\left(0;1\right)$. Khi đó, tổng giá trị các phần tử của S là
		\choice
		{$12$}
		{\True $9$}
		{$6$}
		{$15$}}{
		\begin{tikzpicture}[scale=0.7,font=\footnotesize, line join=round, line cap=round, >=stealth] %Đường cong bậc 3
			%	\draw[style=help lines,step=1] (-2.5,-3) grid (3,3.5);
			\draw[thick, ->] (-4.5,0)--(6.5,0);
			\draw[thick, ->] (0,-2.8)--(0,2.8);
			\draw (6.6,0) node[below] {$x$};
			\draw (0,3) node[above left]{$y$};
			\draw (0,0) node[below left]{$0$};
			\draw[fill] (-2,0) circle (0.5pt)node[below]{$ -2 $};
			\draw[fill] (4,0) circle (0.5pt)node[above]{$ 4$};
			\draw[fill] (0,1) circle (0.5pt)node[right]{$ 1 $};
			\draw[fill] (0,-2) circle (0.5pt)node[left]{$ -2$};
			%	\draw[fill] (3,0) circle (0.5pt)node[below right]{$ 3$};
			\draw[dashed] (-2,0)--(-2,1) --(0,1); 
			\draw[dashed](4,0)--(4,-2) --(0,-2);
			%			\draw[dashed](3,0)--(3,3)--(0,3);
			\draw[line width=1.2pt,smooth,samples=100,domain=-3.8:5.5] plot(\x,{0.0714*(\x)^3-0.1423*(\x)^2-1.0714*(\x)});		
			%\draw[line width=1.2pt,smooth,samples=100,domain=-3.3:2.8] plot(\x,{0.75*(\x)^2+0.5*\x-1});
			%\draw (2.0,2.8) node[left]{$y=f'(x)$};
		\end{tikzpicture}	
	}
	\loigiai
	{
		Xét $g'(x)=-2f'\left(3-2x+m\right)+2x-\left(m+3\right)$.\\
		Xét phương trình $g'(x)=0$, đặt $t=3-2x+m$ thì phương trình trở thành\\ $-2\cdot \left[f'(t)-\dfrac{-t}{2}\right]=0\Leftrightarrow\hoac{
			& t=-2\\ 
			& t=4\\ 
			& t=0.}$ \\
		Từ đó, $g'(x)=0\Leftrightarrow{x_1}=\dfrac{5+m}{2},\,x_2=\dfrac{m+3}{2},x_3=\dfrac{-1+m}{2}$.\\
		Lập bảng xét dấu, đồng thời lưu ý nếu $x>x_1$ thì $t<t_1$ nên $f(x)>0$. Và các dấu đan xen nhau do các nghiệm đều làm đổi dấu đạo hàm nên suy ra $g'(x)\le 0\Leftrightarrow x\in\left[x_2;{x_1}\right]\cup\left(-\infty ;{x_3}\right]$.\\
		Vì hàm số nghịch biến trên $\left(0;1\right)$ nên \\
		$g'(x)\le 0,\,\forall x\in\left(0;1\right)$ từ đó suy ra $\hoac{
			&\dfrac{3+m}{2}\le 0<1\le\dfrac{5+m}{2}\\ 
			& 1\le\dfrac{-1+m}{2}.}$ \\
		và giải ra các giá trị nguyên thuộc $\left(-6;6\right)$ của $m$ là $-3$; $3$; $4$; $5$. }
\end{ex}

\begin{ex}%[2D1G1-3]%Câu 6
	[Chuyên Quang Trung-Bình Phước-Lần 2-2020]
	\immini{
		Cho hàm số $ y=f(x)$ có đạo hàm liên tục trên $\mathbb{R}$ và có đồ thị $ y=f'(x)$ như hình vẽ bên. Đặt $ g(x)=f\left(x-m\right)-\dfrac{1}{2}{\left(x-m-1\right)^2}+2019$, với $ m$ là tham số thực. Gọi $ S$ là tập hợp các giá trị nguyên dương của $ m$ để hàm số $ y=g(x)$ đồng biến trên khoảng $\left(5;6\right)$. Tổng tất cả các phần tử trong $ S$ bằng
		\choice
		{$ 4$}
		{$ 11$}
		{\True $ 14$}
		{$ 20$}}{
		\begin{tikzpicture}[scale=0.9,font=\footnotesize, line join=round, line cap=round, >=stealth] %Đường cong bậc 3
			\draw[thick, ->] (-2.5,0)--(3.7,0);
			\draw[thick, ->] (0,-2.8)--(0,2.8);
			\draw (3.9,0) node[below] {$x$};
			\draw (0,2.9) node[left]{$y$};
			\draw (0,0) node[below left]{$0$};
			\draw[fill] (-1,0) circle (0.5pt)node[above]{$ -1 $};
			\draw[fill] (1,0) circle (0.5pt)node[below]{$ 1$};
			\draw[fill] (3,0) circle (0.5pt)node[below]{$ 3$};
			\draw[fill] (2,0) circle (0.5pt)node[above]{$ 2$};
			\draw[fill] (0,2) circle (0.5pt)node[above left]{$ 2$};
			\draw[fill] (0,-2) circle (0.5pt)node[below left]{$ -2$};
			\draw[dashed] (-1,0)--(-1,-2)--(2,-2)--(2,0); 
			\draw[dashed](3,0)--(3,2)--(0,2);
			%			\draw[dashed](3,0)--(3,3)--(0,3);
			\draw[line width=1.2pt,smooth,samples=100,domain=-1.1:3.1] plot(\x,{1*(\x)^3-3*(\x)^2-0*(\x)+2});		
			%\draw[line width=1.2pt,smooth,samples=100,domain=-3.3:2.8] plot(\x,{0.75*(\x)^2+0.5*\x-1});
			%	\draw (2.0,2.8) node[left]{$y=f'(x)$};
	\end{tikzpicture}	}
	\loigiai
	{
		Ta có $g'(x)=f'\left(x-m\right)-\left(x-m-1\right)$.\\
		Cho $g'(x)=0\Leftrightarrow{f}'\left(x-m\right)=x-m-1$.\\
		Đặt $ x-m=t\Rightarrow f'(t)=t-1$\\
		Khi đó nghiệm của phương trình là hoành độ giao điểm của đồ thị hàm số $ y=f'(t)$ và và đường thẳng $ y=t-1$.
		\begin{center}
			\begin{tikzpicture}[scale=0.9,font=\footnotesize, line join=round, line cap=round, >=stealth] %Đường cong bậc 3
				\draw[thick, ->] (-2.5,0)--(3.7,0);
				\draw[thick, ->] (0,-2.8)--(0,2.8);
				\draw (3.9,0) node[below] {$x$};
				\draw (0,2.9) node[left]{$y$};
				\draw (0,0) node[below left]{$0$};
				\draw[fill] (-1,0) circle (0.5pt)node[above]{$ -1 $};
				\draw[fill] (1,0) circle (0.5pt)node[below]{$ 1$};
				\draw[fill] (3,0) circle (0.5pt)node[below]{$ 3$};
				\draw[fill] (2,0) circle (0.5pt)node[above]{$ 2$};
				\draw[fill] (0,2) circle (0.5pt)node[above left]{$ 2$};
				\draw[fill] (0,-2) circle (0.5pt)node[below left]{$ -2$};
				\draw[dashed] (-1,0)--(-1,-2)--(2,-2)--(2,0); 
				\draw[dashed](3,0)--(3,2)--(0,2);
				%			\draw[dashed](3,0)--(3,3)--(0,3);
				\draw[line width=1.2pt,smooth,samples=100,domain=-1.1:3.1] plot(\x,{1*(\x)^3-3*(\x)^2-0*(\x)+2});		
				%\draw[line width=1.2pt,smooth,samples=100,domain=-3.3:2.8] plot(\x,{0.75*(\x)^2+0.5*\x-1});
				%	\draw (2.0,2.8) node[left]{$y=f'(x)$};
				\coordinate (a) at ($(-1,-2)!1.2!(3,2)$);
				\coordinate (b) at ($(-1,-2)!-.2!(3,2)$);
				\draw[line width=1.2pt,smooth] (a)--(b);
			\end{tikzpicture}
		\end{center}
		Dựa vào đồ thị hàm số ta có được $f'(t)=t-1\Leftrightarrow\hoac{
			& t=-1\\ 
			& t=1\\ 
			& t=3.} $ \\
		Bảng xét dấu của $g'(t)$
		\begin{center}
			\begin{tikzpicture}
				\tkzTabInit[lgt=1.2,espcl=2.5,nocadre]
				{$t$/1, $g'(x)$ /.8} % first column
				{$-\infty$, $-1$,$1$, $3$, $+\infty$} % first row
				\tkzTabLine { ,-,0,+,0,-,0,+, } % second row
				%				\tkzTabLine {,-,z,+,t,+,} % third row
				%				\tkzTabLine {,+,d,-,z,+,} % last row
			\end{tikzpicture}
		\end{center}
		Từ bảng xét dấu ta thấy hàm số $ g(t)$ đồng biến trên khoảng $\left(-1;1\right)$ và $\left(3;+\infty\right)$.\\
		Hay $\hoac{
			&-1<t<1\\ 
			& t>3}\Leftrightarrow\hoac{
			&-1<x-m<1\\ 
			& x-m>3} \Leftrightarrow\hoac{
			& m-1<x<m+1\\ 
			& x>m+3.}$\\
		Để hàm số $ g(x)$ đồng biến trên khoảng $\left(5;6\right)$ thì $\hoac{
			& m-1\le 5<6\le m+1\\ 
			& m+3\le 5<6} \Leftrightarrow\hoac{
			& 5\le m\le 6\\ 
			& m\le 2.}$\\
		Vì $ m$ là các số nguyên dương nên $ S=\left\{ 1;2;5;6\right\}$.\\
		Vậy tổng tất cả các phần tử của $ S$ là $ 1+2+5+6=14$.}
\end{ex}

\begin{ex}%[2D1G1-3]%Câu 7
	\immini{
		Cho hàm số $ y=f(x)$ liên tục có đạo hàm trên $\mathbb{R}$. Biết hàm số $ f'(x)$ có đồ thị cho như hình vẽ bên. Có bao nhiêu giá trị nguyên của $ m$ thuộc $\left[-2019;2019\right]$ để hàm só $ g(x)=f\left(2019^x\right)-mx+2$ đồng biến trên $\left[0;1\right]$.
		\choice
		{$ 2028$}
		{$ 2019$}
		{$ 2011$}
		{\True $ 2020$}}{
		\begin{tikzpicture}[scale=0.9,font=\footnotesize, line join=round, line cap=round, >=stealth] %Đường cong bậc 3
			\draw[thick, ->] (-3.5,0)--(2.5,0);
			\draw[thick, ->] (0,-2.8)--(0,2.8);
			\draw (2.7,0) node[below] {$x$};
			\draw (0,2.9) node[left]{$y$};
			\draw (0,0) node[below left]{$0$};
			%	\draw[fill] (-1,0) circle (0.5pt)node[above]{$ -1 $};
			\draw[fill] (1,0) circle (0.5pt)node[below right]{$ 1$};
			%		\draw[fill] (3,0) circle (0.5pt)node[below]{$ 3$};
			%		\draw[fill] (2,0) circle (0.5pt)node[above]{$ 2$};
			%		\draw[fill] (0,2) circle (0.5pt)node[above left]{$ 2$};
			%		\draw[fill] (0,-2) circle (0.5pt)node[below left]{$ -2$};
			%		\draw[dashed] (-1,0)--(-1,-2)--(2,-2)--(2,0); 
			%		\draw[dashed](3,0)--(3,2)--(0,2);
			\draw[line width=1.2pt,smooth,samples=100,domain=-3.28:1.32] plot(\x,{0.667*(\x)^3+2*(\x)^2-0.667*(\x)-2});		
			%\draw[line width=1.2pt,smooth,samples=100,domain=-3.3:2.8] plot(\x,{0.75*(\x)^2+0.5*\x-1});
			%	\draw (2.0,2.8) node[left]{$y=f'(x)$};
	\end{tikzpicture}	}
	\loigiai{
		Ta có $ g'(x)=2019^x\ln 2019\cdot f'\left(2019^x\right)-m$.\\
		Ta lại có hàm số $ y=2019^x$ đồng biến trên $\left[0;1\right]$.\\
		Với $ x\in\left[0;1\right]$ thì $2019^x\in\left[1;2019\right]$ mà hàm $ y=f'(x)$ đồng biến trên $\left(1;+\infty\right)$ nên hàm $ y=f'\left(2019^x\right)$ đồng biến trên $\left[0;1\right]$.\\
		Mà $2019^x\ge 1;f'\left(2019^x\right)>0\,\forall\,x\in\left[0;1\right]$ nên hàm $ h(x)=2019^x\ln 2019\cdot f'\left(2019^x\right)$ đồng biến trên $\left[0;1\right]$.\\
		Hay $ h(x)\ge h(0)=0,\forall\,x\in\left[0;1\right]$.\\
		Do vậy hàm số $ g(x)$ đồng biến trên đoạn $\left[0;1\right]$$\Leftrightarrow g'(x)\ge 0,\forall\,x\in\left[0;1\right]$\\
		$\Leftrightarrow m\le{2019^x}\ln 2019.f'\left(2019^x\right),\forall\,x\in\left[0;1\right]$ $\Leftrightarrow m\le\underset{x\in\left[0;1\right]}{\min}\,h(x)=h(0)=0$\\
		Vì $ m$ nguyên và $ m\in\left[-2019;2019\right]\Rightarrow $có $ 2020$ giá trị $ m$ thỏa mãn yêu cầu bài toán.}
\end{ex}

\begin{ex}%[2D1G1-3]%Câu 8
	\immini{
		Cho hàm số $y=f(x)$ có đồ thị $f'(x)\,$ như hình vẽ. Có bao nhiêu giá trị nguyên $m\in\left(-2020\,;\,2020\right)$ để hàm số $g(x)=f\left(2x-3\right)\,-\ln \left(1+x^2\right)-2mx$ đồng biến trên $\left(\dfrac{1}{2};2\right)$?
		\choice
		{$ 2020$}
		{\True $ 2019$}
		{$ 2021$}
		{$ 2018$}}{
		\begin{tikzpicture}[scale=0.9,font=\footnotesize, line join=round, line cap=round, >=stealth] %Đường cong bậc 3
			\draw[thick, ->] (-2.5,0)--(2.5,0);
			\draw[thick, ->] (0,-1.8)--(0,5.8);
			\draw (2.7,0) node[below] {$x$};
			\draw (0,5.9) node[left]{$y$};
			\draw (0,0) node[below left]{$0$};
			\draw[fill] (-2,0) circle (0.5pt)node[below]{$ -2 $};
			\draw[fill] (1,0) circle (0.5pt)node[below]{$ 1$};
			\draw[fill] (-1,0) circle (0.5pt)node[below]{$-1$};
			\draw[fill] (0,4) circle (0.5pt)node[above left]{$ 2$};
			%		\draw[fill] (0,2) circle (0.5pt)node[above left]{$ 2$};
			%		\draw[fill] (0,-2) circle (0.5pt)node[below left]{$ -2$};
			\draw[dashed] (-2,0)--(-2,4)--(1,4)--(1,0); 
			%		\draw[dashed](3,0)--(3,2)--(0,2);
			\draw[line width=1.2pt,smooth,samples=100,domain=-2.1:2.1] plot(\x,{-1*(\x)^3+0*(\x)^2+3*(\x)+2});		
			%\draw[line width=1.2pt,smooth,samples=100,domain=-3.3:2.8] plot(\x,{0.75*(\x)^2+0.5*\x-1});
			%	\draw (2.0,2.8) node[left]{$y=f'(x)$};
	\end{tikzpicture}	}
	\loigiai{
		Ta có $g'(x)=2f'\left(2x-3\right)-\dfrac{2x}{1+x^2}-2m$.\\
		Hàm số $ g(x)$ đồng biến trên $\left(\dfrac{1}{2};2\right)$ khi và chỉ khi \\
		$g'(x)\ge 0,\,\,\forall x\in\left(-1;\,2\right)$\\
		$\Leftrightarrow m\le{f}'\left(2x-3\right)-\dfrac{x}{1+x^2},\,\,\forall x\in\left(\dfrac{1}{2};2\right)$\\
		$\Leftrightarrow m\le\underset{x\in\left[\dfrac{1}{2};2\right]}{\min}\,\left[f'\left(2x-3\right)-\dfrac{x}{1+x^2}\right]$. \, \,  $(1)$\\
		Đặt $ t=2x-3$, khi đó $ x\in\left(\dfrac{1}{2};2\right)\Leftrightarrow t\in\left(-2;\,1\right)$.\\
		Từ đồ thị hàm $f'(x)$ suy ra $f'(t)\ge 0,\,\,\forall t\in\left(-2;1\right)$ và $f'(t)=0$ khi $ t=-1$.\\
		Tức là $f'\left(2x-3\right)\ge 0,\,\,\forall x\in\left(\dfrac{1}{2};\,2\right)$$\Rightarrow\underset{x\in\left[\dfrac{1}{2};2\right]}{\min}\,f'\left(2x-3\right)=0$ khi $ x=1$. $(2)$\\
		Xét hàm số $ h(x)=-\dfrac{x}{1+x^2}$ trên khoảng $\left(\dfrac{1}{2};\,2\right)$.\\
		Ta có $h'(x)=\dfrac{x^2-1}{\left(1+x^2\right)^2}$ và\\
		$h'(x)=0\Leftrightarrow{x^2}-1=0\Leftrightarrow x=\pm 1$.\\
		Bảng biến thiên của hàm số $ h(x)$ trên $\left(\dfrac{1}{2};\,2\right)$ như sau
		\begin{center}
			\begin{tikzpicture}
				\tkzTabInit[lgt=1.2,espcl=2.5,nocadre]
				{$x$ /0.7, $h'(x)$ /0.7,$h(x)$ /2.5}
				{$\dfrac{1}{2}$ , $1$,$2$}
				\tkzTabLine{,-,0,+,}
				\tkzTabVar{+/$  $ ,-/$ \-\dfrac{1}{2} $, +/$ $}
			\end{tikzpicture}
		\end{center}
		Từ bảng biến thiên suy ra $ h(x)\ge-\dfrac{1}{2}$$\Rightarrow\underset{x\in\left[\dfrac{1}{2};2\right]}{\min}\,h(x)=-\dfrac{1}{2}$ khi $ x=1$. \, \,  $(3)$\\
		Từ $(1)$, $(2)$ và $(3)$ suy ra $ m\le-\dfrac{1}{2}$.\\
		Kết hợp với $ m\in\mathbb{Z}$, $ m\in\left(-2020;\,2020\right)$ thì $ m\in\left\{-2019;\,-201;\ldots ;-2;-1\right\}$.\\
		Vậy có tất cả $ 2019$ giá trị $ m$ cần tìm.}
\end{ex}

\begin{ex}%[2D1G1-3]%Câu 9
	Cho hàm số $ f(x)$ liên tục trên $\mathbb{R}$ và có đạo hàm $f'(x)=x^2\left(x-2\right)\left(x^2-6x+m\right)$ với mọi $ x\in\mathbb{R}$. Có bao nhiêu số nguyên $ m$ thuộc đoạn $\left[-2020;2020\right]$ để hàm số $ g(x)=f\left(1-x\right)$ nghịch biến trên khoảng $\left(-\infty ;-1\right)$?
	\choice
	{$ 2016$}
	{$ 2014$}
	{\True $ 2012$}
	{$ 2010$}
	\loigiai{
		Ta có \\
		$g'(x)=f'\left(1-x\right)=-\left(1-x\right)^2\left(-x-1\right)\left[\left(1-x\right)^2-6\left(1-x\right)+m\right]$
		$=\left(x-1\right)^2\left(x+1\right)\left(x^2+4x+m-5\right)$.\\
		Hàm số $ g(x)$ nghịch biến trên khoảng $\left(-\infty ;-1\right)$\\
		$\Leftrightarrow{g}'(x)\le 0,\forall x<-1$ $(*)$, (dấu \lq\lq $=$\rq\rq \, xảy ra tại hữu hạn điểm).\\
		Với $ x<-1$ thì $\left(x-1\right)^2>0$ và $ x+1<0$ nên\\
		$(*)$ $\Leftrightarrow{x^2}+4x+m-5\ge 0,\forall x<-1 \Leftrightarrow m\ge-x^2-4x+5,\forall x<-1$.\\
		Xét hàm số $ y=-x^2-4x+5$ trên khoảng $\left(-\infty ;-1\right)$, ta có bảng biến thiên
		\begin{center}
			\begin{tikzpicture}
				\tkzTabInit[lgt=1.8,espcl=2.3]
				{$x$ /1.2, $y'$ /1.2,$y$ /2}
				{$-\infty$ , $-2$,$-1$}
				\tkzTabLine{,+,0,-,}
				\tkzTabVar{-/$ -\infty $ ,+/$9 $, -/$ 8$}
			\end{tikzpicture}
		\end{center}
		Từ bảng biến thiên suy ra $ m\ge 9$.\\
		Kết hợp với $ m$ thuộc đoạn $\left[-2020;2020\right]$ và $ m$ nguyên nên $ m\in\left\{ 9;10;11;\ldots ;2020\right\}$.\\
		Vậy có $ 2012$ số nguyên $ m$ thỏa mãn đề bài.}
\end{ex}

\begin{ex}%[2D1G1-3]%Câu 10
	\immini{
		Cho hàm số $f(x)$ xác định và liên tục trên $ R$. Hàm số $y=f'(x)$ liên tục trên $\mathbb{R}$ và có đồ thị như hình vẽ bên.
		Xét hàm số $g(x)=f\left(x-2m\right)+\dfrac{1}{2}{\left(2m-x\right)^2}+2020$, với $ m$ là tham số thực. Gọi $ S$ là tập hợp các giá trị nguyên dương của $ m$ để hàm số $ y=g(x)$ nghịch biến trên khoảng $\left(3;4\right)$. Hỏi số phần tử của $ S$ bằng bao nhiêu?
		\choice
		{$4$}
		{\True $2$}
		{$3$}
		{Vô số}}
	{
		\begin{tikzpicture}[scale=0.7,>=stealth, font=\footnotesize, line join=round, line cap=round]
			\def\xmin{-3.5} \def\xmax{4.5}
			\def\ymin{-5.2} \def\ymax{4}
			\clip(\xmin,\ymin) rectangle (\xmax,\ymax);
			\draw[->] (\xmin,0)--(\xmax,0) node [below]{$x$};
			\draw[->] (0,\ymin)--(0,\ymax) node [left]{$y$};
			\node at (0,0) [below left]{$O$};
			\path
			(-3.1,3.7) coordinate (A)
			(-3,3) coordinate (B)
			(0,-2) coordinate (C)
			(0.65,-2) coordinate (D)
			(1,-1) coordinate (E)
			(3,-3) coordinate (F)
			(3.4,-5) coordinate (G);
			\draw[smooth]
			(A)..controls +(-88:0.1) and +(93:.1)..
			(B)..controls +(-87:0.3) and +(-100:8.5)..
			(C)..controls +(75:.8) and +(180:.1)..
			(D)..controls +(0:.1) and +(-105:.3)..
			(E)..controls +(70:2) and +(97:0.4)..
			(F)..controls +(-80:.1) and +(90:0.3)..
			(G);
			\draw[dashed] 
			(-3,0)node[below]{$-3$}|-(0,3)node[right]{$3$}
			(1,0)node[above]{$1$}|-(0,-1)node[left]{$-1$}
			(3,0)node[above]{$3$}|-(0,-3)node[below right]{$-3$};
			\fill 
			(0,-2) circle(1.5pt)
			(-3,3) circle(1.5pt)
			(3,-3) circle(1.5pt)
			(1,-1) circle(1.5pt);
			\node at (2.1,-4) {$y=f'(x)$};
		\end{tikzpicture}
	}
	\loigiai{
		Ta có $g'(x)=f'\left(x-2m\right)-\left(2m-x\right)$.		Đặt $h(x)=f'(x)-\left(-x\right)$.\\
		Từ đồ thị hàm số $y=f'(x)$ và đồ thị hàm số $y=-x$ trên hình vẽ suy ra \\
		$h(x)\le 0\Leftrightarrow f'(x)\le-x\Leftrightarrow\hoac{
			&-3\le x\le 1\\ 
			& x\ge 3.}$ 
		\begin{center}
			\begin{tikzpicture}[scale=0.7,>=stealth, font=\footnotesize, line join=round, line cap=round]
				\def\xmin{-3.5} \def\xmax{4.5}
				\def\ymin{-5.2} \def\ymax{4}
				\clip(\xmin,\ymin) rectangle (\xmax,\ymax);
				\draw[->] (\xmin,0)--(\xmax,0) node [below]{$x$};
				\draw[->] (0,\ymin)--(0,\ymax) node [left]{$y$};
				\node at (0,0) [below left]{$O$};
				\path
				(-3.1,3.7) coordinate (A)
				(-3,3) coordinate (B)
				(0,-2) coordinate (C)
				(0.65,-2) coordinate (D)
				(1,-1) coordinate (E)
				(3,-3) coordinate (F)
				(3.4,-5) coordinate (G);
				\draw[smooth]
				(A)..controls +(-88:0.1) and +(93:.1)..
				(B)..controls +(-87:0.3) and +(-100:8.5)..
				(C)..controls +(75:.8) and +(180:.1)..
				(D)..controls +(0:.1) and +(-105:.3)..
				(E)..controls +(70:2) and +(97:0.4)..
				(F)..controls +(-80:.1) and +(90:0.3)..
				(G);
				\draw[dashed] 
				(-3,0)node[below]{$-3$}|-(0,3)node[right]{$3$}
				(1,0)node[above]{$1$}|-(0,-1)node[left]{$-1$}
				(3,0)node[above]{$3$}|-(0,-3)node[below right]{$-3$};
				\fill 
				(0,-2) circle(1.5pt)
				(-3,3) circle(1.5pt)
				(3,-3) circle(1.5pt)
				(1,-1) circle(1.5pt);
				\draw[smooth,samples=300,domain=-3.2:3.7] plot(\x,{-(\x)});
				\node at (2.1,-4) {$y=f'(x)$};
				\node at (-1,2.1) {$y=h(x)$};
			\end{tikzpicture}
		\end{center}
		Ta có $ g'(x)=h\left(x-2m\right)\le 0\Leftrightarrow\hoac{
			&-3\le x-2m\le 1\\ 
			& x-2m\ge 3}\Leftrightarrow\hoac{
			& 2m-3\le x\le 2m+1\\ 
			& x\ge 2m+3.}$.\\
		Suy ra hàm số $ y=g(x)$ nghịch biến trên các khoảng $\left(2m-3;2m+1\right)$ và $\left(2m+3;+\infty\right)$.\\
		Do đó hàm số $ y=g(x)$ nghịch biến trên khoảng $\left(3;4\right)$ $\Leftrightarrow\hoac{
			&\heva{
				& 2m-3\le 3\\ 
				& 2m+1\ge 4}\\ 
			& 2m+3\le 3}\Leftrightarrow\hoac{
			&\dfrac{3}{2}\le m\le 3\\ 
			& m\le 0.}$ \\
		Mặt khác, do $ m$ nguyên dương nên $ m\in\left\{ 2;3\right\}\Rightarrow S=\left\{ 2;3\right\}$. Vậy số phần tử của $ S$ bằng $2$.\\
	}
	
\end{ex}

\begin{ex}%[2D1G1-3]%Câu 11
	Cho hàm số $f(x)$ có đạo hàm trên $\mathbb{R}$ là $f'(x)=\left(x-1\right)\left(x+3\right)$. Có bao nhiêu giá trị nguyên của tham số $m$ thuộc đoạn $\left[-10;20\right]$ để hàm số $y=f\left(x^2+3x-m\right)$ đồng biến trên khoảng $\left(0;2\right)$?
	\choice
	{\True $ 18$}
	{$ 17$}
	{$ 16$}
	{$ 20$}
	\loigiai{
		Ta có $y'=f'\left(x^2+3x-m\right)=\left(2x+3\right){f}'\left(x^2+3x-m\right)$.\\
		Theo đề bài ta có $f'(x)=\left(x-1\right)\left(x+3\right)$\\
		suy ra $f'(x)>0\Leftrightarrow\hoac{
			& x<-3\\ 
			& x>1}$ và $f'(x)<0\Leftrightarrow-3<x<1$ .\\
		Hàm số đồng biến trên khoảng $\left(0;2\right)$ khi $y'\ge 0,\forall x\in\left(0;2\right)$\\
		$\Leftrightarrow\left(2x+3\right){f}'\left(x^2+3x-m\right)\ge 0,\forall x\in\left(0;2\right)$.\\
		Do $x\in\left(0;2\right)$ nên $2x+3>0,\forall x\in\left(0;2\right)$. Do đó, ta có\\
		$y'\ge 0,\forall x\in\left(0;2\right)\Leftrightarrow f'\left(x^2+3x-m\right)\ge 0$\\
		$\Leftrightarrow\hoac{
			&{x^2}+3x-m\le-3\\ 
			&{x^2}+3x-m\ge 1}\Leftrightarrow\hoac{
			& m\ge{x^2}+3x+3\\ 
			& m\le{x^2}+3x-1}$\\
		$\Leftrightarrow\hoac{
			& m\ge\underset{\left[0;2\right]}{\max}\,\left(x^2+3x+3\right)\\ 
			& m\le\underset{\left[0;2\right]}{\min}\,\left(x^2+3x-1\right)} \Leftrightarrow\hoac{
			& m\ge 13\\ 
			& m\le-1}$.\\
		Do $m\in\left[-10;20\right]$, $ m\in\mathbb{Z}$ nên có $ 18$ giá trị nguyên của $m$ thỏa yêu cầu đề bài.}
\end{ex}

\begin{ex}%[2D1G1-3]%Câu 12
	Cho các hàm số $f(x)=x^3+4x+m$ và $g(x)=\left(x^2+2018\right){\left(x^2+2019\right)^2}{\left(x^2+2020\right)^3}$ . Có bao nhiêu giá trị nguyên của tham số $m\in\left[-2020;2020\right]$ để hàm số $g\left(f(x)\right)$ đồng biến trên $\left(2;+\infty\right)$ ?
	\choice
	{$2005$}
	{\True $2037$}
	{$4016$}
	{$4041$}
	\loigiai{
		Ta có $f(x)=x^3+4x+m$ và \\
		$g(x)=\left(x^2+2018\right){\left(x^2+2019\right)^2}{\left(x^2+2020\right)^3}=a_{12}{x^{12}}+a_{10}{x^{10}}+...+a_2x^2+a_0$.\\
		Suy ra $f'(x)=3x^2+4$ , $g'(x)=12a_{12}{x^{11}}+10a_{10}{x^9}+...+2a_2x$.\\
		Và có 
		\begin{eqnarray*}
			\left[g\left(f(x)\right)\right]' &=& f'(x)\left[12a_{12}{\left(f(x)\right)^{11}}+10a_{10}{\left(f(x)\right)^9}+...+2a_2f(x)\right]\\
			&=& f(x)f'(x)\left(12a_{12}{\left(f(x)\right)^{10}}+10a_{10}{\left(f(x)\right)^8}+...+2a_2\right).
		\end{eqnarray*} 
		Dễ thấy $a_{12};{a_{10}};...;{a_2};{a_0}>0$ và $f'(x)=3x^2+4>0$, $\forall x>2$.\\
		Do đó $f'(x)\left(12a_{12}{\left(f(x)\right)^{10}}+10a_{10}{\left(f(x)\right)^8}+...+2a_2\right)>0$ , $\forall x>2$.\\
		Hàm số $g\left(f(x)\right)$ đồng biến trên $\left(2;+\infty\right)$ khi $\left[g\left(f(x)\right)\right]^{'}\ge 0$, $\forall x>2$\\
		$\Rightarrow  f(x)\ge 0$, $\forall x>2 \Leftrightarrow x^3+4x+m\ge 0$, $\forall x>3 \Leftrightarrow  m\ge-x^3-4x$, $\forall x>2$\\
		$ \Rightarrow  m\ge\underset{\left[2;+\infty\right)}{\max}\,\left(-x^3-4x\right)=-16$.\\
		Vì $m\in\left[-2020;2020\right]$ và $m\in\mathbb{Z}$ nên có $2037$ giá trị thỏa mãn $m$ .}
\end{ex}

\begin{ex}%[2D1G1-3]%Câu 13
	Cho hàm số $y=f(x)$ có đạo hàm $f'(x)=x{\left(x+1\right)^2}\left(x^2+2mx+1\right)$ với mọi $x \in \mathbb{R}$. Có bao nhiêu số nguyên âm $m$ để hàm số $g(x)=f\left(2x+1\right)$ đồng biến trên khoảng $\left(3;5\right)$?
	\choice
	{\True $3$}
	{$2$}
	{$4$}
	{$6$}
	\loigiai{
		Ta có $g'(x)=2f'(2x+1)=2(2x+1)(2x+2)^2[(2x+1)^2+2m(2x+1)+1]$. 	Đặt $t=2x+1$\\
		Để hàm số $g(x)$ đồng biến trên khoảng $\left(3;5\right)$ khi và chỉ khi 
		\begin{eqnarray*}
			& & g'(x)\ge 0,\forall x\in\left(3;5\right) \\
			& \Leftrightarrow & t(t^2+2mt+1)\ge 0,\forall t\in\left(7;11\right)\Leftrightarrow{t^2}+2mt+1\ge 0,\,\,\forall t\in\left(7;11\right) \\
			&\Leftrightarrow & 2m\ge\dfrac{-t^2-1}{t},\,\,\,\forall t\in\left(7;11\right)
		\end{eqnarray*}	
		Xét hàm số $h(t)=\dfrac{-t^2-1}{t}$ trên $\left[7;11\right]$, có $h'(t)=\dfrac{-t^2+1}{t^2}$\\
		Bảng biến thiên
		\begin{center}
			\begin{tikzpicture}
				\tkzTabInit[espcl=3,lgt=1.2,nocadre]
				{$t$/0.7,$h'(t)$/0.7,$h(t)$/2.5}
				{$-\infty$,$1$,$11$,$+\infty$}
				\tkzTabLine{, ,,-,,,}
				%	\node (0) at ($(N12)+(0,-3)$) {$-\infty$};
				\node (1) at ($(N22)+(0,-0.8)$) [right] {$-\dfrac{50}{7}$};
				\node (2) at ($(N32)+(0,-2.5)$) [left] {$-\dfrac{122}{11}$};
				
				
				%				\node (3) at ($(N11+(-0.5,0))$) {};
				%				\node (4) at ($(N23)$) {};
				\fill[pattern=north east lines] (7.0,-0.7) rectangle (10,-4.4);
				\fill[pattern=north east lines] (1.5,-0.7) rectangle (4.5,-4.4);
				\draw[->] (1)--(2);	
				\draw[dashed] (4.5,-0.7)--(4.5,-4.4);
				\draw[dashed] (7.0,-0.7)--(7.0,-4.4);	
			\end{tikzpicture}		
		\end{center}
		Dựa vào BBT ta có $2m\ge\dfrac{-t^2-1}{t},\,\,\,\forall t\in\left(7;11\right)\Leftrightarrow 2m\ge\underset{\left[7;11\right]}{\max}\,h(t)\Leftrightarrow m\ge-\dfrac{50}{14}$\\
		Vì $ m\in{\mathbb{Z}^-}\Rightarrow m \in \{-3;-2;-1\}$ .
	}
\end{ex}

\begin{ex}%[2D1G1-3]%Câu 14
	Cho hàm số $y=f(x)$ có bảng biến thiên như sau\\
	\begin{center}
		\begin{tikzpicture}[>=stealth,scale = 1]
			\tkzTabInit[lgt=1,espcl=2.5,nocadre]
			{$x$ /0.7, $y'$ /0.7,$y$ /2.5}
			{$-\infty$,$0$,$2$,$+\infty$}
			\tkzTabLine{ ,-,0,+,0,-,}
			\tkzTabVar{-/$-\infty$, +/$4$,- /$0$, +/{ $+\infty$}}
		\end{tikzpicture}
	\end{center}
	Có bao nhiêu số nguyên $m<2019$ để hàm số $g(x)=f\left(x^2-2x+m\right)$ đồng biến trên khoảng $\left(1;+\infty\right)$?
	\choice
	{\True $2016$}
	{$2015$}
	{$2017$}
	{$2018$}
	\loigiai{
		Ta có $g'(x)=\left(x^2-2x+m\right)'{f}'\left(x^2-2x+m\right)=2\left(x-1\right){f}'\left(x^2-2x+m\right)$ .\\
		Hàm số $y=g(x)$ đồng biến trên khoảng $\left(1;+\infty\right)$ khi và chỉ khi $g'(x)\ge 0,\forall x\in\left(1;+\infty\right)$ và\\
		$g'(x)=0$ tại hữu hạn điểm \\
		$\Leftrightarrow 2\left(x-1\right){f}'\left(x^2-2x+m\right)\ge 0,\forall x\in\left(1;+\infty\right)$\\
		$\Leftrightarrow{f}'\left(x^2-2x+m\right)\ge 0,\forall x\in\left(1;+\infty\right)$ $\Leftrightarrow\hoac{
			&{x^2}-2x+m\ge 2,\forall x\in\left(1;+\infty\right)\\ 
			&{x^2}-2x+m\le 0,\forall x\in\left(1;+\infty\right).}$\\
		Xét hàm số $y=x^2-2x+m$, ta có bảng biến thiên
		\begin{center}
			\begin{tikzpicture}[>=stealth,scale = 1]
				\tkzTabInit[lgt=1,espcl=2.5,nocadre]
				{$x$ /0.7, $y'$ /0.7,$y$ /2.5}
				{$-\infty$,$2$,$+\infty$}
				\tkzTabLine{ ,-,0,+,}
				\tkzTabVar{+/$+\infty$, -/$m-1$, +/{$+\infty$}}
			\end{tikzpicture}
		\end{center}
		Dựa vào bảng biến thiên ta có\\
		TH1: $x^2-2x+m\ge 2,\forall x\in\left(1;+\infty\right)\Leftrightarrow m-1\ge 2\Leftrightarrow m\ge 3$ .\\
		TH2: $x^2-2x+m\le 0,\forall x\in\left(1;+\infty\right)$. Không có giá trị $m$ thỏa mãn.\\
		Vậy có $2016$ số nguyên $m<2019$ thỏa mãn yêu cầu bài toán.}
\end{ex}

\begin{ex}%[2D1G1-3]%Câu 15
	\immini{
		Cho hàm số $ y=f(x)$ có đạo hàm là hàm số $f'(x)$ trên $\mathbb{R}$. Biết rằng hàm số $ y=f'\left(x-2\right)+2$ có đồ thị như hình vẽ bên dưới. Hàm số $ f(x)$ đồng biến trên khoảng nào?
		\choice
		{$\left(-\infty ;3\right),\,\,\left(5;+\infty\right)$}
		{\True $\left(-\infty ;-1\right),\,\,\left(1;+\infty\right)$}
		{$\left(-1;1\right)$}
		{$\left(3;5\right)$}}{
		\begin{tikzpicture}[scale=0.7,font=\footnotesize, line join=round, line cap=round, >=stealth] %Đường cong bậc 3
			\draw[thick, ->] (-0.5,0)--(3.5,0);
			\draw[thick, ->] (0,-1.8)--(0,5.3);
			\draw (3.7,0) node[below] {$x$};
			\draw (0,5.4) node[left]{$y$};
			\draw (0,0) node[below left]{$0$};
			\draw[fill] (3,0) circle (0.5pt)node[below]{$ 3$};
			\draw[fill] (1,0) circle (0.5pt)node[below]{$ 1$};
			\draw[fill] (2,0) circle (0.5pt)node[above]{$2$};
			\draw[fill] (0,2) circle (0.5pt)node[left]{$ 2$};
			\draw[fill] (0,-1) circle (0.5pt)node[left]{$ -1$};
			%		\draw[fill] (0,2) circle (0.5pt)node[above left]{$ 2$};
			%		\draw[fill] (0,-2) circle (0.5pt)node[below left]{$ -2$};
			\draw[dashed] (3,0)--(3,2)--(0,2)--(1,2)--(1,0); 
			\draw[dashed](0,-1)--(2,-1)--(2,0);
			\draw[line width=1.2pt,smooth,samples=100,domain=0.6:3.4] plot(\x,{3*(\x)^2-12*(\x)+11});		
			%\draw[line width=1.2pt,smooth,samples=100,domain=-3.3:2.8] plot(\x,{0.75*(\x)^2+0.5*\x-1});
			%	\draw (2.0,2.8) node[left]{$y=f'(x)$};
	\end{tikzpicture}	}
	\loigiai{	
		Hàm số $ y=f'\left(x-2\right)+2$ có đồ thị $(C)$ như sau:\\
		\begin{center}
			\begin{tikzpicture}[scale=0.7,font=\footnotesize, line join=round, line cap=round, >=stealth] %Đường cong bậc 3
				\draw[thick, ->] (-0.5,0)--(3.5,0);
				\draw[thick, ->] (0,-1.8)--(0,5.3);
				\draw (3.7,0) node[below] {$x$};
				\draw (0,5.4) node[left]{$y$};
				\draw (0,0) node[below left]{$0$};
				\draw[fill] (3,0) circle (0.5pt)node[below]{$ 3$};
				\draw[fill] (1,0) circle (0.5pt)node[below]{$ 1$};
				\draw[fill] (2,0) circle (0.5pt)node[above]{$2$};
				\draw[fill] (0,2) circle (0.5pt)node[left]{$ 2$};
				\draw[fill] (0,-1) circle (0.5pt)node[left]{$ -1$};
				%		\draw[fill] (0,2) circle (0.5pt)node[above left]{$ 2$};
				%		\draw[fill] (0,-2) circle (0.5pt)node[below left]{$ -2$};
				\draw[dashed] (3,0)--(3,2)--(0,2)--(1,2)--(1,0); 
				\draw[dashed](0,-1)--(2,-1)--(2,0);
				\draw[line width=1.2pt,smooth,samples=100,domain=0.6:3.4] plot(\x,{3*(\x)^2-12*(\x)+11});		
				%\draw[line width=1.2pt,smooth,samples=100,domain=-3.3:2.8] plot(\x,{0.75*(\x)^2+0.5*\x-1});
				%	\draw (2.0,2.8) node[left]{$y=f'(x)$};
			\end{tikzpicture}
		\end{center}
		Dựa vào đồ thị $(C)$ ta có\\ $f'\left(x-2\right)+2>2,\forall x\in\left(-\infty ;1\right)\cup\left(3;+\infty\right)\Leftrightarrow{f}'\left(x-2\right)>0,\forall x\in\left(-\infty ;1\right)\cup\left(3;+\infty\right)$ .\\
		Đặt $ x*=x-2$ suy ra $f'\left(x*\right)>0,\forall x*\in\left(-\infty ;-1\right)\bigcup\left(1;+\infty\right)$.\\
		Vậy hàm số $ f(x)$ đồng biến trên khoảng $\left(-\infty ;-1\right),\,\,\left(1;+\infty\right)$.}
\end{ex}

\begin{ex}%[2D1G1-2]%Câu 16
	\immini{
		Cho hàm số $ y=f(x)$ có đạo hàm là hàm số $f'(x)$ trên $\mathbb{R}$. Biết rằng hàm số $ y=f'\left(x+2\right)-2$ có đồ thị như hình vẽ bên dưới. Hàm số $ f(x)$ nghịch biến trên khoảng nào?
		\choice
		{$\left(-3;-1\right),\,\,\left(1;3\right)$}
		{\True $\left(-1;1\right),\,\,\left(3;5\right)$}
		{$\left(-\infty ;-2\right),\,\,\left(0;2\right)$}
		{$\left(-5;-3\right),\,\,\left(-1;1\right)$}}{
		\begin{tikzpicture}[scale=0.7,font=\footnotesize, line join=round, line cap=round, >=stealth] %Đường cong bậc 3
			\draw[thick, ->] (-3.8,0)--(4.0,0);
			\draw[thick, ->] (0,-4.8)--(0,3.5);
			\draw (4.2,0) node[below] {$x$};
			\draw (0,3.7) node[left]{$y$};
			\draw (0,0) node[below left]{$0$};
			\draw[fill] (-3,0) circle (0.5pt)node[above]{$ -3$};
			\draw[fill] (-1,0) circle (0.5pt)node[above]{$ -1$};
			\draw[fill] (1,0) circle (0.5pt)node[above]{$ 1$};
			\draw[fill] (3,0) circle (0.5pt)node[above]{$3$};
			\draw[fill] (0,2) circle (0.5pt)node[above left]{$ 2$};
			\draw[fill] (0,-1) circle (0.5pt)node[above right]{$ -1$};
			%		\draw[fill] (0,2) circle (0.5pt)node[above left]{$ 2$};
			%		\draw[fill] (0,-2) circle (0.5pt)node[below left]{$ -2$};
			\draw[dashed] (-3,0)--(-3,-2)--(3,-2)--(3,0) (-1,0)--(-1,-2) (1,0)--(1,-2) (-3.494,0)--(-3.494,2)--(3.494,2)--(3.494,0); 
			\draw[line width=1.2pt,smooth,samples=100,domain=-3.6:3.6] plot(\x,{0.11*(\x)^4-1.11*(\x)^2-1});		
			%\draw[line width=1.2pt,smooth,samples=100,domain=-3.3:2.8] plot(\x,{0.75*(\x)^2+0.5*\x-1});
			%	\draw (2.0,2.8) node[left]{$y=f'(x)$};
	\end{tikzpicture}	}
	\loigiai{
		Hàm số $ y=f'\left(x+2\right)-2$ có đồ thị $(C)$ như sau
		\begin{center}
			\begin{tikzpicture}[scale=0.7,font=\footnotesize, line join=round, line cap=round, >=stealth] %Đường cong bậc 3
				\draw[thick, ->] (-3.8,0)--(4.0,0);
				\draw[thick, ->] (0,-4.8)--(0,3.5);
				\draw (4.2,0) node[below] {$x$};
				\draw (0,3.7) node[left]{$y$};
				\draw (0,0) node[below left]{$0$};
				\draw[fill] (-3,0) circle (0.5pt)node[above]{$ -3$};
				\draw[fill] (-1,0) circle (0.5pt)node[above]{$ -1$};
				\draw[fill] (1,0) circle (0.5pt)node[above]{$ 1$};
				\draw[fill] (3,0) circle (0.5pt)node[above]{$3$};
				\draw[fill] (0,2) circle (0.5pt)node[above left]{$ 2$};
				\draw[fill] (0,-1) circle (0.5pt)node[above right]{$ -1$};
				%		\draw[fill] (0,2) circle (0.5pt)node[above left]{$ 2$};
				%		\draw[fill] (0,-2) circle (0.5pt)node[below left]{$ -2$};
				\draw[dashed] (-3,0)--(-3,-2)--(3,-2)--(3,0) (-1,0)--(-1,-2) (1,0)--(1,-2) (-3.494,0)--(-3.494,2)--(3.494,2)--(3.494,0); 
				\draw[line width=1.2pt,smooth,samples=100,domain=-3.6:3.6] plot(\x,{0.11*(\x)^4-1.11*(\x)^2-1});		
				%\draw[line width=1.2pt,smooth,samples=100,domain=-3.3:2.8] plot(\x,{0.75*(\x)^2+0.5*\x-1});
				%	\draw (2.0,2.8) node[left]{$y=f'(x)$};
			\end{tikzpicture}
		\end{center}
		Dựa vào đồ thị $(C)$ ta có\\
		$f'\left(x+2\right)-2<-2,\forall x\in\left(-3;-1\right)\bigcup\left(1;3\right)\Leftrightarrow{f}'\left(x+2\right)<0,\forall x\in\left(-3;-1\right)\bigcup\left(1;3\right)$.\\
		Đặt $ x^*=x+2$ suy ra: $f'\left(x^*\right)<0,\forall x^*\in\left(-1;1\right)\bigcup\left(3;5\right)$.\\
		Vậy: Hàm số $ f(x)$ đồng biến trên khoảng $\left(-1;1\right),\,\,\left(3;5\right)$.}
\end{ex}

\begin{ex}%[2D1G1-2]%Câu 17
	\immini{
		Cho hàm số $ y=f(x)$ có đạo hàm là hàm số $f'(x)$ trên $\mathbb{R}$. Biết rằng hàm số $ y=f'\left(x-2\right)+2$ có đồ thị như hình vẽ bên dưới. Hàm số $ f(x)$ nghịch biến trên khoảng nào?
		\choice
		{$\left(-\infty ;2\right)$}
		{\True $\left(-1;1\right)$}
		{$\left(\dfrac{3}{2};\dfrac{5}{2}\right)$}
		{$\left(2;+\infty\right)$}}{
		\begin{tikzpicture}[scale=0.7,font=\footnotesize, line join=round, line cap=round, >=stealth] %Đường cong bậc 3
			\draw[thick, ->] (-0.5,0)--(3.5,0);
			\draw[thick, ->] (0,-1.8)--(0,5.3);
			\draw (3.7,0) node[below] {$x$};
			\draw (0,5.4) node[left]{$y$};
			\draw (0,0) node[below left]{$0$};
			\draw[fill] (3,0) circle (0.5pt)node[below]{$ 3$};
			\draw[fill] (1,0) circle (0.5pt)node[below]{$ 1$};
			\draw[fill] (2,0) circle (0.5pt)node[above]{$2$};
			\draw[fill] (0,2) circle (0.5pt)node[left]{$ 2$};
			\draw[fill] (0,-1) circle (0.5pt)node[left]{$ -1$};
			%		\draw[fill] (0,2) circle (0.5pt)node[above left]{$ 2$};
			%		\draw[fill] (0,-2) circle (0.5pt)node[below left]{$ -2$};
			\draw[dashed] (3,0)--(3,2)--(0,2)--(1,2)--(1,0); 
			\draw[dashed](0,-1)--(2,-1)--(2,0);
			\draw[line width=1.2pt,smooth,samples=100,domain=0.6:3.4] plot(\x,{3*(\x)^2-12*(\x)+11});		
			%\draw[line width=1.2pt,smooth,samples=100,domain=-3.3:2.8] plot(\x,{0.75*(\x)^2+0.5*\x-1});
			%	\draw (2.0,2.8) node[left]{$y=f'(x)$};
	\end{tikzpicture}	}
	\loigiai{
		Hàm số $ y=f'\left(x-2\right)+2$ có đồ thị $(C)$ như sau
		\begin{center}
			\begin{tikzpicture}[scale=0.7,font=\footnotesize, line join=round, line cap=round, >=stealth] %Đường cong bậc 3
				\draw[thick, ->] (-0.5,0)--(3.5,0);
				\draw[thick, ->] (0,-1.8)--(0,5.3);
				\draw (3.7,0) node[below] {$x$};
				\draw (0,5.4) node[left]{$y$};
				\draw (0,0) node[below left]{$0$};
				\draw[fill] (3,0) circle (0.5pt)node[below]{$ 3$};
				\draw[fill] (1,0) circle (0.5pt)node[below]{$ 1$};
				\draw[fill] (2,0) circle (0.5pt)node[above]{$2$};
				\draw[fill] (0,2) circle (0.5pt)node[left]{$ 2$};
				\draw[fill] (0,-1) circle (0.5pt)node[left]{$ -1$};
				%		\draw[fill] (0,2) circle (0.5pt)node[above left]{$ 2$};
				%		\draw[fill] (0,-2) circle (0.5pt)node[below left]{$ -2$};
				\draw[dashed] (3,0)--(3,2)--(0,2)--(1,2)--(1,0); 
				\draw[dashed](0,-1)--(2,-1)--(2,0);
				\draw[line width=1.2pt,smooth,samples=100,domain=0.6:3.4] plot(\x,{3*(\x)^2-12*(\x)+11});		
				%\draw[line width=1.2pt,smooth,samples=100,domain=-3.3:2.8] plot(\x,{0.75*(\x)^2+0.5*\x-1});
				%	\draw (2.0,2.8) node[left]{$y=f'(x)$};
			\end{tikzpicture}
		\end{center}
		Dựa vào đồ thị $(C)$ ta có\\
		$f'\left(x-2\right)+2<2,\forall x\in\left(1;3\right)\Leftrightarrow{f}'\left(x-2\right)<0,\forall x\in\left(1;3\right)$.\\
		Đặt $ x^*=x-2$ thì $f'\left(x^*\right)<0,\forall x^*\in\left(-1;1\right)$.\\
		Vậy: Hàm số $ f(x)$ nghịch biến trên khoảng $\left(-1;1\right)$.\\
		Cách khác:\\
		Tịnh tiến sang trái hai đơn vị và xuống dưới $2$ đơn vị thì từ đồ thị $(C)$ sẽ thành đồ thị của hàm$ y=f'(x)$. Khi đó $f'(x)<0,\forall x\in\left(-1;1\right)$.\\
		Vậy hàm số $ f(x)$ nghịch biến trên khoảng $\left(-1;1\right)$.}
\end{ex}

\begin{ex}%[2D1G1-2]%Câu 18
	Cho hàm số $y=f(x)$ có đạo hàm cấp $ 3$ liên tục trên $\mathbb{R}$ và thỏa mãn $f(x)\cdot f'''(x)=x{\left(x-1\right)^2}{\left(x+4\right)^3}$ với mọi $x\in\mathbb{R}$ và $g(x)=\left[f'(x)\right]^2-2f(x)\cdot f''(x)$. Hàm số $h(x)=g\left(x^2-2x\right)$ đồng biến trên khoảng nào dưới đây?
	\choice
	{$\left(-\infty ;1\right)$}
	{$\left(2;+\infty\right)$}
	{$\left(0;1\right)$}
	{\True $\left(1;2\right)$}
	\loigiai{		
		Ta có $g'(x)=2f''(x){f}'(x)-2f'(x)\cdot f''(x)-2f(x)\cdot f'''(x)=-2f(x)\cdot f'''(x);$\\
		Khi đó $\left(h(x)\right)'=\left(2x-2\right){g}'\left(x^2-2x\right)=-2\left(2x-2\right)\left(x^2-2x\right){\left(x^2-2x-1\right)^2}{\left(x^2-2x+4\right)^3}$\\
		$h'(x)=0\Leftrightarrow\hoac{
			& x=0\\ 
			& x=1\\ 
			& x=2\\ 
			& x=1\pm\sqrt{2}.}$ 
		Ta có bảng xét dấu của $h'(x)$
		\begin{center}
			\begin{tikzpicture}
				\tkzTabInit[lgt=1.2,espcl=2,nocadre]
				{$t$/0.7, $h'(x)$ /.7} % first column
				{$-\infty$, $1-\sqrt{2}$,$0$, $1$,$2$,$1+\sqrt{2}$, $+\infty$} % first row
				\tkzTabLine { ,+,0,-,0,+,0,-,0,+,0,- } % second row
				%				\tkzTabLine {,-,z,+,t,+,} % third row
				%				\tkzTabLine {,+,d,-,z,+,} % last row
			\end{tikzpicture}
		\end{center}
		Suy ra hàm số $h(x)=g\left(x^2-2x\right)$ đồng biến trên khoảng $\left(1;2\right)$.}
\end{ex}

\begin{ex}%[2D1G1-2]%Câu 19
	Cho hàm số $ y=f(x)$ xác định trên $\mathbb{R}$. Hàm số $ y=g(x)=f'\left(2x+3\right)+2$ có đồ thị là một parabol với tọa độ đỉnh $ I\left(2;-1\right)$ và đi qua điểm $ A\left(1;2\right)$. Hỏi hàm số $ y=f(x)$ nghịch biến trên khoảng nào dưới đây?
	\choice
	{\True $\left(5;9\right)$}
	{$\left(1;2\right)$}
	{$\left(-\infty ;9\right)$}
	{$\left(1;3\right)$}
	\loigiai{	
		Xét hàm số $ g(x)=f'\left(2x+3\right)+2$ có đồ thị là một Parabol nên có phương trình dạng $ y=g(x)=a{x^2}+bx+c\,\,\,\,(P)$.\\
		Vì $(P)$ có đỉnh $ I\left(2;-1\right)$ nên $\heva{
			&\dfrac{-b}{2a}=2\\ 
			& g(2)=-1} \Leftrightarrow\heva{
			&-b=4a\\ 
			& 4a+2b+c=-1} \Leftrightarrow\heva{
			& 4a+b=0\\ 
			& 4a+2b+c=-1}$.\\
		Vì $(P)$ đi qua điểm $ A\left(1;2\right)$ nên $ g(1)=2\Leftrightarrow a+b+c=2$.\\
		Ta có hệ phương trình $\heva{
			& 4a+b=0\\ 
			& 4a+2b+c=-1\\ 
			& a+b+c=2} \Leftrightarrow\heva{
			& a=3\\ 
			& b=-12\\ 
			& c=11}$ nên $ g(x)=3x^2-12x+11$.\\
		Đồ thị của hàm $ y=g(x)$ là
		\begin{center}
			\begin{tikzpicture}[scale=0.7,font=\footnotesize, line join=round, line cap=round, >=stealth] %Đường cong bậc 3
				\draw[thick, ->] (-0.5,0)--(3.5,0);
				\draw[thick, ->] (0,-1.8)--(0,5.3);
				\draw (3.7,0) node[below] {$x$};
				\draw (0,5.4) node[left]{$y$};
				\draw (0,0) node[below left]{$0$};
				\draw[fill] (3,0) circle (0.5pt)node[below]{$ 3$};
				\draw[fill] (1,0) circle (0.5pt)node[below]{$ 1$};
				\draw[fill] (2,0) circle (0.5pt)node[above]{$2$};
				\draw[fill] (0,2) circle (0.5pt)node[left]{$ 2$};
				\draw[fill] (0,-1) circle (0.5pt)node[left]{$ -1$};
				%		\draw[fill] (0,2) circle (0.5pt)node[above left]{$ 2$};
				%		\draw[fill] (0,-2) circle (0.5pt)node[below left]{$ -2$};
				\draw[dashed] (3,0)--(3,2)--(0,2)--(1,2)--(1,0) (3.2,2)--(3,2); 
				\draw[dashed](0,-1)--(2,-1)--(2,0);
				\draw[line width=1.2pt,smooth,samples=100,domain=0.6:3.4] plot(\x,{3*(\x)^2-12*(\x)+11});		
				%\draw[line width=1.2pt,smooth,samples=100,domain=-3.3:2.8] plot(\x,{0.75*(\x)^2+0.5*\x-1});
				%	\draw (2.0,2.8) node[left]{$y=f'(x)$};
			\end{tikzpicture}	
		\end{center}
		Theo đồ thị ta thấy $ f'(2x+3)\le 0\Leftrightarrow f'(2x+3)+2\le 2\Leftrightarrow 1\le x\le 3$.\\
		Đặt $ t=2x+3\Leftrightarrow x=\dfrac{t-3}{2}$ khi đó $ f'(t)\le 0\Leftrightarrow 1\le\dfrac{t-3}{2}\le 3\Leftrightarrow 5\le t\le 9$.\\
		Vậy $ y=f(x)$ nghịch biến trên khoảng $\left(5;9\right)$.}
\end{ex}

\begin{ex}%[2D1G1-2]%Câu 20
	\immini{
		Cho hàm số $ y=f(x)$, hàm số $f'(x)=x^3+a{x^2}+bx+c\left(a,b,c\in\mathbb{R}\right)$ có đồ thị như hình vẽ bên.
		Hàm số $ g(x)=f\left(f'(x)\right)$ nghịch biến trên khoảng nào dưới đây?
		\choice
		{$\left(1;+\infty\right)$}
		{\True $\left(-\infty ;-2\right)$}
		{$\left(-1;0\right)$}
		{$\left(-\dfrac{\sqrt{3}}{3};\dfrac{\sqrt{3}}{3}\right)$}}{
		\begin{tikzpicture}[scale=0.8,font=\footnotesize, line join=round, line cap=round, >=stealth] %Đường cong bậc 3
			\draw[thick, ->] (-1.7,0)--(1.7,0);
			\draw[thick, ->] (0,-2.7)--(0,3.0);
			\draw (1.9,0) node[below] {$x$};
			\draw (0,3.2) node[left]{$y$};
			\draw (0,0) node[below left]{$0$};
			\draw[fill] (-1,0) circle (0.5pt)node[above left]{$ -1 $};
			\draw[fill] (1,0) circle (0.5pt)node[below right]{$ 1$};
			\draw[line width=1.2pt,smooth,samples=100,domain=-1.3:1.3] plot(\x,{2.667*(\x)^3+0*(\x)^2-2.667*\x});		
			%\draw[line width=1.2pt,smooth,samples=100,domain=-3.3:2.8] plot(\x,{0.75*(\x)^2+0.5*\x-1});
		\end{tikzpicture}	
	}
	\loigiai{	
		Vì các điểm $\left(-1;0\right),\left(0;0\right),\left(1;0\right)$ thuộc đồ thị hàm số $ y=f'(x)$ nên ta có hệ\\
		$\heva{
			&-1+a-b+c=0\\ 
			& c=0\\ 
			& 1+a+b+c=0} \Leftrightarrow\heva{
			& a=0\\ 
			& b=-1\\ 
			& c=0} \Rightarrow {f}'(x)=x^3-x\Rightarrow f''(x)=3x^2-1$.\\
		Ta có $ g(x)=f\left(f'(x)\right)\Rightarrow{g}'(x)=f'\left(f'(x)\right)\cdot f''(x)$.\\
		Xét \\
		$g'(x)=0\Leftrightarrow{g}'(x)=f'\left(f'(x)\right)\cdot f''(x)=0$\\
		$\Leftrightarrow {f}'\left(x^3-x\right)\left(3x^2-1\right)=0\Leftrightarrow\hoac{
			&{x^3}-x=0\\ 
			&{x^3}-x=1\\ 
			&{x^3}-x=-1\\ 
			& 3x^2-1=0} \Leftrightarrow \hoac{
			& x=\pm 1\\ 
			& x=0\\ 
			& x=x_1(x_1\approx 1,325)\\ 
			& x=x_2(x_2\approx-1,325)\\ 
			& x=\pm\dfrac{\sqrt{3}}{3}.}$\\
		Bảng biến thiên
		\begin{center}
			\begin{tikzpicture}
				\tkzTabInit[lgt=1.2,espcl=2,nocadre]
				{$t$/0.7, $h'(x)$ /.7} % first column
				{$-\infty$, $-1{,}325$,$-1$, $-\dfrac{\sqrt{3}}{3}$,$0$,$\dfrac{\sqrt{3}}{3}$,$1$,$1{,}325$, $+\infty$} % first row
				\tkzTabLine { ,-,0,+,0,-,0,+,0,-,0,+,0,-,0,+, } % second row
				%				\tkzTabLine {,-,z,+,t,+,} % third row
				%				\tkzTabLine {,+,d,-,z,+,} % last row
			\end{tikzpicture}
		\end{center}
		Dựa vào bảng biến thiên ta có $ g(x)$ nghịch biến trên $\left(-\infty ;-2\right)$}
\end{ex}
\Closesolutionfile{ans}
\indapan{10}{ans/CD1/Muc_9_10}
\chapter{KHỐI TRỤ}
\begin{Solution}{1}
C
\end{Solution}
\begin{Solution}{3}
B
\end{Solution}
\begin{Solution}{4}
A
\end{Solution}
\begin{Solution}{5}
A
\end{Solution}
\begin{Solution}{6}
A
\end{Solution}
\begin{Solution}{7}
B
\end{Solution}
\begin{Solution}{8}
A
\end{Solution}
\begin{Solution}{9}
C
\end{Solution}
\begin{Solution}{10}
B
\end{Solution}
\begin{Solution}{11}
C
\end{Solution}
\begin{Solution}{12}
D
\end{Solution}
\begin{Solution}{13}
B
\end{Solution}
\begin{Solution}{14}
D
\end{Solution}
\begin{Solution}{15}
A
\end{Solution}
\begin{Solution}{16}
B
\end{Solution}
\begin{Solution}{17}
C
\end{Solution}
\begin{Solution}{18}
C
\end{Solution}
\begin{Solution}{19}
C
\end{Solution}
\begin{Solution}{20}
B
\end{Solution}
\begin{Solution}{21}
C
\end{Solution}
\begin{Solution}{22}
B
\end{Solution}
\begin{Solution}{23}
D
\end{Solution}
\begin{Solution}{24}
B
\end{Solution}
\begin{Solution}{25}
D
\end{Solution}
\begin{Solution}{26}
D
\end{Solution}
\begin{Solution}{27}
B
\end{Solution}
\begin{Solution}{28}
A
\end{Solution}
\begin{Solution}{29}
C
\end{Solution}
\begin{Solution}{30}
B
\end{Solution}
\begin{Solution}{31}
D
\end{Solution}
\begin{Solution}{32}
B
\end{Solution}
\begin{Solution}{33}
B
\end{Solution}
\begin{Solution}{34}
C
\end{Solution}
\begin{Solution}{35}
D
\end{Solution}
\begin{Solution}{36}
B
\end{Solution}
\begin{Solution}{37}
B
\end{Solution}
\begin{Solution}{38}
A
\end{Solution}
\begin{Solution}{39}
A
\end{Solution}
\begin{Solution}{40}
D
\end{Solution}
\begin{Solution}{41}
C
\end{Solution}
\begin{Solution}{42}
B
\end{Solution}
\begin{Solution}{43}
A
\end{Solution}
\begin{Solution}{44}
A
\end{Solution}
\begin{Solution}{45}
D
\end{Solution}
\begin{Solution}{46}
C
\end{Solution}
\begin{Solution}{47}
A
\end{Solution}
\begin{Solution}{48}
B
\end{Solution}
\begin{Solution}{49}
B
\end{Solution}
\begin{Solution}{50}
B
\end{Solution}
\begin{Solution}{51}
A
\end{Solution}
\begin{Solution}{52}
A
\end{Solution}
\begin{Solution}{53}
C
\end{Solution}
\begin{Solution}{54}
C
\end{Solution}
\begin{Solution}{55}
C
\end{Solution}
\begin{Solution}{56}
B
\end{Solution}
\begin{Solution}{57}
C
\end{Solution}
\begin{Solution}{58}
C
\end{Solution}
\begin{Solution}{59}
B
\end{Solution}
\begin{Solution}{60}
C
\end{Solution}
\begin{Solution}{61}
A
\end{Solution}
\begin{Solution}{62}
B
\end{Solution}
\begin{Solution}{63}
B
\end{Solution}
\begin{Solution}{64}
D
\end{Solution}
\begin{Solution}{65}
D
\end{Solution}
\begin{Solution}{66}
B
\end{Solution}
\begin{Solution}{67}
A
\end{Solution}
\begin{Solution}{68}
D
\end{Solution}

\begin{Solution}{1}
D
\end{Solution}
\begin{Solution}{2}
C
\end{Solution}
\begin{Solution}{3}
C
\end{Solution}
\begin{Solution}{4}
A
\end{Solution}
\begin{Solution}{5}
B
\end{Solution}
\begin{Solution}{6}
D
\end{Solution}
\begin{Solution}{7}
C
\end{Solution}
\begin{Solution}{8}
D
\end{Solution}
\begin{Solution}{9}
A
\end{Solution}
\begin{Solution}{10}
B
\end{Solution}
\begin{Solution}{11}
D
\end{Solution}
\begin{Solution}{12}
A
\end{Solution}
\begin{Solution}{13}
D
\end{Solution}
\begin{Solution}{14}
B
\end{Solution}
\begin{Solution}{15}
B
\end{Solution}
\begin{Solution}{16}
C
\end{Solution}
\begin{Solution}{1}
A
\end{Solution}
\begin{Solution}{2}
B
\end{Solution}
\begin{Solution}{3}
D
\end{Solution}
\begin{Solution}{4}
D
\end{Solution}
\begin{Solution}{5}
C
\end{Solution}
\begin{Solution}{6}
A
\end{Solution}
\begin{Solution}{7}
D
\end{Solution}
\begin{Solution}{8}
B
\end{Solution}
\begin{Solution}{9}
C
\end{Solution}
\begin{Solution}{10}
C
\end{Solution}
\begin{Solution}{1}
D
\end{Solution}
\begin{Solution}{2}
D
\end{Solution}
\begin{Solution}{3}
B
\end{Solution}
\begin{Solution}{4}
C
\end{Solution}
\begin{Solution}{5}
D
\end{Solution}
\begin{Solution}{6}
A
\end{Solution}
\begin{Solution}{7}
C
\end{Solution}
\begin{Solution}{8}
B
\end{Solution}
\begin{Solution}{9}
A
\end{Solution}
\begin{Solution}{10}
C
\end{Solution}
\begin{Solution}{11}
D
\end{Solution}
\begin{Solution}{12}
C
\end{Solution}
\begin{Solution}{13}
A
\end{Solution}
\begin{Solution}{14}
D
\end{Solution}
\begin{Solution}{15}
A
\end{Solution}
\begin{Solution}{16}
A
\end{Solution}
\begin{Solution}{17}
B
\end{Solution}
\begin{Solution}{18}
C
\end{Solution}
\begin{Solution}{19}
C
\end{Solution}
\begin{Solution}{20}
A
\end{Solution}
\begin{Solution}{21}
D
\end{Solution}
\begin{Solution}{22}
C
\end{Solution}
\begin{Solution}{23}
A
\end{Solution}
\begin{Solution}{24}
C
\end{Solution}
\begin{Solution}{25}
A
\end{Solution}
\begin{Solution}{26}
B
\end{Solution}
\begin{Solution}{27}
B
\end{Solution}
\begin{Solution}{28}
D
\end{Solution}
\begin{Solution}{29}
B
\end{Solution}
\begin{Solution}{30}
D
\end{Solution}
\begin{Solution}{31}
D
\end{Solution}
\begin{Solution}{32}
C
\end{Solution}
\begin{Solution}{33}
D
\end{Solution}
\begin{Solution}{34}
C
\end{Solution}
\begin{Solution}{35}
D
\end{Solution}
\begin{Solution}{36}
D
\end{Solution}
\begin{Solution}{37}
D
\end{Solution}
\begin{Solution}{38}
D
\end{Solution}
\begin{Solution}{39}
D
\end{Solution}
\begin{Solution}{40}
C
\end{Solution}
\begin{Solution}{41}
A
\end{Solution}
\begin{Solution}{1}
A
\end{Solution}
\begin{Solution}{2}
B
\end{Solution}
\begin{Solution}{3}
C
\end{Solution}
\begin{Solution}{4}
A
\end{Solution}
\begin{Solution}{5}
A
\end{Solution}
\begin{Solution}{6}
C
\end{Solution}
\begin{Solution}{7}
C
\end{Solution}
\begin{Solution}{8}
B
\end{Solution}
\begin{Solution}{9}
C
\end{Solution}
\begin{Solution}{10}
B
\end{Solution}
\begin{Solution}{11}
A
\end{Solution}
\begin{Solution}{12}
B
\end{Solution}
\begin{Solution}{13}
B
\end{Solution}
\begin{Solution}{14}
B
\end{Solution}
\begin{Solution}{15}
A
\end{Solution}
\begin{Solution}{16}
B
\end{Solution}
\begin{Solution}{17}
A
\end{Solution}
\begin{Solution}{18}
D
\end{Solution}
\begin{Solution}{19}
C
\end{Solution}
\begin{Solution}{20}
C
\end{Solution}
\begin{Solution}{21}
A
\end{Solution}
\begin{Solution}{22}
C
\end{Solution}
\begin{Solution}{23}
C
\end{Solution}
\begin{Solution}{24}
A
\end{Solution}
\begin{Solution}{25}
B
\end{Solution}
\begin{Solution}{26}
B
\end{Solution}
\begin{Solution}{27}
A
\end{Solution}
\begin{Solution}{28}
A
\end{Solution}
\begin{Solution}{29}
C
\end{Solution}
\begin{Solution}{30}
B
\end{Solution}
\begin{Solution}{31}
A
\end{Solution}
\begin{Solution}{32}
C
\end{Solution}
\begin{Solution}{33}
B
\end{Solution}
\begin{Solution}{34}
A
\end{Solution}
\begin{Solution}{35}
B
\end{Solution}
\begin{Solution}{36}
B
\end{Solution}
\begin{Solution}{37}
B
\end{Solution}
\begin{Solution}{38}
D
\end{Solution}
\begin{Solution}{39}
B
\end{Solution}
\begin{Solution}{40}
A
\end{Solution}
\begin{Solution}{41}
D
\end{Solution}
\begin{Solution}{42}
D
\end{Solution}
\begin{Solution}{43}
A
\end{Solution}
\begin{Solution}{44}
D
\end{Solution}
\begin{Solution}{45}
C
\end{Solution}
\begin{Solution}{46}
B
\end{Solution}
\begin{Solution}{47}
A
\end{Solution}
\begin{Solution}{48}
D
\end{Solution}
\begin{Solution}{49}
B
\end{Solution}
\begin{Solution}{50}
B
\end{Solution}
\begin{Solution}{51}
D
\end{Solution}
\begin{Solution}{52}
C
\end{Solution}
\begin{Solution}{53}
C
\end{Solution}
\begin{Solution}{54}
B
\end{Solution}
\begin{Solution}{55}
D
\end{Solution}
\begin{Solution}{56}
B
\end{Solution}
\begin{Solution}{57}
C
\end{Solution}
\begin{Solution}{58}
A
\end{Solution}
\begin{Solution}{59}
A
\end{Solution}
\begin{Solution}{60}
B
\end{Solution}
\begin{Solution}{61}
D
\end{Solution}
\begin{Solution}{62}
D
\end{Solution}
\begin{Solution}{63}
B
\end{Solution}
\begin{Solution}{64}
A
\end{Solution}
\begin{Solution}{65}
D
\end{Solution}
\begin{Solution}{66}
C
\end{Solution}
\begin{Solution}{67}
A
\end{Solution}
\begin{Solution}{68}
A
\end{Solution}
\begin{Solution}{69}
D
\end{Solution}
\begin{Solution}{70}
C
\end{Solution}
\begin{Solution}{71}
B
\end{Solution}
\begin{Solution}{72}
A
\end{Solution}
\begin{Solution}{73}
C
\end{Solution}
\begin{Solution}{74}
C
\end{Solution}
\begin{Solution}{75}
C
\end{Solution}
\begin{Solution}{76}
A
\end{Solution}
\begin{Solution}{77}
C
\end{Solution}
\begin{Solution}{78}
B
\end{Solution}
\begin{Solution}{79}
D
\end{Solution}
\begin{Solution}{80}
B
\end{Solution}

\section{Mức 9,10 điểm}
\setcounter{ex}{0}
\setcounter{dang}{0}
\Opensolutionfile{ans}[ans/CD1/Muc_9_10]
\begin{dang}{Tìm m để hàm số đơn điệu trên các khoảng xác định của nó}
	Đang thiếu bài thầy Jf Câu 1 đến 26 
\end{dang}
\begin{dang}
	{Tìm khoảng đơn điệu của hàm số $g(x) = f\left[ u(x)\right] +v(x)$ khi biết đồ thị hoặc bảng biến thiên của hàm số $y = f'(x)$}
\end{dang}
\begin{ex}[Đề tham khảo 2019]%[2D1K1-2]
	Cho hàm số $f(x)$ có bảng xét dấu của đạo hàm như sau
	\begin{center}
		\begin{tikzpicture}
			\tkzTabInit[nocadre,lgt=1.2,espcl=2,deltacl=0.6]
			{$x$ /0.6,$f'(x)$ /0.6}
			{$-\infty$,$1$,$2$,$3$,$4$,$+\infty$}
			\tkzTabLine{,-,$0$,+,$0$,+,$0$,-,$0$,+,}
		\end{tikzpicture}
	\end{center}
	Hàm số $y=3 f(x+2)-x^3+3 x$ đồng biến trên khoảng nào dưới đây?
	\choice
	{$(-\infty ;-1)$}
	{\True $(-1 ; 0)$}
	{$(0 ; 2)$}
	{$(1 ;+\infty)$}
	\loigiai{
		Ta có $y'=3\left[f'(x+2)-\left(x^2-3\right)\right]$.\\
		Với $x \in(-1 ; 0) \Rightarrow x+2 \in(1 ; 2) \Rightarrow f'(x+2)>0$, lại có $x^2-3<0 \Rightarrow y'>0 ;~ \forall x \in(-1 ; 0)$.\\
		Vậy hàm số $y=3 f(x+2)-x^3+3 x$ đồng biến trên khoảng $(-1 ; 0)$.\\
		Chú ý:\\
		+) Ta xét $x \in(1 ; 2) \subset(1 ;+\infty)
		\Rightarrow x+2 \in(3 ; 4)\\
		\Rightarrow f'(x+2)<0 ;~ x^2-3>0$\\
		Suy ra hàm số nghịch biến trên khoảng $(1 ; 2)$ nên loại hai phương án$(0 ; 2)$ và $(1 ;+\infty)$.\\
		+) Tương tự ta xét
		$x \in(-\infty ;-2) \Rightarrow x+2 \in(-\infty ; 0)\\
		\Rightarrow f'(x+2)<0 ; x^2-3>0 \Rightarrow y'<0 ; ~ \forall x \in(-\infty ;-2)$.\\
		Suy ra hàm số nghịch biến trên khoảng $(-\infty ;-2)$ nên loại$(-\infty ;-1)$.\\
		Vậy hàm số đã cho đồng biến trên khoảng $(-1 ; 0)$.
	}
\end{ex}
\begin{ex}[Đề Tham Khảo 2020 - Lần 1]%[2D1G1-2]
	\immini{
		Cho hàm số $f(x)$. Hàm số $y=f'(x)$ có đồ thị như hình bên. Hàm số $g(x)=f(1-2 x)+x^2-x$ nghịch biến trên khoảng nào dưới đây?
		\choice
		{\True $\left(1 ; \dfrac{3}{2}\right)$}
		{$\left(0 ; \dfrac{1}{2}\right)$}
		{$(-2 ;-1)$}
		{$(2 ; 3)$}
	}
	{
		\begin{tikzpicture}[scale=0.7,>=stealth, font=\footnotesize, line join=round, line cap=round]
			%\def\a{1} \def\b{-6} \def\c{9} \def\d{1} % Hệ số
			\def\xmin{-4} \def\xmax{6}
			\def\ymin{-3} \def\ymax{2} 
			%\draw[color=gray!50,dashed] (\xmin,\ymin) grid (\xmax,\ymax); 
			\draw[->] (\xmin,0)--(\xmax,0) node [below]{$x$};
			\draw[->] (0,\ymin)--(0,\ymax) node [left]{$y$};
			\node at (0,0) [below left]{$O$};
			%\node at (1,3) [below left]{$f'(x)$};
			%\node at (-1.3,4) {$f'(x)$};
			\draw[dashed] (-2,0) node[below]{$-2$}--(-2,1)--(0,1) node[below left]{$1$};
			\draw[dashed] (4,0) node[below left]{$4$}--(4,-2)--(0,-2) node[below left]{$-2$};
			%\draw[dashed] (1,0) node[below]{$1$}--(1,1);
			%\draw[dashed] (-0.5,0) node[below left]{$-0{,}5$}--(-0.5,2.125);
			\clip (\xmin+0.1,\ymin+0.1) rectangle (\xmax-0.5,\ymax-0.1);
			\draw[smooth,samples=300][domain=-4:5.5] plot(\x,{0.071*(\x)^3-0.142*(\x)^2-1.07*(\x)});
		\end{tikzpicture}
	}
	
	\loigiai{
		Ta có : $g(x)=f(1-2 x)+x^2-x \Rightarrow g'(x)=-2 f'(1-2 x)+2 x-1$.\\
		\immini{
			Đặt $t=1-2 x \Rightarrow g'(x)=-2 f'(t)-t$.\\
			$g'(x)=0 \Rightarrow f'(t)=-\dfrac{t}{2}$.\\
			Vẽ đường thẳng $y=-\dfrac{x}{2}$ và đồ thị hàm số $f'(x)$ trên cùng một hệ trục
		}	
		{
			\begin{tikzpicture}[scale=0.7,>=stealth, font=\footnotesize, line join=round, line cap=round]
				%\def\a{1} \def\b{-6} \def\c{9} \def\d{1} % Hệ số
				\def\xmin{-4} \def\xmax{6}
				\def\ymin{-3} \def\ymax{2} 
				%	\draw[color=gray!50,dashed] (\xmin,\ymin) grid (\xmax,\ymax); 
				\draw[->] (\xmin,0)--(\xmax,0) node [below]{$x$};
				\draw[->] (0,\ymin)--(0,\ymax) node [left]{$y$};
				\node at (0,0) [below left]{$O$};
				%\node at (1,3) [below left]{$f'(x)$};
				%\node at (-1.3,4) {$f'(x)$};
				\draw[dashed] (-2,0) node[below]{$-2$}--(-2,1)--(0,1) node[below left]{$1$};
				\draw[dashed] (4,0) node[below]{$4$}--(4,-2)--(0,-2) node[below left]{$-2$};
				%\draw[dashed] (1,0) node[below]{$1$}--(1,1);
				%\draw[dashed] (-0.5,0) node[below left]{$-0{,}5$}--(-0.5,2.125);
				\clip (\xmin+0.1,\ymin+0.1) rectangle (\xmax-0.5,\ymax-0.1);
				\draw[smooth,samples=300][domain=-4:5.5] plot(\x,{0.071*(\x)^3-0.142*(\x)^2-1.07*(\x)});
				\draw[smooth,samples=300][domain=-4:5.5] plot(\x,{(-0.5*(\x)});
			\end{tikzpicture}
		}	Hàm số $g(x)$ nghịch biến $\Rightarrow g'(x) \leq 0 \Rightarrow f'(t) \geq-\dfrac{t}{2}\Rightarrow\hoac{&-2 \leq t \leq 0 \\&t \geq 4.}$\\
		Như vậy $f'(1-2 x) \geq \dfrac{1-2 x}{-2}\Rightarrow\hoac{&-2 \leq 1-2 x \leq 0 \\ &4 \leq 1-2 x}\Rightarrow\hoac{&\dfrac{1}{2}\leq x \leq \dfrac{3}{2}\\ &x \leq-\dfrac{3}{2}.}$\\
		Vậy hàm số $g(x)=f(1-2 x)+x^2-x$ nghịch biến trên các khoảng $\left(\dfrac{1}{2}; \dfrac{3}{2}\right)$ và $\left(-\infty ;-\dfrac{3}{2}\right)$.\\
		Mà $\left(1 ; \dfrac{3}{2}\right) \subset \left(\dfrac{1}{2}; \dfrac{3}{2}\right)$ nên hàm số $g(x)=f(1-2 x)+x^2-x$ nghịch biến trên khoảng $\left(1 ; \dfrac{3}{2}\right)$.
	}
\end{ex}
\begin{ex}[Chuyên Lê Quý Đôn Điện Biên 2019]%[2D1G1-2]
	Cho hàm số $f(x)$ có bảng xét dấu của đạo hàm như sau
	\begin{center}
		\begin{tikzpicture}
			\tkzTabInit[nocadre,lgt=1.2,espcl=2,deltacl=0.6]
			{$x$ /0.6,$f'(x)$ /0.6}
			{$-\infty$,$0$,$1$,$2$,$3$,$+\infty$}
			\tkzTabLine{,+,$0$,-,$0$,-,$0$,+,$0$,-,}
		\end{tikzpicture}
	\end{center}
	Hàm số $y=f(x-1)+x^3-12 x+2019$ nghịch biến trên khoảng nào dưới đây?
	\choice
	{$(1 ;+\infty)$}
	{\True $(1 ; 2)$}
	{$(-\infty ; 1)$}
	{$(3 ; 4)$}
	\loigiai{
		$y'=f'(x-1)+3 x^2-12=f'(t)+3 t^2+6 t-9=f'(t)-\left(-3 t^2-6 t+9\right)$, với $t=x-1$.\\
		\immini{
			Nghiệm của phương trình $y'=0$ là hoành độ giao điểm của các đồ thị hàm số $y=f'(t)$ và $y=-3 t^2-6 t+9$.\\
			Vẽ đồ thị hàm số $y=f'(t)$ và $y=-3 t^2-6 t+9$ trên cùng một hệ trục tọa độ như hình vẽ bên.
		}	
		{		\begin{tikzpicture}[scale=0.5,>=stealth, font=\footnotesize, line join=round, line cap=round]
				\def\a{-3} \def\b{-6} \def\c{9} % Hệ số
				\def\xmin{-9} \def\xmax{7}
				\def\ymin{-3} \def\ymax{13}
				
				%\draw[color=gray!50,dashed] (\xmin,\ymin) grid (\xmax,\ymax);
				
				\draw[->] (\xmin,0)--(\xmax,0) node [below]{$x$};
				\draw[->] (0,\ymin)--(0,\ymax) node [left]{$y$};
				\node at (0,0) [below left]{$O$};
				\clip (\xmin+0.1,\ymin+0.1) rectangle (\xmax-0.5,\ymax-0.1);
				\draw[smooth,samples=300] plot(\x,{\a*(\x)^2+\b*(\x)+\c});
				\node at (1,0) [above right]{$1$};
				\node at (2,0) [below right]{$2$};
				\node at (3,0) [below right]{$3$};
				\node at (-3,-2) [left]{$y=-3t^2-6t+9$};
				\node at (4,0) [below right]{$f'(x)$};
				\draw (-2.2,10).. controls (-1,1.9) and (-0.5,0.8) .. (0,0);
				%\draw (-2,0).. controls (-1.5,-2) and (-0.5,-0) .. (0,0);
				\draw (0,0).. controls (0.4,-0.6) and (0.6,-0.6) .. (0.8,-0.2);
				\draw (0.8,-0.2).. controls (1,0.25) and (1.1,-0.1) .. (1.4,-0.8);
				\draw (1.4,-0.8).. controls (1.6,-1.1) and (1.7,-0.9) .. (2,0);
				\draw (2,0).. controls (2.4,1.1) and (2.6,1.1) .. (3.5,-1);
			\end{tikzpicture}
		}
		Dựa vào đồ thị trên, ta có bảng xét dấu của hàm số $y'=f'(t)-\left(-3 t^2-6 t+9\right)$ như sau $
		\left(t_0<-1\right)$
		\begin{center}
			\begin{tikzpicture}
				\tkzTabInit[nocadre,lgt=2,espcl=2,deltacl=0.6]
				{$x$ /0.6,$y'$ /0.6}
				{$-\infty$,$t_0$,$1$,$+\infty$}
				\tkzTabLine{,+,$0$,-,$0$,+,}
			\end{tikzpicture}
		\end{center}
		Hàm số nghịch biến trên khoảng $t \in\left(t_0 ; 1\right)$.\\
		Do đó hàm số nghịch biến trên khoảng $x \in(1 ; 2) \subset \left(t_0+1 ; 1\right)$.
	}
\end{ex}


\begin{ex}[Chuyên Phan Bội Châu Nghệ An 2019]%[2D1G1-2]
	Cho hàm số $f(x)$ có bảng xét dấu đạo hàm như sau:
	\begin{center}
		\begin{tikzpicture}
			\tkzTabInit[nocadre,lgt=2,espcl=2,deltacl=0.6]
			{$x$ /0.6,$f'(x)$ /0.6}
			{$-\infty$,$1$,$2$,$3$,$4$,$+\infty$}
			\tkzTabLine{,-,$0$,+,$0$,+,$0$,-,$0$,+,}
		\end{tikzpicture}
	\end{center}
	Hàm số $y=2 f(1-x)+\sqrt{x^2+1}-x$ nghịch biến trên những khoảng nào dưới đây
	\choice
	{$(-\infty ;-2)$}
	{$(-\infty ; 1)$}
	{\True $(-2 ; 0)$}
	{$(-3 ;-2)$}
	\loigiai{
		$y'=-2 f'(1-x)+\dfrac{x}{\sqrt{x^2+1}}-1$. \\
		Có $\dfrac{x}{\sqrt{x^2+1}}-1<0,~ \forall x \in(-2 ; 0)$.\\
		Bảng xét dấu:
		\begin{center}
			\begin{tikzpicture}
				\tkzTabInit[nocadre,lgt=2,espcl=2,deltacl=0.6]
				{$x$ /0.7,$f'(1-x)$ /0.7}
				{$-\infty$,$-3$,$-2$,$-1$,$0$,$+\infty$}
				\tkzTabLine{,+,$0$,-,$0$,+,$0$,+,$0$,-,}
			\end{tikzpicture}
		\end{center}
		$\Rightarrow-2 f'(1-x)<0, ~ \forall x \in(-2 ; 0) \\
		\Rightarrow-2 f'(1-x)+\dfrac{x}{\sqrt{x^2+1}}-1<0, ~\forall x \in(-2 ; 0)$.
	}
\end{ex}
\begin{ex}[Sở Vĩnh Phúc 2019]%[2D1G1-2]
	\immini{
		Cho hàm số bậc bốn $y=f(x)$ có đồ thị của hàm số $y=f'(x)$ như hình vẽ bên.\\
		Hàm số $y=3 f(x)+x^3-6 x^2+9 x$ đồng biến trên khoảng nào trong các khoảng sau đây?
		\choice
		{$(0 ; 2)$}
		{$(-1 ; 1)$}
		{$(1 ;+\infty)$}
		{\True $(-2 ; 0)$}
	}
	{
		\begin{tikzpicture}[scale=0.7,>=stealth, font=\footnotesize, line join=round, line cap=round]
			\def\a{0.21} \def\b{0.88} \def\c{-0.58} \def\d{-3} % Hệ số
			\def\xmin{-5} \def\xmax{5}
			\def\ymin{-4} \def\ymax{3} 
			%\draw[color=gray!50,dashed] (\xmin,\ymin) grid (\xmax,\ymax); 
			\draw[->] (\xmin,0)--(\xmax,0) node [below]{$x$};
			\draw[->] (0,\ymin)--(0,\ymax) node [left]{$y$};
			\node at (0,0) [above left]{$O$};
			\node at (-4,0) [below left]{$-4$};
			\node at (-2,0) [below left]{$-2$};
			\node at (0,-3) [below right]{$-3$};
			\draw[dashed] (2,0) node[above right]{$2$}--(2,1) --(0,1) node[above right]{$1$};
			\clip (\xmin+0.1,\ymin+0.1) rectangle (\xmax-0.5,\ymax-0.1);
			\draw[smooth,samples=300] plot(\x,{\a*(\x)^3+\b*(\x)^2+\c*(\x)+\d});
		\end{tikzpicture}
	}
	
	\loigiai{
		Hàm số $f(x)=a x^4+b x^3+c x^2+d x+e,(a \neq 0)$.
		Có $f'(x)=4 a x^3+3 b x^2+2 c x+d$.\\
		Đồ thị hàm số $y=f'(x)$ đi qua các điểm $(-4 ; 0),(-2 ; 0),(0 ;-3),(2 ; 1)$ nên ta có
		$$\heva{&- 2 5 6 a + 4 8 b - 8 c + d = 0\\
			&- 3 2 a + 1 2 b - 4 c + d = 0\\
			&d = - 3\\
			&3 2 a + 1 2 b + 4 c + d = 1}\Leftrightarrow \heva{&
			a=\dfrac{5}{96}\\
			&b=\dfrac{7}{24}\\
			&c=-\dfrac{7}{24}\\
			&d=-3.}
		$$
		Xét hàm số
		$
		y=3 f(x)+x^3-6 x^2+9 x$\\
		Ta có $ y'=3\left(f'(x)+x^2-4 x+3\right)=3\left(\frac{5}{24}x^3+\frac{15}{8}x^2-\frac{55}{12}x\right)
		$\\
		Ta có $y'=0 \Leftrightarrow\hoac{&x=-11 \\&x=0 \\&x=2.}$ \\
		Xét dấu $y'$, ta được hàm số đã cho đồng biến trên các khoảng $(-11 ; 0)$ và $(2 ;+\infty)$.
	}
\end{ex}
\begin{ex}[Học Mãi 2019]%[2D1K1-2]
	\immini
	{Cho hàm số $y=f(x)$ có đạo hàm trên $\mathbb{R}$. Đồ thị hàm số $y=f'(x)$ như hình bên. Hỏi đồ thị hàm số $y=f(x)-2 x$ có bao nhiêu điểm cực trị?
		\choice
		{$4$}
		{\True $3$}
		{$2$}
		{$1$}
	}
	{
		\begin{tikzpicture}[font=\footnotesize,line join=round, line cap=round,>=stealth,scale=0.8]
			\draw[->] (-3.5,0)--(4,0) node[above] {$x$};
			\draw[->] (0,-3)--(0,4) node[left] {$y$};
			%\fill[black] (-2,0)node[below left]{$-2$} circle (1.2pt) (0,0)node[above right]{$O$} circle (1.2pt) (3,0)node[above]{$3$} circle (1.2pt);
			\draw[dashed] (-2,-2)-- (0,-2) node[right]{$-2$};
			\draw[dashed] (2,0) node[below]{$2$}-- (2,2)--(0,2) node[below left]{$2$};
			\node at (0,0) [below left]{$O$};
			\node at (3,0) [below right]{$3$};
			\draw (-3,2.5).. controls (-2.2,-3) and (-1.8,-3) .. (-1.1,0);
			\draw (-1.1,0).. controls (-0.6,2.5) and (-0.4,2.5) .. (0,2);
			\draw (0,2).. controls (0.7,0.5) and (1.1,0.5) .. (1.5,1.5);
			\draw (1.5,1.5).. controls (2,2.5) and (2.8,2.5) .. (3.5,-2.5);
			%\draw (3,0).. controls (3.3,-0.1) and (3.5,-0.5) .. (3.5,-2);
		\end{tikzpicture}
	}
	\loigiai{
		\immini{
			Đặt $g(x)=f(x)-2 x$.\\
			$\Rightarrow g'(x)=f'(x)-2 .
			$\\
			Vẽ đường thẳng $y=2$.\\
			$\Rightarrow$ phương trình $g'(x)=0$ có $3$ nghiệm bội lẻ.\\
			$\Rightarrow$ đồ thị hàm số $y=f(x)-2 x$ có $3$ điểm cực trị.
		}
		{
			\begin{tikzpicture}[font=\footnotesize,line join=round, line cap=round,>=stealth,scale=0.8]
				\draw[->] (-3.5,0)--(4,0) node[above] {$x$};
				\draw[->] (0,-3)--(0,4) node[left] {$y$};
				%\fill[black] (-2,0)node[below left]{$-2$} circle (1.2pt) (0,0)node[above right]{$O$} circle (1.2pt) (3,0)node[above]{$3$} circle (1.2pt);
				\draw[dashed] (-2,-2)-- (0,-2) node[right]{$-2$};
				\draw[dashed] (2,0) node[below]{$2$}-- (2,2)--(0,2) node[below left]{$2$};
				\node at (3,0) [below left]{$3$};
				\draw (-3,2.5).. controls (-2.2,-3) and (-1.8,-3) .. (-1.1,0);
				\draw (-1.1,0).. controls (-0.6,2.5) and (-0.4,2.5) .. (0,2);
				\draw (0,2).. controls (0.7,0.5) and (1.1,0.5) .. (1.5,1.5);
				\draw (1.5,1.5).. controls (2,2.5) and (2.8,2.5) .. (3.5,-2.5);
				\draw (-3.5,2)--(4,2) node[above]{$y=2$};
			\end{tikzpicture}
		}
	}
\end{ex}
\begin{ex}[THPT Hoàng Hoa Thám Hưng Yên 2019]%[2D1G1-2]
	\immini{
		Cho hàm số $y=f(x)$ liên tục trên $\mathbb{R}$. Hàm số $y=f'(x)$ có đồ thị như hình vẽ. 
		Hàm số $g(x)=f(x-1)+\dfrac{2019-2018 x}{2018}$ đồng biến trên khoảng nào dưới đây?
		\choice
		{$(2 ; 3)$}
		{$(0 ; 1)$}
		{\True $(-1 ; 0)$}
		{$(1 ; 2)$}
	}
	{
		\begin{tikzpicture}[scale=1, font=\footnotesize, line join=round, line cap=round, >=stealth]
			\tikzset{label style/.style={font=\footnotesize}}
			\draw[->] (-2,0)--(3,0) node[below left] {$x$};
			\draw[->] (0,-2)--(0,3) node[below left] {$y$};
			\draw[fill=black] (0,0) node [above left] {$O$} circle(1pt);
			\fill (1,1) circle(1pt) (-1,1) circle(1pt) (2,1) circle(1pt);
			\foreach \x in {1,2}
			\draw[thin] (\x,1pt)--(\x,-1pt) node [below] {\footnotesize$\x$};
			\foreach \x in {-1}
			\draw[thin] (\x,1pt)--(\x,-1pt) node [below left] {\footnotesize$\x$};
			\foreach \y in {-1}
			\draw[thin] (1pt,\y)--(-1pt,\y) node [right] {\footnotesize$\y$};
			\foreach \y in {1}
			\draw[thin] (1pt,\y)--(-1pt,\y) node [above left] {\footnotesize$\y$};
			\draw[dashed](-1,0)--(-1,1)--(2,1) (1,1)--(1,0) (2,1)--(2,0);
			\begin{scope}
				\clip (-3,-3) rectangle (3,3);
				\draw[name path=(C)] plot[smooth,tension=0.7] coordinates{(-1.15,3)(-0.5,-1.6)(.8,.88)(1.9,0.8)(2.3,3)};
			\end{scope}
		\end{tikzpicture}
	}	\loigiai{
		Ta có $g'(x)=f'(x-1)-1$.\\
		$
		g'(x) \geq 0 \Leftrightarrow f'(x-1)-1 \geq 0 \Leftrightarrow f'(x-1) \geq 1 \Leftrightarrow \hoac{&x - 1 \leq - 1\\
			&x - 1 \geq 2}\Leftrightarrow \hoac{&
			x \leq 0 \\
			&x \geq 3.}
		$\\
		Từ đó suy ra hàm số $g(x)=f(x-1)+\dfrac{2019-2018 x}{2018}$ đồng biến trên khoảng $(-1 ; 0)$.
	}
\end{ex}

\begin{ex}[(Sở Ninh Bình 2019]%[2D1K1-2]
	Cho hàm số $y=f(x)$ có bảng xét dấu của đạo hàm như sau
	\begin{center}
		\begin{tikzpicture}
			\tkzTabInit[nocadre,lgt=1,espcl=2,deltacl=0.6]
			{$x$ /0.7,$f'(x)$ /0.7}
			{$-\infty$,$-2$,$-1$,$2$,$4$,$+\infty$}
			\tkzTabLine{,+,$0$,-,$0$,+,$0$,-,$0$,+,}
		\end{tikzpicture}
	\end{center}
	Hàm số $y=-2 f(x)+2019$ nghịch biến trên khoảng nào trong các khoảng dưới đây?
	\choice
	{$(-4 ; 2)$}
	{\True $(-1 ; 2)$}
	{$(-2 ;-1)$}
	{$(2 ; 4)$}
	\loigiai{
		Xét $y=g(x)=-2 f(x)+2019$.\\
		Ta có $g'(x)=(-2 f(x)+2019)'=-2 f'(x), g'(x)=0 \Leftrightarrow\hoac{&x=-2 \\&x=-1 \\&x=2 \\&x=4.}$.\\
		Ta có bảng xét dấu của $g'(x)$
		\begin{center}
			\begin{tikzpicture}
				\tkzTabInit[nocadre,lgt=1,espcl=2,deltacl=0.6]
				{$x$ /0.6,$f'(x)$ /0.6}
				{$-\infty$,$-2$,$-1$,$2$,$4$,$+\infty$}
				\tkzTabLine{,-,$0$,+,$0$,-,$0$,+,$0$,+,}
			\end{tikzpicture}
		\end{center}
		Dựa vào bảng xét dấu, ta thấy hàm số $y=g(x)$ nghịch biến trên khoảng $(-1 ; 2)$.
	}
\end{ex}
\begin{ex}[THPT Lương Thế Vinh Hà Nội 2019]%[2D1G1-2]
	\immini{
		Cho hàm số $y=f(x)$. Biết đồ thị hàm số $y=f'(x)$ có đồ thị như hình vẽ bên. 
		Hàm số $y=f \left(3-x^2\right)+2018$ đồng biến trên khoảng nào dưới đây?
		\choice
		{\True $(-1 ; 0)$}
		{$(2 ; 3)$}
		{$(-2 ;-1)$}
		{$(0 ; 1)$}
	}
	{
		\begin{tikzpicture}[scale=0.6,>=stealth, font=\footnotesize, line join=round, line cap=round]
			\def\a{0.065} \def\b{0.32} \def\c{-0.53} \def\d{-0.82} % Hệ số
			\def\xmin{-8} \def\xmax{4}
			\def\ymin{-3} \def\ymax{3} 
			%\draw[color=gray!50,dashed] (\xmin,\ymin) grid (\xmax,\ymax); 
			\draw[->] (\xmin,0)--(\xmax,0) node [below]{$x$};
			\draw[->] (0,\ymin)--(0,\ymax) node [left]{$y$};
			\node at (0,0) [below left]{$O$};
			\node at (-6,0) [below left]{$-6$};
			\node at (-1,0) [below left]{$-1$};
			\node at (2,0) [below right]{$2$};
			\clip (\xmin+0.1,\ymin+0.1) rectangle (\xmax-0.5,\ymax-0.1);
			\draw[smooth,samples=300][domain=-6.5:3.5] plot(\x,{\a*(\x)^3+\b*(\x)^2+\c*(\x)+\d});
		\end{tikzpicture}
	}
	
	\loigiai{
		Ta có $\left[f\left( 3-x^2\right)+2018 \right]'=-2 x \cdot f'\left(3-x^2\right) $.\\
		$
		-2 x \cdot f'\left(3-x^2\right)=0 \Leftrightarrow\hoac{&
			x = 0\\
			&3 - x ^{2}= - 6\\
			&3 - x ^{2}= - 1\\
			&3 - x ^{2}= 2}
		\Leftrightarrow \hoac{
			&x=0 \\
			&x=\pm 3 \\
			&x=\pm 2 \\
			&	x=\pm 1.}
		$\\
		Bảng xét dấu của đạo hàm hàm số đã cho
		\begin{center}
			\begin{center}
				\begin{tikzpicture}
					\tkzTabInit[nocadre,lgt=2.9,espcl=1.5,deltacl=0.6]
					{$x$ /0.7,$f'\left( 3-x^2\right) $/0.7,$-2xf'\left( 3-x^2\right)$/0.8}
					{$-\infty$,$-3$,$-2$,$-1$,$0$,$1$,$2$,$3$,$+\infty$}
					\tkzTabLine{,-,$0$,+,$0$,-,$0$,+,$0$,+,$0$,-,$0$,+,$0$,-}
					\tkzTabLine{,-,$0$,+,$0$,-,$0$,+,$0$,-,$0$,+,$0$,-,$0$,+}
				\end{tikzpicture}
			\end{center}
		\end{center}
		Từ bảng xét dấu suy ra hàm số đồng biến trên $(-1 ; 0)$.
	}
\end{ex}
\begin{ex}[Chuyên Biên Hòa - Hà Nam - 2020]%[2D1G1-2]
	\immini{
		Cho hàm số đa thức $f(x)$ có đạo hàm trên $\mathbb{R}$. Biết $f(0)=0$ và đồ thị hàm số $y=f'(x)$ như hình sau.
		Hàm số $g(x)=\left|4 f(x)+x^2\right|$ đồng biến trên khoảng nào dưới đây?
		\choice
		{$(4 ;+\infty)$}
		{\True $(0 ; 4)$}
		{$(-\infty ;-2)$}
		{$(-2 ; 0)$}
	}	
	{
		\begin{tikzpicture}[scale=0.7,>=stealth, font=\footnotesize, line join=round, line cap=round]
			%\def\a{1} \def\b{-6} \def\c{9} \def\d{1} % Hệ số
			\def\xmin{-4} \def\xmax{6}
			\def\ymin{-3} \def\ymax{2} 
			%\draw[color=gray!50,dashed] (\xmin,\ymin) grid (\xmax,\ymax); 
			\draw[->] (\xmin,0)--(\xmax,0) node [below]{$x$};
			\draw[->] (0,\ymin)--(0,\ymax) node [left]{$y$};
			\node at (0,0) [below left]{$O$};
			%\node at (1,3) [below left]{$f'(x)$};
			%\node at (-1.3,4) {$f'(x)$};
			\draw[dashed] (-2,0) node[below]{$-2$}--(-2,1)--(0,1) node[below left]{$1$};
			\draw[dashed] (4,0) node[below]{$4$}--(4,-2)--(0,-2) node[below left]{$-2$};
			%\draw[dashed] (1,0) node[below]{$1$}--(1,1);
			%\draw[dashed] (-0.5,0) node[below left]{$-0{,}5$}--(-0.5,2.125);
			\clip (\xmin+0.1,\ymin+0.1) rectangle (\xmax-0.5,\ymax-0.1);
			\draw[smooth,samples=300][domain=-4:5.5] plot(\x,{0.071*(\x)^3-0.142*(\x)^2-1.07*(\x)});
		\end{tikzpicture}
	}
	\loigiai{
		\immini{
			Xét hàm số $h(x)=4 f(x)+x^2$ trên $\mathbb{R}$.\\
			Vì $f(x)$ là hàm số đa thức nên $h(x)$ cũng là hàm số đa thức và $h(0)=4 f(0)=0$.\\
			Ta có $h'(x)=4 f'(x)+2 x$. Do đó $h'(x)=0 \Leftrightarrow f'(x)=-\dfrac{1}{2}x$.\\
		}
		{
			\begin{tikzpicture}[scale=0.7,>=stealth, font=\footnotesize, line join=round, line cap=round]
				%\def\a{1} \def\b{-6} \def\c{9} \def\d{1} % Hệ số
				\def\xmin{-4} \def\xmax{6}
				\def\ymin{-3} \def\ymax{2} 
				%\draw[color=gray!50,dashed] (\xmin,\ymin) grid (\xmax,\ymax); 
				\draw[->] (\xmin,0)--(\xmax,0) node [below]{$x$};
				\draw[->] (0,\ymin)--(0,\ymax) node [left]{$y$};
				\node at (0,0) [below left]{$O$};
				%\node at (1,3) [below left]{$f'(x)$};
				%\node at (-1.3,4) {$f'(x)$};
				\draw[dashed] (-2,0) node[below]{$-2$}--(-2,1)--(0,1) node[below left]{$1$};
				\draw[dashed] (4,0) node[below]{$4$}--(4,-2)--(0,-2) node[below left]{$-2$};
				%\draw[dashed] (1,0) node[below]{$1$}--(1,1);
				%\draw[dashed] (-0.5,0) node[below left]{$-0{,}5$}--(-0.5,2.125);
				\clip (\xmin+0.1,\ymin+0.1) rectangle (\xmax-0.5,\ymax-0.1);
				\draw[smooth,samples=300][domain=-4:5.5] plot(\x,{0.071*(\x)^3-0.142*(\x)^2-1.07*(\x)});
				\draw[smooth,samples=300][domain=-4:5.5] plot(\x,{-0.5*(\x)});
			\end{tikzpicture}
		}
		Dựa vào sự tương giao của đồ thị hàm số $y=f'(x)$ và đường thẳng $y=-\dfrac{1}{2}x$, ta có
		$
		h'(x)=0 \Leftrightarrow x \in\{-2 ; 0 ; 4\}.\\
		$
		Bảng biến thiên của hàm số $h(x)$ như sau:
		\begin{center}
			\begin{tikzpicture}
				\tkzTabInit[nocadre,lgt=1.2,espcl=2.5,deltacl=0.6]
				{$x$ /0.6,$y'$ /0.6,$y$ /2}
				{$-\infty$,$-2$,$0$,$4$,$+\infty$}
				\tkzTabLine{,-,$0$,+,$0$,-,$0$,+,}
				\tkzTabVar{+/$+\infty$, -/$y_1$,+/$0$,-/$y_3$,+/$+\infty$}
			\end{tikzpicture}
		\end{center}
		Từ đó suy ra bảng biến thiên của hàm số $g(x)=|h(x)|$.\\
		Dựa vào bảng biến thiên trên, ta thấy hàm số $g(x)$ đồng biến trên khoảng $(0 ; 4)$.
	}
\end{ex}
\begin{ex}[Chuyên Thái Bình - 2020]%[2D1G1-2]
	\immini{
		Cho hàm số $f(x)$ liên tục trên $\mathbb{R}$ có đồ thị hàm số $y=f'(x)$ cho như hình vẽ bên.\\
		Hàm số $g(x)=2 f(|x-1|)-x^2+2 x+2020$ đồng biến trên khoảng nào?
		\choice
		{\True $(0 ; 1)$}
		{$(-3 ; 1)$}
		{$(1 ; 3)$}
		{$(-2 ; 0)$}
	}
	{
		\begin{tikzpicture}[scale=0.7,>=stealth, font=\footnotesize, line join=round, line cap=round]
			%\def\a{1} \def\b{-6} \def\c{9} \def\d{1} % Hệ số
			\def\xmin{-4} \def\xmax{5}
			\def\ymin{-3} \def\ymax{5} 
			%\draw[color=gray!50,dashed] (\xmin,\ymin) grid (\xmax,\ymax); 
			\draw[->] (\xmin,0)--(\xmax,0) node [below]{$x$};
			\draw[->] (0,\ymin)--(0,\ymax) node [left]{$y$};
			\node at (0,0) [below left]{$O$};
			%\node at (1,3) [below left]{$f'(x)$};
			\node at (-1.3,4) {$f'(x)$};
			\draw[dashed] (-1,0) node[above]{$-1$}--(-1,-1)--(0,-1) node[below left]{$-1$};
			\draw[dashed] (1,0) node[below]{$1$}--(1,1)--(0,1) node[below left]{$1$};
			\draw[dashed] (3,0) node[below]{$3$}--(3,3)--(0,3) node[below left]{$3$};
			%\draw[dashed] (1,0) node[below]{$1$}--(1,1);
			%\draw[dashed] (-0.5,0) node[below left]{$-0{,}5$}--(-0.5,2.125);
			\clip (\xmin+0.1,\ymin+0.1) rectangle (\xmax-0.5,\ymax-0.1);
			\draw[smooth,samples=300][domain=-2:4] plot(\x,{-0.5*(\x)^3+1.5*(\x)^2+1.5*(\x)-1.5});
			%\draw[smooth,samples=300] plot(\x,{(\x)^3+(\x)^2-2*(\x)+1});
		\end{tikzpicture}
	}
	\loigiai{
		Ta có đường thẳng $y=x$ cắt đồ thị hàm số $y=f'(x)$ tại các điểm $x=-1 ; x=1 ; x=3$ như hình vẽ sau:
		\begin{center}
			\begin{tikzpicture}[scale=0.7,>=stealth, font=\footnotesize, line join=round, line cap=round]
				%\def\a{1} \def\b{-6} \def\c{9} \def\d{1} % Hệ số
				\def\xmin{-4} \def\xmax{5}
				\def\ymin{-3} \def\ymax{5} 
				%\draw[color=gray!50,dashed] (\xmin,\ymin) grid (\xmax,\ymax); 
				\draw[->] (\xmin,0)--(\xmax,0) node [below]{$x$};
				\draw[->] (0,\ymin)--(0,\ymax) node [left]{$y$};
				\node at (0,0) [below left]{$O$};
				%\node at (1,3) [below left]{$f'(x)$};
				\node at (-1.3,4) {$f'(x)$};
				\node at (4,3.2) {$y=x$};
				\draw[dashed] (-1,0) node[above]{$-1$}--(-1,-1)--(0,-1) node[below left]{$-1$};
				\draw[dashed] (1,0) node[below]{$1$}--(1,1)--(0,1) node[below left]{$1$};
				\draw[dashed] (3,0) node[below]{$3$}--(3,3)--(0,3) node[below left]{$3$};
				%\draw[dashed] (1,0) node[below]{$1$}--(1,1);
				%\draw[dashed] (-0.5,0) node[below left]{$-0{,}5$}--(-0.5,2.125);
				\clip (\xmin+0.1,\ymin+0.1) rectangle (\xmax-0.5,\ymax-0.1);
				\draw[smooth,samples=300][domain=-2:4] plot(\x,{-0.5*(\x)^3+1.5*(\x)^2+1.5*(\x)-1.5});
				\draw[smooth,samples=300] plot(\x,{(\x)});
			\end{tikzpicture}
		\end{center}
		Dựa vào đồ thị của hai hàm số trên ta có $f'(x)>x \Leftrightarrow\hoac{&x<-1 \\ &1<x<3}$ và
		$ f'(x)<x \Leftrightarrow\hoac{&
			-1<x<1 \\
			&x>3.}$\\
		+Trường hợp 1: $x-1<0 \Leftrightarrow x<1$.\\
		Khi đó $g(x)=2 f(1-x)-x^2+2 x+2020$.\\
		Ta có $g'(x)=-2 f'(1-x)+2(1-x)$.
		$$
		g'(x)>0 \Leftrightarrow-2 f'(1-x)+2(1-x)>0 \Leftrightarrow f'(1-x)<1-x \Leftrightarrow\hoac{
			&- 1 < 1 - x < 1\\
			&1 - x > 3} \Leftrightarrow \hoac{&
			0<x<2 \\
			&x<-2.}
		$$
		Kết hợp điều kiện, ta có $g'(x)>0 \Leftrightarrow\hoac{&0<x<1 \\ &x<-2.}$\\
		
		+ Trường hợp 2: $x-1>0 \Leftrightarrow x>1$.\\
		Khi đó ta có $g(x)=2 f(x-1)-x^2+2 x+2020$.\\
		$ g'(x)=2 f'(x-1)-2(x-1)$\\
		$g'(x)>0 \Leftrightarrow 2 f'(x-1)-2(x-1)>0 \Leftrightarrow f'(x-1)>x-1 \Leftrightarrow\hoac{&
			x - 1 < - 1\\
			&1 < x - 1 < 3}\Leftrightarrow \hoac{
			&x<0 \\
			&2<x<4.}$
		Kết hợp điều kiện ta có $g'(x)>0 \Leftrightarrow 2<x<4$.\\
		Vậy hàm số $g(x)=2 f(|x-1|)-x^2+2 x+2020$ đồng biến trên khoảng $(0 ; 1)$.
	}
\end{ex}

\begin{ex}[Chuyên Lào Cai - 2020]%[2D1G1-2]
	\immini{
		Cho hàm số $f'(x)$ có đồ thị như hình bên.\\
		Hàm số $g(x)=f(3 x+1)+9 x^3+\dfrac{9}{2}x^2$ đồng biến trên khoảng nào dưới đây?
		\choice
		{$(-1 ; 1)$}
		{$(-2 ; 0)$}
		{$(-\infty ; 0)$}
		{\True $(1 ;+\infty)$}
	}
	{\begin{tikzpicture}[line join=round, line cap=round,>=stealth,thick,scale=.8]
			\tikzset{label style/.style={font=\footnotesize}}
			\draw[->] (-2.1,0)--(5.1,0) node[below left] {$x$};
			\draw[->] (0,-3.1)--(0,4.1) node[below left] {$y$};
			\draw (0,0) node [below left] {$O$};
			\foreach \x in {1,2,3}
			\draw[thin] (\x,1pt)--(\x,-1pt) node [below] {$\x$};
			\draw[thin](-1,1pt)--(1,-1pt)node[above left]{$-1$};
			\foreach \y in {-2,2}
			\draw[thin] (1pt,\y)--(-1pt,\y) node [above right] {$\y$};
			%\begin{scope}
			\clip (-2,-3) rectangle (5,4);
			\draw[samples=200,domain=-2:4,smooth,variable=\x] plot (\x,{(\x)^3-3*(\x)^2+2});
			%\end{scope}
			\draw[dashed] (-1,0)--(-1,-2)--(0,-2);
			\draw[dashed] (3,0)--(3,2)--(0,2);
			%\begin{scope}[on background layer]\path[white]node{MDD-134};\end{scope}
		\end{tikzpicture}
	}
	\loigiai
	{
		\immini{Xét hàm số $g(x)=f(3 x+1)+9 x^3+\dfrac{9}{2}x^2 \\
			\Rightarrow g'(x)=3 f'(3 x+1)+27 x^2+9 x$.\\
			Hàm số đồng biến  $\Leftrightarrow g'(x)>0 \Leftrightarrow 3 f'(3 x+1)+27 x^2+9 x>0$
			\\
			$
			\Leftrightarrow f'(3 x+1)+3 x(3 x+1)>0 \qquad (*)
			$\\
			Đặt $t=3 x+1$, khi đó  $(*) \Leftrightarrow f'(t)+(t-1) t>0$\\ $\Leftrightarrow f'(t)>-t^2+t$.\\
			Vẽ parabol $y=-x^2+x$ và đồ thị hàm số $f'(x)$ trên cùng một hệ trục
		}
		{
			\begin{tikzpicture}[line join=round, line cap=round,>=stealth,thick,scale=.8]
				\tikzset{label style/.style={font=\footnotesize}}
				\draw[->] (-2.1,0)--(5.1,0) node[below left] {$x$};
				\draw[->] (0,-3.1)--(0,4.1) node[below left] {$y$};
				\draw (0,0) node [below left] {$O$};
				\foreach \x in {1,2,3}
				\draw[thin] (\x,1pt)--(\x,-1pt) node [below] {$\x$};
				\draw[thin](-1,1pt)--(1,-1pt);
				\foreach \y in {-2,2}
				\draw[thin] (1pt,\y)--(-1pt,\y) node [above right] {$\y$};
				%\begin{scope}
				\clip (-2,-3) rectangle (5,4);
				\draw[samples=200,domain=-2:4,smooth,variable=\x] plot (\x,{(\x)^3-3*(\x)^2+2});
				\draw[samples=200,domain=-2:4,smooth,variable=\x] plot (\x,{-(\x)^2+(\x)});
				%\end{scope}
				\draw[dashed] (-1,0) node[above left]{$-1$}--(-1,-2)--(0,-2);
				\draw[dashed] (3,0)--(3,2)--(0,2);
				%\begin{scope}[on background layer]\path[white]node{MDD-134};\end{scope}
			\end{tikzpicture}
		}
		Dựa vào đồ thị ta thấy
		$
		f'(t)>-t^2+t \Leftrightarrow\hoac{&- 1 < t < 1\\
			&t > 2}\Rightarrow \hoac{&
			- 1 < 3 x + 1 < 1\\
			&3 x + 1 > 2} \Leftrightarrow \hoac{&
			\dfrac{-2}{3}<x<0\\
			&x>\dfrac{1}{3}.}
		$}
\end{ex}
\begin{ex}[Sở Phú Thọ-2020]%[2D1G1-2]
	\immini{
		Cho hàm số $y=f(x)$ có đồ thị hàm số $y=f'(x)$ như hình vẽ.\\
		Hàm số $g(x)=f\left(\mathrm{e}^x-2\right)-2020$ nghịch biến trên khoảng nào dưới đây?
		\choice
		{\True $\left(-1 ; \dfrac{3}{2}\right)$}
		{$(-1 ; 2)$}
		{$(0 ;+\infty)$}
		{$\left(\dfrac{3}{2}; 2\right)$}
	}
	{
		\begin{tikzpicture}[scale=0.7,>=stealth, font=\footnotesize, line join=round, line cap=round]
			\def\a{1} \def\b{-3} \def\c{0} \def\d{0} % Hệ số
			\def\xmin{-2} \def\xmax{4}
			\def\ymin{-5} \def\ymax{2} 
			%\draw[color=gray!50,dashed] (\xmin,\ymin) grid (\xmax,\ymax); 
			\draw[->] (\xmin,0)--(\xmax,0) node [below]{$x$};
			\draw[->] (0,\ymin)--(0,\ymax) node [left]{$y$};
			\node at (0,0) [above left]{$O$};
			\node at (3,0) [below right]{$3$};
			\draw[dashed] (2,0) node[above]{$2$}--(2,-4) --(0,-4) node[left]{$-4$};
			\clip (\xmin+0.1,\ymin+0.1) rectangle (\xmax-0.5,\ymax-0.1);
			\draw[smooth,samples=300] plot(\x,{\a*(\x)^3+\b*(\x)^2+\c*(\x)+\d});
		\end{tikzpicture}
	}
	
	\loigiai{
		Dựa vào đồ thị hàm số $y=f'(x)$ suy ra $f'(x) \leq 0 ~ \forall x<3$ và $f'(x)>0 ~ \forall x>3$.
		$
		g'(x)=\mathrm{e}^x f'\left(\mathrm{e}^x-2\right) .
		$
		Hàm số $g(x)=f\left(\mathrm{e}^x-2\right)-2020$ nghịch biến \\ $ \Leftrightarrow g'(x)<0 \Leftrightarrow \mathrm{e}^x f'\left(\mathrm{e}^x-2\right)<0$\\
		$
		\Leftrightarrow f'\left(\mathrm{e}^x-2\right)<0 \Leftrightarrow \mathrm{e}^x-2<3 \Leftrightarrow \mathrm{e}^x<5 \Leftrightarrow x<\ln 5 .
		$\\
		Vậy hàm số đã cho nghịch biến trên $\left(-1 ; \dfrac{3}{2}\right)$.
	}
\end{ex}
\begin{ex}[Lý Nhân Tông - Bắc Ninh - 2020]%[2D1G1-2]
	\immini{
		Cho hàm số $f(x)$ có đồ thị hàm số $f'(x)$ như hình vẽ.\\
		Hàm số $y=f(\cos x)+x^2-x$ đồng biến trên khoảng
		\choice
		{$(-2 ; 1)$}
		{$(0 ; 1)$}
		{\True $(1 ; 2)$}
		{$(-1 ; 0)$}
	}
	{
		\begin{tikzpicture}[scale=1,>=stealth, font=\footnotesize, line join=round, line cap=round]
			\def\a{-0.5} \def\b{0} \def\c{1.5} \def\d{0} % Hệ số
			\def\xmin{-3} \def\xmax{4}
			\def\ymin{-2} \def\ymax{2} 
			%\draw[color=gray!50,dashed] (\xmin,\ymin) grid (\xmax,\ymax); 
			\draw[->] (\xmin,0)--(\xmax,0) node [below]{$x$};
			\draw[->] (0,\ymin)--(0,\ymax) node [left]{$y$};
			\node at (0,0) [above left]{$O$};
			\node at (3,0) [below right]{$3$};
			\draw[dashed] (-2,0) node[below]{$-2$}--(-2,1) --(0,1) node[above right]{$1$} --(1,1)--(1,0) node[below]{$1$};
			\draw[dashed] (-1,0) node[below right]{$-1$}--(-1,-1) --(0,-1) node[above right]{$-1$} --(2,-1)--(2,0) node[below right]{$2$};
			\clip (\xmin+0.1,\ymin+0.1) rectangle (\xmax-0.5,\ymax-0.1);
			\draw[smooth,samples=300][domain=-2:2] plot(\x,{\a*(\x)^3+\b*(\x)^2+\c*(\x)+\d});
		\end{tikzpicture}
	}
	\loigiai{
		Đặt  $g(x)=f(\cos x)+x^2-x$.\\
		Ta có $g'(x)=-\sin x \cdot f'(\cos x)+2 x-1$\\
		Vì $\cos x \in[-1 ; 1]$ nên từ đồ thị $f'(x)$ ta suy ra $f'(\cos x) \in[-1 ; 1]$.\\
		Do đó $\left|-\sin x \cdot f'(\cos x)\right| \leq 1, ~\forall x \in \mathbb{R}$.\\
		Ta suy ra $g'(x)=\sin x \cdot f'(\cos x)+2 x-1 \geq-1+2 x-1=2 x-2$
		$\Rightarrow g'(x)>0, ~\forall x>1$.\\
		Vậy hàm số đồng biến trên $(1 ; 2)$.
	}
\end{ex}
\begin{ex}[THPT Nguyễn Viết Xuân - 2020]%[2D1G1-2]
	\immini{
		Cho hàm số $f(x)$. Hàm số $y=f'(x)$ có đồ thị như hình vẽ.\\
		Hàm số $g(x)=f\left(3 x^2-1\right)-\dfrac{9}{2}x^4+3 x^2$ đồng biến trên khoảng nào dưới đây?
		\choice
		{\True $\left(-\dfrac{2 \sqrt{3}}{3}; \dfrac{-\sqrt{3}}{3}\right)$}
		{$\left(0 ; \dfrac{2 \sqrt{3}}{3}\right)$}
		{$(1 ; 2)$}
		{$\left(-\dfrac{\sqrt{3}}{3}; \dfrac{\sqrt{3}}{3}\right)$} 
	}
	{
		\begin{tikzpicture}[scale=0.6,>=stealth, font=\footnotesize, line join=round, line cap=round]
			\def\a{0.25} \def\b{0.25} \def\c{-2} \def\d{0} % Hệ số
			\def\xmin{-5} \def\xmax{4}
			\def\ymin{-5} \def\ymax{5} 
			%\draw[color=gray!50,dashed] (\xmin,\ymin) grid (\xmax,\ymax); 
			\draw[->] (\xmin,0)--(\xmax,0) node [below]{$x$};
			\draw[->] (0,\ymin)--(0,\ymax) node [left]{$y$};
			\node at (0,0) [above left]{$O$};
			%\node at (3,0) [below right]{$3$};
			\draw[dashed] (-4,0) node[below left]{$-4$}--(-4,-4) --(0,-4) node[above right]{$-4$};
			\draw[dashed] (3,0) node[below right]{$3$}--(3,3) --(0,3) node[above right]{$3$};
			\clip (\xmin+0.1,\ymin+0.1) rectangle (\xmax-0.5,\ymax-0.1);
			\draw[smooth,samples=300] plot(\x,{\a*(\x)^3+\b*(\x)^2+\c*(\x)+\d});
		\end{tikzpicture}
	}
	
	\loigiai
	{
		TXĐ: $\mathscr{D}=\mathbb{R}$.\\
		Ta có $g'(x)=6 x f'\left(3 x^2-1\right)-18 x^3+6 x=6 x\left[f'\left(3 x^2-1\right)-3 x^2+1\right]$.\\
		$
		g'(x)=0 \Leftrightarrow\hoac{
			&x = 0\\
			&f '( 3 x ^{2}- 1 ) = 3 x ^{2}- 1}
		\Leftrightarrow \hoac{
			&x = 0\\
			&3 x ^{2}- 1 = - 4 \text{~(vô nghiệm)}\\
			&3 x ^{2}- 1 = 0\\
			&3 x ^{2}- 1 = 3}\Leftrightarrow \hoac{&x=0 \\
			&x=\pm \dfrac{\sqrt{3}}{3}\\
			&x=\pm \dfrac{2 \sqrt{3}}{3}.}
		$\\
		Bảng xét dấu
		\begin{center}
			\begin{tikzpicture}
				\tkzTabInit[nocadre,lgt=1.2,espcl=2.2,deltacl=0.6]
				{$x$ /1.2,$f'(x)$ /0.7}
				{$-\infty$,$-\dfrac{2 \sqrt{3}}{3}$,$-\dfrac{ \sqrt{3}}{3}$,$0$,$\dfrac{\sqrt{3}}{3}$,$\dfrac{2 \sqrt{3}}{3}$,$+\infty$}
				\tkzTabLine{,-,$0$,+,$0$,-,$0$,+,$0$,-,$0$,+,}
			\end{tikzpicture}
		\end{center}
		Vậy hàm số đồng biến trong khoảng $\left(-\dfrac{2 \sqrt{3}}{3}; \dfrac{-\sqrt{3}}{3}\right)$.}
\end{ex}
\begin{ex}[Trần Phú - Quảng Ninh - 2020]%[2D1G1-2]
	Cho hàm số $f(x)$ có bảng xét dấu của đạo hàm như sau
	\begin{center}
		\begin{tikzpicture}
			\tkzTabInit[nocadre,lgt=1.2,espcl=2,deltacl=0.6]
			{$x$ /0.6,$f'(x)$ /0.6}
			{$-\infty$,$-4$,$-1$,$2$,$7$,$+\infty$}
			\tkzTabLine{,+,$0$,-,$0$,+,$0$,-,$0$,+,}
		\end{tikzpicture}
	\end{center}
	Hàm số $y=f(2 x+1)+\dfrac{2}{3}x^3-8 x+5$ nghịch biến trên khoảng nào dưới đây?
	\choice
	{$(-\infty ;-2)$}
	{$(1 ;+\infty)$}
	{$(-1 ; 7)$}
	{\True $\left(-1 ; \dfrac{1}{2}\right)$}
	\loigiai{
		Ta có $y'=2 f'(2 x+1)+2 x^2-8$.\\
		Xét $y'\leq 0 \Leftrightarrow 2 f'(2 x+1)+2 x^2-8 \leq 0 \Leftrightarrow f'(2 x+1) \leq 4-x^2$.\\
		Đặt $t=2x+1$, ta có $f'(t) \leq \dfrac{-t^2+2 t+15}{4}$.\\
		Vì $\dfrac{-t^2+2 t+15}{4}\geq 0, \forall t \in[-3 ; 5]$.\\
		Mà $f'(t) \leq 0, \forall t \in[-3 ; 2]$.\\
		Nên $f'(t) \leq \dfrac{-t^2+2 t+15}{4}\Rightarrow t \in[-3 ; 2]$.\\
		Suy ra $-3 \leq 2 x+1 \leq 2 \Leftrightarrow-2 \leq x \leq \dfrac{1}{2}$.}
\end{ex}

\begin{ex}[Chuyên Thái Bình - Lần 3 - 2020]%[2D1G1-2]
	\immini{
		Cho hàm số $y=f(x)$ liên tục trên $\mathbb{R}$ có đồ thị hàm số $y=f'(x)$ cho như hình vẽ.\\
		Hàm số $g(x)=2 f(|x-1|)-x^2+2 x+2020$ đồng biến trên khoảng nào?
		\choice
		{\True $(0 ; 1)$}
		{$(-3 ; 1)$}
		{$(1 ; 3)$}
		{$(-2 ; 0)$}
	}
	{
		\begin{tikzpicture}[scale=0.7,>=stealth, font=\footnotesize, line join=round, line cap=round]
			\def\a{-0.333} \def\b{1} \def\c{1.333} \def\d{-1} % Hệ số
			\def\xmin{-3} \def\xmax{5}
			\def\ymin{-3} \def\ymax{5} 
			%\draw[color=gray!50,dashed] (\xmin,\ymin) grid (\xmax,\ymax); 
			\draw[->] (\xmin,0)--(\xmax,0) node [below]{$x$};
			\draw[->] (0,\ymin)--(0,\ymax) node [left]{$y$};
			\node at (0,0) [above left]{$O$};
			%\node at (3,0) [below right]{$3$};
			\draw[dashed] (-1,0) node[above]{$-1$}--(-1,-1) --(0,-1) node[above right]{$-1$};
			\draw[dashed] (1,0) node[below right]{$1$}--(1,1) --(0,1) node[above right]{$1$};
			\draw[dashed] (3,0) node[below right]{$3$}--(3,3) --(0,3) node[above right]{$3$};
			\clip (\xmin+0.1,\ymin+0.1) rectangle (\xmax-0.5,\ymax-0.1);
			\draw[smooth,samples=300] plot(\x,{\a*(\x)^3+\b*(\x)^2+\c*(\x)+\d});
			\draw[smooth,samples=300] plot(\x,{(\x)});
		\end{tikzpicture}
	}
	\loigiai{
		Với $x>1$, ta có $g(x)=2 f(x-1)-(x-1)^2+2021 \Rightarrow g'(x)=2 f'(x-1)-2(x-1)$.\\
		Hàm số đồng biến $\Leftrightarrow 2 f'(x-1)-2(x-1)>0 \Leftrightarrow f'(x-1)>x-1 \quad(*)$.\\
		Đặt $t=x-1$, khi đó $(*) \Leftrightarrow f'(t)>t \Leftrightarrow\hoac{&1<t<3 \\ &t<-1}\Rightarrow\hoac{&2<x<4 \\ &x<0 ~(\text{loại}).}$\\
		Với $x<1$, ta có $g(x)=2 f(1-x)-(1-x)^2+2021 \Rightarrow g'(x)=-2 f'(1-x)+2(1-x)$.\\
		Hàm số đồng biến $\Leftrightarrow-2 f'(1-x)+2(1-x)>0 \Leftrightarrow f'(1-x)<1-x \quad(* *)$.\\
		Đặt $t=1-x$, khi đó $(* *) \Leftrightarrow f'(t)<t \Leftrightarrow\hoac{&-1<t<1 \\ &t>3}\Rightarrow\hoac{&0<x<2 \\ &x<-2}\Rightarrow\hoac{&0<x<1 \\ &x<-2.}$\\
		Vậy hàm số $g(x)$ đồng biến trên các khoảng $(-\infty ;-2),(0 ; 1),(2 ; 4)$.
	}
\end{ex}
\begin{ex}[Sở Phú Thọ - 2020]%[2D1G1-2]
	\immini{
		Cho hàm số $y=f(x)$ có đồ thị hàm số $f'(x)$ như hình vẽ.\\
		Hàm số $g(x)=f\left(1+e^x\right)+2020$ nghịch biến trên khoảng nào dưới đây?
		\choice
		{$(0 ;+\infty)$}
		{$\left(\dfrac{1}{2}; 1\right)$}
		{\True $\left(0 ; \dfrac{1}{2}\right)$}
		{$(-1 ; 1)$}
	}{
		\begin{tikzpicture}[scale=0.7,>=stealth, font=\footnotesize, line join=round, line cap=round]
			\def\a{1} \def\b{-3} \def\c{0} \def\d{0} % Hệ số
			\def\xmin{-2} \def\xmax{4}
			\def\ymin{-5} \def\ymax{2} 
			%\draw[color=gray!50,dashed] (\xmin,\ymin) grid (\xmax,\ymax); 
			\draw[->] (\xmin,0)--(\xmax,0) node [below]{$x$};
			\draw[->] (0,\ymin)--(0,\ymax) node [left]{$y$};
			\node at (0,0) [above left]{$O$};
			\node at (3,0) [below right]{$3$};
			\draw[dashed] (2,0) node[above]{$2$}--(2,-4) --(0,-4) node[left]{$-4$};
			\clip (\xmin+0.1,\ymin+0.1) rectangle (\xmax-0.5,\ymax-0.1);
			\draw[smooth,samples=300] plot(\x,{\a*(\x)^3+\b*(\x)^2+\c*(\x)+\d});
		\end{tikzpicture}
	}
	\loigiai{
		$g'(x)=e^x f'\left(1+e^x\right)$.\\
		Do $e^x>0, \forall x$ nên $g'(x) \leq 0 \Leftrightarrow f'\left(1+e^x\right) \leq 0 \Leftrightarrow 1+e^x \leq 3 \Leftrightarrow x \leq \ln 2$, dấu bằng xảy ra tại hữu hạn điểm.\\
		Nên $g(x)$ nghịch biến trên $(-\infty ; \ln 2)$.\\
		Vì $\left(0 ; \dfrac{1}{2}\right) \subset (-\infty ; \ln 2)$ nên hàm số đã cho nghịch biến trên $\left(0 ; \dfrac{1}{2}\right)$.
	}
\end{ex}

\begin{ex}%[2D1K1-2]
	[THPT Anh Sơn - Nghệ An - 2020]
	Cho hàm số $y=f(x)$ có bảng xét dấu của đạo hàm như sau.
	\begin{center}
		\begin{tikzpicture}
			\tkzTabInit[nocadre,lgt=1.2,espcl=2,deltacl=0.6]
			{$x$ /0.6,$f'(x)$ /0.6}
			{$-\infty$,$-2$,$-1$,$2$,$4$,$+\infty$}
			\tkzTabLine{,+,$0$,-,$0$,+,$0$,-,$0$,+,}
		\end{tikzpicture}
	\end{center}
	Hàm số $y=-2 f(x)+2019$ nghịch biến trên khoảng nào trong các khoảng dưới đây?
	\choice
	{$(2 ; 4)$}
	{$(-4 ; 2)$}
	{$(-2 ;-1)$}
	{\True $(-1 ; 2)$}
	\loigiai{
		Ta có $y'=-2 f'(x)$.\\
		$
		y'=0 \Leftrightarrow-2 f'(x)=0 \Leftrightarrow\hoac{&
			x=-2 \\
			&x=-1 \\
			&x=2 \\
			&x=4.}$\\
		Từ bảng xét dấu của $f'(x)$ ta có
		\begin{center}
			\begin{tikzpicture}
				\tkzTabInit[nocadre,lgt=1,espcl=2,deltacl=0.6]
				{$x$ /0.6,$y'$ /0.6}
				{$-\infty$,$-2$,$-1$,$2$,$4$,$+\infty$}
				\tkzTabLine{,-,$0$,+,$0$,-,$0$,+,$0$,-,}
			\end{tikzpicture}
		\end{center}
		Từ bảng xét dấu ta có hàm số nghịch biến trên khoảng $(-\infty ;-2),(-1 ; 2)$ và $(4 ;+\infty)$.}
\end{ex}

\begin{ex}[THPT Anh Sơn - Nghệ An - 2020]%[2D1G1-2]
	Cho hàm số $f(x)$ xác định và liên tục trên $\mathbb{R}$ và có đạo hàm $f'(x)$ thỏa mãn $f'(x)=(1-x)(x+2) g(x)+2019$ với $g(x)<0, ~\forall x \in \mathbb{R}$ . Hàm số $y=f(1-x)+2019 x+2020$ nghịch biến trên khoảng nào?
	\choice
	{$(1 ;+\infty)$}
	{$(0 ; 3)$}
	{$(-\infty ; 3)$}
	{\True $(3 ;+\infty)$}
	\loigiai{
		Đặt $h(x)=f(1-x)+2019 x+2020$.\\
		Vì hàm số $f(x)$ xác định trên $\mathbb{R}$ nên hàm số $h(x)$ cũng xác định trên $\mathbb{R}$.\\
		Ta có $h'(x)=-f'(1-x)+2019$.\\
		Do $h'(x)=0$ tại hữu hạn điểm nên để tìm khoảng nghịch biến của hàm số $h(x)$, ta tìm các giá trị của $x$ sao cho $h'(x)<0 \Leftrightarrow-f'(1-x)+2019<0$\\ 
		$\Leftrightarrow f'(1-x)-2019>0 \\
		\Leftrightarrow x(3-x) g(1-x)>0 \Leftrightarrow x(3-x)<0(\text{~do~}g(x)<0, \forall x \in \mathbb{R})$\\
		$\Leftrightarrow\hoac{&
			x<0 \\
			&x>3.}$\\
		Vậy hàm số $y=f(1-x)+2019 x+2020$ nghịch biến trên các khoảng $(-\infty ; 0)$ và $(3 ;+\infty).$}
\end{ex}

\begin{ex}%[2D1G1-2]
	Cho hàm số $y=f(x)$ xác định trên $\mathbb{R}$ và có bảng xét dấu đạo hàm như sau:
	\begin{center}
		\begin{tikzpicture}
			\tkzTabInit[nocadre,lgt=2,espcl=2,deltacl=0.6]
			{$x$ /0.6,$f'(x)$ /0.6}
			{$-\infty$,$-1$,$1$,$4$,$+\infty$}
			\tkzTabLine{,-,$0$,+,$0$,-,$0$,+,}
		\end{tikzpicture}
	\end{center}
	Biết $f(x)>2,~ \forall x \in \mathbb{R}$. Xét hàm số $g(x)=f(3-2 f(x))-x^3+3 x^2-2020$. Khẳng định nào sau đây đúng?
	\choice
	{Hàm số $g(x)$ đồng biến trên khoảng $(-2 ;-1)$}
	{Hàm số $g(x)$ nghịch biến trên khoảng $(0 ; 1)$}
	{Hàm số $g(x)$ đồng biến trên khoảng $(3 ; 4)$}
	{\True Hàm số $g(x)$ nghịch biến trên khoảng $(2 ; 3)$}
	\loigiai{
		Ta có $g'(x)=-2 f'(x) f'(3-2 f(x))-3 x^2+6 x$.\\
		Vì $f(x)>2, ~\forall x \in \mathbb{R}$ nên $3-2 f(x)<-1 ~\forall x \in \mathbb{R}$.\\
		Từ bảng xét dấu $f'(x)$ suy ra $f'(3-2 f(x))<0, ~\forall x \in \mathbb{R}$.\\
		Từ đó ta có bảng xét dấu sau:
		\begin{center}
			\begin{tikzpicture}
				\tkzTabInit[nocadre,lgt=4,espcl=1.7,deltacl=0.6]
				{$x$ /0.7,$-f'(x)f'\left( 3-2f(x)\right) $/0.8,$-3x^2+6x$/0.7}
				{$-\infty$,$-1$,$0$,$1$,$2$,$4$,$+\infty$}
				\tkzTabLine{,-,$0$,+,|,+,$0$,-,|,-,$0$,+,}
				\tkzTabLine{,-,|,-,$0$,+,|,+,$0$,-,|,-,}
			\end{tikzpicture}
		\end{center}
		Từ bảng xét dấu trên, loại trừ đáp án suy ra hàm số $g(x)$ nghịch biến trên khoảng $(2 ; 3)$.}
\end{ex}

\begin{ex}%[2D1G1-2]
	Cho hàm số $f(x)$ có bảng biến thiên như sau:
	\begin{center}
		\begin{tikzpicture}
			\tkzTabInit[nocadre,lgt=1.2,espcl=2.5,deltacl=0.6]
			{$x$ /0.7, $f'(x)$ /0.7, $f(x)$ /2.5}
			{$-\infty$,$1$,$2$,$3$,$4$,$+\infty$}
			\tkzTabLine{,+,$0$,-,$0$,+,$0$,-,$0$,+,}
			\tkzTabVar{-/$-\infty$,+/$3$,-/$1$,+/$2$,-/$0$,+/$+\infty$}
		\end{tikzpicture}
	\end{center}
	Hàm số $y=(f(x))^3-3 .(f(x))^2$ nghịch biến trên khoảng nào dưới đây?
	\choice
	{$(1 ; 2)$}
	{$(3 ; 4)$}
	{$(-\infty ; 1)$}
	{\True $(2 ; 3)$}
	\loigiai{
		Ta có $y'=3 \cdot(f(x))^2 \cdot f'(x)-6 \cdot f(x) \cdot f'(x)=3 f(x) \cdot f'(x) \cdot[f(x)-2]. \\
		y'=0 \Leftrightarrow \hoac{&f(x)=0 \Leftrightarrow x \in\left\{x_1, 4 \mid x_1<1\right\}\\
			&f(x)=2 \Leftrightarrow x \in\left\{x_2, x_3, 3, x_4 \mid x_1<x_2<1<x_3<2 ; 4<x_4\right\}\\
			&f'(x)=0 \Leftrightarrow x \in\{1,2,3,4\}.}$\\
		Lập bảng xét dấu ta có
		\begin{center}
			\begin{tikzpicture}
				\tkzTabInit[nocadre,lgt=2,espcl=1.5,deltacl=0.6]
				{$x$ /0.7,$f(x)$ /0.7,$f(x)-2$ /0.7,$f'(x)$/0.7,$y'$/0.7}
				{$-\infty$,$x_1$,$x_2$,$1$,$x_3$,$2$,$3$,$4$,$x_4$,$+\infty$}
				\tkzTabLine{,-,$0$,+,|,+,|,+,|,+,|,+,$0$,+,|,+,|,+,}
				\tkzTabLine{,-,|,-,$0$,+,$0$,+,$0$,-,|,-,$0$,-,|,-,$0$,+}
				\tkzTabLine{,+,|,+,|,+,$0$,-,|,-,$0$,+,$0$,-,$0$,+,|,+}
				\tkzTabLine{,+,$0$,-,$0$,+,$0$,-,$0$,+,$0$,-,$0$,+,$0$,-,$0$,+}
			\end{tikzpicture}
		\end{center}
		
		Do đó hàm số nghịch biến trên khoảng $(2 ; 3)$.
	}
\end{ex}
\begin{ex}%[2D1G1-2]
	Cho hàm số $y=f(x)$ có đồ thị nằm trên trục hoành và có đạo hàm trên $\mathbb{R}$, bảng xét dấu của biểu thức $f'(x)$ như bảng dưới đây.
	\begin{center}
		\begin{tikzpicture}
			\tkzTabInit[nocadre,lgt=1.2,espcl=2,deltacl=0.6]
			{$x$ /0.6,$f'(x)$ /0.6}
			{$-\infty$,$-2$,$-1$,$3$,$+\infty$}
			\tkzTabLine{,-,$0$,+,$0$,-,$0$,+,}
		\end{tikzpicture}
	\end{center}
	Hàm số $y=g(x)=\dfrac{f\left(x^2-2 x\right)}{f\left(x^2-2 x\right)+1}$ nghịch biến trên khoảng nào dưới đây?
	\choice
	{$(-\infty ; 1)$}
	{$\left(-2 ; \dfrac{5}{2}\right)$}
	{\True $(1 ; 3)$}
	{$(2 ;+\infty)$}
	\loigiai{
		$ g'(x)=\dfrac{\left(x^2-2 x\right)'\cdot f'\left(x^2-2 x\right)}{\left(f\left(x^2-2 x\right)+1\right)^2}=\dfrac{(2 x-2) \cdot f'\left(x^2-2 x\right)}{\left(f\left(x^2-2 x\right)+1\right)^2}. \\
		g'(x)=0 \Leftrightarrow\hoac{
			&2 x - 2 = 0\\
			&f '( x ^{2}- 2 x ) = 0}
		\Leftrightarrow \hoac{&x = 1\\
			&x ^{2}- 2 x = - 2\\
			&x ^{2}- 2 x = - 1\\
			&x ^{2}- 2 x = 3}
		\Leftrightarrow \hoac{&x=1 \\
			&x=-1 \\
			&x=3.}
		$\\
		Ta có bảng xét dấu của $g'(x)$
		\begin{center}
			\begin{tikzpicture}
				\tkzTabInit[nocadre,lgt=1.2,espcl=2,deltacl=0.6]
				{$x$ /0.6,$g'(x)$ /0.6}
				{$-\infty$,$-1$,$1$,$3$,$+\infty$}
				\tkzTabLine{,-,$0$,+,$0$,-,$0$,+,}
			\end{tikzpicture}
		\end{center}
		Dựa vào bảng xét dấu ta có hàm số $y=g(x)$ nghịch biến trên các khoảng $(-\infty ;-1)$ và $(1 ; 3)$.}
\end{ex}
\begin{ex}[Liên trường huyện Quảng Xương - Thanh Hóa - 2021]%[2D1G1-2]
	\immini{
		Cho các hàm số $y=f(x)$; $y=g(x)$ liên tục trên $\mathbb{R}$ và có đồ thị các đạo hàm $f'(x) ; g'(x)$ (đồ thị hàm số $y=g'(x)$ là đường đậm hơn) như hình vẽ.\\
		Hàm số $h(x)=f(x-1)-g(x-1)$ nghịch biến trên khoảng nào dưới đây?
		\choice
		{$\left(\dfrac{1}{2}; 1\right)$}
		{$(1 ;+\infty)$}
		{$(2 ;+\infty)$}
		{\True $\left(-1 ; \dfrac{1}{2}\right)$}
	}
	{
		\begin{tikzpicture}[scale=1,>=stealth, font=\footnotesize, line join=round, line cap=round]
			%\def\a{1} \def\b{-6} \def\c{9} \def\d{1} % Hệ số
			\def\xmin{-4} \def\xmax{3}
			\def\ymin{-2} \def\ymax{4} 
			%\draw[color=gray!50,dashed] (\xmin,\ymin) grid (\xmax,\ymax); 
			\draw[->] (\xmin,0)--(\xmax,0) node [below]{$x$};
			\draw[->] (0,\ymin)--(0,\ymax) node [left]{$y$};
			\node at (0,0) [above left]{$O$};
			\node at (1,3) [below left]{$f'(x)$};
			\node at (1.5,3) [below right]{$g'(x)$};
			\draw[dashed] (-2,0) node[above right]{$-2$}--(-2,1);
			\draw[dashed] (1,0) node[below]{$1$}--(1,1);
			\draw[dashed] (-0.5,0) node[below]{$-0{,}5$}--(-0.5,2.125);
			\clip (\xmin+0.1,\ymin+0.1) rectangle (\xmax-0.5,\ymax-0.1);
			\draw[smooth,samples=300][domain=-3:2] plot(\x,{2*(\x)^4+4*(\x)^3-2*(\x)^2-4*(\x)+1});
			\draw[smooth,samples=300,line width=1.2pt] plot(\x,{(\x)^3+(\x)^2-2*(\x)+1});
		\end{tikzpicture}
	}
	
	\loigiai{
		Ta có: $h'(x)=f'(x-1)-g'(x-1)$.\\
		Dựa vào hình vẽ ta có hàm số $h(x)$ nghịch biến\\
		$\Leftrightarrow h'(x)<0 \Leftrightarrow f'(x-1)<g'(x-1)$\\
		$
		\Leftrightarrow\hoac{&- 2 < x - 1 < - \dfrac{1}{2}\\
			&0 < x - 1 < 1}
		\Leftrightarrow \hoac{
			&-1<x<\dfrac{1}{2}\\
			&1<x<2.}$\\
		Do đó hàm số $h(x)$ nghịch biến trên các khoảng $\left(-1 ; \dfrac{1}{2}\right)$ và $(1 ; 2)$.
	}
\end{ex}
\begin{ex}[THPT Quế Võ 1 - Bắc Ninh - 2021] %[2D1G1-2]
	\immini{
		Cho ba hàm số $y=f(x), y=g(x), y=h(x)$. Đồ thị của ba hàm số $y=f'(x), y=g'(x), y=h'(x)$ được cho như hình vẽ.\\
		Hàm số $k(x)=f(x+7)+g(5 x+1)-h\left(4 x+\dfrac{3}{2}\right)$ đồng biến trên khoảng nào dưới đây?
		\choice
		{$\left(-\dfrac{5}{8}; 0\right)$}
		{$\left(\dfrac{5}{8};+\infty\right)$}
		{\True $\left(\dfrac{3}{8}; 1\right)$}
		{$\left(-\dfrac{3}{8}; 1\right)$}
	}
	{
		\begin{tikzpicture}[scale=0.25,>=stealth, font=\footnotesize, line join=round, line cap=round]
			\def\a{-.078} \def\b{1.25} \def\c{0} % Hệ số
			\def\xmin{-4} \def\xmax{25}
			\def\ymin{-8} \def\ymax{18}
			
			%\draw[color=gray!50,dashed] (\xmin,\ymin) grid (\xmax,\ymax);
			
			\draw[->] (\xmin,0)--(\xmax,0) node [below]{$x$};
			\draw[->] (0,\ymin)--(0,\ymax) node [left]{$y$};
			\node at (20,14) [below right]{$y=g'(x)$};
			\node at (18,-2) [below left]{$y=h'(x)$};
			\node at (16,5) [below right]{$y=f'(x)$};
			\node at (0,0) [below left]{$O$};
			\draw[dashed] (3,0) node[below]{$3$}--(3,10)--(0,10) node[left]{$10$};
			\draw[dashed] (8,0) node[below]{$8$}--(8,5)--(0,5) node[left]{$5$};
			\draw[dashed] (4,0) node[below]{$4$}--(4,2)--(0,2) node[left]{$2$};
			\clip (\xmin+0.1,\ymin+0.1) rectangle (\xmax-0.5,\ymax-0.1);
			\draw[smooth,samples=300,domain=-2:18] plot(\x,{\a*(\x)^2+\b*(\x)+\c});
			%\draw[smooth,samples=300,domain=-2:25] plot(\x,{0.02*(\x)^3-0.6*(\x)^2+5.16*(\x)});
			\draw[line width=1.2pt] (-2,5)..controls (1.7,1.5) and (4.5,1.6)..(7,2.6);
			\draw[line width=1.2pt] (7,2.6)..controls (9,3.5) and (12,5)..(20,13);
			\draw (-0.5,-2) -- (0,0)--(3,10).. controls +(65:1) and + (-190:1)..(6,15).. controls +(0:1) and + (-180:1)..(14,-1).. controls +(0:1) and + (+80:1)..(19,16);
			
		\end{tikzpicture}
	}
	\loigiai{
		Ta có $k'(x)=f'(x+7)+5 g'(5 x+1)-4 h'\left(4 x+\dfrac{3}{2}\right)$.\\
		Khi $x \in \left( \dfrac{3}{8};1\right)$ thì $\heva{&7{,}375<x+7<8\\&2{,}875<5x+1<6\\&3<4x+\dfrac{4}{3}<5{,}5}\Leftrightarrow \heva{&f'(x+7)>10\\&g'(5x+1)>2 \Rightarrow 5g'(5x+1)>10  \\&h'\left( 4x+\dfrac{3}{2}\right)<5 \Rightarrow -4h'\left( 4x+\dfrac{3}{2}\right) >-20}.$\\
		Do đó $k'(x)=f'(x+7)+5g'(5x+1)-4h'\left( 4x+\dfrac{3}{2}\right)>0$.\\
		Hàm số $k(x)=f(x+7)+g(5 x+1)-h\left(4 x+\dfrac{3}{2}\right)$ đồng biến trên $\left(\dfrac{3}{8}; 1\right)$.
	}
\end{ex}
\begin{ex}[THPT Thanh Chương 1 - Nghệ An- 2021] %[2D1G1-2]
	Cho hàm số $y=f(x)$ liên tục trên $\mathbb{R}$ có bảng xét dấu đạo hàm như sau
	\begin{center}
		\begin{tikzpicture}
			\tkzTabInit[nocadre,lgt=1.2,espcl=2,deltacl=0.6]
			{$x$ /0.6,$f'(x)$ /0.6}
			{$-\infty$,$1$,$2$,$3$,$4$,$+\infty$}
			\tkzTabLine{,-,$0$,+,$0$,+,$0$,-,$0$,+,}
		\end{tikzpicture}
	\end{center}
	Hàm số $y=3f(2x-1)-4x^3+15x^2-18x+1$ đồng biến trên khoảng nào dưới đây?
	\choice
	{$\left(3;+\infty\right)$}
	{\True $\left(1;\dfrac{3}{2}\right)$}
	{$\left(\dfrac{5}{2}; 3\right)$}
	{$\left(2;\dfrac{5}{2}\right)$}
	\loigiai{
		Ta có $y'=6f'(2x-1)-12x^2+30x-18=6\left[f'(2x-1)-2x^2+5x-3\right] $.\\
		Có $f'(2x-1)=0 \Leftrightarrow \hoac{&2x-1=1\\&2x-1=2\\&2x-1=3\\&2x-1=4} \Leftrightarrow \hoac{&x=1\\&x=\dfrac{3}{2}\\&x=2\\&x=\dfrac{5}{2}.}$
		Ta có bảng xét dấu sau
		\begin{center}
			\begin{tikzpicture}
				\tkzTabInit[nocadre,lgt=3.0,espcl=1.5,deltacl=0.6]
				{$x$ /1.0,$f(x)$ /0.6,$f'(2x-1)$ /0.6,$-2x^2+5x-3$/0.6,$g'(x)$/0.6}
				{$-\infty$,$1$,$\dfrac{3}{2}$,$2$,$\dfrac{5}{2}$,$3$,$4$,$+\infty$}
				\tkzTabLine{,-,$0$,+,|,+,$0$,+,|,+,$0$,-,$0$,+,}
				\tkzTabLine{,-,$0$,+,$0$,+,$0$,-,$0$,+,|,+,|,+,}
				\tkzTabLine{,-,$0$,+,$0$,-,|,-,|,-,|,-,|,-,}
				\tkzTabLine{,-,$0$,+,$0$,,?,,|,,?,?,,?,}
			\end{tikzpicture}
		\end{center}
		Dựa vào bảng xét dấu trên, ta kết luận hàm số đã cho đồng biến trên khoảng $\left( 1; \dfrac{3}{2}\right).$
	}
\end{ex}


\begin{ex}%[2D2G4-3] %Câu 27 
	[THPT Hoàng Hoa Thám-Đà Nẵng-2021]
	Cho hàm số $f(x)$ có bảng xét dấu của $f'(x)$ như sau:\\
	\begin{center}
		\begin{tikzpicture}
			\tkzTabInit[lgt=1.2,espcl=2.3]
			{$x$/0.7, $f'(x)$ /.8} % first column
			{$-\infty$,$-3$,$1$, $2$, $+\infty$} % first row
			\tkzTabLine { ,+,0,-,0,+,0,+ }
		\end{tikzpicture}
	\end{center}	
	Hàm số $y=f\left(2-e^x\right)-\dfrac{1}{3}{e^{3x}}+3e^{2x}-5e^x+1$ đồng biến trên khoảng nào dưới đây?
	\choice
	{$\left(0;\dfrac{3}{2}\right)$}
	{$\left(1;3\right)$}
	{\True $\left(-3;0\right)$}
	{$\left(-4;-3\right)$}
	\loigiai{
		Ta có $y'=-e^x.f'\left(2-e^x\right)-e^{3x}+6e^{2x}-5e^x=e^x\left[-f'\left(2-e^x\right)-e^{2x}+6e^x-5\right]$ .\\
		Đặt $t=2-e^x$, ta được\\
		$y'=\left(2-t\right)\left[-f'(t)-\left(2-t\right)^2+6\left(2-t\right)-5\right]=\left(2-t\right)\left[-f'(t)-t^2-2t+3\right]$ .\\
		$y'=0\Leftrightarrow\left(2-t\right)\left[-f'(t)-t^2-2t+3\right]=0\Leftrightarrow
		\hoac{
			& t=2\\ 
			& f'(t)=-t^2-2t+3.}$\\
		Hàm số $g(x)=-x^2-2x+3$ là parabol có trục đối xứng $x=-1$ và cắt trục hoành tại 2 điểm có hoành độ 
		$\hoac{
			& x=1\\ 
			& x=-3
		}$. Suy ra $f'(t)=-t^2-2t+3\Leftrightarrow \hoac{
			& t=1\\ 
			& t=-3. }$\\
		Bảng xét dấu\\
		\begin{center}
			\begin{tikzpicture}
				\tkzTabInit[lgt=3.9,espcl=2,nocadre]
				{$t$/0.7, $2-t$ /0.8, $-f'(t)-t^2-2t+3$ /0.8, $y'$ /0.8} % first column
				{$-\infty$,$-3$,$1$,$2$,$+\infty$} % first row
				\tkzTabLine { ,+,|,+,|,+,z,-, } % second row
				\tkzTabLine {,-,0,+,0,-,|,-,} % third row
				\tkzTabLine {,-,0,+,0,-,0,+,} % last row
			\end{tikzpicture}
		\end{center}
		Dựa vào bảng xét dấu $y'>0,\forall x\in\left(-3;0\right)$.}
\end{ex}


\begin{ex}%[2D1G1-2]%Câu 28 
	[Sở Lạng Sơn 2022] Cho hàm số $f(x)$ có bảng biến thiên như sau:\\
	\begin{center}
		\begin{tikzpicture}
			\tkzTabInit[espcl=2.5,lgt=1,nocadre]
			{$x$/0.7,$y'$/0.7,$y$/3.5}
			{$-\infty$,$1$,$2$,$3$,$4$,$+\infty$}
			\tkzTabLine{,+,0,-,0,+,0,-,0,+,}
			\node (0) at ($(N12)+(0,-3)$) {$-\infty$};
			\node (1) at ($(N22)+(0,-.5)$) {$3$};
			\node (2) at ($(N32)+(0,-1.7)$) {$1$};
			\node (3) at ($(N42)+(0,-0.7)$) {$2$};
			\node (4) at ($(N52)+(0,-2.3)$) {$0$};
			\node (5) at ($(N62)+(0,-.3)$) {$+\infty$};
			%				\node (8) at ($(N42)+(0,-.5)$) {};
			%				\coordinate (9) at ($(N42)!.6!(N53)+ (-0.5,0)$);
			%				\coordinate (6) at ($(T12)!.6!(T13)$);
			%				\coordinate (7) at ($(T22)!.6!(T23)$);
			\draw[-stealth] (0)--(1);
			\draw[-stealth] (1)--(2);
			\draw[-stealth] (2)--(3);
			\draw[-stealth] (1)--(2);
			\draw[-stealth] (3)--(4);
			\draw[-stealth] (4)--(5);
			%				\draw[->,red] (5)--(8);
			%				\draw[->,red] (8)--(9);
			%				\draw[blue,dashed](6)--(7)node[above left]{$y=0$};
		\end{tikzpicture}		
	\end{center}
	Hàm số $y=\left[f(x)\right]^3-3\left[f(x)\right]^2$ đồng biến trên khoảng nào dưới đây?
	\choice
	{$\left(-\infty\,;1\right)$}
	{$\left(1\,;2\right)$}
	{\True $\left(3\,;4\right)$}
	{$\left(2\,;3\right)$}
	\loigiai{
		Ta có $y'=3f'(x)\left[f^2(x)-2f(x)\right]$. 
		Phương trình $y'=0\Leftrightarrow \hoac{
			&{f}'(x)=0\\ 
			& f(x)=0\\ 
			& f(x)=2.
		}$
		\begin{center}
			\begin{tikzpicture}
				\tkzTabInit[espcl=2.5,lgt=1.5]
				{$x$/0.7,$y'$/0.7,$y$/3.5}
				{$-\infty$,$1$,$2$,$3$,$4$,$+\infty$}
				\tkzTabLine{,+,0,-,0,+,0,-,0,+,}
				\node (0) at ($(N12)+(0,-3)$) {$-\infty$};
				\node (1) at ($(N22)+(0,-.3)$) {$3$};
				\node (2) at ($(N32)+(0,-1.7)$) {$1$};
				\node (3) at ($(N42)+(0,-0.8)$) {$2$};
				\node (4) at ($(N52)+(0,-2.3)$) {$0$};
				\node (5) at ($(N62)+(0,-.3)$) {$+\infty$};
				\node (a) at ($(N11)+(0.65,0.35)$) {$a$};
				\node (b) at ($(N11)+(2.0,0.4)$) {$b$};
				\node (c) at ($(N11)+(3.38,0.35)$) {$c$};
				\node (d) at ($(N11)+(11.85,0.4)$) {$d$};
				\node (6) at ($(N12)+(0,-0.8)$) {};
				\node (7) at ($(N62)+(0,-0.8)$) {};
				\node (8) at ($(N12)+(0,-2.3)$) {};
				\node (9) at ($(N62)+(0,-2.3)$) {};
				%				\node (8) at ($(N42)+(0,-.5)$) {};
				%				\coordinate (9) at ($(N42)!.6!(N53)+ (-0.5,0)$);
				\coordinate (A) at ($(0)!.25!(1)$);
				\coordinate (B) at ($(0)!.8!(1)$);
				\coordinate (C) at ($(1)!.35!(2)$);
				\coordinate (D) at ($(4)!.75!(5)$);
				%				\coordinate (7) at ($(T22)!.6!(T23)$);
				\draw[->] (0)--(1);
				\draw[->] (1)--(2);
				\draw[->] (2)--(3);
				\draw[->] (1)--(2);
				\draw[->] (3)--(4);
				\draw[->] (4)--(5);
				%				\draw[->,red] (5)--(8);
				%				\draw[->,red] (8)--(9);
				\draw[blue,dashed](6)--(7)node[below]{$y=2$} (a)--(A) (b)--(B) (c)--(C) (d)--(D);
				\draw[blue,dashed](8)--(9)node[below left]{$y=0$};
			\end{tikzpicture}		
		\end{center}
		Dựa vào bảng biến thiên, ta thấy $f'(x)=0\Leftrightarrow x\in \{ 1\,;2\,;3\,;4 \}$;\\
		$f(x)=0\Leftrightarrow x=a<1$ hoặc $x=4$;\\
		$f(x)=2\Leftrightarrow \hoac{
			& x=b\,\,\left(a<b<1\right)\\ 
			& x=c\in\left(1\,;2\right)\\ 
			& x=3\\ 
			& x=d>4.
		}$ \\
		Ta lập được bảng xét dấu của $y'$ 
		\begin{center}
			\begin{tikzpicture}
				\tkzTabInit[lgt=1.2,espcl=1.5,nocadre]
				{$x$/1, $f(x)$ /.8} % first column
				{$-\infty$,$a$, $b$, $1$,$c$, $2$,$3$, $4$, $d$, $+\infty$} % first row
				\tkzTabLine { ,+,z,-,z,+,z,-,z,+,z,-,z,+,z,-,z,+, } % second row
				%				\tkzTabLine {,-,z,+,t,+,} % third row
				%				\tkzTabLine {,+,d,-,z,+,} % last row
			\end{tikzpicture}
		\end{center}
		Từ bảng xét dấu, ta thấy hàm số đồng biến trên các khoảng \\
		$\left(-\infty;a\right)$, $\left(b;1\right)$, $\left(c;2\right)$, $\left(3;4\right)$ và $(d;+\infty)$.
	}
\end{ex}

\begin{ex}%[2D1G1-2]%Câu 29 
	[THPT Bùi Thị Xuân – Huế-2022] 
	\immini{
		Cho hàm số $y=f(x)$ là hàm đa thức bậc bốn. Đồ thị hàm số $f'(x+2)$ được cho trong hình vẽ bên. Hàm số 
		$$g(x)=4 f\left(x^2\right)-x^6+5 x^4-4 x^2+1$$
		đồng biến trên khoảng nào dưới đây?
		\choice
		{$(-4 ;-3)$}
		{\True $(2 ;+\infty)$}
		{$(-\sqrt{2};\sqrt{2})$}
		{$(-2 ;-1)$}}{
		\begin{tikzpicture}[scale=0.6,font=\footnotesize, line join=round, line cap=round, >=stealth] %Đường cong bậc 3
			\draw[thick, ->] (-5.3,0)--(5,0);
			\draw[thick, ->] (0,-3.5)--(0,7);
			\draw (5.2,0) node[below] {$x$};
			\draw (0,7.1) node[left]{$y$};
			\draw (0,0) node[below left]{$0$};
			\draw[fill] (-2,0) circle (0.5pt)node[below left]{$ -2 $};
			\draw[fill] (2,0) circle (0.5pt)node[below]{$ 2$};
			\draw[fill] (0,3) circle (0.5pt)node[left]{$ 3 $};
			\draw[fill] (0,1) circle (0.5pt)node[right]{$ 1 $};
			\draw[fill] (0,-1) circle (0.5pt)node[right]{$ -1 $};
			\draw[dashed] (-2,0)--(-2,1) --(0,1); 
			\draw[dashed](2,0)--(2,3)--(0,3);
			\draw[line width=1.2pt,smooth,samples=100,domain=-2.8:4.5] plot(\x,{-0.271*(\x)^3+0.75*(\x)^2+1.583*\x-1});
		\end{tikzpicture}		
	}
	\loigiai{
		$\begin{aligned}
			& g(x)=4f\left(x^2\right)-x^6+5x^4-4x^2+1\Rightarrow g' (x)=8xf'\left(x^2\right)-6x^5+20x^3-8x.\\ 
			& g' (x)=0\Leftrightarrow 8xf'\left(x^2\right)-6x^5+20x^3-8x=0 \\
			& \Leftrightarrow 2x\left[4f'\left(x^2\right)-3x^4+10x^2-4\right]=0\\ 
			&\Leftrightarrow 		\hoac{ 			& 2x=0\\ 
				& 4f'(x^2)-3x^4+10x^2-4=0
			}
			\Leftrightarrow \hoac{	& x=0\\ 
				& f'\left(x^2\right)=\dfrac{3}{4}{x^4}-\dfrac{5}{2}{x^2}+1.}
		\end{aligned}$\\ 
		Xét
		$f'\left(x^2\right)=\dfrac{3}{4}x^4-\dfrac{5}{2}x^2+1$. Đặt $x^2=t+2$, ta có\\
		$ f' (t+2)=\dfrac{3}{4}{(t+2)^2}-\dfrac{5}{2}(t+2)+1=\dfrac{3}{4}\left(t^2+4t+4\right)-\dfrac{5}{2}(t+2)-1=\dfrac{3}{4}{t^2}+\dfrac{1}{2}t-1$\\
		Khi đó số nghiệm của phương trình chính là số giao điểm của đồ thị hàm số $y=f' (t+2)$ và\\
		$ y=\dfrac{3}{4}{t^2}+\dfrac{1}{2}t-1$\\
		Ta có đồ thị 
		\begin{center}
			\begin{tikzpicture}[scale=0.6,font=\footnotesize, line join=round, line cap=round, >=stealth] %Đường cong bậc 3
				\draw[thick, ->] (-5.3,0)--(5,0);
				\draw[thick, ->] (0,-3.5)--(0,7);
				\draw (5.2,0) node[below] {$x$};
				\draw (0,7.1) node[left]{$y$};
				\draw (0,0) node[below left]{$0$};
				\draw[fill] (-2,0) circle (0.5pt)node[below left]{$ -2 $};
				\draw[fill] (2,0) circle (0.5pt)node[below]{$ 2$};
				\draw[fill] (0,3) circle (0.5pt)node[left]{$ 3 $};
				\draw[fill] (0,1) circle (0.5pt)node[right]{$ 1 $};
				\draw[fill] (0,-1) circle (0.5pt)node[right]{$ -1 $};
				\draw[dashed] (-2,0)--(-2,1) --(0,1); 
				\draw[dashed](2,0)--(2,3)--(0,3);
				\draw[line width=1.2pt,smooth,samples=100,domain=-2.8:4.5] plot(\x,{-0.271*(\x)^3+0.75*(\x)^2+1.583*\x-1});		
				\draw[line width=1.2pt,smooth,samples=100,domain=-3.3:2.8] plot(\x,{0.75*(\x)^2+0.5*\x-1});
			\end{tikzpicture}
		\end{center}
		Dựa vào đồ thị ta có $f' (t+2)=\dfrac{3}{4}t^2+\dfrac{1}{2}t-1\Leftrightarrow \hoac{& t=-2\\ & t=0\\ & t=2} \Leftrightarrow\hoac{& x+2=-2\\ & x+2=0\\ & x+2=2} \Leftrightarrow \hoac{& x=-4\\ & x=-2\\ & x=0.}$\\
		Ta có bảng xét dấu $g' (x)$ như sau
		\begin{center}
			\begin{tikzpicture}
				\tkzTabInit[lgt=1.2,espcl=2,nocadre]
				{$x$/0.7, $f(x)$ /.7}
				{$-\infty$, $-4$,$-2$, $0$, $+\infty$} % first row
				\tkzTabLine { ,-,z,+,z,-,z,+, }
			\end{tikzpicture}
		\end{center}
		Vậy hàm số $g(x)=4 f\left(x^2\right)-x^6+5 x^4-4 x^2+1$ đồng biến trên khoảng $(2 ;+\infty)$.}
\end{ex}

\begin{ex}%[2D1G1-2]%Câu 30
	[Chuyên Bắc Ninh 2022] 
	\immini{
		Cho hàm số $ y=f(x)$ liên tục trên $\mathbb{R}$ có đồ thị hàm số $ y=f'(x)$ có đồ thị như hình vẽ bên.
		Hàm số $g(x)=2f\left(\left| x-1\right|\right)-x^2+2x+2020$ đồng biến trên khoảng nào
		\choice
		{$\left(-2;0\right)$}
		{$\left(-3;1\right)$}
		{$\left(1\,;3\right)$}
		{\True $\left(0\,;\,1\right)$}}{
		\begin{tikzpicture}[scale=0.6,font=\footnotesize, line join=round, line cap=round, >=stealth] %Đường cong bậc 3
			\draw[thick, ->] (-3.3,0)--(5,0);
			\draw[thick, ->] (0,-3.0)--(0,5.5);
			\draw (5.2,0) node[below] {$x$};
			\draw (0,5.8) node[left]{$y$};
			\draw (0,0) node[below left]{$0$};
			\draw[fill] (-1,0) circle (0.5pt)node[above]{$ -1 $};
			\draw[fill] (1,0) circle (0.5pt)node[below]{$ 1$};
			\draw[fill] (0,1) circle (0.5pt)node[left]{$ 1 $};
			\draw[fill] (0,-1) circle (0.5pt)node[right]{$ -1 $};
			\draw[fill] (0,3) circle (0.5pt)node[left]{$ 3 $};
			\draw[fill] (3,0) circle (0.5pt)node[below]{$ 3 $};
			\draw[dashed] (-1,0)--(-1,-1) --(0,-1); 
			\draw[dashed](1,0)--(1,1)--(0,1);
			\draw[dashed](3,0)--(3,3)--(0,3);
			\draw[line width=1.2pt,smooth,samples=100,domain=-2.2:4.3] plot(\x,{-0.333*(\x)^3+1*(\x)^2+1.333*\x-1});		
			%\draw[line width=1.2pt,smooth,samples=100,domain=-3.3:2.8] plot(\x,{0.75*(\x)^2+0.5*\x-1});
		\end{tikzpicture}	
	}
	\loigiai{
		Ta có $g(x)=2f\left(\left| x-1\right|\right)-x^2+2x+2020\Leftrightarrow g(x)=2f\left(\left| x-1\right|\right)-\left(x-1\right)^2+2021$.\\
		Xét hàm số $ k\left(x-1\right)=2f\left(x-1\right)-\left(x-1\right)^2+2021$.\\
		Đặt $ t=x-1$\\
		Xét hàm số $ h(t)=2f(t)-t^2+2021$ $\Rightarrow{h}'(t)=2f'(t)-2t$.\\
		Kẻ đường $ y=x$ như hình vẽ.
		\begin{center}
			\begin{tikzpicture}[scale=0.6,font=\footnotesize, line join=round, line cap=round, >=stealth] %Đường cong bậc 3
				\draw[thick, ->] (-3.3,0)--(5,0);
				\draw[thick, ->] (0,-3.0)--(0,5.5);
				\draw (5.2,0) node[below] {$x$};
				\draw (0,5.8) node[left]{$y$};
				%	\draw (0,0) node[below left]{$0$};
				\draw[fill] (-1,0) circle (0.5pt)node[above]{$ -1 $};
				\draw[fill] (1,0) circle (0.5pt)node[below]{$ 1$};
				\draw[fill] (0,1) circle (0.5pt)node[left]{$ 1 $};
				\draw[fill] (0,-1) circle (0.5pt)node[right]{$ -1 $};
				\draw[fill] (0,3) circle (0.5pt)node[left]{$ 3 $};
				\draw[fill] (3,0) circle (0.5pt)node[below]{$ 3 $};
				\draw[dashed] (-1,0)--(-1,-1) --(0,-1); 
				\draw[dashed](1,0)--(1,1)--(0,1);
				\draw[dashed](3,0)--(3,3)--(0,3);
				\draw[line width=1.2pt,smooth,samples=100,domain=-2.2:4.3] plot(\x,{-0.333*(\x)^3+1*(\x)^2+1.333*\x-1});		
				%\draw[line width=1.2pt,smooth,samples=100,domain=-3.3:2.8] plot(\x,{0.75*(\x)^2+0.5*\x-1});
				\draw[line width=1.2pt,smooth,samples=100](-2,-2)--(4,4);
			\end{tikzpicture}
		\end{center}
		Khi đó $h'(t)>0\Leftrightarrow{f}'(t)-t>0\Leftrightarrow{f}'(t)>t$$\Leftrightarrow \hoac{
			& t<-1\\ 
			& 1<t<3.
		}$\\
		Do đó $k'\left(x-1\right)>0\Leftrightarrow \hoac{
			& x-1<-1\\ 
			& 1<x-1<3} \Leftrightarrow \hoac{
			& x<0\\ 
			& 2<x<4.}$\\
		Ta có bảng biến thiên của hàm số $ k\left(x-1\right)=2f\left(x-1\right)-\left(x-1\right)^2+2021$.
		\begin{center}
			\begin{tikzpicture}
				\tkzTabInit[lgt=1.8,espcl=2.3]
				{$x$ /1.2, $k'(x-1)$ /1.2,$k(x-1)$ /2}
				{$-\infty$ , $0$,$2$,$4$, $+\infty$}
				\tkzTabLine{,+,0,-,0,+,0,-,}
				\tkzTabVar{-/$ $ ,+/$ $, -/$ $,+/$ $,-/$ $}
			\end{tikzpicture}
		\end{center}
		Khi đó, ta có bảng biến thiên của $g(x)=2f\left(\left| x-1\right|\right)-\left(x-1\right)^2+2021$ bằng cách lấy đối xứng qua đường thẳng $ x=1$ như sau\\
		\begin{center}
			\begin{tikzpicture}
				\tkzTabInit[lgt=1.2,espcl=2.5,nocadre]
				{$x$ /0.7, $g'(x)$ /0.7,$g(x)$ /2.5}
				{$-\infty$ ,$-2$, $0$,$1$,$2$,$4$, $+\infty$}
				\tkzTabLine{,+,0,-,0,+,0,-,0,+,0,-,}
				\tkzTabVar{-/$ $ ,+/$ $, -/$ $,+/$ $,-/$ $,+/ $ $,-/$ $}
			\end{tikzpicture}
		\end{center}
		Vậy hàm số đồng biến trên $\left(0;1\right)$.}
\end{ex}

\begin{ex}%[2D1G1-2]%Câu 31
	[Chuyên Thái Bình 2022] 
	\immini{
		Cho hàm số $f(x)=a{x^4}+b{x^3}+c{x^2}+dx+a$ có đồ thị hàm số $y=f'(x)$ như hình vẽ bên. Hàm số $y=g(x)=f\left(1-2x\right)f\left(2-x\right)$ đồng biến trên khoảng nào dưới đây?
		\choice
		{$\left(\dfrac{1}{2};\dfrac{3}{2}\right)$}
		{$\left(-\infty ;0\right)$}
		{$\left(0;2\right)$}
		{\True $\left(3;+\infty\right)$}}{
		\begin{tikzpicture}[scale=0.9,font=\footnotesize, line join=round, line cap=round, >=stealth] %Đường cong bậc 3
			\draw[thick, ->] (-2.5,0)--(2.5,0);
			\draw[thick, ->] (0,-2.8)--(0,2.8);
			\draw (2.6,0) node[below] {$x$};
			\draw (0,2.9) node[left]{$y$};
			\draw (0,0) node[below left]{$0$};
			\draw[fill] (-1,0) circle (0.5pt)node[below left]{$ -1 $};
			\draw[fill] (1,0) circle (0.5pt)node[below right]{$ 1$};
			%			\draw[dashed] (-1,0)--(-1,-1) --(0,-1); 
			%			\draw[dashed](1,0)--(1,1)--(0,1);
			%			\draw[dashed](3,0)--(3,3)--(0,3);
			\draw[line width=1.2pt,smooth,samples=100,domain=-1.3:1.3] plot(\x,{3*(\x)^3-3*\x});		
			%\draw[line width=1.2pt,smooth,samples=100,domain=-3.3:2.8] plot(\x,{0.75*(\x)^2+0.5*\x-1});
		\end{tikzpicture}	
	}
	\loigiai{
		Ta có $f'(x)=4a{x^3}+3b{x^2}+2cx+d$, theo đồ thị thì đa thức $f'(x)$ có ba nghiệm phân biệt là $-1,0,1$ nên $f'(x)=4ax\left(x+1\right)\left(x-1\right)=4a{x^3}-4ax\Rightarrow f(x)=a{x^4}-2a{x^2}+a=a{\left(x^2-1\right)^2}$.\\
		Dựa vào đồ thị hàm số $y=f'(x)$ ta có $a>0$ nên $f(x)>0,\forall x\in\mathbb{R}\setminus\left\{\pm 1\right\}$.\\
		$g'(x)=\left[f\left(1-2x\right)\right]'f\left(2-x\right)+f\left(1-2x\right)\left[f\left(2-x\right)\right]'=-2f'\left(1-2x\right)f\left(2-x\right)-f\left(1-2x\right)f'\left(2-x\right)$. Xét $x\in\left(\dfrac{1}{2};\dfrac{3}{2}\right)\Rightarrow
		\heva{		
			& 1-2x\in\left(-2;0\right)\\ 
			& 2-x\in\left(\dfrac{1}{2};\dfrac{3}{2}\right)}$, dấu của $f'(x)$ không cố định trên $\left(\dfrac{1}{2};\dfrac{3}{2}\right)$ nên ta không kết luận được tính đơn điệu của hàm số $g(x)$ trên $\left(\dfrac{1}{2};\dfrac{3}{2}\right)$.\\
		Xét $x\in\left(-\infty ;0\right)\Rightarrow
		\heva{
			& 1-2x\in\left(1;+\infty\right)\\ 
			& 2-x\in\left(2;+\infty\right)} 
		\Rightarrow \heva{
			& f'\left(1-2x\right)>0\\ 
			& f'\left(2-x\right)>0} \Rightarrow g'(x)<0$.\\
		Do đó, hàm số $g(x)$ nghịch biến trên $\left(-\infty ;0\right)$.\\
		$x\in\left(0;2\right)\Rightarrow \heva{
			& 1-2x\in\left(-3;1\right)\\ 
			& 2-x\in\left(0;2\right)}$, dấu của $f'(x)$ không cố định trên $\left(-3;1\right)$ và $\left(0;2\right)$ nên ta không kết luận được tính đơn điệu của hàm số $g(x)$ trên $\left(\dfrac{1}{2};\dfrac{3}{2}\right)$.\\
		Xét $x\in\left(3;+\infty\right)\Rightarrow \heva{
			& 1-2x\in\left(-\infty ;-5\right)\\ 
			& 2-x\in\left(-\infty ;-1\right)} \Rightarrow \heva{
			& f'\left(1-2x\right)<0\\ 
			& f'\left(2-x\right)<0} \Rightarrow g'(x)>0$. \\
		Do đó, hàm số $g(x)$ đồng biến trên $\left(3;+\infty\right)$.}
\end{ex}

\begin{dang}{Bài toán hàm ẩn, hàm hợp liên quan đến tham số và một số bài toán khác}
\end{dang}

\begin{ex}%[2D1G1-3]%Câu 1
	[Chuyên Lê Hồng Phong Nam Định 2019]
	\immini{
		Cho hàm số $ y=f(x)$ có đạo hàm liên tục trên $\mathbb{R}$. Biết hàm số $ y=f'(x)$ có đồ thị như hình vẽ. Gọi $ S$ là tập hợp các giá trị nguyên $ m\in\left[-5\,;\,\text{5}\right]$ để hàm số $ g(x)=f\left(x+m\right)$ nghịch biến trên khoảng $\left(1\,;\,2\right)$. Hỏi $S$ có bao nhiêu phần tử?
		\choice
		{$ 4$}
		{$ 3$}
		{$ 6$}
		{\True $ 5$}}{
		\begin{tikzpicture}[scale=0.9,font=\footnotesize, line join=round, line cap=round, >=stealth] %Đường cong bậc 3
			\draw[thick, ->] (-2.5,0)--(4,0);
			\draw[thick, ->] (0,-2.8)--(0,2.8);
			\draw (4.3,0) node[below] {$x$};
			\draw (0,2.9) node[left]{$y$};
			\draw (0,0) node[below left]{$0$};
			\draw[fill] (-1,0) circle (0.5pt)node[below left]{$ -1 $};
			\draw[fill] (1,0) circle (0.5pt)node[below]{$ 1$};
			\draw[fill] (3,0) circle (0.5pt)node[below right]{$ 3$};
			%			\draw[dashed] (-1,0)--(-1,-1) --(0,-1); 
			%			\draw[dashed](1,0)--(1,1)--(0,1);
			%			\draw[dashed](3,0)--(3,3)--(0,3);
			\draw[line width=1.2pt,smooth,samples=100,domain=-1.65:3.5] plot(\x,{0.33*(\x)^3-(\x)^2-0.333*(\x)+1});		
			%\draw[line width=1.2pt,smooth,samples=100,domain=-3.3:2.8] plot(\x,{0.75*(\x)^2+0.5*\x-1});
		\end{tikzpicture}	
	}
	\loigiai{
		Ta có $g'(x)=f'\left(x+m\right)$. Vì $ y=f'(x)$ liên tục trên $\mathbb{R}$ nên $g'(x)=f'\left(x+m\right)$ cũng liên tục trên $\mathbb{R}$. Căn cứ vào đồ thị hàm số $ y=f'(x)$ ta thấy\\
		$g'(x)<0\Leftrightarrow{f}'\left(x+m\right)<0$ $\Leftrightarrow\hoac{
			& x+m<-1\\ 
			& 1<x+m<3} \Leftrightarrow \hoac{
			& x<-1-m\\ 
			& 1-m<x<3-m.}$\\
		Hàm số $ g(x)=f\left(x+m\right)$ nghịch biến trên khoảng $\left(1\,;\,2\right)$
		$\Leftrightarrow \hoac{
			& 2\le-1-m\\ 
			&\hoac{
				& 3-m\ge 2\\ 
				& 1-m\le 1}} \Leftrightarrow \hoac{
			& m\le-3\\ 
			& 0\le m\le 1.}$\\
		Mà $ m$ là số nguyên thuộc đoạn $\left[-5\,;\,5\right]$ nên ta có $ S=\left\{-5;-4;-3;0;1\right\}$.\\
		Vậy $ S$ có $5$ phần tử.}
\end{ex}

\begin{ex}%[2D1G1-3]%Câu 2
	[Chuyên Nguyễn Bỉnh Khiêm-Quảng Nam-2020] Cho hàm số $ y=f(x)$ có đạo hàm trên $\mathbb{R}$ và bảng xét dấu đạo hàm như hình vẽ sau
	\begin{center}
		\begin{tikzpicture}
			\tkzTabInit[lgt=1.2,espcl=2.5,nocadre]
			{$x$/0.7, $f'(x)$ /2.5} % first column
			{$-\infty$, $-10$,$-2$, $3$,$8$, $+\infty$} % first row
			\tkzTabLine { ,+,z,-,z,+,z,-,z,+, } % second row
			%				\tkzTabLine {,-,z,+,t,+,} % third row
			%				\tkzTabLine {,+,d,-,z,+,} % last row
		\end{tikzpicture}
	\end{center}
	Có bao nhiêu số nguyên $ m$ để hàm số $ y=f\left(x^3+4x+m\right)$ nghịch biến trên khoảng $\left(-1;1\right)$?
	\choice
	{$ 3$}
	{$ 0$}
	{\True $ 1$}
	{$ 2$}
	\loigiai
	{
		Đặt $ t=x^3+4x+m\Rightarrow{t}'=3x^2+4$ nên $ t$ đồng biến trên $\left(-1;1\right)$ và $ t\in\left(m-5;m+5\right)$.\\
		Yêu cầu bài toán trở thành tìm $ m$ để hàm số $ f(t)$ nghịch biến trên khoảng $\left(m-5;m+5\right)$.\\
		Dựa vào bảng biến thiên ta được $\heva{
			& m-5\ge-2\\ 
			& m+5\le 8} \Leftrightarrow \heva{
			& m\ge 3\\ 
			& m\le 3} \Leftrightarrow m=3$.}
\end{ex}

\begin{ex}%[2D1G1-3]%Câu 3
	[Chuyên ĐH Vinh-Nghệ An-2020]
	\immini{
		Cho hàm số $ f(x)$ có đạo hàm trên $\mathbb{R}$và $ f(1)=1$. Đồ thị hàm số $ y=f'(x)$ như hình bên. Có bao nhiêu số nguyên dương $ a$ để hàm số $ y=\left| 4f\left(\sin x\right)+\cos 2x-a\right|$ nghịch biến trên $\left(0;\dfrac{\pi}{2}\right)$?
		\choice
		{$ 2$}
		{\True $ 3$}
		{Vô số}
		{$ 5$}}{
		\begin{tikzpicture}[scale=0.9,font=\footnotesize, line join=round, line cap=round, >=stealth] %Đường cong bậc 3
			\draw[thick, ->] (-2.5,0)--(3,0);
			\draw[thick, ->] (0,-2.8)--(0,2.8);
			\draw (3.1,0) node[below] {$x$};
			\draw (0,2.9) node[left]{$y$};
			\draw (0,0) node[below left]{$0$};
			\draw[fill] (-1,0) circle (0.5pt)node[below]{$ -1 $};
			\draw[fill] (1,0) circle (0.5pt)node[above]{$ 1$};
			%	\draw[fill] (3,0) circle (0.5pt)node[below right]{$ 3$};
			\draw[dashed] (-1,0)--(-1,1); 
			\draw[dashed](1,0)--(1,-1);
			%			\draw[dashed](3,0)--(3,3)--(0,3);
			\draw[line width=1.2pt,smooth,samples=100,domain=-2:2] plot(\x,{.8*(\x)^3+0*(\x)^2-1.8*(\x)});		
			%\draw[line width=1.2pt,smooth,samples=100,domain=-3.3:2.8] plot(\x,{0.75*(\x)^2+0.5*\x-1});
			\draw (2.0,2.8) node[left]{$y=f'(x)$};
		\end{tikzpicture}	
	}
	\loigiai
	{		Đặt $g(x)=\left| 4f\left(\sin x\right)+\cos 2x-a\right|\Rightarrow g(x)=\sqrt{\left[4f\left(\sin x\right)+\cos 2x-a\right]^2}$ .\\
		$\Rightarrow{g}'(x)=\dfrac{\left[4\cos x\cdot f'\left(\sin x\right)-2\sin 2x\right]\left[4f\left(\sin x\right)+\cos 2x-a\right]}{\sqrt{\left[4f\left(\sin x\right)+\cos 2x-a\right]^2}}$.\\
		Ta có $ 4\cos x\cdot f'\left(\sin x\right)-2\sin 2x=4\cos x\left[f'\left(\sin x\right)-\sin x\right]$.\\
		Với $ x\in\left(0;\dfrac{\pi}{2}\right)$ thì $\cos x>0,\sin x\in\left(0;1\right)\Rightarrow{f}'\left(\sin x\right)-\sin x<0$.\\
		Hàm số $ g(x)$ nghịch biến trên $\left(0;\dfrac{\pi}{2}\right)$ khi $ 4f\left(\sin x\right)+\cos 2x-a\ge 0,\forall x\in\left(0;\dfrac{\pi}{2}\right)$\\
		$\Leftrightarrow 4f\left(\sin x\right)+1-2\sin^2x\ge a,\forall x\in\left(0;\dfrac{\pi}{2}\right)$.\\
		Đặt $ t=\sin x$ được $ 4f(t)+1-2t^2\ge a,\forall t\in\left(0;1\right)$ (*).\\
		Xét $ h(t)=4f(t)+1-2t^2\Rightarrow{h}'(t)=4f'(t)-4t=4\left[f'(t)-1\right]$.\\
		Với $ t\in\left(0;1\right)$ thì $h'(t)<0\Rightarrow h(t)$ nghịch biến trên $\left(0;1\right)$.\\
		Do đó (*) $\Leftrightarrow a\le h(1)=4f(1)+1-2.1^2=3$.\\
		Vậy có $3$ giá trị nguyên dương của a thỏa mãn.}
\end{ex}


\begin{ex}%[2D1G1-3]%Câu 4
	[Chuyên Quang Trung-2020]
	\immini{
		Cho hàm số $ y=f(x)$ có đạo hàm liên tục trên $\mathbb{R}$ và có đồ thị $ y=f'(x)$ như hình vẽ. Đặt $ g(x)=f\left(x-m\right)-\dfrac{1}{2}{\left(x-m-1\right)^2}+2019$, với $ m$ là tham số thực. Gọi $ S$ là tập hợp các giá trị nguyên dương của $ m$ để hàm số $ y=g(x)$ đồng biến trên khoảng $\left(5;6\right)$. Tổng tất cả các phần tử trong $ S$ bằng
		\choice
		{$ 4$}
		{$ 11$}
		{\True $ 14$}
		{$ 20$}}{
		\begin{tikzpicture}[scale=0.9,font=\footnotesize, line join=round, line cap=round, >=stealth] %Đường cong bậc 3
			\draw[style=help lines,step=1] (-2.5,-3) grid (3,3.5);
			\draw[thick, ->] (-2.5,0)--(3.5,0);
			\draw[thick, ->] (0,-2.8)--(0,2.8);
			\draw (3.6,0) node[below] {$x$};
			\draw (0,3) node[above left]{$y$};
			\draw (0,0) node[below left]{$0$};
			%\draw[fill] (-1,0) circle (0.5pt)node[below]{$ -1 $};
			\draw[fill] (1,0) circle (0.5pt)node[below left]{$ 1$};
			%	\draw[fill] (3,0) circle (0.5pt)node[below right]{$ 3$};
			\draw[dashed] (-1,0)--(-1,-2) --(2,-2)--(2,0); 
			\draw[dashed](3,0)--(3,2) --(0,2);
			\draw (-1,-2) circle (2pt);
			\draw (3,2) circle (2pt);
			%			\draw[dashed](3,0)--(3,3)--(0,3);
			\draw[line width=1.2pt,smooth,samples=100,domain=-1.1:3.1] plot(\x,{1*(\x)^3-3*(\x)^2-0*(\x)+2});		
			%\draw[line width=1.2pt,smooth,samples=100,domain=-3.3:2.8] plot(\x,{0.75*(\x)^2+0.5*\x-1});
			%\draw (2.0,2.8) node[left]{$y=f'(x)$};
		\end{tikzpicture}	
	}
	\loigiai
	{
		Xét hàm số $ g(x)=f\left(x-m\right)-\dfrac{1}{2}{\left(x-m-1\right)^2}+2019$.\\
		$g'(x)=f'\left(x-m\right)-\left(x-m-1\right)$.\\
		Xét phương trình $g'(x)=0. \quad \quad (1)$\\
		Đặt $ x-m=t$, phương trình $(1)$ trở thành $f'(t)-\left(t-1\right)=0\Leftrightarrow{f}'(t)=t-1. \quad (2)$\\
		Nghiệm của phương trình $(2)$ là hoành độ giao điểm của hai đồ thị hàm số $ y=f'(t)$ và $ y=t-1$.\\
		Ta có đồ thị các hàm số $ y=f'(t)$ và $ y=t-1$ như sau
		\begin{center}
			\begin{tikzpicture}[scale=0.9,font=\footnotesize, line join=round, line cap=round, >=stealth] %Đường cong bậc 3
				\draw[style=help lines,step=1] (-2.5,-3) grid (3,3.5);
				\draw[thick, ->] (-2.5,0)--(3.5,0);
				\draw[thick, ->] (0,-2.8)--(0,2.8);
				\draw (3.6,0) node[below] {$x$};
				\draw (0,3) node[above left]{$y$};
				\draw (0,0) node[below left]{$0$};
				%\draw[fill] (-1,0) circle (0.5pt)node[below]{$ -1 $};
				\draw[fill] (1,0) circle (0.5pt)node[below left]{$ 1$};
				%	\draw[fill] (3,0) circle (0.5pt)node[below right]{$ 3$};
				\draw[dashed] (-1,0)--(-1,-2) --(2,-2)--(2,0); 
				\draw[dashed](3,0)--(3,2) --(0,2);
				\draw (-1,-2) circle (2pt);
				\draw (3,2) circle (2pt);
				%			\draw[dashed](3,0)--(3,3)--(0,3);
				\draw[line width=1.2pt,smooth,samples=100,domain=-1.1:3.1] plot(\x,{1*(\x)^3-3*(\x)^2-0*(\x)+2});		
				%\draw[line width=1.2pt,smooth,samples=100,domain=-3.3:2.8] plot(\x,{0.75*(\x)^2+0.5*\x-1});
				%\draw (2.0,2.8) node[left]{$y=f'(x)$};
				\draw (-2,-3)--(4,3);
			\end{tikzpicture}
		\end{center}
		Căn cứ đồ thị các hàm số ta có phương trình $(2)$ có nghiệm là $\hoac{
			& t=-1\\ 
			& t=1\\ 
			& t=3} \Rightarrow \hoac{
			& x=m-1\\ 
			& x=m+1\\ 
			& x=m+3.}$\\
		Ta có bảng biến thiên của $ y=g(x)$
		\begin{center}
			\begin{tikzpicture}
				\tkzTabInit[lgt=1,espcl=2.5,nocadre]
				{$x$ /0.8, $y'$ /0.8,$y$ /2.5}
				{$-\infty$ , $m-1$,$m+1$,$m+3$, $+\infty$}
				\tkzTabLine{,+,0,-,0,+,0,-,}
				\tkzTabVar{-/$ +\infty$ ,+/$ $, -/$ $,+/$ $,-/$+\infty $}
			\end{tikzpicture}
		\end{center}
		Để hàm số $ y=g(x)$ đồng biến trên khoảng $\left(5;6\right)$ cần $\hoac{
			&\heva{
				& m-1\le 5\\ 
				& m+1\ge 6}\\ 
			& m+3\le 5}\Leftrightarrow\hoac{
			& 5\le m\le 6\\ 
			& m\le 2.}$\\
		Vì $ m\in\mathbb{N}^*\Rightarrow m$ nhận các giá trị $ 1;\,2;\,5;\,6\Rightarrow S=14$.}
\end{ex}

\begin{ex}%[2D1G1-3]%Câu 5
	[Sở Hà Nội-Lần 2-2020] 
	\immini{
		Cho hàm số $y=a{x^4}+b{x^3}+c{x^2}+dx+e,\,\,a\ne 0$. Hàm số $y=f'(x)$ có đồ thị như hình vẽ bên. 
		Gọi S là tập hợp tất cả các giá trị nguyên thuộc khoảng $\left(-6;6\right)$ của tham số $m$ để hàm số $g(x)=f\left(3-2x+m\right)+x^2-\left(m+3\right)x+2m^2$ nghịch biến trên $\left(0;1\right)$. Khi đó, tổng giá trị các phần tử của S là
		\choice
		{$12$}
		{\True $9$}
		{$6$}
		{$15$}}{
		\begin{tikzpicture}[scale=0.7,font=\footnotesize, line join=round, line cap=round, >=stealth] %Đường cong bậc 3
			%	\draw[style=help lines,step=1] (-2.5,-3) grid (3,3.5);
			\draw[thick, ->] (-4.5,0)--(6.5,0);
			\draw[thick, ->] (0,-2.8)--(0,2.8);
			\draw (6.6,0) node[below] {$x$};
			\draw (0,3) node[above left]{$y$};
			\draw (0,0) node[below left]{$0$};
			\draw[fill] (-2,0) circle (0.5pt)node[below]{$ -2 $};
			\draw[fill] (4,0) circle (0.5pt)node[above]{$ 4$};
			\draw[fill] (0,1) circle (0.5pt)node[right]{$ 1 $};
			\draw[fill] (0,-2) circle (0.5pt)node[left]{$ -2$};
			%	\draw[fill] (3,0) circle (0.5pt)node[below right]{$ 3$};
			\draw[dashed] (-2,0)--(-2,1) --(0,1); 
			\draw[dashed](4,0)--(4,-2) --(0,-2);
			%			\draw[dashed](3,0)--(3,3)--(0,3);
			\draw[line width=1.2pt,smooth,samples=100,domain=-3.8:5.5] plot(\x,{0.0714*(\x)^3-0.1423*(\x)^2-1.0714*(\x)});		
			%\draw[line width=1.2pt,smooth,samples=100,domain=-3.3:2.8] plot(\x,{0.75*(\x)^2+0.5*\x-1});
			%\draw (2.0,2.8) node[left]{$y=f'(x)$};
		\end{tikzpicture}	
	}
	\loigiai
	{
		Xét $g'(x)=-2f'\left(3-2x+m\right)+2x-\left(m+3\right)$.\\
		Xét phương trình $g'(x)=0$, đặt $t=3-2x+m$ thì phương trình trở thành\\ $-2\cdot \left[f'(t)-\dfrac{-t}{2}\right]=0\Leftrightarrow\hoac{
			& t=-2\\ 
			& t=4\\ 
			& t=0.}$ \\
		Từ đó, $g'(x)=0\Leftrightarrow{x_1}=\dfrac{5+m}{2},\,x_2=\dfrac{m+3}{2},x_3=\dfrac{-1+m}{2}$.\\
		Lập bảng xét dấu, đồng thời lưu ý nếu $x>x_1$ thì $t<t_1$ nên $f(x)>0$. Và các dấu đan xen nhau do các nghiệm đều làm đổi dấu đạo hàm nên suy ra $g'(x)\le 0\Leftrightarrow x\in\left[x_2;{x_1}\right]\cup\left(-\infty ;{x_3}\right]$.\\
		Vì hàm số nghịch biến trên $\left(0;1\right)$ nên \\
		$g'(x)\le 0,\,\forall x\in\left(0;1\right)$ từ đó suy ra $\hoac{
			&\dfrac{3+m}{2}\le 0<1\le\dfrac{5+m}{2}\\ 
			& 1\le\dfrac{-1+m}{2}.}$ \\
		và giải ra các giá trị nguyên thuộc $\left(-6;6\right)$ của $m$ là $-3$; $3$; $4$; $5$. }
\end{ex}

\begin{ex}%[2D1G1-3]%Câu 6
	[Chuyên Quang Trung-Bình Phước-Lần 2-2020]
	\immini{
		Cho hàm số $ y=f(x)$ có đạo hàm liên tục trên $\mathbb{R}$ và có đồ thị $ y=f'(x)$ như hình vẽ bên. Đặt $ g(x)=f\left(x-m\right)-\dfrac{1}{2}{\left(x-m-1\right)^2}+2019$, với $ m$ là tham số thực. Gọi $ S$ là tập hợp các giá trị nguyên dương của $ m$ để hàm số $ y=g(x)$ đồng biến trên khoảng $\left(5;6\right)$. Tổng tất cả các phần tử trong $ S$ bằng
		\choice
		{$ 4$}
		{$ 11$}
		{\True $ 14$}
		{$ 20$}}{
		\begin{tikzpicture}[scale=0.9,font=\footnotesize, line join=round, line cap=round, >=stealth] %Đường cong bậc 3
			\draw[thick, ->] (-2.5,0)--(3.7,0);
			\draw[thick, ->] (0,-2.8)--(0,2.8);
			\draw (3.9,0) node[below] {$x$};
			\draw (0,2.9) node[left]{$y$};
			\draw (0,0) node[below left]{$0$};
			\draw[fill] (-1,0) circle (0.5pt)node[above]{$ -1 $};
			\draw[fill] (1,0) circle (0.5pt)node[below]{$ 1$};
			\draw[fill] (3,0) circle (0.5pt)node[below]{$ 3$};
			\draw[fill] (2,0) circle (0.5pt)node[above]{$ 2$};
			\draw[fill] (0,2) circle (0.5pt)node[above left]{$ 2$};
			\draw[fill] (0,-2) circle (0.5pt)node[below left]{$ -2$};
			\draw[dashed] (-1,0)--(-1,-2)--(2,-2)--(2,0); 
			\draw[dashed](3,0)--(3,2)--(0,2);
			%			\draw[dashed](3,0)--(3,3)--(0,3);
			\draw[line width=1.2pt,smooth,samples=100,domain=-1.1:3.1] plot(\x,{1*(\x)^3-3*(\x)^2-0*(\x)+2});		
			%\draw[line width=1.2pt,smooth,samples=100,domain=-3.3:2.8] plot(\x,{0.75*(\x)^2+0.5*\x-1});
			%	\draw (2.0,2.8) node[left]{$y=f'(x)$};
	\end{tikzpicture}	}
	\loigiai
	{
		Ta có $g'(x)=f'\left(x-m\right)-\left(x-m-1\right)$.\\
		Cho $g'(x)=0\Leftrightarrow{f}'\left(x-m\right)=x-m-1$.\\
		Đặt $ x-m=t\Rightarrow f'(t)=t-1$\\
		Khi đó nghiệm của phương trình là hoành độ giao điểm của đồ thị hàm số $ y=f'(t)$ và và đường thẳng $ y=t-1$.
		\begin{center}
			\begin{tikzpicture}[scale=0.9,font=\footnotesize, line join=round, line cap=round, >=stealth] %Đường cong bậc 3
				\draw[thick, ->] (-2.5,0)--(3.7,0);
				\draw[thick, ->] (0,-2.8)--(0,2.8);
				\draw (3.9,0) node[below] {$x$};
				\draw (0,2.9) node[left]{$y$};
				\draw (0,0) node[below left]{$0$};
				\draw[fill] (-1,0) circle (0.5pt)node[above]{$ -1 $};
				\draw[fill] (1,0) circle (0.5pt)node[below]{$ 1$};
				\draw[fill] (3,0) circle (0.5pt)node[below]{$ 3$};
				\draw[fill] (2,0) circle (0.5pt)node[above]{$ 2$};
				\draw[fill] (0,2) circle (0.5pt)node[above left]{$ 2$};
				\draw[fill] (0,-2) circle (0.5pt)node[below left]{$ -2$};
				\draw[dashed] (-1,0)--(-1,-2)--(2,-2)--(2,0); 
				\draw[dashed](3,0)--(3,2)--(0,2);
				%			\draw[dashed](3,0)--(3,3)--(0,3);
				\draw[line width=1.2pt,smooth,samples=100,domain=-1.1:3.1] plot(\x,{1*(\x)^3-3*(\x)^2-0*(\x)+2});		
				%\draw[line width=1.2pt,smooth,samples=100,domain=-3.3:2.8] plot(\x,{0.75*(\x)^2+0.5*\x-1});
				%	\draw (2.0,2.8) node[left]{$y=f'(x)$};
				\coordinate (a) at ($(-1,-2)!1.2!(3,2)$);
				\coordinate (b) at ($(-1,-2)!-.2!(3,2)$);
				\draw[line width=1.2pt,smooth] (a)--(b);
			\end{tikzpicture}
		\end{center}
		Dựa vào đồ thị hàm số ta có được $f'(t)=t-1\Leftrightarrow\hoac{
			& t=-1\\ 
			& t=1\\ 
			& t=3.} $ \\
		Bảng xét dấu của $g'(t)$
		\begin{center}
			\begin{tikzpicture}
				\tkzTabInit[lgt=1.2,espcl=2.5,nocadre]
				{$t$/1, $g'(x)$ /.8} % first column
				{$-\infty$, $-1$,$1$, $3$, $+\infty$} % first row
				\tkzTabLine { ,-,0,+,0,-,0,+, } % second row
				%				\tkzTabLine {,-,z,+,t,+,} % third row
				%				\tkzTabLine {,+,d,-,z,+,} % last row
			\end{tikzpicture}
		\end{center}
		Từ bảng xét dấu ta thấy hàm số $ g(t)$ đồng biến trên khoảng $\left(-1;1\right)$ và $\left(3;+\infty\right)$.\\
		Hay $\hoac{
			&-1<t<1\\ 
			& t>3}\Leftrightarrow\hoac{
			&-1<x-m<1\\ 
			& x-m>3} \Leftrightarrow\hoac{
			& m-1<x<m+1\\ 
			& x>m+3.}$\\
		Để hàm số $ g(x)$ đồng biến trên khoảng $\left(5;6\right)$ thì $\hoac{
			& m-1\le 5<6\le m+1\\ 
			& m+3\le 5<6} \Leftrightarrow\hoac{
			& 5\le m\le 6\\ 
			& m\le 2.}$\\
		Vì $ m$ là các số nguyên dương nên $ S=\left\{ 1;2;5;6\right\}$.\\
		Vậy tổng tất cả các phần tử của $ S$ là $ 1+2+5+6=14$.}
\end{ex}

\begin{ex}%[2D1G1-3]%Câu 7
	\immini{
		Cho hàm số $ y=f(x)$ liên tục có đạo hàm trên $\mathbb{R}$. Biết hàm số $ f'(x)$ có đồ thị cho như hình vẽ bên. Có bao nhiêu giá trị nguyên của $ m$ thuộc $\left[-2019;2019\right]$ để hàm só $ g(x)=f\left(2019^x\right)-mx+2$ đồng biến trên $\left[0;1\right]$.
		\choice
		{$ 2028$}
		{$ 2019$}
		{$ 2011$}
		{\True $ 2020$}}{
		\begin{tikzpicture}[scale=0.9,font=\footnotesize, line join=round, line cap=round, >=stealth] %Đường cong bậc 3
			\draw[thick, ->] (-3.5,0)--(2.5,0);
			\draw[thick, ->] (0,-2.8)--(0,2.8);
			\draw (2.7,0) node[below] {$x$};
			\draw (0,2.9) node[left]{$y$};
			\draw (0,0) node[below left]{$0$};
			%	\draw[fill] (-1,0) circle (0.5pt)node[above]{$ -1 $};
			\draw[fill] (1,0) circle (0.5pt)node[below right]{$ 1$};
			%		\draw[fill] (3,0) circle (0.5pt)node[below]{$ 3$};
			%		\draw[fill] (2,0) circle (0.5pt)node[above]{$ 2$};
			%		\draw[fill] (0,2) circle (0.5pt)node[above left]{$ 2$};
			%		\draw[fill] (0,-2) circle (0.5pt)node[below left]{$ -2$};
			%		\draw[dashed] (-1,0)--(-1,-2)--(2,-2)--(2,0); 
			%		\draw[dashed](3,0)--(3,2)--(0,2);
			\draw[line width=1.2pt,smooth,samples=100,domain=-3.28:1.32] plot(\x,{0.667*(\x)^3+2*(\x)^2-0.667*(\x)-2});		
			%\draw[line width=1.2pt,smooth,samples=100,domain=-3.3:2.8] plot(\x,{0.75*(\x)^2+0.5*\x-1});
			%	\draw (2.0,2.8) node[left]{$y=f'(x)$};
	\end{tikzpicture}	}
	\loigiai{
		Ta có $ g'(x)=2019^x\ln 2019\cdot f'\left(2019^x\right)-m$.\\
		Ta lại có hàm số $ y=2019^x$ đồng biến trên $\left[0;1\right]$.\\
		Với $ x\in\left[0;1\right]$ thì $2019^x\in\left[1;2019\right]$ mà hàm $ y=f'(x)$ đồng biến trên $\left(1;+\infty\right)$ nên hàm $ y=f'\left(2019^x\right)$ đồng biến trên $\left[0;1\right]$.\\
		Mà $2019^x\ge 1;f'\left(2019^x\right)>0\,\forall\,x\in\left[0;1\right]$ nên hàm $ h(x)=2019^x\ln 2019\cdot f'\left(2019^x\right)$ đồng biến trên $\left[0;1\right]$.\\
		Hay $ h(x)\ge h(0)=0,\forall\,x\in\left[0;1\right]$.\\
		Do vậy hàm số $ g(x)$ đồng biến trên đoạn $\left[0;1\right]$$\Leftrightarrow g'(x)\ge 0,\forall\,x\in\left[0;1\right]$\\
		$\Leftrightarrow m\le{2019^x}\ln 2019.f'\left(2019^x\right),\forall\,x\in\left[0;1\right]$ $\Leftrightarrow m\le\underset{x\in\left[0;1\right]}{\min}\,h(x)=h(0)=0$\\
		Vì $ m$ nguyên và $ m\in\left[-2019;2019\right]\Rightarrow $có $ 2020$ giá trị $ m$ thỏa mãn yêu cầu bài toán.}
\end{ex}

\begin{ex}%[2D1G1-3]%Câu 8
	\immini{
		Cho hàm số $y=f(x)$ có đồ thị $f'(x)\,$ như hình vẽ. Có bao nhiêu giá trị nguyên $m\in\left(-2020\,;\,2020\right)$ để hàm số $g(x)=f\left(2x-3\right)\,-\ln \left(1+x^2\right)-2mx$ đồng biến trên $\left(\dfrac{1}{2};2\right)$?
		\choice
		{$ 2020$}
		{\True $ 2019$}
		{$ 2021$}
		{$ 2018$}}{
		\begin{tikzpicture}[scale=0.9,font=\footnotesize, line join=round, line cap=round, >=stealth] %Đường cong bậc 3
			\draw[thick, ->] (-2.5,0)--(2.5,0);
			\draw[thick, ->] (0,-1.8)--(0,5.8);
			\draw (2.7,0) node[below] {$x$};
			\draw (0,5.9) node[left]{$y$};
			\draw (0,0) node[below left]{$0$};
			\draw[fill] (-2,0) circle (0.5pt)node[below]{$ -2 $};
			\draw[fill] (1,0) circle (0.5pt)node[below]{$ 1$};
			\draw[fill] (-1,0) circle (0.5pt)node[below]{$-1$};
			\draw[fill] (0,4) circle (0.5pt)node[above left]{$ 2$};
			%		\draw[fill] (0,2) circle (0.5pt)node[above left]{$ 2$};
			%		\draw[fill] (0,-2) circle (0.5pt)node[below left]{$ -2$};
			\draw[dashed] (-2,0)--(-2,4)--(1,4)--(1,0); 
			%		\draw[dashed](3,0)--(3,2)--(0,2);
			\draw[line width=1.2pt,smooth,samples=100,domain=-2.1:2.1] plot(\x,{-1*(\x)^3+0*(\x)^2+3*(\x)+2});		
			%\draw[line width=1.2pt,smooth,samples=100,domain=-3.3:2.8] plot(\x,{0.75*(\x)^2+0.5*\x-1});
			%	\draw (2.0,2.8) node[left]{$y=f'(x)$};
	\end{tikzpicture}	}
	\loigiai{
		Ta có $g'(x)=2f'\left(2x-3\right)-\dfrac{2x}{1+x^2}-2m$.\\
		Hàm số $ g(x)$ đồng biến trên $\left(\dfrac{1}{2};2\right)$ khi và chỉ khi \\
		$g'(x)\ge 0,\,\,\forall x\in\left(-1;\,2\right)$\\
		$\Leftrightarrow m\le{f}'\left(2x-3\right)-\dfrac{x}{1+x^2},\,\,\forall x\in\left(\dfrac{1}{2};2\right)$\\
		$\Leftrightarrow m\le\underset{x\in\left[\dfrac{1}{2};2\right]}{\min}\,\left[f'\left(2x-3\right)-\dfrac{x}{1+x^2}\right]$. \, \,  $(1)$\\
		Đặt $ t=2x-3$, khi đó $ x\in\left(\dfrac{1}{2};2\right)\Leftrightarrow t\in\left(-2;\,1\right)$.\\
		Từ đồ thị hàm $f'(x)$ suy ra $f'(t)\ge 0,\,\,\forall t\in\left(-2;1\right)$ và $f'(t)=0$ khi $ t=-1$.\\
		Tức là $f'\left(2x-3\right)\ge 0,\,\,\forall x\in\left(\dfrac{1}{2};\,2\right)$$\Rightarrow\underset{x\in\left[\dfrac{1}{2};2\right]}{\min}\,f'\left(2x-3\right)=0$ khi $ x=1$. $(2)$\\
		Xét hàm số $ h(x)=-\dfrac{x}{1+x^2}$ trên khoảng $\left(\dfrac{1}{2};\,2\right)$.\\
		Ta có $h'(x)=\dfrac{x^2-1}{\left(1+x^2\right)^2}$ và\\
		$h'(x)=0\Leftrightarrow{x^2}-1=0\Leftrightarrow x=\pm 1$.\\
		Bảng biến thiên của hàm số $ h(x)$ trên $\left(\dfrac{1}{2};\,2\right)$ như sau
		\begin{center}
			\begin{tikzpicture}
				\tkzTabInit[lgt=1.2,espcl=2.5,nocadre]
				{$x$ /0.7, $h'(x)$ /0.7,$h(x)$ /2.5}
				{$\dfrac{1}{2}$ , $1$,$2$}
				\tkzTabLine{,-,0,+,}
				\tkzTabVar{+/$  $ ,-/$ \-\dfrac{1}{2} $, +/$ $}
			\end{tikzpicture}
		\end{center}
		Từ bảng biến thiên suy ra $ h(x)\ge-\dfrac{1}{2}$$\Rightarrow\underset{x\in\left[\dfrac{1}{2};2\right]}{\min}\,h(x)=-\dfrac{1}{2}$ khi $ x=1$. \, \,  $(3)$\\
		Từ $(1)$, $(2)$ và $(3)$ suy ra $ m\le-\dfrac{1}{2}$.\\
		Kết hợp với $ m\in\mathbb{Z}$, $ m\in\left(-2020;\,2020\right)$ thì $ m\in\left\{-2019;\,-201;\ldots ;-2;-1\right\}$.\\
		Vậy có tất cả $ 2019$ giá trị $ m$ cần tìm.}
\end{ex}

\begin{ex}%[2D1G1-3]%Câu 9
	Cho hàm số $ f(x)$ liên tục trên $\mathbb{R}$ và có đạo hàm $f'(x)=x^2\left(x-2\right)\left(x^2-6x+m\right)$ với mọi $ x\in\mathbb{R}$. Có bao nhiêu số nguyên $ m$ thuộc đoạn $\left[-2020;2020\right]$ để hàm số $ g(x)=f\left(1-x\right)$ nghịch biến trên khoảng $\left(-\infty ;-1\right)$?
	\choice
	{$ 2016$}
	{$ 2014$}
	{\True $ 2012$}
	{$ 2010$}
	\loigiai{
		Ta có \\
		$g'(x)=f'\left(1-x\right)=-\left(1-x\right)^2\left(-x-1\right)\left[\left(1-x\right)^2-6\left(1-x\right)+m\right]$
		$=\left(x-1\right)^2\left(x+1\right)\left(x^2+4x+m-5\right)$.\\
		Hàm số $ g(x)$ nghịch biến trên khoảng $\left(-\infty ;-1\right)$\\
		$\Leftrightarrow{g}'(x)\le 0,\forall x<-1$ $(*)$, (dấu \lq\lq $=$\rq\rq \, xảy ra tại hữu hạn điểm).\\
		Với $ x<-1$ thì $\left(x-1\right)^2>0$ và $ x+1<0$ nên\\
		$(*)$ $\Leftrightarrow{x^2}+4x+m-5\ge 0,\forall x<-1 \Leftrightarrow m\ge-x^2-4x+5,\forall x<-1$.\\
		Xét hàm số $ y=-x^2-4x+5$ trên khoảng $\left(-\infty ;-1\right)$, ta có bảng biến thiên
		\begin{center}
			\begin{tikzpicture}
				\tkzTabInit[lgt=1.8,espcl=2.3]
				{$x$ /1.2, $y'$ /1.2,$y$ /2}
				{$-\infty$ , $-2$,$-1$}
				\tkzTabLine{,+,0,-,}
				\tkzTabVar{-/$ -\infty $ ,+/$9 $, -/$ 8$}
			\end{tikzpicture}
		\end{center}
		Từ bảng biến thiên suy ra $ m\ge 9$.\\
		Kết hợp với $ m$ thuộc đoạn $\left[-2020;2020\right]$ và $ m$ nguyên nên $ m\in\left\{ 9;10;11;\ldots ;2020\right\}$.\\
		Vậy có $ 2012$ số nguyên $ m$ thỏa mãn đề bài.}
\end{ex}

\begin{ex}%[2D1G1-3]%Câu 10
	\immini{
		Cho hàm số $f(x)$ xác định và liên tục trên $ R$. Hàm số $y=f'(x)$ liên tục trên $\mathbb{R}$ và có đồ thị như hình vẽ bên.
		Xét hàm số $g(x)=f\left(x-2m\right)+\dfrac{1}{2}{\left(2m-x\right)^2}+2020$, với $ m$ là tham số thực. Gọi $ S$ là tập hợp các giá trị nguyên dương của $ m$ để hàm số $ y=g(x)$ nghịch biến trên khoảng $\left(3;4\right)$. Hỏi số phần tử của $ S$ bằng bao nhiêu?
		\choice
		{$4$}
		{\True $2$}
		{$3$}
		{Vô số}}
	{
		\begin{tikzpicture}[scale=0.7,>=stealth, font=\footnotesize, line join=round, line cap=round]
			\def\xmin{-3.5} \def\xmax{4.5}
			\def\ymin{-5.2} \def\ymax{4}
			\clip(\xmin,\ymin) rectangle (\xmax,\ymax);
			\draw[->] (\xmin,0)--(\xmax,0) node [below]{$x$};
			\draw[->] (0,\ymin)--(0,\ymax) node [left]{$y$};
			\node at (0,0) [below left]{$O$};
			\path
			(-3.1,3.7) coordinate (A)
			(-3,3) coordinate (B)
			(0,-2) coordinate (C)
			(0.65,-2) coordinate (D)
			(1,-1) coordinate (E)
			(3,-3) coordinate (F)
			(3.4,-5) coordinate (G);
			\draw[smooth]
			(A)..controls +(-88:0.1) and +(93:.1)..
			(B)..controls +(-87:0.3) and +(-100:8.5)..
			(C)..controls +(75:.8) and +(180:.1)..
			(D)..controls +(0:.1) and +(-105:.3)..
			(E)..controls +(70:2) and +(97:0.4)..
			(F)..controls +(-80:.1) and +(90:0.3)..
			(G);
			\draw[dashed] 
			(-3,0)node[below]{$-3$}|-(0,3)node[right]{$3$}
			(1,0)node[above]{$1$}|-(0,-1)node[left]{$-1$}
			(3,0)node[above]{$3$}|-(0,-3)node[below right]{$-3$};
			\fill 
			(0,-2) circle(1.5pt)
			(-3,3) circle(1.5pt)
			(3,-3) circle(1.5pt)
			(1,-1) circle(1.5pt);
			\node at (2.1,-4) {$y=f'(x)$};
		\end{tikzpicture}
	}
	\loigiai{
		Ta có $g'(x)=f'\left(x-2m\right)-\left(2m-x\right)$.		Đặt $h(x)=f'(x)-\left(-x\right)$.\\
		Từ đồ thị hàm số $y=f'(x)$ và đồ thị hàm số $y=-x$ trên hình vẽ suy ra \\
		$h(x)\le 0\Leftrightarrow f'(x)\le-x\Leftrightarrow\hoac{
			&-3\le x\le 1\\ 
			& x\ge 3.}$ 
		\begin{center}
			\begin{tikzpicture}[scale=0.7,>=stealth, font=\footnotesize, line join=round, line cap=round]
				\def\xmin{-3.5} \def\xmax{4.5}
				\def\ymin{-5.2} \def\ymax{4}
				\clip(\xmin,\ymin) rectangle (\xmax,\ymax);
				\draw[->] (\xmin,0)--(\xmax,0) node [below]{$x$};
				\draw[->] (0,\ymin)--(0,\ymax) node [left]{$y$};
				\node at (0,0) [below left]{$O$};
				\path
				(-3.1,3.7) coordinate (A)
				(-3,3) coordinate (B)
				(0,-2) coordinate (C)
				(0.65,-2) coordinate (D)
				(1,-1) coordinate (E)
				(3,-3) coordinate (F)
				(3.4,-5) coordinate (G);
				\draw[smooth]
				(A)..controls +(-88:0.1) and +(93:.1)..
				(B)..controls +(-87:0.3) and +(-100:8.5)..
				(C)..controls +(75:.8) and +(180:.1)..
				(D)..controls +(0:.1) and +(-105:.3)..
				(E)..controls +(70:2) and +(97:0.4)..
				(F)..controls +(-80:.1) and +(90:0.3)..
				(G);
				\draw[dashed] 
				(-3,0)node[below]{$-3$}|-(0,3)node[right]{$3$}
				(1,0)node[above]{$1$}|-(0,-1)node[left]{$-1$}
				(3,0)node[above]{$3$}|-(0,-3)node[below right]{$-3$};
				\fill 
				(0,-2) circle(1.5pt)
				(-3,3) circle(1.5pt)
				(3,-3) circle(1.5pt)
				(1,-1) circle(1.5pt);
				\draw[smooth,samples=300,domain=-3.2:3.7] plot(\x,{-(\x)});
				\node at (2.1,-4) {$y=f'(x)$};
				\node at (-1,2.1) {$y=h(x)$};
			\end{tikzpicture}
		\end{center}
		Ta có $ g'(x)=h\left(x-2m\right)\le 0\Leftrightarrow\hoac{
			&-3\le x-2m\le 1\\ 
			& x-2m\ge 3}\Leftrightarrow\hoac{
			& 2m-3\le x\le 2m+1\\ 
			& x\ge 2m+3.}$.\\
		Suy ra hàm số $ y=g(x)$ nghịch biến trên các khoảng $\left(2m-3;2m+1\right)$ và $\left(2m+3;+\infty\right)$.\\
		Do đó hàm số $ y=g(x)$ nghịch biến trên khoảng $\left(3;4\right)$ $\Leftrightarrow\hoac{
			&\heva{
				& 2m-3\le 3\\ 
				& 2m+1\ge 4}\\ 
			& 2m+3\le 3}\Leftrightarrow\hoac{
			&\dfrac{3}{2}\le m\le 3\\ 
			& m\le 0.}$ \\
		Mặt khác, do $ m$ nguyên dương nên $ m\in\left\{ 2;3\right\}\Rightarrow S=\left\{ 2;3\right\}$. Vậy số phần tử của $ S$ bằng $2$.\\
	}
	
\end{ex}

\begin{ex}%[2D1G1-3]%Câu 11
	Cho hàm số $f(x)$ có đạo hàm trên $\mathbb{R}$ là $f'(x)=\left(x-1\right)\left(x+3\right)$. Có bao nhiêu giá trị nguyên của tham số $m$ thuộc đoạn $\left[-10;20\right]$ để hàm số $y=f\left(x^2+3x-m\right)$ đồng biến trên khoảng $\left(0;2\right)$?
	\choice
	{\True $ 18$}
	{$ 17$}
	{$ 16$}
	{$ 20$}
	\loigiai{
		Ta có $y'=f'\left(x^2+3x-m\right)=\left(2x+3\right){f}'\left(x^2+3x-m\right)$.\\
		Theo đề bài ta có $f'(x)=\left(x-1\right)\left(x+3\right)$\\
		suy ra $f'(x)>0\Leftrightarrow\hoac{
			& x<-3\\ 
			& x>1}$ và $f'(x)<0\Leftrightarrow-3<x<1$ .\\
		Hàm số đồng biến trên khoảng $\left(0;2\right)$ khi $y'\ge 0,\forall x\in\left(0;2\right)$\\
		$\Leftrightarrow\left(2x+3\right){f}'\left(x^2+3x-m\right)\ge 0,\forall x\in\left(0;2\right)$.\\
		Do $x\in\left(0;2\right)$ nên $2x+3>0,\forall x\in\left(0;2\right)$. Do đó, ta có\\
		$y'\ge 0,\forall x\in\left(0;2\right)\Leftrightarrow f'\left(x^2+3x-m\right)\ge 0$\\
		$\Leftrightarrow\hoac{
			&{x^2}+3x-m\le-3\\ 
			&{x^2}+3x-m\ge 1}\Leftrightarrow\hoac{
			& m\ge{x^2}+3x+3\\ 
			& m\le{x^2}+3x-1}$\\
		$\Leftrightarrow\hoac{
			& m\ge\underset{\left[0;2\right]}{\max}\,\left(x^2+3x+3\right)\\ 
			& m\le\underset{\left[0;2\right]}{\min}\,\left(x^2+3x-1\right)} \Leftrightarrow\hoac{
			& m\ge 13\\ 
			& m\le-1}$.\\
		Do $m\in\left[-10;20\right]$, $ m\in\mathbb{Z}$ nên có $ 18$ giá trị nguyên của $m$ thỏa yêu cầu đề bài.}
\end{ex}

\begin{ex}%[2D1G1-3]%Câu 12
	Cho các hàm số $f(x)=x^3+4x+m$ và $g(x)=\left(x^2+2018\right){\left(x^2+2019\right)^2}{\left(x^2+2020\right)^3}$ . Có bao nhiêu giá trị nguyên của tham số $m\in\left[-2020;2020\right]$ để hàm số $g\left(f(x)\right)$ đồng biến trên $\left(2;+\infty\right)$ ?
	\choice
	{$2005$}
	{\True $2037$}
	{$4016$}
	{$4041$}
	\loigiai{
		Ta có $f(x)=x^3+4x+m$ và \\
		$g(x)=\left(x^2+2018\right){\left(x^2+2019\right)^2}{\left(x^2+2020\right)^3}=a_{12}{x^{12}}+a_{10}{x^{10}}+...+a_2x^2+a_0$.\\
		Suy ra $f'(x)=3x^2+4$ , $g'(x)=12a_{12}{x^{11}}+10a_{10}{x^9}+...+2a_2x$.\\
		Và có 
		\begin{eqnarray*}
			\left[g\left(f(x)\right)\right]' &=& f'(x)\left[12a_{12}{\left(f(x)\right)^{11}}+10a_{10}{\left(f(x)\right)^9}+...+2a_2f(x)\right]\\
			&=& f(x)f'(x)\left(12a_{12}{\left(f(x)\right)^{10}}+10a_{10}{\left(f(x)\right)^8}+...+2a_2\right).
		\end{eqnarray*} 
		Dễ thấy $a_{12};{a_{10}};...;{a_2};{a_0}>0$ và $f'(x)=3x^2+4>0$, $\forall x>2$.\\
		Do đó $f'(x)\left(12a_{12}{\left(f(x)\right)^{10}}+10a_{10}{\left(f(x)\right)^8}+...+2a_2\right)>0$ , $\forall x>2$.\\
		Hàm số $g\left(f(x)\right)$ đồng biến trên $\left(2;+\infty\right)$ khi $\left[g\left(f(x)\right)\right]^{'}\ge 0$, $\forall x>2$\\
		$\Rightarrow  f(x)\ge 0$, $\forall x>2 \Leftrightarrow x^3+4x+m\ge 0$, $\forall x>3 \Leftrightarrow  m\ge-x^3-4x$, $\forall x>2$\\
		$ \Rightarrow  m\ge\underset{\left[2;+\infty\right)}{\max}\,\left(-x^3-4x\right)=-16$.\\
		Vì $m\in\left[-2020;2020\right]$ và $m\in\mathbb{Z}$ nên có $2037$ giá trị thỏa mãn $m$ .}
\end{ex}

\begin{ex}%[2D1G1-3]%Câu 13
	Cho hàm số $y=f(x)$ có đạo hàm $f'(x)=x{\left(x+1\right)^2}\left(x^2+2mx+1\right)$ với mọi $x \in \mathbb{R}$. Có bao nhiêu số nguyên âm $m$ để hàm số $g(x)=f\left(2x+1\right)$ đồng biến trên khoảng $\left(3;5\right)$?
	\choice
	{\True $3$}
	{$2$}
	{$4$}
	{$6$}
	\loigiai{
		Ta có $g'(x)=2f'(2x+1)=2(2x+1)(2x+2)^2[(2x+1)^2+2m(2x+1)+1]$. 	Đặt $t=2x+1$\\
		Để hàm số $g(x)$ đồng biến trên khoảng $\left(3;5\right)$ khi và chỉ khi 
		\begin{eqnarray*}
			& & g'(x)\ge 0,\forall x\in\left(3;5\right) \\
			& \Leftrightarrow & t(t^2+2mt+1)\ge 0,\forall t\in\left(7;11\right)\Leftrightarrow{t^2}+2mt+1\ge 0,\,\,\forall t\in\left(7;11\right) \\
			&\Leftrightarrow & 2m\ge\dfrac{-t^2-1}{t},\,\,\,\forall t\in\left(7;11\right)
		\end{eqnarray*}	
		Xét hàm số $h(t)=\dfrac{-t^2-1}{t}$ trên $\left[7;11\right]$, có $h'(t)=\dfrac{-t^2+1}{t^2}$\\
		Bảng biến thiên
		\begin{center}
			\begin{tikzpicture}
				\tkzTabInit[espcl=3,lgt=1.2,nocadre]
				{$t$/0.7,$h'(t)$/0.7,$h(t)$/2.5}
				{$-\infty$,$1$,$11$,$+\infty$}
				\tkzTabLine{, ,,-,,,}
				%	\node (0) at ($(N12)+(0,-3)$) {$-\infty$};
				\node (1) at ($(N22)+(0,-0.8)$) [right] {$-\dfrac{50}{7}$};
				\node (2) at ($(N32)+(0,-2.5)$) [left] {$-\dfrac{122}{11}$};
				
				
				%				\node (3) at ($(N11+(-0.5,0))$) {};
				%				\node (4) at ($(N23)$) {};
				\fill[pattern=north east lines] (7.0,-0.7) rectangle (10,-4.4);
				\fill[pattern=north east lines] (1.5,-0.7) rectangle (4.5,-4.4);
				\draw[->] (1)--(2);	
				\draw[dashed] (4.5,-0.7)--(4.5,-4.4);
				\draw[dashed] (7.0,-0.7)--(7.0,-4.4);	
			\end{tikzpicture}		
		\end{center}
		Dựa vào BBT ta có $2m\ge\dfrac{-t^2-1}{t},\,\,\,\forall t\in\left(7;11\right)\Leftrightarrow 2m\ge\underset{\left[7;11\right]}{\max}\,h(t)\Leftrightarrow m\ge-\dfrac{50}{14}$\\
		Vì $ m\in{\mathbb{Z}^-}\Rightarrow m \in \{-3;-2;-1\}$ .
	}
\end{ex}

\begin{ex}%[2D1G1-3]%Câu 14
	Cho hàm số $y=f(x)$ có bảng biến thiên như sau\\
	\begin{center}
		\begin{tikzpicture}[>=stealth,scale = 1]
			\tkzTabInit[lgt=1,espcl=2.5,nocadre]
			{$x$ /0.7, $y'$ /0.7,$y$ /2.5}
			{$-\infty$,$0$,$2$,$+\infty$}
			\tkzTabLine{ ,-,0,+,0,-,}
			\tkzTabVar{-/$-\infty$, +/$4$,- /$0$, +/{ $+\infty$}}
		\end{tikzpicture}
	\end{center}
	Có bao nhiêu số nguyên $m<2019$ để hàm số $g(x)=f\left(x^2-2x+m\right)$ đồng biến trên khoảng $\left(1;+\infty\right)$?
	\choice
	{\True $2016$}
	{$2015$}
	{$2017$}
	{$2018$}
	\loigiai{
		Ta có $g'(x)=\left(x^2-2x+m\right)'{f}'\left(x^2-2x+m\right)=2\left(x-1\right){f}'\left(x^2-2x+m\right)$ .\\
		Hàm số $y=g(x)$ đồng biến trên khoảng $\left(1;+\infty\right)$ khi và chỉ khi $g'(x)\ge 0,\forall x\in\left(1;+\infty\right)$ và\\
		$g'(x)=0$ tại hữu hạn điểm \\
		$\Leftrightarrow 2\left(x-1\right){f}'\left(x^2-2x+m\right)\ge 0,\forall x\in\left(1;+\infty\right)$\\
		$\Leftrightarrow{f}'\left(x^2-2x+m\right)\ge 0,\forall x\in\left(1;+\infty\right)$ $\Leftrightarrow\hoac{
			&{x^2}-2x+m\ge 2,\forall x\in\left(1;+\infty\right)\\ 
			&{x^2}-2x+m\le 0,\forall x\in\left(1;+\infty\right).}$\\
		Xét hàm số $y=x^2-2x+m$, ta có bảng biến thiên
		\begin{center}
			\begin{tikzpicture}[>=stealth,scale = 1]
				\tkzTabInit[lgt=1,espcl=2.5,nocadre]
				{$x$ /0.7, $y'$ /0.7,$y$ /2.5}
				{$-\infty$,$2$,$+\infty$}
				\tkzTabLine{ ,-,0,+,}
				\tkzTabVar{+/$+\infty$, -/$m-1$, +/{$+\infty$}}
			\end{tikzpicture}
		\end{center}
		Dựa vào bảng biến thiên ta có\\
		TH1: $x^2-2x+m\ge 2,\forall x\in\left(1;+\infty\right)\Leftrightarrow m-1\ge 2\Leftrightarrow m\ge 3$ .\\
		TH2: $x^2-2x+m\le 0,\forall x\in\left(1;+\infty\right)$. Không có giá trị $m$ thỏa mãn.\\
		Vậy có $2016$ số nguyên $m<2019$ thỏa mãn yêu cầu bài toán.}
\end{ex}

\begin{ex}%[2D1G1-3]%Câu 15
	\immini{
		Cho hàm số $ y=f(x)$ có đạo hàm là hàm số $f'(x)$ trên $\mathbb{R}$. Biết rằng hàm số $ y=f'\left(x-2\right)+2$ có đồ thị như hình vẽ bên dưới. Hàm số $ f(x)$ đồng biến trên khoảng nào?
		\choice
		{$\left(-\infty ;3\right),\,\,\left(5;+\infty\right)$}
		{\True $\left(-\infty ;-1\right),\,\,\left(1;+\infty\right)$}
		{$\left(-1;1\right)$}
		{$\left(3;5\right)$}}{
		\begin{tikzpicture}[scale=0.7,font=\footnotesize, line join=round, line cap=round, >=stealth] %Đường cong bậc 3
			\draw[thick, ->] (-0.5,0)--(3.5,0);
			\draw[thick, ->] (0,-1.8)--(0,5.3);
			\draw (3.7,0) node[below] {$x$};
			\draw (0,5.4) node[left]{$y$};
			\draw (0,0) node[below left]{$0$};
			\draw[fill] (3,0) circle (0.5pt)node[below]{$ 3$};
			\draw[fill] (1,0) circle (0.5pt)node[below]{$ 1$};
			\draw[fill] (2,0) circle (0.5pt)node[above]{$2$};
			\draw[fill] (0,2) circle (0.5pt)node[left]{$ 2$};
			\draw[fill] (0,-1) circle (0.5pt)node[left]{$ -1$};
			%		\draw[fill] (0,2) circle (0.5pt)node[above left]{$ 2$};
			%		\draw[fill] (0,-2) circle (0.5pt)node[below left]{$ -2$};
			\draw[dashed] (3,0)--(3,2)--(0,2)--(1,2)--(1,0); 
			\draw[dashed](0,-1)--(2,-1)--(2,0);
			\draw[line width=1.2pt,smooth,samples=100,domain=0.6:3.4] plot(\x,{3*(\x)^2-12*(\x)+11});		
			%\draw[line width=1.2pt,smooth,samples=100,domain=-3.3:2.8] plot(\x,{0.75*(\x)^2+0.5*\x-1});
			%	\draw (2.0,2.8) node[left]{$y=f'(x)$};
	\end{tikzpicture}	}
	\loigiai{	
		Hàm số $ y=f'\left(x-2\right)+2$ có đồ thị $(C)$ như sau:\\
		\begin{center}
			\begin{tikzpicture}[scale=0.7,font=\footnotesize, line join=round, line cap=round, >=stealth] %Đường cong bậc 3
				\draw[thick, ->] (-0.5,0)--(3.5,0);
				\draw[thick, ->] (0,-1.8)--(0,5.3);
				\draw (3.7,0) node[below] {$x$};
				\draw (0,5.4) node[left]{$y$};
				\draw (0,0) node[below left]{$0$};
				\draw[fill] (3,0) circle (0.5pt)node[below]{$ 3$};
				\draw[fill] (1,0) circle (0.5pt)node[below]{$ 1$};
				\draw[fill] (2,0) circle (0.5pt)node[above]{$2$};
				\draw[fill] (0,2) circle (0.5pt)node[left]{$ 2$};
				\draw[fill] (0,-1) circle (0.5pt)node[left]{$ -1$};
				%		\draw[fill] (0,2) circle (0.5pt)node[above left]{$ 2$};
				%		\draw[fill] (0,-2) circle (0.5pt)node[below left]{$ -2$};
				\draw[dashed] (3,0)--(3,2)--(0,2)--(1,2)--(1,0); 
				\draw[dashed](0,-1)--(2,-1)--(2,0);
				\draw[line width=1.2pt,smooth,samples=100,domain=0.6:3.4] plot(\x,{3*(\x)^2-12*(\x)+11});		
				%\draw[line width=1.2pt,smooth,samples=100,domain=-3.3:2.8] plot(\x,{0.75*(\x)^2+0.5*\x-1});
				%	\draw (2.0,2.8) node[left]{$y=f'(x)$};
			\end{tikzpicture}
		\end{center}
		Dựa vào đồ thị $(C)$ ta có\\ $f'\left(x-2\right)+2>2,\forall x\in\left(-\infty ;1\right)\cup\left(3;+\infty\right)\Leftrightarrow{f}'\left(x-2\right)>0,\forall x\in\left(-\infty ;1\right)\cup\left(3;+\infty\right)$ .\\
		Đặt $ x*=x-2$ suy ra $f'\left(x*\right)>0,\forall x*\in\left(-\infty ;-1\right)\bigcup\left(1;+\infty\right)$.\\
		Vậy hàm số $ f(x)$ đồng biến trên khoảng $\left(-\infty ;-1\right),\,\,\left(1;+\infty\right)$.}
\end{ex}

\begin{ex}%[2D1G1-2]%Câu 16
	\immini{
		Cho hàm số $ y=f(x)$ có đạo hàm là hàm số $f'(x)$ trên $\mathbb{R}$. Biết rằng hàm số $ y=f'\left(x+2\right)-2$ có đồ thị như hình vẽ bên dưới. Hàm số $ f(x)$ nghịch biến trên khoảng nào?
		\choice
		{$\left(-3;-1\right),\,\,\left(1;3\right)$}
		{\True $\left(-1;1\right),\,\,\left(3;5\right)$}
		{$\left(-\infty ;-2\right),\,\,\left(0;2\right)$}
		{$\left(-5;-3\right),\,\,\left(-1;1\right)$}}{
		\begin{tikzpicture}[scale=0.7,font=\footnotesize, line join=round, line cap=round, >=stealth] %Đường cong bậc 3
			\draw[thick, ->] (-3.8,0)--(4.0,0);
			\draw[thick, ->] (0,-4.8)--(0,3.5);
			\draw (4.2,0) node[below] {$x$};
			\draw (0,3.7) node[left]{$y$};
			\draw (0,0) node[below left]{$0$};
			\draw[fill] (-3,0) circle (0.5pt)node[above]{$ -3$};
			\draw[fill] (-1,0) circle (0.5pt)node[above]{$ -1$};
			\draw[fill] (1,0) circle (0.5pt)node[above]{$ 1$};
			\draw[fill] (3,0) circle (0.5pt)node[above]{$3$};
			\draw[fill] (0,2) circle (0.5pt)node[above left]{$ 2$};
			\draw[fill] (0,-1) circle (0.5pt)node[above right]{$ -1$};
			%		\draw[fill] (0,2) circle (0.5pt)node[above left]{$ 2$};
			%		\draw[fill] (0,-2) circle (0.5pt)node[below left]{$ -2$};
			\draw[dashed] (-3,0)--(-3,-2)--(3,-2)--(3,0) (-1,0)--(-1,-2) (1,0)--(1,-2) (-3.494,0)--(-3.494,2)--(3.494,2)--(3.494,0); 
			\draw[line width=1.2pt,smooth,samples=100,domain=-3.6:3.6] plot(\x,{0.11*(\x)^4-1.11*(\x)^2-1});		
			%\draw[line width=1.2pt,smooth,samples=100,domain=-3.3:2.8] plot(\x,{0.75*(\x)^2+0.5*\x-1});
			%	\draw (2.0,2.8) node[left]{$y=f'(x)$};
	\end{tikzpicture}	}
	\loigiai{
		Hàm số $ y=f'\left(x+2\right)-2$ có đồ thị $(C)$ như sau
		\begin{center}
			\begin{tikzpicture}[scale=0.7,font=\footnotesize, line join=round, line cap=round, >=stealth] %Đường cong bậc 3
				\draw[thick, ->] (-3.8,0)--(4.0,0);
				\draw[thick, ->] (0,-4.8)--(0,3.5);
				\draw (4.2,0) node[below] {$x$};
				\draw (0,3.7) node[left]{$y$};
				\draw (0,0) node[below left]{$0$};
				\draw[fill] (-3,0) circle (0.5pt)node[above]{$ -3$};
				\draw[fill] (-1,0) circle (0.5pt)node[above]{$ -1$};
				\draw[fill] (1,0) circle (0.5pt)node[above]{$ 1$};
				\draw[fill] (3,0) circle (0.5pt)node[above]{$3$};
				\draw[fill] (0,2) circle (0.5pt)node[above left]{$ 2$};
				\draw[fill] (0,-1) circle (0.5pt)node[above right]{$ -1$};
				%		\draw[fill] (0,2) circle (0.5pt)node[above left]{$ 2$};
				%		\draw[fill] (0,-2) circle (0.5pt)node[below left]{$ -2$};
				\draw[dashed] (-3,0)--(-3,-2)--(3,-2)--(3,0) (-1,0)--(-1,-2) (1,0)--(1,-2) (-3.494,0)--(-3.494,2)--(3.494,2)--(3.494,0); 
				\draw[line width=1.2pt,smooth,samples=100,domain=-3.6:3.6] plot(\x,{0.11*(\x)^4-1.11*(\x)^2-1});		
				%\draw[line width=1.2pt,smooth,samples=100,domain=-3.3:2.8] plot(\x,{0.75*(\x)^2+0.5*\x-1});
				%	\draw (2.0,2.8) node[left]{$y=f'(x)$};
			\end{tikzpicture}
		\end{center}
		Dựa vào đồ thị $(C)$ ta có\\
		$f'\left(x+2\right)-2<-2,\forall x\in\left(-3;-1\right)\bigcup\left(1;3\right)\Leftrightarrow{f}'\left(x+2\right)<0,\forall x\in\left(-3;-1\right)\bigcup\left(1;3\right)$.\\
		Đặt $ x^*=x+2$ suy ra: $f'\left(x^*\right)<0,\forall x^*\in\left(-1;1\right)\bigcup\left(3;5\right)$.\\
		Vậy: Hàm số $ f(x)$ đồng biến trên khoảng $\left(-1;1\right),\,\,\left(3;5\right)$.}
\end{ex}

\begin{ex}%[2D1G1-2]%Câu 17
	\immini{
		Cho hàm số $ y=f(x)$ có đạo hàm là hàm số $f'(x)$ trên $\mathbb{R}$. Biết rằng hàm số $ y=f'\left(x-2\right)+2$ có đồ thị như hình vẽ bên dưới. Hàm số $ f(x)$ nghịch biến trên khoảng nào?
		\choice
		{$\left(-\infty ;2\right)$}
		{\True $\left(-1;1\right)$}
		{$\left(\dfrac{3}{2};\dfrac{5}{2}\right)$}
		{$\left(2;+\infty\right)$}}{
		\begin{tikzpicture}[scale=0.7,font=\footnotesize, line join=round, line cap=round, >=stealth] %Đường cong bậc 3
			\draw[thick, ->] (-0.5,0)--(3.5,0);
			\draw[thick, ->] (0,-1.8)--(0,5.3);
			\draw (3.7,0) node[below] {$x$};
			\draw (0,5.4) node[left]{$y$};
			\draw (0,0) node[below left]{$0$};
			\draw[fill] (3,0) circle (0.5pt)node[below]{$ 3$};
			\draw[fill] (1,0) circle (0.5pt)node[below]{$ 1$};
			\draw[fill] (2,0) circle (0.5pt)node[above]{$2$};
			\draw[fill] (0,2) circle (0.5pt)node[left]{$ 2$};
			\draw[fill] (0,-1) circle (0.5pt)node[left]{$ -1$};
			%		\draw[fill] (0,2) circle (0.5pt)node[above left]{$ 2$};
			%		\draw[fill] (0,-2) circle (0.5pt)node[below left]{$ -2$};
			\draw[dashed] (3,0)--(3,2)--(0,2)--(1,2)--(1,0); 
			\draw[dashed](0,-1)--(2,-1)--(2,0);
			\draw[line width=1.2pt,smooth,samples=100,domain=0.6:3.4] plot(\x,{3*(\x)^2-12*(\x)+11});		
			%\draw[line width=1.2pt,smooth,samples=100,domain=-3.3:2.8] plot(\x,{0.75*(\x)^2+0.5*\x-1});
			%	\draw (2.0,2.8) node[left]{$y=f'(x)$};
	\end{tikzpicture}	}
	\loigiai{
		Hàm số $ y=f'\left(x-2\right)+2$ có đồ thị $(C)$ như sau
		\begin{center}
			\begin{tikzpicture}[scale=0.7,font=\footnotesize, line join=round, line cap=round, >=stealth] %Đường cong bậc 3
				\draw[thick, ->] (-0.5,0)--(3.5,0);
				\draw[thick, ->] (0,-1.8)--(0,5.3);
				\draw (3.7,0) node[below] {$x$};
				\draw (0,5.4) node[left]{$y$};
				\draw (0,0) node[below left]{$0$};
				\draw[fill] (3,0) circle (0.5pt)node[below]{$ 3$};
				\draw[fill] (1,0) circle (0.5pt)node[below]{$ 1$};
				\draw[fill] (2,0) circle (0.5pt)node[above]{$2$};
				\draw[fill] (0,2) circle (0.5pt)node[left]{$ 2$};
				\draw[fill] (0,-1) circle (0.5pt)node[left]{$ -1$};
				%		\draw[fill] (0,2) circle (0.5pt)node[above left]{$ 2$};
				%		\draw[fill] (0,-2) circle (0.5pt)node[below left]{$ -2$};
				\draw[dashed] (3,0)--(3,2)--(0,2)--(1,2)--(1,0); 
				\draw[dashed](0,-1)--(2,-1)--(2,0);
				\draw[line width=1.2pt,smooth,samples=100,domain=0.6:3.4] plot(\x,{3*(\x)^2-12*(\x)+11});		
				%\draw[line width=1.2pt,smooth,samples=100,domain=-3.3:2.8] plot(\x,{0.75*(\x)^2+0.5*\x-1});
				%	\draw (2.0,2.8) node[left]{$y=f'(x)$};
			\end{tikzpicture}
		\end{center}
		Dựa vào đồ thị $(C)$ ta có\\
		$f'\left(x-2\right)+2<2,\forall x\in\left(1;3\right)\Leftrightarrow{f}'\left(x-2\right)<0,\forall x\in\left(1;3\right)$.\\
		Đặt $ x^*=x-2$ thì $f'\left(x^*\right)<0,\forall x^*\in\left(-1;1\right)$.\\
		Vậy: Hàm số $ f(x)$ nghịch biến trên khoảng $\left(-1;1\right)$.\\
		Cách khác:\\
		Tịnh tiến sang trái hai đơn vị và xuống dưới $2$ đơn vị thì từ đồ thị $(C)$ sẽ thành đồ thị của hàm$ y=f'(x)$. Khi đó $f'(x)<0,\forall x\in\left(-1;1\right)$.\\
		Vậy hàm số $ f(x)$ nghịch biến trên khoảng $\left(-1;1\right)$.}
\end{ex}

\begin{ex}%[2D1G1-2]%Câu 18
	Cho hàm số $y=f(x)$ có đạo hàm cấp $ 3$ liên tục trên $\mathbb{R}$ và thỏa mãn $f(x)\cdot f'''(x)=x{\left(x-1\right)^2}{\left(x+4\right)^3}$ với mọi $x\in\mathbb{R}$ và $g(x)=\left[f'(x)\right]^2-2f(x)\cdot f''(x)$. Hàm số $h(x)=g\left(x^2-2x\right)$ đồng biến trên khoảng nào dưới đây?
	\choice
	{$\left(-\infty ;1\right)$}
	{$\left(2;+\infty\right)$}
	{$\left(0;1\right)$}
	{\True $\left(1;2\right)$}
	\loigiai{		
		Ta có $g'(x)=2f''(x){f}'(x)-2f'(x)\cdot f''(x)-2f(x)\cdot f'''(x)=-2f(x)\cdot f'''(x);$\\
		Khi đó $\left(h(x)\right)'=\left(2x-2\right){g}'\left(x^2-2x\right)=-2\left(2x-2\right)\left(x^2-2x\right){\left(x^2-2x-1\right)^2}{\left(x^2-2x+4\right)^3}$\\
		$h'(x)=0\Leftrightarrow\hoac{
			& x=0\\ 
			& x=1\\ 
			& x=2\\ 
			& x=1\pm\sqrt{2}.}$ 
		Ta có bảng xét dấu của $h'(x)$
		\begin{center}
			\begin{tikzpicture}
				\tkzTabInit[lgt=1.2,espcl=2,nocadre]
				{$t$/0.7, $h'(x)$ /.7} % first column
				{$-\infty$, $1-\sqrt{2}$,$0$, $1$,$2$,$1+\sqrt{2}$, $+\infty$} % first row
				\tkzTabLine { ,+,0,-,0,+,0,-,0,+,0,- } % second row
				%				\tkzTabLine {,-,z,+,t,+,} % third row
				%				\tkzTabLine {,+,d,-,z,+,} % last row
			\end{tikzpicture}
		\end{center}
		Suy ra hàm số $h(x)=g\left(x^2-2x\right)$ đồng biến trên khoảng $\left(1;2\right)$.}
\end{ex}

\begin{ex}%[2D1G1-2]%Câu 19
	Cho hàm số $ y=f(x)$ xác định trên $\mathbb{R}$. Hàm số $ y=g(x)=f'\left(2x+3\right)+2$ có đồ thị là một parabol với tọa độ đỉnh $ I\left(2;-1\right)$ và đi qua điểm $ A\left(1;2\right)$. Hỏi hàm số $ y=f(x)$ nghịch biến trên khoảng nào dưới đây?
	\choice
	{\True $\left(5;9\right)$}
	{$\left(1;2\right)$}
	{$\left(-\infty ;9\right)$}
	{$\left(1;3\right)$}
	\loigiai{	
		Xét hàm số $ g(x)=f'\left(2x+3\right)+2$ có đồ thị là một Parabol nên có phương trình dạng $ y=g(x)=a{x^2}+bx+c\,\,\,\,(P)$.\\
		Vì $(P)$ có đỉnh $ I\left(2;-1\right)$ nên $\heva{
			&\dfrac{-b}{2a}=2\\ 
			& g(2)=-1} \Leftrightarrow\heva{
			&-b=4a\\ 
			& 4a+2b+c=-1} \Leftrightarrow\heva{
			& 4a+b=0\\ 
			& 4a+2b+c=-1}$.\\
		Vì $(P)$ đi qua điểm $ A\left(1;2\right)$ nên $ g(1)=2\Leftrightarrow a+b+c=2$.\\
		Ta có hệ phương trình $\heva{
			& 4a+b=0\\ 
			& 4a+2b+c=-1\\ 
			& a+b+c=2} \Leftrightarrow\heva{
			& a=3\\ 
			& b=-12\\ 
			& c=11}$ nên $ g(x)=3x^2-12x+11$.\\
		Đồ thị của hàm $ y=g(x)$ là
		\begin{center}
			\begin{tikzpicture}[scale=0.7,font=\footnotesize, line join=round, line cap=round, >=stealth] %Đường cong bậc 3
				\draw[thick, ->] (-0.5,0)--(3.5,0);
				\draw[thick, ->] (0,-1.8)--(0,5.3);
				\draw (3.7,0) node[below] {$x$};
				\draw (0,5.4) node[left]{$y$};
				\draw (0,0) node[below left]{$0$};
				\draw[fill] (3,0) circle (0.5pt)node[below]{$ 3$};
				\draw[fill] (1,0) circle (0.5pt)node[below]{$ 1$};
				\draw[fill] (2,0) circle (0.5pt)node[above]{$2$};
				\draw[fill] (0,2) circle (0.5pt)node[left]{$ 2$};
				\draw[fill] (0,-1) circle (0.5pt)node[left]{$ -1$};
				%		\draw[fill] (0,2) circle (0.5pt)node[above left]{$ 2$};
				%		\draw[fill] (0,-2) circle (0.5pt)node[below left]{$ -2$};
				\draw[dashed] (3,0)--(3,2)--(0,2)--(1,2)--(1,0) (3.2,2)--(3,2); 
				\draw[dashed](0,-1)--(2,-1)--(2,0);
				\draw[line width=1.2pt,smooth,samples=100,domain=0.6:3.4] plot(\x,{3*(\x)^2-12*(\x)+11});		
				%\draw[line width=1.2pt,smooth,samples=100,domain=-3.3:2.8] plot(\x,{0.75*(\x)^2+0.5*\x-1});
				%	\draw (2.0,2.8) node[left]{$y=f'(x)$};
			\end{tikzpicture}	
		\end{center}
		Theo đồ thị ta thấy $ f'(2x+3)\le 0\Leftrightarrow f'(2x+3)+2\le 2\Leftrightarrow 1\le x\le 3$.\\
		Đặt $ t=2x+3\Leftrightarrow x=\dfrac{t-3}{2}$ khi đó $ f'(t)\le 0\Leftrightarrow 1\le\dfrac{t-3}{2}\le 3\Leftrightarrow 5\le t\le 9$.\\
		Vậy $ y=f(x)$ nghịch biến trên khoảng $\left(5;9\right)$.}
\end{ex}

\begin{ex}%[2D1G1-2]%Câu 20
	\immini{
		Cho hàm số $ y=f(x)$, hàm số $f'(x)=x^3+a{x^2}+bx+c\left(a,b,c\in\mathbb{R}\right)$ có đồ thị như hình vẽ bên.
		Hàm số $ g(x)=f\left(f'(x)\right)$ nghịch biến trên khoảng nào dưới đây?
		\choice
		{$\left(1;+\infty\right)$}
		{\True $\left(-\infty ;-2\right)$}
		{$\left(-1;0\right)$}
		{$\left(-\dfrac{\sqrt{3}}{3};\dfrac{\sqrt{3}}{3}\right)$}}{
		\begin{tikzpicture}[scale=0.8,font=\footnotesize, line join=round, line cap=round, >=stealth] %Đường cong bậc 3
			\draw[thick, ->] (-1.7,0)--(1.7,0);
			\draw[thick, ->] (0,-2.7)--(0,3.0);
			\draw (1.9,0) node[below] {$x$};
			\draw (0,3.2) node[left]{$y$};
			\draw (0,0) node[below left]{$0$};
			\draw[fill] (-1,0) circle (0.5pt)node[above left]{$ -1 $};
			\draw[fill] (1,0) circle (0.5pt)node[below right]{$ 1$};
			\draw[line width=1.2pt,smooth,samples=100,domain=-1.3:1.3] plot(\x,{2.667*(\x)^3+0*(\x)^2-2.667*\x});		
			%\draw[line width=1.2pt,smooth,samples=100,domain=-3.3:2.8] plot(\x,{0.75*(\x)^2+0.5*\x-1});
		\end{tikzpicture}	
	}
	\loigiai{	
		Vì các điểm $\left(-1;0\right),\left(0;0\right),\left(1;0\right)$ thuộc đồ thị hàm số $ y=f'(x)$ nên ta có hệ\\
		$\heva{
			&-1+a-b+c=0\\ 
			& c=0\\ 
			& 1+a+b+c=0} \Leftrightarrow\heva{
			& a=0\\ 
			& b=-1\\ 
			& c=0} \Rightarrow {f}'(x)=x^3-x\Rightarrow f''(x)=3x^2-1$.\\
		Ta có $ g(x)=f\left(f'(x)\right)\Rightarrow{g}'(x)=f'\left(f'(x)\right)\cdot f''(x)$.\\
		Xét \\
		$g'(x)=0\Leftrightarrow{g}'(x)=f'\left(f'(x)\right)\cdot f''(x)=0$\\
		$\Leftrightarrow {f}'\left(x^3-x\right)\left(3x^2-1\right)=0\Leftrightarrow\hoac{
			&{x^3}-x=0\\ 
			&{x^3}-x=1\\ 
			&{x^3}-x=-1\\ 
			& 3x^2-1=0} \Leftrightarrow \hoac{
			& x=\pm 1\\ 
			& x=0\\ 
			& x=x_1(x_1\approx 1,325)\\ 
			& x=x_2(x_2\approx-1,325)\\ 
			& x=\pm\dfrac{\sqrt{3}}{3}.}$\\
		Bảng biến thiên
		\begin{center}
			\begin{tikzpicture}
				\tkzTabInit[lgt=1.2,espcl=2,nocadre]
				{$t$/0.7, $h'(x)$ /.7} % first column
				{$-\infty$, $-1{,}325$,$-1$, $-\dfrac{\sqrt{3}}{3}$,$0$,$\dfrac{\sqrt{3}}{3}$,$1$,$1{,}325$, $+\infty$} % first row
				\tkzTabLine { ,-,0,+,0,-,0,+,0,-,0,+,0,-,0,+, } % second row
				%				\tkzTabLine {,-,z,+,t,+,} % third row
				%				\tkzTabLine {,+,d,-,z,+,} % last row
			\end{tikzpicture}
		\end{center}
		Dựa vào bảng biến thiên ta có $ g(x)$ nghịch biến trên $\left(-\infty ;-2\right)$}
\end{ex}
\Closesolutionfile{ans}
\indapan{10}{ans/CD1/Muc_9_10}
\end{document}



