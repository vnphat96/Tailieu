\def\sode{5}
\begin{name}
	{\tenchude}
	{\tendethi}
	{\tentruong}
	{\thoigian}
\end{name}
\Opensolutionfile{ans}[ans/ans-HXN-\sode-T]
\caulc
\begin{ex}%Câu 1
	Cho cấp số cộng $\left(u_n\right)$ có $u_2=3,\,\,u_3=5$. Công sai $d$ của cấp số cộng là:
	\choice
	{1}
	{\True 2}
	{8}
	{4}
	\loigiai{
		Chọn B.\\
		Ta có: $\,u_3=u_2+d\Leftrightarrow 5=3+d\Leftrightarrow d=2$ .}
\end{ex}
\begin{ex}%Câu 2
	\immini[thm]{ Cho hàm số có đồ thị như hình vẽ bên. Hàm số đã cho đồng biến trên khoảng nào sau đây?
		\choice
		{$\left(-\infty\,;\,\,-1\right)$}
		{$\left(-1\,;\,\,1\right)$}
		{$\left(-2\,;\,\,1\right)$}
		{$\left(1\,;\,\,+\infty\right)$}}{\includegraphics[scale=.8]{img/HXN-5.2}}
	\loigiai{
		Chọn B.\\
		Từ đồ thị hàm số, ta thấy hàm số đồng biến trên khoảng $\left(-1\,;\,\,1\right)$.}
\end{ex}
\begin{ex}%Câu 3
	Cho hình lăng trụ đứng $ ABC.A'{B}'{C}'$ có đáy là tam giác vuông cân tại $ B$ với $ AB=a$ và $A'B=a\sqrt{3}$. Thể tích khối lăng trụ $ ABC.A'{B}'{C}'$ là
	\choice
	{$\dfrac{a^3\sqrt{3}}{2}$}
	{$\dfrac{a^3}{6}$}
	{$\dfrac{a^3}{2}$}
	{\True $\dfrac{a^3\sqrt{2}}{2}$}
	\loigiai{
	Chọn D.\\
	Ta có $ A{A}'=\sqrt{A'{B^2}-A{B^2}}=a\sqrt{2}$, $S_{ABC}=\dfrac{1}{2}A{B^2}=\dfrac{a^2}{2}$.\\
	Thể tích khối lăng trụ là $ V=A{A}'\cdot{S_{ABC}}=\dfrac{a^3\sqrt{2}}{2}$.}
\end{ex}
\begin{ex}%Câu 4
	Gọi $ S$ là diện tích hình phẳng giới hạn bởi các đường $ y=\text{e}^x$, $ y=0$, $ x=0$, $ x=2$. Mệnh đề nào dưới đây đúng?
	\choice
	{$ S=\pi\displaystyle\int\limits_0^2\text{e}^{2x}\text{d}x$}
	{\True $ S=\displaystyle\int\limits_0^2\text{e}^x\text{d}x$}
	{$ S=\pi\displaystyle\int\limits_0^2\text{e}^x\text{d}x$}
	{$ S=\pi\displaystyle\int\limits_0^2\text{e}^x\text{d}x$}
	\loigiai{
		Chọn B.\\
		Diện tích hình phẳng giới cần tính là $ S=\displaystyle\int\limits_0^2e^x\text{d}x$.}
\end{ex}
\begin{ex}%Câu 5
	Mặt phẳng đi qua ba điểm $ A\left(0\,;\,\,0\,;\,\,2\right)$, $ B\left(1\,;\,\,0\,;\,\,0\right)$ và $C\left(0\,;\,\,3\,;\,\,0\right)$ có phương trình là
	\choice
	{\True $\dfrac{x}{1}+\dfrac{y}{3}+\dfrac{z}{2}=1$}
	{$\dfrac{x}{1}+\dfrac{y}{3}+\dfrac{z}{2}=-1$}
	{$\dfrac{x}{2}+\dfrac{y}{1}+\dfrac{z}{3}=1$}
	{$\dfrac{x}{2}+\dfrac{y}{1}+\dfrac{z}{3}=-1$}
	\loigiai{
		Chọn A.\\
		Mặt phẳng (ABC) chắn các trục tọa độ Ox, Oy, Oz lần lượt tại $ A\left(0\,;\,\,0\,;\,\,2\right)$, $ B\left(1\,;\,\,0\,;\,\,0\right)$ và $C\left(0\,;\,\,3\,;\,\,0\right)$ nên có phương trình $\dfrac{x}{1}+\dfrac{y}{3}+\dfrac{z}{2}=1$.}
\end{ex}
\begin{ex}%Câu 6
	Nếu $\displaystyle\int\limits_{-1}^2f(x)\text{d}x=5$ thì $\displaystyle\int\limits_{-1}^24f(x)\text{d}x$ bằng:
	\choice
	{\True $ 20$}
	{$ 10$}
	{$\dfrac{5}{2}$}
	{$\dfrac{5}{4}$}
	\loigiai{
		Chọn A.\\
		Ta có: $\displaystyle\int\limits_{-1}^24f(x)\text{d}x=4\displaystyle\int\limits_{-1}^2f(x)\text{d}x=4.5=20$.}
\end{ex}
\begin{ex}%Câu 7
	Trong không gian tọa độ $ Oxyz$, mặt cầu $(S)$ có tâm $ I\left(2\,;\,1;\,-1\right)$ và đường kính 6 có phương trình là
	\choice
	{$(x-2)^2+(y-1)^2+(z+1)^2=36$}
	{\True $(x-2)^2+(y-1)^2+(z+1)^2=9$}
	{$(x+2)^2+(y+1)^2+(z-1)^2=9$}
	{$(x+2)^2+(y+1)^2+(z-1)^2=36$}
	\loigiai{
		Chọn B.\\
		Mặt cầu $ (S)$ có tâm $ I(2;1;-1)$, bán kính $ R=3$ nên có phương trình là\\
		$\left(x-2\right)^2+\left(y-1\right)^2+\left(z+1\right)^2=9$.}
\end{ex}
\begin{ex}%Câu 8
	Một mẫu số liệu ghép nhóm về chiều cao của một lớp (đơn vị là centimét) có phương sai là $ 6,25$. Độ lệch chuẩn của mẫu số liệu đó bằng bao nhiêu cm:
	\choice
	{\True $ 2,5$ }
	{$ 12,5$}
	{$ 3,125$}
	{$ 42,25$}
	\loigiai{
		Chọn A.\\
		Độ lệch chuẩn của mẫu số liệu là: $\sqrt{6,25}=2,5$.}
\end{ex}
\begin{ex}%Câu 9
	Tìm giá trị lớn nhất $ M$ của hàm số $ y=\dfrac{3x-1}{x-3}$ trên đoạn $\left[0\,;\,2\right]$.
	\choice
	{$ M=5$}
	{$ M=-5$}
	{\True $ M=\dfrac{1}{3}$}
	{$ M=-\dfrac{1}{3}$}
	\loigiai{
		Chọn C.\\
		Ta có: $y'=\dfrac{-8}{\left(x-3\right)^2}<\,0\,,\,\,\forall x\in\left[0\,;\,\,2\right]$. Hàm số luôn nghịch biến trên $\left[0\,;\,2\right]$.\\
		Ta tính được: $ y(0)=\dfrac{1}{3}$, $ y(2)=\,-5$.\\
		Do đó giá trị lớn nhất của hàm số trên $\left[0\,;\,2\right]$ là $ M=y(0)=\dfrac{1}{3}$.}
\end{ex}
\begin{ex}%Câu 10
	Cho hai biến cố $ A\,,\,\,B$ với $ 0<P(B)<1.$ Phát biểu nào sau đây là đúng?
	\choice
	{$ P(A)=P\left(\overline{B}\right).P\left(A|B\right)+P(B).P\left(A|\overline{B}\right)$}
	{$ P(A)=P(B).P\left(A|B\right)-P\left(\overline{B}\right).P\left(A|\overline{B}\right)$}
	{$ P(A)=P\left(\overline{B}\right).P\left(A|\overline{B}\right)-P(B).P\left(A|B\right)$}
	{\True $ P(A)=P(B).P\left(A|B\right)+P\left(\overline{B}\right).P\left(A|\overline{B}\right)$}
	\loigiai{
		Chọn D.\\
		Theo công thức xác suất toàn phần ta có: $ P(A)=P(B).P\left(A|B\right)+P\left(\overline{B}\right).P\left(A|\overline{B}\right)$.}
\end{ex}
\begin{ex}%Câu 11
	Một thư viện ghi lại số giờ đọc sách của 50 sinh viên trong một ngày và thu được mẫu số liệu ghép nhóm sau:\\
	\centerline{\begin{tabular}{|c|c|c|c|c|c|}
			\hline
			Nhóm giờ     & $\left[0\,;\,\,1\right)$ & $\left[1\,;\,\,2\right)$ & $\left[2\,;\,\,3\right)$ & $\left[3\,;\,\,4\right)$ & $\left[4\,;\,\,5\right)$ \\
			\hline
			Số sinh viên & 8                        & 11                       & 15                       & 9                        & 7                        \\
			\hline
		\end{tabular}}\\
	Khoảng tứ phân vị của mẫu số liệu ghép nhóm gần nhất với giá trị nào sau đây?
	\choice
	{$1,69$}
	{$1,85$}
	{$2,02$}
	{\True $1,98$}
	\loigiai{
	Chọn D.\\
	Giả sử mẫu số liệu gốc là $x_1;\,\,x_2;\,\,...;\,\,x_{50}$ được xếp theo thứ tự không giảm.\\
	Xét nửa bên trái mẫu số liệu gốc là $x_1;\,\,x_2;\,\,...;\,\,x_{25}$. Tứ phân vị thứ nhất của mẫu số liệu gốc là $x_{13}\in\left[1\,;\,\,2\right)$ nên tứ phân vị thứ nhất của mẫu số liệu ghép nhóm là $Q_1=1+\dfrac{\dfrac{50}{4}-8}{11}.1=\dfrac{31}{22}\approx 1,41$ (giờ).\\
	Xét nửa bên phải mẫu số liệu gốc là $x_{26};\,\,x_2;\,\,...;\,\,x_{50}$.\\
	Tứ phân vị thứ ba của mẫu số liệu gốc là $x_{38}\in\left[3\,;\,\,4\right)$ nên tứ phân vị thứ ba của mẫu số liệu ghép nhóm là $Q_3=3+\dfrac{3.\dfrac{50}{4}-34}{9}.1=\dfrac{61}{18}\approx 3,39$ (giờ).\\
	Khoảng tứ phân vị của mẫu số liệu ghép nhóm: $\Delta Q=Q_3-Q_1\approx 1,98$ (giờ).}
\end{ex}
\begin{ex}%Câu 12
	Tập nghiệm của bất phương trình $\log_5\left(2x-1\right)<\log_5\left(x+2\right)$ là
	\choice
	{$S=\left(3\,;\,\,+\infty\right)$}
	{$S=\left(-\infty\,;\,\,3\right)$}
	{\True $S=\left(\dfrac{1}{2}\,;\,\,3\right)$}
	{$S=\left(-2\,;\,\,3\right)$}
	\loigiai{
		Chọn C.\\
		Ta có: $\log_5\left(2x-1\right)<\log_5\left(x+2\right)\Leftrightarrow\left\{\begin{aligned}
				 & 2x-1>0   \\
				 & 2x-1<x+2 \\
			\end{aligned}\right.\Leftrightarrow\left\{\begin{aligned}
				 & x>\dfrac{1}{2} \\
				 & x<3            \\
			\end{aligned}\right.$ .\\
		Vậy tập nghiệm phương trình $S=\left(\dfrac{1}{2}\,;\,\,3\right)$ .}
\end{ex}
\Closesolutionfile{ans}
\cauds
\Opensolutionfile{ans}[ans/ans-HXN-\sode-TF]
\begin{ex}
	Một người đang bơm khí vào một quả bóng bay với tốc độ $100 $cm$^3/s$. Quả bóng ngày càng to dần nhưng luôn có dạng hình cầu. Đây là loại bóng bóng mà nếu người bơm để bán kính vượt quá $30$cm thì bóng sẽ bể.
	Xét tính đúng sai các mệnh đề sau:
	\choiceTF
	{Sau 10 giây, bán kính quả bóng bóng bằng $6,4 cm$ (làm tròn đến hàng phần chục của $cm$)}
	{\True Người bơm không thể để cho thể tích quả bóng bóng vượt quá $113$ lít (làm tròn đến hàng phần chục của lít)}
	{\True Khi đường kính của quả bóng bóng là $50 cm$ thì bán kính của quả bóng đang tăng với tốc độ $0,01 cm/s$ (làm tròn đến hàng phần trăm của $cm/s$)}
	{Nếu sau khi bơm được 4 giây, người bơm tăng tốc độ bơm thêm $5 cm^3$ trên một giây thì sau 189 giây (làm tròn đến hàng đơn vị của giây), bóng bóng sẽ bể}
	\loigiai{
		\begin{itemchoice}
			\itemch Gọi $V(t), R(t)$, là thể tích và bán kính quả bóng bóng sau $t$ giây, ta có $V(t) = \dfrac{4}{3}\pi R^3(t)$.
			Sau 10 giây, thể tích quả bóng là $V(10) = 100 \times 10 = 1000 cm^3$.
			Ta có $V(10) = \dfrac{4}{3}\pi R_{10}^3 = 1000 \Rightarrow R_{10} \approx 6.2 cm$.
			\itemch Bán kính tối đa của quả bóng bóng là $30 cm$; thể tích tối đa của quả bóng bóng là $\dfrac{4}{3}\pi \cdot 30^3 \approx 113097 cm^3 \approx 113$ lít.
			\itemch Khi bán kính bong bóng bóng bằng $\dfrac{50}{2} = 25 cm$ thì thể tích bong bóng là $\dfrac{4\pi \cdot 25^3}{3} = \dfrac{62500\pi}{3} cm^3$.
			Đạo hàm hai vế của $V(t) = \dfrac{4}{3}\pi R^3(t)$ theo t, ta được: $\dfrac{dV(t)}{dt} = 4\pi R^2\cdot \dfrac{dR}{dt}$.
			Thay $R_t = 25 cm$; $\dfrac{dV(t)}{dt} = 100 cm^3/s$, ta có: $\dfrac{dR}{dt} \approx 0.01 cm/s$.
			\itemch Thể tích bong bóng sau $t+4$ giây ($t \ge 0$) là $V(t) = 100\cdot 4 + \int\limits_{0}^{t}(5t+100)dt$.
			Thể tích tối đa của quả bóng bóng là $\dfrac{4}{3}\pi \cdot 30^3 cm^3$.
			Xét $V(t) = 100\cdot 4 + \int\limits_{0}^{t}(5t+100)dt = \dfrac{4}{3}\pi \cdot 30^3 \Rightarrow t \approx 193$ giây.
		\end{itemchoice}
	}
\end{ex}
\begin{ex}
	\immini[thm]{ Vịnh Hạ Long là một địa danh du lịch được nhiều người biết đến trên thế giới, nơi đây vẫn còn nhiều quần thể đảo lớn nhỏ chưa được khám phá. Một công ty du lịch quyết định khai thác khu vực có một số đảo nhỏ với hình dáng đặc biệt nếu nhìn từ trên xuống; trong số đó có hai hòn đảo mà phần giới hạn lát cắt của nó được mô phỏng như hai đồ thị hàm số trên hình. Với hệ trục tọa độ $Oxy$ thích hợp, đơn vị trên mỗi trục là $100$ mét, đường cong mô tả cho hòn đảo thứ nhất có dạng $y = \log_{a}{x}$ đi qua điểm có tọa độ $(3; 1)$.
	}{\includegraphics[scale=1]{img/HXN-5.14}}
	\choiceTF
	{ Điểm có tọa độ $(9; 3)$ thuộc đường cong $y = \log_{a}{x}$}
	{ Chủ dự án muốn xây dựng một nơi trực tiếp nhìn ra biển để du khách tham quan, ăn uống... Họ đã lựa chọn khu vực tam giác cong $ABC$ như trong hình (đường cong $AC$ tiếp giáp biển); diện tích khu vực này là $536$ $m^2$ (làm tròn đến hàng đơn vị)}
	{\True Chủ dự án đã thuê một số kỹ sư rất giỏi toán (đặc biệt giỏi về hàm số mũ-lôgarit) đi khảo sát khu vực này và họ nhận thấy có thể bồi đắp thêm cho hòn đảo thứ hai để đường cong giáp biển $y = g(x)$ của nó đối xứng với đường cong $y = \log_{a}{x}$ qua đường thẳng $y = x+1$. Khi đó đường cong $g(x) = 1+3 \cdot 3^x$}
	{ Chủ dự án định xây một cây cầu nối liền hai hòn đảo, khoảng cách ngắn nhất theo đường chim bay của cây cầu bằng $285$ $m$ (làm tròn đến hàng đơn vị mét)}
	\loigiai{
		\begin{itemchoice}
			\itemch Đường cong $y = \log_{a}{x}$ đi qua điểm $(3; 1)$ nên $1 = \log_{a}{3} \Rightarrow a = 3$.\\
			Khi đó hàm số trở thành $y = \log_{3}{x}$; đường cong này không đi qua điểm $(9; 3)$.
			\itemch Điểm $A(x_A; -1)$ thuộc đồ thị hàm số $y = \log_3 x \Rightarrow \log_3 x_A = -1 \Rightarrow x_A = 3^{-1} = \dfrac{1}{3}$.\\
			Diện tích tam giác cong $ABC$ là phần hình phẳng được giới hạn bởi hai đồ thị $y=\log_3 x$; $y = -1$ cùng các đường thẳng $x = \dfrac{1}{3}$; $x = 4$.\\
			Do đó diện tích cần tính là $S = \int\limits_{\frac{1}{3}}^{4} |\log_3 x - (-1)| \mathrm{d} x \approx 571$ $m^2$.
			\itemch Gọi $M(x_M; y_M) \in (C_1): y = \log_3 x$ và $N(x; y) \in (C_2): y = g(x)$.
			\begin{center}
				\includegraphics[scale=.8]{img/HXN-5.14a}
			\end{center}
			$M, N$ đối xứng qua $x - y + 1 = 0$ nên ta có:\\
			$\begin{cases} \dfrac{x + x_M}{2} - \dfrac{y + y_M}{2} + 1 = 0 \\ 1 \cdot (x - x_M) + (-1) \cdot (y - y_M) = 0 \end{cases} \Leftrightarrow \begin{cases} x + x_M - y - y_M + 2 = 0 \\ x - x_M - y + y_M = 0 \end{cases} \Rightarrow \begin{cases} y_M = x+1 \\ x_M = y-1 \end{cases}$ hay $M(y-1; x+1)$.\\
			Vì $M \in (C_1)$ nên $x+1 = \log_3 (y-1) \Rightarrow y-1 = 3^{x+1} \Rightarrow y = 3^{x+1} + 1$ hay $y = g(x) = 1 + 3 \cdot 3^x$.\\
			Cách giải khác (nhấn vào link) \hyperlink{Cách giải khác(nhấn vào link)}{https://www.tiktok.com/@tp1.phatvn.68/photo/7517274797793955079}
			\itemch Xét tiếp tuyến của đường cong $y = \log_3 x$ biết tiếp tuyến song song với đường thẳng $y = x+1$.\\
			Hệ số góc tiếp tuyến là $k = 1$; gọi $M(x_0; y_0)$ là tiếp điểm.\\
			Thì $f'(x_0) = \dfrac{1}{x_0 \ln 3} = 1 \Rightarrow x_0 = \dfrac{1}{\ln 3}$; $y_0 = \log_3 \dfrac{1}{\ln 3}$.\\
			Độ dài ngắn nhất cây cầu (theo đường chim bay) bằng hai lần khoảng cách từ $M\left(\dfrac{1}{\ln 3}; \log_3 \dfrac{1}{\ln 3}\right)$ đến đường thẳng $y = x+1$.\\
			Ta có: $d_{\min} = 2 \cdot \dfrac{\left| \dfrac{1}{\ln 3} - \log_3 \dfrac{1}{\ln 3} + 1 \right|}{\sqrt{1^2 + (-1)^2}} \times 100 \approx 282$ $m$.
		\end{itemchoice}
	}
\end{ex}
\begin{ex}
	Hộp A đựng 4 bi xanh và 4 bi trắng, hộp B đựng 6 bi xanh và 3 bi trắng, hộp C không có viên bi nào. Người ta thực hiện liên tiếp ba hành động sau đây hoàn toàn ngẫu nhiên:
	\begin{itemize}
		\item Lấy 1 viên bi từ hộp A bỏ sang hộp B.
		\item Lấy 1 viên bi từ hộp B bỏ sang hộp C.
		\item Lấy 1 viên bi từ hộp A bỏ sang hộp C.
	\end{itemize}
	Xét tính đúng sai các mệnh đề sau:
	\choiceTF
	{Nếu từ hộp A đã lấy 1 bi trắng bỏ sang hộp B thì xác suất để lấy bi trắng từ hộp B bỏ sang hộp C bằng $\dfrac{2}{5}$}
	{\True Xác suất để lấy được bi trắng từ hộp B bỏ sang hộp C bằng $\dfrac{7}{20}$}
	{\True Xác suất để lấy từ C được 2 bi xanh bằng $\dfrac{9}{28}$}
	{\True Xác suất để 2 bi lấy từ hộp C đều là các bi từ hộp A chuyển sang bằng $\dfrac{1}{15}$ biết rằng đó là 2 bi xanh}
	\loigiai{
		\begin{itemchoice}
			\itemch Nếu từ hộp A đã lấy 1 bi trắng bỏ sang hộp B thì khi đó hộp B có 6 bi xanh và 4 bi trắng; xác suất để lấy 1 bi trắng từ hộp B là $\dfrac{4}{10} = \dfrac{2}{5}$.
			\itemch Ta mô phỏng bài toán bởi sơ đồ sau:
			\begin{center}
				\includegraphics[scale=1]{img/HXN-5.15}
			\end{center}
			Ta có: $P(\text{Trắng}_{[B]\to[C]}) = \dfrac{1}{2} \cdot \dfrac{3}{10} + \dfrac{1}{2} \cdot \dfrac{4}{10} = \dfrac{7}{20}$.
			\itemch Ta có: $P(2\text{Xanh}_{[C]}) = \dfrac{1}{2} \cdot \dfrac{7}{10} \cdot \dfrac{3}{7} + \dfrac{1}{2} \cdot \dfrac{6}{10} \cdot \dfrac{4}{7} = \dfrac{9}{28}$.\\
			(Trong đó ta xem kí hiệu $2\text{Xanh}_{[C]}$ là lấy được 2 viên bi xanh từ hộp C).
			\itemch Ta có: $P(2\text{ bi}_{[A]\to[C]} | 2\text{Xanh}_{[C]}) = \dfrac{\dfrac{1}{2} \cdot \dfrac{1}{10} \cdot \dfrac{3}{7}}{\dfrac{9}{28}} = \dfrac{1}{15}$.\\
			(Trong đó ta xem kí hiệu $2\text{ bi}_{[A]\to[C]}$ là lấy từ hộp C đúng 2 viên bi từ hộp A chuyển qua).
		\end{itemchoice}
	}
\end{ex}
\begin{ex}
	Trong không gian $Oxyz$ cho trước, đơn vị trên mỗi trục là mét, có hai chiếc chiến đấu cơ từ hai vị trí $A(40;-15;15)$ và $B(55;-10;65)$ cần đáp xuống hai vị trí thuộc tàu sân bay hải quân để nạp nhiên liệu. Bề mặt chứa các đường băng trên tàu là mặt phẳng $(P)$ có phương trình $3x - y + 2z - 25 = 0$. Xét tính đúng sai các mệnh đề sau:
	\choiceTF
	{\True Đường thẳng qua $A$ và vuông góc với mặt phẳng $(P)$ có phương trình chính tắc là $\dfrac{x - 40}{3} = \dfrac{y + 15}{-1} = \dfrac{z - 15}{2}$}
	{ Tổng khoảng cách từ hai vị trí chiến đấu cơ đến mặt phẳng chứa đường băng là $110$ $m$ (làm tròn đến hàng đơn vị của mét)}
	{\True Tọa độ $A'$ đối xứng với $A$ qua $(P)$ là $A'(-20; 5; -25)$}
	{ Người chỉ huy ở tàu sân bay phát tín hiệu để hai chiến đấu cơ đáp xuống các vị trí $M, N$ cách nhau $5\sqrt{6}$ $m$. Tổng đường bay ngắn nhất $AM + BN$ bằng $115$ $m$ (làm tròn đến hàng đơn vị)}
	\loigiai{
		\begin{itemchoice}
			\itemch Đường thẳng qua $A$ và vuông góc với mặt phẳng $(P)$ có phương trình chính tắc là $\dfrac{x - 40}{3} = \dfrac{y + 15}{-1} = \dfrac{z - 15}{2}$.
			\itemch Ta có: $d(A, (P)) + d(B, (P)) = \dfrac{|3 \cdot 40 - (-15) + 2 \cdot 15 - 0|}{\sqrt{3^2 + (-1)^2 + 2^2}} + \dfrac{|3 \cdot 55 - (-10) + 2 \cdot 65 - 0|}{\sqrt{3^2 + (-1)^2 + 2^2}} = 30\sqrt{14} \approx 112$ $m$.
			\begin{center}
				\includegraphics[scale=.8]{img/HXN-5.16}
			\end{center}
			\itemch Gọi $H$ là hình chiếu vuông góc của $A$ trên $(P)$ thì tọa độ $H$ thỏa hệ phương trình\\ $\begin{cases} \dfrac{x - 40}{3} = \dfrac{y + 15}{-1} = \dfrac{z - 15}{2} \\ 3x - y + 2z - 0 = 0 \end{cases} \Leftrightarrow \begin{cases} x+3y+5=0 \\ 2y+z+15=0 \\ 3x-y+2z=25 \end{cases}$. Giải hệ này ta được $\begin{cases} x=10 \\ y=-5 \\ z=-5 \end{cases}$ hay $H(10; -5; -5)$.\\
			$A'$ đối xứng với $A$ qua $(P)$ nên $H$ là trung điểm của $AA'$. Suy ra $A'(-20; 5; -25)$.
			\itemch Lấy điểm $E$ thỏa mãn $\overrightarrow{AE} = \overrightarrow{MN}$, suy ra $A'M = EN$.\\
			Vì $A'$ cố định mà $A'E = 5\sqrt{6}$ nên $E$ thuộc đường tròn tâm $A'$, bán kính $r = 5\sqrt{6}$; đường tròn này thuộc mặt phẳng $(Q)$ qua $A'$ và song song với $(P)$.\\
			Gọi $K, F$ theo thứ tự là hình chiếu vuông góc của $B$ trên $(P)$, $(Q)$ suy ra $K(-5; 10; 25)$. $HK = 15\sqrt{6}$; $KF = HA' = AH = 10\sqrt{33}$.\\
			Ta có $AM + BN = A'M + BN = EN + BN \ge BE$.\\
			Đẳng thức xảy ra khi $E, N, B$ thẳng hàng theo thứ tự đó ($H, M, N, K$ thẳng hàng).\\
			Ta có: $BE = \sqrt{(\sqrt{20\sqrt{14}+10\sqrt{14}})^2 + (15\sqrt{6} - 5\sqrt{6})^2} = \sqrt{(30\sqrt{14})^2 + (10\sqrt{6})^2} = \sqrt{12600 + 600} = \sqrt{13200} = 20\sqrt{33}$.\\
			Vậy tổng độ dài bé nhất $AM + BN$ là $20\sqrt{33} \approx 115$ $m$.
		\end{itemchoice}
	}
\end{ex}

\Closesolutionfile{ans}
\caukq
\Opensolutionfile{ans}[ans/ans-HXN-\sode-SA]
\begin{ex}%Câu 13
	\immini[thm]{ Một trò chơi điện tử có luật chơi như sau:
		\begin{itemize}
			\item Người chơi xuất phát từ A và đi qua tất cả vị trí B, C, D, E trước khi về lại A để kết thúc lượt chơi của mình. Mỗi vị trí người chơi đi qua đúng 1 lần (trừ điểm A).
			\item Thông số trên mỗi đoạn đường đi gồm: x (huy chương) liên quan đến phần thưởng và y (quái vật) liên quan đến chướng ngại vật; điểm số người chơi đạt được trên mỗi đoạn đường có dạng $3x-2y$ .
		\end{itemize}
		Hỏi tổng số điểm tối đa mà người chơi đạt được là bao nhiêu?
		\shortans{ 44}}{\includegraphics[scale=1]{img/HXN-5.17}}
	\loigiai{
		Người chơi đi qua các con đường hợp lệ cùng với số điểm tương ứng như sau:
		\begin{itemize}
			\item $A \to B \to C \to E \to D \to A$; số điểm là $3(5+6+6+3+4)-2(3+4+6+0+1)=44$.
			\item $A \to B \to C \to D \to E \to A$; số điểm là $3(5+6+4+3+8)-2(3+4+5+0+6)=42$.
			\item $A \to E \to B \to C \to D \to A$; số điểm là $3(8+0+6+4+4)-2(6+1+4+5+1)=32$.
			\item $A \to E \to D \to C \to B \to A$; số điểm là $3(8+3+4+6+5)-2(6+0+5+4+3)=42$.
			\item $A \to D \to C \to B \to E \to A$; số điểm là $3(4+4+6+0+8)-2(1+5+4+1+6)=32$.
			\item $A \to D \to E \to C \to B \to A$; số điểm là $3(4+3+6+6+5)-2(1+0+6+4+3)=44$.
		\end{itemize}
		Số điểm tối đa mà người chơi đạt được là 44.}
\end{ex}
\begin{ex}%Câu 14
	Một hộp phấn không bụi có dạng hình hộp chữ nhật, chiều cao hộp phấn bằng 8,2 cm và đáy của nó có hai kích thước là 8,5 cm; 10,5 cm (xem hình vẽ). Tìm số đo góc phẳng nhị diện $\left[A,\,\,B'{D}',\,\,A'\right]$ (tính theo độ, làm tròn kết quả đến hàng phần chục).
	\begin{center}
		\includegraphics[scale=.5]{img/HXN-5.18}
	\end{center}
	\shortans{ 51,1}
	\loigiai{
	\begin{center}
		\includegraphics[scale=1]{img/HXN-5.18a}
	\end{center}
	Trong mặt phẳng $\left(A'{B}'{C}'{D}'\right)$, kẻ $A'H\bot{B}'{D}'$ tại H.\\
	Ta có: $\left\{\begin{aligned}
		 & {B}'{D}'\bot{A}'H                                                                      \\
		 & {B}'{D}'\perp A{A}'\,\,\left(\text{do}\,\,A{A}'\perp\left(A'{B}'{C}'{D}'\right)\right) \\
	\end{aligned}\right.$$\Rightarrow{B}'{D}'\perp\left(A{A}'H\right)\Rightarrow{B}'{D}'\perp AH$.\\
		Do đó $\widehat{AH{A}'}$ là góc phẳng nhị diện $\left[A,\,\,B'{D}',\,\,A'\right]$.\\
		Tam giác $A'{B}'{C}'$ vuông tại $A'$ có đường cao $A'H$ nên $\dfrac{1}{A'{H^2}}=\dfrac{1}{A'{B'^2}}+\dfrac{1}{A'{D'^2}}\Rightarrow{A}'H=\dfrac{A'{B}'.A'{D}'}{\sqrt{A'{B'^2}+A'{D'^2}}}=\dfrac{357}{2\sqrt{730}}$.\\
		Tam giác $ AH{A}'$ vuông tại $A'$ có $\tan\widehat{AH{A}'}=\dfrac{A{A}'}{A'H}=\dfrac{8,2}{\dfrac{357}{2\sqrt{730}}}\Rightarrow\widehat{AH{A}'}\approx 51,1^\circ$.}
\end{ex}
\begin{ex}%Câu 15
	Lan đang dự tính ghi danh học các lớp kỹ năng Anh ngữ, kỹ năng giao tiếp, kỹ năng quản lí v.v... tại một Hệ thống giáo dục trong thành phố, nơi mỗi lớp học chỉ học một lần mỗi tuần. Cô ấy đang chọn giữa 30 lớp học không trùng nhau. Có 6 lớp để lựa chọn cho mỗi ngày trong tuần, từ thứ Hai đến thứ Sáu. Sau nhiều ngày cân nhắc và tìm kiếm lời khuyên, Lan vẫn chưa thể đưa ra lựa chọn phù hợp. Sau cùng cô quyết định đăng ký 7 lớp được chọn ngẫu nhiên trong số 30 lớp đó, với mọi lựa chọn là đồng xác suất. Xác suất để Lan có lớp học vào tất cả các ngày từ thứ Hai đến thứ Sáu bằng $\dfrac{m}{n}$ (trong đó hai số m, n là nguyên tố cùng nhau). Tính $ m+n$.\\
	\shortans{ 491}
	\loigiai{
		Có hai khả năng chính để Lan có lớp học mỗi ngày trong tuần:\\
		• Trường hợp 1: Có 2 ngày có 2 lớp học, và 3 ngày còn lại có 1 lớp học.\\
		Số khả năng cho trường hợp 1 là $ C_5^2\cdot{\left(C_6^2\right)^2}\cdot{\left(C_6^1\right)^3}$.\\
		(Chọn 2 ngày trong 5 ngày có 2 lớp học, mỗi ngày đó chọn 2 lớp trong số 6 lớp; 3 ngày còn lại mỗi ngày chọn 1 lớp trong 6 lớp → có $\left(C_6^1\right)^3$ cách).\\
		• Trường hợp 2: Có 1 ngày có 3 lớp học, và 4 ngày còn lại mỗi ngày có 1 lớp học.\\
		Số khả năng cho trường hợp 2 là $ C_5^1C_6^3\cdot{\left(C_6^1\right)^4}$.\\
		(Chọn 1 ngày có 3 lớp học trong 5 ngày, chọn 3 lớp trong 6 lớp cho ngày đó; 4 ngày còn lại mỗi ngày chọn 1 lớp → $\left(C_6^1\right)^4$ cách).\\
		Vậy xác suất cần tính là $\dfrac{C_5^2\cdot{\left(C_6^2\right)^2}\cdot{\left(C_6^1\right)^3}+C_5^1C_6^3\cdot{\left(C_6^1\right)^4}}{C_{30}^7}=\dfrac{114}{377}=\dfrac{m}{n}$. Suy ra $m+n=491 $.}
\end{ex}
\begin{ex}%Câu 16
	Một chiến sĩ đặc công đang nấp ở bờ sông, cần phải bơi qua bờ bên kia để tấn công mục tiêu. Có thể xem con sông này là thẳng và có độ rộng 100 m; vận tốc bơi của chiến sĩ bằng một phần ba vận tốc chạy bộ. Biết rằng mục tiêu tấn công cách chiến sĩ 1 km theo đường chim bay; hỏi chiến sĩ phải bơi bao nhiêu mét để đến được mục tiêu nhanh nhất (làm tròn kết quả đến hàng đơn vị)?
	\shortans{106 }
	\begin{center}
		\includegraphics[scale=1.2]{img/HXN-5.20}
	\end{center}
	\loigiai{
		\begin{center}
			\includegraphics[scale=1.2]{img/HXN-5.20a}
		\end{center}
		Gọi C là hình chiếu vuông góc của A (vị trí chiến sĩ xuất phát) đối với bờ bên kia và D thuộc đoạn BC là vị trí mà chiến sĩ sẽ bơi đến trước khi chạy bộ tấn công mục tiêu tại A.
		Ta chuẩn hóa bài toán như sau:
		\begin{itemize}
			\item 1 đơn vị độ dài = $100 m$; khi đó $AC = 1$, $AB = 10$.
			\item Vận tốc bơi trên sông của chiến sĩ là 1 (đơn vị vận tốc); vận tốc chạy của chiến sĩ là 3 (đơn vị vận tốc).
		\end{itemize}
		Đặt $AD = x \in (1; 10) \Rightarrow CD = \sqrt{x^2-1}$; $BC = \sqrt{AB^2 - AC^2} = 3\sqrt{11}$.\\
		$BD = BC - CD = 3\sqrt{11} - \sqrt{x^2-1}$.\\
		Tổng thời gian từ khi chiến sĩ xuất phát đến khi tiếp cận mục tiêu là:
		$$t = \dfrac{AD}{1} + \dfrac{BD}{3} = \dfrac{x}{1} + \dfrac{3\sqrt{11} - \sqrt{x^2-1}}{3} = x + \sqrt{11} - \dfrac{\sqrt{x^2-1}}{3}$$
		Xét hàm $f(x) = x + \sqrt{11} - \dfrac{\sqrt{x^2-1}}{3}$; $x \in (1; 10)$; $f'(x) = 1 - \dfrac{1}{3}\dfrac{2x}{2\sqrt{x^2-1}} = 1 - \dfrac{x}{3\sqrt{x^2-1}}$.\\
		$f'(x) = 0 \Rightarrow 1 - \dfrac{x}{3\sqrt{x^2-1}} = 0 \Rightarrow \dfrac{x}{3\sqrt{x^2-1}} = 1 \Rightarrow x = 3\sqrt{x^2-1} \Rightarrow x^2 = 9(x^2-1) \Rightarrow x^2 = 9x^2 - 9 \Rightarrow 8x^2 = 9 \Rightarrow x^2 = \dfrac{9}{8} \Rightarrow x = \dfrac{3\sqrt{2}}{4}$.\\
		Bảng biến thiên:
		\begin{tabular}{|c|ccccc|}
			\hline
			$x$                           & 1            &            & $\dfrac{3\sqrt{2}}{4}$ &            & 10           \\
			\hline
			$f'(x)$                       & $\vert\vert$ & $-$        & 0                      & $+$        & $\vert\vert$ \\
			\hline
			\rule[-.1in]{0in}{.3in}$f(x)$ &              & $\searrow$ & Min                    & $\nearrow$ &              \\
			\hline
		\end{tabular}
		\\Chiến sĩ tiếp cận mục tiêu nhanh nhất khi $BD = x = \dfrac{3\sqrt{2}}{4}$.\\
		Do đó chiến sĩ sẽ bơi một đoạn $AD = 100 \times \dfrac{3\sqrt{2}}{4} \approx 106 m$.
	}
\end{ex}
\begin{ex}%Câu 17
	\immini[thm]{Một người nghệ sĩ đã vẽ hình chiếc nơ theo một cách khác lạ so với các nhà thiết kế. Anh ta vẽ hình chữ nhật ABCD tâm O có chiều dài bằng 4 dm, chiều rộng bằng 2 dm. Chiếc nơ chính là hình (H) nằm bên trong hình chữ nhật sao cho khi kẻ tia Ot bất kì cắt (H) và cạnh hình chữ nhật lần lượt tại M và N thì $ MN=1$ dm. Tính diện tích chiếc nơ hình (H) đó theo $ d{m^2}$ (làm tròn đến hàng phần chục).
		\shortans{1,52 }}{\includegraphics[scale=1]{img/HXN-5.21}}
	\loigiai{
	\begin{center}
		\includegraphics[scale=1]{img/HXN-5.21a}
	\end{center}
	Xét hình vẽ và các kí hiệu như sau.\\
	Gọi $\varphi=\left(Ox\,,\,\,Ot\right)$ thì $\cos\varphi=\dfrac{OH}{ON}=\dfrac{2}{r_{\varphi}+1}\Rightarrow r_{\varphi} =\dfrac{2}{\cos \varphi} -1$ ; với $r_{\varphi}=OM$ quay quanh gốc O khi $0<\varphi <\widehat{HOA}$ .\\
	Gọi $\theta=\left(Oy\,,\,\,O{t}'\right)$ thì $\cos\theta=\dfrac{OK}{O{N}'}=\dfrac{1}{r_{\theta}+1}$ $\Rightarrow r_{\theta}=\dfrac{1}{\cos \theta}-1 $ ; với $r_{\theta}=O{M}'$ quay quanh gốc O khi $0<\theta <\widehat{AOK}$ .\\
	Ta có: $\tan\widehat{AOK}=\dfrac{AK}{OK}=2\Rightarrow\widehat{AOK}=\arctan 2$ .\\
	Do đó diện tích cần tính là $S=4\cdot\dfrac{1}{2}\cdot\left[\displaystyle\int\limits_0^{\arctan 0,5}{\left(\dfrac{2}{\cos\varphi}-1\right)^2\text{d}\varphi}+\displaystyle\int\limits_0^{act\tan 2}{\left(\dfrac{1}{\cos\theta}-1\right)^2\text{d}\theta}\right]\approx 1,52\,\,d{m^2}$ .}
\end{ex}
\begin{ex}%Câu 18
	Trong không gian với hệ trục tọa độ $Oxyz$ cho ba mặt phẳng: $(P):x-2y+z-1=0$, $(Q):x-2y+z+8=0$, $(R):x-2y+z-4=0$. Một đường thẳng $d$ thay đổi cắt ba mặt phẳng $(P)$, $(Q)$, $(R)$ lần lượt tại $ A$, $ B$, $ C$. Tìm giá trị nhỏ nhất của $ T=A{B^2}+\dfrac{144}{AC}$.\\
	\shortans{108 }
	\loigiai{
	\begin{center}
		\includegraphics[scale=.7]{img/HXN-5.22a}
	\end{center}
	Dựa vào phương trình ba mặt phẳng $(P),\,\,(Q),\,\,(R)$ đã cho, ta thấy chúng song song nhau; so sánh hệ số tự do trong phương trình ba mặt phẳng thì: $-4<-1<8$, do vậy mặt phẳng $(P)$ nằm giữa hai mặt phẳng $(Q),\,\,(R)$.\\
	Ta tính khoảng cách giữa $(P)$ với hai mặt phẳng còn lại:\\ $d\left((P),(Q)\right)=\dfrac{\left| 8-\left(-1\right)\right|}{\sqrt{1^2+\left(-2\right)^2+1^2}}=\dfrac{9}{\sqrt{6}}$ ; $ d\left((P),(R)\right)=\dfrac{\left|-4-\left(-1\right)\right|}{\sqrt{1^2+\left(-2\right)^2+1^2}}=\dfrac{3}{\sqrt{6}}.$\\
	Do vậy $d\left((P),(Q)\right)=3 d\left((P),(R)\right)$.\\
	Gọi $A',\,\,B'$ lần lượt là hình chiếu của $ C$ trên các mặt phẳng $(P),\,\,(Q)$$\Rightarrow C{A}'=\dfrac{3}{\sqrt{6}},\,\,A'{B}'=\dfrac{9}{\sqrt{6}}$. Vì $ A{A}'\text{//}B{B}'$ nên $\dfrac{AC}{AB}=\dfrac{C{A}'}{A'{B}'}=\dfrac{\dfrac{3}{\sqrt{6}}}{\dfrac{9}{\sqrt{6}}}=\dfrac{1}{3}$ hay $AC=\dfrac{1}{3}AC $.\\
		Ta có: $ T=A{B^2}+\dfrac{144}{AC}=A{B^2}+\dfrac{144}{\dfrac{1}{3}AB}=A{B^2}+\dfrac{432}{AB}=A{B^2}+\dfrac{216}{AB}+\dfrac{216}{AB}\overset{AM-GM}{\mathop{\ge}}\,3\sqrt[3]{A{B^2}.\dfrac{216}{AB}.\dfrac{216}{AB}}=108$.\\
		Dấu $ ''=''$ xảy ra khi và chỉ khi $ A{B^2}=\dfrac{216}{AB}\Leftrightarrow A{B^3}=216\Leftrightarrow AB=6$, suy ra $ AC=2$.\\
		Vậy $T_{\min}=108$.}
\end{ex}
\Closesolutionfile{ans}
\inputansbox{6,4,3}{ans/ans-HXN-\sode-T,ans/ans-HXN-\sode-TF,ans/ans-HXN-\sode-SA}