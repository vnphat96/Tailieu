\def\sode{6}
\begin{name}
	{\tenchude}
	{\tendethi}
	{\tentruong}
	{\thoigian}
\end{name}
\caulc
\Opensolutionfile{ans}[ans/ans-HXN-\sode-T]
\begin{ex}%Câu 1
 Trong không gian $ Oxyz$, mặt phẳng đi qua điểm $ K(1\,;\,\,1\,;\,\,1)$ nhận $\vec{u}=(1\,;\,\,0\,;\,\,1)$, $\vec{v}=(1\,;\,\,1\,;\,\,0)$ làm căp vectơ chỉ phương có phương trình tổng quát là
 \choice
 {$ x+y+z-3=0$}
 {$ x-y+z-1=0$}
 {$ x+y-z-1=0$}
 {\True $-x+y+z-1=0$}
 \loigiai{
 Chọn D.\\
 Mặt phẳng có vectơ pháp tuyến $\vec{n}=\left[\vec{u},\vec{v}\right]=\left(-1\,;\\,1\ ;\\,1\right)$.\\
 Phương trình mặt phẳng là $-\left(x-1\right)+1\left(y-1\right)+1\left(z-1\right)=0\Leftrightarrow-x+y+z-1=0$.}
\end{ex}
\begin{ex}%Câu 2
 Cho bảng phân bố tần số ghép nhóm về độ dài của 60 lá dương xỉ trưởng thành như sau:\\
 \centerline{\begin{tabular}{|c|c|c|c|c|}
 \hline
 Độ dài (cm) & $\left[10\,;20\right)$ & $\left[20\,;30\right)$ & $\left[30\,;40\right)$ & $\left[40\,;50\right]$\\
 \hline
 Tần số & $ 8$ & $ 18$ & $ 24$ & $ 10$\\
 \hline
 \end{tabular}}\\
 Tính phương sai của mẫu số liệu ghép nhóm đã cho.
 \choice
 {$s_{}^2=83$}
 {\True $s_{}^2=84$}
 {$s_{}^2=85$}
 {$s_{}^2=86$}
 \loigiai{
 Chọn B.\\
 Ta viết lại bảng trên có bổ sung giá trị đại diện:\\
 \centerline{\begin{tabular}{|c|c|c|c|c|}
 \hline
 Độ dài (cm) & $\left[10\,;20\right)$ & $\left[20\,;30\right)$ & $\left[30\,;40\right)$ & $\left[40\,;50\right]$\\
 \hline
 Giá trị đại diện & $15$ & $25$ & $35$ & $45$\\
 \hline
 Tần số & $ 8$ & $ 18$ & $ 24$ & $ 10$\\
 \hline
 \end{tabular}}\\
 Giá trị trung bình của mẫu số liệu ghép nhóm: $\bar{x}=\dfrac{15\times 8+25\times 18+35\times 24+45\times 10}{60}=31$.\\
 Phương sai mẫu số liệu ghép nhóm là:\\
 $ s_{}^2=\dfrac{8\times{(15-31)^2}+18\times{(25-31)^2}+24\times{(35-31)^2}+10\times{(45-31)^2}}{60}=84$.}
\end{ex}
\begin{ex}%Câu 3
 Cho lăng trụ đều $ABC.A'{B}'{C}'$ . Góc giữa hai vectơ $\overrightarrow{BA}$ và $\overrightarrow{C'{B}'}$ bằng bao nhiêu?
 \begin{center}
\begin{tikzpicture}[scale=.7, font=\footnotesize, line join=round, line cap=round, >=stealth]
    \def\ac{4} % cạnh AC
    \def\ab{2} % cạnh AB
    \def\ben{4} % cạnh bên
    \def\gocnghieng{90} % góc nghiêng cạnh bên
    \def\gocA{50} % góc A của đáy
    \coordinate[label=left:$A$] (A) at (0,0);
    \coordinate[label=right:$C$] (C) at (\ac,0);
    \coordinate[label=below left:$B$] (B) at (-\gocA:\ab);
    \coordinate[label=left:$A'$] (A') at ($(A)+(\gocnghieng:\ben)$);
    \coordinate[label=below left:$B'$] (B') at ($(B)-(A)+(A')$);
    \coordinate[label=right:$C'$] (C') at ($(C)-(A)+(A')$);
    \draw (A')--(A)--(B)--(C)--(C')--(A')--(B')--(C') (B)--(B');
    \draw[dashed] (A)--(C);
    \foreach \diem in {A,B,C,A',B',C'} \fill (\diem)circle(1.5pt);
\end{tikzpicture}

 \end{center}
 \choice
 {$30^\circ $}
 {$60^\circ $}
 {\True $120^\circ $}
 {$90^\circ $}
 \loigiai{
 Chọn C.\\
 Ta có $\left(\overrightarrow{BA}\,,\,\,\,\overrightarrow{C'{B}'}\right)=\left(\overrightarrow{BA}\,,\,\,\overrightarrow{CB}\right)=180^\circ-\widehat{ABC}=180^\circ-60^\circ=120^\circ $ .}
\end{ex}
\begin{ex}%Câu 4
 Thống kê điểm thi đánh giá năng lực của một trường THPT qua thang điểm 100 được cho ở bảng sau:
 \begin{center}
     \begin{tabular}{|l|c|c|c|c|c|}
         \hline
         Điểm & $[0; 20)$ & $[20; 40)$ & $[40; 60)$ & $[60; 80)$ & $[80; 100]$ \\
         \hline
         Số học sinh & 25 & 35 & 37 & 15 & 8 \\
         \hline
     \end{tabular}
 \end{center}
 Trung vị của mẫu số liệu ghép nhóm là giá trị nào sau đây?
 \choice
 {$38,2$}
 {\True $40$}
 {$39,6$}
 {$42$}
 \loigiai{
 Chọn B.\\
 Kích thước mẫu số liệu $ n=25+35+37+15+8=120$.\\
 Trung vị của mẫu số liệu gốc là $\dfrac{x_{60}+x_{61}}{2}$; mà $x_{60}\in\left[20\,;\,\,40\right)\,,\,\,x_{61}\in\left[40\,;\,\,60\right)$ nên trung vị của mẫu số liệu ghép nhóm bằng $M_e$=40.}
\end{ex}
\begin{ex}%Câu 5
 Cho cấp số nhân $\left(u_n\right)$ có tổng $n$ số hạng đầu tiên là $S_n=5^n-1$ với $n\in{\mathbb{N}^*}$ . Tìm số hạng đầu $u_1$ và công bội $q$ của cấp số nhân đó.
 \choice
 {$u_1=5$, $q=4$}
 {$u_1=4$, $q=6$}
 {\True $u_1=4$, $q=5$}
 {$u_1=6$, $q=5$}
 \loigiai{
 Chọn C.\\
 Ta có $u_1=S_1=5^1-1=4$; $u_2=S_2-S_1=(5^2-1)-(5^1-1)=20$.\\
 Công bội cấp số nhân là $ q=\dfrac{u_2}{u_1}=\dfrac{20}{4}=5$.}
\end{ex}
\begin{ex}%Câu 6
 Trong không gian với hệ trục tọa độ $Oxyz$ , cho hai điểm $ A(2;-2;1)$, $B(0;1;2)$. Tọa độ điểm $ M$ thuộc mặt phẳng $\left(Oxy\right)$ sao cho ba điểm $ A,B, M$ thẳng hàng là
 \choice
 {\True $M\left(4\,;\,\,-5\,;\,\,0\right)$}
 {$M\left(2\,;\,\,-3\,;\,\,0\right)$}
 {$M\left(0\,;\,\,0\,;\,\,1\right)$}
 {$M\left(4\,;\,\,5\,;\,\,0\right)$}
 \loigiai{
 Chọn A.\\
 Gọi $ M\left(x\,;\,\,y\,;\,\,0\right)\in\left(Oxy\right)$; $\overrightarrow{AB}=\left(-2\,;\,\,3\,;\,\,1\right);\overrightarrow{AM}=\left(x-2\,;\,\,y+2\,;\,\,-1\right)$.\\
 Ba điểm $ A,B, M$ thẳng hàng $\Leftrightarrow $$\overrightarrow{AB}$ và $\overrightarrow{AM}$ cùng phương $\Leftrightarrow\dfrac{x-2}{-2}=\dfrac{y+2}{3}=\dfrac{-1}{1}$$\Leftrightarrow\left\{\begin{aligned}
 &\dfrac{x-2}{-2}=-1\\ 
 &\dfrac{y+2}{3}=-1\\ 
 \end{aligned}\right.\Leftrightarrow\left\{\begin{aligned}
 & x=4\\ 
 & y=-5\\ 
 \end{aligned}\right.$. Vậy $M\left(4\,;\,\,-5\,;\,\,0\right)$ .}
\end{ex}

\begin{ex}%Câu 7
 Giá trị nhỏ nhất của hàm số $ y=\dfrac{x^2+3}{x-1}$ trên đoạn $\left[2\,;\,\,4\right]$là
 \choice
 {\True $\underset{\left[2\,;\,\,4\right]}{\min}\,y=6$}
 {$\underset{\left[2\,;\,\,4\right]}{\min}\,y=-2$}
 {$\underset{\left[2\,;\,\,4\right]}{\min}\,y=-3$}
 {$\underset{\left[2\,;\,\,4\right]}{\min}\,y=\dfrac{19}{3}$}
 \loigiai{
 Chọn A.\\
 Ta có $y'=\dfrac{2x\left(x-1\right)-\left(x^2+3\right)}{x-1}=\dfrac{x^2-2x-3}{\left(x-1\right)^2}$; $y'=0\Rightarrow{x^2}-2x-3=0\Rightarrow\left[\begin{aligned}
 & x=-1\notin\left(2\,;\,\,4\right)\\ 
 & x=3\in\left(2\,;\,\,4\right)\\ 
 \end{aligned}\right.$.\\
 Ta có:$ y(2)=7$, $ y(3)=6$, $ y(4)=\dfrac{19}{3}$. Vậy $\underset{\left[2\,;\,\,4\right]}{\min}\,y=y(3)=6$.}
\end{ex}

\begin{ex}%Câu 8
 Tập nghiệm của bất phương trình $5^{x-1}\ge{5^{x^2-x-9}}$ là
 \choice
 {\True $\left[-2\,;\,\,4\right]$}
 {$\left[-4\,;\,\,2\right]$}
 {$\left(-\infty\,;\,\,-2\right]\cup\left[4\,;\,\,+\infty\right)$}
 {$\left(-\infty\,;\,\,-4\right]\cup\left[2\,;\,\,+\infty\right)$}
 \loigiai{
 Chọn A.\\
 Ta có: $5^{x-1}\ge{5^{x^2-x-9}}\Leftrightarrow x-1\ge{x^2}-x-9\Leftrightarrow{x^2}-2x-8\le 0\Leftrightarrow-2\le x\le 4$.\\
 Tập nghiệm của bất phương trình là $ S=\left[-2\,;\,\,4\right]$.}
\end{ex}

\begin{ex}%Câu 9
    Diện tích hình phẳng giới hạn bởi hai đường $y = x^2 - 1$ và $y = x - 1$ bằng
    \choice
    {$\dfrac{\pi}{6}$}
    {$\dfrac{13}{6}$}
    {$\dfrac{13\pi}{6}$}
    {\True $\dfrac{1}{6}$}
  \loigiai{
 Chọn D.}
\end{ex}
%
\begin{ex}%Câu 10
 Cho hàm số $y=\dfrac{x-2}{x+3}$ . Mệnh đề nào sau đây đúng?
 \choice
 {Hàm số nghịch biến trên khoảng $\left(-\infty\,;\,\,+\infty\right)$}
 {Hàm số nghịch biến trên từng khoảng $\left(-\infty\,;\,\,-3\right)$ và $\left(-3\,;\,\,+\infty\right)$}
 {\True Hàm số đồng biến trên từng khoảng $\left(-\infty\,;\,\,-3\right)$ và $\left(-3\,;\,\,+\infty\right)$}
 {Hàm số đồng biến trên khoảng $\left(-\infty\,;\,\,+\infty\right)$}
 \loigiai{
 Chọn C.\\
 Tập xác định hàm số $D=\mathbb{R}\setminus\left\{-3\right\}$ . Ta có $y'=\dfrac{5}{\left(x+3\right)^2}>0$ , $\forall x\in D$ .\\
 Vậy hàm số đồng biến trên các khoảng $\left(-\infty\,;\,\,-3\right)$ và $\left(-3\,;\,\,+\infty\right)$ .}
\end{ex}

\begin{ex}%Câu 11
 Cho hai biến cố A và B với $ P(A)=0,3;\,\,P(B)=0,4$ và $ P\left(AB\right)=0,2.$ Xác suất để A hoặc B xảy ra bằng
 \choice
 {$ 0,3$}
 {$ 0,4$}
 {$ 0,6$}
 {\True $ 0,5$}
 \loigiai{
 Chọn D.\\
 Ta có: $ P\left(A\cup B\right)=P(A)+P(B)-P\left(AB\right)=0,3+0,4-0,2=0,5$.}
\end{ex}

\begin{ex}%Câu 12
 Biết $ F(x)=x^2$ là một nguyên hàm của hàm số $ f(x)$ trên $\mathbb{R}$. Giá trị của $\displaystyle\int\limits_1^3\left[1+f(x)\right]\text{d}x$ bằng
 \choice
 {\True $ 10$}
 {$ 8$}
 {$\dfrac{26}{3}$}
 {$\dfrac{32}{3}$}
 \loigiai{
 Chọn A.\\
 Ta có $\displaystyle\int\limits_1^3\left[1+f(x)\right]\text{d}x=\left.\left[x+F(x)\right]\right|_1^3=\left.\left(x+x^2\right)\right|_1^3=12-2=10.$}
 \end{ex}
 
\Closesolutionfile{ans}
\cauds
\Opensolutionfile{ans}[ans/ans-HXN-\sode-TF]
% 
 \begin{ex}%Câu 13
 Khi Mặt Trăng quay quanh Trái Đất, mặt đối diện với Trái Đất thường chỉ được Mặt Trời chiếu sáng một phần. Các pha của Mặt Trăng mô tả mức độ phần bề mặt của nó được Mặt Trời chiếu sáng. Khi góc giữa Mặt Trời, Trái Đất và Mặt Trăng là $\alpha$ ($0^\circ \le \alpha \le 360^\circ$) thì tỉ lệ $F$ của phần Mặt Trăng được chiếu sáng cho bởi công thức $F = \dfrac{1}{2}(1-\cos\alpha)$.
 Biết rằng $F = 0$ khi có trăng mới; $F = 0,25$ khi có trăng lưỡi liềm; $F = 0,5$ khi có trăng bán nguyệt đầu tháng và cuối tháng; $F = 1$ khi trăng tròn. 
 \choiceTF
 {Khi có trăng mới thì $\alpha = 90^\circ$}
 {\True Khi có trăng lưỡi liềm thì $\alpha = 60^\circ$ hoặc $\alpha = 300^\circ$}
 {\True Khi có trăng bán nguyệt đầu tháng hoặc cuối tháng thì $\alpha = 90^\circ$ hoặc $\alpha = 270^\circ$}
 {\True Khi có trăng tròn thì $\alpha = 180^\circ$}
 \loigiai{
 \begin{listEX}
     \item Khi có trăng mới thì $F = \dfrac{1}{2}(1-\cos\alpha) = 0 \Rightarrow \cos\alpha = 1 \Rightarrow \alpha = k360^\circ$, $k \in \mathbb{Z}$; mà $0^\circ \le \alpha \le 360^\circ$ nên $\alpha \in \{0^\circ; 360^\circ\}$.
     \item Khi có trăng lưỡi liềm thì $F = \dfrac{1}{2}(1-\cos\alpha) = 0,25 \Rightarrow \cos\alpha = \dfrac{1}{2}$
     $\Rightarrow \begin{cases} \alpha = 60^\circ + k360^\circ \\ \alpha = -60^\circ + k360^\circ \end{cases}$ ($k \in \mathbb{Z}$); mà $0^\circ \le \alpha \le 360^\circ$ nên $\alpha \in \{60^\circ; 300^\circ\}$.
     \item Khi có trăng bán nguyệt đầu tháng hoặc cuối tháng thì
     $F = \dfrac{1}{2}(1-\cos\alpha) = \dfrac{1}{2} \Rightarrow \cos\alpha = 0 \Rightarrow \alpha = 90^\circ + k180^\circ$, $k \in \mathbb{Z}$;
     mà $0^\circ \le \alpha \le 360^\circ$ nên $\alpha \in \{90^\circ; 270^\circ\}$.
     \item Khi có trăng tròn thì $F = \dfrac{1}{2}(1-\cos\alpha) = 1 \Rightarrow \cos\alpha = -1 \Rightarrow \alpha = 180^\circ + k360^\circ$, $k \in \mathbb{Z}$; mà $0^\circ \le \alpha \le 360^\circ$ nên $\alpha = 180^\circ$.
 \end{listEX}}
 \end{ex}
% 
 \begin{ex}%Câu 14
     Hai nguồn sáng giống hệt nhau được đặt cách nhau 10 mét. Một vật sẽ được đặt tại một điểm $P$ nằm trên một đường thẳng $l$, song song với đường nối hai nguồn sáng và cách đường đó một khoảng $d$ mét (xem hình vẽ).
     Ta muốn đặt điểm $P$ trên đường $l$ sao cho cường độ chiếu sáng tại $P$ là nhỏ nhất. Ta cần biết rằng cường độ chiếu sáng tại điểm đơn lẻ tỉ lệ thuận với cường độ của nguồn và tỉ lệ nghịch với bình phương khoảng cách đến nguồn đó.
     Đặt hệ trục tọa độ $Oxy$ sao cho tâm $O$ trùng với nguồn sáng bên trái, tia $Ox$ chứa đoạn nối hai nguồn sáng, tia $Oy$ hướng lên trên, đơn vị trên mỗi trục là mét.
     \begin{center}
         \includegraphics[scale=1]{img/HXN-6.14}
     \end{center}
     Xét tính đúng sai các mệnh đề sau:
     \choiceTF
     {Khoảng cách từ $P$ đến các nguồn sáng là $r_1 = \sqrt{x^2 + d^2}$; $r_2 = \sqrt{(x+10)^2 + d^2}$}
     {\True Tổng cường độ chiếu sáng tại $P$ là $I(x) = k\left( \dfrac{1}{x^2+d^2} + \dfrac{1}{(x-10)^2+d^2} \right)$; với $k > 0$ là hằng số tỉ lệ}
     {\True Khi $d=5$ mét, cường độ ánh sáng tại $P$ đạt cực tiểu khi $x=5$ mét}
     {\True Khi $d=10$ mét, cường độ ánh sáng không đạt cực tiểu khi $P$ ở vị trí chính giữa của thanh $l$}
 \loigiai{
 \begin{listEX}
     \item Mệnh đề sai.
     Khoảng cách từ $P$ đến các nguồn sáng là $r_1 = \sqrt{x^2+d^2}$; $r_2 = \sqrt{(x-10)^2+d^2}$.
     \item Mệnh đề đúng.
     Cường độ chiếu sáng tại $P$ từ mỗi nguồn là:
     \begin{itemize}
         \item Từ nguồn $O$: $I_1 = \dfrac{k}{r_1^2} = \dfrac{k}{x^2+d^2}$.
         \item Từ nguồn bên phải: $I_2 = \dfrac{k}{r_2^2} = \dfrac{k}{(x-10)^2+d^2}$.
     \end{itemize}
     Tổng cường độ sáng tại $P$ là $I(x) = I_1+I_2 = k\left[\dfrac{1}{x^2+d^2} + \dfrac{1}{(x-10)^2+d^2}\right]$.
     \item Mệnh đề đúng.
     Khi $d=5$ thì $I(x) = k\left[\dfrac{1}{x^2+25} + \dfrac{1}{(x-10)^2+25}\right]$; $I'(x) = k\left[\dfrac{-2x}{(x^2+25)^2} + \dfrac{-2(x-10)}{((x-10)^2+25)^2}\right]$.
     Ta có $I'(x)=0 \Leftrightarrow k\left[\dfrac{-2x}{(x^2+25)^2} + \dfrac{-2(x-10)}{((x-10)^2+25)^2}\right] = 0 \Leftrightarrow x=5$.
     
     Bảng biến thiên:
     \begin{center}
         \begin{tabular}{|c|ccccc|}
             \hline
             $x$ & $0$ & & $5$ & & $10$ \\
             \hline
             $I'(x)$ & & $-$ & $0$ & $+$ & \\
             \hline
         \end{tabular}
     \end{center}
     Ta thấy cường độ ánh sáng tại $P$ đạt cực tiểu khi $x=5$ mét.
     \item Mệnh đề đúng.
     Khi $d=10$ thì $I(x) = k\left[\dfrac{1}{x^2+100} + \dfrac{1}{(x-10)^2+100}\right]$.
     \begin{itemize}
         \item Cách giải 1: Học sinh có thể làm như câu c): lập bảng xét dấu đạo hàm và kết luận.
         \item Cách giải 2: So sánh cường độ sáng tại các điểm đặc biệt.
         \begin{itemize}
             \item Tại $x=0$ thì $I(0) = k\left(\dfrac{1}{100} + \dfrac{1}{200}\right) = \dfrac{3k}{200}$.
             \item Tại $x=5$ thì $I(5) = k\left(\dfrac{1}{200} + \dfrac{1}{100}\right) = \dfrac{2k}{125}$. (Chỗ này trong ảnh là $I(5) = k(\dfrac{1}{200} + \dfrac{1}{100}) = \dfrac{2k}{125}$, nhưng $k(\dfrac{1}{200} + \dfrac{1}{100}) = k(\dfrac{1+2}{200}) = \dfrac{3k}{200}$. Nếu $I(5) = k(\dfrac{1}{125} + \dfrac{1}{25+100-100+25}) = k(\dfrac{1}{125}+\dfrac{1}{50})$? Kiểm tra lại đề gốc của $I(5)$ khi $d=10$. $I(5) = k\left[\dfrac{1}{5^2+100} + \dfrac{1}{(5-10)^2+100}\right] = k\left[\dfrac{1}{25+100} + \dfrac{1}{(-5)^2+100}\right] = k\left[\dfrac{1}{125} + \dfrac{1}{25+100}\right] = k\left[\dfrac{1}{125} + \dfrac{1}{125}\right] = \dfrac{2k}{125}$. Phần tính toán trong ảnh là đúng.)
         \end{itemize}
     \end{itemize}
     Vì $\dfrac{3k}{200} < \dfrac{2k}{125}$ nên cường độ sáng không đạt cực tiểu tại $x=5$ ($P$ không ở chính giữa $l$).
 \end{listEX}}
 \end{ex}
% 
 \begin{ex}%Câu 15
  Hộp A có 5 bi đỏ và 3 bi vàng, hộp B có 2 bi đỏ và 2 bi vàng, hộp C có 2 bi đỏ và 2 bi vàng. Lấy ngẫu nhiên 1 bi từ hộp A bỏ sang hộp B, rồi lấy ngẫu nhiên 2 bi từ hộp B bỏ sang hộp C, sau cùng lấy ngẫu nhiên 3 bi từ hộp C.
 Xét tính đúng sai các mệnh đề sau: 

 \choiceTF
 {Xác suất để lấy được 1 bi vàng từ hộp A bằng $\dfrac{5}{8}$}
 {Xác suất để lấy được 2 bi khác màu từ hộp B bằng $\dfrac{2}{5}$}
 {Xác suất để lấy được cả 3 bi màu đỏ từ hộp C bằng $\dfrac{1}{20}$}
 {\True Biết rằng 3 bi được lấy từ hộp C màu đỏ, xác suất để 3 bi đó vốn thuộc về 3 hộp khác nhau bằng $\dfrac{1}{6}$}
 \loigiai{
 a) Mệnh đề sai.
 Xác suất để lấy được 1 bi vàng từ hộp A bằng $\dfrac{3}{8}$.\\
 b) Mệnh đề sai.
 Ta có sơ đồ bài toán như hình bên. Gọi X là biến cố lấy được hai bi khác màu từ hộp B: $ P(X)=\dfrac{5}{8}\times\dfrac{3}{5}+\dfrac{3}{8}\times\dfrac{3}{5}=\dfrac{3}{5}$.\\
 c) Mệnh đề sai.
 Gọi Y là biến cố lấy được cả 3 bi đỏ từ hộp C, ta có: $ P(Y)=\dfrac{5}{8}\times\left(\dfrac{3}{10}\times\dfrac{C_4^3}{C_6^3}+\dfrac{3}{5}\times\dfrac{C_3^3}{C_6^3}\right)+\dfrac{3}{8}\times\left(\dfrac{1}{10}\times\dfrac{C_4^3}{C_6^3}+\dfrac{3}{5}\times\dfrac{C_3^3}{C_6^3}\right)=\dfrac{3}{40}$.\\
 d) Mệnh đề đúng.
 Gọi Z là biến cố: “Lấy từ hộp C 3 viên bi mà mỗi bi thuộc về mỗi hộp A, B, C trước đó”.\\
 Ta có: $ P\left(Z|Y\right)=\dfrac{P\left(ZY\right)}{P(Y)}=\dfrac{\dfrac{5}{8}\times\dfrac{1\times 2}{C_5^2}\times\dfrac{1\times 1\times 2}{C_6^3}}{\dfrac{3}{40}}=\dfrac{1}{6}$.}
 \end{ex}
%
 \begin{ex}
     \immini[thm]{ Trong một mô hình game 3D, với hệ trục tọa độ $Oxyz$ cho trước, đơn vị trên mỗi trục là mét, mặt đất là mặt phẳng $(Oxy)$. Người chơi cùng với khẩu súng của anh ta được xem như một chất điểm, xuất phát từ vị trí gốc tọa độ $O$ và di chuyển trên mặt đất. Hai mục tiêu nhắm bắn là các vị trí cố định $(5;5;5)$, $(5;10;10)$.
         Để tăng thêm độ hấp dẫn trò chơi, người ta dùng công nghệ hologram tạo ra một quả cầu giả lập bán kính thay đổi, có ánh sáng lấp lánh luôn đi qua hai điểm mục tiêu và lăn trên mặt đất, làm người chơi có phần hoa mắt và khó nhắm bắn trúng các mục tiêu này.
         Xét tính đúng sai các mệnh đề sau:
         }{\includegraphics[scale=1]{img/HXN-6.16}}
   \choiceTF
   {Mục tiêu gần nhất cách người chơi khoảng $8,5~m$ (làm tròn đến hàng phần chục)}
   {\True Biết viên đạn có thể bắn xuyên quả cầu, người chơi cần đứng vị trí $(5;0;0)$ để bắn trúng cùng lúc hai mục tiêu}
   {Gọi $M$ là điểm tiếp xúc của quả cầu với mặt đất thì $M$ luôn thuộc một đường tròn cố định có bán kính bằng $8~m$}
   {\True Khoảng cách ngắn nhất từ vị trí người chơi (gốc tọa độ $O$) đến điểm $M$ bằng $5~m$}
    \loigiai{
    \begin{listEX}
        \item Mệnh đề sai.
        Gọi $A(5;5;5)$, $B(5;10;10) \Rightarrow \overrightarrow{AB}=(0;5;5)=5(0;1;1)$.
        $OA = \sqrt{5^2+5^2+5^2} = 5\sqrt{3}$; $OB = \sqrt{5^2+10^2+10^2} = 15 > OA$.
        Do vậy mục tiêu $A$ gần người chơi nhất, có khoảng cách $OA = 5\sqrt{3} \approx 8,7~m$.
        
        \item Mệnh đề đúng.
        Người chơi cần đứng vị trí $I$ sao cho $I, A, B$ thẳng hàng để bắn trúng cùng lúc hai mục tiêu.
        $AB$ qua $A(5;5;5)$, vectơ chỉ phương $\vec{u}=(0;1;1)$ nên có phương trình $AB: \begin{cases} x=5 \\ y=5+t \\ z=5+t \end{cases}$.
        Gọi $I(5;5+t;5+t) \in AB$; $I \in (Oxy): z=0 \Rightarrow 5+t=0 \Rightarrow t=-5 \Rightarrow I(5;0;0)$.
        
        \item Mệnh đề sai.
        Vì $IM$ là tiếp tuyến của mặt cầu nên $IM^2 = IA \cdot IB$
        $= \sqrt{0^2+5^2+5^2} \cdot \sqrt{0^2+10^2+10^2} = 100$
        $\Rightarrow IM=10$.
        Do vậy $M$ luôn thuộc một đường tròn tâm $I$, bán kính $r=10~m$.
        (Hình vẽ minh họa một mặt cầu tiếp xúc với mặt phẳng $(Oxy)$ tại $M$, với $A, B$ là hai điểm trên mặt cầu, $I$ là tâm đường tròn ngoại tiếp tam giác $ABM'$ với $M'$ là hình chiếu của $M$ trên $AB$, hoặc $I$ là điểm trên $(Oxy)$ sao cho $IA, IB$ là tiếp tuyến tới đường tròn $(M, R)$.)
        
        \item Mệnh đề đúng.
        Ta có $OI=5 < 10 = R$. Vì vậy khoảng cách từ $O$ đến tâm quả cầu ngắn nhất khi $I, O, M$ thẳng hàng theo thứ tự đó.
        Ta có $OM_{\min} = R - OI = 10-5=5$.
        Vậy khoảng cách ngắn nhất từ vị trí người chơi đến điểm tiếp xúc quả cầu, mặt đất bằng $5~m$.
        (Hình vẽ minh họa một đường tròn trên mặt phẳng $(Oxy)$ tâm $I$ bán kính $R$. Điểm $O$ nằm trong đường tròn. $M$ là điểm trên đường tròn sao cho $O, M, I$ thẳng hàng và $OM$ ngắn nhất.)
    \end{listEX}
    \begin{center}
        \includegraphics[scale=.7]{img/HXN-6.16a}
    \end{center}
    }
\end{ex}
\Closesolutionfile{ans}
\caukq
\Opensolutionfile{ans}[ans/ans-HXN-\sode-SA]
% 
 \begin{ex}%Câu 17
 Mai là một cô gái dễ mến, tuy nhiên trong chuyện tình cảm thì cô không phải là người đơn giản. Hôm ấy có người bạn trai hẹn đi ăn trưa, Mai cho biết cô sẽ gặp người đó vào thời điểm kim giờ và kim phút gặp nhau lần đầu tiên kể từ sau 12h trưa. Nếu người bạn trai ấy đến địa điểm hẹn vào lúc 12h30 trưa thì anh ấy sẽ phải chờ Mai bao nhiêu phút (làm tròn đến hàng phần chục)?
 \shortans{35,5}
 \loigiai{
 Mỗi phút kim giờ quét được một góc $\dfrac{2\pi}{60.12}=\dfrac{\pi}{360}$ (rad).\\
 Mỗi phút kim phút quét được một góc $\dfrac{2\pi}{60}=\dfrac{\pi}{30}$ (rad).\\
 Gọi t là thời gian (phút) mà kim giờ và kim phút gặp nhau kể từ 12h trưa.\\
 Ta có $\dfrac{\pi}{360}t+2\pi=\dfrac{\pi}{30}t\Rightarrow t\approx 65,5$ (phút).\\
 Người bạn trai đến điểm hẹn lúc 12h30 nên phải chờ Mai khoảng 35,5 phút.}
 \end{ex}
 
 \begin{ex}%Câu 18
     \immini[thm] {Một vật đựng đồ chơi lắp ghép của trẻ con có dạng hình trụ với chiều cao bằng 15 cm. Người ta nhìn vào bên trong hình trụ này thì thấy có các mảnh ghép hình vuông được đặt vừa khít như hình vẽ. Biết mỗi mảnh ghép hình vuông có cạnh 2 cm. Thể tích vật đựng hình trụ này là bao nhiêu $c{m^3}$ (làm tròn đến hàng đơn vị, bỏ qua độ dày của vỏ hộp).  \shortans{589}}{  \includegraphics[scale=.8]{img/HXN-6.18}}

 \loigiai{
 Dựng hệ trục Oxy như hình vẽ, mỗi hình vuông cạnh 1 đơn vị tương ứng với 2 cm). Gọi phương trình đường tròn đáy hình trụ là $x^2+y^2-2ax-2by+c=0$ với $a^2+b^2-c>0$ .\\
 Các điểm $\left(-4\,;\,\,0\right)\,,\,\,\left(-4\,;\,\,1\right)\,,\,\,\left(2\,;\,\,-2\right)$ thuộc đường tròn đáy nên $\left\{\begin{aligned}
 & 16+0+8a-0+c=0\\ 
 & 16+1+8a-2b+c=0\\ 
 & 4+4-4a+4b+c=0\\ 
 \end{aligned}\right.\Leftrightarrow\left\{\begin{aligned}
 & a=-\dfrac{1}{2}\\ 
 & b=\dfrac{1}{2}\\ 
 & c=-12\\ 
 \end{aligned}\right.$ .\\
 Bán kính đáy hình trụ là: $\sqrt{a^2+b^2-c}=\dfrac{5\sqrt{2}}{2}$ ; bán kính thực tế: $R=\dfrac{5\sqrt{2}}{2}\times 2=5\sqrt{2}$ cm.\\
 Thể tích vật đựng hình trụ là $V=\pi{R^2}h=\pi{\left(5\sqrt{2}\right)^2}\cdot 15=750\pi\approx\,\,c{m^3}$ .}
 \end{ex}
 
 \begin{ex}%Câu 19
     \immini[thm]{ Khi ca sĩ ST bước ra sân khấu, có một đèn pha luôn chiếu sáng vào anh. Đèn pha được đặt ở vị trí B, cách đoạn đường mà ca sĩ đang đi một khoảng BH bằng 10 m. Biết rằng ST Cát đi ra với tốc độ 1 m/s, khi ca sĩ cách điểm H trên con đường khoảng 6 m thì đèn pha đang quay với tốc độ bao nhiêu rad/s (làm tròn đến hàng phần trăm)?
         \shortans{0,07}}{\includegraphics[scale=1]{img/HXN-6.19}}

 \loigiai{
 Gọi $\varphi=\widehat{ABH}$ với $0\le\varphi <\dfrac{\pi}{2}\,\,\,(rad)$ ; đăt $x=HA$ (thay đổi).\\
 Tam giác ABH vuông tại H có $\tan\varphi=\dfrac{AH}{BH}=\dfrac{x}{10}\Rightarrow\,\,\,(1)$ .\\
 Đạo hàm hai vế của (1) theo t, ta được: $\,\,\,(2)$ .\\
 Khi $x=6\,\,m$ thì (1) trở thành $6=10\tan\varphi\Rightarrow\varphi\approx 0,54\,\,rad$ (lưu vào A).\\
 Thay lần lượt $\varphi\approx 0,54\,\,rad$ và $\dfrac{\text{d}x}{\text{d}t}=1$ m/s vào (2) ta được $\dfrac{\text{d}\varphi}{\text{d}t}=\dfrac{5}{68}\approx $ rad/s.}
\end{ex}

\begin{ex}%Câu 20
    \immini[thm]
{ Một con châu chấu nhảy lên cầu thang có 18 bậc. Mỗi lần nhảy con châu chấu có thể nhảy 1 bậc hoặc 2 bậc. Tính xác suất để con châu chấu hoàn thành 18 bậc thang với số lần nhảy 2 bậc không bé hơn số lần nhảy 1 bậc (làm tròn kết quả đến hàng phần trăm).
 \shortans{0,31}}{\includegraphics[scale=1]{img/HXN-6.20}}
\loigiai{
 Gọi x là số bước nhảy 1 bậc, y là số bước nhảy 2 bậc; suy ra $\left\{\begin{aligned}
 & x+2y=18\\ 
 & x\,,\,\,y\in\mathbb{N}\\ 
 \end{aligned}\right.$ .\\
 Các cặp (x ; y) thỏa mãn là $\left(18\,;\,\,0\right)\,,\,\,\left(16\,;\,\,1\right)\,,\,\,\left(14\,;\,\,2\right)\,,\,\,\left(12\,;\,\,3\right)\,,\,\,\left(10\,;\,\,4\right)$ , $\left(8\,;\,\,5\right)\,,\,\,\left(6\,;\,\,6\right)\,,\,\,\left(4\,;\,\,7\right)\,,\,\,\left(2\,;\,\,8\right)\,,\,\,\left(0\,;\,\,9\right)$ .\\
 • Nếu $\left(x\,;\,\,y\right)\in\left\{\left(18\,;\,\,0\right)\,,\,\,\left(0\,;\,\,9\right)\right\}$ thì con châu chấu có 2 cách đi.\\
 • Nếu $\left(x\,;\,\,y\right)=\left(16\,;\,\,1\right)$ thì số cách đi của châu chấu là $C_{17}^1$ .\\
 • Tương tự như vậy các trường hợp còn lại sẽ có số cách là $C_{16}^2+C_{15}^3+C_{14}^4+C_{13}^5+C_{12}^6+C_{11}^7+C_{10}^8=4\,162$ (cách).\\
 Vậy tổng số cách nhảy của châu chấu để thoàn thành 18 bậc cầu thang là $2+C_{17}^1+4162=$ (cách). Số phần tử không gian mẫu là $n\left(\Omega\right)=$ .\\
 Số lần nhảy 2 bậc không bé hơn số lần nhảy 1 bậc nên $\left(x\,;\,\,y\right)\in\left\{\left(6\,;\,\,6\right)\,,\,\,\left(4\,;\,\,7\right)\,,\,\,\left(2\,;\,\,8\right)\,,\,\,\left(0\,;\,\,9\right)\right\}$ .\\
 Gọi A là biến cố thỏa đề bài thì $n(A)=C_{12}^6+C_{11}^7+C_{10}^8+C_9^9=$ .\\
 Do vậy $P(A)=\dfrac{n(A)}{n\left(\Omega\right)}=\dfrac{1\,300}{4\,180}=\dfrac{65}{209}\approx $ .}
 \end{ex}
 
 \begin{ex}%Câu 21
Ông Vượng mới khai phá được một mảnh đất hình chữ nhật, nhà nước chưa cấp sổ nên ông cũng chưa biết rõ diện tích mảnh đất là bao nhiêu, chỉ nhớ rằng bản thân là học sinh giỏi toán 12 năm liền thời phổ thông mà thôi. Mảnh đất của ông Vượng nằm ở một vị trí thuận lợi để trồng trọt vì có một dòng suối nhỏ chảy qua với hình dáng một parabol, dòng suối nhỏ này chia mảnh đất ra làm hai phần có diện tích $S_1\,,\,\,S_2\,\,\,\left(S_1>S_2\right)$ . Riêng mảnh đất có diện tích $S_2$ được xem như hình phẳng giới hạn bởi parabol cùng hai tiếp tuyến vuông góc của parabol đó.
Vào vụ Hè thu, ông Vượng quyết định trồng lúa trên phần đất có diện tích $S_1$ và trồng ớt trên phần đất có diện tích $S_2$ . Dự kiến lợi nhuận mang lại từ việc trồng lúa là $30$ nghìn/ $m^2$ và lợi nhuận từ việc trồng ớt là $40$ nghìn/ $m^2$ (trong một vụ mùa).
Ông quyết định dựng hệ trục Oxy như hình vẽ với gốc O trùng với điểm cực trị của dòng suối dạng parabol, đơn vị trên mỗi trục là 100 mét.
\begin{center}
    \includegraphics[scale=.5]{img/HXN-6.21}
\end{center}
 Tính tổng lợi nhuận theo dự kiến của ông Vượng sau vụ Hè thu này (làm tròn đến hàng đơn vị của triệu đồng), biết rằng diện tích con suối không đáng kể.
\shortans{309}

\loigiai{
Đầu tiên ta đặt $ A\left(a\,;\,\,a^2\right)\,,\,\,B\left(b\,;\,\,b^2\right)\,,\,\,a>0\,,\,\,b<0$ là hai tiếp điểm ứng với hai tiếp tuyến vuông góc của parabol (P).\\
Gọi $d_1\,,\,\,d_2$ lần lượt là các tiếp tuyến của đồ thị $(C)$ tại $ A\,,\,\,B$, khi đó: $\left\{\begin{matrix}
 {d_1}:y=2ax-a^2\\
 {d_2}:y=2bx-b^2\\
\end{matrix}\right.$.\\
Do $d_1\bot{d_2}$ nên $ 2a\cdot 2b=-1\Rightarrow b=-\dfrac{1}{4a}\Rightarrow B\left(-\dfrac{1}{4a}\,;\,\,\dfrac{1}{16a^2}\right)$, khi đó $d_2:y=-\dfrac{x}{2a}-\dfrac{1}{16a^2}$.\\
Gọi $ E=d_2\cap{d_1}$, suy ra $ E\left(\dfrac{4a^2-1}{8a}\,;\,\,-\dfrac{1}{4}\right)$; $ EA=\dfrac{\sqrt{\left(4a+1\right)^3}}{8a}\,;\,\,EB=\dfrac{\sqrt{\left(4a+1\right)^3}}{16a^2}$.\\
Ta có $ EA=2EB$; suy ra $ a=1$. Do đó diện tích mảnh đất $ S=EA\cdot EB=\dfrac{\left(4a+1\right)^3}{128a^3}\left|\begin{aligned}
 &\\ 
 & a=1\\ 
\end{aligned}\right.=$.\\
Khi đó phương trình $d_1:y=2x-1\,;\,\,d_2:y=-\dfrac{x}{2}-\dfrac{1}{16}$ và $ A\left(1\,;\,\,1\right)\,,\,\,B\left(-\dfrac{1}{4}\,;\,\,\dfrac{1}{16}\right)\,,\,\,E\left(\dfrac{3}{8}\,;\,\,-\dfrac{1}{4}\right)$.\\
Diện tích $S_2=\displaystyle\int\limits_{-1/4}^{3/8}{\left[x^2-\left(-\dfrac{x}{2}-\dfrac{1}{16}\right)\right]\text{d}x}+\displaystyle\int\limits_{3/8}^1\left[x^2-\left(2x-1\right)\right]\text{d}x=$ và $S_1=S-S_2=$.\\
Tổng số tiền thu được của ông Vượng sau vụ hè thu bằng $\dfrac{625}{768}\times{10^2}\times 30+\dfrac{125}{768}\times{10^2}\times 40\approx 309\,244$ nghìn đồng $\approx $ triệu đồng.}
\end{ex}

\begin{ex}%Câu 22
    \immini[thm]{Trong không gian $ Oxyz$, cho mặt cầu $(S)$ có tâm $ I\left(-1\,;\,\,0\,;\,\,2\right)$ và đi qua điểm $ A\left(0\,;\,\,1\,;\,\,1\right)$. Xét các điểm $ B\,,\,\,C\,,\,\,D$ thuộc $(S)$ sao cho $ AB,\,\,AC,\,\,AD$ đôi một vuông góc với nhau. Thể tích của khối tứ diện $ ABCD$ có giá trị lớn nhất bằng bao nhiêu (làm tròn đền hàng phần trăm)?
        \shortans{1,33}}{\includegraphics[scale=.7]{img/HXN-6.22}}
\loigiai{
Ta nhận diện được đây là bài toán mặt cầu ngoại tiếp tứ diện có ba cạnh đôi một vuông góc nhau. Bán kính mặt cầu là $ R=IA=\sqrt{3}$.\\
Do $ AB,AC,AD$ đôi một vuông góc với nhau nên $ R=\dfrac{\sqrt{A{B^2}+A{C^2}+A{D^2}}}{2}$.\\
Suy ra $ A{B^2}+A{C^2}+A{D^2}=4R^2=12$.\\
Thể tích tứ diện: $$ V_{ABCD}=\dfrac{1}{6}AB.AC.AD=\dfrac{1}{6}\sqrt{A{B^2}.A{C^2}.A{D^2}}\overset{AM-GM}{\mathop{\le}}\,\dfrac{1}{6}\sqrt{\left(\dfrac{A{B^2}+A{C^2}+A{D^2}}{3}\right)^3}=\dfrac{1}{6}\sqrt{\left(\dfrac{12}{3}\right)^3}=\dfrac{4}{3}$$
Do đó $\left(V_{ABCD}\right)_{Max}=\dfrac{4}{3} $. Dấu đẳng thức xảy ra khi và chỉ khi $ AB=AC=AD=2$.}
\end{ex}
\Closesolutionfile{ans}
\inputansbox{6,4,3}{ans/ans-HXN-\sode-T,ans/ans-HXN-\sode-TF,ans/ans-HXN-\sode-SA}