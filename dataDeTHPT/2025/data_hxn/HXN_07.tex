\def\sode{7}
\begin{name}
	{\tenchude}
	{\tendethi}
	{\tentruong}
	{\thoigian}
\end{name}
\caulc
\Opensolutionfile{ans}[ans/ans-HXN-\sode-T]
\begin{ex}%Câu 1
 \immini[thm]{ Cho hàm số $y=f(x)$ liên tục trên $\left[-1\,;+\infty\right)$ và có đồ thị như hình vẽ. Tìm giá trị lớn nhất của hàm số $y=f(x)$ trên $\left[1\,;\,\,4\right]$ .
 \choice
 {0}
 {1}
 {4}
 {\True 3}}{\includegraphics[scale=.8]{img/HXN-7.1}}
 \loigiai{
 Chọn D.}
\end{ex}
\begin{ex}%Câu 2
 Cho $\log_ab=2$ (với $a>0\,,\,\,b>0\,,\,\,a\ne 1$). Tính $\log_a\left(a\cdot b\right)$ .
 \choice
 {$2$}
 {$4$}
 {$5$}
 {\True $3$}
 \loigiai{
 Chọn D.\\
 Ta có: $\log_a\left(a\cdot b\right)=\log_aa+\log_ab=1+2=3$ .}
\end{ex}
\begin{ex}%Câu 3
 Trong không gian $Oxyz$ , cho mặt cầu $(S):{(x-1)^2}+(y+2)^2+(z-3)^2=16$ . Tâm của $(S)$ có tọa độ là
 \choice
 {$\left(-1\,;\,\,-2\,;\,\,-3\right)$}
 {$\left(1\,;\,\,2\,;\,\,3\right)$}
 {$\left(-1\,;\,\,2\,;\,\,-3\right)$}
 {\True $\left(1\,;\,\,-2\,;\,\,3\right)$}
 \loigiai{
 Chọn D.\\
 Mặt cầu (S) có tâm $I\left(1\,;\,\,-2\,;\,\,3\right)$ , bán kính $R=4$ .}
\end{ex}
\begin{ex}%Câu 4
 Hai đường tiệm cận của đồ thị hàm số $y=\dfrac{2x+1}{x-1}$ tạo với hai trục tọa độ một hình chữ nhật có diện tích bằng bao nhiêu?
 \choice
 {\True $2$}
 {$1$}
 {$3$}
 {$4$}
 \loigiai{
 Chọn A.\\
 Đồ thị hàm số $y=\dfrac{2x+1}{x-1}$ có đường tiệm cận đứng $x=1$ và đường tiệm cận ngang $y=2$ .\\
 Gọi A là giao điểm của tiệm cận đứng với Ox, suy ra $OA=1$ .\\
 Gọi B là giao điểm của tiệm cận ngang với Oy, suy ra $OB=2$ .\\
 Hình chữ nhật cần tính diện tích là hình chữ nhật OAIB với $I\left(1\,;\,\,2\right)$ là tâm đối xứng đồ thị.\\
 Diện tích hình chữ nhật ABCD là $S=OA\cdot OB=2$ .}
\end{ex}
\begin{ex}
 Doanh thu bán hàng trong 20 ngày được lựa chọn ngẫu nhiên của một cửa hàng được ghi lại ở bảng sau (đơn vị: triệu đồng):
 \begin{center}
\begin{tabular}{|c|c|c|c|c|c|}
    \hline
    Doanh thu & $[5;7)$ & $[7;9)$ & $[9;11)$ & $[11;13)$ & $[13;15)$ \\
    \hline
    Số ngày & 2 & 7 & 7 & 3 & 1 \\
    \hline
\end{tabular}
 \end{center}
 Giá trị trung bình của mẫu số liệu ghép nhóm trên thuộc khoảng nào sau đây?
 \choice
 {$[7; 9)$}
 {\True $[9; 11)$}
 {$[11; 13)$}
 {$[13; 15)$}
 \loigiai{
 Chọn B.
 Ta viết lại mẫu số liệu ghép nhóm có thêm giá trị đại diện như sau:
 \begin{center}
 \begin{tabular}{|c|c|c|c|c|c|}
 \hline
 Doanh thu & $[5;7)$ & $[7;9)$ & $[9;11)$ & $[11;13)$ & $[13;15)$ \\
 \hline
 Giá trị đại diện & 6 & 8 & 10 & 12 & 14 \\
 \hline
 Số ngày & 2 & 7 & 7 & 3 & 1 \\
 \hline
 \end{tabular}
 \end{center}
 Giá trị trung bình của mẫu số liệu ghép nhóm là
 $ \bar{x} = \dfrac{2 \cdot 6 + 7 \cdot 8 + 7 \cdot 10 + 3 \cdot 12 + 1 \cdot 14}{20} = \dfrac{12 + 56 + 70 + 36 + 14}{20} = \dfrac{188}{20} = 9,4 $.
 Vì $9,4 \in [9;11)$ nên đáp án đúng là B.
 }
\end{ex}
\begin{ex}%Câu 6
 Cho cấp số nhân $\left(u_n\right)$ với $u_1=2$ và công bội $q=3$ . Tìm số hạng thứ $4$ của cấp số nhân?
 \choice
 {$24$}
 {\True $54$}
 {$162$}
 {$48$}
 \loigiai{
 Chọn B.\\
 Ta có: $u_4=u_1\cdot{q^3}=2\cdot{3^3}=54.$}
\end{ex}
\begin{ex}%Câu 7
 Cho hai hàm số $y=f(x)$ và $y=g(x)$ liên tục trên $\left[a\,;\,\,b\right]$ . Diện tích hình phẳng giới hạn bởi đồ thị của các hàm số $y=f(x)$ , $y=g(x)$ và các đường thẳng $x=a$ , $x=b$ bằng
 \choice
 {$\left|\displaystyle\int\limits_a^b{\left[f(x)-g(x)\right]\text{d}x}\right|$}
 {$\displaystyle\int\limits_a^b{\left| f(x)+g(x)\right|\text{d}x}$}
 {\True $\displaystyle\int\limits_a^b{\left| f(x)-g(x)\right|\text{d}x}$}
 {$\displaystyle\int\limits_a^b{\left[f(x)-g(x)\right]\text{d}x}$}
 \loigiai{
 Chọn C.}
\end{ex}
\begin{ex}%Câu 8
 Điểm kiểm tra 15 phút của lớp 12A được cho bởi bảng sau:\\
 \centerline{\begin{tabular}{|c|c|c|c|c|c|c|c|}
 \hline
 Điểm &[3; 4) &[4; 5) &[5; 6) &[6; 7) &[7; 8) &[8; 9) &[9; 10)\\
 \hline
 Số học sinh & 3 & 8 & 7 & 12 & 7 & 1 & 1\\
 \hline
 \end{tabular}}\\
 Tứ phân vị thứ nhất của mẫu số liệu ghép nhóm trên (làm tròn đến hàng phần trăm) là
 \choice
 {\True $4,84$}
 {$2,10$}
 {$2,09$}
 {$6,94$}
 \loigiai{
 Chọn A.\\
 Mẫu số liệu ghép nhóm có cỡ mẫu là $n=3+8+7+12+7+1+1=39$ .\\
 Tứ phân vị thứ nhất của mẫu số liệu gốc là $x_{10}\in\left[4\,;\,\,5\right)$ ; do đó tứ phân vị thứ nhất của mẫu số liệu ghép nhóm là $Q_1=4+\dfrac{\dfrac{39}{4}-3}{8}.1=\dfrac{155}{32}\approx 4,84$ .}
\end{ex}
\begin{ex}%Câu 9
 Trong không gian $Oxyz$ , cho đường thẳng $\Delta :\left\{\begin{aligned}
 & x=1-2t\\ 
 & y=-1\\ 
 & z=3+t\\ 
 \end{aligned}\right.$ . Vectơ nào sau đây là một vectơ chỉ phương của đường thẳng $\Delta $ ?
 \choice
 {$(-2\,;\,\,-1\,;\,\,1)$}
 {$(1\,;\,\,-1\,;\,\,3)$}
 {\True $(-2\,;\,\,0\,;\,\,1)$}
 {$(2\,;\,\,0\,;\,\,1)$}
 \loigiai{
 Chọn C.\\
 $\Delta $ có vectơ chỉ phương $\vec{u}=\left(-2\,;\,\,0\,;\,\,1\right)$ .}
\end{ex}
\begin{ex}%Câu 10
 Cho tích phân $\displaystyle\int\limits_0^1\left[f(x)+2x\right]\text{d}x=2$ . Khi đó tích phân $\displaystyle\int\limits_0^1f(x)\text{d}x$ bằng ?
 \choice
 {\True $1$}
 {$4$}
 {$2$}
 {$0$}
 \loigiai{
 Chọn A.\\
 Ta có: $\displaystyle\int\limits_0^1\left[f(x)+2x\right]\text{d}x=2\Leftrightarrow\displaystyle\int\limits_0^1f(x)\text{d}x+\left.x^2\right|_0^1=2\Leftrightarrow\displaystyle\int\limits_0^1f(x)\text{d}x+1=2\Leftrightarrow\displaystyle\int\limits_0^1f(x)\text{d}x=1$ .}
\end{ex}
\begin{ex}%Câu 11
 Trong không gian $Oxyz$ , cho ba điểm $A\left(1\,;\,\,1\,;\,\,1\right)$ , $B\left(0\,;\,\,2\,;\,\,1\right)$ và $C\left(1\,;\,\,-1\,;\,\,2\right)$ . Mặt phẳng đi qua $A$ và vuông góc với $BC$ có phương trình là
 \choice
 {$\dfrac{x+1}{1}=\dfrac{y+1}{-3}=\dfrac{z+1}{1}$}
 {$x-3y+z-1=0$}
 {\True $x-3y+z+1=0$}
 {$\dfrac{x-1}{1}=\dfrac{y-1}{-3}=\dfrac{z-1}{1}$}
 \loigiai{
 Chọn C.\\
 Mặt phẳng qua $A\left(1\,;\,\,1\,;\,\,1\right)$ , có vectơ pháp tuyến $\overrightarrow{BC}=\left(1\,;\,\,-3\,;\,\,1\right)$ nên có phương trình\\
 $1(x-1)-3(y-1)+1(z-1)=0\Leftrightarrow x-3y+z+1=0$ .}
\end{ex}
\begin{ex}%Câu 12
 Họ nguyên hàm của hàm số $f(x)=e^{2x}+\dfrac{3}{x}$ là
 \choice
 {$\mathop{\displaystyle\int}f(x)\text{d}x=e^{2x}+3\text{ln}x+C$}
 {\True $\mathop{\displaystyle\int}f(x)\text{d}x=\dfrac{e^{2x}}{2}+3\text{ln}\left| x\right|+C$}
 {$\mathop{\displaystyle\int}f(x)\text{d}x=\dfrac{e^{2x}}{2}+3\text{ln}x+C$}
 {$\mathop{\displaystyle\int}f(x)\text{d}x=e^{2x}+3\text{ln}\left| x\right|+C$}
 \loigiai{
 Chọn B.\\
 Ta có: $\mathop{\displaystyle\int}f(x)\text{d}x=\mathop{\displaystyle\int}\left(e^{2x}+\dfrac{3}{x}\right)\text{d}x=\dfrac{e^{2x}}{2}+3\text{ln}\left| x\right|+C$ .}
 \end{ex}
 \Closesolutionfile{ans}
 \cauds
 \Opensolutionfile{ans}[ans/ans-HXN-\sode-TF]
\begin{ex}
 Cho hàm số $f(x) = \begin{cases} 3 & \text{khi } x \le 1 \\ ax+b & \text{khi } 1 < x < 2 \\ 5 & \text{khi } x \ge 2 \end{cases}$.
 Xét tính đúng sai các mệnh đề sau:
 \choiceTF
 {Hàm số liên tục trên khoảng $(-\infty; 1)$}
 {Hàm số không liên tục trên khoảng $(1; 2)$}
 {Hàm số liên tục tại $x=1$ khi $a+b=5$}
 {\True Hàm số liên tục trên $\mathbb{R}$ khi và chỉ khi $a=2, b=1$}
 \loigiai{
 \begin{listEX}
 \item Mệnh đề đúng.
 Khi $x<1$ thì $f(x)=3$ là hàm hằng số nên $f(x)$ liên tục trên $(-\infty; 1)$.
 \item Mệnh đề sai.
 Khi $x \in (1;2)$ thì $f(x)=ax+b$ là hàm số bậc nhất (nếu $a$ khác $0$) hoặc là hàm số không đổi (nếu $a=0$), do đó $f(x)$ liên tục trên $(1;2)$.
 \item Mệnh đề sai.
 Ta có: $\lim_{x \to 1^+} f(x) = 3$; $f(1)=3$; $\lim_{x \to 1^-} f(x) = \lim_{x \to 1^-} (ax+b) = a+b$.
 Hàm số liên tục tại $x=1$ suy ra $\lim_{x \to 1^+} f(x) = \lim_{x \to 1^-} f(x) = f(1) \Rightarrow a+b=3$.
 \item Mệnh đề đúng.
 Dễ thấy hàm số $f(x)$ liên tục trên các khoảng $(-\infty;1)$, $(1;2)$ và $(2;+\infty)$. Vì vậy hàm số liên tục trên $\mathbb{R}$ khi và chỉ khi hàm số liên tục tại các điểm $x=1; x=2$.
 Ta có: $\lim_{x \to 2^-} f(x) = \lim_{x \to 2^-} (ax+b) = 2a+b$; $\lim_{x \to 2^+} f(x)=5$; $f(2)=5$.
 Hàm số liên tục tại $x=2$ suy ra $\lim_{x \to 2^-} f(x) = \lim_{x \to 2^+} f(x) = f(2) \Rightarrow 2a+b=5$.
 Kết hợp với câu c) ta có hệ phương trình $\begin{cases} a+b=3 \\ 2a+b=5 \end{cases} \Leftrightarrow \begin{cases} a=2 \\ b=1 \end{cases}$.
 \end{listEX}
 }
\end{ex}
\begin{ex}
 Trên một vùng cao nguyên rộng lớn, với hệ tọa độ $Oxyz$ thích hợp, đơn vị trên mỗi trục tọa độ là 5 mét, một con đại bàng đang đậu trên vách đá phẳng được mô hình hóa bởi phương trình $(P): 2x+2y-z+9=0$. Con đại bàng này đang ngắm các mục tiêu là hai con dê núi đang ở các vị trí $A(1;2;-3)$ và $B(-2;-2;1)$.
 \choiceTF
 {\True Con dê ở vị trí $B$ thuộc vách núi đá nơi đại bàng đang đậu}
 {Khoảng cách giữa hai con dê núi là $\sqrt{41}$ mét}
 {Khoảng cách ngắn nhất từ đại bàng đến con dê ở vị trí $A$ bằng 32 mét}
 {\True Đại bàng luôn quan sát hai con dê với một góc $90^\circ$ và con dê ở vị trí $B$ cũng đã biết được sự nguy hiểm sau lưng nó; khoảng cách xa nhất giữa nó với đại bàng bằng 11,2 mét (làm tròn đến hàng phần chục)}
 \loigiai{
 \begin{listEX}
 \item Mệnh đề đúng.
 Thay tọa độ $B$ vào phương trình $(P): 2x+2y-z+9=0$ thì $2(-2)+2(-2)-(1)+9 = -4-4-1+9=0$ (thỏa mãn).
 Do đó con dê ở vị trí $B$ thuộc vách núi đá nơi đại bàng đang đậu.
 \item Mệnh đề sai.
 Ta có: $\overrightarrow{AB}=(-3;-4;4) \Rightarrow AB = \sqrt{(-3)^2+(-4)^2+4^2} = \sqrt{9+16+16} = \sqrt{41}$.
 Khoảng cách thực tế hai con dê là $5 \cdot AB = 5\sqrt{41}$ mét.
 \item Mệnh đề sai.
 Gọi $H$ là hình chiếu của $A$ trên $(P)$. Đường thẳng $AH$ đi qua $A(1;2;-3)$ và nhận $\vec{n_P}=(2;2;-1)$ làm vectơ chỉ phương.
 Phương trình $AH: \begin{cases} x=1+2t \\ y=2+2t \\ z=-3-t \end{cases}$.\\
 Vì $H \in AH$, tọa độ $H$ có dạng $(1+2t; 2+2t; -3-t)$.\\
 Mà $H \in (P)$ nên $2(1+2t)+2(2+2t)-(-3-t)+9=0 \Rightarrow 2+4t+4+4t+3+t+9=0 \Rightarrow 9t+18=0 \Rightarrow t=-2$.
$ \Rightarrow H(-3;-2;-1)$.
 Khoảng cách ngắn nhất từ đại bàng (trên vách đá $P$) đến con dê $A$ chính là khoảng cách từ $A$ đến mặt phẳng $(P)$, tức là $5 \cdot AH = 5 \cdot 6 = 30$ mét.
 \item Mệnh đề đúng.
 Gọi $M$ là vị trí đại bàng trên vách đá (mặt phẳng $(P)$).
 $\begin{cases} BM \perp AH \\ BM \perp AM \end{cases} \Rightarrow BM \perp (AMH) \Rightarrow BM \perp MH$. Do đó $BM \le BH = \sqrt{5}$\\
Vậy khoảng cách lớn nhất là $5 \cdot BH = 5\sqrt{5} \approx 11,2$ mét.
  \end{listEX}
 }
\end{ex}
\begin{ex}
 Cho ba biến cố $A, B, C$, trong đó các cặp biến cố $A$ và $C$ là độc lập, $B$ và $C$ là độc lập, $A$ và $B$ là xung khắc.
 Biết rằng $P(A \cup C) = \dfrac{2}{3}$, $P(B \cup C) = \dfrac{3}{4}$, $P(A \cup B \cup C) = \dfrac{11}{12}$; đặt $a=P(A), b=P(B), c=P(C)$.
 \choiceTF
 {\True $P(A \cap C) = P(A) \cdot P(C)$; $P(A \cap B) = P(A)+P(B)$}
 {$a+c = \dfrac{2}{3}$; $b+c = \dfrac{3}{4}$}
 {$a+b+c-ac = \dfrac{11}{12}$}
 {Xác suất để $A$ xảy ra nếu $B$ hay $C$ xảy ra bằng $\dfrac{1}{9}$}
 \loigiai{
 \includegraphics[scale=.8]{img/HXN-7.15}
 \begin{listEX}
     \item Mệnh đề đúng.\\
     Vì $A, C$ độc lập nên $P(A \cap C) = P(A) \cdot P(C)$; tương tự $P(B \cap C) = P(B) \cdot P(C)$.\\
     Vì $A$ và $B$ xung khắc nên $P(A \cup B) = P(A)+P(B)$ và $P(A \cap B)=0$.
     \item Mệnh đề sai.
     Ta có: $P(A \cup C) = P(A)+P(C)-P(A \cap C)$
     $= P(A)+P(C)-P(A)P(C) = a+c-ac = \dfrac{2}{3}$ (1);\\
     $P(B \cup C) = P(B)+P(C)-P(B \cap C)$
     $= P(B)+P(C)-P(B)P(C) = b+c-bc = \dfrac{3}{4}$ (2).
     \item Mệnh đề sai.\\
     Ta có: $P(A \cup B \cup C) = P(A)+P(B)+P(C)-P(A \cap B)-P(A \cap C)-P(B \cap C)+P(A \cap B \cap C)$
     $= P(A)+P(B)+P(C)-P(A)P(C)-P(B)P(C) = a+b+c-ac-bc = \dfrac{11}{12}$ (3).\\
     (Dễ thấy vì $A$ và $B$ xung khắc nên $P(A \cap B)=0$ và $P(A \cap B \cap C)=0$).
     \item Mệnh đề sai.
     Lấy (3) trừ (1) và (2) ta được $-c = -\dfrac{1}{2} \Rightarrow c=\dfrac{1}{2}$; (2) suy ra $b=\dfrac{1}{2}$; (1) suy ra $a=\dfrac{1}{3}$.\\
     Do đó: $P(A|B \cup C) = \dfrac{P(A \cap (B \cup C))}{P(B \cup C)}$
     $= \dfrac{P((A \cap B) \cup (A \cap C))}{P(B \cup C)}$
     $= \dfrac{P(A \cap C)}{P(B \cup C)} = \dfrac{P(A)P(C)}{P(B \cup C)} = \dfrac{\dfrac{1}{3}\cdot\dfrac{1}{2}}{\dfrac{3}{4}} = \dfrac{\dfrac{1}{6}}{\dfrac{3}{4}} = \dfrac{2}{9}$.
 \end{listEX}
 }
\end{ex}
\begin{ex}
 \immini[thm]{ Tháp giải nhiệt tại nhà máy Nhiệt điện Phả Lại (Tỉnh Hải Dương, Việt Nam) có mặt cắt qua trục theo phương thẳng đứng là một hình hyperbol (H). Tháp có chiều cao là 120 mét, bán kính đáy dưới bằng 40 mét. Một nhóm kỹ sư đã thiết lập hệ trục tọa độ $Oxy$ như hình vẽ sao cho mặt cắt dạng hypebol của tháp nhận $Ox, Oy$ làm các trục đối xứng; lấy đơn vị trên mỗi trục là mét. Biết rằng đoạn giao nhau giữa trục $Ox$ với tháp bằng 30 mét và gốc $O$ ở vị trí có độ cao 80 mét so với mặt đất.
 }{\includegraphics[scale=1]{img/HXN-7.16}}
 
 \choiceTF
 {\True Diện tích đáy dưới của tháp bằng $5027~m^2$ (làm tròn đến hàng đơn vị)}
 {Các điểm $(-20;0), (20;0)$ thuộc hyperbol $(H)$}
 {Phương trình $(H)$ là $\dfrac{x^2}{15^2} - \dfrac{y^2}{11520} = 1$}
 {\True Thể tích của tháp giải nhiệt này bằng $214414~m^3$ (làm tròn đến hàng đơn vị)}
 \loigiai{
 \includegraphics[scale=1]{img/HXN-7.16a}
 \begin{listEX}
     \item Mệnh đề đúng.
     Bán kính đáy dưới của tháp là $R=40~m$.
     Diện tích đáy dưới tháp $S=\pi R^2 = 1600\pi \approx 5027~m^2$.
     \item Mệnh đề sai.
     Hypebol $(H)$ cắt $Ox$ tại các điểm $(-15;0), (15;0)$.
     \item Mệnh đề sai.
     Gọi phương trình chính tắc của $(H)$ là $\dfrac{x^2}{a^2} - \dfrac{y^2}{b^2} = 1$ ($a>0, b>0$).\\
     Ta có $2a=30 \Rightarrow a=15$; do đó $(H): \dfrac{x^2}{15^2} - \dfrac{y^2}{b^2} = 1$.
     $(H)$ qua điểm $A(40;-80)$ nên $\dfrac{40^2}{15^2} - \dfrac{(-80)^2}{b^2} = 1 \Rightarrow b^2 = \dfrac{11520}{11}$.\\
     Phương trình $(H)$ là $\dfrac{x^2}{15^2} - \dfrac{y^2}{\dfrac{11520}{11}} = 1$.
     \item Mệnh đề đúng.
     Khoảng cách từ $O$ đến nóc bằng $120-80=40$ mét.\\
     Từ câu c) ta có $x^2 = 15^2 \left(1+\dfrac{11y^2}{11520}\right)$; với $x=f(y)$.\\
     Thể tích tháp là $V = \pi \int\limits_{-80}^{40} (f(y))^2 \mathrm{d}y = \pi \int\limits_{-80}^{40} 15^2 \left(1+\dfrac{11y^2}{11520}\right) \mathrm{d}y \approx 214414~m^3$.
 \end{listEX}
 }
\end{ex}
\Closesolutionfile{ans}
\caukq
\Opensolutionfile{ans}[ans/ans-HXN-\sode-SA]
% 
 \begin{ex}%Câu 17
     Trong một lễ hội mùa hè, ba người bạn An, Bình và Cường tham gia cuộc thi xếp tháp ly. Luật chơi như sau: người chơi lần lượt xếp ly vào các tầng của một kim tự tháp chung. An bắt đầu, xếp 1 chiếc ly. Đến lượt Bình, cậu xếp 2 chiếc ly. Cường xếp tiếp 3 chiếc ly. Trở lại lượt An, cậu xếp 4 chiếc ly, rồi Bình xếp 5 chiếc, Cường xếp 6 chiếc... Cuộc thi diễn ra sôi nổi cho đến khi số ly không còn đủ để xếp theo quy luật tăng dần, người đến lượt ở vòng cuối sẽ dùng hết số ly còn lại để hoàn thành tầng của mình (hoặc bắt đầu tầng mới nếu có thể). Sau khi cuộc thi kết thúc, An tự hào khoe rằng mình đã góp tay xếp được khoảng 317 chiếc ly vào ngọn tháp. Hỏi tổng cộng cả ba người bạn đã sử dụng bao nhiêu chiếc ly để xây ngọn tháp đó?
 \shortans{ 933}
 \loigiai{
     Số ly mà An đã xếp là $1; 4;7;\ldots$ tạo thành một cấp số cộng với $u_1=1; d=3$\\
     Sau $n$ lượt, tổng số ly mà An đã xếp là: 
     $$S_n=\dfrac{(u_1+u_n)n}{2}=\dfrac{(3n-1)n}{2}$$ 
     Xét $S_n=317\Rightarrow\dfrac{n(3n-1)}{2}=317\Rightarrow 3n^2-n-634=0\Rightarrow n\approx 14,7$ (không thỏa mãn).\\
     Do đó An là người xếp ly cuối cùng. Sau 14 lượt thì An xếp được $S_{14}=\dfrac{14(3\cdot 14-1)}{2}=287$ ; lượt cuối An xếp thêm $317-287=30$ (ly).\\
     Số ly mà Bình đã xếp được là tổng cấp số cộng có số hạng đầu bằng 2, công sai bằng 3.\\
      Số ly mà Cường đã xếp được là tổng cấp số cộng có số hạng đầu bằng 3, công sai bằng 3.\\
     Tổng số ly cả 3 bạn xếp được là $$317+\dfrac{14\left(2\cdot 2+13\cdot 3\right)}{2}+\dfrac{14\left(2\cdot 3+13\cdot 3\right)}{2}=933$$ .
 }
 \end{ex}
 
 \begin{ex}%Câu 18
 Một khối gỗ có hình dạng của một lăng trụ đứng $ABC.A'{B}'{C}'$ , trong đó $AC=1\,\,m,\,\,BC=2\,\,m,$ $\widehat{ACB}=120^\circ $ . Người thợ mộc đánh dấu điểm $M$ nằm chính giữa đoạn $B{B}'$ . Tính khoảng cách giữa hai đường $AM$ và $C{C}'$ và làm tròn đến hàng phần trăm theo đơn vị mét.
  \shortans{0,65 }
 \loigiai{
     \begin{center}
         \includegraphics[scale=.8]{img/HXN-7.18}
     \end{center}
 Ta có: $C{C}'\text{//}B{B}'\Rightarrow C{C}'\text{//}\left(AB{B}'{A}'\right)$ nên $d\left(C{C}',\,\,\left(AB{B}'{A}'\right)\right)=d\left(C,\left(AB{B}'{A}'\right)\right)$ .\\
 Trong mặt phẳng (ABC), kẻ $CH\perp AB$ tại H (1).\\
 $ABC.A'{B}'{C}'$ là hình lăng trụ đứng nên $A{A}'\perp\left(ABC\right)\Rightarrow CH\perp A{A}'$ (2).\\
 Từ (1) và (2) suy ra $CH\perp\left(AB{B}'{A}'\right)$ $\Rightarrow d\left(C,\,\,\left(AB{B}'{A}'\right)\right)=CH$ .\\
 Xét tam giác $ABC$ có $A{B^2}=C{A^2}+C{B^2}-2.CA.CB.\cos 120^\circ=7$ $\Rightarrow AB=\sqrt{7}$ m.\\
 Diện tích tam giác ABC: $S_{\Delta ABC}=\dfrac{1}{2}CA.CB.\sin C=\dfrac{1}{2}AB.CH$\\ $\Rightarrow CH=\dfrac{CA.CB.\sin{120^0}}{AB}=\dfrac{2.\dfrac{\sqrt{3}}{2}}{\sqrt{7}}=\dfrac{\sqrt{21}}{7}$ m.
 Vậy $d\left(C{C}',\,\,\left(AB{B}'{A}'\right)\right)=CH=\dfrac{\sqrt{21}}{7}\,\,m$ .\\
 Ta có AM và $C{C}'$ là hai đường thẳng chéo nhau mà $\left\{\begin{aligned}
 & C{C}'\text{//}\left(AB{B}'{A}'\right)\\ 
 & AM\subset\left(AB{B}'{A}'\right)\\ 
 \end{aligned}\right.$ nên $d\left(C{C}',\,\,AM\right)=d\left(C{C}',\,\,\left(AB{B}'{A}'\right)\right)=\dfrac{\sqrt{21}}{7}\approx\,0,65\,m$ .}
 \end{ex}
 
 \begin{ex}%Câu 19
 Nhà máy $A$ chuyên sản xuất một loại sản phẩm cung cấp cho nhà máy $B$ . Hai nhà máy thoả thuận rằng: Hàng tháng nhà máy $A$ cung cấp cho nhà máy $B$ số lượng sản phẩm theo đơn đặt hàng của $B$ (tối đa $100$ tấn sản phẩm). Nếu số lượng đặt hàng là $x$ tấn sản phẩm thì giá bán cho mỗi tấn sản phẩm là $P(x)=45-0,001x^2$ (triệu đồng).\\
 Chi phí để $A$ sản xuất $x$ tấn sản phẩm trong một tháng bao gồm:
 \begin{itemize}
    \item Chi phí cố định: $100$ triệu đồng.
   \item  Cho phí cho mỗi tấn sản phẩm làm ra: $30$ triệu đồng.
 \end{itemize}
 Hỏi nhà máy $A$ cần bán cho nhà máy $B$ bao nhiêu tấn sản phẩm mỗi tháng để lợi nhuận thu được là lớn nhất? (Làm tròn kết quả đến hàng phần chục).
  \shortans{ 70,7}
 \loigiai{
 Số tiền mà nhà máy $A$ thu được từ việc bán $x$ tấn sản phẩm $\left(0\le x\le 100\right)$ cho nhà máy $B$ là: $R(x)=x.P(x)=x\left(45-0,001x^2\right)=45x-0,001x^3$ (triệu đồng).\\
 Chi phí để $A$ sản xuất $x$ tấn sản phẩm trong một tháng là $C(x)=100+30x$ (triệu đồng).\\
 Lợi nhuận (triệu đồng) mà nhà máy$A$ thu được là:\\
 $P(x)=R(x)-C(x)=45x-0,001x^3-\left(100+30x\right)=-0,001x^3+15x-100$\\
 Xét hàm số $P(x)=-0,001x^3+15x-100$ với $\left(0\le x\le 100\right)$ ta có:\\
 $P'(x)=-0,003x^2+15\,;\,\,P'(x)=0\Rightarrow{x^2}=5000\Rightarrow x=50\sqrt{2}$ .\\
 Ta có $P(0)=-100;\,\,P\left(50\sqrt{2}\right)=500\sqrt{2}-100\approx 607;\,\,P\left(100\right)=400$ .\\
 Vậy nhà máy$A$ thu được lợi nhuận lớn nhất khi bán khoảng $50\sqrt{2}\approx 70,7$ tấn sản phẩm cho nhà máy $B$ mỗi tháng.}
 \end{ex}
 
 \begin{ex}%Câu 20
     \immini[thm]{ Một cái chậu đựng nước hình bán cầu có bán kính bằng 2 dm. Người ta đặt một ống nhựa và cho nước vào chậu với lưu lượng nước không đổi bằng $0,3$ lít/phút. Đến phút thứ 6, tốc độ dâng lên của nước trong chậu bằng bao nhiêu dm/phút (làm tròn đến hàng phần trăm)?
         \shortans{0,05 }}{\includegraphics[scale=.8]{img/HXN-7.20}}

 \loigiai{
     \begin{center}
         \includegraphics[scale=1.5]{img/HXN-7.20a}
     \end{center}
 Sau 6 phút bơm nước thì thể tích trong bát bằng $6 \times 0,3 = 1,8$ lít.
 
 Gọi $h$ là chiều cao tức thời của mực nước trong chậu, thể tích nước tương ứng chiều cao $h$ được tính theo công thức thể tích chỏm cầu $V=\dfrac{1}{3}\pi h^2 (3R-h)$; trong đó $R=2~dm$ nên
 $$ \boxed{V=\dfrac{1}{3}\pi h^2 (6-h) = 2\pi h^2 - \dfrac{1}{3}\pi h^3 \quad (1)} $$
 Xét $V=1,8 \Rightarrow \dfrac{1}{3}\pi h^2 (3 \cdot 2 - h) = 1,8 \Leftrightarrow h \approx 0,56~dm$ (lưu vào A).
 
 Đạo hàm hai vế của (1) theo $t$, ta được:
 $$ \dfrac{dV}{dt} = (4\pi h - \pi h^2)\dfrac{dh}{dt} \quad (2) $$
 Thay $\dfrac{dV}{dt} = 0,3~dm^3/\text{phút}$; $h=A \approx 0,56~dm$ vào (2), ta được: $\dfrac{dh}{dt} \approx 0,05~dm/\text{phút}$.
 
 Vậy tốc độ dâng lên của nước là khoảng $\boxed{0,05}~dm/\text{phút}$.}
\end{ex}

\begin{ex}%Câu 21
 Vào ngày lễ Tổng kết năm học 2024-2025, tại một trường Tiểu học nghèo ở miền núi, có 10 em học sinh hiếu học được vinh dự nhận 20 phần quà từ các anh chị cựu học sinh của trường nay đã thành đạt. Các phần quà này là đồng giá, gồm có: 9 đôi giày, 7 cái áo và 4 cái cặp; những món quà cùng loại thì giống hệt nhau.
 Trong số 10 em học sinh được nhận quà thì có Bình và Minh là đôi bạn rất thân thiết, tính xác suất để đôi bạn này cùng nhận các món quà như nhau.
  \shortans{0,4 }
\loigiai{
 Gọi x là số cặp quà (giày, áo); gọi y là số cặp quà (giày, cặp); gọi z là số cặp quà (áo, cặp).
 
 Ta có: $\begin{cases} x+y=9 \\ x+z=7 \\ y+z=4 \end{cases} \Leftrightarrow \begin{cases} x=6 \\ y=3 \\ z=1 \end{cases}$.
 
 Số cách tặng quà cho 10 học sinh, mỗi người hai phần khác nhau là: $n(\Omega) = C_{10}^6 \times C_4^3 \times C_1^1$.
 (Tức là chọn 6 học sinh trong 10 học sinh để trao (giày, áo); chọn 3 trong 4 học sinh tiếp theo để trao (giày, cặp); 1 học sinh cuối cùng buộc phải nhận món quà còn lại).
 
 Có hai trường hợp để trao quà cho 10 học sinh mà Bình và Minh được nhận quà như nhau:
 \begin{itemize}
     \item \textbf{Trường hợp 1:} Bình và Minh nhận quà (giày, áo).
     Số cách trao quà là $1 \times 1 \times C_8^4 \times C_4^3 \times C_1^1 = \boxed{280}$ (cách).\\
     (Tức là có 1 cách để Bình và Minh nhận quà; chọn 4 học sinh trong 8 học sinh còn lại tiếp theo nhận (giày, áo) $\rightarrow C_8^4$ (cách); chọn 3 học sinh trong 4 học sinh còn lại nhận (giày, cặp) $\rightarrow C_4^3$ (cách)).
     \item \textbf{Trường hợp 2:} Bình và Minh nhận quà (giày, cặp).
     Số cách trao quà là $1 \times 1 \times C_8^1 \times C_7^6 \times C_1^1 = \boxed{56}$ (cách).\\
     (Tức là có 1 cách để Bình và Minh nhận quà; chọn 1 học sinh trong 8 học sinh tiếp theo nhận (giày, cặp) còn lại $\rightarrow C_8^1$ (cách); chọn 6 học sinh trong 7 học sinh còn lại nhận (giày, áo) $\rightarrow C_7^6$ (cách)).
 \end{itemize}
 Số cách trao quà mà Bình và Minh được nhận quà như nhau là $n(A) = 280+56 = \boxed{336}$.\\
 Vậy xác suất cần tính là $n(A) = 280+56 = \boxed{336}$. $P(A) = \dfrac{n(A)}{n(\Omega)} = \dfrac{336}{C_{10}^6 \times C_4^3 \times C_1^1} = \boxed{0.4}$.}
 \end{ex}
 
 \begin{ex}%Câu 22
Trong không gian với trục tọa độ $Oxyz$ , cho ba điểm $A\left(-1\,;\,-4\,;\,\,4\right)$ , $B\left(1\,;\,\,7\,;\,\,-2\right)$ ; $C\left(1\,;\,\,4\,;\,\,-2\right)$ . Mặt phẳng $(P)$ : $2x+by+cz+d=0$ đi qua điểm $A$ sao cho B và C cùng phía so với (P). Đặt $h_1=d\left(B\,,\,\,(P)\right)$ và $h_2=2d\left(C\,,\,\,(P)\right)$ . Khi đó $h_1+h_2$ đạt giá trị lớn nhất. Tính $T=b+c+d$ .
 \shortans{65}
\loigiai{
    \begin{center}
       \includegraphics[scale=1]{img/HXN-7.22}
    \end{center}
Gọi $D$ là điểm đối xứng với $A$ qua $C$ và $I$ là trung điểm $BD$.\\
Suy ra $D(3;12;-8)$, $I\left(2;\dfrac{19}{2};-5\right)$.
Khi đó $h_1+h_2 = d(B,(P)) + d(D,(P)) = 2d(I,(P)) \le 2IA$.\\
Do vậy $h_1+h_2$ đạt giá trị lớn nhất khi $(P)$ qua $A$ và vuông góc với $IA$.\\
$\overrightarrow{IA}=\left(-3;-\dfrac{27}{2};9\right)=-\dfrac{3}{2}(2;9;-6) \Rightarrow (P)$ nhận $\vec{n}=(2;9;-6)$ làm vec tơ pháp tuyến.\\
Phương trình mặt phẳng $(P): 2x+9y-6z+62=0$.\\
Vậy $b=9; c=-6; d=62 \Rightarrow b+c+d=65$. }
\end{ex}
\Closesolutionfile{ans}
\inputansbox{6,4,3}{ans/ans-HXN-\sode-T,ans/ans-HXN-\sode-TF,ans/ans-HXN-\sode-SA}