\begin{name}
	{\tenchude}
	{\tendethi}
	{\tentruong}
	{\thoigian}
\end{name}
\Opensolutionfile{ans}[ans/ans-de1-7]

\begin{ex}%Câu 1.
Cho số phức $z$ thỏa mãn $\dfrac{z}{1-2 i}+\bar{z}=2$. Phần thực của số phức $w=z^2-z$ là:
\choice
{$-5$}
{$3$}
{$2$}
{\True $1$}

\end{ex}
\begin{ex}%Câu 1.
Rút gọn biểu thức $P=a^{3-2\log_a b}(a>0, a \neq 1, b>0)$, ta được:
\choice
{$P=a^2 b^3$}
{$P=a b^2$}
{$P=a^3 b$}
{\True $P=a^3 b^{-2}$}

\end{ex}
\begin{ex}%Câu 2.
Tích phân $\displaystyle\displaystyle\int\limits_0^2 \dfrac{2}{2 x+1} \mathrm{\,d} x$ bằng:
\choice
{\True $\ln 5$}
{$4\ln 5$}
{$2\ln 5$}
{$\dfrac{1}{2} \ln 5$}

\end{ex}
\begin{ex}%Câu 3.
Tập hợp điểm biểu diễn số phức $z$ biết $|z-(3-4 i)|=2$ là:
\choice
{Đường tròn tâm $I(-3; 4); R=4$}
{\True Đường tròn tâm $I(3;-4); R=2$}
{Đường tròn tâm $I(-3; 4); R=2$}
{Đường tròn tâm $I(3;-4); R=4$}

\end{ex}
\begin{ex}%Câu 4.
Thiết diện qua trục của một hình nón tròn xoay là một tam giác vuông cân có diện tích bằng $2 a^2$. Khi đó thể tích của khối nón bằng:
\choice
{$\dfrac{\pi a^3}{3}$}
{$\dfrac{4\sqrt{2} \pi a^3}{3}$}
{\True $\dfrac{2\sqrt{2} \pi a^3}{3}$}
{$\dfrac{\sqrt{2} \pi a^3}{3}$}

\end{ex}
\begin{ex}%Câu 5.
\immini{
Cho hàm số $y=f(x)$, có bảng biến thiên như hình bên. Bảng biến thiên đó là của hàm số nào?
}
{
\begin{tikzpicture}[color=\mauchinh]
\tkzTabInit[nocadre,lgt=1.05,espcl=1.6,deltacl=0.6]
{$x$/0.6,$f'(x)$/0.6,$f(x)$/1.7}{$-\infty$,$0$,$2$,$+\infty$}
\tkzTabLine{,-,0,+,0,-,}
\tkzTabVar{+/$+\infty$,-/$-1$,+/$3$,-/$-\infty$}
\end{tikzpicture}
}
\choice
{$y=x^3-3 x^2-1$}
{$y=x^3+3 x^2-1$}
{$y=-x^3-3 x^2-1$}
{\True $y=-x^3+3 x^2-1$}
\end{ex}
\begin{ex}%Câu 6.
Cho số phức $z=3+2 i$. Tìm phần thực và phần ảo của số phức $z$.
\choice
{Phần thực bằng $3$, phần ảo bằng $-2$}
{Phần thực bằng $-3$, phần ảo bằng $-2$}
{\True Phần thực bằng $3$, phần ảo bằng $2$}
{Phần thực bằng $-3$, phần ảo bằng $2$}

\end{ex}
\begin{ex}%Câu 7.
Tổng bình phương các nghiệm của phương trình $\log_2^2 x=\log_2 \dfrac{x}{4}+$ $4(x \in \mathbb{R})$ là:
\choice
{$\dfrac{81}{4}$}
{$\dfrac{17}{4}$}
{\True $\dfrac{65}{4}$}
{$\dfrac{9}{2}$}

\end{ex}
\begin{ex}%Câu 8.
Tìm điểm $M(x, y)$ thỏa $2 x-1+(3 y+2) i=5-i$.
\choice
{\True $M(3;-1)$}
{$M(2;-1)$}
{$M\left(3; \dfrac{-1}{3}\right)$}
{$M\left(2; \dfrac{1}{3}\right)$}

\end{ex}
\begin{ex}%Câu 9.
Hàm số $y=x^3-3 x^2+4$ đạt cực tiểu tại:
\choice
{$x=0$ và $x=2$}
{$x=0$}
{\True $x=2$}
{$x=4$}

\end{ex}

\begin{ex}%Câu 10.
Cho mặt cầu $(S)\colon x^2+y^2+z^2-2 x+4 y-9=0$. Mặt phẳng $(P)$ tiếp xúc với mặt cầu $(S)$ tại điểm $M(0;-5; 2)$ có phương trình là:
\choice
{$x+3 y-2 z+5=0$}
{$x-2 y-10=0$}
{$-5 y+2 z+9=0$}
{\True $x+3 y-2 z+19=0$}

\end{ex}
\begin{ex}%Câu 11.
Tính giá trị $\left(\dfrac{1}{16}\right)^{-0,75}+\left(\dfrac{1}{8}\right)^{-\frac{4}{3}}$, ta được:
\choice
{$18$}
{$12$}
{\True $24$}
{$16$}

\end{ex}

\begin{ex}%Câu 12.
Có bao nhiêu cách chọn một học sinh từ một nhóm gồm $7$ học sinh nam và $8$ học sinh nưu?
\choice
{$56$}
{\True $15$}
{$8$}
{$7$}

\end{ex}

\begin{ex}%Câu 13.
Cho mặt cầu $(S)\colon x^2+y^2+z^2-2 x+4 y-9=0$. Mặt phẳng $(P)$ tiếp xúc với mặt cầu $(S)$ tại điểm $M(0;-5; 2)$ có phương trình là:
\choice
{$x+3 y-2 z+5=0$}
{\True $x-2 y-10=0$}
{$-5 y+2 z+9=0$}
{$x+3 y-2 z+19=0$}

\end{ex}
\begin{ex}%Câu 14.
Cho hình chóp $S.ABC$ có đáy là tam giác đều cạnh $2 a, SA \perp$ $(ABC), SA=a \sqrt{6}$. Gọii $M$ là trung điểm của $BC$. Khi đó, khoảng cách từ $A$ đến đường thẳng $SM$ bằng:
\choice
{$a \sqrt{11}$}
{$a \sqrt{6}$}
{$a \sqrt{3}$}
{\True $a \sqrt{2}$}

\end{ex}
\begin{ex}%Câu 15.
Trong không gian $O x y z$, tâm của mặt cầu $(S)\colon 3 x^2+3 y^2+3 z^2-$ $6 x+8 y+15 z-3=0$ là:
\choice
{$\left(-3; 4; \dfrac{15}{2}\right)$}
{\True $\left(1;-\dfrac{4}{3};-\dfrac{5}{2}\right)$}
{$\left(1; \dfrac{4}{3};-\dfrac{5}{2}\right)$}
{$\left(3;-4; \dfrac{-15}{2}\right)$}

\end{ex}
\begin{ex}%Câu 16.
Gọi $\varphi$ là góc giữa hai vectơ $\vec{a}=(1; 2; 0)$ và $\vec{b}=(2; 0;-1)$, khi đó $\cos \varphi$ bằng:
\choice
{$-\dfrac{2}{5}$}
{\True $\dfrac{2}{5}$}
{$0$}
{$\dfrac{2}{\sqrt{5}}$}

\end{ex}
\begin{ex}%Câu 17.
Biết hàm số $f(x)$ có đạo hàm là $f'(x)=x(x-1)^2(x-2)^3(x-3)^5$. Hỏi hàm số $f(x)$ có bao nhiêu điểm cực trị?
\choice
{$4$}
{\True $3$}
{$2$}
{$1$}

\end{ex}
\begin{ex}%Câu 18.
Giá trị nhỏ nhất của hàm số $f(x)=x^3-3 x^2-9 x+35$ trên đoạn $[-4; 4]$ là:
\choice
{$\min\limits_{[-4; 4]} f(x)=15$}
{$\min\limits_{[-4; 4]} f(x)=0$}
{\True $\min\limits_{[-4; 4]} f(x)=-41$}
{$\min\limits_{[-4; 4]} f(x)=-50$}

\end{ex}
\begin{ex}%Câu 19.
Thể tích khối lăng trụ tam giác đều có tất cả các cạnh bằng $a$ là:
\choice
{$\dfrac{\sqrt{3} a^3}{2}$}
{$\dfrac{\sqrt{2} a^3}{3}$}
{$\dfrac{\sqrt{2} a^3}{4}$}
{\True $\dfrac{\sqrt{3} a^3}{4}$}

\end{ex}
\begin{ex}%Câu 20.
Mặt cầu có diện tích bằng $16\pi$. Tính thể tích khối cầu.
\choice
{$\dfrac{32\sqrt{3}}{3} \pi$}
{$\dfrac{32\sqrt{3}}{9} \pi$}
{\True $\dfrac{32}{3} \pi$}
{$\dfrac{32}{9} \pi$}

\end{ex}
\begin{ex}%Câu 21.
Cho $\displaystyle\displaystyle\int\limits_0^1\left(\dfrac{1}{x+1}-\dfrac{1}{x+2}\right) \mathrm{d} x=a \ln 2+b \ln 3$ với $a, b$ là các số nguyên. Mệnh đề nào dưới đây đúng?
\choice
{$a-2 b=0$}
{$a+b=2$}
{$a+b=-2$}
{\True $a+2 b=0$}

\end{ex}
\begin{ex}%Câu 22.
Trong một hộp đựng $7$ bi xanh, $5$ bi đỏ và $3$ bi vàng. Lấy ngẫu nhiên $3$ viên bi. Tính xác suất để được ít nhất $2$ bi vàng.
\choice
{$\dfrac{121}{455}$}
{$\dfrac{22}{455}$}
{$\dfrac{50}{455}$}
{\True $\dfrac{37}{455}$}

\end{ex}
\begin{ex}%Câu 23.
Giá trị lớn nhất của hàm số $y=\dfrac{x-1}{x+2}$ trên đoạn $[0; 2]$ là:
\choice
{\True $\dfrac{1}{4}$}
{$0$}
{$-\dfrac{1}{2}$}
{$2$}

\end{ex}
\begin{ex}%Câu 24.
Trong không gian $O x y z$, cho điểm $M(1; 2;-6)$ và đường thẳng $d\colon\heva{&x=1+2 t \\& y=2-3 t \\& z=-3+2 t}$. Hình chiếu vuông góc của điểm $M$ lên đường thẳng $d$ có tọa độ là:
\choice
{\True $(0; 2;-4)$}
{$(1; 0;-2)$}
{$(4; 0;-2)$}
{$(-1; 0; 2)$}

\end{ex}
\begin{ex}%Câu 25.
Tìm tập nghiệm của bất phương trình $\log_{\frac{1}{2}}(3 x+1)>-2$.
\choice
{$S=\left(-\dfrac{1}{3}; 1\right)$}
{$S=(1;+\infty)$}
{$S=(-\infty; 1)$}
{\True $S=\left(-\dfrac{1}{3}; 1\right)$}

\end{ex}
\begin{ex}%Câu 26.
\immini{
Đồ thị hình bên là của hàm số nào?
\choice
{$y=x^4+2 x^2-3$}
{$y=-\dfrac{1}{4} x^4+3 x^2-3$}
{$y=x^4-3 x^2-3$}
{\True $y=x^4-2 x^2-3$}}
{\vspace{-0.6cm}
\begin{tikzpicture}[scale=1, font=\footnotesize, line join=round, line cap=round, >=stealth,y=0.7cm,color=\mauchinh]
\def\xmin{-1.78}\def\xmax{1.78}\def\ymin{-4.2}\def\ymax{1.1}
\draw[->,thick] (\xmin-0.2,0)--(\xmax+0.2,0) node[below] {\footnotesize $x$};
\draw[->,thick] (0,\ymin-0.2)--(0,\ymax+0.2) node[right] {\footnotesize $y$};
\draw (0,0) node [below left] {\footnotesize $O$};
\foreach \x in {-1,1}\draw (\x,0.1)--(\x,-0.1) node [below] {\footnotesize $\x$};
\foreach \y in {-4,-3}\draw (0.1,\y)--(-0.1,\y) node [left] {\footnotesize $\y$};
\clip (\xmin,\ymin) rectangle (\xmax,\ymax);
\draw[thick,smooth,samples=200,domain=\xmin:\xmax] plot (\x,{1*((\x)^4)+-2*((\x)^2)+-3});
\draw[dashed](-1,0)|-(0,-4) (1,0)|-(0,-4);
\end{tikzpicture}
}

\end{ex}
\begin{ex}%Câu 27.
Tập xác định $\mathscr{D}$ của hàm số $y=(x-1)^{\frac{1}{3}}$ là:
\choice
{$\mathscr{D}=\mathbb{R} \backslash\{1\}$}
{$\mathscr{D}=(-\infty; 1)$}
{$\mathscr{D}=\mathbb{R}$}
{\True $\mathscr{D}=(1;+\infty)$}

\end{ex}
\begin{ex}%Câu 28.
Cho cấp số cộng $\left(u_n\right)$ có $u_1=-2$ và công sai $d=3$. Tìm số hạng $u_{10}$.
\choice
{$u_{10}=-29$}
{\True $u_{10}=25$}
{$u_{10}=-2.3^9$}
{$u_{10}=28$}

\end{ex}
\begin{ex}%Câu 29.
\immini{
Cho hàm số $f(x)$ liên tục trên $\mathbb{R}$ và có đồ thị như hình vẽ bên. Khẳng định nào sau đây là đúng?
\choice
{\True Hàm số đồng biến trên $(-1; 0)$ và $(1;+\infty)$}
{Hàm số đồng biến trên $(-1; 0) \cup(1;+\infty)$}
{Hàm số đồng biến trên $(-\infty;-1)$ và $(1;+\infty)$}
{Hàm số đồng biến trên $(-\infty; 0)$ và $(0;+\infty)$}}
{\vspace{-0.6cm}
\begin{tikzpicture}[scale=1, font=\footnotesize, line join=round, line cap=round, >=stealth,color=\mauchinh]
\def\xmin{-1.54}\def\xmax{1.54}\def\ymin{-0.5}\def\ymax{2}
\draw[->,thick] (\xmin-0.2,0)--(\xmax+0.2,0) node[below] {\footnotesize $x$};
\draw[->,thick] (0,\ymin-0.2)--(0,\ymax+0.2) node[right] {\footnotesize $y$};
\draw (0,0) node [below left] {\footnotesize $O$};
\foreach \x in {-1,1}\draw (\x,0.1)--(\x,-0.1) node [below] {\footnotesize $\x$};
\foreach \y in {1}\draw (0.1,\y)--(-0.1,\y) node [left] {\footnotesize $\y$};
\clip (\xmin,\ymin) rectangle (\xmax,\ymax);
\draw[thick,smooth,samples=200,domain=\xmin:\xmax] plot (\x,{1*((\x)^4)+-2*((\x)^2)+1});
\end{tikzpicture}
}

\end{ex}
\begin{ex}%Câu 30.
Gọi $z_1, z_2$ là hai nghiệm phức của phương trình $z^2-z+1=0$. Giá trị của $\dfrac{1}{\left|z_1\right|}+\dfrac{1}{\left|z_2\right|}$ bằng:
\choice
{$0$}
{$4$}
{\True $2$}
{$1$}

\end{ex}
\begin{ex}%Câu 31.
Trong không gian $O x y z$, cho $(P)\colon x+m y+(m-1) z+2=0$ và $(Q)\colon 2 x-y+3 z-4=0$. Giá trị của $m$ để hai mặt phẳng $(P),(Q)$ vuông góc là:
\choice
{\True $m=\dfrac{1}{2}$}
{$m=1$}
{$m=2$}
{$m=-\dfrac{1}{2}$}

\end{ex}
\begin{ex}%Câu 32.
Tính thể tích $V$ của khối tròn xoay tạo thành khi cho miền phẳng $D$ giới hạn bởi các đường $y={\rm e}^{x}, y=0, x=0, x=1$ quay quanh trục $O x$.
\choice
{$V=\dfrac{{\rm e} \pi^2}{2}$}
{$V=\pi$}
{$V=\pi^2$}
{\True $V=\dfrac{\left({\rm e}^2-1\right) \pi}{2}$}

\end{ex}
\begin{ex}%Câu 33.
Trong không gian $O x y z$, PTĐT qua\\ $A(1; 2;-1)$ và vuông góc với mặt phẳng $(P)\colon x+2 y-3 z+1=0$ là:
\choice
{$\dfrac{x-1}{2}=\dfrac{y-2}{3}=\dfrac{z+1}{1}$}
{$\dfrac{x+1}{1}=\dfrac{y+2}{2}=\dfrac{z-1}{-3}$}
{\True $\dfrac{x-2}{1}=\dfrac{y-4}{2}=\dfrac{z+4}{-3}$}
{$\dfrac{x+2}{1}=\dfrac{y+4}{2}=\dfrac{z-4}{-3}$}

\end{ex}
\begin{ex}%Câu 34.
Cho hình chóp $S.ABC$ có đáy $ABC$ là tam giác đều cạnh $a$. Biết $SA \perp(ABC)$ và $SA=a \sqrt{3}$. Tính thể tích $V$ của khối chóp $S.ABC$.
\choice
{$\dfrac{3 a^3}{6}$}
{\True $\dfrac{a^3}{4}$}
{$\dfrac{3 a^3}{4}$}
{$\dfrac{3 a^3}{8}$}

\end{ex}
\begin{ex}%Câu 35.
Cho $(H)$ là khối chóp tứ giác đều có tất cả các cạnh bằng $a$. Thể tích của $(H)$ bằng:
\choice
{$\dfrac{a^3 \sqrt{3}}{2}$}
{\True $\dfrac{a^3 \sqrt{2}}{6}$}
{$\dfrac{a^3}{3}$}
{$\dfrac{a^3 \sqrt{3}}{4}$}

\end{ex}
\begin{ex}%Câu 36.
Với giá trị nào của $x$ thì hàm số $f(x)=\log_6\left(2 x-x^2\right)$ xác định?
\choice
{\True $0<x<2$}
{$-1<x<1$}
{$x<3$}
{$x>2$}

\end{ex}
\begin{ex}%Câu 37.
Cho lăng trụ đứng tam giác $ABC \cdot A'B'C'$ có cạnh $AA'=2 a$, đáy $ABC$ là tam giác vuông cạnh huyền $BC=2 a \sqrt{3}$. Thể tích khối trụ ngoại tiếp hình lăng trụ đã cho bằng:
\choice
{$5\pi a^3$}
{\True $6\pi a^3$}
{$8\pi a^3 \sqrt{2}$}
{$4\pi a^3 \sqrt{3}$}

\end{ex}
\begin{ex}%Câu 38.
Tìm họ nguyên hàm của hàm số $y=\dfrac{1}{(x+1)^2}$.
\choice
{$\displaystyle\displaystyle\int \dfrac{1}{(x+1)^2} \mathrm{\,d} x=\dfrac{2}{(x+1)^3}+C$}
{$\displaystyle\displaystyle\int \dfrac{1}{(x+1)^2} \mathrm{\,d} x=\dfrac{-2}{(x+1)^3}+C$}
{\True $\displaystyle\displaystyle\int \dfrac{1}{(x+1)^2} \mathrm{\,d} x=-\dfrac{1}{x+1}+C$}
{$\displaystyle\displaystyle\int \dfrac{1}{(x+1)^2} \mathrm{\,d} x=\dfrac{1}{x+1}+C$}

\end{ex}
\begin{ex}%Câu 39.
Cho hình nón có diện tích xung quanh bằng $5\pi a^2$ và bán kính đáy bằng $a$. Tính độ dài đường sinh của hình nón đã cho.
\choice
{\True $5 a$}
{$3 a \sqrt{2}$}
{$a \sqrt{5}$}
{$3 a$}

\end{ex}
\begin{ex}%Câu 40.
Trong không gian $O x y z$, cho đường thẳng $d\colon\heva{&x=1+2 t \\& y=2-3 t \\& z=-3+2 t}$, tọa độ một vectơ chỉ phương của đường thẳng $d$ là:
\choice
{\True $(2;-3; 2)$}
{$(1; 2;-3)$}
{$(1;-3; 2)$}
{$(2 t;-3 t; 2 t), t \in \mathbb{R}$}

\end{ex}
\begin{ex}%Câu 41.
Với mọi $a, b, x$ là các số thực dương thoả $\log_2 x=5\log_2 a+3\log_2 b$. Mệnh đề nào dưới đây đúng?
\choice
{$x=3 a+5 b$}
{$x=5 a+3 b$}
{\True $x=a^5 b^3$}
{$x=a^5+b^3$}

\end{ex}
\begin{ex}%Câu 42.
Hàm số $y=x^3+2 x^2+x+1$ nghịch biến trên khoảng nào?
\choice
{\True $\left(-1;-\dfrac{1}{3}\right)$}
{$(-\infty;-1)$}
{$\left(-\dfrac{1}{3};+\infty\right)$}
{$(-\infty;+\infty)$}

\end{ex}
\begin{ex}%Câu 43.
Cho hình lập phương $ABCD \cdot A_1 B_1 C_1 D_1$. Góc giữa hai mặt phẳng nào sau đây bằng $45^{\circ}$?
\choice
{$\left(ADC_1 B_1\right)$ và $\left(A_1 D_1 CB\right)$}
{\True $\left(ABC_1 D_1\right)$ và $(ABCD)$}
{$(ABCD)$ và $\left(AA_1 B_1 B\right)$}
{$\left(ABB_1 A_1\right)$ và $\left(BB_1 C_1 C\right)$}

\end{ex}
\begin{ex}%Câu 44.
Cho $a, b>0; m, n, k \in \mathbb{N}^{*}; m, n, k \geq 2$. Hãy tìm khẳng định sai.
\choice
{$a^{n} \cdot b^{n}=(a. b)^{n}$}
{\True $\sqrt[n]{\sqrt[k]{a}}=\sqrt[n+k]{a}$}
{$a^{n}\colon a^{m}=a^{n-m}$}
{$\sqrt[n]{a^{m}}=a \bar{n}$}

\end{ex}
\begin{ex}%Câu 45.
Có bao nhiêu tập con gồm $3$ phần tử của tập hợp\\ $X=\{1; 2; 3; 4; 7; 8; 9\}$?
\choice
{$A_9^3$}
{$C_9^3$}
{$A_7^3$}
{\True $C_7^3$}

\end{ex}
\begin{ex}%Câu 46.
Giá trị lớn nhất của hàm số $y=x-\dfrac{1}{x}$ trên nửa khoảng $(0; 2]$ là:
\choice
{$\dfrac{1}{2}$}
{$\dfrac{2}{3}$}
{$\dfrac{3}{4}$}
{\True $\dfrac{3}{2}$}

\end{ex}

\begin{ex}%Câu 48.
Cho $\vec{u}=(2;-1; 1), \vec{v}=(m; 3;-1), \vec{w}=(1; 2; 1)$. Với giá trị nào của $m$ thì $3$ vectơ trên đồng phẳng?
\choice
{\True $-\dfrac{8}{3}$}
{$\dfrac{8}{3}$}
{$-\dfrac{3}{8}$}
{$\dfrac{3}{8}$}

\end{ex}
\begin{ex}%Câu 49.
Hàm số $F(x)={\rm e}^{x^3}$ là một nguyên hàm của hàm số:
\choice
{$f(x)=x^3 \cdot {\rm e}^{x^3-1}$}
{\True $f(x)=3 x^2 \cdot {\rm e}^{x^3}$}
{$f(x)=\dfrac{{\rm e}^{x^3}}{3 x^2}$}
{$f(x)={\rm e}^{x^3}$}

\end{ex}
\begin{ex}%Câu 50.
Giao điểm giữa đồ thị $(C)\colon y=\dfrac{x^2-2 x-3}{x-1}$ và đường thẳng $(d):$ $y=x+1$ là:
\choice
{$A(2;-1)$}
{\True $A(-1; 0)$}
{$A(0;-1)$}
{$A(-1; 2)$}

\end{ex}

\Closesolutionfile{ans}
%% \indapan{10}{ans/ans-de1-7}
