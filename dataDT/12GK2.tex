\begin{name}
	{\tenchude}
	{\tendethi}
	{LỚP TOÁN THẦY PHÁT}
	{\thoigian}
\end{name}
\Opensolutionfile{ans}[ans/ansBTTeX1]
\begin{ex}%[Dự án Tex TDM - NHTP - Lê Quân]%[2D3Y2-1]%
Cho hàm số $f(x)$ liên tục và xác định trên khoảng $K$ và $a, b, c$ là các số thực thuộc $K$. Xét các phát biểu
\begin{enumerate}[1)]
\item $\displaystyle\int\limits_{a}^{b}k f(x)\mathrm{\, d}x=k\displaystyle\int\limits_{a}^{b}f(x)\mathrm{\, d}x$ với $\forall k\in \mathbb{R}$.
\item $\displaystyle\int\limits_{a}^{b}f(x)\mathrm{\, d}x=-\displaystyle\int\limits_{b}^{a}f(x)\mathrm{\, d}x$.
\item $\displaystyle\int\limits_{a}^{a}f(x)\mathrm{\, d}x=0$.
\item $\displaystyle\int\limits_{a}^{b}f(x)\mathrm{\, d}x=\displaystyle\int\limits_{a}^{c}f(x)\mathrm{\, d}x+\displaystyle\int\limits_{c}^{b}f(x)\mathrm{\, d}x$.
\end{enumerate}
Số phát biểu đúng là
\choice
{\True $4$}
{$3$}
{$2$}
{$1$}
\loigiai{
Theo tính chất của tích phân thì cả 4 phát biểu trên đều đúng.
}
\end{ex}

\begin{ex}%[Dự án Tex TDM - NHTP - Lê Quân]%[2D3Y1-2]%
Biết $\displaystyle\int\limits_{0}^{a}\dfrac{1}{\sqrt{x+1}+1}\mathrm{\, d}x=6$ với $a$ là số thực dương. Tính tích phân $I=\displaystyle\int\limits_{0}^{a}\dfrac{1}{1+\sqrt{t+1}}\mathrm{\, d}t$.
\choice
{\True $I=6$}
{$I=3$}
{$I=2\sqrt{3}$}
{$I=3\sqrt{2}$}
\loigiai{
Vì $\displaystyle\int\limits_{0}^{a}\dfrac{1}{\sqrt{x+1}+1}\mathrm{\, d}x=\displaystyle\int\limits_{0}^{a}\dfrac{1}{1+\sqrt{t+1}}\mathrm{\, d}t$ nên $\displaystyle\int\limits_{0}^{a}\dfrac{1}{1+\sqrt{t+1}}\mathrm{\, d}t=6$.
}
\end{ex}

\begin{ex}%[2D3Y1-1]%
Họ nguyên hàm $F(x)=\displaystyle\int\dfrac{1}{3x+2}\mathrm{d}x$ là
\choice
{$F(x)=\ln|3x+2|+C$}
{\True $F(x)=\dfrac{1}{3}\ln|3x+2|+C$}
{$F(x)=\dfrac{1}{3}\ln(3x+2)+C$}
{$F(x)=\dfrac{1}{2}\ln|2x+3|+C$}
\loigiai{
Ta có $F(x)=\displaystyle\int\dfrac{1}{3x+2}\mathrm{d}x=\dfrac{1}{3}\ln|3x+2|+C$.
}
\end{ex}

\begin{ex}%[2D3Y2-1]%
Cho $f(x)$ có đạo hàm trên $[-3; 5]$ thỏa $f(-3)=1, f(5)=9$, khi đó $\displaystyle\int\limits_{-3}^5 4 f'(x)\mathrm{\,d}x$ bằng
\choice
{$40$}
{\True $32$}
{$36$}
{$44$}
\loigiai{
Ta có $\displaystyle\int\limits_{-3}^5 4f'(x) \mathrm{\,d}x=4\displaystyle\int\limits_{-3}^5 f'(x)\mathrm{\,d}x=4 f(x)\Big|_{-3}^5=4(f(5)-f(-3))=4(9-1)=32$.
}
\end{ex}

\begin{ex}%[2D3Y1-1]%
Họ nguyên hàm của hàm số $f(x)=\dfrac{1}{\cos^2 x}-\dfrac{1}{\sin^2 x}+2$ là
\choice
{\True $\tan x+\cot x+2x+C$}
{$\tan x-\cot x+2x+C$}
{$-\tan x+\cot x+2x+C$}
{$-\tan x-\cot x+2x+C$}
\loigiai{
Ta có $\displaystyle\int f(x) \mathrm{\,d}x=\displaystyle\int \left( \dfrac{1}{\cos^2 x}-\dfrac{1}{\sin^2 x}+2\right)\mathrm{\,d}x=\tan x+\cot x+2x+C$.
}
\end{ex}

\begin{ex}%[2D3Y2-1]%
Cho hàm số $F(x)$ là một nguyên hàm của hàm số $f(x)$ trên đoạn $[-1; 2]$. Biết rằng $\displaystyle\int\limits_{-1}^2 f(x)\mathrm{\,d}x=1$ và $F(-1)=-1$. Tính $F(2)$.
\choice
{$2$}
{$3$}
{\True $0$}
{$-1$}
\loigiai{
Ta có $\displaystyle\int\limits_{-1}^2 f(x) \mathrm{\,d}x=F(2)-F(-1)\Leftrightarrow F(2)=\displaystyle\int\limits_{-1}^2 f(x) \mathrm{\,d}x+F(-1)=1+(-1)=0$.
}
\end{ex}

\begin{ex}%[2D3Y2-1]%
Cho $ \displaystyle \int\limits_1^3 f(x) \mathrm{\,d}x=2 $ và $ \displaystyle \int\limits_1^3 g(x) \mathrm{\,d}x=1 $, khi đó $ \displaystyle \int\limits_1^3 \left[1008 f(x)+2g(x)\right] \mathrm{\,d}x $ bằng
\choice
{$ 2017 $}
{\True $ 2018 $}
{$ 2019 $}
{$ 2020 $}
\loigiai{
Ta có $ \displaystyle \int\limits_1^3 \left[1008 f(x)+2g(x)\right] \mathrm{\,d}x=1008\cdot 2+2\cdot 1=2018 $.
}
\end{ex}

\begin{ex}%[2D3Y2-1]%
Cho $\displaystyle\int\limits_a^b f(x)\mathrm{\,d}x=2$ và $\displaystyle\int\limits_c^b f(x)\mathrm{\,d}x=3$ với $a<b<c$, khi đó $\displaystyle\int\limits_a^c f(x)\mathrm{\,d}x$ bằng
\choice
{$-2$}
{$5$}
{$1$}
{\True $-1$}
\loigiai{
Ta có $\displaystyle\int\limits_a^c f(x)\mathrm{\,d}x=\displaystyle\int\limits_a^b f(x)\mathrm{\,d}x+\displaystyle\int\limits_b^c f(x) \mathrm{\,d}x=\displaystyle\int\limits_a^b f(x) \mathrm{\,d}x-\displaystyle\int\limits_c^b f(x)\mathrm{\,d}x=2-3=-1$.
}
\end{ex}

\begin{ex}%[2H3Y2-3]%Câu 1
Trong không gian với hệ tọa độ $Oxyz$, phương trình tổng quát của mặt phẳng $(Oxz)$ là
\choice
{$y+z=0$}
{$x-1=0$}
{\True $y=0$}
{$x=0$}
\loigiai{
Phương trình tổng quát của mặt phẳng $(Oxz)$ là $y=0$.
}
\end{ex}

\begin{ex}%[2D3Y1-1]%
Họ nguyên hàm của hàm số $f(x)=x+2^x$ là
\choice
{$1+\dfrac{2^x}{\ln 2}+C$}
{\True $\dfrac{x^2}{2}+\dfrac{2^x}{\ln 2}+C$}
{$\dfrac{x^2}{2}+2^x \ln 2+C$}
{$\dfrac{x^2}{2}+2^x+C$}
\loigiai{
Ta có $\displaystyle\int f(x) \mathrm{\,d}x=\displaystyle\int (x+2^x) \mathrm{\,d}x=\dfrac{x^2}{2}+\dfrac{2^x}{\ln 2}+C$.
}
\end{ex}

\begin{ex}%[2H3Y2-2]%
Trong không gian $O x y z$, mặt phẳng $(P): 2 x+3 y-4 z+7=0$ có một véc-tơ pháp tuyến là
\choice
{$\vec{n}=(-2 ; 3 ;-4)$}
{$\vec{n}=(-2 ;-3 ;-4)$}
{\True $\vec{n}=(2 ; 3 ;-4)$}
{$\vec{n}=(2 ;-3 ;-4)$}
\loigiai{
$(P): 2 x+3 y-4 z+7=0\Rightarrow \text{ VTPT } \vec{n}=(2;3;-4)$.
}
\end{ex}

\begin{ex}%[2D3Y1-2]%
Tính nguyên hàm $I=\displaystyle\int\limits (3+2x)^2\mathrm{\,d}x$ bằng cách đặt $t=3+2x$. Mệnh đề nào dưới đây đúng?
\choice
{$I=\displaystyle\int\limits t^2\mathrm{\,d}t$}
{$I=\dfrac{1}{2}\displaystyle\int\limits t^3\mathrm{\,d}t$}
{$I=\dfrac{1}{6}\displaystyle\int\limits t^3\mathrm{\,d}t$}
{\True $I=\dfrac{1}{2}\displaystyle\int\limits t^2\mathrm{\,d}t$}
\loigiai{
Đặt $t=3+2x \Rightarrow \mathrm{\,d}t=2\mathrm{\,d}x \Rightarrow \mathrm{\,d}x= \dfrac{1}{2}\mathrm{\,d}t $.\\
Khi đó, $I=\displaystyle\int\limits \dfrac{1}{2}t^2\mathrm{\,d}t=\dfrac{1}{2}\displaystyle\int\limits t^2\mathrm{\,d}t$.
}
\end{ex}

\begin{ex}%[2D3Y1-1]%
Họ tất cả các nguyên hàm của hàm số $f(x)=\cos 3x$ là
\choice
{$-\dfrac{\sin 3x}{3}+C$}
{\True $\dfrac{\sin 3x}{3}+C$}
{$\sin 3x+C$}
{$3 \sin 3x+C$}
\loigiai{
Ta có $\displaystyle\int f(x) \mathrm{\,d}x=\displaystyle\int \cos 3x \mathrm{\,d}x=\dfrac{1}{3}\sin 3x+C$.
}
\end{ex}

\begin{ex}%[2D3Y1-1]%
Nguyên hàm của hàm số $f(x)=x^2+\dfrac{2}{x^2}$ là
\choice
{\True $\dfrac{x^3}{3}-\dfrac{2}{x}+C$}
{$\dfrac{x^3}{3}-\dfrac{1}{x}+C$}
{$\dfrac{x^3}{3}+\dfrac{2}{x}+C$}
{$\dfrac{x^3}{3}+\dfrac{1}{x}+C$}
\loigiai{
Ta có $\displaystyle\int f(x) \mathrm{\,d}x=\displaystyle\int \left( x^2+\dfrac{2}{x^2}\right) \mathrm{\,d}x=\dfrac{x^3}{3}-\dfrac{2}{x}+C$.
}
\end{ex}


\begin{ex}%[2D3B2-3]%Câu 471.
Biết $\displaystyle\int_0^{\tfrac{\pi}{4}} x(1+\sin 2 x) \mathrm{\,d}x=\dfrac{a}{b}+\dfrac{c}{d}\pi^2$ với $\dfrac{a}{b}$, $\dfrac{c}{d}$ là phân số tối giản. Khi đó $a+b+c+d$ bằng
\choice
{$36$}
{\True $38$}
{$12$}
{$14$}
\loigiai{
Ta có $\displaystyle\int_0^{\tfrac{\pi}{4}} x(1+\sin 2 x) \mathrm{\,d}x=\displaystyle\int_0^{\tfrac{\pi}{4}} x \mathrm{\,d}x+\displaystyle\int_0^{\tfrac{\pi}{4}} x\sin 2 x \mathrm{\,d}x=I+J$.\\
Tính $I=\displaystyle\int_0^{\tfrac{\pi}{4}} x \mathrm{\,d}x=\dfrac{x^2}{2}\bigg|_0^{\tfrac{\pi}{4}}=\dfrac{1}{32}\pi^2$.\\
Tính $J=\displaystyle\int_0^{\tfrac{\pi}{4}} x\sin 2 x \mathrm{\,d}x$.\\
Đặt $\heva{& u=x\Rightarrow \mathrm{\,d}u=\mathrm{\,d}x \\ & \mathrm{\,d}v=\sin 2x \mathrm{\,d}x\Rightarrow v=-\dfrac{1}{2}\cos 2x.}$\\
Khi đó $J=-\dfrac{1}{2}x\cos 2x\bigg|_0^{\tfrac{\pi}{4}}+\dfrac{1}{2}\displaystyle\int_0^{\tfrac{\pi}{4}} \cos 2x \mathrm{\,d}x=\dfrac{1}{4}\sin 2x\bigg|_0^{\tfrac{\pi}{4}}=\dfrac{1}{4}(1-0)=\dfrac{4}{4}$.\\
Suy ra $I+J=\dfrac{1}{4}+\dfrac{1}{32}\pi^2$.\\
Vậy $a+b+c+d=1+4+1+32=38$.
}
\end{ex}

\begin{ex}%[Tổng ôn LVD-GV soạn Mui Doan-GVPB Khanh Lê]%[2D3B2-1]%
Cho hàm số $ f(x) $ có $ f(0)=-1 $ và  $ f'(x)=\sin x\sin^ 2 2x, \forall x\in\mathbb{R} $. Khi đó $ \displaystyle\int\limits_{0}^{\pi} f(x)\mathrm{\,d}x $ bằng
\choice
{\True $ -\dfrac{7\pi}{15} $}
{$ \dfrac{4\pi}{5} $}
{$ \dfrac{3\pi}{5} $}
{$ \dfrac{-8\pi}{15} $}
\loigiai{
Vì
\allowdisplaybreaks
\begin{eqnarray*}
f'(x)&=&\sin x\sin^ 2 2x\\
&=&\dfrac{1}{2}\sin x (1-\cos 4x)\\
&=&\dfrac{1}{2}\sin x -\dfrac{1}{2}\sin x\cos 4x\\
&=&\dfrac{1}{2}\sin x -\dfrac{1}{4}(\sin 5x-\sin 3x)\\
&=&\dfrac{1}{2}\sin x -\dfrac{1}{4}\sin 5x+\dfrac{1}{4}\sin 3x
\end{eqnarray*}
nên
\begin{eqnarray*}
f(x)&=&\displaystyle\int \left(\dfrac{1}{2}\sin x -\dfrac{1}{4}\sin 5x+\dfrac{1}{4}\sin 3x\right)\mathrm{\,d}x\\
&=&-\dfrac{1}{2}\cos x +\dfrac{1}{20}\cos 5x-\dfrac{1}{12}\cos 3x+C.
\end{eqnarray*}
Do $ f(0)=-1\Rightarrow -\dfrac{1}{2} +\dfrac{1}{20}-\dfrac{1}{12}+C=-1\Rightarrow C=-\dfrac{7}{15}$.\\
Vậy $ f(x)=-\dfrac{1}{2}\cos x +\dfrac{1}{20}\cos 5x-\dfrac{1}{12}\cos 3x-\dfrac{7}{15}$.
Ta có
\allowdisplaybreaks
\begin{eqnarray*}
\displaystyle\int\limits_{0}^{\pi} f(x)\mathrm{\,d}x
&=&\displaystyle\int\limits_{0}^{\pi}\left(-\dfrac{1}{2}\cos x +\dfrac{1}{20}\cos 5x-\dfrac{1}{12}\cos 3x-\dfrac{7}{15}\right)\mathrm{\,d}x\\
&=&\left(-\dfrac{1}{2}\sin x +\dfrac{1}{100}\sin 5x-\dfrac{1}{36}\sin 3x-\dfrac{7}{15}x\right)\bigg|^{\pi}_{0}\\
&=& -\dfrac{7}{15}\pi.
\end{eqnarray*}
}
\end{ex}

\begin{ex}%[Tổng ôn LVD-GV soạn Mui Doan-GVPB Khanh Lê]%[2D3B2-2]%
Cho $ \displaystyle\int\limits_{0}^{\frac{\pi}{2}}\dfrac{\cos x}{\sin^2x-5\sin x+6}\mathrm{\,d}x=a\ln \dfrac{4}{c}+b $ với $ a, c>0 $.  Giá trị của $a+b+c $ bằng
\choice
{$ 0 $}
{$ 1 $}
{$ 3 $}
{\True $ 4 $}
\loigiai{
Đặt $ t=\sin x\Rightarrow \mathrm{\,d}t=\cos x\mathrm{\,d}x $.\\
Đổi cận\\
$ x=0\Rightarrow t=0 $.\\
$ x=\dfrac{\pi}{2}\Rightarrow t=1 $.\\
Ta có
\allowdisplaybreaks
\begin{eqnarray*}
\displaystyle\int\limits_{0}^{\frac{\pi}{2}}\dfrac{\cos x}{\sin^2x-5\sin x+6}\mathrm{\,d}x
&=&\displaystyle\int\limits_{0}^{1}\dfrac{1}{t^2-5t+6}\mathrm{\,d}t \\
&=&\displaystyle\int\limits_{0}^{1}\dfrac{1}{(t-2)(t-3)}\mathrm{\,d}t \\
&=&\displaystyle\int\limits_{0}^{1}\left( \dfrac{-1}{t-2}+ \dfrac{1}{t-3}\right)\mathrm{\,d}t\\
&=&\left(-\ln \vert t-2 \vert+\ln \vert t-3 \vert\right)\bigg|^{1}_{0} \\
&=&-\ln 1+\ln 2-(-\ln 2+\ln 3)\\
&=&2\ln 2-\ln 3\\
&=&\ln \dfrac{4}{3}.
\end{eqnarray*}
Suy ra $ a=1$, $ b=0$, $ c=3 $.\\
Vậy $a+b+c =1+0+3=4$.
}
\end{ex}

\begin{ex}%[Tổng ôn LVD-GV soạn Mui Doan-GVPB Khanh Lê]%[2D3B1-1]%
Cho $ F(x)=\cos 2x-\sin x+C $ là họ nguyên hàm của $ f(x) $. Tính $ f(\pi) $.
\choice
{$ f(\pi) =-3$}
{\True $ f(\pi) =1$}
{ $ f(\pi) =-1$}
{$ f(\pi) =0$}
\loigiai{
Ta có $ F(x)=\cos 2x-\sin x+C \Rightarrow f(x)=F'(x)=-2\sin 2x-\cos x\Rightarrow f(\pi)=1$.
}
\end{ex}

\begin{ex}%[Tổng ôn LVD-GV soạn Mui Doan-GVPB Khanh Lê]%[2D3B2-1]%
Tích phân $ \displaystyle\int\limits_{1}^{2}\dfrac{\mathrm{\,d}x}{3x-2} $ bằng
\choice
{$ 2\ln 2 $}
{$ \ln 2 $}
{\True $ \dfrac{2}{3}\ln 2 $}
{$ \dfrac{1}{3}\ln 2 $}
\loigiai{
Ta có
\allowdisplaybreaks
\begin{eqnarray*}
\displaystyle\int\limits_{1}^{2}\dfrac{\mathrm{\,d}x}{3x-2}
&=&\dfrac{1}{3}\ln \vert 3x-2 \vert\bigg|^{2}_{1}\\
&=&\dfrac{1}{3}\ln4-\dfrac{1}{3}\ln 1\\
&=& \dfrac{2}{3}\ln 2.
\end{eqnarray*}
}
\end{ex}

\begin{ex}%[tex hóa Tổng ôn LVĐ 21-22]%[2H3B1-2]%
Trong không gian $Oxyz$, cho $A(1;2;-1)$, $B(0;-2;3)$. Diện tích tam giác $OAB$ bằng \choice
{$\dfrac{\sqrt{29}}{6} $}
{\True $\dfrac{\sqrt{29}}{2} $}
{$ \dfrac{\sqrt{78}}{2} $}
{$2$}
\loigiai{
Ta có $\vec{OA}= (1;2;-1)$, $\vec{OB}= (0;-2;3)$. Suy ra $\left[\vec{OA}, \vec{OB} \right]= (4;-3;-2)$.\\
Vậy $S_{ABC} =\dfrac{1}{2} \left|\left[\vec{OA}, \vec{OB} \right] \right|=\dfrac{\sqrt{29}}{2}$.}
\end{ex}

\begin{ex}%[Nguyễn Thành Tiến- Dự án Tổng ôn LVĐ 2022]%[2H3B1-1]%
Trong không gian $Oxyz$, cho hai véctơ $\vec{u}=\left(2;-1;4\right)$ và $\vec{v}=\vec{i}-3\vec{k}$. Tích vô hướng $\vec{u}\cdot \vec{v}$ bằng
\choice
{$-11$}
{$-13$}
{$5$}
{\True $-10$}
\loigiai{
Ta có $\vec{u}\left(2;-1;4\right)$ và $\vec{v}\left(1;0;-3\right)$.\\
Suy ra $\vec{u}\cdot \vec{v}=2-12=-10$.
}
\end{ex}

\begin{ex}%[2D3B2-2]%Câu 460.
Cho $\displaystyle\int_0^1\dfrac{\mathrm{\,d}x}{\mathrm{e}^x+1}=a+b\ln\dfrac{1+\mathrm{e}}{2}$ với $a, b$ là các số hữu tỉ. Khi đó $a^3+b^3$ bằng
\choice
{$2$}
{$-2$}
{\True $0$}
{$1$}
\loigiai{
Đặt $u=\mathrm{e}^x+1\Rightarrow \mathrm{\,d}u=\mathrm{e}^x\mathrm{\,d}x$.\\
Đổi cận: $x=0\Rightarrow u=2$; $x=1\Rightarrow u=\mathrm{e}+1$.\\
Suy ra $\displaystyle\int_0^1\dfrac{\mathrm{\,d}x}{\mathrm{e}^x+1}=\displaystyle\int_2^{\mathrm{e}+1} \dfrac{1}{u(u-1)} \mathrm{\,d}u=\left(\ln(u-1)-\ln u\right)\bigg|_2^{\mathrm{e}+1}=1-\ln\dfrac{1+\mathrm{e}}{2}$.\\
Vậy $a^3+b^3=1-1=0$.
}
\end{ex}

\begin{ex}%[2H3B2-3]%Câu 4
Trong không gian với hệ tọa độ $Oxyz$, phương trình mặt phẳng song song và cách $(Oyz)$ một đoạn bằng $2$ là
\choice
{$z \pm 2=0$}
{$y \pm 2=0$}
{$y+z \pm 2=0$}
{\True $x \pm 2=0$}
\loigiai{
Mặt phẳng $(Oyz)$ có phương trình là $x=0$.\\
Phương trình mặt phẳng song song $(Oyz)$ nên có dạng $x+D=0$.\\
Mặt phẳng cách $(Oyz)$ một đoạn bằng $2$ nên $|D|=2 \Leftrightarrow D = \pm 2$.\\
Vậy phương trình mặt phẳng cần tìm là $x \pm 2 =0$.
}
\end{ex}

\begin{ex}%[2D3B2-2]%
Cho $f(x)$ có đạo hàm trên đoạn $[1; 2], f(2)=2, f(4)=2018$, khi đó $\displaystyle\int\limits_1^2 f'(2x)\mathrm{\,d}x$ bằng
\choice
{$-1008$}
{$2018$}
{\True $1008$}
{$-2018$}
\loigiai{
Đặt $t=2x\Rightarrow \dfrac{1}{2}\mathrm{\,d}t=\mathrm{\,d}x$.\\
Đổi cận $\heva{&x=1\Rightarrow t=2\\&x=2\Rightarrow t=4.}$\\
Suy ra $\displaystyle\int\limits_1^2 f'(2x) \mathrm{\,d}x=\dfrac{1}{2}\cdot \displaystyle\int\limits_2^4 f'(t) \mathrm{\,d}t=\dfrac{1}{2} f(t) \Big|_2^4 =\dfrac{1}{2}\left(f(4)-f(2)\right)=\dfrac{1}{2}\left(2018-2\right)=1008$.
}
\end{ex}

\begin{ex}%[Dự án Tex hóa Tư Duy Mở]%[Nhật Thiện]%[2D3B2-2]%
Cho tích phân $I=\displaystyle\int\limits_0^{\frac{\pi}{4}} \dfrac{1+\sin^2x}{2+\cos^2x} \mathrm{\,d}x$, nếu ta dùng một phép đổi biến số đặt $t=\tan x$ thì sẽ thu được tích phân tương ứng là
\choice
{$I=\displaystyle\int\limits_0^1 \dfrac{(2t^1+1)\mathrm{\,d}t}{2t^3+3}\mathrm{\,d}t$}
{$I=\displaystyle\int\limits_0^1 \dfrac{(2t^2+2)\mathrm{\,d}t}{(2t^2+3)(t^2+3)}$}
{$I=\displaystyle\int\limits_0^1 \dfrac{(2t^2+1)(t^2+1)\mathrm{\,d}t}{2t^2+3}$}
{\True $I=\displaystyle\int\limits_0^1 \dfrac{(2t^2+1)\mathrm{\,d}t}{(2t^2+3)(t^2+1)}$}
\loigiai{
Đặt $t=\tan x\Rightarrow \mathrm{\,d}t=(1+\tan^2 x) \mathrm{\,d}x\Rightarrow \dfrac{\mathrm{\,d}t}{(1+t^2)}=\mathrm{\,d}x$. Đổi cận $\heva{&x=0\Rightarrow t=0\\&x=\dfrac{\pi}{4}\Rightarrow t=1.}$\\
Ta có
\allowdisplaybreaks
\begin{eqnarray*}
I&=&\displaystyle\int\limits_0^{\frac{\pi}{4}} \dfrac{1+\sin^2x}{2+\cos^2x} \mathrm{\,d}x =\displaystyle\int\limits_0^{\frac{\pi}{4}} \left[\dfrac{\dfrac{1}{\cos^2x}+\tan^2x}{\dfrac{2}{\cos^2x}+1}\right] \mathrm{\,d}x=\dfrac{1}{2}\displaystyle\int\limits_0^{\frac{\pi}{4}} \left[\dfrac{1+2\tan^2x}{2\tan^2x+3}\right]\mathrm{\,d}x\\
&=&\dfrac{1}{2}\displaystyle\int\limits_0^{\frac{\pi}{4}} \left[\dfrac{1+2t^2}{2t^2+3}\right]\dfrac{\mathrm{\,d}t}{1+t^2}=\displaystyle\int\limits_0^1 \dfrac{(2t^2+1)\mathrm{\,d}t}{(2t^2+3)(t^2+1)}.
\end{eqnarray*}
}
\end{ex}

\begin{ex}%[Tổng ôn LVD-GV soạn Mui Doan-GVPB Khanh Lê]%[2D3B1-1]%
Họ nguyên hàm của hàm số $ f(x)=\dfrac{x+5}{x-1} $ là
\choice
{\True $ x+6\ln \vert x-1 \vert +C$}
{$ x-6\ln \vert x-1 \vert +C$}
{$ x+6\ln ( x-1) +C$}
{$ 6\ln \vert x-1 \vert +C$}
\loigiai{
Ta có
\allowdisplaybreaks
\begin{eqnarray*}
\displaystyle\int f(x)\mathrm{\,d}x
&=&\displaystyle\int \dfrac{x+5}{x-1}\mathrm{\,d}x\\
&=&\displaystyle\int \left(1+\dfrac{6}{x-1}\right)\mathrm{\,d}x\\
&=& x+6\ln \vert x-1 \vert +C.
\end{eqnarray*}
}
\end{ex}

\begin{ex}%[2H3B2-3]%
Trong không gian $O x y z$, cho mặt cầu $(S)$ có đường kính $A B$ với $A(6 ; 2 ;-5), B(-4 ; 0 ; 7)$. Phương trình mặt phẳng $(P)$ tiếp xúc với mặt cầu $(S)$ tại $A$ là
\choice
{$5 x+y-6 z+62=0$}
{\True $5 x+y-6 z-62=0$}
{$5 x-y-6 z-62=0$}
{$5 x+y+6 z+62=0$}
\loigiai{
Gọi $I$ là trung điểm $AB$ ta có $I(1;1;1)$.\\
Mặt phẳng $(P)$ tiếp xúc với mặt cầu $(S)$ tại điểm $A(6 ; 2 ;-5)$ nên $(P)$ đi qua $A(6 ; 2 ;-5)$ và có véc-tơ pháp tuyến là $\vec{n}=\overrightarrow{IA}=\left(5;1;-6\right)$ có phương trình là
$$5(x-6)+1(y-2)-6(z+5)=0\Leftrightarrow 5x+y-6z-62=0.$$
}
\end{ex}

\begin{ex}%[Dự án Tex TDM - NHTP - Lê Quân]%[2D3B2-1]%
Giá trị của tích phân $I=\displaystyle\int\limits_{0}^{1}\dfrac{1}{\sqrt{x+1}}\mathrm{\, d}x$ bằng
\choice
{$\dfrac{207}{250}$}
{$3\sqrt{2}-4$}
{\True $2\sqrt{2}-2$}
{$1+\sqrt{2}$}
\loigiai{
Ta có $I=\displaystyle\int\limits_{0}^{1}\dfrac{1}{\sqrt{x+1}}\mathrm{\, d}x =2\sqrt{x+1}\Big|_0^1=2\sqrt{2}-2.$

}
\end{ex}

\begin{ex}%[Tổng ôn LVD-GV soạn Mui Doan-GVPB Khanh Lê]%[2D3B1-1]%
Cho hàm số $ f(x) $ thỏa mãn $ f'(x)=x-\dfrac{1}{x^2}+2 $ và $ f(1)=3 $. Khi đó hàm số $ f(x) $ là
\choice
{$ \dfrac{x^2}{2}-\dfrac{1}{x}+2x-\dfrac{1}{2} $}
{$ \dfrac{x^2}{2}-\dfrac{1}{x}+2x+\dfrac{3}{2} $}
{\True $ \dfrac{x^2}{2}+\dfrac{1}{x}+2x-\dfrac{1}{2} $}
{$ \dfrac{1}{2}x^2+\dfrac{1}{x}+2 $}
\loigiai{
Vì $ f'(x)=x-\dfrac{1}{x^2}+2 \Rightarrow f(x)=\displaystyle\int f'(x)\mathrm{\,d}x=\dfrac{x^2}{2}+\dfrac{1}{x}+2x+C$.\\
Do $ f(1)=3\Rightarrow \dfrac{7}{2}+C=3\Rightarrow C=- \dfrac{1}{2}$.\\
Vậy $ f(x)=\dfrac{x^2}{2}+\dfrac{1}{x}+2x-\dfrac{1}{2} $.
}
\end{ex}

\begin{ex}%[tex hóa Tổng ôn LVĐ 21-22]%[2H3B1-2]%
Trong không gian $Oxyz$, thể tích khối tứ diện $ABCD$ với $A(1;0;1)$, $B(2;0;-1), $ $C(0;1;3)$ và $D(3; 1; 1)$ bằng \choice
{\True $\dfrac{2}{3} $}
{$4 $}
{$ 2 $}
{$\dfrac{4}{3}$}
\loigiai{ Ta có $\vec{AB}= (1;0;-2)$, $\vec{AC}= (-1;1;2)$, $\vec{AD}= (2; 1; 0)$. \\
Suy ra  $V_{ABCD} =\dfrac{1}{6} \cdot \left|\left[\vec{AB}, \vec{AC} \right] \cdot \vec{AD} \right|=\dfrac{4}{6}=\dfrac{2}{3}$.}
\end{ex}

\begin{ex}%[2D3B2-1]%
Cho $\displaystyle\int\limits_0^2 f(x)\mathrm{\,d}x=1$ và $\displaystyle\int\limits_0^2\left[\mathrm{e}^x-f(x)\right]\mathrm{\,d}x=\mathrm{e}^a-b$ với $a$, $b$ là những số nguyên. Khẳng định nàv sau đây đúng?
\choice
{$a>b$}
{$a<b$}
{\True $a=b$}
{$ab=1$}
\loigiai{
Ta có $\displaystyle\int\limits_0^2 [\mathrm{e}^x-f(x)] \mathrm{\,d}x=\displaystyle\int\limits_0^2 \mathrm{e}^x \mathrm{\,d}x-\displaystyle\int\limits_0^2 f(x)\mathrm{\,d}x=\mathrm{e}^x\Big|_0^2 -1=\mathrm{e}^2-1-1=\mathrm{e}^2-2\Rightarrow \heva{&a=2\\&b=2.}$	\\
Khi đó $a=b$.
}
\end{ex}

\begin{ex}%[Dự án Tex TDM - NHTP - Lê Quân]%[2D3B2-1]%
Cho $\displaystyle\int\limits_{a}^{b}f(x)\mathrm{\, d}x=-4$; $\displaystyle\int\limits_{c}^{b}f(t)\mathrm{\, d}t=3$. Giá trị của tích phân $\displaystyle\int\limits_{a}^{c}f(x)\mathrm{\, d}x$ bằng
\choice
{$-1$}
{$1$}
{$7$}
{\True $-7$}
\loigiai{
Ta có
\[\displaystyle\int\limits_{a}^{c}f(x)\mathrm{\, d}x=\displaystyle\int\limits_{a}^{b}f(x)\mathrm{\, d}x +\displaystyle\int\limits_{b}^{c}f(x)\mathrm{\, d}x=-4-3=-7.\]
}
\end{ex}

\begin{ex}%[2D3B1-3]%
Họ nguyên hàm $F(x)=\displaystyle\int\sin\sqrt{x}\mathrm{\,d}x$ là
\choice
{$F(x)=-\sqrt{x}\cdot\cos\sqrt{x}+\sin\sqrt{x}+C$}
{$F(x)=2\sqrt{x}\cdot\cos\sqrt{x}-2\sin\sqrt{x}+C$}
{$F(x)=4\sqrt{x}\cdot\cos\sqrt{x}+4\sin\sqrt{x}+C$}
{\True $F(x)=-2\sqrt{x}\cdot\cos\sqrt{x}+2\sin\sqrt{x}+C$}
\loigiai{
Đặt $\sqrt{x}=t \Rightarrow x=t^2 \Rightarrow \mathrm{\,d}x=2t\mathrm{\,d}t$.\\
Suy ra $F(x)=\displaystyle\int\sin\sqrt{x}\mathrm{\,d}x=2\displaystyle\int t\cdot\sin t\mathrm{\,d}t$.\\
Đặt $\heva{&u=t\\&\mathrm{\,d}v=\sin t\mathrm{\,d}t} \Rightarrow \heva{&\mathrm{\,d}u=\mathrm{\,d}t\\&v=-\cos t.}$\\
Khi đó
\allowdisplaybreaks
$\begin{aligned}[t]
F(x)&=-2t\cdot\cos t+2\displaystyle\int\cos t\mathrm{\,d}t=-2t\cdot\cos t+2\sin t+C\\
&=-2\sqrt{x}\cdot\cos\sqrt{x}+2\sin\sqrt{x}+C.
\end{aligned}$
}
\end{ex}

\begin{ex}%[2H3B2-3]%
Trong không gian $Oxyz$, cho bốn điểm $A(-1;1;-2)$, $B(1;2;-1)$, $C(1;1;2)$, $D(-1;-1;2)$. Phương trình mặt phẳng $(P)$ đi qua hai điểm $A$, $B$ và song song với đường thẳng $CD$ là
\choice
{\True $x-y-z=0$}
{$x-y-z+2=0$}
{$2x+y+z+2=0$}
{$x-2y-2z-1=0$}
\loigiai{
Ta có $\overrightarrow{AB}=(2;1;1)$, $\overrightarrow{CD}=(-2;-2;0)$.\\
Mặt phẳng $(P)$ đi qua hai điểm $A$, $B$ và song song với đường thẳng $CD$ có véc-tơ pháp tuyến
\[\overrightarrow{n}=\left[\overrightarrow{AB};\overrightarrow{CD}\right]=(2;-2;-2).\]
Phương trình mặt phẳng $(P)$ là
\[2\cdot (x+1)-2\cdot (y-1)-2\cdot (z+2)=0\Leftrightarrow x-y-z=0.\]
}
\end{ex}

\begin{ex}%[2H3B1-3]%
Trong không gian hệ trục tọa độ $O x y z$, cho mặt phẳng $(\alpha): x+2 y-2 z-10=0$. Gọi $(S)$ là mặt cầu có tâm $I=(2 ; 1 ; 3)$ cắt mặt phẳng $(\alpha)$ theo giao tuyến là đường tròn có bán kính bằng $2$. Phương trình mặt cầu $(S)$ tương ứng là
\choice
{\True$(S):(x-2)^{2}+(y-1)^{2}+(z-3)^{2}=20$}
{$(S):(x-2)^{2}+(y-1)^{2}+(z-3)^{2}=12$}
{$(S):(x+2)^{2}+(y+1)^{2}+(z+3)^{2}=12$}
{$(S):(x-2)^{2}+(y-1)^{2}+(z-3)^{2}=16$}
\loigiai{
Khoảng cách từ $I$ đến $(\alpha)$ là $d(I,(\alpha))=\dfrac{|2+2-6-10|}{\sqrt{1+4+4}}=4$.\\
Bán kính mặt cầu là $R=\sqrt{r^2+d(I,(\alpha)^2)=2\sqrt{5}}$.\\
Phương trình mặt cầu $(S):(x-2)^2+(y-1)^2+(z-3)^2=20$.
}
\end{ex}

\begin{ex}%[2D3B1-3]%
Họ nguyên hàm $F(x)=\displaystyle\int\mathrm{e}^{\sqrt{x}}\mathrm{\,d}x$ là
\choice
{$F(x)=\dfrac{1}{2\sqrt{x}}\mathrm{e}^{\sqrt{x}}+C$}
{\True $F(x)=2\mathrm{e}^{\sqrt{x}}\left(\sqrt{x}-1\right)+C$}
{$F(x)=\mathrm{e}^{\tfrac{2}{3}x\sqrt{x}}+C$}
{$F(x)=\mathrm{e}^{\sqrt{x}}\left(\sqrt{x}+2\right)+C$}
\loigiai{
Đặt $\sqrt{x}=t \Rightarrow x=t^2 \Rightarrow \mathrm{\,d}x=2t\mathrm{\,d}t$.\\
Suy ra $F(x)=\displaystyle\int\mathrm{e}^{\sqrt{x}}\mathrm{\,d}x=2\displaystyle\int\mathrm{e}^{t}\mathrm{\,d}t$.\\
Đặt $\heva{&u=t\\&\mathrm{e}^{t}\mathrm{\,d}t=\mathrm{\,d}v} \Rightarrow \heva{&\mathrm{\,d}u=\mathrm{\,d}t\\&v=\mathrm{e}^{t}.}$\\
Khi đó $F(x)=2t\cdot\mathrm{e}^{t}-2\displaystyle\int\mathrm{e}^{t}\mathrm{\,d}t=2t\cdot\mathrm{e}^{t}-2\mathrm{e}^{t}+C=2\mathrm{e}^{\sqrt{x}}\left(\sqrt{x}-1\right)+C$.
}
\end{ex}

\begin{ex}%[tex hóa Tổng ôn LVĐ 21-22]%[2H3B1-1]%
Trong không gian $Oxyz$, cho hình hộp $ABCD.A'B'C'D'$ có $A(-3;2;1)$, $C(4;2;0)$, $B'(-2;1;1)$, $D'(3;5;4)$. Tọa độ đỉnh $A'$ là \choice
{$(-3;3;1) $}
{\True $(-3;3;3) $}
{$ (-3;-3;-3)$}
{$(-3;-3;3)$}
\loigiai{
Gọi $I = AC \cap BD$, $I'= A'C' \cap B'D'$. Khi đó $I\left(\dfrac{1}{2}; 2; \dfrac{1}{2} \right)$ và $I' \left( \dfrac{1}{2}; 3; \dfrac{5}{2}\right)$.\\
Gọi $A'(x;y;z)$. Do $ABCD.A'B'C'D'$ là hình hộp nên
$$\vec{AA'}=\vec{II'}\Leftrightarrow \heva{&x +3=0   \\& y-2 =1 \\&z-1=2}\Leftrightarrow \heva{&x =-3 \\ & y= 3\\ & z =3}. $$
Vậy $A'(-3;3;3)$.
}
\end{ex}

\begin{ex}%[2D3B1-1]%
Hàm số $F(x)=\mathrm{e}^{x^2}$ là một nguyên hàm của hàm số
\choice
{$f(x)=\mathrm{e}^{x^2}$}
{\True $f(x)=2x \cdot \mathrm{e}^{x^2}$}
{$f(x)=\dfrac{\mathrm{e}^{x^2}}{2x}$}
{$f(x)=x^2 \cdot \mathrm{e}^{x^2-1}$}
\loigiai{
Ta có $f(x)=F'(x)=\left(\mathrm{e}^{x^2}\right)'=(x^2)' \mathrm{e}^{x^2}=2x \mathrm{e}^{x^2}$.
}
\end{ex}

\begin{ex}%[2D3B1-2]%
Họ nguyên hàm của hàm số $f(x)=\dfrac{1}{x}(2x-\ln x)$ là
\choice
{\True $2x-\dfrac{\ln^2x}{2}+C$}
{$2x-\dfrac{1}{x^2}+C$}
{$\dfrac{2\ln\left| x\right|}{x}-\dfrac{1}{x}+C$}
{$2x-\dfrac{\ln x}{x}+C$}
\loigiai{
Ta có $f(x)=\dfrac{1}{x}(2x-\ln x)=2-\dfrac{1}{x}\ln x$.\\
Suy ra $\displaystyle\int f(x)\mathrm{\,d}x=\displaystyle\int 2\mathrm{\,d}x-\displaystyle\int \dfrac{\ln x\mathrm{\,d}x}{x}=2x-\displaystyle\int \ln x\mathrm{\,d}(\ln x)=2x-\dfrac{1}{2}{\ln ^2}x+C$.}
\end{ex}

\begin{ex}%[2D3B2-3]%Câu 468.
Biết $\displaystyle\int_0^2 2 x\ln (x+1) \mathrm{\,d}x=a\ln b$ với $a$, $b\in \mathbb{N}^*$ và $b$ là số nguyên tố. Khi đó $6a+7b$ bằng
\choice
{$33$}
{$25$}
{$42$}
{\True $39$}
\loigiai{
Đặt $\heva{& u=\ln (x+1)\Rightarrow \mathrm{\,d}u=\dfrac{1}{x+1}\mathrm{\,d}x \\ & \mathrm{\,d}v=2x \mathrm{\,d}x\Rightarrow v=x^2.}$\\
Khi đó
\allowdisplaybreaks
\begin{eqnarray*}
\displaystyle\int_0^2 2 x\ln (x+1) \mathrm{\,d}x&=&x^2\cdot\ln (x+1)\bigg|_0^2-\displaystyle\int_0^2 \dfrac{x^2}{x+1} \mathrm{\,d}x\\
&=&4\ln 3-\displaystyle\int_0^2 \left(x-1+\dfrac{1}{x+1}\right)\mathrm{\,d}x\\
&=& 4\ln 3 -\left(\dfrac{x^2}{2}-x+\ln|x+1|\right)\bigg|_0^2\\
&=& 3\ln 3.
\end{eqnarray*}
Vậy $6a+7b=6\cdot 3+7\cdot 3=39$.
}
\end{ex}

\begin{ex}%[2H3B1-1]%
Trong không gian với hệ tọa độ $Oxyz$, cho điểm $A(3;4;0)$ và điểm $B(3;-4;0)$. Chu vi của tam giác $OAB$ bằng
\choice
{$12$}
{$20$}
{$10+4\sqrt{2}$}
{\True $18$}
\loigiai{
Ta có $\heva{&OA=\sqrt{3^2+4^2+0^2}=5\\& OB=\sqrt{3^2+(-4)^2+0^2}=5\\& AB=\sqrt{\left(x_B-x_A\right)^2+\left(y_B-y_A\right)^2}=8.}$\\
Suy ra chu vi tam giác $OAB$ bằng $OA+OB+AC=18$.
}
\end{ex}

\begin{ex}%[2D3B2-3]%Câu 463.
Cho tích phân $I=\displaystyle\int_1^{\mathrm{e}} x\ln^2 x \mathrm{\,d}x$. Mệnh đề nào dưới đây đúng?
\choice
{$I=\dfrac{1}{2} x^2\ln^2 x\bigg|_1^{\mathrm{e}}+\displaystyle\int_1^{\mathrm{e}} x\ln x \mathrm{\,d}x$}
{$I=x^2\ln^2 x\bigg|_1^{\mathrm{e}}-2\displaystyle\int_1^{\mathrm{e}} x\ln x \mathrm{\,d}x$}
{$I=x^2\ln^2 x\bigg|_1^{\mathrm{e}}-\displaystyle\int_1^{\mathrm{e}} x\ln x \mathrm{\,d}x$}
{\True $I=\dfrac{1}{2} x^2\ln^2 x\bigg|_1^{\mathrm{e}}-\displaystyle\int_1^{\mathrm{e}} x\ln x \mathrm{\,d}x$}
\loigiai{
Đặt $\heva{& u=\ln^2x\Rightarrow \mathrm{\,d}u=\dfrac{2}{x}\cdot\ln x\mathrm{\,d}x \\ & \mathrm{\,d}v=x \mathrm{\,d}x\Rightarrow v=\dfrac{1}{2}x^2.}$\\
Khi đó $I=\dfrac{1}{2} x^2\ln^2 x\bigg|_1^{\mathrm{e}}-\displaystyle\int_1^{\mathrm{e}} x\ln x \mathrm{\,d}x$.
}
\end{ex}

\begin{ex}%[2D3B1-1]%
Cho biết hàm số $f(x)$ có đạo hàm là $f'(x)$ và có một nguyên hàm là $F(x)$. Khi đó giá trị của nguyên hàm $\displaystyle\int\left[2 f(x)+f'(x)+1\right]\mathrm{\,d}x$ bằng
\choice
{$2 F(x)+x f(x)+C$}
{$2x F(x)+f(x)+x+C$}
{$2x F(x)+x+1$}
{\True $2 F(x)+f(x)+x+C$}
\loigiai{
Ta có $\displaystyle\int\left[2 f(x)+f'(x)+1\right]\mathrm{\,d}x=2F(x)+f(x)+x+C$.
}
\end{ex}

\begin{ex}%[2H3B1-3]%
Trong không gian $ Oxyz $, cho mặt cầu $ (S) $ có tâm nằm trên $ (P):x-2y-z-6=0$ và đi qua ba điểm $ A(1 ; 2 ; 1) $, $ B(-1 ; -2 ; 1) $, $ C(3 ; 2 ; 1) $ có phương trình tương ứng
\choice
{$ (S): (x-1)^2+(y-2)^2+(z-3)^2=19$}
{\True $ (S): (x-2)^2+(y+1)^2+(z+2)^2=19$}
{$ (S): x^2+(y-3)^2+(z+1)^2=25$}
{$ (S): (x-1)^2+(y+1)^2+(z-1)^2=16$}
\loigiai{
Gọi mặt cầu $ (S) $ tâm $ I(a ; b ; c) $ có phương trình là $ x^2+y^2+z^2-2ax-2by-2cz+d=0 $.\\
$ A \in (S) : -2a - 4b -2c +d = -6. \quad (1)$\\
$ B \in (S) : 2a + 4b -2c +d = -6. \quad (2)$\\
$ A \in (S) : -6a - 4b -2c +d = -14. \quad (3)$\\
$ I \in (P) : a - 2b -c -6 = 0. \quad (4)$\\
Từ $ (1), (2), (3) $ và $ (4) $ suy ra $ a=2 $, $ b=-1 $, $ c=-2 $, $ d=-10 $.\\
Bán kính $ R=\sqrt{a^2+b^2+c^2-d}=\sqrt{19} $.\\
Phương trình mặt cầu $ (S): (x-2)^2+(y+1)^2+(z+2)^2=19$.
}
\end{ex}

\begin{ex}%[2D3B1-2]%
Tìm nguyên hàm $F(x)$ của hàm số $f(x)=\cos x\sqrt{\sin x+1}$.
\choice
{$F(x)=\dfrac{1}{3}(\sin x+1)\sqrt{\sin x+1}+C$}
{$F(x)=\dfrac{1-2\sin x-3\sin^2 x}{2\sqrt{\sin x+1}}$}
{\True $F(x)=\dfrac{2}{3}(\sin x+1)\sqrt{\sin x+1}+C$}
{$F(x)=\dfrac{1}{3}\sin x\sqrt{\sin x+1}+C$}
\loigiai{
Ta có $\displaystyle\int\cos x\sqrt{\sin x+1}\mathrm{\,d}x=\displaystyle\int\sqrt{\sin x+1}\mathrm{\,d}(\sin x+1)=\dfrac{2}{3}(\sin x+1)\sqrt{\sin x+1}+C$.
}
\end{ex}

\begin{ex}%[2H3B1-3]%
Trong không gian $O x y z$, cho mặt cầu $(S)$ 'có tâm $I(1 ; 2 ;-4)$ và thể tích bằng $36 \pi$. Phương trình của $(S)$ là
\choice
{\True $(x-1)^{2}+(y-2)^{2}+(z+4)^{2}=9$}
{$(x-1)^{2}+(y-2)^{2}+(z-4)^{2}=9$}
{$(x+1)^{2}+(y+2)^{2}+(z-4)^{2}=9$}
{$(x-1)^{2}+(y-2)^{2}+(z+4)^{2}=3$}
\loigiai{
Mặt cầu có 	thể tích bằng $36 \pi$ nên $\dfrac{4 \pi R^3}{3}=36 \pi \Leftrightarrow R^3=27 \Leftrightarrow R=3$.\\
Phương trình mặt cầu là $(x-1)^{2}+(y-2)^{2}+(z+4)^{2}=9$.
}
\end{ex}

\begin{ex}%[tex hóa Tổng ôn LVĐ 21-22]%[2H3B1-2]%]%
Trong không gian $Oxyz$, cho hình bình hành  $ABCD$ với  $A(1;1;1)$, $B(2;3;4)$ và $C(6;5;2)$. Diện tích hình bình hành $ABCD$  bằng \choice
{$3\sqrt{83} $}
{$\sqrt{83} $}
{$ 83 $}
{\True $2\sqrt{83}$}
\loigiai{ Ta có $\vec{AB}= (1;2;3)$, $\vec{AC}= (5;4;1)$. Suy ra $\left[\vec{AB}, \vec{AC} \right]= (-10;14;-6)$.\\
Suy ra $S_{ABCD} =\left|\left[\vec{AB}, \vec{AC} \right] \right|=2\sqrt{83}$.}
\end{ex}

\begin{ex}%[2D3B1-1]%
Họ nguyên hàm của hàm số $f(x)=(2+\mathrm{e}^{3x})^2$ là
\choice
{\True $4 x+\dfrac{4}{3} \mathrm{e}^{3x}+\dfrac{1}{6} \mathrm{e}^{6 x}+C$}
{$3x+\dfrac{4}{3} \mathrm{e}^{3x}+\dfrac{1}{6} \mathrm{e}^{6 x}+C$}
{$4 x+\dfrac{4}{3} \mathrm{e}^{3x}-\dfrac{1}{6} \mathrm{e}^{6 x}+C$}
{$3x+\dfrac{4}{3} \mathrm{e}^{3x}+\dfrac{5}{6} \mathrm{e}^{6 x}+C$}
\loigiai{
Ta có $\displaystyle\int f(x) \mathrm{\,d}x=\displaystyle\int (4+4\mathrm{e}^{3x}+\mathrm{e}^{6x}) \mathrm{\,d}x=4x+\dfrac{4}{3}\mathrm{e}^{3x}+\dfrac{1}{6}\mathrm{e}^{6x}+C$.
}
\end{ex}


\begin{ex}%[2H2K1-1]%
Cắt khối trụ bởi một mặt phẳng qua trục ta được thiết diện là hình chữ nhật $ABB_1A_1$ có $AB$ và $A_1B_1$ thuộc hai đáy của hình trụ với $AB=4a$ và $AA_1=5a$. Thể tích khối trụ đã cho bằng
\choice
{\True $12\pi a^3$}
{$16\pi a^3$}
{$4\pi a^3$}
{$\dfrac{17\pi a^3}{3}$}
\loigiai{\immini{Ta có $V=\pi R^2h=12\pi a^3$.}
{
\begin{tikzpicture}
\def\h{3}
\def\a{2}
\def\b{1}
\def\gA{0}
\def\gB{-180}
\path 	(180:\a) arc (180:\gA:{\a} and {\b}) coordinate (A)
(180:\a) arc (180:\gB:{\a} and {\b}) coordinate (B)
(0:0) coordinate (O)
\foreach \x in {A,B,O}{($(\x)+(90:\h)$) coordinate (\x_1)};
\fill[gray!50] (A)--(B)--(B_1)--(A_1)--cycle;
\draw[dashed] (180:\a) arc(180:0:{\a} and {\b});
\draw (180:\a) arc(180:360:{\a} and {\b});
\draw ($(180:\a)+(90:\h)$) arc (180:0:{\a} and {\b});
\draw ($(180:\a)+(90:\h)$) arc (180:360:{\a} and {\b});
\draw (180:\a)--($(180:\a)+(90:\h)$) (0:\a)--($(0:\a)+(90:\h)$);
\draw[dashed] (O)--(O_1) (A)--(A_1) (A)--(B);
\draw (B)--(B_1) (B_1)--(A_1);
\draw ($(A)!0.5!(A_1)$)node[right]{$3a$} ($(A)!0.5!(O)$)node[below]{$2a$};
\foreach \x/\g in {A/-90,B/-135,A_1/45,B_1/90,O/-135,O_1/135}
\fill[black] 	(\x) circle (1pt)
($(\g:4mm)+(\x)$) node {$\x$};
\end{tikzpicture}}}
\end{ex}

\begin{ex}%[2H2K1-1]%
Cho hình trụ $ (T) $ có hai đường tròn đáy với tâm lần lượt là $ O $ và $ O' $. Gọi $ AB $, $ CD $ lần lượt là hai đường kính của $ (O) $ và $ (O') $, góc giữa $ AB $ và $ CD $ bằng $ 30^\circ $, $ AB=6 $ và thể tích khối tứ diện $ ABCD $ bằng $ 30 $. Thể tích khối trụ đã cho bằng
\choice
{$180\pi$}
{\True  $90\pi$}
{$30\pi$}
{$45\pi$}
\loigiai{
\immini{
Gọi $h\,$, $V$ lần lượt là chiều cao và thể tích khối trụ $(T)$.\\
$\Rightarrow \mathrm{d}\,\left(AB,CD\right)=h$.\\
Ta có
\allowdisplaybreaks
\begin{eqnarray*}
V_{ABCD}
&=&\dfrac{1}{6}h\cdot\sin\left(AB,CD\right)\cdot AB\cdot CD\\
&=& \dfrac{1}{6}h\cdot\sin 30^\circ\cdot {6^2}\\
\Rightarrow h
&=& \dfrac{6V_{ABCD}}{\sin 30^\circ\cdot{6^2}}=10.
\end{eqnarray*}
Suy ra $V_{(T)}=\pi{\left(\dfrac{AB}{2}\right)^2}\cdot h=90\pi$.
}{
\begin{tikzpicture}[>=stealth,line join=round,line cap=round,font=\footnotesize,scale=0.9]
\def\R{3};
\def\r{0.7};
\def\h{4};
\path
(0,0) coordinate (J)
(90:\h) coordinate (I)
($(I)+(80:\R cm and \r cm)$) coordinate (A)
($(I)+(-100:\R cm and \r cm)$) coordinate (B)
(50:\R cm and \r cm) coordinate (D)
(-130:\R cm and \r cm) coordinate (C);
\pgfresetboundingbox
\draw
($(J)+(180:\R)$) arc(180:360:\R cm and \r cm)
(I) ellipse (\R cm and \r cm) (A)--(B)
($(I)+(0:\R)$)--++(-90:\h) ($(J)+(180:\R)$)--++(90:\h);
\draw[dashed]
($(J)+(0:\R)$) arc(0:180:\R cm and \r cm)
(B)--(C)--(A)--(D)--cycle (I)--(J) (C)--(D);
\foreach \x/\g in {A/90,B/130,C/-120,D/30,J/-90,I/90}\fill[black](\x) circle (1pt)+(\g:3mm)node{$\x$};
\end{tikzpicture}
}
}
\end{ex}

\Closesolutionfile{ans}
