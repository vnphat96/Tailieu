\begin{name}
 {Biên soạn: Lại Thị Hảo \\ Phản biện: Nguyễn Tài Tuệ}
 {Đề ôn tập chương III}
\end{name}

\caulc
\Opensolutionfile{ans}[ans/ans\currfilebase-Phan-I]

\begin{ex}%[2-D3B3-SO-8-2425]%[VN-MT-7, Lại Thị Hảo]%[2D3N1-2]
Xét mẫu số liệu ghép nhóm cho bởi bảng sau:
\begin{center}
\begin{tabular}{|c|c|c|c|c|c|c|}
 \hline
 Nhóm & $[40; 45)$ & $[45; 50)$ & $[50; 55)$ & $[55; 60)$ & $[60; 65)$ & \\
 \hline
 Tần số & $4$ & $11$ & $9$ & $8$ & $8$ & $n=40$ \\
 \hline
\end{tabular}
\end{center}
Khoảng biến thiên của mẫu số liệu ghép nhóm đã cho bằng
\choice
{$5$}
{$40$}
{$6$}
{\True $25$}
\loigiai{
 Ta có đầu mút trái của nhóm $1$ là $a_1=40$, đầu mút phải của nhóm $5$ là $a_6=65$.\\
 Vậy khoảng biến thiên của mẫu số liệu ghép nhóm đó là
 \[R=a_6-a_1=65-40=25.\] 
 }
\end{ex}

\begin{ex}%[2-D3B3-SO-8-2425]%[VN-MT-7, Lại Thị Hảo]%[2D3N1-1]
 Xét mẫu số liệu ghép nhóm cho bởi bảng sau:
\begin{center}
\begin{tabular}{|c|c|c|c|c|c|c|}
 \hline
 Nhóm & $[3; 13)$ & $[13; 23)$ & $[23; 33)$ & $[33; 43)$ & $[43; 53)$ & \\
 \hline
 Tần số & $8$ & $7$ & $10$ & $6$ & $9$ & $n=40$ \\
 \hline
\end{tabular}
\end{center}
 Tần số của nhóm $2$ của mẫu số liệu ghép nhóm đã cho bằng
 \choice
 {$6$}
 {\True $7$}
 {$9$}
 {$40$}
 \loigiai{Tần số của nhóm $2$ của mẫu số liệu ghép nhóm đã cho là $n_2=7$.}
\end{ex}

\begin{ex}%[2-D3B3-SO-8-2425]%[VN-MT-7, Lại Thị Hảo]%[2D3N1-4]
Xét mẫu số liệu ghép nhóm cho bởi bảng như hình bên.
 Tần số tích lũy $c f_2$ của nhóm $2$ của mẫu số liệu ghép nhóm đã cho bằng
 \choice
 {$4$}
 {$11$}
 {\True $15$}
 {$40$}
 \begin{center}
 \begin{tabular}{|c|c|c|c|c|c|c|}
 \hline
 Nhóm & $[17; 21)$ & $[21; 25)$ & $[25; 29)$ & $[29; 33)$ & $[33; 37)$ & \\
 \hline
 Tần số & $5$ & $10$ & $6$ & $7$ & $12$ & $n=40$ \\
 \hline
 \end{tabular}
 \end{center}
\loigiai{
 Ta có $c f_2=n_1+n_2=15$.
}
\end{ex}

\begin{ex}%[2-D3B3-SO-8-2425]%[VN-MT-7, Lại Thị Hảo]%[2D3N1-1]
 Khảo sát thời gian tập thể dục trong ngày của một số học sinh khối $11$ thu được mẫu số liệu ghép nhóm sau:
 \begin{center}
 \begin{tabular}{|c|c|c|c|c|c|}
 \hline
 Thời gian (phút) & $[0; 20)$ & $[20; 40)$ & $[40; 60)$ & $[60; 80)$ & $[80; 100)$\\
 \hline
 Số học sinh & $5$ & $9$ & $12$ & $10$ & $6$ \\
 \hline
 \end{tabular}
 \end{center}
 Giá trị đại diện của nhóm $[60; 80)$ là
 \choice
 {$10$}
 {$20$}
 {\True $70$}
 {$40$}
 \loigiai{
 Giá trị đại diện của nhóm $[60; 80)$ là $\dfrac{60+80}{2}=70$.
 }
\end{ex}

\begin{ex}%[2-D3B3-SO-8-2425]%[VN-MT-7, Lại Thị Hảo]%[2D3N1-1]
 Mẫu số liệu dưới đây ghi lại tốc độ của $40$ ô tô khi đi qua một trạm đo tốc độ (đơn vị: km/h):
 \begin{center}
 \begin{tabular}{|l|c|c|c|c|c|c|}
 \hline
 Tốc độ (km/h) & $[40; 45)$ & $[45; 50)$ & $[50; 55)$ & $[55; 60)$ & $[60; 65)$ & $[65; 70)$\\
 \hline
 Số ô tô & $4$ & $11$ & $7$ & $8$ & $8$ & $2$ \\
 \hline
 \end{tabular}
 \end{center}
 Độ dài của nhóm $[55; 60)$ là
 \choice
 {$10$}
 {$55$}
 {\True $5$}
 {$60$}
 \loigiai{
 Độ dài của nhóm $[55; 60)$ là $60-55=5$.
 }
\end{ex}

\begin{ex}%[2-D3B3-SO-8-2425]%[VN-MT-7, Lại Thị Hảo]%[1D5H1-3]
 Người ta đếm số xe ô tô đi qua một trạm thu phí mỗi phút trong khoảng thời gian từ $9$ giờ đến $9$ giờ $30$ phút sáng. Kết quả được ghi lại ở bảng sau:
 \begin{center}
 \begin{tabular}{|c|c|c|c|c|c|}
 \hline
 Số xe & $[6; 10]$ & $[11; 15]$ & $[16; 20]$ & $[21; 25]$ & $[26; 30]$\\
 \hline
 Số lần & $5$ & $9$ & $3$ & $9$ & $4$ \\
 \hline
 Giá trị đại diện & $8$ & $13$ & $18$ & $23$ & $28$ \\
 \hline
 \end{tabular}
 \end{center}
 Tính số trung bình cộng của mẫu số liệu ghép nhóm trên.
 \choice
 {$18{,}4$}
 {$18{,}7$}
 {$17{,}4$}
 {\True $17{,}7$}
 \loigiai{
 Số xe trung bình đi qua trạm trong mỗi phút xấp xỉ bằng
 \[\overline{x}=\dfrac{5 \cdot 8+9 \cdot 13+3 \cdot 18+9 \cdot 23+4 \cdot 28}{30}=17{,}7.\]
 }
\end{ex}

\begin{ex}%[2-D3B3-SO-8-2425]%[VN-MT-7, Lại Thị Hảo]%[1D5H2-3]
 Xét mẫu số liệu ghép nhóm cho bởi bảng sau:
\begin{center}
 \begin{tabular}{|c|c|c|}
 \hline
 Nhóm & Tần số & Tần số tích lũy\\
 \hline
 $[160; 163)$ & $6$ & $6$ \\
 $[163; 166)$ & $11$ & $17$ \\
 $[166; 169)$ & $9$ & $26$ \\
 $[169; 172)$ & $7$ & $33$ \\
 $[172; 175)$ & $3$ & $36$ \\
 \hline
 & $n=36$ & \\
 \hline
\end{tabular}
\end{center}
Tứ phân vị thứ nhất của mẫu số liệu ghép nhóm đã cho bằng
\choice
{\True $\dfrac{1802}{11}$}
{$163$}
{$9$}
{$\dfrac{329}{2}$}
 \loigiai{
Số phần tử của mẫu là $n=36$.\\
Ta có $\dfrac{n}{4}=\dfrac{36}{4}=9$ mà $6< 9< 17$. Suy ra nhóm $2$ là nhóm đầu tiên có tần số tích lũy lớn hơn hoặc bằng $9$.\\
Xét nhóm $2$ là nhóm $[163; 166)$ có $s=163$, $h=3$, $n_2=11$ và nhóm $1$ là nhóm $[160; 163)$ có $c f_1=6$.\\
Tứ phân vị thứ nhất là
\[Q_1=s+\left(\dfrac{9-c f_1}{n_2}\right) \cdot h=163+\left(\dfrac{9-6}{11}\right) \cdot 3=\dfrac{1802}{11}.\]
 }
\end{ex}

\begin{ex}%[2-D3B3-SO-8-2425]%[VN-MT-7, Lại Thị Hảo]%[1D5H2-3]
 Xét mẫu số liệu ghép nhóm cho bởi bảng sau:
\begin{center}
 \begin{tabular}{|c|c|c|}
\hline
Nhóm & Tần số & Tần số tích lũy\\
\hline
$[40; 45)$ & $5$ & $5$ \\
$[45; 50)$ & $10$ & $15$ \\
$[50; 55)$ & $7$ & $22$ \\
$[55; 60)$ & $9$ & $31$ \\
$[60; 65)$ & $7$ & $38$ \\
$[65; 70)$ & $4$ & $42$ \\
\hline
& $n=42$ & \\
\hline
 \end{tabular}
\end{center}
 Tứ phân vị thứ hai của mẫu số liệu ghép nhóm đã cho bằng
 \choice
 {\True $\dfrac{380}{7}$}
 {$50$}
 {$\dfrac{42}{7}$}
 {$\dfrac{105}{2}$}
\loigiai{
 Số phần tử của mẫu là $n=42$.\\
 Ta có $\dfrac{n}{2}=\dfrac{42}{2}=21$ mà $15< 21< 22$. Suy ra nhóm $3$ là nhóm đầu tiên có tần số tích lũy lớn hơn hoặc bằng $21$.\\
 Xét nhóm $3$ là nhóm $[50; 55)$ có $r=50$, $d=5$, $n_3=7$ và nhóm $2$ là nhóm $[45; 50)$ có $c f_2=15$.\\
 Tứ phân vị thứ hai là
 \[Q_2=r+\left(\dfrac{21-c f_2}{n_3}\right) \cdot d=50+\left(\dfrac{21-15}{7}\right) \cdot 5=\dfrac{380}{7}.\]
}
\end{ex}

\begin{ex}%[2-D3B3-SO-8-2425]%[VN-MT-7, Lại Thị Hảo]%[1D5H2-3]
Doanh thu bán hàng trong $20$ ngày được lựa chọn ngẫu nhiên của một của hàng được ghi lại ở bảng sau (đơn vị: triệu đồng):
\begin{center}
 \begin{tabular}{|c|c|c|c|c|c|}
 \hline
 Doanh thu & $[5; 7)$ & $[7; 9)$ & $[9; 11)$ & $[11; 13)$ & $[13; 15)$\\
 \hline
 Số ngày & $2$ & $7$ & $7$ & $3$ & $1$ \\
 \hline
\end{tabular}
\end{center}
Tứ phân vị thứ ba của mẫu số liệu gần nhất với giá trị nào trong các giá trị dưới đây?
\choice
{$10$}
{\True $11$}
{$12$}
{$13$}
 \loigiai{
Gọi $x_1$, $x_2$, $\ldots$, $x_{20}$ là doanh thu bán hàng trong $20$ ngày xếp theo thứ tự không giảm.\\
Khi đó $x_1$, $x_2 \in[5; 7)$; $x_3, \ldots, x_9 \in[7; 9)$; $x_9, \ldots, x_{16} \in[9; 11)$; $x_{17}, \ldots, x_{19} \in[11; 13)$; $x_{20} \in[13; 15)$.\\
Do đó, tứ phân vị thứ ba của mẫu số liệu thuộc nhóm $[9; 11)$.\\
Ta có $n=20$, $n_m=7$, $C=9$, $u_m=9$, $u_{m+1}=11$. Khi đó
\[Q_3=9+\dfrac{\dfrac{3 \cdot 20}{4}-9}{7}\cdot(11-9) =\dfrac{75}{7}\approx 11.\]
}
\end{ex}

\begin{ex}%[2-D3B3-SO-8-2425]%[VN-MT-7, Lại Thị Hảo]%[2D3H1-3]
Mẫu số liệu đây ghi lại tốc độ của $40$ ô tô khi đi qua một trạm đo tốc độ (đơn vị: km/h) được lập bảng tần số ghép nhóm như sau:
\begin{center}
\begin{tabular}{|c|c|c|c|c|c|c|}
 \hline
 Nhóm & $[40; 45)$ & $[45; 50)$ & $[50; 55)$ & $[55; 60)$ & $[60; 65)$ & $[65; 70)$ \\
 \hline
 Giá trị đại diện & $42{,}5$ & $47{,}5$ & $52{,}5$ & $57{,}5$ & $62{,}5$ & $67{,}5$ \\
 \hline
 Tần số & $4$ & $11$ & $7$ & $8$ & $8$ & $2$ \\
 \hline
\end{tabular}
\end{center}
 Khoảng tứ phân vị của mẫu số liệu trên gần bằng số nào dưới đây?
 \choice
 {$11{,}5$}
 {\True $12{,}3$}
 {$14{,}6$}
 {$23$}
\loigiai{
 Số phần tử của mẫu là $n=40$.\\
 Ta có $\dfrac{n}{4}=\dfrac{40}{4}=10$. Suy ra nhóm $2$ là nhóm đầu tiên có tần số tích lũy lớn hơn hoặc bằng $10$.\\
 Xét nhóm $2$ là nhóm $[45; 50)$ có $r=45$; $d=5$; $n_2=11$ và nhóm $1$ là nhóm $[40; 45)$ có $c f_1=4$.\\
 Áp dụng công thức, ta có $Q_1$ của mẫu số liệu là 
 \[Q_1=45+\left(\dfrac{10-4}{11}\right) \cdot 5\approx 47{,}7\, (\mathrm{km/h}).\]
 Ta có $\dfrac{3n}{4}=30$. Suy ra nhóm $4$ là nhóm đầu tiên có tần số tích lũy lớn hơn hoặc bằng $30$.\\ 
 Xét nhóm $4$ là nhóm $[55; 60)$ có $r=55$; $d=5$; $n_4=8$ và nhóm $3$ là nhóm $\left[50; 55\right)$ có $c f_3=22$.\\
 Áp dụng công thức, ta có $Q_3$ của mẫu số liệu là
 \[Q_3=55+\left(\dfrac{30-22}{8}\right) \cdot 5=60\,(\mathrm{km/h}).\]
 Do đó $\Delta_Q=Q_3-Q_1=60-\dfrac{525}{11}=\dfrac{135}{11} \approx 12{,}3$.
}
\end{ex}

\begin{ex}%[2-D3B3-SO-8-2425]%[VN-MT-7, Lại Thị Hảo]%[2D3H2-2]
Mỗi ngày bác An đều đi bộ để rèn luyện sức khỏe. Quãng đường đi bộ mỗi ngày (đơn vị: km) của bác An trong $20$ ngày được thống kê lại ở bảng sau:
\begin{center}
 \begin{tabular}{|c|c|c|c|c|c|}
\hline
Quãng đường (km) & $[2{,}7; 3{,}0)$ & $[3{,}0; 3{,}3)$ & $[3{,}3; 3{,}6)$ & $[3{,}6; 3{,}9)$ & $[3{,}9; 4{,}2)$\\
\hline
Số ngày & $3$ & $6$ & $5$ & $4$ & $2$ \\
\hline
 \end{tabular}
\end{center}
 Phương sai của mẫu số liệu ghép nhóm là
 \choice
 {$3{,}39$}
 {$11{,}62$}
 {\True $0{,}1314$}
 {$0{,}36$}
\loigiai{
 Ta có bảng sau:
\begin{center}
 \begin{tabular}{|c|c|c|c|c|c|}
\hline
Quãng đường (km) & $[2{,}7; 3{,}0)$ & $[3{,}0; 3{,}3)$ & $[3{,}3; 3{,}6)$ & $[3{,}6; 3{,}9)$ & $[3{,}9; 4{,}2)$\\
\hline
Giá trị đại diện & $2{,}85$ & $3{,}15$ & $3{,}45$ & $3{,}75$ & $4{,}05$ \\
\hline
Số ngày & $3$ & $6$ & $5$ & $4$ & $2$ \\
\hline
 \end{tabular}
\end{center}
 Số trung bình của mẫu số liệu ghép nhóm là
 \[\overline{x}=\dfrac{3 \cdot 2{,}85+6 \cdot 3{,}15+5 \cdot 3{,}45+4 \cdot 3{,}75+2 \cdot 4{,}05}{20}=3{,}39.\]
 Phương sai của mẫu số liệu ghép nhóm là 
 \[s^2=\dfrac{3 (2{,}85-3{,}39)^2+6(3{,}15-3{,}39)^2+5(3{,}45-3{,}39)^2+4(3{,}75-3{,}39)^2+2 (4{,}05-3{,}39)^2}{20}=0{,}1314.\]
}
\end{ex}

\begin{ex}%[2-D3B3-SO-8-2425]%[VN-MT-7, Lại Thị Hảo]%[2D3H2-2]
 Một bác tài xế thống kê lại độ dài quãng đường (đơn vị: km) bác đã lái xe mỗi ngày trong một tháng ở bảng sau:
\begin{center}
 \begin{tabular}{|c|c|c|c|c|c|}
 \hline
 Độ dài quãng đường (km) & $[50; 100)$ & $[100; 150)$ & $[150; 200)$ & $[200; 250)$ & $[250; 300)$\\
 \hline
 Số ngày & $5$ & $10$ & $9$ & $4$ & $2$ \\
 \hline
\end{tabular}
\end{center}
Độ lệch chuẩn của mẫu số liệu ghép nhóm gần bằng
\choice
{$33{,}91$}
{$155{,}15$}
{\True $55{,}68$}
{$36{,}54$}
 \loigiai{
Ta có bảng sau:
\begin{center}
 \begin{tabular}{|c|c|c|c|c|c|}
 \hline
Độ dài quãng đường (km)& $[50; 100)$ & $[100; 150)$ & $[150; 200)$ & $[200; 250)$ & $[250; 300)$\\
 \hline
 Giá trị đại diện & $75$ & $125$ & $175$ & $225$ & $275$ \\
 \hline
 Số ngày & $3$ & $6$ & $5$ & $4$ & $2$\\
 \hline
\end{tabular}
\end{center}
Số trung bình của mẫu số liệu ghép nhóm là
\[\overline{x}=\dfrac{5 \cdot 75+10 \cdot 125+9 \cdot 175+4 \cdot 225+2 \cdot 275}{30}=155.\]
Phương sai của mẫu số liệu ghép nhóm là
\[s^2=\dfrac{5 \cdot (75-155)^2+10 \cdot (125-155)^2+9 \cdot (175-155)^2+4 \cdot (225-155)^2+2 \cdot (275-155)^2}{30}=3\,100.\]
Độ lệch chuẩn của mẫu số liệu ghép nhóm là 
\[s=\sqrt{s^2}=\sqrt{3\,100} \approx 55{,}68.\]
 }
\end{ex}
\Closesolutionfile{ans}

\cauds
\Opensolutionfile{ans}[ans/ans\currfilebase-Phan-II]

\begin{ex}%[2-D3B3-SO-8-2425]%[VN-MT-7, Lại Thị Hảo]%[2D3N1-3]
 Cho bảng số liệu sau:
\begin{center}
 \begin{tabular}{|c|c|c|c|c|c|}
 \hline
 Nhóm & $[20; 25)$ & $[25; 30)$ & $[30; 35)$ & $[35; 40)$ & $[40; 45)$\\
 \hline
 Tần số & $6$ & $6$ & $4$ & $1$ & $1$ \\
 \hline
 \end{tabular}
\end{center}
 \choiceTF
 {\True Khoảng biến thiên của mẫu số liệu ghép nhóm là $25$}
 {\True Tần số của nhóm hai là $6$}
 {Tần số tích lũy của nhóm ba là $4$}
 {Khoảng tứ phân vị của mẫu số liệu ghép nhóm là hiệu giữa tứ phân vị thứ ba và tứ phân vị thứ hai của mẫu số liệu ghép nhóm}
 \loigiai{
\begin{itemchoice}
\itemch \textbf{Đúng}.\\
Khoảng biến thiên của mẫu số liệu ghép nhóm là $R=45-20=25$.
\itemch \textbf{Đúng}.\\
Tần số của nhóm hai (nhóm $[25;30)$) là $6$.
\itemch \textbf{Sai}.\\
Tần số tích lũy của nhóm ba là $6+6+4=16$.
\itemch \textbf{Sai}.\\
Khoảng tứ phân vị của mẫu số liệu ghép nhóm là hiệu giữa tứ phân vị thứ ba và tứ phân vị thứ nhất của mẫu số liệu ghép nhóm.
\end{itemchoice}}
\end{ex}

\begin{ex}%[2-D3B3-SO-8-2425]%[VN-MT-7, Lại Thị Hảo]%[2D3H1-3]
 Một vườn thú ghi lại tuổi thọ (đơn vị: năm) của $20$ con hổ và thu được kết quả như sau:
\begin{center}
 \begin{tabular}{|c|c|c|c|c|c|}
\hline
Tuổi thọ & $[14; 15)$ & $[15; 16)$ & $[16; 17)$ & $[17; 18)$ & $[18; 19)$\\
\hline
Số con hổ & $1$ & $3$ & $8$ & $6$ & $2$ \\
\hline
 \end{tabular}
\end{center}
 \choiceTF
 {\True Khoảng biến thiên của mẫu số liệu ghép nhóm này là $5$}
 {\True Nhóm chứa tứ phân vị thứ nhất là $[16; 17)$}
 {Nhóm chứa tứ phân vị thứ ba là $[18; 19)$}
 {\True Tần số tích lũy của nhóm $[17; 18)$ là $18$}
\loigiai{
\begin{itemchoice}
\itemch \textbf{Đúng}.\\
Khoảng biến thiên $R=19-14=5$.
\itemch \textbf{Đúng}.\\
Cỡ mẫu là $1+3+8+6+2=20$.\\
Gọi $x_1, x_2, \ldots, x_{20}$ là tuổi thọ của $20$ con hổ được sắp xếp theo thứ tự không giảm.\\
Tứ phân vị thứ nhất của mẫu số liệu gốc là $\dfrac{x_5+x_6}{2} \in[16; 17)$ nên nhóm chứa tứ phân vị thứ nhất là $[16; 17)$.
\itemch \textbf{Sai}.\\
Tứ phân vị thứ ba của mẫu số liệu gốc là $\dfrac{x_{15}+x_{16}}{2} \in[17; 18)$. Do đó nhóm chứa tứ phân vị thứ ba là $[17; 18)$.
\itemch \textbf{Đúng}.\\
Tần số tích lũy của nhóm $[17; 18)$ là $1+3+8+6=18$.
\end{itemchoice}
}
\end{ex}

\begin{ex}%[2-D3B3-SO-8-2425]%[VN-MT-7, Lại Thị Hảo]%[1D5H2-2]
Cho mẫu số liệu ghép nhóm về lương của nhân viên trong phòng kế toán tổng hợp một công ty X như sau:
\begin{center}
 \begin{tabular}{|l|c|c|c|c|c|}
\hline
 Lương (triệu đồng) & $[6; 9)$ & $[9; 12)$ & $[12; 15)$ & $[15; 18)$ & $[18; 21)$\\
\hline
Số nhân viên & $6$ & $5$ & $3$ & $2$ & $1$ \\
\hline
\end{tabular}
\end{center}
\choiceTF
{\True Giá trị đại diện của nhóm $[6; 9)$ là $7{,}5$}
{\True Trung bình lương các nhân viên là $11{,}2$ triệu đồng}
{Nhóm chứa trung vị là $[12; 15)$}
{\True Độ dài nhóm $[15; 18)$ là $3$}
\loigiai{
 \begin{itemchoice}
 \itemch \textbf{Đúng}.\\
 Giá trị đại diện của nhóm $[6; 9)$ là $\dfrac{6+9}{2}=7{,}5$.
 \itemch \textbf{Đúng}.\\
 Trung bình lương các nhân viên là
 \[\overline{x}=\dfrac{1}{17}(6\cdot 7{,}5+5\cdot 10{,}5+3\cdot 13{,}5+2\cdot 16{,}5+19{,}5)=11{,}2\, \text{(triệu đồng)}.\]
 \itemch \textbf{Sai}.\\
 Phòng kế toán có $17$ nhân viên. Vì $x_9 \in[9; 12)$ nên nhóm này chứa trung vị.
 \itemch \textbf{Đúng}.\\
 Độ dài nhóm $[15; 18)$ là $18-15=3$.
 \end{itemchoice}
 }
\end{ex}

\begin{ex}%[2-D3B3-SO-8-2425]%[VN-MT-7, Lại Thị Hảo]%[2D3V2-2]
 Cho mẫu số liệu ghép nhóm thống kê chiều cao (đơn vị: cm) của $45$ học sinh lớp 9A như sau:
 \begin{center}
 \begin{tabular}{|c|c|c|c|c|c|}
 \hline
 Nhóm & $[145; 150)$ & $[150; 155)$ & $[155; 160)$ & $[160; 165)$ & $[165; 170)$ \\
 \hline
 Tần số & $8$ & $12$ & $15$ & $6$ & $4$ \\
 \hline
 \end{tabular}
 \end{center}
 \choiceTF
 {Giá trị đại diện của nhóm $[150; 155)$ là $152$\,cm}
 {\True Chiều cao trung bình của học sinh là $155{,}94$\,cm}
 {Phương sai của mẫu số liệu (làm tròn đến hàng phần trăm) là $36{,}04$}
 {\True Độ lệch chuẩn của mẫu số liệu (làm tròn đến hàng phần trăm) là $5{,}85$}
\loigiai{
\begin{itemchoice}
 \itemch \textbf{Sai}.\\
 Giá trị đại diện của nhóm $[150; 155)$ là $\dfrac{150+155}{2}=152{,}5$.
 \itemch \textbf{Đúng}.\\
 Ta có bảng giá trị đại diện như sau:
 \begin{center}
 \begin{tabular}{|l|c|c|}
 \hline
 Nhóm & Giá trị đại diện & Tần số \\
 \hline
 $[145; 150)$ & $147{,}5$ & $8$ \\
 \hline
 $[150; 155)$ & $152{,}5$ & $12$ \\
 \hline
 $[155; 160)$ & $157{,}5$ & $15$ \\
 \hline
 $[160; 165)$ & $162{,}5$ & $6$ \\
 \hline
 $[165; 170)$ & $167{,}5$ & $4$ \\
 \hline
 \end{tabular}
 \end{center}
 Chiều cao trung bình của học sinh là
 \[\overline{x}=\dfrac{147{,}5 \cdot 8+152{,}5\cdot 12+157{,}5\cdot 15+162{,}5\cdot 6+167{,}5\cdot 4}{45}=\dfrac{2\,807}{18} \approx 155{,}94.\]
 \itemch \textbf{Sai}.\\
 Phương sai của mẫu số liệu là 
 \[s^2=\dfrac{8(147{,}5-155{,}94)^2+12(152{,}5-155{,}94)^2+\cdots+4(167{,}5-155{,}94)^2}{45}=\dfrac{2\,774}{81}\approx 34{,}25.\]
 \itemch \textbf{Đúng}.\\
 Độ lệch chuẩn $s=\sqrt{s^2}\approx 5{,}85$.
\end{itemchoice}
 }
\end{ex}
\Closesolutionfile{ans}

\caukq
\Opensolutionfile{ans}[ans/ans\currfilebase-Phan-III]

\begin{ex}%[2-D3B3-SO-8-2425]%[VN-MT-7, Lại Thị Hảo]%[2D3N1-2]
 Cho mẫu số liệu ghép nhóm số tiền điện phải trả trong một tháng của các hộ gia đình ở một khu phố (đơn vị: ngàn đồng) như sau:
\begin{center}
 \begin{tabular}{|l|c|c|c|c|c|c|}
 \hline
 Nhóm & $[375; 450)$ & $[450; 525)$ & $[525; 600)$ & $[600; 675)$ & $[675; 750)$ & $[750; 825]$\\
 \hline
 Tần số & $6$ & $15$ & $10$ & $6$ & $9$ & $4$ \\
 \hline
 \end{tabular}
\end{center}
 Tìm khoảng biến thiên của mẫu số liệu ghép nhóm trên.
 
 \shortans[]{450}
 \loigiai{
 Khoảng biến thiên của mẫu số liệu ghép nhóm trên là $R=a_7-a_1=825-375=450$. 
 }
\end{ex}

\begin{ex}%[2-D3B3-SO-8-2425]%[VN-MT-7, Lại Thị Hảo]%[2D3N1-2]
 Cho mẫu số liệu ghép nhóm về tuổi thọ (đơn vị tính là năm) của một loại bóng đèn mới như sau:
 \begin{center}
 \begin{tabular}{|l|c|c|c|c|}
 \hline
 Tuổi thọ & $[2; 3{,}5)$ & $[3{,}5; 5)$ & $[5; 6{,}5)$ & $[6{,}5; 8)$\\
 \hline
 Số bóng đèn & $8$ & $22$ & $35$ & $15$ \\
 \hline
 \end{tabular}
 \end{center}
 Tìm khoảng biến thiên của mẫu số liệu trên.
 
 \shortans[]{6}
 \loigiai{
 Khoảng biến thiên của mẫu số liệu trên là $8-2=6$. 
 }
\end{ex}

\begin{ex}%[2-D3B3-SO-8-2425]%[VN-MT-7, Lại Thị Hảo]%[2D3N1-4]
Cho bảng tần số ghép nhóm số liệu thống kê chiều cao của $38$ mẫu cây ở một vườn thực vật (đơn vị: centimét) như sau:
\begin{center}
 \begin{tabular}{|c|c|c|c|c|c|c|}
 \hline
 Nhóm & $[30; 40)$ & $[40; 50)$ & $[50; 60)$ & $[60; 70)$ & $[70; 80)$ & \\
 \hline
 Tần số & $4$ & $10$ & $14$ & $6$ & $4$ & $n=38$ \\
 \hline
 \end{tabular}
\end{center}
 Tần số tích luỹ của nhóm $4$ bằng bao nhiêu?
 \par
\shortans[]{34}
 \loigiai{
 Ta có bảng số liệu ghép nhóm như sau:
\begin{center}
 \begin{tabular}{|l|c|c|}
 \hline
 Nhóm & Tần số & Tần số tích lũy \\
 \hline
 $[30; 40)$ & $4$ & $4$ \\
 \hline
 $[40; 50)$ & $10$ & $14$ \\
 \hline
 $[50; 60)$ & $14$ & $28$ \\
 \hline
 $[60; 70)$ & $6$ & $34$ \\
 \hline
 $[70; 80)$ & $4$ & $38$ \\
 \hline
 \end{tabular}
\end{center} 
 Vậy tần số tích luỹ của nhóm $4$ là $34$. 
 }
\end{ex}

\begin{ex}%[2-D3B3-SO-8-2425]%[VN-MT-7, Lại Thị Hảo]%[1D5H1-3]
 Cân nặng của một số quả mít trong một khu vườn được thống kê ở bảng sau:
\begin{center}
 \begin{tabular}{|c|c|c|c|c|c|}
 \hline
 Cân nặng (kg) & $[4; 6)$ & $[6; 8)$ & $[8; 10)$ & $[10; 12)$ & $[12; 14)$\\
 \hline
 Số quả mít & $6$ & $12$ & $19$ & $9$ & $4$ \\
 \hline
 \end{tabular}
\end{center} 
 Tính cân nặng trung bình của một quả mít.
 
 \shortans[]{8{,}72}
 \loigiai{
 Số trung bình cộng của mẫu số liệu ghép nhóm là
 \[\overline{x}=\dfrac{6 \cdot 5+12 \cdot 7+19 \cdot 9+9 \cdot 11+4\cdot 13}{50}=8{,}72.\]
 Vậy cân nặng trung bình của một quả mít là $8{,}72$ kg.
 }
\end{ex}

\begin{ex}%[2-D3B3-SO-8-2425]%[VN-MT-7, Lại Thị Hảo]%[1D5H2-3]
 Để đánh giá chất lượng dịch vụ tài xế công nghệ của hãng X, người ta ghi lại thời gian chờ của các khách hàng được thể hiện trong bảng sau:
\begin{center}
 \begin{tabular}{|c|c|c|c|c|c|}
 \hline
 Thời gian chờ (phút)& $[1; 2{,}5)$ & $[2{,}5; 4)$ & $[4; 5{,}5)$ & $[5{,}5; 7)$ & $[7; 8{,}5)$\\
 \hline
 Lượng khách hàng (tần số) & $10$ & $5$ & $23$ & $6$ & $3$ \\
 \hline
 \end{tabular}
\end{center}
 Tìm tứ phân vị thứ nhất của mẫu số liệu trên (kết quả làm tròn đến hàng phần trăm).
 
 \shortans[]{3{,}03}
 \loigiai{
 Cỡ mẫu là $n=10+5+23+6+3=47$.\\
 Gọi $x_1, \ldots, x_{47}$ là thời gian chờ của $47$ khách hàng và giả sử số liệu gốc này đã được sắp xếp theo thứ tự không giảm.\\
 Tứ phân vị thứ nhất của mẫu số liệu gốc là $x_{12}$ nên nhóm chứa $Q_1$ là nhóm $[2{,}5; 4)$.\\
 Khi đó $Q_1=2{,}5+\dfrac{\dfrac{1 \cdot 47}{4}-10}{5} \cdot 1{,}5=3{,}025\approx 3{,}03$.
 
 }
\end{ex}

\begin{ex}%[2-D3B3-SO-8-2425]%[VN-MT-7, Lại Thị Hảo]%[2D3H2-2]
 Tìm hiểu thời gian sử dụng điện thoại trong một ngày của các bạn học sinh lớp 12A được ghi lại trong bảng sau:
\begin{center}
 \begin{tabular}{|l|c|c|c|c|}
 \hline
 Thời gian (giờ) & $[0; 1{,}5)$ & $[1{,}5; 3)$ & $[3; 4{,}5)$ & $[4{,}5; 6)$\\
 \hline
 Số học sinh & $8$ & $12$ & $6$ & $4$ \\
 \hline
 \end{tabular}
\end{center}
 Tìm phương sai của mẫu số liệu trên.
 
 \shortans[]{2{,}16}
 \loigiai{
 Chọn giá trị đại diện cho các nhóm số liệu, ta có:
\begin{center}
 \begin{tabular}{|l|c|c|c|c|}
 \hline
 Thời gian (giờ) & $[0; 1{,}5)$ & $[1{,}5; 3)$ & $[3; 4{,}5)$ & $[4{,}5; 6)$\\
 \hline
 Giá trị đại diện & $0{,}75$ & $2{,}25$ & $3{,}75$ & $5{,}25$ \\
 \hline
 Số học sinh & $8$ & $12$ & $6$ & $4$ \\ 
 \hline
 \end{tabular}
\end{center}
 Thời gian sử dụng điện thoại trung bình của các bạn lớp 12A là
 \[\overline{x}=\dfrac{1}{30}(8 \cdot 0{,}75+12 \cdot 2{,}25+6 \cdot 3{,}75+4 \cdot 5{,}25)=2{,}55.\]
 Phương sai của mẫu số liệu trên là \[s^2=\dfrac{1}{30}\left(8 \cdot 0{,}75^2+12 \cdot 2{,}25^2+6 \cdot 3{,}75^2+4 \cdot 5{,}25^2\right)-2{,}55^2=2{,}16.\]
 }
\end{ex}
\Closesolutionfile{ans}
\begin{indapan}
	{ans/ans\currfilebase}
\end{indapan}

