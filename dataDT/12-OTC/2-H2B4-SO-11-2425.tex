\begin{name}
	{\tenchude}
	{ĐỀ ÔN TẬP CHƯƠNG II}
	{LỚP TOÁN THẦY PHÁT}
	{\thoigian}
\end{name}

\TN
\Opensolutionfile{ans}[ans/ans\currfilebase-Phan-I]
\begin{ex}%[2-H2B4-SO-11-2425]%[VN-MT-7, Đào Trung Kiên]%[2H2N1-2]
Cho tứ diện $ABCD$. Có bao nhiêu vectơ có điểm đầu là $A$ và điểm cuối là một trong các đỉnh còn lại của tứ diện?
\choice{$1$}
{$2$}
{\True $3$}
{$4$}
\loigiai{
\begin{center}
\begin{tikzpicture}[font=\footnotesize, line join=round, line cap=round, >=stealth, scale=0.9]
 \def\a{4}
 \path (0:0) coordinate (B)
 ++(0:\a) coordinate (D)
 ++(-120:\a/2) coordinate (C)
 ($(B)+(60:\a)$) coordinate (A);
 \draw[dashed] (B)--(D);
 \draw[->] (A)--(C);
 \draw[->] (A)--(D);
 \draw[->] (A)--(B);
 \draw (B)--(C)--(D);
 \foreach \x/\g in {A/90,B/180,C/-45,D/0}
 \fill (\x) circle (1pt)
 ($(\g:3mm)+(\x)$) node {$\x$};
\end{tikzpicture}
\end{center}
Ba vectơ $\overrightarrow{AB}$, $\overrightarrow{AC}$, $\overrightarrow{AD}$.
}
\end{ex}

\begin{ex}%[2-H2B4-SO-11-2425]%[VN-MT-7, Đào Trung Kiên]%[2H2N1-1]
Cho hình lập phương $ABCD.A'B'C'D'$. Hai vectơ nào dưới đây có giá cùng nằm trong mặt phẳng $(ABCD)$?
 \choice{$\overrightarrow{DD'}$, $\overrightarrow{AC}$}
 {$\overrightarrow{AD'}$, $\overrightarrow{AD}$}
 {$\overrightarrow{AD'}$, $\overrightarrow{AC}$}
 {\True $\overrightarrow{AC}$, $\overrightarrow{AD}$}
 \loigiai{
\begin{center}
\begin{tikzpicture}[font=\footnotesize, line join=round, line cap=round, >=stealth, scale=1]
 \def\a{3.5}
 \path (0:0) coordinate (A)
 ++(0:\a) coordinate (D)
 ++(-130:\a/2) coordinate (C)
 ($(A)+(C)-(D)$) coordinate (B)
 ($(A)+(90:\a)$) coordinate (A')
 ($(B)+(90:\a)$) coordinate (B')
 ($(C)+(90:\a)$) coordinate (C')
 ($(D)+(90:\a)$) coordinate (D');
 \draw[dashed] (B)--(A) (A)--(A');
 \draw[dashed,->] (A)--(C);
 \draw[dashed,->] (A)--(D);
 \draw (C)--(C') (D)--(D') (B)--(B') (B)--(C)--(D) (A')--(B')--(C')--(D')--cycle;
 \foreach \x/\g in {A/180,B/180,C/0,D/0,A'/180,B'/180,C'/0,D'/0}
 \fill (\x) circle (1pt)
 ($(\g:4mm)+(\x)$) node {$\x$}; 
\end{tikzpicture}
\end{center}
Hai vectơ $\overrightarrow{AC}$, $\overrightarrow{AD}$ có giá cùng nằm trong mặt phẳng $(ABCD)$.
 }
\end{ex}

\begin{ex}%[2-H2B4-SO-11-2425]%[VN-MT-7, Đào Trung Kiên]%[2H2N1-1]
Cho hình lập phương $ABCD.A'B'C'D'$ có cạnh là $a$. Hai vectơ nào dưới đây có cùng độ dài?
 \choice{$\overrightarrow{DD'}$, $\overrightarrow{AC}$}
 {$\overrightarrow{AD'}$, $\overrightarrow{AD}$}
 {\True $\overrightarrow{AD'}$, $\overrightarrow{AC}$}
 {$\overrightarrow{AC}$, $\overrightarrow{AD}$}
 \loigiai{
\begin{center}
 \begin{tikzpicture}[font=\footnotesize, line join=round, line cap=round, >=stealth, scale=1]
 \def\a{3.5}
 \path (0:0) coordinate (A)
 ++(0:\a) coordinate (B)
 ++(-130:\a/2) coordinate (C)
 ($(A)+(C)-(B)$) coordinate (D)
 ($(A)+(-90:\a)$) coordinate (A')
 ($(B)+(-90:\a)$) coordinate (B')
 ($(C)+(-90:\a)$) coordinate (C')
 ($(D)+(-90:\a)$) coordinate (D');
 \draw[dashed] (B)--(A) (A)--(A')--(B') (A')--(D');
 \draw[->] (A)--(C);
 \draw[->] (A)--(D);
 \draw[dashed,->] (A)--(D');
 \draw (C)--(C')--(B') (D)--(D')--(C') (A)--(B)--(B') (B)--(C)--(D) ;
 \foreach \x/\g in {A/180,D/180,C/0,B/0,A'/180,D'/180,C'/0,B'/0}
 \fill (\x) circle (1pt)
 ($(\g:4mm)+(\x)$) node {$\x$}; 
 \end{tikzpicture}
\end{center}
Vì $\left|\overrightarrow{AD'}\right|=\left|\overrightarrow{AC}\right|=a\sqrt{2}$ nên hai vectơ $\overrightarrow{AD'}$ và $\overrightarrow{AC}$ có cùng độ dài.
 }
\end{ex}

\begin{ex}%[2-H2B4-SO-11-2425]%[VN-MT-7, Đào Trung Kiên]%[2H2N1-1]
Cho hình lập phương $ABCD.A'B'C'D'$ có cạnh là $a$. Vectơ nào bằng vectơ $\overrightarrow{D'C'}$?
 \choice{$\overrightarrow{DD'}$}
 {$\overrightarrow{AD}$}
 {\True $\overrightarrow{AB}$}
 {$\overrightarrow{CD}$}
 \loigiai{
\begin{center}
 \begin{tikzpicture}[font=\footnotesize, line join=round, line cap=round, >=stealth, scale=1]
 \def\a{3.5}
 \path (0:0) coordinate (A)
 ++(0:\a) coordinate (B)
 ++(-130:\a/2) coordinate (C)
 ($(A)+(C)-(B)$) coordinate (D)
 ($(A)+(-90:\a)$) coordinate (A')
 ($(B)+(-90:\a)$) coordinate (B')
 ($(C)+(-90:\a)$) coordinate (C')
 ($(D)+(-90:\a)$) coordinate (D');
 \draw[dashed] (B)--(A) (A)--(A')--(B') (A')--(D');
 \draw[->] (A)--(B);
 \draw[->] (D')--(C');
 \draw (C)--(C')--(B') (A)--(D)--(D') (A)--(B)--(B') (B)--(C)--(D) ;
 \foreach \x/\g in {A/180,D/180,C/0,B/0,A'/180,D'/180,C'/0,B'/0}
 \fill (\x) circle (1pt)
 ($(\g:4mm)+(\x)$) node {$\x$}; 
 \end{tikzpicture}
\end{center}
Vì hai vectơ $\overrightarrow{AB}$ và $\overrightarrow{D'C'}$ có cùng hướng và cùng độ dài nên $\overrightarrow{AB}=\overrightarrow{D'C'}$.
 }
\end{ex}

\begin{ex}%[2-H2B4-SO-11-2425]%[VN-MT-7, Đào Trung Kiên]%[2H2N2-1]
Trong không gian với hệ tọa độ $Oxyz$, cho hai điểm $A(1; -1; 2)$ và $B(2; 1; -4)$. Vectơ $\overrightarrow{AB}$ có tọa độ là
 \choice{\True $(1;2;-6)$}
 {$(1; 0; -6)$}
 {$(-1; -2; 6)$}
 {$(3; 0; -2)$}
 \loigiai{
Ta có $\overrightarrow{AB}=(2-1; 1-(-1); -4-2)\Rightarrow \overrightarrow{AB}=(1; 2; -6)$.
 }
\end{ex}

\begin{ex}%[2-H2B4-SO-11-2425]%[VN-MT-7, Đào Trung Kiên]%[2H2N2-3]
Trong không gian với hệ tọa độ $Oxyz$, cho biểu diễn của vectơ $\overrightarrow{a}$ qua các vectơ đơn vị là $\overrightarrow{a}=2\overrightarrow{i}+\overrightarrow{k}-3\overrightarrow{j}$. Tọa độ của vectơ $\overrightarrow{a}$ là
 \choice{\True $(2; -3; 1)$}
 {$(1; -3; 2)$}
 {$(2; 1; -3)$}
 {$(1; 2; -3)$}
 \loigiai{ Ta có $\overrightarrow{a}=2\cdot\overrightarrow{i}-3\cdot\overrightarrow{j}+1\cdot \overrightarrow{k}$ nên $\overrightarrow{a}=(2; -3; 1)$.
 }
\end{ex}

\begin{ex}%[2-H2B4-SO-11-2425]%[VN-MT-7, Đào Trung Kiên]%[2H2H2-2]
Trong không gian với hệ tọa độ $Oxyz$, cho hình bình hành $ABCD$ với các đỉnh có tọa độ là $A(3; 1; 2)$, $B(1; 0; 1)$, $C(2; 3; 0)$. Tọa độ đỉnh $D$ là
 \choice{$D(1; 1; 0)$}
 {$D(0; 2; -1)$}
 {\True $D(4; 4; 1)$}
 {$D(1; 3; -1)$}
 \loigiai{
Ta có $ABCD$ là hình bình hành nên $\overrightarrow{AD}=\overrightarrow{BC}\Leftrightarrow\heva{&x_D-3=1\\&y_D-1=3\\&z_D-2=-1}\Leftrightarrow\heva{&x_D=4\\&y_D=4\\&z_D=1}\Rightarrow D(4; 4; 1)$.
 }
\end{ex}

\begin{ex}%[2-H2B4-SO-11-2425]%[VN-MT-7, Đào Trung Kiên]%[2H2H2-2]
Trong không gian với hệ tọa độ $Oxyz$, cho vectơ $\overrightarrow{a}=(-3; 2; 1)$ và điểm $A(4; 6; -3)$. Tọa độ điểm $B$ thỏa mãn $\overrightarrow{AB}=\overrightarrow{a}$ là
 \choice{$(-1; -8; 2)$}
 {$(7; 4; -4)$}
 {\True $(1; 8; -2)$}
 {$(-7; -4; 4)$}
 \loigiai{
Đặt $B(x; y; z)$. Ta có $\overrightarrow{AB}=(x-4; y-6; z+3)$.\\
Khi đó $\overrightarrow{AB}=\overrightarrow{a}\Leftrightarrow \heva{&x-4=-3\\&y-6=2\\&z+3=1}\Leftrightarrow \heva{&x=1\\&y=8\\&z=-2.}$\\
Vậy $B(1; 8; -2)$.
 }
\end{ex}

\begin{ex}%[2-H2B4-SO-11-2425]%[VN-MT-7, Đào Trung Kiên]%[2H2H2-3]
Trong không gian với hệ tọa độ $Oxyz$, cho ba vectơ $\overrightarrow{a}=(1; 2; 3)$, $\overrightarrow{b}=(-2; 0; 1)$,  $\overrightarrow{c}=(-1; 0; 1)$. Tìm tọa độ của vectơ $\overrightarrow{n}=\overrightarrow{a}+\overrightarrow{b}+2\overrightarrow{c}-3\overrightarrow{i}$.
 \choice{$\overrightarrow{n}=(6; 2; 6)$}
 {$\overrightarrow{n}=(6; 2; -6)$}
 {$\overrightarrow{n}=(0; 2; 6)$}
 {\True $\overrightarrow{n}=(-6; 2; 6)$}
 \loigiai{Vì $2\overrightarrow{c}=(-2; 0; 2)$ và $-3\overrightarrow{i}=(-3; 0; 0)$ nên $\overrightarrow{n}=\overrightarrow{a}+\overrightarrow{b}+2\overrightarrow{c}-3\overrightarrow{i}$ có tọa độ $(-6; 2; 6)$.
 }
\end{ex}

\begin{ex}%[2-H2B4-SO-11-2425]%[VN-MT-7, Đào Trung Kiên]%[2H2H2-2]
Trong không gian với hệ tọa độ $Oxyz$, cho ba điểm $A(3; 5; -1)$, $B(7; x; 1)$ và $C(9; 2; y)$. Để $A$, $B$, $C$ thẳng hàng thì $x+y$ bằng
 \choice{\True $5$}
 {$6$}
 {$4$}
 {$7$}
 \loigiai{
Ta có $\overrightarrow{AB}=(4; x-5; 2)$, $\overrightarrow{AC}=(6; -3; y+1)$.\\
Vì $\overrightarrow{AB}\neq \overrightarrow{0}$ nên $A$, $B$, $C$ thẳng hàng khi $\overrightarrow{AB}$, $\overrightarrow{AC}$ cùng phương\\
\[\Leftrightarrow \overrightarrow{AB}=k\overrightarrow{AC}\Leftrightarrow \heva{&4=k\cdot 6\\&x-5=k\cdot(-3)\\&2=k(y+1)}\Leftrightarrow \heva{&k=\dfrac{2}{3}\\&x=3\\&y=2.}\]
Vậy $x+y=5$.
 }
\end{ex}

\begin{ex}%[2-H2B4-SO-11-2425]%[VN-MT-7, Đào Trung Kiên]%[2H2H2-4]
Trong không gian với hệ tọa độ $Oxyz$, điểm $M$ thuộc trục $Ox$ và cách đều hai điểm $A(4; 2; -1)$ và $B(2; 1; 0)$ là
 \choice{$M(-4; 0; 0)$}
 {$M(5; 0; 0)$}
 {\True $M(4; 0; 0)$}
 {$M(-5; 0; 0)$}
 \loigiai{$M\in Ox\Rightarrow M(x; 0; 0)$. Ta có $\overrightarrow{MA}=(4-x; 2; -1)$, $\overrightarrow{MB}=(2-x; 1; 0)$.\\
$M$ cách đều hai điểm $A$, $B$ khi \[MA=MB\Leftrightarrow \sqrt{(4-x)^2+2^2+(-1)^2}=\sqrt{(2-x)^2+1^2+0^2}\Leftrightarrow x=4\]
 }
\end{ex}

\begin{ex}%[2-H2B4-SO-11-2425]%[VN-MT-7, Đào Trung Kiên]%[2H2H2-2]
Trong không gian với hệ tọa độ $Oxyz$, cho ba điểm $A(1; 3; 4)$, $B(1; 0; -2)$ và $C(4; 0; 1)$. Tọa độ trọng tâm $G$ của tam giác $ABC$ là
 \choice{$G(3; 0; 2)$}
 {\True $G(2; 1; 1)$}
 {$G(1; 1; 3)$}
 {$G(3; 0; -1)$}
 \loigiai{
Tọa độ trọng tâm của tam giác $ABC$ là $G(2; 1; 1)$.
 }
\end{ex}
\Closesolutionfile{ans}

\TNTF
\Opensolutionfile{ans}[ans/ans\currfilebase-Phan-II]
\begin{ex}%[2-H2B4-SO-11-2425]%[VN-MT-7, Đào Trung Kiên]%[2H2V1-4]
\immini{Một chất điểm ở vị trí $A$ của hình lập phương $ABCD.A'B'C'D'$. Chất điểm chịu tác động bởi ba lực $\overrightarrow{a}$, $\overrightarrow{b}$, $\overrightarrow{c}$ lần lượt cùng hướng với $\overrightarrow{AD}$, $\overrightarrow{AB}$, $\overrightarrow{AC'}$ như hình vẽ bên. Độ lớn của lực $\overrightarrow{a}$, $\overrightarrow{b}$ và $\overrightarrow{c}$ tương ứng là $10$ N, $10$ N và $10\sqrt{3}$ N.
\choiceTF
{$\overrightarrow{a}+\overrightarrow{b}=\overrightarrow{c}$}
{$\left|\overrightarrow{a}+\overrightarrow{b}\right|=20$ (N)}
{\True $\left|\overrightarrow{a}+\overrightarrow{c}\right|=\left|\overrightarrow{b}+\overrightarrow{c}\right|$}
{\True $\left|\overrightarrow{a}+\overrightarrow{b}+\overrightarrow{c}\right|=30$ (N)}
}{ \begin{tikzpicture}[font=\footnotesize, line join=round, line cap=round, >=stealth, scale=0.7]
 \def\a{3.5}
 \path (0:0) coordinate (A)
 ++(0:\a) coordinate (B)
 ++(-130:\a/2) coordinate (C)
 ($(A)+(C)-(B)$) coordinate (D)
 ($(A)+(-90:\a)$) coordinate (A')
 ($(B)+(-90:\a)$) coordinate (B')
 ($(C)+(-90:\a)$) coordinate (C')
 ($(D)+(-90:\a)$) coordinate (D')
 ($(B)!0.5!(A)$) coordinate (M)
 ($(C')!0.5!(A)$) coordinate (N)
 ($(D)!0.5!(A)$) coordinate (P);
 \draw[dashed] (B)--(A)--(C') (A)--(A')--(B') (A')--(D');
 \draw[->,thick] (A)--(M)node[midway, above] {$\overrightarrow{b}$};
 \draw[->] (D')--(C');
 \draw[->,thick] (A)--(N)node[midway, above right] {$\overrightarrow{c}$};
 \draw[->,thick] (A)--(P)node[midway, above left] {$\overrightarrow{a}$};
 \draw (C)--(C')--(B') (B)--(A)--(D)--(D') (A)--(B)--(B') (B)--(C)--(D) ;
 \foreach \x/\g in {A/90,D/180,C/0,B/0,A'/180,D'/180,C'/0,B'/0}
 \fill (\x) circle (1pt)
 ($(\g:4mm)+(\x)$) node {$\x$}; 
 \end{tikzpicture}
}
\loigiai{
Xét hình lập phương $ABCD.A'B'C'D'$ với cạnh bằng $x>0$, ta có $AC'=\sqrt{AB^2+AD^2+AA'^2}=x\sqrt{3}$.\\
Vì $\triangle ADC'$ vuông tại $D$ nên
$\cos\left(\overrightarrow{a},\overrightarrow{c}\right)=\cos \widehat{DAC'}=\dfrac{AD}{AC'}=\dfrac{1}{\sqrt{3}}$.\\
Tương tự, $\triangle ABC'$ vuông tại $B$ nên $\cos\left(\overrightarrow{b},\overrightarrow{c}\right)=\cos\widehat{BAC'}=\dfrac{AB}{AC'}=\dfrac{1}{\sqrt{3}}$.
 \begin{itemchoice}
 \itemch \textbf{Sai}.\\
Giả sử $\overrightarrow{a}+\overrightarrow{b}=\overrightarrow{d}$. Theo quy tắc hình bình hành thì $\overrightarrow{d}$ cùng hướng với $\overrightarrow{AC}$ nên $\overrightarrow{d}$ không cùng phương với $\overrightarrow{AC'}$. Suy ra $\overrightarrow{a}+\overrightarrow{b}=\overrightarrow{c}$ là sai.
 \itemch \textbf{Sai}.\\
Ta có $\left( \overrightarrow{a} + \overrightarrow{b}\right)^2 = {\overrightarrow{a}}^2 + {\overrightarrow{b}}^2 + 2\overrightarrow{a}\cdot \overrightarrow{b} = 10^2+10^2+0 =200$, suy ra $\left|\overrightarrow{a}+\overrightarrow{b}\right|=10\sqrt{2}$.
 \itemch \textbf{Đúng}.\\
Ta có \begin{itemize}
\item $\left(\overrightarrow{a}+\overrightarrow{c}\right)^2=\left|\overrightarrow{a}\right|^2+2\overrightarrow{a}\cdot\overrightarrow{c}+\left|\overrightarrow{c}\right|^2=10^2+2\cdot 10\cdot 10\sqrt{3}\cdot\dfrac{1}{\sqrt{3}}+\left(10\sqrt{3}\right)^2=600$.\\
Suy ra $\left|\overrightarrow{a}+\overrightarrow{c}\right|=\sqrt{600}$.
\item $\left(\overrightarrow{b}+\overrightarrow{c}\right)^2=\left|\overrightarrow{b}\right|^2+2\overrightarrow{b}\cdot\overrightarrow{c}+\left|\overrightarrow{c}\right|^2=10^2+2\cdot 10\cdot 10\sqrt{3}\cdot\dfrac{1}{\sqrt{3}}+\left(10\sqrt{3}\right)^2=600$.\\
Suy ra $\left|\overrightarrow{a}+\overrightarrow{c}\right|=\sqrt{600}$.
 \end{itemize}
Vậy $\left|\overrightarrow{a}+\overrightarrow{c}\right|=\left|\overrightarrow{b}+\overrightarrow{c}\right|$.
 \itemch \textbf{Đúng}.\\
Giả sử lực tổng hợp là $\overrightarrow{m}$, tức là $\overrightarrow{m}=\overrightarrow{a}+\overrightarrow{b}+\overrightarrow{c}$. Do đó
\begin{eqnarray*}
&& \big|\overrightarrow{m}\big|^2=\left(\overrightarrow{a}+\overrightarrow{b}+\overrightarrow{c}\right)^2\\
&\Leftrightarrow& \left|\overrightarrow{m}\right|^2={\overrightarrow{a}}^2+{\overrightarrow{b}}^2+{\overrightarrow{c}}^2+2\overrightarrow{a}\cdot\overrightarrow{b}+2\overrightarrow{b}\cdot\overrightarrow{c}+2\overrightarrow{c}\cdot\overrightarrow{a}\\
&\Leftrightarrow& \left|\overrightarrow{m}\right|^2=10^2+10^2+(10\sqrt{3})^2+2\cdot 10\cdot10\sqrt{3}\cdot\dfrac{1}{\sqrt{3}}+2\cdot 10\cdot10\sqrt{3}\cdot\dfrac{1}{\sqrt{3}}\\
&\Leftrightarrow& \left|\overrightarrow{m}\right|^2=900.
\end{eqnarray*}
Suy ra cường độ lực tổng hợp $\overrightarrow{a}+\overrightarrow{b}+\overrightarrow{c}$ bằng $30$ N.
\end{itemchoice}
}
\end{ex}

\begin{ex}%[2-H2B4-SO-11-2425]%[VN-MT-7, Đào Trung Kiên]%[2H2H2-4]
Trong không gian với hệ tọa độ $Oxyz$, cho ba điểm $A(2; 3; 1)$, $B(-1; 2; 0)$, $C(1; 1; -2)$.
 \choiceTF
 {\True $\overrightarrow{OA}=2\overrightarrow{i}+3\overrightarrow{j}+\overrightarrow{k}$}
 {$\overrightarrow{AB}=(3; -1; -1)$}
 {\True Gọi $D$ là đỉnh của hình bình hành $ABCD$, khi đó $D(4; 2; -1)$}
 {\True $G$ là trọng tâm của tam giác $ABC$, khi đó $OG=\dfrac{\sqrt{41}}{3}$}
 \loigiai{
 \begin{itemchoice}
 \itemch \textbf{Đúng}.\\
Vì $A(2; 3; 1)$ nên $\overrightarrow{OA}=2\overrightarrow{i}+3\overrightarrow{j}+\overrightarrow{k}$.
 \itemch \textbf{Sai}.\\
$\overrightarrow{AB}=(-3; -1; -1)$.
 \itemch \textbf{Đúng}.\\
Gọi $D(x; y; z)$. Khi đó $\overrightarrow{AB}=(-3; -1; -1)$ và $\overrightarrow{DC}=(1-x; 1-y; -2-z)$.\\
Vì $ABCD$ là hình bình hành nên $\overrightarrow{AB}=\overrightarrow{DC}\Leftrightarrow \heva{&-3=1-x\\&-1=1-y\\&-1=-2-z}\Leftrightarrow \heva{&x=4\\&y=2\\&z=-1.}$\\
Vậy $D(4; 2; -1)$.
 \itemch \textbf{Đúng}.\\
Gọi $G(x; y; z)$ là trọng tâm của tam giác $ABC$. Khi đó $\heva{&x=\dfrac{2-1+1}{3}=\dfrac{2}{3}\\&y=\dfrac{3+2+1}{3}=2\\&z=\dfrac{1+0-2}{3}=-\dfrac{1}{3}.}$\\
Vậy $G\left(\dfrac{2}{3}; 2; -\dfrac{1}{3}\right)$ nên $OG=\sqrt{\left(\dfrac{2}{3}\right)^2+2^2+\left(-\dfrac{1}{3}\right)^2}=\dfrac{\sqrt{41}}{3}$.

\end{itemchoice}
 }
\end{ex}

\begin{ex}%[2-H2B4-SO-11-2425]%[VN-MT-7, Đào Trung Kiên]%[2H2H2-4]
Trong không gian với hệ tọa độ $Oxyz$.
 \choiceTF
 {\True Cho hai vectơ $\overrightarrow{u}=m\overrightarrow{i}+2\overrightarrow{j}-3\overrightarrow{k}$, $\overrightarrow{v}=m\overrightarrow{j}+2\overrightarrow{i}+4\overrightarrow{k}$. Biết rằng $\overrightarrow{u}\cdot \overrightarrow{v}=8$, khi đó $m=5$}
 {Góc giữa hai vectơ $\overrightarrow{u}=(1; -2;1)$ và $\overrightarrow{v}=(-2; 1; 1)$ bằng $60^\circ$}
 {\True Cho lăng trụ đứng $ABC.A'B'C'$ có $A(0; 0; 0)$, $B(2; 0; 0)$, $C(0; 2; 0)$ và $A'(0; 0; 2)$. Góc giữa $BC'$ và $A'C$ bằng $90^\circ$}
 {Gọi $\varphi$ là góc giữa hai vectơ $\overrightarrow{a}$ và $\overrightarrow{b}$ (với $\overrightarrow{a}$ và $\overrightarrow{b}$ khác $\overrightarrow{0}$), khi đó $\cos\varphi=\dfrac{|\overrightarrow{a}|\cdot|\overrightarrow{b}|}{\overrightarrow{a}\cdot\overrightarrow{b}}$}
\loigiai{
 \begin{itemchoice}
 \itemch \textbf{Đúng}.\\
Từ giả thiết ta có $\overrightarrow{u}=(m; 2; -3)$, $\overrightarrow{v}=(2; m; 4)$.\\
Do đó $\overrightarrow{u}\cdot\overrightarrow{v}=8\Leftrightarrow 2m+2m-3\cdot 4=8\Leftrightarrow m=5$.
 \itemch \textbf{Sai}.\\
Ta có $\cos(\overrightarrow{u},\overrightarrow{v})=\dfrac{\overrightarrow{u}\cdot\overrightarrow{v}}{|\overrightarrow{u}|\cdot|\overrightarrow{v}|}=\dfrac{-3}{\sqrt{6}\cdot\sqrt{6}}=-\dfrac{1}{2}\Rightarrow \left(\overrightarrow{u},\overrightarrow{v}\right)=120^\circ$.
 \itemch \textbf{Đúng}.\\
Gọi $C'(x; y; z)$, vì $ABC.A'B'C'$ là hình lăng trụ đứng nên $\overrightarrow{AA'}=\overrightarrow{CC'}\Leftrightarrow \heva{&x-0=0\\&y-2=0\\&z-0=2}$.\\
Từ đó ta có $B(2; 0; 0)$, $C'(0; 2; 2)$ nên $\overrightarrow{BC'}=(-2; 2; 2)$.\\
Vì $A'(0; 0; 2)$ và $C(0; 2; 0)$ nên $\overrightarrow{A'C}=(0; 2; -2)$.\\
Từ đó suy ra $\overrightarrow{BC'}\cdot\overrightarrow{A'C}=0$ nên góc giữa $BC'$ và $A'C$ bằng $90^\circ$.
\itemch \textbf{Sai}.\\
Công thức tính côsin của góc giữa hai vectơ $\overrightarrow{a}$ và $\overrightarrow{b}$ (với $\overrightarrow{a}$ và $\overrightarrow{b}$ khác $\overrightarrow{0}$) là  $\cos\left(\overrightarrow{a},\overrightarrow{b}\right)=\dfrac{\overrightarrow{a}\cdot\overrightarrow{b}}{|\overrightarrow{a}|\cdot|\overrightarrow{b}|}$.
\end{itemchoice}
 }
\end{ex}

\begin{ex}%[2-H2B4-SO-11-2425]%[VN-MT-7, Đào Trung Kiên]%[2H2V2-6]
Hình minh họa sơ đồ ngôi nhà Trong không gian với hệ tọa độ $Oxyz$, trong đó nền nhà, bốn bức tường và hai mái nhà đều là hình chữ nhật.
\begin{center}
\begin{tikzpicture}[font=\footnotesize, line join=round, line cap=round, >=stealth, scale=1.2]
 \def\a{3}
 \def\b{5}
 \def\h{3}
 \path (0:0) coordinate (C)
 ++(0:\a) coordinate (B)
 ++(-160:\b) coordinate (O)
 ($(O)+(B)-(C)$) coordinate (A)
 ($(O)+(90:\h)$) coordinate (E)
 ($(B)+(90:\h)$) coordinate (G)
 ($(C)+(90:\h)$) coordinate (H)
 ($(A)+(90:\h)$) coordinate (F)
 ($(A)+(0:1)$) coordinate (x)
 ($(H)+(35:2)$) coordinate (Q)
 ($(E)+(35:2)$) coordinate (P)
 ($(E)+(90:1)$) coordinate (z)
 ($(O)!1.3!(C)$) coordinate (y);
 \draw[dashed] (G)--(H)--(C)--(B) (C)--(O);
 \draw (G)--(Q)--(H)--(E)--(F)--(G)--(B)--(A)--(O)--(E) (F)--(A) (F)--(P)--(E) (P)--(Q);
 \draw [->] (A)--(x);
 \draw [->] (E)--(z);
 \draw [->,dashed] (C)--(y);
 \draw (Q)node[above]{$Q(2; 5; 4)$} (G)node[right]{$G(4; 5; 3)$} (B)node[right]{$B(4; 5; 0)$} (P)node[right]{$P(2; 0; 4)$} (O)node[below]{$O(0; 0; 0)$} (E)node[left]{$E(0; 0; 3)$} (x)node[below]{$x$} (y)node[above]{$y$} (z)node[left]{$z$};
 \foreach \x/\g in {A/-90,C/180,F/0,H/90}
 \fill (\x) circle (1pt)
 ($(\g:4mm)+(\x)$) node {$\x$}; 
 \fill (E) circle (1pt) (Q) circle (1pt) (O) circle (1pt) (G) circle (1pt) (P) circle (1pt) (B) circle (1pt);
\end{tikzpicture}
\end{center}
 \choiceTF
 {\True Tọa độ điểm $F(4; 0; 3)$}
 {Tọa độ vectơ $\overrightarrow{AH}=(4; 5; 3)$}
 {$\overrightarrow{AH}\cdot\overrightarrow{AF}=3$}
 {\True Góc đốc của mái nhà, tức là số đo của góc nhị diện có cạnh là đường thẳng $FG$, hai mặt lần lượt là $(FGQP)$ và $(FGHE)$ bằng $26{,}6^\circ$ (làm tròn đến hàng phần mười của đơn vị độ)}
 \loigiai{
 \begin{itemchoice}
 \itemch \textbf{Đúng}.\\
Vì nền nhà là hình chữ nhật nên $OACB$ là hình chữ nhật, suy ra $x_A=x_B=4, y_C=y_B=5$.\\
Do điểm $A$ nằm trên trục $O x$ nên tọa độ điểm $A(4; 0; 0)$; điểm $C$ nằm trên trục $Oy$ nên tọa độ điểm $C(0; 5; 0)$.\\
Tường nhà là hình chữ nhật nên $OCHE$ là hình chữ nhật, suy ra $y_H=y_C=5$, $z_H=z_E=3$.\\
Do $H$ nằm trên mặt phẳng $(Oyz)$ nên tọa độ điểm $H(0; 5; 3)$.\\
Tứ giác $OAFE$ là hình chữ nhật nên $x_F=x_A=4, z_F=z_E=3$.\\
Do $F$ nằm trên mặt phẳng $(Oxz)$ nên tọa độ điểm $F(4; 0; 3)$.
 \itemch \textbf{Sai}.\\
Ta có toạ độ vectơ $\overrightarrow{AH}=(-4; 5; 3)$.
 \itemch \textbf{Sai}.\\
Ta có $\overrightarrow{AF}=(0; 0; 3)$. Suy ra $\overrightarrow{AH}\cdot\overrightarrow{AF}=0+0+9=9$.
 \itemch \textbf{Đúng}.\\
Để tính góc đốc của mái nhà, ta tính số đo của góc nhị diện có cạnh là đường thẳng $FG$, hai mặt lần lượt là $(FGQP)$ và $(FGHE)$.\\
Do mặt phẳng $(O z x)$ vuông góc với hai mặt phẳng $(FGQP)$ và $(F G H E)$ nên $\widehat{PFE}$ là góc phẳng nhị diện cần tìm.\\
Ta có $\overrightarrow{FP}=(-2; 0; 1), \overrightarrow{FE}=(-4; 0; 0)$ suy ra 
\[\cos\widehat{PFE}=\cos \left(\overrightarrow{FP}, \overrightarrow{FE}\right)=\dfrac{\overrightarrow{FP} \cdot \overrightarrow{FE}}{\left|\overrightarrow{FP}\right|\cdot\left|\overrightarrow{FE}\right|}= \dfrac{(-2)(-4)+0\cdot 0+1\cdot 0}{\sqrt{(-2)^2+0^2+1^2} \cdot \sqrt{(-4)^2+0^2+0^2}}=\dfrac{2 \sqrt{5}}{5}.\]
Do đó, $\widehat{PFE} \approx 26{,}6^{\circ}$.\\
Vậy góc đốc mái nhà khoảng $26{,}6^{\circ}$.
\end{itemchoice}
 }
\end{ex}
\Closesolutionfile{ans}

\TNSA
\Opensolutionfile{ans}[ans/ans\currfilebase-Phan-III]
\begin{ex}%[2-H2B4-SO-11-2425]%[VN-MT-7, Đào Trung Kiên]%[2H2H1-3]
Cho hai vectơ $\overrightarrow{a}$, $\overrightarrow{b}$ thỏa mãn $\left|\overrightarrow{a}\right|=3$, $\left|\overrightarrow{b}\right|=4$, $\left|\overrightarrow{a}+\overrightarrow{b}\right|=6$. Tính $\left|\overrightarrow{a}-\overrightarrow{b}\right|$ (làm tròn kết quả đến hàng phần trăm).
\shortans{3{,}74}
\loigiai{
Ta có $\left|\overrightarrow{a}+\overrightarrow{b}\right|^2=\left(\overrightarrow{a}+\overrightarrow{b}\right)^2=\left|\overrightarrow{a}\right|^2+2\overrightarrow{a}\overrightarrow{b}+\left|\overrightarrow{b}\right|^2\Rightarrow 2\overrightarrow{a}\overrightarrow{b}=\left|\overrightarrow{a}+\overrightarrow{b}\right|^2-\left|\overrightarrow{a}\right|^2-\left|\overrightarrow{b}\right|^2=11$.\\
$\left|\overrightarrow{a}-\overrightarrow{b}\right|^2=\left(\overrightarrow{a}-\overrightarrow{b}\right)^2=\left|\overrightarrow{a}\right|^2-2\overrightarrow{a}\overrightarrow{b}+\left|\overrightarrow{b}\right|^2=9-11+16=14\Rightarrow \left|\overrightarrow{a}-\overrightarrow{b}\right|=\sqrt{14}\approx 3{,}74$.
}
\end{ex}

\begin{ex}%[2-H2B4-SO-11-2425]%[VN-MT-7, Đào Trung Kiên]%[2H2V1-4]
Một chiếc đèn trang trí hình tròn được treo song song với mặt phẳng trần nhà nằm ngang bởi ba sợi dây không giãn $OA$, $OB$, $OC$ đôi một vuông góc (như hình vẽ dưới đây). Biết lực căng của sợi dây tương ứng trên mỗi dây $OA$, $OB$, $OC$ lần lượt là $\overrightarrow{F_1}$, $\overrightarrow{F_2}$, $\overrightarrow{F_3}$ thỏa mãn $\left|\overrightarrow{F_1}\right|=\left|\overrightarrow{F_2}\right|=\left|\overrightarrow{F_3}\right|=16$ (N). Tính trọng lượng (đơn vị: N) của chiếc đèn đó (làm tròn kết quả đến hàng phần mười).
\begin{center}
\begin{tikzpicture}[line join=round, line cap=round,>=stealth,xscale=1,yscale=0.4]
 \path (0:0) coordinate (O')
 ++(0:2) coordinate (C)
 ($(O')+(150:2)$) coordinate (B)
 ($(O')+(220:2)$) coordinate (A)
 ($(O')+(90:8)$) coordinate (O)
 ($(O)!0.5!(A)$) coordinate (A1)
 ($(O)!0.5!(B)$) coordinate (B1)
 ($(O)!0.5!(C)$) coordinate (C1);
 \draw[fill=blue,opacity=0.15] (O') circle (2 cm);
 \draw (O') circle (2 cm);
 \draw (O)--(A) (O)--(B) (O)--(C) (2,0)--(2,-1) (-2,0)--(-2,-1);
 \draw (2,-1) arc (0:-180:2);
 \draw[color=red,dashed] (C)--(O');
 \draw[dashed] (O')--(0,-3);
 \draw[fill=blue,opacity=0.4] (-2,8) rectangle (2,8.6);
 \draw [->] (0,-3)--(0,-7)node[midway, right]{$\overrightarrow{P}$};
 \draw [->] (O)--(A1)node[above right]{$\overrightarrow{F_1}$};
 \draw [->] (O)--(B1)node[above left]{$\overrightarrow{F_2}$};
 \draw [->] (O)--(C1)node[above right]{$\overrightarrow{F_3}$};
 \draw (0,7.5) node[below] {$O$};
 \foreach \x/\g in {A/60,B/100,C/0}
 \fill (\x) circle (1pt)
 ($(\g:5mm)+(\x)$) node {$\x$};
\end{tikzpicture}
\end{center}
 \shortans{27{,}7}
 \loigiai{
\begin{center}
\begin{tikzpicture}[line join=round, line cap=round,>=stealth,scale=1]
 \def\a{3.5}
 \path (0:0) coordinate (O)
 ++(0:\a) coordinate (B)
 ++(-160:\a*1.3) coordinate (A)
 ($(B)+(A)-(O)$) coordinate (E)
 ($(O)+(90:\a)$) coordinate (C)
 ($(E)+(90:\a)$) coordinate (F)
 ($(A)+(90:\a)$) coordinate (A')
 ($(B)+(90:\a)$) coordinate (B');
 \draw[dashed] (O)--(A) (O)--(B) (O)--(C) (O)--(F);
 \draw[thick] (C)--(F) (C)--(A') (C)--(B') (F)--(E) (A)--(E)--(B)--(B')--(F)--(A')--cycle;
 \foreach \x/\g in {O/170,A/-90,E/-90,B/-60,C/100,F/0}
 \fill[black] (\x) circle (1pt)
 ($(\g:4mm)+(\x)$) node {$\x$}; 
\end{tikzpicture}
\end{center}
Gọi $P$ là trọng lượng của đèn, ta có $P=\left|\overrightarrow{F_1}+\overrightarrow{F_2}+\overrightarrow{F_3}\right|=\left|\overrightarrow{OA}+\overrightarrow{OB}+\overrightarrow{OC}\right|$.\\
Vẽ hình vuông $OAEB$, ta có $\overrightarrow{OA}+\overrightarrow{OB}=\overrightarrow{OE}$ (quy tắc hình bình hành).\\
Vẽ hình chữ nhật $OCFE$, ta có $\overrightarrow{OC}+\overrightarrow{OE}=\overrightarrow{OF}$ (quy tắc hình bình hành).\\
Suy ra $P=\left|\overrightarrow{OF}\right|=OF$.\\
Xét hình vuông $OAEB$, cạnh bằng $16$ và có đường chéo $OE=16\sqrt{2}$.\\
Xét tam giác vuông $OEF$, vuông tại $E$, có $OF=\sqrt{OE^2+EF^2}=\sqrt{\left(16\sqrt{2}\right)^2+16^2}=16\sqrt{3}\approx 27{,}7$.\\
Vậy $P\approx 27{,}7$ N.
 }
\end{ex}

\begin{ex}%[2-H2B4-SO-11-2425]%[VN-MT-7, Đào Trung Kiên]%[2H2H2-4]
Trong không gian với hệ tọa độ $Oxyz$, cho hai điểm $B(2; 1; 0)$, $C(1; 4; 5)$. Điểm $M(x; y; z)$ thuộc trục hoành sao cho $MB=MC$. Khi đó giá trị $2x+y+z$ bằng bao nhiêu?
 \shortans{-37}
 \loigiai{
Do điểm $M \in Ox$ nên $M(x; 0; 0)$, ta có 
\begin{eqnarray*}
MB=MC&\Leftrightarrow& MB^2=MC^2\Leftrightarrow (2-x)^2+1^2+0^2=(1-x)^2+4^2+5^2\\
&\Leftrightarrow&x^2-4x+5=x^2-2x+42\Leftrightarrow x=-\dfrac{37}{2}. 
\end{eqnarray*}
Vậy $M\left(-\dfrac{37}{2};0;0\right)\Rightarrow 2x+y+z=-37$.
 }
\end{ex}

\begin{ex}%[2-H2B4-SO-11-2425]%[VN-MT-7, Đào Trung Kiên]%[2H2H1-4]
Trong không gian tọa độ $Oxyz$ cho $\overrightarrow{a}$ và $\overrightarrow{b}$ tạo với nhau một góc $120^{\circ}$. Biết rằng $|\overrightarrow{a}|=4$; $|\overrightarrow{b}|=3$, tính giá trị của biểu thức $A=|\overrightarrow{a}-\overrightarrow{b}|+|\overrightarrow{a}+\overrightarrow{b}|$ ( làm tròn kết quả đến hàng phần trăm).

\shortans{9{,}69}
 \loigiai{
Ta có $|\overrightarrow{a}-\overrightarrow{b}|^2=\left(\overrightarrow{a}-\overrightarrow{b}\right)^2=|\overrightarrow{a}|^2-2 \overrightarrow{a} \cdot \overrightarrow{b}+|\overrightarrow{b}|^2=16-2|\overrightarrow{a}| \cdot|\overrightarrow{b}|\cdot\cos 120^{\circ}+9=37$.\\
Tương tự $|\overrightarrow{a}+\overrightarrow{b}|^2=\left(\overrightarrow{a}+\overrightarrow{b}\right)^2=|\overrightarrow{a}|^2+2 \overrightarrow{a} \cdot \overrightarrow{b}+|\overrightarrow{b}|^2=16+2|\overrightarrow{a}| \cdot|\overrightarrow{b}|\cdot\cos 120^{\circ}+9=13$.\\
Do đó $A=|\overrightarrow{a}-\overrightarrow{b}|+|\overrightarrow{a}+\overrightarrow{b}|=\sqrt{37}+\sqrt{13} \approx 9{,}69$.
 }
\end{ex}

\begin{ex}%[2-H2B4-SO-11-2425]%[VN-MT-7, Đào Trung Kiên]%[2H2V2-6]
Người ta cần lắp một camera phía trên sân bóng để phát sóng truyền hình một trận bóng đá, camera có thể di động để luôn thu được hình ảnh rõ nét về diễn biến trên sân. Các kĩ sư dự định trồng bốn chiếc cột cao 30 m và sử dụng hệ thống cáp gắn vào bốn đầu cột để giữ camera ở vị trí mong muốn.\\
Mô hình thiết kế được xây dựng như sau\\
Trong hệ trục toạ độ $Oxyz$ (đơn vị độ dài trên mỗi trục là $1$ m), các đỉnh của bốn chiếc cột lần lượt là các điểm $M(90; 0; 30)$, $N(90; 120; 30)$, $P(0; 120; 30)$, $Q(0; 0; 30)$.\\
Giả sử $K_0$ là vị trí ban đầu của camera có cao độ bằng $25$ và $K_0M=K_0N=K_0P=K_0Q$. Để theo dõi quả bóng đến vị trí $A$, camera được hạ thấp theo phương thẳng đứng xuống điểm $K_1$ cao độ bằng $19$.\\
Tọa độ của vectơ $\overrightarrow{K_0K_1}=(a; b; c)$ với $a$, $b$, $c$ là các số thực. Tính $P=a+b-c$.
\begin{center}
 \begin{tikzpicture}
 \def\a{5.5}
 \def\b{2}
 \def\h{3.5}
 \path (0:0) coordinate (O)
 ++(0:\a) coordinate (P')
 ($(O)+(220:\b)$) coordinate (M')
 ($(O)+(0:\a+1)$) coordinate (y)
 ($(P')+(M')-(O)$) coordinate (F)
 ($(O)+(90:\h)$) coordinate (Q)
 ($(M')+(90:\h)$) coordinate (M)
 ($(F)+(90:\h)$) coordinate (N)
 ($(P')+(90:\h)$) coordinate (P)
 ($(O)!1.3!(M')$) coordinate (x)
 ($(O)!1.3!(Q)$) coordinate (z)
 ($(O)!0.5!(Q)$) coordinate (Q_1)
 ($(O)!0.7!(Q)$) coordinate (Q_0)
 ($(F)!0.5!(N)$) coordinate (N_1)
 ($(F)!0.7!(N)$) coordinate (N_0)
 ($(O)!0.1!(F)$) coordinate (A)
 ($(O)!0.9!(F)$) coordinate (C)
 ($(M')!0.1!(P')$) coordinate (B)
 ($(M')!0.9!(P')$) coordinate (D);
 \coordinate (H) at (intersection of Q--N and M--P);
 \coordinate (H') at (intersection of F--O and M'--P');
 \coordinate (K_0) at (intersection of Q_0--N_0 and H--H');
 \coordinate (K_1) at (intersection of Q_1--N_1 and H--H');
 \draw [->] (O)--(y) node[below]{$y$};
 \draw [->] (O)--(x) node[below]{$x$};
 \draw [->] (O)--(z) node[left]{$z$};
 \draw [fill=green] (A)--(B)--(C)--(D)--(A);
 \draw[dashed] (M')--(P') (F)--(O) (K_1)--(Q) (K_1)--(M) (K_1)--(P) (K_1)--(N);
 \draw (Q)--(K_0)--(N) (M')--(M)--(K_0)--(P) (N)--(F) (P)--(P');
 \foreach \x/\g in {O/45,F/-90,P/40,Q/60,M/160,K_0/100,K_1/-90,N/0}
 \fill (\x) circle (1pt)
 ($(\g:4mm)+(\x)$) node {$\x$}; 
 \fill[black](4.3,-0.4) node [below right]{$A$} circle (1.5pt);
 \end{tikzpicture}
\end{center} 
\shortans{6}
 \loigiai{
\begin{center}
 \begin{tikzpicture}
 \def\a{5.5}
 \def\b{2}
 \def\h{3.5}
 \path (0:0) coordinate (O)
 ++(0:\a) coordinate (P')
 ($(O)+(220:\b)$) coordinate (M')
 ($(O)+(0:\a+1)$) coordinate (y)
 ($(P')+(M')-(O)$) coordinate (F)
 ($(O)+(90:\h)$) coordinate (Q)
 ($(M')+(90:\h)$) coordinate (M)
 ($(F)+(90:\h)$) coordinate (N)
 ($(P')+(90:\h)$) coordinate (P)
 ($(O)!1.3!(M')$) coordinate (x)
 ($(O)!1.3!(Q)$) coordinate (z)
 ($(O)!0.5!(Q)$) coordinate (Q_1)
 ($(O)!0.7!(Q)$) coordinate (Q_0)
 ($(F)!0.5!(N)$) coordinate (N_1)
 ($(F)!0.7!(N)$) coordinate (N_0)
 ($(O)!0.1!(F)$) coordinate (A)
 ($(O)!0.9!(F)$) coordinate (C)
 ($(M')!0.1!(P')$) coordinate (B)
 ($(M')!0.9!(P')$) coordinate (D);
 \coordinate (H) at (intersection of Q--N and M--P);
 \coordinate (H') at (intersection of F--O and M'--P');
 \coordinate (K_0) at (intersection of Q_0--N_0 and H--H');
 \coordinate (K_1) at (intersection of Q_1--N_1 and H--H');
 \draw [->] (O)--(y) node[below]{$y$};
 \draw [->] (O)--(x) node[below]{$x$};
 \draw [->] (O)--(z) node[left]{$z$};
 \draw [fill=green] (A)--(B)--(C)--(D)--(A);
 \draw[dashed] (M')--(P') (F)--(O) (K_1)--(Q) (K_1)--(M) (K_1)--(P) (K_1)--(N);
 \draw (Q)--(K_0)--(N) (M')--(M)--(K_0)--(P) (N)--(F) (P)--(P');
 \foreach \x/\g in {O/45,F/-90,P/40,Q/60,M/160,K_0/100,K_1/-90,N/0}
 \fill (\x) circle (1pt)
 ($(\g:4mm)+(\x)$) node {$\x$}; 
 \fill[black](4.3,-0.4) node [below right]{$A$} circle (1.5pt);
\end{tikzpicture}

\end{center} 
Gọi $K_0(x; y; 25)$ và $K_1(x; y; 19)$ suy ra $\overrightarrow{K_0K_1}=(0; 0; -6)$.\\
Vậy $a =0$, $b=0$, $c=-6$ nên $P=a+b-c=6$.}
\end{ex}

\begin{ex}%[2-H2B4-SO-11-2425]%[VN-MT-7, Đào Trung Kiên]%[2H2H1-3]
\immini{Cho tứ diện $OABC$ có các cạnh $OA$, $OB$, $OC$ đôi một vuông góc và $OA=OB=OC=1$. Gọi $M$ là trung điểm của cạnh $AB$. Côsin của góc giữa hai vectơ $\overrightarrow{OM}$ và $\overrightarrow{AC}$ bằng $-\dfrac{a}{b}$ với $\dfrac{a}{b}$ là phân số tối giản. Tính $Q = a\cdot b$.}{
\begin{tikzpicture}[line join=round, line cap=round,>=stealth,scale=0.8]
 \def\a{4} %Khai báo cạnh
 \def\h{3}
 \path (0:0) coordinate (O)
 ++(0:\a) coordinate (B)
 ($(O)+(-50:2.4)$) coordinate (A)
 ($(O)+(90:\h)$) coordinate (C)
 ($(A)!0.5!(B)$) coordinate (M);
 \draw (C)--(O)--(A)--(C)--(B)--(A);
 \draw[dashed,->] (O)--(M);
 \draw[dashed] (O)--(B) ;
 \draw[->] (A)--(C);
 \foreach \x /\goc in {A/-90,B/0,C/170,M/-40,O/180}
 \fill (\x) circle (1pt)
 ($(\x)+(\goc:3mm)$) node {$\x$};
% \draw pic[draw,angle radius=2mm]{right angle=B--A--S};%Theo chiều dương
\end{tikzpicture}
}
 \shortans{2}
 \loigiai{
Đặt $\overrightarrow{OA}=\overrightarrow{a}$, $\overrightarrow{OB}=\overrightarrow{b}$, $\overrightarrow{OC}=\overrightarrow{c}$.\\
Khi đó, $\left|\overrightarrow{a}\right|=\left|\overrightarrow{b}\right|=\left|\overrightarrow{c}\right|=1$ và $\overrightarrow{a}\cdot\overrightarrow{b}=\overrightarrow{a}\cdot\overrightarrow{c}=\overrightarrow{b}\cdot\overrightarrow{c}=0$.\\
Ta có $\cos\left(\overrightarrow{OM},\overrightarrow{AC}\right)=\dfrac{\overrightarrow{OM}\cdot\overrightarrow{AC}}{\left|\overrightarrow{OM}\right|\cdot\left|\overrightarrow{AC}\right|}$.\\
Mặt khác, do $\overrightarrow{OM}=\dfrac{1}{2}\left(\overrightarrow{OA}+\overrightarrow{OB}\right)=\dfrac{1}{2}\left(\overrightarrow{a}+\overrightarrow{b}\right)$ và $\overrightarrow{AC}=\overrightarrow{OC}-\overrightarrow{OA}=\overrightarrow{c}-\overrightarrow{a}$ nên 
\[\overrightarrow{OM}\cdot\overrightarrow{AC}=\dfrac{1}{2}\left(\overrightarrow{a}+\overrightarrow{b}\right)\cdot\left(\overrightarrow{c}-\overrightarrow{a}\right)=\dfrac{1}{2}\left(\overrightarrow{a}\cdot\overrightarrow{c}-{\overrightarrow{a}}^2+\overrightarrow{b}\cdot\overrightarrow{c}-\overrightarrow{b}\cdot\overrightarrow{a}\right)=-\dfrac{1}{2}.\]
Ta có $AC=\sqrt{OA^2+OC^2}=\sqrt{2}$, $OM=\dfrac{1}{2}AB=\dfrac{1}{2}\sqrt{OA^2+OC^2}=\dfrac{\sqrt{2}}{2}$.\\
Từ đó $\cos\left(\overrightarrow{OM},\overrightarrow{AC}\right)=-\dfrac{1}{2}$ nên $a=1$ và $b=2$.\\
Vậy $Q=a\cdot b=2$.
 }
\end{ex}
\Closesolutionfile{ans}
 
% \begin{indapan}
% 	{ans/ans\currfilebase}
% \end{indapan}

