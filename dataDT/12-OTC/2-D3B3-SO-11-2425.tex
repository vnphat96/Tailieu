\begin{name}
 {Biên soạn: Nguyen Huynh \\ Phản biện: Nguyễn Kiều Nhã Tú}
 {Đề ôn tập chương III}
\end{name}

\caulc
\Opensolutionfile{ans}[ans/ans\currfilebase-Phan-I]

\begin{ex}%[2-D3B3-SO-11-2425]%[VN-MT-7, Nguyen Huynh]%[1D5H2-3]
 Khảo sát thời gian xem ti vi trong một ngày của một số học sinh khối $11$ thu được mẫu số liệu ghép nhóm sau:
 \begin{center}
 \begin{tabular}{|l|c|c|c|c|c|}
 \hline Thời gian (phút)&$[0 ; 20)$&$[20 ; 40)$&$[40 ; 60)$&$[60 ; 80)$&$[80 ; 100)$\\
 \hline Số học sinh & $5$ & $9$ & $12$ & $10$ & $6$ \\
 \hline
 \end{tabular}
 \end{center}
 Nhóm chứa tứ phân vị thứ nhất là
 \choice
 {$[0 ; 20)$}
 {\True $[20 ; 40)$}
 {$[40 ; 60)$}
 {$[60 ; 80)$}
 \loigiai{
 Cỡ mẫu $n=5+9+12+10+6=42$.\\ 
 Gọi $x_1$, $x_2$, $\ldots$, $x_{42}$ là mẫu số liệu về thời gian xem ti vi trong một ngày của một số học sinh khối $11$ được xếp theo thứ tự không giảm.\\
 Ta có
 $x_1$, $\ldots$, $x_5 \in[0 ; 20)$; $x_6$, $\ldots$, $x_{14} \in[20 ; 40)$; $x_{15}$, $\ldots$, $x_{26} \in[40 ; 60)$; $x_{27}$, $\ldots$, $x_{36} \in[60 ; 80)$; $x_{37}; \ldots$; $x_{42} \in[40 ; 45)$.\\
 Tứ phân vị thứ nhất của mẫu số liệu $x_1$, $x_2$, $\ldots$, $x_{42}$ là $\dfrac{1}{2}(x_{10}+x_{11})$.\\ 
 Do $x_{10}\in [20 ; 40)$ và $x_{11}\in [20 ; 40)$ nên nhóm chứa tứ phân vị thứ nhất là $[20 ; 40)$.}
\end{ex}

\begin{ex}%[2-D3B3-SO-11-2425]%[VN-MT-7, Nguyen Huynh]%[2D3N1-2]
 Cô Hà thống kê lại đường kính thân gỗ của một số cây xoan đào $6$ năm tuổi được trồng ở một lâm trường ở bảng sau:
 \begin{center}
 \begin{tabular}{|c|c|c|c|c|c|}
 \hline Đường kính $(\mathrm{cm})$ &$[40 ; 45)$&$[45 ; 50)$&$[50 ; 55)$&$[55 ; 60)$&$[60 ; 65)$\\
 \hline Tần số & $5$ & $20$ & $18$ & $7$ & $3$ \\
 \hline
 \end{tabular}
 \end{center}
 Hãy tìm khoảng biến thiên của mẫu số liệu ghép nhóm trên.
 \choice
 {\True $25$}
 {$30$}
 {$6$}
 {$69{,}8$}
 \loigiai{Khoảng biến thiên của mẫu số liệu ghép nhóm trên là $65-40=25$.}
\end{ex}

\begin{ex}%[2-D3B3-SO-11-2425]%[VN-MT-7, Nguyen Huynh]%[2D3H1-3]
 Bạn Chi rất thích nhảy hiện đại. Thời gian tập nhảy mỗi ngày trong thời gian gần đây của bạn Chi được thống kê lại ở bảng sau:
 \begin{center}
 \begin{tabular}{|c|c|c|c|c|c|}
 \hline Thời gian(phút)&{$[20 ; 25)$}&{$[25 ; 30)$}&{$[30 ; 35)$}&{$[35 ; 40)$}&{$[40 ; 45)$}\\
 \hline Số ngày & $6$ & $6$ & $4$ & $1$ & $1$ \\
 \hline
 \end{tabular}
 \end{center}
 Khoảng tứ phân vị của mẫu số liệu ghép nhóm là
 \choice
 {$23{,}75$}
 {$27{,}5$}
 {$31{,}88$}
 {\True $8{,}125$}
 \loigiai{Cỡ mẫu $n=18$. \\
 Gọi $x_1$, $x_2$, $\ldots$, $x_{18}$ là mẫu số liệu về thời gian tập nhảy mỗi ngày của bạn Chi được xếp theo thứ tự không giảm.\\
 Ta có
 $x_1$, $\ldots$, $x_6 \in[20 ; 25)$; $x_7$, $\ldots$, $x_{12} \in[25 ; 30)$; $x_{13}$, $\ldots$, $x_{16} \in[30 ; 35)$; $x_{17}$, $\in[35 ; 40)$; $x_{18} \in[40 ; 45)$.\\
 Tứ phân vị thứ nhất của mẫu số liệu gốc là $x_5 \in[20 ; 25)$. Do đó, tứ phân vị thứ nhất của mẫu số liệu ghép nhóm là \[Q_1=20+\dfrac{\dfrac{18}{4}}{6}(25-20)=23{,}75.\]
 Tứ phân vị thứ ba của mẫu số liệu góc là $x_{14} \in[30 ; 35)$. Do đó, tứ phân vị thứ ba của mẫu số liệu ghép nhóm là \[Q_3=30+\dfrac{\dfrac{3\cdot18}{4}-(6+6)}{4}(35-30)=31{,}875.\]
 Khoảng tứ phân vị của mẫu số liệu ghép nhóm là $\Delta_Q=Q_3-Q_1=8{,}125$.}
\end{ex}

\begin{ex}%[2-D3B3-SO-11-2425]%[VN-MT-7, Nguyen Huynh]%[1D5H2-3]
 Trong dịp nghỉ hè bạn Lan rất thích đi bơi. Thời gian đi bơi mỗi ngày trong thời gian gần đây của bạn Lan được thống kê lại ở bảng sau:
 \begin{center}
 \begin{tabular}{|c|c|c|c|c|c|}
 \hline Thời gian (phút) & {$[30 ; 35)$} & {$[35 ; 40)$} & {$[45 ; 50)$} & {$[50 ; 55)$} & {$[55 ; 60)$} \\
 \hline Số ngày & $3$ & $6$ & $4$ & $8$ & $4$ \\
 \hline
 \end{tabular}
 \end{center}
 Nhóm chứa tứ phân vị thứ nhất $Q_1$ là
 \choice
 {$[30 ; 35)$}
 {\True $[35 ; 40)$}
 {$[45 ; 50)$}
 {$[50 ; 55)$}
 \loigiai{Cỡ mẫu là $n=25$.\\
 Gọi $x_1$; $x_2$; $\ldots$; $x_{25}$ là mẫu số liệu về thời gian đi bơi mỗi ngày trong thời gian gần đây của bạn Lan được xếp theo thứ tự không giảm.\\
 Ta có
 $x_1$; $\ldots$; $x_3 \in[30 ; 35)$; $x_4$; $\ldots$; $x_{9} \in[35 ; 40)$; $x_{10}$; $\ldots$; $x_{13} \in[45 ; 50)$; $x_{14}$; $\ldots$; $x_{21}\in[50 ; 55)$; $x_{22}$; $\ldots$; $x_{25} \in[55 ; 60)$.\\ 
 Tứ phân vị thứ nhất của mẫu số liệu gốc là $\frac{x_6+x_7}{2}$. Do $x_6$, $x_7$ đều thuộc nhóm $[35 ; 40)$ nên nhóm này chứa $Q_1$.}
\end{ex}

\begin{ex}%[2-D3B3-SO-11-2425]%[VN-MT-7, Nguyen Huynh]%[1D5H2-3]
 Khảo sát thời gian tập nghe nhạc trong ngày của học sinh lớp 12B thu được mẫu số liệu ghép nhóm sau:
 \begin{center}
 \begin{tabular}{|c|c|c|c|c|c|}
 \hline Thời gian(phút)&{$[0 ; 20)$}&{$[20 ; 40)$}&{$[40 ; 60)$}&{$[60 ; 80)$}&{$[80 ; 100)$}\\
 \hline Số học sinh & $5$ & $10$ & $12$ & $9$ & $4$ \\
 \hline
 \end{tabular}
 \end{center}
 Nhóm chứa tứ phân vị thứ ba $Q_3$ là
 \choice
 {$[20 ; 40)$}
 {$[40 ; 60)$}
 {\True $[60 ; 80)$}
 {$[80 ; 100)$}
 \loigiai{
 Cỡ mẫu là $n=40$.\\
 Gọi $x_1$; $x_2$; $\ldots$; $x_{40}$ là mẫu số liệu về tập nghe nhạc trong ngày của học sinh lớp 12B được xếp theo thứ tự không giảm.\\
 Ta có
 $x_1$; $\ldots$; $x_5 \in[0 ; 20)$; $x_6$; $\ldots$; $x_{15} \in[20 ; 40)$; $x_{16}$; $\ldots$; $x_{27} \in[40 ; 60)$; $x_{28}$; $\ldots$; $x_{36}\in[60 ; 80)$; $x_{37}$; $\ldots$; $x_{40} \in[80 ; 100)$.\\ 
 Tứ phân vị thứ ba của mẫu số liệu gốc là $\dfrac{x_{30}+x_{31}}{2}$. Do $x_{30}$, $x_{31}$ đều thuộc nhóm $[60 ; 80)$ nên nhóm này chứa $Q_3$.}
\end{ex}

\begin{ex}%[2-D3B3-SO-11-2425]%[VN-MT-7, Nguyen Huynh]%[1D5H2-3]
 Một nhóm học sinh thi nhau giải khối rubik $4 \times 4$. Thời gian (đơn vị: giây) hoàn thành của nhóm học sinh được thống kê trong bảng sau:
 \begin{center}
 \begin{tabular}{|c|c|c|c|c|c|}
 \hline Thời gian giải rubik &{$[8 ; 10)$}&{$[10 ; 12)$}&{$[12 ; 14)$}&{$[14 ; 16)$}&{$[16 ; 18)$}\\
 \hline Số học sinh & $4$ & $6$ & $8$ & $4$ & $3$ \\
 \hline
 \end{tabular}
 \end{center}
 Tìm tứ phân vị thứ nhất và tứ phân vị thứ ba của mẫu số liệu.
 \choice
 {\True $Q_1=10{,}75$, $Q_3=14{,}375$}
 {$Q_1=11{,}0625$, $Q_3=14{,}375$}
 {$Q_1=10{,}75$, $Q_3=13{,}83$}
 {$Q_1=10{,}85$, $Q_3=14{,}75$}
 \loigiai{Cỡ mẫu là $n=25$.\\
 Tứ phân vị thứ nhất của mẫu số liệu gốc là $\dfrac{x_6+x_7}{2}\in[10 ; 12)$. Do đó, tứ phân vị thứ nhất của mẫu số liệu ghép nhóm là \[Q_1=10+\dfrac{\dfrac{25}{4}-4}{6}(12-10)=10{,}75.\]
 Tứ phân vị thứ ba của mẫu số liệu là $\dfrac{x_{19}+x_{20}}{2}\in[14 ; 16)$. Do đó, tứ phân vị thứ ba của mẫu số liệu ghép nhóm là \[Q_3=14+\dfrac{\dfrac{3\cdot25}{4}-(4+6+8)}{4}(16-14)=14{,}375.\]}
\end{ex}

\begin{ex}%[2-D3B3-SO-11-2425]%[VN-MT-7, Nguyen Huynh]%[2D3H1-3]
 Mỗi ngày bác Hương đều đi bộ để rèn luyện sức khoẻ. Quãng đường đi bộ mỗi ngày (đơn vị: km) của bác Hương trong $20$ ngày được thống kê lại ở bảng sau:
 \begin{center}
 \begin{tabular}{|c|c|c|c|c|c|}
 \hline
 Quãng đường & $[2{,}7;3{,}0)$ & $[3{,}0;3{,}3)$ & $[3{,}3;3{,}6)$ & $[3{,}6;3{,}9)$ & $[3{,}9;4{,}2)$ \\
 \hline
 Số ngày & $3$ & $6$ & $5$ & $4$ & $2$ \\
 \hline
 \end{tabular}
 \end{center}
 Khoảng tứ phân vị của mẫu số liệu ghép nhóm là
 \choice
 {$0{,}9$}
 {$0{,}975$}
 {$0{,}5$}
 {\True $0{,}575$}
 \loigiai{
 Cỡ mẫu $n=20$.\\
 Gọi $x_1$, $x_2$, $\ldots$, $x_{20}$ là mẫu số liệu về quãng đường đi bộ mỗi ngày của bác Hương trong $20$ ngày được xếp theo thứ tự không giảm.\\
 Ta có $x_1,\ldots,x_3 \in [2{,}7;3{,}0)$; $x_4,\ldots,x_9 \in [3{,}0;3{,}3)$; $x_{10},\ldots,x_{14} \in [3{,}3;3{,}6)$; $x_{15},\ldots,x_{18} \in [3{,}6;3{,}9)$; $x_{19},x_{20} \in [3{,}9;4{,}2)$.\\
 Tứ phân vị thứ nhất của mẫu số liệu gốc là $\dfrac{1}{2} \left(x_5+x_6 \right)\in [3{,}0;3{,}3)$.
 Do đó, tứ phân vị thứ nhất của mẫu số liệu ghép nhóm là \[Q_1=3{,}0+\dfrac{\dfrac{20}{4}-3}{6} (3{,}3-3{,}0)=3{,}1\]
 Tứ phân vị thứ ba của mẫu số liệu là $\dfrac{1}{2} \left(x_{15}+x_{16} \right)\in [3{,}6;3{,}9)$.
 Do đó, tứ phân vị thứ ba của mẫu số liệu ghép nhóm là
 \[Q_3=3{,}6+\dfrac{\dfrac{3\cdot 20}{4}-(3+6+5)}{4} (3{,}9-3{,}6)=3{,}675.
 \]
 Khoảng tứ phân vị của mẫu số liệu ghép nhóm là
 \[\Delta_Q=Q_3-Q_1=0{,}575.
 \]
 }
\end{ex}


\begin{ex}%[2-D3B3-SO-11-2425]%[VN-MT-7, Nguyen Huynh]%[1D5H1-3]
 Doanh thu bán hàng trong $20$ ngày được lựa chọn ngẫu nhiên của một của hàng được ghi lại ở bảng sau (đơn vị: triệu đồng):
 \begin{center}
 \begin{tabular}{|c|c|c|c|c|c|}
 \hline
 Doanh thu & $[5;7)$ & $[7;9)$ & $[9;11)$ & $[11;13)$ & $[13;15)$ \\
 \hline
 Số ngày & $2$ & $7$ & $7$ & $3$ & $1$ \\
 \hline
 \end{tabular}
 \end{center}
 Số trung bình của mẫu số liệu trên thuộc khoảng nào trong các khoảng dưới đây?
 \choice
 {$[7;9)$}
 {\True $[9;11)$}
 {$[11;13)$}
 {$[13;15)$}
 \loigiai{
 Bảng tần số ghép nhóm theo giá trị đại diện là
 \begin{center}
 \begin{tabular}{|c|c|c|c|c|c|}
 \hline
 Doanh thu & $[5;7)$ & $[7;9)$ & $[9;11)$ & $[11;13)$ & $[13;15)$ \\
 \hline
 Giá trị đại diện & $6$ & $8$ & $10$ & $12$ & $14$ \\
 \hline
 Số ngày & $2$ & $7$ & $7$ & $3$ & $1$ \\
 \hline
 \end{tabular}
 \end{center}
 Số trung bình \[\overline{x}=\dfrac{2\cdot6+7\cdot8+7\cdot10+3\cdot12+1\cdot14}{20}=9{,}4.\]
 }
\end{ex}


\begin{ex}%[2-D3B3-SO-11-2425]%[VN-MT-7, Nguyen Huynh]%[2D3H2-2]
 Một siêu thị thống kê số tiền (đơn vị: chục nghìn đồng) mà $44$ khách hàng mua hàng ở siêu thị đó trong một ngày. Số liệu được ghi lại trong bảng sau:
 \begin{center}
 \begin{tabular}{|c|c|c|}
 \hline
 Nhóm & Giá trị đại diện & Tần số \\
 \hline
 $[40;45)$ & $42{,}5$ & $4$ \\
 \hline
 $[45;50)$ & $47{,}5$ & $14$ \\
 \hline
 $[50;55)$ & $52{,}5$ & $8$ \\
 \hline
 $[55;60)$ & $57{,}5$ & $10$ \\
 \hline
 $[60;65)$ & $62{,}5$ & $6$ \\
 \hline
 $[65;70)$ & $67{,}5$ & $2$ \\
 \hline
 & & $n=44$ \\
 \hline
 \end{tabular}
 \end{center}
 Phương sai của mẫu số liệu ghép nhóm trên là
 \choice
 {$53{,}2$}
 {\True $46{,}1$}
 {$30$}
 {$11$} 
 \loigiai{Số trung bình cộng của mẫu số liệu ghép nhóm là
 \[\overline{x}=\dfrac{4\cdot42{,}5+14\cdot47{,}5+8\cdot52{,}5+10\cdot57{,}5+6\cdot62{,}5+2\cdot67{,}5}{44}=\dfrac{585}{11}.\]
 Phương sai của mẫu số liệu ghép nhóm là
 \begin{align*}
 \allowdisplaybreaks
 s^2&=\dfrac{4\left(42{,}5-\dfrac{585}{11} \right)^2+14\left(47{,}5-\dfrac{585}{11} \right)^2+8\left(52{,}5-\dfrac{585}{11} \right)^2+10\left(57{,}5-\dfrac{585}{11} \right)^2}{44} \\
 &+\dfrac{6\left(62{,}5-\dfrac{585}{11} \right)^2+2\cdot \left(67{,}5-\dfrac{585}{11} \right)^2}{44}\approx 46{,}1.
 \end{align*}
 }
\end{ex}


\begin{ex}%[2-D3B3-SO-11-2425]%[VN-MT-7, Nguyen Huynh]%[2D3H2-2]
 Khảo sát chiều cao (đơn vị cm) của học sinh lớp 12A, ta thu được kết quả như sau: 
 \begin{center}
 \begin{tabular}{|c|c|c|c|c|c|}
 \hline
 Kết quả đo & $[150;155)$ & $[155;160)$ & $[160;165)$ & $[165;170)$ & $[170;175)$ \\
 \hline
 Số học sinh & $6$ & $10$ & $14$ & $5$ & $5$ \\
 \hline
 \end{tabular}
 \end{center}
 Độ lệch chuẩn của mẫu số liệu ghép nhóm trên thuộc khoảng nào sau đây
 \choice
 {$\left(5{,}5;6\right)$}
 {\True $\left(6;6{,}5\right)$}
 {$\left(6{,}5;7\right)$}
 {$\left(7;7{,}5\right)$}
 \loigiai{
 Chọn giá trị đại diện cho các nhóm số liệu, ta có
 \begin{center}
 \begin{tabular}{|c|c|c|c|c|c|}
 \hline
 Giá trị đại diện & $152{,}5$ & $157{,}5$ & $162{,}5$ & $167{,}5$ & $172{,}5$ \\
 \hline
 Số học sinh & $6$ & $10$ & $14$ & $5$ & $5$ \\
 \hline
 \end{tabular}
 \end{center}
 Tổng số học sinh tham gia khảo sát là $n=6+10+14+5+5=40$.\\
 Chiều cao trung bình của học sinh trong lớp là \[\overline{x}=\dfrac{152{,}5\cdot 6+157{,}5\cdot10+162{,}5\cdot14+167{,}5\cdot5+172{,}5\cdot5}{40}=161{,}625\approx 161{,}6.\]
 Phương sai của mẫu số liệu trên là
 \begin{align*}
 s^2&=\dfrac{m_1 \left(x_1-\overline{x}\right)^2+\cdots+m_k \left(x_k-\overline{x}\right)^2}{n}\\
 &=\dfrac{6\left(152{,}5-161{,}6\right)^2+10\left(157{,}5-161{,}6\right)^2+14\left(162{,}5-161{,}6\right)^2}{40}\\
 &+\dfrac{5\left(167{,}5-161{,}6\right)^2+6\left(172{,}5-161{,}6\right)^2}{40}\\
 &\approx 36{,}1.
 \end{align*}
 Độ lệch chuẩn của mẫu số liệu trên là $s=\sqrt{s^2}=\sqrt{36{,}1} \approx 6{,}01\in(6;6{,}5)$.
 }
\end{ex}


\begin{ex}%[2-D3B3-SO-11-2425]%[VN-MT-7, Nguyen Huynh]%[2D3N1-1]
 Có bao nhiêu nhận xét đúng trong các nhận xét sau:
 \begin{enumerate}
 \item Khoảng biến thiên của mẫu số liệu ghép nhóm luôn luôn bằng khoảng biến thiên của mẫu số liệu.
 \item Khoảng biến thiên của mẫu số liệu ghép nhóm được dùng để đo mức độ phân tán của mẫu số liệu ghép nhóm.
 \item Khoảng biến thiên của mẫu số liệu ghép nhóm càng lớn thì mẫu số liệu càng phân tán.
 \end{enumerate}
 \choice
 {$0$}
 {$1$}
 {\True $2$}
 {$3$}
 \loigiai{
 \begin{itemize}
 \item Nhận xét ``Khoảng biến thiên của mẫu số liệu ghép nhóm luôn luôn bằng khoảng biến thiên của mẫu số liệu''\, sai vì khoảng biến thiên của mẫu số liệu ghép nhóm xấp xỉ cho khoảng biến thiên của mẫu số liệu. 
 \item Nhận xét ``Khoảng biến thiên của mẫu số liệu ghép nhóm được dùng để đo mức độ phân tán của mẫu số liệu ghép nhóm''\, đúng.
 \item Nhận xét ``Khoảng biến thiên của mẫu số liệu ghép nhóm càng lớn thì mẫu số liệu càng phân tán''\, đúng.
 \end{itemize}
 
 }
\end{ex}


\begin{ex}%[2-D3B3-SO-11-2425]%[VN-MT-7, Nguyen Huynh]%[2D3N1-1]
 Nhận xét nào \textbf{sai} trong các nhận xét sau?
 \begin{enumerate}
 \item Khoảng tứ phân vị của mẫu số liệu ghép nhóm bị ảnh hưởng bởi các giá trị bất thường trong mẫu số liệu.
 \item Khoảng tứ phân vị của mẫu số liệu ghép nhóm xấp xỉ cho khoảng tứ phân vị của mẫu số liệu.
 \item Khoảng tứ phân vị càng lớn thì mẫu số liệu càng phân tán.
 \item Khoảng tứ phân vị được dùng để đo mức độ phân tán của mẫu số liệu ghép nhóm.
 \end{enumerate}
 \choice
 {\True Nhận xét a)}
 {Nhận xét b)}
 {Nhận xét c)}
 {Nhận xét d)}
 \loigiai{\begin{itemize}
 \item Nhận xét ``Khoảng tứ phân vị của mẫu số liệu ghép nhóm bị ảnh hưởng bởi các giá trị bất thường trong mẫu số liệu''\, sai vì khoảng tứ phân vị của mẫu số liệu ghép nhóm chỉ phụ thuộc vào nửa giữa của mẫu số liệu, nên không bị ảnh hưởng bởi các giá trị bất thường và có thể dùng đại lượng này để loại giá trị bất thường.
 \item Nhận xét ``Khoảng tứ phân vị của mẫu số liệu ghép nhóm xấp xỉ cho khoảng tứ phân vị của mẫu số liệu''\, đúng.
 \item Nhận xét ``Khoảng tứ phân vị càng lớn thì mẫu số liệu càng phân tán''\, đúng.
 \item Nhận xét ``Khoảng tứ phân vị được dùng để đo mức độ phân tán của mẫu số liệu ghép nhóm''\, đúng.
 \end{itemize}
 }
\end{ex}

\Closesolutionfile{ans}

\cauds
\Opensolutionfile{ans}[ans/ans\currfilebase-Phan-II]

\begin{ex}%[2-D3B3-SO-11-2425]%[VN-MT-7, Nguyen Huynh]%[2D3H2-2]
 Bảng bảng biểu diễn mẫu số liệu ghép nhóm về nhiệt độ ($^\circ$C) của tỉnh Nghệ An tháng $5$ năm $2024$.
 \begin{center}
 \begin{tabular}{|c|c|c|c|}
 \hline
 Nhóm & Giá trị đại diện & Tần số & Tần số tích lũy \\
 \hline
 $[29;31)$ & $30$ & $1$ & $1$ \\
 \hline
 $[31;33)$ & $32$ & $4$ & $5$ \\
 \hline
 $[33;35)$ & $34$ & $5$ & $10$ \\
 \hline
 $[35;37)$ & $36$ & $13$ & $26$ \\
 \hline
 $[37;39]$ & $38$ & $7$ & $33$ \\
 \hline
 & & $n=30$ & \\
 \hline
 \end{tabular}
 \end{center}
 \choiceTF
 {\True Nhóm $[31;33)$ có tần số bằng $4$}
 {Mốt của mẫu số liệu ghép nhóm đã cho là $13$ (làm tròn đến hàng phần trăm)}
 {\True Khoảng tứ phân vị của mẫu số liệu ghép nhóm trên bằng $2{,}92$ (làm tròn đến hàng phần trăm)}
 {\True Phương sai của mẫu số liệu ghép nhóm trên bằng $4{,}57$ (làm tròn đến hàng phần trăm)} 
 \loigiai{
 \begin{itemchoice}
 \itemch \textbf{Đúng}. 
 \\Nhóm $[31;33)$ có tần số bằng $4$.
 
 \itemch \textbf{Sai}. 
 \\Ta có nhóm $[35;37)$ có tần số lớn nhất nên mốt của mẫu số liệu trên là \[M_{\text{o}}=u+\dfrac{n_i-n_{i-1}}{2n_i-n_{i-1}-n_{i+1}}\cdot g=35+\dfrac{13-5}{2\cdot 13-5-7}\cdot2=36{,}14.\]
 
 \itemch \textbf{Đúng}. 
 \\Ta có số phần tử của mẫu là $n=30$.\\
 Ta có $\dfrac{n}{4}=7{,}5$ nên nhóm $[33;35)$ là nhóm đầu tiên có tần số tích lũy lớn hơn hoặc bằng $7{,}5$.\\
 Nhóm $[33;35)$ có $s=33;h=2; n=5$ và nhóm $2$ là nhóm $[31;33)$ có $cf_1=5$.\\
 Áp dụng công thức ta có tứ phân vị thứ nhất là $Q_1=33+\dfrac{7{,}5-5}{5}\cdot2=34$ ($^\circ$C).\\
 Ta có $\dfrac{3n}{4}=22{,}5$ nên nhóm $[35;37)$ là nhóm đầu tiên có tần số tích lũy lớn hơn hoặc bằng $22{,}5$.\\
 Xét nhóm $4$ là nhóm $[35;37)$ có $t=35$; $l=2$; $n_4=13$ và nhóm $3$ có tần số tích lũy $cf_4=10$.\\
 Áp dụng công thức, ta có tứ phân vị thứ $3$ là $Q_3=35+\dfrac{22{,}5-10}{13}\cdot2=36{,}92$ ($^\circ$C).\\
 Vậy khoảng tứ phân vị của mẫu số liệu ghép nhóm đã cho là \[\Delta Q=Q_3-Q_1=36{,}92-34=2{,}92.\]
 
 \itemch \textbf{Đúng}. 
 \\Ta có số trụng bình cộng của mẫu số liệu ghép nhóm trên là\\
 \[\overline{x}=\dfrac{1}{30} \left(1\cdot30+32\cdot4+34\cdot5+36\cdot13+38\cdot7\right)=35{,}4\,(^\circ \mathrm C).\]
 Vậy phương sai của mẫu số liệu ghép nhóm trên là 
% \begin{eqnarray*}
% s^2&=&\dfrac{1}{30} \left[1\cdot(30-35{,}4)^2+4\cdot(32-35{,}4)^2+5\cdot(34-35{,}4)^2+13\cdot(36-35{,}4)^2+7\cdot(38-35{,}4)^2\right]\\&\approx&4{,}57.
% \end{eqnarray*}
 \[s^2=\dfrac{1}{30} \left(30^2\cdot1+32^2\cdot4+34^2\cdot5+36^2\cdot13+38^2\cdot7\right)-(35{,}4)^2\approx 4{,}57.\]
 \end{itemchoice}
 }
\end{ex}

\begin{ex}%[2-D3B3-SO-11-2425]%[VN-MT-7, Nguyen Huynh]%[2D3H2-2]
 Cho bảng phân bố tần số ghép lớp cân nặng (đơn vị: kg) của các công nhân trong một công ty như sau 
 \begin{center}
 \begin{tabular}{|c|c|c|c|c|c|c|}
 \hline Cân nặng &$[50;52)$&$[52;54)$&$[54;56)$&$[56;58)$&$[58;60)$& Cộng \\
 \hline Tần số & $15$ & $20$ & $45$ & $15$ & $5$ & $100$ \\
 \hline
 \end{tabular}
 \end{center}
 \choiceTF
 {\True Tần suất của nhóm $[52;54)$ là $20$}
 {Số trung vị của mẫu số liệu lớn hơn $54{,}9$}
 {\True Khoảng biến thiên của mẫu số liệu trên là $10$}
 {Độ lệch chuẩn của mẫu số liệu trên là $4{,}35$}
 \loigiai{
 \begin{itemchoice}
 \itemch \textbf{Đúng}. 
 \\Tần số của nhóm $[52;54)$ là $20$.\\
 Tần suất của nhóm $[52;54)$ là $\dfrac{20}{100}\cdot100\%=20\%$.
 
 \itemch \textbf{Sai}. 
 \\Trung vị của mẫu số liệu là $x_3 \in [54;56)$.\\
 Do đó, trung vị của mẫu số liệu ghép nhóm là\\ $M_{\text{e}}=Q_2=54+\dfrac{\dfrac{2\cdot100}{4}-(15+20)}{45} (56-54)=\dfrac{164}{3}\approx54{,}667$.
 \\Do đó trung vị của mẫu số liệu bé hơn $54{,}9$
 \itemch\textbf{Đúng}. 
 \\Khoảng biến thiên của mẫu số liệu là $R=60-50=10$.
 
 \itemch \textbf{Sai}. 
 \\Số trung bình cộng của mẫu số liệu ghép nhóm của công ty là \[\overline{x}=\dfrac{51\cdot15+53\cdot20+55\cdot45+57\cdot15+59\cdot5}{100}=54{,}5.\]
 Phương sai của mẫu số liệu ghép nhóm của công ty là
% \begin{eqnarray*}
% s^2&=&\dfrac{15\cdot \left(51-54{,}5\right)^2+20\cdot \left(53-54{,}5\right)^2+45\cdot \left(55-54{,}5\right)^2+15\cdot \left(57-54{,}5\right)^2+5\cdot(59-54{,}4)^2}{100}\\&=&4{,}35.
% \end{eqnarray*}
 \[s^2=\dfrac{51^2\cdot15+53^2\cdot20+55^2\cdot45+57^2\cdot15+59^2\cdot5}{100}-(54{,}5)^2=4{,}35.\]
 Độ lệch chuẩn của mẫu số liệu ghép nhóm của công ty là $s=\sqrt{s^2}=\sqrt{4{,}35} \approx 2{,}09$.
 \end{itemchoice}
 }
\end{ex}

\begin{ex}%[2-D3B3-SO-11-2425]%[VN-MT-7, Nguyen Huynh]%[2D3V2-3]
 Cho bảng số liệu dưới đây về thời gian (phút) tập thể dục buổi sáng của hai bạn Bình và Chi trong $30$ ngày.
 \begin{center}
 \begin{tabular}{|c|c|c|c|c|c|}
 \hline
 Thời gian & $[15;20)$ & $[20;25)$ & $[25;30)$ & $[30;35)$ & $[35;40)$ \\
 \hline
 Bạn Bình & $5$ & $8$ & $10$ & $4$ & $3$ \\
 \hline
 Bạn Chi & $10$ & $10$ & $5$ & $3$ & $2$ \\
 \hline
 \end{tabular}
 \end{center}
 \choiceTF
 {\True Khoảng biến thiên của mẫu số liệu ghép nhóm về thời gian tập thể dục của Chi là $25$ (phút)}
 {Tứ phân vị thứ nhất của mẫu số liệu ghép nhóm về thời gian tập thể dục buổi sáng của bạn Bình là $Q_1=\dfrac{354}{16}$}
 {\True Khoảng tứ phân vị của mẫu số liệu ghép nhóm về thời gian tập thể dục buổi sáng của bạn Chi là $28{,}75$}
 {\True Phương sai của mẫu số liệu ghép nhóm về thời gian tập thể dục buổi sáng của bạn Bình là $\dfrac{314}{9}$} 
 \loigiai{
 Ta có \begin{center}
 \begin{tabular}{|c|c|c|c|c|c|}
 \hline
 Thời gian & $\left[15;20\right)$ & $\left[20;25\right)$ & $\left[25;30\right)$ & $\left[30;35\right)$ & $\left[35;40\right)$ \\
 \hline
 Giá trị đại diện & $17{,}5$ & $22{,}5$ & $27{,}5$ & $32{,}5$ & $37{,}5$ \\
 \hline
 Bạn Bình & $5$ & $8$ & $10$ & $10$ & $10$ \\
 \hline
 Bạn Chi & $10$ & $10$ & $5$ & $3$ & $2$ \\
 \hline
 \end{tabular}
 \end{center}
 
 \begin{itemchoice}
 \itemch \textbf{Đúng}. 
 \\Khoảng biến thiên của mẫu số liệu ghép nhóm là $40-15=25$ (phút).
 \itemch \textbf{Sai}. 
 \\Xét số liệu của bạn Bình.\\
 Ta có cỡ mẫu $n=30$.\\
 Vì $\dfrac{n}{4}=\dfrac{30}{4}=7{,}5$ và $5< 7{,}5< 5+8$ nên tứ phân vị thứ nhất thuộc nhóm $[20;25)$.\\
 Tứ phân vị thứ nhất của mẫu số liệu về thời gian tập thể dục buổi sáng của bạn Bình là \[Q_1=20+\dfrac{\dfrac{30}{4}-5}{8}\cdot5=\dfrac{345}{16}.\]
 \itemch \textbf{Đúng}. 
 \\Xét số liệu của bạn Chi.\\
 Ta có cỡ mẫu $n=30$.\\
 Vì $\dfrac{n}{4}=\dfrac{30}{4}=7{,}5$ và $7{,}5< 10$ nên tứ phân vị thứ nhất thuộc nhóm $[15;20)$.\\
 Tứ phân vị thứ nhất của mẫu số liệu về thời gian tập thể dục buổi sáng của bạn Chi là\\ \[Q'_1=15+\dfrac{\dfrac{30}{4}-0}{10} \cdot 5=18{,}75.\]
 Vì $\dfrac{3n}{4}=\dfrac{3 \cdot 30}{4}=22{,}5$ và $10+10< 22{,}5< 10+10+5$. Do đó tứ phân vị thứ ba thuộc nhóm $[25;30)$.\\
 Tứ phân vị thứ ba của mẫu số liệu về thời gian tập thể dục buổi sáng của bạn Chi là \[Q'_3=25+\dfrac{\dfrac{3\cdot30}{4}-(10+10)}{5}\cdot5=27{,}5.\]
 Vậy khoảng tứ phân vị là $\Delta_Q'=Q'_3-Q'_1=28{,}75$.
 
 \itemch \textbf{Đúng}. 
 \\Thời gian trung bình bạn Bình tập thể dục buổi sáng là
 \[\overline{x}=\dfrac{17{,}5\cdot5+22{,}5\cdot8+27{,}5\cdot10+32{,}5\cdot4+37{,}5\cdot3}{30}=\dfrac{157}{6} \approx 26{,}17.\]
 Phương sai của mẫu số liệu ghép nhóm về thời gian tập thể dục buổi sáng của bạn Bình là
 % \begin{align*}
 % s_B^2&=\dfrac{5\left(17{,}5-\dfrac{157}{6} \right)^2+8\left(22{,}5-\dfrac{157}{6} \right)^2+10\left(27{,}5-\dfrac{157}{6} \right)^2+4\left(32{,}5-\dfrac{157}{6} \right)^2}{30}\\
 % &+\dfrac{3\left(37,5-\dfrac{157}{6} \right)^2}{30}\\
 % &=\dfrac{314}{9}.
 % \end{align*}
 \[s_B^2=\dfrac{17{,}5^2\cdot5+22{,}5^2\cdot8+27{,}5^2\cdot10+32{,}5^2\cdot4+37{,}5^2\cdot3}{30}-\left(\dfrac{157}{6}\right)^2 =\dfrac{314}{9}.\]
 \end{itemchoice}
 }
\end{ex}

\begin{ex}%[2-D3B3-SO-11-2425]%[VN-MT-7, Nguyen Huynh]%[2D3V2-3]
 Bảng sau đây cho biết chiều cao của các em học sinh lớp 12A và 12B:
 \begin{center}
 \begin{tabular}{|c|c|c|c|c|c|c|}
 \hline
 Chiều cao (cm) & $[145; 150)$ & $[150; 155)$ & $[155; 160)$ & $[160; 165)$ & $[165; 170)$ & $[170; 175)$ \\
 \hline
 Số học sinh của lớp 12A& $2$ & $1$ & $15$ & $11$ & $9$ & $3$ \\
 \hline
 Số học sinh của lớp 12B & $0$ & $1$ & $16$ & $11$ & $10$ & $4$ \\
 \hline 
 \end{tabular}
 \end{center}
 \choiceTF
 {\True Dựa vào khoảng biến thiên của mẫu số liệu ghép nhóm thì chiều cao của học sinh lớp 12A phân tán hơn lớp 12B}
 {Dựa vào khoảng tứ phân vị của mẫu số liệu ghép nhóm thì học sinh lớp 12A có chiều cao phân tán hơn học sinh lớp 12B}
 {Dựa vào phương sai của mẫu số liệu ghép nhóm thì chiều cao của học sinh lớp 12A ít phân tán hơn học sinh lớp 12B}
 {\True Học sinh lớp 12B có chiều cao đồng đều hơn học sinh lớp 12A vì có độ lệch chuẩn nhỏ hơn} 
 \loigiai{
 \begin{itemchoice}
 \itemch \textbf{Đúng}.\\ Với số liệu của học sinh lớp 12A, có khoảng biến thiên $R_A=175-145=30$.\\
 Với số liệu của học sinh lớp 12B, có khoảng biến thiên $R_B=175-150=25$.\\
 Do $R_A > R_B$ nên chiều cao của học sinh lớp 12A phân tán hơn lớp 12B.
 \itemch \textbf{Sai}.\\Với mẫu số liệu ghép nhóm của học sinh lớp 12A.\\
 Cỡ mẫu là $n=2+1+15+11+9+3=41$.\\
 Gọi $x_1$, $x_2$, $x_3$, $\ldots$, $x_{41}$ là mẫu số liệu gồm chiều cao của học sinh lớp 12A được sắp xếp theo thứ tự không giảm.\\
 Tứ phân vị thứ nhất của mẫu số liệu gốc là $\dfrac{1}{2} (x_{10}+x_{11})\in [155; 160)$ nên nhóm chứa tứ phân vị thứ nhất là nhóm $[155; 160)$. Do đó \[Q_1=155+\dfrac{\dfrac{41}{4}-3}{15}\cdot5\approx 157{,}42.\]
 Tứ phân vị thứ ba của mẫu số liệu gốc là $\dfrac{1}{2} (x_{31}+x_{32})\in [165; 170)$ nên nhóm chứa tứ phân vị thứ ba là nhóm $[165; 170)$. Do đó \[Q_3=165+\dfrac{\dfrac{3\cdot41}{4}-29}{9}\cdot5\approx 165{,}97.\]
 Suy ra $\Delta_{Q_A}=Q_3-Q_1 \approx 8{,}55$.\\
 Với mẫu số liệu ghép nhóm của học sinh lớp 12B\\
 Cỡ mẫu là $n=1+16+11+10+4=42$.
 \\Gọi $x_1$, $x_2$, $x_3$, $\ldots$, $x_{42}$ là mẫu số liệu gồm chiều cao của học sinh lớp 12B được sắp xếp theo thứ tự không giảm.\\
 Tứ phân vị thứ nhất của mẫu số liệu gốc là $x_{11} \in [155; 160)$ nên nhóm chứa tứ phân vị thứ nhất là nhóm $[155; 160)$. Do đó \[Q'_1=155+\dfrac{\dfrac{42}{4}-1}{16}\cdot5\approx 157{,}97.\]
 Tứ phân vị thứ ba của mẫu số liệu gốc là $x_{32} \in [165; 170)$ nên nhóm chứa tứ phân vị thứ ba là nhóm $[165; 170)$. Do đó \[Q'_3=165+\dfrac{\dfrac{3\cdot42}{4}-28}{10}\cdot5=166{,}75.\]
 Suy ra $\Delta_{Q_B}=Q'_3-Q'_1 \approx 8{,}78$.\\
 Do $\Delta_{Q_A} < \Delta_{Q_B}$ nên học sinh lớp 12A có chiều cao phân tán ít hơn học sinh lớp 12B.
 
 \itemch \textbf{Sai}.\\ Chọn giá trị đại diện cho các nhóm số liệu ta được bảng
 \begin{center}
 \begin{tabular}{|c|c|c|c|c|c|c|}
 \hline
 Chiều cao (cm) & $[145; 150)$ & $[150; 155)$ & $[155; 160)$ & $[160; 165)$ & $[165; 170)$ & $[170; 175)$ \\
 \hline
 Giá trị đại diện & $147{,}5$ & $152{,}5$ & $157{,}5$ & $162{,}5$ & $167{,}5$ & $172{,}5$ \\
 \hline
 Số học sinh của lớp 12A & $2$ & $1$ & $15$ & $11$ & $9$ & $3$ \\
 \hline
 Số học sinh của lớp 12B & $0$ & $1$ & $16$ & $11$ & $10$ & $4$ \\
 \hline 
 \end{tabular}
 \end{center}
 Chiều cao trung bình của học sinh lớp 12A là \[\overline{x}_A=\dfrac{1}{41} (2\cdot 147{,}5+1\cdot 152{,}5+15\cdot 157{,}5+11\cdot 162{,}5+9\cdot 167{,}5+3\cdot 172{,}5)\approx 161{,}52.\]
 Chiều cao trung bình của học sinh lớp 12B là \[\overline{x}_B=\dfrac{1}{42} (0\cdot 147{,}5+1\cdot 152{,}5+16\cdot 157{,}5+11\cdot 162{,}5+10\cdot 167{,}5+4\cdot 172{,}5)=162{,}5.\]
 Phương sai của mẫu số liệu lớp 12A là
 \begin{eqnarray*}
 s_A^2&=&\dfrac{1}{41} \left(2\cdot 147{,}5^2+1\cdot 152{,}5^2+15\cdot 157{,}5^2+11\cdot 162{,}5^2+9\cdot 167{,}5^2+3\cdot 172{,}5^2\right)-\left(161{,}52\right)^2\\&\approx& 34{,}41.
 \end{eqnarray*}
 
 Phương sai của mẫu số liệu lớp 12B là
 \begin{eqnarray*}
 s_B^2&=&\dfrac{1}{42} \left(0\cdot 147{,}5^2+1\cdot 152{,}5^2+16\cdot 157{,}5^2+11\cdot 162{,}5^2+10\cdot 167{,}5^2+4\cdot 172{,}5^2\right)-\left(162{,}50\right)^2\\&\approx& 27{,}38.
 \end{eqnarray*}
 Vậy dựa vào phương sai của mẫu số liệu ghép nhóm thì chiều cao của học sinh lớp 12A phân tán hơn học sinh lớp 12B.
 
 \itemch \textbf{Đúng}. 
 \\Độ lệch chuẩn của mẫu số liệu lớp 12A là $s_A \approx \sqrt{35{,}83} \approx 5{,}99$.\\
 Độ lệch chuẩn của mẫu số liệu lớp 12B là $s_B \approx \sqrt{27{,}38} \approx 5{,}23$.\\
 Vậy học sinh lớp 12B có chiều cao đồng đều hơn học sinh lớp 12A vì có độ lệch chuẩn nhỏ hơn.
 \end{itemchoice}
 }
\end{ex}
\Closesolutionfile{ans}

\caukq
\Opensolutionfile{ans}[ans/ans\currfilebase-Phan-III]
\begin{ex}%[2-D3B3-SO-11-2425]%[VN-MT-7, Nguyen Huynh]%[2D3N1-2]%[Câu 1]
 Bảng sau đây cho biết chiều cao (đơn vị: cm) của học sinh lớp 5A:
 \begin{center}
 \begin{tabular}{|c|c|c|c|c|c|c|}
 \hline
 Chiều cao & $[85; 90)$ & $[90; 95)$ & $[95; 100)$ & $[100; 105)$ & $[105; 110)$& $[110; 115)$\\
 \hline Tần số & $1$ & $4$ & $8$ & $12$ & $3$ & $2$ \\ \hline
 \end{tabular}
 \end{center}
 Tìm khoảng biến thiên của mẫu số liệu ghép nhóm về chiều cao của học sinh lớp 5A.
 
 \shortans[]{30}
 \loigiai{
 Khoảng biến thiên của mẫu số liệu ghép nhóm trên là $R=115-85=30$.
 }
\end{ex}

\begin{ex}%[2-D3B3-SO-11-2425]%[VN-MT-7, Nguyen Huynh]%[2D3H1-3]
 Bảng sau đây cho biết chiều cao (đơn vị: cm) của học sinh lớp 5A:
 \begin{center}
 \begin{tabular}{|c|c|c|c|c|c|c|}
 \hline
 Chiều cao & $[85; 90)$ & $[90; 95)$ & $[95; 100)$ & $[100; 105)$ & $[105; 110)$& $[110; 115)$\\
 \hline 
 Tần số & $1$ & $4$ & $8$ & $12$ & $3$ & $2$ \\ 
 \hline
 \end{tabular}
 \end{center}
 Tìm khoảng tứ phân vị của mẫu số liệu ghép nhóm về chiều cao của học sinh lớp 5A (làm tròn đến hàng phần mười).
 
 \shortans[]{7{,}4}
 \loigiai{
 Cỡ mẫu là $n=1+4+8+12+3+2=30$.\\
 Gọi $x_1$; $x_2$; $\ldots$; $x_{30}$ là mẫu số liệu gốc về chiều cao của học sinh lớp 5A được sắp xếp theo thứ tự không giảm.\\
 Khi đó 
 $x_1 \in [85;90)$;
 $x_2$, $x_3$, $x_4$, $x_5\in [90;95)$;
 $x_6$, $\ldots$, $x_{13} \in [95;100)$;
 $x_{14}$, $\ldots$, $x_{25} \in [100;105)$;
 $x_{26}$, $x_{27}$, $x_{28} \in [105;110)$;
 $x_{29}$, $x_{30} \in [110;115)$;\\
 Tứ phân vị thứ nhất của mẫu số liệu gốc là $x_{8} \in [95;100)$.\\
 Do đó tứ phân vị thứ nhất của mẫu số liệu ghép nhóm là 
 \[Q_1=95+\dfrac{\dfrac{30}{4}-5}{8} \cdot (100-95)\approx 96{,}56.\]
 Tứ phân vị thứ ba của mẫu số liệu gốc là $x_{23} \in [100;105)$.\\ 
 Do đó tứ phân vị thứ ba của mẫu số liệu ghép nhóm là 
 \[Q_3=100+\dfrac{\dfrac{3\cdot 30}{4}-13}{12} \cdot (105-100)\approx 103{,}96.\]
 Vậy khoảng tứ phân vị của mẫu số liệu trên là $\Delta_Q=Q_3-Q_1\approx 103{,}96-96{,}56=7{,}4$.}
\end{ex}

\begin{ex}%[2-D3B3-SO-11-2425]%[VN-MT-7, Nguyen Huynh]%[2D3H2-2]
 Số người xem trong $60$ buổi chiếu phim của một rạp chiếu phim nhỏ.
 \begin{center}
 \begin{tabular}{|c|c|c|c|c|c|c|c|}
 \hline
 Lớp người xem & $[0; 10)$&$[10; 20)$&$[20; 30)$&$[30; 40)$&$[40; 50)$&$[50; 60]$&Cộng\\
 \hline
 Tần số & $5$&$9$&$11$&$15$&$12$&$8$&$60$\\
 \hline
 \end{tabular}
 \end{center}
 Hãy tính phương sai của mẫu số liệu ghép nhóm trên (kết quả được làm tròn đến hàng đơn vị). 
 \par\shortans[]{220}
 \loigiai{
 \begin{center}
 \begin{tabular}{|c|c|c|c|c|c|c|c|}
 \hline
 Lớp người xem & $[0; 10)$&$[10; 20)$&$[20; 30)$&$[30; 40)$&$[40; 50)$&$[50; 60]$&Cộng\\
 \hline
 Giá trị đại diện & $5$ & $15$ & $25$ & $35$ & $45$ & $55$ & \\
 \hline
 Tần số & $5$&$9$&$11$&$15$&$12$&$8$&$60$\\
 \hline
 \end{tabular}
 \end{center}
 Số trung bình của mẫu số liệu ghép nhóm là
 \begin{eqnarray*}
 \overline{x}&=&\dfrac{n_1c_1+n_2c_2+n_3c_3+n_4c_4+n_5c_5+n_6c_6}{n}\\
 &=&\dfrac{5 \cdot 5+9 \cdot 15+11 \cdot 25+15 \cdot 35+12 \cdot 45+8 \cdot 55}{60}\\
 &=&\dfrac{97}{3}\approx 32{,}3.
 \end{eqnarray*}
 Phương sai của mẫu số liệu ghép nhóm là
% \begin{eqnarray*}
% s_x^2&=&\dfrac{n_1(c_1-\overline{x})^2+n_2(c_2-\overline{x})^2+n_3(c_3-\overline{x})^2+n_4(c_4-\overline{x})^2+n_5(c_5-\overline{x})^2+n_6(c_6-\overline{x})^2}{n}\\
% &=&\dfrac{5(5-32{,}3)^2+9(15-32{,}3)^2+11(25-32{,}3)^2+15(35-32{,}3)^2}{60}+\\
% && +\dfrac{12(45-32{,}3)^2+8(55-32{,}3)^2}{60}\\
% &\approx& 220.
% \end{eqnarray*} 
 \[s_x^2=\dfrac{5 \cdot 5^2+9 \cdot 15^2+11 \cdot 25^2+15 \cdot 35^2+12 \cdot 45^2+8 \cdot 55^2}{60}-\left(\dfrac{97}{3}\right)^2 \approx 220.\] 
 }
\end{ex}

\begin{ex}%[2-D3B3-SO-11-2425]%[VN-MT-7, Nguyen Huynh]%[2D3H2-2]
 Bảng thống kê cự li ném tạ của một vận động viên như sau:
 \begin{center}
 \begin{tabular}{|c|c|c|c|c|c|}
 \hline
 Cự li (m) & $[19; 19{,}5)$&$[19{,}5; 20)$&$[20; 20{,}5)$&$[20{,}5; 21)$&$[21; 21{,}5)$\\
 \hline
 Tần số & $13$ & $45$ & $24$ & $12$ & $6$\\
 \hline
 \end{tabular}
 \end{center} 
 Hãy tính độ lệch chuẩn của mẫu số liệu ghép nhóm trên (kết quả được làm tròn đến hàng phần trăm).
 
 \shortans[]{0{,}53}
 \loigiai{
 \begin{center}
 \begin{tabular}{|c|c|c|c|c|c|}
 \hline
 Cự li (m) & $[19; 19{,}5)$&$[19{,}5; 20)$&$[20; 20{,}5)$&$[20{,}5; 21)$&$[21; 21{,}5)$\\
 \hline
 Giá trị đại diện & $19{,}25$ & $19{,}75$ & $20{,}25$ &$20{,}75$ &$21{,}25$ \\
 \hline
 Tần số & $13$ & $45$ & $24$ & $12$ & $6$\\
 \hline
 \end{tabular}
 \end{center}
 Cỡ mẫu $n=100$.
 \\Số trung bình \[\overline{x}=\dfrac{13 \cdot 19{,}25+45 \cdot 19{,}75+24 \cdot 20{,}25+12 \cdot 20{,}75+6 \cdot 21{,}25}{100}=20{,}015\]
 Phương sai
 \begin{eqnarray*}
 s^2&=&\dfrac{13 \cdot \left(19{,}25-20{,}015\right)^2+45 \cdot \left(19{,}75-20{,}015\right)^2+24 \cdot \left(20{,}25-20{,}015\right)^2}{100}\\
 && +\dfrac{12 \cdot \left(20{,}75-20{,}015\right)^2+6 \cdot \left(21{,}25-20{,}015\right)^2}{100} \\
 &\approx& 0{,}28. 
 \end{eqnarray*}
 Độ lệch chuẩn $s=\sqrt{0{,}28} \approx 0{,}53$.
 }
\end{ex}

\begin{ex}%[2-D3B3-SO-11-2425]%[VN-MT-7, Nguyen Huynh]%[2D3H2-3]
 Thành tích môn nhảy cao của các vận động viên tại một giải điền kinh dành cho học sinh trung học phổ thông như sau:
 \begin{center}
 \begin{tabular}{|c|c|c|c|c|}
 \hline
 Mức xà (cm) & $[170;172)$ & $[172;174)$ & $[174;176)$ & $[176;180)$ \\ \hline
 Số vận động viên & $3$ & $10$ & $6$ & $1$ \\ \hline
 \end{tabular}
 \end{center}
 Tính khoảng tứ phân vị $\Delta_Q$ và độ lệch chuẩn $s$ của mẫu số liệu ghép nhóm trên. Khi đó $\Delta_Q+s$ bằng bao nhiêu (kết quả làm tròn đến hàng phần trăm)?
 \par\shortans[]{3{,}96}
 \loigiai{
 Cỡ mẫu là $n=3+10+6+1=20$.\\
 Gọi $x_1$, $x_2$, $\ldots$, $x_{20}$ là mức xà của $20$ vận động viên được sắp xếp theo thứ tự tăng dần.\\
 Tứ phân vị thứ nhất của mẫu số liệu gốc là $\dfrac{x_5+x_6}{2}\in [172; 174)$. \\
 Do đó, tứ phân vị thứ nhất của mẫu số liệu ghép nhóm là \[Q_1=172+\dfrac{\dfrac{20}{4}-3}{10}\cdot(174-172)=172{,}4.\]
 Tứ phân vị thứ ba của mẫu số liệu gốc là $\dfrac{x_{15}+x_{16}}{2}\in [174; 176)$. \\
 Do đó, tứ phân vị thứ ba của mẫu số liệu ghép nhóm là \[Q_3=174+\dfrac{\dfrac{3\cdot20}{4}-13}{6}\cdot(176-174)\approx 174{,}7.\]
 Do đó khoảng tứ phân vị là $\Delta_Q=Q_3-Q_1\approx 174{,}7-172{,}4\approx 2{,}3$.\\
 Chọn giá trị đại diện cho mẫu số liệu ta có
 \begin{center}
 \begin{tabular}{|c|c|c|c|c|}
 \hline
 Mức xà (cm) &$[170; 172)$ &$[172; 174)$ &$[174; 176)$ &$[176; 180)$\\
 \hline
 Giá trị đại diện & $171$ & $173$ & $175$ & $178$\\
 \hline
 Số vận động viên & $3$ & $10$ & $6$ & $1$\\
 \hline
 \end{tabular}
 \end{center}
 Mức xà trung bình là $\overline{x}=\dfrac{3\cdot171+10\cdot173+6\cdot175+1\cdot178}{20}=173{,}55$.\\
 Phương sai và độ lệch chuẩn
 \[S^2=\dfrac{1}{20}\left(3\cdot171^2+10\cdot173^2+6\cdot175^2+1\cdot178^2\right)-(173{,}55)^2\approx 2{,}75.\]
 Suy ra $s=\sqrt{s^2}=\sqrt{2{,}75}\approx 1{,}66$. 
 Khi đó $\Delta_Q+s\approx 3{,}96$.
 }
\end{ex}

\begin{ex}%[2-D3B3-SO-11-2425]%[VN-MT-7, Nguyen Huynh]%[2D3H1-3]
 Một người ghi lại thời gian đàm thoại của một số cuộc gọi cho kết quả như bảng sau:
 \begin{center}
 \begin{tabular}{|c|c|}
 \hline
 Thời gian (phút) & Số cuộc gọi \\
 \hline
 $0\le t<1$ & $8$ \\
 \hline
 $1\le t<2$ & $17$ \\
 \hline
 $2\le t<3$ & $25$ \\
 \hline
 $3\le t<4$ & $20$ \\
 \hline
 $4\le t<5$ & $10$ \\
 \hline
 \end{tabular}
 \end{center}
 Tìm khoảng tứ phân vị của mẫu số liệu ghép nhóm trên (làm tròn đến hàng phần mười).
 \par\shortans[]{1{,}8}
 \loigiai{ 
 Ta có bảng mẫu số liệu ghép nhóm được viết lại như sau:
 \begin{center}
 \begin{tabular}{|c|c|c|c|c|c|}
 \hline
 Thời gian t (phút) & $[0;1)$ & $[1; 2)$ & $[2; 3)$ & $[3; 4)$ & $[4; 5)$ \\
 \hline
 Số cuộc gọi & $8$ & $17$ & $25$ & $20$ & $10$ \\
 \hline
 \end{tabular}
 \end{center}
 Có cỡ mẫu $n=8+17+25+20+10=80$.\\
 Giả sử $x_1$, $x_2$, $\ldots$, $x_{80}$ là thời gian đàm thoại của $80$ cuộc gọi được sắp xếp theo thứ tự tăng dần.\\
 Tứ phân vị thứ nhất của mẫu số liệu gốc là $\dfrac{x_{20}+x_{21}}{2} \in [1; 2)$ nên nhóm chứa tứ phân vị thứ nhất là $[1; 2)$.
 \[Q_1=1+\dfrac{\dfrac{80}{4}-8}{17} \left(2-1\right)\approx 1{,}7.\]
 Tứ phân vị thứ ba của mẫu số liệu gốc là $\dfrac{x_{60}+x_{61}}{2} \in [3; 4)$ nên nhóm chứa tứ phân vị thứ ba là $[3; 4)$.
 \[Q_3=3+\dfrac{\dfrac{3\cdot 80}{4}-50}{20} \left(4-3\right)=3{,}5.\]
 Khoảng tứ phân vị của mẫu số liệu ghép nhóm là $\Delta_Q=Q_3-Q_1 \approx 1{,}8$. 
 }
\end{ex}
\Closesolutionfile{ans}
\begin{indapan}
	{ans/ans\currfilebase}
\end{indapan}

