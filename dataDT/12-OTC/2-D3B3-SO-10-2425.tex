\begin{name}
 {Biên soạn: Bùi Văn Lợi \\ Phản biện: HP Minh Nguyen}
 {Đề ôn tập chương III}
\end{name}

\TN
\Opensolutionfile{ans}[ans/ans\currfilebase-Phan-I]

\begin{ex}%[2-D3B3-SO-10-2425]%[VN-MT-7, Bùi Văn Lợi]%[2D3N1-2]
Xét mẫu dữ liệu cho bởi bảng sau:
\begin{center}
\begin{tabular}{|c|c|c|c|c|c|c|}
\hline 
Nhóm & $[14;15)$ & $[15;16)$ & $[16;17)$ & $[17;18)$ & $[18;19)$ & \\ 
\hline 
Tần số & $1$ & $3$ & $8$ & $6$ & $2$ & $n=20$ \\ 
\hline 
\end{tabular} 
\end{center}
Khoảng biến thiên của mẫu số liệu ghép nhóm bằng
\choice
{$3$}
{$4$}
{\True $5$}
{$6$}

\loigiai{
Khoảng biến thiên của mẫu số liệu ghép nhóm là $R=19-14=5$.
}
\end{ex}

\begin{ex}%[2-D3B3-SO-10-2425]%[VN-MT-7, Bùi Văn Lợi]%[2D3N1-1]
Xét mẫu số liệu cho bởi bảng sau:
\begin{center}
\begin{tabular}{|c|c|c|c|c|c|c|}
\hline
Nhóm & $[40;45)$ & $[45;50)$ & $[50;55)$ & $[55;60)$ & $[60;65)$ & \\
\hline
Tần số & $4$ & $11$ & $9$ & $n_4$ & $8$ & $n=40$ \\
\hline
\end{tabular}
\end{center}
Tần số $n_4$ của nhóm $4$ trong mẫu số liệu trên bằng
\choice
{$7$}
{\True $8$}
{$9$}
{$10$}

\loigiai{
Ta có $n_4 = 40-(4+11+9+8)=8$.
}
\end{ex}

\begin{ex}%[2-D3B3-SO-10-2425]%[VN-MT-7, Bùi Văn Lợi]%[2D3H1-4]
Xét mẫu số liệu cho bởi bảng sau:
\begin{center}
\begin{tabular}{|c|c|c|c|c|c|c|c|}
\hline
Nhóm & $[0;2)$ & $[2;4)$ & $[4;6)$ & $[6;8)$ & $[8;10)$ & $[10;12)$ &\\
\hline
Tần số & $3$ & $8$ & $12$ & $10$ & $7$ & $5$ & $n=45$\\
\hline
Tần số tích lũy & $3$ & $11$ & $23$ & $33$ & $40$ & $n_6$ &\\
\hline
\end{tabular}
\end{center}
Tần số tích lũy của nhóm $6$ bằng
\choice
{$40$}
{$42$}
{\True $45$}
{$41$}

\loigiai{
Tần số tích lũy của nhóm $6$ bằng $n_1+n_2+n_3+n_4+n_5+n_6=3+8+12+10+7+5=45$.
}
\end{ex}

\begin{ex}%[2-D3B3-SO-10-2425]%[VN-MT-7, Bùi Văn Lợi]%[1D5H2-3]
Xét mẫu số liệu cho bởi bảng sau:
\begin{center}
\begin{tabular}{|c|c|c|c|c|c|c|c|}
\hline Nhóm & $[40;45)$ & $[45;50)$ & $[50;55)$ & $[55;60)$ & $[60;65)$ & $[65;70)$ & \\
\hline Tần số & $5$ & $10$ & $7$ & $9$ & $7$ & $4$ & $n=42$ \\
\hline Tần số tích lũy & $5$ & $15$ & $22$ & $31$ & $38$ & $42$ & \\
\hline
\end{tabular}
\end{center}
Tứ phân vị thứ nhất của mẫu số liệu trên bằng
\choice
{$47{,}5$}
{\True $47{,}75$}
{$48$}
{$48{,}25$}

\loigiai{
Cỡ mẫu là $n=42$.\\
Gọi $x_1,x_2,\cdots,x_{42}$ là mẫu số liệu gốc được xếp theo thứ tự không giảm.\\
Tứ phân vị thứ nhất của mẫu số liệu gốc là $x_{11} \in [45;50)$.\\
Xét nhóm $[45;50)$ có đầu mút trái $s=45$, độ dài $h=5$, tần số $n_2=10$ và $cf_1=5$.\\
Tứ phân vị thứ nhất là $Q_1 = 45 + \left(\dfrac{10{,}5-5}{10}\right) \cdot 5=47{,}75$.
}
\end{ex}

\begin{ex}%[2-D3B3-SO-10-2425]%[VN-MT-7, Bùi Văn Lợi]%[1D5H2-3]
Xét mẫu số liệu cho bởi bảng sau:
\begin{center}
\begin{tabular}{|c|c|c|c|c|c|c|c|}
\hline
Nhóm & $[0;2)$ & $[2;4)$ & $[4;6)$ & $[6;8)$ & $[8;10)$\\
\hline
Tần số & $3$ & $8$ & $12$ & $12$ & $4$\\
\hline
Tần số tích lũy & $3$ & $11$ & $23$ & $35$ & $39$\\
\hline
\end{tabular}
\end{center}
Tứ phân vị thứ hai của mẫu số liệu trên gần nhất với kết quả nào dưới đây?
\choice
{$5{,}52$}
{\True $5{,}42$}
{$4{,}5$}
{$4{,}75$}

\loigiai{
Cỡ mẫu là $n=39$.\\
Gọi $x_1,x_2,\cdots,x_{39}$ là mẫu số liệu gốc được xếp theo thứ tự không giảm.\\
Tứ phân vị thứ hai của mẫu số liệu gốc là $x_{20} \in [4;6)$.\\
Xét nhóm $[4;6)$ có đầu mút trái $r=11$, độ dài $d=2$, tần số $n_3=12$ và $cf_2=11$.\\
Tứ phân vị thứ hai là $Q_2 =4+\left(\dfrac{19{,}5-11}{12}\right) \cdot 2 = \dfrac{65}{12} \approx 5{,}42$.
}
\end{ex}

\begin{ex}%[2-D3B3-SO-10-2425]%[VN-MT-7, Bùi Văn Lợi]%[1D5H2-3]
Cho mẫu số liệu ghép nhóm về tuổi thọ (đơn vị tính là năm) của một loại bóng đèn mới như sau:
\begin{center}
\begin{tabular}{|c|c|c|c|c|c|c|c|}
\hline
Tuổi thọ & $[2;3{,}5)$ & $[3{,}5;5)$ & $[5;6{,}5)$ & $[6{,}5;8)$\\
\hline
Số bóng đèn & $8$ & $22$ & $35$ & $15$\\
\hline
\end{tabular}
\end{center}
Nhóm chứa tứ phân vị thứ ba của mẫu số liệu là
\choice
{$[2;3{,}5)$}
{$[3{,}5;5)$}
{\True $[5;6{,}5)$}
{$[6{,}5;8)$}

\loigiai{
Lập bảng tần số tích lũy
\begin{center}
\begin{tabular}{|c|c|c|c|c|c|c|c|}
\hline
Nhóm & $[2;3{,}5)$ & $[3{,}5;5)$ & $[5;6{,}5)$ & $[6{,}5;8)$\\
\hline
Tần số & $8$ & $22$ & $35$ & $15$\\
\hline
Tần số tích lũy & $8$ & $30$ & $65$ & $80$\\
\hline
\end{tabular}
\end{center}
Cỡ mẫu là $n=8+22+35+15=80$.\\
Gọi $x_1,x_2,\cdots,x_{80}$ là mẫu số liệu gốc về tuổi thọ của $80$ bóng đèn được xếp theo thứ tự không giảm.\\
Tứ phân vị thứ ba của mẫu số liệu gốc là $\dfrac{x_{60}+x_{61}}{2} \in [5;6{,}5)$.\\
Khi đó nhóm chứa tứ phân vị thứ ba là $[5;6{,}5)$.
}
\end{ex}

\begin{ex}%[2-D3B3-SO-10-2425]%[VN-MT-7, Bùi Văn Lợi]%[2D3H1-3]
Bạn Thu rất thích nhảy hiện đại. Thời gian tập nhảy mỗi ngày trong thời gian gần đây của bạn Thu được thống kê lại ở bảng sau:
\begin{center}
\begin{tabular}{|c|c|c|c|c|c|c|c|}
\hline
Thời gian (phút) & $[20;25)$ & $[25;30)$ & $[30;35)$ & $[35;40)$ & $[40;45)$\\
\hline
Số ngày & $6$ & $6$ & $4$ & $1$ & $1$\\
\hline
\end{tabular}
\end{center}
Khoảng tứ phân vị của mẫu số liệu ghép nhóm là
\choice
{\True $8{,}125$}
{$8{,}5$}
{$13{,}5$}
{$4{,}5$}

\loigiai{
Ta lập bảng tần số tích lũy
\begin{center}
\begin{tabular}{|c|c|c|c|c|c|c|c|}
\hline Nhóm & $[20;25)$ & $[25;30)$ & $[30;35)$ & $[35;40)$ & $[40;45)$\\
\hline Tần số & $6$ & $6$ & $4$ & $1$ & $1$\\
\hline Tần số tích lũy & $6$ & $12$ & $16$ & $17$ & $18$\\
\hline
\end{tabular}
\end{center}
Cỡ mẫu là $n=18$.\\
Gọi $x_1,x_2,\cdots,x_{18}$ là mẫu số liệu gốc được xếp theo thứ tự không giảm.\\
Tứ phân vị thứ nhất của mẫu số liệu gốc là $x_5 \in [20;25)$. Khi đó\\
\[Q_1 = 20+\dfrac{4{,}5}{6} \cdot 5 = 23{,}75.\]
Tứ phân vị thứ nhất của mẫu số liệu gốc là $x_{14} \in [30;35)$. Khi đó\\
\[Q_3 = 30+\dfrac{13{,}5-12}{4} \cdot 5 = 31{,}875.\]
Vậy khoảng tứ phân vị của mẫu số liệu ghép nhóm là $\Delta_Q = Q_3-Q_1 = 8{,}125$.
}
\end{ex}

\begin{ex}%[2-D3B3-SO-10-2425]%[VN-MT-7, Bùi Văn Lợi]%[1D5N1-2]
Điều tra về chiều cao của học sinh khối 12 của một trường THPT, ta có kết quả sau:
\begin{center}
\begin{tabular}{|c|c|c|}
\hline
Nhóm & Chiều cao (cm) & Số học sinh\\
\hline
$1$ & $[160;163)$ & $6$\\
\hline
$2$ & $[163;166)$ & $10$\\
\hline
$3$ & $[166;169)$ & $12$\\
\hline
$4$ & $[169;172)$ & $20$\\
\hline
$5$ & $[172;175)$ & $40$\\
\hline
$6$ & $[175;178)$ & $12$\\
\hline
& & $n=100$\\
\hline
\end{tabular}
\end{center}
Giá trị đại diện của nhóm thứ năm là
\choice
{$173$}
{\True $173{,}5$}
{$174{,}5$}
{$175$}

\loigiai{
Giá trị đại diện của nhóm thứ năm là $\dfrac{172+175}{2}=173{,}5$.
}
\end{ex}

\begin{ex}%[2-D3B3-SO-10-2425]%[VN-MT-7, Bùi Văn Lợi]%[1D5N1-2]
Các bạn học sinh lớp 12A trả lời $40$ câu hỏi trong một bài kiểm tra. Kết quả được thống kê ở bảng sau. Hãy tính độ dài mỗi nhóm.
\begin{center}
\begin{tabular}{|c|c|c|c|c|c|}
\hline
Số câu trả lời đúng & $[20;24)$ & $[24;28)$ & $[28;32)$ & $[32;36)$ & $[36;40)$ \\
\hline
Số học sinh & $4$ & $7$ & $5$ & $10$ & $6$ \\
\hline
\end{tabular}
\end{center}
\choice
{$2$}
{$3$}
{\True $4$}
{$5$}

\loigiai{
Độ dài mỗi nhóm là $4$.
}
\end{ex}

\begin{ex}%[2-D3B3-SO-10-2425]%[VN-MT-7, Bùi Văn Lợi]%[1D5N1-3]
Tìm cân nặng trung bình của học sinh lớp 12A cho trong bảng sau, làm tròn đến hàng phần trăm.
\begin{center}
\begin{tabular}{|c|c|c|c|c|c|c|}
\hline
Cân nặng ($\mathrm{kg}$) & $[40{,}5;45{,}5)$ & $[45{,}5;50{,}5)$ & $[50{,}5;55{,}5)$ & $[55{,}5;60{,}5)$ & $[60{,}5;65{,}5)$ & $[65{,}5;70{,}5)$ \\
\hline
Số học sinh & $8$ & $10$ & $5$ & $12$ & $6$ & $3$ \\
\hline
\end{tabular}
\end{center}
\choice
{$54{,}5$}
{$53{,}31$}
{\True $53{,}79$}
{$54{,}45$}

\loigiai{
Trong mỗi khoảng cân nặng, giá trị đại diện là trung bình cộng của hai giá trị đầu mút nên ta có bảng sau
\begin{center}
\begin{tabular}{|c|c|c|c|c|c|c|}
\hline
Giá trị đại diện & $43$ & $48$ & $53$ & $58$ & $63$ & $68$ \\
\hline
Số học sinh & $8$ & $10$ & $5$ & $12$ & $6$ & $3$ \\
\hline
\end{tabular}
\end{center}
Tổng số học sinh là $n=44$. Cân nặng trung bình của học sinh lớp 12A là
\[\overline{x} = \dfrac{8\cdot 43 +10\cdot 48 +5\cdot 53 +12 \cdot 58+6 \cdot 63+ 3\cdot 68}{44} = 53{,}79.\]
}
\end{ex}

\begin{ex}%[2-D3B3-SO-10-2425]%[VN-MT-7, Bùi Văn Lợi]%[2D3H2-2]
Tìm phương sai của mẫu số liệu ghép nhóm được cho ở bảng sau (làm tròn kết quả đến hàng phần mười).
\begin{center}
\begin{tabular}{|c|c|c|c|c|c|c|}
\hline
Nhóm & $[1{,}5;2)$ & $[2;2{,}5)$ & $[2{,}5;3)$ & $[3;3{,}5)$ & $[3{,}5;4)$\\
\hline
Tần số & $4$ & $9$ & $13$ & $8$ & $6$\\
\hline
\end{tabular}
\end{center}
\choice
{\True $0{,}4$}
{$0{,}7$}
{$1{,}2$}
{$0{,}8$}

\loigiai{
Lập bảng giá trị đại diện cho mẫu số liệu như sau:
\begin{center}
\begin{tabular}{|c|c|c|c|c|c|c|}
\hline
Nhóm & $[1{,}5;2)$ & $[2;2{,}5)$ & $[2{,}5;3)$ & $[3;3{,}5)$ & $[3{,}5;4)$\\
\hline
Giá trị đại diện & $1{,}75$ & $2{,}25$ & $2{,}75$ & $3{,}25$ & $3{,}75$\\
\hline
Tần số & $4$ & $9$ & $13$ & $8$ & $6$\\
\hline
\end{tabular}
\end{center}
Ta có $n=40$, số trung bình cộng của mẫu số liệu ghép nhóm là
\[\overline{x} = \dfrac{4\cdot 1{,}75+9\cdot 2{,}25+13\cdot 2{,}75+8\cdot 3{,}25+6\cdot 3{,}75}{40} = 2{,}7875.\]
Phương sai của mẫu số liệu là
\[s^2 = \dfrac{4\cdot 1{,}75^2+9\cdot 2{,}25^2+13\cdot 2{,}75^2+8\cdot 3{,}25^2+6\cdot 3{,}75^2}{40} - (2{,}7875)^2 \approx 0{,}4.\]
}
\end{ex}

\begin{ex}%[2-D3B3-SO-10-2425]%[VN-MT-7, Bùi Văn Lợi]%[2D3H2-2]
Tìm độ lệch chuẩn của mẫu số liệu ghép nhóm được cho ở bảng sau (làm tròn kết quả đến hàng phần mười).
\begin{center}
\begin{tabular}{|c|c|c|c|c|c|c|}
\hline
Nhóm & $[25;35)$ & $[35;45)$ & $[45;55)$ & $[55;65)$ & $[65;75)$ & \\
\hline
Tần số & $9$ & $7$ & $5$ & $10$ & $9$ & $n=40$ \\
\hline
\end{tabular}
\end{center}
\choice
{$15{,}1$}
{$15{,}0$}
{$14{,}8$}
{\True $14{,}9$}

\loigiai{
Ta có bảng thống kê sau:
\begin{center}
\begin{tabular}{|c|c|c|c|c|c|c|}
\hline Nhóm & $[25;35)$ & $[35;45)$ & $[45;55)$ & $[55;65)$ & $[65;75)$ & \\
\hline Giá trị đại diện & $30$ & $40$ & $50$ & $60$ & $70$ &\\
\hline Tần số & $9$ & $7$ & $5$ & $10$ & $9$ & $n=40$ \\
\hline
\end{tabular}
\end{center}
Số trung bình cộng của mẫu số liệu ghép nhóm là
\[
\overline{x} = \dfrac{30\cdot 9+40\cdot 7+50\cdot 5+60\cdot 10+70\cdot 9}{40} = 50{,}75.
\]
Phương sai của mẫu số liệu là
\[
s^2 = \dfrac{30^2\cdot 9+40^2\cdot 7+50^2\cdot 5+60^2\cdot 10+70^2\cdot 9}{40} - (50{,}75)^2=221{,}9375.
\]
Vậy độ lệch chuẩn của mẫu số liệu trên là $s = \sqrt{221{,}9375} \approx 14{,}9$.
}
\end{ex}
\Closesolutionfile{ans}

\TNTF
\Opensolutionfile{ans}[ans/ans\currfilebase-Phan-II]

\begin{ex}%[2-D3B3-SO-10-2425]%[VN-MT-7, Bùi Văn Lợi]%[2D3H1-4]
Cho bảng số liệu sau:
\begin{center}
\begin{tabular}{|c|c|c|c|c|c|c|}
\hline
Nhóm & $[40;45)$ & $[45;50)$ & $[50;55)$ & $[55;60)$ & $[60;65)$ & \\
\hline
Tần số & $9$ & $5$ & $5$ & $4$ & $7$ & $n=30$ \\
\hline
\end{tabular}
\end{center}
\choiceTF
{Khoảng biến thiên của mẫu số liệu ghép trên là $R = 65$}
{\True Tần số của nhóm $5$ là $7$}
{Tần số tích lũy của nhóm $3$ là $15$}
{\True Tần số tích lũy của nhóm $5$ hơn nhóm $2$ là $16$}

\loigiai{
\begin{itemchoice}
\itemch \textbf{Sai}.\\
Khoảng biến thiên của mẫu số liệu ghép nhóm trên là $R=65-40=25$.
\itemch \textbf{Đúng}.\\
Tần số của nhóm $5$ là $7$.
\itemch \textbf{Sai}.\\
Tần số tích lũy của nhóm $3$ là $cf_3 = n_1+n_2+n_3 = 9+5+5 = 19$.
\itemch \textbf{Đúng}.\\
Tần số tích lũy của nhóm $2$ là $cf_2 = n_1+n_2 = 9+5 = 14$.\\
Tần số tích lũy của nhóm $5$ là $cf_5 = n_1+n_2+n_3+n_4+n_5 = 9+5+5+4+7 = 30$.\\
Vậy tần số tích lũy của nhóm $5$ hơn tần số tích lũy của nhóm $2$ là $16$.
\end{itemchoice}
}
\end{ex}

\begin{ex}%[2-D3B3-SO-10-2425]%[VN-MT-7, Bùi Văn Lợi]%[2D3H1-3]
Cho bảng số liệu sau:
\begin{center}
\begin{tabular}{|c|c|c|c|c|c|c|}
\hline Nhóm & $[6;8)$ & $[8;10)$ & $[10;12)$ & $[12;14)$ & $[14;16)$ & \\
\hline Tần số & $3$ & $5$ & $8$ & $10$ & $4$ & $n=30$ \\
\hline
\end{tabular}
\end{center}
\choiceTF
{Tứ phân vị thứ nhất của mẫu số liệu trên là $Q_1 = 9$}
{\True Tứ phân vị thứ hai của mẫu số liệu có giá trị nhỏ hơn $12$}
{\True Tứ phân vị thứ ba của mẫu số liệu có giá trị nằm trong khoảng $[12;14)$}
{Khoảng tứ phân vị của mẫu số liệu ghép nhóm trên là $\Delta_Q = 3{,}75$}

\loigiai{
Ta có bảng số liệu ghép nhóm:
\begin{center}
\begin{tabular}{|c|c|c|c|c|c|c|}
\hline Nhóm & $[6;8)$ & $[8;10)$ & $[10;12)$ & $[12;14)$ & $[14;16)$ & \\
\hline Tần số & $3$ & $5$ & $8$ & $10$ & $4$ & $n=30$ \\
\hline Tần số tích lũy & $3$ & $8$ & $16$ & $26$ & $30$ & \\
\hline
\end{tabular}
\end{center}
\begin{itemchoice}
\itemch \textbf{Sai}.\\
Ta có $\dfrac{n}{4} = 7{,}5$ mà $3<7{,}5<8$ nên nhóm $2$ là nhóm đầu tiên có tần số tích lũy lớn hơn hoặc bằng $7{,}5$.\\
Xét nhóm $[8;10)$ có đầu mút trái $s=8$, độ dài $h=2$, tần số $n_2 = 5$ và $cf_1 =3$.\\
Vậy, tứ phân vị thứ nhất là
$Q_1 = s+\left(\dfrac{7{,}5-cf_1}{n_2}\right) \cdot h = 8+\left(\dfrac{7{,}5-3}{5}\right) \cdot 2 = 9{,}8$.
\itemch \textbf{Đúng}.\\
Ta có $\dfrac{n}{2} =15$ mà $8<15<16$ nên nhóm $3$ là nhóm đầu tiên có tần số tích lũy lớn hơn hoặc bằng $15$.\\
Xét nhóm $[10;12)$ có đầu mút trái $r=10$, độ dài $d=2$, tần số $n_3 = 8$ và $cf_2 =8$.\\
Vậy, tứ phân vị thứ hai là
$Q_2 = r+\left(\dfrac{15-cf_2}{n_3}\right) \cdot d = 10+\left(\dfrac{15-8}{8}\right) \cdot 2 = 11{,}75 < 12$.
\itemch \textbf{Đúng}.\\
Ta có $\dfrac{3n}{4} =22{,}5$ mà $16<22{,}5<26$ nên nhóm $4$ là nhóm đầu tiên có tần số tích lũy lớn hơn hoặc bằng $22{,}5$.\\
Xét nhóm $[12;14)$ có đầu mút trái $t=12$, độ dài $l=2$, tần số $n_4 = 10$ và $cf_3 =16$.\\
Vậy, tứ phân vị thứ ba là
$Q_3 = t+\left(\dfrac{22{,}5-cf_3}{n_4}\right) \cdot l = 12+\left(\dfrac{22{,}5-16}{10}\right) \cdot 2 = 13{,}3$.
\itemch \textbf{Sai}.\\
Khoảng tứ phân vị của mẫu số liệu ghép nhóm trên là $\Delta_Q = Q_3-Q_1 = 3{,}5$.
\end{itemchoice}
}
\end{ex}

\begin{ex}%[2-D3B3-SO-10-2425]%[VN-MT-7, Bùi Văn Lợi]%[1D5H1-3]
Một mẫu số liệu được cho ở dạng bảng tần số ghép nhóm như sau:
\begin{center}
\begin{tabular}{|c|c|c|c|c|c|}
\hline
Nhóm & $[0{,}5;2{,}5)$ & $[2{,}5;4{,}5)$ & $[4{,}5;6{,}5)$ & $[6{,}5;8{,}5)$ & $[8{,}5;10{,}5)$ \\ 
\hline
Tần số & $4$ & $7$ & $16$ & $8$ & $5$ \\
\hline
\end{tabular}
\end{center}
\choiceTF
{\True Nhóm $[4{,}5;6{,}5)$ có giá trị đại diện là $5{,}5$}
{Nhóm $[8{,}5;10{,}5)$ có giá trị đại diện là $9$}
{\True Các nhóm trong mẫu số liệu đều độ dài bằng nhau}
{Số trung bình của mẫu số liệu trên lớn hơn $5{,}5$ (làm tròn kết quả đến hàng phần trăm)}

\loigiai{
Cỡ mẫu của số liệu là $n = 4+7+16+8+5 = 40$.\\
Bảng sau cho biết giá trị đại diện và độ dài của mỗi nhóm
\begin{center}
\begin{tabular}{|c|c|c|c|c|c|}
\hline
Nhóm & $[0{,}5;2{,}5)$ & $[2{,}5;4{,}5)$ & $[4{,}5;6{,}5)$ & $[6{,}5;8{,}5)$ & $[8{,}5;10{,}5)$ \\
\hline
Giá trị đại diện & $1{,}5$ & $3{,}5$ & $5{,}5$ & $7{,}5$ & $9{,}5$ \\
\hline
Độ dài nhóm & $2$ & $2$ & $2$ & $2$ & $2$ \\
\hline
\end{tabular}
\end{center}
\begin{itemchoice}
\itemch \textbf{Đúng}.\\
Nhóm $[4{,}5;6{,}5)$ có giá trị đại diện là $5{,}5$.
\itemch \textbf{Sai}.\\
Nhóm $[8{,}5;10{,}5)$ có giá trị đại diện là $9{,}5$.
\itemch \textbf{Đúng}.\\
Các nhóm trong mẫu số liệu có độ dài bằng nhau.
\itemch \textbf{Sai}.\\
Số trung bình cộng của mẫu số liệu ghép nhóm là
\[ \overline{x} = \dfrac{4\cdot 1{,}5+7\cdot 3{,}5+16\cdot 5{,}5+8\cdot 7{,}5+5\cdot 9{,}5}{40} = \dfrac{113}{20} = 5{,}65.\]
\end{itemchoice}
}
\end{ex}

\begin{ex}%[2-D3B3-SO-10-2425]%[VN-MT-7, Bùi Văn Lợi]%[2D3H2-2]
Mỗi ngày bác Nam đều đi bộ để rèn luyện sức khỏe. Quãng đường đi bộ mỗi ngày (đơn vị: km) của bác Nam trong $25$ ngày được thống kê ở bảng sau:
\begin{center}
\begin{tabular}{|c|c|c|c|c|c|c|}
\hline
Quãng đường (km) & $[1{,}5;2{,}5)$ & $[2{,}5;3{,}5)$ & $[3{,}5;4{,}5)$ & $[4{,}5;5{,}5)$ & $[5{,}5;6{,}5)$ & \\
\hline
Số ngày & $4$ & $2$ & $10$ & $7$ & $2$ & $n=25$\\
\hline
\end{tabular}
\end{center}
Các mệnh đề sau đúng hay sai? Kết quả làm tròn đến hàng phần trăm.
\choiceTF
{\True Độ dài nhóm của mẫu số liệu trên là $1$}
{\True Số trung bình của mẫu số liệu trên là $4{,}04$}
{Phương sai của mẫu số liệu trên là $1{,}6$}
{\True Độ lệch chuẩn của mẫu số liệu trên là $1{,}15$}

\loigiai{
Ta có bảng sau:
\begin{center}
\begin{tabular}{|c|c|c|c|c|c|}
\hline
Nhóm & $[1{,}5;2{,}5)$ & $[2{,}5;3{,}5)$ & $[3{,}5;4{,}5)$ & $[4{,}5;5{,}5)$ & $[5{,}5;6{,}5)$\\
\hline
Giá trị đại diện & $2$ & $3$ & $4$ & $5$ & $6$\\
\hline
Tần số & $4$ & $2$ & $10$ & $7$ & $2$\\
\hline
\end{tabular}
\end{center}
\begin{itemchoice}
\itemch \textbf{Đúng}.\\
Độ dài nhóm của mẫu số liệu là $1$.
\itemch \textbf{Đúng}.\\
Số trung bình của mẫu số liệu là
\[\overline{x} = \dfrac{2\cdot 4+3\cdot 2 +4\cdot 10+5\cdot 7+6\cdot 2}{25} = 4{,}04.\]
\itemch \textbf{Sai}.\\
Phương sai của mẫu số liệu là
\[
s^2 = \dfrac{(2-4{,}04)^2 \cdot 4+ (3-4{,}04)^2 \cdot 2 + (4-4{,}04)^2\cdot 10+(5-4{,}04)^2\cdot 7+(6-4{,}04)^2\cdot 2}{25} \approx 1{,}32.
\]
\itemch \textbf{Đúng}.\\
Độ lệch chuẩn của mẫu số liệu là $s = \sqrt{s^2} \approx 1{,}15$.
\end{itemchoice}
}
\end{ex}
\Closesolutionfile{ans}

\TNSA
\Opensolutionfile{ans}[ans/ans\currfilebase-Phan-III]

\begin{ex}%[2-D3B3-SO-10-2425]%[VN-MT-7, Bùi Văn Lợi]%[2D3N1-2]
Kết quả đo chiều cao của $250$ cây dừa đột biến $3$ năm tuổi ở một viện nghiên cứu được tổng hợp ở bảng sau:
\begin{center}
\begin{tabular}{|c|c|c|c|c|c|}
\hline 
Chiều cao (m$^2$) & {$[8{,}5 ; 8{,}8)$} & {$[8{,}8 ; 9{,}1)$} & {$[9{,}1 ; 9{,}4)$} & {$[9{,}4 ; 9{,}7)$} & {$[9{,}7 ; 10)$} \\
\hline Số cây & $36$ & $45$ & $83$ & $65$ & $21$ \\
\hline
\end{tabular}
\end{center}
Tìm khoảng biến thiên của mẫu số liệu ghép nhóm trên.

\shortans[]{1{,}5}

\loigiai{
Khoảng biến thiên của mẫu số liệu ghép nhóm là
$R=10-8{,}5=1{,}5$.
}
\end{ex}

\begin{ex}%[2-D3B3-SO-10-2425]%[VN-MT-7, Bùi Văn Lợi]%[2D3H1-4]
Một cộng ty bất động sản Đất Vàng thực hiện cuộc khảo sát khách hàng xem họ có
nhu cầu mua nhà ở mức giá nào đã tiến hành dự án xây nhà ở Thăng Long group sắp tới. Kết quả khảo sát $500$ khách hàng được ghi lại ở bảng sau:
\begin{center}
\begin{tabular}{|l|c|c|c|c|c|}
\hline 
Mức giá (triệu đồng) & $[10 ; 14)$ & $[14 ; 18)$ & $[18 ; 22)$ & $[22 ; 26)$ & $[26 ; 30)$ \\
\hline Số khách hàng & $75$ & $105$ & $179$ & $96$ & $45$ \\
\hline
\end{tabular}
\end{center}
Tìm tần số tích lũy của nhóm $[18;22)$.

\shortans[]{359}

\loigiai{
Tần số tích lũy của nhóm $[18;22)$ là $cf_3 = 75+105+179 = 359$.
}
\end{ex}

\begin{ex}%[2-D3B3-SO-10-2425]%[VN-MT-7, Bùi Văn Lợi]%[1D5H2-3]
Bảng dưới đây biểu diễn mẫu số liệu ghép nhóm về chiều cao (đơn vị: cm) của $43$ học sinh trong một lớp học khối $11$ của một trường phổ thông.
\begin{center}
\begin{tabular}{|c|c|c|c|c|c|c|c|}
\hline 
Nhóm & $[150 ; 155)$ & $[155 ; 160)$ & $[160 ; 165)$ & $[165 ; 170)$ & $[170 ; 175)$ & $[175 ; 180)$ & \\
\hline 
Tần số & $5$ & $10$ & $12$ & $9$ & $4$ & $3$ & $n=43$ \\ 
\hline 
\end{tabular} 
\end{center}
Tứ phân vị thứ hai của mẫu số liệu ghép nhóm trên bằng bao nhiêu (kết quả làm tròn đến hàng đơn vị)?

\shortans[]{163}

\loigiai{
Ta có bảng sau:
\begin{center}
\begin{tabular}{|c|c|c|}
\hline Nhóm & Tần số & Tần số tích lũy\\
\hline$[150 ; 155)$ & $5$ & $5$\\
\hline$[155 ; 160)$ & $10$& $15$\\
\hline$[160 ; 165)$ & $12$& $27$\\
\hline$[165 ; 170)$ & $9$ & $36$\\
\hline$[170 ; 175)$ & $4$ & $40$\\
\hline$[175 ; 180)$ & $3$ & $43$\\
\hline & $n=43$ & \\
\hline
\end{tabular}
\end{center}
Số phần tử của mẫu là $n=43$.\\
Ta có $\dfrac{n}{2}=21{,}5$ mà $15<21{,}5<27$ nên nhóm $3$ là nhóm đầu tiên có tần số tích lũy lớn hơn hoặc bằng $21{,}5$.\\
Xét nhóm $3$ là nhóm $[160;165)$ có đầu mút trái $r=160$, độ dài $d=5$, tần số $n_3=12$ và $cf_2=15$.\\
Từ đó ta có tứ phân vị thứ $2$ là
\[Q_2=160+\left(\dfrac{21{,}5-15}{12}\right)\cdot 5 \approx 163~\text{(cm)}.\]
}
\end{ex}

\begin{ex}%[2-D3B3-SO-10-2425]%[VN-MT-7, Bùi Văn Lợi]%[2D3H1-3] 
Bảng sau thống kê khối lượng (đơn vị: gam) một số quả măng cụt được lựa chọn ngẫu nhiên trong một thùng hàng
\begin{center}
\begin{tabular}{|c|c|c|c|c|c|c|}
\hline 
Nhóm & $[80 ; 82)$ & $[82 ; 84)$ & $[84 ; 86)$ & $[86 ; 88)$ & $[88 ; 90)$ & \\ 
\hline 
Tần số & $17$ & $22$ & $26$ & $19$ & $16$ & $n=100$ \\ 
\hline 
\end{tabular} 
\end{center}
Tính khoảng tứ phân vị của mẫu số liệu ghép nhóm trên (kết quả làm tròn đến hàng phần mười).

\shortans[]{4{,}3}

\loigiai{
Ta có bảng sau:
\begin{center}
\begin{tabular}{|c|c|c|}
\hline Nhóm & Tần số & Tần số tích lũy\\
\hline$[80 ; 82)$ & $17$ & $17$\\
\hline$[82 ; 84)$ & $22$ & $39$\\
\hline$[84 ; 86)$ & $26$ & $65$\\
\hline$[86 ; 88)$ & $19$ & $84$\\
\hline$[88 ; 90)$ & $16$ & $100$\\
\hline & $n=100$ & \\
\hline
\end{tabular}
\end{center}
Số phần tử của mẫu là $n=100$.\\
Ta có $\dfrac{n}{4}=25$ mà $17<25<39$.\\ 
Suy ra nhóm $2$ là nhóm đầu tiên có tần số tích lũy lớn hơn hoặc bằng $25$.\\ 
Xét nhóm $2$ là nhóm $[82 ; 84)$ có đầu mút trái $s=82$, độ dài $h=2$, tần số $n_2=22$ và $cf_1=17$.\\
Từ đó ta có tứ phân vị thứ nhất là 
\[Q_1=82+\left(\dfrac{25-17}{22}\right)\cdot 2 = \dfrac{910}{11}.\]
Tương tự, ta có $\frac{3n}{4}=75$ mà $65<75<84$.
Suy ra nhóm $4$ là nhóm có tần số tích lũy lớn hơn hoặc bằng $75$.\\
Khi đó tứ phân vị thứ ba là 
\[Q_3=86+ \left(\dfrac{75-65}{19}\right) \cdot 2 = \dfrac{1654}{19}.\]
Vậy khoảng tứ phân vị của mẫu số liệu ghép nhóm đã cho là 
\[\Delta_Q=Q_3-Q_1 =\dfrac{1654}{19}-\dfrac{910}{11} \approx 4{,}3~\text{(gam)}.\]
}
\end{ex}

\begin{ex}%[2-D3B3-SO-10-2425]%[VN-MT-7, Bùi Văn Lợi]%[1D5H1-3]
Mẫu số liệu cân nặng (đơn vị: gam) của $30$ trái xoài trong một thùng xoài chuẩn bị đem ra thị trường được biểu thị ở bảng sau:
\begin{center}
\begin{tabular}{|c|c|c|c|c|c|}
\hline
Cân nặng (g) & $[400;480)$ & $[480;560)$ & $[560;640)$ & $[640;720)$ & $[720;800)$ \\
\hline
Số quả xoài & $5$ & $3$ & $13$ & $7$ & $2$ \\
\hline
\end{tabular}
\end{center}
Cân nặng trung bình của mỗi quả xoài có dạng $\dfrac{a}{b}$ (gam) với $\dfrac{a}{b}$ là phân số tối giản ($a$, $b \in \mathbb{N}$). Khi đó giá trị của biểu thức $M = 2a-3b$ là bao nhiêu?

\shortans[]{3559}

\loigiai{
Ta có bảng sau:
\begin{center}
\begin{tabular}{|c|c|c|c|c|c|}
\hline
Nhóm & $[400;480)$ & $[480;560)$ & $[560;640)$ & $[640;720)$ & $[720;800)$ \\
\hline
Giá trị đại diện & $440$ & $520$ & $600$ & $680$ & $760$ \\
\hline
Tần số & $5$ & $3$ & $13$ & $7$ & $2$ \\
\hline
\end{tabular}
\end{center}
Số trung bình cộng của mẫu số liệu ghép nhóm trên là
\[
\overline{x}=\dfrac{5 \cdot 440+3\cdot 520+13 \cdot 600+7\cdot 680+2 \cdot 760}{30} = \dfrac{1784}{3}~\text{(gam)}.
\]
Khi đó $\dfrac{a}{b} = \dfrac{1784}{3}$ nên $M =2a-3b=3559$.
}
\end{ex}

\begin{ex}%[2-D3B3-SO-10-2425]%[VN-MT-7, Bùi Văn Lợi]%[2D3H2-2] 
Kiểm tra khối lượng của $40$ bao xi măng (đơn vị: kg) được chọn ngẫu nhiêm trước khi xuất xưởng, ta được mẫu số liệu ghép nhóm sau:
\begin{center}
\begin{tabular}{|c|c|c|c|c|c|c|}
\hline
Khối lượng (kg) & $[48{,}5;49)$ & $[49;49{,}5)$ & $[49{,}5;50)$ & $[50;50{,}5)$ & $[50{,}5;51)$ & $[51;51{,}5)$ \\
\hline
Số bao xi măng & $7$ & $3$ & $8$ & $6$ & $7$ & $9$ \\
\hline
\end{tabular}
\end{center}
Phương sai của mẫu số liệu ghép nhóm trên bằng bao nhiêu (kết quả làm tròn đến hàng phần trăm)?

\shortans[]{0{,}77}

\loigiai{
Ta có bảng sau:
\begin{center}
\begin{tabular}{|c|c|c|c|c|c|c|}
\hline
Nhóm & $[48{,}5;49)$ & $[49;49{,}5)$ & $[49{,}5;50)$ & $[50;50{,}5)$ & $[50{,}5;51)$ & $[51;51{,}5)$ \\
\hline
Giá trị đại diện & $48{,}75$ & $49{,}25$ & $49{,}75$ & $50{,}25$ & $50{,}75$ & $51{,}25$ \\
\hline
Tần số & $7$ & $3$ & $8$ & $6$ & $7$ & $9$ \\
\hline
\end{tabular}
\end{center}
Số trung bình cộng của mẫu số liệu ghép nhóm trên là
\[
\overline{x}=\dfrac{7\cdot 48{,}75+3\cdot 49{,}25+8\cdot 49{,}75+6\cdot 50{,}25+7\cdot 50{,}75+9\cdot 51{,}25}{40} = 50{,}125~\text{(kg)}.
\]
Phương sai của mẫu số liệu ghép nhóm trên là
\[
s^2=\dfrac{7\cdot 48{,}75^2+3\cdot 49{,}25^2+8\cdot 49{,}75^2+6\cdot 50{,}25^2+7\cdot 50{,}75^2+9\cdot 51{,}25^2}{40} - (50{,}125)^2 \approx 0{,}77.
\]
}
\end{ex}
\Closesolutionfile{ans}

\begin{indapan}
	{ans/ans\currfilebase}
\end{indapan}

