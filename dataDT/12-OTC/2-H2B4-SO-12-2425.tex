\begin{name}
 {Biên soạn: Vũ Hồng Toàn \\ Phản biện: Lê Văn Hiếu}
{Đề ôn tập chương II}
\end{name}

\TN
\Opensolutionfile{ans}[ans/ans\currfilebase-Phan-I]

\begin{ex}%[2-H2B4-SO-12-2425 (Nguồn Đề 3 - Bài 4)]%[VN-MT-7, Vũ Hồng Toàn]%[2H2H1-2]
Cho hình chóp $ S.ABCD$ có đáy là hình bình hành tâm $O$. Đẳng thức nào sau đây \textbf{sai}?
\choice
{$\overrightarrow{BC}+\overrightarrow{DA}=\overrightarrow{BA}+\overrightarrow{DC}$}
{$\overrightarrow{SA}+\overrightarrow{SC}=\overrightarrow{SB}+\overrightarrow{SD}$}
{$\overrightarrow{SA}+\overrightarrow{SB}+\overrightarrow{SC}+\overrightarrow{SD}=4\overrightarrow{SO}$}
{\True $\overrightarrow{AB}+\overrightarrow{CD}=\overrightarrow{AC}+\overrightarrow{BD}$}
\loigiai{
\begin{center}
\begin{tikzpicture}[scale=1, font=\footnotesize, line join=round, line cap=round, >=stealth]
\def\a{4} \def\b{2.5} \def\h{4.5} \def\g{30}
\path
(0,0) coordinate (A)
(0:\a) coordinate (D)++
(\g-180:\b) coordinate (C)++(180:\a) coordinate (B)
($(A)!.5!(C)$) coordinate (O)++ (90:\h) coordinate (S)
;
\draw[dashed](B)--(A)--(D)--cycle (A)--(C);
\draw[->,dashed] (S)--(O);
\draw[->,dashed] (S)--(A);
\draw[->] (S)--(B);
\draw[->] (S)--(C);
\draw[->] (S)--(D);
\draw (B)--(C)--(D);
\foreach \x /\gN in {A/160,B/180,C/-20,S/90,O/-90,D/0}
\fill(\x) circle (1pt)($(\x)+(\gN:3mm)$) node {$\x$};
\end{tikzpicture}
\end{center}
Với $O$ là trung điểm của $AC$, ta có $\overrightarrow{SA}+\overrightarrow{SC}=2\overrightarrow{SO}$.\\
Với $O$ là trung điểm của $BD$, ta có $\overrightarrow{SB}+\overrightarrow{SD}=2\overrightarrow{SO}$.\\
Từ đó suy ra
\begin{itemize}
\item $\overrightarrow{SA}+\overrightarrow{SC}=\overrightarrow{SB}+\overrightarrow{SD}$.
\item $\overrightarrow{SA}+\overrightarrow{SB}+\overrightarrow{SC}+\overrightarrow{SD}=4\overrightarrow{SO}$.
\item
$\overrightarrow{BC}+\overrightarrow{DA}=\overrightarrow{BA}+\overrightarrow{AC}+\overrightarrow{DC}+\overrightarrow{CA}=\overrightarrow{BA}+\overrightarrow{DC}+\left(\overrightarrow{AC}+\overrightarrow{CA}\right)=\overrightarrow{BA}+\overrightarrow{DC}$.
\item $\overrightarrow{AB}+\overrightarrow{CD}=\overrightarrow{AC}+\overrightarrow{CB}+\overrightarrow{CB}+\overrightarrow{BD}=\overrightarrow{AC}+\overrightarrow{BD}+2\overrightarrow{CB}$.
\end{itemize}
}
\end{ex}


\begin{ex}%[2-H2B4-SO-12-2425 (Nguồn Đề 3 - Bài 4)]%[VN-MT-7, Vũ Hồng Toàn]%[2H2H1-3]
Cho hình lập phương $ABCD.A_1B_1C_1D_1$ có cạnh $a$. Gọi $M$ là trung điểm $AD$. Giá trị $\overrightarrow{B_1M}\cdot\overrightarrow{BD_1}$ bằng
\choice
{$a^2$}
{\True $\dfrac{1}{2} a^2$}
{$\dfrac{3}{2} a^2$}
{$\dfrac{3}{4} a^2$}
\loigiai{
\begin{center}
\begin{tikzpicture}[scale=1, font=\footnotesize, line join=round, line cap=round, >=stealth]
\def\a{4} \def\b{2.5} \def\h{3.5} \def\g{30}
\path
(0,0) coordinate (A)
(0:\a) coordinate (D)++
(\g-180:\b) coordinate (C)++(180:\a) coordinate (B)
(90:\h) coordinate (A_1) ++(0:\a) coordinate (D_1)++
(\g-180:\b) coordinate (C_1)++(180:\a) coordinate (B_1)
($(A)!.5!(D)$) coordinate (M)
;
\draw[dashed](B)--(A)--(D)--cycle (A_1)--(A)--(C);
\draw[dashed,->] (B_1)--(M);
\draw[dashed,->] (B)--(D_1);
\draw (A_1)--(B_1)--(C_1)--(D_1) --cycle (B_1)--(B)--(C)--(C_1)--(A_1) (C)--(D)--(D_1)--(B_1) ;
\foreach \x /\gN in {A/40,B/180,C/-20,M/60,A_1/90,D_1/0,C_1/90,B_1/180,D/0}
\fill(\x) circle (1pt)($(\x)+(\gN:3mm)$) node {$\x$};
\end{tikzpicture}
\end{center}
Áp dụng quy tắc cộng ta có
\begin{itemize}
 \item $\overrightarrow{B_1M}=\overrightarrow{B_1B}+\overrightarrow{BA }+\overrightarrow{AM}$;
 \item $\overrightarrow{BD_1}=\overrightarrow{BA}+\overrightarrow{AD}+\overrightarrow{DD_1}$.
\end{itemize}
Khi đó
\allowdisplaybreaks
\begin{eqnarray*}
\overrightarrow{B_1M}\cdot \overrightarrow{BD_1}&=&\left(\overrightarrow{B_1 B}+\overrightarrow{BA}+\overrightarrow{AM}\right)\cdot\left(\overrightarrow{BA}+\overrightarrow{AD}+\overrightarrow{DD_1}\right)\\
&=&\overrightarrow{B_1B}\cdot\overrightarrow{BA}+\overrightarrow{B_1B}\cdot\overrightarrow{AD}+\overrightarrow{B_1B}\cdot\overrightarrow{DD_1}\\&&+{\overrightarrow{BA}}^2+\overrightarrow{BA}\cdot \overrightarrow{AD}+\overrightarrow{BA}\cdot \overrightarrow{DD_1}\\&&+\overrightarrow{AM}\cdot\overrightarrow{BA}+\overrightarrow{AM}\cdot\overrightarrow{AD}+\overrightarrow{AM}\cdot\overrightarrow{DD_1}\\
&=&\overrightarrow{B_1B}\cdot\overrightarrow{DD_1}+{\overrightarrow{BA}}^2+\overrightarrow{AM}\cdot\overrightarrow{AD}\\
&=&-{\overrightarrow{B_1B}}^2+{\overrightarrow{BA}}^2+\dfrac{1}{2}{\overrightarrow{AD}}^2\\
&=&-a^2+a^2+\dfrac{a^2}{2} \\
&=&\dfrac{a^2}{2}.
\end{eqnarray*}
}
\end{ex}


\begin{ex}%[2-H2B4-SO-12-2425 (Nguồn Đề 3 - Bài 4)]%[VN-MT-7, Vũ Hồng Toàn]%[2H2N2-3]
Trong KG $Oxyz$, cho điểm $A(3;-1;5)$. Tọa độ của vectơ $\overrightarrow{OA}$ là
\choice
{$(3;1;5)$}
{\True $(3;-1;5)$}
{$(-3;-1;5)$}
{$(-3;1;-5)$}
\loigiai{Ta có
$A(3;-1;5)\Rightarrow \overrightarrow{OA}=(3;-1;5)$.
}
\end{ex}


\begin{ex}%[2-H2B4-SO-12-2425 (Nguồn Đề 3 - Bài 4)]%[VN-MT-7, Vũ Hồng Toàn]%[2H2N2-3]
Trong KG $Oxyz$, cho hai điểm $M\left(\dfrac{1}{2};1;-3\right)$ và $N\left(\dfrac{1}{2};-2;4\right)$. Tọa độ của vectơ $\overrightarrow{MN}$ là
\choice
{$(1;-1;1)$}
{\True $(0;-3;7)$}
{$(0;3;-7)$}
{$\left(\dfrac{1}{2};-\dfrac{1}{2};\dfrac{1}{2} \right)$}
\loigiai{
Ta có $\overrightarrow{MN}=\left(\dfrac{1}{2}-\dfrac{1}{2};(-2)-1;4-(-3)\right)=(0;-3;7)$.
}
\end{ex}


\begin{ex}%[2-H2B4-SO-12-2425 (Nguồn Đề 3 - Bài 4)]%[VN-MT-7, Vũ Hồng Toàn]%[2H2H2-3]
Trong KG $Oxyz$, cho các vectơ $\overrightarrow{a}=(1;2;3)$, $\overrightarrow{b}=(2;1;-3)$ và $\overrightarrow{c}=(-1;1;5)$. Vectơ $\overrightarrow{x}=\overrightarrow{a}-4\overrightarrow{b}+2\overrightarrow{c}$ có tọa độ là
\choice
{$\overrightarrow{x}=(9;0;25)$}
{$\overrightarrow{x}=(-9;0;-25)$}
{$\overrightarrow{x}=(9;0;5)$}
{\True $\overrightarrow{x}=(-9;0;25)$}
\loigiai{
Ta có
\begin{itemize}
\item $-4\overrightarrow{b}=(-8;-4;12)$.
\item $2\overrightarrow{c}=(-2;2;10)$.
\end{itemize}
Vậy $ \overrightarrow{x}=\overrightarrow{a}-4\overrightarrow{b}+2\overrightarrow{c}=(-9;0;25) $.
}
\end{ex}


\begin{ex}%[2-H2B4-SO-12-2425 (Nguồn Đề 3 - Bài 4)]%[VN-MT-7, Vũ Hồng Toàn]%[2H2N2-3]
Trong không gian với hệ toạ độ $Oxyz$, cho hai điểm $A(0;-1;2)$ và $B(1;-2;3)$. Toạ độ của vectơ $3\overrightarrow{AB}$ là
\choice
{$(3;3;3)$}
{\True $(3;-3;3)$}
{$(-3;3;3)$}
{$(-3;-3;3)$}
\loigiai{
Ta có $\overrightarrow{AB}=(1;-1;1)$.\\
Vậy $3\overrightarrow{AB}=(3;-3;3)$.
}
\end{ex}


\begin{ex}%[2-H2B4-SO-12-2425 (Nguồn Đề 3 - Bài 4)]%[VN-MT-7, Vũ Hồng Toàn]%[2H2H2-3]
Trong KG $Oxyz$, cho hai vectơ $\overrightarrow{u}=(3;-1;1)$ và $\overrightarrow{v}=(1;2;-2)$. Độ dài của vectơ $\overrightarrow{u}+\overrightarrow{v}$ là
\choice
{$\sqrt{10}$}
{$\sqrt{11}+3$}
{\True $3\sqrt{2}$}
{$5$}
\loigiai{
Có $\overrightarrow{u}+\overrightarrow{v}=(4;1;-1)$.\\
Độ dài của vectơ $\overrightarrow{u}+\overrightarrow{v}$ là $\left|\overrightarrow{u}+\overrightarrow{v}\right|=\sqrt{4^2+1^2+(-1)^2}=3\sqrt{2}$.
}
\end{ex}


\begin{ex}%[2-H2B4-SO-12-2425 (Nguồn Đề 3 - Bài 4)]%[VN-MT-7, Vũ Hồng Toàn]%[2H2H2-4]
Trong KG $Oxyz$, cho ba điểm $A(-2; 1; 0)$, $B(0;-2; 5)$, $C(6;-2;1)$. Tích vô hướng của hai vectơ $\overrightarrow{AB}$ và $\overrightarrow{BC}$ là
\choice
{$\sqrt{38}\cdot\sqrt{52}$}
{$-\sqrt{38}\cdot\sqrt{52}$}
{$8$}
{\True $-8$}
\loigiai{
Ta có $\overrightarrow{AB}=(2;-3; 5)$ và $\overrightarrow{BC}=(6; 0;-4)$.\\
Tích vô hướng của hai vectơ $\overrightarrow{AB}$ và $\overrightarrow{BC}$ là $\overrightarrow{AB}\cdot\overrightarrow{BC}=2\cdot 6+(-3)\cdot 0+5\cdot (-4)=-8$.
}
\end{ex}


\begin{ex}%[2-H2B4-SO-12-2425 (Nguồn Đề 3 - Bài 4)]%[VN-MT-7, Vũ Hồng Toàn]%[2H2H2-2]
Trong không gian với hệ trục tọa độ $Oxyz$, cho hai điểm $A(2;1;1)$ và $B(-1;2;1)$. Tìm tọa độ $A'$ đối xứng với $A$ qua $B$.
\choice
{$A'(3;4;-3)$}
{\True $A'(-4;3;1)$}
{$A'(4;-3;3)$}
{$A'(4;33)$}
\loigiai{
Vì $A'$ đối xứng với $A$ qua $B$ nên $B$ là trung điểm của $AA'$.\\
Do đó $\heva{&x_B=\dfrac{x_A+x_{A'}}{2}\\&y_B=\dfrac{y_A+y_{A'}}{2}\\&z_B=\dfrac{z_A+z_{A'}}{2}}\Rightarrow\heva{&x_{A'}=2x_B-x_A=2\cdot(-1)-2=-4\\&y_{A'}=2y_B-y_A=2\cdot 2-1=3\\&z_{A'}=2z_B-z_A=2\cdot 1-1=1.}$\\
Vậy $A'(-4;3;1)$.
}
\end{ex}


\begin{ex}%[2-H2B4-SO-12-2425 (Nguồn Đề 3 - Bài 4)]%[VN-MT-7, Vũ Hồng Toàn]%[2H2H2-2]
Trong KG $Oxyz$, cho hai điểm $M(0;0;2)$ và $N(4;-2;6)$. Tìm tọa độ điểm $P$ sao cho $N$ là trung điểm của $MP$.
\choice
{$P(2;-1;4)$}
{$(4;-2;4)$}
{$(2;-1;2)$}
{\True $P(8;-4; 10)$}
\loigiai{
Vì $N$ là trung điểm của $MP$ nên\\
\centerline{$\heva{&x_N=\dfrac{x_M+x_P}{2}\\&y_N=\dfrac{y_M+y_P}{2}\\&z_N=\dfrac{z_M+z_P}{2}}\Rightarrow\heva{&4=\dfrac{0+x_P}{2}\\&-2=\dfrac{0+y_P}{2}\\&6=\dfrac{2+z_P}{2}}\Rightarrow\heva{&x_P=8\\&y_P=-4\\&z_P=10.}$}
Vậy $P(8;-4; 10)$.
}
\end{ex}


\begin{ex}%[2-H2B4-SO-12-2425 (Nguồn Đề 3 - Bài 4)]%[VN-MT-7, Vũ Hồng Toàn]%[2H2H2-2]
Trong KG $Oxyz$, cho tam giác $MNP$ có $M(-1;3;0)$, $N(2;2;1)$, $P(-1;1;2)$. Trọng tâm $G$ của tam giác $MNP$ có tọa độ là
\choice
{\True $(0;2;1)$}
{$(0;6;3)$}
{$(2;0;1)$}
{$(0;-2;1)$}
\loigiai{
Theo công thức tính tọa độ trong tâm của tam giác ta có\\
\centerline{$\heva{&x_G=\dfrac{-1+2-1}{3}\\&y_G=\dfrac{3+2+1}{3}\\&z_G=\dfrac{0+1+2}{3}}\Rightarrow\heva{&x_G=0\\&y_G=2\\&z_G=1.}$}
Vậy $G(0;2;1)$.
}
\end{ex}


\begin{ex}%[2-H2B4-SO-12-2425 (Nguồn Đề 3 - Bài 4)]%[VN-MT-7, Vũ Hồng Toàn]%[2H2H2-6]
Trong không gian chọn hệ trục tọa độ cho trước, đơn vị đo là kilômét, rađa phát hiện một máy bay chiến đấu của Nga di chuyển với vận tốc và hướng không đổi từ điểm $M(600;400;20)$ đến điểm $N(800;500;30)$ trong $30$ phút. Nếu máy bay tiếp tục giữ nguyên vận tốc và hướng bay thì tọa độ của máy bay sau $15$ phút tiếp theo bằng bao nhiêu?
\choice
{$(700;500;30)$}
{$(900;650;55)$}
{\True $(900;550;35)$}
{$(800;540;30)$}
\loigiai{
Gọi $Q(x;y;z)$ là tọa độ của máy bay sau $15$ phút tiếp theo.\\
Ta có $\overrightarrow{MN}=(200;100;10)$ và $\overrightarrow{NQ}=(x-800;y-500;z-30)$.\\
Vì máy bay giữ nguyên hướng bay nên $\overrightarrow{MN}$ và $\overrightarrow{NQ}$ cùng hướng.
Do máy bay tiếp tục giữ nguyên vận tốc và thời gian bay từ $M\to N$ gấp $2$ lần thời gian bay từ $N\to Q$ nên\\
\centerline{$MN=2NQ
\Rightarrow\overrightarrow{MN}=2\overrightarrow{NQ}
\Rightarrow\heva{&200=2(x-800)\\&100=2(y-500)\\&10=2(z-30)}
\Rightarrow\heva{&x=900\\&y=550\\&z=35.}
\Rightarrow Q(900;550;35)$.}
Tọa độ của máy bay sau $15$ phút tiếp theo là $(900;550;35)$ .
}
\end{ex}

\Closesolutionfile{ans}

\TNTF
\Opensolutionfile{ans}[ans/ans\currfilebase-Phan-II]

\begin{ex}%[2-H2B4-SO-12-2425 (Nguồn Đề 3 - Bài 4)]%[VN-MT-7, Vũ Hồng Toàn]%[2H2H2-4]
Cho hình lăng trụ tam giác đều $ABC.A'B'C'$ có $AB=a$ và $AA'=a\sqrt{2}$. Gọi $M$ là trung điểm $BC$.
\choiceTF
{\True $\overrightarrow{AC}=\overrightarrow{AB}+\overrightarrow{BC}$}
{\True $\overrightarrow{A'M}=\overrightarrow{A'A}+\overrightarrow{A'B'}-\overrightarrow{CM}$}
{$\overrightarrow{A'M}\cdot\overrightarrow{AC}=\dfrac{a^2\sqrt{3}}{4}$}
{\True Góc giữa vectơ $\overrightarrow{AB'}$ và $\overrightarrow{BC'}$ bằng $60^\circ$}
\loigiai{
\begin{center}
\begin{tikzpicture}[scale=1, font=\footnotesize, line join=round, line cap=round, >=stealth]
\def\a{4} \def\b{2.2} \def\h{3.5} \def\g{30}
\path
(0,0) coordinate (A)
(0:\a) coordinate (C)++
(\g-180:\b) coordinate (B)
(90:\h) coordinate (A') ++(0:\a) coordinate (C')++
(\g-180:\b) coordinate (B')
($(C)!.5!(B)$) coordinate (M);
\draw[dashed](C)--(A)--(M)--(A');
\draw (A')--(A)--(B)--(B')--cycle (B)--(C)--(C')--(B') (A')--(C');
\foreach \x /\gN in {A/160,B/-90,C/-20,M/-30,A'/90,C'/90,B'/90}
\fill(\x) circle (1pt)($(\x)+(\gN:3mm)$) node {$\x$};
\end{tikzpicture}
\end{center}
Do $ABC.A'B'C'$ là lăng trụ tam giác đều cạnh $a$ nên $\triangle ABC$ đều cạnh $a$ và $AA'\perp (ABC)$.\\
Ta có $M$ là trung điểm của $BC$ nên $AM=\dfrac{a\sqrt{3}}{2}$.
\begin{itemchoice}
\itemch \textbf{Đúng}.\\ Áp dụng quy tắc cộng ta có $\overrightarrow{AC}=\overrightarrow{AB}+\overrightarrow{BC}$.
\itemch \textbf{Đúng}.\\Ta có $\overrightarrow{A'A}+\overrightarrow{A'B'}-\overrightarrow{CM}=\overrightarrow{A'A}+\overrightarrow{AB}+\overrightarrow{BM}=\overrightarrow{A'B}+\overrightarrow{BM}=\overrightarrow{A'M}$.
\itemch \textbf{Sai}.\\Ta có
\allowdisplaybreaks
\begin{eqnarray*}
\overrightarrow{A'M}\cdot \overrightarrow{AC}&=&\left(\overrightarrow{A'A}+\overrightarrow{AM}\right)\cdot\overrightarrow{AC}\\&=&\overrightarrow{A'A}\cdot\overrightarrow{AC}+\overrightarrow{AM}\cdot\overrightarrow{AC}\\&=&\overrightarrow{AM}\cdot\overrightarrow{AC}\\&=&\dfrac{a\sqrt{3}}{2}\cdot a\cdot \cos 30^\circ\\&=&\dfrac{3a^2}{4}.
\end{eqnarray*}
\itemch \textbf{Đúng}.\\Ta có
\begin{itemize}
 \item $\triangle ABB'$ vuông tại $B$ nên $AB'=\sqrt{AB^2+BB'^2}=\sqrt{a^2+2a^2}=a\sqrt{3}$.
 \item $\triangle BCC'$ vuông tại $C$ nên $BC'=\sqrt{BC^2+CC'^2}=\sqrt{a^2+2a^2}=a\sqrt{3}$.
 \item $\overrightarrow{AB}\cdot\overrightarrow{BC}=\left|\overrightarrow{AB}\right|\cdot \left|\overrightarrow{BC}\right|\cdot \cos 120^\circ=a\cdot a\cdot\dfrac{-1}{2}=-\dfrac{a^2}{2}$.
 \item $\overrightarrow{BB'}\cdot\overrightarrow{CC'}=\overrightarrow{BB'}^2=2a^2$.
\end{itemize}
Khi đó
\allowdisplaybreaks
\begin{eqnarray*}
\overrightarrow{AB'}\cdot\overrightarrow{BC'}&=&\left(\overrightarrow{AB}+\overrightarrow{BB'}\right)\cdot\left(\overrightarrow{BC}+\overrightarrow{CC'}\right)\\ &=&\overrightarrow{AB}\cdot\overrightarrow{BC}+\overrightarrow{AB}\cdot\overrightarrow{CC'}+\overrightarrow{BB'}\cdot\overrightarrow{BC}+\overrightarrow{BB'}\cdot\overrightarrow{CC'}\\
&=&-\dfrac{a^2}{2}+0+0+2a^2\\&=&\dfrac{3a^2}{2}.
\end{eqnarray*}
Suy ra $\cos \left(\overrightarrow{AB'},\overrightarrow{BC'}\right)=\dfrac{\overrightarrow{AB'}\cdot\overrightarrow{BC'}}{\left|\overrightarrow{AB'}\right|\cdot\left|\overrightarrow{BC'}\right|}=\dfrac{\dfrac{3a^2}{2}}{a\sqrt{3}\cdot a\sqrt{3}}=\dfrac{1}{2} \Rightarrow \left(\overrightarrow{AB'}, \overrightarrow{BC'}\right)=60^\circ$.
\end{itemchoice}
}
\end{ex}


\begin{ex}%[2-H2B4-SO-12-2425 (Nguồn Đề 3 - Bài 4)]%[VN-MT-7, Vũ Hồng Toàn]%[2H2H2-3]
Trong KG $Oxyz$, cho tam giác $ABC$ có các đỉnh $A(1;-2;0)$, $B(2;1;-2)$, $C(0;3;4)$. 
\choiceTF
{\True Tọa độ của vectơ $\overrightarrow{AB}$ là $(1;3;-2)$}
{\True Tọa độ trọng tâm của tam giác $ABC$ là $G\left(1;\dfrac{2}{3};\dfrac{2}{3} \right)$}
{Tọa độ hình chiếu của điểm $B$ trên mặt phẳng $(Oxy)$ là $H(0;0;-2)$}
{$\overrightarrow{x}=2\overrightarrow{AB}-3\overrightarrow{BC}$. Tọa độ của vectơ $\overrightarrow{x}=(-4;12;14)$}
\loigiai{
\begin{itemchoice}
\itemch \textbf{Đúng}.\\ Ta có $\overrightarrow{AB}=(1;3;-2)$.
\itemch \textbf{Đúng}.\\ Ta có $\heva{&x_G=\dfrac{x_A+x_B+x_C}{3}=1\\&y_G=\dfrac{y_A+y_B+y_C}{3}=\dfrac{2}{3}\\&z_G=\dfrac{z_A+z_B+z_C}{3}=\dfrac{2}{3}}\Rightarrow G\left(1;\dfrac{2}{3};\dfrac{2}{3} \right)$.
\itemch \textbf{Sai}.\\Tọa độ hình chiếu của điểm $B(2;1;-2)$ trên mặt phẳng $(Oxy)$ là $H(2;1;0)$.
\itemch \textbf{Sai}.\\Ta có
\begin{itemize}
\item $\overrightarrow{AB}=(1;3;-2)\Rightarrow 2\overrightarrow{AB}=(2;6;-4)$.
\item $\overrightarrow{BC}=(-2;2;6)\Rightarrow-3\overrightarrow{BC}=(6;-6;-18)$.
\end{itemize}
Vậy $\overrightarrow{x}=2\overrightarrow{AB}-3\overrightarrow{BC}=(8;0;-22)$.
\end{itemchoice}
}
\end{ex}


\begin{ex}%[2-H2B4-SO-12-2425 (Nguồn Đề 3 - Bài 4)]%[VN-MT-7, Vũ Hồng Toàn]%[2H2V2-5]
Trong không gian với hệ trục tọa độ $Oxyz$, cho bốn điểm $A(0;-1; 1)$, $B(-2; 1;-1)$, $C(-1; 3; 2)$, $D(-1; 0; 0)$. 
\choiceTF
{\True Ba điểm $A$, $B$, $C$ không thẳng hàng}
{\True Ba điểm $A$, $B$, $D$ thẳng hàng}
{Côsin của góc giữa $\overrightarrow{AB}$ và $\overrightarrow{CB}$ bằng $-\dfrac{\sqrt{42}}{21}$}
{Bốn điểm $A$, $B$, $C$, $D$ không đồng phẳng}
\loigiai{
\begin{itemchoice}
\itemch \textbf{Đúng}.\\Ta có $\overrightarrow{AB}=(-2; 2;-2)$, $ \overrightarrow{BC}=(1; 2; 3)$.\\
Giả sử tồn tại số $k\ne 0$ sao cho $\overrightarrow{AB}=k\overrightarrow{BC}\Rightarrow \heva{&-2=k\\&2=2k\\&-2=3k.}
$\\Hệ vô nghiệm suy ra không tồn tại $k$.
Vậy ba điểm $A$, $B$, $C$ không thẳng hàng.
\itemch \textbf{Đúng}.\\ Ta có $\overrightarrow{AB}=(-2; 2;-2)$, $ \overrightarrow{BD}=(1;-1; 1)$.\\
Vì $\overrightarrow{AB}=-2\overrightarrow{BD}$.
Suy ra điểm $A$, $B$, $D$ thẳng hàng
\itemch \textbf{Sai}.\\Ta có $\overrightarrow{AB}=(-2; 2;-2)$, $\overrightarrow{CB}=(-1;-2;-3)$.\\
Khi đó $\cos\left (\overrightarrow{AB},\overrightarrow{CB}\right)=\dfrac{\overrightarrow{AB}\cdot\overrightarrow{CB}}{\left|\overrightarrow{AB}\right|\cdot\left|\overrightarrow{CB}\right|} =\dfrac{(-2)\cdot(-1)+2\cdot(-2)+(-2)\cdot(-3)}{\sqrt{12}\cdot\sqrt{14}}=\dfrac{\sqrt{42}}{21}$.
\itemch \textbf{Sai}.\\Ta có $\overrightarrow{AB}=(-2; 2;-2)$, $ \overrightarrow{BD}=(1;-1; 1)$.\\
Vì $\overrightarrow{AB}=-2\overrightarrow{BD}$.
Suy ra điểm $A$, $B$, $D$ thẳng hàng.\\
Khi đó luôn tồn tại một mặt phẳng qua $C$ và chứa đường thẳng đi qua ba điểm $A$, $B$, $D$.\\
Vậy bốn điểm $A$, $B$, $C$, $D$ đồng phẳng.
\end{itemchoice}
}
\end{ex}


\begin{ex}%[2-H2B4-SO-12-2425 (Nguồn Đề 3 - Bài 4)]%[VN-MT-7, Vũ Hồng Toàn]%[2H2V2-5]
Trong không gian với hệ trục tọa độ $Oxyz$, cho ba điểm $A(-1; 2; 1)$; $B(2;-2;4)$; $C(0;-4;1)$. 
\choiceTF
{\True Ba điểm $A$, $B$, $C$ không thẳng hàng}
{\True Biết điểm $D(5;-6;7)$. Khi đó ba điểm $A$, $B$, $D$ thẳng hàng}
{$\cos\left(\overrightarrow{AB},\overrightarrow{AC}\right)=\dfrac{37}{\sqrt{1258}}$}
{Cho $\overrightarrow{u}=(x-1;2y+1;3z-5)$ thoả mãn $\overrightarrow{u}\perp \overrightarrow{AB}$ và $\overrightarrow{u}\perp \overrightarrow{AC}$. Khi đó $x^2+y^2+z^2=2024$}
\loigiai{
\begin{itemchoice}
\itemch \textbf{Đúng}.\\Ta có $\overrightarrow{AB}=(3;-4;3)$, $\overrightarrow{AC}=(1;-6;0)$.\\ Giả sử tồn tại số $k\ne 0$ sao cho $\overrightarrow{AB}=k\overrightarrow{AC}\Rightarrow \heva{&3=k\\&-4=-6k\\&3=0k.}
$\\
Hệ vô nghiệm suy ra không tồn tại $k$. Vậy ba điểm $A$, $B$, $C$ không thẳng hàng.
\itemch \textbf{Đúng}.\\Ta có $\overrightarrow{AB}=(3;-4;3)$, $\overrightarrow{AD}=(6;-8;6)\Rightarrow \overrightarrow{AD}=2\overrightarrow{AB}$.\\ 
Vậy ba điểm $A$, $B$, $D$ thẳng hàng.
\itemch \textbf{Sai}.\\
Ta có $\cos\left(\overrightarrow{AB},\overrightarrow{AC}\right)=\dfrac{\overrightarrow{AB}\cdot\overrightarrow{AC}}{\left|\overrightarrow{AB}\right|\cdot\left|\overrightarrow{AC}\right|}=\dfrac{3\cdot 1+(-4)\cdot (-6)+3\cdot 0}{\sqrt{9+16+9}\cdot\sqrt{1+36+0}}=\dfrac{27}{\sqrt{1\,258}}$.
\itemch \textbf{Sai}.\\Ta có $\overrightarrow{u}\perp \overrightarrow{AB}$ và $\overrightarrow{u}\perp \overrightarrow{AC}$ suy ra $\overrightarrow{u}$ cùng phương với $\left[\overrightarrow{AB},\overrightarrow{AC}\right]=(18;3;-14)$.\\
Xét trường hợp $\overrightarrow{u}=(18;3;-14)$ ta có\\
\centerline{$\heva{&x-1=18\\&2y+1=3\\&3z-5=-14}\Rightarrow\heva{&x=19\\&y=1\\&z=-3.}$}\\
Vậy $x^2+y^2+z^2=19^2+1+9=371$.
\end{itemchoice}
}
\end{ex}

\Closesolutionfile{ans}

\TNSA
\Opensolutionfile{ans}[ans/ans\currfilebase-Phan-III]


\begin{ex}%[2-H2B4-SO-12-2425 (Nguồn Đề 3 - Bài 4)]%[VN-MT-7, Vũ Hồng Toàn]%[2H2V2-4]
Cho hình lập phương $ABCD.A'B'C'D'$ có cạnh là $a$. Gọi $G$ là trọng tâm tam giác $B'C'D'$, $I$ là trung điểm của $AB'$. Tính $\cos\left(\overrightarrow{A'D}, \overrightarrow{IG}\right)$ (làm tròn kết quả đến hàng phần trăm).
\shortans{0{,}14}
\loigiai{
\begin{center}
\begin{tikzpicture}[scale=1, font=\footnotesize, line join=round, line cap=round, >=stealth]
\def\a{4} \def\b{2.5} \def\h{3.5} \def\g{30}
\path
(0,0) coordinate (A)
(0:\a) coordinate (D)++
(\g-180:\b) coordinate (C)++(180:\a) coordinate (B)
(90:\h) coordinate (A') ++(0:\a) coordinate (D')++
(\g-180:\b) coordinate (C')++(180:\a) coordinate (B')
($(A')!.5!(C')$) coordinate (M)
($(C')!2/3!(M)$) coordinate (G)
($(A)!.5!(B')$) coordinate (I)
;
\draw[dashed](B)--(A)--(D) (A')--(A)--(B') (I)--(G) (A')--(D);
\draw (A')--(B')--(C')--(D') --cycle (B')--(B)--(C)--(C')--(A') (C)--(D)--(D')--(B') ;
\foreach \x /\gN in {A/60,B/180,C/-20,G/0,A'/90,D'/0,C'/-40,B'/180,D/0,I/180}
\fill(\x) circle (1pt)($(\x)+(\gN:3mm)$) node {$\x$};
\end{tikzpicture}
\end{center}
Ta có cạnh hình lập phương là $a\Rightarrow A'D=a\sqrt{2}$ và $\overrightarrow{A'D}=\overrightarrow{AD}-\overrightarrow{AA'}$.
\allowdisplaybreaks
\begin{eqnarray*}
\overrightarrow{IG}&=&\overrightarrow{IB'}+\overrightarrow{B'G}\\&=&\dfrac{1}{2} \left(\overrightarrow{AA'}+\overrightarrow{AB}\right)+\dfrac{1}{3} \left(\overrightarrow{B'D'}+\overrightarrow{B'C'}\right)\\&=&\dfrac{1}{2} \left(\overrightarrow{AA'}+\overrightarrow{AB}\right)+\dfrac{1}{3} \left(\overrightarrow{BD}+\overrightarrow{AD}\right) \\ &=&\dfrac{1}{2} \left(\overrightarrow{AA'}+\overrightarrow{AB}\right)+\dfrac{1}{3} \left(2\overrightarrow{AD}-\overrightarrow{AB}\right)\\&=&\dfrac{1}{2} \overrightarrow{AA'}+\dfrac{1}{6} \overrightarrow{AB}+\dfrac{2}{3} \overrightarrow{AD} \\
\Rightarrow{\overrightarrow{IG}}^2&=&\left(\dfrac{1}{2} \overrightarrow{AA'}+\dfrac{1}{6} \overrightarrow{AB}+\dfrac{2}{3} \overrightarrow{AD}\right)^2=\dfrac{13a^2}{18}\\
\Rightarrow IG&=&\dfrac{a\sqrt{26}}{6}.
\end{eqnarray*}
Mà\\
\centerline{$\overrightarrow{A'D}\cdot\overrightarrow{IG}=\left(\dfrac{1}{2} \overrightarrow{AA'}+\dfrac{1}{6} \overrightarrow{AB}+\dfrac{2}{3} \overrightarrow{AD}\right)\cdot\left(\overrightarrow{AD}-\overrightarrow{AA'}\right)=\dfrac{a^2}{6} $.}
Vậy\\
\centerline{$ \cos\left(\overrightarrow{A'D}, \overrightarrow{IG}\right)=\dfrac{\overrightarrow{A'D}\cdot\overrightarrow{IG}}{A'D\cdot IG}=\dfrac{\dfrac{a^2}{6}}{\dfrac{a\sqrt{26}}{6}\cdot a\sqrt{2}}=\dfrac{\sqrt{13}}{26}\approx 0{,}14 $.}
}
\end{ex}


\begin{ex}%[2-H2B4-SO-12-2425 (Nguồn Đề 3 - Bài 4)]%[VN-MT-7, Vũ Hồng Toàn]%[2H2H2-2]
Trong KG $Oxyz$, cho hình hộp $ABCD.A'B'C'D'$. Biết $A(2; 4; 0)$, $B(4; 0; 0)$, $C(-1; 4;-7)$ và $D'(6; 8; 10)$. Tọa độ đỉnh $B'$ của hình hộp có dạng $B'(a;b;c)$. Tính $a+b+c$.
\shortans{30}
\loigiai{
\begin{center}
\begin{tikzpicture}[scale=1, font=\footnotesize, line join=round, line cap=round, >=stealth]
\def\a{3} \def\b{2} \def\h{3} \def\g{30}
\path
(0,0) coordinate (A)
(0:\a) coordinate (D)++
(\g-180:\b) coordinate (C)++(180:\a) coordinate (B)
(80:\h) coordinate (A') ++(0:\a) coordinate (D')++
(\g-180:\b) coordinate (C')++(180:\a) coordinate (B')
;
\draw[dashed](B)--(A)--(D) (A')--(A);
\draw (A')--(B')--(C')--(D') --cycle (B')--(B)--(C)--(C') (C)--(D)--(D') ;
\foreach \x /\gN in {A/60,B/180,C/-20,A'/180,D'/0,C'/-40,B'/180,D/0}
\fill(\x) circle (1pt)($(\x)+(\gN:3mm)$) node {$\x$};
\end{tikzpicture}
\end{center}
Ta có $\overrightarrow{BC}=(-5; 4;-7)$. Gọi $D(x; y; z)\Rightarrow \overrightarrow{AD}=(x-2; y-4; z)$.\\
Vì $ABCD$ là hình bình hành nên\\
\centerline{$\overrightarrow{AD}=\overrightarrow{BC}\Rightarrow\heva{&x-2=-5\\&y-4=4\\&z=-7}\Rightarrow\heva{&x=-3\\&y=8\\&z=-7}\Rightarrow D(-3;8;-7)$.}
Ta có $\overrightarrow{DD'}=(9; 0; 17)$ và $\overrightarrow{BB'}=(a-4; b; c)$.\\
Vì $BB'D'D$ là hình bình hành nên\\
\centerline{$\overrightarrow{BB'}=\overrightarrow{DD'}\Rightarrow \heva{&a-4=9\\&b=0\\&c=17}\Rightarrow\heva{&a=13\\&b=0\\&c=17}\Rightarrow B'(13; 0; 17)$.}
Vậy $a+b+c=13+0+17=30$.
}
\end{ex}


\begin{ex}%[2-H2B4-SO-12-2425 (Nguồn Đề 3 - Bài 4)]%[VN-MT-7, Vũ Hồng Toàn]%[2H2H2-4]
Trong KG $Oxyz$, cho ba điểm $A(1;6;2)$, $B(5; 1; 3)$ và $C(4; 0; 6)$. Biết $\overrightarrow{u}=(14;a;b)$ vuông góc với với cả hai vectơ $\overrightarrow{AB}$ và $\overrightarrow{AC}$. Tính $a-b$.
\shortans{4}
\loigiai{
Ta có $\overrightarrow{AB}=(4;-5; 1)$ và $\overrightarrow{AC}=(3;-6; 4)$.\\
Vì $\overrightarrow{u}$ vuông góc với cả hai vectơ $\overrightarrow{AB}$ và $\overrightarrow{AC}$ nên $\overrightarrow{u}$ cùng phương với vectơ \linebreak $\left[\overrightarrow{AB}, \overrightarrow{AC}\right]=(-14;-13;-9)$.
Do đó tồn tại số thực $k$ để $\overrightarrow{u}=k\left[\overrightarrow{AB}, \overrightarrow{AC}\right]$.\\
Khi đó ta có $k=-1$, $\overrightarrow{u}=(14;13;9)$.\\
Suy ra $a=13$ và $b=9$. Vậy $a-b=13-9=4$.
}
\end{ex}


\begin{ex}%[2-H2B4-SO-12-2425 (Nguồn Đề 3 - Bài 4)]%[VN-MT-7, Vũ Hồng Toàn]%[2H2V2-6]
Hình minh họa sơ đồ ngôi nhà Trong KG $Oxyz$, trong đó nền nhà, bốn bức tường và hai mái nhà đều là hình chữ nhật. Biết tọa độ của vectơ $\overrightarrow{AH}=(a;b;c)$. Tìm $a+b+c$.
\begin{center}
 \begin{tikzpicture}[font=\footnotesize, line join=round, line cap=round, >=stealth, scale=1.2]
 \def\a{3}
 \def\b{5}
 \def\h{3}
 \path (0:0) coordinate (C)
 ++(0:\a) coordinate (B)
 ++(-160:\b) coordinate (O)
 ($(O)+(B)-(C)$) coordinate (A)
 ($(O)+(90:\h)$) coordinate (E)
 ($(B)+(90:\h)$) coordinate (G)
 ($(C)+(90:\h)$) coordinate (H)
 ($(A)+(90:\h)$) coordinate (F)
 ($(A)+(0:1)$) coordinate (x)
 ($(H)+(35:2)$) coordinate (Q)
 ($(E)+(35:2)$) coordinate (P)
 ($(E)+(90:1)$) coordinate (z)
 ($(O)!1.3!(C)$) coordinate (y);
 \draw[dashed] (G)--(H)--(C)--(B) (C)--(O);
 \draw[] (G)--(Q)--(H)--(E)--(F)--(G)--(B)--(A)--(O)--(E) (F)--(A) (F)--(P)--(E) (P)--(Q);
 \draw [->] (A)--(x);
 \draw [->] (E)--(z);
 \draw [->,dashed] (C)--(y);
 \draw [] (Q)node[above]{$Q(2; 5; 4)$} (G)node[right]{$G(4; 5; 3)$} (B)node[right]{$B(4; 5; 0)$} (P)node[right]{$P(2; 0; 4)$} (O)node[below]{$O(0; 0; 0)$} (E)node[left]{$E(0; 0; 3)$} (x)node[below]{$x$} (y)node[above]{$y$} (z)node[left]{$z$};
 \foreach \x/\g in {A/-90,C/180,F/0,H/90}
 \fill[black] (\x) circle (1pt)
 ($(\g:4mm)+(\x)$) node {$\x$}; 
 \end{tikzpicture}
\end{center}
\shortans{4}
\loigiai{
Vì nền nhà là hình chữ nhật nên $OABC$ là hình chữ nhật, suy ra $x_A=x_B=4$, $y_C=y_B=5$.\\
Do điểm $A$ nằm trên trục $Ox$ nên tọa độ điểm $A(4;0;0)$; điểm $C$ nằm trên trục $Oy$ nên tọa độ điểm $C(0;5;0)$.\\
Tường nhà là hình chữ nhật nên $OCHE$ là hình chữ nhật, suy ra $y_H=y_C=5$.\\
Do $H$ nằm trên mặt phẳng $(Oyz)$ nên tọa độ điểm $H(0;5;3)$.\\
Khi đó $\overrightarrow{AH}=(0-4;5-0;3-0)\Rightarrow \overrightarrow{AH}=(-4;5;3)$.
Suy ra $a=-4$, $b=5$, $c=3$.\\
Vậy $a+b+c=-4+5+3=4$.
}
\end{ex}


\begin{ex}%[2-H2B4-SO-12-2425 (Nguồn Đề 3 - Bài 4)]%[VN-MT-7, Vũ Hồng Toàn]%[2H2V2-6]
 \immini{
 Một chiếc ô tô được đặt trên mặt đáy dưới một khung sắt có dạng hình hộp chữ nhật với đáy trên là hình chữ nhật $ABCD$, mặt phẳng $(ABCD)$ song song với mặt mặt phẳng nằm ngang. Khung sắt đó được buộc vào móc $E$ của chiếc cần cẩu sao cho các đoạn dây cáp $EA$, $EB$, $EC$, $ED$ có độ dài bằng nhau và cùng tạo với mặt phẳng $(ABCD)$ một góc $60^\circ$ như hình vẽ. Chiếc cần cẩu kéo khung sắt lên theo phương thẳng đứng. Biết lực căng $\overrightarrow{F_1}$, $\overrightarrow{F_2}$, $\overrightarrow{F_3}$, $\overrightarrow{F_4}$ đều có cường độ $5\, 000$ N và trọng lượng khung sắt là $2\, 000$ N. Biết trọng lượng của chiếc xe ô tô bằng $m\times 9{,}81$ N. Giá trị của $m$ làm tròn đến hàng đơn vị bằng bao nhiêu?
 }
 {
 \begin{tikzpicture}[scale=0.65, font=\footnotesize, line join=round, line cap=round, >=stealth,transform shape]
 \definecolor{bostonuniversityred}{rgb}{0.8, 0.0, 0.0}
 \definecolor{charcoal}{rgb}{0.21, 0.27, 0.31}
 \definecolor{bananayellow}{rgb}{1.0, 0.88, 0.21}
 \definecolor{anti-flashwhite}{rgb}{0.95, 0.95, 0.96}
 % \clip (-6,-3) rectangle (6,3);
 \tikzset{%
 xeoto/.pic={%
 %--------------------------
 \tikzset{xe/.pic={
 \def\N{
 (-2.7,.56)--(-2.5,.56)
 ..controls +(50:1.5) and +(165:1.5) .. (2.1,1.88)--(2.05,2)
 ..controls +(-10:.1) and +(130:.1) .. (3.25,1.75)--(3.15,1.65)
 ..controls +(-4:.2) and +(130:.15) .. (4.05,.7)--(4.25,.75)
 ..controls +(-40:.2) and +(130:.15) .. (4.55,.35)--(4.35,.26)
 ..controls +(-40:.2) and +(130:.15) .. (4.8,-.45)--(4.92,-.4)
 ..controls +(-40:.25) and +(73:.17) .. (4.8,-1.8)--(-4.4,-1.8)
 ..controls +(175:.7) and +(-175:3.2) ..cycle
 ;
 }
 \fill[bostonuniversityred] \N;
 \draw \N;
 \def\K{
 (-2.2,.56)--(3.3,.7)
 ..controls +(100:1.18) and +(43:3) .. cycle
 ;
 }
 \fill[bottom color=charcoal,top color=charcoal!20!white, middle color=charcoal!80!white] \K;
 \draw \K;
 \def\K1{
 (-2.2,.56)
 ..controls +(43:.2) and +(43:.2) .. (-1.58,1.05)--(-1.53,.57)--cycle
 ;
 }
 \draw \K1;
 \fill[charcoal] \K1;
 \def\K2{
 (1.2,1.85)
 ..controls +(-10:.1) and +(160:.1) .. (1.58,1.8)--(1.8,.65)--(1.25,.65)--cycle
 ;
 }
 \draw \K2;
 \fill[charcoal] \K2;
 \def\Kt{
 (-2.5,.56)
 ..controls +(50:1.5) and +(165:1.5) .. (2.1,1.88)--(2.05,2)
 ..controls +(170:2.2) and +(45:1.5) .. (-2.7,.56)--cycle
 ;
 }
 \fill[charcoal!50] \Kt;
 \draw \Kt;
 \def\Ks{
 (3.25,1.75)--(3.15,1.65)
 ..controls +(-4:.2) and +(130:.15) .. (4.05,.7)--(4.22,.75)
 ..controls +(120:.3) and +(-35:.3) .. cycle
 ;
 }
 \fill[charcoal!50] \Ks;
 \draw \Ks;
 %Đèn sau
 \def\D{
 (4.55,.35)--(4.35,.26)
 ..controls +(-40:.2) and +(130:.15) .. (4.8,-.45)--(4.92,-.4)
 ..controls +(110:.2) and +(-40:.15) ..cycle
 ;
 }
 \fill[bananayellow] \D;
 \draw \D;
 \def\M{
 (2.2,-1.3)--(-1.8,-1.4)--(-1.78,-1.7)
 ..controls +(-5:.3) and +(-90:.6) ..cycle
 ;
 }
 \draw \M;
 \fill[charcoal!90] \M;
 \draw (-1.6,.55)
 ..controls +(-170:.5) and +(95:.4) .. (-1.78,-1.7)
 (1.6,.65)
 ..controls +(-30:.5) and +(35:.3) .. (1.7,-1.3)
 ;
 %gương
 \def\G{
 (-1.5,.45)--(-1.4,.6)
 ..controls +(85:1) and +(20:.6) .. (-1.25,.5)--(-1.4,.33)
 ;
 }
 \draw \G;
 \fill[bostonuniversityred] \G;
 %Đèn trước
 \def\Dt{
 (-4.85,-.7)
 ..controls +(75:1) and +(65:.8) .. (-4.5,-.7)
 ..controls +(-115:.6) and +(-105:.4) .. cycle
 ;
 }
 \fill[bananayellow] \Dt;
 \draw \Dt;
 \def\Dt2{
 (-4.85,-.7)
 ..controls +(75:.6) and +(65:.4) .. (-4.7,-.7)
 ..controls +(-115:.3) and +(-105:.2) .. cycle
 ;
 }
 \fill[anti-flashwhite] \Dt2;
 \draw \Dt2;
 \draw[fill=anti-flashwhite] (-4.86,-1.45)--(-4.82,-1.5)--(-4.55,-1.3)
 ..controls +(90:.3) and +(45:.2) .. cycle;
 }}
 \tikzset{banh_xe/.pic={
 \draw[fill=charcoal] (-3.25,-1.65) circle (1) ;
 \draw[fill=anti-flashwhite] (-3.25,-1.65) circle (.7) ;
 \draw[fill=charcoal] (-3.25,-1.65) circle (.4) ;
 }}
 %----------------
 \path
 (0,0)pic[scale=1]{xe}(0,0)pic[scale=1]{banh_xe}(6.9,0)pic[scale=1]{banh_xe};
 %--------------------------------
 }}
 %%%%%%%%%%%%%%%%%%%
 \def\bc{4.25} % cạnh BC
 \def\ba{2} % cạnh BA
 \def\h{3.5} % đường cao
 \def\gocnghieng{90} % góc nghiêng
 \def\gocB{160} % góc B của đáy
 \coordinate (B1) at (0,0);
 \coordinate (A1) at (\gocB:\ba);
 \coordinate (C1) at (\bc,0.25);
 \coordinate (D1) at ($(C1)-(B1)+(A1)$);
 \coordinate[label=above left:$A$] (A) at ($(A1)+(\gocnghieng:\h)$);
 \coordinate[label=below left:$B$] (B) at ($(B1)-(A1)+(A)$);
 \coordinate[label=right:$C$] (C) at ($(C1)-(A1)+(A)$);
 \coordinate[label=above right:$D$] (D) at ($(D1)-(A1)+(A)$);
 \coordinate (E) at ($(A)!0.5!(C)+(\gocnghieng:\h)$);
 %------------
 \draw[->,blue,very thick] (E)--($(E)!0.65!(A)$) node[above left]{$\overrightarrow{F_1}$};
 \draw[->,blue,very thick] (E)--($(E)!0.65!(B)$) node[right]{$\overrightarrow{F_2}$};
 \draw[->,blue,very thick] (E)--($(E)!0.65!(C)$) node[above right]{$\overrightarrow{F_3}$};
 \draw[->,blue,very thick] (E)--($(E)!0.65!(D)$) node[left=2pt]{$\overrightarrow{F_4}$};
 %------------
 \path (E) node[left=1mm]{$E$};
 \draw[blue,very thick] (A)--(B)--(C)--(D)--cycle
 (A1)--(A) (D1)--(D) (C1)--(C)
 (A)--(E)--(B) (C)--(E)--(D);
 \draw[fill=teal] (A1)--(B1)--(C1)--(D1)--cycle;
 \draw[fill=teal!30] (A1)--(B1)--(C1)--++(0,-0.3)--([yshift=-0.3cm]B1)--([yshift=-0.3cm]A1)--cycle;
 \foreach \diem in {A1,B1,C1,D1,A,B,C,D,E} \fill (\diem)circle(1.5pt);
 %phần móc và dây
 \def\r{0.3}\def\rr{0.25}
 \coordinate (tam) at ([yshift=6mm]E);
 \draw[brown,fill=brown,line width=1pt] (tam) circle (\r cm);
 \fill (tam) circle (2pt);
 \draw[brown,line width=1pt] (tam)++(\r,0)--++(0,0.7)
 (tam)++(-\r,0)--++(0,0.7);
 \draw[line width=1.5pt] (tam)--++(0,-1.35*\r) arc(90:370:1mm);
 %%%%%%%%%%%%%%%%%%%
 \pic[scale=0.45,rotate=4] at (1.6,1.3) [pic type = xeoto];
 %--------
 \draw[blue,very thick] (B)--(B1);
 \end{tikzpicture}
 }
 \shortans{1562}
 \loigiai{
 \begin{center}
 \begin{tikzpicture}[scale=1, font=\footnotesize, line join=round, line cap=round, >=stealth]
 \def\a{4} \def\b{2.2} \def\h{4.5} \def\g{40}
 \path
 (0,0) coordinate (A)
 (0:\a) coordinate (D)++
 (-\g:\b) coordinate (C)++(180:\a) coordinate (B)
 ($(A)!.5!(C)$) coordinate (O)++ (90:\h) coordinate (E)
 ;
 \draw[dashed] (E)--(D)--(A)--(C)--(D)--(B) (E)--(O); 
 \draw (E)--(A)--(B)--(C)--(E)--(B);
 \foreach \i/\j [count =\k from 1] in {A/180,B/210,C/-10}{\draw[->] (E)-- ($(E)!0.65!(\i)$) coordinate(\i') node[midway, shift={(\j:4mm)}]{$\overrightarrow{F}_\k$};}
 \draw[->, dashed] (E)-- ($(E)!0.65!(D)$) coordinate(D') node[midway, shift={(250:4mm)}]{$\overrightarrow{F}_4$};
 \draw (A')--(B')--(C');
 \draw[dashed] (B')--(D')--(A')--(C')--(D');
 \draw[->, dashed] (E)--($(E)!0.65!(O)$) coordinate(O');
 \foreach \x /\gN in {A/160,B/-90,C/-20,E/90,O/-90,D/30, A'/160, B'/220, C'/10, D'/40, O'/-55}
 \fill(\x) circle (1pt)($(\x)+(\gN:3mm)$) node {$\x$};
 \end{tikzpicture}
 \end{center}
 Gọi $O$ là hình chiếu vuông góc của $E$ trên $(ABCD)$. Ta có $EA=EB=EC=ED$ nên các tam giác vuông $EOA$, $EOB$, $EOC$, $EOD$ bằng nhau. Suy ra $OA=OB=OC=OD$ hay $O$ là tâm hình chữ nhật $ABCD$.\\ 
 Gọi $A'$, $B'$, $C'$, $D'$ lần lượt là các điểm sao cho $\overrightarrow{EA'}=\overrightarrow{F}_1$, $\overrightarrow{EB'}=\overrightarrow{F}_2$, $\overrightarrow{EC'}=\overrightarrow{F}_3$ và $\overrightarrow{ED'}=\overrightarrow{F}_4$.\\
 Vì $\left|\overrightarrow{F}_1\right| = \left|\overrightarrow{F}_2\right| = \left|\overrightarrow{F}_3\right| = \left|\overrightarrow{F}_4\right| = 5\,000$ N nên $EA'=EB'=EC'=ED'=5\,000$. Do đó $E.A'B'C'D'$ là hình chóp có đáy $A'B'C'D'$ là hình chữ nhật.\\
 Gọi $O'$ là tâm hình chữ nhật $A'B'C'D'$, ta có $O'$ thuộc $EO$.\\ 
 Theo quy tắc hình bình hành $\overrightarrow{F_1}+\overrightarrow{F_3}=2\overrightarrow{EO'}$; $\overrightarrow{F_2}+\overrightarrow{F_4}=2\overrightarrow{EO'}$.\\
 Khi đó
 $\overrightarrow{F_1}+\overrightarrow{F_3}+\overrightarrow{F_2}+\overrightarrow{F_4}=4\overrightarrow{EO'}$.\\
 Các dây cáp $EA$, $EB$, $EC$, $ED$ có độ dài bằng nhau và cùng tạo với mặt phẳng $(ABCD)$ một góc $60^\circ$ nên $\widehat{EA'O'}=60^\circ$. Do đó\\
 \[EO'=EA'\cdot\sin 60^\circ=5\, 000\cdot\dfrac{\sqrt{3}}{2}=2\, 500\sqrt{3}.\]
 Gọi trọng lực của xe và khung sắt là $\overrightarrow{P}$. Vì chiếc xe ô tô và khung sắt ở vị trí cân bằng nên 
 \[\overrightarrow{P}=\overrightarrow{F_1}+\overrightarrow{F_2}+\overrightarrow{F_3}+\overrightarrow{F_4} = 4\overrightarrow{EO'}.\]
 Suy ra trọng lượng của xe và khung sắt là $\left|\overrightarrow{P}\right| = 4\left|\overrightarrow{EO'}\right| = 4\cdot 2\,500\cdot \sqrt{3} = 10\,000\sqrt{3}$ N.\\
 Vì khung sắt có trọng lượng bằng $2\,000$ N nên trọng lượng của xe ô tô là $10\,000\sqrt{3}-2\,000$ N.\\
 Vậy $m=\dfrac{10\,000\sqrt{3}-2\,000}{9{,}81}\approx 1\,562$.
 }
\end{ex}


\begin{ex}%[2-H2B4-SO-12-2425 (Nguồn Đề 3 - Bài 4)]%[VN-MT-7, Vũ Hồng Toàn]%[2H2V2-6]
 \immini{
 Một vật nặng có trọng lượng là $400$ N được đặt trên một khung sắt hình tròn như hình bên. Biết $ABCD$ là hình chữ nhật, mặt phẳng $(ABCD)$ song song với mặt phẳng nằm ngang. Khung sắt được móc vào điểm $S$ sao cho các đoạn dây cáp $SA$, $ SB$, $SC$, $SD$ có độ dài bằng nhau và cùng tạo với mặt phẳng $(ABCD)$ một góc bằng $45^\circ$. Chiếc cần cẩu kéo khung sắt lên theo phương thẳng đứng. Biết trọng lượng của khung sắt là $200$ N; cường độ các lực căng $\overrightarrow{F}_1$, $\overrightarrow{F}_2$, $\overrightarrow{F}_3$, $\overrightarrow{F}_4$ là bằng nhau. Tính cường độ của lực căng $\overrightarrow{F}_1$ (làm tròn đến hàng đơn vị).
 }
 {
 \begin{tikzpicture}[scale=.7,>=stealth, font=\footnotesize, line join=round, line cap=round]
 \tikzset{day/.pic=
 {\draw[shade,bottom color=brown!30,top color= white!30,rounded corners=0.5ex,line width=1.5pt,gray!80]
 (0,0) ellipse ({2.5pt} and {8pt});}
 }
 \def\h{6}
 \def\a{3}
 \def\b{1.5}
 \path
 (0,0) coordinate (O)
 ($(O)+(0,\h)$) coordinate (S)
 ($(O)+(10:\a cm and \b cm)$)coordinate (M)
 ($(O)+(180:\a cm and .8*\b cm)$)coordinate (A)
 ($(O)+(0:\a cm and .8*\b cm)$)coordinate (C)
 ($(O)+(60:\a cm and .8*\b cm)$)coordinate (B)
 ($(O)+(-120:\a cm and .8*\b cm)$)coordinate (D)
 ;
 
 \draw[fill=brown] (M) arc (10:-190:\a cm and \b cm);
 \draw[fill=white] (M) arc (10:-190:\a cm and .8*\b cm);
 \draw [fill=white] (M) arc (10:190:\a cm and .8*\b cm);
 \draw[fill,bottom color=black!30,top color= brown!70, ,left color=black!50] ($(S)-(.5,.3)$) rectangle ($(S)+(.5,.3)$);
 \foreach \m in {0,1,2,...,12}{\pic[rotate=-20] at ($(A)+(.25*\m,.5*\m)$) {day};}
 \foreach \m in {0,1,2,...,14}{\pic[rotate=-15] at ($(D)+(.11*\m,.5*\m)$) {day};}
 \foreach \m in {0,1,2,...,10}{\pic[rotate=5] at ($(B)+(-.15*\m,.5*\m)$) {day};}
 \foreach \m in {0,1,2,...,12}{\pic[rotate=18] at ($(C)+(-.25*\m,.5*\m)$) {day};}
 \foreach \x/\y in {A/180,B/150,C/-45,D/-90,S/90}
 \fill[black] (\x) circle (4pt) ($(\y:7mm)+(\x)$) node {$\x$};
 \foreach \i/\j [count =\k from 1] in {A/180,B/245,C/0,D/-65}{\draw[->] (S)-- ($(S)!0.6!(\i)$) coordinate(\i') node[midway, shift={(\j:5mm)}]{$\overrightarrow{F}_\k$};}
 \end{tikzpicture}
 }
 \shortans{212}
 \loigiai{
 \begin{center}
 \begin{tikzpicture}[scale=1, font=\footnotesize, line join=round, line cap=round, >=stealth]
 \def\a{4.5} \def\b{2.2} \def\h{4.5} \def\g{40}
 \path
 (0,0) coordinate (A)
 (0:\a) coordinate (B)++
 (-\g:\b) coordinate (C)++(180:\a) coordinate (D)
 ($(A)!.5!(C)$) coordinate (O)++ (90:\h) coordinate (S)
 ;
 \draw[dashed] (S)--(B)--(A)--(C)--(B)--(D) (S)--(O); 
 \draw (S)--(A)--(D)--(C)--(S)--(D);
 \foreach \i/\j [count =\k from 1] in {A/180,D/210,C/-10}{\draw[->] (S)-- ($(S)!0.65!(\i)$) coordinate(\i') node[midway, shift={(\j:4mm)}]{$\overrightarrow{F}_\k$};}
 \draw[->, dashed] (S)-- ($(S)!0.65!(B)$) coordinate(B') node[midway, shift={(250:4mm)}]{$\overrightarrow{F}_4$};
 \draw (A')--(D')--(C');
 \draw[dashed] (D')--(B')--(A')--(C')--(B');
 \draw[->, dashed] (S)--($(S)!0.65!(O)$) coordinate(O');
 \foreach \x /\g in {A/160,D/-90,C/-20,S/90,O/-90,B/30, A'/160, D'/220, C'/10, B'/40, O'/-55}
 \fill(\x) circle (1pt)($(\x)+(\g:3mm)$) node {$\x$};
 \end{tikzpicture}
 \end{center}
 Gọi $O$ là hình chiếu vuông góc của $S$ trên $(ABCD)$. Ta có $SA=SB=SC=SD$ nên các tam giác vuông $SOA$, $SOB$, $SOC$, $SOD$ bằng nhau. Suy ra $OA=OB=OC=OD$ hay $O$ là tâm hình chữ nhật $ABCD$.\\ 
 Gọi $A'$, $B'$, $C'$, $D'$ lần lượt là các điểm sao cho $\overrightarrow{SA'}=\overrightarrow{F}_1$, $\overrightarrow{SB'}=\overrightarrow{F}_2$, $\overrightarrow{SC'}=\overrightarrow{F}_3$ và $\overrightarrow{SD'}=\overrightarrow{F}_4$.\\
 Vì $\left|\overrightarrow{F}_1\right| = \left|\overrightarrow{F}_2\right| = \left|\overrightarrow{F}_3\right| = \left|\overrightarrow{F}_4\right|$ nên $SA'=SB'=SC'=SD'$. Do đó $S.A'B'C'D'$ là hình chóp có đáy $A'B'C'D'$ là hình chữ nhật.\\
 Gọi $O'$ là tâm hình chữ nhật $A'B'C'D'$, ta có $O'$ thuộc $SO$.\\
 Ta có
 \[\overrightarrow{F}_1 + \overrightarrow{F}_2 + \overrightarrow{F}_3 + \overrightarrow{F}_4 = \overrightarrow{SA'} + \overrightarrow{SB'} + \overrightarrow{SC'} + \overrightarrow{SD'} = \left(\overrightarrow{SA'} + \overrightarrow{SC'}\right) + \left(\overrightarrow{SB'} + \overrightarrow{SD'}\right) =4\overrightarrow{SO'}.\]
 Gọi $\overrightarrow{P}$ là trọng lực của vật nặng và khung sắt. Do vật và khung sắt ở vị trí cân bằng nên
 \[\overrightarrow{P} = \overrightarrow{F}_1 + \overrightarrow{F}_2 + \overrightarrow{F}_3 + \overrightarrow{F}_4 = 4\overrightarrow{SO'}.\]
 Theo giả thiết ta có $\left|\overrightarrow{P}\right|=400+200=600$ N nên $4\left|\overrightarrow{SO'}\right| =600 \Leftrightarrow SO'=150$.\\
 Lại có $\bigl(SA,(ABCD)\bigr)=\widehat{SAO}=\widehat{SA'O'}=45^\circ$.\\
 Vì $\triangle SO'A'$ vuông tại $O'$ nên $SA'=\dfrac{SO'}{\sin 45^{\circ}}=150\sqrt{2}$.\\
 Vậy cường độ của lực căng $\overrightarrow{F}_1$ là $\left|\overrightarrow{F}_1\right| = 150\sqrt{2}\approx 212$ N.
 }
\end{ex}


\Closesolutionfile{ans}
 
\begin{indapan}
	{ans/ans\currfilebase}
\end{indapan}

