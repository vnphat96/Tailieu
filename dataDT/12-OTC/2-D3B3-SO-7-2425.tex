\begin{name}
	{\tenchude}
	{ĐỀ ÔN TẬP CHƯƠNG III}
	{LỚP TOÁN THẦY PHÁT}
	{\thoigian}
\end{name}
\TN
\Opensolutionfile{ans}[ans/ans\currfilebase-Phan-I]
\begin{ex}%[2-D3B3-SO-7-2425]%[VN-MT-7, Nguyễn Kiều Nhã Tú]%[2D3N1-1]
 Cho mẫu số liệu ghép nhóm
 \begin{center}
 \begin{tabular}{|c|c|c|c|c|c|}
 \hline
 Nhóm & $[a_1;a_2)$ & $\ldots$ & $[a_j;a_{j+1})$ & $\ldots$ &$[a_k;a_{k+1})$ \\
 \hline
 Tần số & $m_1$ & $\ldots$ & $m_i$ & $\ldots$ & $m_k$ \\
 \hline
 \end{tabular}
 \end{center}
 trong đó các tần số $m_1 > 0$, $m_k > 0$ và $n = m_1 + \cdots + m_k$ là cỡ mẫu.\\
 Khoảng biến thiên của mẫu số liệu ghép nhóm trên là
 \choice
 {\True $R = a_{k+1} - a_1$}
 {$R = a_k - a_{k+1}$}
 {$R = a_{k+1} + a_1$}
 {$R = a_k + a_{k+1}$}
 \loigiai{
 Khoảng biến thiên của mẫu số liệu ghép nhóm trên là $R = a_{k+1} - a_1$.
 }
\end{ex}

\begin{ex}%[2-D3B3-SO-7-2425]%[VN-MT-7, Nguyễn Kiều Nhã Tú]%[2D3N1-1]
 Cho mẫu số liệu ghép nhóm có tứ phân vị thứ nhất, thứ hai, thứ ba lần lượt là $Q_1$, $Q_2$ và $Q_3$. Khoảng tứ phân vị của mẫu số liệu ghép nhóm đó bằng
 \choice
 {$Q_2 - Q_1$}
 {$Q_1 - Q_3$}
 {\True $Q_3 - Q_1$}
 {$Q_1 - Q_2$}
 \loigiai{
 Khoảng tứ phân vị của mẫu số liệu ghép nhóm là $Q_3 - Q_1$.
 }
\end{ex}

\begin{ex}%[2-D3B3-SO-7-2425]%[VN-MT-7, Nguyễn Kiều Nhã Tú]%[2D3H1-3]
 Một người ghi lại thời gian đàm thoại của một số cuộc gọi cho kết quả như bảng sau:
 \begin{center}
 \begin{tabular}{|c|c|c|c|c|c|}
 \hline
 Thời gian $t$ (phút) & $[0;1)$ & $[1;2)$ & $[2;3)$ & $[3;4)$ & $[4;5)$ \\
 \hline
 Số cuộc gọi & $8$ & $17$ & $25$ & $20$ & $10$ \\
 \hline
 \end{tabular}
 \end{center}
 Khoảng tứ phân vị của mẫu số liệu trên có giá trị bằng
 \choice
 {\True $\dfrac{61}{34}$}
 {$\dfrac{7}{2}$}
 {$\dfrac{29}{17}$}
 {$\dfrac{177}{34}$}
 \loigiai{
 Cỡ mẫu $n = 80$.\\
 Giả sử $x_1$, $x_2$, $\ldots$, $x_{80}$ là thời gian đàm thoại của $80$ cuộc gọi đã được sắp xếp theo thứ tự không giảm.
 \begin{itemize}
 \item Vì $\dfrac{n}{4} = 20$ và $8 < 20 < 8 + 17$ nên tứ phân vị thứ nhất thuộc nhóm $[1;2)$ và có giá trị là
 \[Q_1 = 1 + \dfrac{20-8}{17}\cdot 1 = \dfrac{29}{17}.\]
 \item Lại có $\dfrac{3n}{4} = 60$ và $8+17+25 < 60 < 8 + 17+25+20$ nên tứ phân vị thứ ba thuộc nhóm $[3;4)$ và có giá trị là
 \[Q_3 = 3 + \dfrac{60-(8+17+25)}{20}\cdot 1 = \dfrac{7}{2}.\]
 \end{itemize}
 Vậy khoảng tứ phân vị là $\Delta_{Q}=Q_3-Q_1=\dfrac{7}{2}-\dfrac{29}{17}=\dfrac{61}{34}$.
 }
\end{ex}

\begin{ex}%[2-D3B3-SO-7-2425]%[VN-MT-7, Nguyễn Kiều Nhã Tú]%[1D5H2-3]
 Sau khi kiểm tra sức khoẻ tổng quát, kết quả số cân nặng của học sinh lớp 12A sĩ số $40$ học sinh được thể hiện trong bảng số liệu sau:
 \begin{center}
 \begin{tabular}{|c|c|c|c|c|c|}
 \hline
 Cân nặng (kg)& $[40;50)$ & $[50;60)$ & $[60;70)$ & $[70;80)$ & $[80;90)$ \\
 \hline
 Số học sinh & $7$ & $12$ & $12$ & $7$ & $2$ \\
 \hline
 \end{tabular}
 \end{center}
 Tứ phân vị thứ nhất của mẫu số liệu trên bằng
 \choice
 {$50$}
 {$50{,}5$}
 {\True $52{,}5$}
 {$55{,}5$}
 \loigiai{
 Cỡ mẫu $n = 40$.\\
 Giả sử $x_1$, $x_2$, $\ldots$, $x_{40}$ là cân nặng của $40$ học sinh đã được sắp xếp theo thứ tự không giảm.\\
 Vì $\dfrac{n}{4} = 10$ và $7 < 10 < 7 + 12$ nên tứ phân vị thứ nhất thuộc nhóm $[50;60)$ và có giá trị là \[Q_1 = 50 + \dfrac{10-7}{12}\cdot 10 = 52{,}5.\]
 }
\end{ex}

\begin{ex}%[2-D3B3-SO-7-2425]%[VN-MT-7, Nguyễn Kiều Nhã Tú]%[1D5H2-3]
 Chỉ số ô nhiễm không khí (AQI) tại thủ đô Hà Nội trong tháng $6/2024$ được thống kê vào $10$h$30$ sáng các ngày trong tháng thể hiện trong bảng số liệu sau:
 \begin{center}
 \begin{tabular}{|c|c|c|c|c|c|}
 \hline
 Chỉ số (AQI) & $[130;145)$ & $[145;160)$ & $[160;175)$ &$[175;190)$ & $[190;205)$ \\
 \hline
 Số ngày & $8$ & $7$ & $6$ & $7$ & $2$ \\
 \hline
 \end{tabular}
 \end{center}
 Tứ phân vị thứ ba của mẫu số liệu trên gần nhất với giá trị nào trong các giá trị sau?
 \choice
 {$175$}
 {$176{,}5$}
 {$180{,}2$}
 {\True $178{,}2$}
 \loigiai{
 Cỡ mẫu $n = 30$.\\
 Giả sử $x_1$, $x_2$, $\ldots$, $x_{30}$ là chỉ số (AQI) của $30$ ngày trong tháng $6/2024$ đã được sắp xếp theo thứ tự không giảm.\\
 Vì $\dfrac{3n}{4} =22{,}5 $ và $8+7+6< 22{,}5 < 8+7+6+7$ nên tứ phân vị thứ ba thuộc nhóm $[175;190)$ và có giá trị là
 \[Q_3 = 175 + \dfrac{22,5-(8+7+6)}{7}\cdot 15 \approx 178{,}2.\]
 }
\end{ex}

\begin{ex}%[2-D3B3-SO-7-2425]%[VN-MT-7, Nguyễn Kiều Nhã Tú]%[2D3N1-2]
 Trong kì thi chọn học sinh giỏi ở cụm trường THPT A, môn Toán có $25$ học sinh tham gia kết quả điểm bài thi của học sinh được thể hiện trong bảng sau:
 \begin{center}
 \begin{tabular}{|c|c|c|c|c|c|}
 \hline
 Điểm bài thi & $[10;12)$ & $[12;14)$ & $[14;16)$ & $[16;18)$ & $[18;20)$ \\
 \hline
 Số lần & $4$ & $6$ & $8$ & $4$ & $3$ \\
 \hline
 \end{tabular}
 \end{center}
 Khoảng biến thiên của mẫu số liệu ghép nhóm nhận giá trị nào trong các giá trị dưới đây?
 \choice
 {$18,5$}
 {$10,5$}
 {$8$}
 {\True $10$}
 \loigiai{
 Khoảng biến thiên của mẫu số liệu là $20 - 10 = 10$.
 }
\end{ex}

\begin{ex}%[2-D3B3-SO-7-2425]%[VN-MT-7, Nguyễn Kiều Nhã Tú]%[1D5H2-3]
 Đo cân nặng $40$ học sinh lớp 12A ta được bảng số liệu như sau:
 \begin{center}
 \begin{tabular}{|c|c|c|c|c|c|c|c|}
 \hline
 Khối lượng (kg) & $[40;45)$ & $[45;50)$ & $[50;55)$ & $[55;60)$ & $[60;65)$ & $[65;70)$ & $[70;75)$ \\
 \hline
 Số học sinh & $4$ & $13$ & $7$ & $5$ & $6$ & $2$ & $1$ \\
 \hline
 \end{tabular}
 \end{center}
 Tứ phân vị thứ nhất của mẫu số liệu ghép nhóm thuộc khoảng nào sau đây?
 \choice
 {$[40;45)$}
 {\True $[45;50)$}
 {$[50;55)$}
 {$[55;60)$}
 \loigiai{
 Cỡ mẫu $n = 40$.\\
 Giả sử $x_1$, $x_2$, $\ldots$, $x_{40}$ là cân nặng của $40$ học sinh lớp 12A đã được sắp xếp theo thứ tự không giảm.\\
 Vì $\dfrac{n}{4} =10$ và $4< 10 < 4+13$ nên tứ phân vị thứ nhất thuộc nhóm $[45;50)$.
 }
\end{ex}

\begin{ex}%[2-D3B3-SO-7-2425]%[VN-MT-7, Nguyễn Kiều Nhã Tú]%[1D5H1-3]
 Thống kê điểm thi đánh giá năng lực của một trường THPT qua thang điểm $120$ môn Toán như sau:
 \begin{center}
 \begin{tabular}{|c|c|c|c|c|c|}
 \hline
 Điểm & $[0;20)$ & $[20;40)$ & $[40;60)$ & $[60;80)$ & $[80;100)$ \\
 \hline
 Số học sinh & $25$ & $35$ & $37$ & $15$ & $8$ \\
 \hline
 \end{tabular}
 \end{center}
 Điểm trung bình của tất cả các học sinh tham gia dự thi thuộc khoảng nào sau đây?
 \choice
 {\True $(40;45)$}
 {$(45;50)$}
 {$(50;55)$}
 {$(55;60)$}
 \loigiai{
 Ta có bảng sau:
 \begin{center}
 \begin{tabular}{|c|c|c|c|c|c|}
 \hline
 Điểm & $[0;20)$ & $[20;40)$ & $[40;60)$ & $[60;80)$ & $[80;100)$ \\
 \hline
 Giá trị đại diện & $10$ & $30$ & $50$ & $70$ & $90$ \\
 \hline
 Số học sinh & $25$ & $35$ & $37$ & $15$ & $8$ \\
 \hline
 \end{tabular}
 \end{center}
 Điểm trung bình của các thí sinh dự thi là
 \[\overline{x} = \dfrac{25 \cdot 10 + 35 \cdot 30 + 37 \cdot 50 + 15 \cdot 70 + 8 \cdot 90}{120} = 41.\]
 Do đó, điểm trung bình của tất cả các học sinh tham gia dự thi thuộc khoảng $(40;45)$.
 }
\end{ex}

\begin{ex}%[2-D3B3-SO-7-2425]%[VN-MT-7, Nguyễn Kiều Nhã Tú]%[2D3H2-2]
 Đo chiều cao các em học sinh khối $10$ ta thu được kết quả trong bảng sau:
 \begin{center}
 \begin{tabular}{|c|c|c|c|c|c|c|}
 \hline
 Chiều cao (cm) & $[150;152)$ & $[152;154)$ & $[154;156)$ & $[156;158)$ & $[158;160)$ & $[160;162]$ \\
 \hline
 Số học sinh & $5$ & $18$ & $40$& $26$ & $8$ & $3$ \\
 \hline
 \end{tabular}
 \end{center}
 Tính phương sai của mẫu số liệu ghép nhóm trên (làm tròn kết quả đến hàng phần mười).
 \choice
 {$4{,}5$}
 {$5{,}6$}
 {\True $4{,}7$}
 {$4{,}8$}
 \loigiai{
 Ta có bảng sau:
 \begin{center}
 \begin{tabular}{|c|c|c|c|c|c|c|}
 \hline
 Chiều cao (cm) & $[150;152)$ & $[152;154)$ & $[154;156)$ & $[156;158)$ & $[158;160)$ & $[160;162]$ \\
 \hline
 Giá trị đại diện & $151$ &$153$ & $155$& $157$ & $159$ & $161$ \\
 \hline
 Số học sinh & $5$ & $18$ & $40$& $26$ & $8$ & $3$ \\
 \hline
 \end{tabular}
 \end{center}
 Giá trị trung bình của mẫu số liệu là
 \[\overline{x} = \dfrac{5 \cdot 151 + 18 \cdot 153 + 40 \cdot 155 + 26 \cdot 157 + 8 \cdot 159 + 3 \cdot 161}{100} = 155{,}46.\]
 Khi đó phương sai của mẫu số liệu là
% \[s^2 =\dfrac{5(151-155{,}46)^2 + 18(153-155{,}46)^2 + \ldots + 3(161-155{,}46)^2}{100}=4{,}7084.\]
 \[s^2 = \dfrac{5 \cdot 151^2 + 18 \cdot 153^2 + 40 \cdot 155^2 + 26 \cdot 157^2 + 8 \cdot 159^2 + 3 \cdot 161^2}{100} -(155{,}46)^2 = 4{,}7084.\]
 }
\end{ex}

\begin{ex}%[2-D3B3-SO-7-2425]%[VN-MT-7, Nguyễn Kiều Nhã Tú]%[2D3N2-1]
 Số đặc trưng nào sau đây \textbf{không sử dụng} để đo mức độ phân tán của mẫu số liệu ghép nhóm?
 \choice
 {Khoảng biến thiên}
 {\True Trung vị}
 {Phương sai}
 {Khoảng tứ phân vị}
 \loigiai{
 \begin{itemize}
 \item Khoảng biến thiên của mẫu số liệu ghép nhóm xấp xỉ cho khoảng biến thiên của mẫu số liệu gốc. Khoảng biến thiên được dùng để đo mức độ phân tán của mẫu số liệu ghép nhóm. Khoảng biến thiên càng lớn thì mẫu số liệu càng phân tán.
 \item Trung vị của mẫu số liệu ghép nhóm xấp xỉ cho trung vị của mẫu số liệu gốc, nó chia mẫu số liệu thành hai phần, mỗi phần chứa $50$\% giá trị. Vậy trung vị không thể hiện mức độ phân tán.
 \item Phương sai của mẫu số liệu ghép nhóm xấp xỉ cho phương sai của mẫu số liệu gốc. Phương sai được dùng để đo mức độ phân tán của mẫu số liệu ghép nhóm xung quanh số trung bình của mẫu số liệu đó. Phương sai càng lớn thì mẫu số liệu càng phân tán.
 \item Khoảng tứ phân vị của mẫu số liệu ghép nhóm xấp xỉ cho khoảng tứ phân vị của mẫu số liệu gốc. Khoảng tứ phân vị được dùng để đo mức độ phân tán của mẫu số liệu ghép nhóm. Khoảng tứ phân vị càng lớn thì mẫu số liệu càng phân tán.
 \end{itemize}
 }
\end{ex}

\begin{ex}%[2-D3B3-SO-7-2425]%[VN-MT-7, Nguyễn Kiều Nhã Tú]%[2D3N2-1]
 Ý nghĩa của độ lệch chuẩn đối với mẫu số liệu ghép nhóm là
 \choice
 {\True dùng độ lệch chuẩn của mẫu số liệu để ước lượng độ lệch chuẩn xung quanh số trung bình của mẫu số liệu đó}
 {cho biết về ý nghĩa trung tâm của mẫu số liệu và cả về độ tán xạ dữ liệu của mẫu số liệu}
 {chia mẫu số liệu thành hai phần, mỗi phần chứa $50$\% giá trị}
 {chia mẫu số liệu thành bốn phần, mỗi phần chứa $25$\% giá trị}
 \loigiai{
 Ý nghĩa độ lệch chuẩn của mẫu số liệu ghép nhóm: Độ lệch chuẩn của mẫu số liệu ghép nhóm xấp xỉ cho độ lệch chuẩn của mẫu số liệu gốc. Độ lệch chuẩn được dùng để đo mức độ phân tán của mẫu số liệu ghép nhóm xung quanh số trung bình của mẫu số liệu đó. Độ lệch chuẩn càng lớn thì mẫu số liệu càng phân tán.
 }
\end{ex}

\begin{ex}%[2-D3B3-SO-7-2425]%[VN-MT-7, Nguyễn Kiều Nhã Tú]%[2D3H2-2]
 Quãng đường đi bộ tập thể dục mỗi ngày (đơn vị: km) của bác An trong $20$ ngày được thống kê lại ở bảng sau:
 \begin{center}
 \begin{tabular}{|c|c|c|c|c|c|}
 \hline
 Quãng đường (km) & $[2{,}2;2{,}6)$ & $[2{,}6;3{,}0)$ & $[3{,}0;3{,}4)$ & $[3{,}4;3{,}8)$ & $[3{,}8;4{,}2)$ \\
 \hline
 Tần số & $3$ & $6$ & $5$ & $5$ & $1$ \\
 \hline
 \end{tabular}
 \end{center}
 Độ lệch chuẩn của mẫu số liệu trên có giá trị xấp xỉ bằng
 \choice
 {$3{,}1$}
 {$0{,}042$}
 {$0{,}206$}
 {\True $0{,}45$}
 \loigiai{
 Ta có bảng sau:
 \begin{center}
 \begin{tabular}{|c|c|c|c|c|c|}
 \hline
 Quãng đường (km) & $[2{,}2;2{,}6)$ & $[2{,}6;3{,}0)$ & $[3{,}0;3{,}4)$ & $[3{,}4;3{,}8)$ & $[3{,}8;4{,}2)$ \\
 \hline
 Giá trị đại diện &$2{,}4$&$2{,}8$&$3{,}2$&$3{,}6$&$4{,}0$\\
 \hline
 Tần số & $3$ & $6$ & $5$ & $5$ & $1$ \\
 \hline
 \end{tabular}
 \end{center}
 Số trung bình của mẫu số liệu ghép nhóm là
 \[\overline{x} = \dfrac{3 \cdot 2{,}4 + 6 \cdot 2{,}8 + 5 \cdot 3{,}2 + 5 \cdot 3{,}6 + 1 \cdot 4{,}0}{20} = 3{,}1.\]
 Phương sai của mẫu số liệu ghép nhóm là
 \[s^2 = \dfrac{3 \cdot (2{,}4 - 3{,}1)^2 + 6 \cdot (2{,}8 - 3{,}1)^2 + 5 \cdot (3{,}2 - 3{,}1)^2 + 5 \cdot (3{,}6 - 3{,}1)^2 + 1 \cdot (4{,}0 - 3{,}1)^2}{20} = 0{,}206.\]
 Độ lệch chuẩn của mẫu số liệu ghép nhóm là
 \[s = \sqrt{0{,}206} \approx 0{,}45.\]
 }
\end{ex}
\Closesolutionfile{ans}

\TNTF
\Opensolutionfile{ans}[ans/ans\currfilebase-Phan-II]

\begin{ex}%[2-D3B3-SO-7-2425]%[VN-MT-7, Nguyễn Kiều Nhã Tú]%[2D3H1-4]
 Thành tích chạy $50$\,m của $30$ em học sinh lớp $10$ trường THPT A (đơn vị: giây) được thống kê như bảng sau:
 \begin{center}
 \begin{tabular}{*{6}{p{2cm}}}
 $6{,}3$ & $6{,}2$ & $6{,}5$ & $6{,}8$ & $6{,}9$ & $8{,}2$ \\
 $6{,}6$ & $6{,}7$ & $7{,}0$ & $7{,}1$ & $7{,}2$ & $8{,}3$ \\
 $7{,}4$ & $7{,}3$ & $7{,}2$ & $7{,}1$ & $7{,}0$ & $8{,}4$ \\
 $7{,}1$ & $7{,}3$ & $7{,}5$ & $7{,}5$ & $7{,}6$ & $8{,}7$ \\
 $7{,}6$ & $7{,}7$ & $7{,}8$ & $7{,}5$ & $7{,}7$ & $7{,}8$. \\
 \end{tabular}
 \end{center}
 \choiceTF
 {\True Tần số của nhóm $[7,0;7,5)$ là $10$}
 {Trung bình mỗi em chạy $50$\,m hết số thời gian là $7{,}5$ (giây)}
 {Khoảng biến thiên của mẫu số liệu ghép nhóm trên là $R=3{,}1$}
 {\True Khoảng tứ phân vị của mẫu số liệu ghép nhóm trên là $\Delta_{Q}=0{,}781$}
 \loigiai{
 \begin{itemchoice}
 \itemch \textbf{Đúng}.\\
 Bảng tần số ghép nhóm của mẫu số liệu trên là
 \begin{center}
 \begin{tabular}{|c|c|c|c|c|c|c|}
 \hline
 Thời gian chạy (giây) & $[6{,}0;6{,}5)$ & $[6{,}5;7{,}0)$ & $[7{,}0;7{,}5)$ & $[7{,}5;8{,}0)$ & $[8{,}0;8{,}5)$ & $[8{,}5;9{,}0)$ \\
 \hline
 Tần số & $2$ & $5$ & $10$ & $9$ & $3$ & $1$ \\
 \hline
 \end{tabular}
 \end{center}
 \itemch \textbf{Sai}.\\
 Tổng số học sinh là $n=30$.\\
 Trung bình mỗi em chạy $50$\,m hết số thời gian là
 \begin{center}
 $\overline{x}=\dfrac{6{,}25 \cdot 2+6{,}75 \cdot 5+7{,}25 \cdot 10+7{,}75 \cdot 9+8{,}25 \cdot 3+8{,}75 \cdot 1}{30}=7{,}4$ (giây).
 \end{center}
 \itemch \textbf{Sai}.\\
 Khoảng biến thiên của mẫu số liệu ghép nhóm trên là $R=9{,}0-6{,}0=3{,}0$.
 \itemch \textbf{Đúng}.
 \begin{itemize}
 \item Cỡ mẫu là $n=30$.\\
 Gọi $x_{1}$,$\ldots$, $x_{30}$ là thời gian chạy $50$\,m của $30$ học sinh và giả sử dãy này đã được sắp xếp theo thứ tự không giảm.\\
 Khi đó, trung vị là $\dfrac{x_{15}+x_{16}}{2}$. Do $2$ giá trị $x_{15}$, $x_{16}$ thuộc nhóm $[7{,}0;7{,}5)$ nên nhóm này chứa trung vị. Do đó
 \[
 M_{\text{e}}=7{,}0+\dfrac{\dfrac{30}{2}-(2+5)}{10} \cdot 0{,}5=7{,}4.
 \]
 Tứ phân vị thứ hai $Q_{2}$ chính là trung vị $M_{\text{e}}$.
 \item Tứ phân vị thứ nhất $Q_{1}$\\
 Nhóm chứa $Q_{1}$ là nhóm $[7{,}0;7{,}5)$. Khi đó $Q_{1}=7{,}0+\dfrac{\dfrac{30}{4}-(2+5)}{10} \cdot 0{,}5=7{,}025$.
 \item Tứ phân vị thứ ba $Q_{3}$\\
 Nhóm chứa $Q_{3}$ là nhóm $[7{,}5;8{,}0)$.\\
 Khi đó $Q_{3}=7{,}5+\dfrac{\dfrac{3\cdot30}{4}-(2+5+10)}{9} \cdot 0{,}5=7{,}80(5) \approx 7{,}806$.
 \item Khoảng tứ phân vị của mẫu số liệu ghép nhóm trên là $\Delta_{Q}=Q_{3}-Q_{1}=7{,}806-7{,}025=0{,}781$.
 \end{itemize}
 \end{itemchoice}
 }
\end{ex}

\begin{ex}%[2-D3B3-SO-7-2425]%[VN-MT-7, Nguyễn Kiều Nhã Tú]%[2D3H2-3]
 Khảo sát thời gian xem điện thoại trong một ngày của một số học sinh khối $12$ thu được mẫu số liệu ghép nhóm sau:
 \begin{center}
 \begin{tabular}{|c|c|c|c|c|c|}
 \hline
 Thời gian (phút) & $[0;20)$ & $[20;40)$ & $[40;60)$ & $[60;80)$ & $[80;100)$ \\
 \hline
 Số học sinh & $4$ & $8$ & $12$ & $10$ & $8$ \\
 \hline
 \end{tabular}
 \end{center}
 \choiceTF
 {\True Tổng số học sinh được khảo sát là $42$}
 {Mốt của mẫu số liệu lớn hơn $54$}
 {\True Khoảng tứ phân vị của mẫu số liệu lớn hơn $38$}
 {\True Phương sai của mẫu số liệu nhỏ hơn $610$}
 \loigiai{
 \begin{itemchoice}
 \itemch \textbf{Đúng}.\\
 Tổng số học sinh được khảo sát là $n=4+8+12+10+8=42$.
 \itemch \textbf{Sai}.\\
 Nhóm có tần số lớn nhất là $[40;60)$.\\
 Mốt của mẫu số liệu là
 \[M_{\text{0}}=40+\dfrac{12-8}{(12-8)+(12-10)} \cdot(60-40) \approx 53{,}3.\]
 \itemch \textbf{Đúng}.\\
 Gọi $x_{1}$, $x_{2}$,$\ldots$, $x_{42}$ là thời gian xem điện thoại trong ngày của $42$ học sinh khối $12$ và giả sử dãy này đã sắp xếp theo thứ tự không giảm.\\
 Khi đó tứ phân vị thứ nhất của mẫu gốc là trung vị của dãy $x_{1}$, $x_{2}$,$\ldots$, $x_{21}$, tức là $x_{11}$. Do đó $Q_{1}$ thuộc nhóm $[20;40)$.\\
 Tứ phân vị thứ ba của mẫu gốc là trung vị của dãy $x_{22}$, $x_{2}$,$\ldots$, $x_{42}$, tức là $x_{32}$. Do đó $Q_{3}$ thuộc nhóm $[60;80)$.\\
 Suy ra $Q_{1}=20+\dfrac{\dfrac{42}{4}-4}{8} \cdot(40-20)=36{,}25$ và
 $Q_{3}=60+\dfrac{\dfrac{3 \cdot 42}{4}-24}{10} \cdot(80-60)=75$.\\
 Khoảng tứ phân vị của mẫu số liệu là $\Delta Q=Q_{3}-Q_{1}=75-36{,}25=38{,}75$.
 \itemch \textbf{Đúng}.\\
 Số trung bình của mẫu số liệu là
 \[
 \overline{x}=\dfrac{4 \cdot 10+8 \cdot 30+12 \cdot 50+10 \cdot 70+8 \cdot 90}{42} \approx 54{,}76.
 \]
 Phương sai của mẫu số liệu là
 \begin{eqnarray*}
 s^{2} & = & \dfrac{4 (10-54{,}76)^{2}+8 (30-54{,}76)^{2}+12 (50-54{,}76)^{2}+10 (70-54{,}76)^{2}+8 (90-54{,}76)^{2}}{42}\\
 & \approx & 605{,}9.
 \end{eqnarray*}
 \end{itemchoice}
 }
\end{ex}

\begin{ex}%[2-D3B3-SO-7-2425]%[VN-MT-7, Nguyễn Kiều Nhã Tú]%[2D3H2-3]
 Một trang trại phân $1\ 000$ quả trứng thành $5$ loại, tùy theo khối lượng (đã được làm tròn) của chúng được thống kê bởi bảng dưới đây:
 \begin{center}
 \begin{tabular}{|c|c|c|c|c|c|}
 \hline
 Khối lượng (gam) & $[30;36)$ & $[36;42)$ & $[42;48)$ & $[48;54)$ & $[54;60)$ \\
 \hline
 Số trứng & $45$ & $190$ & $500$ & $250$ & $15$ \\
 \hline
 \end{tabular}
 \end{center}
 \choiceTF
 {\True Tần suất của khối lượng trứng $[30;36)$ là $19 \%$}
 {Số trung vị của mẫu số liệu là $43$}
 {Khoảng biến thiên của mẫu số liệu $39{,}18$}
 {\True Độ lệch chuẩn của mẫu số liệu là $\dfrac{6 \sqrt{17}}{5}$}
 \loigiai{
 \begin{itemchoice}
 \itemch \textbf{Đúng}.\\
 Tần suất của khối lượng trứng $[30;36)$ là $\dfrac{190}{1\ 000} \cdot 100=19 \%$.
 \itemch \textbf{Sai}.\\
 Nhóm chứa trung vị là nhóm $[42;48)$.
 \[M_{\text{e}}=42+\dfrac{\dfrac{1\ 000}{2}-235}{500} \cdot(48-42)=\dfrac{2\ 259}{50}.\]
 \itemch \textbf{Sai}.\\
 Khoảng biến thiên của mẫu số liệu là $60-30=30$.
 \itemch \textbf{Đúng}.\\
 Ta có bảng sau:
 \begin{center}
 \begin{tabular}{|c|c|c|c|c|c|}
 \hline
 Khối lượng (gam) & $[30 ; 36)$ & $[36 ; 42)$ & $[42 ; 48)$ & $[48 ; 54)$ & $[54 ; 60)$ \\
 \hline
 Số trứng & $45$ & $190$ & $500$ & $250$ & $15$ \\
 \hline
 Giá trị đại diện & $33$ & $39$ & $45$ & $51$ & $57$ \\
 \hline
 \end{tabular}
 \end{center}
 Phương sai là
 \[
 s^{2}=\dfrac{33^{2} \cdot 45+39^{2} \cdot 190+45^{2} \cdot 500+51^{2} \cdot 250+57^{2} \cdot 15}{1000}-45^{2}=24{,}48.
 \]
 Vậy độ lệch chuẩn của mẫu số liệu là $s=\sqrt{24{,}48}=\dfrac{6 \sqrt{17}}{5}$.
 \end{itemchoice}
 }
\end{ex}

\begin{ex}%[2-D3B3-SO-7-2425]%[VN-MT-7, Nguyễn Kiều Nhã Tú]%[2D3V2-3]
 Bảng sau thống kê lại tổng số giờ nắng trong tháng $6$ của các năm từ $2002$ đến $2021$ tại hai trạm quan trắc đặt ở Nha Trang và Quy Nhơn.
 \begin{center}
 \begin{tabular}{|c|c|c|c|c|c|c|}
 \hline
 Số giờ nắng & $[130 ; 160)$ & $[160 ; 190)$ & $[190 ; 220)$ & $[220 ; 250)$ & $[250 ; 280)$ & $[280 ; 310)$ \\
 \hline
 Số liệu ở Nha Trang & $1$ & $1$ & $1$ & $8$ & $7$ & $2$ \\
 \hline
 Số liệu ở Quy Nhơn & $0$ & $1$ & $2$ & $4$ & $10$ & $3$ \\
 \hline
 \end{tabular}
 \end{center}
 \begin{flushright}
 (Nguồn: Tổng cục Thống kê)
 \end{flushright}
 \choiceTF
 {Xét số liệu ở Nha Trang thì khoảng tứ phân vị của mẫu số liệu ghép nhóm là $32{,}64$}
 {\True Nếu so sánh theo khoảng tứ phân vị thì số giờ nắng trong tháng $6$ của Quy Nhơn đồng đều hơn}
 {\True Xét số liệu của Quy Nhơn ta có độ lệch chuẩn của mẫu số liệu ghép nhóm (làm tròn kết quả đến hàng phần trăm) là $30{,}59$}
 {Nếu so sánh theo độ lệch chuẩn thì số giờ nắng trong tháng $6$ của Nha Trang đồng đều hơn}
 \loigiai{
 \begin{itemchoice}
 \itemch \textbf{Sai}.\\
 Cỡ mẫu $n=20$.\\
 Gọi $x_{1}$, $x_{2}$,$\ldots$, $x_{20}$ là mẫu số liệu gốc về số giờ nắng trong tháng $6$ trong $20$ năm của Nha Trang được xếp theo thứ tự không giảm.\\
 Ta có $x_{1} \in[130;160)$; $x_{2} \in[160;190)$; $x_{3} \in[190;220)$; $x_{4}$,$\ldots$, $x_{11} \in[220;250)$; $x_{12}$,$\ldots$,\\
 $x_{18} \in[250;280)$; $x_{19}$, $x_{20} \in[280;310)$.\\
 Tứ phân vị thứ nhất của mẫu số liệu gốc là $\dfrac{1}{2}\left(x_{5}+x_{6}\right) \in[220;250)$. Do đó, tứ phân vị thứ nhất của mẫu số liệu ghép nhóm là \[Q_{1}=220+\dfrac{\dfrac{20}{4}-(1+1+1)}{8}\cdot(250-220)=227{,}5.\]
 Tứ phân vị thứ ba của mẫu số liệu gốc là $\dfrac{1}{2}\left(x_{15}+x_{16}\right) \in[250;280)$. Do đó, tứ phân vị thứ ba của mẫu số liệu ghép nhóm là \[Q_{3}=250+\dfrac{\dfrac{3\cdot20}{4}-(1+1+1+8)}{7}\cdot(280-250)=\dfrac{1870}{7}.\]
 Khoảng tứ phân vị của mẫu số liệu ghép nhóm là $\Delta_{Q}=Q_{3}-Q_{1}\approx 39{,}64$.
 \itemch \textbf{Đúng}.\\
 Gọi $y_{1}$, $y_{2}$,$\ldots$, $y_{50}$ là mẫu số liệu gốc về số giờ nắng trong tháng $6$ trong $20$ năm của Quy Nhơn được xếp theo thứ tự không giảm.\\
 Ta có $y_{1} \in[160;190)$; $y_{2}$, $y_{3} \in[190;220)$; $y_{4}$,$\ldots$, $y_{7} \in[220;250)$; $y_{8} $,$\ldots$, $y_{17} \in[250;280)$;
 $y_{18}$,$\ldots$, $y_{20} \in[280;310)$.\\
 Tứ phân vị thứ nhất của mẫu số liệu gốc là $\dfrac{1}{2}\left(y_{5}+y_{6}\right) \in[220;250)$. Do đó, tứ phân vị thứ nhất của mẫu số liệu ghép nhóm là \[Q_{1}'=220+\dfrac{\dfrac{20}{4}-(1+2)}{4}\cdot(250-220)=235.\]
 Tứ phân vị thứ ba của mẫu số liệu gốc là $\dfrac{1}{2}\left(y_{15}+y_{16}\right) \in[250;280)$. Do đó, tứ phân vị thứ ba của mẫu số liệu ghép nhóm là \[Q_{3}'=250+\dfrac{\dfrac{3\cdot 20}{4}-(1+2+4)}{10}\cdot(280-250)=274.\]
 Khoảng tứ phân vị của mẫu số liệu ghép nhóm là $\Delta_{Q}'=Q_{3}'-Q_{1}'=39$.\\
 Vậy nếu so sánh theo khoảng tứ phân vị thì số giờ nắng trong tháng $6$ của Quy Nhơn đồng đều hơn.
 \itemch \textbf{Đúng}.\\
 Xét số liệu của Nha Trang
 \begin{itemize}
 \item Số trung bình 
 \[\overline{x}_{X}=\dfrac{1\cdot 145+1\cdot 175+1\cdot 205+8\cdot 235+7\cdot 265+2\cdot 295}{20}=242{,}5.\]
 \item Độ lệch chuẩn \[s_{X}=\sqrt{\dfrac{1\cdot 145^{2}+1\cdot 175^{2}+1\cdot 205^{2}+8\cdot 235^{2}+7\cdot 265^{2}+2\cdot 295^{2}}{20}-242{,}5^{2}} \approx 35{,}34.\]
 \end{itemize}
 \itemch \textbf{Sai}.\\
 Xét số liệu của Quy Nhơn
 \begin{itemize}
 \item Số trung bình 
 \[\overline{x}_{Y}=\dfrac{1\cdot 175+2\cdot 205+4\cdot 235+10\cdot 265+3\cdot 295}{20}=253.\]
 \item Độ lệch chuẩn 
 \[s_{Y}=\sqrt{\dfrac{1\cdot 175^{2}+2\cdot 205^{2}+4\cdot 235^{2}+10\cdot 265^{2}+3\cdot 295^{2}}{20}-253^{2}} \approx 30{,}59.\]
 \end{itemize}
 Vậy nếu so sánh theo độ lệch chuẩn thì số giờ nắng trong tháng $6$ của Quy Nhơn đồng đều hơn.
 \end{itemchoice}
 }
\end{ex}
\Closesolutionfile{ans}

\TNSA
\Opensolutionfile{ans}[ans/ans\currfilebase-Phan-III]

\begin{ex}%[2-D3B3-SO-7-2425]%[VN-MT-7, Nguyễn Kiều Nhã Tú]%[2D3N1-2]
 Chỉ số AQI là chỉ số thể hiện chất lượng không khí. Có $5$ thông số ảnh hưởng đến chỉ số AQI là Ozone mặt đất, ô nhiễm phân tử (bụi min PM$2.5$ và PM$10$), CO, NO$_2$, SO$_2$ (với NO$_2$, SO$_2$ là tác nhân gây ra mưa axit). Chỉ số AQI từ $0-50$ là mức tốt, từ $51-100$ là trung bình, từ $101-150$ là không tốt cho các nhóm nhạy cảm, từ $151-200$ là không lành mạnh, từ $201-300$ là rất không tốt, và trên $301$ là rất nguy hiểm. Hà Nội của chúng ta là một trong những thành phố ô nhiễm nhất thế giới. Ngày 5/3/2024 chỉ số AQI của Hà Nội đạt mức $241$ và là thành phố ô nhiễm nhất thế giới ngày hôm đó. Chỉ số AQI của một số các thành phố ngày $24/6/2024$ được cho trong bảng sau:
 \begin{center}
 \begin{tabular}{|c|c|c|c|c|}
 \hline
 Chỉ số AQI& $[0;50)$& $[50;100)$& $[100;150)$& $[150;200)$\\\hline
 Số thành phố& $73$& $47$& $7$& $2$\\\hline
 \end{tabular}
 \end{center}
 Khoảng biến thiên của mẫu số liệu trên là bao nhiêu?
 \par
 \shortans{200}
 \loigiai{Khoảng biến thiên của mẫu số liệu trên là $200-0=200$.}
\end{ex}

\begin{ex}%[2-D3B3-SO-7-2425]%[VN-MT-7, Nguyễn Kiều Nhã Tú]%[2D3V1-3]
 Thống kê lượng khách du lịch đến tỉnh Quảng Ninh từ năm $2007$ đến năm $2023$ (đơn vị: triệu người) cho kết quả như sau:
 \begin{center}
 \begin{tabular}{|c|c|c|c|c|c|c|c|c|}
 \hline
 $3{,}4$& $4{,}2$& $5{,}0$& $5{,}4$& $6{,}2$& $7$& $7{,}5$& $7{,}5$& $7{,}8$\\\hline
 $8{,}3$& $9{,}87$& $12{,}2$& $14$& $8{,}8$& $4{,}4$& $9{,}5$& $15{,}5$&\\\hline
 \end{tabular}
 \end{center}
 Ghép nhóm dãy số liệu trên thành các nhóm có độ dài bằng nhau với nhóm đầu tiên là $[1;5)$ rồi cho biết khoảng tứ phân vị của mẫu số liệu ghép nhóm trên.
 \par
 \shortans[]{4{,}44}
 \loigiai{
 Số lượng khách du lịch đến tỉnh Quảng Ninh được cho dưới bảng sau:
 \begin{center}
 \begin{tabular}{|c|c|c|c|c|}
 \hline
 Lượng khách (triệu người)& $[1;5)$& $[5;9)$& $[9;13)$& $[13;17)$\\\hline
 Số năm& $3$& $9$& $3$& $2$\\\hline
 \end{tabular}
 \end{center}
 Cỡ mẫu là $ n=3+9+3+2=17$.\\
 Gọi $x_1$, $x_2$, $\ldots$, $x_{17}$ là số khách đến Quảng Ninh du lịch và giả sử rằng dãy số liệu gốc này đã được sắp xếp theo thứ tự không giảm. \\
 Tứ phân vị thứ nhất của mẫu số liệu gốc này là $\dfrac{1}{2}\left(x_4+x_5\right)$ nên nhóm chứa tứ phân vị thứ nhất là nhóm $[5;9)$ và ta có
 \[Q_1=5+\dfrac{\dfrac{17}{4}-3}{9}\cdot 4\approx 5{,}56.\]
 Tứ phân vị thứ ba của mẫu số liệu gốc là $\dfrac{1}{2}\left(x_{13}+x_{14}\right)$ nên nhóm chứa tứ phân vị thứ ba là nhóm $[9;13)$ và ta có
 \[Q_3=9+\dfrac{\dfrac{3\cdot 17}{4}-12}{3} \cdot 4=10\]
 Vậy khoảng tứ phân vị của mẫu số liệu ghép nhóm là $\Delta Q=Q_3-Q_1=10-5{,}56=4{,}44$.
 }
\end{ex}

\begin{ex}%[2-D3B3-SO-7-2425]%[VN-MT-7, Nguyễn Kiều Nhã Tú]%[2D3H2-2]
 Chiều dài của $40$ bé sơ sinh $12$ ngày tuổi được chọn ngẫu nhiên ở viện nhi trung ương được nghiên cứu thống kê ở bảng dưới đây:
 \begin{center}
 \begin{tabular}{|c|c|c|c|c|c|c|c|}
 \hline
 Chiều dài (cm)& $[44;46)$& $[46;48)$& $[48;50)$&$[50;52)$& $[52;54)$& $[54;56)$& $[56;58)$\\
 \hline
 Số trẻ& $3$& $3$& $10$&$0$& $15$& $7$& $2$\\
 \hline
 \end{tabular}
 \end{center}
 Tìm độ lệch chuẩn (làm tròn đến hàng phần trăm) của $40$ bé sơ sinh ở bảng thống kê trên.
 \par
 \shortans[]{3{,}28}
 \loigiai{
 Ta có bảng phân bố của mẫu ghép nhóm $40$ bé sơ sinh:
 \begin{center}
 \begin{tabular}{|c|c|c|c|c|c|c|c|}
 \hline
 Chiều dài (cm)& $[44;46)$& $[46;48)$& $[48;50)$&$[50;52)$& $[52;54)$& $[54;56)$& $[56;58)$\\\hline
 Số trẻ& $3$& $3$& $10$&$0$& $15$& $7$& $2$\\\hline
 Chiều dài
 đại diện
 (cm)& $45$& $47$& $49$& $51$& $53$& $55$& $57$\\\hline
 \end{tabular}
 \end{center}
 Chiều dài trung bình của $40$ trẻ là
 \[\overline{x}=\dfrac{45\cdot3+47\cdot3+49\cdot10+51\cdot0+53\cdot15+55\cdot7+57\cdot2}{40}=51{,}5 \text{ (cm)}.\]
 Phương sai của $40$ bé sơ sinh ở bảng thống kê trên là
% \[s^2=\frac{3\cdot {{( 51{,}5-45 )}^2}+3\cdot {{( 51{,}5-47 )}^2}+ +2\cdot {{( 57-51{,}5 )}^2}}{40}=10{,}75.\]
 \[s^2=\dfrac{45^2\cdot3+47^2\cdot3+49^2\cdot10+51^2\cdot0+53^2\cdot15+55^2\cdot7+57^2\cdot2}{40}-51{,}5()^2=10{,}75.\]
 Độ lệch chuẩn của $40$ bé sơ sinh ở bảng thống kê trên là $s=\sqrt{s^2}\approx 3{,}28$.
 }
\end{ex}

\begin{ex}%[2-D3B3-SO-7-2425]%[VN-MT-7, Nguyễn Kiều Nhã Tú]%[2D3H2-2]
 Một công ty bất động sản Đất Vàng thực hiện cuộc khảo sát khách hàng xem họ có nhu cầu mua nhà ở mức giá nào để tiến hành dự án xây nhà ở Thăng Long group sắp tới. Kết quả khảo sát $500$ khách hàng được ghi lại ở bảng sau:
 \begin{center}
 \begin{tabular}{|c|c|c|c|c|c|}
 \hline
 Mức giá
 (triệu đồng)& $[10;14)$& $[14;18)$& $[18;22)$& $[22;26)$& $[26;30)$\\\hline
 Số khách hàng& $75$& $105$& $179$& $96$& $45$\\\hline
 \end{tabular}
 \end{center}
 Độ lệch chuẩn (làm tròn đến hàng phần trăm) của mức giá đất là bao nhiêu?
 \par
 \shortans[]{4{,}64}
 \loigiai{
 Bảng phân bố tần số tần suất của bảng số liệu của công ty bất động sản Đất Vàng như sau:
 \begin{center}
 \begin{tabular}{|c|c|c|c|c|c|}
 \hline
 Mức giá
 (triệu đồng)& $[10;14)$& $[14;18)$& $[18;22)$& $[22;26)$& $[26;30)$\\\hline
 Số khách hàng& $75$& $105$& $179$& $96$& $45$\\\hline
 Mức giá đại diện& $12$& $16$& $20$& $24$& $28$\\\hline
 \end{tabular}
 \end{center}
 Mức giá trung bình của công ty là 
 \[\overline{x}=\dfrac{75 \cdot 12 + 105 \cdot 16 + 179 \cdot 20 + 96 \cdot 24 + 45 \cdot 28}{500}\approx 19{,}45\, \text{(triệu đồng)}.\]
 Phương sai của mức giá là 
 \begin{eqnarray*}
 s^2&=&\dfrac{75 (12-19{,}45)^2+105(16-19{,}45)^2+179 (20-19{,}45)^2 + 96(24-19{,}45)^2 + 45 (28-19{,}45)^2}{500}\\
 &\approx& 21{,}49.
 \end{eqnarray*}
 Độ lệch chuẩn của mức giá $\sqrt {s^2} \approx 4{,}64$.
 }
\end{ex}

\begin{ex}%[2-D3B3-SO-7-2425]%[VN-MT-7, Nguyễn Kiều Nhã Tú]%[2D3H2-2]
 Bạn Minh Nhàn sử dụng điện thoại thông minh để chơi game trong một ngày. Số lần bạn sử dụng điện thoại được thống kê như sau:
 \begin{center}
 \begin{tabular}{|c|c|c|c|c|c|}
 \hline
 Thời gian (đơn vị: h)& $[3; 5)$ & $[5; 7)$ & $[7; 9)$ & $[9; 11)$ & $[11; 13)$ \\
 \hline
 Số lần sử dụng & $2$ & $5$ & $13$ & $8$ & $2$ \\
 \hline
 \end{tabular}
 \end{center}
 Hãy tính tỉ số phần trăm (làm tròn 1 chữ số thập phân) giữa độ lệch chuẩn và giá trị trung bình.
 \par
 \shortans[]{23{,}9}
 \loigiai{
 Ta có bảng sau:
 \begin{center}
 \begin{tabular}{|c|c|c|c|c|c|}
 \hline
 Thời gian (đơn vị: h) & $[3; 5)$ & $[5; 7)$ & $[7; 9)$ & $[9; 11)$ & $[11; 13)$ \\
 \hline
 Giá trị đại diện & $4$ & $6$ & $8$ & $10$ & $12$ \\
 \hline
 Số lần sử dụng & $2$ & $5$ & $13$ & $8$ & $2$ \\
 \hline
 \end{tabular}
 \end{center}
 Xét mẫu số liệu của Minh Nhàn $n=2+5+13+8+2=30$.\\
 Số trung bình của mẫu số liệu ghép nhóm là
 \[
 \overline{x}=\dfrac{2\cdot 4+5\cdot 6+13\cdot 8+8\cdot 10+2\cdot 12}{30}=8{,}2.
 \]
 Phương sai của mẫu số liệu ghép nhóm là
 \[
 s^2=\dfrac{1}{30}\left(2\cdot 4^2+5\cdot 6^2+13\cdot 8^2+8\cdot 10^2+2\cdot 12^2\right)-(8{,}2)^2=3{,}83.
 \]
 Độ lệch chuẩn của mẫu số liệu ghép nhóm là $s=\sqrt{s^2} \approx \sqrt{3{,}83} \approx 1{,}96$.\\
 Vậy tỉ số phần trăm giữa độ lệch chuẩn và giá trị trung bình là
 $\dfrac{1{,}96}{8{,}2}\cdot 100\%\approx 23{,}9\%$.
 }
\end{ex}

\begin{ex}%[2-D3B3-SO-7-2425]%[VN-MT-7, Nguyễn Kiều Nhã Tú]%[2D3H2-2]
 Điều tra chi phí thuê nhà ở hằng tháng của một số nhân viên độc thân, công ty $X$ thu được số liệu dưới đây:
 \begin{center}
 \begin{tabular}{|l|c|c|c|c|c|}
 \hline \begin{tabular}{l}
 {Tiền thuê nhà} \\
 {(trăm nghìn đồng)}
 \end{tabular}
 & $[3 ; 6)$ & $[6 ; 9)$ & $[9 ; 12)$ & $[12 ; 15)$ & $[15 ; 18)$ \\
 \hline {Số nhân viên} & $64$ & $40$ & $84$ & $56$ & $16$ \\
 \hline
 \end{tabular}
 \end{center}
 Tính độ lệch chuẩn chi phí thuê nhà hằng tháng của những nhân viên được điều tra (kết quả làm tròn đến hàng phần chục).
 \par
 \shortans[]{3{,}68}
 \loigiai{
 Bổ sung thêm các giá trị đại diện, ta có bảng sau:
 \begin{center}
 \begin{tabular}{|l|c|c|c|c|c|}
 \hline \begin{tabular}{l}
 {Tiền thuê nhà} \\
 {(trăm nghìn đồng)}
 \end{tabular}
 & $[3 ; 6)$ & $[6 ; 9)$ & $[9 ; 12)$ & $[12 ; 15)$ & $[15 ; 18)$ \\
 \hline { Giá trị đại diện} & $4{,}5$ & $7{,}5$ & $10{,5}$ & $13{,}5$ & $16{,}5$ \\
 \hline { Số nhân viên} & $64$ & $40$ & $84$ & $56$ & $16$ \\
 \hline
 \end{tabular}
 \end{center}
 Từ mẫu số liệu đã cho, ta tính được số trung bình là
 \[\overline{x}=\dfrac{4{,}5\cdot 64 + 7{,}5\cdot 40 + 10{,}5\cdot 84 + 13{,}5 \cdot 56 + 16{,}5\cdot 16}{64+40+84+56+16}\approx9{,}58.\]
 Từ đó ta có phương sai chi phí thuê nhà hàng tháng của những nhân viên được điều tra là
 \begin{eqnarray*}
 s^2&=&\dfrac{64 \left(4{,}5-\overline{x}\right)^2+40\left(7{,}5-\overline{x}\right)^2+84\left(10{,}5-\overline{x}\right)^2+56\left(13{,}5-\overline{x}\right)^2+16 \left(16{,}5-\overline{x}\right)^2}{64+40+84+56+16}\\
 &\approx& 13{,}55.
 \end{eqnarray*}
 Vậy độ lệch chuẩn chi phí thuê nhà hàng tháng của những nhân viên được điều tra là
 \[s=\sqrt{s^2}\approx 3{,}68.\]
 }
\end{ex}
\Closesolutionfile{ans}
% \begin{indapan}
% 	{ans/ans\currfilebase}
% \end{indapan}

