\begin{name}
 {Biên soạn:Nguyễn Tài Tuệ \\ Phản biện: Bùi Văn Lợi}
 {Đề ôn tập chương III}
\end{name}

\TN
\Opensolutionfile{ans}[ans/ans\currfilebase-Phan-I]
\begin{ex}%[2-D3B3-SO-9-2425]%[VN-MT-7, Nguyễn Tài Tuệ]%[2D3N1-1]
\immini{Cho mẫu số liệu ghép nhóm được cho trong bảng bên. 
Gọi $ Q_1$, $Q_2$, $Q_3$ lần lượt là tứ phân vị thứ nhất, tứ phân vị thứ hai và tứ phân vị thứ ba của mẫu số liệu. Khoảng tứ phân vị của mẫu số liệu trên là
\choice
{\True $\Delta_Q=Q_3-Q_1$}
{$\Delta_Q=Q_3-Q_2$}
{$\Delta_Q=Q_2-Q_1$}
{$\Delta_Q=Q_3-\dfrac{3}{2}{Q_1}$}}{
\begin{tabular}{|c|c|}
 \hline Nhóm & Tần số \\
 \hline $[a_1; a_2)$ & $n_1$ \\
 \hline$[a_2; a_3)$ & $n_2$ \\
 \hline$\ldots$ & $\ldots$ \\
 \hline$[a_m; a_{m+1})$ & $n_m$\\
 \hline & $n$ \\
 \hline
\end{tabular}}
\loigiai{
Theo định nghĩa $\Delta_Q=Q_3-Q_1$.}
\end{ex}

\begin{ex}%[2-D3B3-SO-9-2425]%[VN-MT-7, Nguyễn Tài Tuệ]%[2D3N1-1]
\immini{Cho mẫu số liệu ghép nhóm được cho trong bảng bên. 
Khoảng biến thiên của mẫu số liệu trên là
\choice
{$R=a_m-a_1$}
{$R=a_{m+1}-a_m$}
{$R=a_{m+1}-a_2$}
{\True $R=a_{m+1}-a_1$}}{

\begin{tabular}{|c|c|}
 \hline Nhóm & Tần số \\
 \hline $[a_1 ; a_2)$ & $n_1$ \\
 \hline $[a_2 ; a_3)$ & $n_2$ \\
 \hline $\ldots$ & $\ldots$ \\
 \hline $[a_m ; a_{m+1})$ & $n_m$\\
 \hline & $n$ \\
 \hline
\end{tabular}}
\loigiai{
Ta có $R=a_{m+1}-a_1$.
}
\end{ex}

\begin{ex}%[2-D3B3-SO-9-2425]%[VN-MT-7, Nguyễn Tài Tuệ]%[2D3N1-2]
\immini{ Bảng thống kê chiều cao của $40$ mẫu cây ở một vườn thực vật (đơn vị: centimét) được cho trong bảng như hình bên. 
Khoảng biến thiên của mẫu số liệu trên bằng
\choice
{\True $R=60$}
{$R=50$}
{$R=70$}
{$R=10$}}{\begin{tabular}{|c|c|c|}
 \hline Nhóm & Tần số & Tần số tích lūy \\
 \hline$[30 ; 40)$ & $4$ & $4$ \\
 $[40 ; 50)$ & $10$ & $14$ \\
 $[50 ; 60)$ & $14$ & $28$ \\
 $[60 ; 70)$ & $6$ & $34$ \\
 $[70 ; 80)$ & $4$ & $38$ \\
 $[80 ; 90)$ & $2$ & $40$ \\
 \hline & $n=40$ & \\
 \hline
\end{tabular}}
\loigiai{
Khoảng biến thiên của mẫu số liệu là $R=90-30=60$.}
\end{ex}

\begin{ex}%[2-D3B3-SO-9-2425]%[VN-MT-7, Nguyễn Tài Tuệ]%[1D5H2-3]
Thời gian (phút) truy bài trước mỗi buổi học của một số học sinh trong một tuần được ghi lại ở bảng sau:
 \begin{center}
\begin{tabular}{|c|c|c|c|c|c|}
\hline
Thời gian & $[9{,}5 ; 12{,}5)$ & $[12{,}5 ; 15{,}5)$ & $[15{,}5 ; 18{,}5)$ & $[18{,}5 ; 21{,}5)$ & $[21{,}5 ; 24{,}5)$ \\
\hline
Số học sinh & $3$ & $12$ & $15$ & $24$ & $2$ \\
\hline
\end{tabular}
\end{center}
Nhóm chứa tứ phân vị thứ nhất là
\choice
{$[9{,}5;12{,}5)$}
{\True $[12{,}5;15{,}5)$}
{$[15{,}5;18{,}5)$}
{$[18{,}5;21{,}5)$}
\loigiai{
Cỡ mẫu $n=56$.\\
Gọi $x_1$, $x_2,\ldots,x_{56} $ là mẫu số liệu gốc về thời gian truy bài trước mỗi buổi học của $56$ số học sinh trong một tuần được xếp theo thứ tự không giảm.\\
Ta có $\dfrac{x_{14}+x_{15}}{2}\in [12{,}5;15{,}5)$, nên nhóm chứa tứ phân vị thứ nhất là $[12{,}5;15{,}5) $.}
\end{ex}

\begin{ex}%[2-D3B3-SO-9-2425]%[VN-MT-7, Nguyễn Tài Tuệ]%[1D5H2-3]
Khảo sát thời gian tập thể dục trong ngày của một số học sinh khối $11$ thu được mẫu số liệu ghép nhóm sau:
\begin{center}
\begin{tabular}{|c|c|c|c|c|c|}
\hline
Thời gian (phút) & $[0 ; 20)$ & $[20 ; 40)$ & $[40 ; 60)$ & $[60 ; 80)$ & $[80 ; 100)$ \\
\hline
Số học sinh & $5$ & $9$ & $12$ & $10$ & $6$ \\
\hline
\end{tabular}
\end{center}
Nhóm chứa tứ phân vị thứ ba là
\choice
{$[20 ;40)$}
{$[40 ;60)$}
{\True $[60 ;80)$}
{$[80 ;100)$}
\loigiai{
Cỡ mẫu $n=42$.\\
Gọi $x_1$, $x_2,\ldots, x_{42} $ là mẫu số liệu gốc về thời gian tập thể dục trong ngày của $42$ học sinh khối $11$ được xếp theo thứ tự không giảm.\\
Tứ phân vị thứ ba của mẫu số liệu gốc là $x_{32}$ thuộc nhóm $[60 ;80)$.}
\end{ex}

\begin{ex}%[2-D3B3-SO-9-2425]%[VN-MT-7, Nguyễn Tài Tuệ]%[1D5H2-3]
Cho mẫu số liệu ghép nhóm về thời gian (phút) đi từ nhà đến nơi làm việc của các nhân viên của một công ty như sau:
\begin{center}
\begin{tabular}{|c|c|c|c|c|c|c|c|}
\hline
Thời gian & $[15 ; 20)$ & $[20 ; 25)$ & $[25 ; 30)$ & $[30 ; 35)$ & $[35 ; 40)$ & $[40 ; 45)$ & $[45 ; 50)$ \\
\hline
Số nhân viên & $7 $& $14$ & $25$ & $37$ & $21$ & $14$ & $10$ \\
\hline
\end{tabular}
\end{center}
Tứ phân vị thứ nhất $Q_1$ và tứ phân vị thứ ba $Q_3$ của mẫu số liệu ghép nhóm này là
\choice
{\True $Q_1=\dfrac{136}{5}$, $Q_3=\dfrac{800}{21}$}
{$Q_1=\dfrac{1360}{37}$, $Q_3=\dfrac{800}{21}$}
{$Q_1=\dfrac{1360}{37}$, $Q_3=\dfrac{3280}{83}$}
{$Q_1=\dfrac{136}{5}$, $Q_3=\dfrac{3280}{83}$}
\loigiai{
Cỡ mẫu $n=128$.\\
Gọi $x_1$, $x_2,\ldots, x_{128} $ là mẫu số liệu gốc về thời gian đi từ nhà đến nơi làm việc của các nhân viên của một công ty được xếp theo thứ tự không giảm.\\
Tứ phân vị thứ nhất của mẫu số liệu gốc là $\dfrac{x_{32}+x_{33}}{2} \in [25;30)$.\\
Do đó, tứ phân vị thứ nhất của mẫu số liệu ghép nhóm là
\[Q_1= 25+ \dfrac{\dfrac{128}{4} - (7+14)}{25} \cdot (30-25)
=\dfrac{136}{5}.\]
Tứ phân vị thứ ba của mẫu số liệu gốc là $\dfrac{x_{96}+x_{97}}{2}\in[35 ;40)$.\\
Do đó, tứ phân vị thứ ba của mẫu số liệu ghép nhóm là
\[ Q_3=35+\dfrac{\dfrac{3\cdot 128}{4}-(7+14+25+37)}{21}\cdot (40-35)
=\dfrac{800}{21}.\]
}
\end{ex}

\begin{ex}%[2-D3B3-SO-9-2425]%[VN-MT-7, Nguyễn Tài Tuệ]%[1D5H2-3]
Thời gian (phút) truy cập Internet mỗi buổi tối của một số học sinh được cho trong bảng sau:
\begin{center}
\begin{tabular}{|l|c|c|c|c|c|}
\hline
Thời gian (phút) & $[9{,}5 ; 12{,}5)$ & $[12{,}5 ; 15{,}5)$ & $[15{,}5 ; 18{,}5)$ & $[18{,}5 ; 21{,}5)$ & $[21{,}5 ; 24{,}5)$ \\
\hline Số học sinh & $3$ & $12$ & $15$ & $24$ & $2$ \\
\hline
\end{tabular}
\end{center}
Tìm tứ phân vị thứ nhất $Q_1$.
\choice
{$Q_1=15$}
{$Q_1=15{,}5$}
{$Q_1=15{,}2$}
{\True $Q_1=15{,}25$}
\loigiai{
Cỡ mẫu là $n=56$.\\
Tứ phân vị thứ nhất $Q_1$ là $\dfrac{x_{14}+x_{15}}{2}$. Do $x_{14}$, $x_{15}$ đều thuộc nhóm $[12{,}5;15{,}5)$ nên nhóm này chứa $Q_1$ .\\
Do đó, $p=2$; ${a_2}=12{,}5$; ${m_2}=12$; ${m_1}=3$, $a_3-a_2=3$ và ta có
\[ Q_1=12{,}5+\dfrac{\dfrac{56}{4}-3}{12}\cdot 3=15{,}25.\]
}
\end{ex}

\begin{ex}%[2-D3B3-SO-9-2425]%[VN-MT-7, Nguyễn Tài Tuệ]%[1D5H1-3]
Thống kê cân nặng của học sinh lớp 11A cho trong bảng dưới đây:
\begin{center}
\begin{tabular}{|c|c|c|c|c|c|c|}
\hline
Cân nặng & $[40{,}5 ; 45{,}5)$ & $[45{,}5 ; 50{,}5)$ & $[50{,}5 ; 55{,}5)$ & $[55{,}5 ; 60{,}5)$ & $[60{,}5 ; 65{,}5)$ & $[65{,}5 ; 70{,}5)$\\
\hline
Số học sinh & $10$ & $7$ & $16$ & $4$ & $2$ & $3$ \\
\hline
\end{tabular}
\end{center}
Tính cân nặng trung bình của học sinh lớp 11A (kết quả làm tròn đến hàng phần trăm).
\choice
{$50{,}1$}
{$52{,}83$}
{$50{,}81$}
{\True $51{,}81$}
\loigiai{
Trong mỗi khoảng cân nặng, giá trị đại diện là trung bình cộng của giá trị hai đầu mút nên ta có bảng sau:
\begin{center}
\begin{tabular}{|c|c|c|c|c|c|c|}
\hline
Cân nặng (kg) & $43$ & $48$ & $53$ & $58$ & $63$ & $68$ \\
\hline
Số học sinh & $10$ & $7$ & $16$ & $4$ & $2$ & $3$ \\
\hline
\end{tabular}
\end{center}
Tổng số học sinh là $n=42 $.\\
Cân nặng trung bình của học sinh lớp 11A là \\
\[ \overline{x}=\dfrac{10\cdot 43+7\cdot 48+16\cdot 53+4\cdot 58+2\cdot 63+3\cdot 68}{42}\approx 51{,}81~\text{(kg).}\]
}
\end{ex}

\begin{ex}%[2-D3B3-SO-9-2425]%[VN-MT-7, Nguyễn Tài Tuệ]%[2D3H2-2]
\immini{Tìm phương sai của một mẫu số liệu ghép nhóm cho bởi bảng thống kê như hình bên.
\choice
{$ 13{,}24$}
{$15{,}74$}
{$18{,}84$}
{\True $14{,}84$}}{\begin{tabular}{|c|c|c|}
 \hline Lớp chiều cao & Giá trị đại diện & Tần số \\
 \hline$[150 ; 154)$ & $152$ & $25$ \\
 \hline$[154 ; 158)$ & $156$ & $50$ \\
 \hline$[158 ; 162)$ & $160$ & $200$ \\
 \hline$[162 ; 166)$ & $164$ & $175$ \\
 \hline$[166 ; 170)$ & $168$ & $50$ \\
 \hline
\end{tabular}}
\loigiai{
Ta có chiều cao trung bình \\
\[\overline x=\dfrac{1}{500}(152\cdot 25+156\cdot 50+160\cdot 200+164\cdot 175+168\cdot 50)=161{,}4.\] 
Phương sai của mẫu số liệu ghép nhóm là
\begin{align*}
 s^2=\dfrac{1}{500}& \left[25(152-161{,}4)^2+50(156-161{,}4)^2+200(160-161{,}4)^2+175(164-161{,}4)^2\right. \\
 &\left.+50(168-161{,}4)^2\right] = 14{,}84.
\end{align*}
}
\end{ex}

\begin{ex}%[2-D3B3-SO-9-2425]%[VN-MT-7, Nguyễn Tài Tuệ]%[2D3H2-2]
Kết quả khảo sát thời gian sử dụng liên tục (đơn vị: giờ) từ lúc sạc đầy cho đến khi hết pin của một số máy vi tính cùng loại được thống kê ở bảng sau:
\begin{center}
\begin{tabular}{|c|c|c|c|c|}
\hline
Thời gian sử dụng & $[7{,}2 ; 7{,}4)$ & $[7{,}4 ; 7{,}6)$ & $[7{,}6 ; 7{,}8)$& $[7{,}8 ; 8{,}0)$ \\
\hline
Số máy & $2$ & $4$ & $7$ & $6$ \\
\hline
\end{tabular}
\end{center}
Tính độ lệch chuẩn của mẫu số liệu ghép nhóm (kết quả làm tròn đến hàng phần nghìn).
\choice
{$0{,}192$}
{\True $0{,}194$}
{$0{,}037$}
{$0{,}2$}
\loigiai{
Từ bảng thống kê ta có
\begin{center}
\begin{tabular}{|c|c|c|c|c|}
\hline
Thời gian sử dụng & $[7{,}2 ; 7{,}4)$ & $[7{,}4 ; 7{,}6)$ & $[7{,}6 ; 7{,}8)$ & $[7{,}8 ; 8{,}0)$ \\
\hline
Giá trị đại diện & $7{,}3$ & $7{,}5$ & $7{,}7$ & $7{,}9$ \\
\hline
Số máy & $2$ & $4$ & $7$ & $6$ \\
\hline
\end{tabular}
\end{center}
\noindent
Tổng số máy $n=2+4+7+6=19$.\\
Thời gian sử dụng trung bình của pin là $\overline x=\dfrac{2\cdot 7{,}3+4\cdot 7{,}5+7\cdot 7{,}7+6\cdot 7{,}9}{19}=\dfrac{1459}{190}$.\\
Phương sai của mẫu số liệu là $s^2=\dfrac{1}{19}(2\cdot 7{,}3^2+4\cdot 7{,}5^2+7\cdot 7{,}7^2+6\cdot 7{,}9^2)-\left (\dfrac{1459}{190}\right )^2=\dfrac{338}{9025}$.\\
Độ lệch chuẩn của mẫu số liệu là $s=\sqrt{s^2}=\sqrt{\dfrac{ 338}{9025}} =\dfrac{13 \sqrt{2}}{95}\approx 0,194 $.}
\end{ex}

\begin{ex}%[2-D3B3-SO-9-2425]%[VN-MT-7, Nguyễn Tài Tuệ]%[2D3N1-1]
Đại lượng nào đo độ phân tán của nửa giữa của mẫu số liệu, không bị ảnh hưởng nhiều bởi các giá trị ngoại lệ trong mẫu số liệu?
\choice
{Khoảng biến thiên}
{\True Khoảng tứ phân vị}
{Phương sai}
{Độ lệch chuẩn}
\loigiai{
Khoảng tứ phân vị dùng để đo độ phân tán của nửa giữa của mẫu số liệu, không bị ảnh hưởng nhiều bởi các giá trị ngoại lệ trong mẫu số liệu.}
\end{ex}

\begin{ex}%[2-D3B3-SO-9-2425]%[VN-MT-7, Nguyễn Tài Tuệ]%[2D3N2-1]
Để so sánh mức độ phân tán của các mẫu số liệu ghép nhóm có cùng số trung bình ta dùng đại lượng nào?
\choice
{Khoảng biến thiên}
{Khoảng tứ phân vị}
{Trung vị}
{\True Độ lệch chuẩn}
\loigiai{
Để so sánh mức độ phân tán của các mẫu số liệu ghép nhóm có cùng số trung bình ta dùng phương sai và độ lệch chuẩn.
}
\end{ex}
\Closesolutionfile{ans}

\TNTF
\Opensolutionfile{ans}[ans/ans\currfilebase-Phan-II]
\begin{ex}%[2-D3B3-SO-9-2425]%[VN-MT-7, Nguyễn Tài Tuệ]%[2D3H1-2]
Mẫu số liệu dưới đây ghi lại tốc độ của $40$ ô tô khi đi qua một trạm đo tốc độ (đơn vị: km/h):
\begin{center}
\begin{tabular}{llllllllll}
$48{,}5$ & $43$ & $50$ & $55$ & $45$ & $60$ & $53$ & $55{,}5$ & $44$ & $65$ \\
$51$ & $62{,}5$ & $41$ & $44{,}5$ & $57$ & $57$ & $68$ & $49$ & $46{,}5$ & $53{,}5$ \\
$61$ & $49{,}5$ & $54$ & $62$ & $59$ & $56$ & $47$ & $50$ & $60$ & $61$ \\
$49{,}5$ & $52{,}5$ & $57$ & $47$ & $60$ & $55$ & $45$ & $47{,}5$ & $48$ & $61{,}5$
\end{tabular}
\end{center}
\choiceTF
{\True Bảng tần số ghép nhóm cho mẫu số liệu trên có sáu nhóm ứng với sáu nửa khoảng là: 
\centerline{
\begin{tabular}{|c|c|c|c|c|c|c|c|}
 \hline
 Nhóm & $[40 ; 45)$ & $[45 ; 50)$ & $[50 ; 55)$ & $[55 ; 60)$ & $[60 ; 65)$ & $[65 ; 70)$ & \\
 \hline
 Tần số & $4$ & $11$ & $7$ & $8$ & $8$ & $2$ & $n=40$ \\
 \hline
\end{tabular}
}}
{Mẫu số liệu trên có số trung bình là $54{,}875$}
{\True Tứ phân vị của mẫu số liệu trên là $Q_1=47{,}8~\text{(km/h)}$; ${Q_2}=53{,}6 ~\text{(km/h)}$; ${Q_3}=60~\text{(km/h)}$}
{Khoảng biến thiên của mẫu số liệu trên là $25$}
\loigiai{
\begin{itemchoice}
\itemch {\bf Đúng}.\\
Bảng tần số ghép nhóm: 
\begin{center}
\begin{tabular}{|c|c|c|c|c|c|c|c|}
 \hline
 Nhóm & $[40 ; 45)$ & $[45 ; 50)$ & $[50 ; 55)$ & $[55 ; 60)$ & $[60 ; 65)$ & $[65 ; 70)$ & \\
 \hline
 Tần số & $4$ & $11$ & $7$ & $8$ & $8$ & $2$ & $n=40$ \\
 \hline
\end{tabular}
\end{center}
Vậy bảng tần số đã cho đúng.
\itemch {\bf Sai}.\\
Số trung bình
\[
\overline x= \dfrac{4\cdot 42{,}5+11\cdot 47{,}5+7\cdot 52{,}5+8\cdot 57{,}5+8\cdot 62{,}5+2\cdot 67 \cdot 5}{40}= 53{,}875~\text{(km/h).}
\]
\itemch \textbf{Đúng}.\\
Cỡ mẫu là $n=40$.\\
Ta có $\dfrac{n}{2}=20$ nên nhóm $3$ là nhóm đầu tiên có tần số tích lũy lớn hơn hoặc bằng $20$.
Xét nhóm $3$ là nhóm $[50;55)$ có $r=50$, $d=5$, $n_3=7$ và $c{f_2}=15$.
Số trung vị của mẫu số liệu là
\[
M_e=50+\dfrac{20-15}{7}\cdot 5\approx 53{,}6 ~\text{(km/h)}.
\]
Ta có $\dfrac{n}{4}=10$ nên nhóm $2$ là nhóm đầu tiên có tần số tích lũy lớn hơn hoặc bằng $10$.\\
Xét nhóm $2$ là nhóm $[45;50)$ có $r=45$, $d=5$, $n_2=11$ và $cf_1=4$.\\
Tứ phân vị thứ nhất là 
\[
Q_1=45+\dfrac{10-4}{11}\cdot 5\approx 47{,}8~\text{(km/h).}\]
Ta có $\dfrac{3n}{4}=30$ nên nhóm $4$ là nhóm đầu tiên có tần số tích lũy lớn hơn hoặc bằng $30$.\\
Xét nhóm $4$ là nhóm $[(60)$ có $r=55$, $d=5$, $n_4=8$ và $c{f_3}=22$.\\
Tứ phân vị thứ ba là
\[ Q_3=55+\dfrac{30-22}{8}\cdot 5=60~\text{(km/h).} \]
Vậy các tứ phân vị của mẫu số liệu trên là
$Q_1=47{,}8$ (km/h); ${Q_2}=53{,}6$ (km/h); $ Q_3 =60$ (km/h).
\itemch {\bf Sai}.\\
Khoảng biến thiên của mẫu số liệu là
$R=70-40=30$.
\end{itemchoice}
}
\end{ex}

\begin{ex}%[2-D3B3-SO-9-2425]%[VN-MT-7, Nguyễn Tài Tuệ]%[2D3H2-2]
\immini{Bảng bên cho ta bảng tần số ghép nhóm số liệu thống kê cân nặng của $40$ học sinh lớp 12B trong một trường trung học phổ thông (đơn vị: kilôgam). 
Các mệnh đề sau \textbf{đúng} hay \textbf{sai}?
\choiceTF
{\True Số học sinh nặng dưới $50$ (kg) là $12$}
{\True Mốt của mẫu số liệu ghép nhóm trên xấp xỉ bằng $54{,}29$ (kg)}
{Khoảng tứ phân vị của mẫu số liệu ghép nhóm trên là $\dfrac{39}{2}$}
{Phương sai của mẫu số liệu ghép nhóm là $128$}}{
\begin{tabular}{|c|c|}
 \hline
 Nhóm & Số học sinh\\
 \hline $[30 ;40)$ & $2$\\
 \hline $[40 ;50)$ & $10$\\
 \hline $[50 ;60)$ & $16$\\
 \hline $[60 ;70)$ & $8$\\
 \hline $[70 ;80)$ & $2$\\
 \hline $[80 ;90)$ & $2$\\
 \hline & $n=40$\\
 \hline
\end{tabular}}
\loigiai{
\begin{itemchoice}
\itemch {\bf Đúng}.\\
Số học sinh nặng dưới $50$ kg là $2+10=12$. 
\itemch {\bf Đúng}.\\
Nhóm chứa mốt của mẫu số liệu là $[50 ;60)$.\\
Do đó $u_m=50$, $n_m=16$, $n_{m-1}=10$, $n_{m+1}=8$, $u_{m+1}-u_m=60-50=10$.\\
Mốt của mẫu số liệu ghép nhóm là 
\[
M_\text{o}=50+\dfrac{16-10}{(16-10)+(16-8)}\cdot 10=\dfrac{380}{7}\approx 54{,}29~\text{(kg).}
\]
Mốt của mẫu số liệu ghép nhóm trên xấp xỉ bằng $54{,}29$ (kg). 
\itemch \textbf{Sai}.\\
Cỡ mẫu $n=40$.\\
Gọi $x_1, x_2 \in [30 ;40)$; $x_3,\,\ldots,x_{12} \in [40 ;50)$; $x_{13},\,...,x_{28}\in [50 ;60);$ $x_{29},\,...,x_{36}\in [60 ;70);$ $x_{37},x_{38}\in [70 ;80);$ $x_{39},x_{40}\in [80 ;90) $.\\
Tứ phân vị thứ nhất của mẫu số liệu gốc là
$ \dfrac{1}{2}(x_{10}+x_{11})\in[40 ;50)$. 
Do đó, tứ phân vị thứ nhất của mẫu số liệu ghép nhóm là 
\[
Q_1=40+\dfrac{\dfrac{40}{4}-2}{10} \cdot (50-40)=48.
\]
Tứ phân vị thứ ba của mẫu số liệu gốc là
$\dfrac{1}{2}(x_{30}+x_{31})\in[60 ;70)$.
Do đó, tứ phân vị thứ ba của mẫu số liệu ghép nhóm là 
\[
Q_3=60+\dfrac{\dfrac{3\cdot 40}{4}-(2+10+16)}{8}\cdot (70-60)=\dfrac{125}{2}.
\]
Vậy khoảng tứ phân vị của mẫu số liệu ghép nhóm là 
\[
\Delta_Q=\dfrac{125}{2}-48=\dfrac{29}{2}.
\]
\itemch {\bf Sai}. \\ Ta có bảng cân nặng của các em học sinh theo giá trị đại diện:
\begin{center}
\begin{tabular}{|c|c|c|}
\hline
Nhóm & Giá trị đại diện & Tần số\\
\hline $[30 ;40)$ & $35$ & $2$\\
\hline $[40 ;50)$ & $45$ & $10$\\
\hline $[50 ;60)$ & $55$ & $16$\\
\hline $[60 ;70)$ & $65$ & $8$\\
\hline $[70 ;80)$ & $75$ & $2$\\
\hline $[80 ;90)$ & $85$ & $2$\\
\hline & & $n=40$\\
\hline
\end{tabular}
\end{center}
Cỡ mẫu $n=2+10+16+8+2+2=40 $.\\
Số trung bình của mẫu số liệu ghép nhóm là 
\[
\dfrac{35\cdot 2+45\cdot 10+55\cdot 16+65\cdot 8+75\cdot 2+85\cdot 2}{40}=\dfrac{2240}{40}=56 \text{ (kg).}
\]
Phương sai của mẫu số liệu ghép nhóm là 
\[
s^2=\dfrac{1}{40}\left(2\cdot 35^2+10\cdot 45^2+16\cdot 55^2+8\cdot 65^2+2\cdot 75^2+2\cdot 85^2\right)-56^2=3265-3136=129.
\]
\end{itemchoice}
}
\end{ex}

\begin{ex}%[2-D3B3-SO-9-2425]%[VN-MT-7, Nguyễn Tài Tuệ]%[2D3H2-2]
Trong một hội thao, thời gian chạy $200$\,m của một nhóm các vận động viên được ghi lại ở bảng sau:
\begin{center}
\begin{tabular}{|c|c|c|c|c|c|}
\hline
Thời gian (giây) & $[21 ; 21{,}5)$ & $[21{,}5 ; 22)$ & $[22 ; 22{,}5)$ & $[22{,}5 ; 23)$ & $[23 ; 23{,}5)$\\
\hline
Số vận động viên & $5$ & $10$ & $30$ & $45$ & $30$\\
\hline
\end{tabular}
\end{center}
\choiceTF 
{Tần suất của nhóm vận động viên chạy trong khoảng thời gian từ $22$ giây đến dưới $22{,}5$ giây bằng $30\% $}
{\True Số trung vị của mẫu số liệu (làm tròn đến chữ số thập phân thứ $2$) bằng $22{,}67$}
{Khoảng biến thiên của mẫu số liệu bằng $R=2$}
{Độ lệch chuẩn của mẫu số liệu (làm tròn đến chữ số thập phân thứ $2$) bằng $0{,}28$}
\loigiai{
\begin{itemchoice}
\itemch {\bf Sai}.\\ 
Cỡ mẫu $n=5+10+30+45+30=120$.\\
Tần suất của nhóm vận động viên chạy trong khoảng thời gian từ $22$ giây đến dưới $22{,}5$ giây bằng $f_3=\dfrac{n_3}{n}=\dfrac{30}{120}=25\% $.
\itemch {\bf Đúng}.\\
Gọi $x_1$, $x_2$, $ \ldots$, $x_{120}$ là thời gian chạy của $120$ vận động viên và dãy này là một dãy không giảm.\\
Khi đó trung vị là
$\dfrac{x_{60}+x_{61}}{2}$. Do $x_{60},\,x_{61}\in[22{,}5;\,23)$ nên nhóm này chứa trung vị.\\
Ta có 
$M_\text{e}=22{,}5+\dfrac{\dfrac{120}{2}-(5+10+30)}{45}\cdot(23-22{,}5)\approx 22{,}67$.
\itemch {\bf Sai}. \\
Khoảng biến thiên của mẫu số liệu bằng $R=23{,}5-21=2{,}5$.
\itemch {\bf Sai}. \\
Giá trị trung bình của mẫu số liệu là
\[
\overline x=\dfrac{5\cdot 21{,}25+10\cdot 21{,}75+30\cdot 22{,}25+45\cdot 22{,}75+30\cdot 23{,}25}{120}\approx 22{,}60.
\]
Độ lệch chuẩn của mẫu số liệu (làm tròn đến chữ số thập phân thứ $2$) bằng
\[
s=\sqrt{\dfrac{5(-1{,}35)^2+10(-0{,}85)^2+30(-0{,}35)^2+45(0{,}15)^2+30(0{,}65)^2}{120}}\approx 0{,}53.
\]
\end{itemchoice}
}
\end{ex}

\begin{ex}%[2-D3B3-SO-9-2425]%[VN-MT-7, Nguyễn Tài Tuệ]%[2D3V1-3]
Giả sử kết quả khảo sát hai khu vực A và B về độ tuổi kết hôn của một số phụ nữ vừa lập gia đình được cho ở bảng sau:
\begin{center}
\begin{tabular}{|c|c|c|c|c|c|}
\hline
Tuổi kết hôn & $[19;22)$ & $[22;25)$ & $[25;28)$ & $[28;31)$ & $[31;34)$\\
\hline
Số phụ nữ khu vực A & $10$ & $27$ & $31$ & $25$ & $7$\\
\hline
Số phụ nữ khu vực B & $47$ & $40$ & $11$ & $2$ & $0$\\
\hline
\end{tabular}
\end{center}
\choiceTF
{Khoảng biến thiên của mẫu số liệu ghép nhóm ứng với khu vực A là $15$ (tuổi)}
{Khoảng biến thiên của mẫu số liệu ghép nhóm ứng với khu vực B là $12$ (tuổi)}
{Khoảng tứ phân vị của mẫu số liệu ghép nhóm ứng với khu vực A là $\dfrac{61}{3}$ (tuổi)}
{Nếu so sánh theo khoảng tứ phân vị thì phụ nữ ở khu vực B có độ tuổi kết hôn đồng đều hơn}
\loigiai{
\begin{itemchoice}
\itemch \textbf{Đúng}.\\ 
Khoảng biến thiên của mẫu số liệu ghép nhóm ứng với khu vực A là $34-19=15$ (tuổi). 
\itemch \textbf{Đúng}. \\
Khoảng biến thiên của mẫu số liệu ghép nhóm ứng với khu vực B là $31-19=12$ (tuổi)
\itemch \textbf{Sai}. \\ 
Cỡ mẫu $n=100$.\\
Gọi $x_1$, $x_2$, $\ldots$, $x_{100} $ là mẫu số liệu gốc về độ tuổi kết hôn của phụ nữ ở khu vực A được xếp theo thứ tự không giảm.\\
Ta có $x_1$, $x_2$, $\ldots$, $x_{10}\in[19;22)$; ${x_{11}}$, $\ldots$, $ x_{37} \in[22;25)$; $x_{38},\ldots, x_{68} \in[25;28)$; $x_{69},\ldots,x_{93} \in[28;31)$; $x_{94},\ldots, x_{100}\in[31;34)$.\\
Tứ phân vị thứ nhất của mẫu số liệu gốc là
$\dfrac{1}{2}(x_{25}+x_{26})\in[22;25)$. Do đó, tứ phân vị thứ nhất của mẫu số liệu ghép nhóm là $Q_1=22+\dfrac{\dfrac{100}{4}-10}{27}(25-22)=\dfrac{71}{3}$.\\
Tứ phân vị thứ ba của mẫu số liệu gốc là
$\dfrac{1}{2}(x_{75}+x_{76})\in[28;31)$. Do đó, tứ phân vị thứ ba của mẫu số liệu ghép nhóm là $Q_3=28+\dfrac{\dfrac{3\cdot 100}{4}-(10+27+31)}{25}(31-28)=\dfrac{721}{25}$.\\
Khoảng tứ phân vị của mẫu số liệu ghép nhóm là $\Delta_Q=Q_3-Q_1=\dfrac{388}{75}$.
\itemch \textbf{Đúng}.\\
Gọi $y_1$, $y_2,\ldots,y_{100} $ là mẫu số liệu gốc về độ tuổi kết hôn của phụ nữ ở khu vực B được xếp theo thứ tự không giảm.\\
Ta có $y_1$, $y_2,\ldots,y_{47} \in [19;22)$; $ y_{48},\ldots, y_{87} \in[22;25)$; $y_{88}, \ldots, y_{98} \in[25;30)$;$y_{99}$, ${y_{100}}\in[28;31)$.\\
Tứ phân vị thứ nhất của mẫu số liệu gốc là $\dfrac{1}{2}(y_{25}+y_{26})\in[19;22)$.
Do đó, tứ phân vị thứ nhất của mẫu số liệu ghép nhóm là
\[
Q_1^\prime=19+\dfrac{\dfrac{100}{4}}{47}(22-19)=\dfrac{968}{47}
\]
Tứ phân vị thứ ba của mẫu số liệu gốc là $\dfrac{1}{2}(y_{75}+y_{76})\in[22;25)$. Do đó, tứ phân vị thứ ba của mẫu số liệu ghép nhóm là
\[
Q_3^\prime=22+\dfrac{\dfrac{3\cdot 100}{4}-47}{40}(25-22)=\dfrac{241}{10}
\]
Có $\Delta_Q^\prime < \Delta_Q$ nên phụ nữ ở khu vực B có độ tuổi kết hôn đồng đều hơn.
\end{itemchoice}
}
\end{ex}
\Closesolutionfile{ans}

\TNSA
\Opensolutionfile{ans}[ans/ans\currfilebase-Phan-III]
\begin{ex}%[2-D3B3-SO-9-2425]%[VN-MT-7, Nguyễn Tài Tuệ]%[2D3N1-2]
Thời gian tập luyện trong một ngày (tính theo giờ) của một số vận động viên được ghi lại ở bảng sau:
\begin{center}
\begin{tabular}{|c|c|c|c|c|c|}
\hline
Thời gian tập luyện &$[0; 2)$ &$[2; 4)$ &$[4; 6)$ &$[6; 8)$ &$[8; 10)$\\
\hline
Số vận động viên & $3$ & $8$ & $12$ & $12$ & $4$\\
\hline
\end{tabular}
\end{center}
Hãy tìm khoảng biến thiên cho thời gian tập luyện của các vận động viên.
\shortans[]{10}
\loigiai{
Gọi $R$ là khoảng biến thiên của mẫu số liệu ghép nhóm về thời gian tập luyện trong ngày của các vận động viên. Ta có $R=10-0=10$.}
\end{ex}

\begin{ex}%[2-D3B3-SO-9-2425]%[VN-MT-7, Nguyễn Tài Tuệ]%[2D3H1-3]
Một trang báo điện tử thống kê thời gian người sử dụng đọc thông tin trên trang trong mỗi lần truy cập ở bảng sau:\\
\begin{center}
\begin{tabular}{|c|c|c|c|c|c|}
\hline
Thời gian đọc (phút) & $[0; 2)$ &$[2; 4)$ &$[4; 6)$ &$[6; 8)$ &$[8; 10)$\\
\hline
Số lượt truy cập & $45$ & $34$ & $23$ & $18$ & $5$\\
\hline
\end{tabular}
\end{center}
Hãy tìm khoảng tứ phân vị của mẫu số liệu ghép nhóm trên.
\shortans[]{3{,}89}
\loigiai{
Cỡ mẫu là $n=45+34+23+18+5=125$.\\
Gọi $x_1$, $x_2$,\ldots, $x_{125}$ là thời gian đọc thông tin trên trang báo điện tử của $125$ lượt truy cập và giả sử rằng dãy số liệu gốc này đã được sắp xếp theo thứ tự tăng dần.\\
Tứ phân vị thứ nhất của mẫu số liệu gốc là $\dfrac{1}{2}(x_{31}+x_{32})$ nên nhóm chứa tứ phân vị thứ nhất là nhóm $[0; 2)$.\\
Tứ phân vị thứ nhất của mẫu số liệu ghép nhóm là
\[
Q_1 = 0+\dfrac{\dfrac{1\cdot 125}{4}-0}{45}\cdot (2-0)\approx 1{,}39.
\]
Tứ phân vị thứ ba của mẫu số liệu gốc là $\dfrac{1}{2}(x_{94}+x_{95})$ nên nhóm chứa tứ phân vị thứ nhất là nhóm $[4; 6)$.\\
Tứ phân vị thứ ba của mẫu số liệu ghép nhóm là
\[ Q_3=4+\dfrac{\dfrac{3\cdot 125}{4}-(45+34)}{23}\cdot (6-4) \approx 5{,}28.
\]
Vậy khoảng tứ phân vị của mẫu số liệu ghép nhóm là
\[
\Delta_Q=Q_3-Q_1\approx 5{,}28-1{,}39=3{,}89.
\]
}
\end{ex}

\begin{ex}%[2-D3B3-SO-9-2425]%[VN-MT-7, Nguyễn Tài Tuệ]%[2D3H2-2]
Người ta ghi lại tiền lãi (đơn vị: triệu đồng) của một số nhà đầu tư (với số tiền đầu tư như nhau), khi đầu tư vào hai lĩnh vực A, B cho kết quả như sau:
\begin{center}
\begin{tabular}{|c|c|c|c|c|c|}
\hline
Tiền lãi & $[5;10)$ & $[10;15)$ & $[15;20)$ & $[20;25)$ & $[25;30)$\\
\hline
Số nhà đầu tư vào lĩnh vực A & $2$ & $5$ & $8$ & $6$ & $4$\\
\hline
Số nhà đầu tư vào lĩnh vực B & $8$ & $4$ & $2$ & $5$ & $6$\\
\hline
\end{tabular}
\end{center} 
Tính hiệu phương sai $s_B^2-s_A^2$ cho các mẫu số liệu về tiền lãi của các nhà đầu tư ở hai lĩnh vực này. \par
\shortans[]{47{,}7}
\loigiai{
Ta có mẫu số liệu ghép nhóm với giá trị đại diện là:
\begin{center}
\begin{tabular}{|c|c|c|c|c|c|}
\hline
Tiền lãi & $[5;10)$ & $[10;15)$ & $[15;20)$ & $[20;25)$ & $[25;30)$\\
\hline
Giá trị đại diện & $7{,}5$ & $12{,}5$ & $17{,}5$ & $22{,}5$ & $27{,}5$\\
\hline
Số nhà đầu tư vào lĩnh vực A & $2$ & $5$ & $8$ & $6$ & $4$\\
\hline
Số nhà đầu tư vào lĩnh vực B & $8$ & $4$ & $2$ & $5$ & $6$\\
\hline
\end{tabular}
\end{center}
Tiền lãi trung bình khi đầu tư vào lĩnh vực A là
\[
\overline{x}_A=\dfrac{7{,}5\cdot 2+12{,}5\cdot 5+17{,}5\cdot 8+22{,}5\cdot 6+27{,}5\cdot 4}{2+5+8+6+4}
=18{,}5.
\]
Tiền lãi trung bình khi đầu tư vào lĩnh vực B là
\[
\overline{x}_B=\dfrac{7{,}5\cdot 8+12{,}5\cdot 4+17{,}5\cdot 2+22{,}5\cdot 5+27{,}5\cdot 6}{8+4+2+5+6}
=16{,}9.
\]
Phương sai của mẫu số liệu về tiền lãi khi đầu tư vào lĩnh vực A là
\[
s_A^2=\dfrac{1}{25}\left(7{,}5^2\cdot 2+12{,}5^2\cdot 5+17{,}5^2\cdot 8+22{,}5^2\cdot 6+27{,}5^2\cdot 4\right)-18{,}5^2
=34.
\]
Phương sai của mẫu số liệu về tiền lãi khi đầu tư vào lĩnh vực B là \\
\[
s_B^2=\dfrac{1}{25}\left(7{,}5^2\cdot 8+12{,}5^2\cdot 4+17{,}5^2\cdot 2+22{,}5^2\cdot 5+27{,}5^2\cdot 6\right)-16{,}9^2
=64{,}64.
\]
Do đó $s_B^2-s_A^2 = 47{,}7$.
}
\end{ex}

\begin{ex}%[2-D3B3-SO-9-2425]%[VN-MT-7, Nguyễn Tài Tuệ]%[2D3H2-2]
Thời gian hoàn thành một bài kiểm tra trắc nghiệm của một số học sinh lớp $10$ của hai lớp 10A và 10B được ghi lại ở bảng sau:
\begin{center}
\begin{tabular}{|c|c|c|c|c|c|}
\hline
Thời gian (phút) & $[6;7)$ & $[7;8)$ & $[8;9)$ & $[9;10)$ & $[10;11)$\\
\hline
Học sinh lớp $10A$ & $8$ & $10$ & $13$ & $10$ & $9$\\
\hline
Học sinh lớp $10B$ & $4$ & $12$ & $17$ & $14$ & $3$\\
\hline
\end{tabular}
\end{center} 
Tính hiệu độ lệch chuẩn $s_{10A}-s_{10B}$ (kết quả làm tròn đến hàng phần trăm).
\shortans[]{0{,}29}
\loigiai{
Lập lại mẫu số liệu ghép nhóm theo giá trị đại diện, ta được:
\begin{center}
\begin{tabular}{|c|c|c|c|c|c|}
\hline
Giá trị đại diện & $6{,}5$ & $7{,}5$ & $8{,}5$ & $9{,}5$ & $10{,}5$\\
\hline
Học sinh lớp $10A$ & $8$ & $10$ & $13$ & $10$ & $9$\\
\hline
Học sinh lớp $10B$ & $4$ & $12$ & $17$ & $14$ & $3$\\
\hline
\end{tabular}
\end{center}
Cỡ mẫu $n=50$.
\begin{enumerate}
\item Xét số liệu của lớp $10A$.\\
Số trung bình là
$\overline x_{10A}=\dfrac{8\cdot 6{,}5+10\cdot 7{,}5+13\cdot 8{,}5+10\cdot 9{,}5+9\cdot 10{,}5}{50}=8{,}54$.\\
Độ lệch chuẩn là
$s_{10A}=\sqrt{\dfrac{8\cdot 6{,}5^2+10\cdot 7{,}5^2+13\cdot 8{,}5^2+10\cdot 9{,}5^2+9\cdot 10{,}5^2}{50}-8{,}54^2}\approx 1{,}33$.
\item Xét số liệu của lớp $10B$.\\
Số trung bình là
$\overline x_{10B}=\dfrac{4\cdot 6{,}5+12\cdot 7{,}5+17\cdot 8{,}5+14\cdot 9{,}5+3\cdot 10{,}5}{50}=8{,}5$.\\
Độ lệch chuẩn là $s_{10B}=\sqrt{\dfrac{4\cdot 6{,}5^2+12\cdot 7{,}5^2+17\cdot 8{,}5^2+14\cdot 9{,}5^2+3\cdot 10{,}5^2}{50}-8{,}5^2}\approx 1{,}04$.
\end{enumerate}
Do đó $s_{10A}-s_{10B}\approx 0{,}29$.
}
\end{ex}

\begin{ex}%[2-D3B3-SO-9-2425]%[VN-MT-7, Nguyễn Tài Tuệ]%[2D3H2-2]
Giá đóng cửa của một cổ phiếu là giá của cổ phiếu đó cuối một phiên giao dịch. Bảng sau thống kê giá đóng cửa (đơn vị: nghìn đồng) của hai mã cổ phiếu A và B trong $50$ ngày giao dịch liên tiếp:
\begin{center}
\begin{tabular}{|c|c|c|c|c|c|}
\hline
Giá đóng cửa &$[120;122)$ &$[122;124)$ &$[124;126)$ &$[126;128)$ &$[128;130)$\\
\hline
Số ngày giao dịch của cổ phiếu A & $8$ & $9$ & $12$ & $10$ & $11$\\
\hline
Số ngày giao dịch của cổ phiếu B & $16$ & $4$ & $3$ & $6$ & $21$\\
\hline
\end{tabular} 
\end{center} 
Tính tỉ số $\dfrac{s_B^2}{s_A^2}$ (kết quả làm tròn đến hàng phần trăm).
\shortans[]{1{,}65}
\loigiai{
Ta có bảng thống kê giá đóng cửa theo giá trị đại diện như sau:
\begin{center}
\begin{tabular}{|c|c|c|c|c|c|}
\hline
Giá trị đại diện & $121$ & $123$ & $125$ & $127$ & $129$\\
\hline
Số ngày giao dịch của cổ phiếu A & $8$ & $9$ & $12$ & $10$ & $11$\\
\hline
Số ngày giao dịch của cổ phiếu B & $16$ & $4$ & $3$ & $6$ & $21$\\
\hline
\end{tabular}
\end{center}
\begin{enumerate}
\item Xét mẫu số liệu của cổ phiếu A
\begin{itemize}
\item Số trung bình của mẫu số liệu ghép nhóm là
\[
\overline{x}_A=\dfrac{8\cdot 121+9\cdot 123+12\cdot 125+10\cdot 127+11\cdot 129}{50}=125{,}28.
\]
\item Phương sai của mẫu số liệu ghép nhóm là
\[
s_A^2=\dfrac{1}{50}\left(8\cdot 121^2+9\cdot 123^2+12\cdot 125^2+10\cdot 127^2+11\cdot 129^2)-(125{,}28\right)^2
=7{,}5216.
\]
\end{itemize}
\item Xét mẫu số liệu của cổ phiếu B
\begin{itemize}
\item Số trung bình của mẫu số liệu ghép nhóm là
\[
\overline{x}_B=\dfrac{16\cdot 121+4\cdot 123+3\cdot 125+6\cdot 127+21\cdot 129}{50}
=125{,}48.
\]
\item Phương sai của mẫu số liệu ghép nhóm là\\
\[
s_B^2=\dfrac{1}{50}\left(16\cdot 121^2+4\cdot 123^2+3\cdot 125^2+6\cdot 127^2+21\cdot 129^2\right)-(125{,}48)^2
=12{,}4096.
\]
\end{itemize}
\end{enumerate}
Do đó $\dfrac{s_B^2}{s_A^2}\approx 1{,}65$.
}
\end{ex}

\begin{ex}%[2-D3B3-SO-9-2425]%[VN-MT-7, Nguyễn Tài Tuệ]%[2D3H2-2]
Thầy Niên thống kê lại điểm trung bình cuối năm của các học sinh lớp 10A và 10B ở bảng sau:
\begin{center}
\begin{tabular}{|c|c|c|c|c|c|}
\hline
Điểm trung bình &$[5;6)$ &$[6;7)$ &$[7;8)$ &$[8;9)$ &$[9;10)$\\
\hline
Số học sinh lớp 10A & $1$ & $0$ & $11$ & $22$ & $6$\\
\hline
Số học sinh lớp 10B & $0$ & $6$ & $8$ & $14$ & $12$\\
\hline
\end{tabular}
\end{center}
Tính $s_A-s_B$ (kết quả làm tròn đến hàng phần chục).
\shortans[]{-0{,}2}
\loigiai{
Ta có bảng thống kê điểm trung bình theo giá trị đại diện
\begin{center}
\begin{tabular}{|c|c|c|c|c|c|}
\hline
Giá trị đại diện & $5{,}5$ & $6{,}5$ & $7{,}5$ & $8{,}5$ & $9{,}5$\\
\hline
Số học sinh lớp 10A & $1$ & $0$ & $11$ & $22$ & $6$\\
\hline
Số học sinh lớp 10B & $0$ & $6$ & $8$ & $14$ & $12$\\
\hline
\end{tabular}
\end{center}
\begin{enumerate}
\item Xét mẫu số liệu của lớp 10A
\begin{itemize}
\item Số trung bình của mẫu số liệu ghép nhóm là
\[
\overline{x}_A=\dfrac{1\cdot 5{,}5+0.6{,}5+11\cdot 7{,}5+22\cdot 8{,}5+6\cdot 9{,}5}{40}=8{,}3.
\]
\item Phương sai của mẫu số liệu ghép nhóm là 
\[
s_A^2=\dfrac{1}{40}\left(1\cdot 5{,}5^2+0\cdot 6{,}5^2+11\cdot 7{,}5^2+22\cdot 8{,}5^2+6\cdot 9{,}5^2\right)-(8{,}3)^2
=0{,}61.
\]
\item Độ lệch chuẩn của mẫu số liệu ghép nhóm là $s_A=\sqrt{0{,}61}$.
\end{itemize}
\item Xét mẫu số liệu của lớp 10B
\begin{itemize}
\item Số trung bình của mẫu số liệu ghép nhóm là\\
\[
\overline{x}_B=\dfrac{0\cdot 5{,}5+6\cdot 6{,}5+8\cdot 7{,}5+14\cdot 8{,}5+12\cdot 9{,}5}{40}
=8{,}3.
\]
\item Phương sai của mẫu số liệu ghép nhóm là\\
\[s_B^2=\dfrac{1}{40}\left(0\cdot 5{,}5^2+6\cdot 6{,}5^2+8\cdot 7{,}5^2+14\cdot 8{,}5^2+12\cdot 9{,}5^2\right)-(8{,}3)^2
=1{,}06.\]
\item Độ lệch chuẩn của mẫu số liệu ghép nhóm là $s_B=\sqrt{1{,}06}$.
\end{itemize}
\end{enumerate}
Do đó $s_A-s_B \approx -0{,}2$.
}
\end{ex}
\Closesolutionfile{ans}
\begin{indapan}
	{ans/ans\currfilebase}
\end{indapan}

