\begin{name}
 {Biên soạn: Bùi Lương Phúc \\ Phản biện: Trần Bảo Hiên}
{Đề ôn tập chương II}
\end{name}


\TN
\Opensolutionfile{ans}[ans/ans\currfilebase-Phan-I]

\begin{ex}%[2-H2B4-SO-14-2425 (Nguồn Đề 5 - Bài 4- Ôn tập chương II)]%[VN-MT-7, Bùi Lương Phúc]%[2H2H1-1]
\immini{Cho hình hộp $ABCD.A'B'C'D'$ (tham khảo hình bên).
Vectơ $\overrightarrow{u}=\overrightarrow{BB'}+\overrightarrow{BA}+\overrightarrow{BC}$ bằng vectơ nào dưới đây?
\choice[2]
{$\overrightarrow{BD}$}
{\True $\overrightarrow{BD'}$}
{$\overrightarrow{BC}$}
{$\overrightarrow{BA'}$}
}
{\begin{tikzpicture}
[scale=0.8, font=\footnotesize, line join=round, line cap=round, >=stealth]
\coordinate (A) at (0,0);
\coordinate (B) at (0.8,1.3);
\coordinate (C) at (4.5,1.3);
\coordinate (D) at ($(A)+(C)-(B)$);
\coordinate (A') at ($(A)+(0.3,3)$);
\coordinate (B') at ($(A')+(B)-(A)$);
\coordinate (C') at ($(B')+(C)-(B)$);
\coordinate (D') at ($(A')+(D)-(A)$);
\draw (A')--(A)--(D)--(D')--(A')--(B')--(C')--(C)--(D) (C')--(D');
\draw[dashed,->](B)--(B');
\draw[dashed,->] (B)--(A);
\draw[dashed,->] (B)--(C);
\foreach \x/\y in {A/-180,B/180,C/0,D/0,A'/180,B'/180,C'/0,D'/0} \fill[black](\x) circle (1pt) ($(\x)+(\y:3mm)$) node{$\x$};
\end{tikzpicture}}
\loigiai{
Ta có $\overrightarrow{u}=\overrightarrow{BB'}+\overrightarrow{BA}+\overrightarrow{BC}=\overrightarrow{BB'}+\left(\overrightarrow{BA}+\overrightarrow{BC}\right)=\overrightarrow{BB'}+\overrightarrow{BD}=\overrightarrow{BD'}$.
}
\end{ex}

\begin{ex}%[2-H2B4-SO-14-2425 (Nguồn Đề 5 - Bài 4- Ôn tập chương II)]%[VN-MT-7, Bùi Lương Phúc]%[2H2H1-3]
\immini{Cho tứ diện $ABCD$ có $AB=AC=AD$ và $\widehat{BAC}=\widehat{BAD}=60^\circ$. Góc giữa hai vectơ $\overrightarrow{AB}$ và $\overrightarrow{CD}$ có số đo bằng
\choice[2]
{$60^\circ$}
{$45^\circ$}
{\True $90^\circ$}
{$120^\circ$}
}
{ \begin{tikzpicture}
[scale=0.8, font=\footnotesize, line join=round, line cap=round, >=stealth]
\coordinate (A) at (0.4,4);
\coordinate (B) at (-1.5,1.5);
\coordinate (D) at (3,1.5);
\coordinate (C) at (0,0);
\draw (A)--(B)--(C)--(D)--cycle (A)--(C);
\draw pic[draw,angle radius=5mm] {angle = B--A--C};
\draw[thick] pic[draw,angle radius=3.5mm] {angle = B--A--D};
\draw[dashed](B)--(D);
\foreach \x/\y in {A/90,B/-120,C/-90,D/-90} \fill[black](\x) circle (1pt) ($(\x)+(\y:3mm)$) node{$\x$};
\draw(-0.5,2.7)--(-0.6,2.8);
\draw (0.13,2) -- (0.27,2);
\draw (1.65,2.7) -- (1.75,2.8);
\end{tikzpicture}}
\loigiai{
\begin{center}
 \begin{tikzpicture}
 [scale=0.8, font=\footnotesize, line join=round, line cap=round, >=stealth]
 \coordinate (A) at (0.4,4);
 \coordinate (B) at (-1.5,1.5);
 \coordinate (D) at (3,1.5);
 \coordinate (C) at (0,0);
 \draw (A)--(B)--(C)--(D)--cycle (A)--(C);
 \draw pic[draw,angle radius=5mm] {angle = B--A--C};
 \draw[thick] pic[draw,angle radius=3.5mm] {angle = B--A--D};
 \coordinate (m) at ($(A)+(0.3,-0.8)$);
 \coordinate (n) at ($(A)+(-0.4,-1)$);
 \node [shift=(m)]{$60^\circ$};
 \node [shift=(n)]{$60^\circ$};
 \draw[dashed](B)--(D);
 \foreach \x/\y in {A/90,B/-120,C/-90,D/-90} \fill[black](\x) circle (1pt) ($(\x)+(\y:3mm)$) node{$\x$};
 \draw(-0.5,2.7)--(-0.6,2.8);
 \draw (0.13,2) -- (0.27,2);
 \draw (1.65,2.7) -- (1.75,2.8);
 \end{tikzpicture}
\end{center}
Ta có
\begin{align*}
 \overrightarrow{AB}\cdot \overrightarrow{CD}=&\, \overrightarrow{AB}\cdot (\overrightarrow{AD}-\overrightarrow{AC})\\
 =&\, \overrightarrow{AB}\cdot \overrightarrow{AD}-\overrightarrow{AB}\cdot \overrightarrow{AC} \\
 =&\, AB\cdot AD\cdot \cos 60{}^\circ -AB\cdot AC\cdot \cos 60^\circ =0 
\end{align*}
Suy ra $\overrightarrow{AB}\perp\overrightarrow{CD}\Rightarrow \left(\overrightarrow{AB},\overrightarrow{CD}\right)=90^\circ$.
}
\end{ex}

\begin{ex}%[2-H2B4-SO-14-2425 (Nguồn Đề 5 - Bài 4- Ôn tập chương II)]%[VN-MT-7, Bùi Lương Phúc]%[2H2N2-2]
Trong KG $Oxyz$, cho điểm $M$ thỏa mãn $\overrightarrow{OM}=2\overrightarrow{i}+3\overrightarrow{j}-\overrightarrow{k}$. Tọa độ của điểm $M$ là
\choice
{$(2;3;1)$}
{$(-2;-3;1)$}
{$(2;-1;3)$}
{\True $(2;3;-1)$}
\loigiai{
Ta có $\overrightarrow{OM}=2\overrightarrow{i}+3\overrightarrow{j}-\overrightarrow{k}\Rightarrow \overrightarrow{OM}=(2;3;-1)\Rightarrow M=(2;3;-1)$.
}
\end{ex}

\begin{ex}%[2-H2B4-SO-14-2425 (Nguồn Đề 5 - Bài 4- Ôn tập chương II)]%[VN-MT-7, Bùi Lương Phúc]%[2H2N2-3]
Trong KG $Oxyz$, cho vectơ $\overrightarrow{u}=2\overrightarrow{i}-\dfrac{1}{2}\overrightarrow{j}+4\overrightarrow{k}$. Tọa độ của vectơ $\overrightarrow{u}$ là
\choice
{\True $\left(2;-\dfrac{1}{2};4 \right)$}
{$\left(2;\dfrac{1}{2};4 \right)$}
{$(2;1;4)$}
{$\left(\dfrac{1}{2};-2;\dfrac{1}{4} \right)$}
\loigiai{
Áp dụng định lí: $\overrightarrow{u}=(a;b;c)\Leftrightarrow \overrightarrow{u}=a\overrightarrow{i}+b\overrightarrow{j}+c\overrightarrow{k}$. 
Ta có $\overrightarrow{u}=2\overrightarrow{i}-\dfrac{1}{2}\overrightarrow{j}+4\overrightarrow{k}=2\overrightarrow{i}+\left(-\dfrac{1}{2}\right)\overrightarrow{j}+4\overrightarrow{k}$, suy ra tọa độ của vectơ $\overrightarrow{u}$ là $\left(2;-\dfrac{1}{2};4\right)$.
}
\end{ex}

\begin{ex}%[2-H2B4-SO-14-2425 (Nguồn Đề 5 - Bài 4- Ôn tập chương II)]%[VN-MT-7, Bùi Lương Phúc]%[2H2H2-5]
Trong KG $Oxyz$, cho hai vectơ $\overrightarrow{u}=(-1;3;-2)$ và $\overrightarrow{v}=(2;5;-1)$. Vectơ nào dưới đây vuông góc với cả hai vectơ $\overrightarrow{u}$ và $\overrightarrow{v}$?
\choice
{${\overrightarrow{w}_4}=(-8;-9;-1)$}
{\True ${\overrightarrow{w}_2}=(7;-5;-15)$}
{${\overrightarrow{w}_1}=(7;5;-15)$}
{${\overrightarrow{w}_3}=(1;8;-3)$}
\loigiai{
Ta có $\left[ \overrightarrow{u},\overrightarrow{v} \right]=\left(\left| \begin{matrix}
3 & -2 \\
5 & -1 \\
\end{matrix} \right|;\left| \begin{matrix}
-2 & -1 \\
-1 & 2 \\
\end{matrix} \right|;\left| \begin{matrix}
-1 & 3 \\
2 & 5 \\
\end{matrix} \right|\right) \Rightarrow \left[\overrightarrow{u},\overrightarrow{v} \right]= (7;-5;-15)$.\\
Vì $\left[\overrightarrow{u},\overrightarrow{v} \right]$ vuông góc với cả hai vectơ $\overrightarrow{u}$ và $\overrightarrow{v}$ nên vectơ $\overrightarrow{w}_2 =(7;-5;-15)$ vuông góc với cả hai vectơ $\overrightarrow{u}$ và $\overrightarrow{v}$.
}
\end{ex}

\begin{ex}%[2-H2B4-SO-14-2425 (Nguồn Đề 5 - Bài 4- Ôn tập chương II)]%[VN-MT-7, Bùi Lương Phúc]%[2H2N2-3]
Trong KG $Oxyz$, cho vectơ $\overrightarrow{a}=(-1;4;2)$. Toạ độ của vectơ $-2\overrightarrow{a}$ là
\choice
{$(-2;8;4)$}
{\True $(2;-8;-4)$}
{$(-2;-8;-4)$}
{$(2;8;4)$}
\loigiai{
Ta có $-2\overrightarrow{a}=\left((-2)\cdot (-1);(-2)\cdot 4;(-2)\cdot 2\right) \Rightarrow -2\overrightarrow{a}=(2;-8;-4)$.
}
\end{ex}

\begin{ex}%[2-H2B4-SO-14-2425 (Nguồn Đề 5 - Bài 4- Ôn tập chương II)]%[VN-MT-7, Bùi Lương Phúc]%[2H2N2-4]
Trong KG $Oxyz$, cho vectơ $\overrightarrow{u}=(2;-1;2)$. Độ dài của vectơ $\overrightarrow{u}$ bằng
\choice
{$\sqrt{7}$}
{\True $3$}
{$9$}
{$2$}
\loigiai{
Độ dài của vectơ $\overrightarrow{u}$ là
$\left| \overrightarrow{u} \right|=\sqrt{2^2+(-1)^2+2^2}=3$.
}
\end{ex}

\begin{ex}%[2-H2B4-SO-14-2425 (Nguồn Đề 5 - Bài 4- Ôn tập chương II)]%[VN-MT-7, Bùi Lương Phúc]%[2H2N2-4]
Trong KG $Oxyz$, cho hai vectơ $\overrightarrow{u}=(-2;1;5)$ và $\overrightarrow{v}=(0;-3;1)$. Tích vô hướng của hai vectơ $\overrightarrow{u}$ và $\overrightarrow{v}$ bằng
\choice
{$10\sqrt{3}$}
{$0$}
{\True $2$}
{$-2$}
\loigiai{
Tích vô hướng của hai vectơ $\overrightarrow{u}$ và $\overrightarrow{v}$ là\\
\[\overrightarrow{u}\cdot \overrightarrow{v}=-2\cdot 0+1\cdot (-3)+5\cdot 1=2.\]
}
\end{ex}

\begin{ex}%[2-H2B4-SO-14-2425 (Nguồn Đề 5 - Bài 4- Ôn tập chương II)]%[VN-MT-7, Bùi Lương Phúc]%[2H2H2-2]
Trong KG $Oxyz$, cho điểm $A(2;-4;3)$ và vectơ $\overrightarrow{u}=(2;2;7)$. Biết rằng $\overrightarrow{u}=\overrightarrow{AB}$, tính tọa độ điểm $B$.
\choice
{$(2;6;4)$}
{$(1;3;2)$}
{\True $(4;-2;10)$}
{$(2;-1;5)$}
\loigiai{
Gọi $B(x;y;z)$, ta có $\overrightarrow{AB}=(x-2;y+4;z-3)$.\\
Vì $\overrightarrow{u}=\overrightarrow{AB}$ nên ta có \begin{center}
$\heva{&x-2=2 \\&y+4=2 \\&z-3=7}\Rightarrow \heva{&x=4 \\&y=-2 \\&z=10}\Rightarrow B(4;-2;10)$.
\end{center}
}
\end{ex}

\begin{ex}%[2-H2B4-SO-14-2425 (Nguồn Đề 5 - Bài 4- Ôn tập chương II)]%[VN-MT-7, Bùi Lương Phúc]%[2H2N2-2]
Trong KG $Oxyz$, cho hai điểm $A(2;3;-4)$ và $B(0;1;6)$. Trung điểm $M$ của đoạn thẳng $AB$ có tọa độ là
\choice
{$(-2;-2;10)$}
{$(1;2;2)$}
{\True $(1;2;1)$}
{$(2;2;-10)$}
\loigiai{
Gọi trung điểm của đoạn thẳng $AB$ là $M(x;y;z)$, ta có
\[x=\dfrac{1+0}{2}=1, 
y=\dfrac{3+1}{2}=2, 
z=\dfrac{-4+6}{2}=1.\]
Vậy $M=(1;2;1)$.
}
\end{ex}

\begin{ex}%[2-H2B4-SO-14-2425 (Nguồn Đề 5 - Bài 4- Ôn tập chương II)]%[VN-MT-7, Bùi Lương Phúc]%[2H2N2-2]
Trong KG $Oxyz$, cho tam giác $MNP$ có $M(1;1;2)$, $N(-1;0;2)$, $P(3;-7;5)$. Trọng tâm $G$ của tam giác $MNP$ có tọa độ là
\choice
{\True $(1;-2;3)$}
{$(-1;2;3)$}
{$(1;2;3)$}
{$(3;-6;7)$}
\loigiai{
Theo công thức tính tọa độ trọng tâm của tam giác ta có\\
\begin{center}
$\heva{& x_G=\dfrac{1-1+3}{3}=1 \\& y_G=\dfrac{1+0-7}{3}=-2 \\& z_G=\dfrac{2+2+5}{3}=3}\Rightarrow G(1;-2;3)$.
\end{center}
}
\end{ex}

\begin{ex}%[2-H2B4-SO-14-2425 (Nguồn Đề 5 - Bài 4- Ôn tập chương II)]%[VN-MT-7, Bùi Lương Phúc]%[2H2H2-2]
Trong KG $Oxyz$, cho hình hộp $ABCD.A'B'C'D'$ có $A(0;0;0)$, $B(3;0;0)$, $D(0;3;0)$ và $D'(0;3;-3)$. Tìm tọa độ đỉnh $A'$ của hình hộp.
\choice
{\True $(0;0;-3)$}
{$(0;3;0)$}
{$(3;3;0)$}
{$(0;0;3)$}
\loigiai{
\begin{center}
\begin{tikzpicture}
[scale=0.8, font=\footnotesize, line join=round, line cap=round, >=stealth]
\coordinate (A) at (0,0);
\coordinate (B) at (0.8,1.3);
\coordinate (C) at (4.5,1.3);
\coordinate (D) at ($(A)+(C)-(B)$);
\coordinate (A') at ($(A)+(0.3,3)$);
\coordinate (B') at ($(A')+(B)-(A)$);
\coordinate (C') at ($(B')+(C)-(B)$);
\coordinate (D') at ($(A')+(D)-(A)$);
\draw (A')--(A)--(D)--(D')--(A')--(B')--(C')--(C)--(D) (C')--(D');
\draw[dashed] (A)--(B)--(B');
\draw[dashed] (B)--(C);
\foreach \x/\y in {A/-180,B/180,C/0,D/0,A'/180,B'/180,C'/0,D'/0} \fill[black](\x) circle (1pt) ($(\x)+(\y:3mm)$) node{$\x$};
\end{tikzpicture}
\end{center}
Ta có $\overrightarrow{AD}=(0;3;0)$.\\
Gọi $A'(x_2;y_2;z_2)\Rightarrow \overrightarrow{A'D'}=(0-x_2;3-y_2;-3-z_2)$.\\
Vì $ADD'A'$ là hình bình hành $\Rightarrow \overrightarrow{A'D'}=\overrightarrow{AD}\Leftrightarrow
\heva{& 0-x_2=0 \\& 3-y_2=3 \\& -3-z_2=0}
\Leftrightarrow \heva{& x_2=0 \\& y_2=0 \\& z_2=-3}
\Rightarrow {A}'(0;0;\,-3)$.
}
\end{ex}
\Closesolutionfile{ans}


\TNTF
\Opensolutionfile{ans}[ans/ans\currfilebase-Phan-II]

\begin{ex}%[2-H2B4-SO-14-2425 (Nguồn Đề 5 - Bài 4- Ôn tập chương II)]%[VN-MT-7, Bùi Lương Phúc]%[2H2V1-3]
\immini{
Cho tứ diện $ABCD$ có $AB$, $AC$ và $AD$ đôi một vuông góc. Gọi $M$ là trung điểm của cạnh $BC$, $H$ là trung điểm của đoạn thẳng $MD$. Cho $AB=AC=a$.
\choiceTF
{\True $\overrightarrow{DM}=\dfrac{1}{2}\overrightarrow{DB}+\dfrac{1}{2}\overrightarrow{DC}$}
{$\overrightarrow{AH}=\dfrac{1}{4}\overrightarrow{AB}+\dfrac{1}{2}\overrightarrow{AC}+\dfrac{1}{2}\overrightarrow{AD}$}
{\True $\overrightarrow{AB}\cdot \overrightarrow{AH}=\dfrac{1}{4}a^2$}
{Góc giữa hai vectơ $\overrightarrow{AH}$ và $\overrightarrow{BC}$ bằng $60^\circ $}
}{\begin{tikzpicture}
[scale=0.7, font=\footnotesize, line join=round, line cap=round, >=stealth]
\coordinate (A) at (0,0);
\coordinate (B) at (-1,-2);
\coordinate (C) at (6,0);
\coordinate (D) at (0,4);
\draw (B)--(D)--(C);
\draw[->] (B)--(C);
\draw[dashed] (B)--(A)--(C) (A)--(D);
\coordinate (M) at ($(B)!0.5!(C)$);
\coordinate (H) at ($(M)!0.5!(D)$);
\draw (D)--(M);
\draw[->, dashed] (A)--(H);
\foreach \x/\g in {A/140,B/-90,C/0,D/90,M/-90,H/0} \fill[black](\x) circle (1pt) ($(\x)+(\g:5mm)$) node{$\x$};
\end{tikzpicture}}
\loigiai{
\begin{itemchoice}
\itemch \textbf{Đúng}.\\
Vì $M$ là trung điểm của cạnh $BC$ nên \[\overrightarrow{DM}=\dfrac{1}{2}\left(\overrightarrow{DB}+\overrightarrow{DC} \right)=\dfrac{1}{2}\overrightarrow{DB}+\dfrac{1}{2}\overrightarrow{DC}.\]
\itemch \textbf{Sai}.\\
Ta có
$\overrightarrow{AH}
=\dfrac{1}{2}\left(\overrightarrow{AM}+\overrightarrow{AD} \right), \overrightarrow{AM}
=\dfrac{1}{2}\left(\overrightarrow{AB}+\overrightarrow{AC} \right)$
\begin{align*}
\Rightarrow \overrightarrow{AH}
=&\, \dfrac{1}{2}\left[ \dfrac{1}{2}\left(\overrightarrow{AB}+\overrightarrow{AC} \right)+\overrightarrow{AD} \right]\\
=&\, \dfrac{1}{2}\cdot \dfrac{1}{2}\left(\overrightarrow{AB}+\overrightarrow{AC}\right)+\dfrac{1}{2}\overrightarrow{AD}\\
=&\, \dfrac{1}{4}\overrightarrow{AB}+\dfrac{1}{4}\overrightarrow{AC}+\dfrac{1}{2}\overrightarrow{AD}
\end{align*}
Rõ ràng $\overrightarrow{AH}=\dfrac{1}{4}\overrightarrow{AB}+\dfrac{1}{2}\overrightarrow{AC}+\dfrac{1}{2}\overrightarrow{AD}$ sai vì $\overrightarrow{AC} \ne\overrightarrow{0}$.
\itemch \textbf{Đúng}.\\
Ta có
\begin{align*}
\overrightarrow{AB}\cdot \overrightarrow{AH}
=&\, \overrightarrow{AB}\cdot \left[ \dfrac{1}{4}\overrightarrow{AB}+\dfrac{1}{4}\overrightarrow{AC}+\dfrac{1}{2}\overrightarrow{AD} \right]\\
=&\, \overrightarrow{AB}\cdot \left(\dfrac{1}{4}\overrightarrow{AB} \right)+\overrightarrow{AB}\cdot \left(\dfrac{1}{4}\overrightarrow{AC} \right)+\overrightarrow{AB}\cdot \left(\dfrac{1}{2}\overrightarrow{AD} \right)\\ =&\, \dfrac{1}{4}{{\overrightarrow{AB}}^2}+\dfrac{1}{4}\overrightarrow{AB}\cdot \overrightarrow{AC}+\dfrac{1}{2}\overrightarrow{AB}\cdot \overrightarrow{AD}.
\end{align*}
Vì $AB\perp AC$ nên $\overrightarrow{AB}\cdot \overrightarrow{AC}=0$, $AB\perp AD$ nên $\overrightarrow{AB}\cdot \overrightarrow{AD}=0$.\\
Vậy $\overrightarrow{AB}\cdot \overrightarrow{AH}=\dfrac{1}{4}{{\overrightarrow{AB}}^2}=\dfrac{1}{4}{\left| \overrightarrow{AB} \right|^2}=\dfrac{1}{4}a^2$.
\itemch \textbf{Sai}.\\
Ta có $\overrightarrow{BC}=\overrightarrow{AC}-\overrightarrow{AB}$, $\overrightarrow{AH}=\dfrac{1}{4}\overrightarrow{AB}+\dfrac{1}{4}\overrightarrow{AC}+\dfrac{1}{2}\overrightarrow{AD}$.\\
\begin{align*}
\overrightarrow{BC}\cdot \overrightarrow{AH}=&\, \left(\overrightarrow{AC}-\overrightarrow{AB} \right)\cdot \left(\dfrac{1}{4}\overrightarrow{AB}+\dfrac{1}{4}\overrightarrow{AC}+\dfrac{1}{2}\overrightarrow{AD} \right)\\
=&\, \dfrac{1}{4}\overrightarrow{AC}\cdot \overrightarrow{AB}+\dfrac{1}{4}{\overrightarrow{AC}}^2+\dfrac{1}{2}\overrightarrow{AC}\cdot \overrightarrow{AD}-\dfrac{1}{4}{\overrightarrow{AB}}^2-\dfrac{1}{4}\overrightarrow{AB}\cdot \overrightarrow{AC}-\dfrac{1}{2}\overrightarrow{AB}\cdot \overrightarrow{AD} \\
=&\, 0+\dfrac{1}{4}\left| \overrightarrow{AC} \right|^2+0-\dfrac{1}{4}\left| \overrightarrow{AB} \right|^2-0-0\\
=&\, \dfrac{1}{4}a^2-\dfrac{1}{4}a^2=0.
\end{align*} 
Vậy góc giữa hai vectơ $\overrightarrow{AH}$ và $\overrightarrow{BC}$ bằng $90^\circ $.
\end{itemchoice}
}
\end{ex}

\begin{ex}%[2-H2B4-SO-14-2425 (Nguồn Đề 5 - Bài 4- Ôn tập chương II)]%[VN-MT-7, Bùi Lương Phúc]%[2H2H2-3]
Trong KG $Oxyz$, cho hai điểm $M(-4;3;-1)$ và $N(2;-1;-3)$.
\choiceTF
{\True Tọa độ vectơ $\overrightarrow{OM}$ bằng $(-4;3;-1)$}
{\True Điểm đối xứng của $M$ qua trục cao có tọa độ là $(4;-3;-1)$}
{Gọi $G$ là trọng tâm của tam giác $OMN$. \\
Hình chiếu của điểm $G$ trên mặt phẳng $\left(Oxy \right)$ có tọa độ là $\left(0;0;-\dfrac{4}{3} \right)$}
{\True Nếu $\overrightarrow{v}=3\overrightarrow{i}-2\overrightarrow{j}-\overrightarrow{k}$ thì hai vectơ $\overrightarrow{MN}$ và $\overrightarrow{v}$ cùng hướng}
\loigiai{
\begin{itemchoice}

\itemch \textbf{Đúng}.\\
Tọa độ vectơ $\overrightarrow{OM}$ cũng là tọa độ điểm $M(-4;3;-1)$.
\itemch \textbf{Đúng}.\\
Điểm đối xứng của $M$ qua trục cao có tọa độ là $(4;-3;-1)$.
\itemch \textbf{Sai}.\\
Ta có trọng tâm tam giác $OMN$ là $G\left(-\dfrac{2}{3};\dfrac{2}{3};-\dfrac{4}{3} \right)$.\\
Hình chiếu của $G$ trên $\left(Oxy \right)$ có tọa độ là $\left(-\dfrac{2}{3};\dfrac{2}{3};0 \right)$.
\itemch \textbf{Đúng}.\\
$\overrightarrow{MN}=(6;-4;-2)$.\\
Vì $\overrightarrow{v}(3;-2;-1)$ nên $\overrightarrow{MN}=2\overrightarrow{v}$.\\
Vậy $\overrightarrow{MN}$ và $\overrightarrow{v}$ cùng hướng.
\end{itemchoice}
}
\end{ex}

\begin{ex}%[2-H2B4-SO-14-2425 (Nguồn Đề 5 - Bài 4- Ôn tập chương II)]%[VN-MT-7, Bùi Lương Phúc]%[2H2H2-4]
Trong KG $Oxyz$, cho vectơ $\overrightarrow{u}=(1;2;3)$ và điểm $A(2;7;4)$.
\choiceTF
{\True Hình chiếu của điểm $A$ trên trục tung có tọa độ là $(0;7;0)$}
{$\left| 2\overrightarrow{u}-\overrightarrow{j} \right|=8$}
{\True Một điểm $B$ nằm trên mặt phẳng $\left(Oxy \right)$ sao cho $\overrightarrow{AB}$ cùng phương với $\overrightarrow{u}=(1;2;3)$. Khi đó điểm $B$ có tọa độ là $\left(\dfrac{2}{3};\dfrac{13}{3};0 \right)$}
{$\sin^2\left(\overrightarrow{u},\overrightarrow{i} \right) +\sin^2\left(\overrightarrow{u},\overrightarrow{j} \right)+\sin ^2\left(\overrightarrow{u},\overrightarrow{k} \right)=1$}
\loigiai{
\begin{itemchoice}
\itemch \textbf{Đúng}.\\
Hình chiếu của điểm $A(2;7;4)$ trên trục tung có tọa độ là $(0;7;0)$.
\itemch \textbf{Sai}.\\
Ta có $2\overrightarrow{u}=(2;4;6);\overrightarrow{j}=(0;1;0)\Rightarrow 2\overrightarrow{u}-\overrightarrow{j}=(2;3;6)$.\\
Suy ra $\left| 2\overrightarrow{u}-\overrightarrow{j} \right|=\sqrt{2^2+3^2+6^2}=7$.
\itemch \textbf{Đúng}.\\
Vì điểm $B$ nằm trên mặt phẳng $\left(Oxy \right)$ nên ta gọi $B(x;y;0)$\\
$\Rightarrow \overrightarrow{AB}=(x-2;y-7;-4)$.\\
Vì $\overrightarrow{AB}$ cùng phương với $\overrightarrow{u}=(1;2;3)$ nên có số thực $k$ sao cho $\overrightarrow{AB}=k\overrightarrow{u}$
hay
\begin{align*}
\heva{& x-2=k \\& y-7=2k \\& -4=3k} \Leftrightarrow \heva{& k=-\dfrac{4}{3} \\& x=\dfrac{2}{3} \\& y=\dfrac{13}{3}.}
\end{align*}
Vậy $B\left(\dfrac{2}{3};\dfrac{13}{3};0 \right)$.
\itemch \textbf{Sai}.\\
Ta có
$\cos \left(\overrightarrow{u},\overrightarrow{i} \right)=\dfrac{\overrightarrow{u}\cdot \overrightarrow{i}}{\left| \overrightarrow{u} \right|\cdot\left| \overrightarrow{i} \right|}=\dfrac{1\cdot 1+2\cdot 0+3\cdot0}{\sqrt{1^2+2^2+3^2}\cdot 1}=\dfrac{1}{\sqrt{14}}$;\\
Tương tự, $\cos \left(\overrightarrow{u},\overrightarrow{j} \right)=\dfrac{2}{\sqrt{14}}$; $\cos \left(\overrightarrow{u},\overrightarrow{k} \right)=\dfrac{3}{\sqrt{14}}$\\
Vậy 
\begin{align*}
&\sin ^2\left(\overrightarrow{u},\overrightarrow{i} \right)+\sin ^2\left(\overrightarrow{u},\overrightarrow{j} \right)+\sin^2\left(\overrightarrow{u},\overrightarrow{k} \right)\\
=&\, 3-\left[ \cos^2\left(\overrightarrow{u},\overrightarrow{i} \right)+\cos ^2\left(\overrightarrow{u},\overrightarrow{j} \right)+\cos^2\left(\overrightarrow{u},\overrightarrow{k} \right) \right]\\
=&\, 3-\left(\dfrac{1}{14}+\dfrac{4}{14}+\dfrac{9}{14} \right)\\
=&\, 2.
\end{align*}
\end{itemchoice}
}
\end{ex}

\begin{ex}%[2-H2B4-SO-14-2425 (Nguồn Đề 5 - Bài 4- Ôn tập chương II)]%[VN-MT-7, Bùi Lương Phúc]%[2H2H2-4]
Trong KG $Oxyz$, cho bốn điểm $A(0;-2;1)$, $B(1;0;-2)$, $C(3;1;-2)$ và $D(-2;-2;-1)$.
\choiceTF
{Ba điểm $A$, $B$, $D$ thẳng hàng}
{\True Tam giác $ACD$ là tam giác vuông tại $A$}
{\True Góc giữa hai vectơ $\overrightarrow{AB}$ và $\overrightarrow{CD}$ là góc tù}
{Khoảng cách từ điểm $A$ đến đường thẳng $CD$ bằng $\dfrac{3\sqrt{210}}{35}$}
\loigiai{
\begin{itemchoice}
\itemch \textbf{Sai}.\\
Ta có $\overrightarrow{AB}=(1;2;-3)$; $\overrightarrow{AD}=(-2;0;-2)$.\\
Vì $\dfrac{1}{-2}\ne \dfrac{-3}{-2}$ nên
hai vectơ $\overrightarrow{AB}$ và $\overrightarrow{AD}$ không cùng phương. \\
Suy ra ba điểm $A$, $B$, $D$ không thẳng hàng.
\itemch \textbf{Đúng}.\\
Ta có $\overrightarrow{AC}=(3;3;-3)$, $\overrightarrow{AD}=(-2;0;-2)$.\\
$\overrightarrow{AC}\cdot\overrightarrow{AD}=3\cdot (-2)+3\cdot 0+(-3) \cdot (-2)=0\Rightarrow AC\perp AD$.\\
Suy ra tam giác $ACD$ là tam giác vuông tại $A$.
\itemch \textbf{Đúng}.\\
Ta có $\overrightarrow{AB}=(1;2;-3)$, $\overrightarrow{CD}=(-5;-3;1)$.\\
Vì $\overrightarrow{AB}\cdot \overrightarrow{CD}=1\cdot (-5)+2\cdot (-3)+(-3) \cdot 1=-14<0$ nên $\cos\left(\overrightarrow{AB},\overrightarrow{CD} \right)<0\Rightarrow \left(\overrightarrow{AB},\overrightarrow{CD} \right)$ là góc tù.
\itemch \textbf{Sai}.\\
Ta có $AC=\sqrt{3^2+3^2+(-3)^2}=3\sqrt{3}$;\\ $AD=\sqrt{(-2)^2+0^2+2^2}=2\sqrt{2}$;\\
$CD=\sqrt{(-5)^2+(-3)^2+1^2}=\sqrt{35}$.\\
Tam giác $ACD$ là tam giác vuông tại $A$ nên
$S_{ACD}=\dfrac{1}{2}AC\cdot AD=3\sqrt{6}$.\\
Khoảng cách từ điểm $A$ đến đường thẳng $CD$ chính là chiều cao kẻ từ $A$ của tam giác $ACD$.\\
\[\mathrm{d}(A,CD)=\dfrac{2S_{ACD}}{CD}=\dfrac{6\sqrt{6}}{\sqrt{35}}=\dfrac{6\sqrt{210}}{35}.\]
\end{itemchoice}
}
\end{ex}
\Closesolutionfile{ans}


\TNSA
\Opensolutionfile{ans}[ans/ans\currfilebase-Phan-III]

\begin{ex}%[2-H2B4-SO-14-2425 (Nguồn Đề 5 - Bài 4- Ôn tập chương II)]%[VN-MT-7, Bùi Lương Phúc]%[2H2V2-4]
Cho tứ diện $ABCD$ có $AB=AC=AD=a$, $\widehat{BAC}=\widehat{BAD}=60^\circ $ và $\widehat{CAD}=90^\circ $. Gọi $I$ là điểm trên cạnh $AB$ sao cho $AI=3IB$ và $J$ là trung điểm của $CD$. Gọi $\alpha$ là số đo của góc giữa hai vectơ $\overrightarrow{AB}$ và $\overrightarrow{IJ}$, hãy tính $\cos \alpha$ (kết quả làm tròn đến hàng phần mười).
\shortans[4]{-0{,}4}

\loigiai{
\begin{center}
\begin{tikzpicture}[line join = round, line cap=round,>=stealth,font=\footnotesize,scale=1]
\coordinate (0,0) at (A);
\coordinate (B) at (-1.5,-3);
\coordinate (C) at (-0.5,-4);
\coordinate (D) at (3,-3);
\coordinate (I) at ($(A)!0.75!(B)$);
\coordinate (J) at ($(C)!0.5!(D)$);
\fill (A) circle (1pt) node[above] {$A$};
\fill (B) circle (1pt) node[left] {$B$};
\fill (C) circle (1pt) node[below] {$C$};
\fill (D) circle (1pt) node[right] {$D$};
\fill (I) circle (1pt) node[left] {$I$};
\fill (J) circle (1pt) node[below] {$J$};
\draw (A) -- (B)--(C)--(D)--(A)--(C);
\draw[dashed] (B) -- (D) (I)--(J);
\draw (0.5,-3.6)--(0.5,-3.85) (2,-3.15)--(2,-3.4);
\end{tikzpicture}
\end{center}
Ta có 
$\overrightarrow{IJ}=\overrightarrow{IA}+\overrightarrow{AJ}=-\dfrac{3}{4}\overrightarrow{AB}+\dfrac{1}{2}\left(\overrightarrow{AC}+\overrightarrow{AD} \right)=-\dfrac{3}{4}\overrightarrow{AB}+\dfrac{1}{2}\overrightarrow{AC}+\dfrac{1}{2}\overrightarrow{AD}$ nên
\[\overrightarrow{IJ}\cdot \overrightarrow{AB}=\left(-\dfrac{3}{4}\overrightarrow{AB}+\dfrac{1}{2}\overrightarrow{AC}+\dfrac{1}{2}\overrightarrow{AD} \right)\cdot \overrightarrow{AB}=\dfrac{1}{2}\left(\overrightarrow{AC}\cdot \overrightarrow{AB}+\overrightarrow{AD}\cdot \overrightarrow{AB}-\dfrac{3}{2}{\overrightarrow{AB}}^2 \right).\]
Lại có $\overrightarrow{AB}\cdot \overrightarrow{AD}=AB\cdot AD\cdot \cos60^\circ =\dfrac{a^2}{2}$; 
$\overrightarrow{AC}\cdot \overrightarrow{AB}=AC\cdot AB\cdot \cos60^\circ =\dfrac{a^2}{2}$; ${\overrightarrow{AB}}^2=AB^2=a^2$.\\
Suy ra, $\overrightarrow{IJ}\cdot \overrightarrow{AB}=\dfrac{1}{2}\left(\dfrac{a^2}{2}+\dfrac{a^2}{2}-\dfrac{3}{2}a^2 \right)=-\dfrac{a^2}{4}$. \\
Ta có $\widehat{CAD}=90^\circ \Rightarrow \overrightarrow{AC}\cdot \overrightarrow{AD}=0$.
\begin{align*}
IJ^2=&\, {\overrightarrow{IJ}}^2=\dfrac{1}{4}\left(\overrightarrow{AC}+\overrightarrow{AD}-\dfrac{3}{2}\overrightarrow{AB} \right)^2 \\
=&\, \dfrac{1}{4}\left({\overrightarrow{AB}}^2+{\overrightarrow{AC}}^2+\dfrac{9}{4}{\overrightarrow{AB}}^2+2\overrightarrow{AC}\cdot \overrightarrow{AD}-3\overrightarrow{AC}\cdot \overrightarrow{AB}-3\overrightarrow{AB}\cdot \overrightarrow{AD} \right) \\
=&\, \dfrac{1}{4}\left(a^2+a^2+\dfrac{9}{4}a^2+2\cdot 0-3\cdot \dfrac{1}{2}a^2-3\cdot \dfrac{1}{2}a^2 \right)=\dfrac{5a^2}{16}.
\end{align*}
$\Rightarrow IJ=\dfrac{a\sqrt{5}}{4}$.\\
Vậy $\cos \alpha=\dfrac{\overrightarrow{AB}\cdot \overrightarrow{IJ}}{AB\cdot IJ}=\dfrac{-\dfrac{a^2}{4}}{\dfrac{a\sqrt{5}}{4}\cdot a}=-\dfrac{\sqrt{5}}{5}\approx -0{,}4$.
}
\end{ex}

\begin{ex}%[2-H2B4-SO-14-2425 (Nguồn Đề 5 - Bài 4- Ôn tập chương II)]%[VN-MT-7, Bùi Lương Phúc]%[2H2H2-4]
\immini{Trong không gian, xét hệ tọa độ $Oxyz$ có gốc $O$ trùng với vị trí của một giàn khoan trên biển, mặt phẳng $\left(Oxy \right)$ trùng với mặt biển (được coi là phẳng) với trục $Ox$ hướng về phía tây, trục $Oy$ hướng về phía nam và trục $Oz$ hướng thẳng đứng lên trời. Đơn vị đo Trong KG $Oxyz$ lấy theo kilômét. Một chiếc ra đa đặt tại giàn khoan và một chiếc tàu thám hiểm có tọa độ là $(30;25;-15)$ (tham khảo hình bên).\\
Tính khoảng cách theo đơn vị kilômét từ chiếc ra đa đến chiếc tàu thám hiểm (kết quả làm tròn đến hàng phần mười).}{
\begin{tikzpicture}[line join = round, line cap=round,>=stealth,font=\footnotesize,scale=0.7]
\definecolor{xanhdatroi}{RGB}{0,127,255}
\tikzset{icon-cot/.pic={
\path[black!80]
(0,0)coordinate (A)
($(A)+(0,8)$)coordinate (B)
($(B)+(-2,-1)$)coordinate (C)
($(A)+(C)-(B)$)coordinate (D)
($(C)!(A)!(D)$) coordinate (A1)
($2*(A1)-(A)+(0,1)$)coordinate (E)
($(B)+(E)-(A)$) coordinate (F)
;
%%%%%%%%%%%%%%%%%%%%%%%%%%%%%%%%%%%%
\draw (A)--(B)--(C)--(D)--cycle
(D)--(E)--(F)--(C);
\foreach \x/\y in {0/0.1,0.1/0.2,0.2/0.3,0.3/0.4,0.4/0.5,0.5/0.6,0.6/0.7,0.7/0.8,0.8/0.9,0.9/1}
\draw
($(A)!\x!(B)$)--($(D)!\y!(C)$)
($(D)!\x!(C)$)--($(A)!\y!(B)$)
($(E)!\x!(F)$)--($(D)!\y!(C)$)
($(D)!\x!(C)$)--($(E)!\y!(F)$)
($(A)!\x!(B)$)--($(E)!\y!(F)$)
($(E)!\x!(F)$)--($(A)!\y!(B)$)
(B)--(F) (A)--(E);
}}
%%%%%%%%%%%%%%%%%%%%%%%%%%%%%%%%%%%%
\fill[bottom color=blue!80!green!15!black!60, top color=blue!20!green!20, middle color=blue!80!green]
(-5,-3.5) rectangle (6,2);
\fill[inner color=xanhdatroi, outer color=xanhdatroi!50]
(-5,4) rectangle (6,2);

%%%%%%%%%%%%%%%%%%%%%%%%%%%%%%%%
%De
\draw[fill=orange!10,scale=1,yshift=-1.25cm,xshift=-0.3cm,opacity=0.2,decorate, decoration={snake, amplitude=0.4mm, segment length=2mm},scale=1.1]
(-1.5,0.5)--(0.5,-0.5)--(4.5,-0.5) --($(-1.5,0.5)+(4.5,-1)-(0.5,-1)$)--cycle
;
%%%%%%%%%%%%%%%%%%%%%%%%%%%%%%%%
%Chande
\path[black!80]
(0.75,-1.2) pic[scale=0.15]{icon-cot}
(0.75,-1.5) pic[scale=0.15]{icon-cot}
(1.8,-1.2) pic[scale=0.15]{icon-cot}
(1.8,-1.5) pic[scale=0.15]{icon-cot}
(2.8,-1.2) pic[scale=0.15]{icon-cot}
(2.8,-1.5) pic[scale=0.15]{icon-cot}
;
%%%%%%%%%%%%%%%%%%%%%%%%%%%%%%%%
%San
\draw[fill=orange!80!gray!60,scale=1]
(-1.5,0.5)--(0.5,-0.5)--(4.5,-0.5) --($(-1.5,0.5)+(4.5,-1)-(0.5,-1)$)--cycle
;
%%%%%%%%%%%%%%%%%%%%%%%%%%%%%%%%%%%%
%Giankhoan
\path[black!80]
(0.75,0) pic[scale=0.15]{icon-cot}
(0.75,0.2) pic[scale=0.15]{icon-cot}
(1.8,0) pic[scale=0.15]{icon-cot}
(1.8,0.2) pic[scale=0.15]{icon-cot}
(2.8,0) pic[scale=0.15]{icon-cot}
(2.8,0.2) pic[scale=0.15]{icon-cot}
;
%%%%%%%%%%%%%%%%%%%%%%%%%%%%%%%%%%%%
%Hetoado Oxyz
\begin{scope}
\draw[->,line width=1pt,red] (0,0)--(4,-2) node [below]{y};
\draw[->,line width=1pt,red] (0,0)--(-4,0) node [below]{x};
\draw[->,line width=1pt,red] (0,0)--(0,3.5) node [right]{z};
\fill (0,0) circle(1pt) node [above left, red]{O};
\end{scope}
\end{tikzpicture}
}
\shortans[4]{41{,}8}
\loigiai{
Theo đề bài ta có tọa độ của ra đa là $(0;0;0)$, tọa độ của tàu thám hiểm là $(30;25;-15)$.\\
Khi đó khoảng cách giữa ra đa và tàu thám hiểm là
\[\mathrm{d}=\sqrt{(30-0)^2+(25-0)^2+(-15-0)^2}=5\sqrt{70}\approx 41{,}8.\]
Vậy khoảng khoảng cách giữa ra đa và tàu thám hiểm là $41{,}8$\,(km).
}
\end{ex}

\begin{ex}%[2-H2B4-SO-14-2425 (Nguồn Đề 5 - Bài 4- Ôn tập chương II)]%[VN-MT-7, Bùi Lương Phúc]%[2H2V2-5]
Trong KG $Oxyz$, cho tứ diện $ABCD$ có $A(0;1;-1)$, $B(1;1;2)$, $C(1;-1;0)$, $D(0;0;1)$. Biết rằng có một vectơ $\overrightarrow{v}=(a;b;2)$ vuông góc với cả hai vectơ $\overrightarrow{BC}$ và $\overrightarrow{BD}$. Tính $3a+b$.
\shortans[4]{-2}
\loigiai{
Ta có $\overrightarrow{BC}=(0;-2;-2)$, $\overrightarrow{BD}=(-1;-1;-1)$.
\begin{align*}
\left[ \overrightarrow{BC},\overrightarrow{BD} \right]=\left(\left| \begin{matrix}
-2 & -2 \\
-1 & -1 \\
\end{matrix} \right|;\left| \begin{matrix}
-2 & 0 \\
-1 & -1 \\
\end{matrix} \right|;\left| \begin{matrix}
0 & -2 \\
-1 & -1 \\
\end{matrix} \right| \right) \Rightarrow \left[ \overrightarrow{BC},\overrightarrow{BD} \right]=(0;2;-2).
\end{align*}
Khi đó $\left[ \overrightarrow{BC},\overrightarrow{BD} \right]=(0;2;-2)$ là một vectơ vuông góc với cả hai vectơ $\overrightarrow{BC}$, $\overrightarrow{BD}$. \\
Ta có $\left[ \overrightarrow{BC},\overrightarrow{BD} \right]$ và $\overrightarrow{v}=(a;b;2)$ cùng phương nên có số thực $k$ để $\overrightarrow{v}= k\cdot \left[ \overrightarrow{BC},\overrightarrow{BD} \right]$.\\
Suy ra
$\heva{&a= k\cdot 0 \\&b=k\cdot 2 \\&2=k \cdot(-2)}$. \\
Giải ra ta được $a=0$, $b=-2$.\\
Vậy $3a+b=-2$.
}
\end{ex}

\begin{ex}%[2-H2B4-SO-14-2425 (Nguồn Đề 5 - Bài 4- Ôn tập chương II)]%[VN-MT-7, Bùi Lương Phúc]%[2H2V2-4]
Trong không gian với một hệ trục toạ độ cho trước (đơn vị đo lấy theo kilômét), ra đa phát hiện một chiếc máy bay di chuyển với vận tốc và hướng không đổi từ điểm $A(800;500;7)$ đến điểm $B(940;550;9)$ trong $10$ phút. Nếu máy bay tiếp tục giữ nguyên vận tốc và hướng bay thì toạ độ của máy bay sau $5$ phút tiếp theo là $C(x;y;z)$. Tính $x+y+z$.
\shortans[4]{1595}
\loigiai{
Vị trí của máy bay sau $5$ phút tiếp theo là $C(x;y;z)$.\\
Vì hướng của máy bay không đổi nên $\overrightarrow{AB}$ và $\overrightarrow{BC}$ cùng hướng.\\
Do vận tốc của máy bay không đổi và thời gian bay từ $A$ đến $B$ gấp đôi thời gian bay từ $B$ đến $C$ nên $AB=2BC$.\\
Do đó $\overrightarrow{BC}=\dfrac{1}{2}\overrightarrow{AB}$.\\
Mặt khác, $\dfrac{1}{2}\overrightarrow{AB}=\left(\dfrac{940-800}{2};\dfrac{550-500}{2};\dfrac{9-7}{2} \right) \Rightarrow \dfrac{1}{2}\overrightarrow{AB}=(70;25;1)$;\\
Mà $\overrightarrow{BC}=(x-940 ; y-550 ; z-9)$
nên
$\heva{&x-940=70 \\&y-550=25 \\&z-9=1} \Rightarrow \heva{&x=1010 \\&y=575 \\&z=10} \Rightarrow x+y+z=1\,595$.
}
\end{ex}

\begin{ex}%[2-H2B4-SO-14-2425 (Nguồn Đề 5 - Bài 4- Ôn tập chương II)]%[VN-MT-7, Bùi Lương Phúc]%[2H2V2-4]
\immini{Một tấm gỗ tròn được treo song song với mặt phẳng nằm ngang bởi ba sợi dây không dãn xuất phát từ điểm $O$ trên trần nhà và lần lượt buộc vào ba điểm $A$, $B$, $C$ trên tấm gỗ tròn sao cho các lực căng $\overrightarrow{F}_1$, $\overrightarrow{F}_2$, $\overrightarrow{F}_3$ lần lượt trên mỗi dây $OA$, $OB$, $OC$ đôi một vuông góc với nhau và có độ lớn $\left| \overrightarrow{F}_1 \right|=\left| \overrightarrow{F}_2 \right|=\left| \overrightarrow{F}_3 \right|=10$\,N (xem hình vẽ).\\
Tính trọng lượng $P$ của tấm gỗ tròn đó (kết quả làm tròn đến hàng phần mười).}{
\begin{tikzpicture}[line join = round, line cap=round,>=stealth,font=\footnotesize,scale=0.9]
\path
(0,-1) coordinate (O)
(0,-4) coordinate (O')
(0,-8) coordinate (O'')
($(100:2cm and 1cm)+(0,-4)$) coordinate (B)
($(220:2cm and 1cm)+(0,-4)$) coordinate (A)
($(-20:2cm and 1cm)+(0,-4)$) coordinate (C) ;
\filldraw[fill=gray!80, draw=gray] (0,-4.4) ellipse (2cm and 1.1cm);
\filldraw[fill=gray!40, draw=gray] (O') ellipse (2cm and 1cm);
\draw[]
(O)--(A)
(O)--(B) (O)--(C)
;
\filldraw[gray!80, line width=1pt]
($($(O')+(2,0)$)+(0,-0.5)$)--($(O')+(2,0)$)
($($(O')-(2,0)$)+(0,-0.45)$)--($(O')-(2,0)$)
;
\draw[->,line width=1pt] (O)--($(O)!1/2!(A)$)node[left]{$\overrightarrow{F}_1$};
\draw[->,line width=1pt] (O)--($(O)!0.5!(B)$)node[right]{$\overrightarrow{F}_2$};
\draw[->,line width=1pt] (O)--($(O)!1/2!(C)$)
node[right]{$\overrightarrow{F}_3$}
;
\draw[->]
($(270:2cm and 1cm)+(0,-4.52)$)--(O'')node[pos=0.2,right]{$\overrightarrow{P}$}
;
\draw[dashed] ($(270:2cm and 1cm)+(0,-3.9)$)--(O')--(C);
\foreach \p/\r in {O'/180,B/-90,O/90,A/-90,C/-100}
\fill (\p) circle (1.2pt) node[shift={(\r:3mm)}]{$\p$};
\filldraw[fill=gray!80, draw=gray] (-2,-1) rectangle (2,-0.9);
\end{tikzpicture}
}
\shortans[4]{17{,}3}
\loigiai{
\begin{center}
\begin{tikzpicture}
[scale=0.9, font=\footnotesize, line join=round, line cap=round, >=stealth]
\coordinate (O) at (0,0);
\coordinate (A_1) at (-1.5,-2);
\coordinate (M) at (0.5,-4);
\coordinate (C_1) at ($(O)+(M)-(A_1)$);
\coordinate (B_1) at ($(O)+(-0.5,-1.7)$);
\coordinate (D_1) at ($(A_1)+(B_1)-(O)$);
\coordinate (N) at ($(B_1)+(C_1)-(O)$);
\coordinate (Q) at ($(D_1)+(M)-(A_1)$);
\draw (M)--(A_1)--(D_1)--(Q)--(N)--(C_1)--(M)--(Q);
\draw[->,line width=1pt] (O)--(A_1) node[midway, above,xshift=-1mm] {$\overrightarrow{F}_1$};;
\draw[->,line width=1pt] (O) -- (C_1) node[midway, above] {$\overrightarrow{F}_3$};
\draw[dashed] (D_1)--(B_1)--(N);
\draw[dashed,->,line width=1pt] (O)--(B_1) node[midway, below, xshift=2mm] {$\overrightarrow{F}_2$};
\draw[dashed,red,->] (O)--(Q);
\foreach \x/\y in {O/90,B_1/-90,C_1/0,D_1/-90,A_1/120,M/0,N/0,Q/-90} \fill[black](\x) circle (1pt) ($(\x)+(\y:4mm)$) node{$\x$};
\end{tikzpicture}
\end{center}
Gọi $A_1$, $B_1$, $C_1$ lần lượt là các điểm sao cho $\overrightarrow{OA}_1=\overrightarrow{F}_1$, $\overrightarrow{OB}_1=\overrightarrow{F}_2$, $\overrightarrow{OC}_1=\overrightarrow{F}_3$.\\
Lấy các điểm $D_1$, $M$, $N$, $Q$ sao cho $OA_1D_1B_1.C_1MQN$ là hình hộp.\\
Theo quy tắc hình hộp ta có $\overrightarrow{OA}_1+\overrightarrow{OB}_1+\overrightarrow{OC}_1=\overrightarrow{OQ}$.\\
Do các lực căng $\overrightarrow{F}_1$, $\overrightarrow{F}_2$, $\overrightarrow{F}_3$ đôi một vuông góc với nhau và có độ lớn $\left| \overrightarrow{F}_1 \right|=\left| \overrightarrow{F}_2 \right|=\left| \overrightarrow{F}_3 \right|=10$\,N nên hình hộp $OA_1D_1B_1.C_1MQN$ có ba cạnh $OA_1$, $OB_1$, $OC_1$ đôi một vuông góc và đều có độ dài bằng $10$.\\
Vì thế $OA_1D_1B_1.C_1MQN$ là hình lập phương có độ dài cạnh bằng $10$.\\
Suy ra độ dài đường chéo bằng $10\sqrt{3}$.\\
Gọi $\overrightarrow{P}$ là trọng lượng tác dụng lên tấm gỗ.\\
Do tấm gỗ ở vị trí cân bằng nên $\overrightarrow{F}_1+\overrightarrow{F}_2+\overrightarrow{F}_3=\overrightarrow{P}$.\\
Suy ra $\left| \overrightarrow{P} \right|=\left| \overrightarrow{OQ} \right|=10\sqrt{3}\approx 17{,}3$.\\
Vậy trọng lượng của tấm gỗ tròn là $P=\left| \overrightarrow{P} \right|=10\sqrt{3}\approx 17{,}3$\,(N).}
\end{ex}

\begin{ex}%[2-H2B4-SO-14-2425 (Nguồn Đề 5 - Bài 4- Ôn tập chương II)]%[VN-MT-7, Bùi Lương Phúc]%[2H2V2-4]
 \immini{Một chậu cây được đặt trên một giá đỡ có bốn chân với điểm đặt $S(0;0;40)$ và các điểm chạm mặt đất của bốn chân lần lượt là $A(40;0;0)$, $B(0;40;0)$, $C(-40;0;0)$, $D(0;-40;0)$ (đơn vị là cm). Cho biết trọng lực tác dụng lên chậu cây có độ lớn $60$\,N và được phân bố thành bốn lực $\overrightarrow{F}_1,\overrightarrow{F}_2,\overrightarrow{F}_3,\overrightarrow{F}_4$ có độ lớn bằng nhau như hình vẽ. \\
 Tính $\left| \overrightarrow{F}_1+\overrightarrow{F}_2+\overrightarrow{F}_3+3\overrightarrow{F}_4 \right|$ (kết quả làm tròn đến hàng đơn vị).}{
 \hspace*{0.5cm}\begin{tikzpicture}[line join = round, line cap=round,>=stealth,font=\footnotesize,scale=1]
 \tikzset{chauhoa/.pic={\fill[orange!80!black!90] (-1,-1.1) -- (1,-1.1) -- (0.8,-1.8) -- (-0.8,-1.8) -- cycle;
 \draw[thick] (-1,-1.1) -- (1,-1.1) -- (0.8,-1.8) -- (-0.8,-1.8) -- cycle;
 \fill[brown] (-0.3,-1) -- (0.3,-1) -- (0.2,0) -- (-0.2,0) -- cycle;
 \draw[brown, very thick] (-0.3,-1) -- (0.3,-1) -- (0.2,-0.2) -- (-0.2,-0.2) -- cycle;
 \fill[brown!60!black!90] (-0.1,-0.7) circle (0.1);
 \fill[brown!60!black!90] (0.1,-0.9) circle (0.1);
 \fill[brown!60!black!90] (-0.2,-0.5) circle (0.1);
 \fill[brown!50!black!80] (0,-0.3) circle (0.1);
 \fill[green] (0,0.3) ellipse (0.6 and 0.3);
 \fill[green] (-0.5,0.3) ellipse (0.6 and 0.3);
 \fill[green] (0.5,0.3) ellipse (0.6 and 0.3);
 \fill[green] (0,-0.2) ellipse (0.6 and 0.3);
 \draw[green, very thick] (0,0.3) to[out=60,in=120] (0.5,1.5);
 \draw[green, very thick] (0,0.3) to[out=120,in=60] (-0.5,1.5);
 \draw[green, very thick] (0,-0.2) to[out=60,in=120] (0.5,1);
 \draw[green, very thick] (0,-0.2) to[out=120,in=60] (-0.5,1);
 \fill[green] (0.5,1.2) ellipse (0.4 and 0.2);
 \fill[green] (-0.5,1.2) ellipse (0.4 and 0.2);
 \fill[green] (0.5,0.8) ellipse (0.4 and 0.2);
 \fill[green] (-0.5,0.8) ellipse (0.4 and 0.2);}}
 \path
 (0,0) coordinate (O)
 (0,4) coordinate (S)
 (-115:2.3cm and 1.2cm) coordinate (A)
 (-10:2.3cm and 1.2cm) coordinate (B)
 ($2*(O)-(A)$) coordinate (C)
 ($2*(O)-(B)$) coordinate (D)
 ($(S)!0.55!(A)$) coordinate(F1)
 ($(S)!0.55!(B)$) coordinate(F2)
 ($(S)!0.55!(C)$) coordinate(F3)
 ($(S)!0.55!(D)$) coordinate(F4);
 \draw[->, shorten >=-0.5cm, shorten <=-0.5cm](C)--(A) node[below left=10pt]{$x$};
 \draw[->, shorten >=-0.5cm, shorten <=-0.5cm](D)--(B) node[below right=5pt]{$y$};
 \foreach \p in {A,B,C,D} {\draw (S)--(\p);}
 \foreach \f/\g [count =\i from 1] in {F1/200,F2/10,F3/-120,F4/160} {\draw[->] (S)--(\f)node [shift={(\g:0.4)}]{$\overrightarrow{F}_\i$};}
 \draw[->] (O)--(90:5.5) node[left]{$z$};
 \foreach \i/\j in {O/-90, A/-70, B/-90, C/-70, D/-90} \fill (\i) node[shift={(\j:0.3)}]{$\i$} circle(1pt);
 \fill (S) node[shift={(210:0.4)}]{$S$} circle(1pt);
 \draw ($(S)+(0,0.7)$) pic[scale=0.4]{chauhoa};
 \end{tikzpicture}}
 \shortans[4]{95}
 \loigiai{
 \begin{center}
 \begin{tikzpicture}[line join = round, line cap=round,>=stealth,font=\footnotesize,scale=1.2]
 \path
 (0,0) coordinate (O)
 (0,4) coordinate (S)
 (-115:2.5cm and 1.2cm) coordinate (A)
 (-10:2.5cm and 1.2cm) coordinate (B)
 ($2*(O)-(A)$) coordinate (C)
 ($2*(O)-(B)$) coordinate (D)
 ($(S)!0.55!(A)$) coordinate(A')
 ($(S)!0.55!(B)$) coordinate(B')
 ($(S)!0.55!(C)$) coordinate(C')
 ($(S)!0.55!(D)$) coordinate(D')
 ($(A')!1/2!(C')$) coordinate(O');
 \draw (S)--(D)--(A)--(B)--(S)--(A) (B')--(A')--(D');
 \draw[dashed] (S)--(C)--(D)--(B)--(C)--(A) (S)--(O) (D')--(C')--(B') (A')--(C') (B')--(D');
 \foreach \i/\j in {O/-70, A/-90, B/-60, C/0, D/200, S/90, A'/200, B'/20, C'/160, D'/160, O'/-50} \fill (\i) node[shift={(\j:0.3)}]{$\i$} circle(1pt);
 \draw[->, red] (S)--(A') node[midway, left=-4pt] {$\overrightarrow{F}_1$};
 \draw[->, red] (S)--(B') node[midway, right] {$\overrightarrow{F}_2$};
 \draw[->, red] (S)--(C') node[midway, shift={(250:0.3)}] {$\overrightarrow{F}_3$};
 \draw[->, red] (S)--(D') node[midway, left] {$\overrightarrow{F}_4$};
 \end{tikzpicture}
 \end{center}
 Tứ giác $ABCD$ có hai đường chéo bằng nhau và vuông góc với nhau tại trung điểm của mỗi đường nên là hình vuông.\\
 Ta có $\overrightarrow{SA}=(40;0;-40)$, $\overrightarrow{SB}=(0;40;-40)$, $\overrightarrow{SC}=(-40;0;-40)$, $\overrightarrow{SD}=(0;-40;-40)$\\
 $\Rightarrow SA=SB=SC=SD=40\sqrt{2}$. \\
 Do đó $S.ABCD$ là hình chóp tứ giác đều.\\
 Các vectơ $\overrightarrow{F}_1$, $\overrightarrow{F}_2$, $\overrightarrow{F}_3$, $\overrightarrow{F}_4$ có điểm đầu tại $S$ và điểm cuối lần lượt là $A'$ ,$B'$, $C'$, $D'$.\\
 Ta có $SA'=SB'=SC'=SD'$ nên $S.A'B'C'D'$ cũng là hình chóp tứ giác đều.\\
 Gọi $\overrightarrow{F}$ là trọng lực tác dụng lên chậu cây và ${O}'$ là tâm của hình vuông $A'B'C'D'$.\\
 Ta có 
 $\overrightarrow{F}=\overrightarrow{F}_1+\overrightarrow{F}_2+\overrightarrow{F}_3+\overrightarrow{F}_4=\overrightarrow{SA'}+\overrightarrow{SB'}+\overrightarrow{SC'}+\overrightarrow{SD'}=4\overrightarrow{SO'}$.\\
 Mà $\left| \overrightarrow{F} \right|=60\Rightarrow \left| \overrightarrow{SO'} \right|=15$ và $\overrightarrow{SO}=(0;0;-40)$ nên $\left| \overrightarrow{SO} \right|=40$.\\
 Vậy $\overrightarrow{SO'}=\dfrac{15}{40}\overrightarrow{SO}=\dfrac{3}{8}\overrightarrow{SO}$.\\
 Suy ra
 $\overrightarrow{SA'}=\dfrac{3}{8}\overrightarrow{SA}$, 
 $\overrightarrow{SB'}=\dfrac{3}{8}\overrightarrow{SB}$, 
 $\overrightarrow{SC'}=\dfrac{3}{8}\overrightarrow{SC}$ và 
 $\overrightarrow{SD'}=\dfrac{3}{8}\overrightarrow{SD}$.\\
 Do đó $\overrightarrow{F}_1=(15;0;-15)$, $\overrightarrow{F}_2=(0;15;-15)$, $\overrightarrow{F}_3=(-15;0;-15)$, $\overrightarrow{F}_4=(0;-15;-15)$.\\
 Suy ra $\overrightarrow{F}_1+\overrightarrow{F}_2+\overrightarrow{F}_3+3\overrightarrow{F}_4=(0;-30;-90)$. \\
 Vậy $\left| \overrightarrow{F}_1+\overrightarrow{F}_2+\overrightarrow{F}_3+3\overrightarrow{F}_4 \right|=\sqrt{0^2+(-30)^2+(-90)^2}=30\sqrt{10}\approx 95$\,(N).
 }
\end{ex}
\Closesolutionfile{ans}
 
\begin{indapan}
	{ans/ans\currfilebase}
\end{indapan}

