\begin{name}
	{\tenchude}
	{ĐỀ ÔN TẬP CHƯƠNG II}
	{LỚP TOÁN THẦY PHÁT}
	{\thoigian}
\end{name}

\TN
\Opensolutionfile{ans}[ans/ans\currfilebase-Phan-I]

\begin{ex}%[2-H2B4-SO-10-2425 (Nguồn: Bài 4 - Đề 1 - Ôn Tập Chương II)]%[VN-MT-7, Trần Bảo Hiên]%[2H2H1-2]
Cho tứ diện $ABCD$. Gọi $G$ là trọng tâm tam giác $BCD$ và điểm $M$ thuộc cạnh $AB$ sao cho $AM=2BM$. Đẳng thức nào sau đây là đúng?
\choice
{$\overrightarrow{MG}=\overrightarrow{AB}+\overrightarrow{AC}+\overrightarrow{AD}$}
{$\overrightarrow{MG}=\dfrac{1}{3}\overrightarrow{AB}-\dfrac{1}{3}\overrightarrow{AC}-\dfrac{1}{3}\overrightarrow{AD}$}
{\True $\overrightarrow{MG}=-\dfrac{1}{3}\overrightarrow{AB}+\dfrac{1}{3}\overrightarrow{AC}+\dfrac{1}{3}\overrightarrow{AD}$}
{$\overrightarrow{MG}=\dfrac{4}{3}\overrightarrow{AB}-\dfrac{1}{3}\overrightarrow{AC}-\dfrac{1}{3}\overrightarrow{AD}$}
\loigiai{
\begin{center}
\begin{tikzpicture}[scale=0.8,font=\footnotesize,line join=round,line cap=round,>=stealth]
\coordinate (A) at (-1,4);
\coordinate (B) at (-3,0);
\coordinate (C) at (1,-2);
\coordinate (D) at (3,0);
\coordinate (I) at ($(C)!1/2!(D)$);
\coordinate (G) at ($(B)!2/3!(I)$);
\coordinate (M) at ($(A)!2/3!(B)$);
\draw(A)--(B)--(C)--(D)--(A)--(C);
\draw[dashed](I)--(B)--(D) (M)--(G);
\foreach \i/\g in {A/90,B/180,C/-90,D/0,G/-90,M/135}{\fill (\i) circle (1.0pt)($(\i)+(\g:3mm)$) node[scale=1]{$\i$};}
\end{tikzpicture}
\end{center}
Ta có $M$ thuộc cạnh $AB$ và $AM=2BM$ nên $\overrightarrow{AM}=\dfrac{2}{3}\overrightarrow{AB}$.\\
Do $G$ là trọng tâm tam giác $BCD$ nên $3\overrightarrow{AG}=\overrightarrow{AB}+\overrightarrow{AC}+\overrightarrow{AD}$ hay $\overrightarrow{AG}=\dfrac{1}{3}\left(\overrightarrow{AB}+\overrightarrow{AC}+\overrightarrow{AD}\right)$.\\
Mà $\overrightarrow{MG}=\overrightarrow{AG}-\overrightarrow{AM}$ nên $\overrightarrow{MG}=\dfrac{1}{3}\left(\overrightarrow{AB}+\overrightarrow{AC}+\overrightarrow{AD}\right)-\dfrac{2}{3}\overrightarrow{AB}=-\dfrac{1}{3}\overrightarrow{AB}+\dfrac{1}{3}\overrightarrow{AC}+\dfrac{1}{3}\overrightarrow{AD}$.
}
\end{ex}

\begin{ex}%[2-H2B4-SO-10-2425 (Nguồn: Bài 4 - Đề 1 - Ôn Tập Chương II)]%[VN-MT-7, Trần Bảo Hiên]%[2H2H1-2]
Cho hình lập phương $ABCD.EFGH$. Hãy xác định góc giữa cặp vectơ $\overrightarrow{AB}$ và $\overrightarrow{EG}$?
\choice
{$60^\circ$}
{\True $45^\circ$}
{$90^\circ$}
{$120^\circ$}
\loigiai{
\begin{center}
\begin{tikzpicture}[scale=0.8, font=\footnotesize, line join=round, line cap=round, >=stealth]
\path
(0:0) coordinate (B)
(0:4) coordinate (C)
($(B)+(45:2)$) coordinate (A)
($(A)+(C)-(B)$) coordinate (D)
($(A)+(90:3)$) coordinate (E)
($(B)+(90:3)$) coordinate (F)
($(C)+(90:3)$) coordinate (G)
($(D)+(90:3)$) coordinate (H);
\draw[dashed] (B)--(A)--(E) (A)--(D)--(B) (A)--(C);
\draw (E)--(F)--(G)--(H)--(D)--(C)--(B)--(F) (E)--(H) (E)--(G)--(C) (F)--(H);
\foreach \x/\g in {A/170,B/170,C/0,D/0,E/170,F/170,G/0,H/0}
\draw[fill=black] (\x) circle (.5pt)($(\g:.3)+(\x)$) node {$\x$};
\end{tikzpicture}
\end{center}
Ta có $\overrightarrow{EG}=\overrightarrow{AC}$ (do $ACGE$ là hình chữ nhật)
$\Rightarrow\left(\overrightarrow{AB},\overrightarrow{EG}\right)=\left(\overrightarrow{AB},\overrightarrow{AC}\right)=\widehat{BAC}=45^\circ$.
}
\end{ex}

\begin{ex}%[2-H2B4-SO-10-2425 (Nguồn: Bài 4 - Đề 1 - Ôn Tập Chương II)]%[VN-MT-7, Trần Bảo Hiên]%[2H2N2-2]
Trong không gian với hệ tọa độ $Oxyz$, cho điểm $A(2;3;-2)$. Gọi $A_1$ là hình chiếu vuông góc của điểm $A$ lên mặt phẳng $(Oyz)$. Khi đó tọa độ của điểm $A_1$ là
\choice
{$(2;3;0)$}
{$(2;0;0)$}
{$(-2;3;-2)$}
{\True $(0;3;-2)$}
\loigiai{
Hình chiếu vuông góc của điểm $A$ lên mặt phẳng $(Oyz)$ là $A_1(0;3;-2)$.
}
\end{ex}

\begin{ex}%[2-H2B4-SO-10-2425 (Nguồn: Bài 4 - Đề 1 - Ôn Tập Chương II)]%[VN-MT-7, Trần Bảo Hiên]%[2H2H2-2]
Trong không gian với hệ tọa độ $Oxyz$, cho vectơ $\overrightarrow{a}=\left(2;\dfrac{1}{3};-5\right)$ và điểm $M(2;3;4)$. Tọa độ điểm $N$ thỏa mãn $\overrightarrow{MN}=\overrightarrow{a}$ là
\choice
{$\left(2;\dfrac{5}{3};-\dfrac{1}{2}\right)$}
{$\left(0;\dfrac{8}{3};9\right)$}
{\True $\left(4;\dfrac{10}{3};-1\right)$}
{$\left(0;-\dfrac{8}{3};-9\right)$}
\loigiai{
Gọi tọa độ điểm $N$ là $\left(x_N;y_N;z_N\right)$, ta có $\overrightarrow{MN}=\left(x_N-2;y_N-3;z_N-4\right)$.\\
Ta có $\overrightarrow{MN}=\overrightarrow{a}\Leftrightarrow\heva{&x_N-2=2\\&y_N-3=\dfrac{1}{3}\\&z_N-4=-5}\Leftrightarrow\heva{&x_N=2+2\\&y_N=\dfrac{1}{3}+3\\&z_N=-5+4}\Leftrightarrow\heva{&x_N=4\\&y_N=\dfrac{10}{3}\\&z_N=-1.}$\\
Vậy $N\left(4;\dfrac{10}{3};-1\right)$.
}
\end{ex}

\begin{ex}%[2-H2B4-SO-10-2425 (Nguồn: Bài 4 - Đề 1 - Ôn Tập Chương II)]%[VN-MT-7, Trần Bảo Hiên]%[2H2N2-3]
Trong không gian với hệ toạ độ $Oxyz$, cho các vectơ $\overrightarrow{a}=(1;1;2)$ và $\overrightarrow{b}=(-2;0;1)$. Tọa độ của vectơ $\overrightarrow{u}=\overrightarrow{a}-\overrightarrow{b}$ là
\choice
{\True $\overrightarrow{u}=(3;1;1)$}
{$\overrightarrow{u}=(-1;1;1)$}
{$\overrightarrow{u}=(3;1;-3)$}
{$\overrightarrow{u}=(1;3;3)$}
\loigiai{
Ta có $\overrightarrow{u}=\overrightarrow{a}-\overrightarrow{b} \Rightarrow \overrightarrow{u}=\left(1-(-2);1-0;2-1\right) \Rightarrow \overrightarrow{u}=(3;1;1)$.
}
\end{ex}

\begin{ex}%[2-H2B4-SO-10-2425 (Nguồn: Bài 4 - Đề 1 - Ôn Tập Chương II)]%[VN-MT-7, Trần Bảo Hiên]%[2H2H2-2]
Trong không gian với hệ tọa độ $Oxyz$, cho điểm $M(4;1;-2)$ và vectơ $\overrightarrow{u}=(4;-2;6)$. Tìm tọa độ điểm $N$ biết rằng $\overrightarrow{MN}=-\dfrac{1}{2}\overrightarrow{u}$.
\choice
{$(3;3;3)$}
{$(3;-3;3)$}
{\True $(2;2;-5)$}
{$(-3;-3;3)$}
\loigiai{
Ta có $-\dfrac{1}{2}\overrightarrow{u}=(-2;1;-3)$.\\
Gọi tọa độ điểm $N$ là $\left(x_N;y_N;z_N\right)$, ta có $\overrightarrow{MN}=\left(x_N-4;y_N-1;z_N+2\right)$.\\
Ta có $\overrightarrow{MN}=-\dfrac{1}{2}\overrightarrow{u}\Leftrightarrow\heva{&x_N-4=-2\\&y_N-1=1\\&z_N+2=-3}\Leftrightarrow\heva{&x_N=2\\&y_N=2\\&z_N=-5.}$\\
Vậy $N(2;2;-5)$.
}
\end{ex}

\begin{ex}%[2-H2B4-SO-10-2425 (Nguồn: Bài 4 - Đề 1 - Ôn Tập Chương II)]%[VN-MT-7, Trần Bảo Hiên]%[2H2N2-3]
Trong không gian với hệ tọa độ $Oxyz$, cho hai điểm $A(2;-1;4)$, $B(5;3;-8)$. Độ dài của vectơ $\overrightarrow{AB}$ là
\choice
{$5$}
{$8$}
{$9$}
{\True $13$}
\loigiai{
Ta có $\overrightarrow{AB}=(3;4;-12)$.\\
Độ dài của vectơ $\overrightarrow{AB}$ là $\left\vert\overrightarrow{AB}\right\vert=\sqrt{3^2+4^2+(-12)^2}=13$.
}
\end{ex}

\begin{ex}%[2-H2B4-SO-10-2425 (Nguồn: Bài 4 - Đề 1 - Ôn Tập Chương II)]%[VN-MT-7, Trần Bảo Hiên]%[2H2H2-3]
Trong không gian với hệ tọa độ $Oxyz$, cho hai vectơ $\overrightarrow{a}=(1;-2;-3)$, $\overrightarrow{b}=(-2;m-1;2)$. Tìm tham số $m$ để vectơ $\overrightarrow{a}$ vuông góc với vectơ $\overrightarrow{b}$.
\choice
{\True $m=-3$}
{$m=1$}
{$m=5$}
{$m=0$}
\loigiai{
Ta có $\overrightarrow{a}\perp\overrightarrow{b}\Leftrightarrow\overrightarrow{a}\cdot\overrightarrow{b}=0\Leftrightarrow 1\cdot(-2)+(-2)\cdot(m-1)+(-3)\cdot2=0\Leftrightarrow-2-2m+2-6=0\Leftrightarrow m=-3$.
}
\end{ex}

\begin{ex}%[2-H2B4-SO-10-2425 (Nguồn: Bài 4 - Đề 1 - Ôn Tập Chương II)]%[VN-MT-7, Trần Bảo Hiên]%[2H2N2-2]
Trong không gian với hệ tọa độ $Oxyz$, cho điểm $A(4;0;0)$, $B(0;2;0)$. Tâm đường tròn ngoại tiếp tam giác $OAB$ là
\choice
{$I(2;-1;0)$}
{$I\left(\dfrac{4}{3};\dfrac{2}{3};0\right)$}
{$I(-2;1;0)$}
{\True $I(2;1;0)$}
\loigiai{
Ta có $A(4;0;0)\in Ox$, $B(0;2;0)\in Oy$ nên tam giác $OAB$ vuông tại $O$.\\
Do đó, tâm đường tròn ngoại tiếp tam giác $OAB$ là trung điểm $I$ của cạnh $AB$.\\
Vậy $I(2;1;0)$.
}
\end{ex}

\begin{ex}%[2-H2B4-SO-10-2425 (Nguồn: Bài 4 - Đề 1 - Ôn Tập Chương II)]%[VN-MT-7, Trần Bảo Hiên]%[2H2H2-2]
Cho hai điểm $A(1;2;3)$ và $B(3;0;-5)$. Gọi $M$ là điểm đối xứng của $A$ qua $B$. Tọa độ của điểm $M$ là
\choice
{$(2;-2;-8)$}
{\True $(5;-2;-13)$}
{$(2;1;-1)$}
{$(7;2;-7)$}
\loigiai{
Vì $M$ là điểm đối xứng của $A$ qua $B$ nên $B$ là trung điểm của $AM$.
Gọi $M(x_M;y_M;z_M)$, ta có
\begin{center}
$\heva{&x_B=\dfrac{x_A+x_M}{2}\\&y_B=\dfrac{y_A+y_M}{2}\\&z_B=\dfrac{z_A+z_M}{2}}\Leftrightarrow\heva{&x_M=2x_B-x_A\\&y_M=2y_B-y_A\\&z_M=2z_B-z_A}\Leftrightarrow\heva{&x_M=2\cdot3-1=5\\&y_M=2\cdot0-2=-2\\&z_M=2\cdot(-5)-3=-13.}$
\end{center}
Vậy $M(5;-2;-13)$.
}
\end{ex}

\begin{ex}%[2-H2B4-SO-10-2425 (Nguồn: Bài 4 - Đề 1 - Ôn Tập Chương II)]%[VN-MT-7, Trần Bảo Hiên]%[2H2H2-2]
Cho tam giác $MNP$ có $M(-1;3;2)$, $N(2;2;0)$ và $P(-1;1;1)$. Biết $N$ là trọng tâm của tam giác $MNQ$. Điểm $Q$ có tọa độ là
\choice
{\True $(8;2;-3)$}
{$(4;-2;0)$}
{$(2;0;-2)$}
{$(0;-2;-2)$}
\loigiai{
Do $N$ là trọng tâm của của tam giác $MPQ$ nên $\heva{&2=\dfrac{-1-1+x_Q}{3}\\&2=\dfrac{3+1+y_Q}{3}\\&0=\dfrac{2+1+z_Q}{3}}\Leftrightarrow\heva{&x_Q=8\\&y_Q=2\\&z_Q=-3.}$\\
Vậy $N(8;2;-3)$.
}
\end{ex}

\begin{ex}%[2-H2B4-SO-10-2425 (Nguồn: Bài 4 - Đề 1 - Ôn Tập Chương II)]%[VN-MT-7, Trần Bảo Hiên]%[2H2C2-6]
Một chiếc máy ảnh được đặt trên giá đỡ ba chân với điểm đặt $E(0;0;8)$ và các điểm tiếp xúc với mặt đất của ba chân lần lượt là $A_1(0;1;0)$, $A_2\left(\dfrac{\sqrt{3}}{2};-\dfrac{1}{2};0\right)$, $A_3\left(-\dfrac{\sqrt{3}}{2};-\dfrac{1}{2};0\right)$.\\
\begin{center}
\begin{tikzpicture}[line join = round, line cap=round,>=stealth,font=\footnotesize,scale=1]
\draw[dashed,orange] (0,0) ellipse (2cm and 1cm);
\path
(0,0) coordinate (O)
(170:2cm and 1cm) coordinate (Q1)
(-10:2cm and 1cm) coordinate (Q1')
(60:2cm and 1cm) coordinate (Q4)
(-120:2cm and 1cm) coordinate (Q4')
(-70:2cm and 1cm) coordinate (Q2)
(10:2cm and 1cm) coordinate (Q3)
($(0,0)+(0,4)$) coordinate (S)
($(Q1)!-0.5!(Q1')$) coordinate (L1)
($(Q4)!1.5!(Q4')$)coordinate (L2)
($(L1)+(L2)-(O)$) coordinate (L3)
($2*(O)-(L3)$) coordinate (L4)
($2*(L1)-(L3)$) coordinate (L5)
($2*(L2)-(L3)$) coordinate (L6)
(-150:2cm and 1cm) coordinate (A2)
(80:2cm and 1cm) coordinate (A3)
(-70:2cm and 1cm) coordinate (Q2)
(-7:2cm and 1cm) coordinate (A1) ;
\draw[fill=cyan,opacity=0.2] (L3)--(L5)--(L4)--(L6)--cycle;;
\draw[->] ($(Q1)!-0.5!(Q1')$)--($(Q1)!1.5!(Q1')$) node[right]{$y$};
\draw[->] ($(Q4)!-0.5!(Q4')$)--($(Q4)!1.5!(Q4')$) node[below]{$x$};
\fill
(A1) circle(2pt) node[above right]{$A_1$}
(A3) circle(2pt) node[above right]{$A_3$}
(A2) circle(2pt) node[below left]{$A_2$}
(S) circle(2pt) node[below right,xshift=0.4cm]{$E\left(0;0;8\right)$}
(O) circle(2pt) node[below right]{$O$};
\draw[line width=1pt] (S)--(A1) (S)--(A2) (S)--(A3) ;
\draw[->,line width=1.5pt,black] (0,0)--($(O)!1.5!(S)$)node[right]{$z$};
\draw[->,line width=1.5pt,red] (S)--($(S)!0.6!(A1)$) node[above right]{$\overrightarrow{F}_1$};
\draw[->,line width=1.5pt,red] (S)--($(S)!0.6!(A2)$) node[left]{$\overrightarrow{F}_2$};
\draw[->,line width=1.5pt,red] (S)--($(S)!0.5!(A3)$) node[below right]{$\overrightarrow{F}_3$};
\draw ($(S)+(-0.2,0.5)$) node[]{
\begin{tikzpicture}[line join = round, line cap=round,>=stealth,font=\footnotesize,scale=1]
\draw[fill=black] (-0.75,-0.5)rectangle (0.5,0.5);
\draw[fill=cyan] (-0.65,-0.35)rectangle (0.35,-0.25);
\draw[fill=white] (-0.65,0.35)rectangle (0.35,0.25);
\draw[fill=black] (-0.65,0.5)rectangle (-0.5,0.75);
\draw[fill=black] (-0.65,0.65)rectangle (0.5,0.85);
\draw[fill=black] (-0.4,0)--(-1,0.5)--(-1,-0.5)--cycle;
\end{tikzpicture}
};
\end{tikzpicture}
\end{center}
Biết rằng trọng lượng của chiếc máy là $240N$. Tọa độ của lực $\overrightarrow{F_1}$ là
\choice
{\True $\overrightarrow{F_1}=(0;10;-80)$}
{$\overrightarrow{F_1}=(0;10;80)$}
{$\overrightarrow{F_1}=(0;-10;-80)$}
{$\overrightarrow{F_1}=(10;0;-80)$}
\loigiai{
Ta có $\overrightarrow{EA_1}=(0;1;-8)$, $\overrightarrow{EA_2}=\left(\dfrac{\sqrt{3}}{2};-\dfrac{1}{2};-8\right)$, $\overrightarrow{EA_3}=\left(-\dfrac{\sqrt{3}}{2};-\dfrac{1}{2};-8\right)$.\\
Nên $EA_1=EA_2=EA_3=\sqrt{65}$.\\
Mặt khác, $\left\vert\overrightarrow{F_1}\right\vert=\left\vert\overrightarrow{F_2}\right\vert=\left\vert\overrightarrow{F_3}\right\vert$ vì đèn cân bằng và trọng lực của đèn tác dụng đều lên 3 chân của giá đỡ.\\
Do đó $\overrightarrow{F_1}=k\overrightarrow{EA_1}=(0;k;-8k)$, $\overrightarrow{F_2}=k\overrightarrow{EA_2}=\left(\dfrac{\sqrt{3}}{2}k;-\dfrac{1}{2}k;-8k\right)$, $\overrightarrow{F_3}=k\overrightarrow{EA_3}=\left(-\dfrac{\sqrt{3}}{2}k;-\dfrac{1}{2}k;-8k\right)$.\\ $\Rightarrow\overrightarrow{F_1}+\overrightarrow{F_2}+\overrightarrow{F_3}=(0;0;-24k)$.\\
Mà $\overrightarrow{F_1}+\overrightarrow{F_2}+\overrightarrow{F_3}=\overrightarrow{P}=(0;0;-240)\Rightarrow -24k=-240\Rightarrow k=10$.\\
Vậy $\overrightarrow{F_1}=(0;10;-80)$.
}
\end{ex}
\Closesolutionfile{ans}

\TNTF
\Opensolutionfile{ans}[ans/ans\currfilebase-Phan-II]

\begin{ex}%[2-H2B4-SO-10-2425 (Nguồn: Bài 4 - Đề 1 - Ôn Tập Chương II)]%[VN-MT-7, Trần Bảo Hiên]%[2H2V1-2]
Cho hình lập phương $ABCD.A'B'C'D'$ có cạnh bằng $a$. Trên các cạnh $CD$ và $BB'$ ta lần lượt lấy các điểm $M$ và $N$ sao cho $DM=BN=x$ với $0\le x\le a$.
\choiceTF
{\True $\overrightarrow{AC'}=\overrightarrow{AA'}+\overrightarrow{AB}+\overrightarrow{AD}$}
{\True Gọi $K$ là trung điểm $AD$. Khi đó $\overrightarrow{C'K}=\overrightarrow{C'C}+\overrightarrow{C'D'}+\dfrac{1}{2}\overrightarrow{C'B'}$}
{$\overrightarrow{AB}\cdot\overrightarrow{B'D'}=a^2$}
{\True Góc giữa vectơ $\overrightarrow{AC'}$ và $\overrightarrow{MN}$ bằng $90^\circ$}
\loigiai{
\begin{center}
\begin{tikzpicture}[scale=0.85, font=\footnotesize, line join=round, line cap=round, >=stealth]
\path
(0:0) coordinate (B)
(0:4) coordinate (C)
($(B)+(45:2)$) coordinate (A)
($(A)+(C)-(B)$) coordinate (D)
($(A)+(90:4)$) coordinate (A')
($(B)+(90:4)$) coordinate (B')
($(C)+(90:4)$) coordinate (C')
($(D)+(90:4)$) coordinate (D')
($(A)!1/2!(D)$) coordinate (K)
($(D)!2/3!(C)$) coordinate (M)
($(B)!2/3!(B')$) coordinate (N);
\draw[dashed] (B)--(A)--(A') (D)--(A)--(C')--(K) (M)--(N);
\draw (D')--(A')--(B')--(C')--(D')--(D)--(C)--(B)--(B') (C')--(C);
\foreach \x/\g in {A/170,B/170,C/0,D/0,A'/170,B'/170,C'/0,D'/0,K/120,M/0,N/180}
\draw[fill=black] (\x) circle (.5pt)($(\g:.3)+(\x)$) node {$\x$};
\end{tikzpicture}
\end{center}
\begin{itemchoice}
\itemch \textbf{Đúng}.\\
Ta có $\overrightarrow{AC'}=\overrightarrow{AA'}+\overrightarrow{AC}$ (do $A'C'CA$ là hình bình hành).\\
Ngoài ra, ta có $\overrightarrow{AC}=\overrightarrow{AB}+\overrightarrow{AD}$ (do $ABCD$ là hình bình hành).\\
Suy ra $\overrightarrow{AC'}=\overrightarrow{AA'}+\overrightarrow{AB}+\overrightarrow{AD}$.
\itemch \textbf{Đúng}.\\
Ta có
\begin{align*}
\overrightarrow{C'K}=\overrightarrow{C'C}+\overrightarrow{CK}&=\overrightarrow{C'C}+\dfrac{1}{2}\left(\overrightarrow{CA}+\overrightarrow{CD}\right)\\ &=\overrightarrow{C'C}+\dfrac{1}{2}\left(\overrightarrow{C'A'}+\overrightarrow{C'D'}\right)\\
&=\overrightarrow{C'C}+\dfrac{1}{2}\left(\overrightarrow{C'B'}+\overrightarrow{C'D'}+\overrightarrow{C'D'}\right)\\
&=\overrightarrow{C'C}+\dfrac{1}{2}\overrightarrow{C'B'}+\overrightarrow{C'D'}.
\end{align*}
\itemch \textbf{Sai}.\\
Ta có
\begin{align*}
\overrightarrow{AB}\cdot\overrightarrow{B'D'}&=\overrightarrow{AB}\cdot\left(\overrightarrow{B'A'}+\overrightarrow{B'C'}\right)\\
&=\overrightarrow{AB}\cdot\left(-\overrightarrow{AB}+\overrightarrow{AD}\right)\\
&=-\overrightarrow{AB}\cdot\overrightarrow{AB}+\overrightarrow{AB}\cdot\overrightarrow{AD}\\
&=-{\overrightarrow{AB}}^2=-a^2.
\end{align*}
\itemch \textbf{Đúng}.\\
Ta đặt $\overrightarrow{AA'}=\overrightarrow{a}$, $\overrightarrow{AB}=\overrightarrow{b}$, $\overrightarrow{AD}=\overrightarrow{c}$. Ta có $\left\vert\overrightarrow{a}\right\vert=\left\vert\overrightarrow{b}\right\vert=\left\vert\overrightarrow{c}\right\vert=a$.\\
$\overrightarrow{AC'}=\overrightarrow{AA'}+\overrightarrow{AB}+\overrightarrow{AD}$ hay $\overrightarrow{AC'}=\overrightarrow{a}+\overrightarrow{b}+\overrightarrow{c}$.\\
Mặt khác $\overrightarrow{MN}=\overrightarrow{AN}-\overrightarrow{AM}=\left(\overrightarrow{AB}+\overrightarrow{BN}\right)-\left(\overrightarrow{AD}+\overrightarrow{DM}\right)$ với $\overrightarrow{BN}=\dfrac{x}{a}\overrightarrow{a}$ và $\overrightarrow{DM}=\dfrac{x}{a}\overrightarrow{b}$.\\
Do đó $\overrightarrow{MN}=\left(\overrightarrow{b}+\dfrac{x}{a}\overrightarrow{a}\right)-\left(\overrightarrow{c}+\dfrac{x}{a}\overrightarrow{b}\right)=\dfrac{x}{a}\overrightarrow{a}+\left(1-\dfrac{x}{a}\right)\overrightarrow{b}-\overrightarrow{c}$.\\
Ta có $\overrightarrow{AC'}\cdot\overrightarrow{MN}=\left(\overrightarrow{a}+\overrightarrow{b}+\overrightarrow{c}\right)\left[\dfrac{x}{a}\overrightarrow{a}+\left(1-\dfrac{x}{a}\right)\overrightarrow{b}-\overrightarrow{c}\right]$.\\
Vì $\overrightarrow{a}\cdot\overrightarrow{b}=\overrightarrow{a}\cdot\overrightarrow{c}=\overrightarrow{b}\cdot\overrightarrow{c}=0$ nên ta có
\begin{center}
$\overrightarrow{AC'}\cdot\overrightarrow{MN}=\dfrac{x}{a}{\overrightarrow{a}}^2+\left(1-\dfrac{x}{a}\right){\overrightarrow{b}}^2-{\overrightarrow{c}}^2=x\cdot a+\left(1-\dfrac{x}{a}\right)a^2-a^2=0$.
\end{center}
Vậy góc giữa vectơ $\overrightarrow{AC'}$ và $\overrightarrow{MN}$ bằng $90^\circ$.
\end{itemchoice}
}
\end{ex}

\begin{ex}%[2-H2B4-SO-10-2425 (Nguồn: Bài 4 - Đề 1 - Ôn Tập Chương II)]%[VN-MT-7, Trần Bảo Hiên]%[2H2V2-5]
Trong không gian với hệ tọa độ $Oxyz$, cho hình bình hành $ABCD$ có $A(-3;4;2)$, $B(-5;6;2)$, $C(-10;17;-7)$.
\choiceTF
{\True Tọa độ trung điểm của $AB$ là $I(-4;5;2)$}
{\True Tọa độ trọng tâm của tam giác $ABC$ là $G(-6;9;-1)$}
{$\overrightarrow{AB}\cdot\overrightarrow{AD}=10$}
{Tọa độ trực tâm của tam giác $ABD$ là $H(-5;12;4)$}
\loigiai{
\begin{itemchoice}
\itemch \textbf{Đúng}.\\
Gọi $I$ là trung điểm của $AB$. Khi đó $\heva{&x_I=\dfrac{x_A+x_B}{2}=\dfrac{-3+(-5)}{2}=-4\\&y_I=\dfrac{y_A+y_B}{2}=\dfrac{4+6}{2}=5\\&z_I=\dfrac{z_A+z_B}{2}=\dfrac{2+2}{2}=2.}$\\
Vậy $I(-4;5;2)$.
\itemch \textbf{Đúng}.\\
Gọi $G$ là trọng tâm của tam giác $ABC$.\\
Khi đó $\heva{&x_G=\dfrac{x_A+x_B+x_C}{3}=\dfrac{-3+(-5)+(-10)}{3}=-6\\&y_I=\dfrac{y_A+y_B+y_C}{3}=\dfrac{4+6+17}{3}=9\\&z_I=\dfrac{z_A+z_B+z_C}{3}=\dfrac{2+2+(-7)}{3}=-1.}$\\
Vậy $G(-6;9;-1)$.
\itemch \textbf{Sai}.\\
Ta có $\overrightarrow{AB}=(-2;2;0)$, $\overrightarrow{DC}=(-10-x_D;17-y_D;-7-z_D)$.
Vì $ABCD$ là hình bình hành nên $\overrightarrow{AB}=\overrightarrow{DC}\Leftrightarrow\heva{&-10-x_D=-2\\&17-y_D=2\\&-7-z_D=0}\Leftrightarrow\heva{&x_D=-8\\&y_D=15\\&z_D=-7}\Rightarrow D(-8;15;-7)$.\\
$\overrightarrow{AD}=(-5;11;-9)$. Do đó $\overrightarrow{AB}\cdot\overrightarrow{AD}=-2\cdot(-5)+2\cdot11+0\cdot(-9)=32$.
\itemch \textbf{Sai}.\\
Gọi $H(a;b;c)$ là trực tâm tam giác $ABD$. Do $\left[\overrightarrow{AB},\overrightarrow{AD}\right]$ có giá vuông góc với $(ABC)$ nên nó vuông góc với vectơ $\overrightarrow{AH}$.\\
Do đó $\heva{&\overrightarrow{AH}\perp\overrightarrow{BD}\\&\overrightarrow{BH}\perp\overrightarrow{AD}\\&\left[\overrightarrow{AB},\overrightarrow{AD}\right]\cdot\overrightarrow{AH}=0.}$\\
Ta có 
$\overrightarrow{AH}=(a+3;b-4;c-2)$, $\overrightarrow{BH}=(a+5;b-6;c-2)$, $\overrightarrow{BD}=(-3;9;-9)$, $\overrightarrow{AD}=(-5;11;-9)$, $\overrightarrow{AB}=(-2;2;0)$, 
$\left[\overrightarrow{AB},\overrightarrow{AD}\right]=(-18;-18;-12)$.
Suy ra 
\begin{center}
$\heva{&-3(a+3)+9(b-4)-9(c-2)=0\\&-5(a+5)+11(b-6)-9(c-2)=0\\&-18(a+3)-18(b-4)-12(c-2)=0}\Leftrightarrow\heva{&-3a+9b-9c=27\\&-5a+11b-9c=73\\&-18a-18b-12c=-42}\Leftrightarrow\heva{&a=-\dfrac{153}{11}\\&b=\dfrac{100}{11}\\&c=\dfrac{118}{11}.}$
\end{center}
Vậy $H\left(-\dfrac{153}{11};\dfrac{100}{11};\dfrac{118}{11}\right)$.
\end{itemchoice}
}
\end{ex}

\begin{ex}%[2-H2B4-SO-10-2425 (Nguồn: Bài 4 - Đề 1 - Ôn Tập Chương II)]%[VN-MT-7, Trần Bảo Hiên]%[2H2H2-4]
Trong không gian với hệ tọa độ $Oxyz$, cho các điểm $A(2;1;-1)$, $B(3;1;0)$, $C(-1;1;3)$.
\choiceTF
{\True Ba điểm $A$, $B$, $C$ không thẳng hàng}
{\True Ba điểm $A$, $B$, $D(4;1;1)$ thẳng hàng}
{Góc $\widehat{ABC}=45^\circ$}
{\True $\left[\overrightarrow{AB},\overrightarrow{AC}\right]=(0;-7;0)$}
\loigiai{
\begin{itemchoice}
\itemch \textbf{Đúng}.\\
Ta có $\overrightarrow{AB}=(1;0;1)$, $\overrightarrow{AC}=(-3;0;4)$, $\overrightarrow{AB}\ne k\overrightarrow{AC}=(-3k;0;4k)$ với mọi $k$ nên hai vectơ $\overrightarrow{AB}$ và $\overrightarrow{AC}$ không cùng phương. Do đó ba điểm $A$, $B$, $C$ không thẳng hàng.
\itemch \textbf{Đúng}.\\
Ta có $\overrightarrow{AB}=(1;0;1)$, $\overrightarrow{AD}=(2;0;2)\Rightarrow\overrightarrow{AD}=2\overrightarrow{AB}$. Do đó ba điểm $A$, $B$, $D$ thẳng hàng.
\itemch \textbf{Sai}.\\
Ta có $\overrightarrow{BA}=(-1;0;-1)$, $\overrightarrow{BC}=(-4;0;3)$, suy ra
\begin{center}
$\cos\widehat{ABC}=\cos\left(\overrightarrow{BA},\overrightarrow{BC}\right)=\dfrac{\overrightarrow{BA}\cdot\overrightarrow{BC}}{\left\vert\overrightarrow{BA}\right\vert\cdot\left\vert\overrightarrow{BC}\right\vert}=\dfrac{1}{5\sqrt{2}}\Rightarrow\widehat{ABC}\approx 82^\circ$.\\
\end{center}

\itemch \textbf{Đúng}.\\
Ta có $\overrightarrow{AB}=(1;0;1)$, $\overrightarrow{AC}=(-3;0;4)\Rightarrow\left[\overrightarrow{AB},\overrightarrow{AC}\right]=(0;-7;0)$.
\end{itemchoice}
}
\end{ex}

\begin{ex}%[2-H2B4-SO-10-2425 (Nguồn: Bài 4 - Đề 1 - Ôn Tập Chương II)]%[VN-MT-7, Trần Bảo Hiên]%[2H2V2-5]
Trong không gian với hệ tọa độ $Oxyz$, cho các điểm $A(1;1;2)$, $B(3;-1;2)$, $C(2;0;1)$.
\choiceTF
{\True Ba điểm $A$, $B$, $C$ không thẳng hàng}
{\True Điểm $M(a;b;3)$ thỏa mãn ba điểm $A$, $C$, $M$ thẳng hàng thì $a+b=2$}
{Góc $\alpha$ là góc tạo bởi hai vectơ $\overrightarrow{AB}$, $\overrightarrow{BC}$ thì $\cos\alpha=-1$}
{Gọi điểm $M(a;b;3)$ thỏa mãn ba điểm $A$, $C$, $M$ thẳng hàng. Khi đó $\left[\overrightarrow{AB},\overrightarrow{AM}\right]=(1;1;2)$}
\loigiai{
\begin{itemchoice}
\itemch \textbf{Đúng}.\\
Ta có $\overrightarrow{AB}=(2;-2;0)$, $\overrightarrow{BC}=(-1;1;-1)$. Suy ra $\overrightarrow{AB}\ne k\cdot\overrightarrow{BC}$ với mọi $k\in\mathbb{R}$ nên ba điểm $A$, $B$, $C$ không thẳng hàng.
\itemch \textbf{Đúng}.\\
Ta có $\overrightarrow{AC}=(1;-1;-1)$, $\overrightarrow{AM}=(a-1;b-1;1)$.\\
Ba điểm $A$, $C$, $M$ thẳng hàng khi và chỉ khi $\overrightarrow{AB}=k\overrightarrow{AM}\Leftrightarrow\heva{&1=k(a-1)\\&-1=k(b-1)\\&-1=k}\Leftrightarrow\heva{&a=0\\&b=2\\&k=-1.}$\\
Vậy $a+b=2$.
\itemch \textbf{Sai}.\\
\begin{center}
$\cos\alpha=\cos\left(\overrightarrow{AB},\overrightarrow{BC}\right)=\dfrac{\overrightarrow{AB}\cdot\overrightarrow{BC}}{\left\vert\overrightarrow{AB}\right\vert\cdot\left\vert\overrightarrow{BC}\right\vert}=\dfrac{2\cdot(-1)+(-2)\cdot1+0\cdot(-1)}{2\sqrt{2}\cdot\sqrt{3}}=-\dfrac{\sqrt{6}}{3}$.
\end{center}
\itemch \textbf{Sai}.\\
$\overrightarrow{AB}=(2;-2;0)$, $\overrightarrow{AM}=(-1;1;1)$.\\
$\left[\overrightarrow{AB},\overrightarrow{AM}\right]=(-2;-2;0)$.
\end{itemchoice}
}
\end{ex}
\Closesolutionfile{ans}

\TNSA
\Opensolutionfile{ans}[ans/ans\currfilebase-Phan-III]

\begin{ex}%[2-H2B4-SO-10-2425 (Nguồn: Bài 4 - Đề 1 - Ôn Tập Chương II)]%[VN-MT-7, Trần Bảo Hiên]%[2H2V2-4]
Cho hình chóp $S.ABC$ có $SA=SB=SC=AB=AC=a$, $BC=a\sqrt{2}$. Góc giữa hai véctơ $\overrightarrow{AB}$ và $\overrightarrow{SC}$ bằng bao nhiêu độ?
\shortans{120}
\loigiai{
\begin{center}
\begin{tikzpicture}[scale=0.8,font=\footnotesize,line join=round,line cap=round,>=stealth]
\coordinate (S) at ($(A)+(50:3)$);
\coordinate (A) at (0,0);
\coordinate (B) at ($(A)+(-60:3)$);
\coordinate (C) at ($(A)+(0:5)$);
\draw(S)--(A)--(B)--(C)--(S)--(B);
\draw[dashed](A)--(C);
\foreach \i/\g in {S/90,A/180,B/-90,C/0}{\fill (\i) circle (1.0pt) ($(\i)+(\g:3mm)$)node[scale=1]{$\i$};}
\end{tikzpicture}
\end{center}
Tam giác $ABC$ có $AB=AC=a$, $BC=a\sqrt{2}\Rightarrow\triangle ABC$ vuông tại $A\Rightarrow\overrightarrow{AB}\cdot\overrightarrow{AC}=0$.\\
\begin{align*}
\cos\left(\overrightarrow{SC},\overrightarrow{AB}\right)&=\dfrac{\overrightarrow{SC}\cdot\overrightarrow{AB}}{\left\vert\overrightarrow{SC}\right\vert\cdot\left\vert\overrightarrow{AB}\right\vert}=\dfrac{\left(\overrightarrow{SA}+\overrightarrow{AC}\right)\cdot\overrightarrow{AB}}{SC\cdot AB}\\
&=\dfrac{\overrightarrow{SA}\cdot\overrightarrow{AB}+\overrightarrow{AC}\cdot\overrightarrow{AB}}{SC\cdot AB}=\dfrac{SA\cdot AB\cdot \cos120^\circ}{SC\cdot AB}=-\dfrac{1}{2}.
\end{align*}
Suy ra $\left(\overrightarrow{SC},\overrightarrow{AB}\right)=120^\circ$.
}
\end{ex}

\begin{ex}%[2-H2B4-SO-10-2425 (Nguồn: Bài 4 - Đề 1 - Ôn Tập Chương II)]%[VN-MT-7, Trần Bảo Hiên]%[2H2H2-2]
Trong không gian với hệ tọa độ $Oxyz$, cho hình hộp $ABCD.A'B'C'D'$ có $A(1;0;1)$, $B(2;1;2)$, $D(1;-1;1)$, $C'(4;5;-5)$. Giả sử $A'(x;y;z)$, tính $x+y+z$.
\shortans{2}
\loigiai{
\begin{center}
\begin{tikzpicture}[scale=0.8, font=\footnotesize, line join=round, line cap=round, >=stealth]
\path
(0:0) coordinate (B)
(0:4) coordinate (C)
($(B)+(45:2)$) coordinate (A)
($(A)+(C)-(B)$) coordinate (D)
($(A)+(80:3)$) coordinate (A')
($(B)+(80:3)$) coordinate (B')
($(C)+(80:3)$) coordinate (C')
($(D)+(80:3)$) coordinate (D');
\draw[dashed] (B)--(A)--(A') (A)--(D);
\draw (D')--(A')--(B')--(C')--(D')--(D)--(C)--(B)--(B') (C)--(C');
\foreach \x/\g in {A/170,B/170,C/0,D/0,A'/170,B'/170,C'/0,D'/0}
\draw[fill=black] (\x) circle (.5pt)
($(\g:.3)+(\x)$) node {$\x$};
\end{tikzpicture}
\end{center}
Ta có $\overrightarrow{AC'}=(3;5;-6)$, $\overrightarrow{AB}=(1;1;1)$, $\overrightarrow{AD}=(0;-1;0)$.\\
Theo quy tắc hình bình hành ta có $\overrightarrow{AB}+\overrightarrow{AD}+\overrightarrow{AA'}=\overrightarrow{AC'}$, suy ra
\begin{center}
$\overrightarrow{AA'}=\overrightarrow{AC'}-\overrightarrow{AB}-\overrightarrow{AD}\Rightarrow\overrightarrow{AA'}=(2;5;-7)$.
\end{center}
Ta có $A'(x;y;z)\Rightarrow\overrightarrow{AA'}=(2;5;-7)\Leftrightarrow\heva{&x-1=2\\&y=5\\&z-1=-7}\Leftrightarrow\heva{&x=3\\&y=5\\&z=-6}\Rightarrow A'(3;5;-6)$.\\
Vậy $x+y+z=3+5+(-6)=2$.
}
\end{ex}

\begin{ex}%[2-H2B4-SO-10-2425 (Nguồn: Bài 4 - Đề 1 - Ôn Tập Chương II)]%[VN-MT-7, Trần Bảo Hiên]%[2H2V2-5]
Cho hình hộp $ABCD.A'B'C'D'$ có $A(1;0;1)$, $B(2;1;2)$, $D(1;-1;1)$, $C'(4;5;-5)$. Biết rằng có một vectơ $\overrightarrow{v}=(a;b;6)$ vuông góc với cả hai vectơ $\overrightarrow{CC'}$ và $\overrightarrow{C'D'}$. Tính $a+b$.
\shortans{-6}
\loigiai{
\begin{center}
\begin{tikzpicture}[scale=0.8, font=\footnotesize, line join=round, line cap=round, >=stealth]
\path
(0:0) coordinate (B)
(0:4) coordinate (C)
($(B)+(45:2)$) coordinate (A)
($(A)+(C)-(B)$) coordinate (D)
($(A)+(80:3)$) coordinate (A')
($(B)+(80:3)$) coordinate (B')
($(C)+(80:3)$) coordinate (C')
($(D)+(80:3)$) coordinate (D');
\draw[dashed] (B)--(A)--(A') (A)--(D);
\draw (D')--(A')--(B')--(C')--(D')--(D)--(C)--(B)--(B') (C)--(C');
\foreach \x/\g in {A/170,B/170,C/0,D/0,A'/170,B'/170,C'/0,D'/0}
\draw[fill=black] (\x) circle (.5pt)($(\g:.3)+(\x)$) node {$\x$};
\end{tikzpicture}
\end{center}
Ta có $\overrightarrow{AB}=(1;1;1)$, $\overrightarrow{AD}=(0;-1;0)$.\\
Gọi $C(x;y;z)\Rightarrow\overrightarrow{AC}=(x-1;y;z-1)$. Theo quy tắc hình bình hành ta có
\begin{center}
$\overrightarrow{AC}=\overrightarrow{AB}+\overrightarrow{AD}\Leftrightarrow\heva{&x-1=1\\&y=0\\&z-1=1}\Leftrightarrow\heva{&x=2\\&y=0\\&z=2}\Rightarrow\overrightarrow{CC'}=(2;5;-7)$.
\end{center}
Mặt khác $\overrightarrow{C'D'}=\overrightarrow{BA}=(-1;-1;-1)$.\\
Suy ra $\left[\overrightarrow{CC'},\overrightarrow{C'D'}\right]=(-12;9;3)$ là một vectơ thoả mãn yêu cầu bài toán.\\
Ta có $\left[\overrightarrow{CC'},\overrightarrow{C'D'}\right]$ và $\overrightarrow{v}$ cùng phương nên có số thực $k$ để $\left[\overrightarrow{CC'};\overrightarrow{C'D'}\right]=k\cdot\overrightarrow{v}$.\\
Suy ra $\heva{&-12=k\cdot a\\&9=k\cdot b\\&3=k\cdot6}\Leftrightarrow\heva{&a = -24\\&b = 18\\&k=\dfrac{1}{2}.}$\\
Vậy $a+b=-6$.
}
\end{ex}

\begin{ex}%[2-H2B4-SO-10-2425 (Nguồn: Bài 4 - Đề 1 - Ôn Tập Chương II)]%[VN-MT-7, Trần Bảo Hiên]%[2H2V2-6]
Trong một căn phòng dạng hình hộp chữ nhật với chiều dài $8$ m, rộng $6$ m và cao $4$ m có cây quạt treo tường. Cây quạt $A$ treo chính gữa bức tường $8$ m và cách trần $1$ m, cây quạt $B$ treo chính giữa bức tường $6$ m và cách trần $1{,}5$ m. Chọn hệ trục tọa độ $Oxyz$ như hình vẽ bên dưới (đơn vị: mét). Hãy tính độ dài vectơ $\overrightarrow{AB}$ (làm tròn đến hàng đơn vị).
\begin{center}
\definecolor{amber}{rgb}{1.0, 0.75, 0.0}%mau non
\definecolor{antiquebrass}{rgb}{0.8, 0.58, 0.46}%mau da
\definecolor{antiquewhite}{rgb}{0.98, 0.92, 0.84}%mau ao
\definecolor{cadmiumgreen}{rgb}{0.0, 0.42, 0.24}%mau quan
\definecolor{cadetblue}{rgb}{0.37, 0.62, 0.63}%mau but
\definecolor{brown(traditional)}{rgb}{0.59, 0.29, 0.0}%mau giay
\definecolor{brilliantlavender}{rgb}{0.96, 0.73, 1.0}%màu sơn tím
\definecolor{brightube}{rgb}{0.82, 0.62, 0.91}%màu sơn tím đậm
%---------------màu quạt
\definecolor{burntorange}{rgb}{0.8, 0.33, 0.0}
\definecolor{arsenic}{rgb}{0.23, 0.27, 0.29}
\definecolor{battleshipgrey}{rgb}{0.52, 0.52, 0.51}
\begin{tikzpicture}[line join=round, line cap=round,scale=0.7,transform shape,font=\large]
\clip (1,-1) rectangle (16,13);
%\draw[gray!50] (-3,-3) grid (3,4);
\definecolor{burntsienna}{rgb}{0.91, 0.45, 0.32}
\tikzset{mai/.pic={
\def\mainha{
(.5,2)
foreach \n in {1,2,...,22} { -- ++ (0,0) -- ++ (0,1) -- ++ (1,0) -- ++ (0,-1) } -- cycle
(.5,1)
foreach \n in {1,2,...,22} { -- ++ (0,0) -- ++ (0,1) -- ++ (1,0) -- ++ (0,-1) } -- cycle
(.5,0)
foreach \n in {1,2,...,22} { -- ++ (0,0) -- ++ (0,1) -- ++ (1,0) -- ++ (0,-1) } -- cycle
(.5,-1)
foreach \n in {1,2,...,22} { -- ++ (0,0) -- ++ (0,1) -- ++ (1,0) -- ++ (0,-1) } -- cycle
(.5,-2)
foreach \n in {1,2,...,22} { -- ++ (0,0) -- ++ (0,1) -- ++ (1,0) -- ++ (0,-1) } -- cycle
(.5,-3)
foreach \n in {1,2,...,22} { -- ++ (0,0) -- ++ (0,1) -- ++ (1,0) -- ++ (0,-1) } -- cycle
(.5,-4)
foreach \n in {1,2,...,22} { -- ++ (0,0) -- ++ (0,1) -- ++ (1,0) -- ++ (0,-1) } -- cycle;
}
\clip (-4,-3)--(17.5,-3)--(17.5,3)--(.5,3)--cycle;
\draw[white,fill=burntsienna!70] \mainha;
\draw (17.5,-3)--(17.5,3)--(17.5,9);}}
\fill[brilliantlavender] (18,-3)--(18,13)--(12,13)--(12,-3)--cycle;
\fill[brightube] (12,13)--(12,3)--(-18,3)--(-18,13)--cycle;
\path(-2.5,0)pic[scale=1,xslant=-1]{mai};
\path
(12,3) coordinate (O)
(3,3) coordinate (x)
(12,13) coordinate (z)
(15,0) coordinate (y);
\foreach\p/\g in {y/160,x/45, z/-45,O/-90}
{
\node at (\p) [shift=(\g:4mm)] {$\p$};
}
\draw[line width=.5mm,->] (O)--(z) ;
\draw[line width=.5mm,->] (O)--(x);
\draw[line width=.5mm,->] (O)--(y);
\node at ($(O)+(-6,4.5)$) {$8$ m};
\draw[red,line width=.5mm,<->] ($(O)+(0,4)$)--($(x)+(-2,4)$);
%===========================A FAN
\tikzset{fan/.pic={
%chân quạt
\draw[fill=battleshipgrey](-.2,.8)--(.2,.8)--(.4,-2.2)
..controls +(-120:.3) and +(-60:.3) ..(-.4,-2.2)--cycle;
%---Nút bấm
\foreach \i in{-1.95,-1.7}{%-1.45
\draw[fill=arsenic](-.15,\i) rectangle (.15,\i+.15);
}
%-----------------------------------------------------
\draw[black](0,.8) circle (2.25cm);
\draw[black](0,.8) circle (2.15cm);
\draw[black](0,.8) circle (1.42cm);
\draw[black](0,.8) circle (1.48cm);
\draw[fill=black](0,.8) circle (6mm);
\draw[fill=arsenic](0,.8) circle (5mm);
\def\N{
(0,.8)
..controls +(145:1.3) and +(170:1) ..(0,2.8)
..controls +(-10:1.4) and +(-20:1) ..(.6,1.76)
..controls +(160:.4) and +(100:.4) ..(.2,.9)--cycle;
}
\foreach \i/\j/\k in {0/0/0,120/.7/-1.2,240/-.7/-1.2}
{
\draw[black,rotate=\i,shift={(\j,\k)}]\N;
\fill[burntorange,rotate=\i,shift={(\j,\k)}] \N;
}
%lồng quạt
\def\r{2.15}
\foreach \i in {0,15,25,35,...,365}
\draw[double] ($(\i:\r)+(0,.8)$)--(0,.8);
\draw[fill=arsenic](0,.8) circle (3.5mm);
}}
\path
(6,11)pic[scale=.6]{fan}
(14,9)pic[scale=.55,yslant=-.3]{fan};
\end{tikzpicture}
\end{center}
\shortans{5}
\loigiai{
Từ hình vẽ $A\in(Oxz)$ nên $A(x;0;z)$ và $B\in(Oyz)$ nên $B(0;y;z)$.\\
Cây quạt $A$ treo chính giữa bức tường $8$ m và cách trần $1$ m nên $A(4;0;3)$.\\
Cây quạt $B$ treo chính giữa bức tường $6$ m và cách trần $1{,}5$ m nên $B\left(0;3;\dfrac{5}{2}\right)$.\\
Khi đó $\overrightarrow{AB}=\left(-4;3;-\dfrac{1}{2}\right)\Rightarrow\left\vert\overrightarrow{AB}\right\vert=\sqrt{(-4)^2+3^2+\left(-\dfrac{1}{2}\right)^2}\approx 5$ m.
}
\end{ex}

\begin{ex}%[2-H2B4-SO-10-2425 (Nguồn: Bài 4 - Đề 1 - Ôn Tập Chương II)]%[VN-MT-7, Trần Bảo Hiên]%[2H2V2-4]
Một chi tiết trong bộ trang sức được gắn hệ trục tọa độ $Oxyz$ như hình vẽ. Các hình chóp $S.ABCD$ và $I.ABCD$ là các hình chóp đều cạnh $1$ cm. Tính số đo góc nhị diện $[S,CD,I]$ theo đơn vị độ, làm tròn đến hàng đơn vị.
\shortans{109}
\loigiai{
\begin{center}
\begin{tikzpicture}[scale=0.8,font=\footnotesize,line join=round,line cap=round,>=stealth]
\def \a{6}
\def \h{5}
\path (0:0) coordinate (A)
++(0:\a) coordinate (D)
++(-138:0.45*\a) coordinate (C)
($(A)+(C)-(D)$) coordinate (B)
($(A)!0.5!(C)$) coordinate (O)
($(O)+(90:\h)$) coordinate (S)
($(S)!-0.2!(O)$) coordinate (H)
($(C)!-0.4!(A)$) coordinate (K)
($(D)!-0.2!(B)$) coordinate (L)
($(D)!0.5!(C)$) coordinate (M)
($(O)!1!180:(S)$) coordinate (I);
\draw [dashed] (B)--(A)--(D) (A)--(S) (A)--(C) (B)--(D) (S)--(O)--(I)--(A);
\draw (B)--(C)--(D) (B)--(S) (C)--(S) (D)--(S) (C)--(I)--(B) (I)--(D);
\draw[->] (S)--(H)node[right]{$z$};
\draw[->] (C)--(K)node[above]{$x$};
\draw[->] (D)--(L)node[above]{$y$};
\foreach \x/\g in {A/135,B/180,C/80,D/70,S/40,O/50,I/-90,M/150}
\fill (\x) circle (1pt)
($(\g:3mm)+(\x)$) node {$\x$};
\end{tikzpicture}
\end{center}
Ta có $ABCD$ là hình vuông cạnh $1$ cm nên $OC=OD=\dfrac{\sqrt{2}}{2}$.\\
Xét $\triangle SOC$ vuông tại $O$, ta có $OS=\sqrt{SC^2-OC^2}=\sqrt{1^2-\left(\dfrac{\sqrt{2}}{2}\right)^2}=\dfrac{\sqrt{2}}{2}$.\\
Xét $\triangle IOC$ vuông tại $O$, ta có $OI=\sqrt{IC^2-OC^2}=\sqrt{1^2-\left(\dfrac{\sqrt{2}}{2}\right)^2}=\dfrac{\sqrt{2}}{2}$.\\
Vậy $C\left(\dfrac{\sqrt{2}}{2};0;0\right)$, $D\left(0;\dfrac{\sqrt{2}}{2};0\right)$, $S\left(0;0;\dfrac{\sqrt{2}}{2}\right)$, $I\left(0;0;-\dfrac{\sqrt{2}}{2}\right)$.\\
Gọi $M$ là trung điểm của $CD$ thì $M\left(\dfrac{\sqrt{2}}{4};\dfrac{\sqrt{2}}{4};0\right)$.\\
Ta có $\heva{&CD\perp MI\\&CD\perp MS}\Rightarrow[S,CD,I]=\widehat{SMI}$.\\
Ta có $\overrightarrow{MS}=\left(-\dfrac{\sqrt{2}}{4};-\dfrac{\sqrt{2}}{4};\dfrac{\sqrt{2}}{2}\right)$, $\overrightarrow{MI}=\left(-\dfrac{\sqrt{2}}{4};-\dfrac{\sqrt{2}}{4};-\dfrac{\sqrt{2}}{2}\right)$.\\
$\Rightarrow\overrightarrow{MS}\cdot\overrightarrow{MI}=-\dfrac{1}{4}$, $\left\vert\overrightarrow{MS}\right\vert=\dfrac{\sqrt{3}}{2}$, $\left\vert\overrightarrow{MI}\right\vert=\dfrac{\sqrt{3}}{2}$.
\begin{align*}
\cos\widehat{SMI}=\cos\left(\overrightarrow{MS},\overrightarrow{MI}\right)=\dfrac{\overrightarrow{MS}\cdot\overrightarrow{MI}}{\left\vert\overrightarrow{MS}\right\vert\cdot\left\vert\overrightarrow{MI}\right\vert}=\dfrac{-\dfrac{1}{4}}{\dfrac{\sqrt{3}}{2}\cdot\dfrac{\sqrt{3}}{2}}=-\dfrac{1}{3}\Rightarrow\widehat{SMI}\approx109^\circ.
\end{align*}
}
\end{ex}

\begin{ex}%[2-H2B4-SO-10-2425 (Nguồn: Bài 4 - Đề 1 - Ôn Tập Chương II)]%[VN-MT-7, Trần Bảo Hiên]%[2H2C1-4]
\immini[thm]{Một chiếc ô tô được đặt trên mặt đáy dưới của một khung sắt có dạng hình hộp chữ nhật với đáy trên là hình chữ nhật $ABCD$, mặt phẳng $(ABCD)$ song song với mặt phẳng nằm ngang. Khung sắt đó được buộc vào móc $E$ của chiếc cần cẩu sao cho các đoạn dây cáp $EA$, $EB$, $EC$ và $ED$ có độ dài bằng nhau và cùng tạo với mặt phẳng $(ABCD)$ một góc bằng $60^\circ$ (hình minh họa). Chiếc cần cẩu đang kéo khung sắt lên theo phương thẳng đứng. Biết rằng các lực căng $\overrightarrow{F_1}$, $\overrightarrow{F_2}$, $\overrightarrow{F_3}$, $\overrightarrow{F_4}$ đều có cường độ là $4{,}7$ kN và trọng lượng của khung sắt là $3$ kN. Tính trọng lượng của chiếc xe ô tô (làm tròn đến hàng phần chục)?
}
{
\begin{tikzpicture}[>=latex,line join=round, line cap=round,scale=0.8,transform shape]
\definecolor{bostonuniversityred}{rgb}{0.8, 0.0, 0.0}
\definecolor{charcoal}{rgb}{0.21, 0.27, 0.31}
\definecolor{bananayellow}{rgb}{1.0, 0.88, 0.21}
\definecolor{anti-flashwhite}{rgb}{0.95, 0.95, 0.96}
% \clip (-6,-3) rectangle (6,3);
\tikzset{%
xeoto/.pic={%
%--------------------------
\tikzset{xe/.pic={
\def\N{
(-2.7,.56)--(-2.5,.56)
..controls +(50:1.5) and +(165:1.5) .. (2.1,1.88)--(2.05,2)
..controls +(-10:.1) and +(130:.1) .. (3.25,1.75)--(3.15,1.65)
..controls +(-4:.2) and +(130:.15) .. (4.05,.7)--(4.25,.75)
..controls +(-40:.2) and +(130:.15) .. (4.55,.35)--(4.35,.26)
..controls +(-40:.2) and +(130:.15) .. (4.8,-.45)--(4.92,-.4)
..controls +(-40:.25) and +(73:.17) .. (4.8,-1.8)--(-4.4,-1.8)
..controls +(175:.7) and +(-175:3.2) ..cycle;
}
\fill[bostonuniversityred] \N;
\draw \N;
\def\K{
(-2.2,.56)--(3.3,.7)
..controls +(100:1.18) and +(43:3) .. cycle;
}
\fill[bottom color=charcoal,top color=charcoal!20!white, middle color=charcoal!80!white] \K;
\draw \K;
\def\K1{
(-2.2,.56)..controls +(43:.2) and +(43:.2) .. (-1.58,1.05)--(-1.53,.57)--cycle;
}
\draw \K1;
\fill[charcoal] \K1;
\def\K2{
(1.2,1.85)..controls +(-10:.1) and +(160:.1) .. (1.58,1.8)--(1.8,.65)--(1.25,.65)--cycle;
}
\draw \K2;
\fill[charcoal] \K2;
\def\Kt{
(-2.5,.56)..controls +(50:1.5) and +(165:1.5) .. (2.1,1.88)--(2.05,2)
..controls +(170:2.2) and +(45:1.5) .. (-2.7,.56)--cycle;
}
\fill[charcoal!50] \Kt;
\draw \Kt;
\def\Ks{
(3.25,1.75)--(3.15,1.65)
..controls +(-4:.2) and +(130:.15) .. (4.05,.7)--(4.22,.75)
..controls +(120:.3) and +(-35:.3) .. cycle;
}
\fill[charcoal!50] \Ks;
\draw \Ks;
%Đèn sau
\def\D{
(4.55,.35)--(4.35,.26)
..controls +(-40:.2) and +(130:.15) .. (4.8,-.45)--(4.92,-.4)
..controls +(110:.2) and +(-40:.15) ..cycle;
}
\fill[bananayellow] \D;
\draw \D;
\def\M{
(2.2,-1.3)--(-1.8,-1.4)--(-1.78,-1.7)
..controls +(-5:.3) and +(-90:.6) ..cycle;
}
\draw \M;
\fill[charcoal!90] \M;
\draw (-1.6,.55)..controls +(-170:.5) and +(95:.4) .. (-1.78,-1.7)
(1.6,.65)..controls +(-30:.5) and +(35:.3) .. (1.7,-1.3);
%gương
\def\G{ (-1.5,.45)--(-1.4,.6)..controls +(85:1) and +(20:.6) .. (-1.25,.5)--(-1.4,.33);
}
\draw \G;
\fill[bostonuniversityred] \G;
%Đèn trước
\def\Dt{
(-4.85,-.7)..controls +(75:1) and +(65:.8) .. (-4.5,-.7)
..controls +(-115:.6) and +(-105:.4) .. cycle;
}
\fill[bananayellow] \Dt;
\draw \Dt;
\def\Dt2{
(-4.85,-.7)
..controls +(75:.6) and +(65:.4) .. (-4.7,-.7)
..controls +(-115:.3) and +(-105:.2) .. cycle;
}
\fill[anti-flashwhite] \Dt2;
\draw \Dt2;
\draw[fill=anti-flashwhite] (-4.86,-1.45)--(-4.82,-1.5)--(-4.55,-1.3)
..controls +(90:.3) and +(45:.2) .. cycle;
}}
\tikzset{banh_xe/.pic={
\draw[fill=charcoal] (-3.25,-1.65) circle (1) ;
\draw[fill=anti-flashwhite] (-3.25,-1.65) circle (.7) ;
\draw[fill=charcoal] (-3.25,-1.65) circle (.4) ;
}}
%----------------
\path
(0,0)pic[scale=1]{xe}(0,0)pic[scale=1]{banh_xe}(6.9,0)pic[scale=1]{banh_xe};
}}
\def\bc{4.25} % cạnh BC
\def\ba{1.5} % cạnh BA
\def\h{4} % đường cao
\def\gocnghieng{90} % góc nghiêng
\def\gocB{160} % góc B của đáy
\coordinate (B1) at (0,0);
\coordinate (A1) at (\gocB:\ba);
\coordinate (C1) at (\bc,0.25);
\coordinate (D1) at ($(C1)-(B1)+(A1)$);
\coordinate[label=above left:$A$] (A) at ($(A1)+(\gocnghieng:\h)$);
\coordinate[label=below left:$B$] (B) at ($(B1)-(A1)+(A)$);
\coordinate[label=right:$C$] (C) at ($(C1)-(A1)+(A)$);
\coordinate[label=above right:$D$] (D) at ($(D1)-(A1)+(A)$);
\coordinate (E) at ($(A)!0.5!(C)+(\gocnghieng:\h)$);
%------------
\draw[->,blue,very thick] (E)--($(E)!0.4!(A)$) node[above left]{$\overrightarrow{F_1}$};
\draw[->,blue,very thick] (E)--($(E)!0.4!(B)$) node[right]{$\overrightarrow{F_2}$};
\draw[->,blue,very thick] (E)--($(E)!0.4!(C)$) node[above right]{$\overrightarrow{F_3}$}; \draw[->,blue,very thick] (E)--($(E)!0.4!(D)$) node[left=2pt]{$\overrightarrow{F_4}$};
%------------
\path (E) node[left=1mm]{$E$};
\draw[blue,very thick] (A)--(B)--(C)--(D)--cycle
(A1)--(A) (D1)--(D) (C1)--(C)
(A)--(E)--(B) (C)--(E)--(D);
\draw[fill=teal] (A1)--(B1)--(C1)--(D1)--cycle;
\draw[fill=teal!30] (A1)--(B1)--(C1)--++(0,-0.3)--([yshift=-0.3cm]B1)--([yshift=-0.3cm]A1)--cycle;
\foreach \diem in {A1,B1,C1,D1,A,B,C,D,E} \fill (\diem)circle(1.5pt);
%phần móc và dây
\def\r{0.3}\def\rr{0.25}
\coordinate (tam) at ([yshift=6mm]E);
\draw[brown,fill=brown,line width=1pt] (tam) circle (\r cm);
\fill (tam) circle (2pt);
\draw[brown,line width=1pt] (tam)++(\r,0)--++(0,0.7)(tam)++(-\r,0)--++(0,0.7);
\draw[line width=1.5pt] (tam)--++(0,-1.35*\r) arc(90:370:1mm);
%%%%%%%%%%%%%%%%%%%
\pic[scale=0.45,rotate=4] at (1.6,1.3) [pic type = xeoto];
%--------
\draw[blue,very thick] (B)--(B1);
\end{tikzpicture}
}
\shortans{13{,}3}
\loigiai{
\begin{center}
\begin{tikzpicture}[scale=1,font=\footnotesize,line join=round,line cap=round,>=stealth]
\def \a{4.5}
\def \h{4}
\path (0:0) coordinate (A_1)
++(0:\a) coordinate (D_1)
++(-138:0.45*\a) coordinate (C_1)
($(A_1)+(C_1)-(D_1)$) coordinate (B_1)
($(A_1)!0.5!(C_1)$) coordinate (O)
($(O)+(90:\h)$) coordinate (E)
($(A_1)!0.5!(E)$) coordinate (F_1)
($(B_1)!0.5!(E)$) coordinate (F_2)
($(C_1)!0.5!(E)$) coordinate (F_3)
($(D_1)!0.5!(E)$) coordinate (F_4);
\draw [dashed] (B_1)--(A_1)--(D_1) (A_1)--(C_1) (B_1)--(D_1) (E)--(O);
\draw (B_1)--(C_1)--(D_1);
\draw[dashed,->] (E)--(A_1);
\draw[->] (E)--(B_1);
\draw[->] (E)--(C_1);
\draw[->] (E)--(D_1);
\foreach \x/\g in {A_1/-90,B_1/-135,C_1/-45,D_1/45,E/90,O/-90}
\fill (\x) circle (1pt)($(\g:3mm)+(\x)$) node {$\x$};
\foreach \y/\m in {F_1/20,F_2/135,F_3/45,F_4/45}
\draw ($(\m:3mm)+(\y)$) node {$\overrightarrow{\y}$};
\end{tikzpicture}
\end{center}
Gọi $A_1$, $B_1$, $C_1$, $D_1$ lần lượt là các điểm sao cho $\overrightarrow{EA_1}=\overrightarrow{F_1}$, $\overrightarrow{EA_2}=\overrightarrow{F_2}$, $\overrightarrow{EA_3}=\overrightarrow{F_3}$, $\overrightarrow{EA_4}=\overrightarrow{F_4}$.\\
Vì $EA=EB=EC=ED$ và cùng tạo với mặt phẳng $(ABCD)$ một góc bằng $60^\circ$, các lực căng $\overrightarrow{F_1}$, $\overrightarrow{F_2}$, $\overrightarrow{F_3}$, $\overrightarrow{F_4}$ đều có cường độ là $4{,}7$ kN nên $EA_1=EB_1=EC_1=ED_1$ và cùng tạo với mặt phẳng $\left(A_1B_1C_1D_1\right)$ một góc bằng $60^\circ$.\\
Vì $ABCD$ là hình chữ nhật nên $A_1B_1C_1D_1$ cũng là hình chữ nhật.\\
Gọi $O$ là tâm của hình chữ nhật $A_1B_1C_1D_1$.\\
Ta suy ra $EO\perp \left(A_1B_1C_1D_1\right)$.\\
Do đó, $\left(EA_1,\left(A_1B_1C_1D_1\right)\right)=\widehat{EA_1O}=60^\circ$.\\
Ta có $\left\vert\overrightarrow{F_1}\right\vert=\left\vert\overrightarrow{F_2}\right\vert=\left\vert\overrightarrow{F_3}\right\vert=\left\vert\overrightarrow{F_4}\right\vert=4{,}7$ kN nên $EA_1=EB_1=EC_1=ED_1=4{,}7$.\\
Tam giác $EA_1O$ vuông tại $O$ nên $EO=EA_1\cdot\sin\widehat{EA_1O}=2{,}35\sqrt{3}$.\\
Ta có
\begin{align*}
\overrightarrow{F_1}+\overrightarrow{F_2}+\overrightarrow{F_3}+\overrightarrow{F_4}&=\overrightarrow{EA_1}+\overrightarrow{EA_2}+\overrightarrow{EA_3}+\overrightarrow{EA_4}\\
&=\overrightarrow{EO}+\overrightarrow{OA_1}+\overrightarrow{EO}+\overrightarrow{OA_2}+\overrightarrow{EO}+\overrightarrow{OA_3}+\overrightarrow{EO}+\overrightarrow{OA_4}\\
&=4\overrightarrow{EO}+\left(\overrightarrow{OA_1}+\overrightarrow{OC_1}\right)+\left(\overrightarrow{OB_1}+\overrightarrow{OD_1}\right)\\
&=4\overrightarrow{EO}.
\end{align*}
Gọi $\overrightarrow{P}$ là trọng lực của khung sắt có chứa chiếc ô tô. Khi đó ta có 
\begin{align*}
 \overrightarrow{P} = \overrightarrow{F}_1 + \overrightarrow{F}_2 + \overrightarrow{F}_3 + \overrightarrow{F}_4 = 4\overrightarrow{EO}.
\end{align*}
Suy ra trọng lượng của khung sắt có chứa ô tô là $\left|\overrightarrow{P}\right| = 4\left\vert\overrightarrow{EO}\right\vert=4\cdot2{,}35\sqrt{3}=9{,}4\sqrt{3}$ kN.\\
Vì trọng lượng của khung sắt là $3$ kN nên trọng lượng của chiếc xe ô tô là $9{,}4\sqrt{3}-3\approx13{,}3$ kN.
}
\end{ex}
\Closesolutionfile{ans}
 
% \begin{indapan}
% 	{ans/ans\currfilebase}
% \end{indapan}

