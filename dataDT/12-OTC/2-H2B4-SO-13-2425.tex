\begin{name}
 {Biên soạn: Lê Văn Hiếu\\Phản biện: Bùi Lương Phúc}
{Đề ôn tập chương II}
\end{name}

\caulc
\Opensolutionfile{ans}[ans/ans\currfilebase-Phan-I]
\begin{ex}%[2-H2B4-SO-13-2425 (Nguồn Đề 13 - Bài 4)]%[VN-MT-7, Lê Văn Hiếu]%[2H2N1-2]
 Cho hình hộp chữ nhật $ABCD.A'B'C'D'$. Khẳng định nào sau đây đúng?
 \choice
 {$\overrightarrow{AC'}=\overrightarrow{AB}+\overrightarrow{AC}+\overrightarrow{AB}$}
 {\True $\overrightarrow{AC'}=\overrightarrow{AB}+\overrightarrow{AD}+\overrightarrow{AA'}$}
 {$\overrightarrow{AC'}=\overrightarrow{AB'}+\overrightarrow{AC}+\overrightarrow{AD'}$}
 {$\overrightarrow{AC'}=\overrightarrow{AB'}+\overrightarrow{AD'}+\overrightarrow{AA'}$}
 \loigiai{Theo quy tắc hình hộp ta có $\overrightarrow{AC'}=\overrightarrow{AB}+\overrightarrow{AD}+\overrightarrow{AA'}$.}
\end{ex}

\begin{ex}%[2-H2B4-SO-13-2425 (Nguồn Đề 13 - Bài 4)]%[VN-MT-7, Lê Văn Hiếu]%[2H2N1-3]
 Nếu một vật có khối lượng $m$ (kg) thì lực hấp dẫn $\overrightarrow{P}$ của trái đất tác dụng lên vật được xác định theo công thức $\overrightarrow P=m\overrightarrow g$, trong đó $\overrightarrow g$ là vectơ gia tốc rơi tự do có độ lớn $g=9{,}8$ (m/s$^2$). Độ lớn của lực hấp dẫn trái đất tác dụng lên một quả lê có khối lượng $105$ g là
 \choice
 {$102{,}9$ N}
 {$1029$ N}
 {\True $1{,}029$ N}
 {$10{,}29$ N}
 \loigiai{Đổi $105$ g$=0{,}105$ kg.\\
 Độ lớn của lực hấp dẫn của trái đất tác dụng lên quả lê là $\left|\overrightarrow P\right|=m\left|\overrightarrow g\right|=0{,}105\cdot9{,}8=1{,}029$ N.
 }
\end{ex}

\begin{ex}%[2-H2B4-SO-13-2425 (Nguồn Đề 13 - Bài 4)]%[VN-MT-7, Lê Văn Hiếu]%[2H2N2-3]
 Cho biết máy bay $A$ đang bay với vectơ vận tốc $\overrightarrow u=( 300;200;400)$ (đơn vị: km/h). Máy bay $B$ bay ngược hướng và có tốc độ gấp $2$ lần tốc độ của máy bay $A$. Tọa độ vectơ vận tốc $\overrightarrow v$ của máy bay $B$ là
 \choice
 {$\overrightarrow v=(600;400;800)$}
 {$\overrightarrow v=(150;100;200)$}
 {\True $\overrightarrow v=(-600;-400;-800)$}
 {$\overrightarrow v=(-150;-100;-200)$}
 \loigiai{Máy bay $B$ bay ngược hướng và có tốc độ gấp $2$ lần tốc độ của máy bay $A$ nên vectơ vận tốc $\overrightarrow{v}$ ngược hướng với vectơ vận tốc $\overrightarrow{u}$ và $|\overrightarrow{v}|=2\left|\overrightarrow{u}\right|$, do đó $\overrightarrow v=-2\overrightarrow u\Rightarrow\overrightarrow v=(-600;-400;-800)$.
}\end{ex}

\begin{ex}%[2-H2B4-SO-13-2425 (Nguồn Đề 13 - Bài 4)]%[VN-MT-7, Lê Văn Hiếu]%[2H2N2-2]
 Trong không gian $Oxyz$, cho hai điểm $A(-3;2;-1)$, $B(-1;0;5)$. Tọa độ trung điểm $I$ của đoạn thẳng $AB$ là
 \choice
 {$I(-1;1;2)$}
 {$I(2;1;-2)$}
 {$I(-2;-1;2)$}
 {\True $I(-2;1;2)$}
 \loigiai{Tọa độ trung điểm $I$ của đoạn thẳng $AB$ là $I(-2;1;2)$.
}\end{ex}

\begin{ex}%[2-H2B4-SO-13-2425 (Nguồn Đề 13 - Bài 4)]%[VN-MT-7, Lê Văn Hiếu]%[2H2N2-2]
 Trong không gian tọa độ $Oxyz$, biết $\overrightarrow{OM}=2\overrightarrow i-3\overrightarrow j+\overrightarrow k$. Toạ độ của điểm $M$ là
 \choice
 {$(-2;3;-1)$}
 {\True $(2;-3;1)$}
 {$(-3;2;1)$}
 {$(2;1;-3)$}
 \loigiai{Vì $\overrightarrow{OM}=2\overrightarrow i-3\overrightarrow j+\overrightarrow k$ nên $M(2;-3;1)$.
}\end{ex}

\begin{ex}%[2-H2B4-SO-13-2425 (Nguồn Đề 13 - Bài 4)]%[VN-MT-7, Lê Văn Hiếu]%[2H2N2-3]
 Trong không gian với hệ trục tọa độ $Oxyz$, cho hai điểm $A(-2;2;1)$, $B( 0;1;3)$. Toạ độ của vectơ $\overrightarrow{AB}$ là
 \choice
 {\True $\overrightarrow{AB}=(2;-1;2)$}
 {$\overrightarrow{AB}=(-2;3;4)$}
 {$\overrightarrow{AB}=(-2;1;-2)$}
 {$\overrightarrow{AB}=(-2;2;3)$}
 \loigiai{Ta có $\overrightarrow{AB}=(2;-1;2)$.
}\end{ex}

\begin{ex}%[2-H2B4-SO-13-2425 (Nguồn Đề 13 - Bài 4)]%[VN-MT-7, Lê Văn Hiếu]%[2H2N1-3]
 Cho tứ diện đều $ABCD$ có cạnh bằng $2$. Tính $\overrightarrow{AB}\cdot\overrightarrow{CD}$.
 \choice
 {$\overrightarrow{AB}\cdot\overrightarrow{CD}=-4$}
 {$\overrightarrow{AB}\cdot\overrightarrow{CD}=2$}
 {$\overrightarrow{AB}\cdot\overrightarrow{CD}=1$}
 {\True $\overrightarrow{AB}\cdot\overrightarrow{CD}=0$}
 \loigiai{
 Vì $ABCD$ là tứ diện đều nên các tam giác $ABC$ và $ABD$ là các tam giác đều.\\
 Khi đó
 \begin{eqnarray*}
 \overrightarrow{AB}\cdot\overrightarrow{CD}&=&\overrightarrow{AB}\cdot(\overrightarrow{AD}-\overrightarrow{AC})\\
 &=&\overrightarrow{AB}\cdot\overrightarrow{AD}-\overrightarrow{AB}\cdot\overrightarrow{AC}\\
 &=&2\cdot2\cdot\cos60^\circ-2\cdot2\cdot\cos60^\circ=0.
 \end{eqnarray*}
}\end{ex}

\begin{ex}%[2-H2B4-SO-13-2425 (Nguồn Đề 13 - Bài 4)]%[VN-MT-7, Lê Văn Hiếu]%[2H2N2-5]
 Trong không gian với hệ trục tọa độ $Oxyz$, cho hai điểm $M(-5;2;3)$, $I(2;3;1)$. Gọi $N$ là điểm đối xứng với $M$ qua $I$. Tính độ dài đoạn $ON$.
 \choice
 {$ON=6\sqrt2$}
 {$ON=5\sqrt2$}
 {\True $ON=7\sqrt2$}
 {$ON=3\sqrt2$}
 \loigiai{Vì $N$ là điểm đối xứng với $M$ qua $I$ nên $I$ là trung điểm của đoạn $MN$, do đó $N( 9;4;-1)$.\\
 Vậy $ON=\sqrt{9^2+4^2+(-1)^2}=7\sqrt2$.
}\end{ex}

\begin{ex}%[2-H2B4-SO-13-2425 (Nguồn Đề 13 - Bài 4)]%[VN-MT-7, Lê Văn Hiếu]%[2H2N2-5]
 Trong không gian với hệ trục tọa độ $Oxyz$, cho hai vectơ $\overrightarrow a=(1;-2;0)$ và $\overrightarrow b=(-2;3;1)$. Cho các mệnh đề sau.
 \begin{enumEX}{2}
 \item $\overrightarrow a\cdot\overrightarrow b=-8$.
 \item $2\overrightarrow a=(2;-4;1)$.
 \item $\overrightarrow a+\overrightarrow b=(-1;0;-1)$.
 \item $\left|\overrightarrow b\right|=14$.
 \end{enumEX}
 Số mệnh đề đúng là
 \choice
 {\True $1$}
 {$3$}
 {$2$}
 {$4$}
 \loigiai{Ta có
 \begin{enumerate}
 \item $\overrightarrow a\cdot\overrightarrow b=-8$.
 \item $2\overrightarrow a=(2;-4;0)$.
 \item $\overrightarrow a+\overrightarrow b=(-1;1;1)$.
 \item $\left|\overrightarrow b\right|=\sqrt{14}$.
 \end{enumerate}
 Vậy số mệnh đề đúng là $1$.
}\end{ex}

\begin{ex}%[2-H2B4-SO-13-2425 (Nguồn Đề 13 - Bài 4)]%[VN-MT-7, Lê Văn Hiếu]%[2H2N2-5]
 Trong không gian với hệ trục tọa độ $Oxyz$, cho $\overrightarrow a=(1;-2;3)$ và $\overrightarrow b=(2;-1;-1)$. Mệnh đề nào là mệnh đề \textbf{sai}?
 \choice
 {Vectơ $\overrightarrow{u}=(-5;-7;-3)$ cùng vuông góc với vectơ $\overrightarrow a$ và $\overrightarrow b$}
 {Vectơ $\overrightarrow a$ không cùng phương với vectơ $\overrightarrow b$}
 {Vectơ $\overrightarrow a$ không vuông góc với vectơ $\overrightarrow b$}
 {\True $\left|\overrightarrow a\right|=14$}
 \loigiai{
 Ta có $\left[\overrightarrow a,\overrightarrow b\right]=( 5;7;3)$, suy ra vectơ $\overrightarrow{u}=(-5;-7;-3)$ cùng phương với $\left[\overrightarrow a,\overrightarrow b\right]$ nên $\overrightarrow{u}$ vuông góc với hai vectơ $\overrightarrow{a}$ và $\overrightarrow{b}$.\\
 Do $\dfrac12\ne\dfrac{-2}{-1}$ nên vectơ $\overrightarrow a$ không cùng phương với vectơ $\overrightarrow b$.\\
 Do $\overrightarrow a\cdot\overrightarrow b=1\cdot2+(-2)(-1)+3(-1)=1$ nên vectơ $\overrightarrow a$ không vuông góc với vectơ $\overrightarrow b$.\\
 Ta có $\left|\overrightarrow a\right|=\sqrt{1^2+(-2)^2+3^2}=\sqrt{14}$.
}\end{ex}

\begin{ex}%[2-H2B4-SO-13-2425 (Nguồn Đề 13 - Bài 4)]%[VN-MT-7, Lê Văn Hiếu]%[2H2H1-2]
 Cho hình chóp $S.ABCD$ có đáy $ABCD$ là hình bình hành tâm $O$. Trong các mệnh đề sau mệnh đề nào là mệnh đề \textbf{sai}?
 \choice
 {$\overrightarrow{SA}+\overrightarrow{SB}+\overrightarrow{SC}+\overrightarrow{SD}=4\overrightarrow{SO}$}
 {$\overrightarrow{SA}-\overrightarrow{SB}+\overrightarrow{SC}-\overrightarrow{SD}=\overrightarrow0$}
 {\True $\overrightarrow{SA}+\overrightarrow{SB}+\overrightarrow{SC}+\overrightarrow{SD}=\overrightarrow0$}
 {$\overrightarrow{OA}+\overrightarrow{OB}+\overrightarrow{OC}+\overrightarrow{OD}=\overrightarrow0$}
 \loigiai{
 \begin{center}
 \begin{tikzpicture}[scale=1, font=\footnotesize, line join=round, line cap=round, >=stealth]
 \coordinate (A) at (0,0);
 \coordinate (B) at (-2,-1);
 \coordinate (C) at (3,-1);
 \coordinate (D) at ($(A)+(C)-(B)$);
 \coordinate (O) at ($(A)!1/2!(C)$);
 \coordinate (S) at ($(O)+(0,5)$);
 \draw(S)--(B)--(C)--(D)--(S)--(C);
 \draw[dashed](B)--(A)--(D)--(B)(C)--(A)--(S)--(O);
 \foreach \p/\g in {A/150, B/-90, C/-90, D/0, S/90, O/-90}\draw[fill=black] (\p) circle (1pt)node[shift={(\g:.3)},scale=1]{$\p$};
 \end{tikzpicture}
 \end{center}
 \begin{itemize}
 \item Ta có $O$ là trung điểm của $AC$ nên $\overrightarrow{SA}+\overrightarrow{SC}=2\overrightarrow{SO}$.\\
 $O$ là trung điểm của $BD$ nên $\overrightarrow{SB}+\overrightarrow{SD}=2\overrightarrow{SO}$.\\
 Do đó $\overrightarrow{SA}+\overrightarrow{SB}+\overrightarrow{SC}+\overrightarrow{SD}=4\overrightarrow{SO}$ là khẳng định đúng.
 \item $\overrightarrow{SA}-\overrightarrow{SB}+\overrightarrow{SC}-\overrightarrow{SD}=\overrightarrow{BA}+\overrightarrow{DC}=\overrightarrow0$ là khẳng định đúng.
 \item Ta có $\overrightarrow{SA}+\overrightarrow{SB}+\overrightarrow{SC}+\overrightarrow{SD}=4\overrightarrow{SO}$ như chứng minh trên.\\
 Do đó $\overrightarrow{SA}+\overrightarrow{SB}+\overrightarrow{SC}+\overrightarrow{SD}=\overrightarrow0$ là khẳng định sai.
 \item Ta có $O$ là trung điểm của $AC$ nên $\overrightarrow{OA}+\overrightarrow{OC}=\overrightarrow{O}$.\\
 $O$ là trung điểm của $BD$ nên $\overrightarrow{OB}+\overrightarrow{OD}=\overrightarrow0$.\\
 Do đó $\overrightarrow{OA}+\overrightarrow{OB}+\overrightarrow{OC}+\overrightarrow{OD}=\overrightarrow0$ là khẳng định đúng.
 \end{itemize}
 }
\end{ex}

\begin{ex}%[2-H2B4-SO-13-2425 (Nguồn Đề 13 - Bài 4)]%[VN-MT-7, Lê Văn Hiếu]%[2H2V1-4]
 \immini[thm]{
 Một chiếc đèn chùm treo có khối lượng $m=5$ kg được thiết kế với đĩa đèn được giữ bởi bốn đoạn xích $SA$, $SB$, $SC$, $SD$ sao cho $S.ABCD$ là hình chóp tứ giác đều có $\widehat{ASC}=60^\circ$ (Hình bên).\\
 Biết $\overrightarrow{P}=m\overrightarrow g$ trong đó $\overrightarrow g$ là vectơ gia tốc rơi tự do có độ lớn $10$ m/s$^2$, $\overrightarrow{P}$ là trọng lực tác động lên vật có đơn vị là N, $m$ là khối lượng của vật có đơn vị kg. Cho các kết luận dưới đây.
 \begin{enumerate}
 \item $SA$, $SB$ là hai vectơ cùng phương.
 \item $\left|\overrightarrow{SA}\right|=\left|\overrightarrow{SB}\right|=\left|\overrightarrow{SC}\right|=\left|\overrightarrow{SD}\right|$.
 \item Độ lớn của trọng lực $\overrightarrow{P}$ tác động lên chiếc đèn chùm bằng $50$ N.
 \item Độ lớn của lực căng cho mỗi sợi xích bằng $\dfrac{25\sqrt3}6$ N.
 \end{enumerate}
 }{
 \begin{tikzpicture}[scale=.65,>=stealth, font=\footnotesize, line join=round, line cap=round]
 \tikzset{day/.pic=
 {\draw[shade,bottom color=brown!30,top color= white!30,rounded corners=0.5ex,line width=1.5pt,gray!80]
 (0,0) ellipse ({2.5pt} and {8pt});}
 }
 \def\h{6}
 \def\a{3}
 \def\b{1.5}
 \path
 (0,0) coordinate (O)
 ($(O)+(0,\h)$) coordinate (S)
 ($(O)+(10:\a cm and \b cm)$)coordinate (M)
 ($(O)+(180:\a cm and .8*\b cm)$)coordinate (A)
 ($(O)+(0:\a cm and .8*\b cm)$)coordinate (C)
 ($(O)+(60:\a cm and .8*\b cm)$)coordinate (B)
 ($(O)+(-120:\a cm and .8*\b cm)$)coordinate (D)
 ;
 \draw[fill=brown] (M) arc (10:-190:\a cm and \b cm);
 \draw[fill=white] (M) arc (10:-190:\a cm and .8*\b cm);
 \draw [fill=white] (M) arc (10:190:\a cm and .8*\b cm);
 \draw[fill,bottom color=black!30,top color= brown!70, ,left color=black!50] ($(S)-(.5,.3)$) rectangle ($(S)+(.5,.3)$);
 \foreach \m in {0,1,2,...,12}{\pic[rotate=-20] at ($(A)+(.25*\m,.5*\m)$) {day};}
 \foreach \m in {0,1,2,...,14}{\pic[rotate=-15] at ($(D)+(.11*\m,.5*\m)$) {day};}
 \foreach \m in {0,1,2,...,10}{\pic[rotate=5] at ($(B)+(-.15*\m,.5*\m)$) {day};}
 \foreach \m in {0,1,2,...,12}{\pic[rotate=18] at ($(C)+(-.25*\m,.5*\m)$) {day};}
 \foreach \x/\y in {A/180,B/220,C/-45,D/-90,S/90}
 \fill[black] (\x) circle (4pt) ($(\y:7mm)+(\x)$) node {$\x$};
 \end{tikzpicture}
 }
 \noindent
 Số kết luận đúng là
 \choice
 {$1$}
 {\True $2$}
 {$3$}
 {$0$}
 \loigiai{
 \begin{enumerate}
 \item $SA$, $SB$ là hai vectơ cùng phương. \textbf{Sai}
 \item $\left|\overrightarrow{SA}\right|=\left|\overrightarrow{SB}\right|=\left|\overrightarrow{SC}\right|=\left|\overrightarrow{SD}\right|$. \textbf{Đúng}
 \item Độ lớn của trọng lực $\overrightarrow{P}$ tác động lên chiếc đèn chùm là $\left|\overrightarrow{P}\right|=m\cdot\left|\overrightarrow{g}\right|=5\cdot10=50$ N. \textbf{Đúng}
 \item Độ lớn của lực căng cho mỗi sợi xích bằng $\dfrac{25\sqrt3}6$ N. \textbf{Sai}\\
 Ta có $S.ABCD$ là hình chóp tứ giác đều $\Rightarrow SA=SB=SC=SD$.\\
 Mà $\widehat{ASC}=60^\circ\Rightarrow$ tam giác $SAC$ đều.\\
 Gọi $O$ là trung điểm $AC$.\\
 Ta có hợp lực của $4$ lực căng của $4$ sợi xích
 \[\overrightarrow{F}=\overrightarrow{SA}+\overrightarrow{SC}+\overrightarrow{SB}+\overrightarrow{SD}=2\overrightarrow{SO}+2\overrightarrow{SO}=4\overrightarrow{SO}.\]
 Để đèn chùm đứng yên thì hợp lực của các sợi xích phải cân bằng với trọng lực hay $4\overrightarrow{SO}=\overrightarrow{P}$ hay $4SO=50\Leftrightarrow SO=12{,}5$.\\
 Xét tam giác đều $SAC$ có $SA=\dfrac{\sqrt3}2SO=\dfrac{25\sqrt3}4$.\\
 Vậy độ lớn của lực căng cho mỗi sợi xích là $\dfrac{25\sqrt3}4$ N.
 \end{enumerate}
}\end{ex}
\Closesolutionfile{ans}

\cauds
\Opensolutionfile{ans}[ans/ans\currfilebase-Phan-II]
\begin{ex}%[2-H2B4-SO-13-2425 (Nguồn Đề 13 - Bài 4)]%[VN-MT-7, Lê Văn Hiếu]%[2H2H2-4]
 Trong không gian với hệ tọa độ $Oxyz$, cho hai vectơ $\overrightarrow a=(1;2;-2)$ và $\overrightarrow b=(-1;-1;0)$.
 \choiceTF
 {$\left|\overrightarrow a\right|=9$}
 {\True $\overrightarrow a+\overrightarrow b=( 0;1;-2)$}
 {$\overrightarrow a$ và $\overrightarrow b$ cùng phương}
 {\True $\left(\overrightarrow a,\overrightarrow b\right)=135^\circ$}
 \loigiai{
 \begin{itemchoice}
 \itemch \textbf{Sai}.\\
 Ta có $\left|\overrightarrow a\right|=\sqrt{1^2+2^2+(-2)^2}=3$.
 \itemch \textbf{Đúng}.\\
 Ta có $\overrightarrow a+\overrightarrow b=(1-1;2-1;-2+0)\Rightarrow \overrightarrow a+\overrightarrow b=( 0;1;-2)$.
 \itemch \textbf{Sai}.\\
 Ta có $\dfrac1{-1}\ne\dfrac2{-1}$ nên $\overrightarrow a$ và $\overrightarrow b$ không cùng phương.
 \itemch \textbf{Đúng}.\\
 Áp dụng công thức:
 \begin{align*}
 \cos(\overrightarrow a,\overrightarrow b)&=\dfrac{\overrightarrow a\cdot\overrightarrow b}{\left|\overrightarrow a\right|\left|\overrightarrow b\right|}\\
 &=\dfrac{1\cdot(-1)+2\cdot(-1)+(-2)\cdot0}{\sqrt{1^2+2^2+(-2)^2}\cdot\sqrt{(-1)^2+(-1)^2+0^2}}=\dfrac{-3}{3\sqrt2}=-\dfrac1{\sqrt2}.
 \end{align*}
 Suy ra $(\overrightarrow a,\overrightarrow b)=135^\circ$.
 \end{itemchoice}
}\end{ex}

\begin{ex}%[2-H2B4-SO-13-2425 (Nguồn Đề 13 - Bài 4)]%[VN-MT-7, Lê Văn Hiếu]%[2H2H2-4]
 Cho $4$ điểm $A(1;2;0)$, $B(5;1;4)$, $C(7;-2;-2)$, $D(3;m;2)$.
 \choiceTF
 {Độ dài đoạn $AB$ lớn hơn độ dài đoạn $AC$}
 {\True $ m=\dfrac32$ thì $D$ là trung điểm của $AB$}
 {$ m=5$ thì $AB\perp AD$}
 {$ m=-1$ thì $AB\parallel CD$}
 \loigiai{
 \begin{itemchoice}
 \itemch \textbf{Sai}.\\
 Ta có $AB=\sqrt{4^2+(-1)^2+4^2}=\sqrt{33}$ và $AC=\sqrt{6^2+(-4)^2+(-2)^2}=\sqrt{56}$.
 \itemch \textbf{Đúng}.\\
 Tọa độ trung điểm của đoạn $AB$ là $\left(3;\dfrac32;2\right)\Rightarrow m=\dfrac32$.
 \itemch \textbf{Sai}. Với $m=5$, ta có $D(3;5;2)$.\\
 Ta có $AB\perp AD\Leftrightarrow\overrightarrow{AB}\cdot\overrightarrow{AD}=0$.\\
 Vì $\heva{&\overrightarrow{AB}=(4;-1;4)\\
 &\overrightarrow{AD}=(2;3;2)}$ nên $\overrightarrow{AB}\cdot\overrightarrow{AD}=4\cdot2+(-1)\cdot3+4\cdot2=13\ne0$.
 \itemch \textbf{Sai}. Với $m=-1$, ta có $D(3;-1;2)$.\\
 Ta có $\heva{&\overrightarrow{AB}=(4;-1;4)\\
 &\overrightarrow{CD}=(-4;-1;4)} \Rightarrow\overrightarrow{AB}\ne k\overrightarrow{CD}$ (với mọi số thực $k$) $\Rightarrow AB$ không song song với $CD$.
 \end{itemchoice}
}\end{ex}

\begin{ex}%[2-H2B4-SO-13-2425 (Nguồn Đề 13 - Bài 4)]%[VN-MT-7, Lê Văn Hiếu]%[2H2H2-4]
 Trong không gian $Oxyz$, cho các điểm $A( 8;9;2)$, $B( 3;5;1)$ và $C(11;10;4)$.
 \choiceTF
 {Điểm $D$ thỏa mãn $ABCD$ là hình bình hành có tọa độ là $D( 6;6;3)$}
 {\True Độ dài trung tuyến $AM$ bằng $\dfrac{\sqrt{14}}2$}
 {$\widehat{BAC}=30^\circ$}
 {\True Điểm $N$ thuộc mp$( Oxy)$ sao cho ba điểm $A$, $B$, $N$ thẳng hàng có tọa độ là $N(-2;1;0)$}
 \loigiai{
 \begin{itemchoice}
 \itemch \textbf{Sai}. Giả sử $D(x;y;z)$.\\
 Ta có $\overrightarrow{AB}=(-5;-4;-1)$, $\overrightarrow{DC}=(11-x;10-y;4-z)$.\\
 Tứ giác $ABCD$ là hình bình hành $\Leftrightarrow \overrightarrow{AB}=\overrightarrow{DC}\Leftrightarrow\heva{
 &-5=11-x\\
 &-4=10-y\\
 &-1=4-z\\
 }\Leftrightarrow\heva{
 &x=16\\
 &y=14\\
 &z=5.}$\\
 Vậy $D(16;14;5)$.
 \itemch \textbf{Đúng}.\\
 Tọa độ trung điểm $M$ của $BC$ là $M\left(7;\dfrac{15}2;\dfrac52\right)$.\\
 Ta có $\overrightarrow{AM}=\left(-1;-\dfrac32;\dfrac12\right)$. Suy ra $AM=\sqrt{(-1)^2+\left(-\dfrac32\right)^2+\left(\dfrac12\right)^2}=\dfrac{\sqrt{14}}2$.
 \itemch \textbf{Sai}.\\
 Ta có $\overrightarrow{AB}=(-5;-4;-1)$; $\overrightarrow{AC}=( 3;1;2)$.\\
 Do đó
 \begin{align*}
 \cos\widehat{BAC}&=\cos\left(\overrightarrow{AB},\overrightarrow{AC}\right)=\dfrac{\overrightarrow{AB}\cdot\overrightarrow{AC}}{AB\cdot AC}\\
 &=\dfrac{(-5)\cdot3+(-4)\cdot1+(-1)\cdot2}{\sqrt{(-5)^2+(-4)^2+(-1)^2}\cdot\sqrt{3^2+1^2+2^2}}=-\dfrac{\sqrt3}2.
 \end{align*}
 Suy ra $\widehat{BAC}=150^\circ$.
 \itemch \textbf{Đúng}.\\
 Vì $N\in( Oxy)$ nên $N(x;y;0)$. Ta có $\overrightarrow{AB}=(-5;-4;-1)$; $\overrightarrow{AN}=(x-8;y-9;-2)$.\\
 Vì $3$ điểm $A$, $B$, $N$ thẳng hàng nên $\overrightarrow{AB}$ cùng phương với $\overrightarrow{AN}$. Khi đó
 \[\heva{
 &\dfrac{x-8}{-5}=2\\
 &\dfrac{y-9}{-4}=2\\
 }\Leftrightarrow\heva{
 &x-8=-10\\
 &y-9=-8\\
 }\Leftrightarrow\heva{
 &x=-2\\
 &y=1.\\
 }\]
 Vậy $N(-2;1;0)$.
 \end{itemchoice}
}\end{ex}

\begin{ex}%[2-H2B4-SO-13-2425 (Nguồn Đề 13 - Bài 4)]%[VN-MT-7, Lê Văn Hiếu]%[2H2V1-3]
 \immini[thm]{
 Cho hình hộp $ABCD.A'B'C'D'$. Gọi $M$, $N$ là các điểm lần lượt thuộc các đường thẳng $CA$ và $DC'$ sao cho $\overrightarrow{MC}=m\overrightarrow{MA}$, $\overrightarrow{ND}=m\overrightarrow{NC'}$. Đặt $\overrightarrow{BA}=\overrightarrow a$, $\overrightarrow{BB'}=\overrightarrow b$, $\overrightarrow{BC}=\overrightarrow c$.
 \choiceTF
 {$\overrightarrow{BD'}=\overrightarrow a+\overrightarrow b-\overrightarrow c$}
 {\True $\overrightarrow{BM}=\dfrac1{1-m}\overrightarrow c-\dfrac m{1-m}\overrightarrow a$}
 {\True $\overrightarrow{BN}=\dfrac1{1-m}\overrightarrow a-\dfrac m{1-m}\overrightarrow b+\overrightarrow c$}
 {$ m=\dfrac12$ thì $MN\parallel BD'$}
 }{
 \begin{tikzpicture}[scale=0.85,>=stealth, font=\footnotesize, line join=round, line cap=round]
 \coordinate (A) at (0,0);
 \coordinate (B) at (-2,-1);
 \coordinate (C) at ($(B)+(4,0)$);
 \coordinate (D) at ($(A)+(C)-(B)$);
 \coordinate (A') at ($(A)+(0,4)$);
 \coordinate (B') at ($(B)+(A')-(A)$);
 \coordinate (C') at ($(C)+(A')-(A)$);
 \coordinate (D') at ($(D)+(A')-(A)$);
 \coordinate (M) at ($(C)!1/3!(A)$);
 \coordinate (N) at ($(D)!1/3!(C')$);
 \draw(A')--(B')--(C')--(D')--(A')(B)--(B')(C)--(C')(D)--(D')(B)--(C)--(D)--(C');
 \draw[dashed](A)--(A')(A)--(B)(A)--(D)(C)--(A);
 \foreach \p/\g in {A/160, B/-90, C/-90, D/-90, A'/90, B'/90, C'/90, D'/90, M/90, N/90}\draw[fill=black] (\p) circle (1pt)node[shift={(\g:.3)},scale=1]{$\p$};
 \end{tikzpicture}
 }
 \loigiai{
 Dễ thấy $m\ne1$ vì nếu $m=1$, khi đó $\overrightarrow{MC}=\overrightarrow{MA}\Leftrightarrow \overrightarrow{AC}=\overrightarrow{0}\Leftrightarrow A\equiv C$ (vô lý).
 \begin{itemchoice}
 \itemch \textbf{Sai}.\\
 Theo quy tắc hình hộp ta có $\overrightarrow{BD'}=\overrightarrow a+\overrightarrow b+\overrightarrow c$.
 \itemch \textbf{Đúng}.\\
 Ta có
 \allowdisplaybreaks
 \begin{eqnarray*}
 &&\overrightarrow{MC}=m\overrightarrow{MA}\\
 &\Rightarrow&\overrightarrow{BC}-\overrightarrow{BM}=m\overrightarrow{BA}-m\overrightarrow{BM}\\
 &\Rightarrow&(1-m)\overrightarrow{BM}=\overrightarrow{BC}-m\overrightarrow{BA}\\
 &\Rightarrow&\overrightarrow{BM}=\dfrac1{1-m}\overrightarrow{BC}-\dfrac m{1-m}\overrightarrow{BA}=\dfrac1{1-m}\overrightarrow c-\dfrac m{1-m}\overrightarrow a.
 \end{eqnarray*}
 \itemch \textbf{Đúng}.\\
 Tương tự ta có
 \allowdisplaybreaks
 \begin{align*}
 \overrightarrow{BN}&=\dfrac1{1-m}\overrightarrow{BD}-\dfrac m{1-m}\overrightarrow{BC'}\\
 &=\dfrac1{1-m}\overrightarrow a+\dfrac1{1-m}\overrightarrow c-\dfrac m{1-m}(\overrightarrow b+\overrightarrow c)\\
 &=\dfrac1{1-m}\overrightarrow a-\dfrac m{1-m}\overrightarrow b+\overrightarrow c
 \end{align*}
 \itemch \textbf{Sai}.\\
 Ta có
 \allowdisplaybreaks
 \begin{eqnarray*}
 &&\overrightarrow{MN}=\overrightarrow{BN}-\overrightarrow{BM}\\
 &\Rightarrow&\overrightarrow{MN}=\dfrac{1+m}{1-m}\overrightarrow a-\dfrac m{1-m}\overrightarrow b-\dfrac m{1-m}\overrightarrow c.
 \end{eqnarray*}
 Vì $MN\parallel BD'$ nên $\overrightarrow{MN}$ cùng phương $\overrightarrow{BD'}$. Từ đó ta có
 \allowdisplaybreaks
 \begin{eqnarray*}
 &&\overrightarrow{MN}=k\overrightarrow{BD'}\\
 &\Rightarrow&\heva{
 &\dfrac{1+m}{1-m}=k\\
 &-\dfrac m{1-m}=k\\
 &-\dfrac m{1-m}=k\\
 }\\
 &\Rightarrow&m=-\dfrac12.
 \end{eqnarray*}
 \end{itemchoice}
}\end{ex}
\Closesolutionfile{ans}

\caukq
\Opensolutionfile{ans}[ans/ans\currfilebase-Phan-III]
\begin{ex}%[2-H2B4-SO-13-2425 (Nguồn Đề 13 - Bài 4)]%[VN-MT-7, Lê Văn Hiếu]%[2H2H2-3]
 Trong không gian với hệ tọa độ $Oxyz$, cho các điểm $A(1;0;3)$, $B(2;3;-4)$, $C(-3;1;2)$. Gọi $D(x;y;z)$ là điểm sao cho $ABCD$ là hình bình hành. Tính tổng $T=x+y+z$.
 \shortans[]{3}
 \loigiai{
 Ta có $\overrightarrow{AB}=(1;3;-7)$, $\overrightarrow{DC}=(-3-x;1-y;2-z)$.\\
 Tứ giác $ABCD$ là hình bình hành khi
 \[\overrightarrow{AB}=\overrightarrow{DC}\Leftrightarrow \heva{&1=-3-x\\ &3=1-y\\ &-7=2-z} \Leftrightarrow \heva{&x=-4\\ &y=-2\\ &z=9.}\]
 Vậy, $D(-4;-2;9)$.
 Khi đó $T=-4-2+9=3$.
}\end{ex}

\begin{ex}%[2-H2B4-SO-13-2425 (Nguồn Đề 13 - Bài 4)]%[VN-MT-7, Lê Văn Hiếu]%[2H2H2-2]
 Trong không gian với hệ tọa độ $Oxyz$, cho hình vuông $ABCD$ có $B(3;0;8)$, $D(-5;-4;0)$. Tính $\left|\overrightarrow{CA}+\overrightarrow{CB}\right|$ (kết quả làm tròn đến hàng đơn vị).
 \shortans[]{19}
 \loigiai{
 \begin{center}
 \begin{tikzpicture}[scale=1,>=stealth, font=\footnotesize, line join=round, line cap=round]
 \coordinate (A) at (0,0);
 \coordinate (B) at (-4,0);
 \coordinate (C) at (-4,-4);
 \coordinate (D) at ($(C)+(A)-(B)$);
 \coordinate (M) at ($(A)!1/2!(B)$);
 \draw(A)--(B)--(C)--(D)--(A)(B)--(D)(C)--(M);
 \foreach \p/\g in {A/90,B/90,C/180,D/0, M/90}\draw[fill=black] (\p) circle (1pt)node[shift={(\g:.3)},scale=1]{$\p$};
 \end{tikzpicture}
 \end{center}
 Ta có $\overrightarrow{BD}=(-8;-4;-8)$ $\Rightarrow BD=12$ $\Rightarrow AB=\dfrac{12}{\sqrt2}$ $=6\sqrt2$.\\
 Gọi $M$ là trung điểm $AB$ ta có $BM=\dfrac12AB=3\sqrt2$.\\
 Áp dụng định lí Pythagore ta có $MC=\sqrt{BC^2+BM^2}=\sqrt{72+18}=3\sqrt{10}$.\\
 Từ đó $\left|\overrightarrow{CA}+\overrightarrow{CB}\right|$ $=\left| 2\overrightarrow{CM}\right|$ $=2CM$ $=6\sqrt{10}\approx19$.
}\end{ex}

\begin{ex}%[2-H2B4-SO-13-2425 (Nguồn Đề 13 - Bài 4)]%[VN-MT-7, Lê Văn Hiếu]%[2H2H2-4]
 Trong không gian với hệ trục tọa độ $Oxyz$, cho hai vectơ $\overrightarrow a=(1;-2;0)$, $\overrightarrow b=(1;3;-2)$. Tính góc giữa hai vectơ $\overrightarrow a$ và $\overrightarrow b$ (tính theo độ làm tròn đến hàng đơn vị).
 \shortans[]{127}
 \loigiai{Ta có
 \[\cos(\overrightarrow a,\overrightarrow b)=\dfrac{\overrightarrow a\cdot\overrightarrow b}{\left|\overrightarrow a\right|\cdot\left|\overrightarrow b\right|}=\dfrac{1-6}{\sqrt{1^2+(-2)^2+0}\cdot\sqrt{1^2+3^2+(-2)^2}}=\dfrac{-5}{\sqrt5\cdot\sqrt{14}}.\]
 Vậy $(\overrightarrow a,\overrightarrow b)\approx 127^\circ$.
}\end{ex}

\begin{ex}%[2-H2B4-SO-13-2425 (Nguồn Đề 13 - Bài 4)]%[VN-MT-7, Lê Văn Hiếu]%[2H2H2-6]
 \immini[thm]{Trong một phòng học dạng hình hộp chữ nhật, với chiều dài $8$ m, chiều rộng $6$ m và chiều cao $3$ m. Hai bạn An và Bình làm nhiệm vụ trực nhật, mạng nhện cần quét ở góc ngoài cùng trên trần nhà, An bảo không nên đứng ngay vị trí đó ở nền nhà quét vì bụi sẽ rơi xuống người mình. An lại đố Bình ``nếu mình đứng ở giữa nhà quét thì chổi quét nhà dài mấy mét để quét được vị trí mạng nhện, biết đầu cán chổi (vị trí $B$ trên hình vẽ minh họa) cao $1{,}5$ m so với sàn nhà''. Bình trả lời đứng vị trí đó chổi dài $5$ m cũng không tới. Hỏi Bình đã tính được chổi cần dài bao nhiêu mét (làm tròn kết quả đến hàng phần trăm)?
 }{
 \begin{tikzpicture}[scale=.5,>=stealth, font=\footnotesize, line join=round, line cap=round]
 \coordinate (O) at (0,0);
 \coordinate (A') at (-140:3);
 \coordinate (A) at ($(A')+(0,5)$);
 \coordinate (B') at (8,0);
 \coordinate (C') at ($(A')+(8,0)$);
 \coordinate (b) at ($(B')+(0,5)$);
 \coordinate (c) at ($(C')+(0,5)$);
 \coordinate (o') at ($(O)+(0,5)$);
 \coordinate (x) at ($(O)!1.5!(A')$);
 \coordinate (y) at ($(O)!1.25!(B')$);
 \coordinate (z) at ($(O)!1.5!(o')$);
 \coordinate (b') at ($(A')!1/2!(B')$);
 \coordinate (B) at ($(b')+(0,2.5)$);
 \draw[dashed](O)--(A')(O)--(B')(O)--(o');
 \draw[-stealth](A')--(x);
 \draw[-stealth](B')--(y);
 \draw[-stealth](o')--(z);
 \draw[dashed](b')--(B)--(A);
 \draw(A')--(C') node[midway, below]{$8$m}--(B') node[midway, right, shift={(-45:.3)}]{$6$m}--(b)--(c)--(A)--(o')--(b)(C')--(c)(A)--(A');
 \foreach \p/\g in {A/90,x/-90,y/-90,z/0, O/-90, B/0}\draw[fill=black] (\p) circle (1pt)node[shift={(\g:.3)},scale=1]{$\p$};
 \end{tikzpicture}
 }
\shortans[]{5{,}22}
 \loigiai{
 Xét hệ tọa độ $Oxyz$ như hình vẽ, ta có vị trí mạng nhện ở $A(6;0;3)$ vị trí cầm chổi $B\left(3;4;\dfrac32\right)$.\\
 Vậy chổi phải có độ dài $AB=\sqrt{(3-6)^2+(4-0)^2+\left(\dfrac32-3\right)^2}=\dfrac{\sqrt{109}}2\approx 5{,}22$ m.
}\end{ex}

\begin{ex}%[2-H2B4-SO-13-2425 (Nguồn Đề 13 - Bài 4)]%[VN-MT-7, Lê Văn Hiếu]%[2H2V2-6]
 \immini[thm]{
 Với hệ trục tọa độ $Oxyz$ sao cho $O$ nằm trên mặt nước, mặt phẳng $(Oxy)$ là mặt nước, trục $Oz$ hướng lên trên (đơn vị đo: mét), một con chim bói cá đang ở vị trí $C$ cách mặt nước $2$ m, cách mặt phẳng $(Oxz)$, $(Oyz)$ lần lượt là $3$ m và $1$ m phóng thẳng xuống vị trí con cá, biết con cá cách mặt nước $50$ cm, cách mặt phẳng $(Oxz)$, $(Oyz)$ lần lượt là $1$ m và $1{,}5$ m. Gọi $B(a;b;0)$ là điểm lúc chim bói cá vừa tiếp xúc với mặt nước. Tính $T=a+b$.
 }{
 \begin{tikzpicture}[line join=round, line cap=round,scale=1,transform shape, >=stealth]
 \definecolor{columbiablue}{rgb}{0.61, 0.87, 1.0}%màu nước
 \definecolor{arsenic}{rgb}{0.23, 0.27, 0.29}%màu mỏ
 \definecolor{antiquewhite}{rgb}{0.98, 0.92, 0.84}%màu trắng
 \definecolor{cadmiumorange}{rgb}{0.93, 0.53, 0.18}%lông cam
 \definecolor{coolblack}{rgb}{0.0, 0.18, 0.39}%cánh đậm
 \definecolor{brandeisblue}{rgb}{0.0, 0.44, 1.0}%màu xanh đầu
 \definecolor{darkcoral}{rgb}{0.8, 0.36, 0.27}%màu chân
 
 %---------màu vẽ cá
 \definecolor{amber}{rgb}{1.0, 0.49, 0.0}
% \clip (-3,-3.5) rectangle (3.5,3);

 \tikzset{san/.pic={ 
 \path
 (-1.3,-1.5) coordinate (O)
 ($(O)+(-142:2)$) coordinate (y)
 ($(O)+(0:4.7)$) coordinate (x)
 ($(O)+(90:3)$) coordinate (z)
 ($(x)+(y)-(O)$) coordinate (t)
 
 (-.8,-2.2)coordinate (A)
 (1.6,0.8) coordinate (C)
 ($(A)!.13!(C)$) coordinate (B)
 ;
 \fill[columbiablue] (O)--(x)--(t)--(y)--cycle;
 
 \foreach\p/\g/\t in {x/-90/y, y/-90/x, z/0/z}
 {
 \node at (\p) [shift=(\g:2mm)] {\tiny $\t$};
 }
 
 \foreach\p/\g in {A/180,B/0,C/-50,O/-90}
 %\node at (\p) [shift=(\g:2mm)] {\tiny $\p$};
 {
 \draw[fill=black](\p) circle (.5pt) +(\g:2mm)node{\tiny $\p$};
 }
 
 \draw[->] (O)--(x) ;
 \draw[->] (O)--(y);
 \draw[->] (O)--(z);
 \draw[dashed] (A)--(B);
 \draw (B)--(C);
 %---------nước
 \draw (-1,-1.7)
 ..controls +(-120:.5) and +(-160:.5) ..(0,-2)
 (-.9,-1.9)
 ..controls +(-70:.2) and +(-160:.2) ..(-.2,-1.9)
 (-.95,-1.8)
 ..controls +(70:.5) and +(30:.5) ..(0,-1.9)
 (.1,-1.6)
 ..controls +(-20:.2) and +(30:.2) ..(.2,-1.9)
 (-.7,-1.75)
 ..controls +(-170:.2) and +(-160:.3) ..(-.5,-1.9)
 ;
 }}
 
 \path
 (0,0)pic[scale=1]{san}
 ;
 
 \tikzset{chim_boi_ca/.pic={
 %==============cánh trái
 \draw[fill=coolblack] %(-1,1.1)..controls +(90:.3) and +(170:.3) ..
 (-.55,1.4)
 ..controls +(110:.7) and +(100:.3) ..(-.9,1.8)
 ..controls +(135:.3) and +(120:.3) ..(-1.2,1.8)
 ..controls +(145:.25) and +(110:.25) ..(-1.45,1.7)
 ..controls +(165:.15) and +(85:.15) ..(-1.75,1.63)
 ..controls +(-165:.1) and +(85:.1) ..(-1.9,1.5)
 ..controls +(165:.15) and +(85:.1) ..(-2.1,1.3)
 ..controls +(-165:.1) and +(95:.1) ..(-2.2,1.1)
 ..controls +(-160:.1) and +(95:.1) ..(-2.35,1)--(-1,1.1)
 ;
 %======--------------------
 %Tô lông đầu
 \def\L{
 (2,.84)
 ..controls +(170:.2) and +(25:.3) ..(1,.65)
 ..controls +(-145:.5) and +(40:.6) ..(.3,.4)
 ..controls +(-140:.3) and +(-60:.3) ..(0,.6)
 ..controls +(120:.3) and +(-50:.7) ..(-1,1.1)
 ..controls +(90:.3) and +(170:.3) ..(-.55,1.4)%1
 ..controls +(-40:.2) and +(140:.1) ..(-.25,1.2)
 ..controls +(-70:.2) and +(-160:.35) ..(.15,.85)
 ..controls +(75:.7) and +(135:.8) ..(1.9,1.3)--(2.1,1)--cycle
 ;
 }
 %\draw[red]\L;
 \fill[brandeisblue] \L;
 %==============================
 \draw[fill=antiquewhite] (2,1.05)
 ..controls +(165:.2) and +(-35:.2)..(1.7,1.15)
 ..controls +(145:.2) and +(65:.2)..(1.2,1.15)
 ..controls +(-60:.1) and +(165:.1)..(1.4,1)
 ..controls +(-15:.2) and +(-145:.3)..cycle
 ;
 \draw[fill=arsenic] (1.6,1.2)
 ..controls +(155:.18) and +(55:.15)..(1.23,1.17)
 ..controls +(-75:.2) and +(-95:.2)..cycle
 ;
 \fill (1.44,1.14) circle(1mm);
 
 \fill[cadmiumorange] (2,1.05)
 ..controls +(165:.2) and +(-35:.2)..(1.7,1.15)
 ..controls +(145:.2) and +(95:.3)..cycle
 ;
 \fill[cadmiumorange] (2,.84)
 ..controls +(150:.2) and +(-25:.1) ..(1.7,1)
 ..controls +(155:.2) and +(-50:.3) ..(1.2,1.08)
 ..controls +(130:.2) and +(50:.2) ..(.75,1)
 ..controls +(-130:.2) and +(-10:.2) ..(.21,1)
 ..controls +(-120:.1) and +(70:.1) ..(.15,.84)
 ..controls +(-20:.3) and +(-150:.3) ..(.8,.85)
 ..controls +(-10:.3) and +(150:.5) ..(1.7,.85)
 ..controls +(-10:.1) and +(150:.1) ..cycle
 ;
 %==================
 %viền đen đầu
 \def\X{
 (0,.6)
 ..controls +(120:.3) and +(-50:.7) ..(-1,1.1)
 
 (-.55,1.4)%1
 ..controls +(-40:.2) and +(140:.1) ..(-.25,1.2)
 ..controls +(-70:.2) and +(-160:.35) ..(.15,.85)
 ..controls +(75:.7) and +(135:.8) ..(1.9,1.3)
 ;
 }
 \draw[black]\X;
 %====================
 %Tô mỏ
 \def\N{
 (1.9,1.3)
 ..controls +(-35:.4) and +(140:.3) ..(3.4,.78)
 ..controls +(-175:.2) and +(-10:.3) ..(2,.84)
 ..controls +(150:.2) and +(-25:.1) ..(1.7,1)
 ..controls +(20:.2) and +(175:.1) ..(2,1.05)
 ..controls +(-35:.2) and +(155:.1) ..cycle
 ;
 }
 %\draw[red]\N;
 \fill[arsenic] \N;
 %Mỏ
 \def\M{
 (1.9,1.3)
 ..controls +(-35:.4) and +(140:.3) ..(3.4,.78)
 ..controls +(-175:.2) and +(-10:.3) ..(2,.84)
 ..controls +(170:.2) and +(25:.3) ..(1,.65)
 ;
 }
 \draw[black]\M;
 %==============================
 
 %================Cánh phải
 \draw[fill=coolblack]
 %(-2.6,.95)
 %..controls +(-160:.2) and +(145:.3) ..(-1.6,.5)
 %..controls +(-35:.2) and +(145:.3) ..(-1,-.5)
 %..controls +(-35:.2) and +(-145:.2) ..
 (-.6,-.3)%3
 ..controls +(-130:.5) and +(-10:.2) ..(-1,-.95)
 ..controls +(170:.1) and +(-10:.15) ..(-1.2,-1.02)
 ..controls +(170:.1) and +(-40:.15) ..(-1.45,-.95)
 ..controls +(160:.1) and +(-80:.15) ..(-1.6,-.8)
 ..controls +(140:.1) and +(-70:.15) ..(-1.75,-.65)
 ..controls +(140:.1) and +(-70:.15) ..(-1.9,-.5)
 ..controls +(160:.1) and +(-70:.15) ..(-2.05,-.35)
 ..controls +(150:.1) and +(-70:.15) ..(-2.25,-.2)
 ..controls +(-160:.1) and +(-20:.15) ..(-2.5,-.18)
 ..controls +(160:.1) and +(-30:.15) ..(-2.75,-.16)
 ..controls +(160:.1) and +(-60:.15) ..(-2.95,-.05)
 ..controls +(160:.1) and +(-80:.15) ..(-3.2,.05)
 ..controls +(160:.1) and +(-90:.15) ..(-3.35,.15)
 ..controls +(160:.1) and +(-95:.15) ..(-3.55,.25)
 ..controls +(-160:.2) and +(-145:.2) ..(-3.7,.35)
 ..controls +(-160:.2) and +(-160:.4) ..(-3.7,.53)
 ..controls +(-160:.2) and +(-160:.3) ..(-3.85,.65)
 ..controls +(160:.5) and +(-170:.3) ..(-2.6,.95)
 ..controls +(0:.5) and +(120:.3) ..(-.6,-.3)%3
 ;
 %===================
 %===================
 \fill[brandeisblue]
 (-.8,-1.1)%lông đuôi xanh
 ..controls +(-80:.2) and +(140:.2) ..(-.5,-1.5)
 ..controls +(-40:.2) and +(140:.4) ..(-.3,-2.5)
 ..controls +(135:.7) and +(-85:.6) ..cycle
 ;
 \draw (-.3,-2.5)
 ..controls +(135:.7) and +(-85:.6) ..(-.8,-1.1)%lông đuôi xanh
 ;
 %=============đuôi
 \draw[fill=brandeisblue] (-.45,-2.3)
 ..controls +(-95:.2) and +(95:.2) ..(-.4,-2.8)
 --(-.32,-2.8)--(-.25,-2.2)
 (-.25,-2.2)--(-.32,-2.78)--(-.27,-2.78)--(-.16,-2.2)
 ;
 %================
 %Tô lông cam
 \def\C{
 (2,.84)
 ..controls +(170:.2) and +(25:.3) ..(1,.65)
 ..controls +(-145:.5) and +(40:.6) ..(.3,.4)
 ..controls +(-140:.3) and +(-60:.3) ..(0,.6)
 ..controls +(120:.3) and +(-50:.7) ..(-1,1.1)%2
 ..controls +(130:.2) and +(20:.3) ..(-2.6,.95)
 ..controls +(-160:.2) and +(145:.3) ..(-1.6,.5)
 ..controls +(-35:.2) and +(145:.3) ..(-1,-.5)
 ..controls +(-35:.2) and +(-145:.2) ..(-.6,-.3)%3
 ..controls +(-135:.2) and +(100:.2) ..(-.8,-1.1)%lông đuôi xanh
 ..controls +(-80:.2) and +(140:.2) ..(-.5,-1.5)
 ..controls +(-40:.2) and +(140:.4) ..(-.3,-2.5)%đuôi dưới
 ..controls +(40:.4) and +(-150:.5) ..(.5,-1.1)%chân
 ..controls +(-20:.2) and +(160:.2) ..(.8,-1.15)
 ..controls +(150:.1) and +(-70:.1) ..(.65,-.9)
 ..controls +(110:.2) and +(-95:.8) ..(1.26,.6)
 ..controls +(75:.1) and +(-160:.2) ..cycle
 ;
 }
 
 \fill[cadmiumorange] \C;
 \draw[black]\C;

 \draw[black] (-1,1.1)
 ..controls +(130:.2) and +(20:.3) ..(-2.6,.95)
 
 (-.6,-.3)%3
 ..controls +(-135:.2) and +(100:.2) ..(-.8,-1.1)
 ;
 
 %======================chân
 \draw[fill=darkcoral]
 (.5,-1.1)
 ..controls +(-20:.1) and +(170:.1) ..(1.5,-1.1)%chân
 ..controls +(-10:.1) and +(-10:.3) ..(1.2,-1.2)
 ;
 %móng 2
 \draw (1.47,-1.15)%chân
 ..controls +(-10:.1) and +(120:.1) ..(1.63,-1.25)
 ;
 %---------------------
 \draw[fill=darkcoral]
 (.7,-1.15)
 ..controls +(40:.1) and +(170:.1) ..(1.3,-1.08)%chân
 ..controls +(-10:.1) and +(-10:.1) ..(1.2,-1.2)
 ..controls +(170:.1) and +(30:.1) ..(1,-1.2)
 ;
 %móng 2
 \draw (1.27,-1.15)%chân
 ..controls +(-10:.1) and +(120:.1) ..(1.43,-1.25)
 ;
 %------------------------
 \draw[fill=darkcoral]
 (.25,-1.1)
 ..controls +(-20:.2) and +(-170:.1) ..(.5,-1.1)%chân
 ..controls +(-20:.2) and +(160:.2) ..(.8,-1.15)
 ..controls +(-20:.2) and +(160:.2) ..(1.1,-1.2)%móng 1
 ..controls +(-20:.1) and +(-50:.1) ..(1.03,-1.25)
 ..controls +(130:.05) and +(-10:.05) ..(.8,-1.26)
 ..controls +(170:.05) and +(10:.05) ..(.5,-1.28)
 ..controls +(-150:.1) and +(-160:.15) ..(.3,-1.25)
 ..controls +(160:.1) and +(-80:.1) ..(.1,-1.15)
 ;
 %móng 1
 \draw (1.1,-1.25)%móng 1
 ..controls +(-20:.1) and +(95:.1) ..(1.2,-1.35)
 ;
 }}
 %===========Vẽ cá
 \tikzset{ca/.pic={
 %vây
 \def\V{
 (-.35,.74)
 ..controls +(120:.12) and +(40:.22) ..(-.7,.72)--cycle
 (-.7,.32)
 ..controls +(-170:.1) and +(10:.1) ..(-.95,.3)
 ..controls +(60:.1) and +(-140:.1) ..(-.85,.45)--(-.65,.4)--cycle
 (-.3,.32)
 ..controls +(-170:.1) and +(10:.1) ..(-.45,.1)
 ..controls +(-40:.1) and +(-110:.1) ..(-.1,.37)--cycle
 ;
 }
 \fill[amber] \V;
 \draw\V;
 
 %-----------------
 \def\C{
 (-1.25,.83)
 ..controls +(-45:.2) and +(130:.2) ..(-1,.58)
 ..controls +(35:.2) and +(130:.52) ..(.05,.52)
 ..controls +(-90:.1) and +(-110:.1) ..(.05,.52)--(.04,.44)
 ..controls +(-150:.5) and +(-40:.2) ..(-1,.53)
 ..controls +(-140:.1) and +(40:.1) ..(-1.3,.42)
 ..controls +(60:.2) and +(-55:.2) ..cycle
 ;
 }
 
 \fill[amber] \C;
 \draw\C;
 %-----------------
 \def\Cn{
 (-1,.57)
 ..controls +(35:.1) and +(130:.4) ..(.01,.52)
 ..controls +(-140:.4) and +(-40:.2) ..cycle
 ;
 }
 %\draw[ecru!70!black]\Cn;
 \fill[amber!70] \Cn;
 \draw (-.3,.7)
 ..controls +(-120:.1) and +(120:.2) ..(-.3,.34);
 
 \draw[fill=white] (-.22,.55) circle (.08);
 \draw[fill=black] (-.22,.55) circle (.048);
 
 }}

 \path (-1,-2.5)pic[xscale=-.35,yscale=.35]{ca}
 (1.6,1)pic[xscale=-.15,yscale=.15,rotate=-40]{chim_boi_ca}; 
 \end{tikzpicture}
 }
\shortans[]{2{,}8}
 \loigiai{Ta có $A(1{,}5;1;-0{,}5)$ và $C(1;3;2)$ suy ra $\overrightarrow{AC}(-0{,}5;2;2{,}5)$ và $\overrightarrow{AB}=(a-1{,}5;b-1;0{,}5)$.\\
 Vì $A$, $B$, $C$ thẳng hàng nên ta có $\overrightarrow{AB}=k\overrightarrow{AC}$. Suy ra
 \[\heva{&a-1{,}5=k(-0{,}5)\\
 &b-1=2k\\
 &0{,}5=2{,}5k}
 \Leftrightarrow\heva{&k=\dfrac15\\
 &a=1{,}5-\dfrac{0{,}5}{5}=\dfrac75\\
 &b=1+\dfrac{2}{5}=\dfrac75.}
 \]
 Suy ra $B\left(\dfrac75;\dfrac75;0\right)$.\\
 Vậy $T=\dfrac75+\dfrac75=2{,}8$.
}\end{ex}

\begin{ex}%[2-H2B4-SO-13-2425 (Nguồn Đề 13 - Bài 4)]%[VN-MT-7, Lê Văn Hiếu]%[2H2H2-6]
 Một căn phòng dạng hình hộp chữ nhật với chiều dài $8$ m, rộng $6$ m và cao $4$ m có hai chiếc quạt treo tường. Chiếc quạt $A$ treo chính giữa bức tường $8$ m và cách trần $1$ m, chiếc quạt $B$ treo chính giữa bức tường $6$ m và cách trần $1{,}5$ m. (Tham khảo hình vẽ minh họa).
 \begin{center}
 \begin{tikzpicture}[line join=round, line cap=round,scale=.6,transform shape,>=stealth]
 \definecolor{amber}{rgb}{1.0, 0.75, 0.0}%mau non
 \definecolor{antiquebrass}{rgb}{0.8, 0.58, 0.46}%mau da
 \definecolor{antiquewhite}{rgb}{0.98, 0.92, 0.84}%mau ao
 \definecolor{cadmiumgreen}{rgb}{0.0, 0.42, 0.24}%mau quan
 \definecolor{cadetblue}{rgb}{0.37, 0.62, 0.63}%mau but
 \definecolor{brown(traditional)}{rgb}{0.59, 0.29, 0.0}%mau giay
 \definecolor{brilliantlavender}{rgb}{0.96, 0.73, 1.0}%màu sơn tím
 \definecolor{brightube}{rgb}{0.82, 0.62, 0.91}%màu sơn tím đậm
 %---------------màu quạt
 \definecolor{burntorange}{rgb}{0.8, 0.33, 0.0}
 \definecolor{arsenic}{rgb}{0.23, 0.27, 0.29}
 \definecolor{battleshipgrey}{rgb}{0.52, 0.52, 0.51}
 \clip (1,-1) rectangle (16,12);
 %\draw[gray!50] (-3,-3) grid (3,4);
 
 \definecolor{burntsienna}{rgb}{0.91, 0.45, 0.32}
 \tikzset{mai/.pic={
 \def\mainha{
 (.5,2)
 foreach \n in {1,2,...,22} { -- ++ (0,0) -- ++ (0,1) -- ++ (1,0) -- ++ (0,-1) } -- cycle
 
 (.5,1)
 foreach \n in {1,2,...,22} { -- ++ (0,0) -- ++ (0,1) -- ++ (1,0) -- ++ (0,-1) } -- cycle
 
 (.5,0)
 foreach \n in {1,2,...,22} { -- ++ (0,0) -- ++ (0,1) -- ++ (1,0) -- ++ (0,-1) } -- cycle
 
 (.5,-1)
 foreach \n in {1,2,...,22} { -- ++ (0,0) -- ++ (0,1) -- ++ (1,0) -- ++ (0,-1) } -- cycle
 
 (.5,-2)
 foreach \n in {1,2,...,22} { -- ++ (0,0) -- ++ (0,1) -- ++ (1,0) -- ++ (0,-1) } -- cycle
 
 (.5,-3)
 foreach \n in {1,2,...,22} { -- ++ (0,0) -- ++ (0,1) -- ++ (1,0) -- ++ (0,-1) } -- cycle
 
 (.5,-4)
 foreach \n in {1,2,...,22} { -- ++ (0,0) -- ++ (0,1) -- ++ (1,0) -- ++ (0,-1) } -- cycle
 ;
 }
 \clip (0.5,-3)--(17.5,-3)--(17.5,3)--(.5,3)--cycle;
 \draw[white,fill=burntsienna!70] \mainha;
 \draw (17.5,-3)--(17.5,3)--(17.5,9);
 }}
 
 \fill[brilliantlavender] (16,-1)--(16,9)--(12,12)--(12,3)--cycle;
 \fill[brightube] (12,13)--(12,3)--(-18,3)--(-18,13)--cycle;
 \begin{scope}
 \clip[draw] (1,3)--(12,3)--(16,-1)--(5,-1)--cycle;
 \path
 (-2.5,0)pic[scale=1,xslant=-1]{mai}%
 ;
 \end{scope}
 
 \path
 (12,3) coordinate (O)
 (3,3) coordinate (x)
 (12,12) coordinate (z)
 (15,0) coordinate (y)
 ;
 
 \foreach\p/\g in {y/70,x/80, z/-30,O/-120}
 {
 \node at (\p) [shift=(\g:3mm)] {$\p$};
 }
 
 \draw[line width=.5mm,->] (O)--(z) ;
 \draw[line width=.5mm,->] (O)--(x);
 \draw[line width=.5mm,->] (O)--(y); 
 \draw[red,line width=.5mm,<->] ($(O)+(0,4)$)--($(x)+(-2,4)$) node[midway, above]{$8$ m};
 
 %===========================A FAN
 \tikzset{fan/.pic={
 %chân quạt
 \draw[fill=battleshipgrey](-.2,.8)--(.2,.8)--(.4,-2.2)
 ..controls +(-120:.3) and +(-60:.3) ..(-.4,-2.2)--cycle;
 %---Nút bấm
 \foreach \i in{-1.95,-1.7}{%-1.45
 \draw[fill=arsenic](-.15,\i) rectangle (.15,\i+.15);
 }
 %-----------------------------------------------------
 \draw[black](0,.8) circle (2.25cm);
 \draw[black](0,.8) circle (2.15cm);
 
 \draw[black](0,.8) circle (1.42cm);
 \draw[black](0,.8) circle (1.48cm);
 
 \draw[fill=black](0,.8) circle (6mm);
 \draw[fill=arsenic](0,.8) circle (5mm);
 \def\N{
 (0,.8)
 ..controls +(145:1.3) and +(170:1) ..(0,2.8)
 ..controls +(-10:1.4) and +(-20:1) ..(.6,1.76)
 ..controls +(160:.4) and +(100:.4) ..(.2,.9)--cycle
 ;
 }
 \foreach \i/\j/\k in {0/0/0,120/.7/-1.2,240/-.7/-1.2}
 {
 \draw[black,rotate=\i,shift={(\j,\k)}]\N;
 \fill[burntorange,rotate=\i,shift={(\j,\k)}] \N;
 }
 
 %lồng quạt
 \def\r{2.15}
 \foreach \i in {0,15,25,35,...,365}
 \draw[double] ($(\i:\r)+(0,.8)$)--(0,.8);
 
 \draw[fill=arsenic](0,.8) circle (3.5mm);

 }}
 \path
 (6.5,9.5)pic[scale=.6]{fan}
 (14,7.5)pic[scale=.55,yslant=-.3]{fan};
 \end{tikzpicture}
 \end{center}
 Hỏi khoảng cách giữa hai chiếc quạt $A$, $B$ cách nhau bao nhiêu mét (làm tròn đến hàng phần trăm).
 
 \shortans[]{5{,}02}
 \loigiai{
 Chọn hệ trục tọa độ như hình vẽ, khi đó ta có tọa độ quạt $A$ là $A(4;0;3)$ và tọa độ quạt $B$ là $B\left(0;3;\dfrac{5}{2}\right)$.\\
 Khi đó $\overrightarrow{AB}=\left(-4;3;-\dfrac{1}{2}\right)$.\\
 Vậy khoảng cách giữa hai quạt $A$, $B$ là $AB= \sqrt{(-4)^2+3^2+\left(-\dfrac{1}{2}\right)^2} \approx 5{,}02$.}
\end{ex}
\Closesolutionfile{ans}
 
\begin{indapan}
	{ans/ans\currfilebase}
\end{indapan}

