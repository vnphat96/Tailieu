
\begin{name}
	{\tenchude}
	{\tendethi}
	{\tentruong}
	{\thoigian}
\end{name}
\Opensolutionfile{ans}[ans/ans-de4-7]

\begin{ex}%Câu 49.
Số phức liên hợp của số phức $z=2+3 i$ là
\choice
{\True $z=2-3 i$}
{$z=-2-3 i$}
{$z=-2+3 i$}
{$z=2+3 i$}

\end{ex}
\begin{ex}%Câu 1.
Cho hình trụ có diện tích toàn phần bằng $8\pi a^2$ và chiều cao bằng $3 a$. Thể tích khối trụ đã cho là
\choice
{$\pi a^3$}
{\True $3\pi a^3$}
{$8\pi a^3$}
{$6\pi a^3$}

\end{ex}

\begin{ex}%Câu 2. 
$\displaystyle\int 4 x^3 \mathrm{\,d} x$ bằng
\choice
{$4 x^4+C$}
{\True $x^4+C$}
{$\dfrac{1}{4} x^4+C$}
{$12 x^2+C$}

\end{ex}

\begin{ex}%Câu 3.
Với $a$ là số thực dương tùy ý, $\log_3(3 a)$ bằng
\choice
{$3-\log_3 a$}
{$3+\log_3 a$}
{\True $1+\log_3 a$}
{$1-\log_3 a$}

\end{ex}
\begin{ex}%Câu 4.
Bán kính mặt cầu tâm $I(1; 3; 5)$ tiếp xúc với đường thẳng\\ $d\colon\heva{&x=t\\&y=-1-t\\&z=2-t}$
\choice
{\True $\sqrt{14}$}
{$7$}
{$14$}
{$\sqrt{7}$}
\end{ex}
\begin{ex}%Câu 5.
\immini{
Hàm số $y=f(x)$ liên tục trên $\mathbb{R}$ và có bảng biến thiên như hình bên. Biết $f(-4)>f(8)$, khi đó giá trị nhỏ nhất của hàm số đã cho trên $\mathbb{R}$ bằng}
{\begin{tikzpicture}[color=\mauchinh]
\tkzTabInit[nocadre,lgt=1.2,espcl=1.5,deltacl=0.5,lw=0.8]
{$x$/0.6,$f'(x)$/0.6,$f(x)$/2}{$-\infty$,$-4$,$0$,$8$,$+\infty$}
\tkzTabLine{,-,0,+,0,-,0,+,}
\tkzTabVar{+/$+\infty$,-/$f(-4)$,+/$9$,-/$f(8)$,+/$+\infty$}
\end{tikzpicture}
}
\choice
{\True $f(8)$}
{$9$}
{$-4$}
{$f(-4)$}

\end{ex}
\begin{ex}%Câu 6.
Tìm tập xác định $\mathscr{D}$ của hàm số $y=(2 x-3)^{\sqrt{2020}}$ 
\choice
{$\mathscr{D}=\left(\dfrac{3}{2};+\infty\right)$}
{$\mathscr{D}=(0;+\infty)$}
{$\mathscr{D}=\mathbb{R} \backslash\left\{\dfrac{3}{2}\right\}$}
{\True $D=\mathbb{R}$}

\end{ex}
\begin{ex}%Câu 7.
Đạo hàm của hàm số $y=\log x$ là
\choice
{$\dfrac{1}{10\ln x}$}
{$\dfrac{\ln 10}{x}$}
{$\dfrac{1}{x}$}
{\True $\dfrac{1}{x \ln 10}$}

\end{ex}
\begin{ex}%Câu 8.
Thể tích khối lăng trụ có diện tích đáy bằng $3 a^2$ và chiều cao bằng $2 a$ là:
\choice
{$a^3$}
{$2 a^3$}
{\True $6 a^3$}
{$3 a^3$}

\end{ex}
\begin{ex}%Câu 9.
Cho hàm số $y=x^4-1$ có đồ thị là $(C)$. Tiếp tuyến của đồ thị $C.$ tại điểm với hoành độ bằng $0$ có hệ số góc là:
\choice
{$4$}
{\True $0$}
{$-1$}
{$1$}

\end{ex}
\begin{ex}%Câu 10.
Cho $\log_6 45=a+\dfrac{\log_2 5+b}{\log_2 3+c}$ với $a, b, c$ là các số nguyên. Giá trị $a+b+c$ bằng:
\choice
{$3$}
{\True $1$}
{$0$}
{$2$} 
\end{ex}
\begin{ex}%Câu 11.
\immini{
Đường cong trong hình bên là đồ thị của hàm số nào trong bốn hàm số dưới đây?
\choice
{$y=x^3-3 x+2$}
{$y=x^3-3 x^2+2$}
{$y=-x^3+3 x+2$}
{\True $y=-x^3+3 x^2-2$}}
{
\begin{tikzpicture}[scale=1, font=\footnotesize, line join=round, line cap=round, >=stealth,y=0.8cm,color=\mauchinh]
\def\xmin{-1.02}\def\xmax{3.02}\def\ymin{-2.17}\def\ymax{2.19}
\draw[->,thick] (\xmin-0.2,0)--(\xmax+0.2,0) node[below] {\footnotesize $x$};
\draw[->,thick] (0,\ymin-0.2)--(0,\ymax+0.2) node[right] {\footnotesize $y$};
\draw (0,0) node [below left] {\footnotesize $O$};
\foreach \x in {}\draw (\x,0.1)--(\x,-0.1) node [below] {\footnotesize $\x$};
\foreach \y in {}\draw (0.1,\y)--(-0.1,\y) node [left] {\footnotesize $\y$};
\clip (\xmin,\ymin) rectangle (\xmax,\ymax);
\draw[thick,smooth,samples=200,domain=\xmin:\xmax] plot (\x,{-1*((\x)^3)+3*((\x)^2)+0*(\x)+-2});
\end{tikzpicture}
}

\end{ex}
\begin{ex}%Câu 12.
Phương trình đường tiệm cận đứng của đồ thị hàm số $y=\dfrac{2 x-1}{x-2}$ là
\choice
{$y=\dfrac{1}{2}$}
{$x=\dfrac{1}{2}$}
{\True $x=2$}
{$y=2$}

\end{ex}
\begin{ex}%Câu 13.
Cho hình trụ có đường cao bằng $4$ nội tiếp trong mặt cầu có bán kính bằng $4$. Tính tỉ số $\dfrac{V_1}{V_2}$, trong đó $V_1, V_2$ lần lượt là thể tích của khối trụ và khối cầu đã cho.
\choice
{$\dfrac{3}{16}$}
{\True $\dfrac{9}{16}$}
{$\dfrac{7}{16}$}
{$\dfrac{5}{16}$}

\end{ex}
\begin{ex}%Câu 14.
Tập nghiệm của bất phương trình $\log_2(4 x+8)-\log_2 x \leq 3$ là
\choice
{\True $[2;+\infty)$}
{$(-\infty; 2]$}
{$[-3;+\infty)$}
{$[1;+\infty)$}

\end{ex}
\begin{ex}%Câu 15.
Cho hàm số $f(x)$ có đạo hàm $f'(x)=\left(x^2-9\right)\left(x^2-3 x\right)^2, \forall x \in \mathbb{R}$. Gọi $T$ là giá trị cực đại của hàm số đã cho. Chọn khẳng định đúng.
\choice
{$T=f(3)$}
{$T=f(0)$}
{$T=f(9)$}
{\True $T=f(-3)$}

\end{ex}
\begin{ex}%Câu 16.
Tập nghiệm của phương trình $\log_2 x+\log_4 x+\log_{16} x=7$ là
\choice
{$\{4\}$}
{$\{2\sqrt{2}\}$}
{\True $\{16\}$}
{$\{\sqrt{2}\}$}

\end{ex}
\begin{ex}%Câu 17.
Cho số phức $z=a+b i(a, b \in \mathbb{R})$ thoả mãn $z-2+|z|=-4 i$. Tính $S=a+b$.
\choice
{\True $S=-7$}
{$S=7$}
{$S=-1$}
{$S=1$}

\end{ex}
\begin{ex}%Câu 18.
Cho hàm số $y=f(x)$ liên tục trên $\mathbb{R}$ và có đồ thị như hình vẽ sau. Giá trị của $\displaystyle\displaystyle\int\limits_{-4}^4 f(x)\mathrm{\,d}x$ bằng
\choice
{$10$}
{$4$}
{$12$}
{\True $8$}

\end{ex}
\begin{ex}%Câu 19.
Cho hình chóp $S.ABCD$ có đáy là hình vuông cạnh $a$. Cạnh bên $SA$ vuông góc với đáy và có độ dài bằng $2 a$, thể tích khối chóp đã cho bằng
\choice
{$\dfrac{a^3}{6}$}
{$\dfrac{a^3}{4}$}
{\True $\dfrac{2 a^3}{3}$}
{$\dfrac{a^3}{3}$}

\end{ex}
\begin{ex}%Câu 20.
Trong không gian $O x y z$, cho $\vec{a}=(-3; 4; 0)$ và $\vec{b}=(5; 0; 12)$. Côsin của góc giữa $\vec{a}$ và $\vec{b}$ bằng
\choice
{$\dfrac{3}{13}$}
{$\dfrac{5}{6}$}
{$\dfrac{-5}{6}$}
{\True $\dfrac{-3}{13}$}

\end{ex}
\begin{ex}%Câu 21.
Cho số phức $z$ thỏa mãn $z(2-i)+12 i=1$. Tính môđun của số phức $z$.
\choice
{$|z|=29$}
{\True $|z|=\sqrt{29}$}
{$|z|=\dfrac{\sqrt{29}}{3}$}
{$|z|=\dfrac{5\sqrt{29}}{3}$}

\end{ex}
\begin{ex}%Câu 22.
Họ nguyên hàm của hàm số $f(x)=2 x(\sin x+1)$ là
\choice
{$x^2+2 x \cos x-2\sin x+C$}
{$x^2-2 x \cos x-2\sin x+C$}
{$x^2(x-\cos x)+C$}
{\True $x^2-2 x \cos x+2\sin x+C$}

\end{ex}
\begin{ex}%Câu 23.
Hàm số $y=x^4+4 x^2+1$ có bao nhiêu điểm cực trị
\choice
{$2$}
{$3$}
{$0$}
{\True $1$}

\end{ex}
\begin{ex}%Câu 24.
Cho $\displaystyle\displaystyle\int\limits_0^1 \dfrac{x^2+2 x}{(x+1)^3} \mathrm{\,d} x=a+b \ln 2$ với $a, b$ là các số hữu tỷ. Giá trị của $16 a+b$ là
\choice
{$-8$}
{$10$}
{$17$}
{\True $-5$}

\end{ex}
\begin{ex}%Câu 25.
Trong KG $Oxyz$, cho $A(-1; 0; 2)$ và $B(2; 1;-5)$. PTĐT $AB$ là
\choice
{$\dfrac{x-1}{1}=\dfrac{y}{1}=\dfrac{z+2}{-3}$}
{\True $\dfrac{x+1}{3}=\dfrac{y}{1}=\dfrac{z-2}{-7}$}
{$\dfrac{x+1}{1}=\dfrac{y}{1}=\dfrac{z-2}{3}$}
{$\dfrac{x-1}{3}=\dfrac{y}{1}=\dfrac{z+2}{-7}$}

\end{ex}
\begin{ex}%Câu 26.
Số đường tiệm cận của đồ thị hàm số $y=\dfrac{3}{x-2}$ bằng
\choice
{$0$}
{\True $2$}
{$3$}
{$1$}

\end{ex}
\begin{ex}%Câu 27.
Với giá trị nào của $x$ thì hàm số $y=2^{2\log_3 x-\log_3 2 x}$ đạt giá trị lớn nhất?
\choice
{$1$}
{$2$}
{\True $3$}
{$\sqrt{2}$}

\end{ex}
\begin{ex}%Câu 28.
Gọi $z_1, z_2$ là hai nghiệm phức của phương trình $2 z^2+\sqrt{3} z+3=0$. Giá trị của $z_1^2+z_2^2$ bằng
\choice
{$-\dfrac{9}{8}$}
{$3$}
{$\dfrac{3}{18}$}
{\True $-\dfrac{9}{4}$}

\end{ex}
\begin{ex}%Câu 29.
Cho biểu thức $P=x \sqrt[5]{x \sqrt[3]{x \sqrt{x}}}, x>0$. Mệnh đề nào dưới đây đúng
\choice
{$P=x^{\frac{2}{3}}$}
{$P=x^{\frac{3}{10}}$}
{\True $P=x^{\frac{13}{10}}$}
{$P=x^{\frac{1}{2}}$}

\end{ex}
\begin{ex}%Câu 30.
Hàm số $y=\dfrac{x^3}{3}+\dfrac{x^2}{2}-2 x-1$ có giá trị lớn nhất trên đoạn $[0; 2]$ là
\choice
{$0$}
{\True $-\dfrac{1}{3}$}
{$-1$}
{$-\dfrac{13}{6}$}

\end{ex}
\begin{ex}%Câu 31.
Rút ra một lá bài từ bộ bài tú lơ khơ $52$ lá. Xác suất để được lá rô là
{$\dfrac{1}{13}$}
{\True  $\dfrac{1}{4}$}
{ $\dfrac{12}{1.3}$}
{ $\dfrac{3}{4}$}

\end{ex}
\begin{ex}%Câu 32.
Cho tam giác $ABC$ vuông tại $B$ có $AC=2 a, BC=a$, khi quay tam giác $ABC$ quanh cạnh góc vuông $AB$ thì đường gấp khúc $ACB$ tạo thành một hình nón tròn xoay có diện tích xung quanh bằng
\choice
{$4\pi a^2$}
{\True $2\pi a^2$}
{$\pi a^2$}
{$3\pi a^2$}

\end{ex}
\begin{ex}%Câu 33.
Cho hình chóp $S.ABC$ có đáy $ABC$ là tam giác vuông tại $B$ và $BA=BC=a$. Cạnh bên $SA=2 a$ và vuông góc với mặt phẳng đáy. Tính theo $a$ thể tích của khối chóp $S.ABC$.
\choice
{$\dfrac{a^3 \sqrt{3}}{2}$}
{$\dfrac{2 a^3}{3}$}
{$a^3$}
{\True $\dfrac{a^3}{3}$}

\end{ex}
\begin{ex}%Câu 34.
Cho hàm số $y=\dfrac{x^3}{3}-3 x^2+5 x-2$ nghịch biến trên khoảng
\choice
{$(5;+\infty)$}
{\True $(2; 3)$}
{$(1; 6)$}
{$(-\infty; 1)$}

\end{ex}
\begin{ex}%Câu 35.
Trong không gian tọa độ $O x y z$, đường thẳng đi qua điểm $A(3; 0;-4)$ và có vectơ chỉ phương $\vec{u}(5; 1;-2)$ có phương trình:
\choice
{$\dfrac{x+3}{5}=\dfrac{y}{1}=\dfrac{z-4}{-2}$}
{$\dfrac{x+3}{5}=\dfrac{y}{1}=\dfrac{z+4}{-2}$}
{\True $\dfrac{x-3}{5}=\dfrac{y}{1}=\dfrac{z+4}{-2}$}
{$\dfrac{x-3}{5}=\dfrac{y}{1}=\dfrac{z-4}{-2}$}

\end{ex}
\begin{ex}%Câu 36.
Tìm tọa độ $M$ là điểm biểu diễn số phức $z=3-4 i$ 
\choice
{$M(-3;-4)$}
{$M(3; 4)$}
{\True $M(3;-4)$}
{$M(-3; 4)$}

\end{ex}
\begin{ex}%Câu 37.
Tính thể tích $V$ của khối lập phương $ABCD \cdot A_1 B_1 C_1 D_1$ biết diện tích mặt chéo $ACC_1 A_1$ bằng $4\sqrt{2} a^2$.
\choice
{$V=2 a^3$}
{$V=4 a^3$}
{\True $V=8 a^3$}
{$V=16 a^3$}

\end{ex}
\begin{ex}%Câu 38.
Cho hai hàm số $y=f(x)$ và $y=g(x)$ liên tục trên đoạn $[b; a]$. Gọi $D$ là diện tích hình phẳng giới hạn bởi các đồ thị hàm số $y=f(x), y=g(x)$ và hai đường thẳng $x=a, x=b(a>b)$, diện tích của $D$ được tính theo công thức
\choice
{$\displaystyle\displaystyle\int\limits_a^{b}(f(x)-g(x)) \mathrm{d} x$}
{$S=\displaystyle\displaystyle\int\limits_a^{b} f(x) \mathrm{d} x-\displaystyle\displaystyle\int\limits_a^{b} g(x) \mathrm{d} x$}
{$\displaystyle\displaystyle\int\limits_a^{b}|f(x)-g(x)| \mathrm{d} x$}
{\True $\displaystyle\displaystyle\int\limits_b^{a}|f(x)-g(x)| \mathrm{d} x$}

\end{ex}
\begin{ex}%Câu 39.
Cho hàm số $y=\dfrac{x^3}{3}-2 x^2+3 x+\dfrac{2}{3}$ có đồ thị là $(C)$. Tìm toạ độ điểm cực đại của đồ thị hàm số $(C)$.
\choice
{\True $(1; 2)$}
{$\left(3; \dfrac{2}{3}\right)$}
{$(-1; 2)$}
{$(1;-2)$}

\end{ex}

\begin{ex}%Câu 40.
Tiệm cận đứng của đồ thị hàm số $y=\dfrac{x+1}{x+3}$ là
\choice
{$x=3$}
{$x=1$}
{$x=-1$}
{\True $x=-3$}

\end{ex}
\begin{ex}%Câu 41.
Tính đạo hàm của hàm số $y=2^{x+1}$ 
\choice
{$y'=(x+1) 2^{x} \ln 2$}
{$y'=2^{x+1} \log 2$}
{\True $y'=2^{x+1} \ln 2$}
{$y'=\dfrac{2^{x+1}}{\ln 2}$}

\end{ex}
\begin{ex}%Câu 42.
Cho $F(x)$ là một nguyên hàm của hàm số $f(x)={\rm e}^{x}+2 x$ thỏa mãn $F(0)=\dfrac{3}{9}$. Tìm $F(x)$.
\choice
{\True $F(x)={\rm e}^{x}+x^2+\dfrac{1}{2}$}
{$F(x)={\rm e}^{x}+x^2+\dfrac{3}{2}$}
{$F(x)=2 {\rm e}^{x}+x^2-\dfrac{1}{2}$}
{$F(x)={\rm e}^{x}+x^2+\dfrac{5}{2}$}

\end{ex}
\begin{ex}%Câu 43.
Cho hình $\displaystyle\displaystyle\int\limits_2^5 \dfrac{\mathrm{d} x}{x}=\ln a$. Tìm $a$.
\choice
{$\dfrac{2}{5}$}
{$5$}
{$2$}
{\True $\dfrac{5}{2}$}

\end{ex}
\begin{ex}%Câu 44.
Khối trụ tròn xoay có đường cao và bán kính đáy cùng bằng $1$ thì thể tích bằng:
\choice
{$\pi^2$}
{$2\pi$}
{\True $\pi$}
{$\dfrac{1}{3} \pi$}

\end{ex}
\begin{ex}%Câu 45.
Cho $z=3+4 i$, tìm phần thực phần ảo của số phức $\dfrac{1}{z}$:
\choice
{\True Phần thực là $\dfrac{3}{25}$, phần ảo là $\dfrac{-4}{25}$}
{Phần thực là $\dfrac{1}{3}$, phần ảo là $\dfrac{-1}{4}$}
{Phần thực là $\dfrac{3}{5}$, phần ảo là $\dfrac{-4}{5}$}
{Phần thực là $\dfrac{1}{3}$, phần ảo là $\dfrac{1}{4}$}

\end{ex}
\begin{ex}%Câu 46.
Đồ thị hàm số nào sau đây không cắt trục hoành?
\choice
{$y=-x^3-2 x^2-4 x+5$}
{$y=\dfrac{2 x-1}{x+2}$}
{\True $y=x^4+2 x^2+3$}
{$y=-x^4+4 x^2-3$}

\end{ex}
\begin{ex}%Câu 47.
\immini{
Đồ thị như hình vẽ là của hàm số nào trong các hàm số đã cho dưới đây?
\choice
{$f(x)=-x^3+3 x$}
{\True $f(x)=x^3-3 x$}
{$f(x)=x^3-3 x+1$}
{$f(x)=\dfrac{x}{x^2+1}$}}
{\vspace{-0.6cm}
\begin{tikzpicture}[scale=1, font=\footnotesize, line join=round, line cap=round, >=stealth,y=0.8cm,color=\mauchinh]
\def\xmin{-2}\def\xmax{2}\def\ymin{-2.01}\def\ymax{2.01}
\draw[->,thick] (\xmin-0.2,0)--(\xmax+0.2,0) node[below] {\footnotesize $x$};
\draw[->,thick] (0,\ymin-0.2)--(0,\ymax+0.2) node[right] {\footnotesize $y$};
\draw (0,0) node [below left] {\footnotesize $O$};
\foreach \x in {}\draw (\x,0.1)--(\x,-0.1) node [below] {\footnotesize $\x$};
\foreach \y in {}\draw (0.1,\y)--(-0.1,\y) node [left] {\footnotesize $\y$};
\clip (\xmin,\ymin) rectangle (\xmax,\ymax);
\draw[thick,smooth,samples=200,domain=\xmin:\xmax] plot (\x,{1*((\x)^3)+0*((\x)^2)+-3*(\x)+0});
\end{tikzpicture}
}

\end{ex}
\begin{ex}%Câu 48.
Trong không gian tọa độ $O x y z$, cho mặt cầu $(S)\colon x^2+y^2+z^2+$ $4 x-2 y+6 z+5=0$. Mặt cầu $(S)$ có bán kính là:
\choice
{$7$}
{\True $3$}
{$5$}
{$2$}

\end{ex}

\begin{ex}%Câu 50.
Hình tứ diện có số cạnh là
\choice
{\True $6$}
{$5$}
{$4$}
{$3$}

\end{ex}
\Closesolutionfile{ans}
%% \indapan{10}{ans/ans-de4-7}