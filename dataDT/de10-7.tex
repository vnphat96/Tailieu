
\begin{name}
	{\tenchude}
	{\tendethi}
	{\tentruong}
	{\thoigian}
\end{name}
\Opensolutionfile{ans}[ans/ans-de10-7]

\begin{ex}%Câu 40.
Cho số phức $z=1-2 i$. Số phức $(2+3 i) \bar{z}$ bằng
\choice
{$4-7 i$}
{$-8+i$}
{$8+i$}
{\True $-4+7 i$}

\end{ex}
\begin{ex}%Câu 1.
Trong không gian $O x y z$, đường thẳng $d$ qua $M(-3; 5; 6)$ và vuông góc với mặt phẳng $(P)\colon 2 x-3 y+4 z-2=0$ thì đường thẳng $d$ có phương trình là
\choice
{$\dfrac{x-3}{2}=\dfrac{y+5}{-3}=\dfrac{z+6}{4}$}
{$\dfrac{x+3}{2}=\dfrac{y-5}{3}=\dfrac{z-6}{4}$}
{$\dfrac{x+3}{2}=\dfrac{y-5}{-3}=\dfrac{z-6}{-4}$}
{\True $\dfrac{x+3}{2}=\dfrac{y-5}{-3}=\dfrac{z-6}{4}$}

\end{ex}

\begin{ex}%2
Biết $\displaystyle\int\limits_0^1[f(x)+2 x] \mathrm{d} x=5$. Khi đó $\displaystyle\int\limits_0^1 f(x) \mathrm{d} x$ bằng
\choice
{$3$}
{$5$}
{\True $4$}
{$7$}

\end{ex}

\begin{ex}%Câu 3
Trong không gian $O x y z$, cho mặt phẳng $(\alpha)\colon x-2 y+4 z-1=0$. Véctơ nào dưới đây là một vectơ pháp tuyến của $(\alpha)$?
\choice
{$\overrightarrow{n_1}=(1; 2;-4)$}
{$\overrightarrow{n_2}=(1; 2; 4)$}
{\True $\overrightarrow{n_3}=(1;-2; 4)$}
{$\overrightarrow{n_4}=(-1; 2; 4)$}

\end{ex}
\begin{ex}%Câu 4.
Xét cấp số cộng $\left(u_n\right), n \in \mathbb{N}^{*}$ có $u_1=5, u_{12}=38$. Khi đó $u_{10}$ bẵng
\choice
{$u_{10}=35$}
{\True $u_{10}=32$}
{$u_{10}=24$}
{$u_{10}=30$}

\end{ex}
\begin{ex}%Câu 5.
Trong không gian $O x y z$, cho hai vectơ $\vec{u}=(1; 4; 1)$ và $\vec{v}=(-1; 1;-3)$. Góc tạo bởi hai vectơ $\vec{u}$ và $\vec{v}$ là
\choice
{$60^{\circ}$}
{$30^{\circ}$}
{\True $90^{\circ}$}
{$120^{\circ}$}

\end{ex}
\begin{ex}%Câu 6.
Tập nghiệm $S$ của phương trình $4^{x^2}=2^{x+1}$ là
\choice
{$S=\left\{-1; \dfrac{1}{2}\right\}$}
{\True $S=\left\{-\dfrac{1}{2}; 1\right\}$}
{$S=\left\{\dfrac{1-\sqrt{5}}{2}; \dfrac{1+\sqrt{5}}{2}\right\}$}
{$S=\{0; 1\}$}

\end{ex}
\begin{ex}%Câu 7.
Tập nghiệm của bất phương trình $\log_{\frac{1}{2}}(x+1)<\log_{\frac{1}{2}}(2 x-1)$ chứa bao nhiêu số nguyên?
\choice
{\True $1$}
{$0$}
{$\text{vô số}$}
{$2$}

\end{ex}
\begin{ex}%Câu 8.
Gọi $M, m$ lần lượt là giá trị lớn nhất và giá trị nhỏ nhất của hàm số $y=\dfrac{x^2+x+3}{x-2}$ trên $[-2; 1]$ Giá trị của $M+m$ bằng
\choice
{$-5$}
{\True $-6$}
{$-\dfrac{9}{4}$}
{$-\dfrac{25}{4}$}

\end{ex}
\begin{ex}%Câu 9.
Thiết diện qua trục của hình trụ là một hình chữ nhật có diện tích bằng $10$. Diện tích xung quanh của hình trụ đó bằng
\choice
{$5$}
{$5\pi$}
{$10$}
{\True $10\pi$}

\end{ex}
\begin{ex}%Câu 10.
Tất cả các giá trị thực của tham số $m$ để hàm số $y=x^3-3 x^2+$ $m x+2$ đồng biến trên $\mathbb{R}$ là
\choice
{$m \leq 3$}
{$m>3$}
{\True $m \geq 3$}
{$m<3$}

\end{ex}
\begin{ex}%Câu 11.
Hệ số của số hạng chứa $x^5$ trong khai triển đa thức $(2+x)^{15}$ là
\choice
{$2^9 C_{15}^6$}
{\True $2^{10} C_{15}^5$}
{$2^9 C_{15}^5$}
{$2^{10} C_{15}^6$}

\end{ex}
\begin{ex}%Câu 12.
\immini{
Cho hàm số $y=a x^3+b x^2+c x+d$ có đồ thị như hình vẽ. Tập hợp các giá trị của tham số $m$ để phương trình $f(2-x)=$ $m$ có đúng ba nghiệm phân biệt là
\choice
{$(1; 3)$}
{\True $(-1; 3)$}
{$(-1; 1)$}
{$(-3; 1)$}}
{\vspace{-.5cm}
\begin{tikzpicture}[scale=1, font=\footnotesize, line join=round, line cap=round, >=stealth,color=\mauchinh,y=0.7cm]
\def\xmin{-2}\def\xmax{2}\def\ymin{-1}\def\ymax{3.5}
\draw[->,thick] (\xmin-0.2,0)--(\xmax+0.2,0) node[below] {\footnotesize $x$};
\draw[->,thick] (0,\ymin-0.2)--(0,\ymax+0.2) node[right] {\footnotesize $y$};
\draw (0,0) node [below left] {\footnotesize $O$};
\foreach \x in {-1,1}\draw (\x,0.1)--(\x,-0.1) node [below] {\footnotesize $\x$};
\foreach \y in {-1,3}\draw (0.1,\y)--(-0.1,\y) node [left] {\footnotesize $\y$};
\clip (\xmin,\ymin) rectangle (\xmax,\ymax);
\draw[thick,smooth,samples=200,domain=\xmin:\xmax] plot (\x,{1*((\x)^3)+0*((\x)^2)+-3*(\x)+1});
\draw[dashed] (-1,0)--(-1,3)--(0,3);\fill (-1,3) circle (1pt);
\draw[dashed] (1,0)--(1,-1)--(0,-1);\fill (1,-1) circle (1pt);
\end{tikzpicture}
}

\end{ex}
\begin{ex}%Câu 13.
Trong không gian Oxyz, cho mặt cầu $(S)\colon(x+1)^2+(y-2)^2+$ $(z+3)^2=4$. Tâm của $(S)$ có tọa độ là
\choice
{$(-2; 4;-6)$}
{$(-1; 2; 3)$}
{$(2;-4; 6)$}
{\True $(-1; 2;-3)$}

\end{ex}
\begin{ex}%Câu 14.
Cho cấp số cộng $\left(u_n\right)$ với $u_1=11$ và công sai $d=3$. Giá trị của $u_2$ bằng:
\choice
{$\dfrac{11}{3}$}
{$8$}
{$33$}
{\True $14$}

\end{ex}
\begin{ex}%Câu 15.
Với $a$ là số thực dương tùy ý, $\log_4(4 a)$ bằng
\choice
{$4+\log_4 a$}
{$1-\log_4 a$}
{$4-\log_4 a$}
{\True $1+\log_4 a$}

\end{ex}
\begin{ex}%Câu 16.
Tập xác định của hàm số $y=4^{x}$ là
\choice
{\True $\mathbb{R}$}
{$(0;+\infty)$}
{$\mathbb{R} \backslash\{0\}$}
{$[0;+\infty)$}

\end{ex}
\begin{ex}%Câu 17.
Trong không gian $O x y z$, cho đường thẳng $d\colon \dfrac{x-2}{4}=\dfrac{y-1}{-2}=$ $\dfrac{z+3}{1}$. Điểm nào sau đây thuộc $d$?
\choice
{$N(4; 2; 1)$}
{$M(2; 1; 3)$}
{\True $P(2; 1;-3)$}
{$Q(4;-2; 1)$}

\end{ex}
\begin{ex}%Câu 18.
Cho khối lăng trụ có diện tích đáy $B=3$ và chiều cao $h=6$. Thể tích của khối lăng trụ đã cho bằng
\choice
{$3$}
{$9$}
{$6$}
{\True $18$}

\end{ex}
\begin{ex}%Câu 19.
Cho mặt cầu có bán kính $r=4$. Diện tích của mặt cầu đã cho bằng
\choice
{\True $64\pi$}
{$\dfrac{265\pi}{3}$}
{$16\pi$}
{$\dfrac{64\pi}{3}$}

\end{ex}
\begin{ex}%Câu 20.
Nghiệm của phương trình $\log_2(x+8)=5$ là
\choice
{$x=17$}
{\True $x=24$}
{$x=40$}
{$x=2$}

\end{ex}
\begin{ex}%Câu 21.
Biết $\displaystyle\int\limits_2^3 f(x) \mathrm{d} x=4$ và $\displaystyle\int\limits_2^3 g(x) \mathrm{d} x=1$. Khi đó $\displaystyle\int\limits_2^3[f(x)-g(x)] \mathrm{d} x$ 
bằng
\choice
{$-3$}
{$5$}
{$4$}
{\True $3$}

\end{ex}
\begin{ex}%Câu 22.
\immini{
Cho hàm số bậc bốn $y=f(x)$ có đồ thị là đường cong trong hình bên. Số nghiệm thực của phương trình $f(x)=-\dfrac{1}{2}$ là
\choice
{$1$}
{$4$}
{\True $2$}
{$3$}}
{\vspace{-0.5cm}
\begin{tikzpicture}[scale=1, font=\footnotesize, line join=round, line cap=round, >=stealth,color=\mauchinh]
\def\xmin{-1.65}\def\xmax{1.65}\def\ymin{-2.1}\def\ymax{1}
\draw[->,thick] (\xmin-0.2,0)--(\xmax+0.2,0) node[below] {\footnotesize $x$};
\draw[->,thick] (0,\ymin-0.2)--(0,\ymax+0.2) node[right] {\footnotesize $y$};
\draw (0,0) node [below left] {\footnotesize $O$};
\foreach \x in {-1,1}\draw (\x,0.1)--(\x,-0.1) node [below] {\footnotesize $\x$};
\foreach \y in {-2}\draw (0.1,\y)--(-0.1,\y) node [left] {\footnotesize $\y$};
\clip (\xmin,\ymin) rectangle (\xmax,\ymax);
\draw[thick,smooth,samples=200,domain=\xmin:\xmax] plot (\x,{1*((\x)^4)+-2*((\x)^2)+-1});
\draw[dashed] (-1,0)--(-1,-2)--(0,-2);\fill (-1,-2) circle (1pt);
\draw[dashed] (1,0)--(1,-2)--(0,-2);\fill (1,-2) circle (1pt);
\end{tikzpicture}
}

\end{ex}
\begin{ex}%Câu 23.
Cho khối chóp có diện tích đáy $B=2 a^2$, chiều cao $h=6 a$. Thể tích của khối chóp cho bằng
\choice
{$12 a^3$}
{$6 a^3$}
{$2 a^3$}
{\True $4 a^3$}

\end{ex}
\begin{ex}%Câu 24.
Cho hai số phức $z_1=3+2 i, z_2=1-i$. Số phức $z_1-z_2$ bằng
\choice
{\True $2+3 i$}
{$-2+3 i$}
{$2-3 i$}
{$-2-3 i$}

\end{ex}
\begin{ex}%Câu 25.
Cho khối trụ có bán kính $r=4$ và chiều cao $h=3$. Thể tích của khối trụ đã cho bằng
\choice
{\True $48\pi$}
{$24\pi$}
{$4\pi$}
{$16\pi$}

\end{ex}
\begin{ex}%Câu 26.
Trong không gian $O x y z$, cho mặt phẳng $(\alpha)\colon 2 x+4 y-z+3=0$. Vectơ nào dưới đây là một vectơ pháp tuyến của $(\alpha)$?
\choice
{$\overrightarrow{n_3}=(2; 4; 1)$}
{$\overrightarrow{n_4}=(-2; 4; 1)$}
{\True $\overrightarrow{n_1}=(2; 4;-1)$}
{$\overrightarrow{n_3}=(2;-4; 1)$}

\end{ex}
\begin{ex}%Câu 27.
Cho hình nón có bán kính đáy $r=2$ và độ dài đường sinh $l=5$. Diện tích xung quanh của hình nón đã cho bằng
\choice
{\True $10\pi$}
{$20\pi$}
{$\dfrac{20}{3} \pi$}
{$\dfrac{10}{3} \pi$}

\end{ex}
\begin{ex}%Câu 28.
\immini{
Cho hàm số $y=f(x)$ có đồ thị là đường cong trong hình bên. Hàm số đã cho đồng biến trên khoảng nào dưới đây?
\choice
{\True $(0; 1)$}
{$(-\infty; 0)$}
{ $(1;+\infty)$}
{ $(-1; 0)$}}
{\vspace{-0.5cm}
\begin{tikzpicture}[scale=1, font=\footnotesize, line join=round, line cap=round, >=stealth,color=\mauchinh]
\def\xmin{-1.65}\def\xmax{1.65}\def\ymin{-0.5}\def\ymax{2.2}
\draw[->,thick] (\xmin-0.2,0)--(\xmax+0.2,0) node[below] {\footnotesize $x$};
\draw[->,thick] (0,\ymin-0.2)--(0,\ymax+0.2) node[right] {\footnotesize $y$};
\draw (0,0) node [below left] {\footnotesize $O$};
\foreach \x in {-1,1}\draw (\x,0.1)--(\x,-0.1) node [below] {\footnotesize $\x$};
\foreach \y in {1,2}\draw (0.1,\y)--(-0.1,\y) node [left] {\footnotesize $\y$};
\clip (\xmin,\ymin) rectangle (\xmax,\ymax);
\draw[thick,smooth,samples=200,domain=\xmin:\xmax] plot (\x,{-1*((\x)^4)+2*((\x)^2)+1});
\draw[dashed] (-1,0)--(-1,2)--(0,2);\fill (-1,2) circle (1pt);
\draw[dashed] (1,0)--(1,2)--(0,2);\fill (1,2) circle (1pt);
\end{tikzpicture}
}

\end{ex}
\begin{ex}%Câu 29.
Nghiệm của phương trình $2^{2 x-3}=2^{x}$ là
\choice
{\True $x=3$}
{$x=8$}
{$x=-3$}
{$x=-8$}

\end{ex}
\begin{ex}%Câu 30.
Có bao nhiêu cách chọn một học sinh từ một nhóm gồm $5$ học sinh nam và $6$ học sinh nữ?
\choice
{$6$}
{\True $11$}
{$30$}
{$5$}

\end{ex}
\begin{ex}%Câu 31.
\immini{
Đồ thị của hàm số nào dưới đây có dạng như đường cong trong hình bên?
\choice
{$y=x^3-3 x^2-2$}
{$y=x^4-2 x^2-2$}
{$y=-x^4+2 x^2-2$}
{\True $y=-x^3+3 x^2-2$}
}
{\vspace{-.5cm}
\begin{tikzpicture}[scale=1, font=\footnotesize, line join=round, line cap=round, >=stealth,y=0.7cm,color=\mauchinh]
\def\xmin{-1.01}\def\xmax{2.98}\def\ymin{-2}\def\ymax{2.13}
\draw[->,thick] (\xmin-0.2,0)--(\xmax+0.2,0) node[below] {\footnotesize $x$};
\draw[->,thick] (0,\ymin-0.2)--(0,\ymax+0.2) node[right] {\footnotesize $y$};
\draw (0,0) node [below left] {\footnotesize $O$};
\foreach \x in {}\draw (\x,0.1)--(\x,-0.1) node [below] {\footnotesize $\x$};
\foreach \y in {}\draw (0.1,\y)--(-0.1,\y) node [left] {\footnotesize $\y$};
\clip (\xmin,\ymin) rectangle (\xmax,\ymax);
\draw[thick,smooth,samples=200,domain=\xmin:\xmax] plot (\x,{-1*((\x)^3)+3*((\x)^2)+0*(\x)+-2});
\end{tikzpicture}
}

\end{ex}
\begin{ex}%32
\immini{
Cho hàm số $y=f(x)$ có bảng biến thiên như hình bên. Điểm cực đại của hàm số đã cho là
\choice
{$x=3$}
{$x=-1$}
{\True $x=-3$}
{$x=2$}}
{
\begin{tikzpicture}[color=\mauchinh]
\tkzTabInit[nocadre,lgt=1.2,espcl=1.6,deltacl=0.6,lw=1]
{$x$/0.6,$f'(x)$/0.6,$f(x)$/2}{$-\infty$,$-1$,$3$,$+\infty$}
\tkzTabLine{,-,0,+,0,-,}
\tkzTabVar{+/$+\infty$,-/$-3$,+/$2$,-/$-\infty$}
\end{tikzpicture}

}
\end{ex}
\begin{ex}%Câu 33.
Phần thực của số phức $z=-3-4 i$ bằng
\choice
{$3$}
{\True $-3$}
{$4$}
{$-4$}

\end{ex}
\begin{ex}%Câu 34.
Trong không gian $O x y z$, điểm nào dưới đây là hình chiếu vuông góc của điểm $A(1; 4; 2)$ trên mặt phẳng $(O x y)$?
\choice
{$Q(1; 0; 2)$}
{$M(0; 0; 2)$}
{$N(0; 4; 2)$}
{\True $P(1; 4; 0)$}

\end{ex}
\begin{ex}%Câu 35.
Trên mặt phẳng tọa độ, điểm nào dưới đây là điểm biểu diễn số phức $z=-3+4 i$?
\choice
{\True $P(-3; 4)$}
{$N(3; 4)$}
{$Q(4;-3)$}
{$M(4; 3)$}

\end{ex}
\begin{ex}%Câu 36. 
$\displaystyle\int 5 x^4 \mathrm{\,d} x$ bằng
\choice
{$20 x^3+C$}
{$\dfrac{1}{5} x^5+C$}
{$5 x^5+C$}
{\True $x^5+C$}

\end{ex}
\begin{ex}%Câu 37.
Tiệm cận đứng của đồ thị hàm số $y=\dfrac{2 x+2}{x-1}$ là
\choice
{$x=2$}
{\True $x=1$}
{$x=-2$}
{$x=-1$}

\end{ex}
\begin{ex}%Câu 38.
Giá trị nhỏ nhất của hàm số $f(x)=x^4-10 x^2-4$ trên $[0; 9]$ bằng
\choice
{\True $-29$}
{$-13$}
{$-28$}
{$-4$}

\end{ex}
\begin{ex}%Câu 39.
Trong không gian $O x y z$, cho điểm $M(1;-2; 3)$ và mặt phẳng $(P)$: $2 x-y+3 z+1=0$. Phương trình của đường thẳng đi qua $M$ và vuông góc với $(P)$ là
\choice
{$\heva{&x=1-2 t \\& y=-2-t. \\& z=3-3 t}$}
{$\heva{&x=1-2 t \\& y=-2-t. \\& z=3-3 t}$}
{\True $\heva{&x=1-2 t \\& y=-2-t. \\& z=3-3 t}$}
{$\heva{&x=1-2 t \\& y=-2-t. \\& z=3-3 t}$}

\end{ex}

\begin{ex}%Câu 41.
Gọi $D$ là hình phẳng giới hạn bởi các đường $y={\rm e}^{3 x}, y=0, x=0$ và $x=1$. Thể tích của khối tròn xoay tạo thành khi quay $D$ quay quanh $\mathrm{O} x$ bằng
\choice
{$\displaystyle\int\limits_0^1 {\rm e}^{3 x}\mathrm{\,d}x$}
{$\displaystyle\int\limits_0^1 {\rm e}^{6 x}\mathrm{\,d}x$}
{\True $\pi \displaystyle\int\limits_0^1 {\rm e}^{6 x}\mathrm{\,d}x$}
{$\pi \displaystyle\int\limits_0^1 {\rm e}^{3 x}\mathrm{\,d}x$}

\end{ex}
\begin{ex}%Câu 42.
Cắt hình trụ $(T)$ bởi một mặt phẳng qua trục của nó, ta được thiết diện là một hình vuông có cạnh bằng $7$. Diện tích xung quanh của $(T)$ bằng
\choice
{$98\pi$}
{$\dfrac{49\pi}{2}$}
{\True $49\pi$}
{$\dfrac{49\pi}{4}$}

\end{ex}
\begin{ex}%Câu 43.
Với $a, b$ là các số thực dương tùy ý thỏa mãn $\log_2 a-2\log_4 b=3$, mệnh đề nào dưới đây đúng?
\choice
{$a=8 b^4$}
{$a=8 b^2$}
{\True $a=8 b$}
{$a=6 b$}

\end{ex}
\begin{ex}%Câu 44.
Trong không gian $O x y z$, cho điểm $M(2;-1; 4)$ và mặt phẳng $(P)$: $3 x-2 y+z+1=0$. Phương trình mặt phẳng đi qua $M$ và song song với $(P)$ là
\choice
{\True $3 x-2 y+z-12=0$}
{$3 x-2 y+z-12=0$}
{$3 x-2 y+z-12=0$}
{$3 x-2 y+z-12=0$}

\end{ex}
\begin{ex}%Câu 45.
Tập nghiệm của bất phương trình $\log_3\left(18-x^2\right) \geq 2$ là:
\choice
{$(-\infty;-3] \cup[3;+\infty)$}
{$(-\infty; 3]$}
{\True $[-3; 3]$}
{$(0; 3]$}

\end{ex}
\begin{ex}%Câu 46.
Gọi $z_1, z_2$ là hai nghiệm phức của phương trình $z^2+z+2=0$. Khi đó $\left|z_1\right|+\left|z_2\right|$ bằng
\choice
{$2$}
{\True $2\sqrt{2}$}
{$\sqrt{2}$}
{$4$}

\end{ex}
\begin{ex}%Câu 47.
Cho hình hộp chữ nhật $ABCD \cdot A'B'C'D'$ có $AB=BC=a; AA'=$ $\sqrt{6} a$. Góc giữa đường thẳng $A'C$ và mặt phẳng $(ABCD)$ bằng
\choice
{$45^{\circ}$}
{$90^{\circ}$}
{$30^{\circ}$}
{\True $60^{\circ}$}

\end{ex}
\begin{ex}%Câu 48.
Cho hàm số $f'(x)=x(x-1)(x+4)^3, \forall x \in \mathbb{R}$. Số điểm cực đại của hàm số đã cho là
\choice
{$2$}
{\True $1$}
{$4$}
{$3$}

\end{ex}
\begin{ex}%Câu 49.
Số giao điểm của đồ thị hàm số $y=-x^3+6 x$ với trục hoành là:
\choice
{$1$}
{$0$}
{\True $3$}
{$2$}

\end{ex}
\begin{ex}%Câu 50.
Trong không gian $O x y z$, hình chiếu vuông góc của điểm $M(3; 4;-2)$ lên mặt phẳng $(O x z)$ có tọa độ là
\choice
{$Q(3; 0; 0)$}
{$G(3; 4; 0)$}
{$E(0; 4;-2)$}
{\True $F(3; 0;-2)$}

\end{ex}

\Closesolutionfile{ans}
%\indapan{10}{ans/ans-de10-7}