
\begin{name}
	{\tenchude}
	{\tendethi}
	{\tentruong}
	{\thoigian}
\end{name}
\Opensolutionfile{ans}[ans/ans-de9-7]

\begin{ex}%Câu 36.
Số phức nào sau đây có biểu diễn hình học là điểm $M(3; 5)$?
\choice
{$z=3-5 i$}
{$z=-3-5 i$}
{\True $z=3+5 i$}
{$z=-3+5 i$}

\end{ex}
\begin{ex}%Câu 1.
Một hộp sữa có dạng hình trụ và có thể tích bằng $2825\mathrm{\,cm}^3$. Biết chiều cao của hộp sưaa bằng $25\mathrm{\,cm}$. Diện tích toàn phần của hộp sữa đó gần với số nào sau đây nhất?
\choice
{\True $1168\mathrm{\,cm}^2$}
{$1172\mathrm{\,cm}^2$}
{$1182\mathrm{\,cm}^2$}
{$1164\mathrm{\,cm}^2$}

\end{ex}
\begin{ex}%Câu 2.
Tìm nguyên hàm của hàm số $f(x)=\cos 4 x$.
\choice
{$\displaystyle\int f(x) \mathrm{d} x=-\dfrac{1}{4} \sin 4 x+C$}
{$\displaystyle\int f(x) \mathrm{d} x=-4\sin 4 x+C$}
{\True $\displaystyle\int f(x) \mathrm{d} x=\dfrac{1}{4} \sin 4 x+C$}
{$\displaystyle\int f(x) \mathrm{d} x=4\sin 4 x+C$}

\end{ex}
\begin{ex}%Câu 3.
Tập nghiệm của bất phương trình $2^{x+2}<\left(\dfrac{1}{4}\right)^{x}$ là
\choice
{\True $\left(-\infty;-\dfrac{2}{3}\right)$}
{$(-\infty; 0)$}
{$\left(-\dfrac{2}{3};+\infty\right)$}
{$(-\infty; 0) \backslash\{1\}$}

\end{ex}
\begin{ex}%Câu 4.
Trong không gian với hê trục tọa độ $O x y z$, mặt cầu ($S$) có tâm $I(2; 3;-6)$ và bán kính $R=4$ có phương trình là
\choice
{$(x+2)^2+(y+3)^2+(z-6)^2=4$}
{$(x-2)^2+(y-3)^2+(z+6)^2=4$}
{$(x+2)^2+(y+3)^2+(z-6)^2=16$}
{\True $(x-2)^2+(y-3)^2+(z+6)^2=16$}

\end{ex}
\begin{ex}%Câu 5.
Cho hàm số $y$ xác định và liên tục trên $\mathbb{R}$ và có bảng biến thiên như hình bên. 
\begin{center}
\begin{tikzpicture}[color=\mauchinh]
\tkzTabInit[nocadre,lgt=1.2,espcl=2.5,deltacl=0.5,lw=1]
{$x$/0.6,$f'(x)$/0.6,$f(x)$/2}{$-\infty$,$0$,$2$,$+\infty$}
\tkzTabLine{,+,0,-,0,+,}
\tkzTabVar{-/$-\infty$,+/$-1$,-/$3$,+/$+\infty$}
\end{tikzpicture}
\end{center}

Khẳng định nào sau đây đúng?
\choice
{Hàm số có đúng một cực trị}
{\True Hàm số đạt cực đại tại $x=2$ và đạt cực tiểu tại $x=0$}
{Hàm số có giá trị lớn nhất bằng 3 và giá trị nhỏ nhất bằng $-1$}
{Hàm số có giá trị cực tiểu bằng 0}

\end{ex}
\begin{ex}%Câu 6.
Trong không gian với hệ tọa độ $O x y z$, cho hai điểm\\ $A(1; 2;-3), B(-2; 3; 1)$. Đường thẳng đi qua $A(1; 2;-3)$ và song song với $OB$ có phương trình là
\choice
{$\heva{&x=1-2 t \\& y=2+3 t \\& z=-3-t}$}
{$\heva{&x=1-2 t \\& y=2+3 t \\& z=-3-t}$}
{\True $\heva{&x=1-2 t \\& y=2+3 t \\& z=-3-t}$}
{$\heva{&x=1-2 t \\& y=2+3 t \\& z=-3-t}$}

\end{ex}
\begin{ex}%Câu 7.
Trong không gian với hệ tọa độ $O x y z$, cho hai điểm $A(4; 1;-2)$.
Tọa độ điểm đối xứng với $A$ qua mặt phẳng $(O x z)$ là
\choice
{$A'(-4;-1; 2)$}
{$A'(4; 1; 2)$}
{\True $A'(4;-1;-2)$}
{$A'(4;-1; 2)$}

\end{ex}
\begin{ex}%Câu 8.
Cho tam giác $ABC$ vuông tại $A$ có độ dài các cạnh $AB=3 a, AC=$ $4 a$, quay quanh cạnh $AC$. Thể tích của khối nón tròn xoay được tạo thành là
\choice
{$36\pi a^3$}
{$12\pi a^3$}
{\True $16\pi a^3$}
{$\dfrac{100}{3} \pi a^3$}

\end{ex}
\begin{ex}%Câu 9.
Số phức liên hợp $\bar{z}$ của số phức $z=3(2+3 i)-4(2 i-1)$ là
\choice
{$\bar{z}=10+3 i$}
{$\bar{z}=10+i$}
{$\bar{z}=2-i$}
{\True $\bar{z}=10-i$}

\end{ex}
\begin{ex}%Câu 10.
Trong không gian $O x y z$, mặt phẳng nào sau đây nhận $\vec{n}=(1; 2; 3)$ làm véctơ pháp tuyến?
\choice
{$2 x-4 z+6=0$}
{$x+2 y-3 z-1=0$}
{$x-2 y+3 z+1=0$}
{\True $2 x+4 y+6 z+1=0$}

\end{ex}
\begin{ex}%Câu 11.
Cho một cấp số nhân có $u_1=2, q=-2$. Khi đó số hạng $u_5$ bằng bao nhiêu?
\choice
{\True $32$}
{$-32$}
{$-64$}
{$64$}

\end{ex}
\begin{ex}%Câu 12.
Cho số phức $z=2+3 i$. Tìm số phức $w=(3+2 i) z+2\bar{z}$.
\choice
{\True $w=4+7 i$}
{$w=5+7 i$}
{$w=7+4 i$}
{$w=7+5 i$}

\end{ex}
\begin{ex}%Câu 13.
Tổng của giá trị lớn nhất và giá trị nhỏ nhất của hàm số $f(x)=x(2-\ln x)$ trên đoạn $[2; 3]$ bằng
\choice
{$6-3\ln 3+{\rm e}$}
{$10-2\ln 2-3\ln 3+{\rm e}$}
{$10-2\ln 2-3\ln 3$}
{\True $4-2\ln 2+{\rm e}$}
\end{ex}
\begin{ex}%14
\immini{
Cho hàm số $y=f(x)$ xác định, liên tục trên $\mathbb{R}$ và có bảng biến thiên như hình bên. Số nghiệm của phương trình $f(x)+1=0$ là
}
{

  	\begin{tikzpicture}[scale=1,line width=.6pt,color=\mauchinh]
\tkzTabInit[nocadre=true,lgt=1.5,espcl=1.8,deltacl=0.5,lw=1]
{$x$ /.7,$f'(x)$/.7,$f(x)$/1.8}{$-\infty$,$1$,$3$,$+\infty$}
\tkzTabLine{,+,0,-,d,+,}
\tkzTabVar{-/$-\infty$,+/$2$,-/$1$,+/$+\infty$}
\end{tikzpicture}
}
\boncot
{$0$}
{$1$}
{$3$}
{\True $2$}
\end{ex}
\begin{ex}%Câu 15.
Cho $\displaystyle\int\limits_0^{\frac{\pi}{2}}\left({\rm e}^{\cos x}+\sin x\right) \sin x \mathrm{\,d} x=a+b {\rm e}+c \pi$. Khi đó giá trị $a+b+c$ 
\choice
{$\dfrac{6}{5}$}
{$\dfrac{2}{3}$}
{\True $\dfrac{1}{4}$}
{$\dfrac{3}{5}$}

\end{ex}
\begin{ex}%Câu 16.
Tìm tập hợp tất cả các giá trị thực của $m$ để hàm số $y=\ln \left(x^2+4\right)+m x+12$ đồng biến trên $\mathbb{R}$ là
\choice
{$\left(\dfrac{1}{2};+\infty\right)$}
{\True $\left[\dfrac{1}{2};+\infty\right)$}
{$\left(-\dfrac{1}{2}; \dfrac{1}{2}\right)$}
{$\left(-\infty;-\dfrac{1}{2}\right]$}

\end{ex}
\begin{ex}%Câu 17.
Hình chóp $S.ABCD$ có đáy là hình vuông cạnh $a, SD=\dfrac{a \sqrt{13}}{2}$. Hình chiếu của $S$ lên $(ABCD)$ là trung điểm $I$ của $AB$. Thể tích khối chóp là
\choice
{\True $\dfrac{a^3 \sqrt{2}}{3}$}
{$a^3 \sqrt{12}$}
{$\dfrac{2 a^3}{3}$}
{$\dfrac{a^3}{3}$}

\end{ex}
\begin{ex}%Câu 18.
Tìm giá trị thực của tham số $m$ để hàm số $y=x^3-3 x^2+m x$ đạt cực tiểu tại $x=2$.
\choice
{\True $m=0$}
{$m=1$}
{$m=-2$}
{$m=2$}

\end{ex}
\begin{ex}%Câu 19.
Cho hàm số $f(x)=\log_{2019}\left(-x^2+m x-3 m\right)$. Tất cả các giá trị thực của $m$ để hàm số có tập xác định $D=\mathbb{R}$ là
\choice
{$m \in(-12; 0)$}
{$m \in(-1; 12)$}
{$m \in(-\infty; 0) \cup(2;+\infty)$}
{\True $\color{\mauchinh}{\text{Không tồn tại}}$ $m$}

\end{ex}
\begin{ex}%Câu 20.
Trong không gian $O x y z$, cho mặt phẳng $(Q)$ song song mặt phẳng $(P)\colon 2 x-2 y+z-7=0$. Biết mặt phẳng $(Q)$ cắt mặt cầu $(S)\colon x^2+(y-2)^2+(z+1)^2=25$ theo một đường tròn có bán kính $r=3$. Khi đó mặt phẳng $(Q)$ có phương trình là
\choice
{$2 x-2 y+z-7=0$}
{$2 x-2 y+z-17=0$}
{\True $2 x-2 y+z+17=0$}
{$x-y+2 z-7=0$}

\end{ex}
\begin{ex}%Câu 21.
\immini{
Đồ thị sau đây là đồ thị hàm số nào?
\choice
{$y=-x^4+2 x^2$}
{$y=x^4-2 x^2+1$}
{\True $y=x^4-2 x^2$}
{$y=-x^4+2 x^2+1$}}
{\vspace{-0.5cm}
\begin{tikzpicture}[scale=1, font=\footnotesize, line join=round, line cap=round, >=stealth,color=\mauchinh,y=0.8cm]
\def\xmin{-1.56}\def\xmax{1.56}\def\ymin{-1.01}\def\ymax{1.01}
\draw[->,thick] (\xmin-0.2,0)--(\xmax+0.2,0) node[below] {\footnotesize $x$};
\draw[->,thick] (0,\ymin-0.2)--(0,\ymax+0.2) node[right] {\footnotesize $y$};
\draw (0,0) node [below left] {\footnotesize $O$};
\foreach \x in {-1,1}\draw (\x,0.1)--(\x,-0.1) node [below] {\footnotesize $\x$};
\foreach \y in {-1}\draw (0.1,\y)--(-0.1,\y) node [left] {\footnotesize $\y$};
\clip (\xmin,\ymin) rectangle (\xmax,\ymax);
\draw[thick,smooth,samples=200,domain=\xmin:\xmax] plot (\x,{1*((\x)^4)+-2*((\x)^2)+0});
\draw[dashed] (-1,0)--(-1,-1)--(0,-1);\fill (-1,-1) circle (1pt);
\draw[dashed] (1,0)--(1,-1)--(0,-1);\fill (1,-1) circle (1pt);
\end{tikzpicture}
}

\end{ex}
\begin{ex}%Câu 22.
Cho hàm số $y=x \ln \left(x+\sqrt{1+x^2}\right)-\sqrt{1+x^2}$. Mệnh đề nào sau đây sai?
\choice
{Hàm số đồng biến trên khoảng $(0;+\infty)$}
{Hàm số có đạo hàm $y'=\ln \left(x+\sqrt{1+x^2}\right)$}
{Tập xác định của hàm số là $\mathbb{R}$}
{\True Hàm số nghịch biến trên khoảng $(0;+\infty)$}

\end{ex}
\begin{ex}%Câu 23.
Cho hình trụ có hai đáy là hai hình tròn $(O)$ và $\left(O'\right)$, bán kính bằng $a$. Một hình nón có đỉnh là $O'$ và có đáy là hình tròn $(O)$. Biết góc giữa đường sinh của hình nón và mặt đáy là $60^{\circ}$, tỉ số diện tích xung quanh của hình trụ và hình nón bằng
\choice
{$\sqrt{2}$}
{\True $\sqrt{3}$}
{$\dfrac{1}{\sqrt{3}}$}
{$2$}

\end{ex}
\begin{ex}%Câu 24.
Trong không gian với hệ trục tọa độ $O x y z$, gọi $H(a; b; c)$ là hình chiếu vuông góc của $M(2; 0; 1)$ lên đường thẳng $\Delta\colon \dfrac{x-1}{1}=\dfrac{y}{2}=\dfrac{z-2}{1}$. Tính giá trị $a+4 b+c$.
\choice
{$-8$}
{\True $3$}
{$7$}
{$-15$}

\end{ex}
\begin{ex}%Câu 25.
Cho $\displaystyle\int\limits_1^{13} f(x) \mathrm{d} x=2019$. Tính $\displaystyle\int\limits_0^4 f(3 x+1) \mathrm{d} x$.
\choice
{\True $673$}
{$-2019$}
{$2019$}
{$6057$}

\end{ex}
\begin{ex}%Câu 26.
Cho hình chóp $S.ABCD$ có đáy $ABCD$ là hình vuông cạnh $a$, cạnh bên $SA$ vuông góc với đáy, biết $SB=a \sqrt{3}$. Khi đó mặt cầu tâm $A$ tiếp xúc với mặt phẳng $(SBD)$ có bán kính $R$ là
\choice
{$R=a$}
{$R=a \dfrac{2\sqrt{5}}{5}$}
{\True $R=a \sqrt{\dfrac{2}{5}}$}
{$R=a \dfrac{2}{\sqrt{5}}$}

\end{ex}
\begin{ex}%Câu 27.
Cho hai đường thẳng $d$ và $\Delta$ cắt nhau nhưng không vuông góc nhau. Mặt tròn xoay sinh bởi đường thẳng $d$ khi quay quanh $\Delta$ là
\choice
{$\text{Mặt cầu}$}
{$\text{Mặt trụ}$}
{\True $\text{Mặt nón}$}
{$\text{Mặt phẳng}$}

\end{ex}
\begin{ex}%Câu 28.
Trong không gian $O x y z$, vị trí tương đối giữa hai đường thẳng 
$(d_1):\heva{&x=1+2t\\&y=-4-3t\\&z=3+2t}$ và $(d_2):\dfrac{x-5}{3}=\dfrac{y+1}{2}=\dfrac{z-2}{-3}$ là
\choice
{$\text{Cắt nhau}$}
{$\text{Song song}$}
{\True $\text{Chéo nhau}$}
{$\text{Trùng nhau}$}

\end{ex}
\begin{ex}%29
Cho số phức $z=4-3i$, khi đó $|z|$ bằng
\choice
{$\sqrt{7}$}
{$25$}
{$7$}
{\True $4$}
\end{ex}
\begin{ex}%30
\immini{
Cho hàm số $y=f(x)$ xác định trên $\mathbb{R} \setminus \left\{1\right\}$, liên tục trên các khoảng xác định của nó và có bảng biến thiên như hình vẽ bên. Tổng số đường tiệm cận đứng và tiệm cận ngang của đồ thị hàm số đã cho là 
}
{
  	\begin{tikzpicture}[scale=1,line width=.6pt,color=\mauchinh]
\tkzTabInit[nocadre=true,lgt=1.5,espcl=1.6,deltacl=0.5,lw=1]
{$x$ /.7,$f'(x)$/.7,$f(x)$/1.8}{$-\infty$,$-1$,$1$,$+\infty$}
\tkzTabLine{,+,d,+,0,-,}
\tkzTabVar{-/$2$,+D-/$4$/$-\infty$,+/$3$,-/$-1$}
\end{tikzpicture}}
\choice
{\True $3$}
{$1$}
{$0$}
{$2$}

\end{ex}
\begin{ex}%Câu 31.
Trong không gian $O x y z$, hình chiếu của điểm $M(-5; 2; 7)$ trên mặt phẳng tọa độ $O x y$ là điểm $H(a; b; c)$. Khi đó giá trị $a+10 b+5 c$ bằng
\choice
{$0$}
{$35$}
{\True $15$}
{$50$}

\end{ex}
\begin{ex}%Câu 32.
Cho hàm số $y=f(x)$ liên tục trên $\mathbb{R}$ và có bảng biến thiên như hình vẽ bên dưới:
\begin{center}
\begin{tikzpicture}[color=\mauchinh]
\tkzTabInit[nocadre,lgt=1.2,espcl=2.3,deltacl=0.6,lw=1]
{$x$/0.6,$f'(x)$/0.6,$f(x)$/2}{$-\infty$,$1$,$3$,$+\infty$}
\tkzTabLine{,+,0,-,0,+,}
\tkzTabVar{-/$-\infty$,+/$2$,-/$-4$,+/$+\infty$}
\end{tikzpicture}
\end{center}

Hàm số $y=f(x)$ nghịch biến trên khoảng nào dưới đây?
\choice
{\True $(1; 2)$}
{$(4;+\infty)$}
{$(2; 4)$}
{$(-\infty;-1)$}

\end{ex}
\begin{ex}%Câu 33. 
$\displaystyle\int \dfrac{1}{x} \mathrm{\,d} x$ bằng
\choice
{$\dfrac{1}{x^2}+C$}
{$-\dfrac{1}{r^2}+C$}
{\True $\ln |x|+C$}
{$\ln x+C$}

\end{ex}
\begin{ex}%Câu 34.
Trong không gian $O x y z$, mặt phẳng $(P)$ qua điểm $M(2;-1; 3)$ và nhận vectơ pháp tuyến $\vec{n}=(1; 1;-2)$, có phương trình là
\choice
{$2 x-y+3 z+5=0$}
{$x-y-2 z+5=0$}
{$x+y-2 z-5=0$}
{\True $x+y-2 z+5=0$}

\end{ex}
\begin{ex}%Câu 35.
Trong không gian $O x y z$, mặt cầu $(S)$ có phương trình $x^2+y^2+ z^2+2 x-8 y+4 z-4=0$. Bán kính của mặt cầu $(S)$ bằng
\choice
{$\sqrt{5}$}
{$25$}
{\True $5$}
{$\sqrt{17}$}

\end{ex}

\begin{ex}%Câu 37.
Cho hàm số $y=f(x)$ có đạo hàm liên tục trên $\mathbb{R}$ và có bảng biến thiên như hình vẽ bên dưới: 
\begin{center}
\begin{tikzpicture}[color=\mauchinh]
\tkzTabInit[nocadre,lgt=1.2,espcl=2,deltacl=0.6,lw=1]
{$x$/0.6,$f'(x)$/0.6,$f(x)$/2}{$-\infty$,$-1$,$0$,$1$,$+\infty$}
\tkzTabLine{,-,0,+,0,-,0,+,}
\tkzTabVar{+/$+\infty$,-/$-1$,+/$2$,-/$-1$,+/$+\infty$}
\end{tikzpicture}
\end{center}
Giá trị cực tiểu của hàm số bằng
\choice
{\True $-1$}
{$2$}
{$0$}
{$1$}

\end{ex}
\begin{ex}%Câu 38.
Cho hàm số $y=f(x)$ có đạo hàm liên tục trên $\mathbb{R}$ và có bảng biến thiên như hình vẽ bên dưới:
\begin{center}
\begin{tikzpicture}[color=\mauchinh]
\tkzTabInit[nocadre,lgt=1.2,espcl=2,deltacl=0.6,lw=1]
{$x$/0.6,$f'(x)$/0.6,$f(x)$/2}{$-\infty$,$-1$,$0$,$1$,$+\infty$}
\tkzTabLine{,-,0,+,0,-,0,+,}
\tkzTabVar{+/$+\infty$,-/$-1$,+/$2$,-/$-1$,+/$+\infty$}
\end{tikzpicture}
\end{center}
Hàm số nghịch biến trên khoảng nào trong các khoảng dưới đây?
\choice
{\True $\left(0; \dfrac{1}{2}\right)$}
{$(1;+\infty)$}
{$(0;+\infty)$}
{$(-\infty; 0)$}

\end{ex}
\begin{ex}%Câu 39.
Cho $a$ là một số thực dương, khác $1$. Khi đó, $\log_a a^3$ bằng
\choice
{$a^3$}
{\True $3$}
{$\dfrac{1}{3}$}
{$a$}

\end{ex}
\begin{ex}%Câu 40.
Khối bát diện đều cạnh $a$ có thể tích bằng
\choice
{\True $\dfrac{a^3 \sqrt{2}}{3}$}
{$\dfrac{2 a^3 \sqrt{2}}{3}$}
{$a^3$}
{$\dfrac{2 a^3}{3}$}

\end{ex}
\begin{ex}%Câu 41.
Tập xác định $\mathscr{D}$ của hàm số $y=\left(x^2-x\right)^{\sqrt{3}}$ là
\choice
{\True $\mathscr{D}=(-\infty; 0) \cup(1;+\infty)$}
{$\mathscr{D}=\mathbb{R}$}
{$\mathscr{D}=(-\infty; 0] \cup[1;+\infty)$}
{$\mathscr{D}=\mathbb{R} \backslash\{0; 1\}$}

\end{ex}
\begin{ex}%Câu 42.
Trong không gian $O x y z$, mặt phẳng $(P)$ chứa hai đường thẳng $d_1\colon \dfrac{x-2}{2}=\dfrac{y+3}{-1}=\dfrac{z-5}{-3}$ và $d_2\colon \dfrac{x+1}{-2}=\dfrac{y+3}{1}=\dfrac{z-2}{3}$. Khi đó phương trình mặt phẳng $(P)$ là
\choice
{$x-5 y+z-22=0$}
{$x-5 y-z+18=0$}
{$x+3 y-z+12=0$}
{\True $x+5 y-z+18=0$}

\end{ex}
\begin{ex}%Câu 43.
Biết hàm số $y=f(x)$ liên tục và có đạo hàm trên $[0; 2], f(0)= \sqrt{5}, f(2)=\sqrt{11}$. Tích phân $I=\displaystyle\int\limits_0^2 f(x) \cdot f'(x) \mathrm{d} x$ bằng
\choice
{$\sqrt{5}-\sqrt{11}$}
{\True $3$}
{$\sqrt{11}-\sqrt{5}$}
{$6$}

\end{ex}
\begin{ex}%Câu 44.
Cho số phức $z=a+b i,(a, b \in \mathbb{R})$ thỏa mãn $z-2\bar{z}=-1+6 i$. Giá trị $a+b$ bằng
\choice
{\True $3$}
{$-3$}
{$2$}
{$-1$}

\end{ex}
\begin{ex}%Câu 45.
Cho hình phẳng $(D)$ giới hạn bởi các đường $y=\sin x; y=0; x=  0; x=\pi$. Thể tích khối tròn xoay sinh bởi hình $(D)$ quay xung quanh $O x$ bằng
\choice
{$\dfrac{\pi^2}{1000}$}
{$\dfrac{\pi}{1000}$}
{$\dfrac{\pi}{2}$}
{\True $\dfrac{\pi^2}{2}$}

\end{ex}
\begin{ex}%Câu 46.
Cho hàm số $y=f(x)$ có đạo hàm $f'(x)=x^2(x-1)(x+3)(2- x), \forall x \in \mathbb{R}$. Số điểm cực trị của hàm số đã cho là
\choice
{$1$}
{\True $3$}
{$2$}
{$4$}

\end{ex}
\begin{ex}%Câu 47.
Khối nón có chiều cao bằng bán kính đáy và có thể tích bằng $9\pi$, chiều cao của khối nón đó bằng
\choice
{\True $3$}
{$3\sqrt{3}$}
{$\sqrt[3]{9}$}
{$\sqrt{3}$}

\end{ex}
\begin{ex}%Câu 48.
Cho hình lăng trụ đều $ABC \cdot A'B'C'$ có $AB=a, AA'=a \sqrt{3}$. Góc giữa đường thẳng $AC'$ và mặt phẳng $(ABC)$ bằng:
\choice
{$30^{\circ}$}
{\True $60^{\circ}$}
{$90^{\circ}$}
{$45^{\circ}$}

\end{ex}
\begin{ex}%Câu 49.
Nếu $\displaystyle\int\limits_0^1\left[f^2(x)-f(x)\right] \mathrm{d} x=5$ và $\displaystyle\int\limits_0^1[f(x)+1]^2 \mathrm{\,d} x=36$ thì $\displaystyle\int\limits_0^1 f(x) \mathrm{d} x$ bằng
\choice
{\True $10$}
{$31$}
{$5$}
{$30$}

\end{ex}
\begin{ex}%Câu 50.
Trong không gian $Oxyz$, mặt cầu $(S)$ có tâm $I(-2; 5; 1)$ và tiếp xúc với mặt phẳng $(P)\colon 2 x+2 y-z+7=0$ có phương trình là:
\choice
{$(x+2)^2+(y-5)^2+(z-1)^2=\dfrac{25}{9}$}
{$(x-2)^2+(y+5)^2+(z+1)^2=16$}
{$(x+2)^2+(y-5)^2+(z-1)^2=4$}
{\True $(x+2)^2+(y-5)^2+(z-1)^2=16$}

\end{ex}

\Closesolutionfile{ans}
%\indapan{10}{ans/ans-de9-7}