
\begin{name}
	{\tenchude}
	{\tendethi}
	{\tentruong}
	{\thoigian}
\end{name}
\Opensolutionfile{ans}[ans/ans-de7-7]

\begin{ex}%Câu 42.
Phần ảo của số phức $z=3-2 i$ là
\choice
{$3$}
{$-2 i$}
{$2$}
{\True $-2$}

\end{ex}
\begin{ex}%Câu 1.
Biết $y=\log_2 x^5,(x>0)$. Khi đó
\choice
{$y=5\log x$}
{\True $y=5\log_2 x$}
{$y=5+\log_2 x$}
{$y=\dfrac{1}{5} \log_2 x$}

\end{ex}
\begin{ex}%Câu 2.
\immini{
Cho hàm số $f(x)=a x^4+b x^2+c(a \neq 0)$ có đồ thị như hình bên. Số nghiệm của phương trình $f(x)-2=0$ là:
\choice
{$1$}
{$4$}
{\True $2$}
{$3$}}
{
\vspace{-0.5cm}
\begin{tikzpicture}[scale=1, font=\footnotesize, line join=round, line cap=round, >=stealth,color=\mauchinh]
\def\xmin{-1.54}\def\xmax{1.54}\def\ymin{-0.5}\def\ymax{1.86}
\draw[->,thick] (\xmin-0.2,0)--(\xmax+0.2,0) node[below] {\footnotesize $x$};
\draw[->,thick] (0,\ymin-0.2)--(0,\ymax+0.2) node[right] {\footnotesize $y$};
\draw (0,0) node [below left] {\footnotesize $O$};
\foreach \x in {}\draw (\x,0.1)--(\x,-0.1) node [below] {\footnotesize $\x$};
\foreach \y in {1}\draw (0.1,\y)--(-0.1,\y) node [left] {\footnotesize $\y$};
\clip (\xmin,\ymin) rectangle (\xmax,\ymax);
\draw[thick,smooth,samples=200,domain=\xmin:\xmax] plot (\x,{1*((\x)^4)+-2*((\x)^2)+1});
\end{tikzpicture}}

\end{ex}
\begin{ex}%Câu 3.
Trong không gian với hệ tọa độ $O x y z$, cho mặt cầu $(S)\colon(x+1)^2+$ $(y+2)^2+(z-1)^2=4$. Tọa độ tâm $I$ và bán kính $R$ của mặt cầu là:
\choice
{$I=(-1;-2; 1); R=4$}
{$I=(1; 2;-1); R=2$}
{\True $I=(-1;-2; 1); R=2$}
{$I=(1; 2;-1); R=4$}

\end{ex}
\begin{ex}%Câu 4.
Trong không gian với hệ tọa độ $O x y z$, cho đường thẳng $d\colon \dfrac{x+2}{-3}=$ $\dfrac{y-3}{2}=\dfrac{z}{1}$. Vec-tơ chỉ phương của đường thẳng $d$ có tọa độ là
\choice
{$(-2; 3; 0)$}
{\True $(-3; 2; 1)$}
{$(-3; 2;-1)$}
{$(3; 2; 1)$}

\end{ex}
\begin{ex}%Câu 5.
Cho hai số phức $z_1=2-3 i, z_2=-3+7 i$. Khi đó số phức $z_1-z_2$ bằng
\choice
{\True $5-10 i$}
{$-5+10 i$}
{$5+4 i$}
{$-5+4 i$}

\end{ex}
\begin{ex}%Câu 6.
Trong không gian với hệ tọa độ $O x y z$, cho hai điểm $A(2; 0; 5), B(1; 2; 3)$.
Phương trình mặt phẳng $(P)$ qua $A$ và vuông góc với $AB$ là
\choice
{$x-2 y+2 z-3=0$}
{\True $x-2 y+2 z-12=0$}
{$x+2 y+2 z+11=0$}
{$x+2 y+2 z-11=0$}

\end{ex}
\begin{ex}%Câu 7.
Cho số thực $x, y$ thỏa mãn $(2-3 i) x+(3+2 y) i=2-2 i$ là:
\choice
{$x=-1, y=-1$}
{$x=-1, y=1$}
{$x=1, y=1$}
{\True $x=1, y=-1$}

\end{ex}
\begin{ex}%Câu 8.
Trong không gian với hệ tọa độ $O x y z$, cho tam giác $ABC$ có trọng tâm $G(2; 1; 0)$ và $A(1; 1; 0), B(2; 3; 5)$. Tọa độ điểm $C$ là
\choice
{$(3;-1;-5)$}
{$(-12; 0; 8)$}
{$(4; 2;-1)$}
{\True $(-6;-2; 0)$}

\end{ex}
\begin{ex}%Câu 9.
Tập nghiệm của bất phương trình $\log_{\frac{\pi}{3}}(x+2)<0$ là
\choice
{$(-1;+\infty)$}
{\True $(-2;-1)$}
{$(-\infty;-1)$}
{$(-2;+\infty)$}

\end{ex}
\begin{ex}%Câu 10.
Thể tích của khối nón có chiều cao $h$, bán kính đáy $r$ bằng
\choice
{$\dfrac{1}{3} \pi r h^2$}
{$\dfrac{1}{3} \pi r h$}
{\True $\dfrac{1}{3} \pi r^2 h$}
{$\pi r^2 h$}

\end{ex}
\begin{ex}%Câu 11.
Tất cả các giá trị thực của tham số $m$ để hàm số $y=2 x^3+3 m x^2+$ $2 m x-5$ không có cực trị là
\choice
{\True $0\leq m \leq \dfrac{4}{3}$}
{$0<m<\dfrac{4}{3}$}
{$-\dfrac{4}{3}<m<0$}
{$-\dfrac{4}{3} \leq m \leq 0$}

\end{ex}
\begin{ex}%Câu 12.
Xét $\displaystyle\displaystyle\int\limits_{\frac{1}{{\rm e}}}^{{\rm e}}\left(\dfrac{\ln x}{x}\right) \mathrm{d} x$, nếu đặt $t=\ln x$ thì $\displaystyle\displaystyle\int\limits_{\frac{1}{{\rm e}}}^{{\rm e}}\left(\dfrac{\ln x}{x}\right) \mathrm{d} x$ bằng
\choice
{\True $\displaystyle\displaystyle\int\limits_{-1}^1 t \mathrm{\,d} t$}
{$\displaystyle\displaystyle\int\limits_{-1}^1 \dfrac{1}{t} \mathrm{\,d} t$}
{$\displaystyle\displaystyle\int\limits_{-1}^1 \mathrm{\,d} t$}
{$\displaystyle\displaystyle\int\limits_{-1}^1 \dfrac{1}{t^2} \mathrm{\,d} t$}

\end{ex}
\begin{ex}%Câu 13.
Cho hình chóp tứ giác đều $S.ABCD$ với $O$ là tâm của đáy, $AB=$ $a, SO=\dfrac{a \sqrt{6}}{2}$. Góc giữa cạnh $SB$ và mặt phẳng $(ABCD)$ bằng
\choice
{\True $60^{\circ}$}
{$45^{\circ}$}
{$90^{\circ}$}
{$30^{\circ}$}

\end{ex}
\begin{ex}%Câu 14.
Cho hình chóp $S.ABCD$ có đáy $ABCD$ là hình vuông cạnh $a$. Biết cạnh bên $SA=a, SA \perp(ABCD)$. Thể tích của khối chóp $S.ABCD$ bằng
\choice
{$a^3$}
{$\dfrac{9 a^3}{3}$}
{\True $\dfrac{a^3}{3}$}
{$3 a^3$}

\end{ex}
\begin{ex}%Câu 15.
Biết $\log_2 x=6\log_4 a-3\log_2 \sqrt[3]{b}-\log_{\frac{1}{2}} c$, với $a, b, c$ là các số thực dương bất kì. Mệnh đề nào dưới đây đúng?
\choice
{\True $x=\dfrac{a^3 c}{b}$}
{$x=\dfrac{a^3}{b c}$}
{$x=\dfrac{a^3 c}{b^2}$}
{$a^3-b+c$}

\end{ex}
\begin{ex}%Câu 16.
Số giao điểm của đồ thị hàm số $y=x^4-4 x^2+1$ với trục hoành là
\choice
{$1$}
{$3$}
{$2$}
{\True $4$}

\end{ex}
\begin{ex}%Câu 17.
Cho hình chóp $S.ABC$ có tam giác $ABC$ vuông tại $B, SA$ vuông góc với mặt phẳng $(ABC), SA=2, AB=1, BC=\sqrt{3}$. Bán kính $R$ mặt cầu ngoại tiếp hình chóp $S.ABC$ bằng
\choice
{$1$}
{$2\sqrt{2}$}
{\True $\sqrt{2}$}
{$2$}

\end{ex}
\begin{ex}%Câu 18.
Nghiệm của phương trình $3^{x-1}=9$ là
\choice
{$2$}
{$0$}
{\True $3$}
{$1$}

\end{ex}
\begin{ex}%Câu 19.
Trong không gian với hệ tọa độ $O x y z$, mặt cầu $(S)$ có tâm $I(1;-2; 3)$ tiếp xúc với mặt phẳng $(P)\colon x-2 z-5=0$ có phương trình là:
\choice
{$(S)\colon(x-1)^2+(y+2)^2+(z-3)^2=100$}
{$(S)\colon(x+1)^2+(y-2)^2+(z+3)^2=4$}
{\True $(S)\colon(x-1)^2+(y+2)^2+(z-3)^2=20$}
{$(S)\colon(x+1)^2+(y-2)^2+(z+3)^2=20$}

\end{ex}
\begin{ex}%Câu 20.
Cho hàm số $y=\dfrac{x+1}{x^2-4 x-5}$. Số đường tiệm cận đứng của đồ thị hàm số là
\choice
{\True $1$}
{$4$}
{$2$}
{$3$}

\end{ex}
\begin{ex}%Câu 21.
Từ một nhóm học sinh gồm $6$ nam và $8$ nữ, có bao nhiêu cách chọn ra $1$ nam và $1$ nữ?
\choice
{$14$}
{\True $48$}
{$6$}
{$8$}

\end{ex}
\begin{ex}%Câu 22.
Cho cáp số nhân $\left(u_n\right)$ với $u_1=2$ và $u_4=54$. Công bội của cấp số nhân đã cho bằng
\choice
{$6$}
{\True $3$}
{$\dfrac{54}{2}$}
{$-3$}

\end{ex}
\begin{ex}%Câu 23.
Nghiệm của phương trình $\log (4 x-6)=1$ là
\choice
{\True $x=4$}
{$x=5$}
{$x=\dfrac{9}{2}$}
{$x=\dfrac{7}{4}$}

\end{ex}
\begin{ex}%Câu 24.
Cho khối hộp chữ nhật có các kích thước là $2,3,4$. Thể tích của khối hộp đã cho bằng
\choice
{$9$}
{$10$}
{\True $24$}
{$72$}

\end{ex}
\begin{ex}%Câu 25.
Tập xác định của hàm số $y=\log_2(x-3)$ là:
\choice
{$[3;+\infty)$}
{$(-\infty;+\infty)$}
{\True $(3;+\infty)$}
{$(0;+\infty)$}

\end{ex}
\begin{ex}%Câu 26.
Họ tất cả các nguyên hàm của hàm số $f(x)={\rm e}^{2 x}-3 x^2$ là:
\choice
{${\rm e}^{x}+x^3+C$}
{${\rm e}^{2 x}-x^3+C$}
{\True $\dfrac{1}{2} {\rm e}^{2 x}-x^3+C$}
{$2 {\rm e}^{2 x}-x^3+C$}

\end{ex}
\begin{ex}%Câu 27.
Diện tích xung quanh của hình nón có độ dài đường $\sinh l$ và bán kính đáy $r$ bằng:
\choice
{$4\pi r l$}
{$2\pi r l$}
{\True $\pi r l$}
{$\dfrac{1}{3} \pi r l$}

\end{ex}
\begin{ex}%Câu 28.
Cho khối trụ có chiều cao $h=3$ và bán kính đáy $r=2$. Thể tích của khối trụ đã cho bằng:
\choice
{$6\pi$}
{\True $12\pi$}
{$18\pi$}
{$4\pi$}

\end{ex}
\begin{ex}%Câu 29.
Cho mặt cầu có bán kính $R=2$. Thể tích khối cầu đã cho bằng
\choice
{\True $\dfrac{32\pi}{3}$}
{$8\pi$}
{$16\pi$}
{$4\pi$}

\end{ex}
\begin{ex}%Câu 30.
\immini{
Cho hàm số $y=f(x)$ có bảng biến thiên như hình bên. Hàm số đã cho đồng biến trên khoảng nào dưới đây?
\choice
{$(-4;+\infty)$}
{\True $(-\infty; 0)$}
{$(-1; 3)$}
{$(0; 1)$}}
{\vspace{-0.5cm}
\begin{tikzpicture}[color=\mauchinh]
\tkzTabInit[nocadre,lgt=1.2,espcl=1.6,deltacl=0.6,lw=1]
{$x$/0.6,$f'(x)$/0.6,$f(x)$/2}{$-\infty$,$0$,$3$,$+\infty$}
\tkzTabLine{,+,0,-,0,+,}
\tkzTabVar{-/$-\infty$,+/$2$,-/$-4$,+/$+\infty$}
\end{tikzpicture}

}

\end{ex}
\begin{ex}%Câu 31.
Với $a$ là số thực dương tùy ý, $\log_2(\sqrt{a})$ bằng
\choice
{$-\dfrac{1}{2} \log_2 a$}
{$-\dfrac{1}{2}+\log_2 a$}
{$2\log_2 a$}
{\True $\dfrac{1}{2} \log_2 a$}

\end{ex}
\begin{ex}%Câu 32.
Cho khối lăng trụ có diện tích đáy $B=3$ và chiều cao $h=4$. Thể tích của khối lăng trụ đã cho bằng
\choice
{$6$}
{\True $12$}
{$36$}
{$4$}

\end{ex}
\begin{ex}%Câu 33.
\immini{
Cho hàm số $f(x)$ có bảng biến thiên như hình bên. Hàm số đã cho đạt cực tiểu tại điểm nào dưới đây?
\choice
{\True $x=0$}
{$x=1$}
{$x=-1$}
{$x=2$}}
{
\begin{tikzpicture}[color=\mauchinh]
\tkzTabInit[lgt=1.2,espcl=1.4,deltacl=0.6]
{$x$/0.6,$f'(x)$/0.6,$f(x)$/2}{$-\infty$,$-1$,$0$,$1$,$+\infty$}
\tkzTabLine{,+,0,-,0,+,0,-,}
\tkzTabVar{-/$-\infty$,+/$2$,-/$-1$,+/$2$,-/$-\infty$}
\end{tikzpicture}
}

\end{ex}
\begin{ex}%Câu 34.
\immini{
Đồ thị của hàm số nào dưới đây có dạng như đường cong trong hình vẽ bên?
\choice
{$y=-x^3+3 x^2+1$}
{$y=x^4-2 x^2+1$}
{$y=x^3-3 x^2+1$}
{\True $y=-x^4+2 x^2+1$}}
{\vspace{-0.5cm}
\begin{tikzpicture}[scale=1, font=\footnotesize, line join=round, line cap=round, >=stealth,color=\mauchinh,y=0.8cm]
\def\xmin{-1.65}\def\xmax{1.65}\def\ymin{-0.99}\def\ymax{2.2}
\draw[->,thick] (\xmin-0.2,0)--(\xmax+0.2,0) node[below] {\footnotesize $x$};
\draw[->,thick] (0,\ymin-0.2)--(0,\ymax+0.2) node[right] {\footnotesize $y$};
\draw (0,0) node [below left] {\footnotesize $O$};
\foreach \x in {}\draw (\x,0.1)--(\x,-0.1) node [below] {\footnotesize $\x$};
\foreach \y in {}\draw (0.1,\y)--(-0.1,\y) node [left] {\footnotesize $\y$};
\clip (\xmin,\ymin) rectangle (\xmax,\ymax);
\draw[thick,smooth,samples=200,domain=\xmin:\xmax] plot (\x,{-1*((\x)^4)+2*((\x)^2)+1});
\end{tikzpicture}
}

\end{ex}
\begin{ex}%Câu 35.
\immini{
Cho hàm số $y=f(x)$ có bảng biến thiên như hình bên. Đồ thị hàm số có bao nhiêu đường tiệm cận ngang?
\choice
{\True $2$}
{$3$}
{$4$}
{$1$}}
{

\vspace{-0.5cm}
  	\begin{tikzpicture}[scale=1,line width=.6pt,color=\mauchinh]
\tkzTabInit[nocadre=true,lgt=1.6,espcl=1.3,deltacl=0.5,lw=0.8]
{$x$ /.7,$f'(x)$/.7,$f(x)$/1.8}{$-\infty$,$0$,$2$,$4$,$+\infty$}
\tkzTabLine{,+,d,-,0,+,d,+,}
\tkzTabVar{-/$-3$,+D+/$1$/$+\infty$,-/$-2$,+D-/$+\infty$/$-\infty$,+/$3$}
\end{tikzpicture}
}

\end{ex}
\begin{ex}%Câu 36.
Tập nghiệm của bất phương trình $\log (x-1) \leq 1$ là:
\choice
{$(1; 11)$}
{\True $(1; 11]$}
{$[1; 11]$}
{$(-\infty; 11]$}

\end{ex}
\begin{ex}%Câu 37.
\immini{
Cho hàm số bậc ba $y=f(x)$ có bảng biến thiên như hình bên. Số nghiệm của phương trình $2 f(x)-1=0$ là
\choice
{$0$}
{$1$}
{$2$}
{\True $3$}}
{\vspace{-0.5cm}
\begin{tikzpicture}[color=\mauchinh]
\tkzTabInit[nocadre,lgt=1.2,espcl=1.5,deltacl=0.5,lw=1]
{$x$/0.6,$f'(x)$/0.6,$f(x)$/2}{$-\infty$,$-1$,$2$,$+\infty$}
\tkzTabLine{,+,0,-,0,+,}
\tkzTabVar{-/$-\infty$,+/$1$,-/$-2$,+/$+\infty$}
\end{tikzpicture}

}

\end{ex}
\begin{ex}%Câu 38.
Biết $\displaystyle\displaystyle\int\limits_2^3 f(x) \mathrm{d} x=5$. Khi đó $\displaystyle\displaystyle\int\limits_2^3[3-5 f(x)] \mathrm{d} x$ bằng
\choice
{\True $-22$}
{$-28$}
{$-26$}
{$-15$}

\end{ex}
\begin{ex}%Câu 39.
Số phức liên hợp của $3+i$ bằng
\choice
{\True $3-i$}
{$i-3$}
{$-3-i$}
{$3 i$}

\end{ex}
\begin{ex}%Câu 40.
Cho hai số phức $z_1=2-i$ và $z_2=1+i$. Môđun của số phức $z_1+z_2$ bằng
\choice
{$\sqrt{5}+\sqrt{2}$}
{\True $3$}
{$\sqrt{13}$}
{$\sqrt{5}$}

\end{ex}
\begin{ex}%Câu 41.
Trong không gian $O x y z$, hình chiếu vuông góc của điểm\\ $M(2;-2; 1)$ trên mặt phẳng $(0 x y)$ có tọa độ là
\choice
{$(2; 0; 1)$}
{\True $(2;-2; 0)$}
{$(0;-2; 1)$}
{$(0; 0; 1)$}

\end{ex}

\begin{ex}%Câu 43.
Trong không gian $O x y z$, cho mặt phẳng $(\alpha)\colon 3 x-4 z+1=0$. Vectơ nào dưới đây là vectơ pháp tuyến của $(\alpha)$?
\choice
{$\vec{n}_1=(3;-4; 1)$}
{$\overrightarrow{n_3}=(3;-4; 0)$}
{$\overrightarrow{n_2}=(0; 3;-4)$}
{\True $\vec{n}_4=(3; 0;-4)$}

\end{ex}
\begin{ex}%Câu 44.
Trong KG $Oxyz$, cho mặt cầu $(S)$ có phương trình $x^2+$ $y^2+z^2-2 x+4 y-6=0$. Tâm của $(S)$ có tọa độ là
\choice
{\True $(1;-2; 0)$}
{$(-1; 2; 0)$}
{$(2;-4; 0)$}
{$(1;-2; 3)$}

\end{ex}
\begin{ex}%Câu 45.
Trong không gian $O x y z$, đường thẳng đi qua hai điểm $A(1; 0; 1)$ và $B(-1; 2; 2)$ có một vec tơ chỉ phương là
\choice
{\True $\vec{n}_1=(2;-2;-1)$}
{$\vec{n}_2=(0; 2; 2)$}
{$\vec{n}_3=(2; 2;-1)$}
{$\vec{n}_4=(2;-2; 1)$}

\end{ex}
\begin{ex}%Câu 46.
Cho hình chóp $S.ABCD$ có đáy là hình vuông có đường chéo bằng $a \sqrt{2}, SA$ vuông góc với mặt phẳng đáy và $SA=a \sqrt{3}$. Góc giữa mặt phẳng $(SBC)$ và mặt $(ABCD)$ bằng
\choice
{$45^{\circ}$}
{$30^{\circ}$}
{\True $60^{\circ}$}
{$90^{\circ}$}

\end{ex}
\begin{ex}%Câu 47.
\immini{
Cho hàm số $f(x)$, bảng xét dấu của $f'(x)$ như sau:
Hàm số đạt cực tiểu tại điểm nào?
}
{
  	\begin{tikzpicture}[color=\mauchinh]
\tkzTabInit[lgt=1,espcl=1.4,deltacl=0.5]
{$x$ /.7,$y'$/.7}{$-\infty$,$-1$,$0$,$1$,$+\infty$}
\tkzTabLine{,+,0,-,0,-,0,+,}
\end{tikzpicture}}
\choice
{$x=0$}
{$x=-1$}
{\True $x=1$}
{$x=2$}
\end{ex}
\begin{ex}%Câu 48.
Giá trị lớn nhất của hàm số $f(x)=-x^3+12 x^2+1$ trên đoạn $[-1; 1]$ bằng:
\choice
{\True $14$}
{$10$}
{$1$}
{$12$}

\end{ex}
\begin{ex}%Câu 49.
Xét tất cả các số dương $a$ và $b$ thỏa mãn $\log_2 a=\log_8\left(\dfrac{a}{b}\right)$. Mệnh đề nào dưới đây đúng
\choice
{$a=b^2$}
{$a^3 b=1$}
{$3 b=1$}
{\True $a^2 b=1$}

\end{ex}
\begin{ex}%Câu 50.
Tập nghiệm của bất phương trình $\log_2(8 x)\left(\log_2 x-1\right)<0$ là khoảng $(a; b)$. Tính $S=a+b$ 
\choice
{\True $S=\dfrac{17}{8}$}
{$S=-2$}
{$S=2$}
{$S=10$}

\end{ex}

\Closesolutionfile{ans}
%\indapan{10}{ans/ans-de7-7}
