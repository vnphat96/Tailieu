
\begin{name}
	{\tenchude}
	{\tendethi}
	{\tentruong}
	{\thoigian}
\end{name}
\Opensolutionfile{ans}[ans/ans-de16-7]

\begin{ex}%Câu 35.
Cho số phức $z$ thỏa mãn $i z=4+3 i$. Số phức liên hợp của $z$ là:
\choice
{\True $\bar{z}=3+4 i$}
{$\bar{z}=-3-4 i$}
{$\bar{z}=3-4 i$}
{$\bar{z}=-3+4 i$}

\end{ex}
\begin{ex}%Câu 1.
Cho số phức $z=-3+2 i$, số phức $(1-i) \bar{z}$ bằng
\choice
{$-1-5 i$}
{\True $-5+i$}
{$1-5 i$}
{$5-i$}

\end{ex}
\begin{ex}%Câu 2.
Cho hai số phức $z=3+2 i$ và $w=1-4 i$. Số phức $z+w$ bằng
\choice
{$4+2 i$}
{\True $4-2 i$}
{$-2-6 i$}
{$2+6 i$}

\end{ex}
\begin{ex}%Câu 3.
\immini{
Đồ thị của hàm số nào dưới đây có dạng như đường cong trong hình bên?
\choice
{$y=x^3-3 x+1$}
{$y=x^4+4 x^2+1$}
{\True $y=-x^3+3 x+1$}
{$y=-x^4+2 x^2+1$}}
{\vspace{-0.5cm}
\begin{tikzpicture}[scale=1, font=\footnotesize, line join=round, line cap=round, >=stealth,y=0.7cm,color=\mauchinh]
\def\xmin{-2.03}\def\xmax{2.08}\def\ymin{-1.1}\def\ymax{3.25}
\draw[->,thick] (\xmin-0.2,0)--(\xmax+0.2,0) node[below] {\footnotesize $x$};
\draw[->,thick] (0,\ymin-0.2)--(0,\ymax+0.2) node[right] {\footnotesize $y$};
\draw (0,0) node [below left] {\footnotesize $O$};
\foreach \x in {1,2}\draw (\x,0.1)--(\x,-0.1) node [below] {\footnotesize $\x$};
\foreach \y in {3}\draw (0.1,\y)--(-0.1,\y) node [left] {\footnotesize $\y$};
\clip (\xmin,\ymin) rectangle (\xmax,\ymax);
\draw[thick,smooth,samples=200,domain=\xmin:\xmax] plot (\x,{-1*((\x)^3)+0*((\x)^2)+3*(\x)+1});
\draw[dashed] (-1,0)--(-1,-1)--(0,-1);\fill (-1,-1) circle (1pt);
\draw[dashed] (1,0)--(1,3)--(0,3);\fill (1,3) circle (1pt);
\draw(-1,0)node[above]{$-1$} (0,-1)node[right]{$-1$};
\end{tikzpicture}
}

\end{ex}
\begin{ex}%Câu 4.
Nếu $\displaystyle\int\limits_1^4 f(x) \mathrm{d} x=4$ và $\displaystyle\int\limits_1^4 g(x) \mathrm{d} x=-3$ thì $\displaystyle\int\limits_1^4[f(x)-g(x)] \mathrm{d} x$ 
\choice
{$1$}
{$-7$}
{$-1$}
{\True $7$}

\end{ex}
\begin{ex}%Câu 5.
Tiệm cận đứng của đồ thị hàm số $y=\dfrac{x-1}{x+2}$ là đường thẳng có phương trình:
\choice
{$x=2$}
{$x=-1$}
{\True $x=-2$}
{$x=1$}

\end{ex}
\begin{ex}%Câu 6.
Trong không gian $O x y z$, cho mặt cầu $(S)$ có tâm $I(-1; 3; 0)$ và bán kính bằng $2$. Phương trình của mặt cầu $(S)$ là:
\choice
{$(x-1)^2+(y+3)^2+z^2=2$}
{$(x-1)^2+(y+3)^2+z^2=4$}
{\True $(x+1)^2+(y-3)^2+z^2=4$}
{$(x+1)^2+(y-3)^2+z^2=2$}

\end{ex}
\begin{ex}%Câu 7.
Tập nghiệm của bất phương trình $2^{x}>5$ là
\choice
{$\left(-\infty; \log_2 5\right)$}
{$\left(\log_5 2;+\infty\right)$}
{$\left(-\infty; \log_5 2\right)$}
{\True $\left(\log_2 5;+\infty\right)$}

\end{ex}
\begin{ex}%Câu 8.
Thể tích của khối lập phương cạnh $2a$ bằng
\choice
{$a^3$}
{$2 a^3$}
{\True $8 a^3$}
{$4 a^3$}

\end{ex}
\begin{ex}%Câu 9.
Trên khoảng $(0;+\infty)$, đạo hàm của hàm số $y=x^{\frac{5}{3}}$ là:
\choice
{$y=\dfrac{3}{8} x^{\frac{5}{3}}$}
{\True $y=\dfrac{5}{3} x^{\frac{2}{3}}$}
{$y=\dfrac{5}{3} x^{-\frac{2}{3}}$}
{$y=\dfrac{3}{5} x^{\frac{2}{3}}$}

\end{ex}
\begin{ex}%Câu 10.
Trong không gian $O x y z$, cho điểm $A(2;-1; 4)$. Tọa độ của véctơ $\overrightarrow{OA}$ là
\choice
{$(-2; 1; 4)$}
{\True $(2;-1; 4)$}
{$(2; 1; 4)$}
{$(-2; 1;-4)$}

\end{ex}
\begin{ex}%Câu 11.
Nếu $\displaystyle\int\limits_0^3 f(x) \mathrm{d} x=3$ thì $\displaystyle\int\limits_0^3 4 f(x) \mathrm{d} x$ bằng
\choice
{$3$}
{\True $12$}
{$36$}
{$4$}

\end{ex}
\begin{ex}%Câu 12.
Cho cấp số nhân $\left(u_n\right)$ với $u_1=2$ và $u_2=10$. Công bội của cấp số nhân đã cho bằng
\choice
{$-8$}
{$8$}
{\True $5$}
{$\dfrac{1}{5}$}

\end{ex}
\begin{ex}%Câu 13.
Với $n$ là số nguyên dương bất kì, $n \geq 3$, công thức nào dưới đây đúng?
\choice
{$A_n^3=\dfrac{(n-3) !}{n !}$}
{$A_n^3=\dfrac{3!}{(n-3) !}$}
{\True $A_n^3=\dfrac{n !}{(n-3) !}$}
{$A_n^3=\dfrac{n !}{3!(n-3) !}$}

\end{ex}
\begin{ex}%Câu 14.
Cho hàm số $f(x)=x^2+2$. Khẳng định nào dưới đây đúng?
\choice
{$\displaystyle\int f(x) \mathrm{d} x=2 x+C$}
{\True $\displaystyle\int f(x) \mathrm{d} x=\dfrac{x^3}{3}+2 x+C$}
{$\displaystyle\int f(x) \mathrm{d} x=x^2+2 x+C$}
{$\displaystyle\int f(x) \mathrm{d} x=x^3+2 x+C$}

\end{ex}
\begin{ex}%Câu 15.
\immini{
Cho hàm số $y=f(x)$ có bảng biến thiên như hình bên dưới. Giá trị cực tiểu của hàm số đã cho bằng
}
{\vspace{-0.5cm}
\begin{tikzpicture}[color=\mauchinh]
\tkzTabInit[nocadre,lgt=1.2,espcl=1.4,deltacl=0.6]
{$x$/0.6,$f'(x)$/0.6,$f(x)$/1.8}{$-\infty$,$-1$,$0$,$1$,$+\infty$}
\tkzTabLine{,+,0,-,0,+,0,-,}
\tkzTabVar{-/$-\infty$,+/$3$,-/$1$,+/$3$,-/$-\infty$}
\end{tikzpicture}
}
\choice
{$0$}
{$3$}
{\True $1$}
{$-1$}
\end{ex}
\begin{ex}%Câu 16.
Trong không gian $O x y z$, cho mặt phẳng $(P)\colon 2 x+4 y-z-1=0$. Vectơ nào dưới đây là một vectơ pháp tuyến của $(P)$?
\choice
{$\overrightarrow{n_2}=(2;-4; 1)$}
{$\overrightarrow{n_1}=(2; 4; 1)$}
{\True $\overrightarrow{n_3}=(2; 4;-1)$}
{$\overrightarrow{n_4}=(-2; 4; 1)$}

\end{ex}
\begin{ex}%Câu 17.
Phần thực của số phức $z=4-2 i$ bằng
\choice
{$2$}
{$-4$}
{\True $4$}
{$-2$}

\end{ex}
\begin{ex}%Câu 18.
Nghiệm của phương trình $\log_2(5 x)=3$ là
\choice
{\True $x=\dfrac{8}{5}$}
{$x=\dfrac{9}{5}$}
{$x=8$}
{$x=9$}

\end{ex}
\begin{ex}%Câu 19.
Tập xác định của hàm số $y=8^{x}$ là
\choice
{$\mathbb{R} \backslash\{0\}$}
{\True $\mathbb{R}$}
{$[0;+\infty)$}
{$(0;+\infty)$}

\end{ex}
\begin{ex}%Câu 20.
Cho $a>0$ và $a \neq 1$, khi đó $\log_a \sqrt[5]{a}$ bằng
\choice
{\True $\dfrac{1}{5}$}
{$-\dfrac{1}{5}$}
{$5$}
{$-5$}

\end{ex}
\begin{ex}%Câu 21.
Trong không gian $O x y z$ cho đường thẳng $d$ di qua điểm\\ $M(1; 5;-2)$ và có một vecto chỉ phương $\vec{u}=(3;-6; 1)$. Phương trình của $d$ là:
\choice
{$\heva{&x=5+2 t \\& y=2+2 t \\& z=-3+t}$}
{$\heva{&x=5+2 t \\& y=2+2 t \\& z=-3+t}$}
{$\heva{&x=5+2 t \\& y=2+2 t \\& z=-3+t}$}
{\True $\heva{&x=5+2 t \\& y=2+2 t \\& z=-3+t}$}

\end{ex}
\begin{ex}%Câu 22.
Trên mặt phẳng tọa độ, điểm $M(-4; 3)$ là điểm biểu diễn của số phức nào sau đây?
\choice
{$z_3=-4-3 i$}
{$z_4=4+3 i$}
{$z_2=4-3 i$}
{\True $z_1=-4+3 i$}

\end{ex}
\begin{ex}%Câu 23.
Cho hàm số $y=f(x)$ có bảng xét dấu của đạo hàm như sau:
\immini{

Số điểm cực trị của hàm số đã cho là
}
{
 \begin{nscenter}
  	\begin{tikzpicture}[scale=1,line width=.6pt,color=\mauchinh]
\tkzTabInit[nocadre=true,lgt=1,espcl=1.4,deltacl=0.5,lw=0.8]
{$x$ /.7,$y'$/.7}{$-\infty$,$-2$,$-1$,$2$,$4$,$+\infty$}
\tkzTabLine{,+,0,-,0,+,0,-,0,+,}
\end{tikzpicture}
\end{nscenter}
}
\choice
{$3$}
{\True $4$}
{$2$}
{$5$}
\end{ex}
\begin{ex}%Câu 24.
Cho hàm số $f(x)={\rm e}^{x}+4$. Khẳng định nào sau đây đúng?
\choice
{\True $\displaystyle\int f(x) \mathrm{d} x={\rm e}^{x}+4 x+C$}
{$\displaystyle\int f(x) \mathrm{d} x={\rm e}^{x}+C$}
{$\displaystyle\int f(x) \mathrm{d} x={\rm e}^{x-4}+C$}
{$\displaystyle\int f(x) \mathrm{d} x={\rm e}^{x}-4 x+C$}

\end{ex}
\begin{ex}%Câu 25.
\immini{
Cho hàm số $y=f(x)$ có đồ thị là đường cong trong hình bên. Hàm số đã cho nghịch biến trên khoảng nào dưới đây?
\choice
{\True $(-1; 1)$}
{$(1;+\infty)$}
{$(-\infty; 1)$}
{$(0; 3)$}}
{\vspace{-0.5cm}
\begin{tikzpicture}[scale=1, font=\footnotesize, line join=round, line cap=round, >=stealth,y=0.8cm,color=\mauchinh]
\def\xmin{-2.03}\def\xmax{2.03}\def\ymin{-1.25}\def\ymax{3.16}
\draw[->,thick] (\xmin-0.2,0)--(\xmax+0.2,0) node[below] {\footnotesize $x$};
\draw[->,thick] (0,\ymin-0.2)--(0,\ymax+0.2) node[right] {\footnotesize $y$};
\draw (0,0) node [below left] {\footnotesize $O$};
\foreach \x in {-1,1}\draw (\x,0.1)--(\x,-0.1) node [below] {\footnotesize $\x$};
\foreach \y in {-1,1,3}\draw (0.1,\y)--(-0.1,\y) node [left] {\footnotesize $\y$};
\clip (\xmin,\ymin) rectangle (\xmax,\ymax);
\draw[thick,smooth,samples=200,domain=\xmin:\xmax] plot (\x,{1*((\x)^3)+0*((\x)^2)+-3*(\x)+1});
\draw[dashed] (-1,0)--(-1,3)--(0,3);\fill (-1,3) circle (1pt);
\draw[dashed] (1,0)--(1,-1)--(0,-1);\fill (1,-1) circle (1pt);
\end{tikzpicture}
}

\end{ex}
\begin{ex}%Câu 26.
Diện tích $S$ của mặt cầu bán kính $R$ được tính theo công thức nào dưới đây?
\choice
{$S=\pi R^2$}
{$S=16\pi R^2$}
{\True $S=4\pi R^2$}
{$S=\dfrac{4}{3} \pi R^2$}

\end{ex}
\begin{ex}%Câu 27.
Đồ thị hàm số $y=-2 x^3+3 x^2-5$ cắt trục tung tại điểm có tung độ bằng
\choice
{\True $-5$}
{$0$}
{$-1$}
{$2$}

\end{ex}
\begin{ex}%Câu 28.
Cho khối chóp có diện tích đáy $B=8 a^2$ và chiều cao $h=a$. Thể tích khối chóp đã cho bằng
\choice
{$8 a^3$}
{$\dfrac{4}{3} a^3$}
{$4 a^3$}
{\True $\dfrac{8}{3} a^3$}

\end{ex}
\begin{ex}%Câu 29.
Cho khối trụ có bán kính đáy $r=5$ và chiều cao $h=3$. Thể tích của khối trụ đã cho bằng
\choice
{$15\pi$}
{\True $75\pi$}
{$25\pi$}
{$45\pi$}

\end{ex}
\begin{ex}%Câu 30.
Trong không gian $O x y z$, cho điểm $M(2; 1;-2)$ và mặt phẳng $(P)$: $3 x+2 y-z+1=0$. Đường thẳng đi qua $M$ và vuông góc với $(P)$ có phương trình là:
\choice
{\True $\dfrac{x-2}{3}=\dfrac{y-1}{2}=\dfrac{z+2}{-1}$}
{$\dfrac{x-2}{3}=\dfrac{y-1}{2}=\dfrac{z+2}{1}$}
{$\dfrac{x+2}{3}=\dfrac{y+1}{2}=\dfrac{z-2}{1}$}
{$\dfrac{x+2}{3}=\dfrac{y+1}{2}=\dfrac{z-2}{-1}$}

\end{ex}
\begin{ex}%Câu 31.
Cho hình lăng trụ đứng $ABC.A'B'C'$ có tất cả các cạnh bằng nhau (tham khảo hình bên). Góc giữa hai đường thẳng $AB'$ và $CC'$ bằng
\choice
{$30^{\circ}$}
{$90^{\circ}$}
{$60^{\circ}$}
{\True $45^{\circ}$}

\end{ex}
\begin{ex}%Câu 32.
Cho hình chóp $S.ABC$ có đáy là tam giác vuông cân tại $B, AB=$ $4 a$ và $SA$ vuông góc với mặt phẳng đáy. Khoảng cách từ điểm $C$ đến mặt phẳng $(SAB)$ bằng
\choice
{\True $4 a$}
{$4\sqrt{2} a$}
{$2\sqrt{2} a$}
{$2 a$}

\end{ex}
\begin{ex}%Câu 33.
Nếu $\displaystyle\int\limits_0^2 f(x) \mathrm{d} x=4$ thì $\left.\displaystyle\int\limits_0^2[2 f(x)-1)\right] \mathrm{d} x$ bằng
\choice
{$8$}
{$10$}
{$7$}
{\True $6$}

\end{ex}
\begin{ex}%Câu 34.
Biết hàm số $y=\dfrac{x+a}{x-1}$ ($a$ là số thực cho trước và $a \neq-1$) có đồ thị như là $(C)$. Mệnh đề nào dưới đây đúng?
\choice
{$y'<0$ khi $a \leq-1$}
{\True $y'<0$ khi $a>-1$}
{$y'<0$ khi $a<-11$}
{$y'<0$ khi $a \geq 1$}

\end{ex}

\begin{ex}%Câu 36.
Từ một hộp chứa $12$ quả bóng gồm $5$ quả màu đỏ và $7$ quả màu xanh, lấy ngẩu nhiên đồng thời $3$ quả. Xác suất để lấy được $3$ quả màu đỏ bằng
\choice
{\True $\dfrac{1}{22}$}
{$\dfrac{7}{44}$}
{$\dfrac{5}{12}$}
{$\dfrac{2}{7}$}

\end{ex}
\begin{ex}%Câu 37.
Với mọi $a, b$ thỏa mãn $\log_2 a^3+\log_2 b=5$, khẳng định nào dưới đây đúng?
\choice
{\True $a^3 b=32$}
{$a^3 b=25$}
{$a^3+b=25$}
{$a^3+b=32$}

\end{ex}
\begin{ex}%Câu 38.
Trên đoạn $[-1; 2]$ hàm số $y=x^3+3 x^2+1$ đạt giá trị nhỏ nhất tại điểm
\choice
{$x=2$}
{\True $x=0$}
{$x=-1$}
{$x=1$}

\end{ex}
\begin{ex}%Câu 39.
Trong mặt phẳng $O x y z$, cho hai điểm $A(1; 0; 0) B(3; 2; 1)$. Mặt phẳng đi qua $A$ và vuông góc với $AB$ có phương trình là:
\choice
{\True $2 x+2 y+z-2=0$}
{$4 x+2 y+z-17=0$}
{$4 x+2 y+z-4=0$}
{$2 x+2 y+z-11=0$}

\end{ex}
\begin{ex}%Câu 40.
Trên khoảng $(0;+\infty)$, đạo hàm của hàm số $y=x^{\frac{5}{4}}$ là
\choice
{$y'=\dfrac{4}{9} x^{\frac{9}{4}}$}
{$y'=\dfrac{4}{5} x^{\frac{1}{4}}$}
{\True $y'=\dfrac{5}{4} x^{\frac{1}{4}}$}
{$y'=\dfrac{5}{4} x^{-\frac{1}{4}}$}

\end{ex}
\begin{ex}%Câu 41.
Cho khối chóp có diện tích đáy $B=3 a^2$ và chiều cao $h=a$. Thể tích của khối chóp đã cho bằng:
\choice
{$\dfrac{3}{2} a^3$}
{$3 a^3$}
{$\dfrac{1}{3} a^3$}
{\True $a^3$}

\end{ex}
\begin{ex}%Câu 42.
Nếu $\displaystyle\int\limits_1^4 f(x)\mathrm{\,d}x=6\displaystyle\int\limits_1^4 g(x)\mathrm{\,d}x=-5$ thì $\displaystyle\int\limits_1^4[f(x)-g(x)]\mathrm{\,d}x$ bằng:
\choice
{$-1$}
{$-11$}
{$1$}
{\True $11$}

\end{ex}
\begin{ex}%Câu 43.
Tập xác định của hàm số $y=7^{x}$ là:
\choice
{$\mathbb{R} \backslash\{0\}$}
{$[0;+\infty)$}
{$(0;+\infty)$}
{\True $\mathbb{R}$}

\end{ex}
\begin{ex}%Câu 44.
\immini{
Cho hàm số $ y= f(x)$ có bảng biến thiên như hình bên. Giá trị cực đại của hàm số đã cho bằng
\choice
{\True $3$}
{$-1$}
{$-5$}
{$1$}}
{\vspace{-0.5cm}
\begin{tikzpicture}[color=\mauchinh]
\tkzTabInit[nocadre,lgt=1.1,espcl=1.6,deltacl=0.6]
{$x$/0.6,$f'(x)$/0.6,$f(x)$/1.8}{$-\infty$,$-1$,$1$,$+\infty$}
\tkzTabLine{,+,0,-,0,+,}
\tkzTabVar{-/$-\infty$,+/$3$,-/$-3$,+/$+\infty$}
\end{tikzpicture}

}

\end{ex}
\begin{ex}%Câu 45.
Diện tích $S$ của mặt cầu bán kính $R$ được tính theo công thức nào dưới đây?
\choice
{\True $S=4\pi R^2$}
{$S=16\pi R^2$}
{$S=\dfrac{4}{3} \pi R^2$}
{$S=\pi R^2$}

\end{ex}
\begin{ex}%Câu 46.
Trong không gian $O x y z$, cho đường thẳng $d$ đi qua điểm $M(2; 2; 1)$ và có một vecto chỉ phương $\vec{u}=(5; 2;-3)$. Phương trình của $d$ là:
\choice
{$\heva{&x=5+2 t \\& y=2+2 t \\& z=-3+t}$}
{$\heva{&x=5+2 t \\& y=2+2 t \\& z=-3+t}$}
{\True $\heva{&x=5+2 t \\& y=2+2 t \\& z=-3+t}$}
{$\heva{&x=5+2 t \\& y=2+2 t \\& z=-3+t}$}

\end{ex}
\begin{ex}%Câu 47.
\immini{
Cho hàm số $y=f(x)$ có đồ thị là đường cong như hình vẽ. Hàm số đã cho đồng biến trên khoảng nào dưới đây?
\choice
{$(-1; 1)$}
{$(-\infty; 0)$}
{\True $(0; 1)$}
{$(0;+\infty)$}}
{\vspace{-0.5cm}
\begin{tikzpicture}[scale=1, font=\footnotesize, line join=round, line cap=round, >=stealth,y=0.8cm,color=\mauchinh]
\def\xmin{-1.65}\def\xmax{1.65}\def\ymin{-1}\def\ymax{2.3}
\draw[->,thick] (\xmin-0.2,0)--(\xmax+0.2,0) node[below] {\footnotesize $x$};
\draw[->,thick] (0,\ymin-0.2)--(0,\ymax+0.2) node[right] {\footnotesize $y$};
\draw (0,0) node [below left] {\footnotesize $O$};
\foreach \x in {-1,1}\draw (\x,0.1)--(\x,-0.1) node [below] {\footnotesize $\x$};
\foreach \y in {2}\draw (0.1,\y)--(-0.1,\y) node [left] {\footnotesize $\y$};
\clip (\xmin,\ymin) rectangle (\xmax,\ymax);
\draw[thick,smooth,samples=200,domain=\xmin:\xmax] plot (\x,{-1*((\x)^4)+2*((\x)^2)+1});
\draw[dashed] (-1,0)--(-1,2)--(0,2);\fill (-1,2) circle (1pt);
\draw[dashed] (1,0)--(1,2)--(0,2);\fill (1,2) circle (1pt);
\end{tikzpicture}
}

\end{ex}
\begin{ex}%Câu 48.
Với $n$ là số nguyên dương bất kì, $n \geq 5$, công thức nào dưới dây đúng?
\choice
{$A_n^5=\dfrac{n !}{5!(n-5) !}$}
{$A_n^5=\dfrac{5!}{(n-5) !}$}
{\True $A_n^5=\dfrac{n !}{(n-5) !}$}
{$A_n^5=\dfrac{n !}{(n-5) !}$}

\end{ex}
\begin{ex}%Câu 49.
Thể tích của khối lập phương cạnh $4 a$ bằng:
\choice
{\True $64 a^3$}
{$32 a^3$}
{$16 a^3$}
{$8 a^3$}
\end{ex}
\begin{ex}%Câu 50.
Cho hàm số $f(x)=x^2+3$. Khẳng định nào dưới đây đúng?
\choice
{$\displaystyle\int f(x) \mathrm{d} x=x^2+3 x+C$}
{\True $\displaystyle\int f(x) \mathrm{d} x=\dfrac{x^3}{3}+3 x+C$}
{$\displaystyle\int f(x) \mathrm{d} x=x^3+3 x+C$}
{$\displaystyle\int f(x) \mathrm{d} x=2 x+C$}
\end{ex}

\Closesolutionfile{ans}
%\indapan{10}{ans/ans-de16-7}
