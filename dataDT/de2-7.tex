
\begin{name}
	{\tenchude}
	{\tendethi}
	{\tentruong}
	{\thoigian}
\end{name}
\Opensolutionfile{ans}[ans/ans-de2-7]

\begin{ex}%Câu 43.
Số phức $z=7-9 i$ có phần ảo là
\choice
{$-9 i$}
{$9$}
{$9 i$}
{\True $-9$}

\end{ex}
\begin{ex}%Câu 1.
Đường tiệm cận đứng của đồ thị hàm số $y=\dfrac{x}{x-1}$ là
\choice
{\True $x=1$}
{$x=0$}
{$y=1$}
{$y=0$}

\end{ex}
\begin{ex}%Câu 2.
Tập nghiệm của bất phương trình $5^{2 x+1} \leq 25$ là:
\choice
{$\left(-\infty; \dfrac{1}{2}\right)$}
{$\left(-\infty; \dfrac{-1}{2}\right)$}
{$\left(-\infty; \dfrac{-1}{2}\right]$}
{\True $\left(-\infty; \dfrac{1}{2}\right]$}

\end{ex}
\begin{ex}%Câu 3.
\immini{
Cho hàm số $f(x)$ liên tục trên $\mathbb{R}$ và có đồ thị là đường cong như hình vẽ bên. Số nghiệm của phương trình $2 f(x)+1=0$ là 
\choice
{$1$}
{ $2$}
{ $3$}
{\True $4$}
}
{\vspace{-0.6cm}
\begin{tikzpicture}[scale=.6, font=\footnotesize, line join=round, line cap=round, >=stealth,color=\mauchinh]
\def\xmin{-2.85}\def\xmax{2.85}\def\ymin{-3.1}\def\ymax{1.5}
\draw[->,thick] (\xmin-0.2,0)--(\xmax+0.2,0) node[below] {\footnotesize $x$};
\draw[->,thick] (0,\ymin-0.2)--(0,\ymax+0.2) node[right] {\footnotesize $y$};
\draw (0,0) node [below left] {\footnotesize $O$};
\foreach \x in {-2,2}\draw (\x,0.1)--(\x,-0.1) node [below] {\footnotesize $\x$};
\foreach \y in {-3,1}\draw (0.1,\y)--(-0.1,\y) node [left] {\footnotesize $\y$};
\clip (\xmin,\ymin) rectangle (\xmax,\ymax);
\draw[thick,smooth,samples=200,domain=\xmin:\xmax] plot (\x,{1/4*((\x)^4)+-2*((\x)^2)+1});
\draw[dashed] (-2,0)--(-2,-3)--(0,-3);\fill (-2,-3) circle (1pt);
\draw[dashed] (2,0)--(2,-3)--(0,-3);\fill (2,-3) circle (1pt);
\end{tikzpicture}
}
\end{ex}
\begin{ex}%Câu 4.
Cho hàm số $f(x)$ và $g(x)$ liên tục trên $[0;2]$ và  $\displaystyle\int\limits_0^2f(x){\rm d}x=2$, $\displaystyle\int\limits_0^2g(x){\rm d}x=-2$. Tính $\displaystyle\displaystyle\int\limits_0^2[3 f(x)+g(x)]\mathrm{\,d}x$ 
\choice
{\True $4$}
{$8$}
{$12$}
{$6$}

\end{ex}
\begin{ex}%Câu 5.
Cho số phức $z=2+\sqrt{3} i$. Môđun của $z$ bằng.
\choice
{$\sqrt{5}$}
{\True $\sqrt{7}$}
{$7$}
{$5$}

\end{ex}
\begin{ex}%Câu 6.
Cho các số phức $z=2+i$ và $\mathrm{w}=3-2 i$. Phần ảo của số phức $z+2 w$ bằng.
\choice
{$8$}
{$-3 i$}
{$-4$}
{\True $-3$}

\end{ex}
\begin{ex}%Câu 7.
Cho số phức $z=2 i+1$. Điểm nào sau đây là điểm biểu diễn của số phức $\bar{z}$ trên mặt phẳng tọa độ.
\choice
{$H(1; 2)$}
{\True $G(1;-2)$}
{$T(2;-1)$}
{$K(2; 1)$}

\end{ex}
\begin{ex}%Câu 8.
Trong không gian $O x y z$, hình chiếu vuông góc của điểm $M(3; 1; 2)$ trên trục $O y$ là điểm
\choice
{$E(3; 0; 2)$}
{\True $F(0; 1; 0)$}
{$L(0;-1; 0)$}
{$S(-3; 0;-2)$}

\end{ex}
\begin{ex}%Câu 9.
Trong không gian $O x y z$, cho mặt cầu $(S)\colon x^2+y^2+z^2-2 x+4 y+1=$ 0. Tính diện tích của mặt cầu $(S)$.
\choice
{$4\pi$}
{$64\pi$}
{$\dfrac{32\pi}{3}$}
{\True $16\pi$}

\end{ex}
\begin{ex}%Câu 10.
Trong không gian $O x y z$ cho mặt phẳng $(P)\colon 2 x+y-z+3=0$. Điểm nào sau đây không thuộc $(P)$?
\choice
{$V(0;-2; 1)$}
{$Q(2;-3; 4)$}
{\True $T(1;-1; 1)$}
{$I(5;-7; 6)$}

\end{ex}
\begin{ex}%Câu 11.
Trong không gian $O x y z$ cho đường thẳng $d\colon \dfrac{x+1}{1}=\dfrac{y-2}{2}=\dfrac{z}{-2}$ có một vectơ chỉ phương là $\vec{u}=(-1; a; b)$. Tính giá trị của $T=a^2-a b$.
\choice
{\True $T=8$}
{$T=0$}
{$T=2$}
{$T=4$}

\end{ex}
\begin{ex}%Câu 12.
Cho hình chóp $S.ABC$ có $SA$ vuông góc với mặt phẳng $(ABC)$. $SA=1$ và đáy $ABC$ là tam giác đều với độ dài cạnh bằng $2$. Tính góc giữa mặt phẳng $(SBC)$ và mặt phẳng $(ABC)$.
\choice
{$60^{\circ}$}
{$45^{\circ}$}
{\True $30^{\circ}$}
{$90^{\circ}$}

\end{ex}
\begin{ex}%Câu 13.
Cho hàm số $f(x)$ thỏa mãn $f'(x)=x^2(x-1), \forall x \in \mathbb{R}$. Phát biểu nào sau đây là đúng?
\choice
{$f(x)$ có hai điểm cực trị}
{$f(x)$ không có cực trị}
{\True $f(x)$ đạt cực tiểu tại $x=1$}
{$f(x)$ đạt cực tiểu tại $x=0$}

\end{ex}
\begin{ex}%Câu 14.
Giá trị lớn nhất của hàm số $y=\dfrac{x^2-2 x+1}{x+2}$ trên đoạn $[0; 3]$ bằng
\choice
{$0$}
{$\dfrac{1}{2}$}
{$\dfrac{3}{2}$}
{\True $\dfrac{4}{5}$}

\end{ex}
\begin{ex}%Câu 15.
Biết $\log_3 4=a$ và $\mathrm{T}=\log_{12} 18$. Phát biểu nào sau đây đúng?
\choice
{$T=\dfrac{a+2}{2 a+2}$}
{\True $T=\dfrac{a+4}{2 a+2}$}
{$T=\dfrac{\sqrt{a}+2}{a+1}$}
{$T=\dfrac{\sqrt{a}-2}{a+1}$}

\end{ex}
\begin{ex}%Câu 16.
Số giao điểm của đồ thị hàm số $y=x^4-3 x^2+1$ với trục hoành là
\choice
{\True $4$}
{$3$}
{$2$}
{$0$}

\end{ex}
\begin{ex}%Câu 17.
Tập nghiệm của bất phương trình $\log_2{}^2(2 x)+1\leq \log_2\left(x^5\right)$ là
\choice
{$(0; 4]$}
{$(0; 2]$}
{\True $[2; 4]$}
{$[1; 4]$}

\end{ex}
\begin{ex}%Câu 18.
Cho tam giác đều $ABC$ có diện tích bằng $s_1$ và $AH$ là đường cao. Quay tam giác $ABC$ quanh đường thẳng $AH$ ta thu được hình nón có diện tích xung quanh bằng $s_2$. Tính $\dfrac{S_1}{S_2}$.
\choice
{$\dfrac{2\sqrt{3}}{\pi}$}
{\True $\dfrac{\sqrt{3}}{2\pi}$}
{$\dfrac{\sqrt{3}}{\pi}$}
{$\dfrac{4}{\pi \sqrt{3}}$}

\end{ex}
\begin{ex}%Câu 19.
Xét tích phân $I=\displaystyle\displaystyle\int\limits_0^4 {\rm e}^{\sqrt{2 x+1}}\mathrm{\,d}x$, nếu đặt $u=\sqrt{2 x+1}$ thì $I$ bằng
\choice
{$\dfrac{1}{2} \displaystyle\displaystyle\int\limits_1^3 u {\rm e}^{u}\mathrm{\,d}u$}
{$\displaystyle\displaystyle\int\limits_0^4 u {\rm e}^{u}\mathrm{\,d}u$}
{\True $\displaystyle\displaystyle\int\limits_1^3 u {\rm e}^{u}\mathrm{\,d}u$}
{$\dfrac{1}{2} \displaystyle\displaystyle\int\limits_1^3 {\rm e}^{u}\mathrm{\,d}u$}

\end{ex}
\begin{ex}%Câu 20.
Gọi $(H)$ là hình phẳng giới hạn bởi các đồ thị $y=x^2-2 x, y=0$ trong mặt phẳng $\mathrm{Oxy}$. Quay hình $(H)$ quanh trục hoành ta được một khối tròn xoay có thể tích bằng
\choice
{$\displaystyle\displaystyle\int\limits_0^2\left|x^2-2 x\right|\mathrm{\,d}x$}
{$\pi \displaystyle\displaystyle\int\limits_0^2\left|x^2-2 x\right|\mathrm{\,d}x$}
{\True $\pi \displaystyle\displaystyle\int\limits_0^2\left(x^2-2 x\right)^2\mathrm{\,d}x$}
{$\displaystyle\displaystyle\int\limits_0^2\left(x^2-2 x\right)^2\mathrm{\,d}x$}

\end{ex}
\begin{ex}%Câu 21.
Cho số phức $z=a+b i$ (với $a, b \in \mathbb{R}$) thỏa mãn $z(\overline{1+2 i})+i=3$. Tính $T=a+b$.
\choice
{$T=-\dfrac{6}{5}$}
{$T=0$}
{\True $T=2$}
{$T=1$}

\end{ex}
\begin{ex}%Câu 22.
Cho hình nón có chiều cao bằng $a \sqrt{3}$ và đường kính đáy bằng $2 a$. Diện tích xung quanh của hình nón đã cho bằng
\choice
{$8\pi a^2$}
{\True $2\pi a^2$}
{$4\pi a^2$}
{$\pi a^2$}

\end{ex}
\begin{ex}%Câu 23.
Đường thẳng nào dưới đây là tiệm cận ngang của đồ thị hàm số $y=\dfrac{1-6 x}{3 x-1}?$ 
\choice
{$y=2$}
{$y=6$}
{\True $y=-2$}
{$y=\dfrac{1}{3}$}

\end{ex}
\begin{ex}%Câu 24.
\immini{
Điểm nào trong hình vẽ bên dưới là điểm biểu diễn số phức $z=-1+2 i?$ 
\choice
{$P$}
{$N$}
{\True $Q$}
{$M$}}
{
\begin{tikzpicture}[>=stealth, scale=.8,color=\mauchinh,line width=.6pt]
\pgfmathsetmacro{\a}{sqrt(2)}
 \draw[->,thick] (-1.5,0)--(1.5,0) node[above] {$x$};
 \draw[->,thick] (0,-2.3)--(0,2.5) node[right] {$y$};
 \draw (0,0) node [below right] {$O$};
 \foreach \x in {-1,1}
\draw[thin] (\x,1pt)--(\x,-1pt) node [below] {$\x$};
 \foreach \y in {-2,1,2}
 	\draw[thin] (1pt,\y)--(-1pt,\y) node [left] {$\y$};
\draw[dashed](-1,0)|-(0,2) (-1,0)|-(0,1) (1,0)|-(0,2) (1,0)|-(0,-2);
\draw (-1,2)  node[above]{$Q$} (-1,1)node[left]{$M$} (1,-2)node[below]{$N$} (1,2)node[above]{$P$};
 \end{tikzpicture}
}

\end{ex}
\begin{ex}%Câu 25.
Thể tích $V$ của khối lăng trụ có diện tích đáy bằng $3\mathrm{\,m}^2$ và chiều cao bằng $4\mathrm{\,m}$ là
\choice
{\True $V=12\mathrm{\,m}^3$}
{$V=6\mathrm{\,m}^3$}
{$V=4\mathrm{\,m}^3$}
{$36\mathrm{\,m}^3$}

\end{ex}
\begin{ex}%Câu 26.
\immini{
Cho hàm số $y=f(x)$ liên tục trên $\mathbb{R}$ và có đồ thị như hình vẽ. Số nghiệm thực dương phân biệt của phương trình $f(x)=$ $-1$ là
\choice
{\True $2$}
{$4$}
{$3$}
{$1$}}
{\vspace{-0.6cm}
\begin{tikzpicture}[scale=.7, font=\footnotesize, line join=round, line cap=round, >=stealth,color=\mauchinh,y=0.7cm]
\def\xmin{-2.83}\def\xmax{2.83}\def\ymin{-2.1}\def\ymax{2.3}
\draw[->,thick] (\xmin-0.2,0)--(\xmax+0.2,0) node[below] {\footnotesize $x$};
\draw[->,thick] (0,\ymin-0.2)--(0,\ymax+0.2) node[right] {\footnotesize $y$};
\draw (0,0) node [below left] {\footnotesize $O$};
\foreach \x in {-2,2}\draw (\x,0.1)--(\x,-0.1) node [below] {\footnotesize $\x$};
\foreach \y in {-2,2}\draw (0.1,\y)--(-0.1,\y) node [left] {\footnotesize $\y$};
\clip (\xmin,\ymin) rectangle (\xmax,\ymax);
\draw[thick,smooth,samples=200,domain=\xmin:\xmax] plot (\x,{1/4*((\x)^4)+-2*((\x)^2)+2});
\draw[dashed] (-2,0)--(-2,-2)--(0,-2);\fill (-2,-2) circle (1pt);
\draw[dashed] (2,0)--(2,-2)--(0,-2);\fill (2,-2) circle (1pt);
\end{tikzpicture}
}

\end{ex}
\begin{ex}%Câu 27.
Cho hàm số $y=f(x)$ có bảng biến thiên như hình vẽ. 
\immini{
Hàm số $y=f(x)$ có giá trị cực tiểu bằng
\choice
{$3$}
{$1$}
{$-1$}
{\True $0$}}
{
\begin{tikzpicture}[color=\mauchinh]
\tkzTabInit[nocadre,lgt=1.2,espcl=1.6,deltacl=0.6,lw=0.8pt]
{$x$/0.6,$f'(x)$/0.6,$f(x)$/1.7}{$-\infty$,$-1$,$0$,$1$,$+\infty$}
\tkzTabLine{,-,0,+,0,-,0,+,}
\tkzTabVar{+/$+\infty$,-/$0$,+/$3$,-/$0$,+/$+\infty$}
\end{tikzpicture}
}

\end{ex}
\begin{ex}%Câu 28.
Giá trị nhỏ nhất của hàm số $f(x)=x^3+3 x+1$ trên đoạn $[1; 3]$ là
\choice
{$\min\limits_{[1; 3]} f(x)=3$}
{$\min\limits_{[1; 3]} f(x)=6$}
{\True $\min\limits_{[1; 3]} f(x)=5$}
{$\min\limits_{[1; 3]} f(x)=37$}

\end{ex}
\begin{ex}%Câu 29.
Bán kính $r$ của khối trụ có thể tích bằng $9 a^3$ và chiều cao bằng $a$ là:
\choice
{$r=\dfrac{3\sqrt{3} a}{\sqrt{\pi}}$}
{\True $r=\dfrac{3 a}{\sqrt{\pi}}$}
{$r=\dfrac{3\sqrt{3} a}{\pi}$}
{$r=\dfrac{3 a}{\pi}$}

\end{ex}
\begin{ex}%Câu 30.
Trong không gian $O x y z$, cho đường thẳng $d\colon\left\{\begin{aligned}&x=1+t \\& y=3 t \\& z=2-t\end{aligned},(t \in\right.$ $\mathbb{R})$. Điểm nào dưới đây không thuộc đường thẳng $d$?
\choice
{$Q(0;-3; 3)$}
{\True $P(1; 3; 2)$}
{$N(2; 3; 1)$}
{$M(1; 0; 2)$}

\end{ex}
\begin{ex}%Câu 31.
Tính tổng hoành độ của các giao điểm của đồ thị hàm số $y=\dfrac{5 x+11}{x+3}$ và đường thẳng $y=-x-1$ 
\choice
{\True $-9$}
{$5$}
{$3$}
{$-7$}

\end{ex}
\begin{ex}%Câu 32.
Trong không gian $O x y z$, cho mặt cầu $(S)\colon(x-2)^2+(y+1)^2+z^2=$ 10. Tâm $I$ và bán kính $R$ của mặt cầu $(S)$ là:
\choice
{\True $I(2;-1; 0); R=\sqrt{10}$}
{$I(-2; 1; 0); R=\sqrt{10}$}
{$I(2;-1; 0); R=10$}
{$I(-2; 1; 0); R=10$}

\end{ex}
\begin{ex}%Câu 33.
Trong không gian $O x y z$, mặt phẳng $(P)$ đi qua điểm $M(1; 2; 3)$ và vuông góc với đường thẳng $d\colon \dfrac{x}{2}=\dfrac{y-1}{-1}=\dfrac{z+2}{1}$ có phương trình là:
\choice
{\True $2 x-y+z-3=0$}
{$y-2 z+4=0$}
{$2 x-y+z+4=0$}
{$2 x+y+z-7=0$}

\end{ex}
\begin{ex}%Câu 34.
Cấp số nhân $\left(u_n\right)$ với $u_5=5$ và công bội $q=3$ thì $u_6$ bằng
\choice
{$\dfrac{5}{3}$}
{\True $15$}
{$45$}
{$75$}

\end{ex}
\begin{ex}%Câu 35.
Cho hai số phức $z_1=1+i$ và $z_2=-3+2 i$. Tính môđun cùa $z_1+z_2?$ 
\choice
{$\left|z_1+z_2\right|=\sqrt{5}$}
{\True $\left|z_1+z_2\right|=\sqrt{13}$}
{$\left|z_1+z_2\right|=1$}
{$\left|z_1+z_2\right|=5$}

\end{ex}
\begin{ex}%Câu 36.
Cho số phức $z$ thỏa mãn $(1-2 i) z=-2-11 i$.Tính số phức liên hợp của số phức $z$.
\choice
{\True $\bar{z}=4+3 i$}
{$\bar{z}=4-3 i$}
{$\bar{z}=-4-3 i$}
{$\bar{z}=-4+3 i$}

\end{ex}
\begin{ex}%Câu 37.
Số cách lấy $5$ viên bi trong số $20$ viên bi khác nhau là
\choice
{$5!$}
{\True $C_{20}^5$}
{$5^{20}$}
{$A_{20}^5$}

\end{ex}
\begin{ex}%Câu 38.
Biết $z$ là số phức có phần ảo dương và là nghiệm của phương trình $z^2-6 z+10=0$. Tính tổng phần thực và phần ảo của số phức $w=\dfrac{z}{\bar{z}}$.
\choice
{\True $\dfrac{7}{5}$}
{$\dfrac{4}{5}$}
{$\dfrac{1}{5}$}
{$\dfrac{3}{5}$}

\end{ex}
\begin{ex}%Câu 39.
Cho hàm số $f(x)$ có $f'(x)=x(x-3)^2(x-2), \forall x \in \mathbb{R}$. Số điểm cực trị của hàm số đã cho là
\choice
{\True $2$}
{$1$}
{$0$}
{$3$}

\end{ex}
\begin{ex}%Câu 40.
Cho mặt cầu có bán kính $R=2$. Diện tích của mặt cầu đã cho bằng
\choice
{$\dfrac{32\pi}{3}$}
{$32\pi$}
{$\dfrac{16\pi}{3}$}
{\True $16\pi$}

\end{ex}
\begin{ex}%Câu 41.
Nếu $a$ và $b$ là các số thực dương thì $\log_7 a+\log_7 b$ bằng
\choice
{$\log_{14}(a+b)$}
{$\log_7 a \cdot \log_7 b$}
{\True $\log_7(a b)$}
{$\log_7(a+b)$}

\end{ex}
\begin{ex}%Câu 42.
Tập nghiệm của bất phương trình $\left(\dfrac{1}{3}\right)^{x}>1$ là
\choice
{$[0;+\infty)$}
{$(-\infty; 1]$}
{$(0;+\infty)$}
{\True $(-\infty; 0)$}

\end{ex}

\begin{ex}%Câu 44.
Nếu $\displaystyle\displaystyle\int\limits_0^2 \dfrac{f(x)}{3} \mathrm{\,d} x=4$ thì $\displaystyle\displaystyle\int\limits_0^2 f(x) \mathrm{d} x$ bằng:
\choice
{\True $12$}
{$4$}
{$3^4$}
{$\dfrac{4}{3}$}

\end{ex}
\begin{ex}%Câu 45.
Nếu muốn tăng thể tích của một khối lập phương lên gấp $8$ lần thì cạnh của khối lập phương đó phải tăng lên mấy lần?
\choice
{\True $2$ lần}
{$4$ lần}
{$8$ lần}
{$3$ lần}

\end{ex}
\begin{ex}%Câu 46.
Tập nghiệm của bất phương trình $\log_3^2 x-\log_3 x-2>0$ là:
\choice
{$\left(-\infty; \dfrac{1}{3}\right) \cup(9;+\infty)$}
{$(9;+\infty)$}
{$(-\infty;-1) \cup(2;+\infty)$}
{\True $\left(0; \dfrac{1}{3}\right) \cup(9;+\infty)$}
\end{ex}
\begin{ex}%47
\immini{
Cho hàm số $y=f(x)$, có bảng biến như hình vẽ. Hàm số $y=f(x)$ đồng biến trên khoảng
\choice
{$(2;+\infty)$}
{$(1;5)$}
{\True $(0;2)$}
{$(-\infty;0)$}}
{\vspace{-0.6cm}
\begin{tikzpicture}[color=\mauchinh]
\tkzTabInit[nocadre,lgt=1.2,espcl=1.8,deltacl=0.6,lw = 0.6pt]
{$x$/0.6,$f'(x)$/0.6,$f(x)$/2}{$-\infty$,$0$,$2$,$+\infty$}
\tkzTabLine{,-,0,+,0,-,}
\tkzTabVar{+/$+\infty$,-/$1$,+/$5$,-/$-\infty$}
\end{tikzpicture}

}
\end{ex}

\begin{ex}%Câu 48.
Cho hình phẳng $D$ giới hạn bởi các đường $y=5^{x}, y=0, x=-2, x=2$. Thể tích khối tròn xoay tạo thành do hình phẳng $D$ quay quanh trục hoành được tính theo công thức nào dưới đây?
\choice
{\True $V=\pi \displaystyle\displaystyle\int\limits_{-2}^2 25^{x} \mathrm{\,d} x$}
{$V=\displaystyle\displaystyle\int\limits_{-2}^2 5^{2 x} \mathrm{\,d} x$}
{$V=\displaystyle\displaystyle\int\limits_{-2}^2\left|5^{x}\right| \mathrm{d} x$}
{$V=2\pi \displaystyle\displaystyle\int\limits_0^2 5^{2 x} \mathrm{\,d} x$}

\end{ex}
\begin{ex}%Câu 49.
Nếu $\displaystyle\displaystyle\int\limits_a^{b} x\mathrm{\,d}x=a$ thì $3\displaystyle\displaystyle\int\limits_{{\rm e}^{a}}^{{\rm e}^{b}} \dfrac{\ln x}{x}\mathrm{\,d}x$ bằng
\choice
{$\dfrac{3}{a}$}
{$\dfrac{a}{3}$}
{$a$}
{\True $3 a$}

\end{ex}
\begin{ex}%Câu 50.
Trong các các hàm số sau, đồ thị của hàm số nào nhận $x=-1$ làm tiệm cận đứng?
\choice
{$y=\dfrac{x-3}{-x+1}$}
{$y=\dfrac{x-3}{x-1}$}
{\True $y=\dfrac{x+3}{x+1}$}
{$y=\dfrac{x+3}{x-1}$}

\end{ex}
\Closesolutionfile{ans}
%\indapan{10}{ans/ans-de2-7}
