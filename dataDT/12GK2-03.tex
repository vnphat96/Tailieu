\begin{name}
{\tenchude}{\tendethi}{LỚP TOÁN THẦY PHÁT}{\thoigian}
\end{name}

\Opensolutionfile{ans}[ans/ans-2-TT-ChuyenDH-Vinh-L1-NH21-22]
\begin{ex}%[2D3Y2-1]%
Cho $\displaystyle\int\limits_a^b f(x)\mathrm{\,d}x=2$ và $\displaystyle\int\limits_c^b f(x)\mathrm{\,d}x=3$ với $a<b<c$, khi đó $\displaystyle\int\limits_a^c f(x)\mathrm{\,d}x$ bằng
\choice
{$-2$}
{$5$}
{$1$}
{\True $-1$}
\loigiai{
Ta có $\displaystyle\int\limits_a^c f(x)\mathrm{\,d}x=\displaystyle\int\limits_a^b f(x)\mathrm{\,d}x+\displaystyle\int\limits_b^c f(x) \mathrm{\,d}x=\displaystyle\int\limits_a^b f(x) \mathrm{\,d}x-\displaystyle\int\limits_c^b f(x)\mathrm{\,d}x=2-3=-1$.
}
\end{ex}

\begin{ex}%[Nguyễn Thành Tiến- Dự án Tổng ôn LVĐ 2022]%[2H3B1-1]%
Trong không gian $Oxyz$, cho hai véctơ $\vec{u}=\left(2;1;-1\right)$ và $\vec{v}=\left(1;3;m\right)$. Tìm $m$ để $\left(\vec{u},\vec{v}\right)=90^\circ$.
\choice
{$m=-5$}
{\True $m=5$}
{$m=1$}
{$m=-2$}
\loigiai{
Ta có $\left(\vec{u},\vec{v}\right)=90^\circ \Leftrightarrow \vec{u}\perp \vec{v}$\\
$\Leftrightarrow 2+3-m=0\Leftrightarrow m=5$.
}
\end{ex}

\begin{ex}%[Nguyễn Thành Tiến- Dự án Tổng ôn LVĐ 2022]%[2H3B1-1]%
Trong không gian $Oxyz$, cho điểm $A\left(3;2;1\right)$, $B\left(-1;3;2\right)$ và $C\left(2;4;-3\right)$. Tích vô hướng $\vec{AB}\cdot \vec{AC}$ bằng
\choice
{\True $2$}
{$-2$}
{$6$}
{$-6$}
\loigiai{
$\vec{AB}\left(-4;1;1\right)$ và $\vec{AC}-1;2;-4$.\\
Suy ra $\vec{AB}\cdot \vec{AC}=4+2-4=2$.
}
\end{ex}

\begin{ex}%[2D3Y2-1]%
Cho hàm số $F(x)$ là một nguyên hàm của hàm số $f(x)$ trên đoạn $[-1; 2]$. Biết rằng $\displaystyle\int\limits_{-1}^2 f(x)\mathrm{\,d}x=1$ và $F(-1)=-1$. Tính $F(2)$.
\choice
{$2$}
{$3$}
{\True $0$}
{$-1$}
\loigiai{
Ta có $\displaystyle\int\limits_{-1}^2 f(x) \mathrm{\,d}x=F(2)-F(-1)\Leftrightarrow F(2)=\displaystyle\int\limits_{-1}^2 f(x) \mathrm{\,d}x+F(-1)=1+(-1)=0$.
}
\end{ex}

\begin{ex}%[2D3Y2-1]%
Cho $f(x)$ có đạo hàm trên $[1; 3]$ thỏa $f(1)=1, f(3)=m$ và $\displaystyle\int\limits_1^3 f'(x)\mathrm{\,d}x=5$. Khẳng định nào sau đây đúng ?
\choice
{$m \in(-\infty; -3)$}
{$m \in[-3; 3)$}
{\True $m \in[3; 10)$}
{$m \in[10;+ \infty)$}
\loigiai{
Ta có $\displaystyle\int\limits_1^3 f'(x) \mathrm{\,d}x=f(x)\Big|_1^3=f(3)-f(1)=m-1$.\\
Suy ra $m-1=5\Leftrightarrow m=6\in [3;10)$.
}
\end{ex}

\begin{ex}%[2D3Y1-1]%
Họ tất cả các nguyên hàm của hàm số $f(x)=\cos 3x$ là
\choice
{$-\dfrac{\sin 3x}{3}+C$}
{\True $\dfrac{\sin 3x}{3}+C$}
{$\sin 3x+C$}
{$3 \sin 3x+C$}
\loigiai{
Ta có $\displaystyle\int f(x) \mathrm{\,d}x=\displaystyle\int \cos 3x \mathrm{\,d}x=\dfrac{1}{3}\sin 3x+C$.
}
\end{ex}

\begin{ex}%[2H3Y2-3]%
Trong không gian với hệ tọa độ $Oxyz$, cho ba điểm $A(2;1;0)$, $B(-1;2;1)$, $C(2;0;-1)$. Phương trình tổng quát của mặt phẳng $(A B C)$ tương ứng là
\choice
{$x-y+1=0$}
{$x+y-1=0$}
{\True $y-z-1=0$}
{$2x-1=0$}
\loigiai{
Ta có $\overrightarrow{AB}=(-3;1;1)$, $\overrightarrow{AC}=(0;-1;-1)$.\\
Mặt phẳng $(ABC)$ có mợt véc-tơ pháp tuyến $\overrightarrow{n}=[\overrightarrow{AB},\overrightarrow{AC}]=(0;-3;3)=-3(0;1;-1)$.\\
Phương trình mặt phẳng $(ABC)$ có véc-tơ pháp tuyến $\vec{n}$ và qua $A(2;1;0)$ là
$$0(x-2)+1(y-1)-1(z-0)=0\Leftrightarrow y-z-1=0.$$
}
\end{ex}

\begin{ex}%[2D3Y2-1]%
Cho $f(x)$ là hàm số có đạo hàm liên tục trên $\mathbb{R}$ và $f(0)=1$, khi đó $\displaystyle\int\limits_0^x f'(t)\mathrm{\,d}t$ bằng
\choice
{$f(x)+1$}
{$f(x+1)$}
{$f(x)$}
{\True $f(x)-1$}
\loigiai{
Ta có $\displaystyle\int\limits_0^x f'(t) \mathrm{\,d}t=f(t)\Big|_0^x=f(x)-f(0)=f(x)-1$.
}
\end{ex}

\begin{ex}%[2D3B1-2]%
Tính $ I = \displaystyle  \int \sin x \cos x\, \mathrm{d\,} x $.
\choice
{$ I =  \dfrac{\cos 2x}{4} + C $}
{$ I =  - \dfrac{\sin^2 x }{2} + C $}
{\True $ I =  \dfrac{\sin^2 x}{2} + C $}
{$ I =  \dfrac{\cos^2 x}{2} + C $}
\loigiai{
Ta có $ I = \displaystyle  \int \sin x \, \mathrm{d\,} (\sin x) = \dfrac{\sin^2 x}{2} + C $.
}
\end{ex}

\begin{ex}%[Tổng ôn LVD-GV soạn Mui Doan-GVPB Khanh Lê]%[2D3B2-1]%
Với $ m $ là tham số thực, ta có $ \displaystyle\int\limits_{1}^{2} (2mx+1)\mathrm{\,d}x =4$. Khi đó $ m $ thuộc tập hợp nào sau đây?
\choice
{$ (-3;-1) $}
{$ [-1;0) $}
{\True $ [0;2) $}
{$ [2;6) $}
\loigiai{
Ta có
\allowdisplaybreaks
\begin{eqnarray*}
&&\displaystyle\int\limits_{1}^{2} (2mx+1)\mathrm{\,d}x =4\\
&\Leftrightarrow&(mx^2+x)\bigg|^{2}_{1}=4\\
&\Leftrightarrow&4m+2-(m+1)=4\\
&\Leftrightarrow&m=1\in[0;2).
\end{eqnarray*}
}
\end{ex}

\begin{ex}13%[2H3B1-3]%
Trong không gian với hệ trục tọa độ $Oxyz$, mặt cầu $(S)\colon x^{2}+y^{2}+(z+2)^{2}=9$ và điểm $A=(3 ;-1 ; 1)$. Phương trình tham số của đường thẳng $d$ đi qua $A$ cắt $(S)$ theo một dây cung có độ dài lớn nhất tương ứng là
\choice
{\True $\heva{&x=6+3 t \\&y=-2-t \\&z=4+3t}$}
{$\heva{&x=3+3 t \\&y=-1+t \\&z=1+3t\\}$}
{$\heva{&x=3 t \\&y=t \\&z=-2-3 .\\}$}
{$\heva{&x=3-3 t \\&y=-1+t \\&z=1-2t\\}$}

\loigiai{
Dễ dàng thấy được dây cung dài nhất đúng bằng đường kính mặt cầu. Khi đó đường thẳng $d$ đi qua tâm $I$ của mặt cầu $(S)$.\\
Suy ra véc-tơ chỉ phương của đường thẳng $d$ là $\vec{u}=\overrightarrow{AI}=(-3 ; 1 ;-3)$.\\
Đường thẳng $d$ đi qua điểm $A$ có véc-tơ chỉ phương $\overrightarrow{AI}=(-3 ; 1 ;-3)$ có phương trình  $$\heva{&x=6+3t\\&y=-2-t\\&z=4+3t.\\}$$
}
\end{ex}

\begin{ex}%[Tổng ôn LVD-GV soạn Mui Doan-GVPB Khanh Lê]%[2D3B2-1]%
Tích phân $ \displaystyle\int\limits_{0}^{1}\mathrm{e}^{3x+1}\mathrm{\,d}x $ bằng
\choice
{\True $ \dfrac{1}{3}\left(\mathrm{e}^4-\mathrm{e}\right) $}
{$ \dfrac{1}{3}\left(\mathrm{e}^4+\mathrm{e}\right) $}
{$ \mathrm{e}^4-\mathrm{e} $}
{$ \mathrm{e}^3-\mathrm{e} $}
\loigiai{
Ta có
\allowdisplaybreaks
\begin{eqnarray*}
\displaystyle\int\limits_{0}^{1}\mathrm{e}^{3x+1}\mathrm{\,d}x
&=&\dfrac{1}{3}\mathrm{e}^{3x+1}\bigg|^{1}_{0}\\
&=&\dfrac{1}{3}\left(\mathrm{e}^{4}-\mathrm{e}\right).
\end{eqnarray*}
}
\end{ex}

\begin{ex}%[2D3B1-1]%
Họ nguyên hàm của hàm số $f(x)=\mathrm{e}^x(1-3 \mathrm{e}^{-2x})$ là
\choice
{$\mathrm{e}^x-3 \mathrm{e}^{-3x}+C$}
{\True $\mathrm{e}^x+3 \mathrm{e}^{-x}+C$}
{$\mathrm{e}^x-3 \mathrm{e}^{-x}+C$}
{$\mathrm{e}^x+3 \mathrm{e}^{-2x}+C$}
\loigiai{
Ta có $\displaystyle\int f(x) \mathrm{\,d}x=\displaystyle\int \left( \mathrm{e}^x-3\mathrm{e}^{-x}\right)\mathrm{\,d}x =\mathrm{e}^x+3\mathrm{e}^{-x}+C$.
}
\end{ex}

\begin{ex}%[2D3B1-1]%
Biết $F(x)=(ax^2+bx+c)\mathrm{e}^x$ là một nguyên hàm của hàm số $f(x)=(x^2+5x+5)\mathrm{e}^x$. Giá trị biểu thức $2 a+3 b+c$ bằng
\choice
{$6$}
{\True $13$}
{$8$}
{$10$}
\loigiai{
Ta có
\begin{eqnarray*}
f(x)&=&F'(x)=(2ax+b)\mathrm{e}^x+(ax^2+bx+c)\mathrm{e}^x\\
&=&(ax^2+(2a+b)x+b+c)\mathrm{e}^x.
\end{eqnarray*}
Suy ra $\heva{&a=1\\&2a+b=5\\&b+c=5}\Rightarrow \heva{&a=1\\&b=3\\&c=2.}$\\
Giá trị của biểu thức $2a+3b+c=13$.
}
\end{ex}

\begin{ex}%[Tổng ôn LVD-GV soạn Mui Doan-GVPB Khanh Lê]%[2D3B1-1]%
Cho hàm số $ f(x) $ có $ f(0)=2 $ và  $ f'(x)=\left(\mathrm{e}^{-x}+1\right)\mathrm{e}^x, \forall x\in\mathbb{R} $. Khi đó $ \displaystyle\int\limits_0^1 f(x)\mathrm{\,d}x $ bằng
\choice
{$ 1+\dfrac{1}{2}\ln 2 $}
{$ \dfrac{1}{2}\ln 2+\dfrac{\pi}{4} $}
{$\mathrm{e}-\dfrac{1}{2}$}
{\True $\mathrm{e}+\dfrac{1}{2}$}
\loigiai{
Vì $ f'(x)=\left(\mathrm{e}^{-x}+1\right)\mathrm{e}^x$ nên
\allowdisplaybreaks
\begin{eqnarray*}
f(x)&=&\displaystyle\int\left(\mathrm{e}^{-x}+1\right)\mathrm{e}^x\mathrm{d}x\\
&=&\displaystyle\int\left(1+\mathrm{e}^x\right)\mathrm{\,d}x\\
&=&x+\mathrm{e}^{x}+C.
\end{eqnarray*}
Do $ f(0)=2\Rightarrow 1+C=2\Rightarrow C=1$.\\
Vậy $ f(x)=x+\mathrm{e}^{x}+1$.\\
Ta có
\allowdisplaybreaks
\begin{eqnarray*}
\displaystyle\int\limits_0^1 f(x)\mathrm{\,d}x &=&\displaystyle\int\limits_0^1 \left(x+\mathrm{e}^{x}+1\right)\mathrm{\,d}x \\
&=&\left(\dfrac{x^2}{2}+\mathrm{e}^{x}+x\right)\bigg|^1_0\\
&=&\left(\dfrac{1}{2}+\mathrm{e}+1\right)-1\\
&=&\mathrm{e}+\dfrac{1}{2}.
\end{eqnarray*}
}
\end{ex}

\begin{ex}%[2D3Y1-2]%
Tính nguyên hàm $I=\displaystyle\int\limits (3+2x)^2\mathrm{\,d}x$ bằng cách đặt $t=3+2x$. Mệnh đề nào dưới đây đúng?
\choice
{$I=\displaystyle\int\limits t^2\mathrm{\,d}t$}
{$I=\dfrac{1}{2}\displaystyle\int\limits t^3\mathrm{\,d}t$}
{$I=\dfrac{1}{6}\displaystyle\int\limits t^3\mathrm{\,d}t$}
{\True $I=\dfrac{1}{2}\displaystyle\int\limits t^2\mathrm{\,d}t$}
\loigiai{
Đặt $t=3+2x \Rightarrow \mathrm{\,d}t=2\mathrm{\,d}x \Rightarrow \mathrm{\,d}x= \dfrac{1}{2}\mathrm{\,d}t $.\\
Khi đó, $I=\displaystyle\int\limits \dfrac{1}{2}t^2\mathrm{\,d}t=\dfrac{1}{2}\displaystyle\int\limits t^2\mathrm{\,d}t$.
}
\end{ex}

\begin{ex}%[Tổng ôn LVD-GV soạn Mui Doan-GVPB Khanh Lê]%[2D3B2-1]%
Cho $ \displaystyle\int\limits_{3}^{5}\dfrac{\mathrm{\,d}x}{x^2-x}=a\ln 5+b\ln 3+c\ln 2 $ với $ a, b, c\in\mathbb{Q} $.  Giá trị của $b- 2a+3c^2 $ bằng
\choice
{$ -2 $}
{$ 0 $}
{$ 3 $}
{\True $ 6 $}
\loigiai{
Ta có
\allowdisplaybreaks
\begin{eqnarray*}
\displaystyle\int\limits_{3}^{5}\dfrac{\mathrm{\,d}x}{x^2-x}
&=&\displaystyle\int\limits_{3}^{5}\dfrac{\mathrm{\,d}x}{x(x-1)}\\
&=&\displaystyle\int\limits_{3}^{5}\left( \dfrac{-1}{x}+ \dfrac{1}{x-1}\right)\mathrm{\,d}x\\
&=&\left(-\ln \vert x \vert+\ln \vert x-1 \vert\right)\bigg|^{5}_{3} \\
&=&-\ln 5+\ln 4-(-\ln 3+\ln 2)\\
&=&-\ln 5+\ln 3+\ln 2.
\end{eqnarray*}
Suy ra $ a=-1 $, $ b=1 $, $ c=1 $.\\
Vậy $b- 2a+3c^2 =1-2\cdot (-1)+3\cdot 1^2=6$.
}
\end{ex}

\begin{ex}%[2D3Y1-1]%
Biết $F(x)$ là một nguyên hàm của $f(x)=\dfrac{1}{x-1}$ thỏa mãn $F(2)=1$. Giá trị của $F(3)$ bằng
\choice
{$\ln 2-1$}
{\True $\ln 2+1$}
{$\dfrac{1}{2}$}
{$\dfrac{7}{4}$}
\loigiai{
Ta có $\displaystyle\int f(x) \mathrm{\,d}x=\displaystyle\int \dfrac{1}{x-1} \mathrm{\,d}x=\ln|x-1|+C$.\\
Vì $F(2)=1\Rightarrow C=1$.\\
Suy ra $F(x)=\ln|x-1|+1$. Do đó, $F(3)=\ln 2+1$.
}
\end{ex}

\begin{ex}%[2H2K1-1]%
Cho tam giác  $\left(ABC\right)$ vuông tại $A$ có $AB=3AC=12$. Hỏi khi cho hình tam giác $ABC$ quay quanh cạnh $BC$ sẽ thu được khối nón tròn xoay có thể tích tương ứng bằng bao nhiêu?
\choice
{$\dfrac{96\pi}{5}$}
{$24\pi\sqrt{10}$}
{\True $\dfrac{96\pi\sqrt{10}}{5}$}
{$\dfrac{48\pi\sqrt{5}}{5}$}
\loigiai{
\begin{center}
\begin{tikzpicture}[scale=.7, font=\footnotesize, line join=round, line cap=round, >=stealth]
\def\a{3}
\def\b{1.5}
\def\h{5}
\pgfmathsetmacro\g{asin(\b/\h)}
\pgfmathsetmacro\xo{\a *cos(\g)}
\pgfmathsetmacro\yo{\b *sin(\g)}
\draw[dashed] (\xo,\yo)  arc (\g:180-\g:{3} and {1});
\draw 		  (\xo,\yo) arc (360+\g:180-\g:{3} and {1});
\draw (\xo,\yo)--(90:\h) node[above]{$$}--(-\xo,\yo);
\draw[dashed] (0:0)node[below]{$$}--(0:\a/2)node[above]{$r$} --(0:\a);
\draw[dashed] (90:\h)--(90:\h/2)node[left]{$h$}--(0:0);
\coordinate[label=right:$12$](I) at ($(90:\h)!0.5!(0:\a)$);
\coordinate[label=right:$4$](M) at ($(-90:\h/2)!0.5!(0:\a)$);
\fill (180:\a) node[below left]{$$}circle(1pt);
\fill (0:0) node[left]{$H$}circle(2pt);
\fill  (0:\a) node[below right] {$A$} circle(1pt);
\fill (90:\h) node[above]{$B$} circle(1pt);
\draw[dashed] (-90:\h/2)--(0:0);
\draw (-90:\h/2)--(0:\a);
\fill (-90:\h/2) node[below]{$C$} circle(1pt);
\draw (-90:\h/2)--(0:-\a);
\end{tikzpicture}
\end{center}
\begin{itemize}
\item Khi  quay hình tam giác $ABC$ quanh cạnh $BC$ ta được khối tròn xoay là hai khối hình nón có đáy bằng nhau chập đáy vào nhau như hình vẽ minh họa.
\item Ta đi tính bán kính đáy chung chính là đường cao $AH$ hạ từ đỉnh $A$.
\item Ta có $r=AH=\dfrac{AB\cdot AC}{\sqrt{AB^2+AC^2}}=\dfrac{6\sqrt{10}}{5}$.
\item Suy ra thể tích khối tròn xoay thu được là
\item $V=\dfrac{1}{3}\pi \left(AH\right)^2\cdot BH+\dfrac{1}{3}\pi\cdot \left(AH\right)^2\cdot CH=\dfrac{1}{3}\pi\cdot \left(AH\right)^2 \left(BC\right)=\dfrac{1}{3}\pi\cdot \left(\dfrac{6\sqrt{10}}{5}\right)^2\cdot \left(4\sqrt{10}\right)\\=\dfrac{96\pi\sqrt{10}}{5}$.
\end{itemize}
}
\end{ex}

\begin{ex}%[2D3B2-2]%
Cho $\displaystyle\int\limits_4^{10} f(x)\mathrm{\,d}x=18$. Tính $I=\displaystyle\int\limits_1^3 f(3x+1)\mathrm{\,d}x$.
\choice
{$I=\dfrac{18}{5}$}
{\True $I=6$}
{$I=9$}
{$I=\dfrac{15}{6}$}
\loigiai{
Đặt $t=3x+1\Rightarrow \dfrac{1}{3} \mathrm{\,d}t=\mathrm{\,d}x$.\\
Đổi cận $\heva{&x=1\Rightarrow t=4\\&x=3\Rightarrow t=10.}$\\
Suy ra $I=\displaystyle\int\limits_4^{10} f(t) \dfrac{1}{3}\mathrm{\,d}t=\dfrac{1}{3}\displaystyle\int\limits_4^{10} f(t) \mathrm{\,d}t=\dfrac{1}{3}\cdot 18=6$.
}
\end{ex}

\begin{ex}%[2H3B2-3]%
Trong không gian $O x y z$, phương trình mặt phẳng đi qua điểm $M(3 ; 2 ;-1)$ và vuông góc với đường thẳng $d:\heva{&x=1+2t\\&y=-2-3t\\&z=3+5t},t\in\mathbb{R}$ là
\choice
{$2 x-3 y+5 z-5=0$}
{$-2 x+3 y-5 z+5=0$}
{$2 x+3 y+5 z+5=0$}
{\True $2 x-3 y+5 z+5=0$}
\loigiai{
Đường thẳng $d:\heva{&x=1+2t\\&y=-2-3t\\&z=3+5t},t\in\mathbb{R}$ có véc-tơ chỉ phương $\vec{u}=(2;-3;5)$.\\
Mặt phẳng đi qua điểm $M(3 ; 2 ;-1)$ và vuông góc với đường thẳng $d:\heva{&x=1+2t\\&y=-2-3t\\&z=3+5t},t\in\mathbb{R}$ nên có véc-tơ pháp tuyến $\vec{n}=\vec{u}=(2;-3;5)$, có phương trình là
$$2(x-3)-3(y-2)+5(z+1)=0\Leftrightarrow 2x-3y+5z+5=0.$$
}
\end{ex}

\begin{ex}%[2D3Y2-1]%
Giá trị của tích phân $\displaystyle\int\limits_{0}^{\mathrm{e}^{2020}-1} \dfrac{\mathrm{\,d}x}{x+1}$ bằng
\choice
{\True$2020$}
{$2019$}
{$2021$}
{$0$}
\loigiai{
Ta có $\displaystyle\int\limits_{0}^{\mathrm{e}^{2020}-1} \dfrac{\mathrm{\,d}x}{x+1}=\ln(x+1)\bigg|_0^{\mathrm{e}^{2020}-1} =2020$.
}
\end{ex}

\begin{ex}%[2D3B1-2]%
Họ nguyên hàm $F(x)=\displaystyle\int\sin^{6}x\cdot\cos x\mathrm{\,d}x$ là
\choice
{\True $F(x)=\dfrac{\sin^{7}x}{7}+C$}
{$F(x)=\dfrac{\cos^{7}x}{7}+C$}
{$F(x)=\dfrac{\sin^{5}x}{5}+C$}
{$F(x)=\dfrac{\sin^{7}x\cdot\cos x}{8}+C$}
\loigiai{
Ta có $F(x)=\displaystyle\int\sin^{6}x\cdot\cos x\mathrm{\,d}x=\displaystyle\int\sin^{6}x\mathrm{\,d}\left(\sin x\right)=\dfrac{\sin^{7}x}{7}+C$.
}
\end{ex}

\begin{ex}%[2D3B1-1]%
Họ nguyên hàm $F(x)=\displaystyle\int\dfrac{1}{x^2+1}\mathrm{d}x$ là
\choice
{$F(x)=\tan x+C$}
{\True $F(x)=\arctan x+C$}
{$F(x)=\arctan2x+C$}
{$F(x)=-\arctan x+C$}
\loigiai{
Ta có $F(x)=\displaystyle\int\dfrac{1}{x^2+1}\mathrm{d}x=\arctan x+C$.
}
\end{ex}

\begin{ex}%[2D3B1-3]%
Họ nguyên hàm $F(x)=\displaystyle\int(3x-1)\mathrm{e}^x\mathrm{d}x$ là
\choice
{$F(x)=(3x-1)\mathrm{e}^x+C$}
{\True $F(x)=(3x-4)\mathrm{e}^x+C$}
{$F(x)=\dfrac{1}{3}(3x+1)\mathrm{e}^x+C$}
{$F(x)=\left(\dfrac{3}{2}x^2-x\right)\mathrm{e}^x+C$}
\loigiai{
Đặt $\heva{&u=3x-1\\&\mathrm{d}v=\mathrm{e}^x\mathrm{d}x}$, suy ra $\heva{&\mathrm{d}u=3\mathrm{d}x\\&v=\mathrm{e}^x.}$\\
Khi đó $F(x)=(3x-1)\mathrm{e}^x-\displaystyle\int3\mathrm{e}^x\mathrm{d}x=(3x-4)\mathrm{e}^x+C$.
}
\end{ex}

\begin{ex}%[2D3Y1-2]%
Tính nguyên hàm $A=\displaystyle\int\limits \dfrac{1}{x\ln{x}}\mathrm{\,d}x$ bằng cách đặt $t=\ln{x}$. Mệnh đề nào dưới đây đúng?
\choice
{$A=\displaystyle\int\limits \mathrm{\,d}t$}
{$A=\displaystyle\int\limits \dfrac{1}{t^2} \mathrm{\,d}t$}
{$A=\displaystyle\int\limits t \mathrm{\,d}t$}
{\True $A=\displaystyle\int\limits \dfrac{1}{t} \mathrm{\,d}t$}
\loigiai{
Đặt $t=\ln{x}\Rightarrow \mathrm{\,d}t=\dfrac{1}{x}\mathrm{\,d}x$.\\
$A=\displaystyle\int\limits \dfrac{1}{t}\mathrm{\,d}t$.
}
\end{ex}

\begin{ex}%[2H3Y2-2]%
Trong không gian $O x y z$, mặt phẳng $(P): x-2 z+1=0$ có một véc-tơ pháp tuyến là
\choice
{\True $\vec{n}_{1}=(1 ; 0 ;-2)$}
{$\vec{n}_{2}=(1 ;-2 ; 1)$}
{$\vec{n}_{3}=(1 ;-2 ; 0)$}
{$\vec{n}_{4}=(-1 ; 2 ; 0)$}
\loigiai{
$(P): x-2 z+1=0 \Rightarrow \text{ VTPT } \vec{n}=(1;0;-2)$.
}
\end{ex}

\begin{ex}%[2D3Y1-1]%
Họ nguyên hàm của hàm số $f(x)=\dfrac{1}{\cos^2 x}-\dfrac{1}{\sin^2 x}+2$ là
\choice
{\True $\tan x+\cot x+2x+C$}
{$\tan x-\cot x+2x+C$}
{$-\tan x+\cot x+2x+C$}
{$-\tan x-\cot x+2x+C$}
\loigiai{
Ta có $\displaystyle\int f(x) \mathrm{\,d}x=\displaystyle\int \left( \dfrac{1}{\cos^2 x}-\dfrac{1}{\sin^2 x}+2\right)\mathrm{\,d}x=\tan x+\cot x+2x+C$.
}
\end{ex}

\begin{ex}%[2D3B1-3]%
Họ nguyên hàm $F(x)=\displaystyle\int x\mathrm{e}^{2x}\mathrm{d}x$ là
\choice
{$F(x)=(2x-1)\mathrm{e}^{2x}+C$}
{$F(x)=(x-2)\mathrm{e}^{2x}+C$}
{$F(x)=\dfrac{1}{4}(2x+1)\mathrm{e}^{2x}+C$}
{\True $F(x)=\dfrac{1}{2}\left(x-\dfrac{1}{2}\right)\mathrm{e}^{2x}+C$}
\loigiai{
Đặt $\heva{&u=x\\&\mathrm{d}v=\mathrm{e}^{2x}\mathrm{d}x}\Rightarrow\heva{&\mathrm{d}u=\mathrm{d}x\\&v=\dfrac{1}{2}\mathrm{e}^{2x}.}$\\
Khi đó $F(x)=\displaystyle\int x\mathrm{e}^{2x}\mathrm{d}x=\dfrac{1}{2}x\mathrm{e}^{2x}-\displaystyle\int\left(\dfrac{1}{2}\mathrm{e}^{2x}\right)\mathrm{d}x=\dfrac{x\mathrm{e}^{2x}}{2}-\dfrac{\mathrm{e}^{2x}}{4}+C=\dfrac{1}{2}\left(x-\dfrac{1}{2}\right)\mathrm{e}^{2x}+C$.
}
\end{ex}

\begin{ex}%[2H2K1-1]%
Cho hình thang $ABCD$ vuông tại $A$ và $D$, $AD=CD=a$, $AB=2a$. Quay hình thang $ABCD$ quanh cạnh $CD$, thể tích khối tròn xoay thu được bằng
\choice
{$\pi a^3$}
{\True $\dfrac{5\pi a^3}{3}$}
{$\dfrac{\pi a^3}{3}$}
{$\dfrac{4\pi a^3}{3}$}
\loigiai{
\begin{center}
\begin{tikzpicture}[>=stealth,line join=round,line cap=round,font=\footnotesize,scale=0.8,rotate=-90]
\coordinate[label=left:$D$] (A) at (0,0);
\coordinate[label=below:$A$] (B) at (3,0);
\coordinate[label=above:$C$] (D) at (0,3);
\coordinate[label= right:$B$] (C) at (3,7);
\coordinate (E) at (-3,0);
\coordinate (F) at (-3,7);
\coordinate (G) at (0,7);
\fill[pattern=north east lines] (A)--(B)--(C)--(D)--cycle;
\draw
(B)--(C) (E)--(F) (C)--(F) (B)--(E)
(B) arc (0:-180:{3} and {0.5})
(B) arc (0:180:{3} and {0.5})
(C) arc (0:180:{3} and {0.5})
pic [draw=black,fill=gray,angle radius=2mm] {right angle = C--G--A}
pic [draw=black,fill=gray,angle radius=2mm] {right angle = B--C--G}
pic [draw=black,fill=gray,angle radius=2mm] {right angle = C--B--A}
pic [draw=black,fill=gray,angle radius=2mm] {right angle = D--A--B};
\draw[dashed]
(A)--(B) (A)--(D)--(F)--(E) (D)--(C) (D)--(G)
(C)--(F) (D)--(A)--(E)

(C) arc (0:-180:{3} and {0.5})
;
\foreach \diem in {A,B,C,D}	\fill (\diem)circle(1.5pt);
\end{tikzpicture}
\end{center}
Ta có $V=\pi R^2h-\dfrac{1}{3}\pi R^2h=\dfrac{5\pi a^3}{3}$.}
\end{ex}

\begin{ex}%[2D3Y2-1]%
Cho $ \displaystyle \int\limits_1^3 f(x) \mathrm{\,d}x=2 $ và $ \displaystyle \int\limits_1^3 g(x) \mathrm{\,d}x=1 $, khi đó $ \displaystyle \int\limits_1^3 \left[1008 f(x)+2g(x)\right] \mathrm{\,d}x $ bằng
\choice
{$ 2017 $}
{\True $ 2018 $}
{$ 2019 $}
{$ 2020 $}
\loigiai{
Ta có $ \displaystyle \int\limits_1^3 \left[1008 f(x)+2g(x)\right] \mathrm{\,d}x=1008\cdot 2+2\cdot 1=2018 $.
}
\end{ex}

\begin{ex}%[Tổng ôn LVD-GV soạn Mui Doan-GVPB Khanh Lê]%[2D3B2-2]%
Cho hàm số $ f(x) $ có $ f\left(\dfrac{\pi}{2}\right)=\dfrac{1}{2} $ và  $ f'(x)=\dfrac{\cos x}{\sin^3 x}, \forall x\in\left(0;\dfrac{2\pi}{3}\right) $. Khi đó $ \displaystyle\int\limits_{\frac{\pi}{4}}^{\frac{\pi}{2}} f(x)\mathrm{\,d}x $ bằng
\choice
{\True $ \dfrac{\pi}{4}-\dfrac{1}{2} $}
{$ 1 $}
{$ \sqrt{3}-1 $}
{$ \dfrac{\sqrt{3}-1}{2} $}
\loigiai{
Vì $ f'(x)=\dfrac{\cos x}{\sin^3 x}$ nên
\begin{eqnarray*}
f(x)&=& \displaystyle\int \dfrac{\cos x}{\sin^3 x}\mathrm{\,d}x\\
&=& \displaystyle\int \dfrac{\mathrm{\,d}(\sin x)}{\sin^3 x}\\
&=&-\dfrac{1}{2\sin^2 x}+C.
\end{eqnarray*}
Do $ f\left(\dfrac{\pi}{2}\right)=\dfrac{1}{2}\Rightarrow -\dfrac{1}{2}+C=\dfrac{1}{2}\Rightarrow C=1$.\\
Vậy $ f(x)=-\dfrac{1}{2\sin^2 x}+1$.
Ta có
\allowdisplaybreaks
\begin{eqnarray*}
\displaystyle\int\limits_{\frac{\pi}{4}}^{\frac{\pi}{2}} f(x)\mathrm{\,d}x
&=&\displaystyle\int\limits_{\frac{\pi}{4}}^{\frac{\pi}{2}}\left(-\dfrac{1}{2\sin^2 x}+1\right)\mathrm{\,d}x\\
&=&\left(\dfrac{1}{2}\cot x+x \right)\bigg|^{\frac{\pi}{2}}_{\frac{\pi}{4}}\\
&=&0+\dfrac{\pi}{2}-\left(\dfrac{1}{2}+\dfrac{\pi}{4}\right)\\
&=& \dfrac{\pi}{4}-\dfrac{1}{2}.
\end{eqnarray*}
}
\end{ex}

\begin{ex}%[Tổng ôn LVD-GV soạn Mui Doan-GVPB Khanh Lê]%[2D3B2-2]%
Cho hàm số $ f(x) $ có $ f(0)=1 $ và  $ f'(x)=\tan^2 x, \forall x\in\left(-\dfrac{\pi}{2};\dfrac{\pi}{2}\right) $. Khi đó $ \displaystyle\int\limits_{0}^{\frac{\pi}{4}} f(x)\mathrm{\,d}x $ bằng
\choice
{$  \dfrac{1}{2}\ln 2- \dfrac{\pi^2}{16}+ \dfrac{\pi}{4} $}
{$  \dfrac{1}{2}\ln 2 + \dfrac{\pi}{4}$}
{\True $  \dfrac{1}{2}\ln 2- \dfrac{\pi^2}{32}+ \dfrac{\pi}{4} $}
{$  \dfrac{1}{2} $}
\loigiai{
Vì $ f'(x)=\tan^2 x$ nên
\begin{eqnarray*}
f(x)&=& \displaystyle\int \tan^2 x\mathrm{\,d}x\\
&=&\displaystyle\int (\tan^2 x+1-1)\mathrm{\,d}x\\
&=&\tan x-x+C.
\end{eqnarray*}
Do $ f(0)=1\Rightarrow 0+C=1\Rightarrow C=1$.\\
Vậy $ f(x)=\tan x-x+1$.
Ta có
\allowdisplaybreaks
\begin{eqnarray*}
\displaystyle\int\limits_{0}^{\frac{\pi}{4}} f(x)\mathrm{\,d}x
&=&\displaystyle\int\limits_{0}^{\frac{\pi}{4}}\left(\dfrac{\sin x}{\cos x}-x+1\right)\mathrm{\,d}x\\
&=&-\displaystyle\int\limits_{0}^{\frac{\pi}{4}}\dfrac{\mathrm{\,d}(\cos x)}{\cos x}-\displaystyle\int\limits_{0}^{\frac{\pi}{4}} x\mathrm{\,d}x+\displaystyle\int\limits_{0}^{\frac{\pi}{4}} 1\mathrm{\,d}x\\
&=&-\ln\vert \cos x \vert \bigg|^{\frac{\pi}{4}}_{0}-\dfrac{x^2}{2}\bigg|^{\frac{\pi}{4}}_{0}+x\bigg|^{\frac{\pi}{4}}_{0}\\
&=&-\ln \dfrac{1}{\sqrt{2}}+\ln 1-\dfrac{\pi^2}{32}+\dfrac{\pi}{4}\\
&=&\ln \sqrt{2}-\dfrac{\pi^2}{32}+\dfrac{\pi}{4}\\
&=& \dfrac{1}{2}\ln 2-\dfrac{\pi^2}{32}+\dfrac{\pi}{4}.
\end{eqnarray*}
}
\end{ex}

\begin{ex}%[2H3Y2-3]%
Trong không gian $O x y z$, mặt phẳng $(O y z)$ có phương trình là
\choice
{\True $x=0$}
{$z=0$}
{$x+y+z=0$}
{$y=0$}
\loigiai{
Mặt phẳng $(O y z): x=0$.
}
\end{ex}

\begin{ex}%[2H3B2-3]%
Trong không gian $O x y z$, cho hai điểm $A(1;1;2)$ và $B(2;0;1)$. Mặt phẳng đi qua $A$ và vuông góc với $A B$ có phương trình là
\choice
{$x+y-z=0$}
{$x-y-z-2=0$}
{$x+y+z-4=0$}
{\True $x-y-z+2=0$}
\loigiai{
Mặt phẳng đi qua $A$ và vuông góc với $A B$ nên có véc-tơ pháp tuyến $\vec{n}=\overrightarrow{AB}=(1;-1;-1)$. Do đó ta có phương trình là
$$1(x-1)-1(y-1)-1(z-2)=0\Leftrightarrow x-y-z+2=0.$$
}
\end{ex}

\begin{ex}%[2H3B1-3]%
Trong không gian $O x y z$, phương trình mặt cầu $(S)$ có tâm $I(1 ; 2 ; 3)$ và tiếp xúc với mặt phẳng $(Oxy)$ là
\choice
{$(x+1)^{2}+(y+2)^{2}+(z+3)^{2}=9$}
{$(x-1)^{2}+(y-2)^{2}+(z-3)^{2}=14$}
{$(x+1)^{2}+(y+2)^{2}+(z+3)^{2}=14$}
{\True $(x-1)^{2}+(y-2)^{2}+(z-3)^{2}=9$}
\loigiai{
Bán kính $R=\mathrm{d}(I,(Oxy))=|3|=3$.\\
Phương trình mặt cầu là $(x-1)^{2}+(y-2)^{2}+(z-3)^{2}=9$.
}
\end{ex}

\begin{ex}%[2D3B1-1]%
Biết $F(x)=(a x^2+b x+c) \sqrt{2x-3}$ là nguyên hàm của hàm số $f(x)=\dfrac{20 x^2-30 x+11}{\sqrt{2x-3}} \cdot$ Giá trị của $a+b+c$ bằng
\choice
{$5$}
{$6$}
{\True $7$}
{$8$}
\loigiai{
Ta có
\begin{eqnarray*}
f(x)&=&F'(x)=(2ax+b)\sqrt{2x-3}+(ax^2+bx+c)\cdot \dfrac{1}{\sqrt{2x-3}}\\
&=& \dfrac{(2ax+b)(2x-3)+ax^2+bx+c}{\sqrt{2x-3}}\\
&=& \dfrac{5ax^2+(3b-6a)x-3b+c}{\sqrt{2x-3}}.
\end{eqnarray*}
Suy ra $\heva{&5a=20\\&3b-6a=-30\\&-3b+c=11}\Rightarrow \heva{&a=4\\&b=-2\\&c=5.}$\\
Giá trị của $a+b+c=7$.
}
\end{ex}

\begin{ex}%[2D3B2-1]%
Cho $ \displaystyle \int\limits_{-1}^2 f(x) \mathrm{\,d}x=2 $ và $ \displaystyle \int\limits_{-1}^2 g(x) \mathrm{\,d}x=-1 $, khi đó $ \displaystyle \int\limits_{-1}^2 \left[x+2f(x)-3g(x)\right] \mathrm{\,d}x $ bằng
\choice
{$ \dfrac{5}{2} $}
{$ \dfrac{7}{2} $}
{\True $ \dfrac{17}{2} $}
{$ \dfrac{11}{2} $}
\loigiai{
Ta có $ \displaystyle \int\limits_{-1}^2 \left[x+2f(x)-3g(x)\right] \mathrm{\,d}x =\dfrac{x^2}{2}\Big|_{-1}^2 +2\cdot 2-3\cdot (-1)=\dfrac{17}{2}$.
}
\end{ex}

\begin{ex}%[tex hóa Tổng ôn LVĐ 21-22]%[2H3B1-2]%
Trong không gian $Oxyz$, cho $\vec{u} =(1;1;2)$, $\vec{v} =(2;0;m)$. Giá trị của tham số $m$ để $\cos \left(\vec{u}, \vec{v} \right)= \dfrac{4}{\sqrt{30}}$ là \choice
{\True $m=1 $}
{$m=1 $; $m=-11$}
{$ m=-11 $}
{$m=0$}
\loigiai{
Ta có $\cos \left(\vec{u}, \vec{v} \right)= \dfrac{\vec{u} \cdot \vec{v}}{\left|\vec{u} \right|\cdot \left|\vec{u} \right|}= \dfrac{2+ 2m}{\sqrt{6}\cdot \sqrt{4+m^2}}.$\\
Theo giả thiết ta có $$\cos \left(\vec{u}, \vec{v} \right)= \dfrac{4}{\sqrt{30}} \Leftrightarrow \dfrac{2+ 2m}{\sqrt{24+ 6m^2}} = \dfrac{4}{\sqrt{30}}\Leftrightarrow \heva{&m\ge -1 \\& 30(1+m)^2 = 4(24+ 6m^2)}\Leftrightarrow \heva{& m\ ge -1 \\& \hoac{& m=1 \\& m=-11}}\Leftrightarrow m=1. $$}
\end{ex}

\begin{ex}%[2H3K1-1]%
Trong không gian với hệ trục tọa độ $Oxyz$, cho $\triangle ABC$ có tọa độ các điểm $A(-2;2;0)$, $B(1;-2;0), C(2;0;4)$. Gọi $D(a;b;c)$ là chân đường phân giác của góc $\widehat{BAC}$. Giá trị của biểu thức $T=a+b+c$ bằng:
\choice
{\True $\dfrac{24}{11}$}
{$\dfrac{36}{11}$}
{$\dfrac{12}{11}$}
{$\dfrac{48}{11}$}
\loigiai{
\immini
{
Nhận thấy:
\allowdisplaybreaks
\begin{eqnarray*}
&&\dfrac{\overrightarrow{D B}}{A B}+\dfrac{\overrightarrow{D C}}{A C}=\overrightarrow 0\\
&\Leftrightarrow&\dfrac{\overrightarrow{D B}}5+\dfrac{\overrightarrow{D C}}6=\overrightarrow 0\\
&\Leftrightarrow& 6\overrightarrow{D B}+5\overrightarrow{D C}=\overrightarrow 0\\
&\Rightarrow& D=\dfrac{6B+5C}{6+5}=\left(\dfrac{16}{11};-\dfrac{12}{11};\dfrac{20}{11}\right)=(a;b;c)
\end{eqnarray*}
Suy ra $T=a+b+c=\dfrac{24}{11}$.
}
{
\begin{tikzpicture}
\def\ab{3}
\def\bc{5}
\def\gocB{60}
\path
(0,0) coordinate (B)
(\gocB:\ab) coordinate (A)
(\bc,0) coordinate (C)
($(A)!1mm!(B)$) coordinate (B1)
($(A)!1mm!(C)$) coordinate (C1)
($(B1)!.5!(C1)$) coordinate (D1)
($(A)!2!(D1)$) coordinate (D2)
(intersection of A--D2 and B--C) coordinate (D)
;
\draw
pic[draw, angle radius=5mm]{angle=B--A--D}
pic[draw, angle radius=4mm]{angle=D--A--C}
;
\draw (A)--(B)--(C)--cycle (A)--(D);
\foreach \x/\gm in {A/90,B/-90,C/-90,D/-90} \fill (\x) circle (1.5pt) ($(\x)+(\gm:3mm)$)node{$\x$};
\end{tikzpicture}
}
}
\end{ex}

\begin{ex}%[2H3B1-3]%
Trong không gian $O x y z$, cho hai điểm $A(2 ; 1 ; 1), B(0 ; 3 ;-1)$. Mặt cầu $(S)$ đường kính $A B$ có phương trình là
\choice
{$x^{2}+(y-2)^{2}+z^{2}=3$}
{\True $(x-1)^{2}+(y-2)^{2}+z^{2}=3$}
{$(x-1)^{2}+(y-2)^{2}+(z+1)^{2}=9$}
{$(x-1)^{2}+(y-2)^{2}+z^{2}=9$}
\loigiai{
Tâm $I$ mặt cầu là trung điểm của $AB$, ta có $I(1;2;0)$.\\
Bán kính mặt cầu $R=\dfrac{AB}{2}=\sqrt{3}$.\\
Phương trình mặt cầu là $(x-1)^{2}+(y-2)^{2}+z^{2}=3$.
}
\end{ex}

\begin{ex}%[Nguyễn Thành Tiến- Dự án Tổng ôn LVĐ 2022]%[2H3B1-1]%
Trong không gian $Oxyz$, cho các véctơ $\vec{a}=\left(-1;1;0\right)$, $\vec{b}=\left(1;1;0\right)$ và $\vec{c}=\left(1;1;1\right)$. Mệnh đề nào dưới đây sai
\choice
{$\vec{a} \perp \vec{b}$}
{$\left|\vec{c}\right|=\sqrt{3}$}
{\True $\vec{c}\perp \vec{b}$}
{$\left|a\right|=\sqrt{2}$}
\loigiai{
Ta có
\begin{itemize}
\item Ta thấy $\vec{a}\cdot \vec{b}=0\Rightarrow \vec{a}\perp \vec{b}$ là đúng.
\item $\left| \vec{c}\right|=\sqrt{3}$ là đúng.
\item $\vec{c}\cdot \vec{b}\ne 0\Rightarrow \vec{c}\not\perp \vec{b}$. Vậy $\vec{c}\perp \vec{b}$ là sai.
\item $\left|\vec{a}\right|=\sqrt{2}$ là đúng.
\end{itemize}
}
\end{ex}

\begin{ex}%[tex hóa Tổng ôn LVĐ 21-22]%[2H3B1-2]%]%
Trong không gian $Oxyz$, cho hình bình hành  $ABCD$ với  $A(1;1;1)$, $B(2;3;4)$ và $C(6;5;2)$. Diện tích hình bình hành $ABCD$  bằng \choice
{$3\sqrt{83} $}
{$\sqrt{83} $}
{$ 83 $}
{\True $2\sqrt{83}$}
\loigiai{ Ta có $\vec{AB}= (1;2;3)$, $\vec{AC}= (5;4;1)$. Suy ra $\left[\vec{AB}, \vec{AC} \right]= (-10;14;-6)$.\\
Suy ra $S_{ABCD} =\left|\left[\vec{AB}, \vec{AC} \right] \right|=2\sqrt{83}$.}
\end{ex}

\begin{ex}%[tex hóa Tổng ôn LVĐ 21-22]%[2H3B1-2]%
Trong không gian $Oxyz$, cho hình bình hành  $ABCD$ với  $A(2;4;0)$, $B(4;0;0)$,  $C(-1;4;-7)$ và $D(-3;8;-7)$. Diện tích hình bình hành $ABCD$  bằng \choice
{$\sqrt{281} $}
{$\sqrt{181} $}
{\True $ 2\sqrt{181} $}
{$2\sqrt{83}$}
\loigiai{ Ta có $\vec{AB}= (2;-4;0)$, $\vec{AC}= (-3;0;-7)$. Suy ra $\left[\vec{AB}, \vec{AC} \right]= (28;14;-12)$.\\
Suy ra $S_{ABCD} =\left|\left[\vec{AB}, \vec{AC} \right] \right|=2\sqrt{281}$.}

\end{ex}

\begin{ex}%[2D3Y1-1]%
Họ nguyên hàm của hàm số $f(x)=7 x^{6}+\dfrac{1}{x}+\dfrac{1}{x^2}-2$ là
\choice
{$x^{7}+\ln |x|+\dfrac{1}{x}-2x+C$}
{$x^{7}+\ln |x|-\dfrac{1}{x}-2x$}
{\True $x^{7}+\ln |x|-\dfrac{1}{x}-2x+C$}
{$x^{7}+\ln x+\dfrac{1}{x}-2x+C$}
\loigiai{
Ta có $\displaystyle\int f(x) \mathrm{\,d}x=\displaystyle\int \left( 7x^6+\dfrac{1}{x}+\dfrac{1}{x^2}-2\right) \mathrm{\,d}x=x^7+\ln |x|-\dfrac{1}{x}-2x+C$.
}
\end{ex}

\begin{ex}%[Tổng ôn LVD-GV soạn Mui Doan-GVPB Khanh Lê]%[2D3B2-3]%
Cho hàm số $ f(x) $ có $ f(0)=3 $ và  $ f'(x)=\sqrt{\mathrm{e}^{x}}, \forall x\in\mathbb{R} $. Khi đó $ \displaystyle\int\limits_0^2 f(x)\mathrm{\,d}x $ bằng
\choice
{$ 1+\dfrac{1}{2}\ln 2 $}
{\True $4\mathrm{e}-2$}
{$\mathrm{e}-\dfrac{1}{2}$}
{$\mathrm{e}+\dfrac{1}{2}$}
\loigiai{
Vì $ f'(x)=\sqrt{\mathrm{e}^{x}}=\mathrm{e}^{\frac{x}{2}}$ nên	$ f(x)=\displaystyle\int f'(x)\mathrm{\,d}x=2\mathrm{e}^{\frac{x}{2}}+C $.\\
Do $ f(0)=3\Rightarrow 2+C=3\Rightarrow C=1$.\\
Vậy $ f(x)=2\mathrm{e}^{\frac{x}{2}}+1 $.
Ta có
\allowdisplaybreaks
\begin{eqnarray*}
\displaystyle\int\limits_0^2 f(x)\mathrm{\,d}x &=&\displaystyle\int\limits_0^2 \left(2\mathrm{e}^{\frac{x}{2}}+1\right)\mathrm{\,d}x \\
&=&\left(4\mathrm{e}^{\frac{x}{2}}+x\right)\bigg|^2_0\\
&=&\left(4\mathrm{e}+2\right)-4\\
&=&4\mathrm{e}-2.
\end{eqnarray*}
}
\end{ex}

\begin{ex}%[2H3B2-3]%
Trong không gian $Oxyz$, mặt phẳng đi qua điểm $A(1 ; 3 ;-2)$ và song song với mặt phẳng $(P): 2 x-y+3 z+4=0$ là
\choice
{$2 x+y+3 z+7=0$}
{$2 x+y-3 z+7=0$}
{\True $2 x-y+3 z+7=0$}
{$2 x-y+3 z-7=0$}
\loigiai{
Gọi mặt phẳng cần tìm là $(\alpha)$.\\
Vì $(\alpha)\parallel(P): 2 x-y+3 z+4=0$ nên $(\alpha): 2x-y+3z+d=0$ với $d\not=4$.\\
Mặt phẳng $(\alpha)$ đi qua $A(1;3;-2)$ nên $A(1;3;-2)\in (\alpha): 2x-y+3z+d=0\Rightarrow 2\cdot 1-3+3\cdot (-2)+d=0\Rightarrow d=7$ (nhận).\\
Vậy $(\alpha): 2x-y+3z+7=0$.
}
\end{ex}

\begin{ex}%[2H2K1-1]%
Cho một hình quạt có bán kính ${R}=12$ và góc ở tâm bằng $120^{\circ}$. Ta uốn hình quạt này thành một hình nón thì thể tích hình nón thu được bằng
\choice
{\True$\dfrac{128 \pi \sqrt{2}}{3}$}
{$\dfrac{128 \pi}{3}$}
{$\dfrac{108 \pi \sqrt{3}}{3}$}
{$120 \pi$}
\loigiai{Bán kính quạt bằng độ dài đường sinh của nón $l=R=12.$\\Độ dài cung tròn của quạt đúng bằng chu vi đáy của nón (bán kính đáy nón là $r$)
$$\dfrac{2 \pi}{3} \cdot R=2 \pi r \Rightarrow r=\dfrac{R}{3}=\dfrac{12}{3}=4$$
Chiều cao là $h=\sqrt{l^{2}-r^{2}}=\sqrt{12^{2}-4^{2}}=8 \sqrt{2}$
\\Do đó thể tích hình nón là $V=\dfrac{1}{3} \pi r^{2} \cdot h=\dfrac{1}{3} \pi(4)^{2} \cdot 8 \sqrt{2}=\dfrac{128 \pi \sqrt{2}}{3}.$  }
\end{ex}

\begin{ex}%[2D3B2-3]%Câu 467.
Nếu $\displaystyle\int_1^a\ln x \mathrm{\, d}x=1+2 a$ với $a>1$ thì $a$ thuộc khoảng nào sau đây?
\choice
{\True $(18; 21)$}
{$(1; 4)$}
{$(11; 14)$}
{$(6; 9)$}
\loigiai{
Đặt $\heva{& u=\ln x\Rightarrow \mathrm{\,d}u=\dfrac{1}{x}\mathrm{\,d}x \\ & \mathrm{\,d}v= \mathrm{\,d}x\Rightarrow v=x.}$\\
Khi đó $\displaystyle\int_1^a\ln x \mathrm{\,d}x=x\cdot\ln x\bigg|_1^a-\displaystyle\int_1^a x\cdot\dfrac{1}{x} \mathrm{\,d}x=a\ln a-x\bigg|_1^a=a\ln a-a+1$.\\
Theo giả thiết, ta có $a\ln a-a+1=1+2a\Leftrightarrow \ln a=3\Leftrightarrow a=\mathrm{e}^3\in(18;21)$.
}
\end{ex}

\begin{ex}%[Tổng ôn LVD-GV soạn Mui Doan-GVPB Khanh Lê]%[2D3Y1-1]%
Cho $ F(x) $ là một nguyên hàm của hàm số $ f(x)= \mathrm{e}^x+2x$ thỏa $ F(0)=\dfrac{3}{2} $. Tìm $ F(x) $.
\choice
{$ F(x)=\mathrm{e}^x+x^2+\dfrac{5}{2} $}
{$ F(x)=2\mathrm{e}^x+x^2-\dfrac{1}{2} $}
{$ F(x)=\mathrm{e}^x+x^2+\dfrac{3}{2} $}
{\True $ F(x)=\mathrm{e}^x+x^2+\dfrac{1}{2} $}
\loigiai{
Vì $ f(x)= \mathrm{e}^x+2x\Rightarrow F(x)=\mathrm{e}^x+x^2+C$ .\\
Do $ F(0)=\dfrac{3}{2}\Rightarrow 1+C= \dfrac{3}{2}\Rightarrow C=\dfrac{1}{2}$.\\
Vậy $ F(x)=\mathrm{e}^x+x^2+\dfrac{1}{2} $.
}
\end{ex}

\begin{ex}%[2D3B2-3]%Câu 464.
Cho hai số thực $a$ và $b$ thỏa $a<b$ và $\displaystyle\int_a^b x\sin x \mathrm{\,d}x=\pi$, đồng thời $a$ cos $a=0$ và $b\cos b=-\pi$. Khi đó $\displaystyle\int_a^b\cos x \mathrm{\,d}x$ bằng
\choice
{$\dfrac{\pi}{2}$}
{$\pi$}
{$-\pi$}
{\True $0$}
\loigiai{
Đặt $\heva{& u=x\Rightarrow \mathrm{\,d}u=\mathrm{\,d}x \\ & \mathrm{\,d}v=\sin x \mathrm{\,d}x\Rightarrow v=-\cos x.}$\\
Khi đó $\displaystyle\int_a^b x\sin x \mathrm{\,d}x=-x\cos x\bigg|_a^b+\displaystyle\int_a^b \cos x \mathrm{\,d}x=a\cos a-b\cos b+\displaystyle\int_a^b \cos x \mathrm{\,d}x=\pi+\displaystyle\int_a^b \cos x \mathrm{\,d}x=0$.\\
Suy ra $\displaystyle\int_a^b \cos x \mathrm{\,d}x=0$.
}
\end{ex}

\begin{ex}%[Tổng ôn LVD-GV soạn Mui Doan-GVPB Khanh Lê]%[2D3B2-2]%
Xét  $ \displaystyle\int\limits_{0}^{2}x\mathrm{e}^{x^2}\mathrm{\,d}x$, nếu đặt $ u=x^2 $ thì $ \displaystyle\int\limits_{0}^{2}x\mathrm{e}^{x^2}\mathrm{\,d}x$ bằng
\choice
{$ 2 \displaystyle\int\limits_{0}^{2}\mathrm{e}^{u}\mathrm{\,d}u$}
{$ 2 \displaystyle\int\limits_{0}^{4}\mathrm{e}^{u}\mathrm{\,d}u$}
{$ \dfrac{1}{2} \displaystyle\int\limits_{0}^{2}\mathrm{e}^{u}\mathrm{\,d}u$}
{\True $  \dfrac{1}{2}\displaystyle\int\limits_{0}^{4}\mathrm{e}^{u}\mathrm{\,d}u$}
\loigiai{
Nếu đặt $ u=x^2 \Rightarrow \mathrm{\,d}u=2x\mathrm{\,d}x$.\\
Đổi cận\\
$ x=0\Rightarrow u=0 $.\\
$ x=2\Rightarrow u=4 $.\\
Ta có
\allowdisplaybreaks
\begin{eqnarray*}
\displaystyle\int\limits_{0}^{2}x\mathrm{e}^{x^2}\mathrm{\,d}x
&=&\displaystyle\int\limits_{0}^{4}\dfrac{1}{2}\mathrm{e}^u\mathrm{\,d}u\\
&=&\dfrac{1}{2}\displaystyle\int\limits_{0}^{4}\mathrm{e}^u\mathrm{\,d}u.
\end{eqnarray*}
}
\end{ex}

\begin{ex}%[Dự án Tex TDM - NHTP - Lê Quân]%[2D3B2-1]%
Cho $\displaystyle\int\limits_{0}^{3}\dfrac{(x+2)}{x^2+4x+3}\mathrm{\, d}x=\dfrac{a}{b}\cdot \ln 2$ với $a, b$ là các số nguyên dương và phân số $\dfrac{a}{b}$ tối giản. Giá trị của biểu thức $T=2a+3b$ tương ứng bằng
\choice
{$6$}
{$10$}
{$13$}
{\True $12$}
\loigiai{
Ta có
\begin{eqnarray*}
\displaystyle\int\limits_{0}^{3}\dfrac{(x+2)}{x^2+4x+3}\mathrm{\, d}x&=& \displaystyle\int\limits_{0}^{3}\dfrac{x+2}{(x+1)(x+3)}\mathrm{\, d}x\\
&=&\dfrac{1}{2}\displaystyle\int\limits_{0}^{3}\left[\dfrac{1}{x+1}+\dfrac{1}{x+3}\right]\mathrm{\, d}x\\
&=& \dfrac{1}{2}\ln(x+1)(x+3)\bigg|_0^3\\
&=&\dfrac{3}{2}\ln 2.
\end{eqnarray*}
Từ đó ta có $\heva{&a=3	\\&b=2}\Rightarrow T=2a+3b=12$.
}
\end{ex}

\begin{ex}%[2D3B2-3]%
Giả sử hàm số $f(x)$ có đạo hàm liên tục trên đoạn $[0; 1]$ thỏa mãn điêu kiện $f(1)=6$, $\displaystyle\int\limits_0^1 x f'(x)\mathrm{\,d}x=5$. Tính $I=\displaystyle\int\limits_0^1 f(x)\mathrm{\,d}x$.
\choice
{\True $1$}
{$\dfrac{1}{2}$}
{$3$}
{$11$}
\loigiai{
Đặt $\heva{&u=x\Rightarrow \mathrm{\,d}u=\mathrm{\,d}x\\&\mathrm{\,d}v=f'(x) \mathrm{\,d}x\Rightarrow v=f(x).}$\\
Suy ra $\displaystyle\int\limits_0^1 xf'(x) \mathrm{\,d}x=xf(x) \Big|_0^1-\displaystyle\int\limits_0^1 f(x)\mathrm{\,d}x=f(1)-\displaystyle\int\limits_0^1 f(x) \mathrm{\,d}x$.\\
Suy ra $\displaystyle\int\limits_0^1 f(x) \mathrm{\,d}x=f(1)-\displaystyle\int\limits_0^1 xf'(x) \mathrm{\,d}x=6-5=1$.
}
\end{ex}

\begin{ex}%[2D3Y1-1]%
Nguyên hàm của hàm số $f(x)=x^2+\dfrac{2}{x^2}$ là
\choice
{\True $\dfrac{x^3}{3}-\dfrac{2}{x}+C$}
{$\dfrac{x^3}{3}-\dfrac{1}{x}+C$}
{$\dfrac{x^3}{3}+\dfrac{2}{x}+C$}
{$\dfrac{x^3}{3}+\dfrac{1}{x}+C$}
\loigiai{
Ta có $\displaystyle\int f(x) \mathrm{\,d}x=\displaystyle\int \left( x^2+\dfrac{2}{x^2}\right) \mathrm{\,d}x=\dfrac{x^3}{3}-\dfrac{2}{x}+C$.
}
\end{ex}

\begin{ex}%[2D3Y2-1]%
Cho hàm sõ $f(x)=\ln \left|x+\sqrt{x^2+1}\right|$, khi đó $\displaystyle\int\limits_0^1 f'(x)\mathrm{\,d}x$ bằng
\choice
{$\ln \sqrt{2}$}
{\True $\ln (1+\sqrt{2})$}
{$1+\ln \sqrt{2}$}
{$2 \ln 2$}
\loigiai{
Ta có $\displaystyle\int\limits_0^1 f'(x) \mathrm{\,d}x= f(x)\Big|_0^1=\ln \left|x+\sqrt{x^2+1}\right| \Big|_0^1=\ln (1+\sqrt{2})$.
}
\end{ex}

\begin{ex}%[Tổng ôn LVD-GV soạn Mui Doan-GVPB Khanh Lê]%[2D3B1-1]%
Cho hàm số $ f(x) $ có $ f(0)=-2 $ và  $ f'(x)=\left(2\mathrm{e}^x-3\right)\mathrm{e}^x, \forall x\in\mathbb{R} $. Khi đó tổng các nghiệm của phương trình $ f(x)=-2 $ bằng
\choice
{\True $ \ln 2 $}
{$ \dfrac{1}{2} $}
{$ \mathrm{e} $}
{$ \mathrm{e}+ \dfrac{1}{2} $}
\loigiai{
Vì $ f'(x)=\left(2\mathrm{e}^x-3\right)\mathrm{e}^x$ nên
\allowdisplaybreaks
\begin{eqnarray*}
f(x)&=&\displaystyle\int\left(2\mathrm{e}^x-3\right)\mathrm{e}^x\mathrm{d}x\\
&=&\displaystyle\int\left(2\mathrm{e}^{2x}-3\mathrm{e}^x\right)\mathrm{d}x\\
&=&\mathrm{e}^{2x}-3\mathrm{e}^x+C.
\end{eqnarray*}
Do $ f(0)=-2\Rightarrow -2+C=-2\Rightarrow C=0$.\\
Vậy $ f(x)=\mathrm{e}^{2x}-3\mathrm{e}^x$.\\
Ta có
\allowdisplaybreaks
\begin{eqnarray*}
&& f(x)=-2\\
&\Leftrightarrow&\mathrm{e}^{2x}-3\mathrm{e}^x+2=0\\
&\Leftrightarrow&\hoac{&\mathrm{e}^x=1\\&\mathrm{e}^x=2}\\
&\Leftrightarrow&\hoac{&x=0\\&x=\ln 2.}\\
\end{eqnarray*}
Vậy tổng các nghiệm của phương trình là $ \ln 2  $.
}
\end{ex}
\Closesolutionfile{ans}