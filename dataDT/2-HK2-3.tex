\begin{name}
{\tenchude}{\tendethi}{LỚP TOÁN THẦY PHÁT}{\thoigian}
\end{name}
\Opensolutionfile{ans}[ans/ansBTTeX1]
\begin{ex}%[2D3Y1-1]%
Cho hàm số $f(x)=\mathrm{e}^{3x-1}$. Trong các khẳng định sau, khẳng định nào đúng?
\choice
{$\displaystyle\int f(x) \mathrm{\,d}x=3 \mathrm{e}^{3x-1}+C$}
{$\displaystyle\int f(x) \mathrm{\,d}x=-\dfrac{1}{3} \mathrm{e}^{3x-1}+C$}
{\True $\displaystyle\int f(x) \mathrm{\,d}x=\dfrac{1}{3} \mathrm{e}^{3x-1}+C$}
{$\displaystyle\int f(x) \mathrm{\,d}x=\mathrm{e}^{3x-1}+C$}
\loigiai{
Ta có $\displaystyle\int f(x) \mathrm{\,d}x=\displaystyle\int \mathrm{e}^{3x-1} \mathrm{~d}x=\dfrac{1}{3} \mathrm{e}^{3x-1}+C$.
}
\end{ex}

\begin{ex}Câu 15%[2D3Y1-1]%
Cho hàm số $f(x)=\dfrac{1}{\sin^2x}$. Trong các khẳng định sau, khẳng định nào đúng?
\choice
{$\displaystyle\int f(x)\mathrm{\,d}x=\cot x+C$}
{$\displaystyle\int f(x)\mathrm{\,d}x=\tan x+C$}
{\True $\displaystyle\int f(x)\mathrm{\,d}x=-\cot x+C$}
{$\displaystyle\int f(x)\mathrm{\,d}x=-\tan x+C$}
\loigiai{
Ta có $\displaystyle\int f(x)\mathrm{\,d}x=\displaystyle\int \dfrac{1}{\sin^2x}\mathrm{\,d}x=-\cot x+C$.}
\end{ex}

\begin{ex}%[2D3B1-1]%
Họ nguyên hàm của hàm số $f(x)=(2+\mathrm{e}^{3x})^2$ là
\choice
{\True $4 x+\dfrac{4}{3} \mathrm{e}^{3x}+\dfrac{1}{6} \mathrm{e}^{6 x}+C$}
{$3x+\dfrac{4}{3} \mathrm{e}^{3x}+\dfrac{1}{6} \mathrm{e}^{6 x}+C$}
{$4 x+\dfrac{4}{3} \mathrm{e}^{3x}-\dfrac{1}{6} \mathrm{e}^{6 x}+C$}
{$3x+\dfrac{4}{3} \mathrm{e}^{3x}+\dfrac{5}{6} \mathrm{e}^{6 x}+C$}
\loigiai{
Ta có $\displaystyle\int f(x) \mathrm{\,d}x=\displaystyle\int (4+4\mathrm{e}^{3x}+\mathrm{e}^{6x}) \mathrm{\,d}x=4x+\dfrac{4}{3}\mathrm{e}^{3x}+\dfrac{1}{6}\mathrm{e}^{6x}+C$.
}
\end{ex}

\begin{ex}%[2D3B1-1]%
Nếu $\displaystyle\int f(x)\mathrm{\,d}x=\dfrac{x^{4}}{4}+\ln x+C$ thì $f(x)$ bằng
\choice
{\True $x^3+\dfrac{1}{x}$}
{$x^3+\ln x$}
{$\dfrac{x^{4}}{3}+\dfrac{1}{x}$}
{$\dfrac{x^{4}}{12}+\ln x$}
\loigiai{
Ta có $f(x)=\left(\dfrac{x^4}{4}+\ln x+C\right)'=x^3+\dfrac{1}{x}$.
}
\end{ex}

\begin{ex}%[2D3B1-2]%
Họ nguyên hàm $F(x)=\displaystyle\int\dfrac{1}{x^2-16}\mathrm{d}x$ là
\choice
{\True $F(x)=\dfrac{1}{8}\ln\left|\dfrac{x-4}{x+4}\right|+C$}
{$F(x)=\dfrac{1}{8}\ln\left|\dfrac{x+4}{x-4}\right|+C$}
{$F(x)=\dfrac{1}{2}\ln\left|\dfrac{x-4}{x+4}\right|+C$}
{$F(x)=\dfrac{1}{2}\ln\left|\dfrac{x+4}{x-4}\right|+C$}
\loigiai{
Ta có
\begin{eqnarray*}
F(x)&=&\displaystyle\int\dfrac{1}{x^2-16}\mathrm{d}x\\
&=&\displaystyle\int\dfrac{1}{(x-4)(x+4)}\mathrm{d}x\\
&=&\dfrac{1}{8}\displaystyle\int\left(\dfrac{1}{x-4}-\dfrac{1}{x+4}\right)\mathrm{d}\\
&=&\dfrac{1}{8}\left(\ln|x-4|-\ln|x+4|\right)+C\\
&=&\dfrac{1}{8}\ln\left|\dfrac{x-4}{x+4}\right|+C.
\end{eqnarray*}
}
\end{ex}

\begin{ex}%[2D3B1-3]%
Họ nguyên hàm $F(x)=\displaystyle\int(3x-1)\mathrm{e}^x\mathrm{d}x$ là
\choice
{$F(x)=(3x-1)\mathrm{e}^x+C$}
{\True $F(x)=(3x-4)\mathrm{e}^x+C$}
{$F(x)=\dfrac{1}{3}(3x+1)\mathrm{e}^x+C$}
{$F(x)=\left(\dfrac{3}{2}x^2-x\right)\mathrm{e}^x+C$}
\loigiai{
Đặt $\heva{&u=3x-1\\&\mathrm{d}v=\mathrm{e}^x\mathrm{d}x}$, suy ra $\heva{&\mathrm{d}u=3\mathrm{d}x\\&v=\mathrm{e}^x.}$\\
Khi đó $F(x)=(3x-1)\mathrm{e}^x-\displaystyle\int3\mathrm{e}^x\mathrm{d}x=(3x-4)\mathrm{e}^x+C$.
}
\end{ex}

\begin{ex}  Câu 32.%[2D3Y2-1]%
Cho hàm số $ f(x)$ có đạo hàm liên tục trên đoạn $[1;3]$ thỏa mãn $ f(1)=2$ và $ f(3)=9$. Tính $I=\int\limits_1^3{f'(x)\mathrm{\,d}x}$.
\choice
{\True $I=7$}
{$I=18$}
{$I=2$}
{$I=11$}
\loigiai{
$I=\int\limits_1^3{f'(x)\mathrm{\,d}x}=f(x)\big|_1^3=f(3)-f(1)=9-2=7$.}
\end{ex}

\begin{ex}Câu 10%[2D3Y2-1]%
Nếu $\displaystyle\int\limits_{-1}^3 f(x)\mathrm{\,d}x=3$ và $\displaystyle\int\limits_3^5 f(x)\mathrm{\,d}x=2$ thì $\displaystyle\int\limits_{-1}^5 f(x)\mathrm{\,d}x$ bằng
\choice
{$-5$}
{\True $5$}
{$-1$}
{$1$}
\loigiai{
Ta có $\displaystyle\int\limits_{-1}^5 f(x)\mathrm{\,d}x=\displaystyle\int\limits_{-1}^3 f(x)\mathrm{\,d}x+\displaystyle\int\limits_3^5 f(x)\mathrm{\,d}x=5$.
}
\end{ex}

\begin{ex}Câu 32%[2D3B2-1]%
Cho hàm số $f(x)$, biết $f(0)=4$ và $f'(x)=2\cos^2x+1$, $\forall x \in \mathbb{R}$. Khi đó $\displaystyle\int\limits_0^{\tfrac{\pi}{4}} f(x)\mathrm{\,d}x$ bằng
\choice
{$\dfrac{\pi^2+4}{16}$}
{$\dfrac{\pi^2+14\pi}{16}$}
{\True $\dfrac{\pi^2+16\pi +4}{16}$}
{$\dfrac{\pi^2+16\pi +16}{16}$}
\loigiai{
Ta có $f(x)=\displaystyle\int f'(x)\mathrm{\,d}x=\displaystyle\int \left(2\cos^2x+1\right)\mathrm{d}x=\displaystyle\int \left(\cos 2x+2\right)\mathrm{d}x=\dfrac{1}{2}\sin 2x+2x+C$\\
Mà $f(0)=4 \Leftrightarrow C=4$.\\
Vậy $\displaystyle\int\limits_0^{\tfrac{\pi}{4}} f(x)\mathrm{\,d}x=\displaystyle\int\limits_0^{\tfrac{\pi}{4}} \left(\dfrac{1}{2}\sin 2x+2x+4\right)\mathrm{d}x=\left(-\dfrac{1}{4}\cos 2x+x^2+4x\right)\bigg|_0^{\tfrac{\pi}{4}}=\dfrac{\pi^2+16\pi +4}{16}$.}
\end{ex}

\begin{ex}%[Vanle Vo]%[2D3B2-2]%Câu 16
Cho tích phân $\displaystyle\int\limits_0^1f(2x)\mathrm{~d}x=\dfrac{5}{2}$ và $\displaystyle\int\limits_2^3f(x)\mathrm{~d}x=-2$. Tính tích phân $\displaystyle\int\limits_0^3f(x)\mathrm{~d}x$.
\choice
{\True $3$}
{$7$}
{$-10$}
{$-7$}
\loigiai{
Đặt $t=2x\Rightarrow\mathrm{d}t=2dx\Rightarrow \mathrm{d}x=\dfrac{1}{2}\mathrm{~d}t$. Đổi cận $x=0$ thì $ t=0$ và $x=1$ thì $t=2$.\\
Khi đó ta có $\displaystyle\int\limits_0^2f(t)\dfrac{1}{2}\mathrm{~d}t=\dfrac{5}{2}$ hay $\displaystyle\int\limits_0^2f(x)\mathrm{~d}x=5$.\\
Vậy $\displaystyle\int\limits_0^3f(x)\mathrm{~d}x=\displaystyle\int\limits_0^2f(x)\mathrm{~d}x+\displaystyle\int\limits_2^3f(x)\mathrm{~d}x=3$.
}
\end{ex}

\begin{ex}%[Tổng ôn LVD-GV soạn Mui Doan-GVPB Khanh Lê]%[2D3B2-2]%
Cho hàm số $ f(x) $ có $ f(0)= \dfrac{1}{2} $ và  $ f'(x)= \dfrac{2x-1}{\left(x+1\right)^3}, \forall x>-1$. Khi đó $ \displaystyle\int\limits_0^1 f(x)\mathrm{\,d}x $ bằng
\choice
{\True $  \dfrac{7}{4}-2\ln 2 $}
{$ 1-2\ln 2 $}
{$ - \dfrac{1}{2} $}
{$ -1 $}
\loigiai{
Vì $ f'(x)=\dfrac{2x-1}{\left(x+1\right)^3}=\dfrac{2(x+1)-3}{\left(x+1\right)^3}=\dfrac{2}{\left(x+1\right)^2}-\dfrac{3}{\left(x+1\right)^3}$ nên
\begin{eqnarray*}
f(x)&=& \displaystyle\int \left(\dfrac{2}{\left(x+1\right)^2}-\dfrac{3}{\left(x+1\right)^3}\right)\mathrm{\,d}x\\
&=&-\dfrac{2}{x+1}+\dfrac{3}{2(x+1)^2}+C.
\end{eqnarray*}
Do $ f(0)=\dfrac{1}{2}\Rightarrow -2+\dfrac{3}{2}+C=\dfrac{1}{2}\Rightarrow C=1$.\\
Vậy $ f(x)=-\dfrac{2}{x+1}+\dfrac{3}{2(x+1)^2}+1$.
Ta có
\allowdisplaybreaks
\begin{eqnarray*}
\displaystyle\int\limits_0^{1} f(x)\mathrm{\,d}x
&=&\displaystyle\int\limits_0^{1} \left(-\dfrac{2}{x+1}+\dfrac{3}{2(x+1)^2}+1\right)\mathrm{\,d}x\\
&=&\left(-2\ln \vert x+1 \vert -\dfrac{3}{2(x+1)}+x\right)\bigg|^{1}_0\\
&=&-2\ln 2-\dfrac{3}{4}+1-\left(-2\ln 1-\dfrac{3}{2}+0\right)\\
&=&\dfrac{7}{4}-2\ln 2.
\end{eqnarray*}
}
\end{ex}

\begin{ex}%[2D3B2-3]%Câu 464.
Cho hai số thực $a$ và $b$ thỏa $a<b$ và $\displaystyle\int_a^b x\sin x \mathrm{\,d}x=\pi$, đồng thời $a$ cos $a=0$ và $b\cos b=-\pi$. Khi đó $\displaystyle\int_a^b\cos x \mathrm{\,d}x$ bằng
\choice
{$\dfrac{\pi}{2}$}
{$\pi$}
{$-\pi$}
{\True $0$}
\loigiai{
Đặt $\heva{& u=x\Rightarrow \mathrm{\,d}u=\mathrm{\,d}x \\ & \mathrm{\,d}v=\sin x \mathrm{\,d}x\Rightarrow v=-\cos x.}$\\
Khi đó $\displaystyle\int_a^b x\sin x \mathrm{\,d}x=-x\cos x\bigg|_a^b+\displaystyle\int_a^b \cos x \mathrm{\,d}x=a\cos a-b\cos b+\displaystyle\int_a^b \cos x \mathrm{\,d}x=\pi+\displaystyle\int_a^b \cos x \mathrm{\,d}x=0$.\\
Suy ra $\displaystyle\int_a^b \cos x \mathrm{\,d}x=0$.
}
\end{ex}

\begin{ex}%[2D3B2-3]%
Cho hàm số $y=f(x)$ có đạo hàm liên tục trên $[0;1]$, thỏa mãn $\displaystyle\int\limits_0^1{f(x)\mathrm{\,d}x=3}$ và $f(1)=4$. Tích phân $\displaystyle\int\limits_0^1{xf'(x)\mathrm{\,d}x}$ có giá trị là
\choice
{$-\dfrac{1}{2}$}
{$\dfrac{1}{2}$}
{\True $1$}
{$-1$}
\loigiai{
Ta có $\displaystyle\int\limits_0^1{xf'(x)\mathrm{\,d}x}=\displaystyle\int\limits_0^1{x\text{ d}\left(f(x)\right)}=\left. xf(x) \right|_0^1-\displaystyle\int\limits_0^1{f(x)\mathrm{\,d}x}=f(1)-\displaystyle\int\limits_0^1{f(x)\mathrm{\,d}x}=4-3=1$.
}
\end{ex}

\begin{ex}%[2D3Y3-3]%Câu 19
Cho hình phẳng $(D)$ giới hạn bởi các đường $ y=\sin x$; $ y=0$; $ x=0$; $ x=\pi $. Thể tích khối tròn xoay sinh bởi hình $(D)$ quay xung quanh $ Ox$ bằng
\choice
{$\dfrac{\pi ^2}{1000}$}
{$\dfrac{\pi}{1000}$}
{$\dfrac{\pi}{2}$}
{\True $\dfrac{\pi ^2}{2}$}
\loigiai{
Thể tích khối tròn xoay sinh bởi hình $(D)$ quay xung quanh $Ox$ bằng\\
$ V=\pi\displaystyle\displaystyle\int\limits_0^\pi\sin ^2x\text{d}x=\pi\displaystyle\displaystyle\int\limits_0^\pi\dfrac{1-\cos 2x}{2}\text{d}x=\dfrac{\pi}{2}\left.\left(x-\dfrac{\sin 2x}{2}\right)\right|_0^{\pi}=\dfrac{\pi ^2}{2}$. }
\end{ex}

\begin{ex}%[2D3Y3-1]%
Diện tích $S$ của hình phẳng giới hạn bởi các đường $y=2\sin x$, $y=3$, $x=1$ và $x=2$ được tính bởi công thức nào dưới đây?
\choice
{$S=\displaystyle\int\limits_1^2(2\sin x-3)\mathrm{\,d}x$}
{\True $S=\displaystyle\int\limits_1^2|3-2\sin x|\mathrm{\,d}x$}
{$S=\displaystyle\int\limits_1^2(3-2\sin x)^2\mathrm{\,d}x$}
{$S=\pi\displaystyle\int\limits_0^2(2\sin x+3)\mathrm{\,d}x$}
\loigiai{
Diện tích $S$ của hình phẳng là $S=\displaystyle\int\limits_0^2|2\sin x-3|\mathrm{\,d}x$.}
\end{ex}

\begin{ex}%[2D3Y3-1]% ::Cau 6::

Diện tích hình phẳng giới hạn bởi đồ thị hàm số $y=f\left( x \right),y=g\left( x \right)$ liên tục trên đoạn $\left[ a\,;b \right]$ và hai đường thẳng $x=a,x=b$ được xác định theo công thức
\choice
{ $S=\left| \int\limits_a^b{\left[ f\left( x \right)-g\left( x \right) \right]\text{d}x} \right|$}
{\True $S=\int\limits_a^b{\left| f\left( x \right)-g\left( x \right) \right|\text{d}x}$}
{ $S=\pi \int\limits_a^b{\left| f\left( x \right)-g\left( x \right) \right|\text{d}x}$}
{ $S=\int\limits_a^b{\left[ \left| f\left( x \right) \right|-\left| g\left( x \right) \right| \right]\text{d}x}$}
\loigiai{
$S=\int\limits_a^b{\left| f\left( x \right)-g\left( x \right) \right|\text{d}x}\,$}
\end{ex}

\begin{ex}%[2D3B3-1]%
Diện tích hình phẳng giới hạn bởi các đường $y=x^2$ và $y=x+2$ là
\choice
{$S=\dfrac{9}{4}$}
{$S=\dfrac{8}{9}$}
{$S=9$}
{\True $S=\dfrac{9}{2}$}
\loigiai
{
Phương trình hoành độ giao điểm của đồ thị hàm số $y=x^2$ và đường thẳng $y=x+2$ là $$x^2=x+2\Leftrightarrow x^2-x-2=0\Leftrightarrow\hoac{& x=-1 \\ & x=2.}$$
Diện tích hình phẳng giới hạn bởi các đường $y=x^2$ và $y=x+2$ là $$S=\displaystyle\int\limits_{-1}^{2} \left|x^2-x-2\right| \mathrm{\,d}x=\int\limits_{-1}^{2} \left(-x^2+x+2\right) \mathrm{\,d}x=\left(-\dfrac{x^3}{3}+\dfrac{x^2}{2}+2x\right)\Bigg|^2_{-1}=\dfrac{9}{2}.$$
}
\end{ex}

\begin{ex}Câu 34%[2D3B3-3]%
Gọi $(H)$ là hình phẳng giới hạn bởi các đồ thị $y=x^2-2x$, $y=0$ trong mặt phẳng $Oxy$. Quay hình $(H)$ quanh trục hoành ta được một khối tròn xoay có thể tích bằng
\choice
{$\displaystyle\int\limits_0^2\left|x^2-2 x\right|\mathrm{d}x$}
{$\pi\displaystyle\int\limits_0^2\left|x^2-2x\right|\mathrm{d}x$}
{\True $\pi\displaystyle\int\limits_0^2\left(x^2-2x\right)^2\mathrm{d} x$}
{$\displaystyle\int\limits_0^2\left(x^2-2x\right)^2\mathrm{d}x$}
\loigiai{
Ta có $x^2-2 x=0 \Leftrightarrow\hoac{&x=0\\&x=2\\}$.\\
Do đó thể tích khối tròn xoay là $V=\pi\displaystyle\int\limits_0^2\left(x^2-2x\right)^2\mathrm{d}x$.
}
\end{ex}

\begin{ex}%[2D4Y1-1]%Câu 35
Cho hai số phức $z_1=5-2i$ và $z_2=2+3i$. Điểm biểu diễn cho số phức $z_1-z_2$ là
\choice
{\True $M\left( 3;-5 \right)$}
{ $M\left( -3;5 \right)$}
{ $M\left( 3;5 \right)$}
{ $M\left( -3;-5 \right)$}
\loigiai{
Gọi $M\left( x;y \right)$ là điểm biểu diễn số phức $z_1-z_2$.\\
Ta có: $z_1-z_2=\left( 5-2 \right)+\left( -2-3 \right)i=3-5i$.\\
Vậy $M\left( 3;-5 \right)$.\\
}
\end{ex}

\begin{ex}%[2D4Y1-1]%
Phần ảo của số phức $ z=2-3i $  là
\choice
{$ 3 $}
{$ 2 $}
{$ -3i $}
{\True $ -3 $}
\loigiai{
Ta có $ z=2-3i $ nên phần ảo của số phức $ z=2-3i $  là  $ -3 $.
}
\end{ex}

\begin{ex}%[2D4Y1-2]%Câu 20
Trên mặt phẳng tọa độ, điểm biểu diễn số phức $5-7i$ có tọa độ là
\choice
{$\left(5;7\right)$}
{$\left(-5;7\right)$}
{$\left(-5;-7\right)$}
{\True $\left(5;-7\right)$}
\loigiai{
Số phức $5-7i$ có điểm biểu diễn là $\left(5;-7\right)$.
}
\end{ex}

\begin{ex}%[2D4Y1-2]%[Hoàng Nguyên]%
Trên mặt phẳng tọa độ, điểm biểu diễn số phức $z=-3 i$ có tọa độ là
\choice
{\True $(0;-3)$}
{$(-3; 0)$}
{$(0; 3)$}
{$(3; 0)$}
\loigiai{
Điểm biểu diễn số phức $z=-3 i$ có tọa độ là $(0;-3)$.
}
\end{ex}

\begin{ex}%[2D4B1-1]%
Cho hai số phức $z_1 = 1 + 2i$ và $z_2 = 2 - 3i$. Phần ảo của số phức liên hợp $z = 3z_1 - 2z_2$ là
\choice
{$12$}
{\True $-12$}
{$1$}
{$-1$}
\loigiai{Ta có $z = 3z_1 - 2z_2 = 3(1 + 2i) - 2(2-3i) = -1+12i$. Do đó số phức liên hợp của $z$ là $\overline z = -1-12 i$ và có phần ảo là $-12$.}
\end{ex}

\begin{ex}%[2D4B1-2]%
Cho số phức $z$ có điểm biểu diễn trong mặt phẳng tọa độ $O x y$ là điểm $M(3 ;-5)$. Xác định số phức liên hợp $\overline{z}$ của $z$
\choice
{$\overline{z}=-5+3 i$}
{$\overline{z}=5+3 i$}
{\True $\overline{z}=3+5 i$}
{$\overline{z}=3-5 i$}
\loigiai{
Vì $z$ có điểm biểu diễn trong mặt phẳng tọa độ $O x y$ là điểm $M(3 ;-5)$ nên $z=3-5 i \Rightarrow \bar{z}=3+5 i$.
}
\end{ex}

\begin{ex}%[2D4Y2-1]%Câu 19
Cho hai số phức $z=3-4i$ và $w=5+i$. Số phức $z+w$ là
\choice
{$2+5i$}
{$8-5i$}
{$-2-5i$}
{\True $8-3i$}
\loigiai{
Ta có $z+w=3-4i+5+i=8-3i$.
}
\end{ex}

\begin{ex}Câu 30%[2D4Y2-2]%
Cho hai số phức $z_1=1+i$ và $z_2=-1-3i$. Phần thực của số phức $z_1 \cdot \overline{z_2}$ bằng
\choice
{$-2$}
{$2$}
{\True $-4$}
{$4$}
\loigiai{
Số phức $z_1 \cdot \overline{z_2}=(1+i)(-1+3i)=-4+2i$ có phần thực là $-4$.   .
}
\end{ex}

\begin{ex}%[2D4B2-1]%
Cho hai số phức $z=3-4i$và $w=2+3i$. Số phức $z-2w$ bằng
\choice
{ $-1-7i$}
{ $1-10i$}
{ $7+2i$}
{\True $-1-10i$}
\loigiai{
Ta có
$z-2w=3-4i-2\left( 2+3i \right)=3-4i-4-6i=-1-10i$.}
\end{ex}

\begin{ex}%[2D4Y3-1]%
Cho hai số phức $z_1=1+3i$ và $z_2=2i-3$. Số phức $\dfrac{z_1}{z_2}$ bằng
\choice
{$\dfrac{-3-11i}{13}$}
{$-\dfrac{3-11i}{13}$}
{$\dfrac{3+11i}{13}$}
{\True $\dfrac{3-11i}{13}$}
\loigiai{
Ta có $\dfrac{z_1}{z_2}=\dfrac{1+3i}{2i-3}=\dfrac{3-11i}{13}$.
}
\end{ex}

\begin{ex}%[2D4B3-2]%
Cho số phức $z$ thỏa mãn phương trình $(2-i)z+1=3i$. Phần thực của số phức $z$ bằng
\choice
{$-2$}
{\True $-1$}
{$2$}
{$1$}
\loigiai
{
Ta có $(2-i)z+1=3i\Leftrightarrow (2-i)z=-1+3i\Leftrightarrow z=\dfrac{-1+3i}{2-i}\Leftrightarrow z=-1+i$.\\
$\Rightarrow$ Phần thực của số phức $z$ bằng $-1$.
}
\end{ex}

\begin{ex}%[2D4B3-2]%
Số phức $z$ thỏa mãn $iz=6+5i$. Số phức liên hợp của $z$ là
\choice
{$\bar{z}=5-6i$}
{$\bar{z}=-5+6i$}
{\True $\bar{z}=5+6i$}
{$\bar{z}=-5-6i$}
\loigiai{
Ta có $iz=6+5i \Leftrightarrow z=\dfrac{6+5i}{i}=5-6i$.\\
Vậy $\overline{z}=5+6i$.
}
\end{ex}

\begin{ex}%[Vũ Ngọc Phát]%[TDM]%[2D4B3-4]%
Cho số phức $z$ thỏa mãn $|(1+i)z-6|=4$. Quỹ tích điểm biểu diễn số phức $z$ là một đường tròn có phương trình tương ứng là
\choice
{$(x-6)^2+y^2=16$}
{$(x+3)^2+(y-3)^2=4$}
{\True $(x-3)^2+(y+3)^2=8$}
{$(x-1)^2+(y-1)^2=4$}
\loigiai{
Gọi $z=x+yi$, $x$, $y\in \mathbb{R}$.
\begin{eqnarray*}
&& \left| (1+i)z-6 \right|=4\\
&\Leftrightarrow & |1+i|\cdot|z-3+3i|=4 \\
&\Leftrightarrow & |z-3+3i|=2\sqrt{2} \\
&\Leftrightarrow & (x-3)^2+(y+3)^2=8.
\end{eqnarray*}
}
\end{ex}

\begin{ex}%[Trần Nhân Kiệt]%[Dự án TexBook12]%[2D4B4-1]%
Kí hiệu $z_1;$ $z_2$ là hai nghiệm phức của phương trình $3z^2-z+1=0$. Tính $P=\left| z_1 \right|+\left| z_2 \right|$.
\choice
{$P=\dfrac{\sqrt{14}}{3}$}
{$P=\dfrac{2}{3}$}
{$P=\dfrac{\sqrt{3}}{3}$}
{\True $P=\dfrac{2\sqrt{3}}{3}$}
\loigiai{
\begin{itemize}
\item Cách 1\\
Ta có
\begin{eqnarray*}
&&3z^2-z+1=0\Leftrightarrow z^2-\dfrac{1}{3}z+\dfrac{1}{3}=0\\
&\Leftrightarrow& \left( z-\dfrac{1}{6} \right){}^2=-\dfrac{11}{36}\\
&\Leftrightarrow& \left( z-\dfrac{1}{6} \right){}^2=\dfrac{11}{36}{{i}^2}\\
&\Leftrightarrow& \heva{& z=\dfrac{1}{6}+\dfrac{\sqrt{11}}{6}i \\
& z=\dfrac{1}{6}-\dfrac{\sqrt{11}}{6}i}
\end{eqnarray*}
Khi đó $P=\sqrt{{{\left( \dfrac{1}{6} \right)}^2}+{{\left( \dfrac{\sqrt{11}}{6} \right)}^2}}+\sqrt{{{\left( \dfrac{1}{6} \right)}^2}+{{\left( -\dfrac{\sqrt{11}}{6} \right)}^2}}=\dfrac{2\sqrt{3}}{3}$.\\
\item Cách 2\\
Theo tính chất phương trình bậc $2$ với hệ số thực, ta có $z_1;$ $z_2$ là hai số phức liên hợp nên $z_1z_2=\left| z_{1}^2 \right|=\left| z_{2}^2 \right|$. Mà $z_1z_2=\dfrac{1}{3}$ suy ra $\left| z_1 \right|=\left| z_2 \right|=\dfrac{\sqrt{3}}{3}$.\\
Vậy $P=\left| z_1 \right|+\left| z_2 \right|=\dfrac{2\sqrt{3}}{3}$.
\end{itemize}
}
\end{ex}

\begin{ex}%[2D4Y4-1]%
Gọi $z_1,\,z_2$ là hai nghiệm phức của phương trình $2z^2+\sqrt{3}z+3=0$. Giá trị của biểu thức $z_1^2+z_2^2$ bằng
\choice
{$\dfrac{3}{18}$}
{$\dfrac{-9}{8}$}
{$3$}
{\True $\dfrac{-9}{4}$}
\loigiai{
\textbf{Cách 1:} Phương trình $2z^2+\sqrt{3}z+3=0$ có hai nghiệm $z_1=-\dfrac{\sqrt{3}}{4}+\dfrac{\sqrt{21}}{4}i;\,z_2=-\dfrac{\sqrt{3}}{4}-\dfrac{\sqrt{21}}{4}i$.\\
Suy ra biểu thức $z_1^2+z_2^2=\left(-\dfrac{\sqrt{3}}{4}+\dfrac{\sqrt{21}}{4}i\right)^2+\left( -\dfrac{\sqrt{3}}{4}-\dfrac{\sqrt{21}}{4}i \right)^2=-\dfrac{9}{4}$.\\
\textbf{Cách 2:} Áp dụng định lý Vi-et cho phương trình $2z^2+\sqrt{3}z+3=0$. Ta có $\heva{&z_1+z_2=\dfrac{-\sqrt{3}}{2} \\ &z_1\cdot z_2=\dfrac{3}{2}.}$\\
Biểu thức $z_1^2+z_2^2=\left(z_1+z_2\right)^2-2z_1\cdot z_2=\left( \dfrac{-\sqrt{3}}{2} \right)^2-2\cdot\dfrac{3}{2}=\dfrac{-9}{4}$.
}
\end{ex}

\begin{ex}%[2H3Y1-1]%
Trong không gian với hệ trục tọa độ $Oxyz$ cho hai điểm $A(-2; 3;-4)$, $B(4;-3; 3)$. Tính độ dài đoạn thẳng $AB$.
\choice
{\True $AB=11$}
{$AB=(6;-6; 7)$}
{$AB=7$}
{$AB=9$}
\loigiai{
Ta có độ dài đoạn thẳng $AB$ là $AB=\left|\overrightarrow{AB}\right|=\sqrt{6^2+(-6)^2+7^2}=\sqrt{121}=11$.\\
Vậy $AB=11$.}
\end{ex}

\begin{ex}%[2H3B1-3]%
Trong không gian $Oxyz$, gọi $I$ là tâm của mặt cầu $(S)$: $x^2+y^2+z^2+2x-4z-1=0$. Độ dài đoạn $OI$ (với $O$ là gốc tọa độ) bằng
\choice
{$5$}
{\True $\sqrt{5}$}
{$\sqrt{6}$}
{$6$}
\loigiai{
Ta có $I\left(-1;0;2\right) \Rightarrow \overrightarrow{OI}=\left(-1;0;2\right)$.\\
Suy ra $OI=\sqrt{\left(-1\right)^2+0^2+2^2}=\sqrt{5}$}
\end{ex}

\begin{ex}%[2H3Y2-2]%
Trong không gian $Oxyz$, cho mặt phẳng $(P)\colon 2x+3y+2=0$. Vectơ nào sau đây là một vectơ pháp tuyến của mặt phẳng $(P)$?
\choice
{$\overrightarrow{n}=(-2;-3;1)$}
{\True $\overrightarrow{n}=(-2;-3;0)$}
{$\overrightarrow{n}=(2;3;1)$}
{$\overrightarrow{n}=(2;3;2)$}
\loigiai
{
Vectơ pháp tuyến của mặt phẳng $(P)$ là $\overrightarrow{n}=(2;3;0)$.\\
$\Rightarrow$ Mặt phẳng $(P)$ có một vectơ pháp tuyến là $-\overrightarrow{n}=(-2;-3;0)$.
}
\end{ex}

\begin{ex}%[2H3Y2-4]%
Trong không gian với hệ tọa độ $Oxyz$, cho đường thẳng $\Delta \colon \left\{ \begin{aligned}
& x=1+t \\ & y=1+t \\ & z=1+2t \\ \end{aligned} \right.$. Điểm nào sau đây thuộc $\Delta$?
\choice
{\True $M(2;2;3)$}
{$M(1;1;2)$}
{$M(2;2;2)$}
{$M(2;2;-3)$}
\loigiai{
Xét điểm $M(2;2;3)$ ta có: $\Delta\colon \left\{ \begin{aligned}
& 2=1+t \\ & 2=1+t \\ & 3=1+2t \end{aligned} \right.\Leftrightarrow \left\{ \begin{aligned}
& t=1 \\& t=1 \\ & t=1 \end{aligned} \right.\Leftrightarrow t=1\Rightarrow M\in \Delta$.
}
\end{ex}

\begin{ex}%[Dự án tư duy mở 3, Bùi Anh Tuấn]%[2H3B2-3]%
Trong không gian với hệ tọa độ $O x y z$, cho hai điểm $A (3;0;2) $, $ B(-1;2;0) $ và mặt phẳng $(P)\colon x-2y-2=0$. Phương trình mặt phẳng $(Q)$ đi qua $A$, $B$ và vuông góc với $(P)$ là
\choice
{$2x + y -3z +12 =0$}
{\True $2x + y -3z =0$}
{$2x - y + z -2 =0$}
{$3x + y +z -1 =0$}
\loigiai{
Ta có $ \overrightarrow{AB} = (-4;2;-2) = -2 ( 2; -1 ; 1) =-2 \overrightarrow{n_1}$ và véc-tơ pháp tuyến của $(P)$ là $ \overrightarrow{n_2} = (1;-2;0)$.\\
Vì $(Q)$ đi qua $A$, $B$ và vuông góc với $(P)$ nên $(Q)$ có một véc-tơ pháp tuyến là
$$ \overrightarrow{n} = \left[ \overrightarrow{n_1}, \overrightarrow{n_2} \right] = \left( 2; 1; -3\right).   $$
Vậy mặt phẳng $(Q)$ có phương trình là
\begin{eqnarray*}
& & 2 (x +1 ) + 1(y - 2) -3(z-0) = 0 \\
& \Leftrightarrow & 2x + y - 3z  = 0.
\end{eqnarray*}
}
\end{ex}

\begin{ex}%[2H3B2-3]%
Trong không gian $Oxyz$, cho điểm $M\left( 2;0;1 \right)$. Gọi $A,B$ lần lượt là hình chiếu của $M$ trên trục $Ox$ và trên mặt phẳng $\left( Oyz \right)$. Viết phương trình mặt phẳng trung trực của đoạn $AB$.
\choice
{\True  $4x-2z-3=0$}
{ $4x-2y-3=0$}
{ $4x-2z+3=0$}
{ $4x+2z+3=0$}
\loigiai{
Vì $A,B$ lần lượt là hình chiếu của $M\left( 2;0;1 \right)$ trên trục $Ox$ và trên mặt phẳng $\left( Oyz \right)$
nên $A\left( 2;0;0 \right)$ và $B\left( 0;0;1 \right)$.\\
Mặt phẳng trung trực của đoạn $AB$ đi qua trung điểm $I\left( 1;0;\dfrac{1}{2} \right)$ và nhận véc-tơ pháp tuyến $\overrightarrow{n}=\overrightarrow{AB}=\left( -2;0;1 \right)$ có phương trình là $-2\left( x-1 \right)+\left(z-\dfrac{1}{2} \right)=0\Leftrightarrow 4x-2z-3=0$.
}
\end{ex}

\begin{ex}%[2H3Y3-1]%
Trong không gian $Oxyz$ cho đường thằng $\Delta: \dfrac{x-1}{2}=\dfrac{y-2}{2}=\dfrac{z+3}{1}$. Véctơ nào dưới đây là một véctơ chỉ phương của đường thẳng $\Delta$
\choice
{\True $\overrightarrow{u}=\left( 2;2;1\right) $}
{$\overrightarrow{u}=\left( 1;2;-3\right) $}
{$\overrightarrow{u}=\left( -1;-2;3\right) $}
{$\overrightarrow{u}=\left( 2;-2;1 \right) $}
\loigiai{
Véctơ chỉ phương của đường thẳng đã cho là $\overrightarrow{u}=\left( 2;2;1\right) $
}
\end{ex}

\begin{ex}%[2H3Y3-6]%[VU Ngoc Hao-Pb Dai Tran]%
Trong không gian với hệ trục tọa độ $O x y z$, cho hai đường thẳng \\ $d_1\colon \dfrac{x-1}{-2}=\dfrac{y-2}{1}=\dfrac{z+2}{m}$ và đường thẳng $d_2\colon \dfrac{x-2}{3}=\dfrac{y}{-3}=\dfrac{z-3}{1}$. Để hai đường thẳng này vuông góc với nhau thì
\choice
{\True $m \in\{9\}$}
{$m=3$}
{$m=-6$}
{$m=-1$}
\loigiai{
Đường thẳng $d_1 $ có $\mathrm{VTCP}$ là 	$\vec{u}_1=(-2;1;m)$. \\
Đường thẳng $d_2 $ có $\mathrm{VTCP}$ là $\vec{u}_2=(3; -3; 1)$.\\
Để hai đường thẳng này vuông góc với nhau thì $\vec{u}_1\cdot \vec{u}_2 = 0$
$\Rightarrow -6 -3+m=0 \Rightarrow m=9$.
}
\end{ex}

\begin{ex}%[2H3Y3-3]%
Trong không gian $Oxyz$, cho đường thẳng $d\colon \dfrac{x-1}{2}=\dfrac{y-2}{3}=\dfrac{z+1}{-1}$. Điểm nào sau đây thuộc $d$?
\choice
{\True $P\left( 1;2;-1 \right)$}
{$M\left( -1;-2;1 \right)$}
{$N\left( 2;3;-1 \right)$}
{$Q\left( -2;-3;1 \right)$}
\loigiai{
Thay tọa độ điểm $P\left( 1;2;-1 \right)$ vào phương trình đường thẳng $d$ thấy thỏa mãn nên đường thẳng $d$ đi qua điểm $P\left( 1;2;-1 \right)$.}
\end{ex}

\begin{ex}Câu 33%[2H3B3-2]%
Trong không gian $Oxyz$, cho các điểm $A(1;2;0), B(2;0;2), C(2;-1;3), D(1;1;3)$. Đường thẳng đi qua $C$ và vuông góc với mặt phẳng $(ABD)$ có phương trình là
\choice
{$\heva{&x=-2-4t\\&y=-2-3t\\&z=2-t}$}
{$\heva{&x=2+4t\\&y=-1+3t\\&z=3-t}$}
{\True $\heva{&x=-2+4t\\&y=-4+3t\\&z=2+t}$}
{$\heva{&x=4+2t\\&y=3-t\\&z=1+3t}$}
\loigiai{
Đường thẳng cần tìm đi qua $C(2;-1;3)$ và có véc-tơ chỉ phương là $\overrightarrow{u}=\left[\overrightarrow{AB},\overrightarrow{AD}\right]=(-4;-3;-1)$ nên có phương trình tham số là $\heva{&x=2+4t\\&y=-1+3t\\&z=3+t.}$\\
Ta thấy điểm $M(-2;-4;2)$ thuộc đường thẳng đi qua $C$ và vuông góc với mặt phẳng $(ABD)$ nên ta chọn.}
\end{ex}

\begin{ex}%[2H3B3-6]%
Trong không gian $O x y z$, cho đường thẳng $d\colon \heva{&x=t \\& y=1-t\\ &z=-1+2 t}$ và mặt phẳng  $(\alpha)\colon x+3 y+z-2=0$. Khẳng định nào sau đây là đúng?
\choice
{Đường thẳng $d$ cắt mặt phẳng $(\alpha)$}
{\True Đường thẳng $d$ nằm trên mặt phẳng $(\alpha)$}
{Đường thẳng $d$ vuông góc với mặt phẳng $(\alpha)$}
{Đường thẳng $d$ song song với mặt phẳng $(\alpha)$}
\loigiai{
Thay $\heva{&x=t \\ &y=1-t\\& z=-1+2t} $ vào phương trình $x+3 y+z-2=0$  ta được
$t+3(1-t)-1+2 t-2=0 \Leftrightarrow 0 t+0=0$.\\
Suy ra  đường thẳng $d$ nằm trên mặt phẳng $(\alpha)$.
}
\end{ex}

\begin{ex}%[2D3K2-3]% Câu 23
Cho tích phân $\displaystyle\int\limits_{1}^{\mathrm{e}^{\frac{\pi}{2}}} \sin(\ln x) \mathrm{d}x = \dfrac{\mathrm{e}^{\frac{\pi}{2}} +a}{b} $; với $ a $, $ b $ là những số nguyên dương. Giá trị của biểu thức $ T=a+b $ tương ứng bằng
\choice
{\True $ 3 $}
{$ 5 $}
{$ 7 $}
{$ 9 $}
\loigiai{
Đây là dạng tích phân từng phần. Đặt $ \heva{&u=\sin(\ln x)\\&\mathrm{d}v = \mathrm{d}x} \Rightarrow \heva{&\mathrm{d}u = \dfrac{1}{x}\cos(\ln x) \mathrm{d}x\\&v=x.} $\\
Suy ra $ I =  \displaystyle\int\limits_{1}^{\mathrm{e}^{\frac{\pi}{2}}} \sin(\ln x) \mathrm{d}x = x\cdot \sin(\ln x) \bigg|^{\mathrm{e}^{\frac{\pi}{2}}}_1 -  \displaystyle\int\limits_{1}^{\mathrm{e}^{\frac{\pi}{2}}} \cos(\ln x) \mathrm{d}x = \mathrm{e}^{\frac{\pi}{2}} -J $. \quad (1)\\
Với tích phân $ J = \displaystyle\int\limits_{1}^{\mathrm{e}^{\frac{\pi}{2}}} \cos(\ln x) \mathrm{d}x $. Đặt $ \heva{&u=\cos(\ln x)\\&\mathrm{d}v = \mathrm{d}x} \Rightarrow \heva{&\mathrm{d}u = -\dfrac{1}{x}\sin (\ln x) \mathrm{d}x\\&v=x.} $\\
Suy ra $ J = \displaystyle\int\limits_{1}^{\mathrm{e}^{\frac{\pi}{2}}} \cos(\ln x) \mathrm{d}x = x\cdot \cos(\ln x) \bigg|^{\mathrm{e}^{\frac{\pi}{2}}}_1 + \displaystyle\int\limits_{1}^{\mathrm{e}^{\frac{\pi}{2}}} \sin(\ln x) \mathrm{d}x = -1+ I $. \quad (2)\\
Thay (2) vào (1), ta được $ I=\mathrm{e}^{\frac{\pi}{2}} -J = \mathrm{e}^{\frac{\pi}{2}} - (-1+I) \Leftrightarrow I = \dfrac{\mathrm{e}^{\frac{\pi}{2}} +1}{2} = \dfrac{\mathrm{e}^{\frac{\pi}{2}} +a}{b} $.\\
Suy ra $ a=1 $; $ b=2 \Rightarrow T=a+b=3 $.
}
\end{ex}

\begin{ex}%[2D3K2-2]%
Tích phân $I=\displaystyle\int\limits_{1}^{2} \dfrac{(3 x-2)^{2019}}{x^{2021}} \mathrm{\,d}x$ tương ứng bằng
\choice
{\True$\dfrac{4^{2020}-1}{2020}$}
{$\dfrac{2^{2019}-1}{4038}$}
{$\dfrac{2^{2020}-1}{4040}$}
{$\dfrac{2^{2021}-1}{4042}$}
\loigiai{
Ta có $I=\displaystyle\int\limits_{1}^{2} \dfrac{(3 x-2)^{2019}}{x^{2021}} \mathrm{\,d}x=\displaystyle\int\limits_{1}^{2} \dfrac{(3 x-2)^{2019}}{x^{2019}} \cdot \dfrac{\mathrm{\,d}x}{x^{2}}=\displaystyle\int\limits_{1}^{2}\left(3-\dfrac{2}{x}\right)^{2019} \cdot \dfrac{\mathrm{\,d}x}{x^{2}}$.\\
Đặt $t=3-\dfrac{2}{x} \Rightarrow \mathrm{\,d}t=\dfrac{2 \mathrm{\,d}x}{x^{2}}$; đổi cận $\heva{&x=1 \Rightarrow t=1 \\ &x=2 \Rightarrow t=2.}$\\
Suy ra $I=\displaystyle\int\limits_{1}^{2}\left(3-\dfrac{2}{x}\right)^{2019} \cdot \dfrac{\mathrm{\,d}x}{x^{2}}=\displaystyle\int\limits_{1}^{2} t^{2019} \cdot \dfrac{\mathrm{\,d}t}{2}=\dfrac{t^{2020}}{4040}\bigg|_{1} ^{2}=\dfrac{2^{2020}-1}{4040}$.
}
\end{ex}

\begin{ex}%[2D3G2-4]%Câu 5
Cho hàm số $f(x)$ có đạo hàm liên tục và xác định trên đoạn $\left[0;1 \right] $ và thỏa mãn \break $\displaystyle\int\limits_0^1 x\cdot f''(x)\mathrm{\,d}x=-1$; $f'(1)=f(1)$. Giá trị của $f(0)$	bằng
\choice
{\True $ -1 $}
{$ 1 $}
{$ 2 $}
{$ 0 $}
\loigiai{
Với tích phân $A=\displaystyle\int\limits_0^1 x^2\cdot f''(x)\mathrm{\,d}x=-1;$ ta đặt $\heva{&u=x \\& \mathrm{\,d}v=f''(x)\mathrm{\,d}x} \Rightarrow \heva{&\mathrm{d}u=\mathrm{d}x \\& v=f'(x).}$\\
$A=\displaystyle\int\limits_0^1 x\cdot f''(x)\mathrm{\,d}x=-1=\left[x\cdot f'(x)\right]\bigg|_0^1-\displaystyle\int\limits_0^1 f'(x)\mathrm{\,d}x=\left[f'(1)-0\right]-[f(x)]\bigg|_0^1=f'(1)-f(1)+f(0)$.\\Suy ra $-1=f'(1)-f(1)+f(0)=f(0)$.


}
\end{ex}

\begin{ex}%[2D4G5-1]%
Gọi $ M $ là điểm biểu diễn số phức $ z_1=a+(a^2-2a+2)i  $  và $ N $ là điểm biểu diễn số phức $ z_2 $ biết $ |z_2-2-i|=|\overline{z}-6-i| $. Tìm độ dài ngắn nhấn của đoạn $ MN $.
\choice
{$ 2\sqrt{5} $}
{\True $ \dfrac{6\sqrt{5}}{5} $}
{$ 1 $}
{$ 5 $}
\loigiai{
Gọi $M(x ; y)$. Từ điều kiện $z_{1}=a+\left(a^{2}-2 a+2\right) i$ suy ra $M$ thuộc parabol $(P)\colon y=x^{2}-2 x+2$.\\
Gọi $N(x ; y)$. Từ điều kiện $\left|z_{2}-2-i\right|=\left|\overline{z_{2}}-6-i\right|$ suy ra $N$ thuộc đường thẳng $d\colon 2 x-y-8=0$.\\
Gọi $\Delta$ là tiếp tuyến của $(P)$ mà song song với $d\colon  2 x-y-8=0$.\\
Gọi $M\left(x_{0} ; y_{0}\right)$ là tiếp điểm mà tại đó tiếp tuyến $\Delta \parallel d$.\\
Ta có $y'=2 x-2$. Do $\Delta\parallel d$ nên $y'\left(x_{0}\right)=2 \Leftrightarrow 2 x_{0}-2=2 \Leftrightarrow x_{0}=2$ suy ra $y_{0}=2$.\\
Phương trình tiếp tuyến $\Delta$ có dạng $y=y^{\prime}\left(x_{0}\right) \cdot\left(x-x_{0}\right)+y_{0} \Leftrightarrow y=2(x-2)+2 \Leftrightarrow y=2 x-2$.\\
Khi đó: $\min M N=\mathrm{d}(\Delta, d)=\mathrm{d}(A , d)$ với $A \in \Delta$. \\
Chọn $A(1 ; 0)$ ta có $\min M N=\dfrac{|2.1-0-8|}{\sqrt{2^{2}+(-1)^{2}}}=\dfrac{6 \sqrt{5}}{5}$.
\begin{center}
\begin{tikzpicture}[scale=1, line join=round, line cap=round,>=stealth]
%Vẽ hệ trục Oxy
\draw[->,line width = 1pt] (-3,0)--(0,0) node[below right]{$O$}--(7,0) node[below]{$x$};
\draw[->,line width = 1pt] (0,-3)--(0,5.5) node[right]{$y$};
%Vẽ các điểm gióng
\foreach \x in {-2,-1,1,2,3,4,5,6} \draw[thin] (\x,1pt)--(\x,-1pt) node [below] {$\x$};
\foreach \y in {-2,-1,1,2,3,4} \draw[thin] (1pt,\y)--(-1pt,\y) node [left] {$\y$};
%Vẽ đồ thị hàm số
\draw[samples=200,domain=-1:3,smooth] plot (\x, {(\x)^2 - 2*(\x)+2}) node[above]{$(P)\colon y=x^2-2x+2$};
\draw[samples=200,domain=-0.5:3.5,smooth] plot (\x, {2*(\x)-2}) ;
\draw[samples=200,domain=2.5:6.5,smooth] plot (\x, {2*(\x)-8}) ;
\coordinate[label = above left:$M$] (M) at (2,2);
\coordinate[label =  right:$N$] (N) at ($(4,0)!(M)!(5,2)$); % Điểm H là hình chiếu của điểm C trên đường thẳng AB
\draw (M)--(N);
\fill[black] (M) circle (1.5pt);
\fill[black] (N) circle (1.5pt);
\end{tikzpicture}
\end{center}
}
\end{ex}

\begin{ex}%[Vô Văn Tự]%[2H3K2-3]%19
Trong không gian hệ trục tọa độ $Oxyz$, cho hai đường thẳng là $d_1\colon \dfrac{x-1}{a}=\dfrac{y}{2}=\dfrac{z-2}{-2}$ và đường thẳng $d_2\colon \dfrac{x-3}{1}=\dfrac{y}{2}=\dfrac{z}{1}$. Với $a$ là tham số thực. Biết rằng tồn tại mặt phẳng $(P)$ có phương trình $ax+by+cz+d=0$ chứa cả hai đường thẳng $d_1$ và $d_2$. Giá trị của $T=a+b+c+d$ bằng
\choice
{$12$}
{$-7$}
{\True $-8$}
{$-10$}
\loigiai{
Đưa hai đường thẳng về dạng tham số là $d_1\colon \left\{\begin{aligned}&x=1+at_1\\&y=2t_1\\&z=2-2t_1\end{aligned}\right.$ và $d_2\colon \left\{\begin{aligned}&x=3+t_2\\&y=2t_2\\&z=t_2.\end{aligned}\right.$\\
Vì $d_1$ và $d_2$ không song song và không trùng nhau nên tồn tại mặt phẳng chứa cả hai đường thẳng thì hai đường thẳng này phải cắt nhau tại một điểm.\\
Hai đường thẳng cắt nhau khi và chỉ khi hệ phương trình giao điểm phải có nghiệm duy nhất. Khi đó
\[\left\{\begin{aligned}&1+at_1=3+t_2\\&2t_1=2t_2\\&2-2t_1=t_2\end{aligned}\right.\Leftrightarrow \left\{\begin{aligned}&at_1=2+t_2\\&t_1=t_2\\&2t_1+t_2=2.\end{aligned}\right.\]
Từ hai phương trình cuối suy ra được $t_1=t_2=\dfrac{3}{2}$, thay vào phương trình đầu ta được
\[at_1=2+t_2\Leftrightarrow a\cdot\dfrac{2}{3}=2+\dfrac{2}{3}\Leftrightarrow a=4.\]
Khi đó hai đường thẳng là có VTCP lần lượt là $\overrightarrow{u}_1=(4;2;-2)$ và $\overrightarrow{u}_2=(1;2;1)$.\\
Suy ra VTPT của mặt phẳng $(P)$ là
\[\overrightarrow{n}=\left[\overrightarrow{u}_1,\overrightarrow{u}_2\right]=(6;-6;6)=6(1;-1;1).\]
Chọn $(P)$ đi qua điểm $A=(1;0;2)\in d_1$, suy ra phương trình mặt phẳng $(P)$ là
\[1(x-1)-1(y-0)+1(z-2)=0\Leftrightarrow x-y+z-3=0.\]
So sánh với giải thiết
\begin{eqnarray*}
(P)\colon x-y+z-3=ax+by+cz+d=0&\Leftrightarrow&  \dfrac{a}{1}=\dfrac{b}{-1}=\dfrac{c}{1}=\dfrac{d}{-3}=\dfrac{4}{1}\\ &\Leftrightarrow & a=4;b=-4;c=4;d=-12.
\end{eqnarray*}
Suy ra $T=a+b+c+d=-8$.
}
\end{ex}

\begin{ex}%[Nguyễn Thắng-Texbook Oxyz-Số Phức]%[2H3K3-7]%
Trong không gian với hệ tọa độ $O x y z$ viết phương trình mặt phẳng tiếp xúc với mặt cầu $(S)\colon (x-1)^{2}+y^{2}+(z+2)^{2}=6$ đồng thời song song với hai đường thẳng $d_{1}\colon \dfrac{x-2}{3}=\dfrac{y-1}{-1}=\dfrac{z}{-1}$, $d_{2}\colon \dfrac{x}{1}=\dfrac{y+2}{1}=\dfrac{z-2}{-1}$.
\choice
{$\hoac{&x-y+2 z-3=0 \\ &x-y+2 z+9=0}$}
{\True$\hoac{&x+y+2 z-3=0 \\ &x+y+2 z+9=0}$}
{$x+y+2 z+9=0$}
{$x-y+2 z+9=0$}
\loigiai{
Gọi $(P)$ là mặt phẳng cần tìm có véc-tơ pháp tuyến là $\vec{n}$.\\
Đường thẳng $d_{1}$, $d_{2}$ có véc-tơ chỉ phương lần lượt là $\overrightarrow{u}_{1}=(3 ;-1 ;-1)$ và $\overrightarrow{u}_{2}=(1 ; 1 ;-1)$.\\
Mặt cầu $(S):(x-1)^{2}+y^{2}+(z+2)^{2}=6$ có tâm $I(1 ; 0 ;-2)$, bán kính $R=\sqrt{6}$.\\
Do $\heva{&(P) \parallel d_{1} \\ &(P) \parallel d_{2}} \Rightarrow\heva{&\vec{n} \perp \overrightarrow{u}_{1} \\ &\vec{n} \perp \overrightarrow{u}_{2}}$. Suy ra $\vec{n}$ cùng phương với $\left[\overrightarrow{u}_{1}, \overrightarrow{u}_{2}\right]$.\\
Có $\left[\overrightarrow{u}_{1}, \overrightarrow{u}_{2}\right]=(2 ; 2 ; 4)$, nên chọn $\vec{n}=(1 ; 1 ; 2)$.\\
Khi đó phương trình tổng quát của mặt phẳng $(P)$ có dạng: $x+y+2 z+d=0$.\\
Mặt phẳng $(P)$ tiếp xúc với mặt cầu $(S)$ $$\Leftrightarrow \mathrm{d}(I,(P))=R  \Leftrightarrow \dfrac{|1+0+2 \cdot(-2)+d|}{\sqrt{1^{2}+1^{2}+2^{2}}}=\sqrt{6}\Leftrightarrow|d-3|=6 \Leftrightarrow\hoac{&d=9 \\ &d=-3.}$$
Vậy có hai mặt phẳng thỏa mãn đề là $\left(P_{1}\right): x+y+2 z-3=0$ và $\left(P_{2}\right): x+y+2 z+9=0$.
}
\end{ex}



\Closesolutionfile{ans}