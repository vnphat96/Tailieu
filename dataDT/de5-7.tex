
\begin{name}
	{\tenchude}
	{\tendethi}
	{\tentruong}
	{\thoigian}
\end{name}
\Opensolutionfile{ans}[ans/ans-de5-7]

\begin{ex}%Câu 42.
Cho hai số phức $z_1=1+2 i, z_2=-2-i$. Khi đó giá trị $\left|z_1 z_2\right|$ là
\choice
{\True $5$}
{$2\sqrt{5}$}
{$25$}
{$0$}

\end{ex}
\begin{ex}%Câu 1.
Cho $a$ và $b$ là các số dương bất kỳ. Chọn khẳng định sai?
\choice
{\True $\log (10 a b)^2=2+\log a+\log b$}
{$\ln a^2+\ln \sqrt[3]{b}=2\ln a+\dfrac{1}{3} \ln b$}
{$\log a-\log b=\log \dfrac{a}{b}$}
{$\ln a b=\ln a+\ln b$}

\end{ex}
\begin{ex}%Câu 2.
Trong không gian với hệ tọa độ $O x y z$, cho $(P)\colon 2 x+2 y-z+3=0$ và điểm $M(1;-2;-1)$. Khi đó khoảng cách từ điểm $M$ đến mặt phẳng $(P)$ bằng
\choice
{$0$}
{\True $\dfrac{2}{3}$}
{$\dfrac{10}{3}$}
{$\dfrac{8}{3}$}

\end{ex}
\begin{ex}%Câu 3.
Tìm tập nghiệm của bất phương trình $\log_3(x-2) \geq 2$.
\choice
{\True $[11;+\infty)$}
{$(11;+\infty)$}
{$(-\infty; 11)$}
{$(2;+\infty)$}

\end{ex}
\begin{ex}%Câu 4.
Trong hệ trục tọa độ ${O} x y z$ cho mặt phẳng $(\alpha)\colon 2 x-y+3 z-1=0$.
Véctơ nào sau đây là véctơ pháp tuyến của mặt phẳng $(\alpha)$.
\choice
{$\vec{n}=(2; 1; 3)$}
{$\vec{n}=(2; 1;-3)$}
{$\vec{n}=(-2; 1; 3)$}
{\True $\vec{n}=(-4; 2;-6)$}

\end{ex}
\begin{ex}%Câu 5.
Thể tích của khối cầu có bán kính bằng $a$ là:
\choice
{$V=\pi a^3$}
{\True $V=\dfrac{4\pi a^3}{3}$}
{$V=4\pi a^3$}
{$V=2\pi a^3$}

\end{ex}
\begin{ex}%Câu 6.
Cho $a<b<c, \displaystyle\displaystyle\int\limits_a^{b} f(x) \mathrm{d} x=5$ và $\displaystyle\displaystyle\int\limits_c^{b} f(x) \mathrm{d} x=2$. Tính $\displaystyle\displaystyle\int\limits_a^{c} f(x) \mathrm{d} x$.
\choice
{\True $\displaystyle\displaystyle\int\limits_a^{c} f(x) \mathrm{d} x=3$}
{$\displaystyle\displaystyle\int\limits_a^{c} f(x) \mathrm{d} x=-2$}
{$\displaystyle\displaystyle\int\limits_a^{c} f(x) \mathrm{d} x=1$}
{$\displaystyle\displaystyle\int\limits_a^{c} f(x) \mathrm{d} x=7$}

\end{ex}
\begin{ex}%Câu 7.
Trong không gian với hệ tọa độ $O x y z$ cho tam giác $ABC$ có $3$ đỉnh $A(1;-2; 3), B(2; 3; 5), C(4; 1;-2)$. Tính tọa độ trọng tâm $G$ của tam giác $ABC$.
\choice
{$G(6; 4; 3)$}
{$G(8; 6;-30)$}
{$G(7; 2; 6)$}
{\True $G\left(\dfrac{7}{3}; \dfrac{2}{3}; 2\right)$}

\end{ex}
\begin{ex}%Câu 8.
Thể tích của khối tròn xoay khi quay hình phẳng giới hạn bởi đồ thị hàm số $y=x^2-x$ và trục hoành quanh trục hoành là
\choice
{$\dfrac{\pi}{3}$}
{\True $\dfrac{\pi}{30}$}
{$\dfrac{\pi}{15}$}
{$\dfrac{\pi}{5}$}

\end{ex}
\begin{ex}%Câu 9.
Cho $F(x)$ là một nguyên hàm của hàm số $f(x)=\cos (\pi-x)$ và $F(\pi)=0$. Tính $F\left(\dfrac{\pi}{2}\right)$.
\choice
{$F\left(\dfrac{\pi}{2}\right)=-2$}
{\True $F\left(\dfrac{\pi}{2}\right)=-1$}
{$F\left(\dfrac{\pi}{2}\right)=1$}
{$F\left(\dfrac{\pi}{2}\right)=0$}

\end{ex}
\begin{ex}%Câu 10.
Với mọi số thực dương $a, b$ bất kì. Mệnh đề nào dưới đây đúng?
\choice
{\True $\log_{a^2+1} a \geq \log_{a^2+1} b \Leftrightarrow a \geq b$}
{$\log_{\frac{3}{4}} a<\log_{\frac{3}{4}} b \Leftrightarrow a<b$}
{$\log_2\left(a^2+b^2\right)=2\log (a+b)$}
{$\log_2 a^2=\dfrac{1}{2} \log_2 a$}

\end{ex}
\begin{ex}%Câu 11.
Xác định tập nghiệm $S$ của bất phương trình $\ln x^2>\ln (4 x-4)$ 
\choice
{$S=(2;+\infty)$}
{$S=\mathbb{R} \backslash\{2\}$}
{$S=(1;+\infty)$}
{\True $S=(1;+\infty) \backslash\{2\}$}

\end{ex}
\begin{ex}%Câu 12.
Đồ thị của hàm số nào dưới đây có hai tiệm cận đứng?
\choice
{$y=\dfrac{x+1}{x^2+1}$}
{$y=\dfrac{3 x-1}{3 x^2-3 x+2}$}
{\True $y=\dfrac{x-1}{3 x^2-10 x+3}$}
{$y=\dfrac{5 x^2-3 x-2}{x^2-4 x+3}$}

\end{ex}
\begin{ex}%Câu 13.
Cho hình chóp $S.ABCD$ có đáy $ABCD$ là hình chữ nhật tâm $O, AB=a, AD=a \sqrt{3}, SA \perp(ABCD)$. Khoảng cách từ $O$ đến mặt phẳng $(SCD)$ bằng $\dfrac{a \sqrt{3}}{4}$. Tính thể tích $V$ của khối chóp $S.ABCD$.
\choice
{\True $V=\dfrac{a^3 \sqrt{3}}{3}$}
{$V=\dfrac{a^3 \sqrt{3}}{6}$}
{$V=\dfrac{a^3 \sqrt{15}}{10}$}
{$a^3\sqrt{3}$}

\end{ex}
\begin{ex}%Câu 14.
Diện tích toàn phần của hình lập phương có cạnh $3 a$ là
\choice
{$9 a^2$}
{$72 a^2$}
{\True $54 a^2$}
{$36 a^2$}

\end{ex}
\begin{ex}%Câu 15.
Tìm tập xác định của hàm số $y=\log (x+1)$ 
\choice
{$D=(-\infty;-1)$}
{\True $D=(-1;+\infty)$}
{$D=[-1;+\infty)$}
{$D=\mathbb{R} \backslash\{-1\}$}

\end{ex}
\begin{ex}%Câu 16.
Họ nguyên hàm của hàm số $f(x)=4 x^3-\dfrac{1}{x^2}$ là
\choice
{\True $F(x)=x^4+\dfrac{1}{x}+C$}
{$F(x)=12 x^2-\dfrac{1}{x}+C$}
{$F(x)=x^4-\dfrac{1}{x}+C$}
{$F(x)=x^4+\ln \left|x^2\right|+C$}

\end{ex}
\begin{ex}%Câu 17.
Có bao nhiêu cách chọn $5$ học sinh từ $20$ học sinh?
\choice
{$1860480$ cách}
{$120$ cách}
{\True $15504$ cách}
{$100$ cách}

\end{ex}
\begin{ex}%Câu 18.
Cho cấp số cộng $\left(u_n\right)$ có số hạng đầu $u_1=3$ và công sai $d=2$. Giá trị của $u_{10}$ bằng:
\choice
{$24$}
{$23$}
{$22$}
{\True $21$}

\end{ex}
\begin{ex}%Câu 19.
Tìm tập nghiệm của phương trình $3^{x^2+2 x}=1$.
\choice
{$S=\{-1; 3\}$}
{\True $S=\{0;-2\}$}
{$S=\{1;-3\}$}
{$S=\{0; 2\}$}

\end{ex}
\begin{ex}%Câu 20.
Cho hàm số $y=f(x)$ xác định và liên tục trên $\mathbb{R}$, có bảng biến thiên như hình bên. Mệnh đề nào sau đây đúng?
\begin{center}
\begin{tikzpicture}[color=\mauchinh]
\tkzTabInit[nocadre,lgt=1.2,espcl=2.5,deltacl=0.6,lw=0.8]
{$x$/0.6,$f'(x)$/0.6,$f(x)$/2}{$-\infty$,$-1$,$1$,$+\infty$}
\tkzTabLine{,+,0,-,0,+,}
\tkzTabVar{-/$-\infty$,+/$2$,-/$-1$,+/$+\infty$}
\end{tikzpicture}
\end{center}
\choice
{Hàm số nghịch biến trên khoảng $(-\infty; 1)$}
{\True Hàm số đồng biến trên khoảng $(-\infty;-2)$}
{Hàm số nghịch biến trên khoảng $(1;+\infty)$}
{Hàm số đồng biến trên khoảng $(-1;+\infty)$}
\end{ex}
\begin{ex}%Câu 21.
Biến đổi biểu thức $A=\sqrt{a} \cdot \sqrt[3]{a^2}$ về dạng lũy thừa với số mũ hữu tỷ ta được
\choice
{\True $A=a^{\frac{7}{6}}$}
{$A=a^2$}
{$A=a$}
{$A=a^{\frac{7}{2}}$}

\end{ex}
\begin{ex}%Câu 22.
Cho hình trụ có bán kính đáy bằng $5$ và chiều cao bằng $7$. Diện tích xung quanh của hình trụ đã bằng:
\choice
{$\dfrac{175\pi}{3}$}
{$175\pi$}
{\True $70\pi$}
{$35\pi$}

\end{ex}
\begin{ex}%Câu 23.
Cho khối chóp $S.ABC$ có $SA$ vuông góc $(ABC)$ và $SA=2$, tam giác $ABC$ vuông cân tại $A$ và $AB=1$. Thể tích khối chóp $S.ABC$ bằng
\choice
{$\dfrac{1}{6}$}
{\True $\dfrac{1}{3}$}
{$1$}
{$\dfrac{2}{3}$}

\end{ex}
\begin{ex}%Câu 24.
Một khối nón tròn xoay có độ dài đường sinh $l=13(\mathrm{\,cm})$ và bán kính đáy $r=5(\mathrm{\,cm})$. Khi đó thể tích khối nón bằng
\choice
{\True $V=100\pi\left(\mathrm{cm}^3\right)$}
{$V=300\pi\left(\mathrm{cm}^3\right)$}
{$V=\dfrac{325}{3} \pi\left(\mathrm{cm}^3\right)$}
{$V=20\pi\left(\mathrm{cm}^3\right)$}

\end{ex}
\begin{ex}%Câu 25.
Khối cầu có bán kính $R=6$ có thể tích bằng bao nhiêu?
\choice
{$144\pi$}
{\True $288\pi$}
{$48\pi$}
{$72\pi$}

\end{ex}
\begin{ex}%Câu 26.
Bất phương trình sau $\log_2(3 x-1)>3$ có nghiệm là:
\choice
{\True $x>3$}
{$x<3$}
{$\dfrac{1}{3}<x<3$}
{$x>\dfrac{10}{3}$}

\end{ex}
\begin{ex}%Câu 27.
\immini{
Đồ thị của hàm số $y=f(x)$ như hình vẽ bên. Số nghiệm của phương trình $4 f(x)-3=0$ 
\choice
{\True $4$}
{$3$}
{$2$}
{$0$}}
{\vspace{-0.6cm}
\begin{tikzpicture}[scale=1, font=\footnotesize, line join=round, line cap=round, >=stealth,color=\mauchinh,y=0.8cm]
\def\xmin{-1.56}\def\xmax{1.56}\def\ymin{-1.1}\def\ymax{1.2}
\draw[->,thick] (\xmin-0.2,0)--(\xmax+0.2,0) node[below] {\footnotesize $x$};
\draw[->,thick] (0,\ymin-0.2)--(0,\ymax+0.2) node[right] {\footnotesize $y$};
\draw (0,0) node [below left] {\footnotesize $O$};
\foreach \x in {-1,1}\draw (\x,0.1)--(\x,-0.1) node [below] {\footnotesize $\x$};
\foreach \y in {1}\draw (0.1,\y)--(-0.1,\y) node [left] {\footnotesize $\y$};
\clip (\xmin,\ymin) rectangle (\xmax,\ymax);
\draw[thick,smooth,samples=200,domain=\xmin:\xmax] plot (\x,{-1*((\x)^4)+2*((\x)^2)+0});
\draw[dashed] (-1,0)--(-1,1)--(0,1);\fill (-1,1) circle (1pt);
\draw[dashed] (1,0)--(1,1)--(0,1);\fill (1,1) circle (1pt);
\end{tikzpicture}
}

\end{ex}
\begin{ex}%Câu 28.
Nếu $\displaystyle\displaystyle\int\limits_0^1 f(x) \mathrm{d} x=5$ và $\displaystyle\displaystyle\int\limits_2^1 f(x) \mathrm{d} x=2$ thì $\displaystyle\displaystyle\int\limits_0^2 f(x) \mathrm{d} x$ bằng
\choice
{$8$}
{$2$}
{\True $3$}
{$-3$}

\end{ex}
\begin{ex}%Câu 29.
\immini{
Cho hàm số $y=f(x)$ có bảng biến thiên như hình bên. Hàm số đạt cực đại tại điểm nào trong các điểm sao đây?}
{\vspace{-0.5cm}
\begin{tikzpicture}[color=\mauchinh]
\tkzTabInit[nocadre,lgt=1.2,espcl=1.7,deltacl=0.5,lw=0.8]
{$x$/0.6,$f'(x)$/0.6,$f(x)$/1.7}{$-\infty$,$2$,$4$,$+\infty$}
\tkzTabLine{,+,0,-,0,+,}
\tkzTabVar{-/$-\infty$,+/$3$,-/$-2$,+/$+\infty$}
\end{tikzpicture}

}
\choice
{$x=-2$}
{$x=3$}
{\True $x=2$}
{$x=4$}

\end{ex}
\begin{ex}%Câu 30.
\immini{
Đường cong trong hình vẽ là đồ thị của hàm số nào?
\choice
{\True $y=-x^3+3 x$}
{$y=x^3-3 x$}
{$y=-x^2+x+1$}
{$y=x^4-x^2+1$}}
{\vspace{-0.5cm}
\begin{tikzpicture}[scale=.8, font=\footnotesize, line join=round, line cap=round, >=stealth,color=\mauchinh,y=0.8cm]
\def\xmin{-2.01}\def\xmax{2.01}\def\ymin{-2.22}\def\ymax{2.22}
\draw[->,thick] (\xmin-0.2,0)--(\xmax+0.2,0) node[below] {\footnotesize $x$};
\draw[->,thick] (0,\ymin-0.2)--(0,\ymax+0.2) node[right] {\footnotesize $y$};
\draw (0,0) node [below left] {\footnotesize $O$};
\foreach \x in {-1,1}\draw (\x,0.1)--(\x,-0.1) node [below] {\footnotesize $\x$};
\foreach \y in {-2,2}\draw (0.1,\y)--(-0.1,\y) node [left] {\footnotesize $\y$};
\clip (\xmin,\ymin) rectangle (\xmax,\ymax);
\draw[thick,smooth,samples=200,domain=\xmin:\xmax] plot (\x,{-1*((\x)^3)+0*((\x)^2)+3*(\x)+0});
\draw[dashed] (-1,0)--(-1,-2)--(0,-2);\fill (-1,-2) circle (1pt);
\draw[dashed] (1,0)--(1,2)--(0,2);\fill (1,2) circle (1pt);
\end{tikzpicture}
}

\end{ex}
\begin{ex}%Câu 31.
Đường thẳng $x=1$ là tiệm cận đứng của đồ thị hàm số nào sau đây?
\choice
{\True $y=\dfrac{1+x}{1-x}$}
{$y=\dfrac{2 x-2}{x+2}$}
{$y=\dfrac{1+x^2}{1+x}$}
{$y=\dfrac{2 x^2+3 x+2}{2-x}$}

\end{ex}
\begin{ex}%Câu 32.
Trong không gian $O x y z$, hình chiếu vuông góc của điểm $A(2; 3; 4)$ lên trục $O x$ là điểm nào dưới đây?
\choice
{\True $M(2; 0; 0)$}
{$M(0; 3; 0)$}
{$M(0; 0; 4)$}
{$M(0; 2; 3)$}

\end{ex}
\begin{ex}%Câu 33.
Mặt cầu $(S)\colon x^2+y^2+z^2-8 x+10 y-8=0$ có tâm $I$ và bán kính $R$ lần lượt là:
\choice
{$I(4;-5; 4), R=8$}
{$I(4;-5; 0), R=\sqrt{33}$}
{$I(4; 5; 0), R=7$}
{\True $I(4;-5; 0), R=7$}

\end{ex}
\begin{ex}%Câu 34.
Trong không gian với hệ tọa độ $O x y z$, cho mặt phẳng $(P)\colon 3 x-$ $z+2=0$. Vectơ nào dưới đây là một vectơ pháp tuyến của $(P)$?
\choice
{$\vec{n}_1=(-1; 0;-1)$}
{$\vec{n}_2=(3;-1; 2)$}
{$\vec{n}_3=(3;-1; 0)$}
{\True $\vec{n}_4=(3; 0;-1)$}

\end{ex}
\begin{ex}%Câu 35.
Phần thực và phần ảo của số phức $z=1+2 i$ lần lượt là
\choice
{\True $1$ và $2$}
{$1$ và $i$}
{$1$ và $2 i$}
{$2$ và $1$}

\end{ex}
\begin{ex}%Câu 36.
Cho 2 số phức $z_1=1+i$ và $z_2=2-3 i$. Tính modun của số phức $z_1+z_2$ bằng
\choice
{\True $\left|z_1+z_2\right|=\sqrt{13}$}
{$\left|z_1+z_2\right|=\sqrt{5}$}
{$\left|z_1+z_2\right|=1$}
{$\left|z_1+z_2\right|=5$}

\end{ex}
\begin{ex}%Câu 37.
Cho số phức $z=6+17 i$. Điểm biểu diễn của số phức $z$ trên mặt phẳng tọa độ $O x y$ là:
\choice
{$M(-6;-17)$}
{$M(-17;-6)$}
{$M(17; 6)$}
{\True $M(6; 17)$}

\end{ex}
\begin{ex}%Câu 38.
Tìm tập nghiệm của bất phương trình $6^{2 x+1}-13.6^{x}+6\leq 0$.
\choice
{$[-1; 1]$}
{$(-\infty;-1) \cup(1;+\infty)$}
{\True $\left[\log_6 \dfrac{2}{3}; \log_6 \dfrac{3}{2}\right]$}
{ $\left(-\infty; \log_6 2\right)$}

\end{ex}
\begin{ex}%Câu 39.
Tính thể tích khối tròn xoay sinh ra khi quay tam giác đều $ABC$ cạnh bằng $1$ quanh $AB$.
\choice
{$\dfrac{3\pi}{4}$}
{\True $\dfrac{\pi}{4}$}
{$\dfrac{\pi}{8}$}
{$\dfrac{\pi \sqrt{3}}{2}$}

\end{ex}
\begin{ex}%Câu 40.
Nếu đặt $x=a \sin t$ thì tích phân $\displaystyle\displaystyle\int\limits_0^{a} \dfrac{1}{\sqrt{a^2-x^2}} \mathrm{\,d} x,(a>0)$ trở thành tích phân nào dưới đây?
\choice
{\True $\displaystyle\displaystyle\int\limits_0^{\frac{\pi}{2}} \mathrm{\,d} t$}
{$\displaystyle\displaystyle\int\limits_0^{\frac{\pi}{2}} \dfrac{1}{a} \mathrm{\,d} t$}
{$\displaystyle\displaystyle\int\limits_0^{\frac{\pi}{2}} \dfrac{a}{t} \mathrm{\,d} t$}
{$\displaystyle\displaystyle\int\limits_0^{\frac{\pi}{4}} \mathrm{\,d} t$}

\end{ex}
\begin{ex}%Câu 41.
\immini{
Cho đồ thị hàm số $y=f(x)$ như hình vẽ. Diện tích hình phẳng (phần gạch chéo) được tính bởi công thức nào sau đây?
}
{
 \begin{tikzpicture}[>=stealth, scale=.8,samples=200,smooth,color=\mauchinh,line width=.6pt,xscale=1,yscale=1]
\pgfmathsetmacro{\a}{sqrt(2)}
 \draw[->,thick] (-3,0)--(3,0) node[above] {$x$};
 \draw[->,thick] (0,-1.3)--(0,1.3) node[right] {$y$};
 \draw (0,0) node [above right] {$O$};
 \foreach \x in {}
\draw[thin] (\x,1pt)--(\x,-1pt) node [below] {$\x$};
 \foreach \y in {}
 	\draw[thin] (1pt,\y)--(-1pt,\y) node [left] {$\y$};
 \begin{scope}
\draw[domain=-2.38:2.42,smooth,variable=\x]
plot (\x,{1/4*(\x)^3-(\x)});
 \fill[pattern=north east lines,smooth,pattern color=\mauchinh] (-2,0)--plot[domain=-2:2] (\x,{1/4*(\x)^3-(\x)})--(2,0)--cycle;
\end{scope} 
\path
(-2,0)  node[above, xshift=-0.2cm]{$-2$}
(2,0)  node[below, xshift=0.3cm]{$2$}
;
 \end{tikzpicture}}
\choice
{$\displaystyle\int\limits_{-2}^2 f(x) \mathrm{d} x$}
{$\displaystyle\int\limits_0^{-2} f(x) \mathrm{d} x+\displaystyle\int\limits_0^2 f(x) \mathrm{d} x$}
{\True $\displaystyle\int\limits_2^0 f(x) \mathrm{d} x+\displaystyle\int\limits_{-2}^0 f(x) \mathrm{d} x$}
{$\displaystyle\int\limits_{-2}^1 f(x) \mathrm{d} x+\displaystyle\int\limits_1^2 f(x) \mathrm{d} x$}
\end{ex}

\begin{ex}%Câu 43.
Gọi $z_1$ và $z_2$ lần lượt là nghiệm của phương trình $z^2-2 z+5=0$. Tính $F=\left|z_1\right|+\left|z_2\right|$ 
\choice
{\True $2\sqrt{5}$}
{$10$}
{$3$}
{$6$}

\end{ex}
\begin{ex}%Câu 44.
Cho đường thẳng $(\Delta)$: $\heva{&x=1+t \\& y=2-2\\& z=3+t} t(t \in \mathbb{R})$. Điểm $M$ nào sau đây thuộc đường thẳng $(\Delta)$?
\choice
{$M(1;-2; 3)$}
{\True $M(2; 0; 4)$}
{$M(1; 2;-3)$}
{$M(2; 1; 3)$}

\end{ex}
\begin{ex}%Câu 45.
Cho tứ diện đều $ABCD$ cạnh $a, M$ là trung điểm của $BC$. Tính cosin của góc giữa hai đường thẳng $AB$ và $DM$.
\choice
{$\dfrac{\sqrt{3}}{2}$}
{\True $\dfrac{\sqrt{3}}{6}$}
{$\dfrac{\sqrt{3}}{3}$}
{$\dfrac{1}{2}$}

\end{ex}
\begin{ex}%Câu 46.
Cho hàm số $f(x)$ có đạo hàm $f'(x)=x(x-1)(x+2)^3$. Số điểm cực trị của hàm số đã cho là
\choice
{\True $3$}
{$2$}
{$5$}
{$1$}

\end{ex}
\begin{ex}%Câu 47.
Giá trị nhỏ nhất của hàm số $y=x^3+3 x^2$ trên đoạn $[-4;-1]$ là
\choice
{$-4$}
{\True $-16$}
{$0$}
{$4$}

\end{ex}
\begin{ex}%Câu 48.
Cho $a, b, c$ là các số thực dương khác 1 và thỏa mãn $\log_b a=$ $\dfrac{1}{3}, \log_a c=-2$. Giá trị của $\log_a\left(\dfrac{a^4 \sqrt[3]{b}}{c^3}\right)$ bằng
\choice
{$-2$}
{$-\dfrac{2}{3}$}
{$-\dfrac{5}{6}$}
{\True $11$}

\end{ex}
\begin{ex}%Câu 49.
Số giao điểm của đồ thị hàm số $y=x^3+x+2$ và đường thẳng $y=-2 x+1$ là
\choice
{$3$}
{$0$}
{$2$}
{\True $1$}

\end{ex}
\begin{ex}%Câu 50.
Lớp $12\mathrm{\,A}$ có $20$ học sinh nam và $25$ học sinh nữ. Có bao nhiêu cách chọn một đôi song ca gồm $1$ nam và $1$ nữ?
\choice
{$45$}
{$C_{45}^2$}
{$A_{45}^2$}
{\True $500$}

\end{ex}

\Closesolutionfile{ans}
%\indapan{10}{ans/ans-de5-7}
%%23:14:43 20/5/2022Last Modification of contents