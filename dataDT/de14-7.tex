
\begin{name}
	{\tenchude}
	{\tendethi}
	{\tentruong}
	{\thoigian}
\end{name}
\Opensolutionfile{ans}[ans/ans-de14-7]

\begin{ex}%Câu 29.
Cho số phức $z=4-2 i$, môđun của số phức $(1+i) \bar{z}$ bằng
\choice
{\True $2\sqrt{10}$}
{$24$}
{$2\sqrt{6}$}
{$40$}

\end{ex}
\begin{ex}%Câu 1.
Cho khối chóp có diện tích đáy $B$ và chiều cao là $h$. Thể tích $V$ của khối chóp đã cho được tính theo công thức nào dưới đây?
\choice
{\True $V=\dfrac{1}{3} B h$}
{$V=\dfrac{4}{3} B h$}
{$V=B h$}
{$V=3 B h$}

\end{ex}
\begin{ex}%Câu 2.
Điểm nào dưới đây thuộc đồ thị của hàm số $y=x^3+x-2$?
\choice
{Điểm $M(1; 1)$}
{Điêm $N(1; 2)$}
{Điểm $P(1; 3)$}
{\True Điểm $Q(1; 0)$}

\end{ex}
\begin{ex}%Câu 3.
Với $n$ là số nguyên dương bất kì, $n \geq 3$, công thức nào dưới đây đúng?
\choice
{$C_n^3=\dfrac{(n-3) !}{n !}$}
{$C_n^3=\dfrac{3!(n-3) !}{n !}.$}
{$C_n^3=\dfrac{n !}{3!(n-3) !}$}
{\True $C_n^3=\dfrac{n !}{(n-3) !}$}

\end{ex}
\begin{ex}%Câu 4.
Tập nghiệm của bất phương $\log_3(2 x)>2$ là
\choice
{$(0; 4)$}
{\True $\left(\dfrac{9}{2};+\infty\right)$}
{$\left(0; \dfrac{9}{2}\right)$}
{$(4;+\infty)$}

\end{ex}
\begin{ex}%Câu 5.
Trong không gian $O x y z$, cho mặt cầu $(S)\colon(x-1)^2+(y+3)^2+z^2=9$. Tâm của $(S)$ có tọa dộ là
\choice
{\True $(1;-3; 0)$}
{$(1; 3; 0)$}
{$(-1; 3; 0)$}
{$(-1;-3; 0)$}

\end{ex}
\begin{ex}%Câu 6.
\immini{
Hàm số nào dưới đây có đồ thị như đường cong trong hình bên?
\choice
{$y=\dfrac{3 x-1}{x+2}$}
{$y=x^2-2 x$}
{$y=2 x^3+x^2$}
{\True $y=-x^4+2 x^2$}}
{\vspace{-0.5cm}
\begin{tikzpicture}[scale=1, font=\footnotesize, line join=round, line cap=round, >=stealth,color=\mauchinh]
\def\xmin{-1.54}\def\xmax{1.54}\def\ymin{-0.8}\def\ymax{1.2}
\draw[->,thick] (\xmin-0.2,0)--(\xmax+0.2,0) node[below] {\footnotesize $x$};
\draw[->,thick] (0,\ymin-0.2)--(0,\ymax+0.2) node[right] {\footnotesize $y$};
\draw (0,0) node [below left] {\footnotesize $O$};
\foreach \x in {}\draw (\x,0.1)--(\x,-0.1) node [below] {\footnotesize $\x$};
\foreach \y in {}\draw (0.1,\y)--(-0.1,\y) node [left] {\footnotesize $\y$};
\clip (\xmin,\ymin) rectangle (\xmax,\ymax);
\draw[thick,smooth,samples=200,domain=\xmin:\xmax] plot (\x,{-1*((\x)^4)+2*((\x)^2)+0});
\end{tikzpicture}
}

\end{ex}
\begin{ex}%Câu 7.
Trong KG $Oxyz$ cho hai vectơ $\vec{u}(-1; 2; 0)$ và $\vec{v}(1;-2; 3)$. Tọa độ của vectơ $\vec{u}+\vec{v}$ là
\choice
{$(-2; 4;-3)$}
{$(2;-4; 3)$}
{\True $(0; 0; 3)$}
{$(0; 0;-3)$}

\end{ex}
\begin{ex}%Câu 8.
\immini{
Cho hàm số $y= f(x)$ có bảng biến thiên như hình bên. Số điểm cực trị của hàm số đã cho là
\choice
{$1$}
{$3$}
{$0$}
{\True $2$}}
{\vspace{-0.5cm}
\begin{tikzpicture}[color=\mauchinh]
\tkzTabInit[nocadre,lgt=1.2,espcl=1.6,deltacl=0.6]
{$x$/0.6,$f'(x)$/0.6,$f(x)$/1.8}{$-\infty$,$1$,$5$,$+\infty$}
\tkzTabLine{,-,0,+,0,-,}
\tkzTabVar{+/$+\infty$,-/$-3$,+/$5$,-/$-\infty$}
\end{tikzpicture}

}

\end{ex}
\begin{ex}%Câu 9.
Trong không gian $O x y z$, mặt phẳng đi qua $O$ và nhận vectơ $\vec{n}=(2;-1; 4)$ làm vectơ pháp tuyến có phương trình là:
\choice
{$2 x+y-4 z+1=0$}
{$2 x+y-4 z=0$}
{\True $2 x-y+4 z=0$}
{$2 x-y+4 z+1=0$}

\end{ex}
\begin{ex}%Câu 10.
Cho khối lăng trụ có diện tích đáy $B=5 a^2$ và chiều cao là $h=a$. Thể tích của khối lăng trụ đã cho bằng
\choice
{$\dfrac{5}{3} a^3$}
{\True $5 a^3$}
{$\dfrac{5}{6} a^3$}
{$\dfrac{5}{2} a^3$}

\end{ex}
\begin{ex}%Câu 11.
Phần ảo của số phức $z=3-4 i$ bằng
\choice
{$4$}
{$-3$}
{\True $-4$}
{$3$}

\end{ex}
\begin{ex}%Câu 12.
Số phức liên hợp của $z=-2-i$ là
\choice
{\True $\bar{z}=2+i$}
{$\bar{z}=2+i$}
{$\bar{z}=-2+i$}
{$\bar{z}=-2-i$}

\end{ex}
\begin{ex}%Câu 13.
Đạo hàm của hàm số $y=4^{x}$ là
\choice
{$y'=x.4^{x-1}$}
{\True $y'=4^{x} \cdot \ln 4$}
{$y'=\dfrac{4^{x}}{\ln 4}$}
{$y'=4^{x}$}

\end{ex}
\begin{ex}%Câu 14.
Thể tích của khối cầu bán kính $2 a$ bằng
\choice
{$\dfrac{4}{3} \pi a^3$}
{\True $\dfrac{32}{3} \pi a^3$}
{$32\pi a^3$}
{$\dfrac{8}{3} \pi a^3$}

\end{ex}
\begin{ex}%Câu 15.
\immini{
Cho hàm hàm số $y=f(x)$ có bảng xét dấu của đạo hàm như sau: Hàm số đã cho đồng biến trên khoảng nào dưới đây?
\choice
{\True $(-\infty;-2)$}
{$(-2; 2)$}
{$(-2; 0)$}
{$(0;+\infty)$}
}
{\begin{tikzpicture}[scale=1,line width=.6pt,color=\mauchinh]
\tkzTabInit[nocadre=true,lgt=1,espcl=1.4,deltacl=0.5,lw=0.8]
{$x$ /.7,$y'$/.7}{$-\infty$,$-2$,$0$,$2$,$+\infty$}
%{$x$ /.7,$f'(x)$/.7,$f(x)$/1.8}{$-\infty$,$-1$,$1$,$+\infty$}
\tkzTabLine{,+,0,-,0,+,0,-,}
%\tkzTabVar{-/$2$,+D-/$4$/$-\infty$,+/$3$,-/$-1$}
\end{tikzpicture}
}
\end{ex}
\begin{ex}%Câu 16.
Cho hình nón có bán kính đáy $r$ và độ dài đường sinh $l$. Diện tích xung quanh $S_{\rm x q}$ của hình nón đã cho được tính theo công thức nào dưới đây?
\choice
{$S_{\mathrm{xq}}=\dfrac{4}{3} \pi r l$}
{\True $S_{\mathrm{xq}}=\pi r l$}
{$S_{\mathrm{xq}}=4\pi r l$}
{$S_{\mathrm{xq}}=2\pi r l$}

\end{ex}
\begin{ex}%Câu 17.
Với mọi số thực $a$ dương, $\log_3(3 a)$ bằng
\choice
{$3\log_3 a$}
{$1-\log_3 a$}
{$\log_3 a$}
{\True $1+\log_3 a$}

\end{ex}
\begin{ex}%Câu 18.
Nghiệm của phương trình $5^{x}=2$ là:
\choice
{$x=\log_2 5$}
{\True $x=\log_5 2$}
{$x=\dfrac{2}{5}$}
{$x=\sqrt{5}$}

\end{ex}
\begin{ex}%Câu 19.
Cho hàm số $f(x)=2+\cos x$. Khẳng định nào dưới đây đúng?
\choice
{\True $\displaystyle\int f(x) \mathrm{d} x=2 x+\sin x+C$}
{$\displaystyle\int f(x) \mathrm{d} x=2 x+\cos x+C$}
{$\displaystyle\int f(x) \mathrm{d} x=-\sin x+C$}
{$\displaystyle\int f(x) \mathrm{d} x=2 x-\sin x+C$}

\end{ex}
\begin{ex}%Câu 20.
Trong không gian $O x y z$, đường thẳng đi qua hai điểm $M(-2; 1; 3)$ và nhận vectơ $\vec{u}=(2;-3; 4)$ làm vetơ chỉ phương có phương trình là:
\choice
{\True $\dfrac{x+2}{2}=\dfrac{y-1}{-3}=\dfrac{z-3}{4}$}
{$\dfrac{x-2}{2}=\dfrac{y+1}{-3}=\dfrac{z+3}{4}$}
{$\dfrac{x-2}{-2}=\dfrac{y+3}{1}=\dfrac{z-4}{3}$}
{$\dfrac{x+2}{2}=\dfrac{y-1}{3}=\dfrac{z-3}{4}$}

\end{ex}
\begin{ex}%Câu 21.
Cho hàm số $f(x)=4 x^3-2$. Khẳng định nào dưới đây đúng?
\choice
{\True $\displaystyle\int f(x) \mathrm{d} x=x^4-2 x+C$}
{$\displaystyle\int f(x) \mathrm{d} x=4 x^3-2 x+C$}
{$\displaystyle\int f(x) \mathrm{d} x=12 x^2+C$}
{$\displaystyle\int f(x) \mathrm{d} x=x^4+C$}

\end{ex}
\begin{ex}%Câu 22.
\immini{
Cho hàm số $f(x)=a x^4+b x^2+ c(a, b, c \in \mathbb{R})$ có đồ thị là đường cong trong hình bên. Điểm cực tiểu của hàm số đã cho là
\choice
{$x=-1$}
{$x=2$}
{$x=1$}
{\True $x=0$}}
{\vspace{-0.5cm}
\begin{tikzpicture}[scale=1, font=\footnotesize, line join=round, line cap=round, >=stealth,y=0.8cm,color=\mauchinh]
\def\xmin{-1.72}\def\xmax{1.72}\def\ymin{-1}\def\ymax{3.5}
\draw[->,thick] (\xmin-0.2,0)--(\xmax+0.2,0) node[below] {\footnotesize $x$};
\draw[->,thick] (0,\ymin-0.2)--(0,\ymax+0.2) node[right] {\footnotesize $y$};
\draw (0,0) node [below left] {\footnotesize $O$};
\foreach \x in {-1,1}\draw (\x,0.1)--(\x,-0.1) node [below] {\footnotesize $\x$};
\foreach \y in {2,3}\draw (0.1,\y)--(-0.1,\y) node [left] {\footnotesize $\y$};
\clip (\xmin,\ymin) rectangle (\xmax,\ymax);
\draw[thick,smooth,samples=200,domain=\xmin:\xmax] plot (\x,{-1*((\x)^4)+2*((\x)^2)+2});
\draw[dashed] (-1,0)--(-1,3)--(0,3);\fill (-1,3) circle (1pt);
\draw[dashed] (1,0)--(1,3)--(0,3);\fill (1,3) circle (1pt);
\end{tikzpicture}
}

\end{ex}
\begin{ex}%Câu 23.
Nếu $\displaystyle\int\limits_0^1 f(x) \mathrm{d} x=5$ và $\displaystyle\int\limits_1^3 f(x) \mathrm{d} x=2$ thì $\displaystyle\int\limits_0^3 f(x) \mathrm{d} x$ bằng
\choice
{$10$}
{$-3$}
{$3$}
{\True $7$}

\end{ex}
\begin{ex}%Câu 24.
Cho $f$ là hàm số liên tục trên đoạn $[1; 2]$. Biết $F$ là nguyên hàm của $f$ trên đoạn $[1; 2]$ thỏa mãn $F(1)=-2$ và $F(2)=3$. Khi đó $\displaystyle\int\limits_1^2 f(x) \mathrm{d} x$ bằng
\choice
{$-5$}
{$1$}
{$-1$}
{\True $5$}

\end{ex}

\begin{ex}%Câu 25.
Cho $F(x)=\ln x$ là một nguyên hàm của $\dfrac{f(x)}{x^3}$. Tìm nguyên hàm của hàm số $f'(x) \ln x$.
\choice
{$\displaystyle\int f'(x) \ln x \mathrm{\,d} x=x^2 \ln x+\dfrac{x^2}{2}+C$}
{\True $\displaystyle\int f'(x) \ln x \mathrm{\,d} x=x^2 \ln x-\dfrac{x^2}{2}+C$}
{$\displaystyle\int f'(x) \ln x \mathrm{\,d} x=x^2 \ln x+\dfrac{3 x^2}{2}+C$}
{$\displaystyle\int f'(x) \ln x \mathrm{\,d} x=x \ln x-\dfrac{x^2}{2}+C$}

\end{ex}
\begin{ex}%Câu 26.
Trong không gian $O x y z$, cho điểm $A(1; 2;-1)$ và mặt phẳng $(P)$: $2 x+y-3 z+1=0$. Mặt phẳng đi qua $A$ và song song với mặt phẳng $(P)$ có phương trình là:
\choice
{\True $2 x+y-3 z-7=0$}
{$2 x+y-3 z+7=0$}
{$2 x+y+3 z-1=0$}
{$2 x+y+3 z+1=0$}

\end{ex}
\begin{ex}%Câu 27.
Với $a>0$, đặt $\log_2(2 a)=b$, khi đó $\log_2\left(4 a^3\right)$ bằng
\choice
{$3 b+5$}
{$3 b$}
{$3 b+2$}
{\True $3 b-1$}

\end{ex}
\begin{ex}%Câu 28.
Chọn ngẫu nhiên đồng thời hai số từ tập hợp gồm $17$ số nguyên dương đầu tiên. Xác suất để chọn được hai số chắn bằng
\choice
{\True $\dfrac{7}{34}$}
{$\dfrac{9}{34}$}
{$\dfrac{9}{17}$}
{$\dfrac{8}{17}$}

\end{ex}

\begin{ex}%Câu 30.
Trên đoạn $[-4;-1]$, hàm số $y=-x^4+8 x^2-19$ đạt giá trị lớn nhất tại điểm
\choice
{$x=-3$}
{\True $x=-2$}
{$x=-4$}
{$x=-1$}

\end{ex}
\begin{ex}%Câu 31.
Cho hình chóp $S.ABCD$ có tất cả các cạnh bằng nhau. Góc giữa hai đường thẳng $SB$ và $CD$ bằng
\choice
{\True $60^{\circ}$}
{$90^{\circ}$}
{$45^{\circ}$}
{$30^{\circ}$}

\end{ex}
\begin{ex}%Câu 32.
Trong không gian $O x y$, cho hai điểm $M(1; 1;-1)$ và $N(3; 0; 2)$. Đường thẳng $MN$ có phương trình là:
\choice
{$\dfrac{x+1}{4}=\dfrac{y+1}{1}=\dfrac{z-1}{1}$}
{\True $\dfrac{x-1}{2}=\dfrac{y-1}{-1}=\dfrac{z+1}{3}$}
{$\dfrac{x-1}{4}=\dfrac{y-1}{1}=\dfrac{z+1}{1}$}
{$\dfrac{x+1}{2}=\dfrac{y+1}{-1}=\dfrac{z-1}{3}$}

\end{ex}
\begin{ex}%Câu 33.
Hàm số nào dưới đây đồng biến trên $\mathbb{R}$?
\choice
{\True $y=x^3+4 x$}
{$y=x^3-4 x$}
{$y=x^4-2 x^2$}
{$y=\dfrac{4 x-1}{x+1}$}

\end{ex}
\begin{ex}%Câu 34.
Nếu $\displaystyle\int\limits_0^2 f(x) \mathrm{d} x=2\displaystyle\int\limits_0^2[2 x-f(x)] \mathrm{d} x$ bằng
\choice
{\True $2$}
{$8$}
{$6$}
{$0$}

\end{ex}
\begin{ex}%Câu 35.
\immini{
Đồ thị của hàm số nào dưới đây có
dạng đường cong như hình vẽ
\choice
{$y=-x^3-3 x^2+2$}
{$y=-x^4+3 x^2+2$}
{\True $y=x^4-3 x^2+2$}
{$y=x^3-2 x^2-2$}}
{\vspace{-0.5cm}
\begin{tikzpicture}[scale=1, font=\footnotesize, line join=round, line cap=round, >=stealth,y=0.8cm,color=\mauchinh]
\def\xmin{-1.76}\def\xmax{1.76}\def\ymin{-0.5}\def\ymax{2.29}
\draw[->,thick] (\xmin-0.2,0)--(\xmax+0.2,0) node[below] {\footnotesize $x$};
\draw[->,thick] (0,\ymin-0.2)--(0,\ymax+0.2) node[right] {\footnotesize $y$};
\draw (0,0) node [below left] {\footnotesize $O$};
\foreach \x in {}\draw (\x,0.1)--(\x,-0.1) node [below] {\footnotesize $\x$};
\foreach \y in {2}\draw (0.1,\y)--(-0.1,\y) node [left] {\footnotesize $\y$};
\clip (\xmin,\ymin) rectangle (\xmax,\ymax);
\draw[thick,smooth,samples=200,domain=\xmin:\xmax] plot (\x,{1*((\x)^4)+-3*((\x)^2)+2});
\end{tikzpicture}
}

\end{ex}
\begin{ex}%Câu 36.
Cho cấp số nhân $\left(u_n\right)$ có số hạng đầu $u_1=2$ công bội $q=4$. Giá trị của $u_3$ bằng.
\choice
{\True $32$}
{$16$}
{$8$}
{$6$}

\end{ex}
\begin{ex}%Câu 37.
Một tổ có $6$ học sinh nam và $5$ học sinh nữ. Có bao nhiêu cách chọn một học sinh nam và một học sinh nữ để đi tập văn nghệ.
\choice
{$A_{11}^2$}
{\True $30$}
{$C_{11}^2$}
{$11$}

\end{ex}
\begin{ex}%Câu 38.
Họ tất cả các nguyên hàm của hàm số $f(x)=2^{x}+4 x$ là
\choice
{$2^{x} \ln 2+2 x^2+C$}
{\True $\dfrac{2^{x}}{\ln 2}+2 x^2+C$}
{$\dfrac{2^{x}}{\ln 2}+C$}
{$2^{x} \ln 2+C$}

\end{ex}
\begin{ex}%Câu 39.
Cho khối lăng trụ có đáy là hình vuông cạnh $a$ và chiều cao bằng $3 a$. Thể tích của khối lăng trụ đã cho bằng
\choice
{$a^3$}
{$4 a^3$}
{$\dfrac{4}{3} a^3$}
{\True $3 a^3$}

\end{ex}
\begin{ex}%Câu 40.
Nghiệm của phương trình $\log_2(3 x-8)=2$ là
\choice
{$x=-4$}
{$x=12$}
{\True $x=4$}
{$x=-\dfrac{4}{3}$}

\end{ex}
\begin{ex}%Câu 41.
Cho khối trụ có chiều cao bằng $2\sqrt{3}$ và bán kính đáy bằng $2$. Thể tích của khối trụ đã cho bằng
\choice
{$8\pi$}
{\True $8\sqrt{3} \pi$}
{$\dfrac{8\sqrt{3}}{3} \pi$}
{$24\pi$}

\end{ex}
\begin{ex}%Câu 42.
\immini{
Cho hàm số có bảng biến thiên như hình bên dưới. Hàm số đã cho đồng biến trên khoảng nào dưới đây?
\choice
{\True $(1;+\infty)$}
{$(-3;+\infty)$}
{$(-1; 1)$}
{$(-\infty; 1)$}}
{\vspace{-0.5cm}
\begin{tikzpicture}[color=\mauchinh]
\tkzTabInit[nocadre,lgt=1.2,espcl=1.6,deltacl=0.6]
{$x$/0.6,$f'(x)$/0.6,$f(x)$/1.8}{$-\infty$,$-1$,$1$,$+\infty$}
\tkzTabLine{,+,0,-,0,+,}
\tkzTabVar{-/$-\infty$,+/$1$,-/$-3$,+/$+\infty$}
\end{tikzpicture}

}

\end{ex}
\begin{ex}%Câu 43.
Trong không gian $O x y z$, cho hai điểm $A(1; 1;-2), B(3;-4; 1)$. Tọa độ của vectơ $\overrightarrow{AB}$ là
\choice
{$(-2; 5;-3)$}
{$(2; 5; 3)$}
{\True $(2;-5; 3)$}
{$(2; 5;-3)$}

\end{ex}
\begin{ex}%Câu 44.
Phương trình đường tiệm cận đứng của đồ thị hàm số $y=\dfrac{2 x-3}{x-1}$ là:
\choice
{$y=2$}
{$y=1$}
{\True $x=1$}
{$x=2$}

\end{ex}
\begin{ex}%Câu 45.
Cho hình nón có độ dài đường sinh bằng $3 a$ và bán kính đáy bằng $a$. Diện tích xung quanh của hình nón đã cho bằng
\choice
{$12\pi a^2$}
{\True $3\pi a^2$}
{$6\pi a^2$}
{$\pi a^2$}

\end{ex}
\begin{ex}%Câu 46.
Với $a$ là số thực dương khác $1, \log_{a^2}(a \sqrt{a})$ bằng
\choice
{\True $\dfrac{3}{4}$}
{$3$}
{$\dfrac{3}{2}$}
{$\dfrac{1}{4}$}

\end{ex}
\begin{ex}%Câu 47.
Cho khối chóp có diện tích đáy bằng $a^2$ và chiều cao bằng $2 a$. Thể tích của khối chóp đã cho bằng
\choice
{\True $\dfrac{2 a^3}{3}$}
{$2 a^3$}
{$4 a^3$}
{$a^3$}

\end{ex}
\begin{ex}%Câu 48.
Giá trị nhỏ nhất của hàm số $y=x^4-2 x^2-3$ trên đoạn $[-1; 2]$ bằng
\choice
{\True $-4$}
{$0$}
{$5$}
{$-3$}

\end{ex}
\begin{ex}%Câu 49.
Cho $f(x)$ là một hàm số liên tục trên $\mathbb{R}$ và $F(x)$ là một nguyên hàm của hàm số $f(x)$. Biết $\displaystyle\int\limits_1^3 f(x) \mathrm{d} x=3$ và $F(1)=1$. Giá trị của $F(3)$ bằng
\choice
 {\True $4$}
{$2$}
{$-2$}
{$3$}

\end{ex}
\begin{ex}%Câu 50.
Đạo hàm của hàm số $y=\log_3\left(2 x^2-x+1\right)$ là
\choice
{$\dfrac{2 x-1}{\left(2 x^2-x+1\right) \ln 3}$}
{\True $\dfrac{4 x-1}{\left(2 x^2-x+1\right) \ln 3}$}
{$\dfrac{(4 x-1) \ln 3}{\left(2 x^2-x+1\right)}$}
{$\dfrac{4 x-1}{\left(2 x^2-x+1\right)}$}

\end{ex}

\Closesolutionfile{ans}
%\indapan{10}{ans/ans-de14-7}