
\begin{name}
	{\tenchude}
	{\tendethi}
	{\tentruong}
	{\thoigian}
\end{name}
\Opensolutionfile{ans}[ans/ans-de12-7]

\begin{ex}%Câu 36.
Cho hai số phức $z=3+4 i$ và $w=1-i$. Số phức $z-w$ bằng
\choice
{$7+i$}
{$-2-5 i$}
{$4+3 i$}
{\True $2+5 i$}

\end{ex}
\begin{ex}%Câu 1.
\immini{
Cho hàm số $f(x)$ có bảng biến thiên như hình bên dưới
Hàm số đồng biến trên khoảng nào?
\choice
{$(-2; 2)$}
{\True $(0; 2)$}
{$(-2; 0)$}
{$(2;+\infty)$}}
{
\vspace{-0.5cm}
\begin{tikzpicture}[color=\mauchinh]
\tkzTabInit[nocadre,lgt=1.2,espcl=1.3,deltacl=0.6]
{$x$/0.6,$f'(x)$/0.6,$f(x)$/1.8}{$-\infty$,$-2$,$0$,$2$,$+\infty$}
\tkzTabLine{,+,0,-,0,+,0,-,}
\tkzTabVar{-/$-\infty$,+/$3$,-/$2$,+/$3$,-/$-\infty$}
\end{tikzpicture}
}

\end{ex}
\begin{ex}%Câu 2.
Tiệm cận ngang của đồ thị hàm số $y=\dfrac{2 x+1}{x-1}$ là
\choice
{$y=\dfrac{1}{2}$}
{$y=-1$}
{$y=1$}
{\True $y=2$}

\end{ex}
\begin{ex}%Câu 3.
\immini{
Đồ thị của hàm số nào dưới đây có dạng như đường cong như hình bên
\choice
{$y=-x^4+2 x^2$}
{$y=x^3-3 x^2$}
{\True $y=x^4-2 x^2$}
{$y=-x^3+3 x^2$}}
{\vspace{-0.5cm}
\begin{tikzpicture}[scale=1, font=\footnotesize, line join=round, line cap=round, >=stealth,y=1cm,color=\mauchinh]
\def\xmin{-1.56}\def\xmax{1.56}\def\ymin{-1.1}\def\ymax{1.2}
\draw[->,thick] (\xmin-0.2,0)--(\xmax+0.2,0) node[below] {\footnotesize $x$};
\draw[->,thick] (0,\ymin-0.2)--(0,\ymax+0.2) node[right] {\footnotesize $y$};
\draw (0,0) node [above left] {\footnotesize $O$};
\foreach \x in {}\draw (\x,0.1)--(\x,-0.1) node [below] {\footnotesize $\x$};
\foreach \y in {}\draw (0.1,\y)--(-0.1,\y) node [left] {\footnotesize $\y$};
\clip (\xmin,\ymin) rectangle (\xmax,\ymax);
\draw[thick,smooth,samples=200,domain=\xmin:\xmax] plot (\x,{1*((\x)^4)+-2*((\x)^2)+0});
\end{tikzpicture}
}

\end{ex}
\begin{ex}%Câu 4.
Trong không gian $O x y z$, cho mặt cầu $(S)\colon x^2+y^2+(z-1)^2=16$. Bán kính của $(S)$ là
\choice
{$32$}
{$8$}
{\True $4$}
{$16$}

\end{ex}
\begin{ex}%Câu 5.
Trong mặt phẳng tọa độ, biết điểm $M(-2; 1)$ là điểm biểu diễn số phức $z$. Phần thực của $z$ bằng
\choice
{\True $-2$}
{$2$}
{$1$}
{$-1$}

\end{ex}
\begin{ex}%Câu 6.
Tập xác định của hàm số $y=\log_3 x$ là
\choice
{$(-\infty; 0)$}
{\True $(0;+\infty)$}
{$(-\infty;+\infty)$}
{$[0;+\infty)$}

\end{ex}
\begin{ex}%Câu 7.
Có bao nhiêu cách xếp $5$ học sinh thành một hàng dọc?
\choice
{$1$}
{$25$}
{$5$}
{\True $120$}

\end{ex}
\begin{ex}%Câu 8.
Với $a, b$ là các số thực dương tùy ý và $a \neq 1, \log_{a^3} b$ bằng
\choice
{$3+\log_a b$}
{$3\log_a b$}
{$\dfrac{1}{3}+\log_a b$}
{\True $\dfrac{1}{3} \log_a b$}

\end{ex}
\begin{ex}%Câu 9. 
$\displaystyle\int x^4 \mathrm{\,d} x$ bằng
\choice
{\True $\dfrac{1}{5} x^5+C$}
{$4 x^3+C$}
{$x^5+C$}
{$5 x^5+C$}

\end{ex}
\begin{ex}%Câu 10.
Biết $F(x)=x^3$ là một nguyên hàm của hàm số $f(x)$ trên $\mathbb{R}$. Giá trị của $\displaystyle\int\limits_1^3(1+f(x)) \mathrm{d} x$ bằng
\choice
{$20$}
{$22$}
{$26$}
{\True $28$}

\end{ex}
\begin{ex}%Câu 11.
Cho hình nón có bán kính bằng $3$ và góc ở đỉnh bằng $60^{\circ}$. Diện tích xung quanh của hình nón đã cho bằng
\choice
{\True $18\pi$}
{$36\pi$}
{$6\sqrt{3} \pi$}
{$12\sqrt{3} \pi$}

\end{ex}
\begin{ex}%Câu 12.
Diện tích hình phẳng giới hạn bởi hai đường $y=x^2-2$ và $ y= 3 x-2$ bằng
\choice
{\True $\dfrac{9}{2}$}
{$\dfrac{9\pi}{2}$}
{$\dfrac{125}{6}$}
{$\dfrac{125\pi}{6}$}

\end{ex}
\begin{ex}%Câu 13.
Tập nghiệm của bất phương trình $2^{x^2-7}<4$ là
\choice
{\True $(-3; 3)$}
{$(0; 3)$}
{$(-\infty; 3)$}
{$(3;+\infty)$}

\end{ex}
\begin{ex}%Câu 14.
Cho $a$ và $b$ là hai số thực dương thỏa mãn $9^{\log_3(a b)}=4 a$. Giá trị của $a b^2$ bằng
\choice
{$3$}
{$6$}
{$2$}
{\True $4$}

\end{ex}
\begin{ex}%Câu 15.
Trong không gian $O x y z$, cho điểm $M(2;-1; 2)$ và đường thẳng $d\colon \dfrac{x-1}{2}=\dfrac{y+2}{3}=\dfrac{z-3}{1}$. Mặt phẳng đi qua điểm qua $M$ và vuông góc với $d$ có phương trình là
\choice
{\True $2 x+3 y+z-3=0$}
{$2 x-y+2 z-9=0$}
{$2 x+3 y+z+3=0$}
{$2 x-y+2 z+9=0$}

\end{ex}
\begin{ex}%Câu 16.
Cho hình chóp $S$. $ABC$ và có đáy $ABC$ là tam giác vuông tại $B$, $AB=a, BC=3 a; SA$ vuông góc với mặt phẳng đáy và $SA=\sqrt{30} a$ (tham khảo hình bên). Góc giữa đường thẳng $SC$ và mặt đáy bằng
\choice
{$45^{\circ}$}
{$90^{\circ}$}
{\True $60^{\circ}$}
{$30^{\circ}$}

\end{ex}
\begin{ex}%Câu 17.
Cho $z_0$ là nghiệm phức có phần ảo dương của phương trình $z^2+ 4 z+13=0$. Trên mặt phẳng tọa độ, điểm biểu diễn của số phức $1-z_0$ là
\choice
{$P(-1;-3)$}
{$M(-1; 3)$}
{\True $N(3;-3)$}
{$Q(3; 3)$}

\end{ex}
\begin{ex}%Câu 18.
Trong không gian $O x y z$, cho ba điểm $A(1; 2; 0), B(1; 1; 2)$ và $C(2; 3; 1)$. Đường thẳng đi qua $A$ và song song với $BC$ có phương trình là
\choice
{\True $\dfrac{x-1}{1}=\dfrac{y-2}{2}=\dfrac{z}{-1}$}
{$\dfrac{x-1}{3}=\dfrac{y-2}{4}=\dfrac{z}{3}$}
{$\dfrac{x+1}{3}=\dfrac{y+2}{4}=\dfrac{z}{3}$}
{$\dfrac{x+1}{1}=\dfrac{y+2}{2}=\dfrac{z}{-1}$}

\end{ex}
\begin{ex}%Câu 19.
Giá trị nhỏ nhất của hàm số $f(x)=x^3-30 x$ trên đoạn $[2; 19]$ bằng
\choice
{$20\sqrt{10}$}
{$-63$}
{\True $-20\sqrt{10}$}
{$-52$}

\end{ex}
\begin{ex}%Câu 20.
Cho hàm số $f(x)$ liên tục trên $\mathbb{R}$ và có bảng xét dấu của $f'(x)$
\immini{như sau: Số điểm cực tiểu của hàm số đã cho là
}
{\begin{tikzpicture}[scale=1,line width=.6pt,color=\mauchinh]
\tkzTabInit[nocadre=true,lgt=1,espcl=1.3,deltacl=0.5,lw=0.8]
{$x$ /.7,$y'$/.7}{$-\infty$,$-2$,$1$,$2$,$3$,$+\infty$}
\tkzTabLine{,-,0,+,0,-,d,+,0,+,}
\end{tikzpicture}
}
\choice
{$2$}
{$4$}
{\True $3$}
{$1$}
\end{ex}
\begin{ex}%Câu 21.
Tiệm cận ngang của đồ thị hàm số $y=\dfrac{4 x-1}{x+1}$ là đường thẳng có phương trình:
\choice
{$x=4$}
{$y=1$}
{\True $y=4$}
{$y=-1$}

\end{ex}
\begin{ex}%Câu 22.
\immini{
Cho hàm số $y=a x^4+b x^2+c(a, b, c \in$ $\mathbb{R}$) có đồ thị là đường cong trong hình bên.
Điểm cực đại của hàm số đã cho là:
\choice
{$x=1$}
{$x=-1$}
{$x=-2$}
{\True $x=0$}}
{\vspace{-0.5cm}
\begin{tikzpicture}[scale=1, font=\footnotesize, line join=round, line cap=round, >=stealth,color=\mauchinh,y=0.8cm]
\def\xmin{-1.65}\def\xmax{1.65}\def\ymin{-2.1}\def\ymax{1}
\draw[->,thick] (\xmin-0.2,0)--(\xmax+0.2,0) node[below] {\footnotesize $x$};
\draw[->,thick] (0,\ymin-0.2)--(0,\ymax+0.2) node[right] {\footnotesize $y$};
\draw (0,0) node [below left] {\footnotesize $O$};
\foreach \x in {-1,1}\draw (\x,0.1)--(\x,-0.1) node [below] {\footnotesize $\x$};
\foreach \y in {-2,-1}\draw (0.1,\y)--(-0.1,\y) node [left] {\footnotesize $\y$};
\clip (\xmin,\ymin) rectangle (\xmax,\ymax);
\draw[thick,smooth,samples=200,domain=\xmin:\xmax] plot (\x,{1*((\x)^4)+-2*((\x)^2)+-1});
\draw[dashed] (-1,0)--(-1,-2)--(0,-2);\fill (-1,-2) circle (1pt);
\draw[dashed] (1,0)--(1,-2)--(0,-2);\fill (1,-2) circle (1pt);
\end{tikzpicture}
}

\end{ex}
\begin{ex}%Câu 23.
Với mọi số thực $a$ dương, $\log_4(4 a)$ bằng
\choice
{\True $1+\log_4 a$}
{$1-\log_4 a$}
{$\log_4 a$}
{$4\log_4 a$}

\end{ex}
\begin{ex}%Câu 24.
Cho hình nón có bán kính đáy $r$ và độ dài đường sinh $l$. Diện tích xung quanh $S_{x q}$ của hình nón đã cho được tính theo công thức nào dưới đây?
\choice
{\True $S_{\mathrm{xq}}=\pi r l$}
{$S_{\mathrm{xq}}=2\pi r l$}
{$S_{\mathrm{xq}}=4\pi r l$}
{$S_{\mathrm{xq}}=\dfrac{4}{3} \pi r l$}

\end{ex}
\begin{ex}%Câu 25.
Đạo hàm của hàm số $y=3^{x}$ là
\choice
{$y'=\dfrac{3^{x}}{\ln 3}$}
{$y'=3^{x}$}
{$y'=x 3^{x-1}$}
{\True $y'=3^{x} \ln 3$}

\end{ex}
\begin{ex}%Câu 26.
Cho hình chóp có diện tích đáy $B$ và chiều cao $h$. Thể tích $V$ của khối chóp đã cho được tính theo công thức nào dưới đây?
\choice
{\True $V=\dfrac{1}{3} B h$}
{$V=\dfrac{4}{3} B h$}
{$V=3 B h$}
{$V=B h$}

\end{ex}
\begin{ex}%Câu 27.
Tập xác định của hàm số $y=\log_3(x-3)$ là
\choice
{$(-\infty; 3]$}
{\True $(3;+\infty)$}
{$[3;+\infty)$}
{$(-\infty; 3)$}

\end{ex}
\begin{ex}%Câu 28.
\immini{
Điểm nào trong hình bên là điểm
biểu diễn của số phức $z=-2+i$?
\choice
{\True Điểm $P$}
{Điểm $Q$}
{Diểm $M$}
{Điểm $N$}}
{\begin{tikzpicture}[>=stealth, scale=1,samples=200,smooth,color=\mauchinh,line width=.6pt,xscale=1,yscale=1]
\tikzstyle{every node}=[font=\small]
\pgfmathsetmacro{\a}{sqrt(2)}
 \draw[->,thick] (-2.5,0)--(2.5,0) node[above] {$x$};
 \draw[->,thick] (0,-1.3)--(0,1.5) node[right] {$y$};
 \draw (0,0) node [above right] {$O$};
 \foreach \x in {-2,2,1}
\draw[thin] (\x,1pt)--(\x,-1pt) node [below] {$\x$};
 \foreach \y in {-1,1}
 	\draw[thin] (1pt,\y)--(-1pt,\y) node [left] {$\y$};
 \begin{scope}
\end{scope}
\draw[dashed](-2,0)|-(0,1) (1,0)|-(0,-1) (2,0)|-(0,1);

\draw (-2,1) circle (1pt)node[above]{$P$} (2,1) circle (1pt)node[above]{$Q$} (1,-1) circle (1pt)node[below]{$N$};

 \end{tikzpicture}
}

\end{ex}
\begin{ex}%Câu 29.
Thể tích của khối cầu bán kính $4 a$ bằng
\choice
{$\dfrac{4}{3} \pi a^3$}
{\True $\dfrac{256}{3} \pi a^3$}
{$256\pi a^3$}
{$\dfrac{64}{3} \pi a^3$}

\end{ex}
\begin{ex}%Câu 30.
Phần ảo của số phức $z=2-3 i$ bằng
\choice
{$-2$}
{\True $-3$}
{$3$}
{$2$}

\end{ex}
\begin{ex}%Câu 31.
\immini{
Hàm số nào dưới đây có đồ thị như đường
cong trong hình bên?
\choice
{$y=\dfrac{3 x+1}{x+2}$}
{$y=x^2+2 x$}
{$y=2 x^3-x^2$}
{\True $y=x^4-2 x^2$}}
{\vspace{-0.5cm}
\begin{tikzpicture}[scale=1, font=\footnotesize, line join=round, line cap=round, >=stealth,color=\mauchinh]
\def\xmin{-1.54}\def\xmax{1.54}\def\ymin{-1.1}\def\ymax{1.1}
\draw[->,thick] (\xmin-0.2,0)--(\xmax+0.2,0) node[below] {\footnotesize $x$};
\draw[->,thick] (0,\ymin-0.2)--(0,\ymax+0.2) node[right] {\footnotesize $y$};
\draw (0,0) node [below left] {\footnotesize $O$};
\foreach \x in {}\draw (\x,0.1)--(\x,-0.1) node [below] {\footnotesize $\x$};
\foreach \y in {}\draw (0.1,\y)--(-0.1,\y) node [left] {\footnotesize $\y$};
\clip (\xmin,\ymin) rectangle (\xmax,\ymax);
\draw[thick,smooth,samples=200,domain=\xmin:\xmax] plot (\x,{1*((\x)^4)+-2*((\x)^2)+0});
\end{tikzpicture}
}

\end{ex}
\begin{ex}%Câu 32.
Trong không gian Oxyz cho hai vectơ $\vec{u}=(-1; 2; 0)$ và $\vec{v}= (1;-2; 3)$. Tọa độ của vectơ $\vec{u}+\vec{v}$ là
\choice
{$(0; 0;-3)$}
{\True $(0; 0; 3)$}
{$(-2; 4;-3)$}
{$(2;-4; 3)$}

\end{ex}
\begin{ex}%Câu 33.
Nếu $\displaystyle\int\limits_0^1 f(x) \mathrm{d} x=2$ và $\displaystyle\int\limits_1^3 f(x) \mathrm{d} x=5$ thì $\displaystyle\int\limits_0^3 f(x) \mathrm{d} x$ bằng
\choice
{$10$}
{$3$}
{\True $7$}
{$-3$}

\end{ex}
\begin{ex}%Câu 34.
Cho khối lăng trụ có diện tích đáy $B=3 a^2$ và chiều cao $h=a$. Thể tích của khối lăng trụ đã cho bằng
\choice
{$\dfrac{1}{2} a^3$}
{\True $3 a^3$}
{$\dfrac{3}{2} a^3$}
{$a^3$}

\end{ex}
\begin{ex}%Câu 35.
Cho hàm số $f(x)=4 x^3-3$. Khẳng định nào dưới đây đúng?
\choice
{\True $\displaystyle\int f(x) \mathrm{d} x=x^4-3 x+C$}
{$\displaystyle\int f(x) \mathrm{d} x=x^4+C$}
{$\displaystyle\int f(x) \mathrm{d} x=4 x^3-3 x+C$}
{$\displaystyle\int f(x) \mathrm{d} x=12 x^2+C$}

\end{ex}

\begin{ex}%Câu 37.
Với $n$ là số nguyên dương bất kỳ, $n \geq 5$, công thức nào dưới đây đúng?
\choice
{$C_n^5=\dfrac{n !}{(n-5) !}$}
{\True $C_n^5=\dfrac{n !}{5!(n-5) !}$}
{$C_n^5=\dfrac{5!(n-5) !}{n !}$}
{$C_n^5=\dfrac{(n-5) !}{n !}$}

\end{ex}
\begin{ex}%Câu 38.
Cho hàm số $f(x)=4+\cos x$. Khẳng định nào dưới đây đúng?
\choice
{$\displaystyle\int f(x) \mathrm{d} x=-\sin x+C$}
{\True $\displaystyle\int f(x) \mathrm{d} x=4 x+\sin x+C$}
{$\displaystyle\int f(x) \mathrm{d} x=4 x-\sin x+C$}
{$\displaystyle\int f(x) \mathrm{d} x=4 x+\cos x+C$}

\end{ex}
\begin{ex}%Câu 39.
\immini{
Cho hàm số $y= f(x)$ có bảng biến thiên
như hình bên. Số điểm cực
trị của hàm số đã cho là
\choice
{$0$}
{$1$}
{\True $2$}
{$3$}}
{\vspace{-0.5cm}
\begin{tikzpicture}[color=\mauchinh]
\tkzTabInit[nocadre,lgt=1.2,espcl=1.6,deltacl=0.6]
{$x$/0.6,$f'(x)$/0.6,$f(x)$/1.8}{$-\infty$,$1$,$5$,$+\infty$}
\tkzTabLine{,+,0,-,0,+,}
\tkzTabVar{-/$-\infty$,+/$3$,-/$-5$,+/$+\infty$}
\end{tikzpicture}

}

\end{ex}
\begin{ex}%Câu 40.
Cho hàm số $y=f(x)$ có bảng xét dấu của đạo hàm như sau:
\immini{
Hàm số đã cho nghịch biến trên khoảng nào dưới đây?
}
{\begin{tikzpicture}[scale=1,line width=.6pt,color=\mauchinh]
\tkzTabInit[nocadre=true,lgt=1,espcl=1.3,deltacl=0.5,lw=0.8]
{$x$ /.7,$y'$/.7}{$-\infty$,$-2$,$0$,$2$,$+\infty$}
\tkzTabLine{,+,0,-,0,+,0,-,}
\end{tikzpicture}
}
\choice
{$(0;+\infty)$}
{$(-2; 2)$}
{\True $(-2; 0)$}
{$(-\infty;-2)$}
\end{ex}
\begin{ex}%Câu 41.
Trong không gian $O x y z$, đường thẳng đi qua điểm $M(-2; 1; 3)$ và nhận vectơ $\vec{u}=(1;-3; 5)$ làm vectơ chỉ phương có phương trình là:
\choice
{$\dfrac{x-1}{-2}=\dfrac{y+3}{1}=\dfrac{z-5}{3}$}
{$\dfrac{x-2}{1}=\dfrac{y+1}{-3}=\dfrac{z+3}{5}$}
{$\dfrac{x+2}{1}=\dfrac{y-1}{3}=\dfrac{z-3}{5}$}
{\True $\dfrac{x+2}{1}=\dfrac{y-1}{-3}=\dfrac{z-3}{5}$}

\end{ex}
\begin{ex}%Câu 42.
Nghiệm của phương trình $5^{x}=3$ là:
\choice
{$x=\sqrt[3]{5}$}
{$x=\dfrac{3}{5}$}
{$x=\log_3 5$}
{\True $x=\log_5 3$}

\end{ex}
\begin{ex}%Câu 43.
Cho $f(x)$ là hàm số liên tục trên đoạn $[1; 2]$. Biết $F(x)$ là nguyên hàm của $f(x)$ trên đoạn $[1; 2]$ thỏa mãn $F(1)=-2$ và $F(2)=4$. Khi đó $\displaystyle\int\limits_1^2 f(x) \mathrm{d} x$ bằng
\choice
{\True $6$}
{$2$}
{$-6$}
{$-2$}

\end{ex}
\begin{ex}%Câu 44.
Cho cấp số cộng $\left(u_n\right)$ với $u_1=2$ và $u_2=7$. Công sai của cấp số cộng đã cho bằng
\choice
{\True $5$}
{$\dfrac{2}{7}$}
{$-5$}
{$\dfrac{7}{2}$}

\end{ex}
\begin{ex}%Câu 45.
Trong không gian $O x y z$, cho mặt cầu $(S)\colon(x+1)^2+(y-3)^2+z^2=9$. Tâm của $(S)$ có tọa độ là
\choice
{$(1;-3; 0)$}
{\True $(-1; 3; 0)$}
{$(1; 3; 0)$}
{$(-1;-3; 0)$}

\end{ex}
\begin{ex}%Câu 46.
Điểm nào dưới đây thuộc đồ thị của hàm số $y=x^3-x+2$?
\choice
{Điểm $M(1; 1)$}
{\True Điểm $P(1; 2)$}
{Diểm $Q(1; 3)$}
{Điểm $N(1; 0)$}

\end{ex}
\begin{ex}%Câu 47.
Trong không gian $O x y z$, mặt phẳng đi qua $O$ và nhận vectơ $\vec{n}=(1;-2; 5)$ làm vectơ pháp tuyến có phương trình là
\choice
{$x+2 y-5 z=0$}
{$x+2 y-5 z+1=0$}
{\True $x-2 y+5 z=0$}
{$x-2 y+5 z+1=0$}

\end{ex}
\begin{ex}%Câu 48.
Tập nghiệm của bất phương trình $\log_2(3 x)>5$ là
\choice
{$\left(0; \dfrac{32}{3}\right)$}
{\True $\left(\dfrac{32}{3};+\infty\right)$}
{$\left(0; \dfrac{25}{3}\right)$}
{$\left(\dfrac{25}{3};+\infty\right)$}

\end{ex}
\begin{ex}%Câu 49.
Chọn ngẫu nhiên đồng thời hai số từ tập hợp gồm $19$ số nguyên dương đầu tiên. Xác suất để chọn được hai số chẵn bằng
\choice
{$\dfrac{10}{19}$}
{$\dfrac{5}{19}$}
{\True $\dfrac{4}{19}$}
{$\dfrac{9}{19}$}

\end{ex}
\begin{ex}%Câu 50.
Cho hình chóp $S.ABCD$ có tất cả các cạnh bằng nhau. Góc giữa hai đường thẳng $SC$ và $AB$ bằng
\choice
{$90^{\circ}$}
{\True $60^{\circ}$}
{$30^{\circ}$}
{$45^{\circ}$}

\end{ex}

\Closesolutionfile{ans}
%\indapan{10}{ans/ans-de12-7}
%%10:0:44 19/5/2022Last Modification of contents