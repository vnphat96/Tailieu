
\begin{name}
	{\tenchude}
	{\tendethi}
	{\tentruong}
	{\thoigian}
\end{name}
\Opensolutionfile{ans}[ans/ans-de3-7]

\begin{ex}%Câu 45.
Số phức $ z=(2-3 i)-(-5+i) $ có phần ảo bằng
\choice
{$-2$}
{$ -2 i $}
{$ -4 i $}
{\True $-4$}

\end{ex}
\begin{ex}%Câu 1.
Nghiệm của phương trình $\log_2 x=3\log_2 3$ là
\choice
{$x=3$}
{$x=9$}
{\True $x=27$}
{$x=8$}

\end{ex}
\begin{ex}%Câu 2.
Hàm số $G(x)$ là một nguyên hàm của hàm số $g(x)$ trên tập $K$ và $C$ là hằng số thực tùy ý. Khẳng định nào sau đây là đúng?
\choice
{$\displaystyle\int G'(x) \mathrm{d} x=G(x), \forall x \in K$}
{\True $\displaystyle\int g(x) \mathrm{d} x=G(x)+C$}
{$G'(x)=g(x)+C, \forall x \in K$}
{$g'(x)=G(x), \forall x \in K$}

\end{ex}
\begin{ex}%Câu 3.
Trong không gian $O x y z$, đường thẳng đi qua hai điểm $M(2; 1; 0)$ và $N(1;-1; 3)$ nhận vectơ nào dưới đây là một vectơ chỉ phương?
\choice
{$\vec{u}_3=(1; 0; 1)$}
{$\vec{u}_4=(-1; 1; 3)$}
{$\vec{u}_2=(-1; 2; 3)$}
{\True $\vec{u}_1=(1; 2;-3)$}

\end{ex}
\begin{ex}%Câu 4.
Trong không gian $O x y z$, cho ba điểm $M(1; 0;-1), N(2; 1; 1)$ và $P$. Biết $N$ là trung điểm của đoạn $MP$. Tọa độ của điểm $P$ là
\choice
{\True $(3; 2; 3)$}
{$\left(\dfrac{3}{2}; \dfrac{1}{2}; 0\right)$}
{$(1; 1; 2)$}
{$(3; 1; 0)$}

\end{ex}
\begin{ex}%Câu 5.
Cho các số thực dương $a, b$ thỏa mãn $3^{\log_3 a}=\log_3 \sqrt{b}$. Mệnh đề nào sau đây là đúng?
\choice
{$a=\log_3 b$}
{\True $b=9^{a}$}
{$b=6^{a}$}
{$a=2\log_3 b$}

\end{ex}
\begin{ex}%Câu 6.
Tập xác định của hàm số $y=\ln x-2$ là.
\choice
{$(2;+\infty)$}
{$[0;+\infty)$}
{\True $(0;+\infty)$}
{$(1;+\infty)$}

\end{ex}
\begin{ex}%Câu 7.
Trong không gian $O x y z$, cho mặt phẳng $(\alpha)\colon x+3 y-2 z+9=0$. Vectơ nào dưới đây là một vectơ pháp tuyến của mặt phẳng $(\alpha)$?
\choice
{$\overrightarrow{n_3}=(3;-2; 9)$}
{$\overrightarrow{n_4}=(1; 3; 2)$}
{$\overrightarrow{n_2}=(1;-3; 2)$}
{\True $\overrightarrow{n_1}=(1; 3;-2)$}

\end{ex}
\begin{ex}%Câu 8.
Cho số phức $z=2-i$. Điểm biểu diễn của số phức liên hợp của $z$ có tọa độ là
\choice
{$(-1; 2)$}
{$(2;-1)$}
{\True $(2; 1)$}
{$(1;-2)$}

\end{ex}
\begin{ex}%Câu 9.
	\immini
	{
Cho hàm số $y=f(x)$ có bảng biến thiên của đạo hàm như hình bên. Hàm số đã cho có bao nhiêu điểm cực trị?
}
{
\begin{tikzpicture}[color=\mauchinh]
	\tkzTabInit[nocadre,lgt=1.2,espcl=1.5,deltacl=0.6,lw=0.8pt]
	{$x$/0.6,$f'(x)$/0.6,$f(x)$/1.8}{$-\infty$,$-4$,$5$,$6$,$+\infty$}
	\tkzTabLine{,-,0,+,0,-,0,+,}
	\tkzTabVar{+/$+\infty$,-/$-3$,+/$0$,-/$-4$,+/$+\infty$}
\end{tikzpicture}
}

\choice
{$3$}
{$4$}
{\True $0$}
{$2$}
\end{ex}
\begin{ex}%Câu 10.
Tập xác định của hàm số $y=\log_2(1-x)$ là
\choice
{$(1;+\infty)$}
{$(-\infty;-1]$}
{$[1;+\infty)$}
{\True $(-\infty; 1)$}

\end{ex}
\begin{ex}%Câu 11.
Diện tích xung quanh của hình nón có độ dài đường sinh $l$ và bán kính $r$ bằng
\choice
{\True $\pi r l$}
{$2\pi r l$}
{$\dfrac{1}{3} \pi r l$}
{$4\pi r l$}

\end{ex}

\begin{ex}%Câu 12.
Trong không gian $O x y z$, hình chiếu vuông góc của điểm $M(3;-1; 1)$ trên mặt phẳng $(O x y)$ có tọa độ là
\choice
{$(3; 0; 0)$}
{\True $(3;-1; 0)$}
{$(3; 0; 1)$}
{$(0;-1; 1)$}

\end{ex}
\begin{ex}%Câu 13.
\immini{
Đồ thị hàm số nào dưới đây có dạng đường cong như hình bên?
\choice
{$y=-x^3+3 x$}
{\True $y=x^3-3 x$}
{$y=x^3-3 x+1$}
{$y=x^3+3 x$}}
{\vspace{-0.6cm}
\begin{tikzpicture}[scale=.8, font=\footnotesize, line join=round, line cap=round, >=stealth,color=\mauchinh,y=0.8cm]
\def\xmin{-2.01}\def\xmax{2.01}\def\ymin{-2.1}\def\ymax{2.3}
\draw[->,thick] (\xmin-0.2,0)--(\xmax+0.2,0) node[below] {\footnotesize $x$};
\draw[->,thick] (0,\ymin-0.2)--(0,\ymax+0.2) node[right] {\footnotesize $y$};
\draw (0,0) node [below left] {\footnotesize $O$};
\foreach \x in {-1,1}\draw (\x,0.1)--(\x,-0.1) node [below] {\footnotesize $\x$};
\foreach \y in {-2,2}\draw (0.1,\y)--(-0.1,\y) node [left] {\footnotesize $\y$};
\clip (\xmin,\ymin) rectangle (\xmax,\ymax);
\draw[thick,smooth,samples=200,domain=\xmin:\xmax] plot (\x,{1*((\x)^3)+0*((\x)^2)+-3*(\x)+0});
\draw[dashed] (-1,0)--(-1,2)--(0,2);\fill (-1,2) circle (1pt);
\draw[dashed] (1,0)--(1,-2)--(0,-2);\fill (1,-2) circle (1pt);
\end{tikzpicture}
}

\end{ex}
\begin{ex}%Câu 14.
\immini{
Cho hàm số bậc bốn $y=f(x)$ có đồ thị như hình bên. Số nghiệm của phương trình $3 f(x)+1=0$ là
\choice
{$0$}
{$3$}
{\True $2$}
{$4$}}
{\vspace{-0.6cm}
\begin{tikzpicture}[scale=1, font=\footnotesize, line join=round, line cap=round, >=stealth,y=0.9cm,color=\mauchinh]
\def\xmin{-1.65}\def\xmax{1.65}\def\ymin{-2.1}\def\ymax{1.1}
\draw[->,thick] (\xmin-0.2,0)--(\xmax+0.2,0) node[below] {\footnotesize $x$};
\draw[->,thick] (0,\ymin-0.2)--(0,\ymax+0.2) node[right] {\footnotesize $y$};
\draw (0,0) node [below left] {\footnotesize $O$};
\foreach \x in {-1,1}\draw (\x,0.1)--(\x,-0.1) node [below] {\footnotesize $\x$};
\foreach \y in {-2,-1}\draw (0.1,\y)--(-0.1,\y) node [left] {\footnotesize $\y$};
\clip (\xmin,\ymin) rectangle (\xmax,\ymax);
\draw[thick,smooth,samples=200,domain=\xmin:\xmax] plot (\x,{1*((\x)^4)+-2*((\x)^2)+-1});
\draw[dashed] (-1,0)--(-1,-2)--(0,-2);\fill (-1,-2) circle (1pt);
\draw[dashed] (1,0)--(1,-2)--(0,-2);\fill (1,-2) circle (1pt);
\end{tikzpicture}
}

\end{ex}
\begin{ex}%Câu 15.
Nghiệm của phương trình $2^{1-x}=16$ là
\choice
{$x=7$}
{$x=3$}
{\True $x=-3$}
{$-7$}

\end{ex}
\begin{ex}%Câu 16.
Thể tích của khối lập phương có cạnh bằng $3$ bằng
\choice
{$18$}
{$6$}
{$9$}
{\True $27$}

\end{ex}
\begin{ex}%Câu 17.
Trong không gian $O x y z$, mặt cầu có tâm $I(2;-1; 1)$ và tiếp xúc mặt phẳng $(O y z)$ có phương trình là:
\choice
{$(x+2)^2+(y-1)^2+(z+1)^2=4$}
{$(x+2)^2+(y-1)^2+(z+1)^2=2$}
{$(x-2)^2+(y+1)^2+(z-1)^2=2$}
{\True $(x-2)^2+(y+1)^2+(z-1)^2=4$}

\end{ex}
\begin{ex}%Câu 18.
\immini{
Cho hàm số $f(x)$ có bảng biến thiên như hình bên. Hàm số đã cho đồng biến trên khoảng nào dưới đây?
}
{\vspace{-0.4cm}
\begin{tikzpicture}[color=\mauchinh]
\tkzTabInit[nocadre,lgt=1.2,espcl=1.6,deltacl=0.6]
{$x$/0.6,$f'(x)$/0.6,$f(x)$/1.8}{$-\infty$,$-1$,$0$,$1$,$+\infty$}
\tkzTabLine{,+,0,-,0,+,0,-,}
\tkzTabVar{-/$-\infty$,+/$2$,-/$-1$,+/$2$,-/$-\infty$}
\end{tikzpicture}
}
\choice
{\True $(0;1)$}
{$(1;+\infty)$}
{$(-1;0)$}
{$(-\infty;2)$}
\end{ex}
\begin{ex}%Câu 19.
Cho khối chóp có diện tích đáy bằng $6$, chiều cao bằng $3$. Thể tích của khối chóp đã cho bằng
\choice
{$9$}
{$18$}
{\True $6$}
{$36$}

\end{ex}
\begin{ex}%Câu 20.
Tiệm cận đứng của đồ thị hàm số $y=\dfrac{x-1}{x+1}$ là
\choice
{$x=1$}
{\True $x=-1$}
{$y=-1$}
{$y=1$}

\end{ex}
\begin{ex}%Câu 21.
Trong không gian $O x y z$, cho mặt phẳng $(P)\colon x-2 y+2 z-1=0$. Khoảng cách từ điểm $A(1;-2; 1)$ đến mặt phẳng $(P)$ bằng
\choice
{\True $2$}
{$3$}
{$\dfrac{2}{3}$}
{$\dfrac{7}{3}$}

\end{ex}
\begin{ex}%Câu 22.
Cho $\displaystyle\int\limits_0^1 f(x) \mathrm{d} x=2$ và $\displaystyle\int\limits_1^4  f(x) \mathrm{d} x=-5$. Tích phân $\displaystyle\int\limits_0^4 f(x) \mathrm{d} x$ 
bằng 
\choice
{$-3$}
{$3$}
{ $6$} 
{\True $-6$}


\end{ex}
\begin{ex}%Câu 23.
\immini{
Cho hàm số $y=f(x)$ có bảng biến thiên như hình bên. Hàm số đã cho đạt cực tiểu tại
\choice
{$x=1$}
{\True $x=0$}
{$x=2$}
{$x=5$}}
{\vspace{-0.4cm}
\begin{tikzpicture}[color=\mauchinh]
\tkzTabInit[nocadre,lgt=1.2,espcl=1.8,deltacl=0.6]
{$x$/0.6,$f'(x)$/0.6,$f(x)$/2}{$-\infty$,$0$,$2$,$+\infty$}
\tkzTabLine{,-,0,+,0,-,}
\tkzTabVar{+/$+\infty$,-/$1$,+/$5$,-/$-\infty$}
\end{tikzpicture}

}
\end{ex}
\begin{ex}%Câu 24.
Gọi $z_1, z_2$ là hai nghiệm phức của phương trình $z^2+4 z+7=0$. Gọi $M, N$ là các điểm biểu diễn số phức $z_1, z_2$. Tính độ dài đoạn $MN$.
\choice
{$4$}
{\True $2\sqrt{3}$}
{$\sqrt{3}$}
{$\sqrt{6}$}

\end{ex}
\begin{ex}%Câu 25.
Cho cấp số cộng $\left(u_n\right)$ với $u_1=2$ và $u_2=8$. Công sai của cấp số cộng bằng
\choice
{\True $-6$}
{$4$}
{$10$}
{$6$}

\end{ex}
\begin{ex}%Câu 26.
Có bao nhiêu cách chọn hai học sinh từ một nhóm gồm $8$ học sinh?
\choice
{$8^2$}
{\True $C_8^2$}
{$A_8^2$}
{$2^8$}

\end{ex}
\begin{ex}%Câu 27.
Cho khối trụ có chiều cao $h=3$ và bán kính đáy $r=2$. Thể tích của khối trụ đã cho bằng
\choice
{$16\pi$}
{\True $12\pi$}
{$4\pi$}
{$8\pi$}

\end{ex}
\begin{ex}%Câu 28.
Gọi $z_1$ và $z_2$ lần lượt là hai nghiệm phức của phương trình $z^2+ 2 z+6=0$. Giá trị của $\left(z_1+z_2\right)^2$ bằng
\choice
{$-2$}
{$-4$}
{\True $4$}
{$2$}

\end{ex}
\begin{ex}%Câu 29.
Trong không gian $O x y z$, cho đường thẳng $d\colon \dfrac{x+1}{1}=\dfrac{y-3}{2}=\dfrac{z-1}{-1}$. Một vectơ chỉ phương của $d$ là
\choice
{$\overrightarrow{u_4}=(1;-3;-1)$}
{\True $\overrightarrow{u_1}=(1;-1; 2)$}
{$\overrightarrow{u_3}=(1; 2;-1)$}
{$\overrightarrow{u_2}=(-1; 1; 3)$}

\end{ex}
\begin{ex}%Câu 30.
Cho các số thực $a, b$ thỏa mãn $\log_2\left(2^{a} \cdot 4^{b}\right)=\log_4 2$. Khẳng định nào sau đây đúng?
\choice
{\True $2 a+4 b=1$}
{$2+2 b=1$}
{$2 a+4 b=2$}
{$a+2 b=2$}

\end{ex}
\begin{ex}%Câu 31.
Giá trị lớn nhất của hàm số $f(x)=x^3-\dfrac{3}{2} x^2-6 x$ trên khoảng $(0; 1)$ bằng
\choice
{$0$}
{$\dfrac{13}{2}$}
{$-\dfrac{13}{2}$}
{\True Không tồn tại}

\end{ex}
\begin{ex}%Câu 32.
Cho hai hàm số $f(x)$ và $g(x)$ liên tục trên $\mathbb{R}$ và $a, b, c, k$ là các số thực bất kì. Xét các khẳng định sau
\begin{enumerate}
	

\item $\displaystyle\int k f(x) \mathrm{d} x=k \displaystyle\int f(x) \mathrm{d} x$\\
\item  $ \displaystyle\int(f(x))'\mathrm{d} x=f(x)+C $\\
\item  $ \displaystyle\int[f(x)+g(x)]\mathrm{\,d}x=\displaystyle\int f(x) \mathrm{d} x+ \displaystyle\int g(x) \mathrm{d} x$\\
\item $ \displaystyle\int\limits_a^{b} f(x) \mathrm{d} x=\displaystyle\int\limits_a^{c} f(x) \mathrm{d} x-\displaystyle\int\limits_b^{c} f(x){\rm d}x $
\end{enumerate}
\choice
{$3$}
{\True $2$}
{$4$}
{$1$}

\end{ex}
\begin{ex}%Câu 33.
Tập nghiệm của bất phương trình $ \log_{\frac{1}{2}} x \geq 1 $là
\choice
{\True $ \left(0; \dfrac{1}{2}\right] $}
{$ \left[\dfrac{1}{2};+\infty\right) $}
{$ \left(0; \dfrac{1}{2}\right) $}
{$ \left(-\infty; \dfrac{1}{2}\right] $}

\end{ex}
\begin{ex}%Câu 34.
Cho hình chóp $ S.ABC $ có đáy là tam giác vuông cân tại $ B, AC=2a $, $SA \perp(ABC), SA=2 a $. Gọi $ H, K $ lần lượt là hình chiếu vuông góc của $ A $ lên $ SB, SC $. Góc giữa hai mặt phẳng $ (AHK) $ và $ (ABC) $ bằng
\choice
{$ 30^{\circ} $}
{\True $ 45^{\circ} $}
{$ 60^{\circ} $}
{$ 90^{\circ} $}

\end{ex}
\begin{ex}%Câu 35.
Với a là số thực dương tùy ý, $ \log_3^2\left(a^2\right) $ bằng
\choice
{$ 4+\log_3^2 a $}
{$ 2\log_3^2 a $}
{$ 2+\log_3^2 a $}
{\True $ 4\log_3^2 a $}

\end{ex}
\begin{ex}%Câu 36.
\immini{
Cho hàm số $f(x)$ có bảng xét dấu của $ f'(x) $ như sau. Hàm số $f(x)$ đạt cực đại tại điểm
}
{
\begin{tikzpicture}[scale=1,line width=.6pt,color=\mauchinh]
\tkzTabInit[nocadre=true,lgt=1.5,espcl=1.3,deltacl=0.5,lw=0.8]
{$x$ /.7,$f'(x)$/.7}{$-\infty$,$-3$,$-1$,$1$,$+\infty$}
\tkzTabLine{,-,0,+,0,-,0,+,}
\end{tikzpicture}
}
\choice
{$ x=0 $}
{$x=-3$}
{\True $x=-1$}
{$x=1$}
\end{ex}
\begin{ex}%Câu 37.
Cho mặt cầu có bán kính $ R=3 $. Diện tích của mặt cầu đã cho bằng
\choice
{$ 18\pi $}
{$12\pi$}
{\True $ 36\pi $}
{$ 9\pi $}

\end{ex}
\begin{ex}%Câu 38.
Tập nghiệm của bất phương trình $ 4^{x}-3\cdot 2^{x}+2<0 $ là
\choice
{\True $ [0; 1] $}
{$(1;+\infty)$}
{$ (-\infty; 0) $}
{$(0; 1)$}

\end{ex}
\begin{ex}%Câu 39.
Diện tích hình phẳng giới hạn bởi các đường $ y=x^2-5 x+4 $ và $ y=0 $ bằng
\choice
{\True $ \displaystyle\int\limits_1^4\left(-x^2+5 x-4\right)\mathrm{\,d}x $}
{$ \pi \displaystyle\int\limits_1^4\left(x^2-5 x+4\right) \mathrm{d} x $}
{$ \pi \displaystyle\int\limits_1^4\left(-x^2+5 x-4\right)\mathrm{\,d}x $}
{$ \displaystyle\int\limits_1^4\left(x^2-5 x+4\right)\mathrm{\,d}x $}

\end{ex}
\begin{ex}%Câu 40.
Trong không gian, cho hình vuông $ ABCD $ cạnh bằng $2$. Gọi $M, N$ lần lượt là trung điểm của $ AB $ và $ CD $. Khi quay hình vuông $ ABCD $ xung quanh cạnh $MN$ thì đường gấp khúc $ MBCN $ tạo thành một hình tròn xoay. Diện tích xung quanh của hình tròn xoay đó bằng
\choice
{$ 6\pi $}
{$ 2\pi $}
{$8\pi$}
{\True $4\pi$}

\end{ex}
\begin{ex}%Câu 41.
Trong mặt phẳng $ O x y $, tập hợp tất cả các điểm biểu diễn của số phức $z$ thỏa mãn$ |\bar{z}+1-2 i|=1 $ là đường tròn có tọa độ của tâm là
\choice
{$ (-2;-1) $}
{$(2;-1)$}
{\True $ (-1;-2) $}
{$(-1; 2)$}

\end{ex}
\begin{ex}%Câu 42.
Gọi $z_1, z_2$ là hai nghiệm phức của phương trình $ z^2-4 z+13=0 $. Giá trị $ \left|z_1^2\right|+\left|z_2^2\right| $ bằng
\choice
{$10$}
{$ -10 $}
{\True $26$}
{$ -26 $}

\end{ex}
\begin{ex}%Câu 43.
Cho cấp số nhân $\left(u_n\right)$ với$ u_1=-4 $ và công bội $ q=5 $. Tính $ u_4 $
\choice
{$ u_4=200 $}
{$ u_4=600 $}
{$ u_4=800 $}
{\True $ u_4=-500 $}

\end{ex}
\begin{ex}%Câu 44.
Cho hai số phức $ z_1=2+3 i $ và $ z_2=3-i $ phần thực của số phức $ \left(z_1-i\right) z_2 $ bằng
\choice
{\True $8$}
{$3$}
{$-4$}
{$4$}

\end{ex}

\begin{ex}%Câu 46.
Trong không gian $O x y z$, mặt phẳng $(P)$ đi qua điểm $ M(3;-1; 4) $ đồng thời vuông góc với giá của vectơ $ \vec{a}=(1;-1; 2) $ có phương trình là
\choice
{$ 3 x-y+4 z-12=0 $}
{$ 3 x-y+4 z+12=0 $}
{$ x-y+2 z-12=0 $}
{\True $ x-y+2 z+12=0 $}

\end{ex}
\begin{ex}%Câu 47.
Trong một hộp có $3$ bi đỏ, $5$ bi xanh và $7$ bi vàng. Bốc ngẫu nhiên $4$ viên. Xác suất để bốc được đủ $3$ màu là
\choice
{$ \dfrac{8}{13} $}
{$ \dfrac{5}{13} $}
{$ \dfrac{7}{13} $}
{\True $ \dfrac{6}{13} $}

\end{ex}
\begin{ex}%Câu 48.
Cho một hình tứ diện đều cạnh $ a $ có một đỉnh trùng với đỉnh của hình nón tròn xoay còn ba đỉnh còn lại của tứ diện nằm trên đường tròn đáy của hình nón. Diện tích xung quanh của hình nón là
\choice
{\True $ \dfrac{1}{3} \pi a^2 \sqrt{3} $}
{$ \pi a^2 \sqrt{2} $}
{$ \dfrac{1}{2} \pi a^2 \sqrt{3} $}
{$ \dfrac{\pi a^2 \sqrt{2}}{3} $}

\end{ex}
\begin{ex}%Câu 49.
Tìm các số thực $ a $ và $ b $ thỏa mãn $ 4 a i+(2-b i) i=1+6 i $ với $ i $ là đơn vị ảo.
\choice
{\True $ a=1, b=1 $}
{$ a=-\dfrac{1}{4}, b=6 $}
{$ a=-\dfrac{1}{4}, b=-6 $}
{$ a=1, b=-1 $}

\end{ex}
\begin{ex}%âu 50.
Trong không gian với hệ tọa độ $O x y z$, cho vật thể $ (H) $ giới hạn bởi hai mặt phẳng có phương trình $ x=a $ và $ x=b(a<b) $. Gọi $ S(x) $ là diện tích thiết diện của $ (H) $ bị cắt bởi mặt phẳng vuông góc với trục $ O x $ tại điểm có hoành độ là $x$, với $ a \leq x \leq b $. Giả sử hàm số $ y=S(x) $ liên tục trên đoạn $ [a; b] $. Khi đó, thể tích $ V $ của vật thể $ (H) $ được cho bởi công thức
\choice
{$ V=\pi \displaystyle\int\limits_a^{b} S(x) \mathrm{d} x $}
{$ V=\pi \displaystyle\int\limits_a^{b}[S(x)]^2 \mathrm{\,d} x $}
{\True $ V=\displaystyle\int\limits_a^{b} S(x) \mathrm{d} x $}
{$ V=\displaystyle\int\limits_a^{b}[S(x)]^2 \mathrm{\,d} x $}

\end{ex}

\Closesolutionfile{ans}
%% \indapan{10}{ans/ans-de3-7}
