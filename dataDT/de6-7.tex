
\begin{name}
	{\tenchude}
	{\tendethi}
	{\tentruong}
	{\thoigian}
\end{name}
\Opensolutionfile{ans}[ans/ans-de6-7]

\begin{ex}%Câu 50.
Số phức liên hợp của số phức $z=5-4 i$ là
\choice
{\True $\bar{z}=5+4 i$}
{$\bar{z}=4+5 i$}
{$\bar{z}=-5-4 i$}
{$\bar{z}=4+5 i$}

\end{ex}
\begin{ex}%Câu 1.
Cho cấp số cộng $\left(u_n\right)$ có số hạng đầu $u_1=2$, công sai $d=3$. Số hạng thứ $5$ của $\left(u_n\right)$ bằng
\choice
{\True $14$}
{$10$}
{$162$}
{$30$}

\end{ex}
\begin{ex}%Câu 2.
Phương trình $2020^{4 x-8}=1$ có nghiệm là
\choice
{$x=\dfrac{7}{4}$}
{$x=-2$}
{$x=\dfrac{9}{4}$}
{\True $x=2$}

\end{ex}
\begin{ex}%Câu 3.
Cho khối hộp chữ nhật có độ dài ba kích thước lần lượt là $4; 6; 8$. Thể tích khối hộp chữ nhật đã cho bằng
\choice
{$288$}
{$64$}
{\True $192$}
{$96$}

\end{ex}
\begin{ex}%Câu 4.
Tìm tập xác định cảu hàm số $y={\rm e}^{\log \left(-x^2+3 x\right)}$ 
\choice
{$D=\mathbb{R}$}
{$D=(0;3)$}
{$D=(0;+\infty)$}
{\True $D=(3;+\infty)$}

\end{ex}
\begin{ex}%Câu 5.
Cho hàm số $f(x)$ có đạo hàm $f'(x)=\cos x$ và $f(0)=1$. Giá trị $\displaystyle\displaystyle\int\limits_0^{\pi} f(x) \mathrm{d} x$ bằng:
\choice
{$0$}
{$\pi$}
{$2$}
{\True $2+\pi$}

\end{ex}
\begin{ex}%Câu 6.
Cho hình chóp có đáy là hình vuông cạnh bằng $a$ và chiều cao $3 a$. Thể tích của hình hộp đã cho bằng
\choice
{$a^3$}
{$9 a^3$}
{$\dfrac{1}{3} a^3$}
{\True $3 a^3$}

\end{ex}
\begin{ex}%Câu 7.
Diện tích xung quanh hình trụ có độ dài đường sinh bằng $l$ và bán kính đáy bằng $r$ là:
\choice
{$4\pi r l$}
{\True $2\pi r l$}
{$\pi r l$}
{$\dfrac{1}{3} \pi r l$}

\end{ex}
\begin{ex}%Câu 8.
Cho khối cầu có bán kính $R=2$. Thể tích của khối cầu đã cho bằng
\choice
{$16\pi$}
{\True $\dfrac{32\pi}{3}$}
{$32\pi$}
{$2\pi$}

\end{ex}
\begin{ex}%Câu 9.
Với số thực dương $a$ tùy ý, $\log_3 \sqrt{a}$ bằng
\choice
{$2+\log_3 a$}
{$\dfrac{1}{2}+\log_3 a$}
{$2\log_3 a$}
{\True $\dfrac{1}{2} \log_3 a$}

\end{ex}
\begin{ex}%Câu 10.
Tập nghiệm của bất phương trình $\log (x+9)>1$ là
\choice
{$(2;+\infty)$}
{$(11;+\infty)$}
{$(-\infty; 2)$}
{\True $(1;+\infty)$}

\end{ex}
\begin{ex}%Câu 11.
\immini{
Cho hàm số $f(x)$ có bảng biến thiên như hình bên. Hàm số nghịch biến trên khoảng nào dưới đây?
}
{\vspace{-0.4cm}
\begin{tikzpicture}[color=\mauchinh]
\tkzTabInit[nocadre,lgt=1.2,espcl=1.8,deltacl=0.5,lw=0.8]
{$x$/0.6,$f'(x)$/0.6,$f(x)$/1.6}{$-\infty$,$-1$,$1$,$+\infty$}
\tkzTabLine{,+,0,-,0,+,}
\tkzTabVar{-/$-\infty$,+/$4$,-/$0$,+/$+\infty$}
\end{tikzpicture}

}
\choice
{$(0; 4)$}
{$(-\infty;-1)$}
{\True $(-1; 1)$}
{$(0; 2)$}
\end{ex}
\begin{ex}%Câu 12.
Cho khối nón có chiều cao bằng $2 a$ và bán kính đáy bằng $a$. Thể tích của khối nón đã cho bằng
\choice
{$\dfrac{4\pi a^3}{3}$}
{\True $\dfrac{2\pi a^3}{3}$}
{$\dfrac{\pi a^3}{3}$}
{$2\pi a^3$}

\end{ex}
\begin{ex}%Câu 13.
\immini{
Cho hàm số $y=f(x)$ có bảng biến thiên như hình bên dưới. Khẳng định nào sau đây đúng?
}
{
\begin{tikzpicture}[color=\mauchinh]
\tkzTabInit[lgt=1.2,espcl=1.4,deltacl=0.6]
{$x$/0.6,$f'(x)$/0.6,$f(x)$/1.7}{$-\infty$,$-1$,$0$,$1$,$+\infty$}
\tkzTabLine{,-,0,+,0,-,0,+,}
\tkzTabVar{+/$+\infty$,-/$-4$,+/$-3$,-/$-4$,+/$+\infty$}
\end{tikzpicture}
}
\choice
{Hàm số đạt cực tiểu tại $x=-4$}
{Điểm cực đại của đồ thị hàm số là $x=0$}
{Giá trị cực tiểu của hàm số bằng $1$}
{\True Điểm cực đại của đồ thị hàm số là $A(0;-3)$}
\end{ex}
\begin{ex}%Câu 14.
\immini{
Đồ thị của hàm số nào dưới đây có dạng như đường cong trong hình vẽ?
\choice
{$y=x^2-2 x-1$}
{\True $y=x^3-2 x-1$}
{$y=x^4+2 x^2-1$}
{$y=-x^3+2 x-1$}}
{\vspace{-0.5cm}
\begin{tikzpicture}[scale=1, font=\footnotesize, line join=round, line cap=round, >=stealth,color=\mauchinh]
\def\xmin{-1.62}\def\xmax{1.73}\def\ymin{-2.1}\def\ymax{0.69}
\draw[->,thick] (\xmin-0.2,0)--(\xmax+0.2,0) node[below] {\footnotesize $x$};
\draw[->,thick] (0,\ymin-0.2)--(0,\ymax+0.2) node[right] {\footnotesize $y$};
\draw (0,0) node [below left] {\footnotesize $O$};
\foreach \x in {}\draw (\x,0.1)--(\x,-0.1) node [below] {\footnotesize $\x$};
\foreach \y in {}\draw (0.1,\y)--(-0.1,\y) node [left] {\footnotesize $\y$};
\clip (\xmin,\ymin) rectangle (\xmax,\ymax);
\draw[thick,smooth,samples=200,domain=\xmin:\xmax] plot (\x,{1*((\x)^3)+0*((\x)^2)+-2*(\x)+-1});
\end{tikzpicture}
}

\end{ex}
\begin{ex}%Câu 15.
Số đường tiệm cận của đồ thị hàm số $y=\dfrac{x^2-x+1}{x^2-x-2}$ là
\choice
{$2$}
{$1$}
{\True $3$}
{$4$}

\end{ex}
\begin{ex}%Câu 16.
Nếu $\displaystyle\displaystyle\int\limits_1^2 f(x) \mathrm{d} x=5$ và $\displaystyle\displaystyle\int\limits_1^2[2 f(x)+g(x)] \mathrm{d} x=13$ thì $\displaystyle\displaystyle\int\limits_1^2 g(x) \mathrm{d} x$ bằng
\choice
{$-3$}
{$-1$}
{$1$}
{\True $3$}

\end{ex}
\begin{ex}%Câu 17.
\immini{
Cho hàm số $y=f(x)$ có đồ thị $(C)$ như hình vẽ. Số nghiệm thực của phương trình $4 f(x)-7=0$ là
\choice
{$2$}
{$4$}
{\True $3$}
{$1$}}
{\vspace{-0.5cm}
\begin{tikzpicture}[scale=.7, font=\footnotesize, line join=round, line cap=round, >=stealth,color=\mauchinh,y=0.7cm]
\def\xmin{-1.07}\def\xmax{3.02}\def\ymin{-0.69}\def\ymax{4.5}
\draw[->,thick] (\xmin-0.2,0)--(\xmax+0.2,0) node[below] {\footnotesize $x$};
\draw[->,thick] (0,\ymin-0.2)--(0,\ymax+0.2) node[right] {\footnotesize $y$};
\draw (0,0) node [below left] {\footnotesize $O$};
\foreach \x in {2}\draw (\x,0.1)--(\x,-0.1) node [below] {\footnotesize $\x$};
\foreach \y in {4}\draw (0.1,\y)--(-0.1,\y) node [left] {\footnotesize $\y$};
\clip (\xmin,\ymin) rectangle (\xmax,\ymax);
\draw[thick,smooth,samples=200,domain=\xmin:\xmax] plot (\x,{1*((\x)^3)+-3*((\x)^2)+0*(\x)+4});
\end{tikzpicture}
}

\end{ex}
\begin{ex}%Câu 18.
Gọi $\bar{z}$ là số phức liên hợp của số phức $z=-3+4 i$. Tìm phần thực và phần ảo của số phức $\bar{z}$.
\choice
{Số phức $\bar{z}$ có phần thực bằng $-3$ và phần ảo bằng $4$}
{Số phức $\bar{z}$ có phần thực bằng $3$ và phần ảo bằng $4$}
{\True Số phức $\bar{z}$ có phần thực bằng $-3$ và phần ảo bằng $-4$}
{Số phức $\bar{z}$ có phần thực bằng $3$ và phần ảo bằng $-4$}

\end{ex}
\begin{ex}%Câu 19.
Cho số phức $z$ có điểm biểu diễn trong mặt phẳng tọa độ $O x y$ là điểm $M(3;-5)$. Xác định số phức liên hợp $\bar{z}$ của $z$ 
\choice
{$\bar{z}=-5+3 i$}
{\True $\bar{z}=5+3 i$}
{$\bar{z}=3+5 i$}
{$\bar{z}=3-5 i$}

\end{ex}
\begin{ex}%Câu 20.
Cho hai số phức $z_1=2+3 i$ và $z_2=1-i$. Tính modul của số phức $z_1+z_2$.
\choice
{$5$}
{$\sqrt{5}$}
{$13$}
{\True $\sqrt{13}$}

\end{ex}
\begin{ex}%Câu 21.
Trong không gian $O x y z$, hình chiếu vuông góc của điểm $A(1; 2; 3)$ trên mặt phẳng $(O y z)$ có tọa độ là
\choice
{\True $(0; 2; 3)$}
{$(1; 0; 3)$}
{$(1; 0; 0)$}
{$(0; 2; 0)$}

\end{ex}
\begin{ex}%Câu 22.
Trong không gian $O x y z$, tọa độ tâm của mặt cầu $(S)\colon x^2+y^2+$ $z^2-2 x-4 y-6=0$ là.
\choice
{$(2; 4; 0)$}
{\True $(1; 2; 0)$}
{$(1; 2; 3)$}
{$(2; 4; 6)$}

\end{ex}
\begin{ex}%Câu 23.
không gian $O x y z$, cho mặt phẳng $(\alpha)\colon 2 x+3 z-1=0$ véc-tơ nào dưới đây là một véc-tơ pháp tuyến của mặt phẳng $(\alpha)$?
\choice
{$\vec{n}=(2; 3;-1)$}
{$\vec{n}=(2; 3; 0)$}
{\True $\vec{n}=(-2; 0;-3)$}
{$\vec{n}=(2; 0;-3)$}

\end{ex}
\begin{ex}%Câu 24.
Trong không gian $O x y z$, điểm nào dưới đây thuộc đường thẳng $d\colon\left\{\begin{aligned}&x=1+2 t \\& y=3-t \\& z=3 t\end{aligned}?\right.$ 
\choice
{\True $M(1; 3; 0)$}
{$N(1; 3; 3)$}
{$P(2;-1; 0)$}
{$Q(2;-1; 3)$}

\end{ex}
\begin{ex}%Câu 25.
Cho hàm số $y=f(x)$, bảng xét dấu của $f'(x)$ như sau:
Số điểm cực tiểu của hàm số đã cho là
\choice
{$0$}
{\True $2$}
{$1$}
{$3$}

\end{ex}
\begin{ex}%Câu 26.
Cho hình chóp $S.ABCD$ có đáy là hình thoi tâm $O, \triangle ABD$ đều cạnh $a \sqrt{2}, SA$ vuông góc với mặt phẳng đáy và $SA=\dfrac{3 a \sqrt{2}}{2}$. Góc giữa đường thẳng $SO$ và mặt phẳng $(ABCD)$ bằng
\choice
{$45^{\circ}$,}
{$30^{\circ}$}
{\True $60^{\circ}$}
{$90^{\circ}$}

\end{ex}
\begin{ex}%Câu 27.
Giá trị nhỏ nhất của hàm số $f(x)=x^4-10 x^2+1$ trên đoạn $[-3; 2]$ bằng
\choice
{$1$}
{$-23$}
{\True $-24$}
{$-8$}

\end{ex}
\begin{ex}%Câu 28.
Xét tất cả số thực dương $a$ và $b$ thỏa mãn $\log_3 a=\log_{27}\left(a^2 \sqrt{b}\right)$.
Mệnh đề nào dưới đây là đúng?
\choice
{$a=b^2$}
{$a^3=b$}
{$a=b$}
{\True $a^2=b$}

\end{ex}
\begin{ex}%Câu 29.
Số giao điểm của đồ thị hàm số $y=x^4-5 x^2+4$ với trục hoành là
\choice
{$3$}
{\True $4$}
{$2$}
{$1$}

\end{ex}
\begin{ex}%Câu 30.
Tập nghiệm của bất phương trình $9^{\log_9^2 x+} x^{\log_9 x} \leq 18$ là
\choice
{$[1; 9]$}
{\True $\left[\dfrac{1}{9}; 9\right]$}
{$(0; 1] \cup[9;+\infty)$}
{$\left(0; \dfrac{1}{9}\right] \cup[9;+\infty)$}

\end{ex}
\begin{ex}%Câu 31.
Cho mặt cầu $(S)$. Biết rằng khi cắt mặt cầu $(S)$ bởi một mặt phẳng cách tâm một khoảng có độ dài là $3$ thì được giao tuyến là đường tròn $(T)$ có chu vi là $12\pi$. Diện tích của mặt cầu $(S)$ bằng
\choice
{\True $180\pi$}
{$180\sqrt{3} \pi$}
{$90\pi$}
{$45\pi$}

\end{ex}
\begin{ex}%Câu 32.
Cho tích phân $I=\displaystyle\displaystyle\int\limits_0^4 x \sqrt{x^2+9} \mathrm{\,d} x$. Khi đặt $t=\sqrt{x^2+9}$ thì tích phân đã cho trở thành
\choice
{$I=\displaystyle\displaystyle\int\limits_3^5 t \mathrm{\,d} t$}
{$I=\displaystyle\displaystyle\int\limits_0^4 t \mathrm{\,d} t$}
{$I=\displaystyle\displaystyle\int\limits_0^4 t^2 \mathrm{\,d} t$}
{\True $I=\displaystyle\displaystyle\int\limits_3^5 t^2 \mathrm{\,d} t$}

\end{ex}
\begin{ex}%Câu 33.
Diện tích hình phẳng giới hạn bởi $y=x^2, y=0, x=1, x=2$ bằng
\choice
{$\dfrac{4}{3}$}
{\True $\dfrac{7}{3}$}
{$\dfrac{8}{3}$}
{$1$}

\end{ex}
\begin{ex}%Câu 34.
Cho số phức $z=2-3 i$. Mô-đun của số phức $w=2 z+(1+i) \bar{z}$ bằng
\choice
{$4$}
{$2$}
{\True $\sqrt{10}$}
{$2\sqrt{2}$}

\end{ex}
\begin{ex}%Câu 35.
Gọi $z_1, z_2$ là hai nghiệm phức của phương trình $9 z^2+6 z+4=0$. Giá trị của biểu thức $\dfrac{1}{\left|z_1\right|}+\dfrac{1}{\left|z_2\right|}$ bằng
\choice
{$\dfrac{4}{3}$}
{\True $3$}
{$\dfrac{3}{2}$}
{$6$}

\end{ex}
\begin{ex}%Câu 36.
Cho khối lăng trụ có diện tích đáy $B=8$ và chiều cao $h=3$. Thể tích của khối lăng trụ đã cho bằng.
\choice
{$72$}
{$8$}
{$12$}
{\True $24$}

\end{ex}
\begin{ex}%Câu 37.
Trên mặt phẳng tọa độ, điểm biểu diễn số phức $z=-5+8 i$ là điểm nào dưới đây
\choice
{\True $(-5; 8)$}
{$(5; 8)$}
{$(5;-8)$}
{$(-5;-8)$}

\end{ex}
\begin{ex}%Câu 38.
Cho cấp số nhân $\left(u_n\right)$ với số hạng đầu $u_1=-2$ và $u_2=6$. Khi đó công bội $q$ bằng
\choice
{\True $-3$}
{$3$}
{$-12$}
{$4$}

\end{ex}
\begin{ex}%Câu 39.
Giá trị lớn nhất của hàm số $y=\dfrac{2 x-1}{x+2}$ trên đoạn $[-1; 1]$ là
\choice
{\True $\max\limits_{[-1; 1]} y=\dfrac{1}{3}$}
{$\max\limits_{[-1; 1]} y=1$}
{$\max\limits_{[-1; 1]} y=-3$}
{$\max\limits_{[-1;]]} y=-\dfrac{1}{2}$}

\end{ex}
\begin{ex}%Câu 40.
\immini{
Đồ thị hàm số nào dưới đây có dạng đường cong như hình bên dưới?
\choice
{$y=-x^4-2 x^2+3$}
{$y=x^3-3 x+3$}
{$y=-x^4+2 x^2+3$}
{\True $y=x^4-2 x^2+3$}}
{\vspace{-0.5cm}
\begin{tikzpicture}[scale=1, font=\footnotesize, line join=round, line cap=round, >=stealth,y=0.7cm,color=\mauchinh]
\def\xmin{-1.55}\def\xmax{1.55}\def\ymin{-0.5}\def\ymax{4.3}
\draw[->,thick] (\xmin-0.2,0)--(\xmax+0.2,0) node[below] {\footnotesize $x$};
\draw[->,thick] (0,\ymin-0.2)--(0,\ymax+0.2) node[right] {\footnotesize $y$};
\draw (0,0) node [below left] {\footnotesize $O$};
\foreach \x in {-1,1}\draw (\x,0.1)--(\x,-0.1) node [below] {\footnotesize $\x$};
\foreach \y in {2}\draw (0.1,\y)--(-0.1,\y) node [left] {\footnotesize $\y$};
\clip (\xmin,\ymin) rectangle (\xmax,\ymax);
\draw[thick,smooth,samples=200,domain=\xmin:\xmax] plot (\x,{1*((\x)^4)+-2*((\x)^2)+3});
\draw[dashed] (-1,0)--(-1,2)--(0,2);\fill (-1,2) circle (1pt);
\draw[dashed] (1,0)--(1,2)--(0,2);\fill (1,2) circle (1pt);
\end{tikzpicture}
}

\end{ex}
\begin{ex}%Câu 41.
Nếu $\displaystyle\displaystyle\int\limits_0^3 f(x) \mathrm{d} x=3, \displaystyle\displaystyle\int\limits_3^5 f(x) \mathrm{d} x=7$ thì $\displaystyle\displaystyle\int\limits_0^5 f(x) \mathrm{d} x$ bằng
\choice
{$7$}
{$4$}
{\True $10$}
{$-4$}

\end{ex}
\begin{ex}%Câu 42.
Số cách phân công $3$ học sinh trong $12$ học sinh đi lao động là:
\choice
{$P_{12}$}
{$36$}
{\True $C_{12}^3$}
{$A_{12}^3$}

\end{ex}
\begin{ex}%Câu 43.
Họ nguyên hàm của hàm số $f(x)=4 x^3-2020$ là:
\choice
{\True $x^4-2020 x+C$}
{$12 x^3+C$}
{$x^4+C$}
{$4 x^3-2020 x+C$}

\end{ex}
\begin{ex}%Câu 44.
Cho hàm số $f(x)$ có bảng biến biên dưới đây. 
 \begin{center}
  	\begin{tikzpicture}[scale=1,line width=.6pt,color=\mauchinh]
\tkzTabInit[nocadre=true,lgt=1.5,espcl=2.5,deltacl=0.5,lw=0.8]
%{$x$ /.7,$y'$/.7,$f(x)$/2}{$-\infty$,$-2$,$0$,$1$,$+\infty$}
{$x$ /.7,$f'(x)$/.7,$f(x)$/1.8}{$-\infty$,$-2$,$1$,$+\infty$}
\tkzTabLine{,-,d,-,0,+,}
\tkzTabVar{+/$+\infty$,-D+/$-\infty$/$+\infty$,-/$-1$,+/$+\infty$}
\end{tikzpicture}
\end{center}
Mệnh đề nào sau đây là sai?
\choice
{\True Hàm số đã cho nghịch biến trên khoảng $(-\infty;-1)$}
{Hàm số đã cho nghịch biến trên khoảng $(0; 1)$}
{Hàm số đã cho đồng biến trên khoảng $(1;+\infty)$}
{Hàm số đã cho nghịch biến trên khoảng $(-3;-2)$}

\end{ex}
\begin{ex}%Câu 45.
Khối trụ tròn xoay có bán kính đáy bằng a và chiều cao bằng $2{a}$. Thể tích khối trụ bằng:
\choice
{$\pi a^3$}
{$\dfrac{1}{3} \pi a^3$}
{$\dfrac{2}{3} \pi a^3$}
{\True $2\pi a^3$}

\end{ex}
\begin{ex}%Câu 46.
Cho hai số phức $z_1=2+2 i$ và $z_2=2-i$. Mô-đun của số phức $w=z_1+i z_2$ bằng:
\choice
{$3$}
{\True $5$}
{$\sqrt{5}$}
{$25$}

\end{ex}
\begin{ex}%Câu 47.
Đồ thị hàm số nào sau đây có $3$ điểm cực trị?
\choice
{$y=2 x^4+4 x^2+1$}
{$y=x^4+2 x^2-1$}
{$y=-x^4-x^2+1$}
{\True $y=x^4-2 x^2-1$}

\end{ex}
\begin{ex}%Câu 48.
Tập xác định của hàm số $y=\log_3 x$ là
\choice
{$\mathbb{R}$}
{\True $(0;+\infty)$}
{$[0;+\infty)$}
{$\mathbb{R}^{*}$}

\end{ex}
\begin{ex}%Câu 49.
Trong không gian với hệ tọa độ $O x y z$, cho ba điểm $M(1; 0; 0)$, $N(0; 2; 0), P(0; 0;-3)$. Phương trình mặt phẳng $(MNP)$ là
\choice
{$\dfrac{x}{1}-\dfrac{y}{2}-\dfrac{z}{3}=0$}
{$\dfrac{x}{1}-\dfrac{y}{2}+\dfrac{z}{3}=1$}
{$\dfrac{x}{1}+\dfrac{y}{2}+\dfrac{z}{3}=1$}
{\True $\dfrac{x}{1}+\dfrac{y}{2}-\dfrac{z}{3}=1$}

\end{ex}


\Closesolutionfile{ans}
%% \indapan{10}{ans/ans-de6-7}
