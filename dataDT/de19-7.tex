
\begin{name}
	{\tenchude}
	{\tendethi}
	{\tentruong}
	{\thoigian}
\end{name}
\Opensolutionfile{ans}[ans/ans-de19-7]

\begin{ex}%Câu 44.
Phần thực của số phức $z=-5-4 i$ bằng
\choice
{\True $-5$}
{$-4$}
{$4$}
{$5$}
\end{ex}
\begin{ex}%Câu 1.
Trong không gian $O x y z$, cho mặt phẳng $(P)\colon x-2 y+2 z-3=0$. Vectơ nào dưới đây là một vectơ pháp tuyến của $(P)$?
\choice
{$\overrightarrow{n_3}=(1; 2; 2)$}
{\True $\overrightarrow{n_1}=(1;-2; 2)$}
{$\overrightarrow{n_4}=(1;-2;-3)$}
{$\overrightarrow{n_2}=(1; 2;-2)$}

\end{ex}
\begin{ex}%Câu 2.
Trong không gian $O x y z$, cho mặt cầu $(S)$ có tâm $I(0; 1;-2)$ và bán kinh bằng $3$. Phương trình của $(S)$ là:
\choice
{\True $x^2+(y-1)^2+(z+2)^2=9$}
{$x^2+(y+1)^2+(z-2)^2=9$}
{$x^2+(y-1)^2+(z+2)^2=3$}
{$x^2+(y+1)^2+(z-2)^2=3$}

\end{ex}
\begin{ex}%Câu 3.
Cho hàm số $f(x)=x^2+1$. Khẳng định nào dưới đây đúng?
\choice
{$\displaystyle\int f(x) \mathrm{d} x=x^3+x+C$}
{\True $\displaystyle\int f(x) \mathrm{d} x=\dfrac{x^3}{3}+x+C$}
{$\displaystyle\int f(x) \mathrm{d} x=x^2+x+C$}
{$\displaystyle\int f(x) \mathrm{d} x=2 x+C$}

\end{ex}
\begin{ex}%Câu 4.
Cho hàm số $y=f(x)$ có bảng xét dấu của đạo hàm như sau:
\immini{
Số điểm cực trị của hàm số đã cho là:
}
{

 \begin{nscenter}
  	\begin{tikzpicture}[scale=1,line width=.6pt,color=\mauchinh]
\tkzTabInit[nocadre=true,lgt=1,espcl=1.3,deltacl=0.5,lw=0.8]
{$x$ /.7,$y'$/.7}{$-\infty$,$-3$,$-1$,$1$,$2$,$+\infty$}
%{$x$ /.7,$f'(x)$/.7,$f(x)$/1.8}{$-\infty$,$0$,$3$,$+\infty$}
\tkzTabLine{,+,0,-,0,-,0,+,0,-,}
%\tkzTabVar{-/$3$,+D+/$1$/$2$,-/$-2$,+/$+\infty$}
\end{tikzpicture}
\end{nscenter}
}
\choice
{$2$}
{$3$}
{\True $4$}
{$5$}
\end{ex}
\begin{ex}%Câu 5.
Tập xác định của hàm số $y=6^{x}$ là:
\choice
{$[0;+\infty)$}
{$\mathbb{R} \backslash\{0\}$}
{$(0;+\infty)$}
{\True $\mathbb{R}$}

\end{ex}
\begin{ex}%Câu 6.
Nếu $\displaystyle\int\limits_0^3 f(x) \mathrm{d} x=2$ thì $\displaystyle\int\limits_0^3 3 f(x) \mathrm{d} x$ bằng
\choice
{\True $6$}
{$1$}
{$-1$}
{$0$}

\end{ex}
\begin{ex}%Câu 7.
Trên mặt phẳng tọa độ, điểm $M(-2; 3)$ là điểm biểu diễn số phức nào dưới đây?
\choice
{$z_3=2+3 i$}
{$z_4=-2-3 i$}
{\True $z_1=-2+3 i$}
{$z_2=2-3 i$}

\end{ex}
\begin{ex}%Câu 8.
Cho hàm số $f(x)={\rm e}^{x}+3$. Khẳng định nào dưới đây đúng?
\choice
{\True $\displaystyle\int f(x) \mathrm{d} x={\rm e}^{x}+3 x+C$}
{$\displaystyle\int f(x) \mathrm{d} x={\rm e}^{x}+C$}
{$\displaystyle\int f(x) \mathrm{d} x={\rm e}^{x-3}+C$}
{$\displaystyle\int f(x) \mathrm{d} x={\rm e}^{x}-3 x+C$}

\end{ex}
\begin{ex}%Câu 9.
Cho hàm số $y=f(x)$ có đồ thị là đường cong trong hình bên. Hàm số đã cho đồng biến trong khoảng nào dưới đây?
\choice
{$(-\infty; 2)$}
{\True $(0; 2)$}
{$(-2; 2)$}
{$(2;+\infty)$}

\end{ex}
\begin{ex}%Câu 10.
Đồ thị hàm số $y=-x^3+2 x^2-1$ cắt trục tung tại điểm có tung độ bằng
\choice
{$3$}
{$1$}
{\True $-1$}
{$0$}

\end{ex}
\begin{ex}%Câu 11.
Trên khoảng $(0;+\infty)$, đạo hàm của hàm số $y=x \dfrac{4}{3}$ là:
\choice
{$y'=\dfrac{4}{3} x^{\frac{-1}{3}}$}
{\True $y'=\dfrac{4}{3} x^{\frac{1}{3}}$}
{$y'=\dfrac{3}{7} x \dfrac{7}{3}$}
{$y'=\dfrac{3}{4} x^{\frac{1}{3}}$}

\end{ex}
\begin{ex}%Câu 12.
Cho $a>0$ và $a \neq 1$, khi đó $\log_a \sqrt{a}$ bằng
\choice
{$2$}
{$-2$}
{$-\dfrac{1}{2}$}
{\True $\dfrac{1}{2}$}

\end{ex}
\begin{ex}%Câu 13.
Trong không gian $O x y z$ cho điểm $A(3; 2;-4)$. Tọa độ vectơ $\overrightarrow{OA}$ là
\choice
{$(3;-2;-4)$}
{$(-3;-2; 4)$}
{\True $(3; 2;-4)$}
{$(3; 2; 4)$}

\end{ex}
\begin{ex}%Câu 14.
Tập nghiệm của phương trình $2^{x}>3$ là
\choice
{$\left(\log_3 2;+\infty\right)$}
{$\left(-\infty; \log_2 3\right)$}
{$\left(-\infty; \log_3 2\right)$}
{\True $\left(\log_2 3;+\infty\right)$}

\end{ex}
\begin{ex}%Câu 15.
Cho hai số phức $z=1+2 i$ và $w=3-4 i$. Số phức $z+w$ bằng
\choice
{$2-6 i$}
{$4+2 i$}
{\True $4-2 i$}
{$-2+6 i$}

\end{ex}
\begin{ex}%Câu 16.
\immini{
Cho hàm số $y=f(x)$ có bảng biến thiên như hình vẽ bên. Giá trị cực đại của hàm số đã cho bằng
}
{\vspace{-0.5cm}
\begin{tikzpicture}[color=\mauchinh]
\tkzTabInit[nocadre,lgt=1.2,espcl=1.4,deltacl=0.6]
{$x$/0.6,$f'(x)$/0.6,$f(x)$/1.8}{$-\infty$,$-2$,$0$,$2$,$+\infty$}
\tkzTabLine{,-,0,+,0,-,0,+,}
\tkzTabVar{+/$+\infty$,-/$-1$,+/$0$,-/$-1$,+/$+\infty$}
\end{tikzpicture}
}
\choice
{\True $3$}
{$0$}
{$2$}
{$1$}
\end{ex}
\begin{ex}%Câu 17.
Thể tích khối lập phương cạnh $3 a$ bằng
\choice
{\True $27 a^3$}
{$3 a^3$}
{$9 a^3$}
{$a^3$}

\end{ex}
\begin{ex}%Câu 18.
Tiệm cận đứng của đồ thị hàm số $y=\dfrac{2 x+1}{x-1}$ là đường thẳng có phương trình
\choice
{$x=2$}
{\True $x=1$}
{$x=\dfrac{-1}{2}$}
{$-1$}

\end{ex}
\begin{ex}%Câu 19.
Phần thực của số phức $z=3-2 i$ bằng:
\choice
{$2$}
{$-3$}
{\True $3$}
{$-2$}

\end{ex}
\begin{ex}%Câu 20.
Nghiệm của phương trình $\log_3(2 x)=2$ là:
\choice
{\True $x=\dfrac{9}{2}$}
{$x=9$}
{$x=4$}
{$x=8$}

\end{ex}
\begin{ex}%Câu 21.
Với $n$ là số nguyên dương bất kì, $n \geq 2$, công thức nào sau đây đúng?
\choice
{$A_n^2=\dfrac{(n-2) !}{n !}$}
{$A_n^2=\dfrac{2!}{(n-2) !}$}
{$A_n^2=\dfrac{n !}{2!(n-2) !}$}
{\True $A_n^2=\dfrac{n !}{(n-2) !}$}

\end{ex}
\begin{ex}%Câu 22.
Cho khối trụ có bán kính đáy $r=2$ và chiều cao $h=3$. Thể tích của khối trụ đã cho bằng
\choice
{\True $12\pi$}
{$18\pi$}
{$6\pi$}
{$4\pi$}

\end{ex}
\begin{ex}%Câu 23.
Trong không gian $O x y z$, Cho điểm $M(1; 2;-1)$ và mặt phẳng $(P)\colon 2 x+y-3 z+1=0$ Đường thẳng đi qua $M$ và vuông góc với $(P)$ có phương trình là:
\choice
{$\dfrac{x-1}{2}=\dfrac{y-2}{1}=\dfrac{z+1}{1}$}
{\True $\dfrac{x-1}{2}=\dfrac{y-2}{1}=\dfrac{z+1}{-3}$}
{$\dfrac{x+1}{2}=\dfrac{y+2}{1}=\dfrac{z-1}{1}$.}
{$\dfrac{x+1}{2}=\dfrac{y+2}{1}=\dfrac{z-1}{-3}$}

\end{ex}
\begin{ex}%Câu 24.
Cho hình lăng trụ đứng $ABC \cdot A'B'C'$ có tất cả các cạnh bằng nhau. Góc giữa hai đường thẳng $A'B$ và $CC'$ bằng:
\choice
{\True $45^{\circ}$}
{$30^{\circ}$}
{$90^{\circ}$}
{$60^{\circ}$}

\end{ex}
\begin{ex}%Câu 25.
Cho số phức $z$ thỏa mãn $i z=3+2 i$. Số phức liên hợp của $z$ là:
\choice
{\True $\bar{z}=2+3 i$}
{$\bar{z}=-2-3 i$}
{$\bar{z}=-2+3 i$}
{$\bar{z}=2-3 i$}

\end{ex}
\begin{ex}%Câu 26.
Cho hình chóp $S.ABC$ có đáy là tam giác vuông cân tại $C, AC=a$ và $SA$ vuông góc với mặt phẳng đáy. Khoảng cách từ $B$ đến mặt phẳng $(SAC)$ bằng
\choice
{$\dfrac{1}{2} a$}
{$\sqrt{2} a$}
{$\dfrac{\sqrt{2}}{2} a$}
{\True $a$}

\end{ex}
\begin{ex}%Câu 27.
Từ một hộp chứa $10$ quả bóng gồm $4$ quả màu đỏ và $6$ quả màu xanh. Lấy ngẫu nhiên đồng thời $3$ quả. Xác suất để lấy được $3$ quả màu đỏ bằng
\choice
{$\dfrac{1}{5}$}
{$\dfrac{1}{6}$}
{$\dfrac{2}{5}$}
{\True $\dfrac{1}{30}$}

\end{ex}
\begin{ex}%Câu 28.
Với mọi $a, b$ thỏa mãn $\log_2 a^3+\log_2 b=7$, khẳng định nào dưới đây đúng?
\choice
{$a^3+b=49$}
{$a^3 b=128$}
{\True $a^3+b=128$}
{$a^3 b=49$}

\end{ex}
\begin{ex}%Câu 29.
Trong không gian $O x y z$, cho hai điểm $A(0; 0; 1)$ và $B(1; 2; 3)$. Mặt phẳng đi qua $A$ và vuông góc với $AB$ có phương trình là:
\choice
{$x+2 y+2 z-11=0$}
{\True $x+2 y+2 z-2=0$}
{$x+2 y+4 z-4=0$}
{$x+2 y+4 z-17=0$}

\end{ex}
\begin{ex}%Câu 30.
Trên đoạn $[0; 3]$, hàm số $y=x^3-3 x+4$ đạt giá trị nhỏ nhất tại điểm
\choice
{\True $x=1$}
{$x=0$}
{$x=3$}
{$x=2$}

\end{ex}
\begin{ex}%Câu 31. 
$\displaystyle\int 3 x^2 \mathrm{\,d} x$ bằng
\choice
{\True $x^3+C$}
{$6 x+C$}
{$\dfrac{1}{3} x^3+C$}
{$3 x^3+C$}

\end{ex}
\begin{ex}%Câu 32.
Trên mặt phẳng tọa độ, điểm nào dưới đây là điểm biểu diễn số phức $z=3-2 i$?
\choice
{$P(-3; 2)$}
{$M(-2; 3)$}
{$Q(2;-3)$}
{\True $N(3;-2)$}

\end{ex}
\begin{ex}%Câu 33.
Trong không gian $O x y z$, cho đường thẳng $d\colon \dfrac{x-3}{2}=\dfrac{y+1}{4}=$ $\dfrac{z+2}{-1}$. Điểm nào dưới đây thuộc $d$?
\choice
{$M(3; 1; 2)$}
{$N(2; 4; 1)$}
{$P(2; 4;-1)$}
{\True $Q(3;-1;-2)$}

\end{ex}
\begin{ex}%Câu 34.
Trong không gian $O x y z$, cho mặt cầu $(S)\colon(x-1)^2+(y+2)^2+$ $(z+3)^2=4$. Tâm của $(S)$ có tọa độ là
\choice
{\True $(1;-2;-3)$}
{$(-2; 4; 6)$}
{$(2;-4;-6)$}
{$(-1; 2; 3)$}

\end{ex}
\begin{ex}%Câu 35.
Cho hai số phức $z_1=1-3 i$ và $z_2=3+i$. Số phức $z_1-z_2$ bằng
\choice
{\True $-2-4 i$}
{$2+4 i$}
{$2-4 i$}
{$-2+4 i$}

\end{ex}
\begin{ex}%Câu 36.
Cho mặt cầu có bán kính $r=4$. Diện tích của mặt cầu đã cho bằng
\choice
{\True $64\pi$}
{$\dfrac{64\pi}{3}$}
{$\dfrac{256\pi}{3}$}
{$16\pi$}

\end{ex}
\begin{ex}%Câu 37.
Với $a$ là số thực dương tuỳ ý $\log_2(2 a)$ bằng
\choice
{$2-\log_2 a$}
{$2+\log_2 a$}
{\True $1+\log_2 a$}
{$1-\log_2 a$}

\end{ex}
\begin{ex}%Câu 38.
Trong không gian Oxyz, cho mặt phẳng $(\alpha)\colon 2 x-y+3 z+5=0$. Vectơ nào dưới đây là một vectơ pháp tuyến của $(\alpha)$?
\choice
{$\vec{n}_3=(-2; 1; 3)$}
{$\vec{n}_4=(2; 1;-3)$}
{\True $\vec{n}_2=(2;-1; 3)$}
{$\vec{n}_1=(2; 1; 3)$}

\end{ex}
\begin{ex}%Câu 39.
Trong không gian $O x y z$, diểm nào dưới dây là hình chiếu vuông góc của điểm $A(3; 5; 2)$ trên mặt phẳng $(Oxy)$?
\choice
{$M(3; 0; 2)$}
{$P(0; 5; 2)$}
{$Q(0; 0; 2)$}
{\True $N(3; 5; 0)$}

\end{ex}
\begin{ex}%Câu 40.
Có bao nhiêu cách chọn một học sinh từ một nhóm gồm $5$ học sinh nam và $7$ học sinh nữ?
\choice
{\True $12$}
{$5$}
{$7$}
{$35$}

\end{ex}
\begin{ex}%Câu 41.
\immini{
Đồ thị hàm số nào dưới đây có
dạng như đường cong trong hình bên?
\choice
{$y=-x^3+3 x+1$}
{$y=-x^4+2 x^2+1$}
{$y=x^4-2 x^2+1$}
{\True $y=x^3-3 x+1$}}
{
\begin{tikzpicture}[scale=1, font=\footnotesize, line join=round, line cap=round, >=stealth,y=0.7cm,color=\mauchinh]
\def\xmin{-2.03}\def\xmax{2.03}\def\ymin{-1.31}\def\ymax{3.27}
\draw[->,thick] (\xmin-0.2,0)--(\xmax+0.2,0) node[below] {\footnotesize $x$};
\draw[->,thick] (0,\ymin-0.2)--(0,\ymax+0.2) node[right] {\footnotesize $y$};
\draw (0,0) node [below left] {\footnotesize $O$};
\foreach \x in {}\draw (\x,0.1)--(\x,-0.1) node [below] {\footnotesize $\x$};
\foreach \y in {}\draw (0.1,\y)--(-0.1,\y) node [left] {\footnotesize $\y$};
\clip (\xmin,\ymin) rectangle (\xmax,\ymax);
\draw[thick,smooth,samples=200,domain=\xmin:\xmax] plot (\x,{1*((\x)^3)+0*((\x)^2)+-3*(\x)+1});
\end{tikzpicture}
}

\end{ex}
\begin{ex}%Câu 42.
\immini{
Cho hàm số bậc bốn $y=f(x)$ có đồ thị là đường cong trong hình bên. Số nghiệm thực của phương trình $f(x)=\dfrac{1}{2}$ là
\choice
{$3$}
{$1$}
{\True $2$}
{$4$}}
{\vspace{-0.5cm}
\begin{tikzpicture}[scale=1, font=\footnotesize, line join=round, line cap=round, >=stealth,y=0.8cm,color=\mauchinh]
\def\xmin{-1.65}\def\xmax{1.65}\def\ymin{-0.65}\def\ymax{2.65}
\draw[->] (\xmin-0.2,0)--(\xmax+0.2,0) node[below] {\footnotesize $x$};
\draw[->] (0,\ymin-0.2)--(0,\ymax+0.2) node[right] {\footnotesize $y$};
\draw (0,0) node [below left] {\footnotesize $O$};
\foreach \x in {-1,1}\draw (\x,0.1)--(\x,-0.1) node [below] {\footnotesize $\x$};
\foreach \y in {1,2}\draw (0.1,\y)--(-0.1,\y) node [left] {\footnotesize $\y$};
\clip (\xmin,\ymin) rectangle (\xmax,\ymax);
\draw[thick,smooth,samples=200,domain=\xmin:\xmax] plot (\x,{-1*((\x)^4)+2*((\x)^2)+1});
\draw[dashed] (-1,0)--(-1,2)--(0,2);\fill (-1,2) circle (1pt);
\draw[dashed] (1,0)--(1,2)--(0,2);\fill (1,2) circle (1pt);
\end{tikzpicture}
}

\end{ex}
\begin{ex}%Câu 43.
Cho khối chóp có diện tích đáy $B=2 a^2$ và chiều cao $h=9 a$. Thể tích của khối chóp đã cho bằng
\choice
{$9 a^3$}
{\True $6 a^3$}
{$3 a^3$}
{$18 a^3$}

\end{ex}

\begin{ex}%45
Tập xác định của hàm số $y=2^{x}$ là
\choice
{ $\mathbb{R} \backslash\{0\}$}
{ $[0;+\infty)$}
{ $(0;+\infty)$}
{\True $\mathbb{R}$}

\end{ex}
\begin{ex}%Câu 46.
Nghiệm của phương trình $2^{2 x-1}=2^{x}$ là
\choice
{\True $x=1$}
{$x=2$}
{$x=-1$}
{$x=-2$}

\end{ex}
\begin{ex}%Câu 47.
\immini{
Cho hàm số $f(x)$ có bảng biến thiên như hình bên. Điểm cực đại của hàm số đã cho là
\choice
{$x=2$}
{$x=-2$}
{$x=3$}
{\True $x=-1$}}
{
\begin{tikzpicture}[color=\mauchinh]
\tkzTabInit[nocadre,lgt=1.2,espcl=1.6,deltacl=0.6]
{$x$/0.6,$f'(x)$/0.6,$f(x)$/1.8}{$-\infty$,$-1$,$2$,$+\infty$}
\tkzTabLine{,+,0,-,0,+,}
\tkzTabVar{-/$-\infty$,+/$3$,-/$-2$,+/$+\infty$}
\end{tikzpicture}

}

\end{ex}
\begin{ex}%Câu 48.
Cho hình nón có bán kính đáy $r=2$ và độ dài đường sinh $l=5$.
Diện tích xung quanh của hình nón đã cho bằng
\choice
{$\dfrac{50\pi}{3}$}
{\True $10\pi$}
{$\dfrac{10\pi}{3}$}
{$20\pi$}

\end{ex}
\begin{ex}%Câu 49.
Cho khối trụ có bán kính đáy $r=3$ và chiều cao $h=4$. Thể tích khối trụ đã cho bằng
\choice
{$24\pi$}
{$12\pi$}
{$4\pi$}
{\True $36\pi$}

\end{ex}
\begin{ex}%Câu 50.
\immini{
Cho hàm số $y=f(x)$ có đồ thị là
đường cong trong hình bên. Hàm số đã cho
đồng biến trên khoảng nào dưới đây?
\choice
{\True $(-1; 0)$}
{$(0;+\infty)$}
{$(0; 1)$}
{$(-\infty;-1)$}}
{\vspace{-0.5cm}
\begin{tikzpicture}[scale=1, font=\footnotesize, line join=round, line cap=round, >=stealth,y=1cm,color=\mauchinh]
\def\xmin{-1.56}\def\xmax{1.56}\def\ymin{-1.56}\def\ymax{1.56}
\draw[->,thick] (\xmin-0.2,0)--(\xmax+0.2,0) node[below] {\footnotesize $x$};
\draw[->,thick] (0,\ymin-0.2)--(0,\ymax+0.2) node[right] {\footnotesize $y$};
\draw (0,0) node [below left] {\footnotesize $O$};
\foreach \x in {-1,1}\draw (\x,0.1)--(\x,-0.1) node [below] {\footnotesize $\x$};
\foreach \y in {-1}\draw (0.1,\y)--(-0.1,\y) node [left] {\footnotesize $\y$};
\clip (\xmin,\ymin) rectangle (\xmax,\ymax);
\draw[thick,smooth,samples=200,domain=\xmin:\xmax] plot (\x,{1*((\x)^4)+-2*((\x)^2)+0});
\draw[dashed] (-1,0)--(-1,-1)--(0,-1);\fill (-1,-1) circle (1pt);
\draw[dashed] (1,0)--(1,-1)--(0,-1);\fill (1,-1) circle (1pt);
\end{tikzpicture}
}

\end{ex}

\Closesolutionfile{ans}
%% \indapan{10}{ans/ans-de19-7}
