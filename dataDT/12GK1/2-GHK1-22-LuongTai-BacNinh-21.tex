\begin{name}
	{Biên soạn: Võ Thanh Phong - Cao Thành Thái - Đoàn Minh Tân}
	{Đề thi Giữa kỳ I môn Toán THPT Lương Tài - Bắc Ninh, năm 2020 - 2021}
\end{name}
	\setcounter{ex}{0}\setcounter{bt}{0}
	\Opensolutionfile{ans}[ans/ans-2-GHK1-22-LuongTai-BacNinh-21]

\begin{ex}%[Đề KSCL lần 1 THPT Lương Tài - Bắc Ninh, 2020-2021]%[Võ Thanh Phong, 12EX3]%[2D1Y1-1]
	Hàm số $y=x^{3}+3 x^{2}-4$ nghịch biến trên khoảng nào sau đây?
	\choice
	{$(0 ;+\infty)$}
	{$\mathbb{R}$}
	{\True $(-2 ; 0)$}
	{$(-\infty ;-2)$}
	\loigiai{
	Ta có $y'=3x^2+6x$. Khi đó $y'=0\Leftrightarrow \hoac{&x=0\\&x=-2.}$\\
	Bảng biến thiên
	\begin{center}
		\begin{tikzpicture}[>=stealth]
	\tkzTabInit[nocadre=false,lgt=1.2,espcl=2.5,deltacl=0.6]{$x$/.6 ,$y'$/.6,$y$/2}
	{$-\infty$ , $-2$ , $0$ , $+\infty$}
	\tkzTabLine{ , + , $0$ , - , $0$ , + , }
	\tkzTabVar{-/$-\infty$ , +/ , -/ , +/$+\infty$}
	\begin{scope}[on background layer]\path[white]node{MDD-108};\end{scope}
\end{tikzpicture}
	\end{center}
	Dựa vào bảng biến thiên ta có hàm số nghịch biến trên $(-2 ; 0)$.
	}
\end{ex}


\begin{ex}%[Đề KSCL lần 1 THPT Lương Tài - Bắc Ninh, 2020-2021]%[Võ Thanh Phong, 12EX3]%[2D2B4-1]
	Tìm tất cả các giá trị thực của tham số $a$ để biểu thức $B=\log _{3}(2-a)$ có nghĩa.
	\choice
	 {\True $a<2$}
	 {$a>2$}
	 {$a=3$}
	 {$a \leq 2$}
	\loigiai{
		Để biểu thức $B=\log _{3}(2-a)$ có nghĩa thì $2-a>0\Leftrightarrow a<2.$
	}
\end{ex}


\begin{ex}%[Đề KSCL lần 1 THPT Lương Tài - Bắc Ninh, 2020-2021]%[Võ Thanh Phong, 12EX3]%[1H3B3-3]
	Cho hình chóp $S . ABC$ có đáy $ABC$ là tam giác đều cạnh $a$. Hình chiếu vuông góc của $S$ lên $(ABC)$ trùng với trung điểm của cạnh $BC$. Biết tam giác $SBC$ là tam giác đều. Số đo của góc giữa $SA$  và	$(ABC)$ bằng
	\choice
	 {$75^{\circ}$}
	 {\True $45^{\circ}$}
	 {$30^{\circ}$}
	 {$60^{\circ}$}
	\loigiai{
		\immini{Gọi $H$ trung điểm của $SC$ suy ra $SH\perp (ABC)$.\\
		Do tam giác $SBC$ và tam giác $ABC$ là hai tam giác đều cạnh $a$ nên tam giác $SHA$ vuông cân tại $H$.\\
		 Suy ra $\widehat{(SA,(ABC))}=\widehat{SAH}=45^\circ$.}{	\begin{tikzpicture}[scale=0.6, font=\footnotesize, line join=round, line cap=round, >=stealth]
		\def\ac{4} % cạnh AC
		\def\ab{3} % cạnh AB
		\def\h{5} % chiều cao
		\def\gocA{30} % góc A của đáy
		\coordinate[label=left:$C$] (A) at (0,0);
		\coordinate[label=right:$A$] (C) at (\ac,0);
		\coordinate[label=below left:$B$] (B) at (-\gocA:\ab);
		\coordinate[label=below left:$H$] (H) at ($(A)!.5!(B)$); % Thay đổi số 1/2 để đổi vị trí điểm H
		\coordinate[label=above left:$S$] (S) at ($(H)+(90:\h)$);
		\draw (A)--(B)--(C)--(S)--cycle (H)--(S)--(B);
		\draw[dashed] (A)--(C)--(H);		
		\tkzMarkAngle[size=0.3](S,C,H)
		\tkzMarkRightAngles[size=.3](S,H,C S,H,A)
		\tkzDrawPoints[fill=black](A,B,C,H,S)
		\begin{scope}[on background layer]\path[white]node{MDD-108};\end{scope}
\end{tikzpicture}}
	}
\end{ex}


\begin{ex}%[Đề KSCL lần 1 THPT Lương Tài - Bắc Ninh, 2020-2021]%[Võ Thanh Phong, 12EX3]%[2D2Y1-2]
	Cho các số thực $a$, $b$, $m$, $n$ với $a$, $b>0$, $n \neq 0 .$ Mệnh đề nào sau đây \textbf{sai}?
	\choice
	 {$a^mb^m=(a b)^m$}
	 {$\dfrac{a^m}{a^n}=a^{m-n}$}
	 {$\left(a^m\right)^n=a^{m\cdot n}$}
	 {\True $a^m\cdot a^n=a^{m\cdot n}$}
	\loigiai{
		Ta có  $a^m\cdot a^n=a^{m+ n}$.
	}
\end{ex}


\begin{ex}%[Đề KSCL lần 1 THPT Lương Tài - Bắc Ninh, 2020-2021]%[Võ Thanh Phong, 12EX3]%[2D1B3-1]
	Biết giá  trị lớn  nhất và  giá  trị  nhỏ  nhất  của  hàm  số $y=\dfrac{x^3}3+2x^2+3x-4$ trên $[-4 ; 0]$ lần lượt là $M$ và $m$. Giá trị của $M+m$ bằng
	\choice
	 {$-\dfrac 43$}
	 {$\dfrac 43$}
	 {$-4$}
	{\True $-\dfrac{28}3$}
	\loigiai{
		Tập xác định: $\mathscr{D}=\mathbb{R}$.\\
		Ta có $y'=x^2+4x+3$. Khi đó $y'=0\Leftrightarrow \hoac{&x=-1\in [-4 ; 0] \\&x=-3 \in [-4 ; 0].}$\\
		Ta có $f(0)=-4;f(-1)=-\dfrac{16}3;f(-3)=-4;f(-4)=-\dfrac{16}3$.\\
		Suy ra $M=-\dfrac{16}3$ và $m=-4$.\\
		Vậy $M+m=-\dfrac{28}3$.
	}
\end{ex}


\begin{ex}%[Đề KSCL lần 1 THPT Lương Tài - Bắc Ninh, 2020-2021]%[Võ Thanh Phong, 12EX3]%[2D2B5-2]
	Tìm tập nghiệm của phương trình $4^{x^2}=2^{x+1}$.
	\choice
	 {$S=\left\{-1 ;\dfrac 12\right\}$}
	 {$S=\{0 ; 1\}$}
	 {$S=\left\{\dfrac{1-\sqrt 5}2;\dfrac{1+\sqrt 5}2\right\}$}
	 {\True $S=\left\{-\dfrac 12; 1\right\}$}
	\loigiai{Ta có
		$$4^{x^2}=2^{x+1}\Leftrightarrow 2x^2-x-1=0 \Leftrightarrow \hoac{&x=1\\&x=-\dfrac{1}2.} $$
	}
\end{ex}


\begin{ex}%[Đề KSCL lần 1 THPT Lương Tài - Bắc Ninh, 2020-2021]%[Võ Thanh Phong, 12EX3]%[2D1B1-1]
	Cho hàm số $y=f(x)$ có đạo hàm $f'(x)=x^2+1$. Khẳng định nào sau đây đúng?
	\choice
	 {\True Hàm số đồng biến trên $(-\infty ;+\infty)$}
	 {Hàm số nghịch biến trên $(-\infty ; 1)$}
	 {Hàm số nghịch biến trên $(-\infty ;+\infty)$}
	 {Hàm số nghịch biến trên $(-1 ; 1)$}
	\loigiai{
	Ta có 	$f'(x)=x^2+1>0 $ với mọi $x\in \mathbb{R}$ nên hàm số đồng biến trên $(-\infty ;+\infty)$.
	}
\end{ex}


\begin{ex}%[Đề KSCL lần 1 THPT Lương Tài - Bắc Ninh, 2020-2021]%[Võ Thanh Phong, 12EX3]%[2D1B3-1]
	Tìm giá trị nhỏ nhất $m$ của hàm số $y=x^2+\dfrac 2 x$ trên đoạn $\left[\dfrac 12; 2\right]$.
	\choice
	 {\True $m=3$}
	 {$m=5$}
	 {$m=\dfrac{17}4$}
	 {$m=4$}
	\loigiai{
		Ta có $y'=2x-\dfrac{2}{x^2}$. Khi đó  $y'=0\Leftrightarrow 2x-\dfrac{2}{x^2}=0 \Leftrightarrow x=1\in \left[\dfrac 12; 2\right]$.\\
		Suy ra  $f\left(\dfrac{1}{2}\right)=\dfrac{17}{4};f(2)=5;f(1)=3$.\\
		Vậy $m=3$.
	}
\end{ex}


\begin{ex}%[Đề KSCL lần 1 THPT Lương Tài - Bắc Ninh, 2020-2021]%[Võ Thanh Phong, 12EX3]%[2D2Y5-1]
	Nghiệm của phương trình $\log_3(2x-1)=1$ là
	\choice
	 {$x=0$}
	 {$x=3$}
	 {\True $x=2$}
	 {$x=1$}
	\loigiai{
		Điều kiện: $2x-1>0\Leftrightarrow x>\dfrac{1}{2}$.\\
		Ta có $\log_3(2x-1)=1\Leftrightarrow 2x-1=3 \Leftrightarrow x=2$.\\
		Vậy nghiệm của phương trình là $x=2$.
	}
\end{ex}


\begin{ex}%[Đề KSCL lần 1 THPT Lương Tài - Bắc Ninh, 2020-2021]%[Võ Thanh Phong, 12EX3]%[2D2Y3-2]
	Cho các số thực $0<a \neq 1$, $x>0, y>0, \alpha \neq 0$. Mệnh đề nào sau đây \textbf{sai}?
	\choice
	{$\log_a1=0$}
	{$\log_a\left(x^{\alpha}\right)=\alpha\cdot\log_ax $}
	{$\log_a\dfrac xy=\log_ax-\log_ay $}
	{\True $\log_a(x y)=\log_ax\cdot\log_ay$}
	\loigiai{
		Ta có  $\log_a(x y)=\log_ax+\log_ay$.
	}
\end{ex}


\begin{ex}%[Đề KSCL lần 1 THPT Lương Tài - Bắc Ninh, 2020-2021]%[Võ Thanh Phong, 12EX3]%[2H1Y1-2]
	Trong các mệnh đề sau mệnh đề nào đúng?
	\choice
	 {\True Mỗi hình đa diện có ít nhất bốn đỉnh}
	 {Mỗi hình đa diện có ít nhất ba đỉnh}
	 {Số đỉnh của một hình đa diện lớn hơn hoặc bằng số cạnh của nó}
	 {Số mặt của một hình đa diện lớn hơn hoặc bằng số cạnh của nó}
	\loigiai{
		Mỗi hình đa diện có ít nhất bốn đỉnh.
	}
\end{ex}


\begin{ex}%[Đề KSCL lần 1 THPT Lương Tài - Bắc Ninh, 2020-2021]%[Võ Thanh Phong, 12EX3]%[1D2Y2-1]
	Có bao nhiêu số tự nhiên gồm $3$ chữ số đôi một khác nhau được lập từ các chữ số $1$, $2$, $3$, $4$, $5$, $6$?
	\choice
	 {$720$ số}
	 {$90$ số}
	 {$20$ số}
	 {\True $120$ số}
	\loigiai{
		Ta có $\mathrm{A}_6^3=120$ số.
	}
\end{ex}


\begin{ex}%[Đề KSCL lần 1 THPT Lương Tài - Bắc Ninh, 2020-2021]%[Võ Thanh Phong, 12EX3]%[2D1B4-2]
	 Giá trị của $m$ để đường tiệm cận đứng của đồ thị hàm số $y=\dfrac{m x-1}{2x+m}$ đi qua điểm $A(1 ; 2)$ là
	 \choice
	 {$m=2$}
	 {$m=-4$}
	 {$m=-5$}
	 {\True $m=-2$}
	\loigiai{
		Ta có $2x+m=0\Leftrightarrow x=\dfrac{-m}{2}$.\\
		Để đường tiệm cận đứng của đồ thị hàm số  đi qua điểm $A(1 ; 2)$ thì $\dfrac{-m}{2}=1 \Leftrightarrow m=-2$.
	}
\end{ex}


\begin{ex}%[Đề KSCL lần 1 THPT Lương Tài - Bắc Ninh, 2020-2021]%[Võ Thanh Phong, 12EX3]%[2H1Y3-2]
	Thể tích của khối lập phương có cạnh bằng $a$ là
	\choice
	{ $V=\dfrac{a^3}6$}
	 {\True $V=a^3$}
	 {$V=\dfrac{a^3}3$}
	 {$V=\dfrac{2a^3}3$}
	\loigiai{
		Ta có $V=a^3$.
	}
\end{ex}


\begin{ex}%[Đề KSCL lần 1 THPT Lương Tài - Bắc Ninh, 2020-2021]%[Võ Thanh Phong, 12EX3]%[2D1Y1-2]
	\immini{Cho đồ thị hàm số $y=f(x)$ liên tục trên $\mathbb{R}$ và có đồ thị như hình vẽ.
	Hàm số đồng biến trên khoảng nào dưới đây?
	\choice
	 {$(-\infty ; 0)$}
	 {$(2 ;+\infty)$}
	 {\True $(0 ; 2)$}
	 {$(-2 ; 2)$}}{	\begin{tikzpicture}[>=stealth,x=1cm,y=1cm,scale=0.4]
	 \def\a{-1} 
	 \def\b{3}
	 \def\c{0}
	 \def\d{-2}
	 \draw[->] (-5,0) -- (5,0)node[below]{\scriptsize $x$};
	 \draw[->] (0,-5) -- (0,5) node[left] {\scriptsize $y$};
	 \draw[] (2,0) node[below] {\scriptsize $2$};
	 \draw[] (0,2) node[left] {\scriptsize $2$};
	 \draw[] (0,-2) node[left] {\scriptsize $-2$};
	 \draw[] (1,0) node[below left] {\scriptsize $1$};
	 \draw[dashed] (0,2) -- (2,2)--(2,0);
	 \draw (0,0)node[above right]{\scriptsize $O$};
	 \clip (-4,-4)rectangle(4,4);
	 \draw[thick,samples=150,smooth,domain=-4:4] plot(\x,{\a*(\x)^3+(\b)*(\x)^2+(\c)*\x+(\d)});
	 \begin{scope}[on background layer]\path[white]node{MDD-108};\end{scope}
\end{tikzpicture}}
	\loigiai{
		Dựa vào đồ thị hàm số ta có hàm số đồng biến trên  $(0 ; 2)$.
	}
\end{ex}


\begin{ex}%[Đề KSCL lần 1 THPT Lương Tài - Bắc Ninh, 2020-2021]%[Võ Thanh Phong, 12EX3]%[2D1B2-3]
	Tiếp tuyến của đồ thị hàm số $y=\dfrac{x^3}3-2x^2+3x+1$ song song với đường thẳng $y=3 x+1$ có
	phương trình là
	\choice
	 {$y=-\dfrac 13x-1$}
	 {\True $y=3x-\dfrac{29}3$}
	 {$y=3x-\dfrac{29}3, y=3x+1$}
	 {$y=-\dfrac 13x+\dfrac{29}3$}
	\loigiai{
		Ta có $y'=x^2-4x+3$.\\
		Gọi $M(x_0;y_0)$ là tọa độ tiếp điểm.\\
		Do tiếp tuyến của đồ thị hàm số $y=\dfrac{x^3}3-2x^2+3x+1$ song song với đường thẳng $y=3 x+1$ nên 
		$$y'(x_0)=x_0^2-4x_0+3=3\Leftrightarrow \hoac{&x_0=0\\&x_0=4.}$$
		Với $x_0=0\Rightarrow y_0=1$. Suy ra phương trình tiếp tuyến là
		$y=3(x-0)+1=3x+1$ loại vì trùng với đường thẳng đã cho.\\
		Với $x_0=4\Rightarrow y_0=\dfrac{7}{3}$. Suy ra phương trình tiếp tuyến là
		$y=3(x-4)+\dfrac{7}{3}=3x-\dfrac{29}3$ (nhận).
	}
\end{ex}


\begin{ex}%[Đề KSCL lần 1 THPT Lương Tài - Bắc Ninh, 2020-2021]%[Võ Thanh Phong, 12EX3]%[0H3Y1-2]
	Đường thẳng đi qua $A(-1 ; 2),$ nhận $\vec{n}=(2 ;-4)$ làm véc-tơ pháp tuyến có phương trình là
	\choice
	 {\True $x-2y+5=0$}
	 {$x-2y-4=0$}
	 {$x+y+4=0$}
	 {$-x+2y-4=0$}
	\loigiai{
		Phương trình đường thẳng có dạng \\
		$A(x-x_0)+B(y-y_0)=0\Rightarrow 2(x+1)-4(y-2)=0\Leftrightarrow 2x-4y+10=0\Leftrightarrow x-2y+5=0.$
	}
\end{ex}


\begin{ex}%[Đề KSCL lần 1 THPT Lương Tài - Bắc Ninh, 2020-2021]%[Võ Thanh Phong, 12EX3]%[1D2Y2-1]
	Số cách chọn $5$ học sinh trong một lớp có $25$ học sinh nam và $16$ học sinh nữ là
	\choice
	 {$\mathrm{C}_{16}^{5}$}
	 {$\mathrm{A}_{41}^{5}$}
	 {$\mathrm{C}_{25}^{5}$}
	 {\True $\mathrm{C}_{41}^{5}$}
	\loigiai{
		Tổng số học sinh là $25+16=41$ học sinh.\\
		Số cách chọn $5$ học sinh trong số $41$ học sinh là $\mathrm{C}_{41}^{5}$.
	}
\end{ex}


\begin{ex}%[Đề KSCL lần 1 THPT Lương Tài - Bắc Ninh, 2020-2021]%[Võ Thanh Phong, 12EX3]%[2H1Y1-1]
	Trong hình chóp đều, khẳng định nào sau đây đúng?
	\choice
	 {\True Tất cả các cạnh bên bằng nhau}
	 {Tất cả các mặt bằng nhau}
	 {Tất cả các cạnh bằng nhau}
	 {Một cạnh đáy bằng cạnh bên}
	\loigiai{
		Hình chóp đều là hình chóp có tất cả các cạnh bên bằng nhau.
	}
\end{ex}


\begin{ex}%[Đề KSCL lần 1 THPT Lương Tài - Bắc Ninh, 2020-2021]%[Võ Thanh Phong, 12EX3]%[2H1Y3-2]
	Cho khối lăng trụ đứng có cạnh bên bằng $5$, đáy là hình vuông có cạnh bằng $4.$ Thể tích khối lăng trụ là
	\choice
	 {$100$}
	 {$20$}
	 {$64$}
	 {\True $80$}
	\loigiai{
		Ta có diện tích đáy là $B=4^2=16$.\\
		Suy ra thể tích khối lăng trụ là $V=Bh=16\cdot 5=80$ (đvtt).
	}
\end{ex}



\begin{ex}%[Đề KSCL lần 1 THPT Lương Tài - Bắc Ninh, 2020-2021]%[Cao Thành Thái, 12-EX-3-2021]%[2D1Y4-1]
	Đường tiệm cận đứng của đồ thị hàm số $y=\dfrac{2x-3}{x-1}$ là
	\choice
	{$y=2$}
	{$y=3$}
	{\True $x=1$}
	{$x=\dfrac{3}{2}$}
	\loigiai
	{
		Tập xác định của hàm số là $\mathscr{D}=\mathbb{R}\setminus \{1\}$.\\
		Vì $\lim\limits_{x\to 1^+}y= \lim\limits_{x\to 1^+}\dfrac{2x-3}{x-1}=-\infty$ nên đường thẳng $x=1$ là tiệm cận đứng của đồ thị hàm số $y=\dfrac{2x-3}{x-1}$.
	}
\end{ex}

\begin{ex}%[Đề KSCL lần 1 THPT Lương Tài - Bắc Ninh, 2020-2021]%[Cao Thành Thái, 12-EX-3-2021]%[2D1B4-1]
	Đồ thị hàm số nào sau đây \textbf{không} có tiệm cận ngang?
	\choice
	{$y=x-\sqrt{x^2+1}$}
	{$y=\dfrac{2x-1}{x+1}$}
	{$y=\dfrac{x^2-3x+2}{x^2-x-2}$}
	{\True $y=x^4+4x^2-3$}
	\loigiai
	{
		\begin{itemize}
			\item $\lim\limits_{x\to +\infty}\left(x^4+4x^2-3\right) = \lim\limits_{x\to +\infty}\left[x^4\left(1+\dfrac{4}{x^2}-\dfrac{3}{x^4}\right)\right] = +\infty$.\\
			$\lim\limits_{x\to -\infty}\left(x^4+4x^2-3\right) =\lim\limits_{x\to -\infty}\left[x^4\left(1+\dfrac{4}{x^2}-\dfrac{3}{x^4}\right)\right] = +\infty$.\\
			Vậy đồ thị hàm số $y=x^4+4x^2-3$ không có tiệm cận ngang.
			\item $\lim\limits_{x\to +\infty}\dfrac{2x-1}{x+1} = \lim\limits_{x\to +\infty}\dfrac{2-\dfrac{1}{x}}{1+\dfrac{1}{x}} = 2$ nên đường thẳng $y=2$ là tiệm cận ngang của đồ thị hàm số $y=\dfrac{2x-1}{x+1}$.
			\item $\lim\limits_{x\to +\infty}\dfrac{x^2-3x+2}{x^2-x-2} = \lim\limits_{x\to +\infty}\dfrac{1-\dfrac{3}{x}+\dfrac{2}{x^2}}{1-\dfrac{1}{x}-\dfrac{2}{x^2}}=1$ nên đường thẳng $y=1$ là tiệm cận ngang của đồ thị hàm số $y=\dfrac{x^2-3x+2}{x^2-x-2}$.
			\item $\lim\limits_{x\to +\infty}\left(x-\sqrt{x^2+1}\right) = \lim\limits_{x\to +\infty}\dfrac{-1}{x+\sqrt{x^2+1}} = \lim\limits_{x\to +\infty}\dfrac{-\dfrac{1}{x}}{1+\sqrt{1+\dfrac{1}{x^2}}} = 0$ nên đường thẳng $y=0$ là tiệm cận ngang của đồ thị hàm số $y=x-\sqrt{x^2+1}$.
		\end{itemize}
	}
\end{ex}

\begin{ex}%[Đề KSCL lần 1 THPT Lương Tài - Bắc Ninh, 2020-2021]%[Cao Thành Thái, 12-EX-3-2021]%[2D1K5-3]
	\immini
	{
		Cho hàm số $y=x^3-3x$ có đồ thị như hình vẽ. Phương trình $\left|x^3-3x\right|=m^2+m$ có $6$ nghiệm phân biệt khi và chỉ khi
		\choice
		{\True $-2<m<-1$ hoặc $0<m<1$}
		{$-1<m<0$}
		{$m>0$}
		{$m<-2$ hoặc $m>1$}
	}
	{
		\begin{tikzpicture}[scale=0.8, font=\footnotesize, line join=round, line cap=round, >=stealth]
		\draw[->](-2.5,0)--(2.5,0) node[right] {$x$};
		\draw[->](0,-2.5)--(0,2.5) node[right] {$y$};
		\node (0,0) [below left]{$O$};
		\foreach \x in {-2,...,2}
		\draw[shift={(\x,0)},color=black] (0pt,2pt) -- (0pt,-2pt);
		\foreach \y in {-2,...,2}
		\draw[shift={(0,\y)},color=black] (2pt,0pt) -- (-2pt,0pt);
		\draw[samples=100,smooth,domain=-2.05:2.05] plot (\x,{(\x)^3-3*(\x)});
		\draw[dashed] (-1,0) node[below]{$-1$}--(-1,2)--(0,2) node[right]{$2$} (1,0) node[above]{$1$}--(1,-2)--(0,-2) node[left]{$-2$};
		\fill (0,0) circle(1pt) (-1,2) circle(1pt) (1,-2) circle(1pt);
		\begin{scope}[on background layer]\path[white]node{MDD-108};\end{scope}
\end{tikzpicture}
	}
	\loigiai
	{
		\immini
		{
			Số nghiệm của phương trình $\left|x^3-3x\right|=m^2+m$ bằng số giao điểm của đồ thị hàm số $y=\left|x^3-3x\right|$ và đường thẳng $y=m^2+m$.\\
			Đồ thị của hàm số $y=\left|x^3-3x\right|$ gồm phần đồ thị của hàm số $y=x^3-3x$ nằm phía trên trục hoành và phần đối xứng qua trục hoành với phần đồ thị nằm phía dưới trục hoành của đồ thị hàm số $y=x^3-3x$.
		}
		{
			\begin{tikzpicture}[scale=0.9, font=\footnotesize, line join=round, line cap=round, >=stealth]
			\draw[->](-2.5,0)--(2.5,0) node[right] {$x$};
			\draw[->](0,-1)--(0,3) node[right] {$y$};
			\node (0,0) [below left]{$O$};
			\foreach \x in {-2,...,2}
			\draw[shift={(\x,0)},color=black] (0pt,2pt) -- (0pt,-2pt);
			\foreach \y in {1,...,2}
			\draw[shift={(0,\y)},color=black] (2pt,0pt) -- (-2pt,0pt);
			\draw[samples=300,smooth,domain=-2.05:2.05] plot (\x,{abs((\x)^3-3*(\x))});
			\draw[dashed] (-1,0) node[below]{$-1$}--(-1,2)--(0,2) node[above left]{$2$} (1,0) node[below]{$1$}--(1,2)--(0,2);
			\fill (0,0) circle(1pt) (-1,2) circle(1pt) (1,2) circle(1pt);
			\begin{scope}[on background layer]\path[white]node{MDD-108};\end{scope}
\end{tikzpicture}
		}\vspace{0.3cm}
		\noindent
		Do đó, phương trình $\left|x^3-3x\right|=m^2+m$ có $6$ nghiệm phân biệt khi và chỉ khi
		\[\left\{\begin{aligned}&m^2+m>0 \\&m^2+m<2\end{aligned}\right. \Leftrightarrow \left\{\begin{aligned}&\left[\begin{aligned}&m<-1 \\&m>0\end{aligned}\right. \\&-2<m<1\end{aligned}\right. \Leftrightarrow \left[\begin{aligned}&-2<m<-1 \\&0<m<1.\end{aligned}\right.\]
		Vậy $-2<m<-1$ hoặc $0<m<1$ là các giá trị thỏa mãn yêu cầu bài toán.
	}
\end{ex}

\begin{ex}%[Đề KSCL lần 1 THPT Lương Tài - Bắc Ninh, 2020-2021]%[Cao Thành Thái, 12-EX-3-2021]%[2H1B3-2]
	Cho hình lăng trụ đứng $ABCD.A'B'C'D'$ có đáy là hình thoi, biết $AA'=4a$, $AC=2a$, $BD=a$. Thể tích của khối lăng trụ $ABCD.A'B'C'D'$ bằng
	\choice
	{$8a^3$}
	{$\dfrac{8a^3}{3}$}
	{\True $4a^3$}
	{$2a^3$}
	\loigiai
	{
		\immini
		{
			Diện tích của hình thoi $ABCD$ là $S_{ABCD}=\dfrac{1}{2}AC\cdot BD = \dfrac{1}{2}\cdot 2a\cdot a= a^2$.\\
			Thể tích của khối lăng trụ $ABCD.A'B'C'D'$ là
			\[V=S_{ABCD}\cdot AA' = a^2\cdot 4a = 4a^3.\]
		}
		{
			\begin{tikzpicture}[scale=1, font=\footnotesize]
			\path (0:0) coordinate(A) (0:2.5) coordinate(C) (20:1.6) coordinate(B) ($(A)+(C)-(B)$) coordinate(D) ($(A)!.5!(C)$) coordinate(O);
			\foreach \d in {A,B,C,D} \path (\d) + (90:2.5) coordinate(\d');
			\draw (A')--(B')--(C')--(D')--cycle (A')--(A)--(D)--(C)--(C') (D)--(D');
			\draw[dashed] (A)--(B)--(C) (B)--(B') (A)--(C) (B)--(D);
			\foreach \d/\g in {A/180, B/45, C/0, D/-90, A'/180, B'/90, C'/0, D'/80}
			\fill (\d) circle(1pt) + (\g:.3) node{$\d$};
			\tkzMarkRightAngle[size=0.15](C,O,D)
			\begin{scope}[on background layer]\path[white]node{MDD-108};\end{scope}
\end{tikzpicture}
		}
	}
\end{ex}

\begin{ex}%[Đề KSCL lần 1 THPT Lương Tài - Bắc Ninh, 2020-2021]%[Cao Thành Thái, 12-EX-3-2021]%[1D5Y2-2]
	Cho hàm số $y=f(x)$ có đạo hàm liên tục trên khoảng $\mathscr{K}$ và có đồ thị là đường cong $(C)$. Hệ số góc của tiếp tuyến của $(C)$ tại điểm $M(a;b)\in (C)$ là
	\choice
	{\True $k=f'(a)$}
	{$k=f(a)$}
	{$k=f(b)$}
	{$k=f'(b)$}
	\loigiai
	{
		Đạo hàm của hàm số $y=f(x)$ là $y'=f'(x)$.\\
		Hệ số góc của tiếp tuyến của $(C)$ tại điểm $M(a;b)\in (C)$ là $k=f'(a)$.
	}
\end{ex}

\begin{ex}%[Đề KSCL lần 1 THPT Lương Tài - Bắc Ninh, 2020-2021]%[Cao Thành Thái, 12-EX-3-2021]%[2D1Y1-2]
	Cho hàm số $y=f(x)$ có bảng biến thiên như sau
	\begin{center}
		\begin{tikzpicture}
		\tkzTabInit[nocadre=false,lgt=1.5,espcl=2.5,deltacl=.6]
		{$x$/0.6,$y'$/0.6,$y$/2}
		{$-\infty$,$-1$,$1$,$+\infty$}
		\tkzTabLine{,+,0,-,0,+}
		\tkzTabVar{-/$-\infty$,+/$3$,-/$-1$,+/$+\infty$}
		\begin{scope}[on background layer]\path[white]node{MDD-108};\end{scope}
\end{tikzpicture}
	\end{center}
	Mệnh đề nào dưới đây đúng?
	\choice
	{Hàm số đồng biến trên khoảng $(-\infty;1)$}
	{Hàm số nghịch biến trên khoảng $(-1;3)$}
	{Hàm số đồng biến trên khoảng $(-1;+\infty)$}
	{\True Hàm số nghịch biến trên khoảng $(-1;1)$}
	\loigiai
	{
		Dựa vào bảng biến thiên của hàm số $y=f(x)$ thì
		\begin{itemize}
			\item Hàm số đồng biến trên mỗi khoảng $(-\infty;-1)$, $(1;+\infty)$.
			\item Hàm số nghịch biến trên khoảng $(-1;1)$.
		\end{itemize}
	}
\end{ex}

\begin{ex}%[Đề KSCL lần 1 THPT Lương Tài - Bắc Ninh, 2020-2021]%[Cao Thành Thái, 12-EX-3-2021]%[2D1Y2-2]
	Cho hàm số $y=f(x)$ có bảng biến thiên như sau
	\begin{center}
		\begin{tikzpicture}
		\tkzTabInit[nocadre=false,lgt=1.5,espcl=2.5,deltacl=.6]
		{$x$/0.6,$y'$/0.6,$y$/2}
		{$-\infty$,$0$,$2$,$+\infty$}
		\tkzTabLine{,+,0,-,0,+}
		\tkzTabVar{-/$-\infty$,+/$5$,-/$1$,+/$+\infty$}
		\begin{scope}[on background layer]\path[white]node{MDD-108};\end{scope}
\end{tikzpicture}
	\end{center}
	Mệnh đề nào dưới đây đúng?
	\choice
	{Hàm số không có cực trị}
	{\True Hàm số đạt cực đại tại $x=0$}
	{Hàm số đạt cực đại tại $x=5$}
	{Hàm số đạt cực tiểu tại $x=1$}
	\loigiai
	{
		Dựa vào bảng biến thiên của hàm số $y=f(x)$ thì
		\begin{itemize}
			\item Hàm số đạt cực đại tại $x=0$ và giá trị cực đại của hàm số bằng $5$.
			\item Hàm số đạt cực tiểu tại $x=2$ và giá trị cực tiểu của hàm số bằng $1$.
		\end{itemize}
	}
\end{ex}

\begin{ex}%[Đề KSCL lần 1 THPT Lương Tài - Bắc Ninh, 2020-2021]%[Cao Thành Thái, 12-EX-3-2021]%[2D1B2-3]
	Hàm số $y=-x^4+2mx^2+1$ đạt cực tiểu tại $x=0$ khi
	\choice
	{\True $m>0$}
	{$-1\leq m<0$}
	{$m\geq 0$}
	{$m<-1$}
	\loigiai
	{
		Ta có $y'=-4x^3+4mx = 4x\left(-x^2+m\right)$ và $y''=-12x+4m$.\\
		Ta thấy $y'(0)=0$ nên hàm số đã cho đạt cực trị tại $x=0$.\\
		Hàm số đã cho đạt cực tiểu tại $x=0$ khi
		\[y''(0)>0 \Leftrightarrow 4m>0 \Leftrightarrow m>0.\]
		Vậy với $m>0$ thì hàm số $y=-x^4+2mx^2+1$ đạt cực tiểu tại $x=0$.
	}
\end{ex}

\begin{ex}%[Đề KSCL lần 1 THPT Lương Tài - Bắc Ninh, 2020-2021]%[Cao Thành Thái, 12-EX-3-2021]%[0D3B1-1]
	Tập xác định của phương trình $\sqrt{x-1}+\sqrt{x-2}=\sqrt{x-3}$ là
	\choice
	{$[1;+\infty)$}
	{$\mathbb{R}\setminus\{1;2;3\}$}
	{\True $[3;+\infty)$}
	{$(3;+\infty)$}
	\loigiai
	{
		Phương trình $\sqrt{x-1}+\sqrt{x-2}=\sqrt{x-3}$ xác định khi và chỉ khi
		\[\left\{\begin{aligned}&x-1\geq 0 \\&x-2\geq 0 \\&x-3\geq 0\end{aligned}\right. \Leftrightarrow \left\{\begin{aligned}&x\geq 1 \\&x\geq 2 \\&x\geq 3\end{aligned}\right. \Leftrightarrow x\geq 3.\]
		Vậy phương trình đã cho xác định với mọi $x\in [3;+\infty)$.
	}
\end{ex}

\begin{ex}%[Đề KSCL lần 1 THPT Lương Tài - Bắc Ninh, 2020-2021]%[Cao Thành Thái, 12-EX-3-2021]%[2D2B3-1]
	Cho $a$, $b$ là các số thực dương khác $1$ thỏa mãn $\log_a b=\sqrt{3}$. Giá trị của $\log_{\frac{\sqrt{b}}{a}}\left(\dfrac{\sqrt[3]{b}}{\sqrt{a}}\right)$ bằng
	\choice
	{$\sqrt{3}$}
	{\True $-\dfrac{1}{\sqrt{3}}$}
	{$-2\sqrt{3}$}
	{$-\sqrt{3}$}
	\loigiai
	{
		Với mọi $a$, $b$ là các số thực dương khác $1$ thỏa mãn $\log_a b=\sqrt{3}$ ta có
		\allowdisplaybreaks
		\begin{eqnarray*}
			\log_{\frac{\sqrt{b}}{a}}\left(\dfrac{\sqrt[3]{b}}{\sqrt{a}}\right) &=& \dfrac{\log_a \left(\dfrac{\sqrt[3]{b}}{\sqrt{a}}\right)}{\log_a\left(\dfrac{\sqrt{b}}{a}\right)} = \dfrac{\log_a \sqrt[3]{b} - \log_a \sqrt{a}}{\log_a \sqrt{b} - \log_a a} = \dfrac{\dfrac{1}{3}\log_a b - \dfrac{1}{2}}{\dfrac{1}{2}\log_a b-1} = \dfrac{\dfrac{1}{3}\cdot \sqrt{3}-\dfrac{1}{2}}{\dfrac{1}{2}\cdot \sqrt{3}-1}\\
			&=& \dfrac{2\sqrt{3}-3}{6}\cdot \dfrac{2}{\sqrt{3}-2} = \dfrac{-\sqrt{3}\left(\sqrt{3}-2\right)}{6}\cdot \dfrac{2}{\sqrt{3}-2} = -\dfrac{1}{\sqrt{3}}.
		\end{eqnarray*}
	}
\end{ex}

\begin{ex}%[Đề KSCL lần 1 THPT Lương Tài - Bắc Ninh, 2020-2021]%[Cao Thành Thái, 12-EX-3-2021]%[2D2B2-1]
	Tập xác định của hàm số $y=\left(x^2-3x+2\right)^{\pi}$ là
	\choice
	{\True $\mathscr{D}=(-\infty;1)\cup (2;+\infty)$}
	{$\mathscr{D}=(1;2)$}
	{$\mathscr{D}=(-\infty;1]\cup [2;+\infty)$}
	{$\mathscr{D}=\mathbb{R}\setminus \{1;2\}$}
	\loigiai
	{
		Vì $\pi \notin \mathbb{Z}$ nên hàm số $y=\left(x^2-3x+2\right)^{\pi}$ xác định khi và chỉ khi
		\[x^2-3x+2>0 \Leftrightarrow \left[\begin{aligned}&x<1 \\&x>2.\end{aligned}\right.\]
		Vậy tập xác định của hàm số $y=\left(x^2-3x+2\right)^{\pi}$ là $\mathscr{D}=(-\infty;1)\cup (2;+\infty)$.
	}
\end{ex}

\begin{ex}%[Đề KSCL lần 1 THPT Lương Tài - Bắc Ninh, 2020-2021]%[Cao Thành Thái, 12-EX-3-2021]%[1D5B2-2]
	Cho hàm số $y=x^4+2x^2+1$ có đồ thị $(C)$. Phương trình tiếp tuyến của đồ thị $(C)$ tại điểm $M(1;4)$ là
	\choice
	{\True $y=8x-4$}
	{$y=8x+4$}
	{$y=-8x+12$}
	{$y=x+3$}
	\loigiai
	{
		Ta có $y'=4x^3+4x$.\\
		Hệ số góc của tiếp tuyến của đồ thị $(C)$ tại điểm $M(1;4)$ là $y'(1)=8$.\\
		Phương trình tiếp tuyến cần tìm là $y=8(x-1)+4$ hay $y=8x-4$.
	}
\end{ex}

\begin{ex}%[Đề KSCL lần 1 THPT Lương Tài - Bắc Ninh, 2020-2021]%[Cao Thành Thái, 12-EX-3-2021]%[2D1Y2-2]
	\immini
	{
		Cho hàm số $y=f(x)$ có đồ thị như hình vẽ. Khẳng định nào sau đây đúng?
		\choice
		{Đồ thị hàm số có điểm cực tiểu là $(-1;3)$}
		{Đồ thị hàm số có điểm cực tiểu là $(1;1)$}
		{\True Đồ thị hàm số có điểm cực tiểu là $(1;-1)$}
		{Đồ thị hàm số có điểm cực đại là $(1;-1)$}
	}
	{
		\begin{tikzpicture}[scale=0.8, font=\footnotesize, line join=round, line cap=round, >=stealth]
		\draw[->](-2.5,0)--(2.5,0) node[right] {$x$};
		\draw[->](0,-1.5)--(0,3.5) node[right] {$y$};
		\node (0,0) [below left]{$O$};
		\foreach \x in {-2,...,2}
		\draw[shift={(\x,0)},color=black] (0pt,2pt) -- (0pt,-2pt);
		\foreach \y in {-1,...,3}
		\draw[shift={(0,\y)},color=black] (2pt,0pt) -- (-2pt,0pt);
		\draw[samples=100,smooth,domain=-2.05:2.05] plot (\x,{(\x)^3-3*(\x)+1});
		\draw[dashed] (-1,0) node[below]{$-1$}--(-1,3)--(0,3) node[right]{$3$} (1,0) node[above]{$1$}--(1,-1)--(0,-1) node[left]{$-1$};
		\fill (0,0) circle(1pt) (-1,3) circle(1pt) (1,-1) circle(1pt);
		\begin{scope}[on background layer]\path[white]node{MDD-108};\end{scope}
\end{tikzpicture}
	}
	\loigiai
	{
		Đồ thị hàm số $y=f(x)$ có điểm cực tiểu là $(1;-1)$ và điểm cực đại là $(-1;3)$.
	}
\end{ex}

\begin{ex}%[Đề KSCL lần 1 THPT Lương Tài - Bắc Ninh, 2020-2021]%[Cao Thành Thái, 12-EX-3-2021]%[0D3B2-4]
	Tập nghiệm của phương trình $\sqrt{2x-3}=x-3$ là
	\choice
	{$S=\varnothing$}
	{\True $S=\{6\}$}
	{$S=\{2;6\}$}
	{$S=\{2\}$}
	\loigiai
	{
		Ta có
		\[\sqrt{2x-3}=x-3 \Leftrightarrow \left\{\begin{aligned}&x-3\geq 0 \\&2x-3=x^2-6x+9\end{aligned}\right. \Leftrightarrow \left\{\begin{aligned}&x\geq 3 \\&x^2-8x+12=0\end{aligned}\right. \Leftrightarrow \left\{\begin{aligned}&x\geq 3 \\&\left[\begin{aligned}&x=2 \\&x=6\end{aligned}\right.\end{aligned}\right. \Leftrightarrow x=6.\]
		Vậy tập nghiệm của phương trình đã cho là $S=\{6\}$.
	}
\end{ex}

\begin{ex}%[Đề KSCL lần 1 THPT Lương Tài - Bắc Ninh, 2020-2021]%[Cao Thành Thái, 12-EX-3-2021]%[2D2B5-2]
	Phương trình $\left(\dfrac{1}{3}\right)^{x^2-2x-3}=3^{x+1}$ có bao nhiêu nghiệm?
	\choice
	{$3$}
	{\True $2$}
	{$1$}
	{$0$}
	\loigiai
	{
		Ta có
		\[\left(\dfrac{1}{3}\right)^{x^2-2x-3}=3^{x+1} \Leftrightarrow 3^{-x^2+2x+3} = 3^{x+1} \Leftrightarrow -x^2+2x+3=x+1 \Leftrightarrow x^2-x-2=0 \Leftrightarrow \left[\begin{aligned}&x=-1 \\&x=2.\end{aligned}\right.\]
		Vậy phương trình đã cho có hai nghiệm phân biệt $x=-1$, $x=2$.
	}
\end{ex}

\begin{ex}%[Đề KSCL lần 1 THPT Lương Tài - Bắc Ninh, 2020-2021]%[Cao Thành Thái, 12-EX-3-2021]%[1D2K3-2]
	Cho $n\in\mathbb{N}$ thỏa mãn $\mathrm{C}_n^1+\mathrm{C}_n^2+\cdots + \mathrm{C}_n^n=1023$. Hệ số của $x^2$ trong khai triển $\left[(12-n)x+1\right]^n$ thành đa thức là
	\choice
	{$45$}
	{\True $180$}
	{$2$}
	{$90$}
	\loigiai
	{
		Ta có
		\[\mathrm{C}_n^1+\mathrm{C}_n^2+\cdots + \mathrm{C}_n^n=1023 \Leftrightarrow \mathrm{C}_n^0+\mathrm{C}_n^1+\mathrm{C}_n^2+\cdots + \mathrm{C}_n^n=1024 \Leftrightarrow 2^n=2^{10} \Leftrightarrow n=10.\]
		Với $n=10$ thì $\left[(12-n)x+1\right]^n$ trở thành $(2x+1)^{10}$.\\
		Khi đó
		\[(2x+1)^{10} = \sum\limits_{k=0}^{10}\mathrm{C}_{10}^k (2x)^{10-k}\cdot 1^k = \sum\limits_{k=0}^{10}\mathrm{C}_{10}^k 2^{10-k}x^{10-k}.\]
		Số hạng chứa $x^2$ ứng với $10-k=2$ hay $k=8$.\\
		Vậy hệ số của $x^2$ trong khai triển $(2x+1)^{10}$ thành đa thức là $\mathrm{C}_{10}^8\cdot 2^2 = 180$.
	}
\end{ex}

\begin{ex}%[Đề KSCL lần 1 THPT Lương Tài - Bắc Ninh, 2020-2021]%[Cao Thành Thái, 12-EX-3-2021]%[2H1K3-3]
	Cho hình chóp $S.ABCD$ có đáy là hình bình hành và có thể tích là $V$. Gọi $M$ là trung điểm của $SB$. $P$ là điểm thuộc cạnh $SD$ sao cho $SP=2DP$. Mặt phẳng $(AMP)$ cắt cạnh $SC$ tại $N$. Thể tích của khối đa diện $ABCDMNP$ tính theo $V$ là
	\choice
	{$V_{ABCDMNP}=\dfrac{7}{30}V$}
	{$V_{ABCDMNP}=\dfrac{19}{30}V$}
	{$V_{ABCDMNP}=\dfrac{2}{5}V$}
	{\True $V_{ABCDMNP}=\dfrac{23}{30}V$}
	\loigiai
	{
		\immini
		{
			Gọi $O=AC\cap BD$, khi đó $SO=(SBD)\cap (SAC)$.\\
			Trong $(SBD)$, gọi $I=SO\cap MP$.\\
			Khi đó $AI=(AMP)\cap (SAC)$.\\
			Trong $(SAC)$, gọi $N=AI\cap SC$.\\
			Khi đó $N=SC\cap (AMP)$.\\
			Vì bốn điểm $A$, $M$, $N$, $P$ đồng phẳng nên
			\[\dfrac{SA}{SA}+\dfrac{SC}{SN}=\dfrac{SB}{SM}+\dfrac{SD}{SP}.\]
			Suy ra $\dfrac{SC}{SN}=2+\dfrac{3}{2}-1 = \dfrac{5}{2}$ hay $\dfrac{SN}{SC}=\dfrac{2}{5}$.
		}
		{
			\begin{tikzpicture}[scale=1, font=\footnotesize]
			\path (0:0) coordinate(A) (0:4) coordinate(B) (-135:2) coordinate(D) ($(B)+(D)-(A)$) coordinate(C) (95:4.5) coordinate(S) ($(S)!.5!(B)$) coordinate(M) ($(S)!{2/3}!(D)$) coordinate(P) ($(B)!.5!(D)$) coordinate(O) (intersection of M--P and S--O) coordinate (I) (intersection of A--I and S--C) coordinate(N);
			\draw (S)--(B)--(C)--(D)--cycle (S)--(C) (M)--(N)--(P);
			\draw[dashed] (S)--(A) (B)--(A)--(D)--cycle (A)--(C) (M)--(A)--(P)--cycle (S)--(O) (A)--(N);
			\foreach \d/\g in {A/-80, B/0, C/-45, D/-135, S/90, M/45, N/60, P/180, O/-90, I/-30}
			\fill (\d) circle(1pt) + (\g:.3) node{$\d$};
			\begin{scope}[on background layer]\path[white]node{MDD-108};\end{scope}
\end{tikzpicture}
		}
		\noindent
		Khi đó, $\dfrac{V_{SAMN}}{V_{S.ABC}} = \dfrac{SM}{SB}\cdot \dfrac{SN}{SC}=\dfrac{1}{2}\cdot \dfrac{2}{5} = \dfrac{1}{5}$. Suy ra $V_{S.AMN}=\dfrac{1}{5}V_{S.ABC} = \dfrac{1}{10}V_{S.ABCD} = \dfrac{1}{10}V$.\\
		Tương tự, $\dfrac{V_{SANP}}{V_{S.ACD}} = \dfrac{SN}{SC}\cdot \dfrac{SP}{SD}=\dfrac{2}{5}\cdot \dfrac{2}{3} = \dfrac{4}{15}$. Suy ra $V_{S.ANP}=\dfrac{4}{15}V_{S.ACD} = \dfrac{2}{15}V_{S.ABCD} = \dfrac{2}{15}V$.\\
		Vậy $V_{S.AMNP} = V_{S.AMN}+V_{S.ANP} = \dfrac{1}{10}V+\dfrac{2}{15}V= \dfrac{7}{30}V$.\\
		Do đó $V_{ABCDMNP}=\dfrac{23}{30}V$.
	}
\end{ex}

\begin{ex}%[Đề KSCL lần 1 THPT Lương Tài - Bắc Ninh, 2020-2021]%[Cao Thành Thái, 12-EX-3-2021]%[2D1K2-4]
	Biết rằng đồ thị hàm số $f(x)=\dfrac{1}{3}x^3-\dfrac{1}{2}mx^2+x-2$ có giá trị tuyệt đối của hoành độ hai điểm cực trị là độ dài hai cạnh của tam giác vuông có cạnh huyền là $\sqrt{7}$. Hỏi có mấy giá trị của $m$?
	\choice
	{$0$}
	{\True $2$}
	{$3$}
	{$1$}
	\loigiai
	{
		Ta có $f'(x)=x^2-mx+1$.\\
		Hàm số đã cho có hai điểm cực trị khi phương trình $f'(x)=0$ có hai nghiệm phân biệt, tức là
		\[(-m)^2-4\cdot 1\cdot 1>0 \Leftrightarrow m^2-4>0 \Leftrightarrow \left[\begin{aligned}&m<-2 \\&m>2.\end{aligned}\right.\tag{$*$}\]
		Với điều kiện $(*)$ thì phương trình $f'(x)=0$ có hai nghiệm phân biệt $x_1$, $x_2$ thỏa mãn $x_1+x_2=m$ và $x_1x_2=1$.\\
		Theo giả thiết, ta có
		\[|x_1|^2+|x_2|^2=7 \Leftrightarrow x_1^2+x_2^2=7 \Leftrightarrow (x_1+x_2)^2-2x_1x_2=7 \Leftrightarrow m^2-9=0 \Leftrightarrow m=\pm 3.\]
		Kết hợp điều kiện $(*)$, ta được $m=\pm 3$ là các giá trị cần tìm.
	}
\end{ex}

\begin{ex}%[Đề KSCL lần 1 THPT Lương Tài - Bắc Ninh, 2020-2021]%[Cao Thành Thái, 12-EX-3-2021]%[2D1K3-6]
	Người ta cần xây một bể chứa nước sản xuất dạng khối hộp chữ nhật không nắp có thể tích bằng $200\mathrm{\,m}^3$. Đáy bể là hình chữ nhật có chiều dài gấp đôi chiều rộng. Chi phí để xây bể là $300$ nghìn đồng/$\mathrm{m}^2$ ({\it chi phí được tính theo diện tích xây dựng, bao gồm diện tích đáy và diện tích xung quanh, không tính chiều dày của đáy và thành bể}). Chi phí thấp nhất để xây bể (làm tròn đến đơn vị triệu đồng) là
	\choice
	{$46$ triệu đồng}
	{\True $51$ triệu đồng}
	{$75$ triệu đồng}
	{$36$ triệu đồng}
	\loigiai
	{
		\immini
		{
			Gọi $x$, $2x$ ($x>0$) lần lượt là chiều rộng, chiều dài của bể, $h$ ($h>0$) là chiều cao của bể.\\
			Thể tích của bể là
			\[V=x\cdot 2x\cdot h \Rightarrow h=\dfrac{200}{2x^2} = \dfrac{100}{x^2}.\]
			Diện tích xây dựng bể là
		}
		{
			\begin{tikzpicture}[scale=1, font=\footnotesize]
			\path (0:0) coordinate(A) (0:4) coordinate(B) (-130:2) coordinate(D) ($(B)+(D)-(A)$) coordinate(C);
			\foreach \d in {A,B,C,D} \path (\d)+(90:2.5) coordinate(\d');
			\draw (A')--(B')--(C')--(D')--cycle (B')--(B) node[midway, right]{$h$} (B)--(C)--(D)--(D') (C)--(C');
			\draw[dashed] (A)--(D) node[midway, left]{$x$} (A)--(B) node[midway, above]{$2x$} (A)--(A');
			\begin{scope}[on background layer]\path[white]node{MDD-108};\end{scope}
\end{tikzpicture}
		}
		\[2(x+2x)h + x\cdot 2x = 6x\cdot \dfrac{100}{x^2} + 2x^2 = \dfrac{600}{x}+2x^2.\]
		Với mọi $x>0$ ta có
		\[\dfrac{600}{x}+2x^2 = \dfrac{300}{x}+\dfrac{300}{x}+2x^2 \geq 3\sqrt[3]{\dfrac{300}{x}\cdot \dfrac{300}{x}\cdot 2x^2} = 30\sqrt[3]{180}.\]
		Đẳng thức xảy ra khi $\dfrac{300}{x}=2x^2$ hay $x=\sqrt[3]{150}$.\\
		Vậy diện tích xây dựng bể ít nhất bằng $30\sqrt[3]{180}\mathrm{\,m}^2$.\\
		Do đó chi phí thấp nhất để xây bể là $30\sqrt[3]{180}\cdot 300\,000 \approx 50\,815\,946$ đồng.
	}
\end{ex}

\begin{ex}%[Đề KSCL lần 1 THPT Lương Tài - Bắc Ninh, 2020-2021]%[Cao Thành Thái, 12-EX-3-2021]%[0H3K1-2]
	Cho tam giác $ABC$ có $AB\colon 2x-y+4=0$, $AC\colon x-2y-6=0$. Hai điểm $B$ và $C$ thuộc $Ox$. Phương trình phân giác góc ngoài của $\widehat{BAC}$ là
	\choice
	{$3x+3y+10=0$}
	{\True $x+y+10=0$}
	{$3x-3y-2=0$}
	{$x-y+10=0$}
	\loigiai
	{
		Vì $B=AB\cap Ox$ nên $B(-2;0)$.\\
		Vì $C=AC\cap Ox$ nên $C(6;0)$.\\
		Xét phương trình
		\[\dfrac{|2x-y+4|}{\sqrt{2^2+(-1)^2}} = \dfrac{|x-2y-6|}{\sqrt{1^2+(-2)^2}} \Leftrightarrow \left[\begin{aligned}&2x-y+4=x-2y-6 \\&2x-y+4=-x+2y+6\end{aligned}\right. \Leftrightarrow \left[\begin{aligned}&x+y+10=0 \\&3x-3y-2=0.\end{aligned}\right.\]
		Một trong hai phương trình trên là phương trình của đường phân giác ngoài của $\widehat{BAC}$.\\
		Xét sự cùng phía của $B$ và $C$ đối với đường thẳng $d\colon x+y+10=0$.\\
		Ta có $\left(-2+0+10\right)\left(-6+0+10\right)>0$ nên $B$, $C$ nằm cùng phía đối với đường thẳng $d$.\\
		Vậy phương trình phân giác góc ngoài của $\widehat{BAC}$ là $x+y+10=0$.
	}
\end{ex}

\begin{ex}%[Đề KSCL lần 1 THPT Lương Tài - Bắc Ninh, 2020-2021]%[Đoàn Minh Tân, 12-EX-3]%[2D1G5-5]
	\immini{Cho hàm số $y=f(x)$ có đồ thị $f'(x)$ như hình vẽ.\\
		Hàm số $y=f(1-x)+\dfrac{x^2}{2}-x$ nghịch biến trên khoảng
		\choice
		{$(1;3)$}
		{$(-3;1)$}
		{\True $(-2;0)$}
		{$\left(-1;\dfrac{3}{2}\right)$}
	}
	{
		\begin{tikzpicture}[>=stealth,line join=round,line cap=round,font=\footnotesize,scale=0.8]
		\draw[->] (-4,0)--(4,0) node[below]{$x$};
		\draw[->] (0,-6)--(0,4) node[left]{$y$};
		\fill[name=O] (0,0) circle (1.5pt) node[below left] {$O$}; 
		\draw[color=black,thick, smooth, samples=100, domain= -3.1:-1] plot(\x,{(\x)^2-6});
		\draw[color=black,thick, smooth, samples=100, domain= -1:0.4] plot(\x,{2.88*(\x)^3-0.83*(\x)^2+-0.88*(\x)-2.17});
		\draw[color=black,thick, smooth, samples=100, domain= 0.4:3.2] plot(\x,{-4/3*(\x)^2+13/3*(\x)-4});
		\foreach \y/\q in {3/right,-1/left,-3/left,-5/right}
		\fill (0,\y) circle (1.5pt) node[\q]{$\y$};
		\foreach \x/\p in {3/above,1/above right,-3/below,-1/below}
		\fill (\x,0) circle (1.5pt) node[\p]{$\x$};
		\draw[dashed] (-3,0)|-(0,3) (3,0)|-(0,-3) (1,0)|-(0,-1) (1.5,0)|-(0,-0.5) (-1,0)|-(0,-5);
		\fill (-3,3) circle (1.5pt) (3,-3) circle (1.5pt) (1,-1) circle(1.5pt);
		\begin{scope}[on background layer]\path[white]node{MDD-108};\end{scope}
\end{tikzpicture}
	}
	\loigiai{
		\immini{Xét $y=f\left(1-x\right)+\dfrac{x^2}{2}-x$ có $y'=-f'(1-x)+x-1$.\\
			Đặt $t=1-x\Rightarrow x-1=-t$.\\
			Khi đó $y'=-f'(t)-t$.\\
			Xét $y'=0\Leftrightarrow f'(t)=-t$.\\
			Dựa vào đồ thị, đường thẳng $y=-t$ cắt đồ thị hàm số $y=f'(t)$ tại $3$ điểm phân biệt có hoành độ lần lượt là $t=-3$, $t=1$, $t=3$.\\
			Ta có bảng xét dấu của hàm số $y=f'(t)$ như sau
			\begin{center}
				\begin{tikzpicture}
				\tkzTabInit[nocadre=false,lgt=1.2,espcl=2,deltacl=.6]
				{$t$/0.6,$y'$/0.6}
				{$-\infty$,$-3$,$1$,$3$,$+\infty$}
				\tkzTabLine{,-,0,+,0,+,0,-}
				\begin{scope}[on background layer]\path[white]node{MDD-108};\end{scope}
\end{tikzpicture}
			\end{center}
		}
		{\begin{tikzpicture}[>=stealth,line join=round,line cap=round,font=\footnotesize,scale=0.8]
			\draw[->] (-4,0)--(4,0) node[below]{$x$};
			\draw[->] (0,-6)--(0,4) node[left]{$y$};
			\fill[name=O] (0,0) circle (1.5pt) node[below left] {$O$}; 
			\draw[color=black,thick, smooth, samples=100, domain= -3.1:-1] plot(\x,{(\x)^2-6});
			\draw[color=black,thick, smooth, samples=100, domain= -1:0.4] plot(\x,{2.88*(\x)^3-0.83*(\x)^2+-0.88*(\x)-2.17});
			\draw[color=black,thick, smooth, samples=100, domain= 0.4:3.2] plot(\x,{-4/3*(\x)^2+13/3*(\x)-4});
			\draw[color=black,thick, smooth, samples=100, domain= -3.5:4] plot(\x,{-(\x)});
			\foreach \y/\q in {3/right,-1/left,-3/left,-5/right}
			\fill (0,\y) circle (1.5pt) node[\q]{$\y$};
			\foreach \x/\p in {3/above,1/above right,-3/below,-1/below}
			\fill (\x,0) circle (1.5pt) node[\p]{$\x$};
			\draw[dashed] (-3,0)|-(0,3) (3,0)|-(0,-3) (1,0)|-(0,-1) (1.5,0)|-(0,-0.5) (-1,0)|-(0,-5);
			\fill (-3,3) circle (1.5pt) (3,-3) circle (1.5pt) (1,-1) circle(1.5pt);
			\begin{scope}[on background layer]\path[white]node{MDD-108};\end{scope}
\end{tikzpicture}
		}\noindent
		Từ bảng xét dấu, $y'<0\Leftrightarrow \hoac{& t<-3\\ & 1<t<3}\Leftrightarrow \hoac{& 1-x<-3\\ & 1<1-x<3}\Leftrightarrow \hoac{& x>4\\ &-2<x<0.}$\\
		Vậy hàm số $y=f\left(1-x\right)+\dfrac{x^2}{2}-x$ nghịch biến trên các khoảng $(-2;0)$ và $(4;+\infty)$.
	}
\end{ex}
\begin{ex}%[Đề KSCL lần 1 THPT Lương Tài - Bắc Ninh, 2020-2021]%[Đoàn Minh Tân, 12-EX-3]%[2D1K2-1]
	Cho hàm số $y=f(x)$ có đạo hàm $f'(x)=x^2(x-9)(x-4)^2$. Khi đó hàm số $y=f\left(x^2\right)$ nghịch biến trên khoảng nào?
	\choice
	{$(-3;0)$}
	{$(3;+\infty)$}
	{\True $(-\infty;-3)$}
	{$(-2;2)$}
	\loigiai{
		Xét $y=f(x^2)$ có $y'=2xf'(x^2)=2x\cdot x^4(x^2-9)\left(x^2-4\right)^2$.\\
		$y'=0\Leftrightarrow \hoac{& x=0\\ &x=\pm 3\\ & x=\pm 2.}$\\
		Ta có bảng biến thiên của $y=f(x^2)$ như sau
		\begin{center}
			\begin{tikzpicture}
			\tkzTabInit[nocadre=false,lgt=1.2,espcl=2,deltacl=.6]
			{$x$/0.6,$y'$/0.6,$y$/2}
			{$-\infty$,$-3$,$-2$,$0$,$2$,$3$,$+\infty$}
			\tkzTabLine{,-,0,+,0,+,0,-,0,-,0,+}
			\tkzTabVar{+/, -/,R,+/,R, -/,+/}
			\begin{scope}[on background layer]\path[white]node{MDD-108};\end{scope}
\end{tikzpicture}
		\end{center}
		Dựa vào bảng biến thiên, hàm số $y=f(x^2)$ nghịch biến trên các khoảng $(-\infty;-3)$ và $(0;3)$.	
	}
\end{ex}
\begin{ex}%[Đề KSCL lần 1 THPT Lương Tài - Bắc Ninh, 2020-2021]%[Đoàn Minh Tân, 12-EX-3]%[2D1B1-3]
	Tìm tất cả giá trị thực của tham số $m$ để hàm số $y=x^3+x^2+mx+1$ đồng biến trên $(-\infty;+\infty)$.
	\choice
	{$m\ge \dfrac{4}{3}$}
	{$m\le \dfrac{4}{3}$}
	{$m\le \dfrac{1}{3}$}
	{\True $m\ge \dfrac{1}{3}$}
	\loigiai{
		Để hàm số $y=x^3+x^2+mx+1$ đồng biến trên $\mathbb{R}$ thì
		$$y'=3x^2+2x+m\le 0,\, \forall x\in \mathbb{R}\\
		\Leftrightarrow \Delta'=1-3m\le 0 \Leftrightarrow m\ge \dfrac{1}{3}.$$	
	}
\end{ex}
\begin{ex}%[Đề KSCL lần 1 THPT Lương Tài - Bắc Ninh, 2020-2021]%[Đoàn Minh Tân, 12-EX-3]%[2D1G2-6]
	Có bao nhiêu giá trị nguyên dương của tham số $m$ để hàm số $y=\left|3x^4-4x^3-12x^2+m\right|$ có $5$ điểm cực trị?
	\choice
	{$26$}
	{$16$}
	{\True $27$}
	{$44$}
	\loigiai{
		Xét hàm số $f(x)=3x^4-4x^3-12x^2+m$.\\
		$f'(x)=12x^3-12x^2-24x$; $f'(x)=0\Leftrightarrow \hoac{& x=0\\ & x=-1\\ & x=2.}$\\
		Bảng biến thiên của hàm số $y=f(x)$
		\begin{center}
			\begin{tikzpicture}
			\tkzTabInit[nocadre=false,lgt=1.2,espcl=2,deltacl=.6]
			{$x$/0.6,$y'$/0.6,$y$/2}
			{$-\infty$,$-1$,$0$,$2$,$+\infty$}
			\tkzTabLine{,-,0,+,0,-,0,+}
			\tkzTabVar{+/$+\infty$, -/$-5+m$,+/$m$,-/$-32+m$,+/$+\infty$}
			\begin{scope}[on background layer]\path[white]node{MDD-108};\end{scope}
\end{tikzpicture}
		\end{center}
		Hàm số $y=f(x)$ có $3$ điểm cực trị.\\
		Để hàm số $y=\left|f(x)\right|$ có $5$ điểm cực trị thì phương trình $f(x)=0$ phải có hai nghiệm bội lẻ.\\
		Từ bảng biến thiên, ta suy ra $\hoac{& m\le 0\\ & -32+m<0\le-5+m}\Leftrightarrow \hoac{ &m\le 0\\ &5\le m<32.}$\\
		Mà $m$ là số nguyên dương nên $m\in \left\lbrace 5;6;7;\ldots;31\right\rbrace $.\\
		Vậy có $27$ giá trị $m$ thỏa mãn.
	}
\end{ex}
\begin{ex}%[Đề KSCL lần 1 THPT Lương Tài - Bắc Ninh, 2020-2021]%[Đoàn Minh Tân, 12-EX-3]%[2H1Y3-3]
	Cho hình chóp tam giác $S.ABC$ có $SA$, $SB$, $SC$ đôi một vuông góc với nhau và $SA=SB=SC=a$. Tính thể tích của khối chóp $S.ABC$.
	\choice
	{$\dfrac{1}{2}a^3$}
	{$\dfrac{2}{3}a^3$}
	{\True $\dfrac{1}{6}a^3$}
	{$\dfrac{1}{3}a^3$}
	\loigiai{
		Thể tích khối chóp $S.ABC$ là $V_{S.ABC}=\dfrac{1}{6}SA\cdot SB\cdot SC	=\dfrac{1}{6}a^3$.
	}
\end{ex}
\begin{ex}%[Đề KSCL lần 1 THPT Lương Tài - Bắc Ninh, 2020-2021]%[Đoàn Minh Tân, 12-EX-3]%[1H3B5-3]
	Cho hình chóp $S.ABC$ có $SA$, $AB$, $BC$ vuông góc với nhau từng đôi một. Biết $SA=a\sqrt{3}$, $AB=a\sqrt{3}$. Khoảng cách từ $A$ đến $(SBC)$ bằng
	\choice
	{$\dfrac{2a\sqrt{5}}{5}$}
	{\True $\dfrac{a\sqrt{6}}{2}$}
	{$\dfrac{a\sqrt{3}}{2}$}
	{$\dfrac{a\sqrt{2}}{3}$}
	\loigiai{
		\immini{Ta có $\heva{&SA\perp AB\\& SA\perp BC}\Rightarrow SA\perp (ABC)$.\\
			Trong $(SAB)$ kẻ $AH\perp SB$ tại $H$.\\
			Khi đó ta chứng minh được $AH\perp (SBC)$.\\
			Do đó $\mathrm{d}(A,(SBC))=AH$.\\
			Xét $\triangle SAB$ vuông cân tại $A$ có $AH$ là đường cao.\\
			Suy ra $AH$ là đường trung tuyến.\\
			Nên $AH=\dfrac{1}{2}SB=\dfrac{1}{2}\cdot a\sqrt{3}\cdot \sqrt{2}=\dfrac{a\sqrt{6}}{2}$.\\
			Vậy $\mathrm{d}(A,(SBC))=\dfrac{a\sqrt{6}}{2}$.}
		{
			\begin{tikzpicture}[scale=1, line join = round, line cap = round]
			\tikzset{label style/.style={font=\footnotesize}}
			\tkzDefPoints{0/0/A, 2/-2/B, 5/0/C, 0/3/S}
			\tkzDefMidPoint(S,B)\tkzGetPoint{H}
			\tkzDrawSegments(S,A S,B S,C A,H A,B B,C)
			\tkzDrawSegments[dashed](A,C)
			\tkzDrawPoints[fill=black](A,B,C,S,H)
			\tkzMarkRightAngles(S,A,B S,H,A A,B,C)
			\tkzLabelPoints[above](S)
			\tkzLabelPoints[below](A,B,C)
			\tkzLabelPoints[right](H)
			\begin{scope}[on background layer]\path[white]node{MDD-108};\end{scope}
\end{tikzpicture}}
	}
\end{ex}
\begin{ex}%[Đề KSCL lần 1 THPT Lương Tài - Bắc Ninh, 2020-2021]%[Đoàn Minh Tân, 12-EX-3]%[2H1K3-3]
	Cho hình lăng trụ $ABC.A'B'C'$, trên cách cạnh $AA'$, $BB'$ lấy các điểm $M$, $N$ sao cho $AA'=4A'M$, $BB'=4B'N$. Mặt phẳng $(C'MN)$ chia khối lăng trụ thành hai phần. Gọi $V_1$ là thể tích khối chóp $C'.A'B'MN$ và $V_2$ là thể tích khối đa diện $ABCMNC'$. Tính tỉ số $\dfrac{V_1}{V_2}$.
	\choice
	{$\dfrac{V_1}{V_2}=\dfrac{2}{5}$}
	{$\dfrac{V_1}{V_2}=\dfrac{3}{5}$}
	{\True $\dfrac{V_1}{V_2}=\dfrac{1}{5}$}
	{$\dfrac{V_1}{V_2}=\dfrac{4}{5}$}
	\loigiai{
		\immini{Từ $AA'=4A'M$, $BB'=4B'N$ suy ra $A'M=B'N$ nên $A'B'NM$ là hình bình hành.\\
			Ta có $S_{A'B'MN}=\dfrac{1}{4}S_{A'B'BA}$ nên $$V_1=V_{C'.A'B'NM}=\dfrac{1}{4}V_{C'.A'B'BA}=\dfrac{1}{4}\cdot \dfrac{2}{3}V_{ABC.A'B'C'}=\dfrac{1}{6}V_{ABC.A'B'C'}.$$
			Suy ra $V_2=V_{ABCMNC'}=V_{ABC.A'B'C'}-V_1=\dfrac{5}{6}V_{ABC'A'B'C'}$.\\
			Vậy $\dfrac{V_1}{V_2}=\dfrac{1}{5}$.}
		{
			\begin{tikzpicture}[scale=0.7, line join = round, line cap = round]
			\tikzset{label style/.style={font=\footnotesize}}
			\tkzDefPoints{0/0/A, 2/-2/B, 5/0/C, 1/5/A'}
			\tkzDefPointBy[translation = from A to A'](B)	\tkzGetPoint{B'}
			\tkzDefPointBy[translation = from A to A'](C)	\tkzGetPoint{C'}
			\coordinate (M) at ($(A')!0.25!(A)$);
			\coordinate (N) at ($(B')!0.25!(B)$);
			\tkzDrawSegments(A,B B,C A,A' B,B' C,C'  A',B' A',C' B',C' M,N N,C')
			\tkzDrawSegments[dashed](A,C C',M)
			\tkzDrawPoints[fill=black](A,B,C,A',B',C',M,N)
			\tkzLabelPoints[above](A',B',C')
			\tkzLabelPoints[below](A,B,C)
			\tkzLabelPoints[below left](M,N)
			\begin{scope}[on background layer]\path[white]node{MDD-108};\end{scope}
\end{tikzpicture}}
	}
\end{ex}
\begin{ex}%[Đề KSCL lần 1 THPT Lương Tài - Bắc Ninh, 2020-2021]%[Đoàn Minh Tân, 12-EX-3]%	[1H3K5-4]
	Cho hình chóp $S.ABC$ có đáy $ABC$ là tam giác vuông cân tại $A$, $AB=AC=2a$, hình chiếu vuông góc của đỉnh $S$ lên mặt phẳng $(ABC)$ trùng với trung điểm $H$ của cạnh $AB$. Biết $SH=a$, khoảng cách giữa hai đường thẳng $SA$ và $BC$ là
	\choice
	{$\dfrac{a\sqrt{3}}{3}$}
	{\True $\dfrac{2a}{\sqrt{3}}$}
	{$\dfrac{4a}{\sqrt{3}}$}
	{$\dfrac{a\sqrt{3}}{2}$}
	\loigiai{
		\immini{Dựng hình bình hành $ACBD$, khi đó $BC\parallel (SAD)$.\\
			Suy ra $$\mathrm{d}(SA,BC)=\mathrm{d}(BC,(SAD))=\mathrm{d}(B,(SAD))=2\mathrm{d}(H,(SAD)).$$
			Kẻ $HK\perp AD$ tại $K$, $HL\perp SK$ tại $L$.\\
			Ta chứng minh được $HL\perp (SAD)$.\\
			Suy ra $\mathrm{d}(H,(SAD))=HL$.\\
			Xét $\triangle AHK$ có $HK=AH\cdot \sin{\widehat{HAD}}=a\cdot \sin45^{\circ}=\dfrac{a\sqrt{2}}{2}$.\\
			Xét $\triangle SHK$ có $\dfrac{1}{HL^2}=\dfrac{1}{SH^2}+\dfrac{1}{HK^2}\Rightarrow HL=\dfrac{a}{\sqrt{3}}$.\\
			Vậy $\mathrm{d}(SA,BC)=2 HL=\dfrac{2a}{\sqrt{3}}$.}
		{
			\begin{tikzpicture}[scale=0.6, line join = round, line cap = round]
			\tikzset{label style/.style={font=\footnotesize}}
			\tkzDefPoints{0/0/A, -4/-2/D, 2/-2/B, 6/0/C, 1/5/S}
			\coordinate (H) at ($(A)!0.5!(B)$);
			\coordinate (K) at ($(A)!0.25!(D)$);
			\coordinate (L) at ($(S)!0.6!(K)$);
			\tkzDrawSegments(S,D S,B S,C D,B B,C)
			\tkzDrawSegments[dashed](S,A S,H S,K A,B A,C H,L H,K A,D)
			\tkzMarkRightAngles(S,H,A B,A,C H,K,D K,L,H)
			\tkzDrawPoints[fill=black](A,B,C,D,S,H,L)
			\tkzLabelPoints[above](S)
			\tkzLabelPoints[below](B,C,D,H)
			\tkzLabelPoints[left](A,K,L)
			\begin{scope}[on background layer]\path[white]node{MDD-108};\end{scope}
\end{tikzpicture}}	
	}
\end{ex}
\begin{ex}%[Đề KSCL lần 1 THPT Lương Tài - Bắc Ninh, 2020-2021]%[Đoàn Minh Tân, 12-EX-3]%	[2D1K5-3]
	Tìm tất cả giá trị của tham số $m$ để phương trình $x^3-3x^2-m^3+3m^2=0$ có ba nghiệm phân biệt?
	\choice
	{\True $\heva{& -1<m<3\\ & m \ne 0\\ & m\ne 2}$}
	{$\heva{& -1<m<3\\ &m\ne 0}$}
	{$\heva{& -3<m<1\\ &m\ne -2}$}
	{$-3<m<1$}
	\loigiai{
		$$x^3-3x^2-m^3+3m^2=0\Leftrightarrow x^3-3x^2=m^3-3m^2. \quad (1)$$	
		Xét hàm số $y=x^3-3x^2$.\\
		$y'=3x^2-6x$, $y'=0\Leftrightarrow \hoac{& x=0\\ & x=2.}$\\
		Bảng biến thiên
		\begin{center}
			\begin{tikzpicture}
			\tkzTabInit[nocadre=false,lgt=1.2,espcl=2,deltacl=.6]
			{$x$/0.6,$y'$/0.6,$y$/2}
			{$-\infty$,$0$,$2$,$+\infty$}
			\tkzTabLine{,+,0,-,0,+}
			\tkzTabVar{-/$-\infty$, +/$0$,-/$-4$,+/$+\infty$}
			\begin{scope}[on background layer]\path[white]node{MDD-108};\end{scope}
\end{tikzpicture}
		\end{center}
		Dựa vào bảng biến thiên, để phương trình $(1)$ có $3$ nghiệm phân biệt thì
		$$-4<m^3-3m^2<0\Leftrightarrow \heva{& m^3-3m^2>-4\\ & m^3-3m^2<0}\Leftrightarrow \heva{& m>-1\\ & m\ne 2\\ & m<3\\ &m\ne 0}\Leftrightarrow \heva{& -1<m<3\\ & m\ne 2\\ & m\ne 0.}$$
	}
\end{ex}
\begin{ex}%[Đề KSCL lần 1 THPT Lương Tài - Bắc Ninh, 2020-2021]%[Đoàn Minh Tân, 12-EX-3]%	[2D1K3-1]
	Cho hàm số $y=\dfrac{2x-m}{x+2}$ với $m$ là tham số, $m\ne -4$. Biết $\displaystyle \min \limits _{[0;2]}f(x)+\displaystyle\max\limits_{[0;2]}f(x)=-8$. Giá trị của tham số $m$ bằng
	\choice
	{$9$}
	{\True $12$}
	{$10$}
	{$8$}
	\loigiai{
		Hàm số $y=\dfrac{2x-m}{x+2}$ có $y'=\dfrac{4+m}{(x+2)^2}\ne 0$, $\forall m\ne -4$.\\
		Nên hàm số hoặc đồng biến trên $[0;2]$ hoặc nghịch biến trên $[0;2]$.\\
		Từ đó suy ra 
		\begin{eqnarray*}
			& & \displaystyle \min \limits _{[0;2]}f(x)+\displaystyle\max\limits_{[0;2]}f(x)=-8\\
			&\Leftrightarrow& f(0)+f(2)=-8\\
			&\Leftrightarrow& \dfrac{-m}{2}+\dfrac{4-m}{4}=-8\\
			&\Leftrightarrow& m=12.
		\end{eqnarray*}
	}
\end{ex}


\Closesolutionfile{ans}
\begin{indapan}{10}
	{ans/ans-2-GHK1-22-LuongTai-BacNinh-21}
\end{indapan}

