\begin{name}
	{Biên soạn: Thầy BcoTuan, thầy Nguyễn Tiến \& Phản biện: Thầy Nguyễn Tiến, thầy BcoTuan}
	{Đề thi giữa học kỳ 1 năm học 2020-2021, THPT Nguyễn Hiền, Quảng Nam}
\end{name}
\setcounter{ex}{0}\setcounter{bt}{0}
\Opensolutionfile{ans}[ans/ans-2-GHK1-31-NguyenHien-QuangNam-21]

\begin{ex}%[Đề thi giữa học kỳ 1 năm học 2020-2021, THPT Nguyễn Hiền, Quảng Nam]%[BcoTuan, dự án 12EX3-2021]%[2D1Y2-2]
	Cho hàm số $y=f(x)$ có bảng biến thiên như sau
	\begin{center}
		\begin{tikzpicture}
		\tkzTabInit[lgt=1.2,espcl=3]
		{$x$ /1, $y'$ /1, $y$ /2}
		{$-\infty$,$-1$,$1$,$+\infty$}
		\tkzTabLine{ ,+,0,-,0,+, }
		\tkzTabVar{-/$-\infty$,+/$4$,-/$0$,+/$+\infty$}
		\begin{scope}[on background layer]\path[white]node{MDD-108};\end{scope}
\end{tikzpicture}
	\end{center}
	Tìm giá trị cực đại $y_{\text{CĐ}}$ và giá trị cực tiểu $y_{\text{CT}}$ của hàm số đã cho.
	\choice
	{$y_{\text{CĐ}}=4$ và $y_{\text{CT}}=-1$  }
	{$y_{\text{CĐ}}=1$ và $y_{\text{CT}}=0$  }
	{$y_{\text{CĐ}}=-1$ và $y_{\text{CT}}=1$  }
	{\True $y_{\text{CĐ}}=4$ và $y_{\text{CT}}=0$  }
	\loigiai{
		Dựa vào bảng biến thiên, hàm số có $y_{\text{CĐ}}=4$ và $y_{\text{CT}}=0$.
	}
\end{ex}


\begin{ex}%[GHKI, THPT Nguyễn Hiền, Quảng Nam, 2020]%[Nguyễn Tiến, dự án EX3]%[2D1Y1-2]
	Cho hàm số $y=f(x)$ có bảng biến thiên như sau
	\begin{center}
		\begin{tikzpicture}
		\tkzTabInit[nocadre=false,lgt=1.2,espcl=2.5,deltacl=0.6]
		{$x$ /0.6,$f'(x)$ /0.6,$f(x)$ /2}
		{$-\infty$,$2$,$+\infty$}
		\tkzTabLine{,+,d,+,}
		\tkzTabVar{-/$1$,+D-/$+\infty$/$-\infty$,+/$1$}
		\begin{scope}[on background layer]\path[white]node{MDD-108};\end{scope}
\end{tikzpicture}
	\end{center}
	Hàm số đã cho đồng biến trên khoảng nào dưới đây?
	\choice
	{$(-\infty;+\infty)$}
	{\True $(3;+\infty)$}
	{$(1;+\infty)$}
	{$(0;3)$}
	\loigiai{
		Từ bảng biến thiên ta thấy hàm số đồng biến trên khoảng $(-\infty;2)$ và $(2;+\infty)$.\\
		Từ các phương án đã cho, ta chọn hàm số đã cho đồng biến trên khoảng $(3;+\infty)$.
	}
\end{ex}


\begin{ex}%[Đề thi giữa học kỳ 1 năm học 2020-2021, THPT Nguyễn Hiền, Quảng Nam]%[BcoTuan, dự án 12EX3-2021]%[2H1Y3-2]
	Thể tích của khối chóp có diện tích đáy bằng $B$ và chiều cao $h$ được tính theo công thức nào dưới đây?
	\choice
	{$V=\dfrac{1}{2}Bh$}
	{$V=Bh$}
	{\True $V=\dfrac{1}{3}Bh$}
	{$V=3Bh$}
	\loigiai{
		Ta có $V=\dfrac{1}{3}Bh$.
	}
\end{ex}


\begin{ex}%[Đề thi giữa học kỳ 1 năm học 2020-2021, THPT Nguyễn Hiền, Quảng Nam]%[BcoTuan, dự án 12EX3-2021]%[2H1Y1-2]
	Số mặt của một khối bát diện đều là
	\choice
	{$4$}
	{$6$}
	{\True $8$}
	{$9$}
	\loigiai{
		\immini
		{
			Khối bát diện đều có $8$ mặt (tham khảo hình vẽ bên).
		}
		{
			\begin{tikzpicture}[join=round,scale=.9]
			\path (0,0) coordinate (A) ++(35:1.5) coordinate (B) (-15:2.5) coordinate (D) ($ (D)-(A)+(B) $) coordinate (C) ($ (B)+(60:1) $) coordinate (E) ($ (D)-(E)+(B) $) coordinate (E');
			\draw (E)--(D)--(E')--(A)--(E)--(C)--(E') (A)--(D)--(C);
			\draw[dashed] (A)--(B)--(C) (E)--(B)--(E');
			\tkzDrawPoints[fill=black](A,B,C,E,B,D,E')
			\begin{scope}[on background layer]\path[white]node{MDD-108};\end{scope}
\end{tikzpicture}
		}
	}
\end{ex}


\begin{ex}%[Đề thi giữa học kỳ 1 năm học 2020-2021, THPT Nguyễn Hiền, Quảng Nam]%[BcoTuan, dự án 12EX3-2021]%[2H1Y3-2]
	Cho khối lăng trụ có diện tích đáy $S$ và chiều cao $h$. Thể tích $V$ của khối lăng trụ đã cho được tính
	theo công thức nào dưới đây?
	\choice
	{\True $V=Sh$}
	{$V=\dfrac{Sh}{3}$}
	{$V=\dfrac{S}{h}$}
	{$V=\dfrac{1}{2}Sh$}
	\loigiai{
		Ta có $V=Sh$.
	}
\end{ex}


\begin{ex}%[Đề thi giữa học kỳ 1 năm học 2020-2021, THPT Nguyễn Hiền, Quảng Nam]%[BcoTuan, dự án 12EX3-2021]%[2D1Y2-2]
	\immini
	{
		Cho hàm số $y=ax^4+bx^2+c\ (a, b, c\in\mathbb{R})$ có đồ thị như hình vẽ bên. Số điểm cực trị của hàm số đã cho là
		\choice
		{$1$}
		{$2$}
		{$0$}
		{\True $3$}
	}
	{
		\begin{tikzpicture}[line join = round, line cap = round,>=stealth,x = 1cm,y = 1cm,font=\footnotesize]
		%Vẽ hệ trục Oxy
		\draw[->] (-2,0)--(0,0) node[below right]{$O$}--(2,0) node[below]{$x$};
		\draw[->] (0,-2.5)--(0,1.5) node[right]{$y$};
		%Vẽ các điểm trên hệ trục
		\foreach \x in {} \draw[thin] (\x,1pt)--(\x,-1pt) node [below] {$\x$};
		\foreach \y in {} \draw[thin] (1pt,\y)--(-1pt,\y) node [left] {$\y$};
		%Vẽ đồ thị hàm số
		\draw[samples=200,domain=-1.65:1.65,smooth] plot (\x, {(\x)^4 - 2*(\x)^2-1});
		\begin{scope}[on background layer]\path[white]node{MDD-108};\end{scope}
\end{tikzpicture}
	}
	\loigiai{
		Dựa vào đồ thị hàm số, hàm số đã cho có $3$ điểm cực trị.
	}
\end{ex}


\begin{ex}%[GHKI, THPT Nguyễn Hiền, Quảng Nam, 2020]%[Nguyễn Tiến, dự án EX3]%[2H1Y1-3]
	Mặt phẳng nào dưới đây chia khối lập phương $ABCD.A'B'C'D'$ thành hai khối lăng trụ đứng tam giác?
	\choice
	{$(ACD')$}
	{$(AA'D')$}
	{\True $(ACC')$}
	{$(BDC')$}
	\loigiai{
		\immini{
			Mặt phẳng $(ACC')$, cũng chính là mặt phẳng $(ACC'A')$, chia khối lập phương $ABCD.A'B'C'D'$ thành hai khối lăng trụ đứng tam giác là $ABC.A'B'C'$ và $ACD.A'C'D'$.
		}{
			\begin{tikzpicture}[scale=0.7, font=\footnotesize, line join=round, line cap=round, >=stealth]
			\tkzDefPoints{0/0/A,-1.3/-1.1/B,2/-1.1/C}
			\coordinate (D) at ($(A)+(C)-(B)$);
			\coordinate (A') at ($(A)+(0,2.5)$);
			\tkzDefPointsBy[translation=from A to A'](B,C,D){B'}{C'}{D'}
			\tkzDrawPolygon(A',B',B,C,D,D')
			\tkzDrawSegments(B',C' C',D' C,C' A',C')
			\tkzDrawSegments[dashed](A,B A,D A,A' A,C A,C')
			\tkzDrawPoints[fill=black,size=4](A,B,D,C,A',B',C',D')
			\tkzLabelPoints[above](A',D')
			\tkzLabelPoints[below](A,B,C)
			\tkzLabelPoints[left](B')
			\tkzLabelPoints[right](C',D)
			\begin{scope}[on background layer]\path[white]node{MDD-108};\end{scope}
\end{tikzpicture}
		}
	}
\end{ex}


\begin{ex}%[Đề thi giữa học kỳ 1 năm học 2020-2021, THPT Nguyễn Hiền, Quảng Nam]%[BcoTuan, dự án 12EX3-2021]%[2H1Y3-2]
	Thể tích của khối chóp có diện tích đáy $6a^2$ và chiều cao $2a$ bằng
	\choice
	{$\dfrac{4a^3}{3}$}
	{\True $4a^3$}
	{$24a^3$}
	{$9a^3$}
	\loigiai{
		Gọi $V$ là thể tích khối chóp, ta có $V=\dfrac{1}{3}\cdot 6a^2\cdot 2a=4a^3$.
	}
\end{ex}


\begin{ex}%[GHKI, THPT Nguyễn Hiền, Quảng Nam, 2020]%[Nguyễn Tiến, dự án EX3]%[2D1Y3-1]
	Cho hàm số $y=f(x)$ liên tục và có bảng biến thiên trong đoạn $\left[-\sqrt{3};\sqrt{2}\right]$ như hình vẽ bên dưới. Mệnh đề nào dưới đây đúng?
	\begin{center}
		\begin{tikzpicture}
		\tkzTabInit[nocadre=false,lgt=1.2,espcl=2.5,deltacl=0.6]
		{$x$ /0.6,$f'(x)$ /0.6,$f(x)$ /2}
		{$-\sqrt{3}$,$0$,$1$,$\sqrt{2}$}
		\tkzTabLine{,-,$0$,+,$0$,-,}
		\tkzTabVar{+/$\dfrac{5}{2}$, -/$-\dfrac{5}{2}$,+/$\dfrac{3}{2}$,-/$0$}
		\begin{scope}[on background layer]\path[white]node{MDD-108};\end{scope}
\end{tikzpicture}
	\end{center}
	\choice
	{$\min\limits_{\left[-\sqrt{3};\sqrt{2}\right]} f(x)+\max\limits_{\left[-\sqrt{3};\sqrt{2}\right]} f(x)=\dfrac{5}{2}$}
	{$\min\limits_{\left[-\sqrt{3};\sqrt{2}\right]} f(x)+\max\limits_{\left[-\sqrt{3};\sqrt{2}\right]} f(x)=-1$}
	{\True $\min\limits_{\left[-\sqrt{3};\sqrt{2}\right]} f(x)+\max\limits_{\left[-\sqrt{3};\sqrt{2}\right]} f(x)=0$}
	{$\min\limits_{\left[-\sqrt{3};\sqrt{2}\right]} f(x)+\max\limits_{\left[-\sqrt{3};\sqrt{2}\right]} f(x)=-\dfrac{5}{2}$}
	\loigiai{
		Từ bảng biến thiên, ta thấy $\min\limits_{\left[-\sqrt{3};\sqrt{2}\right]} f(x)=-\dfrac{5}{2}$ và $\max\limits_{\left[-\sqrt{3};\sqrt{2}\right]} f(x)=\dfrac{5}{2}$.\\
		Vậy $\min\limits_{\left[-\sqrt{3};\sqrt{2}\right]} f(x)+\max\limits_{\left[-\sqrt{3};\sqrt{2}\right]} f(x)=0$.
	}
\end{ex}


\begin{ex}%[Đề thi giữa học kỳ 1 năm học 2020-2021, THPT Nguyễn Hiền, Quảng Nam]%[BcoTuan, dự án 12EX3-2021]%[2D1Y5-3]
	\immini
	{
		Cho hàm số $f(x)=a x^3+b x^2+c x+d\ (a, b, c, d \in \mathbb{R})$ có đồ thị như hình vẽ bên. Số nghiệm thực của phương trình $4 f(x)+3=0$ là
		\choice
		{$2$}
		{\True $3$}
		{$1$}
		{$0$}
	}
	{
		\begin{tikzpicture}[line join = round, line cap = round,>=stealth,x = 1cm,y = 1cm,font=\footnotesize,scale=.8]
		%Vẽ hệ trục Oxy
		\draw[->] (-2,0)--(0,0) node[below right]{$O$}--(3.5,0) node[below]{$x$};
		\draw[->] (0,-4)--(0,2.5) node[right]{$y$};
		%Vẽ các điểm trên hệ trục
		\foreach \x in {-1,2} \draw[thin] (\x,1pt)--(\x,-1pt) node [above] {$\x$};
		\foreach \y in {-3,1} \draw[thin] (1pt,\y)--(-1pt,\y) node [above left=-2pt] {$\y$};
		%Vẽ đồ thị hàm số
		\draw[samples=200,domain=-1.1:3.1,smooth] plot (\x, {(\x)^3 - 3*(\x)^2+1});
		\draw[dashed] (-1,0)--(-1,-3)--(2,-3)--(2,0);
		\begin{scope}[on background layer]\path[white]node{MDD-108};\end{scope}
\end{tikzpicture}
	}
	\loigiai{
		\immini
		{
			Ta có $4f(x)+3=0\Leftrightarrow f(x)=-\dfrac{3}{4}$.\\
			Số nghiệm của phương trình là số giao điểm của đồ thị hàm số $y=f(x)$ và $y=-\dfrac{3}{4}$.\\
			Dựa vào đồ thị hàm số, phương trình đã cho có $3$ nghiệm thực.
		}
		{
			\begin{tikzpicture}[line join = round, line cap = round,>=stealth,x = 1cm,y = 1cm,font=\footnotesize,scale=.8]
			%Vẽ hệ trục Oxy
			\draw[->] (-2,0)--(0,0) node[below right]{$O$}--(3.5,0) node[below]{$x$};
			\draw[->] (0,-4)--(0,2.5) node[right]{$y$};
			%Vẽ các điểm trên hệ trục
			\foreach \x in {-1,2} \draw[thin] (\x,1pt)--(\x,-1pt) node [above] {$\x$};
			\foreach \y in {-3,1} \draw[thin] (1pt,\y)--(-1pt,\y) node [above left=-2pt] {$\y$};
			%Vẽ đồ thị hàm số
			\draw[samples=200,domain=-1.1:3.1,smooth] plot (\x, {(\x)^3 - 3*(\x)^2+1});
			\draw[dashed] (-1,0)--(-1,-3)--(2,-3)--(2,0);
			\draw[red](-1.5,-.75)--(3.1,-.75) node[below right]{$y=-\dfrac{3}{4}$};
			\begin{scope}[on background layer]\path[white]node{MDD-108};\end{scope}
\end{tikzpicture}
		}
	}
\end{ex}


\begin{ex}%[Đề thi giữa học kỳ 1 năm học 2020-2021, THPT Nguyễn Hiền, Quảng Nam]%[BcoTuan, dự án 12EX3-2021]%[2D1Y4-1]
	Đường thẳng nào dưới đây là tiệm cận ngang của đồ thị hàm số $y=\dfrac{x-2}{2x+1}$?
	\choice
	{$y=-\dfrac{1}{2}$}
	{\True $y=\dfrac{1}{2}$}
	{$x=-\dfrac{1}{2}$}
	{$x=\dfrac{1}{2}$}
	\loigiai{
		Tập xác định $\mathscr{D}=\mathbb{R}\setminus \left\{-\dfrac{1}{2}\right\}$.\\
		Ta có $\lim\limits_{x\to +\infty}\dfrac{x-2}{2x+1}=\dfrac{1}{2}, \lim\limits_{x\to-\infty}\dfrac{x-2}{2x+1}=\dfrac{1}{2}$.\\
		Vậy đồ thị hàm số có đường tiệm cận ngang là $y=\dfrac{1}{2}$.
	}
\end{ex}


\begin{ex}%[GHKI, THPT Nguyễn Hiền, Quảng Nam, 2020]%[Nguyễn Tiến, dự án EX3]%[2D1B2-1]
	Cho hàm số $y=4x^3+6x^2-10$. Mệnh đề nào dưới đây đúng?
	\choice
	{\True Cực tiểu của hàm số bằng $-10$}
	{Cực đại của hàm số bằng $8$}
	{Cực đại của hàm số bằng $-1$}
	{Cực tiểu của hàm số bằng $0$}
	\loigiai{
		Ta có $y'=12x^2+12x$; $y'=0 \Leftrightarrow \hoac{& x=-1\\& x=0.}$\\
		Bảng biến thiên
		\begin{center}
			\begin{tikzpicture}
			\tkzTabInit[nocadre=false,lgt=1.2,espcl=2.5,deltacl=0.6]
			{$x$ /0.6,$y'$ /0.6,$y$ /2}
			{$-\infty$,$-1$,$0$,$+\infty$}
			\tkzTabLine{,+,$0$,-,$0$,+,}
			\tkzTabVar{-/$-\infty$, +/$-8$,-/$-10$,+/$+\infty$}
			\begin{scope}[on background layer]\path[white]node{MDD-108};\end{scope}
\end{tikzpicture}
		\end{center}
		Từ đó ta thấy cực tiểu của hàm số bằng $-10$.
	}
\end{ex}


\begin{ex}%[GHKI, THPT Nguyễn Hiền, Quảng Nam, 2020]%[Nguyễn Tiến, dự án EX3]%[2D1B5-1]
	\immini{
		Cho hàm số $y=\dfrac{ax-b}{cx+2}$ ($a$, $b$, $c\in\mathbb{R}$; $c\neq 0$) có đồ thị như hình vẽ bên. Giá trị của biểu thức $a+b+c$ bằng
		\choice
		{$-4$}
		{$-3$}
		{\True $3$}
		{$5$}
	}{
		\begin{tikzpicture}[scale=0.7, font=\footnotesize, line join=round, line cap=round, >=stealth]
		\def\a{1} \def\b{-3} \def\c{-1} \def\d{2} % Hệ số
		\def\xt{-2} \def\xp{6} \def\yt{2} \def\yd{-4} % x_trái, x_phải, y_trên, y_dưới (giới hạn)
		\draw[->] (\xt,0)--(\xp,0) node [below]{$x$};
		\draw[->] (0,\yd)--(0,\yt) node [left]{$y$};
		\fill (0,0) circle (1.5pt) node[above left]{$O$} (1,0) circle (1.5pt) node[below]{$1$} (2,0) circle (1.5pt) node[below left]{$2$} (3,0) circle (1.5pt) node[below]{$3$} (0,-1) circle (1.5pt) node[above left]{$-1$} (0,-1.5) circle (1.5pt) node[below left]{$-\dfrac{3}{2}$};
		\clip (\xt+0.1,\yd+0.1) rectangle (\xp-0.1,\yt-0.1);
		\draw[smooth,samples=300,domain=\xt:(-\d/\c-0.1)] plot(\x,{(\a*(\x)+\b)/(\c*(\x)+\d)});
		\draw[smooth,samples=300,domain=(-\d/\c+0.1:\xp)] plot(\x,{(\a*(\x)+\b)/(\c*(\x)+\d)});
		\draw[dashed] (-\d/\c,\yd)--(-\d/\c,\yt);
		\draw[dashed] (\xt,\a/\c)--(\xp,\a/\c);
		\begin{scope}[on background layer]\path[white]node{MDD-108};\end{scope}
\end{tikzpicture}
	}
	\loigiai{
		Từ hình vẽ, ta thấy đồ thị hàm số có
		\begin{itemize}
			\item Đường tiệm cận đứng $x=2$, suy ra $-\dfrac{2}{c}=2 \Leftrightarrow c=-1$.
			\item Đường tiệm cận ngang $y=-1$, suy ra $\dfrac{a}{c}=-1 \Leftrightarrow a=-c=1$.
			\item Giao điểm với trục $Oy$ tại điểm $\left(0;-\dfrac{3}{2}\right)$, suy ra $-\dfrac{b}{2}=-\dfrac{3}{2} \Leftrightarrow b=3$.
		\end{itemize}
		Vậy $a+b+c=1+3-1=3$.
	}
\end{ex}


\begin{ex}%[GHKI, THPT Nguyễn Hiền, Quảng Nam, 2020]%[Nguyễn Tiến, dự án EX3]%[2H1B3-2]
	Tính thể tích $V$ của khối lăng trụ tam giác đều có tất cả các cạnh đều bằng $3$.
	\choice
	{\True $V=\dfrac{27\sqrt{3}}{4}$}
	{$V=\dfrac{\sqrt{3}}{4}$}
	{$V=\dfrac{21\sqrt{3}}{4}$}
	{$V=\dfrac{15\sqrt{3}}{4}$}
	\loigiai{
		\immini{
			Ta có $\triangle ABC$ đều cạnh bằng $3$ nên diện tích đáy là
			$$S_{ABC}=\dfrac{AB^2\sqrt{3}}{4}=\dfrac{3^2\sqrt{3}}{4}=\dfrac{9\sqrt{3}}{4}.$$
			Do lăng trụ đều nên $AA'=3$ là chiều cao khối lăng trụ.\\
			Vậy thể tích $V$ của khối lăng trụ đã cho là
			$$V=AA'\cdot S_{ABC}=3\cdot \dfrac{9\sqrt{3}}{4}=\dfrac{27\sqrt{3}}{4}.$$
		}{
			\begin{tikzpicture}[scale=1, font=\footnotesize, line join=round, line cap=round, >=stealth]
			\tkzDefPoints{0/0/A,1.1/-1.5/B,3.5/0/C}
			\coordinate (A') at ($(A)+(0,3.2)$);
			\tkzDefPointsBy[translation=from A to A'](B,C){B'}{C'}
			\tkzDrawPolygon(A,B,C,C',B',A')
			\tkzDrawSegments(A',C' B',B)
			\tkzDrawSegments[dashed](A,C)
			\tkzDrawPoints[fill=black,size=4](A,C,B,A',B',C')
			\tkzLabelPoints[above](B')
			\tkzLabelPoints[below](B)
			\tkzLabelPoints[left](A',A)
			\tkzLabelPoints[right](C',C)
			\begin{scope}[on background layer]\path[white]node{MDD-108};\end{scope}
\end{tikzpicture}
		}
	}
\end{ex}


\begin{ex}%[GHKI, THPT Nguyễn Hiền, Quảng Nam, 2020]%[Nguyễn Tiến, dự án EX3]%[2D1B5-4]
	Gọi $S$ là tập hợp các giá trị nguyên của tham số $m$ để phương trình $2x^3-3x^2-2+\dfrac{m}{4}=0$ có ba nghiệm phân biệt. Tổng các phần tử của $S$ là
	\choice
	{$18$}
	{\True $30$}
	{$-30$}
	{$-18$}
	\loigiai{
		Ta có $2x^3-3x^2-2+\dfrac{m}{4}=0 \Leftrightarrow \dfrac{m}{4}=-2x^3+3x^2+2$.\\
		Phương trình trên là phương trình hoành độ giao điểm của hai đồ thị $(C)\colon y=-2x^3+3x^2+2$ và $d\colon y=\dfrac{m}{4}$.\\
		Số giao điểm của $(C)$ và $d$ là số nghiệm của phương trình đã cho.\\
		Ta có $y'=-6x^2+6x$; $y'=0 \Leftrightarrow \hoac{& x=0\\& x=1.}$\\
		Bảng biến thiên
		\begin{center}
			\begin{tikzpicture}
			\tkzTabInit[nocadre=false,lgt=1.2,espcl=2.5,deltacl=0.6]
			{$x$ /0.6,$y'$ /0.6,$y$ /2}
			{$-\infty$,$0$,$1$,$+\infty$}
			\tkzTabLine{,-,$0$,+,$0$,-,}
			\tkzTabVar{+/$+\infty$, -/$2$,+/$3$,-/$-\infty$}
			\begin{scope}[on background layer]\path[white]node{MDD-108};\end{scope}
\end{tikzpicture}
		\end{center}
		Để phương trình đã cho có ba nghiệm phân biệt thì
		$$2<\dfrac{m}{4}<3 \Leftrightarrow 8<m<12.$$
		Do $m\in\mathbb{Z}$ nên $m\in\{9;10;11\}$.\\
		Vậy tổng các phần tử của $S$ là $9+10+11=30$.
	}
\end{ex}


\begin{ex}%[Đề thi giữa học kỳ 1 năm học 2020-2021, THPT Nguyễn Hiền, Quảng Nam]%[BcoTuan, dự án 12EX3-2021]%[2D1B4-1]
	Số tiệm cận đứng của đồ thị hàm số $y=\dfrac{x-2}{x^2-3x+2}$ là
	\choice
	{\True $1$}
	{$0$}
	{$2$}
	{$3$}
	\loigiai{
		Tập xác định $\mathscr{D}=\mathbb{R}\setminus\{1;2\}$.\\
		Ta có $\lim\limits_{x\to 2}\dfrac{x-2}{x^2-3x+2}=\lim\limits_{x\to 2}\dfrac{1}{x-1}=1$.\\
		Và $\lim\limits_{x\to 1^+}y=\lim\limits_{x\to 1^+}\dfrac{1}{x-1}=+\infty$, $\lim\limits_{x\to 1^-}y=\lim\limits_{x\to 1^-}\dfrac{1}{x-1}=-\infty$.\\
		Vậy đồ thị hàm số có một đường tiệm cận đứng.
	}
\end{ex}


\begin{ex}%[Đề thi giữa học kỳ 1 năm học 2020-2021, THPT Nguyễn Hiền, Quảng Nam]%[BcoTuan, dự án 12EX3-2021]%[2D1B3-1]
	Giá trị lớn nhất của hàm số $y=x^3+3x^2$ trên đoạn $[-2;2]$ bằng
	\choice
	{\True $20$}
	{$-4$}
	{$4$}
	{$-20$}
	\loigiai{
		Hàm số đã cho liên tục trên $[-2;2]$.\\
		Ta có $y'=3x^2+6x, y'=0\Leftrightarrow\hoac{& x=0\in (-2;2) \\ & x=-2\notin (-2;2).}$\\
		Ta có $y(0)=0, y(-2)=4, y(2)=20$. Vậy $\max\limits_{x \in [-2;2]} y=20$.
	}
\end{ex}


\begin{ex}%[GHKI, THPT Nguyễn Hiền, Quảng Nam, 2020]%[Nguyễn Tiến, dự án EX3]%[2D1B3-1]
	Gọi $M$ và $m$ lần lượt là giá trị lớn nhất và giá trị nhỏ nhất của hàm số $y=x\sqrt{2-x^2}$. Giá trị của tích $M$ và $m$ bằng
	\choice
	{$2$}
	{\True $-1$}
	{$0$}
	{$1$}
	\loigiai{
		Hàm số đã cho liên tục trên tập xác định $\mathscr D=\left[-\sqrt{2};\sqrt{2}\right]$.\\
		Ta có $y'=\sqrt{2-x^2}-\dfrac{x^2}{\sqrt{2-x^2}}=\dfrac{2-2x^2}{\sqrt{2-x^2}}$.\\
		Cho $y'=0 \Leftrightarrow x=\pm 1\in \mathscr D$.\\
		Khi đó $y\left(-\sqrt{2}\right)=y\left(\sqrt{2}\right)=0$; $y(1)=1$; $y(-1)=-1$.\\
		Vậy $M=\max\limits_{\mathscr D} y=1$, $m=\min\limits_{\mathscr D} y=-1$. Suy ra $M\cdot m=-1$.
	}
\end{ex}


\begin{ex}%[Đề thi giữa học kỳ 1 năm học 2020-2021, THPT Nguyễn Hiền, Quảng Nam]%[BcoTuan, dự án 12EX3-2021]%[2H1B2-3]
	Số mặt phẳng đối xứng của hình lập phương là
	\choice
	{$8$}
	{$6$}
	{$4$}
	{\True $9$}
	\loigiai{
		Hình lập phương có $9$ mặt phẳng đối xứng.
		\begin{center}
			\begin{tikzpicture}[scale=.7, line join=round, line cap=round]
			\tkzDefPoints{0/0/A,-2/-1/B,1/-1/C}
			\coordinate (D) at ($(A)+(C)-(B)$);
			\coordinate (A') at ($(A)+(0,2.5)$);
			\tkzDefPointsBy[translation=from A to A'](B,C,D){B'}{C'}{D'}
			\tkzDrawPolygon[draw = white, fill= gray, opacity=.2](A,C,C',A')
			\tkzDrawPolygon(A',B',B,C,D,D')
			\tkzDrawSegments(B',C' C',D' C,C')
			\tkzDrawSegments[dashed](A,B A,D A,A')
			\begin{scope}[on background layer]\path[white]node{MDD-108};\end{scope}
\end{tikzpicture}$\quad$$\quad$
			\begin{tikzpicture}[scale=.7, line join=round, line cap=round]
			\tkzDefPoints{0/0/A,-2/-1/B,1/-1/C}
			\coordinate (D) at ($(A)+(C)-(B)$);
			\coordinate (A') at ($(A)+(0,2.5)$);
			\tkzDefPointsBy[translation=from A to A'](B,C,D){B'}{C'}{D'}
			\tkzDrawPolygon[draw = white, fill= gray, opacity=.2](B,D,D',B')
			\tkzDrawPolygon(A',B',B,C,D,D')
			\tkzDrawSegments(B',C' C',D' C,C')
			\tkzDrawSegments[dashed](A,B A,D A,A')
			\begin{scope}[on background layer]\path[white]node{MDD-108};\end{scope}
\end{tikzpicture}$\quad$$\quad$
			\begin{tikzpicture}[scale=.7, line join=round, line cap=round]
			\tkzDefPoints{0/0/A,-2/-1/B,1/-1/C}
			\coordinate (D) at ($(A)+(C)-(B)$);
			\coordinate (A') at ($(A)+(0,2.5)$);
			\tkzDefPointsBy[translation=from A to A'](B,C,D){B'}{C'}{D'}
			\coordinate (M) at ($(C)!1/2!(B)$);
			\coordinate (N) at ($(C')!1/2!(B')$);
			\coordinate (P) at ($(A)!1/2!(D)$);
			\coordinate (Q) at ($(A')!1/2!(D')$);
			\tkzDrawPolygon[draw = white, fill= gray, opacity=.2](M,N,Q,P)
			\tkzDrawPolygon(A',B',B,C,D,D')
			\tkzDrawSegments(B',C' C',D' C,C')
			\tkzDrawSegments[dashed](A,B A,D A,A')
			\begin{scope}[on background layer]\path[white]node{MDD-108};\end{scope}
\end{tikzpicture}\\
			\begin{tikzpicture}[scale=.7, line join=round, line cap=round]
			\tkzDefPoints{0/0/A,-2/-1/B,1/-1/C}
			\coordinate (D) at ($(A)+(C)-(B)$);
			\coordinate (A') at ($(A)+(0,2.5)$);
			\tkzDefPointsBy[translation=from A to A'](B,C,D){B'}{C'}{D'}
			\coordinate (M) at ($(A)!1/2!(B)$);
			\coordinate (N) at ($(A')!1/2!(B')$);
			\coordinate (P) at ($(C)!1/2!(D)$);
			\coordinate (Q) at ($(C')!1/2!(D')$);
			\tkzDrawPolygon[draw = white, fill= gray, opacity=.2](M,N,Q,P)
			\tkzDrawPolygon(A',B',B,C,D,D')
			\tkzDrawSegments(B',C' C',D' C,C')
			\tkzDrawSegments[dashed](A,B A,D A,A')
			\begin{scope}[on background layer]\path[white]node{MDD-108};\end{scope}
\end{tikzpicture}$\quad$$\quad$
			\begin{tikzpicture}[scale=.7, line join=round, line cap=round]
			\tkzDefPoints{0/0/A,-2/-1/B,1/-1/C}
			\coordinate (D) at ($(A)+(C)-(B)$);
			\coordinate (A') at ($(A)+(0,2.5)$);
			\tkzDefPointsBy[translation=from A to A'](B,C,D){B'}{C'}{D'}
			\coordinate (M) at ($(A)!1/2!(A')$);
			\coordinate (N) at ($(B)!1/2!(B')$);
			\coordinate (P) at ($(C)!1/2!(C')$);
			\coordinate (Q) at ($(D)!1/2!(D')$);
			\tkzDrawPolygon[draw = white, fill= gray, opacity=.2](M,N,P,Q)
			\tkzDrawPolygon(A',B',B,C,D,D')
			\tkzDrawSegments(B',C' C',D' C,C')
			\tkzDrawSegments[dashed](A,B A,D A,A')
			\begin{scope}[on background layer]\path[white]node{MDD-108};\end{scope}
\end{tikzpicture}$\quad$$\quad$
			\begin{tikzpicture}[scale=.7, line join=round, line cap=round]
			\tkzDefPoints{0/0/A,-2/-1/B,1/-1/C}
			\coordinate (D) at ($(A)+(C)-(B)$);
			\coordinate (A') at ($(A)+(0,2.5)$);
			\tkzDefPointsBy[translation=from A to A'](B,C,D){B'}{C'}{D'}
			\tkzDrawPolygon[draw = white, fill= gray, opacity=.2](A',D',C,B)
			\tkzDrawPolygon(A',B',B,C,D,D')
			\tkzDrawSegments(B',C' C',D' C,C')
			\tkzDrawSegments[dashed](A,B A,D A,A')
			\begin{scope}[on background layer]\path[white]node{MDD-108};\end{scope}
\end{tikzpicture}\\
			\begin{tikzpicture}[scale=.7, line join=round, line cap=round]
			\tkzDefPoints{0/0/A,-2/-1/B,1/-1/C}
			\coordinate (D) at ($(A)+(C)-(B)$);
			\coordinate (A') at ($(A)+(0,2.5)$);
			\tkzDefPointsBy[translation=from A to A'](B,C,D){B'}{C'}{D'}
			\tkzDrawPolygon[draw = white, fill= gray, opacity=.2](A,D,C',B')
			\tkzDrawPolygon(A',B',B,C,D,D')
			\tkzDrawSegments(B',C' C',D' C,C')
			\tkzDrawSegments[dashed](A,B A,D A,A')
			\begin{scope}[on background layer]\path[white]node{MDD-108};\end{scope}
\end{tikzpicture}$\quad$$\quad$
			\begin{tikzpicture}[scale=.7, line join=round, line cap=round]
			\tkzDefPoints{0/0/A,-2/-1/B,1/-1/C}
			\coordinate (D) at ($(A)+(C)-(B)$);
			\coordinate (A') at ($(A)+(0,2.5)$);
			\tkzDefPointsBy[translation=from A to A'](B,C,D){B'}{C'}{D'}
			\tkzDrawPolygon[draw = white, fill= gray, opacity=.2](A',B',C,D)
			\tkzDrawPolygon(A',B',B,C,D,D')
			\tkzDrawSegments(B',C' C',D' C,C')
			\tkzDrawSegments[dashed](A,B A,D A,A')
			\begin{scope}[on background layer]\path[white]node{MDD-108};\end{scope}
\end{tikzpicture}$\quad$$\quad$
			\begin{tikzpicture}[scale=.7, line join=round, line cap=round]
			\tkzDefPoints{0/0/A,-2/-1/B,1/-1/C}
			\coordinate (D) at ($(A)+(C)-(B)$);
			\coordinate (A') at ($(A)+(0,2.5)$);
			\tkzDefPointsBy[translation=from A to A'](B,C,D){B'}{C'}{D'}
			\tkzDrawPolygon[draw = white, fill= gray, opacity=.2](A,B,C',D')
			\tkzDrawPolygon(A',B',B,C,D,D')
			\tkzDrawSegments(B',C' C',D' C,C')
			\tkzDrawSegments[dashed](A,B A,D A,A')
			\begin{scope}[on background layer]\path[white]node{MDD-108};\end{scope}
\end{tikzpicture}
		\end{center}
	}
\end{ex}


\begin{ex}%[Đề thi giữa học kỳ 1 năm học 2020-2021, THPT Nguyễn Hiền, Quảng Nam]%[BcoTuan, dự án 12EX3-2021]%[2D1B5-1]
	\immini
	{
		Đường cong trong hình vẽ bên là đồ thị của hàm số nào dưới đây?
		\choice
		{$y=-x^4+2x^2+1$}
		{$y=-x^4-2x^2$}
		{$y=-2x^4+8x^2$}
		{\True $y=-x^4+2x^2$}
	}
	{
		\begin{tikzpicture}[line join = round, line cap = round,>=stealth,x = 1cm,y = 1cm,font=\footnotesize]
		%Vẽ hệ trục Oxy
		\draw[->] (-2,0)--(0,0) node[below right]{$O$}--(2,0) node[below]{$x$};
		\draw[->] (0,-1)--(0,2) node[right]{$y$};
		%Vẽ các điểm trên hệ trục
		\foreach \x in {-1,1} \draw[thin] (\x,1pt)--(\x,-1pt) node [below] {$\x$};
		\foreach \y in {1} \draw[thin] (1pt,\y)--(-1pt,\y) node [above left] {$\y$};
		%Vẽ đồ thị hàm số
		\draw[samples=200,domain=-1.5:1.5,smooth] plot (\x, {-(\x)^4 + 2*(\x)^2});
		\draw[dashed](-1,0)--(-1,1)--(1,1)--(1,0);
		\begin{scope}[on background layer]\path[white]node{MDD-108};\end{scope}
\end{tikzpicture}
	}
	\loigiai{
		Ta thấy hàm số có ba điểm cực trị, đi qua các điểm $(0;0)$, $(-1;1)$ và $(1;1)$ nên hàm số thỏa mãn là $y=-x^4+2x^2$.
	}
\end{ex}


\begin{ex}%[Đề thi giữa học kỳ 1 năm học 2020-2021, THPT Nguyễn Hiền, Quảng Nam]%[BcoTuan, dự án 12EX3-2021]%[2D1B5-1]
	\immini
	{
		Đường cong trong hình vẽ bên là đồ thị của hàm số nào dưới đây?
		\choice
		{$y=-x^3+3x+2$}
		{$y=2x^3-6x^2+2$}
		{$y=x^3-3x^2$}
		{\True $y=x^3-3x^2+2$}
	}
	{
		\begin{tikzpicture}[line join = round, line cap = round,>=stealth,x = 1cm,y = 1cm,font=\footnotesize,scale=.8]
		%Vẽ hệ trục Oxy
		\draw[->] (-1.5,0)--(0,0) node[below right]{$O$}--(3.5,0) node[below]{$x$};
		\draw[->] (0,-2.5)--(0,3) node[right]{$y$};
		%Vẽ các điểm trên hệ trục
		\foreach \x in {2} \draw[thin] (\x,1pt)--(\x,-1pt) node [above] {$\x$};
		\foreach \y in {-2,2} \draw[thin] (1pt,\y)--(-1pt,\y) node [above left=-2pt] {$\y$};
		%Vẽ đồ thị hàm số
		\draw[samples=200,domain=-1:3,smooth] plot (\x, {(\x)^3 - 3*(\x)^2+2});
		\draw[dashed] (0,-2)--(2,-2)--(2,0);
		\begin{scope}[on background layer]\path[white]node{MDD-108};\end{scope}
\end{tikzpicture}
	}
	\loigiai{
		Đồ thị trên là của hàm số bậc ba $y=ax^3+bx^2+cx+d$.\\
		Nhánh phải của đồ thị đi lên trên nên $a>0$, đồng thời đồ thị hàm số đi qua điểm $(2;-2)$ nên hàm số là $y=x^3-3x^2+2$.
	}
\end{ex}


\begin{ex}%[Đề thi giữa học kỳ 1 năm học 2020-2021, THPT Nguyễn Hiền, Quảng Nam]%[BcoTuan, dự án 12EX3-2021]%[2D1B1-1]
	Hàm số $y=x^3-3x$ nghịch biến trên khoảng nào dưới đây?
	\choice
	{$(0;+\infty)$}
	{\True $(-1;1)$}
	{$(-\infty;+\infty)$}
	{$(-\infty;-1)$}
	\loigiai{
		Tập xác định $\mathscr{D}=\mathbb{R}$, ta có $y'=3x^2-3, y'=0\Leftrightarrow \hoac{& x=1 \\ & x=-1.}$\\
		Bảng biến thiên
		\begin{center}
			\begin{tikzpicture}
			\tkzTabInit[lgt=1.2,espcl=2.5]
			{$x$ /.9, $y'$ /.9, $y$ /2}
			{$-\infty$,$-1$,$1$,$+\infty$}
			\tkzTabLine{ ,+,0,-,0,+, }
			\tkzTabVar{-/$-\infty$,+/$2$,-/$-2$,+/$+\infty$}
			\begin{scope}[on background layer]\path[white]node{MDD-108};\end{scope}
\end{tikzpicture}
		\end{center}
		Dựa vào bảng biến thiên, hàm số nghịch biến trên khoảng $(-1;1)$.
	}
\end{ex}


\begin{ex}%[GHKI, THPT Nguyễn Hiền, Quảng Nam, 2020]%[Nguyễn Tiến, dự án EX3]%[2H1B3-2]
	Cho hình chóp $S.ABCD$ có đáy $ABCD$ là hình vuông cạnh $a$, $SA$ vuông góc với đáy và $SB=2a$. Tính thể tích $V$ của khối chóp đã cho.
	\choice
	{$V=\dfrac{\sqrt{3}}{6}a^3$}
	{$V=2\sqrt{3}a^3$}
	{\True $V=\dfrac{\sqrt{3}}{3}a^3$}
	{$V=\sqrt{3}a^3$}
	\loigiai{
		\immini{
			Diện tích đáy là $S_{ABCD}=a^2$.\\
			Xét $\triangle SAB$ có $SA=\sqrt{SB^2-AB^2}=\sqrt{4a^2-a^2}=a\sqrt{3}$.\\
			Vậy thể tích của khối chóp đã cho là
			$$V=\dfrac{1}{3}SA\cdot S_{ABCD}=\dfrac{1}{3}\cdot a\sqrt{3}\cdot a^2=\dfrac{a^3\sqrt{3}}{3}.$$
		}{
			\begin{tikzpicture}[scale=0.7, font=\footnotesize, line join=round, line cap=round, >=stealth]
			\tkzDefPoints{0/0/A,-1.3/-1.6/B,2.5/-1.6/C}
			\coordinate (D) at ($(A)+(C)-(B)$);
			\coordinate (S) at ($(A)+(0,3)$);
			\tkzDrawPolygon(S,B,C,D)
			\tkzDrawSegments(S,C)
			\tkzDrawSegments[dashed](A,S A,B A,D)
			\tkzDrawPoints[fill=black,size=4](D,C,A,B,S)
			\tkzLabelPoints[above](S)
			\tkzLabelPoints[below](A,B,C)
			\tkzLabelPoints[right](D)
			\begin{scope}[on background layer]\path[white]node{MDD-108};\end{scope}
\end{tikzpicture}
		}
	}
\end{ex}


\begin{ex}%[Đề thi giữa học kỳ 1 năm học 2020-2021, THPT Nguyễn Hiền, Quảng Nam]%[BcoTuan, dự án 12EX3-2021]%[2D1B1-1]
	Cho hàm số $y=x^4-8x^2+7$. Mệnh đề nào dưới đây đúng?
	\choice
	{Hàm số đồng biến trên các khoảng $(-\infty;-2)$ và $(0;2)$}
	{Hàm số nghịch biến trên các khoảng $(-\infty;-1)$ và $(0;2)$}
	{\True Hàm số đồng biến trên các khoảng $(-2;0)$ và $(2;+\infty)$}
	{Hàm số nghịch biến trên các khoảng $(-2;0)$ và $(2;+\infty)$}
	\loigiai{
		Tập xác định $\mathscr{D}=\mathbb{R}$. Ta có $y'=4x^3-16x$, $y'=0\Leftrightarrow \hoac{& x=0 \\ & x=2\\ &x=-2.}$\\
		Bảng biến thiên
		\begin{center}
			\begin{tikzpicture}
			\tkzTabInit[lgt=1.2,espcl=2.5]
			{$x$/1,$y'$/1,$y$/2}
			{$-\infty$,$-2$,$0$,$2$,$+\infty$}
			\tkzTabLine{ ,-,0,+,0,-,0,+, }
			\tkzTabVar{+/$+\infty$,-/$-9$,+/$0$,-/$-9$,+/$+\infty$}
			\begin{scope}[on background layer]\path[white]node{MDD-108};\end{scope}
\end{tikzpicture}
		\end{center}
		Dựa vào bảng biến thiên, hàm số đồng biến trên các khoảng $(-2;0)$ và $(2;+\infty)$
	}
\end{ex}


\begin{ex}%[GHKI, THPT Nguyễn Hiền, Quảng Nam, 2020]%[Nguyễn Tiến, dự án EX3]%[2D1K5-4]
	\immini{
		Cho hàm số $y=f(x)$ có đồ thị như hình vẽ bên. Có bao nhiêu giá trị nguyên của tham số $m$ để phương trình $\dfrac{1}{3}f\left(\dfrac{x}{2}+1\right)+x=m$ có nghiệm thuộc đoạn $[-2;2]$?
		\choice
		{$6$}
		{\True $4$}
		{$5$}
		{$7$}
	}{
		\begin{tikzpicture}[scale=0.7, font=\footnotesize, line join=round, line cap=round, >=stealth]
		\def\a{-0.25} \def\b{0.75} \def\c{0} \def\d{0} % Hệ số
		\def\xt{-2} \def\xp{4} \def\yt{2.5} \def\yd{-2} % x_trái, x_phải, y_trên, y_dưới (giới hạn)
		\draw[->] (\xt,0)--(\xp,0) node [below]{$x$};
		\draw[->] (0,\yd)--(0,\yt) node [left]{$y$};
		\node at (0,0) [below left]{$O$};
		\fill (0,1) circle (1.5pt) node[left]{$1$} (2,1) circle (1.5pt) (3,0) circle (1.5pt) node[above right]{$3$};
		\foreach \x in {1,2}
		\fill (\x,0) circle (1.5pt) node[below]{$\x$};
		\draw[->] (0,\yd)--(0,\yt) node [left]{$y$};
		\clip (\xt-0.1,\yd+0.1) rectangle (\xp-0.1,\yt-0.1);
		\draw[smooth,samples=300] plot(\x,{\a*(\x)^3+\b*(\x)^2+\c*(\x)+\d});
		\draw[dashed] (2,0)--(2,1)--(0,1);
		\begin{scope}[on background layer]\path[white]node{MDD-108};\end{scope}
\end{tikzpicture}
	}
	\loigiai{
		Đặt $t=\dfrac{x}{2}+1$ suy ra $x=2t-1$, $\forall x\in[-2;2] \Rightarrow t\in[0;2]$.\\
		Ta có phương trình $\dfrac{1}{3}f(t)+2t-1=m$ là phương trình hoành độ giao điểm của hai đồ thị hàm số $(C)\colon y=\dfrac{1}{3}f(t)+2t-1$ và $d\colon y=m$.\\
		Xét hàm số $y=\dfrac{1}{3}f(t)+2t-1$ trên đoạn $[0;2]$ (do $f'(t)\geq 0$, $\forall t\in[0;2]$).\\
		Ta có $y'=\dfrac{1}{3}\cdot f'(t)+2>0$, $\forall t\in[0;2]$.\\
		Bảng biến thiên
		\begin{center}
			\begin{tikzpicture}
			\tkzTabInit[nocadre=false,lgt=1.2,espcl=2.5,deltacl=0.6]
			{$x$ /0.6,$y'$ /0.6,$y$ /2}
			{$0$,$2$}
			\tkzTabLine{,+,}
			\tkzTabVar{-/$y(0)$,+/$y(2)$}
			\begin{scope}[on background layer]\path[white]node{MDD-108};\end{scope}
\end{tikzpicture}
		\end{center}
		Vậy $y(0)<m<y(2) \Leftrightarrow \dfrac{1}{3}\cdot f(0)-1<m<\dfrac{1}{3}\cdot f(2)+3 \Leftrightarrow -1<m<\dfrac{10}{3}$.\\
		Do $m\in\mathbb{Z}$ nên $m\in\{0;1;2;3\}$.\\
		Vậy có $4$ giá trị nguyên của $m$ thỏa yêu cầu bài toán.
	}
\end{ex}


\begin{ex}%[GHKI, THPT Nguyễn Hiền, Quảng Nam, 2020]%[Nguyễn Tiến, dự án EX3]%[2D1K3-1]
	\immini{
		Cho hàm số $y=f(x)$ có đồ thị hàm số $y=f'(x)$ như hình vẽ bên. Biết rằng $f(-1)+f(0)=f(1)+f(4)$. Giá trị nhỏ nhất và giá trị lớn nhất của hàm số $y=f(x)$ trên đoạn $[-1;4]$ lần lượt là
		\choice
		{\True $f(4)$, $f(1)$}
		{$f(0)$, $f(-1)$}
		{$f(1)$, $f(4)$}
		{$f(1)$, $f(-1)$}
	}{
		\begin{tikzpicture}[scale=0.7, font=\footnotesize, line join=round, line cap=round, >=stealth]
		\def\xt{-2} \def\xp{4.5} \def\yt{2.5} \def\yd{-4} % x_trái, x_phải, y_trên, y_dưới (giới hạn)
		\draw[->] (\xt,0)--(\xp,0) node [below]{$x$};
		\draw[->] (0,\yd)--(0,\yt) node [left]{$y$};
		\node at (0,0) [below left]{$O$};
		\fill (-0.75,0) circle (1.5pt) node[above left]{$-1$} (0.75,0) circle (1.5pt) node[below left]{$1$} (3,0) circle (1.5pt) node[below right]{$4$};
		\draw[->] (0,\yd)--(0,\yt) node [left]{$y$};
		\clip (\xt+0.1,\yd+0.1) rectangle (\xp+1,\yt-0.1);
		\draw[smooth,samples=300,domain=\xt:3.2] plot(\x,{(\x+0.75)*(\x-0.75)*(\x-3)})node[right]{$y=f'(x)$};
		\begin{scope}[on background layer]\path[white]node{MDD-108};\end{scope}
\end{tikzpicture}
	}
	\loigiai{
		Từ hình vẽ ta thấy $f'(x)=0 \Leftrightarrow \hoac{& x=\pm 1\\& x=4.}$\\
		Ta có bảng biến thiên sau
		\begin{center}
			\begin{tikzpicture}
			\tkzTabInit[nocadre=false,lgt=1.2,espcl=2.5,deltacl=0.6]
			{$x$ /0.6,$f'(x)$ /0.6,$f(x)$ /2}
			{$-\infty$,$-1$,$1$,$4$,$+\infty$}
			\tkzTabLine{,-,$0$,+,$0$,-,$0$,+,}
			\tkzTabVar{+/$+\infty$, -/$f(-1)$,+/$f(1)$,-/$f(4)$,+/$+\infty$}
			\begin{scope}[on background layer]\path[white]node{MDD-108};\end{scope}
\end{tikzpicture}
		\end{center}
		Khi đó $\max\limits_{[-1;4]} f(x)=f(1)$ và $\min\limits_{[-1;4]} f(x)=\left\lbrace f(-1); f(4) \right\rbrace$.\\
		Lại có $f(-1)+f(0)=f(1)+f(4) \Leftrightarrow f(-1)-f(4)=f(1)-f(0)>0$ (do $f(x)$ tăng trên $(0;1)$).\\
		Do đó $f(-1)>f(4)$, suy ra $\min\limits_{[-1;4]} f(x)=f(4)$.
	}
\end{ex}


\begin{ex}%[GHKI, THPT Nguyễn Hiền, Quảng Nam, 2020]%[Nguyễn Tiến, dự án EX3]%[2H1K3-2]
	Cho lăng trụ $ABC.A'B'C'$ có đáy $ABC$ là tam giác vuông tại $A$, $\widehat{ABC}=30^\circ$. Điểm $M$ là trung điểm cạnh $AB$, tam giác $MA'C$ đều cạnh $2a\sqrt{3}$ và nằm trong mặt phẳng vuông góc với đáy. Thể tích khối lăng trụ $ABC.A'B'C'$ bằng
	\choice
	{\True $\dfrac{72a^3\sqrt{3}}{7}$}
	{$\dfrac{24a^3\sqrt{2}}{7}$}
	{$\dfrac{24a^3\sqrt{3}}{7}$}
	{$\dfrac{72a^3\sqrt{2}}{7}$}
	\loigiai{
		\immini{
			Ta có $\heva{& (MA'C)\cap (ABC)=MC\\& (MA'C)\perp (ABC).}$\\
			Gọi $H$ là trung điểm của $MC$.\\
			Xét $\triangle MA'C$ đều có $A'H$ vừa là đường trung tuyến, vừa là đường cao nên $A'H\perp MC$.\\
			Suy ra $A'H\perp (ABC)$, hay $A'H$ là chiều cao khối lăng trụ đã cho.\\
			Ta có $A'H=\dfrac{2a\sqrt{3}\cdot \sqrt{3}}{2}=3a$.\\
			Xét $\triangle ABC$ có $\tan 30^\circ=\dfrac{AC}{AB} \Rightarrow AC=\dfrac{AB}{\sqrt{3}}$.\\
			Xét $\triangle MAC$ có
		}{
			\begin{tikzpicture}[scale=1, font=\footnotesize, line join=round, line cap=round, >=stealth]
			\tkzDefPoints{0/0/A,1.2/-1.5/B,3.5/0/C}
			\coordinate (M) at ($(A)!1/2!(B)$);
			\coordinate (H) at ($(M)!1/2!(C)$);
			\coordinate (A') at ($(H)+(0,4)$);
			\tkzDefPointsBy[translation = from A to A'](B,C){B'}{C'}
			\tkzDrawPolygon(A,B,C,C',B',A')
			\tkzDrawSegments(A',C' B',B A',M)
			\tkzDrawSegments[dashed](A,C A',H M,C A',C)
			\tkzDrawPoints[fill=black,size=4](A,C,B,A',B',C',M,H)
			\tkzLabelPoints[below right](B')
			\tkzLabelPoints[below](B,H)
			\tkzLabelPoints[left](A',A,M)
			\tkzLabelPoints[right](C',C)
			\begin{scope}[on background layer]\path[white]node{MDD-108};\end{scope}
\end{tikzpicture}
		}
		\noindent
		\allowdisplaybreaks
		\begin{eqnarray*}
			& & AM^2+AC^2=MC^2 \Leftrightarrow \dfrac{AB^2}{4}+\dfrac{AB^2}{3}=12a^2\\
			&\Leftrightarrow & \dfrac{7AB^2}{12}=12a^2 \Leftrightarrow AB=\dfrac{12a}{\sqrt{7}}.
		\end{eqnarray*}
		Nên $AC=\dfrac{AB}{\sqrt{3}}=\dfrac{4a\sqrt{3}}{\sqrt{7}} \Rightarrow S_{ABC}=\dfrac{1}{2}AB\cdot AC=\dfrac{24a^2\sqrt{3}}{7}$.\\
		Vậy $V_{ABC.A'B'C'}=A'H\cdot S_{ABC}=3a\cdot \dfrac{24a^2\sqrt{3}}{7}=\dfrac{72a^3\sqrt{3}}{7}$.
	}
\end{ex}


\begin{ex}%[GHKI, THPT Nguyễn Hiền, Quảng Nam, 2020]%[Nguyễn Tiến, dự án EX3]%[2D1K1-2]
	\immini{
		Cho hàm số $y=f(x)$. Đồ thị hàm số $y=f'(x)$ như hình vẽ bên. Hàm số $g(x)=f(3-2x)$ nghịch biến trên khoảng nào dưới đây?
		\choice
		{$(-1;+\infty)$}
		{$(1;3)$}
		{$(0;2)$}
		{\True $(-\infty;-1)$}
	}{
		\begin{tikzpicture}[scale=0.7, font=\footnotesize, line join=round, line cap=round, >=stealth]
		\def\a{1} \def\b{-2.5} \def\c{-1} \def\d{2.5} % Hệ số
		\def\xt{-2} \def\xp{4} \def\yt{3.5} \def\yd{-2.5} % x_trái, x_phải, y_trên, y_dưới (giới hạn)
		\draw[->] (\xt,0)--(\xp,0) node [below]{$x$};
		\draw[->] (0,\yd)--(0,\yt) node [left]{$y$};
		\node at (0,0) [below left]{$O$};
		\fill (-1,0) circle (1.5pt) node[above left]{$-2$} (1,0) circle (1.5pt) node[above right]{$2$} (2.5,0) circle (1.5pt) node[above left]{$5$};
		\clip (\xt-0.1,\yd+0.1) rectangle (\xp-0.1,\yt-0.1);
		\draw[smooth,samples=300] plot(\x,{\a*(\x)^3+\b*(\x)^2+\c*(\x)+\d});
		\begin{scope}[on background layer]\path[white]node{MDD-108};\end{scope}
\end{tikzpicture}
	}
	\loigiai{
		Ta có $g'(x)=-2\cdot f'(3-2x)$.\\
		$g'(x)<0 \Leftrightarrow f'(3-2x)>0 \Leftrightarrow \hoac{& -2<3-2x<2\\& 3-2x>5} \Leftrightarrow \hoac{& \dfrac{1}{2}<x<\dfrac{5}{2}\\& x<-1.}$\\
		Vậy hàm số $g(x)$ nghịch biến trên $\left(\dfrac{1}{2};\dfrac{5}{2}\right)$ và $(-\infty;-1)$.\\
		Từ các phương án, ta chọn hàm số $g(x)$ nghịch biến trên khoảng $(-\infty;-1)$.
	}
\end{ex}


\begin{ex}%[GHKI, THPT Nguyễn Hiền, Quảng Nam, 2020]%[Nguyễn Tiến, dự án EX3]%[2D1K2-2]
	\immini{
		Cho hàm số $y=f(x)$ có đạo hàm trên $\mathbb{R}$ và có đồ thị như hình vẽ bên. Hàm số $g(x)=f(-x^2+3x)$ có bao nhiêu điểm cực đại?
		\choice
		{$5$}
		{$4$}
		{$6$}
		{\True $3$}
	}{
		\begin{tikzpicture}[scale=0.7, font=\footnotesize, line join=round, line cap=round, >=stealth]
		\def\a{1} \def\b{3} \def\c{0} \def\d{-2} % Hệ số
		\def\xt{-4} \def\xp{2} \def\yt{3} \def\yd{-3} % x_trái, x_phải, y_trên, y_dưới (giới hạn)
		\draw[->] (\xt,0)--(\xp,0) node [below]{$x$};
		\draw[->] (0,\yd)--(0,\yt) node [left]{$y$};
		\node at (0,0) [below left]{$O$}; \node at (-2,0) [below]{$-2$}; \node at (0,-2) [below right]{$-2$}; \node at (0,2) [right]{$2$};
		\fill (-2,2) circle (1.5pt) (0,-2) circle (1.5pt);
		\draw[->] (0,\yd)--(0,\yt) node [left]{$y$};
		\clip (\xt-0.1,\yd+0.1) rectangle (\xp-0.1,\yt-0.1);
		\draw[smooth,samples=300] plot(\x,{\a*(\x)^3+\b*(\x)^2+\c*(\x)+\d});
		\draw[dashed] (-2,0)--(-2,2)--(0,2);
		\begin{scope}[on background layer]\path[white]node{MDD-108};\end{scope}
\end{tikzpicture}
	}
	\loigiai{
		Ta có $g'(x)=(-2x+3)\cdot f'(-x^2+3x)$.\\
		$g'(x)=0 \Leftrightarrow \hoac{& -2x+3=0\\& f'(-x^2+3x)=0} \Leftrightarrow \hoac{& -2x+3=0\\& -x^2+3x=-2\\& -x^2+3x=0} \Leftrightarrow \hoac{& x=\dfrac{3}{2}\\& x=1 \vee x=2\\& x=0 \vee x=3.}$\\
		Bảng biến thiên
		\begin{center}
			\begin{tikzpicture}
			\tkzTabInit[nocadre=false,lgt=1.2,espcl=2.5,deltacl=0.6]
			{$x$ /1,$g'(x)$ /0.6,$g(x)$ /2}
			{$-\infty$,$0$,$1$,$\dfrac{3}{2}$,$2$,$3$,$+\infty$}
			\tkzTabLine{,+,$0$,-,$0$,+,$0$,-,$0$,+,$0$,-,}
			\tkzTabVar{-/,+/,-/,+/,-/,+/,-/}
			\begin{scope}[on background layer]\path[white]node{MDD-108};\end{scope}
\end{tikzpicture}
		\end{center}
		Từ bảng biến thiên, ta thấy hàm số $g(x)$ có $3$ điểm cực đại.
	}
\end{ex}


\begin{ex}%[GHKI, THPT Nguyễn Hiền, Quảng Nam, 2020]%[Nguyễn Tiến, dự án EX3]%[2H1K3-3]
	Cho khối chóp tứ giác đều $S.ABCD$. Gọi $M$ là trung điểm của cạnh $SC$, mặt phẳng $(P)$ chứa $AM$ và song song với $BD$ chia khối chóp thành hai khối đa diện, đặt $V_1$ là thể tích khối đa diện có chứa đỉnh $S$ và $V_2$ là thể tích khối đa diện có chứa đáy $ABCD$. Tính $\dfrac{V_1}{V_2}$.
	\choice
	{\True $\dfrac{V_1}{V_2}=\dfrac{1}{2}$}
	{$\dfrac{V_1}{V_2}=\dfrac{1}{3}$}
	{$\dfrac{V_1}{V_2}=\dfrac{2}{3}$}
	{$\dfrac{V_1}{V_2}=1$}
	\loigiai{
		\immini{
			Gọi $I=AM\cap SO \Rightarrow I$ là trọng tâm $\triangle SAC$.\\
			(do $AM$, $SO$ là hai đường trung tuyến của $\triangle SAC$).\\
			Qua $I$ kẻ $PQ\parallel BD$ và cắt $SB$ tại $P$, cắt $SD$ tại $Q$.\\
			$\Rightarrow \dfrac{SP}{SB}=\dfrac{SQ}{SD}=\dfrac{2}{3}$.\\
			Khi đó $(P)\equiv (APMQ)$ $\Rightarrow V_1=V_{S.APMQ}$.\\
			Ta có $\dfrac{V_{S.APM}}{V_{S.ABC}}=\dfrac{SA}{SA}\cdot \dfrac{SP}{SB}\cdot \dfrac{SM}{SC}=\dfrac{1}{3}$.\\
			$\Rightarrow V_{S.APM}=\dfrac{1}{3}V_{S.ABC}=\dfrac{1}{6}V_{S.ABCD}$.\\
			Tương tự, $V_{S.AQM}=\dfrac{1}{3}V_{S.ADC}=\dfrac{1}{6}V_{S.ABCD}$.\\
			Suy ra $V_1=V_{S.APMQ}=V_{S.APM}+V_{S.AQM}=\dfrac{1}{3}V_{S.ABCD}$.\\
			$\Rightarrow V_2=V_{S.ABCD}-V_1=\dfrac{2}{3}V_{S.ABCD}$. Vậy $\dfrac{V_1}{V_2}=\dfrac{1}{2}$.
		}{
			\begin{tikzpicture}[scale=0.8, font=\footnotesize, line join=round, line cap=round, >=stealth]
			\tkzDefPoints{0/0/A,-2/-1.6/B,2.6/-1.6/C}
			\coordinate (D) at ($(A)+(C)-(B)$);
			\coordinate (O) at ($(A)!1/2!(C)$);
			\coordinate (S) at ($(O)+(0,5)$);
			\coordinate (M) at ($(S)!1/2!(C)$);
			\tkzInterLL(A,M)(S,O)\tkzGetPoint{I}
			\coordinate (P) at ($(S)!2/3!(B)$);
			\coordinate (Q) at ($(S)!2/3!(D)$);
			\tkzDrawPolygon(S,B,C,D)
			\tkzDrawSegments(S,C P,M M,Q)
			\tkzDrawSegments[dashed](A,S A,B A,D A,C B,D S,O A,P A,Q P,Q A,M)
			\tkzDrawPoints[fill=black,size=4](D,C,A,B,S,M,I,P,Q,O)
			\tkzLabelPoints[above](S)
			\tkzLabelPoints[below](A,B,C,O)
			\tkzLabelPoints[right](D,Q)
			\tkzLabelPoints[above right](M)
			\tkzLabelPoints[left](P)
			\tkzLabelPoints[below right](I)
			\begin{scope}[on background layer]\path[white]node{MDD-108};\end{scope}
\end{tikzpicture}
		}
	}
\end{ex}


\begin{ex}%[GHKI, THPT Nguyễn Hiền, Quảng Nam, 2020]%[Nguyễn Tiến, dự án EX3]%[2D1K2-1]
	Cho hàm số $y=f(x)$ có biểu thức đạo hàm $f'(x)=(x+1)(x-1)^2 (x-2)+1$ với mọi $x\in\mathbb{R}$. Hàm số $g(x)=f(x)-x$ có bao nhiêu điểm cực trị?
	\choice
	{$4$}
	{\True $2$}
	{$1$}
	{$3$}
	\loigiai{
		Ta có $g'(x)=f'(x)-1=(x+1)(x-1)^2 (x-2)$; $g'(x)=0 \Leftrightarrow \hoac{& x=-1\\& x=1 \text{ (nghiệm kép)}\\& x=2.}$\\
		Bảng biến thiên
		\begin{center}
			\begin{tikzpicture}
			\tkzTabInit[nocadre=false,lgt=1.2,espcl=2.5,deltacl=0.6]
			{$x$ /0.6,$g'(x)$ /0.6,$g(x)$ /2}
			{$-\infty$,$-1$,$1$,$2$,$+\infty$}
			\tkzTabLine{,+,$0$,-,$0$,-,$0$,+,}
			\tkzTabVar{-/$-\infty$, +/,R,-/,+/$+\infty$}
			\begin{scope}[on background layer]\path[white]node{MDD-108};\end{scope}
\end{tikzpicture}
		\end{center}
		Từ bảng biến thiên, ta thấy hàm số $g(x)$ có hai điểm cực trị.
	}
\end{ex}


\begin{ex}%[GHKI, THPT Nguyễn Hiền, Quảng Nam, 2020]%[Nguyễn Tiến, dự án EX3]%[2H1G3-6]
	Cho hình chóp $S.ABCD$ có đáy $ABCD$ là hình vuông, $AB=1$, cạnh bên $SA=1$ và vuông góc với mặt phẳng đáy $(ABCD)$. Kí hiệu $M$ là điểm di động trên đoạn $CD$ và $N$ là điểm di động trên $CB$ sao cho $\widehat{MAN}=60^\circ$. Thể tích nhỏ nhất của khối chóp $S.AMN$ bằng
	\choice
	{$\dfrac{2+\sqrt{3}}{3}$}
	{$\dfrac{2-\sqrt{3}}{3}$}
	{\True $\dfrac{2\sqrt{3}-3}{3}$}
	{$\dfrac{2\sqrt{3}-3}{9}$}
	\loigiai{
		\immini{
			Đặt $BN=x$, $DM=y$ (với $x$, $y\in (0;1)$).\\
			Ta có $\widehat{MAN}=60^\circ$ nên $\widehat{BAN}+\widehat{MAD}=30^\circ$.\\
			Suy ra $\tan\left(\widehat{BAN}+\widehat{MAD}\right)=\tan 30^\circ$
			\allowdisplaybreaks
			\begin{eqnarray*}
				&\Leftrightarrow & \dfrac{\tan\widehat{BAN}+\tan\widehat{MAD}}{1-\tan\widehat{BAN}\cdot\tan\widehat{MAD}}=\dfrac{1}{\sqrt{3}}\\
				&\Leftrightarrow & \dfrac{x+y}{1-xy}=\dfrac{1}{\sqrt{3}} \Leftrightarrow y=\dfrac{1-\sqrt{3}x}{\sqrt{3}+x}.
			\end{eqnarray*}
			Xét $\triangle ABN$ có $AN=\sqrt{1+x^2}$.
		}{
			\begin{tikzpicture}[scale=1, font=\footnotesize, line join=round, line cap=round, >=stealth]
			\tkzDefPoints{0/0/A,-1.3/-1.6/B,2.5/-1.6/C}
			\coordinate (D) at ($(A)+(C)-(B)$);
			\coordinate (S) at ($(A)+(0,3)$);
			\coordinate (N) at ($(C)!2/5!(B)$);
			\coordinate (M) at ($(C)!3/5!(D)$);
			\tkzDrawPolygon(S,B,C,D)
			\tkzDrawSegments(S,C S,N S,M)
			\tkzDrawSegments[dashed](A,S A,B A,D M,N A,M A,N)
			\tkzDrawPoints[fill=black,size=4](D,C,A,B,S,M,N)
			\tkzLabelPoints[above](S)
			\tkzLabelPoints[left](A)
			\tkzLabelPoints[below](B,C,N)
			\tkzLabelPoints[right](D,M)
			\begin{scope}[on background layer]\path[white]node{MDD-108};\end{scope}
\end{tikzpicture}
		}
		\noindent
		Xét $\triangle MAD$ có $AM=\sqrt{1+y^2}=\sqrt{1+\left(\dfrac{1-\sqrt{3}x}{\sqrt{3}+x}\right)^2}=\dfrac{2\sqrt{x^2+1}}{\sqrt{3}+x}$.\\
		Khi đó
		\allowdisplaybreaks
		\begin{eqnarray*}
			V_{S.AMN}&= & \dfrac{1}{3}\cdot SA\cdot S_{AMN} = \dfrac{1}{3}\cdot SA\cdot\dfrac{1}{2}\cdot AN\cdot AM\cdot \sin\widehat{MAN}\\
			&= & \dfrac{\sqrt{3}}{12}\cdot \sqrt{1+x^2}\cdot \dfrac{2\sqrt{x^2+1}}{\sqrt{3}+x}=\dfrac{\sqrt{3}}{6}\cdot \dfrac{x^2+1}{\sqrt{3}+x}.
		\end{eqnarray*}
		Để thể tích khối chóp $S.AMN$ nhỏ nhất thì $f(x)=\dfrac{x^2+1}{\sqrt{3}+x}$ nhỏ nhất trên khoảng $(0;1)$.\\
		Ta có $f'(x)=\dfrac{2x(\sqrt{3}+x)-x^2-1}{(\sqrt{3}+x)^2}=\dfrac{x^2+2x\sqrt{3}-1}{(\sqrt{3}+x)^2}$; $y'=0 \Leftrightarrow \hoac{& x=2-\sqrt{3} \text{ (nhận)}\\& x=-2-\sqrt{3} \text{ (loại).}}$\\
		Bảng biến thiên
		\begin{center}
			\begin{tikzpicture}
			\tkzTabInit[nocadre=false,lgt=1.2,espcl=2.5,deltacl=0.6]
			{$x$ /0.6,$f'(x)$ /0.6,$f(x)$ /2}
			{$0$,$2-\sqrt{3}$,$1$}
			\tkzTabLine{,-,$0$,+,}
			\tkzTabVar{+/,-/,+/}
			\begin{scope}[on background layer]\path[white]node{MDD-108};\end{scope}
\end{tikzpicture}
		\end{center}
		Vậy $f(x)$ nhỏ nhất khi $x=2-\sqrt{3}$.\\
		Suy ra thể tích khối chóp $S.AMN$ nhỏ nhất là $V_{S.AMN}=\dfrac{\sqrt{3}}{6}\cdot \dfrac{x^2+1}{\sqrt{3}+x}=\dfrac{2\sqrt{3}-3}{3}$.
	}
\end{ex}

\Closesolutionfile{ans}
\begin{indapan}{10}
	{ans/ans-2-GHK1-31-NguyenHien-QuangNam-21}
\end{indapan}

