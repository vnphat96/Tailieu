
\begin{name}
	{\tenchude}
	{\tendethi}
	{\tentruong}
	{\thoigian}
\end{name}
\Opensolutionfile{ans}[ans/ans-de11-7]

\begin{ex}%Câu 50.
Cho hai số phức $z_1=1-2 i$ và $z_2=2+i$. Số phức $z_1+z_2$ bằng
\choice
{$3+i$}
{$-3-i$}
{$3-i$}
{\True $-3+i$}

\end{ex}
\begin{ex}%Câu 1.
\immini{
Cho hàm số $y= f(x)$ có bảng biến thiên như hình bên. Giá trị cực đại của hàm số đã cho bằng
\choice
{$2$}
{$+\infty$}
{\True $11$}
{$1$}}
{\vspace{-0.5cm}
\begin{tikzpicture}[color=\mauchinh]
\tkzTabInit[nocadre,lgt=1.2,espcl=1.6,deltacl=0.5]
{$x$/0.6,$f'(x)$/0.6,$f(x)$/2}{$-\infty$,$-1$,$2$,$+\infty$}
\tkzTabLine{,+,0,-,0,+,}
\tkzTabVar{-/$-\infty$,+/$11$,-/$4$,+/$+\infty$}
\end{tikzpicture}

}

\end{ex}
\begin{ex}%Câu 2.
Cho $\displaystyle\int\limits_2^6 f(x)\mathrm{\,d}x=4$ và $\displaystyle\int\limits_2^6 g(x)\mathrm{\,d}x=5$, khi đó $\displaystyle\int\limits_2^6[3 f(x)-g(x)]\mathrm{\,d}x$ 
bằng
\choice
{$19$}
{$17$}
{$11$}
{\True $7$}

\end{ex}
\begin{ex}%Câu 3.
Đường tiệm cận ngang của đồ thị hàm số $y=\dfrac{3 x-1}{x-2}$ là
\choice
{$y=\dfrac{1}{3}$}
{\True $y=3$}
{$y=-3$}
{$y=2$}

\end{ex}
\begin{ex}%Câu 4.
\immini{
Đường cong trong hình bên là đồ thị của hàm số nào dưới đây?
\choice
{$y=x^4+2 x^2+1$}
{$y=-x^4+1$}
{$y=x^4+1$}
{\True $y=-x^4+2 x^2+1$}}
{
\vspace{-0.5cm}
\begin{tikzpicture}[scale=1, font=\footnotesize, line join=round, line cap=round, >=stealth,color=\mauchinh,y=0.8cm]
\def\xmin{-1.65}\def\xmax{1.65}\def\ymin{-1}\def\ymax{2.2}
\draw[->,thick] (\xmin-0.2,0)--(\xmax+0.2,0) node[below] {\footnotesize $x$};
\draw[->,thick] (0,\ymin-0.2)--(0,\ymax+0.2) node[right] {\footnotesize $y$};
\draw (0,0) node [below left] {\footnotesize $O$};
\foreach \x in {}\draw (\x,0.1)--(\x,-0.1) node [below] {\footnotesize $\x$};
\foreach \y in {}\draw (0.1,\y)--(-0.1,\y) node [left] {\footnotesize $\y$};
\clip (\xmin,\ymin) rectangle (\xmax,\ymax);
\draw[thick,smooth,samples=200,domain=\xmin:\xmax] plot (\x,{-1*((\x)^4)+2*((\x)^2)+1});
\end{tikzpicture}
}

\end{ex}
\begin{ex}%Câu 5.
Khối lăng trụ có đáy là hình chữ nhật có hai kích thước lần lượt là $2 a, 3 a$. Chiều cao của khối trụ là $5 a$. Thể tích của khối trụ bằng
\choice
{\True $30 a^3$}
{$10 a^3$}
{$30 a^2$}
{$10 a^2$}

\end{ex}

\begin{ex}%Câu 6.
Tập nghiệm $S$ của bất phương trình $\left(\dfrac{1}{2}\right)^{x}<32$ là:
\choice
{$S=(-\infty; 5)$}
{\True $S=(-5;+\infty)$}
{$S=(5;+\infty)$}
{$S=(-\infty;-5)$}

\end{ex}
\begin{ex}%Câu 7.
Một mặt cầu có bán kính bằng $a$. Diện tích mặt cầu đó bằng
\choice
{$\dfrac{4\pi a^3}{3}$}
{\True $4\pi a^2$}
{$\dfrac{1}{3} a^3$}
{$a^2$}

\end{ex}
\begin{ex}%Câu 8.
Cho cấp số cộng $\left(u_n\right)$ với $u_2=3$ và $u_3=5$. Số hạng đầu của cấp số cộng bằng
\choice
{\True $1$}
{$\dfrac{3}{2}$}
{$2$}
{$7$}

\end{ex}
\begin{ex}%Câu 9.
Một hình trụ có chiều cao $h$ và bán kính $r$. Thể tích khối trụ đó bằng
\choice
{\True $\pi r^2 h$}
{$\dfrac{1}{3} \pi r^2 h$}
{$2\pi r^2 h$}
{$\dfrac{4}{3} \pi r^2 h$}

\end{ex}
\begin{ex}%Câu 10.
Cho số phức $z_1=1+i, z_2=2-3 i$. Phần ảo của số phức $w=z_1+z_2$ là
\choice
{\True $-2$}
{$-3$}
{$2$}
{$3$}

\end{ex}
\begin{ex}%Câu 11.
Cho hình chóp đều $S.ABCD$ có $AB=2 a, SA=2 a \sqrt{2}$. Góc giữa $SB$ và $(ABCD)$ bằng
\choice
{$30^{\circ}$}
{$75^{\circ}$}
{\True $60^{\circ}$}
{$45^{\circ}$}

\end{ex}
\begin{ex}%Câu 12.
Từ một tổ có $10$ học sinh, có bao nhiêu cách chọn ra hai học sinh?
\choice
{$A_{10}^2$}
{\True $C_{10}^2$}
{$20$}
{$2!$}

\end{ex}
\begin{ex}%Câu 13.
Một hình trụ có độ dài đường sinh bằng $l$ và bán kính đường tròn đáy bằng $R$. Diện tích toàn phần của hình trụ đó bằng
\choice
{$\pi R(R+l)$}
{\True $2\pi R(l+R)$}
{$\pi R l$}
{$4\pi R l$}

\end{ex}
\begin{ex}%Câu 14.
Nếu $\displaystyle\int\limits_1^2 f(x) \mathrm{d} x=3$ thì $\displaystyle\int\limits_1^2 2 f(x) \mathrm{d} x$ bằng
\choice
{$8$}
{\True $6$}
{$3$}
{$4$}

\end{ex}
\begin{ex}%Câu 15.
Hàm số $f(x)$ xác định và liên tục trên $\mathbb{R}$, có bảng biến thiên 
\immini{
như hình bên. Phương trình $f(x)=-1$ có tất cả bao nhiêu nghiệm thực?
\choice
{\True $2$}
{$4$}
{$3$}
{$1$}}
{
\begin{tikzpicture}[color=\mauchinh]
\tkzTabInit[nocadre,lgt=1.2,espcl=1.6,deltacl=0.6]
{$x$/0.6,$f'(x)$/0.6,$f(x)$/1.8}{$-\infty$,$-1$,$1$,$+\infty$}
\tkzTabLine{,+,0,-,0,+,}
\tkzTabVar{-/$-\infty$,+/$2$,-/$-2$,+/$+\infty$}
\end{tikzpicture}

}

\end{ex}
\begin{ex}%Câu 16.
Một khối chóp có diện tích đáy bằng $B$ và chiều cao bằng $h$. Thể tích khối chóp bằng
\choice
{$\dfrac{4}{3} B h$}
{$B h$}
{\True $\dfrac{1}{3} B h$}
{$3 B h$}

\end{ex}
\begin{ex}%Câu 17.
Trong không gian $O x y z$, cho mặt phẳng $(P)\colon x+4 y+3 z-2=0$. Vectơ nào sau đây là vectơ pháp tuyến của mặt phẳng $(P)$?
\choice
{\True $\overrightarrow{n_2}=(1; 4; 3)$}
{$\overrightarrow{n_3}=(-1; 4;-3)$}
{$\overrightarrow{n_4}=(-4; 3;-2)$}
{$\vec{n}_1=(0;-4; 3)$}

\end{ex}
\begin{ex}%Câu 18.
Số phức liên hợp của số phức $z=2-5 i$ là
\choice
{$\bar{z}=-2+5 i$}
{$\bar{z}=2-5 i$}
{$\bar{z}=-2-5 i$}
{\True $\bar{z}=2+5 i$}

\end{ex}
\begin{ex}%Câu 19.
Số giao điểm của đồ thị hàm số $y=x^3-3 x^2-6 x+1$ và trục hoành là
\choice
{$0$}
{\True $3$}
{$2$}
{$1$}

\end{ex}
\begin{ex}%Câu 20.
Cho hàm số $y=f(x)$ liên tục trên $\mathbb{R}$ và có bảng xét dấu đạo 
\immini{
hàm như hình vẽ bên dưới: Đồ thị hàm số $y=f(x)$ có tất cả bao nhiêu điểm cực trị?
}
{
\begin{tikzpicture}[scale=1,line width=.6pt,color=\mauchinh]
\tkzTabInit[nocadre=true,lgt=1,espcl=1.2,deltacl=0.5,lw=0.8]
{$x$ /.7,$y'$/.7}{$-\infty$,$-1$,$0$,$2$,$4$,$+\infty$}
\tkzTabLine{,+,0,-,d,+,0,-,0,+,}
\end{tikzpicture}
}
\choice
{\True $4$}
{$3$}
{$2$}
{$1$}
\end{ex}
\begin{ex}%Câu 21.
Trong không gian $O x y z$, điểm $M(3; 4;-2)$ thuộc mặt phẳng nào dưới dưới đây?
\choice
{$(S)\colon x+y+z+5=0$}
{$(Q)\colon x-1=0$}
{$(P)\colon z-2=0$}
{\True $(R)\colon x+y-7=0$}

\end{ex}
\begin{ex}%Câu 22.
Trong không gian $O x y z$, cho mặt cầu $(S)\colon x^2+y^2+z^2-2 x+  2 y-4 z-2=0$. Diện tích mặt cầu $(S)$ bằng
\choice
{$8\pi$}
{\True $32\pi$}
{$64\pi$}
{$16\pi$}

\end{ex}
\begin{ex}%Câu 23.
Nghiệm của phương trình $\left(\dfrac{2}{5}\right)^{5 x-4}=\left(\dfrac{2}{5}\right)^{x}$ là
\choice
{\True $x=1$}
{$x=-1$}
{$x=\dfrac{2}{3}$}
{$x=\dfrac{4}{3}$}

\end{ex}
\begin{ex}%Câu 24.
Giá trị lớn nhất của hàm số $f(x)=x^4-8 x^2+10$ trên đoạn $[-1; 3]$ bằng
\choice
{\True $19$}
{$3$}
{$13$}
{$-6$}

\end{ex}
\begin{ex}%Câu 25.
Cho $\log_5 2=a, \log_5 3=b$. Khi đó giá trị của $\log_5\left(\dfrac{4}{27}\right)$ bằng
\choice
{\True $2 a-3 b$}
{$3 a-4 b$}
{$3 a+3 b$}
{$2 a+3 b$}

\end{ex}
\begin{ex}%Câu 26.
Tập xác định của hàm số $y=x^{\frac{1}{3}}$ là
\choice
{$\mathbb{R}$}
{$[0;+\infty)$}
{\True $(0;+\infty)$}
{$\mathbb{R} \backslash\{0\}$}

\end{ex}
\begin{ex}%Câu 27.
Số phức $z=2+3 i$ có điểm biểu diễn trên mặt phẳng tọa độ là
\choice
{$Q(2;-3)$}
{\True $N(2; 3)$}
{$M(-2; 3)$}
{$P(-2;-3)$}

\end{ex}
\begin{ex}%Câu 28.
Cho số thực dương $a$ tùy ý, $\log (4 a)-\log (7 a)$ bằng
\choice
{\True $\log 4-\log 7$}
{$-\log (3 a)$}
{$\dfrac{\log (4 a)}{\log (7 a)}$}
{$\dfrac{\log 4}{\log 7}$}

\end{ex}
\begin{ex}%Câu 29.
Bất phương trình $\log_{0,5}(2 x-1)>-2$ có tập nghiệm là
\choice
{$\left(-\infty; \dfrac{5}{2}\right)$}
{$\left[\dfrac{1}{2}; \dfrac{5}{2}\right)$}
{\True $\left(\dfrac{1}{2}; \dfrac{5}{2}\right)$}
{$\left(\dfrac{5}{2};+\infty\right)$}

\end{ex}
\begin{ex}%Câu 30.
Diện tích hình phẳng giới hạn bởi các đường $y=7-2 x^2, y=x^2+4$ bằng
\choice
{$5$}
{$3$}
{\True $4$}
{$\dfrac{5}{2}$}

\end{ex}
\begin{ex}%Câu 31.
Cho số phức $z=-\dfrac{1}{2}+\dfrac{\sqrt{3}}{2} i$. Số phức $1+z+z^2$ bằng
\choice
{\True $0$}
{$1$}
{$2-i \sqrt{3}$}
{$-\dfrac{1}{2}+\dfrac{\sqrt{3}}{2} i$}

\end{ex}
\begin{ex}%Câu 32.
Trong không gian $O x y z$, mặt cầu $(S)$ có tâm là điểm $A(1; 2;-3)$ và đi qua điểm $B(3;-2;-1)$. Phương trình của mặt cầu $(S)$ là
\choice
{$(x-2)^2+y^2+(z+2)^2=24$}
{\True $(x-1)^2+(y-2)^2+(z+3)^2=24$}
{$(x-1)^2+(y-2)^2+(z+3)^2=6$}
{$(x-2)^2+y^2+(z+2)^2=6$}

\end{ex}
\begin{ex}%Câu 33.
Cho $\displaystyle\int\limits_1^{{\rm e}} x \ln x \mathrm{\,d} x=a \cdot {\rm e}^2+b$, với $a, b$ là các số hữu tỉ. Khi đó $a+b$ bằng
\choice
{$0$}
{\True $\dfrac{1}{2}$}
{$\dfrac{3}{2}$}
{$-\dfrac{1}{2}$}

\end{ex}
\begin{ex}%Câu 34.
Biết rằng $z$ là số phức có môđun nhỏ nhất thỏa mãn $(z-1)(\bar{z}+2 i)$ là số thực. Số phức $z$ là
\choice
{$z=1+\dfrac{1}{2} i$}
{$z=\dfrac{3}{5}+\dfrac{4}{5} i$}
{$2 i$}
{\True $z=\dfrac{4}{5}+\dfrac{2}{5} i$}

\end{ex}
\begin{ex}%Câu 35.
Cho hình trụ có bán kính đáy $r=5$ và độ dài đường sinh $l=3$. Diện tích xung quanh của hình trụ đã cho bằng
\choice
{$15\pi$}
{$25\pi$}
{\True $30\pi$}
{$75\pi$}

\end{ex}
\begin{ex}%Câu 36.
Cho khối nón có bán kính $r=2$ chiều cao $h=5$. Thể tích của khối nón đã cho bằng
\choice
{\True $\dfrac{20\pi}{3}$}
{$20\pi$}
{$\dfrac{10\pi}{3}$}
{$10\pi$}

\end{ex}
\begin{ex}%Câu 37.
Biết $\displaystyle\int\limits_1^2 f(x) \mathrm{d} x=2$. Giá trị của $\displaystyle\int\limits_1^3 3 f(x) \mathrm{d} x$ bằng
\choice
{$5$}
{\True $6$}
{$\dfrac{2}{3}$}
{$8$}

\end{ex}
\begin{ex}%Câu 38.
Trong không gian $O x y z$, cho đường thẳng $d\colon \dfrac{x-3}{4}=\dfrac{y+1}{-2}= dfrac{z+2}{3}$. Vecto nào dưới đây là một vecto chỉ phương của $d$ 
\choice
{$\overrightarrow{u_3}=(3;-1;-2)$}
{$\overrightarrow{u_4}=(4; 2; 3)$}
{\True $\overrightarrow{u_2}=(4;-2; 3)$}
{$\overrightarrow{u_1}=(3; 1; 2)$}

\end{ex}
\begin{ex}%Câu 39.
Cho khối cầu có bán kính $r=2$. Thể tích của khối cầu đã cho bằng
\choice
{$16\pi$}
{\True $\dfrac{32\pi}{3}$}
{$32\pi$}
{$\dfrac{8\pi}{3}$}

\end{ex}
\begin{ex}%Câu 40.
Trong không gian $O x y z$, hình chiếu vuông góc của điểm $A(3; 5; 2)$ trên trục $O x$ có tọa độ là
\choice
{$(0; 5; 2)$}
{$(0; 5; 0)$}
{\True $(3; 0; 0)$}
{$(0; 0; 2)$}

\end{ex}
\begin{ex}%Câu 41.
Nghiệm của phương trình $\log_2(x-2)=3$ là:
\choice
{$x=6$}
{$x=8$}
{$x=11$}
{\True $x=10$}

\end{ex}
\begin{ex}%Câu 42.
\immini{
Cho hàm số $f(x)$ có bảng biến thiên như hình bên. Giá trị cực tiểu của hàm số đã cho bằng
\choice
{$2$}
{$-2$}
{$3$}
{\True $-1$}}
{\vspace{-0.5cm}
\begin{tikzpicture}[color=\mauchinh]
\tkzTabInit[nocadre,lgt=1.2,espcl=1.6,deltacl=0.6]
{$x$/0.6,$f'(x)$/0.6,$f(x)$/1.8}{$-\infty$,$-2$,$2$,$+\infty$}
\tkzTabLine{,-,0,+,0,-,}
\tkzTabVar{+/$+\infty$,-/$-1$,+/$3$,-/$-\infty$}
\end{tikzpicture}

}

\end{ex}
\begin{ex}%Câu 43.
Trong không gian $O x y z$, cho $3$ điểm $A(-1; 0; 0), B(0; 2; 0)$ và $C(0; 0; 3)$. Mặt phẳng $(ABC)$ có phương trình là
\choice
{$\dfrac{x}{1}+\dfrac{y}{2}+\dfrac{z}{-3}=1$}
{$\dfrac{x}{1}+\dfrac{y}{-2}+\dfrac{z}{3}=1$}
{\True $\dfrac{x}{-1}+\dfrac{y}{2}+\dfrac{z}{3}=1$}
{$\dfrac{x}{1}+\dfrac{\bar{y}^2}{2}+\dfrac{z}{3}=1$}

\end{ex}
\begin{ex}%Câu 44.
Nghiệm của phương trình $3^{x+1}=9$ là
\choice
{\True $x=1$}
{$x=2$}
{$x=-2$}
{$x=-1$}

\end{ex}
\begin{ex}%Câu 45.
Cho khối hộp chữ nhật có ba kích thước $2; 6; 7$. Thể tích của khối hộp đã cho bằng
\choice
{$28$}
{$14$}
{$15$}
{\True $84$}

\end{ex}
\begin{ex}%Câu 46.
Cho khối chóp có diện tích $B=2$ và chiều cao $h=3$. Thể tích của khốp chóp bằng
\choice
{$12$}
{\True $2$}
{$3$}
{$6$}

\end{ex}
\begin{ex}%Câu 47.
Số phức liên hợp của số phức $z=2-5 i$ là
\choice
{\True $\bar{z}=2+5 i$}
{$\bar{z}=-2+5 i$}
{$\bar{z}=2-5 i$}
{$\bar{z}=-2-5 i$}

\end{ex}
\begin{ex}%Câu 48.
Cho cấp số nhân $\left(u_n\right)$ với $u_1=3$ và công bội $q=4$. Giá trị của $u_2$ bằng
\choice
{$64$}
{$81$}
{\True $12$}
{$\dfrac{3}{4}$}

\end{ex}
\begin{ex}%Câu 49.
\immini{
Cho hàm số bậc ba $y=f(x)$ có đồ thị là đường cong trong hình bên. Số nghiệm thực của phương trình $f(x)=1$ là
\choice
{$1$}
{$0$}
{$2$}
{\True $3$}}
{\vspace{-0.5cm}
\begin{tikzpicture}[scale=1, font=\footnotesize, line join=round, line cap=round, >=stealth,color=\mauchinh,y=0.8cm]
\def\xmin{-2.01}\def\xmax{2.01}\def\ymin{-2.1}\def\ymax{2.1}
\draw[->,thick] (\xmin-0.2,0)--(\xmax+0.2,0) node[below] {\footnotesize $x$};
\draw[->,thick] (0,\ymin-0.2)--(0,\ymax+0.2) node[right] {\footnotesize $y$};
\draw (0,0) node [below left] {\footnotesize $O$};
\foreach \x in {-1,1}\draw (\x,0.1)--(\x,-0.1) node [below] {\footnotesize $\x$};
\foreach \y in {-2,2}\draw (0.1,\y)--(-0.1,\y) node [left] {\footnotesize $\y$};
\clip (\xmin,\ymin) rectangle (\xmax,\ymax);
\draw[thick,smooth,samples=200,domain=\xmin:\xmax] plot (\x,{-1*((\x)^3)+0*((\x)^2)+3*(\x)+0});
\draw[dashed] (-1,0)--(-1,-2)--(0,-2);\fill (-1,-2) circle (1pt);
\draw[dashed] (1,0)--(1,2)--(0,2);\fill (1,2) circle (1pt);
\end{tikzpicture}
}

\end{ex}


\Closesolutionfile{ans}
%% \indapan{10}{ans/ans-de11-7}