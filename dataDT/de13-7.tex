
\begin{name}
	{\tenchude}
	{\tendethi}
	{\tentruong}
	{\thoigian}
\end{name}
\Opensolutionfile{ans}[ans/ans-de13-7]

\begin{ex}%Câu 47.
cho hai số phức $z=4+3 i$ và $w=1-i$. Số phức $z-w$ bằng
\choice
{$5+2 i$}
{$7-i$}
{\True $3+4 i$}
{$-3-4 i$}

\end{ex}
\begin{ex}%Câu 1
Tập nghiệm $S$ của bất phương trình $\left(\dfrac{1}{2}\right)^{x}<32$ là:
\choice
{$S=(-\infty; 5)$}
{\True $S=(-5;+\infty)$}
{$S=(5;+\infty)$}
{$S=(-\infty;-5)$}

\end{ex}
\begin{ex}%Câu 2.
Cho số phức $z=4-i$, mô đun của số phức $(1+i) \bar{z}$ bằng
\choice
{$34$}
{$30$}
{\True $\sqrt{34}$}
{$\sqrt{30}$}

\end{ex}
\begin{ex}%Câu 3.
Nếu $\displaystyle\displaystyle\int\limits_0^2 f(x) \mathrm{d} x=2$ thì $\displaystyle\displaystyle\int\limits_0^2[4 x-f(x)] \mathrm{d} x$ bằng
\choice
{$12$}
{$10$}
{$4$}
{\True $6$}

\end{ex}
\begin{ex}%Câu 4.
Hàm số nào dưới đây đồng biến trên $\mathbb{R}$?
\choice
{$y=\dfrac{3 x-1}{x+1}$}
{$y=x^3-x$}
{$y=x^4-4 x$}
{\True $x^3+x$}

\end{ex}
\begin{ex}%Câu 5.
Trên đoạn $[-4;-1]$, hàm số $y=x^4-8 x^2+13$ đạt giá trị nhỏ nhất tại điểm
\choice
{\True $x=-2$}
{$x=-1$}
{$x=-4$}
{$x=-3$}

\end{ex}
\begin{ex}%Câu 6.
Trong không gian $O x y z$, cho hai điểm $M(1; 2; 1)$ và $N(3; 1;-2)$. Đường thẳng $MN$ có phương trình là:
\choice
{$\dfrac{x+1}{4}=\dfrac{y+2}{3}=\dfrac{z+1}{-1}$}
{\True $\dfrac{x-1}{2}=\dfrac{y-2}{-1}=\dfrac{z-1}{-3}$}
{$\dfrac{x-1}{4}=\dfrac{y-2}{3}=\dfrac{z-1}{-1}$}
{$\dfrac{x+1}{2}=\dfrac{y+2}{-1}=\dfrac{z+1}{-3}$}

\end{ex}
\begin{ex}%Câu 7.
Với $a>0$ đặt $\log_2(2 a)=b$, khi đó $\log_2\left(8 a^4\right)$ bằng
\choice
{$4 b+7$}
{$4 b+3$}
{$4 b$}
{\True $4 b-1$}

\end{ex}
\begin{ex}%Câu 8.
Trong không gian $O x y z$ cho điểm $A(1;-1; 2)$ và mặt phẳng $(P)$: $2 x-y+3 z+1=0$. Mặt phẳng đi qua $A$ và song song với mặt phẳng $(P)$ có phương trình là
\choice
{$2 x+y+3 z+7=0$}
{$2 x+y+3 z-7=0$}
{$2 x-y+3 z+9=0$}
{\True $2 x-y+3 z-9=0$}

\end{ex}
\begin{ex}%Câu 9.
Tập nghiệm của bất phương trình $3^{x}<2$ là
\choice
{\True $\left(-\infty; \log_3 2\right)$}
{$\left(\log_3 2;+\infty\right)$}
{$\left(-\infty; \log_2 3\right)$}
{$\left(\log_2 3;+\infty\right)$}

\end{ex}
\begin{ex}%Câu 10.
Nếu $\displaystyle\displaystyle\int\limits_1^4 f(x) \mathrm{d} x=3$ và $\displaystyle\displaystyle\int\limits_1^4 g(x) \mathrm{d} x=-2$ thì $\displaystyle\displaystyle\int\limits_1^4[f(x)-g(x)] \mathrm{d} x$ bằng
\choice
{$-1$}
{$-5$}
{\True $5$}
{$1$}

\end{ex}
\begin{ex}%Câu 11.
Trong không gian $O x y z$, cho mặt cầu $(S)$ có tâm $I(1;-4; 0)$ và bán kính bằng $3$. Phương trình của $(S)$ là:
\choice
{$(x+1)^2+(y-4)^2+z^2=9$}
{\True $(x-1)^2+(y+4)^2+z^2=9$}
{$(x-1)^2+(y+4)^2+z^2=3$}
{$(x+1)^2+(y-4)^2+z^2=3$}

\end{ex}
\begin{ex}%Câu 12.
Trong không gian $O x y z$, cho đường thẳng $d$ đi qua điểm $M(3;-1; 4)$ và có một vectơ chỉ phương $\vec{u}=(-2; 4; 5)$. Phương trình của $d$ là:
\choice
{$\heva{&x=3-2 t \\& y=-1+4 t \text {.} \\& z=4+5 t}$}
{$\heva{&x=3-2 t \\& y=-1+4 t \text {.} \\& z=4+5 t}$}
{$\heva{&x=3-2 t \\& y=-1+4 t \text {.} \\& z=4+5 t}$}
{\True $\heva{&x=3-2 t \\& y=-1+4 t \text {.} \\& z=4+5 t}$}

\end{ex}
\begin{ex}%Câu 13.
Cho hàm số $y=f(x)$ có bảng xét dấu đạo hàm như sau:
\immini{
Số điểm cực trị của hàm số đã cho là
}
{\begin{tikzpicture}[scale=1,line width=.6pt,color=\mauchinh]
\tkzTabInit[nocadre=true,lgt=1,espcl=1.3,deltacl=0.5,lw=0.8]
{$x$ /.7,$y'$/.7}{$-\infty$,$-2$,$-1$,$1$,$4$,$+\infty$}
\tkzTabLine{,-,0,+,0,-,0,+,0,-,}
\end{tikzpicture}
}
\choice
{$5$}
{$3$}
{$2$}
{\True $4$}
\end{ex}
\begin{ex}%Câu 14.
\immini{
Đồ thị hàm số nào dưới đây có dạng như
đường cong trong hình bên?
\choice
{\True $y=-2 x^4+4 x^2-1$}
{$y=-x^3+3 x-1$}
{$y=2 x^4-4 x^2-1$}
{$y=x^3-3 x-1$}}
{\vspace{-0.5cm}
\begin{tikzpicture}[scale=1, font=\footnotesize, line join=round, line cap=round, >=stealth,color=\mauchinh,y=0.8cm]
\def\xmin{-1.45}\def\xmax{1.45}\def\ymin{-1.45}\def\ymax{1.2}
\draw[->,thick] (\xmin-0.2,0)--(\xmax+0.2,0) node[below] {\footnotesize $x$};
\draw[->,thick] (0,\ymin-0.2)--(0,\ymax+0.2) node[right] {\footnotesize $y$};
\draw (0,0) node [below left] {\footnotesize $O$};
\foreach \x in {}\draw (\x,0.1)--(\x,-0.1) node [below] {\footnotesize $\x$};
\foreach \y in {}\draw (0.1,\y)--(-0.1,\y) node [left] {\footnotesize $\y$};
\clip (\xmin,\ymin) rectangle (\xmax,\ymax);
\draw[thick,smooth,samples=200,domain=\xmin:\xmax] plot (\x,{-2*((\x)^4)+4*((\x)^2)+-1});
\end{tikzpicture}
}

\end{ex}
\begin{ex}%Câu 15.
Đồ thị của hàm số $y=-x^4+4 x^2-3$ cắt trục tung tại điểm có tung độ bằng
\choice
{$0$}
{$3$}
{$1$}
{\True $-3$}

\end{ex}
\begin{ex}%Câu 16.
Với $n$ là số nguyên dương bất kì, $n \geq 4$, công thức nào dưới đây đúng?
\choice
{$A_n^4=\dfrac{(n-4) !}{n !}$}
{$A_n^4=\dfrac{4!}{(n-4) !}$}
{$A_n^4=\dfrac{n !}{4!(n-4) !}$}
{\True $A_n^4=\dfrac{n !}{(n-4) !}$}

\end{ex}
\begin{ex}%Câu 17.
Phần thực của số phức $z=5-2 i$ bằng
\choice
{\True $5$}
{$2$}
{$-5$}
{$-2$}

\end{ex}
\begin{ex}%Câu 18.
Trên khoảng $(0;+\infty)$, đạo hàm của hàm số $y=x^{\frac{5}{2}}$ là
\choice
{$y'=\dfrac{2}{7} x^{\frac{7}{2}}$}
{$y'=\dfrac{2}{5} x^{\frac{3}{2}}$}
{\True $y'=\dfrac{5}{2} x^{\frac{3}{2}}$}
{$y'=\dfrac{5}{2} x^{-\frac{3}{2}}$}

\end{ex}
\begin{ex}%Câu 19.
Cho hàm số $f(x)=x^2+4$. Khẳng định nào dưới đây đúng?
\choice
{$\displaystyle\displaystyle\int f(x) \mathrm{d} x=2 x+C$}
{$\displaystyle\displaystyle\int f(x) \mathrm{d} x=x^2+4 x+C$}
{\True $\displaystyle\displaystyle\int f(x) \mathrm{d} x=\dfrac{x^3}{3}+4 x+C$}
{$\displaystyle\displaystyle\int f(x) \mathrm{d} x=x^3+4 x+C$}

\end{ex}
\begin{ex}%Câu 20.
Trong không gian $O x y z$, cho điểm $A(-2; 3; 5)$. Tọa độ vectơ $\overrightarrow{OA}$ là
\choice
{\True $(-2; 3; 5)$}
{$(2;-3; 5)$}
{$(-2;-3; 5)$}
{$(2;-3;-5)$}

\end{ex}
\begin{ex}%Câu 21.
\immini{
Cho hàm số $y=f(x)$ có bảng biến thiên như hình bên. Giá trị cực tiểu của hàm số đã cho bằng
\choice
{$-1$}
{$5$}
{\True $-3$}
{$1$}}
{\vspace{-0.5cm}
\begin{tikzpicture}[color=\mauchinh]
\tkzTabInit[nocadre,lgt=1.2,espcl=1.6,deltacl=0.6]
{$x$/0.6,$f'(x)$/0.6,$f(x)$/1.8}{$-\infty$,$-1$,$1$,$+\infty$}
\tkzTabLine{,-,0,+,0,-,}
\tkzTabVar{+/$+\infty$,-/$-3$,+/$5$,-/$-\infty$}
\end{tikzpicture}

}

\end{ex}
\begin{ex}%Câu 22.
\immini{
Cho hàm số $y=f(x)$ có đồ thị là
đường cong trong hình bên. Hàm số đã cho
nghịch biến trên khoảng nào dưới đây?
\choice
{\True $(0; 1)$}
{$(-\infty; 0)$}
{$(0;+\infty)$}
{$(-1; 1)$}}
{\vspace{-0.5cm}
\begin{tikzpicture}[scale=1, font=\footnotesize, line join=round, line cap=round, >=stealth,color=\mauchinh,y=0.8cm]
\def\xmin{-1.65}\def\xmax{1.65}\def\ymin{-2.1}\def\ymax{1}
\draw[->,thick] (\xmin-0.2,0)--(\xmax+0.2,0) node[below] {\footnotesize $x$};
\draw[->,thick] (0,\ymin-0.2)--(0,\ymax+0.2) node[right] {\footnotesize $y$};
\draw (0,0) node [below left] {\footnotesize $O$};
\foreach \x in {-1,1}\draw (\x,0.1)--(\x,-0.1) node [below] {\footnotesize $\x$};
\foreach \y in {-2}\draw (0.1,\y)--(-0.1,\y) node [left] {\footnotesize $\y$};
\clip (\xmin,\ymin) rectangle (\xmax,\ymax);
\draw[thick,smooth,samples=200,domain=\xmin:\xmax] plot (\x,{1*((\x)^4)+-2*((\x)^2)+-1});
\draw[dashed] (-1,0)--(-1,-2)--(0,-2);\fill (-1,-2) circle (1pt);
\draw[dashed] (1,0)--(1,-2)--(0,-2);\fill (1,-2) circle (1pt);
\end{tikzpicture}
}

\end{ex}
\begin{ex}%Câu 23.
Nghiệm của phương trình $\log_3(5 x)=2$ là:
\choice
{$x=\dfrac{8}{5}$}
{$x=9$}
{\True $x=\dfrac{9}{5}$}
{$x=8$}

\end{ex}
\begin{ex}%Câu 24.
Nếu $\displaystyle\displaystyle\int\limits_0^3 f(x) \mathrm{d} x=4$ thì $\displaystyle\displaystyle\int\limits_0^3 3 f(x) \mathrm{d} x$ bằng
\choice
{$36$}
{\True $12$}
{$3$}
{$4$}

\end{ex}
\begin{ex}%Câu 25.
Thể tích của khối lập phương cạnh $5 a$ bằng
\choice
{$5 a^3$}
{$a^3$}
{\True $125 a^3$}
{$25 a^3$}

\end{ex}
\begin{ex}%Câu 26.
Tập xác định của hàm số $y=9^{x}$ là
\choice
{\True $\mathbb{R}$}
{$[0;+\infty)$}
{$\mathbb{R} \backslash\{0\}$}
{$(0;+\infty)$}

\end{ex}
\begin{ex}%Câu 27.
Diện tích $S$ của mặt cầu bán kính $R$ được tính theo công thức nào dưới đây?
\choice
{$S=16\pi R^2$}
{\True $y=4\pi R^2$}
{$S=\pi R^2$}
{$S=\dfrac{4}{3} \pi R^3$}

\end{ex}
\begin{ex}%Câu 28.
Tiệm cận đứng của đồ thị hàm số $y=\dfrac{2 x-1}{x-1}$ là đường thẳng có phương trình:
\choice
{\True $x=1$}
{$x=-1$}
{$x=2$}
{$x=\dfrac{1}{2}$}

\end{ex}
\begin{ex}%Câu 29.
Cho $a>0$ và $a \neq 1$, khi đó $\log_a \sqrt[4]{a}$ bằng
\choice
{$4$}
{\True $\dfrac{1}{4}$}
{$-\dfrac{1}{4}$}
{$-4$}

\end{ex}
\begin{ex}%Câu 30.
Cho khối chóp có diện tích đáy $B=5 a^2$ và chiều cao $h=a$. Thể tích của khối chóp đã cho bằng
\choice
{$\dfrac{5}{6} a^3$}
{$\dfrac{5}{2} a^3$}
{$5 a^3$}
{\True $\dfrac{5}{3} a^3$}

\end{ex}
\begin{ex}%Câu 31.
Trong không gian $O x y z$, cho mặt phẳng $(P)\colon 3 x-y+2 z-1=0$. Vectơ nào dưới đây là một vectơ pháp tuyến của $(P)$?
\choice
{$\vec{n}_1=(-3; 1; 2)$}
{\True $\overrightarrow{n_2}=(3;-1; 2)$}
{$\overrightarrow{n_3}=(3; 1; 2)$}
{$\vec{n}_4=(3; 1;-2)$}

\end{ex}
\begin{ex}%Câu 32.
Cho khối trụ có bán kính đáy $r=6$ và chiều cao $h=3$. Thể tích của khối trụ đã cho bằng
\choice
{\True $108\pi$}
{$36\pi$}
{$18\pi$}
{$54\pi$}

\end{ex}
\begin{ex}%Câu 33.
Cho hai số phức $z=4+2 i$ và $w=3-4 i$. Số phức $z+w$ bằng
\choice
{$1+6 i$}
{\True $7-2 i$}
{$7+2 i$}
{$-1-6 i$}

\end{ex}
\begin{ex}%Câu 34.
Cho cấp số nhân $\left(u_n\right)$ với $u_1=3$ và $u_2=9$. Công bội của cấp số nhân đã cho bằng
\choice
{$-6$}
{$\dfrac{1}{3}$}
{\True $3$}
{$6$}

\end{ex}
\begin{ex}%Câu 35.
Cho hàm số $f(x)={\rm e}^{x}+2$. Khẳng định nào dưới đây đúng?
\choice
{$\displaystyle\displaystyle\int f(x) \mathrm{d} x={\rm e}^{x-2}+C$}
{\True $\displaystyle\displaystyle\int f(x) \mathrm{d} x={\rm e}^{x}+2 x+C$}
{$\displaystyle\displaystyle\int f(x) \mathrm{d} x={\rm e}^{x}+C$}
{$\displaystyle\displaystyle\int f(x) \mathrm{d} x={\rm e}^{x}-2 x+C$}

\end{ex}
\begin{ex}%Câu 36.
Trên mặt phẳng tọa độ, điểm $M(-3; 4)$ là điểm biểu diễn của số phức nào dưới đây?
\choice
{$x=1$}
{\True $y=1$}
{$x=a$}
{$x=1-a$}

\end{ex}
\begin{ex}%Câu 37.
Biết hàm số $y=\dfrac{x+a}{x+1}$ ($a$ là số thực cho trước, $a \neq 1$) có đồ thị là $(C)$. Đường tiệm cận ngang của đồ thị hàm số đã cho là:
\choice
{$y'<0, \forall x \neq-1$}
{\True $y'>0, \forall x \neq-1$}
{$y'<0, \forall x \in \mathbb{R}$}
{$y'>0, \forall x \in \mathbb{R}$}

\end{ex}
\begin{ex}%Câu 38.
Từ một hộp chứa $12$ quả bóng gồm $5$ quả màu đỏ và $7$ quả màu xanh, lấy ngẫu nhiên đồng thời $3$ quả. Xác suất để lấy được $3$ quả màu xanh bằng
\choice
{\True $\dfrac{7}{44}$}
{$\dfrac{2}{7}$}
{$\dfrac{1}{22}$}
{$\dfrac{5}{12}$}

\end{ex}
\begin{ex}%Câu 39.
Trên đoạn $[0; 3]$, hàm số $y=-x^3+3 x$ đạt giá trị lớn nhất tại điểm
\choice
{$x=0$}
{$x=3$}
{\True $x=1$}
{$x=2$}

\end{ex}
\begin{ex}%Câu 40.
Trong không gian $O x y z$, cho điểm $M(-1; 3; 2)$ và mặt phẳng $(P)$: $x-2 y+4 z+1=0$. Đường thẳng đi qua $M$ và vuông góc với $(P)$ có phương trình là:
\choice
{$\dfrac{x+1}{1}=\dfrac{y-3}{-2}=\dfrac{z-2}{1}$}
{$\dfrac{x-1}{1}=\dfrac{y+3}{-2}=\dfrac{z+2}{1}$}
{$\dfrac{x-1}{1}=\dfrac{y+3}{-2}=\dfrac{z+2}{4}$}
{\True $\dfrac{x+1}{1}=\dfrac{y-3}{-2}=\dfrac{z-2}{4}$}

\end{ex}
\begin{ex}%Câu 41.
Cho hình chóp $S.ABC$ có đáy là tam giác vuông cân tại $B, AB=$ $2 a$ và $SA$ vuông góc với mặt phẳng đáy. Khoảng cách từ $C$ đến mặt phẳng ($SAB)$ bằng
\choice
{$\sqrt{2} a$}
{\True $2 a$}
{$a$}
{$2\sqrt{2} a$}

\end{ex}
\begin{ex}%Câu 42.
Trong không gian $O x y z$, cho hai điểm $A(1; 0; 0)$ và $B(4; 1; 2)$. Mặt phẳng đi qua $A$ và vuông góc với $AB$ có phương trình là
\choice
{$3 x+y+2 z-17=0$}
{\True $3 x+y+2 z-3=0$}
{$5 x+y+2 z-5=0$}
{$5 x+y+2 z-25=0$}

\end{ex}
\begin{ex}%Câu 43.
Cho số phức $z$ thỏa mãn $i z=5+4 i$. Số phức liên hợp của $z$ là
\choice
{\True $\bar{z}=4+5 i$}
{$\bar{z}=4-5 i$}
{$\bar{z}=-4+5 i$}
{$\bar{z}=-4-5 i$}

\end{ex}
\begin{ex}%Câu 44.
Cho hình lắng trụ đứng $ABC \cdot A'B'C'$ có tất cả các cạnh bằng nhau (tham khảo hình bên). Góc giữa hai đường thẳng $AA'$ và $BC'$ là
\choice
{$30^{\circ}$}
{$90^{\circ}$}
{\True $45^{\circ}$}
{$60^{\circ}$}

\end{ex}
\begin{ex}%Câu 45.
Với mọi $a, b$ thỏa mãn $\log_2 a^3+\log_2 b=6$, khẳng định nào dưới đây đúng?
\choice
{\True $a^3 b=64$}
{$a^3 b=36$}
{$a^3+b=64$}
{$a^3+b=64$}

\end{ex}
\begin{ex}%Câu 46.
Nếu $\displaystyle\displaystyle\int\limits_0^2 f(x) \mathrm{d} x=5$ thì $\displaystyle\displaystyle\int\limits_0^2[2 f(x)-1] \mathrm{d} x$ bằng
\choice
{\True $8$}
{$9$}
{$10$}
{$12$}

\end{ex}

\begin{ex}%Câu 48.
Cho cấp số cộng $\left(u_n\right)$ với $u_1=3$ và $u_2=5$. Công sai của cấp số cộng đã cho bằng
\choice
{$-2$}
{$\dfrac{3}{5}$}
{$\dfrac{5}{3}$}
{\True $2$}

\end{ex}
\begin{ex}%Câu 49.
Tiệm cận ngang của đồ thị hàm số $y=\dfrac{5 x-1}{x+1}$ là đường thẳng có phương trình
\choice
{\True $y=5$}
{$y=1$}
{$y=-5$}
{$y=-1$}

\end{ex}
\begin{ex}%Câu 50.
Tập xác định của hàm số $y=\log_3(x-4)$ là
\choice
{$(-\infty; 4]$}
{$[4;+\infty)$}
{\True $(4;+\infty)$}
{$(-\infty; 4)$}

\end{ex}

\Closesolutionfile{ans}
%indapan{10}{ans/ans-de13-7}