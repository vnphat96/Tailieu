
\begin{name}
	{\tenchude}
	{\tendethi}
	{\tentruong}
	{\thoigian}
\end{name}
\Opensolutionfile{ans}[ans/ans-de20-7]

\begin{ex}%Câu 15.
Cho số phức $z=-2+3 i$, số phức $(1+i) \bar{z}$ bằng
\choice
{$-1+5 i$}
{\True $1-5 i$}
{$-5-i$}
{$5-i$}

\end{ex}
\begin{ex}%Câu 1.
Cho khối lăng trụ có diện tích đáy $B=6$ và chiều cao $h=3$. Thể tích của khối lăng trụ đã cho bằng
\choice
{\True $18$}
{$3$}
{$9$}
{$6$}

\end{ex}
\begin{ex}%Câu 2.
Biết $\displaystyle\displaystyle\int\limits_1^2 f(x) \mathrm{d} x=3\displaystyle\displaystyle\int\limits_1^2 g(x) \mathrm{d} x=2$. Khi đó $\displaystyle\displaystyle\int\limits_1^2[f(x)-g(x)] \mathrm{d} x$ bằng
\choice
{$6$}
{$5$}
{$-1$}
{\True $1$}

\end{ex}
\begin{ex}%Câu 3.
Cho cấp số cộng $\left(u_n\right)$ với $u_1=8$ và công sai $d=3$. Giá trị của $u_2$ bằng
\choice
{$5$}
{$24$}
{\True $11$}
{$\dfrac{8}{3}$}

\end{ex}
\begin{ex}%Câu 4.
Tiệm cận đứng của đồ thị hàm số $y=\dfrac{2 x-2}{x+1}$ là
\choice
{$x=-2$}
{$x=2$}
{\True $x=-1$}
{$x=1$}

\end{ex}
\begin{ex}%Câu 5.
Nghiệm của phương trình $\log_2(x+6)=5$ là
\choice
{$x=38$}
{$x=19$}
{\True $x=26$}
{$x=4$}

\end{ex}
\begin{ex}%Câu 6.
Biết $\displaystyle\displaystyle\int\limits_0^1[f(x)+2 x] \mathrm{d} x=4$. Khi đó $\displaystyle\displaystyle\int\limits_0^1 f(x) \mathrm{d} x$ bằng
\choice
{$6$}
{$4$}
{$2$}
{\True $3$}

\end{ex}
\begin{ex}%Câu 7.
Tập nghiệm của bất phương trình $\log_3\left(36-x^2\right) \geq 3$ là
\choice
{$(0; 3]$}
{$(-\infty;-3] \cup[3;+\infty)$}
{$(-\infty; 3]$}
{\True $[-3; 3]$}

\end{ex}
\begin{ex}%Câu 8.
Số giao điểm của đồ thị hàm số $y=-x^3+3 x$ với trục hoành là
\choice
{\True $3$}
{$0$}
{$1$}
{$2$}

\end{ex}
\begin{ex}%Câu 9.
Với $a, b$ là các số thực dương tùy ý thỏa mãn $\log_3 a-2\log_9 b=3$, mệnh đề nào dưới đây đúng?
\choice
{\True $a=27 b$}
{$a=27 b^4$}
{$a=9 b$}
{$a=27 b^2$}

\end{ex}
\begin{ex}%Câu 10.
Trong không gian $O x y z$, cho điểm $M(2;-1; 3)$ và mặt phẳng $(P)$: $3 x-2 y+z+1=0$. Phương trình của mặt phẳng đi qua $M$ và song song với $(P)$ là
\choice
{\True $2 x-y+3 z-14=0$}
{$2 x-y+3 z+14=0$}
{$3 x-2 y+z+11=0$}
{$3 x-2 y+z-11=0$}

\end{ex}
\begin{ex}%Câu 11.
Giá trị nhỏ nhất của hàm số $f(x)=x^4-10 x^2-2$ trên đoạn $[0; 9]$ bằng
\choice
{$-2$}
{\True $-27$}
{$-11$}
{$-26$}

\end{ex}
\begin{ex}%Câu 12.
Gọi $z_1$ và $z_2$ là hai nghiệm phức của phương trình $z^2-z+2=0$. Khi đó $\left|z_1\right|+\left|z_2\right|$ bằng
\choice
{$\sqrt{2}$}
{$4$}
{\True $2\sqrt{2}$}
{$2$}

\end{ex}
\begin{ex}%Câu 13.
Gọi $D$ là hình phẳng giới hạn bởi các đường $y={\rm e}^{2 x}, y=0, x=0$ và $x=1$. Thể tích của khối tròn xoay tạo thành khi quay $D$ quanh trục $O x$ bằng
\choice
{$\displaystyle\displaystyle\int\limits_0^1 {\rm e}^{2 x} \mathrm{\,d} x$}
{\True $\pi \displaystyle\displaystyle\int\limits_0^1 {\rm e}^{4 x} \mathrm{\,d} x$}
{$\pi \displaystyle\displaystyle\int\limits_0^1 {\rm e}^{2 x} \mathrm{\,d} x$}
{$\displaystyle\displaystyle\int\limits_0^1 {\rm e}^{4 x} \mathrm{\,d} x$}

\end{ex}
\begin{ex}%Câu 14.
Cho hình hộp chữ nhật $ABCD \cdot A'B'C'D'$ có $AB=AA'=a, AD=$ $\sqrt{2} a$. Góc giữa đường thẳng $A'C$ và mặt phẳng $(ABCD)$ bằng
\choice
{$90^{\circ}$}
{$60^{\circ}$}
{$45^{\circ}$}
{\True $30^{\circ}$}

\end{ex}

\begin{ex}%Câu 16.
Cho hàm số $f(x)$ có đạo hàm $f'(x)=x(x+1)(x-4)^3, \forall x \in \mathbb{R}$. Số điểm cực đại của hàm số đã cho là
\choice
{\True $1$}
{$4$}
{$3$}
{$2$}

\end{ex}
\begin{ex}%Câu 17.
Trong không gian $O x y z$, cho điểm $M(1;-2; 2)$ và mặt phẳng $(P)$: $2 x+y-3 z+1=0$. Phương trình của đường thẳng đi qua $M$ và vuông góc với $(P)$ là
\choice
{$\heva{&x=1+t \\& y=-2-2 t \text {.} \\& z=2+t}$}
{\True $\heva{&x=1+t \\& y=-2-2 t \text {.} \\& z=2+t}$}
{$\heva{&x=1+t \\& y=-2-2 t \text {.} \\& z=2+t}$}
{$\heva{&x=1+t \\& y=-2-2 t \text {.} \\& z=2+t}$}

\end{ex}
\begin{ex}%Câu 18.
Cắt hình trụ $(T)$ bởi một mặt phẳng qua trục của nó, ta được thiết diện là một hình vuông cạnh bằng $3$. Diện tích xung quanh của $(T)$ bằng
\choice
{\True $9\pi$}
{$\dfrac{9\pi}{2}$}
{$18\pi$}
{$\dfrac{9\pi}{4}$}

\end{ex}
\begin{ex}%Câu 19.
\immini{
Đồ thị của hàm số nào dưới đây có
dạng đường cong như hình vẽ
\choice
{$y=-x^3-3 x^2+2$}
{$y=-x^4+3 x^2+2$}
{\True $y=x^4-3 x^2+2$}
{$y=x^3-2 x^2-2$}}
{\vspace{-0.5cm}
\begin{tikzpicture}[scale=1, font=\footnotesize, line join=round, line cap=round, >=stealth,y=0.8cm,color=\mauchinh]
\def\xmin{-1.78}\def\xmax{1.78}\def\ymin{-0.5}\def\ymax{2.49}
\draw[->,thick] (\xmin-0.2,0)--(\xmax+0.2,0) node[below] {\footnotesize $x$};
\draw[->,thick] (0,\ymin-0.2)--(0,\ymax+0.2) node[right] {\footnotesize $y$};
\draw (0,0) node [below left] {\footnotesize $O$};
\foreach \x in {}\draw (\x,0.1)--(\x,-0.1) node [below] {\footnotesize $\x$};
\foreach \y in {}\draw (0.1,\y)--(-0.1,\y) node [left] {\footnotesize $\y$};
\clip (\xmin,\ymin) rectangle (\xmax,\ymax);
\draw[thick,smooth,samples=200,domain=\xmin:\xmax] plot (\x,{1*((\x)^4)+-3*((\x)^2)+2});
\end{tikzpicture}
}

\end{ex}
\begin{ex}%Câu 20.
Cho cấp số nhân $\left(u_n\right)$ có số hạng đầu $u_1=2$ công bội $q=4$. Giá trị của $u_3$ bằng.
\choice
{\True $32$}
{$16$}
{$8$}
{$6$}

\end{ex}
\begin{ex}%Câu 21.
Một tổ có $6$ học sinh nam và $5$ học sinh nữ. Có bao nhiêu cách chọn một họcc sinh nam và một học sinh nữ để đi tập văn nghệ.
\choice
{$A_{11}^2$}
{\True $30$}
{$C_{11}^2$}
{$11$}

\end{ex}
\begin{ex}%Câu 22.
Họ tất cả các nguyên hàm của hàm số $f(x)=2^{x}+4 x$ là
\choice
{$2^{x} \ln 2+2 x^2+C$}
{\True $\dfrac{2^{x}}{\ln 2}+2 x^2+C$}
{$2^{x} \ln 2+C$}
{$\dfrac{2^{x}}{\ln 2}+C$}

\end{ex}
\begin{ex}%Câu 23.
Cho khối lăng trụ có đáy là hình vuông cạnh $a$ và chiều cao bằng $3 a$. Thể tích của khối lăng trụ đã cho bằng
\choice
{$a^3$}
{$4 a^3$}
{$\dfrac{4}{3} a^3$}
{\True $3 a^3$}

\end{ex}
\begin{ex}%Câu 24.
Nghiệm của phương trình $\log_2(3 x-8)=2$ là
\choice
{$x=-4$}
{$x=12$}
{\True $x=4$}
{$x=-\dfrac{4}{3}$}

\end{ex}
\begin{ex}%Câu 25.
Cho khối trụ có chiều cao bằng $2\sqrt{3}$ và bán kính đáy bằng $2$. Thể tích của khối trụ đã cho bằng
\choice
{$8\pi$}
{\True $8\sqrt{3} \pi$}
{$\dfrac{8\sqrt{3}}{3} \pi$}
{$24\pi$}

\end{ex}
\begin{ex}%Câu 26.
\immini{
Cho hàm số có bảng biến thiên như hình bên.
Hàm số đã cho đồng biến trên khoảng nào dưới đây?
\choice
{\True $(1;+\infty)$}
{$(-3;+\infty)$}
{$(-1; 1)$}
{$(-\infty; 1)$}}
{\vspace{-0.5cm}
\begin{tikzpicture}[color=\mauchinh]
\tkzTabInit[nocadre,lgt=1.2,espcl=1.6,deltacl=0.6]
{$x$/0.6,$f'(x)$/0.6,$f(x)$/1.8}{$-\infty$,$-1$,$1$,$+\infty$}
\tkzTabLine{,+,0,-,0,+,}
\tkzTabVar{-/$-\infty$,+/$1$,-/$-3$,+/$+\infty$}
\end{tikzpicture}

}

\end{ex}
\begin{ex}%Câu 27.
Trong không gian $O x y z$, cho hai điểm $A(1; 1;-2), B(3;-4; 1)$. Tọa độ của vectơ $\overrightarrow{AB}$ là
\choice
{$(-2; 5;-3)$}
{$(2; 5; 3)$}
{\True $(2;-5; 3)$}
{$(2; 5;-3)$}

\end{ex}
\begin{ex}%Câu 28.
Phương trình đường tiệm cận đứng của đồ thị hàm số $y=\dfrac{2 x-3}{x-1}$ là:
\choice
{$y=2$}
{$y=1$}
{\True $x=1$}
{$x=2$}

\end{ex}
\begin{ex}%Câu 29.
Cho hình nón có độ dài đường sinh bằng $3 a$ và bán kính đáy bằng a. Diện tích xung quanh của hình nón đã cho bằng
\choice
{$12\pi a^2$}
{\True $3\pi a^2$}
{$6\pi a^2$}
{$\pi a^2$}

\end{ex}
\begin{ex}%Câu 30.
Với $a$ là số thực dương khác $1, \log_{a^2}(a \sqrt{a})$ bằng
\choice
{$\dfrac{3}{4}$}
{$3$}
{\True $\dfrac{3}{2}$}
{$\dfrac{1}{4}$}

\end{ex}
\begin{ex}%Câu 31.
Cho khối chóp có diện tích đáy bằng $a^2$ và chiều cao bằng $2 a$. Thể tích của khối chóp đã cho bằng
\choice
{\True $\dfrac{2 a^3}{3}$}
{$2 a^3$}
{$4 a^3$}
{$a^3$}

\end{ex}
\begin{ex}%Câu 32.
Giá trị nhỏ nhất của hàm số $y=x^4-2 x^2-3$ trên đoạn $[-1; 2]$ bằng
\choice
{\True $-4$}
{$0$}
{$5$}
{$-3$}

\end{ex}
\begin{ex}%Câu 33.
Cho $f(x)$ là một hàm số liên tục trên $\mathbb{R}$ và $F(x)$ là một nguyên hàm của hàm số $f(x)$. Biết $\displaystyle\displaystyle\int\limits_1^3 f(x) \mathrm{d} x=3$ và $F(1)=1$. Giá trị của $F(3)$ bẵng
\choice
{\True $4$}
{$2$}
{$-2$}
{$3$}

\end{ex}
\begin{ex}%Câu 34.
Đạo hàm của hàm số $y=\log_3\left(2 x^2-x+1\right)$ là
\choice
{$\dfrac{2 x-1}{\left(2 x^2-x+1\right) \ln 3}$}
{\True $\dfrac{4 x-1}{\left(2 x^2-x+1\right) \ln 3}$}
{$\dfrac{(4 x-1) \ln 3}{\left(2 x^2-x+1\right)}$}
{$\dfrac{4 x-1}{\left(2 x^2-x+1\right)}$}

\end{ex}

\begin{ex}%35
Trong không gian $O x y z$, cho điểm $M(2; 1;-3)$ và mặt phẳng $(P)$: $3 x-2 y+z-3=0$. Phương trình của mặt phẳng đi qua $M$ và song song với $(P)$ là
\choice
{$2 x+y-3 x+14=0$}
{$2 x+y-3 z-14=0$}
{\True $3 x-2 y+z-1=0$}
{$3 x-2 y+z+1=0$}

\end{ex}
\begin{ex}%Câu 36.
Trong không gian $O x y z$, cho hai điểm $A(-1; 1; 0)$ và $B(3; 5;-2)$. Tọa độ trung điểm của đoạn thẳng $AB$ là
\choice
{$(2; 2;-1)$}
{$(2; 6;-2)$}
{$(4; 4;-2)$}
{\True $(1; 3;-1)$}

\end{ex}
\begin{ex}%Câu 37.
\immini{
Cho hàm số $y=f(x)$ có đồ thị như hình vẽ bên dưới.
Số giá trị nguyên của tham số $m$ để đường thẳng $y=m$ cắt đồ thị hàm số đã cho tại ba điểm phân biệt là
\choice
{Vô số}
{\True $3$}
{$0$}
{$5$}}
{\vspace{-0.5cm}
\begin{tikzpicture}[scale=1, font=\footnotesize, line join=round, line cap=round, >=stealth,y=0.7cm,color=\mauchinh]
\def\xmin{-1.04}\def\xmax{3.16}\def\ymin{-0.5}\def\ymax{5.39}
\draw[->,thick] (\xmin-0.2,0)--(\xmax+0.2,0) node[below] {\footnotesize $x$};
\draw[->,thick] (0,\ymin-0.2)--(0,\ymax+0.2) node[right] {\footnotesize $y$};
\draw (0,0) node [below left] {\footnotesize $O$};
\foreach \x in {2}\draw (\x,0.1)--(\x,-0.1) node [below] {\footnotesize $\x$};
\foreach \y in {1,5}\draw (0.1,\y)--(-0.1,\y) node [left] {\footnotesize $\y$};
\clip (\xmin,\ymin) rectangle (\xmax,\ymax);
\draw[thick,smooth,samples=200,domain=\xmin:\xmax] plot (\x,{-1*((\x)^3)+3*((\x)^2)+0*(\x)+1});
\draw[dashed] (2,0)--(2,5)--(0,5);\fill (2,5) circle (1pt);
\end{tikzpicture}
}

\end{ex}
\begin{ex}%Câu 38.
Tập nghiệm của bất phương trình $4^{x^2-2 x} \geq 64$ là
\choice
{\True $(-\infty;-1] \cup[3;+\infty)$}
{$[3;+\infty)$}
{$(-\infty;-1]$}
{$[-1; 3]$}

\end{ex}
\begin{ex}%Câu 39.
Cho hình nón có thiết diện qua trục là tam giác vuông cân có cạnh huyền bằng $a \sqrt{2}$. Diện tích xung quanh của hình nón đã cho bằng
\choice
{$\pi a^2 \sqrt{2}$}
{$\dfrac{\pi a^2}{2}$}
{$\pi a^2$}
{\True $\dfrac{\pi a^2 \sqrt{2}}{2}$}

\end{ex}
\begin{ex}%Câu 40.
Cho hàm số $y=\dfrac{2 x+1}{x-1}$. Tích giá trị lớn nhất và giá trị nhỏ nhất của hàm số đã cho trên đoạn $[-1; 0]$ bằng
\choice
{$\dfrac{3}{2}$}
{$2$}
{\True $\dfrac{-1}{2}$}
{$0$}

\end{ex}
\begin{ex}%Câu 41.
\immini{
Cho hàm số $y= f(x)$ có bảng biến thiên như hình bên. Tổng số đường tiệm cận đứng và tiệm cận ngang của đồ thị hàm số bằng
\choice
{$4$}
{$1$}
{\True $2$}
{$3$}}
{
\vspace{-0.5cm}
 \begin{nscenter}
  	\begin{tikzpicture}[scale=1,line width=.6pt,color=\mauchinh]

\tkzTabInit[nocadre=true,lgt=1.1,espcl=1.6,deltacl=0.5,lw=0.8]
{$x$ /.7,$f'(x)$/.7,$f(x)$/1.8}{$-\infty$,$-1$,$2$,$+\infty$}
\tkzTabLine{,+,d,-,d,+,}
\tkzTabVar{-/$3$,+D+/$+\infty$/$4$,-/$-5$,+/$+\infty$}
\end{tikzpicture}
\end{nscenter}
}

\end{ex}
\begin{ex}%Câu 42.
Số nghiệm của phương trình $\log_3(x+2)+\log_3(x-2)=\log_3 5$ là
\choice
{$2$}
{$3$}
{\True $1$}
{$0$}

\end{ex}
\begin{ex}%Câu 43.
Cho hình chóp $S.ABCD$ có đáy là hình vuông cạnh $a, SA$ vuông góc với mặt phẳng đáy và $SA=a \sqrt{2}$ (tham khảo hình vẽ). Góc giữa đường thẳng $SC$ và mặt phẳng $(ABCD)$ bằng
\choice
{$30^{\circ}$}
{\True $45^{\circ}$}
{$60^{\circ}$}
{$90^{\circ}$}

\end{ex}
\begin{ex}%Câu 44.
Cho hàm số $y=f(x)$ có đạo hàm $f'(x)=x(x+3)(x-1)^2$. Số điểm cực trị của hàm số bằng
\choice
{$0$}
{\True $2$}
{$3$}
{$1$}

\end{ex}
\begin{ex}%Câu 45.
Họ tất cả nguyên hàm của hàm số $f(x)=\dfrac{1}{x}\left(1+\dfrac{x}{\cos ^2 x}\right)$ với $x \in(0;+\infty) \backslash\left\{\dfrac{\pi}{2}+k \pi, k \in \mathbb{Z}\right\}$ là
\choice
{$-\dfrac{1}{x^2}+\tan x+C$}
{\True $\ln x+\tan x+C$}
{$-\dfrac{1}{x^2}-\tan x+C$}
{$\ln x-\tan x+C$}

\end{ex}
\begin{ex}%Câu 46.
Cho khối lăng trụ đứng $ABC \cdot A'B'C'$ có đáy là tam giác vuông tại $B, AB=a, AC=a \sqrt{5}, AA'=2 a \sqrt{3}$. Thể tích khối lăng trụ đã cho bằng
\choice
{\True $2\sqrt{3} a^3$}
{$4\sqrt{3} a^3$}
{$\dfrac{2\sqrt{3} a^3}{3}$}
{$\dfrac{\sqrt{3} a^3}{3}$}

\end{ex}
\begin{ex}%Câu 47.
Trong không gian $O x y z$, cho các vectơ $\vec{a}=(-2;-3; 1)$ và $\vec{b}= (1; 0; 1)$. Côsin góc giữa hai vectơ $\vec{a}$ và $\vec{b}$ bằng
\choice
{\True $-\dfrac{1}{2\sqrt{7}}$}
{$\dfrac{1}{2\sqrt{7}}$}
{$-\dfrac{3}{2\sqrt{7}}$}
{$\dfrac{3}{2\sqrt{7}}$}

\end{ex}
\begin{ex}%Câu 48.
\immini{
Cho hàm số $y=f(x)$ có bảng biến thiên như hình bên. Số nghiệm của phương trình $2 f(x)-11=0$ bằng
}
{\vspace{-0.5cm}
\begin{tikzpicture}[color=\mauchinh]
\tkzTabInit[nocadre,lgt=1.1,espcl=1.4,deltacl=0.6]
{$x$/0.6,$f'(x)$/0.6,$f(x)$/1.8}{$-\infty$,$-\sqrt{6}$,$0$,$\sqrt{6}$,$+\infty$}
\tkzTabLine{,-,0,+,0,-,0,+,}
\tkzTabVar{+/$+\infty$,-/$-4$,+/$5$,-/$-4$,+/$+\infty$}
\end{tikzpicture}

}
\choice
{$3$}
{\True $2$}
{$0$}
{$4$}
\end{ex}
\begin{ex}%Câu 49.
Cho hình chóp $S.ABCD$ có đáy $ABCD$ là hình chữ nhật tâm $O$, cạnh $AB=a, AD=a \sqrt{2}$. Hình chiếu vuông góc của $S$ trên mặt phẳng $(ABCD)$ là trung điểm của đoạn $OA$. Góc giữa $SC$ và mặt phẳng $(ABCD)$ bằng $30^{\circ}$. Khoảng cách từ $C$ đến mặt phẳng $(SAB)$ bằng
\choice
{$\dfrac{9\sqrt{22} a}{44}$}
{\True $\dfrac{3\sqrt{22} a}{11}$}
{$\dfrac{\sqrt{22} a}{11}$}
{$\dfrac{3\sqrt{22} a}{44}$}

\end{ex}
\begin{ex}%Câu 50.
Cho phương trình $16^{x^2}-2\cdot 4^{x^2+1}+10=m$ ($m$ là tham số). Số giá trị nguyên của $m \in[-10; 10]$ để phương trình đã cho có đúng $2$ nghiệm thực phân biệt là
\choice
{$7$}
{\True $9$}
{$8$}
{$1$}

\end{ex}

\Closesolutionfile{ans}
%% \indapan{10}{ans/ans-de20-7}
