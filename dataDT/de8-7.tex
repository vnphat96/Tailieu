
\begin{name}
	{\tenchude}
	{\tendethi}
	{\tentruong}
	{\thoigian}
\end{name}
\Opensolutionfile{ans}[ans/ans-de8-7]

\begin{ex}%Câu 45.
Cho số phức $z=2-3 i$. Môđun của số phức $z$ là
\choice
{\True $\sqrt{13}$}
{$2$}
{$-3$}
{$13$}

\end{ex}
\begin{ex}%Câu 1.
Trong không gian cho hình vuông $ABCD$ cạnh bằng $6 a$. Gọi $M, N$ lần lượt là trung điểm của các cạnh $AB, CD$. Khi quay hình vuông $ABCD$ quanh đường thẳng $MN$ thì đường gấp khúc $MADN$ tạo thành một hình trụ. Diện tích xung quanh hình trụ đó bằng
\choice
{$18\pi a^2$}
{$72\pi a^2$}
{\True $36\pi a^2$}
{$2\pi a^2$}

\end{ex}
\begin{ex}%Câu 2.
Xét $\displaystyle\int\limits_1^{{\rm e}} \dfrac{2\ln x+1}{x} \mathrm{\,d} x$, nếu đặt $t=\ln x$ thì $\displaystyle\displaystyle\int\limits_1^{{\rm e}} \dfrac{2\ln x+1}{x} \mathrm{\,d} x$ bằng
\choice
{$\displaystyle\int\limits_0^2(2 t+1) \mathrm{d} t$} 
{\True $\displaystyle\displaystyle\int\limits_0^1(2t+1){\rm d}t$}
{$\displaystyle\displaystyle\int\limits_1^{{\rm e}}(2 t+1) \mathrm{d} t$}
{$\displaystyle\displaystyle\int\limits_0^{{\rm e}}(2 t+1) \mathrm{d} t$}
\end{ex}
\begin{ex} %3
Thể tích $V$ của khối tròn xoay sinh ra khi cho hình phẳng $S$ giới hạn bởi các đường $y=2-x^2$, trục hoành, trục tung và $x=1$ quay xung quanh trục hoành được tính bởi công thức nào dưới đây?
\choice
{$\displaystyle\int\limits_0^1 \left|2-x^2\right|{\rm d}x$}
{$\displaystyle\int\limits_0^1 \left(2-x^2\right)^2{\rm d}x$}
{$\pi \displaystyle\int\limits_0^1 \left(2-x^2\right){\rm d}x$}
{\True $\pi \displaystyle\int\limits_0^1 \left(2-x^2\right)^2{\rm d}x$}
\end{ex}
\begin{ex}%Câu 4.
Trong tập hợp các số phức, cho số phức $z$ là nghiệm của phương trình $z+2\bar{z}=6+i$. Tính môđun của số phức $z$.
\choice
{\True $\sqrt{5}$}
{$\sqrt{3}$}
{$5$}
{$\sqrt{2}$}

\end{ex}
\begin{ex}%Câu 5.
Gọi $A, B$ là hai điểm biểu diễn hai nghiệm phức $z_1, z_2$ của phương trình $z^2+2 z+7=0$. Tính độ dài đoạn thẳng $AB$.
\choice
{$2$}
{\True $2\sqrt{6}$}
{$24$}
{$4$}

\end{ex}
\begin{ex}%Câu 6.
Có bao nhiêu số tự nhiên có hai chữ số khác nhau mà các chữ số được lấy từ tập hợp $X=\{1; 2; 3; 4; 5\}$.
\choice
{\True $A_5^2$}
{$C_5^2$}
{$5^2$}
{$2^5$}

\end{ex}
\begin{ex}%Câu 7.
Cho cấp số nhân $\left(u_n\right)$ với $u_1=3$, công bội $q=2$. Số hạng thứ hai của cấp số đã cho bằng:
\choice
{\True $6$}
{$5$}
{$8$}
{$9$}

\end{ex}
\begin{ex}%Câu 8.
Tập nghiệm của phương trình $\log_2 x=\log_2(2 x+1)$ là
\choice
{\True $\varnothing$}
{$\{0\}$}
{$\{1\}$}
{$\{-1\}$}

\end{ex}
\begin{ex}%Câu 9.
Cho khối chóp có chiều cao bằng $6$, diện tích đáy bằng $4$. Thể tích khối chóp đã cho bằng
\choice
{\True $8$}
{$24$}
{$10$}
{$12$}

\end{ex}
\begin{ex}%Câu 10.
Tập xác định của hàm số $y=x^{\frac{1}{3}}$ là
\choice
{\True $(0;+\infty)$}
{$[0;+\infty)$}
{$\mathbb{R}$}
{$\mathbb{R} \backslash\{0\}$}

\end{ex}
\begin{ex}%Câu 11.
Cho hàm số $F(x)$ là một nguyên hàm của hàm số $f(x)$ trên đoạn $[a; b]$. Tích phân $\displaystyle\int\limits_a^{b} f(x) \mathrm{d} x$ bằng
\choice
{\True $F(b)-F(a)$}
{$F(a)-F(b)$}
{$f(b)-f(a)$}
{$f(a)-f(b)$}

\end{ex}
\begin{ex}%Câu 12.
Cho khối hộp chữ nhật $ABCD \cdot A'B'C'D'$ có $AB=2, AD=$ $3, AA'=4$. Thể tích khối hộp đã cho bằng
\choice
{\True $24$}
{$20$}
{$9$}
{$8$}

\end{ex}
\begin{ex}%Câu 13.
Cho hình nón có độ dài đường sinh bằng $5$, bán kính đáy bằng $3$. Diện tích toàn phần của hình nón đã cho bằng
\choice
{\True $24\pi$}
{$15\pi$}
{$48\pi$}
{$39\pi$}

\end{ex}
\begin{ex}%Câu 14.
Cho mặt cầu có bán kính $R=3$. Diện tích mặt cầu đã cho bằng
\choice
{\True $36\pi$}
{$9\pi$}
{$18\pi$}
{$24\pi$}

\end{ex}
\begin{ex}%15
\immini{
Cho hàm số $y=f(x)$ có bảng biến thiên như hình vẽ. Hàm số đã cho đồng biến trên khoảng nào dưới đây?
\choice
{\True $(2;+\infty)$}
{$\left(1;+\infty\right)$}
{$\left(-\infty;3\right)$}
{$(-\infty;+\infty)$}}
{
  	\begin{tikzpicture}[scale=1,line width=.6pt,color=\mauchinh]
\tkzTabInit[nocadre=true,lgt=1.5,espcl=2,deltacl=0.5,lw=0.8]
%{$x$ /.7,$y'$/.7}{$-\infty$,$-1$,$0$,$1$,$+\infty$}
{$x$ /.7,$f'(x)$/.7,$f(x)$/1.8}{$-\infty$,$2$,$+\infty$}
\tkzTabLine{,+,d,+,}
\tkzTabVar{-/$1$,+D-/$+\infty$/$-\infty$,+/$+\infty$}
\end{tikzpicture}
}
\end{ex}
\begin{ex}%Câu 16.
Với $a, b$ là các số dương tùy ý, $\log_3\left(a^2 b^5\right)$ bằng
\choice
{\True $2\log_3 a+5\log_3 b$}
{$10\log_3(a b)$}
{$7\log_3(a b)$}
{$10\left(\log_3 a+\log_3 b\right)$}

\end{ex}
\begin{ex}%Câu 17.
Cho khối trụ có chiều cao $h$, bán kính $r$. Thể tích khối trụ đã cho bằng
\choice
{\True $h \pi r^2$}
{$2 h \pi r^2$}
{$\dfrac{h \pi r^2}{3}$}
{$\dfrac{4 h \pi r^2}{3}$}

\end{ex}
\begin{ex}%Câu 18.
\immini{
Cho hàm số $y=f(x)$ có bảng biến thiên như hình vẽ bên dưới:
. Hàm số đã cho đạt cực tiểu tại
\choice
{\True $x=2$}
{$x=-2$}
{$x=0$}
{$x=1$}}
{\vspace{-0.5cm}
\begin{tikzpicture}[color=\mauchinh]
\tkzTabInit[nocadre,lgt=1.2,espcl=1.8,deltacl=0.6,lw=1]
{$x$/0.6,$f'(x)$/0.6,$f(x)$/2}{$-\infty$,$-1$,$1$,$+\infty$}
\tkzTabLine{,+,0,-,0,+,}
\tkzTabVar{-/$-\infty$,+/$2$,-/$-2$,+/$+\infty$}
\end{tikzpicture}

}

\end{ex}
\begin{ex}%Câu 19.
\immini{
Đồ thị hàm số nào dưới đây có dạng như đường cong trong hình vẽ?
\choice
{\True $y=x^4-2 x^2-1$}
{$y=x^4+2 x^2-1$}
{$y=-x^4+2 x^2-1$}
{$y=x^4-2 x^2+1$}}
{\vspace{-0.5cm}
\begin{tikzpicture}[scale=.8, font=\footnotesize, line join=round, line cap=round, >=stealth,color=\mauchinh]
\def\xmin{-1.65}\def\xmax{1.65}\def\ymin{-2.01}\def\ymax{0.97}
\draw[->,thick] (\xmin-0.2,0)--(\xmax+0.2,0) node[below] {\footnotesize $x$};
\draw[->,thick] (0,\ymin-0.2)--(0,\ymax+0.2) node[right] {\footnotesize $y$};
\draw (0,0) node [below left] {\footnotesize $O$};
\foreach \x in {}\draw (\x,0.1)--(\x,-0.1) node [below] {\footnotesize $\x$};
\foreach \y in {}\draw (0.1,\y)--(-0.1,\y) node [left] {\footnotesize $\y$};
\clip (\xmin,\ymin) rectangle (\xmax,\ymax);
\draw[thick,smooth,samples=200,domain=\xmin:\xmax] plot (\x,{1*((\x)^4)+-2*((\x)^2)+-1});
\end{tikzpicture}
}

\end{ex}
\begin{ex}%Câu 20.
Đường tiệm cận đứng của đồ thị hàm số $y=\dfrac{3 x-2}{x+1}$ có phương trình là
\choice
{$x=-2$}
{\True $x=-1$}
{$x=3$}
{$x=1$}

\end{ex}
\begin{ex}%Câu 21.
Số nghiệm nguyên của bất phương trình $\log_4 x<1$ là.
\choice
{\color{\mauchinh}{Vô số}}
{\True $3$}
{$4$}
{$5$}

\end{ex}
\begin{ex}%Câu 22.
\immini{
Cho hàm số bậc bốn $y=f(x)$ có đồ thị như hình vẽ bên. Số nghiệm phân biệt của phương trình $f(x)=1$ là
\choice
{\True $3$}
{$0$}
{$4$}
{$2$}}
{
\vspace{-0.5cm}
\begin{tikzpicture}[scale=1, font=\footnotesize, line join=round, line cap=round, >=stealth,color=\mauchinh]
\def\xmin{-1.45}\def\xmax{1.45}\def\ymin{-1.01}\def\ymax{1.35}
\draw[->,thick] (\xmin-0.2,0)--(\xmax+0.2,0) node[below] {\footnotesize $x$};
\draw[->,thick] (0,\ymin-0.2)--(0,\ymax+0.2) node[right] {\footnotesize $y$};
\draw (0,0) node [below left] {\footnotesize $O$};
\foreach \x in {-1,1}\draw (\x,0.1)--(\x,-0.1) node [below] {\footnotesize $\x$};
\foreach \y in {-1,1}\draw (0.1,\y)--(-0.1,\y) node [left] {\footnotesize $\y$};
\clip (\xmin,\ymin) rectangle (\xmax,\ymax);
\draw[thick,smooth,samples=200,domain=\xmin:\xmax] plot (\x,{2*((\x)^4)+-4*((\x)^2)+1});
\draw[dashed] (-1,0)--(-1,-1)--(0,-1);\fill (-1,-1) circle (1pt);
\draw[dashed] (1,0)--(1,-1)--(0,-1);\fill (1,-1) circle (1pt);
\end{tikzpicture}
}

\end{ex}
\begin{ex}%Câu 23.
Nếu $\displaystyle\displaystyle\int\limits_1^3 f(x) \mathrm{d} x=4$ thì $\displaystyle\displaystyle\int\limits_1^3[f(x)+1] \mathrm{d} x$ bằng
\choice
{\True $6$}
{$5$}
{$4$}
{$2$}

\end{ex}
\begin{ex}%Câu 24.
Cho số phức $z=3-4 i$. Mô $-$ đun của $z$ bằng
\choice
{\True $5$}
{$7$}
{$1$}
{$12$}

\end{ex}
\begin{ex}%Câu 25.
Cho hai số phức $z_1=2+3 i$ và $z_2=3-2 i$. Tọa độ điểm biểu diễn số phức $z_1-z_2$ là
\choice
{\True $(-1; 5)$}
{$(-1; 1)$}
{$(5; 1)$}
{$(1; 5)$}

\end{ex}
\begin{ex}%Câu 26.
Phần ảo của số phức $z=4-5 i$ là
\choice
{\True $-5$}
{$5$}
{$4$}
{$-5 i$}

\end{ex}
\begin{ex}%Câu 27.
Trong hệ tọa độ $(O x y z)$. Hình chiếu vuông góc của điểm $M(1; 2; 3)$ lên trục $O z$ là điểm có tọa độ:
\choice
{\True $(0; 0; 3)$}
{$(1; 2; 0)$}
{$(0; 2; 3)$}
{$(0; 2; 0)$}

\end{ex}
\begin{ex}%Câu 28.
Trong không gian $O x y z$ cho mặt cầu $(S)\colon x^2+y^2+z^2-2 x-$ $4 y+6 z-1=0$. Tâm của $(S)$ có tọa độ là
\choice
{\True $(1; 2;-3)$}
{$(-1;-2; 3)$}
{$(1; 2; 3)$}
{$(1;-2;-3)$}

\end{ex}
\begin{ex}%Câu 29.
Trong không gian Oxyz, mặt phẳng $(P)\colon 2 x-y+z-1=0$ đi qua điểm nào dưới đây?
choice
{\True $(1; 2; 1)$}
{ $(1;-2; 1)$}
{ $(1; 2;-1)$}
{ $(1;-2; 3)$}

\end{ex}
\begin{ex}%Câu 30.
Trong không gian $O x y z$, đường thẳng $\Delta\colon \dfrac{x-1}{-2}=\dfrac{y+2}{3}=\dfrac{z+1}{-1}$ có một véc tơ chỉ phương có tọa độ là
\choice
{\True $(2;-3; 1)$}
{$(1;-2; 1)$}
{$(-2; 3; 1)$}
{$(-1; 2; 1)$}

\end{ex}
\begin{ex}%Câu 31.
Cho hình chóp $S.ABCD$ có đáy hình vuông cạnh $a, SA=\sqrt{6} a$ và vuông góc với mặt phẳng $(ABCD)$ (tham khảo hình vẽ). Góc giữa đường thẳng $SC$ và mặt phẳng $(ABCD)$ bằng
\choice
{\True $60^{\circ}$}
{$30^{\circ}$}
{$90^{\circ}$}
{$45^{\circ}$}

\end{ex}
\begin{ex}%Câu 32.
Cho hàm số $y=f(x)$ có đạo hàm $f'(x)=x^2\left(x^2-1\right)(x+2)$. Số điểm cực đại của hàm số $y=f(x)$ là
\choice
{\True $1$}
{$2$}
{$3$}
{$4$}

\end{ex}
\begin{ex}%Câu 33.
Tổng giá trị nhỏ nhất và giá trị lớn nhất của hàm số $y=-x^3+$ $3 x+3$ trên đoạn $[0; 2]$ bằng
\choice
{\True $6$}
{$4$}
{$8$}
{$5$}

\end{ex}
\begin{ex}%Câu 34.
Cho $1\neq a>0, b>0$ thỏa mãn $\log_2 a=b$ và $\log_a b=\dfrac{3}{b}$. Tổng $a+b$ bằng
\choice
{\True $264$}
{$18$}
{$70$}
{$256$}

\end{ex}
\begin{ex}%Câu 35.
Số giao điểm của đồ thị hàm số $y=x^4-x^2-2^{2020}$ với trục hoành là
\choice
{\True $2$}
{$4$}
{$0$}
{$3$}

\end{ex}
\begin{ex}%Câu 36.
Số nghiệm nguyên của bất phương trình $4^{x}-5\cdot 2^{x}+4<0$ 
\choice
{\True $1$}
{$2$}
{$0$}
{$3$}

\end{ex}
\begin{ex}%Câu 37.
Cắt hình nón bởi một mặt phẳng qua trục thu được thiết diện là một tam giác vuông có diện tích bằng $8$. Diện tích xung quanh của hình nón đã cho bằng
\choice
{$2\sqrt{2} \pi$}
{$4\sqrt{2} \pi$}
{\True $8\sqrt{2} \pi$}
{$16\sqrt{2} \pi$}

\end{ex}
\begin{ex}%Câu 38.
Cho $y=f(x)$ là một hàm số bất kỳ có đạo hàm trên $\mathbb{R}$, đặt $I=\displaystyle\displaystyle\int\limits_0^1 x f'(x) \mathrm{d} x.$ Khẳng định
\choice
{\True $I=f(1)+\displaystyle\displaystyle\int\limits_1^0 f(x) \mathrm{d} x$}
{$I=f(1)+\displaystyle\displaystyle\int\limits_0^1 f(x) \mathrm{d} x$}
{$I=\displaystyle\displaystyle\int\limits_1^0 f(x) \mathrm{d} x-f(1)$}
{$I=\displaystyle\displaystyle\int\limits_0^1 f(x) \mathrm{d} x-f(1)$}

\end{ex}
\begin{ex}%Câu 39.
Diện tích hình phẳng được gạch chéo như hình vẽ bằng
\choice
{\True $\displaystyle\displaystyle\int\limits_{-1}^3\left(-x^2+2 x+3\right)\mathrm{\,d}x$}
{$\displaystyle\displaystyle\int\limits_{-1}^3\left(x^2-2 x-3\right) \mathrm{d} x$}
{$\displaystyle\displaystyle\int\limits_{-1}^3\left(x^2+2 x-3\right)\mathrm{\,d}x$}
{$\displaystyle\displaystyle\int\limits_{-1}^3\left(-x^2+2 x-3\right)\mathrm{\,d}x$}

\end{ex}
\begin{ex}%Câu 40.
Gọi $A$ và $B$ lần lượt là điểm biểu diễn của số phức $z_1=3-2 i$ và $z_2=1+4 i$. Trung điểm của đoạn thẳng $AB$ có tọa độ là
\choice
{\True $(2; 1)$}
{$(4; 2)$}
{$(1;-3)$}
{$(2; 3)$}

\end{ex}
\begin{ex}%Câu 41.
Gọi $z_1, z_2$ là hai nghiệm phức của phương trình $z^2-2 z+3=0$. Mệnh đề nào dưới đây sai?
\choice
{\True $\left|z_1\right|+\left|z_2\right|=2$}
{$\left|z_1\right|=\left|z_2\right|$}
{$z_1 z_2=3$}
{$z_1+z_2=2$}

\end{ex}
\begin{ex}%Câu 42.
Phương trình $\log_2\left(x^2-9 x\right)=3$ có tích hai nghiệm bằng
\choice
{$9$}
{$27$}
{$3$}
{\True $-8$}

\end{ex}
\begin{ex}%Câu 43.
Cho hàm số $y=f(x)$ xác định, liên tục trên $\mathbb{R}$ và có đạo hàm $f'(x)$. Biết rằng $f'(x)$ có đồ thị như hình vẽ bên. Mệnh đề nào sau đây đúng?
\choice
{Hàm số đồng biến trên khoảng $(-2; 0)$}
{\True Hàm số nghịch biến trên khoảng $(0;+\infty)$}
{Hàm số đồng biến trên khoảng $(-\infty; 3)$}
{Hàm số nghịch biến trên khoảng $(-3;-2)$}

\end{ex}
\begin{ex}%Câu 44.
Từ thành phố $A$ đến thành phố $B$ có $2$ con đường, từ thành phố ${B}$ đến thành phố $C$ có $3$ con đường. Hỏi có bao nhiêu cách đi từ thành phố $A$ đến thành phố $C$ nhất định phải qua thành phố $B$?
\choice
{$5$}
{\True $6$}
{$2$}
{$3$}

\end{ex}

\begin{ex}%Câu 46.
Đường tiệm cận đứng và tiệm cận ngang của đồ thị hàm số $y=\dfrac{2 x-3}{x+1}$ tương ứng có phương trình là
\choice
{\True $x=-1$ và $y=2$}
{$x=1$ và $y=2$}
{$x=1$ và $y=-3$}
{$x=2$ và $y=1$}

\end{ex}
\begin{ex}%Câu 47.
Diện tích hình phẳng giới hạn bởi đồ thị hàm số $y=f(x), y=g(x)$ liên tục trên đoạn $[a; b]$ và hai đường thẳng $x=a, x=b$ được xác định theo công thức
\choice
{$S=\left|\displaystyle\displaystyle\int\limits_a^{b}[f(x)-g(x)] \mathrm{d} x\right|$}
{\True $S=\displaystyle\displaystyle\int\limits_a^{b}|f(x)-g(x)| \mathrm{d} x$}
{$S=\pi \displaystyle\displaystyle\int\limits_a^{b}|f(x)-g(x)| \mathrm{d} x$}
{$S=\displaystyle\displaystyle\int\limits_a^{b}[|f(x)|-|g(x)|] \mathrm{d} x$}

\end{ex}
\begin{ex}%Câu 48.
Điểm biểu diễn của số phức $z$ là $M(1; 2)$. Tọa độ của điểm biểu diễn số phức $w=z-2\bar{z}$ là
\choice
{$(2; 3)$}
{$(2; 1)$}
{\True $(-1; 6)$}
{$(2;-3)$}

\end{ex}
\begin{ex}%Câu 49.
Gọi $M, m$ lần lượt là giá trị lớn nhất, nhỏ nhất của hàm số $f(x)=  \dfrac{x+1}{x-1}$ trên $[-3;-1]$. Khi đó $M. m$ bằng
\choice
{$\dfrac{1}{2}$}
{$2$}
{$-4$}
{\True $0$}

\end{ex}
\begin{ex}%Câu 50.
Thể tích của khối lăng trụ tam giác đều có tất cả các cạnh đều bằng $a$ bằng
\choice
{$\dfrac{a^3 \sqrt{2}}{3}$}
{$\dfrac{a^3 \sqrt{2}}{2}$}
{$\dfrac{a^3 \sqrt{3}}{3}$}
{\True $\dfrac{a^3 \sqrt{3}}{4}$}

\end{ex}

\Closesolutionfile{ans}
%% \indapan{10}{ans/ans-de8-7}
