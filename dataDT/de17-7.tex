
\begin{name}
	{\tenchude}
	{\tendethi}
	{\tentruong}
	{\thoigian}
\end{name}
\Opensolutionfile{ans}[ans/ans-de17-7]

\begin{ex}%Câu 30.
Cho số phức $z=5+3 i$. Số phức liên hợp của $z$ là
\choice
{$-5+3 i$}
{$-5-3 i$}
{\True $5-3 i$}
{$5 i-3$}

\end{ex}
\begin{ex}%Câu 1.
Trên mặt phẳng tọa độ, điểm $M(-3; 2)$ là điểm biểu diễn của số phức nào dưới đây?
\choice
{$z_3=3-2 i$}
{$z_4=3+2 i$}
{$z_1=-3-2 i$}
{\True $z_2=-3+2 i$}

\end{ex}
\begin{ex}%Câu 2.
Trong không gian $O x y z$, cho mặt phẳng $(P)\colon-2 x+5 y+z-3=0$.
Vectơ nào dưới đây là một vectơ pháp tuyến của $(P)$?
\choice
{\True $\overrightarrow{n_2}=(-2; 5; 1)$}
{$\overrightarrow{n_1}=(2; 5; 1)$}
{$\overrightarrow{n_4}=(2; 5;-1)$}
{$\overrightarrow{n_3}=(2;-5; 1)$}

\end{ex}
\begin{ex}%Câu 3.
Trong không gian $O x y z$, cho điểm $A(4;-1; 3)$. Tọa độ của vecto $\overrightarrow{OA}$ là
\choice
{$(-4; 1; 3)$}
{\True $(4;-1; 3)$}
{$(-4; 1;-3)$}
{$(4; 1; 3)$}

\end{ex}
\begin{ex}%Câu 4.
\immini{
Đồ thị của hàm số nào dưới đây có dạng
như đường cong trong hình bên?
\choice
{$y=x^3-3 x+1$}
{$y=-2 x^4+4 x^2+1$}
{$y=-x^3+3 x+1$}
{\True $y=2 x^4-4 x^2+1$}}
{\vspace{-0.5cm}
\begin{tikzpicture}[scale=1, font=\footnotesize, line join=round, line cap=round, >=stealth,y=0.8cm,color=\mauchinh]
\def\xmin{-1.44}\def\xmax{1.44}\def\ymin{-1.2}\def\ymax{1.35}
\draw[->,thick] (\xmin-0.2,0)--(\xmax+0.2,0) node[below] {\footnotesize $x$};
\draw[->,thick] (0,\ymin-0.2)--(0,\ymax+0.2) node[right] {\footnotesize $y$};
\draw (0,0) node [below left] {\footnotesize $O$};
\foreach \x in {}\draw (\x,0.1)--(\x,-0.1) node [below] {\footnotesize $\x$};
\foreach \y in {}\draw (0.1,\y)--(-0.1,\y) node [left] {\footnotesize $\y$};
\clip (\xmin,\ymin) rectangle (\xmax,\ymax);
\draw[thick,smooth,samples=200,domain=\xmin:\xmax] plot (\x,{2*((\x)^4)+-4*((\x)^2)+1});
\end{tikzpicture}
}

\end{ex}
\begin{ex}%Câu 5.
Cho cấp số nhân $\left(u_n\right)$ với $u_1=3$ và $u_2=12$. Công bội của cấp số nhân đã cho bằng
\choice
{$9$}
{$-9$}
{$\dfrac{1}{4}$}
{\True $4$}

\end{ex}
\begin{ex}%Câu 6.
Cho $a>0$ và $a \neq 1$, khi đó $\log_a \sqrt[3]{a}$ bằng
\choice
{$-3$}
{\True $\dfrac{1}{3}$}
{$-\dfrac{1}{3}$}
{$3$}

\end{ex}
\begin{ex}%Câu 7.
Đồ thị của hàm số $y=-x^4-2 x^2+3$ cắt trục tung tại điểm có tung độ bằng
\choice
{$1$}
{$0$}
{$2$}
{\True $3$}

\end{ex}
\begin{ex}%Câu 8.
Cho hai số phức $z=5+2 i$ và $w=1-4 i$. Số phức $z+w$ bằng:
\choice
{$6+2 i$}
{$4+6 i$}
{\True $6-2 i$}
{$-4-6 i$}

\end{ex}
\begin{ex}%Câu 9.
Cho hàm số $f(x)={\rm e}^{x}+1$. Khẳng định nào dưới đây đúng?
\choice
{$\displaystyle\int f(x) \mathrm{d} x={\rm e}^{x-1}+C$}
{$\displaystyle\int f(x) \mathrm{d} x={\rm e}^{x}-x+C$}
{\True $\displaystyle\int f(x) \mathrm{d} x={\rm e}^{x}+x+C$}
{$\displaystyle\int f(x) \mathrm{d} x={\rm e}^{x}+C$}

\end{ex}
\begin{ex}%Câu 10.
Cho hàm số $f(x)$ có bảng xét dấu đạo hàm như sau:
\immini{

Số điểm cực trị của hàm số đã cho là

}
{

 \begin{nscenter}
  	\begin{tikzpicture}[scale=1,line width=.6pt,color=\mauchinh]
\tkzTabInit[nocadre=true,lgt=1,espcl=1.4,deltacl=0.5,lw=0.8]
{$x$ /.7,$y'$/.7}{$-\infty$,$-3$,$-3$,$3$,$5$,$+\infty$}
%{$x$ /.7,$f'(x)$/.7,$f(x)$/1.8}{$-\infty$,$0$,$3$,$+\infty$}
\tkzTabLine{,-,0,+,0,-,0,+,0,-,}
%\tkzTabVar{-/$3$,+D+/$1$/$2$,-/$-2$,+/$+\infty$}
\end{tikzpicture}
\end{nscenter}
}
\choice
{$5$}
{$3$}
{$2$}
{\True $4$}
\end{ex}
\begin{ex}%Câu 11.
Nếu $\displaystyle\int\limits_0^3 f(x) \mathrm{d} x=3$ thì $\displaystyle\int\limits_0^3 2 f(x) \mathrm{d} x$ bằng
\choice
{$3$}
{$18$}
{$2$}
{\True $6$}

\end{ex}
\begin{ex}%Câu 12.
Tiệm cận đứng của đồ thị hàm số $y=\dfrac{x+1}{x-2}$ là đường thẳng có phương trình:
\choice
{$x=-1$}
{$x=-2$}
{\True $x=2$}
{$x=1$}

\end{ex}
\begin{ex}%Câu 13.
Trong không gian $O x y z$ cho mặt cầu $(S)$ có tâm $I(0;-2; 1)$ và bán kính bằng 2. Phương trình của $(S)$ là:
\choice
{$x^2+(y+2)^2+(z-1)^2=2$}
{$x^2+(y-2)^2+(z+1)^2=2$}
{$x^2+(y-2)^2+(z+1)^2=4$}
{\True $x^2+(y+2)^2+(z-1)^2=4$}

\end{ex}
\begin{ex}%Câu 14.
Phần thực của số phức $z=6-2 i$ bằng
\choice
{$-2$}
{$2$}
{\True $6$}
{$-6$}

\end{ex}
\begin{ex}%Câu 15.
Tập nghiệm của bất phương trình $2^{x}<5$ là
\choice
{\True $\left(-\infty; \log_2 5\right)$}
{$\left(\log_2 5;+\infty;\right)$}
{$\left(-\infty; \log_5 2\right)$}
{$\left(\log_5 2;+\infty\right)$}

\end{ex}
\begin{ex}%Câu 16.
Nghiệm của phương trình $\log_5(3 x)=2$ là:
\choice
{$25$}
{$\dfrac{32}{3}$}
{$32$}
{\True $\dfrac{25}{3}$}

\end{ex}
\begin{ex}%Câu 17.
Cho khối trụ có bán kính đáy $r=4$ và chiều cao $h=3$. Thể tích của khối trụ đã cho bằng
\choice
{$16\pi$}
{\True $48\pi$}
{$36\pi$}
{$12\pi$}

\end{ex}
\begin{ex}%Câu 18.
Cho hình lăng trụ đứng $ABC \cdot A'B'C'$ có tất cả các cạnh bằng nhau (tham khảo hình bên). Góc giữa hai đường thẳng $AA'$ và $B'C$ bằng
\choice
{$90^{\circ}$}
{\True $45^{\circ}$}
{$30^{\circ}$}
{$60^{\circ}$}

\end{ex}
\begin{ex}%Câu 19.
Trong không gian, cho hai điểm $A(0; 0; 1)$ và $B(2; 1; 3)$. Mặt phẳng đi qua $A$ và vuông góc với $AB$ có phương trình là
\choice
{$2 x+y+2 z-11=0$}
{\True $2 x+y+2 z-2=0$}
{$2 x+y+4 z-4=0$}
{$2 x+y+4 z-17=0$}

\end{ex}
\begin{ex}%Câu 20.
Từ một hộp chứa $10$ quả bóng gồm $4$ quả màu đỏ và $6$ quả màu xanh, lấy ngẫu nhiên đồng thời $3$ quả. Xác suất để lấy được $3$ quả màu xanh bằng
\choice
{\True $\dfrac{1}{6}$}
{$\dfrac{1}{30}$}
{$\dfrac{3}{5}$}
{$\dfrac{2}{5}$}

\end{ex}
\begin{ex}%Câu 21.
Số phức $z$ thỏa mãn $i z=6+5 i$. Số phức liên hợp của $z$ là
\choice
{$\bar{z}=5-6 i$}
{$\bar{z}=-5+6 i$}
{\True $\bar{z}=5+6 i$}
{$\bar{z}=-5-6 i$}

\end{ex}
\begin{ex}%Câu 22.
Biết hàm số $y=\dfrac{x+a}{x+1}$ ($a$ là số thực cho trước, $a \neq 1$) có đồ thị $(C)$. Mệnh đề nào dưới đây đúng?
\choice
{$y'<0$ khi $a>\leq 1$}
{$y'<0$ khi $a<1$}
{\True $y'<0$ khi $a>1$}
{$y'<0$ khi $a \geq 1$}

\end{ex}
\begin{ex}%Câu 23.
Trong không gian $O x y z$, cho điểm $M(2; 1;-1)$ và mặt phẳng $(P)$: $x-3 y+2 z+1=0$.Đường thẳng đi qua $M$ và vuông góc với $(P)$ có phương trình là:
\choice
{$\dfrac{x-2}{1}=\dfrac{y-1}{-3}=\dfrac{z+1}{1}$}
{\True $\dfrac{x-2}{1}=\dfrac{y-1}{-3}=\dfrac{z+1}{2}$}
{$\dfrac{x+2}{1}=\dfrac{y+1}{-3}=\dfrac{z-1}{1}$}
{$\dfrac{x+2}{1}=\dfrac{y+1}{-3}=\dfrac{z-1}{2}$}

\end{ex}
\begin{ex}%Câu 24.
Trên đoạn $[-2; 1]$, hàm số $y=x^3-3 x^2-1$ đạt giá trị lớn nhất tại điểm
\choice
{$x=-2$}
{\True $x=0$}
{$x=-1$}
{$x=1$}

\end{ex}
\begin{ex}%Câu 25.
Khối trụ tròn xoay có thể tích bằng $144\pi$ và có bán kính đáy bằng $6$. Đường sinh của khối trụ bằng
\choice
{\True $4$}
{$6$}
{$12$}
{$10$}

\end{ex}
\begin{ex}%Câu 26.
Trong các hàm số sau đây, hàm số nào nghịch biến trên tập $\mathbb{R}$?
\choice
{$y=\pi^{x}$}
{\True $y=\left(\dfrac{1}{3}\right)^{x}$}
{$y=\sqrt{3}^{x}$}
{$y=3^{x}$}

\end{ex}
\begin{ex}%Câu 27.
Giá trị của tích phân $\displaystyle\int\limits_0^2 2 x \mathrm{\,d} x$ bằng
\choice
{$8$}
{$6$}
{$2$}
{\True $4$}

\end{ex}

\begin{ex}%Câu 28.
Trong không gian $O x y z$, cho mặt phẳng $(P)\colon 3 x-2 y+z-2020=0$. Vectơ nào dưới đây là một vectơ pháp tuyến của mặt phẳng $(P)$ 
\choice
{\True $\overrightarrow{n_2}(3;-2; 1)$}
{$\overrightarrow{n_3}(3; 2; 1)$}
{$\overrightarrow{n_4}(3;-2;-1)$}
{$\vec{n}_1(3; 2;-1)$}

\end{ex}
\begin{ex}%Câu 29.
Trong không gian $O x y z$, cho mặt phẳng $(P)\colon x-2 y+3 z+2020=0$. Vectơ nào dưới đây không phải là một vectơ pháp tuyến của mặt phẳng $(P)?$ 
\choice
{$\vec{n}=(-2; 4;-6)$}
{$\vec{n}=(-1; 2;-3)$}
{$\vec{n}=(1;-2; 3)$}
{\True $\vec{n}=(-2; 3; 2020)$}

\end{ex}

\begin{ex}%Câu 31.
Trong mặt phẳng $(O x y)$, diểm $M$ biểu diễn số phức $z=-1-3 i$ có tọa độ là
\choice
{$M(1;-3)$}
{\True $M(-1;-3)$}
{$M(-1; 3)$}
{$M(1; 3)$}

\end{ex}
\begin{ex}%Câu 32.
Cho các số thực dương $a, b$ và $a \neq 1$. Biểu thức $\log_a a^2 b$ bằng
\choice
{$2\left(1+\log_a b\right)$}
{$2\log_a b$}
{\True $2+\log_a b$}
{$1+\log_a b$}

\end{ex}
\begin{ex}%Câu 33.
Thể tích khối lăng trụ tam giác có chiều cao bằng $2$, cạnh đáy lần lượt bằng $3,4,5$ là:
\choice
{$8$}
{\True $12$}
{$4$}
{$28$}

\end{ex}

\begin{ex}%Câu 34.
Cho khối lăng trụ có diện tích đáy $B=6$ và chiều cao $h=4$. Thể tích của khối lăng trụ đã cho bằng
\choice
{$8$}
{$12$}
{\True $24$}
{$4$}

\end{ex}
\begin{ex}%Câu 35.
Họ tất cả các nguyên hàm của hàm số $f(x)=\sin x+\dfrac{2}{x}$ là
\choice
{$\cos x+2\ln |x|+C$}
{$\cos x-\dfrac{2}{x^2}+C$}
{\True $-\cos x+2\ln |x|+C$}
{$-\cos x-2\ln |x|+C$}

\end{ex}
\begin{ex}%Câu 36.
Trong không gian với hệ tọa độ $O x y z$, cho $\vec{a}=-2\vec{i}+3\vec{j}+5\vec{k}$. Tọa độ của $\vec{a}$ là
\choice
{$(2; 3; 5)$}
{\True $(-2; 3; 5)$}
{$(2; 3;-5)$}
{$(2;-3;-5)$}

\end{ex}
\begin{ex}%Câu 37.
Cho 2 số thực dương $x, y$ thỏa mãn $x \neq 1$ và $\log_x y=3$. Tính $T=\log_{x^3} y^5$.
\choice
{$T=\dfrac{5}{3}$}
{$T=\dfrac{9}{5}$}
{$T=\dfrac{3}{5}$}
{\True $T=5$}

\end{ex}
\begin{ex}%Câu 38.
Trong không gian $O x y z$, cho điểm $M(2;-1; 3)$ và mặt phẳng $(\alpha)$: $2 x-5 y+z-1=0$. Phương trình mặt phẳng nào dưới đây đi qua điểm $M$ và song song với $(\alpha)$.
\choice
{\True $2 x-5 y+z-12=0$}
{$2 x-5 y-z-12=0$}
{$2 x+5 y-z-12=0$}
{$2 x-5 y+z+12=0$}

\end{ex}
\begin{ex}%Câu 39.
\immini{
Cho hàm số $y=f(x)$ 
có đồ thị như hình vẽ bên.
Hàm số đã cho nghịch biến trên
khoảng
\choice
{$(0; 2)$}
{$(-3;-1)$}
{\True $(-1; 0)$}
{$(1; 3)$}}
{

\vspace{-0.5cm}
\begin{nscenter}
 \begin{tikzpicture}[>=stealth, scale=.8,samples=200,smooth,color=\mauchinh,line width=.6pt,xscale=1,yscale=.7]
\tikzstyle{every node}=[font=\small]
\pgfmathsetmacro{\a}{sqrt(2)}
 \draw[->,thick] (-4,0)--(4,0) node[above] {$x$};
 \draw[->,thick] (0,-4)--(0,4) node[right] {$y$};
 \draw (0,0) node [above right] {$O$};
 \foreach \x in {-1,2}
\draw[thin] (\x,1pt)--(\x,-1pt) node [below] {$\x$};
 \foreach \y in {-3,3}
 	\draw[thin] (1pt,\y)--(-1pt,\y) node [left] {$\y$};
 \begin{scope}
\draw[domain=-3:1,smooth,variable=\x]
plot (\x,{-(\x)^2-2*(\x)});
\draw[domain=1:3,smooth,variable=\x]
plot (\x,{-6*(\x)^2+24*(\x)-21});
\end{scope}
\draw[dashed](-3,0)|-(-3,-3) (-1,0)|-(0,1) (1,0)|-(0,-3) (2,0)|-(0,3) (3,0)|-(3,-3);
\path
(-3,0)  node[above, xshift=0cm]{$-3$}
(1,0)  node[above, xshift=0cm]{$1$}
(3,0)  node[above, xshift=0cm]{$3$}
(0,1)  node[right, yshift=0cm]{$1$}
;
 \end{tikzpicture}
\end{nscenter}
}
\end{ex}
\begin{ex}%Câu 40.
Đồ thị hàm số $y=\dfrac{x+2}{2 x+1}$ có đường tiệm cận ngang là đường thẳng nào sau đây?
\choice
{$x=-1$}
{$y=2$}
{\True $y=\dfrac{1}{2}$}
{$x=\dfrac{-1}{2}$}

\end{ex}
\begin{ex}%Câu 41.
Gọi $S$ là tập nghiệm của phương trình $9^{x}-10.3^{x}+9=0$. Tổng các phần tử của $S$ bằng
\choice
{$1$}
{\True $2$}
{$10$}
{$\dfrac{10}{3}$}

\end{ex}

\begin{ex}%42
Cho hàm số $f(x)$ có đạo hàm $f'(x)=x(x+1)(x-4)^3, \forall x \in \mathbb{R}$. Số điểm cực tiểu của hàm số đã cho là
\choice
{$4$}
{$1$}
{$3$}
{\True $2$}
\end{ex}
\begin{ex}%Câu 43.
Cho hàm số $y=f(x)$ có đạo hàm $f'(x)=1, \forall x \in \mathbb{R}$. Mệnh đề nào sau đây đúng?
\choice
{\True $f(-1)<f(2)$}
{$f(-1)=f(2)$}
{$f(-1) \geq f(2)$}
{$f(-1)>f(2)$}

\end{ex}
\begin{ex}%Câu 44.
\immini{
Cho hàm số $y=f(x)$ liên
tục trên $\mathbb{R}$ và có đồ thị như hình vẽ
bên. Biết rằng diện tích miền tô đậm
bằng $\dfrac{37}{12}$ và $\displaystyle\int\limits_{-2}^0 f(x) \mathrm{d} x=\dfrac{14}{3}$. Tính $I=\displaystyle\int\limits_{1}^{\rm e}\dfrac{f\left(\ln x\right)}{x}{\rm d}x$. 
}
{
\vspace{-0.5cm}
\begin{nscenter}
 \begin{tikzpicture}[>=stealth, scale=1,samples=200,smooth,color=\mauchinh,line width=.6pt,xscale=1,yscale=.8]
\tikzstyle{every node}=[font=\small]
\pgfmathsetmacro{\a}{sqrt(2)}
 \draw[->,thick] (-2.5,0)--(2.8,0) node[above] {$x$};
 \draw[->,thick] (0,-2.5)--(0,3.7) node[right] {$y$};
 \draw (0,0) node [above right] {$O$};
 \foreach \x in {-2,-1,1}
\draw[thin] (\x,1pt)--(\x,-1pt) node [below] {$\x$};
 \foreach \y in {-2}
 	\draw[thin] (1pt,\y)--(-1pt,\y) node [left] {$\y$};
 \begin{scope}
\draw[domain=-2:1,smooth,variable=\x]
plot (\x,{(\x)^3+(\x)^2-(\x)+2});
\draw[domain=1:2,smooth,variable=\x]
plot (\x,{-2*(\x)^2+(\x)+4});
\draw[domain=-2:1,smooth,variable=\x]
plot (\x,{(\x)+2});
\fill[pattern=north east lines,smooth,pattern color=blue] plot[domain=-2:1] (\x,{(\x)^3+(\x)^2-(\x)+2}) -- plot[domain=1:-2] (\x,{(\x)+2}) -- cycle; 
\end{scope}
\draw[dashed](-1,0)|-(0,3) (1,0)|-(0,3) (2,0)|-(0,-2);
\path
(2,0)  node[above, xshift=0cm]{$2$}
(0,3)  node[above, xshift=0.1cm]{$3$}
;

 \end{tikzpicture}
\end{nscenter}

}
\choice
{$\dfrac{12}{25}$}
{\True $\dfrac{25}{12}$}
{$\dfrac{8}{3}$}
{$\dfrac{3}{8}$}
\end{ex}
\begin{ex}%Câu 45.
Một cấp số nhân có số hạng thứ $3$ và số hạng thứ $6$ lần lượt là $9$ và $-243$. Khi đó số hạng thứ $8$ của cấp số nhân bằng:
\choice
{$2187$}
{\True $-2187$}
{$729$}
{$243$}

\end{ex}
\begin{ex}%Câu 46.
Tìm hàm số $F(x)$ không là nguyên hàm của hàm số $f(x)=$ $\sin 2 x$.
\choice
{$F(x)=-\cos ^2 x$}
{$F(x)=\sin ^2 x$}
{$F(x)=-\dfrac{1}{2} \cos 2 x$}
{\True $F(x)=-\cos 2 x$}

\end{ex}
\begin{ex}%Câu 47.
\immini{
Cho hàm số $f(x)$ xác định, liên tục trên $\mathbb{R}$ và có bảng biến thiên như hình bên. Đồ thị hàm số $y=f(x)$ cắt đường thẳng $y=-2$ tại bao nhiêu điểm?
}
{\vspace{-0.5cm}
\begin{tikzpicture}[color=\mauchinh]
\tkzTabInit[nocadre,lgt=1.2,espcl=1.4,deltacl=0.6]
{$x$/0.6,$f'(x)$/0.6,$f(x)$/1.8}{$-\infty$,$-1$,$0$,$1$,$+\infty$}
\tkzTabLine{,+,0,-,0,+,0,-,}
\tkzTabVar{-/$-\infty$,+/$3$,-/$-1$,+/$3$,-/$-\infty$}
\end{tikzpicture}
}
\choice
{$0$}
{\True $2$}
{$1$}
{$4$}
\end{ex}
\begin{ex}%Câu 48.
Trong không gian $O x y z$, cho hai điểm $A(-2; 1; 0), B(2; 5;-4)$.
Phương trình mặt cầu đường kính $AB$ là
\choice
{$(x+2)^2+(y-1)^2+z^2=12$}
{$x^2+(y-3)^2+(z+2)^2=48$}
{$(x-4)^2+(y-4)^2+(z+4)^2=48$}
{\True $x^2+(y-3)^2+(z+2)^2=12$}

\end{ex}
\begin{ex}%Câu 49.
Tập nghiệm của bất phương trình $\log_5(3 x+1)<\log_5(25-25 x)$ là
\choice
{$\left(-\dfrac{1}{3}; 1\right)$}
{$\left(-\infty; \dfrac{6}{7}\right)$}
{\True $\left(-\dfrac{1}{3}; \dfrac{6}{7}\right)$}
{$\left(\dfrac{6}{7}; 1\right)$}

\end{ex}
\begin{ex}%Câu 50.
Cho hàm số $y=f(x)$ liên tục trên $[-3; 3]$ và có bảng xét dấu của đạo hàm như sau:
Mệnh đề nào sau đây sai?
\choice
{Hàm số đạt cực tiểu tại $x=1$}
{Hàm số đạt cực đại tại $x=2$}
{Hàm số đạt cực đại tại $x=-1$}
{\True Hàm số đạt cực tiểu tại $x=0$}

\end{ex}

\Closesolutionfile{ans}
%\indapan{10}{ans/ans-de17-7}
