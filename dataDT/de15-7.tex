\begin{name}
{\tenchude}{\tendethi}{\tentruong}{\thoigian}
\end{name}
\Opensolutionfile{ans}[ans/ans-de15-7]

\begin{ex}%Câu 28.
Tìm phần ảo của số phức $z=i(3+8 i)$ 
\choice
{$8$}
{$-8$}
{$3 i$}
{\True $3$}

\end{ex}
\begin{ex}%Câu 1.
\immini{
Trên mặt phẳng tọa độ $O x y$, cho phần hình phẳng được tô đậm như hình bên được giới hạn bởi một đồ thị hàm số đa thức bậc ba và một đường thẳng. Diện tích $S$ của phần tô đậm đó bằng bao nhiêu?
\choice
{$S=8(\mathrm{dvdt})$}
{$S=6(\mathrm{dvdt})$}
{$S=2(\mathrm{dvdt})$}
{\True $S=4(\mathrm{dvdt})$}}
{\begin{tikzpicture}[>=stealth, scale=.8,samples=200,smooth,color=\mauchinh,line width=.6pt,xscale=1,yscale=.9]
\tikzstyle{every node}=[font=\small]
\pgfmathsetmacro{\a}{sqrt(2)}
 \draw[->,thick] (-2.5,0)--(2.5,0) node[above] {$x$};
 \draw[->,thick] (0,-2.5)--(0,2.9) node[right] {$y$};
 \draw (0,0) node [above right] {$O$};
 \foreach \x in {-1}
\draw[thin] (\x,1pt)--(\x,-1pt) node [below] {$\x$};
 \foreach \y in {}
 	\draw[thin] (1pt,\y)--(-1pt,\y) node [left] {$\y$};
 \begin{scope}
\draw[domain=-2.06:2.06,smooth,variable=\x]
plot (\x,{(\x)^3-3*(\x)});
\draw[domain=-2.32:2.5,smooth,variable=\x]
plot (\x,{(\x)});
\fill[pattern=north east lines,smooth,pattern color=\mauchinh] plot[domain=0:2] (\x,{(\x)^3-3*(\x)}) -- plot[domain=0:2] (\x,{(\x)}) -- cycle; 
\end{scope}
\draw[dashed](-2,0)|-(0,-2) (-1,0)|-(0,2) (1,0)|-(0,-2) (2,0)|-(0,2);
\path
(-2,0)  node[above, xshift=-0.2cm]{$-2$}
(2,0)  node[below, xshift=0.1cm]{$2$}
(1,0)  node[above, yshift=0cm]{$1$}
(0,2)  node[above, xshift=0.2cm]{$2$}
(0,-2)  node[below, xshift=0.2cm]{$-2$}
;

 \end{tikzpicture}
}

\end{ex}
\begin{ex}%Câu 2.
Trong không gian $O x y z$, cho hai điểm $A(-1; 1; 0)$ và $B(3; 5;-2)$. Tọa độ trung điểm của đoạn thẳng $AB$ là
\choice
{$(2; 2;-1)$}
{$(2; 6;-2)$}
{$(4; 4;-2)$}
{\True $(1; 3;-1)$}

\end{ex}
\begin{ex}%Câu 3.
\immini{
Cho hàm số $y=f(x)$ có đồ thị như hình vẽ bên. Số giá trị nguyên của tham số $m$ để đường thẳng $y=m$ cắt đồ thị hàm số đã cho tại ba điểm phân biệt là
\choice
{$\text{Vô số}$}
{\True $3$}
{$0$}
{$5$}
}
{\vspace{-0.5cm}
\begin{tikzpicture}[scale=0.7, font=\footnotesize, line join=round, line cap=round, >=stealth,y=0.7cm,color=\mauchinh]
\def\xmin{-1.06}\def\xmax{3.16}\def\ymin{-0.65}\def\ymax{5.61}
\draw[->,thick] (\xmin-0.2,0)--(\xmax+0.2,0) node[below] {\footnotesize $x$};
\draw[->,thick] (0,\ymin-0.2)--(0,\ymax+0.2) node[right] {\footnotesize $y$};
\draw (0,0) node [below left] {\footnotesize $O$};
\foreach \x in {2}\draw (\x,0.1)--(\x,-0.1) node [below] {\footnotesize $\x$};
\foreach \y in {1,5}\draw (0.1,\y)--(-0.1,\y) node [left] {\footnotesize $\y$};
\clip (\xmin,\ymin) rectangle (\xmax,\ymax);
\draw[thick,smooth,samples=200,domain=\xmin:\xmax] plot (\x,{-1*((\x)^3)+3*((\x)^2)+0*(\x)+1});
\draw[dashed] (2,0)--(2,5)--(0,5);\fill (2,5) circle (1pt);
\end{tikzpicture}
}
\end{ex}
\begin{ex}%Câu 4.
Tập nghiệm của bất phương trình $4^{x^2-2 x} \geq 64$ là
\choice
{\True $(-\infty;-1] \cup[3;+\infty)$}
{$[3;+\infty)$}
{$(-\infty;-1]$}
{$[-1; 3]$}

\end{ex}
\begin{ex}%Câu 5.
Cho hình nón có thiết diện qua trục là tam giác vuông cân có cạnh huyền bằng $a \sqrt{2}$. Diện tích xung quanh của hình nón đã cho bằng
\choice
{$\pi a^2 \sqrt{2}$}
{$\dfrac{\pi a^2}{2}$}
{$\pi a^2$}
{\True $\dfrac{\pi a^2 \sqrt{2}}{2}$}

\end{ex}
\begin{ex}%Câu 6.
Cho hàm số $y=\dfrac{2 x+1}{x-1}$. Tích giá trị lớn nhất và giá trị nhỏ nhất của hàm số đã cho trên đoạn $[-1; 0]$ bằng
\choice
{$\dfrac{3}{2}$}
{$2$}
{\True $\dfrac{-1}{2}$}
{$0$}


\end{ex}
\begin{ex}%Câu 7.
\immini{
Cho hàm số $y= f(x)$ có bảng biến thiên như hình bên. Tổng số đường tiệm cận đứng và tiệm cận ngang của đồ thị hàm số bằng
\choice
{$4$}
{$1$}
{\True $2$}
{$3$}}
{\begin{tikzpicture}[scale=1,line width=.6pt,color=\mauchinh]
\tkzTabInit[nocadre=true,lgt=1,espcl=1.6,deltacl=0.5,lw=0.8]
{$x$ /.7,$f'(x)$/.7,$f(x)$/1.8}{$-\infty$,$-1$,$2$,$+\infty$}
\tkzTabLine{,+,d,-,d,+,}
\tkzTabVar{-/$3$,+D+/$+\infty$/$4$,-/$-5$,+/$+\infty$}
\end{tikzpicture}
}

\end{ex}
\begin{ex}%Câu 8.
Số nghiệm của phương trình $\log_3(x+2)+\log_3(x-2)=\log_3 5$ là
\choice
{$2$}
{$3$}
{\True $1$}
{$0$}

\end{ex}
\begin{ex}%Câu 9.
Cho hình chóp $S.ABCD$ có đáy là hình vuông cạnh $a, SA$ vuông góc với mặt phẳng đáy và $SA=a \sqrt{2}$ (tham khảo hình vẽ). Góc giữa đường thẳng $SC$ và mặt phẳng $(ABCD)$ bằng
\choice
{$30^{\circ}$}
{\True $45^{\circ}$}
{$60^{\circ}$}
{$90^{\circ}$}

\end{ex}
\begin{ex}%Câu 10.
Cho hàm số $y=f(x)$ có đạo hàm $f'(x)=x(x+3)(x-1)^2$. Số điểm cực trị của hàm số bằng
\choice
{$0$}
{\True $2$}
{$3$}
{$1$}

\end{ex}
\begin{ex}%Câu 11.
Họ tất cả nguyên hàm của hàm số $f(x)=\dfrac{1}{x}\left(1+\dfrac{x}{\cos ^2 x}\right)$ với $x \in(0;+\infty) \backslash\left\{\dfrac{\pi}{2}+k \pi, k \in \mathbb{Z}\right\}$ là
\choice
{$-\dfrac{1}{x^2}+\tan x+C$}
{\True $\ln x+\tan x+C$}
{$-\dfrac{1}{x^2}-\tan x+C$}
{$\ln x-\tan x+C$}

\end{ex}
\begin{ex}%Câu 12.
Cho khối lăng trụ đứng $ABC \cdot A'B'C'$ có đáy là tam giác vuông tại $B, AB=a, AC=a \sqrt{5}, AA'=2 a \sqrt{3}$. Thể tích khối lăng trụ đã cho bằng
\choice
{\True $2\sqrt{3} a^3$}
{$4\sqrt{3} a^3$}
{$\dfrac{2\sqrt{3} a^3}{3}$}
{$\dfrac{\sqrt{3} a^3}{3}$}

\end{ex}
\begin{ex}%Câu 13.
Trong không gian $O x y z$, cho các vectơ $\vec{a}=(-2;-3; 1)$ và $\vec{b}=  (1; 0; 1)$. Côsin góc giữa hai vectơ $\vec{a}$ và $\vec{b}$ bằng
\choice
{\True $-\dfrac{1}{2\sqrt{7}}$}
{$\dfrac{1}{2\sqrt{7}}$}
{$-\dfrac{3}{2\sqrt{7}}$}
{$\dfrac{3}{2\sqrt{7}}$}

\end{ex}
\begin{ex}%Câu 14.
\immini{
Cho hàm số $y=f(x)$ có bảng biến thiên như hình bên dưới. Số nghiệm của phương trình $2 f(x)-11=0$ bằng
}
{
\begin{tikzpicture}[color=\mauchinh]
\tkzTabInit[nocadre,lgt=1.2,espcl=1.4,deltacl=0.6]
{$x$/0.6,$f'(x)$/0.6,$f(x)$/1.8}{$-\infty$,$-\sqrt{6}$,$0$,$\sqrt{6}$,$+\infty$}
\tkzTabLine{,-,0,+,0,-,0,+,}
\tkzTabVar{+/$+\infty$,-/$-4$,+/$5$,-/$-4$,+/$+\infty$}
\end{tikzpicture}
}
\choice
{$3$}
{\True $2$}
{$0$}
{$4$}
\end{ex}
\begin{ex}%Câu 15.
Cho hình chóp $S.ABCD$ có đáy $ABCD$ là hình chữ nhật tâm $O$, cạnh $AB=a, AD=a \sqrt{2}$. Hình chiếu vuông góc của $S$ trên mặt phẳng $(ABCD)$ là trung điểm của đoạn $OA$. Góc giữa $SC$ và mặt phẳng $(ABCD)$ bằng $30^{\circ}$. Khoảng cách từ $C$ đến mặt phẳng $(SAB)$ bằng
\choice
{$\dfrac{9\sqrt{22} a}{44}$}
{\True $\dfrac{3\sqrt{22} a}{11}$}
{$\dfrac{\sqrt{22} a}{11}$}
{$\dfrac{3\sqrt{22} a}{44}$}

\end{ex}
\begin{ex}%Câu 16.
Cho phương trình $16^{x^2}-2\cdot 4^{x^2+1}+10=m$ ($m$ là tham số). Số giá trị nguyên của $m \in[-10; 10]$ để phương trình đã cho có đúng $2$ nghiệm thực phân biệt là
\choice
{$7$}
{$9$}
{\True $8$}
{$1$}

\end{ex}
\begin{ex}%Câu 17.
Trong không gian $O x y z$, cho điểm $I(2; 4;-3)$. Phương trình mặt cầu có tâm $I$ và tiếp xúc với mặt phẳng $(O x z)$ là
\choice
{$(x-2)^2+(y-4)^2+(z+3)^2=4$}
{$(x-2)^2+(y-4)^2+(z+3)^2=29$}
{$(x-2)^2+(y-4)^2+(z+3)^2=9$}
{\True $(x-2)^2+(y-4)^2+(z+3)^2=16$}

\end{ex}
\begin{ex}%Câu 18.
\immini{
Đường cong trong hình vẽ bên dưới là đồ thị của hàm số nào trong các hàm số dưới đây?
\choice
{$y=-x^3+2 x^2-x-3$}
{$y=x^3+2 x^2-7 x-2$}
{\True $y=x^3-2 x^2+x-2$}
{$y=x^4-2 x^2-3$}}
{\vspace{-0.5cm}
\begin{tikzpicture}[scale=1, font=\footnotesize, line join=round, line cap=round, >=stealth,y=0.7cm,color=\mauchinh]
\def\xmin{-0.59}\def\xmax{2.16}\def\ymin{-3.5}\def\ymax{1}
\draw[->,thick] (\xmin-0.2,0)--(\xmax+0.2,0) node[below] {\footnotesize $x$};
\draw[->,thick] (0,\ymin-0.2)--(0,\ymax+0.2) node[right] {\footnotesize $y$};
\draw (0,0) node [below left] {\footnotesize $O$};
\foreach \x in {}\draw (\x,0.1)--(\x,-0.1) node [below] {\footnotesize $\x$};
\foreach \y in {}\draw (0.1,\y)--(-0.1,\y) node [left] {\footnotesize $\y$};
\clip (\xmin,\ymin) rectangle (\xmax,\ymax);
\draw[thick,smooth,samples=200,domain=\xmin:\xmax] plot (\x,{1*((\x)^3)+-2*((\x)^2)+1*(\x)+-2});
\end{tikzpicture}
}

\end{ex}
\begin{ex}%Câu 19.
Trong không gian với hệ trục tọa độ $O x y z$, viết phương trình mặt cầu $(S)$ biết rằng $(S)$ có một đường kính là $MN$ với $M(2; 5; 6), N(0;-1; 2)$.
\choice
{$(x-1)^2+(y-2)^2+(z-4)^2=56$}
{\True $(x-1)^2+(y-2)^2+(z-4)^2=14$}
{$(x+1)^2+(y+2)^2+(z+4)^2=14$}
{$(x+1)^2+(y+2)^2+(z+4)^2=56$}

\end{ex}
\begin{ex}%Câu 20.
\immini{
Cho hàm số $y=f(x)$ xác định trên $\mathbb{R}$ và có bảng biến thiên như hình vẽ bên. Hỏi hàm số $y=f(x)$ nghịch biến trên khoảng nào?
}
{\begin{tikzpicture}[scale=1,line width=.6pt,color=\mauchinh]
\tkzTabInit[nocadre=true,lgt=1.1,espcl=1.6,deltacl=0.5,lw=0.8]
{$x$ /.7,$f'(x)$/.7,$f(x)$/1.8}{$-\infty$,$0$,$3$,$+\infty$}
\tkzTabLine{,+,d,-,0,+,}
\tkzTabVar{-/$3$,+D+/$1$/$2$,-/$-2$,+/$+\infty$}
\end{tikzpicture}
}
\choice
{$(-3; 1)$}
{$(3;+\infty)$}
{$(-2; 2)$}
{\True $(0; 3)$}
\end{ex}
\begin{ex}%Câu 21.
Cho số phức $z=\dfrac{1}{i}$. Số phức liên hợp của $z$ là
\choice
{$-1$}
{\True $i$}
{$-i$}
{$1$}

\end{ex}
\begin{ex}%Câu 22.
Trong không gian $O x y z$, cho đường thẳng $d\colon\left\{\begin{aligned}&x=2-t \\& y=3\\& z=-1+2 t\end{aligned} \quad(t \in\right. \mathbb{R})$. Vectơ nào sau đây là một vectơ chỉ phương của $d$?
\choice
{\True $\overrightarrow{u_2}=(2; 0;-4)$}
{$\overrightarrow{u_4}=(-1; 0;-2)$}
{$\overrightarrow{u_3}=(-1; 3; 2)$}
{$\overrightarrow{u_1}=(2; 3;-1)$}

\end{ex}
\begin{ex}%Câu 23.
Cho $x, y$ là các số thực thỏa mãn $x \neq 0$ và $\left(3^{x^2}\right)^{3 y}=27^{x}$. Khẳng định nào sau đây là khẳng định đúng?
\choice
{$x^2+3 y=3 x$}
{$3 x y=1$}
{$x^2 y=1$}
{\True $x y=1$}

\end{ex}
\begin{ex}%Câu 24.
Cắt một khối cầu bởi một mặt phẳng đi qua tâm thì được một hình tròn có diện tích bằng $16\pi$. Tính diện tích của mặt cầu giới hạn nên khối cầu đó
\choice
{$16\pi$}
{$4\pi$}
{\True $64\pi$}
{$\dfrac{256\pi}{3}$}

\end{ex}
\begin{ex}%Câu 25.
Đường cao của một hình nón có đường $\sinh$ bằng $7\mathrm{\,cm}$ và đường kính đáy bằng $6\mathrm{\,cm}$ là
\choice
{$1\mathrm{\,cm}$}
{$\sqrt{13} \mathrm{\,cm}$}
{\True $2\sqrt{10} \mathrm{\,cm}$}
{$4\mathrm{\,cm}$}

\end{ex}
\begin{ex}%Câu 26.
Tính mô-đun của số phức $z=5-2 i$ 
\choice
{\True $\sqrt{29}$}
{$7$}
{$\sqrt{21}$}
{$29$}

\end{ex}
\begin{ex}%Câu 27.
Cho hình chóp $S.ABC$ có đáy $ABC$ là tam giác vuông cân ở $B$, cạnh $AC=2 a$. Cạnh $SA$ vuông góc với mặt đáy $(ABC)$, tam giác $SAB$ cân. Tính thể tích hình chóp $S.ABC$ theo $a$.
\choice
{\True $\dfrac{a^3 \sqrt{2}}{3}$}
{$a^3 \sqrt{2}$}
{$2 a^3 \sqrt{2}$}
{$\dfrac{2 a^3 \sqrt{2}}{3}$}

\end{ex}

\begin{ex}%Câu 29.
Một cấp số cộng có $u_2=5$ và $u_3=9$. Khẳng định nào sau là khẳng định đúng?
\choice
{\True $u_4=13$}
{$u_4=36$}
{$u_4=4$}
{$u_4=12$}

\end{ex}
\begin{ex}%Câu 30.
Một hình trụ có bán kính đáy bằng $2$ và diện tích xung quanh bằng $12\pi$. Tính thể tích của khối trụ được giới hạn bởi hình trụ đó.
\choice
{$24\pi$}
{$6\pi$}
{\True $12\pi$}
{$18\pi$}

\end{ex}
\begin{ex}%Câu 31.
Tìm tập nghiệm của bất phương trình $\log_{25} x^2 \leq \log_5(4-x)$.
\choice
{$(-\infty; 2)$}
{$(-\infty; 2]$}
{$(0; 2]$}
{\True $(-\infty; 0) \cup(0; 2]$}

\end{ex}
\begin{ex}%Câu 32.
Trong không gian $O x y z$, tọa độ điểm đối xứng với điểm $Q(2; 7; 5)$ qua mặt phẳng $(O x z)$ là
\choice
{\True $(2;-7; 5)$}
{$(-2;-7;-5)$}
{$(-2; 7;-5)$}
{$(2; 7;-5)$}

\end{ex}
\begin{ex}%Câu 33.
Cho hàm số $y=f(x)$ liên tục trên $\mathbb{R}$ và có bảng xét dấu của 
\immini{
$f'(x)$ như sau: Số điểm cực đại của hàm số $y=f(x)$ là
}
{\begin{tikzpicture}[scale=1,line width=.6pt,color=\mauchinh]
\tkzTabInit[nocadre=true,lgt=1,espcl=1.3,deltacl=0.5,lw=0.8]
{$x$ /.7,$y'$/.7}{$-\infty$,$-2$,$0$,$3$,$5$,$+\infty$}
\tkzTabLine{,+,d,-,0,-,0,+,0,+,}
\end{tikzpicture}
}
\choice
{$0$}
{\True $1$}
{$3$}
{$2$}
\end{ex}
\begin{ex}%Câu 34.
Cho lăng trụ đều $ABC \cdot A'B'C'$ tất cả các cạnh bằng $a$. Gọi $\alpha$ là góc giữa mặt phẳng $\left(A'BC\right)$ và mặt phẳng $(ABC)$. Tính tan $\alpha$.
\choice
{$\tan \alpha=\dfrac{\sqrt{3}}{2}$}
{$\tan \alpha=\sqrt{3}$}
{$\tan \alpha=2$}
{\True $\tan \alpha=\dfrac{2\sqrt{3}}{3}$}

\end{ex}
\begin{ex}%Câu 35.
Cho $a>0$ và đặt $\log_2 a=x$. Tính $\log_8\left(4 a^3\right)$ theo $x$.
\choice
{$\log_8\left(4 a^3\right)=3 x+2$}
{\True $\log_8\left(4 a^3\right)=x+\dfrac{2}{3}$}
{$\log_8\left(4 a^3\right)=9 x+6$}
{$\log_8\left(4 a^3\right)=-\dfrac{3 x+2}{3}$}

\end{ex}
\begin{ex}%Câu 36.
Một hình lập phương có diện tích mỗi mặt bằng $4\mathrm{\,cm}^2$. Tính thể tích của khối lập phương đó.
\choice
{$6\mathrm{\,cm}^3$}
{\True $8\mathrm{\,cm}^3$}
{$2\mathrm{\,cm}^3$}
{$64\mathrm{\,cm}^3$}

\end{ex}
\begin{ex}%Câu 37.
Hàm số $y=x^3-3 x^2+3 x+5$ có số điểm cực trị là
\choice
{$1$}
{$3$}
{\True $0$}
{$2$}

\end{ex}
\begin{ex}%Câu 38.
Tìm họ các nguyên hàm của hàm số $f(x)=6 x^2-\sin 2 x$.
\choice
{$2 x^3+\cos 2 x+C$}
{\True $2 x^3+\dfrac{1}{2} \cos 2 x+C$}
{$2 x^3-\dfrac{1}{2} \cos 2 x+C$}
{$3 x^2+\dfrac{1}{2} \cos 2 x+C$}

\end{ex}
\begin{ex}%Câu 39.
Cho tập hợp $Y$ gồm $5$ điểm phân biệt trên mặt phẳng. Số véc tơ khác $\overrightarrow{0}$ có điểm đầu, điểm cuối thuộc tập $Y$ là
\choice
{$25$}
{$5!$}
{$C_5^2$}
{\True $A_5^2$}

\end{ex}
\begin{ex}%Câu 40.
Cho số phức $z$ và $w$ có diểm biểu diễn trong mặt phẳng $O x y$ lần lượt là $M(2; 1)$ và $N(1; 2)$. Tính mô-đun của số phức $z-w$.
\choice
{$\sqrt{3}$}
{\True $\sqrt{2}$}
{$\sqrt{5}$}
{$2$}

\end{ex}
\begin{ex}%Câu 41.
Trong KG $Oxyz$, véc-tơ $\vec{a}(1; 3;-2)$ vuông góc với véc-tơ nào sau đây?
\choice
{$\vec{q}(1;-1; 2)$}
{$\vec{m}(2; 1; 1)$}
{\True $\vec{p}(1; 1; 2)$}
{$\vec{n}(-2; 3; 2)$}

\end{ex}
\begin{ex}%Câu 42.
Nếu $\displaystyle\int\limits_a^{b} f(x) \mathrm{d} x=2$ và $\displaystyle\int\limits_a^{b} g(x) \mathrm{d} x=3$ thì $\displaystyle\int\limits_a^{b}[5 f(x)-2 g(x)] \mathrm{d} x$ bằng bao nhiêu?
\choice
{$8$}
{$16$}
{\True $4$}
{$11$}

\end{ex}
\begin{ex}%Câu 43.
Khẳng định nào sau đây là khẳng định đúng về tính đơn điệu của hàm số $y=\dfrac{x-3}{x}$?
\choice
{$\text{Hàm số nghịch biến trên tập xác định}$}
{$\text{Hàm số đồng biến trên}$ $\mathbb{R}$}
{$\text{Hàm số nghịch biến trên}$ $(-\infty; 0)$ $\text{và}$ $(0;+\infty)$}
{\True $\text{Hàm số đồng biến trên từng khoảng xác định}$}

\end{ex}
\begin{ex}%Câu 44.
Nghiệm duy nhất của phương trình $4^{x+1}=2\sqrt{2}$ là
\choice
{$x=\dfrac{3}{4}$}
{$x=-\dfrac{3}{4}$}
{$x=\dfrac{1}{4}$}
{\True $x=-\dfrac{1}{4}$}

\end{ex}
\begin{ex}%Câu 45.
Tập xác định của hàm số $y=\ln (4-x)$ là
\choice
{\True $(-\infty; 4)$}
{$(-\infty; 4]$}
{$(4;+\infty)$}
{$(-2; 2)$}

\end{ex}
\begin{ex}%Câu 46.
Gọi $z_1, z_2$ là các nghiệm phức của phương trình $z^2-8 z+26=0$. Tính tích $z_1 z_2$.
\choice
{\True $26$}
{$6$}
{$16-10 i$}
{$8$}

\end{ex}
\begin{ex}%Câu 47.
Trong không gian $O x y z$, mặt phẳng $(P)\colon 3 x-2 z+2=0$ di qua điểm nào sau đây?
\choice
{$A(1; 2; 4)$}
{\True $D(2; 1; 4)$}
{$C(2; 4;-1)$}
{$B(4; 2; 1)$}

\end{ex}
\begin{ex}%Câu 48.
Phương trình đường tiệm cận đứng của đồ thị hàm số $y=\dfrac{\sqrt{10-x}}{x^2-100}$ là
\choice
{$x=-10$}
{\True $x=10$ và $x=-10$}
{$x=10$}
{$x=100$}

\end{ex}
\begin{ex}%Câu 49.
Cho một hình trụ có chiều cao $20\mathrm{\,cm}$. Cắt hình trụ đó bởi một mặt phẳng chứa trục của nó thì được thiết diện là một hình chữ nhật có chu vi $100\mathrm{\,cm}$. Tính thể tích của khối trụ được giới hạn bởi hình trụ đã cho.
\choice
{$300\pi \mathrm{cm}^3$}
{$600\pi \mathrm{cm}^3$}
{\True $4500\pi \mathrm{cm}^3$}
{$6000\pi \mathrm{cm}^3$}

\end{ex}
\begin{ex}%Câu 50.
Trong không gian $O x y z$, gọi $d$ là đường thẳng đi qua điểm $M(2; 1; 1)$, cắt và vuông góc với đường thẳng $\Delta\colon \dfrac{x-2}{-2}=\dfrac{y-8}{1}=\dfrac{z}{1}$. Tìm tọa độ giao điểm của $d$ và mặt phẳng $(O y z)$.
\choice
{$(0;-3; 1)$}
{$(0; 3;-5)$}
{$(1; 0; 0)$}
{\True $(0;-5; 3)$}

\end{ex}

\Closesolutionfile{ans}
%\indapan{10}{ans/ans-de15-7}