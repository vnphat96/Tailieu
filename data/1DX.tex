\chap{CẤP SỐ CỘNG - CẤP SỐ NHÂN\\ TỔ HỢP - XÁC SUẤT}
\section{Cấp số cộng và cấp số nhân}
\Opensolutionfile{ans}[ans/CD4-Tong-On]
\setcounter{ex}{0}% Reset lại số đếm câu hỏi
\begin{ex}%[Tuan Nguyen, dự án tổng ôn 40 câu đầu]%[1D3Y3-3]
Cho cấp số cộng $(u_n)$ có số hạng đầu $u_1=2$, công sai $d=3$. Số hạng thứ 5 của $(u_n)$ bằng
\choice{\True $14$}{$10$}{$162$}{$30$}
\loigiai{
Ta có $u_5=u_1+4d=2+4\cdot 3=14$.
}
\end{ex}
\begin{ex}%[Tuan Nguyen, dự án tổng ôn 40 câu đầu]%[1D3Y3-3]
    Cho cấp số cộng $(u_n)$ với $u_1=-2$ và $u_3=4$. Công sai của cấp số cộng đã cho bằng
    \choice{$6$}{\True $3$}{$2$}{$-2$}
\loigiai{
Ta có $u_3=u_1+2d\Leftrightarrow d=\dfrac{u_3-u_1}{2}=3$.
}
\end{ex}
\begin{ex}%[Tuan Nguyen, dự án tổng ôn 40 câu đầu]%[1D3Y3-3]
Cho cấp số cộng $(u_n)$ biết $u_2=3$ và $u_4=7$. Giá trị của $u_{15}$ bằng
\choice{$27$}{$31$}{$35$}{\True $29$}
\loigiai{
Ta có $u_4=u_2+2d\Leftrightarrow 7=3+2d\Leftrightarrow d=2$.\\
Vậy $u_{15}=u_4+11d=7+11\cdot 2=29$.
}
\end{ex}
\begin{ex}%[Tuan Nguyen, dự án tổng ôn 40 câu đầu]%[1D3Y3-3]
    Cho cấp số cộng $(u_n)$ với $u_1=3$ và $u_{10}=21$. Khi đó $u_4$ bằng
    \choice{\True $9$}{$3$}{$18$}{$10$}
\loigiai{
Ta có $u_{10}=u_1+9d\Leftrightarrow 21=3+9d\Leftrightarrow d=2$.\\
Vậy $u_4=u_1+3d=3+3\cdot 2=9$.
}
\end{ex}
\begin{ex}%[Tuan Nguyen, dự án tổng ôn 40 câu đầu]%[1D3Y3-3]
    Cho một cấp số cộng $(u_n)$ với $u_1=\dfrac{1}{3}$ và $u_8=26$. Công sai $d$ của cấp số cộng đã cho bằng
    \choice{\True $\dfrac{11}{3}$}{$\dfrac{3}{11}$}{$\dfrac{10}{3}$}{$\dfrac{3}{10}$}
\loigiai{
Ta có $u_8=u_1+7d\Leftrightarrow 26=\dfrac{1}{3}+7d\Leftrightarrow d=\dfrac{11}{3}$.
}
\end{ex}
\begin{ex}%[Tuan Nguyen, dự án tổng ôn 40 câu đầu]%[1D3B3-3]
    Cho cấp số cộng $(u_n)$ thỏa mãn $\heva{&u_4=10\\&u_4+u_6=26}$, khi đó công sai $d$ bằng
    \choice
    {$-3$}
    {\True $3$}
    {$5$}
    {$6$}
    \loigiai{
        Gọi $d$ là công sai.Ta có $\heva{&u_4=10\\&u_4+u_6=26}\Leftrightarrow\heva{&u_1+3d=10\\&2u_1+8d=26}\Leftrightarrow\heva{&u_1=1\\&d=3.}$ \\
        Vậy công sai $d=3$.}
\end{ex}
\begin{ex}%[Tuan Nguyen, dự án tổng ôn 40 câu đầu]%[1D3B3-1]
    Cho cấp số cộng $(u_n)$ có $\heva{& u_1+u_6=17 \\ & u_2+u_4=14}$. Công sai $d$ của cấp số cộng đã cho bằng
    \choice
    {$2$}
    {\True $3$}
    {$4$}
    {$5$}
    \loigiai
    { Gọi $d$ là công sai của cấp số cộng.\\
        Ta có $$\heva{& u_1+u_6=17 \\ & u_2+u_4=14}\Leftrightarrow\heva{& u_1+u_1+5d=17 \\ & u_1+d+u_1+3d=14}\Leftrightarrow \heva{& 2u_1+5d=17 \\ & 2u_1+4d=14}\Leftrightarrow\heva{& u_1=1 \\ & d=3.}$$
        Vậy công sai của cấp số cộng đã cho là $d=3$.
    }
\end{ex}
\begin{ex}%[Tuan Nguyen, dự án tổng ôn 40 câu đầu]%[1D3B3-3]
    Cho cấp số cộng $(u_n)$ có $u_1=-5$ và $d=3$. Số $100$ là số hạng thứ mấy của cấp số cộng?
    \choice
    {$15$}
    {$20$}
    {$35$}
    {\True $36$}
    \loigiai{
        Giả sử số $100$ là số hạng thứ $k$ của dãy.\\
        Do $u_k=u_1+(k-1)d$ nên $100=-5+3(k-1) \Leftrightarrow k=36$.\\
        Vậy $100$ là số hạng thứ $36$ của cấp số cộng.
    }
\end{ex}
\begin{ex}%[Tuan Nguyen, dự án tổng ôn 40 câu đầu]%[1D3Y3-3]
    Cho cấp số cộng $(u_n)$, có số hạng đầu $u_1=-5$ và công sai $d=2$. Số $81$ là số hạng thứ bao nhiêu của cấp số cộng?
    \choice
    {$100$}
    {$50$}
    {\True $44$}
    {$75$}
    \loigiai{
        Ta có $u_n=u_1+(n-1)d=-5+(n-1)2=81 \Rightarrow n=44$.}
\end{ex}
\begin{ex}%[Tuan Nguyen, dự án tổng ôn 40 câu đầu]%[1D3B3-5]
    Cho cấp số cộng $(u_n)$ có $u_5=-15$, $u_{20}=60$. Tổng của $10$ số hạng đầu tiên của cấp số cộng này bằng?
    \choice
    {$150$}
    {$250$}
        {\True $-125$}
    {$-200$}
    \loigiai{
        Gọi $u_1$, $d$ lần lượt là số hạng đầu và công sai của cấp số cộng.\\
        Ta có $\heva{&u_5=-15\\&u_{20}=60}\Leftrightarrow\heva{&u_1+4d=-15\\&u_1+19d=60}\Leftrightarrow\heva{&u_1=-35\\&d=5.}$ \\
        Vậy $S_{10}=\dfrac{10}{2}\cdot (2u_1+9d) =5\cdot\left[2\cdot (-35)+9\cdot 5\right] =-125$.}
\end{ex}
\begin{ex}%[Tuan Nguyen, dự án tổng ôn 40 câu đầu]%[1D3B3-5]
    Cho cấp số cộng $(u_n)$ có $u_1=4$ và $d=-5$. Tổng $100$ số hạng đầu tiên của cấp số cộng bằng
    \choice
    {$24350$}
    {\True $-24350$}
    {$-24600$}
    {$24600$}
    \loigiai
    {
        Ta có $S_{100}=100u_1+\dfrac{100\cdot 99}{2}d =100\cdot 4+50\cdot 99\cdot (-5)=-24350$.
    }
\end{ex}
\begin{ex}%[Tuan Nguyen, dự án tổng ôn 40 câu đầu]%[1D3B3-5]
    Cho cấp số cộng $(u_n)$ thỏa $u_2+u_8+u_9+u_{15}=100$. Tổng $16$ số hạng đầu tiên của cấp số cộng đã cho bằng
    \choice
    {$100$}
    {$200$}
    {$300$}
    {\True $400$}
    \loigiai{
        Ta có $u_2+u_8+u_9+u_{15}=100 \Rightarrow 4u_1+30d=100 \Rightarrow 2u_1+15d=50$.\\
        Vậy $S_{16}=8\left(2u_1+15d\right)=400$.    
    }
\end{ex}
\begin{ex}%[Tuan Nguyen, dự án tổng ôn 40 câu đầu]%[1D3B3-5]     
    Cho cấp số cộng $(u_n)$ có $u_1=3$ và công sai $d=4$. Biết tổng $n$ số hạng đầu của dãy số $(u_n)$ là $S_n=253$. Khi đó $n$ bằng
    \choice
    {$9$}
    {\True $11$}
    {$12$}
    {$10$}
    \loigiai{
        Ta có
        \begin{eqnarray*}
        &&S_n=\dfrac{n[2u_1+(n-1)d]}{2}=253\\
        &\Leftrightarrow& \dfrac{n(2\cdot 3+(n-1)\cdot 4)}{2}=253\\
        &\Leftrightarrow &4n^2+2n-506=0\Leftrightarrow \hoac{&n=11\\&n=-\dfrac{23}{2} \text{ (loại).}}  
        \end{eqnarray*} 
        Vậy $n=11$.}
\end{ex}
\begin{ex}%[Tuan Nguyen, dự án tổng ôn 40 câu đầu]%[1D3Y3-4]
Cho các số $1$; $3$; $x$ theo thứ tự lập thành một cấp số cộng. Giá trị của $x$ bằng
\choice{$1$}{$3$}{\True $5$}{$9$}
\loigiai{
Ta có $1+x=2\cdot 3\Leftrightarrow x=5$.
}
\end{ex}
\begin{ex}%[Tuan Nguyen, dự án tổng ôn 40 câu đầu]%[2D2B5-1]
    Xác định số thực $x$ để dãy số $\log 2$, $\log 7$; $\log x$ theo thứ tự đó lập thành một cấp số cộng.
    \choice
    {$x=\dfrac{7}{2}$}
    {$x=\dfrac{2}{49}$}
    {$x=\dfrac{2}{7}$}
    {\True $x=\dfrac{49}{2}$}
    \loigiai
    {
        Điều kiện $x>0$.\\
        Để $\log 2$, $\log 7$; $\log x$ theo thứ tự đó lập thành một cấp số cộng thì
        \begin{align*}
            \log 2+\log x=2\log 7\Leftrightarrow \log 2x=\log 7^2\Leftrightarrow \log 2x=\log 49\Leftrightarrow 2x=49\Leftrightarrow x=\dfrac{49}{2}.
        \end{align*}
    }
\end{ex}
\begin{ex}%[Tuan Nguyen, dự án tổng ôn 40 câu đầu]%[1D3B3-4]
    Biết bốn số $5, x, 15, y$ theo thứ tự lập thành cấp số cộng. Giá trị của biểu thức $3 x+2 y$ bằng
    \choice{$50$}{\True $70$}{$30$}{$80$}
    \loigiai{
        Từ giả thiết ta có
        $$\heva{& 5+15=2x \\ & x+y=30}\Leftrightarrow \heva{& x=10 \\ & y=20.}$$
        Vậy $3x+2y=70$. 
    }
\end{ex}
\begin{ex}%[Tuan Nguyen, dự án tổng ôn 40 câu đầu]%[1D3Y4-3]
    Cho cấp số nhân $(u_n)$ với $u_1=2$ và $u_2=6$. Công bội của cấp số nhân đã cho bằng
    \choice{\True $3$}{$-4$}{$4$}{$-3$}
\loigiai{
Ta có $q=\dfrac{u_2}{u_1}=3$.
}
\end{ex}
\begin{ex}%[Tuan Nguyen, dự án tổng ôn 40 câu đầu]%[1D3Y4-3]
    Cho câp số nhân $(u_n)$ với $u_2=2$ và $u_4=18$. Công bội của cãp số nhân đã cho bằng
    \choice{\True $\pm 3$}{$9$}{$16$}{$\pm 2$}
    \loigiai{
Ta có $u_4=u_2\cdot q^2\Leftrightarrow 18=2q^2\Leftrightarrow q=\pm 3$. 
}
\end{ex}
\begin{ex}%[Tuan Nguyen, dự án tổng ôn 40 câu đầu]%[1D3Y4-3]
    Cho cấp số nhân $(u_n)$ với $u_1=3$, công bội $q=-\dfrac{1}{2}$. Số hạng $u_3$ bằng
    \choice{$\dfrac{3}{2}$}{$-\dfrac{3}{8}$}{$2$}{\True $\dfrac{3}{4}$}
    \loigiai{
Ta có $u_3=u_1q^2=3\cdot \left(-\dfrac{1}{2}\right)^2=\dfrac{3}{4}$.    
}
\end{ex}
\begin{ex}%[Tuan Nguyen, dự án tổng ôn 40 câu đầu]%[1D3Y4-3]
    Cho cấp số nhân $(u_n)$ biết $u_1=1$ và $u_4=64$. Công bội $q$ của cấp số nhân đã cho bằng
    \choice{$21$}{$\pm 4$}{\True $4$}{$2\sqrt{2}$}
    \loigiai{
Ta có $u_4=u_1q^3\Leftrightarrow 64=q^3\Leftrightarrow q=4$.    
}
\end{ex}
\begin{ex}%[Tuan Nguyen, dự án tổng ôn 40 câu đầu]%[1D3Y4-3]
Cho cấp số nhân $(u_n)$ có $u_3=8$, $u_5=32$ và công bội $q>0$. Số hạng thứ $10$ của cấp số nhân đó bằng
\choice{\True $1024$}{$\sqrt{33}$}{$512$}{$-512$}
\loigiai{
Ta có $u_5=u_3q^2\Leftrightarrow q^2=4\Leftrightarrow q=\pm 2$.\\
Vì $q>0$ nên $q=2$. Vậy $u_{10}=u_3\cdot q^7=8\cdot 2^7=1024$.
}
\end{ex}
\begin{ex}%[Tuan Nguyen, dự án tổng ôn 40 câu đầu]%[1D3Y4-3]
    Cho cấp số nhân $(u_n)$ có $u_1=2$ và $u_2=-4$. Số hạng thứ $5$ của cấp số nhân bằng
    \choice{$-16$}{\True $32$}{$-32$}{$16$}
\loigiai{
Ta có $q=\dfrac{u_2}{u_1}=-2$.\\
Vậy $u_5=u_1\cdot q^4=2\cdot (-2)^4=32$.
}
\end{ex}
\begin{ex}%[Tuan Nguyen, dự án tổng ôn 40 câu đầu]%[1D3B4-3]
    Cho cấp số nhân $(u_n)$ có các số hạng thỏa mãn $\heva{&u_1+u_5=33\\&u_2+u_6=66}$. Tìm số hạng đầu $u_1$ và công bội $q$ của cấp số nhân.
    \choice
    {$u_1=2$, $q=2$}
    {\True $u_1=\dfrac{33}{17}$, $q=2$}
    {$u_1=\dfrac{33}{17}$, $p=2$}
    {$u_1=3$, $q=2$}
    \loigiai{
        Áp dụng công thức $u_n=q^{n-1} \cdot u_1$ với $n \geq 2,n \in \mathbb{N}$.\\
        Ta có $\heva{&u_1+u_5=33\\&u_2+u_6=66} \Leftrightarrow \heva{&u_1+u_1 \cdot q^4=33\\&u_1q+u_1q^5=66} \Leftrightarrow \heva{&u_1(1+q^4)=33\quad (1)\\&u_1q(1+q^4)=66.\quad(2)}$\\
        Lấy (2) chia (1) ta được $\dfrac{u_1q(1+q^4)}{u_1(1+q^4)}=\dfrac{66}{33} \Leftrightarrow q=2$. Thay $q=2$ vào (1) ta được $u_1=\dfrac{33}{17}$.}
\end{ex}
\begin{ex}%[Tuan Nguyen, dự án tổng ôn 40 câu đầu]%[1D3B4-3]
    Cho cấp số nhân $(u_n)$ có $\heva{&u_4+u_6=-540\\&u_3+u_5=180}$. Tìm số hạng đầu $u_1$ và công bội $q$ của cấp số nhân.
    \choice
    {\True $u_1=2$, $q=-3$}
    {$u_1=2$, $q=3$}
    {$u_1=-2$, $q=3$}
    {$u_1=-2$, $q=-3$}
    \loigiai{
        Ta có $u_4+u_6=-540$ $\Leftrightarrow (u_3+u_5)q=-540$.\\
        Kết hợp với phương trình thứ hai trong hệ, ta tìm được $q=-3$.\\
        Lại có $u_3+u_5=180$ $\Leftrightarrow u_1(q^2+q^4)=180$.\\
        Vì $q=-3$ nên $u_1=2$.\\
        Vậy $u_1=2$, $q=-3$.}
\end{ex}
\begin{ex}%[Tuan Nguyen, dự án tổng ôn 40 câu đầu]%[1D3B4-5]
    Cho cấp số nhân $(u_n)$ có $u_1=-3$ và $q=-2$. Tính tổng $10$ số hạng đầu tiên của cấp số nhân đã cho.
    \choice
    {$S_{10}=-511$}
    {$S_{10}=-1025$}
    {$S_{10}=1025$}
    {\True $S_{10}=1023$}
    \loigiai{
        Ta có $\heva{&u_1=-3\\&q=-2}\Rightarrow S_{10}=u_1\cdot\dfrac{1-q^{10}}{1-q}=-3\cdot\dfrac{1-(-2)^{10}}{1-(-2)}=1023$.
    }
\end{ex}
\begin{ex}%[Tuan Nguyen, dự án tổng ôn 40 câu đầu]%[1D3B4-5]
    Cho cấp số nhân $(u_n)$ có $u_1=-6$ và $q=-2$. Tổng $n$ số hạng đầu tiên của cấp số nhân đã cho bằng $2046$. Tìm $n$. 
    \choice
    {$n=9$}
    {\True $n=10$}
    {$n=11$}
    {$n=12$}
    \loigiai{
        Ta có $S_n=\dfrac{u_1\cdot\left(1-q^n\right)}{1-q}$. Theo đề bài suy ra
        \[\dfrac{-6\cdot\left(1-\left(-2\right)^n\right)}{1-(-2)}=2046\Leftrightarrow -2\cdot\left(1-2^n\right)=2046\Leftrightarrow 2^n=1024=2^{10}\Leftrightarrow n=10.\]
    }
\end{ex}
\begin{ex}%[Tuan Nguyen, dự án tổng ôn 40 câu đầu]%[1D3B4-5]
    Cho cấp số nhân $(u_n)$ có số hạng đầu $u_1=3$ và công bội $q=2$. Biết rằng tổng của $n$ số hạng đầu tiên bằng $765$, khi đó $n$ bằng.  
    \choice
    {$6$}
    {$7$}
    {\True $8$}
    {$9$}
    \loigiai{
        Với $(u_n)$ là dãy cấp số nhân ta có $S_n=u_1\cdot \dfrac{1-q^n}{1-q}\Rightarrow 765=3\cdot \dfrac{1-2^n}{1-2}\Rightarrow 2^n=256\Rightarrow n=8$.      
    }
\end{ex}
\begin{ex}%[Tuan Nguyen, dự án tổng ôn 40 câu đầu]%[1D3Y4-3]
    Cho cấp số nhân $(u_n)$ thỏa $u_1=1$, $q=2$. Hỏi số $1024$ là số hạng thứ mấy?
    \choice
    {\True $11$}
    {$9$}
    {$8$}
    {$10$}
    \loigiai{
        Công thức số hạng tổng quát của cấp số nhân $u_n=u_1\cdot q^{n-1}$.\\
        Từ đó ta có $1\cdot 2^{n-1}=1024 \Rightarrow n-1=10 \Rightarrow n=11$.
    }
\end{ex}
\begin{ex}%[Tuan Nguyen, dự án tổng ôn 40 câu đầu]%[1D3B4-3]
    Cho cấp số nhân $(v_n)$ có $v_1=-3$ cộng bội $q=-2$. Số $-192$ là số hạng thứ bao nhiêu ?
    \choice{$5$}{$6$}{\True $7$}{$8$}
\loigiai{
Ta có $v_n=v_1\cdot q^{n-1}\Leftrightarrow -192=-3\cdot (-2)^{n-1}\Leftrightarrow n-1=6\Leftrightarrow n=7$.
}
\end{ex}
\begin{ex}%[Tuan Nguyen, dự án tổng ôn 40 câu đầu]%[1D3Y4-3]
    Cho cấp số nhân $(u_n)$ có $u_1=3$ và $q=2$. Số $12288$ là số hạng thứ bao nhiêu của cấp số nhân đã cho?
    \choice
    {$12$}
    {\True $13$}
    {$14$}
    {$11$}
    \loigiai{
        Số hạng tổng quát của cấp số nhân là $u_n=u_1q^{n-1}=3 \cdot 2^{n-1}$.\\
        Vì $u_n=12288$ nên $3 \cdot 2^{n-1}=12288 \Leftrightarrow n=13$.\\
        Do $n=13$ là số nguyên dương nên số $12288$ là số hạng thứ $13$ của cấp số nhân đã cho.
    }
\end{ex}
\begin{ex}%[Tuan Nguyen, dự án tổng ôn 40 câu đầu]%[KSCL, THPT Thanh Thủy, Phú Thọ, 2019]%[Hà Lê, EX3-2019]%[1D3B4-1]
    Tổng tất cả các giá trị của $x$ để ba số $2x-1; x; 2x+1$ theo thứ tự đó lập thành cấp số nhân bằng
    \choice
    {\True $0$}
    {$12$}
    {$5$}
    {$6$}
    \loigiai{
        Ba số $2x-1; x; 2x+1$ theo thứ tự lập thành cấp số nhân khi
        \begin{eqnarray*}
            && x^2=(2x-1)(2x+1)\\
            & \Leftrightarrow & x^2=4x^2-1\\
            & \Leftrightarrow & x^2=\dfrac{1}{3}\\
            & \Leftrightarrow & x =\pm \dfrac{1}{\sqrt{3}}.
        \end{eqnarray*}
Vậy tổng các giá trị của $x$ thỏa mãn là $0$.       
    }
\end{ex}
\begin{ex}%[Tuan Nguyen, dự án tổng ôn 40 câu đầu]%[1D3B4-3]
    Tổng các giá trị thực của $x$ để ba số $1+x$, $9+x$, $33+x$ theo thứ tự đó lập thành một cấp số nhân bằng   
    \choice
    {$4$}
    {\True $3$}
    {$7$}
    {$10$}
    \loigiai{
        Vì ba số $1+x$, $9+x$, $33+x$ theo thứ tự đó lập thành một cấp số nhân nên\\
        \centerline{$(1+x)(33+x)=(9+x)^2\Leftrightarrow 16x-48=0\Leftrightarrow x=3$.}\\
        Vậy tổng các giá trị thực của $x$ bằng $3$.     
    }
\end{ex}
\begin{ex}%[Tuan Nguyen, dự án tổng ôn 40 câu đầu]%[1D3B4-2] 
    Cho cấp số nhân $(u_n)$ thỏa mãn $\heva{&u_4-u_2=36 \\ & u_5-u_3=72}$. Khi đó $u_1+q$ bằng
    \choice 
    { $6$}
    { \True $8$}
    { $11$}
    { $12$}
    \loigiai
    {
        $\heva{&u_4-u_2=36 \\ & u_5-u_3=72} \Leftrightarrow \heva{& u_1q(q^2-1)=36 \\ &u_1q^2(q^2-1)=72}
        \Leftrightarrow \heva{& u_1q(q^2-1)=36 \\ & 36q=72}
        \Leftrightarrow \heva{& q=2 \\& u_1=\dfrac{36}{q(q^2-1)}=6.}$\\
        Khi đó $u_1+q=8$.
    } 
\end{ex}
\begin{ex}%[Tuan Nguyen, dự án tổng ôn 40 câu đầu]%[1D3B4-4]
    Cho ba số $x$, $5$, $2y$ theo thứ tự lập thành cấp số cộng và ba số $x$, $4$, $2y$ theo thứ tự lập thành cấp số nhân thì $|x-2y|$ bằng
    \choice
    {$8$}
    {$9$}
    {\True $6$}
    {$10$}
    \loigiai{
        Ta có ba số $x$, $5$, $2y$ theo thứ tự lập thành cấp số cộng $\Rightarrow x+2y=2\cdot5=10\quad (1)$\\
        Ta có ba số $x$, $4$, $2y$ theo thứ tự lập thành cấp số nhân $\Rightarrow 2xy=4^2 \Rightarrow xy=8\quad (2)$.\\
        Từ $(1)$ và $(2)$ $\Rightarrow \heva{&x=8\\&y=1}$ hoặc $\heva{&x=2\\&y=4}\Rightarrow |x-2y|=6$.
    }
\end{ex}
\begin{ex}%[Tuan Nguyen, dự án tổng ôn 40 câu đầu]%[1D3B4-6]
    Cho ba số $x$, $5$, $3y$ theo thứ tự lập thành một cấp số cộng và ba số $x$, $3$, $3y$ theo thứ tự lập thành một cấp số nhân. Tính $|3y-x|$.
    \choice{\True $8$}{$6$}{$9$}{$10$}
    \loigiai{
        Ba số $x$, $5$, $3y$ theo thứ tự lập thành một cấp số cộng nên $x+3y=10\Rightarrow 3y=10-x$.\\
        Ba số $x$, $3$, $3y$ theo thứ tự lập thành một cấp số nhân nên $3xy=9$.\\
        Suy ra $(10-x)x=9 \Leftrightarrow x^2-10x+9=0 \Leftrightarrow \hoac{&x=1\Rightarrow 3y=9\\&x=9 \Rightarrow 3y=1.}$\\
        Suy ra $|x-3y|=8$.
    }
\end{ex}
\begin{ex}%[Tuan Nguyen, dự án tổng ôn 40 câu đầu]%[1D3K4-3]
    Cho cấp số nhân $(u_n)$ thỏa mãn $\heva{&u_2+u_3+u_4=44\\&u_2^2+u_3^2+u_4^2=1104}$. Giá trị của $u_2u_3+u_3u_4+u_4u_2$ là.  
    \choice
    {$216$}
    {\True $416$}
    {$614$}
    {$164$}
    \loigiai{
        Ta có $u_2+u_3+u_4=44\Rightarrow \heva{&u_2^2+u_2u_3+u_4u_2=44u_2\\&u_2u_3+u_3^2+u_3u_4=44u_3\\&u_4u_2+u_3u_4+u_4^2=44u_4.}$\\
        Cộng các vế của hệ phương trình ta được
        \begin{eqnarray*}
        &&  2\left(u_2u_3+u_3u_4+u_4u_2\right)+\left(u_2^2+u_3^2+u_4^2\right)=44\left(u_2+u_3+u_4\right)\\
        &\Rightarrow & 2\left(u_2u_3+u_3u_4+u_4u_2\right)+1104=44\cdot44\\
        &\Rightarrow & u_2u_3+u_3u_4+u_4u_2=\dfrac{44\cdot44-1104}{2}=416.
        \end{eqnarray*}
    }
\end{ex}
\begin{ex}%[Tuan Nguyen, dự án tổng ôn 40 câu đầu]%[1D3K3-5]
    Một tòa nhà hình tháp có $30$ tầng và tổng cộng có $1890$ phòng, càng lên cao thì số phòng càng giảm, biết rằng  cứ $2$ tầng liên tiếp thì hơn kém nhau $4$ phòng. Quy ước rằng tầng trệt là tầng $1$, tiếp theo lên là tầng số $2$, $3$, $\ldots$. Hỏi tầng số $10$ có bao nhiêu phòng?
    \choice{$55$ phòng}{$50$ phòng}{\True $85$ phòng}{$30$ phòng}
    \loigiai{
        Gọi $u_n$ là số phòng của tầng thứ $n$.
        Theo đề bài ta có
        $u_1-u_2=u_2-u_3=u_3-u_4=\cdots=u_{29}-u_{30}=4$ nên $(u_n)$ là cấp số cộng với công sai $d=-4$.\\
        Ta có \begin{align*}
            S_{30}=1890 
            \Leftrightarrow \dfrac{(2u_1-29 \cdot 4)\cdot 30}{2}=1890
            \Leftrightarrow u_1=121.
        \end{align*}
        Số hạng tổng quát của cấp số cộng là $u_n=u_1+(n-1)d=121-4(n-1)=-4n+125$.\\
        Khi đó tầng $10$ có số phòng là $ u_{10}=-4\cdot 10 +125=85$.   
    }
\end{ex}
\section{Hoán vị - Chỉnh hợp - Tổ hợp}
\begin{ex}%[Tuan Nguyen, dự án tổng ôn 40 câu đầu]%[1D2Y2-1]
    Số hoán vị của $n$ phần tử bằng
    \choice{\True $n!$}{$2n$}{$n^2$}{$n^n$}
    \loigiai{
Số hoán vị của $n$ phần tử bằng $n!$.   
}
\end{ex}
\begin{ex}%[Tuan Nguyen, dự án tổng ôn 40 câu đầu]%[1D2Y2-1]
Công thức tính số tổ hợp chập $k$ của $n$ phần tử là
\choice{$\mathrm{A}_n^k=\dfrac{n!}{(n-k)!}$}{$\mathrm{A}_n^k=\dfrac{n!}{(n-k)!\cdot k!}$}{\True $\mathrm{C}_n^{k}=\dfrac{n!}{(n-k)!k!}$}{$\mathrm{C}_n^k=\dfrac{n!}{(n-k)!}$}
\loigiai{
Công thức tính số tổ hợp chập $k$ của $n$ phần tử là $\mathrm{C}_n^{k}=\dfrac{n!}{(n-k)!k!}$.
}
\end{ex}
\begin{ex}%[Tuan Nguyen, dự án tổng ôn 40 câu đầu]%[1D2Y2-1]
    Kí hiệu $\mathrm{A}_n^k$ là số các chỉnh hợp chập $k$ của $n$ phần tử $(1 \le k \le n)$. Mệnh đề nào đúng?
    \choice{$\mathrm{A}_n^k=\dfrac{n!}{(n+k)!}$}{$\mathrm{A}_n^k=\dfrac{n!}{k!\cdot(n+k)!}$}{$\mathrm{A}_{n}^{k}=\dfrac{n!}{k!\cdot(n-k)!}$}{\True $\mathrm{A}_n^k=\dfrac{n!}{(n-k)!}$}
    \loigiai{
Ta có $\mathrm{A}_n^k=\dfrac{n!}{(n-k)!}$.
}
\end{ex}
\begin{ex}%[Tuan Nguyen, dự án tổng ôn 40 câu đầu]%[1D2Y2-1]
    Có $n$ ($n>0$) phần tử lấy ra $k$ ($0<k<n$) phần tử đem đi sắp xếp theo một thứ tự nào đó, mà khi thay đối thứ tự ta được cách sắp xếp mới. Khi đó số cách sắp xếp là
    \choice{$\mathrm{C}_n^k$}{$\mathrm{A}_k^n$}{\True $\mathrm{A}_n^k$}{$\mathrm{P}_n$}
    \loigiai{
Số cách sắp xếp là số chỉnh hợp chập $k$ của $n$ phần tử nên số cách là     $\mathrm{A}_n^k$.
}
\end{ex}
\begin{ex}%[Tuan Nguyen, dự án tổng ôn 40 câu đầu]%[1D2Y2-1]
    Số chỉnh hợp chập $4$ của $7$ phần tử là
    \choice{$720$}{$35$}{\True $840$}{$24$}
    \loigiai{
Ta có $\mathrm{A}_7^4=840$. 
}
\end{ex}
\begin{ex}%[Tuan Nguyen, dự án tổng ôn 40 câu đầu]%[1D2Y2-1]
    Số chỉnh hợp chập $2$ của $5$ phần tử bằng
    \choice{$10$}{$120$}{\True $20$}{$7$}
\loigiai{
Ta có $\mathrm{A}_5^2=20$.
}
\end{ex}
\begin{ex}%[Tuan Nguyen, dự án tổng ôn 40 câu đầu]%[1D2Y2-1]
    Có bao nhiêu cách sắp xếp $5$ học sinh thành một hàng dọc?
    \choice{$5^5$}{\True $5!$}{$4!$}{$5$}
\loigiai{
Số cách sắp xếp $5$ học sinh thành một hàng dọc là $5!$.
}
\end{ex}
\begin{ex}%[Tuan Nguyen, dự án tổng ôn 40 câu đầu]%[1D2Y2-1]
Cho tập hợp $M$ có $10$ phần tử. Số cách chọn ra hai phần tử của $M$ và sắp xếp thứ tự hai phần tử đó là
\choice{$\mathrm{C}_{10}^2$}{\True $\mathrm{A}_{10}^2$}{$\mathrm{C}_{10}^2+2!$}{$\mathrm{A}_{10}^2+2!$}
\loigiai{
Số cách chọn ra hai phần tử của $M$ là sắp xếp thứ tự hai phần tử đó là $\mathrm{A}_{10}^2$.
}
\end{ex}
\begin{ex}%[Tuan Nguyen, dự án tổng ôn 40 câu đầu]%[1D2Y2-1]
    Cho $A$ là tập hợp gồm $20$ điểm phân biệt. Số đoạn thẳng có hai đầu mút phân biệt thuộc tập $A$ là
    \choice{$170$}{$160$}{\True $190$}{$360$}
\loigiai{
Số đoạn thẳng có hai đầu mút phân biệt thuộc tập $A$ là $\mathrm{C}_{20}^2=190$.
}
\end{ex}
\begin{ex}%[Tuan Nguyen, dự án tổng ôn 40 câu đầu]%[1D2Y2-1]
    Số véc-tơ khác $\overrightarrow{0}$ có điểm đầu, điểm cuối là hai trong 6 đỉnh của lục giác $ABCDEF$ là
    \choice{$\mathrm{P}_6$}{$\mathrm{C}_6^2$}{\True $\mathrm{A}_6^2$}
    {$36$}
    \loigiai{
Số véc-tơ là $\mathrm{A}_6^2$.
}
\end{ex}
\begin{ex}%[Tuan Nguyen, dự án tổng ôn 40 câu đầu]%[1D2Y2-1]
    Có bao nhiêu cách sắp xếp $6$ học sinh theo một hàng dọc?
    \choice{$46656$}{$4320$}{\True $720$}{$360$}
    \loigiai{
Số cách sắp xếp $6$ học sinh theo hàng dọc là $6!=720$. 
}
\end{ex}
\begin{ex}%[Tuan Nguyen, dự án tổng ôn 40 câu đầu]%[1D2Y2-1]
    Cần chọn $3$ người đi công tác từ một tổ có $30$ người, khi đó số cách chọn là
    \choice{$\mathrm{A}_{30}^3$}{$3^{30}$}{$10$}{\True $\mathrm{C}_{30}^3$}
    \loigiai{
Số cách chọn là $\mathrm{C}_{30}^3$.    
}
\end{ex}
\begin{ex}%[Tuan Nguyen, dự án tổng ôn 40 câu đầu]%[1D2Y2-1]
    Một tổ có $10$ học sinh. Hỏi có bao nhiêu cách chọn ra $2$ học sinh từ tổ đó để giữ hai chức vụ tổ trường và tổ phó.
    \choice{\True $\mathrm{A}_{10}^2$}{$\mathrm{C}_{10}^2$}{$\mathrm{A}_{10}^8$}{$10^2$}
\loigiai{
Số cách chọn ra $2$ học sinh từ tổ đó để giữ hai chức vụ tổ trường và tổ phó là $\mathrm{A}_{10}^2$.
}
\end{ex}

%%==========Câu 673
\begin{ex}%[1D2Y2-1]
Cho tập hợp $X$ gồm $10$ phần tử. Số các hoán vị của $10$ phần tử của tập hợp $X$ là
\choice
{\True $10!$}
{$10^{2}$}
{$2^{10}$}
{$10^{10}$}
\loigiai{
Số các hoán vị của $10$ phần tử $10!$
}
\end{ex}

%%==========Câu 674
\begin{ex}%[1D2Y2-1]
Có bao nhiêu cách chọn hai học sinh từ một nhóm gồm $34$ học sinh?
\choice
{$2^{3}$}
{$\mathrm{A}_{34}^{2}$}
{$34^{2}$}
{\True $\mathrm{C}_{34}^{2}$}
\loigiai{
Số cách chọn hai học sinh từ một nhóm gồm $34$ học sinh là $\mathrm{C}_{34}^{2}$.
}
\end{ex}

%%==========Câu 675
\begin{ex}%[1D2Y2-1]
Một nhóm học sinh có $10$ người. Cần chọn $3$ học sinh trong nhóm để làm $3$ công việc là tưới cây, lau bàn và nhặt rác, mỗi người làm một công việc. Số cách chọn là
\choice
{$10^{3}$}
{$3\times 10$}
{$\mathrm{C}_{10}^{3}$}
{\True $\mathrm{A}_{10}^{3}$}
\loigiai{
Mỗi cách chọn là một chỉnh hợp chập $3$ của $10$ phần tử. Số cách chọn là $\mathrm{A}_{10}^{3}$.
}
\end{ex}

%%==========Câu 676
\begin{ex}%[1D2Y2-1]
Có bao nhiêu cách lấy ra $3$ phần tử tùy ý từ một tập hợp có $12$ phần tử?
\choice
{$3^{12}$}
{$12^{3}$}
{$\mathrm{A}_{12}^{3}$}
{\True $\mathrm{C}_{12}^{3}$}
\loigiai{
Mỗi cách chọn là một tổ hợp chập $3$ của $10$ phần tử. Số cách chọn là $\mathrm{C}_{12}^{3}$.
}
\end{ex}

%%==========Câu 677
\begin{ex}%[1D2Y2-1]
Cho tập hợp $A$ có $20$ phần tử, số tập con có $2$ phần tử của $A$ là
\choice
{$2\mathrm{C}_{20}^{2}$}
{$2\mathrm{A}_{20}^{2}$}
{\True $\mathrm{C}_{20}^{2}$}
{$\mathrm{A}_{20}^{2}$}
\loigiai{
Mỗi tập con có hai phần tử của $A$ là là một tổ hợp chập $2$ của $20$ phần tử. Số cách chọn là $\mathrm{C}_{20}^{2}$.
}
\end{ex}

%%==========Câu 678
\begin{ex}%[1D2Y2-1]
Cho tập hợp $S=\{1; 2; 3; 4; 5; 6\}$. Có thế lập được bao nhiêu số tự nhiên gồm bốn chữ số khác nhau lãy từ tập hợp $S$?
\choice
{\True $360$}
{$120$}
{$15$}
{$20$}
\loigiai{
Số các số tự nhiên gồm bốn chữ số khác nhau lãy từ tập hợp $S$ là $\mathrm{A}_{6}^{4}=360$. 
}
\end{ex}

%%==========Câu 679
\begin{ex}%[1D2Y2-1]
Số cách chọn $5$ học sinh trong một lớp có $25$ học sinh nam và $16$ học sinh nữ là
\choice
{$\mathrm{C}_{25}^{5}+\mathrm{C}_{16}^{5}$}
{$\mathrm{C}_{25}^{5}$}
{$\mathrm{A}_{41}^{5}$}
{\True $\mathrm{C}_{41}^{5}$}
\loigiai{
Số cách chọn $5$ học sinh trong một lớp có $25$ học sinh nam và $16$ học sinh nữ là $\mathrm{C}_{41}^{5}$.
}
\end{ex}

%%==========Câu 680
\begin{ex}%[1D2B2-1]
Có $3$ bạn nam và $3$ bạn nữ được xếp vào một ghế dài có $6$ vị trí. Hỏi có bao nhiêu cách xếp sao cho nam và nữ ngồi xen kẽ lẫn nhau?
\choice
{$48$}
{\True $72$}
{$24$}
{$36$}
\loigiai{
Số cách xếp sao cho $3$ bạn nam và $3$ bạn nữ ngồi xen kẽ lẫn nhau là $2\cdot 3!\cdot 3!=72$.
}
\end{ex}

%%==========Câu 681
\begin{ex}%[1D2Y2-1]
Số tập hợp con có $3$ phần tử của một tập hợp có $7$ phần tử là
\choice
{$A^{3}$}
{\True $\mathrm{C}_{7}^{3}$}
{$7$}
{$\dfrac{7!}{3!}$}
\loigiai{
Mỗi tập con có $3$ phần tử của tập có $7$ phần tử là là một tổ hợp chập $3$ của $7$. Số tập con là $\mathrm{C}_{7}^{3}$.
}
\end{ex}

%%==========Câu 682
\begin{ex}%[1D2Y2-1]
Một hộp đựng $2$ viên bi màu vàng và $3$ viên bi màu đỏ. Có bao nhiêu cách lấy ra $2$ viên bi trong hộp?
\choice
{\True $10$}
{$20$}
{$5$}
{$6$}
\loigiai{
Số cách lấy $2$ viên bi từ hộp có $5$ viên bi là $\mathrm{C}_{5}^{2}=10$.
}
\end{ex}

%%==========Câu 683
\begin{ex}%[1D2Y2-1]
Từ tập $A=\{1; 2; 3; 4; 5; 6; 7\}$, có thể lập được bao nhiêu số có $5$ chữ số đôi một khác nhau?
\choice
{$5!$}
{$\mathrm{C}_{7}^{5}$}
{\True $\mathrm{A}_{7}^{5}$}
{$7^{5}$}
\loigiai{
Số ác số có $5$ chữ số đôi một khác nhau từ tập $A=\{1; 2; 3; 4; 5; 6; 7\}$ là $\mathrm{A}_{7}^{5}$.
}
\end{ex}

%%==========Câu 684
\begin{ex}%[1D2Y1-2]
Trong một buổi khiêu vũ có $20$ nam và $18$ nữ. Hỏi có bao nhiêu cách chọn ra một đôi nam nữ để khiêu vũ?
\choice
{$\mathrm{C}_{38}^{2}$}
{$\mathrm{A}_{38}^{2}$}
{$\mathrm{C}_{20}^{2} \mathrm{C}_{18}^{1}$}
{\True $\mathrm{C}_{20}^{1} \mathrm{C}_{18}^{1}$}
\loigiai{
Số cách chọn ra một đôi nam nữ để khiêu vũ là $\mathrm{C}_{20}^{1} \mathrm{C}_{18}^{1}$.
}
\end{ex}

%%==========Câu 685
\begin{ex}%[1D2B2-1]
Một nhóm có $7$ học sinh trong đó có $3$ nam và $4$ nữ. Hỏi có bao nhiêu cách xếp các học sinh trên thành một hàng ngang sao cho các học sinh nữ đứng cạnh nhau?
\choice
{$144$}
{$5040$}
{\True $576$}
{$1200$}
\loigiai{
Số cách xếp thỏa mãn yêu cầu bài toán là $4!\cdot4!=576$.
}
\end{ex}

%%==========Câu 686
\begin{ex}%[1D2Y2-1]
Cho $8$ điểm trong đó không có $3$ điểm nào thẳng hàng. Hỏi có bao nhiêu tam giác mà ba đỉnh của nó được chọn từ $8$ điểm trên?
\choice
{$336$}
{\True $56$}
{$168$}
{$84$}
\loigiai{
Số tam giác mà ba đỉnh của nó được chọn từ $8$ điểm là $\mathrm{C}_{8}^{3}=56$
}
\end{ex}

%%==========Câu 687
\begin{ex}%[1D2Y2-1]
Có bao nhiêu cách chọn $5$ cầu thủ từ $11$ cầu thủ trong một đội bóng để thực hiện đá $5$ quả luân lưu $11$ m, theo thứ tự quả thứ nhất đến quả thứ năm.
\choice
{\True $\mathrm{A}_{11}^{5}$}
{$\mathrm{C}_{11}^{5}$}
{$\mathrm{A}_{11}^{2} \cdot 5!$}
{$\mathrm{C}_{10}^{5}$}
\loigiai{
cách chọn $5$ cầu thủ từ $11$ trong một đội bóng để thực hiện đá $5$ quả luân lưu $11$ m là $\mathrm{A}_{11}^{5}$
}
\end{ex}

%%==========Câu 688
\begin{ex}%[1D2B2-1]
Có $14$ người gồm $8$ nam và $6$ nữ. Số cách chọn $6$ người trong đó có đúng $2$ nữ là
\choice
{$1078$}
{$1414$}
{\True $1050$}
{$1386$}
\loigiai{
Số cách chọn $6$ người trong đó có đúng $2$ nữ là $\mathrm{C}_{6}^{2}\cdot \mathrm{C}_{8}^{4}=1050$. 
}
\end{ex}

%%==========Câu 689
\begin{ex}%[1D2B2-1]
Có bao nhiêu cách xếp $6$ bạn $A, B, C, D, E, F$ vào một ghế dài sao cho bạn $A$, $F$ ngồi ở $2$ đầu ghế?
\choice
{$120$}
{$720$}
{$24$}
{\True $48$}
\loigiai{
Số cách xếp $6$ bạn $A, B, C, D, E, F$ vào một ghế dài sao cho bạn $A$, $F$ ngồi ở $2$ đầu ghế là
$2\cdot 4!=48$.
}
\end{ex}

%%==========Câu 690
\begin{ex}%[1D2Y2-1]
Cho tập hợp $S$ có $10$ phần tử. Số tập con gồm $3$ phần tử của $S$ bằng
\choice
{$\mathrm{A}_{10}^{3}$}
{\True $\mathrm{C}_{10}^{3}$}
{$30$}
{$10^{3}$}
\loigiai{
Số tập con gồm $3$ phần tử của tập có $10$ phần tử bằng $\mathrm{C}_{10}^{3}$.
}
\end{ex}

%%==========Câu 691
\begin{ex}%[1D2Y2-1]
Cần phân công $3$ bạn từ một tổ có $10$ bạn để làm trực nhật. Hỏi có bao nhiêu cách phân công khác nhau?
\choice
{$720$}
{$10^{3}$}
{\True $120$}
{$210$}
\loigiai{
Số cách phân công $3$ bạn từ một tổ có $10$ bạn để làm trực nhật là $\mathrm{C}_{10}^{3}=120$.
}
\end{ex}

%%==========Câu 692
\begin{ex}%[1D2Y2-1]
Số cách sắp xếp $6$ học sinh vào một bàn dài có $10$ chỗ ngồi là
\choice
{$6\cdot \mathrm{A}_{10}^{6}$}
{\True $\mathrm{C}_{10}^{6}$}
{$\mathrm{A}_{10}^{6}$}
{$10\mathrm{P}_{6}$}
\loigiai{
Số cách sắp xếp $6$ học sinh vào một bàn dài có $10$ chỗ ngồi là $\mathrm{C}_{10}^{6}$.
}
\end{ex}

%%==========Câu 693
\begin{ex}%[1D2Y2-1]
Một tổ có $6$ học sinh nam và $9$ học sinh nữ. Hỏi có bao nhiêu cách chọn $5$ học sinh đi lao động, trong đó có $2$ học sinh nam?
\choice
{$\mathrm{C}_{9}^{2} \cdot \mathrm{C}_{6}^{2}$}
{$\mathrm{C}_{6}^{2}+\mathrm{C}_{9}^{3}$}
{$\mathrm{A}_{8}^{2} \cdot A^{3}$}
{\True $\mathrm{C}_{6}^{2} \cdot \mathrm{C}_{9}^{3}$}
\loigiai{
Số cách chọn $5$ học sinh đi lao động, trong đó có $2$ học sinh nam là $\mathrm{C}_{6}^{2} \cdot \mathrm{C}_{9}^{3}$.
}
\end{ex}

%%==========Câu 694
\begin{ex}%[1D2Y2-1]
Cho tập $A=\{0; 1; 2; 3; 4; 5; 6; 7; 8; 9\}$. Có bao nhiêu số tự nhiên gồm $5$ chữ số khác nhau được tạo từ tập $A$?
\choice
{$\mathrm{A}_{10}^{4}$}
{$9\cdot\mathrm{C}_{9}^{4}$}
{\True $9\cdot\mathrm{A}_{9}^{4}$}
{$\mathrm{C}_{10}^{4}$}
\loigiai{
Số các số tự nhiên gồm $5$ chữ số khác nhau được tạo từ tập $A=\{0; 1; 2; 3; 4; 5; 6; 7; 8; 9\}$ là $9\cdot\mathrm{A}_{9}^{4}$.
}
\end{ex}

%%==========Câu 695
\begin{ex}%[1D2B2-1]
Một tổ có $6$ học sinh nam và $9$ học sinh nữ. Hỏi có bao nhiêu cách chọn $6$ học sinh đi lao động, trong đó có đúng $2$ học sinh nam?
\choice
{$\mathrm{C}_{6}^{2}+\mathrm{C}_{9}^{4}$}
{$\mathrm{C}_{6}^{2} \mathrm{C}_{13}^{4}$}
{$\mathrm{A}_{6}^{2} \mathrm{A}_{9}^{4}$}
{\True $\mathrm{C}_{6}^{2} \mathrm{C}_{9}^{4}$}
\loigiai{
Số cách chọn $6$ học sinh đi lao động, trong đó có $2$ học sinh nam là $\mathrm{C}_{6}^{2} \cdot \mathrm{C}_{9}^{4}$.
}
\end{ex}

%%==========Câu 696
\begin{ex}%[1D2Y2-1]
Có bao nhiêu số tự nhiên có $5$ chữ số, các chữ số khác $0$ và đôi một khác nhau?
\choice
{$5!$}
{$9^{5}$}
{$\mathrm{C}_{9}^{5}$}
{\True $\mathrm{A}_{9}^{5}$}
\loigiai{
Số các số tự nhiên có $5$ chữ số, các chữ số khác $0$ và đôi một khác nhau là $\mathrm{A}_{9}^{5}$.
}
\end{ex}

%%==========Câu 697
\begin{ex}%[1D2B2-1]
Cho hai đường thẳng $d_{1}$ và $d_{2}$ song song với nhau. Trên $d_{1}$ lấy $5$ điểm phân biệt, trên $d_{2}$ lấy $7$ điểm phân biệt. Hỏi có bao nhiêu tam giác mà các đỉnh của nó được lấy từ các điểm trên hai đường thằng $d_{1}$ và $d_{2}$?
\choice
{$220$}
{\True $175$}
{$1320$}
{$7350$}
\loigiai{
Số tam giác mà các đỉnh của nó được lấy từ các điểm trên hai đường thằng $d_{1}$ và $d_{2}$ là $\mathrm{C}_{5}^{2} \cdot \mathrm{C}_{7}^{1}+\mathrm{C}_{5}^{1} \cdot \mathrm{C}_{7}^{2}=175$.
}
\end{ex}

%%==========Câu 698
\begin{ex}%[1D2B2-1]
Cho hai đường thằng song song. Trên đường thứ nhất có $10$ điểm, trên đường thứ hai có $15$ điểm, có bao nhiêu tam giác được tạo thành từ các điểm đã cho?
\choice
{\True $1725$}
{$1050$}
{$675$}
{$1275$}
\loigiai{
Số tam giác được tạo thành từ các điểm đã cho là $\mathrm{C}_{10}^{2} \cdot \mathrm{C}_{15}^{1}+\mathrm{C}_{10}^{1} \cdot \mathrm{C}_{15}^{2}=1725$.
}
\end{ex}

%%==========Câu 699
\begin{ex}%[1D2B2-1]
Một nhóm gồm $6$ học sinh nam và $7$ học sinh nữ. Hỏi có bao nhiêu cách chọn từ đó ra $3$ học sinh tham gia văn nghệ sao cho luôn có ít nhất một học sinh nam?
\choice
{$245$}
{$3480$}
{$336$}
{\True $251$}
\loigiai{
cách chọn từ đó ra $3$ học sinh tham gia văn nghệ sao cho luôn có ít nhất một học sinh nam là $\mathrm{C}_{13}^{3}-\mathrm{C}_{7}^{3}=251$.
}
\end{ex}

%%==========Câu 700
\begin{ex}%[1D2B2-1]
Một lớp có $40$ học sinh gồm $25$ nam và $15$ nữ. Giáo viên chủ nhiệm muốn chọn $4$ em trực cờ đỏ. Hỏi có bao nhiêu cách chọn nếu ít nhất phải có một nam?
\choice
{\True $\mathrm{C}_{40}^{4}-\mathrm{C}_{15}^{4}$}
{$\mathrm{C}_{25}^{4}$}
{$\mathrm{C}_{25}^{1} \mathrm{C}_{15}^{3}$}
{$\mathrm{C}_{40}^{4}+\mathrm{C}_{15}^{4}$}
\loigiai{
Số chọn $4$ em trực cờ đỏ sao cho có ít nhất phải có một nam là $\mathrm{C}_{40}^{4}-\mathrm{C}_{15}^{4}$
}
\end{ex}

%%==========Câu 701
\begin{ex}%[1D2Y2-1]
Số đường chéo của đa giác đều có $20$ cạnh là bao nhiêu?
\choice
{\True $170$}
{$190$}
{$360$}
{$380$}
\loigiai{
Số đường chéo của đa giác đều có $20$ cạnh là $\mathrm{C}_{20}^{2}-20=170$.
}
\end{ex}
\section{Xác suất}
%%==========Câu 702
\begin{ex}%[1D2B5-2]
Một hộp chứa $11$ quả cầu gồm $5$ quả cầu màu xanh và $6$ quả cầu màu đỏ. Chọn ngẫu nhiên đồng thời $2$ quả cầu từ hộp đó. Xác suất để chọn ra $2$ quả cầu cùng màu bằng
\choice
{$\dfrac{5}{22}$}
{$\dfrac{6}{11}$}
{\True $\dfrac{5}{11}$}
{$\dfrac{8}{11}$}
\loigiai{
Số cách chọn ngẫu nhiên đồng thời $2$ quả cầu từ hộp có $11$ quả cầu là $n(\Omega)=\mathrm{C}_{11}^{2}$.\\
Số cách chọn được $2$ quả cầu cùng màu là $n(A)=\mathrm{C}_{5}^{2}+\mathrm{C}_{6}^{2}$.\\
Xác suất cần tìm là $\mathrm{P}(A)=\dfrac{n(A)}{n(\Omega)}=\dfrac{5}{11}$.
}
\end{ex}

%%==========Câu 703
\begin{ex}%[1D2B5-2]
Trong hộp có $10$ viên bi xanh và $7$ viên bi đỏ. Lấy ngẫu nhiên $2$ viên bi trong hộp đó. Xác suất sao cho $2$ viên bi lấy ra khác màu bằng
\choice
{$\dfrac{21}{136}$}
{\True $\dfrac{35}{68}$}
{$\dfrac{3}{10}$}
{$\dfrac{21}{40}$}
\loigiai{
Số cách chọn ngẫu nhiên đồng thời $2$ viên bi từ hộp có $17$ viên bi là $n(\Omega)=\mathrm{C}_{17}^{2}$.\\
Số cách chọn được $2$ viên bi khác màu là $n(A)=\mathrm{C}_{10}^{1}\cdot\mathrm{C}_{7}^{1}$.\\
Xác suất cần tìm là $\mathrm{P}(A)=\dfrac{n(A)}{n(\Omega)}=\dfrac{35}{68}$.
}
\end{ex}

%%==========Câu 704
\begin{ex}%[1D2B5-2]
Cho một hộp đựng $12$ viên bi, trong đó có $7$ viên bi màu đỏ, $5$ viên bi màu xanh. Lấy ngẫu nhiên $3$ viên bi từ hộp. Xác suất để $3$ bi được lấy có ít nhất $2$ viên bi màu đỏ bằng
\choice
{\True $\dfrac{7}{11}$}
{$\dfrac{8}{11}$}
{$\dfrac{6}{11}$}
{$\dfrac{5}{11}$}
\loigiai{
Số cách lấy ngẫu nhiên đồng thời $3$ viên bi từ hộp có $12$ viên bi là $n(\Omega)=\mathrm{C}_{12}^{3}$.\\
Số cách lấy được $3$ viên bi sao cho có ít nhất $2$ viên bi màu đỏ là $n(A)=\mathrm{C}_{7}^{2}\cdot\mathrm{C}_{5}^{1}+\mathrm{C}_{7}^{3}$.\\
Xác suất cần tìm là $\mathrm{P}(A)=\dfrac{n(A)}{n(\Omega)}=\dfrac{7}{11}$.
}
\end{ex}

%%==========Câu 705
\begin{ex}%[1D2B5-2]
Một hộp chứa $16$ viên bi trong đó có $7$ viên bi trắng, $6$ viên bi xanh và $3$ viên bi đỏ. Lấy ngẫu nhiên $3$ viên bi. Xác suất đế lấy được ít nhất $1$ viên bi xanh bằng
\choice
{$\dfrac{53}{80}$}
{$\dfrac{3}{14}$}
{\True $\dfrac{11}{14}$}
{$\dfrac{27}{80}$}
\loigiai{
Số cách lấy ngẫu nhiên đồng thời $3$ viên bi từ hộp có $16$ viên bi là $n(\Omega)=\mathrm{C}_{16}^{3}$.\\
Số cách lấy được $3$ viên bi sao cho có ít nhất $1$ viên bi màu xanh là $n(A)=\mathrm{C}_{16}^{3}-\mathrm{C}_{10}^{3}$.\\
Xác suất cần tìm là $\mathrm{P}(A)=\dfrac{n(A)}{n(\Omega)}=\dfrac{11}{14}$.
}
\end{ex}

%%==========Câu 706
\begin{ex}%[1D2B5-2]
Một tổ có $7$ nam và $3$ nữ. Chọn ngẫu nhiên đồng thời $2$ người. Xác suất sao cho $2$ người được chọn có ít nhất $1$ người nữ bằng
\choice
{$\dfrac{12}{15}$}
{$\dfrac{7}{15}$}
{$\dfrac{2}{15}$}
{\True $\dfrac{8}{15}$}
\loigiai{
Số cách chọn ngẫu nhiên đồng thời $2$ người trong $10$ người là $n(\Omega)=\mathrm{C}_{10}^{2}$.\\
Số cách chọn được $2$ người có ít nhất $1$ người nữ là $n(A)=\mathrm{C}_{3}^{1}\cdot\mathrm{C}_{7}^{1}+\mathrm{C}_{3}^{2}$.\\
Xác suất cần tìm là $\mathrm{P}(A)=\dfrac{n(A)}{n(\Omega)}=\dfrac{8}{15}$.
}
\end{ex}

%%==========Câu 707
\begin{ex}%[1D2B5-2]
Chọn ngẫu nhiên $5$ học sinh trong một lớp có $15$ nam và $10$ nữ để tham gia đồng diễn. Tính xác suất sao cho $5$ học sinh được chọn có cả nam lẫn nữ và số học sinh nữ ít hơn số học sinh nam bằng
\choice
{$\dfrac{352}{506}$}
{\True $\dfrac{325}{506}$}
{$\dfrac{235}{506}$}
{$\dfrac{253}{506}$}
\loigiai{
Số cách chọn ngẫu nhiên đồng thời $5$ học sinh trong lớp có $25$ học sinh là $n(\Omega)=\mathrm{C}_{25}^{5}$.\\
Số cách chọn được $5$ học sinh có cả nam lẫn nữ và số học sinh nữ ít hơn số học sinh nam $n(A)=\mathrm{C}_{15}^{4}\cdot\mathrm{C}_{10}^{1}+\mathrm{C}_{15}^{3}\cdot\mathrm{C}_{10}^{2}$.\\
Xác suất cần tìm là $\mathrm{P}(A)=\dfrac{n(A)}{n(\Omega)}=\dfrac{325}{506}$.
}
\end{ex}

%%==========Câu 708
\begin{ex}%[1D2B5-2]
Một hộp đựng $11$ viên bi được đánh số từ $1$ đến $11$. Lấy ngẫu nhiên $4$ viên bi, rồi cộng các số trên các viên bi lại với nhau. Xác suất để kết quả thu được là một số lẻ bằng
\choice
{$\dfrac{31}{32}$}
{\True $\dfrac{16}{33}$}
{$\dfrac{11}{32}$}
{$\dfrac{21}{32}$}
\loigiai{
Số cách chọn ngẫu nhiên $4$ viên bi trong $11$ viên bi là $n(\Omega)=\mathrm{C}_{11}^{4}$.\\
Số cách chọn được $4$ viên bi để tổng các số ghi trên $4$ viên bi là một số lẻ là $n(A)=\mathrm{C}_{6}^{1}\cdot\mathrm{C}_{5}^{3}+\mathrm{C}_{6}^{3}\cdot\mathrm{C}_{5}^{1}$.\\
Xác suất cần tìm là $\mathrm{P}(A)=\dfrac{n(A)}{n(\Omega)}=\dfrac{16}{33}$.
}
\end{ex}

%%==========Câu 709
\begin{ex}%[1D2B5-2]
Lấy ngẫu nhiên một thẻ từ một hộp chứa $20$ thẻ được đánh số từ $1$ đến $20$. Xác suất để lấy được thẻ ghi số chia hết cho $3$ là
\choice
{$\dfrac{1}{20}$}
{\True $\dfrac{3}{10}$}
{$\dfrac{1}{2}$}
{$\dfrac{3}{20}$}
\loigiai{
Số cách chọn ngẫu nhiên $1$ thẻ trong $20$ thẻ là $n(\Omega)=\mathrm{C}_{20}^{1}$.\\
Số cách chọn được thẻ mang số chia hết cho $3$ là $n(A)=\mathrm{C}_{6}^{1}$.\\
Xác suất cần tìm là $\mathrm{P}(A)=\dfrac{n(A)}{n(\Omega)}=\dfrac{3}{10}$.
}
\end{ex}

%%==========Câu 710
\begin{ex}%[1D2B5-2]
Chọn ngẩu nhiên $2$ số khác nhau từ $27$ số nguyên dương đầu tiên. Xác suất để chọn được $2$ số có tổng là một số chẵn bằng
\choice
{\True $\dfrac{13}{27}$}
{$\dfrac{365}{729}$}
{$\dfrac{1}{2}$}
{$\dfrac{14}{27}$}
\loigiai{
Số cách chọn ngẫu nhiên $2$ số trong $27$ số nguyên dương đầu tiên là $n(\Omega)=\mathrm{C}_{27}^{2}$.\\
Số cách chọn được $2$ số có tổng là một số chẵn là $n(A)=\mathrm{C}_{13}^{2}+\mathrm{C}_{14}^{2}$.\\
Xác suất cần tìm là $\mathrm{P}(A)=\dfrac{n(A)}{n(\Omega)}=\dfrac{13}{27}$.
}
\end{ex}

%%==========Câu 711
\begin{ex}%[1D2B5-2]
Cho $14$ tấm thẻ đánh số từ $1$ đển $14$. Chọn ngẫu nhiên $3$ tấm thẻ. Xác suất để tích $3$ số ghi trên $3$ tấm thẻ này chia hết cho $3$ bằng
\choice
{$\dfrac{30}{91}$}
{\True $\dfrac{61}{91}$}
{$\dfrac{31}{91}$}
{$\dfrac{12}{17}$}
\loigiai{
Số cách chọn ngẫu nhiên $3$ tấm thẻ trong $14$ tấm thẻ là $n(\Omega)=\mathrm{C}_{14}^{3}$.\\
Số cách chọn được $3$ tấm thẻ để tích $3$ số ghi trên $3$ tấm thẻ này chia hết cho $3$ là $n(A)=\mathrm{C}_{4}^{1}\cdot\mathrm{C}_{10}^{2}+\mathrm{C}_{4}^{2}\cdot\mathrm{C}_{10}^{1}+\mathrm{C}_{4}^{3}$.\\
Xác suất cần tìm là $\mathrm{P}(A)=\dfrac{n(A)}{n(\Omega)}=\dfrac{61}{91}$.
}
\end{ex}

%%==========Câu 712
\begin{ex}%[1D2B5-2]
Đội văn nghệ của một lớp có $5$ bạn nam và $7$ bạn nữ. Chọn ngẫu nhiên $5$ bạn tham gia biểu diễn văn nghệ. Tính xác suất để $5$ bạn được chọn có đủ nam, nữ và số bạn nam lớn hơn $2$
\choice
{$\dfrac{547}{792}$}
{\True $\dfrac{245}{792}$}
{$\dfrac{210}{792}$}
{$\dfrac{582}{792}$}
\loigiai{
Số cách chọn ngẫu nhiên đồng thời $5$ học sinh trong lớp có $12$ học sinh là $n(\Omega)=\mathrm{C}_{12}^{5}$.\\
Số cách chọn được $5$ học sinh có cả nam lẫn nữ và số học sinh nam lớn hơn $2$ là $n(A)=\mathrm{C}_{5}^{2}\cdot\mathrm{C}_{7}^{3}+\mathrm{C}_{5}^{3}\cdot\mathrm{C}_{7}^{2}+\mathrm{C}_{5}^{4}\cdot\mathrm{C}_{7}^{1}+\mathrm{C}_{5}^{5}$.\\
Xác suất cần tìm là $\mathrm{P}(A)=\dfrac{n(A)}{n(\Omega)}=\dfrac{245}{792}$.
}
\end{ex}

%%==========Câu 713
\begin{ex}%[1D2B5-2]
Một tổ chuyên môn tiếng Anh của trường Đại học X gồm có $7$ thầy giáo và $5$ cô giáo, trong đó thầy Xuân và cô Hạ là vợ chồng. Tổ chọn ngẫu nhiên $5$ người để lập hội đồng chấm thi vấn đáp tiếng Anh B1 khung châu Âu. Xác xuất để sao cho hội đồng có $3$ thầy, $2$ cô và nhất thiết có thầy Xuân hoặc cô Hạ nhưng không có cả hai là
\choice
{$\dfrac{5}{44}$}
{$\dfrac{5}{88}$}
{$\dfrac{85}{792}$}
{\True $\dfrac{85}{396}$}
\loigiai{
Số cách chọn ngẫu nhiên hội đồng có $5$ thầy, cô trong tổng số $12$ thầy, cô là $n(\Omega)=\mathrm{C}_{12}^{5}$.\\
Số cách chọn được hội đồng có $3$ thầy, $2$ cô và nhất thiết có thầy Xuân hoặc cô Hạ nhưng không có cả hai là $n(A)=\mathrm{C}_{6}^{2}\cdot\mathrm{C}_{4}^{2}+\mathrm{C}_{6}^{3}\cdot\mathrm{C}_{4}^{1}$.\\
Xác suất cần tìm là $\mathrm{P}(A)=\dfrac{n(A)}{n(\Omega)}=\dfrac{85}{396}$.
}
\end{ex}

%%==========Câu 714
\begin{ex}%[1D2B5-2]
Trên giá sách có $4$ quyển sách Toán, $3$ quyển sách Lý, $2$ quyển sách Hóa. Lấy ngẫu nhiên $3$ quyển sách. Tính xác suất để trong ba quyển sách lấy ra có ít nhất một quyển sách Toán.
\choice
{$\dfrac{2}{7}$}
{$\dfrac{3}{4}$}
{\True $\dfrac{37}{42}$}
{$\dfrac{10}{21}$}
\loigiai{
Số cách lấy ngẫu nhiên $3$ quyển sách trong $9$ quyển sách là $n(\Omega)=\mathrm{C}_{9}^{3}$.\\
Số cách lấy được $3$ quyển sách trong đó có ít nhất một quyển sách Toán là $n(A)=\mathrm{C}_{9}^{3}-\mathrm{C}_{5}^{3}$.\\
Xác suất cần tìm là $\mathrm{P}(A)=\dfrac{n(A)}{n(\Omega)}=\dfrac{37}{42}$.
}
\end{ex}

%%==========Câu 715
\begin{ex}%[1D2B5-2]
Thầy giáo cho đề cương ôn thi có $20$ câu hỏi. Mỗi đề thi có $4$ câu lấy ngẫu nhiên từ đề cương đó. Một thí sinh đã học thuộc $10$ câu trong đề cương. Xác suất để thí sinh đó rút được đề thi có ít nhất $2$ câu đã học thuộc.
\choice
{$\dfrac{43}{136}$}
{$\dfrac{14}{83}$}
{\True $\dfrac{229}{323}$}
{$\dfrac{118}{231}$}
\loigiai{
Số cách lấy ngẫu nhiên $4$ câu trong $20$ câu là $n(\Omega)=\mathrm{C}_{20}^{4}$.\\
Số cách lấy được $4$ câu trong đó có ít nhất $2$ câu đã thuộc là $n(A)=\mathrm{C}_{20}^{4}-\mathrm{C}_{10}^{1}\cdot\mathrm{C}_{10}^{3}-\mathrm{C}_{10}^{4}$.\\
Xác suất cần tìm là $\mathrm{P}(A)=\dfrac{n(A)}{n(\Omega)}=\dfrac{229}{323}$.
}
\end{ex}

%%==========Câu 716
\begin{ex}%[1D2B5-2]
Giải bóng chuyền quốc tễ VTV Cup có $8$ đội tham gia, trong đó có $2$ đội Việt Nam. Ban tổ chức bốc thăm ngẫu nhiên để chia thành $2$ bảng đấu, mỗi bảng $4$ đội. Xác suất để $2$ đội Việt Nam nằm ở $2$ bảng đấu khác nhau là
\choice
{\True $\dfrac{2}{7}$}
{$\dfrac{5}{7}$}
{$\dfrac{3}{7}$}
{$\dfrac{4}{7}$}
\loigiai{
Số cách bốc thăm ngẫu nhiên để chia thành $2$ bảng đấu là $n(\Omega)=\mathrm{C}_{8}^{4}$.\\
Số cách bốc được $2$ đội Việt Nam nằm ở $2$ bảng đấu khác nhau là $n(A)=\mathrm{C}_{6}^{3}$.\\
Xác suất cần tìm là $\mathrm{P}(A)=\dfrac{n(A)}{n(\Omega)}=\dfrac{2}{7}$.
}
\end{ex}

%%==========Câu 717
\begin{ex}%[1D2B5-2]
Một tố có $9$ học sinh nam và $3$ học sinh nữ. Chia tổ thành $3$ nhóm, mỗi nhóm $4$ người để làm $3$ nhiệm vụ khác nhau. Xác suất khi chia ngẫu nhiên nhóm nào cũng có nữ là
\choice
{$\dfrac{8}{55}$}
{$\dfrac{292}{34650}$}
{$\dfrac{292}{1080}$}
{\True $\dfrac{16}{55}$}
\loigiai{
Số cách chia ngẫu nhiên $12$ học sinh thành $3$ nhóm là $n(\Omega)=\mathrm{C}_{12}^{4}\cdot\mathrm{C}_{8}^{4}$.\\
Số cách chia $12$ học sinh thành $3$ nhóm mà nhóm nào cũng có nữ là $n(A)=\mathrm{C}_{3}^{1}\cdot\mathrm{C}_{9}^{3}\cdot\mathrm{C}_{2}^{1}\cdot\mathrm{C}_{6}^{3}$.\\
Xác suất cần tìm là $\mathrm{P}(A)=\dfrac{n(A)}{n(\Omega)}=\dfrac{16}{55}$.
}
\end{ex}

%%==========Câu 718
\begin{ex}%[1D2B5-2]
Trong cuộc thi \lq\lq  Tìm kiếm tài năng Việt\rq\rq, có $20$ bạn lọt vào vòng chung kết, trong đó có $5$ bạn nữ và $15$ bạn nam. Để sắp xếp vị trí thi đấu, Ban tổ chức chia thành $4$ nhóm $A$, $B$, $C$, $D$, mỗi nhóm $5$ bạn. Tính xác suất để $5$ bạn nữ thuộc cùng một nhóm
\choice
{$\dfrac{1}{3876}$}
{\True $\dfrac{1}{646}$}
{$\dfrac{2}{3465}$}
{$\dfrac{5}{3876}$}
\loigiai{
Số cách chia $20$ bạn thành $4$ nhóm, mỗi nhóm $5$ bạn là $n(\Omega)=\mathrm{C}_{20}^{5}\cdot\mathrm{C}_{15}^{5}\cdot\mathrm{C}_{10}^{5}$.\\
Số cách chia để $5$ bạn nữ thuộc cùng một nhóm là $n(A)=\mathrm{C}_{15}^{5}\cdot\mathrm{C}_{10}^{5}\cdot4!$.\\
Xác suất cần tìm là $\mathrm{P}(A)=\dfrac{n(A)}{n(\Omega)}=\dfrac{1}{646}$.
}
\end{ex}

%%==========Câu 719
\begin{ex}%[1D2B5-2]
Một hộp chứa $10$ quả cầu màu đỏ đánh số từ $1$ đển $10$ và $15$ quả cầu màu xanh được đánh số từ $1$ đến $15$. Chọn ngẫu nhiên $2$ quả cầu. Xác suất để chọn được $2$ quả cầu khác màu và tổng của các số trên $2$ quà cầu là một số lẻ bằng
\choice
{$\dfrac{1}{2}$}
{$\dfrac{1}{5}$}
{\True $\dfrac{1}{4}$}
{$\dfrac{3}{4}$}
\loigiai{
Số cách chọn ngẫu nhiên $2$ quả cầu trong hộp chứa $25$ quả cầu là $n(\Omega)=\mathrm{C}_{25}^{2}$.\\
Số cách chọn được $2$ quả cầu khác màu và tổng của các số trên $2$ quà cầu là một số lẻ là $n(A)=\mathrm{C}_{5}^{1}\cdot\mathrm{C}_{8}^{1}+\mathrm{C}_{5}^{1}\cdot\mathrm{C}_{7}^{1}$.\\
Xác suất cần tìm là $\mathrm{P}(A)=\dfrac{n(A)}{n(\Omega)}=\dfrac{1}{4}$.
}
\end{ex}

%%==========Câu 720
\begin{ex}%[1D2B5-2]
Có $30$ tấm thẻ được đánh số thứ tự từ $1$ đển $30$. Chọn ngẫu nhiên ra $10$ tấm. Tính xác suất để lấy được $5$ tấm thẻ mang số lẻ, $5$ tấm thẻ mang số chẵn trong đó có đúng một tấm thẻ mang số chia hết cho $10$
\choice
{\True $\dfrac{99}{667}$}
{$\dfrac{568}{667}$}
{$\dfrac{33}{667}$}
{$\dfrac{634}{667}$}
\loigiai{
Số cách chọn ngẫu nhiên $10$ tấm thẻ trong $30$ tấm thẻ là $n(\Omega)=\mathrm{C}_{30}^{10}$.\\
Số cách chọn được $5$ tấm thẻ mang số lẻ, $5$ tấm thẻ mang số chẵn trong đó có đúng một tấm thẻ mang số chia hết cho $10$ là $n(A)=\mathrm{C}_{3}^{1}\cdot\mathrm{C}_{12}^{4}\cdot\mathrm{C}_{15}^{5}$.\\
Xác suất cần tìm là $\mathrm{P}(A)=\dfrac{n(A)}{n(\Omega)}=\dfrac{99}{667}$.
}
\end{ex}

%%==========Câu 721
\begin{ex}%[1D2B5-2]
Có $40$ tấm thẻ đánh số thứ tự từ $1$ đến $40$. Chọn ngẫu nhiên ra $10$ tấm thẻ. Tính xác suất để lấy được $5$ tấm thẻ mang số lẻ, $5$ tấm thẻ mang số chẵn trong đó có đúng một thẻ mang số chia hết cho $6$ bằng
\choice
{\True $\dfrac{126}{1147}$}
{$\dfrac{16}{33}$}
{$\dfrac{1787}{2300}$}
{$\dfrac{127}{380}$}
\loigiai{
Số cách chọn ngẫu nhiên $10$ tấm thẻ trong $40$ tấm thẻ là $n(\Omega)=\mathrm{C}_{40}^{10}$.\\
Số cách chọn được $5$ tấm thẻ mang số lẻ, $5$ tấm thẻ mang số chẵn trong đó có đúng một thẻ mang số chia hết cho $6$ là $n(A)=\mathrm{C}_{6}^{1}\cdot\mathrm{C}_{14}^{4}\cdot\mathrm{C}_{20}^{5}$.\\
Xác suất cần tìm là $\mathrm{P}(A)=\dfrac{n(A)}{n(\Omega)}=\dfrac{126}{1147}$.
}
\end{ex}

%%==========Câu 722
\begin{ex}%[1D2B5-2]
Một hộp đựng $9$ thẻ được đánh số $1; 2; 3; 4; 5; 6; 7; 8; 9$. Rút ngẫu nhiên $2$ thẻ và nhân $2$ số ghi trên $2$ thẻ lại với nhau. Tính xác suất để kết quả thu được là một số chẵn.
\choice
{$\dfrac{5}{18}$}
{$\dfrac{1}{6}$}
{$\dfrac{8}{9}$}
{\True $\dfrac{13}{18}$}
\loigiai{
Số cách rút ngẫu nhiên $2$ thẻ trong $9$ thẻ là $n(\Omega)=\mathrm{C}_{9}^{2}$.\\
Số cách rút được $2$ thẻ mà tích $2$ số ghi trên thẻ là một số chắn là $n(A)=\mathrm{C}_{5}^{1}\cdot\mathrm{C}_{4}^{1}+\mathrm{C}_{4}^{2}$.\\
Xác suất cần tìm là $\mathrm{P}(A)=\dfrac{n(A)}{n(\Omega)}=\dfrac{13}{18}$.
}
\end{ex}

%%==========Câu 723
\begin{ex}%[1D2B5-2]
Sau buổi hội nghị, $10$ thành viên ban tố chức đứng thành một hang ngang để chụp hình. Biết rằng có $3$ nữ. Tính xác xuất để $3$ nữ đó luôn cạnh nhau.
\choice
{$\dfrac{1}{5}$}
{\True $\dfrac{1}{15}$}
{$\dfrac{3}{25}$}
{$\dfrac{2}{25}$}
\loigiai{
Số cách xếp ngẫu nhiên $10$ thành viên ban tố chức đứng thành một hang ngang để chụp hình là $n(\Omega)=10!$.\\
Số các xếp để $3$ nữ đó luôn cạnh nhau là $n(A)=8!\cdot 3!$.\\
Xác suất cần tìm là $\mathrm{P}(A)=\dfrac{n(A)}{n(\Omega)}=\dfrac{1}{15}$.
}
\end{ex}

%%==========Câu 724
\begin{ex}%[1D2B5-2]
Một nhóm học sinh gồm $4$ học sinh nam và $4$ học sinh nữ được xếp vào $8$ chiếc ghế kê thành hàng ngang sao cho mỗi ghế có đúng một học sinh ngồi. Xác suất để các bạn học sinh nam và nữ ngồi xen kẽ nhau bằng
\choice
{$\dfrac{1}{70}$}
{\True $\dfrac{1}{35}$}
{$\dfrac{2}{35}$}
{$\dfrac{1}{2}$}
\loigiai{
Số cách xếp ngẫu nhiên $8$ học sinh thành một hang ngang là $n(\Omega)=8!$.\\
Số các xếp để các bạn học sinh nam và nữ ngồi xen kẽ nhau là $n(A)=2\cdot 4!\cdot 4!$.\\
Xác suất cần tìm là $\mathrm{P}(A)=\dfrac{n(A)}{n(\Omega)}=\dfrac{1}{35}$.
}
\end{ex}

%%==========Câu 725
\begin{ex}%[1D2B5-2]
Có $6$ học sinh lớp $11$ và $3$ học sinh lớp $12$ xếp ngẫu nhiên vào $9$ ghế thành một dãy. Tính xác suất để xếp được $3$ học sinh lớp $12$ xen kẽ giữa $6$ học sinh lớp $11$.
\choice
{\True $\dfrac{5}{42}$}
{$\dfrac{3}{11}$}
{$\dfrac{4}{21}$}
{$\dfrac{14}{55}$}
\loigiai{
Số cách xếp ngẫu nhiên $9$ học sinh thành một hang ngang là $n(\Omega)=9!$.\\
Số các xếp để $3$ học sinh lớp $12$ xen kẽ giữa $6$ học sinh lớp $11$ là $n(A)=6!\cdot\mathrm{A}_{5}^{3}$.\\
Xác suất cần tìm là $\mathrm{P}(A)=\dfrac{n(A)}{n(\Omega)}=\dfrac{5}{42}$.
}
\end{ex}

%%==========Câu 726
\begin{ex}%[1D2B5-2]
Có $8$ học sinh nam và $4$ học sinh nữ được xếp thành hàng ngang. Tính xác suất để khi xếp sao cho $2$ học sinh nữ không đứng cạnh nhau?
\choice
{$\dfrac{1}{5}$}
{\True $\dfrac{14}{55}$}
{$\dfrac{5}{12}$}
{$\dfrac{1}{2}$}
\loigiai{
Số cách xếp ngẫu nhiên $12$ học sinh thành một hang ngang là $n(\Omega)=12!$.\\
Số các xếp để $2$ học sinh nữ không đứng cạnh nhau là $n(A)=8!\cdot\mathrm{A}_{9}^{4}$.\\
Xác suất cần tìm là $\mathrm{P}(A)=\dfrac{n(A)}{n(\Omega)}=\dfrac{14}{55}$.
}
\end{ex}

%%==========Câu 727
\begin{ex}%[1D2B5-2]
Từ các chữ số $1; 2; 3; 4; 5; 6; 7; 8$ ta lập các số tự nhiên có $6$ chữ số, mà các chữ số đôi một khác nhau. Chọn ngẫu nhiên một số vừa lập, tính xác suất để chọn được một số có đúng $3$ chữ số lẻ mà các chữ số lẻ xếp kề nhau.
\choice
{$\dfrac{1}{5}$}
{\True $\dfrac{4}{35}$}
{$\dfrac{3}{7}$}
{$\dfrac{4}{7}$}
\loigiai{
Số các số tự nhiên có $6$ chữ số đôi một khác nhau lập từ các chữ số $1; 2; 3; 4; 5; 6; 7; 8$ là $n(\Omega)=\mathrm{A}_{8}^{6}$.\\
Số các số có đúng $3$ chữ số lẻ mà các chữ số lẻ xếp kề nhau là $n(A)=\mathrm{C}_{4}^{3}\cdot\mathrm{C}_{4}^{3}\cdot 4!\cdot 3!$.\\
Xác suất cần tìm là $\mathrm{P}(A)=\dfrac{n(A)}{n(\Omega)}=\dfrac{4}{35}$.
}
\end{ex}

%%==========Câu 728
\begin{ex}%[1D2B5-2]
Xếp ngẫu nhiên $5$ bạn An, Bình, Cường, Dũng, Đông ngồi vào một dãy $5$ ghế thẳng hàng (mỗi bạn ngồi $1$ ghế). Xác suất của biến cố \lq\lq  hai bạn An và Bình không ngồi cạnh nhau\rq\rq bằng
\choice
{\True $\dfrac{3}{5}$}
{$\dfrac{2}{5}$}
{$\dfrac{1}{5}$}
{$\dfrac{4}{5}$}
\loigiai{
Số cách xếp ngẫu nhiên $5$ bạn thành một hang ngang là $n(\Omega)=5!$.\\
Số các xếp để hai bạn An và Bình không ngồi cạnh nhau là $n(A)=5!-4!\cdot 2!$.\\
Xác suất cần tìm là $\mathrm{P}(A)=\dfrac{n(A)}{n(\Omega)}=\dfrac{3}{5}$.
}
\end{ex}




\begin{ex}%[1D2B5-2]%729
    Xếp ngẫu nhiên $2$ quả cầu xanh, $2$ quả cầu đỏ, $2$ quả cầu trắng (các quả cầu này đôi một khác nhau) thành một hàng ngang. Tính xác suất để $2$ quả cầu màu trắng không xếp cạnh nhau?
    \choice
    {\True $\dfrac{2}{3}$}
    {$\dfrac{1}{3}$}
    {$\dfrac{5}{6}$}
    {$\dfrac{1}{2}$}
    \loigiai{
        Không gian mẫu $ n(\Omega)=6!=720 $.\\
        Gọi $ A $   là biến cố hai quả cầu trắng không xếp cạnh nhau.\\
        Khi đó $ \overline{A} $ là biến cố hai quả cầu trắng xếp cạnh nhau.\\
        Ta có $ n(\overline{A})= 2\cdot 5!$.\\
        Suy ra $ P(\overline{A})=\dfrac{n(\overline{A})}{n(\Omega)}=\dfrac{2\cdot 5!}{6!}=\dfrac{1}{3}$.\\
        Vậy $ P(A)=1-\dfrac{1}{3}=\dfrac{2}{3} $.
    }
\end{ex}

\begin{ex}%[1D2B5-2]%730
    Xếp $10$ học sinh gồm $4$ học sinh lớp $12$, ba học sinh lớp $11$ và ba học sinh lớp $10$ vào một hàng ngang gồm $10$ ghế được đánh số từ $1$ đến $10$. Tính xác suất để không có hai học sinh lớp $12$ ngồi cạnh nhau.
    \choice
    {$\dfrac{20}{253}$}
    {$\dfrac{1}{9}$}
    {\True $\dfrac{1}{6}$}
    {$\dfrac{1}{3}$}
    \loigiai{Ta có Không gian mẫu $ n(\Omega)=10!$.\\
        Gọi $A$ là biến cố \lq\lq  $10$ học sinh ngồi vào một hàng ngang gồm $10$ ghế sao cho không có học sinh lớp $12$ ngồi cạnh nhau.\\
        Số cách xếp $6$ học sinh gồm ba học sinh lớp $11$ và ba học sinh lớp $10$ là $6!$.\\
        Sau đó có $\mathrm{A_7^4}$ cách xếp $4$ học sinh lớp $12$ xen kẽ vào $4$ trong $7$ vị trí ở giữa và ở hai đầu của $6$ học sinh đã xếp ở trên.\\
        Suy ra $\mathrm{n\left(A\right)}=\mathrm{A_7^4}\cdot 6!$ .\\
        $ \Rightarrow \mathrm{P\left(A\right)}=\dfrac{\mathrm{n\left(A\right)}}{\mathrm{n\left(\omega\right)}}=\dfrac{1}{6}$.
    }
\end{ex}

\begin{ex}%[1D2B5-2]%731
    Từ $12$ học sinh gồm $5$ học sinh giỏi, $4$ học sinh khá, $3$ học sinh trung bình, giáo viê muốn thành lập $4$ nhóm làm $4$ bài tập lớn khác nhau, mỗi nhóm $3$ học sinh. Tính xác suất để nhóm nào cũng có học sinh giỏi và học sinh khá.
    \choice
    {\True $\dfrac{36}{385}$}
    {$\dfrac{18}{365}$}
    {$\dfrac{72}{385}$}
    {$\dfrac{144}{385}$}
    \loigiai{\begin{itemize}
            \item Xếp vào mỗi nhóm một học sinh khá có $4!$ cách.
            \item Xếp $5$ học sinh giỏi vào $4$ nhóm thì có một nhóm có $2$ học sinh giỏi. Chọn nhóm có $2$ học sinh giỏi có $4$ cách, chọn $2$ học sinh giỏi có $\mathrm{C_5^2}$ cách, xếp $3$ học sinh giỏi còn lại có $3!$ cách.
            \item Xếp $3$ học sinh trung bình có $3!$ cách.
            \item Xác suất để nhóm nào cũng có học sinh giỏi và học sinh khá là $\dfrac{4!\cdot 4\cdot\mathrm{C_5^2}\cdot 31\cdot 3!}{\mathrm{C_{12}^3}\cdot \mathrm{C_9^3}\cdot \mathrm{C_6^3}\cdot \mathrm{C_3^3}}=\dfrac{36}{385}$.
        \end{itemize}
    }
\end{ex}

\begin{ex}%[1D2B5-2]%732
    Đại hội đại biểu toàn quốc lần thứ $XIII$ Đảng Cộng Sản Việt Nam năm $2020$ có $10$ đại biểu trong đó có $A,~B,~C$ tham dự đại hội được xếp vào ngồi một dãy ghế dài $10$ chỗ trống. Tính xác suất để $A$ và $B$ luôn ngồi cạnh nhau nhưng $A$ và $C$ không được ngồi cạnh nhau.
    \choice
    {\True $\dfrac{8}{45}$}
    {$\dfrac{1}{5}$}
    {$\dfrac{1}{6}$}
    {$\dfrac{11}{45}$}
    \loigiai{Số cách xếp $3$ đại biểu $A,~B,~C$  vào $10 $ chỗ trống là $\mathrm{n\left(\omega\right)}=\mathrm{A_{10}^3}=720$.\\
        Gọi $D\colon $ \lq\lq  là biến cố $A$ và $B$ luôn ngồi cạnh nhau nhưng $A$ và $C$ không được ngồi cạnh nhau\rq\rq.\\
        \begin{itemize}
            \item Trường hợp $A$ ngồi đầu dãy 
            \begin{itemize}
                \item $A$ có $2$ cách chọn.
                \item $B$ có $1$ cách chọn.
                \item $C$ có $8$ cách chọn.
                \item Suy ra có $16$ cách chọn.
            \end{itemize}
            \item Trường hợp $A$ ngồi giữa dãy 
            \begin{itemize}
                \item $A$ có $8$ cách chọn.
                \item $B$ có $2$ cách chọn.
                \item $C$ có $7$ cách chọn.
                \item Suy ra có $112$ cách chọn.
            \end{itemize}
            \item $\mathrm{P\left(D\right)=\dfrac{128}{720}}=\dfrac{8}{45}$.
        \end{itemize}
    }
\end{ex}

\begin{ex}%[1D2B5-2]%733
    Có $4$ viên bi xanh được đánh số từ $1$ đến $4$ và $4$ viên bi đỏ cũng được đánh số từ $1$ đến $4$. Xếp $8$ viên bi này thành một hàng ngang. Tính xác suất để không có hai viên bi đỏ nào cạnh nhau đồng thời hai viên bi mang số $1$ luôn cạnh nhau.
    \choice
    {\True $\dfrac{1}{35}$}
    {$\dfrac{3}{70}$}
    {$\dfrac{2}{35}$}
    {$\dfrac{1}{70}$}
    \loigiai{Ta có $\mathrm{n\left(\Omega\right)}=8!$.\\
        Xếp $4$ viên bi xanh trước có $4!$ cách.\\
        Suy ra có $5$ vị trí xếp bi đỏ.\\ 
        Số cách xếp bi đỏ số $1$ cạnh bi xanh số $1$ là $2$ cách.\\
        Xếp $3$ bi đỏ còn lại có $\mathrm{C_4^3}\cdot 3!$.
        Do đó xác suất để không có hai viên bi đỏ nào cạnh nhau đồng thời hai viên bi mang số $1$ luôn cạnh nhau là $\dfrac{\mathrm{C_4^3}\cdot 3!\cdot 2\cdot 4!}{8!}=\dfrac{1}{35}$.
    }
\end{ex}

\begin{ex}%[1D2B5-2]%734
    Gọi $S$ là tập hợp các số tự nhiên có $4$ chữ số khác nhau được tạo từ tập $E=\{1;2;3;4;5\}$. Chọn ngẫu nhiên một số từ tập $S$. Tính xác suất để số được chọn là một số chẵn?
    \choice
    {$\dfrac{3}{4}$}
    {\True $\dfrac{2}{5}$}
    {$\dfrac{3}{5}$}
    {$\dfrac{1}{2}$}
    \loigiai{
        Số phần tử của $ S $ là $ \mathrm{A}_{5}^4=120 $.\\
        Không gian mẫu $ n(\Omega)=120$.\\
        Trong $ 120 $ số của tập $ S $ có $ 72 $ số lẻ và $ 48 $ số chẵn.
        $ P(A)=\dfrac{n(A)}{n(\Omega)}=\dfrac{48}{120}=\dfrac{2}{5}$.       
    }
\end{ex}

\begin{ex}%[1D2G5-5]%735
    Cho tập hợp $S=\{1;2;3;\cdots;19;20\}$ gồm $20$ số tự nhiên từ $1$ đến $20$, lấy ngẫu nhiên $3$ số thuộc $S$ $E=\{1;2;3;4;5\}$. xác suất để $3$ số lấy được lập thành một cấp số cộng bằng
    \choice
    {$\dfrac{7}{38}$}
    {$\dfrac{5}{38}$}
    {$\dfrac{3}{38}$}
    {$\dfrac{1}{114}$}
    \loigiai{
        Số phần tử không gian mẫu $n(\Omega)=\mathrm{C}_{20}^{3}$.\\
        Gọi $a$, $b$, $c$ là ba số lấy ra theo thứ tự đó lập thành cấp số cộng, nên $b=\dfrac{a+c}{2} \in \mathbb{N}$. \\
        Do đó $a$ và $c$ cùng chẵn hoặc cùng lẻ và hơn kém nhau ít nhất 2 đơn vị.\\
        Số cách chọn bộ $\left(a; b ; c\right)$ theo thứ tự đó lập thành cấp số cộng bằng số cặp $\left(a; c\right)$ cùng chẵn hoặc cùng lẻ, số cách chọn là $2$ . $\mathrm{C}_{10}^{2}$.\\ 
        Vậy xác suất cần tính là $P=\dfrac{2\cdot\mathrm{C}_{10}^{2}}{\mathrm{C}_{20}^{3}}=\dfrac{3}{38}$.
    }
\end{ex}
\begin{ex}%[1D2G5-5]%736
    Cho tập số $\{1;2;3;4;\cdots;30\}$. Xác suất lấy ra ba số sao cho ba số đó lập thành một cấp số cộng bằng
    \choice
    {$\dfrac{3}{16}$}
    {$\dfrac{3}{58}$}
    {$\dfrac{45}{812}$}
    {$\dfrac{24}{19}$}
    \loigiai{
        Số phần tử không gian mẫu $n(\Omega)=\mathrm{C}_{30}^{3}$.\\
        Gọi $a$, $ b$, $c$ là ba số lấy ra theo thứ tự đó lập thành cấp số cộng, nên $b=\dfrac{a+c}{2} \in \mathbb{N}$. \\
        Do đó $a$ và $c$ cùng chẵn hoặc cùng lẻ và hơn kém nhau ít nhất 2 đơn vị.\\
        Số cách chọn bộ $\left(a; b ; c\right)$ theo thứ tự đó lập thành cấp số cộng bằng số cặp $\left(a; c\right)$ cùng chẵn hoặc cùng lẻ, số cách chọn là $2$ . $\mathrm{C}_{15}^{2}$.\\ 
        Vậy xác suất cần tính là $P=\dfrac{2\cdot\mathrm{C}_{15}^{2}}{\mathrm{C}_{30}^{3}}=\dfrac{3}{58}$.
    }
\end{ex}
\begin{ex}%[1D2G5-5]%737
    Cho $H=\{n\in \mathbb{N^*}|n\le 100\}$. Chọn ngẫu nhiên ba phần tử thuộc tập $H$. Tính xác suất để chọn được ba phần tử lập thành một cấp số cộng?
    \choice
    {$\dfrac{1}{132}$}
    {$\dfrac{2}{275}$}
    {\True $\dfrac{1}{66}$}
    {$\dfrac{4}{275}$}
    \loigiai{
        Số phần tử không gian mẫu $n(\Omega)=\mathrm{C}_{100}^{3}$.\\
        Gọi $a$, $ b$, $c$ là ba số lấy ra theo thứ tự đó lập thành cấp số cộng, nên $b=\dfrac{a+c}{2} \in \mathbb{N}$. \\
        Do đó $a$ và $c$ cùng chẵn hoặc cùng lẻ và hơn kém nhau ít nhất 2 đơn vị.\\
        Số cách chọn bộ $\left(a; b ; c\right)$ theo thứ tự đó lập thành cấp số cộng bằng số cặp $\left(a; c\right)$ cùng chẵn hoặc cùng lẻ, số cách chọn là $2$ . $\mathrm{C}_{50}^{2}$.\\ 
        Vậy xác suất cần tính là $P=\dfrac{2\cdot\mathrm{C}_{50}^{2}}{\mathrm{C}_{100}^{3}}=\dfrac{1}{66}$.
    }
\end{ex}
\begin{ex}%[1D2G5-5]%738
    Gọi $E$ là tập hợp các số tự nhiên có $3$ chữ số đôi một khác nhau lập từ các chữ số $1;2;3;4;7$. Chọn ngẫu nhiên một phần tử của $E$, xác suất được chọn chia hết cho $3$ bằng
    \choice
    {$\dfrac{3}{7}$}
    {$\dfrac{1}{4}$}
    {$\dfrac{2}{5}$}
    {$\dfrac{3}{5}$}
    \loigiai{
        Số phần tử của tập $E$: $n(E)=\mathrm{A}_{5}^{3}\Rightarrow |\Omega|=\mathrm{A}_{5}^{3}$.\\
        Từ 5 số đã cho ta lập được 4 bộ 3 số có tổng chia hết cho 3 là $\left\{
        1;2;3\right\}$, $\left\{
        1;4;7\right\}$, $\left\{
        2;3;4\right\}$, $\left\{
        2;3;7\right\}$.\\
        Mỗi bộ 3 số này ta lập được $n!=6$ phần tử thuộc $E$, do đó trong tập $E$ có $4\cdot6=24$ số chia hết cho 3.\\
        Gọi $A$ là biến cố \lq\lq  số được chọn từ $E$ chia hết cho 3\lq\lq . Ta có $\left|\Omega_A\right|=24$.\\
        Vậy xác suất cần tính là $\mathrm{P}(A)=\dfrac{24}{\mathrm{A}_{5}^{3}}=\dfrac{2}{5}$.
    }
\end{ex}
\begin{ex}%[1D2B5-2]%739
    Một hộp đựng $11$ tấm thẻ được đánh số từ $1$ đến $11$. Chọn ngẫu nhiên $4$ tấm thẻ từ hộp đó. Gọi $P$ là xác suất để tổng các số ghi trên $4$ tấm thẻ ấy là một số lẻ. Khi đó $P$ bằng
    \choice
    {$\dfrac{1}{12}$}
    {\True $\dfrac{16}{33}$}
    {$\dfrac{10}{33}$}
    {$\dfrac{2}{11}$}
    \loigiai{
        Không gian mẫu $ n(\Omega)=\mathrm{C}_{11}^4=330 $.\\
        Số các viên bi đánh số lẻ là $ 6 $, số các viên bi đánh số chẵn là $ 5 $.
        Gọi $ A $   là biến cố lấy ra  $ 4 $ viên bi có tổng là số lẻ.
        \begin{enumerate}[Trường hợp 1:]
            \item $ 1 $ bi số lẻ, $ 3 $ số chẵn có $ \mathrm{C}_{6}^1\cdot \mathrm{C}_{5}^3 =60$ cách.
            \item $ 3 $ bi số lẻ, $ 1 $ số chẵn có $ \mathrm{C}_{6}^3\cdot \mathrm{C}_{5}^1 =100$ cách.
        \end{enumerate}
        Ta có $ n(A)=60+100=160$ cách.\\
        Xác suất cần tìm $ P(A)=\dfrac{n(A)}{n(\Omega)}=\dfrac{160}{330}=\dfrac{16}{33}$.   
    }
\end{ex}
\begin{ex}%[1D2K5-2]%740
    Cho tập hợp $S=\{1;2;3;\cdots;17\}$ gồm $17$ số nguyên dương đầu tiên. Chọn ngẫu nhiên $3$ phần tử của tập $S$. Tính xác suất để tập hợp con chọn được có tổng các phần tử chia hết cho $3$.
    \choice
    {$\dfrac{27}{34}$}
    {\True $\dfrac{23}{68}$}
    {$\dfrac{9}{34}$}
    {$\dfrac{9}{17}$}
    \loigiai{
        Không gian mẫu $ n(\Omega)=\mathrm{C}_{17}^3=680$.\\
        Trong $ S $ có $ 5 $ số chia hết cho $ 3 $ là $ \{3;6;9;12;15\} $, có $ 6 $ số chia cho $ 3 $ dư $ 1 $ là $ \{1;4;7;10;13;16\} $, có $ 6 $ số chia $ 3 $ dư $ 2 $ là $ \{2;5;8;11;14;17\} $.\\
        Gọi số cần tìm là $ \overline{abc} $, ta có $ a+b+c $ chia hết cho $ 3 $.\\
        Gọi $ A $   là biến cố chọn được $ 3 $ phần tử của $ S $ có tổng các phần tử chia hết cho $ 3 $ là
        \begin{enumerate}[Trường hợp 1:]
            \item Cả $ 3 $ số $ a $, $ b $, $ c $ đều chia hết cho $ 3 $ có  $\mathrm{C}_{5}^3=10$ số.
            \item  Cả $ 3 $ số $ a $, $ b $, $ c $ chia cho $ 3 $ dư $ 1 $ có  $\mathrm{C}_{6}^3=20$ số.
            \item  Cả $ 3 $ số $ a $, $ b $, $ c $ chia cho $ 3 $ dư $ 2 $ có  $\mathrm{C}_{6}^3=20$ số.
            \item  Trong  $ 3 $ số $ a $, $ b $, $ c $ có $ 1 $ số chia hết  cho $ 3 $, có $ 1 $ số chia $ 3 $ dư $ 1 $, có $ 1 $ số chia $ 3 $ dư $ 2 $ \\
            trường hợp này có  $5\cdot 6\cdot 6=180$ số.
        \end{enumerate}
        Ta có $ n(A)=10+20+20+180=230$.\\
        Xác suất cần tìm $ P(A)=\dfrac{n(A)}{n(\Omega)}=\dfrac{230}{680}=\dfrac{23}{68}$.
    }
\end{ex}

\begin{ex}%741
    Trong một hộp có $100$ tấm thẻ được đánh số từ $101$ đến $200$ (mỗi tấm thẻ được đánh một số khác nhau). Lấy ngẫu nhiên đồng thời $3$ tấm thẻ trong hộp. Xác suất để tổng các số ghi trên $3$ tấm thẻ đó là một số chia hết cho $3$ bằng
    \choice
    {$\dfrac{817}{2450}$}
    {$\dfrac{1181}{2450}$}
    {$\dfrac{37026}{161700}$}
    {$\dfrac{808}{2450}$}
    \loigiai{
    }
\end{ex}
\begin{ex}%[1D2G5-5]%742
    Cho đa giác đều $20$ đỉnh. Trong các tứ giác có bốn đỉnh là đỉnh của đa giác, chọn ngẫu nhiên một tứ giác. Xác suất để tứ giác được chọn là hình chữ nhật bằng
    \choice
    {$\dfrac{6}{323}$}
    {\True $\dfrac{3}{323}$}
    {$\dfrac{15}{323}$}
    {$\dfrac{14}{323}$}
    \loigiai{
        Số các chọn $4$ đỉnh của đa giác trong 20 đỉnh của đa giác là $n(\Omega)=\mathrm{C}_{20}^{4}=4845$ cách.\\
        Gọi biến cố A:\lq\lq  Chọn được 4 đỉnh của đa giác được chọn là một hình chữ nhật\rq\rq.\\
        Ta có 20 đỉnh của đa giác nên có thể tạo được 10 đường kính của đường tròn từ 20 đỉnh đó.\\
        Một hình chữ nhật có 4 đỉnh là đỉnh của đa giác được tạo bởi hai đường kính nói trên.\\
        $\Rightarrow$ Số cách chọn 4 đỉnh của đa giác tạo thành hình chữ nhật là $n(A)=\mathrm{C}_{10}^{2}=45$ cách.\\
        Xác suất cần tính là        $\mathrm{P}(A)=\dfrac{n(A)}{n(\Omega)}=\dfrac{45}{4845}=\dfrac{3}{323}$.
    }
\end{ex}
\begin{ex}%[1D2G5-5]%743
    Cho đa giác đều $36$ đỉnh. Chọn ngẫu nhiên $4$ đỉnh trong $36$ đỉnh của đa giác. Tính xác suất để $4$ đỉnh được chọn tạo thành một hình vuông.
    \choice
    {\True $\dfrac{1}{6545}$}
    {$\dfrac{2}{6545}$}
    {$\dfrac{1}{385}$}
    {$\dfrac{2}{385}$}
    \loigiai{
        Chọn ngẫu nhiên 4 đỉnh của hình $(H) \Rightarrow n(\Omega)=C_{36}^{4}=58905$.\\
        Giả sử $A_{1}, A_{2}, A_{3}, \dots, A_{36}$ là $36$ đỉnh của đa giác đều $(H)$. Gọi $O$ là tâm của đa giác đều $(H)$.\\
        $\Rightarrow A_{1} A_{2} \dots A_{36}$ là đa giác đều ngoại tiếp đường tròn $(O)$.\\
        Khi đó ta có $A_{i} O A_{i+1}=\dfrac{360^{\circ}}{36}=10^{\circ}, \forall i=\overline{1 ; 36}$.\\
        Để $A_{x} A_{y} A_{z} A_{t}$ là hình vuông thì $\widehat{A_{x} O A_{y}}=\widehat{A_{y} O A}_{z}=\widehat{A_{z} O A}_{t}=\widehat{A_{t} O A_{x}}=90^{\circ}$.\\
        Ta có $\widehat{A_{1} O A}_{10}=\widehat{A_{10} O A}_{19}=\widehat{A_{19} O A}_{28}=\widehat{A_{28} O A_{1}}=90^{\circ} \Rightarrow A_{1} A_{10} A_{19} A_{28}$ là 1 hình vuông.\\
        Cứ như vậy ta có các hình vuông là $A_{2} A_{11} A_{20} A_{29}, A_{3} A_{12} A_{21} A_{30}, \ldots, A_{9} A_{18} A_{27} A_{36}$.\\
        Gọi $A$ là biến cố: \lq\lq  4 đỉnh được chọn tạo thành hình vuông\rq\rq $\Rightarrow n(A)=9$.\\
        Xác suất cần tính là $\mathrm{\mathrm{P}(A)}=\dfrac{9}{58905}=\dfrac{1}{6545}$.
    }
\end{ex}
\begin{ex}%[1D2G5-5]%744
    Chọn nngẫu nhiên ba đỉnh bất kỳ từ các đỉnh của đa giác đều có $12$ cạnh $A_1A_2\cdots A_{12}$. Tính xác suất để $3$ đỉnh được chọn tạo thành một tam giác cân.
    \choice
    {\True $\dfrac{13}{55}$}
    {$\dfrac{12}{55}$}
    {$\dfrac{3}{11}$}
    {$\dfrac{5}{11}$}
    \loigiai{
        Chọn ngẫu nhiên $3$ trong số $12$ đỉnh của đa giác ta được $1$ tam giác nên $n(\Omega)=\mathrm{C}_{12}^{3}=220$.\\
        Vì đa giác đã cho là đa giác đều có $12$ đỉnh nên từ mỗi đỉnh có thể tìm ra $5$ cặp điểm để cùng với nó tạo ra $1$ tam giác cân, trong đó có $1$ tam giác đều. \\
        Từ $12$ đỉnh của đa giác đều có thể tạo ra $4$ tam giác đều. \\
        Vậy số tam giác cân và đều mà $12$ đỉnh của đa giác đều đó tạo ra là $12.4+4=52$.\\
        Xác suất cần tìm là $\mathrm{\mathrm{P}(A)}=\dfrac{52}{220}=\dfrac{13}{55}$.
    }
\end{ex}
\begin{ex}%[1D2G5-5]%745
    Gọi $X$ là tập hợp các số tự nhiên có tám chữ số đôi một khác nhau. Chọn ngẫu nhiên một số tự nhiên thuộc vào tập $X$. Tính xác suất để chọn được một số thuộc tập $X$ và số đó chia hết cho $9$ bằng
    \choice
    {\True $\dfrac{1}{9}$}
    {$\dfrac{1}{10}$}
    {$\dfrac{1}{8}$}
    {$\dfrac{1}{11}$}
    \loigiai{
        Số phần tử của không gian mẫu $n\left(\Omega\right)=\mathrm{A}_{10}^{8}-\mathrm{A}_{9}^{7}$\\
        Do đó số gồm 8 chữ số phân biệt chia hết cho 9 thì số đó phải không chữ 2 trong 10 chữ số $\{0 ; 1 ; 2 ; 3 ; 4 ; 5 ; 6 ; 7 ; 8 ; 9\}$ và có tổng chia hết cho 9 . Ta có 5 cặp số thỏa mãn: $\{0 ; 9\} ;\{1 ; 8\} ;\{2 ; 7\} ;\{3 ; 6\} ;\{4 ; 5\}$.\\
        Gọi số có 8 chữ số là $\overline{a_{1} a_{2} a_{3} a_{4} a_{5} a_{6} a_{7} a_{8}}$
        \begin{enumEX}[$\bullet$]{1}
            \item Trường hợp 1: Số được lập không chứa cặp số $\{0 ; 9\}$. Khi đó có $8 !$ Số thỏa mãn. 
            \item Trường hợp 2: Số được lập không chứa một trong 4 cặp số $\{1 ; 8\} ;\{2 ; 7\} ;\{3 ; 6\} ;\{4 ; 5\}$. Với mỗi số không chứa 1 trong 4 cặp trên, ta có 7.7! Số được tạo ra thỏa mãn bài toán.
            Do đó số các số gồm 8 chữ số phân biệt không chứa một trong 4 cặp số trên là $7\cdot7! \cdot 4 $.
        \end{enumEX}
        Vậy số các số gồm 8 chữ số phân biệt chia hết cho 9 là $8!+7\cdot7! \cdot 4 $ số.\\
        Xác suất cần tính là $\mathrm{P}=\dfrac{8!+7\cdot7! \cdot4}{\mathrm{A}_{10}^{8}-\mathrm{A}_{9}^{7}}=\dfrac{1}{9}$.
    }
\end{ex}
\Closesolutionfile{ans}
\indapan{10}{ans/CD4-Tong-On}
