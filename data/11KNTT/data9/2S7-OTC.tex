\section*{ÔN TẬP CHƯƠNG VII}
\setcounter{subsection}{0}
\subsection{Bài tập trắc nghiệm}
\Opensolutionfile{ans}[ans/ansTL-11K9-OTC]
%%==========Câu 1
\begin{ex}%[1K9KV-1]
Quy tắc tính đạo hàm nào sau đây là đúng?
\choice 
{$(u+v)'=u'-v'$}
{\True $(uv)'=u'v+uv'$}
{ $\left(\dfrac{1}{v}\right)'=-\dfrac{1}{v^2}$}
{$\left(\dfrac{u}{v}\right)'=\dfrac{u'v+uv'}{v^2}$}
\loigiai{
	\begin{itemize}
	\item $\left(\dfrac{1}{v}\right)'=-\dfrac{v'}{v^2}$ nên $\left(\dfrac{1}{v}\right)'=-\dfrac{1}{v^2}$ sai.
	\item $(u+v)'=u'+v'$ nên $(u+v)'=u'-v'$ sai.
	\item $\left(\dfrac{u}{v}\right)'=\dfrac{u'v-uv'}{v^2}$ nên $\left(\dfrac{u}{v}\right)'=\dfrac{u'v+uv'}{v^2}$ sai.
	\end{itemize}
} 
\end{ex}
%%==========Câu 2
\begin{ex}%[1K9BV-1]
Cho hàm số $f(x)=x^2+\sin^3 x$. Khi đó $f'\left(\dfrac{\pi}{2}\right)$ bằng
\choice 
{\True $\pi$}
{ $2\pi$}
{$\pi +3$}
{$\pi -3$}
\loigiai{ 
$f'(x)=2x+3\sin^2 x\cdot\cos x\Rightarrow f'\left(\dfrac{\pi}{2}\right)=2\cdot \dfrac{\pi}{2}+3\sin^2 \left(\dfrac{\pi}{2}\right) \cdot \cos \left(\dfrac{\pi}{2}\right)=\pi.$
}
\end{ex}
%%==========Câu 3
\begin{ex}%[1K9KV-1]
Cho hàm số $f(x)=\dfrac{1}{3}x^3-x^2-3x+1$. Tập nghiệm của bất phương trình $f'(x)\leq 0$ là
\choice
{$[1;3]$}
{\True $[-1;3]$}
{$[-3;1]$}
{$[-3;-1]$}
\loigiai{
	$f'(x)=x^2-2x-3$.\\
	$f'(x)\leq 0\Leftrightarrow x^2-2x-3\leq 0\Leftrightarrow -1\leq x\leq 3$.
} 
\end{ex}
%%==========Câu 4
\begin{ex}%[1K9KV-1]
Cho hàm số $f(x)=\sqrt{4+3u(x)}$ với $u(1)=7$, $u'(1)=10$. Khi đó $f'(1)$ bằng
\choice 
{$1$}
{$6$}
{\True $3$}
{$-3$}
\loigiai{
$f'(x)=\dfrac{(4+3u(x))'}{2\sqrt{4+3u(x)}}=\dfrac{3u'(x)}{2\sqrt{4+3u(x)}}$\\
$\Rightarrow f'(1)=\dfrac{3u'(1)}{2\sqrt{4+3u(1)}}=\dfrac{3\cdot 10}{2\sqrt{4+3\cdot 7}}=3$.
}
\end{ex}
%%==========Câu 5
\begin{ex}%[1K9BV-1]
Cho hàm số $f(x)=x^2\mathrm{\, e}^{-2x}$. Tập nghiệm của phương trình $f'(x)=0$ là
	\choice 
	{\True $\{0;1\}$}
	{$\{-1;0\}$}
	{ $\{0\}$}
	{$\{1\}$}
\loigiai{
	$f(x)=x^2\mathrm{\, e}^{-2x}$.\\
	$f'(x)=2x\cdot\mathrm{\, e}^{-2x}+(-2x)'\cdot x^2 \cdot \mathrm{\, e}^{-2x}=2x\cdot \mathrm{\, e}^{-2x}-2x^2\cdot \mathrm{\, e}^{-2x}=\left(2x-2x^2\right) \mathrm{\, e}^{-2x}.$\\
	$f'(x)=0\Leftrightarrow 2x-2x^2=0\Leftrightarrow 2x(1-x)=0\Leftrightarrow \hoac{&x=0\\&x=1.}$
	}
\end{ex}
%%==========Câu 6
\begin{ex}%[1K9BV-6]
	Chuyển động của một vật có phương trình $s(t)=\sin \left(0{,}8 \pi t+\dfrac{\pi}{3}\right)$ với $s$ tính bằng centimet và thời gian $t$ tình bằng giây. Tại các thời điểm vận tốc bằng $0$, giá trị tuyệt đối của gia tốc của vật gần với giá trị nào sau đây nhất?
	\choice
	{$4{,}5 \, \text{(cm/s}^2)$}
	{$5{,}5 \, \text{(cm/s}^2)$}
	{$6{,}3 \, \text{(cm/s}^2)$}
	{$7{,}1 \, \text{(cm/s}^2)$}
	\loigiai{
	$s(t)=\sin \left(0{,}8 \pi t+\dfrac{\pi}{3}\right)$\\
	 $v(t)=s'(t)=0{,}8\pi \cos \left(0{,}8 \pi t+\dfrac{\pi}{3}\right)$	.\\
	 $a(t)=s''(t)=-\left(0{,}8\pi\right)^2 \sin \left(0{,}8 \pi t+\dfrac{\pi}{3}\right)$	.\\
	 $v(t)=0\Leftrightarrow t=\dfrac{5}{24}+\dfrac{5}{4} k, \; (k \in \mathbb{Z})$.\\
	 Khi đó $a\left(\dfrac{5}{24}+\dfrac{5}{4} k\right)=-\left(0{,}8\pi \right)^2\sin \left[0{,}8 \pi \cdot \left(\dfrac{5}{24}+\dfrac{5}{4} k\right)+\dfrac{\pi}{3}\right]=-\left(0{,}8\pi \right)^2\sin \left( \dfrac{\pi}{2}+ k\pi \right) $.\\
	 $\Rightarrow \bigg|a\left(\dfrac{5}{24}+\dfrac{5}{4} k\right)\bigg|\approx 6{,}3\, \text{(cm/s}^2)$.
	} 
\end{ex}
%%==========Câu 7
\begin{ex}%[1K9KV-3]
	Cho hàm số $y=x^3-3x^2+4x-1$ có đồ thị là $(C).$ Hệ số góc nhỏ nhất của tiếp tuyến tại một điểm $M$ trên đồ thị $(C) $ là
	\choice 
	{\True $1$}
	{$2$}
	{ $-1$}
	{$3$}
	\loigiai{
	$y'=3x^2-6x+4=3\left(x^2-2x+1\right)+1=3\left(x-1\right)^2+1\geq 1$.\\
	Hệ số góc nhỏ nhất của tiếp tuyến bằng $1$.	
	}
\end{ex}
%%==========Câu 8
\begin{ex}%[1T7Y1-3]
	Cho hàm số $y=x^3-3 x^2$. Tiếp tuyến với đồ thị của hàm số tại điểm $M(-1 ;-4)$ có hệ số góc bằng
	\choice
	{ $-3$}
	{\True $9$ }
	{ $-9$}
	{$72$}
	\loigiai{
	Hệ số góc của tiếp tuyến với đồ thị hàm số $y=x^3-3 x^2$ tại điểm $M(-1 ;-4)$ là \\
	Ta có $f^\prime(x)=3x^2-6x$.\\
	Nên $k=y^\prime(-1)=9$.}
\end{ex}
%%==========Câu 9
\begin{ex}%[1T7Y1-1]
	Hàm số $y=-x^2+x+7$ có đạo hàm tại $x=1$ bằng
	\choice
	{\True $-1$}
	{$7$} 
	{$1$}
	{$6$}
	\loigiai{Ta có $y^\prime=-2x+1$\\
	Đạo hàm của hàm số tại $x=1$ là $y^\prime(1)=-1$.}
\end{ex}
%%==========Câu 10
\begin{ex}%[1T7K1-1]
	Cho hai hàm số $f(x)=2 x^3-x^2+3$ và $g(x)=x^3+\dfrac{x^2}{2}-5$.
	Bất phương trình $f'(x)>g'(x)$ có tập nghiệm là
	\choice
	{$(-\infty ; 0] \cup[1 ;+\infty)$}
	{$(0 ; 1)$}
	{$[0 ; 1]$}
	{\True $(-\infty ; 0) \cup(1 ;+\infty)$}
	\loigiai{
	Ta có $f^\prime(x)=6x^2-2x$ và $g^\prime(x)=3x^2+x$.\\
	Bất phương trình $f'(x)>g'(x)$
	$$\Leftrightarrow 6x^2-2x>3x^2+x\Leftrightarrow 3x^2-3x>0 \Leftrightarrow \hoac{&x>1\\&x<0.}$$
	Vậy $S=(-\infty ; 0) \cup(1 ;+\infty)$. }
\end{ex}
%%==========Câu 11
\begin{ex}%[1T7B1-1]
	Hàm số $y=\dfrac{x+3}{x+2}$ có đạo hàm là
	\choice
	{$y'=\dfrac{1}{(x+2)^2}$}
	{$y'=\dfrac{5}{(x+2)^2}$}
	{\True $y'=\dfrac{-1}{(x+2)^2}$}
	{$y'=\dfrac{-5}{(x+2)^2}$}
	\loigiai{Đạo hàm hàm số $y=\dfrac{x+3}{x+2}$.\\
	Ta có $y^\prime=\dfrac{(x+3)^\prime \cdot (x+2)-(x+2)^\prime \cdot (x+3)}{(x+2)^2}=\dfrac{-1}{(x+2)^2}$.}
\end{ex}
%%==========Câu 12
\begin{ex}%[1T7B1-1]
	Hàm số $y=\dfrac{1}{x+1}$ có đạo hàm cấp hai tại $x=1$ là
	\choice
	{$y^{\prime \prime}(1)=\dfrac{1}{2}$}
	{$y^{\prime \prime}(1)=-\dfrac{1}{4}$}
	{$y^{\prime \prime}(1)=4$}
	{\True $y^{\prime \prime}(1)=\dfrac{1}{4}$}
	\loigiai{Ta có $y^\prime=\dfrac{-1}{(x+1)^2}$ và $y^{\prime\prime}=\dfrac{2}{(x+1)^3}$.\\
	Vậy $y^{\prime\prime}(1)=\dfrac{1}{4}$.}
\end{ex}
\Closesolutionfile{ans}
%%%%%%%%%%%%%%%%
\subsection{Bài tập tự luận}
%%==========Bài 1
\begin{bt}%[1K9BV-1]
	Tính đạo hàm của các hàm số sau
	\begin{listEX}[4]
	\item $y=\left(\dfrac{2x-1}{x+2}\right)^5$;
	\item $y=\dfrac{2x}{x^2+1}$;
	\item $y=e^x\cdot \sin^2x$;
	\item $y=\log \left(x+\sqrt{x}\right)$
	\end{listEX} 	
	\loigiai{
	\begin{itemize}
	\item \begin{eqnarray*}
	y'&=&5\cdot \left(\dfrac{2x-1}{x+2}\right)^4 \cdot \left(\dfrac{2x-1}{x+2}\right)'\\
	&=&5\cdot \left(\dfrac{2x-1}{x+2}\right)^4 \cdot \dfrac{2(x+2)-(2x-1)}{(x+2)^2}\\
	&=&5\cdot \left(\dfrac{2x-1}{x+2}\right)^4 \cdot \dfrac{5}{(x+2)^2}\\
	&=& 25\cdot \left(\dfrac{2x-1}{x+2}\right)^4 \cdot \dfrac{1}{(x+2)^2};
	\end{eqnarray*}
	\item $y'=\dfrac{2(x^2+1)-2x\cdot 2x}{(x^2+1)^2}=\dfrac{2-2x^2}{(x^2+1)^2}$;
	\item $y'=\mathrm{\, e}^x\cdot \sin^2x+2\cdot \mathrm{\, e}^x\cdot \sin x\cdot \cos x=\mathrm{\, e}^x\cdot \left(\sin ^2 x+\sin 2x\right)$;
	\item $y'=\dfrac{\left(x+\sqrt{x}\right)'}{ \left(x+\sqrt{x}\right)\ln 10}=\dfrac{1+\dfrac{1}{2\sqrt{x}}}{ \left(x+\sqrt{x}\right)\ln 10}=\dfrac{2\sqrt{x}+1}{2\sqrt{x}\left(x+\sqrt{x}\right)\ln 10}$.
	\end{itemize} 
	} 
\end{bt}
%%==========Bài 2
\begin{bt}%[1T7B1-1]
	Tính đạo hàm của các hàm số sau:
	\begin{listEX}[3]
		\item $y=3 x^4-7 x^3+3 x^2+1$
		\item $y=\left(x^2-x\right)^3$
		\item $y=\dfrac{4 x-1}{2 x+1}$.
	\end{listEX}
	\loigiai{Đạo hàm của các hàm số sau là
		\begin{enumerate}
			\item $y^\prime=12x^3-21x^2+6x$.
			\item $y^\prime=3(x^2-x)^2(x^2-x)^\prime=3(x^2-x)^2(2x-1)$.
			\item $y^\prime=\dfrac{(4x-1)^\prime\cdot(2x+1)-(2x+1)^\prime\cdot(4x-1)}{(2x+1)^2}=\dfrac{6}{(2x+1)^2}$.
	\end{enumerate}}
\end{bt}
%%==========Bài 3
\begin{bt}%[1T7B1-1]
	Tính đạo hàm của các hàm số sau:
	\begin{listEX}[2]
		\item $y=\left(x^2+3 x-1\right) e^x$;
		\item $y=x^3 \log _2 x$.
	\end{listEX}
	\loigiai{Đạo hàm của các hàm số sau là
		\begin{enumerate}
			\item $y^\prime=(2x+3)e^x+(x^2+3x-1)e^x=e^x(x^2+5x-1)$.
			\item $y^\prime=(x^3)^\prime\ \cdot\log_2 x+x^3\cdot (\log_2 x)^\prime=3x^2\log_2 x+\dfrac{x^3}{x\ln 2}$.
	\end{enumerate}}
\end{bt}
%%==========Bài 4
\begin{bt}%[1T7B1-1]
	Tính đạo hàm của các hàm số sau:
	\begin{listEX}[3]
		\item $y=\tan \left(e^x+1\right)$;
		\item $y=\sqrt{\sin 3 x}$;
		\item $y=\cot \left(1-2^x\right)$.
	\end{listEX}
	\loigiai{Đạo hàm của các hàm số sau là
		\begin{enumerate}
			\item $y^\prime=\dfrac{(e^x+1)^\prime}{\cos^2(e^x+1)}=\dfrac{e^x}{\cos^2(e^x+1}$.
			\item $y^\prime=\dfrac{(\sin3x)^\prime}{2\sqrt{\sin3x}}=\dfrac{3\cos3x}{2\sqrt{\sin3x}}$.
			\item $y^\prime=\dfrac{(1-2^x)^\prime}{\sin^2(1-2^x)}=\dfrac{2^x\ln x}{\sin^2(1-2^x)}$.
	\end{enumerate}}
\end{bt}
%%==========Bài 5
\begin{bt}%[1T7B1-1]
	Tính đạo hàm cấp hai của các hàm số sau:
	\begin{listEX}[2]
		\item $y=x^3-4 x^2+2 x-3$;
		\item $y=x^2 e^x$.
	\end{listEX}
	\loigiai{Đạo hàm cấp $1$ của các hàm số sau là
		\begin{enumerate}
			\item $y^\prime=3x^2-8x+2$.
			\item $y^\prime=2xe^x+x^2e^x=e^x(2x+x^2)$.
		\end{enumerate}
		Đạo hàm cấp $2$ của các hàm số trên là
		\begin{enumerate}
			\item $y^{\prime\prime}=6x-8$.
			\item $y^{\prime\prime}=e^x(2x+x^2)+e^x(2+2x)=e^x(x^2+4x+2)$.
	\end{enumerate}}
\end{bt}
%%==========Bài 6
\begin{bt}%[1K9KV-1]
	Xét hàm số luỹ thừa $y=x^{\alpha}$ với $\alpha$ là số thực.
\begin{enumerate}
	\item Tìm tập xác định của hàm số đã cho.
	\item Bằng cách viết $y=x^{\alpha}=\mathrm{\, e}^{\alpha \cdot \ln x}$, tính đạo hàm của hàm số đã cho.
\end{enumerate}	
	\loigiai{
	\begin{enumerate}
	\item 
	\begin{itemize}
	\item Với $\alpha $ nguyên dương. Khi đó tập xác định $\mathscr{D}=\mathbb{R}$.
	\item Với $\alpha $ nguyên âm hoặc $\alpha = 0$. Khi đó tập xác định $\mathscr{D}=\mathbb{R}\setminus\{0\}$.
	\item Với $\alpha$ không nguyên $\mathscr{D}=(0;+\infty)$.
	\end{itemize}
	\item Bằng cách viết $y=x^{\alpha}=\mathrm{\, e}^{\alpha \cdot \ln x}$, tính đạo hàm của hàm số đã cho.\\
	$y'=\left(\alpha \cdot \ln x\right)'\cdot \mathrm{\, e}^{\alpha \cdot \ln x}=\alpha \cdot\dfrac{1}{x}\cdot \mathrm{\, e}^{\alpha \cdot \ln x}=\dfrac{\alpha }{x}\cdot \mathrm{\, e}^{\alpha \cdot \ln x}=\dfrac{\alpha }{x}\cdot x^{\alpha}$.
	\end{enumerate}	
	} 
\end{bt}
%%==========Bài 7
\begin{bt}%[1K9BV-1]
	Cho hàm số $f(x)=\sqrt{3x+1}$. Đặt $g(x)=f(1)+4(x^2-1)f'(1)$. Tính $g(2)$.
\loigiai{
$f'(x)=\dfrac{(3x+1)'}{2\sqrt{3x+1}}=\dfrac{3}{2\sqrt{3x+1}}$.\\
$f(1)=\sqrt{3\cdot 1+1}=2$.\\
$f'(1)=\dfrac{3}{2\sqrt{3\cdot 1+1}}=\dfrac{3}{4}$.\\
$g(x)=2+4(x^2-1)\cdot \left(\dfrac{3}{4}\right)$.\\
Do đó $g(2)=2+4(2^2-1)\cdot \left(\dfrac{3}{4}\right)=11$.
} 
\end{bt}
%%==========Bài 8
\begin{bt}%[1K9YW-1]
	Cho hàm số $f(x)=\dfrac{x+1}{x-1}$. Tính $f''(1).$
	\loigiai{
	$f'(x)=\dfrac{x-1-(x+1)}{(x-1)^2}=\dfrac{-2}{(x-1)^2}$.\\
	$f''(x)=-\dfrac{-2\cdot 2(x-1)}{(x-1)^4}=\dfrac{4}{(x-1)^3}$.\\
	Vì hàm số không xác định tại $x=1$ nên không tồn tại $f''(1)$.
	}
\end{bt}
%%==========Bài 9
\begin{bt}%[1K9YW-1]
	Cho hàm số $f(x)$ thoả mãn $f(1)=2$ và $f'(x)=x^2f(x)$ với mọi $x$. Tính $f''(1)$.
	\loigiai{
	\begin{eqnarray*}f'(x)=x^2f(x)&\Rightarrow & f''(x)=2x\cdot f(x)+x^2\cdot f'(x)=2x\cdot f(x)+x^2\cdot x^2\cdot f(x)\\
	&\Rightarrow & f''(1)=2\cdot 1\cdot f(1)+ f'(1)=2\cdot 2+1^4\cdot 2=6.
	\end{eqnarray*}
	}
\end{bt}
%%==========Bài 10
\begin{bt}%[1K9BV-2]
	Viết phương trình tiếp tuyến của đồ thị hàm số $y=x^3+3x^2-1$ tại điểm có hoành độ bằng $1$.
	\loigiai{
	$y'=3x^2+6x$.\\
	$y'(1)=9$.\\
	$y(1)=1^3+3\cdot 1^2-1=3$.\\
	Tiếp tuyến của đồ thị hàm số $y=x^3+3x^2-1$ tại điểm có hoành độ bằng $1$ là $$y-3=9\left(x-1\right)\Leftrightarrow y=9x-6$$.
	}
\end{bt}
%%==========Bài 11
\begin{bt}%[1K9KV-2]
	Đồ thị hàm số $y=\dfrac{a}{x}$ ($a$ là hằng số dương) là một đường hyperbol. Chứng minh rằng tiếp tuyến tại một điểm bất kỳ của đường hyperbol đó tạo với các trục toạ độ một tam giác có diện tích không đổi.
	\loigiai{
	$y=\dfrac{a}{x}\Rightarrow y'=-\dfrac{a}{x^2}$.\\
Phương trình tiếp tuyến tại $M(x_0;y_0)$ là
$y=-\dfrac{a}{x_0^2}(x-x_0)+\dfrac{a}{x_0}\Leftrightarrow y=-\dfrac{ax}{x_0^2}+\dfrac{2a}{x_0}$.\\
Suy ra diện tích tam giác $OAB$ là 
$S=\dfrac{1}{2}\cdot\bigg | \dfrac{2a}{x_0}\bigg|\cdot \mid 2x_0\mid=2a^2$ (không đổi).}
\end{bt}
%%==========Bài 12
\begin{bt} 
	\immini{Hình bên dưới biểu diễn đồ thị của ba hàm số. Hàm thứ nhất là hàm vị trí của một chiếc xe ô tô, hàm thứ hai là hàm hàm biểu thị vận tốc và hàm thứ ba làm hàm biểu thị gia tốc của ô tô đó. Hãy xác định đồ thị của mỗi hàm số này và giải thích?}
	{\begin{tikzpicture}[>=stealth]
	\draw[->] (-0.5,0)--(9,0) node[right]{$t$};
	\draw[->] (0,-3.2)--(0,3.5) node[left]{$y$};
	\draw (0,0) node[below left]{$O$};
	\draw (0,0)..controls (3,0.4) and (6,1)..(9,1);
	\draw (0,0)..controls (1,0.2) and (2,1.5)..(4,1.3) ..controls (6,0.9) and (7,0.2)..(9,0.1); 
	\draw (0,0.3)..controls (0.4,0.5) and (1.5,1.9)..(2,2) ..controls (2.5,1.7) and (3,-2.7)..(4,-3)..controls (5,-2.8) and (7,-0.4)..(9,-0.1); 
	\draw (8,1.2) node[right]{$c$};
	\draw (4,1.6) node[right]{$b$};
	\draw (2,2.1) node[right]{$a$};
	\end{tikzpicture}}
	\loigiai{
	\begin{itemize}
	\item Với hàm số a, ta có $a$ có thể nhận giá trị âm hoặc dương (thể hiện tăng tốc hay giảm tốc ) tuỳ thuộc vào thời gian $t$. Do đó $a$ là hàm số biểu thị gia tốc của xe ô tô.
	\item 	Với hàm $b$, ta quan sát thấy $b$ luôn nhận giá trị dương và có thời điểm $b$ tăng có thời điểm $b$ giảm do đó $b$ là hàm số biểu thị vận tốc.
	\item Với hàm số $c$, ta quan sát thấy đồ thị hàm số luôn nhận giá trị dương và luôn đi lên theo thời gian $t$, do đó $c$ là hàm số biểu thị quãng đường.
	\end{itemize}
}
\end{bt}
%%==========Bài 13
\begin{bt}%[1K9KW-3]
	Vị trí của một vật chuyển động thẳng được cho bởi phương trình $s=f(t)=t^3-6t^2+9t$, trong đó $t$ tính bằng giây và $s$ tính bằng mét.
	\begin{enumerate}
	\item Tính vận tốc của vật tại thời điểm $t=2$ giây và $t=4$ giây.
	\item Tại thời điểm nào vật đứng yên?
	\item Tìm gia tốc của vật tại thời điểm $t=4$ giây.
	\item Tính tổng quãng đường vật đi được trong $5$ giây đầu tiên.
	\item[e)] Trong $5$ giây đầu tiên, khi nào vật tăng tốc, khi nào vật giảm tốc?
	\end{enumerate}
	\loigiai{
	$s=f(t)=t^3-6t^2+9t\Rightarrow v(t)=f'(t)=3t^2-12t+9\Rightarrow a(t)=f''(t)=6t-12$.\\
	\begin{enumerate}
	\item Vận tốc của vật tại thời điểm $t=2$ giây là 
	$v(2)=f'(2)=3\cdot 2^2-12\cdot 2+9=-3 \, \text{(m/s})$.\\
	Vận tốc của vật tại thời điểm $t=4$ giây là 	$v(4)=f'(4)=3\cdot 4^2-12\cdot 4+9=9\, \text{(m/s})$.\\
	\item Vật đứng yên khi $v(t)=0\Leftrightarrow 3t^2-12t+9=0\Leftrightarrow \hoac{&t=1\\&t=2.}$
	\item Gia tốc của vật tại thời điểm $t=4$ giây là $a(4)=6\cdot 4-12=12 \, \text{(m/s}^2)$.
	\item Tổng quãng đường vật đi được trong $5$ giây đầu tiên là $s=f(5)=5^3-6\cdot 5^2+9\cdot 5=20 \text{(m)}$
	\item[e)] Trong $5$ giây đầu tiên, khi nào vật tăng tốc, khi nào vật giảm tốc?
	\begin{itemize}
	\item $t=0\Rightarrow a(0)=-12=-12$. Do đó vật giảm tốc.\\
	\item $t=1\Rightarrow a(1)=6-12=-6$. Do đó vật giảm tốc.\\
	\item $t=3\Rightarrow a(3)=18-12=6$. Do đó vật tăng tốc.\\
	\end{itemize}	
	\end{enumerate}
	}
\end{bt}
%%==========Bài 14
\begin{bt}%[1T7Y1-3]
	Cho hàm số $f(x)=x^2-2 x+3$ có đồ thị $(C)$ và điểm $M(-1 ; 6) \in(C)$. Viết phương trình tiếp tuyến với $(C)$ tại điểm $M$.
	\loigiai{Ta có $f^\prime(x)=2x-2$ nên $f^\prime(-1)=-4$.\\
	Phương trình tiếp tuyến của đồ thị $(C)$ tại điểm $M(-1 ; 6)$ là
	$$y=f^\prime(-1)(x+1)+6\Leftrightarrow y=-4x+2.$$}
\end{bt}


%%==========Bài 15
\begin{bt}%[1T7B2-7]
	Một viên sỏi rơi từ độ cao $44,1$ m thì quãng đường rơi được biểu diễn bởi công thức $s(t)=4,9t^2$, trong đó $t$ là thời gian tính bằng giây và $s$ tính bằng mét. Tính:
	\begin{enumerate}
	\item Vận tốc rơi của viên sỏi lúc $t=2$;
	\item Vận tốc của viên sỏi khi chạm đất.
	\end{enumerate}
	\loigiai{Ta có $s'(t)=4,9\cdot 2t$.
	\begin{enumerate}
	\item Vận tốc rơi của viên sỏi lúc $t=2$ là $s'(2)=4,9\cdot 2\cdot 2=19,6$ m/s.
	\item Vận tốc của viên sỏi khi chạm đất.\\
	Thời gian viên sỏi rơi từ độ cao $44,1$ m đến lúc chạm đất
	\[4,9t^2=44,1\ \text{suy ra}\ t=3 \ (\text{vì} \ t>0). \]
	Vận tốc của viên sỏi khi chạm đất là $s'(3)=4,9\cdot 2\cdot 3=29,4$ m/s.
	\end{enumerate}
	}
\end{bt}
%%==========Bài 16
\begin{bt}%[1T7B2-7]
	Một vật chuyển động trên một đường thẳng được xác định bởi công thức $s(t)=2t^3+4t+1$, trong đó $t$ là thời gian tính bằng giây và $s$ tính bằng mét. Tính vận tốc và gia tốc của vật khi $t=1$.
	\loigiai{ Ta có $s'(t)=6t^2+4$, $s''(t)=12t$.\\
	Vận tốc của vật khi $t=1$ là $s'(1)=6\cdot 1^2+4=10$ m/s.\\
	Gia tốc của vật khi $t=1$ là $s''(1)=12\cdot 1=12 ~ m/s^2$.
	}
\end{bt}
%%==========Bài 17
\begin{bt}%[1T7B2-7]
	Dân số $P$ (tính theo nghìn người) của một thành phố nhỏ được tính theo công thức $P(t)=\dfrac{500t}{t^2+9}$, trong đó $t$ là thời gian tính bằng năm. Tìm tốc độ tăng dân số tại thời điểm $t=12$.
	\loigiai{
	$P'(t)=\left(\dfrac{500t}{t^2+9}\right)'=\dfrac{(500t)'(t^2+9)-500t\left(t^2+9\right)'}{\left(t^2+9\right)^2}=\dfrac{500(t^2+9)-100t\cdot 2t}{\left(t^2+9\right)^2}=\dfrac{-500t^2+4500}{(t^2+9)^2}$.\\
	Tốc độ tăng dân số tại thời điểm $t=12$ là $P'(12)\simeq -2,88$ nghìn người/năm.
	}
\end{bt}
%%==========Bài 18
\begin{bt}%[1T7B2-7]
	Hàm số $S(r)=\dfrac{1}{r^4}$ có thể được sử dụng để xác định sức cản $S$ của dòng máu có bán kính $r$ (tính theo milimét) (theo Bách khoa toàn thư Y học "Harrison's internal medicine $21$st edition"). Tìm tốc độ thay đổi của $S$ theo $r$ khi $r=0,8$.
	\loigiai{
	Ta có $S'(r)=\left(\dfrac{1}{r^4}\right)'=\left(r^{-4}\right)'=-4r^{-5}=\dfrac{-4}{r^5}$.\\
	Tìm tốc độ thay đổi của $S$ theo $r$ khi $r=0,8$ là $S'(0,8)\simeq-12,207$.
	}
\end{bt}
%%==========Bài 19
\begin{bt}%[1T7B2-7]
	Nhiệt độ cơ thể của một người trong thời gian bi bệnh được cho bởi công thức $T(t)=-0,1t^2+1,2t+98,6$, trong đó $T$ là nhiệt độ (tính theo đơn vị đo nhiệt độ Fahrenheit) tại thời điểm $t$ (tính theo ngày). Tìm tốc độ thay đổi của nhiệt độ ở thời điểm $t=1,5$.
	\loigiai{ Ta có $T'(t)=-0,2t+1,2$.\\
	Tốc độ thay đổi của nhiệt độ ở thời điểm $t=1,5$ là $T'(1,5)=0,9$.
	}
\end{bt}
%%==========Bài 20
\begin{bt}%[1T7B2-7]
	Hàm số $R(v)=\dfrac{600}{v}$ có thể dùng để xác định nhịp tim $R$ của một người mà tim của người đó có thể lấy đi được $6000~ml$ máu trên mỗi phút và $v~ml$ máu trên mỗi nhịp đập (theo Bách khoa toàn thư Y học "Harrison's internal medicine $21$st edition"). Tìm tốc độ thay đổi của nhịp tim khi lượng máu tim đẩy đi ở một nhịp là $v=80$.
	\loigiai{Ta có $R'(v)=-\dfrac{600}{v^2}$.\\
	Tốc độ thay đổi của nhịp tim khi lượng máu tim đẩy đi ở một nhịp $v=80$ là $R'(80)=-0,09375$. 
	}
\end{bt}	