\setcounter{section}{30}
\section{ĐẠO HÀM}
\subsection{Tóm tắt kiến thức}
\begin{tomtat}
	\subsubsection{Đạo hàm của hàm số tại một điểm}
	\begin{dn}
	Cho hàm số $y=f(x)$ xác định trên khoảng $(a;b)$ và điểm $x_0 \in (a;b)$.
	Nếu tồn tại giới hạn hữu hạn 
	\[\lim\limits_{x \to x_0} \dfrac{f(x)-f(x_0)}{x-x_0}\]
	thì giới hạn đó được gọi là đạo hàm của hàm số $y=f(x)$ tại điểm $x_0$, kí hiệu bởi $f'(x_0)$ (hoặc $y'(x_0)$), tức là 
	\[f'(x_0)=\lim\limits_{x \to x_0} \dfrac{f(x)-f(x_0)}{x-x_0}.\]
	\end{dn}
	\begin{note}
	Để tính đạo hàm của hàm số $y=f(x)$ tại điểm $x_0\in (a;b)$, ta thực hiện theo các bước sau:
	\begin{itemize}
	\item Tính $f(x)-f(x_0)$.
	\item Lập và rút gọn tỉ số $\dfrac{f(x)-f(x_0)}{x-x_0}$ với $x \in (a;b),x \ne x_0$.
	\item Tìm giới hạn $\lim\limits_{x \to x_0} \dfrac{f(x)-f(x_0)}{x-x_0}$.
	\end{itemize}
	\end{note}
	\begin{note}
	Đặt $\Delta x=x-x_0$, khi đó đạo hàm của hàm số đã cho tại điểm $x_0$ có thể tính theo công thức:\\
	$$f'(x_0)=\lim\limits_{\Delta x \to 0} \dfrac{f(x_0+\Delta x)-f(x_0)}{\Delta x}.$$
	$\Delta x$ được gọi là số gia của biến số tại điểm $x_0$.
	\end{note}
	\subsubsection{Đạo hàm của hàm số trên một khoảng}
	\begin{dn}
	Hàm số $y=f(x)$ được gọi là có đạo hàm trên khoảng $(a;b)$ nếu nó có đạo hàm $f'(x)$ tại mọi điểm $x$ thuộc khoảng đó, kí hiệu là $y'=f'(x)$.
	\end{dn}	
	\begin{note}
	Nếu phương trình chuyển động của một vật là $s=f(t)$ thì $v(t)=f'(t)$ là vận tốc tức thời của vật tại thời điểm $t$.
	\end{note}
	\subsubsection{Ý nghĩa hình học của đạo hàm}
	\paragraph{Tiếp tuyến của đồ thị hàm số}
	\begin{dn}
	Tiếp tuyến của đồ thị hàm số $y=f(x)$ tại điểm $P(x_0;f(x_0))$ là đường thẳng đi qua $P$ với hệ số góc $k=\lim\limits_{x \to x_0} \dfrac{f(x)-f(x_0)}{x-x_0}$ nếu giới hạn này tồn tại và hữu hạn, nghĩa là $k=f'(x_0)$. Điểm $P$ gọi là tiếp điểm.
	\end{dn}
	\begin{nx}
	Hệ số góc của tiếp tuyến của đồ thị hàm số $y=f(x)$ tại điểm $P(x_0;f(x_0))$ là đạo hàm $f'(x_0)$.
	\end{nx}
	\paragraph{Phương trình tiếp tuyến}
	\begin{dn}
	Nếu hàm số $y=f(x)$ có đạo hàm tại điểm $x_0$ thì phương trình tiếp tuyến của đồ thị hàm số tại điểm $P(x_0;y_0)$ là 
	$$y-y_0=f'(x_0)(x-x_0),$$
	trong đó $y_0=f(x_0)$.
	\end{dn}	
\end{tomtat}
%%%%%%%%%%%%%%%%%%%
\subsection{Các dạng bài tập}
\begin{dang}{Tính đạo hàm của hàm số bằng định nghĩa}
	Ta sử dụng một trong hai cách tính sau:
	\[f'(x_0)=\lim\limits_{x \to x_0} \dfrac{f(x)-f(x_0)}{x-x_0}\quad \text{ hoặc }\quad f'(x_0)=\lim\limits_{\Delta x \to 0} \dfrac{f(x_0+\Delta x)-f(x_0)}{\Delta x}.\]
\end{dang}
\subsubsection{Ví dụ minh hoạ}
\begin{vd}%[1K9YU1]
	Tính (bằng định nghĩa) đạo hàm của các hàm số sau:
	\begin{listEX}[2]
	\item $y=2x-3$ tại điểm $x_0=1$;
	\item $y=x^2$ tại điểm $x_0=2$;
	\item $y=x^2+1$ tại điểm $x_0=-1$;
	\item $y=x^2+2x$ tại điểm $x_0=3$;
	\item $y=x^2-2x+1$ tại điểm $x_0=-2$;
	\item $y=2x^3$ tại điểm $x_0=1$;
	\item $y=x^3+1$ tại điểm $x_0=2$;
	\item $y=x^3-x$ tại điểm $x_0=3$.
	\end{listEX} 
	\loigiai{
	\begin{listEX}[2]
	\item 
	Ta có 
	$\begin{aligned}[t]
	f^{\prime}(1)
	&=\lim\limits_{x \to 1}\dfrac{f(x)-f(1)}{x-1}\\
	&=\lim\limits_{x \to 1}\dfrac{(2x-3)-(2\cdot 1-3)}{x-1}\\
	&=\lim\limits_{x\to 1}\dfrac{2(x-1)}{x-1}\\
	&=\lim\limits_{x\to 1}2=2
	\end{aligned}
	$.
	\item 
	Ta có 
	$\begin{aligned}[t]
	f^{\prime}(2)
	&=\lim\limits_{x \to 2}\dfrac{f(x)-f(2)}{x-2}\\
	&=\lim\limits_{x \to 2}\dfrac{x^2-2^2}{x-2}\\
	&=\lim\limits_{x \to 2}\dfrac{(x-2)(x+2)}{x-2}\\
	&=\lim\limits_{x \to 2}(x+2)=4
	\end{aligned}
	$.
	\item 
	Ta có 
	$\begin{aligned}[t]
	f^{\prime}(-1)
	&=\lim\limits_{x \to -1}\dfrac{f(x)-f(-1)}{x+1}\\
	&=\lim\limits_{x \to -1}\dfrac{(x^2+1)-\left[(-1)^2+1\right])}{x+1}\\
	&=\lim\limits_{x \to -1}\dfrac{x^2-1}{x+1}\\
	&=\lim\limits_{x \to -1}(x-1)=-2
	\end{aligned}
	$.
	\item 
	Ta có 
	$\begin{aligned}[t]
	f^{\prime}(3)
	&=\lim\limits_{x \to 3}\dfrac{f(x)-f(3)}{x-3}\\
	&=\lim\limits_{x \to 3}\dfrac{(x^2+2x)-(3^2+2\cdot 3)}{x-3}\\
	&=\lim\limits_{x \to 3}\dfrac{(x-3)(x+5)}{x-3}\\
	&=\lim\limits_{x \to 3}(x+5)=8
	\end{aligned}
	$.
	\item 
	Ta có 
	$\begin{aligned}[t]
	f^{\prime}(-2)
	&=\lim\limits_{x \to -2}\dfrac{f(x)-f(-2)}{x+2}\\
	&=\lim\limits_{x \to -2}\dfrac{(x^2-2x+1)-\left[ (-2)^2-2\cdot(-2)+1\right]}{x+2}\\
	&=\lim\limits_{x \to -2}\dfrac{(x+2)(x-4)}{x+2}\\
	&=\lim\limits_{x \to -2}(x-4)=-6
	\end{aligned}
	$.
	\item 
	Ta có 
	$\begin{aligned}[t]
	f^{\prime}(1)
	&=\lim\limits_{x \to 1}\dfrac{f(x)-f(1)}{x-1}\\
	&=\lim\limits_{x \to 1}\dfrac{2x^3-2\cdot 1^3}{x-1}\\
	&=\lim\limits_{x \to 1}\dfrac{2(x^3-1)}{x-1}\\
	&=\lim\limits_{x \to 1}2(x^2+x+1)=6
	\end{aligned}
	$.
	\item 
	Ta có 
	$\begin{aligned}[t]
	f^{\prime}(2)
	&=\lim\limits_{x \to 2}\dfrac{f(x)-f(2)}{x-2}\\
	&=\lim\limits_{x \to 2}\dfrac{(x^3+1)-(2^3+1)}{x-2}\\
	&=\lim\limits_{x \to 2}\dfrac{x^3-2^3}{x-2}\\
	&=\lim\limits_{x \to 2}(x^2+2x+4)=12
	\end{aligned}
	$.
	\item 
	Ta có 
	$\begin{aligned}[t]
	f^{\prime}(3)
	&=\lim\limits_{x \to 3}\dfrac{f(x)-f(3)}{x-3}\\
	&=\lim\limits_{x \to 3}\dfrac{(x^3-x)-(3^3-3)}{x-3}\\
	&=\lim\limits_{x \to 3}\dfrac{(x^3-3^3)-(x-3)}{x-3}\\
	&=\lim\limits_{x \to 3}(x^2+3x+8)=26
	\end{aligned}
	$.
	\end{listEX}
	}
\end{vd}
\begin{vd}%[1D5B1]
	Tính (bằng định nghĩa) đạo hàm của các hàm số sau:
	\begin{listEX}[2]
	\item $f(x)=\dfrac{2}{x}$ tại điểm $x_0=3$;
	\item $f(x)=\sqrt{x}$ tại điểm $x_0=1$;
	\item $f(x)=\dfrac{1}{x-1}$ tại điểm $x_0=2$;
	\item $f(x)=\sqrt{x-1}$ tại điểm $x_0=5$.
	\end{listEX}
	\loigiai{
	\begin{listEX}[2]
	\item Ta có:
	$\begin{aligned}[t]
	f^{\prime}\left(3\right)
	&=\lim\limits_{x \to 3} \dfrac{f(x)-f(3)}{x-3}\\
	&=\lim\limits_{x \to 3} \dfrac{\dfrac{2}{x}-\dfrac{2}{3}}{x-3}\\
	&=\lim\limits_{x \to 3} \dfrac{2(3-x)}{3x\left(x-3\right)}\\
	&=\lim\limits_{x \to 3} \dfrac{-2}{3x}\\
	&=-\dfrac{1}{9}
	\end{aligned}
	.$
	\item %b
	Ta có:
	$\begin{aligned}[t]
	f^{\prime}\left(1\right)
	&=\lim\limits_{x \to 1} \dfrac{f(x)-f(1)}{x-1}\\
	&=\lim\limits_{x \to 1} \dfrac{\sqrt{x}-1}{x-1}\\
	&=\lim\limits_{x \to 1} \dfrac{x-1}{(\sqrt{x}+1)\left(x-1\right)}\\
	&=\lim\limits_{x \to 1} \dfrac{1}{\sqrt{x}+1}\\
	&=\dfrac{1}{2}
	\end{aligned}
	.$
	\item %c
	Ta có:
	$\begin{aligned}[t]
	f^{\prime}\left(2\right)
	&=\lim\limits_{x \to 2} \dfrac{f(x)-f(2)}{x-2}\\
	&=\lim\limits_{x \to 2} \dfrac{\dfrac{1}{x-1}-\dfrac{1}{2-1}}{x-2}\\
	&=\lim\limits_{x \to 2} \dfrac{2-x}{(x-1)\left(x-2\right)}\\
	&=\lim\limits_{x \to 2} \dfrac{-1}{x-1}\\
	&=-1
	\end{aligned}
	.$
	\item %d
	Ta có:
	$\begin{aligned}[t]
	f^{\prime}\left(5\right)
	&=\lim\limits_{x \to 5} \dfrac{f(x)-f(5)}{x-5}\\
	&=\lim\limits_{x \to 5} \dfrac{\sqrt{x-1}-\sqrt{5-1}}{x-5}\\
	&=\lim\limits_{x \to 5} \dfrac{x-5}{(\sqrt{x-1}+2)\left(x-5\right)}\\
	&=\lim\limits_{x \to 5} \dfrac{1}{\sqrt{x-1}+2}\\
	&=\dfrac{1}{4}
	\end{aligned}
	.$
	\end{listEX}
	}
\end{vd}
\begin{vd}%[1K9YU1]
	Sử dụng định nghĩa, tìm đạo hàm của các hàm số sau:
	\begin{listEX}[4]
	\item $y=2x-3$;
	\item $y=x^2$;
	\item $y=x^2+1$;
	\item $y=x^2+2x$;
	\item $y=x^2-2x+1$;
	\item $y=2x^3$;
	\item $y=x^3+1$;
	\item $y=x^3-x$.
	\end{listEX} 
	\loigiai{
	\begin{enumerate}
	\item 
	Với bất kì $x_0$, ta có: 
	$\begin{aligned}[t]
	f^{\prime}(x_0)
	&=\lim\limits_{x \to x_0}\dfrac{f(x)-f(x_0)}{x-x_0}
	=\lim\limits_{x \to x_0}\dfrac{(2x-3)-(2x_0-3)}{x-x_0}\\
	&=\lim\limits_{x\to x_0}\dfrac{2(x-x_0)}{x-x_0}
	=\lim\limits_{x\to x_0}2=2
	\end{aligned}
	.$\\
	Vậy $(2x-3)'=2$ trên $\mathbb{R}$.
	\item 
	Với bất kì $x_0$, ta có: 
	$\begin{aligned}[t]
	f^{\prime}(x_0)
	&=\lim\limits_{x \to x_0}\dfrac{f(x)-f(x_0)}{x-x_0}\\
	&=\lim\limits_{x \to x_0}\dfrac{x^2-x_0^2}{x-x_0}\\
	&=\lim\limits_{x \to x_0}\dfrac{(x-x_0)(x+x_0)}{x-x_0}\\
	&=\lim\limits_{x \to x_0}(x+x_0)=2x_0
	\end{aligned}
	.$\\
	Vậy $(x^2)'=2x$ trên $\mathbb{R}$.
	\item 
	Với bất kì $x_0$, ta có: 
	$\begin{aligned}[t]
	f^{\prime}(x_0)
	&=\lim\limits_{x \to x_0}\dfrac{f(x)-f(x_0)}{x-x_0}\\
	&=\lim\limits_{x \to x_0}\dfrac{(x^2+1)-(x_0^2+1)}{x-x_0}\\
	&=\lim\limits_{x \to x_0}\dfrac{x^2-x_0}{x-x_0}\\
	&=\lim\limits_{x \to x_0}(x+x_0)=2x_0
	\end{aligned}
	.$\\
	Vậy $(x^2+1)'=2x$ trên $\mathbb{R}$.
	\item 
	Với bất kì $x_0$, ta có: 
	$\begin{aligned}[t]
	f^{\prime}(x_0)
	&=\lim\limits_{x \to x_0}\dfrac{f(x)-f(x_0)}{x-x_0}\\
	&=\lim\limits_{x \to x_0}\dfrac{(x^2+2x)-(x_0^2+2x_0)}{x-x_0}\\
	&=\lim\limits_{x \to x_0}\dfrac{(x-x_0)(x+x_0+2)}{x-x_0}\\
	&=\lim\limits_{x \to x_0}(x+x_0+2)=2x_0+2
	\end{aligned}
	.$\\
	Vậy $(x^2+2x)'=2x+2$ trên $\mathbb{R}$.
	\item 
	Với bất kì $x_0$, ta có: 
	$\begin{aligned}[t]
	f^{\prime}(x_0)
	&=\lim\limits_{x \to x_0}\dfrac{f(x)-f(x_0)}{x-x_0}\\
	&=\lim\limits_{x \to x_0}\dfrac{(x^2-2x+1)-(x_0^2-2x_0+1)}{x-x_0}\\
	&=\lim\limits_{x \to x_0}\dfrac{(x^2-x_0^2)-2(x-x_0)}{x-x_0}\\
	&=\lim\limits_{x \to x_0}(x+x_0-2)=2x_0-2
	\end{aligned}
	$.\\
	Vậy $(x^2-2x+1)'=2x-2$ trên $\mathbb{R}$.
	\item 
	Với bất kì $x_0$, ta có: 
	$\begin{aligned}[t]
	f^{\prime}(x_0)
	&=\lim\limits_{x \to x_0}\dfrac{f(x)-f(x_0)}{x-x_0}\\
	&=\lim\limits_{x \to x_0}\dfrac{2x^3-2x_0^3}{x-x_0}\\
	&=\lim\limits_{x \to x_0}\dfrac{2(x^3-x_0^3)}{x-x_0}\\
	&=\lim\limits_{x \to x_0}2(x^2+x_0x+x_0^2)=6x_0^2
	\end{aligned}
	$.\\
	Vậy $(2x^3)'=6x^2$ trên $\mathbb{R}$.
	\item 
	Với bất kì $x_0$, ta có: 
	$\begin{aligned}[t]
	f^{\prime}(x_0)
	&=\lim\limits_{x \to x_0}\dfrac{f(x)-f(x_0)}{x-x_0}\\
	&=\lim\limits_{x \to x_0}\dfrac{(x^3+1)-(x_0^3+1)}{x-x_0}\\
	&=\lim\limits_{x \to x_0}\dfrac{x^3-x_0^3}{x-x_0}\\
	&=\lim\limits_{x \to x_0}(x^2+x_0x+x_0^2)=3x_0^2
	\end{aligned}
	$.\\
	Vậy $(x^3+1)'=3x^2$ trên $\mathbb{R}$.
	\item 
	Với bất kì $x_0$, ta có: 
	$\begin{aligned}[t]
	f^{\prime}(x_0)
	&=\lim\limits_{x \to x_0}\dfrac{f(x)-f(x_0)}{x-x_0}\\
	&=\lim\limits_{x \to x_0}\dfrac{(x^3-x)-(x_0^3-x_0)}{x-x_0}\\
	&=\lim\limits_{x \to x_0}\dfrac{(x^3-x_0^3)-(x-x_0)}{x-x_0}\\
	&=\lim\limits_{x \to x_0}(x^2+x_0x+x_0^2-1)=3x_0^2-1
	\end{aligned}
	$.\\
	Vậy $(x^3-x)'=3x^2-1$ trên $\mathbb{R}$.
	\end{enumerate}
	}
\end{vd}
\begin{vd}%[1D5B1]
	Sử dụng định nghĩa, tìm đạo hàm của các hàm số sau:
	\begin{listEX}[4]
	\item $f(x)=\dfrac{2}{x}$;
	\item $f(x)=\sqrt{x}$;
	\item $f(x)=\dfrac{1}{x-1}$;
	\item $f(x)=\sqrt{x-1}$.
	\end{listEX}
	\loigiai{
	\begin{listEX}[1]
	\item %a
	Với bất kì $x_0 \neq 0$, ta có:
	$\begin{aligned}[t]
	f^{\prime}\left(x_0\right)
	&=\lim\limits_{x \to x_0} \dfrac{f(x)-f(x_0)}{x-x_0}\\
	&=\lim\limits_{x \to x_0} \dfrac{\dfrac{2}{x}-\dfrac{2}{x_0}}{x-x_0}\\
	&=\lim\limits_{x \to x_0} \dfrac{2(x_0-x)}{x x_0\left(x-x_0\right)}\\
	&=\lim\limits_{x \to x_0} \dfrac{-2}{x x_0}\\
	&=-\dfrac{2}{x_0^2}
	\end{aligned}
	.$\\
	Vậy $f^{\prime}(x)=\left(\dfrac{2}{x}\right)^{\prime}=-\dfrac{2}{x^2}$ trên các khoảng $(-\infty ; 0)$ và $(0 ;+\infty)$.
	\item %b
	Với bất kì $x_0 >0$, ta có:
	$\begin{aligned}[t]
	f^{\prime}\left(x_0\right)
	&=\lim\limits_{x \to x_0} \dfrac{f(x)-f(x_0)}{x-x_0}\\
	&=\lim\limits_{x \to x_0} \dfrac{\sqrt{x}-\sqrt{x_0}}{x-x_0}\\
	&=\lim\limits_{x \to x_0} \dfrac{x-x_0}{(\sqrt{x}+\sqrt{x_0})\left(x-x_0\right)}\\
	&=\lim\limits_{x \to x_0} \dfrac{1}{\sqrt{x}+\sqrt{x_0}}\\
	&=\dfrac{1}{2\sqrt{x_0}}
	\end{aligned}
	.$\\
	Vậy $f^{\prime}(x)=\left(\sqrt{x}\right)^{\prime}=\dfrac{1}{2\sqrt{x}}$ trên $(0 ;+\infty)$.	
	\item %c
	Với bất kì $x_0 \neq 1$, ta có:
	$\begin{aligned}[t]
	f^{\prime}\left(x_0\right)
	&=\lim\limits_{x \to x_0} \dfrac{f(x)-f(x_0)}{x-x_0}\\
	&=\lim\limits_{x \to x_0} \dfrac{\dfrac{1}{x-1}-\dfrac{1}{x_0-1}}{x-x_0}\\
	&=\lim\limits_{x \to x_0} \dfrac{(x_0-x)}{(x-1)(x_0-1)\left(x-x_0\right)}\\
	&=\lim\limits_{x \to x_0} \dfrac{-1}{(x-1)(x_0-1)}\\
	&=-\dfrac{1}{(x_0-1)^2}
	\end{aligned}
	.$\\
	Vậy $f^{\prime}(x)=\left(\dfrac{1}{x-1}\right)^{\prime}=-\dfrac{1}{(x-1)^2}$ trên các khoảng $(-\infty ; 1)$ và $(1 ;+\infty)$.
	\item %d
	Với bất kì $x_0 \neq 1$, ta có:
	$\begin{aligned}[t]
	f^{\prime}\left(x_0\right)
	&=\lim\limits_{x \to x_0} \dfrac{f(x)-f(x_0)}{x-x_0}\\
	&=\lim\limits_{x \to x_0} \dfrac{\sqrt{x-1}-\sqrt{x_0-1}}{x-x_0}\\
	&=\lim\limits_{x \to x_0} \dfrac{x-x_0}{(\sqrt{x-1}+\sqrt{x_0-1})\left(x-x_0\right)}\\
	&=\lim\limits_{x \to x_0} \dfrac{1}{\sqrt{x-1}+\sqrt{x_0-1}}\\
	&=\dfrac{1}{2\sqrt{x_0-1}}
	\end{aligned}
	.$\\
	Vậy $f^{\prime}(x)=\left(\sqrt{x-1}\right)^{\prime}=\dfrac{1}{2\sqrt{x-1}}$ trên $(1 ;+\infty)$.
	\end{listEX}
	}
\end{vd}
% \begin{vd}%[1K9KU3]
% 	Giải bài toán sau (bỏ qua sức cản của không khí và làm tròn kết quả đến chữ số thập phân thứ nhất).\\
% 	Nếu một quả bóng được thả rơi tự do từ đài quan sát trên sân thượng của tòa nhà Landmark 81 (Thành phố Hồ Chí Minh) cao $461{,}3$~m xuống mặt đất. Có tính được vận tốc của quả bóng khi nó chạm đất hay không?
% 	\loigiai{
% 	Phương trình chuyển động rơi tự do của quả bóng là $s=f(t)=\dfrac{1}{2}gt^2$ ($g$ là gia tốc rơi tự do, lấy $g=9{,}8$ m/s$^2$). Do vậy, vận tốc của quả bóng tại thời điểm $t$ là $v(t)=f'(t)=gt=9{,}8t$. Mặt khác, vì chiều cao của tòa tháp là $461{,}3$ m nên quả bóng sẽ chạm đất tại thời điểm $t_1$, với $f(t_1)=461{,}3$. Từ đó, ta có:
% 	\[4{,}9t_1^2=461{,}3\Leftrightarrow t_1=\sqrt{\dfrac{462{,}3}{4{,}9}}\text{ (giây).}\]
% 	Vậy vận tốc của quả bóng khi nó chạm đất là 
% 	\[v(t_1)=9{,}8t_1=9{,}8\cdot \sqrt{\dfrac{461{,}3}{4{,}9}}\approx 95{,}1 \text{ (m/s).}\]
% 	}
% \end{vd}
\subsubsection{Bài tập áp dụng}
\begin{bt}%[1C7Y1-2]
	Tính đạo hàm của hàm số $f(x)=\dfrac{1}{x}$ tại $x_0=2$ bằng định nghĩa.
	\loigiai{
	Xét $\Delta x$ là số gia của biến số tại điểm $x_0=2$.\\
	Ta có: $\Delta y=f(2+\Delta x)-f(2)=\dfrac{1}{2+\Delta x}-\dfrac{1}{2}=\dfrac{2-(2+\Delta x)}{2(2+\Delta x)}=\dfrac{-\Delta x}{2(2+\Delta x)}$.\\
	Suy ra: $\dfrac{\Delta y}{\Delta x}=\dfrac{-1}{2(2+\Delta x)}$.\\
	Ta thấy: $\lim\limits_{\Delta x\to 0}\dfrac{\Delta y}{\Delta x}=\lim\limits_{\Delta x\to 0}\dfrac{-1}{2(2+\Delta x)}=\dfrac{-1}{4}$.\\
	Vậy $f^\prime(2)=\dfrac{-1}{4}$.
	}
\end{bt}
\begin{bt}%[1D5B1]
	Tính đạo hàm của hàm số $y=-x^2+3x-2$ tại điểm $x_0=2$ bằng định nghĩa.
	\loigiai{
	Đặt $\Delta x=x-2$. Ta có
	\begin{align*}
	& \Delta y=f(2+\Delta x)-f(2)=[-(2+\Delta x)^2+3(2+\Delta x)-2]-(-2^2+3\cdot 2-2)=-\Delta^2 x-\Delta x;\\
	& \dfrac{\Delta y}{\Delta x}=-\Delta x-1;\\
	& \lim\limits_{\Delta x\to 0}\dfrac{\Delta y}{\Delta x}=\lim\limits_{\Delta x\to 0}(-\Delta x-1)=-1.
	\end{align*}
	Vậy $y'(2)=-1$.
	}
\end{bt}
\begin{bt}%[1D5B1]
	Tính đạo hàm của hàm số $f(x)=\dfrac{1}{x-3}$ tại $x_0=4$.
	\loigiai{
	Đặt $\Delta x=x-4$. Ta có
	\begin{align*}
	& \Delta y=f(4+\Delta x)-f(4)=\dfrac{1}{1+\Delta x}-1=\dfrac{-\Delta x}{1+\Delta x};\\
	& \dfrac{\Delta y}{\Delta x}=\dfrac{-1}{1+\Delta x};\\
	& \lim\limits_{\Delta x\to 0}\dfrac{\Delta y}{\Delta x}=\lim\limits_{\Delta x\to 0}\dfrac{-1}{1+\Delta x}=-1.
	\end{align*}
	Vậy $f'(4)=-1$.
	}
\end{bt}
\begin{bt}%[1D5G1]
	Tính đạo hàm của hàm số $f(x)=\sin 3x$ tại $x_0=\dfrac{\pi}{6}$.
	\loigiai{
	Đặt $\Delta x=x-\dfrac{\pi}{6}$. Ta có
	\begin{align*}
	& \Delta y=f\left(\dfrac{\pi}{6}+\Delta x\right)-f\left(\dfrac{\pi}{6}\right)=\sin\left(\dfrac{\pi}{2}+3\Delta x\right)-\sin\dfrac{\pi}{2}=\cos(3\Delta x)-1;\\
	& \dfrac{\Delta y}{\Delta x}=\dfrac{\cos(3\Delta x)-1}{\Delta x}=-\dfrac{2\sin^2\tfrac{3\Delta x}{2}}{\Delta x};\\
	& \lim\limits_{\Delta x\to 0}\dfrac{\Delta y}{\Delta x}=\lim\limits_{\Delta x\to 0}\dfrac{-2\sin^2\tfrac{3\Delta x}{2}}{\Delta x}=0.
	\end{align*}
	Vậy $f'\left(\dfrac{\pi}{6}\right)=0$.
	}
\end{bt}
\begin{bt}%[1D5K1]
	Tính đạo hàm của hàm số $f(x)=4x-x^2$ tại điểm $x=2$.
	\loigiai{
	Đáp số: $f'(2)=0$.
	}
\end{bt}
\begin{bt}%[1D5K1]
	Tính đạo hàm của hàm số $f(x)=\sqrt{3x+1}$ tại điểm $x=1$.
	\loigiai{
	Đáp số: $f'(1)=\dfrac{3}{4}$.
	}
\end{bt}
\begin{bt}%[1T7B1-1]
	Sử dụng định nghĩa, tìm đạo hàm của các hàm số sau:
	\begin{listEX}[2]
	\item $f(x)=C\,(C$ là hằng số $)$;
	\item $f(x)=\dfrac{1}{x}$ với $x \neq 0$;
	\item $f(x)=x^2$;
	\item $f(x)=cx^2$ với $c$ là hằng số;
	\item $f(x)=x^3$;
	\item $f(x)=3x-5$;
	\item $f(x)=\sqrt{x+2}$;
	\item $f(x)=cosx$.
	\end{listEX}
	\loigiai{
	\begin{enumerate}
	\item Với bất kì $x_0$, ta có:
	$$
	f^{\prime}(x_0)=\lim\limits_{x \to x_0} \dfrac{f(x)-f\left(x_0\right)}{x-x_0}=\lim\limits_{x \to x_0} \dfrac{C-C}{x-x_0}=\lim\limits_{x \to x_0} 0=0 .
	$$
	Vậy $f^{\prime}(x)=(C)^{\prime}=0$ trên $\mathbb{R}$.
	\item Với bất kì $x_0 \neq 0$, ta có:
	$$
	f^{\prime}\left(x_0\right)=\lim\limits_{x \to x_0} \dfrac{\dfrac{1}{x}-\dfrac{1}{x_0}}{x-x_0}=\lim\limits_{x \to x_0} \dfrac{x_0-x}{x x_0\left(x-x_0\right)}=\lim\limits_{x \to x_0} \dfrac{-1}{x x_0}=-\dfrac{1}{x_0^2} .
	$$
	Vậy $f^{\prime}(x)=\left(\dfrac{1}{x}\right)^{\prime}=-\dfrac{1}{x^2}$ trên các khoảng $(-\infty ; 0)$ và $(0 ;+\infty)$.
	\item Với $x_0$ bất kì, ta có 
	$$f^{\prime}\left(x_0\right)=\lim\limits_{x \to x_0} \dfrac{f(x)-f\left(x_0\right)}{x-x_0}=\lim\limits_{x \to x_0} \dfrac{x^2-x_0^2}{x-x_0}=\lim\limits_{x \to x_0}\left(x+x_0\right)=2 x_0.$$
	\item Với $x_0$ bất kì, ta có:
	$$f'(x_0)=\lim\limits_{x \to x_0} \dfrac{cx^2-cx_0^2}{x-x_0}=\lim\limits_{x \to x_0} \dfrac{c(x-x_0)(x+x_0)}{x-x_0}=\lim\limits_{x \to x_0} c(x+x_0)=c(x_0+x_0)=2cx_0.$$
	Vậy hàm số $y=cx^2$ (với $c$ là hằng số) có đạo hàm là hàm số $y'=2cx$.
	\item Với bất kì $x_0$, ta có:\\
	$$f^\prime(x_0)=\lim\limits_{x \to x_0}\dfrac{x^3-{x_0}^3}{x-x_0}=\lim\limits_{x \to x_0}(x^2+xx_0+{x_0}^2)=3{x_0}^2.$$
	Vậy $f^\prime(x)=\left(x^3\right)^\prime=3x^2$ trên $\mathbb{R}$.
	\item $f'(x)=(3x-5)'=3$ trên $\mathbb{R}$.
	\item $f'(x)=\left(\sqrt{x+2}\right)'=\dfrac{1}{2\sqrt{x+2}}$ trên $(-2,+\infty)$.
	\item $f'(x)=(\cos x)'=-\sin x$ trên $\mathbb{R}$.
	\end{enumerate}}
\end{bt}
\begin{bt}%[1T7Y1-4]
	Quãng đường rơi tự do của một vật được biểu diễn bởi công thức $s(t)=4{,}9t^2$ với $t$ là thời gian tính bằng giây và $s$ tính bằng mét. Tính vận tốc tức thời của chuyển động lúc $t=2$.
	\loigiai{
	Đặt: $\Delta t=t-2$.\\
	Ta có:
	\begin{eqnarray*}
	\Delta s&=&f(2+\Delta t)-f(2)\\
	&=&4{,}9\cdot(2+\Delta t)^2-4{,}9\cdot2^2=4{,}9\cdot\left[(2+\Delta t)^2-2^2\right]\\
	&=&4{,}9\cdot \left[\Delta t\cdot(\Delta t+4)\right].
	\end{eqnarray*}
	Vận tốc tức thời tại thời điểm $t=2$.\\
	Suy ra, $f^\prime(2)=\lim\limits_{\Delta t\to 0}\dfrac{\Delta s}{\Delta t}=\lim\limits_{\Delta t\to 0}\dfrac{4,9\cdot \left[\Delta t\cdot(\Delta t+4)\right]}{\Delta t}=\lim\limits_{\Delta t\to 0}4,9\cdot (\Delta t+4)=19,6.$}
\end{bt}
%===================
\begin{dang}{Phương trình tiếp tuyến của đồ thị hàm số }
\end{dang}
\subsubsection{Ví dụ minh hoạ}
\begin{vd}%[1C7Y1-3]
	Cho hàm số $y=-x^2$ có đồ thị $(C)$.
	\begin{enumerate}
	\item Xác định hệ số góc của tiếp tuyến của đồ thị $(C)$ tại điểm có hoành độ bằng $3$.
	\item Viếp phương trình tiếp tuyến của đồ thị $(C)$ tại điểm $M(3;-9)$.
	\end{enumerate}
	\loigiai{
	\begin{enumerate}
	\item Tiếp tuyến của đồ thị $(C)$ tại điểm có hoành độ bằng $3$ có hệ số góc là:
	$$\begin{aligned}[t]
	f^\prime(3)
	=\lim\limits_{x\to 3}\dfrac{f(x)-f(3)}{x-3}
	=\lim\limits_{x\to 3}\dfrac{-x^2-(-3^2)}{x-3}
	=\lim\limits_{x\to 3}(-x-3)=-6.
	\end{aligned}$$
	\item Phương trình tiếp tuyến của đồ thị $(C)$ tại điểm $M(3;-9)$ là
	$$\begin{aligned}[t]
	y=-6(x-3)+(-9)
	\Leftrightarrow y=-6x+9.
	\end{aligned}$$
	\end{enumerate}
	}
\end{vd}
\begin{vd}%[1T7B1-3]
	Cho $(C)$ là đồ thị của hàm số $f(x)=\dfrac{1}{x}$ và điểm $M(1; 1) \in(C)$. Tính hệ số góc của tiếp tuyến của $(C)$ tại điểm $M$ và viết phương trình tiếp tuyến đó.
	\loigiai{
	Ở bài trước ta tính được $f^\prime(x)=-\dfrac{1}{x^2}$.
	\begin{itemize}
	\item Hệ số góc của tiếp tuyến $(C)$ tại $M(1 ; 1)$ là $k=f^\prime(1)=-1.$
	\item Phương trình tiếp tuyến của $(C)$ tại $M(1 ; 1)$ là $y=-x+2.$
	\end{itemize}}
\end{vd}
\begin{vd}%[1K9BU2]
	Cho $(C)$ là đồ thị của hàm số $y=\sqrt{x}$. Viết phương trình tiếp tuyến của $(C)$, biết:
	\begin{listEX}[2]
	\item Tiếp điểm có hoành độ $x_0=4$;
	\item Tiếp điểm có tung độ $y_0=3$.
	\end{listEX}	
	\loigiai{
	Ta có $y'=\dfrac{1}{2\sqrt{x}}$.
	\begin{enumerate}
	\item Tiếp điểm có hoành độ $x_0=4$\\
	Gọi $(x_0;y_0)$ là tọa độ tiếp điểm.\\
	Hệ số góc tiếp tuyến $k=f'(4)=\dfrac{1}{4}$, 
	$y_0=f(4)=2$.\\
	Phương trình tiếp tuyến $y=\dfrac{1}{4}(x-4)+2$ hay $y=\dfrac{1}{4}x+1$.
	\item Tiếp điểm có tung độ $y_0=3$.\\
	Gọi $(x_0;y_0)$ là tọa độ tiếp điểm.\\
	$y_0=3\Leftrightarrow \sqrt{x}=3\Leftrightarrow x=9$.\\
	Hệ số góc của tiếp tuyến $k=f'(9)=\dfrac{1}{6}$.\\
	Phương trình tiếp tuyến $y=\dfrac{1}{6}(x-9)+3$ hay $y=\dfrac{1}{6}x+\dfrac{3}{2}$.
	\end{enumerate}	
	}
\end{vd}
\begin{vd}%[1K9BU2]
	Viết phương trình tiếp tuyến của parabol $y=x^2+2x$, biết:
	\begin{listEX}[2]
	\item Tiếp điểm có hoành độ $x_0=2$;
	\item Tiếp điểm có tung độ $y_0=3$.
	\end{listEX}	
	\loigiai{
	Ta có $y'=2x+2$.
	\begin{enumerate}
	\item Tiếp điểm có hoành độ $x_0=2$\\
	Gọi $(x_0;y_0)$ là tọa độ tiếp điểm.\\
	Hệ số góc tiếp tuyến $k=f'(2)=6$, 
	$y_0=f(2)=8$.\\
	Phương trình tiếp tuyến $y=6(x-2)+8$ hay $y=6x-4$.
	\item Tiếp điểm có tung độ $y_0=3$.\\
	Gọi $(x_0;y_0)$ là tọa độ tiếp điểm.\\
	$y_0=3\Leftrightarrow x_0^2+2x_0=3\Leftrightarrow \hoac{&x_0=1\\&x_0=-3.}$
	\begin{enumerate}[\bf TH1.]
	\item $x_0=1$, 	$k=f'(1)=4$.
	Phương trình tiếp tuyến $y=4(x-1)+3$ hay $y=4x-1$.
	\item $x=-3$, $k=f'(-3)=-4$.
	Phương trình tiếp tuyến $y=-4(x+3)+3$ hay $y=-4x-8$.
	\end{enumerate}
	\end{enumerate}	
	}
\end{vd}
%-----------
\subsubsection{Bài tập áp dụng}
\begin{bt}%[1C7Y1-3]
	Cho hàm số $y=x^2$ có đồ thị $(C)$.
	\begin{enumerate}
	\item Xác định hệ số góc của tiếp tuyến của đồ thị $(C)$ tại điểm có hoành độ bằng $2$.
	\item Viếp phương trình tiếp tuyến của đồ thị $(C)$ tại điểm $M(2;4)$.
	\end{enumerate}
	\loigiai{
	\begin{enumerate}
	\item Tiếp tuyến của đồ thị $(C)$ tại điểm có hoành độ bằng $2$ có hệ số góc là:
	$$\begin{aligned}[t]
	f^\prime(2)
	=\lim\limits_{x\to 2}\dfrac{f(x)-f(2)}{x-2}
	=\lim\limits_{x\to 2}\dfrac{x^2-2^2}{x-2}
	=\lim\limits_{x\to 2}(x+2)=4.
	\end{aligned}$$
	\item Phương trình tiếp tuyến của đồ thị $(C)$ tại điểm $M(2;4)$ là
	$$\begin{aligned}[t]
	y=4(x-2)+4
	\Leftrightarrow y=4x-4.
	\end{aligned}$$
	\end{enumerate}
	}
\end{bt}
\begin{bt}%[1T7B1-3]
	Cho $(C)$ là đồ thị của hàm số $f(x)=\dfrac{1}{x-1}$ và điểm $M(2; 1) \in(C)$. Tính hệ số góc của tiếp tuyến của $(C)$ tại điểm $M$ và viết phương trình tiếp tuyến đó.
	\loigiai{
	Ở bài trước ta tính được $f^\prime(x)=-\dfrac{1}{(x-1)^2}$.
	\begin{itemize}
	\item Hệ số góc của tiếp tuyến $(C)$ tại $M(2 ; 1)$ là $k=f^\prime(2)=-1.$
	\item Phương trình tiếp tuyến của $(C)$ tại $M(2 ; 1)$ là $y=-1(x-2)+1\Leftrightarrow y=-x+3.$
	\end{itemize}}
\end{bt}
\begin{bt}%[1K9BU2]
	Cho $(C)$ là đồ thị của hàm số $y=\sqrt{x-1}$. Viết phương trình tiếp tuyến của $(C)$, biết:
	\begin{listEX}[2]
	\item Tiếp điểm có hoành độ $x_0=2$;
	\item Tiếp điểm có tung độ $y_0=3$.
	\end{listEX}	
	\loigiai{
	Ta có $y'=\dfrac{1}{2\sqrt{x-1}}$.
	\begin{enumerate}
	\item Tiếp điểm có hoành độ $x_0=2$\\
	Gọi $(x_0;y_0)$ là tọa độ tiếp điểm.\\
	Hệ số góc tiếp tuyến $k=f'(2)=\dfrac{1}{2}$, 
	$y_0=f(2)=1$.\\
	Phương trình tiếp tuyến $y=\dfrac{1}{2}(x-2)+1$ hay $y=\dfrac{1}{2}x$.
	\item Tiếp điểm có tung độ $y_0=3$.\\
	Gọi $(x_0;y_0)$ là tọa độ tiếp điểm.\\
	$y_0=3\Leftrightarrow \sqrt{x-1}=3\Leftrightarrow x=10$.\\
	Hệ số góc của tiếp tuyến $k=f'(10)=\dfrac{1}{6}$.\\
	Phương trình tiếp tuyến $y=\dfrac{1}{6}(x-10)+3$ hay $y=\dfrac{1}{6}x+\dfrac{4}{3}$.
	\end{enumerate}	
	}
\end{bt}
\begin{bt}%[1K9BU2]
	Viết phương trình tiếp tuyến của parabol $y=x^2+1$, biết:
	\begin{listEX}[2]
	\item Tiếp điểm có hoành độ $x_0=3$;
	\item Tiếp điểm có tung độ $y_0=5$.
	\end{listEX}	
	\loigiai{
	Ta có $y'=2x$.
	\begin{enumerate}
	\item Tiếp điểm có hoành độ $x_0=3$\\
	Gọi $(x_0;y_0)$ là tọa độ tiếp điểm.\\
	Hệ số góc tiếp tuyến $k=f'(3)=6$, 
	$y_0=f(3)=10$.\\
	Phương trình tiếp tuyến $y=6(x-3)+10$ hay $y=6x-8$.
	\item Tiếp điểm có tung độ $y_0=5$.\\
	Gọi $(x_0;y_0)$ là tọa độ tiếp điểm.\\
	$y_0=5\Leftrightarrow x_0^2+1=5\Leftrightarrow \hoac{&x_0=2\\&x_0=-2.}$
	\begin{enumerate}[\bf TH1.]
	\item $x_0=2$, 	$k=f'(2)=4$.
	Phương trình tiếp tuyến $y=4(x-2)+5$ hay $y=4x-3$.
	\item $x=-2$, $k=f'(-2)=-4$.
	Phương trình tiếp tuyến $y=-4(x+2)+5$ hay $y=-4x-3$.
	\end{enumerate}
	\end{enumerate}	
	}
\end{bt}
%%%%%%%%%%%%%%%%%%%
\subsection{Bài tập rèn luyện}
\begin{bt}%[1K9YU1]
Tính (bằng định nghĩa) đạo hàm của các hàm số sau:
\begin{listEX}[1]
	\item $y=x^2-x$ tại $x_0 =1$;
	\item $y=-x^3$ tại $x_0=-1$;
	\item $f(x)=3x^3-1$ tại $x_0=1$.
\end{listEX}
\loigiai{
	\begin{enumerate}
	\item 
	$y=x^2-x$ tại $x_0 =1$;
	Ta có $f(x)-f(1)=x^2-x-(1^2-1)=x(x-1)$.\\
	Với $x \ne 1, \dfrac{f(x)-f(1)}{x-1}=\dfrac{x(x-1)}{x-1}=x$.\\
	Tính giới hạn $\lim\limits_{x \to 1} \dfrac{f(x)-f(1)}{x-1}=\lim\limits_{x \to 1} x=1$.\\
	Vậy $f'(1)=1$.
	\item 
	$y=-x^3$ tại $x_0=-1$.\\
	Ta có $f(x)-f(-1)=-x^3-1=-(x+1)(x^2-x+1)$.\\
	Với $x \ne 1, \dfrac{f(x)-f(-1)}{x-(-1)}=\dfrac{-(x+1)(x^2-x+1)}{x+1}=-(x^2-x+1)$.\\
	Tính giới hạn $\lim\limits_{x \to -1} \dfrac{f(x)-f(-1)}{x+1}=\lim\limits_{x \to -1} -(x^2-x+1)=-3$.\\
	Vậy $f'(-1)=-3$.
	\item
	Xét $\Delta x=x_0-1$. Ta có: 
	$$\begin{aligned}[t]
	\Delta y
	=&f(1+\Delta x)-f(1)\\
	=&3(1+\Delta x)^3-1-2\\
	=&9\Delta x+9(\Delta x)^2+3(\Delta x)^3\\
	=&\Delta x[9+9\Delta x+3(\Delta x)^2].\\
	\end{aligned}$$
	Suy ra: $\dfrac{\Delta y}{\Delta x}=9+9\Delta x+3(\Delta x)^2$.\\
	Ta thấy: $\lim\limits_{\Delta x\to 0}\dfrac{\Delta y}{\Delta x}=\lim\limits_{\Delta x\to 0}\left(9+9\Delta x+3(\Delta x)^2\right)=9$.\\
	Vậy $f^\prime(1)=9$.
	\end{enumerate}
}
\end{bt}
\begin{bt}%[1K9YU1]
	Sử dụng định nghĩa, tìm đạo hàm của các hàm số sau:
	\begin{listEX}[2]
	\item! $y=kx^2+c$ (với $k,c$ là hằng số);
	\item $y=x^3$.
	\item $f(x)=-x^2$;
	\item $f(x)=x^3-2 x$;
	\item $f(x)=\dfrac{4}{x}$.
	\end{listEX}
\loigiai{
	\begin{enumerate}
	\item $y=kx^2+c$ (với $k,c$ là hằng số).\\
	Với $x_0$ bất kì, ta có:\\
	$\begin{aligned}[t]
	f'(x_0)&=\lim\limits_{x \to x_0} \dfrac{(kx^2+c)-(kx_0^2+c)}{x-x_0}=\lim\limits_{x \to x_0} \dfrac{k(x-x_0)(x+x_0)}{x-x_0}\\
	&=\lim\limits_{x \to x_0} k(x+x_0)=k(x_0+x_0)=2kx_0.\end{aligned}$\\
	Vậy hàm số $y=kx^2+c$ (với $c,k$ là hằng số) có đạo hàm là hàm số $y'=2kx$.
	\item $y=x^3$\\
	Với $x_0$ bất kì, ta có:\\
	$\begin{aligned}[t]
	f'(x_0)&=\lim\limits_{x \to x_0} \dfrac{(x^3)-(x_0^3)}{x-x_0}=\lim\limits_{x \to x_0} \dfrac{(x-x_0)(x^2+xx_0+x_0^2)}{x-x_0}\\
	&=\lim\limits_{x \to x_0} (x^2+xx_0+x_0^2)=(x_0^2+x_0x_0+x_0^2)=3x_0^2.\end{aligned}$\\
	Vậy hàm số $y=x^3$ có đạo hàm là hàm số $y'=3x^2$.
	\item Với bất kì $x_0$, ta có\\
	$f^\prime(x_0)=\lim\limits_{x\to x_0}\dfrac{f(x)-f(x_0)}{x-x_0}=\lim\limits_{x\to x_0}\dfrac{-x^2+{x_0}^2}{x-x_0}=\lim\limits_{x\to x_0}\left[-(x+x_0)\right]=-2x_0$.\\
	Vậy $f^\prime(x)=-2x$ trên $\mathbb{R}$.
	\item Với bất kì $x_0$, ta có\\
	$f^\prime(x_0)=\lim\limits_{x\to x_0}\dfrac{f(x)-f(x_0)}{x-x_0}=\lim\limits_{x\to x_0}\dfrac{\left[\left(x^3-2x\right)-\left({x_0}^3-2x_0\right)\right]}{x-x_0}$\\
	$=\lim\limits_{x\to x_0}\left[x^2+xx_0+{x_0}^2-2\right]=3{x_0}^2-2$.\\
	Vậy $f^\prime(x)=3x^2-2$ trên $\mathbb{R}$.
	\item Với mọi $x_0\ne 0$, ta có\\
	$f^{\prime}\left(x_0\right)=\lim\limits_{x \to x_0} \dfrac{\dfrac{4}{x}-\dfrac{4}{x_0}}{x-x_0}=\lim\limits_{x \to x_0} \dfrac{4(x_0-x)}{x x_0\left(x-x_0\right)}=\lim\limits_{x \to x_0} \dfrac{-4}{x x_0}=-\dfrac{4}{x_0^2}.$\\
	Vậy $f^{\prime}(x)=\left(\dfrac{4}{x}\right)^{\prime}=-\dfrac{4}{x^2}$ trên các khoảng $(-\infty ; 0)$ và $(0 ;+\infty)$.
	\end{enumerate}}
\end{bt}
\begin{bt}%[1T7B1-4]
	Một chuyển động thẳng xác định bởi phương trình $s(t)=4 t^3+6 t+2$, trong đó $s$ tính bằng mét và $t$ là thời gian tính bằng giây. Tính vận tốc tức thời của chuyển động tại $t=2$.
	\loigiai{
	Ta có: $v(t)=s^\prime(t)=12t^2+6$.\\
	Vận tốc tức thời tại $t=2$ là $v(2)=12\cdot 2^2+6=54 \;m/s$.
	}	
\end{bt}
\begin{bt}%[1T7T1-4]
	% \immini{
	Trên Mặt Trăng, quãng đường rơi tự do của một vật được cho bởi công thức $h(t)=0,81 t^2$, với $t$ được tính bằng giây và $h$ tính bằng mét. Hãy tính vận tốc tức thời của vật được thả rơi tự do trên Mặt Trăng tại thời điểm $t=2$.
	% }{
	% \includegraphics[scale=0.3]{HINHVE/CTST/CTST-7_1_1}
	% }
	\loigiai{Vận tốc rơi tức thời tại thời điểm $t=2$ là $v(2)=h^\prime(2)=1{,}62\cdot2=3{,}24$ m/s.
	}
\end{bt}
\begin{bt}%[1K9KU3]
	Một vật được phóng theo phương thẳng đứng lên trên từ mặt đất với vận tốc ban đầu là $19{,}6$ m/s thì độ cao $h$ của nó (tính bằng mét) sau $t$ giây được cho bởi công thức $h=19{,}6t-4{,}9t^2$. Tìm vận tốc của vật khi nó chạm đất.
	\loigiai{
	Phương trình biểu diễn độ cao của vật là $h=19{,}6t-4{,}9t^2$ nên vận tốc của vật theo thời gian $t$ là $v(t)=19{,}6-9{,}8t$.\\
	Khi vật chạm đất thì $h=0$ hay $19{,}6t-4{,}9t^2=0\Leftrightarrow \hoac{&t=0\text{ (loại)}\\&t=4\text{ (nhận)}.}$\\
	Vận tốc của vật khi chạm đất là $v(4)=19{,}6-9{,}8\cdot 4=-19{,}6$ m/s.
	}
\end{bt}
\begin{bt}%[1K9BU2]
	Tìm hệ số góc của tiếp tuyến của parabol $y=x^2$ tại điểm có hoành độ $x_0=-1$.
	\loigiai{
	Ta tính được $(x^2)'=2x$ nên $y'(-1)=2\cdot(-1)=-2$.\\
	Vậy hệ số góc của tiếp tuyến của paraol $y=x^2$ tại điểm có hoành độ $x_0=-1$ là $k=-2$. 
	}
\end{bt}
\begin{bt}%[1K9BU2]
	Viết phương trình tiếp tuyến của parabol $(P)\colon y=3x^2$ tại điểm có hoành độ $x_0=1$.
	\loigiai{
	Ta tính được $y'=6x$. Do đó, hệ số góc của tiếp tuyến là $k=f'(1)=6$. Ngoài ra, ta có $f(1)=3$ nên phương trình tiếp tuyến cần tìm là $y-3=6(x-1)$ hay $y=6x-3$.
	}
\end{bt}	
\begin{bt}%[1T7B1-3]
	Cho hàm số $f(x)=-2 x^2$ có đồ thị $(C)$ và điểm $A(1 ;-2) \in(C)$. Tính hệ số góc của tiếp tuyến với $(C)$ tại điểm $A$.
	\loigiai{Đạo hàm $f^\prime(x)=-4x$.\\
	Hệ số góc tiếp tuyến của $(C)$ tại điểm $A$ là
	$$k=f^\prime(1)=-4\cdot 1=-4.$$
	}
\end{bt}
\begin{bt}%[1T7B1-3]
	Viết phương trình tiếp tuyến của đồ thị hàm số $y=x^3$.
	\begin{enumerate}
	\item Tại điểm $(-1 ; 1)$;
	\item Tại điểm có hoành độ bằng $2$.
	\end{enumerate}
	\loigiai{
	\begin{enumerate}
	\item Ta có $f^\prime(x)=3x^2$ nên $f^\prime(-1)=3$.\\
	Phương trình tiếp tuyến của đồ thị tại điểm $(-1;1)$ là
	$$y=f^\prime(-1)\cdot (x+1)+1\Leftrightarrow y=3x+4.$$
	\item Gọi $A(2;8)$ thuộc đồ thị hàm số $y=x^3$\\
	Ta có $f^\prime(x)=3x^2$ nên $f^\prime(2)=12$.\\
	Phương trình tiếp tuyến của đồ thị tại $A$ là
	$$y=f^\prime(2)\cdot (x-2)+8\Leftrightarrow y=12x-16.$$
	\end{enumerate}}
\end{bt}
\begin{bt}%[1C7Y1-3]
	Cho hàm số $y=-2x^2+x$ có đồ thị $(C)$.
	\begin{enumerate}
	\item Xác định hệ số góc của tiếp tuyến của đồ thị $(C)$ tại điểm có hoành độ bằng $2$.
	\item Viết phương trình tiếp tuyến của đồ thị $(C)$ tại điểm $M(2;-6)$.
	\end{enumerate}
	\loigiai{
	\begin{enumerate}
	\item Tiếp tuyến của đồ thị $(C)$ tại điểm có hoành độ bằng $2$ có hệ số góc là:
	$$\begin{aligned}[t]
	f^\prime(2)
	=&\lim\limits_{x\to 2}\dfrac{f(x)-f(2)}{x-2}\\
	=&\lim\limits_{x\to 2}\dfrac{-2x^2+x-(-6)}{x-2}\\
	=&\lim\limits_{x\to 2}\dfrac{-2x^2+x+6}{x-3}\\
	=&\lim\limits_{x\to 2}\dfrac{(x-2)(-2x-3)}{x-2}\\
	=&\lim\limits_{x\to 2}(-2x-3)=-7.\\
	\end{aligned}$$
	\item Phương trình tiếp tuyến của đồ thị $(C)$ tại điểm $M(2;-6)$ là
	$$\begin{aligned}[t]
	y=&-7(x-2)+(-6)\\
	\Leftrightarrow y=&-7x+8.
	\end{aligned}$$
	\end{enumerate}
	}
\end{bt}
\begin{bt}%[1K9BU2]
Viết phương trình tiếp tuyến của parabol $y=-x^2+4x$, biết:
\begin{enumerate}
	\item Tiếp điểm có hoành độ $x_0=1$;
	\item Tiếp điểm có tung độ $y_0=0$.
\end{enumerate}	
\loigiai{
\begin{enumerate}
	\item Tiếp điểm có hoành độ $x_0=1$\\
	Gọi $(x_0;y_0)$ là tọa độ tiếp điểm.\\
	Ta có $y'=-2x+4$.\\
	Hệ số góc tiếp tuyến $k=f'(1)=2$, 
	$y_0=f(1)=3$.\\
	Phương trình tiếp tuyến $y-3=2(x-1)$ hay $y=2x+1$.
	\item Tiếp điểm có tung độ $y_0=0$.\\
	Gọi $(x_0;y_0)$ là tọa độ tiếp điểm.\\
	$y_0=0\Leftrightarrow -x_0^2+4x_0=0\Leftrightarrow \hoac{&x_0=0\\&x_0=4.}$\\
	TH1: $x_0=0$, 	$k=f'(0)=4$.\\
	Phương trình tiếp tuyến $y-0=4(x-0)$ hay $y=4x$.\\
	TH2: $x=4$, $k=f'(4)=-4$\\
	Phương trình tiếp tuyến $y-0=-4(x-4)$ hay $y=-4x+16$.
\end{enumerate}	
}
\end{bt}
\begin{bt}%[1D5Y1]
	Cho hàm số $y=x^{3}-3x^{2}+2(C)$. Viết phương trình tiếp tuyến của $(C)$:\begin{enumerate}
	\item Tại giao điểm của đồ thị hàm số với trục $Oy$
	\item Tại điểm có tung độ bằng $2$
	\item Tại điểm $M$ mà tiếp tuyến tại $M$ song song với đường thẳng $y=6x+1$
	\item Tiếp tuyến vuông góc với đường thẳng $y=\dfrac{-1}{9}x+3$ 
	\end{enumerate} 
	\loigiai{Ta tính được $y'(x)=3x^{2}-6x$.
	\begin{enumerate}
	\item $(C)$ cắt $Oy$ nên $x=0\Rightarrow y=2$. Vậy tiếp tuyến của $(C)$ tại điểm $A(0;2)$ là $y=y'(0)(x-0)+2\Rightarrow y=2$.
	\item Điểm trên $(C)$ có tung độ bằng $2\Rightarrow$ hoành độ là nghiệm của phương trình $x^{3}-3x^{2}+2=2\Rightarrow\hoac{x=0;y=2;A(0;2)\\x=3;y=2;B(3;2)}$\\
	$\Rightarrow$ phương trình tiếp tuyến $\hoac{y&=2\\y&=9x-24}$ 
	\item Tiếp tuyến song song với $y=6x+1\Rightarrow f'(x_0)=6$ với $x_0$ là hoành độ tiếp điểm.\\
	Giải phương trình ta có $\hoac{x_0=1+\sqrt3\\x_0=1-\sqrt3}\Rightarrow\hoac{y=6x-6-6\sqrt3\\y=6x-6+6\sqrt3}$ 
	\item Tiếp tuyến vuông góc với $y=-\dfrac{1}{9}x+3\Rightarrow f'(x_0)=9\Rightarrow\hoac{&x_0=3\\&x_0=-1}\Rightarrow\hoac{y=9x-25\\y=9x+20}$ 
	\end{enumerate} } 
\end{bt}
%%%%%%%%%
\begin{bt}%[1C7K1-2]
	Chứng minh rằng hàm số $f(x)=|x|$ không có đạo hàm tại điểm $x_0=0$, nhưng có đạo hàm tại mọi điểm $x\neq 0$.
	\loigiai{
	\begin{itemize}
	\item Chứng minh rằng hàm số $f(x)=|x|$ không có đạo hàm tại điểm $x_0=0$.\\
	Xét $\Delta x$ là số gia của biến số tại điểm $x_0=0$.\\
	Ta có: $\Delta y=f(0+\Delta x)-f(0)=|\Delta x|$.\\
	Suy ra: $\dfrac{\Delta y}{\Delta x}=\dfrac{|\Delta x|}{\Delta x}$.\\
	Vì: $\lim\limits_{\Delta x\to 0^+}\dfrac{\Delta y}{\Delta x}=1$ và $\lim\limits_{\Delta x\to 0^-}\dfrac{\Delta y}{\Delta x}=-1$.\\
	Nên hàm số $f(x)=|x|$ không có đạo hàm tại điểm $x_0=0$.
	\item Chứng minh rằng hàm số $f(x)=|x|$ có đạo hàm tại mọi điểm $x\neq0$.\\
	Xét $\Delta x$ là số gia của biến số tại điểm $x\neq0$.\\
	Ta có: 
	\begin{eqnarray*}
	\Delta y=&f(x+\Delta x)-f(x)\\
	=&|x+\Delta x|-|x|\\
	=&\sqrt{\left(x+\Delta x\right)^2}-\sqrt{x^2}\\
	=&\dfrac{\left(x+\Delta x\right)^2-x^2}{\sqrt{\left(x+\Delta x\right)^2}+\sqrt{x^2}}\\
	=&\dfrac{2x\Delta x+(\Delta x)^2}{\sqrt{\left(x+\Delta x\right)^2}+\sqrt{x^2}}.
	\end{eqnarray*}
	Suy ra: $\dfrac{\Delta y}{\Delta x}=\dfrac{2x+\Delta x}{\sqrt{\left(x+\Delta x\right)^2}+\sqrt{x^2}}$.\\
	Ta thấy: $\lim\limits_{\Delta x\to 0}\dfrac{\Delta y}{\Delta x}=\lim\limits_{\Delta x\to 0}\dfrac{2x+\Delta x}{\sqrt{\left(x+\Delta x\right)^2}+\sqrt{x^2}}=\dfrac{2x}{2\sqrt{x^2}}=\dfrac{x}{|x|}$.\\
	Vậy $f^\prime(x)=\dfrac{x}{|x|}$ với $x\neq0$.
	\end{itemize}
	}
\end{bt}
\begin{bt}%[1K9KU4]
	Một kỹ sư thiết kế một đường ray tàu lượn, mà mặt cắt của nó gồm một cung đường cong có dạng parabol, đoạn dốc lên $L_1$ và đoạn dốc xuống $L_2$ là những phần đường thẳng có hệ số góc lần lượt là $0{,}5$ và $-0{,}75$. Để tàu lượn chạy êm và không bị đổi hướng đột ngột, $L_1$ và $L_2$ phải là những tiếp tuyến của cung parabol tại các điểm chuyển tiếp $P$ và $Q$. Giả sử gốc tọa độ đặt tại $P$ và phương trình của paraol là $y=ax^2+bx+c$, trong đó $x$ tính bằng mét.
	\immini{
	\begin{enumerate}
	\item Tìm $c$.
	\item Tính $y'(0)$ và tìm $b$.
	\item Giả sử khoảng cách theo phương ngang giữa $P$ và $Q$ là $40$m.\\
	Tìm $a$.
	\item Tìm chênh lệch độ cao giữa hai điểm chuyển tiếp $P$ và $Q$.
	\end{enumerate}
	}{
	\begin{tikzpicture}[scale=0.5, font=\footnotesize, line join=round, line cap=round, >=stealth]
	\draw [->] (-3,0)--(8,0) node[below]{$x$};
	\draw [->] (0,-3)--(0,3) node[right]{$y$};
	\draw[fill=black] (0,0) circle(1pt) node[above left]{$P$};
	\draw [fill=black] (6,-3/5) circle(1pt) node[below left] {$Q$};
	\clip (-3,-3) rectangle (8,6);
	\draw[name path=c1,smooth, samples=300,dashed,color=red, domain=-2:0] plot (\x,{(1/2)*(\x)});
	\draw[name path=c2,smooth, samples=300, domain=0:6] plot (\x,{(-1)*(1/10)*(\x)^2+(1/2)*(\x)});
	\draw[name path=c3,smooth, samples=300,dashed,color=red, domain=6:8] plot (\x,{(-1)*(7/10)*(\x)+(18/5)});
	\draw (-1,-1) node[below]{$L_1$};
	\draw (6.4,-1.4) node[below]{$L_2$};
	\end{tikzpicture}
	}
	\loigiai{
	\begin{enumerate}
	\item Tìm $c$.\\
	Chọn hệ trục tọa độ như hình, đồ thì đi qua gốc tọa độ $P(0;0)$ nên $c=0$.
	\item Tính $y'(0)$ và tìm $b$.\\
	$y'=2ax^2+b$ nên $y'(0)=b$.\\
	Do hệ số góc tại $P$ bằng $0{,}5$ nên $y'(0)=0{,}5$ hay $b=0{,}5$.
	\item Giả sử khoảng cách theo phương ngang giữa $P$ và $Q$ là $40$ m. Tìm $a$.\\
	Khoảng cách theo phương ngang giữa $P$ và $Q$ là $40$ m nên $x_Q=40$.\\
	Hệ số góc tại $Q$ bằng $-0{,}75$ nên $y'(40)=-0{,}75$ hay $2a\cdot 40^2+0{,}5=-0{,}75 \Leftrightarrow a=\dfrac{-1}{2560}.$ 
	\item Tìm chênh lệch độ cao giữa hai điểm chuyển tiếp $P$ và $Q$.\\
	$y=-\dfrac{1}{2560}x^2+\dfrac{1}{2}x$.\\
	$y_Q=y(40)=\dfrac{155}{8}=19{,}375$ m.\\
	Vậy chênh lệch độ cao giữa $P$ và $Q$ là $19{,}375$ m.
	\end{enumerate}
	}
\end{bt}
\begin{bt}%[1C7K1-2]
	Giả sử chi phí $C$ (USD) để sản xuất $Q$ máy vô tuyến là $C(Q)=Q^2+80Q+3500$.
	\begin{enumerate}
	\item Ta gọi chi phí biên là chi phí gia tăng để sản xuất thêm $1$ sản phẩm từ $Q$ sản phẩm lên $Q+1$ sản phẩm. Giả sử chi phí biên được xác định bởi hàm số $C^\prime(Q)$. Tìm hàm chi phí biên.
	\item Tìm $C^\prime(90)$ và giải thích ý nghĩa kết quả tìm được.
	\item Hãy tính chi phí sản xuất máy vô tuyến thứ $100$.
	\end{enumerate}
	\loigiai{
	\begin{enumerate}
	\item Xét $\Delta Q$ là số gia của biến số tại điểm $Q$.\\
	Ta có: 
	$$\begin{aligned}[t]
	\Delta Q=&C(Q+\Delta Q)-C(Q)\\
	=&\left(Q+\Delta Q\right)^2+80\left(Q+\Delta Q\right)+3500-\left(Q^2+80Q+3500\right)\\
	=&2Q\cdot(\Delta Q)+(\Delta Q)^2+80\Delta Q.
	\end{aligned}$$
	Suy ra: $\dfrac{\Delta y}{\Delta Q}=2Q+\Delta Q+80$.\\
	Ta thấy: $\lim\limits_{\Delta Q\to 0}\left(2Q+\Delta Q+80\right)=2Q+80$.\\
	Vậy $C^\prime(Q)=2Q+80$.
	\item $C^\prime(90)=2\cdot90+80=260$.\\
	Chi phí gia tăng để sản xuất từ $90$ sản phẩm máy vô tuyến lên $91$ máy vô tuyến là $260$ USD.
	\item Chi phí để sản xuất máy vô tuyến thứ $100$ là 
	\begin{center}
	$C(100)=100^2+80\cdot100+3500=21~500$ (USD).
	\end{center}
	Chi phí gia tăng để sản xuất từ $99$ sản phẩm máy vô tuyến lên $100$ máy vô tuyến là
	\begin{center}
	$C^\prime(99)=2\cdot99+80=278$ (USD).
	\end{center}
	Vậy tổng chi phí để sản xuất máy vô tuyến thứ $100$ là
	\begin{center}
	$21~500+278=21~778$ (USD).
	\end{center}
	\end{enumerate}
	}
\end{bt}	