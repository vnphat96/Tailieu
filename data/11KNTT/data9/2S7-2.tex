\setcounter{dang}{0}
\section{CÁC QUY TẮC TÍNH ĐẠO HÀM}
\subsection{Tóm tắt kiến thức}
% \subsubsection{Đạo hàm của một số hàm số thường gặp}
% \paragraph{Đạo hàm của hàm số \boldmath $y=x^n$ $\left(n\in \mathbb{N}^*\right)$}
% \begin{dl}
% 	Hàm số $y=x^n$ $\left(n\in \mathbb{N}\right)$ có đạo hàm trên $\mathbb{R}$ và $\left(x^n\right)'=nx^{n-1}$.
% \end{dl}
% \paragraph{Đạo hàm của hàm số \boldmath $y=\sqrt{x}$}
% \begin{dl}
% 	Hàm số $y=\sqrt{x}$ có đạo hàm trên khoảng $\left(0;+\infty\right)$ và $\left(\sqrt{x}\right)'=\dfrac{1}{2\sqrt{x}}$.
% \end{dl}
\subsubsection{Đạo hàm của tổng hiệu tích thương}
\begin{dl}
Giả sử các hàm số $u=u(x)$, $v=v(x)$ có đạo hàm trên khoảng $\left(a;b\right)$. Khi đó
\begin{enumEX}[\itemCI]{2}
	\item $\left(u+v\right)'=u'+v'$;	
	\item $\left(u-v\right)'=u'-v'$;
	\item $\left(u\cdot v\right)'=u'\cdot v+u\cdot v'$;	
	\item $\left(\dfrac{u}{v}\right)'=\dfrac{u'\cdot v-u\cdot v'}{v^2}$ $\left(v=v(x)\neq 0\right)$. 
\end{enumEX}
\end{dl}
\begin{note}
	\begin{itemize}
	% \item Quy tắc tính đạo hàm của tổng, hiệu có thể áp dụng cho tổng, hiệu của hai hay nhiều hàm số.
	\item Với $k$ là một hằng số, ta có $\left(ku\right)'=k\cdot u'$.
	\item Đạo hàm của hàm số nghịch đảo: $\left(\dfrac{1}{v}\right)'=-\dfrac{v'}{v^2}$ $\left(v=v(x)\neq 0\right)$.
	\end{itemize}
\end{note}
\subsubsection{Đạo hàm của hàm số hợp}
\paragraph{Khái niệm hàm số hợp}
\begin{dl}
	Giả sử $u=g(x)$ là hàm số xác định trên khoảng $(a;b)$, có tập giá trị chứa trong khoảng $(c;d)$ và $y=f(u)$ là hàm số xác định trên khoảng $(c;d)$. Hàm số $y=f(g(x))$ được gọi là hàm số hợp của hàm số $y=f(u)$ với $u=g(x)$.
\end{dl}
% \begin{center}
% 	\begin{tikzpicture}[scale=1,>=stealth,cham/.style={circle,fill,inner sep=1pt}]
% 	\path (.5,0) coordinate (a) (1.3,0) coordinate (x) (2.2,0) coordinate (b) 
% 	(4.3,0) coordinate (c) (5.3,0) coordinate (u) node(u)[cham]{} node[below,xshift=-5pt]{$u=g(x)$}
% 	(6,0) coordinate (d) (9,0) coordinate (y) node(y)[cham]{} node[below,xshift=2pt]{$y=f(u)$};
% 	\draw[thick,->](0,0)--(2.8,0);
% 	\draw[thick,->](3.8,0)--(6.8,0);
% 	\draw[thick,->](8,0)--(10,0);
% 	\foreach \i/\j in {a/-90,x/-110,b/-60,c/100,d/70} {
% 	\path (\i) node(\i)[cham]{}	node[shift={(\j:.25)},scale=1]{$\i$};}
% 	\draw[red, ->] (x) to[out=45,in=135] node[above,black]{$g$}(u);
% 	\draw[red, ->] (u) to[out=45,in=135] node[above,black]{$f$}(y);
% 	\draw[red, dashed,->] (x) to[out=-25,in=-155] node[below,black]{$y=f(g(x))$}(y);
% 	\end{tikzpicture}
% \end{center}
\paragraph{Đạo hàm của hàm số hợp}
\begin{dl}
	Nếu hàm số $u=g(x)$ có đạo hàm $u_x'$ tại $x$ và hàm số $y=f(u)$ có đạo hàm $y_u'$ tại $u$ thì hàm số hợp $y=f(g(x))$ có đạo hàm $y_x'$ tại $x$ là
	$$y_x'=y_u'\cdot u_x'.$$
\end{dl}
% \subsubsection{Đạo hàm của hàm số lượng giác}
% \paragraph{Đạo hàm của hàm số $y=\sin x$}
% \begin{dl}
% \begin{itemize}
% 	\item Hàm số $y=\sin x$ có đạo hàm trên $\mathbb{R}$ và $\left(\sin x\right)'=\cos x$.
% 	\item Đối với hàm số hợp $y=\sin u$, với $u=u(x)$, ta có $(\sin u)'=u'\cdot \cos u$. 
% \end{itemize}
% \end{dl}
% \paragraph{Đạo hàm của hàm số $y=\cos x$}
% \begin{dl} 
% 	\begin{itemize}
% 	\item Hàm số $y=\cos x$ có đạo hàm trên $\mathbb{R}$ và $\left(\cos x\right)'=-\sin x$.
% 	\item Đối với hàm số hợp $y=\cos u$, với $u=u(x)$, ta có $(\cos u)'=-u'\cdot \sin u$. 
% 	\end{itemize}
% \end{dl}
% \paragraph{Đạo hàm của hàm số $y=\tan x$ và $y=\cot x$}
% \begin{dl} 
% 	\begin{itemize}
% 	\item Hàm số $y=\tan x$ có đạo hàm tại mọi $x\neq \dfrac{\pi}{2}+k\pi\,\left(k\in \mathbb{Z}\right)$ và $\left(\tan x\right)'=\dfrac{1}{\cos ^2 x}.$
% 	\item Hàm số $y=\cot x$ có đạo hàm tại mọi $x\neq k\pi\,\left(k\in \mathbb{Z}\right)$ và $\left(\cot x\right)'=-\dfrac{1}{\sin ^2 x}.$
% 	\item Đối với hàm số hợp $y=\tan u$ và $y=\cot u$ với $u=u(x)$, ta có 
% 	$$(\tan u)'=\dfrac{u'}{\cos^2u}; \, (\cot u)'=-\dfrac{u'}{\sin^2u}$$
% 	(giả thiết $\tan u$ và $\cot u$ có nghĩa).
% 	\end{itemize}
% \end{dl}
% \subsubsection{Đạo hàm của hàm số mũ và hàm số lôgarit}
% \paragraph{Giới hạn liên quan đến hàm số mũ và hàm số lôgarit}
% \begin{nx}
% 	Ta có các giới hạn sau:$$\lim\limits_{x \to 0}\left(1+x\right)^{\frac{1}{x}}=\mathrm{e};\,\lim\limits_{x \to 0}\dfrac{\ln (1+x)}{x}=1;\,\lim\limits_{x \to 0}\dfrac{\mathrm{e}^x-1}{x}=1.$$
% \end{nx}
% \paragraph{Đạo hàm của hàm số mũ}
% \begin{dl}
% 	\begin{itemize}
% 	\item Hàm số $y=\mathrm{e}^x$ có đạo hàm trên $\mathbb{R}$ và $\left(\mathrm{e}^x\right)'=\mathrm{e}^x$.\\
% 	Đối với hàm số hợp $y=\mathrm{e}^u$, với $u=u(x)$, ta có $\left(\mathrm{e}^u\right)'=u'\cdot \mathrm{e}^u$.
% 	\item Hàm số $y=a^x\,\left(0<a\neq 1\right)$ có đạo hàm trên $\mathbb{R}$ và $\left(a^x\right)'=a^x\cdot \ln a$.\\
% 	Đối với hàm số hợp $y=a^u$, với $u=u(x)$, ta có $\left(a^u\right)'=u'\cdot a^u\cdot \ln a$.
% 	\end{itemize}
% \end{dl}
% \paragraph{Đạo hàm của hàm số logarit}
% \begin{dl}
% 	\begin{itemize}
% 	\item Hàm số $\ln x$ có đạo hàm trên khoảng $\left(0;+\infty\right)$ và $\left(\ln x\right)'=\dfrac{1}{x}$.\\
% 	Đối với hàm số hợp $y=\ln u$, với $u=u(x)$, ta có $\left(\ln u\right)'=\dfrac{u'}{u}$.
% 	\item Hàm số $\log_a x$ có đạo hàm trên khoảng $\left(0;+\infty\right)$ và $\left(\log_a x\right)'=\dfrac{1}{x\ln a}$.\\
% 	Đối với hàm số hợp $y=\log_a x$, với $u=u(x)$, ta có $\left(\log_a x\right)'=\dfrac{u'}{u\ln a}$.
% 	\end{itemize}
% \end{dl} 
% \begin{note}
% 	Với $x<0$, ta có: 
% 	$$\ln |x|=\ln (-x) \text{ và } [\ln (-x)]'=\dfrac{(-x)'}{-x}=\dfrac{1}{x}.$$
% 	Từ đó ta có:
% 	$$
% 	(\ln |x|)'=\dfrac{1}{x}, \forall x \neq 0
% 	$$
% \end{note}
\subsubsection*{Bảng đạo hàm}
\begin{center}
	\renewcommand{\arraystretch}{2}
	\begin{longtable}{|C{0.45\textwidth}|C{0.45\textwidth}|}
	\hline
	\centerline{\bf Đạo hàm của hàm số sơ cấp cơ bản}& \centerline{\bf Đạo hàm của hàm hợp} \\ [-.5cm]
	\centerline{\bf thường gặp} & \centerline{\bf(ở đây $\mathbf{u=u(x)}$)}\\
	\hline
	$\left(x^n\right)'=n\cdot x^{n -1}$ & $\left(u^n\right)'=n\cdot u^{n-1}\cdot u'$\\
	\hline
	$\left(\dfrac{1}{x}\right)'=-\dfrac{1}{x^2}$& $\left(\dfrac{1}{u}\right)'=-\dfrac{u'}{u^2}$\\
	\hline
	$\left(\sqrt{x}\right)'=\dfrac{1}{2\sqrt{x}}$&
	$\left(\sqrt{u}\right)'=\dfrac{u'}{2\sqrt{u}}$\\
	\hline
	$\left(\sin x\right)'=\cos x$& $\left(\sin u\right)'=u'\cdot \cos u$\\
	\hline
	$\left(\cos x\right)'=-\sin x$& $\left(\cos u\right)'=-u'\cdot\sin u $ \\
	\hline
	$\left(\tan x\right)'=\dfrac{1}{\cos^2 x}$&$\left(\tan u\right)'=\dfrac{u'}{\cos^2 u}$ \\
	\hline
	$\left(\cot x\right)'=-\dfrac{1}{\sin^2 x}$&$\left(\cot u\right)'=-\dfrac{u'}{\sin^2 u}$ \\
	\hline
	$\left(\mathrm{e}^x\right)'=\mathrm{e}^{x}$& $\left(\mathrm{e}^{u}\right)'=u'\cdot \mathrm{e}^{u}$ \\
	\hline
	$\left(a^{x}\right)'=a^{x}\ln a$&$\left(a^{u}\right)=u'\cdot a^{u}\ln a$ \\
	\hline
	$\left(\ln x\right)'=\dfrac{1}{x}$& $\left(\ln u\right)'=\dfrac{u'}{u}$ \\
	\hline
	$\left(\log_{a}x\right)' =\dfrac{1}{x\ln a}$& $\left(\log_{a} u\right)'=\dfrac{u'}{u\ln a}$\\
	\hline
	\end{longtable}
\end{center}
\subsubsection{Đạo hàm cấp hai}
\paragraph{Khái niệm đạo hàm cấp 2}
\begin{dn}
	Giả sử hàm số $y=f(x)$ có đạo hàm tại mỗi điểm $x\in (a;b)$. Nếu hàm số $y'=f'(x)$ lại có đạo hàm tại $x$ thì ta gọi đạo hàm của $y'$ là đạo hàm cấp hai của hàm số $y=f(x)$ tại $x$. Kí hiệu là $y''(x)$.
\end{dn}
\paragraph{Ý nghĩa cơ học của đạo hàm cấp hai}
\begin{dn}
	Một chuyển động có phương trình $s=f(t)$ thì đạo hàm cấp hai (nếu có) của hàm số $f(t)$ là gia tốc tức thời của chuyển động. Ta có $$a(t)=f''(t).$$
\end{dn}
%%%%%%%%%%%%%%%%%%%%
\subsection{Các dạng bài tập}
\begin{dang}{Tính đạo hàm cơ bản}
\end{dang}
\subsubsection{Ví dụ minh hoạ}
\begin{vd}%[1C7Y2-1]
	Tính đạo hàm của hàm số $f(x)=\sin x$ tại điểm $x_0=\dfrac{\pi}{3}$.
	\loigiai{
	Ta có $f'(x)=\cos x$.\\	
	Đạo hàm của hàm số trên tại điểm $x_0=\dfrac{\pi}{3}$ là
	$f'\left(\dfrac{\pi}{3}\right)=\cos \dfrac{\pi}{3}=\dfrac{1}{2}$.	
	}
\end{vd}
\begin{vd}%[1C7Y2-1]
	Tính đạo hàm của hàm số $f(x)=\tan x$ tại điểm $x_0=\dfrac{\pi}{4}$.	
	\loigiai{
	Ta có $f'(x)=\dfrac{1}{\cos^2 x}$ $\left(x \neq \dfrac{\pi}{2}+k \pi,\, k \in \mathbb{Z}\right)$.\\	
	Đạo hàm của hàm số trên tại điểm $x_0=\dfrac{\pi}{4}$ là
	$f'\left(\dfrac{\pi}{4}\right)=\dfrac{1}{\cos^2\left(\dfrac{\pi}{4}\right)}=2$.	
	}
\end{vd}
\begin{vd}%[1C7Y2-1]
	Tính đạo hàm của hàm số $f(x)=\cot x$ tại điểm $x_0=\dfrac{\pi}{2}$.	
	\loigiai{
	Ta có $f'(x)=-\dfrac{1}{\sin^2 x}$ $\left(x \neq k \pi, k \in \mathbb{Z}\right)$.\\
	Đạo hàm của hàm số trên tại điểm $x_0=\dfrac{\pi}{2}$ là: $f'\left(\dfrac{\pi}{2}\right)=-\dfrac{1}{\sin^2\left(\dfrac{\pi}{2}\right)}=-1$.	
	}
\end{vd}
\begin{vd}%[1T7Y2-1]
	Tính đạo hàm của hàm số $y=x^5$ tại điểm $x=2$ và $x=-\dfrac{1}{2}$.
	\loigiai{Ta có $\left(x^5\right)^\prime =5 x^4$. Từ đó, $y^\prime \left(2\right)=5\cdot 2^4=80$ và $y^\prime \left(-\dfrac{1}{2}\right)=5\cdot \left(-\dfrac{1}{2}\right) ^4=\dfrac{5}{16}$.
	}
\end{vd}
\begin{vd}%[1T7Y2-1]
	Tính đạo hàm của hàm số $y=\sqrt{x}$ tại điểm $x=1$ và $x=\dfrac{1}{4}$.
	\loigiai{Ta có $y^\prime=\left(\sqrt{x}\right)^\prime =\dfrac{1}{2\sqrt{x}}$, $x>0$. Từ đó, $y^\prime \left(1\right)=\dfrac{1}{2\sqrt{1}}=\dfrac{1}{2}$ và $y^\prime \left(\dfrac{1}{4}\right)=\dfrac{1}{2\sqrt{\dfrac{1}{4}}}=\dfrac{1}{2\cdot \dfrac{1}{2}}=1$.
	}
\end{vd}
\begin{vd}%[1T7Y2-1]
	Tìm đạo hàm của các hàm số
	\begin{listEX}[2]
	\item $y=\sqrt[4]{x}$ tại $x=1$;
	\item $y=\dfrac{1}{x}$ tại $x=-\dfrac{1}{4}$.
	\end{listEX}
	\loigiai{
	\begin{listEX}
	\item $y=\sqrt[4]{x}$ tại $x=1$.\\
	Ta có $y^\prime =\left(\sqrt[4]{x}\right)^\prime=\left(x^{\frac{1}{4}}\right)^\prime=\dfrac{1}{4}\cdot x^ {\frac{1}{4}-1}=\dfrac{1}{4}\cdot x^ {-\frac{3}{4}}=\dfrac{1}{4\sqrt[4]{x^3}}$.
	Từ đó, $y^\prime(1)=\dfrac{1}{4\sqrt[4]{1^3}}=\dfrac{1}{4}$.
	\item $y=\dfrac{1}{x}$ tại $x=-\dfrac{1}{4}$.\\
	Ta có $y^\prime =\left(\dfrac{1}{x}\right)^\prime=-\dfrac{1}{x^2}$. Từ đó, $y^ \prime \left(-\dfrac{1}{4}\right)=-\dfrac{1}{\left(-\frac{1}{4}\right)^2}=-16$.
	\end{listEX}
	}
\end{vd}
\begin{vd}%[1T7Y2-1]
	Tìm đạo hàm của các hàm số 
	\begin{listEX}[2]
	\item $y=9^x$ tại $x=1$;
	\item $y=\ln x$ tại $x=\dfrac{1}{3}$.
	\end{listEX}
	\loigiai{Tìm đạo hàm của các hàm số 
	\begin{listEX}
	\item $y=9^x$ tại $x=1$.\\
	Ta có $y'=\left(9^ x\right)^\prime =9^x \cdot \ln 9$. Từ đó $y'(1)=9^1\cdot \ln 9=18\ln 3$.
	\item $y=\ln x$ tại $x=\dfrac{1}{3}$.\\
	Ta có $y'=\left(\ln x\right)^\prime =\dfrac{1}{x}\ \left(x>0\right)$. Từ đó $y'\left(\dfrac{1}{3}\right)=\dfrac{1}{\frac{1}{3}}=3$.
	\end{listEX}
	}
\end{vd}
\begin{vd}%[1T7Y2-1]
	Tìm đạo hàm của các hàm số 
	\begin{listEX}[2]
	\item $y=\mathrm{e}^x$ tại $x=2\ln 3$;
	\item $y=\log_5x $ tại $x=2$.
	\end{listEX}
	\loigiai{Tìm đạo hàm của các hàm số 
	\begin{listEX}[2]
	\item $y=\mathrm{e}^x$ tại $x=2\ln 3$.\\
	Ta có $y'=\left(\mathrm{e}^x\right)^\prime =\mathrm{e}^x$. Từ đó, $y^\prime\left(2\ln 3\right)=\mathrm{e}^{2\ln 3}=\left(\mathrm{e}^{\ln 3}\right)^2=3^2=9$.
	\item $y=\log_5x $ tại $x=2$.\\
	Ta có $y'=\left( \log_5 x \right)^\prime =\dfrac{1}{x \cdot \ln 5 } \ \left(x>0\right)$.
	Từ đó $y'(2)=\dfrac{1}{2 \cdot \ln 5 }$.
	\end{listEX}}
\end{vd}
\subsubsection{Bài tập áp dụng}
\begin{bt}%[1C7Y2-1]
	Tính đạo hàm của hàm số $f(x)=\cos x$ tại điểm $x_0=\dfrac{\pi}{6}$.
	\loigiai{
	Ta có $f'(x)=-\sin x$.\\	
	Đạo hàm của hàm số trên tại điểm $x_0=\dfrac{\pi}{6}$ là
	$f'\left(\dfrac{\pi}{6}\right)=-\sin \dfrac{\pi}{6}=-\dfrac{1}{2}$.	
	}
\end{bt}
\begin{bt}%[1T7Y2-1]
	Tính đạo hàm của hàm số $y=\tan x$ tại $x=\dfrac{3\pi}{4}$.
	\loigiai{Ta có $y^\prime= \left(\tan x\right)^\prime =\dfrac{1}{\cos^2x}\ \left(x \neq \dfrac{\pi}{2}+k\pi, k \in \mathbb{Z}\right)$. Vậy $y^\prime\left(\dfrac{3\pi}{4}\right)= \dfrac{1}{\cos^2 \left(\frac{3\pi}{4}\right)}=2$.
	}
\end{bt}
\begin{bt}%[1T7Y2-1]
	Tính đạo hàm của hàm số $y=x^{10}$ tại điểm $x=-1$ và $x=\sqrt[3]{2}$.
	\loigiai{Ta có $\left(x^{10}\right)^\prime =10 x^9$. Từ đó, $y^\prime \left(-1\right)=10\cdot \left(-1\right)^9=-10$ và $y^\prime \left(\sqrt[3]{2}\right)=10\cdot \left(\sqrt[3]{2}\right) ^9=80$.
	}
\end{bt}
\begin{bt}%[1K9YV-1]
	Tính đạo hàm của hàm số $y=\sqrt{x}$ tại các điểm $x=4$ và $x=\dfrac{1}{9}$.
	\loigiai{
	Với mọi $x\in \left(0;+\infty\right)$, ta có $y'=\dfrac{1}{2\sqrt{x}}$. Do đó, $y'(4)=\dfrac{1}{2\sqrt{4}}=\dfrac{1}{4}$ và $y'\left(\dfrac{1}{9}\right)=\dfrac{1}{2\sqrt{\dfrac{1}{9}}}=\dfrac{3}{2}$.
	}
\end{bt}
\begin{bt}%[1T7Y2-1]
	Tìm đạo hàm của các hàm số
	\begin{listEX}[2]
	\item $y=\sqrt[3]{x}$ tại điểm $x=8$;
	\item $y=\dfrac{2}{x}$ tại $x=\dfrac{1}{5}$.
	\end{listEX}
	\loigiai{
	\begin{listEX}[1]
	\item
	Ta có $y'=\left(\sqrt[3]{x}\right)'=\left(x^{\frac{1}{3}}\right)^\prime=\dfrac{1}{3}\cdot x^{\frac{1}{3}-1}=\dfrac{1}{3}\cdot x^{-\frac{2}{3}}=\dfrac{1}{3\sqrt[3]{x^2}}$.\\
	Từ đó, $y'(8)=\dfrac{1}{3\sqrt[3]{8^2}}=\dfrac{1}{12}$.
	\item
	Ta có $y^\prime =\left(\dfrac{2}{x}\right)^\prime=-\dfrac{2}{x^2}$. Từ đó, $y^ \prime \left(\dfrac{1}{5}\right)=-\dfrac{2}{\left(\frac{1}{5}\right)^2}=-50$.
	\end{listEX}
	}
\end{bt}
\begin{bt}%[1C7Y2-1]
	Tính đạo hàm của hàm số $f(x)=2^x$ tại điểm $x_0=1$.
	\loigiai{
	Ta có $f'(x)=2^x \ln 2$.\\	
	Đạo hàm của hàm số trên tại điểm $x_0=1$ là
	$f'(1)=2^1 \ln 2=2 \ln 2$.	
	}
\end{bt}
\begin{bt}%[1C7Y2-1]
	Tính đạo hàm của hàm số $f(x)=\ln x$ tại điểm $x_0=1$.
	\loigiai{
	Ta có $f'(x)=\dfrac{1}{x}$ $(x>0)$.\\	
	Đạo hàm của hàm số trên tại điểm $x_0=1$ là $f'(1)=\dfrac{1}{1}=1$.	
	}
\end{bt}
\begin{bt}%[1C7Y2-1]
	Cho hàm số $f(x)=x^{10}$.
	\begin{enumerate}
	\item Tính đạo hàm của hàm số trên tại điểm $x$ bất kì.
	\item Tính đạo hàm của hàm số trên tại điểm $x_0=1$.	
	\end{enumerate}
	\loigiai{
	\begin{enumerate}
	\item Ta có $f'(x)=\left(x^{10}\right)'=10 x^9$.
	\item Đạo hàm của hàm số tại điểm $x_0=1$ là $f'(1)=10\cdot 1^9=10$.	
	\end{enumerate}	
	}
\end{bt}
%-----------
\begin{dang}{Tính đạo hàm hàm hợp}
\end{dang}
\subsubsection{Ví dụ minh hoạ}
\begin{vd}%[1T7B2-1]
	Tính đạo hàm của các hàm số sau:
	\begin{listEX}[4]
	\item $y=\left(2x^3+3\right)^2$;
	\item $y=\cos 3x$;
	\item $y=\log_2\left(x^2+2\right)$;
	\item $y=\mathrm{e}^{x^2+1}$.
	\end{listEX}
	\loigiai{
	\begin{enumerate}
	\item $y=\left(2x^3+3\right)^2$.\\
	Đặt $u=2x^3+3$ thì $y=u^2$. Ta có $u'_x=6x$ và $y'_u
	=2u$.\\
	Suy ra $y'_x=y'_u\cdot u'_x=2u\cdot 6x=2\left(2x^3+3\right)\cdot 6x =12x\left(2x^3+3\right)$.\\
	Vậy $y'=12x\left(2x^3+3\right)$.
	\item $y=\cos 3x$.\\
	Đặt $u=3x$ thì $y=\cos u$. Ta có $u'_x=3$ và $y'_u
	=-\sin u$.\\
	Suy ra $y'_x=y'_u\cdot u'_x=\left(-\sin u\right)\cdot 6=\left(-\sin 3x\right)\cdot 3 =-3\sin 3x$.\\
	Vậy $y'=-3\sin 3x$.
	\item $y=\log_2\left(x^2+2\right)$.\\
	Đặt $u=x^2+2$ thì $y=\log_2 u$. Ta có $u'_x=2x$ và $y'_u
	=\dfrac{1}{u\ln 2}$.\\
	Suy ra $y'_x=y'_u\cdot u'_x=\dfrac{1}{u\ln 2} \cdot 2x=\dfrac{1}{\left(x^2+2\right)\ln 2}\cdot 2x=\dfrac{2x}{\left(x^2+2\right)\ln 2}$.\\
	Vậy $y'=\dfrac{2x}{\left(x^2+2\right)\ln 2}$.
	\item $y=\mathrm{e}^{x^2+1}$.\\
	Đặt $u=x^2+1$ thì $y=\mathrm{e}^u$. Ta có $u'_x=2x$ và $y'_u
	=\mathrm{e}^u$.\\
	Suy ra $y'_x=y'_u\cdot u'_x=\mathrm{e}^u\cdot 2x=\mathrm{e}^{x^2+1} \cdot 2x =2x\mathrm{e}^{x^2+1}$.\\
	Vậy $y'=2x\mathrm{e}^{x^2+1}$.
	\end{enumerate}}
\end{vd}
\begin{vd}%[1K9BV-1]
	Tính đạo hàm của hàm số sau
	\begin{listEX}[4]
	\item $y=\sin \left(2x+\dfrac{\pi}{8}\right)$;
	\item $y=\cos \left(4x-\dfrac{\pi}{3}\right)$;
	\item $y=\tan \left(2x+\dfrac{\pi}{4}\right)$;
	\item $y=\cot \left(3x-\dfrac{\pi}{6}\right)$.
	\end{listEX}
	\loigiai{
	\begin{listEX}[1]
	\item Ta có $y'=\left(2x+\dfrac{\pi}{8}\right)'\cdot \cos \left(2x+\dfrac{\pi}{8}\right)=2 \cos \left(2x+\dfrac{\pi}{8}\right).$
	\item Ta có $y'=-\left(4x-\dfrac{\pi}{3}\right)'\cdot \sin \left(4x-\dfrac{\pi}{3}\right)=-4\sin \left(4x-\dfrac{\pi}{3}\right).$
	\item Ta có $y'=\dfrac{\left(2x+\dfrac{\pi}{4}\right)'}{\cos^2\left(2x+\dfrac{\pi}{4}\right)}=\dfrac{2}{\cos^2\left(2x+\dfrac{\pi}{4}\right)}.$
	\item Ta có $y'=-\dfrac{\left(3x-\dfrac{\pi}{6}\right)'}{\sin^2\left(3x-\dfrac{\pi}{6}\right)}=\dfrac{-3}{\sin^2\left(3x-\dfrac{\pi}{6}\right)}.$
	\end{listEX}
	}
\end{vd}
\begin{vd}%[1K9BV-1]
	Tính đạo hàm của hàm số sau
	\begin{listEX}[2]
	\item $y=\sqrt{x^2+1}$;
	\item $\dfrac{1}{2x-3}$.
	\end{listEX}
	\loigiai{
	\begin{listEX}[1]
	\item Ta có $y'=\dfrac{\left(x^2+1\right)'}{2\sqrt{x^2+1}}=\dfrac{2x}{2\sqrt{x^2+1}}=\dfrac{x}{\sqrt{x^2+1}}$.
	\item $y'=\left(\dfrac{1}{2x-3}\right)=\dfrac{-(2x-3)'}{(2x-3)^2}=\dfrac{-2}{(2x-3)^2}$.
	\end{listEX}
	}
\end{vd}
\begin{vd}%[1D5B3]
	Tính đạo hàm của các hàm số sau:
	\begin{listEX}[2]
	\item $y=2\tan^{2}{x}$.
	\item $y=3\cot^{3}{x}$.
	\end{listEX}
	\loigiai{
	\begin{enumerate}
	\item $y'=(2\tan^{2}{x})'=2\cdot 2\tan{x}\cdot (\tan{x})'=4\tan{x}\cdot \dfrac{1}{\cos^{2}{x}}=\dfrac{4\tan{x}}{\cos^{2}{x}}$.
	\item $y'=(3\cot^{3}{x})=3\cdot 3\cot^{2}{x}\cdot (\cot{x})'\\=9\cot^2{x}\cdot \dfrac{-1}{\sin^{2}{x}}=\dfrac{-9\cot^2{x}}{\sin^{2}{x}}$.
	\end{enumerate}
	}
\end{vd}
\begin{vd}%[1D5B3]
	Tính đạo hàm của các hàm số sau:
	\begin{listEX}[2]
	\item $y=3\sin{x}+\cos{x}$.
	\item $y=4\sin{x}-5\cos{x}$.
	\end{listEX}
	\loigiai{
	\begin{enumerate}
	\item $y'=(3\sin{x}+\cos{x})'=3\cdot (\sin{x})'+(\cos{x})'\\=3\cdot \cos{x}+(-\sin{x})=3\cos{x}-\sin{x}$.
	\item $y'=(4\sin{x}-5\cos{x})'=4\cdot (\sin{x})'-5\cdot (\cos{x})'\\=4\cdot \cos{x}-5\cdot (-\sin{x})=4\cos{x}+5\sin{x}$.
	\end{enumerate}
	}
\end{vd}
\begin{vd}%[1D5B3]
	Tính đạo hàm của các hàm số $y=\tan{3x}+2\tan{x}$.
	\loigiai{
 $y'=(\tan{3x}+2\tan{x})'=(\tan{3x})'+(2\tan{x})'\\=\dfrac{(3x)'}{\cos^{2}{3x}}+2\cdot \dfrac{1}{\cos^{2}{x}}=\dfrac{3}{\cos^{2}{3x}}+\dfrac{2}{\cos^{2}{x}}$.
	}
\end{vd}
\begin{vd}%[1D5B3]
	Tính đạo hàm của các hàm số $y=\cot{5x}\cos{4x}$.
	\loigiai{
$y'=(\cot{5x}\cos{4x})'=(\cot{5x})'\cdot \cos{4x}+\cot{5x}\cdot (\cos{4x})'\\=\dfrac{-5}{\sin^{2}{5x}}\cdot \cos{4x}+\cot{5x}\cdot (-4\sin{4x})\\[0.3cm]=\dfrac{-5\cos{4x}}{\sin^{2}{5x}}-4\cot{5x}\sin{4x}$.
	}
\end{vd}
\subsubsection{Bài tập áp dụng}
\begin{bt}%[1T7B2-1]
	Tính đạo hàm của các hàm số sau:
	\begin{listEX}[4]
	\item $y=(3x^2+x)^3$;
	\item $y=\sin 2x$;	
	\item $y=\ln \left(x^2+1\right)$;
	\item $y=2^{x^2-x}$.	
	\end{listEX}
	\loigiai{\begin{enumerate}
	\item $y=(3x^2+x)^3$.\\
	Đặt $u=3x^2+x$ thì $y=u^3$. Ta có $u'_x=6x+1$ và $y'_u=3u^2$.\\
	Suy ra $y'_x=y'_u\cdot u'_x=3u^2\cdot (6x+1)=3\left(3x^2+x\right)^2(6x+1)$.\\
	Vậy $y'=3\left(3x^2+x\right)^2(6x+1)$.
	\item $y=\sin 2x$.\\
	Đặt $u=2x$ thì $y=\sin u$. Ta có $u'_x=2$ và $y'_u=\cos u$.\\
	Suy ra $y'_x=y'_u\cdot u'_x=\cos u\cdot 2=\cos 2x\cdot 2 =2\cos 2x$.\\
	Vậy $y'=2\cos 2x$.
	\item Ta có $y'=\dfrac{\left(x^2+1\right)'}{\left(x^2+1\right)}=\dfrac{2x}{\left(x^2+1\right)}.$
	\item Ta có $y'=\left(x^2-x\right)'\cdot 2^{x^2-x}\cdot \ln 2=(2x-1)\ln 2\cdot 2^{x^2-x}.$
	\end{enumerate}}
\end{bt}
\begin{bt}%[1K9BV-1]
	Tính đạo hàm của hàm số sau
	\begin{listEX}[4]
	\item $y=\sin \left(10x-5\right)$;
	\item $y=\cos \left(3-x\right)$;
	\item $y=\tan \left(5x+7\right)$;
	\item $y=\cot \left(4-2x\right)$.
	\end{listEX}
	\loigiai{
	\begin{listEX}[1]
	\item Ta có $y'=\left(10x-5\right)'\cdot \cos \left(10x-5\right)=10 \cos \left(10x-5\right).$
	\item Ta có $y'=-\left(3-x\right)'\cdot \sin \left(3-x\right)=\sin \left(3-x\right).$
	\item Ta có $y'=\dfrac{\left(5x+7\right)'}{\cos^2\left(5x+7\right)}=\dfrac{5}{\cos^2\left(5x+7\right)}.$
	\item Ta có $y'=-\dfrac{\left(4-2x\right)'}{\sin^2\left(4-2x\right)}=\dfrac{2}{\sin^2\left(4-2x\right)}.$
	\end{listEX}
	}
\end{bt}
\begin{bt}%[1K9BV-1]
	Tính đạo hàm của hàm số sau
	\begin{listEX}[2]
	\item $y=\sqrt{3x-5}$;
	\item $\dfrac{1}{3-x}$.
	\end{listEX}
	\loigiai{
	\begin{listEX}[1]
	\item Ta có $y'=\dfrac{\left(3x-5\right)'}{2\sqrt{3x-5}}=\dfrac{3}{2\sqrt{3x-5}}$.
	\item $y'=\left(\dfrac{1}{3-x}\right)'=\dfrac{-(3-x)'}{(3-x)^2}=\dfrac{1}{(3-x)^2}$.
	\end{listEX}
	}
\end{bt}
\begin{bt}%[1D5B3]
	Tính đạo hàm của các hàm số sau:
	\begin{listEX}[2]
	\item $y=\tan^{2}{3x}$.
	\item $y=\cot^{3}{4x}$.
	\end{listEX}
	\loigiai{
	\begin{enumerate}
	\item $y'=(\tan^{2}{3x})'=2\tan{3x}\cdot (\tan{3x})'=2\tan{3x}\cdot (3x)'\cdot \dfrac{1}{\cos^{2}{3x}}=\dfrac{6\tan{3x}}{\cos^{2}{3x}}$.
	\item $y'=(\cot^{3}{4x})'=3\cot^{2}{4x}\cdot (\cot{4x})'=3\cot^{2}{4x}\cdot (4x)'\cdot \dfrac{-1}{\sin^{2}{4x}}=\dfrac{-12\cot^{2}{4x}}{\sin^{2}{4x}}$.
	\end{enumerate}
	}
\end{bt}
\begin{bt}%[1D5B3]
	Tính đạo hàm của các hàm số sau:
	\begin{listEX}[2]
	\item $y=\sin{2x}-3\sin{x}$.
	\item $y=\cos{3x}-4\cos{x}$.
	\end{listEX}
	\loigiai{
	\begin{enumerate}
	\item $y'=(\sin{2x}-3\sin{x})=(\sin{2x})'-(3\sin{x})'\\=(2x)'\cdot \cos{2x}-3\cdot (\sin{x})'=2\cdot \cos{2x}-3\cdot \cos{x}=2\cos{2x}-3\cos{x}$.
	\item $y'=(\cos{3x}-4\cos{x})'=(\cos{3x})'-(4\cos{x})'\\=(3x)'\cdot (-\sin{3x})-4\cdot (\cos{x})'=3\cdot (-\sin{3x})-4\cdot (-\sin{x})=-3\sin{3x}+4\sin{x}$.
	\end{enumerate}
	}
\end{bt}
\begin{bt}%[1D5B3]
	Tính đạo hàm của các hàm số $y=\cot{5x}-4\cot{x}$.
	\loigiai{
	 $y'=(\cot{5x}-4\cot{x})'=(\cot{5x})'-(4\cot{x})'\\=\dfrac{-(5x)'}{\sin^{2}{5x}}-4\cdot \dfrac{-1}{\sin^{2}{x}}=\dfrac{-5}{\sin^{2}{5x}}+\dfrac{4}{\sin^{2}{x}}$.
	}
\end{bt}
\begin{bt}%[1D5B3]
	Tính đạo hàm của các hàm số $y=\sin{x}\cos{3x}$.
	\loigiai{
	$y'=(\sin{x}\cos{3x})'=(\sin{x})'\cdot \cos{3x}+\sin{x}\cdot(\cos{3x})'\\=\cos{x}\cdot\cos{3x}+3\sin{x}\cdot (-\sin{3x})\\=\cos{x}\cos{3x}-3\sin{x}\sin{3x}$.
	}
\end{bt}
%-----------
\begin{dang}{Tính đạo hàm tổng, hiệu, tích, thương}
\end{dang}
\subsubsection{Ví dụ minh hoạ}
\begin{vd}%[1K9BV-1]
	Tính đạo hàm của các hàm số sau:
	\begin{listEX}[2]
	\item $y=\dfrac{1}{3}x^3-x^2+2x+1$;
	\item $y=3x^2-4x+2$.
	\end{listEX}
	\loigiai{
	\begin{enumerate}
	\item $y=3x^2-4x+2$.\\
	$y'=\left(3x^2-4x+2\right)^\prime=\left(3x^2\right)^\prime-\left(4x\right)^\prime+(2)^\prime=3\left(x^2\right)^\prime-4\left(x\right)^\prime+0=3\cdot2x-4\cdot 1=6x-4$.
	\item Ta có
	\begin{eqnarray*}
	y'&=&\dfrac{1}{3}\left(x^3\right)'-\left(x^2\right)'+2\left(x\right)'+1'\\&=&\dfrac{1}{3}\cdot 3x^2-2x+2\\&=&x^2-2x+2.
	\end{eqnarray*}
	\end{enumerate}
	}
\end{vd}
\begin{vd}%[1D5B2]
	Tính đạo hàm của các hàm số sau:
	\begin{multicols}{2}
	\begin{enumerate}
	\item $y=2x^4-\dfrac{1}{3}x^3+2\sqrt{x}-5$.
	\item $y=\left(x^3-1\right)\left(1-x^2\right)$.
	\end{enumerate}
	\end{multicols}
	\loigiai{
	\begin{enumerate}
	\item Ta có $y'=4x^3-x^2+\dfrac{1}{\sqrt{x}}$.
	\item Ta có $y'=\left(x^3-2\right)'\left(1-x^2\right)+\left(x^3-2\right)\left(1-x^2\right)'=3x^2\left(1-x^2\right)+\left(x^3-2\right) \left(-2x\right)=-5x^4+x^3+4x$.
	\end{enumerate}
	}
\end{vd}
\begin{vd}%[1T7B2-1]
	Tính đạo hàm của các hàm số sau:
	\begin{listEX}[2]
	\item $y=x^2\cdot 3^x$;
	\item $y=\dfrac{\sqrt{x}}{\cos x}$.
	\end{listEX}
	\loigiai{\begin{listEX}
	\item 
	$y'=\left(x^2\cdot 3^x\right)'=\left(x^2\right)'\cdot 3^x+x^2\cdot \left(3^x\right)'=2x\cdot 3^x+x^2\cdot 3^x\cdot \ln 3=x\cdot 3^x\left(2+x\ln 3\right)$.
	\item $y'=\left(\dfrac{\sqrt{x}}{\cos x}\right)'=\dfrac{\left(\sqrt{x}\right)'\cdot \cos x-\sqrt{x}\cdot \left(\cos x\right)'}{\cos^2x}=\dfrac{\dfrac{1}{2\sqrt{x}}\cdot\cos x-\sqrt{x}\left(-\sin x \right) }{\cos^2x}
	=\dfrac{\cos x+2x\sin x}{2\sqrt{x}\cos^2x}.
	$
	\end{listEX}
	}
\end{vd}
\begin{vd}%[1D5B2]
	Tính đạo hàm của các hàm số sau:
	\begin{multicols}{3}
	\begin{enumerate}
	\item $y=\dfrac{2x+1}{1-3x}$.
	\item $y=\dfrac{x^2-3x+3}{x-1}$.
	\item $y=\dfrac{1+x-x^2}{1-x+x^2}$.
	\end{enumerate}
	\end{multicols}
	\loigiai{
	\begin{enumerate}
	\item Ta có $y'=\dfrac{\left(2x-1\right)'\left(1-3x\right)-\left(2x-1\right)\left(1-3x\right)'}{\left(1-3x\right)^2}=\dfrac{2\left(1-3x\right)-\left(2x-1\right)\left(-3\right)}{\left(1-3x\right)^2}=\dfrac{5}{\left(1-3x\right)^2}$. 
	\item Ta có $y'=\dfrac{\left(x^2-3x+3\right)'\left(x-1\right)-\left(x^2-3x+3\right)\left(x-1\right)'}{\left(x-1\right)^2}=\dfrac{\left(2x-3\right)\left(x-1\right)-\left(x^2-3x+3\right)}{\left(x-1\right)^2}=\dfrac{x^2-2x}{\left(x-1\right)^2}$.
	\item Ta có $y'=\dfrac{\left(1+x-x^2\right)'\left(1-x+x^2\right)-\left(1+x-x^2\right)\left(1-x+x^2\right)'}{\left(1-x+x^2\right)^2}$\\ $=\dfrac{\left(1-2x\right)\left(1-x+x^2\right)-\left(1+x-x^2\right)\left(-1+2x\right)}{\left(1-x+x^2\right)^2}=\dfrac{2-4x}{\left(1-x+x^2\right)^2}$.
	\end{enumerate}
	}
\end{vd}
\
\subsubsection{Bài tập áp dụng}
\begin{bt}%[1C7Y2-1]
	Tính đạo hàm của mỗi hàm số sau:
	\begin{enumEX}{2}
	\item $f(x)=x^3+x$;
	\item $g(x)=x^4-x^2$.	
	\end{enumEX}
	\loigiai{
	\begin{enumerate}
	\item $f'(x)=\left(x^3\right)'+(x)'=3 x^2+1$.
	\item $g'(x)=\left(x^4\right)'-\left(x^2\right)'=4 x^3-2 x$. 	
	\end{enumerate}	
	}
\end{bt}
\begin{bt}%[1T7B2-1]
	Tính đạo hàm của các hàm số sau:
	\begin{listEX}[2]
	\item $y=x\cdot \log_2x$;
	\item $y=x^3\cdot \mathrm{e}^x$.
	\end{listEX}	
	\loigiai{\begin{listEX}
	\item $y=x\cdot \log_2x$.\\
	$y'=\left(x\cdot \log_2x\right)'=x'\cdot \log_2x+x\cdot \left(\log_2x\right)'=1\cdot \log_2x+x \cdot \dfrac{1}{x\ln 2}=\log_2x+\dfrac{1}{\ln 2}$.
	\item $y=x^3\cdot \mathrm{e}^x$.\\
	$y'=\left(x^3\cdot \mathrm{e}^x\right)'=\left(x^3\right)'\cdot \mathrm{e}^x+x^3\cdot \left(\mathrm{e}^x\right)'=3x^2\cdot \mathrm{e}^x+x^3\cdot \mathrm{e}^x=x^2\mathrm{e}^x\left(3+x\right)$.
	\end{listEX}
	}
\end{bt}
\begin{bt}%[1T7B2-1]
	Tính đạo hàm của các hàm số sau:
	\begin{listEX}[3]
	\item $y=\dfrac{2x+1}{x-1}$;
	\item $y=x\sin x$;
	\item $y=\dfrac{3x+2}{2x-1}$.
	\end{listEX}
	\loigiai{
	\begin{listEX}
	\item Ta có
	\begin{eqnarray*}
	y'&=&\dfrac{\left(2x+1\right)'\cdot (x-1)+\left(2x+1\right)\cdot \left(x+1\right)'}{\left(x-1\right)^2}\\&=&\dfrac{2(x-1)-(2x+1)}{\left(x-1\right)^2}\\&=&-\dfrac{3}{\left(x-1\right)^2}.
	\end{eqnarray*}
	\item $y=x\sin x$.\\
	$y'=\left(x\sin x\right)^\prime=\left(x\right)'\sin x+x \left(\sin x\right)'=1\cdot \sin x+x \cdot \cos x=\sin x+x\cos x$.
	\item $y=\dfrac{3x+2}{2x-1}$.
	\begin{eqnarray*}
	y'&=&\left(\dfrac{3x+2}{2x-1}\right)^\prime=\dfrac{\left(3x+2\right)'\left(2x-1\right)-\left(3x+2\right)\left(2x-1\right)'}{\left(2x-1\right)^2}\\
	&=&\dfrac{3\cdot \left(2x-1\right)-(3x+2)\cdot 2}{\left(2x-1\right)^2}
	=\dfrac{6x-3-6x-4}{\left(2x-1\right)^2}=\dfrac{-7}{\left(2x-1\right)^2}.
	\end{eqnarray*}
	\end{listEX}}
\end{bt}
\begin{bt}%[1D5B2]
	Tính đạo hàm của các hàm số sau:
	\begin{multicols}{2}
	\begin{enumerate}
	\item $y=\dfrac{1}{2}x^5+\dfrac{2}{3}x^4-x^3-\dfrac{3}{2}x^2+4x-5$.
	\item $y=\dfrac{1}{4}-\dfrac{1}{3}x+x^2-0,5x^4$.
	\item $y=\dfrac{x^4}{4}-\dfrac{x^3}{3}+\dfrac{x^2}{2}-x$.
	\item $y=x^5-4x^3+2x-3\sqrt{x}$.
	\end{enumerate}
	\end{multicols}
	\loigiai{
	\begin{multicols}{2}
	\begin{enumerate} 
	\item Có $y'=\dfrac{5}{2}x^4+\dfrac{8}{3}x^3-3x^2-3x+4$.
	\item Có $y'=-\dfrac{1}{3}+2x-2x^3$.
	\item Có $y'=x^3-x^2+x-1$.
	\item Có $y'=5x^4-12x^2+2-\dfrac{3}{2\sqrt{x}}$. 
	\end{enumerate}
	\end{multicols}
	}
\end{bt}
\begin{bt}%[1D5B2]
	Tính đạo hàm của các hàm số sau:
	\begin{enumerate}
	\begin{multicols}{3}
	\item $y=(2x-3)(x^5-2x)$.
	\item $y=x(2x-1)(3x+2)$.
	\item $y=\left(\sqrt{x}+1\right)\left(\dfrac{1}{\sqrt{x}}-1\right)$.
	\item $y=\dfrac{2x-1}{x-1}$.
	\item $y=\dfrac{x^2+x-1}{x-1}$.
	\item $y=\dfrac{2x^2-4x+5}{2x+1}$.
	\item $y=x+1-\dfrac{2}{x+1}$.
	\item $y=\dfrac{5x-3}{x^2+x+1}$.
	\item $y=\dfrac{x^2+x+1}{x^2-x+1}$.
	\end{multicols}
	\end{enumerate}
	\loigiai{
	\begin{enumerate}
	\begin{multicols}{2}
	\item $y'=12x^5-15x^4-8x+6$.
	\item $y'=18x^2+2x-2$.
	\end{multicols} 
	\item Ta có $y=\dfrac{1}{\sqrt{x}}-\sqrt{x}$.
	Suy ra $y'=-\dfrac{\left(\sqrt{x}\right)'}{x}-\dfrac{1}{2\sqrt{x}}=-\dfrac{1}{2x\sqrt{x}}-\dfrac{1}{2\sqrt{x}}$.
	\begin{multicols}{2} 
	\item $y'=\dfrac{-1}{\left(x-1\right)^2}$.
	\item $y'=\dfrac{x^2-2x}{\left(x-1\right)^2}$.
	\item $y'=\dfrac{4x^2+4x-14}{\left(2x+1\right)^2}$.
	\item $y'=1+\dfrac{2}{\left(x+1\right)^2}=\dfrac{x^2+2x+3}{\left(x+1\right)^2}$.
	\item $y'=\dfrac{-5x^2-6x+8}{\left(x^2+x+1\right)^2}$.
	\item $y'=\dfrac{-2x^2+2}{\left(x^2-x+1\right)^2}$.
	\end{multicols}
	\end{enumerate}
	}
\end{bt}
\begin{dang}{Một số ứng dụng của đạo hàm}
\end{dang}
\subsubsection{Ví dụ minh hoạ}
\begin{vd}%[1T7Y2-4]
	Viết phương trình tiếp tuyến của đồ thị hàm số $y=\sqrt{x}$ tại điểm có hoành độ bằng $4$.
	\loigiai{Với $x=4$ thì $y=\sqrt{4}=2$.\\
	Ta có $y^\prime=\left(\sqrt{x}\right)^\prime =\dfrac{1}{2\sqrt{x}}$, $x>0$. Từ đó, $y^\prime \left(4\right)=\dfrac{1}{2\sqrt{4}}=\dfrac{1}{4}$. \\
	Phương trình tiếp tuyến của đồ thị hàm số $y=\sqrt{x}$ tại điểm có hoành độ bằng $4$ là 
	\[y-2=\dfrac{1}{4}\left(x-4\right) \ \text{hay }\ y=\dfrac{1}{4}x+1.\]	
	} 
\end{vd}
\begin{vd}%[1D5B2]
	Cho đường cong $(C):\ y=f(x)=\dfrac{x^2}{2}-4x+1$.
	\begin{enumerate}
	\item Viết phương trình tiếp tuyến của $(C)$ tại điểm có hoành độ $x_0=-2$.
	\item Viết phương trình tiếp tuyến của $(C)$, biết tiếp tuyến có hệ số góc $k=1$.
	\end{enumerate}
	\loigiai{
	\begin{enumerate}
	\item Ta có $f'(x)=x-4$. Với $x_0=-2\Rightarrow y_0=11$.\\
	Do đó, tiếp tuyến cần tìm có phương trình: $y=f'(-2)(x+2)+11=-6x-1$.
	\item Gọi $(x_0;y_0)$ là tiếp điểm. Ta có $f'(x_0)=1\Leftrightarrow x_0-4=1\Leftrightarrow x_0=5\Rightarrow y_0=-\dfrac{13}{2}$.\\
	Vậy, tiếp tuyến có phương trình là $y=1(x-5)-\dfrac{13}{2}=x-\dfrac{23}{2}$.
	\end{enumerate}}
\end{vd}
\begin{vd}%[1D5B2]
	Cho hàm số $y=f(x)=x^3-3x^2+2$ có đồ thị $(C)$. Viết phương trình tiếp tuyến của $(C)$ biết tiếp tuyến song song với đường thẳng $\Delta:\ 3x+y=2$.
	\loigiai{Gọi $(x_0;y_0)$ là tiếp điểm của tiếp tuyến cần tìm.\\
	Vì tiếp tuyến song song với đường thẳng $\Delta:\ y=-3x+2$ nên ta có $$f'(x_0)=-3\Leftrightarrow 3x_0^2-6x_0+3=0\Leftrightarrow x_0=1\Rightarrow y_0=0.$$
	Do vậy, tiếp tuyến có phương trình: $y=-3(x-1)+0=-3x+3$.}
\end{vd}
\begin{vd}%[1D5B2]
	Cho hàm số $y=4x^3-6x^2+1\ (1)$. Viết phương trình tiếp tuyến của đồ thị hàm số $(1)$, biết rằng tiếp tuyến đó đi qua điểm $M(-1;-9)$.
	\loigiai{Ta có $y'=12x^2-12x$. Gọi $(x_0;y_0)$ là tiếp điểm của tiếp tuyến đó.\\
	Khi đó, phương trình tiếp tuyến tương ứng có dạng: $y=y'(x_0)(x-x_0)+y_0$.\\
	Mặt khác, tiếp tuyến đi qua điểm $M(-1;-9)$ nên ta có phương trình $$-9=(12x_0^2-12x_0)(-1-x_0)+4x_0^3-6x_0^2+1\Leftrightarrow (x_0+1)^2(4x_0-5)=0\Leftrightarrow \hoac{&x_0=-1\\&x_0=\dfrac{5}{4}.}$$
	Với $x=-1$ ta tìm được phương trình tiếp tuyến: $y=24x+15$.\\
	Với $x=\dfrac{5}{4}$ ta có phương trình tiếp tuyến: $y=\dfrac{15}{4}x+\dfrac{21}{4}$.}
\end{vd}
\begin{vd}%[1T7B2-7]
	Một vật chuyển động thẳng không đều xác định bởi phương trình $s(t)=t^2-4t+3$, trong đó $s$ là quãng đường tính bằng mét và $t$ là thời gian tính bằng giây. Tính gia tốc của chuyển động tại thời điểm $t=4$.
	\loigiai{Ta có $s'(t)=\left(t^2-4t+3\right)'=2t-4$, $s^{\prime\prime}(t)=(2t-4)'=2$.\\
	Gia tốc của chuyển động tại thời điểm $t=4$ là $s^{\prime\prime}(4)=2$ $ m/s^2$.
	} 
\end{vd}
\begin{vd}%[1D5K2]
	Một vật chuyển động theo quy luật $s=-\dfrac{2}{3} t^3 + 4t^2-1$ với $t$ (giây) là khoảng thời gian tính từ khi vật bắt đầu chuyển động và $s$ (mét) là quãng đường vật di chuyển được trong khoảng thời gian đó. Hỏi trong khoảng thời gian $5$ giây, kể từ khi bắt đầu chuyển động, vận tốc lớn nhất của vật đạt được bằng bao nhiêu?
	\loigiai{
	Vận tốc của chuyển động có phương trình $v=s'=-2t^2+8t$.\\
	Ta có $-2t^2+8t=8-2(t-2)^2\le 8$. Đẳng thức có được khi $t=2$.\\
	Do đó, trong khoảng thời gian $5$ giây, kể từ khi bắt đầu chuyển động, vận tốc lớn nhất của vật đạt được bằng $8$ m/s$^2$.}
\end{vd}
\baitaptl
\setcounter{bt}{0}
\begin{bt}%[1D5B2]
	Viết phương trình tiếp tuyến của đường cong $(C):\ y=f(x)=x(x^2+x-1)+1$ tại điểm có tung độ bằng $-1$.
	\loigiai{
	Đạo hàm $y'=3x^2+2x-1$. Gọi $(x_0;y_0)$ là tiếp điểm, ta có $$y_0=-1\Leftrightarrow x_0(x_0^2+x_0-1)+1=-1\Leftrightarrow (x_0+2)(x_0^2-x_0+1)=0\Leftrightarrow x_0=-2.$$
	Tính được $f'(-2)=7$, ta có phương trình tiếp tuyến cần tim: $y=7x+13$.}
\end{bt}
\begin{bt}%[1D5B2]
	Viết phương trình tiếp tuyến của parabol $(P):\ y=f(x)=-x^2+4x-3$ tại các giao điểm của $(P)$ với trục hoành.
	\loigiai{
	Đạo hàm $y'=f'(x)=-2x+4$. Parabol cắt trục hoành lần lượt tại $x=1$ và $x=3$.\\
	+ Với $x_0=1,y_0=0\Rightarrow f'(1)=2$, ta có tiếp tuyến: $y=2x-2$.\\
	+ Với $x_0=3,y_0=0\Rightarrow f'(3)=-2$, ta có tiếp tuyến: $y=-2x+6$.}
\end{bt}
\begin{bt}%[1D5B2]
	Viết phương trình tiếp tuyến của đồ thị $(C):\ y=\dfrac{x-1}{x+2}$ biết tiếp tuyến vuông góc với đường thẳng $\Delta:\ 3x+y-2=0$.
	\loigiai{
	Tập xác định: $\mathscr{D}=\mathbb{R}\setminus\{-2\}$. Đạo hàm $y'=\dfrac{3}{(x+2)^2}$. Viết lại PTĐT $\Delta:\ y=-3x+2$. Gọi $(x_0;y_0)$ là tiếp điểm của tiếp tuyến cần tìm. Tiếp tuyến vuông góc với đường thẳng $\Delta$ nên $$f'(x_0)=\dfrac{1}{3}\Leftrightarrow \dfrac{3}{(x_0+2)^2}=\dfrac{1}{3}\Leftrightarrow \hoac{&x_0=1\\&x_0=-5.}$$
	Từ đó tìm được hai tiếp tuyến thỏa mãn yêu cầu bài toán là: $y=\dfrac{x-1}{3}$ và $y=\dfrac{x+11}{3}$.}
\end{bt}
\begin{bt}%[1K9KU-3]
	Một vật được phóng lên theo phương thẳng đứng lên trên từ mặt đất với vận tốc ban đầu $v_0=20\,\mathrm{m/s}$. Trong vật lí, ta biết rằng khi bỏ qua sức cản của không khí, độ cao $h$ so với mặt đất (tính bằng mét) của vật tại thời điểm $t$ (giây) sau khi ném được cho bởi công thức sau: $h=v_0 t-\dfrac{1}{2}gt^2$, trong đó $v_0$ là vận tốc ban đầu của vật, $g=9{,}8\,\mathrm{m/s^2}$ là gia tốc rơi tự do. Hãy tính vận tốc của vật khi nó đạt độ cao cực đại và khi nó chạm đất.
	\loigiai{
	Phương trình chuyển động của vật là $h=v_0 t-\dfrac{1}{2}gt^2$.\\
	Vận tốc của vật tại thời điểm $t$ được cho bởi $v(t) = h' = v_0 -gt $.\\
	Vận tốc của vật cao cực đại tại thời điểm $t_1=\dfrac{v_0}{g}$, tại đó vận tốc bằng $v\left(t_1\right)=v_0-gt_1=0$.\\
	Vật chạm đất tại thời điểm $t_2$ mà $h\left(t_2\right)=0$ nên ta có $v_0t_2-\dfrac{1}{2}gt_2^2=0 \Leftrightarrow t_2=0$ (loại) và $t_2=\dfrac{2v_0}{g}$.\\
	Khi chạm đất, vận tốc của vật là $v\left(t_2\right)=v_0-gt_2=-v_0=-20\,\left(\mathrm{m/s}\right)$.\\
	Dấu âm của $v\left(t_2\right)$ thể hiện độ cao của vật giảm với $20\,\mathrm{m/s}$ (tức là chiều chuyển động của vật ngược với chiều dương đẫ chọn).
	}
\end{bt}
\begin{bt}%[1T7B2-7]
	Một hòn sỏi rơi tự do có quãng đường rơi tính theo thời gian $t$ là $s(t)=4,9 t^2$, trong đó $s$ tính bằng mét và $t$ tính bằng giây. Tính gia tốc rơi của hòn sỏi lúc $t=3$.
	\loigiai{
	Ta có $s'(t)=\left(4{,}9 t^2\right)'=4{,}9\cdot 2t=9{,}8t$; $s^{\prime\prime}(t)=(9{,}8t)'=9{,}8$.\\
	Gia tốc rơi của hòn sỏi lúc $t=3$ là $s^{\prime\prime}(3)=9{,}8$ m/s$^2$.
	}
\end{bt}
\begin{dang}{Chứng minh đẳng thức hoặc giải phương trình}
\end{dang}\viduminhhoa
\begin{vd}%[1D5B3]
	Cho hàm số $y=\tan x$. Chứng minh rằng $y'-y^2-1=0$.
	\loigiai{
	Ta có $y'={\left(\tan x\right)}’=\tan^2x+1$. \\
	Suy ra $y'-y^2-1=\tan^2x+1-\tan^2x-1=0$. \\
	Vậy ta có điều phải chứng minh.
	}
\end{vd}
\begin{vd}%[1D5B3]
	Cho hàm số $y=\cot 2x$. Chứng minh rằng $y'+2y^2+2=0$.
	\loigiai{
	Ta có $y'={\left(\cot 2x\right)}’=-2\left(\cot^22x+1\right)$ và $y^2=\cot^22x\Rightarrow 2y^2+2=2\cot^22x+2$.\\
	Suy ra $y'+2y^2+2=-2\left(\cot^2x+1\right)+2\cot^2x+2=0$.\\
	Vậy ta có điều phải chứng minh.
	}
\end{vd}
\begin{vd}%[1D5B3]
	Cho hàm số $y=\sin^6x+\cos^6x+3\sin^2x\cos^2x$. Chứng minh rằng $y'=0$.
	\loigiai{
	Ta có 
	\begin{align*}
	y&=\sin^6x+\cos^6x+3\sin^2x\cos^2x \\ 
	& =\left(\sin^2x+\cos^2x\right)\left[{\left(\sin^2x+\cos^2x\right)}^2-3\sin^2x.\cos^2x\right]+3\sin^2x\cos^2x \\ 
	& =\left(\sin^2x+\cos^2x\right)^2=1.\ 
	\end{align*}
	Suy ra $y'=0$.
	}
\end{vd}
\begin{vd}%[1D5B3]
	Cho hàm số $y=\cos^2x-\sin x$. Giải phương trình $y'=0$.
	\loigiai{
	Ta có $y'=\left(\cos^2x-\sin x\right)’=-2\sin x.\cos x-\cos x$. Suy ra\\
	\begin{align*}
	y'=0&\Leftrightarrow-2\sin x.\cos x-\cos x=0 \\ 
	& \Leftrightarrow \cos x\left(-2\sin x-1\right)=0 \\ 
	& \Leftrightarrow \hoac{
	\cos x=0 \\
	-2\sin x-1=0}\\ 
	& \Leftrightarrow \hoac{
	&x=\dfrac{\pi}{2}+k\pi \\
	&x=-\dfrac{\pi}{6}+k2\pi \\ 
	&x=\dfrac{7\pi}{6}+k2\pi}, \left(k\in \mathbb{Z}\right).
	\end{align*}
	}
\end{vd}
\begin{vd}%[1D5B3]
	Giải phương trình $y'=0$ với $y=3\cos x+4\sin x+5x$.
	\loigiai{
	Ta có $y'=\left(3\cos x+4\sin x+5x\right)'=-3\sin x+4\cos x+5$
	\begin{align*}
	y'=0&\Leftrightarrow-3\sin x+4\cos x+5=0 \\ 
	& \Leftrightarrow 3\sin x-4\cos x=5 \\ 
	& \Leftrightarrow \sin \left(x-\alpha \right)=1 \\ 
	& \Leftrightarrow x=\dfrac{\pi}{2}+\alpha+k2\pi, \text{với} \heva{
	& \sin \alpha =\dfrac{4}{5} \\ 
	& \cos \alpha =\dfrac{3}{5}},\left(k\in \mathbb{Z}\right). 
	\end{align*}
	}
\end{vd}
\begin{vd}%[1D5K3]
	Giải phương trình $y'=0$ với $y=\tan x+\cot x$.
	\loigiai{
	Ta có $y'=\left(\tan x+\cot x\right)’=-\dfrac{\cos 2x}{\cos^2x.\sin^2x}$. Suy ra \\
	$y'=0\Leftrightarrow-\dfrac{\cos 2x}{\cos^2x.\sin^2x}=0\Leftrightarrow \dfrac{\cos 2x}{\sin^22x}=0 \left(*\right)$.\\
	Điều kiện $\sin^22x\ne 0\Leftrightarrow \sin 2x\ne 0\Leftrightarrow x\ne \dfrac{k\pi}{2}$. \\
	Khi đó $\left(*\right)\Leftrightarrow \cos 2x=0\Leftrightarrow x=\dfrac{\pi}{4}+\dfrac{k\pi}{2}$ (thỏa mãn điều kiện). \\
	Vậy phương trình $y'=0$ có các nghiệm là $x=\dfrac{\pi}{4}+\dfrac{k\pi}{2}, (k\in \mathbb{Z})$.
	}
\end{vd}
\begin{vd}%[1D5K3]
	Cho hàm số $f(x)=\dfrac{\sin 3x}{3}-\cos x-\sqrt{3}\left(\sin x-\dfrac{\cos 3x}{3}\right)$. Giải phương trình $f'(x)=0$.
	\loigiai{
	Ta có: $ f'(x)=\cos 3x+\sin x-\sqrt{3}\left(\cos x+\sin 3x\right)$. 
	\begin{align*}
	f'(x)=0&\Leftrightarrow \cos 3x+\sin x-\sqrt{3}\left(\cos x+\sin 3x\right)=0 \\ 
	& \Leftrightarrow \sin x-\sqrt{3}\cos x=\sqrt{3}\sin 3x-\cos 3x \\ 
	& \Leftrightarrow \dfrac{1}{2}\sin x-\dfrac{\sqrt{3}}{2}\cos x=\dfrac{\sqrt{3}}{2}\sin 3x-\dfrac{1}{2}\cos 3x \\ 
	& \Leftrightarrow \sin \left(x-\dfrac{\pi}{3}\right)=\sin \left(3x-\dfrac{\pi}{6}\right) \\ 
	& \Leftrightarrow \hoac{
	x-\dfrac{\pi}{3}&=3x-\dfrac{\pi}{6}+k2\pi \\
	x-\dfrac{\pi}{3}&=\pi-3x+\dfrac{\pi}{6}+k2\pi}
	\Leftrightarrow \hoac{
	&x=-\dfrac{\pi}{12}-k\pi \\
	&x=\dfrac{3\pi}{8}+k\dfrac{\pi}{2}},\left(k\in \mathbb{Z}\right).
	\end{align*}
	}
\end{vd}
\baitaptl
\begin{bt}%[1D5B3]
	Cho hàm số $ y=\cos^2\left(\dfrac{\pi}{3}-x\right)+\cos^2\left(\dfrac{\pi}{3}+x\right)+\cos^2\left(\dfrac{2\pi}{3}-x\right)+\cos^2\left(\dfrac{2\pi}{3}+x\right)-2\sin^2x$. Chứng minh rằng $y'=0$.
	\loigiai{
	Ta có \begin{align*}
	y&=\cos^2\left(\dfrac{\pi}{3}-x\right)+\cos^2\left(\dfrac{\pi}{3}+x\right)+\cos^2\left(\dfrac{2\pi}{3}-x\right)+\cos^2\left(\dfrac{2\pi}{3}+x\right)-2\sin^2x \\ 
	& = 2\cos^2\left(\dfrac{\pi}{3}-x\right)+2\cos^2\left(\dfrac{\pi}{3}+x\right)-2\sin^2x \\ 
	& =1+\cos \left(\dfrac{2\pi}{3}-2x\right)+\cos \left(\dfrac{2\pi}{3}+2x\right)+\cos 2x \\ 
	& =1+2\cos \dfrac{2\pi}{3}.\cos 2x+\cos 2x \\ 
	& =1-\cos 2x+\cos 2x=1 
	\end{align*}
	Suy ra $y'=0$.
	}
\end{bt}
\begin{bt}%[1D5K3]
	Cho hàm số $y=x\sin x$. Chứng minh rằng 
	\begin{listEX}[2]
	\item $xy-2\left(y'-\sin x\right)+x\left(2\cos x-y\right)=0$.
	\item $\dfrac{y'}{\cos x}-x=\tan x$.
	\end{listEX}
	\loigiai{
	\begin{enumerate}
	\item Ta có: $y'=\left(x\sin x\right)'=\sin x+x\cos x$.
	\begin{align*}
	xy-2\left(y'-\sin x\right)+x\left(2\cos x-y\right)&=x^2\sin x-2\left(\sin x+x\cos x-\sin x\right)+x\left(2\cos x-x\sin x\right) \\ 
	& =x^2\sin x-2x\cos x+2x\cos x-x^2\sin x=0.
	\end{align*}
	\item Ta có: $y'=\left(x\sin x\right)'=\sin x+x\cos x$.\\
	$\dfrac{y'}{\cos x}-x=\dfrac{\sin x+x\cos x}{\cos x}-x=\tan x+x-x=\tan x$.
	\end{enumerate}
	}
\end{bt}
\begin{bt}%[1D5B3]
	Giải phương trình $y'=0$ với $y=1-\sin \left(\pi+x\right)+2\cos \left(\dfrac{2\pi+x}{2}\right)$.
	\loigiai{
	Ta có $y'=\left[1-\sin \left(\pi+x\right)+2\cos \left(\dfrac{2\pi+x}{2}\right)\right]’=\cos x+\sin \dfrac{x}{2}$.\\
	$y'=0\Leftrightarrow \cos x+\sin \dfrac{x}{2}=0\Leftrightarrow 2\sin^2\dfrac{x}{2}-\sin \dfrac{x}{2}-1=0\Leftrightarrow \hoac{
	\sin \dfrac{x}{2}&=1 \\ 
	\sin \dfrac{x}{2}&=-\dfrac{1}{2}}$.\\
	Với $\sin \dfrac{x}{2}=1\Leftrightarrow x=\pi+k4\pi.$\\
	Với $\sin \dfrac{x}{2}=-\dfrac{1}{2}\Leftrightarrow \hoac{
	x=-\dfrac{\pi}{3}+k4\pi \\
	x=\dfrac{7\pi}{3}+k4\pi }.$ \\
	Vậy phương trình $y'=0$ có các nghiệm là $x=\pi+k4\pi; x=-\dfrac{\pi}{3}+k4\pi; x=\dfrac{7\pi}{3}+k4\pi, (k\in \mathbb{Z})$.
	}
\end{bt}
\begin{bt}%[1D5B3]
	Giải phương trình $y'=0$ với $y=\sin 2x-2\cos x$.
	\loigiai{
	Ta có $y'=\left(\sin 2x-2\cos x\right)’=2\cos 2x+2\sin x$. Suy ra\\ 
	$y'=0\Leftrightarrow 2\cos 2x+2\sin x=0\Leftrightarrow 2\sin^2x-\sin x-1=0\Leftrightarrow \hoac{
	\sin x=1 \\
	\sin x=-\dfrac{1}{2}}$\\
	Với $\sin x=1\Leftrightarrow x=\dfrac{\pi}{2}+k2\pi.$\\
	Với $\sin x=-\dfrac{1}{2}\Leftrightarrow \hoac{
	x=-\dfrac{\pi}{6}+k2\pi \\
	x=\dfrac{7\pi}{6}+k2\pi }$. \\
	Vậy phương trình $y'=0$ có các nghiệm là $x=\dfrac{\pi}{2}+k2\pi; x=-\dfrac{\pi}{6}+k2\pi; x=\dfrac{7\pi}{6}+k2\pi, (k\in \mathbb{Z})$
	}
\end{bt}
\begin{bt}%[1D5K3]
	Cho hàm số $f(x)=a\sin x+b\cos x+1$ có đạo hàm là $f'(x)$. Tìm $a,b$ biết $f'(0)=\dfrac{1}{2}$ và $f’\left(-\dfrac{\pi}{4}\right)=1$.
	\loigiai{
	Ta có $f’(x) =a\cos x-b\sin x.$ Khi đó\\
	$\heva{
	& f’(0)=\dfrac{1}{2} \\ 
	& f’\left(-\dfrac{\pi}{4}\right)=1}$
	$\Leftrightarrow \heva{
	& a\cos 0-b\sin 0=\dfrac{1}{2} \\ 
	& a\sin \left(-\dfrac{\pi}{4}\right)+b\cos \left(-\dfrac{\pi}{4}\right)+1=1}\Leftrightarrow \heva{
	& a=\dfrac{1}{2} \\ 
	&-\dfrac{\sqrt{2}}{2}a+\dfrac{\sqrt{2}}{2}b=0 \\}\Leftrightarrow \heva{
	& b=\dfrac{1}{2} \\ 
	& a=\dfrac{1}{2}}.$\\
	Vậy $a=\dfrac{1}{2}$ và $b=\dfrac{1}{2}$.
	}
\end{bt}
%=================
\begin{dang}{Tính đạo hàm cấp hai}
\end{dang}
\subsubsection{Ví dụ minh hoạ}
%%==========Ví dụ 1
\begin{vd}%[1K9YV-1]	
	\begin{itemize}
		\item Gọi $g(x)$ là đạo hàm của hàm số $y=\sin \left(2x+\dfrac{\pi}{4}\right)$. Tìm $g(x)$.
		\item Tính đạo hàm của hàm số $y=g(x)$.
	\end{itemize}
	\loigiai{\begin{itemize}
			\item $g(x)=f'(x)=\left(2x+\dfrac{\pi}{4}\right)'\cdot \cos \left(2x+\dfrac{\pi}{4}\right)=2\cos \left(2x+\dfrac{\pi}{4}\right)$ .
			\item $y'=g'(x)=-2\left(2x+\dfrac{\pi}{4}\right)'\cdot \sin \left(2x+\dfrac{\pi}{4}\right) =-4\sin \left(2x+\dfrac{\pi}{4}\right)$.
	\end{itemize}}
\end{vd}
%%==========Ví dụ 2
\begin{vd}%[1C7B3-1]
	Cho hàm số $ f(x)=x^{4}-4x^{2}+3 $.
	\begin{enumerate}
		\item Tìm đạo hàm cấp hai của hàm số tại điểm $ x $ bất kì.
		\item Tính đạo hàm cấp hai của hàm số tại điểm $ x_{0}=-1 $.
	\end{enumerate}
	\loigiai{
		\begin{enumerate}
			\item Ta có $ f'(x)=4x^{3}-8x $ và $ f''(x)=12x^{2}-8 $.
			\item Ta có $ f''(-1)=12\cdot (-1)^{2}-8=4 $.
		\end{enumerate}
	}
\end{vd}
%%==========Ví dụ 3
\begin{vd}%[1T7Y2-2]
	Tính đạo hàm cấp hai của các hàm số:
	\begin{listEX}[4]
		\item $y=3x^2+5x+1$;
		\item $y=\sin x$;
		\item $y=x\cdot \mathrm{\, e}^{2x}$;
		\item $y=\ln (2x+3)$.
	\end{listEX}
	\loigiai{
		\begin{enumerate}
			\item $y=3x^2+5x+1$.\\
			$y'=\left(3x^2+5x+1\right)'=6x+5$, $y^{\prime\prime} =\left(6x+5\right)'=6$.
			\item $y=\sin x$.\\
			$y'=\left(\sin x\right)'=\cos x$, $y^{\prime\prime} =\left(\cos x\right)'=-\sin x$.
			\item $y'=x'\cdot \mathrm{\, e}^{2x}+x\cdot \left(\mathrm{\, e}^{2x}\right)'=\mathrm{\, e}^{2x}+x\cdot (2x)'\mathrm{\, e}^{2x}=\mathrm{\, e}^{2x}+2x\cdot \mathrm{\, e}^{2x}=\mathrm{\, e}^{2x}\cdot \left(1+2x\right)$;\\
			$y''=(2x)'\mathrm{\, e}^{2x}\cdot \left(1+2x\right)+\left(1+2x\right)'\cdot \mathrm{\, e}^{2x}=2\mathrm{\, e}^{2x}\cdot \left(1+2x\right)+2\cdot \mathrm{\, e}^{2x}=\mathrm{\, e}^{2x}\left(4+4x\right)$.
			\item $y'=\dfrac{(2x+3)'}{2x+3}=\dfrac{2}{2x+3}$;\\
			$y''=-\dfrac{2\left(2x+3\right)'}{(2x+3)^2}=-\dfrac{4}{(2x+3)^2}$.
		\end{enumerate}
	}
\end{vd}
%%==========Ví dụ 4
\begin{vd}%[1D5B5]
	Tìm đạo hàm cấp hai của các hàm số sau:
	\begin{listEX}[3]
		\item $y={\left(x^2+1\right)}^3$.
		\item $y=\dfrac{x}{x-2}$.
		\item $y=\dfrac{x^2+x+1}{x+1}$.
	\end{listEX}
	\loigiai{
		\begin{enumerate}
			\item $y=x^6+3x^4+3x^2+1$; $y'=6x^5+12x^3+6x$; 
			$y''=30x^4+36x^2+6$.
			\item $y'=\left(\dfrac{x}{x-2}\right)'=\dfrac{-2}{\left(x-2\right)^2}$; $y''=\left(\dfrac{-2}{\left(x-2\right)^2}\right)'=2\cdot\dfrac{2\left(x-2\right)}{\left(x-2\right)^4}=\dfrac{4}{\left(x-2\right)^3}$.
			\item $y=\dfrac{x^2+x+1}{x+1}=x+\dfrac{1}{x+1}$.\\
			$y'=1-\dfrac{1}{\left(x+1\right)^2}$.\\
			$y''=\dfrac{2}{\left(x+1\right)^3}$.
		\end{enumerate}
	}
\end{vd}
%%==========Ví dụ 5
\begin{vd}%[1D5B5]
	Tìm đạo hàm cấp hai của các hàm số sau:
	\begin{listEX}[4]
		\item $y=\sqrt{2x+5}$.
		\item $y=x\sqrt{x^2+1}$.
		\item $y= \sin x$. 
		\item $y=\tan x$.
	\end{listEX}
	\loigiai{
		\begin{enumerate}
			\item ${y}'=\left(\sqrt{2x+5}\right)'=\dfrac{2}{2\sqrt{2x+5}}=\dfrac{1}{\sqrt{2x+5}}$\\
			$y''=-\dfrac{\left(\sqrt{2x+5}\right)'}{2x+5}=-\dfrac{\dfrac{2}{2\sqrt{2x+5}}}{2x+5}=-\dfrac{1}{\left(2x+5\right)\sqrt{2x+5}}$.
			\item $y'=\sqrt{x^2+1}+x\dfrac{x}{\sqrt{x^2+1}}=\dfrac{2x^2+1}{\sqrt{x^2+1}}$.\\
			$y''=\dfrac{4x\sqrt{x^2+1}-\left(2x^2+1\right)\dfrac{x}{\sqrt{x^2+1}}}{x^2+1}=\dfrac{2x^3+3x}{\left(1+x^2\right)\sqrt{1+x^2}}$.
			\item $y'= \cos x = sin\left (\dfrac{\pi}{2}+x\right )$; $y''= \cos\left (\dfrac{\pi}{2}+x\right )= \sin\left (\pi+x\right )$.
			\item $y'=\dfrac{1}{\cos^2x}$; $y''=-\dfrac{2 \cos x\left(- \sin x\right)}{\cos^4x}=\dfrac{2 \sin x}{\cos^3x}$.
		\end{enumerate}
	}
\end{vd}
%%==========Ví dụ 6
\begin{vd}%[1K9YW-1]
	Tính đạo hàm cấp hai của hàm số $y=x^2+\mathrm{\, e}^{2x-1}$. Từ đó tính $y''(0)$.
	\loigiai{
		$y'=2x+(2x-1)'\mathrm{\, e}^{2x-1}=2x+2\cdot \mathrm{\, e}^{2x-1}$\\
		$y''=2+2\cdot (2x-1)'\mathrm{\, e}^{2x-1}=2+4\mathrm{\, e}^{2x-1}$.\\
		$y''(0)=2+4\mathrm{\, e}^{2\cdot 0-1}=2+4\mathrm{\, e}^{-1}=2+\dfrac{4}{\mathrm{\, e}}=\dfrac{2\mathrm{\, e}+4}{\mathrm{\, e}}$.
	}
\end{vd}
%%==========Ví dụ 7
\begin{vd}%[1D5B5]
	Cho hàm số $h(x)=5\left(x+1\right)^3+4\left(x+1\right)$. Giải phương trình $h''(x)=0$.
	\loigiai{
		$h(x)=5\left(x+1\right)^3+4\left(x+1\right)$;\\
		$h'(x)=15\left(x+1\right)^2+4$;\\
		$h''(x)=30\left(x+1\right)$.\\
		$h''(x)=0\Leftrightarrow x=-1$.
	}
\end{vd}
\subsubsection{Bài tập áp dụng}
%%==========Bài 1
\begin{bt}%[1C7B3-1]
	Cho hàm số $ f(x)=\dfrac{1}{x+2} $.
	\begin{enumerate}
		\item Tìm đạo hàm cấp hai của hàm số tại điểm $ x\ne -2 $.
		\item Tính đạo hàm cấp hai của hàm số tại điểm $ x_{0}=2 $.
	\end{enumerate}
	\loigiai{
		\begin{enumerate}
			\item Với $ x\ne -2 $, ta có\\
			$f'(x)=\left(\dfrac{1}{x+2}\right)'=-\dfrac{(x+2)'}{(x+2)^{2}}=\dfrac{-1}{(x+2)^{2}} $.\\
			$f''(x)=\left[\dfrac{-1}{(x+2)^{2}}\right]'=\dfrac{\left[(x+2)^{2}\right]'}{(x+2)^{4}} =\dfrac{2(x+2)}{(x+2)^{4}} =\dfrac{2}{(x+2)^{3}} $.
			\item Ta có $f''(2)=\dfrac{2}{(2+2)^3}=\dfrac{1}{32}$.
		\end{enumerate}
	}
\end{bt}
%%==========Bài 2
\begin{bt}%[1T7B2-2]
	Tính đạo hàm cấp hai của các hàm số sau:
	\begin{listEX}[4]
		\item $y=x^2-x$;
		\item $y=\cos x$;
		\item $y=2x^4-5x^2+3$;
		\item $y=x\cdot \mathrm{e}^x$.
	\end{listEX}
	\loigiai{\begin{enumerate}
			\item $y=x^2-x$.\\
			$y'=(x^2-x)'=2x-1$, $y^{\prime\prime}=(2x-1)'=2$.
			\item $y=\cos x$.\\
			$y'=(\cos x)'=-\sin x$, $y^{\prime\prime}=(-\sin x)'=-(\sin x)'=-\cos x$.
			\item $y=2x^4-5x^2+3$.\\
			Ta có $y'=\left(2x^4-5x^2+3\right)'=8x^3-10x$; $y^{\prime\prime}=\left(8x^3-10x\right)'=24x^2-10$.
			\item $y=x\cdot \mathrm{e}^x$.\\
			Ta có $y'=\left(x\cdot \mathrm{e}^x\right)'=x'\cdot \mathrm{e}^x+x \cdot \left(\mathrm{e}^x\right)'=1\cdot \mathrm{e}^x+x\cdot \mathrm{e}^x=(1+x)\cdot \mathrm{e}^x$.\\
			$y^{\prime\prime}=\left[(1+x)\cdot \mathrm{e}^x\right]'=(1+x)'\cdot \mathrm{e}^x+(1+x)\cdot \left(\mathrm{e}^x\right)'=\mathrm{e}^x+(1+x)\mathrm{e}^x=(2+x)\mathrm{e}^x$.
		\end{enumerate}
	}
\end{bt}
%%==========Bài 3
\begin{bt}%[1D5B5]
	Tìm đạo hàm cấp hai của các hàm số sau:
	\begin{listEX}[2]
		\item $y=-3x^4+4x^3+5x^2-2x+1$.
		\item $y=\dfrac{4}{5}x^5-3x^2-x+4$. 
	\end{listEX}
	\loigiai{
		\begin{listEX}[2]
			\item $y'=-12x^3+12x^2+10x-2$; $y''=-36x^2+24x+10$.
			\item $y'=4x^4-6x-1$; $y''=16x^3-6$.
		\end{listEX}
	}
\end{bt}
%%==========Bài 4
\begin{bt}%[1D5K5]
	Tìm đạo hàm cấp hai của các hàm số sau:
	\begin{listEX}[4]
		\item $y=-\dfrac{1}{x}$.
		\item $y=\dfrac{1}{x-3}$
		\item $y=\dfrac{-2x^2+3x}{1-x}$.
		\item $y=\dfrac{5x^2-3x-20}{x^2-2x-3}$. 
	\end{listEX}
	\loigiai{
		\begin{enumerate}
			\item $y'=\dfrac{1}{x^2}$; $y''=-\dfrac{2}{x^3}$.
			\item $y'=-\dfrac{1}{\left(x-3\right)^2}$; $y''=\dfrac{2}{{\left(x-3\right)}^3}$.
			\item $y=2x-1+\dfrac{1}{1-x}\Rightarrow y'=2+\dfrac{1}{\left(1-x\right)^2}$ ; $y''=\dfrac{2}{(1-x)^3}$.
			\item $y'=\dfrac{(10x-3)(x^2-2x-3)-(5x^2-3x-20)(2x-2)}{(x^2-2x-3)^2}=\dfrac{-7x^2+10x-31}{(x^2-2x-3)^2}$.\\
			$\begin{aligned}[t]
				y''&=\dfrac{(-14x+10)\cdot(x^2-2x-3)^2-(-7x^2+10x-31)\cdot 2\cdot (x^2-2x-3)\cdot (2x-2)}{(x^2-2x-3)^4} \\
				&=\dfrac{2(7x^3-15x^2+93x-77)}{(x^2-2x-3)^3}.
			\end{aligned}$
		\end{enumerate}
	}
\end{bt}
%%==========Bài 5
\begin{bt}%[1D5K5]
	Tìm đạo hàm cấp hai của các hàm số sau:
	\begin{listEX}[3]
		\item $y=\sqrt{2x+1}$.
		\item $y=x^2\cdot \sqrt{x^3-x}$.
		\item $f(x)=\left(x+1\right)^3$.
	\end{listEX}
	\loigiai{
		\begin{enumerate}
			\item $y'=\dfrac{1}{\sqrt{2x+1}}$; $y''=-\dfrac{1}{\sqrt{(2x+1)^3}}$.
			\item $y'=\dfrac{x^2(7x^2-5)}{2\sqrt{x^3-x}}$; $y''=\dfrac{x^2(35x^4-54x^2+15)}{4\sqrt{(x^3-x)^3}}$.
			\item $f'(x)=3\left(x+1\right)^2$;\\ $f''(x)=6\left(x+1\right)$.
		\end{enumerate}
	}
\end{bt}
%%==========Bài 6
\begin{bt}%[1D5B5]
	Tìm đạo hàm cấp hai của các hàm số sau:
	\begin{listEX}[4]
		\item $y=\cos \left(2x-\dfrac{\pi}{3}\right)$.
		\item $y=\sin 2x$.
		\item $y= \sin^2 2x$.
		\item $y=3\sin x+2 \cos x$.
	\end{listEX}
	\loigiai{
		\begin{enumerate}
			\item $y'=-2 \sin\left(2x-\dfrac{\pi}{3}\right)$; $y''=-4 \cos\left(2x-\dfrac{\pi}{3}\right)$.
			\item $y'=2\cos 2x$; $y''=-4\sin 2x$.
			\item $y'=2 \sin2x\left(2 \cos2x\right)=2 \sin4x$; $y''=8 \cos4x$ .
			\item $y=3\sin x+2 \cos x$; $y'=3\cos x-2\sin x$; $y''=-3\sin x-2\cos x$.
		\end{enumerate}
	}
\end{bt}
%%==========Bài 7
\begin{bt}%[1D5K5]
	Tìm đạo hàm cấp hai của các hàm số sau:
	\begin{listEX}[3]
		\item $y=x\cdot\sin x$.
		\item $y=x^2\cdot\cos^2x$.
		\item $y=\dfrac{\cos x}{x^3+1}$.
	\end{listEX}
	\loigiai{
		\begin{enumerate}
			\item $y'=\sin x+x\cos x$; $y''=2\cos x-x\sin x$.
			\item $y'=2x\cos x (\cos x-x\cdot \sin x)$; $y''=(1-2x^2)\cos 2x-4x\sin 2x+1$.
			\item $y'= -\dfrac{\sin x}{x^3+1}-\dfrac{3x^2\cos x}{(x^3+1)^2}$; $y''=\left (-\dfrac{1}{x^3+1}-\dfrac{6x}{(x^3+1)^2}+\dfrac{18x^4}{(x^3+1)^3}\right )\cos x+\dfrac{6x^2\sin x}{(x^3+1)^2}$.
		\end{enumerate}
	}
\end{bt}
%%==========Bài 8
\begin{bt}%[1D5B5]
	Cho hàm số $f(x)=\sin^3x+x^2$. Tính giá trị $f''\left(\dfrac{\pi}{2}\right)$.
	\loigiai{
		$f'(x)=3 \sin^2x \cos x+2x$; $f''(x)=6 \sin x \cos^2x-3 \sin^3x+2 \Rightarrow f''\left(\dfrac{\pi}{2}\right)=-1$.
	}
\end{bt}
\begin{dang}{Ý nghĩa của đạo hàm cấp hai}
\end{dang}
\subsubsection{Ví dụ minh hoạ}
%%==========Ví dụ 8
\begin{vd}%[1T7B2-7]
	Một hòn sỏi rơi tự do có quãng đường rơi tính theo thời gian $t$ là $s(t)=4,9 t^2$, trong đó $s$ tính bằng mét và $t$ tính bằng giây. Tính gia tốc rơi của hòn sỏi lúc $t=3$.
	\loigiai{
		Ta có $s'(t)=\left(4{,}9 t^2\right)'=4{,}9\cdot 2t=9{,}8t$; $s^{\prime\prime}(t)=(9{,}8t)'=9{,}8$.\\
		Gia tốc rơi của hòn sỏi lúc $t=3$ là $s^{\prime\prime}(3)=9{,}8$ m/s$^2$.
	}
\end{vd}
%%==========Ví dụ 9
% \begin{vd}%[1T7B2-7]
% 	Một vật chuyển động thẳng không đều xác định bởi phương trình $s(t)=t^2-4t+3$, trong đó $s$ là quãng đường tính bằng mét và $t$ là thời gian tính bằng giây. Tính gia tốc của chuyển động tại thời điểm $t=4$.
% 	\loigiai{Ta có $s'(t)=\left(t^2-4t+3\right)'=2t-4$, $s^{\prime\prime}(t)=(2t-4)'=2$.\\
% 		Gia tốc của chuyển động tại thời điểm $t=4$ là $s^{\prime\prime}(4)=2$ $ m/s^2$.
% 	} 
% \end{vd}
% %%==========Ví dụ 10
% \begin{vd}%[1D5B5]
% 	Một chuyển động thẳng xác định bởi phương trình $s=t^3-3t^2+5t+2$, trong đó $t$ tính bằng giây và $s$ tính bằng mét. Tính gia tốc của chuyển động tại thời điểm $t=3$ s.
% 	\loigiai{
% 		Gia tốc tức thời của chuyển động tại thời điểm $t$ chính là đạo hàm cấp hai của phương trình chuyển động tại thời điểm $t$.
% 		\begin{align*}
% 			& s'=\left(t^3-3t^2+5t+2\right)'=3t^2-6t+5 \\ 
% 			& s''=6t-6\Rightarrow s''(3)=12.
% 		\end{align*}
% 	}
% \end{vd}
% %%==========Ví dụ 11
% \begin{vd}%[1K9YW-1]
% 	Một vật chuyển động thẳng có phương trình là $s=2t^2+\dfrac{1}{2}t^4$ ($s$ tính bằng mét, $t$ tính bằng giây). Tìm gia tốc của vật tại thời điểm $t=4$ giây.
% 	\loigiai{
% 		Ta có $v(t)=s'(t)=\left(2t^2+\dfrac{1}{2}t^4\right)'=4t+2t^3$.\\
% 		Suy ra $a(t)=v'(t)=\left(4t+2t^3\right)'=4+6t^2$.\\
% 		Do đó gia tốc của vật tại thời điểm $t=4$ giây là
% 		$a(4)=4+6\cdot 4^2=100 \,\text{(m/s}^2) $ .}
% \end{vd}

\subsubsection{Bài tập vận dụng}
%%==========Bài 9
\begin{bt}%[1C7B3-3]
	Xét dao động điều hòa có phương trình chuyển động $ S(t)=A\cos \left(\omega t+\varphi\right) $, trong đó $ A $, $ \omega $, $ \varphi $ là các hằng số. Tìm gia tốc tức thời tại thời điểm $ t $ của chuyển động đó.
	\loigiai{
		Gọi $ v(t) $ là vận tốc tức thời của chuyển động tại thời điểm $ t $, ta có
		\begin{center}
			$ v(t) =s'(t)=\left[A\cos \left(\omega t+\varphi\right)\right]'=-A\sin \left(\omega t+\varphi\right) $.	
		\end{center}
		Gia tốc tức thời của chuyển động tại thời điểm $ t $ là
		\begin{center}
			$ S''(t)=v'(t)=\left[-A\sin \left(\omega t+\varphi\right)\right]'=-A\omega^{2}\cos(\omega t+\varphi) $.
		\end{center}
	}
\end{bt}
%%==========Bài 10
\begin{bt}%[1K9YW-3]
	Chuyển động của một vật gắn trên con lắc lò xo (khi bỏ qua ma sát và sức cản không khí) được cho bởi phương trình sau: $x(t)= 4 \cos \left( 2 \pi t + \dfrac{\pi}{3} \right) $. Trong đó $x$ tính bằng centimet và thời gian $t$ tính bằng giây. Tìm gia tốc tức thời của vật tại thời điểm $t=5$ giây ( làm tròn kết quả đến hàng đơn vị).
	\loigiai{
		\begin{itemize}
			\item Vận tốc tức thời của chuyển động tại thời điểm $t$ là $$v(t)=x'(t)=-\left(2\pi t+\dfrac{\pi}{3}\right)'\cdot 4\sin \left(2\pi t+\dfrac{\pi}{3}\right)=-8\pi\sin \left(2\pi t+\dfrac{\pi}{3}\right). $$
			\item Gia tốc tức thời tại thời điểm $t$ là 
			$$a(t)=v'(t)=-8\pi\left(2\pi t+\dfrac{\pi}{3}\right)'\cdot \cos \left(2\pi t+\dfrac{\pi}{3}\right)=-16\pi^2\cos \left(2\pi t+\dfrac{\pi}{3}\right). $$
			\item Gia tốc của vật tại thời điểm $t=5$ giây.
			$$a(5)=-16\pi^2 \cos \left(10\pi+\dfrac{\pi}{3}\right)=-16\pi^2\cos \dfrac{\pi}
			{3}\approx -79 \text{(cm/s}^2).$$ 
	\end{itemize}}
\end{bt}
%%==========Bài 11
\begin{bt}%[1D5B5]
	Cho chuyển động thẳng xác định bởi phương trình $s=t^3-3t^2-9t+2$ ($t$ tính bằng giây; $s$ tính bằng mét). Tính gia tốc của chuyển động tại thời điểm $t=2$ s.
	\loigiai{
		Gia tốc tức thời của chuyển động tại thời điểm $t$ chính là đạo hàm cấp hai của phương trình chuyển động tại thời điểm $t$.
		\begin{align*}
			& {s}'=\left(t^3-3t^2-9t+2\right)'=3t^2-6t-9 \\ 
			& s''=6t-6\Rightarrow s''(2)=6.
		\end{align*}
	}
\end{bt}
%%==========Bài 12
\begin{bt}%[1D5B5]
	Cho chuyển động thẳng xác định bởi phương trình $s=t^3-3t^2$ ($t$ tính bằng giây; $s$ tính bằng mét). Tính gia tốc của chuyển động tại thời điểm $t=4$ s.
	\loigiai{
		Gia tốc tức thời của chuyển động tại thời điểm $t$ chính là đạo hàm cấp hai của phương trình chuyển động tại thời điểm $t$.
		$$
		s'=3t^2-6t\Rightarrow s''=6t-6\Rightarrow s''(4)=18.
		$$
	}
\end{bt}
\begin{dang}{Chứng minh đẳng thức chứa đạo hàm cấp 2}
	\begin{itemize}
		\item Tìm các đạo hàm đến cấp cao nhất có mặt trong đẳng thức cần chứng minh. 
		\item Thay thế vào vị trí tương ứng và biến đổi vế này cho bằng vế kia. Từ đó suy ra đẳng thức cần chứng minh.
	\end{itemize}
\end{dang}
\subsubsection{Ví dụ minh hoạ}
%%==========Ví dụ 12
\begin{vd}%[1D5B5]
	Cho hàm số $y=\sqrt{2x-x^2}$. Chứng minh rằng: $y^3.{y}''+1=0.$
	\loigiai{
		Ta có: ${y}'=\dfrac{1-x}{\sqrt{2x-x^2}}$, ${y}''=-\dfrac{1}{\sqrt{{\left(2x-x^2\right)}^3}}$\\
		Thay vào: $y^3.{y}''+1=\sqrt{{\left(2x-x^2\right)}^3}\cdot \dfrac{\left(-1\right)}{\sqrt{{\left(2x-x^2\right)}^3}}+1=-1+1=0$ (đpcm).
	}
\end{vd}
%%==========Ví dụ 13
\begin{vd}%[1D5Y5]
	Cho hàm số $y=\dfrac{x^2+2x+2}{2}\cdot $ Chứng minh rằng: $2y.{y}''-1=({y}')^2.$
	\loigiai{
		Ta có:
		${y}'=x+1,{y}''=1$\\
		Thế vào đẳng thức: $2y.{y}''-1=x^2+2x+1=({y}')^2$ (đpcm).
	}
\end{vd}
%%==========Ví dụ 14
\begin{vd}%[1D5K5]
	Cho hàm số $y=x\sin x$. Chứng minh rằng: $x.y-2\left({y}'-\sin x\right)+x.{y}''=0.$
	\loigiai{
		Ta có: ${y}'=\sin x+x\cos x$;
		${y}''=2\cos x-x\sin x$\\
		$VT=x^2\sin x-2\left(\sin x+x\cos x-\sin x\right)+2x\cos x-x^2\sin x$
		$=-2x\cos x+2x\cos x=0=VP$ (đpcm).
	}
\end{vd}
%%==========Ví dụ 15
\begin{vd}%[1D5K5]
	Cho hàm số $y=\dfrac{x+2}{x-1}\cdot$ Chứng minh biểu thức sau không phụ thuộc $x$.\\
	$P=2{\left({{y}'}\right)}^2-{y}''\left(y-1\right)$ (Giả sử các biểu thức đều có nghĩa).
	\loigiai{
		${y}'=\dfrac{-3}{{\left(x-1\right)}^2}\Rightarrow 2{\left({{y}'}\right)}^2=\dfrac{18}{{\left(x-1\right)}^4}$\\
		${y}''=-3\cdot \dfrac{-2\left(x-1\right)}{{\left(x-1\right)}^4}=\dfrac{6}{{\left(x-1\right)}^3}$\\
		$y-1=\dfrac{3}{x-1}\Rightarrow{y}''\left(y-1\right)=\dfrac{18}{{\left(x-1\right)}^4}$\\
		$P=2{\left({{y}'}\right)}^2-{y}''\left(y-1\right)=\dfrac{18}{{\left(x-1\right)}^4}-\dfrac{18}{{\left(x-1\right)}^4}=0$\\
		Vậy đẳng thức được chứng minh xong.
	}
\end{vd}
%%==========Ví dụ 16
\begin{vd}%[1D5G5]
	Cho hàm số $y=\tan x.$ Chứng minh rằng: $\dfrac{6y}{{{y}''}}-\dfrac{1}{{{y}'}}-\cos 2x=1.$
	\loigiai{
		${y}'=\dfrac{1}{\cos^2x}=1+\tan^2x$; ${y}''=\dfrac{2\sin x}{\cos^3x}=2\tan x\left(1+\tan^2x\right)$\\
		\noindent Do đó:\\ 
		\noindent	$\dfrac{6y}{{{y}''}}-\dfrac{1}{{{y}'}}-\cos 2x =\dfrac{6\tan x}{2\tan x\left(1+\tan^2x\right)}-\dfrac{1}{1+\tan^2x}-\cos 2x$ $ =\dfrac{2}{1+\tan^2x}-\cos 2x=\\
		=2\cos^2x-\left(\cos^2x-\sin^2x\right)=1$ (đpcm).
	}
\end{vd}
\subsubsection{Bài tập áp dụng}
%%==========Bài 13
\begin{bt}%[1D5B5]
	Chứng minh rằng hàm số $y=\sqrt{4x-2x^2}$ thỏa hệ thức: $y^3{y}''+4=0.$
	\loigiai{
		${y}'=\dfrac{2-2x}{\sqrt{4x-2x^2}};{y}''=\dfrac{-4}{{\left(\sqrt{4x-2x^2}\right)}^3}$\\
		$VT={\left(\sqrt{4x-2x^2}\right)}^3\cdot \dfrac{-4}{{\left(\sqrt{4x-2x^2}\right)}^3}+4=0=VP$ (đpcm).
	}
\end{bt}
%%==========Bài 14
\begin{bt}%[1D5Y5]
	Cho hàm số $y=-2+\dfrac{5}{x}\cdot $ Chứng minh rằng: $\dfrac{2{y}'}{x}+{y}''=0.$
	\loigiai{
		${y}'=-\dfrac{5}{x^2}; {y}''=\dfrac{10}{x^3}$\\
		$\dfrac{2{y}'}{x}+{y}''=-\dfrac{10}{x^3}+\dfrac{10}{x^3}=0.$
	}
\end{bt}
%%==========Bài 15
\begin{bt}%[1D5B5]
	Cho $y=\dfrac{x-3}{x+4}.$ Chứng minh rằng: $2{\left({{y}'}\right)}^2=\left(y-1\right){y}''.$
	\loigiai{
		$y=\dfrac{x-3}{x+4}\Rightarrow {y}'=\dfrac{7}{{\left(x+4\right)}^2}\Rightarrow {y}''=-\dfrac{14}{{\left(x+4\right)}^3}\cdot$\\
		Ta có vế trái: $2{\left({{y}'}\right)}^2=\dfrac{98}{{\left(x+4\right)}^4}\cdot $\\
		Và vế phải:$\left(y-1\right){y}''=\left(\dfrac{x-3}{x+4}-1\right)\left[\dfrac{-14}{{\left(x+4\right)}^3}\right]=\dfrac{98}{{\left(x+4\right)}^4}\cdot $\\
		Vậy $2{\left({{y}'}\right)}^2=\left(y-1\right){y}''.$
	}
\end{bt}
%%==========Bài 16
\begin{bt}%[1D5K5]
	Cho hàm số $y=x\cos x.$ Chứng minh rằng: $x.y-2({y}'-\cos x)+x.{y}''=0.$
	\loigiai{
		${y}'=\cos x-x\sin x;{y}''=-2\sin x-x\cos x$ \\
		$VT=x.y-2\left({y}'-\cos x\right)+x.{y}''=x.x\cos x-2\left(\cos x-x\sin x-\cos x\right)+x\left(-2\sin x-x\cos x\right)=$\\$=x^2\cos x+2x\sin x-2x\sin x-x^2\cos x=0=VP$ (đpcm).
	}
\end{bt}
%%==========Bài 17
\begin{bt}%[1D5K5]
	Cho hàm số $y=x\sin x.$ Chứng minh $xy-2{y}'+x{y}''=-2\sin x.$
	\loigiai{
		${y}'=\sin x+x\cos x;{y}''=2\cos x-x\sin x$\\
		$xy-2{y}'+x{y}''=x^2\sin x-2(\sin x+x\cos x)+x(2\cos x-x\sin x)=-2\sin x.$
	}
\end{bt}
%%==========Bài 18
\begin{bt}%[1D5K5]
	Cho hàm số $y=\sin^2x$. Chứng minh rằng: $2y+{y}'\tan x+{y}''-2=0.$
	\loigiai{
		$ {y}'=2\sin x\cos x;{y}''=2\cos^2x-2\sin^2x$ \\ 
		$2y+{y}'\tan x+{y}''-2=0 \\ 
		\Leftrightarrow 2\sin^2x+2\sin x.\cos x.\dfrac{\sin x}{\cos x}+2\cos^2x-2\sin^2x -2=0 \\ 
		\Leftrightarrow 2\sin^2x+2\cos^2x-2=0\Leftrightarrow 0=0$ (đúng). 
	}
\end{bt}
%%%%%%%%%%%%%%%%%%%%
\subsection{Bài tập rèn luyện}
\begin{bt}%[1K9BV-1]
Tính đạo hàm của các hàm số sau:
\begin{listEX}[2]
	\item $y=x^3-3x^2+2x+1$;
	\item $y=x^2-4\sqrt{x}+3$.
\end{listEX}
\loigiai{
\begin{enumerate}
	\item Ta có $y'=3x^2-6x+2.$
	\item Ta có $y'=2x-4\cdot \dfrac{1}{2\sqrt{x}}=2x-\dfrac{2}{\sqrt{x}}.$
\end{enumerate}
}
\end{bt}
\begin{bt}%[1K9BV-1]
Tính đạo hàm của các hàm số sau:
\begin{listEX}[2]
	\item $y=\dfrac{2x-1}{x+2}$;
	\item $y=\dfrac{2x}{x^2+1}$.
\end{listEX}
\loigiai{
\begin{enumerate}
	\item Ta có $y'=\dfrac{2(x+2)-(2x-1)}{(x+2)^2}=\dfrac{5}{(x+2)^2}.$
	\item Ta có $y'=\dfrac{2(x^2+1)-2x\cdot 2x}{(x^2+1)^2}=\dfrac{2-2x^2}{(x^2+1)^2}.$
\end{enumerate}
}
\end{bt}
\begin{bt}%[1T7Y2-1]
	Tính đạo hàm của các hàm số sau
	\begin{listEX}[2]
	\item $y=2x^3-\dfrac{x^2}{2}+4x-\dfrac{1}{3}$;
	\item $y=\dfrac{-2x+3}{x-4}$;
	\item $y=\dfrac{x^2-2x+3}{x-1}$;
	\item $y=\sqrt{5x}$.
	\end{listEX}
	\loigiai{\begin{enumerate}
	\item $y=2x^3-\dfrac{x^2}{2}+4x-\dfrac{1}{3}$.
	\begin{eqnarray*}
	y'&=&\left(2x^3-\dfrac{x^2}{2}+4x-\dfrac{1}{3}\right)'=2\left(x^3\right)'-\dfrac{1}{2} \left(x^2\right)'+4(x)'-0
	=2\cdot 3x^2-\dfrac{1}{2}\cdot 2x+4\cdot 1-0\\&=&6x^2-x+4.
	\end{eqnarray*}
	\item $y=\dfrac{-2x+3}{x-4}$.
	\begin{eqnarray*}
	y'&=&\left(\dfrac{-2x+3}{x-4}\right)'=\dfrac{(-2x+3)'(x-4)-(-2x+3)(x-4)'}{(x-4)^2}=\dfrac{-2(x-4)-(-2x+3)1}{(x-4)^2}\\
	&=&\dfrac{5}{(x-4)^2}.
	\end{eqnarray*}
	\item $y=\dfrac{x^2-2x+3}{x-1}$.
	\begin{eqnarray*}
	y'&=&\left(\dfrac{x^2-2x+3}{x-1}\right)'=\dfrac{\left(x^2-2x+3\right)'(x-1)-\left(x^2-2x+3\right)(x-1)'}{(x-1)^2}\\
	&=&\dfrac{(2x-2)(x-1)-\left(x^2-2x+3\right)1}{(x-1)^2}=\dfrac{\left(2x^2-4x+2\right)-(x^2-2x+3)}{(x-1)^2}\\
	&=&\dfrac{x^2-2x-1}{(x-1)^2}.
	\end{eqnarray*}
	\item $y=\sqrt{5x}$.\\
	$y'=\left(\sqrt{5x}\right)'=\dfrac{(5x)'}{2\sqrt{5x}}=\dfrac{5}{2\sqrt{5x}}=\dfrac{\sqrt{5}}{2\sqrt{x}}$.
	\end{enumerate}
	}
\end{bt}
\begin{bt}%[1T7B2-1]
	Tính đạo hàm của các hàm số sau
	\begin{listEX}[3]
	\item $y=\left(x^2-x\right)\cdot 2^x$;
	\item $y=x^2\cdot \log_3 x$;
	\item $y=\mathrm{e}^{3x+1}$.
	\end{listEX}
	\loigiai{
	\begin{enumerate}
	\item $y=\left(x^2-x\right)\cdot 2^x$.
	\begin{eqnarray*}
	y'&=&\left[\left(x^2-x\right)\cdot 2^x\right]'=\left(x^2-x\right)'\cdot 2^x+\left(x^2-x\right)\cdot\left( 2^x\right)'\\
	&=&(2x-1)\cdot 2^x+
	\left(x^2-x\right)\cdot 2^x\cdot \ln 2\\
	&=&2^x\left[(2x-1)+(x^2-x)\ln 2\right].
	\end{eqnarray*}
	\item $y=x^2\cdot \log_3 x$.
	\begin{eqnarray*}
	y'&=&\left[x^2\cdot \log_3 x\right]'=\left(x^2\right)'\cdot \log_3x+x^2\cdot \left(\log_3 x\right)'=2x\cdot \log_3x+x^2\cdot \dfrac{1}{x\cdot \ln 3}\\
	&=&2x\cdot \log_3x+\dfrac{x}{\ln 3}=x \left(2\log_3x+\dfrac{1}{\ln 3}\right).
	\end{eqnarray*}
	\item $y=\mathrm{e}^{3x+1}$.\\
	$y'=\left(\mathrm{e}^{3x+1}\right)'=(3x+1)'\cdot \mathrm{e}^{3x+1}=3\cdot \mathrm{e}^{3x+1}$.
	\end{enumerate}
	}
\end{bt}
\begin{bt}%[1T7B2-2]
	Tính đạo hàm của các hàm số sau
	\begin{listEX}[2]
	\item $y=2x^4-5x^2+3$;
	\item $y=x\cdot \mathrm{e}^x$.
	\end{listEX}
	\loigiai{\begin{enumerate}
	\item $y=2x^4-5x^2+3$.\\
	Ta có $y'=\left(2x^4-5x^2+3\right)'=8x^3-10x$.
	\item $y=x\cdot \mathrm{e}^x$.\\
	Ta có $y'=\left(x\cdot \mathrm{e}^x\right)'=x'\cdot \mathrm{e}^x+x \cdot \left(\mathrm{e}^x\right)'=1\cdot \mathrm{e}^x+x\cdot \mathrm{e}^x=(1+x)\cdot \mathrm{e}^x$.
	\end{enumerate}
	}
\end{bt}
\begin{bt}%[1T7B2-1]
	Tính đạo hàm của các hàm số sau
	\begin{listEX}[4]
	\item $y=\sin 3x$;
	\item $y=\cos^32x$;
	\item $y=\tan^2x$;
	\item $y=\cot(4-x^2)$.
	\end{listEX}
	\loigiai{
	\begin{enumerate}
	\item $y=\sin 3x$.\\
	$y'=\left(\sin 3x\right)'=\left(3x\right)'\cdot \cos 3x=3\cos 3x$.
	\item $y=\cos^32x$.\\
	$y'=\left[\left(\cos 2x\right)^3\right]^\prime=3\left(\cos 2x\right)^2\cdot \left(\cos 2x\right)'=3\cos^2 2x \cdot (2x)'\cdot\left(-\sin 2x\right)=-6\cos^2 2x\cdot \sin 2x$.
	\item $y=\tan^2x$.\\
	$y'=\left[\left(\tan x\right)^2\right]'=2\tan x\cdot \left(\tan x\right)'=2\tan x\cdot\dfrac{1}{\cos^2x}=\dfrac{2\tan x}{\cos^2x}$.
	\item $y=\cot(4-x^2)$.\\
	$y'=\left[\cot (4-x^2)\right]'=-\dfrac{\left(4-x^2\right)'}{\sin^2(4-x^2)}=\dfrac{2x}{\sin^2(4-x^2)}$.
	\end{enumerate}
	}
\end{bt}
\begin{bt}%[1C7B2-1]
	Tìm đạo hàm của mỗi hàm số sau
	\begin{enumEX}{2}
	\item $y=4 x^3-3 x^2+2 x+10$;
	\item $y=\dfrac{x+1}{x-1}$;
	\item $y=-2 x \sqrt{x}$;
	\item $y=3 \sin x+4 \cos x-\tan x$;
	\item $y=4^x+2 \mathrm{e}^x$;
	\item $y=x \ln x$.	
	\end{enumEX}
	\loigiai{
	\begin{enumerate}
	\item $y=4 x^3-3 x^2+2 x+10\Rightarrow y'=(4x^{3})'-(3x^2)'+(2x)'+(10)'=12x^2-6x+2$.
	\item $y=\dfrac{x+1}{x-1}\Rightarrow y'=\dfrac{(x+1)'(x-1)-(x+1)(x-1)'}{(x-1)^2}=\dfrac{x-1-(x+1)}{(x-1)^2}=\dfrac{-2}{(x-1)^2}$.
	\item $y=-2 x \sqrt{x}\Rightarrow y'=(-2x)'\cdot \sqrt{x}-2x\cdot (\sqrt{x})'=-2\sqrt{x}-2x\cdot \dfrac{1}{2\sqrt{x}}=-2\sqrt{x}-\sqrt{x}=-3\sqrt{x}$.
	\item $y=3 \sin x+4 \cos x-\tan x\Rightarrow y'=(3 \sin x)'+(4 \cos x)'-(\tan x)'=3\cos x -4\sin x -\dfrac{1}{\cos^2x}$.
	\item $y=4^x+2 \mathrm{e}^x\Rightarrow y'=(4^x)'+(2 \mathrm{e}^x)'=4^{x}\ln 4+2\mathrm{e}^x$.
	\item $y=x \ln x\Rightarrow y'=(x)'\ln x+x(\ln x)'=\ln x+x\cdot \dfrac{1}{x}=\ln x +1$.	
	\end{enumerate}
	}
\end{bt}
\begin{bt}%[1C7B2-1]
	Cho hàm số $f(x)=2^{3 x+2}$.
	\begin{enumerate}
	\item Hàm số $f(x)$ là hàm hợp của các hàm số nào?
	\item Tìm đạo hàm của $f(x)$. 	
	\end{enumerate}
	\loigiai{
	\begin{enumerate}
	\item Đặt $3x+2$, ta có $f(x)=2^{u}$.\\
	Vậy $f(x)=2^{3 x+2}$ là hàm hợp của hai hàm số $f(x)=2^{u}$, $u=3x+2$.	
	\item Ta có $f'(x)=\left(2^{3 x+2}\right)'=(3x+2)'\cdot 2^{3 x+2}\cdot \ln 2=3\cdot 2^{3 x+2}\cdot \ln 2 $.	
	\end{enumerate}	
	}
\end{bt}
%--------------------------------------
% Bài 5
\begin{bt}%[1C7B2-1]
	Tìm đạo hàm của mỗi hàm số sau:
	\begin{enumEX}{2}
	\item $y=\sin 3 x+\sin^2 x$;
	\item $y=\log_2(2 x+1)+3^{-2 x+1}$.	
	\end{enumEX}
	\loigiai{
	\begin{enumerate}
	\item Ta có$y=\sin 3 x+\sin^2 x=\sin 3x + \dfrac{1}{2}(1-\cos 2x)$.\\
	Do đó $y'=(\sin 3x)'+\dfrac{1}{2}(1-\cos 2x)'=3\cos 3x +\sin 2x$.
	\item Ta có $y=\log_2(2 x+1)+3^{-2 x+1}$.\\
	Do đó
	\allowdisplaybreaks 
	\begin{eqnarray*}
	y'&=&\left(\log_2(2x+1)\right)'+\left(3^{-2x+1}\right)'\\
	&=&\dfrac{(2x+1)'}{(2x+1)\ln 2}+(-2x+1)'\cdot 3^{-2x+1}\cdot \ln 3\\
	&=& \dfrac{2}{(2x+1)\ln 2}-2\cdot 3^{-2 x+1}\cdot \ln 3.
	\end{eqnarray*}	
	\end{enumerate}	
	}
\end{bt}
%-------------------------------------
\begin{bt}%[1K9BV-1]
Tính đạo hàm của các hàm số sau:
\begin{listEX}[4]
	\item $y=x\sin^2 x$;
	\item $y=\cos^2x+\sin2x$;
	\item $y=\sin3x-3\sin x$;
	\item $y=\tan x+\cot x$.
\end{listEX}
\loigiai{
\begin{enumerate}
	\item Ta có $y'=\sin^2x+x\cdot 2\sin x\cdot \cos x.$
	\item Ta có $y'=-2\cos x\cdot \sin x+2\cos 2x=-\sin 2x+2\cos 2x.$
	\item Ta có $y'=3\cos 3x-3\cos x.$
	\item Ta có $y'=\dfrac{1}{\cos^2 x}-\dfrac{1}{\sin^2 x}=-\dfrac{\cos 2x}{\dfrac{1}{4}\sin^2 2x}=-\dfrac{4\cos 2x}{\sin^2 2x}.$
\end{enumerate}
}
\end{bt}
\begin{bt}%[1K9BV-1]
Tính đạo hàm của các hàm số sau:
\begin{listEX}[2]
	\item $y=2^{3x-x^2}$;
	\item $y=\log_{3}x$.
\end{listEX}
\loigiai{
\begin{enumerate}
\item Ta có $y'=(3-2x)\ln 2 \cdot 2^{3x-x^2}.$
\item Ta có $y'=\dfrac{1}{x\ln 3}.$
\end{enumerate}
}
\end{bt}
\begin{bt}%[1K9KV-1]
Cho hàm số $f(x)=2\sin^2\left(3x-\dfrac{\pi}{4}\right)$. Chứng minh rằng $\left|f'(x)\right|\leq 6$ với mọi $x$.
\loigiai{
Ta có $$f'(x)=4\sin\left(3x-\dfrac{\pi}{4}\right)\cdot \cos\left(3x-\dfrac{\pi}{4}\right)\cdot \left(3x-\dfrac{\pi}{4}\right)'=6\sin2\left(3x-\dfrac{\pi}{4}\right)=6\sin\left(6x-\dfrac{\pi}{2}\right)=-6\cos 6x.$$
Do $\left|\cos6x\right|\leq 1$ với mọi $x\in \mathbb{R}$ suy ra $\left|-6\cos6x\right|\leq 6$ với mọi $x\in \mathbb{R}$.\\
Hay $\left|f'(x)\right|\leq 6$ với mọi $x$ (điều phải chứng minh).
}
\end{bt}
\begin{bt}%[1K9BU-3]
Một vật chuyển động rơi tự do có phương trình $h(t)=100-4{,}9t^2$, ở đó độ cao $h$ so với mặt đất tính bằng mét và thời gian $t$ tính bằng giây. Tính vận tốc của vật: 
\begin{enumerate}
	\item Tại thời điểm $t=5$ giây;
	\item Khi vật chạm đất.
\end{enumerate}
\loigiai{
Một vật chuyển động rơi tự do có phương trình $h(t)=100-4{,}9t^2$ suy ra vận tốc tức thời của vật tại thời điểm $t$ là $v(t)=h'(t)=-9{,}8t$.\\
Khi vật chạm đất thì $h(t)=0$ suy ra $100-4{,}9t^2=0\Leftrightarrow t=\pm \dfrac{10\sqrt{10}}{7}$.\\
Do $t>0$ suy ra $t=\dfrac{10\sqrt{10}}{7}$.
\begin{enumerate}
	\item Tại thời điểm $t=5$ giây thì vận tốc của vật là $v(5)=-9{,}8\cdot 5=-49$ (m/s).
	\item Khi vật chạm đất tức là tại thời điểm $t=\dfrac{10\sqrt{10}}{7}$ có vận tốc là $$v\left(\dfrac{10\sqrt{10}}{7}\right)=-9{,}8\cdot \dfrac{10\sqrt{10}}{7}=-14\sqrt{10}\, \left(\mathrm{m/s}\right).$$
\end{enumerate}
}
\end{bt}
\begin{bt}%[1K9BU-3]
Chuyển động của một hạt trên một dây rung được cho bởi $s(t)=12+0{,}5\sin(4\pi t)$, trong đó $s$ được tính bằng centimét và $t$ tính bằng giây. Tính vận tốc của hạt sau $t$ giây. Vận tốc cực đại của hạt là bao nhiêu?
\loigiai{
\begin{enumerate}
	\item Chuyển động của một hạt trên một dây rung được cho bởi $s(t)=12+0{,}5\sin(4\pi t)$ suy ra vận tốc tức thời của hạt tại thời điểm $t$ là $$v(t)=s'(t)=2\pi\cos (4\pi t).$$
	\item Do $\cos (4\pi t)\leq 1$ với mọi $t\in \mathbb{R}$ suy ra $2\pi\cos (4\pi t)\leq 2\pi$ với mọi $t\in \mathbb{R}$.\\
	Vậy vận tốc cực đại của hạt là $2\pi$.
\end{enumerate}
}
\end{bt}
%--------------------------------------
% Bài 2
\begin{bt}%[1C7Y2-1]
	Cho $u=u(x), v=v(x), w=w(x)$ là các hàm số có đạo hàm tại điểm $x$ thuộc khoảng xác định. Chứng minh rằng $(u \cdot v \cdot w)'=u' \cdot v \cdot w+u \cdot v' \cdot w+u \cdot v \cdot w'$.
	\loigiai{
	Ta có
	\[	VT=(u \cdot v \cdot w)'=u'\cdot (v\cdot w)+u\cdot (v\cdot w)'=u'\cdot v\cdot w + u\cdot v'\cdot w +u\cdot v \cdot w'=VP.	\]
	}
\end{bt}
% Bài 6
\begin{bt}%[1C7B2-3]
	Viết phương trình tiếp tuyến của đồ thị mỗi hàm số sau:
	\begin{enumerate}
	\item $y=x^3-3 x^2+4$ tại điểm có hoành độ $x_0=2$;
	\item $y=\ln x$ tại điểm có hoành độ $x_0=\mathrm{e}$;
	\item $y=\mathrm{e}^x$ tại điểm có hoành độ $x_0=0$.	
	\end{enumerate}
	\loigiai{
	\begin{enumerate}
	\item Ta có $y=x^3-3 x^2+4\Rightarrow y'=3x^2-6x$.\\
	Khi đó $f'(2)=3\cdot 2^{2}-6\cdot 2=0$.\\
	Phương trình tiếp tuyến của đồ thị hàm số tại điểm $M(2,0)$ là $y=0\cdot (x-2)+0$ hay $y=0$.
	\item Ta có $y=\ln x\Rightarrow y'=\dfrac{1}{x}$.\\
	Khi đó $f'(\mathrm{e})=\dfrac{1}{\mathrm{e}}$.\\
	Phương trình tiếp tuyến của đồ thị hàm số tại điểm $N(\mathrm{e},1)$ là $y=\dfrac{1}{\mathrm{e}}(x-\mathrm{e})+1$ hay $y=\dfrac{1}{\mathrm{e}}x$.
	\item Ta có $y=\mathrm{e}^x\Rightarrow y'=\mathrm{e}^{x}$.\\
	Khi đó $f'(0)=1$.\\
	Phương trình tiếp tuyến của đồ thị hàm số tại điểm $P(0,1)$ là $y=1\cdot (x-0)+1$ hay $y=x+1$.	
	\end{enumerate}	
	}
\end{bt}
%--------------------------------------
% Bài 7
\begin{bt}%[1C7T2-6]
	Một viên đạn được bắn lên từ mặt đất theo phương thẳng đứng với tốc độ ban đầu $v_0=196$ m/s (bỏ qua sức cản của không khí). Tìm thời điểm tại đó tốc độ của viên đạn bằng $0$. Khi đó viên đạn cách mặt đất bao nhiêu mét (lấy $g=9,8$ m/s$^2$)?
	\loigiai{
	Cho $Oy$ theo phương thẳng đứng, chiều dương hướng từ mặt đất lên trời, gốc $O$ là vị trí viên đạn bắn lên, khi đó phương trình chuyển động của viên đạn là
	\[y=v_0 t-\dfrac{1}{2} g t^2.\]
	Ta có vận tốc tại thời điểm $t$ là
	\[v=y'(t)=v_0-gt.\]
	Do đó 
	\[v=0 \Leftrightarrow v_0-g t=0 \Leftrightarrow t=\dfrac{v_0}{g}=\frac{196}{9{,}8}=20\,\, (\text{s}).\]
	Vậy khi $t = 20$ s thì viên đạn bắt đầu rơi, lúc đó viên đạn cách mặt đất
	\[
	y=v_0 t-\dfrac{1}{2} g t^2=196\cdot 20-\dfrac{1}{2} 9{,}8\cdot 20^2=1960\,\,\text{m}.
	\]	
	}
\end{bt}
%--------------------------------------
% Bài 8
\begin{bt}%[1C7T2-6]
	Cho mạch điện như hình bên. Lúc đầu tụ điện có điện tích $Q_0$.
	\immini{
	Khi đóng khoá $K$, tụ điện phóng điện qua cuộn dây; điện tích $q$ của tụ điện phụ thuộc vào thời gian $t$ theo công thức $q(t)=Q_0 \sin \omega t$, trong đó $\omega$ là tốc độ góc. Biết rằng cường độ $I(t)$ của dòng điện tại thời điểm $t$ được tính theo công thức $I(t)=q'(t)$. Cho biết $Q_0=10^{-8}$ (C) và $\omega=10^6 \pi$ (rad/s). \\
	Tính cường độ của dòng điện tại thời điểm $t=6$(s) (tính chính xác đến $10^{-5}$ (mA)).	
	}{
	\begin{tikzpicture}[>=stealth,line join=round,line cap=round,font=\footnotesize,scale=1]
	\path[draw] (0,.25)node[left]{$+$}--(0,-.25) (0,0)--++(180:2)--++(-90:2)--++(0:1.6)coordinate (A)++(0:1.2)coordinate (B)--++(0:1.5)--++(90:.75)coordinate (C)++(90:.5)coordinate (D)--++(90:.75)--++(180:2)coordinate (E)--++(90:.2)node[right]{$-$}--++(-90:.4) ;
	\draw [decoration={aspect=0.2, segment length=1mm, amplitude=2mm,coil},decorate] (A) -- (B);
	\draw[fill=black] (C)circle(1.2pt) (D)circle(1.2pt)--($(D)+(-60:.5)$);	
	\end{tikzpicture}
	}
	\loigiai{
	Ta có $q'(t)=(Q_{0}\sin \omega t)'=Q_0\cdot \omega\cdot \cos \omega t$.	\\
	Cường độ của dòng điện tại thời điểm $t=6$ (s) là
	\[I(6)=10^{-8}\cdot 10^{6}\pi \cdot\cos(10^{6}\pi \cdot 6)=\dfrac{\pi}{100}\,\,(\text{A})=31{,}4159\,\,(\text{mA}).\] 
	}
\end{bt}
\begin{bt}%[1T7B2-7]
	Cân nặng trung bình của một bé gái trong độ tuổi từ $0$ đến $36$ tháng có thể được tính gần đúng bởi hàm số $w(t)=0,000758t^3-0,0596t^2+1,82t+8,15$, trong đó $t$ được tính bằng tháng và $w$ được tính bằng pound (\textit{nguồn:https://wwww.cdc.gov.growthcharts/data/who/GrChrt-Boys}). Tính tốc độ thay đổi cân nặng của bé gái đó tại thời điểm $ 10$ tháng tuổi.
	\begin{center}
	%\includegraphics[scale=0.6]{DataCTST/images/hinh-2}
	\begin{tikzpicture}[scale=0.5,font=\scriptsize,>=stealth]
	\draw[->] (-1,0)--(8.2,0) node [below]{$t$};
	\draw[->] (0,-1.4)--(0,7.3) node [left]{$w$};
	\node at (0,0) [below left=-2pt]{$O$};
	\foreach \x in {5,10,...,35}
	\draw (\x/5,2pt)--(\x/5,-2pt)node[below=-2pt]{$\x$};
	\foreach \y in {5,10,...,30}
	\draw (2pt,\y/5)--(-2pt,\y/5)node[left=-2pt]{$\y$};
	\node at (4,-1.1){Tháng tuổi};
	\node at (-1.5,3.6)[rotate=90]{Cân nặng trung bình (pound)};
	\node at (-2,3)[left]{\includegraphics[scale=0.9]{HINHVE/CTST/CTST-7_1_2}};
	\draw (0,1.7)..controls +(66:2) and +(-160:1.5)..(3,5)
	..controls +(20:2) and +(-160:1.5)..(7.3,6.2);
	\end{tikzpicture}
	\end{center}
	\loigiai{
	Ta có	$w'(t)=0,000758\cdot 3t^2-0,0596\cdot 2t+1,82$.\\
	Tốc độ thay đổi cân nặng của bé gái đó tại thời điểm $ 10$ tháng tuổi là $w'(10)=0,8554$ pound/tháng.
	}
\end{bt}
\begin{bt}%[1T7B2-7]
	Một công ty xác định rằng tổng chi phí của họ, tính theo nghìn đô-la, để sản xuất $x$ mặt hàng là $C(x)=\sqrt{5x^2+60}$ và công ty lên kế hoạch nâng sản lượng trong $t$ tháng kể từ nay theo hàm số $x(t)=20t+40$. Chi phí sẽ tăng thế nào sau $4$ tháng kể từ khi công ty thực hiện kế hoạch đó?
	\loigiai{Ta có $C'(x)=\dfrac{(5x^2+60)'}{2\sqrt{2x^2=60}}=\dfrac{5x}{\sqrt{5x^2+50}}$.\\
	Công ty thực hiện kế hoạch nâng sản lượng sau $4$ tháng có sản lượng là $x(4)=20\cdot 4+40=120$.\\
	Chi phí sẽ tăng sau $4$ tháng kể từ khi công ty thực hiện kế hoạch là $C'(120)\simeq 2,235$.	}
\end{bt}
\begin{bt}%[1T7B2-7]
	Trên Mặt Trăng, quãng đường rơi tự đo của một vật được cho bởi công thức $s(t)=0,81t^2$, trong đó $t$ là thời gian được tính bằng giây và $s$ tính bằng mét. Một vật thả rơi từ độ cao $200~m$ phía trên Mặt Trăng. Tại thời điểm $t=2$ sau khi thả vật đó, tính quãng đường vật đã rơi.
	\loigiai{Ta có $s'(t)=1,62t$.\\
	Quãng đường vật đã rơi tại thời điểm $t=2$ sau khi thả vật rơi từ độ cao $200$ m phía trên Mặt Trăng là $s(2)=0,81\cdot 2^2=3,24$ m.
	}
\end{bt}
%----------------
%%==========Bài 19
\begin{bt}%[1K9YW-1]
	Cho hàm số $f(x)=x^2\cdot \mathrm{\, e}^x$. Tính $f''(0)$.
	\loigiai{$f'(x)=2x\cdot \mathrm{\, e}^x+x^2\cdot \mathrm{\, e}^x=\mathrm{\, e}^x\cdot (x^2+2x)$.\\
		$f''(x)=\mathrm{\, e}^x\cdot (x^2+2x)+(2x+2)\cdot \mathrm{\, e}^x=\mathrm{\, e}^x\cdot \left(x^2+4x+2\right)$.\\ 
		$f''(0)=\mathrm{\, e}^0\cdot \left(0^2+4\cdot 0+2\right)=2.$
	} 
\end{bt}
%%==========Bài 20
\begin{bt}%[1K9YW-1]
	Tính đạo hàm cấp hai của các hàm số sau
	\begin{listEX}[2]
		\item $y=\ln (x+1)$.
		\item $y=\tan 2x$.
	\end{listEX}
	\loigiai{ 
		\begin{itemize}
			\item $y'=\dfrac{(x+1)'}{x+1}=\dfrac{1}{x+1}$.\\
			$y''=-\dfrac{(x+1)'}{(x+1)^2}=-\dfrac{1}{(x+1)^2}$.
			\item $y'=\dfrac{(2x)'}{\cos^2 2x}=\dfrac{2}{\cos^2 2x}$.\\
			$y''=-\dfrac{2 (\cos^2 2x)'}{\cos^4 2x}=\dfrac{-2\cdot 2 (2x)'\cdot \cos 2x \cdot (-\sin 2x)}{\cos^4 2x}=\dfrac{4\cdot \sin 4x}{\cos^ 4 2x}$.
		\end{itemize}
	}
\end{bt}
%%==========Bài 21
\begin{bt}%[1C7Y3-1]
	Tìm đạo hàm cấp hai của mỗi hàm số sau
	\begin{enumEX}{3}
		\item $ y=\dfrac{1}{2x+3}$.
		\item $ y=\log_{3}x $.
		\item $ y=2^{x} $.
	\end{enumEX}
	\loigiai{
		\begin{enumerate}
			\item $ y=\dfrac{1}{2x+3}$.\\
			$\Rightarrow y'=-\dfrac{(2x+3)'}{(2x+3)^{2}}=-\dfrac{2}{(2x+3)^{2}}$.\\
			$ \Rightarrow y''=\left[-\dfrac{2}{(2x+3)^{2}}\right]'=\dfrac{2\left[(2x+3)^{2}\right]'}{(2x+3)^{4}}=\dfrac{8(2x+3)}{(2x+3)^{4}}=\dfrac{8}{(2x+3)^{3}} $.
			\item $ y=\log_{3}x $.\\
			$\Rightarrow y'=\dfrac{1}{x\ln 3} $.\\
			$\Rightarrow y''=\left(\dfrac{1}{x\ln 3}\right)'=-\dfrac{(x\ln 3)'}{(x\ln 3)^{2}} =-\dfrac{\ln 3}{(x\ln 3)^{2}} =-\dfrac{1}{x^{2}\ln 3}$.
			\item $ y=2^{x} $.\\
			$ \Rightarrow y'=2^{x}\ln 2 $.\\
			$ \Rightarrow y''=\left(2^{x}\ln 2\right)'=2^{x}\ln ^{2}2 $.
		\end{enumerate}
	}
\end{bt}
%%==========Bài 22
\begin{bt}%[1C7B3-1]
	Tính đạo hàm cấp hai của mỗi hàm số sau
	\begin{enumerate}
		\item $ y=3x^{2}-4x+5 $ tại điểm $ x_{0}=-2 $.
		\item $ y=\log_{3}(2x+1) $ tại điểm $ x_{0}=3 $.
		\item $ y=\mathrm{e}^{4x+3} $ tại điểm $ x_{0}=1 $.
		\item $ y=\sin\left(2x+\dfrac{\pi}{3}\right) $ tại điểm $ x_{0}=\dfrac{\pi}{6} $.
		\item $ y=\cos \left(3x-\dfrac{\pi}{6}\right) $ tại điểm $ x_{0}=0 $.
	\end{enumerate}
	\loigiai{
		\begin{enumerate}
			\item $ y=3x^{2}-4x+5 $ tại điểm $ x_{0}=-2 $.\\
			Ta có $ y'=6x-4 \Rightarrow y''=6$.
			\item $ y=\log_{3}(2x+1) $ tại điểm $ x_{0}=3 $.\\
			Ta có $ y'= \dfrac{2}{(2x+1)\ln 3}\Rightarrow y''=\dfrac{-2\left[(2x+1)\ln 3\right]'}{\left[(2x+1)\ln 3\right]^{2}}=-\dfrac{4\ln 3}{\left[(2x+1)\ln 3\right]^{2}} = - \dfrac{4}{(2x+1)^2\ln 3}$.\\
			Suy ra $ y''(3)=-\dfrac{4\ln 3}{\left[(2\cdot 3+1)\ln 3\right]^{2}}=-\dfrac{4}{49\ln 3} $.
			\item $ y=\mathrm{e}^{4x+3} $ tại điểm $ x_{0}=1 $.\\
			Ta có $ y'=4\mathrm{e}^{4x+3} \Rightarrow y''=16\mathrm{e}^{4x+3}$.\\
			\item $ y=\sin\left(2x+\dfrac{\pi}{3}\right) $ tại điểm $ x_{0}=\dfrac{\pi}{6} $.\\
			Ta có $ y'=2\cos \left(2x+\dfrac{\pi}{3}\right) \Rightarrow y''=-4\sin\left(2x+\dfrac{\pi}{3}\right) $.\\
			Suy ra $ y''\left(\dfrac{\pi}{6}\right)= -4\sin \left(2\cdot\dfrac{\pi}{6}+\dfrac{\pi}{3}\right)=-2\sqrt{3} $.
			\item $ y=\cos \left(3x-\dfrac{\pi}{6}\right) $ tại điểm $ x_{0}=0 $.\\
			Ta có $ y'=-3\sin \left(3x-\dfrac{\pi}{6}\right) \Rightarrow y''= -9\cos \left(3x-\dfrac{\pi}{6}\right)$.\\
			Suy ra $ y''\left(0\right)=-9\cos \left(3\cdot 0-\dfrac{\pi}{6}\right)=\dfrac{-9\sqrt{3}}{2} $.
		\end{enumerate}
	}
\end{bt}
%%==========Bài 23
\begin{bt}%[1K9BW-1]
	Cho hàm số $P(x)=ax^2+bx+3$, ($a$, $b$ là các hằng số ). Tìm $a$, $b$ biết $P'(1)=0$, $P''(1)=-2$.
	\loigiai{$P'(x)=2ax+b$, $P''(x)=2a$.\\
		Giải hệ $\heva{&2a\cdot 1+b=0\\&2a=-2}\Leftrightarrow \heva{&a=-1\\&b=2.}$\\
		Vậy $a=-1,\, b=2$.} 
\end{bt}
%%==========Bài 24
\begin{bt}%[1K9KW-1]
	Cho hàm số $f(x)=2\sin ^2 \left(x+\dfrac{\pi}{4}\right)$. Chứng minh rằng $\bigg|f''(x)\bigg|\leq 4$ với mọi $x$.
	\loigiai{
		$f'(x)=2\cdot 2\cdot \left(x+\dfrac{\pi}{4}\right)'\sin \left(x+\dfrac{\pi}{4}\right)\cdot \cos \left(x+\dfrac{\pi}{4}\right)=2\cdot \sin \left(2x+\dfrac{\pi}{2}\right)$.\\
		$f''(x)=2\cdot \left(2x+\dfrac{\pi}{2}\right)'\cdot \cos \left(2x+\dfrac{\pi}{2}\right)=4\cos \left(2x+\dfrac{\pi}{2}\right)$\\
		$\Rightarrow \bigg|f''(x)\bigg|=4\bigg|\cos \left(2x+\dfrac{\pi}{2}\right)\bigg|$.\\
		Ta có $0\leq \bigg|\cos\left(2x+\dfrac{\pi}{2}\right)\bigg|\leq 1\Leftrightarrow 0\leq 4\bigg|\cos \left(2x+\dfrac{\pi}{2}\right)\bigg|\leq 4 $.
	}
\end{bt}
%%==========Bài 25
\begin{bt}%[1D5K5]
	Cho hàm số $y=\cos^24x$. Chứng minh rằng: $32\left(2y-1\right)+{y}''=0.$
	\loigiai{
		${y}'=2\cos 4x.{\left(\cos 4x\right)}'\Rightarrow {y}'=-8\cos 4x.\sin 4x\Rightarrow {y}'=-4\sin 8x$\\
		${y}''=-32\cos 8x$\\
		$VT=32\left(2y-1\right)+{y}''=32\left(2\cos^24x-1\right)-32\cos 8x=32\cos 8x-32\cos 8x=0=VP.$
	}
\end{bt}
%%==========Bài 26
\begin{bt}%[1D5G5]
	Cho hàm số $y=x\tan x$. Chứng minh rằng: $x^2{y}''-2(x^2+y^2)(1+y)=0.$
	\loigiai{
		$y'=\tan x+x+x\tan^2x$\\
		${y}''=1+\tan^2x+1+\tan^2x+2x\tan x.(1+\tan^2x)=2+2\tan^2x+2x\tan x+2x\tan^3x$\\
		$VT=x^2(2+2\tan^2x+2x\tan x+2x\tan^3x)-2(x^2+x^2\tan^2x)(1+x\tan x)=$ \\ 
		$ =2x^2+2x^2\tan^2x+2x^3\tan x+2x^3\tan^3x-2x^2-2x^3\tan x-2x^2\tan^2x-2x^3\tan^3x=0=VP. $ 
	}
\end{bt}
%%==========Bài 27
\begin{bt}%[1D5K5]
	Cho hàm số $y=\dfrac{\sin^3x+\cos^3x}{1-\sin x\cos x}\cdot $ Chứng minh rằng : ${y}''+y=0.$
	\loigiai{
		Ta có: $y=\dfrac{\left(\sin x+\cos x\right)\left(\sin^2x+\cos^2x-\sin x\cos x\right)}{1-\sin x\cos x}=\sin x+\cos x$\\
		$\Rightarrow {y}'=\cos x-\sin x;{y}''=-\sin x-\cos x$\\
		$\Rightarrow {y}''+y=0.$
	}
\end{bt}
%%==========Bài 28
\begin{bt}%[1K9YW-3]
	Phương trình chuyển động của một hạt được cho bởi công thức $s(t)=10+0{,}5\sin \left(2\pi t+\dfrac{\pi}{5}\right)$, trong đó $s$ tính bằng centimét, $t$ tính bằng giây. Gia tốc của hạt tại thời điểm $t=5$ giây (làm tròn kết quả đến chữ số thập phân thứ nhất).
	\loigiai{ $v(t)=s'(t)=\left(2\pi t+\dfrac{\pi}{5}\right)'\cdot 0{,}5 \cos \left(2\pi t+\dfrac{\pi}{5}\right)=\pi \cos \left(2\pi t+\dfrac{\pi}{5}\right).$\\
		$a(t)=v'(t)=-\pi \cdot \left(2\pi t+\dfrac{\pi}{5}\right)'\cdot \sin \left(2\pi t+\dfrac{\pi}{5}\right)=-2\pi^2 \sin \left(2\pi t+\dfrac{\pi}{5}\right) $.\\ 
		Gia tốc của hạt tại thời điểm $t=5$ giây là $a(5)=-2\pi^2 \sin \left(2\pi 5+\dfrac{\pi}{5}\right)\approx -11{,}6 \, \text{(m/s}^2).$	
	}
\end{bt}
%%==========Bài 29
\begin{bt}%[1C7B3-3]
	Một vật rơi tự do theo phương thẳng đứng có phương trình $ s=\dfrac{1}{2}gt^{2} $, trong đó $ g $ là gia tốc rơi tự do, $ g\approx 9,8 $ m/s$ ^{2} $.
	\begin{enumerate}
		\item Tính vận tốc tức thời của vật tại thời điểm $ t_{0}=2 $ (s).
		\item Tính gia tốc tức thời của vật tại thời điểm $ t_{0}=2 $ (s).
	\end{enumerate}
	\loigiai{
		\begin{enumerate}
			\item Phương trình vận tốc của vật $ v(t)= s'(t)=gt $.\\
			Vận tốc tức thời của vật tại thời điểm $ t_{0}=2 $ là $ v(2)=9,8\cdot 2= 19,6 $ (m/s).
			\item Phương trình gia tốc của vật $ a(t)=v'(t)=g$. Do đó $a(2)=9,8 $ (m/s$^{2}$).
		\end{enumerate}
	}
\end{bt}
%%==========Bài 30
\begin{bt}%[1C7K3-3]
	Một chất điểm chuyển động theo phương trình $ s(t)=t^{3}-3t^{2}+8t+1 $, trong đó $ t>0 $, $ t $ tính bằng giây và $ s(t) $ tính bằng mét. Tìm vận tốc tức thời, gia tốc tức thời của chất điểm
	\begin{enumerate}
		\item Tại thời điểm $ t=3 $ (s).
		\item Tại thời điểm mà chất điểm di chuyển được $ 7 $ (m).
	\end{enumerate}
	\loigiai{
		\begin{enumerate}
			\item Phương trình vận tốc của vật $ v(t)=s'(t)=3t^{2} -6t+8$.\\
			Vận tốc tại thời điểm $ t=3 $ là $ v(3)=3\cdot 3^{2}-6\cdot 3+8=17 $ (m/s).\\
			Phương trình gia tốc của vật $ a(t)=v'(t)= 6t-6$.\\
			Gia tốc tại thời điểm $ t=3 $ là $ a(3)=6\cdot 3-6 =12$ (m/s$ ^{2} $).
			\item Tại thời điểm chất điểm di chuyển được $ 7 $m nên ta có\\
			$ t^{3}-3t^{2}+8t+1=7 \Leftrightarrow t^{3}-3t^{2}+8t-6=0\Leftrightarrow t=1$.\\
			Vận tốc tại thời điểm $ t=1 $ là $ v(1)=3\cdot 1^{2}-6\cdot 1+8= 5 $ (m/s).\\
			Gia tốc tại thời điểm $ t=1 $ là $ a(1)=6\cdot 1-6 =0$ (m/s$^{2} $).
		\end{enumerate}
	}
\end{bt}
%%==========Bài 31
\begin{bt}%[1C7K3-3]
	\immini{
	Một con lắc lò xo dao động điều hòa theo phương ngang trên mặt phẳng không ma sát như hình 7, có phương trình chuyển động $ x=4\sin t $, trong đó $ t $ tính bằng giây và $ x $ tính bằng centimét.
	\begin{enumerate}
		\item Tìm vận tốc tức thời và gia tốc tức thời của con lắc tại thời điểm $ t $ (s).
		\item Tìm vị trí, vận tốc tức thời và gia tốc tức thời của con lắc tại thời điểm $ t=\dfrac{2\pi}{3} $ (s). Tại thời điểm đó, con lắc di chuyển theo hướng nào?
	\end{enumerate}
	}{
	\begin{tikzpicture}[line cap=round,line join=round,>=triangle 45,x=1cm,y=1cm,scale=.7]
			%	\draw[decoration={aspect=0.3, segment length=1.5mm, amplitude=3mm,coil},decorate] (0,1) -- (4,1); 
			\draw [smooth,domain=3.13:15*pi,scale=0.1,samples=100] plot (\x, {4*sin(\x r)+7});
			\fill [pattern = north east lines] (-0.2,0) rectangle (0,4);
			\draw[fill=gray] (4.9,1.5)--(5.5,1.5)--(5.5,0)--(4.9,0)--cycle;
			\draw[->](0,0)--(6,0) ;
			\draw[->] (2.5,2.5)--(2.5,1.2);
			\draw[thick] (0,0) -- (0,4) (2.5,0.1)--(2.5,-0.1) (5.2,0.1)--(5.2,-0.1);
			\draw (2.5,3) node {vị trí cân bằng}
			(2.5,-0.3) node {$O$}
			(5.2,-0.3) node {$x$}
			(6,-0.3) node {$x$}
			(3,-1) node {Hình 7}
			;
			\draw (0,0.7)--(0.3,0.7) (4.72,0.7)--(4.9,0.7);
		\end{tikzpicture}
	}
	\loigiai{
		\begin{enumerate}
			\item Phương trình vận tốc của vật $ v(x)=s'(x)=4\cos t $ (cm/s).\\
			Vận tốc tức thời của vật ở thời điểm $ t $ là $ v(t)=4\cos t $ (cm/s).\\
			Phương trình gia tốc của vật $ a(x)=v'(x)=-4\sin t $ (cm/s$ ^{2} $).\\
			Gia tốc tức thời của vật ở thời điểm $ t $ là $ a(t)=-4\sin t $ (cm/s$ ^{2} $).
			\item Tại thời điểm $ t=\dfrac{2\pi}{3} $ ta có\\
			Vật di chuyển được quãng đường $ x=4\sin \dfrac{2\pi}{3} = 2\sqrt{3}$ (cm).\\
			Vận tốc tức thời tại thời điểm $ t=\dfrac{2\pi}{3} $ là $ v\left(\dfrac{2\pi}{3}\right)=4\cos \dfrac{2\pi}{3}=-2 $ (cm/s).\\
			Gia tốc tức thời tại điểm $ t=\dfrac{2\pi}{3} $ là $ a\left(\dfrac{2\pi}{3}\right)=-4\sin \dfrac{2\pi}{3} =-2\sqrt{3} $.\\
			Tại thời điểm đó, vật di chuyển theo hướng ngược lại với phương $ Ox $.
		\end{enumerate}
	}
\end{bt}