\section{KNTT - Lớp 11 - Ôn tập giữa học kì 1 - Đề 4}

\caulc
\Opensolutionfile{ans}[ans-ABCD-KNTT-Lop11-OTGK1-De4]

\begin{ex}%[KNTT- Lớp 11 - Ôn tập giữa học kì 1 - Đề 4]%[Lê Hồng Phi]%[1D1H2-2]
Cho $\sin \alpha =\dfrac{3}{5}$ và $\dfrac{\pi}{2}<\alpha <\pi$. Tính giá trị $\cos \alpha $.
\choice
{\True $\cos \alpha =-\dfrac{4}{5}$}
{$\cos \alpha =\dfrac{4}{5}$}
{$\cos \alpha =\dfrac{2\sqrt{5}}{5}$}
{$\cos \alpha =-\dfrac{2\sqrt{5}}{5}$}
\loigiai{
Ta có $\dfrac{\pi}{2}<\alpha <\pi $ nên $\cos \alpha <0$.\\
Vậy
$\cos\alpha=-\sqrt{1-\sin^2\alpha} =-\dfrac{4}{5}$.
}
\end{ex}
\begin{ex}%[KNTT- Lớp 11 - Ôn tập giữa học kì 1 - Đề 4]%[Lê Hồng Phi]%[1D1N3-2]
Công thức nào đúng trong các công thức dưới đây:
\choice
{$\sin (a+b)=\sin a\cos b-\cos a\sin b$}
{\True $\cos (a+b)=\cos a\cos b-\sin a\sin b$}
{$\sin (a-b)=\sin a\cos b+\cos a\sin b$}
{$\cos (a-b)=\cos a\cos b-\sin a\sin b$}
\loigiai{
Công thức đúng là $\cos (a+b)=\cos a\cos b-\sin a\sin b$.}
\end{ex}
\begin{ex}%[KNTT- Lớp 11 - Ôn tập giữa học kì 1 - Đề 4]%[Lê Hồng Phi]%[1D1N4-5]
Xác định chu kỳ tuần hoàn của hàm số $y=\sin x$.
\choice
{$k2\pi,k\in \mathbb{Z}$}
{$\pi$}
{\True $2\pi$}
{$\dfrac{2\pi}{3}$}
\loigiai{
Hàm số $y=\sin x$ tuần hoàn với chu kỳ $2\pi$.}
\end{ex}
\begin{ex}%[KNTT- Lớp 11 - Ôn tập giữa học kì 1 - Đề 4]%[Lê Hồng Phi]%[1D1N5-3]
Phương trình $\cos x=-\dfrac{\sqrt{3}}{2}$ có nghiệm là
\choice
{$x=\pm \dfrac{\pi}{3}+k\pi;k\in \mathbb{Z}$}
{$x=\pm \dfrac{\pi}{6}+k\pi; k\in \mathbb{Z}$}
{\True $x=\pm \dfrac{5\pi}{6}+k2\pi; k\in \mathbb{Z}$}
{$x=\pm \dfrac{\pi}{3}+k2\pi; k\in \mathbb{Z}$}
\loigiai{
Ta có $\cos x=-\dfrac{\sqrt{3}}{2}\Leftrightarrow \cos x=\cos \left(\dfrac{5\pi}{6}\right)\Leftrightarrow x=\pm \dfrac{5\pi}{6}+k2\pi $, $k\in \mathbb{Z}$.}
\end{ex}
\begin{ex}%[KNTT- Lớp 11 - Ôn tập giữa học kì 1 - Đề 4]%[Lê Hồng Phi]%[1D1V5-5]
Tổng tất cả các nghiệm thuộc đoạn $\left[0;\pi\right]$ của phương trình $2\sin x-\sqrt{3}=0$ là
\choice
{\True $\pi$}
{$\dfrac{\pi}{3}$}
{$\dfrac{2\pi}{3}$}
{$\dfrac{4\pi}{3}$}
\loigiai{
Ta có $2\sin x-\sqrt{3}=0\Leftrightarrow $ $\sin x=\dfrac{\sqrt{3}}{2}\Leftrightarrow \sin x=\sin \dfrac{\pi}{3}$ $\Leftrightarrow \hoac{&x=\dfrac{\pi}{3}+k2\pi \\&x=\dfrac{2\pi}{3}+k2\pi}$, $k\in \mathbb{Z}$.\\
Các nghiệm của phương trình thuộc đoạn $\left[0;\pi\right]$ là $x=\dfrac{\pi}{3}$; $x=\dfrac{2\pi}{3}$.\\
Vậy tổng tất cả các nghiệm thuộc $\left[0;\pi\right]$ của phương trình là $\dfrac{\pi}{3}+\dfrac{2\pi}{3}=\pi$.}
\end{ex}
\begin{ex}%[KNTT- Lớp 11 - Ôn tập giữa học kì 1 - Đề 4]%[Lê Hồng Phi]%[1D2N1-3]
Cho dãy số $\left(u_n\right)$ có số hạng tổng quát $u_n=\dfrac{n}{2^n-1}$ (với $n\in{\mathbb{N}^{*}}$). Hai số hạng đầu tiên của dãy số đã cho lần lượt là
\choice
{$u_1=\dfrac{1}{2};u_2=\dfrac{2}{3}$}
{$u_1=1;u_2=\dfrac{1}{2}$}
{$u_1=u_2=\dfrac{1}{4}$}
{\True $u_1=1;u_2=\dfrac{2}{3}$}
\loigiai{
Ta có $u_1=\dfrac{1}{2^1-1}=1$; $u_2=\dfrac{2}{2^2-1}=\dfrac{2}{3}$.}
\end{ex}
\begin{ex}%[KNTT- Lớp 11 - Ôn tập giữa học kì 1 - Đề 4]%[Lê Hồng Phi]%[1D2N2-4]
Cho cấp số cộng $\left(u_n\right)$ với $u_1=9$ và công sai $d=2$. Giá trị của $u_2$ bằng
\choice
{\True $11$}
{$\dfrac{9}{2}$}
{$18$}
{$7$}
\loigiai{
Ta có $u_2=u_1+d=9+2=11$.}
\end{ex}
\begin{ex}%[KNTT- Lớp 11 - Ôn tập giữa học kì 1 - Đề 4]%[Lê Hồng Phi]%[1D2N3-4]
Cho cấp số nhân $\left(u_n\right)$ với $u_1=\dfrac{1}{2}$ và công bội $q=2$. Giá trị của $u_{10}$ bằng
\choice
{\True $2^8$}
{$2^9$}
{$\dfrac{1}{2^{10}}$}
{$\dfrac{37}{2}$}
\loigiai{
Ta có $\heva{&u_1=\dfrac{1}{2} \\&q=2}\Rightarrow {u_{10}}=u_1q^9=\dfrac{1}{2}\cdot 2^9=2^8$.}
\end{ex}
\begin{ex}%[KNTT- Lớp 11 - Ôn tập giữa học kì 1 - Đề 4]%[Lê Hồng Phi]%[1D2N3-6]
Cho cấp số nhân $\left(u_n\right)$ với $u_1=3$, $q=-2$. Tính tổng $10$ số hạng đầu tiên của cấp số nhân đã cho.
\choice
{$S_{10}=-3069$}
{\True $S_{10}=-1023$}
{$S_{10}=3069$}
{$S_{10}=1023$}
\loigiai{
Ta có $S_{10}=u_1\cdot \dfrac{1-q^{10}}{1-q}=3\cdot\dfrac{1-(-2)^{10}}{1+2}=-1023$.}
\end{ex}

\begin{ex}%[KNTT- Lớp 11 - Ôn tập giữa học kì 1 - Đề 4]%[Lê Hồng Phi]%[1D5N1-1]
Cho $4$ mẫu số liệu như sau:\\
Mẫu $1$: Số tiền mà $60$ sinh viên chi cho thanh toán cước điện thoại trong tháng:
\begin{center}
\begin{tabular}{|c|c|c|c|c|c|}
\hline Số tiền (nghìn đồng) & {$[0 ; 50)$} & {$[50 ; 100)$} & {$[100 ; 150)$} & {$[150 ; 200)$} & {$[200 ; 250)$} \\
\hline Số sinh viên & 5 & 12 & 23 & 17 & 3 \\
\hline
\end{tabular}
\end{center}
	Mẫu $2$: Thống kê nhiệt độ tại một địa điểm trong $40$ ngày, ta có bảng số liệu sau:
\begin{center}
\begin{tabular}{|c|c|c|c|c|}
\hline Nhiệt độ $\left({ }^{\circ} \mathrm{C}\right)$ & {$[19 ; 22)$} & {$[22 ; 25)$} & {$[25 ; 28)$} & {$[28 ; 31)$} \\
\hline Số ngày & 7 & 15 & 12 & 6 \\
\hline
\end{tabular}
\end{center}
Mẫu $3$: Số sản phẩm một công nhân làm được trong một ngày được cho như sau:
\begin{center}
\begin{tabular}{lllllllllllll}
18 & 25 & 39 & 12 & 54 & 27 & 46 & 25 & 19 & 8 & 36 & 22 & \\
20 & 19 & 17 & 44 & 5 & 18 & 23 & 28 & 25 & 34 & 46 & 27 & 16
\end{tabular}
\end{center}
Mẫu $4$: Thời gian ra sân (giờ) của một cựu cầu thủ ở giải ngoại hạng Anh qua các thời kì được cho như sau:
\begin{center}$\begin{array}{llllllllllllllll}653 & 632 & 609 & 572 & 565 & 535 & 516 & 514 & 508 & 505 & 504 & 504 & 503 & 499 & 496 & 492\end{array}$\end{center}
	Trong 4 mẫu số liệu nói trên, có bao nhiêu mẫu số liệu cho dưới dạng mẫu số liệu ghép nhóm?
\choice
{\True $2$}
{$1$}
{$3$}
{$4$}
\loigiai{
Mẫu $1$ và $2$ là các mẫu số liệu ghép nhóm}
\end{ex}

\begin{ex}%[KNTT- Lớp 11 - Ôn tập giữa học kì 1 - Đề 4]%[Lê Hồng Phi]%[1D5N1-3]
Tìm hiểu thời gian hoàn thành một bài tập (đơn vị: phút) của một số học sinh thu được kết quả sau:
\begin{center}\begin{tabular}{|l|c|c|c|c|c|}
\hline Thời gian (phút) & {$[0 ; 4)$} & {$[4 ; 8)$} & {$[8 ; 12)$} & {$[12 ; 16)$} & {$[16 ; 20)$} \\
\hline Số học sinh & 2 & 4 & 7 & 4 & 3 \\
\hline Giá trị đại diện & 2 & 6 & 10 & 14 & 18 \\
\hline
\end{tabular}
\end{center}
Thời gian trung bình (phút) để hoàn thành bài tập của các em học sinh là
\choice
{$7$}
{$11{,}3$}
{\True $10{,}4$}
{$12{,}5$}
\loigiai{
Thời gian trung bình (phút) để hoàn thành bài tập của các em học sinh là
$$\overline{x}=\dfrac{2\cdot 2+4\cdot 6+7\cdot 10+4\cdot 14+3\cdot 18}{20}=10{,}4 \text{ (phút)}.$$ }
\end{ex}
\begin{ex}%[KNTT- Lớp 11 - Ôn tập giữa học kì 1 - Đề 4]%[Lê Hồng Phi]%[1D5H2-3]
Doanh thu bán hàng trong $20$ ngày được lựa chọn ngẫu nhiên của một của hàng được ghi lại ở bảng sau (đơn vị: triệu đồng):
\begin{center}\begin{tabular}{|c|c|c|c|c|c|}
\hline Doanh thu & {$[5 ; 7)$} & {$[7 ; 9)$} & {$[9 ; 11)$} & {$[11 ; 13)$} & {$[13 ; 15)$} \\
\hline Số ngày & 2 & 7 & 7 & 3 & 1 \\
\hline
\end{tabular}
\end{center}
Tứ phân vị thứ ba của mẫu số liệu gần nhất với giá trị nào trong các giá trị dưới đây?
\choice
{$10$}
{\True $11$}
{$12$}
{$13$}
\loigiai{
Cỡ mẫu là $n=20$.\\
Gọi $x_1$, $x_2$,\ldots, $x_{20}$ là doanh thu bán hàng trong 20 ngày xếp theo thứ tự không giảm.\\
Khi đó $x_1,x_2\in \left[5;7\right)$, $x_3,\ldots,x_9\in \left[7;9\right)$, $x_{10},\ldots,x_{16}\in \left[9;11\right)$, $x_{17},\ldots,x_{19}\in \left[11;13\right)$, $x_{20}\in \left[13;15\right)$.\\
Tứ phân vị thứ ba $Q_3$ của mẫu số liệu là $\dfrac{x_{15}+x_{16}}{2}$. Vì $x_{15}$, $x_{16}$ đều thuộc nhóm $\left[9;11\right)$ nên nhóm này chứa $Q_3$.\\
Do đó $p=3$, $a_3=9$, $m_3=7$, $m_1+m_2=9$, $a_4-a_3=2$.\\
Vậy $Q_3=9+\dfrac{\dfrac{3\cdot 20}{4}-\left(2+7\right)}{7}\cdot\left(11-9\right)=\dfrac{75}{7}\approx 11$.
}
\end{ex}

\Closesolutionfile{ans}

\indapan{6}{ans-ABCD-KNTT-Lop11-OTGK1-De4}

\cauds

\Opensolutionfile{ans}[ans-DS-KNTT-Lop11-OTGK1-De4]

\begin{ex}%[KNTT- Lớp 11 - Ôn tập giữa học kì 1 - Đề 4]%[Lê Hồng Phi]%[1D1V2-2]
Cho $\tan \alpha =2$.
\choiceTF[t]
{\True $\cot \alpha =\dfrac{1}{2}$}
{\True $\sin 2\alpha =\dfrac{4}{5}$}
{Biết $\dfrac{3\pi}{2}<\alpha <\dfrac{5\pi}{2}$. Khi đó $\sin \alpha =-\dfrac{2\sqrt{5}}{5}$}
{Giá trị biểu thức $T=\dfrac{\sin^3\alpha +\cos^3\alpha}{\cos \alpha +\cos 3\alpha}$ là $-\dfrac{1}{2}$}
\loigiai{
\begin{itemchoice}
\itemch {\bfseries Đúng.}\\
Do $\cot \alpha =\dfrac{1}{\tan \alpha}=\dfrac{1}{2}$.
\itemch {\bfseries Đúng.}\\
Vì $\tan \alpha =2\Rightarrow \sin \alpha =2\cos \alpha $ nên $\sin 2\alpha =2\sin \alpha \cos \alpha =4\cos^2\alpha $.\\
Ta lại có $1+\tan^2\alpha =\dfrac{1}{\cos^2\alpha}\Rightarrow {\cos^2}\alpha =\dfrac{1}{1+\tan^2\alpha}=\dfrac{1}{5}$.\\
Vậy $\sin 2\alpha =\dfrac{4}{5}$.
\itemch {\bfseries Sai.}\\
Do $\dfrac{3\pi}{2}<\alpha <\dfrac{5\pi}{2}$ và $\tan \alpha =2$ nên suy ra $\sin \alpha >0,\cos \alpha >0$.\\
Từ $\cos^2\alpha =\dfrac{1}{5}\Rightarrow \cos \alpha =\dfrac{1}{\sqrt{5}}$.\\
Vậy $\sin \alpha =2\cos \alpha =\dfrac{2\sqrt{5}}{5}$.
\itemch {\bfseries Sai.}\\
Do $\tan \alpha =2$ nên $\sin \alpha =2\cos \alpha $. Khi đó
\begin{align*}
 T&=\dfrac{\sin^3\alpha +\cos^3\alpha}{\cos \alpha +\cos 3\alpha}=\dfrac{\left(\sin \alpha +\cos \alpha\right)\left(\sin^2\alpha -\sin \alpha \cos \alpha +\cos^2\alpha\right)}{2\cos \alpha \cos 2\alpha} \\
& =\dfrac{3\cos \alpha \left(1-\dfrac{1}{2}\sin 2\alpha\right)}{2\cos \alpha \cos 2\alpha}=\dfrac{3\left(1-\dfrac{1}{2}\sin 2\alpha\right)}{2\cos 2\alpha}. \\
\end{align*}
Ta lại có $\sin 2\alpha =\dfrac{4}{5}$, $\cos 2\alpha =2\cos^2\alpha -1=2\cdot\dfrac{1}{5}-1=-\dfrac{3}{5}$ nên $$T=\dfrac{3\left(1-\dfrac{1}{2}\cdot\dfrac{4}{5}\right)}{2\left(-\dfrac{3}{5}\right)}=-\dfrac{3}{2}.$$
\end{itemchoice}
}
\end{ex}

\begin{ex}%[KNTT- Lớp 11 - Ôn tập giữa học kì 1 - Đề 4]%[Lê Hồng Phi]%[1D1V5-3]
Cho hàm số $f(x)=2\sin \left(x-\dfrac{\pi}{4}\right)+1$.
\choiceTF[t]
{\True Hàm số $y=\sin x$ là hàm số tuần hoàn với chu kì $2\pi $}
{Phương trình $f(x)=0$ có nghiệm là $x=-\dfrac{\pi}{6}+k2\pi$ $\left(k\in \mathbb{Z}\right)$ và\hfill\break  $x=\dfrac{7\pi}{6}+k2\pi$ $\left(k\in \mathbb{Z}\right)$}
{\True Tập xác định của hàm số $y=\dfrac{f(x)}{\sin x}$ là $\mathbb{R} \setminus \left\{k\pi,k\in \mathbb{Z}\right\}$}
{Giá trị lớn nhất của hàm số $f(x)$ trên đoạn $\left[-\dfrac{\pi}{4};\dfrac{\pi}{4}\right]$ bằng $3$}
\loigiai{
\begin{itemchoice}
\itemch {\bfseries Đúng.}\\
Ta có hàm số $y=\sin x$ là hàm số tuần hoàn với chu kì $2\pi$.
\itemch {\bfseries Sai.}\\
Ta có \begin{align*}
f(x)=0&\Leftrightarrow 2\sin \left(x-\dfrac{\pi}{4}\right)+1=0\\
&\Leftrightarrow \sin \left(x-\dfrac{\pi}{4}\right)=-\dfrac{1}{2}\\
&\Leftrightarrow \sin \left(x-\dfrac{\pi}{4}\right)=\sin \left(-\dfrac{\pi}{6}\right)\\
&\Leftrightarrow \hoac{&x-\dfrac{\pi}{4}=-\dfrac{\pi}{6}+k2\pi \\&x-\dfrac{\pi}{4}=\dfrac{7\pi}{6}+k2\pi}\left(k\in \mathbb{Z}\right)\\
&\Leftrightarrow \hoac{&x=\dfrac{\pi}{12}+k2\pi \\&x=\dfrac{17\pi}{12}+k2\pi}\left(k\in \mathbb{Z}\right).
\end{align*}
\itemch {\bfseries Đúng.}\\
Hàm số $y=\dfrac{f(x)}{\sin x}=\dfrac{2\sin \left(x-\dfrac{\pi}{4}\right)+1}{\sin x}$ có điều kiện xác định là $$\sin x\ne 0\Leftrightarrow x\ne k\pi \left(k\in \mathbb{Z}\right).$$
Vậy tập xác định của hàm số $y=\dfrac{f(x)}{\sin x}$ là $\mathbb{R} \setminus \left\{k\pi,k\in \mathbb{Z}\right\}$.
\itemch {\bfseries Sai.}\\
Với mọi $x\in \left[-\dfrac{\pi}{4};\dfrac{\pi}{4}\right]$, ta có \begin{align*}
x-\dfrac{\pi}{4}\in \left[-\dfrac{\pi}{2};0\right]&\Leftrightarrow \sin \left(x-\dfrac{\pi}{4}\right)\in \left[-1;0\right]\\
&\Leftrightarrow 2\sin \left(x-\dfrac{\pi}{4}\right)\in \left[-2;0\right]\\
&\Leftrightarrow 2\sin \left(x-\dfrac{\pi}{4}\right)+1\in \left[-1;1\right].
\end{align*}
Vậy giá trị lớn nhất của hàm số $f(x)$ trên đoạn $\left[-\dfrac{\pi}{4};\dfrac{\pi}{4}\right]$ bằng $1$.
\end{itemchoice}
}
\end{ex}

\begin{ex}%[KNTT- Lớp 11 - Ôn tập giữa học kì 1 - Đề 4]%[Lê Hồng Phi]%[1D2H3-6]
Cho dãy số $\left(u_n\right)$ là cấp số nhân được cho bởi hệ thức truy hồi $u_1=8$ và $u_{n+1}=4u_n$ với $n\ge 1$.
\choiceTF[t]
{\True Công bội của cấp số nhân là $q=4$}
{Số hạng thứ $9$ của câp số nhân là $131\,072$}
{Số $2\,048$ là số hạng thứ $6$ của cấp số nhân}
{Kết quả tổng $5$ số hạng đầu của cấp số nhân là một nghiệm của phương trình bậc hai $3x^2+8\,186x+5\,456=0$}
\loigiai{
\begin{itemchoice}
\itemch {\bfseries Đúng.}\\
Ta có công bội của cấp số nhân là $q=\dfrac{u_{n+1}}{u_n}=4$.
\itemch {\bfseries Sai.}\\
Số hạng thứ $9$ của cấp số nhân là $u_9=u_1q^{n-1}=8\cdot 4^8=524\,288$.
\itemch {\bfseries Sai.}\\
Ta có $u_n=u_1q^{n-1}$ mà $u_n=2\,048\Rightarrow {q^{n-1}}=256=4^4\Rightarrow n=5$.
\itemch {\bfseries Sai.}\\
Ta có $S_n=\dfrac{u_1\left(1-q^n\right)}{1-q}\Rightarrow S_5=\dfrac{8\left(1-4^5\right)}{1-4}=2\,728$.\\
Phương trình $3x^2+8186x+5456=0\Leftrightarrow \hoac{&x=-\dfrac{2}{3} \\&x=-2\,728.}$\\
Suy ra $2\,728$ không là nghiệm của phương trình $3x^2+8\,186x+5\,456=0$.
\end{itemchoice}
}
\end{ex}

\begin{ex}%[KNTT- Lớp 11 - Ôn tập giữa học kì 1 - Đề 4]%[Lê Hồng Phi]%[1D5V2-3]
Mẫu số liệu ghép nhóm về doanh thu của $50$ cửa hàng của một công ty trong một tháng (đơn vị: triệu đồng) được cho như sau:
\begin{center}
\begin{tabular}{|c|c|c|c|c|c|}
\hline Doanh thu & {$[26 ; 48)$} & {$[48 ; 70)$} & {$[70 ; 92)$} & {$[92 ; 114)$} & {$[114 ; 136)$} \\
\hline Số cửa hàng & 6 & 14 & 17 & 8 & 5 \\
\hline
\end{tabular}
\end{center}
\choiceTF[t]
{\True Cỡ của mẫu số liệu trên là $n=50$}
{Doanh thu trung bình của mỗi cửa hàng của công ty trong tháng đó là 81 triệu đồng}
{\True Số trung vị của mẫu số liệu ghép nhóm là $M_e\approx 76{,}5$}
{Tứ phân vị thứ ba của mẫu số liệu trên là $Q_3=95$}
\loigiai{
\begin{itemchoice}
\itemch {\bfseries Đúng.}\\
Cỡ mẫu là $n=6+14+17+8+5=50$.
\itemch {\bfseries Sai.}\\
Ta có \begin{center}
\begin{tabular}{|l|c|c|c|c|c|}
\hline Doanh thu & {$[26 ; 48)$} & {$[48 ; 70)$} & {$[70 ; 92)$} & {$[92 ; 114)$} & {$[114 ; 136)$} \\
\hline Tần số & 6 & 14 & 17 & 8 & 5 \\
\hline Giá trị đại diện & 37 & 59 & 81 & 103 & 125 \\
\hline
\end{tabular}
\end{center}
Doanh thu trung bình là $$\overline{x}=\dfrac{37\cdot 6+59\cdot 14+81\cdot 17+103\cdot 8+125\cdot 5}{50}=77{,}48\ (\text{triệu đồng}).$$
\itemch {\bfseries Đúng.}\\
Gọi $x_1;x_2;..;x_{50}$ là doanh thu của $50$ cửa hàng xếp theo thứ tự không giảm.\\
Trung vị của mẫu số liệu $x_1;x_2;..;x_{50}$ là $\dfrac{x_{25}+x_{26}}{2}\in \left[70;92\right)$.\\
Ta có $n=50$, $n_p=17$, $u_p=70$, $u_{p+1}=92$ nên
$$M_e=70+\dfrac{\dfrac{50}{2}-\left(6+14\right)}{17}\left(92-70\right)\approx 76{,}5.$$
\itemch {\bfseries Sai.}\\
Xét nửa mẫu số liệu bên phải $x_{26};x_{27};..;x_{50}$ có trung vị là $x_{38}$. Do đó, mẫu số liệu $x_1;x_2;..;x_{50}$ có tứ phân vị thứ $3$ là $x_{38}\in \left[92;114\right)$.\\
Ta có $n=50,n_p=8,u_p=92,u_{p+1}=114$.\\
Tứ phân vị thứ ba của mẫu số liệu trên là $$Q_3=92+\dfrac{\dfrac{3\cdot 50}{4}-\left(6+14+17\right)}{8}.\left(114-92\right)\approx 93{,}4.$$
\end{itemchoice}
}
\end{ex}
\Closesolutionfile{ans}

\indapan{3}{ans-DS-KNTT-Lop11-OTGK1-De4}

\caukq

\Opensolutionfile{ans}[ans-KQ-KNTT-Lop11-OTGK1-De4]

\begin{ex}%[KNTT- Lớp 11 - Ôn tập giữa học kì 1 - Đề 4]%[Lê Hồng Phi]%[1D1H1-4]
Trong một buổi biểu diễn ở rạp xiếc, người nghệ sĩ có một tiết mục giữ thăng bằng và đạp xe $1$ bánh trên $1$ sợi dây dài $30$ m. Hỏi khi người nghệ sĩ đi hết đoạn dây thì bán kính xe đạp quét một góc lượng giác có số đo là bao nhiêu? (Tính theo đơn vị radian) Biết bánh xe đạp có bán kính bằng $0{,}4$ m.
\shortans{$75$}
\loigiai{
Gọi góc lượng giác mà bán kính xe đạp quét được khi đi hết sợi dây là $\alpha$  (rad).\\
Ta có $\ell=\alpha R\Leftrightarrow \alpha=\dfrac{\ell}{R} =\dfrac{30}{0{,}4}=75$ (rad).
}
\end{ex}

\begin{ex}%[KNTT- Lớp 11 - Ôn tập giữa học kì 1 - Đề 4]%[Lê Hồng Phi]%[1D1V5-6]
Số giờ có ánh sáng mặt trời của một thành phố A trong ngày thứ $t$ (ở đây $t$ là số ngày tính từ ngày $1$ tháng giêng) của một năm không nhuận được mô hình hoá bởi hàm số $L(t)=12+2{,}83\sin \left(\dfrac{2\pi}{365}(t-80)\right)$ với $t\in \mathbb{R}$ và $0<t\le 365$. Vào ngày $a$ tháng $b$ trong năm thì thành phố A có ít giờ ánh sáng mặt trời nhất. Tính $a-b$.
\shortans{$8$}
\loigiai{
Ta có \begin{align*}
-1\le \sin \left(\dfrac{2\pi}{365}(t-80)\right)\le 1&\Rightarrow -2{,}83\le 2{,}83\sin \left(\dfrac{2\pi}{365}(t-80)\right)\le 2{,}83\\
& \Rightarrow 9{,}17\le 12+2{,}83\sin \left(\dfrac{2\pi}{365}(t-80)\right)\le 14{,}83{,}\ \forall t\in \mathbb{R}.
\end{align*} 
Ngày thành phố A có ít giờ ánh sáng mặt trời nhất ứng với $\min L(t)=9{,}17$ (h) khi 
\begin{align*}
\sin \left(\dfrac{2\pi}{365}(t-80)\right)=-1&\Leftrightarrow \dfrac{2\pi}{365}(t-80)=-\dfrac{\pi}{2}+k2\pi,\  \left(k\in \mathbb{Z}\right)\\
&\Leftrightarrow t=-\dfrac{45}{4}+365k,\ \left(k\in \mathbb{Z}\right).
\end{align*} 
Vì $0<t\le 365$ nên $k=1$. Suy ra $t=-\dfrac{45}{4}+365=353{,}75$.\\
Như thế, vào ngày thứ $354$ của năm, tức là khoảng ngày $20$ tháng $12$ thì thành phố A sẽ có ít giờ ánh sáng mặt trời nhất.\\
Vậy $a=20$, $b=12$ và $a-b=8$.
}
\end{ex}

\begin{ex}%[KNTT- Lớp 11 - Ôn tập giữa học kì 1 - Đề 4]%[Lê Hồng Phi]%[1D1V5-5]
Có bao nhiêu giá trị nguyên của tham số $m$ để phương trình $\cos^22x=m+1$ có nghiệm?
\shortans{$2$}
\loigiai{
Ta có $\cos^2 2x=m+1\Leftrightarrow \dfrac{1+\cos 4x}{2}=m+1\Leftrightarrow \cos 4x=2m+1$.\\
Phương trình đã cho có nghiệm khi $-1\le 2m+1\le 1\Leftrightarrow -1\le m\le 0$.\\
Do $m\in \mathbb{Z}$ nên $m\in \{-1;0\}$.\\
Vậy có $2$ giá trị nguyên của $m$ thỏa mãn bài toán.
}
\end{ex}

\begin{ex}%[KNTT- Lớp 11 - Ôn tập giữa học kì 1 - Đề 4]%[Lê Hồng Phi]%[1D2H2-7]
Trong sân vận động có tất cả $30$ dãy ghế, dãy đầu tiên có $15$ ghế, mỗi dãy liền sau nhiều hơn dãy trước $4$ ghế. Hỏi sân vận động đó có tất cả bao nhiêu ghế?
\shortans{$2190$}
\loigiai{
Số ghế các dãy trong sân vận động là các số hạng của cấp số cộng có $u_1=15$ và $d=4$.\\
Suy ra $u_{30}=u_1+29d=15+29\cdot 4=131$.\\
Sân vận động đó có số ghế là $S_{30}=\dfrac{30\left(u_1+u_{30}\right)}{2}=\dfrac{30(15+131)}{2}=2\,190$ (ghế).
}
\end{ex}

\begin{ex}%[KNTT- Lớp 11 - Ôn tập giữa học kì 1 - Đề 4]%[Lê Hồng Phi]%[1H4V1-4]
Cho hình chóp $S.ABCD$ có đáy là hình bình hành $ABCD$. Gọi $E$, $F$, $G$ lần lượt là trung điểm của cạnh $SA$, $AB$, $CD$. Gọi $P$ là giao điểm của đường thẳng $EG$ và mặt phẳng $(SDF)$. Tính tỷ số $\dfrac{GP}{PE}$.
\shortans{$2$}
\loigiai{
\immini{Trong mặt phẳng $(ABCD)$, gọi $AG\cap DF=L$.\\
Dễ thấy $ADGF$ là hình bình hành nên $L$ là trung điểm của $AG$.\\
Trong mặt phẳng $(SAG)$, gọi $SL\cap GE=P$. Suy ra
$\heva{&P\in EG \\&P\in SL,SL\subset (SDF).}$\\	
Khi đó $P$ là giao điểm của đường thẳng $EG$ và mặt phẳng $(SDF)$.\\
Mặt khác, $P$ là trọng tâm của tam giác $SAG$ nên $$\dfrac{GP}{PE}=2.$$}{\begin{tikzpicture}[scale=1, font=\footnotesize, line join=round, line cap=round,>=stealth]
\path (0,0)coordinate (B)++(4,0) coordinate (C)++(1.5,1.2) coordinate (D) ($(B)+(D)-(C)$) coordinate (A) (0.7,3) coordinate (h) ($(A)+(h)$) coordinate (S) (barycentric cs:S=1,A=1) coordinate (E) (barycentric cs:B=1,A=1) coordinate (F) (barycentric cs:C=1,D=1) coordinate (G) (intersection of A--G and D--F) coordinate (L) (intersection of E--G and S--L) coordinate (P);
\draw (S)--(B)--(C)--(D)--cycle (G)--(S)--(C);
\draw[dashed] (E)--(G)--(A)--(D)--(F)--(S)--(A)--(B) (S)--(L);
\foreach \p/\g in {A/45,B/240,C/-60, D/0, S/90,E/30,F/-60,G/-30,L/-90,P/210} \fill[black] (\p) circle(1pt)+(\g:0.3) node{$\p$};
\end{tikzpicture}}
}
\end{ex}

\begin{ex}%[KNTT- Lớp 11 - Ôn tập giữa học kì 1 - Đề 4]%[Lê Hồng Phi]%[1H4C3-5]
Cho tứ diện $ABCD$ có $AB=3$, $CD=2$. Gọi $(\alpha)$ là một mặt phẳng song song với $AB$ và $CD$. Biết $(\alpha)$ cắt tứ diện $ABCD$ theo thiết diện là một hình thoi, chu vi của hình thoi đó bằng bao nhiêu?
\shortans{$4{,}8$}
\loigiai{
\immini{Giả sử mặt phẳng $(\alpha)$ cắt cạnh $BC$ tại điểm $M$ sao cho $CM=kCB$.\\
Gọi $N$, $P$, $Q$ lần lượt là giao điểm của $(\alpha)$ với các cạnh $AC$, $AD$, $BD$.\\
Khi đó $(\alpha)\cap (ABC)=MN$ và do $AB\parallel (\alpha)$ nên $MN\parallel AB$.\\
Tương tự $(\alpha)$ lần lượt cắt các mặt phẳng $(BCD)$, $(ACD)$, $(ABD)$ theo các giao tuyến $MQ$, $NP$, $PQ$ và $MQ\parallel CD$, $NP\parallel CD$, $PQ\parallel AB$.}{\begin{tikzpicture}[scale=1, font=\footnotesize, line join=round, line cap=round,>=stealth]
\def\k{0.4}
\path (0,0)coordinate (B) (5,0) coordinate (D) (3,-1.3) coordinate (C) ($(B)+(2,3)$) coordinate (A) ($(C)!{\k}!(B)$) coordinate (M) ($(C)!{\k}!(A)$) coordinate (N) ($(D)!{\k}!(B)$) coordinate (Q) ($(D)!{\k}!(A)$) coordinate (P);
\draw (A)--(B)--(C)--(D)--cycle (A)--(C) (M)--(N)--(P);
\draw[dashed] (B)--(D) (M)--(Q)--(P);
\foreach \p/\g in {B/180,C/-90,D/0,A/90,M/-120,N/180,P/30,Q/-60} \fill[black] (\p) circle(1pt)+(\g:0.3) node{$\p$};
\end{tikzpicture}}\noindent\\
Như thế, thiết diện giữa mặt phẳng $(\alpha)$ và tứ diện $ABCD$ là hình bình hành $MNPQ$.\\
Trong tam giác $ABC$ ta có $MN\parallel AB\Rightarrow \dfrac{MN}{AB}=\dfrac{CM}{CB}\Leftrightarrow \dfrac{MN}{3}=k\Rightarrow MN=3k$.\\
Trong tam giác $BCD$ ta có
$$MQ\parallel CD\Rightarrow \dfrac{MQ}{CD}=\dfrac{BM}{CB}=\dfrac{BC-CM}{BC}\Leftrightarrow \dfrac{MQ}{2}=1-k\Rightarrow MQ=2\left(1-k\right).$$
Hình bình hành $MNPQ$ là hình thoi khi $$MN=MQ\Leftrightarrow 3k=2\left(1-k\right)\Leftrightarrow k=\dfrac{2}{5}\Rightarrow MN=\dfrac{6}{5}.$$
Vậy chu vi hình thoi $MNPQ$ là $4MN=\dfrac{24}{5}$.
}
\end{ex}

\Closesolutionfile{ans}

\indapan{6}{ans-KQ-KNTT-Lop11-OTGK1-De4}