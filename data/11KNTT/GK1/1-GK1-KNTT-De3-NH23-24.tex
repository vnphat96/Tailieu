%=====batdau=====
% Danh sách các câu lỗi ID6 (35 câu): TẤT CẢ chưa được gắn ID6
% Danh sách các câu tự luận hoặc thiếu lệnh \choice (0 câu): KHÔNG CÓ.
% Danh sách các câu thiếu/thừa lệnh \True: KHÔNG CÓ.
% Danh sách các câu có phương án trùng lặp: KHÔNG CÓ.
% Danh sách các câu chưa có lời giải: 13
%=====ketthuc=====
\begin{name}
	{Biên soạn đề: VU Ngoc Hao}
	{Lớp 11 - Đề 16} 
\end{name}
\noindent{\bf\fontfamily{qag}\selectfont\color{violet}A. PHẦN TRẮC NGHIỆM}
\Opensolutionfile{ans}[ans/ans-1-GK1-KNTT-De16-NH23-24]
%%==========Câu 1
\begin{ex}%[1D1Y2-2]
	Trong các giá trị sau, $\cos \alpha$ không thể nhận những giá trị nào?
	\choice
	{$\dfrac{1}{2}$}
	{$\dfrac{2023}{2024}$}
	{\True $\dfrac{\sqrt{2}+2}{2}$}
	{$-\dfrac{\sqrt{3}}{2}$}
	\loigiai{
		$\dfrac{\sqrt{2}+2}{2} > 1$. Suy ra $\cos \alpha$ không nhận giá trị $\dfrac{\sqrt{2}+2}{2}$.	
	}
\end{ex}
%%==========Câu 2
\begin{ex}%[1D1Y2-1]
	Cho $\dfrac{\pi}{2} < \alpha <\pi$. Kết quả nào sau đây đúng?
	\choice
	{\True $\sin \alpha >0$, $\cos \alpha <0$}
	{$\sin \alpha >0$, $\cos \alpha >0$}
	{$\sin \alpha <0$, $\cos \alpha <0$}
	{$\sin \alpha <0$, $\cos \alpha >0$}
	\loigiai{
		Vì $\dfrac{\pi}{2} < \alpha <\pi$, điểm biểu diễn cung $\alpha$ thuộc góc phần tư thứ $II$. Suy ra $\sin \alpha >0$, $\cos \alpha <0$.
	}
\end{ex}
%%==========Câu 3
\begin{ex}%[1D1Y2-1]
Trong các đẳng thức sau, đẳng thức nào đúng?
	\choice
	{\True $\cos \left( \pi - x \right) = -\cos x$}
	{$\cos \left( \pi - x \right) = \cos x$}
	{$\cos \left( \pi + x \right) = \cos (-x)$}
	{$\cos (-x) = -\cos x$}
	\loigiai{
		$\cos \left( \pi - x \right) = -\cos (-x) = -\cos x$.
	}
\end{ex}
%%==========Câu 4
\begin{ex}%[1D1Y2-2]
	Cho $\sin \alpha =\dfrac{3}{5}$ với $\dfrac{\pi}{2} < \alpha < \pi $. Khi đó $\tan \alpha$ nhận giá trị bằng
	\choice
	{$\dfrac{5}{3}$}
	{$- \dfrac{4}{5}$}
	{\True $- \dfrac{3}{4}$}
	{$- \dfrac{4}{3}$}
	\loigiai{
		Ta có $\heva{&\cos^{2} \alpha =\dfrac{16}{25} \\&\dfrac{\pi}{2} < \alpha < \pi} \Rightarrow \cos \alpha = -\dfrac{4}{5} \Rightarrow \tan \alpha = - \dfrac{3}{4}$.
	}
\end{ex}
%%==========Câu 5
\begin{ex}%[1D1Y2-2]
	Cho $\tan \alpha =2$, giá trị của biểu thức $P=\dfrac{4\sin \alpha +5\cos \alpha}{2\sin \alpha - 3\cos \alpha}$ là 
	\choice
	{$12$}
	{\True $13$}
	{$-9$}
	{$-12$}
	\loigiai{
		$P=\dfrac{4\sin \alpha +5\cos \alpha}{2\sin \alpha - 3\cos \alpha} = \dfrac{4\tan \alpha +5}{2\tan \alpha - 3} =\dfrac{4 \cdot 2 +5}{2 \cdot 2 - 3} = 13$.
	}
\end{ex}
%%==========Câu 6
\begin{ex}%[1D1Y3-1]
Với mọi góc lượng giác $a$, $b$. Trong các công thức sau, công thức nào đúng?
	\choice
	{$\cos \left( a+b \right) = \cos a \cdot \cos b + \sin a \cdot \sin b$}
	{$\sin \left( a-b \right) = \sin a \cdot \cos b + \cos a \cdot \sin b$}
	{\True $\cos \left( a-b \right) = \cos a \cdot \cos b + \sin a \cdot \sin b$}
	{$\sin \left( a+b \right) = \sin a \cdot \cos b - \cos a \cdot \sin b$}
	\loigiai{
		Theo công thức cộng $\cos \left( a-b \right) = \cos a \cdot \cos b + \sin a \cdot \sin b$.
	}
\end{ex}
%%==========Câu 7
\begin{ex}%[1D1Y3-2]
	Cho $\cos a =\dfrac{1}{3}$, khi đó giá trị của $\cos 2a$ bằng
	\choice
	{$\dfrac{8}{9}$}
	{$\dfrac{7}{9}$}
	{\True $-\dfrac{7}{9}$}
	{$-\dfrac{8}{9}$}
	\loigiai{
	Ta có $\cos 2a = 2\cos^{2} a -1 = \dfrac{2}{9} - 1=-\dfrac{7}{9}$.
	}
\end{ex}
%%==========Câu 8
\begin{ex}%[1D1B3-2]
	Giá trị của biểu thức $A=\sin \dfrac{\pi}{8} \sin \dfrac{3\pi}{8}$ là
	\choice
	{$A=\sqrt{2}$}
	{\True $A=\dfrac{\sqrt{2}}{4}$}
	{$A=\dfrac{\sqrt{2}}{2}$}
	{$A=-\dfrac{\sqrt{2}}{2}$}
	\loigiai{
		$A=\sin \dfrac{3\pi}{8} \sin \dfrac{\pi}{8} = \dfrac{1}{2} \left[ \cos \left( \dfrac{3\pi}{8} - \dfrac{\pi}{8} \right) - \cos \left( \dfrac{3\pi}{8} + \dfrac{\pi}{8} \right) \right] = \dfrac{1}{2} \left( \cos \dfrac{\pi}{4} - \cos \dfrac{\pi}{2} \right) = \dfrac{\sqrt{2}}{4}$.
	}
\end{ex}
%%==========Câu 9
\begin{ex}%[1D1B3-1]
Rút gọn biểu thức $\cos \left( x+\dfrac{\pi}{4} \right) - \cos \left( x-\dfrac{\pi}{4} \right)$ ta được
	\choice
	{$\sqrt{2}\sin x$}
	{\True $-\sqrt{2}\sin x$}
	{$\sqrt{2}\cos x$}
	{$-\sqrt{2}\cos x$}
	\loigiai{
	\begin{eqnarray*}
	\cos \left( x+\dfrac{\pi}{4} \right) - \cos \left( x-\dfrac{\pi}{4} \right) & = & -2 \sin \left( \dfrac{x+\dfrac{\pi}{4} + x-\dfrac{\pi}{4}}{2} \right) \cdot \sin \left( \dfrac{x+\dfrac{\pi}{4} - x+\dfrac{\pi}{4}}{2} \right)\\
	& = & -2 \sin x \cdot \sin \dfrac{\pi}{4} = -\sqrt{2}\sin x.
	\end{eqnarray*}
	}
\end{ex}
%%==========Câu 10
\begin{ex}%[1D1K3-5]
Nếu tam giác $ABC$ có ba góc $A$, $B$, $C$ thỏa mãn $\sin A=\cos B+\cos C$ thì tam giác $ABC$ là tam giác gì ?
	\choice
	{Tam giác $ABC$ đều}
	{Tam giác $ABC$ cân}
	{\True Tam giác $ABC$ vuông}
	{Tam giác $ABC$ vuông cân}
	\loigiai{
	Xét tam giác $ABC$, ta có
	$$
	A+B+C=\pi \Rightarrow \frac{A}{2}+\frac{B+C}{2}=\frac{\pi}{2} \Rightarrow \frac{B+C}{2}=\frac{\pi}{2}-\frac{A}{2} \Rightarrow \cos \frac{B+C}{2}=\cos \left(\frac{\pi}{2}-\frac{A}{2}\right)=\sin \frac{A}{2} \text {. }
	$$
	Theo đề bài, ta có
	$$
	\begin{aligned}
		& \sin A=\cos B+\cos C \Leftrightarrow 2 \sin \frac{A}{2} \cos \frac{A}{2}=2 \cos \frac{B+C}{2} \cos \frac{B-C}{2} \Leftrightarrow \cos \frac{A}{2}=\cos \frac{B-C}{2} \\
		& \Leftrightarrow
		\hoac{&A = B - C\\&A = C - B}
	    \Leftrightarrow 
	    \hoac{&A + C = B\\&A + B = C}
	    \Rightarrow 
	    \hoac{&B=\dfrac{\pi}{2}\\&C=\dfrac{\pi}{2}.}
		&
	\end{aligned}
	$$
	Vậy tam giác $ABC$ vuông tại $B$ hoặc tại $C$.	
	}
\end{ex}
%%==========Câu 11
\begin{ex}%[1D1Y4-3]
	Hàm số nào sau đây là hàm số chẵn?
	\choice
	{$y=\sin 2x$}
	{$y= x \cdot \cos x$}
	{$y=\cos x \cdot \cot x$}
	{\True $y=\dfrac{\tan x}{\sin x}$}
	\loigiai{
	Ta có $f(-x)=\dfrac{\tan (-x)}{\sin (-x)}=\dfrac{-\tan x}{-\sin x}=\dfrac{\tan x}{\sin x}=f (x)$. Suy ra hàm số $y=\dfrac{\tan x}{\sin x}$ là hàm chẵn.
	}
\end{ex}
%%==========Câu 12
\begin{ex}%[1D1Y4-1]
Tìm tập xác định $\mathscr{D}$ của hàm số $y=\dfrac{2023}{\sin x}$.
	\choice
	{$\mathscr{D} = \mathbb{R} \setminus \left\{ 0 \right\}$}
	{\True $\mathscr{D} = \mathbb{R} \setminus \left\{ k \pi, k\in \mathbb{Z} \right\}$}
	{$\mathscr{D} = \mathbb{R} \setminus \left\{ \dfrac{\pi}{2} + k \pi, k\in \mathbb{Z} \right\}$}
	{$\mathscr{D} = \mathbb{R}$}
	\loigiai{
	Hàm số $y=\dfrac{2023}{\sin x}$ xác định khi: $\sin x \ne 0 \Leftrightarrow x \ne k\pi, k \in \mathbb{Z}$.\\
	Tập xác định $\mathscr{D} = \mathbb{R} \setminus \left\{ k \pi, k\in \mathbb{Z} \right\}$.
	}
\end{ex}
%%==========Câu 13
\begin{ex}%[1D1Y4-2]
Xét sự biến thiên của hàm số $y=\sin x-\cos x$. Trong các kết luận sau, kết luận nào đúng?
	\choice
	{\True Hàm số đã cho đồng biến trên khoảng $\left(-\dfrac{\pi}{4} ; \dfrac{3 \pi}{4}\right)$}
	{Hàm số đã cho đồng biến trên khoảng $\left(\dfrac{3 \pi}{4} ; \dfrac{7 \pi}{4}\right)$}
	{Hàm số đã cho nghịch biến trên khoảng $\left(\pi ; \dfrac{4 \pi}{3}\right)$}
	{Hàm số đã cho nghịch biến trên khoảng $\left(-\dfrac{\pi}{4} ; \dfrac{7 \pi}{4}\right)$}
	\loigiai{
		
	}
\end{ex}
%%==========Câu 14
\begin{ex}%[1D1Y4-4]
Hàm số $y=\sin 2x$ tuần hoàn với chu kì
	\choice
	{$\dfrac{\pi}{3}$}
	{$\dfrac{\pi}{2}$}
	{\True $\pi$}
	{$2 \pi$}
	\loigiai{
	Hàm số $y=\sin 2x$ có chu kỳ: $T=\dfrac{2 \pi}{2}=\pi$.
	}
\end{ex}
%%==========Câu 15
\begin{ex}%[1D1Y4-5]
Tập giá trị của hàm số $y=\sin \left(x+\dfrac{2 \pi}{3}\right)+\sin x$ là
	\choice
	{\True $\left[ -1; 1 \right]$}
	{$\left[ -2; 2 \right]$}
	{$\left[-\dfrac{\pi}{3} ; \dfrac{\pi}{3}\right]$}
	{$\left[-\dfrac{2 \pi}{3} ; \dfrac{2 \pi}{3}\right]$}
	\loigiai{
	Ta có $y=2 \sin \left(x+\dfrac{\pi}{3}\right) \cos \dfrac{\pi}{3}=2 \sin \left(x+\dfrac{\pi}{3}\right) \cdot \dfrac{1}{2}=\sin \left(x+\dfrac{\pi}{3}\right)$.\\
	Suy ra tập giá trị của hàm số là $\left[ -1; 1 \right]$.
	}
\end{ex}
%%==========Câu 16
\begin{ex}%[1D1Y5-3]
Phương trình $2 \sin x-1=0$ có tập nghiệm là
	\choice
	{\True $S=\left\{\dfrac{\pi}{6}+k 2 \pi ; \dfrac{5 \pi}{6}+k 2 \pi, k \in \mathbb{Z}\right\}$}
	{$S=\left\{\dfrac{\pi}{3}+k 2 \pi ;-\dfrac{2 \pi}{3}+k 2 \pi, k \in \mathbb{Z}\right\}$}
	{$S=\left\{\dfrac{\pi}{6}+k 2 \pi ;-\dfrac{\pi}{6}+k 2 \pi, k \in \mathbb{Z}\right\}$}
	{$S=\left\{\dfrac{1}{2}+k 2 \pi, k \in \mathbb{Z}\right\}$}
	\loigiai{
		$$
		2 \sin x-1=0 \Leftrightarrow \sin x=\dfrac{1}{2} \Leftrightarrow
		\hoac{&x=\dfrac{\pi}{6}+k 2 \pi \\&x=\dfrac{5 \pi}{6}+k 2 \pi}, k \in \mathbb{Z}.
		$$
	}
\end{ex}
%%==========Câu 17
\begin{ex}%[1D1Y5-3]
Tìm tất cả các giá trị của tham số $m$ để phương trình $\cos 2x=m$ có nghiệm.
	\choice
	{$m \in \left( -1; 1 \right)$}
	{\True $m \in \left[ -1; 1 \right]$}
	{$m \in \left( -2; 2 \right)$}
	{$m \in \left[ -2; 2 \right]$}
	\loigiai{
		$-1 \leq \cos 2x \leq 1 \Rightarrow$ PT: $\cos 2x = m$ có nghiệm khi và chi khi $m \in \left[ -1; 1 \right]$.
	}
\end{ex}
%%==========Câu 18
\begin{ex}%[1D1Y5-3]
Tìm tất cả họ nghiệm của phương trình $\sin \left(\dfrac{2 x}{3}-\dfrac{\pi}{3}\right)=0$.
	\choice
	{$x=k \pi, k \in \mathbb{Z}$}
	{$x=\dfrac{2 \pi}{3}+\dfrac{k 3 \pi}{2}, k \in \mathbb{Z}$}
	{$x=\dfrac{\pi}{3}+k \pi, k \in \mathbb{Z}$}
	{\True $x=\dfrac{\pi}{2}+\dfrac{k 3 \pi}{2}, k \in \mathbb{Z}$}
	\loigiai{
	$$
	\sin \left(\dfrac{2 x}{3}-\dfrac{\pi}{3}\right)=0 \Leftrightarrow \dfrac{2 x}{3}-\dfrac{\pi}{3}=k \pi \Leftrightarrow x=\dfrac{\pi}{2}+\dfrac{3}{2} k \pi, k \in \mathbb{Z} \text {.}
	$$
	}
\end{ex}
%%==========Câu 19
\begin{ex}%[1D1B5-3]
Giải phương trình $\cos \left( 3x+15^{\circ} \right)=\dfrac{\sqrt{3}}{2}$ ta được
	\choice
	{$\left[\begin{aligned}&x=25^{\circ}+k \cdot 120^{\circ} \\& x=-15^{\circ}+k \cdot 120^{\circ}\end{aligned}, k \in \mathbb{Z}\right.$}
	{$\left[\begin{aligned}&x=5^{\circ}+k \cdot 120^{\circ} \\ &x=15^{\circ}+k \cdot 120^{\circ}\end{aligned}, k \in \mathbb{Z} \right.$}
	{$\left[\begin{aligned}&x=25^{\circ}+k \cdot 120^{\circ} \\& x=15^{\circ}+k \cdot 120^{\circ}\end{aligned}, k \in \mathbb{Z}\right.$}
	{\True $\left[\begin{aligned}&x=5^{\circ}+k \cdot 120^{\circ} \\& x=-15^{\circ}+k \cdot 120^{\circ}\end{aligned}, k \in \mathbb{Z} \right.$}
	\loigiai{
		\begin{eqnarray*}
	& & \cos \left( 3x+15^{\circ} \right)=\dfrac{\sqrt{3}}{2}\\ 
	& \Leftrightarrow & \cos \left( 3x+15^{\circ} \right)=\cos 30^{\circ}\\
	& \Leftrightarrow &  \hoac{&	3 x+15^{\circ}=30^{\circ}+k 360^{\circ}\\&	3 x+15^{\circ}=-30^{\circ}+k 360^{\circ}}\\
	& \Leftrightarrow & \hoac{&3 x=15^{\circ}+k 360^{\circ}\\&3 x=-45^{\circ}+k 360^{\circ}}\\ 
	& \Leftrightarrow & \hoac{&x=5^{\circ}+k 120^{\circ}\\&x=-15^{\circ}+k 120^{\circ}}, k \in \mathbb{Z}.
\end{eqnarray*}	
	}
\end{ex}
%%==========Câu 20
\begin{ex}%[1D1B5-3]
	Các nghiệm của phương trình $\tan \left(x+\dfrac{\pi}{4}\right)-\sqrt{3}=0$ là
	\choice
	{$x= \pm \dfrac{\pi}{3}+k \pi, k \in \mathbb{Z}$}
	{\True $x=\dfrac{\pi}{12}+k \pi, x=-\dfrac{7 \pi}{12}+k \pi k \in \mathbb{Z}$}
	{$x= \pm \dfrac{\pi}{12}+k \pi \quad k \in \mathbb{Z}$}
	{$x=\dfrac{\pi}{12}+k \pi, k \in \mathbb{Z}$}
	\loigiai{
	$$
	\tan ^2\left(x+\dfrac{\pi}{4}\right)-3=0 
	\Leftrightarrow
	\hoac{&\tan \left( x + \dfrac { \pi } { 4 }\right) = \sqrt { 3 }\\&\tan \left( x + \dfrac { \pi } { 4 }\right) = - \sqrt { 3 }}
\Leftrightarrow 
\hoac{&x + \dfrac { \pi } { 4 } = \dfrac { \pi } { 3 } + k \pi  \\& x + \dfrac { \pi } { 4 } = - \dfrac { \pi } { 3 } + k \pi}
\Leftrightarrow 
\hoac{&x=\dfrac{\pi}{12}+k \pi\\&x=-\dfrac{7 \pi}{12}+k \pi}
, k \in \mathbb{Z}.
	$$
	}
\end{ex}
%%==========Câu 21
\begin{ex}%[1D1B5-5]
Tổng các nghiệm thuộc khoảng $\left(-\dfrac{\pi}{2} ; \dfrac{\pi}{2}\right)$ của phương trình $4 \sin^{2} 2 x-1=0$ bằng:
	\choice
	{$\pi$}
	{$\dfrac{\pi}{3}$}
	{\True $0$}
	{$\dfrac{\pi}{6}$}
	\loigiai{
	Ta có $4 \sin^{2} 2x-1=0 \Leftrightarrow 2 \left(1-\cos 4x \right)-1=0 \Leftrightarrow \cos 4 x=\dfrac{1}{2} \Leftrightarrow x= \pm \dfrac{\pi}{12}+k \dfrac{\pi}{2}, k \in \mathbb{Z}$.\\
	Do $x= \pm \dfrac{\pi}{12}+k \dfrac{\pi}{2} \in\left(-\dfrac{\pi}{2} ; \dfrac{\pi}{2}\right) \Rightarrow
	\hoac{&x_1=\dfrac{\pi}{12}\\&x_2=-\dfrac{\pi}{12}\\&x_3=-\dfrac{5 \pi}{12}\\&x_4=\dfrac{5 \pi}{12}.}
$
	}
\end{ex}
%%==========Câu 22
\begin{ex}%[1D2Y1-2]
	Cho dãy số $u_{n}$ với $u_{n}=\dfrac{n+1}{n}$. Tính $u_{5}$.
	\choice
	{$5$}
	{\True $\dfrac{6}{5}$}
	{$\dfrac{5}{6}$}
	{$1$}
	\loigiai{
	Ta có $u_5=\dfrac{5+1}{5}=\dfrac{6}{5}$.
	}
\end{ex}
%%==========Câu 23
\begin{ex}%[1D2Y1-2]
	Cho dãy số $u_n$, biết $u_n=\dfrac{1}{n+1}$. Ba số hạng đầu tiên của dãy số đó lần lượt là những số nào dưới đây?
	\choice
	{\True $\dfrac{1}{2} ; \dfrac{1}{3} ; \dfrac{1}{4}$}
	{$1 ; \dfrac{1}{2} ; \dfrac{1}{3}$}
	{$\dfrac{1}{2} ; \dfrac{1}{4} ; \dfrac{1}{6}$}
	{$1 ; \dfrac{1}{3} ; \dfrac{1}{5}$}
	\loigiai{
	Ta có $u_1=\dfrac{1}{1+1}=\dfrac{1}{2}$ ; $u_2=\dfrac{1}{1+2}=\dfrac{1}{3}$ ; $u_3=\dfrac{1}{1+3}=\dfrac{1}{4}$.
	}
\end{ex}
%%==========Câu 24
\begin{ex}%[1D2Y1-2]
	Cho dãy số $u_n$, biết $u_n=\dfrac{n+1}{2n+1}$. Số $\dfrac{8}{15}$ là số hạng thứ mấy của dãy số?
	\choice
	{$8$}
	{$6$}
	{$5$}
	{\True $7$}
	\loigiai{
	Ta có $\dfrac{n+1}{2n+1}=\dfrac{8}{15} \Leftrightarrow 15 \left( n+1 \right)=8 \left( 2n+1 \right) \Leftrightarrow n=7$.
	Vậy $\dfrac{8}{15}$ là số hạng thứ 7 của dãy.
	}
\end{ex}
%%==========Câu 25
\begin{ex}%[1D2Y1-1]
	Cho dãy số viết dưới dạng khai triển là $1$, $4$, $9$, $16$, $25$. Trong các công thức sau, công thức nào là công thức tổng quát của dãy số trên.
	\choice
	{$u_n=3n-2$}
	{$u_n=n+3$}
	{$u_n=n^2$}
	{\True $u_n=2n^2-1$}
	\loigiai{
	Ta có $1=1^2$; $4=2^2$; $9=3^2$; $16=4^2$; $25=5^2$.
	Vậy dãy $1$, $4$, $9$, $16$, $25$ có công thức tổng quát của dãy $u_n=n^2$.
	}
\end{ex}
%%==========Câu 26
\begin{ex}%[1D2B2-3]
	Trong các dãy số sau, dãy số nào là một cấp số cộng?
	\choice
	{$1$; $-2$; $-4$; $-6$; $-8$; $\ldots$}
	{$1$; $-3$; $-6$; $-9$; $-12$; $\ldots$}
	{$1 $; $-3 $; $-7 $; $-11 $; $-15$; $\ldots$}
	{\True $-1 $; $-3 $; $-5 $; $-7 $; $-9$; $\ldots$}
	\loigiai{
	$$
	\begin{aligned}
		& \text { Ta có} -3=-1+-2 \Leftrightarrow u_2=u_1+-2 \\
		& -5=-3+-2 \Leftrightarrow u_3=u_2+-2 \\
		& -7=-5+-2 \Leftrightarrow u_4=u_1+-2 \\
		& -9=-7+-2 \Leftrightarrow u_5=u_4+-2
	\end{aligned}
	$$
	Vậy dãy $-1 $; $-3 $; $-5 $; $-7 $; $-9$, $\ldots$ là cấp số cộng với $u_1=-1$ và công sai $d=-2$.
	}
\end{ex}
%%==========Câu 27
\begin{ex}%[1D2Y2-2]
Trong các dãy số sau đây, dãy số nào là cấp số cộng?
	\choice
	{$u_n=3n^2+2017$}
	{\True $u_n=3n+2018$}
	{$u_n=3^n$}
	{$u_n=-3^{n+1}$}
	\loigiai{
	Ta có $u_{n+1}-u_n=3 \left( n+1 \right)+2018-\left( 3n +2018 \right)=3 \Leftrightarrow u_{n+1}=u_n+3, \forall n \in \mathbb{N^*}$.\\
	Suy ra: $u_n=3n+2018$ là một cấp số cộng.
	}
\end{ex}
%%==========Câu 28
\begin{ex}%[1D2B2-3]
Cho cấp số cộng $u_n$ biết $u_5=18$ và $4 S_n=S_{2n}$. Giá trị $u_1$ và $d$ là:
	\choice
	{$u_1=2$, $d=3$}
	{\True $u_1=3$, $d=2$}
	{$u_1=2$, $d=2$}
	{$u_1=2$, $d=4$}
	\loigiai{
	Ta có $u_5=18 \Leftrightarrow u_1+4d=18$.\\
	Lại có
	$$
	4S_5=S_{10} \Leftrightarrow 4\left(5 u_1+\frac{5.4}{2} d\right)=10 u_1+\frac{10\cdot 9}{2} d \Leftrightarrow 2 u_1-d=0 .
	$$
	Khi đó ta có hệ phương trình: 
	$\heva{&u_1+4 d=18\\&2 u_1-d=0}
	 \Leftrightarrow
	 \heva{&u_1=2\\&d=4.}$
	}
\end{ex}
%%==========Câu 29
\begin{ex}%[1D2B2-3]
	Cho cấp số cộng $u_n$ biết $u_1+u_3=8$; $u_4=10$. Hỏi công sai của cấp số cộng đã cho bằng?
	\choice
	{\True $d=3$}
	{$d=6$}
	{$d=2$}
	{$d=4$}
	\loigiai{
	Gọi $d$ là công sai của cấp số cộng đã cho. Theo đề bài, ta có
	$$\heva{& u_{1} + u_{3} = 8\\&u_{4} = 10}
	\Leftrightarrow 
	\heva{&2 u _ { 1 } + 2 d = 8\\& u _ { 1 } + 3 d = 1 0 }
    \Leftrightarrow 
    \heva{& u_1=1\\& d=3.}$$
	}
\end{ex}
%%==========Câu 30
\begin{ex}%[1D2B2-6]
Một gia đình cần khoan một cái giếng để lấy nước. Họ thuê một đội khoan giếng nước biết giá của mét khoan đầu tiên là $80.000$ đồng, kể từ mét khoan thứ hai giá của mỗi mét tăng thêm $5.000$ đồng so với giá của mét khoan trước đó. Biết cần phải khoan sâu xuống $50$ m mới có nước. Hỏi phải trả bao nhiêu tiền để khoan giếng đó?
	\choice
	{$10120000$}
	{\True $10125000$}
	{$1012500$}
	{$10000000$}
	\loigiai{
	Áp dụng công thức tính tổng của $n$ số hạng đầu của một cấp số cộng có số hạng đầu $u_1=80.000$ công sai $d=5.000$ ta được số tiền phải trả khi khoan đến mét thứ $n$ là
	$$
	S_n=\frac{n u_1+u_n}{2}=\frac{n\left[2 u_1+ \left( n-1 \right) d\right]}{2}
	$$
	Khi khoan sâu xuống $50$ m, số tiền phài trả là
	$$
	S_n=\frac{50[2\cdot 80000+ \left( 50-1 \right)\cdot 5000]}{2}=10125000 .
	$$
	}
\end{ex}
%%==========Câu 31
\begin{ex}%[1D2Y3-1]
Trong các dãy số sau, dãy số nào là một cấp số nhân?
	\choice
	{\True $128$; $-64$; $32$; $-16$; $8$; $\ldots$}
	{$\sqrt{2}$; $2$; $4$; $4 \sqrt{2}$; $\ldots$}
	{$5$; $6$; $7$; $8$; $\ldots$}
	{$15$; $5$; $1$; $\dfrac{1}{5}$; $\ldots$}
	\loigiai{}
\end{ex}
%%==========Câu 32
\begin{ex}%[1D2Y3-1]
Cho dãy số $u_n$ với $u_n=\dfrac{3}{2} \cdot 5^n$. Khẳng định nào sau đây đúng?
	\choice
	{$u_{n}$ không phải là cấp số nhân}
	{$u_{n}$ là cấp số nhân có công bội $q=5$ và số hạng đầu $u_{1}=\dfrac{3}{2}$}
	{\True $u_n$ là cấp số nhân có công bội $q=5$ và số hạng đầu $u_{1}=\dfrac{15}{2}$}
	{$u_n$ là cấp số nhân có công bội $q=\dfrac{5}{2}$ và số hạng đầu $u_{1}=3$}
	\loigiai{}
\end{ex}
%%==========Câu 33
\begin{ex}%[1D2Y3-1]
Cho cấp số nhân $u_n$ rới công bội $q <0$ và $u_2=4$, $u_4=9$. Tìm $u_{1}$ ?
	\choice
	{$u_1=-\dfrac{8}{3}$}
	{$u_1=\dfrac{8}{3}$}
	{\True $u_1=-6$}
	{$u_1=6$}
	\loigiai{
Vì $q<0$, $u_2>0$ nên $u_3<0$. Do đó $u_3=-\sqrt{u_2 \cdot u_4}=-\sqrt{4\cdot9}=-6$;
$$
u_2^2=u_1 u_3 \Rightarrow u_1=\dfrac{u_2^2}{u_3}=\dfrac{4^2}{-6}=-\dfrac{8}{3}.
$$	
}
\end{ex}
%%==========Câu 34
\begin{ex}%[1D2B3-3]
	Cho cấp số nhân $u_n$ biết $u_1+u_5=51$; $u_2+u_{6}=102$. Hỏi số $12288$ là số hạng thứ mấy của cấp số nhân $u_n$ ?
	\choice
	{Số hạng thứ $10$}
	{Số hạng thứ $11$}
	{Số hạng thứ $12$}
	{\True Số hạg thứ $13$}
	\loigiai{
	Gọi $q$ là công bội của cấp số nhân đã cho. Theo đề bài, tạ có
	$$
	\heva{&u _ { 1 } + u _ { 5 } = 5 1 \\&u _ { 2 } + u _ { 6 } = 1 0 2}
	\Leftrightarrow 
	\heva{&	u_1 \left( 1+q^4 \right)=51\\&	u_1 q \left( 1+q^4 \right)=102}
	\Rightarrow q=2 \Rightarrow u_1=3 \Rightarrow u_n=3 \cdot 2^{n-1} .
	$$
	Mặt khác $u_n=12288 \Leftrightarrow 3\cdot 2^{n-1}=12288 \Leftrightarrow 2^{n-1}=2^{12} \Leftrightarrow n=13$.
	}
\end{ex}
%%==========Câu 35
\begin{ex}%[1D2B3-7]
	Chu kì bán rã của nguyên tố phóng xạ poloni $210$ là $138$ ngày. Tính khối lượng còn lại của $20$ gam poloni $210$ sau $7314$ ngày .
	\choice
	{\True $2{,}22 \cdot 10^{-15}$}
	{$2{,}52 \cdot 10^{-15}$}
	{$3{,}22 \cdot 10^{-15}$}
	{$3{,}52 \cdot 10^{-15}$}
	\loigiai{
	Kí hiệu $u_n$ là khối lượng còn lại của $20$ gam poloni $210$ sau $n$ chu kì bán rã.\\
	Ta có $7314$ ngày gồm $53$ chu kì bán rã. Theo đề bài ra, ta cần tính $u_{53}$.\\
	Từ giả thiết suy ra dãy $\left(u_n\right)$ là một cấp số nhân với số hạng đầu là $u_1=\dfrac{20}{2}=10$ và công bội $q=0{,}5$.\\
	Do đó $u_{53}=10 \cdot\left(\dfrac{1}{2}\right)^{52} \approx 2{,}22 \cdot 10^{-15}$.
	}
\end{ex}

\Closesolutionfile{ans}
\inputans{10}{ans/ans-1-GK1-KNTT-De16-NH23-24}
\noindent{\bf\fontfamily{qag}\selectfont\color{violet}B. PHẦN TỰ LUẬN}
\setcounter{bt}{0}
%%==========Bài 36
\begin{bt}%[1D1B2-2]
	Cho $\sin \alpha=\dfrac{3}{5}$ với $0<\alpha<\dfrac{\pi}{2}$. Tính $\cos \alpha, \tan \alpha$.
	\loigiai{
	Ta có \begin{eqnarray*}
			& & \sin ^2 \alpha+\cos ^2 \alpha=1\\ 
			& \Leftrightarrow & \cos ^2 \alpha=1-\sin ^2 \alpha\\
			& \Leftrightarrow &  \cos ^2 \alpha=1-\left(\dfrac{3}{5}\right)^2\\
			& \Leftrightarrow & \cos ^2 \alpha=\dfrac{16}{25}\\ 
			& \Leftrightarrow & \cos \alpha= \pm \dfrac{4}{5}.
		\end{eqnarray*}	
	Vì $0<\alpha<\dfrac{\pi}{2} \Rightarrow \cos \alpha>0 \Rightarrow \cos \alpha=\dfrac{4}{5}$. Ta có $\tan \alpha=\dfrac{\sin \alpha}{\cos \alpha}=\dfrac{\dfrac{3}{5}}{\dfrac{4}{5}}=\dfrac{3}{4}$.
	}
\end{bt}
%%==========Bài 37
\begin{bt}%[1D2B2-6]
	Người ta trồng $465$ cây trong một khu vườn hình tam giác như sau: Hàng thứ nhất có $1$ cây, hàng thứ hai có $2$ cây, hàng thứ ba có $3$ cây.... Số hàng cây trong khu vườn là bao nhiêu ?
	\loigiai{
	Cách trồng $465$ cây trong một khu vườn hình tam giác như trên lập thành một cấp số cộng $u_n$ với số $u_n$ là số cây ở hàng thứ $n$ và $u_1=1$ và công sai $d=1$.\\
	Tổng số cây trồng được là: $S_n=465 \Leftrightarrow \dfrac{n \left( n+1 \right)}{2}=465 \Leftrightarrow n^2+n-930=0 \Leftrightarrow
	\hoac{&n=30\\&n=-31 \left( \text {Loại } \right)}$. \\
	Như vậy số hàng cây trong khu vườn là 30 .
	}
\end{bt}
%%==========Bài 38
\begin{bt}%[1D1G5-5]
Giải phương trình: $\cot x+\sin x\left(1+\tan x \tan \dfrac{x}{2}\right)=4$.
	\loigiai{
	Điều kiện: $
	\heva{&\sin x \neq 0\\&\cos x \neq 0\\&\cos \dfrac{x}{2} \neq 0}
	\Leftrightarrow
	\heva{&\sin x \neq 0\\&\cos x \neq 0}
    \Leftrightarrow \sin 2 x \neq 0$.
    	\begin{eqnarray*}
    	& & \cot x+\sin x\left(1+\tan x \tan \dfrac{x}{2}\right)=4\\ 
    	& \Leftrightarrow & \dfrac{\cos x}{\sin x}+\sin x\left(1+\dfrac{\sin x}{\cos x} \cdot \dfrac{\sin \dfrac{x}{2}}{\cos \dfrac{x}{2}}\right)=4\\
    	& \Leftrightarrow &  \dfrac{\cos x}{\sin x}+\sin x\left(\dfrac{\cos x \cos \dfrac{x}{2}+\sin x \cdot \sin \dfrac{x}{2}}{\cos x \cdot \cos \dfrac{x}{2}}\right)=4\\
    	& \Leftrightarrow & \dfrac{\cos x}{\sin x}+\sin x \cdot \dfrac{\cos \dfrac{x}{2}}{\cos x \cdot \cos \dfrac{x}{2}}=4\\ 
    	& \Leftrightarrow & \dfrac{\cos x}{\sin x}+\dfrac{\sin x}{\cos x}=4\\
    	& \Leftrightarrow & \dfrac{\cos ^2 x+\sin ^2 x}{\sin x \cdot \cos x}=4\\
        & \Leftrightarrow & 4 \sin x \cdot \cos x=1\\
        & \Leftrightarrow & 2 \sin 2 x=1 \Leftrightarrow \sin 2 x=\dfrac{1}{2}\\
        & \Leftrightarrow & \hoac{&x=\dfrac{\pi}{12}+k \pi\\&x=\dfrac{5 \pi}{12}+k \pi}, k \in \mathbb{Z}.
    \end{eqnarray*}	
So với điều kiện, ta nhận 
$\hoac{&x=\dfrac{\pi}{12}+k \pi \\&x=\dfrac{5 \pi}{12}+k \pi}, k \in \mathbb{Z}$.
	}
\end{bt}


%%==========Bài 39
\begin{bt}%[1D2B3-3]
Người ta thiết kế một cái tháp gồm $11$ tầng. Diện tích bề mặt trên của mỗi tầng bằng nửa diện tích của mặt trên của tầng ngay bên dưới và diện tích mặt trên của tầng $1$ bằng nửa diện tích của đế tháp. Tính diện tích mặt trên cùng.
	\loigiai{
		Diện tích bề mặt của mỗi tầng lập thành một cấp số nhân có công bội $q=\dfrac{1}{2}$ và $u_1=\dfrac{12288}{2}=6144$.\\
		Khi đó diện tích mặt trên cùng là
		$u_{11}=u_{1} q^{10}=\dfrac{6144}{2^{10}}=6$.
	}
\end{bt}