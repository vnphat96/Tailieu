\section{CTST - Lớp 11 - Ôn tập giữa học kì 1 - Đề 2}
\caulc
\Opensolutionfile{ans}[ans-ABCD]
\begin{ex}%[1D1N1-3]%[CTST - Lớp 11 - Ôn tập giữa học kì 1 - Đề 2]%[Võ Thị Thùy Trang]
	\immini{
		 Cho điểm $M$ trên đường tròn lượng giác như hình vẽ.
	Mệnh đề nào sau đây đúng?
	\choice
	{$(OA,OM)=-60^\circ+k360^\circ$, $k\in\mathbb{Z}$}
	{$(OA,OM)=60^\circ+k180^\circ$, $k\in\mathbb{Z}$}
	{$(OA,OM)=-60^\circ+k180^\circ$, $k\in\mathbb{Z}$}
	{\True $(OA,OM)=60^\circ+k360^\circ$, $k\in\mathbb{Z}$}
     }
     {
     	\begin{tikzpicture}[scale=0.75, font=\footnotesize, line join=round, line cap=round, >=stealth]
     	\path (0,0) coordinate (O)
     	(3,0) coordinate (A)
     	(0:0)++(60:3) coordinate (M)
     	;
     	\draw[->] (-4,0) -- (4,0) node[above,blue]{$x$};
     	\draw[->] (0,-4) -- (0.,4) node[left,blue]{$y$};
     	\draw[orange] (O) circle (3cm);
     	\draw[rotate=0,->,green!50!black] (0.5,0) arc (0:60:0.5cm);
     	\draw (0.85,0.15) node[above,blue] {$60^\circ$};
     	\draw (O)--(M);
     	\foreach \p/\r in {A/-45,M/90,O/-135}
     	\fill (\p) circle (1pt) node[shift={(\r:3mm)},blue]{$\p$};
     \end{tikzpicture}
 }
	\loigiai{
		Ta có $(OA,OM)=60^\circ+k360^\circ$, $k\in\mathbb{Z}$.
	}
\end{ex}
\begin{ex}%[1D1H2-2]%[CTST - Lớp 11 - Ôn tập giữa học kì 1 - Đề 2]%[Võ Thị Thùy Trang]%[1D1N2-4]
	Cho góc $\alpha \in \left(\dfrac{\pi}{2};\pi\right)$ thỏa mãn $\sin \alpha =\dfrac{2}{3}$. Tính $\cos \alpha$.
	\choice
	{$\cos \alpha=\dfrac{3}{2}$}
	{$\cos \alpha=\dfrac{3}{\sqrt{5}}$}
	{$\cos \alpha=-\dfrac{1}{3}$}
	{\True $\cos \alpha=-\dfrac{\sqrt{5}}{3}$}
	\loigiai{
		Ta có $\sin^2 \alpha+\cos^2 \alpha=1\Leftrightarrow \cos^2 \alpha=1-\sin^2 \alpha=1-\dfrac{4}{9}=\dfrac{5}{9}\Leftrightarrow\cos \alpha=\pm\dfrac{\sqrt{5}}{3}$.\\
		Do $\alpha \in \left(\dfrac{\pi}{2};\pi\right)$ nên $\cos \alpha < 0\Leftrightarrow\cos \alpha=-\dfrac{\sqrt{5}}{3}$.
	}
\end{ex}
\begin{ex}%[1D1N2-4]%[CTST - Lớp 11 - Ôn tập giữa học kì 1 - Đề 2]%[Võ Thị Thùy Trang]
	Công thức nào dưới đây đúng?
	\choice
	{$\sin^2 \alpha+\cos^2 \alpha=2$}
	{$\cos (\pi-\alpha)=\cos \alpha$}
	{\True $\sin \left(\dfrac{\pi}{2}-\alpha\right)=\cos \alpha$}
	{$\tan \left(\dfrac{\pi}{2}-\alpha\right)=-\cot \alpha$}
	\loigiai{
		Ta có $\sin\left(\dfrac{\pi}{2}-\alpha\right)=\cos \alpha$.
	}
\end{ex}
\begin{ex}%[1D1N4-5]%[CTST - Lớp 11 - Ôn tập giữa học kì 1 - Đề 2]%[Võ Thị Thùy Trang]
	Chu kỳ tuần hoàn của hàm số $y=\cos x$ là
	\choice
	{\True $2\pi$}
	{$\pi$}
	{$4\pi$}
	{$\dfrac{\pi}{2}$}
	\loigiai{
		Chu kỳ tuần hoàn của hàm số $y=\cos x$ là $2\pi$.
	}
\end{ex}
\begin{ex}%[1D1H5-3]%[CTST - Lớp 11 - Ôn tập giữa học kì 1 - Đề 2]%[Võ Thị Thùy Trang]
	Phương trình $\sin 2x=-1$ có bao nhiêu nghiệm thuộc đoạn $[-\pi;2\pi]$?
	\choice
	{\True $3$}
	{$1$}
	{$4$}
	{$2$}
	\loigiai{
		Ta có $\sin 2x=-1\Leftrightarrow 2x=-\dfrac{\pi}{2}+k2\pi\Leftrightarrow x=-\dfrac{\pi}{4}+k\pi~ (k\in \mathbb{Z})$.\\
		Xét trên đoạn $[-\pi;2\pi]$, $-\pi\le x\le 2\pi\Leftrightarrow -\pi\le-\dfrac{\pi}{4}+k\pi\le 2\pi\Leftrightarrow \dfrac{3}{4}\le k\le \dfrac{9}{4}$.\\	
		Mà $k\in \mathbb{Z}$ nên $k\in \{0;1;2\}$.\\
		Vậy phương trình có $3$ nghiệm thuộc đoạn $[-\pi;2\pi]$.
	}
\end{ex}
\begin{ex}%[1D2N1-3]%[CTST - Lớp 11 - Ôn tập giữa học kì 1 - Đề 2]%[Võ Thị Thùy Trang]
	Cho dãy số $(u_n)$, biết $u_1=-1$, $u_{n+1}=un+4$ với $n\ge 1$. Ba số hạng đầu tiên của dãy số đó lần lượt là những số nào dưới đây?
	\choice
	{$-1$; $4$;$ 7$}
	{$-1$; $3$; $5$}
	{$-1$; $4$; $6$}
	{\True $-1$; $3$; $7$}
	\loigiai{
		Ta có $u_1=-1$, $u_2=u_1+4=3$, $u_3=u_2+4=7$.\\
		Vậy ba số hạng đầu tiên của dãy là  $-1$; $3$; $7$.
	}
\end{ex}
\begin{ex}%[1D2N2-4]%[CTST - Lớp 11 - Ôn tập giữa học kì 1 - Đề 2]%[Võ Thị Thùy Trang]
	Cho cấp số cộng $(u_n)$ có $u_1=-0{,}1$; $d=0{,}1$. Số hạng thứ $7$ của cấp số cộng này là
	\choice
	{$1{,}6$}
	{$6$}
	{$0{,}5$}
	{\True $0{,}6$}
	\loigiai{
		Số hạng tổng quát của cấp số cộng $(u_n)$ là\\ $u_n=u_1+(n-1)\cdot d\Rightarrow u_7=-0{,}1+(7-1)\cdot 0{,}1=0{,}5$.
	}
\end{ex}
\begin{ex}%[1D2N3-2]%[CTST - Lớp 11 - Ôn tập giữa học kì 1 - Đề 2]%[Võ Thị Thùy Trang]
	Cho cấp số nhân $(u_n)$ với $u_1=-\dfrac{1}{2}$; $u_7=-32$. Tìm $q$.
	\choice
	{$q=\pm\dfrac{1}{2}$}
	{\True $q=\pm 2$}
	{$q=\pm 4$}
	{$q=\pm 1$}
	\loigiai{
		Áp dụng công thức số hạng tổng quát cấp số nhân ta có\\
		$u_7=u_1\cdot q^{7-1} \Rightarrow u_1\cdot q^6=64\Rightarrow q^6=64\Rightarrow q=\pm 2$.
	}
\end{ex}
\begin{ex}%[1H4N1-1]%[CTST - Lớp 11 - Ôn tập giữa học kì 1 - Đề 2]%[Võ Thị Thùy Trang]
	Trong các khẳng định sau, khẳng định nào đúng?
	\choice
	{Qua $2$ điểm phân biệt có duy nhất một mặt phẳng}
	{Qua $3$ điểm phân biệt bất kì có duy nhất một mặt phẳng}
	{\True Qua $3$ điểm không thẳng hàng có duy nhất một mặt phẳng}
	{Qua $4$ điểm phân biệt bất kì có duy nhất một mặt phẳng}
	\loigiai{
		Có một và chỉ một mặt phẳng đi qua ba điểm không thẳng hàng cho trước.
	}
\end{ex}
\begin{ex}%[1H4N2-2]%[CTST - Lớp 11 - Ôn tập giữa học kì 1 - Đề 2]%[Võ Thị Thùy Trang]
	Cho hai đường thẳng phân biệt không có điểm chung cùng nằm trong một mặt phẳng thì hai đường thẳng đó
	\choice
	{\True song song}
	{chéo nhau}
	{cắt nhau}
	{trùng nhau}
	\loigiai{Cho hai đường thẳng phân biệt không có điểm chung cùng nằm trong một mặt phẳng thì hai đường thẳng đó song song.
	}
\end{ex}
\begin{ex}%[1H4H3-2]%[CTST - Lớp 11 - Ôn tập giữa học kì 1 - Đề 2]%[Võ Thị Thùy Trang]
	Cho hình chóp $S.ABCD$ có đáy $ABCD$ là hình bình hành tâm $O$, $M$ là trung điểm $SA$. Khẳng định nào sau đây là đúng?
	\choice
	{\True $OM\parallel (SCD$}
	{$OM\parallel (SBD$}
	{$OM\parallel (SAB)$}
	{$OM\parallel (SAD$}
	\loigiai{
		\immini{
			Ta có $M$ là trung điểm $SA$; $O$ là trung điểm $AC$ nên $OM$ là đường trung bình của $\triangle SAC$.
			Suy ra $OM\parallel SC$.\\
			Mà $\heva{&SC\subset (SCD)\\&OM \not \subset (SCD)\\&OM\parallel SC}$ nên 
			$OM\parallel (SCD)$.
		}
		{\begin{tikzpicture}[>=stealth,line join=round,line cap=round]
				\draw[dashed,black,scale=1.0] (0,0)coordinate(A)--(-130:1.25)coordinate(B) (2.5,0)coordinate(D)--(A)--($(A)+(80:2.5)$)coordinate(S);
				\draw (B)--($(D)-(A)+(B)$)coordinate(C)--(D)--(S)--(B) (S)--(C);
				\coordinate[label=left:$O$] (O) at ($(A)!0.5!(C)$);
				\coordinate[label=right:$M$] (M) at ($(S)!0.5!(A)$);
				\draw[dashed] (A)--(C) (O)--(M);
				\foreach \diem/\vitrin in {S/above,A/left,B/below left,C/below right,D/right} \fill (\diem)circle(1.0pt)node[\vitrin]{$\diem$};
			\end{tikzpicture}
		}	
	}
\end{ex}
\begin{ex}%[1H4H3-2]%[CTST - Lớp 11 - Ôn tập giữa học kì 1 - Đề 2]%[Võ Thị Thùy Trang]
	Có bao nhiêu mặt phẳng song song với cả hai đường thẳng chéo nhau?
	\choice
	{\True Vô số}
	{$3$}
	{$2$}
	{$1$}
	\loigiai{
		\begin{center}
			\begin{tikzpicture}[>=stealth,line join=round,line cap=round,scale=1.0]
				\draw (0,3)--(5,3)node[above]{$a$};
				\draw (1,1)--(4,1)node[above]{$c$};
				\draw (1,1.6)--(3,0.2)node[above]{$b$};
				\draw (0,0)coordinate(A)--(0:4)coordinate(B)--($(80:2)+(0:4)$)coordinate(C)--(80:2)coordinate(D)--cycle;
				\foreach \diem/\vitrin in {A/left,B/right,C/right,D/left} \fill[black](\diem)circle(1.0pt)node[\vitrin]{$\diem$};
			\end{tikzpicture}
			
		\end{center}
		Gọi hai đường thẳng chéo nhau là $a$ và $b$, $c$ là đường thẳng song song với $a$ và cắt $b$.\\
		Gọi mặt phẳng $(\alpha)\equiv (b,c)$. Do $a\parallel c\Rightarrow a\parallel (\alpha)$.\\
		Giả sử mặt phẳng $(\beta) \parallel (\alpha)$ mà $b\subset (\alpha)\Rightarrow b\parallel (\beta)$.\\
		Mặt khác $a\parallel (\alpha)\Rightarrow a \parallel (\beta)$. Có vô số mặt phẳng $(\beta) \parallel (\alpha)$
		nên có vô số mặt phẳng song song với cả hai đường thẳng chéo nhau.
	}
\end{ex}
\Closesolutionfile{ans}
\indapan{6}{ans-ABCD}
\cauds
\Opensolutionfile{ans}[ans-DS]
	\begin{ex}%[1D1H2-2]%[CTST - Lớp 11 - Ôn tập giữa học kì 1 - Đề 2]%[Võ Thị Thùy Trang]
	Cho $\tan x=-4$ với $\dfrac{\pi}{2}<x<\pi$.
	\choiceTF
	{Giá trị của $\cos x$ là $-\dfrac{1}{8}$}
	{\True Giá trị của $\tan 2x$ là $\dfrac{8}{15}$}
	{Giá trị của $\tan(\dfrac{\pi}{4}+x)$ là $\dfrac{3}{5}$}
	{\True Giá trị của biểu thức $A=\dfrac{2\sin x-5\cos x}{3\cos x+\sin x}$ là $13$}
	\loigiai{
		\begin{itemchoice}
			\itemch \textbf{Sai}.\\
			Ta có
			$\dfrac{1}{\cos^2 x}=\tan^2 x+1=17\Rightarrow\cos x=\dfrac{1}{\sqrt{17}}$.
			\itemch \textbf{Đúng}.\\
			Ta có
			$\tan 2x=\dfrac{2\tan x}{1-\tan^2 x}=\dfrac{2\cdot (-4)}{1-(-4)^2}=\dfrac{8}{15}$.
			\itemch \textbf{Sai}.\\
			Ta có
			$\tan (\dfrac{\pi}{4}+x)=\dfrac{\tan \dfrac{\pi}{4}+\tan x}{1-\tan \dfrac{\pi}{4}\cdot \tan x}=\dfrac{1+(-4)}{1-1\cdot (-4)}=-\dfrac{3}{5}$.
			\itemch \textbf{Đúng}.\\
			Ta có
			$A=\dfrac{2\sin x-5\cos x}{3\cos x+\sin x}=\dfrac{\dfrac{2\sin x}{\cos x}-\dfrac{5\cos x}{\cos x}}{\dfrac{3\cos x}{\cos x}+\dfrac{\sin x}{\cos x}}=\dfrac{2\tan x-5}{3+\tan x}=\dfrac{2.(-4)-5}{3+(-4)}=13$.
		\end{itemchoice}
	}
\end{ex}
	\begin{ex}%[1D1V5-5]%[CTST - Lớp 11 - Ôn tập giữa học kì 1 - Đề 2]%[Võ Thị Thùy Trang]
	Cho hàm số lượng giác $y=2\sin 2x-\dfrac{\pi}{3}+2m-1$.
	\choiceTF
	{\True Hàm số có tập xác định $\mathscr{D}=\mathbb{R}$}
	{\True Với $m=0$ hàm số có tập giá trị $T=[-3;1]$}
	{Với $m=2$ hàm số luôn có giao điểm với trục $Ox$}
	{\True Có $3$ giá trị nguyên của tham số $m$ để đường thẳng $y=3$ cắt đồ thị hàm số}
	\loigiai{
		\begin{itemchoice}
			\itemch \textbf{Đúng}.
			Hàm số có tập xác định là $\mathscr{D}=\mathbb{R}$.
			\itemch \textbf{Đúng}.\\
			Với $m=0$ hàm số trở thành $y=2\sin 2x-\dfrac{\pi}{3}-1$.\\
			Ta có $-1\le \sin \left(2x-\dfrac{\pi}{3}\right)\le 1$ nên $-2\le 2\sin \left(2x-\dfrac{\pi}{3}\right) \le 2$.\\
			Do đó $-3\le 2\sin \left(2x-\dfrac{\pi}{3}\right)-1 \le 1$.\\
			Vậy tập giá trị của hàm số là $T=[-3;1]$.
			\itemch \textbf{Sai}.
			Với $m=2$ hàm số trở thành $y=2\sin (2x-\dfrac{\pi}{3})+3$.\\
			Số giao điểm với trục $Ox$ là số nghiệm của phương trình hoành độ giao điểm của hai hàm số $y=2\sin (2x-\dfrac{\pi}{3})+3$ và $y=0$.\\
			Khi đó ta có phương trình\\ $2\sin (2x-\dfrac{\pi}{3})+3=0
			\Leftrightarrow 2\sin (2x-\dfrac{\pi}{3})=-3\Leftrightarrow\sin (2x-\dfrac{\pi}{3})=-\dfrac{3}{2}$.\\
			Vậy hàm số không có giao điểm với trục $Ox$.
			\itemch \textbf{Đúng}.
			Đường thẳng $y=3$ cắt đồ thị hàm số $y=2\sin 2x-\dfrac{\pi}{3}+2m-1$ khi phương trình hoành độ giao điểm
			$y=2\sin (2x-\dfrac{\pi}{3})+2m-1$ và $y=3$ là\\
			$2\sin (2x-\dfrac{\pi}{3})+2m-1=3$ có nghiệm\\
			$\Leftrightarrow \sin(2x-\dfrac{\pi}{3})=2-m$ có nghiệm.\\
			Phương trình trên có nghiệm khi $|2-m|\le 1\Leftrightarrow 1\le 2-m\le 1\Leftrightarrow 1\le m\le 3$.\\
			Suy ra $m\in \{1;2;3\}$.\\
			Vậy có $3$ giá trị nguyên của tham số $m$ để đường thẳng $y=3$ cắt đồ thị hàm số.
		\end{itemchoice}
	}
\end{ex}
	\begin{ex}%[1D2V2-7]%[CTST - Lớp 11 - Ôn tập giữa học kì 1 - Đề 2]%[Võ Thị Thùy Trang]
	Một sinh viên sau khi ra trường và xin vào làm cho một trung tâm với mức lương khởi điểm là $120$ triệu đồng một năm. Cứ sau mỗi năm, trung tâm trả thêm cho sinh viên $24$ triệu đồng. Gọi $u_n$ là số tiền lương mà sinh viên đó nhận được ở năm thứ $n$.
	\choiceTF
	{\True Số tiền lương sinh viên nhận được ở năm thứ hai là $144$ triệu đồng}
	{Số tiền lương sinh viên nhận được ở năm thứ $10$ là $330$ triệu đồng}
	{Dãy số $(u_n)$ là cấp số cộng có $u_1=120$ và công sai $d=20$}
	{Giả sử, mỗi năm bạn sinh viên chi tiêu tiết kiệm hết $70$ triệu đồng. Vậy sau ít nhất $10$ năm thì sinh viên đó mua được căn chung cư $2$ tỉ đồng}
	\loigiai{
		\begin{itemchoice}
			\itemch\textbf{ Đúng}.\\
			Ta thấy, số tiền lương năm sau hơn năm trước $24$ triệu đồng nên số tiền lương hằng năm $(u_n)$ là cấp số cộng có $u_1=120$ và công sai $d=24$. Do đó
			$u_n=u_1+(n-1)d=120+(n-1)24=24n+96$.\\
			Số tiền lương sinh viên nhận được ở năm thứ hai là $u_2=144$.
			\itemch \textbf{Sai}.\\
			Số tiền lương sinh viên nhận được ở năm thứ $10$ là $u_{10}=24\cdot 10+96=336$.
			\itemch \textbf{Sai}. Vì $d=24$.
			\itemch \textbf{Sai}.\\
			Tổng số tiền bạn sinh viên tiết kiệm được sau $n$ năm là\\
			$S_n=\dfrac{n}{2}[2u_1+(n-1)d]-70n=[2\cdot 120+(n-1)24]-70n=12n^2+38n$.\\
			Ta có $S_n\ge 2\,000\Leftrightarrow 12n^2+338n-2\,000\ge 0\Leftrightarrow \hoac{&n\ge 11{,}42\\&n\le -14{,}59.}$\\
			Do đó sau ít nhất $12$ năm thì sinh viên đó có thể mua được chung cư $2$ tỉ đồng.
		\end{itemchoice}
	}
\end{ex}
	\begin{ex}%[1H4V4-6]%[CTST - Lớp 11 - Ôn tập giữa học kì 1 - Đề 2]%[Võ Thị Thùy Trang]
	Cho hình chóp $S.ABCD$ có đáy $ABCD$ là hình bình hành và $O$ là giao điểm của hai đường chéo của hình bình hành $ABCD$. Gọi $M$, $N$ lần lượt là trung điểm của $SA$ và $SD$. Khi đó
	\choiceTF
	{\True Điểm $O$ là điểm chung của $(OMN)$ và $(ABCD)$}
	{\True $MN\parallel BC$}
	{\True $OM\parallel (SBC)$}
	{Giao tuyến của $(OMN)$ và $(SBC)$ là đường thẳng $d$ song song với hai đường thẳng $MN$ và $BC$}
	\loigiai{
		\begin{center}
			\begin{tikzpicture}[>=stealth,line join=round,line cap=round]
				\draw[dashed,black,scale=1.0] (0,0)coordinate(A)--(-130:1.25)coordinate(B) (2.5,0)coordinate(D)--(A)--($(A)+(80:2.5)$)coordinate(S);
				\draw (B)--($(D)-(A)+(B)$)coordinate(C)--(D)--(S)--(B) (S)--(C);
				\coordinate[label=left:$O$] (O) at ($(A)!0.5!(C)$);
				\coordinate[label=below right:$M$] (M) at ($(S)!0.5!(A)$);
				\coordinate[label=right:$N$] (N) at ($(S)!0.5!(D)$);
				\draw[dashed] (A)--(C) (B)--(D) (O)--(N)--(M)--(O) ;
				\foreach \diem/\vitrin in {S/above,A/left,B/below left,C/below right,D/right} \fill (\diem)circle(1.0pt)node[\vitrin]{$\diem$};
			\end{tikzpicture}
		\end{center}
		\begin{itemchoice}
			\itemch \textbf{Đúng}. Vì $\heva{&O\in (OMN)\\&O=AC\cap BD}\Rightarrow \heva{&O\in (OMN)\\&O\in (ABCD)}\Rightarrow O\in (OMN)\cap (ABCD)$.
			\itemch \textbf{Đúng}. Vì $M$, $N$ lần lượt là trung điểm của $SA$ và $SD$ nên $MN\parallel AD$ mà $ABCD$ là hình bình hành nên $AD\parallel BC$.\\
			Vậy $\heva{&MN\parallel AD\\&AD\parallel BC}\Rightarrow MN\parallel BC$.
			\itemch \textbf{Đúng}. Vì $M$, $O$ lần lượt là trung điểm của $SA$ và $AC$ nên $MO\parallel SC$.\\
			Vậy $\heva{&OM\parallel SC\\&SC\subset (SBC)}\Rightarrow OM \parallel (SBC)$.
			\itemch \textbf{Sai}. Vì $\heva{&MN\parallel BC\\&BC\subset (SBC)}\Rightarrow MN\parallel (SBC)$.\\
			Vậy $\heva{&MN\parallel (SBC)\\&OM\parallel (SBC)\\&MN\cap OM=M\\&MN,\, OM\subset (OMN)}\Rightarrow (OMN)\parallel (SBC)$.\\
			Do đó hai mặt phẳng $(OMN)$ và $(SBC)$ không có đường thẳng giao tuyến.
		\end{itemchoice}
	}
\end{ex}
\Closesolutionfile{ans}

\indapan{3}{ans-DS}

\caukq

\Opensolutionfile{ans}[ans-KQ]
\begin{ex}%[1D1H3-5]%[CTST - Lớp 11 - Ôn tập giữa học kì 1 - Đề 2]%[Võ Thị Thùy Trang]
	Biết $\sqrt{\sin^2x(4+\cot x)+\cos^2x(1+3\tan x)}=\left|a\sin x+b\cos x\right|$. Tính $a-b$.
		\shortans{$1$}
	\loigiai{
		Ta có
		\begin{eqnarray*}
			\sqrt{\sin^2 x(4+\cot x)+\cos^2 x(1+3\tan x)}&=&\sqrt{4\sin^2 x+\sin^2 x \cot x+\cos^2 x+3\cos^2 x \tan x}\\
			&=&\sqrt{4\sin^2 x+\sin^2 x\cdot \dfrac{\cos x}{\sin x}+\cos^2 x+3\cos^2 x\cdot\dfrac{\sin x}{\cos x}}\\
			&=& \sqrt{4\sin^2 x+\sin x\cdot \cos x+\cos^2 x+3\cos x\cdot \sin x}\\
			&=& \sqrt{4\sin^2 x+4\sin x\cdot \cos x+\cos^2 x}\\
			&=& \sqrt{(2\sin x+\cos x)^2}=\left|2\sin x+\cos x\right| 
		\end{eqnarray*}
		Do đó $a=2$, $b=1$. Vậy $a-b=1$
	}
\end{ex}
\begin{ex}%[1D1H3-5]%[CTST - Lớp 11 - Ôn tập giữa học kì 1 - Đề 2]%[Võ Thị Thùy Trang]
	Biết với mọi $x$ thì $\sin^6 x+\cos^6 x=a+b\cos 4x$ $(a,~b\in \mathbb{Q})$.
	\shortans{$1$}
	\loigiai{
		Ta có 
		\begin{eqnarray*}
			\sin^6 x+\cos^6 x &=& \left(\sin^2 x\right)^3+ \left(\cos^2 x\right)^3\\
			&=& \left(\sin^2 x+\cos^2 x\right)^3-3\sin^2 x\cos^2 x\left(\sin^2 x+\cos^ x\right)\\
			&=& 1-3\left(\dfrac{\sin 2x}{2}\right)^2=1-\dfrac{3}{4}\sin^2    2x\\
			&=&1-\dfrac{3}{8}(1-\cos 4x)=\dfrac{5}{8}+\dfrac{3}{8}\cos 4x.	
		\end{eqnarray*}
		Suy ra $a=\dfrac{5}{8}$, $b=\dfrac{3}{8}$.\\
		Vậy $a+b=\dfrac{5}{8}+\dfrac{3}{8}=1$.
	}
\end{ex}
\begin{ex}%[1D1V5-3]%[CTST - Lớp 11 - Ôn tập giữa học kì 1 - Đề 2]%[Võ Thị Thùy Trang]
	Phương trình $\sin \left(\dfrac{2x}{4}-\dfrac{\pi}{4}\right)=\sin \left( \dfrac{3\pi}{4}x+\dfrac{\pi}{4}\right)$ có tổng các nghiệm thuộc khoảng $(0;\pi)$ bằng $a\cdot \pi$. Tìm $a$.
	\shortans{$1$}
	\loigiai{
		Ta có
		\begin{eqnarray*}
			\sin \left(\dfrac{2x}{4}-\dfrac{\pi}{4}\right)=\sin \left( \dfrac{3\pi}{4}x+\dfrac{\pi}{4}\right) &\Leftrightarrow&\hoac{&\dfrac{2x}{4}-\dfrac{\pi}{4}=\dfrac{3\pi}{4}x+\dfrac{\pi}{4}+k2\pi\\&\dfrac{2x}{4}-\dfrac{\pi}{4}=\pi-(\dfrac{3\pi}{4}x+\dfrac{\pi}{4})+k2\pi}\\
			&\Leftrightarrow& \hoac{&x=-\dfrac{\pi}{2}+k2\pi\\&x=\dfrac{2\pi}{3}+k2\pi} (k\in \mathbb{Z}).
		\end{eqnarray*}
		Họ nghiệm $x=-\dfrac{\pi}{2}+k2\pi$ không có nghiệm nào thuộc khoảng $(0;\pi)$.\\
		$\dfrac{2\pi}{3}+k2\pi \in (0;\pi ) \Rightarrow 0< \dfrac{2\pi}{3}+k2\pi < \pi \Leftrightarrow -\dfrac{1}{3} < k < \dfrac{1}{6}\Rightarrow k\in \{0;1\}$.\\
		Vậy phương trình có hai nghiệm thuộc khoảng $(0;\pi)$ là $x=\dfrac{2\pi}{3}$ và $x=\dfrac{5\pi}{6}$.\\
		Từ đó suy ra tổng các nghiệm thuộc khoảng $(0;\pi)$ của phương trình này bằng $\pi$. Vậy $a=1$.
	}
\end{ex}
\begin{ex}%[1D2V2-6]%[CTST - Lớp 11 - Ôn tập giữa học kì 1 - Đề 2]%[Võ Thị Thùy Trang]
	Cho một cấp số cộng $(u_n)$ có số hạng đầu $u_1=1$ và tổng của $100$ số hạng đầu bằng $24\,850$. Khi đó $S=\dfrac{1}{u_1\cdot u_2}+\dfrac{1}{u_2u_3}+\dfrac{1}{u_3\cdot u_4}+\cdots+\dfrac{1}{u_{28} \cdot u_{29}}+\dfrac{1}{u_{29} \cdot u_{30}}=\dfrac{a}{b}$, trong đó $\dfrac{a}{b}$ là phân số tối giản và $a$, $b \in \mathbb{N}^*$. Tính tổng $a+b$.
	\shortans{$175$}
	\loigiai{
		\begin{align*}
			& S_{100}=100 \cdot u_1+\dfrac{100\cdot 99}{2} \cdot d \Leftrightarrow 24850=100 \cdot 1+\dfrac{100\cdot 99}{2} \cdot d \Leftrightarrow 4950 d=14\,750 \Leftrightarrow d=5. \\
			& u_{30}=u_1+29 \cdot d=146 \\
			& S=\dfrac{1}{u_1 \cdot u_2}+\dfrac{1}{u_2 \cdot u_3}+\dfrac{1}{u_3 u_4}+\cdots+\dfrac{1}{u_{28} u_{29}}+\dfrac{1}{u_{29} u_{30}} \\
			& \Rightarrow 5 S=\dfrac{d}{u_1 \cdot u_2}+\dfrac{d}{u_2 u_3}+\dfrac{d}{u_3 \cdot u_4}+\cdots+\dfrac{d}{u_{28} \cdot u_{29}}+\dfrac{d}{u_{29} \cdot u_{30}}\\
			&=\left(\dfrac{1}{u_1}-\dfrac{1}{u_2}\right)+\left(\dfrac{1}{u_2}-\dfrac{1}{u_3}\right)+\left(\dfrac{1}{u_3}-\dfrac{1}{u_4}\right)+\cdots.+\left(\dfrac{1}{u_{2 s}}-\dfrac{1}{u_{29}}\right)+\left(\dfrac{1}{u_{29}}-\dfrac{1}{u_{30}}\right) \\
			&=\dfrac{1}{u_1}-\dfrac{1}{u_{30}}=\dfrac{1}{1}-\dfrac{1}{146}=\dfrac{145}{146} \Rightarrow S=\dfrac{29}{146} \\
			& \Rightarrow a=29,~ b=146 \Rightarrow a+b=175.
		\end{align*}
	}
\end{ex}

\begin{ex}%[1H4V1-4]%[CTST - Lớp 11 - Ôn tập giữa học kì 1 - Đề 2]%[Võ Thị Thùy Trang]
	Cho hình chóp $SABC$. Gọi $M$, $N$ lần lượt là trung điểm của $SA$ và $BC$, $P$ là điểm thuộc cạnh $AB$ sao cho $\dfrac{AP}{AB}=\dfrac{2}{3}$. Đường thẳng $SC$ cắt mặt phẳng $(MNP)$ tại $Q$. Biết ti số $\dfrac{SQ}{SC}=\dfrac{a}{b}$, trong đó $\dfrac{a}{b}$ là phân số tối giản và $a$, $b \in \mathbb{N}^*$. Tổng $a^2+b^2$ bằng bao nhiêu?
	\shortans{$13$}
	\loigiai{
		\immini{
			Trong mặt phẳng $(ABC)$, gọi $E=AC\cap PN$.\\
			Trong mặt phằng ($SAC$), ta có $Q=SC\cap EM$.\\
			Áp dụng định lí Menelaus cho tam giác $ABC$ ta có\\ $\dfrac{EA}{EC} \cdot \dfrac{NC}{NB} \cdot \dfrac{PB}{PA}=1\Rightarrow \dfrac{EA}{EC} \cdot 1\cdot \dfrac{1}{2}=1\Rightarrow \dfrac{EA}{EC}=2$.\\
			Áp dụng định lí Menelaus cho tam giác $SAC$ ta có\\ $\dfrac{EA}{EC} \cdot \dfrac{QC}{QS} \cdot \dfrac{MS}{MA}=1\Rightarrow 2\cdot \dfrac{QC}{QS} \cdot 1=1$ $\Rightarrow \dfrac{QC}{QS}=\dfrac{1}{2}$.\\
			$ \Rightarrow SQ=2QC\Rightarrow SQ=\dfrac{2}{3} SC\Rightarrow \dfrac{SQ}{SC}=\dfrac{2}{3}$.\\
			Suy ra $a=2$, $b=3\Rightarrow a^2+b^2=13$.
		}
		{
			\begin{tikzpicture}[>=stealth,line join=round,line cap=round]
			\draw[black,scale=1.0] (0,0)coordinate(A)--(-35:1.8)coordinate(B) (2.5,0)coordinate(C) ($(A)+(60:2.5)$)coordinate(S)--(B) (S)--(A);
			\coordinate[label=left:$M$] (M) at ($(A)!0.5!(S)$);
			\coordinate (N) at ($(C)!0.5!(B)$);
			\coordinate (P) at ($(A)!2/3!(B)$);
			\coordinate (E) at (intersection of P--N and A--C);
			\coordinate (Q) at (intersection of M--E and S--C);
			\draw (M)--(P) (B)--(N)--(E)--(Q)--(N) (S)--(Q);
			\draw[dashed]  (A)--(C)--(N) (P)--(N)--(M)--(Q)--(C)--(E);
			\foreach \diem/\vitrin in {S/above,A/left,B/below left,P/below,N/below,Q/above right,C/above right,E/right}	\fill (\diem)circle(1.0pt)node[\vitrin]{$\diem$};
		\end{tikzpicture}	
		}
		}
\end{ex}
\begin{ex}%[1H4V4-6]%[CTST - Lớp 11 - Ôn tập giữa học kì 1 - Đề 2]%[Võ Thị Thùy Trang]
	Cho lăng trụ $ABCD.A'B'C'D'$ có đáy $ABCD$ là hình vuông, $AB=1$, $AA'=2$. Gọi $L$ là trung điểm $B'D$, mặt phång $(P)$ qua $L$ và song song $AC$ lần lượt cắt $AA'$, $CC'$, $DD'$ tại $E$, $F$, $K$. Đặt $\dfrac{DK}{DD'}=x$. Khi $(EFK) \parallel\left(MA'C'\right)$ thì $P=\dfrac{2025x}{1005}$ bằng.
	\shortans{$2{,}01$}
	\loigiai{
		\begin{center}
				\begin{tikzpicture}[>=stealth,line join=round,line cap=round,scale=1]
				\draw[dashed] (0,0)coordinate(A)--(-130:1.25)coordinate(B) (2.5,0)coordinate(D)--(A)--($(A)+(90:2.5)$)coordinate(A');
				\draw (B)--($(D)-(A)+(B)$)coordinate(C)--(D)--($(D)+(90:2.5)$)coordinate(D')--(A')--($(B)+(90:2.5)$)coordinate(B')--($(D')-(A')+(B')$)coordinate(C')--(D') (C')--(C) (B')--(B);
				\coordinate (L) at ($(D)!0.5!(B')$);
				\coordinate (O') at ($(A')!0.5!(C')$);
				\coordinate[label=left:$M$] (M) at ($(B')!0.5!(B)$);
				\coordinate (E) at ($(A')!0.75!(A)$);
				\coordinate (F) at ($(C')!0.75!(C)$);
				\coordinate (K) at ($(D')!0.1!(D)$);
				\coordinate (Q) at (intersection of M--O' and B'--D);
				\draw[dashed] (B')--(D) (M)--(O')--(L) (A')--(M)--(C') (L)--(K)--(E)--(F);
				\draw  (A')--(C') (B')--(D') (K)--(F);
				\foreach \diem/\vitrin in {A/left,B/below left,C/below right,D/right,A'/left,B'/below left,C'/below right,D'/right,Q/left,E/below left,F/below left,K/right,L/below,O'/above} \fill (\diem)circle(1.0pt)node[\vitrin]{$\diem$};
			\end{tikzpicture}
		\end{center}
		Trong mặt phång $\left(A'B'C'D'\right)$, $O'=A'C'\cap B'D'$.\\
		Trong mặt phång $\left(B'D'DB\right)$, $Q=B'D\cap MO'$.\\
		$ML$ là đường trung bình của tam giác $B'BD$ nên $ML\parallel BD\parallel B'D' \quad(1)$.\\
		$O'L$ là đường trung bình của tam giác $B'D'D$ nên $LO' \parallel D'D\parallel B'B \quad(2)$\\
		Từ $(1)$,$(2)$ suy ra tư giác $MIO'B'$ là hình bình hành nên $Q$ là trung điểm $B'L\Rightarrow QO'\parallel LD' \quad(3)$\\
		Ta có $EF\parallel A'C'$ nên để $(EFK) \parallel\left(MA'C'\right)$ thì $\Rightarrow QO'\parallel LK \quad(4)$.\\
		Từ $(3)$, $(4)$ suy ra $K\equiv D' \Rightarrow \dfrac{DK}{DD'}=1\Rightarrow x=1$.\\
		Vây $P=\dfrac{2025x}{1005}=\dfrac{2025}{1005} \approx 2{,}01$.}
\end{ex}
\Closesolutionfile{ans}
\indapan{6}{ans-KQ}