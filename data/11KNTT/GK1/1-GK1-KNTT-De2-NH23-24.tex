\section*{ÔN TẬP KIỂM TRA GIỮA KÌ 1 - ĐỀ 02}
\setcounter{ex}{0}\setcounter{bt}{0}
\noindent{\bf\fontfamily{qag}\selectfont\color{violet}A. PHẦN TRẮC NGHIỆM}
\Opensolutionfile{ans}[ans/ans-1-GK1-KNTT-De15-NH23-24]

%%==========Câu 1
\begin{ex}%[1K1Y1-5] 
	$\sin\alpha>0$ khi điểm cuối của cung $\alpha$ trên đường tròn lượng giác thuộc các góc phần tư thứ
	\choice
	{I và III}
	{\True I và II}
	{II và IV}
	{I và IV}
	\loigiai{
		$\sin\alpha>0$ khi điểm cuối của cung $\alpha$ trên đường tròn lượng giác các góc phần tư thứ I và II.	
	}
\end{ex}
	\begin{ex}%[1K1Y1-8]
	Trong các khẳng định sau, khẳng định nào sai?
	\choice
	{$\tan\left(\pi-\alpha\right)=-\tan\alpha$}
	{\True $\tan\left(\pi+\alpha\right)=-\tan\alpha$}
	{$\tan\left(-\alpha\right)=-\tan\alpha$}
	{$\tan\left(\dfrac{\pi}{2}-\alpha\right)=\cot\alpha$}
	\loigiai{
		Vì $\tan\left(\pi+\alpha\right)=\tan\alpha$ nên khẳng định sai là $\tan\left(\pi+\alpha\right)=-\tan\alpha$.	
	}
\end{ex}
	\begin{ex}%[1K1Y1-5]
	Khi biểu diễn cung lượng giác $\alpha$ lên đường tròn lượng giác thì điểm cuối của cung $\alpha$ thuộc góc phần tư thứ ba của đường tròn lượng giác. Khẳng định nào sau đây là \textbf{đúng}?
	\choice
	{$\sin\alpha>0$}
	{$\cos\alpha>0$}
	{\True $\tan\alpha>0$}
	{$\cot\alpha<0$}
	\loigiai{
		Vì khi biểu diễn cung lượng giác $\alpha$ lên đường tròn lượng giác thì điểm cuối của cung $\alpha$ thuộc góc phần tư thứ ba của đường tròn lượng giác nên $\tan\alpha>0$.
	}
\end{ex}
\begin{ex}%[1K1B1-6]
	Cho góc $\alpha$ thỏa mãn $\sin\alpha=\dfrac{4}{5}$ và $\dfrac{\pi}{2}<\alpha<\pi$. Tính $\cos\alpha$.
	\choice
	{$\cos\alpha=\dfrac{3}{5}$}
	{\True $\cos\alpha=-\dfrac{3}{5}$}
	{$\cos\alpha=-\dfrac{1}{5}$}
	{$\cos\alpha=\dfrac{1}{5}$}
	\loigiai{
		Ta có $\cos^2\alpha=1-\sin^2\alpha=1-\left(\dfrac{4}{5}\right)^2 \Leftrightarrow \cos^2\alpha =\dfrac{9}{25} \Leftrightarrow \cos\alpha=\pm\dfrac{3}{5}$.\\
		Mà $\dfrac{\pi}{2}<\alpha<\pi$ nên $\cos\alpha<0$. Vậy $\cos\alpha=-\dfrac{3}{5}$.	
	}
\end{ex}
\begin{ex}%[1K1Y2-1]
	Trong các công thức sau, công thức nào đúng?
	\choice
	{\True $\sin\left(a-b\right)=\sin a\cdot\cos b-\sin b\cdot\cos a$}
	{$\cos\left(a-b\right)=\cos a\cdot\cos b-\sin a\cdot\sin b$}
	{$\sin\left(a+b\right)=\sin a\cdot\cos b-\sin b\cdot\cos a$}
	{$\cos\left(a+b\right)=\cos a\cdot\cos b$ $+$ $\sin a\cdot\sin b$}
	\loigiai{
		Công thức cộng $\sin\left(a-b\right)=\sin a\cdot\cos b-\sin b\cdot\cos a$.	
	}
\end{ex}
\begin{ex}%[1K1K2-1]
	Rút gọn biểu thức $\sin\left(a-17^\circ\right)\cos\left(a+13^\circ\right)-\sin\left(a+13^\circ\right)\cos\left(a-17^\circ\right)$, ta được
\choice
{$\sin2a$}
{$\cos2a$}
{\True $-\dfrac{1}{2}$}
{$\dfrac{1}{2}$}
\loigiai{ 
	Ta có $\sin\left(a-17^\circ\right)\cdot\cos\left(a+13^\circ\right)-\sin\left(a+13^\circ\right)\cdot\cos\left(a-17^\circ\right)=\sin\left[\left(a-17^\circ\right)-\left(a+13^\circ\right)\right]$\\
	$=\sin\left(-30^\circ\right)=-\dfrac{1}{2}$.	
	}
\end{ex}
\begin{ex}%[1K1K2-3]
	Với $\alpha$ là số thực bất kỳ, mệnh đề nào sau đây là mệnh đề đúng? 
	\choice
	{$\cos2\alpha+\cos4\alpha=2\cos2\alpha\cdot\cos6\alpha$}
	{$\sin2\alpha+\sin4\alpha=2\sin\alpha\cdot\cos3\alpha$}
	{$\cos2\alpha-\cos4\alpha=-2\sin3\alpha\cdot\sin\alpha$}
	{\True $\sin2\alpha-\sin4\alpha=-2\cos3\alpha\cdot\sin\alpha$}
	\loigiai{
		Ta có\\
		$\cos2\alpha+\cos4\alpha=2\cos\dfrac{2\alpha+4\alpha}{2}\cdot\cos\dfrac{2\alpha-4\alpha}{2}=2\cos3\alpha\cdot\cos\alpha$. Do đó $\cos2\alpha+\cos4\alpha=2\cos2\alpha\cdot\cos6\alpha$ sai.\\
		$\sin2\alpha+\sin4\alpha=2\sin\dfrac{2\alpha+4\alpha}{2}\cdot\cos\dfrac{2\alpha-4\alpha}{2}=2\sin\alpha.\cos\alpha$. Do đó $\sin2\alpha+\sin4\alpha=2\sin\alpha\cdot\cos3\alpha$ sai.\\
		$\cos2\alpha-\cos4\alpha=-2\sin\dfrac{2\alpha+4\alpha}{2}\cdot\sin\dfrac{2\alpha-4\alpha}{2}=2\sin3\alpha\cdot\sin\alpha$. Do đó $\cos2\alpha-\cos4\alpha=-2\sin3\alpha\cdot\sin\alpha$ sai.\\	
		$\sin2\alpha-\sin4\alpha=2\cos\dfrac{2\alpha+4\alpha}{2}\cdot\sin\dfrac{2\alpha-4\alpha}{2}=-2\cos3\alpha\cdot\cos\alpha$ là đáp án đúng.
	}
\end{ex}
\begin{ex}%[1K1B3-1]
	Tập xác định của hàm số $y=\tan x$ là 
	\choice
	{$\mathscr{D}=[-1;1]$}
	{$\mathscr{D}=\mathbb{R}\setminus\left\{k\pi,k\in\mathbb{Z}\right\}$}
	{\True $\mathscr{D}=\mathbb{R}\setminus\left\{\dfrac{\pi}{2}+k\pi,k\in\mathbb{Z}\right\}$}
	{$\mathscr{D}=\mathbb{R}$}
	\loigiai{
		Điều kiện xác định $\cos x\neq 0 \Leftrightarrow x\neq \dfrac{\pi}{2}+k\pi,k\in\mathbb{Z}$.\\
		Vậy tập xác định của hàm số là $\mathscr{D}=\mathbb{R}\setminus\left\{\dfrac{\pi}{2}+k\pi,k\in\mathbb{Z}\right\}$.
	}
\end{ex}
\begin{ex}%[1K1Y3-4]
	Hàm số $y=\sin x$ tuần hoàn với chu kỳ là 
	\choice
	{$\dfrac{\pi}{2}$}
	{$\dfrac{\pi}{3}$}
	{\True $2\pi$}
	{$\pi$}
	\loigiai{
		Hàm số $y=\sin x$ tuần hoàn với chu kỳ là $2\pi$.
	}
\end{ex}
\begin{ex}%[1K1Y3-3]
	Trong các hàm số sau $y=\sin x$, $y=\cos x$, $y=\tan x$, $y=\cot x$ hàm số nào là hàm số chẵn? 
	\choice
	{\True $y=\cos x$}
	{$y=\tan x$}
	{$y=\cot x$}
	{$y=\sin x$}
	\loigiai{
		Hàm số $y=\cos x$ là hàm số chẵn.
	}
\end{ex}
\begin{ex}%[1K1B3-1]
	Tìm tập xác định $\mathscr{D}$ của hàm số $y=\dfrac{3\tan x-5}{1-\sin^2 x}$.
	\choice
	{$\mathscr{D}=\mathbb{R}\setminus\left\{\dfrac{\pi}{2}+k2\pi,k\in\mathbb{Z}\right\}$}
	{\True $\mathscr{D}=\mathbb{R}\setminus\left\{\dfrac{\pi}{2}+k\pi,k\in\mathbb{Z}\right\}$}
	{$\mathscr{D}=\mathbb{R}\setminus\left\{\pi+k\pi,k\in\mathbb{Z}\right\}$}
	{$\mathscr{D}=\mathbb{R}$}
	\loigiai{
	Hàm số xác định khi và chỉ khi $\heva{&1-\sin^2 x\neq 0\\&\cos x\neq 0}$\\
	$\Leftrightarrow \heva{&\sin^2 x\neq 1\\&\cos x\neq 0} \Leftrightarrow \cos x\neq 0 \Leftrightarrow x\neq\dfrac{\pi}{2}+k\pi,k\in\mathbb{Z}$.\\
	Vậy tập xác định của hàm số là $\mathscr{D}=\mathbb{R}\setminus\left\{\dfrac{\pi}{2}+k\pi,k\in\mathbb{Z}\right\}$. 
	}
\end{ex}
\begin{ex}%[1K1K3-5]
	Gọi $M, m$, lần lượt là giá trị lớn nhất, giá trị nhỏ nhất của hàm số $y=\sqrt{3}\sin 2x - \cos 2x -1$. Giá trị của $M+m$ bằng
	\choice
	{$M+m=0$}
	{\True $M+m=-2$}
	{$M+m=1$}
	{$M+m=-1$}
	\loigiai{
		Ta có $y=\sqrt{3}\sin 2x - \cos 2x -1=2\left(\dfrac{\sqrt{3}}{2}\sin 2x -\dfrac{1}{2}\cos 2x\right)-1= 2\sin\left(2x-\dfrac{\pi}{6}\right)-1$\\
		Vì $\forall x\in\mathbb{R}, -1\leq \sin\left(2x-\dfrac{\pi}{6}\right)\leq1$ nên suy ra $-2 -1\leq 2\sin\left(2x-\dfrac{\pi}{6}\right)-1\leq2-1$.\\
		Do đó, $-3\leq y\leq1,\forall x\in\mathbb{R}$.\\
		Do đó giá trị lớn nhất và giá trị nhỏ nhất của hàm số $y=\sqrt{3}\sin 2x - \cos 2x -1$ lần lượt là $M=1,m=-3$.\\
		Khi $\sin\left(2x-\dfrac{\pi}{6}\right)=1 \Leftrightarrow 2x-\dfrac{\pi}{6}=\dfrac{\pi}{2}+k2\pi \Leftrightarrow x=\dfrac{\pi}{3}+k2\pi,k\in\mathbb{Z}$.\\
		$\sin\left(2x-\dfrac{\pi}{6}\right)=-1 \Leftrightarrow 2x-\dfrac{\pi}{6}=-\dfrac{\pi}{2}+k2\pi \Leftrightarrow x=-\dfrac{\pi}{6}+k\pi,k\in\mathbb{Z}$.\\
		Vậy $M+m=1-3=-2$.
	}
\end{ex}
\begin{ex}%[1K1Y4-2]
	Phương trình nào sau đây có nghiệm?
	\choice
	{$\sin2x=2$}
	{$\cos2x=-2$}
	{\True $\sin3x=\dfrac{2}{3}$}
	{$\cos x =\pi$}
	\loigiai{
		Do $\dfrac{2}{3}\in[-1;1]$ nên phương trình $\sin3x=\dfrac{2}{3}$ có nghiệm.
	}
\end{ex}
\begin{ex}%[1K1B4-5]
	Nghiệm của phương trình $\sin x=\dfrac{1}{2}$ là
	\choice
	{$x=\dfrac{\pi}{6}+k\pi$; $x=\dfrac{5\pi}{6}+k\pi,k\in\mathbb{Z}$}
	{$x=\dfrac{\pi}{6}+k\pi$; $x=\dfrac{-\pi}{6}+k\pi,k\in\mathbb{Z}$}
	{\True $x=\dfrac{\pi}{6}+k2\pi$; $x=\dfrac{5\pi}{6}+k2\pi,k\in\mathbb{Z}$}
	{$x=\dfrac{\pi}{6}+k2\pi$; $x=\dfrac{-\pi}{6}+k2\pi,k\in\mathbb{Z}$}
	\loigiai{
		Ta có $\sin x=\dfrac{1}{2} \Leftrightarrow \sin x=\sin\dfrac{\pi}{6} \Leftrightarrow \hoac{&x=\dfrac{\pi}{6}+k2\pi\\&x=\dfrac{5\pi}{6}+k2\pi}, k\in\mathbb{Z}$. 
	}
\end{ex}
\begin{ex}%[1K1K4-5]
	Tổng tất cả các nghiệm của phương trình $\sin\left(x+\dfrac{\pi}{4}\right)+\cos\left(x-\dfrac{3\pi}{4}\right)=0$ thuộc $\left(0;5\pi\right)$ bằng
	\choice
	{\True $10\pi$}
	{$7\pi$}
	{$6\pi$}
	{$9\pi$}
	\loigiai{
		Ta có\\
		$\sin\left(x+\dfrac{\pi}{4}\right)+\cos\left(x-\dfrac{3\pi}{4}\right)=0 \Leftrightarrow \sin x\cdot\cos\dfrac{\pi}{4}+\cos x\cdot\sin\dfrac{\pi}{4}+\cos x\cdot\cos\dfrac{3\pi}{4}+\sin x\cdot\sin\dfrac{3\pi}{4}=0$\\
		$\Leftrightarrow \sqrt{2}\sin x=0 \Leftrightarrow \sin x=0 \Leftrightarrow x=k\pi,k\in\mathbb{Z}$.\\
		Vì $x\in\left(0;5\pi\right) \Rightarrow 0<k\pi<5\pi \Leftrightarrow 0<k<5$.\\
		Vì $k\in\mathbb{Z} \Rightarrow k\in\left\{1;2;3;4\right\} \Rightarrow x\in\left\{\pi;2\pi;3\pi;4\pi\right\}$.\\
		Khi đó, tổng các nghiệm của phương trình là $S=\pi+2\pi+3\pi+4\pi=10\pi$.	
	}
\end{ex}
\begin{ex}%[1K2B5-2]
	Cho dãy số $\left(u_{n}\right)$ có số hạng tổng quát $u_{n}=n^2-3$. Số hạng thứ $10$ của dãy số là
	\choice
	{$7$}
	{\True $97$}
	{$100$}
	{$103$}
	\loigiai{
		Số hạng thứ $10$ của dãy số là $u_{10}=10^2-3=97$.	
	}
\end{ex}
\begin{ex}%[1K2B5-2]
	Cho dãy số $0;2;4;6;\ldots;304$. Hỏi dãy số trên có bao nhiêu số hạng?
	\choice
	{$304$}
	{$152$}
	{\True $153$}
	{$305$}
	\loigiai{
		Số các số hạng của dãy số là $\dfrac{304-0}{2}+1=153$.	
	}
\end{ex}
\begin{ex}%[1K2B6-1]
	Cho cấp số cộng $1;1;1;\ldots$. Công sai của cấp số cộng trên là
	\choice
	{\True $0$}
	{$1$}
	{$-1$}
	{$\varnothing$}
	\loigiai{
		Cấp số cộng có công sai $d=1-1=0$. Đây là dãy số không đổi.	
	}
\end{ex}
\begin{ex}%[1K2B6-3]
	Cho cấp số cộng $\left(u_{n}\right)$ với $u_{1}=-2$ và công sai $d=3$ thì số hạng $u_{5}$ bằng
	\choice
	{$7$}
	{\True $10$}
	{$5$}
	{$6$}
	\loigiai{
		Áp dụng công thức số hạng thứ $n$ của cấp số cộng $\left(u_{n}\right)$ là $u_{n}=u_{1}+\left(n-1\right)\cdot d$.\\
		Khi đó số hạng $u_{5}=u_{1}+\left(5-1\right)\cdot d=-2+4\cdot3=10$. Vậy $u_{5}=10$.	
	}
\end{ex}
\begin{ex}%[1K2G6-6]
	Vào năm $2023$, nhiệt độ trung bình của thành phố $A$ là khoảng $29{,}5^\circ C$. Giả sử do biến đổi khí hậu nên mỗi năm nhiệt độ trung bình của thành phố $A$ đều tăng thêm khoảng $0{,}1^\circ C$. Hãy ước tính kể từ năm nào thì nhiệt độ trung bình của thành phố $A$ đạt từ $35^\circ C$ trở lên.
	\choice
	{$2076$}
	{$2077$}
	{\True $2078$}
	{$2079$}
	\loigiai{
		Theo bài toán, nhiệt độ trung bình ở mỗi năm của thành phố $A$ lập thành cấp số cộng với công sai là $d=0{,}1\left(^\circ C\right)$ và $u_{1}=29{,}5^\circ C$ là nhiệt độ trung bình của thành phố $A$ vào năm $2023$.\\
		Giả sử số hạng thứ $n$ của cấp số cộng có giá trị lớn hơn hoặc bằng $35$.\\
		Tức là, $u_{n}\geq35^\circ C$ hay $u_{1}+\left(n-1\right)\cdot d\geq35 \Leftrightarrow 29{,}5+\left(n-1\right)\cdot0{,}1\geq35 \Leftrightarrow n\geq56$.\\
		Do đó, kể từ số hạng thứ $56$ trở đi thì chúng đều có giá trị lớn hơn hoặc bằng $35$.\\
		Ta có $u_{1}$ là nhiệt độ trung bình của thành phố $A$ vào năm $2023$.\\
		Nên $u_{55}$ là nhiệt độ trung bình của thành phố $A$ vào năm $\left(2023+56-1\right)=2078$.\\
		Vậy kể từ năm $2078$ thì nhiệt độ trung bình của thành phố $A$ đạt từ $35^\circ C$ trở lên. 	
	}
\end{ex}
\begin{ex}%[1K2K7-1]
	Cho cấp số nhân $\left(u_{n}\right)$ có công bội dương và $u_{2}=\dfrac{1}{5}$, $u_{4}=5$. Tính công bội $q$.
	\choice
	{\True $5$}
	{$25$}
	{$\dfrac{1}{5}$}
	{$125$}
	\loigiai{
		Ta có $\heva{&u_{2}=u_{1}\cdot q=\dfrac{1}{5}\\&u_{4}=u_{1}\cdot q^3=5} \Rightarrow \dfrac{u_{4}}{u_{2}}=q^2=25 \Leftrightarrow q=\pm5$.\\
		Mà cấp số nhân $\left(u_{n}\right)$ có công bội dương nên $q=5$.
	}
\end{ex}
\begin{ex}%[1K2B7-4]
	Tìm $x$ để các số $2;8;x;128$ theo thứ tự đó lập thành một cấp số nhân.
	\choice
	{$16$}
	{$64$}
	{$34$}
	{\True $32$}
	\loigiai{
	Các số $2;8;x;128$ theo thứ tự đó lập thành một cấp số nhân khi $x=\sqrt{8\cdot 128}=32$.
	}
\end{ex}
\begin{ex}%[1K2B7-1]
	Cho cấp số nhân $\left(u_{n}\right)$, biết $u_{1}=1$, $u_{4}=64$. Tính công bội $q$ của cấp số nhân.
	\choice
	{$21$}
	{$\pm4$}
	{\True $4$}
	{$2\sqrt{2}$}
	\loigiai{
		Theo công thức tổng quát của cấp số nhân $u_{4}=u_{1}\cdot q^3 \Leftrightarrow 64=1\cdot q^3 \Leftrightarrow q=4$.
	}
\end{ex}
\begin{ex}%[1K2B7-3]
	Cho cấp số nhân $\left(u_{n}\right)$ với $u_{1}=-1$, $q=\dfrac{-1}{10}$. Số $\dfrac{1}{10^{103}}$ là số hạng thứ mấy của $\left(u_{n}\right)$?
	\choice
	{số hạng thứ $103$}
	{\True số hạng thứ $104$}
	{số hạng thứ $105$}
	{Không là số hạng của cấp số đã cho}
	\loigiai{
		Ta có $u_{n}=u_{1}\cdot q^{n-1} \Rightarrow \dfrac{1}{10^{103}}=-1\cdot \left(\dfrac{-1}{10}\right)^{n-1} \Rightarrow n-1=103 \Rightarrow n=104$.
	}
\end{ex}
\begin{ex}%[1K3Y9-5]
	Mỗi nhóm số liệu ghép nhóm là tập hợp gồm
	\choice
	{Các giá trị của số liệu được ghép nhóm theo nhiều tiêu chí xác định}
	{Các giá trị của số liệu được ghép nhóm theo hai tiêu chí xác định}
	{\True Các giá trị của số liệu được ghép nhóm theo một tiêu chí xác định}
	{Các giá trị của số liệu được ghép nhóm theo ba tiêu chí xác định}
	\loigiai{
		Theo định nghĩa số liệu ghép nhóm: Các giá trị của số liệu được ghép nhóm theo một tiêu chí xác định.
	}
\end{ex}
\begin{ex}%[1K3Y8-1]
	Mẫu số liệu sau cho biết phân bố theo độ tuổi của dân số Việt Nam năm $2019$\\
	\begin{center}
		\begin{tabular}{|c|c|c|c|}
		\hline
		Độ tuổi& Dưới $15$ & Từ $15$ đến $65$ & Từ $65$ trở lên\\
		\hline
		Số người& $23 371 882$ & $65 420 451$ & $7 416 651$\\
		\hline
	\end{tabular}
	\end{center}
	Số dân Việt Nam năm $2019$ là
	\choice
	{$73837102$}
	{$72837102$}
	{$95208984$}
	{\True $96208984$}
	\loigiai{
		Số dân Việt Nam năm $2019$ là $23371882+65420451+7416651= 96208984$.	
	}
\end{ex}
\begin{ex}%[1K3Y9-4]
	Khảo sát thời gian tập thể dục của một số học sinh khối $11$ thu được mẫu số liệu ghép nhóm sau
	\begin{center}
		\begin{tabular}{|c|c|c|c|c|c|}
			\hline
			Thời gian $\left(\text{phút}\right)$& $\left[0;20\right)$ & $\left[20;40\right)$ & $\left[40;60\right)$& $\left[60;80\right)$ & $\left[80;100\right)$\\
			\hline
			Số học sinh & $5$ & $9$ & $12$ & $10$ & $6$\\
			\hline
		\end{tabular}
	\end{center}
	Nhóm chứa mốt của mẫu số liệu trên là
	\choice
	{$\left[20;40\right)$}
	{$\left[60;80\right)$}
	{\True $\left[40;60\right)$}
	{$\left[80;100\right)$}
	\loigiai{
		Mốt $M_0$ chứa trong nhóm $\left[40;60\right)$.		
	}
\end{ex}
\begin{ex}%[1K3K9-3]
	Khảo sát thời gian tập thể dục của một số học sinh khối $11$ thu được mẫu số liệu ghép nhóm sau
	\begin{center}
		\begin{tabular}{|c|c|c|c|c|c|}
			\hline
			Thời gian $\left(\text{phút}\right)$& $\left[0;20\right)$ & $\left[20;40\right)$ & $\left[40;60\right)$& $\left[60;80\right)$ & $\left[80;100\right)$\\
			\hline
			Số học sinh & $5$ & $9$ & $12$ & $10$ & $6$\\
			\hline
		\end{tabular}
	\end{center}
	Nhóm chứa tứ phân vị thứ ba của mẫu số liệu trên là
	\choice
	{$\left[20;40\right)$}
	{\True $\left[60;80\right)$}
	{$\left[40;60\right)$}
	{$\left[80;100\right)$}
	\loigiai{
		Ta có $n=42$ nên tứ phân vị thứ ba của mẫu số liệu trên là $Q_3=x_{33}$.\\
		Mà $x_{33}\in\left[60;80\right)$.\\
		Vậy nhóm chứa tứ phân vị thứ ba của mẫu số liệu trên là nhóm $\left[60;80\right)$.		
	}
\end{ex}
\begin{ex}%[1K3B9-1]
	Khi thống kê chiều cao của $40$ bạn lớp $11A$, ta thu được mẫu số liệu ghép nhóm được cho ở bảng sau (đơn vị: centimét).
	\begin{center}
		\begin{tabular}{|c|c|}
		\hline
		Nhóm& Tần số\\
		\hline
		$\left[155;160\right)$&$5$\\
		\hline
		$\left[160;165\right)$&$12$\\
		\hline
		$\left[165;170\right)$& $16$\\
		\hline
		$\left[170;175\right)$&$7$\\
		\hline
		& $n=40$\\
		\hline
	\end{tabular}
	\end{center}
	Số trung bình cộng bằng
	\choice
	{\True $165{,}6$}
	{$156{,}6$}
	{$155{,}6$}
	{$156{,}5$}
	\loigiai{
		Số trung bình cộng là $\bar{x}=\dfrac{5\cdot157{,}5+12\cdot162{,}5+16\cdot167{,}5+7\cdot172{,}5}{40}\approx165{,}6$.		
	}
\end{ex}
\begin{ex}%[1K3K9-3]
	Cho mẫu số liệu ghép nhóm thống kê thời gian sử dụng điện thoại trước khi ngủ (đơn vị: phút) của một người trong $120$ ngày như ở bảng sau. Xác định các số đặc trưng đo xu thế trung tâm cho mẫu số liệu đó (làm tròn các kết quả đến hàng phần mười).
	\begin{center}
		\begin{tabular}{|c|c|}
			\hline
			Nhóm & Tần số\\
			\hline
			$\left[0;4\right)$& $13$\\
			\hline
			$\left[4;8\right)$& $29$\\
			\hline
			$\left[8;12\right)$& $48$\\
			\hline
			$\left[12;16\right)$& $22$\\
			\hline
			$\left[16;20\right)$& $8$\\
			\hline
			& $n=120$\\
			\hline
		\end{tabular}
	\end{center}
	Giá trị các tứ phân vị thứ nhất, thứ hai và thứ ba lần lượt là
	\choice
	{$9{,}5;12;6{,}3$}
	{\True $6{,}3;9{,}5;12$}
	{$9{,}5;6{,}3;12$}
	{$12;6{,}3;9{,}5$}
	\loigiai{
		Bảng tần số ghép nhóm bao gồm cả tần số tích luỹ được cho như ở bảng
	\begin{center}
			\begin{tabular}{|c|c|c|}
			\hline
			Nhóm & Tần số & Tần số tích lũy\\
			\hline
			$\left[0;4\right)$& $13$ & $13$\\
			\hline
			$\left[4;8\right)$& $29$ & $42$\\
			\hline
			$\left[8;12\right)$& $48$ & $90$\\
			\hline
			$\left[12;16\right)$& $22$ & $112$\\
			\hline
			$\left[16;20\right)$& $8$ & $120$\\
			\hline
			& $n=120$ &\\
			\hline
		\end{tabular}
	\end{center}
		Ta có $\dfrac{n}{2}=60$, $\dfrac{n}{4}=30$, $\dfrac{3n}{4}=90$.\\
		Vì $42<60<90$ nên nhóm $3$ là nhóm đầu tiên có tần số tích luỹ lớn hơn hoặc bằng $60$.\\
		Suy ra trung vị là $M_e=8+\left(\dfrac{60-42}{48}\right)\cdot 4=9{,}5$.\\
		Tứ phân vị thứ hai là $Q_2=M_e=9{,}5$.\\
		Do $13<30<42$ nên nhóm $2$ là nhóm đầu tiên có tần số tích luỹ lớn hơn hoặc bằng $30$. Suy ra tứ phân vị thứ nhất là $Q_1=4+\left(\dfrac{30-13}{29}\right)\cdot 4\approx6{,}3$.\\
		Do $42<90\leq 90$ nên nhóm $3$ là nhóm đầu tiên có tần số tích luỹ lớn hơn hoặc bằng $90$. Suy ra tứ phân vị thứ ba là $Q_3=8+\left(\dfrac{90-42}{48}\right)\cdot 4=12$.		
	}
\end{ex}


\Closesolutionfile{ans}
% \inputans{10}{ans/ans-1-GK1-KNTT-De15-NH23-24}
\noindent{\bf\fontfamily{qag}\selectfont\color{violet}B. PHẦN TỰ LUẬN}
\setcounter{bt}{0}

%%==========Bài 1
\begin{bt}%[1K1K3-1]
	\begin{enumEX}{1}
		\item Tìm tập xác định của hàm số $y=\dfrac{\sqrt{1+\cos{2x}}}{1-\left(\sin x-\cos x\right)^2}$.
		\item Cho góc $\alpha \in (-\pi ; -\dfrac{\pi}{2})$ và $\tan \alpha = 3$. Tìm các GTLG của $\alpha$.
	\end{enumEX}
	\loigiai{
	Hàm số xác định khi và chỉ khi $\heva{&1+\cos{2x}\geq0\\&1-\left(\sin x-\cos x\right)^2\neq0} \Leftrightarrow \heva{&\cos{2x}\geq-1\\&\sin{2x}\neq0}$.\\
	$\cos{2x}\geq-1$ thỏa mãn $\forall x\in\mathbb{R}$.\\
	$\sin{2x}\neq0 \Leftrightarrow 2x\neq k\pi,k\in\mathbb{Z} \Leftrightarrow x\neq k\dfrac{\pi}{2},k\in\mathbb{Z}$.\\
	Vậy tập xác định của hàm số là $\mathscr{D}=\mathbb{R}\setminus\left\{k\dfrac{\pi}{2},k\in\mathbb{Z}\right\}$.
	}
\end{bt}

\begin{bt}%[1K2K6-5]
	Cho cấp số cộng $\left(u_n\right)$ có $u_5=-15$, $u_{20}=60$. Tính tổng $10$ số hạng đầu tiên của cấp số cộng đó.
	\loigiai{
		Gọi $u_1$, $d$ lần lượt là số hạng đầu và công sai của cấp số cộng.\\
		Ta có $\heva{&u_5=-15\\&u_{20}=60} \Leftrightarrow \heva{&u_1+4d=-15\\&u_1+19d=60} \Leftrightarrow \heva{&u_1=35\\&d=5}$.\\
		Vậy $S_{10}=\dfrac{10}{2}\cdot\left(2u_1+9d\right)=5\cdot\left[2\cdot\left(-35\right)+9\cdot5\right]=-125$.
	}
\end{bt}
\begin{bt}
	Số giờ có ánh sáng mặt trời của một thành phố A ở vĩ độ $40^\circ$ bắc trong ngày thứ $t$ của một năm không nhuận được cho bởi hàm số $d\left( t \right) = 3\sin \left[ {\dfrac{\pi }{{182}}\left( {t - 80} \right)} \right] + 12$  với $t \in \mathbb{Z}$ và $0<t \le 365$. Hãy cho biết ngày tháng nào có nhiều giờ có ánh sáng mặt trời nhất và ngày tháng nào có ít giờ có ánh sáng mặt trời nhất trong năm (không nhuận)?
\end{bt}
\begin{bt}%[1K2G7-6]
	Tìm $4$ số hạng đầu của một cấp số nhân biết tổng $3$ số hạng đầu bằng $\dfrac{148}{9}$, đồng thời theo thứ tự chúng là số hạng thứ $1$, thứ $4$, thứ $8$ của một cấp số cộng có công sai khác $0$.
	\loigiai{
		Gọi $4$ số hạng đầu của cấp số nhân đã cho là $u_1$, $u_2$, $u_3$, $u_4$; công bội của cấp số nhân là $ q $, công sai của cấp số cộng là $ d $ $\left(d\neq0\right)$.\\
		Tổng $3$ số hạng đầu của cấp số nhân bằng $\dfrac{148}{9}$ nên $u_1+u_2+u_3=\dfrac{148}{9} \Leftrightarrow u_1+u_1\cdot q+u_1\cdot q^2=\dfrac{148}{9}\left(1\right)$.\\
		Do $u_1$, $u_2$, $u_3$ theo thứ tự chúng là số hạng thứ $1$, thứ $4$, thứ $8$ của một cấp số cộng có công sai $d\neq0$ nên\\
		$\heva{&u_1\cdot q=u_1+3d\left(2\right)\\&u_1\cdot q^2=u_1+7d\left(3\right)}$.\\
		Nhân phương trình $\left(2\right)$ với $ 7 $ và nhân phương trình $\left(3\right)$ với $ 3 $, sau đó trừ hai phương trình theo vế ta được $u_1\left(3q^2-7q+4\right)=0\left(4\right)$.\\
		Từ phương trình $\left(1\right)$ ta có $u_1\neq0$. Khi đó $\left(3\right)\Leftrightarrow 3q^2-7q+4=0 \Leftrightarrow \hoac{&q=1\\&q=\dfrac{4}{3}}$.
		\begin{itemize}
			\item[+)] Với $q=1$, thay vào $\left(1\right)$ suy ra $u_1=u_2=u_3=\dfrac{148}{27}$ (loại do $u_1$, $_2$, $u_3$ theo thứ tự chúng là số
			hạng thứ $1$, thứ $4$, thứ $8$ của một cấp số cộng có công sai $d\neq0$).
			\item[+)] Với $q=\dfrac{4}{3}$, thay vào $\left(1\right)$ suy ra $u_1=4$, $u_2=\dfrac{16}{3}$, $u_3=\dfrac{64}{9}$, $u_4=\dfrac{256}{27}$.
		\end{itemize}
			Vậy $4$ số hạng đầu của cấp số nhân là $u_1=4$, $u_2=\dfrac{16}{3}$, $u_3=\dfrac{64}{9}$, $u_4=\dfrac{256}{27}$.
	}
\end{bt}
\begin{bt}%[Dự án TLDH2-Nhóm Latex, Kiều Ngân]%[2D2B5-6]%Câu 7.
	Một người mỗi tháng đều đặn gửi vào ngân hàng một khoản tiền $T$ theo hình thức lãi kép với lãi suất $0{,}6\%$ mỗi tháng. Biết sau $15$ tháng, người đó có số tiền là $100$ triệu đồng. Hỏi số tiền $T$ gần với số tiền nào nhất trong các số sau?
	\loigiai{
		Với số tiền $T$ gửi đều đặn mỗi tháng theo hình thức lãi kép với lãi suất $r\%$ mỗi tháng, ta có\\
		Sau một tháng, số tiền của người đó là $A_1=T(1+r)$ đồng.\\
		Sau hai tháng, số tiền của người đó là $A_2=[T(1+r)+T](1+r)=T\left[(1+r)^2+(1+r)\right]$ đồng.\\
		Sau ba tháng, số tiền của người đó là
		$$A_3=\left\{T\left[(1+r)^2+(1+r)\right]+T\right\}(1+r)=T\left[(1+r)^3+(1+r)^2+(1+r)\right]\text{ đồng}.$$
		\ldots \\
		Sau mười lăm tháng, số tiền của người đó là
		$$A_{15}=T\left[(1+r)^{15}+(1+r)^{14}+\cdots +(1+r)\right]=\dfrac{T}{r}(1+r)\left[(1+r)^{15}-1\right]\text{ đồng}.$$
		Khi đó $T=\dfrac{A_{15}\cdot r}{(1+r)\left[(1+r)^{15}-1\right]}=\dfrac{10^8\cdot 0{,}006}{1{,}006\cdot (1{,}006^{15}-1)}\approx 6.350.000$ đồng.}
\end{bt}

\Closesolutionfile{ans}
% \inputans{10}{ans/ans-0-GK1-CanhDieu-De1-NH23-24}