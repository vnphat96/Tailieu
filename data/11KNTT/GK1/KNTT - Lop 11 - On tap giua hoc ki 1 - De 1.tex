\begin{name}
	{\tenchude}
	{TOÁN 11}
	{LỚP TOÁN THẦY PHÁT}
	{Thời gian: 90 phút - Không kể thời gian phát đề}
\end{name}

\caulc

\Opensolutionfile{ans}[ans-1-KNTT-GK1-DE1-NLC]
%%%%==============Cau_EX1==============%%%
\begin{ex}%[1D1H1-2]%[KNTT - Lớp 11 - Ôn tập giữa học kì 1 - Đề 1]%[Phạm Văn Long]
Góc có số đo $132^{\circ}$ đổi sang rađian là
\choice
	{\True $\dfrac{11\pi}{15}$}
	{$\dfrac{11}{15}$}
	{$\dfrac{15\pi}{11}$}
	{$\dfrac{15}{11}$}
\loigiai{
		Áp dụng công thức $\alpha^{\circ}=\dfrac{\alpha \pi}{180}$ rad ta được $132^{\circ}=\dfrac{11\pi}{15}$.
		}
\end{ex}
%%%==============HetCau_EX1==============%%%

%%%==============Cau_EX2==============%%%
\begin{ex}%[1D1N1-4]%[KNTT - Lớp 11 - Ôn tập giữa học kì 1 - Đề 1]%[Phạm Văn Long]
	Một đường tròn có đường kính 40 cm. Cung tròn trên đường tròn đó có số đo $1{,}5$. Tính độ dài của cung tròn đó.
	\choice
	{\True $30$ cm}
	{$30$}
	{$60$ cm}
	{$60$}
		\loigiai{
			Áp dụng công thức $l=R\cdot \alpha=20\cdot 1{,}5=30\mathrm{~cm}$.
		}
	\end{ex}
	%%%==============HetCau_EX2==============%%%
	
	%%%==============Cau_EX3==============%%%
	\begin{ex}%[1D1H2-2]%[KNTT - Lớp 11 - Ôn tập giữa học kì 1 - Đề 1]%[Phạm Văn Long]
		Biết $\sin \left(\dfrac{7\pi}{2}+\alpha\right)=\dfrac{3}{7}$. Khi đó giá trị của $\cos \alpha$ bằng
		\choice
		{$\dfrac{3}{7}$}
		{\True $\dfrac{-3}{7}$}
		{$\dfrac{4}{7}$}
		{$\dfrac{2\sqrt{10}}{7}$}
		\loigiai{
			Ta có $\sin \left(\dfrac{7\pi}{2}+\alpha\right)=\sin \left(4\pi-\dfrac{\pi}{2}+\alpha\right)=\sin \left(-\dfrac{\pi}{2}+\alpha\right)=-\cos \alpha$.\\
			Suy ra $\cos \alpha=\dfrac{-3}{7}$.
		}
	\end{ex}
	%%%==============HetCau_EX3==============%%%
	
	%%%==============Cau_EX4==============%%%
	\begin{ex}%[1D1N2-2]%[KNTT - Lớp 11 - Ôn tập giữa học kì 1 - Đề 1]%[Phạm Văn Long]
		Cho $\pi < \alpha < \dfrac{3\pi}{2}$. Khẳng định nào sau đây \textbf{sai}?
		\choice
		{$\sin \alpha < 0$}
		{$\cos \alpha < 0$}
		{$\tan \alpha > 0$}
		{\True $\cot \alpha < 0$}
			\loigiai{
		Vì $\pi < \alpha < \dfrac{3\pi}{2}$ nên $\cot \alpha > 0$.
			}
	\end{ex}
	%%%==============HetCau_EX4==============%%%
	
	%%%==============Cau_EX5==============%%%
	\begin{ex}%[1D1N2-2]%[KNTT - Lớp 11 - Ôn tập giữa học kì 1 - Đề 1]%[Phạm Văn Long]
	Khẳng định nào sau đây sai?
	\choice
		{$\sin \dfrac{\pi}{4}=\dfrac{\sqrt{2}}{2}$}
		{$\cos \dfrac{5\pi}{6}=-\dfrac{\sqrt{3}}{2}$}
		{$\tan 135^{\circ}=-1$}
		{\True $\cot 120^{\circ}=\dfrac{1}{\sqrt{3}}$}
		\loigiai{
			Ta có $\cot 120^{\circ}=-\dfrac{1}{\sqrt{3}}$.	}
			\end{ex}
		%%==============HetCau_EX5==============%%%
			
		%%%==============Cau_EX6==============%%%
	\begin{ex}%[1D1H2-2]%[KNTT - Lớp 11 - Ôn tập giữa học kì 1 - Đề 1]%[Phạm Văn Long]
	Cho $\pi < \alpha < \dfrac{3\pi}{2}$ và $\sin \alpha=-\dfrac{1}{3}$. Khẳng định nào sau đây đúng?
		\choice
	{$\cos \alpha=-\dfrac{2}{3}$}
	{$\cos \alpha=\dfrac{4}{3}$}
	{$\cos \alpha=\dfrac{2\sqrt{2}}{3}$}
	{\True $\cos \alpha=-\dfrac{2\sqrt{2}}{3}$}
	\loigiai{
	Vì $\pi < \alpha < \dfrac{3\pi}{2}$ nên $\cos \alpha < 0\Rightarrow \cos \alpha=-\sqrt{1-\sin ^2\alpha}=-\sqrt{1-\dfrac{1}{9}}=-\dfrac{2\sqrt{2}}{3}$.
		}
	\end{ex}
	%%%==============HetCau_EX6==============%%%
			
	%%%==============Cau_EX7==============%%%
	\begin{ex}%[1D1H3-3]%[KNTT - Lớp 11 - Ôn tập giữa học kì 1 - Đề 1]%[Phạm Văn Long]
	Biến đổi biểu thức $\sin ^2\left(\dfrac{\pi}{4}-\dfrac{x}{2}\right)$ bằng
	\choice
	{\True $\dfrac{1-\sin x}{2}$}
	{$1-\sin x$}
	{$1-\cos x$}
	{$\dfrac{1-\cos x}{2}$}
	\loigiai{
	Ta có $\sin ^2\left(\dfrac{\pi}{4}-\dfrac{x}{2}\right)=\dfrac{1-\cos \left(2\left(\dfrac{\pi}{4}-\dfrac{x}{2}\right)\right)}{2}=\dfrac{1-\cos \left(\dfrac{\pi}{2}-x\right)}{2}=\dfrac{1-\sin x}{2}$.
	}
	\end{ex}
	%%%==============HetCau_EX7==============%%%
	%%%==============Cau_EX8==============%%%
	\begin{ex}%[1D1H4-6]%[KNTT - Lớp 11 - Ôn tập giữa học kì 1 - Đề 1]%[Phạm Văn Long]
	Giá trị lớn nhất của hàm số $y=2\sin x+3$ là
	\choice
	{$-1$}
	{$3$}
	{$4$}
	{\True $5$}
	\loigiai{
	Tập xác định $\mathscr{D}=\mathbb{R}$.\\
	Ta có $-1\leq \sin x \leq 1\Leftrightarrow-2\leq 2\sin x \leq 2\Leftrightarrow 1\leq 2\sin x+3\leq 5$ nên $1\leq y \leq 5, \forall x \in \mathbb{R}$.\\
	Vậy giá trị lớn nhất của hàm số bằng 5 khi $\sin x=1$.
			}
	\end{ex}
	%%%==============HetCau_EX8==============%%%
			
	%%%==============Cau_EX9==============%%%
	\begin{ex}%[1D1H4-2]%[KNTT - Lớp 11 - Ôn tập giữa học kì 1 - Đề 1]%[Phạm Văn Long]
	Số giá trị nguyên dương của $m \leq 10$ để hàm số $y=\sqrt{\sin x-\cos x+m}$ có tập xác định $\mathbb{R}$ là
	\choice
	{$1$}
	{$10$}
	{\True $9$}
	{$8$}
	\loigiai{
	Hàm số đã cho có tập xác định $\mathbb{R}$ khi và chỉ khi
	\allowdisplaybreaks
	\begin{eqnarray*}
	\sin x-\cos x+m \geq 0,\, \forall x\in \mathbb{R} &\Leftrightarrow& \sqrt{2} \sin \left(x-\dfrac{\pi}{4}\right) \geq-m,\, \forall x \in \mathbb{R}\\
	&\Leftrightarrow&-m \leq \min \left\{\sqrt{2} \sin \left(x-\dfrac{\pi}{4}\right)\right\}=-\sqrt{2} \\
	& \Leftrightarrow& m \geq \sqrt{2}\\
	&\Rightarrow& \sqrt{2} \leq m \leq 10.
	\end{eqnarray*}
	Mà $m \in \mathbb{Z} \Rightarrow m \in\{2; 3; 4; 5; \ldots.; 10\}$.\\	
	Vây có $9$ số nguyên $m$ để hàm số đã cho có tập xác định $\mathbb{R}$.
				}
			\end{ex}
	%%%==============HetCau_EX9==============%%%
			
	%%%==============Cau_EX10==============%%%
	\begin{ex}%[1D2H2-2]%[KNTT - Lớp 11 - Ôn tập giữa học kì 1 - Đề 1]%[Phạm Văn Long]
	Trong bốn dãy số sau, có bao nhiêu dãy số lập thành một cấp số cộng?
	\begin{itemize}
		\item \textbf{I)} $10,-2,-14,-26,-38$.
		\item \textbf{II)} $\dfrac{1}{2}, \dfrac{5}{4}, 2, \dfrac{11}{4}, \dfrac{7}{2}$.
		\item \textbf{III)} $\sqrt{1}, \sqrt{2}, \sqrt{3}, \sqrt{4}, \sqrt{5}$.
		\item \textbf{IV)} $1,4,7,10,13$.
	\end{itemize}
	\choice
	{$1$}
	{$2$}
	{\True $3$}
	{$4$}
	\loigiai{
		Dãy số cho ở các ý I); II); IV) là một cấp số cộng vì kể từ số hạng thứ hai, mỗi số hạng đều bằng tổng của số hạng đứng ngay trước nó với một số không đổi lần lượt là $-12; \dfrac{3}{4}; 3$.\\
		Dãy số thứ III không phải là một cấp số cộng vì $\sqrt{3}-\sqrt{2} \neq \sqrt{2}-1$.
			}
		\end{ex}
	%%%==============HetCau_EX10==============%%%
	%%%==============Cau_EX11==============%%%
	\begin{ex}%[1D1H5-3]%[KNTT - Lớp 11 - Ôn tập giữa học kì 1 - Đề 1]%[Phạm Văn Long]
	Phương trình $\sin x=\sin \dfrac{\pi}{8}$ có các họ nghiệm là
	\choice
	{$\hoac{&x=\dfrac{\pi}{8}+k 2\pi\\&x=\dfrac{5\pi}{8}+k 2\pi}, k \in \mathbb{Z}$}
	{\True $\hoac{&x=\dfrac{\pi}{8}+k 2\pi\\&x=\dfrac{7\pi}{8}+k 2\pi}, k \in \mathbb{Z}$}
	{$\hoac{&x=\dfrac{5\pi}{8}+k 2\pi\\&x=\dfrac{7\pi}{8}+k 2\pi}, k \in \mathbb{Z}$}
	{$\hoac{&x=-\dfrac{\pi}{8}+k 2\pi \\&x=-\dfrac{7\pi}{8}+k 2\pi}, k \in \mathbb{Z}$}
	\loigiai{
	Áp dụng công thức nghiệm của phương trình $\sin x=\sin \alpha \Leftrightarrow \hoac{&x=\alpha+k 2\pi \\&x=\pi-\alpha+k 2\pi}, k \in \mathbb{Z}$.\\
	Ta có $\sin x=\sin \dfrac{\pi}{8} \Leftrightarrow\hoac{&x=\dfrac{\pi}{8}+k 2\pi\\&x=\dfrac{7\pi}{8}+k 2\pi}, k \in \mathbb{Z}$.}
	\end{ex}
	%%%==============HetCau_EX11==============%%%
	%%%==============Cau_EX12==============%%%
	\begin{ex}%[1D1V4-8]%[KNTT - Lớp 11 - Ôn tập giữa học kì 1 - Đề 1]%[Phạm Văn Long]
	Người ta xác định được số giờ có ánh sáng mặt trời của tỉnh Bà Rịa Vũng Tàu trong ngày thứ $t$ của một năm không nhuận, được cho bởi một hàm số $\mathrm{d}(t)=4\sin \left[\dfrac{\pi}{182}(t-80)\right]+11$ với $t \in \mathbb{Z}$ và $0< t \leq 365$. Ngày nào trong năm thì tỉnh Bà Rịa Vũng Tàu có số giờ có ánh sáng mặt trời là lớn nhất?
	\choice
	{$68$}
	{$235$}
	{\True $171$}
	{$168$}
	\loigiai{
	Ta có $\sin \left[\dfrac{\pi}{182}(t-80)\right] \leq 1$ suy ra $\mathrm{d}(t)=4\sin \left[\dfrac{\pi}{182}(t-80)\right]+11\leq 4.1+11=15$.\\
	Dấu bằng xảy ra khi và chỉ khi 
	$$\sin \left[\dfrac{\pi}{182}(t-80)\right]=1\Leftrightarrow \dfrac{\pi}{182}(t-80)=\dfrac{\pi}{2}+k 2\pi \Leftrightarrow t=171+364k, k \in \mathbb{Z}.$$
	Mà $0< t \leq 365\Leftrightarrow 0< 171+364k \leq 365\Leftrightarrow \dfrac{-171}{364} < k \leq \dfrac{194}{364}, k \in Z$, suy ra $k=0\Rightarrow t=171$.\\
	Vậy ngày thứ 171 thì thành phố có số giờ có ánh sáng mặt trời là nhiều nhất.}
	\end{ex}
						%%%==============HetCau_EX12==============%%%
\Closesolutionfile{ans}

% \indapan{6}{ans-1-KNTT-GK1-DE1-NLC}

\cauds

\Opensolutionfile{ans}[ans-1-KNTT-GK1-DE1-DS]

\begin{ex}%[1D5V2-3]%[KNTT - Lớp 11 - Ôn tập giữa học kì 1 - Đề 1]%[Phạm Văn Long]
	Bảng số liệu ghép nhóm sau cho biết chiều cao học sinh lớp 11A
	\begin{center}
		\begin{tabular}{|c|c|c|c|c|c|}
			\hline
			\begin{tabular}{c}
					Khoảng chiều cao \\
					$(\mathrm{~cm})$
				\end{tabular} & {$[145; 150)$} & {$[150; 155)$} & {$[155; 160)$} & {$[160; 165)$} & {$[165; 170)$} \\
			\hline
			Số học sinh & 7 & 14 & 10 & 10 & 9 \\
			\hline
		\end{tabular}
	\end{center}
	\choiceTF
		{\True Lớp $11A$ có 50 học sinh}
		{Giá trị đại diện của nhóm $[155; 160)$ là 155}
		{\True Bạn Tú tính giá trị trung bình của bảng số liệu ghép nhóm là $157{,}5$}
		{Tứ phân vị của bảng số liệu ghép nhóm: $Q_1=152; Q_2=157; Q_3=163$}
		\loigiai{
		\begin{itemchoice}
		\itemch Đúng.\\
		Số học sinh lớp $11A$ là 50 học sinh.
		\itemch Sai.\\
		Giá trị đại diện của nhóm $[155; 160)$ là $\dfrac{155+160}{2}=157{,}5$.
		\itemch Đúng.\\
		Giá trị đại diện của các nhóm $[145; 150);[150; 155);[155; 160);[160; 165);[165; 170)$ lần lượt là $147{,}5;\, 152{,}5;\, 157{,}5;\, 162{,}5;\, 167{,}5$. Nên giá trị trung bình.\\
		$$\bar{x}=\dfrac{7.147{,}5+14.152{,}5+10.157{,}5+10.162{,}5+9.167{,}5}{7+14+10+10+9}=157{,}5.$$
		\itemch Sai.\\
		Gọi $x_1; x_2; \ldots; x_{50}$ là chiều cao của 50 học sinh và giả sử dãy này đã được sắp xếp theo thứ tự tăng dần. Khi đó, trung vị là $\dfrac{x_{25}+x_{26}}{2}$. Do hai giá trị $x_{25}; x_{26}$ thuộc nhóm $[155; 160)$ nên nhóm này chứa trung vị.\\
		 Do đó, $p=3; a_3=155; m_3=10; m_1+m_2=7+14=21; a_4-a_3=5$ và ta có\\
		$Q_2=M_e=155+\dfrac{\dfrac{50}{2}-21}{10}\cdot 5=157$.\\
		Tứ phân vị thứ nhất $Q_1$ là $x_{13}$. Do $x_{13}$ thuộc nhóm $[150; 155)$ nên nhóm này chứa $Q_1$.\\
		Do đó, $p=2; a_2=150; m_2=14; m_1=7; a_3-a_2=5$ và ta có 
		$$Q_1=150+\dfrac{\dfrac{50}{4}-7}{14}\cdot 5\approx 151{,}96.$$
		Tứ phân vị thứ ba $Q_3$ là $x_{38}$. Do $x_{38}$ thuộc nhóm $[160; 165)$ nên nhóm này chứa $Q_3$.\\
		Do đó, $p=4; a_4=160; m_4=10; m_1+m_2+m_3=7+14+10=31; a_3-a_2=5$ và ta có
		$$Q_3=160+\dfrac{\dfrac{50\cdot 3}{4}-31}{10}\cdot 5=163{,}25.$$
		\end{itemchoice}}
		\end{ex}
		
	\begin{ex}%[1D2V2-7]%[KNTT - Lớp 11 - Ôn tập giữa học kì 1 - Đề 1]%[Phạm Văn Long]
	Do nhu cầu đi lại của gia đình, anh Bình quyết định thực hiện tích góp tiền để mua một chiếc ôtô \textbf{HONDA CRV} trị giá $1{,}259$ tỉ đồng.
	\begin{itemize}
		\item Đợt thứ nhất: anh Bình đã tích góp theo nguyên tắc tháng sau tích góp nhiều hơn tháng ngay trước đó số tiền là $2$ triệu đồng và cứ như thế đến tháng thứ $10$ anh phải góp $21$ triệu đồng. Đến hết đợt thứ nhất anh Bình có tất cả $624$ triệu đồng.				
		\item Đợt thứ hai kế tiếp: do muốn rút ngắn thời gian mua xe thì số tiền còn lại anh tiếp tục tích góp với tháng đầu là $5$ triệu đồng và mỗi tháng tiếp theo số tiền gấp đôi tháng kề trước nó. Xét tính đúng sai của các khẳng định sau:
	\end{itemize}
	\choiceTF
	{\True Đợt thứ nhất anh Bình tích lũy tiền theo dãy số với cấp số cộng có công sai là $d=2$ triệu và $u_1=3$ triệu}
	{\True Đợt thứ hai anh Bình tích lũy tiền theo dãy số với cấp số nhân có công bội là $q=2$ triệu và $u_1=5$ triệu}
	{Anh Bình tích lũy tiền hết đợt thứ nhất trong $25$ tháng}
	{\True Để đủ tiền mua ôtô thì anh Bình thì anh Bình tích góp ít nhất $31$ tháng}
	\loigiai{
	\begin{itemchoice}
	\itemch Đúng.\\
	Đợt thứ nhất anh Bình tích lũy theo cấp số cộng với công sai $d=2$ triệu.\\
	Theo đề bài ta có
	$$u_{10}=21 \Leftrightarrow u_1+9 d=21 \Leftrightarrow u_1=3 \text{\,triệu}$$
	\itemch Đúng.\\
	Đợt thứ hai anh Bình tích lũy theo cấp số nhân với $u_1=5$ và mỗi tháng tiếp theo số tiền gấp đôi tháng kề trước nó nên công bội $q=2$ triệu.
	\itemch Sai.\\
	Vì hết đợt thứ nhất anh Bình có tất cả 624 triệu đồng nên $S_n=624$ nên ta có
	$$\dfrac{n\left[2 u_1+(n-1) d\right]}{2}=624 \Leftrightarrow n[2 \cdot 3+(n-1) \cdot 2]=1248 \Leftrightarrow 2 n^2+4 n^2-1248=0$$
	Suy ra $n=24$. Vậy Anh Bình tích lũy tiền hết đợt thứ nhất trong $24$ tháng.
	\itemch Đúng.\\
	Số tiền tích lũy đợt $1$ là $624$ nên đợt anh Bình cần tích lũy $1259-624=635$ triệu đồng.\\
	Ta có $S_n=635\Leftrightarrow \dfrac{u_1\left(1-q^n\right)}{1-q}=635\Leftrightarrow n=7$ tháng.\\
	Vậy tổng cộng hai đợt cần có ít nhất $31$ tháng.
	\end{itemchoice}}
	\end{ex}

\Closesolutionfile{ans}

% \indapan{3}{ans-1-KNTT-GK1-DE1-DS}

\caukq

\Opensolutionfile{ans}[ans-1-KNTT-GK1-DE1-KQ]
\begin{ex}%[1D5H2-2]
Một mẫu số liệu có bảng tần số ghép nhóm như sau
\begin{center}
\begin{tabular}{|c|c|c|c|c|c|}
\hline
Nhóm & $[1;5)$ & $[5;9)$ & $[9;13)$ & $[13;17)$ & $[17;21)$ \\
\hline
Tần số & $4$ & $8$ & $13$ & $6$ & $4$ \\
\hline
\end{tabular}
\end{center}
Trung vị của mẫu số liệu ghép nhóm trên bằng bao nhiêu (kết quả làm tròn đến hàng phần chục)?
\shortans{$10{,}7$}
\loigiai{
Cỡ mẫu của mẫu số liệu là $n=4+8+13+6+5=35$.\\
Gọi $x_1, x_2, \ldots, x_{35}$ là mẫu số liệu được sắp xếp theo thứ tự không giảm.\\
Trung vị của mẫu số liệu này là $x_{18} \in [9;13)$.\\
Ta có $n_m=13$; $C=4+8=12$; $u_m=9$; $u_{m+1}=13$.\\
Trung vị của mẫu số liệu ghép nhóm là
\[
M_e=9+\dfrac{\dfrac{35}{2}-12}{13} (13-9)=\dfrac{139}{13} \approx 10{,}7.
\]
}
\end{ex}
\begin{ex}%[1D2V2-4]%[KNTT - Lớp 11 - Ôn tập giữa học kì 1 - Đề 1]%[Phạm Văn Long]
	Cho bốn số $a$, $b$, $c$, $d$ theo thứ tự lập thành cấp số cộng có công sai dương. Biết rằng tổng của bốn số hạng bằng $13$ và tổng của ba số đầu bằng $\dfrac{15}{2}$. Tính tổng ba số cuối.
	\shortans{$12$}
	\loigiai{
		Gọi cấp số cộng có công sai là $x>0$.
		Ta có \allowdisplaybreaks
		\begin{eqnarray*}
			\heva{&a+b+c+d=13\\&a+b+c=\dfrac{15}{2}} &\Leftrightarrow& \heva{&a+(a+x)+(a+2x)+(a+3x)=13\\&a+(a+x)+(a+2x)=\dfrac{15}{2}}\\
			&\Leftrightarrow&\heva{&4a+6x=13\\&a+x=\dfrac{5}{2}}\\
			&\Leftrightarrow&\heva{&a=1\\&x=\dfrac{3}{2}.}
		\end{eqnarray*}
		Do đó $b+c+d=13-a=13-1=12$.
	}
\end{ex}
%%%==============HetBai_BT3==============%%%

%%%==============Bai_BT4==============%%%
\begin{ex}%[1D1V5-3]%[KNTT - Lớp 11 - Ôn tập giữa học kì 1 - Đề 1]%[Phạm Văn Long]
	Cho phương trình lượng giác $\sin x-1=0$. Tổng tất cả các nghiệm của phương trình lượng giác trên $[0; 10\pi]$ có dạng $\dfrac{a \pi}{b}$ với $a$, $b \in \mathbb{N}$, $b > 0$ và $\dfrac{a}{b}$ tối giản. Tích $a b$ bằng
	\shortans{$90$}
	\loigiai{
		Phương trình lượng giác $\sin x-1=0\Leftrightarrow \sin x=1\Leftrightarrow x=\dfrac{\pi}{2}+k 2\pi,\,(k \in \mathbb{Z})$.\\
		Với $\heva{&x=\dfrac{\pi}{2}+k 2\pi \\&x \in[0; 10\pi]} \Rightarrow 0\leq \dfrac{\pi}{2}+k 2\pi \leq 10\pi \Rightarrow k=0; 1; 2; 3; 4$.\\
		Vậy $x=\dfrac{\pi}{2};\, x=\dfrac{\pi}{2}+2\pi;\, x=\dfrac{\pi}{2}+4\pi;\, x=\dfrac{\pi}{2}+6\pi;\, x=\dfrac{\pi}{2}+8\pi$.\\
		Tổng $S=\dfrac{5\pi}{2}+20\pi=\dfrac{45\pi}{2} \Rightarrow a=45, b=2$.\\
		Vậy $ab=90$.
	}
\end{ex}
%%%==============HetBai_BT4==============%%%

%%%==============Bai_BT6==============%%%
\begin{ex}%[1D1V1-6]%[KNTT - Lớp 11 - Ôn tập giữa học kì 1 - Đề 1]%[Phạm Văn Long]
	Một chiếc đồng hồ treo tường có kim giờ dài $5$ cm, vào lúc $12$ giờ trưa cho tới $14$ giờ $15$ cùng ngày thì đầu của kim giờ di chuyển được quãng đường có độ dài là bao nhiêu centimét? (làm tròn đế chữ số thập phân thứ hai)
	\shortans{$5{,}89$}
	\loigiai{
		Khoảng thời gian từ $12$ giờ đến $14$ giờ $15$ cùng ngày là $2{,}25$ tiếng
		Số đo cung của đầu kim giờ quét được từ lúc $12$ giờ đến $14$ giờ $15$ cùng ngày là $2{,}25\cdot \dfrac{\pi}{6}=0,375\pi$.\\
		Quãng đường di chuyển của kim giờ trong khoảng thời gian đó là $0,375\pi \cdot 5\approx 5{,}89\mathrm{~cm}$.
	}
\end{ex}
%%%==============HetBai_BT6==============%%%
\Closesolutionfile{ans}

% \indapan{6}{ans-1-KNTT-GK1-DE1-KQ}
\TL

\begin{ex}%[1D1H3-3]
Cho $\sin a-\cos a=\dfrac{1}{5}\;(90^\circ<a<270^\circ)$. Tính giá trị của biểu thức $\tan 2a$ (làm tròn đến một chữ số thập phân).
% \shortans{$3{,}4$}
\loigiai{
Ta có: $\sin a-\cos a=\dfrac{1}{5}\Leftrightarrow \sin a=\cos a+\dfrac{1}{5}$.\\
Mặt khác: ${\sin ^2}a+{\cos ^2}a=1$.\\
Thay vào ta có: $\left(\cos a+\dfrac{1}{5}\right)^2+{\cos^2} a=1\Leftrightarrow 2{\cos ^2}a+\dfrac{2}{5}\cos a-\dfrac{24}{25}=0\Leftrightarrow \left[ \begin{matrix}
\cos a=\dfrac{-4}{5} \\
\cos a=\dfrac{3}{5}.\\
\end{matrix} \right.$\\
Vì $90^\circ<a<270^\circ\Rightarrow \cos a<0$. Do đó $\cos a=-\dfrac{4}{5}$.\\
Nên $\sin a=\cos a+\dfrac{1}{5}=-\dfrac{3}{5}\Rightarrow \tan a=\dfrac{\sin a}{\cos a}=\dfrac{3}{4}\Rightarrow \tan 2a=\dfrac{2\tan a}{1-{\tan ^2}a}=\dfrac{24}{7}$.\\
Suy ra $\tan 2a\approx 3{,}4$.}
\end{ex}

\begin{ex}%[1D1V5-6]%[24-5-giảng K10-K11, Nguyễn Văn Hồng]
Giả sử một vật dao động điều hoà xung quanh vị trí cân bằng theo phương trình $\break x=2\cos\left(2t+\dfrac{\pi}{4}\right)$. Ở đây, thời gian $t$ tính bằng giây và quãng đường $x$ tính bằng centimét. Hãy cho biết trong thời gian từ $0$ đến $20$ giây, vật đi qua vị trí cân bằng bao nhiêu lần?
\shortans{ $13$}
\loigiai{
Vị trí cân bằng của vật dao động điều hòa là vị trí vật đứng yên, khi đó $x=0$.\\
Xét phương trình $2\cos\left(2t+\dfrac{\pi}{4}\right)=0$ ta có
\[2\cos\left(2t+\dfrac{\pi}{4}\right)=0
\Leftrightarrow
2t+\dfrac{\pi}{4}=\dfrac{\pi}{2}+ k\pi \Leftrightarrow
t= \dfrac{\pi}{8}+ k\dfrac{\pi}{2}, k\in\mathbb{Z}.\]
Trong thời gian từ $0$ đến $20$ giây, tức là $0\le t\le 20$ hay $0\le\dfrac{\pi}{8}+k\dfrac{\pi}{2}\le 20$
$\Leftrightarrow-\dfrac{1}{4}\le k\le\dfrac{160-\pi}{4\pi}$.\\
Mà $k\in\mathbb{Z}$ nên $k\in\left\{{0;1;2;3;\ldots;12}\right\}$.\\
Vậy trong khoảng thời gian từ $0$ đến $20$ giây, vật đi qua vị trí cân bằng $13$ lần.}
\end{ex}

\begin{ex}%[1D2V2-3]
Tìm tổng $50$ số hạng đầu tiên của cấp số cộng $\left({u_n}\right)$, biết $\left\{\begin{aligned}& u_1+u_5-u_3=10 \\
& u_1+u_6=7.
\end{aligned}\right.$
\loigiai{
Ta có
$\heva{&u_1+u_5-u_3=10\\&u_1+u_6=7 }
\Leftrightarrow \heva{&u_1+2d=10\\&2u_1+5d=7 }
\Leftrightarrow\heva{&u_1=36\\&d=-13.}$
Tổng $50$ số hạng đầu tiên của cấp số cộng là
$S_{50}=\dfrac{50}{2}\left[2\cdot u_1+(50-1)d\right]=25\left[72-637\right]=-14125.$
}
\end{ex}

\begin{ex}%[1D2C3-8]
Bố bạn An tặng bạn ấy một máy vi tính trị giá $15$ triệu đồng bằng cách cho bạn ấy tiền hàng tháng theo phương thức: tháng đầu tiên cho $300\,000$ đồng, các tháng từ tháng thứ $2$ trở đi mỗi tháng nhận được số tiền nhiều hơn tháng trước $50\,000$ đồng.
\begin{enumerate}
\item Nếu chọn cách gửi tiết kiệm số tiền được nhận hàng tháng với lãi suất $0,6\%$/tháng thì bạn An gửi bao nhiêu tháng mới đủ mua máy vi tính?
\item Nếu bạn An muốn có ngay máy vi tính để học bằng phương thức mua trả góp hàng tháng bằng số  tiền bố cho với lãi suất ngân hàng là $0,7\%$/tháng thì bạn An mất bao nhiêu tháng để trả đủ số tiền và tháng cuối cùng trả bao nhiêu?
\end{enumerate}
\loigiai{
\begin{enumerate}
\item Nếu chọn cách gửi tiết kiệm số tiền được nhận hàng tháng với lãi suất $0,6\%=0{,}006$/tháng. Khi đó
\begin{itemize}
\item Đầu tháng thứ $1$ số tiền có là  $T_1=300\,000$ đồng.
\item Đầu tháng thứ $2$ số tiền có là  \\
$T_2=T_1\cdot1{,}006+300\,000+50\,000= 300\,000\cdot 1{,}006+300\,000+50\,000$.
\item Đầu tháng thứ $3$ số tiền có là
\begin{eqnarray*}
T_3
&=&T_2\cdot1{,}006+300\,000+2\cdot50\,000\\
&=& 300\,000\cdot(1{,}006)^2+300\,000\cdot1{,}006+50\,000\cdot 1{,}006+300\,000+2\cdot50\,000\\
&=&300\,000(1{,}006^2+1{,}006+1)+ 50\,000(1{,}006+2).
\end{eqnarray*}
\item Đầu tháng thứ $4$ số tiền có là
\begin{eqnarray*}
T_4
&=&T_3\cdot1{,}006+300\,000+3\cdot50\,000\\
&=& 300\,000(1{,}006^3+ 1{,}006^2+1{,}006+1)+ 50\,000(1{,}006^2+2\cdot1{,}006+3).
\end{eqnarray*}
\item Đầu tháng thứ $5$ số tiền có là
\begin{eqnarray*}
T_5
&=&T_4\cdot1{,}006+300\,000+4\cdot50\,000\\
&=& 300\,000(1{,}006^4+ 1{,}006^6+1{,}006^2+1{,}006+1)\\
& & + 50\,000(1{,}006^3+2\cdot1{,}006^2+3\cdot1{,}006+4).
\end{eqnarray*}
\item Đầu tháng thứ $n$ số tiền có là
\begin{eqnarray*}
T_n
&=&T_{n-1}\cdot1{,}006+300\,000+(n-1)\cdot50\,000\\
&=& 300\,000(1{,}006^{n-1}+ 1{,}006^{n-2}+\cdots+1{,}006^2+1{,}006+1)\\
&&
+50\,000\left[1{,}006^{n-2}+2\cdot1{,}006^{n-3}+\cdots+(n-2)\cdot1{,}006+(n-1)\right].
\end{eqnarray*}
\end{itemize}
Thử các giá trị $n$ trong công thức trên, ta thấy $n=20$ thì $T_n=16\,205\,523$. Vậy sau $20$ tháng thì bạn An mới đủ tiền mua máy vi tính.\\
{\bf Sử dụng quy trình bấm máy tính trên máy tính cầm tay như sau:}\\
Vì tháng thứ $n$ số tiền có là
$T_n=T_{n-1}\cdot1{,}006+300\,000+(n-1)\cdot50\,000$.
\begin{itemize}
\item Nhập vào màn hình $X=X+1\colon A=1{,}006\cdot A+300\,000+50\,000(X-1)$.
\item Ấn CALC, gán $X=1$, $A=300\,000$, $=, =, \cdots, =$ đến khi $A$ vươt quá mười lăm triệu.
\end{itemize}
Ta thấy khi $X=20$ thì $A> 15\,000\, 000$.
\item Nếu bạn An muốn có ngay máy vi tính để học bằng phương thức mua trả góp hàng tháng bằng số  tiền bố cho với lãi suất ngân hàng là $0,7\%= 0{,}007$/tháng. Khi đó\\
Vừa mua xong thì An trả luôn bằng tiền nhận được ở tháng đó nên đầu tháng $1$, số tiền còn nợ là  $15\,000\,000 - 300\, 000 = 14\,700\,000$ đồng.
\begin{itemize}
\item Đầu tháng $2$ số tiền còn nợ là\\
$N_2= 14\,700\,000\cdot1{,}007-300\,000-50\,000$ đồng.
\item Đầu tháng $3$ số tiền còn nợ là
\begin{eqnarray*}
N_3
&=&N_2\cdot0{,}007-300\,000-2\cdot50\,000\\
&=& 14\,700\,000\cdot1{,}007^2-300\,000(1{,}007+1)-50\,000(1{,}007+2).
\end{eqnarray*}
\item Đầu tháng $4$ số tiền còn nợ là
\begin{eqnarray*}
N_4
&=&N_3\cdot0{,}007-300\,000-3\cdot50\,000\\
&=&  14\,700\,000\cdot1{,}007^3-300\,000(1{,}007^2+1{,}007^2+1)\\
& &- 50\,000(1{,}007^2+2\cdot1{,}007+3).
\end{eqnarray*}
\item Đầu tháng $5$ số tiền còn nợ là
\begin{eqnarray*}
N_5
&=&N_4\cdot0{,}007-300\,000-4\cdot50\,000\\
&=&  14\,700\,000\cdot1{,}007^4-300\,000(1{,}007^3+1{,}007^2+1{,}007+1)\\
&& -50\,000(1{,}007^3+2\cdot1{,}007^2+3\cdot1{,}007+4).
\end{eqnarray*}
\item Đầu tháng $5$ số tiền còn nợ là
\begin{eqnarray*}
N_n
&=&N_{n-1}\cdot0{,}007-300\,000-(n-1)50\,000\\
&=&  14\,700\,000\cdot1{,}007^{n-1}-300\,000(1{,}007^{n-2}+1{,}007^{n-3}+\cdots+1{,}007+1)\\
&& -50\,000\left[1{,}007^{n-2}+2\cdot1{,}007^{n-3}+\cdots+(n-2)1{,}007+(n-1)\right].
\end{eqnarray*}
\end{itemize}
Thử các giá trị $n$ trong công thức trên, ta thấy $n=21$ thì $N_n<0$. Vậy sau $21$ tháng thì bạn An mới đủ tiền mua máy vi tính.\\
{\bf Sử dụng quy trình bấm máy tính trên máy tính cầm tay như sau:}\\
Vì tháng thứ $n$ số tiền có là
$N_n=N_{n-1}\cdot0{,}007-300\,000-(n-1)50\,000$.
\begin{itemize}
\item Nhập vào màn hình $X=X+1\colon A=1{,}007\cdot A-300\,000-50\,000(X-1)$.
\item Ấn CALC, gán $X=1$, $A=14\,700\,000$, $=, =, \cdots, =$ đến khi $A<0$.
\end{itemize}
Ta thấy khi $X=21$ thì $A=-496\,006$.
\end{enumerate}


}
\end{ex}