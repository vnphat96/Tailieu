\begin{name}
	{\tenchude}
	{TOÁN 11}
	{LỚP TOÁN THẦY PHÁT}
	{Thời gian: 90 phút - Không kể thời gian phát đề}
\end{name}

\caulc

\Opensolutionfile{ans}[ans-ABCD]


\begin{ex}%[1D1H1-2]%[KNTT - Lớp 11 - Ôn tập giữa học kì 1 - Đề 2]%[Don Lee]
	Góc có số đo $270^\circ$ đổi sang radian là
	\choice
	{$\dfrac{5\pi}{6}$}
	{$\dfrac{2\pi}{3}$}
	{$\dfrac{3\pi}{2}$}
	{\True $\dfrac{3\pi}{2}$}
	\loigiai{
		Ta có $270^\circ=\dfrac{270\pi}{180}=\dfrac{3\pi}{2}$.
	}
\end{ex}

\begin{ex}%[1D1H1-4]%[KNTT - Lớp 11 - Ôn tập giữa học kì 1 - Đề 2]%[Don Lee]
	Trên đường tròn đường kính $6$ cm, độ dài cung tròn có số đo bằng $135^\circ$ là
	\choice
	{$14{,}14$ cm}
	{$6{,}28$ cm}
	{$12{,}57$ cm}
	{\True $7{,}07$ cm}
	\loigiai{
		Ta có $135^\circ=\dfrac{135\pi}{180}=\dfrac{3\pi}{4}$. Độ dài cung tròn là $l=R\alpha=\dfrac{3\pi\cdot 3}{4} \approx 7{,}07$ cm.
	}
\end{ex}

\begin{ex}%[1D1N2-4]%[KNTT - Lớp 11 - Ôn tập giữa học kì 1 - Đề 2]%[Don Lee]
	Khẳng định nào sau đây \textbf{sai}?
	\choice
	{$\sin(x-3\pi)=-\sin x$}
	{\True $\tan(x-3\pi)=\tan x$}
	{$\cos(x-3\pi)=-\cos x$}
	{$\cot(x-3\pi)=\cot x$}
	\loigiai{
		Ta có $\tan(x-3\pi)=-\tan x$, do đó khẳng định sai là $\tan(x-3\pi)=\tan x$.
	}
\end{ex}

\begin{ex}%[1D1?2-2]%[KNTT - Lớp 11 - Ôn tập giữa học kì 1 - Đề 2]%[Don Lee]
	Cho góc lượng giác $\alpha=-\dfrac{5\pi}{4}$. Mệnh đề nào sau đây \textbf{sai}?
	\choice
	{$\sin\alpha>0$}
	{$\cot\alpha<0$}
	{\True $\cos\alpha>0$}
	{$\tan\alpha<0$}
	\loigiai{
		Với $\alpha=-\dfrac{5\pi}{4}$ ta có $\sin\alpha>0$, $\cos\alpha<0$, $\tan\alpha<0$, $\cot\alpha<0$.\\
		Vậy mệnh đề sai là $\cos\alpha>0$.
	}
\end{ex}

\begin{ex}%[1D1N2-4]%[KNTT - Lớp 11 - Ôn tập giữa học kì 1 - Đề 2]%[Don Lee]
	Cho góc lượng giác $x$. Khẳng định nào sau đây \textbf{đúng}?
	\choice
	{\True $\sin(\pi+x)=-\sin x$}
	{$\cos(\pi+x)=\cos x$}
	{$\tan(\pi+x)=-\tan x$}
	{$\cot(\pi+x)=-\cot x$}
	\loigiai{
		Vì $x+\pi$ và $x$ là hai góc hơn kém nhau $\pi$ nên ta có $\sin(\pi+x)=-\sin x$.
	}
\end{ex}

\begin{ex}%[1D1H2-3]%[KNTT - Lớp 11 - Ôn tập giữa học kì 1 - Đề 2]%[Don Lee]
	Cho $P=\dfrac{2\sin x+3\cos x}{\sin x+2\cos x}$ với $\cot x=2$. Tính giá trị của $P$.
	\choice
	{$\dfrac{5}{3}$}
	{$\dfrac{2\sqrt{2}}{3}$}
	{\True $\dfrac{8}{5}$}
	{$\dfrac{7}{4}$}
	\loigiai{
		Ta có $\dfrac{2\sin x+3\cos x}{\sin x+2\cos x}=\dfrac{2+3\cot x}{1+2\cot x}=\dfrac{2+3\cdot 2}{1+2\cdot 2}=\dfrac{8}{5}$.
	}
\end{ex}

\begin{ex}%[1D1H2-2]%[KNTT - Lớp 11 - Ôn tập giữa học kì 1 - Đề 2]%[Don Lee]
	Cho $\sin\alpha=\dfrac{1}{3}$ và $\dfrac{\pi}{2}<\alpha<\pi$. Tính $\sin 2\alpha$.
	\choice
	{$-\dfrac{2\sqrt{2}}{9}$}
	{\True $-\dfrac{4\sqrt{2}}{9}$}
	{$\dfrac{2\sqrt{2}}{9}$}
	{$\dfrac{4\sqrt{2}}{9}$}
	\loigiai{
		Vì $\dfrac{\pi}{2}<\alpha<\pi$ nên $\cos\alpha<0$.\\
		Ta có $\cos x=-\sqrt{1-\sin^2 x}=-\sqrt{1-\dfrac{1}{9}}=-\dfrac{2\sqrt{2}}{3}$.\\
		Vậy $\sin 2\alpha=2\sin\alpha \cos\alpha=-\dfrac{4\sqrt{2}}{9}$.
	}
\end{ex}

\begin{ex}%[1D1H4-6] %[KNTT - Lớp 11 - Ôn tập giữa học kì 1 - Đề 2]%[Don Lee]
	Hàm số $y=3\sin\left(x-\dfrac{\pi}{10}\right)-1$ có tập giá trị là
	\choice
	{$[2;4]$}
	{$[-4;2]$}
	{\True $[-4;4]$}
	{$[-3;3]$}
	\loigiai{
		Ta có $-1\le \sin\left(x-\dfrac{\pi}{10}\right)\le 1 \Leftrightarrow -3\le 3\sin\left(x-\dfrac{\pi}{10}\right)\le 3 \Leftrightarrow -4\le \sin\left(x-\dfrac{\pi}{10}\right)-1\le 2$.\\
		Vậy tập giá trị của hàm số $y=3\sin\left(x-\dfrac{\pi}{10}\right)-1$ là $[-4;2]$.
	}
\end{ex}

\begin{ex}%[1D1H4-2]%[KNTT - Lớp 11 - Ôn tập giữa học kì 1 - Đề 2]%[Don Lee]
	Điều kiện xác định của hàm số $y=\dfrac{2\sin x-1}{\cot x}$ là
	\choice
	{$x\neq k\pi$, $k\in \mathbb{Z}$}
	{\True $x\neq \dfrac{k\pi}{2}$, $k\in \mathbb{Z}$}
	{$x\neq \dfrac{k\pi}{3}$, $k\in \mathbb{Z}$}
	{$x\neq \dfrac{k\pi}{4}$, $k\in \mathbb{Z}$}
	\loigiai{
		Điều kiện xác định là $\heva{&\sin x\neq 0\\&\cos x\neq 0} \Rightarrow x\neq \dfrac{k\pi}{2}$, $k\in\mathbb{Z}$.
	}
\end{ex}

\begin{ex}%[1D2H2-6]%[KNTT - Lớp 11 - Ôn tập giữa học kì 1 - Đề 2]%[Don Lee]
	Một cấp số cộng có số hạng tổng quát là $u_n=3n+5$ với $n\in \mathbb{N}^*$. Gọi $S_n$ là tổng $n$ số hạng đầu tiên. Khẳng định nào sau đây \textbf{đúng}?
	\choice
	{$S_n=\dfrac{3^n-1}{2}$}
	{\True $S_n=\dfrac{3n^2+13n}{2}$}
	{$S_n=\dfrac{3n^2+5n}{2}$}
	{$S_n=\dfrac{3n(n+1)}{2}$}
	\loigiai{
		Ta có $u_1=3\cdot 1+5=8$.\\
		Tổng $n$ số hạng đầu tiên là $S_n=\dfrac{n}{2}(u_1+u_n)=\dfrac{n}{2}(8+3n+5)=\dfrac{3n^2+13n}{2}$.
	}
\end{ex}

\begin{ex}%[1D1N5-3]%[KNTT - Lớp 11 - Ôn tập giữa học kì 1 - Đề 2]%[Don Lee]
	Phương trình $\cos x=\cos \dfrac{\pi}{3}$ có nghiệm là
	\choice
	{\True $x=\pm \dfrac{\pi}{3}+k2\pi$, $k\in \mathbb{Z}$}
	{$x=\pm \dfrac{2\pi}{3}+k\pi$, $k\in \mathbb{Z}$}
	{$x=\pm \dfrac{\pi}{4}+k2\pi$, $k\in \mathbb{Z}$}
	{$x=\pm \dfrac{3\pi}{4}+k2\pi$, $k\in \mathbb{Z}$}
	\loigiai{
		Ta có $\cos x=\cos \dfrac{\pi}{3} \Leftrightarrow x=\pm \dfrac{\pi}{3}+k2\pi$, $k\in \mathbb{Z}$.
	}
\end{ex}

\begin{ex}%[1D1H5-6] %[KNTT - Lớp 11 - Ôn tập giữa học kì 1 - Đề 2]%[Don Lee]
	Huyết áp là áp lực máu cần thiết tác động lên thành động mạch nhằm đưa máu đi nuôi dưỡng các mô trong cơ thể. Nhờ lực co bóp của tim và sức cản của động mạch mà huyết áp được tạo ra. Giả sử huyết áp của người đó thay đổi theo thời gian  được cho bởi công thức $p(t)=115+25\sin(160\pi t)$, trong đó $p(t)$ là huyết áp tính theo mmHg và $t$ tính theo phút. Tính chỉ số huyết áp của người đó.
	\choice
	{$100/90$}
	{$150/60$}
	{$120/80$}
	{\True $140/90$}
	\loigiai{
		Ta có\\
		\allowdisplaybreaks
		$\begin{aligned}[t]
			&\quad -1\le \sin 160\pi t\le 1 \Leftrightarrow -25\le 25\sin 160\pi t\le 25\\
			&\Leftrightarrow 90\le 115+25\sin 160\pi t\le 140 \Leftrightarrow 90\leq p(t)\leq 140.
		\end{aligned}$\\
		Vậy huyết áp tâm thu là $140$, huyết áp tâm trương là $90$.\\
		Do đó chỉ số huyết áp của người này là $140/90$.
	}
\end{ex}


\Closesolutionfile{ans}

% \indapan{6}{ans-ABCD}

\cauds

\Opensolutionfile{ans}[ans-DS]

\begin{ex}%[1D5H2-3]%[KNTT - Lớp 11 - Ôn tập giữa học kì 1 - Đề 2]%[Don Lee]
	Tốc độ của $42$ ô tô khi đi qua một trạm đo tốc độ được ghi nhận ở bảng sau
	\begin{center}
		\begin{tabular}{|c|c|c|}
			\hline
			Nhóm      & Tần số & Tần số tích lũy \\ \hline
			$[40;45)$ & $5$      & $5$               \\ \hline
			$[45;50)$ & $10$     & $15 $             \\ \hline
			$[50;55)$ & $7$      & $22$              \\ \hline
			$[55;60)$ & $9$      & $31$              \\ \hline
			$[60;65)$ & $7$      & $38$              \\ \hline
			$[65;70)$ & $4$      & $42$              \\ \hline
			& $n=42$   &                 \\ \hline
		\end{tabular}
	\end{center}
	Xác định tính \textbf{đúng}, \textbf{sai} của các phát biểu sau
	\choiceTF
	{\True Cỡ mẫu của mẫu số liệu là $n=42$}
	{Nhóm $[40;45)$ có giá trị đại diện là $40{,}5$}
	{Số trung bình của mẫu số liệu là $52$}
	{\True Tứ phân vị thứ hai của mẫu số liệu là $Q_2=54{,}3$}
	\loigiai{
		\begin{itemchoice}
			\itemch Đúng.\\
			Cỡ mẫu là $n=42$.
			\itemch Sai.\\
			Nhóm $[40;45)$ có giá trị đại diện là $\dfrac{40+45}{2}=42{,}5$.
			\itemch Sai.\\
			Số trung bình của mẫu số liệu là
			\[n=\dfrac{5\cdot 42{,}5+10\cdot 47{,}5+7\cdot 52{,}5+9\cdot 57{,}5+7\cdot 62{,}5+4\cdot 67{,}5}{42}=54{,}3.\]
			\itemch Đúng.\\
			Ta có $n_1=\dfrac{n}{2}=21$, suy ra nhóm $3$ là nhóm đầu tiên có tần số tích lũy lớn hơn hoặc bằng $21$.\\
			Xét nhóm $3$ là nhóm $[50;55)$ có $r=50$; $d=5$; $n_3=7$ và nhóm $2$ có $cf_2=15$.\\
			Trung vị của mẫu số liệu là $M_e=50+\dfrac{(21-15)}{7}\cdot 5\approx 54{,}3$.\\
			Tứ phân vị thứ hai của mẫu số liệu là $Q_2=54{,}3$.
		\end{itemchoice}
	}
\end{ex}

\begin{ex}%[1D2H3-6]%[KNTT - Lớp 11 - Ôn tập giữa học kì 1 - Đề 2]%[Don Lee]
	Anh Hùng là kỹ sư vừa tốt nghiệp ra trường, anh nộp hồ sơ xin việc vào công ty A. Công ty đề nghị mức lương là $12$ triệu đồng một tháng và cứ sau $9$ tháng thì lương tháng sẽ tăng thêm $10\%$. Hợp đồng ký kết trong $5$ năm. Xác định tính \textbf{đúng}, \textbf{sai} của các phát biểu sau
	\choiceTF
	{\True Tổng lương anh Hùng nhận được trong $3$ tháng đầu tiên là $36$ triệu đồng}
	{\True Số tiền lương anh Hùng nhận được ở tháng thứ $10$ của hợp đồng là $13{,}2$ triệu đồng}
	{Tổng lương anh Hùng nhận được trong $6$ tháng cuối cùng của hợp đồng lớn hơn $130$ triệu đồng}
	{\True Tổng lương anh Hùng nhận được trong $5$ năm lớn hơn $960$ triệu đồng}
	\loigiai{
		\begin{itemchoice}
			\itemch Đúng.\\
			Tổng lương $3$ tháng đầu là $12\cdot 3=36$ triệu đồng.
			\itemch Đúng.\\
			Lương tháng thứ $10$ là $12\cdot 110\%=13{,}2$ triệu đồng.
			\itemch Sai.\\
			Có cấp số nhân với $u_1=108$ triệu, công bội $q=1{,}1$.\\
			Khi đó $5$ năm làm việc tức là $6$ lần $9$ tháng và $6$ tháng còn lại. Tương ứng số tiền nhận được trong $6$ tháng cuối cùng là $\dfrac{6}{9}u_7=\dfrac{6}{9}u_1q^6\approx 128$ triệu đồng.
			\itemch Đúng.\\
			Tổng lương trong $5$ năm là
			\[S_6+\dfrac{6}{9}u_7=\dfrac{u_1\left(1-q^7\right)}{1-q}+\dfrac{6}{9}u_1q^6\approx 960{,}838 \, \text{triệu đồng.}\]
		\end{itemchoice}
	}
\end{ex}

\Closesolutionfile{ans}

% \indapan{3}{ans-DS}

\caukq

\Opensolutionfile{ans}[ans-KQ]

\begin{ex}%[1D2H2-6]%[KNTT - Lớp 11 - Ôn tập giữa học kì 1 - Đề 2]%[Don Lee]
	Cho cấp số cộng $(u_n)$ với $\heva{&u_1-u_3+u_5=15\\&u_1+u_6=27}$. Tính tổng $S_{10}$ của $10$ số hạng đầu tiên.
	\shortans{$75$}
	\loigiai{
		Ta có $\heva{&u_1-u_3+u_5=15\\&u_1+u_6=27} \Leftrightarrow \heva{&u_1+2d=15\\&2u_1+5d=27} \Leftrightarrow \heva{&u_1=21\\&d=-3.}$\\
		Suy ra $S_{10}=\dfrac{10}{2}\cdot [2\cdot 21+9\cdot (-3)]=75$.
	}
\end{ex}

\begin{ex}%[1D1H5-3]%[KNTT - Lớp 11 - Ôn tập giữa học kì 1 - Đề 2]%[Don Lee]
	Tổng các nghiệm của phương trình $\tan\left(2x-\dfrac{\pi}{6}\right)=\dfrac{1}{\sqrt{3}}$ trên đoạn $[0; 2\pi]$ có dạng $\dfrac{m\pi}{3}$. Tìm $m$.
	\shortans{$11$}
	\loigiai{
		Ta có $\tan\left(2x-\dfrac{\pi}{6}\right)=\dfrac{1}{\sqrt{3}} \Leftrightarrow \tan\left(2x-\dfrac{\pi}{6}\right)=\tan\dfrac{\pi}{6} \Leftrightarrow x=\dfrac{\pi}{6}+k\dfrac{\pi}{2}$.\\
		Do $x\in [0;2\pi]$ suy ra $0\le \dfrac{\pi}{6}+k\dfrac{\pi}{2} \Leftrightarrow -\dfrac{1}{3}\le k\le \dfrac{11\pi}{6}$.\\
		Mà $k\in\mathbb{Z}$ suy ra $k\in \{0;1;2;3\}$ suy ra $x\in \left\{\dfrac{\pi}{6}; \dfrac{2\pi}{3}; \dfrac{7\pi}{6}; \dfrac{5\pi}{3}\right\}$.\\
		Tổng các nghiệm trên đoạn $[0; 2\pi]$ là $\dfrac{\pi}{6}+\dfrac{2\pi}{3}+\dfrac{7\pi}{6}+\dfrac{5\pi}{3}=\dfrac{11\pi}{3}$, suy ra $m=11$.
	}
\end{ex}

\begin{ex}%[1D2V2-3]%[KNTT - Lớp 11 - Ôn tập giữa học kì 1 - Đề 2]%[Don Lee]
	Cho hai cấp số cộng hữu hạn, mỗi cấp số có $2024$ số hạng là $4; 7; 10; 13; 16; \ldots$ và $1; 6; 11; 16; 21; \ldots$. Có bao nhiêu số có mặt trong cả hai cấp số cộng?
	\shortans{$404$}
	\loigiai{
		Số hạng tổng quát của cấp số cộng $\left(x_n\right)$ là $x_n=3n+1$.\\
		Số hạng tổng quát của cấp số cộng $\left(y_n\right)$ là $y_m=5m-4$.\\
		Xét $3n+1=5m-4 \Leftrightarrow 3n=5(m-1) \Rightarrow n\mid 5 \Rightarrow n\in \{5;10;15;\ldots;2020\}$.\\
		Vậy có $404$ số chung.
	}
\end{ex}

\begin{ex}%[1D1V1-6]%[KNTT - Lớp 11 - Ôn tập giữa học kì 1 - Đề 2]%[Don Lee]
	Ngày 29/2/2024, lúc 15h30, người đàn ông thấy kim giờ không đi qua số $3$ nữa. Tính đến 12h00 ngày 1/1/2034, kim giờ đi qua số $3$ bao nhiêu lần?
	\shortans{$7277$}
	\loigiai{
		Mỗi ngày, kim giờ đi qua số $3$ hai lần.\\
		Từ 1/3/2024 đến 31/12/2024 có $306$ ngày, kim giờ đi qua số $3$ là $306\cdot 2=702$ lần.\\
		Từ năm $2025$ đến $2033$ có $9$ năm (bao gồm $2$ năm nhuận), nên có $365\cdot 9+2=3287$ ngày, kim giờ đi qua số $3$ là $3287\cdot 2=6574$ lần.\\
		Từ 0h00 đến 12h00 ngày 1/1/2034, kim giờ đi qua số $3$ thêm $1$ lần.\\
		Tổng cộng là $702+6574+1=7277$ lần.
	}
\end{ex}


\Closesolutionfile{ans}

% \indapan{6}{ans-KQ}
\TL

\begin{ex}%[1D1V4-6]%[KNTT - Lớp 11 - Ôn tập giữa học kì 1 - Đề 3]%[Phúc Hậu]
	Giải phương trình $2 \cos \left(2 x+\dfrac{\pi}{6}\right)-\sqrt{3}=0$.
\loigiai{
	\allowdisplaybreaks
	\begin{eqnarray*}
		& & 2 \cos \left(2 x+\dfrac{\pi}{6}\right)-\sqrt{3}=0 \\
		&\Leftrightarrow& \cos \left(2 x+\dfrac{\pi}{6}\right)=\dfrac{\sqrt{3}}{2}\\
		&\Leftrightarrow& \hoac{&2 x+\dfrac{\pi}{6}=\dfrac{\pi}{6}+k2 \pi \\&2 x+\dfrac{\pi}{6}=-\dfrac{\pi}{6}+k 2 \pi},\, k \in \mathbb{Z} \\
		&\Leftrightarrow& \hoac{&x=k \pi \\&x=-\dfrac{\pi}{6}+k \pi},\, k \in \mathbb{Z}.
	\end{eqnarray*}
}
\end{ex}

\begin{ex}%[1D1V5-6]%[CTST - Lớp 11 - Ôn tập giữa học kì 1 - Đề 1]%[Tex hóa: Lê Thị Thúy Hằng]
	Một chất điển dao động điều hòa theo phương trình $x= 2 \cos \left( 2\pi t + \dfrac{\pi}{2} \right)$, $t$ tính bằng giây và $x$ tính bằng cm.Thời điểm đầu tiên vật có li độ lớn nhất bằng bao nhiêu giây?
	% \shortans{$0{,}75$}
	\loigiai
	{
	Với mọi $t \ge 0$ ta có $-1 \le \cos \left( 2\pi t + \dfrac{\pi}{2} \right) \le 1 \Leftrightarrow -2 \le 2 \cos \left( 2\pi t + \dfrac{\pi}{2} \right) \le 2$.\\
	Do đó, li độ lớn nhất là $x=2$ cm xảy ra khi \\
	$ \cos \left( 2\pi t + \dfrac{\pi}{2} \right) =1 \Leftrightarrow 2\pi t + \dfrac{\pi}{2} = k2\pi \Leftrightarrow t = k - \dfrac{1}{4}$, $k \in \mathbb{Z}$.\\
	Vì $t \ge 0$ nên $k- \dfrac{1}{4} \ge 0 \Leftrightarrow k \ge \dfrac{1}{4}$.\\
	Vì $k \in \mathbb{Z}$ nên thời điểm đầu tiên thỏa mãn ứng với $k=1 \Rightarrow t_0 = \dfrac{3}{4} = 0{,}75$ giây.
	}
\end{ex}

\begin{ex}%[1D2V3-8]%[1D3V2-8]%[CTST - Lớp 11 - Ôn tập giữa học kì 1 - Đề 3]%[Nguyễn Mộng Hùng]
	Cho tam giác $OA_1A_2$ vuông tại $A_2$, $A_1A_2=2$ và $\widehat{A_1OA_2}=45^{\circ}$. Lần lượt hạ các đường vuông góc $A_2A_3\perp OA_1$; $A_3A_4\perp OA_2$; $A_4A_5\perp OA_1$; $A_5A_6\perp OA_2$; $\ldots$. Tiếp tục quá trình này tổng cộng $7$ lần, ta nhận được đường gấp khúc $A_1A_2A_3A_4\ldots A_7$. Tính độ dài đường gấp khúc này (Làm tròn đến hàng phần trăm).
	\begin{center}
		\begin{tikzpicture}[declare function={r= 1.3;}]
			\path
			(0,0) coordinate (O)
			(r,0) coordinate (A_6)
			(2*r,0) coordinate (A_4)
			(4*r,0) coordinate (A_2)
			(4*r,4*r) coordinate (A_1)
			($(O)!.5!(A_1)$) coordinate (A_3)
			($(O)!.5!(A_3)$) coordinate (A_5)
			($(O)!.5!(A_5)$) coordinate (A_7)
			;
			\draw (O)--(A_2)--(A_1)--cycle (A_2)--(A_3)--(A_4)--(A_5)--(A_6)--(A_7)
			pic[draw,angle radius=5pt]{right angle= A_2--A_3--A_1}
			pic[draw,angle radius=5pt]{right angle= A_4--A_5--A_3}
			pic[draw,angle radius=5pt]{right angle= A_6--A_7--A_5}
			pic[draw,angle radius=5pt]{right angle= A_3--A_4--A_6}
			pic[draw,angle radius=5pt]{right angle= A_5--A_6--O}
			pic[draw,angle radius=5pt]{right angle= A_1--A_2--O}
			(O)pic[draw,angle radius = 15] {angle = A_2--O--A_1} node[shift={(22:22pt)},scale=.6]{$ 45^\circ $}	
			;
			
			\foreach \t/\g in {O/-90,A_6/-90,A_4/-90,A_2/-90,A_1/90,A_3/110,A_5/110,A_7/110}{
				\draw[fill=white] (\t) circle (1pt) node[shift={(\g:7pt)},font=\scriptsize]{$ \t $};
			}
		\end{tikzpicture}
	\end{center}
	% \shortans{$4{,}83$}
	\loigiai{
		\begin{center}
			\begin{tikzpicture}[declare function={r= 1.3;}]
				\path
				(0,0) coordinate (O)
				(r,0) coordinate (A_6)
				(2*r,0) coordinate (A_4)
				(4*r,0) coordinate (A_2)
				(4*r,4*r) coordinate (A_1)
				($(O)!.5!(A_1)$) coordinate (A_3)
				($(O)!.5!(A_3)$) coordinate (A_5)
				($(O)!.5!(A_5)$) coordinate (A_7)
				;
				\draw (O)--(A_2)--(A_1)--cycle (A_2)--(A_3)--(A_4)--(A_5)--(A_6)--(A_7)
				pic[draw,angle radius=5pt]{right angle= A_2--A_3--A_1}
				pic[draw,angle radius=5pt]{right angle= A_4--A_5--A_3}
				pic[draw,angle radius=5pt]{right angle= A_6--A_7--A_5}
				pic[draw,angle radius=5pt]{right angle= A_3--A_4--A_6}
				pic[draw,angle radius=5pt]{right angle= A_5--A_6--O}
				pic[draw,angle radius=5pt]{right angle= A_1--A_2--O}
				(O)pic[draw,angle radius = 15] {angle = A_2--O--A_1} node[shift={(22:22pt)},scale=.6]{$ 45^\circ $}	
				;
				
				\foreach \t/\g in {O/-90,A_6/-90,A_4/-90,A_2/-90,A_1/90,A_3/110,A_5/110,A_7/110}{
					\draw[fill=white] (\t) circle (1pt) node[shift={(\g:7pt)},font=\scriptsize]{$ \t $};
				}
			\end{tikzpicture}
		\end{center}
		Các góc $\widehat{A_1A_2A_3}, \widehat{A_2A_3A_4}, \widehat{A_3A_4A_5}, \ldots$ đều bằng góc $\widehat{A_1OA_2}$ nên đều có đo là $45^{\circ}$.
		$$
		\begin{aligned}
			& A_2 A_3=A_1 A_2 \cdot \cos 45^{\circ}=2 \cdot \dfrac{\sqrt{2}}{2}=\sqrt{2}\\
			& A_3 A_4=A_2 A_3 \cdot \cos 45^{\circ}=2 \cdot \dfrac{\sqrt{2}}{2} \cdot \dfrac{\sqrt{2}}{2}=2\cdot\left(\dfrac{\sqrt{2}}{2}\right)^2 \\
			& A_4 A_5=A_3 A_4 \cdot \cos 45^{\circ}=2 \cdot\left(\dfrac{\sqrt{2}}{2}\right)^2 \cdot \dfrac{\sqrt{2}}{2}=2 \cdot\left(\dfrac{\sqrt{2}}{2}\right)^3; \ldots
		\end{aligned}
		$$
		% Do đó độ dài các đoạn thẳng $A_1A_2$, $A_2A_3$, $A_3A_4$, $\ldots$ tạo thành cấp số nhân lùi vô hạn với số hạng đầu bằng $2$ với công bội bằng $\dfrac{\sqrt{2}}{2}$.\\
		% Vậy độ dài đường gấp khúc $A_1A_2A_3A_4\ldots$ là $l=\dfrac{2}{1-\tfrac{\sqrt{2}}{2}}=2\sqrt{2}\cdot(\sqrt{2}+1)\approx 6{,}83$.
	}
\end{ex}

\begin{ex}%[1D2V2-7]%[CTST - Lớp 11 - Ôn tập giữa học kì 1 - Đề 3]%[Nguyễn Mộng Hùng]
	Công ty $A$ muốn thuê một mảnh đất trong vòng $15$ năm để làm nhà kho. Có hai công ty môi giới bất động sản $B$ và bất động sản $C$ đều muốn cho thuê. Mỗi công ty, đưa ra phương án cho thuê như sau:\\
	Phương án công ty $B$ trả tiền theo quý, quý đầu tiên là $10$ triệu đồng và từ quý thứ hai trở đi mỗi quý tăng thêm $500\,000$ đồng.\\
	Phương án công ty $C$ trả tiền theo năm, năm đầu tiên thuê đất là $70$ triệu và kể từ năm thứ hai trở đi mỗi năm tăng thêm $3$ triệu đồng.\\
	Công ty $A$ nên lựa chọn thuê đất của công ty môi giới bất động sản nào để chi phí là thấp nhất và số tiền đó bằng bao nhiêu?
	% \shortans{$15$}
	\loigiai{
		Gọi $B_n$, $C_n$ lần lượt là số tiền công ty $A$ cần trả theo cách tính của hai công ty $B$ và $C$
		Theo bài ra, ta có
		\begin{itemize}
			\item $B_n$ là tổng $n$ số hạng đầu tiên của một cấp số cộng với $u_1=10$ triệu đồng và công sai $d=0,5$ triệu đồng.
			\item $C_n$ là tổng $n$ số hạng đầu tiên của một cấp số cộng với $u_1=70$ triệu đồng, công sai $d=3$ triệu đồng.
		\end{itemize}
		Khi đó:\\
		Nếu thuê đất của công ty $B$ trong vòng 15 năm thì số tiền công ty $A$ phải trả là
		$$B_{60}=\dfrac{n}{2}\cdot[2u_1+(n-1)d]=30\cdot(2\cdot 10+59\cdot 0{,}5)=1\,485\, (\text{triệu đồng}).$$
		Nếu thuê đất của công ty $C$ trong vòng 15 năm thì số tiền công ty $A$ phải trả là
		$$C_{15}=\dfrac{n}{2}\cdot[2u_1+(n-1)d]=7{,}5(2\cdot70+14\cdot 3)=1\,365\, (\text{triệu đồng}).$$
		Do đó công ty $A$ nên thuê đất của công ty $C$ và số tiền phải trả là $1\,365$ (triệu đồng).\\
		% Vậy $a+b+c+d=15$.
	}
\end{ex}