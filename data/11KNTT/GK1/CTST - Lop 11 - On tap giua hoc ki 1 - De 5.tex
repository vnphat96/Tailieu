\section{Tên sách - Lớp 1x - Ôn tập giữa học kì 1 - Đề y}

\caulc

\Opensolutionfile{ans}[ans-ABCD]

%---Câu 1
\begin{ex}% [1D1N1-2]%[CTST - Lớp 11 - Ôn tập giữa học kì 1 - Đề 5]%[Sauluoi3105]
Một đường tròn bán kính $7$ cm. Cung tròn có góc ở tâm bằng $54^{\circ}$ có độ dài là
\choice
{\True $\dfrac{21}{10} \pi(\mathrm{~cm})$}
{$\dfrac{11}{20} \pi(\mathrm{~cm})$}
{$\dfrac{63}{20} \pi(\mathrm{~cm})$}
{$\dfrac{20}{11} \pi(\mathrm{~cm})$}
\loigiai{
Ta có: $54^{\circ}=\dfrac{54 \cdot \pi}{180}=\dfrac{3 \pi}{10} \mathrm{rad}$.\\
Vậy độ dài cung tròn là $l=R \alpha=7 \times \dfrac{3 \pi}{10}=\dfrac{21 \pi}{10}\,(\mathrm{~cm})$.
}
\end{ex}

%---Câu 2
\begin{ex}% [1D1H1-2]%[CTST - Lớp 11 - Ôn tập giữa học kì 1 - Đề 5]%[Sauluoi3105]
Cho góc lượng giác $\alpha$ biết $\pi<\alpha<\dfrac{3 \pi}{2}$. Xét dấu $\cos (\pi+\alpha)$ và $\tan \left(\dfrac{3 \pi}{2}-\alpha\right)$. Chọn kết quả đúng.
\choice
{$\left\{\begin{array}{l}\cos (\pi+\alpha)>0 \\ \tan \left(\dfrac{3 \pi}{2}-\alpha\right)<0\end{array}\right.$}
{$\left\{\begin{array}{l}\cos (\pi+\alpha)<0 \\ \tan \left(\dfrac{3 \pi}{2}-\alpha\right)<0\end{array}\right.$}
{\True $\left\{\begin{array}{l}\cos (\pi+\alpha)>0 \\ \tan \left(\dfrac{3 \pi}{2}-\alpha\right)>0\end{array}\right.$}
{$\left\{\begin{array}{l}\cos (\pi+\alpha)<0 \\ \tan \left(\dfrac{3 \pi}{2}-\alpha\right)>0\end{array}\right.$}
\loigiai{
Ta có: $\pi<\alpha<\dfrac{3 \pi}{2} \Rightarrow\left\{\begin{array}{l}2 \pi<\pi+\alpha<\dfrac{5 \pi}{2} \\ 0<\dfrac{3 \pi}{2}-\alpha<\dfrac{\pi}{2}\end{array} \Rightarrow\left\{\begin{array}{l}\cos (\pi+\alpha)>0 \\ \tan \left(\dfrac{3 \pi}{2}-\alpha\right)>0\end{array}\right.\right.$.
}
\end{ex}

%---Câu 3
\begin{ex}%[1D1N3-3]%[CTST - Lớp 11 - Ôn tập giữa học kì 1 - Đề 5]%[Sauluoi3105]
Trong các khẳng định sau, khẳng định nào \textbf{sai}?
\choice
{$\sin 2 a=2 \sin a \cos a$}
{\True $\cos (a+b)=\cos a+\cos b$}
{$\cos 2 a=2 \cos ^2 a-1$}
{$\cos 2 a=1-2 \sin ^2 a$}
\loigiai{
Ta có $\cos (a+b)=\cos a \cos b-\sin a \sin b$.
}
\end{ex}

%---Câu 4
\begin{ex}%[1D1B3-3]%[CTST - Lớp 11 - Ôn tập giữa học kì 1 - Đề 5]%[Sauluoi3105]
Cho $\sin x=\dfrac{4}{5}$ khi đó $\sin \left(x+\dfrac{\pi}{6}\right) \cdot \sin \left(x-\dfrac{\pi}{6}\right)$ bằng
\choice
{$\dfrac{39}{50}$}
{\True $\dfrac{39}{100}$}
{$\dfrac{11}{50}$}
{$\dfrac{11}{100}$}
\loigiai{
Ta có 
\allowdisplaybreaks{
\begin{eqnarray*}
\sin \left(x+\dfrac{\pi}{6}\right) \cdot \sin \left(x-\dfrac{\pi}{6}\right)
&=&\dfrac{1}{2}\left(\cos \dfrac{\pi}{3}-\cos 2 x\right)\\
&=&\dfrac{1}{2}\left(\dfrac{1}{2}-1+2 \sin ^2 x\right)\\
&=&\dfrac{1}{2}\left(2 \sin ^2 x-\dfrac{1}{2}\right)\\
&=&\dfrac{1}{2}\left[2\left(\dfrac{4}{5}\right)^2-\dfrac{1}{2}\right]=\dfrac{39}{100}.
\end{eqnarray*}
}
}
\end{ex}



%---Câu 5
\begin{ex}%[1D1B4-2]%[CTST - Lớp 11 - Ôn tập giữa học kì 1 - Đề 5]%[Sauluoi3105]
Tập xác định của hàm số $y=\dfrac{2 \cos x+3}{\sin ^2 x-1}$ là
\choice
{\True $\mathscr{D}=\mathbb{R} \backslash\left\{\dfrac{\pi}{2}+k \pi, k \in \mathbb{Z}\right\}$}
{$\mathscr{D}=\mathbb{R} \backslash\{k \pi, k \in \mathbb{Z}\}$}
{$\mathscr{D}=\mathbb{R} \backslash\left\{-\dfrac{\pi}{2}+k 2 \pi, k \in \mathbb{Z}\right\}$}
{$\mathscr{D}=\mathbb{R} \backslash\left\{\dfrac{\pi}{2}+k 2 \pi, k \in \mathbb{Z}\right\}$}
\loigiai{
Hàm số xác định $\Leftrightarrow \sin ^2 x-1 \neq 0 \Leftrightarrow \cos x \neq 0 \Leftrightarrow x \neq \dfrac{\pi}{2}+k \pi$.\\
Vậy tập xác định của hàm số là $\mathscr{D}=\mathbb{R} \backslash\left\{\dfrac{\pi}{2}+k \pi, k \in \mathbb{Z}\right\}$.
}
\end{ex}


%---Câu 6
\begin{ex}%[1D1H4-4] %[CTST - Lớp 11 - Ôn tập giữa học kì 1 - Đề 5]%[Sauluoi3105]
Trong các hàm số dưới đây, hàm số nào là hàm số chẵn?
\choice
{$y=\cos x \cdot \sin ^3 x$}
{$y=\dfrac{\cot x}{\tan ^2 x+1}$}
{$y=\cos x \cdot \sin 6 x$}
{\True $y=\sin ^{2022} x \cdot \cos x$}
\loigiai{
\begin{itemize}
\item Với hàm số $y=f(x)=\cos x \cdot \sin ^3 x$.\\
Tập xác định: $\mathscr{D}=\mathbb{R}$.
Với mọi $x \in \mathscr{D}$ thì $f(-x)=\cos (-x) \cdot \sin ^3(-x)=-\cos x \cdot \sin ^3 x \neq f(x)$ $y=\cos x \cdot \sin ^3 x$ không phải hàm số chẵn.
\item Với hàm số $y=f(x)=\dfrac{\cot x}{\tan ^2 x+1}$.\\
Tập xác định: $\mathscr{D}=\mathbb{R} \backslash\left\{\dfrac{k \pi}{2}, k \in \mathbb{Z}\right\}$.
Với mọi $x \in \mathscr{D}$ thì $f(-x)=\dfrac{\cot (-x)}{\tan ^2(-x)+1}=-\dfrac{\cot x}{\tan ^2 x+1} \neq f(x)$ nên hàm số $y=\cos x \cdot \sin ^3 x$ không phải hàm số chẵn.
\item Với hàm số $y=f(x)=\cos x \cdot \sin 6 x$.\\
Tập xác định: $\mathscr{D}=\mathbb{R}$.
Với mọi $x \in \mathscr{D}$ thì $f(-x)=\cos (-x) \cdot \sin [6(-x)]=-\cos x \cdot \sin 6 x \neq f(x)$ nên hàm số $y=\cos x \cdot \sin 6 x$ không phải là hàm số chẵn.
\item Với hàm số $y=f(x)=\sin ^{2022} x \cdot \cos x$.\\
Tập xác định: $\mathscr{D}=\mathbb{R}$.\\
Với mọi $x \in \mathscr{D}$ thì $-x \in \mathscr{D}$ và $f(-x)=[\sin (-x)]^{2022} \cdot \cos (-x)=\sin ^{2022} x \cdot \cos x=f(x)$ nên hàm số $y=\sin ^{2022} x \cdot \cos x$ là hàm số chẵn.
\end{itemize}
}
\end{ex}

%---Câu 7
\begin{ex}%[1D1N4-7]%[CTST - Lớp 11 - Ôn tập giữa học kì 1 - Đề 5]%[Sauluoi3105]
Đồ thị hàm số như hình vẽ sau đây là của hàm số nào?
\begin{center}
		\begin{tikzpicture}[scale=0.7,font=\footnotesize,samples=200,>=stealth]
		\tikzset{declare function={
				xmin=-2.85*pi;xmax=2.85*pi;
				ymin=-4;ymax=4;}}
		\draw[->] (xmin,0)--(xmax,0) node[below]{$x$};
		\draw[->] (0,ymin)--(0,ymax) node[left]{$y$};
		\node at (0,0) [below left=-3pt]{$O$};
		\node at (0.2,3)[right]{$y=\cot x$};
		\clip (xmin+0.1,ymin+0.1) rectangle (xmax-0.2,ymax-0.1);
		\foreach \k in {-5,-4,...,5}{
			\draw plot[domain=\k*pi+0.1:(1+\k)*pi-0.1](\x,{cos(\x r)/sin(\x r)});
			\draw[dashed] (\k*pi,ymin)--(\k*pi,ymax);}
		\draw plot[domain=0.1:pi-0.1](\x,{cos(\x r)/sin(\x r)});
		\foreach \x/\t in {0.5*pi/$\tfrac{\pi}{2}$,1.5*pi/$\tfrac{3\pi}{2}$,
			2.5*pi/$\tfrac{5\pi}{2}$,3.5*pi/$\tfrac{7\pi}{2}$}{
			\node at (\x,0) [shift={(-109:10pt)}]{\normalsize\t};
			\node at (-\x,0) [shift={(-125:11.5pt)}]{\normalsize$-$\t};}
		\foreach \x/\t in {pi/$\pi$,2*pi/$2\pi$,3*pi/$3\pi$,4*pi/$4\pi$}{
			\node at (\x,0) [shift={(-143:8pt)}]{\t};
			\node at (-\x,0) [shift={(-152:11.5pt)}]{$-$\t};}
		\end{tikzpicture}
\end{center}
\choice
{$y=\sin x$}
{$y=\cos x$}
{$y=\tan x$}
{\True $y=\cot x$}
\loigiai{
Đồ thị hàm số trên là đồ thị hàm số $y=\cot x$.
}
\end{ex}


%---Câu 8
\begin{ex}%[1D1N4-3]%[CTST - Lớp 11 - Ôn tập giữa học kì 1 - Đề 5]%[Sauluoi3105]
Cho đồ thị hàm số $y=\sin x$ trên $[-2 \pi ; 2 \pi]$ như hình dưới
	\begin{center}
		\begin{tikzpicture}[scale=0.75,font=\footnotesize,samples=200,>=stealth]
		\tikzset{declare function={
				xmin=-3.4*pi;xmax=3.4*pi;
				ymin=-1.6;ymax=1.8;}}
		\draw[->] (xmin,0)--(xmax,0) node [below]{$x$};
		\draw[->] (0,ymin)--(0,ymax) node [right]{$y$};
		\node at (0,0) [below right=-3pt]{$O$};
		\node at (2.82*pi,0.6) [right]{$y=\sin x$};
		\draw[dashed] (-pi,0)--(-pi,-2.8) (pi,0)--(pi,-2.8);
		\draw[<->] (-pi,-2.2)--(pi,-2.2)node[pos=0.5,below]{$2\pi$};
		\clip (xmin+0.1,ymin+0.1) rectangle (xmax-0.2,ymax-0.1);
		\node at (0,1)[shift={(125:9pt)}]{$1$};
		\node at (0,-1)[shift={(-140:11pt)}]{$-1$};
		\draw plot[domain=xmin:xmax](\x,{sin(\x r)});
		\draw[thick,red] plot[domain=-pi:pi](\x,{sin(\x r)});
		\draw[dashed] (xmin,1)--(xmax,1) (xmin,-1)--(xmax,-1);
		\foreach \x/\t in {0.5*pi/$\tfrac{\pi}{2}$,2.5*pi/$\tfrac{5\pi}{2}$,4.5*pi/$\tfrac{9\pi}{2}$}{
			\draw[fill] (\x,1)circle(1pt) (\x,0)circle(0.5pt);
			\draw[dashed] (\x,1)--(\x,0) node[below=-1pt]{\normalsize\t};
			\draw[fill] (-\x,-1)circle(1pt) (-\x,0)circle(0.5pt);
			\draw[dashed] (-\x,-1)--(-\x,0) node[above=-1pt,xshift=-3]{\normalsize$-$\t};}
		\foreach \x/\t in {1.5*pi/$\tfrac{3\pi}{2}$,3.5*pi/$\tfrac{7\pi}{2}$,5.5*pi/$\tfrac{11\pi}{2}$}{
			\draw[fill] (\x,-1)circle(1pt) (\x,0)circle(0.5pt);
			\draw[dashed] (\x,-1)--(\x,0) node[above=-1pt]{\normalsize\t};
			\draw[fill] (-\x,1)circle(1pt) (-\x,0)circle(0.5pt);
			\draw[dashed] (-\x,1)--(-\x,0) node[below=-1pt,xshift=-3]{\normalsize$-$\t};}
		\foreach \x/\t in {pi/$\pi$,3*pi/$3\pi$,5*pi/$5\pi$}{
			\draw[fill] (\x,0)circle(1pt) (-\x,0)circle(1pt);
			\node at (\x,0) [below=-1pt,xshift=-4]{\t};
			\node at (-\x,0) [below=-1pt,xshift=-7]{$-$\t};}
		\foreach \x/\t in {2*pi/$2\pi$,4*pi/$4\pi$,6*pi/$6\pi$}{
			\draw[fill] (\x,0)circle(1pt) (-\x,0)circle(1pt);
			\node at (\x,0) [below=-1pt,xshift=4]{\t};
			\node at (-\x,0) [below=-1pt,xshift=6]{$-$\t};}
		\end{tikzpicture}
	\end{center}
Hỏi khẳng định nào dưới đây là đúng
\choice
{Hàm số đồng biến trên khoảng $(-2 \pi ;-\pi)$}
{Hàm số đồng biến trên khoảng $(-\pi ; 0)$}
{Hàm số đồng biến trên khoảng $(0 ; 2 \pi)$}
{\True Hàm số đồng biến trên khoảng $\left(-\dfrac{\pi}{2} ; \dfrac{\pi}{2}\right)$}
\loigiai{
Từ đồ thị hàm số suy ra hàm số đồng biến trên $\left(-\dfrac{\pi}{2} ; \dfrac{\pi}{2}\right)$.
}
\end{ex}

%---Câu 9
\begin{ex}%[1D1?5-3]%[CTST - Lớp 11 - Ôn tập giữa học kì 1 - Đề 5]%[Sauluoi3105]
Nghiệm của phương trình $2 \sin x+1=0$ là
\choice
{\True $x=-\dfrac{\pi}{6}+k 2 \pi ; x=\dfrac{7 \pi}{6}+k 2 \pi$}
{$x=\dfrac{\pi}{6}+k 2 \pi ; x=\dfrac{7 \pi}{6}+k 2 \pi$}
{$x=\pi+k 2 \pi ; x=\dfrac{\pi}{8}+k 2 \pi$}
{$x=-\dfrac{\pi}{6}+k 2 \pi ; x=\dfrac{5 \pi}{6}+k 2 \pi$}
\loigiai{
Ta có: $2 \sin x+1=0 \Leftrightarrow \sin x=\dfrac{-1}{2} \Leftrightarrow\left[\begin{array}{l}x=-\dfrac{\pi}{6}+k 2 \pi \\ x=\pi+\dfrac{\pi}{6}+k 2 \pi\end{array} \Leftrightarrow\left[\begin{array}{l}x=-\dfrac{\pi}{6}+k 2 \pi \\ x=\dfrac{7 \pi}{6}+k 2 \pi\end{array} \quad(k \in \mathbb{Z})\right.\right.$\\
Vậy phương trình có nghiệm là $x=-\dfrac{\pi}{6}+k 2 \pi ; x=\dfrac{7 \pi}{6}+k 2 \pi$.
}
\end{ex}


%---Câu 10
\begin{ex}%[1D1N5-3]%[CTST - Lớp 11 - Ôn tập giữa học kì 1 - Đề 5]%[Sauluoi3105]
Nghiệm của phương trình $\tan 5 x=\tan 2 x$ là
\choice
{\True $x=\dfrac{k \pi}{3}, k \in \mathbb{Z}$}
{$x=\dfrac{k 2 \pi}{3}, k \in \mathbb{Z}$}
{$x=k 3 \pi, k \in \mathbb{Z}$}
{$x=k \pi, k \in \mathbb{Z}$}
\loigiai{
Ta có 
$\tan 5 x=\tan 2 x 
\Leftrightarrow 5 x=2 x+k \pi 
\Leftrightarrow 3 x=k \pi 
\Leftrightarrow x=\dfrac{k \pi}{3},(k \in \mathbb{Z})$.
}
\end{ex}





%---Câu 11
\begin{ex}%[1H4H1-4]%[CTST - Lớp 11 - Ôn tập giữa học kì 1 - Đề 5]%[Sauluoi3105]
Cho tứ diện $ABCD$. $G$ là trọng tâm tam giác $BCD$, $M$ là trung điểm $CD$, $I$ là điểm trên đoạn thẳng $AG$, $BI$ cắt mặt phẳng $(ACD)$ tại $J$. Khẳng định nào sau đây \textbf{sai}?
\choice
{$AM=(ACD) \cap(ABG)$}
{$A$, $J$, $M$ thẳng hàng}
{\True $J$ là trung điểm $AM$}
{$DJ=(ACD) \cap(BDJ)$}
\loigiai{
\immini{
Ta có $A \in(ACD) \cap(ABG)$.\\
Ta có $\heva{& M \in BG \\& M \in CD} \Rightarrow M \in(ACD) \cap(ABG)$.\\ 
Nên $AM=(ACD) \cap(ABG)$.\\ 
Nên $AM=(ACD) \cap(ABG)$.\\
Ta có  $A$, $J$, $M$ cùng thuộc hai mặt phẳng phân biệt $(ACD)$, $(ABG)$ nên $A$, $J$, $M$ thẳng hàng.\\ 
Vì $I$ là điểm tùy ý trên $A G$ nên $J$ không phải lúc nào cũng là trung điểm của $A M$.
}{
\begin{tikzpicture}[scale=0.85]
\coordinate[label=left:$B$] (B) at (0,0);
\coordinate[label=right:$D$] (D) at (6,0);
\coordinate[label=below:$C$] (C) at (-70:3);
\coordinate [label=above:$A$] (A) at (1,4);
\coordinate [label=below right:$M$] (M) at ($(C)!0.5!(D)$);
\coordinate [label=below:$G$] (G) at ($(B)!2/3!(M)$);
\coordinate [label=above right:$J$] (J) at ($(A)!0.25!(M)$);
\path 
(intersection of A--G and B--J) coordinate (I) node[below]{$I$};
\draw (A)--(B)--(C)--(A) (C)--(D)--(A) (A)--(M);
\draw[dashed] (B)--(D) (M)--(B) (A)--(G) (B)--(J);
\fill (A) circle(1.5pt) (B) circle(1.5pt) (C) circle(1.5pt) (D) circle(1.5pt)
(G) circle(1.5pt) (I) circle(1.5pt) (J) circle(1.5pt) (M) circle(1.5pt);
\end{tikzpicture}
}
}
\end{ex}




%---Câu 12
\begin{ex}%[1H4H1-3] %[CTST - Lớp 11 - Ôn tập giữa học kì 1 - Đề 5]%[Sauluoi3105]
Cho hình chóp $S.ABCD$ có đáy $ABCD$ là hình bình hành tâm $O$. Gọi $I$ và $J$ lần lượt là trung điểm của $SA$ và $SB$. Khẳng định nào sau đây là \textbf{sai}?
\choice
{$IJCD$ là hình thang}
{$(SAB) \cap(IBC)=IB$}
{$(SBD) \cap(JCD)=JD$}
{\True $(IAC) \cap(JBD)=AO$}
\loigiai{
\immini{
Ta có $\heva{& O \in AC;AC \subset \in(IAC) \\& O \in BD;BD \subset(JBD)}$.\\
Suy ra $O \in(IAC) \cap(JBD)$ \quad (1)\\
Ta có $\heva{& S \in IA; IA \subset (I A C) \\& S \in JB; JB \subset (JBD)}$ .\\
Suy ra $S \in(IAC) \cap(JBD)$ \quad (2)\\
Từ (1) và (2) suy ra $(IAC) \cap(JBD)=SO$.
}{
\begin{tikzpicture}[scale=0.85]
\coordinate[label=below:$A$] (A) at (0,0);
\coordinate[label=below:$B$] (B) at (4,0);
\coordinate[label=below:$D$] (D) at (-150:3);
\coordinate [label=below:$C$] (C) at ($(B)+(D)-(A)$);
\coordinate [label=below:$O$] (O) at ($(C)!0.5!(A)$);
\coordinate [label=above:$S$] (S) at (0,4);
\coordinate [label=left:$I$] (I) at ($(S)!0.5!(A)$);
\coordinate [label=above right:$J$] (J) at ($(S)!0.5!(B)$);
\draw (S)--(D)--(C)--(S) (C)--(B)--(S);
\draw[dashed] (A)--(D) (S)--(A)--(B) (A)--(C) (B)--(D)--(J)--(I)--(C);
\fill (A) circle(1.5pt) (B) circle(1.5pt) (C) circle(1.5pt) (S) circle(1.5pt) (D) circle(1.5pt)
(O) circle(1.5pt) (I) circle(1.5pt) (J) circle(1.5pt);
\end{tikzpicture}
}
}
\end{ex}
\Closesolutionfile{ans}

\indapan{6}{ans-ABCD}

\cauds

\Opensolutionfile{ans}[ans-DS]

%---Câu 13
\begin{ex}%[1D1H2-4]%[CTST - Lớp 11 - Ôn tập giữa học kì 1 - Đề 5]%[Sauluoi3105]
Cho góc $\alpha$ thỏa mãn $\dfrac{-\pi}{2}<\alpha<0$ và $\cot \alpha=-3$.
\choiceTF
{$\sin \alpha>0$}
{\True $\sin \alpha=-\dfrac{\sqrt{10}}{10}$}
{\True $\dfrac{\cos \alpha-\sin \alpha}{\cos ^3 \alpha+3 \sin ^3 \alpha+2 \cos \alpha}=\dfrac{10}{21}$}
{\True $\left[\tan \dfrac{17 \pi}{4}+\tan \left(\dfrac{7 \pi}{2}-\alpha\right)\right]^2+\left[\cot \dfrac{13 \pi}{4}+\cot (7 \pi-\alpha)\right]^2=20$}
\loigiai
{
% Lời giải chung
\begin{itemchoice}
\itemch Sai. Do $\dfrac{-\pi}{2}<\alpha<0$ nên $\sin \alpha<0$.
\itemch Đúng.\\
Từ hệ thức $1+\cot ^2 \alpha=\dfrac{1}{\sin ^2 \alpha}$, suy ra $\sin \alpha=-\sqrt{\dfrac{1}{1+\cot ^2 \alpha}}=-\dfrac{\sqrt{10}}{10}$.
\itemch Đúng.
Xét biểu thức $P=\dfrac{\cos \alpha-\sin \alpha}{\cos ^3 \alpha+3 \sin ^3 \alpha+2 \cos \alpha}$.\\
Chia cả tử và mẫu cho $\sin ^3 \alpha$ ta được
$$
P=\dfrac{\dfrac{\cos \alpha}{\sin ^3 \alpha}-\dfrac{1}{\sin ^2 \alpha}}{\dfrac{\cos ^3 \alpha}{\sin ^3 \alpha}+3+2 \dfrac{\cos \alpha}{\sin ^3 \alpha}}=\dfrac{\cot \alpha \cdot\left(1+\cot ^2 \alpha\right)-\left(1+\cot ^2 \alpha\right)}{\cot ^3 \alpha+3+2 \cdot \cot \alpha \cdot\left(1+\cot ^2 \alpha\right)}.
$$
Thay $\cot \alpha=-3$ vào $P$ ta được $P=\dfrac{10}{21}$.
\itemch Đúng.
Ta có $\tan \dfrac{17 \pi}{4}=\tan \left(\dfrac{\pi}{4}+4 \pi\right)=\tan \dfrac{\pi}{4}=1$; $\tan \left(\dfrac{7 \pi}{2}-\alpha\right)=\cot \alpha$.\\
Và $\cot \dfrac{13 \pi}{4}=\cot \left(\dfrac{\pi}{4}+3 \pi\right)=\cot \dfrac{\pi}{4}=1$; $\cot (7 \pi-\alpha)=-\cot \alpha$.\\
Suy ra 
\allowdisplaybreaks{
\begin{eqnarray*}
&& \left[\tan \dfrac{17 \pi}{4}+\tan \left(\dfrac{7 \pi}{2}-\alpha\right)\right]^2+\left[\cot \dfrac{13 \pi}{4}+\cot (7 \pi-\alpha)\right]^2\\
&=&(1+\cot \alpha)^2+(1-\cot \alpha)^2=2+2 \cot ^2 \alpha=2+2 \cdot(-3)^2=20.
\end{eqnarray*}
}
\end{itemchoice}
}
\end{ex}


%---Câu 14
\begin{ex}%[1D1H3-2]%[CTST - Lớp 11 - Ôn tập giữa học kì 1 - Đề 5]%[Sauluoi3105]
Biết $\sin x=\dfrac{3}{5}$ và $\tan y=\dfrac{7}{24}$, với $x$ và $y$ là các góc nhọn.
\choiceTF
{\True $\tan x=\dfrac{3}{4}$}
{$\cos (x+y)=\dfrac{117}{125}$}
{$\sin (x-y)=\dfrac{33}{65}$}
{$\tan (x+y)=\dfrac{31}{17}$}
\loigiai
{
% Lời giải chung
\begin{itemchoice}
\itemch Đúng. Ta có $\sin ^2 x+\cos ^2 x=1 \Rightarrow \cos x=\sqrt{1-\sin ^2 x}=\sqrt{1-\left(\dfrac{3}{5}\right)^2}=\dfrac{4}{5}$.\\
Suy ra $\tan x=\dfrac{3}{4}$.
\itemch Sai.
Ta có $1+\tan ^2 y=\dfrac{1}{\cos ^2 y} \Rightarrow \cos y=\dfrac{24}{25} \Rightarrow \sin y=\dfrac{7}{25}$.
$\cos (x+y)=\cos x \cos y-\sin x \sin y=\dfrac{4}{5} \cdot \dfrac{24}{25}-\dfrac{3}{5} \cdot \dfrac{7}{25}=\dfrac{3}{5}$.
\itemch Sai. Ta có $\sin (x-y)=\sin x \cos y-\cos x \sin y=\dfrac{3}{5} \cdot \dfrac{24}{25}-\dfrac{4}{5} \cdot \dfrac{7}{25}=\dfrac{44}{125}$.
\itemch Sai. Ta có $\tan (x+y)=\dfrac{\tan x+\tan y}{1-\tan x \tan y}=\dfrac{\dfrac{3}{4}+\dfrac{7}{24}}{1-\dfrac{3}{4} \cdot \dfrac{7}{24}}=\dfrac{4}{3}$
\end{itemchoice}
}
\end{ex}


%---Câu 15
\begin{ex}%[1D1H5-3] %[CTST - Lớp 11 - Ôn tập giữa học kì 1 - Đề 5]%[Sauluoi3105]
Cho phương trình $\cos \left(2 x-\dfrac{\pi}{6}\right)=-\dfrac{\sqrt{2}}{2}$.
\choiceTF
{\True Phương trình có nghiệm $\hoac{& x=\dfrac{11 \pi}{24}+k \pi \\& x=\dfrac{-7 \pi}{24}+k \pi}(k \in \mathbb{Z})$}
{Phương trình có nghiệm âm lớn nhất bằng $\dfrac{-13 \pi}{24}$}
{Trên khoảng $(0 ; \pi)$ phương trình đã cho có $3$ nghiệm}
{\True Tổng các nghiệm của phương trình trong khoảng $(0 ; \pi)$ bằng $\dfrac{7 \pi}{6}$}
\loigiai{
\begin{itemchoice}
\itemch Đúng.
Ta có 
\allowdisplaybreaks{
\begin{eqnarray*}
\cos \left(2 x-\dfrac{\pi}{6}\right)=-\dfrac{\sqrt{2}}{2} 
&\Leftrightarrow& \cos \left(2x-\dfrac{\pi}{6}\right)=\cos \left(\dfrac{3 \pi}{4}\right) \\
&\Leftrightarrow& \hoac{& 2x-\dfrac{\pi}{6}=\dfrac{3 \pi}{4}+k 2 \pi \\& 2 x-\dfrac{\pi}{6}=\dfrac{-3 \pi}{4}+k 2 \pi.}\\
&\Leftrightarrow& \hoac{& x=\dfrac{11 \pi}{24}+k \pi \\ & x=\dfrac{-7 \pi}{24}+k \pi},(k \in \mathbb{Z}).
\end{eqnarray*}
}
\noindent
Ta có $x=\dfrac{11 \pi}{24}+k \pi<0 \Leftrightarrow k<-\dfrac{11}{24}$.\\
Vậy nghiệm âm lớn nhất trong trường hợp này ứng với $k=-1$ là $\dfrac{-13 \pi}{24}$.\\
Ta có $x=\dfrac{-7 \pi}{24}+k \pi<0 \Leftrightarrow k<\dfrac{7}{24}$.\\
Vậy nghiệm âm lớn nhất trong trường hợp này ứng với $k=0$ là $\dfrac{-7 \pi}{24}$.
\itemch Sai. Do đó phương trình có nghiệm âm lớn nhất bằng $\dfrac{-7 \pi}{24}$.
\itemch Sai. 
Ta có $x=\dfrac{11 \pi}{24}+k \pi 
\Rightarrow 0<\dfrac{11 \pi}{24}+k \pi<\pi 
\Leftrightarrow \dfrac{-11}{24}<k<\dfrac{13}{24}$.\\ 
Mà $k \in \mathbb{Z} \Rightarrow k=0$. Suy ra $x=\dfrac{11 \pi}{24}$.\\
Tương tự $x=\dfrac{-7 \pi}{24}+k \pi$ chỉ có $x=\dfrac{17 \pi}{24} \in(0 ; \pi)$.
\itemch Đúng. Vì $\dfrac{11 \pi}{24}+\dfrac{17 \pi}{24}=\dfrac{7 \pi}{6}$.
\end{itemchoice}
}
\end{ex}


%---Câu 16
\begin{ex}%[1H4H1-4]%[CTST - Lớp 11 - Ôn tập giữa học kì 1 - Đề 5]%[Sauluoi3105]
Cho hình chóp $S.ABCD$ với $M$ là một điểm trên cạnh $SC$, $N$ là một điểm trên cạnh $BC$. Gọi $O=AC \cap BD$ và $K=AN \cap CD$.
\choiceTF
{\True $S O$ là giao tuyến của hai mặt phẳng $(SAC)$ và $(SBD)$}
{\True Giao điểm của đường thẳng $AM$ và mặt phẳng $(SBD)$ là điểm nằm trên $SO$}
{\True $K M$ là giao tuyến của hai mặt phẳng $(AMN)$ và $(SCD)$}
{\True Giao điểm của đường thẳng $SD$ và mặt phẳng $(AMN)$ là điểm nằm trên $KM$}
\loigiai{
\begin{itemchoice}
\itemch Đúng.
\immini{
Dễ thấy $S$ là điểm chung của hai mặt phẳng $(S A C)$ và $(S B D)$.\\
Trong mặt phẳng $(A B C D)$, gọi $O=A C \cap B D$.\\
Vì $\heva{&  O \in A C; A C \subset(S A C) \\& O \in BD; BD \subset(S B D)}$.\\
Suy ra $O \in(S A C) \cap(S B D)$.\\
Vậy $S O=(S A C) \cap(S B D)$.
}{
\begin{tikzpicture}[scale=0.85]
\coordinate[label=below:$A$] (A) at (0,0);
\coordinate[label=below:$B$] (B) at (-130:3);
\coordinate[label=below left:$C$] (C) at (-160:5);
\coordinate [label=left:$D$] (D) at (-6.5,0);
\coordinate [label=below:$N$] (N) at ($(C)!0.6!(B)$);
\coordinate [label=above:$S$] (S) at (-1,4.5);
\coordinate [label=left:$M$] (M) at ($(S)!0.45!(C)$);
\path 
(intersection of A--C and B--D) coordinate (O) node[right]{$O$}
(intersection of C--D and A--N) coordinate (K) node[below]{$K$}
(intersection of K--M and S--D) coordinate (H) node[above left]{$H$}
(intersection of S--O and A--M) coordinate (P) node[below]{$P$};
\draw (S)--(B)--(A)--(S) (S)--(D)--(C)--(S) (N)--(B) (C)--(K)--(H) (K)--(N);
\draw[dashed] (A)--(D)--(B) (A)--(C)--(N) (S)--(O) (A)--(N) (M)--(A);
\fill (A) circle(1.5pt) (B) circle(1.5pt) (C) circle(1.5pt) (S) circle(1.5pt) (D) circle(1.5pt)
(O) circle(1.5pt) (K) circle(1.5pt) (H) circle(1.5pt) (P) circle(1.5pt) (N) circle(1.5pt) (M) circle(1.5pt);
\end{tikzpicture}
}
\itemch Đúng. \\
Trong mặt phẳng $(SAC)$, gọi $P=AM \cap SO$.\\
Ta có $\heva{& P \in AM \\& P \in SO; SO \subset(SBD)}$.\\
Suy ra $P=AM \cap(SBD)$.
\itemch Đúng.\\
Xét mặt phẳng phụ $(SCD)$ chứa $SD$. Ta tìm giao tuyến của $(AMN)$ và $(SCD)$.\\
Trong mặt phẳng $(ABCD)$, gọi $K=AN \cap CD$.\\ 
Khi đó: $\heva{& K \in AN; AN \subset(AMN) \\& K \in CD;CD \subset(SCD)} 
\Rightarrow K \in(AMN) \cap(SCD)$.\\
Mặt khác $M \in SC;SC \subset (SCD) \Rightarrow M \in(SCD) \Rightarrow M \in(SCD) \cap(AMN)$.\\
Vậy $KM=(SCD) \cap(AMN)$.
\itemch Đúng.\\
Trong mặt phẳng $(SCD)$ gọi $H=KM \cap SD$.\\
Ta có $\heva{& H \in SD \\& H \in KM;KM \subset(AMN)} \Rightarrow H=SD \cap(AMN)$.
\end{itemchoice}
}
\end{ex}



\Closesolutionfile{ans}

\indapan{3}{ans-DS}

\caukq

\Opensolutionfile{ans}[ans-KQ]
%---Câu 17
\begin{ex}%[1D1V5-3] %[CTST - Lớp 11 - Ôn tập giữa học kì 1 - Đề 5]%[Sauluoi3105]
Giá trị nhỏ nhất của hàm số $y=\cos ^2 x+2 \sin x+2$ bằng bao nhiêu?
\shortans{$0$}
\loigiai{
Ta có 
\allowdisplaybreaks{
\begin{eqnarray*}
y
&=&\cos ^2 x+2 \sin x+2\\
&=& \left(1-\sin ^2 x\right)+2 \sin x+2\\
&=&-\sin ^2 x+2 \sin x+3=-(\sin x-1)^2+4.
\end{eqnarray*}
}
Vì 
\allowdisplaybreaks{
\begin{eqnarray*}
&& -1 \leq \sin x \leq 1 \Rightarrow-2 \leq \sin x-1 \leq 0 \Rightarrow 4 \geq(\sin x-1)^2 \geq 0 \\
& \Rightarrow& -4 \leq-(\sin x-1)^2 \leq 0 \Rightarrow 0 \leq-(\sin x-1)^2+4 \leq 4 \text { hay } 0 \leq y \leq 4.
\end{eqnarray*}
}
Do đó $\min y =0$ khi $\sin x=-1 \Leftrightarrow x=-\dfrac{\pi}{2}+k 2 \pi, k \in \mathbb{Z}$.
}
\end{ex}


%---Câu 18
\begin{ex}%[1D1V1-3]%[CTST - Lớp 11 - Ôn tập giữa học kì 1 - Đề 5]%[Sauluoi3105]
Một đồng hồ treo tường, có kim giờ dài $8$ cm , kim phút dài $10$ cm . Tổng quãng đường mũi kim phút, kim giờ đi được trong $40$ phút bằng bao nhiêu cm ?
\shortans{$44{,}7$}
\loigiai{
Trong 40 phút, kim phút quay được một góc là $\dfrac{4 \pi}{3} \mathrm{rad}$.\\
Quãng đường kim phút đi được là $S_1=\dfrac{4 \pi}{3} \cdot 10=\dfrac{40 \pi}{3}(\mathrm{~cm})$.\\
Trong 40 phút kim giờ quay được một góc là $\dfrac{\pi}{9}$ rad.\\
Quãng đường kim giờ đi được là $S_2=\dfrac{\pi}{9} \cdot 8=\dfrac{8 \pi}{9}(\mathrm{~cm})$.\\
Vậy tổng quãng đường cần tìm là $S=S_1+S_2=\dfrac{40 \pi}{3}+\dfrac{8 \pi}{9}=\dfrac{128}{9} \pi \approx 44,7(\mathrm{~cm})$.
}
\end{ex}


%---Câu 19
\begin{ex}%[1D1V4-6]%[CTST - Lớp 11 - Ôn tập giữa học kì 1 - Đề 5]%[Sauluoi3105]
Mỗi lần tim đập, huyết áp của bạn sẽ tăng và sau đó huyết áp của bạn sẽ giảm khi tim nghỉ giữa các nhịp đập. Huyết áp tối đa và tối thiểu lần lượt được gọi là huyết áp tâm thu và huyết áp tâm trương. Chỉ số huyết áp thường được viết dưới dạng tỉ lệ tâm thu/tâm trương, một người có huyết áp $120 / 80$ được coi là bình thường. Huyết áp của một người được thể hiện qua hàm số $\partial(t)=115+25 \sin (160 \pi t)$, trong đó $\partial(t)$ là huyết áp tính bằng $m m H g$ tại thời điểm $t \geq 0$ tính bằng phút. Độ lệch chỉ số huyết áp của người này so với người bình thường bằng bao nhiêu? .
\shortans{$0{,}06$}
\loigiai{
Ta có
\allowdisplaybreaks{
\begin{eqnarray*}
&& -1 \leq \sin (160 \pi t) \leq 1, \forall t \\
& \Leftrightarrow& 90 \leq 115+25 \sin (160 \pi t) \leq 140, \forall t \\
& \Leftrightarrow& 90 \leq \partial(t) \leq 140, \forall t.
\end{eqnarray*}
}
Khi đó $\max \partial(t)=140$ khi 
$$\sin (160 \pi t)=1 
\Leftrightarrow 160 \pi t=\dfrac{\pi}{2}+k 2 \pi,(k \in \mathbb{Z})
\Leftrightarrow t=\dfrac{1}{320}+\dfrac{k}{80},(k \in \mathbb{Z}).$$
Vì $t \geq 0 \text { nên } t=\dfrac{1}{320}+\dfrac{k}{80},(k \in \mathbb{N})$.\\
Tương tự $\min \partial(t)=90$ khi 
$$\sin (160 \pi t)=-1 
\Leftrightarrow 160 \pi t=-\dfrac{\pi}{2}+k 2 \pi,(k \in \mathbb{Z})
\Leftrightarrow t=-\dfrac{1}{320}+\dfrac{k}{80},(k \in \mathbb{Z})$$
Vì $t \geq 0 \text { nên } t=-\dfrac{1}{320}+\dfrac{k}{80},\left(k \in \mathbb{N}^*\right)$.\\
Khi đó, chỉ số huyết áp của người này là $140 / 90=\dfrac{140}{90}$.
Độ lệch cần tìm $\dfrac{140}{90}-\dfrac{120}{80} \approx 0,06$.
}
\end{ex}







%---Câu 20
\begin{ex}% [1D1V3-2]%[CTST - Lớp 11 - Ôn tập giữa học kì 1 - Đề 5]%[Sauluoi3105]
\immini{
Một hộ dân lắp đặt một camera trên cửa nhà để có thể quan sát được cổng và sân nhà. Cửa nhà tại điểm $A$ và cổng tại điểm $C, A C=6(m)$. Camera được đặt tại vị trí $B$, hình chiếu vuông góc của $B$ lên mặt đất trùng với điểm $A, A B=4(m)$. Để đảm bảo an ninh, camera có thể quan sát được điểm $E$ trên cổng, $C E$ song song $A B, C E=3(m)$ và vị trí điểm $D$ thuộc đoạn $A C$, cách điểm $A$ một khoảng $A D=0,5(m)$. Số đo của $\widehat{D B E}$ bằng bao nhiêu độ.
}{
\begin{tikzpicture}[scale=0.85]
\coordinate[label=below:$A$] (A) at (0,0);
\coordinate[label=left:$C$] (C) at (-6,0);
\coordinate[label=above:$E$] (E) at ($(C)+(0,3)$);
\coordinate[label=above:$B$] (B) at ($(A)+(0,5.5)$);
\coordinate [label=below:$D$] (D) at ($(A)!1/4!(C)$);
\draw (A)--(B)--(E)--(C)--(A) (B)--(D);
\fill (A) circle(1.5pt) (B) circle(1.5pt) (C) circle(1.5pt) (E) circle(1.5pt) (D) circle(1.5pt);
\end{tikzpicture}
}
\shortans{$73{,}4$}
\loigiai{
\begin{itemize}
\item \textbf{Cách 1:}
\immini{
Gọi $F$ là hình chiếu vuông góc của $E$ lên $AB$. Đặt góc $\widehat{A B E}=\alpha$; $\widehat{A B D}=\beta$.\\
Ta có 
$\heva{& EF=AC=6\,(\mathrm{m})\\ & BF=AB-AF=AB-CE=1\,(\mathrm{m}).}$\\
Xét $\triangle B E F$ vuông tại $F$, 
$$B E=\sqrt{B F^2+F E^2}=\sqrt{6^2+1^2}=\sqrt{37}(\mathrm{m}).$$
Áp dụng hệ thức lượng trong tam giác vuông ta có 
$$\sin \alpha=\dfrac{E F}{B E}=\dfrac{6}{\sqrt{37}} ; \cos \alpha=\dfrac{B F}{B E}=\dfrac{1}{\sqrt{37}}.$$
}{
\begin{tikzpicture}[scale=0.75]
\coordinate[label=below:$A$] (A) at (0,0);
\coordinate[label=left:$C$] (C) at (-6,0);
\coordinate[label=above:$E$] (E) at ($(C)+(0,3)$);
\coordinate[label=above:$B$] (B) at ($(A)+(0,5.5)$);
\coordinate [label=below:$D$] (D) at ($(A)!0.35!(C)$);
\coordinate[label=right:$F$] (F) at ($(A)+(E)-(C)$);
\foreach \x/\y/\z in {E/B/A}{
          \path pic[draw,fill=orange!35,angle radius=25pt]{angle= \x--\y--\z};
        }
\foreach \x/\y/\z in {D/B/A}{
          \path pic[draw,fill=blue!35,angle radius=18pt]{angle= \x--\y--\z};
        }
\draw (A)--(B)--(E)--(C)--(A) (B)--(D) (E)--(F);
\fill (A) circle(1.5pt) (B) circle(1.5pt) (C) circle(1.5pt) (E) circle(1.5pt) (D) circle(1.5pt)(F) circle(1.5pt);
\draw (B) node[xshift=-1cm,yshift=-0.8cm]{$\alpha$};
\draw (B) node[xshift=-0.2cm,yshift=-1.2cm]{$\beta$};
\end{tikzpicture}
}
Xét $\triangle A B D$ vuông tại $A$, $BD=\sqrt{A B^2+A D^2}=\sqrt{4^2+\left(\dfrac{1}{2}\right)^2}=\dfrac{\sqrt{65}}{2}$.\\ 
Áp dụng hệ thức lượng trong tam giác vuông ta có 
$\heva{& \sin \beta=\dfrac{A D}{B D}=\dfrac{1}{\sqrt{65}}\\& \cos \beta=\dfrac{A B}{B D}=\dfrac{8}{\sqrt{65}}.}$\\
Áp dụng công thức cộng ta có 
\allowdisplaybreaks{
\begin{eqnarray*}
&& \sin \widehat{D B E}=\sin (\alpha-\beta)=\sin \alpha \cdot \cos \beta-\cos \alpha \cdot \sin \beta\\
&\Leftrightarrow& \sin \widehat{D B E}=\dfrac{6}{\sqrt{37}} \cdot \dfrac{8}{\sqrt{65}}-\dfrac{1}{\sqrt{37}} \cdot \dfrac{1}{\sqrt{65}}=\dfrac{47}{\sqrt{2405}} \Rightarrow \widehat{D B E} \approx 73{,}4^{\circ}.
\end{eqnarray*}
}
\item \textbf{Cách 2:}
Ta có 
$\heva{& \tan \widehat{A B E}=\tan \alpha=\dfrac{E F}{B F}=6\\&  \tan \widehat{A B D}=\tan \beta=\dfrac{A D}{A B}=\dfrac{1}{8}.}$
Suy ra $\tan \widehat{B D E}=\tan (\alpha-\beta)=\dfrac{\tan \alpha-\tan \beta}{1+\tan \alpha \cdot \tan \beta}=\dfrac{47}{14} \Rightarrow \widehat{B D E} \simeq 73{,}4^{\circ}.$
\end{itemize}
}
\end{ex}


%---Câu 21
\begin{ex}% [1D1V3-2]%[CTST - Lớp 11 - Ôn tập giữa học kì 1 - Đề 5]%[Sauluoi3105]
Cho tam giác $A B C$ thỏa mãn $\sin A=\dfrac{\sin B+\sin C}{\cos B+\cos C}$. Số đo của góc $A$ bằng bao nhiêu độ?
\shortans{$90$}
\loigiai{
\begin{itemize}
\item \textbf{Cách 1:}\\
Ta có: $0^{\circ}<A<180^{\circ} \Rightarrow 0^{\circ}<\dfrac{A}{2}<90^{\circ} \Rightarrow \cos \dfrac{A}{2}>0$.\\ 
Khi đó
\allowdisplaybreaks{
\begin{eqnarray*}
\sin A=\dfrac{\sin B+\sin C}{\cos B+\cos C} 
&\Leftrightarrow&  \sin A=\dfrac{2 \sin \left(\dfrac{B+C}{2}\right) \cdot \cos \left(\dfrac{B-C}{2}\right)}{2 \cos \left(\dfrac{B+C}{2}\right) \cdot \cos \left(\dfrac{B-C}{2}\right)}\\
&\Leftrightarrow& \sin A=\dfrac{\sin \left(\dfrac{B+C}{2}\right)}{\cos \left(\dfrac{B+C}{2}\right)}
\Leftrightarrow \sin A=\dfrac{\cos \dfrac{A}{2}}{\sin \dfrac{A}{2}}\\ 
&\Leftrightarrow& 2 \sin \dfrac{A}{2} \cos \dfrac{A}{2}=\dfrac{\cos \dfrac{A}{2}}{\sin \dfrac{A}{2}}.\\
&\Leftrightarrow& 2 \sin ^2 \dfrac{A}{2}=1 \Leftrightarrow \cos A=0 \Leftrightarrow A=90^{\circ}.
\end{eqnarray*}
}
\item \textbf{Cách 2:}
Ta có 
\allowdisplaybreaks{
\begin{eqnarray*}
\sin A=\dfrac{\sin B+\sin C}{\cos B+\cos C} 
&\Leftrightarrow& \sin A(\cos B+\cos C)=\sin B+\sin C\\
&\Leftrightarrow& \dfrac{a}{2 R}\left(\dfrac{c^2+a^2-b^2}{2 c a}+\dfrac{a^2+b^2-c^2}{2 a b}\right)=\dfrac{b+c}{2R}\\ &\Leftrightarrow& b\left(c^2+a^2-b^2\right)+c\left(a^2+b^2-c^2\right)=2 b^2 c+2 c^2 b\\
&\Leftrightarrow& (b+c)\left(b^2+c^2\right)-a^2(b+c)=0 \\
&\Leftrightarrow& (b+c)\left(b^2+c^2-a^2\right)=0 \Leftrightarrow b^2+c^2=a^2.
\end{eqnarray*}
}
Suy ra tam giác $A B C$ vuông tại $A$ hay góc $A=90^{\circ}$.
\end{itemize}
}
\end{ex}


%---Câu 22
\begin{ex}%[1H4V1-4]%[CTST - Lớp 11 - Ôn tập giữa học kì 1 - Đề 5]%[Sauluoi3105]
Cho hình chóp $S.ABCD$ có đáy là hình bình hành, $M$ là trung điểm của $SC$. Gọi $I$ là giao điểm của đường thẳng $AM$ với mặt phẳng $(SBD)$. Biết $AI=k\cdot IM$. Giá trị của $k$ bằng bao nhiêu?
\shortans{$2$}
\loigiai{
\immini{
Trong mặt phẳng $(ABCD)$, gọi $O=AC \cap BD$. Khi đó $(SAC) \cap(SBD)=SO$.\\
Trong mặt phẳng $(SAC)$, gọi $I=AM \cap SO$. Hơn nữa $SO \subset(SBD)$ nên $I=AM \cap(S B D)$.\\
Đáy $ABCD$ là hình bình hành, $O=AC \cap BD$ nên $O$ là trung điểm của $AC$. Ta cũng có $M$ là trung điểm của $SC$ nên $AM$ và $S O$ là hai đường trung tuyến của $\triangle SAC$.\\ 
Do đó $I$ là trọng tâm của $\triangle SAC$. Theo tính chất của trọng tâm tam giác, ta có $IA=2IM$.\\ 
Vậy $k=2$.
}{
\begin{tikzpicture}[scale=0.85]
\coordinate[label=below:$A$] (A) at (0,0);
\coordinate[label=left:$B$] (B) at (-1.5,-2);
\coordinate [label=right:$D$] (D) at (4,0);
\coordinate[label=right:$C$] (C) at ($(B)+(D)-(A)$);
\coordinate [label=above:$S$] (S) at (1,3);
\coordinate[label=below:$O$] (O) at ($(A)!0.5!(C)$);
\coordinate[label=right:$M$] (M) at ($(S)!0.5!(C)$);
\coordinate[label=above left:$I$] (I) at ($(A)!0.5!(C)!1/3!(S)$);
\draw (S)--(B)--(C)--(S) (C)--(D)--(S);
\draw[dashed] (A)--(B) (A)--(S) (A)--(D) (A)--(C) (B)--(D) (S)--(O) (A)--(M);
\fill (A) circle(1.5pt) (B) circle(1.5pt) (C) circle(1.5pt) (S) circle(1.5pt) (D) circle(1.5pt)
(I) circle(1.5pt) (M) circle(1.5pt) (O) circle(1.5pt);
\end{tikzpicture}
}
}
\end{ex}





\Closesolutionfile{ans}

\indapan{6}{ans-KQ}