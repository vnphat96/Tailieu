\def\tenchude{CẤP SỐ CỘNG}
\setcounter{section}{5}
\setcounter{dang}{0}
\setcounter{ex}{0}
\setcounter{bt}{0}
\setcounter{vd}{0}
\section{Cấp số cộng}
\subsection{Tóm tắt lý thuyết}
\begin{tomtat}
	\subsubsection{Định nghĩa}
	Dãy số là cấp số cộng nếu mỗi một số hạng (kể từ số hạng thứ hai) đều bằng tổng của số hạng đứng ngay trước nó với một số không đổi $ d $.\\
	Dãy số $ (u_n) $ là cấp số cộng $ \Leftrightarrow u_{n+1}=u_n+d $, $ \forall n \in \mathbb{N}^* $.\\
	$ d $ là số không đổi, gọi là \textbf{\textit{công sai}} của cấp số cộng.
	\subsubsection{Tính chất}
	Nếu $ (u_n) $ là cấp số cộng thì kể từ số hạng thứ hai (trừ số hạng cuối nếu là cấp số cộng hữu hạn) đều là trung bình cộng của hai số hạng đứng kề nó trong dãy. Tức là $$u_k=\dfrac{u_{k-1}+u_{k+1}}{2}, (\forall k\ge 2, k \in \mathbb{N}^*).$$
	\subsubsection{Số hạng tổng quát}
	Nếu cấp số cộng $ (u_n) $ có số hạng đầu $ u_1 $ và công sai $ d $ thì số hạng tổng quát $ u_n $ được xác định bởi công thức $$u_n=u_1+(n-1)d \text{ với $n\ge 2$}.$$
	\subsubsection{Tổng $ n $ số hạng đầu tiên}
	Cho cấp số cộng $ (u_n) $. Tổng $ n $ số hạng đầu tiên của cấp số cộng kí hiệu là $ S_n=u_1+u_2+\ldots+u_n $.\\
	Khi đó $ S_n $ được tính theo công thức $$ S_n=\dfrac{n(u_1+u_n)}{2}=\dfrac{n}{2}\left[ 2u_1+(n-1)d\right]. $$
\end{tomtat}
\subsection{Các dạng toán thường gặp}
\begin{dang}{Nhận diện cấp số cộng, công sai $ d $, số hạng tổng quát $ u_n $}
	% Dựa theo định nghĩa của cấp số cộng, để nhận diện $ (u_n) $ là cấp số cộng $ \Leftrightarrow u_{n+1}=u_n+d $.\\
	% Khi đó công sai $ d=u_{n+1}-u_{n} $, $ \forall n \in \mathbb{N}^* $.
\end{dang}
\subsubsection{Ví dụ minh hoạ}
\begin{vd}%[NB]%[DCHT Toán 11 - KNTT -Lê Hải Phụng] %[1K2Y6-1]
	Dãy số hữu hạn nào là một cấp số cộng? Vì sao?
	\begin{listEX}[2]
		\item  $-2$, $1$, $4$, $7$, $10$, $13$, $16$.
		\item  $ 1 $, $ -2 $, $ -4 $, $ -6 $, $ -8 $.
	\end{listEX}
	\dapso{ Dãy số 1 là một cấp số cộng, dãy số 2 không là một cấp số cộng.}
	\loigiai{
		\begin{enumerate}
			\item Ta thấy $ u_2=u_1+3 $  do $ 1=(-2)+3 $.\\
			Vì $ u_k=u_{k-1}+d,\ \forall k\geq2$ $\left(\ 1=\left(-2\right)+3;4=1+3;7=4+3;10=7+3;13=10+3;16=13+3\right) $ nên dãy số đã cho là cấp số cộng. 
			\item Ta thấy $ u_2=u_1+(-3) $  do $-2=1+(-3)$.\\
			Vì $ {u_3\neq u}_2+(-3) $ bởi $ \left(\ -4\neq-2+\left(-3\right)\right)\ $ nên dãy số đã cho không là cấp số cộng.
		\end{enumerate}
	}
\end{vd}
\begin{vd}%[TH]%[DCHT Toán 11 - KNTT -Lê Hải Phụng] %[1K2B6-1]
	Trong các dãy số dưới đây, dãy số nào là cấp số cộng? 
	\begin{listEX}[2]
		\item  Dãy số $\left({a_n}\right)$ với ${a_n}=4n-3$;
		\item  Dãy số $\left({c_n}\right)$ với ${c_n}={2018^n}$.
	\end{listEX}
	\dapso{Dãy số 1 là một cấp số cộng, dãy số 2 không là một cấp số cộng.}
	\loigiai{	
		\begin{enumerate}
			\item Ta có $a_{n+1}=4(n+1)-3=4n+1$ nên $a_{n+1}-a_n=(4n+1)-(4n-3)=4$,$\forall n\ge 1.$.\\
			Do đó $(a_n)$ là cấp số cộng.
			\item Ta có $c_{n+1}=2018^{n+1}$ nên $c_{n+1}-c_n=2018^{n+1}-2018^n=2017\cdot 2018^n$ (phụ thuộc vào giá trị của $n$).\\ 
			Suy ra $(c_n)$ không phải là một cấp số cộng.
		\end{enumerate}	
	}
\end{vd}
\begin{vd}%[NB]%[DCHT Toán 11 - KNTT -Lê Hải Phụng] %[1K2Y6-1]
	Cho cấp số cộng $(u_n)$  có công thức số hạng tổng quát $u_n=3n+1$, $n\in\mathbb{N}^\ast$ . Tìm số hạng đầu $u_1$ và công sai $d$?
	\dapso{$u_1=4 $, $d=3$.}
	\loigiai{
		Từ công thức số hạng tổng quát, ta có $ u_1=4 $, $u_2=7$ suy ra $d=u_2-u_1=3$.
	}
\end{vd}

\begin{vd}%[TH]%[DCHT Toán 11 - KNTT -Lê Hải Phụng] %[1K2B6-1]
	Cho cấp số cộng $(u_n)$ với $u_1=3$, $u_2=9$. Công sai của cấp số cộng đã cho bằng bao nhiêu?
	\dapso{$ d=6 $}
	\loigiai{
		Cấp số cộng $(u_n)$ có số hạng tổng quát là $u_n=u_1+(n-1)d$ với $n \ge 2$.\\
		Suy ra $u_2=u_1+d \Leftrightarrow 9=3+d \Leftrightarrow d=6$.\\
		Vậy công sai của cấp số cộng đã cho là $6$.
	}
\end{vd}
\begin{vd}%[VD]%[DCHT Toán 11 - KNTT -Lê Hải Phụng] %[1K2K6-1]
	Tính số hạng đầu $u_1$ và công sai $d$ của một cấp số cộng biết $u_4=10$ và $u_7=19$.
	\dapso{$ u_1=1 $, $ d=3 $.}
	\loigiai{Ta có $ \heva{& u_4=10 \\ & u_7=19} \Leftrightarrow \heva{& u_1+3d=10 \\ & u_1+6d=19} \Leftrightarrow \heva{& u_1=1 \\ & d=3.}$}
\end{vd}

\begin{vd}%[TH]%[Dự án DCHT-11-KNTT]%[Dao-V- Thuy]%[1K2B5-1]
	Xác định số hạng tổng quát của cấp số cộng $(u_n),$ biết $\heva{&u_7=8\\ &d=2.}$
	\dapso{$u_n=2n-6$}
	\loigiai{
		Ta có
		\begin{equation*}
			\heva{&u_7=8\\ &d=2} \Leftrightarrow \heva{&u_1+6d=8\\&d=2} \Leftrightarrow \heva{&u_1=-4\\ &d=2.}
		\end{equation*}
		Vậy công thức tổng quát của cấp số cộng
		\begin{center}
			$u_n=-4+(n-1)2 \Leftrightarrow u_n=2n-6 $ với $n \geq 2.$
		\end{center}	
	}
\end{vd}

% \begin{vd}%[TH]%[Dự án DCHT-11-KNTT]%[Dao-V- Thuy]%[1K2B5-1]
% 	Tìm số hạng đầu và công sai của cấp số cộng $(u_n)$, biết $\heva{&u_1+u_5-u_3=10\\ &u_1+u_6=17.}$
% 	\dapso{$u_1=16$, $d=-3$}
% 	\loigiai{
% 		Ta có
% 		\begin{align*}
% 			\heva{&u_1+u_5-u_3=10\\ &u_1+u_6=17} &\Leftrightarrow \heva{&u_1+u_1+4d-(u_1+2d)=10\\ &u_1+u_1+5d=17}\\ & \Leftrightarrow\heva{&u_1+2d=10 \\ &2u_1+5d=17} \Leftrightarrow \heva{&u_1=16 \\ &d=-3.}
% 		\end{align*}
% 		Vậy $u_1=16$, $d=-3$.
% 	}
% \end{vd}

\begin{vd}%[TH]%[Dự án DCHT-11-KNTT]%[Dao-V- Thuy]%[1K2B5-1]
	Cho cấp số cộng $(u_n)$ với $\heva{&u_1=-9\\ &u_{n-1}=u_n-5}$. Tìm số hạng tổng quát của cấp số cộng $(u_n)$.
	\dapso{$u_n= 5n-14$}
	\loigiai{
		Từ công thức $u_{n-1}=u_n-5 \Leftrightarrow u_n= u_{n-1}+5$, suy ra $d=5$.\\
		Vậy công thức tổng quát của cấp số cộng $(u_n)$ là $u_n=-9 + 5(n-1)= 5n-14$.
	}
\end{vd}

\begin{vd}%[TH]%[Dự án DCHT-11-KNTT]%[Dao-V- Thuy]%[1K2B5-1]
	Cho cấp số cộng $(u_n)$ có $u_{20}=-52$ và $u_{51}=-145$. Hãy tìm số hạng tổng quát của cấp số cộng đó.
	\dapso{$u_n= -3n+8$}
	\loigiai{
		Ta có
		\begin{eqnarray*}
			\heva{&u_{20}=-52 \\&u_{51}=-145} &\Leftrightarrow& \heva{&u_1+19d=-52\\ &u_1+50d=-145} \Leftrightarrow \heva{&u_1=5 \\ &d= -3.}
		\end{eqnarray*}
		Vậy số hạng tổng quát cần tìm là $u_n= u_1+ (n-1)d= 5+(n-1) \cdot (-3)= -3n+8$.
	}
\end{vd}

\begin{vd}%[VD]%[Dự án DCHT-11-KNTT]%[Dao-V- Thuy]%[1K2B5-1]
	Tìm số hạng đầu và công sai của cấp số cộng $(u_n)$, biết
		\begin{listEX}[2]
			\item $\heva{&u_9=5u_2\\ &u_{13}=2u_6+5.}$
			\item $\heva{&u_1-u_3+u_5=10\\ &u_1+u_6=7.}$
		\end{listEX}
	\dapso{$u_1=3$, $d=4$; $u_1=36$, $d=-13$}
	\loigiai
	{
		\begin{enumerate}
			\item Ta có
			\begin{eqnarray*}
				\heva{&u_9=5u_2\\ &u_{13}=2u_6+5} &\Leftrightarrow& \heva{&u_1+8d= 5 \left( u_1+d \right) \\ &u_1+12d = 2 \left( u_1+5d\right) + 5}\\
				&\Leftrightarrow& \heva{&-4u_1+3d=0\\ &-u_1+2d=5} \Leftrightarrow \heva{&u_1=3 \\ &d=4.}
			\end{eqnarray*}
			Vậy $u_1=3$, $d=4$.
			\item Ta có 
			\begin{eqnarray*}
				\heva{&u_1-u_3+u_5=10\\ &u_1+u_6=7} &\Leftrightarrow& \heva{&u_1-\left( u_1+2d\right) + \left( u_1+4d\right) = 10\\ &u_1+ \left( u_1+5d\right) = 7}\\
				&\Leftrightarrow& \heva{&u_1+2d=10\\ &2u_1+5d=7} \Leftrightarrow \heva{&u_1=36 \\ &d=-13.}
			\end{eqnarray*}
			Vậy $u_1=36$, $d=-13$.
		\end{enumerate}
	}
\end{vd}

\begin{vd}%[VD]%[Dự án DCHT-11-KNTT]%[Dao-V- Thuy]%[1K2K5-1]
	Tìm số hạng đầu và công sai của cấp số cộng $(u_n)$, biết
		\begin{listEX}[2]
			\item $\heva{&-u_3+u_7=8\\ &u_2u_7=75.}$
			\item $\heva{&u_5=4u_3\\ &u_2u_6=-11.}$
		\end{listEX}
	\dapso{$\heva{&u_1=3\\ &d=2} \text{ hoặc } \heva{&u_1=-17\\ &d=2}$; $\heva{&u_1=-4\\ &d=3}$ hoặc $\heva{&u_1=4\\ &d=-3}$}
	\loigiai{
		\begin{enumerate}
			\item Ta có
			\begin{eqnarray*}
				\heva{&-u_3+u_7=8\\ &u_2u_7=75} &\Leftrightarrow& \heva{&-\left( u_1+2d\right) + \left( u_1+6d\right) = 8\\ &\left( u_1+d\right) \left( u_1+6d\right) = 75}\\
				&\Leftrightarrow& \heva{&4d=8\\ &u_1^2+7u_1d+6d^2=75}\\
				&\Leftrightarrow& \heva{&d=2\\ &u_1^2+14u_1-51=0}\\
				&\Leftrightarrow& \heva{&u_1=3\\ &d=2} \text{ hoặc } \heva{&u_1=-17\\ &d=2.}
			\end{eqnarray*}
			Vậy $\heva{&u_1=3\\ &d=2} \text{ hoặc } \heva{&u_1=-17\\ &d=2.}$
			\item Ta có 
			\begin{eqnarray*}
				\heva{&u_5=4u_3\\ &u_2u_6=-11} &\Leftrightarrow& \heva{&u_1+4d=4 \left( u_1+2d\right)\\ &\left( u_1+d\right) \left( u_1+5d\right)= -11}\\
				&\Leftrightarrow& \heva{&3u_1+4d=0 &(1)\\ &u_1^2+6du_1+5d^2=-11 &(2)}
			\end{eqnarray*}
			Từ $(1)$ suy ra $3u_1=-4d$. Thay vào $(2)$ ta được
			\begin{eqnarray*}
				9u_1^2+54du_1+45d^2=-99 &\Leftrightarrow& 16d^2 -72d^2+45d^2=-99\\
				&\Leftrightarrow& -11d^2=-99 \Leftrightarrow \hoac{&d=3\\ &d=-3.}
			\end{eqnarray*}
			Với $d=3$, ta có $u_1=-4$.\\
			Với $d=-3$, ta có $u_1=4$.\\
			Vậy $\heva{&u_1=-4\\ &d=3}$ hoặc $\heva{&u_1=4\\ &d=-3.}$
		\end{enumerate}
	}
\end{vd}

\subsubsection{Bài tập tự luận}
 

\begin{bt}%[NB]%[DCHT Toán 11 - KNTT -Lê Hải Phụng] %[1K2Y6-1]
	Trong các dãy số sau, dãy số nào là một cấp số cộng?
	\begin{listEX}[1]
		\item $1$, $-3$, $-7$, $-11$, $-15$, $\ldots$;
		\item $1$, $-2$, $-4$, $-6$, $-8,$ $\ldots$.
		\item $ \dfrac{1}{2} $, $0$, $-\dfrac{1}{2}$, $-1$, $-\dfrac{3}{2}$, $\ldots$
	\end{listEX}
	\dapso{1) và 3) là cấp số cộng.}
	\loigiai{Ta lần lượt đi kiểm tra: $ u_2-u_1=u_3-u_2=u_4-u_3=\ldots $?\\
		Xét từng dãy số thì ta thấy 1) và 3) là cấp số cộng. 
	}
\end{bt}

\begin{bt}%[NB]%[DCHT Toán 11 - KNTT -Lê Hải Phụng] %[1K2Y6-1]
	Trong các dãy số sau, dãy nào là cấp số cộng. Tìm số hạng đầu và công sai của cấp số cộng đó.
	\begin{listEX}[2]
		\item Dãy số $ (u_n) $ với $ u_n=19n-5 $;
		\item Dãy số $ (u_n) $ với $ u_n=n^2+n+1 $. 
	\end{listEX}
	\dapso{Dãy số 1) $ (u_n) $ là một cấp số cộng với số hạng đầu là $ u_1=19\cdot1-5=14 $ và công sai $ d=19 $. Dãy số 2) không là một cấp số cộng.}
	\loigiai{
		\begin{enumerate}
			\item Dãy số $ (u_n) $ với $ u_n=19n-5 $.\\
			Ta có $ u_{n+1}-u_n=19(n+1)-5-(19n-5)=19 $. Vậy $ (u_n) $ là một cấp số cộng với số hạng đầu là $ u_1=19\cdot1-5=14 $ và công sai $ d=19 $.
			\item Dãy số $ (u_n) $ với $ u_n=n^2+n+1 $.\\
			Ta có $ u_{n+1}-u_n=(n+1)^2+(n+1)+1-(n^2+n+1)=2n+2$ phụ thuộc vào $ n $. Vậy $ (u_n) $ không là một cấp số cộng.
	\end{enumerate}}
\end{bt}

\begin{bt}%[TH]%[DCHT Toán 11 - KNTT -Lê Hải Phụng] %[1K2B6-1]
	Cho cấp số cộng $\left(u_n\right)$ với $u_1=3$, $u_2=9$. Công sai của cấp số cộng đã cho bằng bao nhiêu?
	\dapso{Công sai của cấp số cộng đã cho là 6.}
	\loigiai{Cấp số cộng $(u_n)$ có số hạng tổng quát là
		$u_n=u_1+\left(n-1\right)d$ với $n \ge 2$
		(số hạng đầu $u_1$ và công sai $d$)\\
		Suy ra $ u_2=u_1+d\Leftrightarrow9=3+d\Leftrightarrow d=6 $.\\
		Vậy công sai của cấp số cộng đã cho là 6.
	}
\end{bt}


\begin{bt}%[TH]%[Dự án DCHT-11-KNTT]%[Dao-V- Thuy]%[1K2B5-1]
	Xác định công thức tổng quát của cấp số cộng $(u_n)$, biết $\heva{&u_{11}=5\\ &d=-6.}$
	\loigiai{
		Ta có
		\begin{equation*}
			\heva{&u_{11}=5\\ &d=-6} \Leftrightarrow \heva{&u_1+10d=5\\ &d=-6} \Leftrightarrow \heva{&u_1=65\\ &d=-6.}
		\end{equation*}
		Vậy công thức tổng quát của cấp số cộng:
		\begin{center}
			$u_n=65+(n-1).(-6) \Leftrightarrow u_n=-6n+71$  với $n \geq 2.$
		\end{center}	
	}
\end{bt}

\begin{bt}%[TH]%[Dự án DCHT-11-KNTT]%[Dao-V- Thuy]%[1K2B5-1]
	Tìm số hạng đầu và công sai của cấp số cộng $(u_n),$ biết $\heva{&u_2+u_5-u_3=10\\ &u_4+u_6=26.}$
	\loigiai{
		Ta có
		\begin{align*}
			\heva{&u_2+u_5-u_3=10\\ &u_4+u_6=26} 
			&\Leftrightarrow \heva{&u_1+d+u_1+4d-(u_1+2d)=10\\ &u_1+3d+u_1+5d=26}\\ 
			& \Leftrightarrow\heva{&u_1+3d=10 \\ &2u_1+8d=26}
			\Leftrightarrow \heva{&u_1=1 \\&d=3.}
		\end{align*}
		Vậy $u_1=1$, $d=3$.
	}
\end{bt}

\begin{bt}%[TH]%[Dự án DCHT-11-KNTT]%[Dao-V- Thuy]%[1K2B5-1]
	Tìm số hạng đầu và công sai của cấp số cộng, biết
		\begin{listEX}[3]
			\item $\heva{&u_7 = 27\\&u_{15} = 59.}$
			\item $\heva{&u_9 = 5u_2\\&u_{13} = 2u_6 + 5.}$
			\item $\heva{&u_2 + u_4 - u_6 = -7\\&u_8 - u_7 = 2u_4.}$
			\item $\heva{&u_3 - u_7 = -8\\&u_2 \cdot u_7 = 75.}$
			\item $\heva{&u_6 + u_7 = 60\\&u_4^2 + u_{12}^2 = 1170.}$
		\end{listEX}
	\loigiai{
		\begin{enumerate}
			\item Ta có $\heva{&u_7 = 27\\&u_{15} = 59} \Leftrightarrow \heva{&u_1 + 6d = 27\\&u_1 + 14d = 59} \Leftrightarrow \heva{&u_1 = 3\\&d = 4.}$\\
			Vậy số hạng đầu của cấp số cộng là $u_1 = 3$, công sai là $d = 4$.
			\item Ta có $\heva{&u_9 = 5u_2\\&u_{13} = 2u_6 + 5} \Leftrightarrow \heva{&u_1 + 8d = 5u_1 + 5d\\&u_1 + 12d = 2u_1 + 10d + 5} \Leftrightarrow \heva{&4u_1 - 3d = 0\\&-u_1 + 2d = 5} \Leftrightarrow \heva{&u_1 = 3\\&d = 4.}$\\
			Vậy số hạng đầu của cấp số cộng là $u_1 = 3$, công sai là $d = 4$.
			\item Ta có $\heva{&u_2 + u_4 - u_6 = -7\\&u_8 - u_7 = 2u_4} \Leftrightarrow \heva{&u_1 + d + u_1 + 3d - u_1 - 5d = -7\\&u_1 + 7d - u_1 - 6d = 2u_1 + 6d} \Leftrightarrow \heva{&u_1 - d = -7\\&2u_1 + 5d = 0} \Leftrightarrow \heva{&u_1 = -5\\&d = 2.}$\\
			Vậy số hạng đầu của cấp số cộng là $u_1 = -5$, công sai là $d = 2$.
			\item Ta có $\heva{&u_3 - u_7 = -8\\&u_2 \cdot u_7 = 75} \Leftrightarrow \heva{&u_1 + 2d -u_1 - 6d = -8\\&(u_1 + d)(u_1 + 6d) = 75} \Leftrightarrow \heva{&d = 2\\&u_1^2 + 14u_1 - 51 = 0} \Leftrightarrow \heva{&d = 2\\&\hoac{&u_1 = 3\\&u_1 = -17.}}$\\
			Vậy số hạng đầu của cấp số cộng là $u_1 = 3$, công sai là $d = 2$ hoặc $u_1 = -17$, $d = 2$.
			\item Ta có $\heva{&u_6 + u_7 = 60\\&u_4^2 + u_{12}^2 = 1170} \Leftrightarrow \heva{&2u_6 + d = 60&(1)\\&(u_6 - 2d)^2 + (u_6 + 6d)^2 = 1170.&(2)}$\\
			Từ (1), suy ra $d = 60 - 2u_6$, thay vào (2), ta có
			$$(5u_6 - 120)^2 + (360 - 11u_6)^2 = 1170 \Leftrightarrow 146u_6^2 - 9120u_6 + 142830 = 0 \,\, (\text{vô nghiệm}).$$ 
			Vậy không tồn tại cấp số cộng thỏa yêu cầu bài toán.
		\end{enumerate}
	}
\end{bt}
% \begin{bt}%[TH]%[DCHT Toán 11 - KNTT -Lê Hải Phụng] %[1K2B6-1]
% 	Tìm số hạng đầu tiên, công sai của cấp số cộng sau $ \heva{&u_5=19\\&u_9=35.}$
	
% 	\dapso{Số hạng đầu tiên $ u_1=3 $, công sai $ d=4 $.}
% 	\loigiai{Áp dụng công thức $ u_n=u_1+(n-1)d $ ta có $\heva{&u_5=19\\&u_9=35} \Leftrightarrow \heva{&u_1+4d=19\\&u_1+8d=35} \Leftrightarrow \heva{&u_1=3\\&d=4.}$\\
% 	Vậy số hạng đầu tiên $ u_1=3 $, công sai $ d=4 $.}
% \end{bt}

\begin{bt}%[VD]%[1K2K6-1]
	Cho cấp số cộng $ (u_n) $ thỏa mãn $ \heva{&u_2+u_4-u_6=-7\\&u_8+u_7=2u_4} $. Xác định số hạng đầu $ u_1 $ và công sai $ d $ cấp số cộng.        
	
	% \dapso{$ u_1=5 $, $ d=2 $.}
	\loigiai{
		Ta có $ \heva{&u_2+u_4-u_6=-7\\&u_8+u_7=2u_4} \Leftrightarrow \heva{& u_1+d+(u_1+3d)-(u_1+5d)=-7 \\ & u_1+7d-(u_1+6d)=2(u_1+3d)} \Leftrightarrow \heva{& u_1-d=-7 \\ & 2u_1+5d=0} \Leftrightarrow \heva{&u_1=-5\\&d=2.}$}
\end{bt}

\begin{bt}%[VD]%[DCHT Toán 11 - KNTT -Lê Hải Phụng] %[1K2K6-1]
Cho cấp số cộng $ (u_n) $ thỏa mãn $ \heva{&u_2-u_3+u_5=10\\&u_4+u_6=26} $. Xác định số hạng đầu $ u_1 $ và công sai $ d $ cấp số cộng.         
\dapso{$ u_1=1 $, $ d=3 $.}
\loigiai{Ta có $ \heva{&u_2-u_3+u_5=10\\&u_4+u_6=26} \Leftrightarrow \heva{& u_1+d-(u_1+2d)+u_1+4d=10 \\ & u_1+3d+u_1+5d=26} \Leftrightarrow \heva{& u_1+3d=10 \\ & u_1+4d=13} \Leftrightarrow \heva{u_1=1\\d=3.}$}
\end{bt}

\begin{bt}%[VDC]%[DCHT Toán 11 - KNTT -Lê Hải Phụng] %[1K2G6-1]
Tính số hạng đầu $ u_1 $ và công sai $d$ của một cấp số cộng biết $ \heva{&u_1+u_2+u_3=27\\&u_1^2+u_2^2+u_3^2=275} $

\dapso{$ u_1=5 $, $ d=4 $ hoặc $ u_1=13 $, $ d=-4 $.}
\loigiai{Ta có $ \heva{&u_1+u_2+u_3=27\\&u_1^2+u_2^2+u_3^2=275} \Leftrightarrow \heva{&u_2-d+u_2+u_2+d=27\\&(u_2-d)^2+u_2^2+(u_2+d)^2=275}\Leftrightarrow \heva{&u_2=9\\&3u_2^2+2d^2=275.}$\\
Thay $ u_2=9 $ vào $ 3u_2^2+2d^2=275 $ ta được $ d=4 $ hay $ d=-4 $.\\
Vậy $ u_1=5 $, $ d=4 $ hoặc $ u_1=13 $, $ d=-4 $.}
\end{bt}
\subsubsection{Câu hỏi trắc nghiệm}
\Opensolutionfile{ans}[ans/ans-1K2-2-Dang1]

\begin{ex}%[DCHT Toán 11 - KNTT -Lê Hải Phụng] %[1K2Y6-1]
Trong các dãy số sau, dãy số nào là một cấp số cộng?
\choice
{\True $ 1 $; $ -3 $; $ -7 $; $ -11 $; $ -15 $; $ \ldots $}
{$ 1 $; $ -3 $; $ -6 $; $ -9 $; $ -12 $; $ \ldots $}
{$ 1 $; $ -2 $; $ -4 $; $ -6 $; $ -8 $; $ \ldots $}
{$ 1 $; $ -3 $; $ -5 $; $ -7 $; $ -9 $; $ \ldots $}
\loigiai
{
	Ta lần lượt tính khoảng cách $ d $ các phần tử, ta thấy dãy số đáp án A có $ d= -4$.
}
\end{ex}
%Cau2
\begin{ex}%[DCHT Toán 11 - KNTT -Lê Hải Phụng] %[1K2Y6-1]
Dãy số nào sau đây \textbf{không} phải là cấp số cộng?
\choice
{$ -\dfrac{2}{3} $; $ -\dfrac{1}{3} $; $ 0 $; $ \dfrac{1}{3} $; $ \dfrac{2}{3} $; $ 1 $; $ \dfrac{4}{3} $}
{$ 15\sqrt{2} $; $ 12\sqrt{2} $; $ 9\sqrt{2} $; $ 6\sqrt{2} $}
{\True $ \dfrac{4}{5} $; $ 1 $; $ \dfrac{7}{5} $; $ \dfrac{9}{5} $; $ \dfrac{11}{5} $}
{$ \dfrac{1}{\sqrt{3}} $; $ \dfrac{2\sqrt{3}}{3} $; $ \sqrt{3} $; $ \dfrac{4\sqrt{3}}{3} $; $ \dfrac{5}{\sqrt{3}} $}
\loigiai
{
	Ta lần lượt tính khoảng cách $ d $ các phần tử, ta thấy dãy số trừ đáp án C có khoảng cách các phần tử không bằng nhau.
}
\end{ex}
%Cau3
\begin{ex}%[DCHT Toán 11 - KNTT -Lê Hải Phụng] %[1K2Y6-1]
Cho cấp số cộng $ (u_n) $ với $ u_1=2 $ và $ u_2=6 $. Công sai của cấp số cộng đã cho là	
\choice
{\True $ 4 $}
{$ -4 $}
{$ 8 $}
{$ 3 $}
\loigiai
{
	Ta có $ u_2=6 \Leftrightarrow 6=u_1+d \Leftrightarrow d=4 $.
}
\end{ex}
%Cau4
\begin{ex}%[DCHT Toán 11 - KNTT -Lê Hải Phụng] %[1K2Y6-1]
Cho cấp số cộng $ (u_n) $ với $ u_1=-3 $ và $ u_6=27 $. Công sai $ d $ của cấp số cộng đã cho là	
\choice
{$ d=7 $}
{$ d=5 $}
{$ d=8 $}
{\True $ d=6 $}
\loigiai
{
	Ta có $ u_6=27 \Leftrightarrow 27=u_1+5d \Leftrightarrow d=6 $.
}
\end{ex}
%Cau5
\begin{ex}%[DCHT Toán 11 - KNTT -Lê Hải Phụng] %[1K2B6-1]
Cho cấp số cộng $ (u_n) $ với $ u_{17}=33 $ và $ u_{33}=65 $. Công sai của cấp số cộng đã cho là	
\choice
{$ 1 $}
{$ 3 $}
{$ -2 $}
{\True $ 2 $}
\loigiai
{
	Gọi $ u_1 $, $ d $ lần lượt là số hạng đầu và công sai của cấp số cộng $ (u_n) $.\\
	Khi đó, ta có $ u_{17}=u_1+16d $, $ u_{33}=u_1+32d $\\
	Suy ra $ u_{33}-u_{17}=65-33 \Leftrightarrow 16d=32 \Leftrightarrow d=2 $\\
	Vậy công sai bằng $ 2 $.
}
\end{ex}
%Cau6
\begin{ex}%[DCHT Toán 11 - KNTT -Tên GV] %[1K2B6-1]
Cho cấp số cộng có $ u_1=-3 $ và $ d=4 $. Chọn khẳng định đúng trong các khẳng định sau.
\choice
{$ u_5=15 $}
{$ u_4=8 $}
{\True $ u_3=5 $}
{$ u_2=2 $}
\loigiai
{
	Ta có $ u_3=u_1+2d=-3+2\cdot4=5 $.
}
\end{ex}
%Cau7
\begin{ex}%[DCHT Toán 11 - KNTT -Tên GV] %[1K2Y6-1]
Cho cấp số cộng có $ u_1=11 $ và công sai $ d=4 $. Hãy tính $ u_{99} $.
\choice
{$ 401 $}
{\True $ 403 $}
{$ 402 $}
{$ 404 $}
\loigiai
{
	Ta có $ u_{99}=u_1+98d=11+98\cdot4=403 $.
}
\end{ex}
%Cau8
\begin{ex}%[DCHT Toán 11 - KNTT -Tên GV] %[1K2B6-1]
Một cấp số cộng $ (u_n) $ có $ u_{13}=8 $ và $ d=-3 $. Tìm số hạng thứ ba của cấp số cộng $ (u_n) $.
\choice
{$ 50 $}
{$ 28 $}
{\True $ 38 $}
{$ 44 $}
\loigiai
{
	Ta có $ u_{13}=u_1+12d \Leftrightarrow 8=u_1+12\cdot(-3)\Rightarrow u_1=44 \Rightarrow u_{3}=u_1+2d=44-6=38$.
}
\end{ex}
%Cau9
\begin{ex}%[DCHT Toán 11 - KNTT -Tên GV] %[1K2Y6-1]
Cho cấp số cộng $(u_n) $ có số hạng đầu $ u_1=2 $ và công sai $ d=4 $. Hãy tính giá trị $ u_{2019} $ bằng
\choice
{\True $ 8074 $}
{$ 4074 $}
{$ 8078 $}
{$ 4078 $}
\loigiai
{
	Ta có $ u_{2019}=u_1+2018d=2+2018\cdot 4=8074 $.
}
\end{ex}
%Cau10
\begin{ex}%[DCHT Toán 11 - KNTT -Tên GV] %[1K2K6-1]
Cho cấp số cộng $ (u_n) $ có số hạng tổng quát là $ u_n=3n-2 $. Tìm công sai $ d $ của cấp số cộng.
\choice
{\True $ d=3 $}
{$ d=2 $}
{$ d=-2 $}
{$ d=-3 $}
\loigiai
{
	Ta có $ u_{n+1}-u_n=3(n+1)-2-3n+2=3 $. Suy ra công sai $ d=3 $.
}
\end{ex}

\begin{ex}%[Dự án DCHT-11-KNTT]%[Dao-V- Thuy]%[1K2Y5-1]
	Cho cấp số cộng $(u_n)$ có số hạng đầu $u_1$ và công sai $d$. Công thức tìm số hạng tổng quát $u_n$ là 
	\choice
	{\True $u_n=u_1+(n-1)d$}
	{$u_n=u_1+nd$}
	{$u_n=u_1+(n+1)d$}
	{$u_n=nu_1+d$}
	\loigiai{
		Ta có $u_n=u_1+(n-1)d$.
	}
\end{ex}

\begin{ex}%[Dự án DCHT-11-KNTT]%[Dao-V- Thuy]%[1K2Y5-1]
	Cho cấp số cộng $(u_n)$ có $u_1=-3$ và $d=\dfrac{1}{2}$. Khẳng định nào sau đây đúng?
	\choice
	{$u_n=-3+\dfrac{1}{2}(n+1 )$}
	{$u_n=-3+\dfrac{1}{2}n-1$}
	{\True $u_n=-3+\dfrac{1}{2}(n-1)$}
	{$u_n=-3+\dfrac{1}{4}(n-1 )$}
	\loigiai {
		Ta có $\heva{
			&u_1=-3 \\
			& d=\dfrac{1}{2} \\
		}\xrightarrow{CTTQ} u_n=u_1+(n-1 )d=-3+\dfrac{1}{2}(n-1 )$.}
\end{ex}

\begin{ex}%[Dự án DCHT-11-KNTT]%[Dao-V- Thuy]%[1K2Y5-1]
	Cho cấp số cộng $\left(u_n\right)$ xác định bởi $u_n=2n+1$. Xác định số hạng đầu $u_1$ và công sai $d$ của cấp số cộng.
	\choice
	{$u_1=3$, $d=1$}
	{$u_1=1$, $d=1$}
	{\True $u_1=3$, $d=2$}
	{$u_1=1$, $d=2$}
	\loigiai{
		Ta có $u_1=2\cdot 1+1=3$ và $u_2=2\cdot 2+1=5$, nên $d=u_2-u_1=2$.
	}
\end{ex}

\begin{ex}%[Dự án DCHT-11-KNTT]%[Dao-V- Thuy]%[1K2B5-1]
	Cho cấp số cộng $\left(u_n\right)$ có $u_4=-12$, $u_{14}=18$. Tìm số hạng đầu $u_1$ và công sai $d$ của cấp số cộng $\left(u_n\right)$. 
	\choice 
	{$u_1=-20$, $d=-3$}
	{$u_1=-22$, $d=3$ }
	{\True $u_1=-21$, $d=3$}
	{$u_1=-21$, $d=-3$}
	\loigiai{
		Ta có $$\heva{&u_4=u_1+(4-1)d\\&u_{14}=u_1+(14-1)d} \Leftrightarrow \heva{&-12=u_1+3d\\&18=u_1+13d}\Leftrightarrow \heva{&u_1=-12\\&d=3.}$$
	}
\end{ex}

\begin{ex}%[Dự án DCHT-11-KNTT]%[Dao-V- Thuy]%[1K2B5-1]
	Tìm số hạng đầu và công sai của cấp số cộng $(u_n)$ thỏa mãn $\heva{&u_1+u_9=12\\&u_4-3u_2=1.}$
	\choice
	{$u_1=\dfrac{1}{2}$; $d=\dfrac{13}{8}$}
	{$u_1=-1$; $d=\dfrac{13}{8}$}
	{\True $u_1=-\dfrac{1}{2}$; $d=\dfrac{13}{8}$}
	{$u_1=-1$; $d=2$}
	\loigiai{Ta có: $\heva{&u_1+u_9=12\\&u_4-3u_2=1}\Leftrightarrow\heva{&u_1+(u_1+8d)=12\\&(u_1+3d)-3(u_1+d)=1}\Leftrightarrow\heva{&2u_1+8d=12\\&-2u_1=1}\Leftrightarrow\heva{&d=\dfrac{13}{8}\\&u_1=-\dfrac{1}{2}}$}
\end{ex}

\begin{ex}%[Dự án DCHT-11-KNTT]%[Dao-V- Thuy]%[1K2B5-1]
	Cho cấp số cộng $(u_n)$ có $u_4=-12$ và $u_{14} =18$. Khi đó, số hạng đầu tiên $u_1$ và công sai $d$ của cấp số cộng $(u_n)$ lần lượt là
	\choice
	{$u_1=-20$, $d=-3$}
	{$u_1=-22$, $d=3$}
	{\True $u_1=-21$, $d=3$}
	{$u_1=-21$, $d=-3$}
	\loigiai{Ta có: $\heva{&u_4=-12\\&u_{14}=18}\Leftrightarrow\heva{&u_1+3d=-12\\&u_1+13d=18}\Leftrightarrow\heva{&u_1=-21\\&d=3.}$}
\end{ex}

\begin{ex}%[Dự án DCHT-11-KNTT]%[Dao-V- Thuy]%[1K2B5-1]
	Cho cấp số cộng $(u_n )$ có các số hạng đầu lần lượt là $5;\,9;\,13;\,17;\ldots $. Tìm số hạng tổng quát $u_n$ của cấp số cộng.
	\choice
	{$u_n=5n+1$}
	{$u_n=5n-1$}
	{\True $u_n=4n+1$}
	{$u_n=4n-1$}
	\loigiai{
		Cấp số cộng đã cho có $u_1=5$, $ d=u_2-u_1=4 $. Suy ra $u_n=u_1+(n-1 )d=5+4(n-1 )=4n+1$.
		}
\end{ex}

\begin{ex}%[Dự án DCHT-11-KNTT]%[Dao-V- Thuy]%[1K2B5-1]
	Cho cấp số cộng $(u_n)$ có $u_3=15$ và $d=-2$. Tìm $u_n$.
	\choice
	{\True $u_n=-2n+21$}
	{$u_n=-\dfrac{3}{2}n+12$}
	{$u_n=-3n-17$}
	{$u_n=\dfrac{3}{2}{{n}^2}-4$}
	\loigiai {
		Ta có $\heva{ & 15=u_3=u_1+2d \\& d=-2}
		\Leftrightarrow \heva{&u_1=19 \\& d=-2}
		\Rightarrow u_n=u_1+(n-1 )d=-2n+21$.
		}
\end{ex}

\begin{ex}%[Dự án DCHT-11-KNTT]%[Dao-V- Thuy]%[1K2B5-1]
	Trong các dãy số được cho dưới đây, dãy số nào {\bf không} phải là cấp số cộng?
	\choice
	{$u_n=-4n+9$}
	{$u_n=-2n+19$}
	{$u_n=-2n-21$}
	{\True $u_n=-2^n+15$}
	\loigiai {
		Dãy số $u_n=-2^n+15$ không có dạng $an+b$ nên có không phải là cấp số cộng.}
\end{ex}

\begin{ex}%[Dự án DCHT-11-KNTT]%[Dao-V- Thuy]%[1K2B5-1]
	Cho cấp số cộng $(u_n)$ có $u_4=-12$ và $u_{14}=18$. Tìm số hạng đầu tiên $u_1$ và công sai $d$ của cấp số cộng đã cho.
	\choice
	{\True $u_1=-21$; $d=3$}
	{$u_1=-20$; $d=-3$}
	{$u_1=-22$; $d=3$}
	{$u_1=-21$; $d=-3$}
	\loigiai {
		Ta có 
		$\heva{&u_4=-12\\ &u_{14}=18} \Leftrightarrow \heva{
			&u_1+3d=-12\\
			&u_1+13d=18 \\
		}\Leftrightarrow \heva{
			&u_1=-21 \\
			& d=3. \\
		}$}
\end{ex}

\begin{ex}%[Dự án DCHT-11-KNTT]%[Dao-V- Thuy]%[1K2K5-1]
	Cho cấp số cộng $(u_n)$ thoả mãn $\heva{&u_2-u_3+u_5=10\\ &u_3+u_4=17}$. Số hạng đầu tiên và công sai của cấp số cộng đó lần lượt là
	\choice
	{\True $1$ và $3$}
	{$-3$ và $4$}
	{$4$ và $-3$}
	{$-4$ và $-3$}
	\loigiai{
		$\heva{&u_2-u_3+u_5=10\\ &u_3+u_4=17}\Leftrightarrow\heva{&(u_1+d)-(u_1+2d)+(u_1+4d)=10\\&(u_1+2d)+(u_1+3d)=17}\Leftrightarrow\heva{&u_1+3d=10\\&2u_1+5d=17}\Leftrightarrow\heva{&u_1=1\\&d=3.}$}
\end{ex}

\begin{ex}%[Dự án DCHT-11-KNTT]%[Dao-V- Thuy]%[1K2K5-1]
	Cho cấp số cộng $(u_n)$ có công sai $d<0$, $u_{31}+u_{34}=11$ và $(u_{31})^2 + (u_{34})^2=101$. Số hạng tổng quát của $(u_n)$ là
	\choice
	{$u_{n}=86-3n$}
	{$u_{n}=92-3n$}
	{$u_{n}=95-3n$}
	{\True $u_{n}=103-3n$}
	\loigiai{Gọi cấp số cộng $(u_n)$ có công sai $d$.\\
		$(u_{31})^2 + (u_{34})^2=101 \Leftrightarrow \left( {u_{31}+u_{34}}\right)^2-2u_{31}.u_{34}=101$ $\Rightarrow u_{31}.u_{34}=10$.\\
		Do đó, ta có $\heva{&u_{31}+u_{34}=11\\ &u_{31}.u_{34}=10}$ $\Rightarrow \heva{&u_{31}=10 \\ &u_{34}=1}$(vì $d<0$)\\
		$u_{31}+u_{34}=11 \Rightarrow 2u_{31}+3d =11 \Rightarrow d=-3 \,\,\text{và}\,\, u_{1}=100$.\\
		Do đó: $u_{n}=103-3n$.}
\end{ex}
\Closesolutionfile{ans}
% \begin{indapan}{10}
% 	{ans/ans-1K2-2-Dang2}
% \end{indapan}
\begin{dang}{Tổng của $n$ số hạng đầu tiên của một cấp số cộng. Tính chất của cấp số cộng}
	Tổng của $n$ số hạng đầu tiên:	Đặt ${{S}_{n}}={{u}_{1}}+{{u}_{2}}+{{u}_{3}}+\cdots+{{u}_{n}}.$ Khi đó
	\begin{itemize}
		\item [$\bullet$] ${{S}_{n}}=\dfrac{n\left( {{u}_{1}}+{{u}_{n}} \right)}{2}=\dfrac{n\left( {{u}_{2}}+{{u}_{n-1}} \right)}{2}=\dfrac{n\left( {{u}_{3}}+{{u}_{n-2}} \right)}{2}=\cdots$
		\item [$\bullet$] Vì ${{u}_{n}}={{u}_{1}}+\left( n-1 \right)d$ nên công thức trên có thể viết lại là \fbox{${{S}_{n}}=\dfrac{n}{2}\left[2u_1 + \left(n-1\right)d \right]  .$}
	\end{itemize}
	Tính chất của cấp số cộng:
	\begin{itemize}
		\item [\ding{172}] Nếu $a$; $b$; $c$ theo thứ tự lập thành cấp số cộng thì $a+c=2b$.
		\item [\ding{173}] Lưu ý:
		\begin{itemize}
			\item [$\bullet$] Nếu cho ba số liên tiếp của một cấp số cộng, ta có thể xem ba số đó là $$a-d;\quad a; \quad a+d$$
			\item [$\bullet$] Nếu cho bốn số liên tiếp của một cấp số cộng, ta có thể xem ba số đó là $$a-3d;\quad a-d; \quad a+d; \quad a+3d.$$
		\end{itemize}
	\end{itemize}
\end{dang}
\viduminhhoa
\begin{vd}
	Cho một cấp số cộng $(u_n)$ có $u_3 + u_{28} = 100$. Hãy tính tổng của $30$ số hạng đầu tiên của cấp số cộng đó.\dapso{$1500$}
	\loigiai{Ta có $S_{30} = \dfrac{30(u_1 + u_{30})}{2} = \dfrac{30(u_1 + 2d + u_{30} - 2d)}{2} = \dfrac{30(u_3 + u_{28})}{2} = \dfrac{30 \cdot 100}{2} = 1500$.}
\end{vd}\dongcham{7}

\begin{vd}
	Cho một cấp số cộng $(u_n)$ có $S_6 = 18$ và $S_{10} = 110$. Tính $S_{20}$.	\dapso{$ 620 $.}
	\loigiai{
		Giả sử cấp số cộng $(u_n)$ có số hạng đầu là $u_1$ và công sai là $d$.\\
		Ta có $S_6 = 6u_1 + \dfrac{6 \cdot 5}{2}d \Leftrightarrow 6u_1 + 15d = 18$. \quad (1)\\
		$S_{10} = 10u_1 + \dfrac{10 \cdot 9}{2}d \Leftrightarrow 10u_1 + 45d = 110$. \quad (2)\\
		Từ (1) và (2), ta có hệ phương trình $\heva{&6u_1 + 15d = 18\\&10u_1 + 45d = 110} \Leftrightarrow \heva{&u_1 = -7\\&d = 4.}$\\
		Khi đó $S_{20} = 20u_1 + \dfrac{20 \cdot 19}{2}d = 20 \cdot (-7) + 190 \cdot 4 = 620$.
	}
\end{vd}\dongcham{8}


\begin{vd}
	Tìm số hạng đầu và công sai của cấp số cộng, biết
	\begin{tasks}(2)
		\task $\heva{&u_1^2 + u_2^2 + u_3^2 = 155\\&S_3 = 21.}$	\dapso{$u_1 = 9$, $d = -2$ hoặc $u_1 = 5$, $d = 2$.}
		\task $\heva{&S_3 = 12\\&S_5 = 35.}$	\dapso{$u_1 = 1$, $d = 3$.}
	\end{tasks}
	\loigiai{
		\begin{listEX}
			\item $\heva{&u_1^2 + u_2^2 + u_3^2 = 155\\&S_3 = 21} \Leftrightarrow \heva{&u_1^2 + (u_1 + d)^2 + (u_1 + 2d)^2 = 155 &(1)\\&3u_1 + 3d = 21.&(2)}$\\
			Từ (2), ta có $3u_1 + 3d = 21 \Rightarrow d = 7 - u_1$, thay vào (1)
			$$u_1^2 + 7^2 + (14 - u_1)^2 = 155 \Leftrightarrow 2u_1^2 - 28u_1 + 90 = 0 \Leftrightarrow \hoac{&u_1 = 9\\&u_1 = 5.}$$
			Với $u_1 = 9$ thì $d = -2$. Với $u_1 = 5$ thì $d = 2$.\\
			Vậy số hạng đầu của cấp số cộng là $u_1 = 9$, công sai là $d = -2$ hoặc $u_1 = 5$, $d = 2$.
			\item $\heva{&S_3 = 12\\&S_5 = 35} \Leftrightarrow \heva{&3u_1 + 3d = 12\\&5u_1 + 10d = 35} \Leftrightarrow \heva{&u_1 = 1\\&d = 3.}$\\
			Vậy số hạng đầu của cấp số cộng là $u_1 = 1$, công sai là $d = 3$.
	\end{listEX}}
\end{vd}\dongcham{12}

\begin{vd}
	Tìm số hạng tổng quát của cấp số cộng, biết 
	$\heva{&S_4 = 20\\&\dfrac{1}{u_1} + \dfrac{1}{u_2} + \dfrac{1}{u_3} + \dfrac{1}{u_4} = \dfrac{25}{24}}$ và cấp số cộng có công sai là một số nguyên âm.	\dapso{$ u_n=10-2n $.}
	\loigiai{
		$\heva{&S_4 = 20 &(1)\\&\dfrac{1}{u_1} + \dfrac{1}{u_2} + \dfrac{1}{u_3} + \dfrac{1}{u_4} = \dfrac{25}{24}&(2).}$\\
		Từ (1), suy ra $u_1 + u_4 = u_2 + u_3 = 10$ và $u_1 = 5 - \dfrac{3}{2}d$.\\
		Từ (2), ta có 
		\begin{eqnarray*}
			& &\dfrac{u_1 + u_4}{u_1 \cdot u_4} + \dfrac{u_2 + u_3}{u_2 \cdot u_3} = \dfrac{25}{24} \Leftrightarrow \dfrac{10}{u_1(u_1 + 3d)} + \dfrac{10}{(u_1 + d)(u_1 + 2d)} = \dfrac{25}{24}\\
			&\Leftrightarrow & \dfrac{10}{\left(5 - \dfrac{3}{2}d\right)\left(5 + \dfrac{3}{2}d\right)} + \dfrac{10}{\left(5 - \dfrac{1}{2}d\right)\left(5 + \dfrac{1}{2}d\right)} = \dfrac{25}{24} \Leftrightarrow \dfrac{10}{25 - \dfrac{9}{4}d^2} + \dfrac{10}{25 - \dfrac{1}{4}d^2} = \dfrac{25}{24}\\
			&\Leftrightarrow & 10\left(25 - \dfrac{9}{4}d^2 + 25 - \dfrac{1}{4}d^2\right) = \dfrac{25}{24}\left(25 - \dfrac{9}{4}d^2\right)\left(25 - \dfrac{1}{4}d^2\right)\\
			&\Leftrightarrow & \dfrac{75}{128}d^4 - \dfrac{1925}{48}d^2 + \dfrac{3625}{24} = 0 \Leftrightarrow \hoac{&d^2 = \dfrac{580}{9}\\&d^2 = 4} \Leftrightarrow \hoac{&d = \pm \dfrac{2\sqrt{145}}{3}\\&d = \pm 2.}
		\end{eqnarray*}
		Với $d = -2$ thì $u_1 = 8$. Suy ra $u_n=u_1+(n-1)d=10-2n$}
\end{vd}\dongcham{18}

\begin{vd}
	Tính các tổng sau
	\begin{tasks}(2)
		\task $S = 1 + 3 + 5 + \cdots + (2n - 1) + (2n + 1)$.\dapso{$S = (n + 1)^2$}
		\task $S = 100^2 - 99^2 + 98^2 - 97^2 + \cdots + 2^2 - 1^2$.\dapso{$S = 5050$}
	\end{tasks}
	\loigiai{
		\begin{enumEX}{1}
			\item $S = 1 + 3 + 5 + \cdots + (2n - 1) + (2n + 1)$.\\
			Xét cấp số cộng $(u_k)$, $k \in \mathbb{N}^*$ với số hạng đầu là $u_1 = 1$ và công sai là $d = 2$.\\
			Ta có $u_k = u_1 + (k - 1)d \Leftrightarrow 2n + 1 = 1 + 2(k - 1) \Leftrightarrow k = n + 1$.\\
			Vậy $S = \dfrac{k(u_1 + u_k)}{2} = \dfrac{(n + 1)(1 + 2n + 1)}{2} = (n + 1)^2$.
			\item $S = 100^2 - 99^2 + 98^2 - 97^2 + \cdots + 2^2 - 1^2 = 199 + 195 + \cdots + 3$.\\
			Xét cấp số cộng $(u_n)$ có số hạng đầu $u_1 = 199$ và công sai $d = u_2 - u_1 = 195 - 199 = -4$.\\
			Ta có $u_n = u_1 + (n - 1)d \Leftrightarrow 3 = 199 - 4(n - 1) \Leftrightarrow n = 50$.\\
			Khi đó $S = \dfrac{n(u_1 + u_{50})}{2} = \dfrac{50(199 + 3)}{2} = 5050$.
			
		\end{enumEX}
	}
\end{vd}\dongcham{18}

\begin{vd}
	Tìm ba số hạng liên tiếp của một cấp số cộng biết tổng của chúng bằng $27$ và tổng các bình phương của chúng là $293$.\dapso{$4$, $9$, $14$}
	\loigiai{
		Gọi ba số hạng liên tiếp của cấp số cộng là $x - d$, $x$, $x + d$ trong đó $d$ là công sai của cấp số cộng.\\
		Khi đó ta có $x - d + x + x + d = 27 \Leftrightarrow 3x = 27 \Leftrightarrow x = 9$.\\
		Mà $(x - d)^2 + x^2 + (x + d)^2 = 293 \Leftrightarrow (9 - d)^2 + 81 + (9 + d)^2 = 293 \Leftrightarrow 2d^2 -50 = 0 \Leftrightarrow \hoac{&d = 5\\&d = -5.}$\\	
		Với $d = 5$ thì ba số hạng của cấp số cộng là $4$, $9$, $14$.\\
		Với $d = -5$ thì ba số hạng của cấp số cộng là $14$, $9$, $4$.\\
		Vậy ba số hạng liên tiếp của cấp số cộng là $4$, $9$, $14$.
	}
\end{vd}\dongcham{14}

\begin{vd}
	Tìm bốn số hạng liên tiếp của một cấp số cộng, biết tổng của chúng bằng $10$ và tổng bình phương của chúng bằng $30$.\dapso{$1$, $2$, $3$, $4$}
	\loigiai{
		Gọi bốn số hạng liên tiếp của cấp số cộng là $x - 3d$, $x - d$, $x 
		+ d$, $x + 3d$ với $2d$ là công sai của cấp số cộng.\\
		Khi đó ta có $x - 3d + x - d + x + d + x + 3d = 10 \Leftrightarrow 4x = 10 \Leftrightarrow x = \dfrac{5}{2}$.\\
		Mặt khác $$(x - 3d)^2 + (x - d)^2 + (x + d)^2 + (x + 3d)^2 = 30 \Leftrightarrow 4x^2 + 20d^2 = 30 \Leftrightarrow d^2 = \dfrac{1}{4} \Leftrightarrow \hoac{&d = \dfrac{1}{2}\\&d = -\dfrac{1}{2}.}$$
		Với $x = \dfrac{5}{2}$ thì $d = \dfrac{1}{2}$, khi đó bốn số hạng liên tiếp của cấp số cộng là $1$, $2$, $3$, $4$.\\
		Với $x = \dfrac{5}{2}$ thì $d = -\dfrac{1}{2}$, khi đó bốn số hạng liên tiếp của cấp số cộng là $4$, $3$, $2$, $1$.\\
		Vậy bốn số hạng liên tiếp của cấp số cộng là $1$, $2$, $3$, $4$.
	}
\end{vd}\dongcham{14}

\begin{vd}
	Ba góc của một tam giác vuông lập thành một cấp số cộng. Tìm ba góc đó.
	\loigiai{Gọi ba góc của tam giác lần lượt là $A$, $B$, $C$.
		Khi đó ta có $A + B + C = 180^\circ$.\\
		Do ba góc $A$, $B$, $C$ của tam giác theo thứ tự lập thành một cấp số cộng nên $B-A=C-A \Leftrightarrow A + C = 2B$.\\
		Do đó $2B + B = 180^\circ \Rightarrow 3B = 180^\circ \Rightarrow B = 60^\circ$.\\
		Do tam giác $ABC$ vuông nên giả sử $C = 90^\circ$ khi đó công sai $d$ của cấp số cộng là $d = C - B = 30^\circ$.\\
		Vậy góc $A$ của tam giác là $A = 30^\circ$.}
\end{vd}\dongcham{10}

% \begin{vd}
% 	Cho $a$, $b$, $c$ là ba số hạng liên tiếp của một cấp số cộng. Chứng minh rằng
% 	\begin{tasks}(1)
% 		\task $a^2 + 2bc = c^2 + 2ab$.
% 		\task $2(a+b+c)^3 = 9 \left[ a^2(b+c) + b^2(a+c) + c^2(a+b) \right]$.
% 		\task  $b^2 + bc +c^2$, $a^2 + ac + c^2$, $a^2 + ab + b^2$ cũng là một cấp số cộng.
% 	\end{tasks}
% 	\loigiai{
% 		\begin{enumerate}[a)]
% 			\item Vì $a$, $b$, $c$ là ba số liên tiếp của một cấp số cộng nên $a + c = 2b \Rightarrow a = 2b -c$.\\
% 			Do đó
% 			$$a^2 +2bc = (2b-c)^2 + 2bc = 4b^2 - 2bc + c^2 = 2b(2b -c) + c^2 = 2ba + c^2 = c^2 + 2ab.$$
% 			Vậy $a^2 + 2bc = c^2 + 2ab$ (đpcm).
			
% 			\item Vì $a$, $b$, $c$ là ba số liên tiếp của một cấp số cộng nên $a + c = 2b \Rightarrow a = 2b -c$.\\
% 			Do đó
% 			\allowdisplaybreaks
% 			\begin{eqnarray*}
% 				& \mbox{VT}  & = 2(a+b+c)^3 = 2(3b)^3 = 54b^3\\
% 				& \mbox{VP}  & = 9\left[ a^2(b+c) + b^2(a+c) + c^2(a+b) \right] \\
% 				& & = 9\left[ (2b-c)^2(b+c) + b^2(2b-c+c) + c^2(2b-c+b) \right] \\ 
% 				& & = 9\left[ (4b^2 - 4bc+c^2)(b+c) + b^2(2b) + c^2(3b-c) \right] \\
% 				& & = 9\left[ 4b^3 - 4b^2c +bc^2 + 4b^2c - 4bc^2 + c^3 + 2b^3 + 3bc^2 - c^3 \right] \\
% 				& & = 9\cdot (6b^3) = 54b^3 = \mbox{ VT }.
% 			\end{eqnarray*}
% 			Vậy $2(a+b+c)^3 = 9 \left[ a^2(b+c) + b^2(a+c) + c^2(a+b) \right]$ (đpcm).
			
% 			\item Vì ba số $a$, $b$, $c$ theo thứ tự lập thành một cấp số cộng thì $a + c = 2b \Rightarrow a = 2b - c$.\\
% 			Xét
% 			\allowdisplaybreaks
% 			\begin{eqnarray*}
% 				& 2(a^2 + ac + c^2) - (a^2 + ab + b^2) & = a^2 + a(2c-b) + 2c^2 - b^2 \\
% 				& & = (2b-c)^2 + (2b-c)(2c-b) + 2c^2 - b^2 \\
% 				& & = b^2 + bc  +c^2\\
% 				&\Rightarrow (b^2 + bc  +c^2) + (a^2 + ab + b^2) &= 2(a^2 + ac + c^2).
% 			\end{eqnarray*}
% 			Vậy ba số: $b^2 + bc +c^2$, $a^2 + ac + c^2$, $a^2 + ab + b^2$ cũng là một cấp số cộng.
% 		\end{enumerate}
% 	}
% \end{vd}\dongcham{25}

\begin{vd}%[TH]%[Dự án DCHT-11-KNTT]%[Dao-V- Thuy]%[1K2B5-2]
	Xác định $4$ góc của một tứ giác lồi, biết rằng $4$ góc hợp thành cấp số cộng và góc lớn nhất bằng $5$ lần góc nhỏ nhất.
	\dapso{$36^\circ; \, 72^\circ; \, 108^\circ; \, 144^\circ$}
	\loigiai
	{
		Gọi số đo bốn góc cần tìm là $u_1$, $u_2$, $u_3$, $u_4$. Ta có
		\begin{eqnarray*}
			\heva{&u_1+u_2+u_3+u_4=360\\ &u_5=5u_1} \Leftrightarrow \heva{&4u_1+6d=360\\ &4d=4u_1} \Leftrightarrow \heva{&u_1=36\\ &d=36.}
		\end{eqnarray*}
		Vậy số đo bốn góc cần tìm là
		\[
		36^\circ; \, 72^\circ; \, 108^\circ; \, 144^\circ.
		\]
	}
\end{vd}

\subsubsection{Bài tập tự luận}
 

\begin{bt}%[TH]%[Dự án DCHT-11-KNTT]%[Dao-V- Thuy]%[1K2B5-2]
	Giữa các số $10$ và $64$ hãy đặt thêm $17$ số nữa để được một cấp số cộng.
	\dapso{$13; 16; 19; 22; 25; 28; 31; 34; 37; 40; 43; 46; 49; 52; 55; 58; 61$}
	\loigiai{
		Ta có
		\begin{equation*}
			\heva{&u_1=10\\ &u_{19}=64} \Leftrightarrow \heva{&u_1=10\\ &u_1+18d=64} \Leftrightarrow \heva{&u_1=10\\ &d=3.}
		\end{equation*}
		Vậy $17$ số đặt thêm giữa các số $10$ và $64$ để được một cấp số cộng là
		\begin{center}
			13; 16; 19; 22; 25; 28; 31; 34; 37; 40; 43; 46; 49; 52; 55; 58; 61.
		\end{center} 
	}
\end{bt}

\begin{bt}%[TH]%[Dự án DCHT-11-KNTT]%[Dao-V- Thuy]%[1K2B5-2]
	Tổng ba số hạng liên tiếp của một cấp số cộng bằng $2$ và tổng các bình phương của ba số đó bằng $\dfrac{14}{9}$. Xác định ba số đó và tính công sai của cấp số cộng.
	\dapso{$1;\dfrac{2}{3};\dfrac{1}{3}$ ứng với $d=-\dfrac{1}{3}$ hoặc $\dfrac{1}{3};\dfrac{2}{3};1$ ứng với $d=\dfrac{1}{3}$}
	\loigiai{
		Ta có hệ
		\begin{align*}
			&\quad \heva{&u_k+u_{k+1}+u_{k+2}=2\\ &u^2_k+u^2_{k+1}+u^2_{k+2}=\dfrac{14}{9}}
			\Leftrightarrow \heva{&u_k+u_k+d+u_k+2d=2\\ &u^2_k+\left(u_k+d\right)^2 +\left(u_k+2d\right)^2=\dfrac{14}{9}} \\
			&\Leftrightarrow \heva{&3u_k+3d=2\\ &3u^2_k+6u_kd+5d^2=\dfrac{14}{9}}
			\Leftrightarrow \heva{&u_k=1\\ &d=-\dfrac{1}{3}} \text{ hoặc } \heva{&u_k=\dfrac{1}{3}\\ &d=\dfrac{1}{3}.}
		\end{align*}
		Vậy ba số hạng liên tiếp của cấp số cộng thỏa yêu cầu bài toán $1;\dfrac{2}{3};\dfrac{1}{3}$ ứng với $d=-\dfrac{1}{3}$ hoặc $\dfrac{1}{3};\dfrac{2}{3};1$ ứng với $d=\dfrac{1}{3}.$
	}
\end{bt}

\begin{bt}%[TH]%[Dự án DCHT-11-KNTT]%[Dao-V- Thuy]%[1K2B5-2]
	Một cấp số cộng có $7$ số hạng với công sai $d$ dương và số hạng thứ tư bằng $11$. Hãy tìm các số hạng còn lại của cấp số cộng đó, biết hiệu của số hạng thứ ba và số hạng thứ năm bằng $6$.
	\dapso{$u_1=2$; $ u_2=5$; $u_4=11$; $u_6=17$; $u_7=20$}
	\loigiai{
		Gọi số hạng đầu của cấp số cộng là $u_1$, công sai $d$.
		Vì số hạng thứ tư của cấp số cộng bằng $11$ nên ta có $u_4=11$.\\
		Do $d$ dương nên $ u_5>u_3$.\\
		Vì hiệu của số hạng thứ ba và số hạng thứ năm bằng $6$ nên ta có $ u_5-u_3=6$.\\
		Ta có \begin{align*}
			\heva{&u_4=11\\&u_5-u_3=6 }
			\Leftrightarrow \heva{&u_1+3d=11\\&(u_1+4d)-(u_1+2d)=6 }
			\Leftrightarrow \heva{&u_1+3\cdot 3=11\\&d=3 }
			\Leftrightarrow \heva{&u_1=2\\&d=3.}
		\end{align*}
		Vậy các số  hạng còn lại của cấp số cộng là $u_1=2$; $ u_2=5$; $u_4=11$; $u_6=17$; $u_7=20$.
	}
\end{bt}

\begin{bt}%[VD]%[Dự án DCHT-11-KNTT]%[Dao-V- Thuy]%[1K2K5-2]
	Tìm bốn số hạng liên tiếp của một cấp số cộng, biết rằng:
	\begin{enumerate}
		\item Tổng của chúng bằng $10$ và tổng bình phương bằng $70$.
		\item Tổng của chúng bằng $22$ và tổng bình phương bằng $66$.
		\item  Tổng của chúng bằng $36$ và tổng bình phương bằng $504$.
		\item  Chúng có tổng bằng $20$ và tích của chúng bằng $384$.
		\item  Tổng của chúng bằng $ 20$, tổng nghịch đảo của chúng bằng $ \dfrac{25}{24}$ và các số này là những số nguyên.
		\item  Nó là số đo của một tứ giác lồi và góc lớn nhất gấp $5$ lần góc nhỏ nhất.
	\end{enumerate}
	\dapso{$-2$; $ 1$; $ 4$; $7$.} 
	\dapso{không tồn tại bốn số hạng liên tiếp của cấp số cộng thỏa mãn yêu cầu đề bài. $0$; $ 6$; $ 12$; $18$.}
	\dapso{$2$; $ 4$; $ 6$; $8$ hoặc $5-\sqrt{241}$; $ \dfrac{15-\sqrt{241}}{3}$; $ \dfrac{15+\sqrt{241}}{3}$; $5+\sqrt{241}$.} 
	\dapso{$30^\circ$; $70^\circ$; $ 110^\circ$; $150^\circ$.}
	\loigiai{
		\begin{enumerate}
			\item Gọi bốn số hạng liên tiếp của cấp số cộng là $x-3d$; $x-d$; $x+d$, $x+3d$ trong đó $2d$ là công sai.\\
			Theo đề bài ta có 
			\begin{align*}
				&\quad \heva{& (x-3d)+(x-d)+(x+d)+(x+3d)=10\\& (x-3d)^2+(x-d)^2+(x+d)^2+(x+3d)^2=70}
				\Leftrightarrow \heva{& 4x=10 \\& 4x^2+20d^2=70}\\
				&\Leftrightarrow  \heva{&x=\dfrac{5}{2}\\& 4\cdot \left( \dfrac{5}{2}\right) ^2+20d^2=70}
				\Leftrightarrow  \heva{&x=\dfrac{5}{2}\\& d^2=\dfrac{9}{4}}
				\Leftrightarrow  \heva{&x=\dfrac{5}{2}\\& d=\pm \dfrac{3}{2}.}
			\end{align*}
			Vậy bốn số hạng liên tiếp của cấp số cộng là $-2$; $ 1$; $ 4$; $7$.
			\item Gọi bốn số hạng liên tiếp của cấp số cộng là $x-3d$; $x-d$; $x+d$; $x+3d$ trong đó $2d$ là công sai.\\
			Theo đề bài ta có 
			\begin{align*}
				&\quad \heva{& (x-3d)+(x-d)+(x+d)+(x+3d)=22\\& (x-3d)^2+(x-d)^2+(x+d)^2+(x+3d)^2=66}
				\Leftrightarrow \heva{& 4x=22 \\& 4x^2+20d^2=66}\\
				&\Leftrightarrow  \heva{&x=\dfrac{11}{2}\\& 4\cdot \left( \dfrac{11}{2}\right) ^2+20d^2=66}
				\Leftrightarrow  \heva{&x=\dfrac{11}{2}\\& d^2=\dfrac{-11}{4} 
					\ (\text{loại}).}\\
			\end{align*}
			Vậy không tồn tại bốn số hạng liên tiếp của cấp số cộng thỏa mãn yêu cầu đề bài.
			\item Gọi bốn số hạng liên tiếp của cấp số cộng là $x-3d$; $x-d$; $x+d$; $x+3d$ trong đó $2d$ là công sai.\\
			Theo đề bài ta có 
			\begin{align*}
				&\quad \heva{& (x-3d)+(x-d)+(x+d)+(x+3d)=36\\& (x-3d)^2+(x-d)^2+(x+d)^2+(x+3d)^2=504}
				\Leftrightarrow \heva{& 4x=36 \\& 4x^2+20d^2=504}\\
				&\Leftrightarrow  \heva{&x=9\\& 4\cdot 9^2+20d^2=504}
				\Leftrightarrow  \heva{&x=9\\& d^2=9}
				\Leftrightarrow  \heva{&x=9\\& d=\pm 3.}
			\end{align*}
			Vậy  bốn  số hạng liên tiếp của cấp số cộng là $0$; $ 6$; $ 12$; $18$.
			\item Gọi bốn số hạng liên tiếp của cấp số cộng là $x-3d$; $x-d$; $x+d$; $x+3d$ trong đó $2d$ là công sai.\\
			Theo đề bài ta có 
			\begin{align*}
				&\quad \heva{& (x-3d)+(x-d)+(x+d)+(x+3d)=20\\& (x-3d)(x-d)(x+d)(x+3d)=384}
				\Leftrightarrow  \heva{&x=5\\& (x^2-d^2)(x^2-9d^2)=384}\\
				&\Leftrightarrow  \heva{&x=5\\& (25-d^2)(25-9d^2)=384}
				\Leftrightarrow  \heva{&x=5\\& 9d^4-250d^2+241=0}
				\Leftrightarrow  \heva{&x=5\\& \hoac{&d^2=1\\& d^2=\dfrac{241}{9}}}
				\Leftrightarrow  \heva{&x=5\\& \hoac{&d=\pm 1\\& d=\pm \dfrac{\sqrt{241}}{3}.}}
			\end{align*}
			Vậy bốn số hạng liên tiếp của cấp số cộng là $2$; $ 4$; $ 6$; $8$ hoặc $5-\sqrt{241}$; $ \dfrac{15-\sqrt{241}}{3}$; $ \dfrac{15+\sqrt{241}}{3}$; $5+\sqrt{241}$.
			\item Gọi bốn số hạng liên tiếp của cấp số cộng là $x-3d$; $x-d$; $x+d$; $x+3d$ trong đó $2d$ là công sai trong đó $ 2d \in \mathbb{Z}$.\\
			Theo đề bài ta có 
			\begin{align*}
				&\quad \heva{& (x-3d)+(x-d)+(x+d)+(x+3d)=20\\& \dfrac{1}{x-3d}+\dfrac{1}{x-d}+\dfrac{1}{x+d}+\dfrac{1}{x+3d}=\dfrac{25}{24}}
				\Leftrightarrow \heva{& 4x=20 \\& \dfrac{1}{5-3d}+\dfrac{1}{5-d}+\dfrac{1}{5+d}+\dfrac{1}{5+3d}=\dfrac{25}{24}}\\
				&\Leftrightarrow  \heva{&x=5\\& \dfrac{10}{25-9d^2}+\dfrac{10}{25-d^2}=\dfrac{25}{24}}
				\Leftrightarrow  \heva{&x=5\\& 9d^4-250d^2+241=0}\\
				&\Leftrightarrow  \heva{&x=5\\& \hoac{&d^2=1 \\& d^2=\dfrac{241}{9} }}
				\Leftrightarrow  \heva{&x=5\\& \hoac{&d= \pm 1 \ (\text{thỏa mãn})\\& d= \pm \dfrac{\sqrt{241}}{3} \,\,(\text{loại vì} \,  2d \in \mathbb{Z}).}}
			\end{align*}
			Vậy bốn  số hạng  nguyên liên tiếp của cấp số cộng là $2$; $ 4$; $ 6$; $8$.
			\item Gọi bốn số hạng liên tiếp của cấp số cộng xếp theo thứ tự tăng dần  là $x-3d$; $x-d$; $x+d$; $x+3d$ trong đó $2d>0$ là công sai.\\
			Theo đề bài ta có 
			\begin{align*}
				&\quad \heva{& (x-3d)+(x-d)+(x+d)+(x+3d)=360^\circ\\& x+3d=5(x-3d)}\\
				&\Leftrightarrow \heva{& 4x=360^\circ\\& 4x=18d}
				\Leftrightarrow  \heva{&x=90^\circ\\& 4 \cdot 90^\circ=18d}
				\Leftrightarrow  \heva{&x=90^\circ\\& d=20^\circ.}
			\end{align*}
			Vậy bốn  góc của tứ giác lồi lần lượt là  $30^\circ$; $70^\circ$; $ 110^\circ$; $150^\circ$.
		\end{enumerate}
	}
\end{bt}
% \subsubsection{Câu hỏi trắc nghiệm}
% \Opensolutionfile{ans}[ans/ans-1K2-2-Dang3]
% \begin{ex}%[Dự án DCHT-11-KNTT]%[Dao-V- Thuy]%[1K2Y5-2]
% 	Cấp số cộng $(u_n)$ có số hạng đầu $u_1=-5$ và công sai $d=3$. Tính $u_{15}$.
% 	\choice
% 	{$u_{15}=27$}
% 	{\True $u_{15}=37$}
% 	{$u_{15}=47$}
% 	{$u_{15}=57$}
% 	\loigiai{$u_{15}=u_1+14d=-5+14\times 3=37$.}
% \end{ex}

% \begin{ex}%[Dự án DCHT-11-KNTT]%[Dao-V- Thuy]%[1K2Y5-2]
% 	Cho cấp số cộng có các số hạng ban đầu là $1$; $5$; $9$; $13$; $\cdots$. Số hạng thứ $ 6 $ của cấp số cộng này là bao nhiêu?
% 	\choice{\True $ 21$}
% 	{$19 $}
% 	{$ 22$}
% 	{$ 20$}
% 	\loigiai{Ta có $u_1=1$, $d=5-1=4$ nên $u_6=1+5d=1+20=21$.
% 	}
% \end{ex}

% \begin{ex}%[Dự án DCHT-11-KNTT]%[Dao-V- Thuy]%[1K2Y5-2]
% 	Cho cấp số cộng $\left( u_n \right)$ có các số hạng lần lượt là $-4;\,1;\,6;\,x$. Tìm giá trị của $x$.
% 	\choice
% 	{$x=7$}
% 	{$x=10$}
% 	{\True $x=11$}
% 	{$x=12$}
% 	\loigiai{
% 		Dễ thấy $u_1=-4$, $d=5$ nên $u_4=-4+3\cdot 5=11$.
% 	}
% \end{ex}

% \begin{ex}%[Dự án DCHT-11-KNTT]%[Dao-V- Thuy]%[1K2B5-2]
% 	Cho cấp số cộng $(u_n)$ có $u_1=-5$ và $d=3$. Mệnh đề nào sau đây đúng?
% 	\choice
% 	{$u_{15}=34$}
% 	{$u_{15}=45$}
% 	{\True $u_{13}=31$}
% 	{$u_{10}=35$}
% 	\loigiai {
% 		$\heva{
% 			& u_1=-5 \\
% 			& d=3 \\
% 		}\Rightarrow u_n=3n-8\Rightarrow \heva{
% 			& u_{15}=37 \\
% 			& u_{13}=31 \\
% 			& u_{10}=22. \\
% 		}$}
% \end{ex}

% \begin{ex}%[Dự án DCHT-11-KNTT]%[Dao-V- Thuy]%[1K2B5-2]
% 	Cho cấp số cộng có số hạng đầu là $u_1=-\dfrac{1}{2}$, công sai $d=\dfrac{1}{2}$. Trong mỗi bộ gồm năm số hạng dưới đây, bộ năm số nào là các số hạng liên tiếp của dãy này?
% 	\choice
% 	{$-\dfrac{1}{2};\,0;\,1;\,\dfrac{1}{2};\,1$}
% 	{$-\dfrac{1}{2};\,0;\,\dfrac{1}{2};\,0;\,\dfrac{1}{2}$}
% 	{$\dfrac{1}{2};\,1;\,2;\,\dfrac{5}{2};\,\dfrac{7}{2}$}
% 	{\True $1;\,\dfrac{3}{2};\,2;\,\dfrac{5}{2};\,3$}
% 	\loigiai{
% 		Ta có $u_1=-\dfrac{1}{2}$; $u_2=0$; $u_3=\dfrac{1}{2}$, $u_4=1$; $u_5=\dfrac{3}{2}$; $u_6=2$; $u_7=\dfrac{5}{2}$; $u_8=3$.
% 	}
% \end{ex}

% \begin{ex}%[Dự án DCHT-11-KNTT]%[Dao-V- Thuy]%[1K2B5-2]
% 	Cho cấp số cộng $(u_n)$ có $u_7=\dfrac{19}{5}$ và công sai $d=\dfrac{2}{5}$. Tính $u_{10}$.
% 	\choice
% 	{$\dfrac{2}{5}$}
% 	{$\dfrac{19}{5}$}
% 	{\True $5$}
% 	{$\dfrac{27}{5}$}
% 	\loigiai{Ta có: $u_7=u_1+6d\Rightarrow u_1=u_7-6d=\dfrac{19}{5}-6\cdot \dfrac{2}{5}=\dfrac{7}{5}$.\\
% 		Suy ra $u_{10}=u_1+9d=\dfrac{7}{5}+9\cdot \dfrac{2}{5}=5$.}
% \end{ex}

% \begin{ex}%[Dự án DCHT-11-KNTT]%[Dao-V- Thuy]%[1K2B5-2]
% 	Cho cấp số cộng $(u_n)$ có số hạng đầu $u_1=-1$ và công sai $d=-3$. Số hạng thứ $20$ của cấp số cộng này là
% 	\choice
% 	{\True $u_{20}=-58$}
% 	{$u_{20}=60$}
% 	{$u_{20}=-72$}
% 	{$u_{20}=-61$}
% 	\loigiai{Số hạng thứ $20$ là: $u_{20}=u_1+19d=-1+19\cdot (-3)=-58$.}
% \end{ex}

% \begin{ex}%[Dự án DCHT-11-KNTT]%[Dao-V- Thuy]%[1K2B5-2]
% 	Cho cấp số cộng $(u_n)$ có $u_1=-5$ và $d=3$. Số $100$ là số hạng thứ mấy của cấp số cộng?
% 	\choice
% 	{Thứ $15$}
% 	{Thứ $20$}
% 	{Thứ $35$}
% 	{\True Thứ $36$}
% 	\loigiai {
% 		Ta có $\heva{
% 			&u_1=-5 \\
% 			& d=3 \\
% 		}$. Vì $u_n=100 \Rightarrow 100=u_n=u_1+(n-1 )d=3n-8\Leftrightarrow n=36$.
% 	}
% \end{ex}

% \begin{ex}%[Dự án DCHT-11-KNTT]%[Dao-V- Thuy]%[1K2B5-2]
% 	Cho cấp số cộng $(u_n)$ có $u_2=2001$ và $u_5=1995$. Khi đó $u_{1001}$ bằng
% 	\choice
% 	{$u_{1001}=4005$}
% 	{$u_{1001}=4003$}
% 	{\True $u_{1001}=3$}
% 	{$u_{1001}=1$}
% 	\loigiai {
% 		$\heva{
% 			& 2001=u_2=u_1+d \\
% 			& 1995=u_5=u_1+4d \\
% 		}\Leftrightarrow \heva{
% 			&u_1=2003 \\
% 			& d=-2 \\
% 		}\Rightarrow u_{1001}=u_1+1000d=3$.}
% \end{ex}

% \begin{ex}%[Dự án DCHT-11-KNTT]%[Dao-V- Thuy]%[1K2B5-2]
% 	Cho cấp số cộng $(u_n)$ biết $\heva{&u_1+u_3=7\\&u_2+u_4=12}$. Tính $u_{21}$.
% 	\choice
% 	{$u_{21}=1$}
% 	{\True $u_{21}=51$}
% 	{$u_{21}=31$}
% 	{$u_{21}=21$}
% 	\loigiai{
% 		Ta có $\heva{&u_1+u_3=7\\&u_2+u_4=12}\Leftrightarrow\heva{&u_1+u_1+2d=7\\&u_1+d+u_1+3d=12}\Leftrightarrow\heva{&2u_1+2d=7\\&2u_1+4d=12}\Leftrightarrow\heva{&u_1=1\\&d=\dfrac{5}{2}.}$\\
% 		Suy ra $u_{21}=u_1+20d=1+20\cdot\dfrac{5}{2}=1+50=51$.}
% \end{ex}

% \begin{ex}%[Dự án DCHT-11-KNTT]%[Dao-V- Thuy]%[1K2B5-2]
% 	Một cấp số cộng có $7$ số hạng. Biết rằng tổng của số hạng đầu và số hạng cuối bằng $30$, tổng của số hạng thứ ba và số hạng thứ sáu bằng $35$. Tìm số hạng thứ bảy của cấp số cộng đã cho.
% 	\choice
% 	{$u_7=25$}
% 	{\True $u_7=30$}
% 	{$u_7=35$}
% 	{$u_7=40$}
% 	\loigiai{
% 		Theo đề ta có: $\heva{&u_1+u_7=30\\&u_3+u_6=35}\Leftrightarrow \heva{&u_1+(u_1+6d)=30\\&(u_1+2d)+(u_1+5d)=35}\Leftrightarrow \heva{&2u_1+6d=30\\&2u_1+7d=35}\Leftrightarrow \heva{&u_1=0\\&d=5.}$\\
% 		Do đó $u_7=u_1+6d=0+6\cdot 5=30$.}
% \end{ex}

% \begin{ex}%[Dự án DCHT-11-KNTT]%[Dao-V- Thuy]%[1K2G5-2]
% 	Cho dãy số $(u_n)$ có xác định bởi $\heva{&u_1=-2,\\&u_{n+1}=\dfrac{u_n}{1-u_n}} \; (\text{với } n\in\mathbb{N}^*)$ và dãy số $(v_n)$ được xác định bởi $v_n=\dfrac{u_n+1}{u_n}$. Số hạng thứ $2023$ của dãy $(v_n)$ là
% 	\choice
% 	{$ -\dfrac{2023}{3}$}
% 	{$ -\dfrac{4046}{3}$}
% 	{\True$ -\dfrac{4043}{2}$}
% 	{$ -2023$}
% 	\loigiai{
% 		Ta có $v_{n+1}-v_n=\dfrac{u_{n+1}+1}{u_{n+1}}-\dfrac{u_n+1}{u_n}=\dfrac{\dfrac{u_n}{1-u_n}+1}{\dfrac{u_n}{1-u_n}}-\dfrac{u_n+1}{u_n}=\dfrac{1}{u_n}-\dfrac{u_n+1}{u_n}=-1$. Vậy $(v_n)$	là một CSC có công sai $d=-1$.
% 		\\Mặt khác, ta có $v_1=\dfrac{u_1+1}{u_1}=\dfrac{1}{2}$, do đó số hạng tổng quát $v_n=\dfrac{1}{2}+(n-1)(-1)=-n+\dfrac{3}{2}$. \\
% 		Do đó $v_{2023}=-2023+\dfrac{3}{2}=-\dfrac{4043}{2}$.
% 	}
% \end{ex}
% \Closesolutionfile{ans}
% \begin{indapan}{10}
% 	{ans/ans-1K2-2-Dang3}
% \end{indapan}

\begin{dang}{Các bài toán thực tế}
	Các bài toán thực tế về cấp số cộng có thể được giải bằng cách sử dụng công thức của cấp số cộng. Công thức của cấp số cộng là: $ u_n = u_1 + (n-1)d $. Trong đó:
	\begin{itemize}
		\item $ u_n $ là số hạng thứ $ n $ của cấp số cộng.
		\item $ u_1 $ là số hạng đầu tiên của cấp số cộng.
		\item $ d $ là công sai của cấp số cộng.
		\item Một số công thức thường gặp:
		\begin{enumEX}[\faCheckCircleO]{1}
			\item $u_n=\dfrac{u_{n-1}+u_{n+1}}{2}=u_1+(n-1)d$.
			\item $S_n=\dfrac{(u_1+u_n)\cdot n}{2}=\dfrac{2u_1+(n-1)d}{2}\cdot n$.
		\end{enumEX}			
	\end{itemize}
\end{dang}
\subsubsection{Ví dụ minh hoạ}
\begin{vd}%[NB]%[DCHT Toán 11 - KNTT - Nguyễn Hữu Đức] %[1K2B6-6]
	Một người có một khoản tiền gửi ngân hàng với lãi suất 10\% /năm theo hình thức lãi đơn. Nếu sau $ 5 $ năm người đó nhận được tổng số tiền là $ 550 $ triệu đồng thì số tiền gửi ban đầu của người đó là bao nhiêu?
	\dapso{$366{,}67$ triệu đồng.}
	\loigiai{
		Gọi $x$ là số tiền gửi ban đầu của người đó $ (x>0) $.\\
		Sau 5 năm, số tiền nhận được bằng số tiền gốc cộng với lãi suất:
		$$
		x + 0{,}1x \times 5 = 1{,}5x.
		$$
		Theo đề bài, tổng số tiền nhận được sau 5 năm là $550$ triệu đồng, do đó ta có phương trình:
		$$
		1{,}5x = 550.
		$$
		Giải phương trình ta có:
		$$
		x = \frac{550}{1{,}5} \approx 366{,}67.
		$$
		Vậy số tiền gửi ban đầu của người đó là $366{,}67$ triệu đồng.
	}
\end{vd}
\begin{vd}%[DCHT Toán 11 - KNTT - Nguyễn Hữu Đức] %[1K2B6-6]
	Bạn An muốn mua một món quà tặng mẹ nhân ngày mùng $8/3$. Bạn quyết định tiết kiệm từ ngày $1/2/2017$ đến hết ngày $6/3/2017$. Ngày đầu An có $5\,000$ đồng, kể từ ngày thứ hai số tiền An tiết kiệm được ngày sau cao hơn ngày trước mỗi ngày $1\,000$ đồng. Tính số tiền An tiết kiệm được để mua quà tặng mẹ.
	\dapso{$731\,000$ đồng.}
	\loigiai{
		Tính số ngày mà An tiết kiệm được từ ngày $1/2/2017$ đến hết ngày $6/3/2017$:\\
		Số ngày từ ngày $1/2/2017$ đến hết ngày $28/2/2017$ là $28$ ngày.\\
		Số ngày từ ngày $1/3/2017$ đến hết ngày $6/3/2017$ là $6$ ngày.\\
		Vậy An tiết kiệm được $28+6=34$ ngày.\\
		Gọi $u_n$ là số tiền An tiết kiệm được vào ngày thứ $n$ kể từ ngày $1/2/2017$.\\
		Theo đề ta có $u_1=5\,000$ đồng.\\
		Vì ngày sau An tiết kiệm được nhiều hơn ngày trước mỗi ngày $1\,000$ đồng nên $u_n=u_{n-1}+1\,000$, với $n\ge 2$.\\
		Vậy $(u_n)$ là một cấp số cộng với $u_1=5\,000$ và công sai $d=1\,000$.\\
		Tổng số tiền An tiết kiệm được trong $34$ ngày là:
		$$S_{34}=\dfrac{n}{2} \left(2u_1+33d\right)= \dfrac{34}{2} \left(2\cdot 5\,000+33\cdot 1\,000\right)=731\,000.$$
		Vậy số tiền An tiết kiệm được để mua quà tặng mẹ là $731\,000$ đồng.
	}
\end{vd}

\begin{vd}%[TH]%[DCHT Toán 11 - KNTT - Nguyễn Hữu Đức] %[1K2B6-6]
	[Cấp số nhân] Một hội đồng quản trị quyết định tăng lương cho nhân viên hàng năm theo tỷ lệ cố định. Ví dụ, lương của một nhân viên được tăng thêm $ 5 $\% so với năm trước. Hỏi nếu lương của một nhân viên là $ 10 $ triệu đồng/năm vào năm nay, thì lương của nhân viên đó sẽ là bao nhiêu vào năm thứ $ 5 $?
	\dapso{$12{,}1550625 $ triệu đồng/năm.}
	\loigiai{		
		Theo giả thiết, lương của nhân viên được tăng thêm $ 5 $ \% so với năm trước đó.
		\begin{itemize}
			\item Vậy lương của nhân viên vào năm thứ $ 2 $ sẽ là $ 10\cdot(1+0{,}05)=10{,}5 $ triệu đồng/năm.
			\item Tương tự, lương của nhân viên vào năm thứ $ 3 $ sẽ là $ 10{,}5 \cdot(1+0{,}05)=11{,}025 $ triệu đồng/năm.
			\item Lương của nhân viên vào năm thứ $ 4 $ sẽ là $ 11{,}025\cdot (1+0{,}05)=11{,}57625 $ triệu đồng/năm.
			\item Cuối cùng, lương của nhân viên vào năm thứ $ 5 $ sẽ là $ 11{,}57625\cdot(1+0{,}05)=12{,}1550625 $ triệu đồng/năm.
		\end{itemize}
		Vậy lương của nhân viên đó vào năm thứ $ 5 $ sẽ là $ 12{,}1550625 $ triệu đồng/năm.\\
		Chú ý: Lương của nhân viên đó vào năm thứ $ 5 $ sẽ là $ u_5=u_1+4d=10+4\cdot 10\cdot 0{,}05=12 $ triệu đồng chỉ đúng trong trường hợp lương của một nhân viên được tăng thêm $ 5 $\% so với năm đầu tiên.
	}
\end{vd}


\begin{vd}%[TH]%%[DCHT Toán 11 - KNTT - Nguyễn Hữu Đức] %[1K2B6-6]
	Hùng đang tiết kiệm để mua một cây guitar. Trong tuần đầu tiên, anh ta để dành  $ 42 $ đô la, và trong mỗi tuần tiết theo, anh ta đã thêm $ 8 $ đô la vào tài khoản tiết kiệm của mình. Cây guitar Hùng cần mua có giá $ 400 $ đô la. Hỏi vào tuần thứ bao nhiêu thì anh ấy có đủ tiền để mua cây guitar đó?
	\dapso{$ n=46 $.}
	\loigiai{
		Gọi $ n $ là số tuần anh ta đã thêm $ 8 $ đô la vào tài khoản tiết kiệm của mình.\\
		Số tiền anh ta tiết kiệm được sau $ n $ tuần đó là $ S=42+8n $. \\
		Theo bài ra $ S=42+8n\ge 400\Leftrightarrow n\ge 44,75\Rightarrow n=45 $.\\
		Vậy kể cả tuần đầu thì tuần thứ $ 46 $ anh ta có đủ tiền để mua cây guitar đó.
	}
\end{vd}

\begin{vd}%[DCHT Toán 11 - KNTT - Nguyễn Hữu Đức]%[1K2Y6-6]
	[Cấp số nhân] Hàng tháng ông An gửi vào ngân hàng một số tiền như nhau là $5\,000\,000$ đồng (vào ngày đầu mỗi tháng) với lãi suất $0{,}5\%$ một tháng, biết tiền lãi của tháng trước được nhập vào tiền gốc của tháng sau. Hỏi sau $36$ tháng ông An nhận được số tiền vốn và lãi là bao nhiêu? (làm tròn đến hàng đơn vị).
	\dapso{ $ 197\,663\,927 $ đồng.}
	\loigiai{
		Gọi $a$ là số tiền ông An gửi vào hàng tháng, $r$ là lãi suất trên một tháng và $P_n$ là số tiền vốn và lãi ông An nhận được sau $n$ tháng.
		\begin{itemize}
			\item Sau một tháng, ông An có số tiền là $P_1=a+ar=a(1+r)$.
			\item Đầu tháng thứ hai, ông An có số tiền là $P_1+a=a(1+r)+a$.
			\item Sau hai tháng, ông An có số tiền là $P_2=a(1+r)+a+\left[a(1+r)+a\right]r=a\left[(1+r)^2+(1+r)\right]$.
			\item Cuối tháng thứ $36$, ông An có số tiền là
			\begin{align*}
				P_{36}&=a\left[(1+r)^{36}+(1+r)^{35}+\ldots+(1+r)\right]\\
				&=a(1+r)\dfrac{(1+r)^{36}-1}{r}\\
				&=5000000\cdot (1+0{,}005)\cdot\dfrac{(1+0{,}005)^{36}-1}{0{,}005}\\
				&\approx 197\,663\,927 \quad \text{(đồng)}.
			\end{align*}
		\end{itemize}
	}
\end{vd}

\begin{vd}[VDT]%[DCHT Toán 11 - KNTT - Nguyễn Hữu Đức] %[1K2Y6-6]
	Một xưởng có đăng tuyển công nhân với đãi ngộ về lương như sau: Trong quý đầu tiên thì xưởng trả là $ 6 $ triệu đồng/quý và kể từ quý thứ $ 2 $ sẽ tăng lên $ 0{,}5 $ triệu cho $ 1 $ quý. Hỏi với đãi ngộ trên thì sau $ 5 $ năm làm việc tại xưởng, tổng số lương của công nhân đó là bao nhiêu?
	\dapso{ $ 215 $ triệu đồng.}
	\loigiai{
		Gọi $u_n$ (triệu đồng) là số lương của công nhân trong quý thứ $n$.\\
		Theo đề:\\
		Quý đầu: $ u_1 = 6 $ triệu.\\
		Các quý tiếp theo: $ u_{n+1} = u_{n} + 0,5 $ với $\forall n \ge 1$.\\
		Mức lương của công nhân mỗi quý là $ 1 $ số hạng của dãy số $ u_n $. Mặt khác, lương của quý sau hơn lương quý trước là $ 0,5 $ triệu nên dãy số $ u_n $ là một cấp số cộng với công sai $ d = 0{,}5 $.\\
		Ta biết $ 1 $ năm sẽ có $ 4 $ quý nên $ 5 $ năm sẽ có $ 5\cdot 4 = 20 $ quý. Theo yêu cầu của đề bài ta cần tính tổng của $ 20 $ số hạng đầu tiên của cấp số cộng ($ u_n $).\\
		Lương tháng quý $ 20 $ của công nhân: $ u_{20} = 6 + (20 - 1)\cdot 0{,}5 = 15{,}5 $ triệu đồng.\\
		Tổng số lương của công nhân nhận được sau $ 5 $ năm làm việc tại xưởng: $ S_{12}=20\cdot (6+15,5)2=215 $ (triệu đồng).
	}
\end{vd}

% \subsubsection{Bài tập tự luận}
 
%[DCHT Toán 11 - KNTT - Nguyễn Hữu Đức] %[1K2B6-6]
% \begin{bt}%[NB]%[DCHT Toán 11 - KNTT - Nguyễn Hữu Đức] %[1K2B6-6]
% 	Sinh nhật bạn của An vào ngày $ 01 $ tháng năm. An muốn mua một món quà sinh nhật cho bạn nên quyết định bỏ ống heo $ 100 $ đồng vào ngày $ 01 $ tháng $ 01 $ năm $ 2016 $, sau đó cứ liên tục ngày sau hơn ngày trước $ 100 $ đồng. Hỏi đến ngày sinh nhật của bạn, An đã tích lũy được bao nhiêu tiền? (thời gian bỏ ống heo tính từ ngày $ 01 $ tháng $ 01 $ năm $ 2016 $ đến ngày $ 30 $ tháng $ 04 $  năm $ 2016 $).\dapso{$ 738\,100 $ đồng.}
% 	\loigiai{
% 		Từ ngày $1$ tháng $1$ năm $2016$ đến ngày $30$ tháng $4$ năm $2016$ có tổng cộng $31+29+31+30=121$ ngày.\\
% 		Gọi $S$ là số tiền An tích lũy được vào ngày sinh nhật của bạn.\\
% 		Do An bỏ được $100$ đồng vào ngày đầu tiên nên số tiền An tích lũy được vào ngày thứ $n$ là 
% 		$$S= 100 + 100(n-1).$$
% 		Vậy tổng số tiền An tích lũy được là:
% 		$$
% 		S=100 + 200 + \cdots + 12\,100 = \frac{121(100 + 12\,100)}{2} = 738\,100
% 		.$$
% 		Vậy An đã tích lũy được $738\,100$ đồng vào ngày sinh nhật của bạn.}
% \end{bt}

% \begin{bt}%[TH]%[DCHT Toán 11 - KNTT - Nguyễn Hữu Đức] %[1K2Y6-6]
% 	Người ta trồng $ 3\,003 $ cây theo dạng một hình tam giác như sau: hàng thứ nhất trồng $ 1 $ cây, hàng thứ hai trồng $ 2 $ cây, hàng thứ ba trồng $ 3 $ cây... cứ tiếp tục trồng như thế cho đến khi hết số cây. Số hàng cây được trồng là bao nhiêu?
% 	\dapso{$ 77 $ hàng.}
% 	\loigiai{
% 		Tổng số cây trồng được là $1 + 2 + 3 + \cdots + n$, nghĩa là tổng của $n$ số tự nhiên đầu tiên. Ta cần tìm số $n$ để tổng này bằng $3003$.\\
% 		Ta có công thức tổng của $n$ số tự nhiên đầu tiên là:
% 		$$
% 		1 + 2 + 3 + \cdots + n = \frac{n(n+1)}{2}.
% 		$$
% 		Giải phương trình:
% 		$$
% 		\frac{n(n+1)}{2} = 3\,003.
% 		$$
% 		Ta có:
% 		$$
% 		n(n+1) = 6\,006
% 		\Rightarrow n=77.$$
% 		Vậy số hàng cây được trồng là $77$.
% 	}
% \end{bt}
% \begin{bt}%[TH]%[DCHT Toán 11 - KNTT - Nguyễn Hữu Đức] %[1K2B6-6]
% 	Một công ty định mức sản phẩm hàng tháng theo cấp số cộng. Ví dụ, sản lượng hàng tháng của một công ty được tăng thêm $10$ sản phẩm so với tháng trước. Nếu công ty sản xuất được $ 100 $ sản phẩm trong tháng này, hỏi công ty sẽ sản xuất được bao nhiêu sản phẩm trong tháng thứ $ 12 $?
% 	\dapso{$ 210 $ sản phẩm.}
% 	\loigiai{
% 		Công thức cấp số cộng được sử dụng để tính sản lượng hàng tháng của công ty. Nếu công ty sản xuất được $100$ sản phẩm trong tháng này và sản lượng hàng tháng được tăng thêm $10$ sản phẩm so với tháng trước, ta có thể sử dụng công thức sau để tính sản lượng hàng tháng của công ty trong tháng thứ $12$:
% 		$$
% 		a_n = a_1 + (n-1)d.
% 		$$
% 		Trong đó $a_1$ là sản lượng hàng tháng ban đầu, $d$ là công sai và $n$ là số tháng.\\
% 		Với bài toán này, ta có: $a_1 = 100$, $d = 10$, $n = 12$.\\
% 		Sản lượng hàng tháng của công ty trong tháng thứ $12$ là:
% 		$$
% 		a_{12} = a_1 + (n-1)d = 100 + (12-1) \times 10 = 210.
% 		$$
% 		Vậy công ty sẽ sản xuất được $210$ sản phẩm trong tháng thứ $12$.
% 	}
% \end{bt}

\subsubsection{Câu hỏi trắc nghiệm}
\Opensolutionfile{ans}[ans/ans-1K2-2-dang4]
\begin{ex}%[TH]%[DCHT Toán 11 - KNTT - Nguyễn Hữu Đức]%[1K2B6-6]
	Một công ty đang cần tuyển dụng thêm nhân viên. Công ty quyết định tăng số lượng nhân viên hàng tháng theo cấp số cộng. Nếu công ty đã có $ 20 $ nhân viên và quyết định tăng thêm $ 2 $ nhân viên hàng tháng, hỏi sau bao nhiêu tháng công ty sẽ có 50 nhân viên?
	\choice
	{$19$ tháng}
	{\True $16$ tháng}
	{$36$ tháng}
	{$26$ tháng}
	\loigiai{
		Để giải bài toán này, ta có thể sử dụng công thức cấp số cộng:
		$$
		a_n = a_1 + (n-1) \times d.
		$$
		Trong đó $a_1$ là số lượng nhân viên ban đầu, $d$ là số lượng nhân viên tăng hàng tháng và $n$ là số tháng.\\
		Ta cần tìm số tháng $n$ để công ty có được $50$ nhân viên. Thay các giá trị vào công thức cấp số cộng ta có:
		$$
		50 = 20 + (n-1) \times 2.
		$$
		Suy ra:
		$$
		n = \frac{50 - 20}{2} + 1 = 16
		.$$
		Vậy sau $16$ tháng kể từ khi công ty quyết định tăng số lượng nhân viên hàng tháng theo cấp số cộng, công ty sẽ có được $50$ nhân viên.
	}
\end{ex}
\begin{ex}%[VD]%[DCHT Toán 11 - KNTT - Nguyễn Hữu Đức] %[1K2B6-6]
	Một người đang tăng cường luyện tập thể thao hàng ngày. Anh ta quyết định tăng mức độ luyện tập theo cấp số cộng hàng tuần. Nếu anh ta bắt đầu với mức luyện tập $ 30 $ phút mỗi ngày và tăng thêm $ 5 $ phút mỗi ngày, hỏi anh ta sẽ luyện tập được bao lâu để đạt được mức luyện tập $ 60 $ phút mỗi ngày?
	\choice
	{$16$ ngày}
	{\True $6$ ngày}
	{$9$ ngày}
	{$7$ ngày}
	\loigiai{
		Gọi $n$ là số ngày liên tiếp mà người đó tăng mức độ luyện tập. Theo đó, mức độ luyện tập của người đó sau $n$ ngày là:
		$$
		30 + 5n\, \text{(phút)}.
		$$
		Vì để đạt được mức luyện tập $ 60 $ phút mỗi ngày nên:
		$$
		30 + 5n = 60.
		$$
		Từ đó suy ra:
		$$
		n = \frac{60-30}{5} = 6.
		$$
		Vậy người đó cần luyện tập liên tiếp trong $6$ ngày để đạt được mức luyện tập $60$ phút mỗi ngày.	
	}
\end{ex}
\begin{ex}%[VD]%[DCHT Toán 11 - KNTT - Nguyễn Hữu Đức] %[1K2Y6-6]
	Nếu một công ty công nghệ mới thành lập có số lượng người dùng ban đầu là $ 10\,000 $ và mỗi tháng tăng thêm cố định $ 5\,000 $ lượng người dùng, thì sau bao lâu có số lượng người dùng là $ 1 $ triệu. 
	\choice
	{\True $198$ tháng}
	{$197$ tháng}
	{$18$ tháng}
	{$98$ tháng}
	
	\loigiai{
		Ta cần tính số tháng $n$ theo công thức sau:
		$$10\,000 + 5\,000n = 1\,000\,000.$$
		$$\Rightarrow n = \frac{1\,000\,000 - 10\,000}{5\,000} = 198.$$
		Vậy sau khoảng $198$ tháng (khoảng $16$ năm và $6$ tháng), công ty sẽ đạt được $1$ triệu người dùng.
	}
\end{ex}
\begin{ex}%[VDC]%[DCHT Toán 11 - KNTT - Nguyễn Hữu Đức] %[1K2Y6-6]
	Một nhà đầu tư đang đầu tư vào một quỹ đầu tư với mức lợi nhuận cố định hàng năm. Nếu nhà đầu tư đầu tư vào quỹ đầu tư với số tiền ban đầu là $ 20 $ triệu đồng và mức lợi nhuận hàng năm là $ 10 $\%, hỏi số tiền nhà đầu tư sẽ nhận được sau $ 7 $ năm?
	\choice
	{\True $34$ triệu đồng}
	{$14$ triệu đồng}
	{$30$ triệu đồng}
	{$39$ triệu đồng}
	\loigiai{
		Với số tiền ban đầu là $ 20 $ triệu đồng và mức lợi nhuận hàng năm là $ 10 $\%, ta có thể tính được số tiền nhà đầu tư sẽ nhận được sau $ 1 $ năm, sau đó sử dụng cấp số cộng để tính số tiền nhà đầu tư sẽ nhận được sau $ 7 $ năm.
		
		Số tiền nhà đầu tư sẽ nhận được sau $ 1 $ năm là:
		
		$ 20 $ triệu đồng $\times$ $ 10 $\% = $ 2 $ triệu đồng
		
		Số tiền nhà đầu tư sẽ nhận được sau $ 7 $ năm là:
		
		$ 2 $ triệu đồng $\times$ $ 7 $ năm + $ 20 $ triệu đồng = $ 34 $ triệu đồng
		
		Vậy sau $ 7 $ năm, nhà đầu tư sẽ nhận được tổng cộng $ 34 $ triệu đồng.
	}
\end{ex}
\begin{ex}%[VDC]%[DCHT Toán 11 - KNTT - Nguyễn Hữu Đức] %[1K2Y6-6]
	Một công ty sản xuất bánh kẹo tăng sản lượng sản phẩm của mình lên mỗi tháng. Nếu sản lượng ban đầu là $ 1\,000 $ sản phẩm, một sản phẩm lợi nhuận $ 1 $ USD và tăng thêm $ 200 $ sản phẩm mỗi tháng, thì sau bao nhiêu tháng lợi nhuận công ty $ 1 $ triệu đô.
	\choice
	{$8\,000$ tháng}
	{$7\,000$ tháng}
	{$9\,000$ tháng}
	{\True $5\,000$ tháng}
	\loigiai{
		Để tính thời gian công ty đạt được lợi nhuận 1 triệu đô, chúng ta cần biết lợi nhuận của công ty đạt được bao nhiêu sau mỗi tháng.\\
		Giả sử sản lượng ban đầu là $1\,000$ sản phẩm một sản phẩm lợi nhuận $1$ USD và tăng thêm $200$ sản phẩm mỗi tháng. Ta có thể tính được lợi nhuận của công ty sau mỗi tháng như sau:
		\begin{itemize} 
			\item Tháng 1: $1\,000 \times 1 = 1000$ USD.
			\item Tháng 2: $(1\,000 + 200) \times 1 = 1200$ USD.
			\item Tháng 3: $(1\,000 + 2 \times 200) \times 1 = 1\,400$ USD.
			\item Tháng 4: $(1\,000 + 3 \times 200) \times 1 = 1\,600$ USD.
			\item Tháng $n$: $(1\,000 + (n - 1) \times 200) \times 1 = (n - 1) \times 200 + 1\,000$ USD.
		\end{itemize}
		Để tính thời gian để công ty đạt được lợi nhuận $1$ triệu đô, ta giải phương trình sau:
		
		$(n - 1) \times 200 + 1\,000 = 10^6$
		
		$\Rightarrow (n - 1) \times 200 = (10^6 - 1000)$
		
		$\Rightarrow n - 1 = \dfrac{10^6 - 1\,000}{200}$
		
		$\Rightarrow n = \dfrac{10^6 - 1\,000}{200} + 1$
		
		$\Rightarrow n = 5\,001$
		
		Vậy sau $5\,000$ tháng, công ty sẽ đạt được lợi nhuận $1$ triệu đô.
	}
\end{ex}
\begin{ex}%[VDC]%[DCHT Toán 11 - KNTT - Nguyễn Hữu Đức] %[1K2Y6-6]
	Một công ty tăng lương cho nhân viên hàng năm bằng cách thêm một số tiền cố định vào lương của họ. Ví dụ: Nếu lương ban đầu của một nhân viên là $ 10 $ triệu đồng và công ty tăng lương $ 2 $ triệu đồng mỗi năm, thì lương của nhân viên sẽ là bao nhiêu nếu làm cho công ty $ 19 $ năm?
	\choice
	{$ 16 $ triệu đồng}
	{$ 26 $ triệu đồng}
	{$ 28 $ triệu đồng}
	{\True $ 46 $ triệu đồng}
	\loigiai{
		Do tăng lương cho nhân viên hàng năm bằng cách thêm một số tiền cố định nên ta có thể sử dụng công thức tính số hạng thứ $ n $ của cấp số cộng
		$ a_n = a_1 + (n - 1)d $.\\
		Ở bài toán này, ta có:\\
		$ a_1 = 10 $ (triệu đồng) là lương ban đầu của nhân viên.\\
		$ d = 2 $ (triệu đồng) là công sai của cấp số cộng.\\
		$ n = 19  $ là số thứ tự của số hạng.\\
		Ta thay các giá trị này vào công thức trên để tính lương của nhân viên sau $ 19 $ năm:\\
		$ a_{19} = 10 + (19 - 1)2 \Rightarrow$
		$ a_{19} = 46 $ (triệu đồng).\\
		Vậy lương của nhân viên sau $ 19 $ năm làm việc cho công ty là $ 46 $ triệu đồng.
	}
\end{ex}

\begin{ex}%[VDC]%[DCHT Toán 11 - KNTT - Nguyễn Hữu Đức] %[1K2Y6-6]
	Tài sản thường bị khấu hao khiến chúng có tuổi thọ hữu ích giới hạn. Ví dụ, nếu một công ty mua một chiếc xe tải với giá $ 35\,000 $ đô la và nó bị khấu hao với tốc độ không đổi là $ 700 $ đô la mỗi tháng, thì sau bao lâu giá trị của nó còn $ 5\,000 $ đô la.
	\choice
	{$ x = 23 $ tháng}
	{\True  $ x = 43 $ tháng}
	{$ x = 41 $ tháng}
	{$ x = 40 $ tháng}
	\loigiai{
		\textit{Cách 1:} Thời gian để giá trị của chiếc xe tải trên được khấu hao xuống còn $5.000 $ đô la có thể được tính bằng cách sử dụng công thức sau:\\
		Giá trị khởi đầu của chiếc xe tải là $35\,000$
		Giá trị cuối cùng của chiếc xe tải là $5\,000$
		Tốc độ khấu hao tương ứng $700$/tháng\\
		Để tìm ra thời gian cần thiết để giá trị của chiếc xe tải giảm xuống còn $5.000$, ta cần tìm số tháng được khấu hao.\\
		Giả sử số tháng cần khấu hao là $ x $ tháng.\\
		Giá trị của chiếc xe tải sau $ x $ tháng khấu hao được tính bằng:\\ 
		$35\,000 - 700x = 5\,000$.\\
		Giải phương trình trên ta có: $ x \approx 43 $ tháng\\
		Vì vậy, sau $ 43 $ tháng, giá trị của chiếc xe tải sẽ giảm xuống còn $5\,000$.
		Ngoài ra ta có thể giải theo cấp số cộng như sau:\\
		\textit{Cách 2:} Ta có thể sử dụng cộng thức tính số hạng thứ $ n $ của cấp số cộng
		$ a_n = a_{1} + (n - 1)d $
		\begin{itemize}
			\item $ u_1 = 35\,000 $ (đô la) là giá trị ban đầu của xe tải.
			\item $ d = -700 $ (đô la) là công sai của cấp số cộng (âm vì giá trị xe tải giảm).
			\item $ a_n = 5\,000 $ (đô la) là giá trị cuối cùng của xe tải.
		\end{itemize}
		Ta thay các giá trị này vào công thức trên để tính số tháng mà xe tải bị khấu hao đến $ 5\,000 $ đô la:
		$$ 5\,000 = 35\,000 + (n - 1)(-700)\Rightarrow n = 43{,}857.$$
		Vậy sau khoảng $ 43{,}857 $ tháng, tức là khoảng $ 3 $ năm và $ 7 $ tháng, giá trị của xe tải sẽ còn khoảng $ 5\,000 $ đô la.
	}
\end{ex}
\begin{ex}%[VDC]%[DCHT Toán 11 - KNTT - Nguyễn Hữu Đức] %[1K2Y6-6]
	Các thiết bị điện tử như máy tính, điện thoại, hoặc máy ảnh thường bị khấu hao nhanh chóng do sự phát triển của công nghệ mới. Ví dụ, nếu một người mua một máy tính Macbook với giá $ 2\,000 $ đô la và nó bị khấu hao với tốc độ không đổi là $ 100 $ đô la mỗi tháng, thì giá trị của Macbook còn lại $ 1\,000 $ đô la sau bao nhiêu tháng?
	%\dapso{$ 11 $ tháng.}
	\choice
	{$ x = 12 $ tháng}
	{$ x = 43 $ tháng}
	{\True  $ x = 11 $ tháng}
	{$ x = 10 $ tháng}
	\loigiai{
		Để giải bài toán này, ta có thể sử dụng công thức tính số hạng thứ $ n $ của cấp số cộng $ a_n = a + (n - 1)d. $\\
		Ở bài toán này, ta có:\\
		$ a = 2\,000 $ (đô la) là giá trị ban đầu của máy tính Macbook.\\
		$ d = -100 $ (đô la) là công sai của cấp số cộng (âm vì giá trị máy tính giảm).\\
		$ a_n = 1\,000 $ (đô la) là giá trị cuối cùng của máy tính Macbook.\\
		Ta thay các giá trị này vào công thức trên để tính số tháng mà máy tính bị khấu hao đến $ 1\,000 $ đô la:
		$$ 1\,000 = 2\,000 + (n - 1)(-100)\Rightarrow n=11. $$
		Vậy sau $ 11 $ tháng, giá trị của máy tính Macbook sẽ còn $ 1\,000 $ đô la.
	}
\end{ex}
\begin{ex}%[VDC]%[DCHT Toán 11 - KNTT - Nguyễn Hữu Đức] %[1K2Y6-6]
	Ban đầu có 1m$^2$ bèo sinh sôi trên mặt hồ biết tốc độ sinh sôi ngày sau hơn ngày trước $ 0{,}5 $m$^2 $. Biết diện tích mặt hồ nước là $ 120 $m$^2 $ hỏi sau bao lâu bèo phủ đầy mặt hồ?
	%\dapso{$ 238 $ ngày}
	\choice
	{$ x = 120 $ tháng}
	{$ x = 143 $ tháng}
	{\True  $ x = 238 $ tháng}
	{$ x = 130 $ tháng}
	\loigiai{
		Giả sử sau $ x $ ngày, diện tích của bèo phủ đầy mặt hồ là $ S m^2 $.
		
		Theo đề bài, ta biết được rằng:
		\begin{itemize}
			\item Tốc độ sinh sôi của bèo là $ 0{,}5 $m$^2 $/ngày.
			\item Ban đầu, diện tích của bèo là  1 m$^2 $.
			\item Diện tích mặt hồ là  $ 120 $m$^2 $.
		\end{itemize}
		Vậy ta có phương trình sau đây:	$ S = 1 + 0{,}5x. $\\
		Điều kiện để bèo phủ đầy mặt hồ là $ S = 120 $.\\
		$ 1 + 0{,}5x = 120 $ hay	$ 0{,}5x = 119 $ $\Rightarrow x = 238 $ ngày.\\
		Vậy sau $ 238 $ ngày, bèo sẽ phủ đầy mặt hồ.
	}
\end{ex}
\begin{ex}%[VDC]%[DCHT Toán 11 - KNTT - Nguyễn Hữu Đức] %[1K2Y6-6]
	Nhà hát lớn Dạ Cỗ Vĩ Lan ở An Cư có hàng ghế đầu kí hiệu dãy A là $50$ chỗ hàng ghế, sau dãy B là $48$ chỗ và như thế hàng sau ít hơn hàng trước $ 2 $ ghế, biết hàng cuối cùng có $ 10 $ ghế. Tính tổng số dãy ghế và tổng số chỗ ngồi?
	%\dapso{ $21$ dãy và $ 630 $ chỗ.}
	\choice
	{\True  $21$ dãy và $ 630 $ chỗ}
	{$20$ dãy và $ 630 $ chỗ}
	{$11$ dãy và $ 630 $ chỗ}
	{$21$ dãy và $ 930 $ chỗ}
	\loigiai{
		Gọi $n$ là số dãy ghế. Theo đề bài, ta có:
		$$
		\begin{cases}
			S=50 + 48 + \cdots + 10 = \dfrac{50+10}{2}n \\
			S=\dfrac{2.50+(n-1)\cdot (-2)}{2}n
		\end{cases}
		$$
		Từ phương trình đầu tiên, ta có:
		$$
		S = 50 + 48 + \cdots + 10 = \frac{50+10}{2}n = 30n.
		$$
		Từ phương trình thứ hai, ta có:
		$$
		S = \frac{2\cdot 50+(n-1)\cdot(-2)}{2}n = (50 - n + 1)n = (51 - n)n.
		$$
		Do đó, ta có:
		$$
		30n = (51 - n)n
		\Rightarrow n=21.$$
		Vậy $n = 21$ dãy ghế và $ 30\cdot 21=630 $ ghế.
		
	}
\end{ex}
\begin{ex}%[VDC]%[DCHT Toán 11 - KNTT - Nguyễn Hữu Đức] %[1K2Y6-6]
	Người ta trồng  cây theo dạng một hình tam giác như sau: hàng thứ nhất trồng $ 1 $ cây, hàng thứ hai trồng $ 3 $ cây, hàng thứ ba trồng $ 5 $ cây,... cứ tiếp tục trồng như thế cho đến khi hết số cây là $ 6\,561 $. Số hàng cây được trồng là bao nhiêu?
	%\dapso{ $ 81 $ hàng.}
	\choice
	{\True  $ 81 $ hàng}
	{$ 16 $ hàng}
	{$ 100 $ hàng}
	{$ 89 $ hàng}
	\loigiai{
		Để giải bài toán này, ta cần tìm số hàng cây được trồng cho đến khi tổng số cây là $ 2023 $. 
		\begin{itemize}
			\item Hàng thứ nhất trồng $ 1 $ cây. 
			\item Hàng thứ hai trồng $ 3 $ cây ($ 1 $ cây $ + 2 $ cây).
			\item Hàng thứ ba trồng $ 5 $ cây ($ 1 $ cây $ + 2 $ cây $ + 2 $ cây).
			\item ...
		\end{itemize}
		Vậy ta thấy rằng số cây trồng trong hàng thứ $n$ là $(n-1)\cdot 2+1$. \\
		Số cây được trồng trong $n$ hàng đầu tiên là: 
		$$1 + 3 + 5 + ... + (2n-1) = n^2.$$ 
		Để tìm số hàng cây được trồng cho đến khi tổng số cây là $ 6561 $, ta giải phương trình sau:\\ 
		$n^2 = 6\,561.$ 
		Vậy số hàng cây được trồng là $ 81 $ hàng.
	}
\end{ex}
\begin{ex}%[VDC]%[DCHT Toán 11 - KNTT - Nguyễn Hữu Đức] %[1K2Y6-6]
	Người ta thả một $ 1 $ m$^2$ lá bèo vào một hồ nước. Kinh nghiệm cho thấy sau $ x $ giờ, bèo sẽ sinh sôi kín cả mặt hồ $ 500 $ m$^2 $. Biết rằng sau mỗi giờ, lượng lá bèo tăng thêm $ 0{,}5 $ m$^2 $ và tốc độ tăng không đổi tìm $ x $?
	%\dapso{ $999$ giờ.}
	\choice
	{$888$ giờ}
	{$777$ giờ}
	{\True  $999$ giờ}
	{$700$ giờ}
	\loigiai{
		Bài toán này có thể giải bằng cách sử dụng công thức tăng trưởng của bèo. Giả sử lượng lá bèo ban đầu là $ 1 $ m$^2$, sau mỗi giờ lượng lá bèo tăng thêm $ 0{,}5 $ m$^2$. Sau $x$ giờ, lượng lá bèo đã phủ kín mặt hồ $ 500 $ m$^2$. Ta có thể viết phương trình sau:
		$$1 + 0{,}5x = 500.$$
		Giải phương trình ta được:
		$$x = \frac{500-1}{0{,}5} \approx 999.$$
		Vậy sau khoảng $999$ giờ (khoảng 41 ngày), lượng lá bèo sẽ phủ kín mặt hồ $ 500 $ m$^2$.
	}
\end{ex}
\Closesolutionfile{ans}
% \begin{indapan}{10}
% 	{ans/ans-1K2-2-dang6}
% \end{indapan}