\section*{KTTX-2}
\setcounter{ex}{0}\setcounter{bt}{0}
\Opensolutionfile{ans}[ans/ans1C2-CD-1]
\setcounter{ex}{0}
\begin{ex}%[1D3Y3-1]%[Bùi Thanh Tấn]%[Vanle Vo]
Xét xem dãy $u_{n}=\dfrac{2 n+3}{5}$ có phải là cấp số cộng hay không? Nếu phải hãy xác định công sai.
\choice
{$d=\varnothing$}
{\True$d=\dfrac{2}{5}$}
{$d=-3$}
{$d=1$}
\loigiai{
Ta có $u_{n+1}-u_{n}=\dfrac{2}{5}$.\\ Dãy $\left(u_{n}\right)$ là cấp số cộng có công sai $d=\dfrac{2}{5}$.
}
\end{ex}

\begin{ex}%[1D3Y2-3]%[Bùi Thanh Tấn]%[Vanle Vo]
Cho dãy số $\left(u_{n}\right)$ có $u_{n}=-n^{2}+n+1$. Khẳng định nào sau đây là đúng?
\choice
{Là một dãy số tăng}
{Là một dãy số bị chặn}
{$u_{n-1}-u_{n}=1$}
{\True Là một dãy số giảm}
\loigiai{
Ta có
\begin{eqnarray*}
u_{n+1}-u_{n}&=&\left[-(n+1)^{2}+n+1+1\right]-\left[-n^{2}+n+1\right]\\&=&-n^{2}-2 n-1+n+2+n^{2}-n-1=-2 n<0,\, \forall n \geq 1.
\end{eqnarray*}
Do đó $\left(u_{n}\right)$ là một dãy giảm.
}
\end{ex}

\begin{ex}%[1D3Y4-2]
Cho các cấp số nhân với $u_1=\dfrac{-1}{2};{u_7}=-32$. Công bội của cấp số nhân là
\choice
{$\pm \dfrac{1}{2}$}
{ $\pm 4$}
{\True$\pm 2$}
{$\pm 1$}
\loigiai{
Ta có $u_7=u_1q^6\Rightarrow -32=-\dfrac{1}{2}q^6\Rightarrow q=\pm 2$.}
\end{ex}


\begin{ex}%[1D3B4-3]
Cho cấp số nhân có $\heva{&
{u_4}-{u_2}=54 \\&
{u_5}-{u_3}=108 }$. Giá trị ${u_1}$ và q của cấp số nhân là
\choice
{\True $u_1=9$ và $q=2$}
{$u_1=9$ và $q=-2$}
{$u_1=-9$ và $q=2$}
{$u_1=-9$ và $q=-2$}
\loigiai{
Ta có
\[\heva{&
{u_4}-{u_2}=54 \\&
{u_5}-{u_3}=108 }\Leftrightarrow \heva{&
{u_1}q^3-{u_1}q=54 \\&
{u_1}q^4-{u_1}q^2=108.}\]
Ta thấy ${u_1}q^3-{u_1}q\ne 0$ nên chia phương trình $(2)$ cho phương trình $(1)$ ta được $q = 2$.\\
Thay $q=2$ vào phương trình $(1)$ ta tìm được $u_1=9$.}
\end{ex}

\begin{ex}%[Vanle Vo]%[1D3B4-2]
Cho dãy số $\left(u_{n}\right)$ với $u_{n}=3^{\tfrac{n}{2}+1}$. Tìm công bội của dãy số $\left(u_{n}\right)$.
\choice
{$q=\dfrac{3}{2}$}
{\True $q=\sqrt{3}$}
{$q=\dfrac{1}{2}$}
{$q=3$}
\loigiai{
Ta có $\dfrac{u_{n+1}}{u_{n}}=\dfrac{3^{\tfrac{n+1}{2}+1}}{3^{\dfrac{n}{2}+1}}=\sqrt{3},\,\forall\, n \in\mathbb N^{*}$. Suy ra dãy sô là cấp số nhân với $u_{1}=3\sqrt{3}$, $q=\sqrt{3}$.
}
\end{ex}

\begin{ex}%[1D3B3-2]
Cho cấp số cộng có $u_1=\dfrac{1}{4},d=-\dfrac{1}{4}$. Chọn khẳng định đúng trong các khẳng định sau đây?
\choice
{$S_5=\dfrac{5}{4}$}
{$S_5=\dfrac{4}{5}$}
{\True $S_5=-\dfrac{5}{4}$}
{$S_5=-\dfrac{4}{5}$}
\loigiai{
Theo giả thiết $S_5=5u_1+10d=5\cdot \dfrac{1}{4}+10\cdot \left(-\dfrac{1}{4}\right)=-\dfrac{5}{4}$.
}
\end{ex}

\begin{ex}%[Vanle Vo]%[1D3B4-6]
Tìm $x$, $y$ biết các số $x+5y$, $5x+2y$, $8x+y$ lập thành cấp số cộng và các số $(y-1)^{2}$, $xy-1$, $(x+1)^{2}$ lập thành cấp số nhân.
\choice
{$(x; y)\in\left\{\left(-\sqrt{3}; \dfrac {3}{2}\right) ;\left(\sqrt{3} ; \dfrac {\sqrt{3}}{2}\right)\right\}$}
{$(x ; y)\in\left\{\left(\sqrt{3} ;-\dfrac {\sqrt{3}}{2}\right) ;\left(-\sqrt{3} ;-\dfrac {\sqrt{3}}{2}\right)\right\}$}
{$(x ; y)\in\left\{\left(\sqrt{3} ; \dfrac {\sqrt{3}}{2}\right) ;\left(\sqrt{3} ; \dfrac {\sqrt{3}}{2}\right)\right\}$}
{\True $(x ; y)\in\left\{\left(-\sqrt{3} ;-\dfrac {\sqrt{3}}{2}\right) ;\left(\sqrt{3} ; \dfrac {\sqrt{3}}{2}\right)\right\}$}
\loigiai{
Ta có hệ $\heva{&x+5y+8x+y=2(5x+2y)\\&(x+1)^{2}(y-1)^{2}=(x y-1)^{2}.}$\\
Giải hệ này ta tìm được
\[(x; y)\in\left\{\left(-\sqrt{3};-\dfrac {\sqrt{3}}{2}\right) ;\left(\sqrt{3} ; \dfrac {\sqrt{3}}{2}\right)\right\}.\]
}
\end{ex}

\begin{ex}%[1D3B3-6]
Chu vi của một đa giác là $213$ cm, số đo các cạnh của nó lập thành một cấp số cộng với công sai $d=7$ cm. Cạnh lớn nhất bằng $53$ cm. Số cạnh của đa giác đó là
\choice
{ $4$}
{$5$}
{\True $6$}
{$7$}
\loigiai{
Gọi số đo các cạnh của đa giác lần lượt là $u_1$, $u_2$, $\ldots$, $u_n$ với $n\geq 3$, $n\in \mathbb{N}$. Ta có
\begin{eqnarray*}
&&\heva{&S_n=213\\&u_n=53}\\
&\Leftrightarrow & \heva{&\dfrac{n}{2}\left(u_1+53\right)=213\\ &u_1+7(n-1)=53}\\
&\Leftrightarrow & \heva{&nu_1+53n=426\\&u_1+7n=60}\\
&\Leftrightarrow & \heva{&n(60-7n)+53n=426\\&u_1=60-7n}\\
&\Leftrightarrow & \heva{&-7n^2+113n-426=0\\&u_1=60-7n}\\
&\Leftrightarrow &\heva{&\hoac{&n=6~\text{(nhận)}\\ &n=\dfrac{71}{7}~\text{(loại)}}\\ &u_1=60-7n}\\
&\Leftrightarrow &\heva{&n=6\\&u_1=18.}
\end{eqnarray*}
Vậy số cạnh của đa giác là $6$.
}
\end{ex}

\begin{ex}%[1D3B2-3]
Phát biểu nào dưới đây về dãy số $\left(a_{n}\right)$ được cho bởi $a_{n}=2^{n}+n$ là đúng?
\choice
{Dãy số $\left(a_{n}\right)$ là dãy số giảm}
{\True Dãy số $\left(a_{n}\right)$ là dãy số tăng}
{Dãy số $\left(a_{n}\right)$ là dãy không tăng}
{Dãy số $\left(a_{n}\right)$ là dãy không tăng và không giảm}
\loigiai{
Ta có $a_{n+1} - a_n  = 2^{n+1} +n+1 - 2^n -n = 2^n+1 >0$, $\forall n \in \mathbb{N}^*$. \\
Suy ra dãy số $\left(a_{n}\right)$ là dãy số tăng.
}
\end{ex}

\begin{ex}%[Vanle Vo]%[1D3B4-3]
Cho dãy số $\left(u_{n}\right)$ với $u_{n}=3^{\tfrac{n}{2}+1}$. Số $19683$ là số hạng thứ mấy của dãy số.
\choice
{$15$}
{\True $16$}
{$19$}
{$17$}
\loigiai{
Ta có $u_{n}=19683 \Leftrightarrow 3^{\tfrac{n}{2}+1}=3^{9} \Leftrightarrow \dfrac{n}{2}+1=9 \Leftrightarrow n=16$.\\
Vậy số $19683$ là số hạng thứ $16$ của cấp số.
}
\end{ex}

\begin{ex}%[1D3B2-2]
Cho dãy số $(u_n)$ có $u_n=(-1)^{n+1} \cdot \cos  \dfrac{2 \pi}{n} $. Khi đó $u_{12}$ bằng.
\choice
{$\dfrac{1}{2}$}
{$\dfrac{\sqrt{3}}{2}$}
{$-\dfrac{1}{2}$}
{\True $-\dfrac{\sqrt{3}}{2}$}
\loigiai{
Ta có $u_{12}=(-1)^{12+1} \cos \dfrac{2 \pi}{12}=-\dfrac{\sqrt{3}}{2}$.
}
\end{ex}

\begin{ex}%[1D3B4-5]
Tổng 10 số hạng đầu của một cấp số nhân có $u_1=4,{u_{10}}=2048$ là
\choice
{${{S}_{10}}=8184$}
{${{S}_{10}}=4092$}
{\True ${{S}_{10}}=12276$}
{${{S}_{10}}=6138$}
\loigiai{
Ta có ${u_{10}}=u_1\cdot q^9\Rightarrow q=2$. Do đó ${{S}_{10}}=u_1\cdot \dfrac{{q^{10}}-1}{q-1}=4092$.}
\end{ex}

\begin{ex}%[Ngân hàng hỏi 11 - lần 2]%[1D3B3-2]
Dãy số $(u_n)$ là cấp số cộng thỏa mãn $\heva{&u_2+u_5-u_3=10\\&u_1+u_6=18}$. Số hạng đầu và công sai của cấp số cộng là
\choice
{$u_1=3$, $d=2$}
{\True $u_1=4$, $d=2$}
{$u_1=2$, $d=4$}
{$u_1=1$, $d=2$}
\loigiai
{Áp dụng $u_n=u_1+(n-1)d$, ta có
\begin{eqnarray*}
\heva{&u_2+u_5-u_3=10\\&u_1+u_6=18}\Leftrightarrow \heva{&u_1+3d=10\\&2u_1+5d=18}\Leftrightarrow \heva{&u_1=4\\&d=2.}
\end{eqnarray*}
Vậy $u_1=4$, $d=2$.
}
\end{ex}

\begin{ex}%[1D3B4-5]
Cho $S=3+3\cdot 2+3\cdot2^2+\ldots+3\cdot2^n$. Khẳng định nào sau đây đúng với mọi $n$ nguyên dương?
\choice
{$S=3\left( 2^n-1 \right)$}
{$S=3\left( {2^{n+1}}+1 \right)$}
{\True $S=3\left( {2^{n+1}}-1 \right)$}
{$S=3\left( {2^{n-1}}-1 \right)$}
\loigiai{
Ta có $1$, $2$, $2^2$,\ldots, $2^n$ là cấp số nhân với $u_1=1$, $q=2$ nên \[1+2+2^2+\ldots+2^n=\dfrac{1-{2^{n+1}}}{1-2}={2^{n+1}}-1.\]
Vậy $S=3\left( 1+2+2^2+\ldots+2^n \right)=3\left( {2^{n+1}}-1 \right)$.}
\end{ex}

\begin{ex}%[1D3B3-5]
Cho CSC $(u_n)$ có $S_n=3n^2-2n$. Công thức tổng quát của CSC trên là
\choice
{$u_n=20n-19$}
{\True $u_n=6n-5$}
{$u_n=4n-3$}
{$u_n=7n-5$}
\loigiai{
Gọi $d$ là công sai của CSC. Ta có
\[ S_n=3n^2-2n \Leftrightarrow \dfrac{n}{2}[2u_1+(n-1)d]=3n^2-2n \Leftrightarrow 2u_1-d+nd=6n-4\]
Do đó
\[ \heva{& 2u_1-d=-4\\&d=6} \Leftrightarrow \heva{&u_1=1\\&d=6.}\]
Vậy số hạng tổng quát của CSC là $u_n=u_1+(n-1)d=1+6(n-1)=6n-5$, $\forall n\geq 1$.
}
\end{ex}

\begin{ex}%[1D3B3-2]
Cho cấp số cộng $\left(u_n\right)$ thỏa mãn $\heva{
&u_2+u_5=42\\
&u_3+u_{10}=66\\
}$. Tổng của $346$ số hạng đầu là
\choice
{\True $242546$}
{$242000$}
{$241000$}
{$240000$}
\loigiai{
Theo đề ta có
\[\heva{
&u_2+u_5=42\\
&u_3+u_{10}=66\\
}\Leftrightarrow\heva{
& 2u_1+5d=42\\
& 2u_1+11d=66\\
}\Leftrightarrow\heva{
&u_1=11\\
& d=4\\
}\Rightarrow S_{346}=\dfrac{346}{2}\left(2\cdot11+345\cdot4\right)=242546.\]
}
\end{ex}

\begin{ex}%[1D3B3-1]%[Bùi Thanh Tấn]%[Vanle Vo]
Cho cấp số cộng $\left(u_{n}\right)$ thỏa mãn $\heva{&u_{2}-u_{3}+u_{5}=10 \\ &u_{4}+u_{6}=26}$. Xác định công sai.
\choice
{\True$d=3$}
{$d=5$}
{$d=6$}
{$d=4$}
\loigiai{
Ta có $\heva{&u_{1}+d-\left(u_{1}+2 d\right)+u_{1}+4 d=10 \\& u_{1}+3 d+u_{1}+5 d=26} \Leftrightarrow\heva{&u_{1}+3 d=10 \\& u_{1}+4 d=13} \Leftrightarrow \heva{&u_{1}=1\\& d=3.}$
}
\end{ex}

\begin{ex}%[1D3B3-2]
Cho CSC có số hạng tổng quát $u_n=5+\dfrac{n+1}{2}$. Công sai của CSC là
\choice
{$d=1$}
{$d=\dfrac{3}{2}$}
{\True $d=\dfrac{1}{2}$}
{$d=2$}
\loigiai{
Gọi $d$ là công sai của CSC. Ta có\\
\[d=u_n-u_{n-1}=\dfrac{n+1}{2}-\dfrac{n}{2}=\dfrac{1}{2}.\]
Vậy công sai của CSC là $d=\dfrac{1}{2}$.
}
\end{ex}

\begin{ex}%[Dự án ngân hàng hỏi-Mui Doan]%[1D3B2-4]
Xét tính tăng, giảm và bị chặn của dãy số $ (u_n) $ biết $ u_n=\dfrac{2^n}{n!} $, ta thu được kết quả
\choice
{Dãy số tăng, bị chặn trên}
{Dãy số tăng, bị chặn dưới}
{\True Dãy số giảm, bị chặn}
{Dãy số không tăng, không giảm, không bị chặn}
\loigiai{
Ta có   $ \dfrac{u_{n+1}}{u_n}=\dfrac{\dfrac{2^{n+1}}{(n+1)!}}{\dfrac{2^n}{n!}} =\dfrac{2^{n+1}}{(n+1)!}\cdot \dfrac{2^n}{n!}=\dfrac{2}{n+1}<1,\forall n\geq 1$.\\
Mà $ u_n>0, \forall n $ nên $ u_{n+1}<u_n, \forall n\geq 1\Rightarrow $ dãy $ (u_n) $ là dãy số giảm.\\
Vì $ 0<u_n\leq u_1=2, \forall n\geq 1 $ nên dãy $ (u_n) $ là dãy bị chặn.
}
\end{ex}

\begin{ex}%[1D3B3-2]
Cho dãy số $(u_n)$ có $u_1+u_2+\ldots + u_n=\dfrac{n(7-3n)}{2}$. Số hạng tổng quát của $(u_n)$ là
\choice
{\True $u_n=5-3n$, $n\geq 1$}
{$u_n=5+3n$, $n\geq 1$}
{$u_n=2+5n$, $n\geq 1$}
{$u_n=2-n$, $n\geq 1$}
\loigiai{
Gọi $d$ là công sai của CSC. Ta có
\begin{eqnarray*}
&& u_1+u_2+\ldots + u_n=\dfrac{n(7-3n)}{2}\\
&\Leftrightarrow&  \dfrac{n}{2}\left[2u_1+(n-1)d \right]= \dfrac{n(7-3n)}{2}\\
&\Leftrightarrow& 2u_1-d +nd =7-3n\\
&\Leftrightarrow& \heva{&2u_1-d=7\\&d=-3}\\
&\Leftrightarrow& \heva{&u_1=2\\&d=-3.}
\end{eqnarray*}
Vậy $u_n=u_1+(n-1)d=2-3(n-1)=5-3n$, $\forall n\geq 1$.
}
\end{ex}

\begin{ex}%[1D3B3-3]
Độ dài ba cạnh của một tam giác vuông lập thành một cấp số cộng. Nếu cạnh trung bình bằng $6$ thì công sai của cấp số cộng này là
\choice
{$7{,}5$}
{$4{,}5$}
{$0{,}5$}
{\True $1{,}5$}
\loigiai{
Theo giả thiết $\left(6-d\right)^2+6^2=\left(6+d\right)^2\Leftrightarrow 36-12d=12d\Rightarrow d=\dfrac{36}{24}=\dfrac{3}{2}.$
}
\end{ex}

\begin{ex}%[1D3B2-4]%[Bùi Thanh Tấn]%[Vanle Vo]
Xét tính bị chặn của dãy số $u_{n}=4-3 n-n^{2}$.
\choice
{Bị chặn}
{Không bị chặn}
{\True Bị chặn trên}
{Bị chặn dưới}
\loigiai{
Ta có $u_{n}=\dfrac{25}{4}-\left(n+\dfrac{3}{2}\right)^{2}<\dfrac{25}{4} $ nên $\left(u_{n}\right)$ bị chặn trên; dãy $\left(u_{n}\right)$ không bị chặn dưới.
}
\end{ex}

\begin{ex}%[1D3B3-5]
Cho cấp số cộng $\left(u_{n}\right)$ có tổng $5$ số hạng đầu tiên bằng $10$.  Giá trị $u_{3}$ là
\choice
{$4$}
{\True $2$}
{$3$}
{$5$}
\loigiai{
Ta có $u_1+u_2 + \cdots +u_5 =10 \Leftrightarrow 5u_1 + 10d =10 \Leftrightarrow u_3=u_1 +2d =2$.
}
\end{ex}

\begin{ex}%[1D3B4-5]
Trong một cấp số nhân gồm các số hạng dương, hiệu số giữa số hạng thứ $ 5 $ và thứ $ 4 $ là $ 576 $ và hiệu số giữa số hạng  thứ $ 2 $ và số hạng đầu tiên là $ 9 $. Tìm tổng $ 5 $ số hạng đầu tiên của các cấp số nhân này
\choice
{$ 1061 $}
{\True $ 1023 $}
{$ 1024 $}
{$ 768 $}
\loigiai{
Giả sử cấp số nhân trên có số hạng đầu tiên là $ u_1 $.\\
Ta có $u_{5}=q^{4} \cdot u_{1}$ ; $u_{4}=q^{3} \cdot u_{1} \Rightarrow u_{5}-u_{4}=\left(q^{4}-q^{3}\right) u_{1}=576=q^{3}(q-1) u_{1}=576$.\\
Lại có $ u_{2}=q . u_{1} \Rightarrow u_{2}-u_{1}=u_{1}(q-1)=9 . \text { Do đó } q^{3} .9=576 \Rightarrow q^{3}=64 \Rightarrow q=4 \Rightarrow u_{1}=3 $.\\
Suy ra $ S=\dfrac{1-q^5}{1-q}u_1=1023 $.
}
\end{ex}

\begin{ex}%[1D3B4-3]
Cho cấp số nhân $( u_n )$ với $u_1=-1;\,q=\dfrac{-1}{10}$. Số $\dfrac{1}{10^{103}}$ số hạng thứ mấy của $( u_n )$?
\choice
{ Số hạng thứ $105$}
{ Không là số hạng của cấp số đã cho}
{ Số hạng thứ $103$}
{\True Số hạng thứ $104$}
Ta có $u_n=u_1\cdot q^{n-1}$ nên $\dfrac{1}{10^{103}}=-1\cdot \left(-\dfrac{1}{10}\right)^{n-1}\Leftrightarrow (-\dfrac{1}{10})^n=\dfrac{1}{10^{104}}\Rightarrow n=104.$
\loigiai{

}
\end{ex}

\begin{ex}%[1D3B4-4]
Cho cấp số nhân $\dfrac{-1}{5} ; a ; \dfrac{-1}{125} .$ Giá trị của $a$ là
\choice
{$a=\pm \dfrac{1}{\sqrt{5}}$}
{\True $a=\pm \dfrac{1}{25}$}
{$a=\pm \dfrac{1}{5}$}
{$a=\pm 5$}
\loigiai{
Dãy số $\dfrac{-1}{5} ; a ; \dfrac{-1}{125} $ là cấp số nhân khi và chỉ khi $a^2= \left (\dfrac{-1}{5}\right )  \cdot \left ( \dfrac{-1}{125} \right ) $ $\Rightarrow a= \pm \dfrac{1}{25}$.
}
\end{ex}

\begin{ex}%[Dự án ngân hàng hỏi-Mui Doan]%[1D3B2-3]
Xét tính tăng, giảm của dãy số  $ \heva{&u_1=1\\&u_{n+1}=\sqrt[3]{u_n^3+1}, n\geq 1} $. Ta thu được kết quả
\choice
{\True Dãy số tăng}
{Dãy số giảm}
{Dãy số không tăng, không giảm}
{Dãy số khi tăng, khi giảm}
\loigiai{
Ta có   $u_{n+1}=\sqrt[3]{u_n^3+1}\Rightarrow u_{n+1}>\sqrt[3]{u_n^3}=u_n, \forall n\in\mathbb{N^*}\Rightarrow (u_n)$ là dãy số tăng.
}
\end{ex}

\begin{ex}%[1D3B3-2]
Cho cấp số cộng $\left(u_n\right)$ có $u_n=2n+3$. Biết $S_n=320,$ giá trị của $n$ là
\choice
{$n=16$ hoặc $n=-20$}
{$n=15$}
{$n=20$}
{\True $n=16$}
\loigiai{
Ta có $u_1=5$ suy ra $S_n=\dfrac{n\left(5+2n+3\right)}{2}=n^2+4n \Leftrightarrow n^2+4n-320=0\Leftrightarrow \hoac{&n=16\left(\text{nhận}\right)\\&n=-20\left(\text{loại}\right).}$
}
\end{ex}

\begin{ex}%[1D3B4-5]
Cho một cấp số nhân biết $u_1=3$, $q=2$. Tổng của 10 số hạng đầu tiên của cấp số nhân là
\choice
{$3\cdot \left( 1-2^9 \right)$}
{$3\cdot\left( 1-{2^{10}} \right)$}
{$-3\cdot\left( 2^9-1 \right)$}
{\True $3\cdot\left( {2^{10}}-1 \right)$}
\loigiai{
Ta có ${{S}_{10}}=u_1\cdot \dfrac{1-{q^{10}}}{1-q}=3\cdot\dfrac{1-{2^{10}}}{1-2}=3\cdot\left( {2^{10}}-1 \right)$.}
\end{ex}

\begin{ex}%[Dự án ngân hàng hỏi-Mui Doan]%[1D3B4-5]
Tổng $ S=4\cdot 5^{100}\cdot \left(\dfrac{1}{5}+ \dfrac{1}{5^2}+\dfrac{1}{5^3}+\cdot +\dfrac{1}{5^{100}}\right)+1$ có kết quả bằng
\choice
{$ 5^{100}-1 $}
{\True $ 5^{100} $}
{$ 5^{101}-1 $}
{$ 5^{101} $}
\loigiai{
Đặt $ M=\dfrac{1}{5}+ \dfrac{1}{5^2}+\dfrac{1}{5^3}+\cdot +\dfrac{1}{5^{100}} $.\\
Ta có $ 5M=1+\dfrac{1}{5}+ \dfrac{1}{5^2}+\dfrac{1}{5^3}+\cdot +\dfrac{1}{5^{99}} $ .\\
$ \Rightarrow 5M-M=1-\dfrac{1}{5^{100}} \Rightarrow 4M=1-\dfrac{1}{5^{100}}\Rightarrow M=\dfrac{5^{100}-1}{4\cdot 5^{100}}\Rightarrow S=4\cdot 5^{100}\cdot \dfrac{5^{100}-1}{4\cdot 5^{100}}+1=5^{100}$.
}
\end{ex}

\begin{ex}%[1D3B4-3]
Cho cấp số nhân $ (u_n) $ có $ u_1=24 $ và $ \dfrac{u_4}{u_{11}}=16384 $. Sô hạng $ u_{17} $ là
\choice
{$ \dfrac{3}{67108864} $}
{$ \dfrac{3}{368435456} $}
{\True $ \dfrac{3}{536870912} $}
{$ \dfrac{3}{2147483648} $}
\loigiai{
Ta có $ \dfrac{u_{4}}{u_{11}}=16384 \Leftrightarrow \dfrac{u_{1} \cdot q^{3}}{u_{1} \cdot q^{10}}=\dfrac{1}{q^{7}} \Rightarrow q=\sqrt[7]{16384}=4 $.\\
Do vậy $ u_{17}=u_{1} \cdot q^{16}=24 \cdot\left(\dfrac{1}{4}\right)^{16}=\dfrac{3}{536870912} $.
}
\end{ex}

\begin{ex}%[Ngân hàng hỏi 11 - lần 2]%[1D3B4-1]
Trong các dãy số dưới đây, dãy số nào là cấp số cộng?
\choice
{Dãy số $(a_n)$ với $a_n=3^n$, $ \forall n \in \mathbb{N}^*$}
{Dãy số $(b_n)$ với $b_1=1$, $b_{n+1}=2b_n+1$, $ \forall n \in \mathbb{N}^*$}
{\True Dãy số $(c_n)$ với $c_n=(2n+1)^2-4n^2$, $ \forall n \in \mathbb{N}^*$}
{Dãy số $(d_n)$ với $d_1=1$, $d_{n+1}=\dfrac{2020}{d_n+1}$, $ \forall n \in \mathbb{N}^*$}
\loigiai
{\begin{itemize}
\item Ta có $a_{n+1}=3^{n+1}$. Suy ra $a_{n+1}-a_n=3^{n+1}-3^n=2\cdot 3^n$.\\
Vì $a_{n+1}-a_n$ còn phụ thuộc vào $n$ nên dãy $(a_n)$ không là cấp số cộng.
\item Ta có $b_2=3$, $b_3=7$, $b_4=15$. Suy ra $b_2-b_1=2$ và $b_4-b_3=8$ nên $b_2-b_1\neq b_4-b_3$.\\
Do đó dãy $(b_n)$ không là cấp số cộng.
\item Ta có $c_{n+1}=(2n+3)^2-4(n+1)^2$. Suy ra
\begin{center}
$c_{n+1}-c_n=(2n+3)^2-4(n+1)^2-(2n+1)^2-4n^2=2(4n+4)-4(2n+1)=4$.
\end{center}
Vậy dãy $(c_n)$ là cấp số cộng.
\item Ta có $d_2=1010$, $d_3=\dfrac{2020}{1011}$, $d_4=\dfrac{2020}{d_3+1}$ nên $d_2-d_1\neq d_4-d_3$.\\
Do đó dãy $(d_n)$ không là cấp số cộng.
\end{itemize}
}
\end{ex}


\begin{ex}%[1D3K2-4]
Trong các dãy số sau dãy nào bị chặn trên
\choice
{$u_n = 3n^2+1$}
{ $u_n=\dfrac{n+2}{n+1}$}
{ $u_n=(-1)^n n^2$}
{ \True $u_n=3n+2$}
\loigiai{
Ta có $u_n=3n+2$ là dãy số tăng. Thật vậy, \\
$u_{n+1}-u_n=3(n+1)-3n=3>0 \Rightarrow u_{n+1}>u_n$.
}
\end{ex}

\begin{ex}%[1D3K3-2]
Cho cấp số cộng $\left(u_n\right)$ thỏa mãn $\heva{
&{u_5}+3u_3-u_2=-21\\
& 3u_7-2u_4=-34\\
}$. Giá trị của biểu thức $S=u_4+u_5+\cdots+u_{30}$ là
\choice
{\True $-1242$}
{$-1222$}
{$-1276$}
{$-1286$}
\loigiai{
Ta có
\begin{eqnarray*}
& &\heva{
&{u_5}+3u_3-u_2=-21\\
& 3u_7-2u_4=-34} \\
&\Leftrightarrow &\heva{
&{u_1}+4d+3\left(u_1+2d\right)-\left(u_1+d\right)=-21\\
& 3\left(u_1+6d\right)-2\left(u_1+3d\right)=-34} \\
&\Leftrightarrow &\heva{
& 3u_1+9d=-21\\
&{u_1}+12d=-34} \\
&\Leftrightarrow &\heva{
&{u_1}=2\\
& d=-3.\\
}
\end{eqnarray*}
$S=u_4+u_5+\cdots +u_{30}=S_{30}-S_3=\dfrac{30}{2}\left(2\cdot2+29\cdot\left(-3\right)\right)-\dfrac{3}{2}\left(2\cdot2-2\cdot3\right)=-1242$.
}
\end{ex}

\begin{ex}%[Vanle Vo]%[1D3K4-3]
Cho cấp số nhân $\left(u_{n}\right)$ có các số hạng khác không, tổng các giá trị $u_{1}$ thỏa mãn $\heva{&u_{1}+u_{2}+u_{3}+u_{4}=15\\& u_{1}^{2}+u_{2}^{2}+u_{3}^{2}+u_{4}^{2}=85.}$
\choice
{$4$}
{\True $9$}
{$6$}
{$10$}
\loigiai{
Ta có
\begin{eqnarray*}
&&\heva{&u_1(1+q+q^{2}+q^3)=15\\&u_1^2(1+q^2+q^4+q^6=85}\Leftrightarrow\heva{&u_{1} \dfrac{q^{4}-1}{q-1}=15 \\&u_{1}^{2}\dfrac{q^{8}-1}{q^{2}-1}=85}\\
&\Rightarrow&\left(\dfrac{q^{4}-1}{q-1}\right)^{2}\left(\dfrac{q^{2}-1}{q^{8}-1}\right)=\dfrac{45}{17} \Leftrightarrow \dfrac{\left(q^{4}-1\right)(q+1)}{(q-1)\left(q^{4}+1\right)}=\dfrac{45}{17} \Leftrightarrow\hoac{&q=2 \\&q=\dfrac{1}{2}
.}
\end{eqnarray*}
Từ đó ta tìm được $u_{1}=1$, $u_{1}=8$.
}
\end{ex}

\begin{ex}%[1D3K3-2]
Cho cấp số cộng $u_n=5n-2$. Biết $S_n=16040$, số số hạng của cấp số cộng là
\choice
{$79$}
{$3024$}
{\True $80$}
{$100$}
\loigiai{
Ta có cấp số cộng: $u_n=5n-2$ nên $u_1=3,u_2=8,...\Rightarrow d=5$.
\begin{eqnarray*}
& & {S_n}=16040\\
&\Leftrightarrow &\dfrac{n}{2}\left[2u_1+\left(n-1\right)d\right]=16040 \\
&\Leftrightarrow &\dfrac{n}{2}\left[2\cdot3+\left(n-1\right)\cdot5\right]=16040 \\
&\Leftrightarrow &5n^2+n-32080=0 \\
&\Leftrightarrow &\hoac{
& n=80\\
& n=-\dfrac{401}{5}\text{(loại})}\\
&\Leftrightarrow & n=80.
\end{eqnarray*}
}
\end{ex}

\begin{ex}%[1D3K4-3]
Tứ giác $ABCD$ có số đo các góc lập thành một cấp số nhân theo thứ tự $A$, $B$, $C$, $D$. Biết rằng số đo góc $C$ gấp bốn lần số đo góc $A$. Số đo góc $A$ của tứ giác đó bằng
\choice
{\True $24^\circ $}
{$48^\circ $}
{$144^\circ $}
{$72^\circ $}
\loigiai{
Ta có $\heva{&A+B+C+D=360\\&B=Aq\\&C=Aq^2=4A\\&D=Aq^3}\Leftrightarrow \heva{&A(1+q+q^2+q^3)=360\\&B=Aq\\&C=Aq^2\\&D=Aq^3.}$\\
Từ $Aq^2=4A$ ta được $q=2$ thế vào phương trình đầu ta được $A=24^\circ$.
}
\end{ex}


\begin{ex}%[1D3G3-5]%[Bùi Thanh Tấn]%[Vanle Vo]
Cho một cấp số cộng $\left(u_{n}\right)$ là cấp số cộng có $u_{1}=1$ và tổng $100$ số hạng đầu bằng $24850$. Tính $S=\dfrac{1}{u_{1} u_{2}}+\dfrac{1}{u_{2} u_{3}}+\ldots+\dfrac{1}{u_{49} u_{50}}$.
\choice
{$S=\dfrac{9}{246}$}
{$S=\dfrac{4}{23}$}
{$S=123$}
{\True $S=\dfrac{49}{246}$}
\loigiai{
Gọi $d$ là công sai của cấp số đã cho.\\
Ta có $S_{100}=50\left(2 u_{1}+99 d\right)=24850 \Rightarrow d=\dfrac{497-2 u_{1}}{99}=5$.
\begin{eqnarray*}
\Rightarrow 5 S&=&\dfrac{5}{u_{1} u_{2}}+\dfrac{5}{u_{2} u_{3}}+\ldots+\dfrac{5}{u_{49} u_{50}} \\
&=&\dfrac{u_{2}-u_{1}}{u_{1} u_{2}}+\dfrac{u_{3}-u_{2}}{u_{2} u_{3}}+\ldots+\dfrac{u_{50}-u_{49}}{u_{49} u_{50}} \\
&=&\dfrac{1}{u_{1}}-\dfrac{1}{u_{2}}+\dfrac{1}{u_{2}}-\dfrac{1}{u_{3}}+\ldots+\dfrac{1}{u_{48}}-\dfrac{1}{u_{49}}+\dfrac{1}{u_{49}}-\dfrac{1}{u_{50}} \\
&=&\dfrac{1}{u_{1}}-\dfrac{1}{u_{50}}=\dfrac{1}{u_{1}}-\dfrac{1}{u_{1}+49 d}=\dfrac{245}{246}. \\
\Rightarrow S&=&\dfrac{49}{246}.
\end{eqnarray*}
}
\end{ex}

\begin{ex}%[1D3G3-5]%[Bùi Thanh Tấn]%[Vanle Vo]
Cho cấp số cộng $\left(u_{n}\right)$ thỏa mãn $\heva{&u_{2}-u_{3}+u_{5}=10 \\ &u_{4}+u_{6}=26}$.Tính tổng \[S=u_{5}+u_{7}+\ldots+u_{2011}.\]
\choice
{$S=3028123$}
{$S=3021233$}
{\True $S=3028057$}
{$S=3028332$}
\loigiai{
Ta có $\heva{&u_{1}+d-\left(u_{1}+2 d\right)+u_{1}+4 d=10 \\& u_{1}+3 d+u_{1}+5 d=26} \Leftrightarrow\heva{&u_{1}+3 d=10 \\& u_{1}+4 d=13} \Leftrightarrow u_{1}=1,\, d=3$.\\
Ta có $u_{5},\, u_{7}, \ldots,\, u_{2011}$ lập thành cấp số cộng với công sai $d=6$ và có $1003$ số hạng nên \[S=\dfrac{1003}{2}\left(2 u_{5}+1002.6\right)=3028057.\]
}
\end{ex}

\begin{ex}%[1D3G3-6]
Ba cạnh của một tam giác vuông có độ dài là các số nguyên dương lập thành một CSC. Một cạnh có thể có độ dài bằng
\choice
{ $22$}
{$58$}
{\True $81$}
{$91$}
\loigiai{
Gọi độ dài ba cạnh của tam giác lần lượt là $0<a<b<c$. \\
Vì $a$, $b$, $c$ theo thứ tự lập thành cấp số cộng nên
\begin{eqnarray*}
&&a+c=2b\\
&\Leftrightarrow& a^2+2ac+c^2=2(c^2-a^2)\\
&\Leftrightarrow&-c^2+2ac+3a^2=0\\
&\Leftrightarrow& \hoac{& c=-a \\ &c= 3a.}
\end{eqnarray*}
$\bullet ~ c= -a$ vô lí vì $a$, $c>0$.\\
$\bullet ~ c=3a$ nên $c$ chia hết cho $3$. \\
Chỉ có đáp án $81$ thỏa điều kiện trên.\\
Vậy một cạnh của tam giác là $81$.
}
\end{ex}

\Closesolutionfile{ans}