\setcounter{dang}{0}
\setcounter{ex}{0}
\setcounter{bt}{0}
\setcounter{vd}{0}
\section*{Ôn tập chương 2}
\Opensolutionfile{ans}[ans/ans-1K2-Ontapchuong2]
\begin{ex}%[1K2Y5-2]
	Cho dãy số $\left(u_n\right)$, biết $u_n=\left(-1\right)^n.2n$. Mệnh đề nào sau đây sai?
	\choice
	{$u_1=-2$}
	{$u_2=4$}
	{$u_3=-6$}
	{\True $u_4=-8$}
	\loigiai{
		Thay trực tiếp vào kiểm tra, ta có
		\begin{eqnarray*}
			u_1&=&-2.1=-2\\
			u_2&=&(-1)^2.2.2=4\\
			u_3&=&(-1)^3.2.3=-6\\
			u_4&=&(-1)^4.2.4=8.
		\end{eqnarray*}
	}
\end{ex}
\begin{ex}%[1K2Y5-2]
	Cho dãy số $\left(u_n\right)$, biết $u_n=\left(-1\right)^n.\dfrac{2^n}{n}$. Tìm số hạng $u_3$.
	\choice
	{$u_3=\dfrac{8}{3}$}
	{$u_3=2$}
	{$u_3=-2$}
	{\True $u_3=-\dfrac{8}{3}$}
	\loigiai{
		Thay trực tiếp vào kiểm tra, ta có
		\begin{center}
			$u_3=(-1)^3.\dfrac{2^3}{3}=-\dfrac{8}{3}$.
		\end{center}
	}
\end{ex}
\begin{ex}%[1K2Y5-2]
	Cho dãy số $\left(u_n\right)$, biết $u_n=\dfrac{2n+5}{5n-4}$. Số $\dfrac{7}{12}$ là số hạng thứ mấy của dãy số?
	\choice
	{\True $8$}
	{$6$}
	{$9$}
	{$10$}
\end{ex}
\loigiai{
	Ta có
	\allowdisplaybreaks
	\begin{eqnarray*}
		&&u_n=\dfrac{2n+5}{5n-4}\\
		&\Leftrightarrow&\dfrac{7}{12}=	\dfrac{2n+5}{5n-4}\\
		&\Leftrightarrow&24n+60=35n-28\\
		&\Leftrightarrow&11n=88\\
		&\Leftrightarrow&n=8.
	\end{eqnarray*}
	Vậy số $\dfrac{7}{12}$ là số hạng thứ 8.
}
\begin{ex}%[1K2Y5-2]
	Cho dãy số $\left(u_n\right)$, biết $u_n=2^n$. Tìm số hạng $u_{n+1}$.
	\choice
	{\True $u_{n+1}=2^n.2$}
	{$u_{n+1}=2^n+1$}
	{$u_{n+1}=2\left(n+1\right)$}
	{$u_{n+1}=2^n+2$}
	\loigiai{
		Ta có
	}
	\loigiai{
		Thay $n$ bằng $n+1$ trong công thức $u_n$ ta được
		\allowdisplaybreaks
		\begin{eqnarray*}
			u_{n+1}&=&2^{n+1}\\
			& =&2.2^n.
		\end{eqnarray*}
	}
\end{ex}
\begin{ex}%[1K2B5-2]
	Cho dãy số $\left(u_n\right)$, biết $u_n=5^{n+1}$. Tìm số hạng $u_{n-1}$.
	\choice
	{$u_{n-1}=5^{n-1}$}
	{\True $u_{n-1}=5^{n}$}
	{$u_{n-1}=5.5^{n+1}$}
	{$u_{n-1}=5.5^{n-1}$}
	\loigiai{
		Thay $n$ bằng $n-1$ trong công thức $u_n$ ta được
		\allowdisplaybreaks
		\begin{eqnarray*}
			u_{n-1}& = &5^{n-1+1}\\
			& = &5^n.
		\end{eqnarray*}
	}
\end{ex}
\begin{ex}%[1K2Y5-1]
	Cho dãy số có các số hạng đầu là $-2;0;2;4;6;...$. Số hạng tổng quát của dãy số này là công thức nào dưới đây?
	\choice
	{$u_n=-2n$}
	{$u_n=n-2$}
	{$u_n=-2\left(n+1\right)$}
	{\True $u_n=2n-4$}
	\loigiai{
		Kiểm tra $u_1=-2$ ta loại các đáp án B và C. Tương tự kiểm tra $u_2=0$ ta loại đáp án A.
	}
\end{ex}
\begin{ex}%[1K2B5-1]
	Cho dãy số $\left(u_n\right)$, được xác định $\heva{&u_1=\dfrac{1}{2}\\&u_{n+1}=u_n-2}$. Số hạng tổng quát $u_n$ của dãy số là số hạng nào dưới đây?
	\choice
	{$u_n=\dfrac{1}{2}+2\left(n-1\right)$}
	{\True $u_n=\dfrac{1}{2}-2\left(n-1\right)$}
	{$u_n=\dfrac{1}{2}-2n$}
	{$u_n=\dfrac{1}{2}+2n$}
	\loigiai{
		Ta có
		\begin{center}
			$\heva{&u_1=\dfrac{1}{2}\\&u_{n+1}=u_n-2}\Rightarrow\heva{&u_1=\dfrac{1}{2}\\&u_2=-\dfrac{3}{2}\\&u_3=-\dfrac{7}{2}}$
		\end{center} 
		Ta thấy chỉ có đáp án B đều thoả mãn.
	}
\end{ex}
\begin{ex}%[1K2B5-1]
	Cho dãy số $\left(u_n\right)$, được xác định $\heva{&u_1=-2\\&u_{n+1}=-2-\dfrac{1}{u_n}}$. Số hạng tổng quát $u_n$ của dãy số là số hạng nào dưới đây?
	\choice
	{$u_n=\dfrac{-n+1}{n}$}
	{$u_n=\dfrac{n+1}{n}$}
	{\True $u_n=-\dfrac{n+1}{n}$}
	{$u_n=-\dfrac{n}{n+1}$}
	\loigiai{
		Ta có
		\begin{center}
			$\heva{&u_1=-2\\&u_{n+1}=-2-\dfrac{1}{u_n}}\Rightarrow\heva{&u_1=-2\\&u_2=-\dfrac{3}{2}}$
		\end{center}
		Ta thấy chỉ có đáp án C thoả mãn.
	}
\end{ex}
\begin{ex}%[1K2Y5-2]
	Cho cấp số cộng có số hạng đầu $u_1=-\dfrac{1}{2}$, công sai $d=\dfrac{1}{2}$. Năm số hạng liên tiếp đầu tiên của cấp số này là.
	\choice
	{$-\dfrac{1}{2};0;1;\dfrac{1}{2};1$}
	{$-\dfrac{1}{2};0;\dfrac{1}{2};0;\dfrac{1}{2}$}
	{$\dfrac{1}{2};1;\dfrac{3}{2};2;\dfrac{5}{2}$}
	{\True $-\dfrac{1}{2};0;\dfrac{1}{2};1;\dfrac{3}{2}$}
	\loigiai{
		Ta dùng công thức tổng quát $u_n=u_1+(n-1)d=-\dfrac{1}{2}+(n-1)\dfrac{1}{2}=-1+\dfrac{n}{2}$ để tính các số hạng của một cấp số cộng. Ta có
		\begin{center}
			$u_1=-\dfrac{1}{2},u_2=0,u_3=\dfrac{1}{2},u_4=1,u_5=\dfrac{3}{2}$.
		\end{center}
	}
\end{ex}
\begin{ex}%[1K2B6-3]
	Viết ba số hạng xen giữa các số $2$ và $22$ để được một cấp số cộng có năm số hạng.
	\choice
	{\True 
		$7;12;17$}
	{$6;10;14$}
	{$8;13;18$}
	{$6;12;18$}
	\loigiai{
		Giữa $2$ và $22$ có thêm ba số hạng nữa lập thành cấp số cộng, xem như ta có một cấp số cộng có năm số hạng với $u_1=22;u_5=22$, ta cần tìm $u_2,u_3,u_4$. Ta có
		\begin{eqnarray*}
			&&u_5=u_1+4d\\
			&\Leftrightarrow&d=\dfrac{u_5-u_1}{4}\\&\Leftrightarrow&d=5\\
			&\Rightarrow&\heva{&u_2=7\\&u_3=12\\&u_4=17}.
		\end{eqnarray*}
	}
\end{ex}
\begin{ex}%[1K2K6-3]
	Biết các số $C_n^1;C_n^2;C_n^3$ theo thứ tự lập thành một cấp số cộng với $n>3$. Tìm $n$.
	\choice
	{$n=5$}
	{\True $n=7$}
	{$n=9$}
	{$n=11$}
	\loigiai{
		Ba số $C_n^1;C_n^2;C_n^3$ theo thứ tự $u_1;u_2;u_3$ lập thành một cấp số cộng nên
		\begin{eqnarray*}
			&&u_1+u_3=2u_2\\
			&\Leftrightarrow&C_n^1+C_n^3=2C_n^2\\
			&\Leftrightarrow&n+\dfrac{(n-2)(n-1)n}{6}=2.\dfrac{(n-1)n}{2}\\
			&\Leftrightarrow&1+\dfrac{n^2-3n+2}{6}=n-1\\
			&\Leftrightarrow&n^2-9n+14\\
			&\Leftrightarrow&\hoac{&n=2\\&n=7}	
		\end{eqnarray*}
		Kết hợp với điều kiện $n>3$, do đó $n=7$ thoả mãn yêu cầu bài toán.
	}
\end{ex}
\begin{ex}%[1K2B6-2]
	Cho cấp số cộng $\left(u_n\right)$ có các số hạng đầu lần lượt là $5; 9; 13; 17;...$. Tìm số hạng tổng quát $u_n$ của cấp số cộng.
	\choice
	{$u_n=5n+1$}
	{$u_n=5n-1$}
	{\True $u_n=4n+1$}
	{$u_n=4n-1$}
	\loigiai{
		Các số $5; 9; 13; 17 th;...$ theo thứ tự đó lập thành cấp số cộng $\left(u_n\right)$ nên
		\begin{center}
			$\heva{&u_1=5\\&d=u_2-u_1=4}\Rightarrow u_n=u_1+(n-1)d=5+4(n-1)=4n+1$.
		\end{center}
	}
\end{ex}
\begin{ex}%[1K2K6-1]
	Cho cấp số cộng $\left(u_n\right)$ có $u_1=3$ và $d=\dfrac{1}{2}$. Khẳng định nào sau đây đúng?
	\choice
	{$u_n=-3+\dfrac{1}{2}(n+1)$}
	{$u_n=-3+\dfrac{1}{2}n-1$}
	{\True $u_n=-3+\dfrac{1}{2}(n-1)$}
	{$u_n=-3+\dfrac{1}{4}(n-1)$}
	\loigiai{
		Ta có
		\begin{center}
			$\heva{&u_1=-3\\&d=\dfrac{1}{2}}\Rightarrow u_n=u_1+(n-1)d=-3+\dfrac{1}{2}(n-1)$.
		\end{center}
	}
\end{ex}
\begin{ex}%[1K2K6-1]
	Trong các dãy số được cho dưới đây, dãy số nào là cấp số cộng?
	\choice
	{\True $u_7=7-3n$}
	{$u_7=7-3^n$}
	{$u_7=\dfrac{7}{3n}$}
	{$u_7=7.3^n$}
	\loigiai{
		Dãy $\left(u_n\right)$ là cấp số cộng khi và chỉ khi $u_n=an+b$ với $a,b$ là hằng số.
	}
\end{ex}
\begin{ex}%[1K2B6-3]
	Cho cấp số cộng $\left(u_n\right)$ có $u_1=-5$ và $d=3$. Mệnh đề nào sau đây đúng?
	\choice
	{$u_{15}=34$}
	{$u_{15}=45$}
	{\True $u_{13}=31$}
	{$u_{10}=35$}
	\loigiai{
		Ta có
		\begin{center}
			$\heva{&u_1=-5\\&d=3}\Rightarrow u_n=3n-8\Rightarrow\heva{&u_{15}=37\\&u_{13}=31\\&u_{10}=22}$.
		\end{center}
	}
\end{ex}
\begin{ex}%[1K2B6-3]
	Cho cấp số cộng $\left(u_n\right)$ có $d=-2$ và $S_8=72$. Tìm số hạng đầu tiên $u_1$.
	\choice
	{\True $u_1=16$}
	{$u_1=-16$}
	{$u_1=\dfrac{1}{16}$}
	{$u_1=-\dfrac{1}{16}$}
	\loigiai{
		Ta có $\heva{&d=-2\\&S_8=72}\Leftrightarrow\heva{&d=-2\\&8u_1+\dfrac{8.7}{2}d=72}\Rightarrow 8u_1+28.(-2)=72\Leftrightarrow u_1=16$.
	}
\end{ex}
\begin{ex}%[1K2K6-3]
	Một cấp số cộng có số hạng đầu là $1$, công sai là $4$, tổng của n số hạng đầu là $561$. Khi đó số
	hạng thứ $n$ của cấp số cộng đó là $u_n$ có giá trị là bao nhiêu?
	\choice
	{$u_n=57$}
	{$u_n=61$}
	{\True $u_n=65$}
	{$u_n=69$}
	\loigiai{
		Ta có $\heva{&u_1=1,d=4\\&S_n=561}\Leftrightarrow\heva{&u_=1,d=4\\&nu_1+\dfrac{n(n-1)}{2}d=561}\Rightarrow n+\dfrac{n^2-n}{2}.4=561\Leftrightarrow 2n^2-n-561=0\Leftrightarrow n=17$.\\
		Từ đây suy ra $u_{17}=u_1+16d=1+16.4=65$.
	}
\end{ex}
\begin{ex}%[1K2K6-5]
	Tổng $n$ số hạng đầu tiên của một cấp số cộng là $S_n=\dfrac{3n^2-19n}{4}$ với $n\in\mathbb{N}^*$. Tìm số hạng đầu
	tiên $u_1$ và công sai $d$ của cấp số cộng đã cho.
	\choice
	{$u_1=2,d=-\dfrac{1}{2}$}
	{\True $u_1=-4,d=\dfrac{3}{2}$}
	{$u_1=-\dfrac{3}{2},d=-2$}
	{$u_1=\dfrac{5}{2},d=\dfrac{1}{2}$}
	\loigiai{
		Ta có $\dfrac{3n^2-19n}{4}=\dfrac{3}{4}n^2-\dfrac{19n}{4}=S_n=nu_1+\dfrac{n^2-n}{2}d=\dfrac{d}{2}n^2+\left(u_1-\dfrac{d}{2}\right)n$.\\
		Đồng nhất hai vế của phương trình, ta có $\heva{&\dfrac{d}{2}=\dfrac{3}{4}\\&u_1-\dfrac{d}{2}=-\dfrac{19}{4}}\Leftrightarrow\heva{&u_1=-4\\&d=\dfrac{3}{2}}$.
	}
\end{ex}
\begin{ex}%[1K2K6-3]
	Cho cấp số cộng $\left(u_n\right)$ có $u_2=2001$ và $u_5=1995$. Khi đó $u_{1001}$ bằng.
	\choice
	{$u_{1001}=4005$}
	{$u_{1001}=4003$}
	{\True $u_{1001}=3$}
	{$u_{1001}=1$}
	\loigiai{
		Ta có $\heva{&u_2=2001\\&u_5=1995}\Leftrightarrow\heva{&u_1+d=2001\\&u_1+4d=1995}\Leftrightarrow \heva{&u_1=2003\\&d=-2}\Rightarrow u_{1001}=u_1+1000d=3$.
	}
\end{ex}
\begin{ex}%[1K2B6-1]
	Cho cấp số cộng $\left(u_n\right)$ biết $u_n=-1,u_{n+1}=8$. Tính công sai $d$ của cấp số cộng đó.
	\choice
	{$d=-9$}
	{$d=7$}
	{$d=-7$}
	{\True $d=9$}
	\loigiai{
		Ta có $d=u_{n+1}-u_n=8-(-1)=9$.
	}
\end{ex}
\begin{ex}%[1K2K6-5]
	Cho cấp số cộng $\left(u_n\right)$ thỏa mãn $u_2+u_{23}=60$. Tính tổng $S_24$ của $24$ số hạng đầu tiên của
	cấp số cộng đã cho.
	\choice
	{$S_{24}=60$}
	{$S_{24}=120$}
	{\True $S_{24}=720$}
	{$S_{24}=1440$}
	\loigiai{
		Ta có $u_2+u_{23}=60\Leftrightarrow u_1+d+u_1+22d=60\Leftrightarrow 2u_1+23d=60$.\\
		Khi đó $S_{24}=\dfrac{24}{2}\left(u_1+u_{24}\right)=12\left(u_1+u_1+23d\right)=12.60=720$.
	}
\end{ex}
\begin{ex}%[1K2K6-1]
	Một cấp số cộng có $6$ số hạng. Biết rằng tổng của số hạng đầu và số hạng cuối bằng $17$, tổng
	của số hạng thứ hai và số hạng thứ tư bằng $14$. Tìm công sai $d$ của câp số cộng đã cho.
	\choice
	{$d=2$}
	{\True $d=-3$}
	{$d=4$}
	{$d=5$}
	\loigiai{
		Ta có $\heva{&u_1+u_6=17\\&u_2+u_4=14}\Leftrightarrow\heva{&2u_1+5d=17\\&2u_1+6d=14}\Leftrightarrow\heva{&u_1=16\\&d=-3}$.
	}
\end{ex}
\begin{ex}%[1K2K6-1]
	Cho cấp số cộng $\left(u_n\right)$ thỏa mãn $\heva{&u_7-u_3=8\\&u_2u_7=75}$. Tìm công sai $d$ của cấp số cộng đã cho.
	\choice
	{$d=\dfrac{1}{2}$}
	{$d=\dfrac{1}{3}$}
	{\True $d=2$}
	{$d=3$}
	\loigiai{
		Ta có $\heva{&u_7-u_3=8\\&u_2u_7=75}\Leftrightarrow\heva{&u_1+6d-u_1-2d=8\\&(u_1+d)(u_1+6d)=75}\Leftrightarrow\heva{&d=2\\&(u_1+2)(u_1+12)=75}$.
	}
\end{ex}
\begin{ex}%[1K2K6-3]
	Ba góc của một tam giác vuông tạo thành cấp số cộng. Hai góc nhọn của tam giác có số đo
	(độ) là
	\choice
	{$20^\circ$ và $70^\circ$}
	{$45^\circ$ và $45^\circ$}
	{$20^\circ$ và $45^\circ$}
	{\True $30^\circ$ và $60^\circ$}
	\loigiai{
		Ba góc $A,B,C$ của một tam giác vuông theo thứ tự đó $(A<B<C)$ lập thánh cấp số cộng nên $C=90,C+A=2B$.\\
		Ta có $\heva{&A+B+C=180\\&A+C=2B\\&C=90}\Leftrightarrow\heva{&A=30\\&B=60\\&C+90}$.
	}
\end{ex}
\begin{ex}%[1K2K6-3]
	Một tam giác vuông có chu vi bằng $3$ và độ dài các cạnh lập thành một cấp số cộng. Độ dài các
	cạnh của tam giác đó là
	\choice
	{$\dfrac{1}{2};1;\dfrac{3}{2}$}
	{$\dfrac{1}{3};1;\dfrac{5}{3}$}
	{\True $\dfrac{3}{4};1;\dfrac{5}{4}$}
	{$\dfrac{1}{4};1;\dfrac{7}{4}$}
	\loigiai{
		Ba cạnh $a,b,c,(a<b<c)$ của một tam giác theo thứ tự đó lập thành một cấp số cộng.\\
		Ta có $\heva{&a^2+b^2=c^2\\&a+b+c=3\\&a+c=2b}\Leftrightarrow\heva{&a^2+b^2=c^2\\&3b=3\\&a+c=2b}\Leftrightarrow\heva{&a^2+b^2=c^2\\&b=1\\&a=2-c}$.\\
		Từ đây suy ra $a^2+b^2=c^2\Rightarrow (2-c)^2+1=c^2\Leftrightarrow c=\dfrac{5}{4}\Leftrightarrow\heva{&a=\dfrac{3}{4}\\&b=1\\&c=\dfrac{5}{4}}$.
	}
\end{ex}
\begin{ex}%[1K2K6-6]
	Một rạp hát có $30$ dãy ghế, dãy đầu tiên có $25$ ghế. Mỗi dãy sau có hơn dãy trước $3$ ghế. Hỏi rạp
	hát có tất cả bao nhiêu ghế?
	\choice
	{$1635$}
	{$1792$}
	{\True $2055$}
	{$3125$}
	\loigiai{
		Số ghế của mỗi dãy (bắt đầu từ dãy đầu tiên) theo thứ tự đó lập thành một cấp số cộng có $30$ số hạng có công sai $d=3$ và $u_1=25$.\\
		Tổng số ghế là $S_{30}=30u_1+\dfrac{30.29}{2}d=2055$.
	}
\end{ex}
\begin{ex}%[1K2K6-6]
	Người ta trồng $3003$ cây theo một hình tam giác như sau: hàng thứ nhất trồng $1$ cây, hàng thứ hai trồng $2$ cây, hàng thứ ba trồng $3$ cây,... .Hỏi có tất cả bao nhiêu hàng cây?
	\choice
	{$73$}
	{$75$}
	{\True $77$}
	{$79$}
	\loigiai{
		Số cây mỗi hàng (bắt đầu từ hàng thứ nhất) lập thành một cấp số cộng $(u_n)$ có $u_1=1,d=1$. Giả sử có $n$ hàng cây thì $u_1+u_2+...+u_n=S_n$.\\
		Ta có $S_n=1.n+\dfrac{n(n-1)}{2}.1=3003\Leftrightarrow n=77$.
	}
\end{ex}
\begin{ex}%[1K2G6-6]
	Một chiếc đồng hồ đánh chuông, kể từ thời điểm $0$ (giờ) thì sau mỗi giờ thì số tiếng chuông được đánh đúng bằng số giờ mà đồng hồ chỉ tại thời điểm đánh chuông. Hỏi một ngày đồng hồ đó đánh bao nhiêu tiếng chuông?
	\choice
	{$78$}
	{$156$}
	{\True $300$}
	{$48$}
	\loigiai{
		Kể từ lúc $1$ (giờ) đến $24$ (giờ) số tiếng chuông được đánh lập thành cấp số cộng có $24$ số hạng với $u_1=1$, công sai $d=1$. Vậy số tiếng chuông được đánh trong $1$ ngày là $S_{24}=1.24+\dfrac{24.23}{2}.1=300$.
	}
\end{ex}
\begin{ex}%[1K2G6-6]
	Trên một bàn cờ có nhiều ô vuông, người ta đặt $7$ hạt dẻ vào ô đầu tiên, sau đó đặt tiếp vào ô thứ
	hai số hạt nhiều hơn ô thứ nhất là $5$, tiếp tục đặt vào ô thứ ba số hạt nhiều hơn ô thứ hai là $5$,...
	và cứ thế tiếp tục đến ô thứ $n$. Biết rằng đặt hết số ô trên bàn cờ người ta phải sử dụng $25450$
	hạt. Hỏi bàn cờ đó có bao nhiêu ô vuông?
	\choice
	{$98$}
	{\True $100$}
	{$102$}
	{$104$}
	\loigiai{
		Số hạt dẻ trên mỗi ô (bắt đầu từ ô thứ nhất) theo thứ tự đó lập thành cấp số cộng $(u_n)$ có $u_1=7,d=5$. Gọi $n$ là số ô trên bàn cờ thì $u_1+u_2+...+u_n=S_n$.\\
		Ta có $S_n=25450\Leftrightarrow 7n+\dfrac{n(n-1)}{2}.7=25450\Leftrightarrow n=100$.
	}
\end{ex}
\begin{ex}%[1K2G6-6]
	Một gia đình cần khoan một cái giếng để lấy nước. Họ thuê một đội khoan giếng nước đến để khoan giếng nước. Biết giá của mét khoan đầu tiên là $80.000$ đồng, kể từ mét khoan thứ $2$ giá của mỗi mét khoan tăng thêm $5000$ đồng so với giá của mét khoan trước đó. Biết cần phải khoan sâu xuống $50$ mét mới có nước. Vậy hỏi phải trả bao nhiêu tiền để khoan cái giếng đó?
	\choice
	{$5.250.000$ đồng}
	{\True $10.125.000$ đồng}
	{$4.00.000$ đồng}
	{$4.245.000$ đồng}
	\loigiai{
		Giá tiền khoang mỗi mét (bắt đầu từ mét đầu tiên) lập thành cấp số cộng $(u_n)$ có $u_1=80000,d=5000$. Do cần khoang $50$ mét nên tổng số tiền cần trả là $S_{50}=80000.50+\dfrac{50.49}{2}.5000=10125000$.
	}
\end{ex}
\begin{ex}%[1K2Y7-3]
	Một cấp số nhân có hai số hạng liên tiếp là $16$ và $36$. Số hạng tiếp theo là
	\choice
	{$720$}
	{\True$81$}
	{$64$}
	{$56$}
	\loigiai{
		Ta có cấp số nhân $(u_n)$ có $\heva{&u_n=36\\&u_{n+1}=36}\Rightarrow q=\dfrac{u_{n+1}}{u_n}=\dfrac{9}{4}$. Từ đây suy ra $u_{n+2}=u_{n+1}.q=36.\dfrac{9}{4}=81$.
	}
\end{ex}
\begin{ex}%[1K2B7-3]
	Tìm x để các số $2;8;x;128$ theo thứ tự đó lập thành một cấp số nhân.
	\choice
	{$x=14$}
	{\True 
		$x=32$}
	{$x=64$}
	{$x=68$}
	\loigiai{
		Cấp số nhân$ 2;8;x;128$ theo thứ tự đó sẽ là $u_1,u_2,u_3,u_4$.\\
		Ta có $\heva{&\dfrac{u_2}{u_1}=\dfrac{u_3}{u_2}\\&\dfrac{u_3}{u_2}=\dfrac{u_4}{u_3}}\Leftrightarrow\heva{&\dfrac{8}{2}=\dfrac{x}{8}\\&\dfrac{128}{x}=\dfrac{x}{8}}\Leftrightarrow\heva{&x=32\\&x^2=1024}\Rightarrow x=32$.
	}
\end{ex}
\begin{ex}%[1K2K7-3]
	Tìm tất cả giá trị của $x$ để ba số $2x-11;x;2x+1$ theo thứ tự đó lập thành một cấp số nhân.
	\choice
	{\True $x=\pm\dfrac{1}{\sqrt{3}}$}
	{$x=\pm\dfrac{1}{3}$}
	{$x=\pm\sqrt{3}$}
	{$x=\pm3$}
	\loigiai{
		Cấp số nhân $2x-1;x;2x+1$, suy ra $(2x-1)(2x+1)=x^2\Leftrightarrow x=\pm \dfrac{1}{\sqrt{3}}$.
	}
\end{ex}
\begin{ex}%[1K2K7-3]
	Với giá trị $x,y$ nào dưới đây thì các số hạng lần lượt là $-2;x;-18;y$ theo thứ tự đó lập thành cấp số nhân?
	\choice
	{$\heva{&x=6\\&y=-54}$}
	{$\heva{&x=-10\\&y=-26}$}
	{\True $\heva{&x=-6\\&y=-54}$}
	{$\heva{&x=-6\\&y=54}$}
	\loigiai{
		Cấp số nhân $-2;x;-18;y$, suy ra $\heva{&\dfrac{x}{-2}=\dfrac{-18}{x}\\&\dfrac{-18}{x}=\dfrac{y}{-18}}\Leftrightarrow\heva{&x=\pm6\\& y=\pm 54}$. Vậy $(x,y)=(6;54)$ hoặc $(x;y)=(-6;-54)$.
	}
\end{ex}
\begin{ex}%[1K2K7-3]
	Hai số hạng đầu của của một cấp số nhân là $2x+1$ và $4x^2-1$. Số hạng thứ ba của cấp số nhân là.
	\choice
	{$2x-1$}
	{$2x+1$}
	{\True $8x^3-4x^2-2x+1$}
	{$8x^3+4x^2-2x-1$}
	\loigiai{
		Công bội của cấp số nhân là $q=\dfrac{4x^2-1}{2x+1}=2x-1$. Vậy số hạng thứ ba của cấp số nhân là $(4x^2-1)(2x-1)=8x^3-4x^2-2x+1$.
	}
\end{ex}
\begin{ex}%[1K2B7-1]
	Trong các dãy số $(u_n)$ cho bởi số hạng tổng quát nu sau, dãy số nào là một cấp số nhân
	\choice
	{\True 
		$u_n=\dfrac{1}{3^{n-2}}$}
	{$u_n=\dfrac{1}{3^{n}}-1$}
	{$u_n=n+\dfrac{1}{3}$}
	{$u_n=n^2-\dfrac{1}{3}$}
	\loigiai{
		Dãy $u_n=\dfrac{1}{3^{n-2}}=3\left(\dfrac{1}{3}\right)^{n-1}$ là cấp số nhân có $u_1=3,q=\dfrac{1}{3}$.
	}
\end{ex}
\begin{ex}%[1K2B7-1]
	Một cấp số nhân có $6$ số hạng, số hạng đầu bằng $2$ và số hạng thứ sáu bằng $486$. Tìm công bội $q$ của cấp số nhân đã cho.
	\choice
	{\True $q=3$}
	{$q=-3$}
	{$q=2$}
	{$q=-2$}
	\loigiai{
		Ta có $\heva{&u_1=2\\&u_6=486}\Rightarrow u_6=u_1q^5\Leftrightarrow 486=2.q^5\Leftrightarrow q=3$.
	}
\end{ex}
\begin{ex}%[1K2B7-1]
	Cho cấp số nhân $\left(u_n\right)$ có $u_1=-3$ và $q=\dfrac{2}{3}$ Mệnh đề nào sau đây đúng.
	\choice
	{$u_5=-\dfrac{27}{16}$}
	{\True $u_5=-\dfrac{16}{27}$}
	{$u_5=\dfrac{16}{27}$}
	{$u_5=\dfrac{27}{16}$}
	\loigiai{
		Ta có $\heva{&u_1=-3\\&q=\dfrac{2}{3}}\Rightarrow u_5=u_1.q^4=-3.\left(\dfrac{2}{3}\right)^4=-\dfrac{16}{27}$.
	}
\end{ex}
\begin{ex}%[1K2K7-3]
	Cho cấp số nhân $\left(u_n\right)$ có $u_1=3$ và $q=-2$. Số $192$ là số hạng thứ mấy của cấp số nhân đã cho.
	\choice
	{$5$}
	{$6$}
	{\True $7$}
	{Không là số hạng của cấp số đã cho}
	\loigiai{
		Ta có $u_n=u_1.q^{n-1}\Leftrightarrow 192=3.(-2)^{n-1}\Leftrightarrow n=7$.
	}
\end{ex}
\begin{ex}%[1K2K7-3]
	Một cấp số nhân có công bội bằng $3$ và số hạng đầu bằng $5$. Biết số hạng chính giữa là $32805$. Hỏi cấp số nhân đã cho có bao nhiêu số hạng?
	\choice
	{$18$}
	{\True $17$}
	{$16$}
	{$9$}
	\loigiai{
		Ta có $u_n=u_1.q^{n-1}\Leftrightarrow 32805=3.5^{n-1}\Leftrightarrow n=9$. Vậy $u_9$ là số hạng chính giữa của cấp số nhân, nên cấp số nhân đã cho có $17$ số hạng.
	}
\end{ex}
\begin{ex}%[1K2K7-5]
	Cho cấp số nhân $\left(u_n\right)$ có $u_1=-3$ và $q=-2$. Tính tổng $10$ số hạng đầu tiên của cấp số nhân đã cho.
	\choice
	{$S_{10}=-511$}
	{$S_{10}=-1025$}
	{$S_{10}=1025$}
	{\True $S_{10}=1023$}
	\loigiai{
		Ta có $\heva{&u_1=-3\\&q=-2}\Rightarrow S_{10}=u_1.\dfrac{q^{n}-1}{q-1}=(-3).\dfrac{(-2)^{10}-1}{-2-1}=1023$.
	}
\end{ex}
\begin{ex}%[1K2G7-5]
	Cho cấp số nhân có các số hạng lần lượt là $1;4;16;64;...$. Gọi $S_n$ là tổng của $n$ số hạng đầu tiên của cấp số nhân đó. Mệnh đề nào sau đây đúng?
	\choice
	{$S_n=4^{n-1}$}
	{$S_n=\dfrac{n\left(1+4^{n-1}\right)}{2}$}
	{\True $S_n=\dfrac{4^n-1}{3}$}
	{$S_n=\dfrac{4\left(4^n-1\right)}{3}$}
	\loigiai{
		Ta có $\heva{&u_1=-3\\&q=4}\Rightarrow S_n=u_1.\dfrac{q^{n}-1}{q-1}=\dfrac{4^n-1}{3}$.
	}
\end{ex}
\begin{ex}%[1K2G7-1]
	Số hạng thứ hai, số hạng đầu và số hạng thứ ba của một cấp số cộng với công sai khác $0$ theo thứ tự đó lập thành một cấp số nhân với công bội $q$. Tìm $q$.
	\choice
	{$q=2$}
	{\True $q=-2$}
	{$q=-\dfrac{3}{2}$}
	{$q=\dfrac{3}{2}$}
	\loigiai{
		Giả sử ba số hạng $a;b;c$ lập thành cấp số cộng thỏa yêu cầu, khi đó $b;a;c$ theo thứ tự đó lập thành cấp số nhân công bội $q$. Ta có $\heva{&a+c=2b\\&a=bq\\&c=bq^2}\Rightarrow bq+bq^2=2b\Leftrightarrow\heva{&b=0\\&q^2+q-2=0}$.\\
		Nếu $b=0\Rightarrow a=b=c=0$ nên $a;b;c$ là cấp số cộng công sai $d=0$ (vô lí).\\
		Nếu $q^2+q-2=0\Leftrightarrow\hoac{&q=1\\&q=-2}$. Nếu $q=1\Rightarrow a=b=c$ (vô lí), do đó $q=-2$.
	}
\end{ex}
\begin{ex}%[1K2G7-1]
	Cho bố số $a,b,c,d$ biết rằng $a,b,c$ theo thứ tự đó lập thành một cấp số nhân công bội $q>1$,
	còn $b,c,d$ theo thứ tự đó lập thành cấp số cộng. Tìm $q$ biết rằng $a+d=14$ và $b+c=12$.
	\choice
	{$q=\dfrac{18+\sqrt{73}}{24}$}
	{\True $q=\dfrac{19+\sqrt{73}}{24}$}
	{$q=\dfrac{20+\sqrt{73}}{24}$}
	{$q=\dfrac{21+\sqrt{73}}{24}$}
	\loigiai{
		Giả sử $a,b,c$ lập thành cấp số cộng công bội $q$. Khi đó theo giả thiết ta có\\
		$\heva{&b=aq\\&c=aq^2\\&b+d=2c\\&a+d=14\\&c+d=12}\Rightarrow\heva{&aq+d=aq^2,&(1)\\&a+d=14,&(2)\\&a\left(q+q^2\right)=12,&(3)}$.\\
		Nếu $q=0\Rightarrow b=c=d=0$ (vô lý).\\
		Nếu $q=-1\Rightarrow b=-a=-c\Rightarrow b+c=0$ (vô lý).\\
		Vậy $q\ne 0,q\ne -1$, từ $(2)$ và $(3)$, ta có $d=14-a$ và $a=\dfrac{12}{q+q^2}$, thay vào $(1)$, ta được\\
		$\dfrac{12q}{q+q^2}+\dfrac{14q^2+14q-12}{q+q^2}=\dfrac{24q^3}{q+q^2}\Leftrightarrow 12q^3-7q^2-13q+6=0\Leftrightarrow q=\dfrac{19\pm \sqrt{73}}{24}$.\\
		Mà $q>1$ nên $q=\dfrac{19+\sqrt{73}}{24}$.
	}
\end{ex}
\begin{ex}%[1K2G7-5]
	Gọi $S=1+11+111+\cdots+111\ldots1$ ($n$ số $1$) thì $S$ nhận giá trị nào sau đây?
	\choice
	{$S=\dfrac{10^n-1}{81}$}
	{$S=10\cdot\dfrac{10^n-1}{81}$}
	{$S=10\cdot\dfrac{10^n-1}{81}-1$}
	{\True $S=\dfrac{1}{9}\left[10\cdot\dfrac{10^n-1}{9}-1\right]$}
	\loigiai{
		Ta có $S=\dfrac{1}{9}\left(9+99+999+\cdots+999\ldots9\right)=\dfrac{1}{9}\left(10+100+1000+\cdots+100\ldots0-n\right)=\dfrac{1}{9}\left[10\cdot\dfrac{10^n-1}{9}-1\right]$.
	}
\end{ex}
\begin{ex}%[1K2G7-7]
	Biết rằng $S=1+2\cdot3+3\cdot3^2+\cdots+11.3^{10}=a+\dfrac{21\cdot3^b}{4}$. Tính $P=a+\dfrac{b}{4}$.
	\choice
	{$P=1$}
	{$P=2$}
	{\True $P=3$}
	{$P=4$}
	\loigiai{
		Từ giả thiết suy ra $3S=3+2\cdot3^2+3\cdot3^3+\cdots+11\cdot3^{11}$.\\
		Do đó $-2S=S-3S=1+3+3^2+3^3+\cdots+3^{10}-10.3^{11}=\dfrac{1-3^{11}}{1-3}-11\cdot3^{11}\Rightarrow S=\dfrac{1}{4}+\dfrac{21}{4}\cdot3^{11}$.\\
		Vậy $a=\dfrac{1}{4},b=11$, suy ra $P=3$.
	}
\end{ex}
\begin{ex}%[1K2K7-1]
	Một cấp số nhân có ba số hạng là $a,b,c$ (theo thứ tự đó) trong đó các số hạng đều khác $0$ và công bội $q\ne 0$. Mệnh đề nào sau đây là đúng.
	\choice
	{$\dfrac{1}{a^2}=\dfrac{1}{bc}$}
	{\True $\dfrac{1}{b^2}=\dfrac{1}{ac}$}
	{$\dfrac{1}{c^2}=\dfrac{1}{ba}$}
	{$\dfrac{1}{a}+\dfrac{1}{b}=\dfrac{2}{c}$}
	\loigiai{
		Ta có $ac=b^2\Rightarrow \dfrac{1}{b^2}=\dfrac{1}{ac}$
	}
\end{ex}
\begin{ex}%[1K2K7-3]
	Bốn góc của một tứ giác tạo thành cấp số nhân và góc lớn nhất gấp $27$ lần góc nhỏ nhất. Tổng của góc lớn nhất và góc bé nhất bằng.
	\choice
	{$56^\circ$}
	{$102^\circ$}
	{\True $252^\circ$}
	{$168^\circ$}
	\loigiai{
		Giả sử $4$ góc $A, B, C, D$ (với $A<B<C<D$) theo thứ tự đó lập thành cấp số nhân thỏa yêu cầu với công bội $q$.\\
		Ta có $\heva{&A+B+C+D=360\\&D=27A}\Leftrightarrow\heva{&A\left(1+q+q^2+q^3\right)=360\\&Aq^3=27A}\Leftrightarrow\heva{&q=3\\&A=9\\&D=243}\Rightarrow A+D=252$.
	}
\end{ex}
\begin{ex}%[1K2G7-7]
	Người ta thiết kế một cái tháp gồm $11$ tầng. Diện tích bề mặt trên của mỗi tầng bằng nữa diện tích của mặt trên của tầng ngay bên dưới và diện tích mặt trên của tầng $1$ bằng nửa diện tích của đế tháp (có diện tích là $12288m^2$). Tính diện tích mặt trên cùng.
	\choice
	{\True $6m^2$}
	{$8m^2$}
	{$10m^2$}
	{$12m^2$}
	\loigiai{
		Diện tích bề mặt của mỗi tầng (kể từ $1$) lập thành một cấp số nhân có công bội $q=\dfrac{1}{2}$ và
		$u_1=\dfrac{12288}{2}=6144$. Khi đó diện tích mặt trên cùng là $u_{11}=u_1\cdot q^{10}=6144\cdot\left(\dfrac{1}{2}\right)^{10}=6$.
	}
\end{ex}
\begin{ex}%[1K2G7-7]
	Một du khách vào chuồng đua ngựa đặt cược, lần đầu đặt $20000$ đồng, mỗi lần sau tiền đặt gấp
	đôi lần tiền đặt cọc trước. Người đó thua $9$ lần liên tiếp và thắng ở lần thứ $10$. Hỏi du khác trên thắng hay thua bao nhiêu?
	\choice
	{Hoà vốn}
	{Thua $20000$ đồng}
	{\True Thắng $20000$ đồng}
	{Thua $40000$ đồng}
	\loigiai{
		Số tiền du khác đặt trong mỗi lần (kể từ lần đầu) là một cấp số nhân có $u_1=20000$ và công bội $q=2$. Du khách thua trong $9$ lần đầu tiên nên tổng số tiền thua là $S_9=u_1.\dfrac{q^9-1}{q-1}=20000\cdot\dfrac{2^9-1}{2-1}=10220000$.\\
		Số tiền mà du khách thắng trong lần thứ $10$ là $u_{10}=u_1\cdot q^9=20000\cdot2^9=10240000$.\\
		Ta có $u_{10}-S_9=20000>0$ nên du khách thắng $20000$.
	}
\end{ex}
\Closesolutionfile{ans}
% \begin{indapan}{10}
% 	{ans/ans-1K2-Ontapchuong2}
% \end{indapan}