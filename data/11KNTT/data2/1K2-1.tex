
\setcounter{section}{4}
\setcounter{dang}{0}
\setcounter{ex}{0}
\setcounter{bt}{0}
\setcounter{vd}{0}
\section{Dãy số}
\subsection{Tóm tắt lý thuyết}
\begin{tomtat}
	\subsubsection{Định nghĩa dãy số} 
	\begin{itemize}
		\item Mỗi hàm $u$ xác định trên tập các số nguyên dương $\mathbb{N^{*}}$ được gọi là một dãy vô hạn (gọi tắt là dãy số), kí hiệu $u=u(n)$.
		\item 	Ta thường viết $u_n$ thay cho $u(n)$ và kí hiệu dãy số $u=u(n)$ bởi $(u_n)$, do đó dãy số $(u_n)$ được viết dưới dạng khai triển $u_1, u_2, u_3, \ldots, u_n, \ldots$\\
		Số $u_1$ gọi là số hạng đầu, $u_n$ gọi là số hạng thứ $n$ và gọi là số hạng tổng quát của dãy số.
		\item Nếu $\forall n \in \mathrm{N^*}, u_n=c$ thì $(u_n)$ được gọi là dãy số không đổi.
		\item Mỗi hàm $u$ xác định trên tập $\mathrm{M}=\left\{1;2;3;\ldots;m\right\}, \forall m \in \mathrm{N^*}$ được gọi là một dãy số hữu hạn.
		\item Dạng khai triển của dãy hữu hạn là $u_1, u_2, u_3, \ldots, u_m$.\\
		Số $u_1$ gọi là số hạng đầu, số $u_m$ gọi là số hạng cuối.
		
	\end{itemize}
	\subsubsection{Các cách cho một dãy số}
	Một dãy số có thể cho bằng:
	\begin{itemize}
		\item Liệt kê các số hạng (chỉ dùng cho các dãy hữu hạn và có ít số hạng);
		\item Công thức của số hạng tổng quát;
		\item Phương pháp mô tả;
		\item Phương pháp truy hồi.
	\end{itemize}
	\subsubsection{Dãy số tăng, dãy số giảm, dãy số bị chặn}
	\begin{itemize}
		\item Dãy số $(u_n)$ được gọi là dãy số tăng nếu ta có $u_{n+1}>u_n, \forall n \in \mathrm{N^*}$.
		\item Dãy số $(u_n)$ được gọi là dãy số giảm nếu ta có $u_{n+1}<u_n, \forall n \in \mathrm{N^*}$.
		\item Dãy số $(u_n)$ được gọi là bị chặn trên nếu tồn tại số $M$ sao cho $u_n \le M, \forall n \in \mathrm{N^*}$.
		\item Dãy số $(u_n)$ được gọi là bị chặn dưới nếu tồn tại số $m$ sao cho $u_n \ge m, \forall n \in \mathrm{N^*}$.
		\item Dãy số $(u_n)$ được gọi là bị chặn nếu nó vừa bị chặn trên vừa bị chặn dưới, tức là  tồn tại các số $m, M$ sao cho $m \le u_n \le M, \forall n \in \mathrm{N^*}$.
	\end{itemize}
\end{tomtat}

\subsection{Các dạng toán thường gặp}
\begin{dang}{Số hạng tổng quát, biểu diễn dãy số}
	Để tìm số hạng tổng quát của một dãy bất kỳ khi biết một vài số hạng đầu của dãy số ta làm như sau
	\begin{itemize}
		\item Phân tích các số hạng sau theo các số hạng đã biết theo một quy luật nào đó.
		\item Dự đoán số hạng tổng quát 
		\item Kiểm tra bằng cách thay lần lượt các giá trị $n\in \mathrm{N^*}$ vào công thức tổng quát (Chứng minh bằng phương pháp quy nạp).
	\end{itemize}
	Để biểu diễn một dãy số khi biết công thức tổng quát ta lần lượt thay $n\in \mathrm{N^*}$ vào công thức tổng quát để tìm các số hạng thứ nhất, thứ hai, $\ldots$
\end{dang}
\subsubsection{Ví dụ minh hoạ}
\begin{vd}[NB]%[DCHT Toán 11 - KNTT -Tên GV]%[1K2Y1-1]
	Xác định số hạng đầu và số hạng tổng quát của dãy số $(u_n)$ các số tự nhiên lẻ $1, 3, 5, 7, \ldots $
	\dapso{$u_n=2n-1$}	
	\loigiai{Dãy $(u_n)$ có số hạng đầu $u_1=1$ và số hạng tổng quát $u_n=2n-1$.}
\end{vd}
\begin{vd}[NB]%[DCHT Toán 11 - KNTT -Tên GV]%[1K2Y1-1]
	Xác định số hạng đầu và số hạng tổng quát của dãy số $(v_n)$ các số nguyên dương chia hết cho $5$: $5,10,15,20,\ldots$
	\dapso{$v_n=5n$}
	\loigiai{Dãy $(v_n)$ có số hạng đầu $v_1=5$ và số hạng tổng quát $v_n=5n$.}
\end{vd}
% \begin{vd}[NB]%[DCHT Toán 11 - KNTT -Tên GV]%[1K2Y1-1]
% 	Viết năm số hạng đầu và số hạng thứ $100$ của dãy số $(u_n)$ có số hạng tổng quát $u_n=3n-2$.
% 	\dapso{$u_{100}=298$}
% 	\loigiai{Năm số hạng đầu của dãy số là $1,4,7,10,13$.\\
% 		Số hạng thứ $100$ của dãy là $u_{100}=3\cdot100-2=298$.}
% \end{vd}
% \begin{vd}[NB]%[DCHT Toán 11 - KNTT -Tên GV]%[1K2Y1-1]
% 	Cho dãy số xác định bằng hệ thức truy hồi: $u_1=1, u_n=3u_{n-1}+2$ với $n\ge 2$. Viết ba số hạng đầu của dãy số này.
% 	\dapso{$u_1=1, u_2=5, u_3=17$}
% 	\loigiai{Ta có $u_1=1, u_2=3u_1+2=5, u_3=3u_2+2=17$.}
% \end{vd}
% \begin{vd}[NB]%[DCHT Toán 11 - KNTT -Tên GV]%[1K2Y1-1]
% 	Dãy số $(u_n)$ cho bởi hệ thức truy hồi: $u_1=1, u_n=n \cdot u_{n-1}$ với $n \ge 2$. Viết năm số hạng đầu của dãy số và dự đoán công thức tổng quát $u_n$.
% 	\dapso{$u_n=n!$}
% 	\loigiai{Năm số hạng đầu của dãy là
% 		$u_1=1, u_2=2\cdot u_1=2, u_3=3\cdot u_2=6, u_4=4 \cdot u_3= 24, u_5=5 \cdot u_4=124$.\\
% 		Số hạng tổng quát\\
% 		Ta có $ u_2=2\cdot 1, u_3=6=3\cdot 2\cdot 1, u_4=24=4\cdot 3\cdot 2\cdot 1, u_5=124= 5\cdot4\cdot3\cdot2\cdot1  $.\\
% 		Vậy số hạng tổng quát $u_n=n!$.}
% \end{vd}
\subsubsection{Bài tập tự luận}
 
\begin{bt}[NB]%[DCHT Toán 11 - KNTT -Tên GV]%[1K2Y1-1]
	Xét dãy số hữu hạn gồm các số tự nhiên lẻ nhỏ hơn 20, sắp xếp theo thứ tự từ bé đến lớn. Liệt kê tất cả các số hạng của dãy số này, tìm số hạng đầu và số hạng cuối của dãy. 
	\dapso{$u_1=1$, $u_{11}=19$}
	\loigiai{Các số hạng của dãy là $1,3,5,7,9,10,11,13,15,17,19$.\\
		Số hạng đầu của dãy là $u_1=1$.\\
		Số hạng cuối của dãy là $u_{11}=19$.}
\end{bt}
\begin{bt}[TH]%[DCHT Toán 11 - KNTT -Tên GV]%[1K2Y1-1]
	Xét dãy số gồm tất cả các số tự nhiên chia cho $5$ dư $1$. Xác định số hạng tổng quát của dãy số.
	\dapso{$u_n=5n+1$}
	\loigiai{Các số tự nhiên chia cho $5$ dư $1$ gồm các số sau:
		$6,11,16,21, \ldots $\\
		Số hạng tổng quát $u_n=5n+1$.}
\end{bt}
% \begin{bt}[NB]%[DCHT Toán 11 - KNTT -Tên GV] %[1K2Y1-1]
% 	Tìm năm số hạng đầu và số hạng thứ $100$ của dãy $(u_n)$ có số hạng tổng quát $u_n= \dfrac{(-1)^n}{n}$.
% 	\dapso{$u_1=-1, \dfrac{1}{2},-\dfrac{1}{3}, \dfrac{1}{4}, -\dfrac{1}{5} $, $u_{100}=\dfrac{1}{100}$}
% 	\loigiai{
% 		Năm số hạng đầu của dãy là $u_1=-1, \dfrac{1}{2},-\dfrac{1}{3}, \dfrac{1}{4}, -\dfrac{1}{5} $.\\
% 		Số hạng thứ $100$ là $u_{100}=\dfrac{1}{100}$.}
% \end{bt}
\begin{bt}[NB]%[DCHT Toán 11 - KNTT -Tên GV]%[1K2Y1-1]
	Viết năm số hạng đầu của dãy số gồm các số nguyên tố theo thứ tự tăng dần.
	\dapso{$2,3,5,7,11$}
	\loigiai {Năm số hạng đầu của dãy số trên là $2,3,5,7,11$.} 
\end{bt}
% \begin{bt}[NB]%[DCHT Toán 11 - KNTT-Tên GV]%[1K2Y1-1]
% 	Viết năm số hạng đầu của dãy $(u_n)$ với số hạng tổng quát là $u_n=n!$.
% 	\dapso{$1,2,6,24,120$}
% 	\loigiai{Năm số hạng đầu của dãy trên là $1,2,6,24,120$.}
% \end{bt}
\subsubsection{Câu hỏi trắc nghiệm}
\Opensolutionfile{ans}[ans/ans-1K2-1-Dang1]
%Cau1
\begin{ex}%[DCHT Toán 11 - KNTT -Tên GV]%[1K2B1-1]
	Cho dãy số có các số hạng đầu là $5,10,15,20,25, \ldots$ Số hạng tổng quát của dãy số này là
	\choice
	{$ u_n=5(n-1) $}
	{\True$ u_n=5n $}
	{$ u_n=5+n $}
	{$ u_n=5n+1 $}
	\loigiai
	{Ta có $5=5\cdot 1, 10=5 \cdot 2, 15 = 5\cdot 3, 20=5 \cdot 4, 25 = 5\cdot 5, \ldots$\\
		Vậy dãy trên có số hạng tổng quát là $u_n=5n$.
	}
\end{ex}
%Cau2
\begin{ex}%[DCHT Toán 11 - KNTT -Tên GV]%[1K2B1-1]
	Cho dãy số $(u_n)$ với $u_n=\dfrac{an^2}{n+1}$, $a$ là hằng số. $u_{n+1}$ là số hạng nào trong các số hạng sau
	\choice
	{\True $u_{n+1}=\dfrac{a(n+1)^2}{n+2} $}
	{$u_{n+1}=\dfrac{a(n+1)^2}{n+1}$}
	{$u_{n+1}=\dfrac{an^2+1}{n+1}$}
	{$u_{n+1}=\dfrac{an^2}{n+2} $}
	\loigiai
	{Ta có $u_{n+1}=\dfrac{a(n+1)^2}{n+1+1}=\dfrac{a(n+1)^2}{n+2}$.
	}
\end{ex}
%cau3
\begin{ex}%[DCHT Toán 11 - KNTT -Tên GV]%[1K2B1-1]
	Cho dãy số có các số hạng đầu là $8,15,22,29,36, \ldots$ Số hạng tổng quát của dãy số này là
	\choice
	{$ u_n=7n+7 $}
	{$ u_n=7n $}
	{\True $ u_n=7n+1 $}
	{$ u_n$ không viết được dưới dạng công thức }
	\loigiai
	{Ta có $8=7\cdot 1+1, 15=7 \cdot 2+1, 22 = 7\cdot 3+1, 29=7 \cdot 4+1, 36 = 7\cdot 5+1, \ldots$\\
		Vậy dãy trên có số hạng tổng quát là $u_n=7n+1$.
	}
\end{ex}
%Cau4
\begin{ex}%[DCHT Toán 11 - KNTT -Tên GV]%[1K2B1-1]
	Cho dãy số có các số hạng đầu là $0,\dfrac{1}{2},\dfrac{2}{3},\dfrac{3}{4},\dfrac{4}{5}, \ldots$ Số hạng tổng quát của dãy số này là
	\choice
	{$ u_n=\dfrac{n+1}{n}$}
	{\True $ u_n=\dfrac{n}{n+1} $}
	{$ u_n=\dfrac{n-1}{n}$}
	{$ u_n=\dfrac{n^2-n}{n+1}$  }
	\loigiai
	{Ta có $0=\dfrac{0}{0+1}, \dfrac{1}{2}=\dfrac{1}{1+1} ,\dfrac{2}{3} = \dfrac{2}{2+1}, \dfrac{3}{4}=\dfrac{3}{3+1}, \dfrac{4}{5} = \dfrac{4}{4+1}, \ldots$\\
		Vậy dãy trên có số hạng tổng quát là $u_n=\dfrac{n}{n+1}$.
	}
\end{ex}
%Cau5
\begin{ex}%[DCHT Toán 11 - KNTT -Tên GV]%[1K2B1-1]
	Cho dãy số $(u_n)$ với $u_1=1, u_{n+1}=u_n-1$. Số hạng tổng quát $u_n$ của dãy số là số hạng nào dưới đây?
	\choice
	{\True $ u_n=2-n$}
	{$ u_n$ không xác định}
	{$ u_n=1-n$}
	{$ u_n=-n$, với mọi $n$ }
	\loigiai
	{Ta có $u_1=1, u_2=0 ,u_3 = -1, u_4=-2,  \ldots$\\
		Dễ dàng dự đoán được số hạng tổng quát là $u_n=2-n$.
	}
\end{ex}
% %cau6
% \begin{ex}%[DCHT Toán 11 - KNTT -Tên GV]%[1K2B1-1]
% 	Cho dãy số $(u_n)$ với $u_n=\dfrac{2n^2-1}{n^2+3}, \forall n \in \mathrm{N*}$. Số hạng đầu tiên của dãy số là 
% 	\choice
% 	{$ u_1=-\dfrac{1}{3}$}
% 	{$ u_1=\dfrac{2}{3}$}
% 	{$ u_1=\dfrac{1}{3}$}
% 	{\True $ u_1=\dfrac{1}{4}$ }
% 	\loigiai
% 	{Ta có $u_1=\dfrac{2\cdot 1^2-1}{1^2+3}=\dfrac{1}{4}$.
% 	}
% \end{ex}
% %cau7
% \begin{ex}%[DCHT Toán 11 - KNTT -Tên GV]%[1K2B1-1]
% 	Cho dãy số $(u_n)$ với $u_1=-1, u_{n+1}=u_n+3$ với $n \ge 1$. Ba số hạng đầu tiên của dãy số lần lượt là 
% 	\choice
% 	{\True $-1, 2, 5$}
% 	{$ 1, 4, 7$}
% 	{$ 4,7,10$}
% 	{$-1,3,7$ }
% 	\loigiai
% 	{Ta có $u_1=-1, u_2=-1+3=2 ,u_3 = 2+3=5$.
% 	}
% \end{ex}
\Closesolutionfile{ans}
\begin{indapan}{10}
	{ans/ans-1K2-1-Dang1}
\end{indapan}

\begin{dang}{Tìm số hạng cụ thể của dãy số}
	Để tìm số hạng cụ thể của dãy số ta làm như sau
	\begin{itemize} 
		\item Với trường hợp dãy số đã cho biết công thức tổng quát của dãy số thì ta chỉ cần thay giá trị tương ứng của số hạng đó vào công thức tổng quát.
		\item  Với trường hợp dãy số cho bởi công thức truy hồi hoặc dưới dạng thì ta phải tìm lần lượt từ những số hạng đầu tiên cho đến số đứng trước số cần tìm trong dãy.
	\end{itemize}
\end{dang}
\subsubsection{Ví dụ minh hoạ}
\begin{vd}[NB]%[1K2Y5-2]
	Cho dãy số $(u_n),$ biết $u_n=(-1 )^n\cdot \dfrac{2^n}{n}$. Tìm số hạng $u_3$.
	\dapso{$u_3=-\dfrac{8}{3}$}
	\choice
	{\True $u_3=-\dfrac{8}{3}$}
	{$u_3=2$}
	{$u_3=-2$}
	{$u_3=\dfrac{8}{3}$}
	\loigiai{
		Ta có
		$$u_3=(-1)^3\cdot \dfrac{2^3}{3}=-\dfrac{8}{3}.$$}
\end{vd}
\begin{vd}[NB]%[DCHT Toán 11 - KNTT -Nguyễn Long]%[1K2Y5-2]
	Cho dãy số $(u_n)$, biết $u_n=\dfrac{2n^2-1}{n^2+3}$. Tìm số hạng $u_5$.
	\dapso{$u_5=\dfrac{7}{4}$}
	\choice
	{$u_5=\dfrac{1}{4}$}
	{\True $u_5=\dfrac{7}{4}$}
	{$u_5=\dfrac{17}{12}$}
	{$u_5=\dfrac{71}{39}$}
	\loigiai{
		Ta có $u_5=\dfrac{2\cdot 5^2-1}{5^2+3}=\dfrac{49}{28}=\dfrac{7}{4}$.}
	
\end{vd}
\begin{vd}[NB]%[1K2Y5-2]
	Cho dãy số $u_n$ bao gồm các số nguyên tố. Tìm số hạng thứ $5$ của dãy số.
	\dapso{$u_5=11$}
	\loigiai{ 
		Ta có
		$u_1=2,u_2=3,u_3=5,u_4=7,u_5=11$. \\
		Vậy số hạng thứ $5$ của dãy số là $11$.
	}
\end{vd}
\begin{vd}[NB]%[1K2Y5-2]
	Cho dãy số $(u_n) $ thỏa mãn $ \heva{& u_1 = 5 \\& u_{n+1} = u_n+n}$. Tìm số hạng thứ $5$ của dãy số.
	\dapso{$u_5=15$}
	\choice
	{$ 11 $}
	{\True $ 15 $}
	{$ 16 $}
	{$ 12 $}
	\loigiai{
		Ta có $ u_2=u_1+1=6$, $ u_3=u_2+2=8$, $ u_4=u_3+3=11$,  $ u_5=u_4+4=15$.
	}
\end{vd}

\begin{vd}[TH]%[VD 5 SGK KNTT]%[1K2B5-2]
	Cho dãy số xác định bằng hệ thức truy hồi
	$$
	u_1=1, u_n=3 u_{n-1}+2 \text { với } n \geq 2
	$$
	Viết ba số hạng đầu của dãy số này.
	\dapso{$u_5=17$}
	\loigiai{
		Ta có: $u_1=1, u_2=3 u_1+2=3 \cdot 1+2=5, u_3=3 u_2+2=3 \cdot 5+2=17$.
	}
\end{vd}

\begin{vd}[VD]%[1K2B5-2]
	Cho dãy số $\left(u_n\right)\colon\heva{&u_1=5 \\ &u_{n+1}=u_n+n}$. Số $20$ là số hạng thứ mấy trong dãy?
	\dapso{số hạng thứ $6$}
	\loigiai{
		Ta có $u_1=5, u_2=6, u_3=8, u_4=11, u_5=16, u_6=20$.\\
		Vậy số $20$ là số hạng thứ $6$.}
\end{vd}

\subsubsection{Bài tập tự luận}
 
\begin{bt}[NB]%[1K2Y5-2]
	Cho dãy số $u_n=\dfrac{1}{\sqrt{n}+1}$. Tìm số hạng $u_4$.	
	\dapso{$u_4=\dfrac{1}{3}$}
	\loigiai{ Ta có
		$u_4=\dfrac{1}{\sqrt{4}}+1=\dfrac{1}{3}.$		
	}
\end{bt}
%%%
\begin{bt}[NB]%[1K2Y5-2]
	Cho dãy số $(u_n)$ có số hạng tổng quát: $u_n=2 n+\sqrt{n^2+4}$. Tìm số hạng thứ $6$ của dãy số.
	\dapso{$u_6=12+2\sqrt{10}$}
	\loigiai{
		Ta có $u_6=12+2 \sqrt{10}$.
	}
\end{bt}
%%%
\begin{bt}[NB]%[1K2Y5-2]
	Cho dãy số $(u_n)$ xác định bởi: $\heva{&u_1=-1 ; u_2=3 \\&u_{n+1}=5 u_n-6 u_{n-1} \forall n \geq 2}.$ Tìm số hạng thứ $7$ của dãy.
	\dapso{$3261$}
	\loigiai{
		Ta có
		$$
		u_3=5 u_2-6 u_1=21 ;~ u_4=5 u_3-6 u_2=87 ;~ u_5=309 ;~ u_6=1023 ;~ u_7=3261
		$$
		Vậy số hạng thứ $7$ của dãy là $3261$.
	}
\end{bt}
%%%%%%%
\begin{bt}[NB]%[1K2Y5-2]
	Viết năm số hạng đầu của dãy số Fibonacci $\left(F_n\right)$ cho bởi hệ thức truy hồi
	$$
	\heva{
		&F_1=1, F_2=1 \\
		&F_n=F_{n-1}+F_{n-2}~(n \geq 3) .
	}
	$$
	\dapso{$F_3=2,~F_4=3,~F_5=5$}
	\loigiai{
		Ta có $F_3=2,~F_4=3,~F_5=5.$
	}
\end{bt}
%%%
\begin{bt}[NB]%[1K2T5-2]
	Người ta nuôi cấy $5$ con vi khuẩn E-coli trong môi trường nhân tạo. Cứ $30$ phút thì vi khuẩn E-coli sẽ nhân đôi 1 lần. Tính số lượng vi khuẩn thu được sau $1,2,3$ lần nhân đôi.
	\dapso{$u_2=10, u_3=20, u_4=40$}
	\loigiai{
		Đặt $u_1=5$, gọi số vi khuẩn sau $n$ lần phân chia là $u_{n+1}$, khi đó ta có dãy số $(u_n)$ thỏa mãn $$u_1=5, \; u_{n+1}=2u_n$$
		Ta có $u_2=10, u_3=20, u_4=40$.
	}	
\end{bt}
%%%%%
\begin{bt}[TH]%[1K2B5-2]
	Cho dãy số $(u_n)$ được xác định bởi $u_n=\dfrac{n^2+3n+7}{n+1}$.
	\begin{listEX}
		\item Viết năm số hạng đầu của dãy.
		\item Dãy số có bao nhiêu số hạng nhận giá trị nguyên.
	\end{listEX}
	\dapso{$u_1=\dfrac{11}{2}$; $u_2=\dfrac{17}{3}$; $u_3=\dfrac{25}{4}$; $u_4=7$; $u_5=\dfrac{47}{6}$. $u_4=7 $}
	\loigiai{		
		\begin{listEX}
			\item Ta có năm số hạng đầu của dãy
			$u_1=\dfrac{1^2+3.1+7}{1+1}=\dfrac{11}{2}$; $u_2=\dfrac{17}{3}$; $u_3=\dfrac{25}{4}$; $u_4=7$; $u_5=\dfrac{47}{6}$.
			\item Ta có: $u_n=n+2+\dfrac{5}{n+1}$, do đó $u_n$ nguyên khi và chỉ khi $ \dfrac{5}{n+1}$ nguyên hay $ n+1 $ là ước của 5. Điều đó xảy ra khi $ n+1=5\Leftrightarrow n=4 $. Vậy dãy số có duy nhất một số hạng nguyên là $u_4=7 $.
		\end{listEX}		
	}
\end{bt}
\begin{bt}[VD]%[1K2K5-2]
	Cho dãy số $\left(x_n\right)$ thỏa mãn điều kiện $x_1=1, x_{n+1}-x_n=\dfrac{1}{n(n+1)}, n=1,2,3, \ldots$. Số hạng $x_{2023}$ bằng
	\dapso{$x_{2023}=\dfrac{4045}{2023}$}
	\loigiai{
		Ta có
		$$
		\begin{aligned}
			x_{n+1}-x_n=\dfrac{1}{n(n+1)}=\dfrac{1}{n}-\dfrac{1}{n+1} & \Leftrightarrow \sum_{k=1}^{n-1}\left(x_{k+1}-x_k\right)=\sum_{k=1}^{n-1}\left(\dfrac{1}{k}-\dfrac{1}{k+1}\right) \\
			& \Leftrightarrow x_n-x_1=1-\dfrac{1}{n} \\
			& \Leftrightarrow x_n=\dfrac{2n-1}{n} .
		\end{aligned}
		$$
	}
\end{bt}
\begin{bt}[VDC]%[1K2G5-2]
	Cho dãy số $\left(u_n\right)$ biết $\heva{&u_1=99 \\&u_{n+1}=u_n-2 n-1, n \geq 1}$. Hỏi số $-861$ là số hạng thứ mấy?
	\dapso{$-861$ là số hạng thứ $31$}
	\loigiai{
		Ta có
		$$
		\begin{aligned}
			&u_n &=& &u_{n-1}-2 n+1 \\
			&u_{n-1} & = & &u_{n-2}-2 n+3 \\
			&\vdots &\vdots&  &\vdots \\
			&u_3 & = & &u_2-2 n+2 n-5 \\
			&u_2 & = & &u_1-2 n+2 n-3
		\end{aligned}
		$$
		Suy ra
		$$
		\begin{aligned}
			& u_n=u_1-2 n \cdot(n-1)+1+3+5+\cdots+(2 n-5)+(2 n-3) \\
			& u_n=99-2 n^2+2 n+\dfrac{n-1}{2}\cdot[2 \cdot 1+(n-2) \cdot 2]=100-n^2
		\end{aligned}
		$$
		Giả sử $u_n=-861 \Rightarrow n^2=961 \Rightarrow n=31$ (vì $n \in \mathbb{N}$).
		Vậy số $-861$ là số hạng thứ $31$ .}
\end{bt}
\subsubsection{Câu hỏi trắc nghiệm}
\Opensolutionfile{ans}[ans/ans-1K2-1-Dang2]
%Câu 1
\begin{ex}%[1K2Y5-2]
	Cho dãy số $({{u}_{n}} )$, biết ${{u}_{n}}=\dfrac{n}{{{3}^{n}}-1}$. Ba số hạng đầu tiên của dãy số đó lần lượt là những số nào dưới đây?
	\choice
	{$\dfrac{1}{2};\dfrac{1}{4};\dfrac{1}{16}$}
	{$\dfrac{1}{2};\dfrac{2}{3};\dfrac{3}{4}$}
	{ \True $\dfrac{1}{2};\dfrac{1}{4};\dfrac{3}{26}$}
	{$\dfrac{1}{2};\dfrac{1}{4};\dfrac{1}{8}$}
	\loigiai {
		Ta có
		${{u}_{1}}=\dfrac{1}{2};\,\,{{u}_{2}}=\dfrac{2}{{{3}^2}-1}=\dfrac{2}{8}=\dfrac{1}{4};\,\,{{u}_{3}}=\dfrac{3}{{{3}^3}-1}=\dfrac{3}{26}$.}
	
\end{ex}
%%%%%%%%%%
%Câu 2
\begin{ex}%[1K2Y5-2]
	Cho dãy số $({{u}_{n}} ),$ biết ${{u}_{n}}={{(-1 )}^{n}}\cdot 2n$. Mệnh đề nào sau đây {\bf sai}?
	\choice
	{${{u}_{3}}=-6$}
	{${{u}_{2}}=4$}
	{ \True ${{u}_{4}}=-8$}
	{${{u}_{1}}=-2$}
	\loigiai {
		Ta có\\
		${{u}_{1}}=-2\cdot 1=-2;\,\,{{u}_{2}}={{(-1 )}^2}\cdot 2\cdot 2=4,\,\,{{u}_{3}}={{(-1 )}^3}\cdot 2\cdot 3=-6;\,\,{{u}_{4}}={{(-1 )}^4}\cdot 2\cdot 4=8$.\\
		\textbf{Nhận xét:} Dễ thấy ${{u}_{n}}>0$ khi $n$ chẵn và ngược lại nên đáp án $u_4=-8$ sai.}
	
\end{ex}
%%%%%%%%%%
%Câu 3
\begin{ex}%[1K2Y5-2]
	Cho dãy số $({{u}_{n}} )$ xác định bởi $\heva{
		& {{u}_{1}}=2 \\
		& {{u}_{n+1}}=\dfrac{1}{3}({{u}_{n}}+1 ) \\
	}.$ Tìm số hạng ${{u}_{4}}$.
	\choice
	{${{u}_{4}}=\dfrac{2}{3}$}
	{${{u}_{4}}=1$}
	{${{u}_{4}}=\dfrac{14}{27}$}
	{ \True ${{u}_{4}}=\dfrac{5}{9}$}
	\loigiai {
		Ta có
		${{u}_{2}}=\dfrac{1}{3}({{u}_{1}}+1 )=\dfrac{1}{3}(2+1 )=1;\,\,{{u}_{3}}=\dfrac{1}{3}({{u}_{2}}+1 )=\dfrac{2}{3};\,\,{{u}_{4}}=\dfrac{1}{3}({{u}_{3}}+1 )=\dfrac{1}{3}\cdot\left(\dfrac{2}{3}+1\right)=\dfrac{5}{9}$. \\
	}
\end{ex}
%%%%%%%%%%
%Câu 4
\begin{ex}%[1K2Y5-2]
	Cho dãy số $({{u}_{n}} )$, biết $\heva{
		& {{u}_{1}}=-1 \\
		& {{u}_{n+1}}={{u}_{n}}+3 \\
	}$ với $n\ge 0$. Ba số hạng đầu tiên của dãy số đó là lần lượt là những số nào dưới đây?
	\choice
	{\True $-1;\,2;\,5$}
	{$-1;3;7$}
	{$1;\,4;\,7$}
	{$4;\,7;\,10$}
	\loigiai {
		Ta có ${{u}_{1}}=-1;\,\,{{u}_{2}}={{u}_{1}}+3=2;\,\,{{u}_{3}}={{u}_{2}}+3=5$. \\
		\textbf{Nhận xét.} (i) Dùng chức năng “lặp” của MTCT để tính:\\
		Nhập vào màn hình: $X=X+3$ \\
		Bấm CALC và cho $X=-1$ (ứng với ${{u}_{1}}=-1)$ \\
		Để tính ${{u}_{n}}$ cần bấm “=” ra kết quả liên tiếp $n-1$ lần. Ví dụ để tính ${{u}_{2}}$ ta bấm “=” ra kết quả lần đầu tiên, bấm “=” ra kết quả thứ hai chính là ${{u}_{3}},\ldots$\\
		(ii) Vì ${{u}_{1}}=-1$ nên loại các đáp án $u_1=1$, $u_1=4$.\\
		Còn lại các đáp án có $u_1=-1$; để biết đáp án nào ta chỉ cần kiểm tra ${{u}_{2}}$ (vì ${{u}_{2}}$ ở hai đáp án là khác nhau): ${{u}_{2}}={{u}_{1}}+3=2$.
	}
	
\end{ex}
%%%%%%%%%%
%Câu 5
\begin{ex}%[1K2B5-2]
	Cho dãy số $({{u}_{n}} ),$ biết ${{u}_{n}}=\dfrac{2n+5}{5n-4}$. Số $\dfrac{7}{12}$ là số hạng thứ mấy của dãy số?
	\choice
	{$9$}
	{$6$}
	{$10$}
	{\True $8$}
	\loigiai {
		Ta có
		$${{u}_{n}}=\dfrac{2n+5}{5n-4}=\dfrac{7}{12}\Leftrightarrow 24n+60=35n-28\Leftrightarrow 11n=88\Leftrightarrow n=8.$$}
	
\end{ex}
%%%%%%%%%%
%Câu 6
\begin{ex}%[1K2B5-2]
	Cho dãy $(u_n)$ xác định bởi $\heva{& u_1=3 \\& u_{n+1}=\dfrac{u_n}{2}+2}$. Mệnh đề nào sau đây {\bf sai}?
	\choice
	{\True ${{u}_{2}}=\dfrac{5}{2}$}
	{${{u}_{4}}=\dfrac{31}{8}$}
	{${{u}_{3}}=\dfrac{15}{4}$}
	{${{u}_{5}}=\dfrac{63}{16}$}
	\loigiai {
		Ta có $\heva{
			& {{u}_{2}}=\dfrac{{{u}_{1}}}{2}+2=\dfrac{3}{2}+2=\dfrac{7}{2};\,\,{{u}_{3}}=\dfrac{{{u}_{2}}}{2}+2=\dfrac{7}{4}+2=\dfrac{15}{4}. \\
			& {{u}_{4}}=\dfrac{{{u}_{3}}}{2}+2=\dfrac{15}{8}+2=\dfrac{31}{8};\,\,{{u}_{5}}=\dfrac{{{u}_{4}}}{2}+2=\dfrac{31}{16}+2=\dfrac{63}{16}. \\
		}$}
\end{ex}
%%%%%%%%%%
%Câu 7
\begin{ex}%[1K2B5-2]
	Cho dãy số $({{u}_{n}} ),$ với ${{u}_{n}}={{\left(\dfrac{n-1}{n+1} \right)}^{2n+3}}$. Tìm số hạng ${{u}_{n+1}}$.
	\choice
	{${{u}_{n+1}}={{\left(\dfrac{n-1}{n+1} \right)}^{2(n-1 )+3}}$}
	{${{u}_{n+1}}={{\left(\dfrac{n-1}{n+1} \right)}^{2(n+1 )+3}}$ }
	{\True ${{u}_{n+1}}={{\left(\dfrac{n}{n+2} \right)}^{2n+5}}$}
	{${{u}_{n+1}}={{\left(\dfrac{n}{n+2} \right)}^{2n+3}}$}
	\loigiai {
		${{u}_{n}}={{\left(\dfrac{n-1}{n+1} \right)}^{2n+3}}\Rightarrow {{u}_{n+1}}={{\left(\dfrac{(n+1 )-1}{(n+1 )+1} \right)}^{2(n+1 )+3}}={{\left(\dfrac{n}{n+2} \right)}^{2n+5}}$.}
	
\end{ex}
%%%%%%%%%%
%Câu 8
\begin{ex}%[1K2K5-2]
	Cho dãy số $({{a}_{n}} ),$ được xác định $\heva{
		& {{a}_{1}}=3 \\
		& {{a}_{n+1}}=\dfrac{1}{2}{{a}_{n}},~n\ge 1 \\
	}$. Mệnh đề nào sau đây {\bf sai}?
	\choice
	{${{a}_{1}}+{{a}_{2}}+{{a}_{3}}+{{a}_{4}}+{{a}_{5}}=\dfrac{93}{16}$}
	{${{a}_{10}}=\dfrac{3}{512}$}
	{ \True ${{a}_{n}}=\dfrac{3}{{{2}^{n}}}$}
	{${{a}_{n+1}}+{{a}_{n}}=\dfrac{9}{{{2}^{n}}}$}
	\loigiai {
		Ta có ${{a}_{1}}=3;\,{{a}_{2}}=\dfrac{{{u}_{1}}}{2};\,\,{{a}_{3}}=\dfrac{{{u}_{2}}}{2}=\dfrac{{{u}_{1}}}{{{2}^2}};\,\,{{a}_{4}}=\dfrac{{{u}_{3}}}{2}=\dfrac{{{u}_{1}}}{{{2}^3}},\ldots \\
		\Rightarrow {{u}_{n}}=\dfrac{{{u}_{1}}}{{{2}^{n-1}}}=\dfrac{3}{{{2}^{n-1}}}$ nên suy ra đáp án ${{a}_{n}}=\dfrac{3}{{{2}^{n}}}$ sai. \\
		Xét đáp án\\
		${{a}_{1}}+{{a}_{2}}+{{a}_{3}}+{{a}_{4}}+{{a}_{5}}=3\left(1+\dfrac{1}{2}+\dfrac{1}{{{2}^2}}+\dfrac{1}{{{2}^3}}+\dfrac{1}{{{2}^4}}\right)=3.\dfrac{1-{{(\dfrac{1}{2} )}^5}}{1-\dfrac{1}{2}}=\dfrac{93}{16}\Rightarrow $  đúng.\\
		Xét đáp án ${{a}_{10}}=\dfrac{3}{{{2}^{9}}}=\dfrac{3}{512}\Rightarrow $  đúng.\\
		Xét đáp án ${{a}_{n+1}}+{{a}_{n}}=\dfrac{3}{{{2}^{n}}}+\dfrac{3}{{{2}^{n-1}}}=\dfrac{3+3\cdot 2}{{{2}^{n}}}=\dfrac{9}{{{2}^{n}}}\Rightarrow $ đúng.}
	
\end{ex}
%%%%%%%%%%
%Câu 9
\begin{ex}%[1K2K5-2]
	Cho dãy số $(u_n)$ biết $\heva{&u_1=1\\&u_2=4\\&u_{n+2}=3u_{n+1}-2u_n}$ với mọi $n \ge 1$. Giá trị $u_{101}-u_{100}$ là 
	\choice
	{$3\cdot 2^{102} $}
	{$3\cdot 2^{101} $}
	{$3\cdot 2^{100} $}
	{\True $ 3\cdot 2^{99}$}
	\loigiai{
		Theo bài  ta có 
		\begin{eqnarray*}
			&u_{n+2}=3u_{n+1}-2u_n\\
			\Leftrightarrow \,& u_{n+2}=u_{n+1}+2(u_{n+1}-u_n)\\
			\Leftrightarrow \,& u_{n+2}-u_{n+1}=2(u_{n+1}-u_n).
		\end{eqnarray*}
		Với $n=99$ ta có 
		\begin{align*}
			u_{101}-u_{100}&=2(u_{100}-u_{99})\\
			&=2\cdot 2 (u_{99}-u_{98})\\
			&= \ldots\\
			&=2^{99}\cdot(u_2-u_1)=3\cdot2^{99}.
		\end{align*}
	}
\end{ex}
%%%%%%%%%%
%Câu 10
\begin{ex}%[1K2G5-2]
	Cho dãy số $\left(u_n\right)$ thoả mãn $u_1=\sqrt{2}$ và $u_{n+1}=\sqrt{2+u_n}$ với mọi $n\geq 1$. Tìm $u_{2023}$.
	\choice
	{$u_{2023}=\sqrt{2}\cos\dfrac{\pi}{2^{2022}}$}
	{\True $u_{2023}=\sqrt{2}\cos\dfrac{\pi}{2^{2024}}$}
	{$u_{2023}=\sqrt{2}\cos\dfrac{\pi}{2^{2023}}$}
	{$u_{2023}=2$}
	\loigiai{Ta chứng minh bằng phương pháp quy nạp số hạng tổng quát của dãy là $u_n=2\cos\dfrac{\pi}{2^{n+1}}$.\\
		Dễ thấy, với $n=1$ ta có $u_1=\sqrt{2}$ (đúng).\\
		Giả sử mệnh đề đúng với $n=k, \forall k\in \mathbb{N}^\ast$ nghĩa là $u_k=2\cos\dfrac{\pi}{2^{k+1}}$ ta phải chứng minh mệnh đề đúng với $n=k+1$ nghĩa là $u_{k+1}=2\cos\dfrac{\pi}{2^{k+2}}$.\\
		Thật vậy, $u_{k+1}=\sqrt{2+u_k}=\sqrt{2+2\cos\dfrac{\pi}{2^{k+1}}}=\sqrt{4\cos^2\dfrac{\pi}{2^{k+2}}}=2\cos\dfrac{\pi}{2^{k+2}}$.\\
		Áp dụng công thức tổng quát trên ta có $u_{2023}=\sqrt{2}\cos\dfrac{\pi}{2^{2024}}$.
	}
\end{ex}
%%%%%%%%%%
\Closesolutionfile{ans}
\begin{indapan}{10}
	{ans/ans-1K2-1-Dang2}
\end{indapan}
\begin{dang}{Xét tính tăng giảm của dãy số}
	\begin{enumerate}
		\item Phương pháp 1. Xét dấu của hiệu số $u_{n+1}-u_n$.
		\begin{enumerate}
			\item Nếu $u_{n+1}-u_n>0, \forall n \in \mathbb{N}^\ast$ thì $(u_n)$ là dãy số tăng.
			\item Nếu $u_{n+1}-u_n<0, \forall n \in \mathbb{N}^\ast$ thì $(u_n)$ là dãy số giảm.
		\end{enumerate}
		\item Phương pháp 2. Nếu $u_n>0, \forall n\in \mathbb{N}^\ast$ thì ta có thể so sánh thương $\dfrac{u_{n+1}}{u_n}$ với $1$.
		\begin{enumerate}
			\item Nếu $\dfrac{u_{n+1}}{u_n}>1$ thì $(u_n)$ là dãy số tăng.
			\item Nếu $\dfrac{u_{n+1}}{u_n}<1$ thì $(u_n)$ là dãy số giảm.
		\end{enumerate}
		Nếu $u_n<0, \forall n\in \mathbb{N}^\ast$ thì ta có thể so sánh thương $\dfrac{u_{n+1}}{u_n}$ với $1$.
		\begin{enumerate}
			\item Nếu $\dfrac{u_{n+1}}{u_n}<1$ thì $(u_n)$ là dãy số tăng.
			\item Nếu $\dfrac{u_{n+1}}{u_n}>1$ thì $(u_n)$ là dãy số giảm.
		\end{enumerate}
		\item Phương pháp 3. Nếu dãy số $(u_n)$ cho bởi hệ thức truy hồi thì thường dùng phương pháp quy nạp để chứng minh $u_{n+1}>u_n, \forall n \in \mathbb{N}^\ast$ (hoặc $u_{n+1}<u_n \forall n \in \mathbb{N}^\ast$).
	\end{enumerate}
\end{dang}
\subsubsection{Ví dụ minh hoạ}
\begin{vd}[NB]%[1K2Y5-3]
	Xét sự tăng giảm của dãy số $(u_n)$ với $u_n=(-1)^n$.
	\dapso{dãy không tăng không giảm}
	\loigiai{
		Ta có:\\ $u_1=(-1)^1=-1,\,
		u_2=(-1)^2=1,\,
		u_3=(-1)^3=-1.$\\
		Vậy $(u_n)$ là dãy không tăng không giảm.
	}
\end{vd}
\begin{vd}[NB]%[1K2Y5-3]
	Xét tính tăng giảm của dãy số sau $(u_n)$ với $u_n=\dfrac{2n+1}{n+1}$.
	\dapso{dãy số tăng}
	\loigiai
	{
		Ta có: $u_n=\dfrac{2n+1}{n+1}=2-\dfrac{1}{n+1}$.\\
		$u_{n+1}-u_n=\left(2-\dfrac{1}{n+1+1}\right)-\left(2-\dfrac{1}{n+1}\right)=\dfrac{1}{n+1}-\dfrac{1}{n+2}>0, \forall n \in \mathbb{N}^\ast$.\\
		Vậy dãy số $(u_n)$ là dãy số tăng.
	}
\end{vd}
\begin{vd}[TH]%[1K2B5-3]
	Xét tính tăng giảm của dãy số $(u_n)$ với $u_n=\sqrt{n}-\sqrt{n+2}$.
	\dapso{dãy số tăng}
	\loigiai{
		Ta có $u_n=\sqrt{n}-\sqrt{n+2}=\dfrac{-2}{\sqrt{n}+\sqrt{n+2}}$.\\
		Xét hiệu\\ 
		$\begin{aligned}
			u_{n+1}-u_n&=\dfrac{-2}{\sqrt{n+1}+\sqrt{n+3}}-\dfrac{-2}{\sqrt{n}+\sqrt{n+2}}\\
			&=\dfrac{2}{\sqrt{n}+\sqrt{n+2}}-\dfrac{2}{\sqrt{n+1}+\sqrt{n+3}}>0, \forall n\in \mathbb{N}^\ast.
		\end{aligned}$\\
		Vậy $(u_n)$ là dãy số tăng.
	}
\end{vd}

\begin{vd}[TH]%[1K2B5-3]
	Xét tính tăng giảm của dãy số $(u_n)$ với $u_n=\dfrac{n}{3^n}$.
	\dapso{dãy số giảm}
	\loigiai{
		Ta có $u_n=\dfrac{n}{3^n}>0, \forall n \in \mathbb{N}^\ast$.\\
		Xét thương $\dfrac{u_{n+1}}{u_n}=\dfrac{n+1}{3^{n+1}}:\dfrac{n}{3^n}=\dfrac{n+1}{3.n}<1, \forall  n \in \mathbb{N}^\ast$.\\
		Vậy $(u_n)$ là dãy số giảm.
	}
\end{vd}
\begin{vd}[VD]%[1K2K5-3]
	Xét tính tăng giảm của dãy số $(u_n)$ với $ \heva{& u_1=2\\
		& u_{n+1}=\dfrac{3u_n+1}{u_n+1}, n\in \mathbb{N}^\ast.}$
	\dapso{dãy số tăng}
	\loigiai{
		Giả sử $u_{n+1}>u_n , \forall n \in \mathbb{N}^\ast. \qquad (*)$\\
		Ta chứng minh $(*)$ bằng phương pháp quy nạp.
		\begin{itemize}
			\item Với $n=1, u_2=\dfrac{3.2+1}{2+1}=\dfrac{6}{3}=\dfrac{7}{3}>u_1=2.$
			\item Giả sử $(*)$ đúng khi $n=k, k\in \mathbb{N}^\ast$, tức là $u_{k+1}>u_k$.\\
			Ta sẽ chứng minh $(*)$ đúng với $n=k+1$, tức là
			$u_{k+2}>u_{k+1}$.\\
			Thật vậy\\ $u_{k+2}-u_{k+1}=\left(3-\dfrac{2}{u_{k+1}+1}\right)-\left(3-\dfrac{2}{u_k+1}\right)=\dfrac{2}{u_k+1}-\dfrac{2}{u_{k+1}+1}.$\\
			Theo giả thiết quy nạp ta có:\\ $u_{k+1}>u_k \Rightarrow u_{k+1}+1>u_k+1 \Rightarrow \dfrac{2}{u_k+1}>\dfrac{2}{u_{k+1}+1}$.\\
			Vậy $u_{k+2}-u_{k+1}>0$.\\
			Do đó, $(*)$ đúng với mọi số nguyên dương $n$.
		\end{itemize}
		Vậy $(u_n)$ là dãy số tăng.}
\end{vd}
\subsubsection{Bài tập tự luận}
 
\begin{bt}[NB]%[1K2Y5-3]
	Xét tính tăng giảm của dãy số $(u_n)$ với $u_n=\dfrac{\sqrt{2}}{3^n}$.
	\dapso{dãy số giảm}
	\loigiai{
		Ta có $u_n>0, \forall n \in \mathbb{N}^\ast$.\\
		Xét thương $$\dfrac{u_{n+1}}{u_n}=\dfrac{\sqrt{2}}{3^{n+1}}: \dfrac{\sqrt{2}}{\sqrt{3^2}}=\dfrac{3^n}{3^{n+1}}=\dfrac{1}{3}<1.$$
		Vậy $\left(u_n\right)$ là dãy số giảm.
	}
\end{bt}
\begin{bt}[NB]%[1K2Y5-3]
	Xét tính tăng giảm của dãy số $\left(u_n\right)$ với $u_n=\dfrac{1}{n(n+1)}$.
	\dapso{dãy số tăng}
	\loigiai{
		Ta có $u_n=\dfrac{1}{n(n+1)}=\dfrac{1}{n}-\dfrac{1}{n+1}$.
		Xét hiệu:
		$$
		\begin{aligned}
			u_{n+1}-u_n & =\left(\dfrac{1}{n}-\dfrac{1}{n+1}\right)-\left(\dfrac{1}{n+1}-\dfrac{1}{n+2}\right) \\
			& =\dfrac{1}{n}-\dfrac{1}{n+2}>0, \forall n \in \mathbb{N}^\ast
		\end{aligned}
		$$
		Vậy  $\left(u_n\right)$ là dãy số tăng.
	}
\end{bt}
\begin{bt}[TH]%[1K2B5-3]
	Xét tính tăng giảm của dãy số $\left(u_n\right)$ với $u_n=n+\cos ^2 n$.
	\dapso{dãy số tăng}
	\loigiai{
		Xét hiệu
		$$
		\begin{aligned}
			u_{n+1}-u_n & =\left(n+1+\cos ^2(n+1)\right)-\left(n+\cos ^2 n\right) \\
			& =1+\cos ^2(n+1)-\cos ^2 n \\
			& =\cos ^2(n+1)+\sin ^2 n>0, \forall n \in \mathbb{N}^\ast .
		\end{aligned}
		$$
		Vậy $\left(u_n\right)$ là dãy số tăng.
	}
\end{bt}
\begin{bt}[TH]%[1K2B5-3]
	Xét tính tăng giảm của dãy số $(u_n)$ với $u_n=\dfrac{1}{n+1}+\dfrac{1}{n+2}+\ldots+\dfrac{1}{2n}$.
	\dapso{dãy số giảm}
	\loigiai{
		Xét hiệu\\
		$\begin{aligned}
			u_{n+1}-u_n&=\left(\dfrac{1}{n+2}+\dfrac{1}{n+3}+\ldots+\dfrac{1}{2(n+1)}\right)-\left(\dfrac{1}{n+1}+\dfrac{1}{n+2}+\ldots+\dfrac{1}{2n}\right)\\
			&=\dfrac{1}{n+2}-\dfrac{1}{2n+1}-\dfrac{1}{2n+2}\\
			&=\dfrac{1}{2n+2}-\dfrac{1}{2n+1}<0, \forall n\in \mathbb{N}^\ast.
		\end{aligned}$\\
		Vậy $(u_n)$ là dãy số giảm.
	}
	
\end{bt}

\begin{bt}[TH]%[1K2B5-3]
	Xét tính tăng giảm của dãy số $\left(u_n\right)$ với $u_n=\dfrac{1}{n+1}+\dfrac{1}{n+2}+\ldots+\dfrac{1}{2 n}$.
	\dapso{dãy số giảm}
	\loigiai{
		Xét hiệu
		$$
		\begin{aligned}
			u_{n+1}-u_n & =\left(\dfrac{1}{n+2}+\dfrac{1}{n+3}+\ldots+\dfrac{1}{2(n+1)}\right)-\left(\dfrac{1}{n+1}+\dfrac{1}{n+2}+\ldots+\dfrac{1}{2 n}\right) \\
			& =\dfrac{1}{n+2}-\dfrac{1}{2 n+1}-\dfrac{1}{2 n+2} \\
			& =\dfrac{1}{2 n+2}-\dfrac{1}{2 n+1}<0, \forall n \in \mathbb{N}^\ast
		\end{aligned}
		$$
		Vậy $\left(u_n\right)$ là dãy số giảm.
	}
\end{bt}
\begin{bt}[VD]%[1K2K5-3]
	Xét tính tăng giảm của dãy số $\left(u_n\right)$ cho bởi
	$$
	\left(u_n\right)\colon\heva{
		&u_1=1 ; u_2=2 \\
		&u_{n+1}=\sqrt{u_n}+\sqrt{u_{n-1}} \forall n \geq 2
	}
	$$
	\dapso{dãy số tăng}
	\loigiai{
		Ta chứng minh dãy $\left(u_n\right)$ là dãy tăng bằng phương pháp quy nạp.\\
		Dễ thấy $u_1<u_2<u_3$.\\
		Giả sử $u_{k-1}<u_k ~\forall k \geq 2$, ta chứng minh $u_{k+1}>u_k$.\\
		Thật vậy ta có $u_{k+1}=\sqrt{u_k}+\sqrt{u_{k-1}}>\sqrt{u_{k-1}}+\sqrt{u_{n-2}}=u_k$.\\ Vậy $\left(u_n\right)$ là dãy tăng.
	}
	
\end{bt}
\begin{bt}[VD]%[1K2K5-3]
	Cho dãy số $\left(u_n\right)$ biết $u_n=\dfrac{b \cdot2 n^2+1}{n^2+3}$ và $b \in \mathbb{R}$. Hãy xác định $b$ để
	\begin{listEX}[2]
		\item $\left(u_n\right)$ là dãy số giảm.
		\item $\left(u_n\right)$ là dãy số tăng.
	\end{listEX}
	\dapso{$b<\dfrac{1}{6}$ dãy số giảm; $b>\dfrac{1}{6}$ dãy số tăng}
	\loigiai{
		Ta có
		$$
		u_n=2 b+\dfrac{1-6 b}{n^2+3}
		$$
		Xét hiệu $$u_{n+1}-u_n=\dfrac{1-6 b}{(n+1)^2+3}-\dfrac{1-6 b}{n^2+3}=(1-6 b) \cdot\left(\dfrac{1}{(n+1)^2+3}-\dfrac{1}{n^2+3}\right)=A_n.$$
		\begin{listEX}
			\item Để $\left(u_n\right)$ là dãy sỗ giảm thì $A_n<0, \forall n \in \mathbb{N}^\ast$.
			$$
			A_n<0 \Leftrightarrow 1-6 b>0 \Leftrightarrow b<\dfrac{1}{6}
			$$
			\item Để $\left(u_n\right)$ là dãy số tăng thì $A_n>0, \forall n \in \mathbb{N}^\ast$.
			$$
			A_n>0 \Leftrightarrow 1-6 b<0 \Leftrightarrow b>\dfrac{1}{6}.
			$$
		\end{listEX}
		
	}
\end{bt}
\begin{bt}[VDC]%[1K2G5-3]
	Xét tính tăng giảm của dãy số $\left(u_n\right)$ với $u_n=\sin n+\cos n$.
	\dapso{dãy khōng tăng, không giảm}
	\loigiai{
		Ta có: $u_n=\sin n+\cos n=\sqrt{2} \sin \left(n+\dfrac{\pi}{4}\right)$.
		Xét hiệu
		$$
		\begin{aligned}
			u_{n+1}-u_n&=\sqrt{2} \sin \left(n+1+\dfrac{\pi}{4}\right)-\sqrt{2} \sin \left(n+\dfrac{\pi}{4}\right) \\
			&=2 \sqrt{2} \cdot \cos \left(2 n+\dfrac{1}{2}+\dfrac{\pi}{4}\right) \cdot \sin \dfrac{1}{2}=A_n . \\
		\end{aligned}
		$$
		Với  $n=1, A_1>0$. Với  $n=100, A_{100}<100$ . \\
		Vậy $\left(u_n\right)$ là dãy khōng tăng, không giảm.
	}
\end{bt}
\subsubsection{Bài tập trắc nghiệm}
\Opensolutionfile{ans}[ans/ans-1K2-1-Dang3]
%Câu 1
\begin{ex}%[1K2Y5-3]
	Cho các dãy số sau. Dãy số nào là dãy số tăng?
	\choice
	{$1;1;1;1;1;1;\ldots $}
	{$1;\dfrac{1}{2};\dfrac{1}{4};\dfrac{1}{8};\dfrac{1}{16};\ldots $}
	{$1;-\dfrac{1}{2};\dfrac{1}{4};-\dfrac{1}{8};\dfrac{1}{16};\ldots $}
	{ \True $1;3;5;7;9;\ldots $}
	\loigiai {
		Xét đáp án $1;1;1;1;1;1;\ldots$ đây là dãy hằng nên không tăng không giảm.\\
		Xét đáp án $1;-\dfrac{1}{2};\dfrac{1}{4};-\dfrac{1}{8};\dfrac{1}{16};\ldots \Rightarrow {{u}_{1}}>{{u}_{2}}<{{u}_{3}}\Rightarrow $ loại.\\
		Xét đáp án $1;3;5;7;9;\ldots \Rightarrow {{u}_{n}}<{{u}_{n+1}},\,\,n\in {{\mathbb{N}}^{*}}\Rightarrow $ chọn.\\
		Xét đáp án $1;\dfrac{1}{2};\dfrac{1}{4};\dfrac{1}{8};\dfrac{1}{16};\ldots \Rightarrow {{u}_{1}}>{{u}_{2}}>{{u}_{3}}\ldots >{{u}_{n}}>\ldots \Rightarrow $ loại.}
	
\end{ex}
%%%%%%%%%%
%Câu 2
\begin{ex}%[1K2Y5-3]
	Với giá trị nào của $a$ thì dãy số $\left(u_n\right)$ với $u_n=\dfrac{a n-1}{n+2}, \forall n \geq 1$ là dãy số tăng?
	\choice
	{$a>2$}
	{$a<-2$}
	{\True $a>-\dfrac{1}{2}$}
	{$a<-\dfrac{1}{2}$}
	\loigiai{
		Ta có $u_n=a-\dfrac{1+2 a}{n+2}$.\\
		$u_{n+1}-u_n=(1+2 a)\left(\dfrac{1}{n+2}-\dfrac{1}{n+3}\right)$.\\
		Suy ra dãy số đã cho tăng khi $a>-\dfrac{1}{2}$.
	}
\end{ex}
%%%%%%%%%%%%
%Câu 3
\begin{ex}%[1K2Y5-3]
	Trong các dãy $\left(u_n\right)$ sau đây dãy nào là dãy số giảm ?
	\choice
	{$u_n=(-1)^n$}
	{$u_n=2^n$}
	{$u_n=3 n+1$}
	{\True $u_n=\dfrac{1}{3^n}$}
	\loigiai{
		Xét dãy số $\left(u_n\right)$ có $u_n=\dfrac{1}{3^n}$, ta thấy $u_n>0, \forall n \in \mathbb{N}^\ast$ và $\dfrac{u_{n+1}}{u_n}=\dfrac{\dfrac{1}{3^{n+1}}}{\dfrac{1}{3^n}}=\dfrac{1}{3}<1$ nên dãy số $\left(u_n\right)$ này là dãy số giảm.
	}
\end{ex}
%%%%%%%%%%
%Câu 4
\begin{ex}%[1K2B5-3]
	Trong các dãy số $({{u}_{n}} )$ cho bởi số hạng tổng quát ${{u}_{n}}$ sau, dãy số nào là dãy số tăng?
	\choice
	{${{u}_{n}}=\dfrac{1}{n}$}
	{${{u}_{n}}=\dfrac{1}{{{2}^{n}}}$}
	{${{u}_{n}}=\dfrac{n+5}{3n+1}$}
	{ \True ${{u}_{n}}=\dfrac{2n-1}{n+1}$ }
	\loigiai {
		Vì ${{2}^{n}};\,n$ là các dãy dương và tăng nên $\dfrac{1}{{{2}^{n}}};\,\,\dfrac{1}{n}$ là các dãy giảm, do đó loại các đáp án ${{u}_{n}}=\dfrac{1}{{{2}^{n}}}$ và ${{u}_{n}}=\dfrac{1}{n}$.\\
		Xét đáp án ${{u}_{n}}=\dfrac{n+5}{3n+1}\Rightarrow \heva{
			& {{u}_{1}}=\dfrac{3}{2} \\
			& {{u}_{2}}=\dfrac{7}{6} \\
		}\Rightarrow {{u}_{1}}>{{u}_{2}}\Rightarrow $ loại.\\
		Xét đáp án ${{u}_{n}}=\dfrac{2n-1}{n+1}=2-\dfrac{3}{n+1}\Rightarrow  {{u}_{n+1}}-{{u}_{n}}=3\left(\dfrac{1}{n+1}-\dfrac{1}{n+2}\right)>0\Rightarrow$ nhận.}
	
\end{ex}
%%%%%%%%%%
%%%%%%%%%%
%Câu 5
\begin{ex}%[1K2B5-3]
	Trong các dãy số $({{u}_{n}} )$ cho bởi số hạng tổng quát ${{u}_{n}}$ sau, dãy số nào là dãy số giảm?
	\choice
	{${{u}_{n}}={{n}^2}$}
	{${{u}_{n}}=\dfrac{3n-1}{n+1}$}
	{${{u}_{n}}=\sqrt{n+2}$}
	{ \True ${{u}_{n}}=\dfrac{1}{{{2}^{n}}}$}
	\loigiai {
		Vì ${{2}^{n}}$ là dãy dương và tăng nên $\dfrac{1}{{{2}^{n}}}$ là dãy giảm. \\
		Xét ${{u}_{n}}=\dfrac{3n-1}{n+1}\Rightarrow \heva{
			& {{u}_{1}}=1 \\
			& {{u}_{2}}=\dfrac{5}{3} \\
		}\Rightarrow {{u}_{1}}<{{u}_{2}},$ loại.\\
		Hoặc
		${{u}_{n+1}}-{{u}_{n}}=\dfrac{3n+2}{n+2}-\dfrac{3n-1}{n+1}=\dfrac{4}{(n+1 )(n+2 )}>0$ nên $({{u}_{n}} )$ là dãy tăng.\\
		Xét ${{u}_{n}}={{n}^2}\Rightarrow {{u}_{n+1}}-{{u}_{n}}={{(n+1 )}^2}-{{n}^2}=2n+1>0,$ loại.\\
		Xét ${{u}_{n}}=\sqrt{n+2}\Rightarrow {{u}_{n+1}}-{{u}_{n}}=\sqrt{n+3}-\sqrt{n+2}=\dfrac{1}{\sqrt{n+3}+\sqrt{n+2}}>0,$ loại.}
	
\end{ex}

%Câu 6

\begin{ex}%[1K2B5-3]
	Trong các dãy số $(u_{n})$ sau, hãy chọn dãy số tăng.
	\choice
	{\True $u_{n}=(-1)^{2n}(5^{n}+1)$, $n\in \mathbb N^*$}
	{$u_{n}=\dfrac{n}{n^{2}+1}$, $n\in \mathbb N^*$}
	{$u_{n}=(-1)^{n+1}\sin \dfrac{\pi}{n}$, $n\in \mathbb N^*$}
	{$u_{n}=\dfrac{1}{\sqrt{n+1}+n}$, $n\in \mathbb N^*$}
	\loigiai
	{
		Xét dãy số $(u_n)$ với $u_{n}=(-1)^{2n}(5^{n}+1)$, ta có
		\[u_{n+1}-u_n = (-1)^{2n+2}(5^{n+1}+1)-(-1)^{2n}(5^{n}+1) = 5^{n+1}+1-5^n-1 = 4\cdot 5^n>0, \forall n\in\mathbb{N}^\ast.\]
		Vậy dãy trên là dãy số tăng.\\
		Xét các dãy số còn lại
		\begin{itemize}
			\item Với $u_{n}=(-1)^{n+1}\sin \dfrac{\pi}{n}$ ta có $u_1=0$, $u_2=-1$ hay $u_1>u_2$. Vậy dãy số này không là dãy số tăng.
			\item Với $u_{n}=\dfrac{1}{\sqrt{n+1}+n}$ ta có $u_1=\sqrt{2}-1$, $u_2=2-\sqrt{3}$ hay $u_1>u_2$. Vậy dãy số này không là dãy số tăng.
			\item Với $u_{n}=\dfrac{n}{n^{2}+1}$ ta có $u_1=\dfrac{1}{2}$, $u_2=\dfrac{2}{5}$ hay $u_1>u_2$. Vậy dãy số này không là dãy số tăng.
		\end{itemize}
	}
\end{ex}
%%%%%%%%%%
%Câu 7
\begin{ex}%[1K2K5-3]
	Trong các dãy số $({{u}_{n}} )$ cho bởi số hạng tổng quát ${{u}_{n}}$ sau, dãy số nào là dãy số giảm?
	\choice
	{${{u}_{n}}=\dfrac{{{n}^2}+1}{n}$}
	{${{u}_{n}}={{(-1 )}^{n}}\cdot ({{2}^{n}}+1 )$}
	{\True ${{u}_{n}}=\sqrt{n}-\sqrt{n-1}\,$}
	{${{u}_{n}}=\sin n$}
	\loigiai {
		Xét ${{u}_{n}}=\sin n\Rightarrow  {{u}_{n+1}}-{{u}_{n}}=2\cos \left(n+\dfrac{1}{2} \right)\sin \dfrac{1}{2}$ có thể dương hoặc âm phụ thuộc $n$ nên đáp án sai. Hoặc dễ thấy $\sin n$ có dấu thay đổi trên ${{\mathbb{N}}^{*}}$ nên dãy $\sin n$ không tăng, không giảm.\\
		Xét ${{u}_{n}}=\dfrac{{{n}^2}+1}{n}=n+\dfrac{1}{n}\Rightarrow  {{u}_{n+1}}-{{u}_{n}}=1+\dfrac{1}{n+1}-\dfrac{1}{n}=\dfrac{{{n}^2}+n-1}{n(n+1 )}>0$ nên dãy đã cho tăng nên đáp án sai.\\
		Xét ${{u}_{n}}=\sqrt{n}-\sqrt{n-1}=\dfrac{1}{\sqrt{n}+\sqrt{n+1}},$ dãy $\sqrt{n}+\sqrt{n-1}>0$ là dãy tăng nên suy ra ${{u}_{n}}$ giảm. \\
		Xét ${{u}_{n}}={{(-1 )}^{n}}({{2}^{n}}+1 )$ là dãy thay dấu nên không tăng không giảm, nên đáp án đúng.\\
		Cách trắc nghiệm\\
		Xét ${{u}_{n}}=\sin n$ có dấu thay đổi trên ${{\mathbb{N}}^{*}}$ nên dãy này không tăng không giảm.\\
		Xét ${{u}_{n}}=\dfrac{{{n}^2}+1}{n}$, ta có $\heva{
			& n=1\to {{u}_{1}}=2 \\
			& n=2\to {{u}_{2}}=\dfrac{5}{2} \\
		}\Rightarrow {{u}_{1}}<{{u}_{2}}\Rightarrow {{u}_{n}}=\dfrac{{{n}^2}+1}{n}$ không giảm.\\
		Xét ${{u}_{n}}=\sqrt{n}-\sqrt{n-1}$, ta có $\heva{
			& n=1\to {{u}_{1}}=1 \\
			& n=2\to {{u}_{2}}=\sqrt{2}-1 \\
		}\Rightarrow {{u}_{1}}>{{u}_{2}}$ nên dự đoán dãy này giảm.\\
		Xét ${{u}_{n}}={{(-1 )}^{n}}({{2}^{n}}+1 )$ là dãy thay dấu nên không tăng không giảm.\\
		Cách CASIO.\\
		Các dãy $\sin n;\,\,{{(-1 )}^{n}}({{2}^{n}}+1 )$ có dấu thay đổi trên ${{\mathbb{N}}^{*}}$ nên các dãy này không tăng không giảm nên loại các đáp án này.\\
		Xét hai đáp án còn lại, ta chỉ cần kiểm tra một đáp án bằng chức năng $TABLE$.\\
		Chẳng hạn kiểm tra đáp án ${{u}_{n}}=\dfrac{{{n}^2}+1}{n}$, ta vào chức năng $TABLE$ nhập $F(X )=\dfrac{X^2+1}{X}$ với thiết lập $\text{Start}=1,\text{ End}=10,\text{ Step}=1$.\\
		Nếu thấy cột $F(X )$ các giá trị tăng thì loại ${{u}_{n}}=\dfrac{{{n}^2}+1}{n}$ nếu ngược lại nếu thấy cột $F(X )$ các giá trị giảm dần thị chọn ${{u}_{n}}=\dfrac{{{n}^2}+1}{n}$.}
	
\end{ex}
%%%%%%%%%%
%Câu 8
\begin{ex}%[1K2K5-3]
	Mệnh đề nào sau đây đúng?
	\choice
	{Dãy số ${{u}_{n}}=\dfrac{1}{n}-2$ là dãy tăng}
	{\True Dãy số ${{u}_{n}}=2n+\cos \dfrac{1}{n}$ là dãy tăng}
	{Dãu số ${{u}_{n}}=\dfrac{n-1}{n+1}$ là dãy giảm}
	{Dãy số ${{u}_{n}}={{(-1 )}^{n}}({{2}^{n}}+1 )$ là dãy giảm}
	\loigiai {
		Xét đáp án ${{u}_{n}}=\dfrac{1}{n}-2\Rightarrow {{u}_{n+1}}-{{u}_{n}}=\dfrac{1}{n+1}-\dfrac{1}{n}<0\Rightarrow $loại.\\
		Xét đáp án ${{u}_{n}}={{(-1 )}^{n}}({{2}^{n}}+1 )$ là dãy có dấu thay đổi nên không giảm nên loại.\\
		Xét đáp án ${{u}_{n}}=\dfrac{n-1}{n+1}=1-\dfrac{2}{n+1}\Rightarrow {{u}_{n+1}}-{{u}_{n}}=2\left(\dfrac{1}{n+1}-\dfrac{1}{n+2}\right)>0\Rightarrow $ loại.\\
		Xét đáp án ${{u}_{n}}=2n+\cos \dfrac{1}{n}\Rightarrow {{u}_{n+1}}-{{u}_{n}}=\left(2-\cos \dfrac{1}{n+1}\right)+\cos \dfrac{1}{n+2}>0$ chọn.}
	
\end{ex}
%%%%%%%%%%
%Câu 9
\begin{ex}%[1K2K5-3]
	Mệnh đề nào sau đây {\bf sai}?
	\haicot
	{Dãy số ${{u}_{n}}=\dfrac{1-n}{\sqrt{n}}$ là dãy giảm}
	{Dãy số ${{u}_{n}}=n+\sin ^2n$ là dãy tăng}
	{\True Dãy số ${{u}_{n}}={{\left(1+\dfrac{1}{n}\right)}^{n}}$ là dãy giảm}
	{Dãy số ${{u}_{n}}=2{{n}^2}-5$ là dãy tăng}
	\loigiai {
		Xét đáp án \\ ${{u}_{n}}=\dfrac{1-n}{\sqrt{n}}=\dfrac{1}{\sqrt{n}}-\sqrt{n}\Rightarrow {{u}_{n+1}}-{{u}_{n}}=\dfrac{1}{\sqrt{n+1}}-\dfrac{1}{\sqrt{n}}+\sqrt{n}-\sqrt{n+1}<0$ nên dãy $({{u}_{n}} )$ là dãy giảm nên đúng.\\
		Xét đáp án ${{u}_{n}}=2{{n}^2}-5$ là dãy tăng vì ${{n}^2}$ là dãy tăng nên đúng. \\
		Hoặc
		${{u}_{n+1}}-{{u}_{n}}=2(2n+1 )>0$ nên $({{u}_{n}} )$ là dãy tăng.\\
		Xét đáp án ${{u}_{n}}={{\left(1+\dfrac{1}{n}\right)}^{n}}={{\left(\dfrac{n+1}{n} \right)}^{n}}>0\Rightarrow\dfrac{{{u}_{n+1}}}{{{u}_{n}}}=\dfrac{n+2}{n+1}\cdot {{\left(\dfrac{n+2}{n}\right)}^{n}}>1\Rightarrow ({{u}_{n}} )$ là dãy tăng nên sai.\\
		Xét đáp án ${{u}_{n}}=n+\sin ^2n\Rightarrow {{u}_{n+1}}-{{u}_{n}}=(1-\sin ^2(n+1 ) )+\sin ^2n>0$.}
	
\end{ex}
%%%%%%%%%%
%Câu 10%VDC 
\begin{ex}%[Nguyễn Long]%[1K2G5-3]
	Cho dãy $(u_n)\colon\heva{&u_1=1\\&u_{n+1}=\dfrac{n}{2(n+1)}u_n+\dfrac{3(n+2)}{2(n+1)}},n \in \mathbb{N^*}$. Nhận xét nào sau đây đúng
	\choice
	{\True Dãy số $(u_n)$ là dãy số tăng}
	{Dãy số $(u_n)$ là dãy số giảm}
	{Dãy số $(u_n)$ là dãy số không tăng, không giảm}
	{Tất cả các đáp án còn lại đều sai}
	\loigiai{ Ta chứng minh quy nạp $u_n<3, \forall n \in N^*$.\\
		Giả sử mđ đúng với $\mathrm{n}=\mathrm{k}$ khi đó có:
		$$
		u_{k+1}=\dfrac{k}{2(k+1)} u_k+\dfrac{3(k+2)}{2(k+1)}<\dfrac{3 k}{2(k+2)}+\dfrac{3(k+2)}{2(k+1)}=3 .
		$$
		Vậy mệnh đề đúng với $\mathrm{n}=\mathrm{k}+1$.
		Từ đó ta có $$u_{n+1}-u_n=\dfrac{\left(3-u_n\right)(n+2)}{n+1}>0.$$
		Vậy dãy $\left(u_n\right)$ tăng }
\end{ex}
\Closesolutionfile{ans}
\begin{indapan}{10}
	{ans/ans-1K2-1-Dang3}
\end{indapan}

\begin{dang}{Xét tính bị chặn của dãy số}
	\begin{itemize}
		\item Để chứng minh dãy số $(u_n)$ bị chặn trên bởi $M$, ta chứng minh $u_n\le M$, $\forall n\in\mathbb{N}^\ast$.
		\item Để chứng minh dãy số $(u_n)$ bị chặn dưới bởi $m$, ta chứng minh $u_n\ge m$, $\forall n\in\mathbb{N}^\ast$.
		\item Để chứng minh dãy số bị chặn ta chứng minh nó bị chặn trên và bị chặn dưới.
		\begin{itemize}
			\item Nếu dãy số $(u_n)$ tăng thì bị chặn dưới bởi $u_1$.
			\item Nếu dãy số $(u_n)$ giảm thì bị chặn trên bởi $u_1$.
		\end{itemize}
	\end{itemize}
\end{dang}
\subsubsection{Ví dụ minh hoạ}

%ví dụ 1
\begin{vd}[NB]%[1K2Y5-4]%[Trương Đăng Khoa]
	Chứng minh rằng dãy số $(u_n)$ với $u_n=\dfrac{3n}{n^2+9}$ bị chặn trên bởi $\dfrac{1}{2}$.
	\dapso{dãy số đã cho bị chặn trên bởi $\dfrac{1}{2}$}
	\loigiai{
		Với mọi $n\ge 1$, ta có $\dfrac{3n}{n^2+9}\le\dfrac{1}{2}\Leftrightarrow n^2+9\le 6n\Leftrightarrow(n-3)^2\le 0$ (đúng).\\
		Vậy dãy số đã cho bị chặn trên bởi $\dfrac{1}{2}$.
	}
\end{vd}

%ví dụ 2
\begin{vd}[NB]%[1K2Y5-4]%[Trương Đăng Khoa]
	Chứng minh rằng dãy số $(u_n)$ xác đinh bởi $u_n=\dfrac{8n+3}{3n+5}$ là một dãy số bị chặn.
	\dapso{dãy số bị chặn}
	\loigiai{
		Ta có $u_n>0$, $\forall n\ge 1$. Suy ra dãy số bị chặn dưới.\\
		Mặt khác $u_n=\dfrac{8n+3}{3n+5}<\dfrac{8n+3}{3n}=\dfrac{8}{3}+\dfrac{1}{n}<\dfrac{8}{3}+1=\dfrac{11}{3}$. Do đó dãy số bị chặn trên bởi $\dfrac{11}{3}$.\\
		Vậy dãy số đã cho bị chặn.
	}
\end{vd}
%ví dụ 4
\begin{vd}[TH]%[1K2B5-4]%[Trương Đăng Khoa]
	Xét tính bị chặn của dãy số $\left(u_n\right)$ với $u_n=\dfrac{3n+1}{n+3}$.
	\dapso{dãy số bị chặn}
	\loigiai{
		Với $n\in \mathbb{N}^\ast$ ta có $u_n=\dfrac{3n+1}{n+3}>0$.\\
		Nên dãy $\left(u_n\right)$ bị chặn dưới bởi $0$.\\
		Mặt khác $u_n=\dfrac{3n+1}{n+3}=\dfrac{3n+9-8}{n+3}=3-\dfrac{8}{n+3}<3$, $\forall n\in\mathbb{N}^\ast$.\\
		Nên dãy $\left(u_n\right)$ bị chặn trên bởi $3$.\\
		Vậy dãy số $\left(u_n\right)$ bị chặn.
	}
\end{vd}
%ví dụ 3
\begin{vd}[VD]%[1K2K5-4]%[Trương Đăng Khoa]
	Cho dãy số $(u_n)$ xác định bởi $u_1=1$ và $u_{n+1}=\dfrac{u_n+2}{u_n+1}$, $\forall n\ge 1$. Chứng minh rằng dãy $(u_n)$ bị chặn trên bởi sô $\dfrac{3}{2}$ và bị chặn dưới bởi số $1$.	
	\loigiai{
		Ta chứng minh $1\le u_n\le\dfrac{3}{2},\forall n\ge 1$ bằng phương pháp quy nạp.
		\begin{itemize}
			\item Với $n=1$ ta có $1\le u_1\le\dfrac{3}{2}$.
			\item Giả sử $1\le u_n\le\dfrac{3}{2}$ với mọi $n=k\ge 1$, tức là $1\le u_k\le\dfrac{3}{2}$. Ta cần chứng minh $1\le u_{k+1}\le\dfrac{3}{2}$.
		\end{itemize}
		Thật vậy 
		$u_{k+1}=1+\dfrac{1}{u_k+1}$.\\
		Vì $u_k+1>0$ nên $u_{k+1}=1+\dfrac{1}{u_k+1}>1$.\\
		Vì $u_k+1\ge 2$ nên $u_{k+1}=1+\dfrac{1}{u_k+1}\le 1+\dfrac{1}{2}=\dfrac{3}{2}$.\\
		Vậy $1\le u_n\le\dfrac{3}{2}$, $\forall n\ge 1$ hay dãy $(u_n)$ bị chặn trên bởi số $\dfrac{3}{2}$ và bị chặn dưới bởi số $1$.
	}
\end{vd}

%ví dụ 5
\begin{vd}[VD]%[1K2K5-4]%[Trương Đăng Khoa]
	Xét tính bị chặn của dãy số $\left(u_n\right)$ với $u_n=\sin n+ \cos n$.
	\dapso{dãy số bị chặn}
	\loigiai{
		Ta có $\begin{aligned}[t]
			&\ \sin n+\cos n \\
			=&\ \sqrt{2}\left(\dfrac{1}{\sqrt{2}}\sin n+\dfrac{1}{\sqrt{2}}\cos n\right)\\
			=&\ \sqrt{2}\left(\sin n\cdot\cos \dfrac{\pi}{4}+\cos n\cdot\sin \dfrac{\pi}{4}\right)\\
			=&\ \sqrt{2}\sin \left(n+\dfrac{\pi}{4}\right).
		\end{aligned}$\\
		Vì $\begin{aligned}[t]
			&\ -1\leq \sqrt{2}\sin \left(n+\dfrac{\pi}{4}\right) \leq 1\\
			\Rightarrow&\ -\sqrt{2}\leq \sqrt{2} \sin \left(n+\dfrac{\pi}{4}\right)\leq \sqrt{2}\\
			\Rightarrow&\ -\sqrt{2}\leq \sin n+\cos n \leq \sqrt{2},\ \forall n\in\mathbb{N}^\ast\\
			\Rightarrow&\ -\sqrt{2}\leq u_n \leq \sqrt{2},\ \forall n\in\mathbb{N}^\ast.
		\end{aligned}$\\
		Vậy dãy số $\left(u_n\right)$ là dãy số bị chặn.}
\end{vd}
\subsubsection{Bài tập tự luận}
 
%Bài 1 
\begin{bt}[TH]%[1K2B5-4]%[Trương Đăng Khoa]
	Xét tính bị chặn của các dãy số sau 
		\begin{listEX}[3]
			\item $u_n=\dfrac{1}{2n^2-1}$.
			\item 
			$u_n=3\cdot\cos\dfrac{n x}{3}$. 
			\item  $u_n=2n^3+1$.
			\item  $u_n=\dfrac{n^2+2n}{n^2+n+1}$.
			\item  $u_n=n+\dfrac{1}{n}$.
		\end{listEX}
	\loigiai{
		\begin{enumerate}
			\item  $u_n=\dfrac{1}{2n^2-1}$.\\
			Ta có $2n^2-1\ge 1\Rightarrow u_n=\dfrac{1}{2n^2-1}\le 1$, $\forall n\ge 1$.\\
			Vậy dãy số bị chặn trên bởi $1$.\\
			\item $u_n=3\cdot\cos\dfrac{n x}{3}$ có $-1\le\cos\dfrac{n x}{3}\le 1\Rightarrow-3\le 3\cdot\cos\dfrac{n x}{3}\le 3$.\\
			Vậy dãy số bị chặn dưới bởi $-3$ và chặn trên bởi $3$.
			\item  $u_n=2n^3+1$ có $2n^3+1\ge 3$, $\forall n\ge 1$.\\
			Vậy dãy số bị chặn dưới bởi $3$.
			\item $u_n=\dfrac{n^2+2n}{n^2+n+1}$ có $u_n=\dfrac{n^2+2n}{n^2+n+1}=1+\dfrac{n-1}{n^2+n+1}\ge 1$, $\forall n\ge 1$.\\
			Vậy dãy số bị chặn dưới bởi $1$.
			\item  $u_n=n+\dfrac{1}{n}$ có $u_n=n+\dfrac{1}{n}\ge 2\sqrt{n\cdot\dfrac{1}{n}}=2$, $\forall n>0$.\\
			Vậy dãy số bị chặn bởi $2$.
		\end{enumerate}
	}
\end{bt}
%Bài 2
\begin{bt}[VD]%[1K2K5-4]%[Trương Đăng Khoa]
	Xét tính bị chặn của dãy số $(u_n)$ với:
	\begin{listEX}[3]	 
		\item $u_{n}=\dfrac{4}{n}-5$.
		\item $u_{n}=\dfrac{n+4}{n+2}$.
		\item $u_{n}=\dfrac{5}{n^2+1}+\dfrac{n+2}{n+1}+\cos n$.
	\end{listEX}
	\loigiai{
		\begin{enumerate} 
			\item $u_{n}=\dfrac{4}{n}-5$.\\
			Ta có $u_n=\dfrac{4}{n}-5 \le \dfrac{4}{1}-5=-1$, $\forall n \in \mathbb{N}^{*}$ suy ra dãy $(u_n)$ bị chặn trên bởi $-1$.\\
			Mặt khác $u_n=\dfrac{4}{n}-5 \ge -5 \,\, \forall n \in \mathbb{N}^{*}$ suy ra dãy $(u_n)$ bị chặn dưới bởi $-5$.\\
			Vậy dãy $(u_n)$ bị chặn.	
			\item $u_{n}=\dfrac{n+4}{n+2}$.\\
			Ta có $u_n=	\dfrac{n+4}{n+2}=1+\dfrac{2}{n+2}> 1$, $\forall n \in \mathbb{N}^{*}$ suy ra dãy $(u_n)$ bị chặn dưới bởi $1$.\\
			Mặt khác $u_n=\dfrac{n+4}{n+2}=1+\dfrac{2}{n+2} \le 1+\dfrac{2}{1+2}=\dfrac{3}{5}$, $\forall n \in \mathbb{N}^{*}$ suy ra dãy $(u_n)$ bị chặn trên bởi $\dfrac{3}{5}$.\\
			Vậy dãy $(u_n)$ bị chặn.
			\item $u_{n}=\dfrac{5}{n^2+1}+\dfrac{n+2}{n+1}+\cos n$.\\
			Ta có $u_n=	\dfrac{5}{n^2+1}+\dfrac{n+2}{n+1}+\cos n=\dfrac{5}{n^2+1}+1+\dfrac{1}{n+1}+\cos n<5$, $\forall n \in \mathbb{N}^{*}$.\\
			Suy ra dãy $(u_n)$ bị chặn trên bởi $5$.\\
			Mặt khác $u_n=	\dfrac{5}{n^2+1}+\dfrac{n+2}{n+1}+\cos n=\dfrac{5}{n^2+1}+1+\dfrac{1}{n+1}+\cos n>0$, $\forall n \in \mathbb{N}^{*}$.\\
			Suy ra dãy $(u_n)$ bị chặn trên bởi $0$.\\
			Vậy dãy $(u_n)$ bị chặn.
		\end{enumerate}			
	}		
\end{bt}
%Bài 3
\begin{bt}[VDC]%[1K2G5-4]%[Trương Đăng Khoa]
	Xét tính bị chặn của dãy số $u_n=\left(1+\dfrac{1}{n}\right)^n$, $n\in N^\ast$.
	\loigiai{
		Ta có $u_n=\left(1+\dfrac{1}{n}\right)^n>0$, $\forall n\in N^\ast$ nên $(u_n)$ bị chặn dưới $(1)$.\\
		Lại có $\begin{aligned}[t]
			u_n&\ =\left(1+\dfrac{1}{n}\right)^n=\displaystyle\sum\limits_{k=0}^n C_n^k\left(\dfrac{1}{n}\right)^k\\
			&\ =\displaystyle\sum\limits_{k=0}^n\left[\dfrac{n!}{k!\cdot(n-k)!\cdot n^k}\right]\\
			&\ =\displaystyle\sum\limits_{k=0}^n\left[\dfrac{1}{k!}\cdot\dfrac{(n-k+1)}{n}\cdot\dfrac{(n-k+2)}{n}\ldots\dfrac{(n-k+k)}{n}\right]\le\displaystyle\sum\limits_{k=0}^n\dfrac{1}{k!},\, n\in \mathbb{N}^\ast
		\end{aligned}$\\
		Mà $\begin{aligned}[t]
			\displaystyle\sum\limits_{k=0}^n\dfrac{1}{k!}&\ \le 1+1+\dfrac{1}{1\cdot2}+\dfrac{1}{2\cdot3}+\dfrac{1}{3\cdot4}+\ldots+\dfrac{1}{(n-1)\cdot n}\\
			&\ =2+\left(1-\dfrac{1}{2}\right)+\left(\dfrac{1}{2}-\dfrac{1}{3}\right)+\ldots+\left(\dfrac{1}{n-1}-\dfrac{1}{n}\right)\\
			&\ 
			=3-\dfrac{1}{n}<3,\, \forall n\in \mathbb{N}^\ast.
		\end{aligned}$\\
		Suy ra $u_n<3$, $\forall n\in \mathbb{N}^\ast$ nên dãy số $(u_n)$ bị chặn trên $(2)$.\\
		Từ $(1)$ và $(2)$  suy ra dãy số $(u_n)$ bị chặn.}
\end{bt}
%Bài 4
\begin{bt}[VD]%[1K2K5-4]%[Trương Đăng Khoa]
	Cho dãy số $(u_n)$ xác định bởi $u_1=0$ và $u_{n+1}=\dfrac{1}{2}u_n+4$, $ \forall n\geq 1$.
	\begin{enumerate}
		\item Chứng minh dãy $(u_n)$ bị chặn trên bởi số $8$.
		\item Chứng minh dãy $(u_n)$ tăng, từ đó suy ra dãy $(u_n)$ bị chặn.
	\end{enumerate}
	\loigiai{
		\begin{enumerate}
			\item Ta chứng minh $u_n\leq 8$ với mọi $n\geq 1$.
			\begin{itemize}
				\item Khi $n=1$, ta có $u_1=0 <8$.
				\item Giả sử $u_n\leq 8$ với $n=k\geq 1$, tức là $u_k\leq 8$.\\
				Ta cần chứng minh $u_{k+1}\leq 8$.\\
				Thật vậy, $u_{k+1}=\dfrac{1}{2}u_k+4\leq \dfrac{1}{2}\cdot 8+4\leq 8$.
			\end{itemize}
			Vậy $u_n\leq 8$ với mọi $n\geq 1$, hay $(u_n)$ bị chặn trên bởi $8$.
			\item Với mọi $n\geq 1$, ta có $u_{n+1}-u_n=4-\dfrac{1}{2}u_n$. Mà $u_n\leq 8$ nên $u_{n+1}-u_n\geq 0$.\\
			Suy ra $u_n$ là dãy số tăng. Do đó $(u_n)$ bị chặn dưới bởi $u_1=0$.\\
			Kết hợp với câu a, ta được dãy số $(u_n)$ bị chặn.
		\end{enumerate}
	}
\end{bt}
%Bài 5
\begin{bt}[VD]%[1K2K5-4]%[Trương Đăng Khoa]
	Trong các dãy số $(u_n)$ sau, dãy số nào bị chặn trên, bị chặn dưới và bị chặn?
	\begin{listEX}[3]
		\item $u_n=n^2+5$.
		\item $u_n=\dfrac{3n+1}{2n+5}$.
		\item $u_n=(-1)^n\cos \dfrac{\pi}{2n}$.
		\item $u_n=\dfrac{n^2+2n}{n^2+n+1}$.
		\item $u_n=\dfrac{n}{\sqrt{n^2+2n}+n}$.
	\end{listEX}
	\loigiai{
		\begin{enumerate}
			\item Dãy số bị chặn dưới bởi $6$, không bị chặn trên.
			\item Dãy $(u_n)$ bị chặn dưới bởi $0$. Vì $u_n<\dfrac{3n+1}{2n}=\dfrac{3}{2}+\dfrac{1}{2n}<\dfrac{3}{2}+1=\dfrac{5}{2}$ nên dãy số bị chặn trên bởi $\dfrac{5}{2}$. Vậy dãy số bị chặn.
			\item Ta có $|u_n|\leq 1$ nên dãy số bị chặn trên bởi 1, bị chặn dưới bởi $-1$.
			\item Dãy số bị chặn dưới bởi $0$. Vì $u_n<\dfrac{n^2+2n}{n^2}=1+\dfrac{2}{n}\leq 3$ nên dãy số bị chặn trên. Vậy dãy số bị chặn.
			\item Ta có $0<u_n\leq 1$ vậy dãy số bị chặn.
		\end{enumerate}
	}
\end{bt}

\subsubsection{Câu hỏi trắc nghiệm}

\Opensolutionfile{ans}[ans/ans-1K2-1-Dang4]
%Câu 1
\begin{ex}%[1K2K5-4]%[Trương Đăng Khoa]
	Cho dãy số $(u_n)$ xác định bởi $u_1=3$ và $u_{n+1}=\dfrac{u_n+1}{2}$, $\forall n\geq 1$. Mệnh đề nào sau đây là đúng?
	\choice
	{\True Dãy số bị chặn}
	{Dãy số bị chặn trên}
	{Dãy số bị chặn dưới}
	{Dãy số không bị chặn}
	\loigiai{Ta chứng minh $u_n>1, \forall n\geq 1$ bằng phương pháp quy nạp.\\
		Suy ra dãy số bị chặn dưới bởi $1$.\\
		Ta có
		$u_{n+1}-u_n=\dfrac{1-u_n}{2}<0$, $\forall n\geq 1$.\\
		Do đó dãy số này là dãy số giảm nên nó bị chặn trên bởi $u_1=3$.\\
		Vậy dãy số đã cho là dãy số bị chặn.		
	}
\end{ex}
%Câu 2
\begin{ex}%[1K2K5-4]%[Trương Đăng Khoa]
	Cho dãy số $(u_n)$ xác định bởi $u_1=\sqrt{2}$ và $u_{n+1}=\sqrt{2+u_n}$, $\forall n\geq 1$. Mệnh đề nào sau đây là đúng?
	\choice
	{Dãy số bị chặn trên}
	{Dãy số bị chặn dưới}
	{\True Dãy số bị chặn}
	{Dãy số không bị chặn}
	\loigiai{Vì $u_n\geq 0$, $\forall n\geq 1$ nên dãy số bị chặn dưới bởi $0$.\\
		Ta chứng minh $u_n\geq 2, \forall n\geq 1$. Suy ra dãy số bị chặn trên bởi $2$.\\
		Vậy dãy số đã cho là dãy số bị chặn.		
	}
\end{ex}
%Câu 3
\begin{ex}%[1K2K5-4]%[Trương Đăng Khoa]
	Xét tính bị chặn của  dãy số $(u_n)$ với $u_n=\dfrac{1}{1\cdot2}+\dfrac{1}{2\cdot3}+\ldots+\dfrac{1}{n\cdot(n+1)}$.
	\choice{Không bị chặn}{Bị chặn trên}{Bị chặn dưới}{\True Bị chặn}
	\loigiai{
		Ta có $u_n=1-\dfrac{1}{2}+\dfrac{1}{2}-\dfrac{1}{3}+\ldots+\dfrac{1}{n}-\dfrac{1}{n+1}=1-\dfrac{1}{n+1}$.\\
		Do đó $0\leq u_n \leq 1$, $\forall n\geq 1$.\\
		Vậy dãy số đã cho bị chặn.
	}
\end{ex}
%Câu 4
\begin{ex}%[1K2G5-4]%[Trương Đăng Khoa]
	Cho dãy số $(u_n)$ với $u_n=\dfrac{1}{1\cdot4}+\dfrac{1}{2\cdot5}+\ldots+\dfrac{1}{n\cdot(n+3)}$. Dãy số $\left(u_n\right)$ bị chặn dưới và chặn trên lần lượt bởi các số $m$ và $M$ nào dưới đây?
	\choice
	{$m=0$, $M=1$}
	{$m=1$, $M=\dfrac{1}{2}$}
	{$m=1$, $M=\dfrac{10}{19}$}
	{\True $m=0$, $M=\dfrac{11}{18}$}
	\loigiai{
		Rõ ràng $u_n>0$, $\forall n\in\mathbb{N}^\ast$ nên $(u_n)$ bị chặn dưới.\\
		Mặt khác $\dfrac{1}{k(k+3)}=\dfrac{1}{3}\left(\dfrac{1}{k}-\dfrac{1}{k+3}\right)$.\\
		Suy ra $\begin{aligned}[t]
			u_n&\  =\dfrac{1}{3}\bigg[\left(1-\dfrac{1}{4}\right)+\left(\dfrac{1}{2}-\dfrac{1}{5}\right)+\left(\dfrac{1}{3}-\dfrac{1}{6}\right)+\left(\dfrac{1}{4}-\dfrac{1}{7}\right)+\\
			&\  \ldots+\left(\dfrac{1}{n-3}-\dfrac{1}{n}\right)+\left(\dfrac{1}{n-2}-\dfrac{1}{n+1}\right)+\left(\dfrac{1}{n-1}-\dfrac{1}{n+2}\right)+\left(\dfrac{1}{n}-\dfrac{1}{n+3}\right)\bigg]\\
			&\ = \dfrac{1}{3}\left(1+\dfrac{1}{2}+\dfrac{1}{3}-\dfrac{1}{n+1}-\dfrac{1}{n+2}-\dfrac{1}{n+3}\right)<\dfrac{11}{18}, \, \forall n\in\mathbb{N}^\ast.
		\end{aligned}$\\
		Do đó $(u_n)$ bị chặn trên.\\
		Vậy $m=0$, $M=\dfrac{11}{18}$.
	}
\end{ex}
%Câu 5
\begin{ex}%[1K2G5-4]%[Trương Đăng Khoa]
	Cho dãy số $(u_n)$ biết $u_n=\dfrac{1\cdot 3\cdot 5\ldots(2n-1)}{2\cdot 4\cdot 6\cdot 2n}$. Dãy số $\left(u_n\right)$ bị chặn dưới và chặn trên lần lượt bởi các số $m$ và $M$. Tính giá trị biểu thức $m+M$?
	\choice{$\dfrac{1}{\sqrt{2}}$}{\True $\dfrac{1}{\sqrt{3}}$}{$\dfrac{1}{\sqrt{5}}$}{$\dfrac{1}{\sqrt{7}}$}
	\loigiai{
		Xét $ \dfrac{2 k-1}{2 k}<\dfrac{2 k-1}{\sqrt{4 k^2-1}}
		=\dfrac{\sqrt{(2 k-1)^2}}{\sqrt{(2 k-1)(2 k+1)}} =\dfrac{\sqrt{2 k-1}}{\sqrt{2 k+1}}$,  $\forall k \ge 1$.\\
		$\Rightarrow u_n<\dfrac{\sqrt{1}}{\sqrt{3}} \cdot\dfrac{\sqrt{3}}{\sqrt{5}} \cdot\dfrac{\sqrt{5}}{\sqrt{7}}\cdot \ldots \cdot \dfrac{\sqrt{2 n-1}}{\sqrt{2 n+1}}=\dfrac{1}{\sqrt{2 n+1}} \le\dfrac{1}{\sqrt{3}}$,  $\forall n \in\mathbb{N}^\ast$.\\
		$\Rightarrow 0<u_n<\dfrac{1}{\sqrt{3}}$, $\forall n \in\mathbb{N}^\ast$.\\
		Vậy $m+M=0+\dfrac{1}{\sqrt{3}}$.
	}
\end{ex}
%Câu 6
\begin{ex}%[1K2G5-4]%[Trương Đăng Khoa]
	Cho dãy số $(u_n)$, với $u_n=\dfrac{1}{2^2}+\dfrac{1}{3^2}+\ldots+\dfrac{1}{n^2}$, $\forall n=2;3;4;\ldots$. Khẳng định nào sau đây là đúng?
	\choice
	{\True Dãy số bị chặn}
	{Dãy số bị chặn trên}
	{Dãy số bị chặn dưới}
	{Dãy số không bị chặn}
	\loigiai{
		Ta có $u_n>0\Rightarrow(u_n)$ bị chặn dưới bởi $0$.\\
		Mặt khác $\dfrac{1}{k^2}<\dfrac{1}{(k-1) k}=\dfrac{1}{k-1}-\dfrac{1}{k}$, ($k\in\mathbb{N}^\ast$, $k\ge 2$) nên suy ra
		\begin{eqnarray*}
			u_n&<&\dfrac{1}{1 \cdot 2}+\dfrac{1}{2 \cdot 3}+\dfrac{1}{3 \cdot 4}+\cdots+\dfrac{1}{n(n+1)}\\
			&=&1-\dfrac{1}{2}+\dfrac{1}{2}-\dfrac{1}{3}+\dfrac{1}{2}-\dfrac{1}{4}+\cdots+\dfrac{1}{n}-\dfrac{1}{n+1}=1-\dfrac{1}{n+1}<1.
		\end{eqnarray*}
		Nên dãy $(u_n)$ bị chặn trên, do đó dãy $(u_n)$ bị chặn.
	}
\end{ex}
%Câu 7
\begin{ex}%[1K2G5-4]%[Trương Đăng Khoa]
	Cho dãy số $\left(u_n\right)$ và đặt $u_n= \displaystyle \sum_{k=1}^{n} a_k$ với $a_k=\dfrac{1}{4k^2-1}$. Mệnh đề nào sau đây là đúng?
	\choice{$0<u_n <1$}
	{$0\leq u_n\leq \dfrac{1}{2}$}
	{\True $0<u_n<\dfrac{1}{2}$}
	{$0\leq u_n\leq 1$}
	\loigiai{
		\begin{itemize}
			\item
			Ta có $a_k=\dfrac{1}{4k^2-1}=\dfrac{1}{(2k+1)(2k-1)}=\dfrac{1}{2}\cdot\dfrac{(2k+1)-(2k-1)}{(2k+1)(2k-1)}=\dfrac{1}{2}\cdot\left(\dfrac{1}{2k-1}-\dfrac{1}{2k+1}\right)$.\\
			\item Mặt khác $u_n=\displaystyle \sum_{k=1}^{n} a_k$.
			Do đó
			\begin{eqnarray*}
				&u_n&=\dfrac{1}{2}\cdot\left(\dfrac{1}{1}-\dfrac{1}{3}\right)+\dfrac{1}{2}\cdot\left(\dfrac{1}{3}-\dfrac{1}{5}\right)+\ldots + \dfrac{1}{2}\cdot \left(\dfrac{1}{2n-1}-\dfrac{1}{2n+1}\right)\\
				&&=\dfrac{1}{2}\left(\dfrac{1}{1}-\dfrac{1}{2n+1}\right)\\
				&&=\dfrac{1}{2}\cdot\dfrac{2n}{2n+1}=\dfrac{n}{2n+1}.
			\end{eqnarray*}
			\item 
			
			Với mọi $n \in \mathbb{N}^\ast$ thì $u_n>0$ nên dãy số $\left(u_n\right)$ bị chặn dưới.\\
			Ta lại có $u_n=\dfrac{1}{2}\cdot\left(1-\dfrac{1}{2n+1}\right)<\dfrac{1}{2}$.\\
			Vậy dãy số bị chặn.
		\end{itemize}
	}
\end{ex}
%Câu 8
\begin{ex}%[1K2G5-4]%[Trương Đăng Khoa]
	Cho dãy số $\left(u_n\right)$ và đặt $u_n= \displaystyle \sum_{k=1}^{n} a_k$ với $a_k=\dfrac{1}{k(k+4)}$.  Dãy số $\left(u_n\right)$ bị chặn dưới và chặn trên lần lượt bởi các số $m$ và $M$ nào sau đây?
	\choice
	{\True $m=0$, $M=\dfrac{25}{48}$}
	{$m=0$, $M=\dfrac{25}{12}$}
	{$m=1$, $M=\dfrac{1}{4}$}
	{$m=1$, $M=\dfrac{1}{2}$}
	\loigiai{
		Ta có $a_k=\dfrac{1}{k(k+4)}=\dfrac{1}{4}\cdot\dfrac{4}{k(k+4)}=\dfrac{1}{4}\cdot\dfrac{k+4-k}{k(k+4)}=\dfrac{1}{4}\cdot\left(\dfrac{1}{k}-\dfrac{1}{k+4}\right)$.\\
		Mặt khác $u_n=\displaystyle \sum_{k=1}^{n} a_k$.
		Do đó
		\begin{eqnarray*}
			&u_n&=\dfrac{1}{4}\cdot\left(\dfrac{1}{1}-\dfrac{1}{5}\right)+\dfrac{1}{4}.\left(\dfrac{1}{2}-\dfrac{1}{6}\right)+\ldots + \dfrac{1}{4}\cdot\left(\dfrac{1}{n}-\dfrac{1}{n+4}\right)\\
			&&=\dfrac{1}{4}\left(\dfrac{1}{1}+\dfrac{1}{2}+\dfrac{1}{3}+\dfrac{1}{4}-\dfrac{1}{n+1}-\dfrac{1}{n+2}-\dfrac{1}{n+3}-\dfrac{1}{n+4}\right)\\
			&&=\dfrac{1}{4}\left(\dfrac{25}{12}-\dfrac{1}{n+1}-\dfrac{1}{n+2}-\dfrac{1}{n+3}-\dfrac{1}{n+4}\right).
		\end{eqnarray*}
		Với mọi $n \in \mathbb{N}^\ast$ thì $u_n>0$ nên dãy số $\left(u_n\right)$ bị chặn dưới.\\
		Ta lại có $u_n=\dfrac{1}{4}\cdot\left(\dfrac{25}{12}-\dfrac{1}{n+1}-\dfrac{1}{n+2}-\dfrac{1}{n+3}-\dfrac{1}{n+4}\right)<\dfrac{1}{4}\cdot\dfrac{25}{12}=\dfrac{25}{48}$.\\
		Vậy $m=0$, $M=\dfrac{25}{48}$.
	}
\end{ex}
%Câu 9
\begin{ex}%[1K2G5-4]%[Trương Đăng Khoa]
	Xét tính bị chặn của dãy số $\left(u_n\right)$ và đặt $u_n=\displaystyle \sum_{k=1}^{n} a_k$ với $a_k=\dfrac{1}{k(k+1)}$.
	\choice{\True Bị chặn}{Bị chặn dưới}{Bị chặn trên}{Không bị chặn.}
	\loigiai{
		Ta có $a_k=\dfrac{1}{k(k+1)}=\dfrac{1}{k}-\dfrac{1}{k+1}$. Do đó\\
		$u_n=\displaystyle \sum_{k=1}^{n}a_k=\left(1-\dfrac{1}{2}\right)+\left(\dfrac{1}{2}-\dfrac{1}{3}\right)+\ldots+\left(\dfrac{1}{n-1}-\dfrac{1}{n}\right)+\left(\dfrac{1}{n}-\dfrac{1}{n+1}\right)=1-\dfrac{1}{n+1}=\dfrac{n}{n+1}$.\\
		Với mọi $n \in \mathbb{N}^*$ thì $u_n>0$ nên dãy số $\left(u_n\right)$ bị chặn dưới.\\
		Ta lại có $u_n=1-\dfrac{n}{n+1}<1$, $\forall n \in \mathbb{N}^\ast$ nên dãy số $\left(u_n\right)$ bị chặn trên.\\
		Vậy dãy số bị chặn.
	}
\end{ex}
%Câu 10
\begin{ex}%[1K2G5-4]%[Trương Đăng Khoa]
	Cho dãy số $(u_n)$, xác định bởi $\heva{&u_1=6\\&u_{n+1}=\sqrt{6+u_n},\, \forall n\in\mathbb{N}^\ast}$. Mệnh đề nào sau đây là đúng?
	\choice{
		$\sqrt{6}<u_n<2\sqrt{3}$	
	}
	{\True $\sqrt{6}\leq u_n\leq 2\sqrt{3}$}
	{$\sqrt{6}<u_n\leq 2\sqrt{3}$	}
	{$\sqrt{6}\geq u_n<2\sqrt{3}$	}
	\loigiai{
		Ta có 
		$\heva{&u_1=6\\
			&u_{n+1}=\sqrt{6+u_n}} \Rightarrow
		\heva{
			&u_1=6\\
			&u_{n+1} \ge 0 } \Rightarrow u_n \ge 0 \Rightarrow\heva{&u_1=6\\
			&u_{n+1}=\sqrt{6+u_n}\ge\sqrt{6}}
		\Rightarrow u_n \ge\sqrt{6}$.\\
		Ta chứng minh quy nạp $\heva{&u_n\le 2\sqrt{3}\\ &u_1\le 2\sqrt{3}\\ &u_k\le 2\sqrt{3}.}$\\
		$\Rightarrow u_{k+1}=\sqrt{6+u_{k+1}} \le\sqrt{6+2 \sqrt{3}}<\sqrt{6+6}=2 \sqrt{3}$.\\
		Vậy $\sqrt{6} \leq  u_n \leq 2\sqrt{3}$.
	}
\end{ex}
\Closesolutionfile{ans}
\begin{indapan}{10}
	{ans/ans-1K2-1-Dang3}
\end{indapan}

\begin{dang}{Toán thực tế về dãy số}
\end{dang}
\subsubsection{Ví dụ minh hoạ}
% \begin{vd}%[1T2B1-5]%[Trương Đăng Khoa]%Ví dụ 1
% 	Một chồng cột gỗ được xếp thành các lớp, hai lớp liên tiếp hơn kém nhau một cột gỗ.
% 	\begin{center}
% 		\begin{tikzpicture}[font=\footnotesize, line join=round, line cap=round, >=stealth,scale=0.8]
% 			\def\r{0.2}
% 			\def\n{25}
% 			\def\g{110}
% 			\fill[teal!50!green](-6*\r,-3*\r)rectangle(3.5*\n*\r,0.5*\n*\r);
% 			\fill[teal!50!green,opacity=0.25](3.5*\n*\r,3*\r)rectangle(-6*\r,2*\n*\r);
% 			\foreach \j in {0,...,12}{
% 				\pgfmathsetmacro{\m}{\n-\j}
% 				\foreach \i in{0,...,\m}{
% 					\fill[left color=orange, right color=teal!30,draw=brown](2*\i*\r,0)++(60:2*\j*\r)++(\g:\r)--++(\g-90:6)arc(\g:-60:\r)--++(\g-270:6)--cycle;
% 					\fill[orange!20!brown!40,draw=teal](2*\i*\r,0)++(60:2*\j*\r)circle(\r);
% 				}
% 			}
% 		\end{tikzpicture}
% 	\end{center}
% 	\begin{enumerate}
% 		\item  Gọi $u_1=25$ là số cột gỗ có ở hàng dưới cùng của chồng cột gỗ, $u_n$ là số cột gỗ có ở hàng thứ $n$ tính từ dưới lên trên. Xét tính tăng, giảm của dãy số này.
% 		\item  Gọi $v_1=14$ là số cột gỗ có ở hàng trên cùng của chồng cột gỗ, $v_n$ là số cột gỗ có ở hàng thứ $n$ tính từ trên xuống dưới. Xét tinh tăng, giảm của dãy số này.
% 	\end{enumerate}
% 	\loigiai{
% 		\begin{enumerate}
% 			\item Ta có $u_n=26-n>u_{n+1}=26-n-1=25-n$.\\
% 			Vậy dãy số $(u_n)$ là dãy số giảm.
% 			\item Ta có $v_n=13+n<v_{n+1}=13+n+1=14+n$.\\
% 			Vậy dãy số $(u_n)$ là dãy số tăng
% 		\end{enumerate}
% 	}
% \end{vd}

\begin{vd}%[1T2B1-5]%[Trương Đăng Khoa]%Ví dụ 2
	Trên lưới ô vuông, mỗi ô cạnh $1$ đơn vị, người ta vẽ $8$ hình vuông và tô màu khác nhau như hình vẽ. Tìm dãy số biểu diễn độ dài cạnh của $8$ hình vuông đó từ nhỏ đến lớn. Có nhận xét gì về dãy số trên?
	\begin{center}
		\begin{tikzpicture}[scale=0.8]
			\def\r{21}
			\def\hv(#1){
				\ifnum #1= 1\else
				\pgfmathsetmacro{\R}{250*rnd}
				\pgfmathsetmacro{\G}{250*rnd}
				\pgfmathsetmacro{\B}{250*rnd}
				\definecolor{mau}{RGB}{\R,\G,\B}
				\fill[mau!30](0,0)rectangle(\r,\r);
				\draw[red,line width=1pt] (0,0) arc(180:90:\r)(0,0)rectangle(\r,\r);
				\pgfmathtruncatemacro{\k}{#1-1}
				\begin{scope}[shift={(45:\r*sqrt(2))},rotate=-90,scale={(sqrt(5)-1)/2}]
					\hv(\k)
					\pgfmathsetmacro{\n}{int((1/(sqrt(5))*(((1+sqrt(5))/2)^(\k)-(1-(sqrt(5))/2)^(\k)+1)}
					\ifnum \k>1
					\path (\r/2,\r/2)node[scale=1.75]{\color{red}$\n$};
					\else
					\fi
				\end{scope}
				\fi
			}
			\begin{scope}[scale=0.35]
				\hv(9)
				\draw[teal](0,0)grid(34,21);
				\path(21/2,21/2)node[scale=2]{21};
			\end{scope}
		\end{tikzpicture}
	\end{center}
	\loigiai{
		\begin{multicols}{4}
			\begin{itemize}
				\item $u_1=1$.
				\item $u_2=1$.
				\item $u_3=2$.
				\item $u_4=3$.
				\item $u_5=5$.
				\item $u_6=8$.
				\item $u_7=13$.
				\item $u_8=21$.
			\end{itemize}
		\end{multicols}
		Ta có dãy số $\left(u_n\right)\colon\heva{& u_1=1\\ &u_2=1\\ &u_n=u_{n-1}-u_{n-2}.}$
	}
\end{vd}

\begin{vd}%[1C2K1-5]%[Trương Đăng Khoa]% Ví dụ 3
	Chị Mai gửi tiền tiết kiệm vào ngân hàng theo thể thức lãi kép như sau. Lần đầu chị gửi $100$ triệu đồng. Sau đó, cứ hết $1$ tháng chị lại gửi thêm vào ngân hàng $6$ triệu đồng. Biết lãi suất của ngân hàng là $0{,}5\%$ một tháng. Gọi $P_n$ (triệu đồng) là số tiền chị có trong ngân hàng sau $n$ tháng.
	\begin{enumerate}
		\item Tính số tiền chị có trong ngân hàng sau $1$ tháng.
		\item Tính số tiền chị có trong ngân hàng sau $3$ tháng.
		\item Dự đoán công thức của $P_n$ tính theo $n$.
	\end{enumerate}
	\loigiai{
		\begin{enumerate}
			\item Số tiền chị có trong ngân hàng sau $1$ tháng là $P_1=+100+100\cdot0{,}5\%+6=100{,}5+6$ (triệu đồng).
			\item Số tiền chị có trong ngân hàng sau 2 tháng là 
			\begin{eqnarray*}
				P_2&=&100{,}5+6+(100{,}5+6)\cdot 0{,}5\%+6\\
				&=&(100{,}5+6)(1+0{,}5\%)+6\\
				&=& 100{,}5(1+0{,}5\%)+6\cdot(1+0{,}5\%)+6\, (\text{triệu đồng}).
			\end{eqnarray*}
			Số tiền chị có trong ngân hàng sau $3$ tháng là
			\begin{eqnarray*}
				P_3&=&(100{,}5+6)(1+0{,}5 \%)+6+[(100{,}5+6)(1+0{,}5 \%)+6] \cdot 0{,}5 \%+6\\
				&=& 100{,}5 \cdot(1+0{,}5 \%)^2+6(1+0{,}5 \%)^2+6 \cdot(1+0{,}5 \%)+6 \text{(triệu đồng)}.
			\end{eqnarray*}
			\item Số tiền chị có trong ngân hàng sau $4$ tháng là
			\begin{eqnarray*}
				P_4&=&(100{,}5+6)(1+0{,}5 \%)^2+6 \cdot(1+0{,}5 \%)+6+\left[(100{,}5+6)(1+0{,}5 \%)^2+6 \cdot(1+0{,}5 \%)+6\right]\cdot 0{,}5 \%+6\\ 
				&=&100{,}5 \cdot(1+0{,}5 \%)^3+6 \cdot(1+0{,}5 \%)^3+6\cdot(1+0{,}5 \%)^2+6 \cdot(1+0{,}5 \%)+6\, (\text{triệu đồng}).
			\end{eqnarray*}
			Số tiền chị có trong ngân hàng sau $n$ tháng là
			$$P_n=100{,}5 \cdot(1+0{,}5 \%)^{n-1}+6\cdot(1+0{,}5 \%)^{n-1}+6\cdot(1+0{,}5 \%)^{n-2}+6 \cdot(1+0{,}5 \%)^{n-3}+\ldots+6$$ với mọi $n \in\mathbb{N}^\ast$.
		\end{enumerate}
	}
\end{vd}

\begin{vd}%[1K2K5-5]%[Trương Đăng Khoa]% Ví dụ 4
	Anh Thanh vừa được tuyển dụng vào một công ty công nghệ, được cam kết lương năm đầu sẽ là $200$ triệu đồng và lương mỗi năm tiếp theo sẽ được tăng thêm $25$ triệu đồng. Gọi $s_n$ (triệu đồng) là lương vào năm thứ $n$ mà anh Thanh làm việc cho công ty đó. Khi đó ta có
	$$s_1=200,\, s_n=s_{n-1}+25\, \text{với}\, n \ge 2.$$
	\begin{enumerate}
		\item Tính lương của anh Thanh vào năm thứ $5$ làm việc cho công ty.
		\item Chứng minh $(s_n)$ là dãy số tăng. Giải thích ý nghĩa thực tế của kết quả này. 
	\end{enumerate}
	\loigiai{
		\begin{enumerate}
			\item Ta có \begin{eqnarray*}
				s_2&=&s_1+25=200+25=225\\
				s_3&=&s_2+25=225+25=250\\
				s_4&=&s_3+25=250+25=275\\
				s_5&=&s_4+25=275+25=300. 
			\end{eqnarray*}
			Vậy lương của anh Thanh vào năm thứ $5$ làm việc cho công ty là $300$ triệu đồng.
			\item  Ta có $s_n=s_{n-1}+25\Leftrightarrow s_n-s_{n-1}=25>0$ với mọi $n\ge 2$, $n\in\mathbb{N}^\ast$.\\
			Tức là $s_n>s_{n-1}$ với mọi $n\ge 2$, $n\in\mathbb{N}^\ast$.\\
			Vậy $(s_n)$ là dãy số tăng.\\
			Điều này có nghĩa là mức lương hàng năm của anh Thanh tăng dần theo thời gian làm việc.
		\end{enumerate}
	}
\end{vd}

\begin{vd}%[1K2K5-5]%[Trương Đăng Khoa]%Ví dụ 5
	Ông An gửi tiết kiệm $100$ triệu đồng kì hạn $1$ tháng với lãi suất $6\%$ một năm theo hình thức tính lãi kép. Số tiền (triệu đồng) của ông An thu được sau $n$ tháng được cho bởi công thứC 
	$$A_n=100\left(1+\dfrac{0{,}06}{12}\right)^n.$$
	\begin{enumerate}
		\item Tìm số tiền ông An nhận được sau tháng thứ nhất, sau tháng thứ hai.
		\item Tìm số tiền ông An nhận được sau $1$ năm.
	\end{enumerate}
	\loigiai{
		\begin{enumerate}
			\item Số tiền ông An nhận được sau tháng thứ nhất là 
			$$A_1=100\left(1+\dfrac{0{,}06}{12}\right)^1=100{,}5\, \text{(triệu đồng)}.$$
			Số tiền ông An nhận được sau tháng thứ hai là 
			$$A_2=100\left(1+\dfrac{0{,}06}{12}\right)^2=101{,}0025\, \text{(triệu đồng)}.$$
			\item  Số tiền ông An nhận được sau $1$ năm ($12$ tháng) là 
			$$A_{12}=100\left(1+\dfrac{0{,}06}{12}\right)^{12} \approx 106{,}17\, \text{(triệu đồng)}.$$
		\end{enumerate}
	}
\end{vd}

\begin{vd}%[1K2G5-5]%[Trương Đăng Khoa]%Ví dụ 6
	Chị Hương vay trả góp một khoản tiền $100$ triệu đồng và đồng ý trả dần $2$ triệu đồng mỗi tháng với lãi suất $0{,}8\%$ số tiền còn lại của mỗi tháng.
	Gọi $A_n$, ($n\in\mathbb{N}$) là số tiền còn nợ (triệu đồng) của chị Hương sau $n$ tháng.
	\begin{enumerate}
		\item Tìm lần lượt $A_0$, $A_1$, $A_2$, $A_3$, $A_4$, $A_5$, $A_6$ đễ tính số tiền còn nợ của chị Hương sau $6$ tháng.
		\item  Dự đoán hệ thức truy hồi đối với dãy số $(A_n)$.
	\end{enumerate}
	\loigiai{
		\begin{enumerate}
			\item  Ta có $A_0=100$ (triệu đồng).
			\begin{itemize}
				\item Tiền lãi chị Hương phải trả sau $1$ tháng là $100\cdot 0{,}8\%=0{,}8$ (triệu đồng).\\
				Do đó, số tiền gốc chị Hương trả được sau $1$ tháng là $2-0{,}8=1{,}2$ (triệu đồng).\\
				Khi đó, số tiền còn nợ của chị Hương sau $1$ tháng là 
				$A_1=100-1{,}2=98{,}8$ (triệu đồng).
				\item  Tiền lãi chị Hương phải trả sau $2$ tháng là $98{,}8\cdot 0{,}8\%=0{,}7904$ (triệu đồng).\\
				Do đó, số tiền gốc chị Hương trả được sau $2$ tháng là $2-0{,}7904=1{,}2096$ (triệu đồng).\\
				Khi đó, số tiền còn nợ của chị Hương sau $2$ tháng là 
				$A_2=98{,}8-1{,}2096=97{,}5904$ (triệu đồng).
				\item Tiền lãi chị Hương phải trả sau $3$ tháng là $97{,}5904\cdot 0{,}8\%=0{,}7807232$ (triệu đồng).\\
				Do đó, số tiền gốc chị Hương trả được sau $3$ tháng là $2-0{,}7807232=1{,}2192768$ (triệu đồng).\\
				Khi đó, số tiền còn nợ của chị Hương sau $3$ tháng là 
				$A_3=97{,}5904-1{,}2192768=96{,}3711232$ (triệu đồng).
				\item Tiền lãi chị Hương phải trả sau $4$ tháng là $96{,}3711232\cdot 0{,}8\%\approx 0{,}77097$ (triệu đồng).\\
				Do đó, số tiền gốc chị Hương trả được sau $4$ tháng là $2-0{,}77097=1{,}22903$ (triệu đồng).\\
				Khi đó, số tiền còn nợ của chị Hương sau $4$ tháng là 
				$A_4=96{,}3711232-1{,}22903=95{,}1420932$ (triệu đồng).
				\item Tiền lãi chị Hương phải trả sau $5$ tháng là $95{,}1420932\cdot 0{,}8\%\approx 0{,}76114$ (triệu đồng).\\
				Do đó, số tiền gốc chị Hương trả được sau $5$ tháng là $2-0{,}76114=1{,}23886$ (triệu đồng).\\
				Khi đó, số tiền còn nợ của chị Hương sau $5$ tháng là $A_5=95{,}1420932-1{,}23886=93{,}9032332$ (triệu đồng).
				\item Tiền lãi chị Hương phải trả sau $6$ tháng là $93{,}9032332\cdot 0{,}8\%\approx 0{,}75123$ (triệu đồng).\\
				Do đó, số tiền gốc chị Hương trả được sau $6$ tháng là $2-0{,}75123=1{,}24877$ (triệu đồng).\\
				Khi đó, số tiền còn nợ của chị Hương sau $6$ tháng là $A_6=93{,}9032332-1{,}24877=92{,}6544632$ (triệu đồng).
			\end{itemize}
			\item Dự đoán hệ thức truy hồi đối với dãy số $(A_n)$ là 
			\[A_0=100,\, A_n=A_{n-1}-\left(2-A_{n-1} \cdot 0{,}8 \%\right)=1{,}008 A_{n-1}-2\]
		\end{enumerate}
	}
\end{vd}
