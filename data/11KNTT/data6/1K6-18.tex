\section{Lũy thừa với số mũ thực}
\subsection{Tóm tắt lí thuyết}
\begin{tomtat}
\subsubsection{Lũy thừa với số mũ nguyên}
\begin{dn}
	Cho $n$ là một số nguyên dương. Ta định nghĩa
	\begin{itemize}
		\item Với $a$ là số thực tùy ý $$a^n=\underbrace{a\cdot a \cdots a}_{n \text { thừa số } a}.$$
		\item Với $a$ là số thực khác $0$ $$a^0=1; a^{-n}=\dfrac{1}{a^n}.$$
	\end{itemize}
	Trong biểu thức $a^n$, $a$ gọi là \textbf{cơ số}, $n$ gọi là \textbf{số mũ}.
\end{dn}
Lũy thừa với số mũ nguyên có tính chất tương tự như lũy thừa với số mũ nguyên dương.
\begin{tc}
	Với $a\ne 0$, $b\ne 0$ và $m$, $n$ là các số nguyên, ta có
	\begin{multicols}{2}
		\begin{itemize}
			\item $a^m\cdot a^n=a^{m+n}$;
			\item $\dfrac{a^m}{a^n}=a^{m-n}$;
			\item $(a^m)^n=a^{mn}$;
			\item $(ab)^n=a^nb^n$;
			\item $\left(\dfrac{a}{b}\right)^n=\dfrac{a^n}{b^n}$.
		\end{itemize}
	\end{multicols}
\end{tc}
\begin{note}
	\begin{itemize}
		\item Nếu $a>1$ thì $a^m>a^n$ khi và chỉ khi $m>n$.
		\item Nếu $0<a<1$ thì $a^m>a^n$ khi và chỉ khi $m<n$.
	\end{itemize}
\end{note}
\subsubsection{Lũy thừa với số mũ hữu tỉ}
\begin{dn}
	Cho số thực $a$ và số nguyên dương $n$. Số $b$ được gọi là \textbf{căn bậc $n$} của số $a$ nếu $b^n=a$.
\end{dn}
\begin{nx}
	Khi $n$ là số lẻ, mỗi số thực $a$ chỉ có một căn bậc $n$ và kí hiệu là $\sqrt[n]{a}$. Căn bậc $1$ của số $a$ chính là $a$.\\
	Khi $n$ là số chẵn, mỗi số thực dương có đúng hai căn bậc $n$ là hai số đối nhau, giá trị dương kí hiệu là $\sqrt[n]{a}$ (gọi là \textit{căn số học bậc $n$} của $a$), giá trị âm kí hiệu là $-\sqrt[n]{a}$. 
\end{nx}
\begin{tc}
	Giả sử $n$, $k$ là các số nguyên dương, $m$ là số nguyên. Khi đó
	\begin{multicols}{2}
		\begin{itemize}
			\item $\sqrt[n]{a}\cdot\sqrt[n]{b}=\sqrt[n]{ab}$;
			\item $\dfrac{\sqrt[n]{a}}{\sqrt[n]{b}}=\sqrt[n]{\dfrac{a}{b}}$;
			\item $\left(\sqrt[n]{a}\right)^m=\sqrt[n]{a^m}$;
			\item $\sqrt[n]{a^n}=\heva{&a&\text{khi } n \text{ lẻ}\\&|a|&\text{khi } n \text{ chẵn}}$;
			\item $\sqrt[n]{\sqrt[k]{a}}=\sqrt[nk]{a}$.
		\end{itemize}
	\end{multicols}
	Giả thiết các biểu thức ở trên đều có nghĩa.
\end{tc}
\begin{dn}
	Cho số thực $a$ dương và số hữu tỉ $r=\dfrac{m}{n}$, trong đó $m$ là một số nguyên và $n$ là số nguyên dương. Lũy thừa của $a$ với số mũ $r$, kí hiệu $a^r$, xác định bởi $$a^r=a^{\tfrac{m}{n}}=\sqrt[n]{a^m}.$$
\end{dn}
\begin{note}
	Lũy thừa với số mũ hữu tỉ (của một số thực dương) có đầy đủ các tính chất như lũy thừa với số mũ nguyên.
\end{note}
\subsubsection{Lũy thừa với số mũ thực}
\begin{dn}
	Cho $a$ là số thực dương và $\alpha$ là một số vô tỉ. Xét dãy số hữu tỉ $(r_n)$ mà $\lim\limits_{n\to +\infty}r_n=\alpha$. Khi đó, dãy số $\left(a^{r_n}\right)$ có giới hạn xác định và không phụ thuộc vào dãy số hữu tỉ $(r_n)$ đã chọn. Giới hạn đó gọi là \textbf{lũy thừa của $a$ với số mũ $\alpha$}, kí hiệu là $a^{\alpha}$.
	$$a^{\alpha}=\lim\limits_{n\to +\infty}a^{r_n}.$$
\end{dn}
\begin{note}
	Lũy thừa với số mũ thực (của một số dương) có đầy đủ tính chất như lũy thừa với số mũ nguyên.
\end{note}
\subsubsection{Công thức lãi kép}
\begin{dn}
	Lãi kép là phần lãi của kì sau được tính trên số tiền gốc kì trước cộng với phần lãi của kì trước.
\end{dn}
Công thức: Giả sử số tiền gốc là $A$; lãi suất $r\%$ /kì hạn gửi (có thể là tháng, quý hay năm).
\begin{itemize}
	\item Số tiền nhận được cả gốc và lãi sau $n$ kì hạn gửi là $A\left(1+r\right)^n$.
	\item Số tiền lãi nhận được sau $n$ kì hạn gửi là $A\left(1+r\right)^n-A=A\left[\left(1+r\right)^n-1\right]$
\end{itemize} 
\end{tomtat}
\subsection{Các dạng toán thường gặp}

\begin{dang}{Tính giá trị biểu thức chứa  lũy thừa}
	\textit{Cho $a, b$ là các số thực dương, $x, y$ là các số thực tùy ý, ta có}
	\begin{itemize}
		\item ${a}^{x+y}=a^x.a^y$ và ${a}^{x-y}=\dfrac{a^x}{a^y}$.
		\item $a^x.b^x=(a.b)^x; \dfrac{a^x}{b^x}={\left(\dfrac{a}{b}\right)}^x$ và $(a^x)^y={a}^{x.y}$.
		\item Nếu $a>1$ thì $a^x>a^y\Leftrightarrow x>y$
		\item Nếu $0<a<1$ thì $a^x>a^y\Leftrightarrow x<y$
	\end{itemize}
\end{dang}

\subsubsection{Ví dụ mẫu}

\begin{vd}%[1K6YH-1]%Vidu1
	Tính
	\begin{multicols}{4}
	\begin{enumerate}
	\item $\left(\dfrac{1}{5}\right)^{-2}$;
	\item $4^{\tfrac{3}{2}}$;
	\item $\left(\dfrac{1}{8}\right)^{-\tfrac{2}{3}}$;
	\item $\left(\dfrac{1}{16}\right)^{-0{,}75}$.
	\end{enumerate}
	\end{multicols}
	\loigiai{
	\begin{enumerate}
		\item Ta có $\left(\dfrac{1}{5}\right)^{-2}=5^2=25$.
		\item Ta có $4^{\tfrac{3}{2}}=\sqrt{4^3}=\sqrt{\left( 2^2\right) ^3}=\sqrt{\left( 2^3\right) ^2}=2^3=8$.
		\item Ta có $\left(\dfrac{1}{8}\right)^{-\tfrac{2}{3}}=\left(\dfrac{1}{2^3}\right)^{-\tfrac{2}{3}}=\left( 2^3\right) ^{\tfrac{2}{3}}=2^{3\cdot \tfrac{2}{3}}=2^2=4$.
		\item Ta có $\left(\dfrac{1}{16}\right)^{-0{,}75}=16^{0{,}75}=16^{\tfrac{3}{4}}=\left( 2^4\right) ^{\tfrac{3}{4}}=2^{4\cdot \tfrac{3}{4}}=2^3=8$.
	\end{enumerate}
	}
\end{vd}

\begin{vd}%[1K6YH-1]%Vidu2
	Tính giá trị của biểu thức $A=\left(\dfrac{1}{2}\right)^{-8}\cdot 8^{-2}+(0{,}2)^{-4}\cdot 25^{-2}$.
	\loigiai{ Ta có 
		\begin{eqnarray*}
		\left(\dfrac{1}{2}\right)^{-8}\cdot 8^{-2}+(0{,}2)^{-4}\cdot 25^{-2}
		&=&2^8\cdot \dfrac{1}{8^2}+\dfrac{1}{0{,}2^4}\cdot\dfrac{1}{25^2}\\
		&=&2^8\cdot\dfrac{1}{2^6}+\dfrac{1}{0{,}2^4\cdot 5^4}\\
		&=&2^2+\dfrac{1}{\left(0{,}2\cdot 5\right)^4}\\	
		&=&4+1=5.
		\end{eqnarray*}
		Vậy $A=5$.}
\end{vd}

\begin{vd}%[1K6YH-1]%Vidu3
	Tính
	\begin{multicols}{2}
	\begin{enumerate}
	\item $\sqrt[3]{-64}$;
	\item $\sqrt[4]{\dfrac{1}{16}}$.
	\end{enumerate}
	\end{multicols}
	\loigiai{
		\begin{enumerate}
		\item Ta có $\sqrt[3]{-64}=\sqrt[3]{(-4)^3}=-4$.
		\item Ta có $\sqrt[4]{\dfrac{1}{16}}=\sqrt[4]{\left(\dfrac{1}{2}\right)^4}=\dfrac{1}{2}$.
		\end{enumerate}
	}
\end{vd}

\begin{vd}%[1K6YH-1]%Vidu4
	Tính
	\begin{multicols}{2}
	\begin{enumerate}
	\item $\sqrt[5]{4}\cdot\sqrt[5]{-8}$;
	\item $\sqrt[3]{-3\sqrt{3}}$.
	\end{enumerate}
	\end{multicols}
	\loigiai{
		\begin{enumerate}
		\item Ta có $\sqrt[5]{4}\cdot\sqrt[5]{-8}=\sqrt[5]{4\cdot (-8)}=\sqrt[5]{-32}=\sqrt[5]{(-2)^5}=-2$.
		\item Ta có $\sqrt[3]{-3\sqrt{3}}=\sqrt[3]{-\left(\sqrt{3}\right)^3}=\sqrt[3]{\left(-\sqrt{3}\right)^3}=-\sqrt{3}$.
		\end{enumerate}
	}
\end{vd}

\begin{vd}%[1K6BH-1]%Vidu5
	Tính giá trị của biểu thức $A=\dfrac{{15}^{3+\sqrt{2}}}{3^{1+\sqrt{2}}\cdot 5^{2+\sqrt{2}}}$?
	\loigiai{ Ta có 
		$$\dfrac{{15}^{3+\sqrt{2}}}{3^{1+\sqrt{2}}\cdot 5^{2+\sqrt{2}}}=\dfrac{3^{3+\sqrt{2}}\cdot5^{3+\sqrt{2}}}{3^{1+\sqrt{2}}\cdot 5^{2+\sqrt{2}}}=3^2\cdot5^1=45.$$
		Vậy $A=45$.}
\end{vd}

\subsubsection{Bài tập rèn luyện}
\centerline{\fcolorbox{red}{yellow!50}{\bf {BÀI TẬP TỰ LUẬN}}}
\begin{bt}%[1K6YH-1]%Bai1
	Tính
	\begin{multicols}{2}
	\begin{enumerate}
	\item $16^{\tfrac{3}{2}}$;
	\item $8^{\tfrac{-2}{3}}$.
	\end{enumerate}
	\end{multicols}
	\loigiai{
		\begin{enumerate}
		\item Ta có     $16^{\tfrac{3}{2}}=\sqrt{16^3}=\sqrt{(4^2)^3}=\sqrt{(4^3)^2}=4^3=64$.
		\item Ta có   $8^{-\tfrac{2}{3}}=\sqrt[3]{8^{-2}}=\sqrt[3]{\left(2^3\right)^{-2}}=\sqrt[3]{\left(2^{-2}\right)^3}=2^{-2}=\dfrac{1}{4}$.
		\end{enumerate}
	}
\end{bt}

\begin{bt}%[1K6YH-1]%Bai2
	Tính giá trị các biểu thức sau:
	\begin{multicols}{2}
	\begin{enumerate}
	\item $A=\left(\dfrac{1}{625}\right)^{-\tfrac{1}{4}}+{16}^{\tfrac{3}{4}}-2^{-2}\cdot{64}^{\tfrac{1}{3}}$;
	\item $B=\left(\dfrac{1}{16}\right)^{-0,75}+\left(\dfrac{1}{8}\right)^{-\tfrac{4}{3}}$.
	\end{enumerate}
	\end{multicols}
	\loigiai{
		\begin{enumerate}
		\item Ta có  $\left(5^{-4}\right)^{-\tfrac{1}{4}}+\left(2^4\right)^{\tfrac{3}{4}}-2^{-2}\cdot\left(2^6\right)^{\tfrac{1}{3}}=5+8-1=12$.\\
		Vậy $A=12$.
		\item Ta có  $\left(2^{-4}\right)^{-0,75}+\left(2^{-3}\right)^{-\tfrac{4}{3}}=2^3+2^4=8+16=24$.\\
		Vậy $B=24$.
		\end{enumerate}
		}
\end{bt}

\begin{bt}%[1K6BH-1]%Bai3
	Tính giá trị biểu thức $A=\sqrt{\sqrt{5}\cdot\left( \sqrt[4]{\sqrt{5}}:\sqrt{\sqrt[5]{5}} \right)^{10}}$
	\loigiai{ Ta  có 
		$$\sqrt{\sqrt{5}\cdot\left( \sqrt[4]{\sqrt{5}}:\sqrt{\sqrt[5]{5}} \right)^{10}}=\sqrt{{5^{\tfrac{1}{2}}}{{\left( {5^{\tfrac{1}{8}}}:{5^{\tfrac{1}{10}}} \right)}^{10}}}=\sqrt{{5^{\tfrac{1}{2}}}{{\left( {5^{\tfrac{1}{40}}} \right)}^{10}}}=\sqrt{5^{\tfrac{1}{2}}\cdot 5^{\tfrac{1}{4}}}=\sqrt{5^{\tfrac{3}{4}}}=5^{\tfrac{3}{8}}.$$
		Vậy $A=5^{\tfrac{3}{8}}.$
		}
\end{bt}

\begin{bt}%[1K6BH-1]%Bai4
	Tính giá trị của biểu thức $A=\dfrac{{12}^{5+\sqrt{3}}}{2^{5+2\sqrt{3}}\cdot 3^{7+\sqrt{3}}}$?
	\loigiai{ Ta có 
		 $$\dfrac{{12}^{5+\sqrt{3}}}{2^{5+2\sqrt{3}}\cdot 3^{7+\sqrt{3}}}=\dfrac{4^{5+\sqrt{3}}\cdot3^{5+\sqrt{3}}}{2^{5+2\sqrt{3}}\cdot 3^{7+\sqrt{3}}}=\dfrac{2^{10+2\sqrt{3}}\cdot3^{5+\sqrt{3}}}{2^{5+2\sqrt{3}}\cdot3^{7+\sqrt{3}}}=\dfrac{2^5}{3^2}=\dfrac{32}{9}.$$
	 	 Vậy $A=\dfrac{32}{9}$.}
\end{bt}

\begin{bt}%[1K6KH-1]%Bai5
	Tính giá trị của biểu thức $A={\left( 5-2\sqrt{6} \right)^{2020}}{\left( 5+2\sqrt{6} \right)^{2021}}$. 
	\loigiai{ Ta có  
		\begin{eqnarray*}
		\left( 5-2\sqrt{6} \right)^{2020}\left( 5+2\sqrt{6} \right)^{2021}
		&=&\left( 5-2\sqrt{6} \right)^{2020}\left( 5+2\sqrt{6} \right)^{2020}\left( 5+2\sqrt{6} \right)\\
		&=&\left(5^2-\left( 2\sqrt{6} \right)^2 \right)^{2020}\left( 5+2\sqrt{6} \right)\\
		&=&5+2\sqrt{6}.
		\end{eqnarray*}
		Vậy $A=5+2\sqrt{6}$.}
\end{bt}

\begin{bt}%[1K6KH-1]%Bai6
	Chứng minh rằng $\sqrt{4+2\sqrt{3}}-\sqrt{4-2\sqrt{3}}=2$.
	\loigiai{ Ta có 
		\allowdisplaybreaks
		\begin{eqnarray*}
		\sqrt{4+2\sqrt{3}}-\sqrt{4-2\sqrt{3}}&=&\sqrt{3+2\sqrt{3}+1}-\sqrt{3-2\sqrt{3}+1}\\
		&=&\sqrt{\left(\sqrt{3}+1\right)^2}-\sqrt{\left(\sqrt{3}-1\right)^2}\\
		&=&\sqrt{3}+1-\left(\sqrt{3}-1\right)\\
		&=&2.
		\end{eqnarray*}
	}
\end{bt}

\begin{bt}%[1K6KH-1]%Bai7
	Biết $A={\left( 4-2\sqrt{3} \right)^{1010}}{\left( 1+\sqrt{3} \right)^{2020}}$ được viết lại dưới dạng $2^{\alpha }$ với $\alpha \in \mathbb{R}$. Tìm $\alpha$?
	\loigiai{
		Ta có 
		\allowdisplaybreaks
		\begin{eqnarray*}
		\left( 4-2\sqrt{3} \right)^{1010}\left( 1+\sqrt{3} \right)^{2020}
		&=&\left( 4-2\sqrt{3} \right)^{1010}\left({\left( 1+\sqrt{3} \right)^2} \right)^{1010}\\
		&=&\left( 4-2\sqrt{3} \right)^{1010}\left( 4+2\sqrt{3} \right)^{1010}\\
		&=&\left[ \left( 4-2\sqrt{3} \right)\left( 4+2\sqrt{3} \right) \right]^{1010}\\
		&=&\left( 4^2-\left( 2\sqrt{3} \right)^2 \right)^{1010}\\
		&=&4^{1010}\\
		&=&2^{2020}.
		\end{eqnarray*}
		Vậy $A=2^{2020}.$}
\end{bt}

\begin{bt}%[1K6GH-1]%Bai8
	Tính giá trị của biểu thức $A=\left( a+1 \right)^{-1}+\left( b+1 \right)^{-1}$ với  $a=\left( 2+\sqrt{3} \right)^{-1}$, $b=\left( 2-\sqrt{3} \right)^{-1}$.
	\choice
	{ $3+\sqrt{3}$}
	{\True $1$}
	{ $3-\sqrt{3}$}
	{ $2$}
	\loigiai{
		Ta có 
		$$a=\left( 2+\sqrt{3} \right)^{-1}=\dfrac{1}{2+\sqrt{3}}=\dfrac{2-\sqrt{3}}{\left( 2+\sqrt{3} \right)\left( 2-\sqrt{3} \right)}=\dfrac{2-\sqrt{3}}{4-3}=2-\sqrt{3}.$$
		$$b=\left( 2-\sqrt{3} \right)^{-1}=\dfrac{1}{2-\sqrt{3}}=\dfrac{2+\sqrt{3}}{\left( 2-\sqrt{3} \right)\left( 2+\sqrt{3} \right)}=\dfrac{2+\sqrt{3}}{4-3}=2+\sqrt{3}.$$
		Suy ra
		\begin{eqnarray*}
		A
		&=&\left( 2-\sqrt{3}+1 \right)^{-1}+\left( 2+\sqrt{3}+1 \right)^{-1}\\
		&=&\left( 3-\sqrt{3} \right)^{-1}+\left( 3+\sqrt{3} \right)^{-1}\\
		&=&\dfrac{1}{3-\sqrt{3}}+\dfrac{1}{3+\sqrt{3}}\\
		&=&\dfrac{3+\sqrt{3}}{\left( 3-\sqrt{3} \right)\left( 3+\sqrt{3} \right)}+\dfrac{3-\sqrt{3}}{\left( 3+\sqrt{3} \right)\left( 3-\sqrt{3} \right)}\\
		&=&\dfrac{3+\sqrt{3}}{6}+\dfrac{3-\sqrt{3}}{6}\\
		&=&1.
		\end{eqnarray*}
		Vậy $A=1.$}
\end{bt}

\centerline{\fcolorbox{red}{yellow!50}{\bf {CÂU HỎI TRẮC NGHIỆM }}}
\Opensolutionfile{ans}[ans/ans-1K6-1-Dang1]
\begin{ex}%[1K6YH-1]%Cau1
	Tính giá trị của biểu thức $A=2^3\cdot 2^7.$
	\choice
	{$2^{21}$}
	{\True $2^{10}$}
	{$2^{20}$}
	{$2^{18}$}
	\loigiai{
		Ta có $A=2^3\cdot 2^7=2^{3+7}=2^{10}.$
		}
\end{ex}

\begin{ex}%[1K6YH-1]%Cau2
	Chọn khẳng định đúng.
	\choice
	{$\left( 3^2\right)^5=3^7$}
	{\True $\left( 3^2\right)^5=3^{10}$}
	{$\left( 3^2\right)^5=3^{-10}$}
	{$\left( 3^2\right)^5=3^{-7}$}
	\loigiai{
		Ta có $\left( 3^2\right)^5=3^{2\cdot 5}=3^{10}.$
	}
\end{ex}

\begin{ex}%[1K6YH-1]%Cau3
	Tính giá trị của biểu thức $A=27^{\tfrac{2}{3}}+81^{-0{,}75}-25^{0{,}5}$.
	\choice
	{\True $\dfrac{109}{27}$}
	{$\dfrac{112}{27}$}
	{$\dfrac{113}{27}$}
	{$\dfrac{107}{27}$}
	\loigiai{
		Ta có 
		\allowdisplaybreaks
		\begin{eqnarray*}
			27^{\tfrac{2}{3}}+81^{-0{,}75}-25^{0{,}5}&=&\left( 3^3\right) ^{\tfrac{2}{3}}+\left( 3^4\right) ^{-0{,}75}-\left( 5^2\right) ^{0{,}5}\\
			&=&3^{3\cdot\tfrac{2}{3}}+3^{4\cdot (-0{,}75)}-5^{2\cdot (0{,}5)}\\
			&=&3^2+3^{-3}-5^1\\
			&=&\dfrac{109}{27}.
		\end{eqnarray*}
	Vậy $A=\dfrac{109}{27}$.
	}
\end{ex}

\begin{ex}%[1K6YH-1]%Cau4
	Giá trị của biểu thức $A=\left( \dfrac{1}{3} \right)^{-10}\cdot 27^{-3}+\left( \dfrac{1}{5} \right)^{-4}\cdot 25^{-2}+128^{-1}\cdot \left( \dfrac{1}{2} \right)^{-9}$ bằng
	\choice
	{$9$}
	{\True $8$}
	{$4$}
	{$1$}
	\loigiai{
		Ta có 
		\allowdisplaybreaks
		\begin{eqnarray*}
			&&\left( \dfrac{1}{3} \right)^{-10}\cdot 27^{-3}+\left( \dfrac{1}{5} \right)^{-4}\cdot 25^{-2}+128^{-1}\cdot \left( \dfrac{1}{2} \right)^{-9}\\
			&=&\left( {3^{-1}} \right)^{-10}\cdot \left( {3^3} \right)^{-3}+\left( {5^{-1}} \right)^{-4}\cdot \left( {5^2} \right)^{-2}+\left( {2^7} \right)^{-1}\cdot \left(2^{-1} \right)^{-9}\\
			&=& 3^{10}\cdot 3^{-9}+5^4\cdot 5^{-4}+2^{-7}\cdot2^9=3+1+2^2\\
			&=&8.
		\end{eqnarray*}
		Vậy $A=8$.
		}
\end{ex}

\begin{ex}%[1K6BH-1]%Cau5
	Tính giá trị của biểu thức $A=\sqrt{2\sqrt{2\sqrt{2\sqrt{2}}}}$.
	\choice
	{${2^{\dfrac{7}{16}}}$}
	{${2^{\dfrac{15}{8}}}$}
	{\True ${2^{\dfrac{15}{16}}}$}
	{${2^{\dfrac{7}{8}}}$}
	\loigiai{
		Ta có\\
		$$\sqrt{2\sqrt{2\sqrt{2\sqrt{2}}}}=\sqrt{2\sqrt{2\sqrt{{2\cdot 2^{\tfrac{1}{2}}}}}}=\sqrt{2\sqrt{2\sqrt{2^{\tfrac{3}{2}}}}}=\sqrt{2\sqrt{{2\cdot 2^{\tfrac{3}{4}}}}}=\sqrt{2\sqrt{2^{\tfrac{7}{4}}}}=\sqrt{{2\cdot2^{\tfrac{7}{8}}}}=\sqrt{2^{\tfrac{15}{8}}}=2^{\tfrac{15}{16}}.$$
		Vậy $A=2^{\tfrac{15}{16}}$.
	}
\end{ex}

\begin{ex}%[1K6BH-1]%Cau6
	Viết biểu thức $\dfrac{\sqrt{2\sqrt[3]{4}}}{16^{0{,}75}}$ về dạng lũy thừa $2^m$. Tìm $m$?
	\choice
	{$m=\dfrac{13}{6}$}
	{$m=-\dfrac{11}{6}$}
	{\True $m=-\dfrac{13}{6}$}
	{$m=-\dfrac{17}{6}$}
	\loigiai{ Ta có
		$$\dfrac{\sqrt{2\sqrt[3]{4}}}{16^{0{,}75}}=\dfrac{\sqrt{2^{\tfrac{5}{3}}}}{2^3}=2^{-\tfrac{13}{6}}\Rightarrow m=-\dfrac{13}{6}.$$}
\end{ex}

\begin{ex}%[1K6BH-1]%Cau7
	Viết biểu thức $\sqrt{\dfrac{2\sqrt{2}}{\sqrt[4]{8}}}$ về dạng $2^x$ và biểu thức $\dfrac{2\sqrt{8}}{\sqrt[3]{4}}$ về dạng $2^y$. Tính $x^2+y^2$.
	\choice
	{$\dfrac{2023}{576}$}
	{$\dfrac{2021}{576}$}
	{$\dfrac{2015}{576}$}
	{\True $\dfrac{2017}{576}$}
	\loigiai{ Ta có
		$$\sqrt{\dfrac{2\sqrt{2}}{\sqrt[4]{8}}}=\sqrt{\dfrac{2^{\tfrac{3}{2}}}{2^{\tfrac{3}{4}}}}=\sqrt{2^{\tfrac{3}{4}}}=2^{\tfrac{3}{8}}\Rightarrow x=\dfrac{3}{8}.$$ $$\dfrac{2\sqrt{8}}{\sqrt[3]{4}}=\dfrac{2\cdot2^{\tfrac{3}{2}}}{2^{\tfrac{2}{3}}}=2^{\tfrac{11}{6}}\Rightarrow y=\dfrac{11}{6}.$$\\
		Vậy $x^2+y^2=\dfrac{2017}{576}.$}
\end{ex}

\begin{ex}%[1K6KH-1]%Cau8
	Tính giá trị của biểu thức $A=\left( 7+4\sqrt{3} \right)^{2020}\left( 7-4\sqrt{3} \right)^{2019}$.
	\choice
	{$4-7\sqrt{3}$}
	{$4+7\sqrt{3}$}
	{$7-4\sqrt{3}$}
	{\True $7+4\sqrt{3}$}
	\loigiai{
		Ta có 
		\allowdisplaybreaks
		\begin{eqnarray*}
		\left( 7+4\sqrt{3} \right)^{2020}\left( 7-4\sqrt{3} \right)^{2019}
		&=&\left( 7+4\sqrt{3} \right)^{2019}\left( 7-4\sqrt{3} \right)^{2019}\left( 7+4\sqrt{3} \right)\\
		&=&\left[ \left( 7+4\sqrt{3} \right)\left( 7-4\sqrt{3} \right) \right]^{2019}\left( 7+4\sqrt{3} \right)\\  
		&=&\left( 49-48 \right)^{2019}\left( 7+4\sqrt{3} \right)\\
		&=&7+4\sqrt{3}.
		\end{eqnarray*}
		Vậy $A=7+4\sqrt{3}$.
		}
\end{ex}

\begin{ex}%[1K6KH-1]%Cau9
	Giá trị của biểu thức $A=\left( 4+\sqrt{15} \right)^{2020}\left( \sqrt{15}-4 \right)^{2022}$ bằng $a+b\sqrt{15}$. Tính $a+3b$.
	\choice
	{\True $7$}
	{$24$}
	{$39$}
	{$-7$}
	\loigiai{ Ta có
		\begin{eqnarray*}
		A
		&=&\left( 4+\sqrt{15} \right)^{2020} \left( \sqrt{15}-4 \right)^{2022}\\
		&=&\left( 4+\sqrt{15} \right)^{2020} \left( \sqrt{15}-4 \right)^{2020} \left( \sqrt{15}-4 \right)^2\\
		&=&\left[ \left( \sqrt{15}+4 \right)\left( \sqrt{15}-4 \right) \right]^{2020}\left( \sqrt{15}-4 \right)^2\\  
		&=&31-8\sqrt{15}.
		\end{eqnarray*}
		$\Rightarrow a=31,b=-8$.\\
		Vậy $a+3b=7$.}
\end{ex}

\begin{ex}%[1K6GH-1]%Cau10
	Tính giá trị của biểu thức $A=\left( a+1 \right)^{-1}+\left( b+1 \right)^{-1}$ với $a=\left( 1+\sqrt{3} \right)^{-1}$ và $b=\left( 1-\sqrt{3} \right)^{-1}$.
	\choice
	{$2$}
	{\True $-2$}
	{$3$}
	{$-3$}
	\loigiai{ Ta có
		\begin{eqnarray*}
			\left( a+1 \right)^{-1}+\left( b+1 \right)^{-1}
			&=&\left( \left( 1+\sqrt{3} \right)^{-1}+1 \right)^{-1}+\left( \left( 1-\sqrt{3} \right)^{-1}+1 \right)^{-1}\\
			&=&\left[ \dfrac{1}{1+\sqrt{3}}+1 \right]^{-1}+\left[ \dfrac{1}{1-\sqrt{3}}+1 \right]^{-1}\\  
			&=&\left[ \dfrac{2+\sqrt{3}}{1+\sqrt{3}} \right]^{-1}+\left[ \dfrac{2-\sqrt{3}}{1-\sqrt{3}} \right]^{-1}\\
			&=&\dfrac{1+\sqrt{3}}{2+\sqrt{3}}+\dfrac{1-\sqrt{3}}{2-\sqrt{3}}\\
			&=&-2.
		\end{eqnarray*}
		Vậy $A=-2$.
	}
\end{ex}
\Closesolutionfile{ans}
\begin{indapan}{10}
	{ans/ans-1K6-1-Dang1}
\end{indapan}

\begin{dang}{Biến đổi, rút gọn biểu thức chứa lũy thừa}
	Áp dụng các công thức, tính chất ở dạng 1. 
\end{dang}
\subsubsection{Ví dụ mẫu}
\begin{vd}%[DCHT Toán 11 - KNTT -Nguyễn Thắng]%[1K6BH-2]
	Rút gọn biểu thức $P=\sqrt{x\cdot \sqrt[3]{x^2\cdot \sqrt{x^3}}}$, với $x>0$. \dapso{$P=x^{^{\tfrac{13}{12}}}$.}
	\loigiai{
		Ta có $P=\sqrt{x\cdot \sqrt[3]{x^2\cdot \sqrt{x^3}}}=\sqrt{x\cdot \sqrt[3]{x^2\cdot{x^{^{\tfrac{3}{2}}}}}}=\sqrt{x\cdot \sqrt[3]{x^{^{\tfrac{7}{2}}}}}=\sqrt{x\cdot{x^{^{\tfrac{7}{6}}}}}=\sqrt{x^{^{\tfrac{13}{6}}}}=x^{^{\tfrac{13}{12}}}$.}
\end{vd}
\subsubsection{Bài tập rèn luyện}
\centerline{\fcolorbox{red}{yellow!50}{\bf {BÀI TẬP TỰ LUẬN}}}
\begin{bt}%[DCHT Toán 11 - KNTT -Nguyễn Thắng]%[1K6BH-2]
	Rút gọn biểu thức $B=\sqrt{x}\cdot \sqrt[3]{x^2\cdot \sqrt{x}}$ với $x>0$.\dapso{$B==x^{^{\tfrac{4}{3}}}$.}
	\loigiai{
		Ta có $\sqrt{x}\cdot \sqrt[3]{x^2\cdot \sqrt{x}}=x^{^{\tfrac{1}{2}}}\cdot \sqrt[3]{x^2\cdot{x^{^{\tfrac{1}{2}}}}}=x^{^{\tfrac{1}{2}}}\cdot \sqrt[3]{x^{^{\tfrac{5}{2}}}}=x^{^{\tfrac{1}{2}}}\cdot{x^{^{\tfrac{5}{6}}}}=x^{^{\tfrac{1}{2}+\tfrac{5}{6}}}=x^{^{\tfrac{4}{3}}}$.}
\end{bt}

\begin{bt}%[DCHT Toán 11 - KNTT -Nguyễn Thắng]%[1K6KH-2]
	Cho hàm số $f(x)=\dfrac{2025^x}{45+2025^x}$, $x\in \mathbb{R}$. Hãy tính $M=f(a)+f(b-2)$, biết rằng $a+b=3$.\dapso{$M=1$.}
	\loigiai
	{Vì $a+b=3$ nên $b=3-a$. Khi đó
		\allowdisplaybreaks
		\begin{eqnarray*}
			f(a)+f(b-2) &=& f(a)+f(1-a)= \dfrac{2025^a}{45+2025^a}+\dfrac{2025^{1-a}}{45+2025^{1-a}}\\
			&=& \dfrac{2025^a}{45+2025^a}+\dfrac{2025}{45\cdot 2025^a+2025}\\
			&=& \dfrac{2025^a\left(45\cdot 2025^a+2025\right) + 2025\left(45+2025^a\right)}{\left(45+2025^a\right)\left(45\cdot 2025^a+2025\right)}\\
			&=& \dfrac{45\cdot 2025^{2a}+2\cdot 2025^{a+1}+45\cdot 2025}{2025^{a+1}+45\cdot 2025+45\cdot 2025^{2a}+2025^{a+1}}\\
			&=& 1.
		\end{eqnarray*}
	}
\end{bt}
\centerline{\fcolorbox{red}{yellow!50}{\bf {CÂU HỎI TRẮC NGHIỆM }}}
\Opensolutionfile{ans}[ans/ans-1K6-1-Dang2]

\begin{ex}%[DCHT Toán 11 - KNTT -Nguyễn Thắng]%[1K6BH-2]
	Cho biểu thức $P=\sqrt{x\cdot \sqrt[3]{x^2}\cdot \sqrt[k]{x^3}}$, với $x>0$. Biết rằng $P=x^{^{\tfrac{23}{24}}}$, giá trị của $k$ bằng
	\choice
	{$k=6$}
	{$k=2$}
	{$k=3$}
	{\True $k=4$}
	\loigiai{
		Ta có $P=\sqrt{x\cdot \sqrt[3]{x^2}\cdot \sqrt[k]{x^3}}=x^{^{\tfrac{23}{24}}}$ suy ra\\$x\cdot \sqrt[3]{x^2}\cdot \sqrt[k]{x^3}=x^{^{\tfrac{23}{12}}}\Leftrightarrow \sqrt[3]{x^2}\cdot \sqrt[k]{x^3}=x^{^{\tfrac{11}{12}}}{x^2}\cdot \sqrt[k]{x^3}=x^{^{\tfrac{11}{4}}}\Leftrightarrow \sqrt[k]{x^3}=x^{^{\tfrac{11}{4}-2}}\Leftrightarrow{x^{^{\tfrac{3}{k}}}}=x^{^{\tfrac{3}{4}}}\Leftrightarrow k=4$.}
\end{ex}

\begin{ex}%[DCHT Toán 11 - KNTT -Nguyễn Thắng]%[1K6BH-2]
	Cho $0<a<1$, $b>1$. Rút gọn biểu thức sau $\sqrt{{\left(a^\pi +b^\pi\right)}^2-\left(4^{^{\tfrac{1}{\pi}}}ab\right)^\pi}$.
	\choice
	{$2\left({a^\pi -b^\pi}\right)$}
	{\True $b^\pi -a^\pi $}
	{$a^\pi +b^\pi $}
	{$a^\pi -b^\pi $}
	\loigiai{
		Ta có
		\begin{eqnarray*}
			\sqrt{{\left(a^\pi +b^\pi\right)}^2-{\left(4^{^{\tfrac{1}{\pi}}}ab\right)}^\pi} &=& \sqrt{a^{2\pi}+b^{2\pi}+2a^\pi b^\pi -4a^\pi b^\pi}\\& =&\sqrt{\left(a^\pi -b^\pi\right)}^2\\
			& =&\left|a^\pi -b^\pi\right| \\
			& =& b^\pi -a^\pi	
		\end{eqnarray*}
		(vì $0<a<1$, $b>1$).
	}
\end{ex}

\begin{ex}%[DCHT Toán 11 - KNTT -Nguyễn Thắng]%[1K6KH-2]
	Cho biểu thức $P=\left\{a^{^{\tfrac{1}{3}}}\left[a^{^{-\tfrac{1}{2}}} b^{^{-\tfrac{1}{3}}}\left(a^2b^2\right)^{^{\tfrac{2}{3}}}\right]^{^{-\tfrac{1}{2}}}\right\}^6$ với $a$, $b$ là các số dương. Khẳng định nào sau đây là đúng?
	\choice
	{$P=\dfrac{b^3\sqrt {a}}{a}$}
	{$P=\dfrac{\sqrt {a}}{b^3}$}
	{$P=b^3\sqrt {3}$}
	{\True $P=\dfrac{\sqrt {a}}{ab^3}$ }
	\loigiai{
		\begin{eqnarray*}
			P&=&\left\{a^{^{\tfrac{1}{3}}}\left[a^{^{-\tfrac{1}{2}}} b^{^{-\tfrac{1}{3}}}\left(a^2b^2\right)^{^{\tfrac{2}{3}}}\right]^{^{-\tfrac{1}{2}}}\right\}^6\\
			&=& \left\{a^{^{\tfrac{1}{3}}}\left[a^{^{-\tfrac{1}{2}}} b^{^{-\tfrac{1}{3}}}\left(a^{^{\tfrac{4}{3}}}b^{^{\tfrac{4}{3}}}\right)\right]^{^{-\tfrac{1}{2}}} \right\}^6\\
			&=& \left\{a^{^{\tfrac{1}{3}}}\left[a^{^{\tfrac{5}{6}}}b\right]^{^{-\tfrac{1}{2}}} \right\}^6\\
			&=&	\left\{a^{^{\tfrac{1}{3}}}a^{^{-\tfrac{5}{12}}}b^{^{-\tfrac{1}{2}}}\right\}^6\\
			&=&	\left\{a^{^{-\tfrac{1}{12}}}b^{^{-\tfrac{1}{2}}}\right\}^6\\
			&=&	a^{^{-\tfrac{1}{2}}}b^{-3}\\
			&=& \dfrac{1}{\sqrt{a}b^3}=\dfrac{\sqrt {a}}{ab^3}.
		\end{eqnarray*}
	}
\end{ex}
\begin{ex}%[DCHT Toán 11 - KNTT -Nguyễn Thắng]%[1K6BH-2]
	Kết quả rút gọn biểu thức $L=\left [\left (1-2\sqrt{\dfrac{b}{a}}+\dfrac{b}{a}\right ): \left (a^{^{\tfrac{1}{2}}}-b^{^{\tfrac{1}{2}}}\right )^2\right ] \cdot \left (\dfrac{a^{^{\tfrac{1}{4}}}-a^{^{\tfrac{9}{4}}}}{a^{^{\tfrac{1}{4}}}-a^{^{\tfrac{5}{4}}}}-\dfrac{b^{^{-\tfrac{1}{2}}}-b^{^{\tfrac{3}{2}}}}{b^{^{\tfrac{1}{2}}}-b^{^{-\tfrac{1}{2}}}}\right )$ là
	\choice
	{$L=\dfrac{a+b+1}{a}$}
	{$L=\dfrac{a+b+2}{b}$}
	{$L=\dfrac{a+b-1}{a}$}
	{\True $L=\dfrac{a+b+2}{a}$}
	\loigiai{
		$L=\dfrac{\left (1-\sqrt{\dfrac{b}{a}}\right )^2}{(\sqrt{a}-\sqrt{b})^2} \cdot \left (\dfrac{1-a^2}{1-a}-\dfrac{1-b^2}{b-1}\right ) = \dfrac{a+b+2}{a}$.}
\end{ex}
\begin{ex}%[DCHT Toán 11 - KNTT -Nguyễn Thắng]%[1K6BH-2]
	Kết quả của biểu thức $M=\left [\left (a^{^{\tfrac{4}{3}}}: \sqrt[3]{a}\right ) : a^{^{\tfrac{1}{3}}} \cdot \sqrt{a}\right ] + \left [ a^{^{\tfrac{1}{2}}}a^{^{\tfrac{1}{3}}}\cdot \sqrt[6]{a}\cdot \left (\sqrt[3]{a} : a^{^{\tfrac{1}{6}}}\right )\right ]$ là
	\choice
	{$2a^{^{\tfrac{6}{7}}}$}
	{$2a^{^{\tfrac{5}{7}}}$}
	{\True $2a^{^{\tfrac{7}{6}}}$}
	{$2a^{^{\tfrac{5}{6}}}$}
	\loigiai{
		$M=a^{^{\tfrac{4}{3}-\tfrac{1}{3}-\tfrac{1}{3} + \tfrac{1}{2}}} + a^{^{\tfrac{1}{2}+\tfrac{1}{3} +\tfrac{1}{6} +\tfrac{1}{3}-\tfrac{1}{6}}} =a^{^{\tfrac{7}{6}}} + a^{^{\tfrac{7}{6}}} =2a^{^{\tfrac{7}{6}}}$.}
\end{ex}
\begin{ex}%[DCHT Toán 11 - KNTT -Nguyễn Thắng]%[1K6KH-2]
	Cho $a>0$, $b>0$ và biểu thức $T=2(a+b)^{-1} \cdot (ab)^{^{\tfrac{1}{2}}}\cdot\left[1+\dfrac{1}{4}{\left(\sqrt{\dfrac{a}{b}}-\sqrt{\dfrac{b}{a}}\right)}^2\right]^{^{\tfrac{1}{2}}}$. Tính giá trị của $T$.
	\choice
	{$T=\dfrac{2}{3}$}
	{$T=\dfrac{1}{3}$}
	{$T=\dfrac{1}{2}$}
	{\True $T=1$}
	\loigiai{
		Ta có 
		\begin{eqnarray*}
			T&=&2(a+b)^{-1} \cdot (ab)^{^{\tfrac{1}{2}}} \cdot {\left[1+\dfrac{1}{4}{\left(\sqrt{\dfrac{a}{b}}-\sqrt{\dfrac{b}{a}}\right)}^2\right]}^{^{\tfrac{1}{2}}}\\
			&=&2(a+b)^{-1} \cdot (ab)^{^{\tfrac{1}{2}}} \cdot {\left[1+\dfrac{1}{4}\left(\dfrac{a}{b}-2+\dfrac{b}{a}\right)\right]}^{^{\tfrac{1}{2}}}\\
			&=&2(a+b)^{-1} \cdot (ab)^{^{\tfrac{1}{2}}} \cdot {\left[\dfrac{1}{4}{\left(\dfrac{a+b}{\sqrt{ab}}\right)}^2\right]}^{^{\tfrac{1}{2}}}\\
			&=&2 \cdot \dfrac{1}{a+b} \cdot \sqrt{ab}\cdot\dfrac{1}{2} \cdot \dfrac{a+b}{\sqrt{ab}}\\
			&=&1.
		\end{eqnarray*}
	}
	
\end{ex}
\begin{ex}%[DCHT Toán 11 - KNTT -Nguyễn Thắng]%[1K6BH-2]
	Cho biểu thức $P=\dfrac{a^{2+\sqrt{3}}\cdot{\left(a^{1-\sqrt{3}}\right)^{1+\sqrt{3}}}}{a^{1+\sqrt{3}}}$, với $a>0$. Mệnh đề nào sau đây là đúng?
	\choice
	{$P=a^{\sqrt{3}}$}
	{\True $P=\dfrac{1}{a}$}
	{$P=a$}
	{$P=\dfrac{1}{a^{\sqrt{3}}}$}
	\loigiai{
		Ta có $P=\dfrac{a^{2+\sqrt{3}}\cdot{\left(a^{1-\sqrt{3}}\right)^{1+\sqrt{3}}}}{a^{1+\sqrt{3}}}=\dfrac{a^{2+\sqrt{3}}\cdot{a^{(1-\sqrt{3})(1+\sqrt{3})}}}{a^{1+\sqrt{3}}}=\dfrac{a^{2+\sqrt{3}}\cdot{a^{-2}}}{a^{1+\sqrt{3}}}=\dfrac{a^{\sqrt{3}}}{a^{1+\sqrt{3}}}=\dfrac{1}{a}$.}
\end{ex}
\begin{ex}%[DCHT Toán 11 - KNTT -Nguyễn Thắng]%[1K6KH-2]
	Cho biểu thức $P=\sqrt[3]{\dfrac{a}{b}\cdot \sqrt[4]{\dfrac{b}{a}\cdot\sqrt{\dfrac{a}{b}}}}=\left(\dfrac{a}{b}\right)^m$ với $a$; $b>0$. Tìm $m$.
	\choice
	{\True $m=\dfrac{7}{24}$}
	{$m=\dfrac{7}{12}$}
	{$m=-\dfrac{7}{12}$}
	{$m=-\dfrac{7}{24}$}
	\loigiai{
		Đặt $x=\dfrac{a}{b}\Rightarrow \dfrac{b}{a}=x^{-1}$.\\
		Khi đó $P=\sqrt[3]{x\sqrt[4]{x^{-1}{\sqrt{x}}}}=\sqrt[3]{x\sqrt[4]{x^{-1}\cdot{x^{^{\tfrac{1}{2}}}}}}=\sqrt[3]{x\sqrt[4]{x^{^{-\tfrac{1}{2}}}}}=\sqrt[3]{x\cdot x^{^{\tfrac{-1}{8}}}}=\sqrt[3]{x^{^{\tfrac{7}{8}}}}=x^{^{\tfrac{7}{24}}}$.\\
		Do đó $P=\sqrt[3]{\dfrac{a}{b}\cdot \sqrt[4]{\dfrac{b}{a}\cdot\sqrt{\dfrac{a}{b}}}}=\left(\dfrac{a}{b}\right)^{^{\tfrac{7}{24}}}$ suy ra $m=\dfrac{7}{24}$.}
\end{ex}
\begin{ex}%[DCHT Toán 11 - KNTT -Nguyễn Thắng]%[1K6BH-2]
	Cho biểu thức với $Q=\dfrac{a^{^{\tfrac{7}{6}}}\cdot{b^{^{\tfrac{1}{3}}}}}{\sqrt[6]{ab^2}}$; $a, b>0$. Mệnh đề nào sau đây là đúng?
	\choice
	{\True $Q=a$}
	{$Q=\dfrac{a}{b}$}
	{$Q=ab$}
	{$Q=a\sqrt{b}$}
	\loigiai{
		Ta có $Q=\dfrac{a^{^{\tfrac{7}{6}}}\cdot b^{^{\tfrac{1}{3}}}}{\sqrt[6]{ab^2}}=\dfrac{a^{^{\tfrac{7}{6}}}\cdot{b^{^{\tfrac{1}{3}}}}}{(ab^2)^{^{\tfrac{1}{6}}}}=\dfrac{a^{^{\tfrac{7}{6}}}\cdot{b^{^{\tfrac{1}{3}}}}}{a^{^{\tfrac{1}{6}}}\cdot{b^{^{\tfrac{2}{6}}}}}=a$.}
\end{ex}
\begin{ex}%[DCHT Toán 11 - KNTT -Nguyễn Thắng]%[1K6BH-2]
	Cho $x$ là số thực dương, viết biểu thức $Q=\sqrt{x\cdot \sqrt[3]x^2}\cdot \sqrt[6]x$ dưới dạng lũy thừa với số hữu tỉ
	\choice
	{$Q=x^{^{\tfrac{5}{36}}}$}
	{$Q=x^{^{\tfrac{2}{3}}}$}
	{\True $Q=x$}
	{$Q=x^2$}
	\loigiai{
		Ta có $Q=\sqrt{x\cdot \sqrt[3]x^2}\cdot \sqrt[6]x=\sqrt{x\cdot{x^{^{\tfrac{2}{3}}}}}\cdot{x^{^{\tfrac{1}{6}}}}=x^{^{\tfrac{5}{6}}}\cdot{x^{^{\tfrac{1}{6}}}}=x$.}
\end{ex}

\begin{ex}%[DCHT Toán 11 - KNTT -Nguyễn Thắng]%[1K6BH-2]
	Cho biểu thức $P=\sqrt[3]{x\cdot \sqrt[4]{x^2\cdot \sqrt{x^3}}}$ với $x>0$. Mệnh đề nào dưới đây là đúng?
	\choice
	{$P=x^{^{\tfrac{5}{6}}}$}
	{$P=x^{^{\tfrac{2}{3}}}$}
	{\True $P=x^{^{\tfrac{5}{8}}}$}
	{$P=x^{^{\tfrac{3}{4}}}$}
	\loigiai{
		Ta có $P=\sqrt[3]{x\cdot \sqrt[4]{x^2\cdot \sqrt{x^3}}}=\sqrt[3]{x\cdot \sqrt[4]{x^2\cdot x^{^{\tfrac{3}{2}}}}}=\sqrt[3]{x\cdot{\left(x^{^{\tfrac{7}{2}}}\right)^{^{\tfrac{1}{4}}}}}=\left(x^{^{\tfrac{15}{8}}}\right)^{^{\tfrac{1}{3}}}=x^{^{\tfrac{5}{8}}}$.}
\end{ex}

\begin{ex}%[DCHT Toán 11 - KNTT -Nguyễn Thắng]%[1K6BH-2]
	Rút gọn biểu thức $T=\dfrac{a^2\cdot{\left(a^{-2}\cdot b^3\right)^2}\cdot{b^{-1}}}{\left(a^{-1}\cdot b\right)^3\cdot{a^{-5}}\cdot{b^{-2}}}$ với $a$, $b$ là hai số thực dương.
	\choice
	{$T=a^4\cdot b^6$}
	{$T=a^6\cdot b^6$}
	{$T=a^4\cdot b^4$}
	{\True $T=a^6\cdot b^4$}
	\loigiai{
		Ta có $T=\dfrac{a^2\cdot{\left(a^{-2}\cdot b^3\right)^2}\cdot{b^{-1}}}{\left(a^{-1}\cdot b\right)^3\cdot{a^{-5}}\cdot{b^{-2}}}=\dfrac{a^2\cdot{a^{-4}}\cdot b^6\cdot{b^{-1}}}{a^{-3}\cdot b^3\cdot{a^{-5}}\cdot{b^{-2}}}=\dfrac{a^{-2}\cdot b^5}{a^{-8}\cdot b}=a^6\cdot b^4$.}
\end{ex}

\begin{ex}%[DCHT Toán 11 - KNTT -Nguyễn Thắng]%[1K6BH-2]
	Biết rằng $\dfrac{x^{a^2}}{x^{b^2}}=x^9$ với $x>1$ và $a+b=3$. Tính giá trị của biểu thức $P=a-b$.
	\choice
	{$P=1$}
	{\True $P=3$}
	{$P=2$}
	{$P=4$}
	\loigiai{
		Ta có $\dfrac{x^{a^2}}{x^{b^2}}=x^9\Leftrightarrow{x^{a^2-b^2}}=x^9\\\xrightarrow{x>1}a^2-b^2=9\Leftrightarrow (a+b)(a-b)=9\Leftrightarrow a-b=\dfrac{9}{a+b}=\dfrac{9}{3}=3$.}
\end{ex}

\begin{ex}%[DCHT Toán 11 - KNTT -Nguyễn Thắng]%[1K6KH-2]
	Cho $x$, $y>0$. Biết rằng $\sqrt{x\cdot \sqrt[4]{\dfrac{\sqrt[3]x}{x^3}}}=x^m$ và $y^2\cdot \sqrt{y\cdot \sqrt[3]{\dfrac{1}{y^2}}}=y^n$. Tính $m-n$.
	\choice
	{$0$}
	{$2$}
	{$1$}
	{\True$-2$}
	\loigiai{
		Ta có $\sqrt{x\cdot \sqrt[4]{\dfrac{\sqrt[3]x}{x^3}}}=\sqrt{x\cdot \sqrt[4]{\dfrac{x^{^{\tfrac{1}{3}}}}{x^3}}}=\sqrt{x\cdot \sqrt[4]{x^{^{\tfrac{-8}{3}}}}}=\sqrt{x\cdot{x^{^{\tfrac{-2}{3}}}}}=\sqrt{x^{^{\tfrac{1}{3}}}}=x^{^{\tfrac{1}{6}}}\Rightarrow m=\dfrac{1}{6}$.\\
		Lại có $y^2\cdot \sqrt{y\cdot \sqrt[3]{\dfrac{1}{y^2}}}=y^2\cdot \sqrt{y\cdot \sqrt[3]{y^{-2}}}=y^2\cdot \sqrt{y\cdot{y^{^{\tfrac{-2}{3}}}}}=y^2\cdot \sqrt{y^{^{\tfrac{1}{3}}}}=y^2\cdot{y^{^{\tfrac{1}{6}}}}=y^{^{\tfrac{13}{6}}}\Rightarrow n=\dfrac{13}{6}$.\\
		Do đó $m-n=-2$.}
\end{ex}
\begin{ex}%[DCHT Toán 11 - KNTT -Nguyễn Thắng]%[1K6BH-2]
	Đơn giản biểu thức $T=\dfrac{\sqrt{a}-\sqrt{b}}{\sqrt[4]a-\sqrt[4]b}-\dfrac{\sqrt{a}+\sqrt[4]ab}{\sqrt[4]a+\sqrt[4]b}$ ta được
	\choice
	{$T=\sqrt[4]a$}
	{\True$T=\sqrt[4]b$}
	{$T=\sqrt[4]a+\sqrt[4]b$}
	{$T=-\sqrt[4]b$}
	\loigiai{
		Ta có $T=\dfrac{(\sqrt[4]a)^2-(\sqrt[4]b)^2}{\sqrt[4]a-\sqrt[4]b}-\dfrac{\sqrt[4]a(\sqrt[4]a+\sqrt[4]b)}{\sqrt[4]a+\sqrt[4]b}=\sqrt[4]a+\sqrt[4]b-\sqrt[4]a=\sqrt[4]b$.}
\end{ex}

\begin{ex}%[DCHT Toán 11 - KNTT -Nguyễn Thắng]%[1K6GH-2]
	Rút gọn biểu thức $Q=\dfrac{1}{x}\cdot \left(\dfrac{\sqrt{x+1}+\sqrt{x-1}}{\sqrt{x+1}-\sqrt{x-1}}+\dfrac{\sqrt{x+1}-\sqrt{x-1}}{\sqrt{x+1}+\sqrt{x-1}}\right)$ với $x>1$ ta được
	\choice
	{$Q=1$}
	{$Q=2x$}
	{\True $Q=2$}
	{$Q=-2$}
	\loigiai{
		Ta có $(\sqrt{x+1}+\sqrt{x-1})^2+(\sqrt{x+1}-\sqrt{x-1})^2=2x+2\sqrt{x^2-1}+2x-2\sqrt{x^2-1}=4x$,\\
		$(\sqrt{x+1}-\sqrt{x-1})\cdot (\sqrt{1x+1}+\sqrt{x-1})=x+1-x+1=2$.\\
		Suy ra $Q=\dfrac{1}{x}\cdot \dfrac{(\sqrt{x+1}+\sqrt{x-1})^2+(\sqrt{x+1}-\sqrt{x-1})^2}{(\sqrt{x+1}-\sqrt{x-1})\cdot (\sqrt{1x+1}+\sqrt{x-1})}=\dfrac{1}{x}\cdot \dfrac{4x}{2}=2$.}
\end{ex}
\Closesolutionfile{ans}
\begin{indapan}{10}
	{ans/ans-1K6-1-Dang2}
\end{indapan}

\begin{dang}{So sánh các lũy thừa.}
	\begin{itemize}
		\item Nếu $a>1$ thì $a^{\alpha} >a^{\beta} \Leftrightarrow \alpha > \beta$. 
		\item Nếu $0<a<1$ thì $a^{\alpha} >a^{\beta} \Leftrightarrow \alpha < \beta$. 
		\item $\displaystyle \mathrm{e} = \lim_{n \to +\infty} \left (1+\dfrac{1}{n}\right )^n  \approx 2{,}718281 \dots$
		\item Để so sánh $\sqrt[s_1]{a}$ và $\sqrt[s_2]{b}$. Ta sẽ đưa $2$ căn đã cho về cùng bậc $n$ (với $n$ là bội số chung của $s_1$ và $s_2$). Khi đó, hai số so sánh mới lần lượt là $\sqrt[n]{A}$ và $\sqrt[n]{B}$. Từ đó so sánh $A$ và $B$ và suy ra kết quả của $\sqrt[s_1]{a}$ và $\sqrt[s_2]{b}$.
	\end{itemize} 
\end{dang}
\subsubsection{Ví dụ mẫu}
\begin{vd}%[DCHT Toán 11 - KNTT -Nguyễn Thắng]%[1K6YH-3]
	So sánh cặp số: $4^{-\sqrt{3}}$ và $4^{-\sqrt{2}}$.\dapso{$4^{-\sqrt{3}}<4^{-\sqrt{2}}$.}
	\loigiai{
		Ta có  $4>1$ và $-\sqrt{3} < -\sqrt{2} \Rightarrow  4^{-\sqrt{3}}<4^{-\sqrt{2}}$.}
\end{vd}
\begin{vd}%[DCHT Toán 11 - KNTT -Nguyễn Thắng]%[1K6YH-3]
	So sánh cặp số: $\left (\dfrac{1}{2}\right )^{1{,}4}$ và $\left (\dfrac{1}{2}\right )^{\sqrt{2}}$.\dapso{$\left (\dfrac{1}{2}\right )^{1{,}4} > \left (\dfrac{1}{2}\right )^{\sqrt{2}}$.}
	\loigiai{
		Ta có $\dfrac{1}{2} <1$ và $1{,}4 <\sqrt{2} \Rightarrow \left (\dfrac{1}{2}\right )^{1{,}4} > \left (\dfrac{1}{2}\right )^{\sqrt{2}}$.}
\end{vd}
\begin{vd}%[DCHT Toán 11 - KNTT -Nguyễn Thắng]%[1K6BH-3]
	So sánh hai số $\sqrt{17}$ và $\sqrt[3]{28}$.\dapso{$\sqrt{17}> \sqrt[3]{28}$.}
	\loigiai{
		Ta có $\sqrt{17} = \sqrt[6]{17^3}=\sqrt[6]{4913}$ và $\sqrt[3]{28} = \sqrt[6]{28^2} = \sqrt[6]{784} \Rightarrow \sqrt{17}> \sqrt[3]{28}$. }
\end{vd}

\begin{vd}%[DCHT Toán 11 - KNTT -Nguyễn Thắng]%[1K6BH-3]
	So sánh hai số $m,n$ nếu: $3{,}2^m < 3{,}2^n$.\dapso{$m<n$.}
	\loigiai{
		$3{,}2 >1$ suy ra $3{,}2^m < 3{,}2^n \Leftrightarrow m<n$. 	}
\end{vd}
\begin{vd}%[DCHT Toán 11 - KNTT -Nguyễn Thắng]%[1K6BH-3]
	So sánh hai số $a,b$ nếu: $\left (\dfrac{1}{9}\right )^a < \left (\dfrac{1}{9}\right )^b$.\dapso{$a>b$.}
	\loigiai{
		$\dfrac{1}{9}<1$ suy ra $\left (\dfrac{1}{9}\right )^a < \left (\dfrac{1}{9}\right )^b \Leftrightarrow a>b$.}
\end{vd}
\begin{vd}%[DCHT Toán 11 - KNTT -Nguyễn Thắng]%[1K6BH-3]
	Có thể kết luận gì về cơ số $a$ nếu
	\begin{enumerate}[a)]
		\item  $(a-1)^{^{-\tfrac{2}{3}}} < (a-1)^{^{-\tfrac{1}{3}}}$ ;
		\item $(2a+1)^{-3} > (2a+1)^{-1}$ ;
		\item $\left (\dfrac{1}{a}\right)^{-0{,}2}<a^2$.
	\end{enumerate}\dapso{a) $a>2$; b) $\hoac{&a<-1\\&-\dfrac{1}{2}<a<0}$; c) $a<0$.}
	\loigiai{\begin{enumerate}[a)]
			\item Ta có $-\dfrac{2}{3} < -\dfrac{1}{3}$ và $(a-1)^{^{-\tfrac{2}{3}}} < (a-1)^{^{-\tfrac{1}{3}}}$ suy ra $a-1 >1 \Leftrightarrow a>2$.
			\item Ta có $(2a+1)^{-3} > (2a+1)^{-1} \Leftrightarrow \dfrac{1}{(2a+1)^3} > \dfrac{1}{2a+1}\Leftrightarrow \hoac{&2a+1 <-1\\&0<2a+1 <1} \Leftrightarrow \hoac{&a<-1\\&-\dfrac{1}{2}<a<0.}$
			\item Ta có $-\dfrac{1}{3} > -\dfrac{1}{2}$ và $(1-a)^{^{-\tfrac{1}{3}}} > (1-a)^{^{-\tfrac{1}{2}}}$ suy ra $1-a >1 \Leftrightarrow a<0$
	\end{enumerate}}
\end{vd}
\subsubsection{Bài tập rèn luyện}
\centerline{\fcolorbox{red}{yellow!50}{\bf {BÀI TẬP TỰ LUẬN }}}

\begin{bt}%[DCHT Toán 11 - KNTT -Nguyễn Thắng]%[1K6YH-3]
	Cho $\left(\sqrt{2}-1\right)^m < \left(\sqrt{2}-1\right)^n$. So sánh hai số $m$, $n$.\dapso{$m>n$.}
	\loigiai{
		Do $0<\sqrt{2}-1< 1 $ nên $\left(\sqrt{2}-1\right)^m < \left(\sqrt{2}-1\right)^n \Leftrightarrow m>n$.	
	}
\end{bt}
\begin{bt}%[DCHT Toán 11 - KNTT -Nguyễn Thắng]%[1K6YH-3]
	Chứng minh rằng $\left(\dfrac{2}{3}\right)^{\sqrt 2}$ nhỏ hơn $1$.
	\loigiai{
		Ta có $\dfrac{2}{3}<1\Rightarrow\left(\dfrac{2}{3}\right)^{\sqrt 2}<\left(\dfrac{2}{3}\right)^{0}=1$.
	}
\end{bt}
\begin{bt}%[DCHT Toán 11 - KNTT -Nguyễn Thắng]%[1K6BH-3]
	Tìm tập hợp các số thực $a$ sao cho $\left(7+4\sqrt{3}\right)^{a-1}<7-4\sqrt{3}$.\dapso{$a<0$.}
	\loigiai{
		\[\left(7+4\sqrt{3}\right)^{a-1}<7-4\sqrt{3}\Leftrightarrow \left(7+4\sqrt{3}\right)^{a-1}<\left(7+4\sqrt{3}\right)^{-1}.\]
		Mà ta có $7+4\sqrt{3}>1$ nên $\left(7+4\sqrt{3}\right)^{a-1}<\left(7+4\sqrt{3}\right)^{-1}\Leftrightarrow a-1<-1\Leftrightarrow a<0$.
	}
\end{bt}

\begin{bt}%[DCHT Toán 11 - KNTT -Nguyễn Thắng]%[1K6KH-3]
	Giải bất phương trình $(x-2)^{^{-\tfrac{1}{3}}}>(x-2)^{^{-\tfrac{1}{6}}}$.\dapso{$2<x<3$.}
	\loigiai{
		Do $-\dfrac{1}{3}<-\dfrac{1}{6}$ và $-\dfrac{1}{3};-\dfrac{1}{6} \notin\mathbb{Z}$ nên bất phương trình tương đương $\heva{& x-2>0\\	&x-2<1} \Leftrightarrow 2<x<3$.
	}
\end{bt}
\centerline{\fcolorbox{red}{yellow!50}{\bf {CÂU HỎI TRẮC NGHIỆM}}}
\Opensolutionfile{ans}[ans/ans-1K6-1-Dang3]

\begin{ex}%[DCHT Toán 11 - KNTT -Nguyễn Thắng]%[1K6YH-3]
	Cho $a,b>0$ thỏa mãn $a^{^{\tfrac{1}{2}}}>a^{^{\tfrac{1}{3}}}$, $b^{^{\tfrac{2}{3}}}>b^{^{\tfrac{3}{4}}}$. Khi đó, khẳng định nào sau đây là đúng?
	\choice
	{$0<a<1$, $b>1$}
	{\True $0<b<1<a$}
	{$0<a<1$, $0<b<1$}
	{$a>1$, $b>1$}
	\loigiai{
		Ta có $\heva{& a^{^{\tfrac{1}{2}}}>a^{^{\tfrac{1}{3}}}\Leftrightarrow a>1 \left(\text{ do }\dfrac{1}{2}>\dfrac{1}{3}\right) \\ & b^{^{\tfrac{2}{3}}}>b^{^{\tfrac{3}{4}}}\Leftrightarrow 0<b<1 \left(\text{ do }\dfrac{2}{3}>\dfrac{3}{4}\right)}\Leftrightarrow 0<b<1<a$.
	}
\end{ex}

\begin{ex}%[DCHT Toán 11 - KNTT -Nguyễn Thắng]%[1K6YH-3]
	Cho số thực $a>1$ và các số thực $\alpha , \beta$. Kết luận nào sau đây đúng?
	\choice
	{$\dfrac{1}{a^{\alpha}}<0, \, \alpha \in \mathbb{R}$}
	{$a^{\alpha}<1, \, \alpha \in \mathbb{R}$}
	{$a^{\alpha}>1, \, \alpha \in \mathbb{R}$}
	{\True $a^{\alpha}>a^{\beta} \Leftrightarrow \alpha > \beta$}
	\loigiai{
		Theo tính chất của lũy thừa với cơ số $a>1$. Khi đó $a^{\alpha}>a^{\beta} \Leftrightarrow \alpha > \beta$.
	}
	
\end{ex}

\begin{ex}%[DCHT Toán 11 - KNTT -Nguyễn Thắng]%[1K6YH-3]
	Nếu $a^{^{^{\tfrac{1}{5}}}}>a^{^{^{\tfrac{1}{3}}}}$ và $\log_b \dfrac{1}{3}<\log_b \dfrac{1}{2}$ thì
	\choice
	{$\heva{&0<a<1\\&0<b<1}$}
	{$\heva{&a>1\\&b>1}$}
	{\True $\heva{&0<a<1\\&b>1}$}
	{$\heva{&a>1\\&0<b<1}$}
	\loigiai{
		Vì $\dfrac{1}{5}<\dfrac{1}{3}$ nên từ $a^{^{^{\tfrac{1}{5}}}}>a^{^{^{\tfrac{1}{3}}}}$ suy ra $0<a<1$.\\
		Vì $\dfrac{1}{3}<\dfrac{1}{2}$ nên từ $\log_b \dfrac{1}{3}<\log_b \dfrac{1}{2}$ suy ra $b>1$.
	}
\end{ex}
\begin{ex}%[DCHT Toán 11 - KNTT -Nguyễn Thắng]%[1K6YH-3]
	Cho $a$, $b$ là hai số thực dương và $\alpha$, $\beta $ là các số thực. Mệnh đề nào sau đây \textbf{sai}?
	\choice
	{$ a^{\alpha }\cdot a^{\beta }=a^{\alpha +\beta }$}
	{$ a^{\alpha \cdot \beta }=(a^{\alpha })^{\beta }$}
	{\True $ a^{\alpha }+a^{\beta }=a^{\alpha +\beta }$}
	{$ a^{\alpha }\cdot b^{\alpha }=(a\cdot b)^{\alpha }$}
	\loigiai{
		Theo tính chất của lũy thừa với số mũ thực ta có $ a^{\alpha }\cdot a^{\beta }=a^{\alpha +\beta }$, 
		$ a^{\alpha \cdot \beta }=(a^{\alpha })^{\beta }$, $ a^{\alpha }\cdot b^{\alpha }=(a\cdot b)^{\alpha }$ đúng. Suy ra  $ a^{\alpha }+a^{\beta }=a^{\alpha +\beta }$ sai.}
\end{ex}
\begin{ex}%[DCHT Toán 11 - KNTT -Nguyễn Thắng]%[1K6YH-3]
	Điều nào sau đây là đúng?
	\choice
	{$a^m<a^n\Leftrightarrow m>n$}
	{$a^m>a^n\Leftrightarrow m>n$}
	{\True $\left(\dfrac{\pi}4\right)^9>\left(\dfrac{\pi}4\right)^3$}
	{Nếu $0<a<b$ và $a^m<b^m$ thì $m>0$}
	\loigiai{
		Xét các đáp án:
		\begin{itemize}
			\item Đáp án $a^m<a^n\Leftrightarrow m>n$ ta thấy sai khi $a>1$.
			\item Đáp án $a^m>a^n\Leftrightarrow m>n$ ta thấy sai khi $0<a<1$.
			\item Đáp án $\left(\dfrac{\pi}4\right)^9>\left(\dfrac{\pi}4\right)^3$ ta thấy $\dfrac{\pi}{4}<1$ và $9>3$ nên $\left(\dfrac{\pi}4\right)^9<\left(\dfrac{\pi}4\right)^3$. Do đó đáp án này sai.
			\item Đáp án nếu $0<a<b$ và $a^m<b^m$ thì $m>0$ đúng.
		\end{itemize}
	}
\end{ex}
\begin{ex}%[DCHT Toán 11 - KNTT -Nguyễn Thắng]%[1K6YH-3]
	Cho $\pi^{\alpha}<\pi^{\beta}$. Kết luận nào sau đây đúng?
	\choice
	{$\alpha>\beta$}
	{$\alpha\beta=1$}
	{\True $\alpha<\beta$}
	{$\alpha+\beta=0$}
	\loigiai{
		Do $\pi >1$ nên từ $\pi^{\alpha}<\pi^{\beta}$, ta có $\alpha<\beta$. 
	}
\end{ex}

\begin{ex}%[DCHT Toán 11 - KNTT -Nguyễn Thắng]%[1K6YH-3]
	Cho các số thực $a$, $b$ thỏa mãn $0<a<b$. Mệnh đề nào sau đây đúng?
	\choice
	{$a^x<b^x$ với mọi $x\neq 0$}
	{\True $a^x<b^x$ với mọi $x> 0$}
	{$a^x<b^x$ với mọi $x< 0$}
	{$a^x<b^x$ với mọi $x\in\mathbb{R}$}
	\loigiai{
		Theo tính chất lũy thừa, ta có $a^x<b^x$ với mọi $x>0$.
	}
\end{ex}
\begin{ex}%[DCHT Toán 11 - KNTT -Nguyễn Thắng]%[1K6YH-3]
	Với $a>0$, $b>0$ và $\alpha$, $\beta$ là các số thực bất kì, đẳng thức nào sau đây \textbf{sai}?
	\choice
	{$\dfrac{a^{\alpha}}{a^{\beta}}=a^{\alpha-\beta}$}
	{$a^{\alpha} \cdot a^{\beta}=a^{\alpha+\beta}$}
	{\True $\dfrac{a^{\alpha}}{a^{\beta}}=\left(\dfrac{a}{b}\right)^{\alpha-\beta}$}
	{$a^{\alpha} \cdot b^{\alpha}=\left(ab\right)^{\alpha}$}
	\loigiai{
		Theo tính chất của lũy thừa thì mệnh đề sai là $\dfrac{a^{\alpha}}{a^{\beta}}=\left(\dfrac{a}{b}\right)^{\alpha-\beta}$.
	}
\end{ex}

\begin{ex}%[DCHT Toán 11 - KNTT -Nguyễn Thắng]%[1K6YH-3]
	Cho hai số thực $\alpha, \beta $ và số thực dương $a$. Khẳng định nào sau đây là khẳng định \textbf{sai}?
	\choice
	{\True $a^{\alpha +\beta}=a^{\alpha}+a^{\beta}$}
	{$a^{\alpha -\beta}=\dfrac{a^{\alpha}}{a^{\beta}}$}
	{$\left(a^{\alpha}\right)^{\beta}=a^{\alpha\cdot\beta}$}
	{$a^{\alpha\cdot\beta}=\left(a^{\beta}\right)^{\alpha}$}
	\loigiai{
		Ta có $a^{\alpha +\beta}=a^{\alpha}\cdot a^{\beta}$.}
\end{ex}
\begin{ex}%[DCHT Toán 11 - KNTT -Nguyễn Thắng]%[1K6YH-3]
	Cho $\pi^{\alpha}>\pi^{\beta}$. Kết luận nào sau đây là đúng?
	\choice
	{$\alpha \cdot \beta = 1$}
	{\True $\alpha > \beta$}
	{$\alpha < \beta$}
	{$\alpha + \beta = 0$}
	\loigiai{
		Do $\pi>1$ nên từ giả thiết $\pi^{\alpha}>\pi^{\beta}$ ta có $\alpha>\beta$.
	}
\end{ex}
\begin{ex}%[DCHT Toán 11 - KNTT -Nguyễn Thắng]%[1K6YH-3]
	Cho các số thực $a$, $b$ thỏa mãn $\left(\sqrt{2}-1\right)^a>\sqrt{2}+1>\left(\sqrt{2}-1\right)^b$. Khẳng định nào sau đây là đúng?
	\choice
	{$b>a>-1$}
	{$a>b>-1$}
	{\True $a<-1<b$}
	{$a<-1<b$}
	\loigiai{Từ gả thiết ta có $\left(\sqrt{2}-1\right)^a>\left(\sqrt{2}-1\right)^{-1}>\left(\sqrt{2}-1\right)^b$.\\ 
		Vì cơ số $0<\left(\sqrt{2}-1\right)<1$ nên ta suy ra $a<-1<b$.
	}
\end{ex}
\begin{ex}%[DCHT Toán 11 - KNTT -Nguyễn Thắng]%[1K6BH-3]
	Cho $a>1$. Mệnh đề nào sau đây là đúng?
	\choice
	{$\dfrac{\sqrt[3]{a^2}}{a}>1$}
	{\True ${a}^{-\sqrt{3}}>\dfrac{1}{a}^{\sqrt{5}}$}
	{${a}^{^{\tfrac{1}{3}}}>\sqrt{a}$}
	{$\dfrac{1}{a^{2016}}<\dfrac{1}{a^{2017}}$}
	\loigiai{
		$\dfrac{1}{a}^{\sqrt{5}}=a^{-\sqrt{5}}$. Do $a>1$ suy ra $a^{-\sqrt{3}}>a^{-\sqrt{5}}$.
	}  
\end{ex}
\begin{ex}%[DCHT Toán 11 - KNTT -Nguyễn Thắng]%[1K6BH-3]
	Cho $\left(0{,}25\pi\right)^{\alpha}>\left(0{,}25\pi\right)^{\beta}$. Kết luận nào sau đây đúng?
	\choice
	{$\alpha\cdot \beta=1$}
	{$\alpha>\beta$}
	{$\alpha+\beta=0$}
	{\True $\alpha<\beta$}
	\loigiai{
		Do $0<0{,}25\pi<1$ nên $\left(0{,}25\pi\right)^{\alpha}>\left(0{,}25\pi\right)^{\beta}\Leftrightarrow \alpha <\beta$.
	}
\end{ex}
\begin{ex}%[DCHT Toán 11 - KNTT -Nguyễn Thắng]%[1K6BH-3]
	Nếu $a^{^{\tfrac{2017}{2018}}}<a^{^{\tfrac{2018}{2017}}}$ và $\left(\sqrt{2018}-\sqrt{2017}\right)^b>\sqrt{2018}+\sqrt{2017}$ thì
	\choice
	{$a<1,b>-1$}
	{$a>1,b>1$}
	{$a<1,b<-1$}
	{\True $a>1,b<-1$}
	\loigiai{
		Vì $a^{^{\tfrac{2017}{2018}}}<a^{^{\tfrac{2018}{2017}}}$ nên $a>1$ do $\dfrac{2017}{2018}<\dfrac{2018}{2017}.$\\
		Ta có $\left(\sqrt{2018}-\sqrt{2017}\right)^b>\sqrt{2018}+\sqrt{2017}\Leftrightarrow \left(\sqrt{2018}-\sqrt{2017}\right)^b>\left(\sqrt{2018}-\sqrt{2017}\right)^{-1}$.\\
		Suy ra $b<-1$ vì $\sqrt{2018}-\sqrt{2017}<1$.
	}
\end{ex}
\begin{ex}%[DCHT Toán 11 - KNTT -Nguyễn Thắng]%[1K6BH-3]
	Trong các khẳng định sau, khẳng định nào \textbf{sai}?
	\choice
	{$\left(1 - \dfrac{\sqrt{2}}{2}\right)^{2018}<\left(1 - \dfrac{\sqrt{2}}{2}\right)^{2017}$}
	{$2^{\sqrt{2}+ 1}>2^{\sqrt{3}}$}
	{$\left(\sqrt{2}-1\right)^{2017}>\left(\sqrt{2}-1\right)^{2018}$}
	{\True $\left(\sqrt{3}-1\right)^{2018}>\left(\sqrt{3}-1\right)^{2017}$}
	\loigiai{Ta có $0<\sqrt{3}-1<1$ do đó $\left(\sqrt{3}-1\right)^{2018}<\left(\sqrt{3}-1\right)^{2017}$. 
	}
\end{ex}
\begin{ex}%[DCHT Toán 11 - KNTT -Nguyễn Thắng]%[1K6BH-3]
	Tìm điều kiện của $m$ để $(m-1)^{-2\sqrt{3}}>(m-1)^{-3\sqrt{2}}$.
	\choice
	{$0<m<1$}
	{$m>1$}
	{$1<m<2$}
	{\True $m>2$}
	\loigiai
	{Ta có $-2\sqrt{3} = -\sqrt{12}$ và $-3\sqrt{2} = -\sqrt{18}$ nên $-2\sqrt{3} > -3\sqrt{2}$.\\
		Để $(m-1)^{-2\sqrt{3}}>(m-1)^{-3\sqrt{2}}$ thì $m-1>1 \Leftrightarrow m>2$.
	}
\end{ex}
\begin{ex}%[DCHT Toán 11 - KNTT -Nguyễn Thắng]%[1K6BH-3]
	Nếu $\left( 2-\sqrt{3} \right)^{a-1}<2+\sqrt{3}$ thì
	\choice
	{$a\ge 0$}
	{$a\le 1$}
	{\True $a>0$}
	{$a<1$}
	\loigiai{
		Ta có
		\begin{align*}
			2+\sqrt{3}>\left( 2-\sqrt{3} \right)^{a-1}=\left( 2+\sqrt{3} \right)^{1-a}.
		\end{align*}
		Do $2+\sqrt{3}>1$ nên $1>1-a$ hay $a>0$.
	}
\end{ex}
\begin{ex}%[DCHT Toán 11 - KNTT -Nguyễn Thắng]%[1K6BH-3]
	Cho $\left(\sqrt{2}-1\right)^m<\left(\sqrt{2}-1\right)^n$. Tìm mệnh đề đúng.
	\choice
	{\True $m>n$}
	{$m<n$}
	{$m=n$}
	{$m\leq n$}
	\loigiai{Do $0<\sqrt{2}-1<1$ nên $\left(\sqrt{2}-1\right)^m<\left(\sqrt{2}-1\right)^n\Rightarrow m>n$.}
	
\end{ex}
\begin{ex}%[DCHT Toán 11 - KNTT -Nguyễn Thắng]%[1K6BH-3]
	Khẳng định nào sau đây là khẳng định đúng?
	\choice
	{$a^m < a^n \Leftrightarrow m > n$}
	{$a^m > a^n \Leftrightarrow m > n$}
	{$\left(\dfrac{\pi}{4}\right)^9 > \left(\dfrac{\pi}{4}\right)^3$}
	{\True Nếu $0 < a < b$ và $a^m < b^m$ thì $m > 0$}
	\loigiai{
		Ta có nếu $0 < a < b$ và $a^m < b^m$ thì $m > 0$.
	}
	
\end{ex}
\begin{ex}%[DCHT Toán 11 - KNTT -Nguyễn Thắng]%[1K6BH-3]
	Cho $ p, q $ là các số thực thỏa mãn $ m=\left(\dfrac{1}{\mathrm{e}}\right)^{2p-q}, n=\left(\dfrac{1}{\mathrm{e}}\right)^{2q-p} $. Biết $ m>n $, hãy so sánh $ p $ và $ q $.
	\choice
	{$ 2p>q $}
	{$ p>2q $}
	{$ p>q $}
	{\True $ p<q $}
	\loigiai{
		Vì $ 0<\dfrac{1}{\mathrm{e}}<1 $ nên từ $ m>n $ suy ra $ 2p-q<2q-p\Leftrightarrow 3p<3q\Leftrightarrow p<q $.
	}
	
\end{ex}
\begin{ex}%[DCHT Toán 11 - KNTT -Nguyễn Thắng]%[1K6BH-3]
	Cho số thực $m$, số nào trong các số sau \textbf{không} bằng $\left(4^2\right)^m$?
	\choice
	{$\left(2^2\right)^{m}\left(4\right)^m$}
	{$\left(4^m\right)^2$}
	{$\left(2^4\right)^m$}
	{\True $\left(8\right)^m$}
	\loigiai{
		Xét các trường hợp
		\begin{itemize}
			\item $\left(2^2\right)^{m}\left(4\right)^m=\left(4\right)^{m}\left(4\right)^m=\left(4\right)^{2m}=\left(4^2\right)^m$.
			\item $\left(4^m\right)^2=\left(4^2\right)^m$. 
			\item $\left(2^4\right)^m=\left(4^2\right)^m$.
			\item $\left(8\right)^m =2^{3m} \ne 2^{4m}=\left(4^2\right)^m$.
		\end{itemize}
	}
\end{ex}
\begin{ex}%[DCHT Toán 11 - KNTT -Nguyễn Thắng]%[1K6BH-3]
	Mệnh đề nào sau đây là đúng?
	\choice
	{\True $(4-\sqrt{2})^3<(4-\sqrt{2})^4$}
	{$(2-\sqrt{2})^3<(2-\sqrt{2})^4$}
	{$(\sqrt{3}-\sqrt{2})^4<(\sqrt{3}-\sqrt{2})^5$}
	{$(\sqrt{11}-\sqrt{2})^6>(\sqrt{11}-\sqrt{2})^7$}
	\loigiai
	{Ta có $4-\sqrt{2}>1$, $0<\sqrt{3}-\sqrt{2}<1$, $0<2-\sqrt{2}<1$, $\sqrt{11}-\sqrt{2}>1$ nên $(4-\sqrt{2})^3<(4-\sqrt{2})^4$ là đúng.}
\end{ex}
\begin{ex}%[DCHT Toán 11 - KNTT -Nguyễn Thắng]%[1K6BH-3]%
	Cho $x$ là số thực lớn hơn $8$. Mệnh đề nào dưới đây đúng?
	\choice
	{$(x-8)^{-3}>(x-8)^{-4}$}
	{$(x^2)^3<x^5$}
	{\True $\left(\dfrac{x}{6}\right)^4>\left(\dfrac{x}{6}\right)^3$}
	{$\left(\dfrac{1}{x}\right)^{-3}<\left(\dfrac{1}{x}\right)^{-2}$}
	\loigiai{
		Do $x>8$, nên $x-8>0$, suy ra $(x-8)^{-3}>(x-8)^{-4}$ chưa thể khẳng định là đúng được.\\
		Mệnh đề $(x^2)^3<x^5$ là sai vì $(x^2)^3=x^6>x^5$.\\
		Vì $x>8$ nên $\dfrac{x}{6}>1$, nên mệnh đề $\left(\dfrac{x}{6}\right)^4>\left(\dfrac{x}{6}\right)^3$ là đúng.
	}
\end{ex}
\begin{ex}%[DCHT Toán 11 - KNTT -Nguyễn Thắng]%[1K6BH-3]
	Mệnh đề nào sau đây đúng?
	\choice
	{\True $\left(4-\sqrt{2}\right)^3<\left(4-\sqrt{2}\right)^4$}
	{$\left(2-\sqrt{2}\right)^3<\left(2-\sqrt{2}\right)^4$}
	{$\left(\sqrt{3}-\sqrt{2}\right)^4<\left(\sqrt{3}-\sqrt{2}\right)^5$}
	{$\left(\sqrt{11}-\sqrt{2}\right)^6>\left(\sqrt{11}-\sqrt{2}\right)^7$}
	\loigiai{
		\begin{itemize}
			\item Do $4-\sqrt{2} >1 $ nên ta có $\left(4-\sqrt{2}\right)^3<\left(4-\sqrt{2}\right)^4$ là mệnh đề đúng.
			\item $\left(2-\sqrt {2}\right)^3<\left(2-\sqrt{2}\right)^4$ sai do $0<2-\sqrt{2}<1$.
			\item $\left(\sqrt{3}-\sqrt{2}\right)^4<\left(\sqrt{3}-\sqrt{2}\right)^5$ sai do $0<\sqrt{3}-\sqrt{2}<1$.
			\item $\left(\sqrt{11}-\sqrt{2}\right)^6>\left(\sqrt{11}-\sqrt{2}\right)^7$ sai do $\sqrt{11}-\sqrt{2} >1$.
		\end{itemize}
	}
\end{ex}
\begin{ex}%[DCHT Toán 11 - KNTT -Nguyễn Thắng]%[1K6BH-3]
	Cho các số thực $a,b$ thỏa $(\sqrt{2019}-\sqrt{2018})^a>(\sqrt{2019}-\sqrt{2018})^b$. Kết luận nào sau đây đúng?
	\choice
	{$a>b$}
	{\True$a<b$}
	{$a=b$}
	{$a\geq b$}
	\loigiai{
		Ta có $0<\sqrt{2019}-\sqrt{2018}<1$ nên $(\sqrt{2019}-\sqrt{2018})^a>(\sqrt{2019}-\sqrt{2018})^b\Leftrightarrow a<b$.
	}
\end{ex}
\begin{ex}%[DCHT Toán 11 - KNTT -Nguyễn Thắng]%[1K6BH-3]
	Tìm $a$, biết rằng $(3-a)^{^{\tfrac{1}{2}}}>(3-a)^{\sqrt{2}}$.
	\choice
	{\True $2<a<3$}
	{$a<3$}
	{$a>3$}
	{$0<a<1$}
	\loigiai{
		Ta có $\dfrac{1}{2}<\sqrt{2}$, mà $(3-a)^{^{\tfrac{1}{2}}}>(3-a)^{\sqrt{2}}$.\\
		Suy ra $0<3-a<1\Leftrightarrow\heva{& 3-a>0\\& 3-a<1} \Leftrightarrow\heva{& a<3\\& a>2} \Leftrightarrow 2<a<3$.
	}
\end{ex}
\begin{ex}%[DCHT Toán 11 - KNTT -Nguyễn Thắng]%[1K6BH-3]
	Cho hai số thực $a$, $b$ thỏa mãn $0<a<1$, $b>1$. Biết $a^\alpha>b^\alpha$, mệnh đề nào sau đây đúng?
	\choice
	{\True $\alpha<0$}
	{$\alpha>1$}
	{$0<\alpha<1$}
	{$\alpha>-1$}	
	\loigiai{
		Do $0<a<1$, $b>1$ nên $\dfrac{a}{b}<1$. Do đó $a^\alpha>b^\alpha\Leftrightarrow \left( \dfrac{a}{b}\right)^ {\alpha}>1\Leftrightarrow \alpha<0$.
	}
\end{ex}
\begin{ex}%[DCHT Toán 11 - KNTT -Nguyễn Thắng]%[1K6BH-3]
	Cho $ a>1$. Mệnh đề nào sau đây là đúng?
	\choice
	{\True $ a^{-\sqrt{3}}>\dfrac{1}{a^{\sqrt{5}}}$}
	{$\dfrac{\sqrt[3]{a^2}}{a}>1$}
	{$\dfrac{1}{a^{2019}}<\dfrac{1}{a^{2020}}$}
	{$ a^{^{\tfrac{1}{3}}}>\sqrt{a}$}
	\loigiai{
		Với $ a>1$, ta có $ a^x>a^y\Leftrightarrow x>y$, $\forall x, y\in \mathbb{R}$.\\
		Ta có $-\sqrt{3}>-\sqrt{5}\Rightarrow a^{-\sqrt{3}}>a^{-\sqrt{5}}\Rightarrow a^{-\sqrt{3}}>\dfrac{1}{a^{\sqrt{5}}}$.}
\end{ex}

\begin{ex}%[DCHT Toán 11 - KNTT -Nguyễn Thắng]%[1K6BH-3]
	Cho $a$, $b>0$ thỏa mãn $a^{^{\tfrac{1}{2}}}>a^{^{\tfrac{1}{3}}}$, $b^{^{\tfrac{2}{3}}}>b^{^{\tfrac{3}{4}}}$. Khi đó
	\choice
	{$ 0<a<1$, $0<b<1$}
	{$a>0$, $b>1$}
	{$0<a<1$, $b>1$}
	{\True $a>1$, $0<b<1$}
	\loigiai{
		\begin{itemize}
			\item $a^{^{\tfrac{1}{2}}}>a^{^{\tfrac{1}{3}}} \Rightarrow a>1$;
			\item $b^{^{\tfrac{2}{3}}}>b^{^{\tfrac{3}{4}}} \Rightarrow 0<b<1$.
		\end{itemize}	
	}
\end{ex}

\begin{ex}%[DCHT Toán 11 - KNTT -Nguyễn Thắng]%[1K6KH-3]
	Cho $U=2\cdot 2019^{2020}$, $V=2019^{2020}$, $W=2018 \cdot 2019^{2019}$, $X=5 \cdot 2019^{2019}$ và $Y=2019^{2019}$. Số nào trong các số dưới đây là số bé nhất?
	\choice
	{$X-Y$}
	{$U-V$}
	{\True $V-W$}
	{$W-X$}
	\loigiai{
		Ta có 
		\begin{eqnarray*}
			X-Y&=&4\cdot 2019^{2019}\\
			U-V&=&2019\cdot 2019^{2019}\\
			V-W&=&2019^{2019}\\
			W-X&=&2014\cdot 2019^{2019}.	
		\end{eqnarray*}
		Suy ra số nhỏ nhất là $V-W$.
	}
\end{ex}
\Closesolutionfile{ans}
\begin{indapan}{10}
	{ans/ans-1K6-1-Dang3}
\end{indapan}

\begin{dang}{Điều kiện cho luỹ thừa, căn thức.}
	\begin{enumerate}
		\item Xét hàm số $y=\sqrt[\alpha]{u}$. Khi đó:
		\begin{itemize}
			\item Nếu $\alpha$ lẻ thì hàm số xác định khi $u\in \mathbb{R}$.
			\item Nếu $\alpha$ chẵn thì hàm số xác định khi $u\geq 0$, hay $u\in \left[0; +\infty\right)$.
		\end{itemize}	
		\item Xét hàm số $y=u^\alpha$. Khi đó:
		\begin{itemize}
			\item Nếu $\alpha$ là số nguyên dương thì hàm số xác định khi $u\in \mathbb{R}$.
			\item Nếu $\alpha$ là số nguyên âm hoặc bằng 0 thì hàm số xác định khi $u\neq 0$, hay $u\in \mathbb{R}\setminus \left\{0\right\}$.
			\item Nếu $\alpha$ là số không nguyên thì hàm số xác định khi $u> 0$, hay $u\in\left(0; +\infty\right)$.
		\end{itemize}
	\end{enumerate}
\end{dang}
\subsubsection{Ví dụ mẫu}
\begin{vd}%[DCHT Toán 11 - KNTT - Vương Lam Huy]%[1K6YH-4]
	Tìm tập xác định của hàm số $y=\sqrt[3]{x^3-6x^2+2x-3}$.\dapso{$D=\mathbb{R}$.}
	\loigiai{
		Tập xác định: $D=\mathbb{R}$.}
\end{vd}
\begin{vd}%[DCHT Toán 11 - KNTT - Vương Lam Huy]%[1K6YH-4]
	Tìm tập xác định của hàm số $y=\sqrt[4]{x-4}$.\dapso{$D=\left[4; +\infty\right)$.}
	\loigiai{
		Điều kiện xác định: $x-4\geq 0 \iff x\geq 4$. Tập xác định: $D=\left[4; +\infty\right)$.}
\end{vd}
\begin{vd}%[DCHT Toán 11 - KNTT - Vương Lam Huy]%[1K6BH-4]
	Tìm tập xác định của hàm số $y=\sqrt[6]{4-x^2}$.\dapso{$D=\left[-2; 2\right]$.}
	\loigiai{
		Điều kiện xác định: $4-x^2\geq 0 \iff -2\leq x\leq 2$. Tập xác định: $D=\left[-2; 2\right]$.}
\end{vd}
\begin{vd}%[DCHT Toán 11 - KNTT - Vương Lam Huy]%[1K6YH-4]
	Tìm tập xác định của hàm số $y=\left(x^4-6 x-2\right)^{7}$.\dapso{$D=\mathbb{R}$.}
	\loigiai
	{Tập xác định: $D=\mathbb{R}$.}
\end{vd}
\begin{vd}%[DCHT Toán 11 - KNTT - Vương Lam Huy]%[1K6BH-4]
	Tìm tập xác định của hàm số $y=\left(x^2-2 x-3\right)^{-4}$.\dapso{$D=\mathbb{R}\setminus\left\{-1; 3\right\}$.}
	\loigiai
	{Điều kiện xác định: $x^2-2x-3\neq 0 \iff \left[\begin{array}{l} x\neq -1\\x\neq 3 \end{array}\right.$. Tập xác định: $D=\mathbb{R}\setminus \left\{-1; 3\right\}$.}
\end{vd}
\begin{vd}%[DCHT Toán 11 - KNTT - Vương Lam Huy]%[1K6YH-4]
	Tìm tập xác định của hàm số $y=(1-2 x)^{\sqrt{3}-1}$.\dapso{$D=\left(-\infty; \dfrac12\right)$.}
	\loigiai
	{Điều kiện xác định: $1-2x>0\iff x<\dfrac12$. Tập xác định: $D=\left(-\infty; \dfrac12\right)$.}
\end{vd}
\begin{vd}%[DCHT Toán 11 - KNTT - Vương Lam Huy]%[1K6KH-4]
	Tìm tập xác định của hàm số $y=\left(\dfrac{3-x}{x-2}\right)^{\sqrt{2}}$.\dapso{$D=\left(2; 3\right)$.}
	\loigiai{\textbf{Cách 1.}\\
		Điều kiện xác định: $
		\left\{\begin{array}{l} \dfrac{3-x}{x-2}>0 \\x-2\neq 0 \end{array}\right. \iff 		\left\{\begin{array}{l} 		\left[\begin{array}{l}  \left\{\begin{array}{l} 3-x>0 \\x-2> 0 \end{array}\right.  \\\left\{\begin{array}{l} 3-x<0 \\x-2< 0 \end{array}\right. \end{array}\right. \\x\neq 2 \end{array}\right. \iff  		\left[\begin{array}{l}  \left\{\begin{array}{l} x<3 \\x>2 \end{array}\right. \iff 2<x<3  \\ \left\{\begin{array}{l} x>3 \\ x<2 \end{array}\right. \text{(vô lý)} \end{array}\right.$. \\		
		Tập xác định: $D=\left(2; 3\right)$.\\
		\textbf{Cách 2.}\\
		Điều kiện xác định: $
		\left\{\begin{array}{l} \dfrac{3-x}{x-2}>0 \\x-2\neq 0 \end{array}\right.$. Để tìm $x$ thoả mãn $\dfrac{3-x}{x-2}>0$, ta có bảng xét dấu:
		\begin{center}
			\begin{tabular}{|c|ccccccc|}
				\hline
				$x$&$-\infty$&&$2$&&$3$&&$+\infty$ \\
				\hline
				$x-2$&&$-$&$0$&$+$&$|$&$+$&\\
				\hline
				$3-x$&&$+$&$|$&$+$&$0$&$-$&\\
				\hline
				$\dfrac{x-2}{3-x}$&&$-$&$0$&$+$&$\|$&$-$& \\[.1cm]
				\hline	
			\end{tabular}
		\end{center} \vspace{.2cm}	
		Vậy $x\in \left(2;3\right)$. Vậy tập xác định của hàm số là $D= \left(2;3\right)$.}
\end{vd}

\subsubsection{Bài tập rèn luyện}
\centerline{\fcolorbox{red}{yellow!50}{\bf {BÀI TẬP TỰ LUẬN}}}
\begin{bt}%[DCHT Toán 11 - KNTT - Vương Lam Huy]%[1K6YH-4]
	Tìm tập xác định của hàm số $y=\sqrt[3]{x^3-4x+2}$.\dapso{$D=\mathbb{R}$.}
	\loigiai{Tập xác định của hàm số là $D=\mathbb{R}$.}
\end{bt}
\begin{bt}%[DCHT Toán 11 - KNTT - Vương Lam Huy]%[1K6BH-4]
	Tìm tập xác định của hàm số $y=\sqrt[8]{x^2-5x+6}$.\dapso{$D=(-\infty;2]\cup[3; +\infty)$.}
	\loigiai{Điều kiện xác định: $x^2-5x+6\geq 0 \iff \left[\begin{array}{l}x\leq 2 \\ x\geq 3\end{array}\right.$.\\
		Vậy tập xác định của hàm số là $D=(-\infty;2]\cup[3; +\infty)$.}
\end{bt}
\begin{bt}%[DCHT Toán 11 - KNTT - Vương Lam Huy]%[1K6YH-4]
	Tìm tập xác định của hàm số $y=\left(x^2+3x+2\right)^{16}$.\dapso{$D=\mathbb{R}$.}
	\loigiai{Tập xác định của hàm số là $D=\mathbb{R}$.}
\end{bt}
\begin{bt}%[DCHT Toán 11 - KNTT - Vương Lam Huy]%[1K6YH-4]
	Tìm tập xác định của hàm số $y=(32-x^5)^{-8}$. \dapso{$D=\mathbb{R}\setminus \left\{2\right\}$.}
	\loigiai{Điều kiện xác định: $32-x^5\neq 0 \iff x\neq 2$. Vậy tập xác định của hàm số là $D=\mathbb{R}\setminus \left\{2\right\}$.}
\end{bt}
\begin{bt}%[DCHT Toán 11 - KNTT - Vương Lam Huy]%[1K6BH-4]
	Tìm tập xác định của hàm số $y=\left(\dfrac{x-3}{x+2}\right)^{\sqrt{3}-2}$. \dapso{$D= \left(-\infty;-2\right)\cup\left(3;+\infty\right)$.}
	\loigiai{Điều kiện xác định: $\left\{\begin{array}{l}\dfrac{x-3}{x+2}>0\\ x\neq -2\end{array}\right.$. Để tìm $x$ thoả mãn $\dfrac{x-3}{x+2}>0$, ta có bảng xét dấu:\\
		\begin{center}
			\begin{tabular}{|c|ccccccc|}
				\hline
				$x$&$-\infty$&&$-2$&&$3$&&$+\infty$ \\
				\hline
				$x+2$&&$-$&$0$&$+$&$|$&$+$&\\
				\hline
				$x-3$&&$-$&$|$&$-$&$0$&$+$&\\
				\hline
				$\dfrac{x-3}{x+2}$&&$+$&$\|$&$-$&$0$&$+$& \\[.1cm]
				\hline	
			\end{tabular}
		\end{center} \vspace{.2cm}	
		Vậy $x\in \left(-\infty;-2\right)\cup\left(3;+\infty\right)$. Vậy tập xác định của hàm số là $D= \left(-\infty;-2\right)\cup\left(3;+\infty\right)$.}
\end{bt}
\centerline{\fcolorbox{red}{yellow!50}{\bf {BÀI TẬP TRẮC NGHIỆM}}}
\Opensolutionfile{ans}[ans/ans-1K6-1-Dang4]
\begin{ex}%[DCHT Toán 11 - KNTT - Vương Lam Huy]%[1K6BH-4]
	Cho các mệnh đề sau: \\
	I. Tập xác định của hàm số $y=\sqrt{x}$ giống với tập xác định của hàm số $y=x^{^{\tfrac{1}{2}}}$. \\
	II. Tập xác định của hàm số $y=\sqrt[3]{x}$ là $\mathbb{R}$. \\
	III. Tập xác định của hàm số $y=x^{^{\tfrac{1}{4}}}$ là $[0 ;+\infty)$. \\
	Số mệnh đề đúng là:
	\choice
	{$0$}{\True $1$}{$2$}{$3$}
	\loigiai{I. Tập xác định của hàm số $y=\sqrt{x}$ là $[0;+\infty)$; tập xác định của hàm số $x^{^{\tfrac{1}{2}}}$ là $(0;+\infty)$. Mệnh đề I sai.\\
		II. Tập xác định của hàm số $y=\sqrt[3]{x}$ là $\mathbb{R}$. Mệnh đề II đúng. \\
		III. Tập xác định của hàm số $y=x^{^{\tfrac{1}{4}}}$ là $(0 ;+\infty)$. Mệnh đề III sai.}
\end{ex}
\begin{ex}%[DCHT Toán 11 - KNTT - Vương Lam Huy]%[1K6YH-4]
	Tìm tập xác định của hàm số $y=\sqrt[5]{x^2-4x+3}$.
	\choice
	{\True $D=\mathbb{R}$}
	{$D=\mathbb{R} \setminus\left\{1;3\right\}$}
	{$D=\left(-\infty;1\right]\cup\left[3;+\infty\right)$}
	{$D=\left[1;3\right]$}
	\loigiai{Tập xác định của hàm số là $D=\mathbb{R}$.}
\end{ex}
\begin{ex}%[DCHT Toán 11 - KNTT - Vương Lam Huy]%[1K6KH-4]
	Tìm tập xác định của hàm số $y=\sqrt[4]{x^3-4x^2+x+6}$.
	\choice
	{$D=\mathbb{R}$}
	{$D=\mathbb{R}\setminus\left\{-1;2;3\right\}$}
	{\True $D=[-1;2]\cup[3; +\infty)$}
	{$D=(-1;2)\cup(3; +\infty)$}
	\loigiai{Điều kiện xác định: $x^3-4x^2+x+6\geq 0 \iff (x + 1)(x - 2)(x - 3)\geq 0$. Ta có bảng xét dấu:
		\begin{center}
			\begin{tabular}{|c|ccccccccc|}
				\hline
				$x$&$-\infty$&&$-1$&&$2$&&$3$&&$+\infty$ \\
				\hline
				$x+1$&&$-$&$0$&$+$&$|$&$+$&$|$&$+$&\\
				\hline
				$x-2$&&$-$&$|$&$-$&$0$&$+$&$|$&$+$&\\
				\hline
				$x-3$&&$-$&$|$&$-$&$|$&$-$&$0$&$+$&\\
				\hline
				$(x + 1)(x - 2)(x - 3)$&&$-$&$0$&$+$&$0$&$-$&$0$&$+$& \\
				\hline	
			\end{tabular}
		\end{center} \vspace{.2cm}
		Vậy tập xác định của hàm số là $D=[-1;2]\cup[3; +\infty)$.}
\end{ex}
\begin{ex}%[DCHT Toán 11 - KNTT - Vương Lam Huy]%[1K6YH-4]
	Tập xác định của hàm số $x^{\sqrt{2}}$ là:
	\choice
	{$D=\mathbb{R}$}
	{\True $D=(0; +\infty)$}
	{$D=[0; +\infty)$}
	{$D=\mathbb{R} \setminus\{0\}$}
	\loigiai{Điều kiện xác định: $x>0$. Tập xác định: $D=(0; +\infty)$.}
\end{ex}
\begin{ex}%[DCHT Toán 11 - KNTT - Vương Lam Huy]%[1K6YH-4]
	Tập xác định của hàm số $y=(x-1)^{\sqrt{3}}$ là:
	\choice
	{$D=\mathbb{R} \setminus \{1\}$}
	{$D=\mathbb{R}$}
	{\True $D=(1; +\infty)$}
	{$D=(-1; +\infty)$}
	\loigiai{Điều kiện xác định: $x-1>0\iff x>1$. Tập xác định: $D=\left(1; +\infty\right)$.}
\end{ex}
\begin{ex}%[DCHT Toán 11 - KNTT - Vương Lam Huy]%[1K6BH-4]
	Tập xác định của hàm số $y=\left(x^2-1\right)^{-3}$ là
	\choice
	{$D=(-\infty; -1) \cup(1; +\infty)$}
	{$D=(1; +\infty)$}
	{\True $D=\mathbb{R} \setminus \{-1; 1\}$}
	{$D=(-\infty,-1)$}
	\loigiai{Điều kiện xác định: $x^2-1\neq 0\iff x\neq \pm 1$. Tập xác định: $D=\mathbb{R}\setminus \left\{-1; 1\right\}$.}
\end{ex}
\begin{ex}%[DCHT Toán 11 - KNTT - Vương Lam Huy]%[1K6BH-4]
	Tập xác định của hàm số $y=\left[x^2(x+1)\right]^{\sqrt{\pi}}$ là
	\choice
	{$D=(0; +\infty)$}
	{\True $D=(-1; +\infty) \setminus\{0\}$}
	{$D=(-\infty; +\infty)$}
	{$D=(-1; +\infty)$}
	\loigiai{Điều kiện xác định: $x^2(x+1)>0\iff -1<x\neq 0$. Tập xác định: $D=(-1; +\infty) \setminus\{0\}$.}
\end{ex}
\begin{ex}%[DCHT Toán 11 - KNTT - Vương Lam Huy]%[1K6BH-4]
	Tìm $x$ để biểu thức $\left(x^2+2 x+1\right)^{^{-\tfrac{1}{3}}}$ có nghĩa
	\choice
	{$x \in \mathbb{R}$}
	{\True $x \neq -1$}
	{$x<-1$}
	{$x>-1$}
	\loigiai{Điều kiện xác định: $x^2+2x+1>0\iff (x+1)^2>0\iff x\neq -1$. Vậy $x\neq -1$.}
\end{ex}
\begin{ex}%[DCHT Toán 11 - KNTT - Vương Lam Huy]%[1K6BH-4]
	Tập xác định của hàm số $y=\left(-x^2+5 x-4\right)^{^{\tfrac{5}{3}}}$ là
	\choice
	{$D=(-\infty; 1) \cup(4; +\infty)$}
	{$D=[1; 4]$}
	{\True $D=(1; 4)$}
	{$D=\mathbb{R}$}
	\loigiai{Điều kiện xác định: $-x^2+5x-4>0\iff 1<x<4$. Tập xác định: $D=(1; 4)$.} 
\end{ex}
\begin{ex}%[DCHT Toán 11 - KNTT - Vương Lam Huy]%[1K6KH-4]
	Tập xác định của hàm số $y=\left(\dfrac{x^2+2x-3}{x^2-x-2}\right)^{^{\tfrac{3}{5}}}$ là
	\choice
	{\True $D=(-\infty; -3) \cup(-1; 1) \cup(2; +\infty)$}
	{$D=(-\infty; -3] \cup(-1; 1] \cup(2; +\infty)$}
	{$D=(-\infty; -3] \cup[-1; 1] \cup[2; +\infty)$}
	{$D=(-\infty; -3) \cup[-1; 1] \cup(2; +\infty)$}
	\loigiai{Điều kiện xác định: $\left\{\begin{array}{l}\dfrac{x^2+2x-3}{x^2-x-2}>0 \\ x^2-x-2\neq 0 \end{array}\right.\iff \left\{\begin{array}{l} \dfrac{(x-1)(x+3)}{(x+1)(x-2)}>0 \\ (x+1)(x-2)\neq 0 \end{array}\right.\iff \left\{\begin{array}{l} \dfrac{(x-1)(x+3)}{(x+1)(x-2)}>0 \\ x\neq -1 \\ x\neq 2 \end{array}\right.$. \\
		Để tìm $x$ thoả mãn $\dfrac{(x-1)(x+3)}{(x+1)(x-2)}>0$, ta có bảng xét dấu:
		\begin{center}
			\begin{tabular}{|c|ccccccccccc|}
				\hline
				$x$&$-\infty$&&$-3$&&$-1$&&$1$&&$2$&&$+\infty$ \\
				\hline
				$x+3$&&$-$&$0$&$+$&$|$&$+$&$|$&$+$&$|$&$+$&\\
				\hline
				$x+1$&&$-$&$|$&$-$&$0$&$+$&$|$&$+$&$|$&$+$&\\
				\hline
				$x-1$&&$-$&$|$&$-$&$|$&$-$&$0$&$+$&$|$&$+$&\\
				\hline
				$x-2$&&$-$&$|$&$-$&$|$&$-$&$|$&$-$&$0$&$+$&\\
				\hline
				$\dfrac{(x-1)(x+3)}{(x+1)(x-2)}$&&$+$&$0$&$-$&$\|$&$+$&$0$&$-$&$\|$&$+$& \\
				\hline	
			\end{tabular}
		\end{center} \vspace{.2cm}
		Vậy $\left\{ \begin{array}{l} x\in(-\infty; -3) \cup(-1; 1) \cup(2; +\infty) \\ x\neq -1 \\ x\neq 2 \end{array}\right.\iff x \in (-\infty; -3) \cup(-1; 1) \cup(2; +\infty)$.}
\end{ex}

\Closesolutionfile{ans}
% \begin{indapan}{10}
% 	{ans/ans-1K6-1-Dang4}
% \end{indapan}
