%\chapter{Hàm số  lượng giác và phương trình lượng giác}
%\setcounter{section}{18}
\section{Lôgarit}
\subsection{Tóm tắt lý thuyết}
\begin{tomtat}
\subsubsection{Khái niệm lôgarit}
Cho $a$ là một số thực dương khác 1 và $M$ là một số thực dương. Số thực $\alpha$ để $a^\alpha=M$ được gọi là lôgarit cơ số $a$ của $M$ và ký hiệu là $\log_aM$. Khi đó: $\alpha=\log_aM\iff a^\alpha=M$.

\begin{note}
Không có lôgarit của số âm và số 0. Cơ số của lôgarit phải dương và khác 1.
\end{note}
	
\subsubsection{Tính chất của lôgarit}
Giả sử $0<a\neq 1$, $0<b\neq 1$, $M,N>0$ và $\alpha$ là số thực tuỳ ý. Khi đó
\begin{itemize}
\item $\log_a1=0; \log_aa=1; a^{\log_aM}=M;\log_aa^\alpha=\alpha.$

\item $ \log_a(MN)=\log_aM+\log_aN;$

$\log_a\left(\dfrac{M}{N}\right)=\log_aM-\log_aN;$

$\log_aM^\alpha=\alpha\log_aM.$

\item $\log_aM=\dfrac{\log_bM}{\log_ba}$
\end{itemize}
\subsubsection{Lôgarit thập phân và lôgarit tự nhiên}
\begin{itemize}
\item Lôgarit cơ số 10 của một số dương $M$ gọi là lôgarit thập phân của $M$, ký hiệu là $\log M$ hoặc $\lg M$ (đọc là lốc của $M$).

\item Lôgarit cơ số $e$ của một số dương $M$ gọi là lôgarit tự nhiên của $M$, ký hiệu là $\ln M$ (đọc là lôgarit Nêpe của $M$).
\end{itemize}

\subsubsection{Công thức lãi kép liên tục}
\begin{itemize}
    \item Thể thức lãi kép liên tục là thể thức tính lãi kép với lãi suất hàng năm không đổi là $r$, chia mỗi năm thành $m$ kì tính lãi, với $m\to +\infty$.
    \item Số tiền thu được cả vốn lẫn lãi sẽ là $A=Pe^{tr}$, với $t$ là số năm.
\end{itemize}
\end{tomtat}
\subsection{Các dạng toán thường gặp}

%%Thiếu dạng 1 Tính giá trị biểu thức, dạng 2 biến đổi - rút gọn
%Dạng 4 thực tế liên môn
\begin{dang}{Bài toán thực tế, liên môn}
	Áp dụng công thức lãi kép, lãi kép liên tục, các công thức đề bài cho.
\end{dang}
\subsubsection{Ví dụ minh hoạ}
\begin{vd} %bttl [1K6BI-3]
	Biết rằng khi độ cao tăng lên, áp suất không khí sẽ giảm và công thức áp suất dựa trên độ cao là
	$a=15500 \cdot (5-\log p)$
	trong đó $a$ là độ cao so với mực nước biển (tính bằng mét) và $p$ là áp suất không khí (tính bằng pascal).
	Tính áp suất không khí ở đỉnh Everest có độ cao $8850$ $m$ so với mực nước biển.
	\loigiai{
		Ta có 	
		\begin{eqnarray*}
			a=15500 \cdot (5-\log p) 
			\Leftrightarrow 8850=15500 \cdot (5-\log p) 
			\Leftrightarrow 5-\log p = \dfrac{8850}{15500}
			\Leftrightarrow \log p \approx 4 {,}429
			\Leftrightarrow p \approx 10^{4 {,}429}\\
			\Leftrightarrow p \approx 26855{,}44.
		\end{eqnarray*}
		Vậy áp suất không khí ở đỉnh Everest có độ cao $8850$ $m$ so với mực nước biển là $26855{,}44$ pascal.
	}
\end{vd}
\begin{vd}  %[1K6BI-3]
	Mức cường độ âm $L$ đo bằng decibel $(\mathrm{dB})$ của âm thanh có cường độ I (đo bằng \textit{oát trên mét vuông}, kí hiệu là $W / m^2$ ) được định nghĩa như sau:
	$L(I)=10 \log \dfrac{I}{I_0}$
	trong đó $I_0=10^{-12} \mathrm{W}/\mathrm{m}^2$ là cường độ âm thanh nhỏ nhất mà tai người có thể phát hiện được (gọi là \textit{ngưỡng nghe}).\\
	Xác định mức cường độ âm của mỗi âm sau:
	\begin{itemize}
		\item Cuộc trò chuyện bình thường có cường độ $I=10^{-7} \,\mathrm{W}/ \mathrm{m}^2$.
		\item Giao thông thành phố đông đúc có cường độ $\mathrm{I}=10^{-3} \, \mathrm{W}/\mathrm{m}^2$.
	\end{itemize}
	\loigiai
	{\begin{itemize}
			\item	Ta có cuộc trò chuyện bình thường có cường độ âm 
			\begin{equation*}
				L(I)=10 \cdot \log \dfrac{I}{{I_0}}=10 \cdot \log \dfrac{{10^{-7}}}{{10^{-12}}}=50{(\mathrm{dB})}
			\end{equation*}
			\item Ta có giao thông thành phố đông đúc có cường độ âm
			\begin{equation*} L(I)=10\log \dfrac{I}{{I_0}}=10 \cdot \log \dfrac{10^{-3}}{10^{-12}}=90{(\mathrm{dB})}
			\end{equation*}
	\end{itemize}}
\end{vd}
\begin{vd} %bttl  [1K6YI-3]
	Biết rằng khi độ cao tăng lên, áp suất không khí sẽ giảm và công thức áp suất dựa trên độ cao là
	$a=15500 \cdot (5-\log p)$
	trong đó $a$ là độ cao so với mực nước biển (tính bằng mét) và $p$ là áp suất không khí (tính bằng pascal).
	Tính độ cao so với mực nước biển biết áp suất không khí ở nơi đang đứng là $2000$ pascal.
	\loigiai{
		Ta có 	
		\begin{eqnarray*}
			a=15500 \cdot (5-\log p) 
			\Leftrightarrow a=15500 \cdot (5-\log 2000) 
			\Leftrightarrow a \approx 26334{,}04 \, \mathrm{m}.
	\end{eqnarray*}}
\end{vd}
\begin{vd} %[1K6YI-3]
	Bác An gửi tiết kiệm ngân hàng 100 triệu đồng kì hạn 12 tháng, với lãi suất không đổi là $6 \%$ một năm. Khi đó sau $n$ năm gửi thì tổng số tiền bác An thu được (cả vốn lẫn lãi) cho bởi công thức sau:
	$$
	A=100 \cdot(1+0,06)^n \text { (triệu đồng). }
	$$
	Hỏi sau ít nhất bao nhiêu năm, tổng số tiền bác An thu được là không dưới $150$ triệu đồng?
	\loigiai{
		Ta có $A=100 \cdot (1+0{,}06)^n=100\cdot 1{,}06^n$.\\
		Với $A=150$, ta có $100 \cdot 1{,}06^n=150 \Leftrightarrow 1{,}06^n=1{,}5 \Leftrightarrow n \approx 6{,}96.$\\
		Vì gởi tiết kiệm kỳ hạn $12$ tháng nên $n$ phải là số nguyên. Do đó chọn $n=7$.\\
		Vậy sau ít nhất $7$ năm thì bác An được nhận số tiền ít nhất là $150$ triệu đồng.
	}
\end{vd}

\subsubsection{Bài tập trắc nghiệm}
\Opensolutionfile{ans}[ans/ans-1K6-2-Dang3]
\begin{ex}
	Cô Lan gởi $180$ triệu vào ngân hàng với lãi suất $6\%$ một năm. Tính số tiền cô Lan nhận được cả vốn lẫn lãi sau $2$ năm.
	\choice 
	{ $200$ triệu đồng}
	{ $210$ triệu đồng}
	{\True $220{,}95$ triệu đồng}
	{$230$ triệu đồng}
	\loigiai 
	{Ta có công thức $A=180 \cdot e^{2 \cdot 0.06} \approx 220{,}95$ triệu đồng.}
\end{ex}	
\begin{ex} %[1K6YI-3]
	Bác Bình gửi tiết kiệm ngân hàng $200$ triệu đồng kì hạn $12$ tháng, với lãi suất không đổi là $6 \%$ một năm. Khi đó sau $n$ năm gửi thì tổng số tiền bác An thu được (cả vốn lẫn lãi) cho bởi công thức sau:
	$$
	A=200 \cdot(1+0,06)^n \text { (triệu đồng). }
	$$
	Tính tổng số tiền các vốn lẫn lãi bác Bình thu được sau khi gởi tiết kiệm $3$ năm gần nhất với số tiền nào sau đây
	\choice 
	{ $220$ triệu đồng}
	{\True $238$ triệu đồng}
	{ $250$ triệu đồng}
	{$260$ triệu đồng}
	\loigiai{
		Ta có sau $3$ năm thì bác An được nhận số tiền\\
		$A=200 \cdot (1+0{,}06)^n=200\cdot 1{,}06^3 \approx 238{,}2$ triệu đồng.\\
		Vậy tổng số tiền các vốn lẫn lãi bác Bình thu được sau khi gởi tiết kiệm $3$ năm gần nhất với $238$ triệu đồng.
	}
\end{ex}
\begin{ex} % Câu 1. %[1K6KI-3]	
	Biết rằng năm $2009$ dân số Việt Nam là $85{,}847{,}000$ người và tỉ lệ tăng dân số năm đó là $1{,}2\%$ cho biết sự tăng dân số được tuân theo công thức $S=A \cdot e^{Nr}$ ($A$ là dân số năm lấy làm mốc tính, $S$ là dân số sau $N$ năm, $r$ tỉ lệ tăng dân số hằng năm). Nếu cứ tăng dân số với tỉ lệ như vậy thì sau bao nhiêu năm nữa dân số nước ở mức $120$ triệu người
	\choice { $26$ năm} { $27$ năm} {\True $28$ năm} { $29$ năm}
	\loigiai{
		Ta có $S=A \cdot e^{Nr}$ $\Leftrightarrow 120{,}000{,}000=85{,}847{,}000 \cdot {e^{N \cdot 1{,}2\%}}\Leftrightarrow N\approx 28$ năm.\\}
\end{ex}
\begin{ex} % Câu 2 	%[1K6KI-3]	
	Dân số thế giới được dự đoán theo công thức $P(t)=a \cdot e^{bt}$, trong đó $ a$,$ b$ là các hằng số, $ t$ là năm tính dân số. Theo số liệu thực tế, dân số thế giới năm $1950$ là $2560$ triệu người; dân số thế giới năm $1980$ là $3040$ triệu người. Hãy dự đoán dân số thế giới năm $2020$?
	\choice {\True $3823$ triệu} { $5360$ triệu} { $3954$ triệu} { $4017$ triệu}
	\loigiai{
		Từ giả thiết ta có hệ phương trình: 
		\begin{eqnarray}
			\heva{& a\cdot {{e}^{1950b}}=2560 \\ 
				& a\cdot {{e}^{1980b}}=3040.}\\
		\end{eqnarray}
		Chia $(2)$ cho $(1)$ ta được ${{e}^{30b}}=\dfrac{19}{16}
		\Leftrightarrow 30b=\ln \dfrac{19}{16}\Leftrightarrow b=\dfrac{1}{30}\ln \dfrac{19}{16}$.\\
		Thay vào $( 1 )$ ta được: $ a=\dfrac{2560}{{{\left( \dfrac{19}{16} \right)}^{65}}}$.\\
		Vậy $P( 2020 )=\dfrac{2560}{{{\left( \dfrac{19}{16} \right)}^{65}}}\cdot {{e}^{2020\cdot \tfrac{1}{30}\cdot\ln \tfrac{19}{16}}}\approx 3823$ (triệu).\\}
\end{ex}
\begin{ex} % Câu 3 %[1K6KI-3]	
	Sự tăng dân số được ước tính theo công thức ${P}_n={{P}_0} \cdot {{e}^{n\cdot r}}$, trong đó ${\mathrm{P}_0}$ là dân số của năm lấy làm mốc tính, ${\mathrm{P}_n}$ là dân số sau $ n$ năm, $r$ là tỉ lệ tăng dân số hàng năm. Biết rằng năm $2001,$ dân số Việt Nam là $78{,}685{,}800$ triệu và tỉ lệ tăng dân số năm đó là $1{,}7\%$. Hỏi cứ tăng dân số với tỉ lệ như vậy thì đến năm nào dân số nước ta ở mức $100$ triệu người?
	\choice 
	{$2018$}
	{$2017$}
	{$2015$} 
	{\True $2016$}
	\loigiai{
		${\mathrm{P}_n}={\mathrm{P}_0}{{e}^{n\cdot r}}$
		$\Leftrightarrow 100000000=78685800 \cdot {{e}^{n\cdot 1{,}7\%}}$
		$\Leftrightarrow n=\dfrac{\ln \dfrac{1000000}{786{,}858}}{1{,}7\%}\approx 14{,}1$\\
		Sau $15$ năm thì dân số nước ta ở mức $100$ triệu người.\\
		Do đó năm $2016$ dân số nước ta ở mức $100$ triệu người.\\}
\end{ex}
\begin{ex} % Câu 4 %[1K6KI-3]	
	Sự tăng trưởng của loại vi khuẩn tuân theo công thức $S={A \cdot {e^{rt}}}$, trong đó $A$ là số lượng vi khuẩn ban đầu, $ r$ là tỉ lệ tăng trưởng $( r>0 )$, $ t$ là thời gian tăng trưởng (tính theo đơn vị là giờ). Biết số vi khuẩn ban đầu là $100$ con và sau $5$ giờ có $300$ con. Thời gian để vi khuẩn tăng gấp đôi số ban đầu gần đúng nhất với kết quả nào trong các kết quả sau đây.
	\choice 
	{ $3$ giờ $2$ phút}
	{ $3$ giờ $20$ phút}
	{ $3$ giờ $40$ phút}
	{\True $3$ giờ $9$ phút}
	\loigiai{
		Ta có: $300=100 \cdot {e^{5r}}\Leftrightarrow e^{5r}=3\Leftrightarrow 5r=\ln 3\Leftrightarrow r=\dfrac{\ln 3}{5}$.\\
		Gọi thời gian cần tìm là $ t$.\\
		Theo yêu cầu bài toán, ta có: $200=100\cdot {e^{rt}} \Leftrightarrow {e^{rt}}=2$
		$\Leftrightarrow rt=\ln 2\Leftrightarrow t=\dfrac{5\cdot \ln 2}{\ln 3}\approx 3{,}15( h )$.\\
		Vậy $ t=$$3$ giờ $9$ phút.\\}
\end{ex}

\begin{ex} % Câu 5	%[1K6KI-3]	
	Sự tăng trưởng của một loại vi khuẩn tuân theo công thức $S=A\cdot{e^{rt}}$, trong đó $A$ là số lượng vi khuẩn ban đầu, $ r$ là tỉ lệ tăng trưởng, $t$ là thời gian tăng trưởng. Biết rằng số lượng vi khuẩn ban đầu là $300$ con và sau $5$ giờ có $300$ con. Hỏi số con vi khuẩn sau $10$ giờ
	\choice {\True $900$} { $1000$} { $800$} { $850$}
	\loigiai{
		Trước tiên, ta tìm tỉ lệ tăng trưởng mỗi giờ của loại vi khuẩn này.\\
		Từ giả thiết ta có: 
		$300=100 \cdot {{e}^{5r}}
		\Leftrightarrow{r}=\dfrac{ln300-ln100}{5}=\dfrac{ln3}{5}\approx 0{,}219$.\\
		Tức tỉ lệ tăng trưởng của loại vi khuẩn này là $21{,}97\%$ mỗi giờ.\\
		Sau $10$ giờ, từ $100$ con vi khuẩn sẽ có $100\cdot {e^{10\cdot 0{,}2197}}\approx 900$ con.
	}
\end{ex}

\begin{ex} % Câu 6 	%[1K6KI-3]	
	Chu kỳ bán rã của chất phóng xạ Polutolium $P{u^{239}}$ là $24360$ năm. Sự phân hủy này được tính theo công thức $S={A{e^{-rt}}}$, trong đó $A$ là lượng chất phóng xạ ban đầu, $ r$ là tỷ lệ phân hủy hàng năm, $ t$ là thời gian phân hủy, $S$ là lượng còn lại sau thời gian phân hủy $ t$. Hỏi $20 $gam $P{u^{239}}$ sau ít nhất bao nhiêu năm thì còn lại $4$ gam?
	\choice {\True $56563$ năm} { $56562$ năm} { $65664$ năm} { $56561$ năm}
	\loigiai{
		Ta có $10=20 \cdot {e^{r \cdot 24360}}\Rightarrow r=\dfrac{\ln 0{,}5}{24360}$.\\
		Khi đó $4=20\cdot {e^{r\cdot t'}}\Rightarrow t'=\dfrac{\ln 0{,}2}{r}$\\
		$ t'=\dfrac{\ln 0{,}2}{\ln 0{,}5}\cdot 24360=56562{,}16$.\\
		Khi đó sau ít nhất $56563$ năm thì thỏa mãn. }
\end{ex}
\begin{ex} % Câu 7	%[1K6KI-3]	
	Áp suất không khí $P$ (đo bằng milimet thủy ngân, kí hiệu $mmHg$) theo công thức $P={{P}_0}\cdot {e^{kx}}$ $(mmHg)$, trong đó $x$ là độ cao (đo bằng mét), ${P}_0=760$ $( mmHg)$ là áp suất không khí ở mức nước biển $( x=0 )$, $ k$ là hệ số suy giảm. Biết rằng ở độ cao $1000$ $m$ thì áp suất không khí là $672{,}71$  $( mmHg)$. Tính áp suất của không khí ở độ cao $3000$ $m$.
	\choice {\True $527{,}06$ $(mmHg)$} { $530{,}23$$( mmHg$}
	{ $530{,}73$$( mmHg)$} { $545{,}01$$( mmHg)$}
	\loigiai{
		Ở độ cao $1000$ $m$ áp suất không khí là $672{,}71$ $( mmHg)$.\\
		Nên ta có: $672{,}71=760\cdot {e^{1000k}}$\\
		$\Leftrightarrow {e^{1000k}}=\dfrac{672{,}71}{760}$\\
		$\Leftrightarrow k=\dfrac{1}{1000}\ln \dfrac{672{,}71}{760}$.\\
		Áp suất ở độ cao $3000$ $m$ là: $P=760 \cdot {e^{3000k}}$$=760{e^{3000\cdot \tfrac{1}{1000}\ln \tfrac{672{,}71}{760}}}$$\approx 527{,}06$ $(mmHg)$.\\
	}
\end{ex}
\begin{ex} % Câu 8 	%[1K6KI-3]	
	Khi ánh sáng đi qua một môi trường cường độ sẽ giảm dần theo quãng đường truyền $ x$, theo công thức $I( x )={I_0} \cdot {e^{-\mu x}}$, trong đó ${I_0}$ là cường độ của ánh sáng khi bắt đầu truyền vào môi trường và $\mu $ là hệ số hấp thu của môi trường đó. Biết rằng nước biển có hệ số hấp thu $\mu =1{,}4$ và người ta tính được rằng khi đi từ độ sâu $2m$ xuống đến độ sâu $20m$ thì cường độ ánh sáng giảm $ l{{\cdot 10}^{10}}$ lần. Số nguyên nào sau đây gần với $ l$ nhất?
	\choice { $8$} {\True $9$} { $90$} { $10$}
	\loigiai{
		Ta có ở độ sâu $2m$: $I(2)={I_0}{e^{-2{,}8}}$.\\
		Ở độ sâu $20m$: $I(20)={I_0}{e^{-28}}$.\\
		Theo giả thiết $I(20)=l{{\cdot 10}^{10}}\cdot I(2)$ $\Leftrightarrow $ ${e^{-28}}=l{{\cdot 10}^{10}}\cdot {e^{-2{,}8}}$.\\
		$\Leftrightarrow $ $ l={10^{-10}}\cdot {e^{25{,}2}}\approx 8{,}79$.\\
		Vậy $9$ là số nguyên gần với $l$ nhất.}
\end{ex}
\begin{ex} % Câu 9. 	%[1K6KI-3]	
	Một điện thoại đang nạp pin, dung lượng pin nạp được tính theo công thức $Q\left( t \right)={Q_0}.\left( 1-{e^{-t\sqrt {2}}} \right)$ với $ t$ là khoảng thời gian tính bằng giờ và ${Q_0}$ là dung lượng nạp tối đa (pin đầy). Hãy tính thời gian nạp pin của điện thoại tính từ lúc cạn hết pin cho đến khi điện thoại đạt được $90\%$ dung lượng pin tối đa (kết quả được làm tròn đến hàng phần trăm).
	\choice { $ t\approx 1{,}65$ giờ} { $ t\approx 1{,}61$ giờ} {\True $ t\approx 1{,}63$ giờ} { $ t\approx 1{,}50$ giờ}
	\loigiai
	{Ta có\\
		${Q_0}\cdot \left( 1-{e^{-t\sqrt {2}}} \right)=0{,}9\cdot {Q_0}\Leftrightarrow 1-{e^{-t\sqrt {2}}}=0{,}9\Leftrightarrow {e^{-t\sqrt {2}}}=0{,}1\Leftrightarrow t=-\dfrac{\ln ( 0{,}1 )}{\sqrt {2}}\approx 1{,}63$.}
	
\end{ex}
\begin{ex} % Câu 1. %[1K6KI-3]	
	Giá trị còn lại của một chiếc xe theo thời gian khấu hao $t$ được xác định bởi công thức: $V(t)=15000 \cdot {{e}^{-0{,}15}}$ trong đó $V(t)$  được tính bằng USD và $t$  được tính bằng năm. Hỏi sau bao lâu giá trị còn lại của chiếc xe chỉ là $5000$ USD gần nhất với số nào sau đây?
	\choice {\True $7{,}3$ năm} { $9{,}3$  năm} { $6{,}3$  năm} { $8{,}3$  năm}
	\loigiai{
		Ta có: $V( t )=15000 \cdot {e^{-0{,}15t}}\Leftrightarrow {e^{-0{,}15t}}=\dfrac{V( t )}{15000}
		\Leftrightarrow -0{,}15t=\ln \left( \dfrac{V( t )}{15000} \right)
		\Leftrightarrow t=-\dfrac{20}{3}\ln \left( \dfrac{V(t)}{15000} \right)$.\\
		Thay $V(t)=5000$ ta được $ t=-\dfrac{20}{3}\ln \left(\dfrac{5000}{15000} \right)\approx 7{,}324$ năm.\\
	}
\end{ex}
\Closesolutionfile{ans}
% \begin{indapan}{10}
% 	{ans/ans-1K6-2-Dang3}
% \end{indapan}