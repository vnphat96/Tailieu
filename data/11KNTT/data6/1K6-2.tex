\section{Phép tính logarit}
\subsection{Tóm tắt lý thuyết}
\begin{tomtat}
	\subsubsection{Định nghĩa}
	\begin{dn}
		Cho hai số thực dương $a, b$ với $a$ khác 1. Số thực $c$ để $a^{c}=b$ được gọi là lôgarit cơ số $a$ của $b$ và kí hiệu là $\log _{a} b$, nghĩa là
		$$c=\log _{a} b \Leftrightarrow a^{c}=b.$$
	\end{dn}
	\begin{note}
		$\log _{a} b$ xác định khi và chỉ khi $a>0$, $a \neq 1$ và $b>0$.
	\end{note}
	\subsubsection{Tính chất}
	\begin{tc}
		Với số thực dương $a$ khác 1, số thực dương $b$, ta có:
		$$\log _{a} 1=0 ; \quad \log _{a} a=1 ; \quad \log _{a} a^{c}=c ; \quad a^{\log _{a} b}=b.$$
	\end{tc}
	\begin{tc}
		Với ba số thực dương $a, m, n$ và $a \neq 1$, ta có:
		\begin{itemize}
			\item $\log _{a}(m n)=\log _{a} m+\log _{a} n$;
			\item $\log _{a}\left(\dfrac{m}{n}\right)=\log _{a} m-\log _{a} n$. 
		\end{itemize}
	\end{tc}
	\begin{note}
		Ta có:
		$$\log _{a}\left(\frac{1}{b}\right)=-\log _{a} b\; (a>0, a \neq 1, b>0).$$
	\end{note}
	\begin{tc}
		Cho $a>0, a \neq 1, b>0$. Với mọi số thực $\alpha$, ta có:
		$$	\log _a b^\alpha=\alpha \log _a b.	$$
	\end{tc}
	\begin{note}
		Cho $a>0, a \neq 1, b>0$. Với mọi số nguyên dương $n \geq 2$, ta có: $	\log _a \sqrt[n]{b}=\dfrac{1}{n} \log _a b.$
	\end{note}
	\begin{tc}
		Với $a, c$ là hai số thực dương khác 1 và $b$ là số thực dương, ta có:
		$$\log _{a} b=\dfrac{\log _{c} b}{\log _{c} a}.$$
	\end{tc}
	\begin{nx}
		Với $a>0$ và $a \neq 1, b>0$ và $b \neq 1, c>0, \alpha \neq 0$, ta có những công thức sau:
		\begin{itemize}
			\item $\log _{a} b \cdot \log _{b} c=\log _{a} c$;
			\item $\log _{a} b=\dfrac{1}{\log _{b} a}$;
			\item $\log _{a^{\alpha}} b=\dfrac{1}{\alpha} \log _{a} b$.
		\end{itemize}
	\end{nx}
	\subsubsection{Lôgarit thập phân. Lôgarit tự nhiên}
	\begin{itemize}
		\item Lôgarit cơ số 10 của số thực dương $b$ được gọi là lôgarit thập phân của $b$ và kí hiệu là $\log b$ hay $\lg b$.
		\item Lôgarit cơ số $\mathrm{e}$ của số thực dương $b$ được gọi là lôgarit tự nhiên của $b$ và kí hiệu là $\ln b$.
	\end{itemize}
	\subsubsection{Tính lôgarit bằng máy tính cầm tay}
	
%	\begin{center}
%		\renewcommand{\arraystretch}{2}
%		\begin{tabular}{|c|c|c|c|}
%			\hline 
%			Tính   & Bấm phím & Màn hình hiện & Kết quả \\ 
%			\hline 
%			$\log 6{,}52$ & \logk \onek  \zerok \rightk \sixk \dotk \fivek \twok \equalk &  $0{.}8142475957$ & $\log 6{,}52 \approx 0{,}8142$ \\ 
%			\hline 
%			$\ln 6{,}52$ & \ln k \sixk \dotk \fivek \twok \equalk & $1{.}874874376$ & $\ln 6{,}52 \approx 1{,}8749$ \\ 
%			\hline 
%			$\log _{14} 17$ & \logk \onek \fourk \rightk  \onek  \sevenk \equalk & $1{.}073570215$ & $\log _{14} 17 \approx 1{,}0736$ \\ 
%			\hline 
%		\end{tabular} \\
%		(làm tròn kết quả đến chữ số thập phân thứ tư)
%	\end{center}
%	
%	\begin{note}
%		Với máy tính không có phím \key{i} thì để tính $\log _{5} 3$, ta có thể dùng công thức đổi cơ số để đưa về cơ số $10$ hoặc cơ số $\mathrm{e}$. 
%	\end{note}
\end{tomtat}
\setcounter{subsubsection}{0}
\setcounter{ex}{0}
\setcounter{bt}{0}
\subsection{Các dạng toán thường gặp}
\begin{dang}{Tính giá trị biểu thức chứa lôgarít}
\end{dang}
\subsubsection{Ví dụ minh hoạ}
\begin{vd}
	Cho $\log a=4$. Tính giá trị của biểu thức $P=\log (100a^2)$.
	\loigiai{
		Ta có $P=\log 100+\log a^2=2+2\log a=2+4\cdot 2=10$.
	}
\end{vd}
\begin{vd}
	Cho $\log_ab=2$. Tính $\log_a(a^2b)$.
	\loigiai
	{
		Ta có $\log_a(a^2b)=\log_aa^2+\log_ab=2+2=4$.
	}
\end{vd}
\begin{vd}
	Cho $a$ và $b$ là hai số thực dương thỏa mãn $a^3 b^2=32$. Tính giá trị của biểu thức $P=3\log _{2} a+2 \log _{2} b$.
	\loigiai
	{
		Ta xét
		$$a^3b^2=32\Leftrightarrow \log_2(a^3b^2)=\log_232\Leftrightarrow\log_2a^3+\log_2b^2=5\Leftrightarrow 3\log_2a+2\log_2b=5.$$
		Vậy $P=5$.
	}
\end{vd}
\begin{vd}
	Cho $\log_ab=2$, $\log_ac=3$. Tính $Q=\log_a\left(b^2c\right)$.
	\loigiai{
		Ta có $Q=\log_a\left(b^2c\right)=\log_ab^2+\log_ac=2\log_ab+\log_ac=2\cdot2+3=7$.
	}
\end{vd}
\begin{vd}
	Cho $a$ là số thực dương khác $5.$ Tính $I=\log_{\tfrac{a}{5}}\left(\dfrac{a^{3}}{125}\right)$.
	\loigiai{
		Ta có $I=\log _{\tfrac{a}{5}}\left[\left(\dfrac{a}{5}\right)^3\right]=3$.
	}
\end{vd}
\begin{vd}
	Cho $a$, $b$ là hai số thực dương thỏa mãn $a b^{3}=8$. Tính giá trị của $\log _2 a+3 \log _2 b$.
	\loigiai{
		Ta có		$a b^{3}=8\Leftrightarrow\log_2\left(ab^3 \right)=\log_28\Leftrightarrow  \log _2 a+3 \log _2 b=3$	.
	}
\end{vd}
\begin{vd}
	Cho $a>0$ và đặt $\log_2 a=x$. Tính $\log_8 (4a^3)$ theo $x$.
	\loigiai{
		Ta có $\log_8 (4a^3)=\log_8 4+\log_8 a^3=\dfrac{2}{3}+\log_2 a=\dfrac{2}{3}+x$.
	}
\end{vd}
\begin{vd}
	Cho số $a>1$. Tính giá trị biểu thức $P=a^{2\log_a3}$. 	
	\loigiai{
		Ta có $P=a^{2\log_a3}=\left(a^{\log_a3} \right)^2=3^2=9$.		
	}
\end{vd}
\begin{vd}
	Đặt $\log_2 3=a$, $\log_2 5=b$. Tính $\log_5 3$ theo $a$, $b$.
	\loigiai
	{Ta có $\log_5 3=\dfrac{\log_2 3}{\log_2 5}=\dfrac{a}{b}$.
	}
\end{vd}
\begin{vd}
	Tính giá trị biểu thức $Q=\log\dfrac{10}{11}+\log\dfrac{11}{12}+\log\dfrac{12}{13}+\cdots +\log\dfrac{999}{1000}$.	
	\loigiai{
		Ta có $Q=\log\dfrac{10}{11}+\log\dfrac{11}{12}+\log\dfrac{12}{13}+\cdots +\log\dfrac{999}{1000}=\log \left(\dfrac{10}{11}\cdot\dfrac{11}{12}\cdots \dfrac{999}{1000}\right)=\log \dfrac{10}{1000}=-2 $.
	}
\end{vd}
\begin{vd}
	Cho $a$, $b$, $c$ là các số thực dương, $a\neq 1$ và $\log_a b = 5$, $\log_a c = 7$. Tính giá trị của biểu thức $P = \log_{\sqrt{a}}\left(\dfrac{b}{c}\right)$.
	\loigiai
	{
		Ta có $P = \log_{\sqrt{a}}\left(\dfrac{b}{c}\right) = \log_{\sqrt{a}}b - \log_{\sqrt{a}}c = 2\log_ab - 2\log_ac = 2\cdot5 - 2\cdot7 = -4$.
	}
\end{vd}
\begin{vd}
	Cho $a$, $b$, $c$ là các số thực khác $0$ thỏa mãn $4^a=25^b=10^c$. Tính $T=\dfrac{c}{a}+\dfrac{c}{b}$.
	\loigiai{
		Từ $4^a=25^b=10^c\Rightarrow a\log 4=b\log 25=c\Rightarrow T = \dfrac{c}{a}+\dfrac{c}{b}=\log 4+\log 25=\log 100=2$.
	}
\end{vd}
\begin{vd}
	Cho $a$ và $b$ lần lượt là số hạng thứ nhất và thứ chín của một cấp số cộng có công sai $d\ne 0$. Tính giá trị của $\log_2 \left(\dfrac{b-a}{d}\right)$.
	\loigiai{
		Ta có $\log_2 \left(\dfrac{b-a}{d}\right) = \log_{2} \dfrac{u_{1}+8d - u_{1}}{d} =3$.
	}
\end{vd}
\begin{vd}
	Ba số $a+\log_23$; $a+\log_43$; $a+\log_83$ theo thứ tự lập thành cấp số nhân. Tìm công bội của cấp số nhân này.
	\loigiai{
		Ba số $a+\log_23$; $a+\log_43$; $a+\log_83$ theo thứ tự lập thành cấp số nhân nên ta có
		{\allowdisplaybreaks
			\begin{eqnarray*}
				& &(a+\log_23)(a+\log_83)=(a+\log_43)^2\\
				&\Leftrightarrow& \dfrac{1}{12}\log_2^23+\dfrac{1}{3}a\log_23=0\\
				&\Leftrightarrow&
				\dfrac{1}{12}\log_23\left(\log_23+4a\right)	=0\\
				&\Leftrightarrow&
				a=-\dfrac{1}{4}\log_23.
			\end{eqnarray*}
		}
		Vậy công bội $q=\dfrac{-\dfrac{1}{4}\log_23+\dfrac{1}{2}\log_23}{-\dfrac{1}{4}\log_23+\log_23}=\dfrac{1}{3}$.	
	}
\end{vd}

\subsubsection{Bài tập rèn luyện}
	\begin{bt}
		Tính giá trị biểu thức $A=2^{\log_4 9+\log_2 5}$.
		\loigiai
		{
			Ta có $A=2^{\log_4 9+\log_2 5}=2^{\log_4 9}\cdot2^{\log_2 5}=3\cdot5=15$.\\
			Vậy $A=15$.
		}
	\end{bt}
	\begin{bt}
		Cho hai số thực dương $a,\,b$ thỏa mãn $a^2b^3=64$. Tính giá trị của biểu thức $P=2\log_2a+3\log_2b$.
		\loigiai{
			Ta có $\log_2\left(a^2b^3\right)=\log_264\Leftrightarrow 2\log_2a+3\log_2b=6$.\\
			Vậy $P=6$.
		}
	\end{bt}
	\begin{bt}
		Cho $0<a\neq 1.$ Tính giá trị của biểu thức $T=\log_a(a^3)$.
		\loigiai{
			Ta có $T=\log_a(a^3)=3\log_a a=3$.
		}
	\end{bt}
	\begin{bt}
		Cho $a$ là một số thực dương tùy ý và $a\neq 2$. Tính $P=\log_{\tfrac{a}{2}} {\dfrac{a^3}{8}}$.
		\loigiai{
			Ta có $P=\log_{\tfrac{a}{2}} {\dfrac{a^3}{8}}=\log_{\tfrac{a}{2}} \left(\dfrac{a}{2}\right)^3=3$.}
	\end{bt}

	\begin{bt}
		Cho $a$ là số thực dương khác $1$. Tính giá trị của biểu thức $I=\log_a {a^{\frac{1}{2}}}$.
		\loigiai{ Ta có $I=\dfrac{1}{2}\log_a a=\dfrac{1}{2}$. 
		}
	\end{bt}
	\begin{bt}
		Tính giá trị của biểu thức $M = \log_2 \sqrt{2\sqrt{32}}$.
		\loigiai{
			Ta có $M = \log_2 \sqrt{2\sqrt{32}} = \log_2 \sqrt{2\sqrt{2^5}} = \log_2 \sqrt{2 \cdot 2^{\tfrac{5}{2}}} = \log_2 2^{\tfrac{7}{4}} = \dfrac{7}{4}$.
		}
	\end{bt}

	\begin{bt}
		Cho $a$ là số thực dương tùy ý khác $1$, tính giá trị $P=\log _{\sqrt[3]{a}}a^3$.
		\loigiai{
			Ta có $P=\log _{\sqrt[3]{a}}a^3=9\log _aa=9$.
		}
	\end{bt}
	\begin{bt}
		Tính giá trị biểu thức $10^{\log 5}+5^0$.
		\loigiai{
			Ta có $10^{\log 5}+5^0=5+1=6$.
		}
	\end{bt}
	\begin{bt}
		Biết $\log_3 5=a$. Tính $\log_3 45$ theo $a$.
		\loigiai{
			Ta có $\log_3 45=\log_3 5+\log_3 9=2+\log_3 5=2+a$.
		}
	\end{bt}
	\begin{bt}
		Cho $a$ và $b$ là hai số thực dương khác $1$ thỏa mãn $\sqrt{a}=\sqrt[3]{b}$. Tính giá trị $\log _a b$.
		\loigiai{
			Ta có $a$ và $b$ là hai số thực dương  thoả $\sqrt{a}=\sqrt[3]{b}\Leftrightarrow b=a^{\tfrac{3}{2}}$.\\
			Ta được $$\log _a b=\log _a a^{\tfrac{3}{2}}=\dfrac{3}{2}.$$
		}
	\end{bt}
	\begin{bt}
		Với $a$, $b$ là hai số thực dương thỏa mãn $\log a=11$, $\log b=13$. Tính giá trị biểu thức $\log\left(ab^2\right)$.
		\loigiai{Ta có $\log\left(ab^2\right)=\log a+\log b^2=\log a+2\log b=11+2\cdot13=37$.}
	\end{bt}
	
	\begin{bt}
		Cho $a, b$ là các số thực dương lớn hơn 1 thỏa mãn $\log _{a} b=3$. Tính giá trị biểu thức $$P=\log _{a^2b} a^3 - 3\log _{a^2}2 \cdot \log _{4}\left(\dfrac{a}{b}\right).$$
		\loigiai{
			Ta có $P=\dfrac{1}{\log _{a^3} (a^2b)} - \dfrac{3}{2}\log _{a}2 \cdot \dfrac{1}{2}\log _{2}\left(\dfrac{a}{b}\right)=\dfrac{3}{\log _{a} a^2 + \log _{a} b} -\dfrac{3}{4}\log_{a} \left(\dfrac{a}{b}\right)=\dfrac{3}{2 + \log _{a} b} -\dfrac{3}{4}\left(1- \log_{a} b\right)$.\\
			Mà $\log _{a} b=3$ nên $P=\dfrac{21}{10}$.
		}
	\end{bt}
	\begin{bt}
		Cho hai số dương $a,b$ với $a\ne 1$, thỏa mãn $\log_{a^2}b+\log_a{b^2}=2$. Tính $\log_ab$. 
		\loigiai{
			Ta có $\log_{a^2}b+\log_a{b^2}=2\Leftrightarrow\dfrac{1}{2}{\log_a}b+2\log_ab=2\Leftrightarrow\dfrac{5}{2}{\log_a}b=2\Leftrightarrow{\log_a}b=\dfrac{4}{5}$. }
	\end{bt}
	\begin{bt}
		Biết rằng $ a=\log_23$, $b=\log_25$. Hãy biểu diễn $\log_{45}4$ theo $ a$ và $b$.
		\loigiai{
			Ta có:\\
			$\log_{45}4=\dfrac{\log_24}{\log_245}=\dfrac{2\log_22}{\log_2\left(5\cdot3^2\right)}=\dfrac{2}{\log_23^2+\log_25}=\dfrac{2}{2\log_23+\log_25}=\dfrac{2}{2a+b}$.}
	\end{bt}

	\begin{bt}
		Cho biểu thức $f(x)=\log_2\left(x-\dfrac{1}{2}+\sqrt{x^2-x+\dfrac{17}{4}}\right)$, $(0<x<1)$. Tính giá trị của biểu thức
		$$T=f\left(\dfrac{1}{1000}\right)+f\left(\dfrac{2}{1000}\right)+\cdots+f\left(\dfrac{999}{1000}\right).$$
		\loigiai{
			Ta có $f(x)=\log_2\left(x-\dfrac{1}{2}+\sqrt{x^2-x+\dfrac{17}{4}}\right)=\log_2\left(x-\dfrac{1}{2}+\sqrt{\left(x-\dfrac{1}{2}\right)^2+4}\right)$.\\
			$f(1-x)=\log_2\left(1-x-\dfrac{1}{2}+\sqrt{\left(1-x-\dfrac{1}{2}\right)^2+4}\right)=\log_2 \left(\sqrt{\left(x-\dfrac{1}{2}\right)^2+4}-\left(x-\dfrac{1}{2}\right)\right)$.\\
			Khi đó
			\allowdisplaybreaks
			\begin{eqnarray*}
				f(x)+f(1-x)&=&\log_2\left(x-\dfrac{1}{2}+\sqrt{\left(x-\dfrac{1}{2}\right)^2+4}\right)+\log_2 \left(\sqrt{\left(x-\dfrac{1}{2}\right)^2+4}-\left(x-\dfrac{1}{2}\right)\right)\\
				&=&\log_2\left[\left(\sqrt{\left(x-\dfrac{1}{2}\right)^2+4}+x-\dfrac{1}{2}\right)\cdot \left(\sqrt{\left(x-\dfrac{1}{2}\right)^2+4}-\left(x-\dfrac{1}{2}\right)\right)\right]\\
				&=&\log_2 4=2.
			\end{eqnarray*}
			Do đó
			\allowdisplaybreaks
			\begin{eqnarray*}
				&&f\left(\dfrac{1}{1000}\right)+f\left(1-\dfrac{1}{1000}\right)=f\left(\dfrac{1}{1000}\right)+f\left(\dfrac{999}{1000}\right)=2\\
				&&f\left(\dfrac{2}{1000}\right)+f\left(1-\dfrac{2}{1000}\right)=f\left(\dfrac{2}{1000}\right)+f\left(\dfrac{998}{1000}\right)=2\\
				&&\cdots\\
				&&f\left(\dfrac{499}{1000}\right)+f\left(1-\dfrac{499}{1000}\right)=f\left(\dfrac{499}{2021}\right)+f\left(\dfrac{501}{1000}\right)=2.
			\end{eqnarray*}
			Vậy
			\allowdisplaybreaks
			\begin{eqnarray*}
				T&=&f\left(\dfrac{1}{1000}\right)+f\left(\dfrac{2}{1000}\right)+\cdots+f\left(\dfrac{999}{1000}\right)\\
				&=&\left[f\left(\dfrac{1}{1000}\right)+f\left(\dfrac{999}{1000}\right)\right]+\left[f\left(\dfrac{2}{1000}\right)+f\left(\dfrac{998}{1000}\right)\right]+\cdots+\left[f\left(\dfrac{499}{1000}\right)+f\left(\dfrac{501}{1000}\right)\right]+f\left(\dfrac{1}{2}\right)\\
				&=& 499\cdot 2+1=999.
			\end{eqnarray*}
		}
	\end{bt}

\subsubsection{Bài tập trắc nghiệm}
\Opensolutionfile{ans}[ans/Chuong6-2-logarit-Dang-1]
	\begin{ex}
		Với $a$ là số thực dương khác $1$ tuỳ ý, giá trị $\log_{a^2}a^3$ bằng
		\choice
		{$8$}
		{$6$}
		{$\dfrac{2}{3}$}
		{\True $\dfrac{3}{2}$}
		\loigiai{
			Với $a$ là số thực dương khác $1$, ta có $\log_{a^2}a^3=\dfrac{3}{2}\log_aa=\dfrac{3}{2}$.}
	\end{ex}
	\begin{ex}
		Cho $a \ne 1$ là số thực dương và $P=\log_{\sqrt[3]{a}}a^3$. Mệnh đề nào sau đây đúng?
		\choice
		{$P=3$}
		{$P=1$}
		{$P=\dfrac{1}{3}$}
		{\True$P=9$}
		\loigiai{
			$P=\log_{\sqrt[3]{a}}a^3=3\log_{a}a^3=9\log_{a}a=9 \cdot 1=9$.
		}
	\end{ex}

	\begin{ex}
		Cho $a>0$, $a\ne1$. Biểu thức $a^{\log_a{a^2}}$ bằng
		\choice
		{$2$}
		{\True $a^2$}
		{$2a$}
		{$2^a$}
		\loigiai{
			Ta có $a^{\log_a{a^2}}=a^2$.
		}
	\end{ex}

	\begin{ex}
		Giá trị của $\log _{\frac{1}{a}} \sqrt[3]{a^{7}}($ với $a>0, a \neq 1)$ bằng
		\choice
		{\True $-\dfrac{7}{3}$}
		{$\dfrac{2}{3}$}
		{$\dfrac{5}{3}$}
		{$4$}
		\loigiai{
			Ta có $\log _{\frac{1}{a}} \sqrt[3]{a^{7}} = \log_{a^{-1}} a^{\frac{3}{7}} = -\dfrac{3}{7}\log_a a =-\dfrac{3}{7}$.
		}
	\end{ex}
	\begin{ex}
		Với $a$ là số thực dương tùy ý, $\ln \left(\mathrm{e}a^{\pi}\right)$ bằng
		\choice
		{$1+a\ln \pi$}
		{$1+\ln \pi+\ln a$}
		{$1-\pi\ln a$}
		{\True $1+\pi\ln a$}
		\loigiai{
			Với $a>0$, ta có $\ln \left(\mathrm{e}a^{\pi}\right)=\ln \mathrm{e}+\ln a^{\pi}=1+\pi\ln a$.
		}
	\end{ex}
	\begin{ex}
		Với $a$ là số thực dương khác $1$ tùy ý, $\log_{a^5} a^4$ bằng
		\choice
		{$\dfrac{1}{5}$}
		{\True $\dfrac{4}{5}$}
		{$20$}
		{$\dfrac{5}{4}$}
		\loigiai{
			Với $0<a\neq 1$, ta có $\log_{a^5} a^4=\dfrac{4}{5}$.
		}
	\end{ex}

	\begin{ex}
		Với $a$ là thực dương tùy ý, $\ln (5a)-\ln (3a)$ bằng
		\choice
		{$\dfrac{\ln 5}{\ln 3}$}
		{$\dfrac{\ln (5a)}{\ln (3a)}$}
		{$\ln (2a)$}
		{\True $\ln \dfrac{5}{3}$}
		\loigiai{
			Ta có $\ln (5a)-\ln (3a)=\ln \dfrac{5a}{3a}=\ln \dfrac{5}{3}$.
		}
	\end{ex}
	\begin{ex} 
		Cho $\log_a b=2$ với $a$, $b>0$, $a$ khác $1$. Khẳng định nào sau đây \textbf{sai}?
		\choice
		{$\log_a(ab)=3$}
		{\True $\log _{a}\left(a b^{2}\right)=3$}
		{$\log_a b^2=4$}
		{$\log _{a}\left(a^2 b\right)=4$}
		\loigiai{
			$\log _{a}\left(a b^{2}\right)=3$ sai vì $\log _{a}\left(a b^{2}\right)=\log_a a+\log_a b^2=1+2\log_ab=1+2\cdot2=5$.
		}
	\end{ex}
	\begin{ex}
		Với $ a$ là số thực dương tùy ý, $\log_{\sqrt 3}{a^{1010}}$ bằng
		\choice
		{\True $ 2020\log_3a$}
		{$ 1010+2\log_3a$}
		{$ 1010+\dfrac{1}{2}{\log_3}a$}
		{$ 505\log_3a$}
		\loigiai{
			$\log_{\sqrt 3}{a^{1010}}=\log_{3^{\textstyle{1\over 2}}}{a^{1010}}=2\cdot 1010\cdot\log_3a=2020\log_3a$.
		}
	\end{ex}

	\begin{ex}
		Giá trị của $\log_a\dfrac{1}{a^3}$ với $a>0$ và $a\neq 1$ bằng
		\choice
		{\True $-3$}
		{$3$}
		{$-\dfrac{1}{3}$}
		{$\dfrac{1}{3}$}
		\loigiai{
			Ta có $\log_a\dfrac{1}{a^3}=-3\log_aa=-3$.
		}
	\end{ex}
	\begin{ex}
		Với mọi số thực $a$ dương, $\log _2^2a^2$ bằng
		\choice
		{$2 \log _2^2a$}
		{\True $-4 \log _2^2a$}
		{$2 \log _2a^2$}
		{$4 \log _2a$}
		\loigiai{
			Ta có $\log _2^2a^2=\left(2 \log _2a\right)^2=4 \log _2^2a$}
	\end{ex}
	\begin{ex}
		Với các số thực dương $a,b$ bất kì. Khẳng định nào sau đây là khẳng định đúng?
		\choice
		{\True $\log(ab)=\log a+\log b$}
		{$\log(ab)=\log (a+b)$}
		{$\log\left(\dfrac{a}{b}\right)=\log_b a$}
		{$\log\left(\dfrac{a}{b}\right)=\log(a-b)$}
		\loigiai
		{
			Với các số thực dương $a,b$ ta có $\log(ab)=\log a+\log b$.
		}
	\end{ex}
	\begin{ex}
		Giá trị của $\log_216$ bằng
		\choice
		{$3$}
		{\True $4$}
		{$-3$}
		{$-4$}
		\loigiai{
			Ta có $\log_216=\log_22^4=4\log_22=4$.}
	\end{ex}
	\begin{ex}
		Cho $a$, $b$ là các số thực dương thỏa mãn $a\neq 1$ và $\log_a b=3$. Tính $\log_a (a^2b)$.
		\choice
		{$4$}
		{$3$}
		{\True $5$}
		{$6$}
		\loigiai{
			Ta có $\log_a (a^2b)=\log_a a^2+\log_a b=2+3=5$.	
		}
	\end{ex}
	\begin{ex}
		Tính giá trị của $A=\log_2\left(\dfrac{8 \cdot 2^5}{\sqrt[3]{2} \cdot 4^{-3}}\right)^2$.
		\choice
		{$\dfrac{25}{3}$}
		{$\dfrac{164}{6}$}
		{\True $\dfrac{82}{3}$}
		{$\dfrac{716}{3}$}
		\loigiai{
			Ta có $\left(\dfrac{8 \cdot 2^5}{\sqrt[3]{2} \cdot 4^{-3}}\right)^2=2^{\left[3+5-\tfrac{1}{3}-(-6)\right]\cdot 2}=2^{\tfrac{82}{3}}$.
			Do đó $A=\dfrac{82}{3}$.
		}
	\end{ex}
	\begin{ex}
		Cho $ a,b$ là các số thực dương khác 1 thỏa mãn $\text{lo}{\text{g}_2}a=2$ và $\text{lo}{\text{g}_4}b=3$. Giá trị biểu thức $ P=\log_a\left(a^2b\right)$ bằng
		\choice
		{$ P=10$}
		{\True $ P=5$}
		{$ P=2$}
		{$ P=1$}
		\loigiai{
			Ta có $ P=\log_a\left(a^2b\right)=\dfrac{\log_2\left(a^2b\right)}{\log_2a}=\dfrac{2\log_2a+\log_2b}{\log_2a}=\dfrac{2\log_2a+2\log_4b}{\log_2a}=\dfrac{2\cdot 2+2\cdot 3}{2}=5$.
		}
	\end{ex}
	\begin{ex}
		Cho các số thực dương $a$, $b$, $c$ với $a\neq 1$, thỏa mãn $\log_a b=3$, $\log_a c=-2$. Khi đó $\log_a\left(a^3b^2\sqrt{c}\right)$ bằng
		\choice
		{$5$}
		{\True $8$}
		{$13$}
		{$10$}
		\loigiai{
			Ta có
			\[\log_a\left(a^3b^2\sqrt{c}\right)=\log_a a^3+\log_a b^2+\log_a\sqrt{c}=3+2\log_a b+\dfrac{1}{2}\log_a c=3+6-1=8.\]
		}
	\end{ex}
	\begin{ex}
		Biết $\log\left(xy^3\right)=\log\left(x^2y\right)=1$. Giá trị của $\log(xy)$ bằng
		\choice
		{$\dfrac{1}{2}$}
		{\True $\dfrac{3}{5}$}
		{$1$}
		{$\dfrac{5}{3}$}
		\loigiai
		{
			Từ giả thiết suy ra $x>0$, $y>0$. Ta có
			\[\left\{\begin{aligned}&\log\left(xy^3\right)=1 \\&\log\left(x^2y\right) = 1\end{aligned}\right. \Leftrightarrow \left\{\begin{aligned}&\log x+3\log y=1 \\&2\log x+\log y=1\end{aligned}\right. \Leftrightarrow \left\{\begin{aligned}&\log x=\dfrac{2}{5} \\&\log y=\dfrac{1}{5}.\end{aligned}\right.\]
			Khi đó, $\log(xy)=\log x+\log y = \dfrac{2}{5}+\dfrac{1}{5}=\dfrac{3}{5}$.
		}
	\end{ex}
	\begin{ex}
		Cho $a$ là số dương khác $1$. Khi đó giá trị của $P=a^{\log_{a\cdot\sqrt[3]{a}}16}$ là
		\choice
		{$48$}
		{\True $8$}
		{$3^{16}$}
		{$16$}
		\loigiai
		{
			Ta có $P=a^{\log_{a\cdot\sqrt[3]{a}}16}=a^{\log_{a\cdot a^{\frac{1}{3}}}16}=a^{\log_{a^{\frac{4}{3}}}16}=a^{\frac{3}{4}\cdot\log_{a}16}=a^{\log_{a}8}=8$.
		}
	\end{ex}

	\begin{ex}
		Cho các số $a$, $b$, $c$ thỏa mãn $\log_a3=2$, $\log_b3=\dfrac{1}{4}$ và $\log_{abc}3=\dfrac{2}{15}$. Giá trị của $\log_c3$ bằng	
		\choice
		{\True $\dfrac{1}{3}$}
		{$3$}
		{$2$}
		{$\dfrac{1}{2}$}
		\loigiai{
			Ta có 
			\begin{eqnarray*}
				&&\log_{abc}3=\dfrac{1}{\log_3abc}=\dfrac{1}{\log_3a+\log_3b+\log_3c}\\
				&\Rightarrow&\dfrac{2}{15}=\dfrac{1}{\dfrac{1}{2}+4+\log_3c}\\
				&\Rightarrow& \dfrac{9}{2}+\log_3c=\dfrac{15}{2}\\
				&\Rightarrow&\log_3c=3\\
				&\Rightarrow&\log_c3=\dfrac{1}{3}.
			\end{eqnarray*}	
		}
	\end{ex}
	\begin{ex}
		Giá trị của biểu thức $M=\log_22+\log_2 4+\log_28+\cdots+\log_2256$ bằng
		\choice
		{$56$}
		{$8\log_2256$}
		{\True $36$}
		{$48$}
		\loigiai{Ta có $M=\log_22+\log_2 4+\log_28+\cdots+\log_2256=1+2+3+\cdots+8=\dfrac{8\cdot(1+8)}{2}=36$.}
	\end{ex}
	\begin{ex}
		Giá trị của biểu thức $P=\left(\mathrm{e}^3\right)^{\log_{\mathrm{e}}5}$ bằng
		\choice
		{$16$}
		{\True $125$}
		{$32$}
		{$5$}
		\loigiai{
			Ta có $P=\left(\mathrm{e}^3\right)^{\log_{\mathrm{e}}5}=\left(5^3\right)^{\log_{\mathrm{e}}\mathrm{e}}=5^3=125$.
		}
	\end{ex}
	\begin{ex}
		Tìm giá trị của biểu thức $A=\log_2 \left(2\sin \dfrac{\pi}{12}\right) + \log_2 \cos \dfrac{\pi}{12}$.
		\choice
		{$3$}
		{$-2$}
		{\True $-1$}
		{$2$}
		\loigiai{
			Ta có $A=\log_2 \left(2\sin \dfrac{\pi}{12}\right) + \log_2 \cos \dfrac{\pi}{12} = \log_2 \left(2\sin \dfrac{\pi}{12} \cos \dfrac{\pi}{12} \right) = \log_2 \sin \dfrac{\pi}{6} = \log_2 \dfrac{1}{2}=-1.$
		}
	\end{ex}
	\begin{ex}
		Cho	$a$, $b$, $c$,  là các số thực dương, khác $1$ và thỏa mãn $\log_ab^2=x$, $\log_{b^2}\sqrt{c}=y$. Giá trị của $\log_ca$ bằng
		\choice
		{$\dfrac{xy}{2}$}
		{$2xy$}
		{\True $ \dfrac{1}{2xy}$}
		{ $\dfrac{2}{xy}$}
		\loigiai{
			Với $a$, $b$, $c$ là các số dương,		ta có $\log_ab^2=x\Rightarrow \log_ab=\dfrac{x}{2},\ \log_{b^2}\sqrt{c}=y\Rightarrow \log_bc=y$.\\
			Vậy $\log_ca=\dfrac{1}{\log_ac}=\dfrac{1}{\log_ab\cdot \log_bc}=\dfrac{1}{\dfrac{x}{2}\cdot 4y}=\dfrac{1}{2xy}$.
		}
	\end{ex}

	\begin{ex}
		Cho $\log_ax=2$, $\log_bx=5$ với $a$, $b$ là các số thực lớn hơn $1$. Giá trị của $\log_{\frac{a^2}{b}}x$ bằng	
		\choice
		{$\dfrac{6}{5}$}
		{$\dfrac{5}{6}$}
		{\True $\dfrac{5}{4}$}
		{$\dfrac{4}{5}$}
		\loigiai{
			Ta có $\log_{\frac{a^2}{b}}x=\dfrac{1}{\log_x\dfrac{a^2}{b}}=\dfrac{1}{\log_xa^2-\log_xb}=\dfrac{1}{2\log_xa-\log_xb}=\dfrac{1}{2\cdot \dfrac{1}{2}-\dfrac{1}{5}}=\dfrac{5}{4}$.
		}
	\end{ex}
	\begin{ex}
		Cho $a$ và $b$ là hai số thực dương thỏa mãn $a^4 b=16$. Giá trị của $4 \log _2 a+\log _2 b$ bằng
		\choice
		{\True $ 4 $}
		{$ 2 $}
		{$ 16 $}
		{$ 8 $}
		\loigiai{
			Lấy logarit cơ số $ 2 $ hai vế ta được $ 4\log_2a+\log_2b=4 $.	
		}
	\end{ex}
	
	\begin{ex}
		Cho các số thực dương $ a$ và $ b$ thỏa mãn $a^2-16b=0$. Tính giá trị của biểu thức $$P=\log_{\sqrt{2}}a-\log_2b.$$
		\choice
		{$ P=2$}
		{\True $ P=4$}
		{$ P=16$}
		{$ P=\sqrt{2}$}
		\loigiai{
			Ta có $b=\dfrac{a^2}{16}$ nên
			$$P=\log_{\sqrt{2}}a-\log_2\dfrac{a^2}{16}=2\log_2a-2\log_2a+\log_216=\log_216=4.$$
		}
	\end{ex}
	
	\begin{ex}
		Cho các số thực $a$, $b$, $c$ thuộc khoảng $(1;+\infty)$ và $\log_{\sqrt{a}}^2b+\log_b c\cdot \log_b\left(\dfrac{c^2}{b}\right)+9\log_ac=4\log_a b$. Giá trị của biểu thức $\log_a b+\log_b c^2$ bằng
		\choice
		{$2$}
		{$\dfrac{1}{2}$}
		{$3$}
		{\True $1$}
		\loigiai{
			Ta có
			\begin{eqnarray*}
				&&\log_{\sqrt{a}}^2b+\log_b c\cdot \log_b\left(\dfrac{c^2}{b}\right)+9\log_ac=4\log_a b\\
				&\Leftrightarrow &4\log_a^2 b-4\log_a b+\log_b c\cdot \left(2\log_b c -1\right)+9\log_a b\cdot \log_b c=0\\
				&\Leftrightarrow & 4\log_a^2 b+9\log_a b\cdot \log_b c+\log_b^2 c-\left(4\log_a b+\log_b c\right)=0\\
				&\Leftrightarrow & \left(4\log_a b+\log_b c\right)\left(\log_a b+2\log_b c-1\right)=0\\
				&\Leftrightarrow & \log_a b+2\log_b c-1=0\text{ (do }4\log_a b+\log_b c>0)\\
				&\Leftrightarrow & \log_a b+\log_b c^2=1.
			\end{eqnarray*} 
		}
	\end{ex}
	\begin{ex}
		Cho cấp số cộng $(u_n)$ có tất cả số hạng đều dương và $9\left(u_1+u_2+\cdots+u_{2050}\right)=4\left(u_1+u_2+\cdots+u_{3075}\right)$. Giá trị nhỏ nhất của biểu thức $P=\log_3^2 u_{14}+\log_3^2 u_{41}-\log_3^2 u_{122}$.
		\choice
		{\True $-4$}
		{$-2$}
		{$1$}
		{$3$}
		\loigiai{
			Ta có
			\begin{eqnarray*}
				&&9\left(u_1+u_2+\cdots+u_{2050}\right)=4\left(u_1+u_2+\cdots+u_{3075}\right)\\
				&\Leftrightarrow &9\left(2050u_1+2100225d\right)=4\left(3075u_1+4726275d\right)\\
				&\Leftrightarrow &d=2u_1.
			\end{eqnarray*}
			Khi đó
			\begin{eqnarray*}
				\log_3^2(u_{14})+\log_3^2(u_{41})-\log_3^2(u_{122})&=&\log_3^2(27u_1)+\log_3^2(81u_1)-\log_3^2(243u_1)\\
				&=&\left(3+\log_3 u_1\right)^2+\left(4+\log_3 u_1\right)^2-\left(5+\log_3 u_1\right)^2.
			\end{eqnarray*}
			Đặt $t=\log_3 u_1$. Khi đó $P=(t+3)^2+(t+4)^2-(t+5)^2=t^2+4t=(t+2)^2-4\geq -4$.\\
			Dấu \lq\lq  =\rq\rq \text{} xảy ra khi và chỉ khi $t=-2\Leftrightarrow \log_3 u_1=-2\Leftrightarrow u_1=\dfrac{1}{9}\Rightarrow d=2u_1=\dfrac{2}{9}$.\\
			Vậy giá trị nhỏ nhất của $P$ là $-4$.
		}
	\end{ex}

	\begin{ex}
		Cho các số thực $a,\ b,\ c$ thỏa mãn $a^{\log_3 7} = 27$, $b^{\log_7 11} = 49$, $c^{\log_{11} 25} =\sqrt{11}$. Giá trị của biểu thức $A=a^{\left( \log_3 7 \right)^2} + b^{\left( \log_7 11 \right)^2} + c^{\left( \log_{11} 25 \right)^2}$ là
		\choice
		{$129$}
		{$519$}
		{\True $469$}
		{$729$}
		\loigiai{
			Ta có
			\begin{eqnarray*}
				& A & = \left( a^{\log_3 7} \right) ^ {\log_3 7} + \left( b^{\log_7 11} \right) ^ {\log_7 11} + \left( c ^{\log_{11} 25} \right) ^ {\log_{11} 25}\\
				& & = 27^{\log_3 7} + 49^{\log_7 11} + \sqrt{11}^{\log_{11} 25}\\
				& & = \left( 3^{\log_3 7} \right)^3 + \left( 7^{\log_7 11} \right)^2 + \left( 11^{\log_{11} 25} \right)^{\frac{1}{2}}\\
				& & = 7^3 + 11^2 + 25^{\frac{1}{2}} = 469.
			\end{eqnarray*}
		}
	\end{ex}
\Closesolutionfile{ans}
% \begin{center}
% 	\fbox{\color{red}\fontfamily{qag}\bfseries\selectfont ĐÁP ÁN TRẮC NGHIỆM}
% \end{center}
% \inputansbox{10}{ans/Chuong6-2-logarit-Dang-1}

\begin{dang}{Biến đổi, rút gọn, biểu diễn biểu thức chứa lôgarít}
\end{dang}
\subsubsection{Ví dụ minh hoạ}
\begin{vd}
	Với $a$, $x$, $y$ là các số thực dương tùy ý và khác $1$. Rút gọn biểu thức $P=\dfrac{x^{\log _{a} y}}{y^{\log _{a} x}}$.
	\loigiai{
		Ta có $x^{\log _{a} y}=y^{\log _{a} x}$ nên $P=1$.
	}
\end{vd}
\begin{vd} 
	Biết $\log _{7} 2=m$, biểu diễn biểu thức $\log _{49} 28$ theo $m$.	
	\loigiai{
		Ta có $\log _{49} 28=\log _{7^2} 28=\dfrac{1}{2}\log_728=\dfrac{\log_74+1}{2}=\dfrac{2\log_72+1}{2}=\dfrac{2m+1}{2}$.
	}
\end{vd}
\begin{vd}
	Với $ a $ là số thực dương tùy ý, rút gọn biểu thức $\log_{3}\left(a^5\right)$.
	\loigiai{
		Do $a>0$ nên ta có $\log_{3}\left(a^5\right)=5\log_{3}a$.		
	}
\end{vd}
\begin{vd} 
	Cho các số thực $ a $, $ b $. Rút gọn biểu thức $ A=\log_2 2^a+\log_2 2^b $.	\loigiai{
		Ta có $ A=\log_2 2^a+\log_2 2^b=a\log_22+b\log_22=a+b$.
	}
\end{vd}
\begin{vd}
	Biết rằng $\log _2 3=a, \log _2 5=b$. Tính $\log _{45} 4$ theo $a$ và $b$.
	\loigiai{Ta có $\log _{45} 4=\dfrac{1}{\log _{4} 45}=\dfrac{2}{\log _{2} (5\cdot 9)}=\dfrac{2}{\log _{2} 5+2\log _{2} 3}=\dfrac{2}{2a+b}$.
	}
\end{vd}
\begin{vd}
	Cho hai số dương $a$, $b$ với $a\ne 1$. Đặt $M=\log_{\sqrt{a}}\sqrt[3]{b}$. Tính $M$ theo $N=\log_ab$.
	\loigiai{
		Ta có $M=\log_{\sqrt{a}}\sqrt[3]{b}=\log_{a^{\tfrac{1}{2}}}b^{\tfrac{1}{3}}=\dfrac{2}{3}\log_ab=\dfrac{2}{3}N$.	
	}
\end{vd}
\begin{vd}
	Biểu diễn $\log_{120}600$ theo $a=\log_23$ và $b=\log_35$.
	\loigiai{ Từ giả thiết ta có $ ab=\log_2 3\cdot \log_3 5=\log_2 5 $.\\
		Suy ra $ \log_{120}{600}=\dfrac{\log_2\left( 2^3\cdot3\cdot5^2\right)  }{\log_2 \left( 2^3\cdot3\cdot5\right)}=\dfrac{3+\log_2 3+2\log_2 5}{3+\log_2 3+\log_2 5}=\dfrac{3+a+2ab}{3+a+ab}$.
	}
\end{vd}

\begin{vd}
	Cho $\log5=a$. Tính $\log25000$ theo $a$.
	\loigiai
	{
		Ta có $\log25000=\log(5^2\cdot 10^3)=\log5^2+\log10^3=2\log5+3=2a+3$.
	}
\end{vd}
\begin{vd}
	Cho $a=\log 2, b=\log 3$. Tính $\log \sqrt[7]{0{,}432}$  theo $a$ và  $b$. 
	\loigiai{
		Ta có
		\begin{align*}
			\log \sqrt[7]{0{,}432} = \dfrac{1}{7} (\log 432 - 3) = \dfrac{1}{7}\left  [\log \left (2^4\cdot 3^3\right ) - 3 \right ] =\dfrac{4a+3b-3}{7}.
		\end{align*}
	}
\end{vd}
\begin{vd}
	Rút gọn biểu thức $M=2\log_{\sqrt{2}}(4x)-12\log_4\sqrt{x}+\log_{\tfrac{1}{2}}\dfrac{8}{x}$ với $x>0$.
	\loigiai
	{Ta có \begin{eqnarray*}
			M&=&2\log_{\sqrt{2}}(4x)-12\log_4\sqrt{x}+\log_{\tfrac{1}{2}}\dfrac{8}{x}\\
			&=&2\log_{2^\frac{1}{2}} (4x)-12\log_{2^2}\left(x^\frac{1}{2}\right)+\log_{2^{-1}}\left(\dfrac{8}{x}\right)\\
			&=&4\left(\log_2 4+\log_2 x\right)-3\log_2 x-\left(\log_2 8-\log_2 x\right)\\
			&=&5+2\log_2 x.
		\end{eqnarray*}
	}
\end{vd}
\begin{vd}
	Cho $\log_2 5=m$, $\log_3 5=n$. Khi đó $\log_6 5$ tính theo $m$ và $n$ là
	\loigiai{
		Ta có $\log_6 5=\dfrac{1}{\log_5 6}=\dfrac{1}{\log_5 2+\log_5 3}=\dfrac{1}{\dfrac{1}{m}+\dfrac{1}{n}}=\dfrac{mn}{m+n}$.	
	}
\end{vd}
\begin{vd}
	Cho $\log_{2}m=a$ và $A=\log_{m} \left(8m\right)$ với $m>0$, $m\ne 1$. Tìm mối liên hệ giữa $A$ và $a$.
	\loigiai{
		Ta có $A= \log_{m} (8m) = \log_{m}m + \log_{m}8 =1+3\log_{m}2 = 1 + 3\cdot\dfrac{1}{\log_{2}m} = 1+\dfrac{3}{a} = \dfrac{3+a}{a}$.\\
		Vậy $A=\dfrac{3+a}{a}$.
	}
\end{vd}
\begin{vd}
	Cho các số thực dương $x$, $a$, $b$, $c$ thỏa mãn $$\log x = 2 \log(2a) - 2 \log b - 4 \log \sqrt[4]{c}.$$ Biểu diễn $x$ theo $a$, $b$, $c$.
	\loigiai{
		Ta có $\log x = 2 \log(2a) - 2 \log b - 4 \log \sqrt[4]{c} \Leftrightarrow 
		\log x = \log(2a)^2 -\log b^2 - \log c   \Leftrightarrow x = \dfrac{4a^2}{b^2 c} $.\\
		Vậy $x = \dfrac{4a^2}{b^2 c}$.
	}
\end{vd}
\begin{vd}
	Cho các số nguyên $a,\,b,\,c$ thỏa mãn $a+\dfrac{b+\log_25}{c+\log_23}=\log_645$. Tính tổng $a+b+c$.
	\loigiai{
		Ta có $\log_645=\log_62\cdot\log_245=\dfrac{2\log_23+\log_25}{1+\log_23}=2+\dfrac{-1+\log_25}{1+\log_23}$.\\
		Suy ra $a=2,\,b=-1,\,c=1\Rightarrow a+b+c=2$.
	}
\end{vd}
\begin{vd}
	Cho $G=10^{10^{100}}$. Đặt $x=\log_{10} G$; $y=\log_x G$, khi đó $\log_y G$ có thể biểu diễn dưới dạng $\dfrac{m}{n}$ trong đó $m$, $n$ là các số nguyên dương và ước chung lớn nhất của chúng bằng $1$. Tính tổng các chữ số của số $m+n$.
	\loigiai{
		Ta có 
		{\allowdisplaybreaks 
			\begin{eqnarray*}  
				G=10^{10^{100}}
				& \Rightarrow & x=\log_{10} G = \log_{10} 10^{10^{100}}=10^{100} \\
				& \Rightarrow  & y=\log_x G=\log_{10^{100}} 10^{10^{100}}= \dfrac{ 10^{100}}{100}=10^{98}  \\ 
				& \Rightarrow &  \log_y G= \log_{10^{98}} 10^{10^{100}}=\dfrac{10^{100}}{98}=\dfrac{10^{99} \cdot 5}{49}=\dfrac{500{\ldots}0}{49}.
		\end{eqnarray*}}
		Vậy tổng các chữ số của số $m+n$ bằng $5+4+9=18$.
	}
\end{vd}

\subsubsection{Bài tập rèn luyện}
\begin{bt}
	Cho $a$ là số thực dương tùy ý, đặt $\log _{3} a=\alpha$. Biểu diễn biểu thức $P=\log _{\frac{1}{3}} a-\log _{\sqrt{3}} a$ theo $\alpha$.
	\loigiai{
		Ta có 
		\[
		P=\log _{\frac{1}{3}} a-\log _{\sqrt{3}} a
		= \log _{3^{-1}} a-\log _{3^{\frac{1}{2}}} a
		=  -\log _{3} a-2\log _{3} a
		= -3\log _{3} a=-3\alpha.
		\]
	}
\end{bt}

\begin{bt}
	Đặt $\log _5 3=a$. Biểu diễn $\log _{\tfrac{1}{25}} 81$ theo $a$.
	\loigiai{
		Ta có $\log _{\tfrac{1}{25}} 81=\log_{5^{-2}}3^4=-\dfrac{4}{2}\cdot\log_53=-2a$.
	}
\end{bt}
\begin{bt}
	Cho $a$ là số thực dương khác $1$ và $x,y$ là các số thực dương thỏa mãn ${{\log }_a}x=-1$ và ${{\log }_a}y=4$. Rút gọn biểu thức $P={{\log }_a}( {x^2}{y^3} )$.
	\loigiai{
		Ta có $P={{\log }_a}( {x^2}{y^3} )={{\log }_a}{x^2}+{{\log }_a}{y^3}=2{{\log }_a}x+3{{\log }_a}y=2\cdot ( -1 )+3\cdot 4=10$.
	}
\end{bt}
\begin{bt}
	Cho các số thực $ x,\, y ,\, z >1$ và $ \log_{xy}(yz)=2 $. Rút gọn biểu thức $\log_{\frac{z}{y}}\left(x\right)^4 +\log_{\frac{z}{x}}(xy)$.
	\loigiai
	{
		Ta có \[ \log_{xy}(yz)=2\Leftrightarrow yz=x^2y^2\Rightarrow \heva{& \dfrac{z}{y}=x^2\\&\dfrac{z}{x}=xy .}\]
		Suy ra $ \log_{\frac{z}{y}}\left(x\right) ^4 +\log_{\frac{z}{x}}(xy)=\log_{x^2} (x)^4+\log_{xy}(xy)=3$.
	}
\end{bt}
\begin{bt}
	Cho $\log_{12}3=a$. Tính $\log_{24}18$ theo $a$.
	\loigiai{
		Ta có $a=\log_{12}3=\dfrac{1}{\log_3 12}=\dfrac{1}{2\log_3 2+1}\Rightarrow{\log_3}2=\dfrac{1-a}{2a}$.\\
		Có $\log_{24}18=\dfrac{\log_3 18}{\log_3 24}=\dfrac{2+\log_32}{1+3\log_32}=\dfrac{2+\dfrac{1-a}{2a}}{1+3\cdot \dfrac{1-a}{2a}}=\dfrac{3a+1}{3-a}$.
	}
\end{bt}
\begin{bt}
	Đặt $a=\ln 2$ và $b=\ln 3$. Biểu diễn $S=\ln \dfrac{1}{2}+\ln \dfrac{2}{3}+\ln \dfrac{3}{4}+\cdots+\ln \dfrac{71}{72}$ theo $a$ và $b$.
	\loigiai{
		Ta có
		\begin{eqnarray*}
			S&=&\ln \dfrac{1}{2}+\ln \dfrac{2}{3}+\ln \dfrac{3}{4}+\cdots+\ln \dfrac{71}{72}\\
			&=& \ln \left(\dfrac{1}{2}\cdot \dfrac{2}{3}\cdot \dfrac{3}{4}\cdots \dfrac{71}{72}\right)=\ln \dfrac{1}{72}\\
			&=&-\ln \left(2^3\cdot 3^2\right)=-3\ln 2-2\ln 3=-3a-2b.
		\end{eqnarray*}
	}
\end{bt}
\begin{bt}
	Rút gọn biểu thức $Q=\left(y^{\log_2 3}\right)^{\log_5 2}$ (với $y>0$).
	\loigiai
	{
		Ta có $Q=\left(y^{\log_2 3}\right)^{\log_5 2}=y^{\log_5 2\cdot\log_2 3}=y^{\log_5 3}$.
	}
\end{bt}
\begin{bt}
	Cho $\log_2a=x$ và $\log_2b=y$ với $a>0$, $b>0$ và $a\ne b$. Tìm biểu diễn của $\log_{a^{-2}b^3}(a^4b)$ theo $x$ và $y$.
	\loigiai{
		Ta có $\log_{a^{-2}b^3}(a^4b)=\dfrac{\log_2(a^4b)}{\log_2(a^{-2}b^3)}=\dfrac{\log_2a^4+\log_2b}{\log_2a^{-2}+\log_2b^3}=\dfrac{4\log_2a+\log_2b}{-2\log_2a+3\log_2b}=\dfrac{4x+y}{-2x+3y}$.
	}
\end{bt}
\begin{bt}
	Cho $\log_2 3 = a$. Biểu diễn biểu thức $\log_9 2$ theo $a$.
	\loigiai{
		Ta có $\log_9 2 = \dfrac{1}{\log_2 9}= \dfrac{1}{2\log_2 3} =\dfrac{1}{2a} $.
	}
\end{bt}
\begin{bt}
	Cho $0 < x \neq 1$, $0 < y$ thỏa mãn $\log_2 x = y$ và $\log_x y = \dfrac{3}{y}$. Tính tổng $x + y$.
	\loigiai{
		Ta có $\log_x y = \dfrac{3}{y} \Leftrightarrow \dfrac{\log_2 y}{\log_2 x} = \dfrac{3}{y} \Leftrightarrow \dfrac{\log_2 y}{y} = \dfrac{3}{y} \Leftrightarrow y = 8$, khi đó $x = 256$.\\
		Do đó $x + y = 264$.
	}
\end{bt}
\begin{bt}
	Cho các số thực dương $a$, $b$ thỏa mãn $\ln a=x$, $\ln b=y$. Tính $P=\ln \left(a^3b^2\right)$.
	\loigiai{
		Với $a$, $b$ là các số dương, ta có 
		$$P=\ln \left(a^3 b^2\right)=\ln \left(a^3\right)+\ln \left(b^2\right)= 3\ln a+2\ln b= 3x+2y.$$
	}
\end{bt}
\begin{bt}
	Biết $\log_{15}20=a+\dfrac{2\log_32+b}{\log_35+c}$ với $a$, $b$, $c\in\mathbb{Z}$. Tính $T=a+b+c$.
	\loigiai{
		Ta có \begin{eqnarray*}
			\log_{15}20&=&\log_{15}(2^2\cdot 5)=2\log_{15}2+\log_{15}5\\
			&=& \dfrac{2}{\log_215}+\dfrac{1}{\log_515}=\dfrac{2}{\log_23+\log_25}+\dfrac{1}{1+\log_53}\\
			&=&\dfrac{2}{\dfrac{1}{\log_32}+\log_25}+\dfrac{1}{1+\dfrac{1}{\log_53}}\\
			&=& \dfrac{2\log_32}{1+\log_25\cdot \log_32}+\dfrac{\log_53}{\log_53+1}\\
			&=& \dfrac{2\log_32}{1+\log_35}+\dfrac{\log_53}{\log_53+1}=\dfrac{2\log_32+\log_53}{\log_53+1}\\
			&=&\dfrac{2\log_32-1+\log_53+1}{\log_53+1}=1+\dfrac{2\log_32-1}{\log_53+1}.
		\end{eqnarray*}
		Vậy $a=1$, $b=-1$, $c=1$, suy ra $T=a+b+c=1$.
	}
\end{bt}
\begin{bt}
	Cho $\log_23=x$ và $\log_25=y$. Biết rằng $\log_{20}15=\dfrac{ax+by}{cy+2}$ với $a$, $b$, $c$ là các số nguyên dương. Tính $P=a+b+c$.
	\loigiai{
		Ta có $\log_{20}15=\dfrac{\log_215}{\log_220}=\dfrac{\log_23+\log_25}{\log_24+\log_25}=\dfrac{x+y}{2+y}$.\\
		Vậy nên $a=b=c=1$, suy ra $P=a+b+c=3$.
	}
\end{bt}

\begin{bt}
	Cho tam giác $ABC$ vuông tại $A$ và $AD$ là đường cao. Biết $AB=\log y$, $AC=\log 3$, $AD=\log x$, $BC=\log 9$. Tính $\dfrac{y}{x}$.
	\loigiai{
		\begin{center}
			\begin{tikzpicture}[scale=.7,>=stealth,font=\footnotesize,line cap=round,line join=round]
				\coordinate (A) at (0,0);
				\fill[black] (A) circle(1.5pt) node[above]{$A$};
				\coordinate (C) at (2,-4);
				\fill[black] (C) circle(1.5pt) node[below]{$C$};
				\coordinate (B)at (-8,-4);
				\fill[black] (B) circle(1.5pt) node[below]{$B$};
				\draw (0,0)--(-8,-4)--(2,-4)--(0,0);
				\draw pic[draw,angle radius=1.5mm]{right angle=C--A--B};%Theo chiều dương
				\coordinate (D) at (0,-4);
				\fill[black] (D) circle(1.5pt) node[below]{$D$};
				\draw (0,0)--(0,-4);
			\end{tikzpicture}
		\end{center}
		Xét $\triangle ABC$ vuông tại $A$ có
		\allowdisplaybreaks
		\begin{eqnarray*}
			AB^2+AC^2=BC^2 &\Leftrightarrow & \log^2 y+\log^2 3=\log^2 9\\
			&\Leftrightarrow & \log^2 y+\log^2 3=4\log^2 3\\
			&\Leftrightarrow & \log^2 y=3\log^2 3\\
			&\Leftrightarrow & \log y=\sqrt{3}\log 3 \text{ (vì } \log y=AB>0)\\
			&\Leftrightarrow & y=10^{\sqrt{3}\log 3}=10^{\log 3^{\sqrt{3}}}=3^{\sqrt{3}}.
		\end{eqnarray*}
		Xét $\triangle ABC$ vuông tại $A$ có đường cao $AD$, ta có
		\allowdisplaybreaks
		\begin{eqnarray*}
			AB\cdot AC=AD\cdot BC &\Leftrightarrow & \log y\cdot \log 3=\log x\cdot \log 9\\
			&\Leftrightarrow & \log y\cdot \log 3=2\log x\cdot \log 3\\
			&\Leftrightarrow & \log y=2\log x \Leftrightarrow \sqrt{3}\log 3=2\log x\\
			&\Leftrightarrow & \log x=\dfrac{\sqrt{3}}{2}\log 3 = \log 3^{\tfrac{\sqrt{3}}{2}} \Leftrightarrow x=3^{\tfrac{\sqrt{3}}{2}}.
		\end{eqnarray*}
		Vậy $y=x^2 \Leftrightarrow \dfrac{y}{x}=x=3^{\tfrac{\sqrt{3}}{2}}$.
	}
\end{bt}

\subsubsection{Bài tập trắc nghiệm}
\Opensolutionfile{ans}[ans/Chuong6-2-logarit-Dang-2]

\begin{ex}
	Với mọi số thực $a$ dương, $\log_2 {\dfrac{4}{a}} $ bằng
	\choice
	{$ \log_2 a - 2 $}
	{$4-\log_2 a$}	
	{\True $2-\log_2 a$}
	{$2+\log_2 a$}
	\loigiai{
		Ta có $\log_2 {\dfrac{4}{a}} = \log_2 4 - \log_2 a = 2 - \log_2 a $.
	}
\end{ex}

\begin{ex}
	Với $a$ là số thực dương tùy ý, $\log\left(10a^2\right)$ bằng
	\choice
	{$20\log a$}
	{\True $1+2\log a$}
	{$1+\left(\log a\right)^2$}
	{$10\log a$}
	\loigiai{
		Với $a$ là số thực dương tùy ý, ta có $\log\left(10a^2\right)=\log 10+\log a^2=1+2\log a$.  	
	}
\end{ex}
\begin{ex}
	Với $a$ là số thực dương tùy ý, $\log \dfrac{5 a}{2}+\log \dfrac{4}{a}$ bằng
	\choice
	{\True $1 $}
	{$ 10 $}
	{$\log \dfrac{5 a}{2} \cdot \log \dfrac{4}{a}$}
	{$\ln 10$}
	\loigiai{
		Ta có $\log \dfrac{5 a}{2}+\log \dfrac{4}{a}=\log \left(\dfrac{5 a}{2} \cdot \dfrac{4}{a}\right)=\log 10=1$.
	}
\end{ex}

\begin{ex}
	Với $a, b$ là hai số dương tùy ý thì $\log \left(a^3 b^2\right)$ có giá trị bằng biểu thức nào sau đây?
	\choice
	{$3 \log a+\dfrac{1}{2} \log b $}
	{$2 \log a+3 \log b $}
	{\True $3 \log a+2 \log b $}
	{$3\left(\log a+\dfrac{1}{2} \log b\right) $}
	\loigiai{
		Ta có 	$\log \left(a^3 b^2\right)=3 \log a+2 \log b$.
	}
\end{ex}
\begin{ex}
	Cho $a$ và $b$ là hai số thực dương thỏa mãn $2\log_2b-3\log_2a=2$. Khẳng định nào sau đây đúng?
	\choice
	{$2b-3a=2$}
	{\True $b^2=4a^3$}
	{$2b-3a=4$}
	{$b^2-a^2=4$}
	\loigiai{
		Ta có $2\log_2b-3\log_2a=2\Leftrightarrow 2\log_2b=2+3\log_2a\Leftrightarrow \log_2 b^2=\log_2 4+\log_2 a^3\Leftrightarrow b^2=4a^3$.
	}
\end{ex}
\begin{ex}
	Giả sử $a$, $b$ là các số thực dương bất kỳ. Biểu thức $\ln \dfrac{a}{b^2}$ bằng
	\choice
	{$\ln a+\dfrac{1}{2}\ln b$}
	{$\ln a+2\ln b$}
	{\True $\ln a-2\ln b$}
	{$\ln a-\dfrac{1}{2}\ln b$}
	\loigiai
	{
		Với mọi số thực dương $a$, $b$ ta có $\ln \dfrac{a}{b^2}=\ln a-\ln b^2 = \ln a-2\ln b$.
	}
\end{ex}
\begin{ex}
	Cho $a=\log_3 4$. Khi đó $\log_3 36$ bằng
	\choice
	{$a+4$}
	{$2a+4$}
	{\True $a+2$}
	{$a+9$}
	\loigiai{
		Ta có $\log_3 36 = \log_3 (4\cdot 9) = \log_3 4 + \log_3 9 = a+2$.
	}
\end{ex}

\begin{ex}	
	Xét $a,b$ là các số thực dương thỏa mãn $4\log_2a+2\log_4b=1$. Khẳng định nào sau đây là đúng?
	\choice
	{\True $a^4b=2$}
	{ $a^4b=1$}
	{ $a^4{b^2}=2$}
	{ $a^4{b^2}=4$}
	\loigiai{
		Ta có $$4\text{lo}{{\text{g}}_2}a+2\text{lo}{{\text{g}}_4}b=1\Leftrightarrow 4\log_2a+\log_2b=1\Leftrightarrow \log_2{a^4}+\log_4b=1 \Leftrightarrow \log_2{a^4}b = 1 \Leftrightarrow {a^4}b = 2.$$}
\end{ex}

\begin{ex}
	Với mọi $a$, $b$, $x$ là các số thực dương thoả mãn $\log_{2}x=5\log_{2}a+3\log_{2}b$. Mệnh đề nào dưới đây đúng
	\choice
	{$x=5a+3b$}
	{$x=a^5+b^3$}
	{\True $x=a^5b^3$}
	{$x=3a+5b$}
	\loigiai{
		Ta có	$\log_{2}x=5\log_{2}a+3\log_{2}b=\log_{2}a^5+\log_{2}b^3=\log_{2}{\left(a^5b^3\right)}\Rightarrow x=a^5b^3$.
	}
\end{ex}

\begin{ex} 
	Cho $a$ và $b$ là hai số thực dương thỏa mãn $\sqrt{a}\cdot b^3=27$. Giá trị của $\log_{3} a+6\log_{3}b$ bằng
	\choice 
	{ $3$}
	{\True $6$}
	{ $9$}
	{ $1$}
	\loigiai{ Ta có $\sqrt{a}\cdot b^3=27\Leftrightarrow \sqrt{a}=\left(  \dfrac{3}{b} \right)^3 \Rightarrow \log _{3}\sqrt{a}=\log_3\left(\dfrac{3}{b}\right)^3$\\
		$\Rightarrow \dfrac{1}{2}\log_3 a=3( 1-\log_3 b)\Rightarrow \log_3 a+3\log_3 b=6$		
	} 
\end{ex}
\begin{ex} 
	Với mọi số thực dương $x$, $\log_3\left( \dfrac{x^3}{3}\right) $ bằng
	\choice
	{\True $3\log_3 x-1$}
	{$\log_3 x-1$}
	{$\log_3 x$}
	{$3\log_3 x+1$}
	\loigiai{
		Với mọi số thực dương $x$, ta có $\log_3\left( \dfrac{x^3}{3}\right)=\log_3\left(x^3 \right)- \log_33=3\log_3 x-1$.	
	}
\end{ex}
\begin{ex} 
	Với mọi $a$, $b$ thỏa mãn $2\log_9a-3\log_3b=1$, mệnh đề nào sau đây đúng?
	\choice
	{$a=\dfrac{3}{b^3}$}
	{$2a-3b=1$}
	{$a^2=3b^3$}
	{\True $a=3b^3$}
	\loigiai{
		Ta có 
		\begin{align*}
			2\log_9a-3\log_3b=1 & \Leftrightarrow 2\log_{3^2}a-3\log_3b=1 \\
			&\Leftrightarrow \log_3a-\log_3b^3=1\\
			&\Leftrightarrow \log_3\left(\dfrac{a}{b^3} \right) =1\\
			&\Leftrightarrow \dfrac{a}{b^3} =3\Leftrightarrow a=3b^3.
		\end{align*}
	}
\end{ex}
\begin{ex}
	Cho hai số thực dương $a$, $b$ bất kì thỏa mãn $9\log^2a+4\log^2b=12\log a\cdot \log b$. Khẳng định nào dưới đây đúng?
	\choice
	{$3a=2b$}
	{$2a=3b$}
	{$a^2=b^3$}
	{\True $a^3=b^2$}
	\loigiai{
		Ta có 
		\begin{eqnarray*}
			&&9\log^2a+4\log^2b=12\log a\cdot \log b\\
			&\Leftrightarrow& \left(3\log a\right)^2+\left(2\log b\right)^2-2\cdot 3\log a\cdot 2\log b=0\\
			&\Leftrightarrow& \left(3\log a-2\log b\right)^2=0\\
			&\Leftrightarrow& 3\log a=2\log b \\
			&\Leftrightarrow& \log a^3=\log b^2\\
			&\Leftrightarrow& a^3=b^2.
		\end{eqnarray*}	
	}
\end{ex}
\begin{ex}
	Đặt $x=\log_214$. Biết $\log_{98}32=\dfrac{a}{bx-c}$ với $a$, $b$, $c$ là những số tự nhiên và biểu thức là tối giản. Giá trị của biểu thức $S=2a+3b+5c$ là
	\choice
	{\True $21$}
	{$16$}
	{$17$}
	{$26$}
	\loigiai{Ta có $x=\log_214=\log_22+\log_27\Leftrightarrow \log_27=x-1$.\\
		Ta có $\log_{98}32=\dfrac{\log_232}{\log_298}=\dfrac{\log_22^5}{\log_22+\log_27^2}=\dfrac{5}{1+2(x-1)}=\dfrac{5}{2x-1}$.\\
		Vậy nên $a=5$, $b=2$, $c=1$, suy ra $S=2a+3b+5c=21$.}
\end{ex}
\begin{ex}
	Cho $a$, $b$ là các số thực dương tùy ý và $a\ne 1$. Đặt $P=\log_ab^3 + \log_{a^2}b^6$. Mệnh đề nào sau
	đây là đúng?
	\choice
	{$P=9\log_a b$}
	{\True $P=6\log_a b$}
	{$P=27\log_a b$}
	{$P=15\log_a b$}
	\loigiai
	{
		Ta có $P=\log_ab^3 + \log_{a^2}b^6=3\log_a b+\dfrac{6}{2} \log_a b=6\log_a b$.
	}
\end{ex}
\begin{ex}
	Đặt $\log\limits_23=a$. Khi đó $\log\limits_{12}18$ bằng
	\choice
	{$\dfrac{2+a}{1+2a}$}
	{\True $\dfrac{1+2a}{2+a}$}
	{$a$}
	{$\dfrac{1+3a}{2+a}$}
	\loigiai{
		Ta có $\log\limits_{12}18=\dfrac{\log_2 18}{\log_2 12}=\dfrac{\log_2\left(2\cdot 3^2\right)}{\log_2 \left(2^2\cdot 3\right)}=\dfrac{\log_2 2+2\log_2 3}{2\log_2 2+\log_2 3}=\dfrac{1+2a}{2+a}$.
	}
\end{ex}
\begin{ex}
	Với mọi số thực dương $x$, $y$ thỏa mãn $x^2+y^2=8xy$, mệnh đề nào dưới đây đúng? 
	\choice
	{\True $\log(x+y)=\dfrac{1}{2}(1+\log x+\log y)$}
	{$\log(x+y)=\dfrac{1}{2}(\log x+\log y)$}
	{$\log(x+y)=1+\log x+\log y$}
	{$\log(x+y)=\dfrac{1}{2}+\log x+\log y$}
	\loigiai{Ta có $x^2+y^2=8xy\Leftrightarrow(x+y)^2=10xy$. Lấy logarit thập phân hai vế ta được
		\begin{eqnarray*}
			\log(x+y)^2=\log(10xy)&\Leftrightarrow&2\log(x+y)=1+\log x+\log y\\
			&\Leftrightarrow&\log(x+y)=\dfrac{1}{2}(1+\log x+\log y).
	\end{eqnarray*}}
\end{ex}
\begin{ex}
	Cho các số thực dương $x$, $y$ thỏa mãn $\log(xy^2)=5$ và $\log(x^3y)=10$. Tính $P=\log(xy)$.
	\choice
	{\True $P=4$}
	{$P=2$}
	{$P=5$}
	{$P=1$}
	\loigiai{Từ giả thiết ta có $\heva{&\log x+2\log y=5\\&3\log x+\log y=10}\Leftrightarrow\heva{&\log x=3\\&\log y=1.}$\\
		Vậy ta có $P=\log xy=\log x+\log y=4$.}
\end{ex}
\begin{ex}
	Nếu $\log_52=m$ thì $\log\left(2^{20}\cdot 5^{19}\right)$ bằng	
	\choice
	{$\dfrac{19m+20}{m+1}$}
	{$\dfrac{20m-19}{m+1}$}
	{$\dfrac{19m+20}{m-1}$}
	{\True $\dfrac{20m+19}{m+1}$}
	\loigiai{
		Ta có $\log\left(2^{20}\cdot 5^{19}\right)=\dfrac{\log_5\left(2^{20}\cdot 5^{19}\right)}{\log_5 (10)}=\dfrac{20\log_5 2+19}{1+\log_5 2}=\dfrac{20m+19}{m+1}$.
	}
\end{ex}

\begin{ex}
	Cho $a$, $b$, $c$ là các số lớn hơn $1$, đặt $\log_ab=m$, $\log_ac=n$. Khi đó $\log_a\left(ab^2c^5\right)$ bằng
	\choice
	{\True $1+2m+5n$}
	{$1+\dfrac{1}{2}m+\dfrac{1}{5}n$}
	{$1+\dfrac{1}{5}m+\dfrac{1}{2}n$}
	{$1+5m+2n$}
	\loigiai{
		Ta có $\log_a\left(ab^2c^5\right)=\log_aa+\log_ab^2+\log_ac^5=1+2\log_ab+5\log_ac=1+2m+5n$.
	}
\end{ex}
\begin{ex}
	Cho tam giác $ABC$ có $BC=a$, $CA=b$, $AB=c$. Nếu $a$, $b$, $c$ theo thứ tự lập thành một cấp số nhân thì
	\choice
	{$\ln \sin A\cdot\ln \sin C=2\ln \sin B$}
	{\True $\ln \sin A+\ln \sin C=2\ln \sin B$}
	{$\ln \sin A\cdot\ln \sin C=(\ln \sin B)^2$}
	{$\ln \sin A+\ln \sin C=\ln (2\sin B)$}
	\loigiai{Do $a$, $b$, $c$ theo thứ tự lập thành một cấp số nhân nên 
		\begin{eqnarray*}
			a\cdot c=b^2&\Leftrightarrow&\sin A\cdot\sin C=\sin^2B\\
			&\Leftrightarrow&\ln (\sin A\cdot\sin C)=\ln \left(\sin^2B\right)\\
			&\Leftrightarrow&\ln \sin A+\ln \sin C=2\ln \sin B.
		\end{eqnarray*}
	}
\end{ex}
\begin{ex}
	Xét các số thực dương $a$ và $b$ thỏa mãn $\log_5\left(5^a\cdot 25^b\right)=5^{\log_5a+\log_5b+1}$. Mệnh đề nào dưới đây đúng?
	\choice
	{$a+2b=ab$}
	{\True $a+2b=5ab$}
	{$2ab-1=a+b$}
	{$a+2b=2ab$}
	\loigiai{
		Ta có $\log_5\left(5^a\cdot 25^b\right)=5^{\log_5a+\log_5b+1}\Leftrightarrow \log_55^a+\log_55^{2b}=5^{\log_5(5ab)}\Leftrightarrow a+2b=5ab$.}
\end{ex}
\begin{ex}
	cho ba số thực dương $a$, $b$, $c$ khác $1$ thỏa $\log_a b + \log_c b = \log_a 2016 \cdot \log_c b$. Khẳng định nào sau đây là đúng?	
	\choice
	{$bc=2016$}
	{$ab=2016$}
	{\True $ac=2016$}
	{$abc=2016$}
	\loigiai{
		\begin{eqnarray*}
			& & \log_a b + \log_c b = \log_a 2016 \cdot \log_c b\\
			& \Leftrightarrow & \log_a b + \log_c b = \log_a 2016 \cdot \dfrac{1}{\log_b c}\\
			& \Leftrightarrow & 	\log_a b \cdot \log_b c + \log_b c \cdot \log_c b = \log_a 2016\\
			& \Leftrightarrow &  \log_a c + 1 = \log_a 2016 \\
			& \Leftrightarrow & \log_a ca = \log_a 2016 \\
			& \Leftrightarrow & ac = 2016.
		\end{eqnarray*}	
	}
\end{ex}
\begin{ex}
	Cho $\log_2 5=a$, $\log_5 3=b$, biết $\log_{24} 15=\dfrac{ma+ab}{n+ab}$, với $m, n\in \mathbb{Z}$. Tính $S=m^2+n^2$.
	\choice
	{$S=2$}
	{\True $S=10$}
	{$S=5$}
	{$S=13$}
	\loigiai{
		Ta có
		\begin{eqnarray*}
			\log_{24}15&=&\log_{24}3+\log_{24}5=\dfrac{1}{\log_3 24}+\dfrac{1}{\log_5 24}\\
			&=&\dfrac{1}{1+3\log_3 2}+\dfrac{1}{\log_5 3+3\log_5 2}=\dfrac{1}{1+\dfrac{3}{ab}}+\dfrac{1}{b+\dfrac{3}{a}}\\
			&=& \dfrac{ab}{ab+3}+\dfrac{a}{ab+3}=\dfrac{a+ab}{3+ab}.
		\end{eqnarray*}
		Suy ra $m=1$, $n=3$, do đó $S=1^2+3^2=10$.
	}
\end{ex}
\begin{ex}
	Cho các số thực $a$, $b$ với $ab>0$. Mệnh đề nào dưới đây \textbf{sai?}
	\choice
	{$\ln \left(\dfrac{a}{b}\right)=\ln |a|+\ln |b|^{-1}$}
	{$\log(ab)=\log|a|+\log|b|$}
	{$\log a^4=4\log|a|$}
	{\True $\log(ab)=\log a+\log b$}
	\loigiai
	{
		Do $ab>0$ nên $\log(ab)=\log|a|+\log|b|$.\\
		Khi $a>0$, $b>0$ ta có $\log(ab)=\log a+\log b$.
	}
\end{ex}

\begin{ex}
	Đặt $\log_2 5=a$, tính giá trị của $\log_4 1250$ theo $a$.
	\choice
	{$2(1+4a)$}
	{\True $\dfrac{1+4a}{2}$}
	{$2(1-4a)$}
	{$\dfrac{1-4a}{2}$}
	\loigiai{
		Ta có $\log_4 1250=\dfrac{1}{2}\log_2\left(2\cdot 5^4\right)=\dfrac{1}{2}\left(1+4\log_2 5\right)=\dfrac{1}{2}\left(1+4a\right).$
	}
\end{ex}
\begin{ex}
	Cho $\log_2 3 = a$, $\log_2 5 = b$, khi đó $\log_5 675$ được biểu diễn theo $a$, $b$ là đáp án nào sau đây?
	\choice
	{\True $\dfrac{3a + 2b}{b} $}
	{$ \dfrac{ab + b}{2+3a} $}
	{$ \dfrac{a^3 + b^2}{b} $}
	{ $\dfrac{a + ab}{3 + 2a}$ }
	\loigiai{
		Ta có $\log_5 675 = \dfrac{\log_2 675}{\log_2 5} = \dfrac{\log_2 (3^3\cdot 5^2)}{\log_2 5} = \dfrac{3\log_2 3 + 2\log_2 5}{\log_2 5} = \dfrac{3a+2b}{b}$.		
	}
\end{ex}

\begin{ex}
	Biết rằng $\log_2 3=a$, $\log_2 5=b$. Tính $\log_{45} 4$ theo $a$, $b$.
	\choice
	{$\dfrac{2a+b}{2}$}
	{$\dfrac{2b+a}{2}$}
	{\True $\dfrac{2}{2a+b}$}
	{$2ab$}
	\loigiai{
		Ta có
		\allowdisplaybreaks
		\begin{eqnarray*}
			\log_{45} 4&= & \dfrac{1}{\log_{4} 45}=\dfrac{2}{\log_{2} 45}\\
			&= & \dfrac{2}{2\log_{2} 3+\log_{2} 5}=\dfrac{2}{2a+b}.
		\end{eqnarray*}
	}
\end{ex}

\begin{ex}
	Biết $\log_2 3=a$, $\log_3 5=b$. Khi đó $\log_{15} 12$ bằng
	\choice
	{$\dfrac{a+2}{b+1}$}
	{$\dfrac{ab+1}{a+2}$}
	{\True $\dfrac{a+2}{a(b+1)}$}
	{$\dfrac{a(b+1)}{a+2}$}
	\loigiai
	{
		Ta có $\log_{15}12=\dfrac{\log_3 12}{\log_3 15}=\dfrac{1+2\log_3 2}{1+\log_3 5}=\dfrac{1+\dfrac{2}{2\log_2 3}}{1+\log_3 5}=\dfrac{1+\dfrac{2}{a}}{1+b}=\dfrac{a+2}{a(b+1)}$.
	}
\end{ex}
\begin{ex}
	Cho $a$, $b$ là các số thực dương thỏa $\log_4a+\log_4b^2=5$ và $\log_4a^2+\log_4b=7$ thì tích $ab$ nhận giá trị bằng
	\choice
	{$16$}
	{\True $2^8$}
	{$2^9$}
	{$2^{18}$}
	\loigiai{
		Ta có $\log_4a+\log_4b^2=5 \Leftrightarrow ab^2=4^5$. \hfill (1)\\
		Mặt khác $\log_4a^2+\log_4b=7 \Leftrightarrow a^2b =4^7$. \hfill (2)\\
		Từ $ (1) $ và $ (2) $ suy ra $ a^3\cdot b^3  = 4^{12}\Leftrightarrow a\cdot b =4^4=2^8$.
	}
\end{ex}

\begin{ex}
	Cho $\log_{18} 6=\dfrac{a+\log_3 2}{b+\log_3 2}$, với $a$, $b$ là các số nguyên. Giá trị của $a+b$ bằng
	\choice
	{$4$}
	{$5$}
	{\True $3$}
	{$2$}
	\loigiai{
		Ta có $\log_{18} 6=\dfrac{\log_{3}6}{\log_{3}18}=\dfrac{\log_{3}\left(2\cdot 3\right)}{\log_{3}\left(2\cdot 3^2\right)}=\dfrac{\log_{3}2+\log_{3}3}{\log_{3}2+\log_{3}3^2}=\dfrac{1+\log_{3}2}{2+\log_{3}2}$.\\
		Suy ra $a=1$ và $b=2$.\\
		Vậy $a+b=3$.
	}
\end{ex}
\begin{ex}
	Biết $\log a=b$ và $\ln 10=c$. Giá trị của $\log _{10 e}(10 a)$ bằng
	\choice
	{$\dfrac{ab+b}{1+c}$}
	{$\dfrac{ab+a}{1+c}$}
	{$\dfrac{bc+b}{1+c}$}
	{\True $\dfrac{b c+c}{1+c}$}
	\loigiai{
		$\log a=b \Leftrightarrow a=10^b$.\\
		Ta có $T=\log _{10 e}(10 a)=\dfrac{\ln \left({10\cdot10^b}\right) }{\ln {(10e)}}=\dfrac{\ln 10+b.\ln 10}{\ln 10+\ln e}=\dfrac{c+bc}{c+1}$.}
\end{ex}
\begin{ex}
	Cho $\log3=a$, $\log2=b$. Khi đó giá trị của $\log_{125}30$ được tính theo $a$, $b$ là 
	\choice
	{\True $\dfrac{1+a}{3(1-b)}$}
	{$\dfrac{4(3-a)}{3-b}$}
	{$\dfrac{a}{3+b}$}
	{$\dfrac{a}{3+a}$}
	\loigiai{
		Ta có
		\begin{eqnarray*}
			\log_{125}30&=& \dfrac{\log 30}{\log 125}=\dfrac{1+\log 3}{3\log 5}\\
			&=& \dfrac{1+\log3}{3\left(1-\log 2\right)}=\dfrac{1+a}{3(1-b)}.
		\end{eqnarray*}
	}
\end{ex}
\begin{ex}
	Cho số thực $\alpha$ thỏa mãn $9^{\alpha}+9^{-\alpha}=23$. Giá trị của biểu thức $\dfrac{5+3^{\alpha}+3^{-\alpha}}{1-3^{-\alpha}-3^{\alpha}}$ bằng	
	\choice
	{$\dfrac{1}{2}$}
	{\True $-\dfrac{5}{2}$}
	{$\dfrac{3}{2}$}
	{$2$}
	\loigiai{
		Ta có: $9^{a}+9^{-a}=23 \Leftrightarrow 3^{2 a}+3^{-2 a}=23 \Leftrightarrow\left(3^{a}+3^{-a}\right)^{2}-2=23 \Leftrightarrow\left(3^{a}+3^{-a}\right)^{2}=25$\\
		$\Leftrightarrow 3^{a}+3^{-a}=5\left(\mbox{vì}\, 3^{a}+3^{-a}>0\right)$.\\
		Vậy $\dfrac{5+3^{a}+3^{-a}}{1-3^{-a}-3^{a}}=\dfrac{5+5}{1-5}=-\dfrac{5}{2}$.
	}
\end{ex}
\begin{ex}
	Cho $a>0$, $b>0$ và $a$ khác $1$ thỏa mãn $\log_{a}b=\dfrac{b}{4}$; $\log_{2}a=\dfrac{16}{b}$. Tính tổng $a+b$.
	\choice
	{$32$}
	{$16$}
	{\True $18$}
	{$10$}
	\loigiai{Ta có $\log_{2}a \cdot \log_{a}b=\dfrac{16}{b} \cdot \dfrac{b}{4} \Rightarrow \log_{2}b=4 \Rightarrow b=16$.\\
		Vậy $\log_{2}a=\dfrac{16}{16}=1\Rightarrow a=2 \Rightarrow a+b=18$.
	}
\end{ex}

\begin{ex}
	Cho $a$, $b$ là các số thực dương thỏa mãn $a^2+b^2=14ab$, biểu thức $\log _{2}(a+b)$ bằng
	\choice
	{$2\left(\log_2a+\log_2b\right)$}
	{\True $\dfrac{1}{2}\left(4+\log_2a+\log_2b\right)$}
	{$4+\dfrac{1}{2}\left(\log_2a+\log_2b\right)$}
	{$\left(4+\log_2a+\log_2b\right)$}
	\loigiai{
		\begin{eqnarray*}
			a^2+b^2=14ab&\Leftrightarrow& (a+b)^2=16ab\\
			&\Rightarrow& \log_2(a+b)^2=\log_2 (16ab)\\
			&\Rightarrow& 2\log_2(a+b)=4+\log_2 a+\log_2 b\\
			&\Rightarrow& \log_2 (a+b)=\dfrac{1}{2}(4+\log_2a+\log_2b)
		\end{eqnarray*}
	}
\end{ex}
\begin{ex}
	Giả sử $a$, $b$ là các số thực sao cho $x^3+y^3=a\cdot 10^{3z}+b\cdot 10^{2z}$ đúng với mọi số thực dương $x$, $y$, $z$ thỏa mãn $\log(x+y)=z$ và $\log(x^2+y^2)=z+1$. Giá trị của $a+b$ bằng
	\choice
	{$\dfrac{31}{2}$}
	{\True $\dfrac{29}{2}$}
	{$-\dfrac{31}{2}$}
	{$-\dfrac{25}{2}$}
	\loigiai{
		Ta có $\heva{&\log(x+y)=z\\&\log (x^2+y^2)=z+1}\Leftrightarrow \heva{&x+y=10^z\\&x^2+y^2=10^{z+1}=10\cdot 10^z}\Rightarrow x^2+y^2=10(x+y)$.\\
		Khi đó \begin{eqnarray*}
			& &x^3+y^3=a\cdot 10^{3z}+b\cdot 10^{2z}\Leftrightarrow (x+y)(x^2+y^2-xy)=a(10^z)^3+b(10^z)^2\\
			&\Leftrightarrow& (x+y)(x^2+y^2-xy)=a(x+y)^3+b(x+y)^2\Leftrightarrow x^2+y^2-xy=a(x+y)^2+b(x+y)\\
			&\Leftrightarrow& x^2+y^2-xy=a(x^2+2xy+y^2)+\dfrac{b}{10}(x^2+y^2)\Leftrightarrow x^2+y^2-xy=\left(a+\dfrac{b}{10}\right)(x^2+y^2)+2axy.
		\end{eqnarray*}
		Đồng nhất hệ số, ta được $\heva{&a+\dfrac{b}{10}=1\\&2a=-1}\Rightarrow \heva{&a=-\dfrac{1}{2}\\&b=15}$. Vậy $a+b=\dfrac{29}{2}$.
		
	}
\end{ex}

\begin{ex}
	Cho các số $a$, $b>0$, $a\neq 1$ thỏa mãn $\log_{ab} \dfrac{a}{b}=\dfrac{1}{3}$. Giá trị của $\log_{a^3} \left(ab^6\right)$ bằng
	\choice
	{$\dfrac{8}{3}$}
	{$\dfrac{13}{4}$}
	{$\dfrac{8}{9}$}
	{\True $\dfrac{4}{3}$}
	\loigiai{
		Ta có $\log_{ab} \dfrac{a}{b} = \log_{ab} a-\log_{ab} b \Leftrightarrow \dfrac{1}{3} = \dfrac{1}{1+\log_{a} b}-\dfrac{1}{\log_{b} a+1}$.\\
		Đặt $\log_{a} b=t$, khi đó ta có
		\allowdisplaybreaks
		\begin{eqnarray*}
			\dfrac{1}{1+t}-\dfrac{1}{\dfrac{1}{t}+1} = \dfrac{1}{3} &\Leftrightarrow & \dfrac{1}{1+t}-\dfrac{t}{1+t} = \dfrac{1}{3}\\
			&\Leftrightarrow & \dfrac{1-t}{1+t} = \dfrac{1}{3} \Rightarrow t=\dfrac{1}{2}.
		\end{eqnarray*}
		Vậy $\log_{a^3} \left(ab^6\right)=\log_{a^3} a+\log_{a^3} b^6=\dfrac{1}{3}+2\log_{a} b=\dfrac{1}{3}+2t=\dfrac{4}{3}$.
	}
\end{ex}
\begin{ex}
	Cho $x,y$ và $z$ là các số thực lớn hơn $1$ và gọi $w$ là số thực dương sao cho $\log_x{w}=24$, $\log_y{40}$ và $log_{xyz}{w}=12$. Tính $\log_z{w}$.
	\choice
	{$52$}
	{$-60$}
	{\True $60$}
	{$-52$}
	\loigiai
	{
		$\log_x{w}=24 \Rightarrow \log_w{x}=\dfrac{1}{24}$\\
		$\log_y{w}=40 \Rightarrow \log_w{y}=\dfrac{1}{40}$\\
		Lại do\\
		$\log_{xyz}{w}=12 \Leftrightarrow \dfrac{1}{\log_w{xyz}}=12 \Leftrightarrow \dfrac{1}{\log_w{x}+\log_w{y}+\log_w{z}}=12$
		$\Leftrightarrow \dfrac{1}{\dfrac{1}{24}+\dfrac{1}{40}+\log_w{z}}=12 \Leftrightarrow \log_w{z}=\dfrac{1}{60}$ $\Leftrightarrow \log_z{w}=60$.
	}
\end{ex}

\begin{ex}
	Cho các số thực dương $x$, $y$ thỏa mãn $\sqrt{\log x} + \sqrt{\log y} + \log \sqrt{x} + \log \sqrt{y} = 100$ và $\sqrt{\log x}$, $ \sqrt{\log y}$, $\log \sqrt{x}$,  $ \log \sqrt{y}$ là các số nguyên dương. Khi đó kết quả $xy$ bằng
	\choice
	{$10^{200}$}
	{$10^{100}$}
	{\True $10^{164}$}
	{$10^{144}$}
	\loigiai{ 
		Ta có 
		$$\sqrt{\log x} + \sqrt{\log y} + \log \sqrt{x} + \log \sqrt{y} = 100 \Leftrightarrow (\sqrt{\log x}+1)^2 + (\sqrt{\log y}+1)^2 = 202.$$
		Vì $\sqrt{\log x}$, $\sqrt{\log y}$ là các số nguyên dương nên $(\sqrt{\log x}+1)^2$ và $(\sqrt{\log y}+1)^2$ là các số nguyên dương.\\ Do đó cần phân tích $202$ thành tổng của $2$ số chính phương.\\
		Mặt khác, ta có $202=9^2+11^2$ và vai trò của $x$, $y$ như nhau nên ta chỉ cần xét trường hợp: 
		$$\heva{&(\sqrt{\log x}+1)^2 = 81 \\& (\sqrt{\log y}+1)^2 = 121}\Leftrightarrow \heva{&\sqrt{\log x}=8\\&\sqrt{\log y}=10}\Leftrightarrow \heva{&x = 10^{64} \\& y = 10^{100}} \Rightarrow xy=10^{164}.$$
	}
\end{ex}

\Closesolutionfile{ans}
% \begin{center}
% 	\fbox{\color{red}\fontfamily{qag}\bfseries\selectfont ĐÁP ÁN TRẮC NGHIỆM}
% \end{center}
% \inputansbox{10}{ans/Chuong6-2-logarit-Dang-2}
%%%%%%% Thầy Hữu Bình
\begin{dang}{Toán thực tế, liên môn}
	\begin{itemize}
		\item Chỉ số hay độ pH của một dung dịch được tính theo công thức: $\mathrm{pH}=-\log  \left[\mathrm{H}^{+}\right]$ với $\left[\mathrm{H}^{+}\right]$ là nồng độ ion hydrogen. Người ta đo được nồng độ ion hydrogen của một cốc nước cam là $10^{-4}$, nước dừa là $10^{-5}$ (nồng độ tính bằng mol $\mathrm{L}^{-1}$ ).
		\item {\bf Công thức lãi kép theo $N$ kì hạn} \\
		Nếu đem gửi ngân hàng một số vốn ban đầu là $P$ theo thể thức lãi kép với lãi suất hằng năm không đổi là $r$ và chia mỗi năm thành $m$ kì tính lãi thì sau $t$ năm (tức là sau $t m=N$ kì hạn) số tiền thu được (cả vốn lẫn lãi) là
		$$
		A_m=P\left(1+\dfrac{r}{m}\right)^{N} .
		$$
		\item {\bf Công thức lãi kép liên tục}\\
		Với số vốn ban đầu là $P$, theo thể thức lãi kép liên tục, lãi suất hằng năm không đổi là $r$ thì sau $t$ năm, số tiền thu được cả vốn lẫn lãi sẽ là
		$$
		A=P \mathrm{e}^{t r}.
		$$
	\end{itemize}
\end{dang}
\subsubsection{Ví dụ minh hoạ}
\begin{vd}%[Chân trời sáng tạo]%[1T6K2-3]
	Trong hóa học, độ  pH  của một dung dịch được tính theo công thức $\text{pH}=-\log[\text{H}^+] $, trong đó $ [\text{H}^+] $ là nồng độ  H$^+ $ (ion hydro) tính bằng mol/L. Các dung dịch có  pH  bé hơn $ 7 $ thì có tính acid, có pH  lớn hơn $ 7 $ thì có tính kiềm, có  pH  bằng $ 7 $ thì trung tính.
	\begin{listEX}[1]
		\item[a)] Tính độ  pH  của dung dịch có nồng độ H$^+ $ là $ 0{,}0001 $ mol/L. Dung dịch này có tính acid, hay kiềm hay trung tính?
		\item[b)] Dung dịch $ A $ có nồng độ  H$^+ $ gấp đôi nồng độ  H$^+ $ của dung dịch $ B $.\\
		Độ pH  của dung dịch nào lớn hơn và lớn hơn bao nhiêu? Làm tròn kết quả đến hàng phần nghìn.
	\end{listEX}
	\loigiai{
		\begin{listEX}[1]
			\item[a)] $ \text{pH}=-\log 0{,}0001=-\log 10^{-4} =4\log 10=4$.\\
			Do $ 4<7 $ nên dung dịch có tính acid.
			\item[b)] Kí hiệu pH$_A $,  pH$_B $ lần lượt là độ  pH  của hai dung dịch $ A $ và $ B $; $ [\text{H}^+]_A $, $ [\text{H}^+]_B $ lần lượt là nồng độ của hai dung dịch $ A $ và $ B $. Ta có
			$$\text{pH}_A=-\log[\text{H}^+]_A=-\log\left(2[\text{H}^+]_B\right)=-\log 2-\log[\text{H}^+]_B=-\log 2+\text{pH}_B.$$ 
			Suy ra $ \text{pH}_B-\text{pH}_A=\log 2\approx 0{,}301 $.\\
			Vậy dung dịch $ B $ có độ  pH  lớn hơn và lớn hơn khoảng $ 0{,}301 $.
		\end{listEX}
	}
\end{vd}

\begin{vd}%[Kết nối tri thức]%[1K6KI-3]
	Bác An gửi tiết kiệm ngân hàng $100$ triệu đồng kì hạn $12$ tháng, với lãi suất không đổi là $6 \%$ một năm. Khi đó sau $n$ năm gửi thì tổng số tiền bác An thu được (cả vốn lẫn lãi) cho bởi công thức sau:
	$$
	A=100 \cdot(1+0{,}06)^n \text { (triệu đồng). }
	$$
	Hỏi sau ít nhất bao nhiêu năm, tổng số tiền bác An thu được là không dưới $150$ triệu đồng?
	\loigiai{
		Ta có: $A=100 \cdot(1+0{,}06)^n=100 \cdot 1{,}06^n$.\\
		Với $A=150$, ta có: $100 \cdot 1{,}06^n=150$ hay $1{,}06^n=1{,}5$, tức là $n=\log _{1{,}06} 1{,}5 \approx 6{,}96$.\\
		Vì gửi tiết kiệm kì hạn $12$ tháng (tức là $1$ năm) nên $n$ phải là số nguyên.\\ Do đó ta chọn $n=7$.\\
		Vậy sau ít nhất $7$ năm thì bác An nhận được số tiền không dưới $150$ triệu đồng.
	}
\end{vd}
\begin{vd}% %[1C6T2-3]
	Một vi khuẩn có khối lượng khoảng $5 \cdot 10^{-13}$ gam và cứ 20 phút vi khuẩn đó tự nhân đôi một lần \textit{(Nguồn: Câu hỏi và bài tập vi sinh học, NXB ĐHSP, 2008)}. Giả sử các vi khuẩn được nuôi trong các điều kiện sinh trưởng tối ưu và mỗi con vi khuẩn đều tồn tại trong ít nhất 60 giờ. Hỏi sau bao nhiêu giờ khối lượng do tế bào vi khuẩn này sinh ra sẽ đạt tới khối lượng của Trái Đất (lấy khối lượng của Trái Đất là $6 \cdot 10^{27} \mathrm{gam}$) (làm tròn kết quả đến hàng đơn vị)?
	\loigiai{
		Số lần phân chia: $N=N_0 \cdot 2^n \Rightarrow n=\log_2\left(\dfrac{N}{N_0}\right)=\log_2\left(\dfrac{6 \cdot 10^{27}}{5 \cdot 10^{-13}}\right)=\log_2\left(1{,}2\cdot 10^{40}\right)\approx 133$.\\
		Thời gian cần thiết là: $133: 3\approx 44{,}3$ giờ.\\
		Vậy sau 45 giờ thì khối lượng do tế bào vi khuẩn này sinh ra sẽ đạt tới khối lượng của Trái Đất.
	}
\end{vd}
\subsubsection{Bài tập rèn luyện}
\begin{bt} %[1C6T2-3]
	Trong nuôi trồng thuỷ sản, độ $\mathrm{pH}$ của môi trường nước sẽ ảnh hưởng đến sức khoẻ và sự phát triển của thuỷ sản. Độ $\mathrm{pH}$ thích hợp cho nước trong đầm nuôi tôm sú là từ $7{,}2$ đến $8{,}8$ và tốt nhất là trong khoảng từ $7{,}8$ đến $8{,}5$. Phân tích nồng độ $\left[\mathrm{H}^{+}\right]$ trong một đầm nuôi tôm sú, ta thu được $\left[\mathrm{H}^{+}\right]=8 \cdot 10^{-8}$ \textit{(Nguồn: \href{https://nongnghiep.farmvina.com}{https://nongnghiep.farmvina.com})}. Hỏi độ pH của đầm đó có thích hợp cho tôm sú phát triển không?
	\loigiai{
		Độ pH của đầm là: $\mathrm{pH}=-\log  \left[\mathrm{H}^{+}\right]=-\log \left( 8\cdot 10^{-8}\right)\approx 7{,}1$.\\
		Do vậy, độ pH của đầm không thích hợp cho tôm sú phát triển.
	}
\end{bt}
\begin{bt}%[Manda La-Dự án tex sách 11-KNTT]%[1K6KI-3]
	Biết thời gian cần thiết (tính theo năm) để tăng gấp đôi số tiền đầu tư theo thể thức lãi kép liên tục với lãi suất không đổi $r$ mỗi năm được cho bởi công thức sau:
	$$
	t=\dfrac{\ln 2}{r}.
	$$
	Tính thời gian cần thiết để tăng gấp đôi một khoản đầu tư khi lãi suất là $6 \%$ mỗi năm (làm tròn kết quả đến chữ số thập phân thứ nhất).
	\loigiai{
		Ta có: $r=6 \%=0{,}06$. Do đó thời gian cần thiết để tăng gấp đôi khoản đầu tư là
		$$
		t=\dfrac{\ln 2}{r}=\dfrac{\ln 2}{0{,}06} \approx 11{,}6\, (\text {năm}) .
		$$
	}
\end{bt}
\begin{bt}%[Tex hóa SGK CTST,T1/23]%[1T6K2-3]
	Độ lớn $ M $ của một trận động đất theo thang Richter được tính theo công thức $ M=\log \dfrac{A}{A_0} $, trong đó $ A $ là biên độ lớn nhất ghi được bởi máy đo địa chấn, $ A_0 $ là biên độ tiêu chuẩn được sử dụng để hiệu chỉnh độ lệch gây ra bởi khoảng cách của máy đo địa chấn so với tâm chấn ($ \mathrm{A}_0=1 \mu m$).
	\begin{listEX}[1]
		\item[a)] Tính độ lớn của trận động đất có biên độ $A$ bằng
		\begin{listEX}[2]
			\item[i)] $ 10^{5{,}1}A_0 $;
			\item[ii)] $ 65~000 A_0$.  
		\end{listEX}
		\item[b)] Một trận động đất tại địa điểm $ N $ có biên độ lớn nhất gấp ba lần biên độ lớn nhất của trận động đất tại địa điểm $ P $. So sánh độ lớn của hai trận động đất.
	\end{listEX}
	\loigiai{
		\begin{listEX}[1]
			\item[a)] Độ lớn của trận động đất có biên độ $ A $ là
			\begin{listEX}[1]
				\item[i)] $A= 10^{5{,}1}A_0 \Rightarrow M=\log \dfrac{10^{5{,}1}A_0}{A_0}=5{,}1$;
				\item[ii)] $A= 65~000 A_0\Rightarrow M=\log \dfrac{65~000 A_0}{A_0} =\log (65\cdot 10^3)\approx 4{,}81 $.  
			\end{listEX}
			\item[b)] Gọi $ M_N $ và $ M_P $ lần lượt là độ lớn của các trận động đất tại địa điểm $ N $ và $ P $.\\
			Gọi $ A $ là biên độ lớn nhất ghi được bởi máy đo địa chấn tại địa điểm  $ P $. Ta có
			$$M_P=\log \dfrac{A}{A_0};\quad M_N=\log \dfrac{3A}{A_0}=\log 3+\log \dfrac{A}{A_0}.$$
			Do $ \log 3\approx 0{,}3>0 $ nên $ M_N>M_P $.
		\end{listEX}
	}
\end{bt}


\begin{bt}%[Tex hóa SGK CTST,T1/23]%[1T6B2-3]
	\hfill
	\begin{listEX}[1]
		\item[a)] Nước cất có nồng độ  H$^+ $ là $ 10^{-7} $ mol/L. Tính nồng độ pH của nước cất.
		\item[b)] Một dung dịch có nồng độ  H$^+ $ gấp $ 20 $ lần nồng độ  H$^+ $ của nước cất. Tính pH của dung dịch đó. 
	\end{listEX}
	\loigiai{
		\begin{listEX}[1]
			\item[a)] Ta có $ \text{pH}=-\log[\text{H}^+]=-\log 10^{-7} =7 $.
			\item[b)] Nồng độ $ \text{H}^+ $ của dung dịch là $ 20\cdot 10^{-7} $ mol/L. Độ pH của dung dịch là $$ \text{pH}=-\log[20\cdot 10^{-7}]\approx 5{,}7  .$$ 
		\end{listEX}
	}
\end{bt}
\begin{bt}%[1K6KI-3]
	Biết rằng khi độ cao tăng lên, áp suất không khí sẽ giảm và công thức tính áp suất dựa trên độ cao là $$ a=15500(5-\log p), $$ trong đó $a$ là độ cao so với mực nước biển (tính bằng mét) và $p$ là áp suất không khí (tính bằng pascal). Tính áp suất không khí ở đỉnh Everest có độ cao $8850 \mathrm{~m}$ so với mực nước biển.
	\loigiai{Đỉnh Everest có độ cao $8850 \mathrm{~m}$ so với mực nước biển suy ra
		$$8850=15500(5-\log p)\Leftrightarrow \log p=\dfrac{1373}{310}\Leftrightarrow p=26855{,}44\mathrm{~(pascal)}.$$
		Áp suất không khí ở đỉnh Everest là $p=26855{,}44\mathrm{~(pascal)}$.
	}
\end{bt}

\begin{bt}%[1K6KI-3]
	Mức cường độ âm $L$ đo bằng decibel ($\mathrm{dB}$) của âm thanh có cường độ $I$ (đo bằng oát trên mét vuông, kí hiệu là $\mathrm{W}/\mathrm{m}^2$) được định nghĩa như sau: $$ L(I)=10 \log \dfrac{I}{I_0}, $$  trong đó $I_0=10^{-12} \mathrm{~W}/\mathrm{m}^2$ là cường độ âm thanh nhỏ nhất mà tai người có thể phát hiện được (gọi là ngưỡng nghe). Xác định mức cường độ âm của mỗi âm sau:
	\begin{enumerate}
		\item Cuộc trò chuyện bình thường có cường độ $I=10^{-7} \mathrm{~W} / \mathrm{m}^2$.
		\item  Giao thông thành phố đông đúc có cường độ $I=10^{-3} \mathrm{~W} / \mathrm{m}^2$. 
	\end{enumerate}
	\loigiai{
		\begin{enumerate}
			\item Cuộc trò chuyện bình thường có cường độ $I=10^{-7} \mathrm{~W} / \mathrm{m}^2$ có mức cường độ âm là
			$$ L(I)=10 \log \dfrac{10^{-7}}{10^{-12}}=50~(\mathrm{dB}).$$ 
			\item Giao thông thành phố đông đúc có cường độ $I=10^{-3} \mathrm{~W} / \mathrm{m}^2$
			có mức cường độ âm là
			$$ L(I)=10 \log \dfrac{10^{-3}}{10^{-12}}=90~(\mathrm{dB}).$$
		\end{enumerate}
	}
\end{bt}
\subsubsection{Bài tập trắc nghiệm}
\Opensolutionfile{ans}[ans/Chuong6-2-logarit-Dang-3]
\begin{ex}
	Sự phân rã của các chất phóng xạ được biểu diễn theo công thức $m(t)=m_0\mathrm{e}^{-\lambda t}$, $t=\dfrac{\ln 2}{T}$, trong đó $m_0$ là khối lượng ban đầu của chất phóng xạ (tại thời điểm $t=0$), $m(t)$ là khối lượng chất phóng xạ tại thời điểm $t$, $T$ là chu kì bán rã ({\it tức là khoảng thời gian để một nửa khối lượng chất phóng xạ biến thành chất khác}). Khi phân tích một mẫu gỗ từ công trình kiến trúc cổ, các nhà khoa học thấy rằng khối lượng cacbon phóng xạ $^ { 14 }_6 C$ trong mẫu gỗ đã mất $45\%$ so với lượng $^ { 14 }_6 C$ ban đầu của nó. Hỏi công trình kiến thúc đó có niên đại khoảng bao nhiêu năm? Cho biết chu kì bán rã của $^ { 14 }_6 C$ là khoảng $5730$ năm.
	\choice
	{\True $4942$ (năm)}
	{$5157$ (năm)}
	{$3561$ (năm)}
	{$6601$ (năm)}
	\loigiai{
		Từ công thức $m(t)=m_0\mathrm{e}^{-\lambda t}$, $\lambda=\dfrac{\ln 2}{T}$ và $m(t)=0{,}55 m_0$ ta suy ra\\
		$0{,}55=\mathrm{e}^{-\frac{\ln 2}{5730}t}\Leftrightarrow 0{,}55=\left(\dfrac{1}{2}\right)^{\frac{t}{5730}}\Rightarrow t=5730\cdot \log_{\frac{1}{2}} 0{,}55 \approx  4942$ (năm).
	}
\end{ex}

\begin{ex}
	Biết rằng năm $2001$, dân số Việt Nam là $78685800$ người và tỉ lệ tăng dân số năm đó là $1{,}7\%$. Cho biết sự tăng dân số được ước tính theo công thức $S= A\cdot \mathrm{e}^{Nr}$ (trong đó $A$ là dân số của năm lấy làm mốc tính, $S$ là dân số sau $N$ năm, $r$ là tỉ lệ tăng dân số hàng năm). Hỏi nếu cứ tăng dân số với tỉ lệ như vậy thì đến năm nào dân số nước ta ở mức $120$ triệu người?
	\choice
	{$2020$}
	{$2022$}
	{$2025$}
	{\True $2026$}
	\loigiai{
		Ta có $S= A\cdot \mathrm{e}^{Nr}$ nên $N= \dfrac{1}{r} \ln \dfrac{S}{A}$.\\
		Để dân số nước ta ở mức $120$ triệu người thì cần số năm $N= \dfrac{100}{1{,}7}\cdot \ln \dfrac{120\cdot 10^6}{78685800} \approx 25$.\\
		Lúc đấy là năm $2001 + 25 = 2026$.
	}
\end{ex}

\begin{ex}
	Một người gửi số tiền $80$ triệu đồng vào một ngân hàng với lãi suất $6{,}2\%$/năm. Cứ sau mỗi năm, số tiền lãi sẽ sinh ra được nhập vào vốn ban đầu để tính lãi cho năm tiếp theo. Hỏi sau ít nhất bao nhiêu năm thì người đó sẽ lĩnh được số tiền cả vốn lẫn lãi là $100$ triệu đồng? (Giả thiết lãi suất không đổi trong suốt thời gian gửi.)
	\choice
	{$3$ năm}
	{$2$ năm}
	{\True $4$ năm}
	{$5$ năm}
	\loigiai{Số tiền người đó nhận được (gốc + lãi) theo công thức $T=A\cdot \left(1+r\right)^n $.\\
		Theo giả thiết $100=80\left(1+\dfrac{6{,}2}{100}\right)^n\Leftrightarrow \left(1{,}062\right)^n=\dfrac{5}{4}\Leftrightarrow n=\log_{1{,}062} \left(\dfrac{5}{4}\right)\approx 3{,}7$.\\
		Vậy cần ít nhất $4$ năm để người đó nhận được $100$ triệu.}
\end{ex}

\begin{ex}
	Một người gửi $50$ triệu đồng vào một ngân hàng với lãi suất $6\%$/ năm. Biết nếu không rút tiền khỏi ngân hàng thì cứ mỗi năm số tiền lãi sẽ được cộng vào gốc để tính lãi cho năm tiếp theo. Hỏi sau ít nhất bao nhiêu năm thì tổng số tiền cả gốc lẫn lãi của người đó nhiều là $100$ triệu? Giả định trong suốt thời gian gửi, lãi suất không thay đổi và người đó không rút tiền ra.
	\choice
	{$14$ năm}
	{$11$ năm }
	{\True $12$ năm}
	{$13$ năm}
	\loigiai{
		Gọi $S_n$ là số tiền cả gốc lẫn lãi sau khi hết năm thứ $n$.\\
		Ta có $S_n=50(1+0{,}06)^n$ triệu đồng. Yêu cầu bài toán tương đương
		\[S_n= 100\Leftrightarrow 50\cdot(1{,}06)^n= 100\Leftrightarrow n= \log_{1{,}06}2\approx 11{,}8.\]
		Vậy là $n=12$.
	}
\end{ex}

\begin{ex}
	Một người gửi tiết kiệm với lãi suất $8{,}4\%$/năm và lãi hàng năm được nhập vào vốn. Sau bao nhiêu năm người đó thu được gấp đôi số tiền ban đầu?
	\choice
	{\True $9$}
	{$6$}
	{$8$}
	{$7$}
	\loigiai
	{
		Số tiền gửi ban đầu $A$ đồng với lãi suất $r$\%, sau $n$ năm được tính theo công thức $P_n=A(1+r)^n$.\\
		Theo đề bài ta có phương trình
		\begin{eqnarray*}
			& & A(1+r)^n=2A\\
			& \Leftrightarrow & 1{,}084^n=2\\
			& \Leftrightarrow & n=\log_{1{,}084} 2\approx 8{,}594.
		\end{eqnarray*}
		Vậy sau $9$ năm người gửi thu được gấp đôi số tiền ban đầu.
	}
\end{ex}
\begin{ex}
	Cường độ một trận động đất $M$ (richter) được cho bởi công thức $M=\log A-\log A_0$, với $A$ là biên độ rung chấn tối đa và $A_0$ là một biên độ chuẩn (hằng số). Đầu thế kỷ $20$, một trận động đất ở San Francisco có cường độ $8{,}3$ độ richter. Trong cùng năm đó, trận động đất khác ở Nam Mỹ có biên độ rung chấn tối đa gấp $4$ lần biên độ rung chấn tối đa của trận động đất ở San Francisco. Tính cường độ của trận động đất ở Nam Mỹ (làm tròn đến $1$ chữ số thập phân).
	\choice
	{$33{,}2$ richter}
	{$12{,}3$ richter}
	{ \True $8{,}9$ richter}
	{$2{,}1$ richter}
	\loigiai{Do trận động đất khác ở Nam Mỹ có biên độ rung chấn tối đa gấp $4$ lần biên độ rung chấn tối đa của trận động đất ở San Francisco nên cường độ của trận động đất ở đó là $$\log 4A - \log A_0 = \log 4 + \log A - \log A_0 =\log 4 + M \approx 8{,}9.$$	
	}
\end{ex}

\begin{ex}
	Một người gửi tiết kiệm $ 200 $ triệu đồng với lãi suất $5 \%$ một năm và lãi hàng năm được nhập vào vốn. Sau $9$ năm nhận được số tiền cả gốc và lãi nhận được là bao nhiêu?
	\choice
	{$210,55$ triệu}
	{\True $310,27$ triệu }
	{$300$ triệu}
	{$352,58$ triệu}
	\loigiai
	{
		Sau $9$ năm, số tiền nhận được là
		$200\cdot 10^6(1+5\%)^9=310,27$ triệu.
	}
\end{ex}

\begin{ex}
	Áp suất không khí $ P $(đo bằng milimet thủy ngân, kí hiệu là mmHg) suy giảm mũ so với độ cao $ x $(đo bằng mét), tức là $ P $ giảm theo công thức $ P=P_0\cdot \text{e}^{xi}$, trong đó $ P_0=760 $ mmHg là áp suất của mực nước biển $(x=0)$, $ i $ là hệ số suy giảm. Biết rằng ở độ cao $ 1000 $ m thì áp suất của không khí là $ 672{,}71 $ mmHg. Hỏi áp suất không khí ở độ cao $ 3000 $ m gần bằng số nào dưới đây nhất?
	\choice
	{$530{,}23 $ mmHg}
	{\True $527{,}06 $ mmHg}
	{$554{,}38 $ mmHg}
	{$428{,}2 $ mmHg}
	\loigiai{
		Ở độ cao $ 1000 $ m, áp suất của không khí là $ 672{,}71 $ mmHg. Thay vào công thức, ta có
		\[672{,}71=760\cdot \text{e}^{1000i}\Leftrightarrow 1000i=\ln \dfrac{672{,}71}{760}\Leftrightarrow i=\dfrac{1}{1000}\ln \dfrac{672{,}71}{760}.\]
		Từ đó suy ra áp suất không khí ở độ cao $ 3000 $ m là
		\[P=760\cdot \text{e}^{3000i}=760\cdot \text{e}^{3\ln \tfrac{672{,}71}{760}}\approx 527{,}06.\]
	}
\end{ex}

\begin{ex}
	Một nguồn âm đẳng hướng phát ra từ điểm $O$. Mức cường độ âm tại điểm $M$ cách $O$ một khoảng $R$ được tính bởi công thức $L_M =\log\dfrac{k}{R^2}$ (Ben), với $k>0$ là hằng số. Biết điểm $O$ thuộc đoạn thẳng $AB$ và mức cường độ âm thanh tại $A$ và $B$ lần lượt là $L_A=4{,}3$ (Ben) và $l_B =5$ (Ben). Tính mức cường độ âm tại trung điểm của $AB$ (làm tròn đến hai chữ số thập phân).
	\choice
	{$4{,}65$ (Ben)}
	{$4{,}58$ (Ben)}
	{\True $5{,}42$ (Ben)}
	{$9{,}40$ (Ben)}
	\loigiai{
		Ta có $4{,}3=\log\dfrac{k}{R_A^2} \Rightarrow R_A=OA  =\sqrt{\dfrac{k}{10^{4{,}3}}}$ và $5=\log\dfrac{k}{R_B^2} \Rightarrow R_B = OB =\sqrt{\dfrac{k}{10^{5}}}$.\\
		Do $O$ thuộc đoạn $AB$ nên $AB =R_A+R_B$. 
		\immini{
			Gọi $I$ là trung điểm của $AB$.
		}{
			\begin{tikzpicture}[scale=1, font=\footnotesize, line join=round, line cap=round, >=stealth]
				\path
				(0,0) coordinate (A)
				(5,0) coordinate (B)
				($(A)!0.5!(B)$) coordinate (I)
				($(A)!0.7!(B)$) coordinate (O)
				;
				\draw (A)--(B);
				\foreach \x/\g in
				{A/90,B/90,O/90,I/90}
				\fill[black](\x) circle (1.3pt)
				($(\x)+(\g:3mm)$) node{$\x$};
			\end{tikzpicture}
		}
		\noindent
		Không mất tính tổng quát, giả sử $O$ gần $B$ hơn so với $A$. Khi đó,
		\[R_I=OI=OA-IA=R_A-\dfrac{R_A+R_B}{2}=\dfrac{R_A-R_B}{2}=\dfrac{\sqrt{k}}{2}\left(\dfrac{1}{\sqrt{10^{4{,}3}}}-\dfrac{1}{\sqrt{10^5}}\right).\]
		Tính được $L_I=\log \dfrac{k}{R_I^2}=\log \dfrac{4}{\left(\dfrac{1}{\sqrt{10^{4{,}3}}}-\dfrac{1}{\sqrt{10^5}}\right)^2}\approx 5{,}42$ (Ben).
	}	
\end{ex}
\Closesolutionfile{ans}
% \begin{center}
% 	\fbox{\color{red}\fontfamily{qag}\bfseries\selectfont ĐÁP ÁN TRẮC NGHIỆM}
% \end{center}
% \inputansbox{10}{ans/Chuong6-2-logarit-Dang-3}
