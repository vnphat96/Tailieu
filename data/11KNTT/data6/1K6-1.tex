\setcounter{chapter}{5}
\setcounter{section}{17}
\setcounter{subsubsection}{0}
\setcounter{ex}{0}
\setcounter{bt}{0}
\chap{Hàm số mũ. Hàm số logarit}
\section{Phép tính lũy thừa với số mũ thực}

\Opensolutionfile{ans}[ANS/ans-Chuong6-Bai1]

\subsection{Tóm tắt lý thuyết}
\begin{tomtat}
	\subsubsection{Lũy thừa}
	\begin{enumerate}
		\item \textbf{Lũy thừa với số mũ nguyên dương:}
		\begin{itemize}
			\item $a$ là số thực tùy ý, $n$ nguyên dương: $a^n=\underbrace{a\cdot a\cdots a}_{n \text{ chữ số}}.$
			\item $a\ne 0$: $a^0=1$, $a^{-n}=\dfrac{1}{a^n}$.
		\end{itemize}
		\item \textbf{Căn bậc $n$ ($n$ nguyên dương):} của số $a$ là $b$ thỏa mãn $b^n=a$.
		\item \textbf{Lũy thừa với số mũ hữu tỉ:} $a^{\frac{m}{n}}=\sqrt[n]{a^m}$.
		\item \textbf{Lũy thừa với số mũ thực:} $a^{\alpha}=\lim\limits_{n \to +\infty} a^{r_n}$, với $\lim\limits_{n \to +\infty} r_n=\alpha$.
	\end{enumerate}
	\subsubsection{Tính chất lũy thừa}
	\begin{itemize}
		\item Với $a\ne 0$, $b\ne 0$, $m, n$ là số thực:
		\begin{eqnarray*}
			&a^m\cdot a^n=a^{m+n}; &\left(a^m\right)^n=a^{mn};\\
			&\left(\dfrac{a}{b}\right)^m=\dfrac{a^m}{b^m}; &\dfrac{a^m}{a^n}=a^{m-n};\\
			&(ab)^m=a^m\cdot b^m; &
		\end{eqnarray*}
		\item Với $n, k$ nguyên dương, $m$ là số nguyên:
		\begin{eqnarray*}
			& \sqrt[n]{a}\cdot \sqrt[n]{b}=\sqrt[n]{ab};& \dfrac{\sqrt[n]{a}}{\sqrt[n]{b}}=\sqrt[n]{\dfrac{a}{b}}\\
			&\left(\sqrt[n]{a}\right)^m=\sqrt[n]{a^m};&\sqrt[n]{\sqrt[k]{a}}=\sqrt[nk]{a};\\
			&\sqrt[n]{a^n}=\heva{&a&\text{ khi } n \text{ lẻ}\\&\left|a\right|&\text{ khi } n \text{ chẵn.}}&
		\end{eqnarray*}
	\end{itemize}
\end{tomtat}
\setcounter{subsubsection}{0}
\setcounter{ex}{0}
\setcounter{bt}{0}
\subsection{Các dạng toán thường gặp}
\begin{dang}{Tính giá trị biểu thức chứa lũy thừa}
	Biến đổi các cơ số về nguyên tố, sử dụng các công thức để rút gọn và tính giá trị biểu thức.
\end{dang}
\subsubsection{Ví dụ minh hoạ}
\begin{vd} 
	Tính giá trị các biểu thức
	\begin{listEX}[3]
		\item $2^{-4}$;
		\item $9\cdot \left(\dfrac{3}{4}\right)^{-2}$;
		\item $\left(\dfrac{1}{2}\right)^{-2}:\left(\sqrt{3}\right)^0$.
	\end{listEX}
	\loigiai{
	\begin{enumerate}
		\item $2^{-4}=\dfrac{1}{2^4}=\dfrac{1}{16}$;
		\item $9\cdot \left(\dfrac{3}{4}\right)^{-2}=9\cdot \dfrac{1}{\left(\dfrac{3}{4}\right)^2}=9\cdot \dfrac{1}{\dfrac{9}{16}}=9\cdot \dfrac{16}{9}=16$;
		\item $\left(\dfrac{1}{2}\right)^{-2}:\left(\sqrt{3}\right)^0=\dfrac{1}{\left(\dfrac{1}{2}\right)^2}:1=\dfrac{1}{\dfrac{1}{4}}=4$.
	\end{enumerate}
	}
\end{vd}
\begin{vd} 
	Tính giá trị của biểu thức $$A=\left(\dfrac{1}{2}\right)^{-12}\cdot 8^{-3}+(0{,}2)^{-4}\cdot 25^{-2}+243^{-1}\cdot \left(\dfrac{1}{3}\right)^{-6}.$$
	\loigiai{
	Ta có
	\begin{eqnarray*}
		A&=& \left(\dfrac{1}{2}\right)^{-12}\cdot 8^{-3}+(0{,}2)^{-4}\cdot 25^{-2}+243^{-1}\cdot \left(\dfrac{1}{3}\right)^{-6}\\
		&=& 2^{12}\cdot \dfrac{1}{8^3}+\left(\dfrac{1}{5}\right)^{-4}\cdot \dfrac{1}{25^2}+\dfrac{1}{243^1}\cdot 3^6\\
		&=& \dfrac{2^{12}}{2^9}+\dfrac{5^4}{5^4}+\dfrac{3^6}{3^5}\\
		&=&2^3+1+3=12.
	\end{eqnarray*}
	}
\end{vd}
\begin{vd} 
	Tính giá trị biểu thức $$A=\left(\dfrac{1}{2}\right)^{-8}\cdot 8^{-2}+(0{,}2)^{-4}\cdot 25^{-2}.$$
	\loigiai{
	Ta có
	\begin{eqnarray*}
		A&=& 2^8\cdot \dfrac{1}{8^2}+\dfrac{1}{0{,}2^4}\cdot \dfrac{1}{25^2}\\
		&=&2^8\cdot \dfrac{1}{2^6}+\dfrac{1}{0{,}2^4\cdot 5^4}\\
		&=& 2^2+\dfrac{1}{\left(0{,}2\cdot 5\right)^4}=4+1=5.
	\end{eqnarray*}
	}
\end{vd}
\subsubsection{Bài tập rèn luyện}
\setcounter{bt}{0}
\begin{bt}
	Tính giá trị các biểu thức
	\begin{listEX}[2]
		\item $\left(\dfrac{3}{4}\right)^{-2}\cdot 3^2\cdot 12^0$;
		\item $\left(2^{-2}\cdot 5^2\right)^{-2}:\left(5\cdot 5^{-5}\right)$.
	\end{listEX}
	\loigiai{
	\begin{enumerate}
		\item $\left(\dfrac{3}{4}\right)^{-2}\cdot 3^2\cdot 12^0=\left(\dfrac{4}{3}\right)^2\cdot 3^2\cdot 1=\dfrac{4^2}{3^2}\cdot 3^2=4^2=16$.
		\item $\left(2^{-2}\cdot 5^2\right)^{-2}:\left(5\cdot 5^{-5}\right)
		=2^4\cdot 5^{-4}:5^{-4}=2^4=16$.
	\end{enumerate}
	}
\end{bt}
\begin{bt}
	Tính giá trị các biểu thức
	\begin{listEX}[3]
		\item $\left(\dfrac{1}{256}\right)^{-0{,}75}+\left(\dfrac{1}{27}\right)^{-\frac{4}{3}}$;
		\item $\left(\dfrac{1}{49}\right)^{-1{,}5}-\left(\dfrac{1}{125}\right)^{-\frac{2}{3}}$;
		\item $\left(4^{3+\sqrt{3}}-4^{\sqrt{3}-1}\right)\cdot 2^{-2\sqrt{3}}$.
	\end{listEX}
	\loigiai{
	\begin{enumerate}
		\item Ta có $$\left(\dfrac{1}{256}\right)^{-0{,}75}+\left(\dfrac{1}{27}\right)^{-\frac{4}{3}}
		=\left(\dfrac{1}{2^8}\right)^{-0{,}75}+\left(\dfrac{1}{3^3}\right)^{-\frac{4}{3}}
		=\left(2^{-8}\right)^{-\frac{3}{4}}+\left(3^{-3}\right)^{-\frac{4}{3}}=2^6+3^4=64+81=145.$$
		\item $\left(\dfrac{1}{49}\right)^{-1{,}5}-\left(\dfrac{1}{125}\right)^{-\frac{2}{3}}
		=\left(\dfrac{1}{7^2}\right)^{-\frac{3}{2}}-\left(\dfrac{1}{5^3}\right)^{-\frac{2}{3}}
		=\left(7^{-2}\right)^{-\frac{3}{2}}-\left(5^{-3}\right)^{-\frac{2}{3}}=7^3-5^2=318$.
		\item $\left(4^{3+\sqrt{3}}-4^{\sqrt{3}-1}\right)\cdot 2^{-2\sqrt{3}}=\left(2^{6+2\sqrt{3}}-2^{2\sqrt{3}-2}\right)\cdot 2^{-2\sqrt{3}}=2^{6}-2^{-2}=64-\dfrac{1}{4}=\dfrac{255}{4}$.
	\end{enumerate}	
	}
\end{bt}
\begin{bt}
	Thực hiện các phép tính
	\begin{enumerate}
		\item $27^{\frac{2}{3}}+81^{-0{,}75}-25^{0{,}5}$;
		\item $4^{2-3\sqrt{7}}\cdot 8^{2\sqrt{7}}$.
	\end{enumerate}
	\loigiai{
		\begin{enumerate}
			\item \allowdisplaybreaks
			\begin{eqnarray*}
				27^{\tfrac{2}{3}}+81^{-0{,}75}-25^{0{,}5}&=&(3^3)^{\tfrac{2}{3}}+(3^4)^{-0{,}75}-(5^2)^{0{,}5}\\
				&=&3^{3\cdot\tfrac{2}{3}}+3^{4\cdot (-0{,}75)}-5^{2\cdot (0{,}5)}\\
				&=&3^2+3^{-3}-5^1\\
				&=&\dfrac{109}{27}.
			\end{eqnarray*}
			\item \allowdisplaybreaks
			\begin{eqnarray*}
				4^{2-3 \sqrt{7}} \cdot 8^{2 \sqrt{7}}&=&(2^2)^{2-3 \sqrt{7}} \cdot (2^3)^{2 \sqrt{7}}\\
				&=&2^{2\left(2-3 \sqrt{7}\right)} \cdot 2^{6 \sqrt{7}}\\
				&=&2^{4-6\sqrt{7}+6\sqrt{7}}\\
				&=&16.
			\end{eqnarray*}
		\end{enumerate}
	}
\end{bt}
\begin{bt}
	Biết $4^a=\dfrac{1}{5}$. Tính giá trị các biểu thức
	\begin{listEX}[2]
		\item $16^{\alpha}+16^{-\alpha}$;
		\item $\left(2^{\alpha}+2^{-\alpha}\right)^2$.
	\end{listEX}
	\loigiai{
	\begin{enumerate}
		\item $16^\alpha+16^{-\alpha}=\left(4^\alpha\right)^2+\dfrac{1}{\left(4^\alpha\right)^2}=\left(\dfrac{1}{5}\right)^2+\dfrac{1}{\left(\dfrac{1}{5}\right)^2}=\dfrac{1}{25}+25=\dfrac{626}{25}$.
		\item $\left(2^\alpha+2^{-\alpha}\right)^2=4^\alpha+2+4^{-\alpha}=\dfrac{1}{5}+2+\dfrac{1}{\dfrac{1}{5}}=\dfrac{36}{5}$.
	\end{enumerate}	
	}
\end{bt}
\begin{bt}%[1C6T1-1]
	Định luật thứ ba của Kepler về quỹ đạo chuyển động cho biết cách ước tính khoảng thời gian $P$ (tính theo năm Trái Đất) mà một hành tinh cần để hoàn thành một quỹ đạo quay quanh Mặt Trời. Khoảng thời gian đó được xác định bởi một hàm số $P=d^{\frac{3}{2}}$, trong đó $d$ là khoảng cách từ hành tinh đó đến Mặt Trời tính theo đơn vị thiên văn AU ($1$ AU là khoảng cách từ Trái Đất đến Mặt Trời, tức là $1$ AU khoảng $93\,000\,000$ dặm) (\textit{Nguồn: R.I.Charles et al.,Algebra 2, Pearson}). Hỏi Sao Hỏa quay quanh Mặt Trời thì mất bao nhiêu năm Trái Đất (làm tròn kết quả đến hàng phần trăm)? Biết khoảng cách từ Sao Hỏa đến Mặt Trời là $1{,}52$ AU.
	\loigiai{
		Sao Hỏa quay quanh Mặt Trời thì mất thời gian $P=1{,}52^{\frac{3}{2}}\approx 1{,}87$ AU.
	}
\end{bt}
\begin{bt}%[1K6TH-1]
	Nếu một khoản tiền gốc $P$ được gửi ngân hàng với lãi suất hằng năm $r$ ($r$ được biểu thị dưới dạng số thập phân), được tính lãi $n$ lần trong một năm, thì tổng số tiền $A$ nhận được (cả vốn lẫn lãi) sau $N$ kì gửi cho bởi công thức sau:
	$$
	A=P\left(1+\dfrac{r}{n}\right)^N .
	$$
	Hỏi nếu bác An gửi tiết kiệm số tiền $120$ triệu đồng theo kì hạn $6$ tháng với lãi suất không đổi là $5 \%$ một năm, thì số tiền thu được (cả vốn lẫn lãi) của bác An sau $2$ năm là bao nhiêu?
	\loigiai{
		Ta có $2$ năm là $24$ tháng ứng với $N=4$ kì hạn.\\
		Do kì hạn là $6$ tháng nên mỗi năm được tính lãi $n=2$ lần.\\
		Vậy số tiền cả vốn lẫn lãi bác An nhận được sau $2$ năm là $A=120\left(1+\dfrac{0{,}05}{2}\right)^4\approx 132{,}457$ triệu đồng.
	}
\end{bt}
\begin{bt}%[1K6TH-1]
	Năm $2021$, dân số của một quốc gia ở châu Á là $19$ triệu người. Người ta ước tính rằng dân số của quốc gia này sẽ tăng gấp đôi sau $30$ năm nữa. Khi đó dân số $A$ (triệu người) của quốc gia đó sau $t$ năm kể từ năm $2021$ được ước tính bằng công thức $A=19 \cdot 2^{\tfrac{t}{30}}$. Hỏi với tốc độ tăng dân số như vậy thì sau $20$ năm nữa dân số của quốc gia này sẽ là bao nhiêu? (Làm tròn kết quả đến chữ số hàng triệu).
	\loigiai{
		Dân số của quốc gia này sau $20$ năm là $A=19\cdot 2^{\tfrac{20}{30}}\approx 30$ triệu người.
	}
\end{bt}
\begin{bt}%[1K6TH-1]
	Với một chỉ vàng, giả sử người thợ lành nghề có thể dát mỏng thành lá vàng rộng $1$ m$^2$ và dày khoảng $1{,}94 \cdot 10^{-7}$ m. Đồng xu $5\ 000$ đồng dày $2{,}2 \cdot 10^{-3}$ m. Cần chồng bao nhiêu lá vàng như trên để có độ dày bằng đồng xu loại $5\ 000$ đồng? Làm tròn kết quả đến chữ số hàng trăm.
	\loigiai{
		Số lá vàng cần chồng là
		\[ \dfrac{2{,}2\cdot 10^{-3}}{1{,}94 \cdot 10^{-7}}\approx11\ 300. \]
	}
\end{bt}
\subsubsection{Bài tập trắc nghiệm}
	\setcounter{ex}{0}
	\begin{ex}
		Cho số dương $a$ và $m,n\in \mathbb{R}$. Mệnh đề nào sau đây đúng?
		\choice
		{$a^m\cdot a^n= \left( a^m\right)^n $}
		{\True $a^m\cdot a^n=a^{m+n}$}
		{$a^m\cdot a^n=a^{mn}$}
		{$a^m\cdot a^n=a^{m-n}$}
		\loigiai{
			Từ công thức hàm lũy thừa thì công thức đúng là $a^m\cdot a^n=a^{m+n}$.
		}
	\end{ex}
	\begin{ex}
		Cho $0<a\neq 1$ và các số thực $\alpha$, $\beta$. Khẳng định nào sau đây là khẳng định \textbf{sai}?
		\choice
		{$\dfrac{a^\alpha}{a^\beta}=a^{\alpha-\beta}$}
		{$\left(a^\alpha\right)^\beta=a^{\alpha\beta}$}
		{\True $a^\alpha\cdot a^\beta=a^{\alpha\beta}$}
		{$a^\alpha\cdot a^\beta=a^{\alpha+\beta}$}
		\loigiai{
			Theo tính chất lũy thừa ta suy ra mệnh đề $a^\alpha\cdot a^\beta=a^{\alpha\beta}$ là sai.
		}	
	\end{ex}
	\begin{ex}
		Cho $a$ là số thực tùy ý, $\left(a^{3}\right)^{2}$ bằng
		\choice
		{$a^5$}
		{$a$}
		{$a^9$}
		{\True $a^6$}
		\loigiai{
			Cho $a$ là số thực tùy ý, ta có $\left(a^{3}\right)^{2}=a^{3\cdot2}=a^{6}$.
		}
	\end{ex}
	\begin{ex}
		Cho các số nguyên dương $m$, $n$ và số thực dương $a$. Mệnh đề nào sau đây \textbf{sai}?
		\choice
		{$\sqrt[m]{\sqrt[n]{a}}=\sqrt[nm]{a}$}
		{$\sqrt[n]{a}\cdot\sqrt[m]{a}=\sqrt[mn]{a^{m+n}}$}
		{\True $\sqrt[n]{a}\cdot\sqrt[m]{a}=\sqrt[n+m]{a}$}
		{$(\sqrt[n]{a})^m=\sqrt[n]{a^m}$}
		\loigiai{
			Ta có
			\begin{itemize}
				\item $(\sqrt[n]{a})^m=\sqrt[n]{a^m}$.
				\item $\sqrt[m]{\sqrt[n]{a}}=\sqrt[nm]{a}$.
				\item $\sqrt[n]{a}\cdot\sqrt[m]{a}=a^{\tfrac{1}{n}}\cdot a^{\tfrac{1}{m}}=a^{\tfrac{1}{n}+\tfrac{1}{m}}=a^{\tfrac{m+n}{mn}}=\sqrt[mn]{a^{m+n}}$.
			\end{itemize}
		}
		
	\end{ex}
	\begin{ex}
		Tính giá trị của biểu thức $A=\dfrac{6^{3+\sqrt{5}}}{2^{2+\sqrt{5}}\cdot 3^{1+\sqrt{5}}}$.
		\choice
		{$6^{-\sqrt{5}}$}
		{$9$}
		{\True $18$}
		{$1$}
		\loigiai{
			Ta có $A=\dfrac{2^{3+\sqrt{5}}\cdot 3^{3+\sqrt{5}}}{2^{2+\sqrt{5}}\cdot 3^{1+\sqrt{5}}}=2\cdot 3^2=18$.
		}		
	\end{ex}
	\begin{ex}
		Tính giá trị của biểu thức $P = 4^4 \cdot 8^{11} \cdot 2^{2017}$.
		\choice
		{$ P = 2^{2407} $}
		{$ P = 2^{2054} $}
		{\True $ P = 2^{2058} $}
		{$ P = 2^{2032} $}
		\loigiai{
			Ta có $ P = 4^4 \cdot 8^{11} \cdot 2^{2017} = 2^8 \cdot 2^{33} \cdot 2^{2017} = 2^{2058} $.
		}
	\end{ex}
	
	\begin{ex}
		Tính giá trị của biểu thức $P=3^{10}\cdot 27^{-3}+0{,}2^{-4}\cdot 25^{-2}+128^{-1}\cdot 2^9+0{,}1^{-5}\cdot 0{,}2^5$.
		\choice
		{$P=32$}
		{\True $P=40$}
		{$P=30$}
		{$P=38$}
		\loigiai{
			Ta có $P=3^{10}\cdot 3^{-9}+0{,}2^{-4}\cdot 5^{-4}+2^{-7}\cdot 2^9+10^5\cdot 0{,}2^5=3^1+1^{-4}+2^2+2^5=40$.
		}	
	\end{ex}
	\begin{ex}
		Cho số thực $a>1$. Nếu $a^{3x}=2$ thì $2a^{9x}$ bằng
		\choice
		{$6$}
		{$12$}
		{\True $16$}
		{$8$}
		\loigiai{
			$$a^{3x}=2\Leftrightarrow a^{9x}=2^3\Leftrightarrow 2a^{9x}=2\cdot8=16.$$
		}
	\end{ex}
	\begin{ex}
		Giá trị của biểu thức $A=\left( 2+\sqrt{3}\right) ^{2019}\left( 2-\sqrt{3}\right) ^{2020}$ bằng
		\choice
		{$A=1$}
		{\True $A=2-\sqrt{3}$}
		{$A=\left( 2-\sqrt{3}\right) ^{2019}$}
		{$A=2+\sqrt{3}$}
		\loigiai{
			Ta có $A=\left( 2+\sqrt{3}\right) ^{2019}\left( 2-\sqrt{3}\right) ^{2020}=\left[ \left( 2+\sqrt{3}\right) \left( 2-\sqrt{3}\right)\right]  ^{2019}(2-\sqrt{3})=2-\sqrt{3}.$
		}
	\end{ex}
	\begin{ex}
		Tính giá trị của biểu thức $P=\left( 7+4\sqrt{3}\right) ^{2020}\left( 4\sqrt{3}-7\right) ^{2019}$.
		\choice
		{$P= 7+4\sqrt{3}$}
		{$P= 7-4\sqrt{3}$}
		{$P=1$}
		{\True$P=-7-4\sqrt{3}$}
		\loigiai{
			\begin{eqnarray*}
				P&=&\left( 7+4\sqrt{3}\right) ^{2020}\left( 4\sqrt{3}-7\right) ^{2019}=\left( 7+4\sqrt{3}\right) \left( 7+4\sqrt{3}\right) ^{2019}\left( 4\sqrt{3}-7\right) ^{2019}\\
				&=&\left( 7+4\sqrt{3}\right) \left[ \left( 7+4\sqrt{3}\right) \left( 4\sqrt{3}-7\right)\right]  ^{2019}=\left( 7+4\sqrt{3}\right) (-1)^{2019}=-7-4\sqrt{3}.
			\end{eqnarray*}
		}	
	\end{ex}
	\begin{ex}
		Cho $P=\left(5-2\sqrt{6}\right)^{2018}\left(5+2\sqrt{6}\right)^{2019}$. Ta có
		\choice
		{\True $P\in (9;11)$}
		{$P\in (3;7)$}
		{$P\in (7;9)$}
		{$P\in (7;9)$}
		\loigiai{
			Ta có $P=(5-2\sqrt{6})^{2018}(5+2\sqrt{6})^{2018}(5+2\sqrt{6})=\left[(5+2\sqrt{6})(5-2\sqrt{6})\right]^{2018}(5+2\sqrt{6})=5+2\sqrt{6}$.\\
			Suy ra $P\in (9;11)$.
		}
	\end{ex}
	\begin{ex}
		Cho $x$, $y$ là hai số nguyên thỏa mãn $3^{x}\cdot 6^{y}=\dfrac{2^{15}\cdot 6^{40}}{9^{59}\cdot 12^{25}}$. Tính giá trị $xy$.
		\choice
		{$-445$}
		{$-755$}
		{\True $-540$}
		{$-425$}
		\loigiai{
			Ta có $\dfrac{2^{15}\cdot 6^{40}}{9^{59}\cdot 12^{25}}=\dfrac{2^{15}\cdot 2^{40}\cdot 3^{40}}{3^{118}\cdot 3^{25}\cdot 2^{50}}=2^{5}\cdot 3^{-103}$.\\
			Mà $3^{x}\cdot 6^{y}=3^{x+y}\cdot 2^{y}$, suy ra $\heva{&y=5\\ &x+y=-103}\Rightarrow\heva{&x=-108\\ &y=5}\Rightarrow xy=-540$.
		}
	\end{ex}
	\begin{ex}
		Cho hàm số $f(x)=\dfrac{4^x}{2+4^x}$ ($x\in\mathbb{R}$). Biết $a+b=5$ với $a$, $b$ là hai số thực, hãy tính $K=f(a)+f(b-4)$.
		\choice
		{\True $K=1$}
		{$K=\dfrac{3}{4}$}
		{$K=\dfrac{128}{129}$}
		{$K=\dfrac{512}{513}$}
		\loigiai{
			Ta có
			\begin{eqnarray*}
				K & = & f(a)+f(b-4)\\
				& = & f(a)+f(1-a)\\
				& = & \dfrac{4^a}{2+4^a}+\dfrac{4^{1-a}}{2+4^{1-a}}\\
				& = & \dfrac{2\cdot 4^a+4+2\cdot 4^{1-a}+4}{(2+4^a)(2+4^{1-a})}\\
				& = & \dfrac{2(a^4+4^{1-a})+8}{2(4^a+4^{1-a})+8}=1.
			\end{eqnarray*}
		}
	\end{ex}
	\setcounter{subsubsection}{0}
	\setcounter{ex}{0}
	\setcounter{bt}{0}
\begin{dang}{Rút gọn biểu thức chứa lũy thừa}
	Sử dụng các tính chất của lũy thừa để chuyển về cùng một cơ số, rồi bằng cách đặt nhân tử chung hằng đẳng thức...để rút gọn biểu thức.
\end{dang}
\subsubsection{Ví dụ minh hoạ}
\begin{vd} 
	Rút gọn biểu thức $A=\dfrac{6^{2+\sqrt{5}}\cdot 2^{1-\sqrt{5}}}{3^{3+\sqrt{5}}}$.
	\loigiai{
	Ta có
	\begin{eqnarray*}
		A&=&\dfrac{6^{2+\sqrt{5}}\cdot 2^{1-\sqrt{5}}}{3^{3+\sqrt{5}}}\\
		&=& \dfrac{(2\cdot 3)^{2+\sqrt{5}}\cdot 2^{1-\sqrt{5}}}{3^{3+\sqrt{5}}}\\
		&=&2^{2+\sqrt{5}}\cdot 3^{2+\sqrt{5}}\cdot 2^{1-\sqrt{5}}\cdot 3^{-3-\sqrt{5}}\\
		&=& 2^{2+\sqrt{5}+1-\sqrt{5}}\cdot 3^{2+\sqrt{5}-3-\sqrt{5}}=2^3\cdot 3^{-1}=\dfrac{8}{3}.
	\end{eqnarray*}
	}
\end{vd}
\begin{vd} 
	Rút gọn các biểu thức
	\begin{enumerate}
		\item $\dfrac{a^\frac{7}{3}-a^{\frac{1}{3}}}{a^{\frac{4}{3}}-a^{\frac{1}{3}}}-\dfrac{a^\frac{5}{3}-a^{-\frac{1}{3}}}{a^{\frac{2}{3}}-a^{-\frac{1}{3}}}$, $(a>0, a\ne 1)$;
		\item $\dfrac{\left(\sqrt[4]{a^3b^2}\right)^4}{\sqrt[3]{\sqrt{a^12b^6}}}$, $(a>0,b>0)$.
	\end{enumerate}
	\loigiai{
	\begin{enumerate}
		\item Với $a>0$, $a \ne 1$, ta có
		\allowdisplaybreaks
		\begin{eqnarray*}
		\dfrac{a^\frac{7}{3}-a^{\frac{1}{3}}}{a^{\frac{4}{3}}-a^{\frac{1}{3}}}-\dfrac{a^\frac{5}{3}-a^{-\frac{1}{3}}}{a^{\frac{2}{3}}-a^{-\frac{1}{3}}} &=& \dfrac{a^{\frac{1}{3}}\left(a^2-1\right)}{a^{\frac{1}{3}}\left(a-1\right)}-\dfrac{a^{-\frac{1}{3}}\left(a^2-1\right)}{a^{-\frac{1}{3}}\left(a+1\right)}\\
		&=& \dfrac{a^{\frac{1}{3}}(a-1)(a+1)}{a^{\frac{1}{3}}\left(a-1\right)}-\dfrac{a^{-\frac{1}{3}}(a-1)(a+1)}{a^{-\frac{1}{3}}\left(a+1\right)}\\
		&=& (a+1)-(a-1)=2.
		\end{eqnarray*}
		\item Với $a>0$, $b>0$ ta có
		$$
			\dfrac{\left(\sqrt[4]{a^3b^2}\right)^4}{\sqrt[3]{\sqrt{a^{12}b^6}}}=\dfrac{a^3b^2}{\sqrt[3]{a^6b^3}}=\dfrac{a^3b^2}{a^2b}=ab.
		$$
	\end{enumerate}
	}
\end{vd}
\begin{vd} 
	Rút gọn biểu thức $A=\dfrac{\left(a^{\sqrt{2}-1}\right)^{1+\sqrt{2}}}{a^{\sqrt{5}-1}\cdot a^{3-\sqrt{5}}}$.
	\loigiai{
	Ta có $$A=\dfrac{\left(a^{\sqrt{2}-1}\right)^{1+\sqrt{2}}}{a^{\sqrt{5}-1}\cdot a^{3-\sqrt{5}}}
	=\dfrac{a^{-1}}{a^2}=a^{-3}=\dfrac{1}{a^3}.$$
	}
\end{vd}
\subsubsection{Bài tập rèn luyện}
	\setcounter{bt}{0}
	\begin{bt}
	Rút gọn các biểu thức
	\begin{listEX}[3]
		\item $a^{\frac{1}{3}}\cdot a^{\frac{1}{2}}\cdot a^{\frac{7}{6}}$;
		\item $a^{\frac{2}{3}}\cdot a^{\frac{1}{4}}:a^{\frac{1}{6}}$;
		\item $\left(\dfrac{3}{2}a^{-\frac{3}{2}}b^{-\frac{1}{2}}\right)\left(-\dfrac{1}{3}a^{\frac{1}{2}}b^{\frac{3}{2}}\right)$.
	\end{listEX}
	\loigiai{
	\begin{enumerate}
		\item $a^{\tfrac{1}{3}} a^{\tfrac{1}{2}} a^{\tfrac{7}{6}}=a^{\tfrac{1}{3}+\tfrac{1}{2}+\tfrac{7}{6}}=a^2$.
		\item $a^{\tfrac{2}{3}} a^{\tfrac{1}{4}}: a^{\tfrac{1}{6}}=a^{\tfrac{2}{3}+\tfrac{1}{4}-\tfrac{1}{6}}=a^{\tfrac{3}{4}}$.
		\item $\left(\dfrac{3}{2} a^{-\tfrac{3}{2}} b^{-\tfrac{1}{2}}\right)\left(-\dfrac{1}{3} a^{\tfrac{1}{2}} b^{\tfrac{3}{2}}\right)=\dfrac{3}{2}\cdot \left(-\dfrac{1}{3}\right)a^{\tfrac{-3}{2}+\tfrac{1}{2}}\cdot b^{-\tfrac{1}{2}+\tfrac{3}{2}}=-\dfrac{1}{2}a^{-1}b$.	
	\end{enumerate}		
	}
\end{bt}
\begin{bt}
	Rút gọn các biểu thức sau
	\begin{listEX}[2]
		\item $A=\dfrac{x^5y^{-2}}{x^3y}$, với $x,y\ne 0$;
		\item $B=\dfrac{x^{\frac{1}{3}}\sqrt{y}+y^{\frac{1}{3}}\sqrt{x}}{\sqrt[6]{x}+\sqrt[6]{y}}$, với $x,y$ dương.
	\end{listEX}
	\loigiai{
	\begin{enumerate}
		\item $A=\dfrac{x^5 y^{-2}}{x^3 y}=x^2y^{-3}$.
		\item $B=\dfrac{x^{\frac{1}{3}}\sqrt{y}+y^{\frac{1}{3}}\sqrt{x}}{\sqrt[6]{x}+\sqrt[6]{y}}
		=\dfrac{x^{\frac{1}{3}}y^{\frac{1}{3}}\left(x^{\frac{1}{6}}+y^{\frac{1}{6}}\right)}{x^{\frac{1}{6}}y^{\frac{1}{6}}}
		=x^{\frac{1}{3}}y^{\frac{1}{3}}$.
	\end{enumerate}
	}
\end{bt}
\begin{bt}
	Cho số thực dương $a$. Rút gọn các biểu thức
	\begin{listEX}[2]
		\item $\dfrac{a^{\frac{4}{3}}\left(a^{-\frac{1}{3}}+a^{\frac{2}{3}}\right)}{a^{\frac{1}{4}}\left(a^{\frac{3}{4}}+a^{-\frac{1}{4}}\right)}$;
		\item $\dfrac{a^{\frac{1}{5}}\left(\sqrt[5]{a^4}-\sqrt[5]{a^{-1}}\right)}{a^{\frac{2}{3}}\left(\sqrt[3]{a}-\sqrt[3]{a^{-2}}\right)}$.
	\end{listEX}
	\loigiai{
	\begin{enumerate}
		\item \allowdisplaybreaks
		\begin{eqnarray*}
			\dfrac{a^{\frac{4}{3}}\left(a^{-\frac{1}{3}}+a^{\frac{2}{3}}\right)}{a^{\frac{1}{4}}\left(a^{\frac{3}{4}}+a^{-\frac{1}{4}}\right)}
			= \dfrac{a^{\frac{4}{3}}\cdot a^{-\frac{1}{3}}\left(1+a\right)}{a^{\frac{1}{4}}\cdot a^{-\frac{1}{4}}\left(a+1\right)}
			=\dfrac{a}{a^0}=a.
		\end{eqnarray*}
		\item \allowdisplaybreaks
		\begin{eqnarray*}
			\dfrac{a^{\frac{1}{5}}\left(\sqrt[5]{a^4}-\sqrt[5]{a^{-1}}\right)}{a^{\frac{2}{3}}\left(\sqrt[3]{a}-\sqrt[3]{a^{-2}}\right)}
			=\dfrac{a^{\frac{1}{5}}\left(a^{\frac{4}{5}}-a^{-\frac{1}{5}}\right)}{a^{\frac{2}{3}}\left(a^{\frac{1}{3}}-a^{-\frac{2}{3}}\right)}
			=\dfrac{a^{\frac{1}{5}}\cdot a^{-\frac{1}{5}}\left(a-1\right)}{a^{\frac{2}{3}}\cdot a^{-\frac{2}{3}}\left(a-1\right)}
			=\dfrac{1}{1}=1
		\end{eqnarray*}
	\end{enumerate}
	}
\end{bt}
\begin{bt}
	Tại một xí nghiệp, công thức $P(t)=500 \cdot\left(\dfrac{1}{2}\right)^{\tfrac{t}{3}}$ được dùng để tính giá trị còn lại (tính theo triệu đồng) của một chiếc máy sau thời gian $t$ (tính theo năm) kể từ khi đưa vào sử dụng.
	\begin{enumerate}
		\item Tính giá trị còn lại của máy sau $2$ năm; sau $2$ năm $3$ tháng.
		\item Sau $1$ năm đưa vào sử dụng, giá trị còn lại của máy bằng bao nhiêu phần trăm so với ban đầu?
	\end{enumerate}
	\loigiai{
		\begin{enumerate}
			\item Giá trị còn lại của máy sau $t=2$ năm là $P=500\cdot \left(\dfrac{1}{2}\right)^{\tfrac{2}{3}}\approx315$.\\
			Giá trị còn lại của máy sau sau $2$ năm $3$ tháng ($t=\dfrac{9}{4}$ năm) là $P=500\cdot \left(\dfrac{1}{2}\right)^{\tfrac{\tfrac{9}{4}}{3}}=500\cdot \left(\dfrac{1}{2}\right)^{\tfrac{3}{4}}\approx297$.\\
			\item 
			Ban đầu giá trị của máy là $P_0=500\cdot \left(\dfrac{1}{2}\right)^0=500$.\\
			Giá trị còn lại của máy sau $1$ năm sử dụng: $P=500\cdot \left(\dfrac{1}{2}\right)^{\tfrac{1}{3}}=396{,}85$.\\
			Suy ra $\dfrac{P}{P_0}=79{,}37\%$.
		\end{enumerate}
	}	
\end{bt}
\subsubsection{Bài tập trắc nghiệm}
	\setcounter{ex}{0}
	\begin{ex}
		Với $a$ là số thực dương, rút gọn $P=a\sqrt[3]{\sqrt{a}}$ ta được
		\choice
		{$P=a^{\tfrac{5}{6}}$}
		{$P=a^{\tfrac{11}{6}}$}
		{\True $P=a^{\tfrac{7}{6}}$}
		{$P=a^{\tfrac{6}{7}}$}
		\loigiai{
			Ta có $P=a\sqrt[3]{\sqrt{a}}=a\sqrt[3]{a^{\tfrac{1}{2}}}=a\cdot a^{\tfrac{1}{6}}=a^{\tfrac{7}{6}}$.
		}
	\end{ex}
	\begin{ex}
		Rút gọn biểu thức $P=b^{\frac{1}{2}}\cdot b^{\frac{1}{3}}\cdot \sqrt[6]{b}$ với $b>0$.
		\choice
		{\True $P=b$}
		{$P=b^{\frac{3}{11}}$}
		{$P=b^{\frac{1}{36}}$}
		{$P=b^{\frac{2}{3}}$}
		\loigiai{
			$P=b^{\frac{1}{2}}\cdot b^{\frac{1}{3}}\cdot \sqrt[6]{b}=b^{\frac{1}{2}+\frac{1}{3}+\frac{1}{6}}=b$.
		}
	\end{ex}
	\begin{ex}
		Rút gọn biểu thức $P=x^{\frac{1}{3}}\cdot\sqrt[6]{x}$ với $x>0$.
		\choice
		{ $P=x^{\frac{1}{8}}$}
		{$P=x^2$}
		{$P=x^{\frac{2}{9}}$}
		{\True $P=\sqrt{x}$}
		\loigiai{
			Ta có $P=x^{\frac{1}{3}}\cdot\sqrt[6]{x}=x^{\frac{1}{3}}\cdot x^{\frac{1}{6}}=x^{\frac{1}{2}}=\sqrt{x}$.
		}
	\end{ex}
	\begin{ex}
		Rút gọn biểu thức $\sqrt{81a^4b^2}$ ta được
		\choice
		{\True $9a^2|b|$}
		{$-9a^2b$}
		{$9a^2b$}
		{Kết quả khác}
		\loigiai{
			Ta có $\sqrt{81a^4b^2}=9a^2|b|$.
		}	
	\end{ex}
	\begin{ex}
		Cho biểu thức $\sqrt[5]{8\sqrt{2\sqrt[3]{2}}}=2^{\frac{m}{n}}$, trong đó $\dfrac{m}{n}$ có dạng phân số tối giản. Gọi $P=m^2+n^2$. Khẳng định nào sau đây đúng?
		\choice
		{$P\in(330;340)$}
		{$P\in(350;360)$}
		{\True $P\in(340;350)$}
		{$P\in(360;370)$}
		\loigiai{
			Ta có $\sqrt[5]{8\sqrt{2\sqrt[3]{2}}}=\left(8\cdot \left(2\cdot 2^{\frac{1}{3}} \right)^\frac{1}{2}  \right)^\frac{1}{5}=\left(2^3\cdot 2^\frac{2}{3} \right)^\frac{1}{5}=\left(2^\frac{11}{3} \right)^\frac{1}{5}=2^\frac{11}{15} $.\\
			Do đó $m=11$, $n=15$ $\Rightarrow P=m^2+n^2=346$.
		}
	\end{ex}
	\begin{ex}
		Rút gọn biểu thức $Q=b^{\tfrac{5}{3}}:\sqrt[3]{b}$ với $b>0$.
		\choice
		{\True $Q=b^{\tfrac{4}{3}}$}
		{$Q=b^{\tfrac{5}{9}}$}
		{$Q=b^2$}
		{$Q=b^{-\tfrac{4}{3}}$}
		\loigiai{
			$Q=b^{\tfrac{5}{3}}:\sqrt[3]{b}=b^{\tfrac{5}{3}-\tfrac{1}{3}}=b^{\tfrac{4}{3}}$.
		}
		
	\end{ex}
	
	\begin{ex}
		Cho $a$ là số thực dương tùy ý và $a$ khác $1$, đặt $A=\dfrac{a^{\sqrt{7}}\cdot a^{\sqrt{7}}}{(a^2)^{\sqrt{7}}}$. Mệnh đề nào dưới đây đúng?
		\choice
		{\True $A=1$}
		{$A=\dfrac{2}{a^{\sqrt{7}}}$}
		{$A=\sqrt{7}$}
		{$A=a$}
		\loigiai{
			Ta có $A=\dfrac{a^{\sqrt{7}}\cdot a^{\sqrt{7}}}{(a^2)^{\sqrt{7}}}=\dfrac{a^{2\sqrt{7}}}{a^{2\sqrt{7}}}=1$.
		}
	\end{ex}
	\begin{ex}
		Cho $x$, $y$ là các số thực thỏa mãn $x\neq 0$ và $\left(3^{x^2}\right)^{3y}=27^x$. Khẳng định nào sau đây là khẳng định đúng?
		\choice
		{$x^2 y=1$}
		{$x^2+3y=3x$}
		{\True $xy=1$}
		{$3xy=1$}
		\loigiai{
			Ta có
			\allowdisplaybreaks
			\begin{eqnarray*}
				& & \left(3^{x^2}\right)^{3y}=27^x \Leftrightarrow 3^{3x^2 y}=3^{3x}\\
				&\Leftrightarrow & 3x^2 y=3x \Leftrightarrow xy=1.
			\end{eqnarray*}
		}
	\end{ex}
	\begin{ex}
		Rút gọn biểu thức $P=\dfrac{\sqrt[6]{x}\sqrt[3]{x^4}\sqrt[4]{x}}{\sqrt{x^3}}$ với $x$ là số thực dương.
		\choice
		{$x^{\tfrac{13}{15}}$}
		{\True $x^{\tfrac{1}{4}}$}
		{$x^{\tfrac{1}{6}}$}
		{$x^{\tfrac{13}{18}}$}
		\loigiai{
			Ta có $P=\dfrac{\sqrt[6]{x}\sqrt[3]{x^4}\sqrt[4]{x}}{\sqrt{x^3}}=x^{\tfrac{1}{6}+\tfrac{4}{3}+\tfrac{1}{4}-\tfrac{3}{2}}=x^{\tfrac{1}{4}}$.
		}
	\end{ex}
	\begin{ex}
		Rút gon biểu thức $\mathrm{P} = \dfrac{a^{\sqrt{7} + 1}\cdot a^{2 - \sqrt{7}}}{\left(a^{\sqrt{2} - 2}\right)^{\sqrt{2} + 2}}$ với $a > 0$.
		\choice
		{$  P = a^4 $}
		{$ P = a^2  $}
		{\True $  P = a^5 $}
		{$  P = a^3 $}
		\loigiai{
			Với $a > 0$ thì $\mathrm{P} = \dfrac{a^{\sqrt{7} + 1}\cdot a^{2 - \sqrt{7}}}{\left(a^{\sqrt{2} - 2}\right)^{\sqrt{2} + 2}} = \dfrac{a^{\sqrt{7} + 1 + 2 - \sqrt{7}}}{a^{(\sqrt{2} - 2)(\sqrt{2} + 2)}} = \dfrac{a^3}{a^{-2}} = a^5$.
		}
	\end{ex}
	\begin{ex}
		Cho $a$ là một số thực dương. Rút gọn biểu thức: $P=\dfrac{ \left(a^{\sqrt{7}-3}\right)^{\sqrt{7}+3}}{a^{\sqrt{11}-4} \cdot a^{5-\sqrt{11}}}$ ?
		\choice
		{$P=a^3$}
		{$P=a^{2\sqrt{7}-1}$}
		{\True $P=\dfrac{1}{a^3}$}
		{$P=a^2$}
		\loigiai{
			\begin{eqnarray*}
				P&=&\dfrac{ \left(a^{\sqrt{7}-3}\right)^{\sqrt{7}+3}}{a^{\sqrt{11}-4} \cdot a^{5-\sqrt{11}}} \\
				&=& \dfrac{a^{\left(\sqrt{7}-3\right) \left(\sqrt{7}+3\right)}}{a^{\sqrt{11}-4+5-\sqrt{11}}}\\
				&=& \dfrac{a^{7-9}}{a}\\
				&=& \dfrac{1}{a^3}
			\end{eqnarray*}
		}
	\end{ex}
	
	\begin{ex}
		Rút gọn biểu thức $A=\dfrac{\sqrt[3]{a^7}\cdot a^{\frac{11}{3}}}{a^4\cdot \sqrt[7]{a^{-5}}}$ với $a>0$ ta được kết quả $A=a^{\frac{m}{n}}$ trong đó $m$, $n \in \mathbb{N}^*$ và $\dfrac{m}{n}$ là phân số tối giản. Khẳng định nào sau đây đúng?
		\choice
		{\True $m^2-n^2=312$}
		{$m^2+n^2=543$}
		{$m^2-n^2=-312$}
		{$m^2+n^2=409$}
		\loigiai{
			Ta có
			$$A=\dfrac{\sqrt[3]{a^7}\cdot a^{\tfrac{11}{3}}}{a^4\cdot \sqrt[7]{a^{-5}}}=\dfrac{a^{\tfrac{7}{3}+\tfrac{11}{3}}}{a^{4-\tfrac{5}{7}}} =a^{\tfrac{19}{7}}.$$
			Vậy $m^2-n^2=312$.
		}
	\end{ex}
	\begin{ex}
		Cho các số dương $a$, $b$. Rút gọn biểu thức $Q=\dfrac{a^{\tfrac{4}{3}}b+ab^{\tfrac{4}{3}}} {\sqrt[3]{a}+\sqrt[3]{b}}$.
		\choice
		{$Q=\sqrt[3]{ab}$}
		{$Q=2ab$}
		{$Q=\sqrt{ab}$}
		{\True $Q=ab$}
		\loigiai{
			Ta có:\\
			$$Q=\dfrac{a^{\tfrac{4}{3}}b+ab^{\tfrac{4}{3}}} {\sqrt[3]{a}+\sqrt[3]{b}}
			=\dfrac{ab\left(a^{\tfrac{1}{3}}+b^{\tfrac{1}{3}}\right)}{\left(a^{\tfrac{1}{3}}+b^{\tfrac{1}{3}}\right)}
			=ab.$$
		}
	\end{ex}
		
	\begin{ex}
	Cho $5^x + 5^{-x} = a$. Rút gọn biểu thức $M = \dfrac{25^x + 25^{-x} + 1}{5^x + 5^{-x} + 1}$ bằng
	\choice
	{$  a + 1 $}
	{\True $  a - 1 $}
	{$  a^2 + 1 $}
	{$  a^2 - 1 $}
	\loigiai{
		Ta có $5^x + 5^{-x} = a \Leftrightarrow 25^x + 25^{-x} + 2 = a^2 \Leftrightarrow 25^x + 25^{-x} = a^2 - 2$.\\
		Do đó $M = \dfrac{25^x + 25^{-x} + 1}{5^x + 5^{-x} + 1} = \dfrac{a^2 - 1}{a + 1} = a - 1$.
	}
	\end{ex}	
	\begin{ex}
		Cho $f(x) = \mathrm{e}^{\sqrt{1 + \tfrac{1}{x^2} + \tfrac{1}{(x+1)^2}}}$. Biết rằng $f(1) \cdot f(2) \cdot f(3) \cdots f(2019) = \mathrm{e}^{\tfrac{m}{n}}$, với $m$, $n$ là các số tự nhiên và $\dfrac{m}{n}$ tối giản. Tính $m - n^2$.
		\choice
		{$m -n^2 = 2018$}
		{$m -n^2 = 1$}
		{$m -n^2 = - 2018$}
		{\True $m -n^2 = -1$}
		\loigiai{
			Ta có
			\begin{eqnarray*}
				1 + \dfrac{1}{x^2} + \dfrac{1}{(x+1)^2} & = & \dfrac{x^2 (x+1)^2 + (x+1)^2 + x^2}{x^2(x+1)^2}\\
				& = & \dfrac{x^2(x+1)^2 + 2x(x+1) + 1}{x^2(x+1)^2} \\
				& = & \dfrac{[x(x+1)+1]^2}{x^2(x+1)^2} \\
				& = &  \left[ \dfrac{x(x+1)+1}{x(x+1)} \right]^2.
			\end{eqnarray*}
			Suy ra $f(x) = \mathrm{e}^{\tfrac{x(x+1)+1}{x(x+1)}} = \mathrm{e}^{1 + \tfrac{1}{x} - \tfrac{1}{x+1}}$ vì $x>0$. \\
			Khi đó \[f(1) \cdot f(2) \cdots f(2019) = \mathrm{e}^{1 + 1 - \tfrac{1}{2}} \cdot \mathrm{e}^{1 + \tfrac{1}{2} - \tfrac{1}{3}} \cdots \mathrm{e}^{1 + \tfrac{1}{2018} - \tfrac{1}{2019}} = \mathrm{e}^{2019 + 1 - \tfrac{1}{2020}} = \mathrm{e}^{2020 - \tfrac{1}{2020}} = \mathrm{e}^{\tfrac{2020^2 - 1}{2020}}.\]
			Suy ra $m = 2020^2 - 1$ và $n = 2020 \Rightarrow m - n^2 = -1$.
		}
		
	\end{ex}
\setcounter{subsubsection}{0}
\setcounter{ex}{0}
\setcounter{bt}{0}

\begin{dang}{So sánh biểu thức lũy thừa}
	Biến đổi các biểu thức về cùng cơ số hoặc cùng số mũ, từ đó, dựa vào tính chất lũy thừa để so sánh.
\end{dang}
\subsubsection{Ví dụ minh hoạ}
\begin{vd} 
	Không sử dụng máy tính, hãy so sánh $3^{\sqrt{8}}$ và $3^3$.
	\loigiai{
	Ta có $3=\sqrt{9}>\sqrt{8}$. Vì cơ số $3$ lớn hơn $1$ nên $3^{\sqrt{8}}<3^3$.
	}
\end{vd}
\begin{vd} 
	Không sử dụng máy tính, hãy so sánh các số $8^{\sqrt{3}}$ và $4^{2\sqrt{3}}$.
	\loigiai{
	Ta có $8^{\sqrt{3}}=\left(2^3\right)^{\sqrt{3}}=2^{3\sqrt{3}}$ và $4^{2\sqrt{3}}=(2^2)^{2\sqrt{3}}=2^{4\sqrt{3}}$.\\
	Vì $3\sqrt{3}<4\sqrt{3}$ và $2>1$ nên $2^{3\sqrt{3}}<2^{4\sqrt{3}}$.\\
	Vậy $8^{\sqrt{3}} < 4^{2\sqrt{3}}$.
	}
\end{vd}

\subsubsection{Bài tập rèn luyện}
	\setcounter{bt}{0}
\begin{bt}%[1C6B1-3]
	Không sử dụng máy tính cầm tay, hãy so sánh các số sau
	\begin{listEX}[3]
		\item $\sqrt{42}$ và $\sqrt[3]{51}$;
		\item $16^{\sqrt{3}}$ và $4^{3\sqrt{2}}$;
		\item $(0{,}2)^{\sqrt{16}}$ và $(0{,}2)^{\sqrt[3]{60}}$.
	\end{listEX}
	\loigiai{
		\begin{enumerate}
			\item Ta có $\sqrt{42}=42^{\frac{1}{2}}$ suy ra $42^3=\left(42^{\frac{1}{2}}\right)^6$ và $\sqrt[3]{51}=51^{\frac{1}{3}}$ suy ra $51^2=\left(51^{\frac{1}{3}}\right)^6$.\\
			Mà $42^3>51^2$ suy ra $\sqrt{42}>\sqrt[3]{51}$.
			\item Ta có $16^{\sqrt{3}}=4^{2\sqrt{3}}$ và $4^{3\sqrt{2}}$.\\
			Do $(2\sqrt{3})^2=12$ và $(3\sqrt{2})^2=32$, nên $2\sqrt{3}<3\sqrt{2}$.\\
			Mặt khác cơ số $4>1$ nên $16^{\sqrt{3}}<4^{3\sqrt{2}}$;
			\item Ta có $(\sqrt{16})^6=16^3$, $(\sqrt[3]{60})^6=60^2$.\\
			Suy ra $\sqrt{16}>\sqrt[3]{60}$ mà cơ số $0{,}2<1$ nên $(0{,}2)^{\sqrt{16}}<(0{,}2)^{\sqrt[3]{60}}$.
		\end{enumerate}
	}
\end{bt}
\begin{bt}%[1K6BH-3]
	Không sử dụng máy tính cầm tay, hãy so sánh
	\begin{enumerate}
		\item $5^{6\sqrt{3}}$ và $5^{3\sqrt{6}}$;
		\item $\left(\dfrac{1}{2}\right)^{\tfrac{-4}{3}}$ và $\sqrt{2}\cdot 2^{\tfrac{2}{3}}$.
	\end{enumerate}
	\loigiai{
		\begin{enumerate}
			\item Ta có $6\sqrt{3}=\sqrt{6^2\cdot 3}=\sqrt{108}$ và $3\sqrt{6}=\sqrt{3^2\cdot 6}=\sqrt{54}$.\\
			Do $3\sqrt{6}=\sqrt{54}<\sqrt{108}=6\sqrt{3}$ và cơ số $5>1$ nên $5^{3\sqrt{6}}<5^{6\sqrt{3}}$.
			\item Ta có $\left(\dfrac{1}{2}\right)^{\tfrac{-4}{3}}=2^{\tfrac{4}{3}}$ và $\sqrt{2}\cdot 2^{\tfrac{2}{3}}=2^{\tfrac{1}{2}+\tfrac{2}{3}}=2^{\tfrac{7}{6}}$.\\
			Do $\dfrac{7}{6}<\dfrac{8}{6}=\dfrac{4}{3}$ và cơ số $2>1$ nên $2^{\tfrac{7}{6}}<2^{\tfrac{4}{3}}$ hay $\sqrt{2}\cdot 2^{\tfrac{2}{3}}<\left(\dfrac{1}{2}\right)^{\tfrac{-4}{3}}$.
		\end{enumerate}	
	}
\end{bt}
\subsubsection{Bài tập trắc nghiệm}
	\setcounter{ex}{0}
	\begin{ex}
		Cho $ \pi^\alpha>\pi^\beta $ với $ \alpha, \beta\in\mathbb{R} $. Mênh đề nào dưới đây đúng?
		\choice
		{$ \alpha<\beta $}
		{\True $ \alpha>\beta $}
		{$ \alpha\leq\beta $}
		{$ \alpha=\beta $}
		\loigiai{
			Ta có $ \pi^\alpha>\pi^\beta \Rightarrow \alpha>\beta $.
		}
	\end{ex}
	\begin{ex}
		Cho $a$ và $b$ thuộc khoảng $\left(0; 1\right)$ và $\alpha$, $\beta$ là những số thực tùy ý. Khẳng định nào sau đây là khẳng định \textbf{sai}?
		\choice
		{$\left(a^\alpha\right)^\beta = \left(a^\beta\right)^\alpha$}
		{\True $a^\alpha > a^\beta \Leftrightarrow \alpha > \beta$}
		{$a^\alpha a^\beta =a^{\alpha + \beta}$}
		{$a^\alpha b^\alpha = (ab)^\alpha$}
		\loigiai{
			Do $0< a<1$ nên $a^\alpha > a^\beta \Leftrightarrow \alpha < \beta$. Vậy khẳng định  $a^\alpha > a^\beta \Leftrightarrow \alpha > \beta$  là sai.
		}
	\end{ex}
	\begin{ex}
		Sắp xếp các số $a=\sqrt{2^3}$, $b=4$, $c=\sqrt[3]{2}$ theo thứ tự từ nhỏ đến lớn ta được
		\choice
		{$a<b<c$}
		{\True $c<a<b$}
		{$c<b<a$}
		{$b<a<c$}
		\loigiai{
		Ta có $a=\sqrt{2^3}=2^{\frac{3}{2}}$, $b=4=2^2$, $c=\sqrt[3]{2}=2^{\frac{2}{3}}$.\\
		Vì $2>1$ và $\dfrac{2}{3}<\dfrac{3}{2}<2$ nên $c<a<b$.
		}
	\end{ex}
	
	\begin{ex}%
		Cho biết $( x-2)^{-\frac{1}{3}}>(x-2)^{-\frac{1}{6}}$, khẳng định nào sau đây đúng?
		\choice
		{$x>1$}
		{\True $2<x<3$}
		{$x>2$}
		{$0<x<1$}
		\loigiai{
			Do $-\dfrac{1}{3}<-\dfrac{1}{6}$ và $-\dfrac{1}{3};-\dfrac{1}{6} \notin\mathbb{Z}$ nên bất phương trình tương đương $\heva{& x-2>0\\&x-2<1} \Leftrightarrow 2<x<3$.
		}
	\end{ex}
	\begin{ex}
		Cho $a>0$, $b>0$ thỏa $a^{\frac{1}{2}}>a^{\frac{1}{3}}$ và $b^{\frac{2}{3}}>b^{\frac{3}{4}}$. Khi đó
		\choice
		{\True $a>1$, $0<b<1$}
		{$a>1$, $b>1$}
		{$0<a<1$, $0<b<1$}
		{$0<a<1$, $b>1$}
		\loigiai{
			Ta có $a>0$ và $a^{\frac{1}{2}}>a^{\frac{1}{3}}$ nên $a>1$; $b>0$ và $b^{\frac{2}{3}}>b^{\frac{3}{4}}$ nên $0<b<1$.
		}
	\end{ex}

\Closesolutionfile{ans}
