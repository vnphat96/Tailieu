\setcounter{section}{1}
\section{CÔNG THỨC CỘNG XÁC SUẤT}
\subsection{Trọng tâm kiến thức}
\begin{tomtat}
\subsubsection{Biến cố hợp}
\begin{boxdn}
	{Cho hai biến cố $A$ và $B$. Biến cố \lq\lq$A$ hoặc $B$ xảy ra\rq\rq, kí hiệu là $A\cup B$ được gọi là \textbf{\textit{biến cố hợp}} của $A$ và $B$}
\end{boxdn}
\begin{note}
	{Biến cố $A \cup B$ xảy ra khi có ít nhất một trong hai biến cố $A$ và $B$ xảy ra. Tập hợp mô tả biến cố $A \cup B$ là hợp của hai tập hợp mô tả biến cố $A$ và biến cố $B$.}
\end{note}
\subsubsection{Quy tắc cộng xác suất}
\paragraph{Quy tắc cộng cho hai biến cố xung khắc}
\begin{boxdn}
	{Cho hai biến cố xung khắc $A$ và $B$. Khi đó: $P(A \cup B)=P(A)+P(B)$.}
\end{boxdn}
\paragraph{Quy tắc cộng cho hai biến cố bất kì}
\begin{boxdn}
	{Cho hai biến cố $A$ và $B$. Khi đó $P(A \cup B)=P(A)+P(B)-P(A B)$.}
\end{boxdn}        
\end{tomtat}
%%%%%%%%%%%%%%%%%%%%%%%
\subsection{Các dạng bài tập}
\begin{dang}{Biến cố hợp}
\end{dang}
%%==========Ví dụ 1
\begin{vd}%[1K8BR-3]
	Một hộp đựng $15$ tấm thẻ cùng loại được đánh số từ $1$ đến $15$. Rút ngẫu nhiên một tấm thẻ trong hộp. Gọi $E$ là biến cố \lq\lq Số ghi trên tấm thẻ là số lẻ\rq\rq; $F$ là biến cố \lq\lq Số ghi trên tấm thẻ là số nguyên tố\rq\rq.
	\begin{enumerate}
		\item Mô tả không gian mẫu.
		\item Nêu nội dung của biến cố hợp $G=E \cup F$. Hỏi $G$ là tập con nào của không gian mẫu?
	\end{enumerate}
	\loigiai{
		\begin{enumerate}
			\item Không gian mẫu $\Omega=\{1 ; 2 ; 3 ; 4 ; 5 ; 6 ; 7 ; 8 ; 9 ; 10 ; 11 ; 12 ; 13 ; 14 ; 15\}$.
			\item $E \cup F$ là biến cố \lq\lq Số ghi trên tấm thẻ là số lẻ hoặc số nguyên tố\rq\rq.\\
			Ta có $E=\{1 ; 3 ; 5 ; 7 ; 9 ; 11 ; 13 ; 15\} ; F=\{2 ; 3 ; 5 ; 7 ; 11 ; 13\}$.\\
			Vậy $G=E \cup F=\{1 ; 2 ; 3 ; 5 ; 7 ; 9 ; 11 ; 13 ; 15\}$.
		\end{enumerate}
	}
\end{vd}
%%==========Ví dụ 2
\begin{vd}%[1K8BR-3]
	Một tổ trong lớp 11B có $4$ học sinh nữ là Hương, Hồng, Dung, Phương và $5$ học sinh nam là Sơn, Tùng, Hoàng, Tiến, Hải. Trong giờ học, giáo viên chọn ngẫu nhiên một học sinh trong tổ đó lên bảng để kiểm tra bài.\\
	Xét các biến cố sau:\\
	$H$: \lq\lq Học sinh đó là một bạn nữ\rq\rq;\\
	$K$: \lq\lq Học sinh đó có tên bắt đầu là chữ cái $\mathrm{H}$\rq\rq.
	\begin{enumerate}
		\item Mô tả không gian mẫu.
		\item Nêu nội dung của biến cố hợp $M=H \cup K$. Mỗi biến cố $H, K, M$ là tập con nào của không gian mẫu?
	\end{enumerate}
	\loigiai{
		\begin{enumerate}
			\item $\Omega=\{\text{Hương, Hồng, Dung, Phương, Sơn, Tùng, Hoàng, Tiến, Hải}\}$.
			\item $M=H \cup K$ là biến cố \lq\lq Học sinh được chọn là một bạn nữ hoặc một bạn nam\rq\rq. \\
			Ta có $H=\{\text{Hương, Hồng, Dung, Phương}\}$;\\
			$K=\{\text{Sơn, Tùng, Hoàng, Tiến, Hải}\}$.\\
			Vậy $M=\Omega=\{\text{Hương, Hồng, Dung, Phương, Sơn, Tùng, Hoàng, Tiến, Hải}\}$.
		\end{enumerate}
	}
\end{vd}
%--------------------
\begin{dang}{Công thức cộng xác suất cho hai biến cố xung khắc}
\end{dang}
\begin{vd}%[1K8BS-1]%[1K8BS-3]
	Một hộp đựng $9$ tấm thẻ cùng loại được ghi số từ $1$ đến $9$. Rút ngẫu nhiên đồng thời hai tấm thẻ từ trong hộp. Xét các biến cố sau:\\
	$A$: \lq\lq Cả hai tấm thẻ đều ghi số chẵn\rq\rq;\\
	$B$: \lq\lq Chỉ có một tấm thẻ ghi số chẵn\rq\rq;\\
	$C$: \lq\lq Tích hai số ghi trên hai tấm thẻ là một số chẵn\rq\rq.
	\begin{enumEX}{2}
	\item Chứng minh rằng $C=A \cup B$.
	\item Tính $\mathrm{P}(C)$.
	\end{enumEX}
	\loigiai{
	\begin{enumEX}{1}
	\item Biến cố $C$ xảy ra khi và chỉ khi trong hai tấm thẻ có ít nhất một tấm thẻ ghi số chẵn. Nếu cả hai tấm thẻ ghi số chẵn thì biến cố $A$ xảy ra. Nếu chỉ có một tấm thẻ ghi số chẵn thì biến cố $B$ xảy ra. Vậy $C$ là biến cố hợp của $A$ và $B$.
	\item Hai biến cố $A$ và $B$ là xung khắc. Do đó $\mathrm{P}(C)=\mathrm{P}(A \cup B)=\mathrm{P}(A)+\mathrm{P}(B)$.\\
	Ta cần tính $\mathrm{P}(A)$ và $\mathrm{P}(B)$.
	Không gian mẫu $\Omega$ là tập hợp tất cả các tập con có hai phần tử của tập $\{1 ; 2 ; \ldots ; 9\}$. Do đó $n(\Omega)=\mathrm{C}_{9}^{2}=36$. \\
	Tính $\mathrm{P}(A)$: Biến cố $A$ là tập hợp tất cả các tập con có hai phần tử của tập $\{2 ; 4 ; 6 ; 8\}$. Do đó $n(A)=\mathrm{C}_{4}^{2}=6$. Suy ra $\mathrm{P}(A)=\dfrac{n(A)}{n(\Omega)}=\dfrac{6}{36}$.\\
	Tính $\mathrm{P}(B)$ : Mỗi phần tử của $B$ được hình thành từ hai công đoạn: 
	\begin{itemize}
	\item Công đoạn 1: Chọn một số chẵn từ tập $\{2 ; 4 ; 6 ; 8\}$. Có $4$ cách chọn. 
	\item Công đoạn 2: Chọn một số lẻ từ tập $\{1 ; 3 ; 5 ; 7 ; 9\}$. Có $5$ cách chọn.
	\end{itemize}
	Theo quy tắc nhân, tập $B$ có $4 \cdot 5=20$ (phần tử). \\
	Do đó $n(B)=20$. Suy ra $\mathrm{P}(B)=\dfrac{n(B)}{n(\Omega)}=\dfrac{20}{36}$.\\
	Vậy $\mathrm{P}(C)=\mathrm{P}(A)+\mathrm{P}(B)=\dfrac{6}{36}+\dfrac{20}{36}=\dfrac{26}{36}=\dfrac{13}{18}$.
	\end{enumEX}
	}
\end{vd}
\begin{vd}%[1T9K2-4]
	Một đội tình nguyện gồm 9 học sinh khối 10 và 7 học sinh khối 11. Chọn ra ngẫu nhiên 3 người trong đội. Tính xác suất của biến cố \lq\lq Cả 3 người được chọn học cùng một khối\rq\rq.
	\loigiai{
	Gọi $A$ là biến cố \lq\lq Cả 3 học sinh được chọn đều thuộc khối 10 \rq\rq và $B$ là biến cố \lq\lq Cả 3 học sinh được chọn đều thuộc khối 11\rq\rq. Khi đó $A \cup B$ là biến cố \lq\lq Cả 3 người được chọn học cùng một khối\rq\rq. Do $A$ và $B$ là hai biến cố xung khắc nên $P(A \cup B)=P(A)+P(B)$.\\
	Ta thấy $P(A)=\dfrac{\mathrm{C}_9^3}{\mathrm{C}_{16}^3}$ và $P(B)=\dfrac{\mathrm{C}_7^3}{C_{16}^3}$, nên $P(A \cup B)=\dfrac{\mathrm{C}_9^3+\mathrm{C}_7^3}{\mathrm{C}_{16}^3}=\dfrac{17}{80}$.
	}
\end{vd}
\begin{dang}{Công thức cộng xác suất cho 2 biến cố bất kì}
\end{dang}
\begin{vd}%[1K8BS-3]
	Ở một trường trung học phổ thông $X$, có $19 \%$ học sinh học khá môn Ngữ văn, $32 \%$ học sinh học khá môn Toán, $7 \%$ học sinh học khá cả hai môn Ngữ văn và Toán. Chọn ngẫu nhiên một học sinh của trường $X$. Xét hai biến cố sau:\\
	$A$: \lq\lq Học sinh đó học khá môn Ngữ văn\rq\rq;\\
	$B$: \lq\lq Học sinh đó học khá môn Toán\rq\rq.
	Hãy tính tỉ lệ học sinh học khá môn Ngữ văn hoặc học khá môn Toán của trường $X$.
	\loigiai{
	Theo đề bài, ta có
	$$
	\mathrm{P}(A)=19 \%=0,19 ; \mathrm{P}(B)=32 \%=0,32 \text { và } \mathrm{P}(A B)=7 \%=0,07 \text {. }
	$$
	Theo công thức cộng xác suất, ta có
	$$
	\mathrm{P}(A \cup B)=\mathrm{P}(A)+\mathrm{P}(B)-\mathrm{P}(A B)=0{,}19+0{,}32-0{,}07=0{,}44 .
	$$
	Do đó, xác suất để chọn ngẫu nhiên một học sinh của trường $X$ học khá môn Ngữ văn hoặc học khá môn Toán là $0{,}44$.\\
	Vậy tỉ lệ học sinh học khá môn Ngữ văn hoặc học khá môn Toán của trường $X$ là $44 \%$.
	}
\end{vd}
\begin{vd}%[1T9K2-4]
	Một hộp chứa 100 tấm thẻ cùng loại được đánh số lần lượt từ 1 đến 100. Chọn ngẫu nhiên 1 thẻ từ hộp. Tính xác suất của biến cố \lq\lq Số ghi trên thẻ được chọn chia hết cho 3 hoặc 5\rq\rq.
	\loigiai{
	Gọi $A$ là biến cố \lq\lq Số ghi trên thẻ được chọn chia hết cho 3\rq\rq và $B$ là biến cố \lq\lq Số ghi trên thẻ được chọn chia hết cho 5\rq\rq.\\
	$A \cup B$ là biến cố \lq\lq Số ghi trên thẻ được chọn chia hết cho 3 hoặc 5\rq\rq.\\
	Từ 1 đến 100 có 33 số chia hết cho 3 nên $P(A)=\dfrac{33}{100}=0,33$.\\
	Từ 1 đến 100 có 20 số chia hết cho 5 nên $P(B)=\dfrac{20}{100}=0,2$.\\
	Một số chia hết cho cả 3 và 5 khi nó chia hết cho 15. Từ 1 đến 100 có 6 số chia hết cho 15 nên\\
	\centerline{$P(A B)=\dfrac{6}{100}=0,06$}.\\
	Vậy $P(A \cup B)=P(A)+P(B)-P(A B)=0,33+0,2-0,06=0,47$.
	}
\end{vd}
%=====================
\subsection{Bài tập rèn luyện}
\begin{bt}%[1T9B1-2]%[1T9Y1-2]
	Hộp thứ nhất chứa $3$ tấm thẻ cùng loại được đánh số lần lượt từ $1$ đến $3$. Hộp thứ hai chứa $5$ tấm thẻ cùng loại được đánh số lần lượt từ $1$ đến $5$. Lấy ra ngẫu nhiên từ mỗi hộp $1$ thẻ. Gọi $A$ là biến cố \lq\lq Tổng các số ghi trên $2$ thẻ bằng $6$\rq\rq, $B$ là biến cố \lq\lq Tích các số ghi trên $2$ thẻ là số lẻ\rq\rq. \\
	Hãy tìm một biến cố khác rỗng và xung khắc với cả hai biến cố $A$ và $B$.	
	\loigiai{
	Một biến cố khác rỗng và xung khắc với cả hai biến cố $A$ và $B$ là \\
	$C$ là biến cố \lq\lq Tích các số ghi trên $2$ thẻ là số chẵn và tổng khác 6\rq\rq .\\
	Suy ra $C=\{(1;2);(1;4);(2;1);(2;2);(2;3);(2;5);(3;2);(3;4)\}$.
	}
\end{bt}
\begin{bt}%[1K8KS-3]%8.6. 
Một hộp đựng $8$ viên bi màu xanh và $6$ viên bi màu đỏ, có cùng kích thước và khối lượng. Bạn Sơn lấy ngẫu nhiên một viên bi từ hộp (lấy xong không trả lại vào hộp). Tiếp đó đến lượt bạn Tùng lấy ngẫu nhiên một viên bi từ hộp đó. Tính xác suất để bạn Tùng lấy được viên bi màu xanh.
\loigiai{
Tổng số bi trong hộp là $6+8=14$. Do đó số phần tử của không gian mẫu là $n(\Omega)=\mathrm{A}^2_{14}= 182$.\\
Gọi $A$ là biến cố: \lq\lq Bạn Sơn lấy được viên bi màu xanh, sau đó Tùng lấy được viên bi màu xanh\rq\rq và $B$ là biến cố: \lq\lq Bạn Sơn lấy được viên bi màu đỏ, sau đó Tùng lấy được viên bi màu xanh\rq\rq. \\
Suy ra $A\cup B$ là biến cố: \lq\lq Bạn Sơn lấy ngẫu nhiên một viên bi từ hộp (lấy xong không trả lại vào hộp) và tiếp đó đến lượt bạn Tùng lấy ngẫu nhiên một viên bi từ hộp đó\rq\rq.\\
 $A$ và $B$ là hai biến cố xung khắc, vì Sơn không thể lấy một viên bi vừa màu xanh và vừa màu đỏ được. Suy ra 
 $\mathrm{P}(A\cup B)=\mathrm{P}(A)+\mathrm{P}(B)$.\\
Ta có $n(A)= 8 \cdot 7 =56 \Rightarrow \mathrm{P}(A)=\dfrac{n(A)}{n(\Omega)} = \dfrac{56}{182}$
và $n(B)= 6 \cdot 8 =48\Rightarrow \mathrm{P}(B)=\dfrac{n(B)}{n(\Omega)} = \dfrac{48}{182} $ .\\
$\Rightarrow \mathrm{P}(A\cup B)= \dfrac{56}{182}+\dfrac{48}{182} = \dfrac{104}{182}=\dfrac{4}{7}$.
}
\end{bt}
\begin{bt}%[1K8KS-3]%8.7. 
	Lớp $11A$ của một trường có $40$ học sinh, trong đó có $14$ bạn thích nhạc cổ điển, $13$ bạn thích nhạc trẻ và $5$ bạn thích cả nhạc cổ điển và nhạc trẻ. Chọn ngẫu nhiên một bạn trong lớp. Tính xác suất để:
	\begin{enumEX}{1}
	\item Bạn đó thích nhạc cổ điển hoặc nhạc trẻ;
	\item Bạn đó không thích cả nhạc cổ điển và nhạc trẻ.
	\end{enumEX}
	\loigiai{
	Gọi 	$A$ là biến cố: \lq\lq Học sinh thích nhạc cổ điển\rq\rq;\\
	 	$B$ là biến cố: \lq\lq Học sinh thích nhạc trẻ\rq\rq.\\
	Khi đó $A \cap B$ là biến cố: \lq\lq Học sinh thích cả nhạc cổ điển và nhạc trẻ\rq\rq; \\
	$A \cup B$ là biến cố: \lq\lq Học sinh hoặc thích nhạc cổ điển hoặc nhạc trẻ\rq\rq;\\
	$\overline{A \cup B}$ là biến cố: \lq\lq Học sinh không thích cả nhạc cổ điển và nhạc trẻ\rq\rq.\\
	Ta có $n(A)=14,\, n(B)=13, \, n(A\cap B)=5$ và $n(\Omega) =40$. 
	\begin{enumEX}{1}
	\item Theo công thức cộng xác suất, ta có	
	\begin{align*}
	\mathrm{P}(A \cup B) &= \mathrm{P}(A) + \mathrm{P}(B) - \mathrm{P}(A \cap B) \\
	 &= \frac{n(A)}{n(\Omega)} + \frac{n(B)}{n(\Omega)} - \frac{n(A \cap B)}{n(\Omega)} \\
	&= \frac{14}{40} + \frac{13}{40} - \frac{5}{40} = \frac{22}{40} = 0{,}55.
	\end{align*}	
	Vậy xác suất để bạn chọn được một bạn thích nhạc cổ điển hoặc nhạc trẻ là $0{,}55$.	
	\item Theo tính chất xác suất đối lập, ta có	
$$\mathrm{P}(\overline{A \cup B}) = 1 - \mathrm{P}(A \cup B) = 1 - 0{,}55 = 0{,}45.$$	
	Vậy xác suất để bạn chọn được một bạn không thích cả nhạc cổ điển và nhạc trẻ là $0{,}45$.
	\end{enumEX}	
	}
\end{bt}
\begin{bt}%[1K8KS-3] %8.8. 
Một khu phố có $50$ hộ gia đình nuôi chó hoặc nuôi mèo, trong đó có $18$ hộ nuôi chó, $16$ hộ nuôi mèo và $7$ hộ nuôi cả chó và mèo. Chọn ngẫu nhiên một hộ trong khu phố trên. Tính xác suất để
\begin{enumEX}{1}
\item Hộ đó nuôi chó hoặc nuôi mèo;
\item Hộ đó không nuôi cả chó và mèo.
\end{enumEX}	
	\loigiai{
Gọi 	$A$ là biến cố: \lq\lq Hộ gia đình nuôi chó\rq\rq;\\
$B$ là biến cố: \lq\lq Hộ gia đình nuôi mèo\rq\rq.\\
Khi đó $A \cap B$ là biến cố: \lq\lq Hộ gia đình nuôi cả chó và mèo\rq\rq; \\
$A \cup B$ là biến cố: \lq\lq Hộ gia đình hoặc nuôi chó hoặc mèo\rq\rq;\\
$\overline{A \cup B}$ là biến cố: \lq\lq Hộ gia đình không nuôi cả chó và mèo\rq\rq.\\
Ta có $n(A)=18,\, n(B)=16, \, n(A\cap B)=7$ và $n(\Omega) =50$. 
\begin{enumEX}{1}
	\item Theo công thức cộng xác suất, ta có	
	\begin{align*}
	\mathrm{P}(A \cup B) &= \mathrm{P}(A) + \mathrm{P}(B) - \mathrm{P}(A \cap B) \\
	&= \frac{n(A)}{n(\Omega)} + \frac{n(B)}{n(\Omega)} - \frac{n(A \cap B)}{n(\Omega)} \\
	&= \frac{18}{50} + \frac{16}{50} - \frac{7}{50} = \frac{27}{50} = 0{,}54.
	\end{align*}	
Vậy xác suất để chọn được một hộ gia đình nuôi chó hoặc nuôi mèo là $0{,}54$.
	\item Theo tính chất xác suất đối lập, ta có	
	$$	\mathrm{P}(\overline{A \cup B}) = 1 - \mathrm{P}(A \cup B) = 1 - \frac{7}{50} = \frac{43}{50} = 0{,}86.$$	
	Vậy xác suất để chọn được một hộ không nuôi cả chó và mèo là $0{,}86$.	
\end{enumEX}	
	}
\end{bt}
\begin{bt}%[1K8KS-3]%8.9. 
Một nhà xuất bản phát hành hai cuốn sách $A$ và $B$. Thống kê cho thấy có $50 \%$ người mua sách $A$; $70 \%$ người mua sách $B$; $30 \%$ người mua cả sách $A$ và sách $B$. Chọn ngẫu nhiên một người mua. Tính xác suất để:
\begin{enumEX}{1}
\item Người mua đó mua ít nhất một trong hai sách $A$ hoặc $B$;
\item Người mua đó không mua cả sách $A$ và sách $B$.
\end{enumEX}	
	\loigiai{
Gọi 	$A$ là biến cố: \lq\lq Người mua mua sách $A$\rq\rq;\\
$B$ là biến cố: \lq\lq Người mua mua sách $A$\rq\rq.\\
Khi đó $A \cap B$ là biến cố: \lq\lq Người mua cả sách $A$ và sách $B$\rq\rq; \\
$A \cup B$ là biến cố: \lq\lq Người mua ít nhất hoặc sách $A$ hoặc sách $B$\rq\rq;\\
$\overline{A \cup B}$ là biến cố: \lq\lq Người mua không mua cả sách $A$ và sách $B$\rq\rq.\\
Ta có $	\mathrm{P}(A) = 0{,}5, \, \mathrm{P}(B) = 0{,}7, \, \mathrm{P}(A\cap B) = 0{,}3.$. 
\begin{enumEX}{1}
	\item Theo công thức cộng xác suất, ta có	
	\begin{align*}
	\mathrm{P}(A \cup B) &= \mathrm{P}(A) + \mathrm{P}(B) - \mathrm{P}(A \cap B) \\
	&= 0{,}5 + 0{,}7 - 0{,}3 = 0{,}9.
	\end{align*}	
Vậy xác suất người mua ít nhất một trong hai cuốn sách $A$ hoặc $B$ là $0.9$.
	\item Theo tính chất xác suất đối lập, ta có	
	$$	\mathrm{P}(\overline{A \cup B}) = 1 - \mathrm{P}(A \cup B) = 1 - 0{,}9 = 0{,}1.$$	
Vậy xác suất người không mua cả hai cuốn sách $A$ và $B$ là $0{,}1$.	
\end{enumEX}	
	}
\end{bt}
\begin{bt}%[1K8KS-3]%8.10. 
	Tại các trường trung học phổ thông của một tỉnh, thống kê cho thấy có $63 \%$ giáo viên môn Toán tham khảo bộ sách giáo khoa $A, 56 \%$ giáo viên môn Toán tham khảo bộ sách giáo khoa B và $28,5 \%$ giáo viên môn Toán tham khảo cả hai bộ sách giáo khoa A và $B$. Tính tỉ lệ giáo viên môn Toán các trường trung học phổ thông của tỉnh đó không tham khảo cả hai bộ sách giáo khoa $A$ và $B$.
	\loigiai{
	Gọi 	$A$ là biến cố: \lq\lq Giáo viên môn Toán tham khảo bộ sách giáo khoa $A$\rq\rq;\\
	$B$ là biến cố: \lq\lq Giáo viên môn Toán tham khảo bộ sách giáo khoa $B$\rq\rq.\\
	Khi đó $A \cap B$ là biến cố: \lq\lq Giáo viên môn Toán tham khảo cả bộ sách giáo khoa $A$ và bộ sách giáo khoa $B$\rq\rq; \\
	$A \cup B$ là biến cố: \lq\lq Giáo viên môn Toán tham khảo hoặc bộ sách giáo khoa $A$ hoặc bộ sách giáo khoa $B$\rq\rq.\\
	$\overline{A \cup B}$ là biến cố: \lq\lq Giáo viên môn Toán không tham khảo cả bộ sách giáo khoa $A$ và bộ sách giáo khoa $B$\rq\rq.\\
	Ta có $	\mathrm{P}(A) = 0{,}63, \, \mathrm{P}(B) = 0{,}56, \, \mathrm{P}(A\cap B) = 0{,}285$. \\
	 Theo công thức cộng xác suất, ta có	
	\begin{align*}
	\mathrm{P}(A \cup B) &= \mathrm{P}(A) + \mathrm{P}(B) - \mathrm{P}(A \cap B) \\
	&= 0{,}63 + 0{,}56 - 0{,}285 = 0{,}905.
	\end{align*}	
 Theo tính chất xác suất đối lập, ta có	
	$$	\mathrm{P}(\overline{A \cup B}) = 1 - \mathrm{P}(A \cup B) = 1 - 0{,}905= 0{,}0905.$$
	Vậy tỉ lệ giáo viên môn Toán không tham khảo cả hai bộ sách giáo khoa $A$ và $B$ là $9{,}05 \%$.
	}
\end{bt}