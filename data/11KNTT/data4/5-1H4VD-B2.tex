\subsection{PHÂN LOẠI, PHƯƠNG PHÁP GIẢI TOÁN}
\begin{dang} {Xét vị trí tương đối của hai đường thẳng}
	Cho hai đường thẳng $a$ và $b$ phân biệt. Xét vị trí tương đối của $a$ với $b$:
	\begin{itemize}
		\item Nếu $a$ và $b$ không đồng phẳng thì $a$ và $b$ chéo nhau.
		\item Nếu $a$ và $b$ đồng phẳng thì xét số điểm chung của $a$ và $b$. Nếu $a$ và $b$ không có điểm chung thì $a\parallel b$. Nếu $a$ và $b$ có một điểm chung thì $a$ và $b$ cắt nhau.
	\end{itemize}
\end{dang}
\begin{vd}
	Cho hình chóp $S.ABCD$ có đáy $ABCD$ là hình bình hành. Xét vị trí tương đối của các cặp đường thẳng sau
	\begin{listEX}[3]
		\item [a)] $AB$ và $CD$.
		\item [b)] $SA$ và $SC$.
		\item [c)] $SA$ và $BC$.
	\end{listEX}
\end{vd}
\begin{vd}
	Cho tứ diện $ABCD$ có $M, N$ lần lượt là trung điểm của $AB, AC$. Xét vị trí tương đối của các cặp đường thẳng sau
	\begin{listEX}[3]
		\item [a)] $MN$ và $BC$.
		\item [b)] $AN$ và $CD$.
		\item [c)] $MN$ và $CD$.
	\end{listEX}
\end{vd}
\begin{dang}{Chứng minh hai đường thẳng song song}
	\indamm{Phương pháp thường dùng:}
	\begin{itemize}
		\item [\ding{172}] Sử dụng các kết quả của hình học phẳng như:
		      \begin{itemize}
			      \item  Cặp cạnh đối hình bình hành thì song song nhau;...
			      \item  Đường trung bình của tam giác thì song song và bằng nửa cạnh đáy.
		      \end{itemize}
		\item [\ding{173}] Sử dụng tỉ lệ (Định lý Thales)
		      \begin{itemize}
			      \item Nếu $\dfrac{AE}{AB}=\dfrac{AF}{AC}$ thì $EF \parallel BC$.
			      \item Chú ý tỉ lệ trọng tâm:  $AG=\dfrac{2}{3}AM$.
		      \end{itemize}
	\end{itemize}
\end{dang}

\begin{vd}
	Cho tứ diện $ABCD$ có $I$, $J$ lần lượt là trọng tâm của tam giác $ABC$ và $ABD$. Chứng minh rằng $IJ \parallel CD$.
	\loigiai{\immini{Gọi $E$ là trung điểm $AB$. Ta có $\heva{&I\in CE\\&J\in DE}\Rightarrow IJ$ và $CD$ đồng phẳng.\\
			Vì $I$, $J$ lần lượt là trọng tâm của tam giác $ABC$ và $ABD$ nên $$\dfrac{EI}{EC}=\dfrac{EJ}{ED}=\dfrac{1}{3}.$$
			Theo định lí đảo Thales suy ra $IJ\parallel CD$ (đpcm).
		}{	\begin{tikzpicture}[scale=0.6,font=\footnotesize,line join=round,line cap=round,>=stealth]
				\tkzDefPoints{0/0/B, 3/4/A, 5/-3/C, 6/0/D}
				\coordinate (E) at ($(A)!0.5!(B)$);
				\coordinate (I) at ($(E)!1/3!(C)$);
				\coordinate (J) at ($(E)!1/3!(D)$);
				\tkzDrawPolygon(A,B,C,D)
				\tkzDrawSegments(A,C E,C)
				\tkzDrawSegments[dashed](E,D I,J B,D)
				\tkzDrawPoints[fill=black](A,B,C,D,E,I,J)
				\tkzLabelPoints[above](A,J)
				\tkzLabelPoints[right](D,I)
				\tkzLabelPoints[left](B,E)
				\tkzLabelPoints[below](C)
				\tkzMarkSegments[mark=||](E,A E,B)
			\end{tikzpicture}}}
\end{vd}
\begin{vd}
	Cho hình chóp $S.ABCD$ có đáy $ABCD$ là hình thang với $AB$ là đáy lớn và $AB=2CD$. Gọi $M, N$ lần lượt là trung điểm của các cạnh $SA$ và $SB$. Chứng minh rằng $NC\parallel MD$.
\end{vd}
\begin{vd}
	Cho tứ diện $ABCD$. Gọi $I, J$ lần lượt là trung điểm của các cạnh $BC, CD$. Trên cạnh $AC$ lấy điểm $K$. Gọi $M$ là giao điểm của $BK$ và $AI$, $N$ là giao điểm của $DK$ và $AJ$. Chứng minh rằng $MN\parallel BD$.
\end{vd}
\begin{vd}
	Cho tứ diện $ABCD$. Gọi $M$, $N$, $P$, $Q$, $R$, $S$ lần lượt là trung điểm của $AB$, $CD$, $BC$, $AD$, $AC$, $BD$.
	\begin{tasks}(1)
		\task Chứng minh $MPNQ$ là hình bình hành.
		\task Chứng minh ba đoạn thẳng $MN$, $PQ$, $RS$ cắt nhau tại trung điểm $G$ của mỗi đoạn.
	\end{tasks}
	\loigiai{
		\immini{
			\begin{enumerate}[a)]
				\item Vì $MP$ là đường trung bình của $\triangle ABC$ nên
				      $\heva{& MP \parallel AC\\& MP=\dfrac{1}{2}AC.} \qquad (1)$\\
				      Vì $NQ$ là đường trung bình của $\triangle ACD$ nên
				      $\heva{& NQ \parallel AC\\& NQ=\dfrac{1}{2}AC.} \qquad (2)$\\
				      Từ $(1)$ và $(2)$ suy ra $\heva{& MP \parallel NQ\\& MP=NQ.}$\\
				      Do đó, $MPNQ$ là hình bình hành.
				\item 	Do $MPNQ$ là hình bình hành nên $MN$, $PQ$ cắt nhau tại trung điểm $G$ của mỗi đoạn.\\
				      Mặt khá, chứng minh tương tự ta được $PSQR$ là hình bình hành nên $PQ$, $RS$ cắt nhau tại trung điểm $G$ của mỗi đoạn.\\
				      Vậy $MN$, $PQ$, $RS$ cắt nhau tại trung điểm $G$ của mỗi đoạn.
			\end{enumerate}
		}{	\begin{tikzpicture}[scale=0.6,font=\footnotesize,line join=round,line cap=round,>=stealth]
				\tikzset{label style/.style={font=\footnotesize}}
				\tkzDefPoints{0/0/B, 3/6/A, 1.3/-3/C, 8/0/D}
				\coordinate (M) at ($(A)!0.5!(B)$);
				\coordinate (N) at ($(C)!0.5!(D)$);
				\coordinate (P) at ($(B)!0.5!(C)$);
				\coordinate (Q) at ($(A)!0.5!(D)$);
				\coordinate (R) at ($(A)!0.5!(C)$);
				\coordinate (S) at ($(B)!0.5!(D)$);
				\coordinate (G) at ($(M)!0.5!(N)$);
				\tkzDrawPolygon(A,B,C,D)
				\tkzDrawSegments(A,C M,P Q,N P,R R,Q)
				\tkzDrawSegments[dashed](B,D R,S M,N P,Q M,Q P,N P,S S,Q)
				\tkzDrawPoints[fill=black](A,B,C,D,M,N,P,Q,R,S,G)
				\tkzLabelPoints[above](A,G)
				\tkzLabelPoints[right](D)
				\tkzLabelPoints[left](B,R)
				\tkzLabelPoints[below](C)
				\tkzLabelPoints[below right](S,N)
				\tkzLabelPoints[below left](P)
				\tkzLabelPoints[above right](Q)
				\tkzLabelPoints[above left](M)
			\end{tikzpicture}}}
\end{vd}

\begin{vd}
	Cho hình chóp $S.ABCD$ có đáy là hình bình hành. Gọi $M,N,P,Q$ lần lượt là trung điểm $BC,CD,SB,SD$.
	\begin{tasks}(1)
		\task Chứng minh rằng $MN\parallel PQ$.
		\task Gọi $I$ là trọng tâm của tam giác $ABC$, $J$ thuộc $SA$ sao cho $\dfrac{JS}{JA}=\dfrac{1}{2}$. Chứng minh $IJ\parallel SM$.
	\end{tasks}
	\loigiai{
		\begin{enumerate}[a)]
			\immini{	\item Chứng minh $MN\parallel PQ$.\\
			      Ta có: $MN\parallel BD$ ($MN$ là đường trung bình của $\Delta BCD$).\\
			      và $PQ\parallel BD$ ($PQ$ là đường trung bình của $\Delta SBD$).\\
			      Suy ra $MN\parallel PQ$
			\item Chứng minh $IJ\parallel SM$.\\
			      $\dfrac{AJ}{AS}=\dfrac{2}{3}$.\\
			      $ \dfrac{JS}{JA}=\dfrac{1}{2}$.\\
			      $\dfrac{AI}{AM}=\dfrac{2}{3}$ ($I$ là trọng tâm của $\Delta ABC$).\\
			      Suy ra $\dfrac{AJ}{AS}=\dfrac{AI}{AM}$.\\
			      Theo định lí Viet đảo ta có $IJ\parallel SM$.}{\begin{tikzpicture}
				      \tkzDefPoints{-1/4/S,-2/0/B,2/0/C} %Định nghĩa các toạ đô dịnh cơ sở
				      \tkzDefPointBy[rotation = center B angle 60](C)
				      \tkzGetPoint{A1}
				      \tkzDefPointsBy[homothety=center B ratio 0.5](A1){A}
				      \tkzDefPointBy[translation = from B to A](C)
				      \tkzGetPoint{D}
				      \tkzDefMidPoint(S,B)
				      \tkzGetPoint{P}
				      \tkzDefMidPoint(C,B)
				      \tkzGetPoint{M}
				      \tkzDefMidPoint(C,D)
				      \tkzGetPoint{N}
				      \tkzDefMidPoint(S,D)
				      \tkzGetPoint{Q}
				      \tkzDefPointsBy[homothety=center S ratio 0.3333](A){J}
				      \tkzInterLL(A,M)(B,D)
				      \tkzGetPoint{I}
				      \tkzLabelPoints[left](P)
				      \tkzLabelPoints[above](S)
				      \tkzLabelPoints[right](N,Q,J)
				      \tkzLabelPoints[above left](A,B)
				      \tkzLabelPoints[above](I)
				      %\tkzLabelPoints[right](A,B,C,D,N,M,P)
				      \tkzLabelPoints[right](D)
				      \tkzLabelPoints[below](C,M)
				      \tkzDrawSegments[dashed](A,B S,A A,D Q,P A,C D,B I,J A,M M,N)
				      \tkzDrawSegments(S,B S,C S,D S,M B,C C,D)
			      \end{tikzpicture}}
		\end{enumerate}
	}
\end{vd}

\begin{dang}{Xác định giao tuyến $d$ của hai mặt phẳng cắt nhau}
	\indamm{Ta thực hiện một trong hai cách sau đây:}
	\begin{itemize}
		\item [\iconCH] \indamm{Cách 1:}Tìm hai điểm chung phân biệt (đã xét ở bài học trước)
		\item [\iconCH] \indamm{Cách 2:} Tìm 1 điểm chung. Sau đó nếu hai mặt phẳng có cặp đường thẳng song song nhau thì giao tuyến $d$ sẽ đi qua điểm chung và song song (hoặc trùng) với một trong hai đường thẳng đó.
	\end{itemize}
\end{dang}

\begin{vd}
	Cho tứ diện $ABCD$. Trên $AB$, $AC$ lần lượt lấy $M$, $N$ sao cho $\dfrac{AM}{AB}=\dfrac{AN}{AC}$. Tìm giao tuyến của hai mặt phẳng $(DBC)$ và $(DMN)$.
	\loigiai{
		\begin{center}
			\begin{tikzpicture}[>=stealth,line join=round,line cap=round,font=\footnotesize,scale=1]
				\coordinate (A) at (2,4.5);
				\coordinate (B) at (0,0);
				\coordinate (C) at (6,0);
				\coordinate (D) at (3,-2);
				\coordinate (M) at ($(A)!.7!(B)$);
				\coordinate (N) at ($(A)!.7!(C)$);
				\coordinate (X) at ($(C)+(D)-(B)$);
				\coordinate (Y) at ($(D)!.5!(X)$);
				\coordinate (Z) at ($(D)!-0.3!(X)$);
				\draw (A)--(B)--(D)--(C)--(A)--(D)--(M) (D)--(N) (Y)--(Z);
				\draw[dashed] (M)--(N) (B)--(C);
				\foreach \p/\pos in {A/above,B/left,C/right,D/below,M/above left, N/above right}
				\fill (\p) circle(1pt)node[\pos]{\p};
				\draw (Y) node[above] {$x$};
			\end{tikzpicture}
		\end{center}
		Trong tam giác $ABC$, theo giả thiết $\dfrac{AM}{AB}=\dfrac{AN}{AC}$ suy ra $MN\parallel BC$. \\
		Ta có $\left\{\begin{aligned}
				 & D\in (DBC)\cap (DMN)            \\
				 & BC\subset (DBC),MN\subset (DMN) \\
				 & BC\parallel MN
			\end{aligned}\right. $ suy ra \\
		$(DBC)\cap (DMN)=Dx\parallel BC\parallel MN$.
	}
\end{vd}

\begin{vd}
	Cho tứ diện $ABCD$. Gọi $M$, $N$ lần lượt là trung điểm của $AD$ và $BD$; $G$ là trọng tâm tam giác $ABC$. Tìm giao tuyến của hai mặt phẳng $(ABC)$ và $(MNG)$.
	\loigiai{
		\begin{center}
			\begin{tikzpicture}[>=stealth,line join=round,line cap=round,font=\footnotesize,scale=1]
				\coordinate (A) at (4,4);
				\coordinate (B) at (0,0);
				\coordinate (C) at (6,0);
				\coordinate (D) at (5,-2);
				\coordinate (M) at ($(A)!.5!(D)$);
				\coordinate (N) at ($(B)!.5!(D)$);
				\coordinate (X) at ($(A)!.5!(B)$);
				\coordinate (G) at ($(C)!2/3!(X)$);
				\coordinate (P) at ($(C)!2/3!(A)$);
				\coordinate (Q) at ($(C)!2/3!(B)$);
				\draw (A)--(B)--(D)--(C)--(A)--(D) (M)--(N);
				\draw[dashed] (M)--(G)--(N) (P)--(Q) (C)--(X) (B)--(C);
				\foreach \p/\pos in {A/above, B/left, D/below, C/right, M/right, N/below, P/right, G/above, Q/above}
				\fill (\p) circle(1pt) node[\pos]{\p};
			\end{tikzpicture}
		\end{center}
		Do $M$, $N$ lần lượt là trung điểm của $AD$ và $BD$ nên $MN\parallel AB$. \\
		Ta có $\left\{\begin{aligned}
				 & G\in (ABC)\cap (MNG)            \\
				 & AB\subset (ABC),MN\subset (MNG) \\
				 & AB\parallel MN
			\end{aligned}\right. $ suy ra \\
		$(ABC)\cap (MNG)=PQ\parallel AB\parallel MN$, \\
		với $PQ$ qua $G$ và song song với $AB$.
	}
\end{vd}

\begin{vd}
	Cho hình chóp $S.ABCD$ có đáy $ABCD$ là hình bình hành. Điểm $M$ thuộc cạnh $SA$. Điểm $E$, $F$ lần lượt là trung điểm của $AB$ và $BC$.
	\begin{enumEX}{2}
		\item Tìm $(SAB) \cap (SCD)$.
		\item Tìm $(MBC) \cap (SAD)$.
		\item Tìm $(MEF) \cap (SAC)$.
		\item Tìm $AD \cap (MEF)$.
		\item Tìm $SD \cap (MEF)$.
		%	\item Tìm thiết diện của hình chóp cắt bởi $(MEF)$.
	\end{enumEX}
	\loigiai{
		\begin{enumerate}
			\immini{	\item $\heva{& S \in (SAB) \cap (SCD)\\& AB \subset (SAB),~ CD \subset (SCD)\\& AB \parallel CD}$\\
			      $\Rightarrow (SAB) \cap (SCD) =Sx ~\text{với}~Sx \parallel AB \parallel CD$.
			\item $\heva{& M \in (MBC) \cap (SAD)\\& BC \subset (MBC),~ AD \subset (SAD)\\& BC \parallel AD}$\\
			      $\Rightarrow (MBC) \cap (SAD) =My ~\text{với}~My \parallel BC \parallel AD$.
			\item $\heva{& M \in (MEF) \cap (SAC)\\& EF \subset (MEF),~ AC \subset (SAC)\\& EF \parallel AC}$\\
			      $\Rightarrow (MEF) \cap (SAC) =Mz ~\text{với}~Mz \parallel EF \parallel AC$.
			      }{	\begin{tikzpicture}[scale=0.7,font=\footnotesize,line join=round,line cap=round,>=stealth]
				      \tikzset{label style/.style={font=\footnotesize}}
				      \tkzDefPoints{0/0/A, -3/-3/B, 6/0/D}
				      \coordinate (C) at ($(B)+(D)-(A)$);
				      \coordinate (S) at ($(A)+(1,6)$);
				      \coordinate (M) at ($(S)!2/5!(A)$);
				      \coordinate (E) at ($(A)!1/2!(B)$);
				      \coordinate (F) at ($(B)!1/2!(C)$);
				      \tkzDefPointBy[translation=from A to B](S) \tkzGetPoint{x}
				      \coordinate (U) at ($(x)!5/4!(S)$);
				      \tkzDefPointBy[translation=from A to D](M) \tkzGetPoint{y}
				      \coordinate (V) at ($(y)!5/4!(M)$);
				      \tkzInterLL(y,V)(S,B)    \tkzGetPoint{Z}
				      \tkzInterLL(y,V)(S,D)    \tkzGetPoint{W}
				      \tkzDefPointBy[translation=from A to C](M) \tkzGetPoint{z}
				      \coordinate (T) at ($(z)!2/5!(M)$);
				      \tkzInterLL(z,M)(S,C)    \tkzGetPoint{K}
				      \tkzInterLL(E,F)(A,D)    \tkzGetPoint{I}
				      \tkzInterLL(E,F)(S,B)    \tkzGetPoint{J}
				      \tkzInterLL(A,D)(S,B)    \tkzGetPoint{L}
				      \tkzInterLL(I,M)(S,B)    \tkzGetPoint{a}
				      \tkzInterLL(I,M)(S,D)    \tkzGetPoint{N}
				      \tkzDrawPolygon(S,B,C,D)
				      \tkzDrawSegments(S,C x,U y,W V,Z K,T I,L I,J I,a K,N K,F)
				      \tkzDrawSegments[dashed](S,A A,B A,D B,M C,M Z,W M,E E,F M,F A,C M,K A,L E,J a,N)
				      \tkzDrawPoints[fill=black](S,A,B,C,D,E,F,M,I,N,K)
				      \tkzLabelPoints[above](S,x,y)
				      \tkzLabelPoints[right](D,K)
				      \tkzLabelPoints[left](I)
				      \tkzLabelPoints[above left](A,M)
				      \tkzLabelPoints[above right](N)
				      \tkzLabelPoints[above right,,xshift=-1cm,yshift=0.8cm](z)
				      \tkzLabelPoints[below](B,C,F)
				      \tkzLabelPoints[below,xshift=-0.1cm,yshift=-0.2cm](E)
			      \end{tikzpicture}}
			\item Trong $(ABCD)$, gọi $I=EF \cap AD$.\\
			      Mà $EF \subset (MEF)$ nên $AD \cap (MEF)=I$.
			\item Trong $(SAD)$, gọi $N=SD \cap IM$.\\
			      Mà $IM \subset (MEF)$ nên $SD \cap (MEF)=N$.
			      %	\item Thiết diện của hình chóp cắt bởi $(MEF)$ là ngũ giác $MNKFE$.
		\end{enumerate}
	}
\end{vd}

\subsection{BÀI TẬP TỰ LUYỆN}
\begin{bt}
	Cho hình chóp $S.ABCD$ có đáy $ABCD$ là hình bình hành tâm $O$. Gọi $M$, $N$ lần lượt là trung điểm của $SA$, $SD$. Chứng minh
	\begin{listEX}[2]
		\item $MN\parallel AD$ và $MN\parallel BC$;
		\item $MO\parallel SC$ và $NO\parallel SB$.
	\end{listEX}
	\loigiai{
		\begin{center}
			\begin{tikzpicture}[line join = round, line cap = round,>=stealth,font=\footnotesize,scale=1]
				\tkzDefPoints{0/0/A}
				\coordinate (B) at ($(A)+(5,0)$);
				\tkzDefShiftPoint[A](-150:3){D}
				\coordinate (C) at ($(B)+(D)-(A)$);
				\tkzInterLL(A,C)(B,D)    \tkzGetPoint{O}
				\coordinate (S) at ($(A)+(-0.2,4)$);
				\coordinate (M) at ($(S)!0.5!(A)$);
				\coordinate (N) at ($(S)!0.5!(D)$);
				\tkzDrawPolygon(S,B,C,D)
				\tkzDrawSegments(S,C)
				\tkzDrawSegments[dashed](A,S A,B A,D A,C B,D M,N O,M O,N)
				\tkzMarkSegments[mark=||,size=2pt](M,A M,S)
				\tkzMarkSegments[mark=|,size=2pt](N,S N,D)
				\tkzDrawPoints[fill=black](A,B,D,C,O,S,M,N)
				\tkzLabelPoints[above](S)
				\tkzLabelPoints[below](O)
				\tkzLabelPoints[left](A,D,N)
				\tkzLabelPoints[above left](M)
				\tkzLabelPoints[right](B)
				\tkzLabelPoints[below right](C)
			\end{tikzpicture}
		\end{center}
		\begin{enumerate}
			\item Xét tam giác $SAD$ có
			      \begin{itemize}
				      \item $M$ là trung điểm của $SA$ (giả thiết);
				      \item $N$ là trung điểm của $SD$ (giả thiết).
			      \end{itemize}
			      Suy ra $MN$ là đường trung bình của $\triangle SAD$. Do đó $MN\parallel AD$.\\
			      Ta có $\heva{&MN\parallel AD~ (\text{chứng minh trên})\\ &BC\parallel AD ~(ABCD \text{ là hình bình hành})}\Rightarrow MN\parallel BC$.
			\item Xét tam giác $ASC$ có
		\end{enumerate}
		\begin{itemize}
			\item $M$ là trung điểm của $SA$ (giả thiết);
			\item $O$ là trung điểm của $AC$ ($O$ là tâm của hình bình hành $ABCD$).
		\end{itemize}
		Suy ra $OM$ là đường trung bình của $\triangle SAC$. Do đó $MO\parallel SC$.\\
		Tương tự, $NO$ là đường trung bình của $\triangle SDB$ nên $NO\parallel SB$.
	}
\end{bt}

\begin{bt}
	Cho hình chóp $S.ABCD$ có đáy $ABCD$ là hình bình hành tâm $O$. Gọi $M$, $N$ lần lượt là trung điểm của $AB$, $AD$. Gọi $I$, $J$, $G$ lần lượt là trọng tâm của các tam giác $SAB$, $SAD$ và $AOD$. Chứng minh
	\begin{listEX}[2]
		\item $IJ\parallel MN$;
		\item $IJ\parallel BD$ và $GJ\parallel SO$.
	\end{listEX}
	\loigiai{
		\begin{center}
			\begin{tikzpicture}[line join = round, line cap = round,>=stealth,font=\footnotesize,scale=1]
				\tkzDefPoints{0/0/A}
				\coordinate (D) at ($(A)+(5,0)$);
				\tkzDefShiftPoint[A](30:2.5){B}
				\coordinate (C) at ($(B)+(D)-(A)$);
				\tkzInterLL(A,C)(B,D)    \tkzGetPoint{O}
				\coordinate (S) at ($(B)+(0.2,3)$);
				\coordinate (M) at ($(A)!0.5!(B)$);
				\coordinate (N) at ($(A)!0.5!(D)$);
				\tkzDefPointBy[homothety = center S ratio 2/3](M)    \tkzGetPoint{I}
				\tkzDefPointBy[homothety = center S ratio 2/3](N)    \tkzGetPoint{J}
				\tkzDefPointBy[homothety = center O ratio 2/3](N)    \tkzGetPoint{G}
				\tkzDrawPolygon(S,A,D,C)
				\tkzDrawSegments(S,D S,N)
				\tkzDrawSegments[dashed](O,S B,S B,A B,C B,D A,C S,M M,N O,N I,J J,G)
				\tkzMarkSegments[mark=||,pos=.5,size=2pt](N,A N,D)
				\tkzMarkSegments[mark=|,pos=.5,size=2pt](M,A M,B)
				\tkzDrawPoints[fill=black](A,B,D,C,O,S,I,J,G)
				\tkzLabelPoints[above](S)
				\tkzLabelPoints[below](N,B)
				\tkzLabelPoints[left](A,M,I)
				\tkzLabelPoints[above right](J)
				\tkzLabelPoints[right](C,G)
				\tkzLabelPoints[below right](D,O)
			\end{tikzpicture}
		\end{center}
		\begin{enumerate}
			\item Xét tam giác $SMN$ có
			      \begin{itemize}
				      \item $SI=\dfrac{2}{3}SM$ ($I$ là trọng tâm của $\triangle SAB$);
				      \item $SJ=\dfrac{2}{3}SN$ ($J$ là trọng tâm của $\triangle SAD$).
			      \end{itemize}
			      suy ra $IJ\parallel MN$ (định lý Ta-lét đảo).
			\item Vì $MN$ là đường trung bình của $\triangle ABD$ nên $MN\parallel BD$.\\
			      Mà $IJ\parallel MN$ (chứng minh trên) nên $IJ\parallel BD$.\\
			      Xét tam giác $SON$ có
			      \begin{itemize}
				      \item $NG=\dfrac{1}{3}NO$ ($G$ là trọng tâm của $\triangle AOD$);
				      \item $NJ=\dfrac{1}{3}SN$ ($J$ là trọng tâm của $\triangle SAD$).
			      \end{itemize}
			      suy ra $GJ\parallel SO$ (định lý Ta-lét đảo).
		\end{enumerate}
	}
\end{bt}

\begin{bt}
	Cho hình chóp $SABCD$ có đáy $ABCD$ là hình thang đáy lớn $AB$. Gọi $E$,
	$F$ lần lượt là trung điểm của $SA$ và $SB$.
	\begin{listEX}[3]
		\item Chứng minh $EF \parallel CD$.
		\item Tìm $I = AF \cap (SCD)$.
		\item Chứng minh $SI \parallel AB \parallel CD$.
	\end{listEX}
	\loigiai{
		\begin{center}
			\begin{tikzpicture}[scale=1, line join=round, line cap=round]
				\tkzDefPoints{0/0/A,4/0/B,2.5/-1.6/C,1/3/S}
				\tkzDefPointBy[translation=from B to A](C)\tkzGetPoint{D'}
				\tkzDefPointBy[homothety=center C ratio 0.9](D')\tkzGetPoint{D}
				\coordinate (E) at ($(S)!0.5!(A)$);
				\coordinate (F) at ($(S)!0.5!(B)$);
				\tkzDefPointBy[translation=from A to B](S)\tkzGetPoint{x}
				\tkzInterLL(A,F)(S,x)\tkzGetPoint{I}
				\tkzInterLL(S,C)(I,D)\tkzGetPoint{K}

				\tkzDrawPolygon(S,B,C,D)
				\tkzDrawSegments(S,C I,D F,I S,I)
				\tkzDrawSegments[dashed](A,S A,B A,D A,F E,F)
				\tkzDrawPoints[fill=black,size=4](D,C,A,B,S,E,F,I)

				\tkzLabelPoints[above](S,F)
				\tkzLabelPoints[below](A,B,C)
				\tkzLabelPoints[right](D,E)
				\tkzLabelPoints[left](I)
			\end{tikzpicture}
		\end{center}
		\begin{listEX}[]
			\item Ta có $EF$ là đường trung bình của tam giác $SAB$ nên $EF
				\parallel AB$\\
			mà $AB \parallel CD$ (hai đáy của hình thang)\\
			nên $EF \parallel CD$.
			\item Hai mặt phẳng $(SAB)$ và $(SCD)$ có $AB \parallel CD$ nên giao
			tuyến là đường thẳng $Sx \parallel AB \parallel CD$.\\
			Kéo dài $AF$ cắt $Sx$ tại $I$.\\
			Ta thấy $I$ là điểm chung của $AF$ và $(SCD)$.
			\item Theo ý \circled{\textbf{2}}.
		\end{listEX}
	}
\end{bt}

\begin{bt}
	Cho hình chóp $S.ABCD$ có đáy $ABCD$ là hình bình hành. Gọi $M$, $N$ lần lượt là trung điểm của $SA$, $SB$. Gọi $P$ là một điểm trên cạnh $BC$. Tìm giao tuyến của
	\begin{listEX}[3]
		\item $(SBC)$ và $(SAD)$;
		\item $(SAB)$ và $(SCD)$;
		\item $(MNP)$ và $(ABCD)$.
	\end{listEX}
	\loigiai{
		\begin{center}
			\begin{tikzpicture}[line join = round, line cap = round,>=stealth,font=\footnotesize,scale=1]
				\tkzDefPoints{0/0/A}
				\coordinate (D) at ($(A)+(5,0)$);
				\tkzDefShiftPoint[A](30:2.5){B}
				\coordinate (C) at ($(B)+(D)-(A)$);
				\tkzInterLL(A,C)(B,D)    \tkzGetPoint{O}
				\coordinate (S) at ($(A)+(0.5,4)$);
				\coordinate (M) at ($(S)!0.5!(A)$);
				\coordinate (N) at ($(S)!0.5!(B)$);
				\coordinate (P) at ($(B)!0.6!(C)$);
				\coordinate (Q) at ($(A)!0.6!(D)$);
				%
				\coordinate (E) at ($(A)+(S)-(B)$);
				\coordinate (y) at ($(S)+(B)-(A)$);
				\coordinate (F) at ($(E)!0.8!(S)$);
				%
				\draw (0,4)--(4,4) node at (4.2,4){$x$};
				\tkzDrawPolygon(S,A,D,C)
				\tkzDrawPolygon[fill=cyan,dashed](M,N,P)
				\tkzDrawSegments(S,D F,y)
				\tkzDrawSegments[dashed](B,S B,A B,C B,D A,C P,Q)
				\tkzMarkSegments[mark=||,pos=.5,size=2pt](N,S N,B)
				\tkzMarkSegments[mark=|,pos=.5,size=2pt](M,A M,S)
				\tkzDrawPoints[fill=black](A,B,D,C,O,S,M,N,Q,P)
				\tkzLabelPoints[above](S,O,y)
				\tkzLabelPoints[below](Q)
				\tkzLabelPoints[left](M,B)
				\tkzLabelPoints[above right](P,N)
				\tkzLabelPoints[right](C,D)
				\tkzLabelPoints[below left](A)
			\end{tikzpicture}
		\end{center}
		\begin{enumerate}
			\item Ta có
			      \begin{itemize}
				      \item $(SBC)\cap (ABCD)=BC$;
				      \item $(SAD)\cap (ABCD)=AD$;
				      \item $AD\parallel BC$ ($ABCD$ là hình bình hành).
			      \end{itemize}
			      Mà $S$ là điểm chung của 2 mặt phẳng $(SBC)$ và $(SAD)$ nên giao tuyến của 2 mặt phẳng $(SBC)$ và $(SAD)$ là đường thẳng $Sx\parallel BC\parallel AD$.
		\end{enumerate}
		\begin{enumerate}
			\setcounter{enumi}{1}
			\item Giao tuyến của hai mặt phẳng $(SAB)$ và $(SCD)$ là đường thẳng $Sy\parallel AB\parallel CD$.
			\item Vì $MN\parallel AB$ ($MN$ là đường trung bình của $\triangle SAB$) nên qua $P$ kẻ $PQ\parallel AB~(Q\in AD)$. Khi đó giao tuyến của hai mặt phẳng $(MNP)$ và $(ABCD)$ là đường thẳng $PQ$.
		\end{enumerate}
	}
\end{bt}
\begin{bt}
	Cho tứ diện $SABC$. Gọi $E$ và $F$ lần lượt là trung điểm của các cạnh $SB$ và $AB$, $G$ là một điểm trên cạnh $AC$. Tìm giao tuyến của các cặp mặt phẳng sau
	\begin{listEX}[2]
		\item $(SAC)$ và $(EFC)$;
		\item $(SAC)$ và $(EFG)$.
	\end{listEX}
	\loigiai{
		\begin{center}
			\begin{tikzpicture}[line join = round, line cap = round,>=stealth,font=\footnotesize,scale=1]
				\tkzDefPoints{0/0/A}
				\coordinate (C) at ($(A)+(5,0)$);
				\tkzDefShiftPoint[A](-30:3){B}
				\coordinate (S) at ($(A)+(1.5,4)$);
				\coordinate (E) at ($(S)!0.5!(B)$);
				\coordinate (F) at ($(A)!0.5!(B)$);
				\coordinate (G) at ($(A)!0.13!(C)$);
				\coordinate (H) at ($(S)!0.13!(C)$);
				\coordinate (x) at ($(S)+(C)-(A)$);
				\coordinate (D) at ($(A)+(C)-(S)$);
				\coordinate (K) at ($(D)!0.8!(C)$);
				\tkzDrawPolygon(S,A,B,C)
				\tkzDrawSegments(S,B K,x E,F E,C)
				\tkzDrawSegments[dashed](A,C G,H F,C G,E F,G)
				\tkzMarkSegments[mark=||,pos=.5,size=2pt](E,S E,B)
				\tkzMarkSegments[mark=|,pos=.5,size=2pt](F,A F,B)
				\tkzDrawPoints[fill=black](B,C,A,S,E,F,G,H)
				\tkzLabelPoints[above](S)
				\tkzLabelPoints[below](B)
				\tkzLabelPoints[left](A)
				\tkzLabelPoints[right](C,x)
				\tkzLabelPoints[above right](H,E)
				\tkzLabelPoints[below left](F)
				\tkzLabelPoints[above left](G)
			\end{tikzpicture}
		\end{center}
		\begin{enumerate}
			\item Ta có
			      \begin{itemize}
				      \item $(SAC)\cap (SAB)=SA$;
				      \item $(EFC)\cap (SAB)=EF$;
				      \item $SA\parallel EF$ ($EF$ là đường trung bình của $\triangle SAB$).
			      \end{itemize}
			      Do đó giao tuyến của 2 mặt phẳng $(SAC)$ và $(EFC)$ sẽ song song với $SA$ và $EF$.\\
			      Mà $C$ là điểm chung của 2 mặt phẳng $(SAC)$ và $(EFC)$ nên giao tuyến của chúng là đường thẳng $Cx\parallel SA\parallel EF$.

			\item Vì $EF\parallel SA$ ($EF$ là đường trung bình của $\triangle SAB$) nên qua $G$ kẻ $GH\parallel SA~(H\in SC)$. Khi đó giao tuyến của hai mặt phẳng $(SAC)$ và $(EFG)$ là đường thẳng $GH$.
		\end{enumerate}
	}
\end{bt}

\begin{bt}
	Cho hình chóp $S.ABCD$ có đáy $ABCD$ là hình bình hành. Gọi $G$ là trọng tâm tam giác $ABD$, $N$ là trung điểm $SG$. Tìm giao tuyến của hai mặt phẳng $(ABN)$ và $(SCD)$.
	\loigiai{
		\begin{center}
			\begin{tikzpicture}[>=stealth,line join=round,line cap=round,font=\footnotesize,scale=0.9]
				\coordinate (S) at (0,5);
				\coordinate (A) at (-3,-1.5);
				\coordinate (B) at (0,0);
				\coordinate (C) at (5,0);
				\coordinate (D) at ($(A)+(C)-(B)$);
				\coordinate (O) at (intersection of A--C and B--D);
				\coordinate (G) at ($(A)!2/3!(O)$);
				\coordinate (N) at ($(S)!.5!(G)$);
				\coordinate (P) at (intersection of A--N and S--C);
				\coordinate (X) at ($(P)+(D)-(C)$);
				\coordinate (Q) at (intersection of P--X and S--D);
				\draw (S)--(A)--(D)--(C)--(S)--(D) (P)--(Q);
				\draw[dashed] (S)--(B)--(A)--(O) (B)--(D) (S)--(G) (B)--(C) (A)--(P);
				\foreach \p/\pos in {S/above, A/below, D/below, C/right, B/above right, G/below, P/right, Q/left, N/left, O/below}
				\fill (\p) circle(1pt) node[\pos]{\p};
			\end{tikzpicture}
		\end{center}
		Trong mặt phẳng $(SAC)$, gọi $P=AN\cap SC$. Ta có \\
		$P\in AN$ mà $AN\subset (ABN)$ suy ra $P\in (ABN)$. \\
		$P\in SC$ mà $SC\subset (SCD)$ suy ra $P\in (SCD)$. \\
		Do đó $P\in (ABN)\cap (SCD)$. \\
		Ta có $\left\{\begin{aligned}
				 & P\in (ABN)\cap (SCD)            \\
				 & AB\subset (ABN),CD\subset (SCD) \\
				 & AB\parallel CD
			\end{aligned}\right. $ suy ra \\
		$(ABN)\cap (SCD)=PQ\parallel CD\parallel AB$.
	}
\end{bt}

\begin{bt}
	Cho tứ diện $ABCD$. Gọi $M, N$ theo thứ tự là trung điểm của $AB, BC$ và $Q$ là một điểm nằm trên cạnh $AD$ ($QA\neq QD$) và $P$ là giao điểm của $CD$ với mặt phẳng $(MNQ)$. Chứng minh rằng $PQ\parallel MN$ và $PQ\parallel AC$.
	\loigiai{
		\immini{
			Vì $QA\neq QD$ nên gọi $K=QM\cap BD$ suy ra $KN \cap CD=P$.\\
			Theo định lý về giao tuyến ba mặt phẳng\\
			Ta xét ba mặt phẳng $(ABC)$ $(ACD)$ và $(MNQ)$.\\
			Ta có: $\heva{&(ABC)\cap (ACD)=AC\\&(ABC)\cap (MNQ)=MN\\&(ACD)\cap (MNQ)=QP}$.\\
			Vậy $AC \parallel MN $ nên $AC\parallel QP\parallel NM$.}
		{\begin{tikzpicture}%[scale=1]
				%\tkzInit[xmin=-5,ymin=-1.5,xmax=3,ymax=2]
				%\tkzClip
				\tkzDefPoints{-3/0/B,2/0/D,-1/3/A} % Định nghĩa các toạ đô dịnh cơ sở
				\tkzDefPointsBy[homothety=center B ratio 0.6](D){B1}
				\tkzDefPointBy[rotation = center B angle -30](B1)
				\tkzGetPoint{C}
				\tkzDefMidPoint(A,B)
				\tkzGetPoint{M}
				\tkzDefMidPoint(C,B)
				\tkzGetPoint{N}
				\tkzDefPointsBy[homothety=center A ratio 0.3](D){Q}
				\tkzInterLL(M,Q)(B,D)
				\tkzGetPoint{K}
				\tkzInterLL(K,N)(C,D)
				\tkzGetPoint{P}
				\tkzLabelPoints[above](A)
				\tkzLabelPoints[above right](Q)
				\tkzLabelPoints[above left](K,B,M)
				\tkzLabelPoints[right](D,P)
				\tkzLabelPoints[below](C,N)
				\tkzDrawSegments[dashed](B,D M,Q M,B B,N B,K N,P)
				\tkzDrawSegments(A,M A,C A,D A,M Q,P C,D N,C M,K N,K M,N)
			\end{tikzpicture}}
	}
\end{bt}

\begin{bt}
	Cho hình chóp $ S.ABCD $ có đáy $ ABCD $ là hình thang với $ AD $ là đáy lớn và $ AD=2BC $. Gọi $ M $, $ N $, $ P $ lần lượt thuộc các đoạn $ SA $, $ AD $, $ BC $ sao cho $ MA=2MS $, $ NA=2ND $, $ PC=2PB $.
	\begin{itemize}
		\item[a)] Tìm giao tuyến của các cặp mặt phẳng sau: $ (SAD) $ và $ (SBC) $, $ (SAC) $ và $ (SBD) $.
		\item[b)] Xác định giao điểm $ Q $ của $ SB $ với $ (MNP) $.
		\item[c)] Gọi $ K $ là trung điểm của $ SD $. Chứng minh $ CK=(MQK)\cap (SCD) $.
	\end{itemize}
	\loigiai{
		\begin{center}
			\begin{tikzpicture}[line join=round, line cap=round,thick,scale=0.8]
				\tikzset{label style/.style={font=\footnotesize}}
				\coordinate (A) at (0,0);
				\coordinate (B) at (-0.8,-3);
				\coordinate (D) at (10,0);
				\tkzDefPointWith[colinear = at B,K=0.5](A,D)
				\tkzGetPoint{C}
				\coordinate (S) at ($(A)+(-0.5,6)$);
				\coordinate (N) at ($(A)!0.6667!(D)$);
				\coordinate (M) at ($(S)!0.3333!(A)$);
				\coordinate (P) at ($(B)!0.3333!(C)$);

				\tkzInterLL(A,C)(B,D) \tkzGetPoint{O}
				\tkzInterLL(N,P)(B,A) \tkzGetPoint{E}
				\tkzInterLL(E,M)(B,S) \tkzGetPoint{Q}
				\coordinate (K) at ($(S)!0.5!(D)$);
				\coordinate (K') at ($(S)!0.5!(A)$);
				\coordinate (t) at ($(S)+(K)-(K')$);
				\tkzInterLL(K,M)(t,S) \tkzGetPoint{F}
				\tkzDrawSegments(S,B S,C B,C C,D S,D S,t Q,E C,K P,E B,E F,C S,F)
				\tkzDrawSegments[dashed,thin](S,A A,B A,C A,D B,D S,O N,P M,Q M,N F,K Q,K K,K')
				\tkzDrawPoints[fill=black,size=3pt](S,A,B,C,D,M,N,P,O,E,Q,K,F,K')
				\tkzLabelPoints[above](S,N,K,t)
				\tkzLabelPoint[above](F){$ F\equiv F' $}
				\tkzLabelPoints[above left](A)
				\tkzLabelPoints[below](C,P,E)
				\tkzLabelPoints[right](D)
				\tkzLabelPoints[below left](Q)
				\tkzLabelPoints[below right](K',B)
				\tkzLabelPoints[above right](M,O)
			\end{tikzpicture}
		\end{center}
		\begin{listEX}
			\item Vì $ \heva{&S\in (SAD) \cap (SBC)\\&AD\subset (SAD) \text{ và } BC\subset (SBC) \\& AD\parallel BC} $\\ nên $ (SAD)\cap (SBC) = St \parallel AD \parallel BC $.\\
			Gọi $ O=AC\cap BD\Rightarrow \heva{&O\in AC\subset (SAC) \\ &O\in BD\subset (SBD)} $
			suy ra $ SO=(SAC)\cap (SBD) $.
			\item Gọi $ E=NP\cap AB $ và $ Q=EM\cap SB $.
			Vì $ \heva{&Q\in SB \\&Q\in ME\subset (MNP)} $ nên $ Q=SB\cap (MNP) $.
			\item Gọi $ F=MK\cap St $ và $ F'=QC\cap St $. Dựa vào các vị trí các điểm $ Q $, $ C $, $ M $ và $ K $ của giả thiết cho, dễ thấy $ F $ và $ F' $ cùng nằm về một phía so với mặt phẳng $ (SAB) $.\\
			Trong mặt phẳng $ (SF'BC) $, áp dụng định lý Thales (để ý rằng $ SF'\parallel BC $) ta có
			\[ \dfrac{QS}{QB}=\dfrac{BC}{SF'}=\dfrac{1}{2}. \tag {1}\]
			Gọi $ K' $ là trung điểm của $ SA $. suy ra $ \dfrac{MK'}{MS}=\dfrac{1}{2}. $\\
			Trong mặt phẳng $ (SFAD) $, áp dụng định lý Thales (để ý rằng $ SF\parallel KK' $) ta có
			\[ \dfrac{MK'}{MS}=\dfrac{KK'}{SF}=\dfrac{1}{2}. \tag {2}\]
			Từ (1), (2) và $ AD=2BC $ suy ra $ SF=SF' $. Do đó $ F\equiv F' $, suy ra bốn điểm $ Q $, $ C $, $ M $ và $ K $ đồng phẳng.\\
			Vậy $ CK=(MQK)\cap (SCD) $.
		\end{listEX}
	}
\end{bt}

\begin{bt}
	Cho hình chóp $ S.ABCD $ có $ O $ là tâm của hình bình hành $ ABCD $, điểm $ M $ thuộc cạnh $ SA $ sao cho $ SM=2MA $, $ N $ là trung điểm của $ AD $.
	\begin{itemize}
		\item[a)] Tìm giao tuyến của mặt phẳng $ (SAD) $ và $ (MBC) $.
		\item[b)] Tìm giao điểm $ I $ của $ SB $ và $ (CMN) $, giao điểm $ J $ của $ SA $ và $ (ICD) $.
		\item[c)] Chứng minh ba đường thẳng $ ID $, $ JC $, $ SO $ cắt nhau tại $ E $. Tính tỉ số $ \dfrac{SE}{SO} $.
	\end{itemize}
	\loigiai{
		\begin{center}
			\begin{tikzpicture}[line join=round, line cap=round,thick,scale=0.8]
				\tikzset{label style/.style={font=\footnotesize}}
				\coordinate (A) at (0,0);
				\coordinate (B) at (-2.5,-2);
				\coordinate (D) at (8,0);
				\coordinate (C) at ($(B)+(D)-(A)$);
				\coordinate (O) at ($(A)!0.5!(C)$);
				\coordinate (S) at ($(O)+(-1,7)$);
				\coordinate (M) at ($(S)!0.6667!(A)$);
				\coordinate (P) at ($(S)!0.6667!(D)$);
				\coordinate (N) at ($(A)!0.5!(D)$);
				\coordinate (S') at ($(S)+(A)-(D)$);
				\coordinate (t) at ($(S)+(P)-(M)$);
				\tkzInterLL(S,S')(M,N) \tkzGetPoint{F}
				\tkzInterLL(S,B)(C,F) \tkzGetPoint{I}
				\coordinate (I') at ($(I)+(D)-(C)$);
				\tkzInterLL(I,I')(S,A) \tkzGetPoint{J}
				\tkzInterLL(S,O)(I,D) \tkzGetPoint{E}
				\tkzDrawSegments(S,B S,C B,C C,F S,F C,D S,D P,C S,t)
				\tkzDrawSegments[dashed,thin](S,A A,B B,M A,C A,D S,O B,D M,P F,N C,N I,J I,D J,C C,M)
				\tkzDrawPoints[fill=black,size=3pt](S,A,B,C,D,O,M,N,P,F,I,J,E)
				\tkzLabelPoints[above](S,F,N,t)
				\tkzLabelPoints[above right](J,M,E)
				\tkzLabelPoints[left](A)
				\tkzLabelPoints[below left](I)
				\tkzLabelPoints[below](B,C,O)
				\tkzLabelPoints[right](D,P)
			\end{tikzpicture}
		\end{center}
		\begin{listEX}
			\item Vì $ \heva{& M\in (MBC) \cap (SAD) \\ & BC \subset (MBC) \text{ và } AD \subset (SAD) \\& BC\parallel AD}$\\ nên $ (SAD)\cap (MBC)=MP\parallel BC \parallel AD $ (với $ P\in SD $).
			\item Vì $ \heva{& S\in (SAD)\cap (SBC)\\& AD\subset (SAD) \text{ và } BC\subset (SBC) \\ &AD \parallel BC} $ \\ nên $ (SAD) \cap (SBC) =St \parallel AD \parallel BC $.\\
			Gọi $ F=MN\cap St $; $ I=CF\cap SB $.\\
			Vì $ \heva{&I\in SB \\ &I\in CF \subset (CMN)} $ nên $ I=SB\cap (CMN) $.\\
			Qua $ I $ kẻ đường thẳng song song với $ AB $  cắt $ SA $ tại $ J $.\\
			Vì $ \heva{&J\in SA \\ & J\in JI\subset (ICD) (\text{vì } IJ\parallel CD \Rightarrow (IJCD)\equiv (ICD))} $ nên $ J=SA\cap (ICD) $.
			\item Xét $ 3 $ mặt phẳng $ (SAC) $, $ (SBD) $ và $ (CDJI) $, ta có
			$$\heva{& SO=(SAC)\cap (SBD)\\& ID=(SBD)\cap (CDJI)\\& JC=(SAC)\cap (CDJI).}$$
			Do đó ba đường thẳng $ ID $, $ JC $, $ SO $ đồng quy. Gọi điểm đồng quy là $ E $.\\
			Trong mặt phẳng $ (SFAD) $, áp dụng định lý Thales (để ý rằng $ AN\parallel SF $) ta có
			\[ \dfrac{MA}{MS}=\dfrac{AN}{SF}=\dfrac{1}{2}. \]
			Suy ra $ SF=AD=BC $ và $ SFBC $ là hình bình hành.\\
			$ I=SB\cap CF $ nên $ I $ là trung điểm của $ SB $.\\
			$ \triangle SBD $ có $ DI $ và $ SO $ là trung tuyến nên $ E $ là trọng tâm của $ \triangle SBD $.\\
			Vậy $ \dfrac{SE}{SO}=\dfrac{2}{3} $.
		\end{listEX}
	}
\end{bt}
\begin{bt}
	Cho hình chóp $S.ABCD$ có đáy $ABCD$ là hình bình hành. Gọi $M, N, P, Q$ lần lượt là trung điểm của các cạnh $AB, BC, CD, DA$; gọi $I, J, K, L$ lần lượt là trung điểm của các đoạn thẳng $SM, SN, SP, SQ$.
	\begin{itemize}
		\item [a)] Chứng minh rằng bốn điểm $I, J, K, L$ đồng phẳng và tứ giác $IJKL$ là hình bình hành.
		\item [b)] Chứng minh rằng $IK\parallel BC$.
		\item [c)] Xác định giao tuyến của hai mặt phẳng $\left(IJKL\right)$ và $\left(SBC\right)$.
	\end{itemize}
\end{bt}

