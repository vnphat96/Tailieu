\subsection{Các dạng toán thường gặp}
\begin{dang}{Tìm giao tuyến bằng cách kẻ song song}
		Sử dụng tính chất
	$\heva{&(\alpha)\parallel(\beta)\\ &(\gamma)\cap (\alpha)=a\\ &(\gamma)\cap (\beta)=b}\Rightarrow a\parallel b$
\end{dang}
\begin{vd}%[1H2B4-2]
	Cho hình chóp $S.ABCD$ đáy $ABCD$ là hình bình hành tâm $H$. Mặt phẳng $(P)$ đi qua $H$ và song song với $(SAB)$. Tìm giao tuyến của
	\begin{enumerate} 
		\item Mặt phẳng $(P)$ và mặt phẳng $(ABCD)$.
		\item Mặt phẳng $(P)$ và mặt phẳng $(SBC)$.
	\end{enumerate}
	\loigiai{
		\immini{
			\begin{enumerate} 
				\item Giao tuyến của mặt phẳng $(P)$ và mặt phẳng $(ABCD)$.\\
				$\heva{&(P)\parallel (SAB)\\ &(ABCD)\cap (SAB)=AB\\ &(P)\cap (ABCD)=H}$\\
				$\Leftrightarrow (P)\cap (ABCD)=EF$ với $\heva{&EF \text{ qua } H\\ & EF\parallel AB\\& E\in BC, F\in AD}$.
				\item Giao tuyến của mặt phẳng $(P)$ và mặt phẳng $(SBC)$.\\
				$\heva{&(P)\parallel (SAB)\\ &(SBC)\cap (SAB)=SB\\ &(P)\cap (SBC)=E}$\\
				$\Leftrightarrow (P)\cap (ABCD)=EK$ với $\heva{&EK \cap SC= K\\ & EK\parallel SB}$.
			\end{enumerate}
		}{
			\begin{tikzpicture}[scale=0.5,line cap=round,line join =round]
				\tkzDefPoints{0/0/A,-3/-3/B, 10/0/D}
				\coordinate (S) at($(A)+(-1,5)$);
				\coordinate (C) at($(B)+(D)-(A)$);
				\tkzInterLL (A,C)(B,D)\tkzGetPoint{H}
				\coordinate (x) at($(H)+(B)-(A)$);
				\tkzInterLL(H,x)(C,B)\tkzGetPoint{E}
				\tkzInterLL(H,x)(D,A)\tkzGetPoint{F}
				\coordinate (y) at($(E)+(S)-(B)$);
				\tkzInterLL(E,y)(S,C)\tkzGetPoint{K}
				\tkzDrawSegments(S,B B,C C,D D,S S,C E,K)
				\tkzDrawSegments[dashed](S,A A,B A,D E,F A,C B,D)
				%	\tkzLabelPoints[above left]()
				\tkzLabelPoints[above right](A,S,F)
				%	\tkzLabelPoints[above]()
				\tkzLabelPoints[below](B,E,H)
				\tkzLabelPoints[below left](K)
				\tkzLabelPoints[below right](C,D)
				\tkzDrawPoints[fill=black](A,B,C,E,F,H,K)
			\end{tikzpicture}
		}
	}
\end{vd}

\begin{vd}%[1H2B4-2]
	Cho hình chóp $S.ABCD$ có đáy $ABCD$ là hình bình hành. Gọi $M$ là điểm bất kỳ trên $AB$. Gọi $(\alpha)$ là măt phẳng qua $M$ và song song với $(SBC)$. Tìm giao tuyến của $(\alpha)$ với cắt mặt của hình chóp.
	\loigiai{
		\immini{
			Ta có
			\begin{itemize}
				\item $\heva{&(\alpha)\parallel (SBC)\\ & (SBC)\cap (ABCD)=BC\\ &(\alpha)\cap (ABCD)=M}$\\
				$\Leftrightarrow (\alpha)\cap (ABCD)=MN$ với $\heva{&MN \cap CD= N\\ & MN\parallel BC}$.
				\item $\heva{&(\alpha)\parallel (SBC)\\ & (SBC)\cap (SCD)=SC\\ &(\alpha)\cap (SCD)=N}$\\
				$\Leftrightarrow (\alpha)\cap (SCD)=NP$ với $\heva{&NP \cap SD= P\\ & NP\parallel SC}$.
			\end{itemize}
		}{
			\begin{tikzpicture}[scale=0.5,line cap=round,line join =round]
				\tkzDefPoints{0/0/A, -3/-3/B, 10/0/D}
				\coordinate (S) at($(A)+(-1,5)$);
				\coordinate (C) at($(B)+(D)-(A)$);
				\coordinate (M) at($(A)!0.4!(B)$);
				\coordinate (x) at($(M)+(S)-(A)$);
				\tkzInterLL(M,x)(S,B)\tkzGetPoint{Q}
				\coordinate (y) at($(M)+(D)-(A)$);
				\tkzInterLL(M,y)(D,C)\tkzGetPoint{N}
				\coordinate (t) at($(Q)+(N)-(M)$);
				\tkzInterLL(Q,t)(S,D)\tkzGetPoint{P}
				\tkzDrawSegments(S,B B,C C,D D,S S,C N,P)
				\tkzDrawSegments[dashed](S,A A,B A,D M,N M,Q Q,P)
				\tkzLabelPoints[above left](M,Q)
				\tkzLabelPoints[above right](A,S,P)
				\tkzLabelPoints[above]()
				\tkzLabelPoints[below](B)
				\tkzLabelPoints[below left]()
				\tkzLabelPoints[below right](C,D,N)
				\tkzDrawPoints[fill=black](A,B,C,M,N,P,Q)
			\end{tikzpicture}
		}
		\begin{itemize}
			\item $\heva{&(\alpha)\parallel (SBC)\\ & (SBC)\cap (SAB)=SB\\ &(\alpha)\cap (SAB)=M} \Leftrightarrow (\alpha)\cap (SAB)=MQ$ với $\heva{&MQ \cap SA= Q\\ & MQ\parallel BC}$.
			\item 
		\end{itemize}
		Suy ra, $(P)\cap (SAD)=PQ$.
	}
\end{vd}
\begin{vd}%[1H2K4-2]
	Cho hình hộp chữ nhật $ABCD.A’B’C’D’$. Gọi $M$, $N$, $P$ lần lượt là trung điểm các cạnh $AB$, $AD$, $A’D’$. Xác định giao tuyến của $(MNP)$ và các mặt $(A’B’C’D’)$, $(AA’B’B)$.
	\loigiai{
		\immini{
			Ta có $M$, $N$, $P$ lần lượt là trung điểm các cạnh $AB$, $AD$, $A’D’$. 
			\\ $\Rightarrow \heva{&MN\parallel BD\parallel B’D’\\ &NP\parallel AA’\parallel DD’}\Rightarrow (MNP)\parallel (BDD’B’)$\\
			Khi đó: $\heva{&(MNP)\parallel (BDD’B’)\\ & (BDD’B’)\cap (A’B’C’D’)=B’D’\\ &(MNP)\cap (A’B’C’D)=P}$\\
			$\Leftrightarrow (MNP)\cap (A’B’C’D’)=PQ$ với $\heva{&PQ \cap A’B’= Q\\ & PQ\parallel B’D’}$.\\
			Suy ra $(MNP)\cap (ABB’A’)=MQ$. 
		}{
			\begin{tikzpicture}[scale=0.4,line cap=round,line join =round]
				\tkzDefPoints{0/0/A', -3/-3/B', 10/0/D'}
				\coordinate (A) at($(A')+(0,6)$);
				\coordinate (B) at($(B')+(A)-(A')$);
				\coordinate (C') at($(D')+(B')-(A')$);
				\coordinate (C) at($(C')+(B)-(B')$);
				\coordinate (D) at($(C)+(A)-(B)$);
				\coordinate (M) at($(A)!0.5!(B)$);
				\coordinate (N) at($(A)!0.5!(D)$);
				\coordinate (P) at($(A')!0.5!(D')$);
				\coordinate (x) at($(M)+(P)-(N)$);
				\tkzInterLL(P,x)(A',B')\tkzGetPoint{Q}
				\tkzDrawSegments(A,B B,C B',B C',C B',C' A,D D,C D,D' C',D' M,N)
				\tkzDrawSegments[dashed](A',A A',B' B',C' D',A' N,P P,M P,Q Q,M)
				\tkzLabelPoints[above left](A,B,B',A',M,Q)
				\tkzLabelPoints[above right](D,D',N)
				\tkzLabelPoints[above]()
				\tkzLabelPoints[below]()
				\tkzLabelPoints[below left]()
				\tkzLabelPoints[below right](C,C',P)
				\tkzDrawPoints[fill=black,size=5pt](A,B,C,A',B',C',D,D',M,N,P,Q)
			\end{tikzpicture}
		}
	}
\end{vd}


\subsubsection{Bài tập tự luận}
%\subsubsection{Bài tập tự luận cơ bản}






\begin{bt}%[1H2B4-5]
	Cho hình chóp $S.ABCD$, đáy $ABCD$ là hình thang có $AB \parallel CD$. Gọi $M$ là trung điểm của $SD$, $(P)$ là mặt phẳng qua $M$ và song song với $(SAB)$. Xác định giao điểm của $(P)$ và $SC$.
	\loigiai{
		\immini{Hai mặt phẳng $(P)$ và $(SCD)$ có điểm $M$ chung.\\
			Giả sử $(P)$ cắt $SC$ tại điểm $N$. Khi đó $MN=(P)\cap (SCD)$. Do $(P)\parallel (SAB)$ nên $AB\parallel (P)$. Lại có $CD\parallel AB$, $CD\not\subset (P)$ nên $CD\parallel (P)$.\\
			Từ đó suy ra $MN\parallel CD$, mà $M$ là trung điểm của $SD$ nên $N$ là trung điểm của $SC$.}
		{\begin{tikzpicture}[scale=1, font=\footnotesize,line join=round, line cap=round, >=stealth]
				\tikzset{label style/.style={font=\footnotesize}}
				\tkzDefPoints{0/0/A,5/0/B,0.8/-1.4/C}
				\coordinate (D) at ($(C)+0.6*(B)-0.6*(A)$);
				\coordinate (S) at ($(A)+(1.2,2.5)$);
				\coordinate (M) at ($(S)!1/2!(D)$);
				\coordinate (N) at ($(S)!1/2!(C)$);
				\coordinate (P) at ($(A)!1/2!(C)$);
				\coordinate (Q) at ($(B)!1/2!(D)$);
				\tkzDrawPolygon(S,A,C,D,B)
				\tkzDrawSegments(S,C S,D M,N M,Q N,P)
				\tkzDrawSegments[dashed](A,B P,Q)
				\tkzDrawPoints[fill=black](D,C,A,B,S,M,N,P,Q)
				\tkzLabelPoints[above](S)
				\tkzLabelPoints[left](A,N)
				\tkzLabelPoints[below](C,D)
				\tkzLabelPoints[right](B)
				\tkzLabelPoints[above right](M)
	\end{tikzpicture}}}
\end{bt}

\begin{bt}%[1H2B4-6]
	Cho hình chóp $S.ABCD$ đáy $ABCD$ là hình bình hành tâm $H$. Mặt phẳng $(P)$ đi qua $H$ và song song với $(SAB)$. Tìm giao tuyến của
	\begin{enumerate} 
		\item Mặt phẳng $(P)$ và mặt phẳng $(ABCD)$.
		\item Mặt phẳng $(P)$ và mặt phẳng $(SBC)$.
		\item Thiết diện cắt hình chóp $S.ABCD$ bởi mặt phẳng $(P)$.
	\end{enumerate}
	\loigiai{
		\immini{
			\begin{enumerate} 
				\item Ta có $H$ là một điểm chung của mặt phẳng $(P)$ và mặt phẳng $(ABCD)$, $(P)\parallel (SAB)$, $(ABCD)\cap (SAB)=AB$ nên giao tuyến của $(P)$ và $(ABCD)$ là đường thẳng qua $H$, song song với $AB$, cắt $BC$ và $AD$ lần lượt tại $E$, $F$.\\
				Do $H$ là tâm hình bình hành $ABCD$ nên $E$, $F$ là trung điểm của $BC$ và $AD$.
				\item Theo cách dựng trên ta có $(P)$ và $(SBC)$ có điểm chung $E$. Mà $(P)\parallel (SAB)$, $(SBC)\cap (SAB)=SB$ nên giao tuyến của $(P)$ và $(SBC)$ là đường thẳng qua $E$, song song với $SB$, cắt $SC$ tại $K$. Khi đó $EK$ là đường trung bình của tam giác $SBC$ nên $K$ là trung điểm của $SC$.
			\end{enumerate}
		}{
			\begin{tikzpicture}[scale=1, font=\footnotesize,line join=round, line cap=round, >=stealth]
				\tikzset{label style/.style={font=\footnotesize}}
				\tkzDefPoints{0/0/A, -1.5/-1.6/B, 5/0/D}
				\coordinate (S) at($(A)+(-0.5,3)$);
				\coordinate (C) at($(B)+(D)-(A)$);
				\tkzInterLL (A,C)(B,D)\tkzGetPoint{H}
				\coordinate (E) at ($(B)!1/2!(C)$);
				\coordinate (F) at ($(A)!1/2!(D)$);
				\coordinate (K) at ($(S)!1/2!(C)$);
				\coordinate (I) at ($(S)!1/2!(D)$);
				\tkzDrawSegments(S,B B,C C,D D,S S,C E,K I,K)
				\tkzDrawSegments[dashed](S,A A,B A,D E,F A,C B,D I,F)
				\tkzDrawPoints[fill=black](S,A,B,C,D,E,F,H,I,K)
				\tkzLabelPoints[shift={(-70:0.3)}](H)
				\tkzLabelPoints[above right](F)
				\tkzLabelPoints[above](S,I)
				\tkzLabelPoints[below](B,E,C)
				\tkzLabelPoints[left](A,K)
				\tkzLabelPoints[right](D)
			\end{tikzpicture}
		}
		\begin{enumerate}
			\setcounter{enumi}{1}
			\item Tương tự ta có giao tuyến của $(P)$ và $(SAD)$ là đường thẳng qua $F$, song song với $SA$, cắt $SD$ tại $I$. Khi đó $I$ là trung điểm của $SD$.\\
			Từ các kết quả trên ta có $(P)\cap (ABCD)=EF$, $(P)\cap (SAD)=FI$, $(P)\cap (SCD)=IK$, $(P)\cap (SBC)=KE$ nên thiết diện cắt hình chóp $S.ABCD$ bởi mặt phẳng $(P)$ là tứ giác $EFIK$.\\
			Lại do $EF\parallel CD$, $IK\parallel CD$ (do $IK$ là đường trung bình của tam giác $SCD$) nên $IK\parallel EF$. Vậy $EFIK$ là hình thang.
		\end{enumerate}
	}
\end{bt}




\begin{dang}{Xác định giao điểm của một đường thẳng với mặt phẳng (dùng tính chất song song)}
	Để tìm giao điểm của đường thẳng $a$ và mặt phẳng $(P)$. Ta cần tìm một mặt phẳng $(Q)$ chứa đường thẳng $a$ và cắt mặt phẳng $(P)$ theo giao tuyến là đường thẳng $\Delta$. Khi đó giao điểm của đường thẳng $a$ và $\Delta$ chính là giao điểm của đường thẳng $a$ và mặt phẳng $(P)$.	
\end{dang}
\subsubsection{Ví dụ minh hoạ}
\begin{vd}%[1H2Y3-5]
	Cho tứ diện $ABCD$ có $G$ là trọng tâm tam giác $BCD$. Biết mặt phẳng $(\alpha)$ chứa $BG$ và song song với $AC$. Tìm giao điểm $K$ của $AD$ và mặt phẳng $(\alpha)$.
	\loigiai{
		\immini
		{
			Gọi $I$ là trung điểm của $CD$.\\
			Ta có
			$\heva{&I \in (\alpha) \cap (ACD)\\ &AC \parallel (\alpha) \\ & AC \subset (ACD)} \Rightarrow (\alpha) \cap (ACD)=Ix \parallel AC$.\\
			Trong mặt phẳng $(ACD)$, gọi $Ix \cap AD = K$.\\
			Ta có
			$\heva{&K\in AD\\
				&K \in Ix \subset (\alpha) \Rightarrow K \in (\alpha)} \Rightarrow K = AD \cap (\alpha)$.
		}
		{
			\begin{tikzpicture}[scale=0.5, line join = round, line cap = round]
				\tikzset{label style/.style={font=\footnotesize}}
				\tkzDefPoints{0/0/B,7/0/C,2/-3/D,3/5/A}
				\tkzCentroid(D,B,C)  \tkzGetPoint{G}
				\coordinate (I) at ($(C)!0.5!(D)$);
				\coordinate (K) at ($(A)!0.5!(D)$);
				\tkzDrawPolygon(A,B,D,C)
				\tkzDrawSegments(A,D B,K I,K)
				\tkzDrawSegments[dashed](B,C B,I)
				\tkzDrawPoints[fill=black](A,B,C,D,G,I,K)
				\tkzLabelPoints[above](A)
				\tkzLabelPoints[above right](K)
				\tkzLabelPoints[below](D,G,I)
				\tkzLabelPoints[left](B)
				\tkzLabelPoints[right](C)
			\end{tikzpicture}
			
		}
	}
\end{vd}
\begin{vd}%[1H2B3-5]
	Cho hình chóp $S.ABCD$ có đáy $ABCD$ là hình bình hành. Mặt phẳng $(\alpha)$ qua $BD$ và song song với $SA$. Tìm giao điểm $K$ của mặt phẳng $(\alpha)$ và $SC$.
	\loigiai{
		\immini
		{
			Gọi $O$ là tâm hình bình hành $ABCD$. Khi đó, $O\in (\alpha)\cap (SAC)$. \\	
			Ta có $\heva{&SA \parallel (\alpha)\\
				&SA \subset (ASC)\\
				&O \in (\alpha)\cap (SAC)	
			} \Rightarrow  (\alpha)\cap (SAC)=Ox \parallel SA$.\\
			Trong mặt phẳng $(SAC)$, gọi $K=SC \cap Ox$.\\ Khi đó 
			$\heva{&K \in SC\\
				&K \in Ox \subset (\alpha) \Rightarrow K \in (\alpha)} \Rightarrow K = SC \cap (\alpha)$.
		}
		{
			\begin{tikzpicture}[scale=0.4, line join = round, line cap = round]
				\tikzset{label style/.style={font=\footnotesize}}
				\tkzDefPoints{0/0/D,7/0/C,3/3/A}
				\coordinate (B) at ($(A)+(C)-(D)$);
				\coordinate (S) at ($(A)+(0.5,6)$);
				\coordinate (O) at ($(A)!0.5!(C)$);
				\coordinate (K) at ($(S)!0.5!(C)$);
				\tkzDrawPolygon(S,B,C,D)
				\tkzDrawSegments(S,C K,B K,D)
				\tkzDrawSegments[dashed](A,S A,B A,D A,C B,D O,K)
				\tkzDrawPoints[fill=black](D,C,A,B,S,K,O)
				\tkzLabelPoints[above](S)
				\tkzLabelPoints[left](A,D)
				\tkzLabelPoints[right](B,C)
				\tkzLabelPoints[above right](K)
				\tkzLabelPoints[below](O)
			\end{tikzpicture}
		}
	}
\end{vd}	
\begin{vd}%[1H2K3-5]
	Cho hình chóp $S.ABCD$ có đáy
	$ABCD$ là hình thang đáy lớn $AB$. Gọi $M$ là một điểm trên $CD$, $(\alpha)$ là mặt phẳng qua $M$ và song song với $SA$ và $BC$.
	Tìm giao điểm $Q$ của $SC$ và $(\alpha)$.
	\loigiai{
		\immini
		{				
			Ta có $\heva{& (\alpha) \parallel BC, BC \subset (ABCD) \\ & M \in (\alpha) \cap (ABCD)}$ \\
			$ \Rightarrow (\alpha) \cap (ABCD) = MN \parallel BC$ và $N \in AB$).\\
			Gọi $I$ là giao điểm của $AC$ và $MN$ ta có \\
			$\heva{&SA \parallel (\alpha)\\& SA \subset (SAC)\\& I\in (\alpha) \cap (SAC)} \Rightarrow  (\alpha) \cap (SAC) = Ix \parallel SA$.\\
			Trong mặt phẳng $(SAC)$ gọi $Q$ là giao điểm của $Ix$ và $SC$.\\ Ta có 
			$\heva{&Q\in SC\\&Q \in Ix \subset (\alpha) \Rightarrow I \in (\alpha)} \Rightarrow Q = SC \cap (\alpha)$.
		}
		{
			\begin{tikzpicture}[scale=1, line join=round, line cap=round]
				\tkzDefPoints{0/0/A, 1.5/-1.3/D, 4.3/-1.3/C, 5/0/B, 3.8/3.3/S}
				\coordinate (M) at ($(D)!0.4!(C)$);
				\tkzDefPointBy[translation=from C to B](M)\tkzGetPoint{N}
				\coordinate (P) at ($(B)!0.336!(S)$);
				\coordinate (Q) at ($(C)!0.336!(S)$);
				\tkzInterLL(A,C)(M,N) \tkzGetPoint{I}
				\tkzDefLine[parallel=through I](A,S) \tkzGetPoint{d1}
				\tkzInterLL(I,d1)(S,C) \tkzGetPoint{Q}		
				\tkzDrawPoints[fill=black](S,A,B,C,D,M,N,I,Q)
				\tkzDrawSegments[dashed](A,B M,N I,Q A,C)
				\tkzDrawSegments(A,D B,C C,D S,A S,B S,C S,D)
				\tkzLabelPoints[below](C,D,M)
				\tkzLabelPoints[above](S)
				\tkzLabelPoints[left](A)
				\tkzLabelPoints[right](B,Q)
				\tkzLabelPoints[above left](N,I)
			\end{tikzpicture}
		}
	}
\end{vd}

\subsubsection{Bài tập tự luận}
%\subsubsection{Bài tập tự luận cơ bản}

\begin{bt}%[1H2B3-5]
	Cho hình chóp $S.ABCD$, đáy $ABCD$ là hình bình hành. Gọi $M,\,N,\,P$ lần lượt là trung điểm của các cạnh $AB,\,AD,\,SB$.
	\begin{enumerate}
		\item Chứng minh $BD \parallel (MNP)$.
		\item Tìm giao điểm của $(MNP)$ với $BC$.
		\item Tìm giao tuyến của hai mặt phẳng $(MNP)$ và $(SBD)$.
	\end{enumerate}
	\loigiai{
		\begin{enumerate}
			\immini{	\item  $\triangle ABD $ có $MN$ là đường trung bình nên $MN\parallel BD$ và $MN=\dfrac{1}{2}BD$.\\
				Ta có	$\heva{&BD\parallel MN\\&MN\subset (MNP)\\&BD\not\subset (MNP)}\Rightarrow BD\parallel (MNP)$.
				\item 	Trong $(ABCD)$, dựng $H=MN\cap BC$, ta có\\
				$$\heva{&H\in BC\\&H\in MN,MN\subset (MNP)}\Rightarrow H=(MNP)\cap BC.$$}{\begin{tikzpicture}[scale=0.6, font=\footnotesize, line join=round, line cap=round, >=stealth]
					\tkzDefPoints{0/0/A, 5/0/D, -2/-2/B, 0.8/4.5/S}
					\tkzDefPointBy[translation=from A to D](B)\tkzGetPoint{C}
					\tkzInterLL(A,C)(B,D) \tkzGetPoint{O}
					\tkzDefMidPoint(A,B)\tkzGetPoint{M};
					\tkzDefMidPoint(A,D)\tkzGetPoint{N};
					\tkzDefMidPoint(S,B)\tkzGetPoint{P};
					\tkzInterLL(A,C)(M,N) \tkzGetPoint{I}
					\tkzDefLine[parallel=through P](D,B) \tkzGetPoint{p}
					\tkzInterLL(P,p)(S,D) \tkzGetPoint{Q} 
					\tkzDefLine[parallel=through I](S,A) \tkzGetPoint{i}
					\tkzInterLL(I,i)(S,C) \tkzGetPoint{K} 
					\tkzInterLL(M,N)(B,C) \tkzGetPoint{H} 
					\tkzInterLL(M,N)(B,S) \tkzGetPoint{L} 
					\tkzDrawPoints[fill=black](S,A,B,C,D,M,N,P,Q,K,H)	
					\tkzDrawSegments(B,C C,D S,B S,C S,D P,K Q,K H,B H,P)
					\tkzDrawSegments[dashed](A,B S,A A,C B,D A,D)	
					\tkzDrawSegments[dashed](M,N M,P N,P Q,P N,Q H,M)
					\tkzLabelPoints[left](A,P,H)
					\tkzLabelPoints[below](B)
					\tkzLabelPoints[right](C,D,Q)
					\tkzLabelPoints[above](S)	
					\tkzLabelPoints[above right](N)
					\tkzLabelPoints[below right,xshift=-0.2cm](M)
					\tkzLabelPoints[xshift=-0.4,yshift=10](K)
			\end{tikzpicture}}
			\item Gọi $\Delta =(MNP)\cap (SBD)$, ta có
			$\heva{&P\in (SBD)\\&P\in (MNP)}\Rightarrow P\in \Delta$.\\
			Ta có
			$$\heva{&MN\parallel BD\\& MN\subset (MNP),\,(BD)\subset (SBD)\\&(MNP)\cap (SBD)=\Delta}\Rightarrow \Delta \parallel MN.$$
			Vậy $\Delta$ là đường thẳng qua $P$ và song song với $MN$.\\
			Gọi $Q=\Delta \cap SD$, ta được $(MNP)\cap (SBD)=PQ$.	
		\end{enumerate}
	}
\end{bt}

\begin{bt}%[1H2K3-5]
	Cho hình chóp $S.ABCD$ có đáy là một hình thang với đáy lớn $AD$. Gọi $M$, $N$, $P$ lần lượt là trung điểm của $AB$, $CD$, $SA$.
	\begin{enumerate}
		\item Xác định giao điểm $Q$ của cạnh $SD$ với $(MNP)$.
		\item Chứng minh $SB$, $SC$ cùng song song với $(MNP)$.
		\item Chứng minh $MP$ và $NQ$ cắt nhau. Chứng minh giao điểm của $MP$ và $NQ$, giao điểm của $AB$ và $CD$ và điểm $S$ thẳng hàng.
	\end{enumerate}
	\loigiai{
		\immini{
			\begin{enumerate}
				\item Ta có $AD \parallel MN$ nên $AD \parallel (MNP)$. Mà $AD \subset (SAD)$ nên giao tuyến của $(MNP)$ và $(SAD)$ là đường thẳng qua $P$, song song với $AD$, cắt $SD$ tại $Q$. Suy ra $Q$ là giao điểm của $SD$ và $(MNP)$.
				\item Tam giác $SAB$ có $MP$ là đường trung bình nên $SB \parallel MP$, do đó $SB \parallel (MNP)$.\\
				Theo cách dựng điểm $Q$, suy ra $Q$ là trung điểm của $SD$. Do đó $NQ$ là đường trung bình của tam giác $SCD$, suy ra $SC \parallel NQ$. Suy ra $SC \parallel (MNP)$.
				\item Ta có $MN \parallel PQ$ (vì cùng song song với $AD$). Mặt khác ta có $PQ =\dfrac{1}{2}AD$, $MN=\dfrac{AD+BC}{2}>\dfrac{AD}{2}=PQ$ nên $MNQP$ là hình thang. Vậy $MP$ và $NQ$ có thể cắt nhau tại $I$.\\
				Theo cách dựng thì $I$ là một điểm chung của $(SAB)$ và $(SCD)$. Tương tự, giao điểm $J$ của $AB$ và $CD$ cũng là điểm chung của $(SAB)$ và $(SCD)$. Do đó $I$, $S$ và $J$ cùng thuộc giao tuyến của $(SAB)$ và $(SCD)$.\\
				Vậy $I$, $S$, $J$ thẳng hàng.
		\end{enumerate}}{\begin{tikzpicture}[scale=0.4, font=\footnotesize, line join=round, line cap=round, >=stealth]
				\tkzDefPoints{0/0/A,7/0/D,2/-3/B,6/-3/C}
				\coordinate (S) at ($(A)+(2,5)$);
				\coordinate (M) at ($(A)!0.5!(B)$);
				\coordinate (N) at ($(C)!0.5!(D)$);
				\coordinate (P) at ($(A)!0.5!(S)$);
				\coordinate (Q) at ($(S)!0.5!(D)$);
				\tkzInterLL(M,P)(N,Q) \tkzGetPoint{I}
				\tkzInterLL(A,B)(C,D) \tkzGetPoint{J}
				\tkzDrawPolygon(S,A,B,C,D)
				\tkzDrawSegments(S,B S,C M,P N,Q M,I N,I S,I A,J D,J)
				\tkzDrawSegments[dashed](A,S A,B A,D M,N P,Q)
				\tkzDrawPoints[fill=black](D,C,A,B,S,M,N,P,Q,I,J)
				\tkzLabelPoints[above](I)
				\tkzLabelPoints[left](S,P,A,M)
				\tkzLabelPoints[right](N,Q,D)
				\tkzLabelPoints[below](B,C,J)
			\end{tikzpicture}
		}	
	}
\end{bt}


\subsubsection{Bài tập trắc nghiệm}
\Opensolutionfile{ans}[ans/ans-1K1-3-Dang3]

\begin{ex}%[1H2Y3-1]
	Trong các khẳng định sau, khẳng định nào là đúng?
	\choice
	{Nếu đường thẳng $d$ song song mặt phẳng $(P)$ thì trong $(P)$ có duy nhất một đường thẳng $a$ song song với $d$}
	{Nếu đường thẳng $d$ song song với mặt phẳng $(P)$ thì $d$ song song với mọi đường thẳng nằm trong $(P)$}
	{\True Nếu đường thẳng $d$ song song với mặt phẳng $(P)$ thì trong $(P)$ tồn tại đường thẳng $a$ song song với $d$}
	{Nếu đường thẳng $d$ song song mặt phẳng $(P)$, đường thẳng $a$ bất kỳ nằm trong $(P)$ thì $a$ và $d$ chéo nhau}
	\loigiai{
		Khẳng định đúng là \lq\lq  Nếu đường thẳng $d$ song song với mặt phẳng $(P)$ thì trong $(P)$ tồn tại đường thẳng $a$ song song với $d$\rq\rq.
	}
\end{ex}
\begin{ex}%[1H2B3-1]
	Trong không gian, đường thẳng $a$ song song với mặt phẳng $(P)$ nếu
	\choice
	{$a\not\subset (P)$}
	{$\heva{&a\parallel b\\ &b\subset (P)}$}
	{$\heva{&a\parallel b\\&b\not\subset (P)}$}
	{\True $\heva{&a\parallel b\\&b\subset(P)\\&a\not\subset (P)}$}
	\loigiai{Đường thẳng $a$ song song với mặt phẳng $(P)$ khi và chỉ khi $a$ không nằm trong $(P)$, đồng thời $a$ song song với một đường thẳng $b$ nằm trong $(P)$.}
\end{ex}



\begin{ex}%[1H2B3-3]
	Cho hình chóp $S.ABCD$, đáy $ABCD$ là hình bình hành. Giao tuyến của hai mặt phẳng $(SAD)$ và $(SBC)$ là đường thẳng song song với đường thẳng nào sau đây?
	\choice
	{$AC$}
	{$DC$}
	{$BD$}
	{\True $AD$}
	\loigiai{
		\immini{
			Giao tuyến của $2$ mặt phẳng chứa $2$ đường thẳng song song với nhau là đường thẳng đi qua $1$ điểm chung của $2$ mặt phẳng đó và song song với $2$ đường thẳng song song trên. Mà $AD\parallel BC$ nên giao tuyến của hai mặt phẳng $(SAD)$ và $(SBC)$ là đường thẳng qua $S$ và song song với $AD$.
		}{
			\begin{tikzpicture}[scale=1, font=\footnotesize, line join=round, line cap=round,>=stealth]
				\tkzInit[xmin=-0.5, xmax=5.5, ymin=-0.5, ymax=4]
				\tkzClip
				\tkzDefPoints{0/0/A,3.5/0/B,5/1.5/C,2/3.5/S}
				\tkzDefPointBy[translation=from B to A](C)\tkzGetPoint{D}
				\tkzDefPointBy[translation=from D to A](S)\tkzGetPoint{d}
				\tkzDefPointBy[homothety=center d ratio 1.3](S)\tkzGetPoint{d'}
				\tkzDrawPoints[fill=black](A,B,C,D,S)
				\tkzDrawSegments(A,B B,C S,A S,B S,C d,d')
				\tkzDrawSegments[dashed](C,D D,A S,D)
				\tkzLabelPoints[above](S,d)
				\tkzLabelPoints[below](A,B,D)
				\tkzLabelPoints[right](C)
			\end{tikzpicture}
		}
	}
\end{ex}

\begin{ex}%[1H2B3-3] 
	Cho tứ diện $ABCD$. Gọi $I,J$ lần lượt là trung điểm của $BC$ và $BD$. Giao tuyến của hai mặt phẳng $(AIJ)$ và $(ACD)$ là đường nào sau đây?
	\choice
	{Đường thẳng $d$ đi qua $A$ và $d \parallel BC$}
	{Đường thẳng $d$ đi qua $A$ và $d \parallel BD$}
	{\True Đường thẳng $d$ đi qua $A$ và $d \parallel CD$}
	{Đường thẳng $d$ đi qua $A$ và giao điểm của $IJ$ và $CD$}
	\loigiai{		\immini
		{
			$I$, $J$ lần lượt là trung điểm của $BC$ và $BD$ nên $IJ \parallel CD$. Do đó giao tuyến của $(AIJ)$ với $(ACD)$ là đường thẳng đi qua $A$ và song song với $CD$.
		}{
			\begin{tikzpicture}[scale=1, font=\footnotesize, line join=round, line cap=round, >=stealth]
				\coordinate (C) at (0,0);
				\coordinate (B) at (2.2,-1.5);
				\coordinate (A) at (4,0);
				\coordinate (O) at ($1/3*(B)+1/3*(C)+1/3*(A)$);
				\coordinate (D) at ($(O)+(0,3)$);
				\coordinate (J) at ($(D)!0.5!(B)$);
				\coordinate (I) at ($(C)!0.5!(B)$);
				\draw (D)--(C)--(B)--(D)--(A)--(B) (I)--(J)--(A);
				\draw [dashed] (C)--(A)--(I) ;
				\tkzLabelPoints[below](B,C,A)
				\tkzLabelPoints[above](D)
				\tkzLabelPoints[left](J)
				\tkzLabelPoints[below left=-3pt](I)
				\tkzDrawPoints[fill=black,size=3pt](A,B,C,D,I,J)
			\end{tikzpicture}
	}}
\end{ex}

\begin{ex}%[1H2B3-3]
	Cho tứ diện $ABCD$ và ba điểm $P$, $Q$, $R$ lần lượt nằm trên cạnh các $AB$, $CD$, $BC$ (không trùng với các đỉnh của tứ diện $ABCD$) sao cho $PR\parallel AC$. Khi đó giao tuyến của hai mặt phẳng $(PQR)$ và $(ACD)$ song song với đường thẳng nào trong các đường thẳng sau?
	\choice
	{$BD$}
	{$CD$}
	{$CB$}
	{\True $AC$}
	\loigiai{	\immini{ 
			Ta có $\heva{&Q\in(PQR)\cap (ACD)\\&PR\subset (PQR)\\&AC\subset (ACD)\\&PR\parallel AC}$	 nên $(PQR)\cap (ACD)=Qx\parallel AC$.\\
			Vậy giao tuyến của hai mặt phẳng $(PQR)$ và $(ACD)$ song song với đường thẳng $AC$.
		}
		{\begin{tikzpicture}[scale=0.6, font=\footnotesize, line join=round, line cap=round, >=stealth]
				\tkzDefPoints{0/0/B, 2/-2/C, 6/0/D}
				\coordinate (Q) at ($(C)!.7!(D)$);
				\coordinate (E) at ($(B)!.3!(Q)$); 
				\coordinate (A) at ($(E)+(0,5)$);
				\coordinate (P) at ($(B)!.4!(A)$);
				\coordinate (R) at ($(B)!.4!(C)$);
				\coordinate (S) at ($(A)!.7!(D)$);
				\tkzDrawSegments[dashed](B,D R,Q P,S)
				\tkzDrawPolygon(A,B,C)
				\tkzDrawPoints(A,B,C,D,P,Q,R,S)
				\tkzDrawSegments(D,C D,A P,R S,Q)
				\tkzLabelPoints[left](B,P)
				\tkzLabelPoints[right](D,S)
				\tkzLabelPoints[below right](Q)
				\tkzLabelPoints[below left](C,R)
				\tkzLabelPoints[above](A)
			\end{tikzpicture}
		}
		
	}
\end{ex}



\begin{ex}%[1H2K3-5] 
	Cho hình chóp $S.ABCD$, đáy là hình thang ($AB\parallel CD$). Gọi $M$ là trung điểm của $SB$. Mặt phẳng qua $DM$, song song với $AB$ cắt đường thẳng $SC$ tại $Q$. Tính tỉ số $\dfrac{SC}{SQ}$.
	\choice
	{$\dfrac12$}
	{$2$}
	{\True $1$}
	{$\dfrac32$}
	\loigiai{
		\immini{
			Gọi $(\alpha)$ là mặt phẳng qua $DM$ và song song với $AB$. \\
			Vì $AB\parallel(\alpha)$ và $AB\subset(ABCD)$ nên giao tuyến của $(\alpha)$ và $(ABCD)$ là đường thẳng qua $D$ và song song với $AB$. Ta có $DC\parallel AB$ nên $DC$ chính là giao tuyến của $(\alpha)$ và $(ABCD)$. Do đó $(\alpha)$ cắt $SC$ tại $C$, tức là $Q$ trùng với $C$. Vì vậy $\dfrac{SC}{SQ}=1$. }{
			\begin{tikzpicture}[scale=1, line join=round, line cap=round,>=stealth]
				\tikzset{label style/.style={font=\footnotesize}}
				\pgfmathsetmacro\h{1.5}
				\pgfmathsetmacro\goc{75}
				\tkzDefPoint(0,0){A}
				\tkzDefShiftPoint[A](\goc:1.5*\h){S}
				\tkzDefShiftPoint[A](0:2*\h){B}
				\tkzDefShiftPoint[A](-0.5*\goc:0.7*\h){D}
				\tkzDefShiftPoint[D](0:\h){C}
				\coordinate (P) at ($(S)!1/2!(A)$);
				\coordinate (N) at ($(A)!1/2!(S)$);
				\coordinate (M) at ($(S)!1/2!(B)$);
				\pgfresetboundingbox
				\tkzDrawPoints[fill=black](A,B,C,D,S,M,N)
				\tkzDrawSegments[dashed](D,M A,B M,N)
				\tkzDrawSegments(B,C C,D S,B S,D S,C A,D A,S M,C D,N)
				\tkzLabelPoints[below](C,D)
				\tkzLabelPoints[below left](N)
				\tkzLabelPoints[above](S)
				\tkzLabelPoints[left](A)
				\tkzLabelPoints[right](B,M)
			\end{tikzpicture}
		}
	}
\end{ex}

\begin{ex}%[1H2K3-2]
	Cho tứ diện $ABCD$. Gọi $G$, $G'$ lần lượt là trọng tâm $\Delta ABD$ và $\triangle BCD$. Khẳng định nào sau đây là sai?	
	\choice
	{$GG'\parallel(ACD)$}
	{\True $GG'\parallel BD$}
	{$GG'\parallel(ABC)$}
	{$GG'\parallel AC$}
	\loigiai{
		\immini{
			Ta có $GG'$ cắt mặt phẳng $(ABD)$ tại $G$. Do đó $GG'$ không thể song song được với $BD$ nằm trong mặt phẳng $(ABD)$.
		}{
			\begin{tikzpicture}[scale=.8, line join=round, line cap=round,>=stealth]
				\tkzDefPoints{0/0/B,3/-2/C,5/0/A}
				\coordinate (I) at ($(C) !0.5! (A) $);
				\coordinate (D) at ($(I)+(0,5)$);
				\tkzCentroid(A,B,D) \tkzGetPoint{G}
				\coordinate (G') at ($1/3*(B)+1/3*(C)+1/3*(D)$);
				\tkzDefMidPoint(A,D) \tkzGetPoint{N}
				\tkzDefMidPoint(C,D) \tkzGetPoint{M}
				\tkzDrawSegments(D,A D,B D,C A,C B,C M,N M,B)
				\tkzDrawSegments[dashed](A,B B,N G,G')
				\tkzDrawPoints[fill=black](A,B,C,D,G,G')
				\tkzLabelPoints(C,A,N)
				\tkzLabelPoints[above](D,G)
				\tkzLabelPoints[left](B)
				\tkzLabelPoints[below](G')
				\tkzLabelPoints[below right](M, N)
			\end{tikzpicture}	
	}}
\end{ex}

\begin{ex}%[1H2K4-4]
	Cho hình chóp $S.ABC$. Gọi $M$ là trung điểm của $SB$, mặt phẳng $(\alpha)$ đi qua $M$ và song song với mặt phẳng $(ABC)$ cắt $SA, SC$ lần lượt tại $N, P$. Khẳng định nào đúng? 
	\choice{$(\alpha)\not\equiv(MNP)$}
	{$MP$ cắt $BC$}
	{$MN$ cắt $AC$}
	{\True $MP \parallel BC$}
	\loigiai{
		\immini{
			Theo cách xác định mặt phẳng $(\alpha)$, ta có $MP \parallel BC$.
		}{
			\begin{tikzpicture}[scale=0.5, font=\footnotesize, line join=round, line cap=round, >=stealth]
				\tkzDefPoints{0/0/A, 7/0/C, 4/5/S, 3/-3/B}
				\tkzLabelPoints[left](A)
				\tkzLabelPoints[below](B)
				\tkzLabelPoints[right](C)
				\tkzLabelPoints[above](S)
				\tkzDefMidPoint(B,S) \tkzGetPoint{M}\tkzLabelPoints[below right](M)
				\tkzDefMidPoint(C,S) \tkzGetPoint{P}\tkzLabelPoints[above right](P)
				\tkzDefMidPoint(A,S) \tkzGetPoint{N}\tkzLabelPoints[above left](N)
				\tkzDrawSegments(B,C A,B A,S S,C S,B M,N M,P)
				\tkzDrawSegments[dashed](A,C N,P)
				\tkzDrawPoints[fill=black](A,S,B,C,M,N,P)
			\end{tikzpicture}
		}
	}
\end{ex}

\begin{ex}%[1H2G4-3]
	Cho lăng trụ tam giác $ABC.A'B'C'$. Gọi $M, N$ lần lượt là trung điểm của $BB'$, $CC'$. Đường thẳng qua đi qua trọng tâm $I$ của tam giác $ABC$ cắt $A'B$ và $MN$ lần lượt tại $P$, $Q$. Khi đó tỉ số $\dfrac{IP}{IQ}$ bằng
	\choice
	{$\dfrac{3}{5}$}
	{$\dfrac{5}{2}$}
	{\True $\dfrac{2}{5}$}
	{$\dfrac{5}{3}$}
	\loigiai{
		\immini{
			Qua $I$ kẻ đường thẳng song song với $BC$ cắt $AC, AB$ tại $H$, $K$. Suy ra $KH \parallel MN$, do đó $M$, $N$, $K$, $H$, $I$ đồng phẳng.\\
			Gọi $P=A'B \cap MH$, $Q=IP \cap MN$.\\
			Kẻ $MD \parallel A'B'$ $(D \in A'B)$ ta được $\dfrac{HP}{PM}=\dfrac{HB}{DM}=\dfrac{\dfrac{1}{3}\cdot AB}{\dfrac{1}{2}\cdot A'B'}=\dfrac{2}{3}$.\\
			Do đó $\dfrac{IP}{PQ}=\dfrac{HP}{PM}=\dfrac{2}{3}$. Vậy $\dfrac{IP}{PQ}=\dfrac{2}{5}$.
		}{
			\begin{tikzpicture}[scale=1, font=\footnotesize, line join=round, line cap=round, >=stealth]
				\tikzset{hidden/.style = {thick, dashed}}
				\tkzDefPoints{0/0/A',2/-2/B',3/0/C', 0/5/A, 2/3/B, 3/5/C}
				\tkzCentroid(A,B,C)\tkzGetPoint{I}
				\tkzDefBarycentricPoint(A=1,B=2)\tkzGetPoint{H}
				\tkzDefBarycentricPoint(A=1,C=2)\tkzGetPoint{K}
				\tkzDefMidPoint(B,B')\tkzGetPoint{M}
				\tkzDefMidPoint(C,C')\tkzGetPoint{N}
				\tkzDefMidPoint(A',B)\tkzGetPoint{D}
				\tkzDefMidPoint(A,C)\tkzGetPoint{E}
				\tkzInterLL(A',B)(H,M)\tkzGetPoint{P}
				\tkzInterLL(I,P)(N,M)\tkzGetPoint{Q}
				\tkzDrawPoints[fill=black](A,B,C,A',B',C',Q,P,D,H,N,I,K,M)
				\tkzLabelPoints[left](I,H)
				\tkzLabelPoints[above](K)
				\tkzLabelPoints[below](B',Q)
				\tkzLabelPoints[left](A,A',P,D)
				\tkzLabelPoints[right](B,C,C',M,N)
				\tkzDrawPoints(A,B,C,A',B',C',I,H,K,M,N,D,P,Q)
				\tkzDrawSegments(A,B B,C C,A A,A' B,B' C,C' A',B' B',C' C',A' H,K M,N A',B H,M M,Q M,D B,E)
				\tkzDrawSegments[dashed](A',C' K,N I,P Q,P)
			\end{tikzpicture}
		}
	}
\end{ex}

\begin{ex}%[1H2G1-3]
	Cho hình chóp $S.ABCD$ có đáy $ABCD$ là hình thang với $AB\parallel CD$ và $AB=2CD$. Gọi  $G$ là trọng tâm của tam giác $SBC$, $H$ là giao điểm của $DG$ và $(SAC)$. Tỉ số $\dfrac{GH}{GD}$ bằng
	\choice
	{$\dfrac{1}{2}$}
	{$\dfrac{3}{5}$}
	{\True $\dfrac{2}{5}$}
	{$\dfrac{2}{3}$}
	\loigiai
	{\immini{Gọi $M$ là trung điểm của $BC$, $N$ là giao điểm của $AC$ và $MD$.\\
			Khi đó, $(SMD)\cap (SAC)=SN$.\\
			Vì $DG\subset (SMD)$ và $DG\cap (SAC)=H$ nên $H\in SN$.\\
			Gọi $E$ là điểm đối xứng với $D$ qua $C$ thì tứ giác $ABED$ là hình bình hành. Suy ra $EM\parallel AC$. Do đó, $N$ là trung điểm của $MD$.\\
			Dựng đường thẳng đi qua $G$ song song với $MN$ và cắt $SN$ tại $K$.\\
			Khi đó, $\dfrac{SG}{SM}=\dfrac{KG}{NM}$ và $\dfrac{HD}{HG}=\dfrac{DN}{KG}$.\\
			Suy ra $\dfrac{SG}{SM}\cdot\dfrac{NM}{ND}\cdot\dfrac{HD}{HG}=\dfrac{KG}{NM}\cdot\dfrac{NM}{ND}\cdot\dfrac{ND}{KG}=1$.\medskip\\
			Mặt khác $\dfrac{SG}{SM}=\dfrac{1}{3}$ và $\dfrac{NM}{ND}=1$ nên $\dfrac{2}{3}\cdot 1\cdot\dfrac{HD}{HG}=1\Leftrightarrow\dfrac{HD}{HG}=\dfrac{3}{2}$.\medskip\\
			Vậy $\dfrac{GH}{GD}=\dfrac{2}{5}$.}{\begin{tikzpicture}[scale=1, font=\footnotesize, line join=round, line cap=round,>=stealth]
				\tkzDefPoints{0/0/D, 2.5/0/C, 4/2/B, 2/4/h};
				\coordinate (A) at ($(B)+2*(D)-2*(C)$);
				\coordinate (M) at ($(C)!0.5!(B)$);
				\coordinate (E) at ($(D)!2!(C)$);
				\coordinate (S) at ($(A)+(h)$);
				\tkzInterLL(A,C)(D,M) \tkzGetPoint{N};
				\tkzCentroid(S,B,C)\tkzGetPoint{G};
				\tkzInterLL(S,N)(D,G) \tkzGetPoint{H};
				\coordinate (K) at ($(S)!2/3!(N)$);
				\tkzDrawPolygon(S,B,C,D,A);
				\tkzDrawSegments(S,C S,D S,M B,E C,E M,E);
				\tkzDrawSegments[dashed](A,B A,C B,D S,N D,G D,M G,K);
				\tkzDrawPoints[fill=black](S,A,B,C,D,M,G,N,H,E,K);
				\tkzLabelPoints[above](S);
				\tkzLabelPoints[below](D,C,M,N,E);
				\tkzLabelPoints[right](B,G);
				\tkzLabelPoints[left](A,H,K);
		\end{tikzpicture}}
	}
\end{ex}
\Closesolutionfile{ans}