\subsection{Hệ thống bài tập trắc nghiệm}
\Opensolutionfile{ans}[ans/ans1-C4B12-5]
\begin{dang}{Câu hỏi lý thuyết}
\end{dang}
	\begin{ex}%[Chuyên đề Toán 11 - 2023]%[Dương Phước Sang, dự án WTB - Toán 11 - new 2023]%[1K4BB-1]
		Cho đường thẳng $a$ nằm trong mặt phẳng $(\alpha)$. Giả sử $b \not\subset (\alpha)$. Mệnh đề nào sau đây đúng?
		\choice
		{Nếu $b \parallel (\alpha)$ thì $b \parallel a$}
		{Nếu $b$ cắt $(\alpha)$ thì $b$ cắt $a$}
		{\True Nếu $b \parallel a$ thì $b \parallel (\alpha)$}
		{Nếu $b \parallel (\alpha)$ và $(\beta)$ chứa $b$ thì $(\beta)$ sẽ cắt $(\alpha)$ theo giao tuyến là đường thẳng song song với $b$}
		\loigiai{
			Theo lý thuyết $\heva{&b \not \subset (\alpha)\\&b \parallel a\\&a \subset (\alpha)} \Rightarrow b \parallel (\alpha)$.
		}
	\end{ex}
	\begin{ex}%[Chuyên đề Toán 11 - 2023]%[Dương Phước Sang, dự án WTB - Toán 11 - new 2023]%[1K4BB-1]
		Cho các mệnh đề sau
		\begin{enumerate}[1.]
			\item Nếu đường thẳng $a$ song song với mặt phẳng $(P)$ thì $a$ song song với mọi đường thẳng nằm trong $(P)$.
			\item Giữa hai đường thẳng chéo nhau có duy nhất một mặt phẳng chứa đường thẳng này và song song với đường thẳng kia.
			\item Hai đường thẳng phân biệt cùng song song với một mặt phẳng thì chúng song song với nhau.
			\item Nếu đường thẳng $\Delta$ song song với mặt phẳng $(P)$ và $(P)$ cắt đường thẳng $a$ thì $\Delta$ cắt $a$.
			\item Đường thẳng song song với mặt phẳng nếu nó song song với một đường thẳng nằm trong mặt phẳng đó.
		\end{enumerate}
		Trong các mệnh đề trên, số các mệnh đề \textbf{sai} là
		\choice
		{$1$}
		{$2$}
		{$3$}
		{\True $4$}
		\loigiai{
			Các mệnh đề \textbf{sai} là: $1$, $3$, $4$, $5$.
		}
	\end{ex}

	\begin{ex}%[Chuyên đề Toán 11 - 2023]%[Dương Phước Sang, dự án WTB - Toán 11 - new 2023]%[1K4BB-1]
		Mệnh đề nào \textbf{sai} trong các mệnh đề sau?
		\choice
		{\True Qua một điểm nằm ngoài một mặt phẳng cho trước có một và chỉ một đường thẳng song song với mặt phẳng đã cho}
		{Nếu mặt phẳng $(\alpha)$ chứa hai đường thẳng cắt nhau $a$, $b$ và $a$, $b$ cùng song song với mặt phẳng $(\beta)$ thì $(\alpha)$ song song với $(\beta)$}
		{Hai mặt phẳng phân biệt cùng song song với mặt phẳng thứ ba thì song song với nhau}
		{Hai mặt phẳng song song chắn trên hai cát tuyến song song những đoạn thẳng bằng nhau}
		\loigiai{
			Qua một điểm nằm ngoài một mặt phẳng cho trước có vô số đường đường thẳng song song với mặt phẳng đã cho, các đường thẳng đó cùng nằm trên một mặt phẳng.
		}
	\end{ex}

	\begin{ex}%[Chuyên đề Toán 11 - 2023]%[Dương Phước Sang, dự án WTB - Toán 11 - new 2023]%[1K4BB-1]
		Cho hai đường thẳng phân biệt $a$, $b$ và mặt phẳng $(\alpha)$. Giả sử $a \parallel (\alpha)$ và $b \parallel (\alpha)$. Mệnh đề nào sau đây đúng?
		\choice
		{$a$ và $b$ không có điểm chung}
		{$a$ và $b$ hoặc song song hoặc chéo nhau}
		{$a$ và $b$ chéo nhau}
		{\True $a$ và $b$ hoặc song song hoặc chéo nhau hoặc cắt nhau}
		\loigiai{
			$a$ và $b$ hoặc song song hoặc chéo nhau hoặc cắt nhau.
		}
	\end{ex}

	\begin{ex}%[Chuyên đề Toán 11 - 2023]%[Dương Phước Sang, dự án WTB - Toán 11 - new 2023]%[1K4BB-1]
		Cho đường thẳng $a$ song song với mặt phẳng $(P)$ và $b$ là đường thẳng nằm trong $(P)$. Khi đó trường hợp nào sau đây \textbf{không} thể xảy ra?
		\choice
		{$a$ song song $b$}
		{\True $a$ cắt $b$}
		{$a$ và $b$ chéo nhau}
		{$a$ và $b$ không có điểm chung}
		\loigiai{
			Vì $a \parallel (P)$ nên $a$ không điểm chung với mặt phẳng $(P)$.\\
			Mà $b \subset (P)$ nên $a$ không điểm chung với $b$ tức $a$ không thể cắt $b$.
		}
	\end{ex}

	\begin{ex}%[Chuyên đề Toán 11 - 2023]%[Dương Phước Sang, dự án WTB - Toán 11 - new 2023]%[1K4BA-1]
		Hai đường thẳng cùng song song với đường thẳng thứ ba thì chúng
		\choice
		{\True hoặc song song hoặc trùng nhau}
		{chéo nhau}
		{trùng nhau}
		{song song}
		\loigiai{
			Hai đường thẳng cùng song song với đường thẳng thứ ba thì hai đường thẳng đó song song hoặc trùng nhau.
		}
	\end{ex}

	\begin{ex}%[Chuyên đề Toán 11 - 2023]%[Dương Phước Sang, dự án WTB - Toán 11 - new 2023]%[1K4BB-1]
		Trong không gian, cho các mệnh đề sau
		\begin{enumerate}[I.]
			\item Hai đường thẳng phân biệt cùng song song với một mặt phẳng thì song song với nhau.
			\item Hai mặt phẳng phân biệt chứa hai đường thẳng song song cắt nhau theo giao tuyến song song với hai đường thẳng đó.
			\item Nếu đường thẳng $a$ song song với đường thẳng $b$, đường thẳng $b$ nằm trên mặt phẳng $(P)$ thì $a$ song song với $(P)$.
			\item Qua điểm $A$ không thuộc mặt phẳng $(\alpha)$, kẻ được đúng một đường thẳng song song với $(\alpha)$.
		\end{enumerate}
		Số mệnh đề đúng là
		\choice
		{$2$}
		{\True $0$}
		{$1$}
		{$3$}
		\loigiai{
			\begin{enumerate}[I.]
				\item Hai đường thẳng phân biệt cùng song song với một mặt phẳng thì song song với nhau.\\
				Đây là một mệnh đề \textbf{sai} vì hai đường thẳng này có thể chéo nhau hoặc cắt nhau.
				\item Hai mặt phẳng phân biệt chứa hai đường thẳng song song cắt nhau theo giao tuyến song song với hai đường thẳng đó.\\
				Đây là một mệnh đề \textbf{sai} vì giao tuyến có thể hoặc song song với hai đường thẳng đó, hoặc trùng với một trong hai đường thẳng đó.
				\item Nếu đường thẳng $a$ song song với đường thẳng $b$, đường thẳng $b$ nằm trên mặt phẳng $(P)$ thì $a$ song song với $(P)$.\\
				Đây là một mệnh đề \textbf{sai} vì $a$ còn có thể thuộc $(P)$.
				\item Qua điểm $A$ không thuộc mặt phẳng $(\alpha)$, kẻ được đúng một đường thẳng song song với $(\alpha)$.\\
				Đây là một mệnh đề sai, vì qua $A$ ta sẽ kẻ được vô số đường song song với $(\alpha)$, các đường này đều nằm trên $(\beta)$ đi qua $A$ và song song với $(\alpha)$.
			\end{enumerate}
		}
	\end{ex}

	\begin{ex}%[Chuyên đề Toán 11 - 2023]%[Dương Phước Sang, dự án WTB - Toán 11 - new 2023]%[1K4BB-1]
		Chọn khẳng định đúng trong các khẳng định sau.
		\choice
		{Hai đường thẳng phân biệt cùng song song với một mặt phẳng thì song song với nhau}
		{\True Nếu $a \parallel (P)$ thì tồn tại trong $(P)$ đường thẳng $b$ để $b \parallel a$}
		{Nếu $\heva{&a \parallel (P)\\&b \subset (P)}$ thì $a \parallel b$}
		{Nếu $a \parallel (P)$ và đường thẳng $b$ cắt mặt phẳng $(P)$ thì hai đường thẳng $a$ và $b$ cắt nhau}
		\loigiai{
			Nếu $a \parallel (P)$ thì tồn tại trong $(P)$ đường thẳng $b$ để $b \parallel a$ là mệnh đề đúng.
		}
	\end{ex}

	\begin{ex}%[Chuyên đề Toán 11 - 2023]%[Dương Phước Sang, dự án WTB - Toán 11 - new 2023]%[1K4BB-1]
		Cho mặt phẳng $(\alpha)$ và đường thẳng $d \not\subset (\alpha)$. Khẳng định nào sau đây là \textbf{sai}?
		\choice
		{Nếu $d \parallel (\alpha)$ thì trong $(\alpha)$ tồn tại đường thẳng $\Delta$ sao cho $\Delta \parallel d$}
		{\True Nếu $d \parallel (\alpha)$ và $b \subset (\alpha)$ thì $b \parallel d$}
		{Nếu $d \cap (\alpha)=A$ và $d' \subset (\alpha)$ thì $d$ và $d'$ hoặc cắt nhau hoặc chéo nhau}
		{Nếu $d \parallel c$, $c \subset (\alpha)$ thì $d \parallel (\alpha)$}
		\loigiai{
			Nếu $d \parallel (\alpha)$ và $b \subset (\alpha)$ thì $b \parallel d$ là mệnh đề sai vì $b$ và $d$ có thể chéo nhau.
		}
	\end{ex}

	\begin{ex}%[Chuyên đề Toán 11 - 2023]%[Dương Phước Sang, dự án WTB - Toán 11 - new 2023]%[1K4BB-1]
		Cho các mệnh đề sau
		\begin{enumerate}[1.]
			\item Nếu $a \parallel (P)$ thì $a$ song song với mọi đường thẳng nằm trong $(P)$.
			\item Nếu $a \parallel (P)$ thì $a$ song song với một đường thẳng nào đó nằm trong $(P)$.
			\item Nếu $a \parallel (P)$ thì có vô số đường thẳng nằm trong $(P)$ song song với $a$.
			\item Nếu $a \parallel (P)$ thì có một đường thẳng $d$ nào đó nằm trong $(P)$ sao cho $a$ và $d$ đồng phẳng.
		\end{enumerate}
		Số mệnh đề đúng là
		\choice
		{$2$}
		{\True $3$}
		{$4$}
		{$1$}
		\loigiai{
			\begin{enumerate}[1.]
				\item Nếu $a \parallel (P)$ thì $a$ song song với mọi đường thẳng nằm trong $(P)$ là mệnh đề sai.
				\item Nếu $a \parallel (P)$ thì $a$ song song với một đường thẳng nào đó nằm trong $(P)$ là mệnh đề đúng.
				\item Nếu $a \parallel (P)$ thì có vô số đường thẳng nằm trong $(P)$ song song với $a$ là mệnh đề đúng.
				\item Nếu $a \parallel (P)$ thì có một đường thẳng $d$ nào đó nằm trong $(P)$ sao cho $a$ và $d$ đồng phẳng là mệnh đề đúng.
			\end{enumerate}
			Vậy có $3$ mệnh đề đúng.
		}
	\end{ex}

	\begin{ex}%[Chuyên đề Toán 11 - 2023]%[Dương Phước Sang, dự án WTB - Toán 11 - new 2023]%[1K4BB-1]
		Trong các khẳng định sau khẳng định nào \textbf{sai}?
		\choice
		{\True Nếu một đường thẳng song song với một trong hai mặt phẳng song song thì nó song song với mặt phẳng còn lại}
		{Nếu một đường thẳng cắt một trong hai mặt phẳng song song thì nó cắt mặt phẳng còn lại}
		{Nếu hai đường thẳng song song thì chúng cùng nằm trên một mặt phẳng}
		{Nếu hai mặt phẳng phân biệt cùng song song với một mặt phẳng thì chúng song song với nhau}
		\loigiai{
			Giả sử $(\alpha)$ song song với $(\beta)$. Một đường thẳng $a$ song song với $(\beta)$ có thể nằm trên $(\alpha)$.
		}
	\end{ex}

	\begin{ex}%[Chuyên đề Toán 11 - 2023]%[Dương Phước Sang, dự án WTB - Toán 11 - new 2023]%[1K4BB-1]
		Tìm khẳng định \textbf{sai} trong các khẳng định sau đây
		\choice
		{Nếu hai mặt phẳng song song cùng cắt mặt phẳng thứ ba thì hai giao tuyến tạo thành song song với nhau}
		{Ba mặt phẳng đôi một song song chắn trên hai đường thẳng chéo nhau những đoạn thẳng tương ứng tỉ lệ}
		{Nếu mặt phẳng $(P)$ song song với mặt phẳng $(Q)$ thì mọi đường thẳng nằm trên mặt phẳng $(P)$ đều song song với mặt phẳng $(Q)$}
		{\True Nếu mặt phẳng $(P)$ có chứa hai đường thẳng phân biệt và hai đường thẳng đó cùng song song song với mặt phẳng $(Q)$ thì mặt phẳng $(P)$ song song với mặt phẳng $(Q)$}
		\loigiai{
			\immini{
				Ví dụ $(SAD)$ chứa $MN$ và $PQ$ cùng song song với $(ABCD)$ nhưng $(SAD)$ cắt $(ABCD)$.}
			{\vspace{-0.5cm}
				\begin{tikzpicture}[scale=0.7, font=\footnotesize, line join=round, line cap=round]
					\foreach \x\y\t in {0/0/A,-1.7/-1.6/B,2.5/-1.6/C,-0.6/2.8/S}
					\coordinate (\t) at (\x,\y);
					\coordinate (D) at ($(A)+(C)-(B)$);
					\coordinate (M) at ($(S)!1/2!(A)$);
					\coordinate (N) at ($(S)!1/2!(D)$);
					\coordinate (P) at ($(S)!3/4!(A)$);
					\coordinate (Q) at ($(S)!3/4!(D)$);
					\draw (S)--(B)--(C)--(S)--(D)--(C);
					\draw[dashed](B)--(A)--(D) (S)--(A) (M)--(N) (P)--(Q);
					\foreach \t/\g in {S/90,A/160,B/-140,C/-45,D/0,M/180,N/30,P/180,Q/30}
					\draw[fill=black] (\t) circle(1pt)
					node[shift={(\g:7pt)}]{$\t$};
			\end{tikzpicture}}
		}
	\end{ex}

	\begin{ex}%[Chuyên đề Toán 11 - 2023]%[Dương Phước Sang, dự án WTB - Toán 11 - new 2023]%[1K4BB-1]
		Trong các mệnh đề sau, mệnh đề nào đúng?
		\choice
		{Hai đường thẳng cùng song song với một mặt phẳng thì song song với nhau}
		{Hai đường thẳng cùng song song với một mặt phẳng thì trùng nhau}
		{Hai đường thẳng cùng song song với một mặt phẳng thì chéo nhau}
		{\True Hai đường thẳng cùng song song với một mặt phẳng có thể chéo nhau, song song, cắt nhau hoặc trùng nhau}
		\loigiai{
			Hai đường thẳng cùng song song với một mặt phẳng có thể chéo nhau, song song, cắt nhau hoặc trùng nhau.
		}
	\end{ex}

	\begin{ex}%[Chuyên đề Toán 11 - 2023]%[Dương Phước Sang, dự án WTB - Toán 11 - new 2023]%[1K4BB-1]
		Cho các giả thiết sau đây. Giả thiết nào kết luận đường thẳng $a$ song song với mặt phẳng $(\alpha)$?
		\choice
		{$a \parallel b$ và $b \subset (\alpha)$}
		{$a \parallel (\beta)$ và $(\beta) \parallel (\alpha)$}
		{$a \parallel b$ và $b \parallel (\alpha)$}
		{\True $a \cap (\alpha)=\varnothing$}
		\loigiai{
			Chọn $a \cap (\alpha)=\varnothing$.
		}
	\end{ex}

	\begin{ex}%[Chuyên đề Toán 11 - 2023]%[Dương Phước Sang, dự án WTB - Toán 11 - new 2023]%[1K4BB-1]
		Cho hai mặt phẳng $(P)$, $(Q)$ cắt nhau theo giao tuyến là đường thẳng $d$. Đường thẳng $a$ song song với cả hai mặt phẳng $(P)$, $(Q)$. Khẳng định nào sau đây đúng?
		\choice
		{$a$, $d$ trùng nhau}
		{$a$, $d$ chéo nhau}
		{\True $a$ song song $d$}
		{$a$, $d$ cắt nhau}
		\loigiai{
			Sử dụng hệ quả: Nếu hai mặt phẳng phân biệt cùng song song với một đường thẳng thì giao tuyến của chúng cũng song song với đường thẳng đó.
		}
	\end{ex}

	\begin{ex}%[Chuyên đề Toán 11 - 2023]%[Dương Phước Sang, dự án WTB - Toán 11 - new 2023]%[1K4BB-1]
		Cho ba đường thẳng đôi một chéo nhau $a$, $b$, $c$. Gọi $(P)$ là mặt phẳng qua $a$, $(Q)$ là mặt phẳng qua $b$ sao cho giao tuyến của $(P)$ và $(Q)$ song song với $c$. Có nhiều nhất bao nhiêu mặt phẳng $(P)$ và $(Q)$ thỏa mãn yêu cầu trên?
		\choice
		{Vô số mặt phẳng $(P)$ và $(Q)$}
		{Một mặt phẳng $(P)$, vô số mặt phẳng $(Q)$}
		{Một mặt phẳng $(Q)$, vô số mặt phẳng $(P)$}
		{\True Một mặt phẳng $(P)$, một mặt phẳng $(Q)$}
		\loigiai{
			\immini{
				Vì $c$ song song với giao tuyến của $(P)$ và $(Q)$ nên $c \parallel (P)$ và $c \parallel (Q)$.\\
				Khi đó, $(P)$ là mặt phẳng chứa $a$ và song song với $c$, mà $a$ và $c$ chéo nhau nên chỉ có một mặt phẳng như vậy.\\
				Tương tự cũng chỉ có một mặt phẳng $(Q)$ chứa $b$ và song song với $c$.\\
				Vậy có nhiều nhất một mặt phẳng $(P)$ và một mặt phẳng $(Q)$ thỏa yêu cầu bài toán.}
			{\vspace{-0.5cm}
				\begin{tikzpicture}[scale=0.8, font=\footnotesize, line join=round, line cap=round]
					\foreach \x\y\t in {0/0/A, 0/4/B, 2/3/C,-2.5/3.2/E}
					\coordinate (\t) at (\x,\y);
					\coordinate (D) at ($(A)+(C)-(B)$);
					\coordinate (F) at ($(A)+(E)-(B)$);
					\draw (A)--(B)--(C)--(D)--(A)--(F)--(E)--(B);
					\draw (-0.5,0.8)--(-1.9,3) node[below left=-2pt]{$a$};
					\draw (0.5,0.2)--(1.6,2.5) node[left]{$b$};
					\draw (2.4,-0.5)--(2.4,3.8)node[left]{$c$};
					\path (F) pic[draw,angle radius=13]{angle=A--F--E}node[shift={(50:8pt)}]{$P$};
					\path (D) pic[draw,angle radius=13]{angle=C--D--A}node[shift={(125:8pt)}]{$Q$};
			\end{tikzpicture}}
		}
	\end{ex}
\Closesolutionfile{ans}
\begin{indapan}{10}
	{ans/ans1-C4B12-5}
\end{indapan}
\begin{dang}{Đường thẳng song song với mặt phẳng}
\end{dang}
\Opensolutionfile{ans}[ans/ans1-C4B12-6]
	\begin{ex}%[Chuyên đề Toán 11 - 2023]%[Dương Phước Sang, dự án WTB - Toán 11 - new 2023]%[1K4BB-2]
		Cho hình chóp tứ giác $S.ABCD$. Gọi $M$, $N$ lần lượt là trung điểm của $SA$ và $SC$. Mệnh đề nào sau đây đúng?
		\choice
		{$MN \parallel (SAB)$}
		{$MN \parallel (SBC)$}
		{$MN \parallel (SBD)$}
		{\True $MN \parallel (ABCD)$}
		\loigiai{
			\immini{
				Vì $MN$ là đường trung bình của tam giác $SAC \Rightarrow MN \parallel AC$.\\
				Mặt khác $AC \subset (ABCD) \Rightarrow MN \parallel (ABCD)$.}
			{\vspace{-0.5cm}
				\begin{tikzpicture}[scale=0.55, font=\footnotesize, line join=round, line cap=round]
					\foreach \x\y\t in {0/0/A,1.2/-1.7/B,4/-1.2/C,5/0/D,2.4/3.8/S}
					\coordinate (\t) at (\x,\y);
					\coordinate (M) at ($(S)!1/2!(A)$);
					\coordinate (N) at ($(S)!1/2!(C)$);
					\draw (B)--(S)--(A)--(B)--(C)--(D)--(S)--(C);
					\draw[dashed] (C)--(A)--(D) (M)--(N);
					\foreach \t/\g in {S/90,A/180,B/-95,C/-40,D/0,M/140,N/10}
					\draw[fill=black] (\t) circle(1pt)
					node[shift={(\g:7.5pt)}]{$\t$};
			\end{tikzpicture}}
		}
	\end{ex}

	\begin{ex}%[Chuyên đề Toán 11 - 2023]%[Dương Phước Sang, dự án WTB - Toán 11 - new 2023]%[1K4BB-2]
		Cho hình chóp $S.ABCD$. Gọi $M$, $N$ lần lượt là trọng tâm tam giác $SAB$ và tam giác $SCD$. Khi đó $MN$ song song với mặt phẳng
		\choice
		{$(SAC)$}
		{$(SBD)$}
		{$(SAB)$}
		{\True $(ABCD)$}
		\loigiai{
			\immini{
				Gọi $E$ và $F$ lần lượt là trung điểm của $AB$ và $CD$.\\
				Do $M$, $N$ là trọng tâm $\triangle SAB$, $\triangle SCD$ nên $S$, $M$, $E$ thẳng hàng; $S$, $N$, $F$ thẳng hàng.\\
				Xét $\vartriangle SEF$ có $\dfrac{SM}{SE}=\dfrac{2}{3}=\dfrac{SN}{SF}$ nên theo định lý Ta lét ta có $MN \parallel EF$.\\
				Mà $EF \subset (ABCD)$ nên $MN \parallel (ABCD)$.}
			{\vspace{-0.5cm}
				\begin{tikzpicture}[scale=0.8, font=\footnotesize, line join=round, line cap=round]
					\foreach \x\y\t in {0/0/A,1.2/-1.7/B,4/-1.2/C,5/0/D,2.4/3.8/S}
					\coordinate (\t) at (\x,\y);
					\coordinate (E) at ($(A)!1/2!(B)$);
					\coordinate (F) at ($(C)!1/2!(D)$);
					\coordinate (M) at ($(S)!2/3!(E)$);
					\coordinate (N) at ($(S)!2/3!(F)$);
					\draw (B)--(S)--(A)--(B)--(C)--(D)--(S)--(C) (E)--(S)--(F);
					\draw[dashed](A)--(D) (E)--(F) (M)--(N);
					\foreach \t/\g in {S/90,A/180,B/-95,C/-40,D/0,E/-150,F/-30,M/150,N/30}
					\draw[fill=black] (\t) circle(1pt)
					node[shift={(\g:7pt)}]{$\t$};
			\end{tikzpicture}}
		}
	\end{ex}

	\begin{ex}%[Chuyên đề Toán 11 - 2023]%[Dương Phước Sang, dự án WTB - Toán 11 - new 2023]%[1K4BB-2]
		Cho hình chóp $S.ABC$. Gọi $M$, $N$ lần lượt là trung điểm của các cạnh $SB$, $SC$. Trong các khẳng định sau, khẳng định nào đúng?
		\choice
		{\True $MN \parallel (ABC)$}
		{$MN \parallel (SAB)$}
		{$MN \parallel (SAC)$}
		{$MN \parallel (SBC)$}
		\loigiai{
			\immini{
				Theo giả thiết thì $M$, $N$ lần lượt là trung điểm của $SB$, $SC$ nên $MN$ là đường trung bình của $\triangle SBC$, do đó $MN \parallel BC$.\\
				Vì $\heva{&MN \not\subset (ABC)\\&BC \subset (ABC)\\&MN \parallel BC}$ nên $MN \parallel (ABC)$.}
			{\vspace{-0.5cm}
				\begin{tikzpicture}[scale=0.7, font=\footnotesize, line join=round, line cap=round]
					\foreach \x\y\t in {0/0/A,1.3/-1.6/B,4.5/0/C,1/3.5/S}
					\coordinate (\t) at (\x,\y);
					\coordinate (M) at ($(S)!1/2!(B)$);
					\coordinate (N) at ($(S)!1/2!(C)$);
					\draw (S)--(B)--(A)--(S)--(C)--(B) (M)--(N);
					\draw[dashed](A)--(C);
					\foreach \t/\g in {S/90,A/180,B/-90,C/0,M/180,N/40}
					\draw[fill=black] (\t) circle(1pt)
					node[shift={(\g:7pt)}]{$\t$};
			\end{tikzpicture}}
		}
	\end{ex}

	\begin{ex}%[Chuyên đề Toán 11 - 2023]%[Dương Phước Sang, dự án WTB - Toán 11 - new 2023]%[1K4BB-2]
		Cho hình chóp $S.ABCD$ có đáy là hình bình hành. Gọi $I$ và $J$ lần lượt là trung điểm của $SC$ và $BC$. Chọn khẳng định đúng trong các khẳng định sau
		\choice
		{$IJ \parallel (SAC)$}
		{\True $JI \parallel (SAB)$}
		{$JI \parallel (SBC)$}
		{$JI \parallel (SAD)$}
		\loigiai{
			\immini{
				Ta có $JI \parallel SB$, $SB \subset (SAB)$.\\
				Vậy $JI \parallel (SAB)$.}
			{\vspace{-0.5cm}
				\begin{tikzpicture}[scale=0.7, font=\footnotesize, line join=round, line cap=round]
					\foreach \x\y\t in {0/0/A,-1.7/-1.6/B,2.5/-1.6/C,0.4/2.8/S}
					\coordinate (\t) at (\x,\y);
					\coordinate (D) at ($(A)+(C)-(B)$);
					\coordinate (I) at ($(S)!1/2!(C)$);
					\coordinate (J) at ($(B)!1/2!(C)$);
					\draw (S)--(B)--(C)--(S)--(D)--(C) (I)--(J);
					\draw[dashed](B)--(A)--(D) (S)--(A);
					\foreach \t/\g in {S/90,A/160,B/-140,C/-45,D/0,I/40,J/-90}
					\draw[fill=black] (\t) circle(1pt)
					node[shift={(\g:7pt)}]{$\t$};
			\end{tikzpicture}}
		}
	\end{ex}

	\begin{ex}%[Chuyên đề Toán 11 - 2023]%[Dương Phước Sang, dự án WTB - Toán 11 - new 2023]%[1K4BB-2]
		Cho hình chóp $S.ABCD$ có đáy $ABCD$ là hình thoi. Gọi $H$, $I$, $K$ lần lượt là trung điểm của $SA$, $AB$, $CD$. Khẳng định nào sau đây đúng?
		\choice
		{\True $HK \parallel (SBC)$}
		{$HK \parallel (SBD)$}
		{$HK \parallel (SAC)$}
		{$HK \parallel (SAD)$}
		\loigiai{
			\immini{
				Ta có $HI$ là đường trung bình của tam giác $SAB$ nên 
				$$HI \parallel SB \subset (SBC) \Rightarrow HI \parallel (SBC).$$
				Lại có $I$, $K$ lần lượt là trung điểm $AB$, $CD$ nên 
				$$IK \parallel BC \subset (SBC) \Rightarrow IK \parallel (SBC).$$
				Từ đó, ta có $(HIK) \parallel (SBC)$.\\
				Mà $HK \subset (HIK)$ nên $HK \parallel (SBC)$.}
			{\vspace{-0.5cm}
				\begin{tikzpicture}[scale=0.7, font=\footnotesize, line join=round, line cap=round]
					\foreach \x\y\t in {0/0/A,-1.7/-1.6/B,2.5/-1.6/C,0.2/2.8/S}
					\coordinate (\t) at (\x,\y);
					\coordinate (D) at ($(A)+(C)-(B)$);
					\coordinate (H) at ($(S)!1/2!(A)$);
					\coordinate (I) at ($(B)!1/2!(A)$);
					\coordinate (K) at ($(C)!1/2!(D)$);
					\draw (S)--(B)--(C)--(S)--(D)--(C);
					\draw[dashed](B)--(A)--(D) (S)--(A) (H)--(I)--(K)--(H);
					\foreach \t/\g in {S/90,A/160,B/-140,C/-45,D/0,H/30,I/-70,K/-40}
					\draw[fill=black] (\t) circle(1pt)
					node[shift={(\g:7pt)}]{$\t$};
			\end{tikzpicture}}
		}
	\end{ex}

	\begin{ex}%[Chuyên đề Toán 11 - 2023]%[Dương Phước Sang, dự án WTB - Toán 11 - new 2023]%[1K4KB-2]
		Cho tứ diện $ABCD$, $G$ là trọng tâm $\triangle ABD$ và $M$ là điểm trên cạnh $BC$ sao cho $BM=2MC$. Đường thẳng $MG$ song song với mặt phẳng nào sau đây?
		\choice
		{\True $(ACD)$}
		{$(ABC)$}
		{$(ABD)$}
		{$(BCD)$}
		\loigiai{
			\immini{
				Gọi $P$ là trung điểm của $AD$.\\
				Ta có $\dfrac{BM}{BC}=\dfrac{BG}{BP}=\dfrac{2}{3} \Rightarrow MG \parallel CP$.\\
				Mà $\heva{&CP \subset (ACD)\\&MG \not\subset (ACD)}$ nên $MG \parallel (ACD)$.}
			{\vspace{-0.5cm}
				\begin{tikzpicture}[scale=0.75, font=\footnotesize, line join=round, line cap=round]
					\foreach \x\y\t in {0/0/B,1.1/-1.6/C,4.5/0/D,1/3.5/A}
					\coordinate (\t) at (\x,\y);
					\coordinate (M) at ($(B)!2/3!(C)$);
					\coordinate (P) at ($(D)!1/2!(A)$);
					\coordinate (G) at ($(B)!2/3!(P)$);
					\coordinate (m) at ($(D)!1/2!(B)$);
					\draw (M)--(A)--(B)--(C)--(A)--(D)--(C)--(P);
					\draw[dashed](B)--(D) (P)--(B) (A)--(m) (M)--(G);
					\foreach \t/\g in {A/90,B/180,C/-90,D/0,M/-160,G/70,P/40}
					\draw[fill=black] (\t) circle(1pt)
					node[shift={(\g:7pt)}]{$\t$};
			\end{tikzpicture}}
		}
	\end{ex}

	\begin{ex}%[Chuyên đề Toán 11 - 2023]%[Dương Phước Sang, dự án WTB - Toán 11 - new 2023]%[1K4KB-2]
		Cho tứ diện $ABCD$. Gọi $G$ là trọng tâm của tam giác $ABD$, $Q$ thuộc cạnh $AB$ sao cho $AQ=2QB$ và $P$ là trung điểm của $AB$. Khẳng định nào sau đây đúng?
		\choice
		{$GQ \parallel (ACD)$}
		{\True $GQ \parallel (BCD)$}
		{$GQ$ cắt $(BCD)$}
		{$Q$ thuộc mặt phẳng $(CDP)$}
		\loigiai{
			\immini{
				Gọi $M$ là trung điểm của $BD$.\\
				Vì $G$ là trọng tâm tam giác $ABD$ nên $\dfrac{AG}{AM}=\dfrac{2}{3}$.\\
				Điểm $Q \in AB$ sao cho $AQ=2QB \Leftrightarrow \dfrac{AQ}{AB}=\dfrac{2}{3}$.\\
				Suy ra $\dfrac{AG}{AM}=\dfrac{AQ}{AB} \Rightarrow GQ \parallel BD$.\\
				Mặt khác $BD$ nằm trong mặt phẳng $(BCD)$ suy ra $GQ \parallel (BCD)$.}
			{\vspace{-0.5cm}
				\begin{tikzpicture}[scale=0.75, font=\footnotesize, line join=round, line cap=round]
					\foreach \x\y\t in {0/0/B,1.3/-1.6/C,4.5/0/D,1/3.5/A}
					\coordinate (\t) at (\x,\y);
					\coordinate (M) at ($(B)!1/2!(D)$);
					\coordinate (P) at ($(B)!1/2!(A)$);
					\coordinate (G) at ($(A)!2/3!(M)$);
					\coordinate (Q) at ($(A)!2/3!(B)$);
					\draw (A)--(B)--(C)--(A)--(D)--(C);
					\draw[dashed](B)--(D)--(P) (A)--(M) (G)--(Q);
					\foreach \t/\g in {A/90,B/180,C/-90,D/0,M/-90,G/30,P/170,Q/170}
					\draw[fill=black] (\t) circle(1pt)
					node[shift={(\g:7pt)}]{$\t$};
			\end{tikzpicture}}
		}
	\end{ex}

	\begin{ex}%[Chuyên đề Toán 11 - 2023]%[Dương Phước Sang, dự án WTB - Toán 11 - new 2023]%[1K4KC-2]
		\immini{
			Cho hình lăng trụ $ABCD.A'B'C'D'$ có hai đáy là các hình bình hành. Các điểm $M$, $N$, $P$ lần lượt là trung điểm của cạnh $AD$, $BC$, $CC'$. Trong các mệnh đề sau có bao nhiêu mệnh đề \textbf{sai}?
			\begin{enumerate}[i)]
				\item $A'B' \parallel (MNP)$.
				\item $(MNP) \parallel (BC'D')$.
				\item $(MNP) \parallel (B'C'D')$.
				\item $DD'$ cắt mp $(MNP)$.
			\end{enumerate}
			Trong các mệnh đề trên có bao nhiêu mệnh đề \textbf{sai}?}
		{\vspace{-0.5cm}
			\begin{tikzpicture}[scale=0.96, font=\footnotesize, line join=round, line cap=round]
				\def\h{4}
				\foreach \x\y\t in {0/0/A,-0.6/-1./B,2.6/-1./C}
				\coordinate (\t) at (\x,\y);
				\coordinate (D) at ($(A)+(C)-(B)$);
				\coordinate (A') at ($(A)+(0,2.5)$);
				\coordinate (B') at ($(B)+(0,2.5)$);
				\coordinate (C') at ($(C)+(0,2.5)$);
				\coordinate (D') at ($(D)+(0,2.5)$);
				\coordinate (M) at ($(D)!1/2!(A)$);
				\coordinate (N) at ($(B)!1/2!(C)$);
				\coordinate (P) at ($(C)!1/2!(C')$);
				\coordinate (Q) at ($(D)!1/2!(D')$);
				\draw (B')--(A')--(D')--(C')--(B')--(B)--(C)--(D)--(D') (C')--(C) (P)--(N);
				\draw[dashed](B)--(A)--(D) (A)--(A') (N)--(M)--(P);
				\foreach \t/\g in {A/170,B/-150,C/-70,D/0,A'/100,B'/170,C'/-20,D'/50,M/100,N/-90,P/0}
				\draw[fill=black] (\t) circle(1pt)
				node[shift={(\g:7pt)}]{$\t$};
		\end{tikzpicture}}
		\choice
		{$4$}
		{$2$}
		{$3$}
		{\True $1$}
		\loigiai{
			\immini{
				Ta có $\heva{&A'B' \parallel AB\\&AB \parallel MN} \Rightarrow A'B' \parallel MN \Rightarrow A'B' \parallel (MNP)$.\\
				Ta có $\heva{&MN \parallel C'D'\\&NP \parallel BC'} \Rightarrow (MNP) \parallel (BC'D')$.\\
				Ta có $\heva{&(MNP) \cap (ABCD)=MN\\&(B'C'D') \parallel (ABCD)} \Rightarrow (MNP)$ cắt $(B'C'D')$.\\
				Ta có $\heva{&(MNP) \parallel (BC'D')\\&DD' \cap (BC'D')=D'} \Rightarrow DD' \cap (MNP)=Q$.\\
				Vậy chỉ có mệnh đề iii) sai.}
			{\vspace{-0.5cm}
				\begin{tikzpicture}[scale=1, font=\footnotesize, line join=round, line cap=round]
					\def\h{4}
					\foreach \x\y\t in {0/0/A,-0.6/-1./B,2.6/-1./C}
					\coordinate (\t) at (\x,\y);
					\coordinate (D) at ($(A)+(C)-(B)$);
					\coordinate (A') at ($(A)+(0,2.5)$);
					\coordinate (B') at ($(B)+(0,2.5)$);
					\coordinate (C') at ($(C)+(0,2.5)$);
					\coordinate (D') at ($(D)+(0,2.5)$);
					\coordinate (M) at ($(D)!1/2!(A)$);
					\coordinate (N) at ($(B)!1/2!(C)$);
					\coordinate (P) at ($(C)!1/2!(C')$);
					\coordinate (Q) at ($(D)!1/2!(D')$);
					\draw (B')--(A')--(D')--(C')--(B')--(B)--(C)--(D)--(D') (C')--(C) (Q)--(P)--(N);
					\draw[dashed](B)--(A)--(D) (D')--(A)--(A') (N)--(M)--(Q);
					\foreach \t/\g in {A/170,B/-150,C/-70,D/0,A'/100,B'/170,C'/-20,D'/50,M/100,N/-90,P/0,Q/0}
					\draw[fill=black] (\t) circle(1pt)
					node[shift={(\g:7pt)}]{$\t$};
			\end{tikzpicture}}
		}
	\end{ex}

	\begin{ex}%[Chuyên đề Toán 11 - 2023]%[Dương Phước Sang, dự án WTB - Toán 11 - new 2023]%[1K4BB-2]
		Cho hai hình bình hành $ABCD$ và $ABEF$ nằm trong hai mặt phẳng khác nhau lần lượt có tâm $O$ và $O'$. Mệnh đề nào sau đây \textbf{sai}?
		\choice
		{$OO' \parallel (ADF)$}
		{$OO' \parallel (BCE)$}
		{\True $OO' \parallel (ACE)$}
		{$OO' \parallel (DCEF)$}
		\loigiai{
			\immini{
				$OO' \parallel (ADF)$ đúng vì 
				$\heva{&OO' \parallel DF, DF \subset (ADF)\\&OO' \not \subset (ADF)} \Rightarrow OO' \parallel (ADF)$.\\
				$OO' \parallel (BCE)$ đúng vì 
				$\heva{&OO' \parallel EC, EC \subset (BCE)\\&OO' \not \subset (BCE)} \Rightarrow OO' \parallel (BCE)$.\\
				$OO' \parallel (ACE)$ sai vì 
				$OO' \subset (ACE)$.\\
				$OO' \parallel (DCEF)$ đúng vì 
				$\heva{&OO' \parallel EC, EC \subset (DCEF)\\&OO' \not \subset (DCEF)} \Rightarrow OO' \parallel (DCEF)$.}
			{\vspace{-0.5cm}
				\begin{tikzpicture}[scale=0.9, font=\footnotesize, line join=round, line cap=round]
					\foreach \x\y\t in {0.3/3.4/B,0/0/A,1.2/-1/F,3.5/0/D}
					\coordinate (\t) at (\x,\y);
					\coordinate (E) at ($(F)+(B)-(A)$);
					\coordinate (C) at ($(D)+(B)-(A)$);
					\coordinate (O) at ($(D)!1/2!(B)$);
					\coordinate (O') at ($(F)!1/2!(B)$);
					\draw (F)--(A)--(B)--(C)--(D)--(F)--(E)--(C) (F)--(B)--(E)--(A);
					\draw[dashed](C)--(A)--(D)--(B) (O)--(O');
					\foreach \t/\g in {A/180,F/-90,D/0,B/170,E/80,C/0,O/80,O'/180}
					\draw[fill=black] (\t) circle(1pt)
					node[shift={(\g:7pt)}]{$\t$};
			\end{tikzpicture}}
		}
	\end{ex}

	\begin{ex}%[Chuyên đề Toán 11 - 2023]%[Dương Phước Sang, dự án WTB - Toán 11 - new 2023]%[1K4BB-2]
		Cho hình chóp $S.ABCD$, có đáy $ABCD$ là hình bình hành tâm $O$. Gọi $H$, $K$ lần lượt là trung điểm của $BC$, $CD$. Mệnh đề nào dưới đây \textbf{sai}?
		\choice
		{$HK \parallel (SBD)$}
		{$OK \parallel (SAD)$}
		{$OH \parallel (SAB)$}
		{\True $HK \parallel (SAB)$}
		\loigiai{
			\immini{
				Ta có $HK\not \subset (SBD)$.\\
				Ta thấy $HK$ là đường trung bình của tam giác $BCD$ nên $HK \parallel BD$.\\
				Mà $BD \subset (SBD)$ nên $HK \parallel (SBD)$.\\
				Ta có $OK \not \subset (SAD)$.\\
				Ta thấy $OK$ là đường trung bình của tam giác $ACD$ nên $OK \parallel AD$.\\
				Mà $AD \subset (SAD)$ do đó $OK \parallel (SAD)$.\\
				Ta có $OH\not \subset (SAB)$.\\
				Ta thấy $OH$ là đường trung bình của tam giác $ABC$ nên $OH \parallel AB$.\\
				Mà $AB \subset (SAB)$ do đó $OH \parallel (SAB)$.\\
				Trong $(ABCD)$ ta thấy $AB \cap HK$ mà $AB \subset (SAB)$ nên $HK \nparallel (SAB)$.}
			{\vspace{-0.2cm}
				\begin{tikzpicture}[scale=0.85, font=\footnotesize, line join=round, line cap=round]
					\foreach \x\y\t in {0.5/3/S,0/0/A,-1.3/-1.6/B,2.5/-1.6/C}
					\coordinate (\t) at (\x,\y);
					\coordinate (D) at ($(A)+(C)-(B)$);
					\coordinate (O) at ($(A)!1/2!(C)$);
					\coordinate (H) at ($(B)!1/2!(C)$);
					\coordinate (K) at ($(D)!1/2!(C)$);
					\draw (S)--(C)--(B)--(S)--(D)--(C);
					\draw[dashed](S)--(A)--(C) (D)--(B)--(A)--(D) (H)--(O)--(K)--(H);
					\foreach \t/\g in {S/90,A/160,B/-130,C/-60,D/0,O/90,H/-90,K/-20}
					\draw[fill=black] (\t) circle(1pt)
					node[shift={(\g:7pt)}]{$\t$};
			\end{tikzpicture}}
		}
	\end{ex}

	\begin{ex}%[Chuyên đề Toán 11 - 2023]%[Quan Ón, dự án WTB - Toán 11 - new 2023]%[1K4BB-2]
		Cho lăng trụ $ABC.A'B'C'$. Gọi $M$, $N$ lần lượt là trung điểm $AA'$ và $B'C'$. Khi đó đường thẳng $AB'$ song song với mặt phẳng
		\choice
		{$(A'MN)$}
		{$(C'MN)$}
		{\True $(A'CN)$}
		{$(CMN)$}
		\loigiai{
			\immini{
				Gọi $H$, $K$ lần lượt là trung điểm của $A'B'$, $A'C$.\\
				Ta có: $HM$ là đường trung bình $\triangle A'B'A$ $\Rightarrow HM \parallel AB'$.$\quad (1)$\\
				Hơn nữa, ta có $HN$, $MK$ lần lượt là đường trung bình $\triangle A'B'C'$, $\triangle A'AC$.\\
				$\Rightarrow \heva{&HN \parallel A'C',\,HN=\dfrac{1}{2}A'C'\\&MK \parallel AC,\,MK=\dfrac{1}{2}AC}$ mà $\heva{&A'C' \parallel AC\\&A'C'=AC}$ nên $\heva{&HN \parallel MK\\&HN=MK}$\\
				$\Rightarrow HNKM$ là hình bình hành.\\
				$\Rightarrow HM \parallel NK$. $\quad (2)$\\
				Từ $(1)$ và $(2)$ suy ra: $AB' \parallel NK \Rightarrow AB' \parallel (A'NC)$.
			}{
				\begin{tikzpicture}[scale=1, font=\footnotesize, line join=round, line cap=round, >=stealth]
					\path
					(0,0) coordinate (B)
					(1.5,-1.4) coordinate (A)
					(4,0) coordinate (C)
					(1.5,2.6) coordinate (A')
					(0,4) coordinate (B')
					(4,4) coordinate (C')
					($(A')!0.5!(A)$) coordinate (M)
					($(A')!0.5!(C)$) coordinate (K)
					($(B')!0.5!(C')$) coordinate (N)
					($(B')!0.5!(A')$) coordinate (H);
					\draw (C')--(A')--(B')--(C')--(C)--(A)--(B)--(B')--(A)--(A') (M)--(H)--(N) (A')--(C) (M)--(K);
					\draw[dashed] (B)--(C)--(N)--(K);
					\foreach \d/\g in {A/-90,B/-135,C/-45,A'/-135,B'/135,C'/45,M/-45,N/90,K/0,H/-135} \fill (\d)node[shift={(\g:0.3)}]{$\d$} circle(1pt);
				\end{tikzpicture}
			}
		}
	\end{ex}

	\begin{ex}%[Chuyên đề Toán 11 - 2023]%[Quan Ón, dự án WTB - Toán 11 - new 2023]%[1K4BB-2]
		Cho hình chóp $S.ABCD$ có đáy $ABCD$ là hình bình hành. Gọi $M$ là một điểm trên cạnh $SA$, mặt phẳng $(\alpha)$ qua $M$ song song với $SB$ và $AC$. Mặt phẳng $(\alpha)$ cắt $AB$, $BC$, $SC$, $SD$, $BD$ lần lượt tại $N$, $E$, $F$, $I$, $J$. Khẳng định nào sau đây là đúng?
		\choice
		{$MN \parallel (SCD)$}
		{$EF \parallel (SAD)$}
		{$NF \parallel (SAD)$}
		{\True $IJ \parallel (SAB)$}
		\loigiai{
			\begin{center}
				\begin{tikzpicture}[>=stealth,line join=round,line cap=round,font=\footnotesize,scale=1]	
					\path
					(0,0) coordinate (A)
					(4,0) coordinate (D)
					(-1.5,-1.5) coordinate (B)
					($(B)-(A)+(D)$) coordinate (C)
					(1,3) coordinate (S)
					($(A)!0.4!(B)$) coordinate (N)
					($(C)!0.4!(B)$) coordinate (E)
					($(A)!0.4!(S)$) coordinate (M)
					($(C)!0.4!(S)$) coordinate (F)
					($(S)!1/3!(D)$) coordinate (I)
					(intersection of B--D and N--E) coordinate (J);
					\draw (C)--(D)--(S)--(C)--(B)--(S) (E)--(F)--(I);
					\draw[dashed] (S)--(A)--(D)--(B)--(A)--(C) (E)--(N)--(M)--(I)--(J);
					\foreach \d/\g in {A/30,B/-135,C/-45,D/0,S/90,E/-90,F/0,I/30,M/-45,N/-90,J/-100} \fill (\d)node[shift={(\g:0.3)}]{$\d$} circle(1pt);
				\end{tikzpicture}
			\end{center}
			Ta có: $\heva{&IJ=(\alpha) \cap (SBD)\\&(\alpha) \parallel SB \subset (SBD)} \Rightarrow (\alpha) \cap (SBD)=IJ \parallel SB$.\\
			Mà $SB \subset (SAB) \Rightarrow IJ \parallel (SAB)$.
		}
	\end{ex}

	\begin{ex}%[Chuyên đề Toán 11 - 2023]%[Quan Ón, dự án WTB - Toán 11 - new 2023]%[1K4BB-2]
		Cho hình chóp $S.ABCD$ có đáy $ABCD$ là hình thoi tâm $O$. Gọi $I$ là trung điểm của $BC$, $K$ thuộc cạnh $SD$ sao cho $SK = \dfrac{1}{2}KD$, $M$ là giao điểm của của $BD$ và $AI$. Khẳng định nào sau đây là đúng?
		\choice
		{$MK \parallel (SCD)$}
		{$MK \parallel (SBD)$}
		{$MK \parallel (ABCD)$}
		{\True $MK \parallel (SAB)$}
		\loigiai{
			\immini{
				Ta có
				\begin{itemize}
					\item $MK \cap (SCD)=K$ nên $MK \parallel (SCD)$ sai.
					\item $MK \subset (SBD)$ nên $MK \parallel (SBD)$ sai.
					\item $MK \cap (ABCD)=M$ nên $MK \parallel (ABCD)$ sai.
					\item $M$ là trọng tâm tam giác $ABC$, do đó $BM=\dfrac{2}{3}BO=\dfrac{1}{3}BD$\\
					Suy ra $\dfrac{DK}{DS}=\dfrac{DM}{DB}=\dfrac{2}{3} \Rightarrow MK \parallel SB$ mà $SB \subset (SAB)$.\\
					Vậy $MK \parallel (SAB)$.
				\end{itemize}
			}{
				\begin{tikzpicture}[>=stealth,line join=round,line cap=round,font=\footnotesize,scale=1]	
					\path
					(0,0) coordinate (A)
					(4,0) coordinate (D)
					(-1.5,-1) coordinate (B)
					($(B)-(A)+(D)$) coordinate (C)
					(1,3) coordinate (S)
					($(B)!0.5!(C)$) coordinate (I)
					($(S)!1/3!(D)$) coordinate (K)
					(intersection of B--D and A--C) coordinate (O)
					(intersection of B--D and A--I) coordinate (M);
					\draw (C)--(D)--(S)--(C)--(B)--(S);
					\draw[dashed] (S)--(A)--(D)--(B)--(A)--(C) (A)--(I) (M)--(K);
					\foreach \d/\g in {A/30,B/-135,C/-45,D/0,S/90,O/-90,I/-90,M/155,K/30} \fill (\d)node[shift={(\g:0.3)}]{$\d$} circle(1pt);
				\end{tikzpicture}
			}
			
		}
	\end{ex}

	\begin{ex}%[Chuyên đề Toán 11 - 2023]%[Quan Ón, dự án WTB - Toán 11 - new 2023]%[1K4BB-2]
		Cho hình chóp $S.ABCD$ có đáy $ABCD$ là hình thang, đáy lớn $AB$. Gọi $P,Q$ lần lượt là hai điểm nằm trên cạnh $SA$ và $SB$ sao cho $\dfrac{SP}{SA}=\dfrac{SQ}{SB}=\dfrac{1}{3}$. Khẳng định nào sau đây là đúng?
		\choice
		{$PQ$ cắt $(ABCD)$}
		{$PQ \subset (ABCD)$}
		{\True $PQ \parallel (ABCD)$}
		{$PQ$ và $CD$ chéo nhau}
		\loigiai{
			\immini{
				Vì $\dfrac{SP}{SA}=\dfrac{SQ}{SB}=\dfrac{1}{3}$ nên $PQ \parallel AB$.\\
				Ta có $\heva{&PQ \parallel AB\\&AB \subset (ABCD)\\&PQ\not\subset(ABCD)} \Rightarrow PQ \parallel (ABCD)$.
			}{
				\begin{tikzpicture}[>=stealth,line join=round,line cap=round,font=\footnotesize,scale=1]	
					\path
					(0,0) coordinate (A)
					(4,0) coordinate (B)
					(1.4,-1) coordinate (D)
					(3.2,-1) coordinate (C)
					(0.3,2.5) coordinate (S)
					($(S)!1/3!(A)$) coordinate (P)
					($(S)!1/3!(B)$) coordinate (Q);
					\draw (C)--(D)--(S)--(C)--(B)--(S)--(A)--(D);
					\draw[dashed] (A)--(B) (P)--(Q);
					\foreach \d/\g in {A/180,B/0,C/-45,D/-135,S/90,P/180,Q/0} \fill (\d)node[shift={(\g:0.3)}]{$\d$} circle(1pt);
				\end{tikzpicture}
			}
		}
	\end{ex}

	\begin{ex}%[Chuyên đề Toán 11 - 2023]%[Quan Ón, dự án WTB - Toán 11 - new 2023]%[1K4BB-2]
		Cho tứ diện $ABCD$. Gọi $G_1$ và $G_2$ lần lượt là trọng tâm các tam giác $BCD$ và $ACD$. Khẳng định nào sau đây \textbf{sai}?
		\choice
		{$G_1G_2 \parallel (ABD)$}
		{$G_1G_2 \parallel (ABC)$}
		{$BG_1$, $AG_2$ và $CD$ đồng quy}
		{\True $G_1G_2=\dfrac{2}{3}AB$}
		\loigiai{
			\immini{
				Gọi $M$ là trung điểm $CD$.\\
				Vì $G_1$ và $G_2$ lần lượt là trọng tâm các tam giác $BCD$ và $ACD$ nên 
				$$\heva{&G_1 \in BM;\dfrac{MG_1}{MB}=\dfrac{1}{3}\\&G_2 \in AM;\dfrac{MG_2}{MA}=\dfrac{1}{3}.}$$
				Xét tam giác $ABM$, ta có $\dfrac{MG_1}{MB }= \dfrac{MG_2}{MA} = \dfrac{1}{3} \Rightarrow G_1G_2 \parallel AB$.\\
				Suy ra $ \dfrac{G_1G_2}{AB}=\dfrac{MG_1}{MB}=\dfrac{1}{3} \Rightarrow G_1G_2=\dfrac{1}{3}AB$.\\
				Vậy $G_1G_2=\dfrac{2}{3}AB$ sai.\\
				Hơn nữa, ta có
				\begin{itemize}
					\item Vì $G_1G_2 \parallel AB \Rightarrow G_1G_2 \parallel (ABD)$.
					\item Vì $G_1G_2 \parallel AB \Rightarrow G_1G_2 \parallel (ABC)$.
					\item Ba đường $BG_1$, $AG_2$ và $CD$ đồng quy tại $M$.
				\end{itemize}
			}{
				\begin{tikzpicture}[>=stealth,line join=round,line cap=round,font=\footnotesize,scale=1]	
					\path
					(1,2) coordinate (A)
					(0,0) coordinate (B)
					(4,0) coordinate (D)
					(1.7,-1.5) coordinate (C)
					($(C)!0.5!(D)$) coordinate (M)
					($(M)!1/3!(B)$) coordinate (G1)
					($(M)!1/3!(A)$) coordinate (G2);
					\draw (C)--(D)--(A)--(C)--(B)--(A)--(M);
					\draw[dashed] (M)--(B)--(D) (G1)--(G2);
					\foreach \d/\g in {A/60,B/-135,C/-90,D/-45,M/-45} \fill (\d)node[shift={(\g:0.3)}]{$\d$} circle(1pt);
					\fill (G1)node[shift={(-80:0.3)}]{$G_1$} circle(1pt);
					\fill (G2)node[shift={(40:0.3)}]{$G_2$} circle(1pt);
				\end{tikzpicture}
			}
		}
	\end{ex}

	\begin{ex}%[Chuyên đề Toán 11 - 2023]%[Quan Ón, dự án WTB - Toán 11 - new 2023]%[1K4BB-2]
		Cho hình chóp $S.ABCD$ có đáy $ABCD$ là hình bình hành. $M$, $N$, $K$ lần lượt là trung điểm của $DC$, $BC$, $SA$. Gọi $H$ là giao điểm của $AC$ và $MN$. Trong các khẳng định sau, khẳng định nào \textbf{sai}?
		\choice
		{$MN$ chéo $SC$}
		{$MN \parallel (SBD)$}
		{\True $MN \parallel (ABCD)$}
		{$MN \cap (SAC)=H$}
		\loigiai{
			\begin{center}
				\begin{tikzpicture}[>=stealth,line join=round,line cap=round,font=\footnotesize,scale=1]	
					\path
					(0,0) coordinate (A)
					(4,0) coordinate (D)
					(-1.5,-1) coordinate (B)
					($(B)-(A)+(D)$) coordinate (C)
					(1,2) coordinate (S)
					($(D)!0.5!(C)$) coordinate (M)
					($(B)!0.5!(C)$) coordinate (N)
					($(S)!0.5!(A)$) coordinate (K)
					(intersection of M--N and A--C) coordinate (H);
					\draw (C)--(D)--(S)--(C)--(B)--(S);
					\draw[dashed] (S)--(A)--(D)--(B)--(A)--(C) (M)--(N);
					\foreach \d/\g in {A/170,B/-135,C/-45,D/0,S/90,N/-90,M/-45,K/0} \fill (\d)node[shift={(\g:0.3)}]{$\d$} circle(1pt);
					\fill (H)node[shift={(60:0.25)}]{$H$} circle(1pt);
				\end{tikzpicture}
			\end{center}
			Vì $MN \subset (ABCD)$ nên $MN$ không song song với mặt phẳng $(ABCD)$ do đó $MN \parallel (ABCD)$ sai.
		}
	\end{ex}

	\begin{ex}%[Chuyên đề Toán 11 - 2023]%[Quan Ón, dự án WTB - Toán 11 - new 2023]%[1K4KB-2]
		Cho hai hình bình hành $ABCD$ và $ABEF$ không cùng nằm trong một mặt phẳng. Gọi $O_1$, $O_2$ lần lượt là tâm của $ABCD$, $ABEF$. $M$ là trung điểm của $CD$. Chọn khẳng định \textbf{sai} trong các khẳng định sau
		\choice
		{\True $MO_2$ cắt $(BEC)$}
		{$O_1O_2$ song song với $(BEC)$}
		{$O_1O_2$ song song với $(EFM)$}
		{$O_1O_2$ song song với $(AFD)$}
		\loigiai{
			\begin{center}
				\begin{tikzpicture}[>=stealth,line join=round,line cap=round,font=\footnotesize,scale=1]	
					\path
					(0,0) coordinate (A)
					(4,0) coordinate (B)
					(1.5,1.5) coordinate (D)
					($(B)-(A)+(D)$) coordinate (C)
					(1.5,-1.5) coordinate (F)
					($(B)-(A)+(F)$) coordinate (E)
					($(D)!0.5!(C)$) coordinate (M)
					($(A)!2!(M)$) coordinate (x)
					($(B)!1.5!(C)$) coordinate (y)
					($(A)!0.5!(C)$) coordinate (O1)
					($(A)!0.5!(E)$) coordinate (O2)
					(intersection of A--x and B--y) coordinate (J);
					\draw (F)--(A)--(D)--(M)--(J)--(E)--(F)--(D) (E)--(M)--(F);
					\draw[dashed] (F)--(B)--(A)--(M)--(C)--(J) (E)--(B)--(C)--(E)--(A)--(C) (M)--(O1)--(O2)--(M) (D)--(B);
					\foreach \d/\g in {A/180,B/-90,C/110,D/120,E/-45,F/-135,M/90,J/30} \fill (\d)node[shift={(\g:0.3)}]{$\d$} circle(1pt);
					\fill (O1)node[shift={(-125:0.3)}]{$O_1$} circle(1pt);
					\fill (O2)node[shift={(-75:0.3)}]{$O_2$} circle(1pt);
				\end{tikzpicture}
			\end{center}
			Gọi $J$ là giao điểm của $AM$ và $BC$.\\
			Ta có $MO_1 \parallel AD \parallel BC \Rightarrow MO_1 \parallel CJ$.\\
			Mà $O_1$ là trung điểm của $AC$ nên $M$ là trung điểm của $AJ$.\\
			Suy ra $MO_2$ là đường trung bình của tam giác $AJE$.\\
			Do đó $MO_2 \parallel EJ$ mà $EJ \subset (BEC)$.\\
			Từ đó suy ra $MO_2 \parallel (BEC)$.\\
			Vậy $MO_2$ không cắt $(BEC)$.
		}
	\end{ex}

	\begin{ex}%[Chuyên đề Toán 11 - 2023]%[Quan Ón, dự án WTB - Toán 11 - new 2023]%[1K4BB-2]
		Cho hình chóp $S.ABCD$ có đáy $ABCD$ là hình chữ nhật. Gọi $M$, $N$ theo thứ tự là trọng tâm $\triangle SAB$, $\triangle SCD$. Khi đó $MN$ song song với mặt phẳng
		\choice
		{$(SAC)$}
		{$(SBD)$}
		{$(SAB)$}
		{\True $(ABCD)$}
		\loigiai{
			\immini{
				Gọi $E$ và $F$ lần lượt là trung điểm của $AB$ và $CD$.\\
				Do $M$, $N$ là trọng tâm tam giác $SAB$, $SCD$ nên $S$, $M$, $E$ thẳng hàng và $S$, $N$, $F$ thẳng hàng.\\
				Xét $\triangle SEF$, ta có $\dfrac{SM}{SE} = \dfrac{SN}{SF} = \dfrac{2}{3}$ nên $MN \parallel EF$.\\
				Mà $EF \subset (ABCD)$ nên $MN \parallel (ABCD)$.
			}{
				\begin{tikzpicture}[>=stealth,line join=round,line cap=round,font=\footnotesize,scale=1]	
					\path
					(0,0) coordinate (A)
					(4,0) coordinate (D)
					(-1.5,-1) coordinate (B)
					($(B)-(A)+(D)$) coordinate (C)
					(-0.3,2.5) coordinate (S)
					($(A)!0.5!(B)$) coordinate (E)
					($(C)!0.5!(D)$) coordinate (F)
					($(S)!2/3!(E)$) coordinate (M)
					($(S)!2/3!(F)$) coordinate (N);
					\draw (C)--(D)--(S)--(C)--(B)--(S)--(F);
					\draw[dashed] (S)--(A)--(D)--(B)--(A)--(C) (M)--(N) (S)--(E)--(F)--(S);
					\foreach \d/\g in {A/170,B/-135,C/-45,D/0,S/90,N/30,M/30,E/160,F/-45} \fill (\d)node[shift={(\g:0.3)}]{$\d$} circle(1pt);
				\end{tikzpicture}
			}
		}
	\end{ex}

	\begin{ex}%[Chuyên đề Toán 11 - 2023]%[Quan Ón, dự án WTB - Toán 11 - new 2023]%[1K4BB-2]
		Cho hình chóp $S.ABCD$ có đáy là hình bình hành. Các điểm $I$, $J$ lần lượt là trọng tâm các tam giác $SAB$, $SAD$. $M$ là trung điểm $CD$. Chọn mệnh đề đúng trong các mệnh đề sau
		\choice
		{$IJ \parallel (SCD)$}
		{$IJ \parallel (SBM)$}
		{$IJ \parallel (SBC)$}
		{\True $IJ \parallel (SBD)$}
		\loigiai{
			\immini{
				Gọi $N,P$ lần lượt là trung điểm của cạnh $AB,AD$.\\
				Xét $\triangle SNP$ có $\dfrac{SI}{SN}=\dfrac{SJ}{SP}=\dfrac{2}{3} \Rightarrow IJ \parallel NP$.\\
				Xét $\triangle ABD$ có $NP$ là đường trung bình trong tam giác $\Rightarrow NP \parallel BD$.\\
				Suy ra $IJ \parallel BD$.\\
				Ta có $\heva{&IJ \not\subset (SBD)\\&IJ \parallel BD\\&BD \subset (SBD)} \Rightarrow IJ \parallel (SBD)$.
			}{
				\begin{tikzpicture}[>=stealth,line join=round,line cap=round,font=\footnotesize,scale=1]	
					\path
					(0,0) coordinate (A)
					(4,0) coordinate (D)
					(-1.5,-1) coordinate (B)
					($(B)-(A)+(D)$) coordinate (C)
					(-0.3,2.5) coordinate (S)
					($(A)!0.5!(B)$) coordinate (N)
					($(A)!0.5!(D)$) coordinate (P)
					($(C)!0.5!(D)$) coordinate (M)
					($(S)!2/3!(N)$) coordinate (I)
					($(S)!2/3!(P)$) coordinate (J);
					\draw (C)--(D)--(S)--(C)--(B)--(S);
					\draw[dashed] (S)--(A)--(D)--(B)--(A) (I)--(J) (S)--(N)--(P)--(S);
					\foreach \d/\g in {A/45,B/-135,C/-45,D/0,S/90,N/170,M/-45,I/45,J/45,P/45} \fill (\d)node[shift={(\g:0.3)}]{$\d$} circle(1pt);
				\end{tikzpicture}
			}
		}
	\end{ex}

	\begin{ex}%[Chuyên đề Toán 11 - 2023]%[Quan Ón, dự án WTB - Toán 11 - new 2023]%[1K4BB-2]
		Cho hình chóp $S.ABCD$ có đáy $ABCD$ là hình bình hành tâm $O$, $M$ là trung điểm $SA$. Khẳng định nào sau đây là đúng?
		\choice
		{\True $OM \parallel (SCD)$}
		{$OM \parallel (SBD)$}
		{$OM \parallel (SAB)$}
		{$OM \parallel (SAD)$}
		\loigiai{
			\immini{
				Ta có $M$ là trung điểm $SA$ và $O$ là trung điểm $AC$ $\Rightarrow OM$ là đường trung bình $\triangle SAC \Rightarrow OM \parallel SC$.\\
				Như vậy 
				$\heva{&OM \not\subset (SCD)\\&OM \parallel SC\\&SC \subset (SCD)} \Rightarrow MO \parallel (SCD)$.
			}{
				\begin{tikzpicture}[>=stealth,line join=round,line cap=round,font=\footnotesize,scale=1]	
					\path
					(0,0) coordinate (A)
					(4,0) coordinate (D)
					(-1.5,-1) coordinate (B)
					($(B)-(A)+(D)$) coordinate (C)
					(0.3,2.5) coordinate (S)
					($(S)!0.5!(A)$) coordinate (M)
					(intersection of B--D and A--C) coordinate (O);
					\draw (C)--(D)--(S)--(C)--(B)--(S);
					\draw[dashed] (S)--(A)--(D)--(B)--(A)--(C) (M)--(O);
					\foreach \d/\g in {A/180,B/-135,C/-45,D/0,S/90,O/-90,M/10} \fill (\d)node[shift={(\g:0.3)}]{$\d$} circle(1pt);
				\end{tikzpicture}
			}
		}
	\end{ex}

	\begin{ex}%[Chuyên đề Toán 11 - 2023]%[Quan Ón, dự án WTB - Toán 11 - new 2023]%[1K4BB-2]
		Cho hình chóp $S.ABCD$ có đáy là hình thang, $AB \parallel CD$ và $AB = 2CD$. Lấy $E$ thuộc cạnh $SA$, $F$ thuộc cạnh $SC$ sao cho $\dfrac{SE}{SA}=\dfrac{SF}{SC}=\dfrac{2}{3}$. Khẳng định nào dưới đây đúng?
		\choice
		{Đường thẳng $EF$ song song với mặt phẳng $(SAC)$}
		{Đường thẳng $EF$ cắt đường thẳng $AC$}
		{\True Đường thẳng $AC$ song song với mặt phẳng $(BEF)$}
		{Đường thẳng $CD$ song song với mặt phẳng $(BEF)$}
		\loigiai{
			\begin{center}
				\begin{tikzpicture}[>=stealth,line join=round,line cap=round,font=\footnotesize,scale=1]	
					\path
					(0,0) coordinate (B)
					(4,0) coordinate (A)
					(1.4,-1) coordinate (C)
					(3.2,-1) coordinate (D)
					(1.3,2.5) coordinate (S)
					($(S)!2/3!(A)$) coordinate (E)
					($(S)!2/3!(C)$) coordinate (F);
					\draw (C)--(D)--(S)--(C)--(B)--(S)--(A)--(D) (F)--(B);
					\draw[dashed] (B)--(A)--(C) (B)--(E)--(F);
					\foreach \d/\g in {A/0,B/180,C/-135,D/-45,S/60,E/30,F/-135} \fill (\d)node[shift={(\g:0.3)}]{$\d$} circle(1pt);
				\end{tikzpicture}
			\end{center}
			Vì $\dfrac{SE}{SA} = \dfrac{SF}{SC} = \dfrac{2}{3}$ nên $EF \parallel AC$.\\
			Mà $EF \subset (BEF)$, $AC \not\subset (BEF)$ nên $AC$ song song với mặt phẳng $(BEF)$.
		}
	\end{ex}

	\begin{ex}%[Chuyên đề Toán 11 - 2023]%[Quan Ón, dự án WTB - Toán 11 - new 2023]%[1K4BB-2]
		Cho tứ diện $ABCD$. Gọi $G$ là trọng tâm tam giác $ABD$. $M$ là điểm trên cạnh $BC$ sao cho $MB = 2MC$. Khi đó đường thẳng $MG$ song song với mặt phẳng nào dưới đây?
		\choice
		{\True $(ACD)$}
		{$(BCD)$}
		{$(ABD)$}
		{$(ABC)$}
		\loigiai{
			\immini{
				Ta có $MB = 2MC \Rightarrow \dfrac{BM}{BC} = \dfrac{2}{3}$.\\
				Gọi $E$ là trung điểm $AD$.\\
				Khi đó, vì $G$ là trọng tâm tam giác $ABD$ nên $\dfrac{BG}{BE} = \dfrac{2}{3}$.\\
				Xét tam giác $BEC$, ta có $\dfrac{BM}{BC} = \dfrac{BG}{BE} = \dfrac{2}{3} \Rightarrow GM \parallel EC$.\\
				Hơn nữa, ta có $EC \subset (ACD)$ nên $GM \parallel (ACD)$.
			}{
				\begin{tikzpicture}[>=stealth,line join=round,line cap=round,font=\footnotesize,scale=1]	
					\path
					(1,2) coordinate (A)
					(0,0) coordinate (B)
					(4,0) coordinate (D)
					(1.7,-1.5) coordinate (C)
					($(A)!0.5!(D)$) coordinate (E)
					($(B)!2/3!(E)$) coordinate (G)
					($(B)!2/3!(C)$) coordinate (M);
					\draw (C)--(D)--(A)--(C)--(B)--(A) (C)--(E);
					\draw[dashed] (E)--(B)--(D) (G)--(M);
					\foreach \d/\g in {A/60,B/-135,C/-90,D/-45,M/-110,E/30} \fill (\d)node[shift={(\g:0.3)}]{$\d$} circle(1pt);
					\fill (G)node[shift={(80:0.3)}]{$G$} circle(1pt);
				\end{tikzpicture}
			}
		}
	\end{ex}

	\begin{ex}%[Chuyên đề Toán 11 - 2023]%[Quan Ón, dự án WTB - Toán 11 - new 2023]%[1K4BB-2]
		Cho hình chóp $S.ABCD$ có đáy là hình bình hành. Gọi $M$, $N$ lần lượt là trung điểm của $SC$ và $SD$. Mệnh đề nào sau đây là đúng?
		\choice
		{$MN \parallel (SBD)$}
		{\True $MN \parallel (SAB)$}
		{$MN \parallel (SAC)$}
		{$MN \parallel (SCD)$}
		\loigiai{
			\begin{center}
				\begin{tikzpicture}[>=stealth,line join=round,line cap=round,font=\footnotesize,scale=1]	
					\path
					(0,0) coordinate (A)
					(4,0) coordinate (D)
					(-1.5,-1) coordinate (B)
					($(B)-(A)+(D)$) coordinate (C)
					(0.3,2.5) coordinate (S)
					($(S)!0.5!(C)$) coordinate (M)
					($(S)!0.5!(D)$) coordinate (N);
					\draw (C)--(D)--(S)--(C)--(B)--(S) (M)--(N);
					\draw[dashed] (S)--(A)--(D)--(B)--(A)--(C);
					\foreach \d/\g in {A/180,B/-135,C/-45,D/0,S/90,M/-135,N/20} \fill (\d)node[shift={(\g:0.3)}]{$\d$} circle(1pt);
				\end{tikzpicture}
			\end{center}
			Ta có $M$, $N$ lần lượt là trung điểm của $SC$ và $SD$ nên $MN$ là đường trung bình trong $\triangle SCD \Rightarrow MN \parallel CD$.\\
			Vì $ABCD$ là hình bình hành nên $CD \parallel AB$.\\
			Do đó $MN \parallel AB$.\\
			Mà $AB \subset (SAB)$ nên $MN \parallel (SAB)$.
		}
	\end{ex}

	\begin{ex}%[Chuyên đề Toán 11 - 2023]%[Quan Ón, dự án WTB - Toán 11 - new 2023]%[1K4BB-2]
		Cho lăng trụ $ABC.A'B'C'$. Gọi $M$, $N$ lần lượt là trung điểm của $A'B'$ và $CC'$. Khi đó $CB'$ song song với
		\choice
		{\True $(AC'M)$}
		{$(BC'M)$}
		{$A'N$}
		{$AM$}
		\loigiai{
			\immini{
				Gọi $G$ là giao điểm của $AC'$ và $A'C$ $\Rightarrow G$ là trung điểm của $A'C$.\\
				Hơn nữa, ta có $M$ là trung điểm của $AB'$ nên $MG$ là đường trung bình của tam giác $A'CB'$.\\
				Do đó $CB' \parallel MG$ mà $MG \subset (AC'M)$ $\Rightarrow CB' \parallel (AC'M)$.
			}{
				\begin{tikzpicture}[scale=1, font=\footnotesize, line join=round, line cap=round, >=stealth]
					\path
					(0,0) coordinate (A')
					(1.97,-1.47) coordinate (B')
					(4,0) coordinate (C')
					(1,4) coordinate (A)
					(2.97,2.53) coordinate (B)
					(5,4) coordinate (C)
					($(A')!0.5!(B')$) coordinate (M)
					($(C)!0.5!(C')$) coordinate (N)
					($(A)!0.5!(C')$) coordinate (G);
					\draw (C)--(A)--(B)--(C)--(C')--(B')--(B) (A')--(B')--(C) (A')--(A)--(M);
					\draw[dashed] (A)--(C')--(A')--(C) (M)--(G);
					\foreach \d/\g in {A/180,B/90,C/0,A'/-135,B'/-90,C'/-45,M/-90,N/0,G/90} \fill (\d)node[shift={(\g:0.3)}]{$\d$} circle(1pt);
				\end{tikzpicture}
			}
		}
	\end{ex}

	\begin{ex}%[Chuyên đề Toán 11 - 2023]%[Quan Ón, dự án WTB - Toán 11 - new 2023]%[1K4KB-2]
		Cho hình chóp $S.ABCD$ có đáy $ABCD$ là hình thang với đáy lớn $AD$, $AD=2BC$. Gọi $M$ là điểm thuộc cạnh $SD$ sao cho $MD=2MS$. Gọi $O$ là giao điểm của $AC$ và $BD$. Khi đó, $OM$ song song với mặt phẳng
		\choice
		{$(SAD)$}
		{$(SBD)$}
		{\True $(SBC)$}
		{$(SAB)$}
		\loigiai{
			\immini{
				Ta có $ABCD$ là hình thang với đáy lớn $AD$, $AD=2BC$ nên\\
				$AD \parallel BC$, $AC \cap BD = O \Rightarrow \dfrac{OC}{OA}=\dfrac{OB}{OD}=\dfrac{BC}{AD}=\dfrac{1}{2} \Rightarrow \dfrac{DO}{DB}=\dfrac{2}{3}$.\\
				Mặt khác, vì $MD = 2MS$ nên $\dfrac{DM}{DS}=\dfrac{2}{3}$.\\
				$\Rightarrow \dfrac{DO}{DB} = \dfrac{DM}{DS} \Rightarrow OM \parallel SB$.\\
				Mà $SB \subset (SBC)$, $OM \not\subset (SBC)$.\\
				Nên $OM \parallel (SBC)$.
			}{
				\begin{tikzpicture}[>=stealth,line join=round,line cap=round,font=\footnotesize,scale=1]	
					\path
					(0,0) coordinate (A)
					(4,0) coordinate (D)
					(0.8,-1) coordinate (B)
					(2.8,-1) coordinate (C)
					(0.8,2.5) coordinate (S)
					($(S)!1/3!(D)$) coordinate (M)
					(intersection of B--D and A--C) coordinate (O);
					\draw (C)--(D)--(S)--(C)--(B)--(S)--(A)--(B);
					\draw[dashed] (C)--(A)--(D)--(B) (O)--(M);
					\foreach \d/\g in {A/180,B/-135,C/-45,D/0,S/90,M/30} \fill (\d)node[shift={(\g:0.3)}]{$\d$} circle(1pt);
					\fill (O)node[shift={(-90:0.2)}]{$O$} circle(1pt);
				\end{tikzpicture}
			}
		}
	\end{ex}

	\begin{ex}%[Chuyên đề Toán 11 - 2023]%[Quan Ón, dự án WTB - Toán 11 - new 2023]%[1K4KB-2]
		Cho hình hộp $ABCD.A'B'C'D'$ có tất cả các mặt là hình vuông cạnh $a$. Các điểm $M$, $N$ lần lượt nằm trên $AD'$, $DB$ sao cho $AM = DN = x$ $\left( 0 < x < a\sqrt{2} \right)$. Khi $x$ thay đổi, đường thẳng $MN$ luôn song song với mặt phẳng cố định nào sau đây?
		\choice
		{$(CB'D')$}
		{\True $(A'BC)$}
		{$(AD'C)$}
		{$(BA'C')$}
		\loigiai{
			\begin{center}
				\begin{tikzpicture}[>=stealth,line join=round,line cap=round,font=\footnotesize,scale=0.8]
					% Day ABCD
					%\tkzDefPoints{0/0/A, -1/-1.5/B, 4/0/D, 0/3/A'}
					\path
					(0,0) coordinate (A)
					(4,0) coordinate (D)
					(-1,-1.5) coordinate (B)
					($(B)+(D)-(A)$) coordinate (C)
					(0,3) coordinate (A');
					% Day A’B’C’D’
					\coordinate (B') at ($(B)+(A')-(A)$);
					\coordinate (D') at ($(D)+(A')-(A)$);
					\coordinate (C') at ($(B')+(D')-(A')$);
					\coordinate (M) at ($(A)!0.5!(C)$);
					\path
					($(A)!0.4!(D')$) coordinate (M)
					($(D)!0.4!(B)$) coordinate (N);
					\draw (D')--(C)--(C')--(D')--(A') (C')--(B')--(A') (B')--(B)--(C)--(D)--(D');
					\draw[dashed] (A)--(D') (A)--(D) (A)--(A') (A)--(B)--(A')--(C) (B)--(D);
					\draw[dashed, color= red] (M)--(N);
					\foreach \d/\g in {A/180,B/-135,C/-45,D/0,A'/90,B'/180,C'/0,D'/90,M/-30,N/-90} \fill (\d)node[shift={(\g:0.3)}]{$\d$} circle(1pt);
				\end{tikzpicture}
			\end{center}
			\textbf{Cách 1: Sử dụng định lí Ta-lét thuận}\\
			Vì $AD \parallel A'D'$ nên tồn tại $(P)$ là mặt phẳng qua $AD$ và song song với mp $(A'D'CB)$.\\
			$(Q)$ là mặt phẳng qua $M$ và song song với mp $(A'D'CB)$.\\
			Giả sử $(Q)$ cắt $DB$ tại $N'$.\\
			Theo định lí Ta-lét ta có $\dfrac{AM}{AD'} = \dfrac{DN'}{DB}$. $\hfill (*)$.\\
			Mà các mặt của hình hộp là hình vuông cạnh $a$ nên $AD'= DB = a\sqrt{2}$.\\
			Từ $(*)$ ta có $AM = DN'$ $\Rightarrow DN' = DN$ $\Rightarrow N' \equiv N$ $\Rightarrow MN \subset (Q)$.\\
			$(Q) \parallel (A'D'CB)$ suy ra $MN$ luôn song song với mặt phẳng cố định $(A'D'CB)$ hay $(A'BC)$.\\
			\textbf{Cách 2: Sử dụng định lí Ta-lét đảo}\\
			Từ giả thiết ta có $\dfrac{AM}{DN} = \dfrac{MD'}{NB} = \dfrac{AD'}{DB}$.\\
			Suy ra $AD$, $MN$ và $D'B$ luôn song song với một mặt phẳng.\\
			Vậy $MN$ luôn song song với một mặt phẳng $(P)$, mà $(P)$ song song với $AD$ và $D'B$.\\
			Mặt phẳng này chính là mp $(A'D'CB)$ hay $(A'BC)$.
		}
	\end{ex}

	\begin{ex}%[Chuyên đề Toán 11 - 2023]%[Quan Ón, dự án WTB - Toán 11 - new 2023]%[1K4KB-2]
		Cho hình hộp $ABCD. A'B'C'D'$. Trên các cạnh $AA'$, $BB'$, $CC'$ lần lượt lấy ba điểm $M$, $N$, $P$ sao cho $\dfrac{A'M}{AA'} = \dfrac{1}{3}$; $\dfrac{B'N}{BB'} = \dfrac{2}{3}$; $\dfrac{C'P}{CC'} = \dfrac{1}{2}$. Biết mặt phẳng $(MNP)$ cắt cạnh $DD'$ tại $Q$. Tính tỉ số $\dfrac{D'Q}{DD'}$.
		\choice
		{\True $\dfrac{1}{6}$}
		{$\dfrac{1}{3}$}
		{$\dfrac{5}{6}$}
		{$\dfrac{2}{3}$}
		\loigiai{
			\immini{
				Gọi độ dài cạnh bên của hình hộp là $a$.\\
				Giao tuyến của mặt phẳng $(MNP)$ với $(CDD'C')$ là đường thẳng đi qua $P$ và song song với $MN$\\
				Gọi $P'$ là trung điểm $BB'$ và $Q' \in AA'$, $MN \parallel P'Q'$. \\
				Khi đó tứ giác $MNP'Q'$ là hình bình hành và $NP'=\dfrac{2}{3}a-\dfrac{1}{2}a=\dfrac{1}{6}a$.\\
				Do đó
				$MQ'=\dfrac{1}{6}a$.\\
				Suy ra $Q'A' = MA' - MQ' = \dfrac{1}{6}a$.\\
				Vậy $\dfrac{A'Q'}{AA'} = \dfrac{D'Q}{DD'} = \dfrac{1}{6}$.
			}{
				\begin{tikzpicture}[>=stealth,line join=round,line cap=round,font=\footnotesize,scale=1]
					% Day A'B'C'D'
					%\tkzDefPoints{0/0/A, -1/-1.5/B, 4/0/D, 0/3/A'}
					\path
					(0,0) coordinate (B')
					(4,0) coordinate (C')
					(-1,-1) coordinate (A')
					($(C')-(B')+(A')$) coordinate (D')
					(-1,3) coordinate (A);
					% Day A’B’C’D’
					\coordinate (B) at ($(B')+(A)-(A')$);
					\coordinate (D) at ($(D')+(A)-(A')$);
					\coordinate (C) at ($(B')+(D)-(A')$);
					\coordinate (P) at ($(C)!0.5!(C')$);
					\coordinate (P') at ($(B)!0.5!(B')$);
					\path
					($(A')!1/3!(A)$) coordinate (M)
					($(B)!1/3!(B')$) coordinate (N)
					($(D')!1/6!(D)$) coordinate (Q)
					($(A')!1/6!(A)$) coordinate (Q');
					\draw (A)--(B)--(C)--(D)--(A)--(A')--(D')--(C')--(C) (D')--(D) (M)--(Q)--(P);
					\draw[dashed] (A')--(B')--(C') (B)--(B') (M)--(N)--(P) (Q')--(P');
					\foreach \d/\g in {A/135,B/135,C/45,D/135,A'/-90,B'/-90,C'/-45,D'/-90,M/180,N/180,P/0,P'/0,Q/-15,Q'/180} \fill (\d)node[shift={(\g:0.3)}]{$\d$} circle(1pt);
				\end{tikzpicture}
			}
		}
	\end{ex}
\Closesolutionfile{ans}
\begin{indapan}{10}
	{ans/ans1-C4B12-6}
\end{indapan}