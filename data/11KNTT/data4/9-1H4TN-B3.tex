\subsection{BÀI TẬP TRẮC NGHIỆM}
\Opensolutionfile{ans}[ans/1H4.B3]
\setcounter{ex}{0}

\begin{ex} Trong không gian cho mặt phẳng $(\alpha)$ và $A$ không thuộc $(\alpha)$. Qua điểm $A$ có thể dựng được bao nhiêu đường thẳng song song với $(\alpha)$?
	\choice
	{Duy nhất}
	{\True Vô số }
	{$2$}
	{$4$}
	\loigiai{}
\end{ex}
\begin{ex}%[1H2B3-1]
	Trong không gian cho đường thẳng $\Delta$ và điểm $O$ không nằm trong $\Delta$. Qua điểm $O$ cho trước, có bao nhiêu mặt phẳng song song với đường thẳng $\Delta$?
	\choice
	{\True Vô số}
	{ $3$}
	{ $1$}
	{ $2$}
	\loigiai{
		Gọi $d$ là đường thẳng qua $O$ và song song với $\Delta$. Khi đó có vô số mặt phẳng chứa $d$ và không chứa $\Delta$. Vậy có vô số mặt phẳng qua $O$ và song song với $\Delta$. }
\end{ex}

\begin{ex}%[1H2B3-1]
	Có bao nhiêu mặt phẳng song song với cả hai đường thẳng chéo nhau?
	\choice
	{\True Vô số}
	{$1 $}
	{$2 $}
	{$3 $}
	\loigiai{}
\end{ex}

\begin{ex}%[1H2B3-1]
	Cho hai đường thẳng phân biệt $a$, $b$ và mặt phẳng $\left(\alpha \right)$. Giả sử $a\parallel \left(\alpha \right), b\subset \left(\alpha \right)$. Khi đó
	\choice
	{$a\parallel b $}
	{$a,b$ chéo nhau}
	{$a,b$ cắt nhau}
	{\True $a\parallel b$ hoặc $a,b$ chéo nhau}
	\loigiai{
		\begin{center}
			\begin{tikzpicture}[scale=1]
				\tkzDefPoints{0/0/A', 4/0/B', 1/2/D', 1/1/A, 4/1/B, 0/2.5/M}
				\coordinate (C') at ($(B')+(D')-(A')$);
				\tkzDefPointBy[translation = from A' to B'](M) \tkzGetPoint{N}
				\tkzLabelSegment[pos=.3](M,N){ $a$}
				\tkzLabelSegment[pos=.3](A,B){ $b$}
				%\tkzDrawSegments[dashed](A,A' A',B' A',C')
				\tkzDrawPolygon(A',B',C',D')
				\tkzDrawSegments(A,B M,N)
				%\tkzLabelPoints[right](B)
				%        \tkzLabelPoints[above](A,B)
				\tkzMarkAngles[size=.7](B',A',D')
				\tkzLabelAngle[pos=0.4](D',A',B'){\footnotesize $\alpha$ }
			\end{tikzpicture}
			\hspace{2cm}
			\begin{tikzpicture}[scale=1]
				\tkzDefPoints{0/0/A', 4/0/B', 1/2/D', 1/1/A, 4/1/B, 0/2.5/M, 1/.5/P, 4/1.75/Q}
				\coordinate (C') at ($(B')+(D')-(A')$);
				\tkzDefPointBy[translation = from A' to B'](M) \tkzGetPoint{N}
				\tkzLabelSegment[pos=.3](M,N){ $a$}
				\tkzLabelSegment[pos=.9](A,B){ $c$}
				\tkzLabelSegment[pos=.1, below](P,Q){ $b$}
				%\tkzDrawSegments[dashed](A,A' A',B' A',C')
				\tkzDrawPolygon(A',B',C',D')
				\tkzDrawSegments(A,B M,N P,Q)
				%\tkzLabelPoints[right](B)
				%        \tkzLabelPoints[above](A,B)
				\tkzMarkAngles[size=.7](B',A',D')
				\tkzLabelAngle[pos=0.4](D',A',B'){\footnotesize $\alpha$ }
			\end{tikzpicture}
		\end{center}
		Vì $a\parallel \left(\alpha \right)$ nên tồn tại đường thẳng $c\subset \left(\alpha \right)$ thỏa mãn $a\parallel c $. Suy ra $b,c$ đồng phẳng và xảy ra các trường hợp sau:
		\begin{itemize}
			\item  Nếu $b$ song song hoặc trùng với $c$ thì $a\parallel b$.
			\item  Nếu $b$ cắt $c$ thì $b$ cắt $\left(\beta \right)\equiv \left({a,c}\right)$ nên $a,b$ không đồng phẳng. Do đó $a,b$ chéo nhau.
		\end{itemize}
	}
\end{ex}

\begin{ex}%[1H2B3-1]
	Cho hai đường thẳng phân biệt $a$, $b$ và mặt phẳng $\left(\alpha \right)$. Giả sử $a\parallel b$ và  $b\parallel \left(\alpha \right)$. Khẳng định nào sau đây là khẳng định đúng?
	\choice
	{$a\parallel \left(\alpha \right) $}
	{$a\subset \left(\alpha \right) $}
	{\True $a\parallel \left(\alpha \right)$ hoặc $a\subset \left(\alpha \right) $}
	{$a$ cắt $\left(\alpha \right) $}
	\loigiai{
	}
\end{ex}

\begin{ex}%[1H2B3-1]
	Cho đường thẳng $a$ nằm trong mặt phẳng $\left(\alpha \right)$ và đường thẳng $b$ không thuộc $\left(\alpha \right)$. Mệnh đề nào sau đây đúng?
	\choice
	{Nếu $b\parallel \left(\alpha \right)$ thì $b\parallel a $}
	{\True Nếu $b\parallel a$ thì $b\parallel \left(\alpha \right) $}
	{Nếu $b$ cắt $\left(\alpha \right)$ và $\left(\beta \right)$ chứa $b$ thì giao tuyến của $\left(\alpha \right)$ và $\left(\beta \right)$ là đường thẳng cắt cả $a$ và $b $.
	}
	{Nếu $b$ cắt $\left(\alpha \right)$ thì $b$ cắt $a $}
	\loigiai{
		\begin{itemize}
			\item  A sai. Nếu $b\parallel \left(\alpha \right)$ thì $b\parallel a$ hoặc $a,b$ chéo nhau.
			\item B sai. Nếu $b$ cắt $\left(\alpha \right)$ thì $b$ cắt $a$ hoặc $a,b$ chéo nhau.
			\item  D sai. Nếu $b$ cắt $\left(\alpha \right)$ và $\left(\beta \right)$ chứa $b$ thì giao tuyến của $\left(\alpha \right)$ và $\left(\beta \right)$ là đường thẳng cắt $a$ hoặc song song với $a$.
		\end{itemize}
	}
\end{ex}


\begin{ex}%[1H2B3-1]
	Cho hai đường thẳng chéo nhau $a$ và $b$. Khẳng định nào sau đây \textbf{sai}?
	\choice
	{\True Có duy nhất một mặt phẳng song song với $a$ và $b $}
	{Có vô số đường thẳng song song với $a$ và cắt $b $}
	{Có duy nhất một mặt phẳng qua $a$ và song song với $b $}
	{Có duy nhất một mặt phẳng qua điểm $M$, song song với $a$ và $b$ (với $M$ là điểm cho trước)}
	\loigiai{
		Có có vô số mặt phẳng song song với 2 đường thẳng chéo nhau.
		}
\end{ex}


\begin{ex}%[1H2B3-1]
	Cho $d\parallel \left(\alpha \right)$, mặt phẳng $\left(\beta \right)$ qua $d$ cắt $\left(\alpha \right)$ theo giao tuyến $d'$. Khẳng định nào sau đây là đúng?
	\choice
	{$d$ cắt $d'$}
	{\True $d\parallel d' $}
	{$d$ và $d'$ chéo nhau}
	{$d\equiv d' $}
	\loigiai{
		Ta có $d'=\left(\alpha \right)\cap \left(\beta \right)$. Do $d$ và $d'$ cùng thuộc $\left(\beta \right)$ nên $d$ cắt $d'$ hoặc $d\parallel d'$.
		Nếu $d$ cắt $d'$. Khi đó, $d$ cắt $\left(\alpha \right)$ (mâu thuẫn với giả thiết).
		Vậy $d\parallel d'$.}
\end{ex}

\begin{ex}%[1H2Y3-2]
	Cho hình chóp tứ giác $S.ABCD$. Gọi $M$ và $N$ lần lượt là trung điểm của $SA$ và $SC$. Khẳng định nào sau đây đúng?
	\choice  
	{\True $MN \parallel (ABCD)$}
	{$MN \parallel (SAB)$} 
	{$MN \parallel (SCD)$}
	{$MN \parallel (SBC)$}
	\loigiai{
		\immini{Xét tam giác $SAC$ có $M, N$ lần lượt là trung điểm của $SA, SC$.\\ Suy ra $MN \parallel AC$ nên $MN \parallel (ABCD)$.}{\begin{tikzpicture}[scale=0.8, line join=round, line cap=round,thick]
				\tikzset{label style/.style={font=\footnotesize}}
				\coordinate (S) at (1,2.5);
				\coordinate (A) at (0,0);
				\coordinate (B) at (1,-0.9);
				\coordinate (D) at (3.5,0);
				\coordinate (C) at (2.5,-1.3);
				\coordinate (M) at ($(S)!0.5!(A)$);
				\coordinate (N) at ($(S)!0.5!(C)$);
				%\draw[name path=MN,dashed] (M)--(N);
				\tkzDrawSegments(S,A S,B S,C S,D A,B B,C C,D)
				\tkzDrawSegments[dashed](A,D M,N A,C)
				\tkzDrawPoints[fill=black,size=2pt](S,A,B,C,D,M,N)
				\tkzLabelPoints[above](S)
				\tkzLabelPoints[left](A,M)
				\tkzLabelPoints[below](B,C)
				\tkzLabelPoints[right](D,N)
		\end{tikzpicture}}
		
		
		
	}
\end{ex}

\begin{ex}%[1H2B3-2]
	Cho hình chóp $S.ABCD$ có đáy $ABCD$ là hình bình hành, $M$ và $N$ là hai điểm trên $SA,SB$ sao cho $\dfrac{SM}{SA}=\dfrac{SN}{SB}=\dfrac{1}{3} $. Vị trí tương đối giữa $MN$ và $\left({ABCD}\right)$ là
	\choice
	{$MN$ và $\left({ABCD}\right)$ chéo nhau}
	{\True $MN$ song song $\left({ABCD}\right)$}
	{$MN$ nằm trong $\left({ABCD}\right) $}
	{$MN$ cắt $\left({ABCD}\right) $}
	\loigiai{
		Theo định lí Talet, ta có $\dfrac{SM}{SA}=\dfrac{SN}{SB}$ suy ra $MN$ song song với $AB. $ \\
		Mà $AB$ nằm trong mặt phẳng $\left({ABCD}\right)$ suy ra $MN \parallel \left({ABCD}\right) $.}
\end{ex}

\begin{ex}%[1H2Y3-3]
	\immini[thm]{Cho hình chóp $S.ABCD$ có đáy $ABCD$ là hình bình hành. Tìm giao tuyến của hai mặt phẳng $(SAD)$ và $(SBC)$.
		\choice
		{Là đường thẳng đi qua đỉnh $S$ và song song với đường thẳng $BD$}
		{Là đường thẳng đi qua đỉnh $S$ và tâm $O$ của đáy}
		{\True Là đường thẳng đi qua đỉnh $S$ và song song với đường thẳng $BC$}
		{Là đường thẳng đi qua đỉnh $S$ và song song với đường thẳng $AB$}
	}
	{\begin{tikzpicture}[scale=0.3,>=stealth]
			\tkzDefPoints{0/0/A, 8/0/B,-3/-3/D, 2/6/S}
			\tkzDefPointBy[translation = from A to B](D)\tkzGetPoint{C}
			\tkzDrawPoints(A,B,C,D,S)
			\tkzLabelPoints[below](A,B,C,D)
			\tkzLabelPoints[above](S)
			\tkzDrawSegments (S,B S,C S,D D,C C,B)
			\tkzDrawSegments[dashed](D,A A,B S,A)
		\end{tikzpicture}
	}
	\loigiai{
		\immini{Do hai mặt phẳng $(SAD)$ và $(SBC)$ có chung điểm $S$ và có hai đường thẳng $AD$, $BC$ song song với nhau nên giao tuyến của hai mặt phẳng $(SAD)$ và $(SBC)$ là đường thẳng đi qua đỉnh $S$ và song song với đường thẳng $BC$.
		}{
			\begin{tikzpicture}[scale=0.3,>=stealth]
				\tkzDefPoints{0/0/A, 8/0/B,-3/-3/D, 2/6/S}
				\tkzDefPointBy[translation = from A to B](D)\tkzGetPoint{C}
				\tkzDefPointBy[translation = from A to D](S)\tkzGetPoint{s}
				\tkzDefPointBy[homothety = center S ratio -0.7](s)\tkzGetPoint{s1}
				\tkzDrawPoints(A,B,C,D,S)
				\tkzLabelPoints[below](A,B,C,D)
				\tkzLabelPoints[above](S)
				\tkzDrawSegments (S,B S,C S,D D,C C,B s,s1)
				\tkzDrawSegments[dashed](D,A A,B S,A)
			\end{tikzpicture}
	}}
\end{ex}

\begin{ex}%[Hà Lê]%[1H2B3]
	Cho tứ diện $ABCD$ có $I$, $J$ lần lượt là trung điểm của $BC$, $BD$. Giao tuyến của mặt phẳng $(AIJ)$ và $(ACD)$ là
	\haicot
	{đường thẳng $d$ đi qua $A$ và song song với $BC$}
	{đường thẳng $d$ đi qua $A$ và song song với $BD$}
	{\True đường thẳng $d$ đi qua $A$ và song song với $CD$}
	{đường thẳng $AB$}
\end{ex}

\begin{ex}%[1H2B3-2]
	Cho tứ diện $ABCD$. Gọi $G$ là trọng tâm của tam giác $ABD,\ Q$ thuộc cạnh $AB$ sao cho $AQ=2QB$, $P$ là trung điểm của $AB $, $M$ là trung điểm của $BD$. Khẳng định nào sau đây đúng?
	\choice
	{$Q \in \left({CDP}\right) $}
	{$QG$ cắt $\left({BCD}\right) $}
	{$MP \parallel \left({BCD}\right) $}
	{\True $GQ \parallel \left({BCD}\right) $}
	\loigiai{
		\immini{
			Vì $G$ là trọng tâm tam giác $ABD$ $\Rightarrow \dfrac{AG}{AM}=\dfrac{2}{3} $.\\
			Điểm $Q\in AB$ sao cho $AQ=2QB\Leftrightarrow \dfrac{AQ}{AB}=\dfrac{2}{3} $. Suy ra $\dfrac{AG}{AM}=\dfrac{AQ}{AB}\xrightarrow{}GQ \parallel BD. $\\
			Mặt khác $BD$ nằm trong mặt phẳng $\left({BCD}\right)$ suy ra $GQ \parallel \left({BCD}\right) $.
		}{
			\begin{tikzpicture}[scale=1]
			\tkzDefPoints{1/3/A, 0/0/B, 2/-2/C, 5/0/D}
			\tkzCentroid(A,B,D)\tkzGetPoint{G}
			\coordinate (M) at ($(B)!.5!(D)$);
			\coordinate (Q) at ($(A)!.67!(B)$);
			\coordinate (P) at ($(A)!.5!(B)$);
			\tkzDrawSegments[dashed](A,M B,D Q,G P,D)
			\tkzDrawPolygon(A,B,C,D)
			\tkzDrawSegments(A,C)
			\tkzLabelPoints[left](B,P,Q)
			\tkzLabelPoints[right](D)
			\tkzLabelPoints[above](A)
			\tkzLabelPoints[below](M,C)
			\tkzLabelPoints[above right](G)
			\tkzDrawPoints(A,B,C,D,M,P,Q,G)
			\end{tikzpicture}
		}
	}
\end{ex}


\begin{ex}%[1H2K3-2]
	Cho hai hình bình hành $ABCD$ và $ABEF$ không cùng nằm trong một mặt phẳng. Gọi $O$, $O_1$ lần lượt là tâm của $ABCD$, $ABEF $; $M$ là trung điểm của $CD. $ Khẳng định nào sau đây \textbf{sai}?
	\choice
	{$OO_1 \parallel \left({BEC}\right) $}
	{$OO_1 \parallel \left({EFM}\right) $}
	{\True $MO_1$ cắt $\left({BEC}\right) $}
	{$OO_1 \parallel \left({AFD}\right) $}
	\loigiai{
		\immini{
			Xét tam giác $ACE$ có $O,O_1$ lần lượt là trung điểm của $AC,AE $.\\
			Suy ra $OO_1$ là đường trung bình trong tam giác $ACE\Rightarrow OO_1 \parallel EC. $\\
			Tương tự, $OO_1$ là đường trung bình của tam giác $BFD$ nên $OO_1 \parallel FD. $\\
			Vậy $OO_1 \parallel \left({BEC}\right), OO_1 \parallel \left({AFD}\right)$ và $OO_1 \parallel \left({EFC}\right)$. Chú ý rằng: $\left({EFC}\right)\equiv \left({EFM}\right) $.
		}{
			\begin{tikzpicture}[scale=1]
				\tkzDefPoints{0/0/A, 4/0/B, 1/2/D, 1.5/-2.5/F}
				\coordinate (C) at ($(D)+(B)-(A)$);
				\coordinate (E) at ($(F)+(B)-(A)$);
				\coordinate (M) at ($(C)!0.5!(D)$);
				\tkzInterLL(A,C)(B,D)\tkzGetPoint{O}
				\tkzInterLL(A,E)(B,F)\tkzGetPoint{O_1}
				\tkzDrawSegments[dashed](B,D A,C A,E B,F O,O_1 A,B B,E B,C)
				\tkzDrawPolygon(D,C,E,F)
				\tkzDrawSegments(A,D A,F)
				\tkzLabelPoints[left](A)
				\tkzLabelPoints[right](B)
				\tkzLabelPoints[below](E,F,O_1)
				\tkzLabelPoints[above](C,D,O,M)
				\tkzDrawPoints(A,B,C,D,O,E,F,O_1,M)
			\end{tikzpicture}
		}
	}
\end{ex}

\begin{ex}%[HK1 THPT Quốc Thái, An Giang 2018]%[Lê Nguyễn Viết Tường-DA 11HK1-18]%[1H2K3-4]%
	Cho hình chóp $S.ABCD$ có đáy là hình bình hành. Thiết diện của hình chóp khi cắt bởi mặt phẳng đi qua trung điểm $M$ của cạnh $AB$ và song song với $BD,SA$ là hình gì?
	\choice
	{Ngũ giác}
	{Hình thang}
	{Tam giác}
	{\True Hình bình hành}
	\loigiai{
		\immini{
			Gọi $(\alpha)$ là mặt phẳng đi qua $M$ và song song với $BD,SA$.
			Ta có
			$BD\parallel (\alpha),BD\subset (ABCD),(\alpha)\cap (ABCD)=Mx$
			$\Rightarrow Mx\parallel BD\Rightarrow Mx$ cắt $AD$ tại $N$ trong $(ABCD)$.
			$SA\parallel (\alpha),SA\subset (SAD),(\alpha)\cap (SAD)=Ny$
			$\Rightarrow Ny\parallel SA\Rightarrow Ny$ cắt $SD$ tại $P$ trong $(SAD)$.
			$SA\parallel (\alpha),SA\subset (SAB),(\alpha)\cap (SAB)=Mt$
			$\Rightarrow Mt\parallel SA\Rightarrow Mt$ cắt $SB$ tại $Q$ trong $(SAB)$.
			Vậy thiết diện là hình bình hành $MNPQ$.
		}{
			\begin{tikzpicture}[line join = round, line cap = round]
				\tkzDefPoints{0/0/D,1.5/1.6/A,5.5/1.6/B,4/0/C,2.5/6/S}
				\tkzDefMidPoint(A,B)\tkzGetPoint{M}
				\tkzDefMidPoint(A,D)\tkzGetPoint{N}
				\tkzDefMidPoint(S,D)\tkzGetPoint{P}
				\tkzDefMidPoint(S,B)\tkzGetPoint{Q}
				\tkzLabelPoints[](M,N)
				\tkzLabelPoints[above](S)
				\tkzLabelPoints[above right](A)
				\tkzLabelPoints[right](B,C,Q)
				\tkzLabelPoints[left](D,P)
				\tkzDrawPoints[fill=black](S,A,B,C,D,M,N,P,Q)
				\tkzDrawSegments[](S,B S,C S,D B,C C,D)
				\tkzDrawSegments[dashed](S,A A,B A,D M,N N,P B,D M,Q Q,P)
			\end{tikzpicture}
		}
	}
\end{ex}
% \centerline{---HẾT---}
\Closesolutionfile{ans}

