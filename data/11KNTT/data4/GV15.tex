\begin{dang}{Xác định thiết diện bằng cách kẻ song song}
	Để xác định được thiết diện của một hình chóp hoặc hình lăng trụ cắt bởi mặt phẳng $(P)$ ta thường dựng các đoạn giao tuyến của $(P)$ với các mặt của hình chóp hoặc hình lăng trụ đó. Khi ấy, ta sử dụng định lí sau:
	\begin{dl}
	 Cho hai mặt phẳng song song. Nếu một mặt phẳng cắt mặt phẳng này thì cũng cắt mặt phẳng kia theo hai giao tuyến song song với nhau.
	\end{dl}
\end{dang}
\subsubsection{Ví dụ mẫu}
\begin{vd}%[1K4BC-5]%[DCHT Toán 11 - KNTT -Nguyễn Đức Lợi]
	Cho hình chóp $S.ABC$. Gọi $M$ là trung điểm $SA$. Xác định thiết diện của hình chóp cắt bởi mặt phẳng $(P)$ đi qua $M$ và song song với mặt phẳng $(ABC)$.
	\loigiai{
	\immini{
	Vì $\heva{
			& (P)\parallel (ABC)\\
			& (SAB) \cap (ABC)=AB\\
			& M \text{ là một điểm chung của } (P) \text{ và } (SAB)}$ nên giao tuyến của $(P)$ và $(SAB)$ là đường thẳng đi qua $M$, song song với $AB$, cắt $SB$ tại trung điểm $N$ của $SB$.\\
	Tương tự, giao tuyến của $(P)$ và $(SAC)$ là đường thẳng đi qua $M$, song song với $AC$, cắt $SC$ tại trung điểm $P$ của $SC$.\\
	Vậy thiết diện của hình chóp $S.ABC$ cắt bởi mặt phẳng $(P)$ là tam giác $MNP$.
	}
	{
	\begin{tikzpicture}[line join=round, line cap=round,thick,scale=0.7]
	\coordinate (S) at (0,4);
	\coordinate (A) at (-2,0);
	\coordinate (B) at (0,-2);
	\coordinate (C) at (3,0);
	\coordinate (M) at ($(S)!0.5!(A)$);
	\coordinate (N) at ($(S)!0.5!(B)$);
	\coordinate (P) at ($(S)!0.5!(C)$);
	\fill[fill=gray!20](M)--(N)--(P)--cycle;
	\draw(S)--(A) (S)--(B) (S)--(C) (B)--(C) (A)--(B) (M)--(N)--(P);
	\draw[dashed,thin](A)--(C) (M)--(P);
	\foreach \i/\g in {S/90,A/180,B/-90,C/0, M/180, N/-30, P/45}{\draw[fill=white](\i) circle (1.5pt) ($(\i)+(\g:3mm)$) node[scale=1]{$\i$};}
	\end{tikzpicture}
	}
	}
\end{vd}
%================================================================
\begin{vd}%[1K4BC-5]%[DCHT Toán 11 - KNTT -Nguyễn Đức Lợi]
	Cho hình hộp $ABCD.A'B'C'D'$. Gọi $M$ là trung điểm của $A'B'$. Tìm thiết diện của hình hộp cắt bởi mặt phẳng $(P)$ đi qua $M$ và song song với mặt phẳng $(A'C'C)$. Thiết diện là hình gì?
	\loigiai{
	\immini{
	Vì $(P)$ song song với $(A'C'C)$ nên $(P)$ song song với $(A'C'CA)$.\\
	Suy ra giao tuyến của $(P)$ với $(A'B'C'D')$ là đường thẳng đi qua $M$, song song với $A'C'$ và cắt $B'C'$ tại trung điểm $N$ của $B'C'$.\\
	Tương tự, giao tuyến của $(P)$ với $(B'C'CB)$ là đường thẳng đi qua $N$, song song với $C'C$ và cắt $BC$ tại trung điểm $P$ của $BC$.\\
	Giao tuyến của $(P)$ với $(A'B'BA)$ là đường thẳng đi qua $M$, song song với $A'A$ và cắt $AB$ tại trung điểm $Q$ của $AB$.\\
	Vậy thiết diện cần tìm là tứ giác $MNPQ$.\\
	}
	{
	\begin{tikzpicture}[line join=round, line cap=round,thick,scale=1]
	\coordinate (A) at (0,0);
	\coordinate (B) at (-2,-2);
	\coordinate (D) at (5,0);
	\coordinate (C) at ($(B)+(D)-(A)$);
	\coordinate (A') at ($(A)+(0,3)$);
	\coordinate (B') at ($(B)+(0,3)$);
	\coordinate (C') at ($(C)+(0,3)$);
	\coordinate (D') at ($(D)+(0,3)$);
	\coordinate (M) at ($(A')!0.5!(B')$);
	\coordinate (N) at ($(C')!0.5!(B')$);
	\coordinate (P) at ($(C)!0.5!(B)$);
	\coordinate (Q) at ($(A)!0.5!(B)$);
	\fill[fill=gray!20](M)--(N)--(P)--(Q)--cycle;
	\draw(A')--(B')--(C')--(D')--(A') (B')--(B) (D')--(D) (C')--(C) (B)--(C)--(D) (M)--(N)--(P) (A')--(C');
	\draw[dashed,thin](A')--(A)--(C) (D)--(A)--(B) (A)--(D) (P)--(Q)--(M);
	\foreach \i/\g in {A/-90,B/-90,C/-90,D/0,A'/90,B'/180,C'/90,D'/0, M/90, N/45, P/-90, Q/180}{\draw[fill=white](\i) circle (1.5pt) ($(\i)+(\g:3mm)$) node[scale=1]{$\i$};}
	\end{tikzpicture}
	}
	\noindent
	Do $NP$ song song và bằng $CC'$; $MQ$ song song và bằng $AA'$ nên tứ giác $MNPQ$ là hình bình hành.
	}
\end{vd}
%================================================================
\begin{vd}%[1K4KC-5]%[DCHT Toán 11 - KNTT -Nguyễn Đức Lợi]
	Cho hình hộp $ABCD.A'B'C'D'$. Hai điểm $M, N$ lần lượt nằm trên hai cạnh $AD, CC'$ sao cho $\dfrac{AM}{MD}=\dfrac{CN}{NC'}$. Xác định thiết diện của hình hộp cắt bởi mặt phẳng qua $MN$ và song song với $(ACB')$
	\loigiai{
	\begin{center}
	\begin{tikzpicture}[line join=round, line cap=round,thick,scale=1]
	\coordinate (A) at (0,0);
	\coordinate (B) at (-2,-2);
	\coordinate (D) at (8,0);
	\coordinate (C) at ($(B)+(D)-(A)$);
	\coordinate (A') at ($(A)+(0,4)$);
	\coordinate (B') at ($(B)+(0,4)$);
	\coordinate (C') at ($(C)+(0,4)$);
	\coordinate (D') at ($(D)+(0,4)$);
	\coordinate (M) at ($(A)!3/5!(D)$);
	\coordinate (N) at ($(C)!3/5!(C')$);
	\coordinate (I) at ($(A)!3/5!(A')$);
	\coordinate (E) at ($(C)!3/5!(D)$);
	\coordinate (F) at ($(B')!3/5!(C')$);
	\coordinate (K) at ($(B')!3/5!(A')$);
	\fill[fill=gray!20](M)--(E)--(N)--(F)--(K)--(I)--cycle;
	\draw(A')--(B')--(C')--(D')--(A') (C)--(B')--(B) (D')--(D) (C')--(C) (B)--(C)--(D) (E)--(N)--(F)--(K);
	\draw[dashed,thin](D)--(A')--(A)--(C) (D)--(A)--(B) (B')--(A)--(D) (K)--(I)--(M)--(E);
	\foreach \i/\g in {A/-90,B/-90,C/-90,D/0,A'/90,B'/180,C'/90,D'/0, M/-110, N/0,I/180, E/0, F/90, K/180}{\draw[fill=white](\i) circle (1.5pt) ($(\i)+(\g:3mm)$) node[scale=1]{$\i$};}
	\end{tikzpicture}
	\end{center}
	Gọi $I$ là điểm trên $AA'$ sao cho $\dfrac{AI}{IA'}=\dfrac{AM}{MD}$ suy ra $IM \parallel A'D$ suy ra $IM \parallel CB'$\\
	Lại có $\dfrac{AI}{IA'}=\dfrac{CN}{NC'}$ suy ra $IN \parallel AC$ suy ra $(MNI) \parallel (ACB')$. Do đó, mặt phẳng $(MNI)$ là mặt phẳng đi qua $M$, $N$ và song song vưới mặt phẳng $(ACB')$.\\
	Qua $M$ kẻ $ME \parallel AC$; qua $N$ kẻ $NF \parallel B'C'$, qua $F$ kẻ $FK \parallel A'C'$.\\
	Khi đó, thiết diện cần tìm là đa giác $MENFKI$.
	}
\end{vd}
%================================================================
\begin{vd}%[1K4KC-5]%[DCHT Toán 11 - KNTT -Nguyễn Đức Lợi]
	Cho hình chóp $S.ABCD$ có $\triangle SAD$ đều. Đáy $ABCD$ là hình thang có $AD\parallel BC$, $AB=BC=CD=1$, $AD=2$. Gọi $(P)$ là mặt phẳng qua điểm $M$ nằm trên cạnh $AB$ và song song với $(SAD)$ . Đặt $BM=x$, $(0<x<1)$. Tìm $x$ để thiết diện của $S.ABCD$ cắt bởi $(P)$ có diện tích bằng một nửa diện tích tam giác $SAD$.
	\loigiai{
	\immini{
	Qua $M$ kẻ đường thẳng song song với $AD$, cắt $CD$ tại $N$.\\
	Qua $M$ kẻ đường thẳng song song với $SA$, cắt $SB$ tại $Q$.\\
	Qua $Q$ kẻ đường thẳng song song với $BC$, cắt $SC$ tại $P$.\\
	Suy ra $(MNPQ)$ là mặt phẳng qua $M$ và song song với $(SAD)$. \\
	Do đó, thiết diện của hình chóp $S.ABCD$ cắt bởi mặt phẳng $(P)$ là tứ giác $MNPQ$.\\
	Bởi vì $MN$ và $PQ$ cùng song song với $BC$ (hoặc $AD$) nên tứ giác $MNPQ$ là hình thang.\\
	Tam giác $SAD$ đều, cạnh $AD=2$ nên $SA=SD=2$.\\
	Áp dụng định lí Thales trong các tam giác $SAB$, $SBC$, $SCD$ ta được
	}
	{
	\begin{tikzpicture}[scale=0.9,line join=round, line cap=round,thick]
	\coordinate (S) at (3,4);
	\coordinate (A) at (0,0);
	\coordinate (B) at (1,-2);
	\coordinate (C) at (4,-2);
	\coordinate (D) at (5,0);
	\coordinate (M) at ($(A)!3/5!(B)$);
	\coordinate (N) at ($(D)!3/5!(C)$);
	\coordinate (P) at ($(S)!3/5!(C)$);
	\coordinate (Q) at ($(S)!3/5!(B)$);
	\coordinate (I) at (intersection of A--C and M--N);
	\fill[fill=gray!20](M)--(N)--(P)--(Q)--cycle;
	\draw(S)--(A) (S)--(B) (S)--(C) (S)--(D) (B)--(C) (A)--(B)--(C)--(D) (M)--(Q)--(P)--(N);
	\draw[dashed,thin](A)--(D) (M)--(N) (A)--(C);
	\foreach \i/\g in {S/90,A/180,B/-90,C/-90,D/0, M/180, N/0, P/60, Q/150, I/-90}{\draw[fill=white](\i) circle (1.5pt) ($(\i)+(\g:3mm)$) node[scale=1]{$\i$};}
\end{tikzpicture}
	}
	\noindent
	$x=\dfrac{BM}{BA}=\dfrac{MQ}{SA}=\dfrac{BQ}{BS}=\dfrac{CP}{CS}=\dfrac{CN}{CD}=\dfrac{NP}{SD}$ và $\dfrac{PQ}{BC}=\dfrac{SQ}{SB}=\dfrac{AM}{AB}=1-x$.\\
	Suy ra $MQ=NP=2x$, $CN=x$ và $PQ=1-x$.
	\immini{Tiếp tục áp dụng định lí Thales trong các tam giác $ABC$, $ACD$ ta được\\
	$1-x=\dfrac{AM}{BC}=\dfrac{MI}{BC}$, $x=\dfrac{CN}{CD}=\dfrac{IN}{AD}$, $\Rightarrow MI=1-x$ và $IN=2x$\\
	$ \Rightarrow MN=x+1$.\\
	Gọi $H$, $K$ lần lượt là hình chiếu vuông góc của $P$, $Q$ trên $MN$.\\
	Ta có $MK=HN=\dfrac{MN-PQ}{2}=\dfrac{(x+1)-(1-x)}{2}=x$.
	}{
	\begin{tikzpicture}[scale=0.8,line join=round, line cap=round,thick]
	\coordinate (Q) at (1,3);
	\coordinate (P) at (4,3);
	\coordinate (M) at (0,0);
	\coordinate (N) at (5,0);
	\draw (Q)--(K) (P)--(H);
	\draw(M)--(N)--(P)--(Q)--cycle;
	\coordinate (K) at ($(M)!(Q)!(N)$);
	\coordinate (H) at ($(M)!(P)!(N)$);
	\foreach \i/\g in {Q/90,P/90,N/-90,M/-90,K/-90,H/-90}
	{\draw[fill=white](\i) circle (1.5pt) ($(\i)+(\g:3mm)$) node[scale=1]{$\i$};}
\end{tikzpicture}
	}
	\noindent
	Suy ra $QK=\sqrt{QM^2-MK^2}=\sqrt{4x^2-x^2}=x\sqrt{3}$\\	$\Rightarrow S_{MNPQ}=\dfrac{(PQ+MN)\cdot QK}{2} = \dfrac{(1-x+ 1+x)\cdot \sqrt{3}x}{2} = \sqrt{3}x$.\\
	Từ giả thiết ta có $S_{MNPQ} = \dfrac{1}{2}S_{SBD} \Leftrightarrow \sqrt{3}x = \dfrac{1}{2}\dfrac{(2x)^2 \sqrt{3}}{4}  \Leftrightarrow \sqrt{3}x = \dfrac{x^2 \sqrt{3}}{2}$	$\Leftrightarrow x = \dfrac{\sqrt{3}}{2}$ (vì $0 < x < 1$).\\
	Vậy $x=\dfrac{\sqrt{3}}{2}$.
	}
\end{vd}
%================================================================
\begin{vd}%[1K4KC-5]%[DCHT Toán 11 - KNTT -Nguyễn Đức Lợi]
	Cho hình chóp $S.ABCD$ đáy là hình bình hành tâm $O$ có $AC=a$, $BD=b$. Tam giác $SBD$ là tam giác đều. Một mặt phẳng $(\alpha)$ di động song song với mặt phẳng $(SBD)$ và đi qua điểm $I$ trên đoạn $AC$. 
	\begin{enumerate}
		\item Xác định thiết diện của hình chóp với mặt phẳng $(\alpha)$. 
		\item Tính diện tích thiết diện theo $a,b$ và $x=AI$.
	\end{enumerate}
	\loigiai{
		\begin{enumerate}
			\item 
			Xác định thiết diện của hình chóp với mặt phẳng $(\alpha)$. 
			\immini{
			Nếu $I \in OA$ thì ta có $$\left\{ \begin{array}{ll} (\alpha)\parallel (SBD) \\
			(ABD) \cap (SBD)=BD\\
			I \in (\alpha) \cap (ABD)
			\end{array}  \Rightarrow (\alpha) \cap (ABD)=MN\right. .$$
			với $MN$ qua $I$ và $MN \parallel BD$.\\
			Tương tự $(\alpha)$ cắt $(SAB)$ theo đoạn giao tuyến $MP$ song song với $SB$, cắt $(SAD)$ theo đoạn giao tuyến $NP \parallel SD$.\\
			Thiết diện là tam giác đều $MNP$ (vì đồng dạng với tam giác đều $SBD$)	
			}{
			\begin{tikzpicture}[line join=round, line cap=round,thick]
			\coordinate (A) at (0,0);
			\coordinate (B) at (2,-2);
			\coordinate (D) at (5,0);
			\coordinate (C) at ($(B)+(D)-(A)$);
			\coordinate (O) at ($(A)!1/2!(C)$);
			\coordinate (S) at ($(O)+(0.5,5)$);
			\coordinate (I) at ($(A)!1/6!(C)$);
			\coordinate (M) at ($(A)!1/3!(D)$);
			\coordinate (N) at ($(A)!1/3!(B)$);
			\coordinate (P) at ($(A)!1/3!(S)$);
			\coordinate (J) at ($(O)!1/2!(C)$);
			\coordinate (H) at ($(C)!1/2!(D)$);
			\coordinate (L) at ($(C)!1/2!(B)$);
			\coordinate (K) at ($(C)!1/2!(S)$);
			\fill[fill=gray!20](M)--(N)--(P)--cycle;
			\fill[fill=gray!20](K)--(H)--(L)--cycle;
			\draw(S)--(A) (S)--(B) (S)--(C) (A)--(B) (B)--(C) (N)--(P) (K)--(L);
			\draw[dashed,thin](A)--(C) (A)--(D) (C)--(D) (S)--(D) (S)--(O) (B)--(D) (P)--(M)--(N) (K)--(H)--(L);
			\foreach \i/\g in {S/90,A/180,B/-90,C/-90,D/140,O/-90, M/-70, N/-130, P/90, H/-90, L/-90, K/45}{\draw[fill=white](\i) circle (1.5pt) ($(\i)+(\g:3mm)$) node[scale=1]{$\i$};}
			\end{tikzpicture}
			}	
			\noindent
			Nếu $I \in OC$ thì thiết diện là tam giác đều $HKL$ có các cạnh tương ứng song song với cạnh tam giác $SBD$.
			\item Tính diện tích thiết diện theo $a,b$ và $x=AI$.\\
			Ta có $S_{\triangle BCD}=\dfrac{b^2\sqrt{3}}{4}$.\\
			Nếu $I$ nằm giữa $O$ và $A$ thì $0<x<\dfrac{a}{2}$.\\
			Khi đó $\dfrac{S_{\triangle MNP}}{S_{\triangle BCD}}=\left(\dfrac{MN}{BD}\right)^2$. Do $MN \parallel BD$, suy ra $\dfrac{MN}{BD}=\dfrac{AI}{AO}=\dfrac{2x}{a} \Rightarrow S_{\triangle MNP}=\dfrac{b^2x^2\sqrt{3}}{a^2}$.\\
			Nếu $I$ nằm giữa $O$ và $C$  thì $\dfrac{a}{2}<x<a$.\\
			Khi đó $\dfrac{S_{\triangle HKL}}{S_{\triangle BCD}}=\left(\dfrac{HL}{BD}\right)^2$. Do $HL \parallel BD\Rightarrow \dfrac{HL}{BD}=\dfrac{CI}{CO}=\dfrac{2(a-x)}{a} \Rightarrow S_{\triangle MNP}=\dfrac{b^2(a-x)^2\sqrt{3}}{a^2}$.			
		\end{enumerate}
		
	}
\end{vd}

\subsubsection{Bài tập rèn luyện}
\centerline{\fcolorbox{red}{yellow!50}{\bf {BÀI TẬP TỰ LUẬN }}}

\begin{bt}%[1K4BC-5]%[DCHT Toán 11 - KNTT -Nguyễn Đức Lợi]
	Cho hình hộp $ABCD.A'B'C'D'$. Gọi $K$ là trung điểm $CD$. Gọi $(P)$ là mặt phẳng đi qua $C'$ và song song với $(A'AK)$. Xác định thiết diện của hình hộp cắt bởi mặt phẳng $(P)$.
	\loigiai{
	\immini{
	Gọi $H$ là trung điểm $C'D' \Rightarrow KH \parallel AA'\Rightarrow H\in (A'AK)$.\\
	Qua $C'$ kẻ đường thẳng song song với $A'H$, cắt $A'B'$ tại $M$.\\
	Qua $C$ kẻ đường thẳng song song với $AK$, cắt $AB$ tại $N$.\\
	Suy ra mặt phẳng $(CC'MN)$ là mặt phẳng đi qua $C'$ và song song với $(A'AK)$.\\
	Thiết diện cần tìm là hình bình hành $CC'MN$.
	}{
	\begin{tikzpicture}[line join=round, line cap=round,thick,scale=0.8]
	\coordinate (A) at (0,0);
	\coordinate (B) at (-2,-2);
	\coordinate (D) at (5,0);
	\coordinate (C) at ($(B)+(D)-(A)$);
	\coordinate (A') at ($(A)+(0,3)$);
	\coordinate (B') at ($(B)+(0,3)$);
	\coordinate (C') at ($(C)+(0,3)$);
	\coordinate (D') at ($(D)+(0,3)$);
	\coordinate (M) at ($(A')!0.5!(B')$);
	\coordinate (K) at ($(C)!0.5!(D)$);
	\coordinate (N) at ($(A)!0.5!(B)$);
	\coordinate (H) at ($(C')!0.5!(D')$);
	\fill[fill=gray!20](M)--(C')--(C)--(N)--cycle;
	\draw(A')--(B')--(C')--(D')--(A') (B')--(B) (D')--(D) (C')--(C) (B)--(C)--(D) (M)--(C') (K)--(H)--(A');
	\draw[dashed,thin](A')--(A) (D)--(A)--(B) (A)--(D) (C)--(N)--(M) (A')--(K)--(A);
	
	\foreach \i/\g in {A/-90,B/-90,C/-90,D/0,A'/90,B'/180,C'/90,D'/0, M/120, K/-45, N/180,H/90}{\draw[fill=white](\i) circle (1.5pt) ($(\i)+(\g:3mm)$) node[scale=1]{$\i$};}
	\end{tikzpicture}
	}
	}
\end{bt}

\begin{bt}%[1K4BC-5]%[DCHT Toán 11 - KNTT -Nguyễn Đức Lợi]
	Cho hình chóp $S.ABCD$ có đáy là hình thang, đáy lớn $AD=3a$, $AB=BC=a$. Mặt bên $(SAD)$ là tam giác cân đỉnh $S$ với $SA=2a$, gọi $M$ là điểm thuộc cạnh $AB$ và không trùng với $A$, $B$. Mặt phẳng $(\alpha)$ đi qua $M$ và song song với $(SAD)$. Xác định thiết diện của chóp với mặt phẳng $(\alpha)$. Thiết diện là hình gì?
	\loigiai{
	\immini{
	Qua $M$ kẻ đường thẳng song song với $AD$, cắt $CD$ tại $N$.\\
	Qua $M$ kẻ đường thẳng song song với $SA$, cắt $SB$ tại $Q$.\\
	Qua $Q$ kẻ đường thẳng song song với $BC$, cắt $SC$ tại $P$.\\
	Suy ra $(MNPQ)$ là mặt phẳng qua $M$ và song song với $(SAD)$. \\
	Do đó, thiết diện của hình chóp $S.ABCD$ cắt bởi mặt phẳng $(P)$ là tứ giác $MNPQ$.\\
	Bởi vì $MN$ và $PQ$ cùng song song với $BC$ (hoặc $AD$) nên tứ giác $MNPQ$ là hình thang.\\
	Áp dụng định lí Thales trong các tam giác $SAB$, $SBC$, $SCD$ ta được
	$\dfrac{BM}{BA}=\dfrac{MQ}{SA}=\dfrac{BQ}{BS}=\dfrac{CP}{CS}=\dfrac{CN}{CD}=\dfrac{NP}{SD}$.\\
	Suy ra $\dfrac{MQ}{SA} = \dfrac{NP}{SD}$, mà tam giác $SAD$ cân tại $S$. Do đó $MQ=NP$.\\
	Vậy $MNPQ$ là hình thang cân.
	}{
	\begin{tikzpicture}[scale=0.9,line join=round, line cap=round,thick]
	\coordinate (S) at (3,4);
	\coordinate (A) at (0,0);
	\coordinate (B) at (1,-2);
	\coordinate (C) at (4,-2);
	\coordinate (D) at (5,0);
	\coordinate (M) at ($(A)!3/5!(B)$);
	\coordinate (N) at ($(D)!3/5!(C)$);
	\coordinate (P) at ($(S)!3/5!(C)$);
	\coordinate (Q) at ($(S)!3/5!(B)$);
	\fill[fill=gray!20](M)--(N)--(P)--(Q)--cycle;
	\draw(S)--(A) (S)--(B) (S)--(C) (S)--(D) (B)--(C) (A)--(B)--(C)--(D) (M)--(Q)--(P)--(N);
	\draw[dashed,thin](A)--(D) (M)--(N);
	\foreach \i/\g in {S/90,A/180,B/-90,C/-90,D/0, M/180, N/0, P/60, Q/150}{\draw[fill=white](\i) circle (1.5pt) ($(\i)+(\g:3mm)$) node[scale=1]{$\i$};}
	\end{tikzpicture}
	}
	}
\end{bt}

\begin{bt}%[1K4BC-5]%[DCHT Toán 11 - KNTT -Nguyễn Đức Lợi]
	Cho hình chóp $S.ABCD$ có đáy là hình thang $ABCD$ có $AD \parallel BC, AD=2BC$. Gọi $E$ là trung điểm $AD$ và $O$ giao điểm của $AC$ và $BE$; $I$ là một điểm di động trên cạnh $AC$ khác $A$ cà $C$. Qua $I$ vẽ mặt phẳng $(\alpha)$ song song với $(SBE)$. Tìm thiết diện tạo bởi $(\alpha)$ và hình chóp $S.ABCD$.
	\loigiai{
	\immini{
	\begin{itemize}
		\item Trường hợp 1: Điểm $I $ nằm giữa $A$ và $O$.\\
		Qua $I$ kẻ đường thẳng song song với $BE$, cắt $AB$, $AD$ lần lượt tại $M$ và $N$.\\
		Qua $M$ kẻ đường thẳng song song với $SB$, cắt $SA$ tại $P$.\\
		Suy ra $(MNP)$ là mặt phẳng đi qua $I$ và song song với $(SBE)$. Do đó, thiết diện của hình chóp cắt bởi mặt phẳng $(\alpha)$ là tam giác $MNP$.
		\item Trường hợp 2: Điểm $I $ nằm giữa $C$ và $O$.\\
		Qua $I$ kẻ đường thẳng song song với $BE$, cắt $BC$, $AD$ lần lượt tại $H$ và $U$.\\
		Qua $H$ kẻ đường thẳng song song với $SB$, cắt $SC$ tại $K$.\\
		Qua $U$ kẻ đường thẳng song song với $SE$, cắt $SD$ tại $V$.\\
		Suy ra $(HKVU)$ là mặt phẳng đi qua $I$ và song song với $(SBE)$. Do đó, thiết diện của hình chóp cắt bởi mặt phẳng $(\alpha)$ là tứ giác $HKVU$.
	\end{itemize}
	}{
	\begin{tikzpicture}[scale=0.9,line join=round, line cap=round,thick]
	\coordinate (S) at (2,4);
	\coordinate (A) at (0,0);
	\coordinate (B) at (1,-2);
	\coordinate (C) at (4,-2);
	\coordinate (D) at (5,0);
	\coordinate (E) at ($(A)!1/2!(D)$);
	\coordinate (O) at (intersection of A--C and B--E);
	\coordinate (I) at ($(A)!1/2!(O)$);
	\coordinate (M) at ($(A)!1/2!(B)$);
	\coordinate (N) at ($(A)!1/2!(E)$);
	\coordinate (P) at ($(A)!1/2!(S)$);
	\coordinate (J) at ($(C)!1/2!(O)$);
	\coordinate (H) at ($(C)!1/2!(B)$);
	\coordinate (K) at ($(C)!1/2!(S)$);
	\coordinate (U) at ($(D)!1/2!(E)$);
	\coordinate (V) at ($(D)!1/2!(S)$);
	\fill[fill=gray!20](M)--(N)--(P)--cycle;
	\fill[fill=gray!20](H)--(K)--(V)--(U)--cycle;
	\draw (S)--(A) (S)--(B) (S)--(C) (S)--(D) (B)--(C) (A)--(B)--(C)--(D) (P)--(M) (H)--(K)--(V);
	\draw[dashed,thin](A)--(D) (M)--(N) (A)--(C) (S)--(E)--(B) (M)--(N)--(P) (V)--(U)--(H);

	\foreach \i/\g in {S/90,A/180,B/-90,C/-90,D/0, E/-90, O/-90, M/180, N/-40, P/90, I/-90, H/-90, K/160, U/-45, V/45}{\draw[fill=white](\i) circle (1.5pt) ($(\i)+(\g:3mm)$) node[scale=1]{$\i$};}
\end{tikzpicture}
	}
	}
\end{bt}

\begin{bt}%[1K4BC-5]%[DCHT Toán 11 - KNTT -Nguyễn Đức Lợi]
	Cho tứ diện $ABCD$ có $G$ là trọng tâm của tam giác $BCD$. Gọi $O$ là trung điểm của đoạn thẳng $AG$. Thiết diện của tứ diện cắt bởi mặt phẳng đi qua $O$ và song song với mặt phẳng $(ABC)$ là tam giác $MNP$. Gọi $S_1$, $S_2$ lần lượt là diện tích của hai tam giác $MNP$ và $ABC$. Tính tỉ số $\dfrac{S_1}{S_2}$.
	\loigiai{
	\immini{
	Gọi $I$ là trung điểm $BC$. Trong mặt phẳng $(AID)$, qua $O$ kẻ đường thẳng song song với $AI$, cắt $AD$ tại $P$ và cắt $DI$ tại $J$.\\
	Qua $J$, kẻ đường thẳng song song với $MN$, cắt $CD$ tại $N$ và cắt $BD$ tại $M$.\\
	Thiết diện cần tìm là tam giác $MNP$.	\\
	Trong tam giác $AIG$ có $OJ \parallel AI$, hơn nữa $O$ là trung điểm $AG$ nên $J$ là trung điểm $IG$. Do đó $IJ = \dfrac{1}{2} IG = \dfrac{1}{6}ID\Rightarrow \dfrac{JD}{ID}=\dfrac{5}{6}$.\\
	Suy ra hai tam giác $MNP$ và $BCA$ đồng dạng theo tỉ số $k=\dfrac{MN}{BC}=\dfrac{JD}{ID}=\dfrac{5}{6}$.\\
	Vậy $\dfrac{S_1}{S_2}=k^2=\dfrac{25}{36}$.
	}
	{
	\begin{tikzpicture}[line join=round, line cap=round,thick]
	\coordinate (A) at (1.5,4);
	\coordinate (B) at (-3,0);
	\coordinate (C) at (0,-2);
	\coordinate (D) at (3,0);
	\coordinate (I) at ($(B)!1/2!(C)$);
	\coordinate (G) at ($(D)!2/3!(I)$);
	\coordinate (O) at ($(A)!1/2!(G)$);
	\coordinate (J) at ($(D)!5/6!(I)$);
	\coordinate (M) at ($(D)!5/6!(B)$);
	\coordinate (N) at ($(D)!5/6!(C)$);
	\coordinate (P) at ($(D)!5/6!(A)$);
	\fill[fill=gray!20](M)--(N)--(P)--cycle;
	\draw (I)--(A)--(B)--(C)--(D)--(A)--(C) (P)--(N);
	\draw[dashed,thin](B)--(D)--(I) (N)--(M)--(P)--(J) (A)--(G);
	\foreach \i/\g in {A/90,B/180,C/-90,D/0,I/-170, M/120, N/-70, J/-90, G/-90,P/80, O/0}{\draw[fill=white](\i) circle (1.5pt) ($(\i)+(\g:3mm)$) node[scale=1]{$\i$};}
	\end{tikzpicture}
	}
	}
\end{bt}

\begin{bt}%[1K4BC-5]%[DCHT Toán 11 - KNTT -Nguyễn Đức Lợi]
	Cho hình chóp $S.ABC$ có $M$ là điểm di động trên cạnh $SA$ sao cho $\dfrac{SM}{SA}=k$, với $0<k<1, k\in \mathbb{R}$. Gọi $(P)$ là mặt phẳng đi qua $M$ và song song với mặt phẳng $(ABC)$. Tìm $k$ để mặt phẳng $(\alpha)$ cắt hình chóp $S.ABC$ theo một thiết diện có diện tích bằng nửa diện tích của tam giác $ABC$.
	\loigiai{
	\immini{
	Qua $M$ kẻ đường thẳng song song với $AB$, cắt $SB$ tại $N$.\\
	Qua $M$ kẻ đường thẳng song song với $AC$, cắt $SC$ tại $P$.\\
	Thiết diện của hình chóp $S.ABC$ cắt bởi mặt phẳng $(P)$ là tam giác $MNP$.\\
	Ta có hai tam giác $MNP$ và $ABC$ đồng dạng theo tỉ số $\dfrac{MN}{AB}=\dfrac{SM}{SA}=k$.\\
	Suy ra $\dfrac{S_{MNP}}{S_{ABC}}=k^2 =\dfrac{1}{2} \Rightarrow k=\dfrac{\sqrt{2}}{2}$.
	}
	{
	\begin{tikzpicture}[line join=round, line cap=round,thick,scale=0.7]
	\coordinate (S) at (0,4);
	\coordinate (A) at (-2,0);
	\coordinate (B) at (0,-2);
	\coordinate (C) at (3,0);
	\coordinate (M) at ($(S)!0.5!(A)$);
	\coordinate (N) at ($(S)!0.5!(B)$);
	\coordinate (P) at ($(S)!0.5!(C)$);
	\fill[fill=gray!20](M)--(N)--(P)--cycle;
	\draw(S)--(A) (S)--(B) (S)--(C) (B)--(C) (A)--(B) (M)--(N)--(P);
	\draw[dashed,thin](A)--(C) (M)--(P);
	\foreach \i/\g in {S/90,A/180,B/-90,C/0, M/180, N/-30, P/45}{\draw[fill=white](\i) circle (1.5pt) ($(\i)+(\g:3mm)$) node[scale=1]{$\i$};}
	\end{tikzpicture}
	}
	}
\end{bt}

\begin{bt}%[1K4KC-5]%[DCHT Toán 11 - KNTT -Nguyễn Đức Lợi]
Cho hình chóp $S.ABCD$ có đáy là hình thoi cạnh $a$, tam giác $SAD$ là tam giác đều. Gọi $M$ là một điểm thuộc cạnh $AB, AM=x$ với $ 0<x<a $. Mặt phẳng $(P)$ đi qua $M$ song song với  $(SAD)$. Tính diện tích thiết diện của hình chóp $S.ABCD$ cắt bởi mặt phẳng $(P)$.
\loigiai{
\immini{
Do mặt phẳng $(P)$ đi qua điểm $M$ và song song với $(SAD)$ nên cắt các cạnh của hình chóp bằng các giao tuyến đi qua $M$ và song song với mặt phẳng $(SAD)$.\\
Do $ABCD$ là hình thoi và tam giác $SAD$ đều nên thiết diện thu được là hình thang cân $ MNFE$ với $MN \parallel EF, ME=NF$.\\
Khi đó $MN=a, \dfrac{EF}{BC}=\dfrac{SF}{SC}=\dfrac{MA}{AB}=\dfrac{x}{a} \Rightarrow EF=x$ và $MF=a-x$.\\
Kẻ đường cao $FH$, $H\in MN$.\\
Ta có $FH =\sqrt{MF^{2}-\left(\dfrac{MN-EF}{2}\right)^2} =\dfrac{\sqrt{3}}{2} (a-x) $.\\
Vậy diện tích hình thang là $ S=\dfrac{\sqrt{3}}{4} (a^2-x^2)$.
}{
\begin{tikzpicture}[line join=round, line cap=round,thick]
\coordinate (B) at (0,0);
\coordinate (C) at (-1.5,-2);
\coordinate (A) at (3,0);
\coordinate (D) at ($(A)+(C)-(B)$);
\coordinate (M) at ($(A)!1/2!(B)$);
\coordinate (S) at (0.5,2);
\coordinate (N) at ($(C)!1/2!(D)$);
\coordinate (E) at ($(S)!1/2!(B)$);
\coordinate (F) at ($(S)!1/2!(C)$);
\coordinate (H) at ($(M)!3/5!(N)$);
\fill[fill=gray!20](M)--(N)--(F)--(E)--cycle;
\draw(S)--(A)--(D)--(C)--(S)--(D) (N)--(F);
\draw[dashed,thin](S)--(B)--(C) (A)--(B) (H)--(F)--(E)--(M)--(N);
\foreach \i/\g in {S/90,A/0,B/-90,C/-90,D/-90, M/60, N/-90, F/180, E/35,H/-70}{\draw[fill=white](\i) circle (1.5pt) ($(\i)+(\g:3mm)$) node[scale=1]{$\i$};}
\end{tikzpicture}
}
}
\end{bt}

\begin{bt}%[1K4GC-5]%[DCHT Toán 11 - KNTT -Nguyễn Đức Lợi]
Cho hình chóp $S.ABCD$ có đáy là hình thang $AB \parallel CD$, $AB=2CD$. Điểm $M$ thuộc cạnh $AD$ ($M$ không trùng với $A$ và $D$) sao cho $\dfrac{MA}{MD}=x$. Gọi $(\alpha)$ là mặt phẳng qua $M$ và song song với $(SAB)$. Tìm $x$ để diện tích thiết diện của hình chóp cắt bởi mặt phẳng $(\alpha)$ bằng một nửa diện tích tam giác $SAB$.
\loigiai{
\immini{
Dễ thấy thiết diện là hình thang $MNPQ$ (như hình vẽ) với $MN \parallel SA$, $NP\parallel CD$, $PQ \parallel SB$, $QM\parallel AB$.\\
Gọi $E$ là giao điểm của $MN$ và $PQ$.\\
 Ta có: $QM=\dfrac{MD}{AD} \cdot AB+\dfrac{AM}{AD} \cdot CD$\\ $=\dfrac{1}{x+1}AB+\dfrac{x}{x+1}CD=\dfrac{x+2}{2(x+1)}AB$.\\
 Hai tam giác $SAB$ và $EMQ$ đồng dạng nên\\
 $\dfrac{S_{EMQ}}{S_{SAB}}=\left(\dfrac{MQ}{AB} \right)^2=\dfrac{(x+2)^2}{4(x+1)^2}$. \hfill (1)\\
 Vì $\dfrac{NP}{CD}=\dfrac{NS}{SD}=\dfrac{AM}{AD}=\dfrac{x}{x+1}$ suy ra $NP=\dfrac{x}{x+1}CD=\dfrac{x}{2(x+1)}AB$.\\
 Do đó $\dfrac{NP}{QM}=\dfrac{x}{x+2}$ và $\dfrac{S_{EPN}}{S_{EMQ}}=\left(\dfrac{NP}{MQ} \right)^2=\dfrac{x^2}{(x+2)^2}$\\
 $\Rightarrow \dfrac{S_{MNPQ}}{S_{EMQ}}=1-\dfrac{x^2}{(x+2)^2} =\dfrac{4x+4}{(x+2)^2}$. \hfill (2)\\
 Từ (1) và (2) suy ra $\dfrac{S_{MNPQ}}{S_{ABC}}=\dfrac{4x+4}{4(x+1)^2}=\dfrac{1}{x+1}$.\\
 Vậy $S_{MNPQ}=\dfrac{1}{2}S_{SAB} \Leftrightarrow \dfrac{1}{x+1}=\dfrac{1}{2} \Leftrightarrow x=1$.
}{
\begin{tikzpicture}[scale=0.9,line join=round, line cap=round,thick]
	\coordinate (S) at (3,4);
	\coordinate (A) at (0,0);
	\coordinate (D) at (1,-2);
	\coordinate (C) at (4,-2);
	\coordinate (B) at (5,0);
	\coordinate (M) at ($(A)!3/5!(D)$);
	\coordinate (Q) at ($(B)!3/5!(C)$);
	\coordinate (P) at ($(S)!3/5!(C)$);
	\coordinate (N) at ($(S)!3/5!(D)$);
	\coordinate (E) at (intersection of M--N and Q--P);
	\fill[fill=gray!20](M)--(N)--(P)--(Q)--cycle;
	\draw (S)--(A) (S)--(D) (S)--(C) (S)--(B) (A)--(D)--(C)--(B) (M)--(E)--(Q);
	\draw[dashed,thin](A)--(B) (M)--(Q) (P)--(N);
	\foreach \i/\g in {S/90,A/180,D/-110,C/-70,B/0, M/180, Q/0, P/60, N/150, E/120}{\draw[fill=white](\i) circle (1.5pt) ($(\i)+(\g:3mm)$) node[scale=1]{$\i$};}
	\end{tikzpicture}
}
}
\end{bt}

\centerline{\fcolorbox{red}{yellow!50}{\bf {CÂU HỎI TRẮC NGHIỆM (Tầm 10 - 20 câu theo theo tỉ lệ 4:3:2:1)}}}
\Opensolutionfile{ans}[ans/ans-1K4-13-Dang4]

\begin{ex}%[1K4BC-5]%[DCHT Toán 11 - KNTT -Nguyễn Đức Lợi]
Cho hình vuông $ABCD$ và tam giác đều $SAB$ nằm trong hai mặt phẳng khác nhau. Gọi $M$ là điểm di động trên đoạn $AB$. Mặt phẳng $(\alpha)$ qua $M$ song song với $(SBC)$ cắt hình chóp $S.ABCD$ theo thiết diện là
\choice
{Hình tam giác}
{Hình vuông}
{Hình bình hành}
{\True Hình thang}
\loigiai{
\immini{
Do $(\alpha)\parallel (SBC)$ nên $(\alpha)$ cắt mặt phẳng $(ABCD)$ theo giao tuyến $MN\parallel BC,\  (N\in CD)$.\\
Hoàn toàn tương tự, ta có $(\alpha)$ cắt các mặt phẳng $(SAB)$ và $(SCD)$ theo các giao tuyến $MQ, NP$ ở đó $Q\in SA$ và $MQ\parallel SB$; $P\in SD$ và $NP\parallel SC$.\\
Mặt phẳng $(\alpha)$ cắt hình chóp $S.ABCD$ theo thiết diện là tứ giác $MNPQ$.\\
Dễ thấy $PQ\parallel AD, PQ\neq AD$ từ đó suy ra $MNPQ$ là hình thang.
}
{
\begin{tikzpicture}[line join=round, line cap=round,thick]
\coordinate (A) at (0,0);
\coordinate (D) at (-1.5,-2);
\coordinate (B) at (3,0);
\coordinate (C) at ($(B)+(D)-(A)$);
\coordinate (M) at ($(A)!1/2!(B)$);
\coordinate (S) at (0.5,2);
\coordinate (N) at ($(C)!1/2!(D)$);
\coordinate (P) at ($(S)!1/2!(D)$);
\coordinate (Q) at ($(S)!1/2!(A)$);
\fill[fill=gray!20](M)--(N)--(P)--(Q)--cycle;
\draw(S)--(B)--(C)--(D)--(S)--(C) (N)--(P);
\draw[dashed,thin](S)--(A)--(B) (A)--(D) (P)--(Q)--(M)--(N);
\foreach \i/\g in {S/90,A/-90,B/0,C/-90,D/-90, M/60, N/-90, P/180, Q/35}{\draw[fill=white](\i) circle (1.5pt) ($(\i)+(\g:3mm)$) node[scale=1]{$\i$};}
\end{tikzpicture}
}
}
\end{ex}

\begin{ex}%[1K4BC-5]%[DCHT Toán 11 - KNTT -Nguyễn Đức Lợi]
Cho hình chóp $ S.ABCD $ có đáy $ ABCD $ là hình bình hành tâm $ O $. Gọi $ M  $ là điểm bất kì trên đoạn thẳng $ SO $. Mặt phẳng $ (\alpha) $  qua $ M $ và song song với $ (ABCD) $. Thiết diện của hình chóp $ S.ABCD $ cắt bởi mặt phẳng $ (\alpha) $ là hình gì?
\choice
{\True Hình bình hành}
{Hình tam giác}
{Hình ngũ giác}
{Hình thang cân}
\loigiai{
\immini{
Xét mặt phẳng $ (SAC) $ và mặt phẳng $ (\alpha) $, có $ M $ là điểm chung và $ AC \parallel  (\alpha) $ nên giao tuyến là đường thẳng đi qua $ M  $ và song song với $ AC $ cắt $ SA,SC $ lần lượt tại $ E,G $.
Tương tự $ (\alpha) $ cắt $ SB,SD $ lần lượt tại $ F,H $.
Khi đó thiết diện cần tìm là tứ giác $ EFGH. $
Ta có $ EF \parallel GH $ và $ FG\parallel EH $ nên thiết diện là hình bình hành.
}{
\begin{tikzpicture}[line join=round, line cap=round,thick]
\coordinate (A) at (0,0);
\coordinate (D) at (-1.5,-2);
\coordinate (B) at (3,0);
\coordinate (C) at ($(B)+(D)-(A)$);
\coordinate (O) at ($(A)!1/2!(C)$);
\coordinate (S) at (0.5,2);
\coordinate (M) at ($(S)!7/10!(O)$);
\coordinate (E) at ($(S)!7/10!(A)$);
\coordinate (G) at ($(S)!7/10!(C)$);
\coordinate (F) at ($(S)!7/10!(B)$);
\coordinate (H) at ($(S)!7/10!(D)$);
\fill[fill=gray!20](E)--(F)--(G)--(H)--cycle;
\draw(S)--(B)--(C)--(D)--(S)--(C) (F)--(G)--(H);
\draw[dashed,thin](O)--(S)--(A)--(B)--(D) (C)--(A)--(D) (H)--(E)--(F) (E)--(G) (F)--(H);
\foreach \i/\g in {S/90,A/-90,B/0,C/-90,D/-90, M/-120, E/45, F/60, G/0, O/-90, H/180}{\draw[fill=white](\i) circle (1.5pt) ($(\i)+(\g:3mm)$) node[scale=1]{$\i$};}
\end{tikzpicture}
}
}
\end{ex}

\begin{ex}%[1K4KC-5]%[DCHT Toán 11 - KNTT -Nguyễn Đức Lợi]
Cho hình chóp $S.ABCD$ có đáy $ABCD$ là hình bình hành tâm $O$, $AB = 8$, $SA= SB = 6$. Gọi $(P)$ là mặt phẳng đi qua $O$ và song song với $(SAB)$. Tính diện tích của thiết diện của $(P)$ và hình chóp $S.ABCD$.
\choice
{$12$}
{$5\sqrt{5}$}
{\True $6\sqrt{5}$}
{$13$}
\loigiai{
\immini{
Qua $O$ dựng đường thẳng $PQ\parallel AB \Rightarrow P$, $Q$ lần lượt là trung điểm của $AD$ và $BC$. \\
Qua $P$ dựng đường thẳng $PN\parallel SA \Rightarrow N$ là trung điểm của $SD$.\\
Qua $Q$ dựng đường thẳng $QM\parallel SB \Rightarrow M$ là trung điểm của $SC$.\\
Nối $M$ và $N$ $\Rightarrow$ thiết diện của $(P)$ và hình chóp $S.ABCD$ là tứ giác $MNPQ$.\\
Vì $PQ\parallel CD, MN\parallel CD\Rightarrow PQ\parallel MN$ nên tứ giác $MNPQ$ là hình thang.\\
Ta có $PQ = AB = 8$, $MN = \dfrac{1}{2}AB =4, MQ = NP =\dfrac{1}{2}SA=3$. Suy ra $MNPQ$ là hình thang cân.\\
Gọi $H$ là chân đường cao hạ từ đỉnh $M$ của hình thang $MNPQ$. Khi đó ta có $HQ = \dfrac{1}{4}PQ = 2\Rightarrow MH = \sqrt{MQ^2 - HQ^2}=\sqrt{5}$.\\
Vậy diện tích của thiết diện cần tìm là $$S=\dfrac{(MN + PQ)\cdot MH}{2}=6\sqrt{5}.$$
}{
\begin{tikzpicture}[line join=round, line cap=round,thick]
\coordinate (A) at (0,0);
\coordinate (D) at (-1.5,-2);
\coordinate (B) at (3,0);
\coordinate (C) at ($(B)+(D)-(A)$);
\coordinate (M) at ($(S)!1/2!(C)$);
\coordinate (O) at ($(A)!1/2!(C)$);
\coordinate (S) at (0.5,3);
\coordinate (N) at ($(S)!1/2!(D)$);
\coordinate (P) at ($(A)!1/2!(D)$);
\coordinate (Q) at ($(B)!1/2!(C)$);
\coordinate (H) at ($(O)!1/2!(Q)$);
\fill[fill=gray!20](M)--(N)--(P)--(Q)--cycle;
\draw(S)--(B)--(C)--(D)--(S)--(C) (N)--(M)--(Q);
\draw[dashed,thin](S)--(A)--(B)--(D) (C)--(A)--(D) (N)--(P)--(Q) (M)--(H);
\foreach \i/\g in {S/90,A/-90,B/0,C/-90,D/-90, M/60, N/180, P/-90, Q/0, O/-90, H/-80}{\draw[fill=white](\i) circle (1.5pt) ($(\i)+(\g:3mm)$) node[scale=1]{$\i$};}
\end{tikzpicture}
}
}
\end{ex}

\begin{ex}%[1K4BC-5]%[DCHT Toán 11 - KNTT -Nguyễn Đức Lợi]
Cho hình hộp $ABCD.A'B'C'D'$. Gọi $I$ là trung điểm của cạnh $AB$, $(\alpha)$ là mặt phẳng đi qua $I$ và song song với mặt phẳng $(BDD')$. Thiết diện của hình hộp cắt bởi mặt phẳng $(\alpha)$ là hình gì?
\choice
{Hình tam giác}
{\True Hình bình hành}
{Hình ngũ giác}
{Hình thang}
\loigiai{
\immini{
Qua $I$ kẻ đường thẳng song song với $BD$, cắt $AD$ tại $K$.\\
Qua $K$ kẻ đường thẳng song song với $DD'$, cắt $A'D'$ tại $K'$.\\
Qua $I$ kẻ đường thẳng song song với $BB'$, cắt $A'B'$ tại $I'$.\\
Thiết diện cần tìm là hình bình hành $IKK'I'$.
}
{
\begin{tikzpicture}[line join=round, line cap=round,thick]
\coordinate (A') at (0,0);
\coordinate (B') at (-1,-1);
\coordinate (C') at (3,-1);
\coordinate (D') at (4,0);
\coordinate (A) at (0,2);
\coordinate (B) at (-1,1);
\coordinate (C) at (3,1);
\coordinate (D) at (4,2);
\coordinate (I) at (-0.5,1.5);
\coordinate (I') at (-0.5,-0.5);
\coordinate (K) at (2,2);
\coordinate (K') at (2,0);
\fill[fill=gray!20](I)--(K)--(K')--(I')--cycle;
\draw [dashed] (B')--(A')--(D') (A)--(A');
\draw (B')--(C')--(D');
\draw (A)--(B)--(C)--(D)--(A) (B)--(B') (C)--(C') (D)--(D');
\draw [dashed] (D')--(B') (I)--(I')--(K')--(K);
\draw (B)--(D) (I)--(K);

\foreach \i/\g in {A/90,B/180,C/0,D/0,A'/40,B'/180,C'/-45,D'/0, I/120, K/90, K'/50, I'/180}{\draw[fill=white](\i) circle (1.5pt) ($(\i)+(\g:3mm)$) node[scale=1]{$\i$};}
\end{tikzpicture}
}
}
\end{ex}

\begin{ex}%[1K4BC-5]%[DCHT Toán 11 - KNTT -Nguyễn Đức Lợi]
Cho tứ diện đều $SABC$. Gọi $I$ là trung điểm của đoạn $AB$, $M$ là điểm di động trên đoạn $AI$. Qua $M$ vẽ mặt phẳng $(\alpha)$ song song với $(SIC)$. Thiết diện tạo bởi $(\alpha)$ với tứ diện $SABC$ là
\choice
{\True tam giác cân tại $M$}
{hình bình hành}
{tam giác đều}
{hình thoi}
\loigiai{
\immini{
Trong mặt phẳng $(SAB)$, qua $M$ kẻ đường thẳng song song với $SI$ cắt $SA$ tại $P$.\\
Trong mặt phẳng $(ABC)$, qua $M$ kẻ đường thẳng song song với $IC$ cắt $AC$ tại $N$.\\
Thiết diện là tam giác $MNP$. Ta có
$$\dfrac{MP}{SI}=\dfrac{MN}{CI}\Rightarrow MP=MN\quad (\text{vì } SI=CI).$$
Vậy thiết diện là tam giác $MNP$ cân tại $M$.
}{
\begin{tikzpicture}[line join=round, line cap=round,thick]
\coordinate (S) at (2.5,4);
\coordinate (A) at (0,0);
\coordinate (B) at (2,-2);
\coordinate (C) at (6,0);
\coordinate (I) at ($(A)!.5!(B)$);
\coordinate (M) at ($(A)!.57!(I)$);
\coordinate (P) at ($(A)!.57!(S)$);
\coordinate (N) at ($(A)!.57!(C)$);
\fill[fill=gray!20](M)--(N)--(P)--cycle;
\draw (S)--(A)--(B)--(C)--(S)--(B) (S)--(I) (M)--(P);
\draw[dashed] (A)--(C)--(I) (M)--(N)--(P);

\foreach \i/\g in {S/90,A/180,C/0,B/-90,M/180,I/180,N/45,P/130}{\draw[fill=white](\i) circle (1.5pt) ($(\i)+(\g:3mm)$) node[scale=1]{$\i$};}
\end{tikzpicture}
}
}
\end{ex}

\begin{ex}%[1K4KC-5]%[DCHT Toán 11 - KNTT -Nguyễn Đức Lợi]
Cho tứ diện đều $ABCD$ cạnh $a$ và $G$ là trọng tâm tam giác $ABC$. Cắt tứ diện bởi mặt phẳng $(P)$ qua $G$ và song song với mặt phẳng $(BCD)$ thì diện tích thiết diện bằng bao nhiêu?
\choice
{$\dfrac{a^2\sqrt{3}}{16}$}
{$\dfrac{a^2\sqrt{3}}{4}$}
{\True $\dfrac{a^2\sqrt{3}}{9}$}
{$\dfrac{a^2\sqrt{3}}{18}$}
\loigiai{
\immini{
Trong mặt phẳng $(ABC)$ kẻ đường thẳng qua $G$ và song song với $BC$ cắt $AC,AB$ lần lượt tại $H,K$.\\
Trong mặt phẳng $(ACD)$ kẻ đường thẳng qua $H$ và song song với $CD$ cắt $AD$ tại $I$.\\
Thiết diện cần tìm là tam giác $KHI$.\\
Ta có $\triangle HKI\backsim \triangle BCD$ theo tỉ số đồng dạng bằng $\dfrac{2}{3}$.\\
Do đó $S_{KHI}=\dfrac{4}{9}S_{BCD}=\dfrac{4}{9}\dfrac{a^2\sqrt{3}}{4}=\dfrac{a^2\sqrt{3}}{9}$.
}{
	\begin{tikzpicture}[line join=round, line cap=round,thick,scale=0.9]
	\coordinate (A) at (0,4);
	\coordinate (B) at (-2,0);
	\coordinate (C) at (0,-2);
	\coordinate (D) at (3,0);
	\coordinate (M) at ($(B)!1/2!(C)$);
	\coordinate (G) at ($(A)!2/3!(M)$);
	\coordinate (H) at ($(A)!2/3!(C)$);
	\coordinate (K) at ($(A)!2/3!(B)$);
	\coordinate (I) at ($(A)!2/3!(D)$);
	\fill[fill=gray!20](H)--(K)--(I)--cycle;
	\draw(A)--(B)--(C)--(D)--(A)--(C) (A)--(M) (K)--(H)--(I);
	\draw[dashed,thin](B)--(D) (K)--(I);
	\foreach \i/\g in {A/90,B/180,C/-90,D/0, M/190, G/190,K/180, H/-40, I/45}{\draw[fill=white](\i) circle (1.5pt) ($(\i)+(\g:3mm)$) node[scale=1]{$\i$};}
	\end{tikzpicture}
}
}
\end{ex}

\begin{ex}%[1K4KC-5]%[DCHT Toán 11 - KNTT -Nguyễn Đức Lợi]
Cho hình chóp $S.ABCD$ có đáy $ABCD$ là hình bình hành tâm $O$. Tam giác $SBD$ đều. Một mặt phẳng $(P)$ song song với $(SBD)$ và qua điểm $I$ thuộc cạnh $AC$ ( không trùng với $A$ hoặc $C$). Thiết diện của $(P)$ và hình chóp là hình gì?
\choice
{Hình bình hành}
{\True Tam giác đều}
{Tam giác vuông}
{Tam giác cân không đều}
\loigiai{
\immini{
{\bf TH1:} Điểm  $I$ thuộc đoạn $OC$. Gọi $M,N,E$ lần lượt là giao điểm của $(P)$ với các đường thẳng $CD,BC,SC$. Ta có thiết diện của $(P)$ và hình chóp là tam giác $EMN$.\\
$\heva{&(P) \parallel (SBD)\\& (ABCD) \cap (SBD) = BD \\ & (P) \cap (ABCD)=MN } \quad \Rightarrow MN \parallel BD$\\
$\Rightarrow \dfrac{MN}{BD}=\dfrac{CM}{CD}=t \Rightarrow MN = t\cdot BD$.\\
$\heva{&(P) \parallel (SBD)\\& (SCD) \cap (SBD) = SD \\ & (P) \cap (SCD)=ME } \quad \Rightarrow ME \parallel SD$ \\
$\Rightarrow \dfrac{ME}{SD}=\dfrac{CM}{CD} = t\Rightarrow ME = t\cdot SD$. \\
Tương tự $EN \parallel SB$ và $EN = t \cdot SB$
}{
\begin{tikzpicture}[line join=round, line cap=round,thick]
			\coordinate (A) at (0,0);
			\coordinate (B) at (2,-2);
			\coordinate (D) at (5,0);
			\coordinate (C) at ($(B)+(D)-(A)$);
			\coordinate (O) at ($(A)!1/2!(C)$);
			\coordinate (S) at ($(O)+(0.5,5)$);
			\coordinate (I) at ($(A)!1/6!(C)$);
			\coordinate (M) at ($(C)!1/2!(D)$);
			\coordinate (N) at ($(C)!1/2!(B)$);
			\coordinate (E) at ($(C)!1/2!(S)$);
			\fill[fill=gray!20](E)--(M)--(N)--cycle;
			\draw(S)--(A) (S)--(B) (S)--(C) (A)--(B) (B)--(C) (E)--(N);
			\draw[dashed,thin](A)--(C) (A)--(D) (C)--(D) (S)--(D) (S)--(O) (B)--(D) (E)--(M)--(N);
			
			\foreach \i/\g in {S/90,A/180,B/-90,C/-90,D/140,O/-90, M/-90, N/-90, E/45, I/-80}{\draw[fill=white](\i) circle (1.5pt) ($(\i)+(\g:3mm)$) node[scale=1]{$\i$};}
		\end{tikzpicture}
}
\noindent Mà $SB=BD=SD\Rightarrow MN=EM=EN$, suy ra tam giác $MNE$ là tam giác đều.\\
{\bf TH2:} Điểm $I$ thuộc $OA$. Làm tương tự như trên ta được thiết diện của $(P)$ và hình chóp cũng là một tam giác đều.
}
\end{ex}
\begin{ex}%[1K4BC-5]%[DCHT Toán 11 - KNTT -Nguyễn Đức Lợi]
Cho tứ diện đều $ S.ABC $. Gọi $ I $ và $ J $ lần lượt là trung điểm của $ AB $ và $ SC $. Xét $ M $ là một điểm di động trên đoạn thẳng $ AI $. Qua $ M $ kẻ mặt phẳng $ \left(\alpha\right) $ song song với $ \left(CIJ\right) $. Khi đó thiết diện của mặt phẳng $ \left(\alpha\right) $ và tứ diện $ S.ABC $ là hình gì?
\choice
{ Tam giác đều}
{ Hình bình hành}
{\True Tam giác cân tại $ M $}
{ Hình thang cân}
\loigiai
{\immini
{
Nhận xét mặt phẳng $ (CIJ) $ chính là mặt phẳng $ (SCI) $.\\
Qua $M$ kẻ đường thẳng song song với $IC$, cắt $AC$ tại $K$.\\
Qua $M$ kẻ đường thẳng song song với $SI$, cắt $SA$ tại $P$.\\
Thiết diện là tam giác $ MPK $.\\
Ta  có $ CI = SI \neq SC $ nên tam giác $ SCI  $ cân tại $ I $.\\
Mặt khác $\triangle PKM \sim \triangle SCI $.\\
Do đó tam giác $ PKM $ cân tại $M$.
}
{
\begin{tikzpicture}[line join=round, line cap=round,thick]
\coordinate (A) at (0,0);
\coordinate (B) at (2,-1.5);
\coordinate (C) at (4,0);
\coordinate (S) at (1,4);
\coordinate (I) at ($(B)!1/2!(A)$);
\coordinate (J) at ($(S)!1/2!(C)$);
\coordinate (M) at ($(A)!1/3!(I)$);
\coordinate (K) at ($(A)!1/3!(C)$);
\coordinate (P) at ($(A)!1/3!(S)$);
\fill[fill=gray!20](P)--(M)--(K)--cycle;
\draw (I)--(S)--(C)--(B)--(A)--(S)--(B) (P)--(M);
\draw[dashed,thin](A)--(C)--(I)--(J) (M)--(K)--(P);

\foreach \i/\g in {S/90,A/180,B/-90,C/-90, M/-90, K/90, P/180, I/-80,J/45}{\draw[fill=white](\i) circle (1.5pt) ($(\i)+(\g:3mm)$) node[scale=1]{$\i$};}
\end{tikzpicture}
}
}
\end{ex}

\begin{ex}%[1K4KC-5]%[DCHT Toán 11 - KNTT -Nguyễn Đức Lợi]
Cho hình chóp tứ giác đều $S.ABCD$ có cạnh đáy bằng $10$. Gọi $M$ là điểm trên $SA$ sao cho $\dfrac{SM}{SA}=\dfrac{2}{3}$. Một mặt phẳng $(\alpha)$ đi qua $M$ song song với $AB$ và $AD$, cắt hình chóp theo một tứ giác có diện tích là
\choice
{$\dfrac{20}{3}$}
{\True $\dfrac{400}{9}$}
{$\dfrac{4}{9}$}
{$\dfrac{16}{9}$}
\loigiai{
\immini{
Từ $\heva{&AB\parallel (\alpha)\\&AB \subset (SAB)\\&M\in (SAB)\cap (\alpha)}\Rightarrow (\alpha) \cap (SAB)=Mx\parallel AB$.\\
Từ $\heva{&AD\parallel (\alpha)\\&AD \subset (SAD)\\&M\in (SAD)\cap (\alpha)}\Rightarrow (\alpha) \cap (SAD)=My\parallel AD$.\\
$Mx,My$ cắt $SB,SD$ lần lượt tại $N,Q$.\\
Từ $\heva{&CD\parallel (\alpha)\\&CD \subset (SCD)\\&Q\in (SCD)\cap (\alpha)}\Rightarrow (\alpha) \cap (SCD)=Qz\parallel CD$.\\
$Qz$ cắt $SC$ tại $P$.\\
Thiết diện cần tìm là tứ giác $MNPQ$.\\
Dễ thấy $\dfrac{2}{3}=\dfrac{SM}{SA}=\dfrac{MN}{AB}=\dfrac{NP}{BC}=\dfrac{PQ}{CD}=\dfrac{QM}{DA}$.\\
$\Rightarrow MN=NP=PQ=QM=\dfrac{20}{3}$.\\
}
{
\begin{tikzpicture}[scale=0.8,line join=round, line cap=round,thick]
\coordinate (S) at (4.5,6);
\coordinate (A) at (3,1);
\coordinate (B) at (1,-1);
\coordinate (C) at (6,-1);
\coordinate (D) at (8,1);
\coordinate (M) at ($(S)!2/3!(A)$);
\coordinate (N) at ($(S)!2/3!(B)$);
\coordinate (P) at ($(S)!2/3!(C)$);
\coordinate (Q) at ($(S)!2/3!(D)$);
\fill[fill=gray!20](M)--(N)--(P)--(Q)--cycle;
\draw (S)--(B)--(C)--(D)--(S)--(C) (N)--(P)--(Q);
\draw[dashed,thin](B)--(A)--(D) (S)--(A) (Q)--(M)--(N);

\foreach \i/\g in {S/90,A/180,B/-90,C/-90, M/120, N/180, P/-30, Q/20,D/0}{\draw[fill=white](\i) circle (1.5pt) ($(\i)+(\g:3mm)$) node[scale=1]{$\i$};}
\end{tikzpicture}
}
\noindent
Mặt khác $MN \parallel AB, NP\parallel BC\Rightarrow \widehat{(MN,NP)}=\widehat{\left(AB,BC\right)}=90^\circ$
$\Rightarrow MNPQ$ là hình vuông với cạnh $\dfrac{20}{3}\Rightarrow S_{MNPQ}=\left(\dfrac{20}{3}\right)^2=\dfrac{400}{9}$.
}
\end{ex}

\begin{ex}%[1K4KC-5]%[DCHT Toán 11 - KNTT -Nguyễn Đức Lợi]
Cho hình chóp $ S.ABCD $ có đáy $ ABCD $ là hình thang cân với cạnh bên $ BC = 3 $, hai đáy $ AB = 8 $, $ CD = 4 $. Mặt phẳng $ (P) $ song song với $ (ABCD) $ và cắt cạnh $ SA $ tại $ M $ sao cho $ SA = 3SM $. Diện tích thiết diện của $ (P) $ và hình chóp $ S{.}ABCD $ bằng bao nhiêu?
\choice
{ $ \dfrac{2 \sqrt{5}}{9} $}
{\True $ \dfrac{2 \sqrt{5}}{3} $ }
{ $ \dfrac{7 \sqrt{3}}{3} $}
{$ \dfrac{7 \sqrt{3}}{9} $ }
\loigiai{
\immini{
Gọi $ M $, $ N $, $ P $ lần lượt là giao điểm của $ (P) $ với các cạnh $ SB $, $ SC $, $ SD $.\\ Suy ra thiết diện của $ (P) $ với hình chóp là tứ giác $ MNPQ $.\\
Do $ (P) \parallel (ABCD)$ nên $ \dfrac{MN}{AB} = \dfrac{NP}{BC} = \dfrac{PQ}{CD} = \dfrac{SM}{SA} = \dfrac{1}{3} \cdot$\\
Suy ra $ MN = \dfrac{8}{3} $, $ NP = 1 $, $ PQ = \dfrac{4}{3} $, $ QM = 1 $.\\
Gọi $ H $ là hình chiếu của $ P $ trên cạnh $ MN $.\\
Do $ MNPQ $ là hình thang cân nên\\
$ HN = \dfrac{MN - PQ}{2} = \dfrac{2}{3}  $
$ \Rightarrow PH = \sqrt{PN^2  - HN^2} = \dfrac{\sqrt 5}{3} \cdot$\\
Vậy $ S_{MNPQ} = \dfrac{1}{2} \cdot PH \cdot \left(MN +PQ\right) = \dfrac{2\sqrt{5}}{3} \cdot$
}{
\begin{tikzpicture}[scale=0.9,line join=round, line cap=round,thick]
	\coordinate (S) at (2,4);
	\coordinate (A) at (0,0);
	\coordinate (D) at (1,-2);
	\coordinate (C) at (4,-2);
	\coordinate (B) at (5,0);
	\coordinate (M) at ($(S)!1/3!(A)$);
	\coordinate (N) at ($(S)!1/3!(B)$);
	\coordinate (P) at ($(S)!1/3!(C)$);
	\coordinate (Q) at ($(S)!1/3!(D)$);
	\coordinate (H) at ($(M)!1/2!(N)$);
	\fill[fill=gray!20](M)--(N)--(P)--(Q)--cycle;
	\draw (C)--(S)--(A)--(D)--(C)--(B)--(S)--(D) (M)--(Q)--(P)--(N);
	\draw[dashed,thin](A)--(B) (M)--(N) (P)--(H);
	
	\foreach \i/\g in {S/90,A/180,B/0,C/-90,D/-90, M/180, Q/-45, P/-135, N/0, H/90}{\draw[fill=white](\i) circle (1.5pt) ($(\i)+(\g:3mm)$) node[scale=1]{$\i$};}
\end{tikzpicture}
}
}
\end{ex}

\begin{ex}%[1K4KC-5]%[DCHT Toán 11 - KNTT -Nguyễn Đức Lợi]
Cho hình hộp $ABCD.A'B'C'D'$ và điểm $M$ nằm giữa hai điểm $A$ và $B$. Gọi $(P)$ là mặt phẳng đi qua $M$ và song song với mặt phẳng $(AB'D')$. Mặt phẳng $(P)$ cắt hình hộp theo thiết diện là hình gì?
\choice
{Hình tứ giác}
{Hình tam giác}
{\True Hình lục giác}
{Hình ngũ giác}
\loigiai{
\immini{
Nhận thấy $(BC'D)\parallel (AB'D')\Rightarrow (BC'D)\parallel (AB'D')\parallel (P)$.\hfill $(1)$\\
Do $(1)$, ta giả sử $(P)$ cắt $BB'$ tại $N$, suy ra $(P)\cap (ABB'A')= MN$, kết hợp với $(AB'D')\cap (ABB'A')= AB'$ suy ra $MN\parallel AB'$, suy ra $N$ thuộc cạnh $BB'$.\\
Tương tự, giả sử $(P)\cap B'C'= P$ suy ra $(P)\cap (BCC'B')= NP$. Kết hợp với $(1)$ suy ra $NP\parallel BC'$.\\
Tương tự, $(P)\cap C'D'= Q$ sao cho $PQ\parallel B'D'$; $(P)\cap DD' = G$ sao cho $QG\parallel C'D$; $(P)\cap AD = H$ sao cho $GH\parallel AD'$.\\
Từ đó suy ra thiết diện là lục giác $MNPQGH$.
}{
\begin{tikzpicture}[line join=round, line cap=round,thick,scale=0.8]
\coordinate (A') at (0,0);
\coordinate (B') at (-2,-2);
\coordinate (D') at (4,0);
\coordinate (C') at ($(B')+(D')-(A')$);
\coordinate (A) at ($(A')+(0,4)$);
\coordinate (B) at ($(B')+(0,4)$);
\coordinate (C) at ($(C')+(0,4)$);
\coordinate (D) at ($(D')+(0,4)$);
\coordinate(M)at($(A)!0.4!(B)$);
\coordinate(N)at($(B')!0.4!(B)$);
\coordinate(P)at($(B')!0.4!(C')$);
\coordinate(Q)at($(D')!0.4!(C')$);
\coordinate(G)at($(D')!0.4!(D)$);
\coordinate(H)at($(A)!0.4!(D)$);
\fill[fill=gray!20](M)--(N)--(P)--(Q)--(G)--(H)--(M);
\draw (B)--(C)--(D)--(A)--(B) (C)--(C')--(D')--(D) (Q)--(G)--(H)--(M) (B)--(D)--(C')--(B')--(B);
\draw[dashed,thin]  (A')--(B')--(A) (B')--(D')--(A')--(A)  (M)--(N)--(P)--(Q) (B)--(C');
\foreach \i/\g in {A'/-90,B'/-90,C'/-90,D'/0,A/90,B/180,C/90,D/0, M/170, N/180, P/-90, Q/-30, G/0, H/90}{\draw[fill=white](\i) circle (1.5pt) ($(\i)+(\g:3mm)$) node[scale=1]{$\i$};}
\end{tikzpicture}
}}
\end{ex}

\begin{ex}%[1K4KC-5]%[DCHT Toán 11 - KNTT -Nguyễn Đức Lợi]
Cho hình chóp $S.ABCD$ có đáy là hình thang $ABCD$, $AB \parallel CD$, $AB=2CD$, $M$ là điểm thuộc cạnh $AD$, $(\alpha)$ là mặt phẳng qua $M$ và song song với mặt phẳng $(SAB)$. Biết diện tích thiết diện của hình chóp cắt bởi mặt phẳng $(\alpha)$ bằng $\dfrac{2}{3}$ diện tích tam giác $SAB$, tính tỉ số $x=\dfrac{MA}{MD}$.
\choice
{$x=1$}
{$x=\dfrac{2}{3}$}
{$x=\dfrac{3}{2}$}
{\True $x=\dfrac{1}{2}$}
\loigiai{
\begin{center}
 \begin{tikzpicture}[scale=.8]
%\tkzDefPoints{0/0/A, -2.5/-2/D, 1/-2/C, 8/0/B}
\coordinate (A) at (0,0);
\coordinate (D) at (-2.5,-2);
\coordinate (C) at (1,-2);
\coordinate (B) at (8,0);
\coordinate (O) at ($(A)!0.5!(C)$);
\coordinate (S) at ($(O)+(0,5)$);
\coordinate (M) at ($(A)!.4!(D)$);
\coordinate (N) at ($(B)!.4!(C)$);
\coordinate (P) at ($(S)!.4!(C)$);
\coordinate (Q) at ($(S)!.4!(D)$);
\coordinate (K) at ($(A)!.25!(B)$);
\coordinate (H) at ($(K)!.4!(C)$);
\fill[fill=gray!20](M)--(N)--(P)--(Q)--(M);
\draw  (C)--(D)--(S)--(C) (Q)--(P)--(N) (S)--(B)--(C);
\draw[dashed, thin] (P)--(H) (N)--(M)--(Q) (C)--(K)--(S)--(A)--(B) (S)--(K)--(C) (P)--(H) (C)--(A)--(D);
\foreach \i/\g in {S/90,A/130,B/0,C/-90,D/180,M/-90,N/-60, P/40, Q/180, K/-60, H/-30}{\draw[fill=white](\i) circle (1.5pt) ($(\i)+(\g:3mm)$) node[scale=1]{$\i$};}
\end{tikzpicture}
\end{center}
Ta có $\begin{cases}
(\alpha)\parallel (SAB) \\
(ABCD)\cap (SAB)=AB\\
M\in (\alpha)\cap (ABCD)
\end{cases} $ suy ra giao tuyến của $(\alpha)$ và $(ABCD)$ là đường thẳng qua $M$ và song song với $AB$, đường thẳng này cắt $BC$ tại $N$.\\
Tương tự giao tuyến của $(\alpha) $ và $(SBC)$ là đường thẳng qua $N$ song song $SB$ cắt $SC$ tại $P$, giao tuyến của $(\alpha) $ và $(SCD)$ là đường thẳng qua $P$ song song $CD$ cắt $SD$ tại $Q$. Thiết diện của $S.ABCD$ khi cắt bởi $(\alpha)$ là hình thang $MNPQ$.\\
Đặt $CD=a$, ta có  $\dfrac{PQ}{CD}=\dfrac{SQ}{SD}=\dfrac{AM}{AD}=\dfrac{x}{x+1}\Leftrightarrow PQ=\dfrac{ax}{x+1}$. \\
Trong hình thang $ABCD$ ta có $MN=\dfrac{x}{x+1}CD+\dfrac{1}{x+1}AB=\dfrac{a(x+2)}{x+1}$.\\
Gọi $K$ là hình chiếu của $S$ lên $AB$, $H$ là giao của $MN$ và $CK$, khi đó $PH \parallel SK$ và do đó $PH \perp MN$, thêm nữa $\dfrac{PH}{SK}=\dfrac{CH}{CK}=\dfrac{DM}{DA}=\dfrac{1}{x+1}$
nên $\dfrac{S_{MNPQ}}{S_{ABC}}=\dfrac{(PQ+MN)\cdot PH}{SK\cdot AB}=\dfrac{1}{x+1}$.\\
Theo giả thiết $\dfrac{1}{x+1}=\dfrac{2}{3}\Leftrightarrow x=\dfrac{1}{2}$. 
}
\end{ex}

\begin{ex}%[1K4KC-5]%[DCHT Toán 11 - KNTT -Nguyễn Đức Lợi]
Cho hình chóp $S.ABCD$ có đáy là hình vuông tâm $O$, cạnh $a$, các cạnh bên đều bằng $2a$. Gọi $(\alpha)$ là mặt phẳng đi qua $O$ và song song với mặt phẳng $(SBC)$. Tính chu vi $P$ của thiết diện tạo bởi mặt phẳng $(\alpha)$ và hình chóp $S.ABCD$.
\choice
{\True $P=\dfrac{7a}{2}$}
{$P=\dfrac{11a}{2}$}
{$P=\dfrac{9a}{2}$}
{$P=\dfrac{5a}{2}$}
\loigiai{
\immini{
Do $(\alpha)\parallel (SBC)$ nên $(\alpha)$ cắt mặt phẳng $(ABCD)$ theo giao tuyến đi qua $O$, song song với $BC$ và cắt $AB$ tại $M$, cắt $CD$ tại $N$. \\
Hoàn toàn tương tự, ta có $(\alpha)$ cắt các mặt phẳng $(SAB)$ và $(SCD)$ theo các giao tuyến $MQ, NP$ ở đó $Q\in SA$ và $MQ\parallel SB$; $P\in SD$ và $NP\parallel SC$.\\
Mặt phẳng $(\alpha)$ cắt hình chóp $S.ABCD$ theo thiết diện là tứ giác $MNPQ$.\\
Lại có $O$ là trung điểm $AC$ nên $M$, $N$, $P$, $Q$ lần lượt là trung điểm của $AB$, $CD$, $SD$, $SA$.\\
Suy ra $MN=a$, $NP=\dfrac{1}{2}SC=a$, $MQ=\dfrac{1}{2}SB=a$, $PQ=\dfrac{1}{2}AD=\dfrac{a}{2}$.\\
}{
\begin{tikzpicture}[line join=round, line cap=round,thick]
\coordinate (A) at (0,0);
\coordinate (D) at (-1.5,-2);
\coordinate (B) at (3,0);
\coordinate (C) at ($(B)+(D)-(A)$);
\coordinate (M) at ($(A)!1/2!(B)$);
\coordinate (S) at (0.5,2);
\coordinate (N) at ($(C)!1/2!(D)$);
\coordinate (P) at ($(S)!1/2!(D)$);
\coordinate (Q) at ($(S)!1/2!(A)$);
\coordinate (O) at ($(C)!1/2!(A)$);
\fill[fill=gray!20](M)--(N)--(P)--(Q)--cycle;
\draw(S)--(B)--(C)--(D)--(S)--(C) (N)--(P);
\draw[dashed,thin](S)--(A)--(B) (A)--(D) (P)--(Q)--(M)--(N);
\foreach \i/\g in {S/90,A/-90,B/0,C/-90,D/-90, M/60, N/-90, P/180, Q/35,O/-90}{\draw[fill=white](\i) circle (1.5pt) ($(\i)+(\g:3mm)$) node[scale=1]{$\i$};}
\end{tikzpicture}
}
\noindent
Vậy chu vi của thiết diện là
$P=MN+NP+PQ+QM=a+a+\dfrac{a}{2}+a=\dfrac{7a}{2}$.
}
\end{ex}

\begin{ex}%[1K4GC-5]%[DCHT Toán 11 - KNTT -Nguyễn Đức Lợi]
Cho hình chóp $S.ABCD$ có đáy là hình thang $ABCD$, $AB \parallel CD$, $AB=2CD$. $M$ là điểm thuộc cạnh $AD$, $\left(\alpha \right)$ là mặt phẳng qua $M$ và song song với mặt phẳng $\left(SAB\right)$. Biết diện tích thiết diện của hình chóp cắt bởi mặt phẳng $\left(\alpha \right)$ bằng $\dfrac{2}{3}$ diện tích tam giác $SAB$. Tính tỉ số $x=\dfrac{MA}{MD}\cdot$
\choice
{$x=1$}
{$x=\dfrac{3}{2}$}
{$x=\dfrac{2}{3}$}
{\True $x=\dfrac{1}{2}$}
\loigiai{
\immini{
Dễ thấy, thiết diện là hình thang $MNPQ$ (như hình vẽ) với $MN \parallel AB$, $MQ\parallel SA$, $PQ \parallel CD$, $NP\parallel SB$.\\
Gọi $E=AD \cap BC$ và $F=MQ \cap NP$.\\
Đặt $MD=1$, vì $x=\dfrac{MA}{MD}$ nên $MA=x$, $AD=x+1$.\\
$\dfrac{QP}{DC}=\dfrac{SQ}{SD}=\dfrac{AM}{AD}=\dfrac{x}{x+1}\Rightarrow \dfrac{QP}{AB}=\dfrac{x}{2\left(x+1\right)}$.\\
Mà $\dfrac{MN}{AB}=\dfrac{EM}{EA}=\dfrac{x+2}{2x+2}$.\\
$\Rightarrow \dfrac{QP}{MN}=\dfrac{x}{x+2}$.\\
$\dfrac{S_{FPQ}}{S_{FMN}}=\dfrac{FQ}{FM}\cdot \dfrac{FP}{FN}={\left(\dfrac{QP}{MN}\right)}^2={\left(\dfrac{x}{x+2}\right)}^2=\dfrac{x^2}{(x+2)^2}$.\\
$\dfrac{S_{FMN}}{S_{SAB}}=\dfrac{FM}{SA}\cdot\dfrac{FN}{SB}=\dfrac{EM}{EA}\cdot\dfrac{EN}{EB}=\left(\dfrac{MN}{AB}\right)^2=\left(\dfrac{x+2}{2x+2}\right)^2$.\\
$\Rightarrow S_{FMN}=\dfrac{(x+2)^2}{(2x+2)^2}\cdot S_{SAB}$. \\
$\Rightarrow \dfrac{S_{FPQ}}{S_{SAB}}=\left(\dfrac{x}{x+2}\right)^2\cdot \left(\dfrac{x+2}{2x+2}\right)^2=\dfrac{x^2}{(2x+2)^2}$.\\
$\Rightarrow S_{FPQ}=\dfrac{x^2}{(2x+2)^2}\cdot S_{SAB}$
$\Rightarrow S_{MNPQ}=S_{FMN}-S_{FPQ}$\\
$=\left[\dfrac{(x+2)^2}{(2x+2)^2}-\dfrac{x^2}{(2x+2)^2}\right]\cdot S_{SAB}=\dfrac{4x+4}{(2x+2)^2}\cdot S_{SAB}$.\\
$\Rightarrow \dfrac{S_{MNPQ}}{S_{SAB}}=\dfrac{4x+4}{(2x+2)^2}$.
}{
\begin{tikzpicture}[scale=0.9,line join=round, line cap=round,thick]
	\coordinate (S) at (3,4);
	\coordinate (A) at (0,0);
	\coordinate (D) at (1,-2);
	\coordinate (C) at (4,-2);
	\coordinate (B) at (5,0);
	\coordinate (M) at ($(A)!3/5!(D)$);
	\coordinate (N) at ($(B)!3/5!(C)$);
	\coordinate (P) at ($(S)!3/5!(C)$);
	\coordinate (Q) at ($(S)!3/5!(D)$);
	\coordinate (E) at (intersection of A--D and B--C);
	\coordinate (F) at (intersection of M--Q and N--P);
	\fill[fill=gray!20](M)--(N)--(P)--(Q)--cycle;
	\draw (E)--(S)--(A) (S)--(D) (S)--(C) (S)--(B) (A)--(E)--(B) (M)--(F)--(N);
	\draw[dashed,thin](A)--(B) (M)--(N) (D)--(C) (P)--(Q);
	\foreach \i/\g in {S/90,A/180,D/-110,C/-70,B/0, M/180, N/0, P/60, Q/150, E/-90, F/120}{\draw[fill=white](\i) circle (1.5pt) ($(\i)+(\g:3mm)$) node[scale=1]{$\i$};}
	\end{tikzpicture}
}
\noindent
Theo đề bài ta có $\dfrac{S_{MNPQ}}{S_{SAB}}=\dfrac{2}{3}\Leftrightarrow \dfrac{4x+4}{(2x+2)^2}=\dfrac{2}{3}\Leftrightarrow 8x^2+4x-4=0\Leftrightarrow \hoac{
& x=\dfrac{1}{2} \\
& x=-1 \\}$.\\
Vì $x>0$ nên $x=\dfrac{1}{2}.$
}
\end{ex}
\Closesolutionfile{ans}
\begin{indapan}{10}
	{ans/ans-1K4-13-Dang4}
\end{indapan}

