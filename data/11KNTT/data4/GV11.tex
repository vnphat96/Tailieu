\begin{dang}{Xác định thiết diện bằng cách kẻ song song}
	Để tìm thiết diện của mặt phẳng $(P)$ qua điểm $M$ và song song với hai cạnh $a$, $b$ của hình chóp ta làm như sau
	\begin{itemize}
		\item Xét các mặt của hình chóp chứa cạnh $a$.
		\item Trong mặt phẳng đang xét, từ $M$ vẽ đường thẳng song song với $a$.
		\item Xét các mặt của hình chóp chứa cạnh $b$.
		\item Trong mặt phẳng đang xét, từ $M$ vẽ đường thẳng song song với $b$.
	\end{itemize}
	Nối các đoạn giao tuyến lại ta được thiết diện của mặt phẳng $(P)$ với hình chóp.
\end{dang}
\subsubsection{Ví dụ mẫu}
%ví dụ 1
\begin{vd}%[Dự án Đề cương 11-Mui Doan]%[1T4K3-5]
	Cho hình chóp $S.ABCD$.  $M$, $N$ là hai điểm trên $AB$, $CD$. Mặt phẳng $(P)$ qua $MN$ và song song với $SA$.	Xác định thiết diện của $(P)$ với hình chóp.
	\loigiai{
	\immini{
	Mặt phẳng $(SAB)$ chứa đường thẳng $SA$ song song với mặt phẳng $(P)$ nên mặt phẳng $(SAB)$ cắt mặt phẳng $(P)$ theo giao tuyến song song với $SA$. Vẽ $MK\parallel SA$ ($K\in SB$) thì $MK$ là giao tuyến của $(P)$ với mặt phẳng $(SAB)$.\\
	Trong $(ABCD)$ ta có $AC\cap MN=E\Rightarrow E\in (P)$.\\
	Mặt phẳng $(SAC)$ chứa đường thẳng $SA$ song song với mặt phẳng $(P)$ nên mặt phẳng $(SAC)$ cắt mặt phẳng $(P)$ theo giao tuyến song song với $SA$. Vẽ $EH\parallel SA$ ($H\in SC$) thì $EH$ là giao tuyến của $(P)$ với mặt phẳng $(SAC)$.\\
	Khi đó giao tuyến của mặt phẳng $(SCD)$  với mặt phẳng $(P)$ là $NH$; $HK$ là giao tuyến của $(P)$ với $(SBC)$.\\
	Tứ giác $MNHK$  là thiết diện của $(P)$ với hình chóp $S.ABCD$.
	}
	{
	\begin{tikzpicture}[declare function={a=2.65;b=5;c=5;h=3;},line join=round]
		\path (0,0) coordinate (A)
		(-60:a) coordinate (B)
		(-35:c) coordinate (C)
		(b,0) coordinate (D)
		($(A)+(0.5,h)$) coordinate (S)
		($(A)!0.6!(B)$) coordinate (M)
		($(C)!0.7!(D)$) coordinate (N)
		($(S)!0.6!(B)$) coordinate (K)
		(intersection of A--C and M--N) coordinate (E)
		($(E)+(S)-(A)$) coordinate (x)
		(intersection of S--C and E--x) coordinate (H)
			;
		\fill [blue,opacity=0.3] (M)--(N)--(H)--(K);
		\draw[dashed] (A)--(D) (M)--(N) (A)--(C) (E)--(H);
		\draw (A)--(B)--(C)--(D)--(S)--(B) (S)--(A) (S)--(C) (N)--(H)--(K)--(M);
		\foreach \t/\g in {S/90,A/180,B/-90,C/-90,D/0,M/180,N/0,K/180,E/-90,H/90}{
			\draw[fill=black] (\t) circle (1pt) node[shift={(\g:7pt)},font=\scriptsize]{$ \t $};
		}
	\end{tikzpicture}
	}
	}
\end{vd}
% ví dụ 2
\begin{vd}%[Dự án Đề cương 11-Mui Doan]%[1T4K3-5]
	Cho hình chóp $S.ABCD$.  $M$, $N$ là hai điểm trên $SB$, $CD$. Mặt phẳng $(P)$ qua $MN$ và song song với $SC$.	Xác định thiết diện của $(P)$ với hình chóp.
	\loigiai{
	\immini{
	Mặt phẳng $(SBC)$ chứa đường thẳng $SC$ song song với mặt phẳng $(P)$ nên mặt phẳng $(SBC)$ cắt mặt phẳng $(P)$ theo giao tuyến song song với $SC$. Vẽ $MK\parallel SC$ ($K\in BC$) thì $MK$ là giao tuyến của $(P)$ với mặt phẳng $(SBC)$.\\
	Mặt phẳng $(SCD)$ chứa đường thẳng $SC$ song song với mặt phẳng $(P)$ nên mặt phẳng $(SCD)$ cắt mặt phẳng $(P)$ theo giao tuyến song song với $SC$. Vẽ $NP\parallel SC$ ($P\in SD$) thì $NP$ là giao tuyến của $(P)$ với mặt phẳng $(SCD)$.\\
	Trong $(ABCD)$ ta có $AC\cap NK=E\Rightarrow E\in (P)$.\\
	Mặt phẳng $(SAC)$ chứa đường thẳng $SC$ song song với mặt phẳng $(P)$ nên mặt phẳng $(SAC)$ cắt mặt phẳng $(P)$ theo giao tuyến song song với $SC$. Vẽ $EH\parallel SC$ ($H\in SA$) thì $EH$ là giao tuyến của $(P)$ với mặt phẳng $(SAC)$.\\
	Khi đó giao tuyến của mặt phẳng $(SAD)$  với mặt phẳng $(P)$ là $PH$; $HM$ là giao tuyến của $(P)$ với $(SAB)$.\\
	Vậy thiết diện của $(P)$ với hình chóp $S.ABCD$ là	ngũ giác $MKNPH$.
	}
	{
	\begin{tikzpicture}[declare function={a=2.65;b=5;c=5;h=3;},line join=round]
		\path (0,0) coordinate (A)
		(-60:a) coordinate (B)
		(-35:c) coordinate (C)
		(b,0) coordinate (D)
		($(A)+(0.5,h)$) coordinate (S)
		($(S)!0.4!(B)$) coordinate (M)
		($(C)!0.7!(D)$) coordinate (N)
		($(C)!0.4!(B)$) coordinate (K)
		($(S)!0.7!(D)$) coordinate (P)
		(intersection of A--C and N--K) coordinate (E)
		($(E)+(S)-(C)$) coordinate (x)
		(intersection of A--S and E--x) coordinate (H)
		;
		\fill [blue,opacity=0.3] (M)--(K)--(N)--(P)--(H);
		\draw[dashed] (A)--(D) (M)--(N) (A)--(C) (N)--(K) (E)--(H)--(P);
		\draw (A)--(B)--(C)--(D)--(S)--(B) (S)--(A) (S)--(C) (N)--(P) (H)--(M)--(K);
		\foreach \t/\g in {S/90,A/180,B/-90,C/-90,D/0,M/180,N/0,K/-90,P/90,E/0,H/180}{
			\draw[fill=black] (\t) circle (1pt) node[shift={(\g:7pt)},font=\scriptsize]{$ \t $};
		}
	\end{tikzpicture}	
	}	
	}
\end{vd}
%ví dụ 3

\begin{vd}%[Dự án Đề cương 11-Mui Doan]%[1T4K3-5]
	Cho tứ diện $ABCD$, điểm $E$ nẳm giữa hai điểm $A$ và $C$. Gọi $P$ là mặt phẳng qua $E$ và song song với hai đường thẳng $AB$, $CD$. Xác định các giao tuyến của $(P)$ và các mặt của tứ diện. Hình tạo bởi các giao tuyến là hình gì?
	\loigiai{
	\immini{
	Mặt phẳng $(ABC)$ chứa đường thẳng $AB$ song song với mặt phẳng $(P)$ nên mặt phẳng $(ABC)$ cắt mặt phẳng $(P)$ theo giao tuyến song song với $AB$. Vẽ $EF\parallel AB$ ($F\in BC$) thì $EF$ là giao tuyến của $(P)$ với mặt phẳng $(ABC)$.\\
	Hai mặt phẳng $(ACD)$ và $(BCD)$ cùng chứa đường thẳng $CD$ song song với mặt phẳng $(P)$ nên chúng cắt mặt phẳng $(P)$ theo giao tuyến song song với $CD$. Vẽ $EF$, $GH$ lần lượt là giao tuyến của mặt phẳng $(P)$ với hai mặt phẳng $(ACD)$ và $(BCD)$. Khi đó $GH$ là giao tuyến của $(P)$ với mặt phẳng $(ABD)$.\\
	Mặt phẳng $(ABD)$ chứa đường thẳng $AB$ song song với mặt phẳng $(P)$ nên giao tuyến $GH$ của $(ABD)$ và $(P)$ song song với $AB$. Tứ giác $EFGH$ có $EF\parallel GH$ (vì cùng song song với $AB$) và $EH\parallel FG$ (vì cùng song song với $CD$) nên nó là hình bình hành.
	}
	{
	\begin{tikzpicture}[declare function={a=2.65;b=4.5;h=3;},line join=round]
		\path (0,0) coordinate (B)
		(-60:a) coordinate (C)
		(b,0) coordinate (D)
		($(B)+(0.5,h)$) coordinate (A)
		($(A)!0.4!(C)$) coordinate (E)
		($(B)!0.4!(C)$) coordinate (F)
		($(B)!0.4!(D)$) coordinate (G)
		($(A)!0.4!(D)$) coordinate (H)
		;
		\fill [blue,opacity=0.3] (E)--(F)--(G)--(H);
		\draw[dashed] (B)--(D) (F)--(G)--(H);
		\draw (B)--(C)--(D) (B)--(A)--(C) (A)--(D) (F)--(E)--(H) ;
		\foreach \t/\g in {A/90,B/180,C/-90,D/0,E/180,F/180,G/-90,H/90}{
			\draw[fill=black] (\t) circle (1pt) node[shift={(\g:7pt)},font=\scriptsize]{$ \t $};
		}
	\end{tikzpicture}
	}
	}
\end{vd}	
%Ví dụ 4
\begin{vd}%[Dự án Đề cương 11-Mui Doan]%[1T4K3-5]
	Cho tứ diện $ABCD$. Gọi $I$, $J$ lần lượt là trung điểm của $AB$ và $CD$. Mặt phẳng $(P)$ đi qua một điểm $M$ trên đoạn $IJ$ và song song với $AB$ và $CD$. Xác định thiết diện của $(P)$ với tứ diện $ABCD$.
	\loigiai{
		\immini{
			Mặt phẳng $(ICD)$ chứa đường thẳng $CD$ song song với mặt phẳng $(P)$ nên mặt phẳng $(ICD)$ cắt mặt phẳng $(P)$ theo giao tuyến song song với $CD$. Qua $M$ vẽ $PQ\parallel CD$ ($P\in IC$; $Q\in ID$) thì $PQ$ là giao tuyến của $(P)$ với mặt phẳng $(ICD)$.\\
			Mặt phẳng $(ABC)$ chứa đường thẳng $AB$ song song với mặt phẳng $(P)$ nên mặt phẳng $(ABC)$ cắt mặt phẳng $(P)$ theo giao tuyến song song với $AB$. Qua $P$ vẽ $EF\parallel AB$ ($E\in BC$; $F\in AC$) thì $EF$ là giao tuyến của $(P)$ với mặt phẳng $(SCD)$.\\
			Mặt phẳng $(ACD)$ chứa đường thẳng $CD$ song song với mặt phẳng $(P)$ nên mặt phẳng $(ACD)$ cắt mặt phẳng $(P)$ theo giao tuyến song song với $CD$. Vẽ $FG\parallel CD$ ($G\in AD$) thì $FG$ là giao tuyến của $(P)$ với mặt phẳng~$(ACD)$.
					}
		{
			\begin{tikzpicture}[declare function={a=2.65;b=4.5;h=3;},line join=round]
				\path (0,0) coordinate (B)
				(-60:a) coordinate (C)
				(b,0) coordinate (D)
				($(B)+(0.5,h)$) coordinate (A)
				($(A)!0.5!(B)$) coordinate (I)
				($(D)!0.5!(C)$) coordinate (J)
				($(I)!0.5!(J)$) coordinate (M)
				($(M)+(D)-(J)$) coordinate (x)
				(intersection of I--C and M--x) coordinate (P)
				(intersection of I--D and M--x) coordinate (Q)
				($(P)+(B)-(I)$) coordinate (y)
				(intersection of A--C and P--y) coordinate (F)
				(intersection of B--C and P--y) coordinate (E)
				($(F)+(D)-(C)$) coordinate (z)
				(intersection of A--D and F--z) coordinate (G)
				(intersection of B--D and G--Q) coordinate (H)
				;
				\fill [blue,opacity=0.3] (E)--(F)--(G)--(H);
				\draw[dashed] (B)--(D) (I)--(J) (I)--(D) (P)--(Q) (E)--(H)--(G);
				\draw (B)--(C)--(D) (B)--(A)--(C) (A)--(D) (I)--(C) (E)--(F)--(G);
				\foreach \t/\g in {A/90,B/180,C/-90,D/0,I/180,J/-30,M/90,P/180,Q/30,E/180,F/160,G/90,H/-90}{
					\draw[fill=black] (\t) circle (1pt) node[shift={(\g:7pt)},font=\scriptsize]{$ \t $};
				}
			\end{tikzpicture}	
		}
	\noindent
	Mặt phẳng $(ABC)$ chứa đường thẳng $AB$ song song với mặt phẳng $(P)$ nên mặt phẳng $(BCD)$ cắt mặt phẳng $(P)$ theo giao tuyến song song với $CD$. Vẽ $EH\parallel CD$ ($H\in BD$) thì $EH$ là giao tuyến của $(P)$ với mặt phẳng~$(BCD)$.\\
	Khi đó giao tuyến của mặt phẳng $(ABD)$  với mặt phẳng $(P)$ là $GH$.\\
	Ta có $Q\in (ABD)\cap (P)\Rightarrow Q\in QH$; $EF\parallel GH$; $FG\parallel EH$.\\
	Vậy thiết diện của $(P)$ với tứ diện $ABCD$ là	hình bình hành $EFGH$.		
	}
\end{vd}
%Ví dụ 5
\begin{vd}%[Dự án Đề cương 11-Mui Doan]%[1T4K3-5]
	Cho hình chóp $S.ABCD$ có đáy $ABCD$ là hình bình hành. Gọi $O$ là giao điểm của hai đường chéo hình bình hành. Một mặt phẳng $(P)$ qua $O$ đồng thời song song với $SA$ và $CD$. Tìm thiết diện tạo bởi $(P)$ và hình chóp.
	\loigiai{
		\immini{
		Mặt phẳng $(SAC)$ chứa đường thẳng $SA$ song song với mặt phẳng $(P)$ nên mặt phẳng $(SAC)$ cắt mặt phẳng $(P)$ theo giao tuyến song song với $SA$. Vẽ $OP\parallel SA$ ($P\in SC$) thì $OP$ là giao tuyến của $(P)$ với mặt phẳng $(SAC)$.\\
		Mặt phẳng $(ABCD)$ chứa đường thẳng $CD$ song song với mặt phẳng $(P)$ nên mặt phẳng $(ABCD)$ cắt mặt phẳng $(P)$ theo giao tuyến song song với $CD$. Qua $O$ vẽ $MN\parallel CD$ ($M\in AD$; $N\in BC$) thì $MN$ là giao tuyến của $(P)$ với mặt phẳng $(ABCD)$.
				}{
		\begin{tikzpicture}[declare function={a=2.2;c=5;h=3.7;D=45;},line join=round]
			%	c: cạnh DC
			%	a: cạnh DA
			%h: đường cao
			%D: góc D của đáy
			\path
			(0,0) coordinate (D)
			(D:a) coordinate (A)
			(c,0) coordinate (C)
			($(A)+(C)-(D)$) coordinate (B)
			($(A)+(70:h)$)  coordinate (S)
			($(S)!0.5!(C)$)  coordinate (P)
			($(A)!0.5!(C)$)  coordinate (O)
			($(A)!0.5!(D)$)  coordinate (M)
			($(B)!0.5!(C)$)  coordinate (N)
			($(P)+(D)-(C)$) coordinate (x)
			(intersection of P--x and S--D) coordinate (Q)
			;
			\fill [blue,opacity=0.3] (M)--(N)--(P)--(Q);
			\draw[dashed] (B)--(A)--(D)--cycle (S)--(A)--(C) (Q)--(M)--(N) (O)--(P);
			\draw (S)--(B)--(C)--(D)--cycle (S)--(C) (N)--(P)--(Q);
			\foreach \t/\g in {A/180,B/0,C/0,D/180,S/90,P/60,O/-90,M/-90,N/0,Q/180}{
				\draw[fill=black] (\t) circle (1pt) node[shift={(\g:7pt)},font=\scriptsize]{$ \t $};
			}
		\end{tikzpicture}
		}
	\noindent
	Mặt phẳng $(SCD)$ chứa đường thẳng $CD$ song song với mặt phẳng $(P)$ nên mặt phẳng $(SCD)$ cắt mặt phẳng $(P)$ theo giao tuyến song song với $CD$. Vẽ $PQ\parallel CD$ ($Q\in SD$) thì $PQ$ là giao tuyến của $(P)$ với mặt phẳng $(SCD)$.\\
	Ta có $NP$ là giao tuyến của mặt phẳng $(P)$ với $(SBC)$.\\
	Tứ giác $MPNQ$ có $MN\parallel PQ$ (vì cùng song song với $CD$)  nên nó là hình thang.	
	}
\end{vd}
\subsubsection{Bài tập rèn luyện}
%Bài 1
\begin{bt}%[Dự án Đề cương 11-Mui Doan]%[1T4K3-5]
	Cho tứ diện $ABCD$. Gọi $M$ là điểm bất kỳ trên cạnh $BC$. $(P)$ là mặt phẳng qua $M$ và song song với hai đường thẳng $AB$, $CD$. Xác định thiết diện của $(P)$ với tứ diện $ABCD$.
	\loigiai{
	\immini{
	Mặt phẳng $(ABC)$ chứa đường thẳng $AB$ song song với mặt phẳng $(P)$ nên mặt phẳng $(ABC)$ cắt mặt phẳng $(P)$ theo giao tuyến song song với $AB$. Vẽ $MN\parallel AB$ ($N\in AC$) thì $MN$ là giao tuyến của $(P)$ với mặt phẳng $(ABC)$.\\
	Hai mặt phẳng $(ACD)$ và $(BCD)$ cùng chứa đường thẳng $CD$ song song với mặt phẳng $(P)$ nên chúng cắt mặt phẳng $(P)$ theo giao tuyến song song với $CD$. Vẽ $MQ$, $NP$ lần lượt là giao tuyến của mặt phẳng $(P)$ với hai mặt phẳng $(ACD)$ và $(BCD)$. Khi đó $PQ$ là giao tuyến của $(P)$ với mặt phẳng $(ABD)$.\\
	Mặt phẳng $(ABD)$ chứa đường thẳng $AB$ song song với mặt phẳng $(P)$ nên giao tuyến $PQ$ của $(ABD)$ và $(P)$ song song với $AB$. 
	}
	{
	\begin{tikzpicture}[declare function={a=2.65;b=4.5;h=3;},line join=round]
		\path (0,0) coordinate (B)
		(-60:a) coordinate (C)
		(b,0) coordinate (D)
		($(B)+(0.5,h)$) coordinate (A)
		($(A)!0.4!(C)$) coordinate (N)
		($(B)!0.4!(C)$) coordinate (M)
		($(B)!0.4!(D)$) coordinate (Q)
		($(A)!0.4!(D)$) coordinate (P)
		;
		\fill [blue,opacity=0.3] (N)--(M)--(Q)--(P);
		\draw[dashed] (B)--(D) (M)--(Q)--(H);
		\draw (B)--(C)--(D) (B)--(A)--(C) (A)--(D) (M)--(N)--(P) ;
		\foreach \t/\g in {A/90,B/180,C/-90,D/0,N/180,M/180,Q/-90,P/90}{
			\draw[fill=black] (\t) circle (1pt) node[shift={(\g:7pt)},font=\scriptsize]{$ \t $};
		}
	\end{tikzpicture}
	}
	\noindent
	Tứ giác $MNPQ$ có $MN\parallel PQ$ (vì cùng song song với $AB$) và $NP\parallel MQ$ (vì cùng song song với $CD$) nên nó là hình bình hành.
	}
\end{bt}

%Bài 2
\begin{bt}%[Dự án Đề cương 11-Mui Doan]%[1T4K3-5]
	Cho tứ diện $ABCD$ và điểm $M$ thuộc  cạnh $AB$. $(P)$ là mặt phẳng qua $M$, song song với hai đường thẳng $BC$ và $AD$. Xác định thiết diện của $(P)$ với tứ diện $ABCD$.
	\loigiai{
	\immini{
		Mặt phẳng $(ABC)$ chứa đường thẳng $BC$ song song với mặt phẳng $(P)$ nên mặt phẳng $(ABC)$ cắt mặt phẳng $(P)$ theo giao tuyến song song với $BC$. Vẽ $MN\parallel BC$ ($N\in AC$) thì $MN$ là giao tuyến của $(P)$ với mặt phẳng $(ABC)$.\\
		Mặt phẳng $(ABD)$ chứa đường thẳng $AD$ song song với mặt phẳng $(P)$ nên mặt phẳng $(ABD)$ cắt mặt phẳng $(P)$ theo giao tuyến song song với $AD$. Vẽ $MQ\parallel AD$ ($Q\in BD$) thi $MQ$ là giao tuyến của mặt phẳng $(P)$ với mặt phẳng $(ABD)$.\\
		Mặt phẳng $(BCD)$ chứa đường thẳng $BC$ song song với mặt phẳng $(P)$ nên mặt phẳng $(BCD)$ cắt mặt phẳng $(P)$ theo giao tuyến song song với $BC$. Vẽ $QP\parallel BC$ ($P\in CD$) thì $PQ$ là giao tuyến của $(P)$ với mặt phẳng $(BCD)$.
			}
	{
		\begin{tikzpicture}[declare function={a=2.65;b=4.5;h=3;},line join=round]
			\path (0,0) coordinate (B)
			(-60:a) coordinate (C)
			(b,0) coordinate (D)
			($(B)+(0.5,h)$) coordinate (A)
			($(A)!0.4!(B)$) coordinate (M)
			($(A)!0.4!(C)$) coordinate (N)
			($(D)!0.4!(B)$) coordinate (Q)
			($(D)!0.4!(C)$) coordinate (P)
			;
			\fill [blue,opacity=0.3] (N)--(M)--(Q)--(P);
			\draw[dashed] (B)--(D) (M)--(Q)--(P);
			\draw (B)--(C)--(D) (B)--(A)--(C) (A)--(D) (M)--(N)--(P) ;
			\foreach \t/\g in {A/90,B/180,C/-90,D/0,M/180,N/180,Q/90,P/0}{
				\draw[fill=black] (\t) circle (1pt) node[shift={(\g:7pt)},font=\scriptsize]{$ \t $};
			}
		\end{tikzpicture}
	}	
	\noindent
	 Khi đó $NP$ là giao tuyến của $(P)$ với mặt phẳng $(ACD)$.\\
	Mặt phẳng $(ACD)$ chứa đường thẳng $AD$ song song với mặt phẳng $(P)$ nên giao tuyến $NP$ của $(ACD)$ và $(P)$ song song với $AD$. Tứ giác $MNPQ$ có $MN\parallel PQ$ (vì cùng song song với $BC$) và $NP\parallel MQ$ (vì cùng song song với $AD$) nên nó là hình bình hành.
	}
\end{bt}
%Bài 3
	\begin{bt}%[Dự án Đề cương 11-Mui Doan]%[1T4K3-5]
		Cho tứ diện $ABCD$. Gọi $M$ là trung điểm cạnh $AD$; $N$ là điểm bất kỳ trên cạnh $BC$. Gọi $(P)$ là mặt phẳng chứa đường thẳng $MN$ và song song với $CD$. Xác định thiết diện của  $(P)$ với tứ diện $ABCD$.
		\loigiai{
		\immini{
	Mặt phẳng $(ACD)$ chứa đường thẳng $CD$ song song với mặt phẳng $(P)$ nên mặt phẳng $(ACD)$ cắt mặt phẳng $(P)$ theo giao tuyến song song với $CD$. Vẽ $MP\parallel CD$ ($P\in CD$) thì $MP$ là giao tuyến của $(P)$ với mặt phẳng $(ACD)$.\\
	Mặt phẳng $(BCD)$ chứa đường thẳng $CD$ song song với mặt phẳng $(P)$ nên mặt phẳng $(BCD)$ cắt mặt phẳng $(P)$ theo giao tuyến song song với $CD$. Vẽ $NQ\parallel CD$ ($Q\in BD$) thì $NQ$ là giao tuyến của $(P)$ với mặt phẳng $(BCD)$.\\
	Khi đó $MQ$ là giao tuyến của $(P)$ với $(ABD)$; $NP$ là giao tuyến của $(P)$ với $(ABC)$.\\
	Tứ giác $MPNQ$ có $MP\parallel NQ$ (vì cùng song song với $CD$)  nên nó là hình thang.	
	}
	{
	\begin{tikzpicture}[declare function={a=2.65;b=4.5;h=3;},line join=round]
		\path (0,0) coordinate (B)
		(-60:a) coordinate (C)
		(b,0) coordinate (D)
		($(B)+(0.5,h)$) coordinate (A)
		($(A)!0.5!(D)$) coordinate (M)
		($(B)!0.6!(C)$) coordinate (N)
		($(A)!0.5!(C)$) coordinate (P)
		($(N)+(D)-(C)$) coordinate (x)
		(intersection of N--x and B--D) coordinate (Q)
		;
		\fill [blue,opacity=0.3] (M)--(P)--(N)--(Q);
		\draw[dashed] (B)--(D) (M)--(N) (N)--(Q)--(M);
		\draw (B)--(C)--(D) (B)--(A)--(C) (A)--(D) (N)--(P)--(M);
		\foreach \t/\g in {A/90,B/180,C/-90,D/0,M/30,N/180,P/180,Q/-90}{
			\draw[fill=black] (\t) circle (1pt) node[shift={(\g:7pt)},font=\scriptsize]{$ \t $};
		}
	\end{tikzpicture}
	}
	}
	\end{bt}

%Bài 4
\begin{bt}%[Dự án Đề cương 11-Mui Doan]%[1T4K3-5]
	Cho tứ diện $ABCD$. Gọi $M$ là điểm nằm trong tam giác  $ABC$, $(P)$ là mặt phẳng đi qua $M$ và song song với các đường thẳng $AB$ và $CD$. Xác định thiết diện của  $(P)$ với tứ diện $ABCD$.
	\loigiai{
		\immini{
		Trong $(ABC)$ ta có $ AM\cap BC=N$.\\
		Mặt phẳng $(ABC)$ chứa đường thẳng $AB$ song song với mặt phẳng $(P)$ nên mặt phẳng $(ABC)$ cắt mặt phẳng $(P)$ theo giao tuyến song song với $AB$. Qua $M$ vẽ $HK\parallel AB$ ($H\in BC$; $K\in AC$) thì $HK$ là giao tuyến của $(P)$ với mặt phẳng $(ABC)$.\\
		Mặt phẳng $(ACD)$ chứa đường thẳng $CD$ song song với mặt phẳng $(P)$ nên mặt phẳng $(ACD)$ cắt mặt phẳng $(P)$ theo giao tuyến song song với $CD$. Vẽ $KE\parallel CD$ ($E\in AD$) thì $KE$ là giao tuyến của $(P)$ với mặt phẳng $(ACD)$.\\
		Mặt phẳng $(BCD)$ chứa đường thẳng $CD$ song song với mặt phẳng $(P)$ nên mặt phẳng $(BCD)$ cắt mặt phẳng $(P)$ theo giao tuyến song song với $CD$. Vẽ $HF\parallel CD$ ($F\in BD$) thì $HF$ là giao tuyến của $(P)$ với mặt phẳng $(BCD)$.
				}
		{
			\begin{tikzpicture}[declare function={a=2.65;b=4.5;h=3;},line join=round]
				\path (0,0) coordinate (B)
				(-60:a) coordinate (C)
				(b,0) coordinate (D)
				($(B)+(0.5,h)$) coordinate (A)
				($(B)+(0.5,0.5)$) coordinate (M)
				(intersection of A--M and B--C) coordinate (N)
				($(M)+(B)-(A)$) coordinate (x)
				(intersection of M--x and B--C) coordinate (H)
				(intersection of M--x and A--C) coordinate (K)
				($(K)+(D)-(C)$) coordinate (y)
				(intersection of K--y and A--D) coordinate (E)
				($(H)+(D)-(C)$) coordinate (z)
				(intersection of H--z and B--D) coordinate (F)
				;
				\fill [blue,opacity=0.3] (H)--(K)--(E)--(F);
				\draw[dashed] (B)--(D) (H)--(F)--(E);
				\draw (B)--(C)--(D) (B)--(A)--(C) (A)--(D) (A)--(N) (H)--(K)--(E);
				\foreach \t/\g in {A/90,B/180,C/-90,D/0,M/180,N/-100,H/180,K/70,E/90,F/-90}{
					\draw[fill=black] (\t) circle (1pt) node[shift={(\g:7pt)},font=\scriptsize]{$ \t $};
				}
			\end{tikzpicture}
		}
	\noindent
	Tứ giác $KEFH$ có $KE\parallel HF$ (vì cùng song song với $CD$) và $HK\parallel FE$ (vì cùng song song với $AB$) nên nó là hình bình hành.	
	}
\end{bt}
%Bài 5
\begin{bt}%[Dự án Đề cương 11-Mui Doan]%[1T4K3-5]
	Cho tứ diện $ABCD$. Gọi $M$ là điểm nằm trong tam giác  $BCD$, $(P)$ là mặt phẳng đi qua $M$ và song song với các đường thẳng $AB$ và $CD$. Xác định thiết diện của  $(P)$ với tứ diện $ABCD$.
	\loigiai{
		\immini{
			Trong $(BCD)$ ta có $ BM\cap CD=N$.\\
			Mặt phẳng $(BCD)$ chứa đường thẳng $CD$ song song với mặt phẳng $(P)$ nên mặt phẳng $(BCD)$ cắt mặt phẳng $(P)$ theo giao tuyến song song với $CD$. Qua $M$ vẽ $HK\parallel CD$ ($H\in BC$; $K\in BD$) thì $HK$ là giao tuyến của $(P)$ với mặt phẳng $(BCD)$.\\
			Mặt phẳng $(ABD)$ chứa đường thẳng $AB$ song song với mặt phẳng $(P)$ nên mặt phẳng $(ABD)$ cắt mặt phẳng $(P)$ theo giao tuyến song song với $AB$. Vẽ $KE\parallel AB$ ($E\in AD$) thì $KE$ là giao tuyến của $(P)$ với mặt phẳng $(ABD)$.\\
			Mặt phẳng $(ABC)$ chứa đường thẳng $AB$ song song với mặt phẳng $(P)$ nên mặt phẳng $(ABC)$ cắt mặt phẳng $(P)$ theo giao tuyến song song với $AB$. Vẽ $HF\parallel AB$ ($F\in AC$) thì $HF$ là giao tuyến của $(P)$ với mặt phẳng $(ABC)$.
		}
		{
			\begin{tikzpicture}[declare function={a=2.65;b=4.5;h=3;},line join=round]
				\path (0,0) coordinate (B)
				(-60:a) coordinate (C)
				(b,0) coordinate (D)
				($(B)+(0.5,h)$) coordinate (A)
				($(C)+(0.6,1.2)$) coordinate (M)
				(intersection of B--M and D--C) coordinate (N)
				($(M)+(C)-(D)$) coordinate (x)
				(intersection of M--x and B--C) coordinate (H)
				(intersection of M--x and B--D) coordinate (K)
				($(K)+(A)-(B)$) coordinate (y)
				(intersection of K--y and A--D) coordinate (E)
				($(H)+(A)-(B)$) coordinate (z)
				(intersection of H--z and A--C) coordinate (F)
				;
				\fill [blue,opacity=0.3] (H)--(K)--(E)--(F);
				\draw[dashed] (B)--(D) (B)--(N) (H)--(K)--(E);
				\draw (B)--(C)--(D) (B)--(A)--(C) (A)--(D) (H)--(F)--(E) ;
				\foreach \t/\g in {A/90,B/180,C/-90,D/0,M/-90,N/0,H/180,K/-90,E/90,F/180}{
					\draw[fill=black] (\t) circle (1pt) node[shift={(\g:7pt)},font=\scriptsize]{$ \t $};
				}
			\end{tikzpicture}
		}
		\noindent
		Tứ giác $KEFH$ có $KE\parallel HF$ (vì cùng song song với $AB$) và $HK\parallel FE$ (vì cùng song song với $CD$) nên nó là hình bình hành.	
	}
\end{bt}

%Bài 6
\begin{bt}%[Dự án Đề cương 11-Mui Doan]%[1T4K3-5]
	Cho hình chóp $S.ABCD$ có đáy $ABCD$ là hình bình hành. Gọi $M$ là điểm trên cạnh $SC$ và $(P)$ là mặt phẳng chứa $AM$ và song song với $BD$. Xác định thiết diện của  $(P)$ với hình chóp $S.ABCD$.
	\loigiai{
	\immini{
	Gọi $O=AC\cap BD$.\\
	Trong $(SAC)$ ta có $AM\cap SO=K$.\\
	Mặt phẳng $(SBD)$ chứa đường thẳng $BD$ song song với mặt phẳng $(P)$ nên mặt phẳng $(SBD)$ cắt mặt phẳng $(P)$ theo giao tuyến song song với $BD$. Qua $K$ vẽ $EF\parallel BD$ ($E\in SB$; $F\in SD$) thì $EF$ là giao tuyến của $(P)$ với mặt phẳng $(SBD)$.\\
	Khi đó $AF$ là giao tuyến của $(P)$ với $(SAD)$; $FM$ là giao tuyến của $(P)$ với $(SCD)$; $MF$ là giao tuyến của $(P)$ với $(SBC)$; $AE$ là giao tuyến của $(P)$ với $(SAB)$.\\
		}	
	{
	\begin{tikzpicture}[declare function={a=2.2;c=5;h=3.7;D=45;},line join=round]
		%	c: cạnh DC
		%	a: cạnh DA
		%h: đường cao
		%D: góc D của đáy
		\path
		(0,0) coordinate (D)
		(D:a) coordinate (A)
		(c,0) coordinate (C)
		($(A)+(C)-(D)$) coordinate (B)
		($(A)+(70:h)$)  coordinate (S)
		($(S)!0.4!(C)$)  coordinate (M)
		 ($(A)!0.5!(C)$)  coordinate (O)
		 (intersection of A--M and S--O) coordinate (K)
		 ($(K)+(B)-(O)$) coordinate (x)
		 (intersection of K--x and S--B) coordinate (E)
		 (intersection of K--x and S--D) coordinate (F)
		 ;
		 \fill [blue,opacity=0.3] (A)--(F)--(M)--(E);
		 \draw[dashed] (B)--(A)--(D)--cycle (S)--(A)--(C) (S)--(O) (A)--(M) (E)--(F)--(A)--(E);
		\draw (S)--(B)--(C)--(D)--cycle (S)--(C) (F)--(M)--(E);
		\foreach \t/\g in {A/180,B/0,C/0,D/180,S/90,M/60,O/-90,K/-30,E/0,F/180}{
			\draw[fill=black] (\t) circle (1pt) node[shift={(\g:7pt)},font=\scriptsize]{$ \t $};
		}
	\end{tikzpicture}
	}
\noindent
	Thiết diện của $(P)$ với hình chóp $S.ABCD$ là tứ giác $AFME$.
	}
\end{bt}

%Bài 7
\begin{bt}%[Dự án Đề cương 11-Mui Doan]%[1T4K3-5]
	Cho hình chóp $S.ABCD$ có đáy $ABCD$ là hình chữ nhật tâm $O$. $M$ là trung điểm $OC$. $(P)$ là mặt phẳng qua $M$ và song song với $SA$ và $BD$. Xác định thiết diện của  $(P)$ với hình chóp $S.ABCD$.
	\loigiai{
	\immini{
	Mặt phẳng $(SAC)$ chứa đường thẳng $SA$ song song với mặt phẳng $(P)$ nên mặt phẳng $(SAC)$ cắt mặt phẳng $(P)$ theo giao tuyến song song với $SA$. Vẽ $MN\parallel SA$ ($N\in SC$) thì $MN$ là giao tuyến của $(P)$ với mặt phẳng $(SAC)$.\\
	Mặt phẳng $(ABCD)$ chứa đường thẳng $BD$ song song với mặt phẳng $(P)$ nên mặt phẳng $(ABCD)$ cắt mặt phẳng $(P)$ theo giao tuyến song song với $BD$. Qua $M$ vẽ $PQ\parallel BD$ ($P\in DC$, $Q\in BC$) thì $PQ$ là giao tuyến của $(P)$ với mặt phẳng $(ABCD)$.
		}	
	{
	\begin{tikzpicture}[declare function={a=2.2;c=5;h=3.7;D=45;},line join=round]
		%	c: cạnh DC
		%	a: cạnh DA
		%h: đường cao
		%D: góc D của đáy
		\path
		(0,0) coordinate (D)
		(D:a) coordinate (A)
		(c,0) coordinate (C)
		($(A)+(C)-(D)$) coordinate (B)
		($(A)+(70:h)$)  coordinate (S)
		($(O)!0.5!(C)$)  coordinate (M)
		($(A)!0.5!(C)$)  coordinate (O)
		($(M)+(S)-(A)$) coordinate (x)
		(intersection of M--x and S--C) coordinate (N)
		($(M)+(B)-(O)$) coordinate (y)
		(intersection of M--y and C--D) coordinate (P)
		(intersection of M--y and C--B) coordinate (Q)
		;
		\fill [blue,opacity=0.3] (N)--(P)--(Q);
		\draw[dashed] (B)--(A)--(D)--cycle (S)--(A)--(C) (S)--(O) (P)--(Q) (M)--(N);
		\draw (S)--(B)--(C)--(D)--cycle (S)--(C) (P)--(N)--(Q) ;
		\foreach \t/\g in {A/180,B/0,C/0,D/180,S/90,M/-90,O/-120,N/30,P/-90,Q/0}{
			\draw[fill=black] (\t) circle (1pt) node[shift={(\g:7pt)},font=\scriptsize]{$ \t $};
		}
	\end{tikzpicture}	
	}	
	\noindent
	Khi đó $MP$ là giao tuyến của $(P)$ với $(SCD)$; $NQ$ là giao tuyến của $(P)$ với $(SBC)$. \\
	Vậy thiết diện là $\triangle NPQ$.
	}
\end{bt}
%Bài 8
\begin{bt}%[Dự án Đề cương 11-Mui Doan]%[1T4K3-5]
	Cho hình chóp $S.ABCD$ có đáy $ABCD$ là hình bình hành. $M$ là điểm di động trên đoạn $AB$. $(P)$ là mặt phẳng qua $M$ và song song với $SA$ và $BC$. Xác định thiết diện của  $(P)$ với hình chóp $S.ABCD$.
	\loigiai{
	\immini{
	Mặt phẳng $(SAB)$ chứa đường thẳng $SA$ song song với mặt phẳng $(P)$ nên mặt phẳng $(SAB)$ cắt mặt phẳng $(P)$ theo giao tuyến song song với $SA$. Vẽ $MN\parallel SA$ ($N\in SB$) thì $MN$ là giao tuyến của $(P)$ với mặt phẳng $(SAB)$.\\
	Hai mặt phẳng $(SBC)$ và $(ABCD)$ cùng chứa đường thẳng $BC$ song song với mặt phẳng $(P)$ nên chúng cắt mặt phẳng $(P)$ theo giao tuyến song song với $BC$. Vẽ $NP$, $MQ$ lần lượt là giao tuyến của mặt phẳng $(P)$ với hai mặt phẳng $(SBC)$ và $(ABCD)$. Khi đó $PQ$ là giao tuyến của $(P)$ với mặt phẳng $(SCD)$.
			}	
	{
	\begin{tikzpicture}[declare function={a=2.2;c=5;h=3.7;D=45;},line join=round]
		%	c: cạnh DC
		%	a: cạnh DA
		%h: đường cao
		%D: góc D của đáy
		\path
		(0,0) coordinate (D)
		(D:a) coordinate (A)
		(c,0) coordinate (C)
		($(A)+(C)-(D)$) coordinate (B)
		($(A)+(70:h)$)  coordinate (S)
		($(A)!0.6!(B)$)  coordinate (M)
		($(M)+(S)-(A)$)  coordinate (x)
		(intersection of M--x and S--B) coordinate (N)
		($(D)!0.6!(C)$)  coordinate (Q)
		($(N)+(C)-(B)$) coordinate (y)
		(intersection of N--y and S--C) coordinate (P)
		;
		\fill [blue,opacity=0.3] (M)--(N)--(P)--(Q);
		\draw[dashed] (B)--(A)--(D) (S)--(A) (N)--(M)--(Q);
		\draw (S)--(B)--(C)--(D)--cycle (S)--(C) (N)--(P)--(Q);
		\foreach \t/\g in {A/180,B/0,C/0,D/180,S/90,M/-60,N/90,Q/-90,P/180}{
			\draw[fill=black] (\t) circle (1pt) node[shift={(\g:7pt)},font=\scriptsize]{$ \t $};
		}
	\end{tikzpicture}
	}
	\noindent
	Mặt phẳng $(SBC)$ chứa đường thẳng $BC$ song song với mặt phẳng $(P)$ nên giao tuyến $NP$ của $(SBC)$ và $(P)$ song song với $BC$. \\
	Mặt phẳng $(ABCD)$ chứa đường thẳng $BC$ song song với mặt phẳng $(P)$ nên giao tuyến $MQ$ của $(ABCD)$ và $(P)$ song song với $BC$. \\
	Tứ giác $MNPQ$ có $MQ\parallel NP$ (vì cùng song song với $BC$)  nên nó là hình thang.
	}
\end{bt}
%Bài 9
\begin{bt}%[Dự án Đề cương 11-Mui Doan]%[1T4K3-5]
		Cho hình chóp $S.ABCD$ có đáy $ABCD$ là hình bình hành. $M$ là điểm thuộc đoạn $SA$. $(P)$ là mặt phẳng qua $M$ và song song với $SC$ và $AD$. Xác định thiết diện của  $(P)$ với hình chóp $S.ABCD$.
	\loigiai{
		\immini{
		Mặt phẳng $(SAC)$ chứa đường thẳng $SC$ song song với mặt phẳng $(P)$ nên mặt phẳng $(SAC)$ cắt mặt phẳng $(P)$ theo giao tuyến song song với $SC$. Vẽ $MK\parallel SC$ ($K\in AC$) thì $MK$ là giao tuyến của $(P)$ với mặt phẳng $(SAC)$.\\
		Mặt phẳng $(SAD)$ chứa đường thẳng $AD$ song song với mặt phẳng $(P)$ nên mặt phẳng $(SAD)$ cắt mặt phẳng $(P)$ theo giao tuyến song song với $AD$. Vẽ $MQ\parallel AD$ ($Q\in SD$) thì $MQ$ là giao tuyến của mặt phẳng $(P)$ với mặt phẳng $(SAD)$.
		}	
		{
		\begin{tikzpicture}[declare function={a=2.2;c=5;h=3.7;D=45;},line join=round]
			%	c: cạnh DC
			%	a: cạnh DA
			%h: đường cao
			%D: góc D của đáy
			\path
			(0,0) coordinate (D)
			(D:a) coordinate (A)
			(c,0) coordinate (C)
			($(A)+(C)-(D)$) coordinate (B)
			($(A)+(70:h)$)  coordinate (S)
			($(S)!0.7!(A)$)  coordinate (M)
			($(C)!0.7!(A)$)  coordinate (K)
			($(K)+(D)-(A)$) coordinate (x)
			(intersection of K--x and A--B) coordinate (N)
			(intersection of K--x and C--D) coordinate (P)
			($(P)+(S)-(C)$) coordinate (y)
			(intersection of P--y and S--D) coordinate (Q)
			;
			\fill [blue,opacity=0.3] (M)--(N)--(P)--(Q);
			\draw[dashed] (B)--(A)--(D) (S)--(A)--(C) (M)--(K) (Q)--(M)--(N)--(P);
			\draw (S)--(B)--(C)--(D)--cycle (S)--(C) (P)--(Q);
			\foreach \t/\g in {A/180,B/0,C/0,D/180,S/90,M/30,K/-90,N/40,P/-90,Q/160}{
				\draw[fill=black] (\t) circle (1pt) node[shift={(\g:7pt)},font=\scriptsize]{$ \t $};
			}
		\end{tikzpicture}	
		}
	\noindent
	Mặt phẳng $(ABCD)$ chứa đường thẳng $AD$ song song với mặt phẳng $(P)$ nên mặt phẳng $(ABCD)$ cắt mặt phẳng $(P)$ theo giao tuyến song song với $AD$. Qua $K$ vẽ $NP\parallel AD$ ($N\in AB$; $P\in CD$) thì $NP$ là giao tuyến của mặt phẳng $(P)$ với mặt phẳng $(ABCD)$.\\
	Khi đó $MN$ là giao tuyến của mặt phẳng $(P)$ với mặt phẳng $(SAB)$.\\
	Tứ giác $MNPQ$ có $MQ\parallel NP$ (vì cùng song song với $AD$)  nên nó là hình thang.	
	}
\end{bt}
%Bài 10
\begin{bt}%[Dự án Đề cương 11-Mui Doan]%[1T4K3-5]
	Cho hình chóp $S.ABCD$ có đáy $ABCD$ là hình bình hành. $M$ là điểm di động trên cạnh $AD$. $(P)$ là mặt phẳng qua $M$ và song song với $SA$ và $CD$. Xác định thiết diện của  $(P)$ với hình chóp $S.ABCD$.
	\loigiai{
	\immini{
	Mặt phẳng $(SAD)$ chứa đường thẳng $SA$ song song với mặt phẳng $(P)$ nên mặt phẳng $(SAD)$ cắt mặt phẳng $(P)$ theo giao tuyến song song với $SA$. Vẽ $MN\parallel SA$ ($N\in SD$) thì $MN$ là giao tuyến của $(P)$ với mặt phẳng $(SAD)$.\\
	Mặt phẳng $(SCD)$ chứa đường thẳng $CD$ song song với mặt phẳng $(P)$ nên mặt phẳng $(SCD)$ cắt mặt phẳng $(P)$ theo giao tuyến song song với $CD$. Vẽ $NP\parallel CD$ ($P\in SC$) thì $NP$ là giao tuyến của mặt phẳng $(P)$ với mặt phẳng $(SCD)$.
	}	
	{
	\begin{tikzpicture}[declare function={a=2.2;c=5;h=3.7;D=45;},line join=round]
		%	c: cạnh DC
		%	a: cạnh DA
		%h: đường cao
		%D: góc D của đáy
		\path
		(0,0) coordinate (D)
		(D:a) coordinate (A)
		(c,0) coordinate (C)
		($(A)+(C)-(D)$) coordinate (B)
		($(A)+(70:h)$)  coordinate (S)
		($(A)!0.4!(D)$)  coordinate (M)
		($(B)!0.4!(C)$)  coordinate (Q)
		($(M)+(S)-(A)$) coordinate (x)
		(intersection of M--x and S--D) coordinate (N)
		($(N)+(C)-(D)$) coordinate (y)
		(intersection of N--y and S--C) coordinate (P)
		;
		\fill [blue,opacity=0.3] (M)--(N)--(P)--(Q);
		\draw[dashed] (B)--(A)--(D) (S)--(A)--(C) (Q)--(M)--(N);
		\draw (S)--(B)--(C)--(D)--cycle (S)--(C) (N)--(P)--(Q) ;
		\foreach \t/\g in {A/180,B/0,C/0,D/180,S/90,M/-90,Q/0,N/90,P/0}{
			\draw[fill=black] (\t) circle (1pt) node[shift={(\g:7pt)},font=\scriptsize]{$ \t $};
		}
	\end{tikzpicture}	
	}
	\noindent
	Mặt phẳng $(ABCD)$ chứa đường thẳng $CD$ song song với mặt phẳng $(P)$ nên mặt phẳng $(ABCD)$ cắt mặt phẳng $(P)$ theo giao tuyến song song với $CD$. Vẽ $MQ\parallel CD$ ($Q\in BC$) thì $MQ$ là giao tuyến của mặt phẳng $(P)$ với mặt phẳng $(ABCD)$.\\
	Khi đó $PQ$ là giao tuyến của mặt phẳng $(P)$ với mặt phẳng $(SBC)$.\\
	Tứ giác $MNPQ$ có $MQ\parallel NP$ (vì cùng song song với $CD$)  nên nó là hình thang.	
	}
\end{bt}
% Bài 11
\begin{bt}%[Dự án Đề cương 11-Mui Doan]%[1T4K3-5]
	Cho hình chóp $S.ABCD$ có đáy là hình thang $(AB\parallel CD)$. Gọi $E$ là một điểm nằm giữa $S$ và $A$. Gọi $(P)$ là mặt phẳng qua $E$ và song song với hai đường thẳng $AB$, $AD$. Xác định giao tuyến của $(P)$ và các mặt bên của hình chóp. Hình tạo bởi các giao tuyến là hình gì?
	\loigiai{
		\immini{
			Mặt phẳng $(SAB)$ chứa đường thẳng $AB$ song song với mặt phẳng $(P)$ nên mặt phẳng $(SAB)$ cắt mặt phẳng $(P)$ theo giao tuyến song song với $AB$. Vẽ $EF\parallel AB$ ($F\in SB$) thì $EF$ là giao tuyến của $(P)$ với mặt phẳng $(SAB)$.\\
			Mặt phẳng $(SAD)$ chứa đường thẳng $AD$ song song với mặt phẳng $(P)$ nên mặt phẳng $(SAD)$ cắt mặt phẳng $(P)$ theo giao tuyến song song với $AD$. \\Vẽ $EH\parallel AD$ ($H\in SD$) thì $EH$ là giao tuyến của $(P)$ với mặt phẳng $(SAD)$.
			}
		{
			\begin{tikzpicture}[declare function={a=2.65;b=5;h=3;},line join=round]
				\path (0,0) coordinate (A)
				(b,0) coordinate (B)
				(-30:a) coordinate (C)
				($(A)+(C)-(B)$) coordinate (x)
				($(x)!0.7!(C)$) coordinate (D)
				($(A)+(1,h)$) coordinate (S)
				($(S)!0.4!(A)$) coordinate (E)
				($(S)!0.4!(B)$) coordinate (F)
				($(S)!0.4!(D)$) coordinate (H)
				($(S)!0.4!(C)$) coordinate (G)
				;
				\fill [blue,opacity=0.3] (E)--(F)--(G)--(H);
				\draw[dashed] (A)--(B) (E)--(F);
				\draw (B)--(C)--(D)--(A) (B)--(S)  (D)--(S)--(C) (S)--(A) (E)--(H)--(G)--(F);
				\foreach \t/\g in {A/180,B/0,C/-90,D/-90,S/90,E/180,F/0,H/200,G/-30}{
					\draw[fill=white] (\t) circle (1pt) node[shift={(\g:7pt)},font=\scriptsize]{$ \t $};
				}
			\end{tikzpicture}
		}
		\noindent
		Mặt phẳng $(SCD)$ chứa đường thẳng $CD\parallel AB$ song song với mặt phẳng $(P)$ nên mặt phẳng $(SCD)$ cắt mặt phẳng $(P)$ theo giao tuyến song song với $CD$. Vẽ $HG\parallel CD$ ($G\in SC$) thì $HG$ là giao tuyến của $(P)$ với mặt phẳng $(SCD)$. Khi đó $GF$ là giao tuyến của $(P)$ với $(SBC)$.\\
		Tứ giác $EFGH$ có $EF\parallel GH$ (vì cùng song song với $AB$)  nên nó là hình thang.
	}
\end{bt}
%Bài 12
\begin{bt}%[Dự án Đề cương 11-Mui Doan]%[1T4K3-5]
	Cho hình chóp $S.ABCD$ có đáy $ABC$ là hình thang đáy lớn là $AB$. Gọi $M$ là một điểm trên cạnh $CD$. $(P)$ là mặt phẳng qua $M$ và song song với $SA$ và $BC$. Xác định thiết diện của  $(P)$ với hình chóp $S.ABCD$.
	\loigiai{
		\immini{
			Mặt phẳng $(ABCD)$ chứa đường thẳng $BC$ song song với mặt phẳng $(P)$ nên mặt phẳng $(ABCD)$ cắt mặt phẳng $(P)$ theo giao tuyến song song với $BC$. Vẽ $MN\parallel BC$ ($N\in AB$) thì $MN$ là giao tuyến của $(P)$ với mặt phẳng $(ABCD)$.\\
			Mặt phẳng $(SAB)$ chứa đường thẳng $SA$ song song với mặt phẳng $(P)$ nên mặt phẳng $(SAB)$ cắt mặt phẳng $(P)$ theo giao tuyến song song với $SA$. Vẽ $NP\parallel SA$ ($P\in SB$) thì $NP$ là giao tuyến của $(P)$ với mặt phẳng $(SAB)$.\\
		}	
		{
			\begin{tikzpicture}[declare function={a=2.65;b=5;h=3;},line join=round]
				\path (0,0) coordinate (A)
				(b,0) coordinate (B)
				(-30:a) coordinate (C)
				($(A)+(C)-(B)$) coordinate (x)
				($(x)!0.7!(C)$) coordinate (D)
				($(A)+(1,h)$) coordinate (S)
				($(C)!0.6!(D)$) coordinate (M)
				($(M)+(B)-(C)$) coordinate (N)
				($(N)+(S)-(A)$) coordinate (x)
				(intersection of N--x and S--B) coordinate (P)
				($(P)+(C)-(B)$) coordinate (y)
				(intersection of P--y and S--C) coordinate (Q)
				;
				\fill [blue,opacity=0.3] (M)--(N)--(P)--(Q);
				\draw[dashed] (A)--(B) (M)--(N)--(P);
				\draw (B)--(C)--(D)--(A) (B)--(S)  (D)--(S)--(C) (S)--(A) (M)--(Q)--(P);
				\foreach \t/\g in {A/180,B/0,C/-90,D/-90,S/90,M/-90,N/-70,P/90,Q/180}{
					\draw[fill=white] (\t) circle (1pt) node[shift={(\g:7pt)},font=\scriptsize]{$ \t $};
				}
			\end{tikzpicture}
		}
		\noindent
		Mặt phẳng $(SBC)$ chứa đường thẳng $BC$ song song với mặt phẳng $(P)$ nên mặt phẳng $(SBC)$ cắt mặt phẳng $(P)$ theo giao tuyến song song với $BC$. Vẽ $PQ\parallel BC$ ($Q\in SC$) thì $PQ$ là giao tuyến của $(P)$ với mặt phẳng $(SBC)$.\\
		Khi đó $MQ$ là giao tuyến của $(P)$ với $(SCD)$.\\
		Tứ giác $MNPQ$ có $MN\parallel PQ$ (vì cùng song song với $BC$)  nên nó là hình thang.	
	}
\end{bt}
%Bài 13
\begin{bt}%[Dự án Đề cương 11-Mui Doan]%[1T4K3-5]
	Cho tứ diện đều $S.ABC$. Gọi $I$; $M$ lần lượt là trung điểm $AB$ và $AI$. $(P)$ là mặt phẳng đi qua $M$ và song song với mặt phẳng $SIC$. Xác định thiết diện của  $(P)$ với tứ diện $S.ABC$.
	\loigiai{
		\immini{
			Ta có  $(P)\parallel (SIC)\Rightarrow (P)\parallel SI, SC, IC$.\\
			Mặt phẳng $(ABC)$ chứa đường thẳng $IC$ song song với mặt phẳng $(P)$ nên mặt phẳng $(ABC)$ cắt mặt phẳng $(P)$ theo giao tuyến song song với $IC$. Vẽ $MN\parallel IC$ ($N\in AC$) thì $MN$ là giao tuyến của $(P)$ với mặt phẳng $(ABC)$.\\
			Mặt phẳng $(SAB)$ chứa đường thẳng $SI$ song song với mặt phẳng $(P)$ nên mặt phẳng $(SAB)$ cắt mặt phẳng $(P)$ theo giao tuyến song song với $SI$. Vẽ $MP\parallel SI$ ($P\in SA$) thì $MP$ là giao tuyến của $(P)$ với mặt phẳng $(SAB)$.\\
			Khi đó $PN$ là giao tuyến của mặt phẳng $(P)$ với mặt phẳng $(SAC)$.\\
			}
		{
			\begin{tikzpicture}[declare function={r=3;}]
				\path (160:{r} and {r*0.35}) coordinate(A)
				(260:{r} and {r*0.5}) coordinate (B)
				(20:{r} and {r*0.35})coordinate (C)
				(90:{r*1.25}) coordinate (S)
				($(A)!.5!(B)$) coordinate (I)
				($(A)!.5!(I)$) coordinate (M)
				($(A)!.5!(C)$) coordinate (N)
				($(A)!.5!(S)$) coordinate (P)
				;
				\fill [blue,opacity=0.3] (M)--(N)--(P);
				\draw[dash pattern=on 2pt off 2 pt] (A)--(C)--(I) (M)--(N)--(P);
				\draw (A)--(B)--(C)--(S)--cycle (S)--(B) (S)--(I) (M)--(P);
				\foreach \t/\g in {A/180,C/0,B/-90,S/90,I/190,M/190,N/60,P/120}{
					\draw[fill=black] (\t) circle (1pt) node[shift={(\g:7pt)},font=\scriptsize]{$ \t $};
				}
			\end{tikzpicture}
		}
		\noindent
		Mặt phẳng $(SAC)$ chứa đường thẳng $SC$ song song với mặt phẳng $(P)$ nên mặt phẳng $(SAC)$ cắt mặt phẳng $(P)$ theo giao tuyến song song với $SC$. Suy ra $PN\parallel SC$. \\
		Ta có $MN$ là đường trung bình của $\triangle AIC$ nên $MN=\dfrac{IC}{2}$;\\
		$MP$ là đường trung bình của $\triangle SAI$ nên $MP=\dfrac{SI}{2}$;\\
		Mà $IC=SI$ (do $S.ABC$ là tứ diện đều) nên $MN=MP$.\\
		Thiết diện là tam giác $MNP$ cân tại $M$.
	}
\end{bt}
%Bài 14
\begin{bt}%[Dự án Đề cương 11-Mui Doan]%[1T4K3-5]
	Cho hình chóp $S.ABCD$ có đáy $ABCD$ là hình vuông. $M$ là điểm di động trên đoạn $AB$. $(P)$ là mặt phẳng qua $M$ và song song với mặt phẳng $(SAD)$. Xác định thiết diện của  $(P)$ với hình chóp $S.ABCD$.
	\loigiai{
		\immini{
			Ta có $(P)\parallel (SAD)\Rightarrow (P)\parallel SA, SD, AD$.\\
			Mặt phẳng $(SAB)$ chứa đường thẳng $SA$ song song với mặt phẳng $(P)$ nên mặt phẳng $(SAB)$ cắt mặt phẳng $(P)$ theo giao tuyến song song với $SA$. Vẽ $MN\parallel SA$ ($N\in SB$) thì $MN$ là giao tuyến của $(P)$ với mặt phẳng $(SAB)$.\\
			Hai mặt phẳng $(SBC)$ và $(ABCD)$ lần lượt chứa đường thẳng $BC$; $AD$ song song với mặt phẳng $(P)$ nên chúng cắt mặt phẳng $(P)$ theo giao tuyến song song với $BC\parallel AD$. Vẽ $NP$, $MQ$ lần lượt là giao tuyến của mặt phẳng $(P)$ với hai mặt phẳng $(SBC)$ và $(ABCD)$. Khi đó $PQ$ là giao tuyến của $(P)$ với mặt phẳng $(SCD)$.
		}	
		{
			\begin{tikzpicture}[declare function={a=2.2;c=5;h=3.7;D=45;},line join=round]
				%	c: cạnh DC
				%	a: cạnh DA
				%h: đường cao
				%D: góc D của đáy
				\path
				(0,0) coordinate (D)
				(D:a) coordinate (A)
				(c,0) coordinate (C)
				($(A)+(C)-(D)$) coordinate (B)
				($(A)+(70:h)$)  coordinate (S)
				($(A)!0.6!(B)$)  coordinate (M)
				($(M)+(S)-(A)$)  coordinate (x)
				(intersection of M--x and S--B) coordinate (N)
				($(D)!0.6!(C)$)  coordinate (Q)
				($(N)+(C)-(B)$) coordinate (y)
				(intersection of N--y and S--C) coordinate (P)
				;
				\fill [blue,opacity=0.3] (M)--(N)--(P)--(Q);
				\draw[dashed] (B)--(A)--(D) (S)--(A) (N)--(M)--(Q);
				\draw (S)--(B)--(C)--(D)--cycle (S)--(C) (N)--(P)--(Q);
				\foreach \t/\g in {A/180,B/0,C/0,D/180,S/90,M/-60,N/90,Q/-90,P/180}{
					\draw[fill=black] (\t) circle (1pt) node[shift={(\g:7pt)},font=\scriptsize]{$ \t $};
				}
			\end{tikzpicture}
		}
		\noindent
		Mặt phẳng $(SBC)$ chứa đường thẳng $BC$ song song với mặt phẳng $(P)$ nên giao tuyến $NP$ của $(SBC)$ và $(P)$ song song với $BC$. \\
		Mặt phẳng $(ABCD)$ chứa đường thẳng $BC$ song song với mặt phẳng $(P)$ nên giao tuyến $MQ$ của $(ABCD)$ và $(P)$ song song với $BC$. \\
		Mặt phẳng $(SCD)$ chứa đường thẳng $SD$ song song với mặt phẳng $(P)$ nên giao tuyến $PQ$ của $(SCD)$ và $(P)$ song song với $SD$. \\
		Tứ giác $MNPQ$ có $MQ\parallel NP$ (vì cùng song song với $BC$)  nên nó là hình thang.
	}
\end{bt}
%Bài 15
\begin{bt}%[Dự án Đề cương 11-Mui Doan]%[1T4K3-5]
	Cho hình chóp tứ giác đều $S.ABCD$ có cạnh đáy bằng $10$. $M$ là điểm trên cạnh  $SA$ sao cho $\dfrac{SM}{SA}=\dfrac{2}{3}$. $(P)$ là mặt phẳng qua $M$ và song song với mặt phẳng $AB$ và $AD$. Xác định thiết diện của  $(P)$ với hình chóp $S.ABCD$. Tính diện tích thiết diện này.
	\loigiai{
		\immini{
			Mặt phẳng $(SAB)$ chứa đường thẳng $AB$ song song với mặt phẳng $(P)$ nên mặt phẳng $(SAB)$ cắt mặt phẳng $(P)$ theo giao tuyến song song với $AB$. Vẽ $MN\parallel AB$ ($N\in SB$) thì $MN$ là giao tuyến của $(P)$ với mặt phẳng $(SAB)$.\\
			Mặt phẳng $(SAD)$ chứa đường thẳng $AD$ song song với mặt phẳng $(P)$ nên mặt phẳng $(SAD)$ cắt mặt phẳng $(P)$ theo giao tuyến song song với $AD$. Vẽ $MQ\parallel AD$ ($Q\in SD$) thì $MQ$ là giao tuyến của mặt phẳng $(P)$ với mặt phẳng $(SAD)$.\\
				}	
		{
			\begin{tikzpicture}[declare function={a=2.2;c=5;h=4;D=30;},line join=round]
				%	c: cạnh DC
				%	a: cạnh DA
				%h: đường cao
				%D: góc D của đáy
				\path
				(0,0) coordinate (D)
				(D:a) coordinate (A)
				(c,0) coordinate (C)
				($(A)+(C)-(D)$) coordinate (B)
				($(A)!.5!(C)$) coordinate (O)
				($(O)+(0,h)$)  coordinate (S)
				($(S)!2/3!(A)$)  coordinate (M)
				($(S)!2/3!(B)$)  coordinate (N)
				($(S)!2/3!(C)$)  coordinate (P)
				($(S)!2/3!(D)$)  coordinate (Q)
				;
				\fill [blue,opacity=0.3] (M)--(N)--(P)--(Q);
				\draw[dashed] (B)--(A)--(D) (A)--(C) (B)--(D) (S)--(A) (N)--(M)--(Q) (S)--(O);
				\draw (S)--(B)--(C)--(D)--cycle (S)--(C) (N)--(P)--(Q) ;
				\foreach \t/\g in {A/180,B/0,C/0,D/180,S/90,M/180,O/-90,N/0,P/0,Q/180}{
					\draw[fill=black] (\t) circle (1pt) node[shift={(\g:7pt)},font=\scriptsize]{$ \t $};
				}
			\end{tikzpicture}
		}
		\noindent
		Mặt phẳng $(SBC)$ chứa đường thẳng $BC\parallel AD$ song song với mặt phẳng $(P)$ nên mặt phẳng $(SBC)$ cắt mặt phẳng $(P)$ theo giao tuyến song song với $BC$. Vẽ $NP\parallel BC$ ($P\in SC$) thì $NP$ là giao tuyến của mặt phẳng $(P)$ với hai mặt phẳng $(SBC)$.\\
		Khi đó $QP$ là giao tuyến của mặt phẳng $(P)$ với mặt phẳng $(SCD)$.\\
		Mặt phẳng $(SCD)$ chứa đường thẳng $CD\parallel AB$ song song với mặt phẳng $(P)$ nên mặt phẳng $(SCD)$ cắt mặt phẳng $(P)$ theo giao tuyến song song với $CD$.  Suy ra $QP\parallel CD$.\\
		Tứ giác $MNPQ$ có $MQ\parallel NP$ (vì cùng song song với $BC$); $MN\parallel QP$ (vì cùng song song với $AB$)  nên nó là hình bình hành.\\
		Mặt khác, $AB\perp AD\Rightarrow MN\perp MQ$; $\dfrac{MN}{AB}=\dfrac{SM}{SA}=\dfrac{2}{3}\Rightarrow MN=10\cdot \dfrac{2}{3}=\dfrac{20}{3}$.\\
		Tương tự $MQ=\dfrac{20}{3}$.\\
		Vậy $MNPQ$ là hình vuông và có diện tích là $\dfrac{20}{3}\cdot \dfrac{20}{3}=\dfrac{400}{3}$.
	}
\end{bt}


