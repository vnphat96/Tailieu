\section{Phép chiếu song song}
\subsection{Tóm tắt lý thuyết}
\begin{tomtat}
\begin{center}
\begin{tikzpicture}[scale=1, font=\footnotesize, line join=round, line cap=round, >=stealth]
	\tkzInit[xmin=-3.5,xmax=5,ymin=-1, ymax=5] \tkzClip[space=.1]
	\tkzDefPoints{-3/0/A, 3/0/B, 4/2/C, -2/2/D, 0/1/Q, -1/2/P, -0.2/4/M, 3/1/x}
	\tkzDefLine[parallel=through M,K=1](P,Q)\tkzGetPoint{N}
	\tkzInterLL(M,N)(Q,x) \tkzGetPoint{M'}
	\tkzInterLL(A,D)(P,Q) \tkzGetPoint{y}
	\tkzInterLL(A,B)(P,Q) \tkzGetPoint{y'}
	\tkzInterLL(M,M')(B,C) \tkzGetPoint{t}
	\tkzDefLine[parallel=through y',K=1](P,Q)\tkzGetPoint{z}
	\tkzDefLine[parallel=through t,K=1](P,Q)\tkzGetPoint{t'}
	\tkzDrawSegments(A,B B,C C,D D,A P,Q M,N P,y N,M' y',z t,t')
	\tkzDrawSegments[dashed](Q,y' M',t)
	\coordinate (i) at ($(M')!1.2!(M)$);
	\tkzDrawSegment(M,i)
	\tkzDrawPoints(M,M')
	\tkzLabelSegment[above left=0.4](P,y){$\Delta$}
	\tkzLabelPoints[above right](M',M)
	\tkzMarkAngle[size=0.8cm,opacity=.5,mark=](B,A,D)
	\tkzLabelAngle[pos=-0.5](D,A,B){$\alpha$}
\end{tikzpicture}
\end{center}
\subsubsection{Định nghĩa}
\begin{dn}
	Cho mặt phẳng $(\alpha)$ và đường thẳng $\Delta$ cắt $(\alpha)$. Với mỗi điểm $M$ trong không gian, ta xác định điểm $M'$ như sau:
\begin{itemize}
	\item Nếu $M$ thuộc $\Delta$ thì $M'$ là giao điểm của $(\alpha)$ và $\Delta$.
	\item Nếu $M$ không thuộc $\Delta$ thì $M'$ là giao điểm của $(\alpha)$ và đường thẳng qua $M$ song song với $\Delta$.
\end{itemize}
	Điểm $M'$ được gọi là hình chiếu song song của điểm $M$ trên mặt phẳng $(\alpha)$ theo phương $\Delta$. Phép đặt tương ứng mỗi điểm $M$ với hình chiếu $M'$ của nó được gọi là phép chiếu song song lên $(\alpha)$ theo phương $\Delta$.\\
	Mặt phẳng $(\alpha)$ được gọi là mặt phẳng chiếu, phương $\Delta$ được gọi là phương chiếu.
\end{dn}

\begin{dn}
	Cho hình $\mathcal{H}$. Tập hợp $\mathcal{H}'$ các hình chiếu $M'$ của các điểm $M$ thuộc $\mathcal{H}$ qua phép chiếu song song được gọi là hình chiếu của $\mathcal{H}$ qua phép chiếu song song đó.
\end{dn}
\subsubsection{Tính chất}
\begin{itemize}
	\item Phép chiếu song song biến ba điểm thẳng hàng thành ba điểm thẳng hàng và không làm thay đổi thứ tự ba điểm đó. Phép chiếu song song biến đường thẳng thành đường thẳng, tia thành tia, đoạn thẳng thành đoạn thẳng.
	\item Phép chiếu song song biến hai đường thẳng song song thành hai đường thẳng song song hoặc trùng nhau.
	\item Phép chiếu song song giữ nguyên tỉ số độ dài của hai đoạn thẳng cùng nằm trên một đường thẳng hoặc nằm trên hai đường thẳng song song.
\end{itemize}
\end{tomtat}