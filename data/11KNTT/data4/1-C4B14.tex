\section{Phép chiếu song song}
\subsection{Tóm tắt lí thuyết}
\subsubsection{Phép chiếu song song}
Cho mặt phẳng $(\alpha )$ và đường thẳng $\Delta $ cắt $(\alpha )$. Với mỗi điểm $M$ trong không gian, ta xác định điểm $M'$ như sau:
\begin{itemize}
	\item Nếu điểm $M\in \Delta $ thì $M'$ là giao điểm của $(\alpha )$ với $\Delta $.
	\item Nếu điểm $M\notin \Delta $ thì $M'$ là giao điểm của $(\alpha )$ với đường thẳng đi qua $M$ và song song $\Delta$.
	\begin{center}
	\begin{tikzpicture}
	\def\a{4}
	\path 	(0:0) coordinate (A)
			++(0:\a) coordinate (D)
			++(-130:\a/2) coordinate (C)
			($(A)+(C)-(D)$) coordinate (B);
	\path (A)++(-50:.2*\a)coordinate (M)
			++(110:.4*\a) coordinate (N)
			++(-70:.8*\a) coordinate (Nt)
			(intersection of M--N and B--C)coordinate (Mt);
	\path (A)++(30:.45*\a)coordinate (P)
			($(M)-(N)+(P)$)coordinate (Q)
			($(Mt)-(M)+(Q)$)coordinate (Qt)
			($(Nt)-(Mt)+(Qt)$)coordinate (Pt);
	\draw[dashed] 	(M)--(Mt) (Q)--(Qt);
	\draw[thick] (A)--(B)--(C)--(D)--cycle (M)--(N) (Nt)--(Mt) (P)--(Q) (Pt)--(Qt);
\end{tikzpicture}
	\end{center}
	\begin{itemize}
		\item Điểm $M'$ được gọi là hình chiếu song của điểm $M$ trên mặt phẳng $(\alpha )$ theo phương $\Delta $.
		\item Mặt phẳng $(\alpha )$ gọi là mặt phẳng chiếu. Phương $\Delta $ gọi là phương chiếu.
	\end{itemize}
	\item Phép đặt tương ứng mỗi điểm $M$ trong không gian với hình chiếu $M'$ của nó trên mặt phẳng $(\alpha )$ được gọi là phép chiếu song song lên $(\alpha )$ theo phương $\Delta $.
	\item Nếu $\mathscr{H}$ là một hình nào đó thì tập hợp $\mathscr{H'}$ các hình chiếu $M'$ của tất cả những điểm $M$ thuộc $\mathscr{H}$ được gọi là hình chiếu của $\mathscr{H}$ qua phép chiếu song song nói trên.
\end{itemize}
\begin{note}
	Nếu một đường thẳng có phương trùng với phương chiếu thì hình chiếu của đường thẳng đó là một điểm.
\end{note}
\subsubsection{Tính chất của phép chiếu song song}
\begin{itemize}
	\item Phép chiếu song song biến ba điểm thẳng hàng thành ba điểm thẳng hàng và không làm thay đổi thứ tự ba điểm đó.
	\item Phép chiếu song song biến đường thẳng thành đường thẳng, biến tia thành tia, biến đoạn thẳng thành đoạn thẳng.
	\item Phép chiếu song song biến hai đường thẳng song song thành hai đường thẳng song song hoặc trùng nhau.
	\item Phép chiếu song song không làm thay đổi tỉ số độ dài của hai đoạn thẳng nằm trên hai đường thẳng song song hoặc cùng nằm trên một đường thẳng.
\end{itemize}
\subsubsection{Hình biểu diễn của một hình không gian trên mặt phẳng}
Hình biểu diễn của một hình $\mathbf{H}$ trong không gian là hình chiếu song song của hình $\mathbf{H}$ trên một mặt phẳng theo một phương chiếu nào đó hoặc hình đồng dạng với hình chiếu đó.
Hình biểu diễn của các hình thường gặp:
\begin{itemize}
	\item {\bf Tam giác}. Một tam giác bất kì bao giờ cũng có thể coi là hình biểu diễn của một tam giác có dạng tùy ý cho trước
	\item {\bf Hình bình hành.} Một hình bình hành bất kì bao giờ cũng có thể coi là hình biểu diễn của một hình bình hành có dạng tùy ý cho trước
	\item {\bf Hình thang}. Một hình thang bất kì bao giờ cũng có thể coi là hình biểu diễn của một hình thang tùy ý cho trước, miễn là tỉ số độ dài hai đáy của hình biểu diễn phải bằng tỉ số độ dài hai đáy của hình thang ban đầu.
	\item {\bf Hình tròn}. Người ta thường dùng hình elip để biểu diễn cho hình tròn.
\end{itemize}
\subsection{Bài tập trắc nghiệm}
\Opensolutionfile{ans}[ans/ans1-C4B14]
\begin{ex}%Câu 1
	Hình chiếu của hình chữ nhật không thể là hình nào trong các hình sau?
	\choice
		{Hình chữ nhật}
		{\True Hình thang}
		{Hình bình hành}
		{Hình thoi}
	\loigiai{
		Hình chiếu của hình chữ nhật không thể là hình thang.}
\end{ex}
\begin{ex}%Câu 2
	Cho hình lăng trụ $ABC.A'B'C'$, gọi $I$, $I'$ lần lượt là trung điểm của $AB$, $A'B'$. Qua phép chiếu song song đường thẳng $AI'$, mặt phẳng chiếu $\left( A'B'C' \right)$ biến $I$ thành
	\choice
	{$A'$}
	{$C'$}
	{\True $B'$}
	{$I'$}
	\loigiai{
		\begin{center}
			\begin{tikzpicture}
				\def\a{5} 	\def\h{4.5}
				\path 	(0:0) coordinate (A')
						++(0:\a) coordinate (C')
						++(-150:3*\a/4) coordinate (B')
						($(A')+(78:\h)$) coordinate (A)
						($(B')+(78:\h)$) coordinate (B)
						($(C')+(78:\h)$) coordinate (C)
						($(A)!.5!(B)$)coordinate (I)
						($(A')!.5!(B')$)coordinate (I');
				\draw[dashed] 	(A')--(C');
				\draw[thick]	(C)--(C') 	(B)--(B') (A)--(I') (B')--(I)	(A)--(A') 	(A)--(B)--(C)--cycle (A')--(B')--(C');
				\foreach \x/\g in 					{A/180,B/-45,C/0,A'/180,B'/-45,C'/0,I/30,I'/-120}
				\fill[black] 	(\x) circle (1pt)
				($(\g:3mm)+(\x)$) node {$\x$};	
\end{tikzpicture}
		\end{center}
		Ta có $\heva{& AI\parallel B'I'\\& AI=B'I'}\Rightarrow AIB'I'$ là hình bình hành.\\
		Suy ra qua phép chiếu song song đường thẳng $AI'$, mặt phẳng chiếu $\left( A'B'C' \right)$ biến điểm $I$ thành điểm $B'$.}
\end{ex}
\begin{ex}%Câu 3
	Cho tứ diện $ABCD$. Gọi $M$ là trung điểm của $AD$. Hình chiếu song song của điểm $M$ theo phương $AC$ lên mặt phẳng $\left( BCD \right)$ là điểm nào sau đây?
	\choice
	{$D$}
	{\True Trung điểm của $CD$}
	{Trung điểm của $BD$}
	{Trọng tâm tam giác $BCD$}
	\loigiai{
		\begin{center}
			\begin{tikzpicture}
				\def\a{4}
				\path 	(0:0) coordinate (B)
					++(0:\a) coordinate (D)
					++(-120:\a/2) coordinate (C)
					($(B)+(70:\a)$) coordinate (A)
					($(A)!.5!(D)$)coordinate (M)
					($(C)!.5!(D)$)coordinate (N);
				\draw[dashed] 	(B)--(D);
				\draw[thick] 	(B)--(A)--(D) (A)--(C)
					(B)--(C)--(D) (M)--(N);
				\foreach \x/\g in {A/90,B/180,C/-45,D/0,M/45,N/-45}
				\fill[black] 	(\x) circle (1pt)
				($(\g:3mm)+(\x)$) node {$\x$};	
\end{tikzpicture}
		\end{center}
		Gọi $N$ là trung điểm của cạnh $CD$.\\
		Khi đó $MN$ là đường trung bình của $\triangle ADC$ nên $MN\parallel AC$.\\ Do đó, hình chiếu song song của $M$ theo phương $AC$ lên mặt phẳng $\left( BCD \right)$ là điểm $N$.}
\end{ex}
\begin{ex}%Câu 4
	Qua phép chiếu song song, tính chất nào không được bảo toàn?
	\choice
	{\True Chéo nhau}
	{Đồng qui}
	{Song song}
	{Thẳng hàng}
	\loigiai{
		Do hai đường thẳng qua phép chiếu song song ảnh của chúng sẽ cùng thuộc một mặt phẳng.\\
		Suy ra tính chất chéo nhau không được bảo toàn.}
\end{ex}
\begin{ex}%Câu 5
	Trong các mệnh đề sau mệnh đề nào \textbf{sai}?
	\choice
	{Phép chiếu song song biến đường thẳng thành đường thẳng, biến tia thành tia, biến đoạn thẳng thành đoạn thẳng}
	{\True Phép chiếu song song biến hai đường thẳng song song thành hai đường thẳng song song}
	{Phép chiếu song song biến ba điểm thẳng hàng thành ba điểm thẳng hàng và không thay đổi thứ tự của ba điểm đó}
	{Phép chiếu song song không làm thay đổi tỉ số độ dài của hai đoạn thẳng nằm trên hai đường thẳng song song hoặc cùng nằm trên một đường thẳng}
	\loigiai{
		Tính chất của phép chiếu song song.\\
		Phép chiếu song song biến hai đường thẳng song song thành hai đường thẳng song song hoặc trùng nhau.\\
		Suy ra phương án \lq\lq  Phép chiếu song song biến hai đường thẳng song song thành hai đường thẳng song song\rq\rq sai vì chúng có thể trùng nhau.}
\end{ex}
\begin{ex}%Câu 6
	Cho hình lăng trụ $ABC.A'B'C'$, qua phép chiếu song song đường thẳng $CC'$, mặt phẳng chiếu $\left( A'B'C' \right)$ biến $M$ thành $M'$. Trong đó $M$ là trung điểm của $BC$. Chọn mệnh đề đúng?
	\choice
	{$M'$ là trung điểm của $A'B'$}
	{\True $M'$ là trung điểm của $B'C'$}
	{$M'$ là trung điểm của $A'C'$}
	{Cả ba đáp án trên đều sai}
	\loigiai{
		Ta có phép chiếu song song đường thẳng $CC'$, biến $C$ thành $C'$, biến $B$ thành $B'$.\\
		Do $M$ là trung điểm của $BC$ suy ra $M'$ là trung điểm của $B'C'$.}
\end{ex}
\begin{ex}%Câu 7
	Cho hình lăng trụ $ABC.A'B'C'$, gọi $I$, $I'$ lần lượt là trung điểm của $AB$, $A'B'$. Qua phép chiếu song song đường thẳng $AI'$, mặt phẳng chiếu $\left( A'B'C' \right)$ biến $I$ thành
	\choice
	{$A'$}
	{\True $B'$}
	{$C'$}
	{$I'$}
	\loigiai{
		\begin{center}
			\begin{tikzpicture}
				\def\a{5} 	\def\h{4.5}
				\path 	(0:0) coordinate (A')
				++(0:\a) coordinate (C')
				++(-150:3*\a/4) coordinate (B')
				($(A')+(78:\h)$) coordinate (A)
				($(B')+(78:\h)$) coordinate (B)
				($(C')+(78:\h)$) coordinate (C)
				($(A)!.5!(B)$)coordinate (I)
				($(A')!.5!(B')$)coordinate (I');
				\draw[dashed] 	(A')--(C');
				\draw[thick]	(C)--(C') 	(B)--(B') (A)--(I') (B')--(I)	(A)--(A') 	(A)--(B)--(C)--cycle (A')--(B')--(C');
				\foreach \x/\g in 					{A/180,B/-45,C/0,A'/180,B'/-45,C'/0,I/30,I'/-120}
				\fill[black] 	(\x) circle (1pt)
				($(\g:3mm)+(\x)$) node {$\x$};	
			\end{tikzpicture}
		\end{center}
		Ta có $\heva{& AI\parallel B'I' \\& AI=B'I'}\Rightarrow AIB'I'$ là hình bình hành.\\
		Suy ra qua phép chiếu song song đường thẳng
		$AI'$, mặt phẳng chiếu $\left( A'B'C' \right)$ biến điểm $I$
		thành điểm $B'$.}
\end{ex}
\begin{ex}%Câu 8
	Cho tam giác $ABC$ ở trong mặt phẳng $\left( \alpha \right)$ và phương $\ell$. Biết hình chiếu của tam giác $ABC$ lên mặt phẳng $\left( P \right)$ là một đoạn thẳng. Khẳng định nào sau đây đúng?
	\choice
	{$\left( \alpha \right)\parallel \left( P \right)$}
	{$\left( \alpha \right)\equiv \left( P \right)$}
	{\True $\left( \alpha \right)\parallel \ell$ hoặc $\left( \alpha \right)\supset \ell$}
	{A, B, C đều sai}
	\loigiai{
		\begin{itemize}
			\item Hình chiếu của tam giác $ABC$ vẫn là một tam giác trên mặt phẳng $\left( P \right)$.
			\item Hình chiếu của tam giác $ABC$ vẫn là tam giác $ABC$.
			\item Khi phương chiếu $\ell$ song song hoặc được chứa trong mặt phẳng $\left( \alpha \right)$ thì hình chiếu của tam giác là đoạn thẳng trên mặt phẳng $\left( P \right)$. Nếu giao tuyến của hai mặt phẳng $\left( \alpha \right)$ và $\left( P \right)$ là một trong ba cạnh của tam giác $ABC$.
		\end{itemize}
	}
\end{ex}
\begin{ex}%Câu 9
	Khẳng định nào sau đây đúng?
	\choice
	{\True Hình chiếu song song của một hình chóp cụt có thể là một hình tam giác}
	{Hình chiếu song song của một hình chóp cụt có thể là một đoạn thẳng}
	{Hình chiếu song song của một hình chóp cụt có thể là một hình chóp cụt}
	{Hình chiếu song song của một hình chóp cụt có thể là một điểm}
	\loigiai{
		Qua phép chiếu song song chỉ có thể biến hình chóp cụt thành một đa giác.
	}
\end{ex}
\begin{ex}%Câu 10
	Trong các mệnh đề sau mệnh đề nào \textbf{sai}?
	\choice
		{Hình chiếu song song của hai đường thẳng chéo nhau có thể song song với nhau}
		{Một đường thẳng có thể trùng với hình chiếu của nó}
		{\True Hình chiếu song song của hai đường thẳng chéo nhau có thể trùng nhau}
		{Một tam giác bất kỳ đều có thể xem là hình biểu diễn của một tam giác cân}
	\loigiai{
		\begin{itemize}
			\item Hình chiếu song song của hai đường thẳng chéo nhau có thể song song với nhau đúng vì khi đó hình chiếu của chúng cùng nằm trên một mặt phẳng.
			\item Một đường thẳng có thể trùng với hình chiếu của nó đúng vì mặt phẳng chiếu chứa đường thẳng đã cho.
			\item Hình chiếu song song của hai đường thẳng chéo nhau có thể trùng nhau sai vì hình chiếu của chúng chỉ có thể song song hoặc cắt nhau.
			\item Một tam giác bất kỳ đều có thể xem là hình biểu diễn của một tam giác cân đúng - tính chất phép chiếu song song.
		\end{itemize}
	}
\end{ex}
\begin{ex}%Câu 11
	Qua phép chiếu song song biến ba đường thẳng song song thành.
	\choice
		{Ba đường thẳng đôi một song song với nhau}
		{Một đường thẳng}
		{Thành hai đường thẳng song song}
		{\True Cả ba trường hợp trên}
		\loigiai{
			Tính chất phép chiếu song song.}
\end{ex}
\begin{ex}%Câu 12
	Khẳng định nào sau đây đúng?
	\choice
		{Hình chiếu song song của hình lập phương $ABCD.A'B'C'D'$ theo phương $AA'$ lên mặt phẳng $\left( ABCD \right)$ là hình bình hành}
		{\True Hình chiếu song song của hình lập phương $ABCD.A'B'C'D'$ theo phương $AA'$ lên mặt phẳng $\left( ABCD \right)$ là hình vuông}
		{Hình chiếu song song của hình lập phương $ABCD.A'B'C'D'$ theo phương $AA'$ lên mặt phẳng $\left( ABCD \right)$ là hình thoi}
		{Hình chiếu song song của hình lập phương $ABCD.A'B'C'D'$ theo phương $AA'$ lên mặt phẳng $\left( ABCD \right)$ là một tam giác}
	\loigiai{
			Qua phép chiếu song song đường thẳng $AA'$ lên mặt phẳng $\left( ABCD \right)$ sẽ biến $A'$ thành $A$, biến $B'$ thành $B$, biến $C'$ thành $C$, biến $D'$ thành $D$. Nên hình chiếu song song của hình lập phương $ABCD.A'B'C'D'$ là hình vuông.}
\end{ex}
\begin{ex}%Câu 13
	Hình chiếu của hình vuông không thể là hình nào trong các hình sau?
	\choice
		{Hình vuông}
		{Hình bình hành}
		{\True Hình thang}
		{Hình thoi}
	\loigiai{
			Tính chất của phép chiếu song song.}
\end{ex}
\begin{ex}%Câu 14
	Trong các mện đề sau mệnh đề nào \textbf{sai}?
	\choice
		{\True Một đường thẳng luôn cắt hình chiếu của nó}
		{Một tam giác bất kỳ đề có thể xem là hình biểu diễn của một tam giác cân}
		{Một đường thẳng có thể song song với hình chiếu của nó}
		{Hình chiếu song song của hai đường thẳng chéo nhau có thể song song với nhau}
	\loigiai{
			Khi mặt phẳng chiếu song song với đường thẳng đã cho thì đường thẳng đó song song với hình chiếu của nó.}
\end{ex}
\begin{ex}%Câu 15
	Nếu đường thẳng $a$ cắt mặt phẳng chiếu $\left( P \right)$ tại điểm $A$ thì hình chiếu của $a$ sẽ là:
	\choice
		{Điểm $A$}
		{Trùng với phương chiếu}
		{Đường thẳng đi qua $A$}
		{\True Đường thẳng đi qua $A$ hoặc chính $A$}
	\loigiai{
		\begin{itemize}
			\item Nếu phương chiếu song song hoặc trùng với đường thẳng $a$ thì hình chiếu là điểm $A$.
			\item Nếu phương chiếu không song song hoặc không trùng với đường thẳng $a$ thì hình chiếu là đường thẳng đi qua điểm $A$.
		\end{itemize}
	}
\end{ex}
\begin{ex}%Câu 16
	Giả sử tam giác $ABC$ là hình biểu diễn của một tam giác đều. Hình biểu diễn của tâm đường tròn ngoại tiếp tam giác đều là
	\choice
		{Giao điểm của hai đường trung tuyến của tam giác $ABC$}
		{\True Giao điểm của hai đường trung trực của tam giác $ABC$}
		{Giao điểm của hai đường đường cao của tam giác $ABC$}
		{Giao điểm của hai đường phân giác của tam giác $ABC$}
	\loigiai{
		Tâm của đường tròn ngoại tiếp tam giác là giao của ba đường trung trực.}
\end{ex}
\begin{ex}%Câu 17
	Cho hình chóp $S.ABCD$ có đáy là hình bình hành. $M$ là trung điểm của $SC$. Hình chiếu song song của điểm $M$ theo phương $AB$ lên mặt phẳng $\left( SAD \right)$ là điểm nào sau đây?
	\choice
		{$S$}
		{\True Trung điểm của $SD$}
		{$A$}
		{$D$}
	\loigiai{
		Giả sử $N$ là ảnh của $M$ theo phép chiếu song song đường thẳng $AB$ lên mặt phẳng $\left( SAD \right)$.\\
			Suy ra $MN\parallel AB \Rightarrow MN\parallel CD$. Do $M$ là trung điểm của $SC \Rightarrow N$ là trung điểm của $SD$.}
\end{ex}
\begin{ex}%Câu 18
	Cho hình chóp $S.ABCD$ có đáy là hình bình hành. Hình chiếu song song của điểm $A$ theo phương $AB$ lên mặt phẳng $\left( SBC \right)$ là điểm nào sau đây?
	\choice
		{$S$}
		{Trung điểm của $BC$}
		{\True $B$}
		{$C$}
	\loigiai{
		Do $AB\cap \left( SBC \right)=\left\{ A \right\}$ suy ra hình chiếu song song của điểm $A$ theo phương $AB$ lên mặt phẳng $\left( SBC \right)$ là điểm $B$.}
\end{ex}
\begin{ex}%Câu 19
	Cho lăng trụ $ABC.A'B'C'$. Gọi $M$ là trung điểm của $AC$. Khi đó hình chiếu song song của điểm $M$ lên $\left( AA'B' \right)$ theo phương chiếu $CB$ là
	\choice
		{Trung điểm $BC$}
		{\True Trung điểm $AB$}
		{Điểm $A$}
		{Điểm $B$}
	\loigiai{
		\begin{center}
			\begin{tikzpicture}
				\def\a{5}	\def\h{4.5}
				\path 	(0:0) coordinate (A)
						++(0:\a) coordinate (C)
						++(-150:\a/2) coordinate (B)
						($(B)!0.5!(C)$) coordinate (M)
						($(A)!2/3!(M)$) coordinate (G)
						($(G)+(90:\h)$) coordinate (A')
						($(A')+(C)-(A)$) coordinate (C')
						($(C')+(B)-(C)$) coordinate (B')
						($(A)!.5!(C)$) coordinate (M)
						($(A)!.5!(B)$) coordinate (N); 
				\draw[dashed,thick] (A)--(C) (M)--(N);
				\draw[thick] (C)--(C') 	(B)--(B') 	(A)--(A') 
					(A)--(B)--(C) (A')--(B')--(C')--cycle;
				\foreach \x/\g in {A/180,B/-45,C/0,A'/180,B'/90,C'/0,M/90,N/-90}
				\fill[black] 	(\x) circle (1pt)
				($(\g:3mm)+(\x)$) node {$\x$};
\end{tikzpicture}
		\end{center}
		Gọi $N$ là trung điểm của $AB$. Ta có: $MN\parallel CB$.\\
		Vậy hình chiếu song song của điểm $M$ lên $\left( AA'B' \right)$ theo phương chiếu $CB$ là điểm $N$.}
\end{ex}
\begin{ex}%Câu 20
	Cho hình hộp chữ nhật $ABCD.A'B'C'D'$. Gọi $O=AC\cap BD$ và $O'=A'C'\cap B'D'$. Điểm $M$, $N$ lần lượt là trung điểm của $AB$ và $CD$ Qua phép chiếu song song theo phương $AO'$ lên mặt phẳng $\left( ABCD \right)$ thì hình chiếu của tam giác $C'MN$ là
	\choice
		{\True Đoạn thẳng $MN$}
		{Điểm $O$}
		{Tam giác $CMN$}
		{Đoạn thẳng $BD$}
	\loigiai{
		\begin{center}
		\begin{tikzpicture}
			\def\a{3} \def\b{2} \def\h{3}
			\path 	(0:0) coordinate (A)
				++(0:\a) coordinate (D)
				++(-140:\b) coordinate (C)
				($(A)+(C)-(D)$) coordinate (B)
				($(A)+(90:\h)$) coordinate (A')
				($(B)+(90:\h)$) coordinate (B')
				($(C)+(90:\h)$) coordinate (C')
				($(D)+(90:\h)$) coordinate (D')
				($(A)!.5!(B)$) coordinate (M)
				($(C)!.5!(D)$) coordinate (N)
				(intersection of A--C and B--D)coordinate (O)
				(intersection of A'--C' and B'--D')coordinate (O');
			\draw[dashed] 	(B)--(A)--(D)	(A)--(A') (A)--(C) (B)--(D) (C')--(M)--(N) (C')--(O) (A)--(O');
			\draw[thick] 	(C)--(C') 	(D)--(D') 	(B)--(B')	(C)--(C') (B)--(C)--(D) (A')--(B')--(C')--(D')--cycle
			(A')--(C') (B')--(D') (C')--(N);
			\foreach \x/\g in {A/-90,B/180,C/0,D/0,A'/180,B'/180,C'/0,D'/0,O/-90,O'/90,M/180,N/0}
		\fill[black] 	(\x) circle (1pt)
					($(\g:3mm)+(\x)$) node {$\x$};	
\end{tikzpicture}
		\end{center}
		Ta có: $O'C'=AO$ và $O'C'\parallel AO$ nên tứ giác $O'C'OA$ là hình bình hành $\Rightarrow O'A\parallel C'O$.\\
		Do đó hình chiếu của điểm $O'$ qua phép chiếu song song theo phương $O'A$ lên mặt phẳng $\left( ABCD \right)$ là điểm $O$.\\
		Mặt khác điểm $M$ và $N$ thuộc mặt phẳng $\left( ABCD \right)$ nên hình chiếu của $M$ và $N$ qua phép chiếu song song theo phương $O'A$ lên mặt phẳng $\left( ABCD \right)$ lần lượt là điểm $M$ và $N$.\\
		Vậy qua phép chiếu song song theo phương $AO'$ lên mặt phẳng $\left( ABCD \right)$ thì hình chiếu của tam giác $C'MN$ là đoạn thẳng $MN$.}
\end{ex}
\begin{ex}%Câu 21
	Cho hình hộp $ABCD.A'B'C'D'$. Xác định các điểm $M,N$ tương ứng trên các đoạn $AC',B'D'$ sao cho $MN$ song song với $BA'$ và tính tỉ số $\dfrac{MA}{MC'}$.
	\choice
		{\True $2$}
		{$3$}
		{$4$}
		{1}
	\loigiai{
		\begin{center}
		\begin{tikzpicture}
			\def\a{4} \def\h{3}
	\path 	(0:0) coordinate (A')
			++(0:\a) coordinate (D')
			++(-65:\a/2) coordinate (C')
			($(A')+(C')-(D')$) coordinate (B')	
			($(A')!0.25!(C')$) coordinate (H)
			($(H)+(90:\h)$) coordinate (A)
			($(A)+(C')-(A')$) coordinate (C)
			($(C)+(B')-(C')$) coordinate (B)
			($(A)+(D')-(A')$) coordinate (D)
			($2*(A')-(B')$) coordinate (K)
			($(C')!1/3!(A)$) coordinate (M)
			(intersection of B'--D' and K--C')coordinate (N);
	\draw[dashed] 	(A')--(D')--(C')	(D)--(D') (A)--(C') (K)--(N)--(M) (B')--(D') (K)--(C');
	\draw[thick] 	(C)--(C') 	(A)--(A') 	(B)--(B')  (A')--(B')--(C') (A)--(B)--(C)--(D)--cycle (B)--(A')--(K)--(A);
	\foreach \x/\g in {A/90,B/-45,C/0,D/0,A'/180,B'/180,C'/0,D'/0,K/90,N/-90,M/-90}
			\fill[black] 	(\x) circle (1pt)
			($(\g:3mm)+(\x)$) node {$\x$};
\end{tikzpicture}
		\end{center}
		Xét phép chiếu song song lên mặt phẳng $\left( A'B'C'D' \right)$ theo phương chiếu $BA'$.\\
		Ta có $N$ là ảnh của $M$ hay $M$ chính là giao điểm của $B'D'$ và ảnh $AC'$ qua phép chiếu này.\\
		Do đó ta xác định $M,N$ như sau:
		\begin{itemize}
			\item Trên $A'B'$ kéo dài lấy điểm $K$ sao cho $A'K=B'A'$ thì $ABA'K$ là hình bình hành nên $AK\parallel BA'$ suy ra $K$ là ảnh của $A$ trên $AC'$ qua phép chiếu song song.
			\item Gọi $N=B'D'\cap KC'$. Đường thẳng qua $N$ và song song với $AK$ cắt $AC'$ tại $M$. Ta có $M,N$ là các điểm cần xác định.
		\end{itemize}
	Theo định lí Thales, ta có $\dfrac{MA}{MC'}=\dfrac{NK}{NC'}=\dfrac{KB'}{C'D'}=2$. }
\end{ex}
\begin{ex}%Câu 22
	Cho hình hộp $ABCD.A'B'C'D'$. Gọi $M,N$ lần lượt là trung điểm của $CD$ và $CC'$.
	\begin{enumerate}[a)]
		\item Xác định đường thẳng $\Delta $ đi qua $M$ đồng thời cắt $AN$ và $A'B$.
		\item Gọi $I,J$ lần lượt là giao điểm của $\Delta $ với $AN$ và $A'B$. Hãy tính tỉ số $\dfrac{IM}{IJ}$.
	\end{enumerate}
	\choice
		{$2$}
		{$3$}
		{$4$}
		{\True $1$}
	\loigiai{
		\begin{center}
		\begin{tikzpicture}
			\def\a{3} \def\b{2} \def\h{3.5}
	\path 	(0:0) coordinate (B)
			++(0:\a) coordinate (C)
			++(-130:\b) coordinate (D)
			($(B)+(D)-(C)$) coordinate (A)
			($(A)+(90:\h)$) coordinate (A')
			($(B)+(90:\h)$) coordinate (B')
			($(C)+(90:\h)$) coordinate (C')
			($(D)+(90:\h)$) coordinate (D')
			($(C)!.5!(D)$) coordinate (M)
			($(C)!.5!(C')$) coordinate (N)
			($(A)!.5!(N)$) coordinate (I)
			($2*(C)-(M)$) coordinate (N')
			($2*(I)-(M)$) coordinate (J)
			(intersection of A--N' and B--M)coordinate (I');
	\draw[dashed] 	(A)--(B)--(C)	(B)--(B') (A')--(B) (N)--(A)--(N') (B)--(M) (I)--(I') (M)--(J);
	\draw[thick] 	(C)--(C') 	(D)--(D') 	(A)--(A') (N)--(N')
			(A)--(D)--(C)--(N')
			(A')--(B')--(C')--(D')--cycle;
	\foreach \x/\g in {A/180,B/45,C/0,D/0,A'/180,B'/180,C'/0,D'/0,N/0,N'/0,M/0,I'/-90,I/90,J/180}
	\fill[black] 	(\x) circle (1pt)
	($(\g:3mm)+(\x)$) node {$\x$};
	\draw[dashed] 	(I)--(J) node[pos=0.5,sloped,above]{$\Delta$};	
\end{tikzpicture}
		\end{center}
		\begin{enumerate}[a)]
			\item Giả sử đã dựng được đường thẳng $\Delta $ cắt cả $AN$ và $BA'$.\\
			Gọi $I,J$ lần lượt là giao điểm của $\Delta $ với $AN$ và $BA'$.\\
			Xét phép chiếu song song lên $\left( ABCD \right)$ theo phương chiếu $A'B$.\\
			Khi đó ba điểm $J,I,M$ lần lượt có hình chiếu là $B,I',M$.\\
			Do $J,I,M$ thẳng hàng nên $B,I',M$ cũng thẳng hàng.\\
			Gọi $N'$ là hình chiếu của $N$ thì $AN'$ là hình chiếu của $AN$.\\
			Vì $I\in AN\Rightarrow I'\in AN'\Rightarrow I'=BM\cap AN'$.\\
			Từ phân tích trên suy ra cách dựng: 
			\begin{itemize}
				\item  Lấy $I'=AN'\cap BM$.
				\item Trong $\left( ANN' \right)$ dựng $II'\parallel NN'$ cắt $AN$ tại $I$.
				\item Vẽ đường thẳng $MI$, đó chính là đường thẳng cần dựng.
			\end{itemize}
		\item Ta có $MC=CN'$ suy ra $MN'=CD=AB$.\\
		Do đó $I'$ là trung điểm của $BM$.\\
		Mặt khác $II'\parallel JB$ nên $II'$ là đường trung bình của tam giác $MBJ$, suy ra $IM=IJ\Rightarrow \dfrac{IM}{IJ}=1$.
		\end{enumerate}
	}
\end{ex}
\begin{ex}%Câu 23
	Cho hình lăng trụ tam giác $ABC.A'B'C'$, gọi $M,N,P$ lần lượt là tâm của các mặt bên $\left( ABB'A' \right)$, $\left( BCC'B' \right)$ và $\left( ACC'A' \right)$. Qua phép chiếu song song đường thẳng $BC'$ và mặt phẳng chiếu $\left( AB'C \right)$ khi đó hình chiếu của điểm $P$?
	\choice
		{\True Trung điểm của $AN$}
		{Trung điểm của $AM$}
		{Trung điểm của $B'N$}
		{Trung điểm của $B'M$}
	\loigiai{}
\end{ex}
\Closesolutionfile{ans}
\begin{indapan}{10}
	{ans/ans1-C4B14}
\end{indapan}