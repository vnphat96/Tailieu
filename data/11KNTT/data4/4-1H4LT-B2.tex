\setcounter{section}{10} \setcounter{dang}{0}
\newpage
\section{HAI ĐƯỜNG THẲNG SONG SONG}
\subsection{KIẾN THỨC CẦN NHỚ}
\subsubsection{VỊ TRÍ TƯƠNG ĐỐI CỦA HAI ĐƯỜNG THẲNG}
Trong không gian, cho hai đường thẳng $a$ và $b$.
\begin{enumerate}[\iconMT]
	\item \indam{Các trường hợp có thể xảy ra:}
	\begin{itemize}
		\item [$\bullet$] Nếu $a$ và $b$ đồng phẳng (cùng thuộc một mặt phẳng) thì chúng có các khả năng: cắt nhau; song song nhau hoặc trùng nhau.
		\item [$\bullet$] Nếu $a$ và $b$ không đồng phẳng (không tồn tại mặt phẳng chưa được cả $a$ và $b$) thì ta nói $a$ và $b$ chéo nhau.
	\end{itemize}
	\begin{tabular}{cccc}
		\begin{tikzpicture}[scale=0.5,font=\small]
			\tkzDefPoints{0/0/A, 5/0/B, 6/3/C}
			\coordinate (D) at ($(A)+(C)-(B)$);
			\tkzDrawPolygon(A,B,C,D)
			\tkzMarkAngle[size=.85](B,A,D)
			\draw (A) node[above right]{$\alpha$};
			\tkzDefPoints{1/2/E, 4.5/0.5/F, 0.8/1.5/G, 5/2.5/H}
			\draw (E)--(F) (G)--(H);
			\tkzInterLL(E,F)(G,H) \tkzGetPoint{M}
			\tkzDrawPoints[size=5,fill=black](M)
			\tkzLabelPoints[above](M)
			\draw (F) node[above]{$a$};
			\draw (H) node[below]{$b$};
		\end{tikzpicture}
		&\begin{tikzpicture}[scale=0.5,font=\small]
			\tkzDefPoints{0/0/A, 5/0/B, 6/3/C}
			\coordinate (D) at ($(A)+(C)-(B)$);
			\tkzDrawPolygon(A,B,C,D)
			\tkzMarkAngle[size=.85](B,A,D)
			\draw (A) node[above right]{$\alpha$};
			\tkzDefPoints{0.8/1.5/G, 5/2.5/H, 1/0.5/I}
			\coordinate (K) at ($(H)+(I)-(G)$);
			\draw (G)--(H) (I)--(K);
			\draw ($(G)!0.8!(H)$) node[above]{$a$};
			\draw ($(I)!0.8!(K)$) node[above]{$b$};
		\end{tikzpicture}
		&\begin{tikzpicture}[scale=0.5,font=\small]
			\tkzDefPoints{0/0/A, 5/0/B, 6/3/C}
			\coordinate (D) at ($(A)+(C)-(B)$);
			\tkzDrawPolygon(A,B,C,D)
			\tkzMarkAngle[size=.85](B,A,D)
			\draw (A) node[above right]{$\alpha$};
			\tkzDefPoints{0.5/0.8/G, 5/2.5/H}
			\draw (G)--(H);
			\draw ($(G)!0.2!(H)$) node[above]{$a$};
			\draw ($(G)!0.2!(H)$) node[below]{$b$};
		\end{tikzpicture}
		&\begin{tikzpicture}[scale=0.5,font=\small]
			\tkzDefPoints{0/0/A, 5/0/B, 6/3/C}
			\coordinate (D) at ($(A)+(C)-(B)$);
			\tkzDrawPolygon(A,B,C,D)
			\tkzMarkAngle[size=.85](B,A,D)
			\draw (A) node[above right]{$\alpha$};
			\tkzDefPoints{0.8/0.5/G, 4.5/1.5/H, 2/3.5/E, 2.5/-0.5/F}
			\tkzInterLL(A,B)(E,F) \tkzGetPoint{K}
			\coordinate (I) at ($(E)!0.4!(F)$);
			\draw (G)--(H) (E)--(I) (K)--(F);
			\draw[dashed] (I)--(K);
			\draw ($(E)!0.1!(F)$) node[above right]{$a$};
			\draw ($(G)!0.9!(H)$) node[above]{$b$};
			\draw [fill=black] (I) circle(1pt);
			\tkzLabelPoints[left](I)
		\end{tikzpicture}\\
		\small * $a$ cắt $b$ & \small * $a$ song song $b$ & \small * $a$ trùng $b$ & \small * $a$ chéo $b$\\
		\small * Kí hiệu $a \cap b = M$ & \small * Kí hiệu $a \parallel b $  & \small * Kí hiệu 	$a \equiv b$ & \small * $a$, $b$ không điểm chung
	\end{tabular}
	\item \indam{Chú ý:}
		Cho hai đường thẳng $a$ và $b$ phân biệt.
		\begin{itemize}
			\item [$\bullet$] Khi kiểm tra hai đường thẳng $a$ và $b$ \textbf{song song} hay \textbf{cắt nhau} thì trước tiên chúng phải đồng phẳng (cùng thuộc một mặt phẳng nào đó);
			\item [$\bullet$] Khi $a$ và $b$ không có điểm chung thì chúng có thể song song hoặc chéo nhau. Vấn đề này các bạn hay bị nhầm lẫn, cần chú ý. 
		\end{itemize}
\end{enumerate}

\subsubsection{CÁC ĐỊNH LÝ VÀ HỆ QUẢ CẦN NHỚ}
\begin{enumerate}[\iconMT]
	\item \indam{Định lý 1:} Trong không gian, qua một điểm không nằm trên đường thẳng cho trước, có một và chỉ một đường thẳng song song với đường thẳng đã cho.
	\item \indam{Định lý 2:} Hai đường thẳng phân biệt cùng song song với đường thẳng thứ ba thì song song với nhau.
\immini{	\item \indam{Định lý 3:} Nếu ba mặt phẳng phân biệt đôi một cắt nhau theo ba giao tuyến phân biệt thì ba giao tuyến đó hoặc đồng quy hoặc đôi một song song với nhau.}{
\begin{tikzpicture}[line cap=round,line join=round,x=1.0cm,y=1.0cm, font=\small]
	\begin{scope}[>=stealth,scale=0.5]
		\tikzset{label style/.style={font=\footnotesize}}
		\tkzDefPoints{0/0/A,0/5/B, 3/4/C,3/-1/D, -3/3/E, -3/-2/F, 0/4/I}
		\coordinate (J) at ($(A)!0.77!(F)$);
		\coordinate (M) at ($(A)!0.5!(D)$);
		\coordinate (c) at ($(I)!0.5!(B)$);
		\coordinate (a) at ($(I)!0.5!(J)$);
		\coordinate (b) at ($(I)!0.5!(M)$);
		\tkzDrawSegments[dashed](A,M A,J A,I)
		\tkzDrawSegments(F,J E,F E,B B,I B,C C,D D,M M,I M,J I,J)
		\tkzMarkAngles[size=0.8cm](F,E,B)
		\tkzLabelAngles[pos=0.5,rotate=30](B,E,F){$\alpha$}
		\tkzMarkAngles[size=0.8cm](B,C,D)
		\tkzLabelAngles[pos=0.5,rotate=320](D,C,B){$\beta$}
		\tkzMarkAngles[size=0.8cm](M,J,I)
		\tkzLabelAngles[pos=0.5,rotate=30](M,J,I){$\gamma$}
		%\tkzMarkAngles[size=0.5cm]
		\tkzLabelPoints[right](b,c)
		\tkzLabelPoints[left](a)
		
	\end{scope}
	\begin{scope}[xshift=5cm,>=stealth,scale=0.5]
		\tikzset{label style/.style={font=\footnotesize}}
		\tkzDefPoints{0/0/A,0/5/B, 3/4/C,3/-1/D, -3/3/E, -3/-2/F, 0/4/I}
		\coordinate (J) at ($(A)!0.6!(F)$);
		\coordinate (P) at ($(B)!0.6!(E)$);
		\coordinate (Q) at ($(B)!0.5!(C)$);
		\tkzInterLL(A,B)(Q,P)    \tkzGetPoint{I}
		\coordinate (M) at ($(A)!0.5!(D)$);
		\coordinate (c) at ($(I)!0.5!(B)$);
		\coordinate (a) at ($(I)!0.5!(J)$);
		\coordinate (b) at ($(I)!0.5!(M)$);
		\tkzDrawSegments[dashed](A,M A,J A,I)
		\tkzDrawSegments(F,J E,F E,B B,I B,C C,D D,M M,Q M,J J,P P,Q)
		\tkzMarkAngles[size=0.8cm](F,E,B)
		\tkzLabelAngles[pos=0.5,rotate=30](B,E,F){$\alpha$}
		\tkzMarkAngles[size=0.8cm](B,C,D)
		\tkzLabelAngles[pos=0.5,rotate=320](D,C,B){$\beta$}
		\tkzMarkAngles[size=0.8cm](M,J,I)
		\tkzLabelAngles[pos=0.5,rotate=30](M,J,P){$\gamma$}
		%\tkzMarkAngles[size=0.5cm]
		\tkzLabelPoints[right](b,c)
		\tkzLabelPoints[left](a)
	\end{scope} 
\end{tikzpicture}}
	\begin{note}
		\textbf{Hệ quả:} Nếu hai mặt phẳng phân biệt lần lượt chứa hai đường thẳng song song thì giao tuyến của chúng (nếu có) cũng song song với hai đường thẳng đó hoặc trùng với một trong hai đường thẳng đó.

	\end{note}
\end{enumerate}

	