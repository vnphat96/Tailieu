\subsection{Hệ thống bài tập trắc nghiệm}
\setcounter{dang}{0}
\Opensolutionfile{ans}[ans/ans-1-C4B11-Dang3]
\begin{dang}{Câu hỏi lý thuyết}
\end{dang}
\begin{ex}%[1C4Y2-1]%Câu 1
Trong các phát biểu sau, phát biểu nào đúng?
\choice
{Hai đường thẳng không có điểm chung thì song song với nhau}
{Hai đường thẳng không có điểm chung thì chéo nhau}
{Hai đường thẳng phân biệt không cắt nhau thì song song}
{\True Hai đường thẳng không nằm trên cùng một mặt phẳng thì chéo nhau}
\loigiai{
Phương án “Hai đường thẳng không có điểm chung thì song song với nhau” sai vì hai đường thẳng có thể chéo nhau.\\
Phương án “Hai đường thẳng không có điểm chung thì chéo nhau” sai vì hai đường thẳng có thể song song.\\
Phương án “Hai đường thẳng phân biệt không cắt nhau thì song song” sai vì hai đường thẳng có thể chéo nhau.}
\end{ex}

\begin{ex}%[1C4Y2-1]%Câu 2
Cho hai đường thẳng phân biệt $a$ và $b$ trong không gian. Có bao nhiêu vị trí tương đối giữa $a$ và $b$?
\choice
{\True $ 3$}
{$ 1$}
{$ 2$}
{$ 4$}
\loigiai{
Hai đường thẳng phân biệt $a$ và $b$ trong không gian có những vị trí tương đối sau:
\begin{itemize}
\item Hai đường thẳng phân biệt $a$ và $ b$ cùng nằm trong một mặt phẳng thì chúng có thể song song hoặc cắt nhau.
\item Hai đường thẳng phân biệt $a$ và $b$ không cùng nằm trong một mặt phẳng thì chúng chéo nhau.
\end{itemize}	
Vậy chúng có $3$ vị trí tương đối là song song hoặc cắt nhau hoặc chéo nhau.
}
\end{ex}

\begin{ex}%[1C4Y2-1]%Câu 3
Trong các mệnh đề sau, mệnh đề nào đúng?
\choice
{Hai đường thẳng không có điểm chung thì song song với nhau}
{\True Hai đường thẳng chéo nhau thì không có điểm chung}
{Hai đường thẳng không song song thì cắt nhau}
{Hai đường thẳng không cắt nhau và không song song thì chéo nhau}
\loigiai{
Phương án A sai do hai đường thẳng không có điểm chung có thể chéo nhau.\\
Phương án C sai do hai đường thẳng không song song thì có thể trùng nhau hoặc chéo nhau.\\
Phương án D sai do hai đường thẳng không cắt nhau và không song song với nhau thì có thể trùng nhau.}
\end{ex}

\begin{ex}%[1C4Y2-1]%Câu 4
	Trong các mệnh đề sau, mệnh đề nào đúng?\\
	Trong không gian:
	\choice
	{Hai đường thẳng không có điểm chung thì song song}
	{Hai đường thẳng không có điểm chung thì chéo nhau}
	{Hai đường thẳng không song song, không cắt nhau thì chéo nhau}
	{\True Hai đường thẳng song song khi và chỉ khi chúng nằm trong cùng một mặt phẳng và không có điểm chung}
	\loigiai{
		Hai đường thẳng song song khi và chỉ khi chúng nằm trong cùng một mặt phẳng và không có điểm chung.}
\end{ex}

\begin{ex}%[1C4Y2-1]%Câu 5
Trong các khẳng định sau, có bao nhiêu khẳng định sai?
\begin{enumerate}[(I)]
	\item 	Hai đường thẳng chéo nhau thì chúng có điểm chung.
	\item 
	Hai đường thẳng không có điểm chung là hai đường thẳng song song hoặc chéo nhau.
	\item Hai đường thẳng song song với nhau khi chúng ở trên cùng một mặt phẳng.
	\item Khi hai đường thẳng ở trên hai mặt phẳng phân biệt thì hai đường thẳng đó chéo nhau.
\end{enumerate}
	\choice
	{$1$}
	{$2$}
	{\True $3$}
	{$4$}
\loigiai{
	\begin{itemize}
		\item (I)sai do hai đường thẳng chéo nhau thì chúng không có điểm chung.
	\item (II)	đúng.
	\item (III)	sai do có thể xảy ra trường hợp hai đường thẳng đó hoặc cắt nhau hoặc trùng nhau.
		\item (IV) sai do có thể xảy ra trường hợp hai đường thẳng đó song song.
	\end{itemize}
		Vậy có 3 khẳng định sai.}
\end{ex}

\begin{ex}%[1C4Y2-1]%Câu 6
Trong không gian, cho hai đường thẳng $ a$ và $ b$ chéo nhau. Một đường thẳng $ c$ song song với $ a$. Khẳng định nào sau đây là đúng?
\choice
{$ b$ và $ c$ chéo nhau}
{$ b$ và $ c$ cắt nhau}
{\True $ b$ và $ c$ chéo nhau hoặc cắt nhau}
{$ b$ và $ c$ song song với nhau}
	\loigiai{
Phương án A sai vì $b, c$ có thể cắt nhau.\\
Phương án B sai vì $b,c$ có thể chéo nhau.\\
Phương án D sai vì nếu $ b$ và $ c$ song song thì $ a$ và $ b$ song song hoặc trùng nhau.}
\end{ex}

\begin{ex}%[1C4Y4-1]%Câu 7
Cho ba mặt phẳng phân biệt cắt nhau từng đôi một theo ba giao tuyến $d_1,d_2,d_3$ trong đó $d_1$ song song với $d_2$. Khi đó vị trí tương đối của $d_2$ và $d_3$ là?
\choice
{Chéo nhau}
{Cắt nhau}
{\True Song song}
{trùng nhau}
\loigiai{
Ba mặt phẳng cắt nhau theo ba giao tuyến phân biệt thì ba giao tuyến đó hoặc đôi một song song hoặc đồng quy.}
\end{ex}

\begin{ex}%[1C4Y2-1]%Câu 8
Trong các mệnh đề sau, mệnh đề nào đúng?
\choice
{Hai đường thẳng không có điểm chung thì chéo nhau}
{\True Hai đường thẳng chéo nhau thì không có điểm chung}
{Hai đường thẳng không song song thì chéo nhau}
{Hai đường thẳng không cắt nhau và không song song thì chéo nhau}
\loigiai{
Đáp án A sai do hai đường thẳng không có điểm chung có thể song song với nhau.\\
Đáp án C sai do hai đường thẳng không song song thì có thể trùng nhau hoặc cắt nhau.\\
Đáp án D sai do hai đường thẳng không cắt nhau và không song song với nhau thì có thể trùng nhau.\\
Đáp án B đúng.}
\end{ex}

\begin{ex}%[1C4Y3-1]%Câu 9
Cho đường thẳng $ a$ song song với mặt phẳng $\left(\alpha\right)$. Nếu $\left(\beta\right)$ chứa $ a$ và cắt $\left(\beta\right)$ theo giao tuyến là $ b$ thì $ a$ và $ b$ là hai đường thẳng
\choice
{cắt nhau}
{trùng nhau}
{chéo nhau}
{\True song song với nhau}
\loigiai{
}
\end{ex}

\begin{ex}%[1C4Y2-2]%Câu 10
Cho hình tứ diện $ABCD$ . Khẳng định nào sau đây đúng?
\choice
{$AB$ và $ CD$ cắt nhau}
{\True $AB$ và $ CD$ chéo nhau}
{$AB$ và $ CD$ song song}
{Tồn tại một mặt phẳng chứa $AB$ và $ CD$}
\loigiai{
Do $ABCD$ là hình tứ diện nên bốn điểm$ A,\;B,\;C,\;D$ không đồng phẳng.}
\end{ex}

\begin{ex}%[1C4Y2-1]%Câu 11
Trong các khẳng định sau, khẳng định nào đúng?
\choice
{Hai đường thẳng không có điểm chung thì chéo nhau}
{Hai đường thẳng phân biệt không cắt nhau thì song song}
{\True Hai đường thẳng không cùng nằm trên một mặt phẳng thì chéo nhau}
{Hai đường thẳng không có điểm chung thì song song với nhau}
\loigiai{
}
\end{ex}

\begin{ex}%[1C4B2-2]%Câu 12
Cho hai đường thẳng chéo nhau $a$ và $ b$. Lấy $ A$, $ B$ thuộc $a$ và $ C$, $ D$ thuộc $ b$. Khẳng định nào sau đây đúng khi nói về hai đường thẳng $ AD$ và $ BC$?
\choice
{Cắt nhau}
{Song song nhau}
{Có thể song song hoặc cắt nhau}
{\True Chéo nhau}
\loigiai{
\immini{Ta có $a$ và $ b$ là hai đường thẳng chéo nhau nên $a$ và $ b$ không đồng phẳng.\\
		Giả sử $ AD$ và $ BC$ đồng phẳng.
		\begin{itemize}
			\item Nếu $ AD\cap BC=M$ thì $M\in\left(ABCD\right)\Rightarrow M\in\left(a;b\right)$.\\
		Mà $a$ và $ b$ không đồng phẳng, do đó không tồn tại điểm $ M$.
		\item Nếu $ AD \parallel BC$ thì  $a$ và $ b$ đồng phẳng.
\end{itemize}
		Vậy điều giả sử là sai. Do đó $ AD$ và $ BC$ chéo nhau.}{\begin{tikzpicture}[line join=round, line cap=round,>=stealth,thick]
			%%mp 1
			\coordinate (I) at (0,0);
			\coordinate (J) at (2,2);
			\coordinate (K) at (5,0);
			\coordinate (L) at ($(J) + (5,0)$);
	\draw[name path=AB] (I)--(J)--(L)--(K)--(I);
			%%mp2
				\coordinate (I') at (0,3);
			\coordinate (J') at (2,5);
			\coordinate (K') at (5,3);
			\coordinate (L') at ($(J') + (5,0)$);
			\draw[name path=AB] (I')--(J')--(L')--(K')--(I');
			%%%%ve duong thang b
				\coordinate (E) at (2,1.5);
				\coordinate (F) at (5,0.5);
					 \coordinate (D) at ([scale around={1/3:(E)}] F);
					 \coordinate (C) at ([scale around={2/3:(E)}] F);
				\draw[name path=AB] (E)--(F);
				%%%%%ve duong thang a
				 \coordinate (E') at (2,3.5);
				\coordinate (F') at (5,4.5);
				\coordinate (A) at ([scale around={1/4:(E')}] F');
				\coordinate (B) at ([scale around={3/4:(E')}] F');
				\draw[name path=AB] (E')--(F');
				%%%
				\draw[name path=AB] (A)--(D);
				\draw[name path=AB] (B)--(C);
				%nhan diem
				\foreach \i in {C,D,A,B}{\fill[](\i) circle (1.5pt);}
				\foreach \i in {A,B}{\draw(\i) node[scale=0.9, above]{$\i$};}
				\foreach \i in {C,D}{\draw(\i) node[scale=0.9, below]{$\i$};}
	\end{tikzpicture}}
	}
\end{ex}

\begin{ex}%[1C4Y2-1]%Câu 13
Trong không gian cho ba đường thẳng phân biệt $ a$, $ b$, $ c$ trong đó $ a$ song song với $ b$. Khẳng định nào sau đây sai?
	\choice
{Tồn tại duy nhất một mặt phẳng chứa cả hai đường thẳng $a$ và $ b$}
{Nếu $ b$ song song với $ c$ thì $ a$ song song với $ c$}
{Nếu điểm $ A$ thuộc $ a$ và điểm $ B$ thuộc $ b$ thì ba đường thẳng $ a$, $ b$ và $ AB$ cùng ở trên một mặt phẳng}
{\True Nếu $ c$ cắt $ a$ thì $ c$ cắt $ b$}
\loigiai{
Mệnh đề “Nếu $ c$ cắt $ a$ thì $ c$ cắt $ b$” là mệnh đề sai, vì $ c$ và $ b$ có thể chéo nhau.}
\end{ex}

\begin{ex}%[1C4B2-1]%Câu 14
Cho đường thẳng $a$ nằm trên $mp(P)$, đường thẳng $b$ cắt $(P)$ tại $O$ và $O$ không thuộc $a$. Vị trí tương đối của $a$ và $b$ là
\choice
{\True chéo nhau}
{cắt nhau}
{song song với nhau}
{trùng nhau}
\loigiai{
\immini{Do đường thẳng $a$ nằm trên $mp(P)$ , đường thẳng $b$ cắt $(P)$ tại $O$ và $O$ không thuộc $a$ nên đường thẳng $a$ và đường thảng $b$ không đồng phẳng nên vị trí tương đối của $a$ và $b$ là chéo nhau.}{\begin{tikzpicture}[line join=round, line cap=round,>=stealth,thick]
				%%mp 1
\coordinate (I) at (0,0);
\coordinate (J) at (2,2);
\coordinate (K) at (5,0);
\coordinate (A) at (3,3);
\coordinate (O) at (4,1);
\coordinate (L) at ($(J) + (5,0)$);
\draw[name path=AB] (I)--(J)--(L)--(K)--(I);
\draw[name path=AB] (A)--(O);
%%%%ve duong thang b
\coordinate (a) at (1,0.5);
\coordinate (E) at (3,1.5);
	\draw[name path=AB] (E)--(a);
				  \draw pic["$P$",draw,angle eccentricity=0.8,angle radius=0.7cm]{angle=K--I--J};
	\draw (3.3,2.3)node[right]{$b$}	;		  
			%nhan diem
				\foreach \i in {a,O}{\fill[](\i) circle (1.5pt);}
				\foreach \i in {a}{\draw(\i) node[scale=0.9, above]{$\i$};}
				\foreach \i in {O}{\draw(\i) node[scale=0.9, right]{$\i$};}
				  
	\end{tikzpicture}}	
}
\end{ex}

\begin{ex}%[1C4B2-1]%Câu 15
Cho hai đường thẳng chéo nhau $ a$, $ b$ và điểm $ M$ không thuộc $ a$ cũng không thuộc $ b$. Có nhiều nhất bao nhiêu đường thẳng đi qua $ M$ và đồng thời cắt cả $ a$ và $ b$?
\choice
{$ 4$}
{$ 3$}
{$ 2$}
{\True $ 1$}
\loigiai{
Gọi $(P)$ là mặt phẳng qua $ M$ và chứa $ a$; $(Q)$ là mặt phẳng qua $ M$ và chứa $ b$.\\
Giả sử tồn tại đường thẳng $ c$ đi qua $ M$ và đồng thời cắt cả $ a$ và $ b$ suy ra $\heva{&c\in(P)\\&c\in(Q)}\Rightarrow c=(P)\cap(Q)$.\\
Mặt khác nếu có một đường thẳng $ c'$ đi qua $ M$ và đồng thời cắt cả $ a$ và $ b$ thì $ a$ và $ b$ đồng phẳng.\\
Do đó có duy nhất một đường thẳng đi qua $ M$ và đồng thời cắt cả $ a$ và $ b$.}
\end{ex}

\begin{ex}%[1C4B2-1]%Câu 16
Trong không gian cho đường thẳng $ a$ chứa trong mặt phẳng $(P)$ và đường thẳng $ b$ song song với mặt phẳng $(P)$. Mệnh đề nào sau đây là đúng?
\choice
{$a\parallel b$}
{\True $ a$, $ b$ không có điểm chung}
{$ a$, $ b$ cắt nhau}
{$ a$, $ b$ chéo nhau}
\loigiai{
	\immini{
\begin{itemize}
\item $ b{\rm{//}}(P)$ thì $ b$ có thể song song với $ a$ mà $ b$ cũng có thể chéo $ a$.
\item $ b{\rm{//}}(P)$$\Rightarrow b\cap(P)=\varnothing $ $\Rightarrow b\cap a=\varnothing $. 
\end{itemize}
Vậy $ a$, $ b$ không có điểm chung.}{\begin{tikzpicture}[line join=round, line cap=round,>=stealth,thick]
	%%mp 1
	\coordinate (O) at (0,0);
	\coordinate (B) at (2,2);
	\coordinate (A) at ($(O)!0.5!(B)$);
	 \coordinate (D) at ($(B) + (5,0)$);
	\coordinate (E) at (5,0);
	\coordinate (A') at (1,0.5);
	  \coordinate (B') at (5.5,1.5);
	 \coordinate (C) at ([scale around={3/4:(B)}] D);
 \coordinate (D') at ([scale around={4:(A')}] A);
  \coordinate (Tempt1) at ($(A')+(4.5,1)$);
 \coordinate (Tempt2) at ($(D')+(4.5,1)$);
 \draw[name path=a] (Tempt1)--(Tempt2);
 \coordinate (Tempt3) at ($(A')+(0,2)$);
 \coordinate (Tempt4) at ($(B')+(0,2)$);
 \draw[name path=a] (Tempt3)--(Tempt4);
	\draw[name path=AB] (O)--(A) (A')--(D') (C)--(D)--(E)--(O) ;
	\draw[name path=AB][dashed] (A)--(B)--(C);
	\draw[name path=AB] (A')--(B');
	%%%%ve duong thang b
	\coordinate (a) at (1,0.5);
	\coordinate (E) at (3,1.5);
	 	\draw pic["$Q$",draw,angle eccentricity=0.6,angle radius=0.7cm]{angle=A'--D'--Tempt4};
	\draw pic["$P$",draw,angle eccentricity=0.8,angle radius=0.7cm]{angle=K--I--J};
		  \draw (3,1)node[below]{$a$} (3,2.9)node[above]{$b$}; 
	%nhan diem
%	\foreach \i in {a,O}{\fill[](\i) circle (1.5pt);}
%	\foreach \i in {a}{\draw(\i) node[scale=0.9, above]{$\i$};}
%	\foreach \i in {O}{\draw(\i) node[scale=0.9, right]{$\i$};}
\end{tikzpicture}\begin{tikzpicture}[line join=round, line cap=round,>=stealth,thick]
%%mp 1
\coordinate (O) at (0,0);
\coordinate (B) at (2,2);
\coordinate (C) at (2.4,3);
\coordinate (C') at (6.4,3);
\coordinate (D) at ($(B) + (5,0)$);
\coordinate (E) at (5,0);
\coordinate (A') at (2,0.5);
\coordinate (B') at (5.2,1.5);
\draw[name path=AB] (O)--(B)--(D)--(E)--(O) ;
 \draw[name path=AB] (A')--(B') (C)--(C');
%%%%ve duong thang b
\draw pic["$P$",draw,angle eccentricity=0.8,angle radius=0.7cm]{angle=K--I--J};
\draw (3.5,1)node[below]{$a$} (4,2.9)node[above]{$b$}; 
%nhan diem
%	\foreach \i in {a,O}{\fill[](\i) circle (1.5pt);}
%	\foreach \i in {a}{\draw(\i) node[scale=0.9, above]{$\i$};}
%	\foreach \i in {O}{\draw(\i) node[scale=0.9, right]{$\i$};}
\end{tikzpicture}}
}
\end{ex}

\begin{ex}%[1C4Y2-1]%Câu 17
Trong các mệnh đề sau, mệnh đề nào đúng?
\choice
{Trong không gian hai đường thẳng không có điểm chung thì chéo nhau}
{\True Trong không gian hai đường thẳng lần lượt nằm trên hai mặt phẳng phân biệt thì chéo nhau}
{Trong không gian hai đường thẳng phân biệt không song song thì chéo nhau}
{Trong không gian hai đường chéo nhau thì không có điểm chung}
\loigiai{
Áp dụng định nghĩa hai đường thẳng được gọi là chéo nhau nếu chúng không đồng phẳng.
}
\end{ex}
\Closesolutionfile{ans}
\begin{indapan}{10} 
	{ans/ans-1-C4B11-Dang3}
\end{indapan}
\begin{dang}{Một số bài toán liên quan đến hai đường thảng song song}
\end{dang}
\Opensolutionfile{ans}[ans/ans-1-C4B11-Dang4]
\begin{ex}%[1C4B2-2]%Câu 18
Cho hình chóp $ S.ABCD$ có đáy $ ABCD$là hình bình hành tâm $ O$. Gọi $ I,$ $ J$ lần lượt là trung điểm $ SA$, $ SC$. Đường thẳng $ IJ$ song song với đường thẳng nào trong các đường thẳng sau?
\choice
{\True $ AC$}
{$ BC$}
{$ SO$}
{$ BD$}
\loigiai{
\immini{Do $ IJ$ là đường trung bình của tam giác $ SAC\Rightarrow IJ\parallel AC$.}{\begin{tikzpicture}[line join=round, line cap=round,thick,scale=0.6]
\coordinate (A) at (0,0);
 		\coordinate (B) at (2,-2);
	 		\coordinate (D) at (5,0);
	 		\coordinate (C) at ($(B)+(D)-(A)$);
	 		\coordinate (O) at ($(A)!0.5!(C)$);
	 		\coordinate (S) at ($(O)+(0,7)$);
	 		\coordinate (I) at ($(S)!0.5!(A)$);
	 		\coordinate (J) at ($(S)!0.5!(C)$);
	 		\draw(S)--(A) (S)--(B) (S)--(C) (A)--(B) (B)--(C);
	 		\draw[dashed,thin](A)--(C) (A)--(D) (C)--(D) (S)--(D) (B)--(D) (I)--(J);
	 		 
	 		\foreach \i/\g in {S/90,A/180,B/-90,C/-90,D/0,O/-90,I/180,J/0}{\draw[fill=white](\i) circle (1.5pt) ($(\i)+(\g:3mm)$) node[scale=1]{$\i$};}
	 \end{tikzpicture}}
}
\end{ex}

\begin{ex}%[1C4B2-2]%Câu 19
Cho hình chóp $ S.ABC$ và $ G, K$ lần lượt là trong tâm tam giác $ SAB$, $SBC$. Khẳng định nào sau đây là đúng?
\choice
{$ GK\parallel AB$}
{$ GK\parallel BC$}
{\True $ GK\parallel AC$}
{$ GK\parallel SB$}
\loigiai{
\immini{Gọi $ M,N$ lần lượt là trung điểm của $ AB,BC$. \\
Khi đó 	$\dfrac{SG}{SM}=\dfrac{2}{3}$ và $\dfrac{SK}{SN}=\dfrac{2}{3}$ suy ra $\dfrac{SG}{SM}=\dfrac{SK}{SN}$.\\
		Suy ra $ GK\parallel MN$ mà $ MN\parallel AC$.\\
Nên $GK\parallel AC$.}{\begin{tikzpicture}[line join=round, line cap=round,thick,scale=0.6]
			\coordinate (S) at (0,4);
			\coordinate (A) at (-2,0);
			\coordinate (B) at (0,-3);
			\coordinate (C) at (3,0);
			\coordinate (M) at ($(A)!0.5!(B)$);
			\coordinate (N) at ($(C)!0.5!(B)$);
			\coordinate (G) at ([scale around={2/3:(S)}] M);
			\coordinate (K) at ([scale around={2/3:(S)}] N);
			\draw(S)--(A) (S)--(B) (S)--(C) (B)--(C) (A)--(B) (S)--(M) (S)--(N);
			\draw[dashed,thin](A)--(C) (M)--(N) (G)--(K);
			\foreach \i/\g in {S/90,A/180,B/-90,C/0,M/180,N/0,G/180,K/0}{\draw[fill=white](\i) circle (1.5pt) ($(\i)+(\g:4mm)$) node[scale=1]{$\i$};}
	\end{tikzpicture}}
	
}
\end{ex}

\begin{ex}%[1C4B2-2]%Câu 20
Cho hình chóp $ S.ABCD$ có $ AD$ không song song với $ BC$. Gọi $M; N; P; Q; R; T$ lần lượt là trung điểm $AC; BD; BC; CD; SA$ và $SD$. Cặp đường thẳng nào sau đây song song với nhau?
\choice
{$ MP$ và $ RT$}
{\True $ MQ$ và $ RT$}
{$ MN$ và $ RT$}
{$ PQ$ và $ RT$}
\loigiai{
 \immini{Ta có: $M$, $Q$ lần lượt là trung điểm của $AC$, $CD$\\
$\Rightarrow MQ$ là đường trung bình của tam giác $ CAD\Rightarrow MQ\,\parallel\,AD\,\,\,\,(1)$\\
Ta có: $R$, $T$ lần lượt là trung điểm của $SA$, $SD$\\
$\Rightarrow RT$ là đường trung bình của tam giác $SAD\Rightarrow RT\,\parallel\,AD\,\,\,(2)$\\
Từ $(1),(2)$ suy ra: $ MQ\,\parallel\,RT.$}{\begin{tikzpicture}[line join=round, line cap=round,thick,scale=0.6]
			\coordinate (S) at (1,4);
			\coordinate (A) at (0,0);
			\coordinate (B) at (1,-2);
			\coordinate (C) at (3,-3);
			\coordinate (D) at (6,0);
			\coordinate (M) at ($(A)!0.5!(C)$);
			\coordinate (N) at ($(B)!0.5!(D)$);
			\coordinate (P) at ($(B)!0.5!(C)$);
			\coordinate (Q) at ($(D)!0.5!(C)$);
			\coordinate (R) at ($(A)!0.5!(S)$);
			\coordinate (T) at ($(S)!0.5!(D)$);
			\draw(S)--(A) (S)--(B) (S)--(C) (S)--(D) (B)--(C) (A)--(B)--(C)--(D);
			\draw[dashed,thin](A)--(D) (A)--(C) (B)--(D) (R)--(T) (M)--(P)--(Q)--(M) (M)--(N);
			\foreach \i/\g in {S/90,A/180,B/-90,C/-90,D/0,M/180,N/0,P/180,Q/0,R/180,T/0}{\draw[fill=black](\i) circle (1.5pt) ($(\i)+(\g:3.5mm)$) node[scale=1]{$\i$};}
	\end{tikzpicture}}
}
\end{ex}

\begin{ex}%[1C4B2-2]%Câu 21
Cho hình chóp $ S.ABCD$ có đáy là hình bình hành. Gọi $G_1$; $G_2$ lần lượt là trọng tâm của $\Delta SAB$; $\Delta SAD$. Khi đó $G_1G_2$ song song với đường thẳng nào sau đây?
\choice
{$ CD$}
{\True $ BD$}
{$ AD$}
{$ AB$}
\loigiai{
\immini{Gọi $ N$ là trung điểm của $ SA$.\\
		Vì $G_1$; $G_2$ lần lượt là trọng tâm của $\Delta SAB$; $\Delta SAD$ nên ta có: $\dfrac{N{G_1}}{NB}=\dfrac{N{G_2}}{ND}=\dfrac{1}{3}$$\Rightarrow{G_1}{G_2}//BD$.}{\begin{tikzpicture}[line join=round, line cap=round,thick,scale=1]
			\coordinate (A) at (0,0);
			\coordinate (B) at (-2.8,-3);
			\coordinate (D) at (7,0);
			\coordinate (C) at ($(B)+(D)-(A)$);
			\coordinate (S) at ($(A)+(0,5)$);
			\coordinate (N) at ($(A)!0.5!(S)$);
			\coordinate (G_1) at ([scale around={1/3:(N)}] B);
			\coordinate (G_2) at ([scale around={1/3:(N)}] D);
			\draw(S)--(B) (S)--(C) (S)--(D) (B)--(C)--(D);
			\draw[dashed,thin](S)--(A) (A)--(B) (A)--(D) (B)--(D) (N)--(B) (N)--(D) (G_1)--(G_2);
			\foreach \i/\g in {S/90,A/-90,B/-90,C/-90,D/0,N/180,G_1/180,G_2/90}{\draw[fill=white](\i) circle (1.5pt) ($(\i)+(\g:3mm)$) node[scale=1]{$\i$};}
	\end{tikzpicture}}
	}
\end{ex}

\begin{ex}%[1C4B2-2]%Câu 22
Cho hình chóp $ S.ABCD$ có đáy $ ABCD$ là hình chữ nhật. Gọi $ M$, $N$ lần lượt là trung điểm của $ AB$, $CD$ và $G_1$, $G_2$ lần lượt là trọng tâm của các cạnh tam giác $SAB$, $SCD$. Trong các đường thẳng sau đây, đường thẳng nào không song song với $G_1G_2$?
\choice
{$ AD$}
{$ BC$}
{\True $ SA$}
{$ MN$}
\loigiai{
 \immini{Gọi $ M$, $N$ lần lượt là trung điểm của $AB$, $CD$ và $G_1$, $G_2$ lần lượt là trọng tâm của các tam giác $ SAB$, $ SCD$ nên $G_1\in SM,G_2\in SN$\\
Và $\dfrac{S{G_1}}{SM}=\dfrac{S{G_2}}{SN}=\dfrac{1}{3}$. Suy ra $G_1G_2\parallel MN\parallel \left(AD\parallel BC\right)$.}{\begin{tikzpicture}[line join=round, line cap=round,thick]
\coordinate (A) at (0,0);
			\coordinate (B) at (-2.8,-3);
			\coordinate (D) at (7,0);
			\coordinate (C) at ($(B)+(D)-(A)$);
			\coordinate (S) at ($(A)+(0,5)$);
			\coordinate (M) at ($(A)!0.5!(B)$);
			\coordinate (N) at ($(C)!0.5!(D)$);
			\coordinate (G_1) at ([scale around={2/3:(S)}] M);
			\coordinate (G_2) at ([scale around={2/3:(S)}] N);
			\draw(S)--(B) (S)--(C) (S)--(D) (B)--(C)--(D) (S)--(N);
			\draw[dashed,thin](S)--(A) (A)--(B) (A)--(D) (S)--(M) (M)--(N) (G_1)--(G_2);
		 	\foreach \i/\g in {S/90,A/-90,B/-90,C/-90,D/0,M/180,N/0,G_1/180,G_2/0}{\draw[fill=white](\i) circle (1.5pt) ($(\i)+(\g:3mm)$) node[scale=1]{$\i$};}
	\end{tikzpicture}}
	
}
\end{ex}

\begin{ex}%[1C4B2-2]%Câu 23
Cho hình chóp $ S.ABCD$ có đáy là hình bình hành. Gọi $ A'$, $B'$, $C'$, $D'$ lần lượt là trung điểm của các cạnh $ SA$, $SB$, $SC$, $SD$. Đường thẳng không song song với $ A'B'$ là
\choice
{$ C'D'$}
{$ AB$}
{$ CD$}
{\True $ SC$}
\loigiai{
 \immini{Ta có $ C'D'\parallel CD$; $A'B'\parallel AB$, mà $ AB\parallel CD.$ \\
 Do đó 
		$ A'B' \parallel C'D'$ và $ A'B'\parallel CD$.}{\begin{tikzpicture}[line join=round, line cap=round,thick]
			\coordinate (A) at (0,0);
			\coordinate (B) at (-2.8,-3);
			\coordinate (D) at (7,0);
			\coordinate (C) at ($(B)+(D)-(A)$);
			\coordinate (S) at ($(A)+(0,5)$);
			\coordinate (A') at ($(A)!0.5!(S)$);
			\coordinate (B') at ($(S)!0.5!(B)$);
			 \coordinate (C') at ($(S)!0.5!(C)$);
			 	\coordinate (D') at ($(S)!0.5!(D)$);
			\draw(S)--(B) (S)--(C) (S)--(D) (B)--(C)--(D) (B')--(C')--(D');
			\draw[dashed,thin](S)--(A) (A)--(B) (A)--(D) (A')--(B') (A')--(D')  ;
			\foreach \i/\g in {S/90,A/-90,B/-90,C/-90,D/0,A'/180,B'/180,C'/0,D'/0}{\draw[fill=white](\i) circle (1.5pt) ($(\i)+(\g:3mm)$) node[scale=1]{$\i$};}
	\end{tikzpicture}}
	
}
\end{ex}

\begin{ex}%[1C4B2-2]%Câu 24
Cho tứ diện $ ABCD$ và $ M$, $N$ lần lượt là trọng tâm của tam giác $ ABC$, $ABD$. Khẳng định nào sau đây là đúng?
\choice
{\True $ MN\parallel CD$}
{$ MN \parallel AD$}
{$ MN \parallel BD$}
{$ MN\parallel CA$}
\loigiai{
 \immini{
 	\begin{itemize}
 		\item Dễ thấy $ MN,\,AD$ là hai đường thẳng chéo nhau nên loại.
\item 	Dễ thấy $ MN,\,BD$ là hai đường thẳng chéo nhau nên loại.
		\item Dễ thấy $ MN,\,CA$ là hai đường thẳng chéo nhau nên loại.
\end{itemize}}{\begin{tikzpicture}[line join=round, line cap=round,thick,scale=0.6]
			\coordinate (A) at (0,4);
			\coordinate (C) at (-2,0);
			\coordinate (B) at (0,-3);
			\coordinate (D) at (3,0);
			\coordinate (I) at ($(C)!0.5!(B)$);
			\coordinate (J) at ($(D)!0.5!(B)$);
			\coordinate (M) at ([scale around={2/3:(A)}] I);
			\coordinate (N) at ([scale around={2/3:(A)}] J);
			\draw(A)--(C) (A)--(B) (A)--(D) (B)--(C) (C)--(B) (A)--(I) (A)--(J) (B)--(D);
			\draw[dashed,thin](D)--(C) (I)--(J) (M)--(N);
			\foreach \i/\g in {A/90,C/180,B/-90,D/0,M/180,N/0,I/180,J/0}{\draw[fill=white](\i) circle (1.5pt) ($(\i)+(\g:4mm)$) node[scale=1]{$\i$};}
	\end{tikzpicture}}
	}
\end{ex}

\begin{ex}%[1C4B2-2]%Câu 25
Cho hình chóp $S.ABCD$ đáy là hình bình hành tâm $O$, $I$ là trung điểm của $SC$, xét các mệnh đề:
\begin{itemize}
	\item [(I)]	Đường thẳng $IO$ song song với $SA$.
	\item [(II)] Mặt phẳng $\left(IBD\right)$ cắt hình chóp $S.ABCD$ theo thiết diện là một tứ giác.
	\item[(III)] Giao điểm của đường thẳng $AI$ với mặt phẳng $\left(SBD\right)$ là trọng tâm của tam giác $\left(SBD\right)$.
	\item [(IV)] Giao tuyến của hai mặt phẳng $\left(IBD\right)$ và $\left(SAC\right)$ là $IO$.
\end{itemize}
Số mệnh đề đúng trong các mệnh để trên là
	\choice
	{$2$}
	{$4$}
	{\True $3$}
	{$1$}
\loigiai{
\begin{itemize}
	\item[(I)]
	Mệnh đề đúng vì $IO$ là đường trung bình của tam giác $SAC$.
		\item [(II)] Mệnh đề sai vì tam giác $IBD$ chính là thiết diện của hình chóp $S.ABCD$ cắt bởi mặt phẳng $\left(IBD\right)$.
		\item [(III)] Mệnh đề đúng vì giao điểm của đường thẳng $AI$ với mặt phẳng $\left(SBD\right)$ là giao điểm của $AI$ với $SO$.
	\item [(IV)]	Mệnh đề đúng vì $I,\,O$ là hai điểm chung của 2 mặt phẳng $\left(IBD\right)$ và $\left(SAC\right)$.
\end{itemize}
		Vậy số mệnh đề đúng trong các mệnh để trên là $3$.}
\end{ex}

\begin{ex}%[1C4B2-2]%Câu 26
Cho tứ diện $ ABCD$. Gọi $ I$ và $ J$ lần lượt là trọng tâm $\Delta ABC$ và $\Delta ABD$. Mệnh đề nào dưới đây đúng?
\choice
{\True $ IJ$ song song với $ CD$}
{$ IJ$ song song với $ AB$}
{$ IJ$ chéo nhau với $ CD$}
{$ IJ$ cắt $ AB$}
\loigiai{
\immini{Gọi $ E$ là trung điểm $ AB$.\\
		Vì $ I$ và $ J$ lần lượt là trọng tâm tam giác $ ABC$ và $ ABD$ nên: $\dfrac{EI}{EC}=\dfrac{EJ}{ED}=\dfrac{1}{3}$\\
		Suy ra: $ IJ//CD$.}{\begin{tikzpicture}[line join=round, line cap=round,thick]
			\coordinate (A) at (0,4);
			\coordinate (B) at (-2,0);
			\coordinate (C) at (0,-3);
			\coordinate (D) at (3,0);
		\coordinate (E) at ($(A)!0.5!(B)$);
			\coordinate (I) at ([scale around={1/3:(E)}] C);
			\coordinate (J) at ([scale around={1/3:(E)}] D);
			\draw(A)--(C) (A)--(B) (A)--(D) (C)--(D) (C)--(B) (E)--(C);
			\draw[dashed,thin](B)--(D) (E)--(D) (I)--(J);
			\foreach \i/\g in {A/90,C/180,B/-90,D/0,E/180,I/180,J/0}{\draw[fill=white](\i) circle (1.5pt) ($(\i)+(\g:3mm)$) node[scale=1]{$\i$};}
	\end{tikzpicture}}
	}
\end{ex}

\begin{ex}%[1C4B2-2]%Câu 27
Cho hình chóp $ S.ABCD$ có đáy $ ABCD$ là hình thang với đáy lớn$ AD$, $ AD=2BC$. Gọi $ G$ và $ G'$ lần lượt là trọng tâm tam giác $ SAB$ và $ SAD$. Đường thẳng $GG'$ song song với đường thẳng
\choice
{$ AB$}
{$ AC$}
{\True $ BD$}
{$ SC$}
\loigiai{
 \immini{Gọi $ H$ và $ K$ lần lượt là trung điểm cạnh $ AB;\,AD$. Với $ G$ và $ G'$ lần lượt là trọng tâm tam giác $ SAB$ và $ SAD$ ta có: $\dfrac{SG}{SH}=\dfrac{SG'}{SK}=\dfrac{2}{3}\Rightarrow GG'//HK$.\\
		Mà $ HK//BD$ ($ HK$ là đường trung bình tam giác $ ABD$.\\
		Từ và suy ra $ GG'$song song với $ BD.$}{\begin{tikzpicture}[line join=round, line cap=round,thick]
			\coordinate (A) at (0,0);
			\coordinate (B) at (-2,-3);
			\coordinate (D) at (7,0);
			\coordinate (C) at (3.5,-3 );
			\coordinate (S) at ($(A)+(0,5)$);
			\coordinate (H) at ($(A)!0.5!(B)$);
			\coordinate (K) at ($(A)!0.5!(D)$);
			\coordinate (G) at ([scale around={2/3:(S)}] H);
			\coordinate (G') at ([scale around={2/3:(S)}] K);
			\draw(S)--(B) (S)--(C) (S)--(D) (B)--(C)--(D);
			\draw[dashed,thin](S)--(A) (A)--(B) (A)--(D) (B)--(D) (S)--(H) (S)--(K) (H)--(K) (G)--(G');
			 \foreach \i/\g in {S/90,A/-90,B/-90,C/-90,D/0,H/180,K/90,G/180,G'/0}{\draw[fill=white](\i) circle (1.5pt) ($(\i)+(\g:3mm)$) node[scale=1]{$\i$};}
	\end{tikzpicture}}
	
}
\end{ex}

\begin{ex}%[1C4B2-2]%Câu 28
Cho tứ diện $ABCD$. Gọi $G$ và $ E$ lần lượt là trọng tâm của tam giác $ABD$ và $ABC$. Mệnh đề nào dưới đây đúng
\choice
{$ GE$ và $CD$ chéo nhau}
{\True $ GE\parallel CD$}
{$GE$ cắt $ AD$}
{$ GE$ cắt $CD$}
\loigiai{
 \immini{Gọi $ M$ là trung điểm của $ AB$.\\ Trong tam giác $ MCD$ có $\dfrac{MG}{MD}=\dfrac{ME}{MC}=\dfrac{1}{3}$ suy ra $ GE{\rm{//}}CD$}{ \begin{tikzpicture}[line join=round, line cap=round,thick]
 		\coordinate (A) at (0,4);
 		\coordinate (B) at (-2,0);
 		\coordinate (C) at (0,-3);
 		\coordinate (D) at (3,0);
 		\coordinate (M) at ($(A)!0.5!(B)$);
 		\coordinate (E) at ([scale around={1/3:(M)}] C);
 		\coordinate (G) at ([scale around={1/3:(M)}] D);
 		\draw(A)--(C) (A)--(B) (A)--(D) (C)--(D) (C)--(B) (M)--(C);
 		\draw[dashed,thin](B)--(D) (M)--(D) (E)--(G);
 		\foreach \i/\g in {A/90,C/180,B/-90,D/0,M/180,E/180,G/0}{\draw[fill=white](\i) circle (1.5pt) ($(\i)+(\g:3mm)$) node[scale=1]{$\i$};}
 \end{tikzpicture}}
 }
\end{ex}

\begin{ex}%[1C4B2-2]%Câu 29
Cho hình tứ diện $ ABCD$, lấy điểm $ M$ tùy ý trên cạnh $AD$ $\left(M\ne A,\;D\right)$. Gọi $(P)$ là mặt phẳng đi qua $ M$ song song với mặt phẳng $\left(ABC\right)$ lần lượt cắt $ BD$, $DC$ tại $ N$, $ P$. Khẳng định nào sau đây sai?
\choice
{\True $ MN\parallel AC$}
{$ MP \parallel AC$}
{$ MP\parallel \left(ABC\right)$}
{$ NP\parallel BC$}
\loigiai{
\immini{Do $(P)\parallel \left(ABC\right)$ nên $ AB\parallel (P)$.\\
Lại có $\heva{&	MN=(P)\cap\left(ABD\right)\\
			&AB\subset\left(ABD\right),\;AB\parallel (P).}$\\
Suy ra, $MN\parallel AB$, mà $ AB$ cắt $ AC$ nên $ MN\parallel AC$ là sai.}{ \begin{tikzpicture}[line join=round, line cap=round,thick]
			\coordinate (A) at (0,4);
			\coordinate (B) at (-2,0);
			\coordinate (C) at (-1,-2);
			\coordinate (D) at (3,0);
			\coordinate (M) at ([scale around={1/3:(A)}] D);
			\coordinate (N) at ([scale around={1/3:(B)}] D);
			\coordinate (P) at ([scale around={1/3:(C)}] D);
			\draw(A)--(C) (A)--(B) (A)--(D) (C)--(D) (C)--(B) (M)--(P);
			\draw[dashed,thin](B)--(D) (M)--(N) (N)--(P);
			\foreach \i/\g in {A/90,C/180,B/-90,D/0,M/0,N/-90,P/-90}{\draw[fill=white](\i) circle (1.5pt) ($(\i)+(\g:3mm)$) node[scale=1]{$\i$};}
	\end{tikzpicture}}
	
}
\end{ex}

\begin{ex}%[1C4B2-2]%Câu 30
Cho tứ diện $ ABCD$. Gọi $ I,J$ lần lượt là trọng tâm của các tam giác $ ABC$, $ABD$. Đường thẳng $IJ$ song song với đường thẳng:
	\choice
{$ CM$ trong đó $ M$ là trung điểm $ BD$}
{$ AC$}
{$ DB$}
{\True $ CD$}
\loigiai{
 \begin{itemize}
 	\item \textbf{Cách 1:}\\
Gọi $ E$ là trung điểm của $ AB$. Ta có $\heva{&I\in CE\\&J\in DE}$ nên suy ra $IJ$ và $ CD$ đồng phẳng.\\
Do $ I,J$ lần lượt là trọng tâm của các tam giác $ ABC$, $ABD$ nên ta có: $\dfrac{EI}{EC}=\dfrac{EJ}{ED}=\dfrac{1}{3}$. Suy ra $IJ\parallel CD$.
\item \textbf{Cách 2:}\\
Gọi $ M,N$ lần lượt là trung điểm của $ BD$ và $ BC$. Suy ra $MN\parallel CD$ .\\
		Do $ I,J$ lần lượt là trọng tâm của các tam giác $ ABC,ABD$ nên ta có:$\dfrac{AI}{AN}=\dfrac{AJ}{AM}=\dfrac{2}{3}$. Suy ra $IJ\parallel MN$ .\\
		Từ và suy ra $IJ\parallel CD$.
\item \textbf{Cách 3:}\\
Có lẽ trong ví dụ này cách này hơi dài, song chúng tôi vẫn sẽ trình bày ở đây, để các bạn có thể hiểu và vận dụng cách 3 hợp lí trong các ví dụ khác.\\
Dễ thấy, bốn điểm $D$ , $C$ , $I$ , $J$ đồng phẳng.\\
Ta có: $\heva{&\left(DCIJ\right)\cap\left(AMN\right)=IJ\\&\left(DCIJ\right)\cap\left(BCD\right)=CD\\&\left(AMN\right)\cap\left(BCD\right)=MN\\&MN\parallel CD.}$\\
Suy ra $IJ\parallel CD\parallel MN$.
	
\end{itemize}
}
\end{ex}

\begin{ex}%[1C4K2-2]%Câu 31
Cho hình chóp $ S.ABCD$ có đáy $ ABCD$ là hình chữ nhật. Gọi $ M,N$ theo thứ tự là trọng tâm $\Delta SAB$; $\Delta SCD$. Gọi I là giao điểm của các đường thẳng $ BM$; $CN$. Khi đó tỉ số $\dfrac{SI}{CD}$ bằng
\choice
{\True $ 1$}
{$\dfrac{1}{2}$}
{$\dfrac{2}{3}$}
{$\dfrac{3}{2}$}
\loigiai{
\immini{Gọi $E$ và $F$ lần lượt là trung điểm $AB$ và $CD$.\\
Ta có $ I=BM\cap CN$ nên  $\heva{&I\in BM\subset\left(SAB\right)\\&I\in CN\subset\left(SCD\right)}$\\ 
hay $I\in\left(SAB\right)\cap\left(SCD\right).$\\
	Mà $ S\in\left(SAB\right)\cap\left(SCD\right)$. Do đó $\left(SAB\right)\cap\left(SCD\right)=SI.$\\
Lại có: $\left.\begin{array}{l}
		AB//CD\\
		AB\subset\left(SAB\right)\\
		CD\subset\left(SCD\right)\\
		\left(SAB\right)\cap\left(SCD\right)=SI
	\end{array}\right\}\Rightarrow SI//AB//CD$.\\
Vì $ SI//CD$ nên $ SI//CF$.\\
Theo định lý Ta – let ta có: $\dfrac{SI}{CF}=\dfrac{SN}{NF}=2$\\ Do đó, $SI=2CF=CD$.\\
Vậy $\dfrac{SI}{CD}=1$.}{\begin{tikzpicture}[line join=round, line cap=round,thick]
		\coordinate (A) at (0,0);
		\coordinate (B) at (-2.5,-2);
		\coordinate (D) at (7,0);
		\coordinate (C) at ($(B)+(D)-(A)$);
		\coordinate (S) at ($(A)+(0,5)$);
		\coordinate (E) at ($(A)!0.5!(B)$);
		\coordinate (F) at ($(C)!0.5!(D)$);
		\coordinate (M) at ([scale around={2/3:(S)}] E);
		\coordinate (N) at ([scale around={2/3:(S)}] F);
		\coordinate (I) at (intersection of B--M and C--N);
		 
		\draw(S)--(B) (S)--(C) (S)--(D) (B)--(C)--(D) (S)--(F) (I)--(C) (S)--(I);
		\draw[dashed,thin](S)--(A) (A)--(B) (A)--(D) (S)--(E) (E)--(F) (M)--(N) (I)--(B);
		 \foreach \i/\g in {S/90,A/-90,B/-90,C/-90,D/0,E/180,F/0,M/180,N/0,I/90}{\draw[fill=white](\i) circle (1.5pt) ($(\i)+(\g:3mm)$) node[scale=1]{$\i$};}
\end{tikzpicture}}

}
\end{ex}
%%%%
%Câu 32

\begin{ex}%[Dự án Tex Toán 11 WTB]%[Thầy Lê Vũ Hải]%[1C4B2-2]
	Cho tứ diện $ABCD$. Điểm $P$, $Q$ lần lượt là trung điểm của $AB$, $CD$. Điểm $R$ nằm trên cạnh $BC$ sao cho $BR=2RC$. Gọi $S$ là giao điểm của mặt phẳng $(PQR)$ và $AD$. Khi đó
	\choice
	{$SA=2SD$}
	{\True $SA=2SD$}
	{$SA=SD$}
	{$2SA=3SD$}
	\loigiai{
		\begin{center}
			\begin{tikzpicture}
				\pgfmathsetmacro\hsr{2/3}
				\coordinate (B) at (0,0);
				\coordinate (C) at (3,-1.5);
				\coordinate (D) at (4.5,0);
				\coordinate (A) at (1.5,3);
				\coordinate (P) at ($(A)!0.5!(B)$);
				\coordinate (Q) at ($(C)!0.5!(D)$);
				\coordinate (R) at ($(B)!\hsr!(C)$);
				\coordinate (S) at ($(A)!\hsr!(D)$);
				\coordinate (F) at ($(B)!2!(D)$);
				\draw (A)--(B)--(C)--(Q) (A)--(S) (A)--(C) (F)--(S) (F)--(D) (F)--(Q) (P)--(R) (S)--(Q);
				\draw[dashed] (P)--(S) (R)--(Q) (S)--(D)--(Q) (B)--(D);
				\foreach \x in{A,B,C,D,P,Q,R,S,F}{
					\fill (\x) circle (1pt);
				}
				\path 
				node at (A) [above]{$A$}
				node at (B) [below left]{$B$}
				node at (C) [below]{$C$}
				node at (D) [above right]{$D$}
				node at (P) [above left]{$P$}
				node at (Q) [below right]{$Q$}
				node at (R) [below left]{$R$}
				node at (S) [above right]{$S$}
				node at (F) [right]{$F$}
				;
			\end{tikzpicture}
		\end{center}
		Gọi $F=BD\cap RQ$. Nối $P$ với $F$ cắt $AD$ ở $S$.\\
		Ta có $\dfrac{DF}{FB}\cdot \dfrac{BR}{RC} \cdot \dfrac{CQ}{QD}=1 \Rightarrow \dfrac{DF}{FB}=\dfrac{RC}{BR}=\dfrac{1}{2}$.\\
		Tương tự ta có $\dfrac{DF}{FB}\cdot \dfrac{BP}{PA} \cdot \dfrac{AS}{SD}=1 \Rightarrow \dfrac{SA}{SD}=\dfrac{FB}{DF}=2 \Rightarrow SA=2SD$.
	}
\end{ex}

\begin{ex}%[Dự án Tex Toán 11 WTB]%[Thầy Lê Vũ Hải]%[1C4K2-2]
	Cho hình chóp $S.ABCD$ có đáy là hình bình hành. Gọi $N$ là trung điểm của cạnh $SC$. Lấy điểm $M$ đối xứng với $B$ qua $A$. Gọi giao điểm của đường thẳng $MN$ và mặt phẳng $(SAD)$ là $G$. Tính tỷ số $\dfrac{GM}{GN}$.
	\choice
	{$\dfrac{1}{2}$}
	{$\dfrac{1}{3}$}
	{\True $2$}
	{$3$}
	\loigiai{
		\begin{center}
			\begin{tikzpicture}
				\coordinate (A) at (0,0);
				\coordinate (B) at ($(A)+(4,0)$);
				\coordinate (C) at ($(B)+(-2,-2)$);
				\coordinate (D) at ($(A)+(-2,-2)$);
				\coordinate (S) at ($(A)+(1.5,4)$);
				\coordinate (O) at ($(A)!0.5!(C)$);
				
				\coordinate (I) at ($(A)!0.5!(B)$);
				\coordinate (N) at ($(S)!0.5!(C)$);
				\coordinate (M) at ($(A)!-1!(B)$);
				\path[name path=OM] (O)--(M);
				\path[name path=AD] (A)--(D);
				\path[name path=SD] (S)--(D);
				\path[name intersections={of=OM and AD,by=K}];
				\path[name path=SK] (S)--(K);
				\path[name path=MN] (M)--(N);
				\path[name path=AM] (A)--(M);
				
				\path[name intersections={of=SK and MN,by=G}];
				\path[name intersections={of=SD and MN,by=G'}];
				\path[name intersections={of=AM and SD,by=A'}];			
				\path[name intersections={of=OM and SD,by=K'}];
				
				\draw (S)--(B)--(C)--(D)--(S)--(C)--(D) (C)--(B) (K')--(M)--(A') (M)--(G');
				\draw[dashed] (A)--(C) (A)--(D) (S)--(A)--(B)--(D) (A')--(B) (K')--(O) (S)--(K) (G')--(N) (O)--(I); 
				\foreach \x in {A,B,C,D,S,N,M,K,O,G,I}{
					\fill (\x) circle (1pt);
				}
				\path 
				node at (A) [above right]{$A$}
				node at (B) [right]{$B$}
				node at (C) [below right]{$C$}
				node at (D) [below left]{$D$}
				node at (M) [left]{$M$}
				node at (N) [above right]{$N$}
				node at (O) [below]{$O$}
				node at (I) [above right]{$I$}
				node at (K) [below]{$K$}
				node at (S) [above]{$S$}
				node at (G) [above left]{$G$}
				; 
			\end{tikzpicture}
		\end{center}
		Gọi giao điểm của $AC$ và $BD$ là $O$ và kẻ $OM$ cắt $AD$ tại $K$.\\
		Do $O$ là trung điểm $AC$, $N$ là trung điểm $SC$ nên $ON \parallel SA$. Vậy hai mặt phẳng $(MON)$ và $(SAD)$ cắt nhau tại giao tuyến $GK$ song song với $NO$. \\
		Áp dụng định lí Ta-lét cho $GK \parallel ON$, ta có $\dfrac{GM}{GN}=\dfrac{KM}{KO}$. \quad (1)\\
		Gọi $I$ là trung điểm $AB$, vì $O$ là trung điểm $BD$ nên theo tính chất đường trung bình thì $OI \parallel AD$.\\
		Vậy theo định lý Ta-lét ta có $\dfrac{KM}{KO}=\dfrac{AM}{AI}=\dfrac{AB}{AI}=2$. \quad (2) \\
		Từ (1) và (2) ta có $\dfrac{GM}{GN}=2$.
	}
\end{ex}

\begin{ex}%[Dự án Tex Toán 11 WTB]%[Thầy Lê Vũ Hải]%[1C4B2-2]
	Cho tứ diện $ABCD$. Các điểm $P$, $Q$ lần lượt là trung điểm của $AB$ và $CD$; điểm $R$ nằm trên cạnh $BC$ sao cho $BR=2RC$. Gọi $S$ là giao điểm của mặt phẳng $(PQR)$ và cạnh $AD$. Tính tỷ số $\dfrac{SA}{SD}$.
	\choice
	{$\dfrac{7}{3}$}
	{\True $2$}
	{$\dfrac{5}{3}$}
	{$\dfrac{3}{2}$}
	\loigiai{
		\begin{center}
			\begin{tikzpicture}
				\pgfmathsetmacro\hsr{2/3}
				\coordinate (B) at (0,0);
				\coordinate (C) at (3,-1.5);
				\coordinate (D) at (4.5,0);
				\coordinate (A) at (1.5,3);
				\coordinate (P) at ($(A)!0.5!(B)$);
				\coordinate (Q) at ($(C)!0.5!(D)$);
				\coordinate (R) at ($(B)!\hsr!(C)$);
				\coordinate (S) at ($(A)!\hsr!(D)$);
				\coordinate (F) at ($(B)!2!(D)$);
				\draw (A)--(B)--(C)--(Q) (A)--(S) (A)--(C) (F)--(S) (F)--(D) (F)--(Q) (P)--(R) (S)--(Q);
				\draw[dashed] (P)--(S) (R)--(Q) (S)--(D)--(Q) (B)--(D);
				\foreach \x in{A,B,C,D,P,Q,R,S,F}{
					\fill (\x) circle (1pt);
				}
				\path 
				node at (A) [above]{$A$}
				node at (B) [below left]{$B$}
				node at (C) [below]{$C$}
				node at (D) [above right]{$D$}
				node at (P) [above left]{$P$}
				node at (Q) [below right]{$Q$}
				node at (R) [below left]{$R$}
				node at (S) [above right]{$S$}
				node at (F) [right]{$F$}
				;
			\end{tikzpicture}
		\end{center}
		Gọi $F=BD\cap RQ$. Nối $P$ với $F$ cắt $AD$ ở $S$.\\
		Ta có $\dfrac{DF}{FB}\cdot \dfrac{BR}{RC} \cdot \dfrac{CQ}{QD}=1 \Rightarrow \dfrac{DF}{FB}=\dfrac{RC}{BR}=\dfrac{1}{2}$.\\
		Tương tự ta có $\dfrac{DF}{FB}\cdot \dfrac{BP}{PA} \cdot \dfrac{AS}{SD}=1 \Rightarrow \dfrac{SA}{SD}=\dfrac{FB}{DF}=2$.
	}
\end{ex}

\begin{ex}%[Dự án Tex Toán 11 WTB]%[Thầy Lê Vũ Hải]%[1C4B2-2]
	Cho tứ diện $ABCD$. Lấy ba điểm $P$, $Q$, $R$ lần lượt trên ba cạnh $AB$, $CD$, $BC$ sao cho $PR \parallel AC$ và $CQ=2QD$. Gọi giao điểm của đường thẳng $AD$ và mặt phẳng $(PQR)$ là $S$. Khẳng định nào dưới đây đúng?
	\choice
	{$AS=3DS$}
	{\True $AD=3DS$}
	{$AD=2DS$}
	{$AS=DS$}
	\loigiai{
		\begin{center}
			\begin{tikzpicture}
				\pgfmathsetmacro\hsr{2/3}
				\coordinate (B) at (0,0);
				\coordinate (C) at (3,-1.5);
				\coordinate (D) at (4.5,0);
				\coordinate (A) at (1.5,3);
				\coordinate (P) at ($(B)!\hsr!(A)$);
				\coordinate (Q) at ($(C)!\hsr!(D)$);
				\coordinate (R) at ($(B)!\hsr!(C)$);
				\coordinate (S) at ($(A)!\hsr!(D)$);
				\coordinate (x) at ($(Q)!2!(S)$);
				\draw (A)--(B)--(C)--(Q) (A)--(D) (A)--(C) (Q)--(x) (P)--(R);
				\draw[dashed] (P)--(Q) (R)--(Q) (D)--(Q) (B)--(D);
				\foreach \x in{A,B,C,D,P,Q,R,S}{
					\fill (\x) circle (1pt);
				}
				\path 
				node at (A) [above]{$A$}
				node at (B) [below left]{$B$}
				node at (C) [below]{$C$}
				node at (D) [above right]{$D$}
				node at (P) [above left]{$P$}
				node at (Q) [below right]{$Q$}
				node at (R) [below left]{$R$}
				node at (S) [above right]{$S$}
				node at (x) [right]{$x$}
				;
			\end{tikzpicture}
		\end{center}
		Ta có $\heva{& Q \in (PQR) \cap (ACD) \\& PR \subset (PQR)\\ & AC \subset (ACD) \\& PQ \parallel AC} \Rightarrow (PQR)\cap (ACD)=Qx$ với $Qx \parallel PR \parallel AC$.\\
		Gọi $S=Qx \cap AD \Rightarrow S=(PQR) \cap AD$.\\
		Xét tam giác $ACD$ có $QS \parallel AC$.\\
		Ta có $\dfrac{SD}{AD}=\dfrac{QD}{CD}=\dfrac{1}{3} \Rightarrow AD=3SD$.
	}
\end{ex}

\begin{ex}%[Dự án Tex Toán 11 WTB]%[Thầy Lê Vũ Hải]%[1C4B2-2]
	Cho tứ diện $ABCD$. Gọi $K$, $L$ lần lượt là trung điểm $AB$ và $BC$. Lấy $N$ là điểm thuộc đoạn $CD$ sao cho $CN=2ND$. Gọi $P$ là giao điểm của $AD$ với $(KLN)$. Tính tỷ số $\dfrac{PA}{PD}$.
	\choice
	{$\dfrac{PA}{PD}=\dfrac{1}{2}$}
	{$\dfrac{PA}{PD}=\dfrac{2}{3}$}
	{$\dfrac{PA}{PD}=\dfrac{3}{2}$}
	{\True $\dfrac{PA}{PD}=2$}
	\loigiai{
		\begin{center}
			\begin{tikzpicture}
				\pgfmathsetmacro\hsr{2/3}
				\coordinate (B) at (0,0);
				\coordinate (C) at (3,-1.5);
				\coordinate (D) at (4.5,0);
				\coordinate (A) at (1.5,3);
				\coordinate (K) at ($(A)!0.5!(B)$);
				\coordinate (N) at ($(C)!\hsr!(D)$);
				\coordinate (L) at ($(B)!0.5!(C)$);
				\coordinate (P) at ($(A)!\hsr!(D)$);
				\coordinate (I) at ($(B)!2!(D)$);
				\draw (A)--(B)--(C)--(N) (A)--(P) (A)--(C) (I)--(P) (I)--(D) (I)--(N);
				\draw[dashed] (K)--(P) (L)--(N) (P)--(D)--(N) (B)--(D) (K)--(L) (P)--(N);
				\foreach \x in{A,B,C,D,K,N,L,P,I}{
					\fill (\x) circle (1pt);
				}
				\path 
				node at (A) [above]{$A$}
				node at (B) [below left]{$B$}
				node at (C) [below]{$C$}
				node at (D) [above right]{$D$}
				node at (K) [above left]{$K$}
				node at (N) [below right]{$N$}
				node at (L) [below left]{$L$}
				node at (P) [above right]{$P$}
				node at (I) [right]{$I$}
				;
			\end{tikzpicture}
		\end{center}
		Giả sử $LN \cap BD=I$. Nối $KI$ cắt $AD$ tại $P$, suy ra $AD \cap (KLN)=P$.\\
		Ta có $KL \parallel AC \Rightarrow PN \parallel AC$ $\Rightarrow \dfrac{PA}{PD}=\dfrac{NC}{ND}=2$.
	}
\end{ex}

\begin{ex}%[Dự án Tex Toán 11 WTB]%[Thầy Lê Vũ Hải]%[1C4B2-2]
	Cho tứ diện $ABCD$, $M$ là điểm thuộc $BC$ sao cho $MC=2MB$. Gọi $N,P$ lần lượt là trung điểm $BD$ và $AD$. Điểm $Q$ là giao điểm của $AC$ với $(MNP)$. Tính $\dfrac{QC}{QA}$.
	\choice
	{$\dfrac{QC}{QA}=\dfrac{3}{2}$}
	{$\dfrac{QC}{QA}=\dfrac{5}{2}$}
	{\True $\dfrac{QC}{QA}=2$}
	{$\dfrac{QC}{QA}=\dfrac{1}{2}$}
	\loigiai{
		\begin{center}
			\begin{tikzpicture}
				\pgfmathsetmacro\hsr{2/3}
				\coordinate (B) at (0,0);
				\coordinate (C) at (3,-1.5);
				\coordinate (D) at (4.5,0);
				\coordinate (A) at (1.5,3);
				\coordinate (M) at ($(C)!\hsr!(B)$);
				\coordinate (N) at ($(B)!0.5!(D)$);
				\coordinate (P) at ($(A)!0.5!(D)$);
				\coordinate (Q) at ($(C)!\hsr!(A)$);
				\draw (A)--(B)--(C)--(D)--(A)--(C) (M)--(Q)--(P);
				\draw[dashed] (B)--(D) (P)--(N)--(M);
				\foreach \x in{A,B,C,D,M,N,P,Q}{
					\fill (\x) circle (1pt);
				}
				\path 
				node at (A) [above]{$A$}
				node at (B) [below left]{$B$}
				node at (C) [below]{$C$}
				node at (D) [above right]{$D$}
				node at (M) [below]{$M$}
				node at (N) [below]{$N$}
				node at (P) [above right]{$P$}
				node at (Q) [left]{$Q$}
				;
			\end{tikzpicture}
		\end{center}
		Ta có $NP \parallel AB \Rightarrow AB \parallel (MNP)$.\\
		Mặt khác $AB \subset (ABC)$, $(ABC)$ và $(MNP)$ có điểm $M$ chung nên giao tuyến của $(ABC)$ và $(MNP)$ là đường thẳng $MQ \parallel AB$, $Q\in AC$.\\
		Vậy $\dfrac{QC}{QA}=\dfrac{MC}{MB}=2$.
	}
\end{ex}

\begin{ex}%[Dự án Tex Toán 11 WTB]%[Thầy Lê Vũ Hải]%[1C4K2-2]
	Cho hình chóp $S.ABCD$ có đáy là hình bình hành. Gọi $M$, $N$ lần lượt là trung điểm $AB$, $AD$ và $G$ là trọng tâm tam giác $SBD$. Mặt phẳng $(MNG)$ cắt $SC$ tại điểm $H$. Tính $\dfrac{SH}{SC}$.
	\choice
	{\True $\dfrac{2}{5}$}
	{$\dfrac{1}{4}$}
	{$\dfrac{1}{3}$}
	{$\dfrac{2}{3}$}
	\loigiai{
		\begin{center}
			\begin{tikzpicture}[scale=1]
				\pgfmathsetmacro\hsr{2/3}
				\coordinate (A) at (0,0);
				\coordinate (B) at ($(A)+(5,0)$);
				\coordinate (C) at ($(B)+(-2,-2)$);
				\coordinate (D) at ($(A)+(-2,-2)$);
				\coordinate (S) at ($(A)+(1,4)$);
				\coordinate (O) at ($(A)!0.5!(C)$);
				\coordinate (M) at ($(A)!0.5!(B)$);
				\coordinate (N) at ($(A)!0.5!(D)$);
				\coordinate (G) at ($(S)!\hsr!(O)$);
				\coordinate (E) at ($(M)!0.5!(N)$);
				\coordinate (e) at ($(E)!2.5!(G)$);
				
				\path[name path=EG] (E)--(e);
				\path[name path=SC] (S)--(C);				
				\path[name intersections={of=EG and SC,by=H}];
				\coordinate (I) at ($(S)!0.5!(G)$);
				\coordinate (J) at ($(S)!0.5!(H)$);
				\draw (S)--(B)--(C)--(D)--(S)--(C)--(D) (C)--(B);
				\draw[dashed] (A)--(C) (A)--(D) (S)--(A)--(B)--(D) (M)--(N) (E)--(H)
				(N)--(G)--(M) (S)--(O) (A)--(J);
				\foreach \x in {A,B,C,D,S,O,E,G,M,N,H,I,J}{
					\fill (\x) circle (1pt);
				}
				\path 
				node at (A) [left]{$A$}
				node at (B) [right]{$B$}
				node at (C) [below right]{$C$}
				node at (D) [below left]{$D$}
				node at (O) [below]{$O$}
				node at (S) [above]{$S$}
				node at (M) [above right]{$M$}
				node at (N) [above left]{$N$}
				node at (E) [below]{$E$}
				node at (G) [left]{$G$}
				node at (H) [right]{$H$}
				node at (I) [left]{$I$}
				node at (J) [above right]{$J$}
				; 
				\begin{scope}[shift={(7,-2)}]
					\coordinate (S) at (2,5); 
					\coordinate (A) at (0,0);
					\coordinate (C) at (6,0);
					\coordinate (O) at ($(A)!0.5!(C)$);
					\coordinate (E) at ($(A)!0.5!(O)$);
					\coordinate (H) at ($(S)!0.4!(C)$);
					\coordinate (G) at ($(S)!\hsr!(O)$);
					\coordinate (J) at ($(S)!0.5!(H)$);
					\coordinate (I) at ($(S)!0.5!(G)$);
					
					\draw (S)--(A)--(C)--(S)--(O) (A)--(J) (E)--(H);
					\foreach \x in {S,A,C,O,G,H,I,J}{
						\fill (\x) circle (1pt);
					}
					\path 
					node at (A) [below left]{$A$}
					node at (O) [below]{$O$}
					node at (C) [below right]{$C$}
					node at (E) [below]{$E$}
					node at (S) [above]{$S$}
					node at (J) [above right]{$J$}
					node at (H) [above right]{$H$}
					node at (I) [left]{$I$}
					; 
				\end{scope}
			\end{tikzpicture}
		\end{center}
		Trong mặt phẳng $(ABCD)$, gọi $E=MN \cap AC$.\\
		Trong mặt phẳng $(SAC)$, gọi $H=EG \cap SC$.\\
		Ta có $\heva{& H \in EG, EG \subset (MNG)\\ & H \in SC} \Rightarrow H=SC \cap (MNG)$.\\
		Gọi $I$, $J$ lần lượt là trung điểm $SG$ và $SH$.\\
		Ta có $\heva{&IJ \parallel HG \\& IA \parallel GE} \Rightarrow A,I,J$ thẳng hàng.\\
		Xét $\triangle ACJ$ có $EH \parallel AJ \Rightarrow \dfrac{CH}{HJ}=\dfrac{CE}{EA}=3 \Rightarrow CH=3HJ$.\\
		Lại có $SH=2HJ$ nên $SC=5HJ$.\\
		Vậy $\dfrac{SH}{SC}=\dfrac{2}{5}$.
	}
\end{ex}

\begin{ex}%[Dự án Tex Toán 11 WTB]%[Thầy Lê Vũ Hải]%[1C4K2-2]
Cho hình chóp $S.ABC$. Bên trong tam giác $ABC$ ta lấy một điểm $O$ bất kì. Từ $O$ ta dựng các đường thẳng lần lượt song song với $SA$, $SB$, $SC$ và cắt các mặt $(SBC)$, $(SCA)$, $(SAB)$ theo thứ tự tại $A'$, $B'$, $C'$. Khi đó tổng tỷ số $T=\dfrac{OA'}{SA}+\dfrac{OB'}{SB}+\dfrac{OC'}{SC}$ bằng bao nhiêu?
	\choice
	{$T=3$}
	{$T=\dfrac{3}{4}$}
	{\True $T=1$}
	{$T=\dfrac{1}{3}$}
	\loigiai{
		\begin{center}
			\begin{tikzpicture}
				%\pgfmathsetmacro\hsr{2/3}
				\coordinate (A) at (0,0);
				\coordinate (B) at (3.5,-2);
				\coordinate (C) at (6,0);
				\coordinate (S) at (2.6,5); 
				\coordinate (P) at ($(A)!0.45!(B)$);
				\coordinate (M) at ($(B)!0.4!(C)$);
				\path[name path=PC] (P)--(C);
				\path[name path=AM] (A)--(M);
				\path[name intersections={of=PC and AM,by=O}];
				\coordinate (K) at ($(B)!3!(O)$);
				\path[name path=BK] (B)--(K);
				\path[name path=AC] (A)--(C);
				\path[name intersections={of=AC and BK,by=N}];
				
				\coordinate (A') at ($(S)!0.6!(M)$);
				\coordinate (C') at ($(S)!0.75!(P)$);
				\coordinate (B') at ($(S)!0.65!(N)$);
				\draw (S)--(A)--(B)--(C)--(S)--(B) (S)--(P) (S)--(M);
				\draw[dashed] (B)--(N) (A)--(M) (P)--(C) (S)--(O) (A)--(C) (S)--(N) (O)--(A') (O)--(C') (O)--(B');
				\foreach \x in{S,B,C,A,O,M,N,P,A',C',B'}{
					\fill (\x) circle (1pt);
				}
				\path 
				node at (A) [left]{$A$}
				node at (B) [below]{$B$}
				node at (C) [right]{$C$}
				node at (S) [above]{$S$}
				node at (O) [below]{$O$}
				node at (M) [below right]{$M$}
				node at (N) [below]{$N$}
				node at (P) [below left]{$P$}
				node at (A') [above right]{$A'$}
				node at (B') [left]{$B'$}
				node at (C') [left]{$C'$}
				;
			\end{tikzpicture}
		\end{center}
		Gọi $M$, $N$, $P$ lần lượt là giao điểm của $AO$ và $BC$, $BO$ và $AC$, $CO$ và $AB$.\\
		Ta có $\dfrac{OA'}{SA}=\dfrac{MO}{MA}=\dfrac{S_{CMO}}{S_{CMA}}=\dfrac{S_{BMO}}{S_{BMA}} = \dfrac{S_{CMO}+S_{BMO}}{S_{CMA}+S_{BMA}}=\dfrac{S_{OBC}}{S_{ABC}}$.\\
		$\dfrac{OB'}{SB}=\dfrac{NO}{NB}=\dfrac{S_{ANO}}{S_{ANB}} = \dfrac{S_{CNO}}{S_{CNB}} = \dfrac{S_{ANO}+S_{CNO}}{S_{ANB}+S_{CNB}}=\dfrac{S_{OAC}}{S_{ABC}}$.\\
		$\dfrac{OC'}{SC} = \dfrac{PO}{PC}=\dfrac{S_{APO}}{S_{APC}} = \dfrac{S_{BPO}}{S_{BPC}}= \dfrac{S_{APO}+S_{BPO}}{S_{APC}+S_{BPC}}=\dfrac{S_{OAB}}{S_{ABC}}$.\\
		Từ đó $T=\dfrac{OA'}{SA}+\dfrac{OB'}{SB}+\dfrac{OC'}{SC} =\dfrac{S_{OBC}}{S_{ABC}} + \dfrac{S_{OAC}}{S_{ABC}} + \dfrac{S_{OAB}}{S_{ABC}} =\dfrac{S_{ABC}}{S_{ABC}}=1$.
	}
\end{ex}
\Closesolutionfile{ans}
\begin{indapan}{10} 
	{ans/ans-1-C4B11-Dang4}
\end{indapan}
\begin{dang}{Sử dụng yếu tố song song để tìm giao tuyến}
\end{dang}
\Opensolutionfile{ans}[ans/ans-1-C4B11-Dang5]
\begin{ex}%[Dự án Tex Toán 11 WTB]%[Thầy Lê Vũ Hải]%[1C4B2-3]
	Cho hình chóp $S.ABCD$ có đáy $ABCD$ là hình bình hành. Gọi $M$, $N$ lần lượt là trung điểm của $SB$, $SD$. Khi đó giao tuyến của hai mặt phẳng $(CMN)$ và $(ABCD)$ là
	\choice
	{Đường thẳng $CI$, với $I=MN \cap BD$}
	{Đường thẳng $MN$}
	{Đường thẳng $BD$}
	{\True Đường thẳng $d$ đi qua $C$ và $d \parallel BD$}
	\loigiai{
		\begin{center}
			\begin{tikzpicture}
				\coordinate (A) at (0,0);
				\coordinate (B) at ($(A)+(5,0)$);
				\coordinate (C) at ($(B)+(-2,-2)$);
				\coordinate (D) at ($(A)+(-2,-2)$);
				\coordinate (S) at ($(A)+(1.5,4)$);
				\coordinate (O) at ($(A)!0.5!(C)$);				
				\coordinate (I) at ($(A)!0.5!(B)$);
				\coordinate (N) at ($(S)!0.5!(B)$);
				\coordinate (M) at ($(S)!0.5!(D)$);
				
				%\path[name intersections={of=SK and MN,by=G}];
				\draw (S)--(B)--(C)--(D)--(S)--(C)--(D) (C)--(B) (M)--(C)--(N) ($(C)+(-3.5,-1)$)--(C)--($(C)+(3.5,1)$);
				\draw[dashed] (A)--(C) (A)--(D) (S)--(A)--(B)--(D) (M)--(N); 
				\foreach \x in {A,B,C,D,S,O,M,N}{
					\fill (\x) circle (1pt);
				}
				\path 
				node at (A) [above left]{$A$}
				node at (B) [right]{$B$}
				node at (C) [below right]{$C$}
				node at (D) [below left]{$D$}
				node at (M) [left]{$M$}
				node at (N) [right]{$N$}
				node at (O) [below]{$O$}
				node at (S) [above]{$S$}
				node at ($(C)+(3.5,1)$) [below]{$d$}
				; 
			\end{tikzpicture}
		\end{center}
		Ta có $M$, $N$ là trung điểm của $SB$, $SD$ nên $MN$ là đường trung bình của tam giác $SBD$.\\
		Suy ra $MN \parallel BD$.\\
		Ta có $\heva{& C \in (CMN) \cap (ABCD)\\ & MN \subset (CMN) \\& BD \subset (ABCD) \\& MN \parallel BD} \Rightarrow (CMN) \cap (ABCD) =d \parallel MN \parallel BD$ ($d$ đi qua $C$).
	}
\end{ex}

\begin{ex}%[Dự án Tex Toán 11 WTB]%[Thầy Lê Vũ Hải]%[1C4B2-3]
	Cho hình chóp $S.ABCD$ có đáy $ABCD$ là hình thang với $AD \parallel BC$. Gọi $M$ là trung điểm của $SC$. Gọi $d$ là giao tuyến của hai mặt phẳng $(SBC)$ và $(MAD)$. Kết luận nào sau đây là sai?
	\choice
	{$d$ cắt $SB$}
	{$d\parallel AD$}
	{\True $d$ cắt $SA$}
	{$d$ và $AC$ chéo nhau}
	\loigiai{
		\begin{center}
			\begin{tikzpicture}
				\coordinate (A) at (0,0);
				\coordinate (D) at ($(A)+(6,0)$);
				\coordinate (C) at ($(D)+(-2,-2)$);
				\coordinate (B) at ($(A)+(1,-2)$);
				\coordinate (S) at ($(A)+(1.5,4)$);		
				\coordinate (I) at ($(A)!0.5!(D)$);
				\coordinate (N) at ($(S)!0.5!(B)$);
				\coordinate (M) at ($(S)!0.5!(C)$);
				\coordinate (m) at ($(N)!4!(M)$);
				\coordinate (n) at ($(N)!-2!(M)$);
				%\path[name intersections={of=SK and MN,by=G}];
				\draw (S)--(D)--(C)--(B)--(S)--(C)--(B) (C)--(B) (M) (A)--(B) (S)--(A)  (M)--(N) (M)--(D) (m)--(n);
				\draw[dashed] (A)--(C) (A)--(D) (A)--(M); 
				\foreach \x in {A,B,C,D,S,M,N}{
					\fill (\x) circle (1pt);
				}
				\path 
				node at (A) [above left]{$A$}
				node at (B) [below left]{$B$}
				node at (C) [below right]{$C$}
				node at (D) [right]{$D$}
				node at (M) [above right]{$M$}
				node at (N) [above left]{$N$}
				node at (S) [above]{$S$}
				node at ($(M)+(4,0)$) [above]{$d$}
				; 
			\end{tikzpicture}
		\end{center}
		Ta có $\heva{& M \in (SBC)\cap (MAD) \\ &BC \parallel AD \\& d=(SBC)\cap (MAD)} \Rightarrow d$ đi qua $M$, $d \parallel BC$.\\
		Do đó $d$ cắt $SB$, $d$ và $SA$ chéo nhau
	}
\end{ex}

\begin{ex}%[Dự án Tex Toán 11 WTB]%[Thầy Lê Vũ Hải]%[1C4B2-3]
	Cho hình chóp $S.ABCD$ có đáy là hình bình hành. Gọi $M$ là trung điểm $SA$, $(\alpha)$ là mặt phẳng đi qua $M$ và song song với mặt phẳng $(ABCD)$, $d=(\alpha) \cap (SAB)$. Khi đó
	\choice
	{$d$ là đường thẳng đi qua $M$ và song song $AD$}
	{$d$ là đường thẳng đi qua $M$ và song song $BC$}
	{$d$ là đường thẳng đi qua $M$ và song song $AC$}
	{\True $d$ là đường thẳng đi qua $M$ và song song $AB$}
	\loigiai{
		\begin{center}
			\begin{tikzpicture}
				\coordinate (A) at (0,0);
				\coordinate (B) at ($(A)+(5,0)$);
				\coordinate (C) at ($(B)+(-2,-2)$);
				\coordinate (D) at ($(A)+(-2,-2)$);
				\coordinate (S) at ($(A)+(1.5,4)$);
				\coordinate (O) at ($(A)!0.5!(C)$);				
				\coordinate (I) at ($(A)!0.5!(B)$);
				\coordinate (N) at ($(S)!0.5!(B)$);
				\coordinate (M) at ($(S)!0.5!(A)$);
				\coordinate (m) at ($(M)+(-3,0)$);
				\coordinate (n) at ($(N)+(3.5,0)$);
				%\path[name intersections={of=SK and MN,by=G}];
				\path[name path=mn] (m)--(n);
				\path[name path=SD] (S)--(D);
				\path[name path=SB] (S)--(B);
				\path[name intersections={of=mn and SD,by=d}];
				\path[name intersections={of=mn and SB,by=b}];
				\draw (S)--(B)--(C)--(D)--(S)--(C)--(D) (C)--(B) (m)--(d) (b)--(n);
				\draw[dashed] (A)--(C) (A)--(D) (S)--(A)--(B)--(D) (d)--(b); 
				\foreach \x in {A,B,C,D,S,O,M,N}{
					\fill (\x) circle (1pt);
				}
				\path 
				node at (A) [above left]{$A$}
				node at (B) [right]{$B$}
				node at (C) [below right]{$C$}
				node at (D) [below left]{$D$}
				node at (M) [below right]{$M$}
				node at (N) [above right]{$N$}
				node at (O) [below]{$O$}
				node at (S) [above]{$S$}
				node at ($(N)+(3.5,0)$) [above]{$d$}
				; 
			\end{tikzpicture}
		\end{center}
		Vì $\heva{&(\alpha) \parallel (ABCD)\\ & (SAB)\cap (ABCD)=AB\\ & M\in (SAB)\cap (\alpha)\\ & (\alpha) \cap (SAB)=d} \Rightarrow d$ đi qua $M$ và song song $AB$.
	}
\end{ex}

\begin{ex}%[Dự án Tex Toán 11 WTB]%[Thầy Lê Vũ Hải]%[1C4B2-3]
	Cho hình chóp $S.ABCD$ có đáy là hình bình hành. Giao tuyến của $(SAB)$ và $(SCD)$ là 
	\choice
	{Đường thẳng qua $S$ và song song với $AD$}
	{\True Đường thẳng qua $S$ và song song với $CD$}
	{Đường $SO$ với $O$ là tâm hình bình hành}
	{Đường thẳng qua $S$ và cắt $AB$}
	\loigiai{
		\begin{center}
			\begin{tikzpicture}[scale=0.8]
				\coordinate (A) at (0,0);
				\coordinate (B) at ($(A)+(5,0)$);
				\coordinate (C) at ($(B)+(-2,-2)$);
				\coordinate (D) at ($(A)+(-2,-2)$);
				\coordinate (S) at ($(A)+(1.5,4)$);			
				\draw (S)--(B)--(C)--(D)--(S)--(C)--(D) (C)--(B) ($(S)+(-3.5,0)$)--($(S)+(3.5,0)$);
				\draw[dashed](A)--(D) (S)--(A)--(B); 
				\foreach \x in {A,B,C,D,S}{
					\fill (\x) circle (1pt);
				}
				\path 
				node at (A) [above left]{$A$}
				node at (B) [right]{$B$}
				node at (C) [below right]{$C$}
				node at (D) [below left]{$D$}
				node at (S) [above]{$S$}
				node at ($(S)+(3.5,0)$) [above]{$d$}
				; 
			\end{tikzpicture}
		\end{center}
		Ta có $\heva{& AB \subset (SAB)\\& CD\subset (SCD)\\& AB \parallel CD \\& S \in (SAB) \cap (SCD)}\Rightarrow (SAB)\cap (SCD)=d$, với $d$ qua $S$ và song song $AB$, $CD$.
	}
\end{ex}

\begin{ex}%[Dự án Tex Toán 11 WTB]%[Thầy Lê Vũ Hải]%[1C4B2-3]
	Cho hình chóp $S.ABCD$ có đáy là hình bình hành. Mệnh đề nào sau đây là sai?
	\choice
	{\True $(SAD)\cap (SBC)$ là đường thẳng qua $S$ và song song $AC$}
	{$(SAB)\cap (SAD)=SA$}
	{$AD \parallel (SBC)$}
	{$SA$ và $CD$ chéo nhau}
	\loigiai{
		\begin{center}
			\begin{tikzpicture}[scale=0.8]
				\coordinate (A) at (0,0);
				\coordinate (B) at ($(A)+(5,0)$);
				\coordinate (C) at ($(B)+(-2,-2)$);
				\coordinate (D) at ($(A)+(-2,-2)$);
				\coordinate (S) at ($(A)+(1.5,4)$);			
				\draw (S)--(B)--(C)--(D)--(S)--(C)--(D) (C)--(B) ($(S)+(-4,-4)$)--($(S)+(2,2)$);
				\draw[dashed](A)--(D) (S)--(A)--(B); 
				\foreach \x in {A,B,C,D,S}{
					\fill (\x) circle (1pt);
				}
				\path 
				node at (A) [above left]{$A$}
				node at (B) [right]{$B$}
				node at (C) [below right]{$C$}
				node at (D) [below left]{$D$}
				node at (S) [above]{$S$}
				node at ($(S)+(2,1)$) [above]{$d$}
				; 
			\end{tikzpicture}
		\end{center}
		Ta có $\heva{& AD \subset (SAD)\\& BC \subset (SBC) \\ & AD \parallel BC\\& (SAD) \cap (SBC)=d} \Rightarrow d$ qua $S$ và song song $AD$, $BC$.
	}
\end{ex}

\begin{ex}%[Dự án Tex Toán 11 WTB]%[Thầy Lê Vũ Hải]%[1C4B2-3]
	Cho hình chóp $S.ABCD$ có đáy là hình bình hành. Gọi $I$, $J$ là trung điểm $AB$ và $CB$. Khi đó giao tuyến của $2$ mặt phẳng $(SAB)$ và $(SCD)$ là đường thẳng song song với
	\choice
	{$AD$}
	{$IJ$}
	{$BJ$}
	{\True $BI$}
	\loigiai{
		\begin{center}
			\begin{tikzpicture}[scale=0.8]
				\coordinate (A) at (0,0);
				\coordinate (B) at ($(A)+(6,0)$);
				\coordinate (C) at ($(B)+(-2,-2)$);
				\coordinate (D) at ($(A)+(-2,-2)$);
				\coordinate (S) at ($(A)+(1.5,4)$);	
				\coordinate (I) at ($(A)!0.5!(B)$);
				\coordinate (J) at ($(B)!0.5!(C)$);		
				\draw (S)--(B)--(C)--(D)--(S)--(C)--(D) (C)--(B) ($(S)+(-3.5,0)$)--($(S)+(3.5,0)$);
				\draw[dashed](A)--(D) (S)--(A)--(B); 
				\foreach \x in {A,B,C,D,S,I,J}{
					\fill (\x) circle (1pt);
				}
				\path 
				node at (A) [above left]{$A$}
				node at (B) [right]{$B$}
				node at (C) [below right]{$C$}
				node at (D) [below left]{$D$}
				node at (S) [above]{$S$}
				node at ($(S)+(3.5,0)$) [above]{$d$}
				node at (I) [below left]{$I$}
				node at (J) [below right]{$J$}
				; 
			\end{tikzpicture}
		\end{center}
		Ta có $\heva{& AB \subset (SAB)\\& CD\subset (SCD)\\& AB \parallel CD \\& S \in (SAB) \cap (SCD)}\Rightarrow (SAB)\cap (SCD)=d$, với $d$ qua $S$ và song song $AB$, $CD$.\\
		Vậy giao tuyến cần tìm song song $BI$.
	}
\end{ex}

\begin{ex}%[Dự án Tex Toán 11 WTB]%[Thầy Lê Vũ Hải]%[1C4B2-3]
	Cho hình chóp $S.ABCD$ có đáy $(ABCD)$ là hình bình hành. Gọi $d$ là giao tuyến của hai mặt phẳng $(SAD)$ và $(SBC)$. Khẳng định nào sau đây là đúng?
	\choice
	{Đường thẳng $d$ đi qua $S$ và song song với $AB$}
	{Đường thẳng $d$ đi qua $S$ và song song với $DC$}
	{\True Đường thẳng $d$ đi qua $S$ và song song với $BC$}
	{Đường thẳng $d$ đi qua $S$ và song song với $BD$}
	\loigiai{
		\begin{center}
			\begin{tikzpicture}[scale=0.8]
				\coordinate (A) at (0,0);
				\coordinate (B) at ($(A)+(5,0)$);
				\coordinate (C) at ($(B)+(-2,-2)$);
				\coordinate (D) at ($(A)+(-2,-2)$);
				\coordinate (S) at ($(A)+(1.5,4)$);			
				\draw (S)--(B)--(C)--(D)--(S)--(C)--(D) (C)--(B) ($(S)+(-4,-4)$)--($(S)+(2,2)$);
				\draw[dashed](A)--(D) (S)--(A)--(B); 
				\foreach \x in {A,B,C,D,S}{
					\fill (\x) circle (1pt);
				}
				\path 
				node at (A) [above left]{$A$}
				node at (B) [right]{$B$}
				node at (C) [below right]{$C$}
				node at (D) [below left]{$D$}
				node at (S) [above]{$S$}
				node at ($(S)+(2,1)$) [above]{$d$}
				; 
			\end{tikzpicture}
		\end{center}
		Ta có $\heva{& BC \subset (SBC)\\& AD\subset (SAD)\\& BC \parallel AD \\& S \in (SBC) \cap (SAD)}\Rightarrow (SBC)\cap (SAD)=d$, với $d$ qua $S$ và song song $BC$, $AD$.\\
	}
\end{ex}

\begin{ex}%[Dự án Tex Toán 11 WTB]%[Thầy Lê Vũ Hải]%[1C4B2-3]
	Cho hình chóp $S.ABCD$ có đáy là hình thang ($AB \parallel CD$). Gọi $I$, $K$ lần lượt là trung điểm $AD$ và $BC$; $G$ là trọng tâm tam giác $SAB$. Khi đó giao tuyến của hai mặt phẳng $(IKG)$ và $(SAB)$ là
	\choice
	{Đường thẳng qua $S$ và song song $AB$, $IK$}
	{Đường thẳng qua $S$ và song song $AD$}
	{Đường thẳng qua $G$ và song song $BC$}
	{\True Đường thẳng qua $G$ và song song $AB$, $IK$}
	\loigiai{
		\begin{center}
			\begin{tikzpicture}
				\pgfmathsetmacro\k{2/3}
				\coordinate (A) at (0,0);
				\coordinate (B) at ($(A)+(6,0)$);
				\coordinate (C) at ($(B)+(-1.5,-2)$);
				\coordinate (D) at ($(A)+(1,-2)$);
				\coordinate (S) at ($(A)+(1.5,4)$);		
				\coordinate (I) at ($(A)!0.5!(D)$);
				\coordinate (K) at ($(B)!0.5!(C)$);
				\coordinate (J) at ($(A)!0.5!(B)$);
				\coordinate (G) at ($(S)!\k!(J)$);
				\coordinate (E) at ($(S)!\k!(A)$);
				\coordinate (F) at ($(S)!\k!(B)$);
				%\path[name intersections={of=SK and MN,by=G}];
				\draw (S)--(D)--(C)--(B)--(S)--(C)--(B) (C)--(B) (A)--(D) (S)--(A) (E)--(I) (F)--(K);
				\draw[dashed] (A)--(B) (I)--(K) (E)--(F) (S)--(J); 
				\foreach \x in {A,B,C,D,S,I,K,G,J,E,F}{
					\fill (\x) circle (1pt);
				}
				\path 
				node at (A) [above left]{$A$}
				node at (B) [right]{$B$}
				node at (C) [below right]{$C$}
				node at (D) [below left]{$D$}
				node at (S) [above]{$S$}
				node at (J) [below]{$J$}
				node at (G) [above left]{$G$}
				node at (E) [left]{$E$}
				node at (F) [above right]{$F$}
				node at (I) [below left]{$I$}
				node at (K) [below right]{$K$}
				; 
			\end{tikzpicture}
		\end{center}
		Ta có $\heva{& IK \subset (IKG)\\& AB\subset (SAB)\\& IK \parallel AB \\& G \in (IKG) \cap (SAB)}\Rightarrow (IKG)\cap (SAB)=d$, với $d$ qua $G$ và song song $AB$, $IG$.\\
	}
\end{ex}

\begin{ex}%[Dự án Tex Toán 11 WTB]%[Thầy Lê Vũ Hải]%[1C4B2-3]
	Cho hình chóp $S.ABCD$ có đáy là hình thang $ABCD$ $(AB \parallel CD)$. Gọi $E$, $F$ lần lượt là trung điểm của $AD$ và $BC$. Giao tuyến của hai mặt phẳng $(SAB)$ và $(SCD)$ là
	\choice
	{Đường thẳng đi qua $S$ và giao điểm của hai đường thẳng $AB$ và $SC$}
	{Đường thẳng đi qua $S$ và song song $AD$}
	{Đường thẳng đi qua $S$ và song song $AF$}
	{\True Đường thẳng đi qua $S$ và song song $EF$}
	\loigiai{
		\begin{center}
			\begin{tikzpicture}
				\pgfmathsetmacro\k{2/3}
				\coordinate (A) at (0,0);
				\coordinate (B) at ($(A)+(6,0)$);
				\coordinate (C) at ($(B)+(-1.5,-2)$);
				\coordinate (D) at ($(A)+(1,-2)$);
				\coordinate (S) at ($(A)+(1.5,4)$);		
				\coordinate (E) at ($(A)!0.5!(D)$);
				\coordinate (F) at ($(B)!0.5!(C)$);
				%\path[name intersections={of=SK and MN,by=G}];
				\draw (S)--(D)--(C)--(B)--(S)--(C)--(B) (C)--(B) (A)--(D) (S)--(A) ($(S)-(2,0)$)--($(S)+(4,0)$);
				\draw[dashed] (A)--(B) (I)--(K) (E)--(F);
				\foreach \x in {A,B,C,D,S,E,F}{
					\fill (\x) circle (1pt);
				}
				\path 
				node at (A) [above left]{$A$}
				node at (B) [right]{$B$}
				node at (C) [below right]{$C$}
				node at (D) [below left]{$D$}
				node at (S) [above]{$S$}
				node at (E) [left]{$E$}
				node at (F) [below right]{$F$}
				node at ($(S)+(3,0)$)[above]{$d$}
				; 
			\end{tikzpicture}
		\end{center}
		Ta có $\heva{& AB \subset (SAB)\\& CD\subset (SCD)\\& AB \parallel CD \\& S \in (SAB) \cap (SCD)}\Rightarrow (SAB)\cap (SCD)=d$, với $d$ qua $S$ và song song $AB$, $CD$.\\
		Lại có $AB \parallel EF$ nên giao tuyến song song $EF$.
	}
\end{ex}

\begin{ex}%[Dự án Tex Toán 11 WTB]%[Thầy Lê Vũ Hải]%[1C4B2-3]
	Cho tứ diện $S.ABCD$ có đáy $ABCD$ là hình thang ($AB \parallel CD$). Gọi $M$, $N$, $P$ lần lượt là trung điểm $BC$, $AD$ và $SA$. Giao tuyến của hai mặt phẳng $(SAB)$ và $(MNP)$ là
	\choice
	{Đường thẳng qua $M$ và song song $BC$}
	{\True Đường thẳng qua $P$ và song song $AB$}
	{Đường thẳng $PM$}
	{Đường thẳng qua $S$ và song song $AB$}
	\loigiai{
		\begin{center}
			\begin{tikzpicture}
				\pgfmathsetmacro\k{2/3}
				\coordinate (A) at (0,0);
				\coordinate (B) at ($(A)+(6,0)$);
				\coordinate (C) at ($(B)+(-1.5,-2)$);
				\coordinate (D) at ($(A)+(1,-2)$);
				\coordinate (S) at ($(A)+(1.5,4)$);		
				\coordinate (N) at ($(A)!0.5!(D)$);
				\coordinate (M) at ($(B)!0.5!(C)$);
				\coordinate (P) at ($(S)!0.5!(A)$);
				%\path[name intersections={of=SK and MN,by=G}];
				\draw (S)--(D)--(C)--(B)--(S)--(C)--(B) (C)--(B) (A)--(D) (S)--(A) (P)--(N);
				\draw[dashed] (A)--(B) (P)--(M)--(N);
				\foreach \x in {A,B,C,D,S,M,N,P}{
					\fill (\x) circle (1pt);
				}
				\path 
				node at (A) [above left]{$A$}
				node at (B) [right]{$B$}
				node at (C) [below right]{$C$}
				node at (D) [below left]{$D$}
				node at (S) [above]{$S$}
				node at (N) [left]{$N$}
				node at (M) [below right]{$M$}
				node at (P) [above left]{$P$}
				; 
			\end{tikzpicture}
		\end{center}
		Ta có $\heva{& MN \parallel AB \\& MN \subset (PMN), AB \subset (SAB) \\& P \in (PMN) \cap (SAB) \\& (PMN) \cap (SAB)=d} \Rightarrow d$ đi qua $P$ và song song $MN$, $AB$.
	}
\end{ex}

\begin{ex}%[Dự án Tex Toán 11 WTB]%[Thầy Lê Vũ Hải]%[1C4B2-3]
	Cho hình chóp $S.ABCD$ có đáy $ABCD$ là hình thang ($AB \parallel CD$). Gọi $I$, $J$ lần lượt là trung điểm của $AD$ và $BC$; $G$ là trọng tâm tam giác $SAB$. Giao tuyến của hai mặt phẳng $(SAB)$ và $(IJG)$ là 
	\choice
	{Đường thẳng qua $S$ và song song $AB$}
	{\True Đường thẳng qua $G$ và song song $DC$}
	{$SC$}
	{Đường thẳng qua $G$ và cắt $BC$}
	\loigiai{
		\begin{center}
			\begin{tikzpicture}
				\pgfmathsetmacro\k{2/3}
				\coordinate (A) at (0,0);
				\coordinate (B) at ($(A)+(6,0)$);
				\coordinate (C) at ($(B)+(-1.5,-2)$);
				\coordinate (D) at ($(A)+(1,-2)$);
				\coordinate (S) at ($(A)+(1.5,4)$);		
				\coordinate (I) at ($(A)!0.5!(D)$);
				\coordinate (J) at ($(B)!0.5!(C)$);
				\coordinate (M) at ($(A)!0.5!(B)$);
				\coordinate (G) at ($(S)!\k!(M)$);
				\coordinate (E) at ($(S)!\k!(A)$);
				\coordinate (F) at ($(S)!\k!(B)$);
				%\path[name intersections={of=SK and MN,by=G}];
				\draw (S)--(D)--(C)--(B)--(S)--(C)--(B) (C)--(B) (A)--(D) (S)--(A) (E)--(I) (F)--(J) ($(E)+(-2,0)$)--(E) (F)--($(F)+(2,0)$);
				\draw[dashed] (A)--(B) (I)--(K) (E)--(F) (S)--(M); 
				\foreach \x in {A,B,C,D,S,I,J,G,M,E,F}{
					\fill (\x) circle (1pt);
				}
				\path 
				node at (A) [above left]{$A$}
				node at (B) [right]{$B$}
				node at (C) [below right]{$C$}
				node at (D) [below left]{$D$}
				node at (S) [above]{$S$}
				node at (M) [below]{$M$}
				node at (G) [below left]{$G$}
				%node at (E) [left]{$E$}
				node at ($(F)+(2,0)$) [above right]{$x$}
				node at (I) [below left]{$I$}
				node at (J) [below right]{$J$}
				; 
			\end{tikzpicture}
		\end{center}
		Ta có $\heva{& IJ \parallel AB\\& IJ \subset (GIJ), AB \subset (SAB)  \\ &G \in (GIJ) \cap (SAB)} \Rightarrow Gx=(GIJ) \cap (SAB)$, $Gx \parallel AB \parallel CD$.
	}
\end{ex}

%%%%Từ trang 25
\begin{ex}%[1K4BA-3]
	Cho hình chóp $S . A B C D$ có đáy $A B C D$ là hình thang, $A D\parallel B C$. Giao tuyến của $(S A D)$ và $(S B C)$ là
	\choice
	{Đường thẳng đi qua $S$ và song song với $A B$}
	{Đường thẳng đi qua $S$ và song song với $CD$}
	{Đường thẳng đi qua $S$ và song song với $A C$}
	{\True Đường thẳng đi qua $S$ và song song với $A D$}
	\loigiai{
		\immini{Ta có hai mặt phẳng $(S A D)$ và $(S B C)$ có $1$ điểm chung là $S$ và lần lượt chứa hai đường thẳng $A D$ và $B C$ song song nhau nên giao tuyến $d$ của hai mặt phẳng $(S A D)$ và $(S B C)$ đi qua $S$ và song song $A D$,  $B C$.
		}{
			\begin{tikzpicture}[scale=0.7, font=\footnotesize, line join=round, line cap=round, >=stealth]
				\def\a{4}
				\path 	(0:0) coordinate (A)
				++(0:\a+1) coordinate (D)
				++(-120:\a-2) coordinate (C)
				($(A)+(C)-(D)$) coordinate (B')		
				($(A)+(80:\a)$) coordinate (S);%giao điểm O
				\coordinate (B) at ($(C)!1/3!(B')$);
				\path ($(S)+(D)-(A)$) coordinate (M);			
				\coordinate (d) at ($(M)!1.3!(S)$);
				\draw[dashed] (A)--(D);
				\draw 			(S)--(A)--(B)--(C)--(D)
				(B)--(S)	(C)--(S)	(D)--(S) (d)--(M);
				\foreach \x/\g in {A/135,B/-135,C/-45,D/45,S/90}
				\fill[black] 	(\x) circle (1pt)
				($(\g:3mm)+(\x)$) node {$\x$};	
				\foreach \x/\g in {d/90}
				\fill[black] 	(\x) 
				($(\g:3mm)+(\x)$) node {$\x$};	
		\end{tikzpicture}	}			
	}
\end{ex}

\begin{ex}%[1K4BA-3]
	Cho hình chóp $S . A B C D$, đáy $A B C D$ là hình bình hành. Giao tuyến của hai mặt phẳng $(S A D)$ và $(S B C)$ là đường thẳng song song với đường thẳng nào sau đây?	
	\choice
	{\True $AD$}
	{$AC$}
	{$DC$}
	{$BD$}
	\loigiai{
		\immini{ Ta có $A D \parallel B C \Rightarrow(S A D) \cap(S B C)=d$, với $d$ là đường thẳng đi qua $S$ và song song với $A D$.	}{
			\begin{tikzpicture}[scale=0.7, font=\footnotesize, line join=round, line cap=round, >=stealth]
				\def\a{4}
				\path 	(0:0) coordinate (A)
				++(0:\a+2) coordinate (D)
				++(-130:\a/2) coordinate (C)
				($(A)+(C)-(D)$) coordinate (B)
				($(A)+(80:\a)$) coordinate (S);%giao điểm O
				% ($(A)!0.5!(S)$) coordinate (I);%k là tỉ số AB' trên AB
				%($(C)!0.5!(S)$) coordinate (J);%k là tỉ số AB' trên AB 
				\path ($(S)+(D)-(A)$) coordinate (M);			
				\coordinate (d) at ($(M)!1.3!(S)$);			
				\draw[dashed] 	(B)--(A)--(D)--(B) (S)--(A);
				\draw			(B)--(C)--(D)--(S)
				(B)--(S)	(S)--(C) (d)--(M);
				\foreach \x/\g in {A/185,B/-135,C/-45,D/45,S/90}
				\fill[black] 	(\x) circle (1pt)
				($(\g:3mm)+(\x)$) node {$\x$};	
				\foreach \x/\g in {d/90}
				\fill[black] 	(\x) 
				($(\g:3mm)+(\x)$) node {$\x$};	
		\end{tikzpicture}}		
	}
\end{ex}

\begin{ex}%[1K4BA-3]
	Cho hình chóp $S . A B C$. Gọi $M$ và $N$ lần lượt là trung điểm của $A B$ và $A C$. Giao tuyến của hai mặt phẳng $(S M N)$ và $(S B C)$ là một đường thẳng song song với đường thẳng nào sau đây?
	\choice
	{$AC$}
	{\True  $BC$}
	{$AB$}
	{$SA$}
	\loigiai{
		\immini{Xét $\triangle A B C$ có $M$ và $N$ lần lượt là trung điểm của $A B$ và $A C$ nên $M N$ là đường trung bình suy ra $M N \parallel B C$.\\
			Ta có: $\heva{&S \in(S M N) \cap(S B C) \\& M N \subset(S M N) ; B C \subset(S B C)\\& MN\parallel BC} \Rightarrow(S M N) \cap(S B C)=S x \parallel M N \parallel B C$.
		}{
			\begin{tikzpicture}[scale=0.7, font=\footnotesize, line join=round, line cap=round, >=stealth]
				\def\a{4}
				\path 	(0:0) coordinate (A)
				++(0:\a) coordinate (C)
				++(-120:\a/2) coordinate (B)
				($(A)+(70:\a)$) coordinate (S);
				\coordinate (N) at ($(A)!0.5!(C)$);
				\coordinate (M) at ($(A)!0.5!(B)$);
				\path ($(S)+(B)-(C)$) coordinate (x);			
				\coordinate (d') at ($(x)!1.3!(S)$);		
				%			\path[name intersections={of=RQ and BD,by=F}];
				%			\path[name intersections={of=MF and NE,by=I}];	
				\draw[dashed] 	(A)--(C) (S)--(N)--(M);
				\draw 			(B)--(A)--(S)--(C)--(B)--(S)--(M) (x)--(d');
				\foreach \x/\g in {A/180,B/-90,C/0,N/60,M/-120,S/110}
				\fill[black] 	(\x) circle (1pt)
				($(\g:3mm)+(\x)$) node {$\x$};
				\foreach \x/\g in {x/90}
				\fill[black] 	(\x) 
				($(\g:3mm)+(\x)$) node {$\x$};	
				%Hình chóp S.ABC có SA vuông góc đáy	
		\end{tikzpicture}	}			
	}
\end{ex}

\begin{ex}%[1K4BA-3]
	Cho hình chóp $S \cdot A B C D$ có đáy là hình bình hành tâm $O . M$ là một điểm bất kì thuộc cạnh $S C$ , $H$ là giao điểm của $A M$ và mặt phẳng $(S B D)$. Trong các khẳng định sau khẳng định nào đúng?
	\choice
	{$H$ là giao điểm của $A M$ và $S D$}
	{$H$ là giao điểm của $A M$ và $S B$}
	{$H$ là giao điểm của $A M$ và $B D$}
	{\True $H$ là giao điểm của $A M$ và $S O$}
	\loigiai{
		\immini{ Gọi $O=A C \cap B D$. Ta có $(S A C) \cap(S B D)=S O$.\\
			Trong mặt phẳng $(S A C)$, kẻ $A M \cap S O=\{H\}$.\\
			Ta có: $\heva{&H \in A M \\ &H \in S O \subset(S B D)} \Rightarrow H=A M \cap(S B D)$.}{
			\begin{tikzpicture}[scale=0.7, font=\footnotesize, line join=round, line cap=round, >=stealth]
				\def\a{4}
				\path 	(0:0) coordinate (A)
				++(0:\a+2) coordinate (D)
				++(-130:\a/2) coordinate (C)
				($(A)+(C)-(D)$) coordinate (B)
				($(A)+(80:\a)$) coordinate (S)
				(intersection of A--C and B--D) coordinate (O);%giao điểm O
				% ($(A)!0.5!(S)$) coordinate (I);%k là tỉ số AB' trên AB
				%($(C)!0.5!(S)$) coordinate (J);%k là tỉ số AB' trên AB 
				\coordinate (M) at ($(S)!0.4!(C)$);	
				\path (intersection of A--M and S--O) coordinate (H);				
				\draw[dashed] 	(B)--(A)--(D)--(B)	(M)--(A)--(S)--(O) (A)--(C);
				\draw			(B)--(C)--(D)--(S)
				(B)--(S)	(S)--(C);
				\foreach \x/\g in {A/185,B/-135,C/-45,D/45,S/90,H/180,M/70,O/-90}
				\fill[black] 	(\x) circle (1pt)
				($(\g:3mm)+(\x)$) node {$\x$};	
		\end{tikzpicture}}		
	}
\end{ex}
\Closesolutionfile{ans}
\begin{indapan}{10} 
	{ans/ans-1-C4B11-Dang5}
\end{indapan}
\begin{dang}{Sử dụng yếu tố song song tìm thiết diện}
\end{dang}
\Opensolutionfile{ans}[ans/ans-1-C4B11-Dang6]
\begin{ex}%[1K4BA-5]
	Cho tứ diện $A B C D$. Gọi $M, N, P, Q$ lần lượt là trung điểm của các cạnh $A B, A D, C D$, $B C$. Tìm điều kiện để $M N P Q$ là hình thoi.
	\choice
	{$AB=BC$}
	{$BC=AD$}
	{\True $AC=BD$}
	{$AB=CD$}
	\loigiai{
		\immini{Xét tam giác $AB D$ có $M N$ là đường trung bình nên $M N \parallel B D$, $M N=\dfrac{1}{2} B D$.\\ Tương tự tam giác $B C D$ có $P Q$ là đường trung bình nên $P Q \parallel B D$, $P Q=\dfrac{1}{2} B D$.\\
			Tứ giác $M N P Q$ có $M N \parallel P Q, M N=P Q$ suy ra tứ giác $M N P Q$ là hình bình hành. \\
			Để $M N P Q$ là hình thoi thì $M N=M Q$ hay $B D=A C$.}{\begin{tikzpicture}[scale=0.7, font=\footnotesize, line join=round, line cap=round, >=stealth]
				\def\a{4}
				\path 	(0:0) coordinate (B)
				++(0:\a) coordinate (D)
				++(-120:\a/2) coordinate (C)
				($(B)+(70:\a)$) coordinate (A);
				\coordinate (P) at ($(D)!0.5!(A)$);
				\coordinate (N) at ($(B)!0.5!(D)$);
				\coordinate (M) at ($(C)!2/3!(B)$);
				\coordinate (Q) at ($(C)!2/3!(A)$);
				%			\path[name intersections={of=RQ and BD,by=F}];
				%				\path[name intersections={of=PF and AD,by=S}];
				%(intersection of R--Q and B--D) coordinate (F);		
				\draw[dashed] 	(B)--(D) (M)--(N)--(P);
				\draw 			(B)--(A)--(D)--(C)--(A)--(B)--(C) (M)--(Q)--(P);
				\foreach \x/\g in {A/90,B/180,C/-45,D/40,P/40,Q/160,N/-70,M/-100}
				\fill[black] 	(\x) circle (1pt)
				($(\g:3mm)+(\x)$) node {$\x$};
				%Hình chóp S.ABC có SA vuông góc đáy	
			\end{tikzpicture}	
		}		
	}
\end{ex}

\begin{ex}%[1K4BA-5]
	Cho hình chóp $S . A B C D$, đáy $A B C D$ là hình bình hành. Gọi $M$ là trung điểm của $sa$. Thiết diện của mặt phẳng $(M C D)$ với hình chóp $S . A B C D$ là hình gì?
	\choice
	{Tam giác}
	{Hình bình hành}
	{\True Hình thang}
	{Hình thoi}
	\loigiai{
		Gọi $N$ là trung điểm của $S B$. Do $M N\parallel A B, A B \parallel C D \Rightarrow M N \parallel C D$.\\
		Như vậy suy ra $N$ thuộc mặt phẳng $(M C D)$.
		Ta có: $\left\{\begin{array}{l}(M C D) \cap(S A D)=M D \\ (M C D) \cap(S A B)=M N \\ (M C D) \cap(S B C)=N C \\ (M C D) \cap(A B C D)=C D\end{array}\right.$\\
		Vậy tứ giác $M N C D$ là thiết diện của hình chóp bị cắt bởi mặt phẳng $(M C D)$.\\
		Kết hợp với $M N / / C D$, suy ra $M N C D$ là hình thang.		
	}
\end{ex}

\begin{ex}%[1K4KA-5]
	Cho hình chóp $S . A B C D$ có đáy $A B C D$ là hình thang, $A D \parallel B C, A D=2 B C . M$ là trung điểm của $S A$. Mặt phẳng $(M B C)$ cắt hình chóp theo thiết diện là	
	\choice
	{\True Hình bình hành}
	{Tam giác}
	{Hình chữ nhật}
	{Hình thang}
	\loigiai{
		\immini{ Ta có $(B M C) \cap(A B C D)=B C$, $(B M C) \cap(S A B)=B M$. \\
			$ (B M C) \cap(S A D)=Mx$,  $Mx \parallel A D \parallel B C$, $Mx \cap S D=N$, $(B M C) \cap(S C D)=N C.$\\
			Suy ra thiết diện của hình chóp cắt bởi mặt phẳng $(M B C)$ là tứ giác $B M N C$.
			Ta có $\left\{\begin{array}{l}M N=\dfrac{1}{2} A D \\ M N \parallel A D\end{array}\right.$ suy ra $\left\{\begin{array}{l}M N=B C \\ M N \parallel BC\end{array}\right.$ nên thiết diện $B M N C$ là hình bình hành.
		}{
			\begin{tikzpicture}[scale=0.7, font=\footnotesize, line join=round, line cap=round, >=stealth]
				\def\a{4}
				\path 	(0:0) coordinate (A)
				++(0:\a+1) coordinate (D)
				++(-120:\a-2) coordinate (C)
				($(A)+(C)-(D)$) coordinate (B')		
				($(A)+(80:\a)$) coordinate (S)
				($(D)+(S)-(A)$) coordinate (d);%giao điểm O
				\coordinate (B) at ($(C)!1/3!(B')$);
				\coordinate (M) at ($(A)!0.5!(S)$);
				\coordinate (N) at ($(S)!0.5!(D)$);
				\draw[dashed] (D)--(A) (M)--(N);
				\draw 			(M)--(B)--(A)--(S)--(B)--(C)--(D)--(S)
				(N)--(C)--(S);
				\foreach \x/\g in {A/135,B/-90,C/-45,D/45,S/90,N/40,M/130}
				\fill[black] 	(\x) circle (1pt)
				($(\g:3mm)+(\x)$) node {$\x$};	
		\end{tikzpicture}	}				
	}
\end{ex}

\begin{ex}%[1K4KA-5]
	Cho tứ diện $ABCD$. Trên các cạnh $AB$, $AD$ lần lượt lấy các điểm $M$, $N$ sao cho $\dfrac{A M}{A B}=\dfrac{A N}{A D}=\dfrac{1}{3}$ Gọi $P$, $Q$ lần lượt là trung điểm các cạnh $CD$, $CB$. Khẳng định nào sau đây là đúng
	\choice
	{Tứ giác $MNPQ$ là hình bình hành}
	{\True Tứ giác $MNPQ$ là một hình thang nhưng không phải hình bình hành}
	{Bốn điểm $M, N, P, Q$ đồng phẳng}
	{Tứ giác $MNPQ$ không có cặp cạnh đối nào song song}
	\loigiai{
		Ta có $\dfrac{A M}{A B}=\dfrac{A N}{A D}=\dfrac{1}{3} \Rightarrow M N \parallel B D$ và $\dfrac{M N}{B D}=\dfrac{1}{3}$.\\
		Mặt khác vì $P Q$ là đường trung bình của tam giác $B C D \Rightarrow P Q=\dfrac{1}{2} B D, P Q \parallel B D$.\hfill{(2)}\\
		Từ suy ra tứ giác $MNPQ$ là hình thang, nhưng không là hình bình hành.		
	}
\end{ex}

\begin{ex}%[1K4KA-5]
	Cho hình lập phương $A B C D \cdot A' B' C' D'$, $A C \cap B D=O$, $A' C' \cap B' D'=O'$. Gọi $M, N, P$ lần lượt là trung điểm các cạnh $A B, B C, C C'$. Khi đó thiết diện do mặt phẳng $(M N P)$ cắt hình lập phương là hình
	\choice
	{Tam giác}
	{Tứ giác}
	{Ngũ giác}
	{\True Lục giác}
	\loigiai{
		\immini{Ta có $\heva{&M N \parallel A C \\ &N P \parallel A B'} \Rightarrow(M N P) \parallel\left(A B' C\right)$.\\
			$\Rightarrow(M N P)$ cắt hình lập phương theo thiết diện là lục giác.}{
			\begin{tikzpicture}[scale=0.7, font=\footnotesize, line join=round, line cap=round, >=stealth]
				\def\a{4}
				\path 	(0:0) coordinate (A)
				++(0:\a) coordinate (D)
				++(-130:\a/2) coordinate (C)
				($(A)+(C)-(D)$) coordinate (B)
				($(A)+(90:\a)$) coordinate (A')
				($(B)+(90:\a)$) coordinate (B')
				($(C)+(90:\a)$) coordinate (C')
				($(D)+(90:\a)$) coordinate (D');
				\coordinate (M) at ($(A)!0.5!(D)$);
				\coordinate (N) at ($(D)!0.5!(C)$);	
				\coordinate (Q) at ($(B')!0.5!(C')$);
				\coordinate (P) at ($(C)!0.5!(C')$);
				\coordinate (R) at ($(A')!0.5!(B')$);	
				\coordinate (S) at ($(A)!0.5!(A')$);		
				\draw[dashed] 	(B)--(A)--(D)--(B)	(C)--(A)--(A') (M)--(N) (R)--(S)--(M);
				\draw	(C)--(C')--(A') 	(D)--(D')--(B') 	(B)--(B') 	(B)--(C)--(D) (N)--(P)--(Q)--(R) (A')--(B')--(C')--(D')--cycle;
				\foreach \x/\g in {A/180,B/180,C/0,D/0,A'/180,B'/180,C'/0,D'/0,M/-120,N/-30,Q/-120,R/120,S/-130,P/30}
				\fill[black] 	(\x) circle (1pt)
				($(\g:4mm)+(\x)$) node {$\x$};	
		\end{tikzpicture}}		
	}
\end{ex}

\begin{ex}%[1K4KA-5]
	Cho hình chóp $S . A B C D$ có đáy $A B C D$ là một hình bình hành. Gọi $M$ là trung điểm của $S D$, điểm $N$ nằm trên cạnh $S B$ sao cho $S N=2 N B$ và $O$ là giao điểm của $A C$ và $B D$. Khẳng định nào sau đây sai?
	\choice
	{\True Thiết diện của hình chóp $S . A B C D$ với mặt phẳng ( $A M N)$ là một hình thang}
	{Đường thẳng $M N$ cắt mặt phẳng $(A B C D)$}
	{Hai đường thẳng $M N$ và $S C$ chéo nhau}
	{Hai đường thẳng $M N$ và $S O$ cắt nhau}
	\loigiai{
		\immini{ $M N$ không song song với $B D$. Suy ra trong $(S B D)$ ta có $M N$ cắt $B D$. Do đó đáp án \,\lq\lq  Đường thẳng $M N$ cắt mặt phẳng $(A B C D)$\rq\rq\, đúng.\\
			Hai đường thẳng $M N$ và $S C$ chéo nhau. Hiển nhiên đúng do $S . A B C D$ là hình chóp. Do đó đáp án \,\lq\lq  Hai đường thẳng $M N$ và $S C$ chéo nhau\rq\rq\, đúng.\\
			Hai đường thẳng $M N$ và $S O$ cắt nhau vì chúng cùng nằm trong mặt phẳng $(S B D)$. Do đó đáp án \,\lq\lq  Hai đường thẳng $M N$ và $S O$ cắt nhau\rq\rq\, đúng.}{
			\begin{tikzpicture}[scale=0.7, font=\footnotesize, line join=round, line cap=round, >=stealth]
				\def\a{4}
				\path 	(0:0) coordinate (A)
				++(0:\a+2) coordinate (D)
				++(-130:\a/2) coordinate (C)
				($(A)+(C)-(D)$) coordinate (B)
				($(A)+(80:\a)$) coordinate (S)
				(intersection of A--C and B--D) coordinate (O);%giao điểm O
				% ($(A)!0.5!(S)$) coordinate (I);%k là tỉ số AB' trên AB
				%($(C)!0.5!(S)$) coordinate (J);%k là tỉ số AB' trên AB 
				\coordinate (N) at ($(S)!0.5!(D)$);
				\coordinate (M) at ($(S)!0.5!(B)$);			
				\draw[dashed] 	(B)--(A)--(D)--(B)	(M)--(A)--(N)--(M) (O)--(S)--(A)--(C);
				\draw			(B)--(C)--(D)--(S)
				(B)--(S)	(S)--(C);
				\foreach \x/\g in {A/185,B/-135,C/-45,D/45,S/90,N/70,M/110,O/-90}
				\fill[black] 	(\x) circle (1pt)
				($(\g:3mm)+(\x)$) node {$\x$};	
		\end{tikzpicture}}			
	}
\end{ex}

\begin{ex}%[1K4KA-5]
	Cho tứ diện $A B C D$. Gọi $M$ là trung điểm của $A B$. Cắt tứ diện $A B C D$ bới mặt phẳng đi qua $M$ và song song với $B C$ và $A D$, thiết diện thu được là hình gì?
	\choice
	{Tam giác đều}
	{Tam giác vuông}
	{\True Hình bình hành}
	{Ngũ giác}
	\loigiai{
		\immini{Gọi $\alpha$ là mặt phẳng đi qua $M$ và song song với $B C$ và $A D$.\\
			Xét $(\alpha)$ và $(A B D)$ có $\left\{\begin{array}{l}M \in(\alpha) \cap(A B D) \\ (\alpha) \parallel A D\end{array}\right.$ nên $(\alpha) \cap(A B D)=M Q$ với $Q$ là trung điểm $B D$.\\
			Xét $(\alpha)$ và $(M N P Q)$ có $\left\{\begin{array}{l}Q \in(\alpha) \cap(B C D) \\ (\alpha) \parallel B C\end{array}\right.$ nên $(\alpha) \cap(B C D)=Q P$ với $P$ là trung điểm $C D$.\\
			Xét $(\alpha)$ và $(A C D)$ có $\left\{\begin{array}{l}P \in(\alpha) \cap(A C D) \\ (\alpha) \parallel A D\end{array}\right.$ nên $(\alpha) \cap(A C D)=N P$ với $N$ là trung điểm $A C$.\\
			Mà $M N$, $P Q$ là hai đường trung bình của tam giác $A B C$ và $D B C$.
			Nên ta có $\left\{\begin{array}{l}M N \| P Q \\ M N=P Q.\end{array}\right.$\\M
			Vậy thiết diện là hình bình hành $M N P Q$.}{
			\begin{tikzpicture}[scale=0.7, font=\footnotesize, line join=round, line cap=round, >=stealth]
				\def\a{4}
				\path 	(0:0) coordinate (B)
				++(0:\a) coordinate (D)
				++(-120:\a/2) coordinate (C)
				($(B)+(70:\a)$) coordinate (A);
				\coordinate (P) at ($(D)!0.5!(C)$);
				\coordinate (N) at ($(A)!0.5!(C)$);
				\coordinate (M) at ($(A)!0.5!(B)$);
				\coordinate (Q) at ($(B)!0.5!(D)$);
				%			\path[name intersections={of=RQ and BD,by=F}];
				%				\path[name intersections={of=PF and AD,by=S}];
				%(intersection of R--Q and B--D) coordinate (F);		
				\draw[dashed] 	(B)--(D) (M)--(Q)--(P);
				\draw 			(B)--(A)--(D)--(C)--(A)--(B)--(C) (M)--(N)--(P);
				\foreach \x/\g in {A/90,B/180,C/-45,D/40,P/-40,Q/-100,N/60,M/120}
				\fill[black] 	(\x) circle (1pt)
				($(\g:3mm)+(\x)$) node {$\x$};
				%Hình chóp S.ABC có SA vuông góc đáy	
		\end{tikzpicture}	}		
	}
\end{ex}

\begin{ex}%[1K4KA-5]
	Cho hình chóp $S . A B C D$, có đáy $A B C D$ là hình bình hành. Gọi $M$ là trung điểm của $S D, N$ là điểm trên cạnh $S B$ sao cho $S N=2 S B, O$ là giao điểm của $A C$ và $B D$. Khẳng định nào sau đây \textbf{sai} ?
	\choice
	{Đường thẳng $M N$ cắt mặt phẳng $(A B C D)$}
	{\True Thiết diện của hình chóp $S . A B C D$ với mặt phẳng $(A M N)$ là một hình thang}
	{Hai đường thẳng $M N$ và $S O$ cắt nhau}
	{Hai đường thẳng $M N$ và $S C$ chéo nhau}
	\loigiai{
		\immini{ $M N \cap B D=I \Rightarrow M N \cap(A B C D)=I$. nên \,\lq\lq  Đường thẳng $M N$ cắt mặt phẳng $(A B C D)$\rq\rq\, đúng.\\
			Hai đường thẳng $M N$ và $S O$ cắt nhau do cùng nằm trong mặt phẳng ( $S B D)$ và không song song nên \,\lq\lq  Hai đường thẳng $M N$ và $S O$ cắt nhau\rq\rq\, đúng.\\
			Hai đường thẳng $M N$ và $S C$ chéo nhau vì không cùng nằm trong một mặt phẳng nên \,\lq\lq  Hai đường thẳng $M N$ và $S C$ chéo nhau\rq\rq\, đúng.}{
			\begin{tikzpicture}[scale=0.7, font=\footnotesize, line join=round, line cap=round, >=stealth]
				\def\a{4}
				\path 	(0:0) coordinate (A)
				++(0:\a+2) coordinate (D)
				++(-130:\a/2) coordinate (C)
				($(A)+(C)-(D)$) coordinate (B)
				($(A)+(80:\a)$) coordinate (S)
				(intersection of A--C and B--D) coordinate (O);%giao điểm O
				% ($(A)!0.5!(S)$) coordinate (I);%k là tỉ số AB' trên AB
				%($(C)!0.5!(S)$) coordinate (J);%k là tỉ số AB' trên AB 
				\coordinate (M) at ($(S)!0.5!(D)$);
				\coordinate (N) at ($(S)!2/3!(B)$);		
				\path (intersection of M--N and B--D) coordinate (I);			
				\draw[dashed] 	(B)--(A)--(D)--(B)	(M)--(A)--(N)--(M) (O)--(S)--(A)--(C);
				\draw			(N)--(I)--(B)--(C)--(D)--(S)
				(B)--(S)	(S)--(C);
				\foreach \x/\g in {A/185,B/-45,C/-45,D/45,S/90,N/70,M/110,O/-90,I/120}
				\fill[black] 	(\x) circle (1pt)
				($(\g:3mm)+(\x)$) node {$\x$};	
		\end{tikzpicture}}		
	}
\end{ex}

\begin{ex}%[1K4KA-5]
	Cho hình chóp tứ giác $S . A B C D$, có đáy $A B C D$ là hình bình hành. Gọi $M, N, P$ lần lượt là trung điểm của các cạnh $S A, S B$ và $B C$. Thiết diện tạo bởi mặt phẳng $(M N P)$ và hình chóp $S . A B C D$ là	
	\choice
	{Tứ giác $M N P K$ với $K$ là điểm tuỳ ý trên cạnh $A D$}
	{Tam giác $M N P$}
	{Hình bình hành $M N P K$ với $K$ là điểm trên cạnh $A D$ mà $P K \parallel A B$}
	{\True Hình thang $M N P K$ với $K$ là điểm trên cạnh $A D$ mà $P K / / A B$.}
	\loigiai{
		\immini{ Vì $M N \parallel A B \Rightarrow A B \parallel(M N P)$ mà $A B \subset(A B C D)$ nên $m p(M N P)$ cắt $m p(A B C D)$ theo giao tuyến là đường thẳng qua $P$ và song song với $A B$.\\
			Trong $m p(A B C D)$, qua $P$ kẻ đường thẳng song song với $A B$ cắt $A D$ tại $K \Rightarrow M N \parallel P K$.\\
			Vậy thiết diện tạo bởi mặt phẳng $(M N P)$ và hình chóp $S . A B C D$ là hình thang $M N P K$ với $K$ là điểm trên cạnh $A D$ mà $P K \parallel A B$.}{
			\begin{tikzpicture}[scale=0.7, font=\footnotesize, line join=round, line cap=round, >=stealth]
				\def\a{4}
				\path 	(0:0) coordinate (A)
				++(0:\a+2) coordinate (D)
				++(-130:\a/2) coordinate (C)
				($(A)+(C)-(D)$) coordinate (B)
				($(A)+(80:\a)$) coordinate (S)
				(intersection of A--C and B--D) coordinate (O);%giao điểm O
				% ($(A)!0.5!(S)$) coordinate (I);%k là tỉ số AB' trên AB
				%($(C)!0.5!(S)$) coordinate (J);%k là tỉ số AB' trên AB 
				\coordinate (M) at ($(S)!0.5!(A)$);
				\coordinate (N) at ($(S)!0.5!(B)$);		
				\coordinate (P) at ($(B)!0.5!(C)$);
				\coordinate (K) at ($(A)!0.5!(D)$);		
				\path (intersection of M--N and B--D) coordinate (I);			
				\draw[dashed] 	(B)--(A)--(D)--(B)	(N)--(M)--(K)--(P) (S)--(A);
				\draw			(N)--(P) (B)--(C)--(D)--(S)
				(B)--(S)	(S)--(C);
				\foreach \x/\g in {A/185,B/-45,C/-45,D/45,S/90,N/120,M/30,P/-90,K/-30}
				\fill[black] 	(\x) circle (1pt)
				($(\g:3mm)+(\x)$) node {$\x$};	
		\end{tikzpicture}}				
	}
\end{ex}

\begin{ex}%[1K4KA-5]
	Cho hình chóp $S . A B C D$ có đáy $A B C D$ là hình bình hành tâm $O$. Gọi $M$ là trung điểm của $O B,(\alpha)$ là mặt phẳng đi qua $M$, song song với $A C$ và song song với $S B$. Thiết diện của hình chóp $S . A B C D$ khi cắt bởi mặt phẳng $(\alpha)$ là hình gì?
	\choice
	{Lục giác}
	{\True Ngũ giác}
	{Tam giác}
	{Tứ giác}
	\loigiai{
		\immini{Ta có:
			$$
			\left\{\begin{array}{l}
				M \in(\alpha) \cap(A B C D) \\
				(A B C D) \supset A C / /(\alpha)
			\end{array} \Rightarrow(\alpha) \cap(A B C D)=d_1 \text { đi qua } M \text { và song song với } A C\right. \text {. }
			$$
			Trong $(A B C D)$, gọi $I, H$ lần lượt là giao điểm của $d_1$ với $A B$ và $B C$. Khi đó, $I$ và $H$ lần lượt là trung điểm của $A B$ và $B C$.\\
			Ta lại có:
			$$
			\left\{\begin{array}{l}
				I \in(\alpha) \cap(S A B) \\
				(S A B) \supset S B / /(\alpha)
			\end{array} \Rightarrow(\alpha) \cap(A B)=d_2 \text { đi qua } I \text { và song song với } S B\right. \text {. }
			$$
			Trong $(S A B)$, gọi $J$ là giao điểm của $d_2$ với $S A$. Khi đó, $J$ là trung điểm của $S A$.
		}{
			\begin{tikzpicture}[scale=0.7, font=\footnotesize, line join=round, line cap=round, >=stealth]
				\def\a{4}
				\path 	(0:0) coordinate (A)
				++(0:\a+2) coordinate (D)
				++(-130:\a/2) coordinate (C)
				($(A)+(C)-(D)$) coordinate (B)
				($(A)+(80:\a)$) coordinate (S)
				(intersection of A--C and B--D) coordinate (O);%giao điểm O
				% ($(A)!0.5!(S)$) coordinate (I);%k là tỉ số AB' trên AB
				%($(C)!0.5!(S)$) coordinate (J);%k là tỉ số AB' trên AB 
				\coordinate (M) at ($(O)!0.5!(B)$);
				\coordinate (I) at ($(A)!0.5!(B)$);		
				\coordinate (H) at ($(B)!0.5!(C)$);
				\coordinate (J) at ($(A)!0.5!(S)$);		
				\coordinate (L) at ($(S)!0.5!(C)$);
				\coordinate (K) at ($(D)!3/4!(S)$);					
				\draw[dashed] 	(B)--(A)--(D)--(B)--(D)	(H)--(I)--(J)--(K)--(M) (S)--(A)--(C);
				\draw			(H)--(L)--(K) (B)--(C)--(D)--(S)
				(B)--(S)	(S)--(C);
				\foreach \x/\g in {A/-95,B/-45,C/-45,D/45,S/90,H/-90,M/-90,I/-90,K/30,J/180,L/0,O/-90}
				\fill[black] 	(\x) circle (1pt)
				($(\g:3mm)+(\x)$) node {$\x$};	
		\end{tikzpicture}}	
		\noindent Ta cũng có $\left\{\begin{array}{l}H \in(\alpha) \cap(S B C) \\ (S B C) \supset S B \parallel(\alpha)\end{array} \Rightarrow(\alpha) \cap(S B C)=d_3\right.$ đi qua $H$ và song song với $S B$.\\
		Trong $(S B C)$, gọi $L$ là giao điểm của $d_3$ với $S C$. Khi đó, $L$ là trung điểm của $S C$.\\
		Mặt khác
		$\left\{\begin{array}{l}M \in(\alpha) \cap(S B D) \\ (S B D) \supset S B \parallel(\alpha)\end{array} \Rightarrow(\alpha) \cap(S B D)=d_4\right.$ đi qua $M$ và song song với $S B$.\\
		Trong $(S B C)$, gọi $K$ là giao điểm của $d_4$ với $S D$.\\
		Vậy thiết diện của hình chóp $S . A B C D$ khi cắt bởi mặt phẳng $(\alpha)$ là ngũ giác $H I J K L$.			
	}
\end{ex}

\begin{ex}%[1K4KA-5]
	Cho tứ diện $A B C D$. Gọi $M, N$ lần lượt là trung điêm của $A B, A C . E$ là điểm trên cạnh $C D$ với $E D=3 E C$. Thiết diện tạo bởi mặt phẳng $(M N E)$ và tứ diện $A B C D$ là
	\choice
	{Tam giác $M N E$}
	{Tứ giác $M N E F$ với $E$ là điểm bất kì trên cạnh $B D$}
	{Hình bình hành $M N E F$ với $E$ là điểm trên cạnh $B D$ mà $E F \parallel B C$}
	{\True Hình thang $M N E F$ với $E$ là điểm trên cạnh $B D$ mà $E F \parallel B C$}
	\loigiai{
		\immini{Do $M, N$ lần lượt là trung điêm của $A B, A C \Rightarrow M N \parallel B C$.\\
			Ta có
			$$
			\heva{&
				E \in(M N E) \cap(B C D) \\
				&M N \subset(M N E), B C \subset(B C D)\\&M N \parallel B C} \Rightarrow(M N E) \cap(B C D)=E F \parallel M N \parallel B C \quad(F \in B D).
			$$
			Ta có: $(M N E) \cap(A B C)=M N$, $(M N E) \cap(A C D)=N E$, $(M N E) \cap(B C D)=E F$, $(M N E) \cap(A B D)=F M$.\\
			Vậy thiết diện là hình thang $M N E F$.\\
			Xét $\triangle C A D$ có $\dfrac{C N}{C A}=\dfrac{1}{2} \neq \dfrac{C E}{C D}=\dfrac{1}{4} \Rightarrow E N \cap A D=I$.
		}{
			\begin{tikzpicture}[scale=0.7, font=\footnotesize, line join=round, line cap=round, >=stealth]
				\def\a{4}
				\path 	(0:0) coordinate (B)
				++(0:\a) coordinate (D)
				++(-120:\a/2) coordinate (C)
				($(B)+(70:\a)$) coordinate (A);
				\coordinate (E) at ($(D)!3/4!(C)$);
				\coordinate (N) at ($(A)!0.5!(C)$);
				\coordinate (M) at ($(A)!0.5!(B)$);
				\coordinate (F) at ($(D)!3/4!(B)$);
				%			\path[name intersections={of=RQ and BD,by=F}];
				%			\path[name intersections={of=MF and NE,by=I}];
				\path(intersection of M--F and N--E) coordinate (I);		
				\draw[dashed] 	(B)--(D) (E)--(F)--(M);
				\draw 			(B)--(A)--(D)--(C)--(A)--(B)--(C) (E)--(I)--(M)--(N);
				\foreach \x/\g in {A/90,B/180,C/-45,D/40,E/-40,F/40,N/60,M/120,I/180}
				\fill[black] 	(\x) circle (1pt)
				($(\g:3mm)+(\x)$) node {$\x$};
				%Hình chóp S.ABC có SA vuông góc đáy	
		\end{tikzpicture}	}	
		\noindent Ta có
		$\left.\begin{array}{l}(M N E) \cap(A B D)=F M \\ (A B D) \cap(A C D)=A D \\ (M N E) \cap(A C D)=E N \\ E N \cap A D=I\end{array}\right\} \Rightarrow M N, A D, F M$ dồng qui tại $I$.
		Do đó $M N E F$ không thể là hình bình hành.	
	}
\end{ex}

\begin{ex}%[1K4KA-5]
	Cho hình chóp $S . A B C D$ với các cạnh đáy là $A B, C D$. Gọi $I, J$ lần lượt là trung điểm của các cạnh $A D, B C$ và $G$ là trọng tâm tam giác $S A B$. Tìm $k$ với $A B=k C D$ để thiết diện của mặt phẳng $(G I J)$ với hình chóp $S . A B C D$ là hình bình hành.
	\choice
	{$K=4$}
	{$K=2$}
	{$K=1$}
	{\True $k=3$}
	\loigiai{
		\immini{Dễ thấy giao tuyến của hai mặt phẳng $(G I J)$ và $(S A B)$ là đường thẳng $G x$ đi qua $G$ và song song với các đường thẳng $A B, I J$. Giao tuyến $G x$ cắt $S A$ tại $M$ và cắt $S B$ tại $N$.\\
			Thiết diện của mặt phẳng $(G I J)$ với hình chóp $S . A B C D$ là hình thang $I J N M$ vì $I J \parallel M N$.\\
			$I J$ là đường trung bình của hình thang $A B C D$ nên ta có
			$$
			I J=\dfrac{A B+C D}{2}=\dfrac{k C D+C D}{2}=\dfrac{k+1}{2} C D .
			$$
			$G$ là trọng tâm tam giác $S A B$ nên $M N=\dfrac{2}{3} A B=\dfrac{2}{3} k C D$.\\
			Để $I J N M$ là hình bình hành ta cần phải có $I J=M N$
			$$
			\Leftrightarrow \dfrac{k+1}{2} C D=\dfrac{2}{3} k C D \Leftrightarrow \dfrac{k+1}{2}=\dfrac{2 k}{3} \Leftrightarrow k=3.
			$$
		}{
			\begin{tikzpicture}[scale=0.7, font=\footnotesize, line join=round, line cap=round, >=stealth]
				\def\a{4}
				\path 	(0:0) coordinate (A)
				++(0:\a+1) coordinate (B)
				++(-120:\a-2) coordinate (C)
				($(A)+(C)-(B)$) coordinate (B')		
				($(A)+(80:\a)$) coordinate (S)
				($(B)+(S)-(A)$) coordinate (d);%giao điểm O
				\coordinate (D) at ($(C)!1/3!(B')$);
				\coordinate (I) at ($(A)!0.5!(D)$);
				\coordinate (J) at ($(C)!0.5!(B)$);
				\coordinate (M) at ($(S)!2/3!(A)$);
				\coordinate (N) at ($(S)!2/3!(B)$);
				\coordinate (G) at ($(N)!0.5!(M)$);
				\coordinate (K) at ($(A)!0.5!(B)$);
				\draw[dashed] (A)--(B) (N)--(M) (S)--(K) (I)--(J);
				\draw 			(A)--(S)--(B)--(C)--(D)--(S)
				(A)--(D)	(C)--(S) (N)--(J) (M)--(I);
				\foreach \x/\g in {A/135,B/0,C/-45,D/45,S/90,N/40,M/120,J/-40,I/-120,G/-120,K/-90}
				\fill[black] 	(\x) circle (1pt)
				($(\g:3mm)+(\x)$) node {$\x$};	
		\end{tikzpicture}	}				
	}
\end{ex}

\begin{ex}%[1K4KA-5]
	Cho tứ diện $A B C D$. Gọi $M$ và $N$ lần lượt là trung điểm của $A B$ và $A C . E$ là điển trên cạnh $C D$ với $E D=3 E C$. Thiết diện tạo bởi mặt phẳng $(M N E)$ và tứ diện $A B C D$ là
	\choice
	{Tam giác $M N E$}
	{Tứ giác $M N E F$ với $F$ là điểm bất kì trên cạnh $B D$}
	{Hình bình hành $M N E F$ với $F$ là điểm bất kì trên cạnh $B D$ mà $E F$ song song với $B C$}
	{\True Hình thang $M N E F$ với $F$ là điểm trên cạnh $B D$ mà $E F$ song song với $B C$}
	\loigiai{
		
	}
\end{ex}

\begin{ex}%[1K4KA-5]
	Cho hình chóp $S . A B C D$ có đáy là hình bình hành. Gọi $M, N, I$ lần lượt là trung điểm của $S A, S B, B C$ điểm $G$ nằm giữa $S$ và $I$ sao cho $\dfrac{S G}{S I}=\dfrac{3}{4}$. Thiết diện của hình chóp $S . A B C D$ với mặt phẳng $(M N G)$ là	
	\choice
	{\True hình thang}
	{hình tam giác}
	{hình bình hành}
	{hình ngũ giác}
	\loigiai{
		\immini{ Xét trong mặt phẳng $(S B C)$ ta có $N G \cap B C=\{P\}$.\\
			Vì $M N \parallel A B$ nên $(M N G) \cap(A B C D)$ theo giao tuyến đi qua $P$ song song với $A B, C D$ và cắt $A D$ tại $Q$.\\
			Do đó: $\left\{\begin{array}{l}(M N G) \cap(S A B)=M N \\ (M N G) \cap(S B C)=N P \\ (M N G) \cap(A B C D)=P Q \\ (M N G) \cap(S A D)=Q M\end{array}\right.$\\
			Suy ra: Thiết diện của hình chóp $S . A B C D$ với mặt phẳng $(M N G)$ là tứ giác $M N P Q$.}{
			\begin{tikzpicture}[scale=0.7, font=\footnotesize, line join=round, line cap=round, >=stealth]
				\def\a{4}
				\path 	(0:0) coordinate (A)
				++(0:\a+2) coordinate (D)
				++(-130:\a/2) coordinate (C)
				($(A)+(C)-(D)$) coordinate (B)
				($(A)+(80:\a)$) coordinate (S)
				(intersection of A--C and B--D) coordinate (O);%giao điểm O
				% ($(A)!0.5!(S)$) coordinate (I);%k là tỉ số AB' trên AB
				%($(C)!0.5!(S)$) coordinate (J);%k là tỉ số AB' trên AB 
				\coordinate (M) at ($(S)!0.5!(A)$);
				\coordinate (N) at ($(S)!0.5!(B)$);		
				\coordinate (I) at ($(B)!0.5!(C)$);
				\coordinate (G) at ($(S)!3/4!(I)$);		
				\path (intersection of G--N and B--C) coordinate (P);
				\path	($(A)+(P)-(B)$) coordinate (Q);				
				\draw[dashed] 	(B)--(A)--(D)--(B)	(N)--(M)--(Q)--(P) (S)--(A);
				\draw			(N)--(P) (B)--(C)--(D)--(S)--(I)
				(B)--(S)	(S)--(C);
				\foreach \x/\g in {A/185,B/-45,C/-45,D/45,S/90,N/120,M/30,P/-90,I/-90,G/50,Q/30}
				\fill[black] 	(\x) circle (1pt)
				($(\g:3mm)+(\x)$) node {$\x$};	
		\end{tikzpicture}}				
	}
\end{ex}
\Closesolutionfile{ans}
\begin{indapan}{10} 
	{ans/ans-1-C4B11-Dang6}
\end{indapan}