%\section{Hai đường thẳng song song}
\begin{dang}{Bài toán quỹ tích và điểm cố định}
	
\end{dang}

\subsubsection{Ví dụ mẫu}
\begin{vd}%[1K4BA-7]
	Trong không gian $Oxyz$ cho hình chóp $S.ABCD$ đáy $ABCD$ là hình chữ nhật. Tìm tập hợp tất cả các giao điểm của hai mặt phẳng $(SAB)$ và $(SCD)$?
	\loigiai{
	\begin{center}
		\begin{tikzpicture}[line join=round, line cap=round,thick]
			\coordinate (A) at (0,0);
			\coordinate (B) at (2,-2);
			\coordinate (D) at (5,0);
			\coordinate (C) at ($(B)+(D)-(A)$);
			\coordinate (O) at ($(A)!0.5!(C)$);
			\coordinate (S) at ($(O)+(0,5)$);
			\draw(S)--(A) (S)--(B) (S)--(C) (A)--(B) (B)--(C);
			\draw[dashed,thin](A)--(D) (C)--(D) (S)--(D);
			\foreach \i/\g in {S/90,A/180,B/-90,C/-90,D/0}{\draw[fill=black](\i) circle (1pt) ($(\i)+(\g:3mm)$) node[scale=1]{$\i$};}
		\end{tikzpicture}
	\end{center}
	Ta có $S$ là điểm chung thứ nhất của $(SAB)$ và $(SCD)$.\\
	Vì $AB\parallel CD$ nên tập hợp tất cả các giao điểm của hai mặt phẳng $(SAB)$ và $(SCD)$ là đường thẳng song song với $AB,CD$ và đi qua điểm $ S $.}
\end{vd}
\begin{vd}%[1K4BA-7]
	Trong không gian $Oxyz$ cho hình chóp $S.ABCD$ đáy $ABCD$ là hình thoi. Gọi $M,N$ lần lượt là trung điểm của $SB,SD$. Tìm tập hợp tất cả các điểm là giao điểm của hai mặt phẳng $(AMN)$ và $(ABD)$?\\
	\loigiai{
	\begin{center}
		\begin{tikzpicture}[line join=round, line cap=round,thick]
			\coordinate (A) at (0,0);
			\coordinate (B) at (2,-2);
			\coordinate (D) at (4,0);
			\coordinate (C) at ($(B)+(D)-(A)$);
			\coordinate (O) at ($(A)!0.5!(C)$);
			\coordinate (S) at ($(O)+(0,5)$);
			\coordinate (M) at ($(S)!0.5!(B)$);
			\coordinate (N) at ($(S)!0.5!(D)$);
			\draw(S)--(A) (S)--(B) (S)--(C) (A)--(B) (B)--(C) (A)--(M);
			\draw[dashed,thin](A)--(D) (C)--(D) (S)--(D) (A)--(N) (M)--(N) (B)--(D);
			\foreach \i/\g in {S/90,A/180,B/-90,C/-90,D/0,M/0,N/180}{\draw[fill=black](\i) circle (1pt) ($(\i)+(\g:3mm)$) node[scale=1]{$\i$};}
		\end{tikzpicture}
	\end{center}
	Ta có $A$ là điểm chung thứ nhất của $(AMN)$ và $(ABD)$.\\
	Vì $M,N$ lần lượt là trung điểm của $SB,SD$ nên $MN$ là đường trung bình của tam giác $SBD$. Suy ra $MN\parallel BD$.\\
	Gọi $ d$ là đường thằng đi qua $A$ và song song $MN,BD$. Đường thẳng $ d $ chính là tập hợp tất cả các giao điểm của hai mặt phẳng $(AMN)$ và $(ABD)$.}
\end{vd}
\begin{vd}%[1K4KA-7]
	Trong mặt phẳng tọa độ $Oxyz$ cho hình chóp $S.ABC$. Gọi $K\in SA$ sao cho $SK=\dfrac {1}{2}KA$, $ H\in SB$ sao cho $ SH=\dfrac{1}{2}HB $. Tìm tập hợp tất cả các điểm chung giữa $ (CHK) $ và $ (ABC) $? \\
	\loigiai{
	\begin{center}
		\begin{tikzpicture}[line join=round, line cap=round,thick]
			\coordinate (S) at (0,4);
			\coordinate (A) at (-2,0);
			\coordinate (B) at (2,-2);
			\coordinate (C) at (3,0);
			\coordinate (K) at ($(S)!1/3!(A)$);
			\coordinate (H) at ($(S)!1/3!(B)$);
			\draw(S)--(A) (S)--(B) (S)--(C) (B)--(C) (A)--(B) (K)--(H)--(C);
			\draw[dashed,thin](A)--(C) (K)--(C);
			\foreach \i/\g in {S/90,A/180,B/-90,C/0,K/180,H/0}{\draw[fill=white](\i) circle (1.5pt) ($(\i)+(\g:3mm)$) node[scale=1]{$\i$};}
		\end{tikzpicture}
	\end{center}
	Theo định lý Ta-let ta được: $ HK \parallel AB $. \\
	Ta có $C$ là điểm chung thứ nhất của $(CHK)$ và $(ABC)$.\\
	Vì $AB\parallel KH$ nên tập hợp tất cả các giao điểm của hai mặt phẳng $(CHK)$ và $(ABC)$ là đường thẳng song song với $AB,KH$ và đi qua điểm $ C $.}
\end{vd}
\subsubsection{Bài tập rèn luyện}
\centerline{\fcolorbox{red}{yellow!50}{\bf {CÂU HỎI TRẮC NGHIỆM}}}
\Opensolutionfile{ans}[ans/ans-1K4A-7]
\begin{ex}%[1K4YA-7] 
	Cho hình chóp $S.ABCD$ có đáy $ABCD$ là hình bình hành tâm $O$. Tập hợp các giao điểm của hai mặt phẳng $(SAB)$ và $(SCD)$ là đường thẳng nào dưới đây?
	\choice
	{\True Đường thẳng đi qua $S$ và song song $AB$}
	{ Đường thẳng đi qua $S$ và song song với $AD$}
	{ Đường thẳng $SO$}
	{ Đường thẳng $AC$}
	\loigiai{
		
	}
\end{ex}
\begin{ex}%[1K4KA-7] 
	Cho tứ diện $ABCD$. Gọi $M,N$ lần lượt là trung điểm của $AB$ và $AC$; gọi $E$ là điểm thuộc $ CD $ sao cho $ ED=3EC $. Quỹ tích các điểm chung của mặt phẳng $ (MNE) $ và tứ diện $(ABCD)$ là
	\choice
	{ Tam giác $ MNE $}		
	{ Tứ giác $ MNEF $ là trung điểm của $ BD $}
	{ Hình bình hành $ MNEF $ với $ F $ là điểm trên cạnh $ BD $ mà $ EF \parallel BC $}
	{\True Hình thang $ MNEF $ với $ F $ là điểm trên cạnh $ BD $ mà $ EF \parallel BC $}
	\loigiai{
		+ Tam giác $ ABC $ có $ M,N $ là trung điểm của $ AB, AC $. \\
		Suy ra $ MN $ là đường trung bình của tam giác ABC nên $ MN \parallel BC $ .\\
		+ Ta tìm giao tuyến của $ (MNE) $ và $ (BCD) $. \\
		$\heva{& MN//BC \\ & E~\text{chung}}\Rightarrow Ex=(MNE) \cap (BCD)$. \\
		Gọi giao điểm của tia $Ex$ và $ BD $ là $ F $. \\
		Do đó: $ MN \parallel EF $ suy ra bốn điểm $ M,N,E,F $ đồng phẳng và $ MNEF $ là hình thang. \\
		Vậy quỹ tích các điểm tạo bởi $ (MNE) $ và tứ diện $ ABCD $ là hình thang $ MNEF $ cần tìm}
\end{ex}
\begin{ex}%[1K4KA-7] 
	Cho hình chóp $S.ABCD$, với đáy $ABCD$ là tứ giác lồi. Tập hợp tất cả các điểm $ M $ tạo thành thiết diện của mặt phẳng $( \alpha )$ tuỳ ý với hình chóp \textbf{không} thể là
	\choice
	{ Tam giác}
	{\True Lục giác}
	{ Ngũ giác}
	{ Tứ giác}
	\loigiai{
		
	}
\end{ex}
\begin{ex}%[1K4KA-7] 
	Cho hình chóp $S.ABCD$ đáy $ABCD$ là hình chữ nhật. Gọi $M,N,P$ lần lượt là trung điểm của $ AB, BC $ và $ SB $. Tập hợp tất cả các giao điểm của hai mặt phẳng $ (ACP) $ và $ (SMN) $ là
	\choice
	{\True Một đường thẳng song song với $ AC $}
	{ Một đường thẳng song song $ MC $}
	{ Một đường thằng song song $ AN $}
	{ Một đường thẳng song song $ BD $}
	\loigiai{
	+ Gọi $G_1,G_2$ lần lượt là giao điểm của $SM\cap AP;SN\cap CP$. \\
	+ Xét tam giác $ APC $, theo định lý Ta-lét ta có $G_1G_2 \parallel AC$. \\
	}
\end{ex}
\begin{ex}%[1K4BA-7] 
	Cho tứ diện $ABCD$ có $I,J$ lần lượt là trung điểm của $BC,BD$. Tập hợp tất cả các giao điểm của mặt phẳng $(AIJ)$ và $(ACD)$ là
	\choice
	{ đường thẳng $d$ đi qua $A$ và song song với $BC$}
	{ đường thẳng $d$ đi qua $A$ và song song với $BD$}
	{\True đường thẳng $ d$ đi qua $A$ và song song với $CD$}
	{ đường thẳng $AB$}
	\loigiai{
	Do $IJ$ là đường trung bình của tam giác $BCD$ nên $IJ \parallel CD$\\
	$\Rightarrow ( AIJ )\cap ( ACD )=d$ thì $ d \parallel CD \parallel IJ$\\
	Mà $A\in ( AIJ )\cap ( ACD )$ nên $ d$ đi qua $A$ và song song với $CD$}
\end{ex}
\begin{ex}%[1K4GA-7] 
	Cho hình chóp $S.ABCD$ có đáy $ABCD$ là hình bình hành. Lấy điểm $M$ trên $SA$ ($M$ khác $S$ và $A$). Mặt $( MBC )$ cắt hình chóp $S.ABCD$ theo thiết diện là hình gì?
	\choice
	{ Tam giác}
	{\True Hình thang}
	{ Hình bình hành}
	{ Hình chữ nhật}
	\loigiai{
	Do $BC \parallel AD \Rightarrow (MBC) \cap (SAD)=MN$. \\
	$\Rightarrow MN \parallel AD$ (với $N \in SD$). \\
	Do $MN<AD$ suy ra $MN<BC$ và $MN \parallel BC$. \\
	Vậy: Thiết diện của khối chóp khi bởi $(MBC)$ là hình thang $MNCB$.}
\end{ex}
\begin{ex}%[1K4GA-7] 
	Cho tứ diện $ABCD$. Điểm $M$ thuộc đoạn $AC$ ($M$ khác $A$, $M$ khác $C$). Mặt phẳng $(\alpha)$ đi qua $M$ và song song với $AB$ và $AD$. Tập hợp điểm chung giữa $(\alpha)$ và hình chóp $ABCD$ là hình gì?
	\choice
	{\True Hình tam giác}
	{ Hình bình hành}
	{ Hình vuông}
	{ Hình chữ nhật}
	\loigiai{
	Trong $( ACD )$ kẻ $MN \parallel AD, N\in CD$.\\
	Trong $( ABC )$ kẻ $MP \parallel AB, P\in BC$.\\
	Từ đó suy ra $(\alpha)=(MNP)$. Mà thiết diện của $(MNP)$ và tứ diện $(ABCD)$ là hình tam giác.}
\end{ex}
\begin{ex}%[1K4BA-7] 
	Cho hình chóp $S.ABCD$ có đáy $ABCD$ là hình thang, $AD$ song song $BC$ và $AD=2BC$. Gọi $M$ là trung điểm $SA$. Tập hợp tất cả các giao điểm của $(MBC)$ và $(SAD)$ là
	\choice
	{$MN$ với $N$ là điểm thuộc đoạn $SD$ sao cho $SN=2ND$}
	{\True $MN$ với $N$ là trung điểm $SD$}
	{$MN$ với $N$ là trọng tâm tam giác $SCD$}
	{$MN$ với $N$ là điểm thuộc đoạn $SD$ sao cho $SN=\dfrac{1}{3}ND$}
	\loigiai{
	Do $AD \parallel BC$ $ \Rightarrow (MBC) \cap (SAD)=MN$ nên $MN \parallel BC \parallel AD (N \in SD)$. \\
	$M$ là trung điểm $SA \Rightarrow MN$ là đường trung bình của tam giác $SAD \Rightarrow N$ là trung điểm $SD$.}
\end{ex}
\begin{ex}%[1K4KA-7] 
	Cho tứ diện $ABCD$ có $AB=CD$. Mặt phẳng $( \alpha )$ qua trung điểm của cạnh $AC$, song song với $AB$ và $CD$. Tập hợp các điểm chung giữa tứ diện $ABCD$ và $(\alpha)$ là hình gì?
	\choice
	{ Hình tam giác}
	{ Hình vuông}
	{\True Hình thoi}
	{ Hình chữ nhật}
	\loigiai{
	Gọi $E$ là trung điểm $AC$. Kẻ $EH \parallel CD$ và $EM \parallel AB$ với $H \in AD$, $M \in BC$. \\
	Kẻ $MG \parallel CD (G\in CD) \Rightarrow $ thiết diện của tứ diện cắt bởi $(\alpha)$ là tứ giác $EMGH$ có các cặp cạnh đối song song $\Rightarrow $ là hình bình hành\\
	Lại có $AB=CD \Rightarrow EM=EH$.\\
	$\Rightarrow EMGH$ là hình thoi.}
\end{ex}
\begin{ex}%[1K4KA-7] 
	Cho tứ diện $ABCD$, gọi $M,N,P$ lần lượt là trung điểm của $AB,BC,CD$. Tập hợp các điểm chung giữa mặt phẳng $(MNP)$ và tứ diện $ ABCD $ là hình gì?
	\choice
	{ Hình tam giác}
	{\True Hình bình hành}
	{ Hình ngũ giác}
	{ Hình tam giác cân}
	\loigiai{
	$MN \parallel AC$, $NP \parallel BD$. \\
	Kẻ $PQ$ song song với $MN (Q \in AD)$. \\
	$MN \parallel PQ$ và $MN=\dfrac{1}{2}PQ \Rightarrow MNPQ$ là hình bình hành.\\
	}
\end{ex}

\Closesolutionfile{ans}
\begin{indapan}{10}
	{ans/ans-1K4A-7}
\end{indapan}

\begin{dang}{Bài toán thực tế}
	
\end{dang}

\subsubsection{Ví dụ mẫu}
\begin{vd}%[1K4YA-8]
	Cho khối rubik có dạng được minh họa như hình bên dưới. Hãy kể tên các cạnh song song với cạnh $CD$.
	\begin{center}
	\begin{tikzpicture}[line join=round, line cap=round,>=stealth,thick]
		\coordinate (A) at (1,0.5);
		\coordinate (B) at (0,0);
		\coordinate (C) at (3,0);
		\coordinate (D) at (4,0.5);
		\coordinate (A') at (1,3.5);
		\coordinate (B') at (0,3);
		\coordinate (C') at (3,3);
		\coordinate (D') at (4,3.5);
		\draw[] (B)--(B')--(C')--(C)--cycle (B')--(A')--(D')--(D)--(C) (C')--(D');
		\draw[dashed] (B)--(A) (A')--(A) (A)--(D);
		\foreach \i/\g in {A/180,B/-90,C/-90,D/0,A'/90,B'/90,C'/90,D'/90}{\draw[fill=black](\i) circle (1pt) ($(\i)+(\g:3mm)$) node[scale=1]{$\i$};}
	\end{tikzpicture}
	\end{center}
	\loigiai{
	Các cạnh song song với cạnh $CD$ là $C'D'$, $AB$, $A'B'$.}
\end{vd}
\begin{vd}%[1K4BA-8]
	Một chiếc lều được minh họa như hình bên dưới.
	\begin{center}
		\begin{tikzpicture}[line join=round, line cap=round,thick]
			\def\canh{3}
			\coordinate (B) at (0,0);
			\coordinate (C) at (\canh,0);
			\coordinate (A) at ($(B) + (60:\canh)$);
			\coordinate (B') at (4,2);
			\coordinate (C') at (\canh+4,2);
			\coordinate (A') at ($(B') + (60:\canh)$);
			\draw(A)--(B)--(C)--cycle (A')--(C') (A)--(A') (C)--(C');
			\draw[dashed](A')--(B')--(C') (B)--(B');
			\draw pic["$S$",draw,angle eccentricity=0.5,angle radius=0.8cm]{angle=C--B--A};
			\draw pic["$Q$",draw,angle eccentricity=0.5,angle radius=0.7cm]{angle=C'--C--A};
			\draw pic["$R$",draw,angle eccentricity=0.7,angle radius=1.2cm]{angle=B'--C'--C};
			\draw pic["$P$",draw,angle eccentricity=0.5,angle radius=0.5cm]{angle=A'--B'--B};
		\end{tikzpicture}
	\end{center}
	\begin{enumEX}[a)]{1}
		\item Tìm ba mặt phẳng cắt nhau từng đôi một theo ba giao tuyến song song.
		\item Tìm ba mặt phẳng cắt nhau từng đôi một theo ba giao tuyến đồng quy.
	\end{enumEX}
	\loigiai{
	\begin{enumEX}[a)]{1}
		\item Ba mặt phẳng cắt nhau từng đôi một theo ba giao tuyến song song: $(P)$, $(Q)$ và $(R)$.
		\item Ba mặt phẳng cắt nhau từng đôi một theo ba giao tuyến đồng quy: $(S)$, $(Q)$ và $(R)$.
	\end{enumEX}
	}
\end{vd}
\begin{vd}%[1K4YA-8]
	Khối rubik tam giác được minh họa như hình bên dưới. Hãy kể tên các cặp cạnh chéo nhau?
	\begin{center}
		\begin{tikzpicture}[line join=round, line cap=round,thick]
			\coordinate (S) at (0,4);
			\coordinate (A) at (-2,0);
			\coordinate (B) at (0,-3);
			\coordinate (C) at (3,0);
			\draw(S)--(A) (S)--(B) (S)--(C) (B)--(C) (A)--(B);
			\draw[dashed,thin](A)--(C);
			\foreach \i/\g in {S/90,A/180,B/-90,C/0}{\draw[fill=black](\i) circle (1.5pt) ($(\i)+(\g:3mm)$) node[scale=1]{$\i$};}
		\end{tikzpicture}
	\end{center}
	\loigiai{
	Các cặp cạnh chéo nhau: $AD$ và $BC$, $AB$ và $CD$, $AC$ và $BD$.}
\end{vd}
\begin{vd}%[1K4YA-8]
	Quan sát một phần căn phòng (Hình bên dưới). Hãy cho biết vị trí tương đối của các cặp đường thẳng $ a$ và $ b$; $ a$ và $ c$; $ b$ và $ c$?
	\begin{center}
		\begin{tikzpicture}[line join=round, line cap=round,>=stealth,thick]
			\coordinate (A) at (0,0);
			\coordinate (B) at (0,3);
			\coordinate (C) at (4,3);
			\coordinate (D) at (4,0);
			\coordinate (A') at (-1,-1.5);
			\coordinate (B') at (-1,4);
			\coordinate (C') at (5,4);
			\coordinate (D') at (5,-1.5);
			\coordinate (M) at (-1,-1);
			\coordinate (N) at (-1,3.5);
			\coordinate (K) at (5,3.5);
			\coordinate (H) at (5,-1);
			\draw[] (A)--(B)--(C)--(D)--cycle;
			\draw[] (A')--(B')--(C')--(D')--cycle;
			\draw[] (A)--(M) (B)--(N) (C)--(K) (D)--(H);
			\fill[pattern=checkerboard](A)--(D)--(H)--(D')--(A')--(M)--cycle;
			\draw[color=red, line width=1.5pt] (B)node[above right]{$ b $}--(C) ;
			\draw[color=green!50!black, line width=2pt] (A)node[above right]{$ a $}--(D) ;
			\draw[color=blue, line width=1.5pt] (C)node[above]{$ c $}--(K) ;
		\end{tikzpicture}
	\end{center}
	\loigiai{
	$ a$ và $ b$ song song với nhau. \\
	$ a$ và $ c$ chéo nhau. \\
	$ b$ và $ c$ cắt nhau.}
\end{vd}
\begin{vd}%[1K4YA-8]
	Kim tự tháp Ai Cập được minh họa là hình chóp $S.ABCD$ có đáy $ABCD$ là hình vuông.	Xét vị trí tương đối của các cặp đường thẳng $AB$ và $CD$, $SA$ và $SC$, $SA$ và $BC$?
	\begin{center}
		\begin{tikzpicture}[line join=round, line cap=round,thick]
			\coordinate (A) at (0,0);
			\coordinate (B) at (2,-2);
			\coordinate (D) at (5,0);
			\coordinate (C) at ($(B)+(D)-(A)$);
			\coordinate (O) at ($(A)!0.5!(C)$);
			\coordinate (S) at ($(O)+(0,5)$);
			\draw(S)--(A) (S)--(B) (S)--(C) (A)--(B) (B)--(C);
			\draw[dashed,thin](A)--(D) (C)--(D) (S)--(D);
			\foreach \i/\g in {S/90,A/180,B/-90,C/-90,D/0}{\draw[fill=black](\i) circle (1pt) ($(\i)+(\g:3mm)$) node[scale=1]{$\i$};}
		\end{tikzpicture}
	\end{center}
	\loigiai{
	$AB$ và $CD$ song song với nhau.\\
	$SA$ và $SC$ cắt nhau tại $S$.\\
	$SA$ và $BC$ chéo nhau.}
\end{vd}
\subsubsection{Bài tập rèn luyện}
\centerline{\fcolorbox{red}{yellow!50}{\bf {CÂU HỎI TRẮC NGHIỆM}}}
\Opensolutionfile{ans}[ans/ans-1K4A-8]
\begin{ex}%[1K4YA-8]
	Cho khối rubik có dạng được minh họa như hình bên dưới. Khẳng định nào dưới đây là đúng?
	\begin{center}
		\begin{tikzpicture}[line join=round, line cap=round,>=stealth,thick]
			\coordinate (A) at (1,0.5);
			\coordinate (B) at (0,0);
			\coordinate (C) at (3,0);
			\coordinate (D) at (4,0.5);
			\coordinate (A') at (1,3.5);
			\coordinate (B') at (0,3);
			\coordinate (C') at (3,3);
			\coordinate (D') at (4,3.5);
			\draw[] (B)--(B')--(C')--(C)--cycle (B')--(A')--(D')--(D)--(C) (C')--(D');
			\draw[dashed] (B)--(A) (A')--(A) (A)--(D);
			\foreach \i/\g in {A/180,B/-90,C/-90,D/0,A'/90,B'/90,C'/90,D'/90}{\draw[fill=black](\i) circle (1pt) ($(\i)+(\g:3mm)$) node[scale=1]{$\i$};}
		\end{tikzpicture}
	\end{center}
	\choice
	{\True $ AB \parallel C'D' $}
	{$ AB \parallel A'D' $}
	{$ AC \parallel C'D' $}
	{$ AC \parallel A'B' $}
	\loigiai{
		
	}
\end{ex}
\begin{ex}%[1K4YA-8]
	Cho khối rubik có dạng được minh họa như hình bên dưới. Khẳng định nào dưới đây là đúng?
	\begin{center}
		\begin{tikzpicture}[line join=round, line cap=round,>=stealth,thick]
			\coordinate (A) at (1,0.5);
			\coordinate (B) at (0,0);
			\coordinate (C) at (3,0);
			\coordinate (D) at (4,0.5);
			\coordinate (A') at (1,3.5);
			\coordinate (B') at (0,3);
			\coordinate (C') at (3,3);
			\coordinate (D') at (4,3.5);
			\draw[] (B)--(B')--(C')--(C)--cycle (B')--(A')--(D')--(D)--(C) (C')--(D');
			\draw[dashed] (B)--(A) (A')--(A) (A)--(D);
			\foreach \i/\g in {A/180,B/-90,C/-90,D/0,A'/90,B'/90,C'/90,D'/90}{\draw[fill=black](\i) circle (1pt) ($(\i)+(\g:3mm)$) node[scale=1]{$\i$};}
		\end{tikzpicture}
	\end{center}
	\choice
	{$ AB $ và $ A'B' $ chéo nhau}
	{\True $ AB $ và $ A'C' $ chéo nhau}
	{$ AC $ và $ A'C' $ chéo nhau}
	{$ BD $ và $ B'D' $ chéo nhau}
	\loigiai{
		
	}
\end{ex}
\begin{ex}%[1K4BA-8]
	Một chiếc lều được minh họa như hình bên dưới. Ba mặt phẳng nào nào dưới đây cắt nhau từng đôi một theo ba giao tuyến song song?
	\begin{center}
		\begin{tikzpicture}[line join=round, line cap=round,thick]
			\def\canh{3}
			\coordinate (B) at (0,0);
			\coordinate (C) at (\canh,0);
			\coordinate (A) at ($(B) + (60:\canh)$);
			\coordinate (B') at (4,2);
			\coordinate (C') at (\canh+4,2);
			\coordinate (A') at ($(B') + (60:\canh)$);
			\draw(A)--(B)--(C)--cycle (A')--(C') (A)--(A') (C)--(C');
			\draw[dashed](A')--(B')--(C') (B)--(B');
			\draw pic["$S$",draw,angle eccentricity=0.5,angle radius=0.8cm]{angle=C--B--A};
			\draw pic["$Q$",draw,angle eccentricity=0.5,angle radius=0.7cm]{angle=C'--C--A};
			\draw pic["$R$",draw,angle eccentricity=0.7,angle radius=1.2cm]{angle=B'--C'--C};
			\draw pic["$P$",draw,angle eccentricity=0.5,angle radius=0.5cm]{angle=A'--B'--B};
		\end{tikzpicture}
	\end{center}
	\choice
	{$ (S),(Q),(P) $}
	{$ (S),(Q),(R) $}
	{$ (P),(S),(R) $}
	{\True $ (P),(Q),(R) $}
	\loigiai{
		
	}
\end{ex}
\begin{ex}%[1K4BA-8]
	Một chiếc lều được minh họa như hình bên dưới. Khẳng định nào dưới đây là \textbf{sai}?
	\begin{center}
		\begin{tikzpicture}[line join=round, line cap=round,thick]
			\def\canh{3}
			\coordinate (B) at (0,0);
			\coordinate (C) at (\canh,0);
			\coordinate (A) at ($(B) + (60:\canh)$);
			\coordinate (B') at (4,2);
			\coordinate (C') at (\canh+4,2);
			\coordinate (A') at ($(B') + (60:\canh)$);
			\draw(A)--(B)--(C)--cycle (A')--(C') (A)--(A') (C)--(C');
			\draw[dashed](A')--(B')--(C') (B)--(B');
			\draw pic["$S$",draw,angle eccentricity=0.5,angle radius=0.8cm]{angle=C--B--A};
			\draw pic["$Q$",draw,angle eccentricity=0.5,angle radius=0.7cm]{angle=C'--C--A};
			\draw pic["$R$",draw,angle eccentricity=0.7,angle radius=1.2cm]{angle=B'--C'--C};
			\draw pic["$P$",draw,angle eccentricity=0.5,angle radius=0.5cm]{angle=A'--B'--B};
		\end{tikzpicture}
	\end{center}
	\choice
	{$ (S),(Q),(P) $ cắt nhau từng đôi một theo ba giao tuyến đồng quy}
	{\True $ (S),(Q),(R) $ cắt nhau từng đôi một theo ba giao tuyến song song}
	{$ (P),(S),(R) $ cắt nhau từng đôi một theo ba giao tuyến đồng quy}
	{$ (P),(Q),(R) $ cắt nhau từng đôi một theo ba giao tuyến đồng quy}
	\loigiai{
		
	}
\end{ex}
\begin{ex}%[1K4YA-8]
	Quan sát một phần căn phòng (Hình bên dưới). Hãy cho biết vị trí tương đối của cặp đường thẳng $ a$ và $ c$?
	\begin{center}
		\begin{tikzpicture}[line join=round, line cap=round,>=stealth,thick]
			\coordinate (A) at (0,0);
			\coordinate (B) at (0,3);
			\coordinate (C) at (4,3);
			\coordinate (D) at (4,0);
			\coordinate (A') at (-1,-1.5);
			\coordinate (B') at (-1,4);
			\coordinate (C') at (5,4);
			\coordinate (D') at (5,-1.5);
			\coordinate (M) at (-1,-1);
			\coordinate (N) at (-1,3.5);
			\coordinate (K) at (5,3.5);
			\coordinate (H) at (5,-1);
			\draw[] (A)--(B)--(C)--(D)--cycle;
			\draw[] (A')--(B')--(C')--(D')--cycle;
			\draw[] (A)--(M) (B)--(N) (C)--(K) (D)--(H);
			\fill[pattern=checkerboard](A)--(D)--(H)--(D')--(A')--(M)--cycle;
			\draw[color=red, line width=1.5pt] (B)node[above right]{$ b $}--(C) ;
			\draw[color=green!50!black, line width=2pt] (A)node[above right]{$ a $}--(D) ;
			\draw[color=blue, line width=1.5pt] (C)node[above]{$ c $}--(K) ;
		\end{tikzpicture}
	\end{center}
	\choice
	{cắt nhau}
	{trùng nhau}
	{\True chéo nhau}
	{song song}
	\loigiai{}
\end{ex}

\Closesolutionfile{ans}
\begin{indapan}{10}
	{ans/ans-1K4A-8}
\end{indapan}

