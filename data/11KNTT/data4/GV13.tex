%\chapter{Hàm số  lượng giác và phương trình lượng giác}
\section{Hai mặt phẳng song song}
%============
\subsection{Tóm tắt lý thuyết}
\begin{tomtat}
%\begin{khung}
\subsubsection{Định nghĩa}
Hai mặt phẳng được gọi là song song nếu chúng không có điểm chung.\\
Hai mặt phẳng $(\alpha)$ và $(\beta)$ song song được ký hiệu là $(\alpha) \parallel (\beta)$.\\
\begin{center}
\begin{tikzpicture}[line join = round, line cap = round,>=stealth,font=\footnotesize,scale=0.7]
\path (0,-1)coordinate(A)+(0:6)coordinate(B)+(-150:3)coordinate(D)($(B)+(D)-(A)$)coordinate(C) (90:1.5)coordinate(M)+(0:6)coordinate(N)+(-150:3)coordinate(Q)($(N)+(Q)-(M)$)coordinate(P);
\draw (A)--(B)--(C)--(D)--cycle (M)--(N)--(P)--(Q)--cycle; 
\draw pic[draw,blue,fill=green!50,opacity=.5,angle radius=10mm,angle eccentricity=0.8,"$\alpha$"] {angle = C--D--A};
\draw  pic[draw,blue,fill=green!50,opacity=.5,angle radius=10mm,angle eccentricity=0.8,"$\beta$"] {angle = P--Q--M}; 
\end{tikzpicture}
\end{center}
\begin{tc}
	\immini{Cho hai mặt phẳng song song, nếu một đường thẳng nằm trong mặt phẳng này thì đường thẳng đó sẽ song song với mặt phẳng kia.
	\begin{center}
	\fbox{$\heva{&
		(\alpha)\parallel(\beta) \\&
		d \subset(\alpha)
		} \Rightarrow d \parallel(\beta)
		$}
	\end{center}}{\begin{tikzpicture}[>=stealth,x=1cm,y=1cm,scale=.85]
		\coordinate (A) at (-4,-2);
		\coordinate (B) at (0,-2);
		\coordinate (C) at (1,0);
		\coordinate (D) at (-3,0);
		\draw(A)--(B)--(C)--(D)--cycle;
	%	\fill (-1.5,2) circle (1.0pt) node[above]{$I$};
		\draw (-3,2) node[above]{$d$};
		%\draw (-2.5,2.5) node[above]{$b$};
		\coordinate(A) at (-4,-2);
		\coordinate (B) at (0,-2);
		\coordinate (C) at (1,0);
		\coordinate (D) at (-3,0);
		\draw(A)--(B)--(C)--(D)--cycle;
		\coordinate (A) at (-4,-2);
		\coordinate (B) at (0,-2);
		\coordinate (C) at (1,0);
		\coordinate (D) at (-3,0);
		\draw(A)--(B)--(C)--(D)--cycle;
		\coordinate(A1) at (-4,1);
		\coordinate (B1) at (0,1);
		\coordinate (C1) at (1,3);
		\coordinate (D1) at (-3,3);
		\draw (-3,2)--(0,2) ;
		\draw pic[draw,blue,fill=green!50,opacity=.5,angle radius=5mm,angle eccentricity=0.65,"$\beta$"] {angle = B--A--D};
		%\draw pic[draw,blue,angle radius=4mm] {angle = B--A--D} ($($(A)!5mm!(B)$)!.5!($(A)!4mm!(C)$)$)node[above right]{$(\beta)$};
		\draw pic[draw,blue,fill=green!50,opacity=.5,angle radius=5mm,angle eccentricity=0.65,"$\alpha$"] {angle = B1--A1--D1};
		%\draw pic[draw,blue,angle radius=4mm] {angle = B1--A1--D1} ($($(A1)!4mm!(B1)$)!.5!($(A1)!4mm!(C1)$)$)node[above right]{$(\alpha)$};
		\draw(A1)--(B1)--(C1)--(D1)--cycle;
		\end{tikzpicture}}
\end{tc}
\subsubsection{Các định lý}
\begin{dl}
\immini
	{Nếu mặt phẳng $(\alpha)$ chứa hai đường thẳng cắt nhau $a$, $b$ và $a$, $b$ cùng song song với mặt phẳng $(\beta)$ thì $(\alpha)$ song song với $(\beta)$.
		\begin{center}
		\fbox{$\heva{ &a \parallel(\beta) \\ &b \parallel(\beta)\\ &a,\,  b \subset(\alpha) \\& a \cap b=I } \Rightarrow(\alpha) \parallel(\beta)$.}
		\end{center}
	}
	{
		\begin{tikzpicture}[>=stealth,x=1cm,y=1cm,scale=.8]
		\coordinate (A) at (-4,-2);
		\coordinate (B) at (0,-2);
		\coordinate (C) at (1,0);
		\coordinate (D) at (-3,0);
		\draw(A)--(B)--(C)--(D)--cycle;
		\fill (-1.5,2) circle (1.0pt) node[above]{$I$};
		\draw (-3,2) node[above]{$a$};
		\draw (-2.5,2.5) node[above]{$b$};
		\coordinate(A1) at (-4,1);
		\coordinate (B1) at (0,1);
		\coordinate (C1) at (1,3);
		\coordinate (D1) at (-3,3);
		\draw (-3,2)--(0,2) (-2.5,2.5)--(-.5,1.5);
		\draw pic[draw,blue,fill=green!50,opacity=.5,angle radius=5mm,angle eccentricity=0.65,"$\beta$"] {angle = B--A--D};
		\draw pic[draw,blue,fill=green!50,opacity=.5,angle radius=5mm,angle eccentricity=0.65,"$\alpha$"] {angle = B1--A1--D1};
		\draw(A1)--(B1)--(C1)--(D1)--cycle;
		\end{tikzpicture}
	}
\end{dl}
\begin{dl}
Qua một điểm nằm ngoài một mặt phẳng cho trước có một và chỉ một mặt phẳng song song với mặt phẳng đã cho.

\end{dl}
\begin{dl}
\immini{Cho hai mặt phẳng song song, nếu một mặt phẳng cắt mặt phẳng này thì cũng cắt mặt phẳng kia và hai giao tuyến song song với nhau.
\begin{center}
		\fbox{$\heva{& (\alpha) \parallel (\beta) \\& (\gamma) \cap (\alpha) = a } \Rightarrow \heva{& (\gamma) \cap (\beta)= b \\&a \parallel b.}$}
		\end{center}
}{\begin{tikzpicture}[line join=round,line cap=round,line width=.6pt,font=\footnotesize,scale=.7]
	\coordinate (A) at (0,0);
	\coordinate (D) at ($(A)+(0:5)$);
	\coordinate (B) at ($(A) + (-120:2)$);
	\coordinate (C) at ($(D)-(A)+(B)$);
	
	\coordinate (A') at ($(A)+(90:2.5)$);
	\coordinate (B') at ($(B)+(90:2.5)$);
	\coordinate (C') at ($(C)+(90:2.5)$);
	\coordinate (D') at ($(D)+(90:2.5)$);
	
	\coordinate (M) at (2,-1);
	\coordinate (N) at (1,-3);
	\coordinate (Q) at ($(M)+(90:5)$);
	\coordinate (P) at ($(N)+(90:5)$);
	
	\coordinate (M1) at (intersection of M--Q and A--D) {};
	\coordinate (M2) at (intersection of M--Q and C'--B') {};
	\coordinate (M3) at (intersection of M--Q and A'--D') {};
	\coordinate (M4) at (intersection of N--P and C'--B') {};
	\coordinate (M5) at (intersection of N--P and A--D) {};
	\coordinate (M6) at (intersection of N--P and B--C) {};
	\coordinate (M7) at (intersection of Q--P and A'--D') {};
	\coordinate (M8) at (intersection of N--M and B--C) {};
	\draw (M5)--(A)--(B)--(C)--(D)--(M1)--(M6) (M7)--(A')--(B')--(C')--(D')--(M3)--(M4) (M3)--(Q)--(P)--(N)--(M8) (M1)--(M2);
	\draw[dashed] (M7)--(M3)  (M3)--(M2) (M5)--(M1)--(M)--(M8)   ;
	\draw pic[draw,blue,fill=green!50,opacity=.5,angle radius=5mm,angle eccentricity=0.65,"$\alpha$"] {angle = C'--B'--A'};
	\draw pic[draw,blue,fill=green!50,opacity=.5,angle radius=5mm,angle eccentricity=0.65,"$\beta$"] {angle = C--B--A};
	\draw pic[draw,blue,fill=green!50,opacity=.5,angle radius=5mm,angle eccentricity=0.65,"$\gamma$"] {angle = P--Q--M};
	\draw ($(M3)!1/2!(M4)$) node[above]{$a$};
	\draw ($(M1)!1/2!(M6)$) node[above]{$b$};
\end{tikzpicture}}
\immini
{
\begin{dl}\textbf{(Định lý Thales)}
Ba mặt phẳng đôi một song song chắn trên hai cát tuyến bất kì những đoạn thẳng tương ứng tỉ lệ.\\
Trên hình vẽ bên, ta có \begin{center}\fbox{$\dfrac{AB}{A'B'}=\dfrac{BC}{B'C'}=\dfrac{AC}{A'C'}$.} \end{center}
\end{dl}
}
{
\begin{tikzpicture}[line join = round, line cap = round,>=stealth,font=\footnotesize,scale=0.65]
\def \dai{4}
\def\rong{2}
\def \goc{30}
\def \kcach{1.5}	
\draw (\goc-180:\rong)coordinate(A4)--(0,0)coordinate(A1)--+(0:\dai)coordinate(A2)--($(A2)+(A4)-(A1)$)coordinate(A3)--cycle (-120:\kcach)coordinate(B1)+(\goc-180:\rong)coordinate(B4)--(B1)--+(0:1.3*\dai)coordinate(B2)--($(B2)+(B4)-(B1)$)coordinate(B3)--cycle 
(-120:2*\kcach)coordinate(C1)+(\goc-180:\rong)coordinate(C4)--(C1)--+(0:1.5*\dai)coordinate(C2)--($(C2)+(C4)-(C1)$)coordinate(C3)--cycle 
(A1)++(-90:0.4)coordinate(A)--+(5:1.7)coordinate(A') (A)+(-110:\kcach)coordinate(B)--($(A')+(-70:\kcach)$)coordinate(B')
(A)+(-110:2*\kcach)coordinate(C)--($(A')+(-70:2*\kcach)$)coordinate(C')
($(C)!1.5!(A)$)coordinate(a)($(A)!1.5!(C)$)coordinate(a') ($(C')!1.5!(A')$)coordinate(b)($(A')!1.5!(C')$)coordinate(b') (intersection of a--a' and A3--A4)coordinate(I1) (intersection of b--b' and A3--A4)coordinate(I2) (intersection of a--a' and B3--B4)coordinate(I3) (intersection of b--b' and B3--B4)coordinate(I4)
(intersection of a--a' and C3--C4)coordinate(I5) (intersection of b--b' and C3--C4)coordinate(I6)
(a)--(A)--(A')--(b) (I1)--(B)--(B')--(I2) (I3)--(C)--(C')--(I4) 
(a')--(I5) (b')--(I6);
\draw[dashed] (A)--(I1) (A')--(I2) (B)--(I3) (B')--(I4) (I5)--(C) (I6)--(C');
\draw pic[draw,blue,fill=green!50,opacity=.5,angle radius=3mm,angle eccentricity=1.5,"$\alpha$"] {angle = A3--A4--A1};
\draw  pic[draw,blue,fill=green!50,opacity=.5,angle radius=3mm,angle eccentricity=1.5,"$\beta$"] {angle = B3--B4--B1};
\draw  pic[draw,blue,fill=green!50,opacity=.5,angle radius=3mm,angle eccentricity=1.5,"$\gamma$"] {angle = C3--C4--C1};
\foreach \d/\g in {A/180,A'/0,B/180,B'/0,C/180,C'/0,a/180,b/0}\draw (\d)+(\g:.3)node{\small$\d$};
\end{tikzpicture}
}
\end{dl}
\subsubsection{Hình lăng trụ và hình hộp}
Hình lăng trụ $A_1A_2 \ldots A_n . A_1' A_2' \ldots A_n'$ có
\immini{
\begin{itemize}
  \item Hai đáy là hai đa giác lồi $A_1A_2 \ldots A_n$ và $ A_1' A_2' \ldots A_n'$, nằm ở hai mặt phẳng song song nhau.
  \item Các điểm $A_1$, $A_2$, \ldots , $A_n$, $A_1'$, $A_2',$ \ldots, $A_n'$ được gọi là các đỉnh của hình lăng trụ.
  \item Các đoạn thẳng $A_1A_1'$, $A_2A_2'$, \ldots, $A_n A_n'$ được gọi là các cạnh bên của hình lăng trụ.
  \item Các đoạn thẳng $A_1A_2$, $A_2A_3$, \ldots, $A_n A_1$, $A_1'A_2'$, $A_2'A_3'$, \ldots, $A_n' A_1'$ được gọi là các cạnh đáy của hình lăng trụ.
  \item Các tứ giác $A_1A_1'A_2'A_2$, $A_2A_2'A_3'A_3$, \ldots, $A_nA_n'A_1'A_1$ được gọi là các mặt bên của hình lăng trụ, đồng thời chúng là các hình bình hành.
\end{itemize}}{\begin{tikzpicture}[line join=round,line cap=round,line width=.6pt,font=\footnotesize,scale=.7]
	\coordinate (A) at (0,0);
	\coordinate (D) at ($(A)+(0:5.5)$);
	\coordinate (B) at ($(A) + (-120:2)$);
	\coordinate (C) at ($(D)-(A)+(B)$);
	
	\coordinate (A') at ($(A)+(80:5)$);
	\coordinate (B') at ($(B)+(80:5)$);
	\coordinate (C') at ($(C)+(80:5)$);
	\coordinate (D') at ($(D)+(80:5)$);
	
	\coordinate (A_1) at (0,-1);
	\coordinate (A_2) at (2,-1.5);
	\coordinate (A_3) at (4,-1);
	\coordinate (A_4) at (3,-0.5);
	\coordinate (A_5) at (1,-0.5);
	
	\coordinate (A_1') at ($(A_1) +(80:5)$);
	\coordinate (A_2') at ($(A_2) +(80:5)$);
	\coordinate (A_3') at ($(A_3) +(80:5)$);
	\coordinate (A_4') at ($(A_4) +(80:5)$);
	\coordinate (A_5') at ($(A_5) +(80:5)$);
	\coordinate (M_1') at (intersection of A_1--A_1' and B'--C') {};
	\coordinate (M_2') at (intersection of A_2--A_2' and B'--C') {};
	\coordinate (M_3') at (intersection of A_3--A_3' and B'--C') {};
	\coordinate (N_1) at (intersection of A_1--A_1' and A--D) {};
	\coordinate (N_2) at (intersection of A_3--A_3' and A--D) {};
	\draw pic[draw,blue,fill=green!50,opacity=.5,angle radius=5mm,angle eccentricity=0.65,"$\alpha$"] {angle = C--B--A};
	\draw pic[draw,blue,fill=green!50,opacity=.5,angle radius=5mm,angle eccentricity=0.65,"$\alpha '$"] {angle = C'--B'--A'};
	\draw (A_1)--(A_2)--(A_3) (N_1)--(A)--(B)--(C)--(D)--(N_2) 
	(A_1)--(M_1') (A_2)--(M_2') (A_3)--(M_3') (A_1')--(A_2')--(A_3')--(A_4')--(A_5')--cycle  (A')--(B')--(C')--(D')--cycle;
	\draw[dashed] (A_5)--(A_5') (A_4)--(A_4') (A_1)--(A_5)--(A_4)--(A_3) (M_1')--(A_1') (A_2')--(M_2') (A_3')--(M_3') (N_1)--(N_2);
	\foreach \p/\r in {A_1/180,A_2/120, A_3/0,A_4/-90,  A_5/-90, A_1'/180, A_2'/90, A_3'/0, A_4'/90, A_5'/90}
\fill (\p) circle (1.5pt) node[shift={(\r:2mm)}]{\small $\p$};
\end{tikzpicture}}
\begin{note}
	Tên của hình lăng trụ được gọi theo tên của đa giác đáy.
	\begin{itemize}
  \item Hình lăng trụ $ABC.A'B'C'$ được gọi là hình lăng trụ tam giác.
  \item Hình lăng trụ $ABCD.A'B'C'D'$ được gọi là hình lăng trụ tứ giác.
\end{itemize}

\end{note}

Hình lăng trụ $ABCD.A'B'C'D'$ có đáy là hình bình hành được gọi là hình hộp.
\immini{
\begin{itemize}
  \item Các cặp điểm  $A$ và $C'$, $D$ và $B'$, $C$ và $A'$, $B$ và $D'$ được gọi là các đỉnh đối diện của hình hộp.
  \item Các đoạn thẳng $AC'$, $B'D$, $A'C$, $BD'$ được gọi là các đường chéo của hình hộp.
  \item Các cặp hình bình hành $ABCD$ và $A'B'C'D'$, $AA'B'B$ và $DD'C'C$, $AA'D'D$ và $BB'C'C$ được gọi là các mặt đối diện của hình hộp.
\end{itemize}}{\begin{tikzpicture}[scale=0.6, font=\footnotesize, line join=round, line cap=round, >=stealth]
%\draw[color=gray,dash pattern=on 1pt off 1pt,xstep=1.0cm,ystep=1.0cm] (-5.2,-5.2) grid (5.2,5.2);
\coordinate (A) at (0,0);
\coordinate (D) at (5,0);
\coordinate (B) at (-2,-1.5);
\coordinate (C) at ($(B)-(A)+ (D)$);
\coordinate (B') at ($(B)+(70:5)$);
\coordinate (C') at ($(C)+(70:5)$);
\coordinate (D') at ($(D)+(70:5)$);
\coordinate (A') at ($(A)+(70:5)$);
\draw (B)--(C)--(D) (B)--(B') (C)--(C') (D)--(D') (A')--(B')--(C')--(D')--cycle  (A')--(A')  ; 
\draw[dashed] (A)--(A') (B)--(A)--(D);

\foreach \p/\r in {A/180,B/-90, C/0,A'/180,  B'/180, D/0, C'/0, D'/0}
\fill (\p) circle (1.5pt) node[shift={(\r:3mm)}]{$\p$};
\end{tikzpicture}}

	
\end{tomtat}





\subsection{Các dạng toán thường gặp}
\begin{dang}{Chứng minh hai mặt phẳng song song}
\textbf{Phương pháp} Chứng minh hai mặt phẳng song song $(\alpha) \parallel (\beta)$.\\
\immini{Ta chứng minh mặt phẳng $(\alpha)$ có hai đường thẳng \textbf{CẮT NHAU} và lần lượt song song với mặt phẳng $(\beta)$.\\
Cụ thể $$\heva{&a \parallel (\beta) \;  \\&b \parallel (\beta) \\&a, b \subset (\alpha) \\&a \cap b =I
} \Rightarrow (\alpha) \parallel (\beta).$$}{\begin{tikzpicture}[>=stealth,x=1cm,y=1cm,scale=.8]
		\coordinate (A) at (-4,-2);
		\coordinate (B) at (0,-2);
		\coordinate (C) at (1,0);
		\coordinate (D) at (-3,0);
		\draw(A)--(B)--(C)--(D)--cycle;
		\fill (-1.5,2) circle (1.0pt) node[above]{$I$};
		\draw (-3,2) node[above]{$a$};
		\draw (-2.5,2.5) node[above]{$b$};
		\draw(A)--(B)--(C)--(D)--cycle;
		\coordinate(A1) at (-4,1);
		\coordinate (B1) at (0,1);
		\coordinate (C1) at (1,3);
		\coordinate (D1) at (-3,3);
		\draw (-3,2)--(0,2) (-2.5,2.5)--(-.5,1.5)  ;
		\draw pic[draw,blue,fill=green!50,opacity=.5,angle radius=5mm,angle eccentricity=0.65,"$\beta$"] {angle = B--A--D};
		\draw pic[draw,blue,fill=green!50,opacity=.5,angle radius=5mm,angle eccentricity=0.65,"$\alpha$"] {angle = B1--A1--D1};
		\draw(A1)--(B1)--(C1)--(D1)--cycle;
		\end{tikzpicture}}
Ngoài ra, dựa vào phương pháp chứng minh đường thẳng song song với mặt phẳng, ta còn có thể chứng minh hai mặt phẳng song song như sau: \textbf{chứng minh hai đường thẳng cắt nhau nằm trong mặt phẳng này, song song với hai đường thẳng (cắt nhau) nằm trong mặt phẳng kia.}\\
\immini{Cụ thể
$$\heva{& a \parallel c\\ & b\parallel d\\& a, \, b \subset (\alpha) \\ & c, \, d \subset (\beta) \\& a\cap b = I} \Rightarrow (\alpha) \parallel (\beta).$$}{\begin{tikzpicture}[>=stealth,x=1cm,y=1cm,scale=.8]
		\coordinate (A) at (-4,-2);
		\coordinate (B) at (0,-2);
		\coordinate (C) at (1,0);
		\coordinate (D) at (-3,0);
		\draw(A)--(B)--(C)--(D)--cycle;
		\fill (-1.5,2) circle (1.0pt) node[above]{$I$};
		\draw (-3,2) node[above]{$a$};
		\draw (-2.5,2.5) node[above]{$b$};
		\draw (-3,-1) node[above]{$c$};
		\draw (-2.5,- 0.5) node[above]{$d$};
		\draw(A)--(B)--(C)--(D)--cycle;
		\coordinate(A1) at (-4,1);
		\coordinate (B1) at (0,1);
		\coordinate (C1) at (1,3);
		\coordinate (D1) at (-3,3);
		\draw (-3,2)--(0,2) (-2.5,2.5)--(-.5,1.5) (-3,-1)--(0,-1) (-2.5,-0.5)--(-.5,-1.5) ;
		\draw pic[draw,blue,fill=green!50,opacity=.5,angle radius=5mm,angle eccentricity=0.65,"$\beta$"] {angle = B--A--D};
		\draw pic[draw,blue,fill=green!50,opacity=.5,angle radius=5mm,angle eccentricity=0.65,"$\alpha$"] {angle = B1--A1--D1};
		\draw(A1)--(B1)--(C1)--(D1)--cycle;
		\end{tikzpicture}}
\end{dang}
\subsubsection{Ví dụ mẫu}
\begin{vd}%[Nguyễn Trần Phong]%[1K4BC-2]
Cho hình chóp $S.ABC$. Gọi $M$, $N$, $P$ lần lượt là trung điểm các cạnh $SA$, $SB$, $SC$ và $K$ là điểm bất kỳ trên cạnh $MN$. Chứng minh rằng $(MNP) \parallel (ABC)$. Từ đó suy ra $PK \parallel (ABC)$.
\loigiai{\immini{
Ta có $MN$ là đường trung bình của $\triangle SAB$ nên $MN \parallel AB$.\\
Khi đó $\heva{& MN \parallel AB \\ & AB \subset (ABC) \\& MN \not\subset (ABC)} \Rightarrow MN \parallel (ABC)$.\\
Tương tự, $\heva{& NP \parallel BC \text{ (do } NP \text{là đường trung bình }\triangle SBC)\\& BC \subset (ABC) \\& NP \not\subset (ABC)} \\ \Rightarrow NP \parallel (ABC)$.\\
Vậy $\heva{& MN \parallel (ABC) \\& NP \parallel (ABC) \\& MN, \, NP \subset (MNP)\\& MN \cap NP =N}\Rightarrow (MNP) \parallel (ABC).$\\
Mặt khác, do $PK \subset (MNP)$ nên $PK \parallel (ABC).$}{\begin{tikzpicture}[scale=0.8, font=\footnotesize, line join=round, line cap=round, >=stealth]
%\draw[color=gray,dash pattern=on 1pt off 1pt,xstep=1.0cm,ystep=1.0cm] (-5.2,-5.2) grid (5.2,5.2);
\coordinate (A) at (0,0);
\coordinate (C) at (5,0);
\coordinate (B) at (2,-1.5);
\coordinate (S) at ($(A)+ (70:5)$);
\coordinate (M) at ($(S)!1/2!(A)$);
\coordinate (N) at ($(S)!1/2!(B)$);
\coordinate (P) at ($(S)!1/2!(C)$);
\coordinate (K) at ($(M)!1/3!(N)$);
%\draw pic[draw,angle radius=2mm] {right angle = S--H--K}; 
%\draw pic[draw,angle radius=2mm] {right angle = H--I--K};
%\draw pic[draw,angle radius=2mm] {right angle = H--K--D};
%\coordinate (I) at (intersection of O--N and A--B) {};
\draw (A)--(B)--(C)--(S)--cycle (C)--(S)--(B) (M)--(N)--(P); 
\draw[dashed] (A)--(C) (M)--(P)--(K);

\foreach \p/\r in {A/180,B/-90, C/0,S/90, M/180, N/180, P/0, K/-90}
\fill (\p) circle (1.5pt) node[shift={(\r:3mm)}]{$\p$};
\end{tikzpicture}} }
	
\end{vd}


\begin{vd}%[Nguyễn Trần Phong]%[1K4BC-2]
Cho hình chóp $S.ABCD$ có đáy là hình bình hành tâm $O$. Gọi $M$, $N$ lần lượt là trung điểm của $SA$, $SB$.
\begin{listEX}
\item Chứng minh $(OMN) \parallel (SCD)$.
\item Gọi $K$ là điểm bất kỳ trên $MN$. Chứng minh $OK \parallel (SCD)$.	
\end{listEX} 
	\loigiai{
	\begin{listEX}
	\immini{\item Chứng minh $(OMN) \parallel (SCD)$.\\
	Ta có $OM$ là đường trung bình của $\triangle SAC$ nên $OM \parallel SC$.\\
	Khi đó $\heva{& OM \parallel SC \\ & SC \subset (SCD) \\& OM \not\subset (SCD)} \Rightarrow OM \parallel (SCD).$\\
	Tương tự, $\heva{& ON \parallel SD \text{ (do }ON \text{là đường trung bình tam giác } SBD)  \\ & SD \subset (SCD) \\& ON \not\subset (SCD) }\\ \Rightarrow ON \parallel (SCD)$.\\
	Vậy $\heva{&OM, \, ON \parallel (SCD)  \\& OM, \, ON \subset (OMN)\\& OM \cap ON = O }\Rightarrow (OMN) \parallel (SCD).$
	\item Chứng minh $OK \parallel (SCD)$.
	}{\begin{tikzpicture}[scale=1, font=\footnotesize, line join=round, line cap=round, >=stealth]
\def\bc{4}\def\ba{2} \def\h{4} 
\def\gocB{30} % góc B của đáy
\coordinate[label=below left:$B$] (B) at (0,0);
\coordinate[label=above left:$A$] (A) at (\gocB:\ba);
\coordinate[label=below:$C$] (C) at (\bc,0);
\coordinate[label=right:$D$] (D) at ($(C)-(B)+(A)$);
\coordinate[label=above:$S$] (S) at ($(A)+(70:\h)$);
\coordinate[label=right:$M$] (M) at ($(A)!1/2!(S)$); 
\coordinate[label=left:$N$] (N) at ($(B)!1/2!(S)$);
\coordinate[label=below:$O$] (O) at ($(A)!1/2!(C)$);
\coordinate[label=above:$K$] (K) at ($(N)!3/4!(M)$);
\draw  (B)--(C)--(D)--(S)--cycle (S)--(C) ;
\draw[dashed] (O)--(N)--(M)--cycle (A)--(D) (S)--(A)--(B) (A)--(C) (B)--(D) (O)--(K);
\foreach \diem in {M,N,A,B,C,D,O,K,S}	\fill (\diem)circle(1.0pt);
\end{tikzpicture}}
\noindent Ta có $\heva{& (OMN) \parallel (SCD) \\& OK \subset (OMN)} \Rightarrow OK \parallel (SCD)$.
\end{listEX}}
\end{vd}



\begin{vd}%[Nguyễn Trần Phong]%[1K4BC-2]
Cho hình chóp $S.ABCD$ với đáy $ABCD$ là hình thang mà $AD \parallel BC$ và $AD = 2BC$. Gọi $M$, $N$ lần lượt là trung điểm của $SA$ và $AD$. Chứng minh: $(BMN) \parallel (SCD)$, từ đó suy ra $BM \parallel (SCD)$.
\loigiai{ 
\immini{ Ta có: $M$, $N$ lần lượt là trung điểm của $SA$ và $AD$.\\
$\Rightarrow MN $ là đường trung bình của $\triangle SAD \Rightarrow MN \parallel SD$.\\
Ta có: $\heva{&MN \parallel SD\\&SD \subset (SCD)\\&MN \not\subset (SCD)} \Rightarrow MN \parallel (SCD)$.\\
Mà có $2ND = AD = 2BC$ và $ND \parallel BC$ nên suy ra $BNDC$ là hình bình hành $\Rightarrow BN \parallel CD$.
}
{\begin{tikzpicture}[line join=round,line cap=round,line width=.6pt,font=\footnotesize,scale=.8]
	\coordinate[label=below left:$B$] (B) at (0,0);
	\coordinate[label=above right:$N$] (N) at (1,1);
	\coordinate[label=below right:$C$] (C) at (4,0);
	\coordinate[label=above right:$D$] (D) at ($(C)-(B)+(N)$);
	\coordinate[label=above left:$S$] (S) at ($(N)+(80:4)$);
	\coordinate[label=below left:$A$] (A) at (-3,1);
	\coordinate[label=above left:$M$] (M) at ($(A)!0.5!(S)$);
	\filldraw[pattern=dots] (S)--(C)--(D)--cycle;
	\fill[pattern=dots] (N)--(B)--(M)--cycle;
	\draw (B)--(C)--(D)--(S)--cycle (S)--(C) (A)--(B) (S)--(A) (M)--(B);
	\draw[dashed] (N)--(D) (S)--(N)--(B) (N)--(A) (B)--(N)--(M);
	\fill (N)circle(1.5pt) (B)circle(1.5pt) (C)circle(1.5pt) (D)circle(1.5pt) (S)circle(1.5pt) (A)circle(1.5pt) (M)circle(1.5pt);
\end{tikzpicture}
}
\noindent Ta có: $\heva{&NB \parallel CD\\&CD \subset (SCD)\\&NB \not\subset (SCD)} \Rightarrow NB \parallel (SCD)$.\\
Khi đó: $\heva{&MN, \, NB \parallel (SCD)\\&MN, \, NB \subset (BMN)\\& MN \cap NB = N} \Rightarrow (BMN) \parallel (SCD)$.\\
Mặt khác, do $BM \subset (BMN)$ nên $BM \parallel (SCD)$.
}
\end{vd}

\begin{vd}%[Nguyễn Trần Phong]%[1K4BC-2]
Cho hình chóp $S.ABCD$ có đáy là hình thang, đáy lớn $AD$ gấp đôi đáy bé $BC$. Gọi $O= AC \cap BD$, $M$ thuộc cạnh $SA$ sao cho $AM= 2 MS$ và $N$ thuộc cạnh $SB$ sao cho $2BN= NS$.
\begin{listEX}
\item	Chứng minh rằng $ (OMN) \parallel (SCD)$.
\item Gọi $d = (OMN) \cap (ABCD)$, $P= d \cap AD$, $Q= d \cap BC$. Chứng minh tứ giác $PQCD$ là hình bình hành.
\end{listEX}

 
\loigiai{
\begin{listEX}
\immini{
\item Chứng minh rằng $ (OMN) \parallel (SCD)$.\\
Ta có $AM=2 MS \Rightarrow \dfrac{AM}{AS}=\dfrac{2}{3}$.\\
$2BN= NS \Rightarrow \dfrac{BN}{BS}=\dfrac{1}{3}$.\\
Xét $\triangle OAB$ và $\triangle OBC$ có $$\heva{& AD \parallel BC\\ & AD= 2 BC}\Rightarrow \heva{&OA = 2OC \\& OD= 2 OB} \Rightarrow \heva{& \dfrac{AO}{AC} = \dfrac{2}{3} \\& \dfrac{BO}{BD}=\dfrac{1}{3}.}$$
}{\begin{tikzpicture}[line join=round,line cap=round,line width=.6pt,font=\footnotesize,scale=.8]
	\coordinate[label=below left:$B$] (B) at (0,0);
	\coordinate[label=below right:$C$] (C) at (4,0);
	\coordinate (P) at (1,1);
	\coordinate[label=above right:$D$] (D) at ($(C)-(B)+(P)$);
	\coordinate[label=above left:$S$] (S) at ($(P)+(80:4)$);
	\coordinate[label=below left:$A$] (A) at (-3,1);
	\coordinate[label=above left:$M$] (M) at ($(A)!2/3!(S)$);
	\coordinate[label=above:$O$] (O) at ($(A)!2/3!(C)$);
	\coordinate[label=above left:$N$] (N) at ($(B)!1/3!(S)$);
	\coordinate[label=above left:$P$] (P) at ($(D)!1/3!(A)$);
	\coordinate[label=below:$Q$] (Q) at ($(B)!1/3!(C)$);
	\filldraw[pattern=dots] (S)--(C)--(D)--cycle;
	\fill[pattern=dots] (N)--(O)--(M)--cycle;
	\draw (B)--(C)--(D)--(S)--cycle (S)--(C) (A)--(B) (S)--(A) (N)--(M);
	\draw[dashed] (A)--(D) (S)--(N)--(B)  (M)--(O)--(N) (A)--(C) (B)--(D) (P)--(Q);
	\fill (N)circle(1.5pt) (B)circle(1.5pt) (C)circle(1.5pt) (D)circle(1.5pt) (S)circle(1.5pt) (A)circle(1.5pt) (M)circle(1.5pt) (O)circle(1.5pt) (P)circle(1.5pt) (Q)circle(1.5pt);
\end{tikzpicture}}
\noindent 
Trong tam giác $SAC$ có $\dfrac{AM}{AS} =\dfrac{AO}{AC}=\dfrac{2}{3}$ nên $OM \parallel SC$.\\
Trong tam giác $ SBD$ có $\dfrac{BN}{BS} =\dfrac{BO}{BD}=\dfrac{1}{3}$ nên $ON \parallel SD$.\\
Như vậy, $$\heva{&  OM \parallel SC \\& ON \parallel SD \\& OM, \, ON \subset (OMN) \\& SC,\,  SD \subset (SCD) \\& OM \cap ON = O} \Rightarrow (OMN) \parallel (SCD).$$
\item Chứng minh tứ giác $ PQCD$ là hình bình hành.
\\
Ta có $\heva{& (OMN) \parallel (SCD) \\& (ABCD) \cap (SCD) = CD \\& (ABCD) \cap (OMN) = d} \Rightarrow d \parallel CD$, trong đó $d$ đi qua $O \in (ABCD) \cap (OMN)$.\\
Xét tứ giác $PQCD$ có $\heva{& PQ \parallel CD \\& PD \parallel CQ} \Rightarrow PQCD$ là hình bình hành.
\end{listEX}}	
\end{vd}

\begin{vd}%[Nguyễn Trần Phong]%[1K4BC-2]
Cho hình lăng trụ $ABC. A'B'C'$. Gọi $M$, $N$, $M'$ lần lượt là trung điểm các cạnh $AB$, $AC$ và $A'B'$. 
\begin{listEX}
\item Chứng minh $(MNM') \parallel (BCC'B')$.
\item Tìm giao điểm $N'$ của $A'C'$ và $(MNM')$. Tứ giác $MNN'M'$ là hình gì?	
\end{listEX}
\loigiai{
\begin{listEX}
\immini{
\item 	Chứng minh $(MNM') \parallel (BCC'B')$.\\
Ta có $\heva{& MM' \parallel BB' \text{ (tính chất đường trung bình hình bình hành } AA'B'B) \\& MN \parallel BC \text{ (tính chất đường trung bình tam giác }ABC)\\& MM', \, MN \subset (MNM')\\& BB', \, BC \subset (BB'C'C)\\& MN \cap MM' = M }\\ \Rightarrow (MNM') \parallel (BCC'B')$.
}{\begin{tikzpicture}[scale=0.7, font=\footnotesize, line join=round, line cap=round, >=stealth]
%\draw[color=gray,dash pattern=on 1pt off 1pt,xstep=1.0cm,ystep=1.0cm] (-5.2,-5.2) grid (5.2,5.2);
\coordinate (A) at (0,0);
\coordinate (C) at (5,0);
\coordinate (B) at (2,-2);
\coordinate (A') at ($(A) +(70:5)$);
\coordinate (B') at ($(B)+(70:5)$);
\coordinate (C') at ($(C)+(70:5)$);
\coordinate (M) at ($(A)!1/2!(B)$);
\coordinate (N) at ($(A)!1/2!(C)$);
\coordinate (M') at ($(A')!1/2!(B')$);
\coordinate (N') at ($(A')!1/2!(C')$);
%\draw pic[draw,angle radius=2mm] {right angle = S--H--K}; 
%\draw pic[draw,angle radius=2mm] {right angle = H--I--K};
%\draw pic[draw,angle radius=2mm] {right angle = H--K--D};
%\coordinate (I) at (intersection of O--N and A--B) {};
\draw (A)--(B)--(C)  (A')--(B')--(C')--cycle (A)--(A') (B)--(B') (C)--(C') (M')--(N') (M)--(M') ; 
\draw[dashed] (A)--(C) (M)--(N) (N)--(N') (M')--(N);

\foreach \p/\r in {A/180,B/-90, C/0,A'/90, B'/-30, C'/-90, M/180, N/120, M'/180, N'/90}
\fill (\p) circle (1.5pt) node[shift={(\r:3mm)}]{$\p$};
\end{tikzpicture}}
\item Tìm giao điểm $N'$ của $A'C'$ và $(MNM')$. Tứ giác $MNN'M'$ là hình gì?
\\ Ta có $\heva{& (MNM') \parallel (BCC'B') \\ & (A'B'C') \cap (BCC'B') = B'C'\\& M' \in (A'B'C') \cap (MNM')}$\\
Suy ra $(A'B'C') \cap (MNM') = x'M'x \parallel B'C'$.\\
Trong $(A'B'C')$, gọi $N' = A'C' \cap x'M'x \Rightarrow \heva{& N' \in A'C' \\& N' \in x'M'x \subset (MNM')}\Rightarrow N' = A'C' \cap (MNM')$.\\
Suy ra $M'N'$ là đường trung bình tam giác $A'B'C' \Rightarrow  M'N' =\dfrac{1}{2} B'C'$.\\
Mà $MN$ là đường trung bình tam giác $ABC$ nên $MN \parallel BC$ và $MN =\dfrac{1}{2} BC$.\\
Lại do $BC \parallel B'C'$ và $BC= B'C'$ nên $MN \parallel M'N'$ và $MN= M'N'$.\\
Vậy tứ giác $MNN'M'$  là hình bình hành.
\end{listEX}
}
\end{vd}

\begin{vd}%[Nguyễn Trần Phong]%[1K4BC-2]
Cho hình chóp $S.ABC$, trên cạnh $SA$ lấy hai điểm $A_1$, $A_2$ sao cho $ A_2A_1 = 2 A_1A$. Gọi $(P)$ và $(Q)$ là hai mặt phẳng lần lượt đi qua $A_1$, $A_2$, đồng thời cùng song song với $(ABC)$. Mặt phẳng $(P)$ cắt các cạnh $SB$, $SC$ lần lượt tại $B_1$, $C_1$; mặt phẳng $(Q)$ lần lượt cắt các cạnh $SB$, $SC$ tại $B_2$, $C_2$. Chứng minh $B_2B_1= 2 B_1B$ và $C_2C_1 = 2 C_1C$.
\loigiai{\immini{
Áp dụng định lý Thales cho ba mặt phẳng đôi một song song $(P)$, $(Q)$, $(ABC)$ và hai cát tuyến $SA$, $SB$ ta được $$\dfrac{A_2A_1}{A_1A} =\dfrac{B_2B_1}{B_1B}.$$
\noindent Vì $\dfrac{A_2A_1}{A_1A}= 2$ (do $A_2A_1=2A_1A$) nên $\dfrac{B_2B1}{B_1B}=2 \Rightarrow B_2B_1=2B_1B$.\\
Chứng minh tương tự, ta được $C_2C_1=2C_1C$.}{\begin{tikzpicture}[scale=0.8, font=\footnotesize, line join=round, line cap=round, >=stealth]
%\draw[color=gray,dash pattern=on 1pt off 1pt,xstep=1.0cm,ystep=1.0cm] (-5.2,-5.2) grid (5.2,5.2);
\coordinate (A) at (0,0);
\coordinate (C) at (5,0);
\coordinate (B) at (2,-2);
\coordinate (S) at ($(A)+(70:5)$);
\coordinate (A_1) at ($(A)!1/5!(S)$);
\coordinate (A_2) at ($(A)!3/5!(S)$);
\coordinate (B_1) at ($(B)!1/5!(S)$);
\coordinate (B_2) at ($(B)!3/5!(S)$);
\coordinate (C_1) at ($(C)!1/5!(S)$);
\coordinate (C_2) at ($(C)!3/5!(S)$);
%\coordinate (H) at (intersection of S--M and C--N) {};
\draw (B)--(A)--(S)--(C)--cycle (A_1)--(B_1)--(C_1) (A_2)--(B_2)--(C_2) (S)--(B); 
\draw[dashed] (A)--(C) (A_1)--(C_1) (A_2)--(C_2);

\foreach \p/\r in {A/180,B/-90, C/0, A_1/180, S/90,  A_2/180, B_1/160, B_2/180, C_1/0, C_2/0}
\fill (\p) circle (1.5pt) node[shift={(\r:3mm)}]{$\p$};
\end{tikzpicture}}}
\end{vd}

\subsubsection{Bài tập rèn luyện}
\begin{tcolorbox}[colback=red!5!white,colframe=red!75!black] \begin{center}
\textbf{ BÀI TẬP TỰ LUẬN} \end{center} \end{tcolorbox}
%%Bài 1
\begin{bt}%[Nguyễn Trần Phong]%[1K4BC-2]
	Cho hình chóp $S.ABCD$ có đáy $ABCD$ là hình bình hành. Gọi $M$ là điểm trên $AD$ và $(\alpha)$ là mặt phẳng qua $M$ và song song với $AB$ và $SC$.
	\begin{listEX}
		\item Chứng minh hai mặt phẳng $(\alpha)$ và $(SDC)$ song song.
		\item Chứng minh $SD$ song song với mặt phẳng $(\alpha)$.
	\end{listEX}
\loigiai{
\begin{listEX}
	\immini
	{
\item Chứng minh hai mặt phẳng $(\alpha)$ và $(SDC)$ song song.\\
Ta có $\heva{& SC \parallel (\alpha) \\& CD \parallel (\alpha) \text{ (do } CD \parallel AB \parallel (\alpha))\\& CD, \, SD \subset (SCD) \\& CD \cap SC = C}\Rightarrow (\alpha) \parallel (SCD).$
	\item Chứng minh $SD$ song song với mặt phẳng $(\alpha)$.\\
	Ta có $\heva{& (SCD)\parallel(\alpha)\\ & SD\subset (SCD)}\Rightarrow SD\parallel  (\alpha)$.}
	{
\begin{tikzpicture}[scale=0.8, font=\footnotesize, line join=round, line cap=round, >=stealth]
\def\bc{4} \def\ba{2} 
\def\h{4} 
\def\gocB{30} % góc B của đáy
\coordinate[label=below left:$B$] (B) at (0,0);
\coordinate[label=above left:$A$] (A) at (\gocB:\ba);
\coordinate[label=below:$C$] (C) at (\bc,0);
\coordinate[label=right:$D$] (D) at ($(C)-(B)+(A)$);
\coordinate[label=above:$S$] (S) at ($(A)+(90:\h)$);
\coordinate[label=above:$M$] (M) at ($(D)!2/3!(A)$); 
\coordinate (N) at ($(C)!2/3!(B)$); 
\coordinate (P) at ($(S)!2/3!(B)$); 
\draw (P)--(N) (B)--(C)--(D)--(S)--cycle (S)--(C);
\draw[dashed] (P)--(M)--(N) (A)--(D) (S)--(A)--(B);
\foreach \diem in {N,M,P,A,B,C,D,S}	\fill (\diem)circle(1.0pt);
\end{tikzpicture}
	}
\end{listEX}



}
\end{bt}

%Bài 2
\begin{bt}%[Nguyễn Trần Phong]%[1K4BC-2]
Cho hình chóp $S.ABC$. Gọi $G_1$, $G_2$, $G_3$ lần lượt là trọng tâm các tam giác $SAB$, $SBC$ và $SAC$. Chứng minh rằng $(G_1G_2G_3) \parallel (ABC)$.	\loigiai{
\immini{
Gọi $M$, $N$, $P$ lần lượt là trung điểm $AB$, $BC$ và $AC$.\\
Khi đó do $G_1$, $G_2$ lần lượt là trọng tâm $\triangle SAB$ và $\triangle SBC$ nên $$\dfrac{SG_1}{SM}=\dfrac{SG_2}{SN}=\dfrac{2}{3}.$$
Suy ra $G_1G_2 \parallel MN$.\\
Ta có $\heva{& G_1G_2 \parallel MN \\& MN \subset (ABC) \\& G_1G_2 \not\subset (ABC)} \Rightarrow G_1G_2 \parallel (ABC)$.\\
Tương tự, do $G_2$, $G_3$ lần lượt là trọng tâm $\triangle SBC$ và $\triangle SAC$ nên $$\dfrac{SG_2}{SN}=\dfrac{SG_3}{SP}=\dfrac{2}{3}.$$
Suy ra $G_2G_3 \parallel NP$.\\
Ta có $\heva{& G_2G_3 \parallel NP \\& NP \subset (ABC) \\& G_2G_3 \not\subset (ABC)} \Rightarrow G_2G_3 \parallel (ABC)$.\\
Vậy $\heva{& G_1G_2, \, G_2G_3 \parallel (ABC) \\& G_1G_2, \, G_2G_3 \subset (G_1G_2G_3) \\& G_1G_2 \cap G_2G_3 =G_2}\Rightarrow (G_1G_2G_3) \parallel (ABC).$
}{\begin{tikzpicture}[scale=0.9, font=\footnotesize, line join=round, line cap=round, >=stealth]
%\draw[color=gray,dash pattern=on 1pt off 1pt,xstep=1.0cm,ystep=1.0cm] (-5.2,-5.2) grid (5.2,5.2);
\coordinate (A) at (0,0);
\coordinate (C) at (5,0);
\coordinate (B) at (2,-1.5);
\coordinate (S) at ($(A)+ (70:5)$);
\coordinate (M) at ($(A)!1/2!(B)$);
\coordinate (N) at ($(B)!1/2!(C)$);
\coordinate (P) at ($(A)!1/2!(C)$);
\coordinate (G_1) at ($(S)!2/3!(M)$);
\coordinate (G_2) at ($(S)!2/3!(N)$);
\coordinate (G_3) at ($(S)!2/3!(P)$);
%\draw pic[draw,angle radius=2mm] {right angle = S--H--K}; 
%\draw pic[draw,angle radius=2mm] {right angle = H--I--K};
%\draw pic[draw,angle radius=2mm] {right angle = H--K--D};
%\coordinate (I) at (intersection of O--N and A--B) {};
\draw (A)--(B)--(C)--(S)--cycle (C)--(S)--(B) (S)--(M) (S)--(N); 
\draw[dashed] (A)--(C) (S)--(P) (G_1)--(G_2)--(G_3)--cycle (M)--(N)--(P)--cycle;

\foreach \p/\r in {A/180,B/-90, C/0,S/90, M/180, N/-20, P/30, G_1/180, G_2/0, G_3/30}
\fill (\p) circle (1.5pt) node[shift={(\r:3mm)}]{$\p$};
\end{tikzpicture}}}
\end{bt}

%%Bài 3
\begin{bt}%[Nguyễn Trần Phong]%[1K4BC-2]
Cho hình chóp $S.ABCD$ có đáy $ABCD$ là hình bình hành tâm $O$. Gọi $M$, $N$, $P$ lần lượt là trung điểm $SA$, $SB$, $SD$ và $K$, $I$ là trung điểm của $BC$, $OM$. Chứng minh rằng
\begin{enumEX}{3}
\item $(OMN) \parallel (SCD)$.
\item $(PMN) \parallel (ABCD)$.
\item $KI \parallel (SCD)$.
\end{enumEX}
\loigiai{
\immini{a) Chứng minh: $(OMN) \parallel (SCD)$.\\
Ta có: $M$, $O$ lần lượt là trung điểm của $SA$ và $AC$.\\
$\Rightarrow OM $ là đường trung bình của $\triangle SAC \Rightarrow OM \parallel SC$.\\
Ta có: $\heva{&OM \parallel SC\\&SC \subset (SCD)\\&OM \not\subset (SCD)} \Rightarrow OM \parallel (SCD)$.\\
Ta có: $N$, $O$ lần lượt là trung điểm của $SA$ và $BD$.\\
$\Rightarrow ON $ là đường trung bình của $\triangle SBD \Rightarrow ON \parallel SD$.\\
Tương tự: $\heva{&ON \parallel SD\\&SD \subset (SCD)\\&ON \not\subset (SCD)} \Rightarrow ON \parallel (SCD)$.\\
Khi đó: $\heva{&OM \parallel (SCD)\\&ON \parallel (SCD)\\&OM, \, ON \subset (OMN) \\& OM \cap ON = O} \Rightarrow (OMN) \parallel (SCD)$.}{
	\begin{tikzpicture}[line join=round,line cap=round,line width=.6pt,font=\footnotesize,scale=1]
		\coordinate[label=below left:$B$] (B) at (0,0);
		\coordinate[label=above left:$A$] (A) at (1,1.5);
		\coordinate[label=below right:$C$] (C) at (4,0);
		\coordinate[label=above right:$D$] (D) at ($(C)-(B)+(A)$);
		\coordinate[label=above left:$S$] (S) at ($(A)+(80:4)$);
		\coordinate[label=above right :$M$] (M) at ($(A)!1/2!(S)$);
		\coordinate[label=above left :$N$] (N) at ($(S)!1/2!(B)$);
		\coordinate[label=above right :$P$] (P) at ($(S)!1/2!(D)$);
		\coordinate[label=above right :$O$] (O) at ($(B)!1/2!(D)$);
		\coordinate[label=above right :$I$] (I) at ($(O)!1/2!(M)$);
		\coordinate[label=below :$K$] (K) at ($(B)!1/2!(C)$);	
		\draw (B)--(C)--(D)--(S)--cycle (S)--(C);
		\draw[dashed] (A)--(D) (S)--(A)--(B) (A)--(C) (B)--(D) (O)--(M) (O)--(N) (N)--(M) (P)--(M) (N)--(P) (K)--(I) (K)--(M) (O)--(K);
		\fill (A)circle(1.5pt) (B)circle(1.5pt) (C)circle(1.5pt) (D)circle(1.5pt) (M)circle(1.5pt) (N)circle(1.5pt) (P)circle(1.5pt) (O)circle(1.5pt) (I)circle(1.5pt) (K)circle(1.5pt);
	\end{tikzpicture}} \noindent
b) Chứng minh: $(PMN) \parallel (ABCD)$.\\
Ta có: $M$, $N$ lần lượt là trung điểm của $SA$ và $SB$.\\
$\Rightarrow MN $ là đường trung bình của $\triangle SAB \Rightarrow MN \parallel AB$.\\
Ta có: $\heva{&MN \parallel AB \\& AB \subset (ABCD)\\&MN \not\subset (ABCD)} \Rightarrow MN \parallel (ABCD)$.\\
Ta có: $N$, $P$ lần lượt là trung điểm của $SB$ và $SD$.\\
$\Rightarrow NP $ là đường trung bình của $\triangle SBD \Rightarrow NP \parallel BD$.\\
Tương tự: $\heva{&NP \parallel BD\\&BD \subset (ABCD)\\&NP \not\subset (ABCD)} \Rightarrow NP \parallel (ABCD)$.\\
Khi đó: $\heva{&MN, \, NP \parallel (ABCD)\\& MN, \, NP \subset (MNP)\\& MN \cap NP = N} \Rightarrow (PMN) \parallel (ABCD)$.\\
c) Chứng minh: $KI \parallel (SCD)$.\\
Ta có: $M$, $O$ lần lượt là trung điểm của $SA$ và $AC$.\\
$\Rightarrow OM $ là đường trung bình của $\triangle SAC \Rightarrow OM \parallel SC$.\\
Ta có: $\heva{&OM \parallel SC\\&SC \subset (SCD)\\&OM \not\subset (SCD)} \Rightarrow OM \parallel (SCD)$.\\
Ta có: $K$, $O$ lần lượt là trung điểm của $BC$ và $BD$.\\
$\Rightarrow OK $ là đường trung bình của $\triangle BCD \Rightarrow OK \parallel CD$.\\
Tương tự: $\heva{&OK \parallel CD\\&CD \subset (SCD)\\&OK \not\subset (SCD)} \Rightarrow OK \parallel (SCD)$.\\
Khi đó: $\heva{&OM, \, OK \parallel (SCD)\\&OM, \, OK \subset (OMK) \\& OM \cap OK = O} \Rightarrow (OMK) \parallel (SCD)$.\\
Do $KI \subset (OMK) \Rightarrow KI \parallel (SCD)$.

}	
\end{bt}
%Bài 4
\begin{bt}%[Nguyễn Trần Phong]%[1K4BC-2]
Cho hình chóp $S.ABCD$ có đáy $ABCD$ là hình bình hành tâm $O$. Gọi $M$, $N$ lần lượt là trung điểm $SA$, $SD$.
\begin{listEX}
\item Chứng minh: $(OMN) \parallel (SBC)$.
\item Gọi $P$, $Q$, $R$ lần lượt là trung điểm $AB$, $ON$, $SB$. Chứng minh: $PQ \parallel (SBC)$.
\item Chứng minh: $(MOR) \parallel (SCD)$.
\end{listEX}
\allowdisplaybreaks
\loigiai{
\begin{listEX}
\item Chứng minh: $(OMN) \parallel (SBC)$.
\immini{Ta có: $M$, $O$ lần lượt là trung điểm của $SA$ và $AC$.\\
$\Rightarrow OM $ là đường trung bình của $\triangle SAC \Rightarrow OM \parallel SC$.\\
Ta có: $\heva{&OM \parallel SC\\&SC \subset (SBC)\\&OM \not\subset (SBC)} \Rightarrow OM \parallel (SBC)$.\\
Ta có: $N$, $O$ lần lượt là trung điểm của $SD$ và $AC$.\\
$\Rightarrow ON $ là đường trung bình của $\triangle SBD \Rightarrow ON \parallel SB$.\\
Tương tự: $\heva{&ON \parallel SB\\&SB \subset (SBC)\\&ON \not\subset (SBC)} \Rightarrow ON \parallel (SBC)$.\\
Khi đó: $\heva{&OM \parallel (SBC)\\&ON \parallel (SBC)\\&OM, \, ON \subset (OMN)\\& OM \cap ON = O} \Rightarrow (OMN) \parallel (SBC)$.}{
	\begin{tikzpicture}[line join=round,line cap=round,line width=.6pt,font=\footnotesize,scale=1]
		\coordinate[label=below left:$B$] (B) at (0,0);
		\coordinate[label=above left:$A$] (A) at (1,1.5);
		\coordinate[label=below right:$C$] (C) at (4,0);
		\coordinate[label=above right:$D$] (D) at ($(C)-(B)+(A)$);
		\coordinate[label=above left:$S$] (S) at ($(A)+(80:4)$);
		\coordinate[label=above right :$M$] (M) at ($(A)!1/2!(S)$);
		\coordinate[label=above right :$N$] (N) at ($(S)!1/2!(D)$);
		\coordinate[label=above :$P$] (P) at ($(A)!1/2!(B)$);
		\coordinate[label=below :$O$] (O) at ($(B)!1/2!(D)$);
		\coordinate[label=above left :$Q$] (Q) at ($(O)!1/2!(N)$);
		\coordinate[label=left :$R$] (R) at ($(B)!1/2!(S)$);
		\draw (B)--(C)--(D)--(S)--cycle (S)--(C);
		\draw[dashed] (A)--(D) (S)--(A)--(B) (A)--(C) (B)--(D) (N)--(M) (O)--(M) (N)--(O) (Q)--(P) (O)--(P) (N)--(P) (O)--(R) (M)--(R);
		\fill (A)circle(1.5pt) (B)circle(1.5pt) (C)circle(1.5pt) (D)circle(1.5pt) (M)circle(1.5pt) (N)circle(1.5pt) (O)circle(1.5pt) (P)circle(1.5pt) (Q)circle(1.5pt) (R)circle(1.5pt);
	\end{tikzpicture}}
\item Gọi $P$, $Q$, $R$ lần lượt là trung điểm $AB$, $ON$, $SB$. Chứng minh: $PQ \parallel (SBC)$.\\
Ta có: $O$, $P$ lần lượt là trung điểm của $BD$ và $AB$.\\
$\Rightarrow OP $ là đường trung bình của $\triangle ABC \Rightarrow OP \parallel BC$.\\
Ta có: $\heva{&OP \parallel BC\\&BC \subset (SBC)\\&OP \not\subset (SBC)} \Rightarrow OP \parallel (SBC)$.\\
Ta có: $N$, $O$ lần lượt là trung điểm của $SD$ và $BD$.\\
$\Rightarrow ON $ là đường trung bình của $\triangle SBD \Rightarrow ON \parallel SB$.\\
Tương tự: $\heva{&ON \parallel SB\\&SB \subset (SBC)\\&ON \not\subset (SBC)} \Rightarrow ON \parallel (SBC)$.\\
Khi đó: $\heva{&OP \parallel (SBC)\\&ON \parallel (SBC)\\&OP, \, ON \subset (OPN) \\& OP \cap ON = O} \Rightarrow (OPN) \parallel (SBC)$.\\
Do $PQ \subset (OPN) \Rightarrow PQ \parallel (SBC)$.
\item Chứng minh: $(MOR) \parallel (SCD)$.\\
Ta có: $M$, $O$ lần lượt là trung điểm của $SA$ và $AC$.\\
$\Rightarrow OM $ là đường trung bình của $\triangle SAC \Rightarrow OM \parallel SC$.\\
Ta có: $\heva{&OM \parallel SC\\&SC \subset (SCD)\\&OM \not\subset (SCD)} \Rightarrow OM \parallel (SCD)$.\\
Ta có: $R$, $O$ lần lượt là trung điểm của $SB$ và $BD$.\\
$\Rightarrow OR $ là đường trung bình của $\triangle SBD \Rightarrow OR \parallel SD$.\\
Tương tự: $\heva{&OR \parallel SD\\&SD \subset (SCD)\\&OR \not\subset (SCD)} \Rightarrow OR \parallel (SCD)$.\\
Khi đó: $\heva{&OM \parallel (SCD)\\&OR \parallel (SCD)\\&OM, \, OR \subset (MOR) \\& OM \cap OR = O} \Rightarrow (MOR) \parallel (SCD)$.
\end{listEX}
}	
\end{bt}
%Bài 5
\begin{bt}%[Nguyễn Trần Phong]%[1K4BC-2]
Cho hai hình bình hành $ABCD$ và $ABEF$ có chung cạnh $AB$ và không đồng phẳng. Gọi $I$, $J$, $K$ lần lượt là trung điểm các cạnh $AB$, $CD$, $EF$.
\begin{listEX}
\item Chứng minh: $(ADF) \parallel (BCE)$.
\item Chứng minh: $(DIK) \parallel (JBE)$.
\end{listEX}
\loigiai{
\begin{listEX}
\item Chứng minh: $(ADF) \parallel (BCE)$.
\immini{	
Do $ABCD$ là hình bình hành nên $AD \parallel BC$.\\
Ta có: $\heva{&AD \parallel BC\\&BC \subset (BCE)\\&AD \not\subset (BCE)} \Rightarrow AD \parallel (BCE)$.\\
Tương tự: Do $ABEF$ là hình bình hành nên $AF \parallel BE$.\\
Ta có: $\heva{&AF \parallel BE\\&BE \subset (BCE)\\&AF \not\subset (BCE)} \Rightarrow AF \parallel (BCE)$.\\
Khi đó: $\heva{&AD \parallel (BCE)\\& AF \parallel (BCE)\\&AD, \, AF \subset (ADF)\\& AD \cap AF = A} \Rightarrow (ADF) \parallel (BCE)$.}{\begin{tikzpicture}[line join=round,line cap=round,line width=.6pt,font=\footnotesize,scale=1]
		\coordinate[label=below left:$D$] (D) at (0,0);
		\coordinate[label=above left:$A$] (A) at (1.5,.8);
		\coordinate[label=below right:$C$] (C) at (5,0);
		\coordinate[label=above right:$B$] (B) at ($(C)-(D)+(A)$);
		\coordinate (H) at ($(A)!1/2!(D)$);
		\coordinate (H') at ($(C)!1/2!(B)$);
		\coordinate[label=above :$E$] (E) at ($(H')+(90:4)$);
		\coordinate[label=above left:$F$] (F) at ($(H)+(90:4)$);
		\coordinate[label=above right:$I$] (I) at ($(A)!1/2!(B)$);
		\coordinate[label=below:$J$] (J) at ($(C)!1/2!(D)$);
		\coordinate[label=above:$K$] (K) at ($(E)!1/2!(F)$);
		\draw (D)--(C)--(E)--(F)--cycle (C)--(B) (E)--(B) (K)--(D) (J)--(E);
		\draw[dashed] (A)--(B) (F)--(A)--(D) (I)--(D) (I)--(K) (J)--(B);
		\fill[pattern=dots] (K)--(I)--(D)--cycle (J)--(B)--(E)--cycle;
		\fill (A)circle(1.5pt) (D)circle(1.5pt) (C)circle(1.5pt) (B)circle(1.5pt) (F)circle(1.5pt) (E)circle(1.5pt) (I)circle(1.5pt) (J)circle(1.5pt) (K)circle(1.5pt);
	\end{tikzpicture}}
\item Chứng minh: $(DIK) \parallel (JBE)$.\\
Do $I$, $K$ lần lượ là trung điểm của $AB$ và $EF$ nên trong hình bình hành $ABEF$ ta có $IK \parallel BE$.\\
Suy ra $\heva{&IK \parallel BE\\&BE \subset (JBE)\\&IK \not\subset (JBE)} \Rightarrow IK \parallel (JBE)$.\\
Mặt khác ta có $\heva{& BI \parallel DJ \\& BI= DJ =\dfrac{1}{2} AB}$ nên tứ giác $DJBI$ là hình bình hành.\\
Suy ra $DI \parallel BJ$\\
Do $\heva{&DI \parallel JB\\&JB \subset (JBE)\\&DI \not\subset (JBE)} \Rightarrow DI \parallel (JBE)$.\\
Khi đó: $\heva{&IK \parallel (JBE)\\&DI \parallel (JBE)\\&IK, \, DI \subset (DIK) \\& IK \cap DI = I} \Rightarrow (DIK) \parallel (JBE)$.\\
\end{listEX}
}	
\end{bt}
%Bài 6
\begin{bt}%[Nguyễn Trần Phong]%[1K4BC-2]
Cho các hình bình hành $ABCD$, $ABEF$ nằm trên hai mặt phẳng khác nhau. Trên
các đường chéo $AC$, $BF$ lấy các điểm $M$, $N$ sao cho $MC = 2AM$, $NF = 2BN$. Qua $M$, $N$ lần lượt kẻ các đường thẳng song song với cạnh $AB$, cắt các cạnh $AD$, $AF$ theo thứ tự tại $M_1$, $N_1$.
\begin{listEX}
\item Chứng minh: $MN \parallel DE$.
\item Chứng minh: $M_1 N_1 \parallel (DEF)$.
\item Chứng minh: $(MNM_1 N_1) \parallel (DEF)$.
\end{listEX}
\loigiai{
\begin{listEX}
\item Chứng minh: $MN \parallel DE$.
\immini{
Gọi $O$ là tâm hình bình hành $ABCD$.\\
Trong $\triangle ABD$ co $AO$ là đường trung tuyến và $\dfrac{AM}{AO} = \dfrac{2AM}{AC} = \dfrac{2AM}{3AM} = \dfrac{2}{3}
\Rightarrow M $ là trọng tâm $\triangle ABD$.\\
Tương tự: $N$ là trọng tâm $\triangle ABE$.\\
Gọi $K$ là trung điểm $AB$.\\
Trong $\triangle DKE$ ta có $$\dfrac{KM}{KD} = \dfrac{KN}{KE} = \dfrac{1}{3} \Rightarrow MN \parallel DE.$$}{\begin{tikzpicture}[line join=round,line cap=round,line width=.6pt,font=\footnotesize,scale=1]
		\coordinate[label=below left:$D$] (D) at (0,0);
		\coordinate[label=below:$A$] (A) at (2.40,1.48);
		\coordinate[label=below right:$C$] (C) at (5,0);
		\coordinate[label=above right:$B$] (B) at ($(C)-(D)+(A)$);
		\coordinate (H) at ($(A)!1/2!(D)$);
		\coordinate (H') at ($(C)!1/2!(B)$);
		\coordinate[label=above :$E$] (E) at ($(H')+(90:4)$);
		\coordinate[label=above left:$F$] (F) at ($(H)+(90:4)$);
		\coordinate[label=below:$M$] (M) at ($(A)!1/3!(C)$);
		\coordinate[label=above :$N$] (N) at ($(F)!2/3!(B)$);
		\coordinate[label=below :$K$] (K) at ($(A)!1/2!(B)$);
		\coordinate[label=below:$O$] (O) at ($(A)!1/2!(C)$);
		\coordinate[label=below:$M_1$] (M1) at ($(A)!1/3!(D)$);
		\coordinate[label=above right:$N_1$] (N1) at ($(A)!1/3!(F)$);
		\fill[pattern=dots] (M)--(N)--(N1)--(M1)--cycle;
		\draw (D)--(C)--(E)--(F)--cycle (C)--(B) (E)--(B) (E)--(D);
		\draw[dashed] (A)--(B) (F)--(A)--(D) (C)--(A) (F)--(B) (M)--(N) (K)--(D) (E)--(K) (M1)--(N1) (M1)--(M) (N)--(N1);
		\fill (A)circle(1.5pt) (D)circle(1.5pt) (C)circle(1.5pt) (B)circle(1.5pt) (F)circle(1.5pt) (E)circle(1.5pt) (M)circle(1.5pt) (N)circle(1.5pt) (K)circle(1.5pt) (O)circle(1.5pt) (M1)circle(1.5pt) (N1)circle(1.5pt);
	\end{tikzpicture}}
\item Chứng minh: $M_1 N_1 \parallel (DEF)$.\\
Trong $\triangle DAB$ ta có $MM_1 \parallel AK \Rightarrow \dfrac{DM_1}{DA} = \dfrac{DM}{DK} = \dfrac{2}{3} \Rightarrow \dfrac{AM_1}{AD} = \dfrac{1}{3}$.\\
Tương tự: Trong $\triangle FAB$ ta có $NN_1 \parallel AB \Rightarrow \dfrac{FN_1}{FA} = \dfrac{FN}{FB} = \dfrac{2}{3} \Rightarrow \dfrac{AN_1}{AF} = \dfrac{1}{3}$.\\
Suy ra trong $\triangle ADF$ có $$\dfrac{AM_1}{AD} = \dfrac{AN_1}{AF} = \dfrac{1}{3} \Rightarrow M_1N_1 \parallel DF.$$
Do $DF \subset (DEF) \Rightarrow M_1N_1 \parallel  (DEF)$.
\item Chứng minh: $(MNM_1 N_1) \parallel (DEF)$.\\
Ta có $\heva{&MN \parallel DE\\& DE \subset (DEF)\\&MN \not\subset (DEF)} \Rightarrow MN \parallel (DEF)$.\\
Mặt khác $\heva{&NN_1 \parallel AB\\& AB \parallel EF} \Rightarrow NN_1 \parallel EF$.\\
Suy ra $\heva{&NN_1 \parallel EF\\& EF \subset (DEF)\\&NN_1 \not\subset (DEF)} \Rightarrow NN_1 \parallel (DEF)$.\\
Khi đó: $\heva{&MN \parallel (DEF)\\&NN_1 \parallel (DEF)\\&MN, \, NN_1 \subset (MNM_1 N_1)\\& MN \cap NN_1 = N} \Rightarrow (MNM_1 N_1) \parallel (DEF)$.
\end{listEX}
}	
\end{bt}


%Bài 7
\begin{bt}%[Nguyễn Trần Phong]%[1K4BC-2]
\immini{
Trong mặt phẳng $(P)$, cho hình bình hành $ABCD$. Vẽ các nửa đường thẳng song song nhau, cùng nằm một phía đối với $(P)$ đồng thời đi qua các điểm $A$, $B$, $C$, $D$. Mặt phẳng $(P')$ cắt bốn nửa đường thẳng nói trên tại $A'$, $B'$, $C'$, $D'$ như hình vẽ bên.
\begin{enumEX}{1}
\item Chứng minh mp$(AA', BB')$ song song với mp$(CC', DD')$.
\item Chứng minh tứ giác $A'B'C'D'$ là hình bình hành.	
\item Gọi $O=AC \cap BD$ và $O'= A'C' \cap B'D'$. Chứng minh $AA'+CC'=BB'+DD'$.
\end{enumEX}}{\begin{tikzpicture}[scale=0.7, font=\footnotesize, line join=round, line cap=round, >=stealth]
%\draw[color=gray,dash pattern=on 1pt off 1pt,xstep=1.0cm,ystep=1.0cm] (-5.2,-5.2) grid (5.2,5.2);
\coordinate (A) at (0,0);
\coordinate (D) at (5,0);
\coordinate (B) at (-2,-1.5);
\coordinate (C) at ($(B)-(A)+ (D)$);
\coordinate (B') at ($(B)+(90:4)$);
\coordinate (C') at ($(C)+(90:5)$);
\coordinate (D') at ($(D)+(90:6)$);
\coordinate (O') at ($(B')!1/2!(D')$);
\coordinate (O) at ($(B)!1/2!(D)$);
\coordinate (A') at ($(C')!2!(O')$);
%\draw pic[draw,angle radius=2mm] {right angle = S--H--K}; 
%\draw pic[draw,angle radius=2mm] {right angle = H--I--K};
%\draw pic[draw,angle radius=2mm] {right angle = H--K--D};
%\coordinate (I) at (intersection of O--N and A--B) {};
\draw (B)--(C)--(D) (B)--($(B')+(90:1)$) (C)--($(C')+(90:1)$) (D)--($(D')+(90:1)$) (A')--(B')--(C')--(D')--cycle (B')--(D') (A')--(C') (A')--($(A')+(90:1)$); 
\draw[dashed] (A)--(A') (B)--(A)--(D) (A)--(C) (B)--(D) (O)--(O');

\foreach \p/\r in {A/180,B/-90, C/0,O'/90, A'/180, O/-90, B'/180, D/0, C'/0, D'/0}
\fill (\p) circle (1.5pt) node[shift={(\r:3mm)}]{$\p$};
\end{tikzpicture}}
\loigiai{
\begin{listEX}
\item Chứng minh mp$(AA', BB')$ song song với mp$(CC', DD')$.\\
Ta có $\heva{&AA' \parallel DD' \\& AB \parallel CD \text{ (do } ABCD \text{ là hình bình hành})\\& AA', \, AB \subset (AA', BB')\\& DD', \, CD \subset (CC', DD') \\& AA' \cap AB= A } \Rightarrow (AA', BB') \parallel (CC', DD').$
\item Chứng minh tứ giác $A'B'C'D'$ là hình bình hành.
\\
Ta có $\heva{& (AA', BB') \parallel (CC', DD')\\& (P') \cap (AA', BB')= A'B'\\ & (P') \cap (CC', DD') = C'D'} \Rightarrow A'B' \parallel C'D'$.
\\ Chứng minh tương tự ta được $A'D' \parallel B'C'$.\\
Do đó tứ giác $A'B'C'D'$ là hình bình hành.\\
\item Chứng minh $AA'+CC'=BB'+DD'$.\\
	Xét hình thang $ACC'A'$ ta có $AA'+CC'=2OO'$ (Đường trung bình hình thang).\\
	Xét hình thang $BDD'B'$ ta có $BB'+DD'=2OO'$ (Đường trung bình hình thang).\\
	Khi đó ta suy ra được $AA'+CC'=BB'+DD'$.	
\end{listEX}}
\end{bt}

%%Bài 8
\begin{bt}%[Nguyễn Trần Phong]%[1K4BC-2]
Cho hình chóp $S.ABC$ có $G$ là trọng tâm của tam giác $ABC$. Trên đoạn $SA$ lấy hai điểm $M$, $N$ sao cho $SM = MN = NA$.
\begin{listEX}
\item Chứng minh: $GM \parallel (SBC)$.
\item Gọi $D$ là điểm đối xứng của $A$ qua $G$. Chứng minh: $(MCD) \parallel (NBG)$.
\item Gọi $H = DM \cap (SBC)$. Chứng minh $H$ là trọng tâm $\triangle SBC$.
\end{listEX}
\loigiai{
\begin{listEX}
\immini{\item Chứng minh: $GM \parallel (SBC)$.\\
Gọi $I$ là trung điểm cạnh $BC$.\\
Trong $\triangle SAI$ có $\dfrac{AM}{AS} = \dfrac{AG}{AI} = \dfrac{2}{3} \Rightarrow GM \parallel SI $.\\
Do $\heva{&GM \parallel SI\\&SI \subset (SBC)\\&GM \not\subset (SBC)} \Rightarrow GM \parallel (SBC)$.
\item Gọi $D$ là điểm đối xứng của $A$ qua $G$. Chứng minh: $(MCD) \parallel (NBG)$.\\
Do $D$ là điểm đối xứng của $A$ qua $G$.\\
Khi đó tứ giác $BGCD$ là hình bình hành do $BC$ và $GD$ cắt nhau tại $I$ là trung điểm của mỗi đường.\\
Suy ra $DC \parallel BG$.\\
Mặt khác trong $\triangle MAD$ ta có $$\dfrac{AN}{AM} = \dfrac{AG}{AD} = \dfrac{1}{2} \Rightarrow NG \parallel MD.$$
Khi đó: $\heva{&CD \parallel BG \\&MD \parallel NG\\&CD,\, MD \subset (MCD)\\& BG, \, NG \subset (NBG) \\& CD \cap MD = D} \Rightarrow (MCD) \parallel (NBG)$.}{\begin{tikzpicture}[line join=round,line cap=round,line width=.6pt,font=\footnotesize,scale=1.4]
		\coordinate[label=left:$A$] (A) at (0,0);
		\coordinate[label=below left:$B$] (B) at (1,-1);
		\coordinate[label=right:$C$] (C) at (4,0);
		\coordinate[label=below:$I$] (I) at ($(B)!.5!(C)$);
		\coordinate[label=below:$G$] (G) at ($(A)!2/3!(I)$);
		\coordinate[label=above left:$S$] (S) at ($(G)+(90:4)$);
		\coordinate[label=left:$M$] (M) at ($(S)!1/3!(A)$);
		\coordinate[label=left:$N$] (N) at ($(S)!2/3!(A)$);		
		\coordinate[label=below:$D$] (D) at ($(G)!2!(I)$);
		\coordinate[label=right:$H$] (H) at ($(S)!2/3!(I)$);
		\fill[pattern=dots] (M)--(C)--(D)--cycle;
		\fill[pattern=dots] (N)--(B)--(G)--cycle; 
		\draw (A)--(B)--(C)--(S)--cycle (S)--(B) (I)--(S) (I)--(D) (B)--(N) (C)--(D) (D)--(H) (D)--(B);
		\draw[dashed] (C)--(A) (G)--(S) (M)--(G) (I)--(A) (N)--(G) (B)--(G) (M)--(C) (M)--(H) (G)--(C);
		\fill (A)circle(1pt) (B)circle(1pt) (C)circle(1pt) (S)circle(1pt) (G)circle(1pt) (M)circle(1pt) (N)circle(1pt) (I)circle(1pt) (D)circle(1pt) (H)circle(1pt);
	\end{tikzpicture}}
\item Gọi $H = DM \cap (SBC)$. Chứng minh $H$ là trọng tâm $\triangle SBC$.\\
Gọi $H = DM \cap SI$ trong $(SAI)$.\\
Do $SI \subset (SBC) \Rightarrow H$ là giao điểm của $MD$ và $(SBC)$.\\
Do $HI \parallel MG \Rightarrow \dfrac{HI}{MG} = \dfrac{DI}{DG} = \dfrac{1}{2} \Rightarrow HI = \dfrac{1}{2} MG$.\\
Mặt khác $MG \parallel SI \Rightarrow \dfrac{MG}{SI} = \dfrac{AM}{AS} = \dfrac{2}{3} \Rightarrow MG = \dfrac{2}{3} SI$.\\
Khi đó $HI = \dfrac{1}{2}MG = \dfrac{1}{2} \cdot \dfrac{2}{3} SI = \dfrac{1}{3} SI$.\\
Suy ra $H$ là trọng tâm $\triangle SBC$.
\end{listEX}
}	
\end{bt}

%%Bài 9
\begin{bt}%[Nguyễn Trần Phong]%[1K4BC-2]
	Cho hình chóp $S.ABCD$ có đáy là hình thang, $AD \parallel BC$. Gọi $M$ là trọng tâm tam giác $SAD$, $N$ thuộc cạnh $AC$ sao cho $2NA= NC$, $P$ thuộc cạnh $CD$ sao cho $PC= 2PD$. 
	\begin{listEX}
	\item Tìm giao điểm của $(MNP)$ và $SD$.
	\item Chứng minh $(MNP)\parallel (SBC)$.
	\end{listEX}
\loigiai{
\begin{listEX}
\immini{
\item	Tìm giao điểm của $(MNP)$ và $SD$.\\
Trong tam giác $ADC$ có $\dfrac{CN}{CA}=\dfrac{CP}{CD}=\dfrac{2}{3}$ nên $$NP \parallel AD \parallel BC.$$
Ta có $\heva{& M \in (MNP) \cap (SAD) \\ & NP \parallel AD \\& NP \subset (MNP)\\& AD \subset (SAD) }\\ \Rightarrow (MNP)\cap (SAD) = x'Mx \parallel AD \parallel NP$.\\
Trong $(SAD)$, gọi $H= x'Mx \cap SD\\  \Rightarrow \heva{& H \in x'Mx \subset (MNP) \\& H \in  SD} \Rightarrow H= SD \cap (MNP)$.}{\begin{tikzpicture}[line join=round,line cap=round,line width=.6pt,font=\footnotesize,scale=.9]
	\coordinate (B) at (0,0);
	\coordinate (C) at (4,0);
	\coordinate (A) at (-1,2);
	\coordinate (D) at ($(A) + (0:6)$);
	\coordinate (S) at ($(A) + (70:5)$);
	\coordinate (E) at ($(A)!1/2!(D)$);
	\coordinate (M) at ($(S)!2/3!(E)$);
	\coordinate (N) at ($(A)!1/3!(C)$);
	\coordinate (P) at ($(D)!1/3!(C)$);
	\coordinate (H) at ($(S)!2/3!(D)$);
	\draw (S)--(A)--(B)--(C)--(D)--cycle (B)--(S)--(C)   (P)--(H) ;
	\draw[dashed] (D)--(A)--(C)   (P)--(N)--(M)--(H) (P)--(M) (S)--(E)  ;
	
\foreach \p/\r in {A/180,B/-90, C/-90, D/0, S/90, M/40, N/180, P/0, H/0, E/50}
\fill (\p) circle (1.5pt) node[shift={(\r:3mm)}]{$\p$};

\end{tikzpicture}}
\item Chứng minh $(MNP)\parallel (SBC)$.\\
Ta có $(MNP) \equiv (MNPH)$.\\
Gọi $E$ là trung điểm $AD$.\\
Trong tam giác $SED$ có $\heva{& MH \parallel ED \\& \dfrac{SM}{SE}=\dfrac{2}{3}} \Rightarrow \dfrac{SH}{SD}=\dfrac{2}{3}$.\\
Suy ra $\dfrac{SH}{SD}=\dfrac{CP}{CD}=\dfrac{2}{3}$.\\
Do đó $PH \parallel SC$.\\
Khi đó $\heva{& MH \parallel BC \\& PH \parallel SC \\ & MH, \, PH \subset (MNPH) \\ & BC, \, SC \subset (SBC) \\& MH \cap PH = H} \Rightarrow (MNPH) \parallel (SBC)$ hay $(MNP) \parallel (SBC)$.
\end{listEX}
}
\end{bt}


%Bài 10
\begin{bt}%[Nguyễn Trần Phong]%[1K4KC-2]
	Cho hình chóp $S.ABCD$ có đáy $ABCD$ là hình thang, $AD \parallel BC $, $AD = 2BC$. Gọi $G$, $H$ lần lượt là trọng tâm tam giác $SAB$, $SCD$ và $O$ là giao điểm của $AC$ và $BD$. Gọi $I$ là trung điểm cạnh $AD$.	
	\begin{listEX}
	\item Chứng minh $IG \parallel (SCD)$.
	\item Chứng minh $(OGH) \parallel (SBC)$.	
	\end{listEX}
\loigiai{
\begin{listEX}
	\immini{
	\item Chứng minh $IG \parallel (SCD)$.\\
	Gọi $M$ là trung điểm $SA$. Khi đó $IM$ là đường trung bình của tam giác $SAD$ nên $IM \parallel SD$.\\
	Mặt khác xét tứ giác $BIDC$ có \\ $\heva{& ID \parallel BC \\& ID= BC=\dfrac{1}{2} AD} \Rightarrow BIDC$ là hình bình hành.\\
	Do đó $BI \parallel CD$.\\
	Ta có $\heva{& BI \parallel CD \\& IM \parallel SD  \\& IB, \, IM \subset (IBM)\\& SD, \, CD \subset (SCD) \\& IM \cap IB =I} \Rightarrow (BIM) \parallel (SCD)$.\\
	Mà $IG \subset (IBM)$ nên $IG \parallel (SCD)$. 
			
		}{\begin{tikzpicture}[line join=round,line cap=round,line width=.6pt,font=\footnotesize,scale=.9]
	\coordinate (B) at (0,0);
	\coordinate (C) at (4,0);
	\coordinate (I) at (1,1);
	\coordinate (D) at ($(C)-(B)+(I)$);
	\coordinate (S) at ($(I)+(80:4)$);
	\coordinate (A) at (-3,1);
	\coordinate (M) at ($(A)!1/2!(S)$);
	\coordinate (O) at ($(A)!2/3!(C)$);
	\coordinate (G) at ($(B)!2/3!(M)$);
	\coordinate (N) at ($(S)!1/2!(B)$);
	\coordinate (P) at ($(S)!1/2!(C)$);
	\coordinate (H) at ($(D)!2/3!(P)$);
	\draw (B)--(C)--(D)--(S)--cycle (S)--(C) (A)--(B) (S)--(A)  (B)--(M) (D)--(P) (A)--(N);
	\draw[dashed] (A)--(D)   (G)--(I)--(B) (A)--(C) (B)--(D) (I)--(M) (O)--(G)--(H)--cycle ;
	
\foreach \p/\r in {A/180,B/-90, C/-90, D/0, S/90,  O/90, M/120, G/180, I/60, N/0, P/180, H/0}
\fill (\p) circle (1.5pt) node[shift={(\r:3mm)}]{$\p$};

\end{tikzpicture}}
\item Chứng minh $(OGH) \parallel (SBC)$.\\
Gọi $N$, $P$ lần lượt là trung điểm $SB$ và $SC$.
\\ Xét hai tam giác $OAD$ và $OBC$ có $$\heva{& AD \parallel BC \\& AD=2 BC} \Rightarrow \heva{& OA= 2OC \\& OD= 2 OB} \Rightarrow  \dfrac{AO}{AC}=\dfrac{DO}{DB}=\dfrac{2}{3}.$$
Do $G$, $H$ lần lượt là trọng tâm hai tam giác $SAB$ và $SCD$ nên $\dfrac{AG}{AN}=\dfrac{DH}{DP}=\dfrac{2}{3}$.\\
Xét tam giác $ANC$ có $\dfrac{AG}{AN}=\dfrac{AO}{AC}=\dfrac{2}{3}$ nên $GO \parallel CN$.\\
Xét tam giác $DPB$ có $\dfrac{DH}{DP}=\dfrac{DO}{DP}=\dfrac{2}{3}$ nên $OH \parallel BP$.\\
Khi đó $\heva{& GO \parallel CN \\& HO \parallel BP \\& GO, \, HO \subset (OGH) \\& CN, \, BP \subset (SBC) \\& GO \cap HO=O } \Rightarrow (OGH)\parallel (SBC)$.
\end{listEX}}
\end{bt}


%Bài 11
\begin{bt}%[Nguyễn Trần Phong]%[1K4KC-2]
Cho hình chóp $S.ABCD$ có đáy là hình thang với $AD \parallel BC$, $AD= 2 BC$. Gọi $I$ là trung điểm AD, $M$ là trung điểm $SD$ và $H$, $K$ lần lượt nằm trên cạnh $SA$, $SB$ sao cho $SH= 2 HA$ và $3BK= SB$.
\begin{listEX}
\item Chứng minh $(IMC) \parallel (SAB)$.
\item Tìm giao điểm $G$ của $(MAB)$ và $SC$.
\item Chứng minh $(GHK) \parallel (ABCD)$.	
\end{listEX}
\loigiai{
\begin{listEX}
\immini{\item 	Chứng minh $(IMC) \parallel (SAB)$.\\
Xét tứ giác $AICB$ có $$\heva{&AI\parallel BC  \\& AI= BC=\dfrac{1}{2} AD}.$$
Suy ra tứ giác $AICB$ là hình bình hành và do đó $AB \parallel CI$.\\
Khi đó $\heva{& CI \parallel AB \\& IM \parallel SA \text{ (tính chất đường trung bình } \triangle SAD)\\& IC, \, IM \subset (IMC)\\& SA, \, AB \subset (SAB) \\& IM \cap IC = I}$\\
Suy ra $(IMC) \parallel (SAB).$
}{\begin{tikzpicture}[line join=round,line cap=round,line width=.6pt,font=\footnotesize,scale=1]
	\coordinate (B) at (0,0);
	\coordinate (C) at (4,0);
	\coordinate (I) at (1,1);
	\coordinate (D) at ($(C)-(B)+(I)$);
	\coordinate (S) at ($(I)+(80:4)$);
	\coordinate (A) at (-3,1);
	\coordinate (M) at ($(D)!1/2!(S)$);
	\coordinate (G) at ($(S)!2/3!(C)$);
	\coordinate (H) at ($(S)!2/3!(A)$);
	\coordinate (K) at ($(S)!2/3!(B)$);
	\coordinate (N) at (intersection of A--B and C--D) {};
	\draw (B)--(C)--(D)--(S)--cycle (S)--(C) (A)--(B) (S)--(A) (B)--(N)--(D) (C)--(M)--(N) (H)--(K)--(G);
	\draw[dashed] (A)--(D)  (C)--(I)--(M) (G)--(H);
	
\foreach \p/\r in {A/180,B/-90, C/-90, D/0, S/90, H/180, K/200,  M/0, I/90, N/0, G/220}
\fill (\p) circle (1.5pt) node[shift={(\r:3mm)}]{$\p$};

\end{tikzpicture}}
\item Tìm giao điểm $G$ của $(MAB)$ và $SC$.\\
Trong $(ABCD),$ gọi $N= AB \cap CD \Rightarrow \heva{& N \in AB, \, AB \subset (MAB) \\& N \in CD, \, CD \subset (SCD)} \Rightarrow N \in (MAB) \cap (SCD)$.\\
Mà $M \in (MAB) \cap (SCD)$ nên $MN = (MAB) \cap (SCD)$.\\
Trong $(SCD)$, gọi $G= MN \cap SC \Rightarrow \heva{& G \in MN, \, MN \subset (MAB)\\ & G \in SC} \Rightarrow G= (MAB) \cap SC.$
\item Chứng minh $(GHK) \parallel (ABCD)$.\\
Xét tam giác $NAD$ có $\heva{& BC \parallel AD \\ & AD= 2 BC} \Rightarrow ND=2 NC$ hay $C$ là trung điểm đoạn thẳng $ND$.\\
Khi đó trong tam giác $SDN$ có hai đường trung tuyến $NM$ và $SC$ cắt nhau tại $G$ nên $G$ là trọng tâm tam giác $SDN$.\\
Suy ra $\dfrac{SG}{SC}=\dfrac{2}{3}$.\\
Mà $\dfrac{SK}{SB}=\dfrac{2}{3}$ (do $SB= 3 BK$) nên $\dfrac{SK}{SB}=\dfrac{SG}{SC} \Rightarrow GK \parallel BC$.\\
Trong tam giác $SAB$ ta có $\dfrac{SH}{SA}=\dfrac{SK}{SB}=\dfrac{2}{3}$ nên $HK \parallel AB$.\\
Khi đó $\heva{& HK \parallel AB  \\& GK \parallel BC \\& HK, \, GK \subset (GHK) \\ & AB, \, BC \subset (ABCD) \\& GK \cap HK = K} \Rightarrow (GHK) \parallel (ABCD).$
\end{listEX}}
\end{bt}

%Bài 12
\begin{bt}%[Nguyễn Trần Phong]%[1K4BC-2]
Cho hình hộp  $ABCD.A'B'C'D'$. Gọi $M$, $N$, $M'$ lần lượt là trung điểm $ AD$, $BC$, $ A'D'$.
\begin{listEX}
\item Chứng minh $(MNM') \parallel (DCC'D')$.
\item Tìm giao điểm $N'$ của $(MNM')$ và $B'C'$. Chứng minh $ABNM.A'B'N'M'$ là hình hộp.	
\end{listEX}
\loigiai{\begin{listEX}
\immini{\item Chứng minh $(MNM') \parallel (DCC'D')$.\\
Ta có\\ $\heva{& MM' \parallel DD' \text{ (tính chất đường trung bình hình bình hành } AA'D'D) \\& MN \parallel CD \text{ (tính chất đường trung bình hình bình hành }ABCD)\\& MM', \, MN \subset (MNM')\\& DD', \, CD \subset (CDD'C')\\& MN \cap MM' = M }\\ \Rightarrow (MNM') \parallel (CDD'C')$.


}{\begin{tikzpicture}[scale=0.6, font=\footnotesize, line join=round, line cap=round, >=stealth]
%\draw[color=gray,dash pattern=on 1pt off 1pt,xstep=1.0cm,ystep=1.0cm] (-5.2,-5.2) grid (5.2,5.2);
\coordinate (A) at (0,0);
\coordinate (D) at (5,0);
\coordinate (B) at (-2,-1.5);
\coordinate (C) at ($(B)-(A)+ (D)$);
\coordinate (B') at ($(B)+(70:5)$);
\coordinate (C') at ($(C)+(70:5)$);
\coordinate (D') at ($(D)+(70:5)$);
\coordinate (A') at ($(A)+(70:5)$);
\coordinate (M) at ($(A)!1/2!(D)$);
\coordinate (N) at ($(B)!1/2!(C)$);
\coordinate (M') at ($(A')!1/2!(D')$);
\coordinate (N') at ($(B')!1/2!(C')$);
%\draw pic[draw,angle radius=2mm] {right angle = S--H--K}; 
%\draw pic[draw,angle radius=2mm] {right angle = H--I--K};
%\draw pic[draw,angle radius=2mm] {right angle = H--K--D};
%\coordinate (I) at (intersection of O--N and A--B) {};
\draw (B)--(C)--(D) (B)--(B') (C)--(C') (D)--(D') (A')--(B')--(C')--(D')--cycle  (A')--(A') (N)--(N')--(M')  ; 
\draw[dashed] (A)--(A') (B)--(A)--(D) (M')--(M)--(N);

\foreach \p/\r in {A/180,B/-90, C/0,M'/90, A'/180, N'/90, B'/180, D/0, C'/0, D'/0, M/40, N/-90}
\fill (\p) circle (1.5pt) node[shift={(\r:3mm)}]{$\p$};
\end{tikzpicture}}	
\item Tìm giao điểm $N'$ của $(MNM')$ và $B'C'$. Chứng minh $DCNM.D'C'N'M'$ là hình hộp.
\\Ta có $\heva{& (MNM') \parallel (CDD'C') \\ & (A'B'C'D') \cap (CDD'C') = C'D'\\& M' \in (A'B'C'D') \cap (MNM')}$\\
Suy ra $(A'B'C'D') \cap (MNM') = x'M'x \parallel C'D'$.\\
Trong $(A'B'C'D')$, gọi $N' = B'C' \cap x'M'x \Rightarrow \heva{& N' \in B'C' \\& N' \in x'M'x \subset (MNM')}\Rightarrow N' = B'C' \cap (MNM')$.\\	
Lại do $M'$ là trung điểm $A'B'$ nên $N'$ là trung điểm $B'C'$.\\
Ta có hai tứ giác $CDMN$ và $C'D' M'N'$ nằm trong hai mặt phẳng song song. \quad (1)\\
Mà $NN'$, $MM'$ lần lượt là đường trung bình của hình bình hành $BCC'B'$ và $ADD'A'$ nên $NN' \parallel CC'$ và $MM' \parallel DD'$.\\
Lại do $CC' \parallel DD'$ nên $NN' \parallel CC' \parallel DD' \parallel MM'$. \quad (2)\\
Xét tứ giác $D'C'N'M'$ có $\heva{& M'N' \parallel D'C'\\& C'N' \parallel D'M'} \Rightarrow D'C' N'M'$ là hình bình hành. \quad (3)\\
Tương tự, ta chứng minh được tứ giác $DCNM$ là hình bình hành. \quad (4)
\\ Từ (1), (2), (3), (4) ta suy ra $DCNM.D'C'N'M'$ là hình hộp.
\end{listEX}
}
\end{bt}

\begin{tcolorbox}[colback=red!5!white,colframe=red!75!black] \begin{center}
\textbf{ BÀI TẬP TRẮC NGHIỆM} \end{center} \end{tcolorbox}
\setcounter{ex}{0}
\Opensolutionfile{ans}[ans/ans-haimpsongsong]

%TN1
\begin{ex}%[Nguyễn Trần Phong]%[1K4YC-1]
	Trong các khẳng định sau, khẳng định nào đúng?
	\choice
	{ Qua một điểm có vô số mặt phẳng song song với một mặt phẳng cho trước}
	{ Qua một điểm nằm ngoài một mặt phẳng, có vô số mặt phẳng song song với mặt phẳng đã cho}
	{\True Qua một điểm nằm ngoài một mặt phẳng, tồn tại duy nhất một mặt phẳng song song với mặt phẳng đã cho}
	{Qua một điểm tồn tại duy nhất một mặt phẳng song song với một mặt phẳng cho trước}
	\loigiai{
		Qua một điểm nằm ngoài một mặt phẳng, có một và chỉ một mặt phẳng song song với mặt phẳng đó.
	}
\end{ex}

%TN2
\begin{ex}%[Nguyễn Trần Phong]%[1K4YC-1]
	Cho các mệnh đề sau:\\
	\textbf{1.} Hai mặt phẳng phân biệt cùng song song với một đường thẳng thì chúng song song với nhau.\\
	\textbf{2.} Hai mặt phẳng cùng song song với một mặt phẳng thứ ba thì chúng song song với nhau.\\
	\textbf{3.} Bất kì đường thẳng nào cắt một trong hai mặt phẳng song song thì nó cũng cắt mặt phẳng còn lại.\\
	Số mệnh đề \textbf{sai} là
	\choice
	{0}
	{1}
	{3}
	{\True 2}
	\loigiai{
		\begin{enumerate}[+]
			\item Mệnh đề 	`` Hai mặt phẳng phân biệt cùng song song với một đường thẳng thì chúng song song với nhau.'' là một khẳng định sai vì chúng có thể cắt nhau.
			\item Mệnh đề ``Hai mặt phẳng cùng song song với một mặt phẳng thứ ba thì chúng song song với nhau." là một mệnh đề sai vì hai mặt này có thể trùng nhau.
			\item Mệnh đề ``Bất kì đường thẳng nào cắt một trong hai mặt phẳng song song thì nó cũng cắt mặt phẳng còn lại." là một mệnh đề đúng.
		\end{enumerate}
	}
\end{ex}

%TN3
\begin{ex}%[Nguyễn Trần Phong]%[1K4YC-1]
	Trong các mệnh đề dưới đây, mệnh đề nào đúng?
	\choice
	{\True Nếu hai mặt phẳng $(\alpha)$ và $(\beta)$ song song với nhau thì mọi đường thẳng nằm trong $(\alpha)$ đều song song với $(\beta)$}
	{Nếu hai mặt phẳng $(\alpha)$ và $(\beta)$ song song với nhau thì bất kì đường thẳng nào nằm trong $(\alpha)$ cũng song song với bất kì đường thẳng nào nằm trong $(\beta)$}
	{Nếu hai đường thẳng phân biệt $a$ và $b$ song song lần lượt nằm trong hai mặt phẳng $(\alpha)$ và $(\beta)$ phân biệt thì $(\alpha)\parallel (\beta) $}
	{Nếu đường thẳng $d $ song song với mặt phẳng $(\alpha)$ thì nó song song với mọi đường thẳng nằm trong $(\alpha)$}
	\loigiai{
		Mệnh đề: \lq \lq Nếu hai mặt phẳng $(\alpha)$ và $(\beta)$ song song với nhau thì mọi đường thẳng nằm trong $(\alpha)$ đều song song với $(\beta)$.\rq \rq là đúng.
	}
\end{ex}


%TN4
\begin{ex}%[Nguyễn Trần Phong]%[1K4YC-1]
	Đặc điểm nào sau đây là đúng với hình lăng trụ?
	\choice
	{Hình lăng trụ có tất cả các mặt bên bằng nhau}
	{Đáy của hình lăng trụ là hình bình hành}
	{\True Hình lăng trụ có tất cả các mặt bên là hình bình hành}
	{Hình lăng trụ có tất cả các mặt là hình bình hành}
	\loigiai{
		Hình lăng trụ là một đa diện có hai mặt đáy là các đa giác bằng nhau và những mặt bên là các hình bình hành.
	}
\end{ex}

%TN5
\begin{ex}%[Nguyễn Trần Phong]%[1K4YC-1]
	Trong các điều kiện sau, điều kiện nào kết luận mp $\left(\alpha\right)$ $\parallel$ mp $\left(\beta\right)$ ?
	\choice
	{\True $\left(\alpha\right)\parallel a$ và $\left(\alpha\right)\parallel b$ với $a, b$ là hai đường thẳng cắt nhau thuộc $\left(\beta\right)$}
	{$\left(\alpha\right)\parallel a$ và $\left(\alpha\right)\parallel b$ với $a, b$ là hai đường thẳng phân biệt thuộc $\left(\beta\right)$}
	{$\left(\alpha\right)\parallel a$ và $\left(\alpha\right)\parallel b$ với $a, b$ là hai đường thẳng phân biệt cùng song song với $\left(\beta\right)$}
	{$\left(\alpha\right)\parallel \gamma$ và $\left(\beta\right)\parallel \gamma$ ($\left(\gamma\right)$ là mặt phẳng nào đó)}
	\loigiai{
		Ta có $\heva{ & a \parallel (\alpha) \\& b \parallel (\alpha) \\ & a, \, b \text{ cắt nhau trong } (\beta)}\Rightarrow (\alpha) \parallel (\beta).$
	}
\end{ex}


%TN6
\begin{ex}%[Nguyễn Trần Phong]%[1K4BC-2]
	Cho hình tứ diện $ABCD$. Gọi $I$, $J$, $K$ lần lượt là trọng tâm của các tam giác $ABD$, $ACD$, $ABC$ và $M$, $N$, $P$ lần lượt là trung điểm của các cạnh $BD$, $CD$, $BC$. Khẳng định nào sau đây là \textbf{đúng}?
	\choice
	{$(DJK) \parallel (ABC)$}
	{\True $(IJK)\parallel (BCD)$}
	{$(KMN)\parallel (ABC)$}
	{$(IJK) \parallel (AMD)$}
	\loigiai{
		\immini{
		Ta có $\dfrac{AK}{AP}=\dfrac{AI}{AM}=\dfrac{AJ}{AN}=\dfrac{2}{3}$ nên $KJ \parallel PN$ và $IJ \parallel MN$.\\
		Khi đó $\heva{& KJ \parallel PN \\& IJ \parallel MN \\& KJ, \, IJ \subset (KIJ) \\& PN,\, MN \subset (BCD) \\& KJ \cap IJ = J}$\\
		Suy ra $(KIJ) \parallel (BCD)$.}{\begin{tikzpicture}[scale=0.8, font=\footnotesize, line join=round, line cap=round, >=stealth]
%\draw[color=gray,dash pattern=on 1pt off 1pt,xstep=1.0cm,ystep=1.0cm] (-5.2,-5.2) grid (5.2,5.2);
\coordinate (B) at (0,0);
\coordinate (D) at (5,0);
\coordinate (C) at (2,-1.5);
\coordinate (A) at ($(B)+ (70:5)$);
\coordinate (M) at ($(B)!1/2!(D)$);
\coordinate (N) at ($(D)!1/2!(C)$);
\coordinate (P) at ($(B)!1/2!(C)$);
\coordinate (I) at ($(A)!2/3!(M)$);
\coordinate (J) at ($(A)!2/3!(N)$);
\coordinate (K) at ($(A)!2/3!(P)$);
%\draw pic[draw,angle radius=2mm] {right angle = S--H--K}; 
%\draw pic[draw,angle radius=2mm] {right angle = H--I--K};
%\draw pic[draw,angle radius=2mm] {right angle = H--K--D};
%\coordinate (I) at (intersection of O--N and A--B) {};
\draw (A)--(B)--(C)--(D)--cycle (A)--(C) (A)--(N) (A)--(P); 
\draw[dashed] (B)--(D) (A)--(M) (I)--(J)--(K)--cycle  (M)--(N)--(P)--cycle;
 
\foreach \p/\r in {A/180,B/-90, C/0,D/90, M/150, N/-20, P/180, I/40, J/0, K/180}
\fill (\p) circle (1.5pt) node[shift={(\r:3mm)}]{$\p$};
\end{tikzpicture}}
	}
\end{ex}

%TN7
\begin{ex}%[Nguyễn Trần Phong]%[1K4BC-2]
	Cho hình hộp $ABCD.A'B'C'D'$, $AC$ cắt $BD$ tại $O$ còn $A'C'$ cắt $B'D'$ tại $O'$. Khi đó $(AB'D')$ sẽ song song mặt phẳng nào dưới đây?
	\choice
	{$(A'OC') $}
	{$(BDA') $}
	{\True $(BDC') $}
	{$(BCD) $}
	\loigiai{
		\immini{Ta có $\heva{ &B'D'\parallel BD\\&AB'\parallel DC'\\&B'D',AB'\subset (AB'D')\\ &DC',BD\subset (BDC')\\& B'D' \cap AB' = B'} \Rightarrow (AB'D')\parallel (BDC').$
		}{\begin{tikzpicture}[scale=0.6, font=\footnotesize, line join=round, line cap=round, >=stealth]
%\draw[color=gray,dash pattern=on 1pt off 1pt,xstep=1.0cm,ystep=1.0cm] (-5.2,-5.2) grid (5.2,5.2);
\coordinate (A) at (0,0);
\coordinate (D) at (5,0);
\coordinate (B) at (-1.5,-2);
\coordinate (C) at ($(B)-(A)+ (D)$);
\coordinate (B') at ($(B)+(70:5)$);
\coordinate (C') at ($(C)+(70:5)$);
\coordinate (D') at ($(D)+(70:5)$);
\coordinate (A') at ($(A)+(70:5)$);
\coordinate (O) at ($(A)!1/2!(C)$);
\coordinate (O') at ($(A')!1/2!(C')$);
%\draw pic[draw,angle radius=2mm] {right angle = S--H--K}; 
%\draw pic[draw,angle radius=2mm] {right angle = H--I--K};
%\draw pic[draw,angle radius=2mm] {right angle = H--K--D};
%\coordinate (I) at (intersection of O--N and A--B) {};
\draw (B)--(C)--(D) (B)--(B') (C)--(C') (D)--(D') (A')--(B')--(C')--(D')--cycle  (A')--(A')  (B')--(D')  (D)--(C')--(B); 
\draw[dashed] (A)--(A') (B)--(A)--(D) (D')--(A)--(B') (B)--(D);

\foreach \p/\r in {A/180,B/-90, C/0, A'/180, B'/180, D/0, C'/0, D'/0, O/40, O'/-90}
\fill (\p) circle (1.5pt) node[shift={(\r:3mm)}]{$\p$};
\end{tikzpicture}}
	}
\end{ex}

%TN8

\begin{ex}%[Nguyễn Trần Phong]%[1K4BC-2]
Cho hình hộp $ABCD.{A}'{B}'{C}'{D}'$ có các cạnh bên $A{A}',B{B}',C{C}',D{D}'.$ Khẳng định nào dưới đây \textbf{sai}? 
\choice
{$\left(A{A}'{B}'B\right)\parallel \left(D{D}'{C}'C\right)$}
{\True$\left(B{A}'{D}'\right)\parallel \left(AD{C}'\right)$}
{${A}'{B}'CD$ là hình bình hành}
{$B{B}'{D}'D$ là một tứ giác}
\loigiai{
\immini{
Ta có $\heva{& BD' \subset (BA'D') \\& AC' \subset (ADC')\\& AC' \cap BD' \neq \emptyset} \Rightarrow (BA'D') \not\parallel (ADC')$.
}{\begin{tikzpicture}[scale=0.6, font=\footnotesize, line join=round, line cap=round, >=stealth]
%\draw[color=gray,dash pattern=on 1pt off 1pt,xstep=1.0cm,ystep=1.0cm] (-5.2,-5.2) grid (5.2,5.2);
\coordinate (A) at (0,0);
\coordinate (D) at (5,0);
\coordinate (B) at (-1.5,-2);
\coordinate (C) at ($(B)-(A)+ (D)$);
\coordinate (B') at ($(B)+(70:5)$);
\coordinate (C') at ($(C)+(70:5)$);
\coordinate (D') at ($(D)+(70:5)$);
\coordinate (A') at ($(A)+(70:5)$);
\coordinate (O) at ($(A)!1/2!(C)$);
\coordinate (O') at ($(A')!1/2!(C')$);
%\draw pic[draw,angle radius=2mm] {right angle = S--H--K}; 
%\draw pic[draw,angle radius=2mm] {right angle = H--I--K};
%\draw pic[draw,angle radius=2mm] {right angle = H--K--D};
%\coordinate (I) at (intersection of O--N and A--B) {};
\draw (B)--(C)--(D) (B)--(B') (C)--(C') (D)--(D') (A')--(B')--(C')--(D')--cycle  (A')--(A') (C')--(D) ; 
\draw[dashed] (A)--(A') (B)--(A)--(D) (A')--(B)--(D') (A)--(C') ;

\foreach \p/\r in {A/180,B/-90, C/0, A'/180, B'/180, D/0, C'/0, D'/0, O/40, O'/-90}
\fill (\p) circle (1.5pt) node[shift={(\r:3mm)}]{$\p$};
\end{tikzpicture}

}}
\end{ex}


%TN9
\begin{ex}%[Nguyễn Trần Phong]%[1K4BC-2]
	Cho hình chóp $S.ABCD$ có đáy là hình thang ($AB \parallel CD$) và $AB=2CD$. Gọi $I$, $J$ lần lượt là trung điểm của $SB$ và $AB$. Mặt phẳng nào song song với mặt phẳng $(SAD)$?
	\choice
	{$(BCI)$}
	{$(BIJ)$}
	{\True $(CIJ)$}
	{$(SJC)$}
	\loigiai{
		\immini{
			Vì $I$, $J$ lần lượt là trung điểm của $SB$ và $AB$ nên $IJ\parallel SA$.\\
			Do $J$ là trung điểm $AB$ và $AB=2CD$ nên $AJ=CD$.\\
			Mà $AJ \parallel CD$ nên $AJCD$ là hình bình hành.\\ Bởi vậy $CJ\parallel AD$. \\
			Ta có $\heva{& IJ \parallel SA \\& IC \parallel AD\\ & IJ, \, IC \subset (CIJ) \\& SA, \, AD \subset (SAD) \\ & CI \cap IJ = I } \Rightarrow (CIJ)\parallel (SAD)$.}{
			\begin{tikzpicture}[line join=round,line cap=round,line width=.6pt,font=\footnotesize,scale=.9]
	\coordinate (D) at (0,0);
	\coordinate (C) at ($(D)+(0:4)$);
	\coordinate (A) at (-3,1);
	\coordinate (B) at ($2*(C)-2*(D)+(A)$);
	\coordinate (S) at ($(D)+(80:4)$);
	\coordinate (I) at ($(S)!1/2!(B)$);
	\coordinate (J) at ($(A)!1/2!(B)$);
	\draw (S)--(A)--(D)--(C)--(B)--cycle (D)--(S)--(C)--(I);
	\draw[dashed] (A)--(B) (I)--(J)--(C) ;
	
\foreach \p/\r in {A/180,B/-90, C/-90, D/-90, S/90,   I/20, J/60}
\fill (\p) circle (1.5pt) node[shift={(\r:3mm)}]{$\p$};

\end{tikzpicture}
			
		}
	}
\end{ex}


%TN10

\begin{ex}%[Nguyễn Trần Phong]%[1K4BC-2]
	Cho hình lăng trụ $ABC.A'B'C'$, gọi $I,J,K$ lần lượt là trọng tâm $\triangle ABC$, $\triangle ACC'$ và $\triangle AB'C'$. Mặt phẳng nào sau đây song song với $(IJK)$?
	\choice
	{$(BC'A)$}
	{$(AA'B)$}
	{\True $(BB'C)$}
	{$(CC'A)$}
	\loigiai{
		\immini
		{Gọi $M,N,P$ lần lượt là trung điểm $BC$, $CC'$ và $B'C'$.\\
			Ta có $\dfrac{AK}{AP}=\dfrac{AJ}{AN}=\dfrac{AI}{AM}=\dfrac{2}{3}$.\\
			Suy ra $IK \parallel PM$ và $KJ \parallel PN$.\\
			Khi đó $\heva{&IK \parallel PM\\ &KJ \parallel PN \\ & IK, \, KJ \subset (IJK) \\& PM, \, PN \subset (BCC'B')\\& IK \cap KJ = K }$.\\
			 Suy ra $(IJK)\parallel(BCC'B')$ hay $(IJK)\parallel (BB'C).$}
		{\begin{tikzpicture}[scale=0.7, font=\footnotesize, line join=round, line cap=round, >=stealth]
%\draw[color=gray,dash pattern=on 1pt off 1pt,xstep=1.0cm,ystep=1.0cm] (-5.2,-5.2) grid (5.2,5.2);
\coordinate (A) at (0,0);
\coordinate (C) at (5,0);
\coordinate (B) at (1.5,-2);
\coordinate (A') at ($(A) +(80:5)$);
\coordinate (B') at ($(B)+(80:5)$);
\coordinate (C') at ($(C)+(80:5)$);
\coordinate (M) at ($(B)!1/2!(C)$);
\coordinate (N) at ($(C)!1/2!(C')$);
\coordinate (P) at ($(B')!1/2!(C')$);
\coordinate (I) at ($(A)!2/3!(M)$);
\coordinate (J) at ($(A)!2/3!(N)$);
\coordinate (K) at ($(A)!2/3!(P)$);
%\draw pic[draw,angle radius=2mm] {right angle = S--H--K}; 
%\draw pic[draw,angle radius=2mm] {right angle = H--I--K};
%\draw pic[draw,angle radius=2mm] {right angle = H--K--D};
%\coordinate (I) at (intersection of O--N and A--B) {};
\draw (A)--(B)--(C)  (A')--(B')--(C')--cycle (A)--(A') (B)--(B') (C)--(C')   ; 
\draw[dashed] (A)--(C) (A)--(M) (P)--(A)--(N) (I)--(J)--(K)--cycle ;
\fill[pattern=dots] (I)--(J)--(K)--cycle;
\foreach \p/\r in {A/180,B/-90, C/0,A'/90, B'/180, C'/-90, M/0, N/0, P/-20, I/-90, J/0, K/0}
\fill (\p) circle (1.5pt) node[shift={(\r:3mm)}]{$\p$};
\end{tikzpicture}

		}
	}
\end{ex}

%TN11
\begin{ex}%[Nguyễn Trần Phong]%[1K4GC-2]
	Cho hình lăng trụ $ABC.A'B'C'$. Gọi $M, N, P$ là 3 điểm lần lượt nằm trên ba đoạn $AB', AC', B'C$ sao cho $\dfrac{AM}{AB'}=\dfrac{C'N}{AC'}=\dfrac{CP}{CB'}=x$. Để $(MNP) \parallel (A'BC')$ thì $x$ bằng bao nhiêu?
	\choice{$x=\dfrac{1}{2}$}
	{\True $x=\dfrac{1}{3}$}
	{$x=\dfrac{2}{3}$}
	{$x=\dfrac{1}{4}$}
	\loigiai{ 
		\immini{
		Gọi $O = AB' \cap A'B$.\\
			Ta có $\dfrac{AM}{AB'}=\dfrac{CP}{CB'}=x$, suy ra $MP \parallel AC \parallel A'C'$, do đó $(MNP) \cap (ACC')$ là đường thẳng qua $N$ và song song với $A'C'$ cắt $CC', AA'$ lần lượt tại $R, S$.\\ Vậy $(MNP)$ chính là mặt phẳng $(MPRS)$.\\
			Để $(MNP) \parallel (A'BC')$ thì cần $MS \parallel A'B$.\\
			 Suy ra  $\dfrac{AM}{AO}=\dfrac{AS}{AA'}$.\\
			 Mà $\heva{&\dfrac{AM}{AO}= 2 \cdot \dfrac{AM}{AB'}= 2x\\& \dfrac{AS}{AA'}=1-x}$.\\
			 Do đó ta có $2x=1-x \Leftrightarrow x=\dfrac{1}{3}$.
			  
		}{\begin{tikzpicture}[scale=0.7, font=\footnotesize, line join=round, line cap=round, >=stealth]
%\draw[color=gray,dash pattern=on 1pt off 1pt,xstep=1.0cm,ystep=1.0cm] (-5.2,-5.2) grid (5.2,5.2);
\coordinate (A) at (0,0);
\coordinate (C) at (5,0);
\coordinate (B) at (1.5,-2);
\coordinate (A') at ($(A) +(80:5)$);
\coordinate (B') at ($(B)+(80:5)$);
\coordinate (C') at ($(C)+(80:5)$);
\coordinate (M) at ($(A)!1/3!(B')$);
\coordinate (N) at ($(C')!1/3!(A)$);
\coordinate (P) at ($(C)!1/3!(B')$);
\coordinate (S) at ($(A')!1/3!(A)$);
\coordinate (R) at ($(C')!1/3!(C)$);
\coordinate (O) at ($(A')!1/2!(B)$);
%\draw pic[draw,angle radius=2mm] {right angle = S--H--K}; 
%\draw pic[draw,angle radius=2mm] {right angle = H--I--K};
%\draw pic[draw,angle radius=2mm] {right angle = H--K--D};
%\coordinate (I) at (intersection of O--N and A--B) {};
\draw (A)--(B)--(C)  (A')--(B')--(C')--cycle (A)--(A') (B)--(B') (C)--(C')  (A)--(B') (B')--(C) (A')--(B)--(C'); 
\draw[dashed] (C')--(A)--(C) (M)--(N)--(P)--cycle (R)--(S);
\fill[pattern=dots] (M)--(P)--(R)--(S)--cycle;
\foreach \p/\r in {A/180,B/-90, C/0,A'/90, B'/90, C'/90, M/180, N/-40, P/0, S/180, R/0, O/180}
\fill (\p) circle (1.5pt) node[shift={(\r:3mm)}]{$\p$};
\end{tikzpicture}
}
	}
\end{ex}

%TN12
\begin{ex}%[Nguyễn Trần Phong]%[1K4GC-2]
Cho hình chóp $S.ABCD$ có đáy là hình bình hành tâm $O$ và $AC=a$, $BD=b$. Tam giác $SBD$ đều. Gọi $(P)$ là mặt phăng di động đi qua điểm $I$ trên đoạn $OC$, song song với $(SBD)$. Đặt $AI=x$ $\left(\dfrac{a}{2} < x <a \right)$, cắt các cạnh $BC$, $CD$, $SC$ lần lượt tại $E$, $F$, $G$. Diện tích tam giác $EFG$ bằng 
 \choice
 {$\dfrac{b^2 (a-x)^2 \sqrt{2}}{a^2} $}
 {$\dfrac{b^2 (a+x)^2\sqrt{3}}{a^2}$}
 {$\dfrac{b^2 (a+x)^2}{a^2 \sqrt{3}} $}
 {\True $ \dfrac{b^2 (a-x)^2 \sqrt{3}}{a^2}$}
 \loigiai{\immini{
 Ta có $\heva{& (P) \parallel (SBD) \\& (P) \cap (ABCD) = EF\\& (SBD) \cap (ABCD)= EF} \Rightarrow EF \parallel BD$.\\
 Suy ra $\dfrac{EF}{BD}=\dfrac{CE}{CB}=\dfrac{CF}{CD}$. \quad (*)\\
 Chứng minh tương tự ta được $\heva{& EG \parallel SB \Rightarrow  \dfrac{EG}{SB}=\dfrac{CE}{CB} \\& GF\parallel SD \Rightarrow \dfrac{GF}{SD}=\dfrac{CF}{CD}.}$  \quad (**)\\
  Từ (*) và (**) ta suy ra $\dfrac{EF}{BD}= \dfrac{EG}{SB}= \dfrac{GF}{SD}$.\\
 Lại do $SB=SD= BD$ nên $EG= GF=EF$ hay tam giác $GEF$ đều.\\
 Ta có $\dfrac{EF}{BD}=\dfrac{CI}{CO}\Rightarrow EF= \dfrac{CI}{CO} \cdot BD = \dfrac{2(a-x)}{a} \cdot b$.\\
 Vậy $S_{GEF }=\dfrac{EF^2 \sqrt{3}}{4}= \dfrac{b^2\cdot (a-x)^2 \sqrt{3}}{a^2}$.}{ \begin{tikzpicture}[line join=round,line cap=round,line width=.6pt,font=\footnotesize,scale=.9]
	\coordinate (D) at (0,0);
	\coordinate (C) at ($(D)+(0:4)$);
	\coordinate (A) at (1.5,2);
	\coordinate (B) at ($(C)-(D)+(A)$);
	\coordinate (S) at ($(A)+(80:5)$);
	\coordinate (O) at ($(A)!1/2!(C)$);
	\coordinate (I) at ($(C)!1/3!(O)$);
	\coordinate (E) at ($(C)!1/3!(B)$);
	\coordinate (F) at ($(C)!1/3!(D)$);
	\coordinate (G) at ($(C)!1/3!(S)$);
	\draw (S)--(D)--(C)--(B)--cycle (D)--(S)--(C) (F)--(G)--(E);
	\draw[dashed] (S)--(A)--(D) (A)--(B) (E)--(F) (A)--(C) (B)--(D)   ;
	\fill[pattern=dots] (G)--(E)--(F)--cycle;
\foreach \p/\r in {A/180,B/-90, C/-90, D/-90, S/90, G/70, F/-90, E/-90, I/90, O/180}
\fill (\p) circle (1.5pt) node[shift={(\r:3mm)}]{$\p$};

\end{tikzpicture}} 
 }
\end{ex}
\Closesolutionfile{ans}
%\inputansbox{10}{ans/ans-haimpsongsong}

