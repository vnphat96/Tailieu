
\begin{dang}{Tìm thiết diện của hình $\mathscr{(H)}$ khi cắt bởi mặt phẳng $(P)$}
	\phuongphap Ta tìm các đoạn giao tuyến nối tiếp nhau của mặt phẳng $(\alpha)$ với các mặt của hình chóp cho đến khi khép kín thành một đa giác phẳng. Đa giác đó là thiết diện cần tìm và các đoạn giao tuyến chính là các cạnh của thiết diện.
\end{dang}
\subsection*{VÍ DỤ MINH HỌA}
\setcounter{vd}{0}
\begin{vd}%[1K4K0-5]
	Cho tứ diện $ABCD$. Trên các đoạn $CA, CB, BD$ cho lần lươt các điểm $M, N, P$ sao cho $MN$ không song song với $AB$. Gọi $(\alpha)$ là mặt phẳng xác định bởi ba điểm $M, N, P$. Dựng thiết diện tạo bởi $(\alpha)$ và tứ diện $ABCD$.
\loigiai{
Ta có $M, N, P \in (\alpha) \Rightarrow (\alpha)\equiv(MNP)$.
\begin{itemize}
	\item Ta có $(MNP) \cap (BCD)=MP$\hfill{(1)}
	\item Tương tự $(MNP) \cap (ABC)=MN$\hfill{(2)}
	\item Xét $(MNP)$ và $(ACD)$:\\
		\immini{$\heva{&N\in AC, AC \subset (ACD) \Rightarrow N \in (ACD)\\&N \in (MNP)} \Rightarrow N \in (ACD) \cap (MNP)$.\\
		Trong $(BCD)$, gọi $E=MP \cap CD$, ta có\\ $\heva{&E\in MP ,MP \subset (MNP) \Rightarrow E \in (MNP)\\&E\in CD,CD \subset (ACD) \Rightarrow E \in (ACD)} \Rightarrow E \in (MNP) \cap (ACD)$.\\
		Suy ra $(MNP) \cap (ACD)=NP$\hfill{(3)}}
		{\begin{tikzpicture}[scale=0.8, font=\footnotesize, line join=round, line cap=round, >=stealth]
			\def\ac{4} % cạnh AC
			\def\ab{2} % cạnh AB
			\def\h{4} % chiều cao
			\def\gocB{50} % góc A của đáy
			\coordinate[label=left:$B$] (B) at (0,0);
			\coordinate[label=right:$D$] (D) at (\ac,0);
			\coordinate[label=below left:$C$] (C) at (-\gocB:\ab);
			\coordinate[label=below left:$M$] (M) at ($(B)!1/3!(C)$);
			\coordinate[label=below:$P$] (P) at ($(B)!2/3!(D)$);
			\coordinate (I) at ($(B)!1/2!(M)$);
			\coordinate[label=above:$A$] (A) at ($(I)+(90:\h)$);
				\coordinate[label=below left:$N$] (N) at ($(A)!1/3!(C)$);
			\coordinate[label=below:$E$] (E) at ($(C)!3/2!(D)$);
			\coordinate[label=above right:$F$] (F) at ($(A)!3/5!(D)$);
			\fill[lightgray,opacity=0.9] (M)--(N)--(F)--(P)--cycle;
			\draw[thick] (B)--(C)--(D)--(A)--cycle (A)--(C) (C)--(E) (N)--(E) (M)--(N);
			\draw[thick][dashed] (M)--(B) (B)--(D) (M)--(E) (F)--(P);
				\foreach \diem in {B,A,C,M,P,E,N,F}	\fill (\diem)circle(1.0pt);
			
			\end{tikzpicture}}
	\item  Trong $(ACD)$, gọi $F=NE \cap AD$. Suy ra $(MNP) \cap (ACD)=NF$\hfill{(4)}\\
Từ $(1),(2),(3),(4)$ ta có thiết diện tạo bởi $(\alpha)$ và tứ diện $ABCD$ là tứ giác $MNFP$.
\end{itemize}
	}
\end{vd}
\begin{vd}%[1K4K0-5]
	Cho hình chóp $S.ABC$. Trên cạnh $SA, SB$ lần lượt lấy $M, N$ sao cho $MN$ không song song với $AB$. Gọi $P$ là điểm thuộc miền trong tam giác $ABC$. Xác định thiết diện khi cắt hình chóp bởi mặt phẳng $(MNP)$.
	\loigiai{
		\immini{Ta có $MN=(MNP)\cap (SAB)$\hfill{(1)}\\
		Trong mp $(SAB)$: gọi $Q=MN\cap AB$.\\
		Trong mp $(ABCD)$	 kéo dài $QP$ cắt $CB$, $CA$ lần lượt tại $R, T$.\\
		Khi đó \begin{align*}
		&NR=(MNP)\cap (SBC)&\hfill{(2)}\\
		&RJ=(MNP)\cap (ABC)&\hfill{(3)}\\
		&TM=(MNP)\cap (SAC)&\hfill{(4)}
		\end{align*}
		Từ (1), (2), (3), (4) ta có thiết diện là tứ giác $MNRJ$.}
		{\begin{tikzpicture}[scale=1, font=\footnotesize, line join=round, line cap=round, >=stealth]
			\def\ac{4} % cạnh AC
			\def\ab{2} % cạnh AB
			\def\h{5} % chiều cao
			\def\gocA{50} % góc A của đáy
			\coordinate[label=left:$A$] (A) at (0,0);
			\coordinate[label=right:$C$] (C) at (\ac,0);
			\coordinate[label=below left:$B$] (B) at (-\gocA:\ab);
			\coordinate[label=below left:$Q$] (Q) at ($(A)!2!(B)$);
			\coordinate[label=above right:$J$] (J) at ($(C)!1/2!(A)$);
			\coordinate[label=above:$S$] (S) at ($(B)+(90:\h)$);
			\coordinate[label=left:$N$] (N) at ($(S)!2/3!(B)$);
			\coordinate[label=below right:$R$] (R) at ($(B)!1/3!(C)$);
					\coordinate[label=below left:$M$] (M) at ($(S)!1/2!(A)$);
			\coordinate[label=right:$P$] (P) at ($(J)!1/2!(R)$);
			\draw[thick] (S)--(A)--(B) (C)--(S) (S)--(B) (A)--(Q) (Q)--(R) (M)--(Q) (N)--(R)--(C);
			\draw[thick][dashed] (M)--(A)--(C) (M)--(J) (B)--(R)--(J) (M)--(P);
			\foreach \diem in {A,B,C,S,J,M,N,Q,R,P}	\fill (\diem)circle(1.0pt);
			\end{tikzpicture}}
	}
\end{vd}
\begin{vd}%[1K4K0-5]
	Cho hình chóp $S.ABC$. Gọi $K, N$ lần lượt là trung điểm $SA, BC$ và $M$ là điểm thuộc đoạn $SC$ sao cho $3SM=2MC$.
	\begin{enumerate}
		\item Tìm thiết diện của $(KMN)$ và hình chóp.
		\item  Mặt phẳng $(KMN)$ cắt $AB$ tại $I$. Đặt $IA=kIB$. Tìm $k$.
	\end{enumerate}
	\loigiai{
		\begin{enumerate}
			\item Tìm thiết diện của $(KMN)$ và hình chóp.\\
			\immini{	$MN=(KMN)\cap (SBC)$\hfill{(1)}\\
				$MK=(KMN)\cap (SAC)$\hfill{(2)}\\
				Trong mp $(SAC)$ gọi $J=MK\cap AC$.\\
				Trong mp $(ABC)$ gọi $I=JN\cap AB$. Khi đó\\
				$KI=(KMN)\cap (SAB)$\hfill{(3)}\\
				$IN=(KMN)\cap (ABC)$\hfill{(4)}\\
				Từ (1), (2), (3), (4) ta có thiết diện là tứ giác $MNIK$.}
			{\begin{tikzpicture}[scale=1, font=\footnotesize, line join=round, line cap=round, >=stealth]
				\def\ac{3} % cạnh AC
				\def\ab{2} % cạnh AB
				\def\h{4} % chiều cao
				\def\gocA{50} % góc A của đáy
				\coordinate[label=above left:$A$] (A) at (0,0);
				\coordinate[label=right:$C$] (C) at (\ac,0);
				\coordinate[label=below left:$B$] (B) at (-\gocA:\ab);
				\coordinate[label=above:$S$] (S) at ($(B)+(90:\h)$);
				\coordinate[label=above right:$M$] (M) at ($(S)!2/5!(C)$);
				\coordinate[label=above left:$K$] (K) at ($(A)!1/2!(S)$);
				\coordinate[label=below:$N$] (N) at ($(B)!1/2!(C)$);
				\coordinate[label=below:$J$] (J) at ($(C)!3!(A)$);
				\coordinate[label=below:$I$] (I) at ($(A)!1/3!(B)$);
				\draw[thick] (I)--(B)--(C)--(S)--cycle (S)--(B) (M)--(N) (K)--(J)--(I) (S)--(K)--(I);
				\draw[thick][dashed] (I)--(A)--(C) (K)--(N) (M)--(K)--(A)--(J) (N)--(I);
				\foreach \diem in {A,B,C,S,M,N,K,J,I}	\fill (\diem)circle(1.0pt);
				\end{tikzpicture}}
			\item  Mặt phẳng $(KMN)$ cắt $AB$ tại $I$. Đặt $IA=kIB$. Tìm $k$.\\
			Ta có $3SM=2MC \Rightarrow SM=\dfrac{2}{3}MC=\dfrac{2}{3}(SC-SM) \Rightarrow \dfrac{5}{3} \Rightarrow SM=\dfrac{2}{5}SC$.\\
			Theo định lí Meneleus trong tam giác $SAC$: $\dfrac{MS}{MC}\cdot\dfrac{KS}{KA}\cdot\dfrac{JC}{JA}=1 \Rightarrow \dfrac{JA}{JC}=\dfrac{2}{3}$.\\
			Theo định lí Meneleus trong $\triangle ABC$: $\dfrac{NC}{NB}\cdot\dfrac{IB}{IA}\cdot\dfrac{JA}{JC}=1 \Rightarrow \dfrac{IB}{IA}=\dfrac{3}{2} \Rightarrow IA=\dfrac{2}{3}IB \Rightarrow k=\dfrac{2}{3}$.
		\end{enumerate}
	}
\end{vd}
\begin{vd}%[1K4K0-5]
	Cho hình chóp $ABCD$. Trên cạnh $AB$ lấy điểm $M$, điểm $N$ trên $BC$ thỏa $BN=2NC$, $P$ là trung điểm $CD$. Xác định thiết diện khi cắt hình chóp bởi mặt phẳng $(MNP)$.
	\loigiai{
		\immini{Ta có $MN=(MNP)\cap (ABC)$\hfill{(1)}\\
		$NP=(MNP)\cap (BCD)$\hfill{(2)}\\
		Trong mặt phẳng $(BCD)$: gọi $I=NP\cap BD$.\\
		Trong mặt phẳng $(ABD)$: gọi $J=MI\cap AD \Rightarrow MJ=(MNP)\cap (ABD)$\hfill{(3)}\\
		$JP=(MNP)\cap (ACD)$\hfill{(4)}\\
		Từ (1), (2), (3), (4) ta có thiết diện là tứ giác $MNPJ$.}
		{\begin{tikzpicture}[line join=round, line cap=round,thick,scale=0.8]
			\tikzset{label style/.style={font=\footnotesize}}
			\coordinate[label=above left:$A$] (A) at (2,3);
			\coordinate[label=below left:$B$] (B) at (0,0);
			\coordinate[label=above right:$D$] (D) at (4,0);
			\coordinate[label=below:$C$] (C) at (1,-2);
			\coordinate[label=above left:$M$] (M) at ($(A)!1/2!(B)$);
			\coordinate[label=above left:$N$] (N) at ($(B)!2/3!(C)$);
			\coordinate[label=below right :$P$] (P) at ($(C)!1/2!(D)$);
			\draw[thick] (A)--(B)--(C)--(D)--cycle (M)--(N) (A)--(C);
			\draw[thick][dashed] (B)--(D) (M)--(P)--(N);
			\foreach \diem in {A,B,C,D,P,M,N}	\fill (\diem)circle(1.5pt);
			\end{tikzpicture}}
	}
\end{vd}
\begin{vd}%[1K4K0-5] CHƯA GÁN ID
	Cho hình chóp $S.ABCD$ có đáy là hình thang đáy lớn $AD$. Lấy điểm $M$ trên  cạnh $SB$. Tìm thiết diện tạo bởi mặt phẳng $(AMD)$ và hình chóp.
	\loigiai{
		\immini{ Ta có $AM=(AMD)\cap (SAB)$\hfill{(1)}\\
		$AD=(AMD)\cap (ABCD)$\hfill{(2)}\\
		Trong mặt phẳng $(SBC)$, dựng đường thẳng qua $M$ song song $BC$ và cắt $SC$ tại $N\\
		 \Rightarrow MN=(AMD)\cap (SAB)$, $ND=(AMD)\cap (SAB)$\hfill{(3)}\\
		Từ (1), (2), (3) ta có thiết diện là hình thang $ADNM$ (vì $MN\parallel BC\parallel AD$).}
		{\begin{tikzpicture}[scale=0.6,line join=round, line cap=round,thick]
			\tikzset{label style/.style={font=\footnotesize}}
			\coordinate[label=above left:$S$] (S) at (2.5,5);
			\coordinate[label=above left:$A$] (A) at (0,0);
			\coordinate[label=below left:$B$] (B) at (1,-2);
			\coordinate[label=above right:$D$] (D) at (6,0);
			\coordinate[label=below:$C$] (C) at (4,-2);
			\coordinate[label=above left:$M$] (M) at ($(S)!1/2!(B)$);
			\coordinate[label=above left:$N$] (N) at ($(S)!1/2!(C)$);
			\draw[thick] (A)--(B)--(C)--(D)--(S)--cycle (S)--(B) (S)--(C) (A)--(M)--(N)--(D);
			\draw[thick][dashed] (A)--(D) (M)--(D);
			\foreach \diem in {A,B,C,D,S,M,N}	\fill (\diem)circle(1.5pt);
			\end{tikzpicture}}
	}
\end{vd}
\subsection*{BÀI TẬP LUYỆN TẬP}
\begin{bt}%[1K4K0-5]
	Cho hình chóp $S.ABCD$ có đáy là hình bình hành. Gọi $M,\,N,\,P$ lần lượt là trung điểm của các cạnh $CB,\,CD,\,SA$. Tìm thiết diện tạo bởi mặt phẳng $\left(MNP\right)$ và hình chóp.
	\loigiai{
	\immini{Trong mặt phẳng $\left(ABC\right)$, kẻ $MN\cap AD=K$, $MN\cap AB=I$ $\Rightarrow K,\,I\in\left(MNP\right)$.\\
	Trong mặt phẳng $\left(SAD\right)$, kẻ $KP\cap SD=H \Rightarrow H\in\left(MNP\right)$.\\
	Trong mặt phẳng $\left(SAB\right)$ , kẻ $IP\cap SB=L$ $\Rightarrow L\in\left(MNP\right)$.}
		{\begin{tikzpicture}[scale=0.6,line join=round, line cap=round,thick]
			\tikzset{label style/.style={font=\footnotesize}}
			\coordinate[label=above left:$S$] (S) at (0.5,4);
			\coordinate[label=above left:$A$] (A) at (0,0);
			\coordinate[label=below right:$B$] (B) at (-3,-3);
			\coordinate[label=above right:$D$] (D) at (6,0);
			\coordinate[label=below:$C$] (C) at (3,-3);
			\coordinate[label=below right:$M$] (M) at ($(C)!1/2!(B)$);
			\coordinate[label=below right:$N$] (N) at ($(C)!1/2!(D)$);
			\coordinate[label=left:$P$] (P) at ($(S)!1/2!(A)$);
			\coordinate[label=above left:$I$] (I) at ($(A)!1.5!(B)$);
			\coordinate[label=above left:$K$] (K) at ($(A)!1.5!(D)$);
			\coordinate[label=above left:$L$] (L) at ($(B)!1/3!(S)$);
			\coordinate[label=above right:$H$] (H) at ($(D)!1/3!(S)$);
			\draw[thick] (S)--(L)--(M)--(C)--(N)--(H)--(S) (S)--(C) (S)--(C) (I)--(M) (N)--(K) (I)--(L) (H)--(K) (H)--(N) (L)--(M);
			\draw[thick][dashed] (I)--(B) (L)--(B)--(M) (D)--(K) (S)--(A)--(B) (A)--(D) (H)--(D)--(N) (P)--(M)--(N)--cycle (P)--(L) (P)--(H);
			\foreach \diem in {A,B,C,D,S,M,N,P,I,K}	\fill (\diem)circle(1.5pt);
			\end{tikzpicture}}
		\allowdisplaybreaks
		Khi đó ta có \begin{eqnarray*}
		&MN=(MNP)\cap (ABCD)&(1)\\
		&NH=(MNP)\cap (SCD)&(2)\\
		&HP=(MNP)\cap (SAD)&(3)\\
		&PL=(MNP)\cap (SAB)&(4)\\
		&LM=(MNP)\cap (SBC)&(5)
	\end{eqnarray*}
	Từ (1), (2), (3), (4), (5) ta có thiết diện là ngũ giác $MNHPL$.
		}
\end{bt}
\begin{bt}%[1K4K0-5]
	Cho hình chóp $S.ABCD$ có đáy là hình thang đáy lớn $AD$. Gọi $H, K$ là trung điểm $SB$ và $AB$, $M$ là điểm lấy trong hình thang $ABCD$ sao cho đường thẳng $KM$ cắt đường thẳng $AD$, cạnh $CD$. Tìm thiết diện của hình chóp với $(HKM)$.
	\loigiai{
		\immini{Ta có $HK=(HKM)\cap (SAB)$\hfill{(1)}\\
		Trong mặt phẳng $(ABCD)$, gọi $N, I$ lần lượt là giao điểm $KM$ với $CD, BC \Rightarrow KN=(HKM)\cap (ABCD)$\hfill{(2)}\\
		Trong $(SCD)$, gọi $L=HI\cap SC \Rightarrow NL=(HKM)\cap (SCD), LH=(HKM)\cap (SBC)$\hfill{(3)}\\
		Từ (1), (2), (3) ta có thiết diện là hình thang $KNLH$.}
		{\begin{tikzpicture}[scale=0.8, font=\footnotesize, line join=round, line cap=round, >=stealth]
				\tikzset{label style/.style={font=\footnotesize}}
				\coordinate[label=above left:$S$] (S) at (2.5,5);
				\coordinate[label=above left:$A$] (A) at (0,0);
				\coordinate[label=below left:$B$] (B) at (1,-2);
				\coordinate[label=above right:$D$] (D) at (6,0);
				\coordinate[label=below:$C$] (C) at (4,-2);
				\coordinate[label=above left:$H$] (H) at ($(S)!1/2!(B)$);
				\coordinate[label=below left:$K$] (K) at ($(A)!1/2!(B)$);
				\coordinate[label=below left:$I$] (I) at ($(B)!1.5!(C)$);
				\coordinate[label=below left:$M$] (M) at ($(I)!1/2!(K)$);
				\coordinate[label=below left:$L$] (L) at ($(S)!3/4!(C)$);
				\coordinate[label=above:$N$] (N) at ($(C)!1/8!(D)$);
				\coordinate (F) at (intersection cs:first line={(L)--(I)}, second line={(C)--(D)});
				\draw[thick] (S)--(A)--(B)--(C) (F)--(D)--(S) (S)--(B) (S)--(C) (H)--(K) (C)--(I) (I)--(H);
				\draw[thick][dashed] (A)--(D) (M)--(H) (K)--(I) (L)--(N) (C)--(F);
				\foreach \diem in {A,B,C,D,S,H,K,I,L,M,N}	\fill (\diem)circle(1.5pt);
			
				\end{tikzpicture}
		}
	}
\end{bt}
\begin{bt}%[1K4K0-5]
	Cho hình chóp $S.ABCD$ có đáy $ABCD$ là hình thang với $AB\parallel CD, AB>CD$. Gọi $I, J$ theo thứ tự là trung điểm $SB$ và $SC$.
	\begin{enumerate}
		\item Xác định giao tuyến của $(SAD)$ và $(SBC)$.
		\item Tìm giao điểm của đường thẳng $SD$ với $(AIJ)$.
		\item Xác định thiết diện của hình chóp $S.ABCD$ cắt bở mặt phẳng $(AIJ)$.
	\end{enumerate}
	\loigiai{
		\begin{enumerate}
			\item Xác định giao tuyến của $(SAD)$ và $(SBC)$.
			\immini{$S\in(SAD)\cap (SBC)$\hfill{(1)}\\
			Trong mặt phẳng $(ABCD)$, gọi $N=AD\cap BC\\
			 \Rightarrow L\in (SAD)\cap (SBC)$\hfill{(2)}\\
			Từ (1), (2) ta có $SL=(SAD)\cap (SBC)$.}
			{\begin{tikzpicture}[scale=0.6,line join=round, line cap=round,thick]
				\tikzset{label style/.style={font=\footnotesize}}
				\coordinate[label=above left:$S$] (S) at (2.5,5);
				\coordinate[label=above left:$A$] (A) at (0,0);
				\coordinate[label=below left:$D$] (D) at (1,-2);
				\coordinate[label=above right:$B$] (B) at (6,0);
				\coordinate[label=below:$C$] (C) at (4,-2);
				\coordinate[label=above right:$I$] (I) at ($(S)!1/2!(B)$);
				\coordinate[label=below right:$J$] (J) at ($(S)!1/2!(C)$);
				\coordinate[label=below right:$L$] (L) at ($(A)!2!(D)$);
				\coordinate[label=below right:$K$] (K) at ($(S)!1/2!(L)$);
				\coordinate[label=above left:$M$] (M) at ($(S)!2/3!(D)$);
				\draw[thick] (S)--(A)--(D)(C)--(B)--(S)--cycle (S)--(D) (S)--(C) (I)--(J) (C)--(L)--(D) (S)--(L) (J)--(K)--(A);
				\draw[thick][dashed] (A)--(B) (A)--(I) (M)--(J) (D)--(C);
				\foreach \diem in {A,B,C,D,S,I,J,L,K,M}	\fill (\diem)circle(1.5pt);
				\end{tikzpicture}}
			\item Tìm giao điểm của đường thẳng $SD$ với $(AIJ)$.\\
				Trong mặt phẳng $(SBC)$, gọi $K=SL\cap IJ$.\\
				Trong mặt phẳng $(SAD)$, gọi $M=SD\cap AK \Rightarrow M\in SD\cap (AIJ)$.
			\item Xác định thiết diện của hình chóp $S.ABCD$ cắt bở mặt phẳng $(AIJ)$.\\
				Ta có $AM=(AIJ)\cap (SAD)$; 
				$MJ=(AIJ)\cap (SCD)$; 
				$IJ=(AIJ)\cap (SBC)$; 
				$IA=(AIJ)\cap (SAB)$.\\
				Vậy thiết diện là tứ giác $AMJI$.
		\end{enumerate}
	}
\end{bt}

\begin{bt}%Bài 1%[1K4K0-5]
	Cho hình chóp S.ABCD. Lấy một điểm $M$ thuộc miền trong tam giác $S B C$. Lấy một điểm $N$ thuộc miền trong tam giác $S C D$.
	\begin{enumerate}[a)]
		\item  Tìm giao điểm của $MN$ với $(SAC)$.
		\item  Tìm giao điểm của $SC$ với $(AMN)$.
		\item  Tìm thiết diện của hình chóp $S.ABCD$ với $(AMN)$.
	\end{enumerate}
	\loigiai{ 
		\immini{\begin{enumerate}[a)]
				\item \textit{Tìm giao điểm của $MN$ với $(SAC)$}.\\Gọi $E$ là giao điểm của đường thẳng $SM$ và cạnh $BC$, $F$ là giao điểm của đường thẳng $SN$ và cạnh $CD$ và gọi $O$ là giao điểm của $EF$ và $AC$.
				\begin{itemize}
					\item  Chọn mặt phẳng phụ chứa $MN$ là $(SEF)$.
					\item $(SAC)\cap (SEF)=SO$.
					\item Trong mặt phẳng $(SEF)$ hai đường thẳng $MN$ và $SO$ phải cắt nhau, gọi giao điểm này là $I$ thì $I$ chính là giao điểm của $MN$ với $(SAC)$.
				\end{itemize}
				\item \textit{Tìm giao điểm của $SC$ với $(AMN)$.}
				\begin{itemize}
					\item  Chọn mặt phẳng phụ chứa $SC$ là $(AMN)$.
					\item $(SAC)\cap (AMN)=AI$.
					\item Trong mặt phẳng $(SAC)$ hai đường thẳng $AI$ và $SC$ phải cắt nhau, gọi giao điểm này là $J$ thì $J$ chính là giao điểm của $SC$ với $(AMN)$.
				\end{itemize}
		\end{enumerate}}
		{\begin{tikzpicture}[scale=1, font=\footnotesize, line join=round, line cap=round, >=stealth]
				\def\ad{6} % cạnh AD
				\def\ab{2} % cạnh AB
				\def\bc{3.5} % chéo AC
				\def\as{4} % cạnh AS
				\def\gocA{50} % góc A của đáy
				\def\gocB{125} % góc B của đáy
				\coordinate (A) at (0,0);
				\coordinate(B) at (-\gocA:\ab);
				\coordinate (C) at ($(B)+(180-\gocA-\gocB:\bc)$);
				\coordinate (D) at (\ad,0);
				\coordinate (S) at (60:\as); % chỉnh 75 và as để thay đổi S
				\coordinate(E) at ($(B)!0.3!(C)$);
				\coordinate (M) at ($(S)!0.75!(E)$);	
				\coordinate (F) at ($(D)!0.4!(C)$);
				\coordinate (N) at ($(S)!0.5!(F)$);	
				\coordinate (O) at (intersection cs:first line={(E)--(F)}, second line={(C)--(A)});
				\coordinate (I) at (intersection cs:first line={(M)--(N)}, second line={(S)--(O)});				
				\coordinate(J) at (intersection cs:first line={(A)--(I)}, second line={(C)--(S)});
				\coordinate(P) at (intersection cs:first line={(J)--(N)}, second line={(S)--(D)});
				\coordinate (Q) at (intersection cs:first line={(J)--(M)}, second line={(S)--(B)});
				
				\draw[thick] (A)--(B)--(C)--(D)--(S)--cycle (B)--(S)--(C)
				(S)--(E)
				(S)--(F) (Q)--(B);
				\draw[thick][dashed] (A)--(D) (A)--(C) (E)--(F) (S)--(O)
				(A)--(M)--(N)--(A)--(J);
				%	\foreach \diem in {A,B,C,D,S,E,M,N,F,I,O}	\fill (\diem)circle(1.0pt);
				\foreach \i/\j in {A/150,B/-35,C/-45,D/0,E/-90,F/-50,I/-110,O/45,M/-30,N/-5,S/90,J/0}\fill[black] (\i) circle (1pt) ($(\i)+(\j:2.5mm)$)node{$\i$};
				%	\fill[cyan,opacity=0.5] (A)--(Q)--(J)--(P)--cycle;
				\fill[pattern=dots] (A)--(N)--(M)--cycle;	
			\end{tikzpicture}
		}
		\immini{c) \textit{Tìm thiết diện của hình chóp $S.ABCD$ với $(AMN)$.}\\
			\textbf{Trường hợp 1}: Đường thẳng $MJ$ cắt cạnh $SB$ tại $Q$ và đường thẳng $JN$ cắt cạnh $SD$ tại $P$.\\ Ta thấy thiết diện là tứ giác $AQJP$.}
		{\begin{tikzpicture}[scale=1, font=\footnotesize, line join=round, line cap=round, >=stealth]
				\def\ad{6} % cạnh AD
				\def\ab{2} % cạnh AB
				\def\bc{3.5} % chéo AC
				\def\as{4} % cạnh AS
				\def\gocA{50} % góc A của đáy
				\def\gocB{125} % góc B của đáy
				\coordinate (A) at (0,0);
				\coordinate(B) at (-\gocA:\ab);
				\coordinate (C) at ($(B)+(180-\gocA-\gocB:\bc)$);
				\coordinate (D) at (\ad,0);
				\coordinate (S) at (60:\as); % chỉnh 75 và as để thay đổi S
				\coordinate(E) at ($(B)!0.3!(C)$);
				\coordinate (M) at ($(S)!0.75!(E)$);	
				\coordinate (F) at ($(D)!0.4!(C)$);
				\coordinate (N) at ($(S)!0.5!(F)$);	
				\coordinate (O) at (intersection cs:first line={(E)--(F)}, second line={(C)--(A)});
				\coordinate (I) at (intersection cs:first line={(M)--(N)}, second line={(S)--(O)});				
				\coordinate(J) at (intersection cs:first line={(A)--(I)}, second line={(C)--(S)});
				\coordinate(P) at (intersection cs:first line={(J)--(N)}, second line={(S)--(D)});
				\coordinate (Q) at (intersection cs:first line={(J)--(M)}, second line={(S)--(B)});
				
				\draw[thick] (A)--(B)--(C)--(D)--(S)--cycle (B)--(S)--(C)
				(S)--(E)
				(S)--(F) (J)--(P) (J)--(Q)--(B) (A)--(Q);
				\draw[thick][dashed] (A)--(D) (A)--(C) (E)--(F) (S)--(O)
				(A)--(M)--(N)--(A)--(J) (P)--(A);
				%	\foreach \diem in {A,B,C,D,S,E,M,N,F,I,O}	\fill (\diem)circle(1.0pt);
				\foreach \i/\j in {A/150,B/-35,C/-45,D/0,E/-90,F/-50,I/150,O/45,M/-30,N/-5,S/90,J/0,P/0,Q/0}\fill[black] (\i) circle (1pt) ($(\i)+(\j:2.5mm)$)node{$\i$};
				\fill[cyan,opacity=0.5] (A)--(Q)--(J)--(P)--cycle;
				\fill[pattern=dots] (A)--(N)--(M)--cycle;				
		\end{tikzpicture}}	
		\immini{\textbf{Trường hợp 2}: Đường thẳng $MJ$ cắt cạnh $BC$ tại $Q$ và đường thẳng $JN$ cắt cạnh $SD$ tại $P$.\\
			Ta thấy thiết diện là tứ giác $AQJP$.}
		{\begin{tikzpicture}[scale=1, font=\footnotesize, line join=round, line cap=round, >=stealth]
				\def\ad{6} % cạnh AD
				\def\ab{2} % cạnh AB
				\def\bc{3.5} % chéo AC
				\def\as{4} % cạnh AS
				\def\gocA{50} % góc A của đáy
				\def\gocB{125} % góc B của đáy
				\coordinate (A) at (0,0);
				\coordinate(B) at (-\gocA:\ab);
				\coordinate (C) at ($(B)+(180-\gocA-\gocB:\bc)$);
				\coordinate (D) at (\ad,0);
				\coordinate (S) at (60:\as); % chỉnh 75 và as để thay đổi S
				\coordinate(E) at ($(B)!0.6!(C)$);
				\coordinate (M) at ($(S)!0.8!(E)$);	
				\coordinate (F) at ($(D)!0.6!(C)$);
				\coordinate (N) at ($(S)!0.4!(F)$);	
				\coordinate (O) at (intersection cs:first line={(E)--(F)}, second line={(C)--(A)});
				\coordinate (I) at (intersection cs:first line={(M)--(N)}, second line={(S)--(O)});				
				\coordinate(J) at (intersection cs:first line={(A)--(I)}, second line={(C)--(S)});
				\coordinate(P) at (intersection cs:first line={(J)--(N)}, second line={(S)--(D)});
				\coordinate (Q) at (intersection cs:first line={(J)--(M)}, second line={(C)--(B)});
				
				\draw[thick] (A)--(B)--(C)--(D)--(S)--cycle (B)--(S)--(C)
				(S)--(E)
				(S)--(F) (J)--(P) (J)--(Q)--(B);
				\draw[thick][dashed] (A)--(D) (A)--(C) (E)--(F) (S)--(O)
				(A)--(M)--(N)--(A)--(J) (P)--(A) (A)--(Q);
				%	\foreach \diem in {A,B,C,D,S,E,M,N,F,I,O}	\fill (\diem)circle(1.0pt);
				\foreach \i/\j in {A/150,B/-35,C/-45,D/0,E/-90,I/150,O/45,M/-30,N/-5,S/90,J/0,P/0,Q/-90,F/0}\fill[black] (\i) circle (1pt) ($(\i)+(\j:2.5mm)$)node{$\i$};
				\fill[cyan,opacity=0.5] (A)--(Q)--(J)--(P)--cycle;
				\fill[pattern=dots] (A)--(N)--(M)--cycle;			
		\end{tikzpicture}}	
		\immini{\textbf{Trường hợp 3}: Đường thẳng $MJ$ cắt cạnh $SB$ tại $Q$ và đường thẳng $JN$ cắt cạnh $CD$ tại $P$.\\
			Ta thấy thiết diện là tứ giác $AQJP$.}
		{\begin{tikzpicture}[scale=0.8, font=\footnotesize, line join=round, line cap=round, >=stealth]
				\def\ad{6} % cạnh AD
				\def\ab{2} % cạnh AB
				\def\bc{3.5} % chéo AC
				\def\as{4} % cạnh AS
				\def\gocA{50} % góc A của đáy
				\def\gocB{125} % góc B của đáy
				\coordinate (A) at (0,0);
				\coordinate(B) at (-\gocA:\ab);
				\coordinate (C) at ($(B)+(180-\gocA-\gocB:\bc)$);
				\coordinate (D) at (\ad,0);
				\coordinate (S) at (60:\as); % chỉnh 75 và as để thay đổi S
				\coordinate(E) at ($(B)!0.4!(C)$);
				\coordinate (M) at ($(S)!0.34!(E)$);	
				\coordinate (F) at ($(D)!0.6!(C)$);
				\coordinate (N) at ($(S)!0.75!(F)$);	
				\coordinate (O) at (intersection cs:first line={(E)--(F)}, second line={(C)--(A)});
				\coordinate (I) at (intersection cs:first line={(M)--(N)}, second line={(S)--(O)});				
				\coordinate(J) at (intersection cs:first line={(A)--(I)}, second line={(C)--(S)});
				\coordinate(P) at (intersection cs:first line={(J)--(N)}, second line={(C)--(D)});
				\coordinate(Q) at (intersection cs:first line={(J)--(M)}, second line={(S)--(B)});
				
				\draw[thick] (A)--(B)--(C)--(D)--(S)--cycle (B)--(S)--(C)
				(S)--(E) (J)--(P)
				(S)--(F) (A)--(Q)--(J);
				\draw[thick][dashed] (A)--(D) (A)--(C) (E)--(F) (S)--(O)
				(A)--(M)--(N)--(A)--(J) (A)--(P);
				%	\foreach \diem in {A,B,C,D,S,E,M,N,F,I,O}	\fill (\diem)circle(1.0pt);
				\foreach \i/\j in {A/150,B/-35,C/-45,D/0,E/-90,F/-50,I/-100,O/45,M/40,N/80,S/90,J/80,P/0,Q/120}\fill[black] (\i) circle (1pt) ($(\i)+(\j:2.5mm)$)node{$\i$};
				\fill[cyan,opacity=0.5] (A)--(Q)--(J)--(P)--cycle;
				\fill[pattern=dots] (A)--(N)--(M)--cycle;
	\end{tikzpicture}}}
\end{bt}
\begin{bt}%Bài 2%[1K4K0-5]
	Cho hình chóp $S.ABCD$ có đáy $ABCD$ là hình bình hành tâm $O$. Gọi $M$ là trung điểm của $SB$, $G$ là trọng tâm tam giác $SAD$.
	\begin{enumerate}
		\item Tìm giao điểm $I$ của $GM$ với $(ABCD)$. Chứng minh $IC=2ID$.
		\item Tìm giao điểm $J$ của $(OMG)$ với $AD$. Đặt $JA=k \cdot JD$. Tìm $k$.
		\item Tìm giao điểm $K$ của $(OMG)$ với $SA$. Đặt $KA=p \cdot KS$. Tìm $p$.
		\item Tìm thiết diện tạo bởi $(OMG)$ với hình chóp $S.ABCD$.
	\end{enumerate}
	\loigiai{
		a) Tìm giao điểm $I$ của $GM$ với $(ABCD)$. Chứng minh $I C=2ID$.\\Gọi $N$ là trung điểm của $AD$.
		\begin{itemize}
			\item  Chọn mặt phẳng phụ chứa $MG$ là $(SBN)$.
			\item $(SBN)\cap (ABCD)=BN$.
			\begin{itemize}
				\item $MG$ và $BN$ đồng phẳng.
				\item $\dfrac{SM}{SB}=\dfrac{1}{2}\neq \dfrac{SG}{GN}=\dfrac{2}{3}.$
			\end{itemize}			 
		\end{itemize}
		Suy ra hai đường thẳng $MG$ và $BN$ phải cắt nhau. Đó chính là giao điểm $I$ của $MG$ với $(ABCD)$.\\
		Áp dụng định lí Menelaus cho tam giác $SBN$ với ba điểm thẳng hàng $M, G, I$ ta có
		$$\dfrac{SM}{MB}\cdot\dfrac{BI}{IN}\cdot\dfrac{NG}{GS}=1\Leftrightarrow \dfrac{BI}{IN}=2\Leftrightarrow BI=2BN$$
		Suy ra $N$ là trung điểm của $BE$.\\
		Gọi $E$ là điểm đối xứng với $C$ qua $D$ thì $EC=2ED$ đồng thời $N$ là trung điểm của $BE$.\\
		Do đó, hai điểm $I$ và $E$ trùng nhau. Vậy ta có $IC=2ID$.\\
		\immini{
			b) \textit{Tìm giao điểm $J$ của $(OMG)$ với $AD$.\\ Đặt $JA=k \cdot JD$. Tìm $k$.}
			\begin{itemize}
				\item  Chọn mặt phẳng phụ chứa $AD$ là $(ABCD)$.
				\item $(ABCD)\cap (OMG)=OI$.
				\item Trong mặt phẳng $(ABCD)$ hai đoạn thẳng $AD$ và $OI$ phải cắt nhau, đó là giao điểm $J$ của $AD$ với $(OMG)$.
				\item Tam giác $ACI$ có $IO$ và $AD$ là hai đường trung tuyến, $J$ là giao điểm của hai đoạn thẳng này nên $J$ là trọng tâm. Suy ra $JA=2JD$, hay $k=2$.
		\end{itemize}}
		{\begin{tikzpicture}[scale=0.8, font=\footnotesize, line join=round, line cap=round, >=stealth]
				\def\a{2.5} \def\b{1.2} \def\c{5} \def\d{3.5} \def\e{6}
				\path (0,0) coordinate (B)
				(\a,\b) coordinate (A)
				(\c,0) coordinate (C)
				(\d,\e) coordinate (S)
				($(A)+(C)-(B)$) coordinate (D)
				($(S)!.5!(B)$) coordinate (M)	
				($(A)!.5!(D)$) coordinate (N)	
				($(S)!.(2/3)!(N)$) coordinate (G)	
				($(A)!.5!(C)$) coordinate (O); 
				\coordinate(I) at (intersection cs:first line={(B)--(N)}, second line={(M)--(G)});
				\coordinate(P) at (intersection cs:first line={(G)--(I)}, second line={(S)--(D)});
				\coordinate(Q) at (intersection cs:first line={(N)--(I)}, second line={(S)--(D)});
				\coordinate(J) at (intersection cs:first line={(O)--(I)}, second line={(A)--(D)});
				\coordinate (K) at (intersection cs:first line={(A)--(S)}, second line={(G)--(J)});
				
				\draw[thick] (C)--(S)--(B)--(C)--(D)--(S) (D)--(I)--(S) ;
				\draw[thick][dashed,thick](S)--(A) (B)--(A)--(D) (S)--(N) (M)--(I) (B)--(I) (A)--(C) (M)--(O)--(I) (O)--(G) (J)--(K) (A)--(I);
				\foreach \i/\j in {A/150,B/-135,C/-45,D/-20,M/150,S/90,N/70,G/70,I/70,O/230,J/-90, K/195}
				\fill[black] (\i) circle (1pt) ($(\i)+(\j:3mm)$)node{$\i$};
				%	\fill[cyan,opacity=0.5] (M)--(K)--(I)--(O)--cycle;}
		\end{tikzpicture}}
		\vspace{0.5cm}
		\noindent c) \textit{Tìm giao điểm $K$ của $(OMG)$ với $SA$. Đặt $KA=p \cdot KS$. Tìm $p$.}
		\begin{itemize}			
			\item  Chọn mặt phẳng phụ chứa $SA$ là $(SAD)$.
			\item $(SAD)\cap (OMG)=GJ$.
			\item Trong mặt phẳng $(SAD)$ hai đường thẳng $SA$ và $GJ$ phải cắt nhau, đó là giao điểm $K$ của $SA$ với $(OMG)$.
			\item 	Áp dụng định lí Menelaus cho tam giác $SAN$ với ba điểm thẳng hàng $K, G, J$ ta có
			$$\dfrac{SK}{KA}\cdot\dfrac{AJ}{JN}\cdot\dfrac{NG}{GS}=1\Leftrightarrow \dfrac{SK}{KA}=\dfrac{1}{2}.$$
		\end{itemize}
	}
	\immini{\textit{c) Tìm thiết diện tạo bởi $(OMG)$ với hình chóp $S.ABCD$}.\\ Gọi $Q$ là giao điểm của đường thẳng $OI$ và cạnh $BC$. Thiết diện mà mặt phẳng $(OMG)$ cắt hình chóp $S.ABCD$ là tứ giác $QMKJ$}
	{\begin{tikzpicture}[scale=1, font=\footnotesize, line join=round, line cap=round, >=stealth]
			\def\a{2.5} \def\b{1.2} \def\c{5} \def\d{3.5} \def\e{6}
			\path (0,0) coordinate (B)
			(\a,\b) coordinate (A)
			(\c,0) coordinate (C)
			(\d,\e) coordinate (S)
			($(A)+(C)-(B)$) coordinate (D)
			($(S)!.5!(B)$) coordinate (M)	
			($(A)!.5!(D)$) coordinate (N)	
			($(S)!.(2/3)!(N)$) coordinate (G)	
			($(A)!.5!(C)$) coordinate (O)
			; 
			\coordinate(I) at (intersection cs:first line={(B)--(N)}, second line={(M)--(G)});
			\coordinate(P) at (intersection cs:first line={(G)--(I)}, second line={(S)--(D)});
			\coordinate(Q) at (intersection cs:first line={(N)--(I)}, second line={(S)--(D)});
			\coordinate(J) at (intersection cs:first line={(O)--(I)}, second line={(A)--(D)});
			\coordinate (K) at (intersection cs:first line={(A)--(S)}, second line={(G)--(J)});
			\coordinate(Q) at (intersection cs:first line={(O)--(I)}, second line={(B)--(C)});
			\draw[thick] (C)--(S)--(B)--(C)--(D)--(S) (D)--(I)--(S) (M)--(Q) ;
			\draw[thick][thick][dashed,thick](S)--(A) (B)--(A)--(D) (S)--(N) (M)--(I) (B)--(I) (A)--(C) (M)--(O)--(I)--(K) (O)--(G) (M)--(K)--(J) (I)--(A) (O)--(Q);
			\foreach \i/\j in {A/150,B/-135,C/-45,D/0,M/150,S/90,N/70,G/70,I/70,O/230,J/-90, Q/-90,K/45}
			\fill[black] (\i) circle (1pt) ($(\i)+(\j:3mm)$)node{$\i$};
			\fill[cyan,opacity=0.5] (Q)--(M)--(K)--(J)--cycle;
			\fill[pattern=dots] (O)--(G)--(M)--cycle;	
		\end{tikzpicture} 
	}
\end{bt}
\begin{bt}%Bài 1%[1K4K0-5]
	Cho hình chóp $S.ABCD$ có đáy $ABCD$ là hình bình hành tâm $O$. Gọi $M$, $N$, $P$ lần lượt là trung điểm của $SB$, $SD$ và $OC$.
	\begin{enumerate}[a)]
		\item  Tìm giao tuyến của $(MNP)$ với $(SAC)$ và $(ABCD)$.
		\item  Tìm giao điểm của $SA$ và $(MNP)$.
		\item  Xác định thiết diện của hình chóp với $(MNP)$. Tính tỉ số mà $(MNP)$ chia các cạnh $SA$, $BC$ và $CD$. 
	\end{enumerate}
	\loigiai{
		\immini{\begin{enumerate}[a)]
				\item  \textit{Tìm giao tuyến của $(MNP)$ và $(SAC)$.}
				\begin{itemize}
					\item $P$ là điểm chung thứ nhất của $(MNP)$ và $(SAC)$.
					\item Trong tam giác $SAC$, hai đoạn thẳng $MN$ và $SO$ phải cắt nhau. Gọi giao điểm này là $I$ thì $I$ là điểm chung thứ hai của $(MNP)$ và $(SAC)$.			
				\end{itemize}
				Vậy $(MNP) \cap (SAC)=PI$.		
		\end{enumerate}}
		{\begin{tikzpicture}[scale=1, font=\footnotesize, line join=round, line cap=round, >=stealth]
				\def\a{2.9} \def\b{1.2} \def\c{6} \def\d{3.9} \def\e{6}
				\path (0,0) coordinate (B)
				(\a,\b) coordinate (A)
				(\c,0) coordinate (C)
				(\d,\e) coordinate (S)
				($(A)+(C)-(B)$) coordinate (D)	
				($(S)!.5!(B)$) coordinate (M)
				($(S)!.5!(D)$) coordinate (N)
				($(A)!.5!(C)$) coordinate (O)
				($(O)!.5!(C)$) coordinate (P)
				($(D)!.5!(C)$) coordinate (Q)
				($(B)!.5!(C)$) coordinate (T)
				; 
				\coordinate(I) at (intersection cs:first line={(M)--(N)}, second line={(S)--(O)});
				\coordinate(R) at (intersection cs:first line={(P)--(I)}, second line={(S)--(A)});
				\draw[thick] (C)--(S)--(B)--(C)--(D)--(S) (M)--(T) (Q)--(N) ;
				\draw[thick][dashed,thick](S)--(A) (B)--(A)--(D) (A)--(C) (B)--(D) (M)--(N)--(P)--(M) (S)--(O) (P)--(R) (M)--(R)--(N) (T)--(Q)node[pos=.70,above]{$d$};
				\foreach \i/\j in {A/45,B/-135,C/-45,D/0,N/45,M/180,S/90, O/-90, R/170,T/-90,Q/-90, P/-140,I/45}\fill[black] (\i) circle (1pt) ($(\i)+(\j:3mm)$)node{$\i$};
				\fill[cyan,opacity=0.5] (M)--(T)--(Q)--(N)--(R)--cycle;
				\fill[pattern=dots] (P)--(N)--(M)--cycle;
		\end{tikzpicture}}
		\begin{enumerate}[b)]
			\item[$*$] \textit{Tìm giao tuyến của  $(MNP)$ và $(ABCD)$}
			\begin{itemize}
				\item $MN$ là đường trung bình của tam giác $SBD$ nên $MN \parallel BD$.
				\item Hai mặt phẳng $(MNP)$, $(ABCD)$ phân biệt, có điểm chung $P$ và lần lượt chứa hai đường thẳng song song $MN$, $BD$ nên chúng cắt nhau theo một giao tuyến $d$ qua $P$ và $d$ song song với $BD$.
			\end{itemize}
			\item \textit{ Tìm giao điểm của $SA$ và $(MNP)$.}
			\begin{itemize}
				\item  Chọn mặt phẳng phụ chứa $SA$ là $(SAC)$.
				\item $(SAC)\cap (MNP)=PI$.
				\item Trong mặt phẳng $(SAC)$ hai đường thẳng $PI$ và $SA$ phải cắt nhau, gọi giao điểm này là $R$ thì $R$ chính là giao điểm của $SA$ với $(MNP)$.
			\end{itemize}
			\item[$\text{c})$]  \textit{Xác định thiết diện của hình chóp với $(MNP)$. Tính tỉ số mà $(MNP)$ chia các cạnh $SA$, $BC$ và $CD$.}
			\begin{itemize}
				\item  Gọi $T=d\cap BC$, $Q=d\cap CD$. Ta thấy, thiết diện mà mặt phẳng $(MNP)$ cắt hình chóp $S.ABCD$ là ngũ giác $MTQNR$.
				\item 	Áp dụng định lí Menelaus cho tam giác $SAO$ với ba điểm thẳng hàng $R, I, P$ ta có
				$$\dfrac{SR}{RA}\cdot\dfrac{AP}{PO}\cdot\dfrac{OI}{IS}=1\Leftrightarrow \dfrac{SR}{RA}=\dfrac{1}{3}.$$
				\item $\heva{&d \parallel BD \\
					& P\;\text{là trung điểm của} \;OC}$ $\Rightarrow$ $\heva{&T\;\text{là trung điểm của}\;CB \\& Q\;\text{là trung điểm của}\;CD}\Rightarrow \dfrac{CT}{TB}=\dfrac{CQ}{QD}=1$.
			\end{itemize}
			Vậy các tỉ số cần tìm là  $\dfrac{SR}{RA}=\dfrac{1}{3}, \dfrac{CT}{TB}=\dfrac{CQ}{QD}=1.$
	\end{enumerate}}
\end{bt}
\begin{bt}%Bài 1%[1K4K0-5]
	Cho hình chóp $S.ABCD$ có đáy $ABCD$ là hình bình hành. Gọi $K$ là trọng tâm của tam giác $S AC$ và $I$, $J$ lần lượt là trung điểm của $CD$ và $SD$.
	\begin{enumerate}[a)]
		\item  Tìm giao điểm $H$ của đường thẳng $IK$ với mặt phẳng $(SAB)$.
		\item  Xác định thiết diện tạo bởi mặt phẳng $(IJK)$ với hình chóp.
	\end{enumerate}
	\loigiai{
		a) Tìm giao điểm $H$ của đường thẳng $IK$ với mặt phẳng $(SAB)$.	
		\immini{\begin{itemize}
				\item  Gọi $O$ là tâm hình bình hành $ABCD$. Chọn mặt phẳng phụ chứa $IK$ là $(SOI)$.
				\item Gọi $E$ là trung điểm của $AB$ thì $(ABCD)\cap (OMG)=OI$.
				\item Trong mặt phẳng $(SOI)$ hai đường thẳng $IK$ và $SE$ phải cắt nhau, đó là giao điểm $H$ của $IK$ với $(SAB)$.		
		\end{itemize}}
		{\begin{tikzpicture}[scale=1, font=\footnotesize, line join=round, line cap=round, >=stealth]
				\def\a{2.8} \def\b{1.2} \def\c{6} \def\d{3.5} \def\e{6}
				\pgfmathsetmacro\k{2/3}
				\path (0,0) coordinate (B)
				(\a,\b) coordinate (A)
				(\c,0) coordinate (C)
				(\d,\e) coordinate (S)
				($(A)+(C)-(B)$) coordinate (D)	
				($(A)!.5!(C)$) coordinate (O)
				($(S)!\k!(O)$) coordinate (K)
				($(D)!.5!(C)$) coordinate (I)
				($(A)!.5!(B)$) coordinate (E)
				($(S)!.5!(D)$) coordinate (J)
				($(S)!.5!(E)$) coordinate (H)
				; 
				\coordinate (R) at (intersection cs:first line={(H)--(B)}, second line={(S)--(A)});
				\draw[thick] (C)--(S)--(B)--(C)--(D)--(S)--(I) ;
				\draw[thick][dashed,thick](S)--(A) (B)--(A)--(D) (A)--(C) (B)--(D)  (S)--(O) (I)--(E)--(S) (I)--(H);
				\foreach \i/\j in {A/45,B/-135,C/-45,D/0,S/90, O/50,K/-115,I/-30,E/-90,H/170}
				\fill[black] (\i) circle (1pt) ($(\i)+(\j:2.7mm)$)node{$\i$};
				%	\fill[cyan,opacity=0.5] (B)--(R)--(J)--(I)--cycle;
				\fill[pattern=dots] (S)--(E)--(I)--cycle;
		\end{tikzpicture}}
		\immini{
			\begin{itemize}
				\item Đoạn $SO$ là đường trung tuyến tam giác $SAC$ và của tam giác $SBD$ nên $K$ cũng là trọng tâm của tam giác $SAD$. Suy ra ba điểm $B, K, J$ thẳng hàng.
				\item Trong mặt phẳng $(SAB)$, hai đường thẳng hai đường thẳng $BH$ và $SA$ phải cắt nhau, gọi $R$ là giao điểm.
				\item Thiết diện cần tìm là tứ giác $BIJR$. 
		\end{itemize}}
		{\begin{tikzpicture}[scale=1, font=\footnotesize, line join=round, line cap=round, >=stealth]
				\def\a{2.8} \def\b{1.2} \def\c{6} \def\d{3.5} \def\e{6}
				\pgfmathsetmacro\k{2/3}
				\path (0,0) coordinate (B)
				(\a,\b) coordinate (A)
				(\c,0) coordinate (C)
				(\d,\e) coordinate (S)
				($(A)+(C)-(B)$) coordinate (D)	
				($(A)!.5!(C)$) coordinate (O)
				($(S)!\k!(O)$) coordinate (K)
				($(D)!.5!(C)$) coordinate (I)
				($(A)!.5!(B)$) coordinate (E)
				($(S)!.5!(D)$) coordinate (J)
				($(S)!.5!(E)$) coordinate (H)
				; 
				\coordinate (R) at (intersection cs:first line={(H)--(B)}, second line={(S)--(A)});
				\draw[thick] (C)--(S)--(B)--(C)--(D)--(S)--(I)--(J) ;
				\draw[thick][dashed,thick](S)--(A) (B)--(A)--(D) (A)--(C) (B)--(D)  (S)--(O) (I)--(E)--(S) (I)--(H)--(J)--(B)--(R)--(J) (B)--(I);
				\foreach \i/\j in {A/45,B/-135,C/-45,D/0,S/90, R/20, O/50,K/-115,I/-30,J/0,E/-90,H/170}
				\fill[black] (\i) circle (1pt) ($(\i)+(\j:2.7mm)$)node{$\i$};
				\fill[cyan,opacity=0.5] (B)--(R)--(J)--(I)--cycle;
				%	\fill[pattern=dots] (P)--(N)--(M)--cycle;
		\end{tikzpicture}}
	}
\end{bt}
\begin{bt}%Bài 1%[1K4K0-5]
	Hình chóp $S.ABCD$ có đáy $ABCD$ không là hình thang, điểm $P$ nằm trong tam giác $SAB$ và điểm $M$ thuộc cạnh $SD$ sao cho $MD=2MS$.
	\begin{enumerate}[a)]
		\item  Tìm giao tuyến của hai mặt phẳng $(SAB)$ và $(PCD)$.
		\item  Tìm giao điểm của $SC$ với mặt phẳng $(ABM)$.
		\item  Gọi $N$ là trung điểm của $AD .$ Tìm thiết diện tạo bởi $(MNP)$ và hình chóp.
	\end{enumerate}
	\loigiai{
		\immini{\begin{enumerate}[a)]
				\item  \textit{Tìm giao tuyến của hai mặt phẳng $(SAB)$ và $(PCD)$.}
				\begin{itemize}
					\item $P$ là điểm chung thứ nhất của $(SAB)$ và $(PCD)$.
					\item  Theo giả thiết, tứ giác $ABCD$ không phải là hình thang nên hai đường thẳng $AB$ và $CD$ cắt nhau. Gọi $O$ là giao điểm của $AB$ và $CD$ thì $O$ là điểm chung thứ hai của $(SAB)$ và $(PCD)$.
					\item Vậy $(SAB)\cap (PCD)=PO$.		
				\end{itemize}
				\item  \textit{Tìm giao điểm của $SC$ với mặt phẳng $(ABM)$.}
				\begin{itemize}
					\item  Chọn mặt phẳng phụ chứa $SC$ là $(SCD)$.
					\item $(SCD)\cap (ABM)=OM$.
					\item Trong tam giác $SCD$, hai đoạn thẳng $OM$ và $SC$ phải cắt nhau, gọi giao điểm là $R$ thì đó là giao điểm của $SC$ và $(ABM)$.	
				\end{itemize}
			\end{enumerate}
		}
		{\begin{tikzpicture}[scale=1, font=\footnotesize, line join=round, line cap=round, >=stealth]
				\def\ad{7} % cạnh AD
				\def\ab{2.5} % cạnh AB
				\def\bc{3} % chéo AC
				\def\as{4} % cạnh AS
				\def\gocA{55} % góc A của đáy
				\def\gocB{120} % góc B của đáy
				\pgfmathsetmacro\k{1/3}
				\coordinate (A) at (0,0);
				\coordinate(B) at (-\gocA:\ab);
				\coordinate (C) at ($(B)+(180-\gocA-\gocB:\bc)$);
				\coordinate (D) at (\ad,0);
				\coordinate (S) at (60:\as); % chỉnh 75 và as để thay đổi S
				\coordinate(E) at ($(B)!0.6!(C)$);
				\coordinate (M) at ($(S)!\k!(D)$);	
				\coordinate (F) at ($(A)!0.4!(B)$);
				\coordinate (P) at ($(S)!0.5!(F)$);	
				
				
				\coordinate (O) at (intersection cs:first line={(A)--(B)}, second line={(C)--(D)});
				\coordinate(Q) at (intersection cs:first line={(P)--(O)}, second line={(S)--(A)});
				\coordinate (R) at (intersection cs:first line={(M)--(O)}, second line={(S)--(C)});
				\draw[thick] (S)--(A)--(B) (C)--(D)--(S)--cycle (B)--(S)--(C)
				(S)--(F) (B)--(O) (C)--(O)--(Q) (M)--(O);
				\draw[thick][dashed] (A)--(D) (A)--(C)--(P)--(D) (B)--(C)(A)--(M)--(B)--(D)  ;
				%	\foreach \diem in {A,B,C,D,S,E,M,N,F,I,O}	\fill (\diem)circle(1.0pt);
				\foreach \i/\j in {S/90,A/150,B/180,C/-45,D/0,M/20,P/180,O/-90,Q/150,R/0}\fill[black] (\i) circle (1pt) ($(\i)+(\j:2.5mm)$)node{$\i$};
				%	\fill[cyan,opacity=0.5] (A)--(M)--(B)--cycle;
				\fill[pattern=dots] (P)--(C)--(D)--cycle;
		\end{tikzpicture}}
		\vspace{0.6cm}
		\noindent c) \textit{Gọi $N$ là trung điểm của $AD .$ Tìm thiết diện tạo bởi $(MNP)$ và hình chóp.}
		\immini{\begin{itemize}
				\item Trong mặt phẳng $SAD$, ta có $$\dfrac{MD}{MS}=\dfrac{2}{3}\neq\dfrac{1}{2}=\dfrac{DN}{DA}$$ nên hai đường thẳng $MN$ và $SA$ cắt nhau, gọi $U$ là giao điểm.
				\item Trong mặt phẳng $(SAB)$, đường thẳng $UP$ phải cắt hai đoạn thẳng $SB$ và $AB$. Gọi các giao điểm theo thứ tự là $X$, $T$.
				\item  Cuối cùng, ta cần tìm giao điểm của $(MNP)$ và cạnh $SC$.
				\begin{itemize}
					\item  Chọn mặt phẳng phụ chứa $SC$ là $(SAC)$.
					\item Gọi $L=AC\cap TN $ thì $(SAC)\cap (AMN)=UL$.
					\item Trong tam giác $SAC$, hai đoạn thẳng $UL$ và $SC$ phải cắt nhau, gọi giao điểm là $Y$ thì đó là giao điểm của $SC$ và $(AMN)$.	
				\end{itemize}
				\item Vậy thiết diện mà mặt phẳng $(MNP)$ cắt hình chóp $S.ABCD$ là ngũ giác $MNTXY$.
		\end{itemize}}
		{\begin{tikzpicture}[scale=1, font=\footnotesize, line join=round, line cap=round, >=stealth]
				\def\ad{6} % cạnh AD
				\def\ab{2.5} % cạnh AB
				\def\bc{3} % chéo AC
				\def\as{4} % cạnh AS
				\def\gocA{55} % góc A của đáy
				\def\gocB{120} % góc B của đáy
				\pgfmathsetmacro\k{1/3}
				\coordinate (A) at (0,0);
				\coordinate(B) at (-\gocA:\ab);
				\coordinate (C) at ($(B)+(180-\gocA-\gocB:\bc)$);
				\coordinate (D) at (\ad,0);
				\coordinate (S) at (60:\as); % chỉnh 75 và as để thay đổi S
				\coordinate(E) at ($(B)!0.6!(C)$);
				\coordinate (M) at ($(S)!\k!(D)$);	
				\coordinate (F) at ($(A)!0.8!(B)$);
				\coordinate (P) at ($(S)!0.6!(F)$);	
				\coordinate (N) at ($(A)!0.5!(D)$);			
				\coordinate (O) at (intersection cs:first line={(A)--(B)}, second line={(C)--(D)});
				\coordinate(Q) at (intersection cs:first line={(P)--(O)}, second line={(S)--(A)});
				\coordinate (U) at (intersection cs:first line={(M)--(N)}, second line={(S)--(A)});
				\coordinate (T) at (intersection cs:first line={(U)--(P)}, second line={(A)--(B)});
				\coordinate (X) at (intersection cs:first line={(S)--(B)}, second line={(P)--(U)});
				\coordinate(L) at (intersection cs:first line={(A)--(C)}, second line={(T)--(N)});
				\coordinate(Y) at (intersection cs:first line={(U)--(L)}, second line={(C)--(S)});		
				\coordinate (V) at (intersection cs:first line={(A)--(B)}, second line={(M)--(N)});
				\draw[thick] (S)--(A)--(B) (C)--(D)--(M) (B)--(S)
				(B)--(C) (A)--(U)--(M) (U)--(T) (U)--(Y)--(C) (X)--(Y)--(M);
				\draw[thick][dashed] (A)--(D) (A)--(C) (B)--(D) (T)--(N)  (M)--(N) (L)--(Y)--(S)--(M)--(N)--(P)--(M);
				%	\foreach \diem in {A,B,C,D,S,E,M,N,F,I,O}	\fill (\diem)circle(1.0pt);
				\foreach \i/\j in {S/90,A/150,B/180,C/-45,D/0,M/20,P/160,N/-90,U/150,T/180,X/180,L/-90,Y/40}\fill[black] (\i) circle (1pt) ($(\i)+(\j:2.5mm)$)node{$\i$};
				\fill[cyan,opacity=0.5] (M)--(N)--(T)--(X)--(Y)--cycle;
				\fill[pattern=dots] (P)--(M)--(N)--cycle;
		\end{tikzpicture}}
	}
\end{bt}
\begin{bt}%Bài 1%[1K4K0-5]]
	Cho hình chóp $S.ABCD$ có đáy $ABCD$ là hình bình hành tâm $O$. Trên các cạnh $SB, SD$ ta lần lượt lấy các điểm $M$ và $N$ thỏa $S B=3 S M$ và $3SN=2SD$. 
	\begin{enumerate}
		\item Tìm giao tuyến của hai mặt phẳng $(AMN)$ và $(SCD)$.
		\item Tìm thiết diện của mặt phẳng $(AMN)$ và hình chóp 
		$S.ABCD$.
		\item Gọi $K$ là giao điểm của $IN$ và $CD$. Tính tỉ số $\dfrac{KC}{KD}$.
	\end{enumerate}
	\loigiai{
		\immini{\begin{enumerate}
				\item \textit{Tìm giao tuyến của hai mặt phẳng $(AMN)$ và $(SCD)$.}
				\begin{itemize}
					\item  $N$ là điểm chung thứ nhất của $(AMN)$ và $(SCD)$.
					\item Gọi $P=MN\cap SO $, $I=AP\cap SC$, thì $I$ là điểm chung thức hai của hai mặt phẳng này.
				\end{itemize}
				Vậy $(AMN)\cap(SCD)=NI$.
				\item \textit{Tìm  thiết diện của mặt phẳng $(AMN)$ và hình chóp $S.ABCD$.}
				Từ câu (a), ta có thiết diện mà mặt phẳng $(AMN)$ cắt hình chóp $S.ABCD$ là tứ giác $AMIN$.	
		\end{enumerate}}
		{\begin{tikzpicture}[scale=1, font=\footnotesize, line join=round, line cap=round, >=stealth]
				\def\a{2.5} \def\b{1.2} \def\c{5} \def\d{4} \def\e{5}
				\pgfmathsetmacro\k{1/3}
				\pgfmathsetmacro\t{2/3}
				\path (0,0) coordinate (B)
				(\a,\b) coordinate (C)
				(\c,0) coordinate (A)
				(\d,\e) coordinate (S)
				($(A)+(C)-(B)$) coordinate (D)
				($(S)!\k!(B)$) coordinate (M)
				($(S)!\t!(D)$) coordinate (N); 
				\coordinate(O) at (intersection cs:first line={(A)--(C)}, second line={(B)--(D)});	
				\coordinate(P) at (intersection cs:first line={(S)--(O)}, second line={(M)--(N)});
				\coordinate(I) at (intersection cs:first line={(A)--(P)}, second line={(C)--(S)});
				\coordinate(K) at (intersection cs:first line={(I)--(N)}, second line={(C)--(D)});
				
				\draw[thick] (A)--(S)--(B)--(A)--(D)--(S) (N)--(A)--(M) (N)--(K)--(D);
				\draw[thick][dashed,thick](S)--(C) (B)--(C)--(D) (M)--(N) (A)--(C) (B)--(D) (S)--(O) (A)--(I) (M)--(I)--(N);
				\foreach \i/\j in {A/-90,B/-135,C/180,D/-90,M/150,N/90,S/180,O/-90,I/30,P/210,K/-90}\fill[black] (\i) circle (1pt) ($(\i)+(\j:3mm)$)node{$\i$};
				\fill[cyan,opacity=0.5] (A)--(M)--(I)--(N)--cycle;
		\end{tikzpicture}}
		\vspace{0.6cm}
		c) \textit{Gọi $K$ là giao điểm của $IN$ và $CD$. Tính tỉ số $\dfrac{KC}{KD}$.}
		\immini{
			\begin{itemize}
				\item Áp dụng định lí Menelaus cho tam giác $SBD$ với ba điểm thẳng hàng $M, N, Q$ ta có
				$$\dfrac{SM}{MB}\cdot\dfrac{BQ}{QD}\cdot\dfrac{DN}{NS}=1\Leftrightarrow \dfrac{BQ}{QD}=\dfrac{1}{4}.$$
				\item Áp dụng định lí Menelaus cho tam giác $SOD$ với ba điểm thẳng hàng $P, N, Q$ ta có
				$$\dfrac{SP}{PO}\cdot\dfrac{OQ}{QD}\cdot\dfrac{DN}{NS}=1\Leftrightarrow \dfrac{SP}{PO}=\dfrac{4}{5}.$$		
		\end{itemize}}
		{\begin{tikzpicture}[>=stealth,line join=round,line cap=round,line width=0.6pt,font=\footnotesize,scale=1]
				\coordinate(B) at (0,0);
				\coordinate(D) at (4,0);
				\coordinate(S) at (1,3);
				\pgfmathsetmacro\k{1/3}
				\pgfmathsetmacro\t{2/3}
				\coordinate (M) at ($(S)!\k!(B)$);
				\coordinate (N) at ($(S)!\t!(D)$);
				\coordinate (O) at ($(B)!0.5!(D)$);
				\coordinate(Q) at (intersection cs:first line={(M)--(N)}, second line={(B)--(D)});
				\coordinate(P) at (intersection cs:first line={(S)--(O)}, second line={(M)--(N)});	
				\draw[thick] (S)--(B)--(D)--cycle (M)--(Q) (S)--(O) (D)--(Q);
				\foreach \i/\j in {S/90,B/-135,M/150,N/90,P/210,D/-90,O/-90,Q/-90}\fill[black] (\i) circle (1pt) ($(\i)+(\j:3mm)$)node{$\i$};
		\end{tikzpicture}}
		\begin{itemize}
			\item Áp dụng định lí Menelaus cho tam giác $SCO$ với ba điểm thẳng hàng $I, P, A$ ta có
			$$\dfrac{SI}{IC}\cdot\dfrac{CA}{AO}\cdot\dfrac{OP}{PS}=1\Leftrightarrow \dfrac{SI}{IC}=\dfrac{2}{5}.$$
			\item Áp dụng định lí Menelaus cho tam giác $SCD$ với ba điểm thẳng hàng $I, N, K$ ta có
			$$\dfrac{SI}{IC}\cdot\dfrac{CK}{KD}\cdot\dfrac{DN}{NS}=1\Leftrightarrow \dfrac{CK}{KD}=5.$$
		\end{itemize}
	}
\end{bt}
\begin{bt}%Bài 1%[1K4K0-5]
	Cho tứ diện đều $ABCD$ có cạnh bằng $a$. Gọi $I$ là trung điểm của $AD$, $J$ là điểm đối xứng với $D$ qua $C$, $K$ là điểm đối xứng với $D$ qua $B$. Xác định thiết diện của hình tứ diện khi cắt bởi mặt phẳng $(IJK)$ và tính diện tích của thiết diện này.
	\loigiai{
		\immini{\begin{itemize}
				\item Gọi $M$ là giao điểm của $IK$ và $AB$, $N$ là giao điểm của $IJ$ và $AC$.
				\item Thiết diện mà mặt phẳng $IJK$ cắt tứ diện là tam giác $IMN$.
				\item Vì $M$, $N$ theo thứ tự là trọng tâm các tam giác $AKD$, $ADJ$ nên $\dfrac{AM}{AB}=\dfrac{AN}{AC}=\dfrac{2}{3}$ và $MN=\dfrac{2a}{3}$.
				\item Vì $AB=AC=a$ nên $AM=AN=\dfrac{2a}{3}$.
				\item Áp dụng Định lí côsin cho tam giác $AMI$ ta được $$IM^2=AM^2+AI^2-2AM\cdot AI\cdot\cos\widehat{AMI}=\dfrac{13a^2}{36}.$$ 
				Suy ra $IM=IN=\dfrac{a\sqrt{13}}{6}$.
				\item $p=\dfrac{IM+IN+MN}{2}=\dfrac{a\sqrt{13}}{6}+\dfrac{a}{3}$.
		\end{itemize}}
		{\begin{tikzpicture}[line cap=round,line join=round, >=stealth,scale=1.2]
				\def\gocvg(#1,#2,#3){\draw[thick][gray]($(#2)!6pt!(#1)$)--($($(#2)!6pt!(#1)$)+($(#2)!6pt!(#3)$)-(#2)$)--($(#2)!6pt!(#3)$);}
				\def \a{1} \def \b{-1} \def \c{4} \def \h{3.5}
				\path (.5,.5)coordinate(A) 
				+(\a,\b)coordinate(B)
				+(\c,0)coordinate(C)
				($(A)!2/3!($(B)!1/2!(C)$)$)coordinate(G) %tọa độ G là trọng tâm ABC
				+(0,\h)coordinate(D);
				\coordinate (I) at ($(A)!0.5!(D)$);	
				\coordinate (J) at ($(C)!-1!(D)$);	
				\coordinate (K) at ($(B)!-1!(D)$);	
				\coordinate (M) at (intersection cs:first line={(A)--(B)}, second line={(I)--(K)});
				\coordinate (N) at (intersection cs:first line={(A)--(C)}, second line={(I)--(J)});
				\coordinate (P) at (intersection cs:first line={(I)--(J)}, second line={(C)--(B)});
				
				\foreach \i/\j in {A/180,B/-60,C/0,D/90,G/-90,I/180,J/-90,K/-90,M/210,N/70}\fill[black] (\i) circle (1pt) ($(\i)+(\j:2mm)$)node{$\i$};	
				%\draw[thick][ultra thin,color=gray] (-.5,-1.5) grid (5.5,4.5);
				\draw[thick] [dashed] (C)--(G)--(A)--(C)(D)--(G)--(B) (I)--(P) (M)--(N);
				\draw[thick] (D)--(A)--(B)--(D)--(C)--(B)--(K)--(I) (C)--(J)--(P);	
				\gocvg(D,G,B) \gocvg(D,G,A)\gocvg(D,G,C)	
				\fill[cyan,opacity=0.5] (M)--(I)--(N)--cycle;	
		\end{tikzpicture}}
		\begin{itemize}
			\item Áp dụng công thức Hê-rông, ta có diện tích tam giác $IMN$ là
			$$S=\sqrt{p(p-IM)(p-IN)(p-MN)}=\sqrt{\left(\dfrac{a \sqrt{13}}{6}+\dfrac{a}{3}\right) \cdot \dfrac{a}{3} \cdot \frac{a}{3} \cdot\left(\dfrac{a \sqrt{13}}{6}-\dfrac{a}{3}\right)}=\dfrac{a^{2}}{6}.$$
		\end{itemize}
	}
\end{bt}
%--
\begin{dang}{Chứng minh ba điểm thẳng hàng}
\phuongphap 
	Để chứng minh ba điểm $A$, $B$, $C$ thẳng hàng, ta cần chứng minh ba điểm này lần lượt thuộc hai mặt phẳng phân biệt $(\alpha)$ và $(\beta)$. Nghĩa là chúng cùng thuộc giao tuyến $d$ của hai mặt phẳng $(\alpha)$ và $(\beta)$ nên chúng thẳng hàng.	
	\begin{center}
		\begin{tikzpicture}[scale=1, font=\footnotesize, line join=round, line cap=round, >=stealth]
			%\draw[thick] (-4,-4) grid (4,4);
			\path
			(-3,0) coordinate (A)
			(-4,2) coordinate (B)
			(0,2) coordinate (C)
			(1,0) coordinate (D)
			(0,-2) coordinate (A')
			(-4,-2) coordinate (B')
			(-2,0) coordinate (C')node[below]{$A$}
			(-1,0) coordinate (D')node[below]{$B$}
			(0,0) coordinate (I)node[below]{$C$}
			;
			\draw[thick] (A)--(B)--(C)--(D)--(A')--(B')--(A)--(D);
			\draw[thick](A)--(B)--(C)
			pic [draw] {angle=A--B--C};%vẽ góc
			\draw[thick] (150:4)node[below left]{$\beta$};
			\draw[thick](A')--(B')--(A)
			pic [draw] {angle=A'--B'--A};%vẽ góc
			\draw[thick] (-155:3.8)node[below left]{$\alpha$};
			\draw[thick][dashed] (A)--(D)node[pos=0.1,above,rotate=0]{$ d $};
			\foreach\p in{C',D',I}\draw[thick][fill=black](\p)circle(1pt);
		\end{tikzpicture}
	\end{center}
\end{dang}
\subsection*{VÍ DỤ MINH HỌA}
\begin{vd}%[1K4K0-6]
	Cho tứ diện $SABC$. Trên các cạnh $SA$, $SB$, $SC$ lần lượt lấy các điểm $M$, $N$, $P$ sao cho
	$MN$ cắt $AB$ tại $I$, $NP$ cắt $BC$ tại $J$ và $MP$ cắt $AC$ tại $K$. Chứng minh rằng ba điểm
	$I$, $J$, $K$ thẳng hàng.
	\loigiai{ 
		\begin{center}
			\begin{tikzpicture}[scale=1, font=\footnotesize, line join=round, line cap=round, >=stealth]
				\path
				(0,0) coordinate (A)node[below]{$A$}
				($(A)+(-60:2.5)$) coordinate (B)node[below]{$B$}
				($(A)+(0:4)$) coordinate (C)node[right]{$C$}
				($(A)+(70:4)$) coordinate (S)node[right]{$S$}
				($(C)!2/3!(S)$) coordinate (P)node[above]{$P$}
				($(A)+(70:1.5)$) coordinate (M)node[right]{$M$}
				($(B)!1/5!(S)$) coordinate (N)node[right]{$N$}
				($(M)!2!180:(P)$)coordinate (Mt)
				($(A)!1!180:(C)$)coordinate (At)
				($(N)!2!180:(M)$)coordinate (Nt)
				($(B)!1!180:(A)$)coordinate (Bt)
				($(N)!1!180:(P)$)coordinate (Nt')
				($(B)!1!180:(C)$)coordinate (Bt')
				;
				\path[name path=a] (M)--(Mt);
				\path[name path=b] (A)--(At);
				\path [name intersections={of=a and b,by=K}];
				\path(K) coordinate (K)node[left]{$K$};
				\path[name path=c] (N)--(Nt);
				\path[name path=d] (B)--(Bt);
				\path [name intersections={of=c and d,by=I}];
				\path(I) coordinate (I)node[left]{$I$};
				\path[name path=e] (N)--(Nt');
				\path[name path=f] (B)--(Bt');
				\path [name intersections={of=e and f,by=J}];
				\path(J) coordinate (J)node[left]{$J$};
				\draw[thick](S)--(A)--(B)--(C)--(S)--(B) (M)--(K)--(A) (M)--(I)--(B) (P)--(J)--(B);
				\draw[thick][dashed](A)--(C) (P)--(M);
				\foreach\p in{P,M,N,K,I,J}\draw[thick][fill=black](\p)circle(1pt);
			\end{tikzpicture}
		\end{center}		
		Gọi đường thẳng $d$ là giao tuyến của hai mặt phẳng $(MNP)$ và $(ABC)$.\\
		Ta có $K=MP\cap AC \Rightarrow \heva{&K \in MP,MP\subset (MNP)\Rightarrow K\in(MNP)\\&K \in AC,AC\subset (ABC)\Rightarrow K\in(ABC)}\Rightarrow K\in d\quad (1)$\\
		Tương tự:\\
		$$I=MN\cap AB \Rightarrow \heva{&I \in MN,MN\subset (MNP)\Rightarrow I\in(MNP)\\&I \in AB,AB\subset (ABC)\Rightarrow I\in(ABC)}\Rightarrow I\in d\quad (2)$$
		$$J=NP\cap BC \Rightarrow \heva{&J \in NP,NP\subset (MNP)\Rightarrow J\in(MNP)\\&J \in BC,BC\subset (ABC)\Rightarrow J\in(ABC)}\Rightarrow J\in d\quad (3)$$
		Từ (1),(2),(3), suy ra $I$, $J$, $K$ cùng thuộc $d$ $\Rightarrow$ ba điểm $I$, $J$, $K$ thẳng hàng (đpcm).	
	}
\end{vd}
\begin{vd}%[1K4K0-6]%[1H2B1-5]
	Cho tứ diện $ABCD$ có $G$ là trọng tâm tam giác $BCD$. Gọi $M$, $N$, $P$ lần lượt là trung
	điểm của các cạnh $AB$, $BC$, $CD$.
	\begin{enumerate}
		\item Xác định giao tuyến của $(ADN)$ và $(ABP)$.
		\item Gọi $I=AG\cap MP$ và $J=CM\cap AN$. Chứng minh $D$, $I$, $J$ thẳng hàng.
	\end{enumerate}	
	\loigiai{ 
		\begin{center}
			\begin{tikzpicture}[scale=1, font=\footnotesize, line join=round, line cap=round, >=stealth]
				\path
				(0,0) coordinate (B)node[left]{$B$}
				($(A)+(-60:2.5)$) coordinate (C)node[below]{$C$}
				($(A)+(0:4)$) coordinate (D)node[right]{$D$}
				($(B)!1/2!(C)$) coordinate (N)node[right]{$N$}
				($(C)!1/2!(D)$) coordinate (P)node[left]{$P$}
				($(B)!2/3!(P)$) coordinate (G)node[below]{$G$}
				($(G)+(90:4)$) coordinate (A)node[above]{$A$}
				($(A)!1/2!(B)$) coordinate (M)node[left]{$M$}
				;
				\path[name path=a] (M)--(C);
				\path[name path=b] (A)--(N);
				\path [name intersections={of=a and b,by=J}];
				\path(J) coordinate (J)node[left]{$J$};
				\path[name path=c] (A)--(G);
				\path[name path=d] (M)--(P);
				\path [name intersections={of=c and d,by=I}];
				\path(I) coordinate (I)node[left]{$I$};
				\draw[thick] (A)--(B)--(C)--(D)--(A)--(C) (A)--(N) (C)--(M)(A)--(P);
				\draw[thick][dashed](B)--(D)--(N) (B)--(P) (A)--(G) (M)--(P);
				\foreach\p in{A,B,C,D,M,N,P,G,I,J}\draw[thick][fill=black](\p)circle(1pt);
			\end{tikzpicture}
		\end{center}
		\begin{enumerate}
			\item Ta có $\heva{A\in(ADN)\\A\in(ABP)}\Rightarrow A\in(ADN)\cap(ABP)\quad (1)$\\
			Tương tự $$G=BP\cap DN\Rightarrow\heva{&G\in DN,DN\subset(ADN)\Rightarrow G\in(ADN)\\&G\in BP,BP\subset(ABP)\Rightarrow G\in(ABP)}\Rightarrow G\in(ADN)\cap(ABP)\quad (2)$$
			Từ $(1)$, $(2)$ suy ra $(ADN)\cap(ABP)=AG$.
			\item Gọi đường thẳng $d$ là giao tuyến của hai mặt phẳng $(DCM)$ và $(ADN)$.\\
			Ta có $\heva{D \in (DCM)\\D \in (ADN)}\Rightarrow D\in d\quad (1)$\\
			Tương tự:\\
			$$I=AG\cap MP \Rightarrow \heva{&I \in AG,AG\subset (ADN)\Rightarrow I\in(ADN)\\&I \in MP,MP\subset (DCM)\Rightarrow I\in(DCM)}\Rightarrow I\in d\quad (2)$$
			$$J=CM\cap AN \Rightarrow \heva{&J \in CM,CM\subset (DCM)\Rightarrow J\in(DCM)\\&J \in AN,AN\subset (ADN)\Rightarrow J\in(ADN)}\Rightarrow J\in d\quad (3)$$
			Từ (1),(2),(3), suy ra $D$, $I$, $J$ cùng thuộc $d$ $\Rightarrow$ ba điểm $D$, $I$, $J$ thẳng hàng (đpcm).
		\end{enumerate}	
	}
\end{vd}
\begin{vd}%[1K4K0-6]%[1H2K1-5]
	Cho hình chóp $S.ABCD$ có đáy $ABCD$ là hình bình hành tâm $O$, hai điểm $M$, $N$ lần lượt
	là trung điểm của $SB$, $SD$, điểm $P$ thuộc $SC$ và không là trung điểm của $SC$.
	\begin{enumerate}
		\item Tìm giao điểm của $SO$ với $(MNP)$.
		\item Tìm giao điểm $Q$ của $SA$ với mặt phẳng $(MNP)$.
		\item Gọi $F$, $G$, $H$ lần lượt là giao điểm của $QM$ và $AB$, $QP$ và $AC$, $QN$ và $AD$. Chứng
		minh ba điểm $F$, $G$, $H$ thẳng hàng.
	\end{enumerate}	
	\loigiai{ 
		\begin{center}
			\begin{tikzpicture}[scale=1, font=\footnotesize, line join=round, line cap=round, >=stealth]
				\path
				(0,0) coordinate (A)node[left]{$A$}
				($(A)+(-120:2)$) coordinate (B)node[left]{$B$}
				($(B)+(0:4)$) coordinate (C)node[below]{$C$}
				($(A)+(0:4)$) coordinate (D)node[left]{$D$}
				($(B)+(85:5)$) coordinate (S)node[left]{$S$}
				($(S)!1/2!(B)$) coordinate (M)node[left]{$M$}
				($(S)!1/2!(D)$) coordinate (N)node[above]{$N$}
				($(C)!1/4!(S)$) coordinate (P)node[right]{$P$}
				;
				\path[name path=a] (A)--(C);
				\path[name path=b] (B)--(D);
				\path [name intersections={of=a and b,by=O}];
				\path(O) coordinate (O)node[left]{$O$};
				\path[name path=c] (M)--(N);
				\path[name path=d] (S)--(O);
				\path [name intersections={of=c and d,by=I}];
				\path(I) coordinate (I)node[left]{$I$};
				\path
				($(I)!1!180:(P)$)coordinate (It)
				;
				\path[name path=e] (I)--(It);
				\path[name path=f] (S)--(A);
				\path [name intersections={of=e and f,by=Q}];
				\path(Q) coordinate (Q)node[left]{$Q$};
				\path
				($(N)!5!-180:(Q)$)coordinate (Nt)
				($(D)!3!-180:(A)$)coordinate (Dt)
				($(P)!2!-180:(Q)$)coordinate (Pt)
				($(C)!2!-180:(A)$)coordinate (Ct)
				($(M)!5!-180:(Q)$)coordinate (Mt)
				($(B)!3!-180:(A)$)coordinate (Bt)
				;
				\path[name path=g] (N)--(Nt);
				\path[name path=h] (D)--(Dt);
				\path [name intersections={of=g and h,by=F}];
				\path(F) coordinate (F)node[above]{$F$};
				\path[name path=i] (P)--(Pt);
				\path[name path=k] (C)--(Ct);
				\path [name intersections={of=i and k,by=G}];
				\path(G) coordinate (G)node[below]{$G$};
				\path[name path=u] (M)--(Mt);
				\path[name path=v] (B)--(Bt);
				\path [name intersections={of=u and v,by=H}];
				\path(H) coordinate (H)node[left]{$H$};
				\draw[thick](S)--(B)--(C)--(D)--(S)--(C) (M)--(P)--(N)--(F)--(D) (P)--(G)--(C) (M)--(H)--(B);
				\draw[thick][dashed](S)--(A)--(B) (A)--(D) (A)--(C) (B)--(D) (M)--(N) (S)--(O) (P)--(I)--(Q)--(M) (Q)--(N);
				\foreach\p in{A,B,C,D,S,M,N,P,O,I,Q,F,G,H}\draw[thick][fill=black](\p)circle(1pt);
			\end{tikzpicture}
		\end{center}
		\begin{enumerate}
			\item Trong $(SBD)$ ta có $\heva{&SO\cap MN=\{I\}\\&MN\subset(MNP)}\Rightarrow SO\cap(MNP)=\{I\}$.
			\item Chọn $mp(SAC)$ chứa $SA$.
			Ta có $(SAC)\cap(MNP)=PI$. Kẻ $PI\cap SA=\{Q\}$.\\
			Suy ra $SA\cap(MNP)=\{Q\}$.
			\item Gọi đường thẳng $d$ là giao tuyến của hai mặt phẳng $(MNPQ)$ và $(ABCD)$.\\
			Ta có 
			$$F=QM\cap AB \Rightarrow \heva{&F \in QM,QM\subset (MNPQ)\Rightarrow F\in(MNPQ)\\&F \in AB,AB\subset (ABCD)\Rightarrow F\in(ABCD)}\Rightarrow F\in d\quad (1)$$
			$$G=PQ\cap AC \Rightarrow \heva{&G \in PQ,PQ\subset (MNPQ)\Rightarrow G\in(MNPQ)\\&G \in AC,AC\subset (ABCD)\Rightarrow G\in(ABCD)}\Rightarrow G\in d\quad (2)$$
			$$H=NQ\cap AD \Rightarrow \heva{&H \in NQ,NQ\subset (MNPQ)\Rightarrow H\in(MNPQ)\\&H \in AD,AD\subset (ABCD)\Rightarrow H\in(ABCD)}\Rightarrow H\in d\quad (3)$$
			Từ (1),(2),(3), suy ra $F$, $G$, $H$ cùng thuộc $d$ $\Rightarrow$ ba điểm $F$, $G$, $H$ thẳng hàng (đpcm).
		\end{enumerate}	
	}
\end{vd}
\begin{vd}%[1K4K0-6]%[1H2B1-5]
	Cho hình chóp $S.ABCD$ có $AD$ không song song với $BC$. Lấy $M$ thuộc $SB$ và $O$ là
	giao điểm $AC$ với $BD$.
	\begin{enumerate}
		\item Tìm giao điểm $N$ của $SC$ với $(ADM)$.
		\item Gọi $I=AN\cap DM$. Chứng minh $S$, $I$, $O$ thẳng hàng.
	\end{enumerate}
	\loigiai{ 
		\immini{
			\begin{enumerate}
				\item Chọn $mp(SBC)$ chứa $SC$. Gọi $H=AD\cap BC$.\\
				Ta có $(SBC)\cap(ADM)=MH$. Kẻ $MH\cap SC=N$.
				Suy ra $SC\cap(ADM)=N$.	
				\item Gọi đường thẳng $d$ là giao tuyến của hai mặt phẳng $(SAC)$ và $(SBD)$.\\
				Ta có 
				$\heva{&S\in(SAC)\\&S\in(SBD)}\Rightarrow S\in d\quad (1)$.\\
				Tương tự
				$$I=AN\cap SO \Rightarrow \heva{I \in AN,AN\subset (SAC)\Rightarrow I\in(SAC)\\I \in SO,SO\subset (SBD)\Rightarrow I\in(SBD)}\Rightarrow I\in d\quad (2)$$
				$$O=AC\cap BD \Rightarrow \heva{&O \in AC,AC\subset (SAC)\Rightarrow O\in(SAC)\\&O \in BD,BD\subset (SBD)\Rightarrow O\in(SBD)}\Rightarrow O\in d\quad (3)$$
				Từ (1),(2),(3), suy ra $S$, $I$, $O$ cùng thuộc $d$ $\Rightarrow$ ba điểm $S$, $I$, $O$ thẳng hàng (đpcm).
		\end{enumerate}}{\begin{tikzpicture}[scale=0.8, font=\footnotesize, line join=round, line cap=round, >=stealth]
				\path
				(0,0) coordinate (A)node[left]{$A$}
				($(A)+(0:4)$) coordinate (B)node[right]{$B$}
				($(A)+(-40:4)$) coordinate (C)node[below]{$C$}
				($(A)+(-60:2)$) coordinate (D)node[left]{$D$}
				($(A)+(80:4)$) coordinate (S)node[left]{$S$}
				($(B)!1/4!(S)$) coordinate (M)node[right]{$M$}
				($(D)!2!180:(A)$) coordinate (Dt)
				($(C)!2!180:(B)$) coordinate (Ct)
				;
				\path[name path=a] (A)--(C);
				\path[name path=b] (B)--(D);
				\path [name intersections={of=a and b,by=O}];
				\path(O) coordinate (O)node[below]{$O$};
				\path[name path=c] (D)--(Dt);
				\path[name path=d] (C)--(Ct);
				\path [name intersections={of=c and d,by=H}];
				\path(H) coordinate (H)node[left]{$H$};
				\path[name path=e] (S)--(C);
				\path[name path=f] (M)--(H);
				\path [name intersections={of=e and f,by=N}];
				\path(N) coordinate (N)node[left]{$N$};
				\path[name path=g] (A)--(N);
				\path[name path=h] (D)--(M);
				\path [name intersections={of=g and h,by=I}];
				\path(I) coordinate (I)node[above]{$I$};
				\draw[thick] (S)--(A)--(D)(C)--(B)--(S)--(D) (S)--(C)--(H)--(D) (M)--(H);
				\draw[thick][dashed](A)--(B) (A)--(C) (B)--(D)--(C) (A)--(M)--(D) (A)--(N);
				\foreach\p in{A,B,C,D,S,M,O,H,N,I}\draw[thick][fill=black](\p)circle(1pt);
		\end{tikzpicture}}
	}
\end{vd}
\begin{vd}%[1K4K0-6]%[1H2B1-5]
	Cho tứ diện $ABCD$ có $G$ là trọng tâm tam giác $BCD$. Gọi $M$, $N$, $P$ lần lượt là trung
	điểm của $AB$, $BC$, $CD$.
	\begin{enumerate}
		\item Tìm giao tuyến của $(ADN)$ và $(ABP)$.
		\item Gọi $I=AG\cap MP$ và $J=CM\cap AN$ Chứng minh $D$, $I$, $J$ thẳng hàng.
	\end{enumerate}
	\loigiai{ 
		\begin{center}
			\begin{tikzpicture}[scale=1, font=\footnotesize, line join=round, line cap=round, >=stealth]
				\path
				(0,0) coordinate (B)node[left]{$B$}
				($(A)+(-60:2.5)$) coordinate (C)node[below]{$C$}
				($(A)+(0:4)$) coordinate (D)node[right]{$D$}
				($(B)!1/2!(C)$) coordinate (N)node[right]{$N$}
				($(C)!1/2!(D)$) coordinate (P)node[left]{$P$}
				($(B)!2/3!(P)$) coordinate (G)node[below]{$G$}
				($(G)+(90:4)$) coordinate (A)node[above]{$A$}
				($(A)!1/2!(B)$) coordinate (M)node[left]{$M$}
				;
				\path[name path=a] (M)--(C);
				\path[name path=b] (A)--(N);
				\path [name intersections={of=a and b,by=J}];
				\path(J) coordinate (J)node[left]{$J$};
				\path[name path=c] (A)--(G);
				\path[name path=d] (M)--(P);
				\path [name intersections={of=c and d,by=I}];
				\path(I) coordinate (I)node[left]{$I$};
				\draw[thick] (A)--(B)--(C)--(D)--(A)--(C) (A)--(N) (C)--(M)(A)--(P);
				\draw[thick][dashed](B)--(D)--(N) (B)--(P) (A)--(G) (M)--(P);
				\foreach\p in{A,B,C,D,M,N,P,G,I,J}\draw[thick][fill=black](\p)circle(1pt);
			\end{tikzpicture}
		\end{center}
		\begin{enumerate}
			\item Ta có 
			$\heva{&A\in(ADN)\\&A\in(ABP)}\Rightarrow A\in (ADN)\cap (ABP) \quad (1)$.\\
			Tương tự
			$$G=BP\cap DN \Rightarrow \heva{&G \in BP,BP\subset (ABP)\Rightarrow G\in(ABP)\\&G \in DN,DN\subset (ADN)\Rightarrow G\in(ADN)}\Rightarrow G\in(ADN)\cap (ABP) \quad (2)$$
			Từ $(1)$, $(2)$ $\Rightarrow (ADN)\cap (ABP)=AG$.
			\item Gọi đường thẳng $d$ là giao tuyến của hai mặt phẳng $(DMC)$ và $(ADN)$.\\
			Ta có 
			$\heva{&D\in(DMC)\\&D\in(ADN)}\Rightarrow D\in d\quad (1)$.\\
			Tương tự
			$$I=AG\cap MP \Rightarrow \heva{&I \in AG,AG\subset (ADN)\Rightarrow I\in(ADN)\\&I \in MP,MP\subset (DMC)\Rightarrow I\in(DMC)}\Rightarrow I\in d\quad (2)$$
			$$J=CM\cap AN \Rightarrow \heva{&J \in CM,CM\subset (DMC)\Rightarrow J\in(DMC)\\&J \in AN,AN\subset (ADN)\Rightarrow J\in(ADN)}\Rightarrow J\in d\quad (3)$$
			Từ (1),(2),(3), suy ra $D$, $I$, $J$ cùng thuộc $d$ $\Rightarrow$ ba điểm $D$, $I$, $J$ thẳng hàng (đpcm).
		\end{enumerate}	
	}
\end{vd}
\subsection*{BÀI TẬP LUYỆN TẬP}
\begin{bt}%[1K4K0-6]%[1H2K1-5]
	Cho hình chóp $S.ABCD$. Gọi $E$, $F$, $H$ lần lượt là các điểm thuộc cạnh $SA$, $SB$, $SC$.
	\begin{enumerate}
		\item Tìm giao điểm $K=SD\cap (EFH)$.
		\item Gọi $O=AC\cap BD$ và $I=EH\cap FK$. Chứng minh: $S$, $I$, $O$ thẳng hàng.
		\item Gọi $M=AD\cap BC$ và $N=EK\cap FH$. Chứng minh: $S$, $M$, $N$ thẳng hàng.
		\item Gọi $P=AB\cap CD$ và $Q=EF\cap HK$. Chứng minh: $S$, $P$, $Q$ thẳng hàng.
	\end{enumerate}
	\loigiai{ 
		\begin{center}
			\begin{tikzpicture}[scale=1, font=\footnotesize, line join=round, line cap=round, >=stealth]
				\path
				(0,0) coordinate (A)node[left]{$A$}
				($(A)+(0:4)$) coordinate (B)node[right]{$B$}
				($(A)+(-40:4)$) coordinate (C)node[below]{$C$}
				($(A)+(-60:2)$) coordinate (D)node[left]{$D$}
				($(A)+(80:4)$) coordinate (S)node[left]{$S$}
				($(S)!1/3!(A)$) coordinate (E)node[above left]{$E$}
				($(S)!1/3!(B)$) coordinate (F)node[right]{$F$}
				($(S)!1/3!(C)$) coordinate (H)node[right]{$H$}
				;
				\path[name path=a] (A)--(C);
				\path[name path=b] (D)--(B);
				\path [name intersections={of=a and b,by=O}];
				\path(O) coordinate (O)node[below]{$O$};
				\path[name path=c] (S)--(O);
				\path[name path=d] (E)--(H);
				\path [name intersections={of=c and d,by=I}];
				\path(I) coordinate (I)node[right]{$I$};
				\path ($(I)!1!-180:(F)$) coordinate (It);
				\path[name path=e] (S)--(D);
				\path[name path=f] (I)--(It);
				\path [name intersections={of=e and f,by=K}];
				\path(K) coordinate (K)node[left]{$K$};
				\path 
				($(D)!2!180:(A)$) coordinate (Dt)
				($(C)!1!180:(B)$) coordinate (Ct)
				($(A)!1!180:(B)$) coordinate (At)
				($(D)!3!180:(C)$) coordinate (Dt')
				($(E)!1!180:(F)$) coordinate (Et)
				($(K)!5!180:(H)$) coordinate (Kt)
				($(K)!3!180:(E)$) coordinate (Kt')
				($(H)!2!180:(F)$) coordinate (Ht)
				;
				\path[name path=g] (D)--(Dt);
				\path[name path=h] (C)--(Ct);
				\path [name intersections={of=g and h,by=M}];
				\path(M) coordinate (M)node[left]{$M$};
				\path[name path=i] (A)--(At);
				\path[name path=j] (D)--(Dt');
				\path [name intersections={of=i and j,by=P}];
				\path(P) coordinate (P)node[left]{$P$};
				\path[name path=u] (E)--(Et);
				\path[name path=v] (K)--(Kt);
				\path [name intersections={of=u and v,by=Q}];
				\path(Q) coordinate (Q)node[left]{$Q$};
				\path[name path=p] (K)--(Kt');
				\path[name path=r] (H)--(Ht);
				\path [name intersections={of=p and r,by=N}];
				\path(N) coordinate (N)node[right]{$N$};
				\draw[thick] (C)--(B)--(S)--(D) (S)--(C) (F)--(H) (S)--(M) (C)--(M)--(D)--(P) (Q)--(K)--(N)--(H) (S)--(P);
				\draw[thick][dashed](A)--(B) (A)--(C) (B)--(D) (H)--(E)--(F) (S)--(O) (K)--(F) (A)--(D)--(C)(P)--(A)(S)--(A)(E)--(Q)
				;
				\foreach\p in{A,B,C,D,S,E,F,H,O,I,K,M,P,Q,N}\draw[thick][fill=black](\p)circle(1pt);
			\end{tikzpicture}
		\end{center}
		\begin{enumerate}
			\item Gọi $O=AC\cap BD$, $I=SO\cap EF$. Chọn $(SBD)$ chứa $SD$.\\
			Ta có $$\heva{&F\in (EFH)\\&F\in SB,SB\subset (SBD)\Rightarrow F\in(SBD)}\Rightarrow F\in(SBD)\cap(EFH)\quad(1)$$ 
			$$I=SO\cap EH \Rightarrow \heva{&I \in SO,SO\subset (SBD)\Rightarrow I\in(SBD)\\&I \in EH,EH\subset (EFH)\Rightarrow I\in(EFH)}\Rightarrow I\in(SBD)\cap(EFH)\quad(2)$$	
			Từ $(1)$, $(2)$ suy ra $(SBD)\cap(EFH)=IF$. Kẻ $IF\cap SD=K$ suy ra $SD\cap(EFH)=K$.
			\item Gọi đường thẳng $d$ là giao tuyến của hai mặt phẳng $(SAC)$ và $(SBD)$.\\
			Ta có 
			$\heva{&S\in(SAC)\\&S\in(SBD)}\Rightarrow S\in d\quad (1)$.\\
			Tương tự
			$$O=AC\cap BD \Rightarrow \heva{&O \in AC,AC\subset (SAC)\Rightarrow O\in(SAC)\\&O \in BD,BD\subset (SBD)\Rightarrow O\in(SBD)}\Rightarrow O\in d\quad (2)$$
			$$I=EH\cap FK \Rightarrow \heva{&I \in EH,EH\subset (SAC)\Rightarrow I\in(SAC)\\&I \in FK,FK\subset (SBD)\Rightarrow I\in(SBD)}\Rightarrow I\in d\quad (3)$$
			Từ (1),(2),(3), suy ra $S$, $O$, $I$ cùng thuộc $d$ $\Rightarrow$ ba điểm $S$, $O$, $I$ thẳng hàng (đpcm).
			\item Gọi đường thẳng $d$ là giao tuyến của hai mặt phẳng $(SAD)$ và $(SBC)$.\\
			Ta có 
			$\heva{&S\in(SAD)\\&S\in(SBC)}\Rightarrow S\in d\quad (1)$.\\
			Tương tự
			$$M=AD\cap BC \Rightarrow \heva{&M\in AD,AD\subset (SAD)\Rightarrow M\in(SAD)\\&M \in BC,BC\subset (SBC)\Rightarrow M\in(SBC)}\Rightarrow M\in d\quad (2)$$
			$$N=EK\cap FH \Rightarrow \heva{&N \in EK,EK\subset (SAD)\Rightarrow N\in(SAD)\\&N \in FH,FH\subset (SBC)\Rightarrow N\in(SBC)}\Rightarrow N\in d\quad (3)$$
			Từ (1),(2),(3), suy ra $S$, $M$, $N$ cùng thuộc $d$ $\Rightarrow$ ba điểm $S$, $M$, $N$ thẳng hàng (đpcm).
			\item Gọi đường thẳng $d$ là giao tuyến của hai mặt phẳng $(SAB)$ và $(SDC)$.\\
			Ta có 
			$\heva{&S\in(SAB)\\&S\in(SDC)}\Rightarrow S\in d\quad (1)$.\\
			Tương tự
			$$P=AB\cap CD \Rightarrow \heva{&P\in AB,AB\subset (SAB)\Rightarrow P\in(SAB)\\&P \in CD,CD\subset (SDC)\Rightarrow P\in(SDC)}\Rightarrow P\in d\quad (2)$$
			$$Q=EF\cap HK \Rightarrow \heva{&Q \in EF,EF\subset (SAB)\Rightarrow Q\in(SAB)\\&Q \in HK,HK\subset (SDC)\Rightarrow Q\in(SDC)}\Rightarrow Q\in d\quad (3)$$
			Từ (1),(2),(3), suy ra $S$, $P$, $Q$ cùng thuộc $d$ $\Rightarrow$ ba điểm $S$, $P$, $Q$ thẳng hàng (đpcm).
		\end{enumerate}	
	}
\end{bt}
\begin{bt}%[1K4K0-6]
	Cho tứ diện $ABCD$. Gọi $M$, $N$, $P$ lần lượt là các điểm thuộc cạnh $AB$, $AC$, $BD$ và $MN\cap BC=I$, $MP\cap AD=J$, $NJ\cap IP=K$. Chứng minh $C$, $D$, $K$ thẳng hàng.
	\loigiai{ 
		\begin{center}
			\begin{tikzpicture}[scale=1, font=\footnotesize, line join=round, line cap=round, >=stealth]
				\path
				(0,0) coordinate (B)node[left]{$B$}
				($(A)+(-60:2.5)$) coordinate (C)node[below]{$C$}
				($(A)+(0:4)$) coordinate (D)node[right]{$D$}
				($(B)+(80:4)$) coordinate (A)node[above]{$A$}
				($(B)!1/3!(A)$) coordinate (M)node[above left]{$M$}
				($(A)!1/3!(C)$) coordinate (N)node[above right]{$N$}
				($(D)!1/2!(B)$) coordinate (P)node[above]{$P$}
				($(B)!1!180:(C)$) coordinate (Bt)
				($(M)!3!180:(N)$) coordinate (Mt)
				($(D)!1!180:(A)$) coordinate (Dt)
				($(P)!3!180:(M)$) coordinate (Pt);			
				\path[name path=a] (B)--(Bt);
				\path[name path=b] (M)--(Mt);
				\path [name intersections={of=a and b,by=I}];
				\path(I) coordinate (I)node[left]{$I$};
				\path[name path=c] (D)--(Dt);
				\path[name path=d] (P)--(Pt);
				\path [name intersections={of=c and d,by=J}];
				\path(J) coordinate (J)node[right]{$J$};
				\path[name path=e] (N)--(J);
				\path[name path=f] (C)--(D);
				\path [name intersections={of=e and f,by=K}];
				\path(K) coordinate (K)node[right]{$K$};
				\coordinate(Q) at (intersection cs:first line={(M)--(J)}, second line={(C)--(D)});
				\draw[thick] (A)--(B)--(C)--(D)--(A)--(C) (N)--(J)--(D)(B)--(I)--(N) (Q)--(J);
				\draw[thick][dashed](B)--(D)(J) (I)--(K) (M)--(Q);
				\foreach\p in{A,B,C,D,M,N,P,I,J,K}\draw[thick][fill=black](\p)circle(1pt);
			\end{tikzpicture}
		\end{center}
		Gọi đường thẳng $d$ là giao tuyến của hai mặt phẳng $(BCD)$ và $(ADC)$.\\
		Ta có 
		$\heva{&C\in(BCD)\\&C\in(ADC)}\Rightarrow C\in d\quad (1)$, $\heva{&D\in(BCD)\\&D\in(ADC)}\Rightarrow D\in d.\quad (2)$\\
		Tương tự
		$$K=NJ\cap PI \Rightarrow \heva{&K\in NJ,NJ\subset (ADC)\Rightarrow K\in(ADC)\\&K \in PI,PI\subset (BCD)\Rightarrow K\in(BCD)}\Rightarrow K\in d.\quad (3)$$
		Từ (1),(2),(3), suy ra $C$, $D$, $K$ cùng thuộc $d$ $\Rightarrow$ ba điểm $C$, $D$, $K$ thẳng hàng (đpcm).			
	}
\end{bt}
\begin{bt}%[1K4K0-6]%[1H2B1-3]%[1H2K1-5]
	Cho hình chóp $S.ABCD$. Gọi $I$ và $J$ là hai điểm trên hai cạnh $AD$, $SB$.
	\begin{enumerate}
		\item Tìm giao tuyến của $(SBI)$ và $(SAC)$. Tìm giao điểm $K$ của $IJ$ và $(SAC)$.
		\item Tìm giao tuyến của $(SBD)$ và $(SAC)$. Tìm giao điểm $L$ của $DJ$ và $(SAC)$.
		\item Gọi $O=AD\cap BC$, $M=OJ\cap SC$. Chứng minh rằng: $A$, $K$, $L$, $M$ thẳng hàng.
	\end{enumerate}
	\loigiai{ 
		\immini{	
			\begin{enumerate}
				\item Ta có $\heva{S\in(SBI)\\S\in(SAC)}\Rightarrow (SBI)\cap(SAC)=S\quad(1)$\\
				Gọi $H=BI\cap AC\Rightarrow \heva{&H\in BI\subset (SBI)\Rightarrow H\in(SBI)\\&H\in AC\subset (SAC)\Rightarrow H\in(SAC)}\Rightarrow H\in(SBI)\cap(SAC)\quad(2)$\\
				Từ (1),(2) suy ra $(SBI)\cap(SAC)=SH$.\\
				Chọn $(SBI)$ chứa $IJ$, $(SBI)\cap(SAC)=SH$. Gọi $K=IJ\cap SH$ suy ra $IJ\cap(SAC)=K$.
				\item Ta có $\heva{S\in(SBD)\\S\in(SAC)}\Rightarrow (SBD)\cap(SAC)=S\quad(1)$\\
				Gọi $G=BD\cap AC\Rightarrow \heva{&G\in BD\subset (SBD)\Rightarrow G\in(SBD)\\&G\in AC\subset (SAC)\Rightarrow G\in(SAC)}\\ \Rightarrow G\in(SBD)\cap(SAC)\quad(2)$\\
				Từ (1),(2) suy ra $(SBD)\cap(SAC)=SG$.\\
				Chọn $(SBD)$ chứa $DJ$, $(SBD)\cap(SAC)=SG$. Gọi $L=DJ\cap SG$ suy ra $DJ\cap(SAC)=L$.	
		\end{enumerate}}	{\begin{tikzpicture}[scale=1, font=\footnotesize, line join=round, line cap=round, >=stealth]
				\path
				(0,0) coordinate (A)node[left]{$A$}
				($(A)+(0:4)$) coordinate (B)node[right]{$B$}
				($(A)+(-40:4)$) coordinate (C)node[below]{$C$}
				($(A)+(-60:2)$) coordinate (D)node[left]{$D$}
				($(A)+(80:4)$) coordinate (S)node[left]{$S$}
				($(B)!2/3!(S)$) coordinate (J)node[right]{$J$}
				($(D)!0.6!(A)$) coordinate (I)node[left]{$I$}
				;
				\path[name path=a] (B)--(I);
				\path[name path=b] (A)--(C);
				\path [name intersections={of=a and b,by=H}];
				\path(H) coordinate (H)node[below]{$H$};
				\path[name path=c] (B)--(D);
				\path[name path=b] (A)--(C);
				\path [name intersections={of=c and b,by=G}];
				\path(G) coordinate (G)node[below]{$G$};
				\path(H) coordinate (H)node[below]{$H$};
				\path[name path=d] (S)--(H);
				\path[name path=e] (I)--(J);
				\path [name intersections={of=d and e,by=K}];
				\path(K) coordinate (K)node[above left]{$K$};
				\path[name path=f] (S)--(G);
				\path[name path=g] (D)--(J);
				\path [name intersections={of=f and g,by=L}];
				\path(L) coordinate (L)node[above right]{$L$};
				\path
				($(D)!2!180:(A)$) coordinate (Dt)
				($(C)!2!180:(B)$) coordinate (Ct)
				;
				\path[name path=h] (D)--(Dt);
				\path[name path=i] (C)--(Ct);
				\path [name intersections={of=h and i,by=O}];
				\path(O) coordinate (O)node[right]{$O$};
				\path[name path=j] (S)--(C);
				\path[name path=k] (O)--(J);
				\path [name intersections={of=j and k,by=M}];
				\path(M) coordinate (M)node[right]{$M$};
				\draw[thick](S)--(A)--(D) (C)--(B)--(S)--(D) (S)--(C)--(O)--(D) (O)--(J);
				\draw[red,thick][dashed] (A)--(M);
				\draw[thick][dashed](A)--(B) (I)--(J) (I)--(B) (A)--(C) (B)--(D) (S)--(H) (S)--(G) (D)--(J) (D)--(C);
				\foreach\p in{J,I,H,G,K,L,O,M}\draw[thick][fill=black](\p)circle(1pt);
				
		\end{tikzpicture}}
		\vspace{-1cm}
		\begin{enumerate}[c)]
			\item Gọi đường thẳng $d$ là giao tuyến của hai mặt phẳng $(SAC)$ và $(AJO)$.\\
			Ta có 
			$\heva{&A\in(SAC)\\&A\in(AJO)}\Rightarrow A\in d\quad (1)$.\\
			Tương tự
			$$K=IJ\cap SH \Rightarrow \heva{&K\in IJ\subset (AJO)\Rightarrow K\in(AJO)\\&K \in SH\subset (SAC)\Rightarrow K\in(SAC)}\Rightarrow K\in d\quad (2)$$
			$$L=DJ\cap SG \Rightarrow \heva{&L \in DJ\subset (AJO)\Rightarrow L\in(AJO)\\&L \in SG\subset (SAC)\Rightarrow L\in(SAC)}\Rightarrow L\in d\quad (3)$$
			$$M=OJ\cap SC \Rightarrow \heva{&M \in OJ\subset (AJO)\Rightarrow M\in(AJO)\\&M \in SC\subset (SAC)\Rightarrow M\in(SAC)}\Rightarrow M\in d\quad (4)$$
			Từ (1),(2),(3),(4) suy ra $A$, $K$, $L$, $M$ cùng thuộc $d$ $\Rightarrow$ bốn điểm $A$, $K$, $L$, $M$ thẳng hàng (đpcm).
		\end{enumerate}
	}
\end{bt}
\begin{bt}%[1K4K0-6]%[1H2K1-5]
	Cho tứ giác $ABCD$ có các cạnh đối đôi một không song song và điểm $S\not\in(ABCD)$.
	Lấy điểm $I$ thuộc cạnh $AD$, lấy điểm $J$ thuộc cạnh $SB$.
	\begin{listEX}[2]
		\item Tìm $K=IJ\cap (SAC)$
		\item Tìm $L=DJ\cap (SAC)$
		\item* Gọi $O=AD\cap BC$, $M=OJ\cap SC$. Chứng minh rằng: $K$, $L$, $M$ thẳng hàng.
	\end{listEX}
	\loigiai{ 
		\immini{		
			\begin{enumerate}
				\item Chọn $(SBI)$ chứa $IJ$. Gọi $H=AC\cap BI$ \\ $\Rightarrow$ $(SBI)\cap (SAC)=SH$.\\
				Gọi $IJ\cap SG=K$ suy ra $IJ\cap(SAC)=K$.
				\item Chọn $(SBD)$ chứa $DJ$. Gọi $G=AC\cap BD$\\ $\Rightarrow$ $(SBD)\cap (SAC)=SG$.\\
				Gọi $DJ\cap SH=L$ suy ra $DJ\cap(SAC)=L$.
				\item Gọi đường thẳng $d$ là giao tuyến của hai mặt phẳng $(SAC)$ và $(OMD)$.\\
				Ta có 
				$$K=IJ\cap (SAC) \Rightarrow \heva{&K\in IJ\subset (OMD)\Rightarrow K\in(OMD)\\&K \in(SAC)}\Rightarrow K\in d\quad (1)$$
				$$L=DJ\cap (SAC) \Rightarrow \heva{&L \in DJ\subset (OMD)\Rightarrow L\in(OMD)\\&L \in (SAC)}\Rightarrow L\in d\quad (2)$$
				$$M=OJ\cap SC \Rightarrow \heva{&M \in OJ\subset (OMD)\Rightarrow M\in(OMD)\\&M \in SC\subset (SAC)\Rightarrow M\in(SAC)}\Rightarrow M\in d.\quad (3)$$		
		\end{enumerate}	}{	\begin{tikzpicture}[scale=0.8, font=\footnotesize, line join=round, line cap=round, >=stealth]
				\path
				(0,0) coordinate (A)node[left]{$A$}
				($(A)+(0:4)$) coordinate (B)node[right]{$B$}
				($(A)+(-40:4)$) coordinate (C)node[below]{$C$}
				($(A)+(-60:2)$) coordinate (D)node[left]{$D$}
				($(A)+(80:4)$) coordinate (S)node[left]{$S$}
				($(B)!2/3!(S)$) coordinate (J)node[left]{$J$}
				($(D)!1/3!(A)$) coordinate (I)node[left]{$I$}
				;
				\path[name path=a] (B)--(I);
				\path[name path=b] (A)--(C);
				\path [name intersections={of=a and b,by=H}];
				\path(H) coordinate (H)node[below]{$H$};
				\path[name path=c] (B)--(D);
				\path[name path=b] (A)--(C);
				\path [name intersections={of=c and b,by=G}];
				\path(G) coordinate (G)node[below]{$G$};
				\path(H) coordinate (H)node[below]{$H$};
				\path[name path=d] (S)--(H);
				\path[name path=e] (I)--(J);
				\path [name intersections={of=d and e,by=K}];
				\path(K) coordinate (K)node[above left]{$K$};
				\path[name path=f] (S)--(G);
				\path[name path=g] (D)--(J);
				\path [name intersections={of=f and g,by=L}];
				\path(L) coordinate (L)node[above right]{$L$};
				\path
				($(D)!2!180:(A)$) coordinate (Dt)
				($(C)!2!180:(B)$) coordinate (Ct)
				;
				\path[name path=h] (D)--(Dt);
				\path[name path=i] (C)--(Ct);
				\path [name intersections={of=h and i,by=O}];
				\path(O) coordinate (O)node[right]{$O$};
				\path[name path=j] (S)--(C);
				\path[name path=k] (O)--(J);
				\path [name intersections={of=j and k,by=M}];
				\path(M) coordinate (M)node[right]{$M$};
				\draw[red, thick][dashed](A)--(K)--(L)--(M);
				\draw[thick](S)--(A)--(D) (C)--(B)--(S)--(D) (S)--(C)--(O)--(D) (O)--(J);		
				\draw[thick][dashed](A)--(B) (I)--(J) (I)--(B) (A)--(C) (B)--(D) (S)--(H) (S)--(G) (D)--(J) (D)--(C);
				\foreach\p in{J,I,H,G,K,L,O,M}\draw[thick][fill=black](\p)circle(1pt);	
		\end{tikzpicture}}
		Từ (1),(2),(3) suy ra $K$, $L$, $M$ cùng thuộc $d$ nên chúng thẳng hàng (đpcm).
	}
\end{bt}
\begin{bt}%[1K4K0-6]%[1H2B1-3]%[1H2K1-5]
	Cho hình chóp $S.ABCD$ có đáy $ABCD$ là hình bình hành tâm $O$. Gọi $M$, $N$ lần lượt
	là trung điểm của $SA$, $SC$.
	\begin{enumerate}
		\item Tìm giao tuyến của $(BMN)$ với các mặt phẳng $(SAB)$ và $(SBC)$.
		\item Tìm $I=SO\cap (BMN)$ và $K=SD\cap (BMN)$
		\item Tìm $E=AD\cap (BMN)$ và $F=CD\cap (BMN)$
		\item Chứng minh rằng ba điểm $B$, $E$, $F$ thẳng hàng.
	\end{enumerate}
	\loigiai{
		\immini
		{a)Ta có 
			$$\heva{&B\in(BMN)\\&B\in(SAB)}\Rightarrow B\in(BMN)\cap(SAB)\quad(1)$$
			$$\heva{&M\in(BMN)\\&M\in SA\subset(SAB)}\Rightarrow M\in(BMN)\cap(SAB)\quad(2)$$
			Từ (1),(2) suy ra $(BMN)\cap(SAB)=BM$.
		}{	\begin{tikzpicture}[scale=0.8, font=\footnotesize, line join=round, line cap=round, >=stealth]
				\path
				(0,0) coordinate (A)node[below]{$A$}
				($(A)+(-150:2)$) coordinate (B)node[below]{$B$}
				($(B)+(0:4)$) coordinate (C)node[below]{$C$}
				($(A)+(0:4)$) coordinate (D)node[right]{$D$}
				($(B)+(85:5)$) coordinate (S)node[left]{$S$}
				($(S)!1/2!(A)$) coordinate (M)node[left]{$M$}
				($(S)!1/2!(C)$) coordinate (N)node[right]{$N$}
				;
				\path[name path=a] (A)--(C);
				\path[name path=b] (B)--(D);
				\path [name intersections={of=a and b,by=O}];
				\path(O) coordinate (O)node[below]{$O$};
				\path[name path=c] (S)--(O);
				\path[name path=d] (M)--(N);
				\path [name intersections={of=c and d,by=I}];
				\path(I) coordinate (I)node[right]{$I$};
				\path
				($(I)!1!180:(B)$) coordinate (It)
				;
				\path[name path=e] (I)--(It);
				\path[name path=f] (S)--(D);
				\path [name intersections={of=e and f,by=K}];
				\path(K) coordinate (K)node[right]{$K$};
				\path
				($(N)!7!180:(K)$) coordinate (Nt)
				($(C)!2!180:(D)$) coordinate (Ct)
				($(M)!4!180:(K)$) coordinate (Mt)
				($(A)!1!180:(D)$) coordinate (At)
				;
				\path[name path=g] (N)--(Nt);
				\path[name path=h] (C)--(Ct);
				\path [name intersections={of=g and h,by=F}];
				\path(F) coordinate (F)node[below]{$F$};
				\path[name path=i] (M)--(Mt);
				\path[name path=k] (A)--(At);
				\path [name intersections={of=i and k,by=E}];
				\path(E) coordinate (E)node[left]{$E$};
				
				\coordinate (P) at (intersection cs:first line={(S)--(B)}, second line={(E)--(K)});
				\coordinate (Q) at (intersection cs:first line={(S)--(B)}, second line={(A)--(E)});
				\coordinate (T) at (intersection cs:first line={(B)--(C)}, second line={(K)--(F)});
				
				\draw[thick](S)--(B) (C)--(D)--(S)--(C) (N)--(B) (K)--(F)--(C) (P)--(E)--(F);
				\draw[thick][dashed](S)--(A)--(B)--(C) (Q)--(D) (E)--(A)--(C) (B)--(D) (B)--(M)--(N) (S)--(O) (B)--(I)--(K) (K)--(P);
				\foreach\p in{A,B,C,D,S,M,N,O,I,K,F,E}\draw[thick][fill=black](\p)circle(1pt);
		\end{tikzpicture}}
		
		Ta có 
		$$\heva{&B\in(BMN)\\&B\in(SBC)}\Rightarrow B\in(BMN)\cap(SBC).\quad(1)$$
		$$\heva{&N\in(BMN)\\&N\in SC\subset(SBC)}\Rightarrow N\in(BMN)\cap(SBC).\quad(2)$$
		Từ (1),(2) suy ra $(BMN)\cap(SBC)=BN$.
		\begin{enumerate}[b)]
			\item Trong $(SAC)$ gọi $I=SO\cap MN$ mà $MN\subset(BMN)$, suy ra $I=SO\cap(BMN)$.\\
			Chọn $(SBD)$ chứa $SD$, ta có $(SBD)\cap(BMN)=BI$. Kẻ $BI\cap SD=K$. Suy ra $K=SD\cap(BMN)$.
			\item Trong $(SAD)$ gọi $E=KM\cap AD$ mà $KM\subset(BMN)$, suy ra $E=AD\cap(BMN)$.\\
			Trong $(SCD)$ gọi $F=CD\cap KN$ mà $KN\subset(BMN)$, suy ra $F=CD\cap(BMN)$.
			\item Gọi đường thẳng $d$ là giao tuyến của hai mặt phẳng $(KEF)$ và $(DEF)$.\\
			Ta có 
			$$\heva{E\in(KEF)\\E \in(DEF)}\Rightarrow E\in d\quad (1)$$
			$$\heva{F\in(KEF)\\F \in(DEF)}\Rightarrow F\in d\quad (2)$$
			$$B=BD\cap BK \Rightarrow \heva{B \in BD\subset (DEF)\Rightarrow B\in(DEF)\\B \in BK\subset (KEF)\Rightarrow B\in(KEF)}\Rightarrow B\in d\quad (3)$$
			Từ (1),(2),(3) suy ra $B$, $E$, $F$, cùng thuộc $d$ $\Rightarrow$ ba điểm $B$, $E$, $F$ thẳng hàng (đpcm).
	\end{enumerate}	}
\end{bt}

\begin{bt} %[1K4K0-6]
	Cho hình chóp $S.ABCD$. Gọi $M, N$ lần lượt là hai điểm nằm trên hai cạnh $BC$ và $SD$.
	\begin{enumerate}
		\item Tìm giao điểm $I$ của $BN$ và $(SAC)$.
		\item Tìm giao điểm $J$ của $MN$ và $(SAC)$.
		\item Chứng minh $I, J, C$ thẳng hàng.
		\item Xác định thiết diện của mặt phẳng $(BCN)$ với hình chóp.
	\end{enumerate}
	\loigiai{ 
		\begin{center}
			\begin{tikzpicture}[scale=1, font=\footnotesize, line join=round, line cap=round, >=stealth]
				\coordinate[label= above left:$A$] (A) at (0,0);
				\coordinate[label=below:$B$] (B) at (2,-3);
				\coordinate[label=below:$C$] (C) at (5,-2);
				\coordinate[label=right:$D$] (D) at (6,0);
				\coordinate[label=above:$S$] (S) at (2,4);
				\coordinate[label=below:$M$] (M) at ($(B)!1/3!(C)$);
				\coordinate[label=above:$N$] (N) at ($(S)!1/3!(D)$);
				\coordinate[label=above right:$O$] (O) at ($(C)!1/4!(A)$);
				\coordinate[label=left:$I$] (I) at ($(S)!1/2!(O)$);
				\coordinate[label=below:$H$] (H) at ($(C)!2/3!(O)$);
				\coordinate[label=above :$J$] (J) at ($(S)!2/3!(H)$);
				\coordinate[label=right:$K$] (K) at (12,0);
				\coordinate[label=left:$P$] (P) at ($(S)!1/5!(A)$);
				\path[name path=ac] (A)--(C);
				\path[name path=bd] (B)--(D);
				\path[name intersections={of= ac and bd,by=O}]; 
				\draw[thick] (S)--(A)--(B)--(C)(S) (B)--(S)--(C)--(K)(C)--(N)(B)--(P)(S)--(N)--(K);
				\draw[thick][dashed] (A)--(D)--(B)--(N)(A)--(C)(S)--(O)(M)--(D)--(K) (S)--(H)(M)--(N)(C)--(D)--(N)--(P);
				\foreach \diem in {A,B,C,D,S,N,M,O,K,P}	\fill (\diem)circle(1.0pt);
			\end{tikzpicture}
		\end{center}
		\begin{enumerate}
			\item Tìm giao điểm $I$ của $BN$ và $(SAC)$.
			\begin{itemize}
				\item $S \in (SAC) \cap (SBD)$.
				\item Trong $(ABCD)$, gọi $O=AC \cap BD$.\\
				$\Rightarrow \heva{&O \in AC \subset (SAC)\\&O \in BD \subset (SBD)} \Rightarrow O \in (SAC) \cap \ (SBD)$.\\
				$\Rightarrow SO=(SAC) \cap (SBD)$.\\
				\item Trong $(SBD)$, gọi $I=BN \cap SO \Rightarrow I \in SO \subset (SAC)$.
			\end{itemize}
			Vậy $I=	BN \cap (SAC)$.
			\item Tìm giao điểm $J$ của $MN$ và $(SAC)$.
			\begin{itemize}
				\item $ S \in (SMD) \cap (SAC)$.
				\item  Trong $(ABCD)$, gọi $H=MD \cap AC$.\\
				$ \Rightarrow \heva{&H \in MD \subset (SMD)\\&H \in AC \subset (SAC)}\Rightarrow H \in (SMD) \subset (SAC)$.\\
				$\Rightarrow SH = (SMD) \cap (SAC)$.
				\item  Trong $(SMD)$, gọi $J = SH \cap MN \Rightarrow J \in SH \subset (SAC)$.
			\end{itemize}
			Vậy $J =MN \cap (SAC)$. 
			\item Chứng minh $I, J, C$ thẳng hàng.
			\begin{itemize}
				\item $\heva{&I \in (SAC) \\& I \in BN \subset (BCN)} \Rightarrow I \in (SAC) \cap (BCN)$.
				\item $\heva{&J \in (SAC) \\& J \in MN \subset (BCN)} \Rightarrow J \in (SAC) \cap (BCN)$.
				\item $\heva{&C \in (SAC) \\& C \in  (BCN)} \Rightarrow C \in (SAC) \cap (BCN)$.
			\end{itemize}
			$\Rightarrow I, J, C \in (SAC) \cap (BCN)$.\\
			Vậy $I, J, C$ thẳng hàng.
			\item Xác định thiết diện của mặt phẳng $(BCN)$ với hình chóp.
			\begin{itemize}
				\item $N \in (BCN) \cap (SAD)$.
				\item  Trong $(ABCD)$, gọi $K = BC \cap AD \Rightarrow \heva{&K \in BC \subset (BCN)\\&K \in AD \subset (SAD)} \Rightarrow K \in (BCN) \cap (SAD)$.\\
				$\Rightarrow NK =(BCN) \cap (SAD)$.
				\item Trong $(SAD)$, gọi $P =NK \cap SA \Rightarrow P = SA \cap (BCN)$.
			\end{itemize}
			Vậy thiết diện của mặt phẳng $(BCN)$ với hình chóp $S.ABCD$ là tứ giác $BCNP$.
		\end{enumerate}
	}
\end{bt}
%------------Bài 8------------
\begin{bt}%[1K4K0-6]
	Cho tứ diện $ABCD$ có $K$ là trung điểm của $AB$. Lấy $I, J$ lần lượt thuộc $AC,BD$ sao cho $IA=2IC$ và $JB=3JD$.
	\begin{enumerate}
		\item Tìm giao điểm $E$ của $AD$ và $(IJK)$.
		\item Tìm giao tuyến $d$ của $(IJK)$ và $(BCD)$.
		\item Gọi $O$ là giao điểm của $d$ với $CD$. Chứng minh $I, O, E$ thẳng hàng.
		\item Tính các tỉ số $\dfrac{OI}{OE}$ và $\dfrac{OC}{OD}$.
	\end{enumerate}
	\loigiai{
		\begin{center}
			\begin{tikzpicture}[scale=1, font=\footnotesize, line join=round, line cap=round, >=stealth]
				\coordinate[label= above left:$B$] (B) at (0,0);
				\coordinate[label=above:$A$] (A) at (2,5);
				\coordinate[label=below:$C$] (C) at (2,-3);
				\coordinate[label=right:$D$] (D) at (6,0);
				\coordinate[label=left:$K$] (K) at ($(B)!1/2!(A)$);
				\coordinate[label=left:$I$] (I) at ($(A)!2/3!(C)$);
				\coordinate[label=above right:$J$] (J) at ($(B)!3/4!(D)$);
				\coordinate[label=below:$E$] (E) at ($(A)!1.5!(D)$);
				\coordinate[label=below:$F$] (F) at ($(B)!2!(C)$);
				\coordinate[label=left:$O$] (O) at (intersection cs:first line={(I)--(E)}, second line={(C)--(D)});
				
				\draw[thick] (A)--(B)--(C)--(A)--(D)(A)--(C)--(F)--(I)--(K)(D)--(E)(I)--(O)--(E) (D)--(O)--(F);
				\draw[thick][dashed] (B)--(D) (O)--(C)(K)--(J)--(E)(O)--(J);
				\foreach \diem in {A,B,C,D,I,J,K,E,F,O}	\fill (\diem)circle(1.0pt);
				
			\end{tikzpicture}
		\end{center}
		
		\begin{enumerate}
			\item Tìm giao điểm $E$ của $AD$ và $(IJK)$.
			\begin{itemize}
				\item $\dfrac{BK}{BA}=\dfrac{1}{2}; \dfrac{BJ}{BD}=\dfrac{3}{4} \Rightarrow \dfrac{BK}{BA}\neq \dfrac{BJ}{BD} \Rightarrow KJ$ không song song với $AD$.
				\item 	Trong $(ABD$), gọi $E$ là giao điểm của $KJ$ và $AD$.\\
				$\Rightarrow \heva{&E \in KJ \subset (IJK)\\&E \in AD}$
			\end{itemize}
			Vậy $E =AD \cap (IJK)$.
			\item Tìm giao tuyến $d$ của $(IJK)$ và $(BCD)$.
			\begin{itemize}
				\item  $ J \in (IJK) \cap (BCD)$.
				\item 	$\dfrac{AK}{AB} \neq \dfrac{AI}{AC} \Rightarrow KI$ không song song $BC$.\\
				Trong $(ABC)$, gọi $F=IK \cap BC$.\\
				$ \Rightarrow \heva{& F\in IK \subset (IJK)\\&F \in BC \subset (BCD)}	$\\
				$\Rightarrow F \in (IJK) \cap (BCD)$.
			\end{itemize}
			Vậy $ d = FJ =(IJK) \cap (BCD)$.
			\item Gọi $O$ là giao điểm của $d$ với $CD$. Chứng minh $I, O, E$ thẳng hàng.
			\begin{itemize}
				\item  $ \heva{&I \in (IJK)\\& I \in AC \subset (ACD)} \Rightarrow I \in (IJK) \cap (ACD)$.
				\item 	$\heva{&O \in FJ \subset (IJK)\\&O \in CD \subset (ACD)} \Rightarrow O \in (IJK) \cap (ACD)$.
				\item $\heva{&E \in KJ \subset (IJK)\\&E \in AD \subset (ACD)} \Rightarrow E \in (IJK) \cap (ACD)$.
			\end{itemize}
			Vậy $I, O, E$ thẳng hàng.
			\item Tính các tỉ số $\dfrac{OI}{OE}$ và $\dfrac{OC}{OD}$.
			\begin{itemize}
				\item Áp dụng định lý Menelaus trong tam giác $ABC$ với ba điểm thẳng hàng $K, I, F$ ta có$\colon$\\ $\dfrac{AK}{KB}\cdot \dfrac{BF}{FC}\cdot \dfrac{CI}{IA}=1 \Rightarrow \dfrac{BF}{FC}=2$.\\
				Áp dụng định lý Menelaus trong tam giác $DBC$ với ba điểm thẳng hàng $J, O, F$ ta có$\colon$\\
				$\dfrac{DJ}{JB}\cdot \dfrac{BF}{FC}\cdot \dfrac{CO}{OD}= 1 \Rightarrow \dfrac{CO}{OD}=\dfrac{3}{2}$.\\
				Vậy $\dfrac{OC}{OD}=\dfrac{3}{2}$.
				\item 	Áp dụng định lý Menelaus trong tam giác $ABD$ với ba điểm thẳng hàng $K, J, E$ ta có$\colon$\\
				$\dfrac{BK}{KA}\cdot \dfrac{AE}{ED}\cdot \dfrac{DJ}{JB}= 1 \Rightarrow \dfrac{AE}{ED}=3$.\\
				Áp dụng định lý Menelaus trong tam giác $AIE$ với ba điểm thẳng hàng $C, O, D$ ta có$\colon$\\
				$\dfrac{ED}{DA}\cdot \dfrac{AC}{CI} \cdot \dfrac{IO}{OE}= 1 \Rightarrow \dfrac{IO}{OE}=\dfrac{2}{3}$.\\
				
			\end{itemize}	
		\end{enumerate}
	}
\end{bt}
%------------Bài 9
\begin{bt}%[1K4K0-6]
	Cho hình chóp $S.ABCD$ có đáy $ABCD$ là hình thang, $AD$ là đáy lớn và $AD=2BC$. Gọi $M, N$ lần lượt là trung điểm của $SB, SC$; $O$ là giao điểm của $AC$ và $BD$.
	\begin{enumerate}
		\item Tìm giao tuyến của $(ABN)$ và $(SCD)$.
		\item Tìm giao điểm $P$ của $DN$ và $(SAB)$.
		\item Gọi $K$ là giao điểm của $AN$ và $DM$. Chứng minh $S, K, O$ thẳng hàng. Đặt $KS=k\cdot KO$. Tìm $k$.
	\end{enumerate}
	\loigiai{
		\begin{enumerate}
			\item Tìm giao tuyến của $(ABN)$ và $(SCD)$.
			\immini{
				
				Ta có$\colon N \in (ABN) \cap (SCD) \,\,(1)$.\\
				Trong $(ABCD)$, gọi $I = AB \cap CD$.\\$ \Rightarrow I \in (ABN) \cap (SCD) \,\,(2)$.\\
				Từ $(1), (2)$ suy ra $IN = (ABN) \cap (SCD)$.
			}{
				\begin{tikzpicture}[scale=0.7, font=\footnotesize, line join=round, line cap=round, >=stealth]
					\coordinate[label= above left:$A$] (A) at (0,0);
					\coordinate[label=below:$B$] (B) at (-3,-2);
					\coordinate[label=below:$C$] (C) at (0,-2);
					\coordinate[label=right:$D$] (D) at (6,0);
					\coordinate[label=above:$S$] (S) at (2,4);
					\coordinate[label= above left:$I$] (I) at (-6,-4);
					\coordinate[label=above right:$N$] (N) at ($(S)!1/2!(C)$);
					\coordinate[label=above:$M$] (M) at ($(S)!1/2!(B)$); 
					\path[name path=ac] (A)--(C);
					\path[name path=bd] (B)--(D);
					\path[name intersections={of= ac and bd,by=O}]; 
					\coordinate[label=above left:$O$] (O) at (O);
					\draw[thick] (B)--(C)--(D)--(S)-- (B)--(I)--(C) (S)--(C)(B)--(N);
					\draw[thick][dashed] (S)--(A)--(D)--(B)--(A)--(C)(N)--(A);
					\foreach \diem in {A,B,C,D,S,O, M,N,I}	\fill (\diem)circle(1.0pt);
					
				\end{tikzpicture}
			}
			\item Tìm giao điểm $P$ của $DN$ và $(SAB)$.
			\immini{
				
				Trong $(SCD)$, gọi $P = DN \cap SI$\\$ \Rightarrow \heva{&P \in DN \\&I \in SI \subset (SAB)}$.\\
				Vậy $P = DN \cap (SAB)$.	
			}{
				\begin{tikzpicture}[scale=0.7, font=\footnotesize, line join=round, line cap=round, >=stealth]
					\coordinate[label= above left:$A$] (A) at (0,0);
					\coordinate[label=below:$B$] (B) at (-3,-2);
					\coordinate[label=below:$C$] (C) at (0,-2);
					\coordinate[label=right:$D$] (D) at (6,0);
					\coordinate[label=above:$S$] (S) at (2,4);
					\coordinate[label= above left:$I$] (I) at (-6,-4);
					\coordinate[label= above left:$P$] (P) at($(S)!1/3!(I)$);;
					\coordinate[label=above right:$N$] (N) at ($(S)!1/2!(C)$);
					\coordinate[label=below:$M$] (M) at ($(S)!1/2!(B)$); 
					\path[name path=ac] (A)--(C);
					\path[name path=bd] (B)--(D);
					\path[name intersections={of= ac and bd,by=O}]; 
					\coordinate[label=above left:$O$] (O) at (O);
					\draw[thick] (I)--(C)--(D)--(S) (S)--(C)(D)--(N) (I)--(S)(N)--(P);
					\draw[thick][dashed] (S)--(A)--(D)--(B)--(A)--(C)--(B)--(N)--(A)(S)--(B)--(I);
					\foreach \diem in {A,B,C,D,S,O, M,N,I,P}	\fill (\diem)circle(1.0pt);
					
				\end{tikzpicture}
				
			}		
			\item Gọi $K$ là giao điểm của $AN$ và $DM$. Chứng minh $S, K, O$ thẳng hàng. Đặt $KS=k.KO$. Tìm $k$.
			\begin{center}
				\begin{tikzpicture}[scale=1, font=\footnotesize, line join=round, line cap=round, >=stealth]
					\coordinate[label= above left:$A$] (A) at (0,0);
					\coordinate[label=below:$B$] (B) at (-3,-2);
					\coordinate[label=below:$C$] (C) at (0,-2);
					\coordinate[label=right:$D$] (D) at (6,0);
					\coordinate[label=above:$S$] (S) at (2,4);
					\coordinate[label= above left:$I$] (I) at (-6,-4);
					\coordinate[label= above left:$P$] (P) at($(S)!1/3!(I)$);
					\coordinate[label= below right:$K$] (K) at($(S)!3/5!(O)$);
					\coordinate[label=above right:$H$] (H) at ($(S)!5/6!(C)$);
					\coordinate[label=above right:$N$] (N) at ($(S)!1/2!(C)$);
					\coordinate[label=below:$M$] (M) at ($(S)!1/2!(B)$); 
					\path[name path=ac] (A)--(C);
					\path[name path=bd] (B)--(D);
					\path[name intersections={of= ac and bd,by=O}]; 
					\coordinate[label=above left:$O$] (O) at (O);
					\draw[thick] (I)--(C)--(D)--(S) (S)--(C)(D)--(N) (I)--(S);
					\draw[thick][dashed] (O)--(S)--(A)--(D)--(B)--(A)--(C)(N)--(A)(S)--(B)--(I)(D)--(M)(O)--(H)(B)--(C);
					\foreach \diem in {A,B,C,D,S,O, M,N,I,P,H}	\fill (\diem)circle(1.0pt);
					
				\end{tikzpicture}
			\end{center}
			\begin{itemize}
				\item $S \in (SAC) \cap (SBD)$.
				\item $\heva{&K \in AN \subset (SAC)\\&K \in DM \subset (SBD)} \Rightarrow K \in (SAC) \cap (SBD)$.
				\item $\heva{&O \in AC \subset (SAC)\\&O \in DB \subset (SBD)} \Rightarrow O \in (SAC) \cap (SBD)$.\\
				Vậy $S,K,O$ thẳng hàng.
				\item $ABCD$ là hình thang có $AD = 2BC \Rightarrow \dfrac{BC}{AD}=\dfrac{OC}{OA}=\dfrac{1}{2}$.\\
				Trong  $\Delta SAC$, kẻ $OH \parallel AN \Rightarrow \dfrac{CH}{CN}=\dfrac{1}{3} \Rightarrow \dfrac{SK}{SO}=\dfrac{SN}{SH}=\dfrac{3}{5}$.\\
				Vậy $k =\dfrac{3}{5}$.
			\end{itemize}
		\end{enumerate}
	}
\end{bt}
%-------------Bài 10
\begin{bt}%[1K4K0-6]
	Cho hình chóp $S.ABCD$ có đáy $ABCD$ là hình bình hành tâm $O$. Gọi $M, N$ lần lượt là trung điểm của $SA,SC$. Gọi $(P)$ là mặt phẳng qua $M, N$ và $B$.
	\begin{enumerate}
		\item Tìm giao tuyến của $(P)$ với các mặt phẳng $(SAB), (SBC), (SCD), (SAD)$.
		\item Tìm $E=DA \cap (P), F=DC \cap (P)$.
		\item Chứng tỏ rằng $E, F, B$ thẳng hàng.
	\end{enumerate}
	\loigiai{
		\begin{enumerate}
			\item Tìm giao tuyến của $(P)$ với các mặt phẳng $(SAB), (SBC), (SCD), (SAD)$.
			\begin{center}
				\begin{tikzpicture}[scale=0.8, font=\footnotesize, line join=round, line cap=round, >=stealth]
					\coordinate[label= above left:$A$] (A) at (0,0);
					\coordinate[label=below:$B$] (B) at (-3,-2);
					\coordinate[label=below:$C$] (C) at (3,-2);
					\coordinate[label=right:$D$] (D) at (6,0);
					\coordinate[label=above:$S$] (S) at (2,4);
					\coordinate[label=above:$M$] (M) at ($(A)!1/2!(S)$);
					\coordinate[label=above right:$N$] (N) at ($(S)!1/2!(C)$);
					\coordinate[label=below:$O$] (O) at ($(A)!1/2!(C)$); 
					\coordinate[label=left:$I$] (I) at ($(S)!1/2!(O)$);
					\coordinate[label=above right:$K$] (K) at ($(S)!1/3!(D)$);
					\draw[thick] (B)--(C)--(D)--(S)-- (B)--(N) (S)--(C);
					\draw[thick][dashed] (S)--(A)--(D)--(B)--(A)--(C) (B)--(M)--(N) (O)--(S)(B)--(I)--(K);
					\foreach \diem in {A,B,C,D,S,M,N,O,I,K}	\fill (\diem)circle(1.0pt);
				\end{tikzpicture}
			\end{center}
			\begin{itemize}
				\item $MB = (P) \cap (SAB)$.
				\item $NB = (P)  \cap (SBC)$.
				\item Tìm giao tuyến của $(P)$ và $(SCD)$.
				\begin{itemize}
					\item $N \in (P) \cap (SCD) $.
					\item Trong $(SAC)$, gọi $I = SO \cap MN; K = BI \cap SD$.\\
					$\Rightarrow \heva{&K \in BI \subset (P)\\&K \in SD \subset (SCD)} \Rightarrow K \in (P) \cap (SCD) $.
				\end{itemize}	
				Vậy $NK = (MNB) \cap (SCD)$.
				\item Tìm giao tuyến của $(P)$ và $(SAD)$.
				\begin{itemize}
					\item $M \in (P) \cap (SAD)$.
					\item $K \in (P) \cap (SAD)$.\\
					Vậy $MK = (P) \cap (SAD)$.
				\end{itemize}
			\end{itemize}
			
			\item Tìm $ E=DA \cap (P), F=DC \cap (P)$.
			\begin{center}
				\begin{tikzpicture}[scale=0.8, font=\footnotesize, line join=round, line cap=round, >=stealth]
					\coordinate[label= above left:$A$] (A) at (0,0);
					\coordinate[label=below:$B$] (B) at (-3,-2);
					\coordinate[label=below:$C$] (C) at (3,-2);
					\coordinate[label=right:$D$] (D) at (6,0);
					\coordinate[label=above:$S$] (S) at (2,4);
					\coordinate[label=above:$M$] (M) at ($(A)!1/2!(S)$);
					\coordinate[label=above right:$N$] (N) at ($(S)!1/2!(C)$);
					\coordinate[label=below:$O$] (O) at ($(A)!1/2!(C)$); 
					\coordinate[label=left:$I$] (I) at ($(S)!1/2!(O)$);
					\coordinate[label=above right:$K$] (K) at ($(S)!1/3!(D)$);
					\coordinate[label=above right:$E$] (E) at (-6,0);
					\coordinate[label= below:$F$] (F) at (0,-4);
					\draw[thick] (C)--(D)--(S)-- (B)--(N) (S)--(C)(M)--(E)(K)--(N)--(F)--(C) (E)--(F)--(B);
					\draw[thick][dashed] (S)--(A)--(D)--(B)--(A)--(C) (B)--(M)--(N) (O)--(S)(B)--(I)--(K)--(M)(E)--(A)(B)--(C);
					\foreach \diem in {A,B,C,D,S,M,N,O,I,K,E,F}	\fill (\diem)circle(1.0pt);
				\end{tikzpicture}
			\end{center}
			\begin{itemize}
				\item Gọi $E = MK \cap AD \Rightarrow E = AD \cap (P)$.
				\item Gọi $F = NK \cap CD \Rightarrow F = CD \cap (P)$.
			\end{itemize}
			\item Chứng tỏ rằng $E, F, B$ thẳng hàng.
			\begin{itemize}
				\item $E \in (P) \cap (ABCD)$.
				\item $F \in (P) \cap (ABCD)$.
				\item $B \in (P) \cap (ABCD)$.
			\end{itemize}
			Vậy $E, F, B$ thẳng hàng.
		\end{enumerate}
	}
\end{bt}
%--------------Bài 11
\begin{bt}%[1K4K0-6]
	Cho hình chóp $S.ABCD$ có đáy $ABCD$ là tứ giác có các cặp cạnh đối không song song nhau. Gọi $M, E$ lần lượt là trung điểm của $SA, AC$ và $F \in CD$ sao cho $CD=3CF$.
	\begin{enumerate}
		\item Tìm giao tuyến của $(SAB)$ và $(SCD)$.
		\item Tìm giao điểm $N$ của $SD$ và $(MEF)$. Đặt $NS=k.ND$. Tìm $k$.
		\item Gọi $H=SE \cap CM$ và $K=MF \cap NE$. Chứng minh $D, H, K$ thẳng hàng.
		\item Tính các tỉ số $\dfrac{HM}{HC}; \dfrac{HS}{HE}; \dfrac{KM}{KF}; \dfrac{KN}{KE}; \dfrac{KH}{KD}$.
	\end{enumerate}
	\loigiai{
		\begin{center}
			\begin{tikzpicture}[scale=1, font=\footnotesize, line join=round, line cap=round, >=stealth]
				\coordinate[label= above left:$A$] (A) at (0,0);
				
				\coordinate[label=below:$C$] (C) at (5,-2);
				\coordinate[label=right:$D$] (D) at (6,0);
				\coordinate[label=above:$S$] (S) at (2,4);
				\coordinate[label=above left:$M$] (M) at ($(S)!1/2!(A)$);
				\coordinate[label=below:$E$] (E) at ($(A)!1/2!(C)$);
				\coordinate[label=above right:$O$] (O) at (3.5,-5);
				\coordinate[label=left:$B$] (B) at ($(A)!0.6!(O)$);
				\coordinate[label=right:$F$] (F) at ($(C)!1/3!(D)$);
				\coordinate[label=left:$I$] (I) at (-6,0);
				\coordinate[label=right:$N$] (N) at ($(S)!1/3!(D)$);
				\coordinate[label=left:$H$] (H) at ($(M)!1/3!(C)$);
				\coordinate[label=above right:$K$] (K) at ($(E)!3/7!(N)$);
				\coordinate (P) at (intersection cs:first line={(A)--(B)}, second line={(I)--(E)});
				\draw[thick] (S)--(M) (S)--(B) (C)--(D)--(S) (B)--(S)--(C) (O)--(C)(I)--(M)--(P) (I)--(P) (P)--(O)--(S) (M)--(O);
				\draw[thick][dashed] (I)--(A)--(D)--(B)--(C)(A)--(C)(S)--(O) (M)--(N)(S)--(E)--(N)(C)--(H)--(M)--(F)(H)--(K)--(D) (A)--(M)
				(P)--(F) (M)--(D)(P)--(A);
				\foreach \diem in {A,B,C,D,S,O,M,E,F,I,N,H,K}	\fill (\diem)circle(1.0pt);
				\fill[cyan,opacity=0.5] (M)--(D)--(O)--cycle;	
			\end{tikzpicture}
		\end{center}
		\begin{enumerate}
			\item Tìm giao tuyến của $(SAB)$ và $(SCD)$.\\
			$S \in (SAB) \cap (SCD)$.\\
			Trong $(ABCD)$, gọi $O=AB \cap CD \Rightarrow \heva{&O \in AB \subset (SAB)\\&S \in CD \subset (SCD)} \Rightarrow O \in (SAB)\cap (SCD)$.\\
			Vậy $SO =(SAB) \cap (SCD)$.
			\item Tìm giao điểm $N$ của $SD$ và $(MEF)$. Đặt $NS=k\cdot ND$. Tìm $k$.\\
			Ta có $\colon \dfrac{CE}{CA} \neq \dfrac{CF}{CD} \Rightarrow EF$ không song song với $AD$.\\
			Gọi $I = EF \cap AD \Rightarrow I \in (SAD) \cap (MEF)$.\\
			Mặt khác, $M \in (SAD) \cap (MEF)$.\\
			$\Rightarrow IM = (SAD)\cap (MEF)$. \\
			Gọi $N  = IM \cap SD \Rightarrow \heva{&N \in IM \subset (MEF)\\&N \in SD} \Rightarrow N = SD \cap (MEF)$.\\
			
			
			
			
			\item Gọi $H=SE \cap CM$ và $K=MF \cap NE$. Chứng minh $D, H, K$ thẳng hàng.
			\begin{itemize}
				\item $H=SE \cap CM \Rightarrow \heva{&H \in SE \subset (SED)\\&H \in CM \subset (CDM)} \Rightarrow H \in (SED) \cap (CDM)$.
				\item $K = MF \cap NE \Rightarrow \heva{&K \in MF \subset (CDM)\\&K \in NE \subset (SED)} \Rightarrow K \in (CDM) \cap (SED)$.
				\item $H \in (CDM) \cap (SED)$.	
			\end{itemize}
			Vậy $D, H, K$ thẳng hàng.
			\item Tính các tỉ số $\dfrac{HM}{HC}; \dfrac{HS}{HE}; \dfrac{KM}{KF}; \dfrac{KN}{KE}; \dfrac{KH}{KD}$.
			\begin{itemize}
				\item Xét tam giác $SAC$ có $CM$ và $SE$ là hai đường trung tuyến cắt nhau tại điểm $H$.\\
				$ \Rightarrow H$ là trọng tâm của $\Delta SAC$.\\
				$\Rightarrow \dfrac{HM}{HC}=\dfrac{1}{2}; \dfrac{HS}{HE}=2$.
				\item Áp dụng định lý Menelaus trong $\Delta IND$ với ba điểm thẳng hàng $M, A, S$ ta có$\colon$\\$\dfrac{IA}{AD}\cdot \dfrac{DS}{SN}\cdot \dfrac{NM}{MI}=1 \Rightarrow \dfrac{NM}{MI}=\dfrac{1}{3}$.\\
				Áp dụng định lý Menelaus trong $\Delta IFD$ với ba điểm thẳng hàng $C, A, E$ ta có$\colon$\\$\dfrac{IA}{AD}\cdot \dfrac{DC}{CF}\cdot \dfrac{FE}{EI}=1 \Rightarrow \dfrac{FE}{EI}=\dfrac{1}{3}$.\\
				\\Áp dụng định lý Menelaus trong $\Delta NIE$ với ba điểm thẳng hàng $M, H, F$ ta có$\colon$\\$\dfrac{NM}{MI}\cdot \dfrac{IF}{FE}\cdot \dfrac{EK}{KN}=1 \Rightarrow \dfrac{EK}{KN}=\dfrac{3}{4}$.\\
				Vậy $\dfrac{KN}{KE}=\dfrac{4}{3}$.
				\item 
				Áp dụng định lý Menelaus trong $\Delta IFM$ với ba điểm thẳng hàng $E, K, N$ ta có$\colon$\\$\dfrac{FE}{EI}\cdot \dfrac{IN}{NM}\cdot \dfrac{MK}{KF}=1 \Rightarrow \dfrac{MK}{KF}=\dfrac{3}{4}$.\\
				Vậy $\dfrac{KM}{KF}=\dfrac{3}{4}$.
				\item Áp dụng định lý Menelaus trong $\Delta SHD$ với ba điểm thẳng hàng $E, K, N$ ta có$\colon$\\$\dfrac{DN}{NS}\cdot \dfrac{SE}{EH}\cdot \dfrac{HK}{KD}=1 \Rightarrow \dfrac{HK}{KD}=\dfrac{1}{6}$.\\
			\end{itemize}
		\end{enumerate}
		
	}
\end{bt}

\begin{dang}{Chứng minh ba đường thẳng đồng quy}
\phuongphap	Để chứng minh ba đường thẳng $a, b, c$ đồng quy ta làm theo các bước sau:
	\begin{itemize}
		\item Chọn mặt phẳng $(P)$ chứa đường thẳng $a$ và $b$.
		\item Tìm mặt phẳng $(Q)$ chứa $a$ và $(R)$ chứa $b$ sao cho $(Q) \cap(R)=c \Rightarrow I \in c$.
	\end{itemize}
	\immini{Suy ra: $a, b, c$ đồng quy tại $I$.\\
		Nghĩa là: $\heva{&a \subset(P), b \subset(P), I=a \cap b \\ &a=(P) \cap(Q) \\ &b=(P) \cap(R) \\ &c=(Q) \cap(R)} \Rightarrow a, b, c \text{ đồng quy tại } I.$}{
		\begin{tikzpicture}[scale=0.6, font=\footnotesize, line join=round, line cap=round, >=stealth]
			\coordinate (A) at (0,0);
			\coordinate (B) at (0,5);
			\coordinate (C) at (4,-1);
			\coordinate	(D) at (4,4);
			\coordinate (E) at (-3,-2);
			\coordinate (F) at (-3,3);
			\coordinate [label=left:$I$] (M) at ($(B)!1/3!(A)$);
			\coordinate (N) at ($(E)!1/3!(A)$);
			\coordinate (K) at ($(C)!1/3!(A)$);
			\coordinate [label=left:$a$] (a) at ($(M)!1/2!(N)$);
			\coordinate [label=right:$b$] (b) at ($(M)!1/2!(K)$);
			\coordinate [label=left:$c$] (c) at ($(M)!1/2!(A)$);
			\draw[thick] (B)--(D)--(C)--(K);
			\draw[thick] (B)--(F)--(E)--(N);
			\draw[thick] (B)--(M);
			\draw[thick] (N)--(M)--(K)--(N);
			\draw[thick][dashed] (M)--(A)--(K);
			\draw[thick][dashed] (A)--(N);
			\draw[thick] pic ["$Q$", draw=black,fill=gray!20,angle eccentricity=.6,angle radius=20] {angle=E--F--B};
			\draw[thick] pic ["$P$", draw=black,fill=gray!20,angle eccentricity=.6,angle radius=20] {angle=M--K--N};
			\draw[thick] pic ["$R$", draw=black,fill=gray!20,angle eccentricity=.6,angle radius=20] {angle=B--D--C};
			%\coordinate[label=left:$I$](I) at (intersection of A--B and C--D); %giao điểm
	\end{tikzpicture}}
\end{dang}
\subsection*{VÍ DỤ MINH HỌA}
\begin{vd}%[1K4G0-6]
	Cho hình chóp $S.ABCD$ có $AB$ không song song $CD$. Gọi $M$ là trung điểm $SC$ và $O$ là giao điểm $AC$ với $BD$.
	\begin{enumerate}
		\item Tìm giao điểm $N$ của $SD$ với $(MAB)$.
		\item Chứng minh ba đường thẳng $SO, AM, BN$ đồng quy.
	\end{enumerate}
	\loigiai{
		\begin{enumerate}
			\item Tìm giao điểm $N$ của $SD$ với $(MAB)$.\\
			\immini{
				\begin{itemize}
					\item Chọn mặt phẳng phụ $(SCD)$ chứa $SD$.\\
					Xét $(SCD)$ và $(MAB)$:
					\item Trong $(ABCD)$, gọi $E = AB \cap CD$.\\
					Ta có: $\heva{&E \in AB, AB \subset(ABM) \Rightarrow E \in(ABM) \\ 
						&E \in CD, CD \subset(SCD) \Rightarrow E \in(SCD)}$\\
					$\Rightarrow E \in(ABM) \cap(SCD)$ \quad\quad $(1)$
					\item Mà $\heva{&M \in(ABM) \\ &M \in SC, SC \subset(SCD) \Rightarrow M \in(SCD)}$\\
					$\Rightarrow M \in(ABM) \cap (SCD)$.\quad\quad $(2)$\\
					Từ $(1),(2) \Rightarrow(ABM) \cap (SCD)= EM$.
					\item Trong $(SCD)$, gọi $N=SD \cap EM$.\\ Khi đó:
					$\heva{&N \in SD \\ &N \in EM, EM \subset(ABM)}$\\ 
					$\Rightarrow N = SD \cap (ABM)$.
			\end{itemize}}{
				\begin{tikzpicture}[scale=0.6, font=\footnotesize, line join=round, line cap=round, >=stealth]
					\coordinate[label=left:$A$] (A) at (-3,0);
					\coordinate[label=left:$B$] (B) at (-2,-2);
					\coordinate[label=below:$C$] (C) at (2,-4);
					\coordinate[label=right:$D$] (D) at (4,0);
					\coordinate[label=above:$S$] (S) at (0,6);
					%\coordinate (F) at (-3,3);
					\coordinate[label=right:$M$] (M) at ($(S)!1/2!(C)$);
					%\coordinate (N) at ($(E)!1/3!(A)$);
					%\coordinate (K) at ($(C)!1/3!(A)$);
					\draw[thick] (A)--(B) (C)--(D)--(S)--(A);
					\draw[thick] (B)--(S)--(C);
					\draw[thick] (S)--(D);
					\draw[thick][dashed] (B)--(C)--(A) (A)--(D);
					\draw[thick][dashed] (B)--(D) (A)--(M);
					\coordinate[label=below:$O$](O) at (intersection of A--C and B--D); %giao điểm
					\coordinate[label=below:$E$](E) at (intersection of A--B and C--D); %giao điểm
					\draw[thick] (E)--(B) (E)--(C);
					\coordinate[label=right:$N$](N) at (intersection of E--M and S--D); %giao điểm
					\draw[thick] (E)--(N);
					\draw[thick][dashed] (S)--(O) (B)--(N);
					\coordinate[label=above left :$I$](I) at (intersection of A--M and S--O); %giao điểm
					\foreach \diem in {A,B,C,D,N,M,E,O,S,I}	\fill (\diem)circle(1.5pt);
			\end{tikzpicture}}
			\item Chứng minh ba đường thẳng $SO, AM, BN$ đồng quy.
			\begin{nx}
				$AM \subset (SAC), BN \subset (SBD)$ nên ta quan tâm đến hai mặt phẳng này.
			\end{nx}
			Xét $(SAC)$ và $(SBD)$ có:
			\begin{itemize}
				\item $S \in(SAC) \cap (SBD) \quad\quad\quad\quad (3)$
				\item Ta có: $\heva{&O \in AC, AC \subset (SAC) \Rightarrow O \in(SAC) \\ 
					&O \in BD, BD \subset(SBD) \Rightarrow O \in(SBD)} \Rightarrow O \in (SAC) \cap (SBD)$ $\quad (4)$\\
				Từ $(3),(4) \Rightarrow(SAC) \cap(SBD)=SO.$
				\item Mặt khác, trong $(ABM),$ gọi $I= AM \cap BN$.\\
				$\heva{&I \in AM \subset(SAC) \\ &I \in BN \subset(SBD)} \Rightarrow I \in(SAC) \cap(SBD) \Rightarrow I \in SO.$
			\end{itemize}
			Do đó $SO, AM, BN$ đồng quy.
		\end{enumerate}
	}
\end{vd}
\begin{vd}%[1K4G0-6]
	Cho hình chóp $S.ABCD$ có đáy $ABCD$ là tứ giác lồi. Lấy $M$ trên cạnh $SC$. Gọi $N$ là giao điểm của $SB$ và $(ADM)$. Gọi $O$ là giao điểm $AC$ và $BD$. Chứng minh rằng $SO, AM, DN$ đồng qui.
	\loigiai{
		\begin{itemize}
			\item Tìm $N$ là giao điểm của $SB$ và $(ADM)$.
			\immini{
				\begin{itemize}
					\item Chọn mặt phẳng phụ $(SBC)$ chứa $SB$.\\
					Xét $(SBC)$ và $(ADM)$:
					\item Trong $(ABCD)$, gọi $E = AD \cap BC$.\\
					Ta có: $\heva{&E \in BC, BC \subset(SBC) \Rightarrow E \in(SBC) \\ 
						&E \in AD, AD \subset(ADM) \Rightarrow E \in(ADM)}$\\
					$\Rightarrow E \in(SBC) \cap (ADM)$ \quad\quad $(1)$
					\item Mà $\heva{&M \in(ADM) \\ &M \in SC, SC \subset(SBC) \Rightarrow M \in (SBC)}$\\
					$\Rightarrow M \in(SBC) \cap (ADM)$.\quad\quad $(2)$\\
					Từ $(1),(2) \Rightarrow(SBC) \cap (ADM)= EM$.
					\item Trong $(SBC)$, gọi $N= SB \cap EM$.\\ Khi đó:
					$\heva{&N \in SB \\ &N \in EM, EM \subset(ADM)}$\\ 
					$\Rightarrow N = SB \cap (ADM)$.
			\end{itemize}}{
				\begin{tikzpicture}[scale=0.6, font=\footnotesize, line join=round, line cap=round, >=stealth]
					\coordinate[label=below left:$A$] (A) at (-3,0);
					\coordinate[label=left:$B$] (B) at (-2,-2);
					\coordinate[label=below:$C$] (C) at (2,-4);
					\coordinate[label=right:$D$] (D) at (4,0);
					\coordinate[label=above:$S$] (S) at (0,6);
					\coordinate[label=above right:$M$] (M) at ($(S)!1/2!(C)$);
					\coordinate[label=below:$O$](O) at (intersection of A--C and B--D); %giao điểm
					\coordinate[label=below:$E$](E) at (intersection of A--D and B--C); %giao điểm
					\coordinate[label=above left:$N$](N) at (intersection of S--B and M--E); %giao điểm
					\coordinate[label=below right:$I$](I) at (intersection of S--O and M--A); %giao điểm
					\draw[thick] (C)--(D)--(S);
					\draw[thick] (B)--(S)--(C) (M)--(E);
					\draw[thick] (C)--(E)--(S)--(D)--(M);
					\draw[thick][dashed] (S)--(A) (A)--(B) (C)--(A)--(E) (M)--(A)--(D);
					\draw[thick][dashed] (B)--(D) (S)--(O);
					\foreach \diem in {A,B,C,D,M,E,N,O,I}	\fill (\diem)circle(1.5pt);
			\end{tikzpicture}}
			\item Chứng minh rằng $SO, AM, DN$ đồng qui.
			\begin{nx}
				$AM \subset (SAC), DN \subset (SBD)$ nên ta quan tâm đến hai mặt phẳng này.
			\end{nx}
			Xét $(SAC)$ và $(SBD)$ có:
			\begin{itemize}
				\item $S \in(SAC) \cap (SBD) \quad\quad\quad\quad (3)$
				\item Ta có: $\heva{&O \in AC, AC \subset (SAC) \Rightarrow O \in(SAC) \\ 
					&O \in BD, BD \subset(SBD) \Rightarrow O \in(SBD)} \Rightarrow O \in (SAC) \cap (SBD)$ $\quad (4)$\\
				Từ $(3),(4) \Rightarrow(SAC) \cap(SBD)=SO.$
				\item Mặt khác, trong $(ADM),$ gọi $I= AM \cap DN$.\\
				$\heva{&I \in AM \subset(SAC) \\ &I \in DN \subset(SBD)} \Rightarrow I \in(SAC) \cap(SBD) \Rightarrow I \in SO.$
			\end{itemize}
			Do đó $SO, AM, DN$ đồng quy.
		\end{itemize}
	}
\end{vd}

\begin{vd}%[1K4G0-6]
	Cho hình chóp $S.ABCD$ có đáy $ABCD$ là tứ giác mà $AB$ không song song với $CD$. Trên cạnh $SC$ lấy $E$ không trùng với $S$ và $C$.
	\begin{enumerate}
		\item Tìm giao điểm $F$ của $SD$ và $(ABE)$.
		\item Chứng minh ba điểm $AB, CD, EF$ đồng quy.
	\end{enumerate}
	\loigiai{
		\begin{enumerate}
			\item Tìm giao điểm $F$ của $SD$ và $(ABE)$.
			\immini{\begin{itemize}
					\item Chọn mặt phẳng phụ $(SCD)$ chứa $SD$.\\
					Xét $(SCD)$ và $(ABE)$:
					\item Trong $(ABCD)$, gọi $I = AB \cap CD$.\\
					Ta có: $\heva{&I \in CD, CD \subset(SCD) \Rightarrow I \in(SCD) \\ 
						&I \in AB, AB \subset(ABE) \Rightarrow I \in(ABE)}$\\
					$\Rightarrow I \in(SCD) \cap (ABE)$ \quad\quad $(1)$
					\item Mà $\heva{&E \in(ABE) \\ &E \in SC, SC \subset(SCD) \Rightarrow E \in (SCD)}$\\
					$\Rightarrow E \in(SCD) \cap (ABE)$.\quad\quad $(2)$\\
					Từ $(1),(2) \Rightarrow(SCD) \cap (ABE)= EI$.
					\item Trong $(SCD)$, gọi $F= SD \cap EI$.\\ Khi đó:
					$\heva{&F \in SD \\ &F \in EI, EI \subset(ABE)}$\\ 
					$\Rightarrow F = SD \cap (ABE)$.
			\end{itemize}}{
				\begin{tikzpicture}[scale=0.6, font=\footnotesize, line join=round, line cap=round, >=stealth]
					\coordinate[label=below left:$A$] (A) at (-3,0);
					\coordinate[label=left:$B$] (B) at (-2,-2);
					\coordinate[label=below right:$C$] (C) at (2,-4);
					\coordinate[label=below right:$D$] (D) at (4,0);
					\coordinate[label=above:$S$] (S) at (0,6);
					\coordinate[label=above right:$E$] (E) at ($(S)!1/2!(C)$);
					\coordinate[label=below:$I$](I) at (intersection of A--B and C--D); %giao điểm
					\coordinate[label=right:$F$](F) at (intersection of I--E and S--D); %giao điểm
					%\coordinate[label=above left:$N$](N) at (intersection of S--B and M--E); %giao điểm
					%\coordinate[label=below right:$I$](I) at (intersection of S--O and M--A); %giao điểm
					\draw[thick] (A)--(B) (C)--(D)--(S)--(A) (B)--(I)--(C) (I)--(F);
					\draw[thick] (B)--(S)--(C) (B)--(E);
					%	\draw[thick] (C)--(E)--(S)--(D)--(M);
					\draw[thick][dashed] (D)--(A)--(E) (B)--(C);
					\foreach \diem in {A,B,C,D,S,E,I,F}	\fill (\diem)circle(1.5pt);
			\end{tikzpicture}}
			\item Chứng minh ba điểm $AB, CD, EF$ đồng quy.\\
			Theo cách tìm giao điểm $F$ ở trên thì đường thẳng $AB, CD, EF$ đồng quy tại điểm $I$.
		\end{enumerate}
	}
\end{vd}
\begin{vd} %[1K4K0-6]
	Cho tứ diện $ABCD$. Lấy $M, N, P$ lần lượt trên các cạnh $AB, AC, BD$ sao cho $MN$ cắt $BC$ tại $I$, $MP$ cắt $AD$ tại $J$. Chứng minh: $PI, NJ, CD$ đồng quy.
	\loigiai{
		\immini{Ta có
			\begin{itemize}
				\item Trong $(BCD)$, gọi $K= CD \cap IP$.\\ Khi đó:
				$\heva{&K \in CD \subset(ACD) \\ &K \in PI \subset(MNP)}$\\
				$\Rightarrow K \in (ACD) \cap (MNP) \quad \quad (1)$.
				\item Lại có  $\heva{&N \in AC \subset(ACD) \\ &N \in (MNP)}$\\
				$\Rightarrow K \in (ACD) \cap (MNP) \quad \quad (2)$.
				\item $\heva{&J \in AD \subset (ACD) \\ &J \in MP \subset (MNP)}$\\
				$\Rightarrow J \in (ACD) \cap (MNP) \quad \quad (3)$.\\
			\end{itemize}
			Từ $(1)$, $(2)$ và $(3)$ suy ra $J, N, K$ thẳng hàng hay ba đường thẳng $PI, NJ, CD$ đồng quy.}{
			\begin{tikzpicture}[scale=0.8, font=\footnotesize, line join=round, line cap=round, >=stealth]
				\coordinate[label=left:$A$] (A) at (-3,0);
				\coordinate[label=right:$B$] (B) at (3,0);
				\coordinate[label=above:$D$] (D) at (-1,4);
				\coordinate[label=below:$C$] (C) at (1,-2);
				%\coordinate (F) at (-3,3);
				\coordinate[label=above:$M$] (M) at ($(A)!1/3!(B)$);
				\coordinate[label=left:$N$] (N) at ($(A)!3/4!(C)$);
				\coordinate[label=right:$P$] (P) at ($(B)!1/2!(D)$);
				\coordinate[label=below:$I$](I) at (intersection of M--N and B--C); %giao điểm
				\coordinate[label=below:$J$](J) at (intersection of M--P and A--D); %giao điểm
				\coordinate[label=right:$K$](K) at (intersection of N--J and C--D);
				\coordinate(E) at (intersection of M--P and C--A);
				\draw[thick] (A)--(N) (B)--(D)--(A) (C)--(D) (N)--(I)--(C)--(B) (P)--(I) (J)--(A) (J)--(K);
				\draw[thick][dashed] (A)--(B) (M)--(N)--(C) (J)--(M)--(P);
				\foreach \diem in {A,B,C,D,N,M,J,K,P,I}	\fill (\diem)circle(1.5pt);
			\end{tikzpicture}
		}
	}
\end{vd}
\begin{vd} %[1K4K0-6]
	Cho hình chóp $S.ABCD$ có $AB \cap CD=E$ và $AD \cap BC=K$. Gọi $M, N, P$ lần lượt là trung điểm của $SA, SB, SC$.
	\begin{enumerate}
		\item Tìm giao tuyến của $(SAC)$ và $(SBD)$.
		\item Tìm giao tuyến của $(MNP)$ và $(SBD)$.
		\item Tìm giao điểm của $Q$ của $SD$ và $(MNP)$.
		\item Gọi $H=MN \cap PQ$. Chứng minh: $S, H, E$ thẳng hàng.
		\item Chứng minh: $SK, QM, NP$ đồng quy.
	\end{enumerate}
	\loigiai{
		\begin{enumerate}
			\item Tìm giao tuyến của $(SAC)$ và $(SBD)$.
			\immini{ \begin{itemize}
					\item Có $S \in(SBD) \cap(SAC) \quad (1)$
					\item Trong mặt phẳng $(ABCD)$, gọi $I = AC \cap BD$\\
					$\Rightarrow\heva{&I \in BD \subset (SBD) \\ &I \in AC \subset (SAC)}$\\
					$\Rightarrow I \in (SBD) \cap (SAC) \quad (2)$
				\end{itemize}
				Từ (1) và (2) suy ra $(SBD) \cap (SAC)= SI$
				\item Tìm giao tuyến của $(MNP)$ và $(SBD)$.
				\begin{itemize}
					\item Có $N \in(SBD) \cap(MNP) \quad \quad (3)$
					\item Trong mặt phẳng $(SAC)$ gọi $J=MP \cap SI$\\
					$\Rightarrow\heva{&J \in MP \subset(MNP) \\ &J \in SI \subset(SBD)}$\\
					$\Rightarrow J \in (MNP) \cap(SBD) \quad \quad (4)$
				\end{itemize}
				Từ (3) và (4) suy ra $(SBD) \cap(MNP)=NJ$.}{
				\begin{tikzpicture}[scale=0.6, font=\footnotesize, line join=round, line cap=round, >=stealth]
					\coordinate[label=below left:$A$] (A) at (-3,0);
					\coordinate[label=below left:$B$] (B) at (-2,-2);
					\coordinate[label=below right:$C$] (C) at (2,-4);
					\coordinate[label=below right:$D$] (D) at (4,0);
					\coordinate[label=above:$S$] (S) at (0,6);
					\coordinate[label=above right:$M$] (M) at ($(S)!1/2!(A)$);
					\coordinate[label=below left:$N$] (N) at ($(S)!1/2!(B)$);
					\coordinate[label=right:$P$] (P) at ($(S)!1/2!(C)$);
					\coordinate[label=below:$E$](E) at (intersection of A--B and C--D); %giao điểm
					%\coordinate[label=right:$F$](F) at (intersection of I--E and S--D); %giao điểm
					\coordinate[label=above left:$I$](I) at (intersection of A--C and B--D); %giao điểm
					\coordinate[label=below left:$K$](K) at (intersection of B--C and D--A); %giao điểm
					\coordinate[label=below right:$H$](H) at (intersection of M--N and S--E); %giao điểm
					\coordinate[label=right:$Q$](Q) at (intersection of S--D and H--P); %giao điểm
					\coordinate[label=above left:$R$](R) at (intersection of M--Q and S--K); %giao điểm
					\coordinate[label=above right:$J$](J) at (intersection of N--Q and M--P); %giao điểm
					\draw[thick] (C)--(D)--(S) (N)--(H);
					\draw[thick] (B)--(S)--(C) (B)--(E)--(C) (E)--(S)--(K)--(B) (H)--(Q) (N)--(R);
					%	\draw[thick] (C)--(E)--(S)--(D)--(M);
					\draw[thick][dashed] (S)--(A) (A)--(B) (D)--(A)--(E) (B)--(C) (K)--(D) (A)--(C) (B)--(D) (N)--(P) (R)--(Q) (M)--(N) (I)--(S) (M)--(P) (N)--(Q);
					\foreach \diem in {A,B,C,D,S,E,I,J,P,M,N,Q,K,R}	\fill (\diem)circle(1.5pt);
			\end{tikzpicture}}
			\item Tìm giao điểm của $Q$ của $SD$ và $(MNP)$.
			\begin{itemize}
				\item Trong mặt phẳng $(SBD)$, gọi
				$Q=SD \cap NJ$ \\
				$ \Rightarrow\heva{& Q \in SD \\ & Q \in NJ \subset (MNP)}\\ \Rightarrow Q= SD \cap(MNP)$.
			\end{itemize}
			\item Gọi $H=MN \cap PQ$. Chứng minh: $S, H, E$ thẳng hàng.
			\begin{itemize}
				\item Có $SE=(SAB) \cap (SCD)$
				\item Theo giả thuyết có $H=MN \cap PQ$\\
				$\Rightarrow\heva{H \in MN \subset(SAB) \\ H \in PQ \subset (SCD)}\\
				\Rightarrow H \in (SAB) \cap (SCD)\Rightarrow H \in SE$
			\end{itemize}
			Suy ra ba điểm điểm $S, H, E$ thẳng hàng.
			\item Chứng minh: $SK, QM, NP$ đồng quy.
			\begin{itemize}
				\item Có $SK = (SAD) \cap (SBC)$
				\item Theo giả thuyết có $ R = MQ \cap NP$\\
				$\Rightarrow \heva{ R \in MQ \subset(SAD) \\ R \in NP \subset(SBC)}\\
				\Rightarrow R \in(SAD) \cap (SBC) \Rightarrow R \in SK$ 
			\end{itemize}
			Suy ra ba đường thẳng $SK, MQ, NP$ đồng quy tại điểm $R$.
		\end{enumerate}
	}
\end{vd}
\subsection*{BÀI TẬP LUYỆN TẬP}
\begin{bt}%[1K4G0-6]%[1H2G1-6]
	Cho tứ diện $SABC$ với $I$ trung điểm của $SA$, $J$ là trung điểm của $BC$. Gọi $M$ là điểm di động trên $IJ$ và $N$ là điểm di động trên $SC$.
	\begin{enumerate}
		\item Xác định giao điểm $P$ của $MC$ và $(SAB)$.
		\item Tìm giao tuyến của $(SMP)$ và $(ABC)$.
		\item Tìm giao điểm $E$ của $MN$ và $(ABC)$.
		\item Gọi $F=IN \cap AC$. Chứng minh rằng đường thẳng $EF$ luôn đi qua một điểm cố định khi $M$, $N$ di động.
	\end{enumerate}
	\loigiai{
		\begin{enumerate}
			\item Xác định giao điểm $P$ của $MC$ và $(SAB)$.\\
			Chọn mặt phẳng $(BCI)$ chứa $MC$.
			\immini{\begin{itemize} 
					\item Có $IB=(SAB) \cap(BCI)$
					\item Trong mặt phẳng $(BCI)$, gọi $P = CM \cap BI$.\\
					$ \Rightarrow\heva{& P \in CM \\ & P \in BI \subset(SAB)}$\\
					$ \Rightarrow P = CM \cap(SAB)$.
				\end{itemize}
				\item Tìm giao tuyến của $(SMP)$ và $(ABC)$.
				\begin{itemize}
					\item Có $\heva{& C \in(ABC) \\ & C \in PM \subset(SMP)} $\\
					$\Rightarrow C \in(ABC) \cap (SMP)\quad \quad (1)$.
					\item Trong mặt phẳng $(SAB)$ gọi $H= SP \cap AB$\\
					$\Rightarrow \heva{& H \in SP \subset (SMP) \\ &H \in AB \subset(ABC)}$\\
					$\Rightarrow H \in(ABC) \cap(SMP)\quad \quad (2)$
				\end{itemize}
				Từ $(1)$ và $(2)$ suy ra $(ABC) \cap (SMP) = CH$.}{
				\begin{tikzpicture}[scale=0.8, font=\footnotesize, line join=round, line cap=round, >=stealth]
					\coordinate[label=above left:$A$] (A) at (-3,0);
					\coordinate[label=right:$C$] (C) at (3,0);
					\coordinate[label=above:$S$] (S) at (-1,6);
					\coordinate[label=below:$B$] (B) at (0,-3);
					%\coordinate (F) at (-3,3);
					\coordinate[label=above:$I$] (I) at ($(A)!1/2!(S)$);
					\coordinate[label=right:$J$] (J) at ($(B)!1/2!(C)$);
					\coordinate[label=above right:$M$] (M) at ($(I)!1/2!(J)$);
					\coordinate[label=right:$N$] (N) at ($(S)!1/4!(C)$);
					\coordinate[label=below:$F$](F) at (intersection of A--C and I--N); %giao điểm
					\coordinate[label=left:$P$](P) at (intersection of M--C and I--B); %giao điểm
					\coordinate[label=below:$H$](H) at (intersection of S--P and A--B); %giao điểm
					\coordinate[label=below:$E$](E) at (intersection of H--C and F--J); %giao điểm
					\coordinate (X) at (intersection of F--J and A--B); %giao điểm
					%\coordinate[label=right:$K$](K) at (intersection of N--J and C--D);
					%\coordinate(E) at (intersection of M--P and C--A);
					\draw[thick] (S)--(I) (X)--(B) (C)--(B)--(S) (C)--(S) (F)--(I)--(B) (N)--(J) (X)--(F) (S)--(H);
					\draw[thick][dashed] (I)--(A)--(X) (F)--(C) (N)--(I)--(J) (C)--(P) (X)--(J) (H)--(C) (M)--(N)--(E);
					\foreach \diem in {A,B,C,S,N,M,J,J,I,E,F,P,H}	\fill (\diem)circle(1.5pt);
				\end{tikzpicture}
			}
			\item Tìm giao điểm $E$ của $MN$ và $(ABC)$.
			\begin{itemize}
				\item Trong mặt phẳng $(SHC)$, gọi $E=MN \cap CH$\\
				$\Rightarrow \heva{& E \in MN \\ &E \in CH \subset (ABC)}$\\
				$ \Rightarrow E=MN \cap (ABC).$
			\end{itemize}
			\item Gọi $F=IN \cap AC$. Chứng minh rằng đường thẳng $EF$ luôn đi qua một điểm cố định khi $M$, $N$ di động.
			\begin{itemize}
				\item Có $F= IN \cap AC$\\
				$\Rightarrow \heva{&F \in IN \subset(IJN) \\ &F \in AC \subset (ABC)} \Rightarrow F \in (IJN) \cap (ABC) \quad \quad (3)$
				\item Có $E = MN \cap CH$\\
				$\Rightarrow \heva{&E \in MN \subset(IJN) \\ &E \in CH \subset(ABC)} \Rightarrow E \in(IJN) \cap (ABC)$. \quad \quad (4) \\
				Từ $(3)$ và $(4)$ suy ra $(IJN) \cap (ABC)= EF$.\\
				Ngoài ra có $J \in (IJN) \cap (ABC)$ hay $J \in EF$.
			\end{itemize}
			Kết luận đường thẳng $EF$ luôn đi qua điểm $J$ cố định khi $M, N$ thay đổi.
		\end{enumerate}
	}
\end{bt}
\begin{bt} %[1K4G0-6]
	Cho tứ diện $ABCD$. Gọi $I$ và $K$ là trung điểm của $AB$ và $CD$. Gọi $J$ là một điểm trên đoạn $AD$ sao cho $AD=3JD$.
	\begin{enumerate}
		\item Tìm giao điểm $F$ của $IJ$ và $(BCD)$.
		\item Tìm giao điểm $E$ của $(IJK)$ và đường thẳng $BC$. Tính tỉ số: $\dfrac{EB}{EC}$.
		\item Chứng minh ba đường thẳng $AC, KJ, IE$ đồng quy tại điểm $H$. Tính $\dfrac{HC}{HA}$. 
		\item Chứng minh $EJ  \parallel  HF$ và đường thẳng $IK$ đi qua trung điểm của đoạn $HF$.
		\item Gọi $O$ là trung điểm $IK$ và $G$ là trọng tâm của tam giác $BCD$. Chứng minh ba điểm $A, O, G$ thẳng hàng. Tính tỉ số: $\dfrac{OA}{OG}$.
	\end{enumerate}
	\loigiai{
		\begin{center}
			\begin{tikzpicture}[scale=0.6, font=\footnotesize, line join=round, line cap=round, >=stealth]
				\coordinate[label=left:$B$] (B) at (-3,0);
				\coordinate[label=above:$D$] (D) at (5,0);
				\coordinate[label=above:$A$] (A) at (-1,6);
				\coordinate[label=below right:$C$] (C) at (0,-2);
				%\coordinate (F) at (-3,3);
				\coordinate[label=left:$I$] (I) at ($(A)!1/2!(B)$);
				\coordinate[label=above right:$K$] (K) at ($(D)!1/2!(C)$);
				\coordinate[label=above right:$J$] (J) at ($(D)!1/3!(A)$);
				\coordinate[label=above right:$O$] (O) at ($(I)!1/2!(K)$);
				\coordinate[label=above right:$G$] (G) at ($(B)!2/3!(K)$);
				\coordinate[label=right:$N$] (N) at ($(C)!1/3!(A)$);
				\coordinate[label=below:$F$](F) at (intersection of B--D and I--J); %giao điểm
				\coordinate[label=above right:$L$] (L) at ($(I)!1/2!(F)$);
				\coordinate[label=below left:$E$](E) at (intersection of F--K and B--C); %giao điểm
				\coordinate[label=below:$H$](H) at (intersection of I--E and A--C); %giao điểm
				%\coordinate[label=left:$M$](M) at (intersection of M--C and I--B); %giao điểm
				\coordinate[label=below:$M$] (M) at ($(E)!1/2!(F)$);
				\coordinate[label=below:$X$] (X) at ($(H)!1/2!(F)$);
				\draw[thick] (A)--(B)--(C) (J)--(A)--(C) (F)--(K)--(C) (I)--(H)--(J)--(F)--(H)--(C) (N)--(E) (K)--(X);
				\draw[thick][dashed] (K)--(D) (J)--(I)--(K)--(E) (B)--(F) (D)--(L) (B)--(K) (A)--(G) (D)--(M) (E)--(J)--(D);
				\foreach \diem in {A,B,C,D,M,N,J,J,I,E,F,H,O,G,L,X}	\fill (\diem)circle(1.5pt);
			\end{tikzpicture}
		\end{center}
		\begin{enumerate}
			\item Tìm giao điểm $F$ của $IJ$ và $(BCD)$.
			\begin{itemize}
				\item Trong mặt phẳng $(ABD)$, gọi $F=IJ \cap BD$ có\\ $\heva{&F \in IJ \\ &F \in BD \subset(BCD)}$\\
				$\Rightarrow F= IJ \cap(BCD)$.
			\end{itemize}
			\item Tìm giao điểm $E$ của $(IJK)$ và đường thẳng $BC$. Tính tỉ số: $\dfrac{EB}{EC}$.
			\begin{itemize}
				\item Trong mặt phẳng $(BCD)$, gọi $E = JK \cap BC$ có\\
				$\heva{&E \in BC \\ &E \in FK \subset(IJK)} \Rightarrow E = BC \cap (IJK)$.
				\item Trong mặt phẳng $(ABD),$ dựng $DL  \parallel  AB, L \in IJ$, 
				có $\Delta JAI \backsim \Delta JDL \text{ (góc- góc) }$\\ 
				$\Rightarrow \dfrac{AI}{DL}=\dfrac{JA}{JD}=2 \Rightarrow AI = 2DL$
				\item Trong $\Delta BIF$ có $\overrightarrow{DL}=\dfrac{1}{2} \overrightarrow{BI} \Rightarrow D$ trung điểm của $BF$.
				\item Trong mặt phẳng $(BCD)$, dựng $DM  \parallel  BC, M \in EF \Rightarrow BE = 2DM$.\\
				Ngoài ra có $\Delta KDM=\Delta KCE \text{ (góc - cạnh - góc) }.\\ \Rightarrow CE = DM$.
			\end{itemize}
			Vậy $\dfrac{EB}{EC}=2$.
			\item Chứng minh ba đường thẳng $AC, KJ, IE$ đồng quy tại điểm $H$. Tính $\dfrac{HC}{HA}$.
			\begin{itemize}
				\item Trong mặt phẳng $(IJK)$, gọi $H = JK \cap IE$, có \\
				$\heva{& H \in IE \subset (ABC) \\ &H \in JK \subset(ACD)} \Rightarrow H \in(ABC) \cap (ACD)$.\\
				Hay $H$ thuộc giao tuyến $AC$ của mặt phẳng $(ABC)$ với mặt phẳng $(ACD)$.
			\end{itemize}
			Kết luận 3 đường thẳng $AC, JK$ và $IE$ đồng quy tại điểm $H$.
			\begin{itemize}
				\item Trong mặt phẳng $(ABC)$, dựng $EN  \parallel  AB, N \in AC$.
				\item Trong $\Delta ABC$ có $\dfrac{ CE}{CB}=\dfrac{EN}{AB}=\dfrac{1}{3} \Rightarrow \dfrac{EN}{AI}=\dfrac{2}{3}$.
				\item Trong $\Delta HAI$ có $\dfrac{ HE }{HI}=\dfrac{EN}{AI}=\dfrac{2}{3} \Rightarrow E$ là trọng tâm của $\Delta ABH$.
			\end{itemize}
			Vậy $\dfrac{HC}{HA}=\dfrac{1}{2}$.
			\item Chứng minh $EJ  \parallel  HF$ và đường thẳng $IK$ đi qua trung điểm của đoạn $HF$.\\
			\begin{itemize}
				\item Trong $\Delta IHF$ có $\dfrac{IJ}{IE}=\dfrac{IE}{IH}=\dfrac{2}{3}$ (tính chất trọng tâm) $\Rightarrow {EJ}  \parallel  EH$ (định lý đảo Thalet).
				\item Gọi $X$ là giao điểm của $IK$ và $HF$. Theo định lý Ce-Va ta có:\\
				$$\dfrac{EI}{EH} \cdot \dfrac{XH}{XF} \cdot \dfrac{JF}{JI}=1 \Rightarrow \dfrac{XH}{XF} =1.$$
			\end{itemize}
			Vậy $X$ là trung điểm của $HF$ hay $IK$ qua trung điểm của $HF$. 
			\item Gọi $O$ là trung điểm $IK$ và $G$ là trọng tâm của tam giác $BCD$. Chứng minh ba điểm $A, O, G$ thẳng hàng. Tính tỉ số: $\dfrac{OA}{OG}$.
			\begin{itemize}
				\item Ta có
				\begin{eqnarray*}
					\overrightarrow{AG} & =& \overrightarrow{AB}+\overrightarrow{BG}=\overrightarrow{AB}+\dfrac{2}{3} \overrightarrow{BK}=\overrightarrow{AB}+\dfrac{2}{3}(\overrightarrow{AK}-\overrightarrow{AB})=\dfrac{1}{3} \overrightarrow{AB}+\dfrac{2}{3} \overrightarrow{AK} \\
					&=& \dfrac{2}{3} \overrightarrow{AI}+\dfrac{2}{3} \overrightarrow{AK}=\dfrac{2}{3} \left(\overrightarrow{AI}+\overrightarrow{AK}\right) \quad \quad (1) .
				\end{eqnarray*}
				\item Lại có: $\overrightarrow{AO}=\dfrac{1}{2}(\overrightarrow{AI}+\overrightarrow{AK} \quad \quad (2)$.
			\end{itemize}
			Từ $(1)$ và $(2)$ suy ra $\overrightarrow{AG}=\dfrac{4}{3} \overrightarrow{AO}$.\\
			Kết luận ba điểm $A, O, G$ thẳng hàng và $\dfrac{OA}{OG}=3$.
		\end{enumerate}
		
	}
\end{bt}
%--------------Bài 12
\begin{bt}%[1K4G0-6]
	Cho tứ diện $ABCD$. Trên các cạnh $AB, AC, BD$ lần lượt lấy ba điểm $E, F, G$ sao cho $AB=3AE; AC=2AF; DB=4DG$.
	\begin{enumerate}
		\item Tìm giao tuyến của hai mặt phẳng $(EFG)$ và $(BCD)$.
		\item Tìm giao điểm $H$ của đường thẳng $CD$ với $(EFG)$. Tính tỉ số $\dfrac{HC}{HD}$.
		\item Tìm giao điểm $I$ của đường thẳng $AD$ và $(EFG)$. Tính tỉ số $\dfrac{IA}{ID}$.
		\item Chứng minh ba đường thẳng $FI$, $GM$, $CD$ đồng quy.
		\item Gọi $J$ là trung điểm của $BC, AJ$ cắt $EF$ tại $K$. Tính tỉ số $\dfrac{AK}{AJ}$.
	\end{enumerate}
	\loigiai{
		\begin{center}
			\begin{tikzpicture}[scale=0.8, font=\footnotesize, line join=round, line cap=round, >=stealth]
				\coordinate[label= above left:$B$] (B) at (0,0);
				\coordinate[label=above:$A$] (A) at (2,8);
				\coordinate[label=below:$C$] (C) at (3,-2);
				\coordinate[label=right:$D$] (D) at (8,0);
				\coordinate[label=left:$E$] (E) at ($(A)!1/3!(B)$);
				\coordinate[label=left:$F$] (F) at ($(A)!1/2!(C)$);
				\coordinate[label=above right :$G$] (G) at ($(B)!0.82!(D)$);
				\coordinate[label=above right :$M$] (M) at ($(B)!2!(C)$);
				
				\coordinate[label=above right :$I$] (I) at ($(A)!1.76!(D)$);
				\coordinate[label=above right :$J$] (J) at ($(B)!1/2!(C)$);
				\coordinate[label=left :$K$] (K) at ($(A)!2/5!(J)$);
				\coordinate (P) at (intersection cs:first line={(E)--(M)}, second line={(C)--(D)});
				\coordinate[label=left:$H$] (H) at (intersection cs:first line={(G)--(M)}, second line={(P)--(D)});		
				\coordinate(Q) at (intersection cs:first line={(E)--(I)}, second line={(C)--(D)});
				
				\draw[thick] (A)--(B)--(C)--(A)--(D)--(H) (I)--(M) (A)--(C)  (E)--(F)(C)--(M)--(F)(D)--(I)(A)--(J) (H)--(M) (I)--(F);
				\draw[thick][dashed] (B)--(D) (C)--(H) (E)--(G)--(F)(H)--(G)(E)--(Q)--(I);
				\foreach \diem in {A,B,C,D,E,F,G,M,H,I,J,K}	\fill (\diem)circle(1.0pt);
				\fill[cyan,opacity=0.5] (M)--(E)--(I)--cycle;
			\end{tikzpicture}
		\end{center}
		
		\begin{enumerate}
			\item Tìm giao tuyến của hai mặt phẳng $(EFG)$ và $(BCD)$.
			\begin{itemize}
				\item $G \in (EFG) \cap (BCD)$.
				\item Ta có$\colon \dfrac{AE}{AB} \neq \dfrac{AF}{AC} \Rightarrow $ EF cắt $BC$ tại $M \Rightarrow M \in (EFG)\cap (BCD)$
			\end{itemize}
			Vậy $MG =(EFG)\cap (BCD)$.
			\item Tìm giao điểm $H$ của đường thẳng $CD$ với $(EFG)$. Tính tỉ số $\dfrac{HC}{HD}$.
			\begin{itemize}
				\item Trong $(BCD)$, gọi $H =MG \cap CD \Rightarrow H =CD \cap (EFG)$.
				\item Áp dụng định lý Menelaus trong $\Delta BCA$ với ba điểm thẳng hàng $M, E, F$ ta có$\colon$\\$\dfrac{AE}{EB}\cdot \dfrac{BM}{MC}\cdot \dfrac{CF}{FA}=1 \Rightarrow \dfrac{BM}{MC}=2$.
				\\Áp dụng định lý Menelaus trong $\Delta BCD$ với ba điểm thẳng hàng $M, H, G$ ta có$\colon$\\$\dfrac{DG}{DB}\cdot \dfrac{BM}{MC}\cdot \dfrac{CH}{HD}=1 \Rightarrow \dfrac{CH}{HD}=\dfrac{3}{2}$.
			\end{itemize}
			\item Tìm giao điểm $I$ của đường thẳng $AD$ và $(EFG)$. Tính tỉ số $\dfrac{IA}{ID}$.
			\begin{itemize}
				\item Ta có$\colon \dfrac{BE}{BA}\neq \dfrac{BG}{BD} \Rightarrow EG$ cắt $AD$ tại điểm $I \Rightarrow I = AD \cap (EFG)$.
				\item Áp dụng định lý Menelaus trong $\Delta ABD$ với ba điểm thẳng hàng $E, G, I$ ta có$\colon$\\$\dfrac{BE}{EA}\cdot \dfrac{AI}{ID}\cdot \dfrac{DG}{GB}=1 \Rightarrow \dfrac{AI}{ID}=\dfrac{3}{2}$.
			\end{itemize}
			\item Chứng minh ba đường thẳng $FI$, $GM$, $CD$ đồng quy.
			\begin{itemize}
				\item $F\in (EFG)\cap (ACD)$.
				\item $H \in (EFG)\cap (ACD)$.
				\item $I \in (EFG)\cap (ACD)$.
			\end{itemize}
			Suy ra ba điểm $F$, $H$, $I$ thẳng hàng.\\
			Vậy ba đường thẳng $FI$, $GM$, $CD$ đồng quy.
			\item Gọi $J$ là trung điểm của $BC, AJ$ cắt $EF$ tại $K$. Tính tỉ số $\dfrac{AK}{AJ}$.\\
			Áp dụng định lý Menelaus trong $\Delta ABJ$ với ba điểm thẳng hàng $E, F, M$ ta có$\colon$\\$\dfrac{AE}{EB}\cdot \dfrac{BM}{MJ}\cdot \dfrac{JK}{KA}=1 \Rightarrow \dfrac{JK}{KA}=\dfrac{3}{2}$.\\
			Vậy $\dfrac{AK}{AJ}=\dfrac{2}{3}$.
		\end{enumerate}
	}
	
\end{bt}