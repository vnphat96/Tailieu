\begin{dang}{Tìm thiết diện bằng cách kẻ song song}
	\Opensolutionfile{ans}[ans/ans-1K4-2-Dang4]
\end{dang}
\begin{vd}%[1K4BA-5]
	Cho tứ diện $ABCD$. Trên các cạnh $AB,BC,CD$ lần lượt lấy các điểm $P,Q,R$ sao cho $AP=\dfrac{1}{3}AB$, $BC=3QC$ và $R$ không trùng với $C,D$. Gọi $PQRS$ là thiết diện của mặt phẳng $(PQR)$ với tứ diện $ABCD$. Khi đó $PQRS$ là hình gì?
	\loigiai{
		\immini{
			Từ giả thiết, ta có $\dfrac{BP}{BA}=\dfrac{BQ}{BC}=\dfrac{2}{3} \,\Rightarrow\; PQ \parallel AC$.\\
			Ta có $\heva{&PQ\parallel AC\\&(PQRS) \supset PQ, (ACD) \supset AC \\&(PQRS)\cap(ACD)=RS} \;\Rightarrow\, RS\parallel AC$.\\
			Suy ra, $PQRS$ là hình thang.\\
			Nếu $AB=CB$ và $AD=CD$ thì $\triangle{ABD}=\triangle{CBD}$\\
			$\Rightarrow\, QR=PS$, khi đó $PQRS$ là hình thang cân.
		}{\begin{tikzpicture}[scale=1, font=\footnotesize, line join=round, line cap=round, >=stealth]
			\def\ac{4} % cạnh AC
			\def\ab{2} % cạnh AB
			\def\as{4} % cạnh AS
			\def\gocA{50} % góc A của đáy
			\coordinate[label=left:$B$] (B) at (0,0);
			\coordinate[label=right:$D$] (D) at (\ac,0);
			\coordinate[label=below left:$C$] (C) at (-\gocA:\ab);
			\coordinate[label=above:$A$] (A) at (70:\as);
			\path 
			(barycentric cs:A=2,B=1) coordinate (P)
			($(A)!0.5!(D)$) coordinate (S)
			($(C)!0.5!(D)$) coordinate (R)
			(barycentric cs:C=2,B=1) coordinate (Q)
			;
			\draw[dashed] (B)--(D) (P)--(S) (Q)--(R);
			\draw(A)--(C) (A)--(B) (A)--(D) (B)--(C)--(D)(P)--(Q) (S)--(R);
			\foreach \diem in {A,B,C,D}\fill (\diem)circle(1.5pt);
			\foreach \x/\y in{P/120,Q/180,R/-90,S/0}{\draw[fill=black](\x) circle(1.5pt)++(\y:0.3)node{$\x$};}
		\end{tikzpicture}
		}
	}
\end{vd}

\begin{vd}%[1K4BA-5]
	Cho hình chóp $S.ABCD$ có đáy $ABCD$ là hình thang, $AD\parallel BC$, $AD=2BC$. Gọi $M$ là trung điểm $SA$. Mặt phẳng $(MBC)$ cắt hình chóp $S.ABCD$ theo thiết diện là hình gì?
	\loigiai{\immini{Gọi $N$ là giao của $SD$ và mặt phẳng $(MBC)$. Do các mặt phẳng $(MBC)$ và $(SAD)$ lần lượt chứa hai đường song song là $BC$ và $AD$, nên giao tuyến của chúng cũng song song với hai đường đó, tức $MN\parallel AD.$ Suy ra $N$ là trung điểm của $SD.$
			Khi đó, $MN$ là đường trung bình của tam giác $SAD$, suy ra $MN=\dfrac12 AD=BC.$ Vậy, thiết diện $BCNM$ là một hình bình hành.
		}{
			\begin{tikzpicture}[scale=1, font=\footnotesize, line join=round, line cap=round, >=stealth]
				\def\ad{4} % cạnh AD
				\def\ab{2} % cạnh AB
				\def\bc{2} % chéo AC
				\def\as{4} % cạnh AS
				\def\gocA{50} % góc A của đáy
				\def\gocB{130} % góc B của đáy
				\coordinate[label=left:$A$] (A) at (0,0);
				\coordinate[label=below left:$B$] (B) at (-\gocA:\ab);
				\coordinate[label=below right:$C$] (C) at ($(B)+(180-\gocA-\gocB:\bc)$);
				\coordinate[label=right:$D$] (D) at (\ad,0);
				\coordinate[label=above:$S$] (S) at (75:\as);
				\path
				($(A)!0.5!(S)$) coordinate (M)
				($(S)!0.5!(D)$) coordinate (N)
				;
				\draw (A)--(B)--(C)--(D)--(S)--cycle (B)--(S)--(C) (M)--(B) (C)--(N);
				\draw[dashed] (A)--(D) (M)--(N);
				\foreach \diem in {A,B,C,D,S}	\fill (\diem)circle(1.5pt);
				\foreach \x/\y in{M/180,N/0}{\draw[fill=black](\x) circle(1.5pt)++(\y:0.3)node{$\x$};}
			\end{tikzpicture}
			
	}
	}
\end{vd}
%
\begin{vd}%[1K4YA-5]
	Cho tứ diện $ABCD$. Gọi $M, N$ lần lượt là trung điểm của $AB, AC$, $E$ là điểm trên cạnh $CD$ sao cho $ED = 3EC$. Thiết diện tạo bởi mặt phẳng $(MNE)$ và tứ diện $ABCD$ là hình gì?
	\loigiai{
		\immini{$\bullet$ Thiết diện là tứ giác $MNEF$\\
			$\bullet$ $MN \parallel EF$ và $MN \ne EF$ nên $MNEF$ là hình thang.
		}{\begin{tikzpicture}[scale=1, font=\footnotesize, line join=round, line cap=round, >=stealth]
			\def\ac{4} % cạnh AC
			\def\ab{2} % cạnh AB
			\def\as{4} % cạnh AS
			\def\gocA{50} % góc A của đáy
			\coordinate[label=left:$B$] (B) at (0,0);
			\coordinate[label=right:$D$] (D) at (\ac,0);
			\coordinate[label=below left:$C$] (C) at (-\gocA:\ab);
			\coordinate[label=above:$A$] (A) at (40:\as);
			\path 
			(barycentric cs:C=3,D=1) coordinate (E)
			(barycentric cs:B=3,D=1) coordinate (F)
			($(A)!0.5!(B)$) coordinate (M)
			($(C)!0.5!(A)$) coordinate (N)
			(barycentric cs:C=2,B=1) coordinate (Q)
			;
			\draw[dashed] (B)--(D) (E)--(F) (M)--(F);
			\draw(A)--(C) (A)--(B) (A)--(D) (B)--(C)--(D) (M)--(N)--(E);
			\foreach \diem in {A,B,C,D}\fill (\diem)circle(1.5pt);
			\foreach \x/\y in{E/-90,F/-90,N/0,M/90}{\draw[fill=black](\x) circle(1.5pt)++(\y:0.3)node{$\x$};}
		\end{tikzpicture}
			
		}
	}
\end{vd}
%

\subsubsection{Bài tập rèn luyện}
\begin{bt}%[1K4BA-5]
	Cho hình chóp $S.ABCD$ có đáy $ABCD$ là hình thang với đáy lớn $AB$, $M$ là trung điểm $CD$. Mặt phẳng $(\alpha)$ qua $M$ song song với $BC$ và $SA$. $(\alpha)$ cắt $AB$, $SB$ lần lượt tại $N$ và $P$. Thiết diện của hình chóp $S.ABCD$ với mặt phẳng $(\alpha)$ là hình gì?
	\loigiai{
		\immini{
			Trong mặt phẳng $(ABCD)$ qua $M$ kẻ đường thẳng song song với $BC$ cắt $AB$ tại $N$.\\
			Trong mặt phẳng $(SAB)$ qua $N$ kẻ đường thẳng song song với $SA$ cắt $SB$ tại $P$.\\
			Trong mặt phẳng $(SBC)$ qua $P$ kẻ đường thẳng song song với $BC$ cắt $SC$ tại $Q$.\\
			Khi đó thiết diện là mặt phẳng $(MNPQ)$.\\
			Ta thấy $MNPQ$ là hình thang có đáy lớn là $MN$.}
		{\begin{tikzpicture}[scale=1, font=\footnotesize, line join=round, line cap=round, >=stealth]
				\def\ad{4} % cạnh AD
				\def\ab{2} % cạnh AB
				\def\bc{2} % chéo AC
				\def\as{4} % cạnh AS
				\def\gocA{50} % góc A của đáy
				\def\gocB{130} % góc B của đáy
				\coordinate[label=left:$B$] (B) at (0,0);
				\coordinate[label=below left:$C$] (C) at (-\gocA:\ab);
				\coordinate[label=below right:$D$] (D) at ($(C)+(180-\gocA-\gocB:\bc)$);
				\coordinate[label=right:$A$] (A) at (\ad,0);
				\coordinate[label=above:$S$] (S) at (40:\as);
				\path
				($(C)!0.5!(D)$) coordinate (M)
				(barycentric cs:B=3,S=1) coordinate (P)
				(barycentric cs:C=3,S=1) coordinate (Q)
				(barycentric cs:B=3,A=1) coordinate (N)
				;
				\draw (A)--(D)--(C)--(B)--(S)--cycle (P)--(Q) (Q)--(M)(B)--(S)--(C);
				\draw[dashed] (A)--(B) (P)--(N)--(M);
				\foreach \diem in {A,B,C,D,S}	\fill (\diem)circle(1.5pt);
				\foreach \x/\y in{M/-90,N/-120,P/90,Q/0}{\draw[fill=black](\x) circle(1.5pt)++(\y:0.3)node{$\x$};}
			\end{tikzpicture}
		}
	}
\end{bt}
\begin{bt}%[1K4BA-5]
	Cho tứ diện $ABCD$. Gọi $M$ và $N$ lần lượt là trung điểm $AB$ và $AC$, $E$ là điểm trên cạnh $CD$ với $ED=3EC$. Thiết diện tạo bởi mặt phẳng $(MNE)$ và tứ diện $ABCD$ là hình gì?
	\loigiai{
		\immini{
			$M$ và $N$ lần lượt là trung điểm $AB$ và $AC$ nên $MN \parallel BC$ và $MN=\dfrac{1}{2}BC$.\\
			Qua $E$ kẻ đường thẳng song song với $BC$, cắt $BD$ tại $F$ thì $EF \parallel MN$.\\
			Ta có $\dfrac{EF}{BC}=\dfrac{DE}{DC}=\dfrac{3}{4} \Rightarrow EF=\dfrac{3}{4}BC > MN$.\\
			Vậy thiết diện tạo bởi mặt phẳng $(MNE)$ và tứ diện $ABCD$ là là hình thang $MNEF$.
		}{
			\begin{tikzpicture}[scale=1, font=\footnotesize, line join=round, line cap=round, >=stealth]
				\def\ac{4} % cạnh AC
				\def\ab{2} % cạnh AB
				\def\as{4} % cạnh AS
				\def\gocA{50} % góc A của đáy
				\coordinate[label=left:$B$] (B) at (0,0);
				\coordinate[label=right:$D$] (D) at (\ac,0);
				\coordinate[label=below left:$C$] (C) at (-\gocA:\ab);
				\coordinate[label=above:$A$] (A) at (40:\as);
				\path 
				(barycentric cs:C=3,D=1) coordinate (E)
				(barycentric cs:B=3,D=1) coordinate (F)
				($(A)!0.5!(B)$) coordinate (M)
				($(C)!0.5!(A)$) coordinate (N)
				(barycentric cs:C=2,B=1) coordinate (Q)
				;
				\draw[dashed] (B)--(D) (E)--(F) (M)--(F);
				\draw(A)--(C) (A)--(B) (A)--(D) (B)--(C)--(D) (M)--(N)--(E);
				\foreach \diem in {A,B,C,D}\fill (\diem)circle(1.5pt);
				\foreach \x/\y in{E/-90,F/-90,N/0,M/90}{\draw[fill=black](\x) circle(1.5pt)++(\y:0.3)node{$\x$};}
			\end{tikzpicture}
		}
	}
\end{bt}
%
\begin{bt}%[1K4BA-5]
	Cho tứ diện $ABCD$. Gọi $I$ và $J$ lần lượt là trung điểm của $BC$ và $BD$; $E$ là điểm thuộc $AD$ khác với $A$ và $D$. Thiết diện của hình tứ diện khi cắt bởi mặt phẳng $(IJE)$ là hình gì?
	\loigiai{
		\immini{
			$\bullet$  Vì $I$ và $J$ lần lượt là trung điểm của $BC$ và $BD$ nên $IJ \parallel CD$. \\
			$\bullet$  $IJ \subset (IJE),CD \subset (ACD),E \in (IJE) \cap (ACD)$ nên giao tuyến của hai mặt phẳng $(IJE)$ và $(ACD)$ là đường thẳng $d$ qua $E$ và song song với $CD$.\\
			$\bullet$  Gọi $F=d \cap AC$, ta có: tứ giác $IJEF$ là thiết diện của $(IJE)$ với tứ diện $ABCD$.\\
			$\bullet$  Vì $EF \parallel IJ$ nên $IJEF$ là hình thang.}{
			\begin{tikzpicture}[scale=1, font=\footnotesize, line join=round, line cap=round, >=stealth]
				\def\ac{4} % cạnh AC
				\def\ab{2} % cạnh AB
				\def\as{4} % cạnh AS
				\def\gocA{50} % góc A của đáy
				\coordinate[label=left:$B$] (B) at (0,0);
				\coordinate[label=right:$D$] (D) at (\ac,0);
				\coordinate[label=below left:$C$] (C) at (-\gocA:\ab);
				\coordinate[label=above:$A$] (A) at (70:\as);
				\path 
				(barycentric cs:C=1,A=3) coordinate (F)
				(barycentric cs:D=1,A=3) coordinate (E)
				($(C)!0.5!(B)$) coordinate (I)
				($(B)!0.5!(D)$) coordinate (J)
				;
				\draw[dashed] (B)--(D) (E)--(F) (I)--(J)--(E);
				\draw(A)--(C) (A)--(B) (A)--(D) (B)--(C)--(D)(F)--(I) ;
				\foreach \diem in {A,B,C,D}\fill (\diem)circle(1.5pt);
				\foreach \x/\y in{E/0,F/180,I/180,J/-90}{\draw[fill=black](\x) circle(1.5pt)++(\y:0.3)node{$\x$};}
			\end{tikzpicture}
			}
	}
\end{bt}
\subsubsection{Bài tập trắc nghiệm}
\Opensolutionfile{ans}[ans/ans-1]
\begin{ex}%[1K4YA-5]
	Cho tứ diện $A B C D$. Gọi $M, N$ lần lượt là trung điểm của hai cạnh $B C, C D$. Mặt phẳng $(AMN)$ cắt tứ diện $ABCD$ theo thiết diện là
	\choice
	{\True tam giác}
	{tứ giác}
	{ngũ giác}
	{hình thang}
	\loigiai{
		\immini {Ta có $ \heva{&BD \parallel MN\\&MN\subset (AMN) \Rightarrow BD \parallel (AMN).} $}
		{\begin{tikzpicture}[scale=.6, font=\footnotesize, line join=round, line cap=round, >=stealth]
				\def\ac{4} % cạnh AC
				\def\ab{2} % cạnh AB
				\def\as{4} % cạnh AS
				\def\gocA{50} % góc A của đáy
				\coordinate[label=left:$B$] (B) at (0,0);
				\coordinate[label=right:$D$] (D) at (\ac,0);
				\coordinate[label=below left:$C$] (C) at (-\gocA:\ab);
				\coordinate[label=above:$A$] (A) at (70:\as);
				\draw (A)--(B)--(C)--(D)--cycle (A)--(C);
				\draw[dashed] (B)--(D);
				\coordinate[label=below left:$M$] (M) at ($(B)!.5!(C)$);
				\coordinate[label=below:$N$] (N) at ($(D)!.5!(C)$);
				\draw[dashed] (M)--(N);
				\draw (A)--(M) (N)--(A);
				\foreach \diem in {A,B,C,D,M,N}\fill (\diem)circle(.8pt);
		\end{tikzpicture}}	
	}
\end{ex}
\begin{ex}%[1K4BA-5]
	Cho hình chóp $S.ABCD$ có đáy $ABCD$ là hình thang, $AD\parallel BC$, $AD=2BC$. Gọi $M$ là trung điểm $SA$. Mặt phẳng $(MBC)$ cắt hình chóp $S.ABCD$ theo thiết diện là
	\choice
	{\True một hình bình hành}
	{một tam giác}
	{một hình tứ giác (không là hình thang)}
	{một hình thang (không là hình bình hành)}
	\loigiai{\immini{Gọi $N$ là giao của $SD$ và mặt phẳng $(MBC)$. Do các mặt phẳng $(MBC)$ và $(SAD)$ lần lượt chứa hai đường song song là $BC$ và $AD$, nên giao tuyến của chúng cũng song song với hai đường đó, tức $MN\parallel AD.$ Suy ra $N$ là trung điểm của $SD.$
			
			Khi đó, $MN$ là đường trung bình của tam giác $SAD$, suy ra $MN=\dfrac12 AD=BC.$ Vậy, thiết diện $BCNM$ là một hình bình hành.
		}{\begin{tikzpicture}[scale=1, font=\footnotesize,>=stealth]%<DTools>
				%Gán số liệu.
				\def\canhAD{4};\def\canhBA{2};\def\canhCB{2};\def\gocBAD{-50};\def\h{3};\def\xdinhS{2};
				%Gán tọa độ.
				\coordinate (A) at (0,0);
				\coordinate (B) at ($(A)+(\gocBAD:\canhBA)$);
				\coordinate (C) at ($(B)+(0:\canhCB)$);
				\coordinate (D) at ($(A)+(0:\canhAD)$);
				\coordinate (S) at ($(A)+(\xdinhS,\h)$);
				\path
				($(S)!0.5!(A)$) coordinate (M)
				($(S)!0.5!(D)$) coordinate (N)
				;
				%Vẽ khối chóp S.ABCD.
				\draw (B)--(S)--(C)--cycle (S)--(A)--(B) (S)--(D)--(C)(M)--(B)(N)--(C);
				\draw[dashed] (A)--(D)(M)--(C)(M)--(N);
				%Gán nhãn.
				\foreach \x/\y in {A/180,B/-90,C/-90,D/0,S/90,M/90,N/90}{\fill (\x) circle(1pt) ($(\x)+(\y:0.3cm)$) node{$\x$};}
		\end{tikzpicture}}
	}
\end{ex}

\begin{ex}%[1K4BA-5]
	Cho tứ diện $ABCD$. Gọi $I$, $J$ và $K$ lần lượt là trung điểm của $AC$, $BC$ và $BD$. Mặt phẳng $(IJK)$ cắt tứ diện $ABCD$ theo thiết diện là
	\choice
	{hình chữ nhật}
	{\True hình bình hành}
	{hình vuông}
	{hình thang}
	\loigiai{
		\immini{Ta có điểm $K$ là điểm chung của hai mặt phẳng $(ABD)$ và $(IJK)$. \\
			Mặt khác ta có $IJ \parallel AB$, $IJ \subset (IJK)$, $AB \subset (ABD)$. \\
			Suy ra giao tuyến của hai mặt phẳng $(ABD)$ và $(IJK)$ là đường thẳng đi qua điểm $K$ và song song với $AB$. Ta đặt là $Kx$.\\
			Trong $(ABD)$, gọi $H=AD\cap Kx$.\\
			Ta có $\heva{&IJ=KH=\dfrac{1}{2}AB\\&JK=IH=\dfrac{1}{2}CD\\}$.\\
			Suy ra tứ giác $IJKH$ là hình bình hành.}
		{\begin{tikzpicture}[line width=0.6pt,scale=1.2]
				\coordinate[label=left:{$B$}] (B) at (0,0);
				\coordinate[label=below:{$C$}] (C) at (1.4,-1.3);
				\coordinate[label=right:{$D$}] (D) at (4,0);
				\coordinate[label={$A$}] (A) at (1.8,3);
				\coordinate[label=above right:{$I$}] (I) at ($(A)!1/2!(C)$);
				\coordinate[label=below left:{$J$}] (J) at ($(B)!1/2!(C)$);
				\coordinate[label=below right:{$K$}] (K) at ($(B)!1/2!(D)$);
				\coordinate (x) at ($(A)!1/2!(D)$);
				\path 
				(intersection of K--x and A--D) coordinate (H)
				;
				\draw (A)--(B)--(C)--(A)--(D)--(C) (I)--(J) (x)--($(x)!-0.5!(K)$)node[right]{$x$}(I)--(H)node[right]{$H$};
				\draw[dashed] (B)--(D) (I)--(K)--(J) ($(K)!-0.3!(x)$)--(x);
		\end{tikzpicture}}
	}
\end{ex}

\begin{ex}%[1K4BA-5]
	Cho tứ diện $ABCD$. Trên các cạnh $AB,BC,CD$ lần lượt lấy các điểm $P,Q,R$ sao cho $AP=\dfrac{1}{3}AB$, $BC=3QC$ và $R$ không trùng với $C,D$. Gọi $PQRS$ là thiết diện của mặt phẳng $(PQR)$ với tứ diện $ABCD$. Khi đó $PQRS$ là
	\choice
	{hình thang cân}
	{\True hình thang}
	{một tứ giác không có cặp cạnh đối nào song song}
	{hình bình hành}
	\loigiai{
		\immini{
			Từ giả thiết, ta có $\dfrac{BP}{BA}=\dfrac{BQ}{BC}=\dfrac{2}{3} \,\Rightarrow\; PQ \parallel AC$.\\
			Ta có $\heva{&PQ\parallel AC\\&(PQRS) \supset PQ, (ACD) \supset AC \\&(PQRS)\cap(ACD)=RS} \;\Rightarrow\, RS\parallel AC$.\\
			Suy ra, $PQRS$ là hình thang.
		}{\begin{tikzpicture}[scale=1, font=\footnotesize,>=stealth]%<DTools>
				%Gán số liệu.
				\def\canhBD{4};\def\canhCB{2};\def\gocCBD{-50};\def\h{3};
				%Gán tọa độ.
				\coordinate (B) at (0,0);
				\coordinate (C) at ($(B)+(\gocCBD:\canhCB)$);
				\coordinate (D) at ($(B)+(0:\canhBD)$);
				\coordinate (O) at ($ 1/3*(B)+1/3*(C)+1/3*(D) $);
				\coordinate (A) at ($(O)+(90:\h)$);
				\path
				(barycentric cs:A=2,B=1) coordinate (P)
				(barycentric cs:C=2,B=1) coordinate (Q)
				(barycentric cs:D=2,C=1) coordinate (R)
				(barycentric cs:D=2,A=1) coordinate (S)
				;
				%Vẽ khối chóp A.BCD đều.
				\draw (A)--(C) (A)--(B)--(C) (A)--(D)--(C)(P)--(Q)(R)--(S);
				\draw[dashed] (B)--(D)(P)--(R)(Q)--(R)(P)--(S);
				%Gán nhãn.
				\foreach \x/\y in {A/90,B/180,C/-90,D/0,P/90,Q/180,R/-60,S/80}{\fill (\x) circle (1pt) ($(\x)+(\y:0.3cm)$) node{$\x$};}
			\end{tikzpicture}
			
		}
	}
\end{ex}

\begin{ex}%[1K4KA-5]
	Cho hình chóp $S.ABCD$ có đáy $ABCD$ là hình bình hành tâm $I$. Gọi $M$, $N$, $P$ lần lượt là trung điểm các đoạn thẳng $BC$, $CD$ và $SI$. Thiết diện của hình chóp $S.ABCD$ với mặt phẳng $(MNP)$ là
	\choice
	{Một tam giác}
	{Một lục giác}
	{\True Một ngũ giác}
	{Một tứ giác}
	\loigiai{
		\immini{Trong $(ABCD) \colon MN \cap AC=K$.\\
			$\heva{&P \in (SBD) \cap (MNP) \\& MN \parallel BD \\& MN \subset (MNP) \\& BD \subset (SBD)}$\\
			$\Rightarrow (MNP) \cap (SCD)=d \parallel MN \parallel BD$ \\
			$d \cap SB=E$; $d \cap SD=F$.\\
			Trong $(SAC) \colon KP \cap SA=G$. Ta có\\
			$(MNP) \cap (ABCD)=MN$; $(MNP) \cap (SBC)=ME$}{
			\begin{tikzpicture}[scale=1, font=\footnotesize,>=stealth]%<DTools>
				%Gán số liệu.
				\def\canhAD{4};\def\canhBA{2};\def\gocBAD{-130};\def\h{3};\def\xdinhS{1};
				%Gán tọa độ.
				\coordinate (A) at (0,0);
				\coordinate (B) at ($(A)+(\gocBAD:\canhBA)$);
				\coordinate (C) at ($(B)+(0:\canhAD)$);
				\coordinate (D) at ($(A)+(0:\canhAD)$);
				\coordinate (S) at ($(A)+(\xdinhS,\h)$);
				\path
				(intersection of A--C and B--D) coordinate (I)
				($(B)!0.5!(C)$) coordinate (M)
				($(C)!0.5!(D)$) coordinate (N)
				($(S)!0.5!(I)$) coordinate (P)
				(intersection of M--N and A--C) coordinate (K)
				($(N)+(P)-(M)$) coordinate (E1)
				(intersection of P--E1 and S--D) coordinate (F)
				(intersection of P--E1 and S--B) coordinate (E)
				(intersection of K--P and S--A) coordinate (G)
				;
				%Vẽ khối chóp S.ABCD.
				\draw (B)--(S)--(C)--cycle (S)--(D)--(C)(S)--(M)(S)--(N)(E)--(M)(F)--(N);
				\draw[dashed] (A)--(D) (S)--(A)--(B)(A)--(C)(B)--(D)(P)--(N)(P)--(M)(S)--(I)(M)--(N)
				(E)--(F)(G)--(E)(G)--(F);
				%Gán nhãn.
				\foreach \x/\y in {A/180,B/-90,C/-90,D/0,S/90,M/-90,N/0,P/90,I/-90,K/-90,E/90,F/90,G/0}{\fill (\x) circle(1pt) ($(\x)+(\y:0.3cm)$) node{$\x$};}
		\end{tikzpicture}}
		\noindent $(MNP) \cap (SAB)=EG$; $(MNP) \cap (SAD)=GF$ \\
		$(MNP) \cap (SCD)=FN$.\\
		Thiết diện là ngũ giác $MNFGE$.}
\end{ex}
\begin{ex}%[1K4BA-5]
	Cho hình chóp $S.ABCD$ có đáy $ABCD$ là hình bình hành. Gọi $M$, $N$, $P$ lần lượt là các trung điểm của các cạnh $BC$, $AD$ và $SD$. Thiết diện của hình chóp $S.ABCD$ khi cắt bởi mặt phẳng $(MNP)$ là
	\choice
	{\True Một hình thang}
	{Một ngũ giác}
	{Một tam giác}
	{Một hình bình hành}
	\loigiai{
		\immini{
			Ta có $P \in (MNP) \cap (SCD)$ và $MN \parallel CD$\\ $\Rightarrow (SCD) \cap (MNP)=PQ \parallel MN \parallel CD$ với $Q$ là trung điểm của $SC$.\\
			Khi đó	$(SBC) \cap (MNP)=MQ$, $(ABCD) \cap (MNP)=MN$, \\
			$(SAD) \cap (MNP)=NP$.\\
			Do đó thiết diện của hình chóp $S.ABCD$ khi cắt bởi mặt phẳng $(MNP)$ là hình tứ giác $MNPQ$ \\
			Mà $MN \parallel PQ \Rightarrow MNPQ$ là hình thang.}{
			\begin{tikzpicture}[scale=1, font=\footnotesize,>=stealth]%<DTools>
				%Gán số liệu.
				\def\canhAD{4};\def\canhBA{2};\def\gocBAD{-130};\def\h{3};\def\xdinhS{1};
				%Gán tọa độ.
				\coordinate (A) at (0,0);
				\coordinate (B) at ($(A)+(\gocBAD:\canhBA)$);
				\coordinate (C) at ($(B)+(0:\canhAD)$);
				\coordinate (D) at ($(A)+(0:\canhAD)$);
				\coordinate (S) at ($(A)+(\xdinhS,\h)$);
				\path
				($(B)!0.5!(C)$) coordinate (M)
				($(A)!0.5!(D)$) coordinate (N)
				($(S)!0.5!(D)$) coordinate (P)
				($(S)!0.5!(C)$) coordinate (Q)
				;
				%Vẽ khối chóp S.ABCD.
				\draw (B)--(S)--(C)--cycle (S)--(D)--(C)(P)--(Q)(Q)--(M);
				\draw[dashed] (A)--(D) (S)--(A)--(B)(M)--(N)--(P)--cycle;
				%Gán nhãn.
				\foreach \x/\y in {A/180,B/-90,C/-90,D/0,S/90,N/-90,P/90,Q/90,M/-90}{\fill (\x) circle(1pt) ($(\x)+(\y:0.3cm)$) node{$\x$};}
	\end{tikzpicture}}}
\end{ex}

\begin{ex}%[1K4BA-5]
	Cho hình chóp $S.ABCD$ có đáy $ABCD$ là hình bình hành tâm $O$. Gọi $I$, $J$ lần lượt là trung điểm $SA$, $SB$, $M$ là giao điểm của $IC$ và $JD$. Khẳng định nào sau đây \textbf{SAI}?
	\choice
	{$IJ \parallel CD$}
	{\True $ ID \parallel JC$}
	{$(SAC) \cap (SBD) = SO$}
	{$IJ = \dfrac{1}{2} AB$}
	\loigiai{
		\immini{
			\begin{itemize}
				\item Ta có $IJ$ là đường trung bình tam giác $SAB$ nên $IJ \parallel AB$ và $IJ = \dfrac{1}{2}AB$.\\
				Do đó $IJ \parallel CD$ (do $ABCD$ là hình bình hành nên $AB \parallel CD$).
				\item Giao tuyến của hai mặt phẳng $(SAC)$ và $(SBD)$ là $SO$.
				\item Do $IJ \parallel CD$ và $IJ = \dfrac{1}{2}CD$ nên $IJCD$ là hình thang đáy là $IJ$ và $CD$ nên $DI$ và $CJ$ cắt nhau.
		\end{itemize}}{
			\begin{tikzpicture}[scale=0.7, line join = round, line cap = round]
				
			\end{tikzpicture}
		}
	}
\end{ex}

\begin{ex}%[1K4BA-5]
	Cho hình chóp $S.ABCD$ có đáy $ABCD$ là hình bình hành. Gọi $M$ là trung điểm cạnh $SC$. Thiết diện của hình chóp $S.ABCD$ khi cắt bởi mặt phẳng $(ABM)$ là
	\choice
	{Một hình bình hành}
	{Một hình thang cân}
	{\True Một hình thang}
	{Một hình thoi}
	\loigiai{
		\immini{Ta có $(SBC) \cap (ABM)=BM$ \\
			Ta lại có $M \in (MAB) \cap (SCD)$ và $AB \parallel CD$\\
			$\Rightarrow (SCD) \cap (ABM)=MN \parallel CD \parallel AB$ \\
			Và $(SAD) \cap (ABM)=AN$ \\
			Do đó thiết diện của hình chóp $S.ABCD$ khi cắt bởi mặt phẳng $(ABM)$ là hình tứ giác $ABMN$ \\
			Mà $AB \parallel MN \Rightarrow ABMN$ là hình thang.}{
			\begin{tikzpicture}[scale=1, font=\footnotesize,>=stealth]%<DTools>
				%Gán số liệu.
				\def\canhAD{4};\def\canhBA{2};\def\gocBAD{-130};\def\h{3};\def\xdinhS{1};
				%Gán tọa độ.
				\coordinate (A) at (0,0);
				\coordinate (B) at ($(A)+(\gocBAD:\canhBA)$);
				\coordinate (C) at ($(B)+(0:\canhAD)$);
				\coordinate (D) at ($(A)+(0:\canhAD)$);
				\coordinate (S) at ($(A)+(\xdinhS,\h)$);
				\path
				($(S)!0.5!(C)$) coordinate (M)
				($(S)!0.5!(D)$) coordinate (N)
				;
				%Vẽ khối chóp S.ABCD.
				\draw (B)--(S)--(C)--cycle (S)--(D)--(C)(B)--(M)(M)--(N);
				\draw[dashed] (A)--(D) (S)--(A)--(B)(A)--(M)(A)--(N);
				%Gán nhãn.
				\foreach \x/\y in {A/180,B/-90,C/-90,D/0,S/90,M/0,N/90}{\fill (\x) circle(1pt) ($(\x)+(\y:0.3cm)$) node{$\x$};}
	\end{tikzpicture}}}
\end{ex}
\begin{ex}%[1K4KA-5]
	Cho tứ diện $ABCD$, điểm $E$ nằm giữa hai điểm $A$ và $C$. Gọi $(P)$ là mặt phẳng qua $E$ và song song với hai đường thẳng $AB$, $CD$. Thiết diện của tứ diện $(ABCD)$ khi cắt bởi mặt phẳng $(P)$ là hình gì?
	\choice
	{Hình vuông}
	{Hình thoi}
	{Hình thang cân}
	{\True hình bình hành}
	\loigiai{
		\immini{Mặt phẳng $(ABC)$ chứa đường thẳng $AB$ song song với mặt phẳng $(P)$ nên mặt phẳng $(ABC)$ cắt mặt phẳng $(P)$ theo giao tuyến song song với $AB$.\\
			Vẽ $EF\parallel AB$ ($F$ thuộc $BC$) thì $EF$ là giao tuyến của $(P)$ và $(ABC)$.\\
			Hai mặt phẳng $(ACD)$ và $(BCD)$ cùng chứa đường thẳng $CD$ song song với mặt phẳng $(P)$ nên chúng cắt mặt phẳng $(P)$ theo giao tuyến song song với $CD$.\\
			Vẽ $EH$, $FG$ song song với $CD$ ($H$ thuộc $AD$, $G$ thuộc $BD$) thì $EH$, $FG$ lần lượt là giao tuyến của mặt phẳng $(P)$ với hai mặt phẳng $(ACD)$, $(BCD)$. Khi đó $GH$ là giao tuyến của $(P)$ và $(A B D)$.\\
			Mặt phẳng $(ABD)$ chứa đường thẳng $AB$ song song với mặt phẳng $(P)$ nên giao tuyến $GH$ của $(ABD)$ và $(P)$ song song với $AB$.\\
			Tứ giác $EFGH$ có $E F\parallel GH$ (vì cùng song song với $AB$ ) và $EH\parallel FG$  (vì cùng song song với $CD$ ) nên nó là hình bình hành.}
		{\begin{tikzpicture}[scale=.9,font=\footnotesize,line join=round,line cap=round,>=stealth]
				\path 
				(0,0)coordinate(B) 
				(4,0)coordinate(D) 
				(2.2,-2)coordinate(C) 
				(1,4)coordinate(A)
				($(A)!.5!(C)$)coordinate(E) 
				($(B)!.5!(C)$)coordinate(F) 
				($(B)!.5!(D)$)coordinate(G) 
				($(A)!.5!(D)$)coordinate(H) 		 
				;
				\draw (A)--(B)--(C)--(D)--cycle (C)--(A) (F)--(E)--(H)
				;
				\draw [dashed](B)--(D) (F)--(G)--(H)
				;					 
				\draw 
				pic[draw,angle radius=6mm]{angle=E--H--G}
				;		
				\path (H)+(-120:4.5mm)node{$P$}
				;
				\fill[cyan,opacity=0.15] (E)--(F)--(G)--(H)--cycle;
				\foreach \x/\g in {A/90,B/180,C/-90,D/0,E/180/,F/-95,G/-90,H/90}\fill[black] (\x) circle (1pt) +(\g:.3)node{$\x$};
		\end{tikzpicture}}	
	}
\end{ex}
\begin{ex}%[1K4BA-5]
	Cho hình chóp $S. A B C D$ có đáy là hình thang $(A B \parallel C D)$. Gọi $E$ là một điểm nằm giữa $S$ và $A$. Gọi $(P)$ là mặt phẳng qua $E$ và song song với hai đường thẳng $A B, A D$. Xác định giao tuyến của $(P)$ và các mặt bên của hình chóp. Hình tạo bởi các giao tuyến là hình gì?
	\choice
	{\True Hình thang}
	{Hình thang cân}
	{Hình bình hành}
	{Hình chữ nhật}
	\loigiai{
		\immini  {$ \bullet $ Ta có $ \heva{&AB \parallel (P)\\&AC \subset (SAB)\\&(P)\cap (SAB)=EF \quad (F\in SB)} $\\
			$ \Rightarrow AB \parallel EF.\quad (1) $\\
			$ \bullet $ Tương tự $(P)\cap (SAD)=EM; (P)\cap (SCD)=MN; (P)\cap (SBC)=NF  $. \\Do đó 
			$ AD \parallel EM \quad (M\in SD). \quad (2) $\\
			$ MN \parallel AB \quad (N\in SC). \quad (3) $\\
			Từ $ (1); (2)$ và $(3) $, ta được hình tạo bởi $ EMNF $ là hình thang.}
		{\begin{tikzpicture}[>=stealth,line join=round,line cap=round,font=\footnotesize,scale=.6]
				\path (0,0)coordinate[label=left:$A$](A) (5,0)coordinate[label=right:$B$](B) (3,-2)coordinate[label=below:$C$](C) (1,-2)coordinate[label=below:$D$](D) (2,4)coordinate[label=above:$S$](S) ;
				\coordinate[label=left:$E$] (E) at ($(S)!.7!(A)$);
				\draw ($(E)+1/2*(A)-0*(D)$)--($(E)+0.7*(D)-1/2*(A)$); %Đường thẳng qua A song song với BC
				\coordinate[label=left:$M$] (M) at ($(S)!.7!(D)$);
				\coordinate[label=right:$N$] (N) at ($(S)!.7!(C)$);
				\coordinate[label=right:$F$] (F) at ($(S)!.7!(B)$);
				\draw (S)--(A)--(D)--(C)--(B)--(S) (D)--(S)--(C) (M)--(N)--(F);
				\draw[dashed] (A)--(B) (E)--(F);
				\foreach \diem in {S,A,B,C,D,E,M,N,F}\fill (\diem)circle(.8pt);
				
		\end{tikzpicture}}
	}
\end{ex}
\Closesolutionfile{ans}
\begin{indapan}{10}
	{ans/ans-1}
\end{indapan}

