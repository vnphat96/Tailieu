\section{Hai đường thẳng song song}
\subsection{Lý thuyết}
\subsubsection{Vị trí tương đối của hai đường thẳng}
\begin{dn}
	Cho hai đường thẳng $a$ và $b$ trong không gian.
	\begin{itemize}
		\item Nếu $a$ và $b$ cùng nằm trong một mặt phẳng thì ta nói $a$ và $b$ đồng phẳng. Khi đó, $a$ và $b$ có thể cắt nhau, song song với nhau hoặc trùng nhau.
		\item Nếu $a$ và $b$ không cùng nằm trong bất kì mặt phẳng nào thì ta nói $a$ và $b$ chéo nhau. Khi đó, ta cũng nói $a$ chéo với $b$, hoặc $b$ chéo với $a$.
	\end{itemize}
\end{dn}
Do đó: Cho hai đường thẳng $a$ và $b$ trong không gian. Khi đó, giữa hai đường thẳng sẽ có $4$ vị trí tương đối
\begin{center}
	\begin{tabular}{cc}
		\begin{tikzpicture}[scale=1, font=\footnotesize, line join=round, line cap=round, >=stealth]
			\coordinate (A) at (0,0);
			\coordinate (B) at (3,0);
			\coordinate (C) at (4,2);
			\coordinate (D) at ($(A)+(C)-(B)$);
			\draw (A)--(B)--(C)--(D)--cycle;
			\draw (0.7,0.8)--(2.7,1.9) node[midway,below]{$a$};
			\draw (1.4,0.5)--(3.7,1.5) node[midway,below]{$b$};
		\end{tikzpicture}
		&
		\begin{tikzpicture}[line join = round, line cap = round,>=stealth,font=\footnotesize,scale=1]
			\coordinate (A) at (0,0);
			\coordinate (B) at (3,0);
			\coordinate (C) at (4,2);
			\coordinate (D) at ($(A)+(C)-(B)$);
			\draw (A)--(B)--(C)--(D)--cycle;
			\draw[name path=a] (0.8,0.6)--(3.5,1.9) node[below]{$a$};
			\draw[name path=b] (1.5,1.8)--(2.9,0.5) node[below]{$b$};
			\path[name intersections={of=a and b}] (intersection-1) coordinate (I);
			\draw[fill=red] (I) circle (1pt) node[below]{$I$};
		\end{tikzpicture}
		\\
		$a$ song song $b$
		&
		$a$ cắt $b$ tại $I$\\
		\begin{tikzpicture}[scale=0.6, font=\footnotesize, line join=round, line cap=round, >=stealth]
			\def\bc{5} % cạnh BC
			\def\ba{3} % cạnh BA
			\def\gocB{60} % góc B của đáy
			\coordinate (B) at (0,0);
			\coordinate (A) at (\gocB:\ba);
			\coordinate (C) at (\bc,0);
			\coordinate (D) at ($(C)-(B)+(A)$);
			\coordinate (M) at (2,1);
			\coordinate (N) at (4.5,1.7);
			\coordinate (P) at (2.5,2);
			\coordinate (Q) at (4,0.7);
			\node at (N)[above right]{$a \equiv b$};
			\draw (A)--(B)--(C)--(D)--(A) ($(N)!1.2!(P)$)--($(P)!1.2!(N)$) ;
		\end{tikzpicture}
		&
		\begin{tikzpicture}[line join = round, line cap = round,>=stealth,font=\footnotesize,scale=.7]
			\coordinate (A) at (0,0);
			\coordinate (B) at (3,0);
			\coordinate (C) at (4,2);
			\coordinate (D) at ($(A)+(C)-(B)$);
			\coordinate (A') at ($(A)+(4,0)$);
			\coordinate (B') at ($(B)+(4,0)$);
			\coordinate (C') at ($(C)+(4,0)$);
			\coordinate (D') at ($(D)+(4,0)$);	
			\draw (A')--(B')--(C')--(D')--cycle;
			\draw (4.7,0.8)--(6.2,1.9) node[midway,below]{$a$};
			\coordinate (E) at (6,3.0);	
			\coordinate (F) at (7,-0.5);
			\path[name path=l] (A')--(B');
			\path[name path=m] (E)--(F);
			\path[name intersections={of=l and m}] (intersection-1) coordinate (I);
			\coordinate (G) at ($(E)!0.4!(F)$);	
			\draw  (E)--(G);
			\draw[dashed] (G)--(I) node[midway,right]{$b$};
			\draw (I)--(F);
		\end{tikzpicture}
		\\
		$a\equiv b$& $a$ và $b$ chéo nhau\\
	\end{tabular}
\end{center}
\begin{dn}
	\begin{itemize}
		\item Hai đường thẳng gọi là đồng phẳng nếu chúng cùng nằm trong một mặt phẳng.
		\item Hai đường thẳng gọi là chéo nhau nếu chúng không đồng phẳng.
		\item Hai đường thẳng gọi là song song nếu chúng đồng phẳng và không có điểm chung.
		\item Có đúng một mặt phẳng chứa hai đường thẳng song song.
	\end{itemize}
\end{dn}
\subsubsection{Tính chất hai đường thẳng song song}
\begin{tc}
	Trong không gian, qua một điểm không nằm trên một đường thẳng cho trước, có một và chỉ một đường thẳng song song với đường thẳng đó.
\end{tc}
\begin{tc}
	Trong không gian hai đường thẳng phân biệt cùng song song với đường thẳng thứ ba thì song song với nhau.
\end{tc}
\begin{dl}
	Nếu ba mặt phẳng đôi một cắt nhau theo ba giao tuyến phân biệt thì ba giao tuyến ấy đồng quy hoặc đôi một song song.
	\begin{center}
		\begin{tikzpicture}[scale=1, font=\footnotesize, line join=round, line cap=round, >=stealth]
			\path 
			(0,0) coordinate (A)
			(2.5,1.5) coordinate (B)
			(5,1) coordinate (C)
			(0,4.5) coordinate (x);
			\coordinate (D) at ($(A)+(x)$);
			\coordinate (E) at ($(B)+(x)$);
			\coordinate (F) at ($(C)+(x)$);
			\coordinate (M) at ($(A)!.35!(B)$);
			\coordinate (N) at ($(C)!.4!(B)$);
			\coordinate (I) at ($(E)!.3!(B)$);
			\path 
			($(I)!0.5!(M)$) coordinate (a)
			($(I)!0.5!(N)$) coordinate (b)
			($(I)!0.5!(E)$) coordinate (c);
			\draw[draw=none, fill=magenta!20] (D)--(A)--(B)--(E)--cycle;
			\draw[draw=none, fill=cyan!20] (E)--(F)--(C)--(B)--cycle;
			\draw[draw=none, fill=blue!20] (I)--(M)--(N)--cycle;
			\draw (A)--(D)--(E)--(F)--(C)--(N)--(M)--cycle;
			\draw (M)--(I) (N)--(I) (E)--(I);
			\draw[dashed] (I)--(B) (B)--(M) (B)--(N);
			\foreach \x/\y/\z/\t/\i in {B/A/D/P/magenta, F/C/B/Q/cyan, N/M/I/R/blue}{
				\path pic[draw,fill=\i!40,angle radius=19pt,angle eccentricity=0.7]{angle= \x--\y--\z};}
			%\foreach \i in {A,B,C,D,x,E,F,M,N,I}\draw (\i) circle (1pt) node {$\i$};
			\foreach \i/\j in {a/135, b/-170, c/180}\draw (\i) ++(\j:0.2) node {$\i$};
		\end{tikzpicture} \quad \quad \quad \quad
		\begin{tikzpicture}[scale=1, font=\footnotesize, line join=round, line cap=round, >=stealth]
			\path 
			(0,0) coordinate (A)
			(2.5,1.5) coordinate (B)
			(5,1) coordinate (C)
			(0,4.5) coordinate (x);
			\coordinate (D) at ($(A)+(x)$);
			\coordinate (E) at ($(B)+(x)$);
			\coordinate (F) at ($(C)+(x)$);
			\coordinate (M) at ($(A)!.35!(B)$);
			\coordinate (N) at ($(C)!.4!(B)$);
			\coordinate (M') at ($(M)+(x)$);
			\coordinate (N') at ($(N)+(x)$);
			\path 
			(intersection of B--E and M'--N') coordinate (I);
			\draw[draw=none, fill=magenta!20] (D)--(A)--(B)--(E)--cycle;
			\draw[draw=none, fill=cyan!20] (E)--(F)--(C)--(B)--cycle;
			\draw[draw=none, fill=blue!20] (M')--(M)--(N)--(N')--cycle;
			\draw (A)--(D)--(E)--(F)--(C)--(N)--(M)--cycle;
			\draw (M)--(M') (M')--(N') (N')--(N) (E)--(I);
			\draw[dashed] (I)--(B) (B)--(M) (B)--(N);
			\path 
			($(M')!0.5!(M)$) coordinate (c)
			($(E)!0.5!(B)$) coordinate (a)
			($(N')!0.5!(N)$) coordinate (b);
			\foreach \x/\y/\z/\t/\i in {B/A/D/P/magenta, F/C/B/Q/cyan, N/M/M'/R/blue}{
				\path pic[draw,fill=\i!40,angle radius=19pt,angle eccentricity=0.7]{angle= \x--\y--\z};}
			\foreach \i/\j in {a/0, b/0, c/00}\draw (\i) ++(\j:0.2) node {$\i$};
		\end{tikzpicture}
	\end{center}
\end{dl}
\begin{note}
	Nếu hai mặt phẳng phân biệt lần lượt chứa hai đường thẳng song song thì giao tuyến của chúng song song với hai đường thẳng đó.
\end{note}
\subsection{Hệ thống bài tập}
\setcounter{dang}{0}
\begin{dang}{Chứng minh hai đường thẳng song song}
	Cách 1: Sử dụng tính chất đường trung bình, định lí Ta-let để chứng minh hai đường thẳng song song.\\
	Cách 2: Chứng minh hai đường thẳng đó cùng song song với đường thẳng thứ ba.\\
	Cách 3: Áp dụng định lí giao tuyến của 3 mặt phẳng và hệ quả quả nó.
\end{dang}
\subsubsection{Bài tập tự luận}
\begin{bt}%[Dự án Toán 11-WTB-1]%[Nguyễn Tấn Linh]%[1H1B2-2]
	Cho tứ diện $ABCD$ có $I;J$ lần lượt là trọng tâm của tam giác $ABC$, $ABD$. Chứng minh rằng: $IJ\parallel CD$.
\loigiai{
	\immini
	{
		Gọi $M$ là trung điểm của $AB$.\\
	Xét tam giác $ABC$ có: $\dfrac{MI}{MC}=\dfrac{1}{3}$.\\
	Xét tam giác $ABD$ có: $\dfrac{MJ}{MD}=\dfrac{1}{3}$.\\
	Do $\dfrac{MI}{MC}=\dfrac{MJ}{MD}=\dfrac{1}{3}$ $\Rightarrow IJ\parallel CD$.
	}
	{
		\begin{tikzpicture}[>=stealth,line join=round,line cap=round,font=\footnotesize,scale=1]
			\coordinate (B) at (0,0);
			\coordinate (C) at (1,-1);
			\coordinate (D) at (4,0);
			\coordinate (A) at (1.2,3);
			\coordinate (M) at ($(A)!0.5!(B)$);
			\coordinate (I) at ($(C)!2/3!(M)$);
			\coordinate (J) at ($(D)!2/3!(M)$);
			\draw (B)--(C)--(D)--(A)--cycle (A)--(C)--(M);
			\draw[dashed] (B)--(D)--(M) (I)--(J);
			\foreach \diem/\goc in {B/180,C/-90,D/0,A/90,M/180,I/180,J/90} \fill[black](\diem) circle (1pt) ($(\diem)+(\goc:3mm)$) node{$\diem$};
		\end{tikzpicture}
	}
}
\end{bt}
\begin{bt}%[Dự án Toán 11-WTB-1]%[Nguyễn Tấn Linh]%[1H1B2-2]
	Cho tứ diện $ABCD$. Gọi $M,N,P,Q,R,S$ lần lượt là trung điểm của $AB,CD,BC,AD,AC,BD$. Chứng minh $MPNQ$ là hình bình hành. Từ đó suy ra ba đoạn $MN,PQ,RS$ cắt nhau tại trung điểm $G$ của mỗi đoạn.
\loigiai{
	Ta có $MQ$ là đường trung bình của tam giác $ABD$ $\Rightarrow \heva{&MQ\parallel DB\\&MQ=\dfrac{1}{2}BD.}$\hfill $\left(1\right)$
	\immini
	{
		$NP$ là đường trung bình của tam giác $BCD$ $\Rightarrow \heva{&PN\parallel BD\\&PN=\dfrac{1}{2}BD.}$\hfill $\left(2\right)$\\
	Từ $\left(1\right);\left(2\right)$ $\Rightarrow PN\parallel QM$ và $PN=QM$.\\
	Vậy $MPNQ$ là hình bình hành.\\
	$\Rightarrow MN$ và $PQ$ cắt nhau tại trung điểm $G$ của mỗi đường.\\
	Chứng minh tương tự, ta có: $QRPS$ là hình bình hành.\\
	$\Rightarrow QP$ và $RS$ cắt nhau tại trung điểm $G$ của mỗi đường.\\
	Vậy $MN,PQ,RS$ cắt nhau tại trung điểm $G$ của mỗi đoạn.
	}
	{
		\begin{tikzpicture}[>=stealth,line join=round,line cap=round,font=\footnotesize,scale=1]
			\coordinate (B) at (0,0);
			\coordinate (C) at (1,-1);
			\coordinate (D) at (4,0);
			\coordinate (A) at (2.3,3);
			\coordinate (M) at ($(A)!0.5!(B)$);
			\coordinate (N) at ($(C)!0.5!(D)$);
			\coordinate (P) at ($(B)!0.5!(C)$);
			\coordinate (Q) at ($(A)!0.5!(D)$);
			\coordinate (S) at ($(B)!0.5!(D)$);
			\coordinate (R) at ($(A)!0.5!(C)$);
			\coordinate (G) at ($(R)!0.5!(S)$);
			\draw (B)--(C)--(D)--(A)--cycle (A)--(C) (M)--(P) (N)--(Q);
			\draw[dashed] (B)--(D) (Q)--(M)--(N)--(P)--(Q) (R)--(S);
			\foreach \diem/\goc in {B/180,C/-90,D/0,A/90,M/180,Q/0,P/-90,N/-90,R/0,S/-90,G/0} \fill[black](\diem) circle (1pt) ($(\diem)+(\goc:3mm)$) node{$\diem$};
		\end{tikzpicture}
	}
}
\end{bt}
\subsubsection{Bài tập trắc nghiệm}
\Opensolutionfile{ans}[ans/ans-1-C4B11-Dang1]
\begin{ex}%[Dự án Toán 11-WTB-1]%[Nguyễn Tấn Linh]%[1H1Y2-2]
	Cho hai đường thẳng phân biệt không có điểm chung cùng nằm trong một mặt phẳng thì hai đường thẳng đó
	\choice
	{\True song song}
	{chéo nhau}
	{cắt nhau}
	{trùng nhau}
	\loigiai{
	}
\end{ex}
\begin{ex}%[Dự án Toán 11-WTB-1]%[Nguyễn Tấn Linh]%[1H1Y2-2]
	Trong các mệnh đề sau, mệnh đề nào đúng?
	\choice
	{\True Hai đường thẳng không có điểm chung là hai đường thẳng song song hoặc chéo nhau}
	{Hai đường thẳng chéo nhau khi chúng không có điểm chung}
	{Hai đường thẳng song song khi chúng ở trên cùng một mặt phẳng}
	{Khi hai đường thẳng ở trên hai mặt phẳng thì hai đường thẳng đó chéo nhau}
	\loigiai{
	}
\end{ex}
\begin{ex}%[Dự án Toán 11-WTB-1]%[Nguyễn Tấn Linh]%[1H1Y2-2]
	Trong các mệnh đề sau, mệnh đề nào đúng?
	\choice
	{Hai đường thẳng không có điểm chung thì chéo nhau}
	{Hai đường thẳng lần lượt nằm trên hai mặt phẳng phân biệt thì chéo nhau}
	{Hai đường thẳng phân biệt không song song thì chéo nhau}
	{\True Hai đường thẳng chéo nhau thì không có điểm chung}
	\loigiai{
	}
\end{ex}
\begin{ex}%[Dự án Toán 11-WTB-1]%[Nguyễn Tấn Linh]%[1H1Y2-2]
	Chọn mệnh đề sai trong các mệnh đề sau:
	\choice
	{Hai đường thẳng phân biệt có không quá một điểm chung}
	{Hai đường thẳng cắt nhau thì không song song với nhau}
	{\True Hai đường thẳng không có điểm chung thì song song với nhau}
	{Hai đường thẳng chéo nhau thì không có điểm chung}
	\loigiai{
	}
\end{ex}
\begin{ex}%[Dự án Toán 11-WTB-1]%[Nguyễn Tấn Linh]%[1H1Y2-2]
	Mệnh đề nào sau đây là mệnh đề đúng?
	\choice
	{Hai đường thẳng phân biệt không song song thì chéo nhau}
	{Hai đường thẳng nằm trong hai mặt phẳng phân biệt thì chúng chéo nhau}
	{\True Hai đường thẳng nằm trong một mặt phẳng thì chúng không chéo nhau}
	{Hai đường thẳng phân biệt không cắt nhau thì chéo nhau}
	\loigiai{
	}
\end{ex}
\begin{ex}%[Dự án Toán 11-WTB-1]%[Nguyễn Tấn Linh]%[1H1Y2-2]
	Mệnh đề nào đúng?
	\choice
	{\True Hai đường thẳng phân biệt cùng nằm trong một mặt phẳng thì không chéo nhau}
	{Hai đường thẳng phân biệt không cắt nhau thì chéo nhau}
	{Hai đường thẳng phân biệt không song song thì chéo nhau}
	{Hai đường thẳng phân biệt lần lượt thuộc hai mặt phẳng khác nhau thì chéo nhau}
	\loigiai{
	}
\end{ex}
\begin{ex}%[Dự án Toán 11-WTB-1]%[Nguyễn Tấn Linh]%[1H1Y2-2]
	Chọn mệnh đề đúng.
	\choice
	{\True Không có mặt phẳng nào chứa hai đường thẳng $a$ và $b$ thì ta nói $a$ và $b$ chéo nhau}
	{Hai đường thẳng song song nhau nếu chúng không có điểm chung}
	{Hai đường thẳng cùng song song với một đường thẳng thứ ba thì song song với nhau}
	{Hai đường thẳng cùng song song với một mặt phẳng thì song song với nhau}
	\loigiai{
	}
\end{ex}
\begin{ex}%[Dự án Toán 11-WTB-1]%[Nguyễn Tấn Linh]%[1H1Y2-2]
	Cho hai đường thẳng chéo nhau $a$ và $b$. Có bao nhiêu mặt phẳng chứa $a$ và song song với $b$?
	\choice
	{Vô số}
	{\True $1$}
	{$2$}
	{$0$}
	\loigiai{
	}
\end{ex}
\begin{ex}%[Dự án Toán 11-WTB-1]%[Nguyễn Tấn Linh]%[1H1Y2-2]
	Cho $a;b$ là hai đường thẳng song song với nhau. Chọn khẳng định sai:
	\choice
	{Hai đường thẳng $a$ và $b$ cùng nằm trong một mặt phẳng}
	{Nếu $c$ là đường thẳng song song với $a$ thì $c$ song song hoặc trùng với $b$}
	{Mọi mặt phẳng cắt $a$ đều cắt $b$}
	{\True Mọi đường thẳng cắt $a$ đều cắt $b$}
	\loigiai{
	}
\end{ex}
\begin{ex}%[Dự án Toán 11-WTB-1]%[Nguyễn Tấn Linh]%[1H1Y2-2]
	Cho hai đường thẳng $a$ và $b$. Điều kiện nào sau đây đủ để kết luận $a$ và $b$ chéo nhau ?
	\choice
	{$a$ và $b$ không có điểm chung}
	{$a$ và $b$ là hai cạnh của một hình tứ diện}
	{$a$ và $b$ nằm trên hai mặt phẳng phân biệt}
	{\True $a$ và $b$ không cùng nằm trên bất kỳ mặt phẳng nào}
	\loigiai{
	}
\end{ex}
\begin{ex}%[Dự án Toán 11-WTB-1]%[Nguyễn Tấn Linh]%[1H1Y2-2]
	Trong không gian, hai đường thẳng không đồng phẳng chỉ có thể:
	\choice
	{Song song với nhau}
	{Cắt nhau}
	{Trùng nhau}
	{\True Chéo nhau}
	\loigiai{
	}
\end{ex}
\begin{ex}%[Dự án Toán 11-WTB-1]%[Nguyễn Tấn Linh]%[1H1Y2-2]
	Trong không gian, nếu hai đường thẳng không có điểm chung thì ta có thể kết luận gì về hai đường thẳng đó ?
	\choice
	{Song song với nhau}
	{Chéo nhau}
	{Cùng thuộc một mặt phẳng}
	{\True Hoặc song song hoặc chéo nhau}
	\loigiai{
	}
\end{ex}
\begin{ex}%[Dự án Toán 11-WTB-1]%[Nguyễn Tấn Linh]%[1H1Y6-1]
	Mệnh đề nào sau đây là \textbf{sai}? Qua một phép chiếu song song, hình chiếu của hai đường thẳng chéo nhau có thể là:
	\choice
	{\True Hai đường thẳng chéo nhau}
	{Hai đường thẳng cắt nhau}
	{Hai đường thẳng song song với nhau}
	{Hai đường thẳng phân biệt}
	\loigiai{
	}
\end{ex}
\begin{ex}%[Dự án Toán 11-WTB-1]%[Nguyễn Tấn Linh]%[1H1Y6-1]
	Mệnh đề nào sau đây \textbf{sai}? Qua một phép chiếu song song, hình chiếu của hai đường thẳng cắt nhau có thể là:
	\choice
	{Hai đường thẳng cắt nhau}
	{\True Hai đường thẳng song song với nhau}
	{Hai đường thẳng trùng nhau}
	{Hai đường thẳng phân biệt}
	\loigiai{
	}
\end{ex}
\begin{ex}%[Dự án Toán 11-WTB-1]%[Nguyễn Tấn Linh]%[1H1Y2-2]
	Trong không gian, cho ba đường thẳng $a;b;c$. Trong các mệnh đề sau đây, mệnh đề nào đúng?
	\choice
	{Nếu hai đường thẳng cùng chéo với một đường thẳng thứ ba thì chúng chéo nhau}
	{Nếu hai đường thẳng cùng song song với đường thẳng thứ ba thì chúng song song với nhau}
	{\True Nếu $a\parallel b$ và $b;c$ chéo nhau thì $a$ và $c$ chéo nhau hoặc cắt nhau}
	{Nếu $a$ và $b$ cắt nhau, $b$ và $c$ cắt nhau thì $a$ và $c$ cắt nhau hoặc song song}
	\loigiai{
	}
\end{ex}
\begin{ex}%[Dự án Toán 11-WTB-1]%[Nguyễn Tấn Linh]%[1H1Y2-2]
	Cho các mệnh đề sau:
	\begin{enumerate}[(I)]
		\item Hai đường thẳng song song thì đồng phẳng.
		\item Hai đường thẳng không có điểm chung thì chéo nhau.
		\item Hai đường thẳng chéo nhau thì không có điểm chung.
		\item Hai đường thẳng chéo nhau thì không đồng phẳng.
	\end{enumerate}
	Có bao nhiêu mệnh đề đúng?
	\choice
	{$1$}
	{\True $3$}
	{$4$}
	{$2$}
	\loigiai{
	}
\end{ex}
\begin{ex}%[Dự án Toán 11-WTB-1]%[Nguyễn Tấn Linh]%[1H1Y2-2]
	Trong không gian cho hai đường thẳng song song $a$ và $b$. Kết luận nào sau đây đúng?
	\choice
	{Nếu $c$ cắt $a$ thì $c$ cắt $b$}
	{Nếu $c$ chéo $a$ thì $c$ chéo $b$}
	{Nếu $c$ cắt $a$ thì $c$ chéo $b$}
	{\True Nếu đường thẳng $c$ song song với $a$ thì $c$ song song hoặc trùng $b$}
	\loigiai{
	}
\end{ex}
\begin{ex}%[Dự án Toán 11-WTB-1]%[Nguyễn Tấn Linh]%[1H1Y2-2]
	Trong không gian, cho $3$ đường thẳng $a, b, c$, biết $a\parallel b$, $a$ và $c$ chéo nhau. Khi đó hai đường thẳng $b$ và $c$
	\choice
	{Trùng nhau hoặc chéo nhau}
	{\True Cắt nhau hoặc chéo nhau}
	{Chéo nhau hoặc song song}
	{Song song hoặc trùng nhau}
	\loigiai{
	Giả sử $b\parallel c\Rightarrow c\parallel a$.}
\end{ex}
\begin{ex}%[Dự án Toán 11-WTB-1]%[Nguyễn Tấn Linh]%[1H1Y2-2]
	Nếu ba đường thẳng không cùng nằm trong một mặt phẳng và đôi một cắt nhau thì ba đường thẳng đó
	\choice
	{\True đồng quy}
	{tạo thành tam giác}
	{trùng nhau}
	{cùng song song với một mặt phẳng}
	\loigiai{
		\begin{center}
			\begin{tikzpicture}
				\tkzDefPoints{0/0/A,0/5/B, 3/4/C,3/-1/D, -3/3/E, -3/-2/F, 0/4/I}
				\coordinate (J) at ($(A)!0.77!(F)$);
				\coordinate (M) at ($(A)!0.5!(D)$);
				\tkzDrawSegments[dashed](A,M A,J A,I)
				\tkzDrawSegments(F,J E,F E,B B,I B,C C,D D,M M,I M,J I,J)
				\tkzMarkAngles[size=0.8cm](F,E,B)
				\tkzLabelAngles[pos=0.4](B,E,F){$\beta$}
				\tkzMarkAngles[size=0.8cm](B,C,D)
				\tkzLabelAngles[pos=-0.4](B,C,D){$\gamma$}
				\tkzMarkAngles[size=0.9cm](M,J,I)
				\tkzLabelAngles[pos=0.5,yshift=+0.1cm](M,J,I){$\alpha$}
				\tkzLabelSegment[left](J,I){$a$}
				\tkzLabelSegment[right](M,I){$b$}
				\tkzLabelSegment[right](B,I){$c$}
			\end{tikzpicture}
		\end{center}
	Đặt $\left(\alpha\right)\equiv \left(a;b\right);\left(\beta\right)\equiv \left(a;c\right);\left(\gamma\right)\equiv \left(b;c\right)$
	Ta thấy, ba mặt phẳng $\left(\alpha\right);\left(\beta\right);\left(\gamma\right)$ cắt nhau theo ba giao tuyến phân biệt và ba giao tuyến $\left(a\right);\left(b\right);\left(c\right)$ đôi một cắt nhau nên chúng đồng quy tại $M$.
}
\end{ex}
\begin{ex}%[Dự án Toán 11-WTB-1]%[Nguyễn Tấn Linh]%[1H1Y2-2]
	Cho một tứ diện. Số cặp đường thẳng chứa cạnh của tứ diện đó mà chéo nhau là?
	\choice
	{$1$}
	{$2$}
	{\True $3$}
	{$4$}
	\loigiai{
	}
\end{ex}
\begin{ex}%[Dự án Toán 11-WTB-1]%[Nguyễn Tấn Linh]%[1H1Y2-2]
	Cho hình bình hành $ABCD$. Qua đỉnh $A$, kẻ đường thẳng $a$ song song với $BD$ và qua đỉnh $C$ kẻ đường thẳng $b$ không song song với $BD$. Khi đó
	\choice
	{Đường thẳng $a$ và đường thẳng $b$ chéo nhau}
	{Đường thẳng $a$ và đường thẳng $b$ cắt nhau}
	{Đường thẳng $a$ và đường thẳng $b$ không có điểm chung}
	{\True Nếu $a$ và $b$ không chéo nhau thì chúng cắt nhau}
	\loigiai{
	}
\end{ex}
\begin{ex}%[Dự án Toán 11-WTB-1]%[Nguyễn Tấn Linh]%[1H1Y2-2]
	Cho hai đường thẳng $a;b$ chéo nhau. Một đường thẳng $c$ song song với $a$. Có bao nhiêu vị trí tương đối giữa $b$ và $c$?
	\choice
	{$1$}
	{\True $2$}
	{$3$}
	{$4$}
	\loigiai{
	Nếu $c\parallel b$ thì $a\parallel b$ $\Rightarrow $ $c$ cắt $b$ hoặc $c$ và $b$ chéo nhau.}
\end{ex}
\begin{ex}%[Dự án Toán 11-WTB-1]%[Nguyễn Tấn Linh]%[1H1B2-2]
	Cho tứ diện $ABCD$, gọi $M$ và $N$ lần lượt là trung điểm các cạnh $AB$ và $CD$. Gọi $G$ là trọng tâm tam giác $BCD$. Đường thẳng $AG$ cắt đường thẳng nào trong các đường thẳng dưới đây?
	\choice
	{\True Đường thẳng $MN$}
	{Đường thẳng $CM$}
	{Đường thẳng $DN$}
	{Đường thẳng $CD$}
	\loigiai{
	\immini
	{
		Do $AG$ và $MN$ cùng nằm trong mặt phẳng $\left(ABN\right)$ nên hai đường thẳng cắt nhau.
	}
	{
		\begin{tikzpicture}[>=stealth,line join=round,line cap=round,font=\footnotesize,scale=1]
			\coordinate (B) at (0,0);
			\coordinate (C) at (1,-1);
			\coordinate (D) at (4,0);
			\coordinate (A) at (1.2,3);
			\coordinate (M) at ($(A)!0.5!(B)$);
			\coordinate (N) at ($(C)!0.5!(D)$);
			\coordinate (G) at ($(B)!2/3!(N)$);
			\draw (B)--(C)--(D)--(A)--cycle (A)--(C) (A)--(N);
			\draw[dashed] (B)--(D) (B)--(N)--(M) (A)--(G);
			\foreach \diem/\goc in {B/180,C/-90,D/0,A/90,G/-90,N/-90,M/180} \fill[black](\diem) circle (1pt) ($(\diem)+(\goc:3mm)$) node{$\diem$};
		\end{tikzpicture}
	}
}
\end{ex}
\begin{ex}%[Dự án Toán 11-WTB-1]%[Nguyễn Tấn Linh]%[1H1B2-2]
	\immini
	{
		Cho hình hộp $ABCD.EFGH$. Mệnh đề nào sau đây \textbf{sai}?
	\choice
	{$BG$ và $HD$ chéo nhau}
	{$BF$ và $AD$ chéo nhau}
	{$AB$ song song với $HG$}
	{\True $CG$ cắt $HE$}
	}
	{
		\begin{tikzpicture}[>=stealth,line join=round,line cap=round,font=\footnotesize,scale=1]
			\def\ngang{3.5}
			\def \doc{3}
			\coordinate (F) at (0,0);
			\coordinate (E) at (1,1.2);
			\coordinate (G) at (\ngang,0);
			\coordinate (H) at ($(G)-(F)+(E)$);
			\coordinate (A) at ($(E)+(90:\doc)$);
			\coordinate (B) at ($(F)-(E)+(A)$);
			\coordinate (C) at ($(G)-(E)+(A)$);
			\coordinate (D) at ($(H)-(E)+(A)$);
			\draw (B)--(F)--(G)--(H)--(D)--(A)--(B)--(C)--(D) (G)--(C);
			\draw[dashed] (E)--(H) (A)--(E)--(F);
			\foreach \diem/\goc in {E/160,F/-90,G/-90,H/0,B/180,C/-25,D/90,A/90} \fill[black](\diem) circle (1pt) ($(\diem)+(\goc:3mm)$) node{$\diem$};
		\end{tikzpicture}
	}
	\loigiai{
	Do $CG$ và $HE$ không cùng nằm trong một mặt phẳng nên hai đường thẳng này chéo nhau}
\end{ex}
\begin{ex}%[Dự án Toán 11-WTB-1]%[Nguyễn Tấn Linh]%[1H1B2-2]
	Cho tứ diện $ABCD$, gọi $I$ và $J$ lần lượt là trọng tâm của tam giác $ABD$ và $ABC$. Đường thẳng $IJ$ song song với đường nào?
	\choice
	{$AB$}
	{\True $CD$}
	{$BC$}
	{$AD$}
	\loigiai{
	\immini
	{
		Gọi $N,M$ lần lượt là trung điểm của $BC, BD$.\\
	$\Rightarrow $ $MN$ là đường trung bình của tam giác $BCD$ $\Rightarrow MN\parallel CD$.\hfill$\left(1\right)$\\
	$J;I$ lần lượt là trọng tâm các tam giác $ABC$ và $ABD$\\
	$\Rightarrow \dfrac{AI}{AM}=\dfrac{AJ}{AN}=\dfrac{2}{3}\Rightarrow IJ\parallel MN$.\hfill$\left(2\right)$\\
	Từ $\left(1\right)$ và $\left(2\right)$ suy ra $IJ\parallel CD$.
	}
	{
		\begin{tikzpicture}[>=stealth,line join=round,line cap=round,font=\footnotesize,scale=1]
			\coordinate (B) at (0,0);
			\coordinate (D) at (1,-1);
			\coordinate (C) at (4,0);
			\coordinate (A) at (1.2,3);
			\coordinate (M) at ($(B)!0.5!(D)$);
			\coordinate (N) at ($(B)!0.5!(C)$);
			\coordinate (I) at ($(A)!2/3!(M)$);
			\coordinate (J) at ($(A)!2/3!(N)$);
			\draw (B)--(D)--(C)--(A)--cycle (A)--(D) (A)--(M);
			\draw[dashed] (B)--(C) (M)--(N)--(A) (I)--(J);
			\foreach \diem/\goc in {B/180,D/-90,C/0,A/90,I/180,J/0,M/-90,N/45} \fill[black](\diem) circle (1pt) ($(\diem)+(\goc:3mm)$) node{$\diem$};
		\end{tikzpicture}
	}
}
\end{ex}
\begin{ex}%[Dự án Toán 11-WTB-1]%[Nguyễn Tấn Linh]%[1H1B2-2]
	Cho tứ diện $ABCD$. Gọi $M,N$ là hai điểm phân biệt cùng thuộc đường thẳng $AB$; $P,Q$ là hai điểm phân biệt cùng thuộc đường thẳng $CD$. Xác định vị trí tương đối của $MQ$ và $NP$.
	\choice
	{$MQ$ cắt $NP$}
	{$MQ\parallel NP$}
	{$MQ\equiv NP$}
	{\True $MQ,NP$ chéo nhau}
	\loigiai{
	\immini
	{
		Xét mặt phẳng $\left(ABP\right)$.\\
	Ta có $M,N$ thuộc $AB\Rightarrow M,N$ thuộc mặt phẳng $\left(ABP\right)$.\\
	Mặt khác: $CD\cap \left(ABP\right)=P$.\\
	Mà: $Q\in CD\Rightarrow Q\notin \left(ABP\right)\Rightarrow M,N,P,Q$ không đồng phẳng\\
	$\Rightarrow MQ$ và $NP$ chéo nhau.
	}
	{
		\begin{tikzpicture}[>=stealth,line join=round,line cap=round,font=\footnotesize,scale=1]
			\coordinate (B) at (0,0);
			\coordinate (C) at (1,-1);
			\coordinate (D) at (4,0);
			\coordinate (A) at (1.2,3);
			\coordinate (M) at ($(A)!0.3!(B)$);
			\coordinate (N) at ($(A)!0.7!(B)$);
			\coordinate (P) at ($(C)!0.3!(D)$);
			\coordinate (Q) at ($(C)!0.7!(D)$);
			\draw (B)--(C)--(D)--(A)--cycle (A)--(C);
			\draw[dashed] (B)--(D) (M)--(P)--(B) (N)--(Q);
			\foreach \diem/\goc in {B/180,C/-90,D/0,A/90,M/180,N/180,P/-90,Q/-90} \fill[black](\diem) circle (1pt) ($(\diem)+(\goc:3mm)$) node{$\diem$};
		\end{tikzpicture}
	}
}
\end{ex}
\begin{ex}%[Dự án Toán 11-WTB-1]%[Nguyễn Tấn Linh]%[1H1B2-2]
	Cho hình chóp $S.ABCD$ có đáy $ABCD$ là hình bình hành tâm $O$. Gọi $I,J$ lần lượt là trung điểm của $SA$ và $SC$. Đường thẳng $IJ$ song song với đường thẳng nào?
	\choice
	{$BC$}
	{\True $AC$}
	{$SO$}
	{$BD$}
	\loigiai{
	\immini
	{
		Dễ dàng thấy được: $IJ$ là đường trung bình của tam giác $SAC$ $\Rightarrow IJ\parallel AC$.
	}
	{
		\begin{tikzpicture}[>=stealth,line join=round,line cap=round,font=\footnotesize,scale=1]
			\coordinate (B) at (0,0);
			\coordinate (A) at (1,1.2);
			\coordinate (C) at (3.5,0);
			\coordinate (D) at ($(C)-(B)+(A)$);
			\coordinate (S) at ($(A)+(100:3.5)$);
			\coordinate (I) at ($(S)!0.5!(A)$);
			\coordinate (J) at ($(S)!0.5!(C)$);
			\coordinate (O) at ($(A)!0.5!(C)$);
			\draw (B)--(C)--(D)--(S)--cycle (S)--(C);
			\draw[dashed] (A)--(D) (S)--(A)--(B) (A)--(C) (B)--(D) (I)--(J);
			\foreach \diem/\goc in {A/180,B/-90,C/-90,D/0,S/90,I/180,J/0,O/-90} \fill[black](\diem) circle (1pt) ($(\diem)+(\goc:3mm)$) node{$\diem$};
		\end{tikzpicture}
	}
}
\end{ex}
\begin{ex}%[Dự án Toán 11-WTB-1]%[Nguyễn Tấn Linh]%[1H1B2-2]
	Trong mặt phẳng $\left(P\right)$, cho hình bình hành $ABCD$. Vẽ các tia $Bx,Cy,Dz$ song song với nhau, nằm cùng phía với mặt phẳng $\left(ABCD\right)$, đồng thời không nằm trong mặt phẳng $\left(ABCD\right)$. Một mặt phẳng đi qua $A$, cắt $Bx,Cy,Dz$ tương ứng tại $B',C',D'$ sao cho $BB'=2$, $DD'=4$. Tính $CC'$.
	\choice
	{$6$}
	{$8$}
	{$2$}
	{\True $3$}
	\loigiai{
	\immini
	{
		Ta có $AB'C'D'$ là hình bình hành.\\
	$AC'\cap BD'=I$ và $AC\cap BD=O$ $\Rightarrow OI$ là đường trung bình của tam giác $ACC'$ $\Rightarrow CC'=2\mathrm{O}I$.\\
	$BB'D'D$ là hình thang có $OI$ là đường trung bình\\
	$\Rightarrow OI=\dfrac{BB'+DD'}{2}=3$.\\
	Vậy $CC'=6$.
	}
	{
		\begin{tikzpicture}[line join = round, line cap = round,>=stealth,font=\footnotesize,scale=1]
			\foreach \x/\y/\diem in {0/0/A,4/0/D,1.5/1.7/B} \coordinate (\diem) at (\x,\y);
			\coordinate (C) at ($(B)+(D)-(A)$);
			\coordinate (x) at ($(B)+(100:4)$);
			\coordinate (z) at ($(D)!1.5!($(x)+(D)-(B)$)$);
			\coordinate (y) at ($(x)+(C)-(B)$);
			\coordinate (B') at ($(B)!0.3!(x)$);
			\coordinate (D') at ($(D)!0.35!(z)$);
			\coordinate (I) at ($(B')!0.5!(D')$);
			\coordinate (O) at ($(B)!0.5!(D)$);
			\coordinate (C') at (intersection of A--I and  C--y);
			\draw (A)--(D)--(C) (A)--(B')--(C')--(D')--(A)--(C') (B')--(D') (D)--(z) (C)--(y) (B')--(x);
			\draw[dashed] (A)--(B)--(C) (B)--(B') (O)--(I) (A)--(C) (B)--(D);
			\foreach \diem/\goc in {A/-90,D/-90,C/0,B/180,B'/180,D'/0,C'/0,I/90,O/-90} \fill[black](\diem) circle (1pt) ($(\diem)+(\goc:3mm)$) node{$\diem$};
			\foreach \x in {x,y,z}\node at (\x)[left]{$\x$};
		\end{tikzpicture}
	}
}
\end{ex}
\begin{ex}%[Dự án Toán 11-WTB-1]%[Nguyễn Tấn Linh]%[1H1B2-2]
	Cho tứ diện $ABCD$. Gọi $G$ và $E$ lần lượt là trọng tâm của tam giác $ABD$ và $ABC$. Mệnh đề nào dưới đây đúng?
	\choice
	{\True $GE\parallel CD$}
	{$GE$ cắt $AD$}
	{$GE$ cắt $CD$}
	{$GE$ và $CD$ chéo nhau}
	\loigiai{
	\immini
	{
		Ta có $\dfrac{AG}{AI}=\dfrac{AE}{AJ}=\dfrac{2}{3}$ $\Rightarrow EG\parallel IJ$
	Mà $IJ\parallel CD$
	$\Rightarrow EG\parallel CD$.
	}
	{
		\begin{tikzpicture}[>=stealth,line join=round,line cap=round,font=\footnotesize,scale=1]
			\coordinate (B) at (0,0);
			\coordinate (D) at (1,-1);
			\coordinate (C) at (4,0);
			\coordinate (A) at (1.2,3);
			\coordinate (J) at ($(B)!0.5!(D)$);
			\coordinate (I) at ($(B)!0.5!(C)$);
			\coordinate (E) at ($(A)!2/3!(J)$);
			\coordinate (G) at ($(A)!2/3!(I)$);
			\draw (B)--(D)--(C)--(A)--cycle (A)--(D) (A)--(J);
			\draw[dashed] (B)--(C) (J)--(I)--(A) (E)--(G);
			\foreach \diem/\goc in {B/180,D/-90,C/0,A/90,E/180,G/0,J/-90,I/45} \fill[black](\diem) circle (1pt) ($(\diem)+(\goc:3mm)$) node{$\diem$};
		\end{tikzpicture}
	}
}
\end{ex}
\begin{ex}%[Dự án Toán 11-WTB-1]%[Nguyễn Tấn Linh]%[1H1B2-2]
	Cho tứ diện $ABCD$. Trên các cạnh $AB,AD$ lần lượt lấy các điểm $M,N$ sao cho $\dfrac{AM}{AB}=\dfrac{AN}{AD}=\dfrac{1}{3}$. Gọi $P,Q$ lần lượt là trung điểm các cạnh $CD,CB$. Mệnh đề nào sau đây đúng
	\choice
	{\True Tứ giác$MNPQ$ là một hình thang}
	{Tứ giác $MNPQ$ là hình bình hành}
	{Bốn điểm $M,N,P,Q$ không đồng phẳng}
	{Tứ giác $MNPQ$ không có các cặp cạnh đối nào song song}
	\loigiai{
	\immini
	{
		Xét tam giác $ABD$ có: $\dfrac{AM}{AB}=\dfrac{AN}{AD}=\dfrac{1}{3}$ $\Rightarrow MN\parallel BD$.\\
	Xét tam giác $BCD$ có: $PQ$ là đường trung bình của tam giác $\Rightarrow PQ\parallel BD$.\\
	Vậy $PQ\parallel MN$ $\Rightarrow MNPQ$ là hình thang.
	}
	{
		\begin{tikzpicture}[>=stealth,line join=round,line cap=round,font=\footnotesize,scale=1]
			\coordinate (B) at (0,0);
			\coordinate (C) at (1,-1);
			\coordinate (D) at (4,0);
			\coordinate (A) at (1.2,3);
			\coordinate (M) at ($(A)!1/3!(B)$);
			\coordinate (N) at ($(A)!1/3!(D)$);
			\coordinate (Q) at ($(B)!1/2!(C)$);
			\coordinate (P) at ($(D)!1/2!(C)$);
			\draw (B)--(C)--(D)--(A)--cycle (A)--(C) (M)--(Q) (N)--(P);
			\draw[dashed] (B)--(D) (M)--(N) (Q)--(P);
			\foreach \diem/\goc in {B/180,C/-90,D/0,A/90,M/180,N/0,Q/-90,P/-90} \fill[black](\diem) circle (1pt) ($(\diem)+(\goc:3mm)$) node{$\diem$};
		\end{tikzpicture}
	}
}
\end{ex}
\begin{ex}%[Dự án Toán 11-WTB-1]%[Nguyễn Tấn Linh]%[1H1B2-2]
	Cho hai đường thẳng chéo nhau $a$ và $b$. Lấy $A,B$ thuộc $a$ và $C,D$ thuộc $b$. Khẳng định nào sau đây đúng khi nói về hai đường thẳng $AD$ và $BC$?
	\choice
	{Có thể song song hoặc cắt nhau}
	{Cắt nhau}
	{Song song nhau}
	{\True Chéo nhau}
	\loigiai{
	\immini
	{
		Theo giả thiết, $a$ và $b$ chéo nhau $\Rightarrow $ $a$ và $b$ không đồng phẳng.\\
	Giả sử $AD$ và $BC$ đồng phẳng.\\
	Nếu $AD\cap BC=I\Rightarrow I\in \left(ABCD\right)\Rightarrow I\in \left(a;b\right)$. Mà $a$ và $b$ không đồng phẳng, do đó, không tồn tại điểm $I$.\\
	Nếu $AD\parallel BC\Rightarrow $ $a$ và $b$ đồng phẳng.\\
	Vậy điều giả sử là sai. Do đó $AD$ và $BC$ chéo nhau.
	}
	{
		\begin{tikzpicture}[line join = round, line cap = round,>=stealth,font=\footnotesize,scale=.9]
			\foreach \x/\y/\diem in {0/0/y,6/0/z,1.5/2/x} \coordinate (\diem) at (\x,\y);
			\coordinate (x') at ($(x)+(85:3)$);
			\coordinate (y') at ($(y)+(x')-(x)$);
			\coordinate (z') at ($(x')+(z)-(x)$);
			\coordinate (A) at ($(x')+(-60:1.5)$);
			\coordinate (B) at ($(A)+(10:3)$);
			\coordinate (D) at ($(x)+(-30:1.5)$);
			\coordinate (C) at ($(D)+(-20:1.5)$);
			\coordinate (a) at (intersection of A--D and  y'--z');
			\coordinate (b) at (intersection of B--C and  y'--z');
			\draw (x)--(y)--(z) (x')--(y')--(z') (D)--(a) (C)--(b);
			\draw[dashed] (A)--(a) (B)--(b);
			\draw ($(A)!1.3!(B)$) node[above]{$a$}--($(B)!1.3!(A)$);
			\draw ($(D)!1.5!(C)$) node[above]{$b$}--($(C)!1.3!(D)$); 
			\foreach \diem/\goc in {A/90,B/90,D/-90,C/-90} \fill[black](\diem) circle (1pt) ($(\diem)+(\goc:3mm)$) node{$\diem$};
		\end{tikzpicture}
	}
}
\end{ex}
\begin{ex}%[Dự án Toán 11-WTB-1]%[Nguyễn Tấn Linh]%[1H1B2-2]
	Cho tứ diện $ABCD$ với $M,N,P,Q$ lần lượt là trung điểm của $AC,BC,BD,AD$. Tìm điều kiện để $MNPQ$ là hình thoi.
	\choice
	{$AB=BC$}
	{$BC=AD$}
	{$AC=BD$}
	{\True $AB=CD$}
	\loigiai{
	\immini
	{
		Xét tam giác $ABC$ có: $MN=\dfrac{1}{2}AB$.\\
	Xét tam giác $ABD$ có: $PQ=\dfrac{1}{2}AB\Rightarrow MN=PQ$.\\
	Chứng minh tương tự, ta có: $MQ=NP$.\\
	Vậy $MNPQ$ là hình bình hành.\\
	Để $MNPQ$ là hình thoi $\Leftrightarrow MN=NP\Leftrightarrow AB=CD$.
	}
	{
		\begin{tikzpicture}[>=stealth,line join=round,line cap=round,font=\footnotesize,scale=1]
			\coordinate (B) at (0,0);
			\coordinate (C) at (1,-1);
			\coordinate (D) at (4,0);
			\coordinate (A) at (1.2,3);
			\foreach \td/\a/\b in {M/A/C,Q/A/D,N/B/C,P/B/D}\coordinate (\td) at ($(\a)!0.5!(\b)$);
			\draw (B)--(C)--(D)--(A)--cycle (A)--(C) (N)--(M)--(Q);
			\draw[dashed] (B)--(D) (N)--(P)--(Q);
			\foreach \diem/\goc in {B/180,C/-90,D/0,A/90,N/-90,P/-90,Q/0,M/180} \fill[black](\diem) circle (1pt) ($(\diem)+(\goc:3mm)$) node{$\diem$};
		\end{tikzpicture}
	}
}
\end{ex}
\begin{ex}%[1K4KA-2]
	Cho hình chóp $S.ABCD$. Gọi $A'$, $B'$, $C'$, $D'$ lần lượt là trung điểm của các cạnh $SA$, $SB$, $SC$, $SD$. Trong các đường thẳng sau đây, đường thẳng nào không song song với $A'B'$?
	\choice
	{$AB$}
	{$CD$}
	{$C'D'$}
	{\True $SC$}
	\loigiai{
		\begin{center}
			\begin{tikzpicture}[scale=0.7, font=\footnotesize, line join=round, line cap=round, >=stealth]
				\coordinate (S) at (.5,4);
				\coordinate (A) at (0,0); 
				\coordinate (B) at (-2,-2);
				\coordinate (C) at (3,-2);
				\coordinate (D) at ($(A)+(C)-(B)$); 
				\coordinate (A') at ($(A)!.5!(S)$);
				\coordinate (B') at ($(B)!.5!(S)$);
				\coordinate (C') at ($(C)!.5!(S)$);
				\coordinate (D') at ($(D)!.5!(S)$);
				\draw (S)--(B)--(C)--(D)--(S)--(C) (B')--(C')--(D');
				\draw [dashed](S)--(A)--(B) (A)--(D) (B')--(A')--(D');
				\foreach \x/\g in {S/90,A/170,B/-90,C/-90,D/0,B'/180,A'/-60,D'/60, C'/0} \fill (\x) circle (1pt) +(\g:3mm) node{$\x$};
			\end{tikzpicture}
		\end{center}
		Do $A'B'$ và $SC$ không đồng phẳng nên $A'B'$ và $SC$ không song song nhau.
	}
\end{ex}

\begin{ex}%[1K4KA-2]
	Cho tứ diện $ABCD$. Các điểm $M$, $N$ lần lượt là trung điểm $BD$, $AD$. Các điểm $H$, $G$ lần lượt là trọng tâm các tam giác $BCD$; $ACD$. Đường thẳng $HG$ chéo với đường thẳng nào sau đây?
	\choice
	{$M N$}
	{\True $C D$}
	{$C N$}
	{$A B$}
	\loigiai{
		\begin{center}
			\begin{tikzpicture}[scale=1, font=\footnotesize, line join=round, line cap=round, >=stealth]
				\coordinate (A) at (1.5,4);
				\coordinate (D) at (0,0); 
				\coordinate (B) at (4,0);
				\coordinate (C) at (1,-1.6);
				\coordinate (O) at ($(C)!.5!(D)$);
				\coordinate (H) at ($(B)!2/3!(O)$);
				\coordinate (M) at ($(D)!.5!(B)$);
				\coordinate (N) at ($(D)!.5!(A)$);
				\coordinate (G) at ($(A)!2/3!(O)$);
				\draw (A)--(D)--(C)--(B)--(A)--(C) (A)--(O);
				\draw [dashed](O)--(B)--(D) (G)--(H) (N)--(M);
				\foreach \x/\g in {A/90,D/170,B/0,C/-90,G/180,M/60,N/180,O/180, H/-90} \fill (\x) circle (1pt) +(\g:3mm) node{$\x$};
			\end{tikzpicture}
		\end{center}
		Do $\dfrac{O G}{O A}=\dfrac{O H}{O B}=\dfrac{1}{3} \Rightarrow H G\parallel A B$.\\
		Xét tam giác $A B D$ có: $M N \parallel A B \Rightarrow H G \parallel M N$.\\
		Lại có $H G \cap C N=G$.\\
		Vậy $H G$ và $C D$ chéo nhau.
	}
\end{ex}

\begin{ex}%[1K4KA-2]
	Cho hình chóp $S.ABCD$ có đáy $A B C D$ là một hình thang với đáy $A D$ và $B C$. Biết $AD=a$, $BC=b$. Gọi $I$ và $J$ lần lượt là trọng tâm các tam giác $SAD$ và $SBC$. Mặt phẳng $(ADJ)$ cắt $SB$, $SC$ lần lượt tại $M$, $N$. Mặt phẳng $(B C I)$ cắt $SA$, $SD$ tại $P$, $Q$. Khẳng định nào sau đây là đúng?
	\choice
	{\True $M N$ song sonng với $P Q$}
	{$M N$ chéo với $P Q$}
	{$M N$ cắt với $P Q$}
	{$M N$ trùng với $P Q$}
	\loigiai{
		\begin{center}
			\begin{tikzpicture}[scale=1, font=\footnotesize, line join=round, line cap=round, >=stealth]
				\coordinate (S) at (1,3);
				\coordinate (A) at (0,0);
				\coordinate (B) at (1,-2);
				\coordinate (C) at (3.3,-2);
				\coordinate (D) at (6,0);
				\coordinate (i) at ($(A)!.5!(D)$);
				\coordinate (I) at ($(S)!2/3!(i)$);
				\coordinate (j) at ($(B)!.5!(C)$);
				\coordinate (J) at ($(S)!2/3!(j)$);
				\coordinate (M) at ($(S)!2/3!(B)$);
				\coordinate (N) at ($(S)!2/3!(C)$);
				\coordinate (P) at ($(S)!2/3!(A)$);
				\coordinate (Q) at ($(S)!2/3!(D)$);
				\draw(S)--(A) (S)--(B) (S)--(C) (S)--(D) (B)--(C) (A)--(B)--(C)--(D) (M)--(N);
				\draw[dashed,thin](A)--(D)--(J)--(A) (P)--(Q) (C)--(I)--(B) ;
				\foreach \i/\g in {S/90,A/180,B/-90,C/-90,D/0,I/90,J/-90,M/-120,N/-90,P/180,Q/0}\fill (\i) circle (1pt) +(\g:3mm) node{$\i$};
			\end{tikzpicture}
		\end{center}		
		Ta có $\heva{&M N=(A D J) \cap(S B C) \\& A D \subset(J A D) ; B C \subset(S B C) \\ &A D \parallel B C}\Rightarrow M N \parallel A D \parallel B C$.\\
		Tương tự: $\heva{&P Q=(I B C) \cap(S A D) \\ &A D \subset(S A D) ; B C \subset(I B C) \\ &A D \parallel B C}\Rightarrow P Q \parallel A D \parallel B C$.\\
		Vậy $M N \parallel P Q$.
	}
\end{ex}
\Closesolutionfile{ans}
\begin{indapan}{10} 
	{ans/ans-1-C4B11-Dang1}
\end{indapan}

\begin{dang}{Tìm giao tuyến của hai mặt phẳng}
	\begin{itemize}
		\item Cách 1: Tìm hai điểm chung phân biệt của hai mặt phẳng.
		\item Cách 2: Nếu hai mặt phẳng $(P)$; $(Q)$ lần lượt chứa hai đường thẳng song song $a$, $b$ và có $1$ điểm chung $M$ thì $(P) \cap(Q)=M x$ với $M x \parallel(a) \parallel(b)$.
	\end{itemize}
\end{dang}

\subsubsection{Bài tập tự luận}
\begin{bt}%[1K4KA-3]
	Cho hình chóp $S.ABCD$ có đáy là hình bình hành. Điểm $M$ thuộc cạnh $SA$, điểm $E$ và $F$ lần lượt là trung điểm của $AB$ và $BC$.
	\begin{enumerate}
		\item Xác định giao tuyến của hai mặt phẳng $(SAB)$ và $(SCD)$.
		\item Xác định giao tuyến của hai mặt phẳng $(MBC)$ và $(SAD)$.
		\item Xác định giao tuyến của hai mặt phẳng $(MEF)$ và $(SAC)$.
	\end{enumerate}
	\loigiai{
		\begin{enumerate}
			\immini{\item 
				Ta có $\heva{&S \in(SAB) \cap(SCD) \\ &AB \subset(SAB); CD \subset(SCD) \\ &AB \parallel CD}\\
				\Rightarrow S x=(SAB) \cap(SCD) \text { với } Sx \parallel AB \parallel CD$.
				\item 
				Ta có $\heva{&M \in SA \subset(SAD) \\ &M \in(MBC)}\Rightarrow M \in(MBC) \cap(SAD)$.\\
				Khi đó $\heva{&M \in(MBC) \cap(SAD) \\ &BC \subset(SBC); AD \subset(SAD)  \\ &BC \parallel AD}\\
				\Rightarrow M y=(MBC) \cap(SAD) \text { với } My \parallel BC \parallel AD$.}{\begin{tikzpicture}[scale=0.9, font=\footnotesize, line join=round, line cap=round, >=stealth]
					\coordinate (S) at (0,4);
					\coordinate (A) at (0,0); 
					\coordinate (B) at (-2,-2);
					\coordinate (C) at (4,-2);
					\coordinate (D) at ($(A)+(C)-(B)$); 
					\coordinate (E) at ($(A)!.5!(B)$);
					\coordinate (F) at ($(B)!.5!(C)$);
					\coordinate (M) at ($(S)!.4!(A)$);
					\coordinate (X) at ($(S)+(B)-(A)$);
					\coordinate (X1) at ($(S)!-.3!(X)$);
					\coordinate (T) at ($(S)!.4!(C)$);
					\coordinate (T1) at ($(M)!1.5!(T)$);
					\coordinate (Y) at ($(S)!.4!(D)$);
					\coordinate (Y1) at ($(M)!1.5!(Y)$);
					\draw (S)--(B)--(C)--(D)--(S)--(C) (S)--(X)node[above left]{$x$} (S)--(X1) (T)--(T1)node[right]{$t$} (Y)--(Y1)node[right]{$y$};
					\draw [dashed](S)--(A) (B)--(A)--(D) (A)--(C) (M)--(E)--(F)--(M) (B)--(M)--(C) (M)--(T) (M)--(Y);
					\foreach \x/\g in {S/90,A/170,B/-90,C/-90,D/0,E/180,F/-90,M/60} \fill (\x) circle (1pt) +(\g:3mm) node{$\x$};
			\end{tikzpicture}}
			\item Ta có $\heva{&M \in SA \subset(SAC) \\ &M \in(MEF)} \Rightarrow M \in(MEF) \cap(SAC)$.\\
			Vì $EF$ là đường trung bình của tam giác $ABC$ nên $EF \parallel AC$.\\
			Do $\heva{&M \in(MEF) \cap(SAC) \\ &EF \subset(MEF); AC \subset(SAC) \\ &E F \parallel AC}\Rightarrow Mt=(MEF) \cap(SAC) \text { với } EF \parallel AC \parallel Mt$.
		\end{enumerate}
	}
\end{bt}

\begin{bt}%[1K4KA-3]
	Cho hình chóp $S.ABCD$. Mặt đáy là hình thang có cạnh đáy lớn $AD$, $AB$ cắt $CD$ tại $K$, điểm $M$ thuộc cạnh $SD$.
	\begin{enumerate}
		\item Xác định giao tuyến $(d)$ của $(SAD)$ và $(SBC)$. Tìm giao điểm $N$ của $KM$ và $(SBC)$.
		\item Chứng minh rằng $AM$, $BN$, $(d)$ đồng quy.
	\end{enumerate}
	\loigiai{
		\immini{\begin{enumerate}
				\item 
				Ta có
				$\heva{&S \in(SAD) \cap(SBC) \\ &AD \subset(SAD); BC \subset(SBC) \\& AD \parallel BC}\\
				\Rightarrow S x=(SAD) \cap(SBC) \text { với } S x \parallel AD \parallel B C\Rightarrow(d) \equiv S x$.\\
				Trong $(SCD)$ gọi $N=KM \cap SC \Rightarrow\heva{&N \in KM \\ &N \in SC \subset(SBC)}\\
				\Rightarrow N=KM \cap(SBC)$.
				\item Ta có $(d)=(SAD) \cap(SBC)$.\\
				Trong $(AMK)$ gọi $O$ là giao điểm của $AM$ và $BN$.\\
				Suy ra $\heva{&O \in AM \subset(SAD) \\& O \in BN \subset(SBC)} \Rightarrow O \in(d)$.\\
				Vậy ba đường thẳng $(d)$; $BN$; $AM$ đồng quy tại $O$.
		\end{enumerate}}{\begin{tikzpicture}[scale=1, font=\footnotesize, line join=round, line cap=round, >=stealth]
				\coordinate (S) at (1,4);
				\coordinate (A) at (0,0); 
				\coordinate (D) at (5,0);
				\coordinate (K) at (2,-2);
				\coordinate (B) at ($(A)!0.6!(K)$);
				\coordinate (C) at ($(D)!0.6!(K)$);
				\coordinate (X) at ($(S)+(D)-(A)$); 
				\coordinate (X1) at ($(X)!1.3!(S)$);
				\coordinate (O) at ($(X)!0.6!(S)$);
				\path (intersection of O--A and S--D) coordinate (M);
				\path (intersection of M--K and S--C) coordinate (N);
				\path (intersection of N--O and S--D) coordinate (P);
				\draw (S)--(A)--(K)--(D)--(P) (S)--(K) (B)--(S)--(C) (X)node[above]{$x$}--(X1) (N)--(K) (N)--(O);
				\draw [dashed](A)--(D) (B)--(C) (A)--(O) (B)--(N) (S)--(P) (M)--(N);
				\foreach \x/\g in {S/90,A/170,D/0,C/-20,B/180,K/-90,O/90,M/90,N/0} \fill (\x) circle (1pt) +(\g:3mm) node{$\x$};
		\end{tikzpicture}}
	}
\end{bt}
\subsubsection{Bài tập trắc nghiệm}
\Opensolutionfile{ans}[ans/ans-1-C4B11-Dang2]
\begin{ex}
	Nếu hai mặt phẳng phân biệt lần lượt chứa hai đường thẳng song song thì giao tuyến của chúng (nếu có) sẽ
	\choice
	{Song song với hai đường thẳng đó}
	{\True Song song với hai đường thẳng đó hoặc trùng với một trong hai đường thẳng đó}
	{Trùng với một trong hai đường thẳng đó}
	{Cắt một trong hai đường thẳng đó}
	\loigiai{
		
	}
\end{ex}

\begin{ex}%[1K4BA-2]
	Cho hình chóp $S.ABCD$, đáy $ABCD$ là hình bình hành. Điểm $M$ thuộc cạnh $SC$ sao cho $SM=3MC$, $N$ là giao điểm của $SD$ và $(MAB)$. Khi đó, hai đường thẳng $CD$ và $MN$ là hai đường thẳng
	\choice
	{Cắt nhau}
	{Chéo nhau}
	{\True Song song}
	{Có hai điểm chung}
	\loigiai{
		\immini{Ta có $\heva{&M \in(MAB) \cap(SCD) \\ &A B \subset(MAB); CD \subset(SCD) \\ &AB \parallel CD}\Rightarrow Mx=(MAB) \cap(SCD) \text { với } Mx \parallel CD \parallel AB$.\\
			Gọi $N=Mx \cap SD$ trong $(SCD) \Rightarrow N=SD \cap(MAB)$.\\
			Vậy $MN$ song song với $CD$.}{\begin{tikzpicture}[scale=0.8, font=\footnotesize, line join=round, line cap=round, >=stealth]
				\coordinate (S) at (1,4);
				\coordinate (A) at (0,0); 
				\coordinate (D) at (-3,-2);
				\coordinate (C) at (2,-2);
				\coordinate (B) at ($(A)+(C)-(D)$); 
				\coordinate (M) at ($(S)!3/4!(C)$);
				\coordinate (N) at ($(S)!3/4!(D)$);
				\coordinate (X) at ($(M)!1.3!(N)$);
				\draw (S)--(D)--(C)--(B)--(S)--(C) (M)--(X) node[left]{$x$};
				\draw [dashed](S)--(A) (D)--(A)--(B) (A)--(M);
				\foreach \x/\g in {S/90,A/170,D/-90,C/-90,B/0,M/20,N/150} \fill (\x) circle (1pt) +(\g:3mm) node{$\x$};
		\end{tikzpicture}}
	}
\end{ex}

\begin{ex}%[1K4YA-2]
	Cho hình chóp $S.ABCD$ có đáy là hình chữ nhật. Mặt phẳng $(P)$ cắt các cạnh $SA$, $SB$, $SC$, $SD$ lần lượt tại $M$, $N$, $P$, $Q$. Gọi $I$ là giao điểm của $MQ$ và $NP$. Câu nào sau đây đúng?
	\choice
	{$SI \parallel AB$}
	{$SI \parallel AC$}
	{\True $SI \parallel AD$}
	{$SI \parallel BD$}
	\loigiai{
		\immini{Ta có $S I=(SBC) \cap(SAD)$.\\
			Khi đó $\heva{&SI=(SAD) \cap(SBC) \\ &A D \subset(SAD); BC \subset(SBC)\\ &AD \parallel BC}\Rightarrow SI \parallel BC \parallel AD$.}{\begin{tikzpicture}[scale=0.8, font=\footnotesize, line join=round, line cap=round, >=stealth]
				\coordinate (S) at (0,4);
				\coordinate (A) at (0,0); 
				\coordinate (B) at (-2,-2);
				\coordinate (C) at (3,-2);
				\coordinate (D) at ($(A)+(C)-(B)$); 
				\coordinate (Y) at ($(S)+(D)-(A)$);
				\coordinate (I) at ($(S)+(A)-(D)$);
				\coordinate (P) at ($(S)!.8!(C)$);
				\coordinate (Q) at ($(S)!.7!(D)$);
				\path (intersection of S--B and P--I) coordinate (N);
				\path (intersection of S--A and Q--I) coordinate (M);
				\draw (S)--(B)--(C)--(D)--(S)--(C) (S)--(I)--(P);
				\draw [dashed](S)--(A) (B)--(A)--(D) (I)--(Q);
				\foreach \x/\g in {S/90,A/170,B/-90,C/-90,D/0,I/180,P/0,N/200,Q/20,M/30} \fill (\x) circle (1pt) +(\g:3mm) node{$\x$};
		\end{tikzpicture}}
	}
\end{ex}

\begin{ex}%[1K4BA-2]
	Cho hình chóp $S.ABCD$ có đáy là hình thang đáy lớn là $CD$. Gọi $M$ là trung điểm của cạnh $SA$, $N$ là giao điểm của cạnh $SB$ và mặt phẳng $(MCD)$. Mệnh đề nào sau đây là mệnh đề đúng?
	\choice
	{$M N$ và $S D$ cắt nhau}
	{\True $M N \parallel C D$}
	{$M N$ và $S C$ cắt nhau}
	{$M N$ và $C D$ chéo nhau}
	\loigiai{
		\immini{Ta có $\heva{&MN=(MCD) \cap(SAB) \\ &CD \subset(MCD); AB \subset(SAB)  \\ &CD \parallel AB}\Rightarrow MN \parallel CD \parallel AB$.}{\begin{tikzpicture}[scale=0.8, font=\footnotesize, line join=round, line cap=round, >=stealth]
				\coordinate (S) at (1,5);
				\coordinate (D) at (0,0); 
				\coordinate (A) at (2,-2);
				\coordinate (C) at (6,0);
				\coordinate (B) at (4,-2);
				\coordinate (M) at ($(S)!.5!(A)$);
				\coordinate (N) at ($(S)!.5!(B)$);
				\coordinate (X) at ($(M)!1.8!(N)$);
				\draw (S)--(C)--(B)--(A)--(D)--(S)--(A)--(B)--(S) (D)--(M)--(X)node[right]{$x$};
				\draw [dashed](D)--(C)--(M);
				\foreach \x/\g in {S/90,A/-90,B/-90,C/0,D/180,M/150,N/60} \fill (\x) circle (1pt) +(\g:3mm) node{$\x$};
		\end{tikzpicture}}
		
	}
\end{ex}

\begin{ex}%[1K4BA-2]
	Mệnh đề nào sau đây đúng?
	\choice
	{\True Nếu một mặt phẳng cắt một trong hai đường thẳng song song thì mặt phẳng đó sẽ cắt đường thẳng còn lại.}
	{Hai mặt phẳng lần lượt đi qua hai đường thẳng song song thì cắt nhau theo một giao tuyến song song với một trong hai đường thẳng đó}
	{Nếu một đường thẳng cắt một trong hai đường thẳng song song thì đường thẳng đó sẽ cắt đường thẳng còn lại}
	{ Hai mặt phẳng có một điểm chung thì cắt nhau theo một giao tuyến đi qua điểm chung đó}
	\loigiai{
		
	}
\end{ex}

\begin{ex}%[1K4KA-3]
	Cho hình chóp $S.ABCD$ có đáy $ABCD$ là hình bình hành. Gọi $d$ là giao tuyến của hai mặt phẳng $(S A D)$ và $(S B C)$. Khẳng định nào sau đây đúng?
	\choice
	{\True $d$ qua $S$ và song song với $BC$}
	{$d$ qua $S$ và song song với $DC$}
	{$d$ qua $S$ và song song với $AB$}
	{$d$ qua $S$ và song song với $BD$}
	\loigiai{
		\immini{Ta có $\heva{&S\in (SAD) \cap(SBC) \\
				&AD \subset(SAD), BC \subset(SBC) \\
				&AD \parallel BC}\Rightarrow(SAD) \cap(SBC)=Sx \parallel AD \parallel BC$.}{\begin{tikzpicture}[scale=0.8, font=\footnotesize, line join=round, line cap=round, >=stealth]
				\coordinate (S) at (1,4);
				\coordinate (A) at (0,0); 
				\coordinate (B) at (-3,-2);
				\coordinate (C) at (2,-2);
				\coordinate (D) at ($(A)+(C)-(B)$); 
				\coordinate (Y) at ($(S)+(D)-(A)$);
				\coordinate (Y1) at ($(S)!-.3!(Y)$);
				\draw (S)--(B)--(C)--(D)--(S)--(C) (S)--(Y)node[above left]{$d$} (S)--(Y1);
				\draw [dashed](S)--(A) (B)--(A)--(D) (A)--(C) (B)--(D);
				\foreach \x/\g in {S/90,A/170,B/-90,C/-90,D/0} \fill (\x) circle (1pt) +(\g:3mm) node{$\x$};
		\end{tikzpicture}}
	}
\end{ex}

\begin{ex}%[1K4KA-3]
	Cho tứ diện $ABCD$. Gọi $I$ và $J$ theo thứ tự là trung điểm của $AD$ và $AC$, $G$ là trọng tâm tam giác $BCD$. Giao tuyến của hai mặt phẳng $(GIJ)$ và $(BCD)$ là đường thẳng
	\choice
	{qua $I$ và song song với $AB$}
	{qua $J$ và song song với $BD$}
	{\True qua $G$ và song song với $CD$}
	{qua $G$ và song song với $BC$}
	\loigiai{
		\immini{Ta có
			$\heva{&G\in(GIJ)\cap(BCD)\\
				&IJ \subset(GIJ), CD \subset(BCD)\\&IJ \parallel C D}\Rightarrow(GIJ) \cap(BCD)=Gx \parallel IJ \parallel CD$.}{\begin{tikzpicture}[scale=1, font=\footnotesize, line join=round, line cap=round, >=stealth]
				\coordinate (C) at (0,0);
				\coordinate (A) at (2,4); 
				\coordinate (B) at (4,-2);
				\coordinate (D) at (5,0);
				\coordinate (I) at ($(A)!.5!(D)$);
				\coordinate (J) at ($(A)!.5!(C)$);
				\coordinate (M) at ($(B)!.5!(C)$);
				\coordinate (G) at ($(D)!2/3!(M)$);
				\coordinate (X) at ($(B)!2/3!(C)$);
				\coordinate (Y) at ($(B)!2/3!(D)$);
				\fill[green!50,smooth] (I)--(J)--(G)--cycle;
				\draw (A)--(C)--(B)--(D)--(A)--(B);
				\draw [dashed](C)--(D) (D)--(M) (I)--(J)--(G)--(I) (X) node[above right]{$x$}--(Y);
				\foreach \x/\g in {A/90,B/-90,C/180,D/0,I/0,J/180,M/-90,G/-90} \fill (\x) circle (1pt) +(\g:3mm) node{$\x$};
		\end{tikzpicture}}
	}
\end{ex}

\begin{ex}%[1K4BA-2]
	Cho ba mặt phẳng phân biệt $(\alpha)$, $(\beta)$, $(\gamma)$ có $(\alpha) \cap(\beta)=d_1$; $(\beta) \cap(\gamma)=d_2$; $(\alpha) \cap(\gamma)=d_3$. Khi đó ba đường thẳng $d_1$, $d_2$, $d_3$.
	\choice
	{Đôi một cắt nhau}
	{Đôi một song song}
	{Đồng quy}
	{\True Đôi một song song hoặc đồng quy}
	\loigiai{
		Nếu ba mặt phẳng đôi một cắt nhau theo ba giao tuyến phân biệt thì ba giao tuyến ấy hoặc đồng quy hoặc đôi một song song.
	}
\end{ex}

\begin{ex}%[1K4KA-5]
	Cho hình chóp $S.ABCD$ có đáy $ABCD$ là hình bình hành. Gọi $I$ là trung điểm $SA$. Thiết diện của hình chóp $S.ABCD$ cắt bởi mặt phẳng $(IBC)$ là
	\choice
	{Tam giác $IBC$}
	{\True Hình thang $IBCJ$ ($J$ là trung điểm $SD)$}
	{Hình thang $IGBC$ ($G$ là trung điểm $SB$)}
	{Tứ giác $IBCD$}
	\loigiai{
		\immini{Ta có $\heva{&I \in(IBC) \cap(SAD) \\ &BC \subset(IBC), AD \subset(SAD) \\ &BC \parallel AD}\Rightarrow(I B C) \cap(SAD)=Ix \parallel BC \parallel AD$.\\
			Trong mặt phẳng $(SAD)$ gọi $Ix \cap SD=J \Rightarrow IJ \parallel BC$.\\
			Vậy thiết diện của hình chóp $S.ABCD$ cắt bởi mặt phẳng $(IBC)$ là hình thang $IBCJ$.}{\begin{tikzpicture}[scale=0.8, font=\footnotesize, line join=round, line cap=round, >=stealth]
				\coordinate (S) at (1,4);
				\coordinate (A) at (0,0); 
				\coordinate (B) at (-3,-2);
				\coordinate (C) at (2,-2);
				\coordinate (D) at ($(A)+(C)-(B)$);
				\coordinate (I) at ($(S)!.5!(A)$);
				\coordinate (J) at ($(S)!.5!(D)$);
				\fill[green!50,smooth] (B)--(I)--(J)--(C)--cycle;
				\draw (S)--(B)--(C)--(D)--(S)--(C)--(J);
				\draw [dashed](S)--(I)node[midway,sloped]{|}--(A)node[midway,sloped]{|} (B)--(A)--(D) (A)--(C) (B)--(D) (B)--(I)--(J);
				\foreach \x/\g in {S/90,A/170,B/-90,C/-90,D/0,I/150,J/0} \fill (\x) circle (1pt) +(\g:3mm) node{$\x$};
		\end{tikzpicture}}
	}
\end{ex}

\begin{ex}%[1K4KA-5]
	Cho tứ diện $ABCD$, $M$ và $N$ lần lượt là trung điểm $AB$ và $AC$. Mặt phẳng $(\alpha)$ qua $MN$ cắt tứ diện $ABCD$ theo thiết diện là đa giác $(T)$. Khẳng định nào sau đây đúng?
	\choice
	{$(T)$ là hình chữ nhật}
	{$(T)$ là tam giác}
	{$(T)$ là hình thoi}
	{\True $(T)$ là tam giác hoặc hình thang hoặc hình bình hành}
	\loigiai{
		\begin{center}
			\begin{tikzpicture}[scale=1, font=\footnotesize, line join=round, line cap=round, >=stealth]
				\coordinate (B) at (0,0); 
				\coordinate (D) at (7,0);
				\coordinate (C) at (2,-2);
				\coordinate (A) at (3,4);
				\coordinate (M) at ($(A)!.5!(B)$);
				\coordinate (N) at ($(A)!.5!(C)$);
				\coordinate (K) at ($(A)!.4!(D)$);
				\fill[green!50,smooth] (M)--(N)--(K)--cycle;	
				\draw (A)--(M)node[midway,sloped]{|}--(B)node[midway,sloped]{|}--(C)--(N)node[midway,sloped]{||}--(A)node[midway,sloped]{||}--(D)--(C) (M)--(N)--(K);
				\draw [dashed](B)--(D) (M)--(K);
				\foreach \x/\g in {A/90,B/180,C/-90,D/0,M/180,N/-60,K/20} \fill (\x) circle (1pt) +(\g:3mm) node{$\x$};
			\end{tikzpicture}
			\begin{tikzpicture}[scale=1, font=\footnotesize, line join=round, line cap=round, >=stealth]
				\coordinate (B) at (0,0); 
				\coordinate (D) at (7,0);
				\coordinate (C) at (3,-2);
				\coordinate (A) at (4,4);
				\coordinate (M) at ($(A)!.5!(B)$);
				\coordinate (N) at ($(A)!.5!(C)$);
				\coordinate (I) at ($(D)!.8!(B)$);
				\coordinate (J) at ($(D)!.8!(C)$);
				\fill[green!50,smooth] (M)--(N)--(J)--(I)--cycle;	
				\draw (A)--(M)node[midway,sloped]{|}--(B)node[midway,sloped]{|}--(C)--(N)node[midway,sloped]{||}--(A)node[midway,sloped]{||}--(D)--(C) (M)--(N)--(J);
				\draw [dashed](B)--(D) (M)--(I)--(J);
				\foreach \x/\g in {A/90,B/180,C/-90,D/0,M/180,N/30,I/150,J/-90} \fill (\x) circle (1pt) +(\g:3mm) node{$\x$};
			\end{tikzpicture}
		\end{center}
		Trường hợp $(\alpha) \cap AD=K$
		thì $(T)$ là tam giác $MNK$.\\
		Trường hợp $(\alpha) \cap(BCD)=IJ$, với $I \in BD$, $J\in CD$; $I$, $J$ không trùng $D$. Suy ra $(T)$ là tứ giác.
	}
\end{ex}

\begin{ex}%[1K4KA-3]
	Gọi $G$ là trọng tâm tứ diện $ABCD$. Giao tuyến của mặt phẳng $(ABG)$ và mặt phẳng $(CDG)$ là
	\choice
	{Đường thẳng đi qua trung điểm hai cạnh $BC$ và $AD$}
	{\True Đường thẳng đi qua trung điểm hai cạnh $AB$ và $CD$}
	{Đường thẳng đi qua trung điểm hai cạnh $AC$ và $BD$}
	{Đường thẳng $CG$}
	\loigiai{
		\begin{center}
			\begin{tikzpicture}[scale=1, font=\footnotesize, line join=round, line cap=round, >=stealth]
				\coordinate (B) at (0,0); 
				\coordinate (D) at (7,0);
				\coordinate (C) at (2,-2);
				\coordinate (A) at (3,4);
				\coordinate (M) at ($(A)!.5!(B)$);
				\coordinate (N) at ($(C)!.5!(D)$);
				\coordinate (G) at ($(M)!.5!(N)$);	
				\draw (A)--(M)node[midway,sloped]{||}--(B)node[midway,sloped]{||}--(C)--(N)node[midway,sloped]{|}--(D)node[midway,sloped]{|}--(A)--(C);
				\draw [dashed](B)--(D) (M)--(N);
				\foreach \x/\g in {A/90,B/180,C/-90,D/0,M/180,N/-90,G/20} \fill (\x) circle (1pt) +(\g:3mm) node{$\x$};
			\end{tikzpicture}
		\end{center}
	}
\end{ex}

\begin{ex}%[1K4KA-3]
	Cho Cho hình chóp $S.ABCD$ có đáy là hình bình hành. Qua $S$ kẻ $Sx$; $Sy$ lần lượt song song với $AB$, $AD$. Gọi $O$ là giao điểm của $AC$ và $BD$. Khi đó, khẳng định nào dưới đây đúng?
	\choice
	{Giao tuyến của $(SAC)$ và $(SBD)$ là đường thẳng $Sx$}
	{Giao tuyến của $(SBD)$ và $(SAC)$ là đường thẳng $Sy$}
	{\True Giao tuyến của $(SAB)$ và $(SCD)$ là đường thẳng $Sx$}
	{Giao tuyến của $(SAD)$ và $(SBC)$ là đường thẳng $Sx$}
	\loigiai{
		\immini{Ta có $\heva{&S \in(SAB) \cap(SCD) \\ &AB \subset(SAB); CD \subset(SCD) \\ &AB \parallel CD}\\\Rightarrow Sx=(SAB) \cap(SCD) \text { với } S x \parallel AB \parallel CD$.}{\begin{tikzpicture}[scale=0.8, font=\footnotesize, line join=round, line cap=round, >=stealth]
				\coordinate (S) at (1,4);
				\coordinate (A) at (0,0); 
				\coordinate (B) at (-3,-2);
				\coordinate (C) at (2,-2);
				\coordinate (D) at ($(A)+(C)-(B)$);
				\coordinate (O) at ($(A)!.5!(C)$); 
				\coordinate (X) at ($(S)+(B)-(A)$);
				\coordinate (Y) at ($(S)+(D)-(A)$);
				\coordinate (X1) at ($(S)!-.3!(X)$);
				\coordinate (Y1) at ($(S)!-.3!(Y)$);
				\draw (S)--(B)--(C)--(D)--(S)--(C) (S)--(X)node[left]{$x$} (S)--(Y)node[above]{$y$} (S)--(X1) (S)--(Y1);
				\draw [dashed](S)--(A) (B)--(A)--(D) (A)--(C) (B)--(D);
				\foreach \x/\g in {S/90,A/170,B/-90,C/-90,D/0,O/-90} \fill (\x) circle (1pt) +(\g:3mm) node{$\x$};
		\end{tikzpicture}}
	}
\end{ex}

\begin{ex}%[1K4KA-3]
	Cho hình chóp $S.ABCD$ có đáy $ABCD$ là hình bình hành. Mặt phẳng $(\alpha)$ qua $AB$ và cắt cạnh $SC$ tại $M$ ở giữa $S$ và $C$. Xác định giao tuyến $d$ giữa mặt phẳng $(\alpha)$ và $(SCD)$.
	\choice
	{Đường thẳng $d$ qua $M$ song song với $AC$}
	{\True Đường thẳng $d$ qua $M$ song song với $CD$}
	{Đường thẳng $d$ trùng với $MA$}
	{Đường thẳng $d$ trùng với $MD$}
	\loigiai{
		\immini{Ta có $\heva{&M \in(\alpha) \cap(SCD) \\ &AB \subset(\alpha); CD \subset(SCD) \\ &AB \parallel CD}\Rightarrow M x=(SCD) \cap(\alpha) \text { với } Mx \parallel AB \parallel CD$.\\
			Vậy $Mx \equiv(d)$.}{\begin{tikzpicture}[scale=0.8, font=\footnotesize, line join=round, line cap=round, >=stealth]
				\coordinate (S) at (1,4);
				\coordinate (A) at (0,0); 
				\coordinate (B) at (-2,-2);
				\coordinate (C) at (2,-2);
				\coordinate (D) at ($(A)+(C)-(B)$);
				\coordinate (O) at ($(A)!.5!(C)$); 
				\coordinate (M) at ($(S)!.6!(C)$);
				\coordinate (N) at ($(S)!.6!(D)$);
				\coordinate (X) at ($(M)!1.3!(N)$);
				\coordinate (X1) at ($(M)!-.4!(X)$);
				\fill[green!50,smooth] (A)--(M)--(B)--cycle;
				\draw (S)--(B)--(C)--(D)--(S)--(C) (B)--(M)--(X)node[right]{$x$} (M)--(X1);
				\draw [dashed](S)--(A) (B)--(A)--(D) (A)--(C) (B)--(D) (M)--(A);
				\foreach \x/\g in {S/90,A/170,B/-90,C/-90,D/0,O/-90,M/160} \fill (\x) circle (1pt) +(\g:3mm) node{$\x$};
		\end{tikzpicture}}
	}
\end{ex}

\begin{ex}%[1K4KA-5]
	Cho tứ diện $ABCD$. Gọi $M$ và $N$ lần lượt là trung điểm của $AB$, $AC$. Gọi $E$ là điểm trên cạnh $CD$ với $ED=3EC$. Thiết diện tạo bởi mặt phẳng $(MNE)$ và tứ diện $ABCD$ là
	\choice
	{Tam giác $MNE$}
	{Tứ giác $MNEF$ với điểm $F$ bất kỳ trên cạnh $BD$}
	{Hình bình hành $MNEF$ với $F$ là điểm trên cạnh $BD$ thỏa mãn $EF \parallel BC$}
	{\True Hình thang $MNEF$ với $F$ là điểm trên cạnh $BD$ thỏa mãn $EF \parallel BC$}
	\loigiai{
		\immini{Ta có $\heva{&E \in(MNE) \cap(BCD) \\ &MN \subset(MNE); BD \subset(BCD) \\&MN \parallel BD}\Rightarrow Ex=(MNE) \cap(BCD) \text { với } E x \parallel BD \parallel MN$.\\
			Trong $(BCD)$ gọi $F=Ex \cap BC \Rightarrow EF=(BCD) \cap(MNE)$.\\
			Mặt khác $\heva{&MN=(MNE) \cap(ABD) \\ &NE=(MNE) \cap(ACD) \\ &MF=(MNE) \cap(ABC.}$\\
			Vậy thiết diện của mặt phẳng $(MNE)$ và tứ diện $ABCD$ là hình thang $MNEF$.}{
			\begin{tikzpicture}[scale=1, font=\footnotesize, line join=round, line cap=round, >=stealth]
				\coordinate (B) at (0,0); 
				\coordinate (D) at (7,0);
				\coordinate (C) at (2,-2);
				\coordinate (A) at (4,4);
				\coordinate (E) at ($(D)!.7!(C)$); 
				\coordinate (F) at ($(B)!.7!(C)$); 
				\coordinate (M) at ($(A)!.5!(B)$);
				\coordinate (N) at ($(A)!.5!(D)$);
				\coordinate (X) at ($(E)!1.4!(F)$);
				\coordinate (X1) at ($(F)!1.4!(E)$);
				\draw (A)--(M)node[midway,sloped]{||}--(B)node[midway,sloped]{||}--(C)--(D)--(N)node[midway,sloped]{|}--(A)node[midway,sloped]{|}--(C) (M)--(F) (N)--(E) (F)--(X)node[left]{$x$}(E)--(X1);
				\draw [dashed](B)--(D) (M)--(N) (M)--(E)--(F);
				\foreach \x/\g in {A/90,B/180,C/-90,E/-90,D/0,F/-100,M/180,N/0} \fill (\x) circle (1pt) +(\g:3mm) node{$\x$};
		\end{tikzpicture}}
	}
\end{ex}
\Closesolutionfile{ans}
\begin{indapan}{10} 
	{ans/ans-1-C4B11-Dang2}
\end{indapan}