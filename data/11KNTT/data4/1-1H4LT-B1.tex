\setcounter{section}{9} \setcounter{dang}{0}
\section{ĐƯỜNG THẲNG VÀ MẶT PHẲNG TRONG KHÔNG GIAN}
\subsection{KIẾN THỨC CẦN NHỚ}
\subsubsection{CÁC KHÁI NIỆM MỞ ĐẦU}
\begin{enumerate}[\iconMT]
	\item \indam{Mặt phẳng:} Để biểu diễn mặt phẳng, người ta dùng hình bình hành hay một miền góc
	\begin{listEX}[2]
		\item [] \begin{tikzpicture}[scale=0.7, line join=round, line cap=round]
			\tkzDefPoints{0/0/A,4/0/B,5.5/2/C,1.5/2/D}
			\tkzDrawPolygon(A,B,C,D)
			\draw (A)--(B)--(C)--(D)--cycle;
			\tkzMarkAngles[size=0.7cm,arc=l](B,A,D)
			\tkzLabelAngles[pos=0.4,rotate=30](B,A,D){\footnotesize$P$}
			\node[right] at (-0.4,-1) {Kí hiệu $(P)$ hoặc mp$(P)$};
		\end{tikzpicture}
		\item [] \begin{tikzpicture}[scale=0.7, line join=round, line cap=round]
			\tkzDefPoints{0/0/A,4/0/B,1.5/2/D}
			\tkzDrawSegments(A,B A,D)
			\tkzMarkAngles[size=0.7cm,arc=l](B,A,D)
			\tkzLabelAngles[pos=0.4,rotate=30](B,A,D){\footnotesize$\alpha$}
			\node[right] at (-0.4,-1) {Kí hiệu $(\alpha)$ hoặc mp$(\alpha)$};
		\end{tikzpicture}
	\end{listEX}
	\item \indam{Điểm thuộc mặt phẳng:}
	Cho điểm $A$, $B$ và mặt phẳng $(\alpha)$.
		\begin{tcolorbox}[colframe=\maudl,colback=cyan!3!white,boxrule=0.5mm]
	\immini{
			\begin{itemize}
				\item [\ding{172}] Khi $A$ thuộc mặt phẳng $(\alpha)$, ta kí hiệu $A \in (\alpha)$.
				\item [\ding{173}] Khi $B$ không thuộc mặt phẳng $(\alpha)$, ta kí hiệu $B \notin (\alpha)$.
				\begin{note}
					Dấu hiệu nhận biết $A \in (\alpha)$ là điểm $A$ thuộc một đường thẳng nằm trong $(\alpha)$
				\end{note}
			\end{itemize}
			}{
		\begin{tikzpicture}[scale=0.7, font=\footnotesize,>=stealth]
			\path
			%	Vẽ mp
			(0,0) coordinate (M)
			(5,0) coordinate (N)
			(6,2) coordinate (P)
			(1,2) coordinate (Q)
			(2,3) coordinate (B)
			(3,1) coordinate (A)
			(3.5,0) coordinate (A')
			;
			\draw (M)--(N)--(P)--(Q)--cycle (1.5,4)--(A) (A')--(4,-1);
			\draw[dashed] (A)--(A');
			\foreach \x/\g in {B/0,A/60}\draw[fill=black] (\x) circle (.05) +(\g:.5)node{\footnotesize$\x$};
			\draw
			pic["$P$",draw,angle radius=6mm]{angle=N--M--Q};
			\end{tikzpicture}
		}
	\end{tcolorbox}
		\item \indam{Biểu diễn hình không gian lên một mặt phẳng:}
		\begin{tcolorbox}[colframe=\maudl,colback=cyan!3!white,boxrule=0.5mm]
			\begin{itemize}
				\item[\ding{172}] Dùng nét vẽ liền để biểu diễn cho những đường trông thấy và dùng nét đứt đoạn (- - - -) để biểu diễn cho những đường bị che khuất.
				\item[\ding{172}]  Quan hệ thuộc, song song được giữ nguyên, nghĩa là
				\begin{itemize}
					\item Nếu hình thực tế điểm $A$ thuộc đường thẳng $\Delta$ thì hình biểu diễn phải giữ nguyên quan hệ đó.
					\item Nếu hình thực tế hai đường thẳng song song thì hình biểu diễn phải giữ nguyên quan hệ đó.
				\end{itemize}
			\end{itemize}
		\end{tcolorbox}
	\indamm{Hình biểu diễn của các mô hình không gian thường gặp:}\
	\begin{center}
		\begin{tikzpicture}[scale=0.5, line join=round, line cap=round]
			\tkzDefPoints{0/0/B,1.3/-1.6/C,4.5/0/D,1/3.5/A}
			\tkzDrawPolygon(A,B,C,D)
			\tkzDrawSegments(A,C)
			\tkzDrawSegments[dashed](B,D)
			\tkzDrawPoints[fill=black,size=4](A,B,C,D)
			\tkzLabelPoints[above,font=\footnotesize](A)
			\tkzLabelPoints[below,font=\footnotesize](C)
			\tkzLabelPoints[left,font=\footnotesize](B)
			\tkzLabelPoints[right,font=\footnotesize](D)
			\node[below right] at (0,-2.4) {Hình tứ diện};
		\end{tikzpicture}
		\begin{tikzpicture}[scale=0.5, line join=round, line cap=round]
			\tkzDefPoints{0/0/A,-1.3/-1.6/B,2.5/-1.6/C}
			\coordinate (D) at ($(A)+(C)-(B)$);
			\coordinate (S) at ($(A)+(100:3)$);
			\tkzDrawPolygon(S,B,C,D)
			\tkzDrawSegments(S,C)
			\tkzDrawSegments[dashed](A,S A,B A,D)
			\tkzDrawPoints[fill=black,size=4](D,C,A,B,S)
			\tkzLabelPoints[above,font=\footnotesize](S)
			\tkzLabelPoints[below,font=\footnotesize](A,B,C)
			\tkzLabelPoints[right,font=\footnotesize](D)
			\node[below] at (0.6,-2.4) {Hình chóp tứ giác đáy hbh};
		\end{tikzpicture}
		\begin{tikzpicture}[scale=0.5, line join=round, line cap=round]
			\tkzDefPoints{0/0/A,-1.3/-1.1/B,2/-1.1/C}
			\coordinate (D) at ($(A)+(C)-(B)$);
			\coordinate (A') at ($(A)+(0,2.5)$);
			\tkzDefPointsBy[translation=from A to A'](B,C,D){B'}{C'}{D'}
			\tkzDrawPolygon(A',B',B,C,D,D')
			\tkzDrawSegments(B',C' C',D' C,C')
			\tkzDrawSegments[dashed](A,B A,D A,A')
			\tkzDrawPoints[fill=black,size=4](A,B,D,C,A',B',C',D')
			\tkzLabelPoints[above,font=\footnotesize](A',D')
			\tkzLabelPoints[below,font=\footnotesize](A,B,C)
			\tkzLabelPoints[left,font=\footnotesize](B')
			\tkzLabelPoints[right,font=\footnotesize](C',D)
			\node[below] at (0.9,-2.4) {Hình lập phương, hộp chữ nhật};
		\end{tikzpicture}
	\end{center}	
\end{enumerate}
\subsubsection{CÁC TÍNH CHẤT THỪA NHẬN}
Xét trong không gian, ta thừa nhận các tính chất sau:
\begin{enumerate}[\iconMT]
	\item \indam{Tính chất 1:} Có một và chỉ một đường thẳng đi qua hai điểm phân biệt.
	\item \indam{Tính chất 2:} Có một và chỉ một mặt phẳng đi qua ba điểm không thẳng hàng.
	\item \indam{Tính chất 3:} Tồn tại $4$ điểm không cùng thuộc một mặt phẳng.
	\begin{tcolorbox}[colframe=\maudl,colback=cyan!3!white,boxrule=0.5mm]
		Một mặt phẳng hoàn toàn xác định nếu biết ba điểm không thẳng hàng thuộc mặt phẳng đó. Ta kí hiệu mặt phẳng đi qua ba điểm không thẳng hàng $A$, $B$, $C$ là $(A B C)$. Nếu có nhiều điểm cùng thuộc một mặt phẳng thì ta nói những điểm đó đồng phẳng. Nếu \textit{không} có mặt phẳng nào chứa các điểm đó thì ta nói những điểm đó \textit{không đồng phẳng}.
	\end{tcolorbox}
	\item \indam{Tính chất 4:} Nếu một đường thẳng có hai điểm thuộc một mặt phẳng thì tất cả các điểm của đường thẳng đều thuộc mặt phẳng đó.\\
	\begin{tcolorbox}[colframe=\maudl,colback=cyan!3!white,boxrule=0.5mm]
		Cho đường thẳng $d$ và mặt phẳng $(\alpha)$.
		\begin{itemize}
			\item [\ding{172}] Khi $d$ nằm trong $(\alpha)$, ta kí hiệu $d \subset (\alpha)$ hoặc $(P) \supset d$ .\quad (không được viết  $d \in (\alpha)$ nhé!!!)
			\item [\ding{173}] Khi $d$ không nằm trong $(\alpha)$, ta kí hiệu $d \not\subset (\alpha)$.
			\begin{note}
				Dấu hiệu nhận biết $d \subset (\alpha)$  là trên $d$ có hai điểm phân biệt thuộc $(\alpha)$
			\end{note}
		\end{itemize}
	\end{tcolorbox}
	\item \indam{Tính chất 5:} Nếu hai mặt phẳng phân biệt có điểm chung thì các điểm chung của hai mặt phẳng là một đường thẳng đi qua điểm chung đó.
	\immini{
	\begin{tcolorbox}[colframe=\maudl,colback=cyan!3!white,boxrule=0.5mm]
		Đường thẳng chung $d$ (nếu có) của hai mặt phẳng phân biệt $(P)$ và $(Q)$ được gọi là giao tuyến của hai mặt phẳng đó và kí hiệu là $d=(P) \cap(Q)$.
	\end{tcolorbox}}{
\begin{tikzpicture}[declare function={a=2.5;},font=\footnotesize]
	\path 
	(0,0) coordinate (a)
	(0,-a) coordinate (b)
	(b)+(-20:a/1.5) coordinate (c)
	($(a)+(c)-(b)$) coordinate (d)
	(b)+(220:a/1.5) coordinate (e)
	($(a)+(e)-(b)$) coordinate (f)
	($(a)!.35!(b)$) coordinate (M)
	;
	\draw (a)--(b)--(c)--(d)--cycle
	(b)--(e)--(f)--(a)
	;
	\draw pic[draw, angle radius=5mm]{angle=e--f--a};
	\path (f)+(-28:8pt) node{$\alpha$};
	\draw pic[draw, angle radius=5mm]{angle=a--d--c};
	\path (d)+(225:8pt) node{$\beta$};
	\path ($(a)!.7!(b)$)node[right]{$d$};
	\draw[fill=black] (M)node[right]{$M$} circle (1pt);
\end{tikzpicture}}
\item \indam{Tính chất 6:} Trên mỗi mặt phẳng các kết quả đã biết trong hình học phẳng đều đúng.
\end{enumerate}

\subsubsection{CÁCH XÁC ĐỊNH MỘT MẶT PHẲNG}
Ba cách xác định một mặt phẳng
\begin{tcolorbox}[colframe=\maudl,colback=cyan!3!white,boxrule=0.5mm]
	\begin{itemize}
		\item Một mặt phẳng được xác định nếu biết nó đi qua ba điểm $A,B,C$ không thẳng hàng của mặt phẳng, kí hiệu $\left(ABC\right) $.
		\item Một mặt phẳng được xác định nếu biết nó đi qua một đường thẳng $d$ và một điểm $A$ không thuộc $d,$ kí hiệu $\left(A,d\right) $.
		\item Một mặt phẳng được xác định nếu biết nó đi qua hai đường thẳng $a,b$ cắt nhau, kí hiệu $\left(a,b\right) $.
	\end{itemize}
\end{tcolorbox}
\subsubsection{HÌNH CHÓP VÀ HÌNH TỨ DIỆN}
\begin{enumerate}[\iconMT]
	\item \indam{Hình chóp:}
	\begin{itemize}
		\item [\iconCH] \indamm{Định nghĩa:} Cho đa giác $A_1A_2\ldots A_n$ và cho điểm $S$ nằm ngoài mặt phẳng chứa đa giác đó. Nối $S$ với các đỉnh $A_1,A_2,\ldots ,A_n$ ta được $n$ miền đa giác $SA_1A_2,SA_2A_3,\ldots ,SA_{n-1}A_n $.
		Hình gồm $n$ tam giác đó và đa giác $A_1A_2A_3\ldots A_n$ được gọi là hình chóp $S.A_1A_2A_3\ldots A_n $.
		\immini{\item[\iconCH] \indamm{Các tên gọi:}
			\begin{itemize}
				\item Điểm $S$ gọi là đỉnh của hình chóp.
				\item Đa giác $A_1A_2\ldots A_n$ gọi là mặt đáy của hình chóp.
				\item Các đoạn thẳng $A_{1}A_{2},A_{2}A_{3},\ldots,A_{n-1}A_{n}$ gọi là các cạnh đáy của hình chóp.
				\item Các đoạn thẳng $SA_{1},SA_{2},\ldots,SA_{n}$ gọi là các cạnh bên của hình chóp. 
				\item Các miền tam giác $SA_1A_2,SA_2A_3,\ldots ,SA_{n-1}A_n$ gọi là các mặt bên của hình chóp.	 
		\end{itemize}}{
			\begin{tikzpicture}[scale=1.3, line join=round, line cap=round]
				\tkzDefPoints{0/0/A1,0.3/-1/A2,1/-1.6/A3,1.7/-1.5/A4,2.2/-1/A5,2.5/0.3/A6,1/2/S,-1.5/-2/P,-0.7/0.3/Q,3/-2/R}
				\tkzDrawPolygon(S,A1,A2,A3,A4,A5,A6)
				\tkzDrawSegments(S,A2 S,A3 S,A4 S,A5 P,Q P,R)
				\tkzDrawSegments[dashed](A1,A6)
				\tkzDrawPoints[fill=black](S,A1,A2,A3,A4,A5,A6)
				\tkzLabelPoints[above](S)
				\tkzLabelPoints[above right](P)
				\tkzLabelPoint[left](A1){$A_1$}
				\tkzLabelPoint[left](A2){$A_2$}
				\tkzLabelPoint[below](A3){$A_3$}
				\tkzLabelPoint[below](A4){$A_4$}
				\tkzLabelPoint[right](A5){$A_5$}
				\tkzLabelPoint[right](A6){$A_6$}
				\tkzMarkAngle[size=0.6cm,opacity=.4,draw=black,mksize=2](R,P,Q)
		\end{tikzpicture}}
	\end{itemize}
	\item \indam{Hình tứ diện:}
	\begin{itemize}
		\item[\iconCH] \indamm{Định nghĩa:} 
		Cho bốn điểm $A, B, C, D$ không đồng phẳng. Hình gồm bốn tam giác $ABC$, $ACD$, $ABD$, $BCD$ được gọi là hình tứ diện và được kí hiệu là $ABCD$.
		\item[\iconCH] \indamm{Chú ý:} 
	\immini{	\begin{itemize}
			\item Hai cạnh không có đỉnh chung gọi là hai cạnh đối diện, đỉnh không nằm trên một mặt được gọi là đỉnh đối diện với mặt đó.
			\item Hình chóp tam giác còn được gọi là hình tứ diện. 
			\item Hình tứ diện có bốn mặt là những tam giác đều hay có tất cả các cạnh bằng nhau được gọi là hình tứ diện đều. 
		\end{itemize}
	}{
	\begin{tikzpicture}[scale=0.6, line join=round, line cap=round]
		\tkzDefPoints{0/0/B,1.3/-1.6/C,4.5/0/D,1/3.5/A}
		\tkzDrawPolygon(A,B,C,D)
		\tkzDrawSegments(A,C)
		\tkzDrawSegments[dashed](B,D)
		\tkzDrawPoints[fill=black,size=4](A,B,C,D)
		\tkzLabelPoints[above,font=\footnotesize](A)
		\tkzLabelPoints[below,font=\footnotesize](C)
		\tkzLabelPoints[left,font=\footnotesize](B)
		\tkzLabelPoints[right,font=\footnotesize](D)
		\node[below right] at (0,-2.4) {Hình tứ diện};
\end{tikzpicture}}
	\end{itemize}
\end{enumerate}
