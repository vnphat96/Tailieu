\subsection{PHÂN LOẠI, PHƯƠNG PHÁP GIẢI TOÁN}
\begin{dang}{Các quan hệ cơ bản}
	\begin{itemize}
		\item [\ding{172}] Chứng minh điểm $A$ thuộc $(\alpha)$: Ta chứng tỏ điểm $A$ thuộc đường thẳng $\Delta$ nằm trong $\alpha$, nghĩa là
		$$A \in \Delta, \Delta \subset (\alpha) \Rightarrow A \in (\alpha).$$
		\item [\ding{173}] Chứng minh đường thẳng $d$ nằm trong $(\alpha)$: Ta chứng tỏ $d$ có hai điểm phân biệt cùng thuộc $(\alpha)$, nghĩa là
		$$\heva{&A \in (\alpha), B \in (\alpha) \\& A,\,B \in d} \Rightarrow d \subset (\alpha).$$
		\item [\ding{174}] Chứng minh $A$ là điểm chung của hai mặt phẳng $(\alpha)$ và $(\beta)$:Ta thường sử dụng một trong hai cách sau
		$$\heva{&A \in (\alpha) \\& A \in (\beta)} \Rightarrow A \in (\alpha) \cap  (\beta)
		\text{ hoặc } 
		\heva{&d \subset (\alpha) \\& \Delta \subset (\beta)\\& d \cap \Delta = A} \Rightarrow A \in (\alpha) \cap  (\beta).$$
	\end{itemize}
\end{dang}
\begin{vd}
	Cho tam giác $ABC$ và điểm $S$ không thuộc mặt phẳng $\left(ABC\right)$. Lấy $D, E$ là các điểm lần lượt thuộc các cạnh $SA, SB \quad (D, E \text{ khác } S)$.
	\begin{itemize}
		\item [a)] Đường thẳng $DE$ có nằm trong mặt phẳng $\left(SAB\right)$ không?
		\item [b)] Giả sử $DE$ cắt $AB$ tại $F$. Chứng minh rằng $F$ là điểm chung của hai mặt phẳng $\left(SAB\right)$ và $\left(CDE\right)$.
	\end{itemize}
\end{vd}
\begin{vd}
	Cho hình chóp $S.ABCD$, gọi $O$ là giao điểm của $AC$ và $BD$. Lấy $M, N$ lần lượt thuộc các cạnh $SA, SC$.
	\begin{itemize}
		\item[a)] Chứng minh rằng đường thẳng $MN$ nằm trong mặt phẳng $\left(SAC\right)$.
		\item[b)] Chứng minh rằng $O$ là điểm chung của hai mặt phẳng $\left(SAC\right)$ và $\left(SBD\right)$.
	\end{itemize}
\end{vd}
\begin{vd}
	Cho hình tứ diện $ABCD$. Gọi $I$ là trung điểm cạnh $CD$. Gọi $M, N$ lần lượt là trọng tâm của các tam giác $BCD, CDA$.
	\begin{itemize}
		\item[a)] Chứng minh rằng các điểm $M, N$ thuộc mặt phẳng $\left(ABI\right)$.
		\item[b)] Gọi $G$ là giao điểm của $AM$ và $BN$. Chứng minh rằng $\dfrac{GM}{GA}=\dfrac{GN}{GB}=\dfrac{1}{3}$.
		\end{itemize}
\end{vd} 
\begin{dang}{Xác định giao tuyến của hai mặt phẳng}
	Cho hai mặt phẳng $(\alpha)$ và $(\beta)$ cắt nhau. Để xác định giao tuyến của chúng, ta đi tìm hai điểm chung phân biệt. Cụ thể, ta thường gặp một trong ba trường hợp sau:
\begin{itemize}
		\item [\ding{172}] Hai mặt phẳng $(\alpha)$ và $(\beta)$ có sẵn hai điểm chung phân biệt: Khi đó giao tuyến là đường thẳng qua hai điểm chung đó.
		\item [\ding{173}] Hai mặt phẳng  $(\alpha)$ và $(\beta)$ thấy trước một điểm chung $A$:
			\begin{itemize}
				\item [$\bullet$] $A$ là điểm chung thứ nhất hay $A \in (\alpha) \cap (\beta)$.
				\item [$\bullet$] Ta tìm điểm chung thứ 2: Trong $ (\alpha)$ tìm một đường thẳng $d_1$, trong $ (\beta)$ tìm một đường thẳng $d_2$ sao cho chúng có thể cắt nhau (đồng phẳng).			Gọi $B = d_1 \cap d_2$, suy ra $B \in (\alpha) \cap (\beta)$. Vậy $AB=(\alpha) \cap (\beta)$.
		\end{itemize}
		\item [\ding{174}] Hai mặt phẳng  $(\alpha)$ và $(\beta)$ chưa thấy điểm chung: Ta mở rộng mặt phẳng để tìm điểm chung tương tự như cách tìm điểm chung ở mục số \ding{173}.
	\end{itemize}
\end{dang}

\begin{vd}
	\immini{Cho tứ giác $ABCD$ sao cho các cạnh đối không song song với nhau. Lấy một điểm $S$ không thuộc mặt phẳng $\left(ABCD\right)$. Xác định giao tuyến của 
		\begin{itemize}
			\item [a)] Mặt phẳng $\left(SAC\right)$ và mặt phẳng $\left(SBD\right)$. 
			\item [b)] Mặt phẳng $\left(SAB\right)$ và mặt phẳng $\left(SCD\right)$.
			\item [c)] Mặt phẳng $\left(SAD\right)$ và mặt phẳng $\left(SBC\right)$.  
		\end{itemize}
	}{
		\begin{tikzpicture}[scale=0.6, font=\footnotesize, line join=round, line cap=round, >=stealth]
			\tkzDefPoints{0/0/A, 1/-1.5/D, 6/0/B, 4/-2.5/C, 2/3.5/S}
			\tkzLabelPoints[above](S)
			\tkzLabelPoints[above left](A)
			\tkzLabelPoints[below left](D)
			\tkzLabelPoints[right](B,C)
			\tkzDrawPoints[fill=black](A,B,C,D)
			\tkzDrawSegments[dashed](B,A)
			\tkzDrawSegments(S,A S,B S,C S,D A,D D,C C,B)
	\end{tikzpicture}}
	\loigiai{
		\begin{itemize}
			\immini{	
				\item[a)] Gọi $H$ là giao điểm của $AC$ với $BD$. 
				Khi đó 
				$$\heva{&H\in AC\\&H\in BD}\Rightarrow H\in \left(SAC\right)\cap\left(SBD\right)\quad (1).$$
				Dễ thấy 
				$S\in \left(SAC\right)\cap\left(SBD\right)\quad (2).$\\
				Từ $(1)$ và $(2)$ suy ra $SH=\left(SBD\right)\cap\left(SAC\right)$.
				\item[b)] Gọi $K$ là giao điểm của hai đường thẳng $CD$ và $AB$.\\
				Khi đó 
				$\heva{&K\in AB\\&K\in CD}\Rightarrow K\in \left(SAB\right)\cap\left(SCD\right)\quad (3)$.\\
				Dễ thấy 
				$S\in \left(SAB\right)\cap\left(SCD\right)\quad (4)$.\\
				Từ $(3)$ và $(4)$ suy ra $SK=\left(SAB\right)\cap\left(SCD\right)$.}{
				\begin{tikzpicture}[scale=0.6, font=\scriptsize, line join=round, line cap=round, >=stealth]
					\tkzDefPoints{0/0/A, 1/-1.5/D, 6/0/B, 4/-2.5/C, 2/4/S}
					\tkzInterLL(A,C)(B,D)\tkzGetPoint{H}
					\tkzInterLL(A,B)(C,D)\tkzGetPoint{K}
					\tkzInterLL(A,D)(B,C)\tkzGetPoint{L}
					\tkzLabelPoints[above](S)
					\tkzLabelPoints[above left](A)
					\tkzLabelPoints[below left](D)
					\tkzLabelPoints[below](H,K,L)
					\tkzLabelPoints[right](B,C)
					\tkzDrawPoints[fill=black](A,B,C,D,S,H,K,L)
					\tkzDrawSegments[dashed](B,K A,C B,D S,H S,A A,D C,D)
					\tkzDrawSegments(S,C S,B S,D D,K S,K B,C D,L C,L S,L)
			\end{tikzpicture}}
			\item[c)] Gọi $L$ là giao điểm của hai đường thẳng $AD$ và $BC$.\\
			Khi đó $\heva{&L\in AD\\&K\in BC}\Rightarrow L\in \left(SAD\right)\cap\left(SBC\right)\quad (5)$.
			Mặt khác $S\in \left(SAD\right)\cap\left(SBC\right)\quad (6)$.\\
			Từ $(5)$ và $(6)$ suy ra $SL=\left(SAD\right)\cap\left(SBC\right)$.
		\end{itemize}
	}
\end{vd}

\begin{vd}
	\immini{Cho tứ diện $ABCD$. Lấy các điểm $M$ thuộc cạnh $AB$, $N$  thuộc cạnh $AC$ sao cho $MN$ cắt $BC$. Gọi $I$ là điểm bên trong tam giác $BCD$. Tìm giao tuyến của mặt phẳng $\left(MNI\right)$ với các mặt phẳng $\left(ABC\right)$, $\left(BCD\right)$, $\left(ABD\right)$, $\left(ACD\right)$.
	}{
	\begin{tikzpicture}[scale=0.7, font=\footnotesize,>=stealth]
		\path
		%	Vẽ mp
		(0,0) coordinate (B)
		(5,0) coordinate (C)
		(1.5,-1.5) coordinate (D)
		(1,4) coordinate (A)
		(2.3,-0.7) coordinate (I)
		($(A)!0.6!(B)$)coordinate (M)
		($(C)!0.7!(A)$)coordinate (N)
		;
		\draw (B)--(A)--(D)--(C)--(A) (B)--(D);
		\draw[dashed] (M)--(N)--(I)--cycle (B)--(C);
		\foreach \x/\g in {B/180,A/90,C/0,D/-90,M/180,I/30,N/30}\draw[fill=black] (\x) circle (.05) +(\g:.5)node{\footnotesize$\x$};
\end{tikzpicture}}
	\loigiai{
		\begin{itemize}
			\immini{
				\item [a)] Dễ thấy $(MNI) \cap (ABC)=MN$.
				\item [b)] Tìm $(MNI) \cap (BCD)$.
				\begin{itemize}
					\item [$\bullet$] Gọi $H$ là giao điểm của $MN$ và $BC$. Suy ra 
					$$H\in \left(MNI\right)\cap\left(BCD\right)\quad (1).$$
					\item [$\bullet$]Do $I$ là điểm trong $\triangle BCD$ nên 
					$$I\in \left(MNI\right)\cap\left(BCD\right)\quad (2).$$
				\end{itemize}
				Từ $(1)$ và $(2)$ suy ra $IH=\left(MNI\right)\cap\left(BCD\right)$.}{
				\begin{tikzpicture}[scale=0.8, font=\footnotesize, line join=round, line cap=round, >=stealth]
					\tkzDefPoints{0/0/B, 2/-2/C, 6/0/D, 2/4/A, 3/-1/I}
					\tkzDefBarycentricPoint(A=1,B=2)
					\tkzGetPoint{M}
					\tkzDefBarycentricPoint(A=3,C=2)
					\tkzGetPoint{N}
					\tkzInterLL(M,N)(C,B)\tkzGetPoint{H}
					\tkzLabelPoints[left](H)
					\tkzLabelPoints[below left](B,I)
					\tkzLabelPoints[above left](M)
					\tkzLabelPoints[above](A)
					\tkzLabelPoints[below](C)
					\tkzLabelPoints[right](D,N)
					\tkzDrawPoints[fill=black](A,B,C,D,M,N,H,I)
					\tkzDrawSegments[dashed](B,D H,I N,I M,I)
					\tkzDrawSegments(A,C A,B A,D C,D C,H N,H)
			\end{tikzpicture}}
			\immini{\item [c)] Tìm $(MNI) \cap (ABD)$.
				\begin{itemize}
					\item [$\bullet$] Gọi $E=IH\cap BD$. Ta có
					$$\heva{&E\in BD\\&E\in IH}\Rightarrow E\in \left(MNI\right)\cap\left(ABD\right)\quad (3).$$
					\item [$\bullet$] Dễ thấy $M\in \left(ABD\right)\cap\left(MNI\right)\quad (4)$.
				\end{itemize}.\\
				Từ $(3)$ và $(4)$ suy ra $ME=\left(ABD\right)\cap\left(MNI\right)$.
			}{
				\begin{tikzpicture}[scale=0.8, font=\footnotesize, line join=round, line cap=round, >=stealth]
					\tkzDefPoints{0/0/B, 2/-2/C, 6/0/D, 2/4/A, 3/-1/I}
					\tkzDefBarycentricPoint(A=1,B=2)
					\tkzGetPoint{M}
					\tkzDefBarycentricPoint(A=3,C=2)
					\tkzGetPoint{N}
					\tkzInterLL(M,N)(C,B)\tkzGetPoint{H}
					\tkzInterLL(I,H)(B,D)\tkzGetPoint{E}
					\tkzInterLL(I,H)(C,D)\tkzGetPoint{F}
					\tkzLabelPoints[left](H)
					\tkzLabelPoints[below left](B,I)
					\tkzLabelPoints[above left](M)
					\tkzLabelPoints[above](A)
					\tkzLabelPoints[below](C,E)
					\tkzLabelPoints[right](D,N)
					\tkzLabelPoints[below right](F)
					\tkzDrawPoints[fill=black](A,B,C,D,M,N,H,I,E,F)
					\tkzDrawSegments[dashed](B,D H,F M,E M,I N,I)
					\tkzDrawSegments(A,C A,B A,D C,D C,H N,H N,F)
			\end{tikzpicture}}
			\item [d)] Tìm $(MNI) \cap (BCD)$.
			\begin{itemize}
				\item [$\bullet$] Gọi $F=IH\cap CD$. Ta có
				$$\heva{&F\in CD\\&F\in IH}\Rightarrow F\in \left(MNI\right)\cap\left(ACD\right)\quad (5).$$
				\item [$\bullet$] Mặt khác: $N\in AC$ nên $N\in \left(ACD\right)$. Suy ra
				$N\in \left(MNI\right)\cap\left(ACD\right)\quad (6)$.\\
				Từ $(5)$ và $(6)$ suy ra $NF=\left(ACD\right)\cap\left(MNI\right)$.
			\end{itemize}
		\end{itemize}
	}
\end{vd}

\begin{vd}
	Cho tứ diện $ABCD$. Gọi $I$, $J$ lần lượt là trung điểm các cạnh $AD$, $BC$.  
	\begin{itemize}
		\item [a)] Tìm giao tuyến của hai mặt phẳng $\left(IBC\right)$ và mặt phẳng $\left(JAD\right)$. 
		\item [b)] Lấy điểm $M$ thuộc cạnh $AB$, $N$ thuộc cạnh $AC$ sao cho $M$, $N$ không là trung điểm. Tìm giao tuyến của  mặt phẳng $\left(IBC\right)$ và mặt phẳng $\left(DMN\right)$. 
	\end{itemize}
	\loigiai{
		\immini{
			\begin{itemize}
				\item [a)] Do giả thiết $I\in AD$ nên $I\in \left(JAD\right)$. 
				Suy ra 
				$$I\in \left(BCI\right)\cap\left(ADJ\right)\quad (1).$$
				Tương tự, ta có $J\in \left(BCI\right)\cap\left(ADJ\right)\quad (2)$.\\
				Từ $(1)$ và $(2)$ suy ra $IJ=\left(BCI\right)\cap \left(ADJ\right)$.
				\item [b)] Gọi $E=DM\cap BI$. Khi đó 
				$$\heva{&E\in BI\\&E\in DM}\Rightarrow E\in \left(MND\right)\cap\left(IBC\right)\quad (3).$$
				Tương tự, gọi $F=DN\cap CI$ suy ra
				$$F\in \left(BCI\right)\cap\left(MND\right)\quad (4).$$
				Từ $(3)$ và $(4)$ suy ra $EF=\left(BCI\right)\cap \left(MND\right)$.
			\end{itemize}
		}{\begin{tikzpicture}[scale=0.8, font=\footnotesize, line join=round, line cap=round, >=stealth]
				\tkzDefPoints{0/0/B, 2/-2/C, 6/0/D, 2/4/A}
				\tkzDefMidPoint(A,D)\tkzGetPoint{I}
				\tkzDefMidPoint(B,C)\tkzGetPoint{J}
				\tkzDefBarycentricPoint(A=1,B=3)
				\tkzGetPoint{M}
				\tkzDefBarycentricPoint(A=3,C=2)
				\tkzGetPoint{N}
				\tkzInterLL(D,N)(C,I)\tkzGetPoint{F}
				\tkzInterLL(D,M)(B,I)\tkzGetPoint{E}
				\tkzLabelPoints[below left](B,J)
				\tkzLabelPoints[above left](M)
				\tkzLabelPoints[above](A)
				\tkzLabelPoints[below](C,E)
				\tkzLabelPoints[right](D,N)
				\tkzLabelPoints[above right](I,F)
				\tkzDrawPoints[fill=black](A,B,C,D,I,J,M,N,E,F)
				\tkzDrawSegments[dashed](B,D B,I D,J M,D I,J E,F)
				\tkzDrawSegments(A,C A,B A,D C,D B,C C,I A,J M,N N,D)
			\end{tikzpicture}
		}
	}
\end{vd}

\begin{vd}
	Cho tứ diện $ABCD$, $M$ là một điểm bên trong tam giác $ABD$, $N$ là một điểm bên trong tam giác $ACD$. Tìm giao tuyến của các cặp mặt phẳng sau
	\begin{enumEX}[]{2}
		\item $(AMN)$ và $(BCD)$.
		\item $(DMN)$ và $(ABC)$.
	\end{enumEX}
	\loigiai{
		\immini
		{
			\begin{enumEX}[]{1}
				\item Tìm $(AMN)\cap (BCD)$.\\
				Trong $(ABD)$, gọi $E=AM\cap BD$.\\
				Ta có $\heva{& E\in AM\subset (AMN) \\ & E\in BD\subset (BCD)}\Rightarrow E\in (AMN)\cap (BCD)$ $(1)$.\\
				Trong $(ACD)$, gọi $F=AN\cap CD$.\\
				Ta có $\heva{& F\in AN\subset (AMN) \\ & F\in CD\subset (BCD)}\Rightarrow F\in (AMN)\cap (BCD)$ $(2)$.\\
				Từ $(1)$ và $(2)$ suy ra $(AMN)\cap (BCD)=EF$.
				\item Tìm $(DMN)\cap (ABC)$.\\
				Trong $(ABD)$, gọi $P=DM\cap AB$.\\
				Ta có $\heva{& P\in DM\subset (DMN) \\ & P\in AB\subset (ABC)}\Rightarrow P\in (DMN)\cap (ABC)$ $(3)$.\\
				Trong $(ACD)$, gọi $Q=DN\cap AC$.\\
				Ta có $\heva{& Q\in DN\subset (DMN) \\ & Q\in AC\subset (ABC)}\Rightarrow Q\in (DMN)\cap (ABC)$ $(4)$.\\
				Từ $(3)$ và $(4)$ suy ra $(DMN)\cap (ABC)=PQ$.
			\end{enumEX}
		}
		{
			\begin{tikzpicture}
				[scale=1, font=\footnotesize, line join=round, line cap=round, >=stealth]
				\tkzDefPoints{0/0/B,3/-2/C,5/0/D,2/4.5/A}
				\coordinate (E) at ($(B)!1/3!(D)$);
				\coordinate (F) at ($(C)!2/3!(D)$);
				\coordinate (P) at ($(A)!1.3/3!(B)$);
				\coordinate (Q) at ($(A)!3/5!(C)$);
				\tkzInterLL(A,E)(P,D)\tkzGetPoint{M}
				\tkzInterLL(A,F)(Q,D)\tkzGetPoint{N}
				\tkzDrawPolygon(A,B,C,D)
				\tkzDrawSegments(A,C A,F P,Q Q,D)
				\tkzDrawSegments[dashed](B,D A,E E,F P,D)
				\tkzDrawPoints[fill=black](A,B,C,D,Q,P,M,N,E,F)
				\tkzLabelPoints[above](A)
				\tkzLabelPoints[below](C,E)
				\tkzLabelPoints[left](B,P,Q)
				\tkzLabelPoints[right](D,F)
				\tkzLabelPoints[right](M)
				\tkzLabelPoints[right](N)
			\end{tikzpicture}
		}
	}
\end{vd}

\begin{vd}
	\immini{Cho hình chóp $S.ABCD$ đáy là hình bình hành tâm $O$. Gọi $M$, $N$, $P$ lần lượt là trung điểm của cạnh $BC$, $CD$, $SA$. Tìm giao tuyến của 
		\begin{tasks}(1)
			\task $(MNP)$ và $(SAB)$.
			\task $(MNP)$ và $(SBC)$.
			\task $(MNP)$ và $(SAD)$.
			\task $(MNP)$ và $(SCD)$.
	\end{tasks}}{
		\begin{tikzpicture}
			[scale=1, font=\footnotesize, line join=round, line cap=round, >=stealth]
			\tkzDefPoints{0/0/A,-1.3/-1.6/B,2.5/-1.6/C}
			\coordinate (D) at ($(A)+(C)-(B)$);
			\coordinate (S) at ($(A)+(0.5,2.5)$);
			\coordinate (P) at ($(S)!0.5!(A)$);
			\coordinate (M) at ($(B)!0.5!(C)$);
			\coordinate (N) at ($(C)!0.5!(D)$);
			\tkzDrawPoints[fill=black](D,C,A,B,S,M,N,P)
			\tkzDrawPolygon(S,B,C,D)
			\tkzDrawPolygon[dashed](M,N,P)
			\tkzDrawSegments[dashed](A,B A,D A,S)
			\tkzDrawSegments(S,C)
			\tkzLabelPoints[above](S)
			\tkzLabelPoints[below](A,B,C,D,M,N)
			\tkzLabelPoints[above right](P)
	\end{tikzpicture}}
	\loigiai{
		\begin{itemize}
			\immini{\item[a)] Tìm $(MNP)\cap (SAB)$.
				\begin{itemize}
					\item [$\bullet$] Ta có $P\in (MNP)\cap (SAB)$ $(1)$.
					\item [$\bullet$] Gọi $F=MN \cap AB $ thì $\heva{& F\in MN\subset (MNP) \\ & F\in AB\subset (SAB)}$\\
					nên $F\in (MNP)\cap (SAB) \quad (2)$.
				\end{itemize}
				Từ $(1)$ và $(2)$ suy ra $(MNP)\cap (SAB)=PF$.
				\item[b)] Tìm $(MNP)\cap (SBC)$.
				\begin{itemize}
					\item [$\bullet$] Ta có $M\in (MNP)\cap (SBC)$ $(3)$.
					\item [$\bullet$] Gọi $K=PF \cap SB $ thì $\heva{& K\in PF\subset (MNP) \\ & K\in SB\subset (SBC)}$\\
					nên $K\in (MNP)\cap (SBC) \quad (4)$.
				\end{itemize}
				Từ $(3)$ và $(4)$ suy ra $(MNP)\cap (SBC)=MK$.}{
				\begin{tikzpicture}
					[scale=1, font=\footnotesize, line join=round, line cap=round, >=stealth]
					\tkzDefPoints{0/0/A,-1.3/-1.6/B,2.5/-1.6/C}
					\coordinate (D) at ($(A)+(C)-(B)$);
					\coordinate (S) at ($(A)+(0.5,2.5)$);
					\coordinate (P) at ($(S)!0.5!(A)$);
					\coordinate (M) at ($(B)!0.5!(C)$);
					\coordinate (N) at ($(C)!0.5!(D)$);
					\tkzInterLL(A,B)(M,N)\tkzGetPoint{F}
					\tkzInterLL(S,B)(F,P)\tkzGetPoint{K}
					\tkzDrawPoints[fill=black](D,C,A,B,S,M,N,P,F,K)
					\tkzDrawPolygon[dashed](M,N,P)
					\tkzDrawSegments[dashed](A,B A,D A,S K,P F,B B,K B,M)
					\tkzDrawSegments(S,C M,F F,K M,K S,K S,D C,M C,D)
					\tkzLabelPoints[above](S)
					\tkzLabelPoints[left](K)
					\tkzLabelPoints[below](A,B,C,D,M,N,F)
					\tkzLabelPoints[above right](P)
			\end{tikzpicture}	}
			\immini{\item[c)] Tìm $(MNP)\cap (SAD)$.
				\begin{itemize}
					\item [$\bullet$] Ta có $P\in (MNP)\cap (SAD)$ $(5)$.
					\item [$\bullet$] Gọi $E=MN \cap AD $, suy ra 
					$$ E\in (MNP)\cap (SAD) \quad (6).$$
				\end{itemize}
				Từ $(5)$ và $(6)$ suy ra $(MNP)\cap (SAD)=EP$.
				\item[d)] Tìm $(MNP)\cap (SCD)$.
				\begin{itemize}
					\item [$\bullet$] Ta có $N\in (MNP)\cap (SCD)$ $(7)$.
					\item [$\bullet$] Gọi $H=PE \cap SD$, suy ra
					$$H\in (MNP)\cap (SCD) \quad (8).$$
				\end{itemize}
				Từ $(7)$ và $(8)$ suy ra $(MNP)\cap (SCD)=HN$.
			}{
				\begin{tikzpicture}
					[scale=1, font=\footnotesize, line join=round, line cap=round, >=stealth]
					\tkzDefPoints{0/0/A,-1.3/-1.6/B,2.5/-1.6/C}
					\coordinate (D) at ($(A)+(C)-(B)$);
					\coordinate (S) at ($(A)+(0.5,2.5)$);
					\coordinate (P) at ($(S)!0.5!(A)$);
					\coordinate (M) at ($(B)!0.5!(C)$);
					\coordinate (N) at ($(C)!0.5!(D)$);
					\tkzInterLL(A,D)(M,N)\tkzGetPoint{E}
					\tkzInterLL(S,D)(E,P)\tkzGetPoint{H}
					\tkzDrawPoints[fill=black](D,C,A,B,S,M,N,P,E,H)
					\tkzDrawPolygon[dashed](M,N,P)
					\tkzDrawSegments[dashed](A,B A,D A,S H,P E,D D,H D,N)
					\tkzDrawSegments(S,C N,E E,H N,H S,B S,C S,H C,M B,C C,N)
					\tkzLabelPoints[above](S)
					\tkzLabelPoints[above right](H)
					\tkzLabelPoints[below](A,B,C,D,M,N,E)
					\tkzLabelPoints[above right](P)
			\end{tikzpicture}	}
		\end{itemize}
		
	}
\end{vd}

\begin{vd}
	Cho hình chóp $S.ABCD$ có đáy $ABCD$ là hình bình hành. Gọi $M,P$ lần lượt là trung điểm của $SA, BC$. $N$ là điểm trên cạnh $SB$ sao cho $BN=\dfrac{1}{4}BS$. Xác định giao tuyến của mặt phẳng $(MNP)$ với các mặt phẳng
	\begin{tasks}(3)
		\task $(ABCD)$.
		\task $(SAD)$.
		\task $(SCD)$.
	\end{tasks}
	\loigiai{
		\immini{a) Gọi $I$ là giao điểm của $MN$ và $AB$, khi đó ta có $\heva{&I\in MN\\&I\in AB}\Rightarrow \heva{&I\in (MNP)\\&I\in (ABCD)}.$\hfill (1)\\
			Hiển nhiên $\heva{&P\in (MNP)\\&P\in (ABCD)}.$\hfill (2)\\
			Từ (1) và (2) suy ra $PI$ là giao tuyến của các mặt phẳng $(MNP)$ và $(ABCD)$.\\
			b) Gọi $K$ là giao điểm của $IP$ với $AD$, khi đó $\heva{&K\in IP\\&K\in AD}\Rightarrow \heva{&K\in (MNP)\\&K\in (SAD)}.$\hfill (3)\\
			Hiển nhiên $\heva{&M\in (MNP)\\&M\in (ABCD)}.$\hfill (4)\\
			Từ (3) và (4) suy ra $MK$ là giao tuyến của các mặt phẳng $(MNP)$ và $(ABCD)$.\\
			c) Gọi $Q$ là giao điểm của $IP$ và $CD$, $R$ là giao điểm của $MK$ và $SD$. Khi đó ta chứng minh được $QR$ là giao tuyến của các mặt phẳng $(MNP)$ và $(SCD)$.
		}
		{
			\begin{tikzpicture}[scale=1]
				\tkzDefPoints{0/0/A, -2/-2/B, 4/0/D}
				\coordinate (C) at ($(B)+(D)-(A)$);
				\coordinate (S) at ($(A)+(0,5)$);
				\coordinate (M) at ($(A)!0.5!(S)$);
				\coordinate (P) at ($(B)!0.5!(C)$);
				\coordinate (Q) at ($(C)!0.5!(D)$);
				\tkzInterLL(P,Q)(A,B) \tkzGetPoint{I}
				\tkzInterLL(P,Q)(A,D) \tkzGetPoint{K}
				\tkzInterLL(M,I)(S,B) \tkzGetPoint{N}
				\tkzInterLL(M,K)(S,D) \tkzGetPoint{R}
				\tkzDrawSegments(S,N N,I I,P P,N P,C C,Q C,S Q,R Q,K K,R R,S)
				\tkzDrawSegments[dashed](S,A A,I B,P B,N A,K Q,P R,D M,N M,R Q,D)
				\tkzLabelPoints[above](S,R)
				\tkzLabelPoints[below](B,I,P,Q,C,K,D,A)
				\tkzLabelPoints[left](N)
				\tkzLabelPoints[above right](M)
			\end{tikzpicture}
		}
	}
\end{vd}

\begin{dang}{Tìm giao điểm của đường thẳng và mặt phẳng}
	\begin{note}
		Muốn tìm giao điểm của đường thẳng $d$ và mặt phẳng $(P)$ (phân biệt, không song song), ta tìm giao điểm của $d$ với một đường thẳng $a$ nằm trong $(P)$. Xét hai khả năng: 
	\end{note}
	\immini{
		\begin{itemize}
			\item [\ding{172}] Nếu đường thẳng $a$ dễ tìm, nghĩa là có sẵn $a \subset (P)$ và $a$ cắt được $d$. Khi đó
			\begin{itemize}
				\item [$\bullet$] Gọi $M=d \cap a$ thì $\heva{& M \in d\\& M \in a \subset (P)}$.
				\item [$\bullet$] Vậy $M = d \cap (P)$.
			\end{itemize}
			\item [\ding{173}] Nếu đường thẳng $a$ khó tìm, ta thực hiện các bước sau:
			\begin{itemize}
				\item [$\bullet$] Tìm một mặt phẳng $(Q)$ chứa đường thẳng $d$ và dễ tìm giao tuyến với $(P)$;
				\item [$\bullet$] Tìm $(Q) \cap (P)= a$.
				\item [$\bullet$] Tìm $M=d \cap a$, suy ra $M = d \cap (P)$.
			\end{itemize}
		\end{itemize}
	}{\vspace{1cm}
		\begin{tikzpicture}[line cap=round, line join=round,font=\footnotesize,>=stealth, scale=0.9]
			\tikzset{label style/.style={font=\footnotesize}}
			\tkzDefPoints{0/0/a, 4.5/0/b, 4/2/c, 4.2/2.3/b'}
			\tkzDefBarycentricPoint(a=1,c=1,b=-1)\tkzGetPoint{d}
			\tkzDefBarycentricPoint(a=0.75,b=0.25)\tkzGetPoint{a'}
			\tkzDefBarycentricPoint(d=0.75,c=0.25)\tkzGetPoint{d'}
			\tkzDefBarycentricPoint(d'=1,b'=1,a'=-1)\tkzGetPoint{c'}
			\tkzInterLL(a',b')(c,d)\tkzGetPoint{x}
			\tkzDefMidPoint(a',d')\tkzGetPoint{M}
			\tkzDefLine[parallel=through M](a',b')\tkzGetPoint{y}
			\tkzDrawPolygon(a',b',c',d')
			\tkzDrawSegments(a,b b,c d,a c,x d,d')
			\tkzDrawSegments[dashed](x,d')
			\tkzDrawSegments[add=0 and -0.3](M,y)
			\tkzLabelSegments[above, pos=0.7](M,y){$d$}
			\tkzDrawSegments[add=0.2 and -1, dashed](M,y)
			\tkzMarkAngle[size=0.5](c',b',a')
			\tkzLabelAngle[pos=-0.25](a',b',c'){$Q$}
			\tkzMarkAngle[size=0.5](b,a,d)
			\tkzLabelAngle[pos=0.25](b,a,d){$P$}
			\tkzLabelSegments[pos=0.8, right](a',d'){$a$}
			
			\tkzDrawPoints[fill=black](M)
			\tkzLabelPoints[below right](M)
		\end{tikzpicture}
	}
\end{dang}

\begin{vd}
	Cho tứ diện $ABCD$. Gọi $M,N$ lần lượt là trung điểm của $AC$ và $BC$. $K$ là điểm nằm trên $BD$ sao cho $KD<KB$. 
	\begin{tasks}
		\task Tìm giao điểm của $CD$ với mặt phẳng $(MNK)$.
		\task Tìm giao điểm của $AD$ với mặt phẳng $(MNK)$.
	\end{tasks}
	\loigiai{
		\begin{enumerate}[\faPencilSquareO]
			\immini{	\item Tìm giao điểm của $CD$ với mp$(MNK)$.
				\begin{note}
					Dễ thấy trong mặt phẳng $(MNK)$ có đường thẳng $NK$ có thể cắt được đường $CD$. Nên ta giải như sau:
				\end{note}
				\begin{itemize}
					\item [$\bullet$] Do $KD<KC$ nên $K$ không là trung điểm của $BD$, suy ra $NK$ cắt $CD$; 
					\item [$\bullet$] Gọi $I=CD \cap NK$, ta có
					$$\heva{&I \in CD\\& I \in NK, NK \subset (MNK)}\Rightarrow I=CD \cap (MNK).$$
				\end{itemize}
			}{
				\begin{tikzpicture}[line cap=round, line join=round,font=\footnotesize,>=stealth, scale=1.1]
					\tikzset{label style/.style={font=\footnotesize}}
					\tkzDefPoints{0/0/B, 4/0/D, 3/-1.5/C, 1/2.5/A}
					\tkzDefBarycentricPoint(B=0.2,D=0.8)\tkzGetPoint{K}
					\tkzDefMidPoint(A,C)\tkzGetPoint{M}
					\tkzDefMidPoint(B,C)\tkzGetPoint{N}
					\tkzInterLL(N,K)(C,D)\tkzGetPoint{I}
					\tkzInterLL(I,M)(A,D)\tkzGetPoint{H}
					\tkzDrawSegments[dashed](B,D M,K N,I)
					\tkzDrawSegments(A,B B,C C,I I,M A,D A,C M,N)
					
					\tkzDrawPoints[fill=black](A,B,C,D,M,N,K,H,I)
					\tkzLabelPoints[above](A)
					\tkzLabelPoints[left](B,M)
					\tkzLabelPoints[below](C,K)
					\tkzLabelPoints[below right](D)
					\tkzLabelPoints[above right](I,H)
					\tkzLabelPoints[below left](N)
			\end{tikzpicture}}
			\item Tìm giao điểm của $AD$ và $(MNK)$.\\
			Trong mặt phẳng $(ACD)$, gọi $H=AD \cap MI$. Ta có
			$$\heva{&H \in AD\\& H \in MI, MI \subset (MNK)}\Rightarrow H=AD \cap (MNK).$$		
	\end{enumerate}}
\end{vd}

\begin{vd}
	Cho tứ diện $ABCD$. trên cạnh $AC$ và $AD$ lấy hai điểm $M$, $N$ sao cho $AC=3AM$ và $AN=\dfrac{2}{3}AD$. Gọi $O$ là điểm bên trong tam giác $(BCD)$.
	\begin{tasks}(1)
		\task Tìm giao điểm của $BC$ với $(OMN)$.
		\task Tìm giao điểm của $BD$ với $(OMN)$.
	\end{tasks}
	\loigiai{
		
		\begin{enumerate}[a)]
			\immini{
				\item Tìm giao điểm của $BC$ với $(OMN)$.\\
				Xét $BC \subset (BCD)$. Ta đi tìm giao tuyến của $(BCD)$ với $(OMN)$.
				\begin{itemize}
					\item [$\bullet$] Gọi $I= CD \cap MN \Rightarrow I \in (BCD) \cap (OMN) \quad (1)$.\\
					\item [$\bullet$] Mặt khác $O \in (BCD) \cap (OMN) \quad (2)$.
				\end{itemize}
				Từ (1) và (2), suy ra $OI=(BCD) \cap (OMN)$.\\
				Trong $(BCD)$, gọi $P=OI \cap BC$. Ta có 
				$$\heva{&P \in BC\\&P \in OI, OI \subset (OMN)}\Rightarrow P=BC \cap (OMN).$$
			}{\vspace{0.6cm}
				\begin{tikzpicture}[line cap=round, line join=round,font=\footnotesize,>=stealth, scale=1]
					\tikzset{label style/.style={font=\footnotesize}}
					\tkzDefPoints{0/0/B, 4/0/D, 1.6/-1.2/C, 1/3/A}
					\tkzDefBarycentricPoint(A=0.35,D=0.65)\tkzGetPoint{N}
					\tkzDefBarycentricPoint(A=0.6,C=0.4)\tkzGetPoint{M}
					\tkzDefBarycentricPoint(B=0.35,C=0.3,D=0.35)\tkzGetPoint{O}
					\tkzInterLL(M,N)(C,D)\tkzGetPoint{I}
					\tkzInterLL(B,C)(O,I)\tkzGetPoint{P}
					\tkzInterLL(D,B)(O,I)\tkzGetPoint{Q}
					\tkzDrawSegments[dashed](B,D I,O O,M O,N O,P)
					\tkzDrawSegments(A,B B,C C,D A,C M,I A,D I,D P,M)
					\tkzLabelPoints[above](A,Q)
					\tkzLabelPoints[left](B,M)
					\tkzLabelPoints[below](C,O)
					\tkzLabelPoints[below right](D)
					\tkzLabelPoints[above right](N,I)
					\tkzLabelPoints[below left](P)
					\tkzDrawPoints[fill=black](A,B,C,D,M,N,P,Q,I,O)
				\end{tikzpicture}
			}
			\item Tìm giao điểm của $BD$ với $(OMN)$.\\
			Trong $(BCD)$, gọi $Q=OI \cap BD$. Ta có 
			$\heva{&Q \in BD\\&Q \in OI, OI \subset (OMN)}\Rightarrow Q=BD \cap (OMN).$
		\end{enumerate}	
	}
\end{vd}

\begin{vd}
	Cho hình chóp $S.ABCD$ có đáy là hình bình hành. Gọi $M$ là trung điểm của $SC$.
	\begin{itemize}
		\item [a)] Tìm giao điểm $I$ của đường thẳng $AM$ và mặt phẳng $\left(SBD\right)$. Chứng minh $IA=2IM$.
		\item [b)] Tìm giao điểm $E$ của đường thẳng $SD$ và mặt phẳng $\left(ABM\right)$.
		\item [c)] Gọi $N$ là một điểm tuỳ ý trên cạnh $AB$. Tìm giao điểm của đường thẳng $MN$ và mặt phẳng $\left(SBD\right)$.
	\end{itemize}
\end{vd}


\begin{vd}
	Cho tứ giác $ABCD$ và một điểm $S$ không thuộc mặt phẳng $(ABCD)$. Trên đoạn $AB$ lấy một điểm $M$, trên đoạn $SC$ lấy một điểm $N$ ($M,N$ không trùng với các đầu mút).
	\begin{tasks}(1)
		\task Tìm giao điểm của đường thẳng $AN$ với mặt phẳng $(SBD)$.
		\task Tìm giao điểm của đường thẳng $MN$ với mặt phẳng $(SBD)$.
	\end{tasks}
	\loigiai{
		\immini{\begin{enumerate}[a)]
				\item \textbf{Tìm giao điểm của đường thẳng $AN$ với mặt phẳng $(SBD)$.}
				\begin{itemize}
					\item Chọn mặt phẳng phụ $(SAC)\supset AN$. Ta tìm giao tuyến của $(SAC)$ và $(SBD)$.\\
					Trong $(ABCD)$ gọi $P=AC\cap BD$. Suy ra $(SAC)\cap(SBD)=SP$.
					\item Trong $(SAC)$ gọi $I=AN\cap SP$.\\
					$\heva{&I\in AN \\&I\in SP, SP\subset (SBD)}\Rightarrow I=AN\cap (SBD)$.
				\end{itemize}
				\item \textbf{Tìm giao điểm của đường thẳng $MN$ với mặt phẳng $(SBD)$.}
				\begin{itemize}
					\item Chọn mặt phẳng phụ $(SMC)\supset MN$. Ta tìm giao tuyến của $(SMC)$ và $(SBD)$.\\
					Trong $(ABCD)$ gọi $Q=MC\cap BD$. Suy ra $(SMC)\cap(SBD)=SQ$.
					\item Trong $(SMC)$ gọi $J=MN\cap SQ$.\\
					$\heva{&J\in MN \\&J\in SQ, SQ\subset (SBD)}\Rightarrow J=MN\cap (SBD)$.
				\end{itemize}
			\end{enumerate}
		}{
			\begin{tikzpicture}[line cap=round, line join=round,font=\footnotesize,>=stealth, scale=1]
				\tikzset{label style/.style={font=\footnotesize}}
				\tkzDefPoints{0/0/A, 4.5/0/D, 1.2/3/S, 0.9/-1.5/B, 3.5/-1.3/C}
				\tkzDefBarycentricPoint(A=0.3,B=0.7)\tkzGetPoint{M}
				\tkzDefBarycentricPoint(S=0.6,C=0.4)\tkzGetPoint{N}
				\tkzInterLL(A,C)(B,D)\tkzGetPoint{P}
				\tkzInterLL(S,P)(A,N)\tkzGetPoint{I}
				\tkzInterLL(C,M)(B,D)\tkzGetPoint{Q}
				\tkzInterLL(S,Q)(M,N)\tkzGetPoint{J}
				
				\tkzDrawSegments[dashed](S,P S,Q A,D A,C A,N C,M M,N B,D)
				\tkzDrawSegments(S,A S,B S,C S,D S,M A,B B,C C,D)
				
				\tkzLabelPoints[above](S)
				\tkzLabelPoints[left](A,J)
				\tkzLabelPoints[below left](B,M)
				\tkzLabelPoints[below right](C)
				\tkzLabelPoints[right](D)
				\tkzLabelPoints[above right](P,N)
				\tkzLabelPoints[above left](Q)
				\draw (I)+(-0.2,0.15) node{$I$};
				\tkzDrawPoints[fill=black](S,A,B,C,D,M,N,P,Q,I,J)
			\end{tikzpicture}
		} 
	}
\end{vd}

\begin{dang}{Chứng minh ba điểm thẳng hàng}
\begin{note}
	Muốn chứng minh ba điểm $A$, $B$, $C$ thẳng hàng, ta chứng minh ba điểm đó lần lượt thuộc hai mặt phẳng phân biệt $(\alpha)$ và $(\beta)$, nghĩa là chúng cùng nằm trên một đường giao tuyến.
\end{note}
	% \immini{	\begin{itemize}
	% 		\item [$\bullet$]  Ta có $A=a \cap b$, mà $a \subset (\alpha)$, $b \subset (\beta)$ nên
	% 		$$A \in (\alpha) \cap (\beta) \quad (1)$$
	% 		\item [$\bullet$] Tương tự ta cũng tìm xem $B$ và $C$ tương ứng là giao của cặp đường thẳng nào nằm trong $(\alpha)$ và $(\beta)$. Từ đó, suy ra
	% 		$$B \in (\alpha) \cap (\beta) \quad (2)$$
	% 		và 
	% 		$$C \in (\alpha) \cap (\beta) \quad(3)$$
	% 	\end{itemize}
	% }	
	% {\begin{tikzpicture}[scale=0.4, font=\footnotesize,line join=round, line cap=round,>=stealth]
	% 		\tikzset{label style/.style={font=\footnotesize}}
	% 		\tkzDefPoints{0/0/M,7/0/N,10/-2/P,3/-2/Q,6/2/X,3/5/Y, 2.5/3/E, 5.5/-1/F}
	% 		\tkzInterLL(M,N)(Q,X)\tkzGetPoint{T}
	% 		\coordinate (A) at ($(M)!0.15!(Q)$);
	% 		\coordinate (B) at ($(M)!0.45!(Q)$);
	% 		\coordinate (C) at ($(M)!0.84!(Q)$);
	% 		\tkzInterLL(A,F)(X,Q)\tkzGetPoint{I}
	% 		\tkzDrawPoints[size=5,fill=black](A,B,C)
	% 		\tkzLabelPoints[below](A,B,C)
	% 		\tkzLabelSegment(I,F){$b$}
	% 		\tkzLabelSegment(A,E){$a$}
	% 		\tkzLabelAngles[pos=0.6,rotate=30](M,Y,X){$\alpha$}
	% 		\tkzLabelAngles[pos=0.9,rotate=10](N,P,Q){$\beta$}
	% 		\tkzDrawPolygon(M,Q,X,Y)
	% 		\tkzDrawSegments(M,Q Q,P N,P T,N A,E I,F)
	% 		\tkzDrawSegments[dashed](M,T A,I)
	% 		\tkzMarkAngles[size=1cm,arc=l](M,Y,X)
	% 		\tkzMarkAngles[size=1.35cm,arc=l](N,P,Q)
	% 	\end{tikzpicture}\hspace{-2cm}}
	% 	Từ (1), (2), (3) suy ra $A$, $B$, $C$ cùng thuộc đường giao tuyến nên chúng thẳng hàng.
	
\end{dang}
\begin{vd}
	Cho tứ diện $ABCD$ có $G$ là trọng tâm tam giác $BCD$, Gọi $M$, $N$, $P$ lần lượt là trung điểm của $AB$, $BC$, $CD$.
	\begin{tasks}(1)
		\task Tìm giao tuyến của $(AND)$ và $(ABP)$.
		\task Gọi $I=AG\cap MP$, $J=CM\cap AN$. Chứng minh $D$, $I$, $J$ thẳng hàng.
	\end{tasks}
	\loigiai{
		\begin{center}
			\begin{tikzpicture}[scale=1, font=\footnotesize,line join=round, line cap=round,>=stealth]
				\tkzDefPoints{0/0/B,2/-2/C,6/0/D,2/4/A}
				\tkzDefMidPoint(A,B)\tkzGetPoint{M}
				\tkzDefMidPoint(B,C)\tkzGetPoint{N}
				\tkzDefMidPoint(C,D)\tkzGetPoint{P}
				\tkzInterLL(B,P)(D,N)\tkzGetPoint{G}
				\tkzInterLL(A,G)(M,P)\tkzGetPoint{I}
				\tkzInterLL(C,M)(A,N)\tkzGetPoint{J}
				\tkzDrawSegments(A,B B,C C,D D,A A,C A,N C,M A,P)
				\tkzDrawSegments[dashed](B,D B,P D,N M,P D,J A,G D,M)
				\tkzDrawPoints[fill=black](A,B,C,D,M,N,P,I,J,G)
				\tkzLabelPoints[left](M,J)
				\tkzLabelPoints[above right](I)
				\tkzLabelPoints[above](A)
				\tkzLabelPoints[below](D,C,B,N,P,G)
			\end{tikzpicture}
		\end{center}
		\begin{enumerate}[a)]
			\item Tìm giao tuyến của $(AND)$ và $(ABP)$.\\
			$A\in (ABP)\cap (ADN)$.\hfill $(1)$\\
			Ta có $G=BP\cap DN$, có $\heva{&G\in BP,\, BP\subset (ABP)\\&G\in DN,\, DN\subset (ADN)}\Rightarrow G\in (ABP)\cap (ADN)$. \hfill $(2)$\\
			Từ $(1)$ và $(2)$ ta có $AG=(ABP)\cap (ADN)$.
			\item Chứng minh $D$, $I$, $J$ thẳng hàng.\\
			$I=AG\cap MP$, $AG\subset (ADG)$, $MP\subset (DMN)\Rightarrow I\in(ADG)\cap (DMN)$.\hfill $(3)$\\
			$J=CM\cap AN$, $AN\subset (ADG)$, $CM\subset (DMN)\Rightarrow J\in(ADG)\cap (DMN)$.\hfill $(4)$\\
			$D\in (ADG)\cap (DMN)$. \hfill $(5)$\\
			Từ $(3)$, $(4)$, $(5)$ suy ra ba điểm $D$, $I$, $J$ thuộc giao tuyến của hai mặt phẳng $(ADG)$ và $(DMN)$.\\
			Vậy ba điểm $D$, $I$, $J$ thẳng hàng.
		\end{enumerate}	
	}
\end{vd}
\begin{vd}
	Cho hình chóp $S.ABCD$ có đáy là hình bình hành. Gọi $O$ là giao điểm của $AC$ và $BD$ ; $M, N$ lần lượt là trung điểm của $SB, SD$; $P$ thuộc đoạn $SC$ và không là trung điểm của $SC$.
	\begin{itemize}
		\item [a)] Tìm giao điểm $E$ của đường thẳng $SO$ và mặt phẳng $\left(MNP\right)$.
		\item [b)] Tìm giao điểm $Q$ của đường thẳng $SA$ và mặt phẳng $\left(MNP\right)$.
		\item [c)] Gọi $I, J, K$ lần lượt là giao điểm của $QM$ và $AB$, $QP$ và $AC$, $QN$ và $AD$. Chứng minh rằng $I, J, K$ thẳng hàng.
	\end{itemize}
\end{vd}

\begin{dang}{Vận dụng thực tiễn}
\end{dang}
\begin{vd}
	Giải thích tại sao ghế bốn chân có thể bị khập khiễng còn ghế ba chân thì không. 
	\loigiai{
	Dựa vào tính chất được thừa nhận của hình học không gian, có một và chỉ một mặt phẳng đi qua ba điểm không thẳng hàng cho trước. 
	Do đó qua bốn điểm có thể không cùng nằm trên một phẳng phẳng.
	
}
\end{vd}
\begin{vd}
	Giải thích tại sao chân máy ảnh có thể đặt ở hầu hết các loại hình mà vẫn đứng vững.
	\loigiai{
	Dựa vào tính chất được thừa nhận của hình học không gian, có một và chỉ một mặt phẳng đi qua ba điểm không thẳng hàng cho trước. 
	Do đó giá đỡ ba chân của máy ảnh khi đặt trên mặt đất không bị cập kênh. 
	
}
\end{vd}
\begin{vd}
	Hãy giải thích tại sao phần giao nhau giữa 2 vách tường nhà luôn là 1 đường thẳng 
	\loigiai{
Do mỗi vách tường nhà là 1 phần của mặt phẳng nên phần giao nhau là giao tuyến của của 2 mặt phẳng tức là 1 đường thẳng.

}
\end{vd}
\begin{vd}
	Hãy giải thích vì sao khi gấp đôi một tờ giấy thì nếp gấp luôn là 1 đường thẳng 
	\loigiai{
	Do khi gấp đôi tờ giấy thì mỗi phần của tờ giấy trở thành một phần của mặt phẳng khác nhau và nếp gấp là phần chung tức là giao tuyến của 2 mặt phẳng đó.
}
\end{vd}

\subsection{BÀI TẬP TỰ LUYỆN}


\begin{bt}
	Cho tứ diện $ABCD$. Trên $AB$, $AC$ lấy $2$ điểm $M$, $N$ sao cho $MN$ không song song $BC$. Gọi $O$ là một điểm trong tam giác $BCD$.
	\begin{tasks}(1)
		\task Tìm giao tuyến của $(OMN)$ và $(BCD)$.
		\task Tìm giao điểm của $DC$, $BD$ với $(OMN)$.
	%	\task Tìm thiết diện của $(OMN)$ với hình chóp.
	\end{tasks}
	\loigiai{
		\begin{center}
			\begin{tikzpicture}[scale=1, font=\footnotesize, line join=round, line cap=round, >=stealth]
				\tkzDefPoints{0/0/B,1.3/-1.6/C,4.5/0/D,1/3.5/A,2/-0.5/O}
				\coordinate (M) at ($(A)!2/3!(B)$);
				\coordinate (N) at ($(A)!0.4!(C)$);
				\tkzInterLL(M,N)(B,C)\tkzGetPoint{H}
				\tkzInterLL(B,D)(H,O)\tkzGetPoint{I}
				\tkzInterLL(C,D)(H,O)\tkzGetPoint{J}
				\tkzDrawSegments(C,D A,B A,D A,C H,N H,C N,J)
				\tkzDrawSegments[dashed](B,D H,J N,O M,I M,O)
				\tkzDrawPoints[fill=black](A,B,C,D,M,N,H,O,I,J)
				\tkzLabelPoints[above](A)
				\tkzLabelPoints[below](C,O,I,J)
				\tkzLabelPoints[left](B,M,H)
				\tkzLabelPoints[right](D,N)
			\end{tikzpicture}
		\end{center}
		\begin{enumerate}[a)]
			\item Tìm $(OMN)\cap (BCD)$.
			Trong $(ABC)$, gọi $H=MN\cap BC$.\\
			Ta có $\heva{& H\in MN\subset (MNO) \\ & H\in BC\subset (BCD)}\Rightarrow H\in (BCD)\cap (MNO)$ $(1)$.\\
			Mặt khác $O\in (BCD)\cap (MNO)$ $(2)$.\\
			Từ $(1)$ và $(2)$ suy ra $(BCD)\cap (MNO)=HO$.
			\item Tìm $DC\cap (OMN)$ và $BD\cap (OMN)$.\\
			Trong $(BCD)$, gọi $I=BD\cap HO$.\\
			Ta có $\heva{& I\in BD \\ & I\in HO\subset (MNO)}\Rightarrow I=BD\cap (MNO)$.\\
			Trong $(BCD)$, gọi $J=CD\cap HO$.\\
			Ta có $\heva{& J\in CD \\ & J\in HO\subset (MNO)}\Rightarrow J=CD\cap (MNO)$.
		%	\item Tìm thiết diện của $(OMN)$ và hình chóp.\\
		%	Ta có $\heva{& (ABC)\cap (MNO)=MN \\ & (ABD)\cap (MNO)=MI\\&(ACD)\cap (MNO)=NJ\\&(BCD)\cap (MNO)=IJ}.$ Vậy thiết diện cần tìm là tứ giác $MNJI$.
		\end{enumerate}
	}
\end{bt}

\begin{bt}
	Cho hình chóp $S.ABCD$. Gọi $O$ là giao điểm của $AC$ và $BD$. $M$, $N$, $P$ lần lượt là các điểm trên $SA$, $SB$, $SD$.
	\begin{tasks}(1)
		\task Tìm giao điểm $I$ của $SO$ với mặt phẳng $(MNP)$.
		\task Tìm giao điểm $Q$ của $SC$ với mặt phẳng $(MNP)$.
	\end{tasks}
	\loigiai{
		\begin{enumerate}[a)]	
			\item Tìm giao điểm $I$ của $SO$ với mặt phẳng $(MNP)$.
			\immini{Trong mặt phẳng $(SBD)$, gọi $I=SO\cap NP$, có
				$$\heva{&I\in SO\\&I\in NP\subset (MNP)}\Rightarrow I=SO\cap (MNP).$$
				\item Tìm giao điểm $Q$ của $SC$ với mặt phẳng $(MNP)$.\\
				$\bullet$ Chọn mặt phẳng phụ $(SAC)\supset SC$.\\
				$\bullet$ Tìm giao tuyến của $(SAC)$ và $(MNP)$.\\
				Ta có $\heva{&M\in(MNP)\\&M\in SA,\, SA\subset (SAC)}\Rightarrow M\in(MNP)\cap (SAC)$. \hfill $(1)$\\
				Và $\heva{&I\in SP,\, SP\subset(MNP)\\&I\in SO,\, SO\subset (SAC)}\Rightarrow I\in(MNP)\cap (SAC)$. \hfill $(2)$\\
			}
			{\begin{tikzpicture}[scale=1, font=\footnotesize,line join=round, line cap=round,>=stealth]
					\tkzDefPoints{-1/-2/B,-2/0/A,3/0/D,2/-1.5/C,0/4/S}
					\coordinate (M) at ($(A)!0.3!(S)$);
					\coordinate (N) at ($(B)!0.4!(S)$);
					\coordinate (P) at ($(D)!0.6!(S)$);
					\tkzInterLL(A,C)(B,D)\tkzGetPoint{O}
					\tkzInterLL(S,O)(N,P)\tkzGetPoint{I}
					\tkzInterLL(M,I)(S,C)\tkzGetPoint{Q}
					\tkzDrawSegments(A,B B,C C,D S,A S,B S,C S,D M,N N,Q Q,P)
					\tkzDrawSegments[dashed](A,D A,C B,D M,Q N,P M,P S,O)
					\tkzDrawPoints[fill=black](A,B,C,D,S,O,M,N,P,I,Q)
					\tkzLabelPoints[left](B,A,M,N)
					\tkzLabelPoints[right](C,D,P,Q)
					\tkzLabelPoints[above](S)
					\tkzLabelPoints[above left](I)
					\tkzLabelPoints[below](O)
			\end{tikzpicture}}
			Từ $(1)$ và $(2)$ có $(MNP)\cap (SAC)=MI$.\\
			$\bullet$ Trong mặt phẳng $(SAC)$ gọi $Q=SC\cap MI$, có $\heva{&Q\in SC\\&Q\in MI,\, MI\subset(MNP)}\Rightarrow Q=SC\cap (MNP)$.
			
		\end{enumerate}
	}
\end{bt}

\begin{bt}%[Dự án HHKG 11, 2018, TranTony]%[1H2B1-4]
	Cho hình chóp $S.ABCD$ có đáy $ABCD$ là hình thang với $AB$ song song với $CD$. O là giao điểm của hai đường chéo, $M$ thuộc $SB$.
	\begin{tasks}(1)
		\task Xác định giao tuyến của các cặp mặt phẳng: $(SAC)$ và $(SBD)$; $(SAD)$ và $(SBC)$.
		\task Tìm giao điểm $SO\cap (MCD)$; $SA\cap (MCD)$.
	\end{tasks}
	\loigiai{
		\centerline{\begin{tikzpicture}[scale=1,font=
				\footnotesize,line join=round,line cap=round, >=stealth]
				\tkzDefPoints{0/0/A,7/0/B,2/4/S,2/-3/H}
				\tkzDefPointBy[homothety=center A ratio 0.6](H)\tkzGetPoint{D}
				\tkzDefPointBy[homothety=center B ratio 0.6](H)\tkzGetPoint{C}
				\tkzDefPointBy[homothety=center S ratio 0.4](B)\tkzGetPoint{M}
				\tkzInterLL(A,C)(B,D)    \tkzGetPoint{O}
				\tkzInterLL(S,O)(D,M)    \tkzGetPoint{I}
				\tkzInterLL(S,A)(C,I)    \tkzGetPoint{J}
				\tkzDrawSegments[dashed](A,B A,C B,D S,O D,C D,M C,J)
				\tkzDrawSegments(S,A S,B S,D S,H S,C A,H B,H C,M)
				\tkzDrawPoints(A,B,C,S,O,H,D,M,I,J)
				\tkzLabelPoints[left](A,H,D,S,J)
				\tkzLabelPoints[below right](B,C)
				\tkzLabelPoints[right](M,I)
				\tkzLabelPoints[below](O)
		\end{tikzpicture}}\\
		\begin{enumerate}[a)]
			\item Xác định giao tuyến của $(SAC)$ và $(SBD)$.\\
			Ta có $S$ là điểm chung thứ nhất và $O$ là điểm chung thứ hai của hai mặt phẳng $(SAC)$ và $(SBD)$.\\
			Vậy $(SAC)\cap (SBD)=SO$.\\
			Xác định giao tuyến của $(SAD)$ và $(SBC)$.\\
			Ta có $S\in (SAD)\cap (SBC)$. \hfill (1)\\
			Trong mặt phẳng $(ABCD)$ gọi $H=AD\cap BC$, có $\heva{&H\in AD,AD\subset (SAD)\\& H\in BC, BC\subset (SBC)}\Rightarrow H\in (SAD)\cap (SBC)$. \hfill (2)\\
			Từ $(1)$ và $(2)$ suy ra $(SAD)\cap (SBC)=SH$.
			\item Tìm giao điểm $SO\cap (MCD)$; $SA\cap (MCD)$.\\
			Gọi $I=SO\cap DM$ (vì $SO, DM\subset (SBD)$).\\
			Ta có $\heva{&I\in SO\\&I\in DM ,DM\subset (MCD)}\Rightarrow I=SO\cap (MCD)$.\\
			Gọi $J=SA\cap CI$ (vì $SA, CI\subset (SAC)$).\\
			Ta có $\heva{&J\in SA\\& J\in CI, CI\subset (MCD)}\Rightarrow J=SA\cap (MCD)$.
		\end{enumerate}
	}
\end{bt}

\begin{bt}
	Cho hình chóp $S.ABCD$ có đáy $ABCD$ là hình bình hành tâm $O$. Gọi $M$, $N$ lần lượt là trung điểm của $AB$, $SC$.
	\begin{tasks}(2)
		\task Tìm $I=AN\cap (SBD)$.
		\task Tìm $K=MN\cap (SBD)$.
		\task Tính tỉ số $\dfrac{KM}{KN}$.
		\task Chứng minh $B, I, K$ thẳng hàng. Tính $\dfrac{IB}{IK}$.
	\end{tasks}
	\loigiai{
		\begin{center}
			\begin{tikzpicture}[scale=0.8,font=\footnotesize,line join=round,line cap=round, >=stealth]
				\tkzDefPoints{0/0/A,1/-2.7/B,8.6/-2.7/C, 4/4.8/S}
				\tkzDefPointBy[translation=from B to A](C)\tkzGetPoint{D}
				\tkzDefMidPoint(A,B)\tkzGetPoint{M}
				\tkzDefMidPoint(S,C)\tkzGetPoint{N}
				\tkzInterLL(A,C)(B,D)\tkzGetPoint{O}
				\tkzInterLL(S,O)(A,N)\tkzGetPoint{I}
				\tkzInterLL(M,N)(B,I)\tkzGetPoint{K}
				\tkzDrawSegments[dashed](A,D S,O A,C B,D M,N A,N B,I)
				\tkzDrawSegments(A,B B,C C,D S,A S,B S,C S,D B,N)
				\tkzDrawPoints(A,B,C,D,S,M,I,N,O,K)
				\tkzLabelPoints[left](A,B,M,S)
				\tkzLabelPoints[below](O)
				\tkzLabelPoints[above left](I)
				\tkzLabelPoints[below right](K,N)
				\tkzLabelPoints[right](C,D)
				
			\end{tikzpicture}
		\end{center}
		\begin{enumerate}[a)]
			\item Tìm $I=AN\cap (SBD)$.\\
			Trước hết ta tìm giao tuyến của mp$(SAC)$ và mp$(SBD)$. Ta có $S\in (SAC)\cap (SBD)$. \hfill (1)\\
			Có $\heva{&O\in AC, AC\subset (SAC)\\&O\in BD, BD\subset (SBD)}\Rightarrow O\in (SAC)\cap (SBD)$. \hfill (2)\\
			Từ $(1)$ và $(2)$ suy ra $SO=(SAC)\cap (SBD)$.\\
			Gọi $I=SO\cap AN$ (vì $SO, AN\subset (SAC)$). Suy ra $I=AN\cap (SBD)$.
			\item Tìm $K=MN\cap (SBD)$.\\
			Chọn mp$(ABN)$ chứa $MN$. Tìm giao tuyến của mp$(ABN)$ và mp$(SBD)$.\\ Có $\heva{&I\in SO, SO\subset (SBD)\\&I\in AN, AN\subset (ABN)}\Rightarrow I\in (ABN)\cap (SBD)$. \hfill (3)\\
			Có $B\in (ABN)\cap (SBD)$. \hfill (4)\\
			Từ $(3)$ và $(4)$ suy ra $BI=(ABN)\cap (SBD)$; $K=BI\cap MN$. Khi đó $K=MN\cap (SBD)$.
			\item Tính tỉ số $\dfrac{KM}{KN}$.
			\begin{center}
				\begin{tikzpicture}[scale=1,font=
					\footnotesize,line join=round,line cap=round, >=stealth]
					\tkzDefPoints{0/0/A,6/0/N,4/-2/B,2/0/Q}
					\tkzDefMidPoint(B,A)\tkzGetPoint{M}
					\tkzDefMidPoint(Q,N)\tkzGetPoint{I}
					\tkzInterLL(M,N)(B,I)\tkzGetPoint{K}
					\tkzDrawSegments(A,B B,N N,A Q,M M,N B,I)
					\tkzDrawPoints(A,N,I,K,Q)
					\tkzLabelPoints[left](A,M)
					\tkzLabelPoints[below right](K,B)
					\tkzLabelPoints[above](Q,I,N)
					\tkzMarkSegments[mark=||](I,N I,Q Q,A)
					\tkzMarkSegments[mark=|](M,A M,B)
				\end{tikzpicture}
			\end{center}
			Gọi $Q$ là trung điểm của $AI$. Ta có $AQ=QI=IN$ (vì $I$ là trọng tâm tam giác $SAC$). Có $MQ$ là đường trung bình của tam giác $ABI$. Suy ra $MQ\parallel BI$. Ta có $IK$ là đường trung bình tam giác $MNQ$. Vậy $K$ là trung điểm $MN$. Suy ra $\dfrac{KM}{KN}=1$.
			\item Chứng minh $B, I, K$ thẳng hàng. Tính tỉ số $\dfrac{IB}{IK}$.\\
			Theo cách tìm giao tuyến của câu 2 thì ba điểm $B$, $K$, $I$ thẳng hàng.\\
			Trong tam giác $ABI$, có $QM=\dfrac{1}{2}BI\Rightarrow IB=4IK\Leftrightarrow \dfrac{IB}{IK}=4$.
		\end{enumerate}
	}
\end{bt}

\begin{bt}
	Cho hình chóp $S.ABCD$ với đáy $ABCD$ là hình bình hành. Gọi $M$ là điểm bất kỳ thuộc $SB$, $N$ thuộc miền trong tam giác $S\Delta SCD$.
	\begin{tasks}(1)
		\task Tìm giao điểm của $MN$ và mặt phẳng $\left(ABCD\right)$
		\task Tìm $SC\cap \left(AMN\right)$ và $SD\cap \left(AMN\right)$
		\task Tìm $SA\cap \left(CMN\right)$
	\end{tasks}
	\loigiai{
		\begin{center}
			\begin{tikzpicture}[scale=0.7, line join=round, line cap=round,>=stealth]
				\tkzDefPoints{0/0/A,-1.2/-3/B,4.8/-3/C,6/0/D,0/5.2/S}
				\coordinate (M) at ($(S)!0.22!(B)$);
				%\coordinate (K) at ($(S)!0.3!(C)$);
				\coordinate (I) at ($(C)!0.6!(D)$);
				\coordinate (N) at ($(S)!0.4!(I)$);
				\coordinate (Q) at ($(S)!0.3!(A)$);
				\tkzInterLL(B,I)(M,N)\tkzGetPoint{H}
				\tkzInterLL(A,C)(B,I)\tkzGetPoint{O}
				\tkzInterLL(S,O)(C,Q)\tkzGetPoint{E}
				\tkzInterLL(A,E)(S,C)\tkzGetPoint{K}
				\tkzInterLL(K,N)(S,D)\tkzGetPoint{P}
				\tkzDrawPoints[fill=black](S,A,B,C,D,E,M,N,K,I,O,H,P,Q)
				\tkzLabelPoints[above](S)
				\tkzLabelPoints[left](A)
				\tkzLabelPoints[below left](B,M,E)
				\tkzLabelPoints[below right](K,N,P,C,I)
				\tkzLabelPoints[above  right](D,H)
				\tkzLabelPoints[right](Q)
				\tkzLabelPoints[below](O)
				\tkzDrawSegments(S,B S,C S,D B,C C,D S,I C,N N,H C,M H,I K,N N,P)
				\tkzDrawSegments[dashed](S,A A,B A,D A,C B,I M,N A,N S,O C,Q A,K)
			\end{tikzpicture}
		\end{center}
		\begin{enumerate}[a)]
			\item Tìm giao điểm của $MN$ và $\left(ABCD\right)$.\\
			Gọi $I=SN \cap CD$ (vì $SN,\,CD \subset \left(SCD\right)$). Chọn mặt phẳng $\left(SBI\right)$ chứa $MN$. Ta có $B$ và $I$ là hai điểm chung của hai mặt phẳng $\left(SBI\right)$ và $\left(ABCD\right)$. Vậy $\left(SBI\right)\cap \left(ABCD\right)=BI$.\\
			Gọi $H=MN \cap BI$ (vì $MN,\,BI\subset \left(SBI\right)$)
			Ta có $\heva{& H \in MN \\ & H \in BI,\, BI \subset \left(ABCD\right)}$ $\Rightarrow H=MN\cap \left(ABCD\right)$
			\item Tìm $SC\cap \left(MAN\right)$.\\
			Đầu tiên ta tìm giao tuyến của mặt phẳng $\left(SAC\right)$ và $\left(SBI\right)$. Gọi $O=AC \cap BI$ (vì $AC,\,BI \subset \left(ABCD\right)$).\\
			Ta có $S$ và $O$ là hai điểm chung của hai mặt phẳng $\left(SAC\right)$ và $\left(SBI\right)$.\\
			Vậy $SO=\left(SAC\right)\cap \left(SBI\right)$.\\
			Gọi $E=SO \cap MN$ (vì $SO,\,MN \subset \left(SBI\right)$). Chọn mặt phẳng $\left(SAC\right)$ chứa $SC$. Tìm giao tuyến của hai mặt phẳng $\left(SAC\right)$ và $\left(AMN\right)$
			\item
		\end{enumerate}
	}
\end{bt}

\begin{bt}%[Dự án HHKG 11, 2018, Chu Duc Minh]%[1H2K1-5]
	Cho tứ diện $ABCD$	. Gọi $M$ là trung điểm $AB$, $K$ là trọng tâm của tam giác $ACD$. 
	\begin{tasks}(1)
		\task Xác định giao tuyến của $(AKM)$ và $(BCD)$. 
		\task Tìm giao điểm $H$ của $MK$ và mp$(BCD)$. Chứng minh $K$ là trọng tâm của tam giác $ABH$. 
		\task Trên $BC$ lấy điểm $N$. Tìm giao điểm $P, Q$ của $CD$, $AD$ với mp$(MNK)$. 
	\end{tasks}
	\loigiai{
		\begin{center}
			\begin{tikzpicture}[scale=0.6, line join=round, line cap=round]
				\tkzDefPoints{0/0/B,1.3/-1.6/C,4.5/0/D,1/3.5/A}
				\coordinate (M) at ($1/2*(A)+1/2*(B)$);
				\coordinate (N) at ($1/3*(B)+2/3*(C)$);
				\coordinate (K) at ($1/3*(A)+1/3*(C)+1/3*(D)$);
				\coordinate (G) at ($1/2*(C)+1/2*(D)$);
				\tkzInterLL(M,K)(B,G)\tkzGetPoint{H}
				\tkzInterLL(M,N)(A,C)\tkzGetPoint{E}
				\tkzInterLL(E,K)(C,D)\tkzGetPoint{P}	
				\tkzInterLL(E,K)(A,D)\tkzGetPoint{Q}
				\tkzInterLL(M,Q)(N,P)\tkzGetPoint{F}
				\tkzDrawPolygon(A,B,C,D)
				\tkzDrawSegments(A,C A,G H,K H,G E,M E,A E,Q F,N F,M)
				\tkzDrawSegments[dashed](B,D M,K B,G N,P M,Q F,B)
				\tkzDrawPoints[fill=black,size=4](A,B,C,D,K,G,H,E)
				\tkzLabelPoints[above](A)
				\tkzLabelPoints[left](F, E)
				\tkzLabelPoints[above left](B,M)
				\tkzLabelPoints[below left](N)
				\tkzLabelPoints[below right](C,P,G)
				\tkzLabelPoints[right](D,K,Q,H)
			\end{tikzpicture}
		\end{center}
		\begin{enumerate}[a)]
			\item \textbf{Xác định giao tuyến của $(AKM)$ và $(BCD)$.} \\
			Gọi $G = AK \cap CD$  (vì $AK, CD \subset (ACD)$). \\
			Ta có $\heva{&G \in AK, AK \subset (AKM) \\ & G \in CD, CD \subset (BCD)}$\\
			$\Rightarrow G \in (AKM) \cap (BCD). \quad (1)$\\
			$B \in (ABG) \cap (BCD). \quad (2)$\\
			Từ $(1)$ và $(2)$ suy ra $(ABG) \cap (BCD) = BG$. 
			\item \textbf{Tìm giao điểm $H$ của $MK$ và} mp$(BCD)$. \\
			Trong mp$(ABG)$, gọi $H = MK \cap BG$, \\ có $\heva{& H \in MK \\& H \in BG, BG \subset (BCD)}$\\
			$\Rightarrow H = MK \cap (BCD)$.\\
			\textbf{Chứng minh $K$ là trọng tâm của tam giác $ABH$.}
			\begin{center}
				\begin{tikzpicture}[scale=1, line join=round, line cap=round]
					\tkzDefPoints{1/3/A,0/0/B,5/0/H}
					\coordinate (G) at ($1/2*(B)+1/2*(H)$);
					\coordinate (M) at ($1/2*(A)+1/2*(B)$);
					\coordinate (K) at ($1/3*(A)+1/3*(B)+1/3*(H)$);
					\coordinate (L) at ($2*(G)-(K)$);
					\tkzDrawPolygon(A,B,H)
					\tkzDrawSegments(A,L M,H L,B)
					\tkzDrawPoints[fill=black,size=4](A,B,H,G,L,M,K)
					\tkzLabelPoints[above](A)
					\tkzLabelPoints[left](B,M)
					\tkzLabelPoints[above right](K)
					\tkzLabelPoints[below right](H,L)
					\tkzLabelPoints[below left](G)
				\end{tikzpicture}
			\end{center}
			Vì $K$ là trọng tâm của tam giác $ACD$ nên $K$ chia đoạn $AG$ thành ba phần bằng nhau. \\
			Gọi $L$ là điểm đối xứng của $K$ qua $G$ thì $K$ là trung điểm của $AL$. \\ Trong $\triangle ABL$, $MK$ là đường trung bình của tam giác.\\
			Ta có $\triangle BGL = \triangle HGK$(g.c.g) $\Rightarrow BG= HG$.\\ Vậy $K$ là trọng tâm của tam giác $ABH$. 
			\item \textbf{Tìm giao điểm $P, Q$ của $CD, AD$ với} mp$(MNK)$. \\
			Trong mp$(ABC)$ gọi $E = MN \cap AC$. Trong mp$(ACD)$ đường thẳng $EK$ cắt $CD$ và $AD$ lần lượt tại $P, Q$, thì $P$ và $Q$ chính là giao điểm của $CD$ và $AD$ với mp$(MNK)$. 
		\end{enumerate}
	}
\end{bt}

\begin{bt}%[Dự án HHKG 11, 2018, Nguyễn Thành Tiến]%[1H2B1-1]%[1H2K1-3]
	Cho tứ giác $ABCD$ và $ S \not\in \left(ABCD\right)$. Gọi $I,\,J$ là hai điểm trên $AD$ và $SB$, $AD$ cắt $BC$ tại $O$ và $OJ$ cắt $SC$ tại $M$.
	\begin{tasks}(1)
		\task Tìm giao điểm $K=IJ \cap \left(SAC\right)$.
		\task Xác định giao điểm $L=DJ \cap \left(SAC\right)$.
		\task Chứng minh $A,\,K,\,L,\,M$ thẳng hàng.
	\end{tasks}
	\loigiai{
		\begin{center}
			\begin{tikzpicture}[scale=0.8, line join=round, line cap=round,>=stealth]
				\tkzDefPoints{0/0/A, 7/0/B,2/-2.1/D, 4.8/-1.6/C, 2/4/S}
				\coordinate (I) at ($(A)!0.3!(D)$);
				\coordinate (J) at ($(S)!0.33!(B)$);
				\tkzDrawPoints[fill=black](S,A,B,C,D,I,J)
				\tkzLabelPoints[above](S)
				\tkzLabelPoints[left](A)
				\tkzLabelPoints[right](B)
				\tkzLabelPoints[below right](C)
				\tkzLabelPoints[below left](D)
				\tkzLabelPoints[below left](I)
				\tkzLabelPoints[above right](J)
				\tkzDrawSegments(S,A S,B S,C S,D A,D B,C S,I)
				\tkzDrawSegments[dashed](A,C A,B B,D C,D I,J D,J I,B) 		
				\tkzInterLL(A,D)(B,C)\tkzGetPoint{O}
				\tkzInterLL(S,C)(O,J)\tkzGetPoint{M}
				\tkzInterLL (A,C)(B,I)\tkzGetPoint{E}
				\tkzInterLL(A,C)(B,D)\tkzGetPoint{F}
				\tkzInterLL(A,M)(I,J)\tkzGetPoint{K}
				\tkzInterLL(D,J)(S,F)\tkzGetPoint{L}
				\tkzDrawPoints[fill=black](O,E,M,F,K,L)
				\tkzDrawSegments[dashed](S,E A,M)
				\tkzLabelPoints[below](O)
				\tkzLabelPoints[right](M)
				\tkzLabelPoints[below](E)
				\tkzLabelPoints[below](F)
				\tkzLabelPoints[above left](L)
				\tkzDrawSegments(O,D O,C O,J)
			\end{tikzpicture}
		\end{center}	
		\begin{enumerate}[a)]
			\item Tìm giao điểm $K=IJ \cap \left(SAC\right)$.\\
			Chọn mặt phẳng phụ $\left(SIB\right)$ chứa $IJ$.\\
			Tìm giao tuyến của $\left(SIB\right)$ và $\left(SAC\right)$.\\
			có $S \in \left(SBI\right) \cap \left(SAC\right)$\hfill (1)\\
			Trong mặt phẳng $\left(ABCD\right)$ gọi $E=AC \cap BI$, ta có$\colon$\\
			$\heva{& E \in AC, \, AC \subset \left(SAC\right) \\ &E \in BI,\, BI \subset \left(SBI\right)}$ $\Rightarrow E=\left(SAC\right) \cap \left(SBI\right)$ \hfill (2)\\
			Từ $(1)$ và $(2)$ suy ra $SE=\left(SBI\right) \cap \left(SAC\right)$.\\
			Trong mặt phẳng $\left(SIB\right)$, gọi $K=IJ \cap SE$.\\
			Ta có $\heva{& K \in IJ \\ & K \in SE, SE \subset \left(SAC\right)}$ $\Rightarrow K=IJ \cap \left(SAC\right)$
			\item Xác định giao điểm $L=DJ \cap \left(SAC\right)$.\\
			Chọn mặt phẳng phụ $\left(SBD\right)$ chứa $DJ$. Tìm giao tuyến của $\left(SBD\right)$ với $\left(SAC\right)$.\\
			Ta có $S \in \left(SBD\right) \cap \left(SAC\right) $ \hfill (3)\\
			\noindent Trong mặt phẳng $\left(ABCD\right)$ gọi $F=AC \subset BD$. Suy ra $F$ là điểm chung thứ hai của hai mặt phẳng $\left(SBD\right)$ và $\left(SAC\right)$. \hfill (4)\\
			Từ $(3)$ và $(4)$ suy ra $SF=\left(SBD\right)\subset \left(SAC\right)$. Trong mặt phẳng $\left(SBD\right)$ gọi $L=DJ \cap SF $.\\
			Vậy
			$\heva{& L \in DJ \\ & L \in SF,\, SF \subset \left(SAC\right)}$ $\Rightarrow L=DJ \cap \left(SAC\right)$
			\item Chứng minh $A,\, K,\, L, \, M$ thẳng hàng.\\
			Ta có $A \in \left(SAC\right)\cap \left(AJO\right)$ \quad (3)\\
			và $\heva{& K \in IJ,\, IJ \subset \left(AJO\right) \\ & K \in SE,\, SE \subset \left(SAC\right)}$ $\Rightarrow K \in \left(SAC\right)\cap \left(AJO\right)$. \quad (4)\\
			có $\heva{& L \in DJ,\, DJ \subset \left(AJO\right) \\ & L \in SF,\, SF \subset \left(SAC\right)}$ $\Rightarrow L \in \left(SAC\right) \cap \left(AJO\right)$ \quad (5)\\
			có $\heva{& M \in JO,\, JO \subset \left(AJO\right) \\ & M \in SC,\, SC \subset \left(SAC\right)}$ $\Rightarrow M \in \left(SAC\right)\cap \left(AJO\right)$ \quad (6)\\
			Từ (3), (4), (5) và (6) suy ra bốn điểm $A,\,K,\, L,\,M$ cùng thuộc giao tuyến của hai mặt phẳng $\left(SAC\right)$ và $\left(AJO\right)$. Vậy $A,\,K,\,L,\,M$ thẳng hàng.
		\end{enumerate}
	}	
\end{bt}

\begin{bt}
	Cho hình chóp $S.ABCD$ có đáy $ABCD$ và hình bình hành. Gọi $G$ là trọng tâm của tam giác $SAD$, $M$ là trung điểm của $SB$. 
	\begin{tasks}(1)
		\task Tìm giao điểm $N$ của $MG$ và mặt phẳng $(ABCD)$. 
		\task Chứng minh ba điểm $C, D, N$ thẳng hàng và $D$ là trung điểm của $CN$. 
	\end{tasks}
	\loigiai{
		\begin{center}
			\begin{tikzpicture}[scale=1, line join=round, line cap=round]
				\tkzDefPoints{0/0/A,-1.7/-1.6/B,2.5/-1.6/C}
				\coordinate (D) at ($(A)+(C)-(B)$);
				\coordinate (S) at ($(A)+(-0.5,3)$);
				\coordinate (G) at ($1/3*(S)+1/3*(A)+1/3*(D)$);
				\coordinate (M) at ($1/2*(S)+1/2*(B)$);
				\coordinate (E) at ($1/2*(A)+1/2*(D)$);
				\coordinate (F) at ($1/2*(B)+1/2*(M)$);
				\tkzInterLL(B,E)(M,G)\tkzGetPoint{N}
				\tkzInterLL(B,N)(S,D)\tkzGetPoint{X}
				\tkzInterLL(M,N)(S,D)\tkzGetPoint{Y}
				
				\tkzDrawPolygon(S,B,C,D)
				\tkzDrawSegments(S,C S,N Y,N X,N N,D)
				\tkzDrawSegments[dashed](A,S A,B A,D S,E F,E B,X M,Y)
				\tkzDrawPoints[fill=black,size=4](D,C,A,B,S,M,G,N,E,F)
				\tkzLabelPoints[above](S)
				\tkzLabelPoints[left](A,M,F)
				\tkzLabelPoints[below](B,C,E)
				\tkzLabelPoints[right](D,N)
				\tkzLabelPoints[above right](G)
			\end{tikzpicture}
		\end{center}
		\begin{enumerate}[a)]
			\item Trong mặt phẳng chứa $MG$, gọi $N$ là giao điểm của $MG$ và $BE$. 
			Vì $BE$ thuộc mặt phẳng $(ABCD)$, nên $N$ thuộc $(ABCD)$. Vậy $N$ là giao điểm của $MG$ và mặt phẳng $(ABCD)$.
			\item Trong mặt phẳng $(SBN)$, kẻ $EF \parallel MN$ ($F$ thuộc $SB$). \\
			Trong tam giác $SEF$ có $MG \parallel EF$ nên $$\dfrac{SM}{MF} = \dfrac{SG}{GE} = 2 \Rightarrow SM = 2MF \Leftrightarrow BM = 2MF.$$
			Vậy $F$ là trung điểm của $BM$. 
		\end{enumerate}	
		Trong $\triangle  BMN$ có $EF \parallel MN$ nên  $\dfrac{BF}{FM} = \dfrac{BE}{EN} = 1 \Rightarrow BE = EN$. Vậy $E$ là trung điểm của $BN$. 
		
		Dễ dàng chứng minh $\triangle AEB = \triangle DEN$ (c.g.c) $\Rightarrow \widehat{ABE} = \widehat{END}$.\\ Hai góc này bằng nhau theo trường hợp so le trong nên $AB \parallel DN$,  mà $AB \parallel CD$ nên $C, D, N$ thẳng hàng. 
		
		$ED$ là đường trung bình của tam giác $NBC$ suy ra $D$ là trung điểm của $CN$. 
		
	}
\end{bt}

\begin{bt}
	Cho hình chóp $S.ABCD$, đáy $ABCD$ là hình bình hành tâm $O$. Gọi $M$ là trung điểm của $SC$. 
	\begin{tasks}(1)
		\task Xác định giao tuyến của $(ABM)$ và $(SCD)$. 
		\task Gọi $N$ là trung điểm của $BO$. Xác định giao điểm $I$ của $(AMN)$ với $SD$. Chứng minh $\dfrac{SI}{ID} = \dfrac{2}{3}$. 
		%Tìm thiết diện của hình chóp $S.ABCD$ cắt bởi mặt phẳng $(AMN)$. 
	\end{tasks}
	\loigiai{
		\begin{center}
			\begin{tikzpicture}[scale=1, line join=round, line cap=round]
				\tkzDefPoints{0/0/A,-3/-2.4/B,2.4/-2.4/C}
				\coordinate (D) at ($(A)+(C)-(B)$);
				\coordinate (O) at ($(A)!1/2!(C)$);
				\coordinate (S) at ($(A)+(0.5,3)$);
				\coordinate (M) at ($1/2*(S)+1/2*(C)$);
				\coordinate (N) at ($1/2*(B)+1/2*(O)$);
				\coordinate (H) at ($1/2*(S)+1/2*(D)$);
				\tkzInterLL(A,M)(S,O)\tkzGetPoint{K}
				\tkzInterLL(N,K)(S,D)\tkzGetPoint{I}
				\tkzInterLL(A,N)(B,C)\tkzGetPoint{L}
				\coordinate (P) at ($2/3*(I)+1/3*(D)$);
				\tkzFillPolygon[color=gray!50!](A,I,M,L)
				\tkzDrawPolygon(S,B,C,D)
				\tkzDrawSegments(S,C M,H M,I L,M)
				\tkzDrawSegments[dashed](A,S A,B A,D A,C B,D S,O A,M N,I A,I A,L)
				\tkzDrawPoints[fill=black,size=4](D,C,A,B,S,N,O,K,I,P,H,L)
				\tkzLabelPoints[above](S)
				\tkzLabelPoints[left](A)
				\tkzLabelPoints[below](B,C,L,O)
				\tkzLabelPoints[right](D,M)
				\tkzLabelPoints[above right](I,H)
				\tkzLabelPoints[left](K)
			\end{tikzpicture}
		\end{center}
		\begin{enumerate}[a)]
			\item Xác định giao tuyến của $(ABM)$ và $(SCD)$. \\
			Ta có $\heva{& M \in (ABM) \cap (SCD)\\ &AB \parallel CD\\ &AB \subset (ABM), CD \subset (SCD)}\Rightarrow (ABM) \cap (SCD) = MH$ ($MH \parallel AB \parallel CD$.)
			\item Xác định giao điểm $I$ của $(AMN)$  và $SD$
			Ta có $(SAC) \cap (SBD) = SO$. Gọi $K = AM \cap SO$  ($AM, SO \subset (SAC)$). \\
			\textbf{Tìm giao tuyến $(AMN)$ và $(SBD)$.}\\
			Ta có $\heva{&N \in (AMN)\\ &N \in BD, BD \subset (SBD)} \Rightarrow N \in (AMN) \cap (SBD). \quad (1)$\\
			$\heva{&K \in AM, AM \subset (AMN)\\ & K \in BD, BD \subset (SBD)} \Rightarrow K \in (AMN) \cap (SBD). \quad (2)$\\
			Từ $(1)$ và $(2)$ suy ra $(AMN) \cap (SBD) = NK$. 
			$NK$ cắt $SD$ tại điểm $I$, thì $I$ chính là giao điểm của $(AMN)$ và $SD$. \\
			\begin{center}
				\begin{center}
					\begin{tikzpicture}[scale=1, line join=round, line cap=round]
						\tkzDefPoints{0/0/B,1/4/S,5/0/D}
						\coordinate (N) at ($(B)!1/4!(D)$);
						\coordinate (O) at ($(B)!1/2!(D)$);
						\coordinate (I) at ($3/5*(S)+2/5*(D)$);
						\coordinate (P) at ($(I)!1/3!(D)$);
						\coordinate (K) at ($(S)!2/3!(O)$);
						\tkzDrawPolygon(S,B,D)
						\tkzDrawSegments(O,P N,I S,O)
						\tkzDrawPoints[fill=black,size=4](S,B,D,I,N,P,K)
						\tkzLabelPoints[above](S)
						\tkzLabelPoints[below](B,D,O,N)
						\tkzLabelPoints[above right](I,P)
						\tkzLabelPoints[left](K)
					\end{tikzpicture}
				\end{center}
			\end{center}
			Trong mặt phẳng $(SBD)$, từ $O$ dựng $OP \parallel NI (P \in SD)$. \\
			Trong $\triangle DNI$, có $OP \parallel DI$ nên có $\dfrac{DO}{ON} = \dfrac{DP}{PI} = \dfrac{2}{1} = 2 \Rightarrow DP = 2PI. \quad (3)$\\
			Trong $\triangle SOP$ có $KI \parallel OP$ nên có $\dfrac{SK}{KO} = \dfrac{SI}{PI} = \dfrac{2}{1} = 2 \Rightarrow SI = 2PI. \quad (4)$ 
			($K$) là trọng tâm của $\triangle SAC$. 
			Từ $(3)$ và $(4)$ suy ra $\dfrac{IS}{ID} = \dfrac{2}{3}$. 
			
		%	\textbf{Thiết diện của hình chóp bị cắt bởi mặt phẳng $(AMN)$.} \\
		%	Gọi $L$ là giao điểm của $AN$ và $BC$. Kết luận thiết diện là tứ giác $ALMI$. 
		\end{enumerate}
	}
\end{bt}


\begin{bt}
	Cho tứ diện $SABC$. Gọi $I$, $H$ lần lượt là trung điểm của $SA$, $AB$. Trên cạnh $SC$ lấy điểm $K$ sao cho $CK=3SK$.
	\begin{tasks}(1)
		\task Tìm giao điểm $F$ của $BC$ với mặt phẳng $(IHK)$. Tính tỉ số $\dfrac{FB}{FC}$.
		\task Gọi $M$ là trung điểm của đoạn thẳng $IH$. Tìm giao điểm của $KM$ và mặt phẳng $(ABC)$.
	\end{tasks}
	\loigiai{
		\begin{center}
			{\begin{tikzpicture}[scale=1, font=\footnotesize,line join=round, line cap=round,>=stealth]
					\tkzDefPoints{-2/0/A,0/-2.5/B,3/0/C,1/4/S}
					\coordinate (I) at ($(A)!0.5!(S)$);
					\coordinate (H) at ($(A)!0.5!(B)$);
					\coordinate (M) at ($(I)!0.5!(H)$);
					\coordinate (K) at ($(S)!1/4!(C)$);
					\coordinate (D) at ($(S)!0.5!(C)$);
					\coordinate (N) at ($(B)!0.5!(C)$);
					\tkzInterLL(A,C)(K,I)\tkzGetPoint{E}
					\tkzInterLL(E,H)(B,C)\tkzGetPoint{F}
					\tkzInterLL(K,M)(E,H)\tkzGetPoint{J}
					\tkzDrawSegments(H,B B,C S,I S,B S,C E,I K,F E,H I,H M,J)
					\tkzDrawSegments[dashed](A,H A,C A,N H,F H,K A,D M,K E,A A,I I,K)
					\tkzDrawPoints[fill=black](A,B,C,S,I,H,M,K,E,F,D,J,N)
					\tkzLabelPoints[left](A,B,E)
					\tkzLabelPoints[right](C,M,K,F,D,N)
					\tkzLabelPoints[above](S)
					\tkzLabelPoints[above left](I)
					\tkzLabelPoints[below](H,J)
					\tkzMarkSegments[mark=|](S,K K,D)
					\tkzMarkSegments[mark=||](A,H H,B)
					\tkzMarkSegments[mark=|||](A,I S,I)
			\end{tikzpicture}}
		\end{center}
		\begin{enumerate}[a)]
			\item Tìm giao điểm $F$ của $BC$ với mặt phẳng $(IHK)$. Tính tỉ số $\dfrac{FB}{FC}$.\\
			$\bullet$ Ta tìm giao tuyến của $(ABC)$ và $(IHK)$ trước.\\
			Gọi $E=AC\cap KI$ ($AC,\,KI\subset (SAC)$), ta có\\
			$\heva{&E\in AC,\, AC\subset (ABC)\\&E\in KI,\, KI\subset (IHK)}\Rightarrow E\in(ABC)\cap (IHK)$. \hfill $(1)$\\
			$\heva{&H\in (IHK)\\&H\in AB,\, AB\subset (ABC)}\Rightarrow H\in(ABC)\cap (IHK)$. \hfill $(2)$\\
			Từ $(1)$ và $(2)$ suy ra $EH=(ABC)\cap (IHK)$.\\
			$\bullet$ Gọi $F=EH\cap BC$ ($EH,\,BC\subset (ABC)$), có
			$$\heva{&F\in BC\\&F\in EH, \, EH\subset (IHK)}\Rightarrow F=BC\cap (IHK).$$
			Gọi $D$ là trung điểm của $SC$, ta có $IK$ là đường trung bình của $\triangle SAD$.\\
			Trong $\triangle CEK$ có $\dfrac{CA}{AE}=\dfrac{CD}{DK}=2\Rightarrow CA=2CK$.\\
			Trong mặt phẳng $(ABC)$ kẻ $AN\parallel EF$ ($N\in BC$).
			Ta có
			\begin{eqnarray*}
				&& HF\parallel AN\Rightarrow \dfrac{BH}{HA}=\dfrac{BF}{FN}=1\Rightarrow BF=FN.\\
				&& EF\parallel AN\Rightarrow \dfrac{CA}{AE}=\dfrac{CN}{NF}=2\Rightarrow CN=2NF.
			\end{eqnarray*}
			Do đó $\dfrac{FB}{FC}=\dfrac{FB}{FN+NC}=\dfrac{FB}{3FB}=\dfrac{1}{3}$.
			\item Tìm giao điểm của $KM$ và mặt phẳng $(ABC)$.\\
			Ta có $KM\subset (IHK)$. Gọi $J=KM\cap EH$ ($ EH,\, KM\subset (IHK)$).\\
			Ta có $\heva{&J\in KM\\&J\in EH,\, EH\subset (ABC)}\Rightarrow J=KM\cap (ABC)$.
		\end{enumerate}
	}
\end{bt}