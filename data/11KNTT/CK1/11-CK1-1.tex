\begin{name}
	{\tenchude}
	{\tendethi}
	{\tentruong}
	{\thoigian}
\end{name}
\setcounter{ex}{0}\setcounter{bt}{0}

\caulc
\Opensolutionfile{ans}[ans/11-CK1-Chuyen-Hung-Vuong-Phan-1]
\begin{ex}%Câu 7% [1D1N2-2]
	Cho $\dfrac{\pi}{2}<\alpha <\pi $. Mệnh đề nào dưới đây đúng?
	\choice
	{\True $\sin\alpha > 0$}
	{$\cot\alpha > 0$}
	{$\cos\alpha > 0$}
	{$\tan\alpha > 0$}
	\loigiai{
		Với $\dfrac{\pi}{2}<\alpha <\pi $ thì $\sin >0$.
	}
\end{ex}
\begin{ex}%Câu 10%[1D1H3-3]
	Cho $\sin\alpha=\dfrac{2}{3}$. Tính $\cos 2\alpha$.
	\choice
	{$-\dfrac{1}{9}$}
	{$\dfrac{1}{3}$}
	{$-\dfrac{1}{3}$}
	{\True $\dfrac{1}{9}$}
	\loigiai{
		Ta có $\cos2\alpha=1-2\sin^2\alpha=1-2\cdot\left(\dfrac{2}{3}\right)^2=\dfrac{1}{9}$.
	}
\end{ex}

\begin{ex}%[Mức độ 1]%[Dự án BG-K11-Lần-2]%[Ngô Quang Anh]%[1D1N4-2]
Tập xác định của hàm số $y=\tan4x$ là
\choice
{$\mathbb{R}\setminus\left\{\dfrac{k\pi}{2} \,\middle|\, k\in\mathbb{Z}\right\}$}
{\True $\mathbb{R}\setminus\left\{\dfrac{\pi}{8}+\dfrac{k\pi}{4} \,\middle|\, k\in\mathbb{Z}\right\}$}
{$\mathbb{R}\setminus\left\{\dfrac{k\pi}{4} \,\middle|\, k\in\mathbb{Z}\right\}$}
{$\mathbb{R}\setminus\left\{\dfrac{\pi}{8}+\dfrac{k\pi}{2}\,\middle|\, k\in\mathbb{Z}\right\}$}
\loigiai{
Hàm số xác định khi $4x\ne\dfrac{\pi}{2}+k\pi\Leftrightarrow x\ne\dfrac{\pi}{8}+\dfrac{k\pi}{4}$, $k\in\mathbb{Z}$.\\
Vậy tập xác định của hàm số đã cho là $\mathscr{D} = \mathbb{R}\setminus\left\{\dfrac{\pi}{8}+\dfrac{k\pi}{4} \,\middle|\, k\in\mathbb{Z}\right\}$.
}
\end{ex}

\begin{ex}%[Đề thi HK 1 Lớp 11- THPT Núi Thành - Quảng Nam, Năm học 2023-2024]%[Tran Quoc, 10-11-HK1-2024]%[1D1N5-1]
Khẳng định nào sau đây là đúng?
\choice
{\True $\sin a=\sin b \Leftrightarrow \hoac{&a=b+k 2 \pi \\ &a=\pi-b+k 2 \pi}, k \in \mathbb{Z}$}
{$\sin a=\sin b \Leftrightarrow \hoac{&a=b+k 2 \pi \\ &a=-b+k 2 \pi}, k \in \mathbb{Z}$}
{$\sin a=\sin b \Leftrightarrow \hoac{&a=b+k \pi \\ &a=\pi-b+k \pi}, k \in \mathbb{Z}$}
{$\sin a=\sin b \Leftrightarrow \hoac{&a=b+k \pi \\ &a=-b+k \pi}, k \in \mathbb{Z}$}
\loigiai{Khẳng định đúng là $\sin a=\sin b \Leftrightarrow \hoac{&a=b+k 2 \pi \\ &a=\pi-b+k 2 \pi}, k \in \mathbb{Z}$.}
\end{ex}

\begin{ex}%Câu 2%[1H4N4-3] 
	Cho hình chóp $ S.ABCD$ có đáy $ ABCD$ là hình chữ nhật (tham khảo hình vẽ). Gọi $ E$, $F$ theo thứ tự là trung điểm $ AB$, $CD$. $M$, $N$ theo thứ tự là trọng tâm $\triangle SAB$, $\triangle SCD$.
	\begin{center}
		\begin{tikzpicture}[scale=1, font=\footnotesize, line join=round, line cap=round, >=stealth]
			\path
			(0,0) coordinate (A)
			(-1.8,-1.5) coordinate (B)
			(4,0) coordinate (D)
			($(B)+(D)-(A)$) coordinate (C)
			(0.7,3) coordinate (S)
			($(B)!0.5!(A)$) coordinate (E)
			($(C)!0.5!(D)$) coordinate (F)
			($(S)!.5!(E)$) coordinate (M)
			($(S)!0.5!(F)$) coordinate (N)
			;
			\draw 
			(S)--(B)--(C)--(D)--(S)--(C) (S)--(F)
			;
			\draw[dashed] 
			(S)--(A) (S)--(E)--(F) (M)--(N)
			(B)--(A)--(D)
			;
			\foreach \p/\g in {A/170, B/-90, C/-90, D/0, S/90, E/-90, F/-10, N/20, M/-60}
			\draw[fill=black] (\p) circle (1pt) node[shift=(\g:3mm)] {$\p$};
			%	\pic[draw,angle radius=2mm]{angle=A--H--O};
		\end{tikzpicture}
	\end{center}
	Khẳng định nào sau đây sai?
	\choice
	{\True $MN\parallel (SEF)$}
	{$MN\parallel AD$}
	{$MN\parallel EF$}
	{$MN \parallel (ABCD)$}
	\loigiai{
	Ta có $MN$ là đường trung bình tam giác $SEF$ nên $MN\parallel EF$.\\
	Suy ra $MN\parallel AD$ và $MN\parallel (ABCD)$.
	Khẳng định sai là $MN\parallel (SEF)$.
}
\end{ex}


\begin{ex}%Câu 4%[1D2N1-4] 
	Dãy số nào trong các dãy số sau là dãy số tăng?
	\choice
	{$-1$; $1$; $-1$; $1$; $-1$}
	{\True $1$; $3$; $5$; $7$; $9$}
	{$-2$; $-4$; $-6$; $-8$}
	{$\dfrac{1}{2}$; $\dfrac{1}{22}$; $\dfrac{1}{222}$; $\dfrac{1}{2222}$, $\ldots$}
	\loigiai{
	Dãy số tăng là $1$; $3$; $5$; $7$; $9$.	
	}
\end{ex}

\begin{ex}%[Dự án tài liệu ôn tập THPT 2025]%[Nguyễn Tấn Linh]%[1D2N2-2]
Trong các dãy số $(u_n)$ với số hạng tổng quát sau, dãy số nào là cấp số cộng?
\choice
{$u_n=5^n$}
{\True $u_n=2-5n$}
{$u_n=5^n-2$}
{$u_n=5+n^2$}
\loigiai{
Xét $u_n=2-5n \Rightarrow u_{n+1}=2-5(n+1)$.\\
Ta có $u_{n+1}-u_n=2-5(n+1)-(2-5n)=-5$ (hằng số).\\
Vậy $u_n=2-5n$ là cấp số cộng có số hạng đầu $u_1=2$ và công sai $d=-5$.}
\end{ex}

\begin{ex}%[1D2H2-2]%[CTST - Lớp 11 - Ôn tập cuối học kì 1 - Đề 6]%[Lâm Tú]
Cho một cấp số cộng $\left(u_n\right)$ có $u_1=5$ và tổng của 40 số hạng đầu là $3320$. Tìm công sai của cấp số cộng đó.
\choice
{$-4$}
{$8$}
{$-8$}
{\True $4$}
\loigiai{
Gọi $d$ là công sai của cấp số cộng.\\
Ta có tổng 40 số hạng đầu của cấp số cộng là
\begin{eqnarray*}
& & S_{40}=\dfrac{40\left(2u_1+39d\right)}{2}=3320\\
&\Leftrightarrow & \dfrac{40\left(2.5+39d\right)}{2}=3320\\
&\Leftrightarrow & d=4.
\end{eqnarray*}
}
\end{ex}

\begin{ex}%Câu 5% [1D5N1-2] 
	Cho mẫu số liệu ghép nhóm về thời gian đi từ nhà đến nơi làm việc của các nhân viên một công ty như sau:
	\begin{center}
		\begin{tabular}{|c|c|c|c|c|c|c|c|}
			\hline
			Thời gian (phút) & $[15;20)$ & $[20;25)$ & $[25;30)$ & $[30;35)$ & $[35;40)$ & $[40;45)$ & $[45;50)$\\
			\hline
			Số nhân viên & $6$ & $14$ & $25$ & $37$ & $21$ & $13$ & $9$\\
			\hline
	\end{tabular}
\end{center}
	Giá trị đại diện của nhóm $\left[30;35\right)$ là
	\choice
	{\True $ 32{,}5$}
	{$35$}
	{$5$}
	{$30$}
	\loigiai{
		Giá trị đại diện của nhóm $\left[30;35\right)$ là $\dfrac{30+35}{2}=32{,}5$.
	}
\end{ex}

\begin{ex}%[1D5H1-2]%[Dự án 11 HVA 2024-2025]%[Bồ Văn Hậu]
Khảo sát thời gian tự học bài ở nhà của một số em học sinh lớp $11$ thu được mẫu ghép nhóm số lượng như sau:
\begin{center}
\begin{tabular}{|c|c|c|c|c|c|}
\hline
Thời gian & {$[0; 30)$} & {$[30; 60)$} & {$[60; 90)$} & {$[90; 120)$} & {$[120; 150)$} \\
\hline
Số học sinh & $9$ & $10$ & $9$ & $15$ & $7$ \\
\hline
\end{tabular}
\end{center}
Mốt của số liệu ghép nhóm trên là
\choice
{\True $102{,}86$}
{$102{,}85$}
{$102{,}8$}
{$102{,}9$}
\loigiai{
Mốt của số liệu ghép nhóm là $90+\dfrac{15-9}{15-9+15-7} \cdot 30 \approx 102{,}86$.
}
\end{ex}

\begin{ex}%Câu 9%[1H4N4-1]
	Trong không gian, khẳng định nào sau đây \textbf{sai}?
	\choice
	{\True Qua một điểm nằm ngoài mặt phẳng, có một và chỉ một đường thẳng song song với mặt phẳng đó}
	{Tồn tại bốn điểm không cùng thuộc một mặt phẳng}
	{Nếu hai mặt phẳng phân biệt có 1 điểm chung thì chúng có vô số điểm chung}
	{Một mặt phẳng hoàn toàn xác định nếu biết nó chứa hai đường thẳng cắt nhau}
	\loigiai{
		Qua một điểm nằm ngoài mặt phẳng, có một và chỉ một mặt phẳng song song với mặt phẳng đó.
	}
\end{ex}
\begin{ex}%[1D3N1-1]%[CD - Lớp 11-Ôn tập cuối học kì 1-Đề 2]%[Duong Xuan Loi]
Cho dãy số $(u_{n})$ có $\lim\limits_{n \to +\infty}u_{n}=2$. Tính giới hạn $\lim\dfrac{3u_{n}-1}{2 u_{n}+5}$?
\choice
{$-\dfrac{1}{5}$}
{\True $\dfrac{5}{9}$}
{$\dfrac{3}{2}$}
{$+\infty$}
\loigiai{
Từ $ \lim\limits_{n \to +\infty}u_{n}=2$ ta có $ \lim\dfrac{3u_{n}-1}{2u_{n}+5} =\dfrac{3\cdot 2-1}{2\cdot 2+5}=\dfrac{5}{9}$.}
\end{ex}
\Closesolutionfile{ans}
% \indapan{7}{ans/11-CK1-Chuyen-Hung-Vuong-Phan-1}

\cauds
\Opensolutionfile{ans}[ans/11-CK1-Chuyen-Hung-Vuong-Phan-2]

\begin{ex}%[1H4H3-2]%[Nguyễn Sơn Thành] 
\immini[thm]{Cho hai hình bình hành $ABCD$ và $ABEF$ không cùng nằm trong một mặt phẳng và có tâm lần là $O$ và $O^\prime$. Gọi $M$, $N$ theo thứ tự là hai điểm trên các cạnh $AE$, $BD$ sao cho $AM=\dfrac{1}{3} AE$, $B N=\dfrac{1}{3}BD$ (tham khảo hình vẽ).}{\begin{tikzpicture}[line join=round,line cap=round,>=stealth,font=\footnotesize,scale=.7]
		\path 
		(0:0) coordinate (A)
		(0:5) coordinate (B)
		(-120:2.5) coordinate (D)
		($(B)+(D)-(A)$) coordinate (C)
		(110:3.5) coordinate (F)
		($(B)+(F)-(A)$) coordinate (E)
		($(A)!.333!(E)$) coordinate (M)
		($(B)!.3333!(D)$) coordinate (N)
		($(A)!.5!(B)$) coordinate (I)
		(intersection of A--C and B--D) coordinate (O)
		(intersection of A--E and B--F) coordinate (O')
		;
		\draw (E)--(F)--(D)--(C)--cycle  (B)--(E) (B)--(C);
		\draw[dashed] (A)--(B) (B)--(F) (A)--(D) (A)--(F) (A)--(E) (A)--(C) (B)--(D)(O)--(O') (M)--(N);
		\foreach \x/\g in {A/65,B/30,C/-90,D/-90,E/90,F/90,O/-90,O'/90,M/90,N/90}\draw[fill=white] (\x) circle (.5pt)+(\g:.3)node{$\x$};
\end{tikzpicture}}	

	\choiceTF
	{\True $OO^\prime$ song song với mặt phẳng $(ADF)$}
	{$OO^\prime$ cắt mặt phắng $(BCE)$}
	{\True $MN$ song song với $CF$}
	{\True $M N$ song song với mặt phẳng ( $CDFE)$}
	\loigiai{
		\begin{center}
			\begin{tikzpicture}[line join=round,line cap=round,>=stealth,font=\footnotesize,scale=.7]
				\path 
				(0:0) coordinate (A)
				(0:5) coordinate (B)
				(-120:2.5) coordinate (D)
				($(B)+(D)-(A)$) coordinate (C)
				(110:3.5) coordinate (F)
				($(B)+(F)-(A)$) coordinate (E)
				($(A)!.333!(E)$) coordinate (M)
				($(B)!.3333!(D)$) coordinate (N)
				($(A)!.5!(B)$) coordinate (I)
				(intersection of A--C and B--D) coordinate (O)
				(intersection of A--E and B--F) coordinate (O')
				;
				\draw (E)--(F)--(D)--(C)--cycle  (B)--(E) (B)--(C)--(F);
				\draw[dashed] (A)--(B) (B)--(F) (A)--(D) (A)--(F) (A)--(E) (A)--(C) (B)--(D)(C)--(I)--(F) (M)--(N) (O)--(O') ;
				\foreach \x/\g in {A/65,B/30,C/-90,D/-90,E/90,F/90,O/-90,O'/90,M/90,N/90,I/90}\draw[fill=white] (\x) circle (.5pt)+(\g:.3)node{$\x$};
			\end{tikzpicture}
			\end{center}
		\begin{itemchoice}
			\itemch \textbf{Đúng}. Vì 
			$\heva{&OO'\parallel DF\\&DF \subset \left(ADF\right)\\&OO'\not\subset \left(ADF\right)}\Rightarrow OO'\parallel \left(ADF\right)$.
			\itemch \textbf{Sai}. Vì 
			$\heva{&OO'\parallel CE\\&CE \subset \left(BCE\right)\\&OO'\not\subset \left(BCE\right)}\Rightarrow OO'\parallel \left(BCE\right)$.
			\itemch \textbf{Đúng}. Gọi $I$ là trung điểm $AB$. Ta có $\dfrac{IN}{IC}=\dfrac{IM}{IF}=\dfrac{1}{3}\Rightarrow MN \parallel CF$.
			\itemch \textbf{Đúng}. 
			Ta có $\heva{& MN\parallel CF\\&CF \subset \left(CDFE\right)\\&MN \not\subset \left(CDFE\right)}\Rightarrow MN \parallel \left(CDFE\right)$.
		\end{itemchoice}
	}
\end{ex}
\begin{ex}%[1D3V3-3]%[KNTT - Lớp 11 - Ôn tập cuối học kì 1 - Đề 3]%[Võ Thị Thùy Trang]
Cho hàm số $f(x)=\heva{&\dfrac{\sqrt{x+3}-2}{x-1}&& \text{khi}~ x > 1\\
&-x^2+m&&\text{khi}~ x\leq 1}$.
\choiceTF
{\True Hàm số xác định trên $\mathbb{R}$}
{\True $f(1)=-1+m$}
{$\lim\limits_{x \to 1^{-}} f(x)=1+m$}
{Hàm số liên tục tại $x=1$ khi $m=-\dfrac{3}{4}$}
\loigiai{
\begin{itemchoice}
\itemch \textbf{Đúng}.\\
Hàm số xác định trên $\mathbb{R}$.
\itemch \textbf{Đúng}.\\
Ta có $f(1)=-1+m$.
\itemch \textbf{Sai}.\\
Ta có $\lim\limits_{x \to 1^{-}} f(x)=\lim\limits_{x \to 1^{-}}\left(-x^2+m\right)=-1+m$.
\itemch \textbf{Sai}.\\
Ta có
\begin{itemize}
\item $f(1)=-1+m$.
\item $\lim\limits_{x \to 1^{-}} f(x)=\lim\limits_{x \to 1^{-}}\left(-x^2+m\right)=-1+m$.
\item
\allowdisplaybreaks
\begin{eqnarray*}
\lim\limits_{x \to 1^{+}} f(x)&=&\lim\limits_{x \to 1^{+}} \dfrac{\sqrt{x+3}-2}{x-1}\\
&=&\lim\limits_{x \to 1^{+}} \dfrac{(\sqrt{x+3}-2)(\sqrt{x+3}+2)}{(x-1)(\sqrt{x+3}+2)}\\
&=&\lim\limits_{x \to 1^{+}} \dfrac{(\sqrt{x+3})^2-2^2}{(x-1)(\sqrt{x+3}+2)}\\
&=&\lim\limits_{x \to 1^{+}} \dfrac{x-1}{(x-1)(\sqrt{x+3}+2)}\\
&=&\lim\limits_{x \to 1^{+}} \dfrac{1}{\sqrt{x+3}+2}=\dfrac{1}{4}.
\end{eqnarray*}
\end{itemize}
Hàm số liên tục tại $x=1$ khi và chỉ khi $\lim\limits_{x \to 1^{+}} f(x)=\lim\limits_{x \to 1^{-}} f(x)=f(1) \Leftrightarrow-1+m=\dfrac{1}{4} \Leftrightarrow m=\dfrac{5}{4}$.
\end{itemchoice}
}
\end{ex}

\Closesolutionfile{ans}
% \indapan{3}{ans/11-CK1-Chuyen-Hung-Vuong-Phan-2}

\caukq
\Opensolutionfile{ans}[ans/11-CK1-Chuyen-Hung-Vuong-Phan-3]

%%%==============Bai_BT3==============%%%
\begin{ex}%[1D1H5-6]
	Một quả bóng được ném xiên một góc $\alpha~\left(0^{\circ} < \alpha < 90^{\circ}\right)$ từ mặt đất với tốc độ $v_0$ (m/s). Khoảng cách theo phương ngang từ vị trí ban đầu của quả bóng đến vị trí bóng chạm đất được tính theo công thức $d=\dfrac{v_0^2\sin 2\alpha}{10}$. Nếu tốc độ ban đầu của quả bóng là $10\mathrm{~m} / s$ thì tồn tại hai góc ném $\alpha_1$, $\alpha_2~\left(\alpha_1 > \alpha_2\right)$ để khoảng cách $d$ là $5$ m. Hiệu số đo hai góc ném $\alpha_1-\alpha_2$ bằng bao nhiêu độ?
	
	\shortans{60}
	\loigiai{
	Ta có $d = \dfrac{v_0^2 \sin 2\alpha}{10}$.\\
	Trong bài toán, $v_0 = 10$ m/s và $d = 5$ m, nên ta có	
	\[5 = \dfrac{(10)^2 \sin 2\alpha}{10}\Leftrightarrow 
	\sin 2\alpha = \dfrac{1}{2}\Leftrightarrow\hoac{&2\alpha=30^\circ\\&2\alpha=150^\circ}\Leftrightarrow\hoac{&\alpha=15^\circ\\&\alpha=75^\circ.}\]	
	Hiệu số giữa hai góc ném $ \alpha_1 - \alpha_2$ là	
	\[\alpha_1 - \alpha_2 = 75^\circ - 15^\circ = 60^\circ.\]	
	Vậy hiệu số đo hai góc ném là $60^\circ$.
	}
\end{ex}
%%%==============HetBai_BT3==============%%%

%%%==============Bai_BT4==============%%%
\begin{ex}%[1D2H3-8]
	Theo thống kê của Chi cục Dân số Hà Nội, tính đến năm $2024$, dân số thủ đô Hà Nội ước tính đạt khoảng $8{,}5$ triệu người và tốc độ tăng trưởng dân số là $1{,}26\%$. Nếu tốc độ tăng trưởng dân số này được giữ nguyên hàng năm, hãy ước tính dân số của thủ đô Hà Nội vào năm $2030$ (tính theo đơn vị triệu người, làm tròn đến hàng phần trăm).
	
	\shortans{9,15}
	\loigiai{
	Bài toán yêu cầu tính dân số của Hà Nội sau $6$ năm (từ năm $2024$ đến năm $2030$) với tốc độ tăng trưởng hàng năm là $1{,}26\%$. Ta giải bài toán theo các bước sau
	\begin{itemize}
		\item Bước 1: Sử dụng công thức tính dân số tăng trưởng theo cấp số nhân.\\
	Công thức tính dân số sau $n$ năm là
	\[P = P_0 \cdot (1 + r)^n.\]
	Trong đó: 
		\begin{itemize}
			\item $P_0$ là dân số ban đầu ($8{,}5$ triệu người),  
			\item $r$ là tốc độ tăng trưởng hàng năm ($1{,}26\% = 0{,}0126$),  
			\item $n$ là số năm ($n = 2030 - 2024 = 6$).  
		\end{itemize}
	\item Bước 2: Thay số vào công thức
	\[P = 8{,}5 \cdot (1 + 0{,}0126)^6 \approx 9{,}1545 \, (\text{triệu người}).\]
	\item Bước 4: Làm tròn đến hàng phần trăm
	Dân số năm $2030$ được làm tròn là $9{,}15$ triệu người.
\end{itemize}
	Dân số thủ đô Hà Nội vào năm $2030$ ước tính khoảng $9{,}15$ triệu người.
	}
\end{ex}
%%%==============HetBai_BT4==============%%%

%%%==============Bai_BT5==============%%%
% \begin{ex}%[1H4V1-5]
% 	Cho hình chóp $S. ABCD$ có đáy là hình bình hành tâm $O, SAB$ là tam giác đều cạnh $4, M, N$ theo thứ tự là trung điểm $SC, SD$. Hình tạo bởi các đoạn giao tuyến của $(OMN)$ và các mặt của hình chóp có diện tích $S$. Tính $S^2$.
	
% 	\shortans{27}
% 	\loigiai{
% 	\begin{center}
% 		\begin{tikzpicture}[declare function={gocx=80; goc=-150; a=5; b=a/2; h=4;}]
% 			\path (0,0) coordinate (A)--+(gocx:h) coordinate (S)
% 			(a,0) coordinate (B)
% 			(goc:b) coordinate (D)
% 			+(a,0) coordinate (C)
% 			($(A)!.5!(C)$) coordinate (O)
% 			($(S)!.5!(C)$) coordinate (M)
% 			($(S)!.5!(D)$) coordinate (N)
% 			($(B)!.5!(C)$) coordinate (Q)
% 			($(A)!.5!(D)$) coordinate (P)
% 			;
% 			\draw[dashed] (S)--(A) (D)--(A)--(B) (A)--(C) (B)--(D)
% 			(M)--(O)--(N)--(P)--(Q)
% 			;
% 			\draw (S)--(D)--(C)--(S)--(B)--(C)--cycle
% 			(Q)--(M)--(N) 
% 			;
% 			\foreach \x/\goc in {S/90,A/150,B/0,C/-60,D/-180,O/-90,M/0,N/180,P/180,Q/0}
% 			\draw[fill=black] (\x) node[shift={(\goc:7pt)},font=\scriptsize]{$\x$} circle (1pt);
% 		\end{tikzpicture}
% 	\end{center}	
% 	Ta có $\heva{&O\in (OMN)\cap (ABCD)\\&MN\subset (OMN)\\&CD\subset (ABCD)\\&MN\parallel CD}\Rightarrow (OMN)\cap (ABCD)=d$ với $\heva{&d~\text{qua}~O\\&d\parallel MN \parallel CD.} $\\
% 	Ta có 
% 	\begin{itemize}
% 		\item $(OMN)\cap  (SCD)=MN$;
% 		\item $(OMN)\cap  (SAD)=NP$;
% 		\item $(OMN)\cap  (ABCD)=PQ$;
% 		\item $(OMN)\cap  (SBC)=QM$.
% 	\end{itemize}
% 	Mặt cắt của hình chóp với $(OMN)$ là  $MNPQ$.\\
% 	Ta có $SA=AB=SB=4 $ suy ra $CD=4$.\\
% 	$MN=OP=OQ=2$.\\
% 	$NP=OM=\dfrac{1}{2}SA=2$.\\
% 	$ON=MQ=\dfrac{1}{2}SB=2$.\\
% 	Vậy $S=S_{\triangle ONP}+S_{\triangle OMN}+S_{\triangle OMQ}=3\cdot \dfrac{2^2\cdot \sqrt{3}}{4}=3\sqrt{3}$. \\
% 	Vậy $S^2=27$.
% }
% \end{ex}
%%%==============HetBai_BT5==============%%%

%%%==============Bai_BT6==============%%%
\begin{ex}%[1H4V4-4]
	Cho hình chóp $S. ABCD$ có đáy $ABCD$ là hình thang $(AB\parallel CD)$, $G$ là trọng tâm của tam giác $(SAB)$, mặt phẳng $(\alpha)$ qua $G$ và song song với $(SCD)$. Gọi $E$, $F$ theo thứ tự là giao điểm của $(\alpha)$ với $SA$, $AD$. Tính tỉ số $\dfrac{SD}{EF}$.
	
	\shortans{3}
	\loigiai{
	\begin{center}
		\begin{tikzpicture}[declare function={goc=-60;gocx=85;a=5; b=0.4*a; c=0.4*a; h=4;}] 
			\path (0,0) coordinate (A)
			(a,0) coordinate (B)
			(goc:b) coordinate (D)
			+(c,0) coordinate (C)
			(gocx:h) coordinate (S)
			($(A)!.5!(B)$) coordinate (M)
			($(S)!2/3!(M)$) coordinate (G)
			($(G)+(B)-(A)$) coordinate (x)
			(intersection of G--x and S--A) coordinate (E)
			(intersection of G--x and S--B) coordinate (H)
			($(E)+(D)-(S)$) coordinate (y)
			(intersection of E--y and A--D) coordinate (F)
			($(B)!1/3!(C)$) coordinate (K)
			;
			\draw[dashed] (A)--(B) (S)--(M) (E)--(H) (F)--(K);
			\draw (S)--(A)--(D) (S)--(D)--(C)--(S)--(B)--(C)--cycle (E)--(F) (H)--(K);
			\foreach \x/\goc in {S/90,A/-180,B/0,C/-60,D/-100,M/-90,G/30,E/180,F/180,H/30,K/0}
			\draw[fill=black] (\x) node[shift={(\goc:7pt)},font=\scriptsize]{$\x$} circle (1pt);
		\end{tikzpicture}
	\end{center}
Ta có $(\alpha)\parallel (SCD)$ suy ra $\heva{&(\alpha) \parallel CD\parallel AB\\&(\alpha)\parallel SD\\&(\alpha)\parallel SC.} $\\
Ta có $\heva{&(\alpha)\parallel SD\\&SD\subset (SAD)\\&(\alpha)\cap (SAD)=EF}\Rightarrow EF\parallel SD $.\\
$\heva{&(\alpha)\parallel AB\\&AB\subset (SAB)\\&(\alpha)\cap (SAB)=EH}\Rightarrow EH\parallel AB $.\\
$\heva{&(\alpha)\parallel AB\\&AB\subset (ABCD)\\&(\alpha)\cap (ABCD)=FK}\Rightarrow FK\parallel AB $.\\
$\heva{&(\alpha)\parallel SC\\&SC\subset (SBC)\\&(\alpha)\cap (SBC)=KH}\Rightarrow KH\parallel SC $.\\
Vì $EG\parallel AM$ nên $\dfrac{SE}{SA}=\dfrac{SG}{SM}=\dfrac{2}{3}$.\\
Vì $EF\parallel SD$ nên $\dfrac{SD}{EF}=\dfrac{SA}{AE}=3$.
}
\end{ex}
%%%==============HetBai_BT6==============%%%

\begin{ex}%[Mức độ 3]giảng K10-11, Nguyễn Đắc Giáp]%[1D3V3-4]
Cho hàm số $f(x)=\heva{
&\dfrac{3-\sqrt{9-x}}{x} &&\text{khi } 0<x\leq 9\\
&m &&\text{khi } x\leq 0}$. Giá trị của tham số $m$ để hàm số liên tục trên $[0;9]$ bằng (làm tròn đến hàng phần trăm)
\shortans{$0{,}17$}
\loigiai{
\begin{itemize}
\item Với $0<x<9$ thì $f(x)=\dfrac{3-\sqrt{9-x}}{x}$ hàm số xác định và liên tục.
\item $\lim\limits_{x\to {9^-}}f(x)=\lim\limits_{x\to {9^-}}{\dfrac{3-\sqrt{9-x}}{x}}=\dfrac{1}{3}=f(9)$.
\item Xét tại $x=0$ ta có $f(0)=m=\lim\limits_{x\to {0^-}}f(x)$.\\
$\lim\limits_{x\to {0^+}}f(x)=\lim\limits_{x\to {0^+}}{\dfrac{3-\sqrt{9-x}}{x}}=\lim\limits_{x\to 0^+}{\dfrac{1}{3+\sqrt{9-x}}}=\dfrac{1}{6}$.
\end{itemize}
Để hàm số liên tục trên $[0;9]$ thì $f(0)=\dfrac{1}{6}$ hay $m=\dfrac{1}{6}\approx 0{,}17$.
}
\end{ex}
\Closesolutionfile{ans}
% \indapan{7}{ans/11-CK1-Chuyen-Hung-Vuong-Phan-3}
\TL

\begin{ex}%[1D3H2-3]
Tính các giới hạn sau:
\begin{enumerate}
	\item $\lim\limits_{x \to  2}\dfrac{x^2+3 x-10}{x^2+2 x-3}$
	\item $\lim\limits_{x\to 1} \dfrac{\sqrt{x^2-3x+6}-x-1}{5x-5}$
\end{enumerate} 
\loigiai{
	\begin{enumerate}
		\item $ \lim\limits_{x \to  2}\dfrac{x^2+3 x-10}{x^2-3 x+2}
=\lim\limits_{x \to  2}\dfrac{(x-2)(x+5)}{(x-2)(x-1)}
=\lim\limits_{x \to  2}\dfrac{x+5}{x-1}
=\dfrac{2+5}{2-1}
=7.$
\item $\begin{aligned}[t] \lim\limits_{x\to 1} \dfrac{\sqrt{x^2-3x+6}-x-1}{5x-5} &= \lim\limits_{x\to 1} \dfrac{(x^2-3x+6)-(x+1)^2}{(5x-5)(\sqrt{x^2-3x+6}+x+1)} = \lim\limits_{x\to 1} \dfrac{-5x+5}{(5x-5)(\sqrt{x^2-3x+6}+x+1)} \\ &= \lim\limits_{x\to 1} \dfrac{-1}{\sqrt{x^2-3x+6}+x+1} = \dfrac{-1}{\sqrt{4}+2} = -\dfrac{1}{4} \end{aligned}$.
	\end{enumerate}
}
\end{ex}

\begin{ex}%[1D2V3-8]
	\immini{Cho tam giác $OMN$ vuông cân tại $O$, $OM=ON=2$. Trong tam giác $OMN$, vẽ hình vuông $O A_1 B_1 C_1$ sao cho các đỉnh $A_1$, $B_1$, $C_1$ lần lượt nằm trên các cạnh $OM$, $MN$, $ON$ (Hình bên). Trong tam giác $A_1 M B_1$, vẽ hình vuông $A_1 A_2 B_2 C_2$ sao cho các đỉnh $A_2$, $B_2$, $C_2$ lần lượt nằm trên các cạnh $A_1 M$, $M B_1$, $A_1 B_1$. Tiếp tục quá trình đó, ta được một dãy các hình vuông. Tính tổng diện tích các hình vuông này.}
	{\begin{tikzpicture}[scale=1, font=\footnotesize, line join=round, line cap=round, >=stealth]
			\path
			(0,0) coordinate (O)
			(4,0) coordinate (M)
			(0,4) coordinate (N)
			($(O)!0.5!(M)$) coordinate (A_1)
			(A_1)++(90:2) coordinate (B_1)
			($(O)!0.5!(N)$) coordinate (C_1)
			($(A_1)!0.5!(M)$) coordinate (A_2)
			(A_2)++(90:1) coordinate (B_2)
			($(A_1)!0.5!(B_1)$) coordinate (C_2)
			($(A_2)!0.5!(M)$) coordinate (A_3)
			(A_3)++(90:0.5) coordinate (B_3)
			($(A_2)!0.5!(B_2)$) coordinate (C_3);
			\draw (N)--(O)--(M)--(N) (A_1)--(B_1)--(C_1) (A_2)--(B_2)--(C_2) (A_3)--(B_3)--(C_3);
			\path (A_3)--(M)node[midway,above]{$\ldots$};
			\pic[draw,angle radius=1.5mm,angle eccentricity=1.5] {right angle = N--O--M};
			\foreach \p/\g in {A_1/-90,A_2/-90,A_3/-90,B_1/30,B_2/30,B_3/30,C_1/180,C_2/180,C_3/180,M/0,N/90,O/-135}\draw[fill=black] (\p) circle (1pt)node[shift={(\g:.3)}]{$\p$};
		\end{tikzpicture}}
	\loigiai{
		Độ dài cạnh của các hình vuông lần lượt là
		\[
			OA_1=1;\ A_1A_2=\dfrac12 OA_1;\ A_2A_3=\dfrac12 A_1A_2;\ldots
		\]
		Đặt $S_1$ là diện tích hình vuông $OA_1B_1C_1$, $S_n$ là diện tích hình vuông $A_{n-1}A_nB_nC_n$ với $n\ge 2$. Diện tích của các hình vuông lần lượt là
		\[
			\begin{aligned}
				 & S_1=OA_1^2=1^2=1,                                                                                                                   \\
				 & S_2=A_1A_2^2=\dfrac14 S_1                                                                                                           \\
				 & S_3=a_3^2=\left[\left(\dfrac{1}{2}\right)^3\right]^2=\left[\left(\dfrac{1}{2}\right)^2\right]^3=\left(\dfrac{1}{4}\right)^3, \ldots
			\end{aligned}
		\]
		Các diện tích $S_1$, $S_2$, $S_3$, $\ldots$ tạo thành cấp số nhân lùi vô hạn với số hạng đầu là $S_1=\dfrac{1}{4}$ và công bội bằng $\dfrac{1}{4}$.
		Do đó, tổng diện tích các hình vuông là $S=\dfrac{1}{4} \cdot \dfrac{1}{1-\dfrac{1}{4}}=\dfrac{1}{3}$.
	}
\end{ex}

\begin{ex}%[1H4C4-6]
	Cho hình chóp $S.ABCD$, đáy là hình bình hành tâm $O$. Gọi $M$, $N$ lần lượt là trung điểm của $SA$ và $CD$.
	\begin{enumerate}
		\item Tìm giao điểm của đường thẳng $SD$ và mặt phẳng $(OMN)$.
		\item Tính diện tích thiết diện mặt phẳng $(OMN)$ với hình chóp $S.ABCD$ biết mặt bên $SBC$ là tam giác đều cạnh $a$.
	\end{enumerate}
	\loigiai{
		\begin{center}
			\begin{tikzpicture}[line join = round, line cap = round, thick, font = \footnotesize, scale = 1]
				\path
				(0:0) coordinate (A)
				+(0:5) coordinate (B)
				+(-140:2.5) coordinate (D)
				+(80:4) coordinate (S)
				($(B)+(D)-(A)$) coordinate (C);
				\path (intersection of A--C and B--D) coordinate (O);
				\path ($(A)!.5!(S)$) coordinate (M);
				\path ($(C)!.5!(D)$) coordinate (N);
				\path ($(S)!.5!(D)$) coordinate (I);
				\path ($(A)!.5!(B)$) coordinate (E);
				\draw[dashed]
				(A)--(B) (A)--(D) (A)--(S);
				\draw[dashed] (O)--(M)--(N)--(E)--(M) (C)--(A) (I)--(M);
				\draw
				(D)--(C)--(B) (I)--(N)
				(S)--(B) (S)--(C) (S)--(D);
				\foreach \x/\g in {A/180,B/0,C/-45,D/-135,S/90,M/30,N/-90,I/150,E/45,O/-75}
				\fill (\x) circle (1.5pt)
				+(\g:3.5mm) node {$\x$};
			\end{tikzpicture}
		\end{center}
		\begin{enumerate}
			\item Gọi $I$ là trung điểm $SD$. Suy ra $MI$ là đường trung bình của tam giác $SAD$, nên $MI \parallel AD$.\\
			      Mặt khác $ON \parallel AD$ (vì $O$, $N$ lần lượt là trung điểm của $AC$, $CD$).\\
			      Do đó, $MI \parallel ON$.\\
				  Do vậy $MI \subset (OMN)$.\\
			      Vậy giao tuyến của đường thẳng $SD$ và mặt phẳng $(OMN)$ là $I$.
			\item $(OMN) \cap(ABCD)=ON$. Cho $ON \cap AB=E$.\\
			      $(OMN) \cap(SAB)=ME$.\\
			      $(OMN) \cap(SAD)=MI$. Do $MI \parallel AD \parallel ON$.\\
			      $(OMN) \cap(SCD)=NI$.\\
			      Các giao tuyến trên tạo ra tứ giác $MINE$. Vì $MI \parallel NE$ nên tứ giác $MINE$ là hình thang.
				  Ta có tam giác $SBC$ đều cạnh $a$ nên có diện tích $S_{SBC}=\dfrac{a^2\sqrt{3}}{4}$.\\
			      Lại có hai tam giác $MOE$ và $SBC$ đồng dạng nhau (vì $\dfrac{OM}{SC}=\dfrac{OE}{BC}=\dfrac{ME}{SB}=\dfrac{1}{2}$) nên tam giác $MOE$ cũng là tam giác đều và có cạnh bằng $\dfrac{a}{2}$. Do đó tam giác $MOE$ có diện tích $S_{MOE}=\dfrac{1}{4}S_{SBC}=\dfrac{a^2\sqrt{3}}{16}$.\\
			      Mặt khác, $ABCD$ là hình bình hành nên $NE=AD=a$ và $IM= \dfrac{1}{2}AD$ nên $S_{MINE}=3S_{MOE}=\dfrac{3a^2\sqrt{3}}{16}$.\\
			      Vậy diện tích thiết diện mặt phẳng $(OMN)$ với hình chóp $S.ABCD$ là $\dfrac{3a^2\sqrt{3}}{16}$.
		\end{enumerate}
	}
\end{ex}
