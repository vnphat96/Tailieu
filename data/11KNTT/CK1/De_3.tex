\begin{name}
	{\tenchude}
	{\tendethi}
	{\tentruong}
	{\thoigian}
\end{name}
\setcounter{ex}{0}\setcounter{bt}{0}
\begin{ex}%[1D2H2-3]%[Dự án đề kiểm tra Toán 11 GHKI NH23-24- Huỳnh Quy]%[THPT Tân Bình - Tp HCM]
	Cấp số cộng $(u_n)$ có số hạng đầu tiên $u_1=2$ và công sai $d=3$. Số hạng $u_3$ bằng
	\choice
	{$6$}
	{\True $8$}
	{$10$}
	{$9$}
	\loigiai{
		Ta có $u_3=u_1+2d=2+2\cdot 3=8$.
	}
\end{ex}

\begin{ex}%[1D3N1-2]%[Dự án đề kiểm tra Toán 11 HKI NH23-24- Tư Đô Nguyên]%[Đề 11 - CTST]
	$\lim\dfrac{1}{n^3}$ bằng
	\choice
	{\True $0$}
	{$2$}
	{$4$}
	{$5$}
	\loigiai{Ta có $\lim\dfrac{1}{n^3}=0$.}
\end{ex}

\begin{ex}%[1D2H2-6] %[Dự án 11 HVA 2024-2025]%[Trần Hưng]
	Cho cấp số cộng $(u_n)$ với $u_2=3$ và $u_5=12$. Giá trị $759$ là tổng của bao nhiêu số hạng đầu của cấp số cộng?
	\choice
	{\True $23$}
	{$25$}
	{$17$}
	{$27$}
	\loigiai{
		Ta có
		$\heva{&u_2=u_1+d=3\\&u_5=u_1+4d=12}\Leftrightarrow \heva{&u_1=0\\&d=3}$.\\
		Gọi tổng của $n$, $(n>0, n\in\mathbb{N})$ số hạng đầu bằng $759$ suy ra
		\begin{eqnarray*}
			&&S_n=\dfrac{n\left[2u_1+(n-1)d\right]}{2}=759\\
			&\Leftrightarrow& \dfrac{n[0+(n-1)\cdot 3]}{2}=759\\
			&\Leftrightarrow& n(n-1)=506\\
			&\Leftrightarrow& \hoac{&n=-22\quad \text{(loại)}\\&n=23.}
		\end{eqnarray*}
	}
\end{ex}

\begin{ex}%[1-TK-HK1-KN1-2424]%[VN-MT-9,Nguyễn Tuấn]%[1H4N2-1]%[1]
	Cho hai đường thẳng phân biệt $a$ và $b$ trong không gian. Có bao nhiêu vị trí tương đối giữa $a$ và $b$?
	\choice
	{$4$}
	{$2$}
	{\True $3$}
	{$1$}
	\loigiai{
		Vì hai đường thẳng $a$ và $b$ phân biệt nên hai đường thẳng có $3$ vị trí tương đối: cắt nhau, song song, chéo nhau.}
\end{ex}

\begin{ex}%[1D1N4-2]
	Tập $D=\mathbb{R}\backslash \left\{ \dfrac{k\pi }{2}\middle| k\in \mathbb{Z} \right\}$ là tập xác định của hàm số nào sau đây?
	\choice
	{ $y=\cot x$}
	{\True  $y=\cot 2x$}
	{ $y=\tan x$}
	{ $y=\tan 2x$}
	\loigiai{Hàm số $y=\cot 2x$ xác định khi $2x\ne k\pi $ $\Leftrightarrow x\ne \dfrac{k\pi }{2}$.}
\end{ex}

\begin{ex}%[1D3N2-1]
	Giả sử $\lim\limits_{x \rightarrow x_0} f(x)=L$, $\lim\limits_{x \rightarrow x_0} g(x)=L\, (L, M \in \mathbb{R})$. Chọn đáp án \textbf{sai}.
	\choice
	{$\lim\limits_{x \rightarrow x_0}[f(x)+g(x)]=L+M$}
	{$\lim\limits_{x \rightarrow x_0}[f(x)-g(x)]=L-M$}
	{$\lim\limits_{x \rightarrow x_0}[f(x)\cdot g(x)]=L\cdot M$}
	{\True $\lim\limits_{x \rightarrow x_0} \dfrac{f(x)}{g(x)}=\dfrac{L}{M}$}
	\loigiai{Ta có $\lim\limits_{x \rightarrow x_0} \dfrac{f(x)}{g(x)}=\dfrac{L}{M}$ (nếu $M \neq 0$).
	}
\end{ex}

\begin{ex}%[1D1N3-1]
	Trong các công thức sau, công thức nào đúng?
	\choice
	{\True $\sin 2\alpha=2\sin \alpha \cos \alpha$}
	{$\sin 2\alpha=2\sin \alpha$}
	{$\sin 2\alpha=\sin \alpha+\cos \alpha$}
	{$\sin 2\alpha=\cos^2\alpha-\sin^2\alpha$}
	\loigiai{

	}
\end{ex}

\begin{ex}%[Pj08-0-GK1-NH23-24--TeamTeXHoa--LeQuan]%[1D1H2-2]
	Cho $\sin\alpha=\dfrac{1}{3}$. Giá trị của $\cos 2\alpha $ bằng
	\choice
	{$\dfrac{2\sqrt 2}{3}$}
	{$-\dfrac{2\sqrt 2}{3}$}
	{\True $\dfrac{7}{9}$}
	{$-\dfrac{7}{9}$}
	\loigiai{
		Ta có $\cos 2\alpha=1-2\sin ^2\alpha=1-2\left(\dfrac{1}{3}\right)^2=\dfrac{7}{9}$.}
\end{ex}

\begin{ex}%[Mức độ 1]%[TVN005]%[BG K11, Nguyễn Văn Nay]%[1H4N4-2]
	Cho hình hộp $ABCD.A'B'C'D'$. Mệnh đề nào sau đây là mệnh đề \textbf{sai}?
	\choice
	{$(BA'C')\parallel (ACD')$}
	{$(ADD'A')\parallel (BCC'B')$}
	{$(BA'D)\parallel (CB'D')$}
	{\True $(ABA')\parallel (CB'D')$}
	\loigiai{
		\immini{
			Ta có
			$\heva{&BA'\parallel CD'\\&A'C'\parallel AC}
				\Rightarrow (BA'C')\parallel (ACD')$. \\
			$\heva{&AD\parallel BC\\&AA'\parallel BB'}
				\Rightarrow (ADD'A')\parallel (BCC'B')$.\\
			$\heva{&BD\parallel B'D'\\&A'D\parallel B'C}
				\Rightarrow (BA'D)\parallel (CB'D')$.
		}
		{
			\begin{tikzpicture}[line cap=round,line join=round,>=stealth,scale=1]
				\coordinate[label=left:{$A$}] (A) at (0,0);
				\coordinate[label=left:{$D$}] (D) at (0.8,1);
				\coordinate[label=below:{$B$}] (B) at ($(A)+(3,0)$);
				\coordinate[label=left:{$A'$}] (A') at ($(A)+(-0.4,2.5)$);
				\coordinate[label=right:{$C$}] (C) at ($(B)+(D)-(A)$);
				\coordinate[label=right:{$B'$}] (B') at ($(B)+(A')-(A)$);
				\coordinate[label=right:{$C'$}] (C') at ($(C)+(A')-(A)$);
				\coordinate[label={$D'$}] (D') at ($(D)+(A')-(A)$);
				\draw (A')--(A)--(B)--(C)--(C')--(D')--(A')--(B')--(C') (B')--(B);
				\draw[dashed] (D')--(D)--(A) (C)--(D);
			\end{tikzpicture}
		}
		\noindent Mặt khác $B'\in (ABA')\cap (CB'D') \Rightarrow $ $(ABA')\parallel (CB'D')$ là mệnh đề sai.
	}
\end{ex}

\begin{ex}%[1H4H1-3]%[Dự án đề ôn tập Toán Khối 11 HK1 NH23-24-Dot 1-Vương Quốc Phong]%[CTST - Đề số 3]
	Cho hình chóp $S.ABCD$ có đáy là hình thang $ABCD$ ($AB \cap CD = O$). Khẳng định nào sau đây \textbf{sai}?
	\choice
	{Hình chóp $S.ABCD$ có $4$ mặt bên}
	{Giao tuyến của hai mặt phẳng $(SAC)$ và $(SBD)$ là $SO$}
	{Giao tuyến của hai mặt phẳng $(SAD)$ và $(SBC)$ là $SI$ ($I$ là giao điểm của $AD$ và $BC$)}
	{\True Giao tuyến của hai mặt phẳng $(SAB)$ và $(SAD)$ là đường trung bình của $ABCD$}
	\loigiai{
		Ta có $(SAB) \cap (SAD) = SA$ và $SA$ không thể là đường trung bình của hình thang $ABCD$.
	}
\end{ex}

\begin{ex}%[Pj08-0-GK1-NH23-24--TeamTeXHoa--LeQuan]%[1D2H3-3]
	Cho cấp số nhân có các số hạng lần lượt là $ 3;\, 9;\, 27;\, 81;\ldots$. Tìm số hạng tổng quát $u_n$ của cấp số nhân đã cho.
	\choice
	{$u_n=3^{n-1}$}
	{\True $u_n=3^n$}
	{$u_n=3^{n+1}$}
	{$u_n=3+3^n$}
	\loigiai{
		Dãy số  $ 3;\, 9;\, 27;\, 81;\ldots$ là một cấp số nhân có $u_1=3$ và công bội $q=3$.\\
		$\Rightarrow u_n=u_1\cdot q^{n-1}=3\cdot 3^{n-1}=3^n$.
	}
\end{ex}

\begin{ex}%[Pj08-0-GK1-NH23-24--TeamTeXHoa--LeQuan]%[1D2N1-4]
	Trong các dãy số có số hạng tổng quát sau, dãy số nào {\bf không} là dãy số tăng, cũng {\bf không} là dãy số giảm?
	\choice
	{$u_n=n$}
	{$v_n=2n$}
	{$x_n=\dfrac{1}{n}$}
	{\True $w_n=\dfrac{\left(-1\right)^n}{n}$}
	\loigiai{
		\begin{itemize}
			\item Xét  $u_n=n\Rightarrow u_{n+1}=n+1\Rightarrow u_{n+1}-u_n=1>0\,\forall n\in \mathbb{N}^*$.\\
			      Vậy dãy số $(u_n)$ là dãy số tăng.
			\item Xét $v_n=2n\Rightarrow v_{n+1}=2(n+1)\Rightarrow v_{n+1}-v_n=21>0\,\forall n\in \mathbb{N}^*$.\\
			      Vậy dãy số $(v_n)$ là dãy số tăng.
			\item Xét $x_n=\dfrac{1}{n}\Rightarrow x_{n+1}=\dfrac{1}{n+1}\Rightarrow x_{n+1}-x_n<0 1>0\,\forall n\in \mathbb{N}^*$.\\
			      Vậy dãy số $(x_n)$ là dãy số giảm.
			\item Xét $w_n=\dfrac{(-1)^n}{n}$ có $w_1=-1$, $w_2=\dfrac{1}{2}$, $w_3=-\dfrac{1}{3}$.\\
			      Suy ra $ \heva{&w_1<w_2 \\&w_2>w_3}$ nên dãy $(w_n)$ không phải là dãy tăng, cũng không phải là dãy giảm.
		\end{itemize}
	}
\end{ex}

\begin{ex} %[1D2N3-1]
	Trong các dãy số sau, dãy số nào không phải là một cấp số nhân?
	\choice{$2;4;8;16;\ldots$}
	{$1;-1;1;-1;\ldots$}
	{\True $1^2;2^2;3^2; 4^2;\ldots$}
	{$a;a^3;a^5;a^7;\ldots \; \left(a\ne 0\right)$}
	\loigiai{
		Dãy $1^2;2^2;3^2; 4^2;\cdots$ không phải là một cấp số nhân.
	}
\end{ex}

\begin{ex}%[1H4H2-2]
	Cho hình chóp $S.ABCD$, đáy $ABCD$ là hình bình hành. Điểm $M$ thuộc cạnh $SC$, $N$ là giao điểm của $SD$ và $\left(MAB \right)$. Khi đó, hai đường thẳng $CD$ và $MN$ là hai đường thẳng
	\choice
	{Cắt nhau}
	{\True Song song}
	{Chéo nhau}
	{Có hai điểm chung}
	\loigiai{
		\begin{center}
			\begin{tikzpicture}[scale=1,line join=round, line cap=round, >=stealth,font=\footnotesize]
				\def\a{4} \def\b{3.5}
				\path
				(0,0) coordinate (D)
				(\a,0) coordinate (C)
				({\a+1.5},1.5) coordinate (B)
				(1.5,1.5) coordinate (A)
				(2,\b) coordinate (S)
				($(S)!0.7!(C)$) coordinate (M)
				($(S)!0.7!(D)$) coordinate (N)
				;

				\draw
				(D) -- (C) -- (B) -- (S) -- (D)
				(S) -- (C)
				(B) -- (M) -- (N) -- ([turn]0:1cm) node[above]{$x$}
				;

				\draw[dashed]
				(B) -- (A) -- (D)
				(S) -- (A) -- (M)
				;
				\fill
				(A) circle (1pt) node[left]{$A$}
				(B) circle (1pt) node[right]{$B$}
				(C) circle (1pt) node[below]{$C$}
				(D) circle (1pt) node[below]{$D$}
				(M) circle (1pt) node[below left]{$M$}
				(N) circle (1pt) node[above left]{$N$}
				(S) circle (1pt) node[above]{$S$}
				;
			\end{tikzpicture}
		\end{center}
		Ta có $MN$ là giao tuyến của hai mặt phẳng $\left(MAB \right)$ và $\left(SCD \right)$.\\
		Mặt khác $\heva{&AB \subset \left(MAB \right)\\&CD \subset \left(SCD \right)\\&AB \parallel CD} \Rightarrow MN \parallel CD$.
		Vậy $MN$ song song với $CD$.
	}
\end{ex}

\begin{ex}%[1H4H1-2]
	Hình lăng trụ lục giác có bao nhiêu mặt bên?
	\choice
	{$5$}
	{\True $6$}
	{$3$}
	{$4$}
	\loigiai{
	Hình lăng trụ lục giác có $6$ mặt bên.
	}
	\end{ex}

\begin{ex}%[1D3N3-1]
	Cho hàm số $y=f\left(x\right)$ liên tục trên $(a;b)$. Điều kiện cần và đủ để hàm số liên tục trên $\left[a; b\right]$ là
	\choice
	{${\mathop{\lim}\limits_{x\to a^{+}}} f\left(x\right)=f\left(a\right)$ và ${\mathop{\lim}\limits_{x\to b^{+}}} f\left(x\right)=f\left(b\right)$}
	{${\mathop{\lim}\limits_{x\to a^{-}}} f\left(x\right)=f\left(a\right)$ và ${\mathop{\lim}\limits_{x\to b^{-}}} f\left(x\right)=f\left(b\right)$}
	{\True ${\mathop{\lim}\limits_{x\to a^{+}}} f\left(x\right)=f\left(a\right)$ và ${\mathop{\lim}\limits_{x\to b^{-}}} f\left(x\right)=f\left(b\right)$}
	{${\mathop{\lim}\limits_{x\to a^{-}}} f\left(x\right)=f\left(a\right)$ và ${\mathop{\lim}\limits_{x\to b^{+}}} f\left(x\right)=f\left(b\right)$}
	\loigiai{
		Theo định nghĩa hàm số liên tục trên đoạn $\left[a; b\right]$ nếu
		hàm số $y=f\left(x\right)$ liên tục trên $(a;b)$ và ${\mathop{\lim}\limits_{x\to a^{+}}} f\left(x\right)=f\left(a\right)$ và ${\mathop{\lim}\limits_{x\to b^{-}}} f\left(x\right)=f\left(b\right)$.

	}
\end{ex}

\begin{ex}%[Pj10-Đề 07 HK1 NH2023-2024 CTST]%[Nguyễn Văn Nay]%[1H4N6-1]
	Cho các đường thẳng không song song với phương chiếu. Khẳng định nào sau đây là đúng?
	\choice
	{Phép chiếu song song biến hai đường thẳng song song thành hai đường thẳng song song}
	{Phép chiếu song song có thể biến hai đường thẳng song song thành hai đường thẳng cắt nhau}
	{Phép chiếu song song có thể biến hai đường thẳng song song thành hai đường thẳng chéo nhau}
	{\True Phép chiếu song song biến hai đường thẳng song song thành hai đường thẳng song song hoặc trùng nhau}
	\loigiai{
		Theo tính chất của phép chiếu song song, phép chiếu song song biến hai đường thẳng song song thành hai đường thẳng song song hoặc trùng nhau.
	}
\end{ex}

\begin{ex}%[1D3H1-2]%[Dự án EX-CK1 (10,11 mới)- Lâm Chính]
	Tìm giới hạn $\lim \limits{n \to +\infty}\dfrac{2^{n+1}+4^n}{3^n+4^{n+1}}$.
	\choice
	{$\dfrac{1}{2}$}
	{\True $\dfrac{1}{4}$}
	{$0$}
	{$+\infty$}
	\loigiai{
	Ta có $\lim \limits{n \to +\infty}\dfrac{2^{n+1}+4^n}{3^n+4^{n+1}}=\lim \limits{n \to +\infty}\dfrac{2\cdot2^n+4^n}{3^n+4\cdot4^n}=\lim \limits{n \to +\infty}\dfrac{2\cdot\left(\dfrac{2}{4} \right)^n+1}{\left(\dfrac{3}{4} \right)^n+4}=\dfrac{2\cdot0+1}{0+4}=\dfrac{1}{4}$.

	}
\end{ex}

\begin{ex}%[1D3H2-2]
	Cho $\lim\limits u_n=-3$, $\lim\limits v_n=2$. Khi đó $\lim\limits \left(u_n-v_n\right)$ bằng
	\choice
	{\True $-5$}
	{$-1$}
	{$5$}
	{$1$}
	\loigiai{
		Ta có $\lim\limits \left(u_n-v_n\right)=\lim \limits{n \to +\infty}u_n-\lim \limits{n \to +\infty}v_n
			=-3-2=-5$.}
\end{ex}

\begin{ex}%[1D1N5-1]
	Phương trình $\sin x=\sin \alpha $ có tập nghiệm là:
	\choice
	{ $S=\left\{ \alpha +k2\pi |k\in \mathbb{Z} \right\}$}
	{ $S=\left\{ \alpha +k\pi |k\in \mathbb{Z} \right\}$}
	{ $S=\left\{ \alpha +k2\pi;-\alpha +k2\pi |k\in \mathbb{Z} \right\}$}
	{\True  $S=\left\{ \alpha +k2\pi;\pi -\alpha +k2\pi |k\in \mathbb{Z} \right\}$}
	\loigiai{
		$
			\begin{array}{llll}          & \sin x=\sin \alpha                                          \\
             \Leftrightarrow & \hoac{                                    & x=\alpha +k2\pi \\
                             & x= \pi -\alpha +k2\pi},\,k\in \mathbb{Z}.                   \\
			\end{array}$}
\end{ex}

\begin{ex}%[1D5H1-4]
	Người ta ghi lại tuổi thọ của một số con muỗi cái trong phòng thí nghiệm cho kết quả như sau
	\begin{center}
		\begin{tabular}{|c|c|c|c|c|c|}
			\hline
			Tuổi thọ (ngày) & $[0;20)$ & $[20;40)$ & $[40;60)$ & $[60;80)$ & $[80;100)$ \\
			\hline
			Số lượng        & $5$      & $12$      & $23$      & $31$      & $29$       \\ \hline
		\end{tabular}
	\end{center}
	Muỗi cái có tuổi thọ khoảng bao nhiêu ngày là nhiều nhất?
	\choice
	{$80$ ngày}
	{$66$ ngày}
	{\True $76$ ngày}
	{$96$ ngày}
	\loigiai{
	Nhóm chứa mốt của mẫu số liệu ghép nhóm trên là nhóm 4: $[60;80)$.\\
	Do đó $u_4=60$, $n_4=31$, $n_3=23$, $n_5=29$ $u_5-u_4=80-60=20$.\\
	Vậy mốt của mẫu số liệu trên là
	\allowdisplaybreaks
	\begin{eqnarray*}
		M_e &=& u_4+\dfrac{n_4-n_3}{(n_4-n_3)+(n_4-n_5)}\cdot (u_5-u_4)\\
		&=& 60+\dfrac{31-23}{(31-23)+(31-29)}\cdot (20)=76.
	\end{eqnarray*}
	Muỗi cái có tuổi thọ nhiều nhất là $76$ ngày.
	}
\end{ex}

\begin{ex} %[1D3N1-1]
	Khẳng định nào sau đây là đúng?
	\choice
	{Ta nói dãy số $\left(u_n \right)$ có giới hạn là số $a$ (hay $u_n$ dần tới $a$) khi $n \to +\infty$, nếu $\underset{x \to +\infty}{\lim}{\left(u_n + a \right)} = 0$}
	{Ta nói dãy số $\left(u_n \right)$ có giới hạn là $0$ khi $n$ dần tới vô cực, nếu $\left|u_n \right|$ có thể lớn hơn một số dương tùy ý, kể từ một số hạng nào đó trở đi}
	{Ta nói dãy số $\left( {{u}_{n}} \right)$ có giới hạn $+\infty $ khi $n\to +\infty $ nếu ${{u}_{n}}$ có thể nhỏ hơn một số dương bất kì, kể từ một số hạng nào đó trở đi}
	{\True Ta nói dãy số $\left( {{u}_{n}} \right)$ có giới hạn $+\infty $ khi $n\to +\infty $ nếu ${{u}_{n}}$ có thể lớn hơn một số dương bất kì, kể từ một số hạng nào đó trở đi}
	\loigiai{
		Ta nói dãy số $\left( {{u}_{n}} \right)$ có giới hạn $+\infty $ khi $n\to +\infty $ nếu ${{u}_{n}}$ có thể lớn hơn một số dương bất kì, kể từ một số hạng nào đó trở đi}
\end{ex}

\begin{ex}%[1D5H1-2]%[Dự án 11 HVA 2024-2025]%[Bồ Văn Hậu]
	Thời gian đề học sinh hoàn thành một câu hỏi thi được cho như sau:
	\begin{center}
		\begin{tabular}{|l|c|c|c|c|c|}
			\hline
			Thời gian (phút) & {$[0{,}5; 10{,}5)$} & {$[10{,}5; 20{,}5)$} & {$[20{,}5; 30{,}5)$} & {$[30{,}5; 40{,}5)$} & {$[40{,}5; 50{,}5)$} \\
			\hline
			Só học sinh      & $2$                 & $10$                 & $6$                  & $4$                  & $3$                  \\
			\hline
		\end{tabular}
	\end{center}
	Tìm mốt của mẫu số liệu ghép nhóm này.
	\choice
	{$17{,}42$}
	{$14{,}56$}
	{\True $17{,}16$}
	{$12{,}67$}
	\loigiai{
	Tần số lớn nhất là $10$ nên nhóm chứa mốt là nhóm $[10{,}5; 20{,}5]$.\\
	Ta có $u_m=10{,}5$; $u_{m+1}=20{,}5$; $n_m=10$; $n_{m+1}=6$; $n_{m-1}=2$; $u_{m+1}-u_m=10$.\\
	Do đó $M_0=10{,}5+\dfrac{10-2}{(10-2)+(10-6)} \cdot 10=17{,}16$.
	}
\end{ex}

\begin{ex}%[1H4H6-3]%[KNTT - Lớp 11 - Ôn tập cuối học kì 1 - Đề 6]%[Tex hóa: Lê Thị Thúy Hằng]
	Cho hình lăng trụ $ABC.A'B'C'$. Gọi $I$, $I'$ lần lượt là trung điểm của $AB$, $A'B'$. Qua phép chiếu song song theo phương $AI'$, mặt phẳng chiếu $(A'B'C')$ biến $I$ thành điểm nào?
	\choice
	{$A'$}
	{\True $B'$}
	{$C'$}
	{$I'$}
	\loigiai{
		\begin{center}
			\begin{tikzpicture}[scale=0.6,>=stealth,font=\footnotesize]
				\path
				(0:0) coordinate (A)
				++(0:4) coordinate (B)
				++(-160:3) coordinate (C)
				\foreach \x in {A,B,C}{(\x)++(-110:5) coordinate (\x')}
				($(A)!0.5!(B)$) coordinate (I)
				($(A')!0.5!(B')$) coordinate (I')
				;
				\draw (A)--(C)--(C')--(A')--cycle
				(C)--(B)--(B')--(C') (A)--(B);
				\draw[dashed] (A')--(B')--(I) (A)--(I');
				\foreach \x/\g in {A/150,B/0,C/-30,A'/180,B'/0,C'/-90,I/90,I'/40}
				\draw[fill=black] (\x) circle (0.02) +(\g:0.4)
				node{$\x$};
			\end{tikzpicture}
		\end{center}
		Ta có $AI \parallel B'I'$ và $AI = B'I'$ nên $AIB'I'$ là hình bình hành. Suy ra qua phép chiếu song song theo phương $AI'$, mặt phẳng chiếu $(A'B'C')$ biến điểm $I$ thành $B'$.
	}
\end{ex}

\begin{ex}%[Team Tex Hóa -- Dương Văn Đức]%[1D3V3-3]
	Tìm $m$ để hàm số $f(x)=\heva{&\dfrac{x^2-1}{x-1}& \text{khi~} x\ne 1\\& m+2& \text{khi~}x=1}$ liên tục tại điểm $x_0=1$.
	\choice
	{$m=3$}
	{\True $m=0$}
	{$m=4$}
	{$m=1$}
	\loigiai{
		Ta có $\lim \limits_{x\to1}\dfrac{x^2-1}{x-1} =\lim \limits_{x\to1}\dfrac{\left(x-1\right)\left(x+1\right)}{\left(x-1\right)} =\lim \limits_{x\to1}\left(x+1\right)=2$.\\
		Để hàm số liên tục tại $x_0=1$ khi và chỉ khi $\lim \limits_{x\to 1} f(x)=f(1)\Leftrightarrow 2=m+2\Leftrightarrow m=0$.}
\end{ex}

\begin{ex}%[1D5N1-2]%[Dự án 11 HVA 2024-2025]%[Đoàn Hiển]
	Cho bảng khảo sát về cân nặng học sinh trong lớp.
	\begin{table}[h]
		\centering
		\begin{tabular}{|c|c|c|c|c|c|c|}
			\hline
			Cân nặng (kg) & [45;50) & [50;55) & [55;60) & [60;65) & [65;70) \\
			\hline
			Số học sinh   & $2$     & $14$    & $11$    & $10$    & $3$     \\  \hline
		\end{tabular}
	\end{table}\\
	Khoảng cân nặng mà số học sinh chiếm nhiều nhất là
	\choice
	{$[60;65)$}
	{$[55;60)$}
	{\True $[50;55)$}
	{$[60;65)$}
	\loigiai{Có $2$ học sinh có cân nặng từ $45$ kg đến dưới $50$ kg.\\
		Có $14$ học sinh có cân nặng từ $50$ kg đến dưới $55$ kg.\\
		Có $11$ học sinh có cân nặng từ $55$ kg đến dưới $60$ kg.\\
		Có $10$ học sinh có cân nặng từ $60$ kg đến dưới $654$ kg.\\
		Có $3$ học sinh có cân nặng từ $65$ kg đến $70$ kg.\\
		Vậy khoảng cân nặng từ $50$ kg đến dưới $55$ kg chiếm nhiều học sinh nhất.}
\end{ex}

\begin{ex}%[1D1H4-6]
	Tập giá trị của hàm số $y=\sin^2x+2\cos^2x$ là
	\choice
	{$T=\left[0;3\right]$}
	{$T=\left[0;2\right]$}
	{\True $T=\left[1;2\right]$}
	{$T=\left[1;3\right]$}
	\loigiai{
		Ta có $y=\sin^2x+2\cos^2x=1+\cos^2x$.\\
		Ta có $0\le \cos^2x\le 1$ nên $1\le 1+\cos^2x\le 2$.\\
		Vậy tập giá trị của hàm số $y=\sin^2x+2\cos^2x$ là $T=\left[1;2\right]$.
	}
\end{ex}

\begin{ex}%[1D2H1-3]
	Cho dãy số $\left(u_n\right)$, biết $u_n=\dfrac{2 n+5}{5 n-4}$. Số $\dfrac{7}{12}$ là số hạng thứ mấy của dãy số?
	\choice
	{\True $8$}
	{$6$}
	{$9$}
	{$10$}
	\loigiai{
		Giả sử $u_n = \dfrac{7}{12} \Leftrightarrow \dfrac{2 n+5}{5 n-4} = \dfrac{7}{12} \Leftrightarrow n = 8.$
	}
\end{ex}

\begin{ex}%[Pj10-1-HK1-NH23-24(CD)--TeamTeXHoa--Võ Thị Thùy Trang]%[1D1H4-3]
	Hàm số $y=\sin x$ đồng biến trên khoảng nào dưới đây?
	\choice
	{$\left(-\pi;\dfrac{\pi}{2}\right)$}
	{\True $\left(-\dfrac{\pi}{2};0\right)$}
	{$\left(0;\pi \right)$}
	{$\left(\dfrac{\pi}{2};\pi \right)$}
	\loigiai{
		Dựa vào đồ thị hàm số $y=\sin x$ ta thấy đồ thị hướng đi lên từ trái sang phải trên $\left(-\dfrac{\pi}{2};0\right)$.\\
		Nên hàm số đồng biến trên khoảng $\left(-\dfrac{\pi}{2};0\right)$.  }
\end{ex}

\begin{ex}%[Dự án BG K10-K11]%[Dao-V- Thuy]%[1H4H1-4]
	Cho hình chóp tứ giác $S.ABCD$ với đáy $ABCD$ có các cạnh đối diện không song song với nhau và $M$ là một điểm trên cạnh $SA$. Tìm giao điểm của đường thẳng $MC$ và mặt phẳng $(SBD)$.
	\choice
	{Điểm $H$, trong đó $I=AC\cap BD$, $H=MA\cap SI$}
	{Điểm $F$, trong đó $I=AC\cap BD$, $F=MD\cap SI$}
	{\True Điểm $K$, trong đó $I=AC\cap BD$, $K=MC\cap SI$}
	{Điểm $V$, trong đó $I=AC\cap BD$, $V=MB\cap SI$}
	\loigiai{
		\immini{
			Trong $(ABCD)$ gọi $I=AC\cap BD$.\\
			Trong $(SAC)$ gọi $K=MC\cap SI$.\\
			Ta có $K\in SI\subset(SBD)$ và $K\in MC$ nên $K=MC\cap(SBD)$.}{\begin{tikzpicture}[scale=0.7,>=stealth, font=\footnotesize, line join=round, line cap=round]
				\coordinate (A) at (0,0);
				\coordinate (E) at (1.3,-1.6);
				\coordinate (D) at (4.5,0);
				\coordinate (A) at (0,0);
				\coordinate (S) at (1,3.5);
				\coordinate (B) at ($(A)!0.5!(E)$);
				\coordinate (C) at ($(E)!0.3!(D)$);
				\coordinate (M) at ($(A)!0.4!(S)$);
				\coordinate (I) at (intersection of A--C and B--D);
				\coordinate (N) at (intersection of M--E and S--B);
				\coordinate (K) at (intersection of M--C and S--I);
				\draw (A)--(S)--(B) (S)--(C) (S)--(D) (A)--(B) (C)--(D) (B)--(C);
				\draw[dashed] (B)--(D) (A)--(C) (A)--(D) (M)--(C) (S)--(I);
				\foreach \p/\g in {S/90,C/-90,I/-90,B/180,A/180,M/180, D/0, K/45} \draw[fill] (\p) circle(.5pt)node [shift={(\g:.3)}] {$\p$};
			\end{tikzpicture}}}%<MyLT3>
\end{ex}

\begin{ex}.%[1H4V1-3]
	Cho tứ giác $A B C D$ và một điểm $S$ không thuộc mặt phẳng $(A B C D)$. Trên đoạn $S C$ lấy một điểm $M$ không trùng với $S$ và $C$. Gọi $N$ là giao điểm của đường thẳng $S D$ với mặt phẳng $(A B M)$. Khi đó $A N$ là giao tuyến của hai mặt phẳng nào sau đây?
	\choice
	{  $A N=(A B M) \cap(S B C)$}
	{ $A N=(A B M) \cap(S C D)$}
	{\True $A N=(A B M) \cap(S A D)$}
	{ $A N=(A B M) \cap(S A C)$}
	\loigiai{
		\immini{Ta có $B \in(A B M) \cap(S B D).\quad(1)$\\
			Gọi $O=A C \cap B D, K=A M \cap S O$. Khi đó\\
			$
				\heva{&K \in A M \subset(A B M) \\
					&K \in S O \subset(S B D)} \Rightarrow K \in(A B M) \cap(S B D).\quad(2)    $\\
			Từ (1) và $(2)$ suy ra $(A B M) \cap(S B D)=B K$.\\
			Trong mặt phẳng $(S B D)$, gọi $N=B K \cap S D$. Khi đó\\
			$
				\heva{  &N \in S D \\
					&N \in B K \subset(A B M)}
				\Rightarrow N=(A B M) \cap S D .$\\
			Dễ thấy $ A N=(A B M) \cap (S A D).$}
		{\begin{tikzpicture}[>=stealth,line join=round, line cap=round, scale=1]
				\coordinate[label=above left:$S$] (S) at (1,4) ;
				\coordinate [label=below left:$B$](B) at (0,-0.5) ;
				\coordinate [label=below left:$A$](A) at (-1,1) ;
				\coordinate [label=below:$C$](C) at (3,-1.5) ;
				\coordinate [label=right:$D$](D) at (5,1);
				\coordinate[label=below  right:$M$](M) at ($(S)!1/3!(C)$);
				\coordinate [label=below:$O$] (O) at (intersection of A--C and B--D);
				\coordinate [label=below right:$K$] (K) at (intersection of A--M and S--O);
				\coordinate [label=above right:$N$] (N) at (intersection of B--K and S--D);
				%   \coordinate[label=above right:$N$] (N) at ($(S)!0.5!(D)$);
				\draw (S)--(B)--(C)--(D)--cycle (B)--(A)--(S)--(C) ;
				\draw[dashed] (B)--(D)--(A)--(C) (S)--(O) (A)--(M) (B)--(N)
				;
				\foreach \diem in {S,A,B,C,D, M, N,O,K} \fill (\diem)circle(1.5pt);
			\end{tikzpicture}
		}}

\end{ex}

\begin{ex}%[1H4H3-2]
	Cho hình chóp tứ giác $S.ABCD$. Gọi $M$ và $N$ lần lượt là trung điểm của $SA$ và $SC$. Đường thẳng $MN$ song song với mặt phẳng nào dưới đây?
	\choice
	{Mặt phẳng $(SCD)$}
	{Mặt phẳng $(SAB)$}
	{Mặt phẳng $(SBC)$}
	{\True Mặt phẳng $(ABCD)$}
	\loigiai{
		\immini{Ta có $M$ và $N$ lần lượt là trung điểm của $SA$ và $SC$ nên $MN$ là đường trung bình của $\triangle SAC$, suy ra $MN\parallel AC.$\\
			Khi đó, $\left\{\begin{aligned}
					 & MN\parallel AC       \\
					 & AC\subset (ABCD)     \\
					 & MN\not\subset (ABCD) \\
				\end{aligned} \right.\Rightarrow MN\parallel (ABCD)$.}
		{
			\begin{tikzpicture}[line join=round, line cap = round, >=stealth, scale=0.7,font=\footnotesize]
				\path
				(0,0) coordinate (A)
				(5,0) coordinate (D)
				(1,-2.5) coordinate (B)
				(4,-3) coordinate (C)
				(2,3.3) coordinate (S)
				($(S)!0.5!(A)$) coordinate (M)
				($(S)!0.5!(C)$) coordinate (N)
				;
				\draw[dashed] (A)--(D) (M)--(N) (A)--(C);
				\draw (S)--(A)--(B)--(C)--(D)--cycle (S)--(B) (S)--(C);
				\foreach \x/\g in
					{S/90,A/165,S/90,B/-90,C/-90,D/0,M/180,N/45,D/0}
				\fill[black](\x) circle (1pt)
				($(\x)+(\g:3mm)$)node{$\x$};

			\end{tikzpicture}
		}
	}
\end{ex}

\begin{ex}%[Pj10-Đề 07 HK1 NH2023-2024 CTST]%[Nguyễn Văn Nay]%[1H4N3-1]
	Cho hai mặt phẳng $(P)$, $(Q)$ cắt nhau theo giao tuyến là đường thẳng $d$. Đường thẳng $a$ song song với cả hai mặt phẳng $(P)$, $(Q)$. Khẳng định nào sau đây đúng?
	\choice
	{$a$, $d$ trùng nhau}
	{$a$, $d$ chéo nhau}
	{\True $a$ song song $d$}
	{$a$, $d$ cắt nhau}
	\loigiai{
		Sử dụng hệ quả: Nếu hai mặt phẳng phân biệt cùng song song với một đường thẳng thì giao tuyến của chúng cũng song song với đường thẳng đó.
	}
\end{ex}

\begin{ex}%[1H4H1-3]
	Cho hình chóp tứ giác $S.ABCD$ và $M$ là một điểm thuộc cạnh $SC$ ($M$ khác $S$ và $C$). Giả sử hai đường thẳng $AB$ cà $CD$ cắt nhau tại $N$. Giao tuyến của hai mặt phẳng $\left(ABM\right)$ và $\left(SCD\right)$ cắt đường thẳng nào trong các đường thẳng sau
	\choice
	{\True $SD$}
	{$SA$}
	{$AD$}
	{$AC$}
	\loigiai{
		\immini{$M$ là điểm chung thứ nhất của hai mặt phẳng $\left(ABM\right)$ và $\left(SCD\right).\quad (1)$\\
			Do $AB$ và $CD$ cắt nhau tại $N$ nên \\
			$N\in AB\subset\left(ABM\right)\Rightarrow N\in \left(ABM\right);\\
				N\in CD\subset\left(SDC\right)\Rightarrow N\in \left(SDC\right)$.\\
			Vậy $N$ là điểm chung thứ hai của hai mặt phẳng $\left(ABM\right)$ và $\left(SCD\right).\quad(2)$\\
			Từ $(1)$ và $(2)$ ta có $MN$ là giao tuyến của hai mặt phẳng $\left(ABM\right)$ và $\left(SCD\right)$.\\
			$MN$ cắt $SD$ trong mặt phẳng $SCD$.}
		{\begin{tikzpicture}[>=stealth,line join=round,line cap=round,font=\footnotesize,scale=0.7]
				\path
				(2,6) coordinate (S)
				(-2,0) coordinate (A)
				(7,0) coordinate (D)
				(0,-3) coordinate (N)
				($(D)!0.4!(N)$) coordinate (C)
				($(A)!0.6!(N)$) coordinate (B)
				($(S)!0.6!(C)$) coordinate (M)
				(intersection of M--N and S--D) coordinate (K)
				;
				\draw[dashed] (M)--(A)--(D) (B)--(C);
				\draw (S)--(A)--(N)--(D)--cycle (N)--(K) (M)--(B)--(S)--(C);
				\foreach \d/\g in {S/90, B/-150,A/180,C/-45,D/0,M/-20,K/30,N/-90} \fill (\d) circle(1pt) + (\g:.4) node{$\d$};
			\end{tikzpicture}}
	}
\end{ex}

\begin{ex}%[1H4N4-1]%[Pj10-1-HK1-NH23-24-TeamTeXHoa-VUNgocHao]
	Hai mặt phẳng được gọi là song song nếu
	\choice
	{Có một đường thẳng nằm trong mặt phẳng này và song song với mặt phẳng kia}
	{Chúng có duy nhất một điểm chung}
	{Chúng có ít nhất hai điểm chung}
	{\True Chúng không có điểm chung}
	\loigiai{Hai mặt phẳng được gọi là song song nếu chúng không có điểm chung.
	}
\end{ex}

\TL
\begin{ex}%[1D1H5-3]%[Dự án đề kiểm tra Toán 11 GHKI NH23-24- Võ Thị Thùy Trang]%[THPT Võ Thị Sáu - Tp HCM]
	Giải phương trình sau $\sin 2x+3\cos x=0$.
	\loigiai{
		Ta có\\
		\allowdisplaybreaks
		$\begin{aligned}
				                & \sin 2x+3\cos x=0                                   \\
				\Leftrightarrow & 2\sin x\cos x+3\cos x=0                             \\
				\Leftrightarrow & \cos x(2\sin x+3)=0                                 \\
				\Leftrightarrow & \hoac{                                   & \cos x=0 \\& 2\sin x+3=0}\\
				\Leftrightarrow & \hoac{                                   & \cos x=0 \\& \sin x=-\dfrac{3}{2}\text{ (loại)}}\\
				\Leftrightarrow & x=\dfrac{\pi }{2}+k\pi ,k\in \mathbb{Z}.
			\end{aligned}$
	}
\end{ex}

\begin{ex}%[Pj08-0-GK1-NH23-24--TeamTeXHoa--LeQuan]%[1D3V2-5]
	Tính giới hạn  $\lim\limits_{x\to 1}\dfrac{x^3-\sqrt{3x-2}}{x^2-1}$
	\loigiai{
		Ta có
		\begin{eqnarray*}
			\lim\limits_{ x\to 1}\dfrac{x^3-\sqrt{3x-2}}{x^2-1}
			&=&\lim\limits_{ x\to 1}\dfrac{x^6-3x+2}{\left(x^2-1\right)\left(x^3+\sqrt{3x-2}\right)}\\
			&=& \lim\limits_{x\to 1}\dfrac{x^6-1-3x+3}{\left(x-1\right)\left(x+1\right)\left(x^3+\sqrt{3x-2}\right)}\\
			&=&\lim\limits_{ x\to 1}\dfrac{\left(x^3+1\right)\left(x^3-1\right)-3\left(x-1\right)}{\left(x-1\right)\left(x+1\right)\left(x^3+\sqrt{3x-2}\right)}\\
			&=&\lim\limits_{ x\to 1}\dfrac{\left(x-1\right)\left[\left(x^3+1\right)\cdot\left(x^2+x+1\right)-3\right]}{\left(x-1\right)\left(x+1\right)\left(x^3+\sqrt{3x-2}\right)}\\
			&=& \lim\limits_{ x\to 1}\dfrac{\left[\left(x^3+1\right)\cdot\left(x^2+x+1\right)-3\right]}{\left(x+1\right)\left(x^3+\sqrt{3x-2}\right)}\\
			&=&\dfrac{\left(1+1\right)\cdot\left(1+1+1\right)-3}{\left(1+1\right)\cdot\left(1+1\right)}\\
			&=&\dfrac{3}{4}.
		\end{eqnarray*}
	}\end{ex}

\begin{ex}%[1-HK1]%[VN-MT-9, Nguyen Huynh]%[1D3V1-5]
	Tam giác mà ba đỉnh của nó là ba trung điểm ba cạnh của tam giác $ ABC$ được gọi là tam giác trung bình của tam giác $ ABC$. Ta xây dựng dãy các tam giác $A_1B_1C_1$, $A_2B_2C_2$, $A_3B_3C_3$, $\dots$ sao cho $A_1B_1C_1$ là một tam giác đều cạnh bằng $3$ và với mỗi số nguyên dương $ n\geq 2$, tam giác $A_n{B_n}{C_n}$ là tam giác trung bình của tam giác $A_{n-1}{B_{n-1}}{C_{n-1}}$. Với mỗi số nguyên dương $ n$, kí hiệu $S_n$ tương ứng là diện tích hình tròn ngoại tiếp tam giác $A_n{B_n}{C_n}$. Tổng $ S=S_1+S_2+\dots+S_n+\dots=a\pi$. Tìm $a$.
	\loigiai{
	Vì dãy các tam giác $A_1B_1C_1$, $A_2B_2C_2$, $A_3B_3C_3$, $\dots$ là các tam giác đều nên bán kính đường tròn ngoại tiếp các tam giác bằng $\dfrac{\text{cạnh}\times\sqrt{3}}{3}$.\\ Với $ n=1$ thì tam giác đều $A_1B_1C_1$ có cạnh bằng $3$ nên đường tròn ngoại tiếp tam giác $A_1B_1C_1$có bán kính $R_1=3.\dfrac{\sqrt{3}}{3}$ $\Rightarrow{S_1}=\pi{\left(3\cdot\dfrac{\sqrt{3}}{3}\right)^2}$.\\ Với $ n=2$ thì tam giác đều $A_2B_2C_2$ có cạnh bằng $\dfrac{3}{2}$ nên đường tròn ngoại tiếp tam giác $A_2B_2C_2$ có bán kính $R_2=3\cdot\dfrac{1}{2}\cdot\dfrac{\sqrt{3}}{3}$ $\Rightarrow{S_2}=\pi{\left(3\cdot\dfrac{1}{2}\cdot\dfrac{\sqrt{3}}{3}\right)^2}$.\\ Với $ n=3$ thì tam giác đều $A_3B_3C_3$ có cạnh bằng $\dfrac{3}{4}$ nên đường tròn ngoại tiếp tam giác $A_2B_2C_2$ có bán kính $R_3=3\cdot\dfrac{1}{4}\cdot\dfrac{\sqrt{3}}{3}$ $\Rightarrow{S_3}=\pi{\left(3\cdot\dfrac{1}{4}\cdot\dfrac{\sqrt{3}}{3}\right)^2}$.\\ Như vậy tam giác đều $A_n{B_n}{C_n}$ có cạnh bằng $ 3\cdot\left(\dfrac{1}{2}\right)^{n-1}$ nên đường tròn ngoại tiếp tam giác $A_n{B_n}{C_n}$ có bán kính $R_n=3\cdot\left(\dfrac{1}{2}\right)^{n-1}\cdot\dfrac{\sqrt{3}}{3}$ $\Rightarrow{S_n}=\pi{\left(3\cdot\left(\dfrac{1}{2}\right)^{n-1}\cdot\dfrac{\sqrt{3}}{3}\right)^2}$.\\ Khi đó, ta được dãy $S_1$, $S_2$, $ \dots S_n\dots$ là một cấp số nhân lùi vô hạn với số hạng đầu $u_1=S_1=3\pi $ và công bội $ q=\dfrac{1}{4}$.\\ Do đó, tổng $ S=S_1+S_2+\dots+S_n+\dots$$=\dfrac{u_1}{1-q}=4\pi $.\\ Suy ra $ a=4$.
	}
\end{ex}

\begin{ex}%[1C4K46]
	Cho hình chóp $ S.ABCD $ có đáy $ ABCD $ là hình thang, đáy lớn $ AD=2BC $ và $ O $ là giao điểm của hai đường chéo đáy. Gọi $ E $, $ F $ lần lượt là trung điểm $ SA $, $ SD $ và $ G $ là trọng tâm tam giác $ SCD $.
	\begin{enumerate}
		\item Mặt phẳng $ (P) $ đi qua $ E $, $ F $ và song song với $ SB $. Giả sử $ (P) $ cắt cạnh $ CD $, $ AB $ lần lượt tại $ P $, $ Q $. Chứng minh $ EQ\parallel SB $. Tứ giác $ EFPQ $ là hình gì? Chứng minh $ BE\parallel (SCD) $ và $ GO\parallel (SBC) $.
		\item Tìm giao điểm $ M $ của $ SB $ và $ (CDE) $. Chứng minh $ \dfrac{S_{\triangle SME}}{S_{\triangle SMF}}=\dfrac{S_{\triangle SAB}}{S_{\triangle SBD}} $ và $ SM\cdot BD=SB\cdot DO $.
	\end{enumerate}
	\loigiai{
		%         \begin{center}
		%             \begin{tikzpicture}[scale=1, font=\footnotesize, line join=round, line cap=round, >=stealth]
		%             \tikzset{label style/.style={font=\footnotesize}}
		%             \pgfmathsetmacro\h{2}
		%             \pgfmathsetmacro\goc{65}
		%             \pgfmathsetmacro\xmin{-2.5*\h}
		%             \pgfmathsetmacro\xmax{8*\h}
		%             \pgfmathsetmacro\ymin{-1.2*\h}
		%             \pgfmathsetmacro\ymax{2.8*\h}
		%             \clip(\xmin-0.01,\ymin-0.01)rectangle(\xmax+0.01,\ymax+0.01);
		%             \begin{scope}
		%             \tkzDefPoint(0,0){B}
		%             \tkzDefShiftPoint[B](0:1.5*\h){C}
		%             \tkzDefShiftPoint[B](1.9*\goc:1.1*\h){A}
		%             \tkzDefShiftPoint[A](0:3*\h){D}
		%             \tkzDefShiftPoint[B](1.3*\goc:2.5*\h){S}
		%             \tkzDefLine[parallel=through S](B,C)\tkzGetPoint{s}
		%             \coordinate(Q) at ($(A)!1/2!(B)$);
		%             \coordinate(E) at ($(A)!1/2!(S)$);
		%             \coordinate(F) at ($(D)!1/2!(S)$);
		%             \coordinate(P) at ($(D)!1/2!(C)$);
		%             \coordinate(N) at ($(S)!1/2!(C)$);
		%             \tkzInterLL(A,B)(D,C)\tkzGetPoint{H}
		%             \tkzInterLL(A,C)(B,D)\tkzGetPoint{O}
		%             \tkzInterLL(B,S)(H,E)\tkzGetPoint{M}
		%             \tkzInterLL(N,D)(F,C)\tkzGetPoint{G}
		%             \tkzDrawPoints[fill=black](A,B,C,D,S,P,Q,E,F,H,O,G,M,N)
		%             \draw (S)--(A)--(H)--(D)--cycle (B)--(S)--(H) (Q)--(E)--(B) (E)--(H) (B)--(N)--(D) (S)--(C)--(F)--(P);
		%             \draw[dashed] (C)--(A)--(D)--(B)--(C)--(E)--(F)--(M) (G)--(O) (P)--(Q);
		% %			\tkzDrawLines[add = 0.5 and 0.3,end=$x$](S,s)
		%             \tkzLabelPoints[below](H,O)
		%             \tkzLabelPoints[above](S)
		%             \tkzLabelPoints[left](A,Q,B,E,N)
		%             \tkzLabelPoints[right](D,P,C,F,M)
		%             \tkzLabelPoints[above left](G)
		%             \end{scope}
		%             \begin{scope}[xshift=12 cm,yshift=0 cm]
		%             \tkzDefPoint(0,0){B}
		%             \tkzDefShiftPoint[B](170:2*\h){A}
		%             \tkzDefShiftPoint[B](1.6*\goc:1.5*\h){S}
		%             \coordinate(E) at ($(A)!1/2!(S)$);
		%             \coordinate(M) at ($(B)!1/3!(S)$);
		%             \tkzDrawAltitude(S,B)(A)\tkzGetPoint{a}
		%             \tkzDrawAltitude(S,B)(E)\tkzGetPoint{e}
		%             \tkzDrawPoints[fill=black](A,B,S,E,M)
		%             \draw (S)--(A)--(B)--cycle (E)--(M);
		%             \tkzLabelPoints[above](S)
		%             \tkzLabelPoints[left](A,E)
		%             \tkzLabelPoints[right](B,M)
		%             \tkzLabelSegment[pos=0.8,xshift=-0.1cm](E,e){$h_{2}$}
		%             \tkzLabelSegment[pos=0.5,xshift=-0.2cm](A,a){$h_{1}$}
		%             \tkzMarkRightAngles[size=0.2](E,e,S A,a,S)
		%             \end{scope}
		%             \end{tikzpicture}
		%         \end{center}
		\begin{enumerate}
			\item Ta có $ \heva{&EQ=(P)\cap(SAB)\\ &SB\subset (SAB),\ SB\parallel (P)}$ nên $ EQ\parallel SB $.\\
			      Lại có $ \heva{&QP=(P)\cap (ABCD)\\ &AD\parallel EF\\ &AD\subset (ABCD), EF\subset (P)} $ nên $ QP\parallel EF\parallel AD $. Suy ra $ EFPQ $ là hình thang.\\
			      Vì $ \heva{&BC=\dfrac{AD}{2}\\ &BC\parallel AD} $ và $ \heva{&EF=\dfrac{AD}{2}\\ &EF\parallel AD} $ nên $ BCFE $ là hình bình hành, suy ra $ BE\parallel CF $.\\
			      Mà $ CF\subset (SCD) $ nên $ BE\parallel (SCD) $.\\
			      Ta có $ \triangle OAD\backsim\triangle OCB $ nên $ \dfrac{OA}{OC}=\dfrac{OD}{OB}=\dfrac{AD}{BC}=2 $. Gọi $ N $ là trung điểm của $ SC $.\\
			      Trong tam giác $ BDN $ có $ \dfrac{SG}{SN}=\dfrac{DO}{DB}=\dfrac{2}{3} $ nên $ OG\parallel BN $. Mà $ BN\subset (SBC) $ nên $ OG\parallel (SBC) $.
			\item Vì $ H $ và $ E $ là hai điểm chung của hai mặt phẳng $ (SAB) $ và $ (ECD) $ nên $ (SAB)\cap (ECD)=EH $.\\
			      Gọi $ M=EH\cap SB\Rightarrow M=SB\cap (ECD) $.\\
			      Trong $ \triangle HAD $ có $ BC\parallel AD $ nên ta có $ \dfrac{HB}{HA}=\dfrac{HC}{HD}=\dfrac{BC}{AD}=\dfrac{1}{2} $, suy ra $ B $ là trung điểm $ AH $. Do đó $ M $ là trọng tâm tam giác $ SAH $.\\
			      Tam giác $ SAB $ được vẽ lại ở hình 2.\\
			      Ta có $ S_{\triangle SME}=\dfrac{1}{2}\cdot h_{2}\cdot SM=\dfrac{1}{2}\cdot \dfrac{h_{1}}{2}\cdot \dfrac{2}{3}SB=\dfrac{1}{3}\cdot \dfrac{1}{2}h_{1}\cdot SB=\dfrac{1}{3}S_{\triangle SAB}$\\
			      $\Rightarrow \dfrac{S_{\triangle SME}}{S_{\triangle SAB}}=\dfrac{1}{3} $.\hfill(*)\\
			      Tương tự ta cũng có $\dfrac{S_{\triangle SMF}}{S_{\triangle SBD}}=\dfrac{1}{3} $.\hfill(**)\\
			      Từ (*) và (**) ta suy ra $ \dfrac{S_{\triangle SME}}{S_{\triangle SAB}}=\dfrac{S_{\triangle SMF}}{S_{\triangle SBD}}\Rightarrow \dfrac{S_{\triangle SME}}{S_{\triangle SMF}}=\dfrac{S_{\triangle SAB}}{S_{\triangle SBD}} $.\\
			      Theo chứng minh trên thì $ \dfrac{SM}{SB}=\dfrac{DO}{DB}\Rightarrow SM\cdot BD=SB\cdot DO $.
		\end{enumerate}
	}
\end{ex}
\Closesolutionfile{ans}
