\section*{ÔN TẬP KIỂM TRA CUỐI KÌ 1 - ĐỀ 03}
\setcounter{ex}{0}\setcounter{bt}{0}
\Opensolutionfile{ans}[ans/ansBTTeX3]

\noindent\textbf{I. PHẦN TRẮC NGHIỆM:}
\begin{ex}%[1D2H2-3]%[Dự án đề kiểm tra Toán 11 GHKI NH23-24- Huỳnh Quy]%[THPT Tân Bình - Tp HCM]
Cấp số cộng $(u_n)$ có số hạng đầu tiên $u_1=2$ và công sai $d=3$. Số hạng $u_3$ bằng
\choice
{$6$}
{\True $8$}
{$10$}
{$9$}
\loigiai{
Ta có $u_3=u_1+2d=2+2\cdot 3=8$.
}
\end{ex}

\begin{ex}%[1D3N1-2]%[Dự án đề kiểm tra Toán 11 HKI NH23-24- Tư Đô Nguyên]%[Đề 11 - CTST]
$\lim\dfrac{1}{n^3}$ bằng
\choice
{\True $0$}
{$2$}
{$4$}
{$5$}
\loigiai{Ta có $\lim\dfrac{1}{n^3}=0$.}
\end{ex}

\begin{ex}%[Dự án TeX GKI 11, Đoàn Hùng]%[1D2H2-6]
Tìm tổng $S$ của $100$ số nguyên dương đầu tiên và đều chia $5$ dư $1$.
\choice
{\True $24850$ }
{ $25100$ }
{ $50200$ }
{ $5001$ }
\loigiai{
Các số chia $5$ dư $1$ tạo thành cấp số cộng có $u_1=1$ và $d=5$, do đó
$S_{100}=\dfrac{100\cdot (2u_1+99d)}{2}=\dfrac{100\cdot (2\cdot 1+99\cdot 5)}{2}=24850.$
}
\end{ex}

\begin{ex}%[Pj10-1-GK1-NH23-24--TeamTeXHoa--Lâm Chính]%[1H4N2-2]
\immini{
Cho hình hộp $ABCD.A'B'C'D'$. Đường thẳng $AB$ song song với đường thẳng nào?
\choice
{\True $C'D'$}
{$BD$}
{$CC'$}
{$D'A'$}
}{
\begin{tikzpicture}[scale=0.6, font=\footnotesize,line join=round, line cap=round, >=stealth]
\coordinate (A) at (0,0);
\coordinate (B) at (4,0);
\coordinate (D) at (-2,-2);
\coordinate (C) at ($(B)+(D)-(A)$);
\coordinate (H) at (1,3);
\foreach \i in {A,B,C,D}{\coordinate (\i') at ($(\i)+(H)$);}
\draw (A')--(B')--(C')--(D')--cycle;
\draw (B)--(B') (C)--(C') (D)--(D')  (B)--(C)--(D);
\draw[dashed,thin](B)--(A)--(A') (A)--(D);
\foreach \i/\g in {A'/90,B'/90,C'/90,D'/90,A/-90,B/-90,C/-90,D/-90}{\draw[fill=black](\i) circle (1.5pt) ($(\i)+(\g:4mm)$) node[scale=1]{$\i$};}
\end{tikzpicture}}

\loigiai{
Ta có $AB \parallel C'D'$.
}
\end{ex}

\begin{ex}%[1D1N4-2]
Tập $D=\mathbb{R}\backslash \left\{ \dfrac{k\pi }{2}\middle| k\in \mathbb{Z} \right\}$ là tập xác định của hàm số nào sau đây?
\choice
{ $y=\cot x$}
{\True  $y=\cot 2x$}
{ $y=\tan x$}
{ $y=\tan 2x$}
\loigiai{Hàm số $y=\cot 2x$ xác định khi $2x\ne k\pi $ $\Leftrightarrow x\ne \dfrac{k\pi }{2}$.}
\end{ex}

\begin{ex}%[1D3N2-1]
Giả sử $\lim\limits_{x \rightarrow x_0} f(x)=L$, $\lim\limits_{x \rightarrow x_0} g(x)=L\, (L, M \in \mathbb{R})$. Chọn đáp án \textbf{sai}.
\choice
{$\lim\limits_{x \rightarrow x_0}[f(x)+g(x)]=L+M$}
{$\lim\limits_{x \rightarrow x_0}[f(x)-g(x)]=L-M$}
{$\lim\limits_{x \rightarrow x_0}[f(x)\cdot g(x)]=L\cdot M$}
{\True $\lim\limits_{x \rightarrow x_0} \dfrac{f(x)}{g(x)}=\dfrac{L}{M}$}
\loigiai{Ta có $\lim\limits_{x \rightarrow x_0} \dfrac{f(x)}{g(x)}=\dfrac{L}{M}$ (nếu $M \neq 0$).
}
\end{ex}

\begin{ex}%[1D1N3-1]
Trong các công thức sau, công thức nào đúng?
\choice
{\True $\sin 2\alpha=2\sin \alpha \cos \alpha$}
{$\sin 2\alpha=2\sin \alpha$}
{$\sin 2\alpha=\sin \alpha+\cos \alpha$}
{$\sin 2\alpha=\cos^2\alpha-\sin^2\alpha$}
\loigiai{

}
\end{ex}

\begin{ex}%[Pj08-0-GK1-NH23-24--TeamTeXHoa--LeQuan]%[1D1H2-2]
Cho $\sin\alpha=\dfrac{1}{3}$. Giá trị của $\cos 2\alpha $ bằng
\choice
{$\dfrac{2\sqrt 2}{3}$}
{$-\dfrac{2\sqrt 2}{3}$}
{\True $\dfrac{7}{9}$}
{$-\dfrac{7}{9}$}
\loigiai{
Ta có $\cos 2\alpha=1-2\sin ^2\alpha=1-2\left(\dfrac{1}{3}\right)^2=\dfrac{7}{9}$.}
\end{ex}

\begin{ex}%[Pj10-Đề 07 HK1 NH2023-2024 CTST]%[Nguyễn Văn Nay]%[1H4H4-2]
Cho hình hộp $ABCD.A'B'C'D'$. Mặt phẳng $(AB'D')$ song song với mặt phẳng nào trong các mặt phẳng sau đây?
\choice
{$(BCA')$}
{\True $(BC'D)$}
{$(A'C'C)$}
{$(BDA')$}
\loigiai{
\immini{
Do $ADC'B'$ là hình bình hành nên $AB'\parallel DC'$, và $ABC'D'$ là hình bình hành nên $AD'\parallel BC'$ nên $(AB'D')\parallel (BC'D)$.
}
{
\begin{tikzpicture}[scale=1, font=\footnotesize, line join=round, line cap=round, >=stealth]
\def\a{1.6} \def\b{3} \def\h{2.5} \def\g{-120}
\path
(0:0) coordinate (A)
(\g:\a) coordinate (B)
(0:\b) coordinate (D)
($(B)+(D)-(A)$) coordinate (C)
;
\foreach \x in {A,B,C,D} \coordinate (\x') at ($(\x)+(90:\h)$);
\draw (B')--(A')--(D')--(C')--(B')--(B)--(C)--(D)--(D') (C)--(C')
(D)--(C')--(B) (B')--(D');
\draw[dashed] (B)--(A)--(D)--(B) (A)--(A')
(B)--(A')--(D) (C)--(A') (B')--(A)--(D')
;
\foreach \i/\g in {A/-80,B/-90,C/-90,D/0,A'/90,D'/90,B'/180,C'/5}
\fill[black] (\i) circle(1pt) ($(\i)+(\g:3mm)$) node{$\i$};
\end{tikzpicture}
}
}
\end{ex}

\begin{ex}%[1H4H1-3]%[Dự án đề ôn tập Toán Khối 11 HK1 NH23-24-Dot 1-Vương Quốc Phong]%[CTST - Đề số 3]
Cho hình chóp $S.ABCD$ có đáy là hình thang $ABCD$ ($AB \cap CD = O$). Khẳng định nào sau đây \textbf{sai}?
\choice
{Hình chóp $S.ABCD$ có $4$ mặt bên}
{Giao tuyến của hai mặt phẳng $(SAC)$ và $(SBD)$ là $SO$}
{Giao tuyến của hai mặt phẳng $(SAD)$ và $(SBC)$ là $SI$ ($I$ là giao điểm của $AD$ và $BC$)}
{\True Giao tuyến của hai mặt phẳng $(SAB)$ và $(SAD)$ là đường trung bình của $ABCD$}
\loigiai{
Ta có $(SAB) \cap (SAD) = SA$ và $SA$ không thể là đường trung bình của hình thang $ABCD$.
}
\end{ex}

\begin{ex}%[Pj08-0-GK1-NH23-24--TeamTeXHoa--LeQuan]%[1D2H3-3]
Cho cấp số nhân có các số hạng lần lượt là $ 3;\, 9;\, 27;\, 81;\ldots$. Tìm số hạng tổng quát $u_n$ của cấp số nhân đã cho.
\choice
{$u_n=3^{n-1}$}
{\True $u_n=3^n$}
{$u_n=3^{n+1}$}
{$u_n=3+3^n$}
\loigiai{
Dãy số  $ 3;\, 9;\, 27;\, 81;\ldots$ là một cấp số nhân có $u_1=3$ và công bội $q=3$.\\
$\Rightarrow u_n=u_1\cdot q^{n-1}=3\cdot 3^{n-1}=3^n$.
}
\end{ex}

\begin{ex}%[Pj08-0-GK1-NH23-24--TeamTeXHoa--LeQuan]%[1D2N1-4]
Trong các dãy số có số hạng tổng quát sau, dãy số nào {\bf không} là dãy số tăng, cũng {\bf không} là dãy số giảm?
\choice
{$u_n=n$}
{$v_n=2n$}
{$x_n=\dfrac{1}{n}$}
{\True $w_n=\dfrac{\left(-1\right)^n}{n}$}
\loigiai{
\begin{itemize}
\item Xét  $u_n=n\Rightarrow u_{n+1}=n+1\Rightarrow u_{n+1}-u_n=1>0\,\forall n\in \mathbb{N}^*$.\\
Vậy dãy số $(u_n)$ là dãy số tăng.
\item Xét $v_n=2n\Rightarrow v_{n+1}=2(n+1)\Rightarrow v_{n+1}-v_n=21>0\,\forall n\in \mathbb{N}^*$.\\
Vậy dãy số $(v_n)$ là dãy số tăng.
\item Xét $x_n=\dfrac{1}{n}\Rightarrow x_{n+1}=\dfrac{1}{n+1}\Rightarrow x_{n+1}-x_n<0 1>0\,\forall n\in \mathbb{N}^*$.\\
Vậy dãy số $(x_n)$ là dãy số giảm.
\item Xét $w_n=\dfrac{(-1)^n}{n}$ có $w_1=-1$, $w_2=\dfrac{1}{2}$, $w_3=-\dfrac{1}{3}$.\\
Suy ra $ \heva{&w_1<w_2 \\&w_2>w_3}$ nên dãy $(w_n)$ không phải là dãy tăng, cũng không phải là dãy giảm.
\end{itemize}
}
\end{ex}

\begin{ex} %[1D2N3-1]
Trong các dãy số sau, dãy số nào không phải là một cấp số nhân?
\choice{$2;4;8;16;\ldots$}
{$1;-1;1;-1;\ldots$}
{\True $1^2;2^2;3^2; 4^2;\ldots$}
{$a;a^3;a^5;a^7;\ldots \; \left(a\ne 0\right)$}
\loigiai{
Dãy $1^2;2^2;3^2; 4^2;\cdots$ không phải là một cấp số nhân.
}
\end{ex}

\begin{ex}%[1H4H2-2]
Cho hình chóp $S.ABCD$, đáy $ABCD$ là hình bình hành. Điểm $M$ thuộc cạnh $SC$, $N$ là giao điểm của $SD$ và $\left(MAB \right)$. Khi đó, hai đường thẳng $CD$ và $MN$ là hai đường thẳng
\choice
{Cắt nhau}
{\True Song song}
{Chéo nhau}
{Có hai điểm chung}
\loigiai{
\begin{center}
\begin{tikzpicture}[scale=1,line join=round, line cap=round, >=stealth,font=\footnotesize]
\def\a{4} \def\b{3.5}
\path
(0,0) coordinate (D)
(\a,0) coordinate (C)
({\a+1.5},1.5) coordinate (B)
(1.5,1.5) coordinate (A)
(2,\b) coordinate (S)
($(S)!0.7!(C)$) coordinate (M)
($(S)!0.7!(D)$) coordinate (N)
;

\draw
(D) -- (C) -- (B) -- (S) -- (D)
(S) -- (C)
(B) -- (M) -- (N) -- ([turn]0:1cm) node[above]{$x$}
;

\draw[dashed]
(B) -- (A) -- (D)
(S) -- (A) -- (M)
;
\fill
(A) circle (1pt) node[left]{$A$}
(B) circle (1pt) node[right]{$B$}
(C) circle (1pt) node[below]{$C$}
(D) circle (1pt) node[below]{$D$}
(M) circle (1pt) node[below left]{$M$}
(N) circle (1pt) node[above left]{$N$}
(S) circle (1pt) node[above]{$S$}
;
\end{tikzpicture}
\end{center}
Ta có $MN$ là giao tuyến của hai mặt phẳng $\left(MAB \right)$ và $\left(SCD \right)$.\\
Mặt khác $\heva{&AB \subset \left(MAB \right)\\&CD \subset \left(SCD \right)\\&AB \parallel CD} \Rightarrow MN \parallel CD$.
Vậy $MN$ song song với $CD$.
}
\end{ex}

\begin{ex}%[1H4H1-2]
Cho tứ diện $ABCD$, gọi $M$ và $N$ lần lượt là trung điểm các cạnh $AB$ và $CD$. Gọi $G$ là trọng tâm tam giác $BCD$. Đường thẳng $AG$ cắt đường thẳng nào trong các đường thẳng dưới đây?
\begin{center}
\begin{tikzpicture}[>=stealth,line join=round,line cap=round,font=\footnotesize,scale=1]
\path
(0,0) coordinate (B)
(1.5,-2) coordinate (C)
(5,0) coordinate (D)
($(B)+(1,4)$) coordinate (A)
($(A)!0.5!(B)$) coordinate (M)
($(C)!0.5!(D)$) coordinate (N)
;
\draw
(B)--(C)--(A)--(D)
;
\draw[dashed] (B)--(D)(M)--(N);
\draw (A)edge node[midway, sloped, rotate=90, anchor=center] {$ - $}(M);
\draw (M)edge node[midway, sloped, rotate=90, anchor=center] {$ - $}(B);
\draw (D)edge node[midway, sloped, rotate=90, anchor=center] {$ = $}(N);
\draw (N)edge node[midway, sloped, rotate=90, anchor=center] {$ = $}(C);
\foreach \x/\y in {A/95,B/190,C/-10,D/10,M/180,N/-20}
\draw[fill=black] (\x) circle (1.1pt) + (\y:0.5cm) node {$\x$};
\end{tikzpicture}
\end{center}
\choice
{\True $MN$}
{$CM$}
{$DN$}
{$CD$}
\loigiai{
\begin{center}
\begin{tikzpicture}[>=stealth,line join=round,line cap=round,font=\footnotesize,scale=1]
\path
(0,0) coordinate (B)
(1.5,-2) coordinate (C)
(5,0) coordinate (D)
($(B)+(1,4)$) coordinate (A)
($(A)!0.5!(B)$) coordinate (M)
($(C)!0.5!(D)$) coordinate (N)
($(B)!2/3!(N)$) coordinate (G)
;
\draw
(B)--(C)--(A)--(D)(A)--(N)
;
\draw[dashed] (M)--(N)--(B)--(D) (A)--(G);
\draw (A)edge node[midway, sloped, rotate=90, anchor=center] {$ - $}(M);
\draw (M)edge node[midway, sloped, rotate=90, anchor=center] {$ - $}(B);
\draw (D)edge node[midway, sloped, rotate=90, anchor=center] {$ = $}(N);
\draw (N)edge node[midway, sloped, rotate=90, anchor=center] {$ = $}(C);
\foreach \x/\y in {A/95,B/190,C/-10,D/10,M/180,N/-10,G/-90}
\draw[fill=black] (\x) circle (1.1pt) + (\y:0.5cm) node {$\x$};
\end{tikzpicture}
\end{center}
Do $AG$ và $MN$ cùng nằm trong mặt phẳng $\left(ABN\right)$ nên hai đường thẳng cắt nhau.
}
\end{ex}

\begin{ex}%[1D3N3-1]
Cho hàm số $y=f\left(x\right)$ liên tục trên $(a;b)$. Điều kiện cần và đủ để hàm số liên tục trên $\left[a; b\right]$ là
\choice
{${\mathop{\lim}\limits_{x\to a^{+}}} f\left(x\right)=f\left(a\right)$ và ${\mathop{\lim}\limits_{x\to b^{+}}} f\left(x\right)=f\left(b\right)$}
{${\mathop{\lim}\limits_{x\to a^{-}}} f\left(x\right)=f\left(a\right)$ và ${\mathop{\lim}\limits_{x\to b^{-}}} f\left(x\right)=f\left(b\right)$}
{\True ${\mathop{\lim}\limits_{x\to a^{+}}} f\left(x\right)=f\left(a\right)$ và ${\mathop{\lim}\limits_{x\to b^{-}}} f\left(x\right)=f\left(b\right)$}
{${\mathop{\lim}\limits_{x\to a^{-}}} f\left(x\right)=f\left(a\right)$ và ${\mathop{\lim}\limits_{x\to b^{+}}} f\left(x\right)=f\left(b\right)$}
\loigiai{
Theo định nghĩa hàm số liên tục trên đoạn $\left[a; b\right]$ nếu
hàm số $y=f\left(x\right)$ liên tục trên $(a;b)$ và ${\mathop{\lim}\limits_{x\to a^{+}}} f\left(x\right)=f\left(a\right)$ và ${\mathop{\lim}\limits_{x\to b^{-}}} f\left(x\right)=f\left(b\right)$.

}
\end{ex}

\begin{ex}%[Pj10-Đề 07 HK1 NH2023-2024 CTST]%[Nguyễn Văn Nay]%[1H4N6-1]
Cho các đường thẳng không song song với phương chiếu. Khẳng định nào sau đây là đúng?
\choice
{Phép chiếu song song biến hai đường thẳng song song thành hai đường thẳng song song}
{Phép chiếu song song có thể biến hai đường thẳng song song thành hai đường thẳng cắt nhau}
{Phép chiếu song song có thể biến hai đường thẳng song song thành hai đường thẳng chéo nhau}
{\True Phép chiếu song song biến hai đường thẳng song song thành hai đường thẳng song song hoặc trùng nhau}
\loigiai{
Theo tính chất của phép chiếu song song, phép chiếu song song biến hai đường thẳng song song thành hai đường thẳng song song hoặc trùng nhau.
}
\end{ex}

\begin{ex}%[1D3H1-2]%[Dự án EX-CK1 (10,11 mới)- Lâm Chính]
Tìm giới hạn $\lim \dfrac{2^{n+1}+4^n}{3^n+4^{n+1}}$.
\choice
{$\dfrac{1}{2}$}
{\True $\dfrac{1}{4}$}
{$0$}
{$+\infty$}
\loigiai{
Ta có $\lim \dfrac{2^{n+1}+4^n}{3^n+4^{n+1}}=\lim \dfrac{2\cdot2^n+4^n}{3^n+4\cdot4^n}=\lim \dfrac{2\cdot\left(\dfrac{2}{4} \right)^n+1}{\left(\dfrac{3}{4} \right)^n+4}=\dfrac{2\cdot0+1}{0+4}=\dfrac{1}{4}$.

}
\end{ex}

\begin{ex}%[1D3H2-2]
Cho $\lim\limits u_n=-3$, $\lim\limits v_n=2$. Khi đó $\lim\limits \left(u_n-v_n\right)$ bằng
\choice
{\True $-5$}
{$-1$}
{$5$}
{$1$}
\loigiai{
Ta có $\lim\limits \left(u_n-v_n\right)=\lim u_n-\lim v_n
=-3-2=-5$.}
\end{ex}

\begin{ex}%[1D1N5-1]
Phương trình $\sin x=\sin \alpha $ có tập nghiệm là:
\choice
{ $S=\left\{ \alpha +k2\pi |k\in \mathbb{Z} \right\}$}
{ $S=\left\{ \alpha +k\pi |k\in \mathbb{Z} \right\}$}
{ $S=\left\{ \alpha +k2\pi;-\alpha +k2\pi |k\in \mathbb{Z} \right\}$}
{\True  $S=\left\{ \alpha +k2\pi;\pi -\alpha +k2\pi |k\in \mathbb{Z} \right\}$}
\loigiai{
$
\begin{array}{llll} &\sin x=\sin \alpha \\
\Leftrightarrow & \hoac{&x=\alpha +k2\pi \\
&x= \pi -\alpha +k2\pi},\,k\in \mathbb{Z}. \\
\end{array}$}
\end{ex}

\begin{ex}%[1D5H1-4]
Người ta ghi lại tuổi thọ của một số con muỗi cái trong phòng thí nghiệm cho kết quả như sau
\begin{center}
\begin{tabular}{|c|c|c|c|c|c|}
\hline
Tuổi thọ (ngày) & $[0;20)$& $[20;40)$&$[40;60)$& $[60;80)$& $[80;100)$\\
\hline
Số lượng & $5$&$12$&$23$&$31$&$29$\\ \hline
\end{tabular}
\end{center}
Muỗi cái có tuổi thọ khoảng bao nhiêu ngày là nhiều nhất?
\choice
{$80$ ngày}
{$66$ ngày}
{\True $76$ ngày}
{$96$ ngày}
\loigiai{
Nhóm chứa mốt của mẫu số liệu ghép nhóm trên là nhóm 4: $[60;80)$.\\
Do đó $u_4=60$, $n_4=31$, $n_3=23$, $n_5=29$ $u_5-u_4=80-60=20$.\\
Vậy mốt của mẫu số liệu trên là
\allowdisplaybreaks
\begin{eqnarray*}
M_e &=& u_4+\dfrac{n_4-n_3}{(n_4-n_3)+(n_4-n_5)}\cdot (u_5-u_4)\\
&=& 60+\dfrac{31-23}{(31-23)+(31-29)}\cdot (20)=76.
\end{eqnarray*}
Muỗi cái có tuổi thọ nhiều nhất là $76$ ngày.
}
\end{ex}

\begin{ex} %[1D3N1-1]
Khẳng định nào sau đây là đúng?
\choice
{Ta nói dãy số $\left(u_n \right)$ có giới hạn là số $a$ (hay $u_n$ dần tới $a$) khi $n \to +\infty$, nếu $\underset{x \to +\infty}{\lim}{\left(u_n + a \right)} = 0$}
{Ta nói dãy số $\left(u_n \right)$ có giới hạn là $0$ khi $n$ dần tới vô cực, nếu $\left|u_n \right|$ có thể lớn hơn một số dương tùy ý, kể từ một số hạng nào đó trở đi}
{Ta nói dãy số $\left( {{u}_{n}} \right)$ có giới hạn $+\infty $ khi $n\to +\infty $ nếu ${{u}_{n}}$ có thể nhỏ hơn một số dương bất kì, kể từ một số hạng nào đó trở đi}
{\True Ta nói dãy số $\left( {{u}_{n}} \right)$ có giới hạn $+\infty $ khi $n\to +\infty $ nếu ${{u}_{n}}$ có thể lớn hơn một số dương bất kì, kể từ một số hạng nào đó trở đi}
\loigiai{
Ta nói dãy số $\left( {{u}_{n}} \right)$ có giới hạn $+\infty $ khi $n\to +\infty $ nếu ${{u}_{n}}$ có thể lớn hơn một số dương bất kì, kể từ một số hạng nào đó trở đi}
\end{ex}

\begin{ex}%[1D5H1-3]%[Dự án đề kiểm tra Toán 11 GHKI NH23-24- Nguyễn Văn Sang]%[ĐỀ KNTT SỐ 5]
Cân nặng của $28$ học sinh của một lớp $11$ được cho như sau
\begin{center}
\begin{tabular}{*{14}{c}}
55{,}4 & 62{,}6 & 54{,}2 & 56{,}8 & 58{,}8 & 59{,}4 & 60{,}7 & 58 & 59{,}5 & 63{,}6 & 61{,}8 & 52{,}3 & 63{,}4 & 57{,}9 \\
49{,}7 & 45{,}1 & 56{,}2 & 63{,}2 & 46{,}1 & 49{,}6 & 59{,}1 & 55{,}3 & 55{,}8 & 45{,}5 & 46{,}8 & 54 & 49{,}2 & 52{,}6 \\
\end{tabular}
\end{center}
Số trung bình của mẫu số liệu ghép nhóm trên xấp xỉ bằng
\choice
{\True $55{,}6$}
{$65{,}5$}
{$48{,}8$}
{$57{,}7$}
\loigiai{
Khoảng biến thiên của mẫu số liệu trên là $R=63{,}6-45{,}1 = 18{,}5 $.\\
Độ dài mỗi nhóm là $L>\dfrac{R}{k}=3{,}7 $.\\
Ta chọn $L=4{,}0$ và chia dữ liệu thành các nhóm và có bảng giá trị đại diện như sau
\begin{center}
\begin{tabular}{|l|c|c|c|c|c|}
\hline
Nhóm & $[45 ; 49 )$ & $[49 ; 53 )$ & $[53 ; 57 )$ & $[57 ; 61 )$ & $[61 ; 65 )$ \\
\hline
Giá Trị Đại Diện & $47 $ & $51 $ & $55 $ & $59 $ & $63 $ \\
\hline
Tần Số &$ 4$ & $5$ & $7$ & $7$ & $5$ \\
\hline
\end{tabular}
\end{center}
Giá trị trung bình của bảng số liệu là
$\overline{x}=\dfrac{4 \cdot 47 + 5 \cdot 51 + 7 \cdot 55 + 7 \cdot 59 + 5 \cdot 63{,}0}{28} \approx 55{,}57.$
}
\end{ex}

\begin{ex}%[1H4H6-2]%[Dự án đề ôn tập khối 11 HKI NH2023-2024-Đơt 1-TheHung Nguyen]%[CD-Đề số 1]
\immini{Cho hình chóp $S . A B C D$ có đáy là hình bình hành, gọi $M$ là trung điểm của $S C$ (như hình vẽ).
Hình chiếu song song của điểm $M$ theo phương $A C$ lên mặt phẳng $(S A D)$ là điểm nào sau đây?
\choice
{Trung điểm của $S B$}
{Trung điểm của $S D$}
{Điểm $D$}
{\True Trung điểm của $S A$}}{\begin{tikzpicture}[scale=.5, font=\footnotesize, line join=round, line cap=round, >=stealth]
\def\a{3}
\def\b{2.5}
\def\h{3.5}
\def\g{35}
\path
(0,0) coordinate (A)
(0:\a) coordinate (B)
--++(\g:\b) coordinate (C)
--++(180:\a)coordinate (D)
;
\coordinate (O) at ($(A)!1/2!(C)$);
\coordinate (S) at ($(O)+(0,\h)$);
\coordinate (M) at ($(S)!1/2!(C)$);
\draw (S)--(A)--(B)--(C) (B)--(S)--(C);
\draw[dashed] (B)--(D)--(S)  (A)--(D)--(C);
\foreach \x/\g in {A/-90,B/-90,C/0,D/180,S/90,M/70}  \fill (\x) circle (1pt)+(\g:.25)node {$\x$};
\end{tikzpicture}}
\loigiai{
\immini{Gọi $N$ là trung điểm $S A$.\\
Khi đó $M N \parallel A C$ nên hình chiếu song song của điểm $M$ lên mặt phẳng $(S A D)$ là trung điểm $S A$.}{\begin{tikzpicture}[scale=.75, font=\footnotesize, line join=round, line cap=round, >=stealth]
\def\a{3}
\def\b{2.5}
\def\h{3.5}
\def\g{35}
\path
(0,0) coordinate (A)
(0:\a) coordinate (B)
--++(\g:\b) coordinate (C)
--++(180:\a)coordinate (D)
;
\coordinate (O) at ($(A)!1/2!(C)$);
\coordinate (S) at ($(O)+(0,\h)$);
\coordinate (M) at ($(S)!1/2!(C)$);
\coordinate (N) at ($(S)!1/2!(A)$);
\draw (S)--(A)--(B)--(C) (B)--(S)--(C);
\draw[dashed] (B)--(D)--(S)  (A)--(D)--(C)--(A) (M)--(N);
\foreach \x/\g in {A/-90,B/-90,C/0,D/180,S/90,M/70,N/110}  \fill (\x) circle (1pt)+(\g:.25)node {$\x$};
\end{tikzpicture}}

}
\end{ex}

\begin{ex}%[Team Tex Hóa -- Dương Văn Đức]%[1D3V3-3]
Tìm $m$ để hàm số $f(x)=\heva{&\dfrac{x^2-1}{x-1}& \text{khi~} x\ne 1\\& m+2& \text{khi~}x=1}$ liên tục tại điểm $x_0=1$.
\choice
{$m=3$}
{\True $m=0$}
{$m=4$}
{$m=1$}
\loigiai{
Ta có $\lim \limits_{x\to1}\dfrac{x^2-1}{x-1} =\lim \limits_{x\to1}\dfrac{\left(x-1\right)\left(x+1\right)}{\left(x-1\right)} =\lim \limits_{x\to1}\left(x+1\right)=2$.\\
Để hàm số liên tục tại $x_0=1$ khi và chỉ khi $\lim \limits_{x\to 1} f(x)=f(1)\Leftrightarrow 2=m+2\Leftrightarrow m=0$.}
\end{ex}

\begin{ex}%[1D5N1-2]%[Dự án đề kiểm tra Toán 11 GHKI NH23-24- Nguyễn Văn Sang]%[ĐỀ KNTT SỐ 5]
Mẫu số liệu sau cho biết cân nặng của học sinh lớp $12$ trong một lớp
\begin{center}
\begin{tabular}{|c|c|c|c|}
\hline Cân nặng $(\mathrm{kg})$ & Dưới $55$ & Từ $55$ đến $65$ & Trên $65$ \\
\hline Số học sinh & $23$ & $15$ & $2$ \\
\hline
\end{tabular}
\end{center}
Số học sinh của lớp đó là bao nhiêu?
\choice
{\True $40$}
{$35$}
{$23$}
{$38$}
\loigiai{
Số học sinh của lớp là $n=23+15+2=40$ học sinh.
}
\end{ex}

\begin{ex}%[1D1H4-6]
Tập giá trị của hàm số $y=\sin^2x+2\cos^2x$ là
\choice
{$T=\left[0;3\right]$}
{$T=\left[0;2\right]$}
{\True $T=\left[1;2\right]$}
{$T=\left[1;3\right]$}
\loigiai{
Ta có $y=\sin^2x+2\cos^2x=1+\cos^2x$.\\
Ta có $0\le \cos^2x\le 1$ nên $1\le 1+\cos^2x\le 2$.\\
Vậy tập giá trị của hàm số $y=\sin^2x+2\cos^2x$ là $T=\left[1;2\right]$.
}
\end{ex}

\begin{ex}%[1D2H1-3]
Cho dãy số $\left(u_n\right)$, biết $u_n=\dfrac{2 n+5}{5 n-4}$. Số $\dfrac{7}{12}$ là số hạng thứ mấy của dãy số?
\choice
{\True $8$}
{$6$}
{$9$}
{$10$}
\loigiai{
Giả sử $u_n = \dfrac{7}{12} \Leftrightarrow \dfrac{2 n+5}{5 n-4} = \dfrac{7}{12} \Leftrightarrow n = 8.$
}
\end{ex}

\begin{ex}%[Pj10-1-HK1-NH23-24(CD)--TeamTeXHoa--Võ Thị Thùy Trang]%[1D1H4-3]
Hàm số $y=\sin x$ đồng biến trên khoảng nào dưới đây?
\choice
{$\left(-\pi;\dfrac{\pi}{2}\right)$}
{\True $\left(-\dfrac{\pi}{2};0\right)$}
{$\left(0;\pi \right)$}
{$\left(\dfrac{\pi}{2};\pi \right)$}
\loigiai{
Dựa vào đồ thị hàm số $y=\sin x$ ta thấy đồ thị hướng đi lên từ trái sang phải trên $\left(-\dfrac{\pi}{2};0\right)$.\\
Nên hàm số đồng biến trên khoảng $\left(-\dfrac{\pi}{2};0\right)$.  }
\end{ex}

\begin{ex}%[1H4H1-4]
Cho tứ diện $ABCD$. Gọi $G$ là trọng tâm tam giác $BCD$, $M$ là trung điểm $CD$, $I$ là điểm ở trên đoạn thẳng $AG$, $BI$ cắt mặt phẳng $(ACD)$ tại $J$. Khẳng định nào sau đây \textbf{sai}?
\choice
{$AM=(ACD) \cap(ABG)$}
{$A$, $J$, $M$ thẳng hàng}
{\True $J$ là trung điểm của $AM$}
{$DJ=(ACD) \cap(BDJ)$}
\loigiai{
\immini{    Ta có $A$ là điểm chung thứ nhất giữa hai mặt phẳng $(A C D)$ và $(G A B)$.\\
Do $BG \cap CD=M \Rightarrow \heva{&M \in BG \subset(ABG) \\ &M \in CD \subset(ACD)} \Rightarrow \heva{&M \in(ABG)\\ &M \in(ACD)}$\\
$\Rightarrow M$ là điểm chung thứ hai giữa hai mặt phẳng $(ACD)$ và $(GAB)$.\\
$\Rightarrow(ABG) \cap(ACD)=AM.$\\
Ta có $\heva{&BI \subset(ABG)\\& AM\subset(ABM)\\&(ABG) \equiv(ABM)} \Rightarrow AM$, $BI$ đồng phẳng.\\
$\Rightarrow J=B I \cap A M \Rightarrow A$, $J$, $M$ thẳng hàng.
}{
\begin{tikzpicture}[scale=0.7, font=\footnotesize,line join=round, line cap=round, >=stealth]
\coordinate (B) at (-2,0);
\coordinate (C) at (2,-2.5);
\coordinate (D) at (4,0);
\coordinate (M) at ($(C)!0.5!(D)$);
\coordinate (G) at ($(B)!2/3!(M)$);
\coordinate (A) at ($(G)+(0,5)$);
\coordinate (J) at ($(A)!2/3!(M)$);
\coordinate (I) at (intersection of B--J and A--G);
\foreach \i in {B,C,D,M}{\draw (A)--(\i);}
\draw (B)--(C)--(D);
\draw[dashed,thin]  (M)--(B)--(D) (A)--(G) (B)--(J);
\foreach \i/\g in {A/90,B/180,C/-90,D/0,M/-60,G/-90,I/-30,J/30}{\draw[fill=black](\i) circle (1.5pt) ($(\i)+(\g:4mm)$) node[scale=1]{$\i$};}
\end{tikzpicture}
}
\noindent   Ta có $\heva{&D J \subset(ACD) \\ &DJ \subset(BDJ)} \Rightarrow DJ=(ACD) \cap(BDJ).$\\
Điểm $I$ di động trên $AG$ nên $J$ có thể không phải là trung điểm của $AM$.
}
\end{ex}

\begin{ex}.%[1H4V1-3]
Cho tứ giác $A B C D$ và một điểm $S$ không thuộc mặt phẳng $(A B C D)$. Trên đoạn $S C$ lấy một điểm $M$ không trùng với $S$ và $C$. Gọi $N$ là giao điểm của đường thẳng $S D$ với mặt phẳng $(A B M)$. Khi đó $A N$ là giao tuyến của hai mặt phẳng nào sau đây?
\choice
{  $A N=(A B M) \cap(S B C)$}
{ $A N=(A B M) \cap(S C D)$}
{\True $A N=(A B M) \cap(S A D)$}
{ $A N=(A B M) \cap(S A C)$}
\loigiai{
\immini{Ta có $B \in(A B M) \cap(S B D).\quad(1)$\\
Gọi $O=A C \cap B D, K=A M \cap S O$. Khi đó\\
$
\heva{&K \in A M \subset(A B M) \\
&K \in S O \subset(S B D)} \Rightarrow K \in(A B M) \cap(S B D).\quad(2)    $\\
Từ (1) và $(2)$ suy ra $(A B M) \cap(S B D)=B K$.\\
Trong mặt phẳng $(S B D)$, gọi $N=B K \cap S D$. Khi đó\\
$
\heva{  &N \in S D \\
&N \in B K \subset(A B M)}
\Rightarrow N=(A B M) \cap S D .$\\
Dễ thấy $ A N=(A B M) \cap (S A D).$}
{\begin{tikzpicture}[>=stealth,line join=round, line cap=round, scale=1]
\coordinate[label=above left:$S$] (S) at (1,4) ;
\coordinate [label=below left:$B$](B) at (0,-0.5) ;
\coordinate [label=below left:$A$](A) at (-1,1) ;
\coordinate [label=below:$C$](C) at (3,-1.5) ;
\coordinate [label=right:$D$](D) at (5,1);
\coordinate[label=below  right:$M$](M) at ($(S)!1/3!(C)$);
\coordinate [label=below:$O$] (O) at (intersection of A--C and B--D);
\coordinate [label=below right:$K$] (K) at (intersection of A--M and S--O);
\coordinate [label=above right:$N$] (N) at (intersection of B--K and S--D);
%   \coordinate[label=above right:$N$] (N) at ($(S)!0.5!(D)$);
\draw (S)--(B)--(C)--(D)--cycle (B)--(A)--(S)--(C) ;
\draw[dashed] (B)--(D)--(A)--(C) (S)--(O) (A)--(M) (B)--(N)
;
\foreach \diem in {S,A,B,C,D, M, N,O,K} \fill (\diem)circle(1.5pt);
\end{tikzpicture}
}}

\end{ex}

\begin{ex}%[1H4H3-2]
Cho hình chóp tứ giác $S.ABCD$. Gọi $M$ và $N$ lần lượt là trung điểm của $SA$ và $SC$. Đường thẳng $MN$ song song với mặt phẳng nào dưới đây?
\choice
{Mặt phẳng $(SCD)$}
{Mặt phẳng $(SAB)$}
{Mặt phẳng $(SBC)$}
{\True Mặt phẳng $(ABCD)$}
\loigiai{
\immini{Ta có $M$ và $N$ lần lượt là trung điểm của $SA$ và $SC$ nên $MN$ là đường trung bình của $\triangle SAC$, suy ra $MN\parallel AC.$\\
Khi đó, $\left\{\begin{aligned}
& MN\parallel AC \\
& AC\subset (ABCD) \\
& MN\not\subset (ABCD) \\
\end{aligned} \right.\Rightarrow MN\parallel (ABCD)$.}
{
\begin{tikzpicture}[line join=round, line cap = round, >=stealth, scale=0.7,font=\footnotesize]
\path
(0,0) coordinate (A)
(5,0) coordinate (D)
(1,-2.5) coordinate (B)
(4,-3) coordinate (C)
(2,3.3) coordinate (S)
($(S)!0.5!(A)$) coordinate (M)
($(S)!0.5!(C)$) coordinate (N)
;
\draw[dashed] (A)--(D) (M)--(N) (A)--(C);
\draw (S)--(A)--(B)--(C)--(D)--cycle (S)--(B) (S)--(C);
\foreach \x/\g in
{S/90,A/165,S/90,B/-90,C/-90,D/0,M/180,N/45,D/0}
\fill[black](\x) circle (1pt)
($(\x)+(\g:3mm)$)node{$\x$};

\end{tikzpicture}
}
}
\end{ex}

\begin{ex}%[Pj10-Đề 07 HK1 NH2023-2024 CTST]%[Nguyễn Văn Nay]%[1H4N3-1]
Cho hai mặt phẳng $(P)$, $(Q)$ cắt nhau theo giao tuyến là đường thẳng $d$. Đường thẳng $a$ song song với cả hai mặt phẳng $(P)$, $(Q)$. Khẳng định nào sau đây đúng?
\choice
{$a$, $d$ trùng nhau}
{$a$, $d$ chéo nhau}
{\True $a$ song song $d$}
{$a$, $d$ cắt nhau}
\loigiai{
Sử dụng hệ quả: Nếu hai mặt phẳng phân biệt cùng song song với một đường thẳng thì giao tuyến của chúng cũng song song với đường thẳng đó.
}
\end{ex}

\begin{ex}%[1H4N1-3]
Cho hình chóp $S.ABCD$, đáy $ABCD$ là hình thang có $2$ đáy là $AD$ và $BC$. Gọi $M$, $N$ lần lượt là trung điểm của $SB$, $SC$, $O$ là giao điểm của $AC$ và $BD$. Giao tuyến của hai mặt phẳng $(AMN)$ và $(SBD)$ là
\choice
{$DN$}
{\True $DM$}
{$OM$}
{$SO$}
\loigiai{
\immini{ Ta có $MN$ là đường trung bình của tam giác $SBC$, suy ra $MN\parallel BC$.\\
Ta lại có $BC\parallel AD$, suy ra $MN\parallel AD$.\\
Khi đó $(AMN) \equiv (AMND)\Rightarrow (AMN)\cap (SBD) = MD$.}
{\begin{tikzpicture}[line join=round, line cap=round,thick,scale =0.7]
\coordinate (S) at (1,4);
\coordinate (A) at (-1,0);
\coordinate (D) at (-1,-3);
\coordinate (C) at (3,-3);
\coordinate (B) at (3,0);
\coordinate (M) at ($(S)!0.5!(B)$);
\coordinate (N) at ($(S)!0.5!(C)$);
\draw(S)--(A) (S)--(D) (S)--(C) (S)--(B) (A)--(D)--(C)--(B) (M)--(N);
\draw[dashed,thin](A)--(B) (B)--(D) (A)--(C) (A)--(M) (A)--(N);
\path (intersection of A--C and B--D) coordinate (O) node[below]{$O$}; % AC\cap BD = O
\foreach \i/\g in {S/90,A/180,D/-90,C/-90,B/0,M/0,N/0}{\draw[fill=white](\i) circle (1.5pt) ($(\i)+(\g:3mm)$) node[scale=1]{$\i$};}
\end{tikzpicture}
}
}
\end{ex}

\begin{ex}%[1H4N4-1]%[Pj10-1-HK1-NH23-24-TeamTeXHoa-VUNgocHao]
Hai mặt phẳng được gọi là song song nếu
\choice
{Có một đường thẳng nằm trong mặt phẳng này và song song với mặt phẳng kia}
{Chúng có duy nhất một điểm chung}
{Chúng có ít nhất hai điểm chung}
{\True Chúng không có điểm chung}
\loigiai{Hai mặt phẳng được gọi là song song nếu chúng không có điểm chung.
}
\end{ex}


\noindent\textbf{II. PHẦN TỰ LUẬN:}
\begin{ex}%[1D1H5-3]%[Dự án đề kiểm tra Toán 11 GHKI NH23-24- Võ Thị Thùy Trang]%[THPT Võ Thị Sáu - Tp HCM]
Giải phương trình sau $\sin 2x+3\cos x=0$.
\loigiai{
Ta có\\
\allowdisplaybreaks
$\begin{aligned}
&\sin 2x+3\cos x=0\\
\Leftrightarrow& 2\sin x\cos x+3\cos x=0\\
\Leftrightarrow& \cos x(2\sin x+3)=0\\
\Leftrightarrow& \hoac{& \cos x=0 \\& 2\sin x+3=0}\\
\Leftrightarrow& \hoac{& \cos x=0 \\& \sin x=-\dfrac{3}{2}\text{ (loại)}}\\
\Leftrightarrow& x=\dfrac{\pi }{2}+k\pi ,k\in \mathbb{Z}.
\end{aligned}$
}
\end{ex}

\begin{ex}%[Pj08-0-GK1-NH23-24--TeamTeXHoa--LeQuan]%[1D3V2-5]
Tính giới hạn  $\lim\limits_{x\to 1}\dfrac{x^3-\sqrt{3x-2}}{x^2-1}$
\loigiai{
Ta có
\begin{eqnarray*}
\lim\limits_{ x\to 1}\dfrac{x^3-\sqrt{3x-2}}{x^2-1}
&=&\lim\limits_{ x\to 1}\dfrac{x^6-3x+2}{\left(x^2-1\right)\left(x^3+\sqrt{3x-2}\right)}\\
&=& \lim\limits_{x\to 1}\dfrac{x^6-1-3x+3}{\left(x-1\right)\left(x+1\right)\left(x^3+\sqrt{3x-2}\right)}\\
&=&\lim\limits_{ x\to 1}\dfrac{\left(x^3+1\right)\left(x^3-1\right)-3\left(x-1\right)}{\left(x-1\right)\left(x+1\right)\left(x^3+\sqrt{3x-2}\right)}\\
&=&\lim\limits_{ x\to 1}\dfrac{\left(x-1\right)\left[\left(x^3+1\right)\cdot\left(x^2+x+1\right)-3\right]}{\left(x-1\right)\left(x+1\right)\left(x^3+\sqrt{3x-2}\right)}\\
&=& \lim\limits_{ x\to 1}\dfrac{\left[\left(x^3+1\right)\cdot\left(x^2+x+1\right)-3\right]}{\left(x+1\right)\left(x^3+\sqrt{3x-2}\right)}\\
&=&\dfrac{\left(1+1\right)\cdot\left(1+1+1\right)-3}{\left(1+1\right)\cdot\left(1+1\right)}\\
&=&\dfrac{3}{4}.
\end{eqnarray*}
}\end{ex}

\begin{ex}%[1D3T1-5]
Tại một nhà máy, người ta đo được rằng $80\%$ lượng nước sau khi sử dụng được xử lí và tái sử dụng. Với $100$ m$^3$ ban đầu được sử dụng lần đầu tại nhà máy, khi quá trình xử lí và tái sử dụng lặp lại mãi mãi, nhà máy sử dụng được tổng lượng nước là bao nhiêu?
\loigiai{
Tổng lượng nước sử dụng là $100+100 \cdot 0{,}8+100 \cdot(0{,}8)^2+100 \cdot(0{,}8)^3+\ldots=100 \cdot \dfrac{1}{1-0{,}8}=500~\left(\mathrm{m}^3\right).$
}
\end{ex}

\begin{ex} %[1H4C4-6]
Hai hình vuông $ABCD$ và $ABEF$ ở trong hai mặt phẳng khác nhau. Trên các đường chéo $AC$ và $BF$ lần lượt lấy các điểm $M,N$ sao cho $AM=BN$. Các đường thẳng song song với $AB$ vẽ từ $M,N$ lần lượt cắt $AD,AF$ tại ${M}',{N}'$.
\begin{enumerate}
    \item Chứng minh $\left( BCE \right)\text{//}\left( ADF \right)$.
    \item Chứng minh $\left( DEF \right)\text{//}\left( MN{N}'{M}' \right)$. 
    \item Gọi $I$ là trung điểm của $MN$. Tìm tập hợp điểm $I$ khi $M,N$ thay đổi trên $AC$ và $BF$. 
\end{enumerate}

\end{ex}
\Closesolutionfile{ans}
