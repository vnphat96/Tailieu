\begin{name}
	{\tenchude}
	{\tendethi}
	{\tentruong}
	{\thoigian}
\end{name}

\caulc
\Opensolutionfile{ans}[ans/10-CK1-Chuyen-Hung-Vuong-Phan-1]
\begin{ex}%[1D1N4-2]%Câu 1
	Tập xác định của hàm số $y=\tan x$ là
	\choice
	{$\mathscr{D}=\mathbb{R}$}
	{\True $\mathscr{D}=\mathbb{R}\setminus \{\dfrac{\pi}{2}+k\pi |k\in \mathbb{Z}\}$}
	{$\mathscr{D}=\mathbb{R}\setminus \{k\pi |k\in \mathbb{Z}\}$}
	{$\mathscr{D}=\mathbb{R}\setminus \{0\}$}
	\loigiai{
		Ta có $y=\tan x=\dfrac{\sin x}{\cos x}$ nên hàm số xác định khi $\cos x\ne 0\Leftrightarrow x\ne \dfrac{\pi }{2}+k\pi, k\in \mathbb{Z}$.
		}
\end{ex}

\begin{ex}%[1D1H5-3]%Câu 2
	Nghiệm của phương trình $\cos 2x=-1$ là
	\choice
	{$x=\pi +k2\pi$, $k\in \mathbb{Z}$}
	{$x=\pi +k\pi$, $k\in \mathbb{Z}$}
	{\True $x=\dfrac{\pi}{2}+k\pi$, $k\in \mathbb{Z}$}
	{$x=\dfrac{k\pi}{2}$, $k\in \mathbb{Z}$}
	\loigiai{
			Ta có $\cos 2x=-1 \Leftrightarrow 2x=\pi +k2\pi \Leftrightarrow x=\dfrac{\pi }{2}+k\pi$, $k\in \mathbb{Z}$.
			}
\end{ex}

\begin{ex}%[1D2H2-3]%Câu 3
	Cho cấp số cộng $\left(u_n\right)$ có $u_1=1$, $u_2=4$. Tìm $u_4$.
	\choice
	{$u_4=5$}
	{$u_4=13$}
	{\True $u_4=9$}
	{$u_4=12$}
\loigiai{Ta có $u_1=1$, $u_2=4\Rightarrow d=u_2-u_1=3$.\\
Khi đó $u_4=u_1+3d=1+3\cdot3=9$.}
\end{ex}

\begin{ex}%[1D2H3-3]%Câu 4
	Tổng của chín số hạng đầu của cấp số nhân $20$; $10$; $5$; $\ldots$ với kết quả làm tròn đến hàng phần mười là
	\choice
	{\True $S_9\approx 39{,}9$}
	{$S_9\approx 39{,}8$}
	{$S_9\approx 59{,}9$}
	{$S_9\approx 40$}
	\loigiai{
		Ta có $u_1=20$, $u_2=10\Rightarrow q=\dfrac{u_2}{u_1}=\dfrac{1}{2}$.\\
Khi đó $S_9=u_1\cdot \dfrac{1-q^9}{1-q}\approx 39{,}9$.}
\end{ex}

\begin{ex}%[1D3H1-2]%Câu 5
	Dãy số nào sau đây có giới hạn bằng $0$?
	\choice
	{\True $u_n=\dfrac{3n}{2n^2+1}$}
	{$u_n=2n-5n^2$}
	{$u_n=\dfrac{5n+1}{7n+13}$}
	{$u_n=\dfrac{2n^2+3n-1}{n^2+7n-3}$}
	\loigiai{
		Ta có $\lim\dfrac{3n}{2n^2+1}=\lim\dfrac{3n}{n^2\left(2+\dfrac{1}{n^2}\right)}=\lim\dfrac{3}{n\left(2+\dfrac{1}{n^2}\right)}=0$.}
\end{ex}


\begin{ex}%[1D3H2-1]%Câu 6
	Nếu $\lim\limits_{x\to 3}f(x)=4$ thì $\lim\limits_{x\to 3}\left[4-3f(x)\right]$ bằng
	\choice
	{$9$}
	{$8$}
	{\True $-8$}
	{$-9$}
	\loigiai{
		Ta có $\lim\limits_{x\to 3}\left[4-3f(x)\right]=4-3 \cdot 4=-8$.}
\end{ex}


\begin{ex}%[1D3H2-4]%Câu 7
	Tính $\lim\limits_{x\to-\infty}\left(3x^3-2x^2+5\right)$.
	\choice
	{$3$}
	{$0$}
	{$+\infty$}
	{\True $-\infty$}
	\loigiai{
		Ta có $\lim\limits_{x\to-\infty}\left(3x^3-2x^2+5\right)=\lim\limits_{x\to-\infty}\left[x^3\left(3-\dfrac{2}{x}+\dfrac{5}{x^3}\right)\right]=-\infty$.}
\end{ex}

\begin{ex}%[1D3H3-3]%Câu 8
	Hàm số nào sau đây gián đoạn tại $x=1$?
	\choice
	{$y=\dfrac{x^2+2x-1}{x+1}$}
	{$y=\dfrac{2x+1}{x^2+1}$}
	{$y=x^3+x+1$}
	{\True $y=\dfrac{x+3}{x^2-1}$}
	\loigiai{
		Do hàm số $y=\dfrac{x+3}{x^2-1}$ không xác định tại $x=1$ nên hàm số gián đoạn tại $x=1$.}
\end{ex}

\begin{ex}%[1H4H1-3]%Câu 9
	Cho tứ diện $ABCD$. Gọi $I$ và $J$ theo thứ tự là trung điểm của $AD$ và $AC$, $G$ là trọng tâm tam giác $BCD$. Giao tuyến của hai mặt phẳng $\left(GIJ\right)$ và $\left(BCD\right)$ là đường thẳng
	\choice
	{qua $I$ và song song với $AB$}
	{qua $J$ và song song với $BD$}
	{\True qua $G$ và song song với $CD$}
	{qua $G$ và song song với $BC$}
	\loigiai{
		\begin{center}
			\begin{tikzpicture}
				\def\a{4}
				\path 	(0:0) coordinate (B)
						++(0:\a) coordinate (D)
						++(-120:\a/2) coordinate (C)
						($(B)+(70:\a)$) coordinate (A)
						($(A)!0.5!(C)$) coordinate (I)
						($(A)!0.5!(D)$) coordinate (J)
						($(D)!0.5!(C)$) coordinate (M)
						($(B)!2/3!(M)$) coordinate (G)
						($(B)!2/3!(C)$) coordinate (G1)
						($(B)!2/3!(D)$) coordinate (G2);
				\draw[dashed,thick] (B)--(D) (G1)--(G2) (B)--(M);
				\draw[thick] (B)--(A)--(D) (A)--(C) (B)--(C)--(D) (I)--(J);
				\foreach \x/\g in {A/90,B/180,C/-45,D/0,I/180,J/0,G/120}
						\fill[black] (\x) circle (1.5pt)
						($(\g:3mm)+(\x)$) node {$\x$};
			\end{tikzpicture}
		\end{center}
		Gọi $d$ là giao tuyến của $\left(GIJ\right)$ và $\left(BCD\right)$.\\
		Ta có $G\in\left(GIJ\right)\cap\left(BCD\right)$, $IJ \parallel CD$, $IJ \subset \left(GIJ\right)$, $CD \subset \left(BCD\right)$.\\
		Suy ra $d$ đi qua $G$ và song song với $CD$.}
\end{ex}

\begin{ex}%[1H4H3-2]%Câu 10
	Cho hình chóp $S.ABCD$ có đáy là hình bình hành. Các điểm $I$, $J$ lần lượt là trọng tâm các tam giác $SAB$ và $SAD$. Gọi $M$ là trung điểm $CD$. Chọn mệnh đề \textbf{đúng} trong các mệnh đề sau
	\choice
	{\True $IJ \parallel \left(SBD\right)$}
	{$IJ \parallel \left(SBM\right)$}
	{$IJ \parallel \left(SCD\right)$}
	{$IJ \parallel \left(SBC\right)$}
	\loigiai{
		\begin{center}
			\begin{tikzpicture}
				\def\a{6}
				\path (0:0) coordinate (A)
				++(0:\a) coordinate (D)
				++(-150:\a/2) coordinate (C)
				($(A)+(C)-(D)$) coordinate (B)
				($(A)+(110:\a)$) coordinate (S)
				(intersection of A--C and B--D) coordinate (O)
				($(A)!0.5!(B)$) coordinate (N)
				($(A)!0.5!(D)$) coordinate (P)
				($(D)!0.5!(C)$) coordinate (M)
				($(S)!2/3!(N)$) coordinate (I)
				($(S)!2/3!(P)$) coordinate (J);
				\draw[dashed,thick] (B)--(A)--(D) (A)--(S) (S)--(N) (S)--(P) (I)--(J) (N)--(P);
				\draw[thick] (B)-- (C)--(D) (B)--(S) (C)--(S) (D)--(S);
				\foreach \x/\g in {A/45,B/-135,C/-45,D/45,S/90,I/180,J/30,N/170,P/40,M/-30}
				\fill[black] (\x) circle (1pt) ($(\g:3mm)+(\x)$) node {$\x$};
			\end{tikzpicture}
		\end{center}
		Gọi $N, P$ lần lượt là trung điểm của $AB,AD$.\\
		Ta có $\dfrac{SI}{SN}=\dfrac{SJ}{SP}=\dfrac{1}{3} \Rightarrow IJ \parallel NP$ mà $NP \parallel BD$ suy ra $IJ \parallel \left(SBD\right)$.
		}
\end{ex}


\begin{ex}%[1H4H6-1]%Câu 11
	Cho tam giác $ABC$ ở trong mặt phẳng $(\alpha)$ và phương $l$. Biết hình chiếu (theo phương $l$) của tam giác $ABC$ lên mặt phẳng $(P)$ là một đoạn thẳng. Khẳng định nào sau đây \textbf{đúng}?
	\choice
{$(\alpha) \parallel (P)$}
{$(\alpha)\equiv (P)$}
{\True $(\alpha) \parallel l$ hoặc $(\alpha) \supset l$}
{$l \subset (P)$}
\loigiai
{Khi phương chiếu $l$ thỏa mãn $(\alpha) \parallel l$ hoặc $(\alpha) \supset l$ thì các đoạn thẳng $AB$, $BC$, $CA$ có hình chiếu lên $(P)$ nằm trên giao tuyến của $(\alpha)$ và $(P)$.
}
\end{ex}

\begin{ex}%[1D5N1-3]%Câu 12
Bảng thống kê số lỗi chính tả trong bài kiểm tra giữa học kì $1$ môn Ngữ Văn của học sinh khối $11$ như sau:
	\begin{center}
	\begin{tabular}{|c|c|c|c|c|c|c|}
		\hline
		Số lỗi &$\left[1;2\right)$&$[3;4)$&$[5;6)$&$[7;8)$&$[9;10)$&\\
		\hline
		Số bài &$122$&$75$&$14$&$5$&$2$&$N=218$\\
		\hline
	\end{tabular}
\end{center}
Số trung bình cộng của mẫu số liệu trên (làm tròn đến hàng phần chục) là
\choice
{$2{,}6$}
{$2{,}9$}
{\True $2{,}7$}
{$2{,}8$}
\loigiai{
	\begin{center}
	\begin{tabular}{|c|c|c|c|c|c|c|}
		\hline
		Số lỗi&$[1;2)$&$[3;4)$&$[5;6)$&$[7;8)$&$[9;10)$&\\
		\hline
		Giá trị đại diện&$1{,}5$&$3{,}5$&$5{,}5$&$7{,}5$&$9{,}5$&\\
		\hline
		Số bài&$122$&$75$&$14$&$5$&$2$&$N=218$\\
		\hline
	\end{tabular}
\end{center}
Số trung bình cộng của mẫu số liệu trên là\\
$$\overline x=\dfrac{122\cdot 1{,}5+75\cdot 3{,}5+14\cdot 5{,}5+5\cdot 7{,}5+2\cdot 9{,}5}{218}=\dfrac{579}{218}\approx 2{,}7.$$
}
\end{ex}

\Closesolutionfile{ans}
% \indapan{7}{ans/10-CK1-Chuyen-Hung-Vuong-Phan-1}

\Opensolutionfile{ans}[ans/ans-0-B15-DS]
\cauds
\begin{ex}%Câu 1.%[1D3H1-2]
	 Cho hai dãy số $\left(u_n\right)$, $\left(v_n\right)$ với $u_n=4\cdot3^n-7^{n+1}$; $v_n=7^n$. Khi đó
\choiceTF
{\True $\lim \dfrac{1}{v_n}=0$}
{\True $\lim v_n=+\infty$}
{\True $\lim \dfrac{u_n-v_n}{3 u_n+2 v_n}=\dfrac{8}{19}$}
{$\lim u_n=+\infty$}
\loigiai{\begin{itemchoice}
\itemch \textbf{Đúng}.  
 $\lim \dfrac{1}{v_n}=\lim \left(\dfrac{1}{7}\right)^n=0$. 
\itemch \textbf{Đúng}. 
$\lim 7^n=+\infty$. 
\itemch \textbf{Đúng}. 
\allowdisplaybreaks
\begin{eqnarray*}
	\lim \dfrac{u_n-v_n}{3 u_n+2 v_n}&=&\lim \dfrac{4 \cdot 3^n-8 \cdot 7^n}{12 \cdot 3^n-19\cdot7^n}\\
	&=&\lim \dfrac{4 \cdot\left(\dfrac{3}{7}\right)^n-8}{12 \cdot\left(\dfrac{3}{7}\right)^n-19}\\
	&=&\lim \dfrac{0-8}{0-19}=\dfrac{8}{19}.
	\end{eqnarray*} 
\itemch \textbf{Sai}. 
 $\lim u_n=\lim \left(4\cdot3^n-7^{n+1}\right)=\lim 7^n\left[4\left(\dfrac{3}{7}\right)^n-7\right]$.\\
$\lim 7^n=+\infty$; $\lim \left[4\left(\dfrac{3}{7}\right)^n-7\right]=-7<0 \Rightarrow \lim u_n=-\infty$. 
\end{itemchoice}
}
\end{ex}
\begin{ex}%Câu 2. %[1D3H2-2]
	Cho hàm số $f(x)=\heva{&x-2 & \text{ khi } x<-1 \\ &\sqrt{x^2+1} & \text{ khi } x \geq-1}$. Khi đó.
\choiceTF
{$\lim\limits_{x \rightarrow-2} f(x)=\sqrt{5}$}
{\True $\lim\limits_{x \rightarrow-1^-} f(x)=-3$}
{\True $\lim\limits_{x \rightarrow-1^+} f(x)=\sqrt{2}$}
{Hàm số tồn tại giới hạn khi $x \rightarrow-1$}
\loigiai{\begin{itemchoice}
		\itemch \textbf{Sai}. 
		 Ta có $\lim\limits_{x \rightarrow-2} f(x)=\lim\limits_{x \rightarrow-2}(x-2)=-2-2=-4$.
\itemch \textbf{Đúng}. 
Ta có $\lim\limits_{x \rightarrow-1^-} f(x)=\lim\limits_{x \rightarrow-1^-}(x-2)=-1-2=-3$.
\itemch \textbf{Đúng}. 
Khi đó $\lim\limits_{x \rightarrow-1^+} f(x)=\lim\limits_{x \rightarrow-1^+} \sqrt{x^2+1}=\sqrt{(-1)^2+1}=\sqrt{2}$. 
	\itemch \textbf{Sai}.  
Vì $\lim\limits_{x \rightarrow-1^-} f(x) \neq \lim\limits_{x \rightarrow-1^+} f(x)$ (hay $-3 \neq \sqrt{2}$ ) nên không tồn tại $\lim\limits_{x \rightarrow-1} f(x)$. 
\end{itemchoice}}
\end{ex}
\begin{ex}% Câu 3. %[1D3H3-4]
	Cho hàm số $y=f(x)=\heva{&\dfrac{x^2-2025}{x-45} & \text { khi } x \neq 45 \\& 2 m+4 & \text { khi } x=45}$ ($m$ là tham số). Khi đó.
	\choiceTF
{Tập xác định của hàm số $\mathbb{R} \backslash\{45\}$}
{\True $\lim\limits_{x \rightarrow 45} f(x)=90$}
{\True Hàm số liên tục tại $x=20$ với mọi $m$}
{Hàm số liên tục trên $\mathbb{R}$ khi $m=44$}
\loigiai{\begin{itemchoice}
		\itemch \textbf{Sai}.
		Hàm số  $f ( x )$ xác định trên $\mathbb{R}$.
\itemch \textbf{Đúng}. 
Ta có $\lim\limits_{x \rightarrow 45} \dfrac{x^2-2025}{x-45}=\lim\limits_{x \rightarrow 45}(x+45)=90$. 
\itemch \textbf{Đúng}.
Ta có $f(20)=65$
$\lim\limits_{x \rightarrow 20} \dfrac{x^2-2025}{x-45}=\lim\limits_{x \rightarrow 20}(x+45)=65=f(20)$, nên $f(x)$ liên tục tại $x=20$.
\itemch \textbf{Sai}. 
Với $x \neq 45$ thì $f(x)=\dfrac{x^2-2025}{x-45}$ hàm số xác định trên khoảng $(-\infty ; 45) \cup(45 ;+\infty)$.\\
Suy ra hàm số liên tục trên từng khoảng $(-\infty ; 45)$ và $(45 ;+\infty)$.\\
Hàm số liên tục trên $\mathbb{R}$ khi và chỉ khi hàm số liên tục tại $x=45$ khi và chỉ khi
\allowdisplaybreaks
 \begin{eqnarray*}
&&\lim\limits_{x \rightarrow 45} f(x)=f(45)\\
&\Leftrightarrow &\lim\limits_{x \rightarrow 45} \dfrac{x^2-2025}{x-45}=2 m+4\\
& \Leftrightarrow &\lim\limits_{x \rightarrow 45} \dfrac{(x-45)(x+45)}{x-45}=2 m+4\\
& \Leftrightarrow &\lim\limits_{x \rightarrow 45}(x+45)=2 m+4\\
&\Leftrightarrow& 90=2 m+45 \Leftrightarrow m=43.
\end{eqnarray*}
 Suy ra $m=43$ thì hàm số liên tục trên $\mathbb{R}$. 
\end{itemchoice}}
\end{ex}
\begin{ex}%Câu 4. %[1H4H3-4]
	Cho hình chóp $S.ABCD$ có đáy $ABCD$ là hình bình hành. Lấy điểm $M$ trên canh $AD$ sao cho $AD=3AM$. Goi $G$, $N$ theo thứ tư là trọng tâm các tam giác $SAB$, $ABC$. Khi đó.
	\choiceTF
{Giao tuyến của hai mặt phẳng $(SAB)$ và $(SCD)$ là đường thẳng đi qua $S$ và song song với $AC$, $BD$}
{$\dfrac{DN}{DB}=\dfrac{1}{3}$}
{\True $MN$ song song với mặt phẳng $(SCD)$}
{$NG$ cắt với mặt phẳng $(SAC)$}
\loigiai{
	\begin{center}
		\begin{tikzpicture}[>=stealth,line join=round,line cap=round,font=\footnotesize,scale=1]
\def\ca{3.5}
\path
(0,0)coordinate(A)
(\ca,0)coordinate(D)
(-140:\ca-1)coordinate(B)
($(B)+(D)-(A)$)coordinate(C)
($(A)+(1,\ca)$)coordinate(S)
($(A)!1/3!(D)$)coordinate(M)
($(B)!0.5!(D)$)coordinate(O)
($(B)!2/3!(O)$)coordinate(N)
($(B)!0.5!(A)$)coordinate(P)
($(S)!2/3!(P)$)coordinate(G)
($(S)+(B)-(A)$)coordinate(x)
($(S)+(A)-(B)$)coordinate(y)
;
\draw(S)--(B)--(C)--(D)--cycle
(S)--(C)
(x)--(y)
;
\draw[dashed](S)--(A)--(B)--(D)
(D)--(A)--(C)
(S)--(P)--(C)
(G)--(N)--(M)
;
\path(x)node[below right]{$x$};
\foreach \x/\g in {A/60,D/0,C/-90,B/-90,S/90,M/90,O/-90,N/-90,G/0,P/120}\fill[black](\x)circle(1pt)+(\g:.3)node{$\x$};
\end{tikzpicture}
\end{center}
	\begin{itemchoice}
		\itemch \textbf{Sai}.
Ta có $\heva{&S \in(SAB) \cap(SCD) \\ &AB \parallel CD \\ &AB \subset(SAB), CD \subset(SCD)} \Rightarrow(SAB) \cap(SCD)=Sx \parallel AB \parallel CD$.
\itemch \textbf{Sai}. 
 Gọi $O$ là tâm hình bình hành $ABCD$.\\
Vi $N$ là trọng tâm của $\triangle ABC$ nên $BN=\dfrac{2}{3} BO=\dfrac{2}{3} \cdot \dfrac{1}{2} BD=\dfrac{1}{3} BD \Rightarrow \dfrac{DN}{DB}=\dfrac{2}{3}$. 
\itemch \textbf{Đúng}. 
Ta có $AD=3 AM \Rightarrow \dfrac{DM}{DA}=\dfrac{2}{3}$.\\
Xét tam giác $ADB$, ta có $\dfrac{DM}{DA}=\dfrac{DN}{DB}=\dfrac{2}{3}$ nên $MN \parallel AB \Rightarrow MN \parallel CD$, mà $CD \subset (SCD) \Rightarrow MN \parallel(SCD)$.
\itemch \textbf{Sai}. 
Goi $P$ là trung điểm $AB$. Tam giác $S P C$ có
$\dfrac{P G}{P S}=\dfrac{P N}{P C}=\dfrac{1}{3}$ (tính chất trọng tâm). \\
$\Rightarrow N G \parallel S C$, $S C \subset(S A C) \Rightarrow N G \parallel(S A C)$. 
\end{itemchoice}}
\end{ex}

\Closesolutionfile{ans}
% \indapan{4}{ans/ans-0-B15-DS}

\caukq
\Opensolutionfile{ans}[ans/11-CK1-Toan-tu-tam-De-1-Phan-3]
\begin{ex}%[1D2V3-8]
	    Bạn Nam thả một quả bóng cao su từ độ cao 15$m$ so với mặt đất, mỗi lần chạm đất quả bóng lại nảy lên một độ cao bằng bốn phần năm độ cao lần rơi trước. Biết rằng quả bóng luôn chuyển động vuông góc với mặt đất. Tổng quãng đường quả bóng đã di chuyển được (từ lúc thả bóng cho đến lúc bóng không nảy nữa, kết quả làm tròn đến hàng đơn vị).	
\loigiai{  
Ta có quãng đường bóng di chuyển được bằng tổng quãng đường bóng này lên và quãng đường bóng rơi xuống.  
Vì mỗi lần bóng nảy lên bằng $\dfrac{4}{5}$ lần này trước nên ta có tổng quãng đường bóng này lên là  
$$S_1=15\cdot \dfrac{4}{5}+15\cdot \left(\dfrac{4}{5}\right)^2+15\cdot \left(\dfrac{4}{5}\right)^3+\ldots+15\cdot \left(\dfrac{4}{5}\right)^n+\ldots $$
Đây là tổng của cấp số nhân lùi vô hạn có số hạng đầu $u_1=15\cdot \dfrac{4}{5}=12$ và công bội $q=\dfrac{4}{5}$.\\
Suy ra
\[
S_1=\dfrac{12}{1-\dfrac{4}{5}}=60.
\]
Tổng quãng đường bóng rơi xuống bằng khoảng cách độ cao ban đầu và tổng quãng đường bóng này lên, nên là
\[
S_2=15+15\cdot\dfrac{4}{5}+15\cdot\left(\dfrac{4}{5}\right)^2+\ldots+15\cdot \left(\dfrac{4}{5}\right)^n+\ldots
\]
Đây là tổng của cấp số nhân lùi vô hạn với số hạng đầu $u_1=15$ và công bội $q=\dfrac{4}{5}$.\\
Suy ra
\[
S_2=\dfrac{15}{1-\dfrac{4}{5}}=75.
\]
Vậy tổng quãng đường bóng bay là
\[
S=S_1+S_2=135\,(m).
\]
}
\shortans{$135$}
\end{ex}

\begin{ex}%[1D3C1-3]
	Cho dãy số $\{u_n\}$ biết $u_1=\dfrac{1}{2}$ và $u_{n+1}=\dfrac{u_n}{1+(n+2)u_n}$.
	Khi đó $\lim\limits_{n\to\infty}\left(u_1+u_2+\ldots+u_n\right)$  bằng bao nhiêu? \textit{(Kết quả làm tròn đến hàng phần trăm)}.	
	\loigiai{  
	Đặt $v_1=\dfrac{1}{u_1} \Rightarrow v_1=\dfrac{2}{1}=2$; 
	$v_{n+1}=\dfrac{1+(n+2)u_n}{u_n}=\dfrac{1}{u_n}+(n+2)$.\\	
	Khi đó ta có
	\[
	\begin{aligned}
		& \begin{cases}
			v_1=2 \\
			v_2=v_1+(1+2) \\
			v_3=v_2+(2+2) \\
			\dots \\
			v_n=v_{n-1}+[(n-1)+2]
		\end{cases}
		\quad \Leftrightarrow \quad
		\begin{cases}
			v_1=2\\
			v_2=v_1+3\\
			v_3=v_2+4\\
			\dots \\
			v_n=v_{n-1}+(n+1).
		\end{cases}
	\end{aligned}
	\]	
	Cộng vế theo vế các biểu thức trên, ta được
	\[
	v_1+v_2+v_3+\dots+v_n=2+(v_1+3)+(v_2+4)+\dots+(v_{n-1}+n+1).
	\]
	
	\[\Leftrightarrow
	v_n=2+3+4+\dots+(n+1)=\dfrac{n \left[2+(n+1)\right]}{2}=\dfrac{n(n+3)}{2}.
	\]		
	Suy ra dãy số $v_n$ có số hạng tổng quát là $v_n=\dfrac{n(n+3)}{2} \Rightarrow u_n=\dfrac{2}{n(n+3)}$.\\	
	Ta có
		\begin{eqnarray*}
	u_1+u_2+\dots+u_n&=& \dfrac{2}{1\cdot 4}+\dfrac{2}{2\cdot 5}+\dots+\dfrac{2}{n(n+3)}=2\cdot \left[\dfrac{1}{1\cdot 4}+\dfrac{1}{2\cdot 5}+\dots+\dfrac{1}{n(n+3)}\right]\\
		&=&2\cdot \left[\dfrac{1}{3}\left(1-\dfrac{1}{4}\right)+\dfrac{1}{3}\left(\dfrac{1}{2}-\dfrac{1}{5}\right)+\dots+\dfrac{1}{3}\left(\dfrac{1}{n}-\dfrac{1}{n+3}\right)\right]\\
		&=&\dfrac{2}{3}\cdot \left[\left(1+\dfrac{1}{2}+\dots+\dfrac{1}{n}\right)-\left(\dfrac{1}{4}+\dfrac{1}{5}+\dots+\dfrac{1}{n+3}\right)\right]\\
		&=&\dfrac{2}{3}\cdot \left[\left(1+\dfrac{1}{2}+\dots+\dfrac{1}{n}\right)-\left(\dfrac{1}{4}+\dfrac{1}{5}+\dots+\dfrac{1}{n+3}\right)\right]\\
		&=& \dfrac{2}{3}\cdot \left[\left(1+\dfrac{1}{2}+\dots+\dfrac{1}{n}\right)-\left(\dfrac{1}{n+1}+\dfrac{1}{n+2}+\dfrac{1}{n+3}\right)\right]\\
		&=&\dfrac{2}{3}\cdot \dfrac{11n^3+48n^2+49n}{6n^3+11n^2+6}.
	\end{eqnarray*}
	Vậy
	\[
	\lim_{n\to\infty} \left(u_1+u_2+\dots+u_n\right)=\lim_{n\to\infty} \dfrac{11n^3+48n^2+49n}{9n^3+54n^2+99n+54}=\dfrac{11}{9} \approx 1{,}22.
	\]		
	}
	\shortans{$1{,}22$}
\end{ex}

\begin{ex}%[1D3V1-3]
	Kết quả của giới hạn $	\lim\limits_{x\to-\infty} \left(\sqrt{4x^2-x+2}+2x-1\right)$
	bằng bao nhiêu? \textit{(Kết quả làm tròn đến hàng phần mười).}	
	\loigiai{  
		Ta có
		\allowdisplaybreaks
		\begin{eqnarray*}
			\lim_{x\to-\infty} \left(\sqrt{4x^2-x+2}+2x-1\right)&=&=\lim_{x\to\infty} \dfrac{4x^2-x+2-(2x-1)^2}{\sqrt{4x^2-x+2}-2x+1}\\
			&=&\lim_{x\to-\infty} \dfrac{3x+1}{\sqrt{4x^2-x+2}-2x+1}\\
			&=&\lim_{x\to-\infty} \dfrac{3x+1}{-x \sqrt{4-\dfrac{1}{x}+\dfrac{2}{x^2}}-2x+1}\\
			&=&\lim_{x\to-\infty} \dfrac{3x+1}{x \left(-\sqrt{4-\dfrac{1}{x}+\dfrac{2}{x^2}}-2+\dfrac{1}{x}\right)}\\
			&=& \lim_{x\to-\infty} \dfrac{3+\dfrac{1}{x}}{-\sqrt{4-\dfrac{1}{x}+\dfrac{2}{x^2}}-2+\dfrac{1}{x}}
			=-\dfrac{3}{4} =-0{,}75.
		\end{eqnarray*}		
		}
	\shortans{$-0{,}75$}
\end{ex}

\begin{ex}%[1D3H3-3]
	Cho hàm số 
	\[
	f(x) =
	\begin{cases} 
		\dfrac{x^3+8x+m}{x-1}, & \text{khi } x \neq 1, \\
		n, & \text{khi } x=1,
	\end{cases}
	\]
	với $m, n$ là các tham số thực. Biết rằng hàm số $f(x)$ liên tục tại $x=1$, khi đó hãy tính giá trị của $m+n$?.	
	\loigiai{  
		TXĐ $D=\mathbb{R}$.\\		
		Hàm số liên tục tại $x=1$ khi và chỉ khi $\lim_{x\to 1} f(x)=f(1)$.\\
		Mà $f(1)=n$ là số hữu hạn, suy ra $\lim\limits_{x\to 1} f(x)$ hữu hạn nên $x=1$ là nghiệm của 
		\[
		x^3+8x+m=0 \Rightarrow m=-9.
		\]
		Khi đó
		\[
		\lim_{x\to1} f(x)=\lim_{x\to1} \dfrac{x^3+8x+9}{x-1}=\lim_{x\to1} \dfrac{(x-1)(x^2+x+9)}{x-1}=\lim_{x\to1} (x^2+x+9)=11.
		\]
		Suy ra $n=11$. Vậy $m+n=-9+11=2$.
	}
	\shortans{$2$}
\end{ex}

\begin{ex}%[1H4H6-5]
	Cho hình chóp $S.ABCD$ có đáy là hình bình hành tâm $O$. Gọi $M, N, P$ lần lượt là trung điểm của $BC, CD, SD$. Gọi $Q$ là giao điểm của $SA$ với mặt phẳng $(MNP)$. Tỉ số $\dfrac{SQ}{SA}$.	
	\loigiai{  		
			\begin{center}
			\begin{tikzpicture}[>=stealth,line join=round,line cap=round,font=\footnotesize,scale=1]
				\def\ca{4.5}
				\path
				(0,0)coordinate(A)
				(\ca,0)coordinate(D)
				(-140:\ca-1)coordinate(B)
				($(B)+(D)-(A)$)coordinate(C)
				($(A)+(-1,6)$)coordinate(S)
				($(B)!0.5!(C)$)coordinate(M)
				($(B)!0.5!(D)$)coordinate(O)
				($(O)!0.5!(C)$)coordinate(E)
				($(C)!0.5!(D)$)coordinate(N)
				($(S)!0.5!(D)$)coordinate(P)
				($(S)!1/4!(A)$)coordinate(Q)
				($(S)+(B)-(A)$)coordinate(x)
				;
				\draw(S)--(B)--(C)--(D)--cycle
				(S)--(C)
				(P)--(N)
				;
				\draw[dashed](S)--(A)--(B)--(D)
				(D)--(A)--(C)
				(N)--(M)--(P)
				(Q)--(E)
				;
				\foreach \x/\g in {A/60,D/0,C/-90,B/-90,S/90,M/-90,O/-90,N/-90,Q/180,P/90,E/-90}\fill[black](\x)circle(2pt)+(\g:.3)node{$\x$};
			\end{tikzpicture}
		\end{center}
	\noindent		
	Trong $(ABCD)$ gọi $E=MN \cap AC$.\\	
	Trong $(SAC)$ vẽ $EQ \parallel SC$ với $Q \in SA$.\\	
	Ta có
	\[
	\begin{cases} 
		QE \parallel PN (\parallel SC) \\
		PN \subset (MNP)
	\end{cases}
	\Rightarrow Q \in (MNP).
	\]	
	\[
	E \in MN \subset (MNP)
	\Rightarrow Q=SA \cap (MNP).
	\]	
	Ta có $MN$ là đường trung bình của $\triangle BCD$ nên $MN \parallel BD$ hay $ME \parallel BO$.\\	
	Suy ra $E$ là trung điểm của $OC$.\\	
	Khi đó
	\[
	\dfrac{CE}{CO}=\dfrac{1}{2} \Rightarrow \dfrac{CE}{CA}=\dfrac{1}{4}.
	\]	
	Xét $\triangle SAC$, ta có $QE \parallel SC$ nên
	\[
	\dfrac{SQ}{SA}=\dfrac{CE}{CA}=\dfrac{1}{4}=0{,}25.
	\]	
	}
	\shortans{$0{,}25$}
\end{ex}

\begin{ex}%[1H4H6-5]
	Cho hình chóp $S.ABCD$ có đáy $ABCD$ là hình bình hành. Gọi $M$ là trung điểm cạnh $BC$, $(\alpha)$ là mặt phẳng qua $A, M$ và song song với $SD$. Mặt phẳng $(\alpha)$ cắt $SB$ tại $N$, tính tỉ số $\dfrac{SN}{SB}$ (kết quả làm tròn đến hàng phần trăm)?
	\loigiai{  
		\begin{center}
		\begin{tikzpicture}[>=stealth,line join=round,line cap=round,font=\footnotesize,scale=1]
			\def\ca{4.5}
			\path
			(0,0)coordinate(A)
			(\ca,0)coordinate(D)
			(-140:\ca-1)coordinate(B)
			($(B)+(D)-(A)$)coordinate(C)
			($(A)+(1.5,4)$)coordinate(S)
			($(B)!0.5!(C)$)coordinate(M)
			($(B)!0.5!(D)$)coordinate(O)
			($(S)!0.7!(B)$)coordinate(N)
			%($(B)!0.6!(O)$)coordinate(I)
			;
			\path[name path=line1] (A)--(M); % Đường thẳng AM
			\path[name path=line2] (B)--(D); % Đường thẳng BD
			\path[name intersections={of=line1 and line2, by=I}]; % Giao điểm I
			\draw(S)--(B)--(C)--(D)--cycle
			(S)--(C)
			(N)--(M)
			;
			\draw[dashed](S)--(A)--(B)--(D)
			(D)--(A)--(C)
			(N)--(A)--(M)
			(N)--(I)
			;
			\foreach \x/\g in {A/160,D/0,C/-90,B/-90,S/90,M/-90,O/-90,N/180,I/-20}\fill[black](\x)circle(2pt)+(\g:.3)node{$\x$};
		\end{tikzpicture}
	\end{center}
	\noindent
	Gọi $I$ là giao điểm của $AM$ và $BD$ nên $I$ là trọng tâm tam giác $ABC$.\\	
	Suy ra
	\[
	\dfrac{BI}{BO}=\dfrac{2}{3} \Rightarrow \dfrac{BI}{BD}=\dfrac{1}{3}.
	\]	
	Ta có $(\alpha)$ và mặt phẳng $(SBD)$ có chung điểm $I$, $(\alpha) \parallel SD$, $SD \subset (SBD)$.\\	
	Nên giao tuyến của $(\alpha)$ và $(SBD)$ là đường thẳng qua $I$ song song với $SD$ cắt $SB$ tại $N$.\\	
	Ta có tam giác $BIN$ đồng dạng với tam giác $BDS$.\\	
	Suy ra
	\[
	\dfrac{BN}{BS}=\dfrac{BI}{BD}=\dfrac{1}{3} \quad \text{hay} \quad \dfrac{SN}{SB}=\dfrac{ID}{BD}=\dfrac{2}{3} \approx 0{,}67.
	\]	
	}
	\shortans{$0{,}67$}
\end{ex} 
\Closesolutionfile{ans}
% \indapan{7}{ans/11-CK1-Toan-tu-tam-De-1-Phan-3}