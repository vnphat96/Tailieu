\begin{name}
	{\tenchude}
	{\tendethi}
	{\tentruong}
	{\thoigian}
\end{name}
\setcounter{ex}{0}\setcounter{bt}{0}
\begin{ex}%[1D2N1-4]%[Dự án đề ôn tập Toán khối 11 HKI NH23-24-Dot 1-Xuan Vy Pham]%[CTST-Đề số 4]
	Dãy số $(u_n)$ được gọi là dãy số tăng nếu với mọi số tự nhiên $n \ge 1$ ta luôn có
	\choice
	{$u_{n+1}=u_n$}
	{$u_{n+1} \ge u_n$}
	{$u_{n+1} < u_n$}
	{\True $u_{n+1} > u_n$}
	\loigiai{Theo định nghĩa, dãy số $(u_n)$ được gọi là dãy số tăng nếu với mọi số tự nhiên $n \ge 1$ ta luôn có $u_{n+1} > u_n$.}
\end{ex}

\begin{ex}%[1H4N3-1]%[Dự án đề ôn tập Toán khối 11 NH23-24-Đợt 1- Bùi Thanh Cương]%[CTST-Đề số 08]
	Cho hai đường thẳng phân biệt $ a$, $ b$ và mặt phẳng $\left(\alpha\right)$. Giả sử $ a \parallel \left(\alpha\right)$ và $ b\parallel\left(\alpha\right)$. Mệnh đề nào sau đây đúng?
	\choice
	{$a$ và $ b$ không có điểm chung}
	{$ a$ và $ b$ hoặc song song hoặc chéo nhau}
	{$ a$ và $ b$ chéo nhau}
	{\True $a$ và $ b$ hoặc song song hoặc chéo nhau hoặc cắt nhau}
	\loigiai{
		Hai đường thẳng $a$ và $b$ hoặc song song hoặc chéo nhau hoặc cắt nhau.
	}

\end{ex}

\begin{ex}%[1H4H2-2]%[Dự án đề ôn tập HKI Toán 11 NH23-24- Dương Phước Sang]%[CTST]
	Cho tứ diện $ABCD$. Gọi $I$, $J$ lần lượt là trọng tâm các tam giác $ABC$ và $ABD$. Chọn khẳng định đúng trong các khẳng định sau.
	\choice
	{\True $IJ$ song song với $CD$}
	{$IJ$ song song với $AB$}
	{$IJ$ chéo $CD$}
	{$IJ$ cắt $AB$}
	\loigiai{
		\immini{
			Gọi $M$ là trung điểm cạnh $AB$.\\
			Do $I$, $J$ lần lượt là trọng tâm $\triangle ABC$ và $\triangle ABD$ nên
			\[\dfrac{MI}{MC}=\dfrac{MJ}{MD}=\dfrac{1}{3}.\]
			Từ đó suy ra $IJ \parallel CD$.
		}
		{\vspace{-0.5cm}
			\begin{tikzpicture}[scale=0.7, font=\footnotesize, line join=round, line cap=round]
				\foreach \x\y\t in {0/0/B,1.3/-1.6/C,4.5/0/D,1/3.5/A}
				\coordinate (\t) at (\x,\y);
				\coordinate (M) at ($(A)!1/2!(B)$);
				\coordinate (I) at ($(M)!1/3!(C)$);
				\coordinate (J) at ($(M)!1/3!(D)$);
				\draw (A)--(B)--(C)--(A)--(D)--(C)--(M);
				\draw[dashed](B)--(D)--(M) (I)--(J);
				\foreach \t/\g in {A/90,B/180,C/-90,D/0,M/170,I/190,J/50}
				\draw[fill=black] (\t) circle(1pt)
				node[shift={(\g:7pt)}]{$\t$};
			\end{tikzpicture}}
	}
\end{ex}

\begin{ex}%[1D3H2-2]%[Dự án đề ôn tập Toán khối 11 HKI NH23-24-Dot 1-Võ Thị Thùy Trang]%[CTST - Đề số 16]
	Kết quả của giới hạn $\lim\limits_{x\to 5} \dfrac{x-5}{x-2}$ là
	\choice
	{\True $0$}
	{$1$}
	{$-1$}
	{$2$}
	\loigiai{
		Ta có $\lim\limits_{x\to 5} \dfrac{x-5}{x-2}=\dfrac{5-5}{5-2}=0$.
	}
\end{ex}

\begin{ex}%[1D1N4-2]%[Dự án đề kiểm tra Toán 11 HHKI NH23-24- Phạm Văn Long]%[Thi thử - KNTT]
	Tìm tập xác định $\mathscr{D}$ của hàm số $y=\cot x$.
	\choice{$\mathscr{D}=\mathbb{R}$}
	{$\mathscr{D}=\mathbb{R}\backslash \left\{0\right\}$}
	{\True $\mathscr{D}=\mathbb{R}\backslash \left\{k\pi, k\in \mathbb{Z}\right\}$}
	{$\mathscr{D}=\mathbb{R}\backslash \left\{\dfrac{\pi}{2}+k\pi, k\in \mathbb{Z}\right\}$}
	\loigiai{
		Tập xác định của hàm số $y=\cot x$ là $\mathscr{D}=\mathbb{R}\backslash \left\{k\pi, k\in \mathbb{Z}\right\}$.
	}
\end{ex}

\begin{ex}%[1H4H1-2]
	Cho tứ diện $ABCD$, gọi $M$ và $N$ lần lượt là trung điểm các cạnh $AB$ và $CD$. Gọi $G$ là trọng tâm tam giác $BCD$. Đường thẳng $AG$ cắt đường thẳng nào trong các đường thẳng dưới đây?
	\begin{center}
		\begin{tikzpicture}[>=stealth,line join=round,line cap=round,font=\footnotesize,scale=1]
			\path
			(0,0) coordinate (B)
			(1.5,-2) coordinate (C)
			(5,0) coordinate (D)
			($(B)+(1,4)$) coordinate (A)
			($(A)!0.5!(B)$) coordinate (M)
			($(C)!0.5!(D)$) coordinate (N)
			;
			\draw
			(B)--(C)--(A)--(D)
			;
			\draw[dashed] (B)--(D)(M)--(N);
			\draw (A)edge node[midway, sloped, rotate=90, anchor=center] {$ - $}(M);
			\draw (M)edge node[midway, sloped, rotate=90, anchor=center] {$ - $}(B);
			\draw (D)edge node[midway, sloped, rotate=90, anchor=center] {$ = $}(N);
			\draw (N)edge node[midway, sloped, rotate=90, anchor=center] {$ = $}(C);
			\foreach \x/\y in {A/95,B/190,C/-10,D/10,M/180,N/-20}
			\draw[fill=black] (\x) circle (1.1pt) + (\y:0.5cm) node {$\x$};
		\end{tikzpicture}
	\end{center}
	\choice
	{\True $MN$}
	{$CM$}
	{$DN$}
	{$CD$}
	\loigiai{
		\begin{center}
			\begin{tikzpicture}[>=stealth,line join=round,line cap=round,font=\footnotesize,scale=1]
				\path
				(0,0) coordinate (B)
				(1.5,-2) coordinate (C)
				(5,0) coordinate (D)
				($(B)+(1,4)$) coordinate (A)
				($(A)!0.5!(B)$) coordinate (M)
				($(C)!0.5!(D)$) coordinate (N)
				($(B)!2/3!(N)$) coordinate (G)
				;
				\draw
				(B)--(C)--(A)--(D)(A)--(N)
				;
				\draw[dashed] (M)--(N)--(B)--(D) (A)--(G);
				\draw (A)edge node[midway, sloped, rotate=90, anchor=center] {$ - $}(M);
				\draw (M)edge node[midway, sloped, rotate=90, anchor=center] {$ - $}(B);
				\draw (D)edge node[midway, sloped, rotate=90, anchor=center] {$ = $}(N);
				\draw (N)edge node[midway, sloped, rotate=90, anchor=center] {$ = $}(C);
				\foreach \x/\y in {A/95,B/190,C/-10,D/10,M/180,N/-10,G/-90}
				\draw[fill=black] (\x) circle (1.1pt) + (\y:0.5cm) node {$\x$};
			\end{tikzpicture}
		\end{center}
		Do $AG$ và $MN$ cùng nằm trong mặt phẳng $\left(ABN\right)$ nên hai đường thẳng cắt nhau.
	}
\end{ex}

\begin{ex}%[1D3N2-1]%[Dự án đề ôn tập Toán khối 11 NH23-24-Đợt 1- Bùi Thanh Cương]%[CTST-Đề số 08]
	Cho hai hàm số $ f(x)$, $g(x)$ thỏa mãn $\lim\limits_{x\to 2}f(x)=5$ và $\lim\limits_{x\to 2}g(x)=1$. Giá trị của $\lim\limits_{x\to 2}\left[f(x)\cdot g(x)\right]$ bằng
	\choice
	{\True $ 5$}
	{$ 6$}
	{$ 1$}
	{$-1$}
	\loigiai{
		Ta có $\lim\limits_{x\to 2}\left[f(x)\cdot g(x)\right]= \lim\limits_{x\to 2}f(x)\cdot \lim\limits_{x\to 2}g(x)=5\cdot 1=5$.
	}

\end{ex}

\begin{ex}%[1D3N3-1]%[Dự án đề kiểm tra Toán 11 GHKI NH23-24- Nguyễn Văn Sang]%[ĐỀ KNTT SỐ 5]
	Hàm số nào sau đây liên tục trên $\mathbb{R}$?
	\choice
	{\True $y=x^3-3x+1$}
	{$y=\sqrt{x-4}$}
	{$y=\tan x$}
	{$y=\sqrt{x}$}
	\loigiai{
		Ta có hàm số $y=x^3-3x+1$ liên tục trên $\mathbb{R}$ vì có tập xác định $\mathscr{D}=\mathbb{R}$.
	}
\end{ex}

\begin{ex}%[1H4N4-1]%[Dự án đề ôn tập Toán khối 11 HKI NH23-24-Dot 1-Xuan Vy Pham]%[CTST-Đề số 4]
	Hãy chọn câu \textbf{đúng}:
	\choice
	{ Nếu hai mặt phẳng song song thì mọi đường thẳng nằm trên mặt phẳng này đều song song với mọi đường thẳng nằm trên mặt phẳng kia}
	{Nếu hai mặt phẳng $(P)$ và $(Q)$ lần lượt chứa hai đường thẳng song song thì chúng song song với nhau}
	{Hai mặt phẳng cùng song song với một đường thẳng thì song song với nhau}
	{\True Hai mặt phẳng phân biệt không song song thì cắt nhau}
	\loigiai{Trong không gian, hai mặt phẳng có ba vị trí tương đối là song song, trùng nhau, cắt nhau. Do đó hai mặt phẳng phân biệt không song song thì cắt nhau.}
\end{ex}

\begin{ex}%[Pj10-Đề 07 HK1 NH2023-2024 CTST]%[Nguyễn Văn Nay]%[1H4H4-2]
	Cho hình hộp $ABCD.A'B'C'D'$. Mặt phẳng $(AB'D')$ song song với mặt phẳng nào trong các mặt phẳng sau đây?
	\choice
	{$(BCA')$}
	{\True $(BC'D)$}
	{$(A'C'C)$}
	{$(BDA')$}
	\loigiai{
		\immini{
			Do $ADC'B'$ là hình bình hành nên $AB'\parallel DC'$, và $ABC'D'$ là hình bình hành nên $AD'\parallel BC'$ nên $(AB'D')\parallel (BC'D)$.
		}
		{
			\begin{tikzpicture}[scale=1, font=\footnotesize, line join=round, line cap=round, >=stealth]
				\def\a{1.6} \def\b{3} \def\h{2.5} \def\g{-120}
				\path
				(0:0) coordinate (A)
				(\g:\a) coordinate (B)
				(0:\b) coordinate (D)
				($(B)+(D)-(A)$) coordinate (C)
				;
				\foreach \x in {A,B,C,D} \coordinate (\x') at ($(\x)+(90:\h)$);
				\draw (B')--(A')--(D')--(C')--(B')--(B)--(C)--(D)--(D') (C)--(C')
				(D)--(C')--(B) (B')--(D');
				\draw[dashed] (B)--(A)--(D)--(B) (A)--(A')
				(B)--(A')--(D) (C)--(A') (B')--(A)--(D')
				;
				\foreach \i/\g in {A/-80,B/-90,C/-90,D/0,A'/90,D'/90,B'/180,C'/5}
				\fill[black] (\i) circle(1pt) ($(\i)+(\g:3mm)$) node{$\i$};
			\end{tikzpicture}
		}
	}
\end{ex}

\begin{ex}%[1D2H1-3]
	Cho dãy số $\left(u_n\right)$, biết $u_n=\dfrac{2n+5}{5n-4}$. Số $\dfrac{7}{12}$ là số hạng thứ mấy của dãy số?
	\choice
	{$6$}
	{\True $8$}
	{$9$}
	{$10$}
	\loigiai{
		Ta có
		$\begin{aligned}[t] u_n=\dfrac{7}{12} & \Leftrightarrow\dfrac{2n+5}{5n-4}=\dfrac{7}{12}\,\left(n\in\mathbb{N}^*\right) \\
                                     & \Leftrightarrow 24n+60=35n-28                                                  \\
                                     & \Leftrightarrow 11n=88                                                         \\
                                     & \Leftrightarrow n=8.
			\end{aligned}$}
\end{ex}

\begin{ex}%[1H4H1-4]
	Cho tứ diện $ABCD$. Gọi $G$ là trọng tâm tam giác $BCD$, $M$ là trung điểm $CD$, $I$ là điểm ở trên đoạn thẳng $AG$, $BI$ cắt mặt phẳng $(ACD)$ tại $J$. Khẳng định nào sau đây \textbf{sai}?
	\choice
	{$AM=(ACD) \cap(ABG)$}
	{$A$, $J$, $M$ thẳng hàng}
	{\True $J$ là trung điểm của $AM$}
	{$DJ=(ACD) \cap(BDJ)$}
	\loigiai{
		\immini{	Ta có $A$ là điểm chung thứ nhất giữa hai mặt phẳng $(A C D)$ và $(G A B)$.\\
			Do $BG \cap CD=M \Rightarrow \heva{&M \in BG \subset(ABG) \\ &M \in CD \subset(ACD)} \Rightarrow \heva{&M \in(ABG)\\ &M \in(ACD)}$\\
			$\Rightarrow M$ là điểm chung thứ hai giữa hai mặt phẳng $(ACD)$ và $(GAB)$.\\
			$\Rightarrow(ABG) \cap(ACD)=AM.$\\
			Ta có $\heva{&BI \subset(ABG)\\& AM\subset(ABM)\\&(ABG) \equiv(ABM)} \Rightarrow AM$, $BI$ đồng phẳng.\\
			$\Rightarrow J=B I \cap A M \Rightarrow A$, $J$, $M$ thẳng hàng.
		}{
			\begin{tikzpicture}[scale=0.7, font=\footnotesize,line join=round, line cap=round, >=stealth]
				\coordinate (B) at (-2,0);
				\coordinate (C) at (2,-2.5);
				\coordinate (D) at (4,0);
				\coordinate (M) at ($(C)!0.5!(D)$);
				\coordinate (G) at ($(B)!2/3!(M)$);
				\coordinate (A) at ($(G)+(0,5)$);
				\coordinate (J) at ($(A)!2/3!(M)$);
				\coordinate (I) at (intersection of B--J and A--G);
				\foreach \i in {B,C,D,M}{\draw (A)--(\i);}
				\draw (B)--(C)--(D);
				\draw[dashed,thin]  (M)--(B)--(D) (A)--(G) (B)--(J);
				\foreach \i/\g in {A/90,B/180,C/-90,D/0,M/-60,G/-90,I/-30,J/30}{\draw[fill=black](\i) circle (1.5pt) ($(\i)+(\g:4mm)$) node[scale=1]{$\i$};}
			\end{tikzpicture}
		}
		\noindent	Ta có $\heva{&D J \subset(ACD) \\ &DJ \subset(BDJ)} \Rightarrow DJ=(ACD) \cap(BDJ).$\\
		Điểm $I$ di động trên $AG$ nên $J$ có thể không phải là trung điểm của $AM$.
	}
\end{ex}

\begin{ex}%[1D1N5-1]
	Công thức nghiệm của phương trình $\sin x=\sin \alpha$ là?
	\choice
	{\True $\hoac{&x=\alpha+k 2 \pi \\ &x=\pi-\alpha+k 2 \pi}; k \in \mathbb{Z}$}
	{$\hoac{&x=\alpha+k \pi \\ &x=\pi-\alpha+k \pi} ; k \in \mathbb{Z}$}
	{$\hoac{&x=\alpha+k \pi \\ &x=-\alpha+k \pi} ; k \in \mathbb{Z}$}
	{$\hoac{&x=\alpha+k 2 \pi \\ &x=-\alpha+k 2 \pi} ; k \in \mathbb{Z}$}
	\loigiai{
		Ta có $\sin x = \sin \alpha \Leftrightarrow \hoac{&x=\alpha+k 2 \pi \\ &x=\pi-\alpha+k 2 \pi}; k \in \mathbb{Z}.$
	}
\end{ex}

\begin{ex}%[1D1H2-2]
	Cho $\sin a=-\dfrac{4}{5}$, $3\pi <a<\dfrac{7\pi}{2}$. Tính $\tan a$.
	\choice
	{\True $\dfrac{4}{3}$}
	{$\dfrac{3}{4}$}
	{$-\dfrac{3}{5}$}
	{$-\dfrac{5}{3}$}
	\loigiai{
		Vì $3\pi<a<\dfrac{7\pi}{2}$ nên $\cos a <0$, $\tan a >0$, $\cot a >0$.\\
		Ta có $\sin^2 a +\cos^2 a=1 \Rightarrow \cos^2 a=1-\sin^2 a=1-\dfrac{16}{25}=\dfrac{9}{25} \Rightarrow \cos a=\pm \dfrac{3}{5}$.\\
		Vì $\cos a <0$ nên $\cos a=-\dfrac{3}{5}$.\\
		Do đó $\tan a =\dfrac{\sin a}{\cos a}=\dfrac{4}{3}$.
	}
\end{ex}

\begin{ex}%[1D5H1-3]
	Doanh thu bán hàng trong $20$ ngày được lựa chọn ngẫu nhiên của một cửa hàng được ghi lại ở bảng sau (đơn vị: triệu đồng)
	\begin{center}
		\begin{tabular}{|l|c|c|c|c|c|}
			\hline  Doanh thu & $[5 ; 7)$ & $[7 ; 9)$ & $[9 ; 11)$ & $[11 ; 13)$ & $[13 ; 15)$ \\
			\hline  Số ngày   & $2$       & $7$       & $7$        & $3$         & $1$         \\
			\hline
		\end{tabular}
	\end{center}
	Số trung bình của mẫu số liệu trên thuộc khoảng nào trong các khoảng dưới đây?
	\choice
	{$\left[7;9\right)$}
	{\True $\left[9;11\right)$}
	{$\left[11;13\right)$}
	{$\left[13;15\right)$}
	\loigiai{
	Bảng tần số ghép nhóm theo giá trị đại diện là\\
	\begin{center}
		\begin{tabular}{|l|c|c|c|c|c|}
			\hline  Doanh thu        & $[5 ; 7)$ & $[7 ; 9)$ & $[9 ; 11)$ & $[11 ; 13)$ & $[13 ; 15)$ \\
			\hline  Giá trị đại diện & $6$       & $8$       & $10$       & $12$        & $14$        \\
			\hline Số ngày           & $2$       & $7$       & $7$        & $3$         & $1$         \\
			\hline
		\end{tabular}
	\end{center}
	Số trung bình $\overline{x}=\dfrac{2\cdot 6+7\cdot8+7\cdot 10+3\cdot 12+1\cdot 14}{20}=9{,}4$
	}
\end{ex}

\begin{ex}%[1H4N1-3]
	Cho hình chóp $S.ABCD$, đáy $ABCD$ là hình thang có $2$ đáy là $AD$ và $BC$. Gọi $M$, $N$ lần lượt là trung điểm của $SB$, $SC$, $O$ là giao điểm của $AC$ và $BD$. Giao tuyến của hai mặt phẳng $(AMN)$ và $(SBD)$ là
	\choice
	{$DN$}
	{\True $DM$}
	{$OM$}
	{$SO$}
	\loigiai{
		\immini{ Ta có $MN$ là đường trung bình của tam giác $SBC$, suy ra $MN\parallel BC$.\\
			Ta lại có $BC\parallel AD$, suy ra $MN\parallel AD$.\\
			Khi đó $(AMN) \equiv (AMND)\Rightarrow (AMN)\cap (SBD) = MD$.}
		{\begin{tikzpicture}[line join=round, line cap=round,thick,scale =0.7]
				\coordinate (S) at (1,4);
				\coordinate (A) at (-1,0);
				\coordinate (D) at (-1,-3);
				\coordinate (C) at (3,-3);
				\coordinate (B) at (3,0);
				\coordinate (M) at ($(S)!0.5!(B)$);
				\coordinate (N) at ($(S)!0.5!(C)$);
				\draw(S)--(A) (S)--(D) (S)--(C) (S)--(B) (A)--(D)--(C)--(B) (M)--(N);
				\draw[dashed,thin](A)--(B) (B)--(D) (A)--(C) (A)--(M) (A)--(N);
				\path (intersection of A--C and B--D) coordinate (O) node[below]{$O$}; % AC\cap BD = O
				\foreach \i/\g in {S/90,A/180,D/-90,C/-90,B/0,M/0,N/0}{\draw[fill=white](\i) circle (1.5pt) ($(\i)+(\g:3mm)$) node[scale=1]{$\i$};}
			\end{tikzpicture}
		}
	}
\end{ex}

\begin{ex}%[Pj10-1-GK1-NH23-24--TeamTeXHoa--Lâm Chính]%[1H4N2-2]
	\immini{
		Cho hình hộp $ABCD.A'B'C'D'$. Đường thẳng $AB$ song song với đường thẳng nào?
		\choice
		{\True $C'D'$}
		{$BD$}
		{$CC'$}
		{$D'A'$}
	}{
		\begin{tikzpicture}[scale=0.6, font=\footnotesize,line join=round, line cap=round, >=stealth]
			\coordinate (A) at (0,0);
			\coordinate (B) at (4,0);
			\coordinate (D) at (-2,-2);
			\coordinate (C) at ($(B)+(D)-(A)$);
			\coordinate (H) at (1,3);
			\foreach \i in {A,B,C,D}{\coordinate (\i') at ($(\i)+(H)$);}
			\draw (A')--(B')--(C')--(D')--cycle;
			\draw (B)--(B') (C)--(C') (D)--(D')  (B)--(C)--(D);
			\draw[dashed,thin](B)--(A)--(A') (A)--(D);
			\foreach \i/\g in {A'/90,B'/90,C'/90,D'/90,A/-90,B/-90,C/-90,D/-90}{\draw[fill=black](\i) circle (1.5pt) ($(\i)+(\g:4mm)$) node[scale=1]{$\i$};}
		\end{tikzpicture}}

	\loigiai{
		Ta có $AB \parallel C'D'$.
	}
\end{ex}

\begin{ex}%[Pj08-1-GK1-NH23-24--TeamTeXHoa--Pau lHieu Nguyen] %[1H4H1-3]
	\immini{Cho hình chóp $S.ABCD$ có đáy không là hình thang. Gọi $M$ là trung điểm của $SA$, $N$ là giao điểm của $AB$ và $CD$, $Q$ là giao điểm của $MN$ và $SB$ (xem hình vẽ).
		Giao tuyến của hai mặt phẳng $(MCD)$ và $(SBC)$ là
		\choice
		{$CD$}
		{\True $QC$}
		{$MQ$}
		{$SB$}
	}
	{
		\begin{tikzpicture}[>=stealth,line join=round,line cap=round,font=\footnotesize,scale=1]
			\tikzset{declare function={a=3.5;b=1.7;h=2.5;goc=-60;}}
			\coordinate (A) at (0:0);
			\coordinate (D) at (0:a);
			\coordinate (B) at (goc:b);
			\coordinate (C) at (-28:a-.8);
			\coordinate (S) at ($(a/2,h)$);
			\coordinate (M) at ($(S)!.5!(A)$);
			\coordinate (N) at (intersection of A--B and D--C);
			\coordinate (Q) at (intersection of M--N and S--B);
			\draw (B)--(S)--(A)--(B) (C)--(D)--(S)--(C) (M)--(N) (B)--(N)--(C);
			\draw[dashed] (A)--(D) (B)--(C) (D)--(M)--(C)--(Q);
			\foreach \d/\g in {A/180,B/-130,C/-60,D/0,S/90,M/150,N/-90,Q/180}
			\fill[black](\d) circle (0.6pt)+(\g:.25)node[scale=.85]{$\d$};
		\end{tikzpicture}
	}
	\loigiai{
		Ta có $\left\{\begin{array}{l}
				C\in (SBC) \\
				C\in (MCD)
			\end{array}\right.\Rightarrow C\in (SBC)\cap (MCD)$ \hfill(1)\\
		Lại có: $Q=SB\cap MN$\\
		$\Rightarrow\left\{\begin{array}{l}
				Q\in SB\subset (SBC) \\
				Q\in MN\subset (MND)\equiv (MCD)
			\end{array}\right.\Rightarrow Q\in (SBC)\cap (MCD)$ \hfill(2)\\
		Từ (1) và (2) suy ra $QC=(SBC)\cap (MCD)$.
	}
\end{ex}

\begin{ex} %[1D3N1-1]
	Cho hai dãy $\left(u_n \right)$ và $\left(v_n \right)$ thỏa mãn $\lim u_n=2$ và $\lim v_n=3$. Giá trị của $\lim \left(u_n\cdot v_n \right)$ bằng
	\choice{$5$}
	{\True $6$}
	{$-1$}
	{$1$}
	\loigiai{
		Ta có $\lim \left(u_n\cdot v_n \right)=\lim u_n \cdot \lim v_n=2\cdot 3=6$.
	}
\end{ex}

\begin{ex}%[Dự án TeX GKI 11, Đoàn Hùng]%[1D2H3-3]
	Cho cấp số nhân $\left(u_n\right)$ có các số hạng lần lượt là $3 ; 9 ; 27 ; 81 ; \ldots$. Tìm số hạng tổng quát $u_n$ của cấp số nhân $\left(u_n\right)$.
	\choice
	{ $u_n=3^{n-1}$}
	{\True $u_n=3^n$}
	{ $u_n=3^{n+1}$}
	{ $u_n=3+3^n$}
	\loigiai{
		Cấp số nhân $\left(u_n\right)$ có các số hạng lần lượt là $3 ; 9 ; 27 ; 81 ; \ldots$.\\
		Do đó cấp số nhân $\left(u_n\right)$ có $u_1=3$ và $q=3$, do đó số hạng tổng quát là $u_n=3\cdot 3^{n-1}=3^n$.
	}
\end{ex}

\begin{ex}%[1H4H3-2]
	Cho hình chóp tứ giác $S.ABCD$. Gọi $M$, $N$ lần lượt là trung điểm của $SA$ và $SC$.
	\begin{center}
		\begin{tikzpicture}[line join = round, line cap = round, thick, font = \small, scale = 0.8]
			\path
			(0:0) coordinate (A)
			+(0:5) coordinate (B)
			+(-30:4.5) coordinate (C)
			+(-50:2) coordinate (D)
			+(50:5) coordinate (S);
			\path ($(A)!.5!(S)$) coordinate (M);
			\path ($(S)!.5!(C)$) coordinate (N);
			\draw[dashed]
			(A)--(B) (M)--(N);
			\draw
			(S)--(A)--(D)--(C)--(B)--cycle
			(S)--(D) (S)--(C);
			\foreach \x/\g in {A/180,B/0,C/-90,D/-90,S/90,M/150,N/0}
			\fill (\x) circle (1.5pt)
			+(\g:3mm) node {$\x$};
		\end{tikzpicture}
	\end{center}
	Mệnh đề nào sau đây đúng.
	\choice
	{$MN \parallel (SAB)$}
	{$MN \parallel (SBC)$}
	{\True $MN \parallel (ABCD)$}
	{$MN \parallel (SBD)$}
	\loigiai{
		\immini
		{Ta có $MN$ là đường trung bình của tam giác $(SAC)$ nên $MN \parallel AC$.\\
			Mà $\heva{& AC \subset (ABCD)\\& MN \not \subset (ABCD)}$ suy ra $MN \parallel (ABCD)$.}
		{\begin{tikzpicture}[line join = round, line cap = round, thick, font = \small, scale = 0.7]
				\path
				(0:0) coordinate (A)
				+(0:5) coordinate (B)
				+(-30:4.5) coordinate (C)
				+(-50:2) coordinate (D)
				+(50:5) coordinate (S);
				\path ($(A)!.5!(S)$) coordinate (M);
				\path ($(S)!.5!(C)$) coordinate (N);
				\draw[dashed]
				(C)--(A)--(B) (M)--(N);
				\draw
				(S)--(A)--(D)--(C)--(B)--cycle
				(S)--(D) (S)--(C);
				\foreach \x/\g in {A/180,B/0,C/-90,D/-90,S/90,M/150,N/0}
				\fill (\x) circle (1.5pt)
				+(\g:3mm) node {$\x$};
			\end{tikzpicture}}
	}
\end{ex}

\begin{ex}%[1D3N1-2]%[Dự án đề ôn tập Toán Khối 11 HK1 NH23-24-Dot 1-Vương Quốc Phong]%[CTST - Đề số 3]
	$\lim \dfrac{1}{2n+5}$ bằng
	\choice
	{$\dfrac{1}{2}$}
	{\True $0$}
	{$+\infty$}
	{$\dfrac{1}{5}$}
	\loigiai{
		Ta có $\lim \dfrac{1}{2n+5} = 0$.
	}
\end{ex}

\begin{ex}%[1D5H1-4]
	Khảo sát chiều cao của một số học sinh khối 11 thu được mẫu số liệu ghép nhóm sau
	\begin{center}
		\begin{tabular}{|l|c|c|c|c|c|}
			\hline
			Khoảng chiều cao (cm) & $[145;150)$ & $[150;155)$ & $[155;160)$ & $[160;165)$ & $[165;170)$ \\
			\hline
			Số học sinh           & $7$         & $14$        & $10$        & $10$        & $9$         \\\hline
		\end{tabular}
	\end{center}
	Tính mốt của mẫu số liệu ghép nhóm này (làm tròn kết quả đến hàng phần trăm).
	\choice
	{$160$}
	{$152{,}25$}
	{\True $152{,}18$}
	{$170$}
	\loigiai{
	Tần số lớn nhất là $14$ nên nhóm chứa mốt là nhóm $[150 ; 155)$.\\
	Ta có nhóm có tần số lớn nhất là nhóm $i=2$; giá trị bên trái của nhóm $2$ là $a_2=150$ với tần số $n_2=14$; tần số nhóm trước nó là $n_1=7$ và tần số nhóm sau là $n_3=10$; độ dài nhóm $2$ là $h=5$.\\
	Do đó $M_0=a_2+\left(\dfrac{n_i-n_{i-1}}{2n_i-n_{i-1}-n_{i+1}}\cdot h \right) =150+\dfrac{14-7}{(14-7)+(14-10)} \cdot5 \approx 153{,}18$.
	}
\end{ex}

\begin{ex}%[1D1H4-6]%[Dự án đề ôn tập Toán 11 HKI NH2023-2024 Dot1-Le Hung Thang]%[CD-Đề số 2]
	Tìm giá trị nhỏ nhất và giá trị lớn nhất của hàm số $y=3-4\cos\left(2x+\dfrac{\pi}{6}\right)$.
	\choice
	{\True $-1$ và $7$}
	{$3$ và $7$}
	{$-1$ và $1$}
	{$1$ và $7$}
	\loigiai{
		Đặt $y=f(x)=3-4\cos\left(2x+\dfrac{\pi}{6}\right)$.\\
		Với $\forall x\in\mathbb{R}$ ta có
		\allowdisplaybreaks \vspace*{-0.7cm}
		\begin{eqnarray*}
			& & -1\le\cos\left(2x+\dfrac{\pi}{6}\right)\le 1\\
			& \Leftrightarrow & 4\ge -4\cos\left(2x+\dfrac{\pi}{6}\right)\ge -4\\
			& \Leftrightarrow & 7\ge 3-4\cos\left(2x+\dfrac{\pi}{6}\right)\ge -1\\
			& \Leftrightarrow &  7\ge y \ge -1.
		\end{eqnarray*}
		Vậy $\min\limits_{x\in\mathbb{R}}f(x)=-1$.\\
		$f(x)=-1 \Leftrightarrow \cos\left(2x+\dfrac{\pi}{6}\right)=1\Leftrightarrow 2x+\dfrac{\pi}{6}=k2\pi\Leftrightarrow x=-\dfrac{\pi}{12}+k\pi $, $k\in\mathbb{Z}$.\\
		Và $\max\limits_{x\in\mathbb{R}}f(x)=7$.\\
		$f(x)=7 \Leftrightarrow \cos\left(2x+\dfrac{\pi}{6}\right)=-1\Leftrightarrow 2x+\dfrac{\pi}{6}=\pi+k2\pi\Leftrightarrow x=\dfrac{5\pi}{12}+k\pi $, $k\in\mathbb{Z}$.\\
	}
\end{ex}

\begin{ex}%[1D3H1-2]%[Dự án đề kiểm tra Toán 11 HKI NH23-24-Đợt 1- Phạm Phương]%[CTST-Đề số 5]
	%[TH]
	Giá trị của $A=\lim \dfrac{2n+1}{n-2}$ bằng
	\choice
	{$+\infty$}
	{$-\infty$}
	{\True $2$}
	{$1$}
	\loigiai{
		Ta có $A=\lim \dfrac{2n+1}{n-2}=\lim \dfrac{\dfrac{2n}{n}+\dfrac{1}{n}}{\dfrac{n}{n}-\dfrac{2}{n}}=\lim \dfrac{2+\dfrac{1}{n}}{1-\dfrac{2}{n}}=\dfrac{2+0}{1-0}=2$.
	}
\end{ex}

\begin{ex}%[1D5N1-2]
	Khảo sát khối lượng $30$ củ khoai tây ngẫu nhiên thu hoạch được ở một nông trường
	\begin{center}
		\begin{tabular}{|c|c|}
			\hline
			Khối lượng (gam) & Số củ khoai tây \\
			\hline
			$[70{;}80)$      & 4               \\
			$[80{;}90)$      & 5               \\
			$[90{;}100)$     & 12              \\
			$[100{;}110)$    & 6               \\
			$[110{;}120)$    & 3               \\
			\hline
			Cộng             & 30              \\
			\hline
		\end{tabular}
	\end{center}
	Số củ khoai tây đạt chuẩn loại I (từ $90$ gam đến dưới $100$ gam) là
	\choice
	{$5$}
	{\True $12$}
	{$6$}
	{$4$}
	\loigiai{
		Số củ khoai tây đạt chuẩn loại I là 12 .
	}
\end{ex}

\begin{ex}%[1D1H4-3]%[Dự án đề kiểm tra Toán 11 HKI NH23-24-Đợt 1- Phạm Phương]%[CTST-Đề số 5]
	%[TH]
	Hàm số nào sau đây nghịch biến trên khoảng $\left(0;\pi \right)$?
	\choice
	{$y=\sin x$}
	{\True $y=\cos x$}
	{$y=\tan x$}
	{$y=\cot x$}
	\loigiai{
		Hàm số $y=\cos x$ và $y=\cot x$ nghịch biến trên khoảng $\left(0;\pi \right)$.
	}
\end{ex}

\begin{ex}%[Dự án TeX GKI 11, Đoàn Hùng]%[1D2H2-6]
	Tìm tổng $S$ của $100$ số nguyên dương đầu tiên và đều chia $5$ dư $1$.
	\choice
	{\True $24850$ }
	{ $25100$ }
	{ $50200$ }
	{ $5001$ }
	\loigiai{
		Các số chia $5$ dư $1$ tạo thành cấp số cộng có $u_1=1$ và $d=5$, do đó
		\[S_{100}=\dfrac{100\cdot (2u_1+99d)}{2}=\dfrac{100\cdot (2\cdot 1+99\cdot 5)}{2}=24850.\]
	}
\end{ex}

\begin{ex}%[Dự án 10 - Team Tex hóa - Nguyễn Mộng Hùng]%[1D2N3-1]
	Hàm số nào trong các hàm số dưới đây liên tục tại $x=2$?
	\choice
	{$y=\dfrac{x+2}{x-2}$}
	{$y=\sqrt{x-5}$}
	{\True $y=x^5-x^3+1$}
	{$y=\dfrac{1}{x^2-4}$}
	\loigiai{
		Hàm số liên tục tại $x=2\Rightarrow x\in\mathscr{D}$ của hàm số.\\
		Mà $x=2\notin\mathscr{D}$ của các hàm số $y=\dfrac{x+2}{x-2}$, $y=\sqrt{x-5}$, $y=\dfrac{1}{x^2-4}$.\\
		Vậy hàm số $y=x^5-x^3+1$ liên tục tại $x=2$, vì có $\mathscr{D}=\mathbb{R}$.
	}
\end{ex}

\begin{ex}%[1D2H2-3]
	Tổng $n$ số hạng đầu tiên của một cấp số cộng là $S_n = n^2 + 4n$ với $n\in\mathbb{N^*}$. Tìm số hạng tổng quát $u_n$ cấp sô cộng đã cho.
	\choice
	{\True $u_n = 2n +3$}
	{$u_n = 3n + 2$}
	{$u_n =5\cdot 3^{n-1}$}
	{$u_n = 5\cdot\left(\dfrac{8}{5}\right)^{n-1}$}
	\loigiai{
		Ta có $S_1 =u_1 = 5$, $S_2 = u_1 + u_2 = 12$. Suy ra $u_2 = 7$ và $d = 2$. Khi đó $u_n = u_1 + (n-1)d = 5 +(n-1)2 = 2n +3$.\\
		Vậy $u_n = 2n +3$.
	}
\end{ex}

\begin{ex}%[1D3V3-3]
	Cho hàm số $f(x)$ xác định và liên tục trên $\mathbb{R}$. Biết khi $x\ne 1$ thì $f(x)=\dfrac{x^2-5x+6}{x-2}$. Giá trị $f(1)$ là
	\choice
	{\True $-2$}
	{$-1$}
	{$1$}
	{$2$}
	\loigiai{Do hàm số liên tục trên $\mathbb{R}$ nên liên tục tại $x=1$, suy ra
		\[f(1)=\lim\limits_{x\to1}f(x)=\lim\limits_{x\to1}\dfrac{x^2-5x+6}{x-2}=\lim\limits_{x\to1}(x-3)=-2.\]}
\end{ex}

\begin{ex}%[1H4N6-1]%[Dự án đề kiểm tra Toán 11 HKI NH23-24- Nguyen Huynh]%[THPT - Tp HCM]
	Qua phép chiếu song song, tính chất nào không được bảo toàn?
	\choice{\True Chéo nhau}
	{Đồng quy}
	{Song song}
	{Thẳng hàng}
	\loigiai{Phép chiếu song song không bảo toàn tính chất chéo nhau. }
\end{ex}

\begin{ex}%[1H4V1-3]%[Lê Xuân Hòa ]
	\immini{Cho hình chóp $S.ABCD$ có đáy $ABCD$ là hình bình hành tâm $O$. Tìm giao tuyến của hai mặt phẳng $(SAB)$ và$(SCD)$.}
	{
		\begin{tikzpicture}[line join=round, line cap=round,thick,scale=0.6]
			\coordinate (A) at (0,0);
			\coordinate (B) at (2,-2);
			\coordinate (D) at (5,0);
			\coordinate (C) at ($(B)+(D)-(A)$);
			\coordinate (S) at (3,4);
			\draw(S)--(A) (S)--(B) (S)--(C) (A)--(B) (B)--(C);
			\draw[dashed,thin](A)--(D) (C)--(D) (S)--(D);
			\tkzLabelPoints[above ](S)
			\tkzLabelPoints[below](B)
			\tkzLabelPoints[right](D,C)
			\tkzLabelPoints[left](A)
		\end{tikzpicture}
	}
	\choice
	{Là đường thẳng đi qua đỉnh $S$ và tâm $O$ đáy}
	{Là đường thẳng đi qua đỉnh $S$ và song song với đường thẳng $AC$}
	{Là đường thẳng đi qua đỉnh $S$ và song song với đường thẳng $AD$}
	{\True Là đường thẳng đi qua đỉnh $S$ và song song với đường thẳng $AB$}
	\loigiai{
		Xét hai mặt phẳng $SAB$ và $SAC$ có $S$ chung và $AB\parallel CD$ .
		Nên giao tuyến của hai mặt phẳng  $SAB$ và $SAC$  là đường thẳng đi qua đỉnh $S$ và song song với đường thẳng $AB$ .
	}
\end{ex}

\begin{ex}%[1D1N3-1]%[Dự án đề ôn tập Toán khối 11 NH23-24-Đợt 1- Bùi Thanh Cương]%[CTST-Đề số 08]
	Công thức nào sau đây đúng?
	\choice
	{$\cos\left(a+b\right)=\sin a\sin b+\cos a\cos b$}
	{$\cos\left(a+b\right)=\sin a\sin b-\cos a\cos b$}
	{$\sin\left(a-b\right)=\sin a\cos b+\cos a\sin b$}
	{\True $\sin\left(a+b\right)=\sin a\cos b+\cos a\sin b$}
	\loigiai{
		Công thức cộng $\sin\left(a+b\right)=\sin a\cos b+\cos a\sin b$ đúng.
	}

\end{ex}

\begin{ex}%[1H4H6-2]%[Dự án đề ôn tập khối 11 HKI NH2023-2024-Đơt 1-TheHung Nguyen]%[CD-Đề số 1]
	\immini{Cho hình chóp $S . A B C D$ có đáy là hình bình hành, gọi $M$ là trung điểm của $S C$ (như hình vẽ).
		Hình chiếu song song của điểm $M$ theo phương $A C$ lên mặt phẳng $(S A D)$ là điểm nào sau đây?
		\choice
		{Trung điểm của $S B$}
		{Trung điểm của $S D$}
		{Điểm $D$}
		{\True Trung điểm của $S A$}}{\begin{tikzpicture}[scale=.5, font=\footnotesize, line join=round, line cap=round, >=stealth]
			\def\a{3}
			\def\b{2.5}
			\def\h{3.5}
			\def\g{35}
			\path
			(0,0) coordinate (A)
			(0:\a) coordinate (B)
			--++(\g:\b) coordinate (C)
			--++(180:\a)coordinate (D)
			;
			\coordinate (O) at ($(A)!1/2!(C)$);
			\coordinate (S) at ($(O)+(0,\h)$);
			\coordinate (M) at ($(S)!1/2!(C)$);
			\draw (S)--(A)--(B)--(C) (B)--(S)--(C);
			\draw[dashed] (B)--(D)--(S)  (A)--(D)--(C);
			\foreach \x/\g in {A/-90,B/-90,C/0,D/180,S/90,M/70}  \fill (\x) circle (1pt)+(\g:.25)node {$\x$};
		\end{tikzpicture}}
	\loigiai{
		\immini{Gọi $N$ là trung điểm $S A$.\\
			Khi đó $M N \parallel A C$ nên hình chiếu song song của điểm $M$ lên mặt phẳng $(S A D)$ là trung điểm $S A$.}{\begin{tikzpicture}[scale=.75, font=\footnotesize, line join=round, line cap=round, >=stealth]
				\def\a{3}
				\def\b{2.5}
				\def\h{3.5}
				\def\g{35}
				\path
				(0,0) coordinate (A)
				(0:\a) coordinate (B)
				--++(\g:\b) coordinate (C)
				--++(180:\a)coordinate (D)
				;
				\coordinate (O) at ($(A)!1/2!(C)$);
				\coordinate (S) at ($(O)+(0,\h)$);
				\coordinate (M) at ($(S)!1/2!(C)$);
				\coordinate (N) at ($(S)!1/2!(A)$);
				\draw (S)--(A)--(B)--(C) (B)--(S)--(C);
				\draw[dashed] (B)--(D)--(S)  (A)--(D)--(C)--(A) (M)--(N);
				\foreach \x/\g in {A/-90,B/-90,C/0,D/180,S/90,M/70,N/110}  \fill (\x) circle (1pt)+(\g:.25)node {$\x$};
			\end{tikzpicture}}

	}
\end{ex}

\TL
\begin{ex}%[1 điểm]%[1D1H5-3]%[Dự án đề kiểm tra Toán 11 GHKI NH23-24 - Don Lee]%[THPT Trần Khai Nguyên - Tp HCM]
	Giải phương trình $\sin\left(x+\dfrac{\pi}{6}\right)=\dfrac{1}{2}$.
	\loigiai{
		Ta có \allowdisplaybreaks
		\begin{eqnarray*}
			&&\sin\left(x+\dfrac{\pi}{6}\right)=\dfrac{1}{2}\\
			&\Leftrightarrow&\hoac{& x+\dfrac{\pi}{6}=\dfrac{\pi}{6}+k2\pi \\ & x+\dfrac{\pi}{6}=\pi-\dfrac{\pi}{6}+2\pi} \quad (k \in \mathbb{Z})\\
			&\Leftrightarrow&\hoac{& x=k2\pi \\ & x=\dfrac{2\pi}{3}+k2\pi} \quad (k \in\mathbb{Z}).
		\end{eqnarray*}
		Vậy phương trình có các nghiệm là $\hoac{& x=k2\pi \\ & x=\dfrac{2\pi}{3}+k2\pi}\;\left(k\in\mathbb{Z}\right)$.
	}
\end{ex}

\begin{ex}%[3]%[Pj08-0-GK1-NH23-24--TeamTeXHoa--VoVanTu]%[1D3V2-5]
	Tính giới hạn $\lim\limits_{x\to 0}\dfrac{2\sqrt{1+x}-\sqrt[3]{8-x}}{x}$.
	\loigiai{Ta có $\lim\limits_{x\to 0}\dfrac{2\sqrt{1+x}-\sqrt[3]{8-x}}{x}=\lim\limits_{x\to 0}\dfrac{\left(2\sqrt{1+x}-2\right)+\left(2-\sqrt[3]{8-x}\right)}{x}$.\\
		Mà $\heva{&
				\lim\limits_{x\to 0}\dfrac{2\sqrt{1+x}-2}{x}=\lim\limits_{x\to 0}\dfrac{2x}{x\left(\sqrt{1+x}+1\right)}=\lim\limits_{x\to 0}\dfrac{2}{\sqrt{1+x}+1}=1\\
				&\lim\limits_{x\to 0}\dfrac{2-\sqrt[3]{8-x}}{x}=\lim\limits_{x\to 0}\dfrac{1}{4+2\sqrt[3]{8-x}+\left(\sqrt[3]{8-x}\right)^2}=\dfrac{1}{12}.}$
		\\
		Nên $\lim\limits_{x\to 0}\dfrac{2\sqrt{1+x}-\sqrt[3]{8-x}}{x}=\lim\limits_{x\to 0}\dfrac{2\sqrt{1+x}-2}{x}+\lim\limits_{x\to 0}\dfrac{2-\sqrt[3]{8-x}}{x}=1+\dfrac{1}{12}=\dfrac{13}{12}.$
	}
\end{ex}

\begin{ex}%[1D2V3-8]
	\immini{Cho tam giác $OMN$ vuông cân tại $O$, $OM=ON=2$. Trong tam giác $OMN$, vẽ hình vuông $O A_1 B_1 C_1$ sao cho các đỉnh $A_1$, $B_1$, $C_1$ lần lượt nằm trên các cạnh $OM$, $MN$, $ON$ (Hình bên). Trong tam giác $A_1 M B_1$, vẽ hình vuông $A_1 A_2 B_2 C_2$ sao cho các đỉnh $A_2$, $B_2$, $C_2$ lần lượt nằm trên các cạnh $A_1 M$, $M B_1$, $A_1 B_1$. Tiếp tục quá trình đó, ta được một dãy các hình vuông. Tính tổng diện tích các hình vuông này.}
	{\begin{tikzpicture}[scale=1, font=\footnotesize, line join=round, line cap=round, >=stealth]
			\path
			(0,0) coordinate (O)
			(4,0) coordinate (M)
			(0,4) coordinate (N)
			($(O)!0.5!(M)$) coordinate (A_1)
			(A_1)++(90:2) coordinate (B_1)
			($(O)!0.5!(N)$) coordinate (C_1)
			($(A_1)!0.5!(M)$) coordinate (A_2)
			(A_2)++(90:1) coordinate (B_2)
			($(A_1)!0.5!(B_1)$) coordinate (C_2)
			($(A_2)!0.5!(M)$) coordinate (A_3)
			(A_3)++(90:0.5) coordinate (B_3)
			($(A_2)!0.5!(B_2)$) coordinate (C_3);
			\draw (N)--(O)--(M)--(N) (A_1)--(B_1)--(C_1) (A_2)--(B_2)--(C_2) (A_3)--(B_3)--(C_3);
			\path (A_3)--(M)node[midway,above]{$\ldots$};
			\pic[draw,angle radius=1.5mm,angle eccentricity=1.5] {right angle = N--O--M};
			\foreach \p/\g in {A_1/-90,A_2/-90,A_3/-90,B_1/30,B_2/30,B_3/30,C_1/180,C_2/180,C_3/180,M/0,N/90,O/-135}\draw[fill=black] (\p) circle (1pt)node[shift={(\g:.3)}]{$\p$};
		\end{tikzpicture}}
	\loigiai{
		Độ dài cạnh của các hình vuông lần lượt là
		\[
			OA_1=1;\ A_1A_2=\dfrac12 OA_1;\ A_2A_3=\dfrac12 A_1A_2;\ldots
		\]
		Đặt $S_1$ là diện tích hình vuông $OA_1B_1C_1$, $S_n$ là diện tích hình vuông $A_{n-1}A_nB_nC_n$ với $n\ge 2$. Diện tích của các hình vuông lần lượt là
		\[
			\begin{aligned}
				 & S_1=OA_1^2=1^2=1,                                                                                                                   \\
				 & S_2=A_1A_2^2=\dfrac14 S_1                                                                                                           \\
				 & S_3=a_3^2=\left[\left(\dfrac{1}{2}\right)^3\right]^2=\left[\left(\dfrac{1}{2}\right)^2\right]^3=\left(\dfrac{1}{4}\right)^3, \ldots
			\end{aligned}
		\]
		Các diện tích $S_1$, $S_2$, $S_3$, $\ldots$ tạo thành cấp số nhân lùi vô hạn với số hạng đầu là $S_1=\dfrac{1}{4}$ và công bội bằng $\dfrac{1}{4}$.
		Do đó, tổng diện tích các hình vuông là $S=\dfrac{1}{4} \cdot \dfrac{1}{1-\dfrac{1}{4}}=\dfrac{1}{3}$.
	}
\end{ex}

\begin{ex}%[1H4C4-6]
	Cho hình chóp $S.ABCD$, đáy là hình bình hành tâm $O$. Gọi $M$, $N$ lần lượt là trung điểm của $SA$ và $CD$.
	\begin{enumerate}[a.]
		\item Chứng minh $(OMN) \parallel(SBC)$.
		\item Gọi $I$ là trung điểm của $SD$, $J$ là một điểm trên $(ABCD)$ cách đều $AB$ và $CD$. Chứng minh $IJ \parallel(SAB)$.
		\item Xác định giao tuyến của mặt phẳng $(OMN)$ với các mặt của hình chóp.
	\end{enumerate}
	\loigiai{
		\begin{center}
			\begin{tikzpicture}[line join = round, line cap = round, thick, font = \footnotesize, scale = 1]
				\path
				(0:0) coordinate (A)
				+(0:5) coordinate (B)
				+(-140:2.5) coordinate (D)
				+(80:4) coordinate (S)
				($(B)+(D)-(A)$) coordinate (C);
				\path (intersection of A--C and B--D) coordinate (O);
				\path ($(A)!.5!(S)$) coordinate (M);
				\path ($(C)!.5!(D)$) coordinate (N);
				\path ($(S)!.5!(D)$) coordinate (I);
				\path ($(A)!.5!(D)$) coordinate (H);
				\path ($(B)!.5!(C)$) coordinate (K);
				\path ($(A)!.5!(B)$) coordinate (E);
				\path (barycentric cs:O=2,H=1) coordinate (J);
				\draw[dashed]
				(A)--(B) (A)--(D) (A)--(S);
				\draw[dashed] (O)--(M)--(N)--(E)--(M) (C)--(A) (J)--(I)--(H)--(K)--(I)--(M);
				\draw
				(D)--(C)--(B) (I)--(N)
				(S)--(B) (S)--(C) (S)--(D);
				\foreach \x/\g in {A/180,B/0,C/-45,D/-135,S/90,M/30,N/-90,I/150,H/-90,K/-30,E/45,O/-75,J/-90}
				\fill (\x) circle (1.5pt)
				+(\g:3.5mm) node {$\x$};
			\end{tikzpicture}
		\end{center}
		\begin{enumerate}[a.]
			\item Do $O$, $M$ lần lượt là trung điểm của $AC$, $SA$ nên $OM$ là đường trung bình của tam giác $SAC$ ứng với cạnh $SC \Rightarrow OM \parallel SC$.\\
			      Mà $SC \subset(SBC) \Rightarrow OM \parallel(SBC)$ $(1)$.\\
			      Tương tự $ON\parallel BC \subset(SBC) \Rightarrow ON\parallel(SBC)$.\\
			      Từ $(1)$ và $(2)$ suy ra $(OMN) \parallel(SBC)$.\\
			\item Gọi $H$, $K$ lần lượt là trung điểm của $AD$ và $BC$. \\
			      Do $J \subset(ABCD)$ và $\mathrm{d}(J, AB)=\mathrm{d}(J, CD)$ nên $J \in HK \Rightarrow IJ \subset(IHK)$.\\
			      Ta có $\heva{&IH \parallel SA \\&HK \parallel AB\\& IH \cap HK = H}$ do đó $(IHK) \parallel(SAB)$.\\
			      Vậy $\heva{&IJ \subset(IHK) \\ &(IHK) \parallel(SAB)} \Rightarrow IJ \parallel(SAB)$.
			\item $(OMN) \cap(ABCD)=ON$. Cho $ON \cap AB=E$.\\
			      $(OMN) \cap(SAB)=ME$.\\
			      $(OMN) \cap(SAD)=MI$. Do $MI \parallel AD \parallel ON$.\\
			      $(OMN) \cap(SCD)=NI$.\\
			      Các giao tuyến trên tạo ra tứ giác $MINE$. Vì $MI \parallel NE$ nên tứ giác $MINE$ là hình thang.
		\end{enumerate}
	}
\end{ex}
\Closesolutionfile{ans}
