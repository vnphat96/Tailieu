\begin{name}
	{\tenchude}
	{\tendethi}
	{\tentruong}
	{\thoigian}
\end{name}

\caulc
\Opensolutionfile{ans}[ans/11-CK1-Toan-tu-tam-De-3-Phan-1]
\begin{ex}%[1D1N1-2]
Nếu một cung tròn có số đo bằng $\dfrac{5\pi}{3}$ thì số đo bằng độ của cung tròn đó bằng
\choice
{\True $300^\circ$}
{$600^\circ$}
{$120^\circ$}
{$135^\circ$}
\loigiai{
Ta có $\dfrac{5\pi}{3}=300^\circ$.}
\end{ex}
\begin{ex}%[1D1H3-2]]
Cho hai góc nhọn $a$ và $b$. Biết $\cos a=\dfrac{5}{13}$, $\cos b=\dfrac{3}{5}$. Khi đó giá trị $\sin (a+b)$ bằng
\choice
{$-\dfrac{56}{65}$}
{\True $\dfrac{56}{65}$}
{$\dfrac{16}{65}$}
{$-\dfrac{16}{65}$}
\loigiai{
Vì $a$ và $b$ là hai góc nhọn nên $\sin a>0$ và $\sin b>0$.\\
Khi đó\\
$\sin a=\sqrt{1-\cos ^2 a}=\sqrt{1-\left(\dfrac{5}{13}\right)^2}=\dfrac{12}{13}$.\\
$\sin b=\sqrt{1-\cos ^2 b}=\sqrt{1-\left(\dfrac{3}{5}\right)^2}=\dfrac{4}{5}$.\\
$\sin (a+b)=\sin a \cos b+\cos a \sin b=\dfrac{12}{13} \cdot \dfrac{3}{5}+\dfrac{5}{13} \cdot \dfrac{4}{5}=\dfrac{56}{65}$.}
\end{ex}

\begin{ex}%[1D1H4-2]
Tập xác định của hàm số $y=\tan \left(2 x-\dfrac{\pi}{3}\right)$ là
\choice
{$\mathscr{D}=\mathbb{R} \backslash\left\{\dfrac{\pi}{6}+\dfrac{k \pi}{2}, k \in \mathbb{Z}\right\}$}
{$\mathscr{D}=\mathbb{R} \backslash\left\{\dfrac{\pi}{6}+k \pi, k \in \mathbb{Z}\right\}$} 
{\True $\mathscr{D}=\mathbb{R} \backslash\left\{\dfrac{5 \pi}{12}+\dfrac{k \pi}{2}, k \in \mathbb{Z}\right\}$}
{$\mathscr{D}=\mathbb{R} \backslash\left\{\dfrac{5 \pi}{12}+k \pi, k \in \mathbb{Z}\right\}$}
\loigiai{
Điều kiện $\cos \left(2 x-\dfrac{\pi}{3}\right) \neq 0 \Leftrightarrow 2 x-\dfrac{\pi}{3} \neq \dfrac{\pi}{2}+k \pi \Leftrightarrow x \neq \dfrac{5 \pi}{12}+\dfrac{k\pi}{2}(k\neq 0)$.\\
Vậy tập xác định của hàm số $y=\tan \left(2 x-\dfrac{\pi}{3}\right)$ là $\mathscr{D}=\mathbb{R} \backslash\left\{\dfrac{5 \pi}{12}+\dfrac{k \pi}{2}, k \in \mathbb{Z}\right\}$.
}
\end{ex}

\begin{ex}%[1D1H5-5]
Phương trình $\tan \left(2x+30^\circ\right)=\cot \left(3x-40^\circ\right)$ có bao nhiêu nghiệm trong khoảng $\left(0^\circ; 180^\circ\right)$?
\choice
{$2$}
{$3$}
{$4$}
{\True $5$}
\loigiai{
Điều kiện $ \heva{& 2 x+30^\circ \neq 90^\circ+k180^\circ\\& 3x-40^\circ\neq k 180^\circ} \Leftrightarrow \heva{&x\neq 30^\circ+k 90^\circ\\& x \neq\dfrac{40^\circ}{3}+k60^\circ}$ $(k \in \mathbb{Z})$.
\begin{eqnarray*}
& & \tan\left(2x+30^\circ\right)=\cot\left(3x-40^\circ\right) \\	&\Leftrightarrow & \tan \left(2x+30^\circ\right)=\tan\left[90^\circ-\left(3 x-40^\circ\right)\right]\\
&\Leftrightarrow & \tan \left(2 x+30^{\circ}\right)=\tan \left(130^{\circ}-3 x\right)\\
&\Leftrightarrow & 2x+30^{\circ}=130^{\circ}-3 x+k 180^{\circ}\\
&\Leftrightarrow & 5 x=100^{\circ}+k 180^{\circ}\\
&\Leftrightarrow& x=20^{\circ}+k36^{\circ}(k \in \mathbb{Z}).
\end{eqnarray*}
Vì $0^\circ<x<180^\circ \Leftrightarrow 0^\circ<20^\circ+k36^\circ<180^\circ \Leftrightarrow-20^\circ<k36^\circ<160^\circ \Leftrightarrow -\dfrac{5}{9}<k<\dfrac{40}{9} \Rightarrow k\in \left\{0;1;2;3;4\right\}$.
Vậy phương trình có $5$ nghiệm.
}
\end{ex}

\begin{ex}%[1D2N1-3]
Cho dãy số $ (u_n) $, biết $u_n=\dfrac{n+1}{2n+1}$. Số $\dfrac{8}{15}$ là số hạng thứ mấy của dãy số?
\choice
{$5$}
{$6$}
{\True $7$}
{$8$}
\loigiai{
Ta có $u_n=\dfrac{n+1}{2n+1}=\dfrac{8}{15}\Leftrightarrow 15(n+1)-8(2n+1)=0\Leftrightarrow -n+7=0\Leftrightarrow n=7$.}
\end{ex}

\begin{ex}%[1D2H2-4]
Viết ba số hạng xen giữa các số $ 2$ và $ 22$ để được một cấp số cộng có năm số hạng.
\choice
{$6$; $10$; $14$}
{$8$; $13$; $18$}
{$6$; $12$; $18$}
{\True $7$; $12$; $17$}
\loigiai{
Giữa $ 2$ và $ 22$ có thêm ba số hạng nữa lập thành cấp số cộng, xem như ta có một cấp số cộng có năm số hạng với $ u_1=2$; $ u_5=22$; ta cần tìm $u_2$; $u_3$; $u_4$.\\
Ta có $u_5=u_1+4d\Leftrightarrow d=\dfrac{u_5-u_1}{4}=\dfrac{22-2}{4}=5$.\\
Vậy $\heva{&u_2=u_1+d=7\\&u_3=u_1+2d=12\\&u_4=u_1+3d=17.}$
}
\end{ex}

\begin{ex}%[1D3N1-1]
Trong các khẳng định dưới đây có bao nhiêu khẳng định đúng?
\begin{enumerate}[(I)]
\item $\lim n^k=+\infty$ với $k$ nguyên dương.
\item $\lim q^n=+\infty$ nếu $|q|<1$.
\item $\lim q^n=+\infty$ nếu $q>1$.
\end{enumerate}
\choice
{$0$}
{$1$}
{\True $2$}
{$3$}
\loigiai{
	\begin{enumerate}[(I)]
	\item  $\lim n^k=+\infty$ với $k$ nguyên dương $\Rightarrow$ (I) là khẳng định đúng.
	\item  $\lim q^n=+\infty$ nếu $|q|<1 \Rightarrow$ (II) là khẳng định sai vì  $\lim q^n=0$ nếu $|q|<1$.
	\item $\lim q^n=+\infty$ nếu $q>1 \Rightarrow$ (III) là khẳng định đúng.
	\end{enumerate}
Vậy số khẳng định đúng là $2$.}
\end{ex}

\begin{ex}%[1D3N2-2]
Cho hàm số $f(x)=\dfrac{2x-1}{x^3-x}$. Kết luận nào sau đây đúng?
\choice
{Hàm số liên tục tại $x=-1$}
{Hàm số liên tục tại $x=0$}
{Hàm số liên tục tại $x=1$}
{\True Hàm số liên tục tại $x=\dfrac{1}{2}$}
\loigiai{Tại $x=\dfrac{1}{2}$, ta có $\lim\limits_{x\rightarrow \frac{1}{2}} f(x)=\lim\limits_{x \rightarrow \frac{1}{2}} \dfrac{2 x-1}{x^3-1}=0=f\left(\dfrac{1}{2}\right)$.\\ Vậy hàm số liên tục tại $x=2$.
}
\end{ex}

\begin{ex}%[1H4N1-4]
Cho $4$ điểm $A$, $B$, $C$, $D$ không cùng nằm trên một mặt phẳng. Trên $AB$, $AD$ lần lượt lấy các điểm $M$, $N$ sao cho $MN$ cắt $BD$ tại $I$. Điểm $I$ không thuộc mặt phẳng nào sau đây?
\choice
{$(ABD)$}
{$(BCD)$}
{$(CMN)$}
{\True $(ACD)$}
\loigiai{
Vì $I=MN\cap BD$ nên $ I \in (ABD), I\in(BCD), I\in(CMN)$.
	\begin{center}
	\begin{tikzpicture}[scale=1,font=\footnotesize,line join=round,line cap=round,>=stealth] 
		\path 
		(0,0) coordinate (B) 
		(1.5,-1.5) coordinate (C) 
		(5,0) coordinate (D) 
		(2,3) coordinate (A)
		($(A)!0.5!(B)$) coordinate (M)
	%	($(C)!0.5!(D)$) coordinate (N)
		($(A)!0.7!(D)$) coordinate (N)
		(intersection of M--N and B--D) coordinate (I); 
		\draw (A)--(B)--(C)--cycle (A)--(D)--(C) (D)--(I)--(N); 
		\draw[dashed] (B)--(D) (M)--(N); 
		\foreach \p/\g in {A/90,B/180,C/-90,D/-45,M/135,N/60,I/0} 
		\fill[black](\p) circle (1pt) ($(\p)+(\g:3mm)$) node{$\p$}; 
	\end{tikzpicture}
\end{center}
} 
\end{ex}

\begin{ex}%[1H4N2-2]
Cho tứ diện $ABCD$. Gọi $J$, $I$ lần lượt là trọng tâm các tam giác $ABC$ và $ABD$. Chọn khẳng định đúng trong các khẳng định sau?
\choice
{\True $IJ$ song song với $CD$}
{$IJ$ song song với $AB$}
{$IJ$ chéo $CD$}
{$IJ$ cắt $AB$}
\loigiai{
	\begin{center}
		
		\begin{tikzpicture}[line cap=round,line join=round,font=\footnotesize,thick]
			\def\g{.5}
			\path
			(0,0) coordinate (B)
			++(0:5) coordinate (D)
			++(-150:4.3) coordinate (C)
			(1.5,4) coordinate (A)
			($(C)!\g!(B)$) coordinate (M)
			($(B)!\g!(D)$) coordinate (N)
			($(A)!2/3!(M)$) coordinate (I)
			($(A)!2/3!(N)$) coordinate (J)
			%($(B)! (A)!90: (C)$) coordinate (K)
			;
			\draw [dashed] (B)--(D) (M)--(N)--(A) (I)--(J)
			;
			\draw (A)--(B)--(C)--(D) (D)--(A)--(C) (M)--(A);
			%\draw[shorten <=-2cm] (A)--(K) node [above] {$x$};
			\foreach \x/\g in {A/90,B/180,D/-90,C/-90,M/180,N/30,I/180,J/0}
			\draw [fill=black] (\x) circle (1.3pt) + (\g:.3) node {$\x$};
		\end{tikzpicture}	
	\end{center}
Gọi $M$, $N$ lần lượt là trung điểm của $BD$, $BC$.\\
$\Rightarrow MN$ là đường trung bình của tam giác $BCD$\\
$\Rightarrow MN \parallel CD$ (1).\\
$J$, $I$ lần lượt là trọng tâm các tam giác $ABC$ và $ABD$.\\
$\Rightarrow \dfrac{AI}{AM}=\dfrac{AJ}{AN}=\dfrac{2}{3} \Rightarrow IJ\parallel MN$ (2).\\
Từ (1) và (2) suy ra $IJ \parallel CD$.}
\end{ex}

\begin{ex}%[1D5N1-3]
Cho mẫu số liệu ghép nhóm về thống kê thời gian hoàn thành (phút) một bài kiểm tra trực tuyến của $100$ học sinh, ta có bảng số liệu sau:
\begin{center}
\begin{tabular}{|c|c|c|c|c|c|c|}
	\hline Thời gian (phút) & {$[33;35)$} & {$[35;37)$} & {$[37;39)$} & {$[39;41)$} & {$[41;43)$} & {$[43;45)$} \\
	\hline Số học sinh & $4$ & $13$ & $38$ & $27$ &$14$ & $4$ \\
	\hline
\end{tabular}
\end{center}
Thời gian trung bình để $100$ học sinh hoàn thành bài kiểm tra là
\choice
{\True $38{,}92$ phút}
{$38{,}29$ phút}
{$39{,}28$ phút}
{$39{,}82$ phút}
\loigiai{
Ta có bảng tần số ghép nhóm theo giá trị đại diện của mỗi nhóm
\begin{center}
\begin{tabular}{|c|c|c|c|c|c|c|}
	\hline Nhóm & {$[33 ; 35)$} & {$[35 ; 37)$} & {$[37 ; 39)$} & {$[39 ; 41)$} & {$[41 ; 43)$} & {$[43 ; 45)$} \\
	\hline Giá trị đại diện & $34$ & $36$ & $38$ & $40$ & $42$ & $44$ \\
	\hline Tần số & $4$ & $13$ & $38$ & $27$ &$14$ & $4$ \\
	\hline
\end{tabular}
\end{center}

Thời gian trung bình để $100$ học sinh hoàn thành bài kiểm tra là $$\bar{x}=\dfrac{4\cdot34+13\cdot36+38\cdot38+27\cdot 40+14\cdot42+4\cdot44}{100}=38{,}92 \,\,\text{(phút)}.$$}
\end{ex}

\begin{ex}%[1D5H2-2]
Cho mẫu số liệu ghép nhóm về chiều cao của $25$ cây dừa giống như sau:
\begin{center}
	\begin{tabular}{|c|c|c|c|c|c|}
		\hline Chiều cao (cm) & {$[0;10)$} & {$[10;20)$} & {$[20;30)$} & {$[30;40)$} & {$[40;50)$} \\
		\hline Số cây & $4$ & $6$ & $7$ & $5$ & $3$ \\
		\hline
	\end{tabular}
\end{center}
Trung vị của mẫu số liệu ghép nhóm này là
\choice
{$M_e=\dfrac{175}{7}$}
{$M_e=\dfrac{165}{5}$}
{\True $M_e=\dfrac{165}{7}$}
{$M_e=\dfrac{165}{3}$}
\loigiai{
Cỡ mẫu $n=4+6+7+5+3=25$.\\
$x_1, x_2, \ldots, x_{25}$ là chiều cao của $25$ cây dừa giống được sắp xếp theo thứ tự không giảm.\\
Khi đó, trung vị là $x_{13}$.\\
%Do $x_{13}$ thuộ̣c nhóm $\left[20;30\right)$ nên nhóm này chứa trung vị.\\ 
Do đó $p=3$, $a_3=20$, $m_3=7$, $m_1+m_2=10$, $a_4-a_3=10$.\\ 
Do đó $M_e=20+\dfrac{\dfrac{25}{2}-10}{7}\cdot10=\dfrac{165}{7}$.
}
\end{ex}
\Closesolutionfile{ans}
% \indapan{6}{ans/11-CK1-Toan-tu-tam-De-3-Phan-1}

\cauds
\Opensolutionfile{ans}[ans/11-CK1-Toan-tu-tam-De-3-Phan-2]
\begin{ex}%[1D1H5-3]%Câu 1
	Cho hai hàm số $ y=\sin\left(x+\dfrac{\pi}{4}\right)$ và $ y=\sin x$. Khi đó:
	\choiceTFt
	{\True Phương trình hoành độ giao điểm của hai đồ thị hàm số là $\sin \left(x+\dfrac{\pi}{4}\right)=\sin x$}
	{\True Hoành độ giao điểm của hai đồ thị là $x=\dfrac{3\pi}{8}+k\pi\,\,\left(k\in\mathbb{Z}\right)$}
	{Khi $x\in\left[0;2\pi \right] $ thì hai đồ thị hàm số cắt nhau tại ba điểm}
	{\True Khi $x\in \left[0;2\pi \right] $ thì một giao điểm của hai đồ thị hàm số có tọa độ là $\left(\dfrac{5\pi}{8};\sin \dfrac{5\pi}{8}\right)$}
	\loigiai{
		\begin{itemchoice}
			\itemch \textbf{Đúng}.\\ Phương trình hoành độ giao điểm của hai đồ thị hàm số là $\sin\left(x+\dfrac{\pi}{4}\right)=\sin x$.
			\itemch \textbf{Đúng}.\\ Hoành độ giao điểm của hai đồ thị là
			\begin{align*}
				\sin \left(x+\dfrac{\pi}{4}\right)=\sin x \Leftrightarrow \hoac{&x+\dfrac{\pi}{4}=x+k2\pi\\&x+\dfrac{\pi}{4}=\pi-x+k2\pi} \,\, \left(k\in \mathbb{Z}\right) \Leftrightarrow x=\dfrac{3\pi}{8}+k\pi \,\,\left(k\in \mathbb{Z}\right).
			\end{align*}
			\itemch \textbf{Sai}.\\ Khi $ x\in[0;2\pi]$ thì hai đồ thị hàm số cắt nhau tại hai điểm.\\
			Vì $ x\in[0;2\pi]\Rightarrow x\in\left\{{\dfrac{3\pi}{8};\dfrac{11\pi}{8}}\right\}$.\\
			Với $ x=\dfrac{3\pi}{8}\Rightarrow y=\sin\dfrac{3\pi}{8}\approx 0{,}92$.\\ 
			Với $ x=\dfrac{11\pi}{8}\Rightarrow y=\sin\dfrac{11\pi}{8}\approx-0{,}92$.
			\itemch \textbf{Đúng}.\\ Khi $ x\in[0;2\pi]$ toạ độ giao điểm của hai đồ thị hàm số là $\left(\dfrac{3\pi}{8};\sin\dfrac{3\pi}{8}\right),\left(\dfrac{11\pi}{8};\sin\dfrac{11\pi}{8}\right)$.
		\end{itemchoice}
	}
\end{ex}

\begin{ex}%[1D2H3-6]%Câu 2
	Cho cấp số nhân $\left(u_n\right)$, biết $u_1+u_5=51;{u_2}+u_6=102$. Khi đó:
	\choiceTFt
	{\True Số hạng $u_1=3$}
	{Số hạng $u_4=48$}
	{Số $12\,288$ là số hạng thứ $12$ của cấp số nhân $\left(u_n\right)$}
	{\True Tổng $ 8$ số hạng đầu của cấp số nhân $\left(u_n\right)$ là $765$}
	\loigiai{
		\begin{itemchoice}
			\itemch \textbf{Đúng}.\\
			Gọi $q$ là công bội của cấp số nhân đã cho, ta có\\
			$\heva{
				&u_1+u_5=51\\
				&u_2+u_6=102} \Leftrightarrow 
			\heva{
				&u_1+u_1q^4=51\\
				&u_1q+u_1q^5=102}\Leftrightarrow
			\heva{
				&u_1\left(1+q^4\right)=51&(1)\\
				&u_1q\left(1+q^4\right)=102&(2)}$\\
			Nhận xét: Nếu $u_1=0$ hay $ q=0$ thì $(1)$ và $(2)$ đều không thoả mãn vì vậy ta có $u_1q\ne 0$.\\
			Chia theo vế $(2)$ cho $(1)$ ta được: $ q=2$.\\
			Thay $ q=2$ vào $(1)$ suy ra $u_1=\dfrac{51}{1+2^4}=3$.
			\itemch \textbf{Sai}.\\
			Công thức số hạng tổng quát của cấp số nhân: $u_n=3\cdot{2^{n-1}}$.\\
			Suy ra $u_4=3\cdot 2^3=24$
			\itemch \textbf{Sai}.\\
			Xét $u_n=12\,288\Leftrightarrow{3\cdot 2^{n-1}}=12\,288\Leftrightarrow{2^{n-1}}=2^{12}\Leftrightarrow n=13$.\\
			Vậy $12\,288$ là số hạng thứ $13$ của cấp số nhân đã cho.
			\itemch \textbf{Đúng}.\\
			Tổng $8$ số hạng đầu của cấp số nhân là\\ $S_8=\dfrac{u_1\left(1-q^8\right)}{1-q}=\dfrac{3\cdot \left(1-2^8\right)}{1-2}=765$.
		\end{itemchoice}
		
		
		
	}
\end{ex}

\begin{ex}%[1D3V3-3]%Câu 3
	Cho hai hàm số $f(x)=\heva{
			&\dfrac{x^2-4}{x-2}&{\text{khi }x\ne 2}\\
			&{4{,}5}&{\text{khi }x=2}}$ và $g(x)=\dfrac{2}{x-1}$. Khi đó:
\choiceTFt
{\True Hàm số $g(x)$ liên tục tại điểm $x_0=2$}
{\True Giới hạn $\lim\limits_{x\to 2}f(x)=4$}
{Hàm số $f(x)$ liên tục tại điểm $x_0=2$}
{Hàm số $y=\dfrac{f(x)}{g(x)}$ liên tục tại điểm $x_0=2$}
	\loigiai{
		\begin{itemchoice}
			\itemch \textbf{Đúng}.\\
			Ta có $\heva{&g(2)=\dfrac{2}{2-1}=2\\ &\lim\limits_{x\to 2}g(x)=\lim\limits_{x\to 2}\dfrac{2}{x-1}=2.}$\\
			Suy ra $\lim\limits_{x\to 2}g(x)=g(2)$.\\
			Vậy hàm số $ g(x)$ liên tục tại điểm $x_0=2$.
			\itemch\textbf{ Đúng}.\\
			Ta có $\lim\limits_{x\to 2}f(x)=\lim\limits_{x\to 2}\dfrac{x^2-4}{x-2}=\lim\limits_{x\to 2}\dfrac{(x-2)(x+2)}{x-2}=\lim\limits_{x\to 2}(x+2)=4$.
			\itemch \textbf{Sai}.\\
			Ta có $\heva{&f(2)=4{,}5\\&\lim\limits_{x\to 2}f(x)=4.}$\\
			Suy ra $\lim\limits_{x\to 2}f(x)\ne f(2)$.\\
			Vậy hàm số $ f(x)$ không liên tục tại điểm $x_0=2$.
			\itemch \textbf{Sai}.\\
			Xét hàm số $ y=\dfrac{f(x)}{g(x)}$ trên khoảng $\left(1;3\right)$, ta có
			\begin{center}
				$y=\dfrac{f(x)}{g(x)}=\heva{
				&\dfrac{\left(x^2-4\right)\left(x-1\right)}{2\left(x-2\right)}&&\text{khi }x\in\left(1;3\right)\setminus\left\{ 2\right\}\\
				&4{,}5&&\text{khi }x=2.}$
			\end{center}
			Xét $\heva{&\lim\limits_{x\to 2}\dfrac{f(x)}{g(x)}=\lim\limits_{x\to 2}\dfrac{\left(x+2\right)\left(x-1\right)}{2}=2\\
			&\dfrac{f(2)}{g(2)}=4{,}5.}$\\
			Suy ra $\lim\limits_{x\to 2}\dfrac{f(x)}{g(x)}\ne\dfrac{f(2)}{g(2)}$.\\
			Vậy hàm số $ y=\dfrac{f(x)}{g(x)}$ không liên tục tại $ x=2$.
		\end{itemchoice}
	}
\end{ex}

\begin{ex}%[1H4H1-4]%Câu 4
	Cho tứ giác $ ABCD$ có $ AC$ và $ BD$ giao nhau tại $ O$ và một điểm $ S$ không thuộc mặt phẳng $\left(ABCD\right)$. Trên đoạn $ SC$ lấy một điểm $ M$ không trùng với $ S$ và $C$, $K=AM\cap SO$. Khi đó:
	\choiceTFt
	{$SO$ là giao tuyến của hai mặt phẳng $(SAC)$ và $(ABC)$}
	{\True $SO$ là giao tuyến của hai mặt phẳng $(SAC)$ và $(SBD)$}
	{\True Giao điểm của đường thẳng $SO$ với mặt phẳng $(ABM)$ là điểm $K$}
	{Giao điểm của đường thẳng $SD$ với mặt phẳng $(ABM)$ là điểm $N$ thuộc đường thẳng $AK$}
	\loigiai{
	\begin{center}
		\begin{tikzpicture}[line join=round, line cap=round,>=stealth,font=\footnotesize,scale=1]
		\path
		(0,0) coordinate (A)
		(0:6) coordinate (D)
		(-60:2) coordinate (B)
		(-25:5) coordinate (C)
		(60:5) coordinate (S)
		($(S)!2/5!(C)$) coordinate (M)
		(intersection of A--C and B--D) coordinate (O)
		(intersection of A--M and S--O) coordinate (K)
		($(B)!2!(K)$) coordinate (x)
		(intersection of B--x and S--D) coordinate (N)
		;
		\draw (S)--(A)--(B)--(C)--(D)--(S)--(B)--(M) (S)--(C);
		\draw[dashed] (A)--(D)--(B) (C)--(A)--(M) (S)--(O) (B)--(N);
		\foreach \i/\g in {S/90,A/180,B/-90,C/-90,D/0,M/0,N/60,K/160,O/-90}{\draw[fill=black](\i) circle (1pt) ($(\i)+(\g:3mm)$) node[scale=1]{$\i$};}
	\end{tikzpicture}
	\end{center}
		\begin{itemchoice}
			\itemch \textbf{Sai}.\\
			Ta thấy $A$ là điểm chung của cả hai mặt phẳng $\left(SAC\right)$ và $\left(ABC\right)$.\\
			Tương tự $C$ cũng là điểm chung của cả hai mặt phẳng $\left(SAC\right)$ và $\left(ABC\right)$.\\
			Do đó $AC$ là giao tuyến của hai mặt phẳng $\left(SAC\right)$ và $\left(ABC\right)$.
			\itemch\textbf{ Đúng}.\\
			$ SO$ là giao tuyến của hai mặt phẳng $\left(SAC\right)$ và $\left(SBD\right)$.
			\itemch \textbf{Sai}.\\
			Tìm giao điểm của $ SO$ và $\left(ABM\right)$.\\
			Trong mặt phẳng $\left(SAC\right)$, gọi $ K=AM\cap SO$.\\
			Ta có $\heva{&K\in AM\\ &AM\subset\left(ABM\right)\\&K\in SO}
			\Rightarrow K=SO\cap\left(ABM\right)$.
			\itemch \textbf{Sai}.\\
			Tìm giao điểm của $ SD$ và $\left(ABM\right)$.\\
			Xét mặt phẳng phụ $\left(SBD\right)$ chứa $SD$.\\
			Dễ thấy $B$ là điểm chung của hai mặt phẳng $\left(SBD\right)$ và $\left(ABM\right)$.\\
			Ta có $\heva{&K\in AM\\&AM\subset\left(ABM\right)\\&K\in SO\\&SO\subset\left(SBD\right)}
			\Rightarrow K\in\left(SBD\right)\cap\left(ABM\right)$.\\ Do đó $ BK=\left(SBD\right)\cap\left(ABM\right)$.\\
			Trong mặt phẳng $\left(SBD\right)$ gọi $ N=BK\cap SD$.\\
			Do $\heva{&N\in SD\\
				&N\in BK\\&BK\subset\left(ABM\right)}
				\Rightarrow N=SD\cap\left(ABM\right)$.
		\end{itemchoice}
		}
\end{ex}
\Closesolutionfile{ans}
% \indapan{2}{ans/11-CK1-Toan-tu-tam-De-3-Phan-2}

\caukq
\Opensolutionfile{ans}[ans/11-CK1-Toan-tu-tam-De-3-Phan-3]
\begin{ex}%[1D1V4-8] 
	Bạn Nam tham gia trò chơi vòng quay mặt trời tại một công viên. Khi bắt đầu trò chơi, Nam ngồi vào cabin số 1. Độ cao so với mặt đất của cabin số 1 trên vòng quay vào thời điểm $t$ giây sau khi bắt đầu chuyển động được cho bởi công thức $h(t)=10+20 \sin \left(\dfrac{\pi}{5} t\right)~(\text{m})$. Sau bao nhiêu giây thì Nam đạt độ cao $30$ (m) lần đầu tiên?
\shortans{$2{,}5$}
\loigiai{
Ta có 
\allowdisplaybreaks
\begin{eqnarray*}
&&h(t)=10+20 \sin \left(\dfrac{\pi}{5} t\right)=30 \Leftrightarrow \sin \left(\dfrac{\pi}{5} t\right)=1.\\&&
\dfrac{\pi}{5} t=\dfrac{\pi}{2}+k 2 \pi \Leftrightarrow t=\dfrac{5}{2}+10 k~(k \in \mathbb{Z}).
\end{eqnarray*}
Nam đạt độ cao $30$ (m) lần đầu tiên ứng với nghiệm $t$ dương nhỏ nhất của phương trình là $t=\dfrac{5}{2}$.\\
Vậy sau $2{,}5$ giây thì Nam đạt độ cao $30$ (m) lần đầu tiên.
}
\end{ex}
\begin{ex}%[1D2V2-7]
	Để chuẩn bị khoan giếng phục vụ cho trang trại của mình, anh Hải đã tham khảo giá của hai cơ sở khoan giếng như sau:
\begin{itemize}
\item Cơ sở $1$: Giá mét khoan đầu tiên là $120\,000$ đồng một mét và kể từ mét khoan thứ hai, giá của mỗi mét sau tăng thêm $10\,000$ đồng so với giá của mét khoan ngay trước đó.
\item Cơ sở $2$: Giá của mét khoan đầu tiên là $80\,000$ đồng một mét và kể từ mét khoan thứ hai, giá của mỗi mét khoan sau tăng thêm $12\,000$ đồng so với giá của mét khoan ngay trước đó.
\end{itemize}
Anh Hải muốn thuê khoan giếng với độ sâu là $50$ m để phục vụ trang trại.
Giả thiết chất lượng và thời gian khoan giếng của hai cơ sở là như nhau. Anh Hải nên chọn cơ sở nào để tiết kiệm chi phí nhất?
\shortans{$1$}
\loigiai{
\begin{itemize}
\item Cơ sở $1$: Giá của mỗi mét khoan theo thứ tự lập thành cấp số cộng với số hạng đầu $u_1=120\,000$ và công sai $d=10\,000$.\\
Tổng số tiền anh Hải phải trả là
$$S_{50}=\dfrac{50}{2}\left[2 u_1+49 \cdot d\right]=\dfrac{50}{2}[2\cdot 120\,000+49\cdot 10\,000]=18\,250\,000~\text{đồng}.$$
\item Cơ sở $2$: Giá của mỗi mét khoan theo thứ tự lập thành cấp số cộng với số hạng đầu $u_1=80\,000$ và công sai $d=12\,000$.\\
Tổng số tiền anh Hải phải trả là
$$
S_{50}=\dfrac{50}{2}\left[2 u_{1}+49 \cdot d\right]=\dfrac{50}{2}[2 \cdot 80\,000+49 \cdot 12\,000]=18\,700\,000~\text{đồng}.
$$
\end{itemize}
Vậy anh Hải nên chọn cơ sở $1$ để tiết kiệm chi phí hơn.
}
\end{ex}
\begin{ex}%[1D2H3-8] 
	Một người vào trường đua ngựa đặt cược, anh ta nghĩ ra một chiến lược, đó là lần đầu anh ta đặt cược $3 \$ $, nếu thua cược anh ta sẽ gấp 2 số tiền cược so với lần trước đó đến khi nào thắng cược thì thôi. Anh ta đã thua $13$ lần liên tiếp và thắng cược ở lần thứ $14$. Sau đó anh ta rời khỏi trường đua. Biết rằng nếu thắng anh ta sẽ nhận được số tiền bằng đúng số tiền cược bỏ ra. Khi ra về anh ta lãi bao nhiêu tiền?
\shortans{$3$}
\loigiai{
Số tiền cược của các lần liên tiếp là một cấp số nhân với $u_1=3$ và $q=2$.\\
Anh ta thua $13$ lần liên tiếp, tổng số tiền thua là
$$S_{13}=u_1+u_2+\cdots+u_{13}=\dfrac{u_1\left(1-q^{13}\right)}{1-q}=\dfrac{3\left(1-2^{13}\right)}{1-2}=24\,573 \$.$$
Số tiền anh ta cược ở lần thứ $14$ (cũng là số tiền anh ta thắng được)
$$u_{14}=2 \cdot u_{13}=2 \cdot 3 \cdot 2^{12}=24\,576 \$.$$
Số tiền anh ta nhận được: $u_{14}-S_{13}=24\,576-24\,573=3 \$ $.\\
Vậy anh ta đã lãi $3 \$ $.
}
\end{ex}
\begin{ex}%[1D3V1-6] 
	Từ độ cao $63$ m của tháp nghiêng Pi-sa ở Italia, người ta thả một quả bóng cao su xuống đất. Giả sử mỗi lần chạm đất quả bóng lại nảy lên độ cao bằng $\dfrac{1}{10}$ độ cao mà quả bóng đạt được ngay trước đó. Tính độ dài hành trình của quả bóng từ thời điểm ban đầu cho đến khi nó nằm yên trên mặt đất.
\shortans{$77$}
\loigiai{
Ta thấy:
Ban đầu bóng cao $63$ m nên chạm đất lần $1$ bóng di chuyển quãng đường $S_1=63~(\mathrm{m})$.\\ Từ lúc chạm đất lần một đến chạm đất lần hai bóng di chuyển được quãng đường là $$S_2=2\left(S_1 \cdot \dfrac{1}{10}\right)=2 \cdot  \left(63 \cdot \dfrac{1}{10}\right)=\dfrac{63}{5}$$ (do độ cao lần hai bằng $\dfrac{1}{10}$ độ cao ban đầu).\\
Từ lúc chạm đất lần hai đến chạm đất lần ba bóng di chuyển được quãng đường là $S_3=S_2 \dfrac{1}{10}$ (do độ cao lần ba bằng $\dfrac{1}{10}$ độ cao lần hai). \\Cứ tiếp tục như vậy kéo dài ra vô hạn thì ta có được tổng quãng đường mà bóng cao su đã di chuyển là
\allowdisplaybreaks
\begin{eqnarray*}
S&=&S_{1}+S_{2}+S_{3}+\cdots\\&=&S_{1}+S_{2}+S_{2} \cdot \dfrac{1}{10}+S_{2} \cdot\left(\dfrac{1}{10}\right)^{2}+\cdots\\&=&S_{1}+\dfrac{S_{2}}{1-\dfrac{1}{10}}=\dfrac{63}{5} \cdot \dfrac{10}{9}=77~(\mathrm{m}).
\end{eqnarray*}
Vậy quãng đường di chuyển của bóng là $77$ m.
}
\end{ex}
\begin{ex}%[1H4V3-8] 
	Một khối gỗ có các mặt đều là một phần của mặt phẳng với $(A B C D) \parallel (E F M H)$, $C K \parallel D H$. Khối gỗ bị hỏng một góc như hình minh họa bên. Bác thợ mộc muốn làm đẹp khối gỗ bằng cách cắt khối gỗ theo mặt phẳng $(\alpha)$ đi qua điểm $K$ và song song với mặt phẳng $(A B C D)$. Biết $C K=80 \mathrm{~cm}$, $D H=128 \mathrm{~cm}$, $B F=1 \mathrm{~m}$. Giả sử $(\alpha)$ cắt $B F$ tại $I$.
Em hãy giúp bác thợ tính độ dài đoạn $B I$ \textit{(tính theo đơn vị centimet)}.
\begin{center}
\begin{tikzpicture}[scale=.7]
\coordinate [label=above left:$B$](B) at (3,2);
	\coordinate [label=below:$A$](A) at (0,0);
	\coordinate [label=below:$D$](D) at (10,0);
	\coordinate[label=right:$C$] (C) at ($(D)-(A)+(B)$);
	\coordinate [label=above left:$E$](E) at (0,7);
\coordinate[label=above:$F$] (F) at ($(E)-(A)+(B)$);
	\coordinate[label=above:$M$] (M) at ($(F)+(5,0)$);
	\coordinate[label=below left:$H$] (H) at ($(E)+(7,0)$);
	\coordinate (L) at ($(H)-(D)+(C)$);
	\coordinate [label=right:$K$](K) at ($(C)! .6 ! (L)$);
\draw[smooth](F)--(E)--(A)--(D)--(C) (K)--(C) (H)--(D) (F)--(M)(E)--(H);
	\draw[dashed](A)--(B)--(C) (F)--(B);
	\draw (H).. controls ($(H)+(.75,-.75)$) and ($(H)+(1,0)$) .. ($(H)+(1.5,-.7)$)
	($(H)+(1.5,-.7)$).. controls ($(H)+(2,-.5)$) and ($(H)+(2.5,-.75)$) .. ($(H)+(3,-.7)$)
	($(H)+(3,-.7)$).. controls ($(H)+(3.5,-.5)$) and ($(H)+(3.75,-.75)$) .. (K)
	(H).. controls ($(H)+(.11,.5)$) and ($(H)+(.8,.91)$) .. ($(H)+(.6,1)$)
	($(H)+(.6,1)$).. controls ($(H)+(.7,1.2)$) and ($(H)+(.8,1.5)$) .. (M)
	(M).. controls ($(M)+(.9,-.8)$) and ($(M)+(.8,-1)$) .. ($(M)+(1.3,-.8)$)	
	($(M)+(1.3,-.8)$).. controls ($(M)+(1.5,-1)$) and ($(M)+(2,-1.2)$) .. (K);
		\end{tikzpicture}
\end{center}
\shortans{$62{,}5$}
\loigiai{
\begin{center}
\begin{tikzpicture}[scale=.7]
\coordinate [label=above left:$B$](B) at (3,2);
	\coordinate [label=below:$A$](A) at (0,0);
	\coordinate [label=below:$D$](D) at (10,0);
	\coordinate[label=right:$C$] (C) at ($(D)-(A)+(B)$);
	\coordinate [label=above left:$E$](E) at (0,7);
\coordinate[label=above:$F$] (F) at ($(E)-(A)+(B)$);
	\coordinate[label=above:$M$] (M) at ($(F)+(5,0)$);
	\coordinate[label=below left:$H$] (H) at ($(E)+(7,0)$);
	\coordinate (L) at ($(H)-(D)+(C)$);
	\coordinate [label=right:$K$](K) at ($(C)! .6 ! (L)$);
	\coordinate [label=right:$J$](J) at ($(D)! .6 ! (H)$);
	\coordinate [label=above left:$I$](I) at ($(B)! .6 ! (F)$);
	\coordinate (P) at ($(A)! .6 ! (E)$);
\draw[smooth](F)--(E)--(A)--(D)--(C) (K)--(C) (H)--(D) (F)--(M)(E)--(H) (K)--(J)--(P);
	\draw[dashed](A)--(B)--(C) (F)--(B) (K)--(I)--(P);
	\draw (H).. controls ($(H)+(.75,-.75)$) and ($(H)+(1,0)$) .. ($(H)+(1.5,-.7)$)
	($(H)+(1.5,-.7)$).. controls ($(H)+(2,-.5)$) and ($(H)+(2.5,-.75)$) .. ($(H)+(3,-.7)$)
	($(H)+(3,-.7)$).. controls ($(H)+(3.5,-.5)$) and ($(H)+(3.75,-.75)$) .. (K)
	(H).. controls ($(H)+(.11,.5)$) and ($(H)+(.8,.91)$) .. ($(H)+(.6,1)$)
	($(H)+(.6,1)$).. controls ($(H)+(.7,1.2)$) and ($(H)+(.8,1.5)$) .. (M)
	(M).. controls ($(M)+(.9,-.8)$) and ($(M)+(.8,-1)$) .. ($(M)+(1.3,-.8)$)	
	($(M)+(1.3,-.8)$).. controls ($(M)+(1.5,-1)$) and ($(M)+(2,-1.2)$) .. (K);
		\end{tikzpicture}
\end{center}
Đổi $1$ m $=100$ cm. \\Gọi $J$ là giao điểm của $(\alpha)$ và $D H$.\\
Do $(\alpha)  \parallel (A B C D) \Rightarrow J K  \parallel  C D$ mà $C K  \parallel  D H \Rightarrow C D J K$ là hình bình hành \\$\Rightarrow D J=C K=80 \mathrm{~cm}$.\\
Do $3$ mặt phẳng $(E F M H)$, $(\alpha)$, $(A B C D)$ là đôi một song song. Nên áp dụng định lí Thalès
trong không gian, ta có $3$ mặt phẳng đó chắn trên $2$ cát tuyến $B F$, $D H$ những đoạn thẳng
tương ứng tỉ lệ, tức là $$\dfrac{B I}{B F}=\dfrac{D J}{D H} \Rightarrow \dfrac{B I}{100}=\dfrac{80}{128} \Rightarrow B J=62{,}5 \mathrm{~cm}.$$
}
\end{ex}
\begin{ex}%[1D5H2-3] 
	Doanh thu bán hàng trong $20$ ngày được lựa chọn ngẫu nhiên của một của hàng được ghi lại ở bảng sau (đơn vị: triệu đồng):
\begin{center}
\begin{tabular}{|c|c|c|c|c|c|}
\hline Doanh thu & {$[5 ; 7)$} & {$[7 ; 9)$} & {$[9 ; 11)$} & {$[11 ; 13)$} & {$[13 ; 15)$} \\
\hline Số ngày & $2$ & $7$ & $7$ & $3$ & $1$ \\
\hline
\end{tabular}
\end{center}
Gọi các tứ phân vị của mẫu số liệu ghép nhóm trên là $Q_1$, $Q_2$, $Q_3$. Hãy tính giá trị biểu thức $T=Q_1-Q_2+2 Q_3$.
\shortans{$20$}
\loigiai{
Cỡ mẫu là $n=20$.\\
Tứ phân vị thứ nhất $Q_1$ của mẫu số liệu ghép nhóm đã cho là trung bình của giá trị thứ $10$ và $11$.
Nên $Q_1$ thuộc nhóm $[7 ; 9)$ (là nhóm chứa các giá trị từ $x_3 \rightarrow x_9$).\\
Ta có $Q_1=7+\dfrac{\dfrac{n}{4}-2}{7} \cdot(7-5)=\dfrac{55}{7}$.\\
Tứ phân vị thứ hai $Q_2$ của mẫu số liệu ghép nhóm đã cho là trung bình của giá trị thứ $5$ và $6$. Nên $Q_2$ thuộc nhóm $\left[9 ; 11\right.$) (là nhóm chứa các giá trị từ $x_{10} \rightarrow x_{16}$).\\
Ta có $Q_2=9+\dfrac{\dfrac{n}{2}-(2+7)}{7} \cdot (11-9)=\dfrac{65}{7}$.\\
Tứ phân vị thứ ba $Q_{3}$ của mẫu số liệu ghép nhóm đã cho là trung bình của giá trị thứ $15$ và $16$.
Nên $Q_3$ cũng thuộc nhóm $\left[9 ; 11\right.$) (là nhóm chứa các giá trị từ $x_{10} \rightarrow x_{16}$).\\Ta có $Q_3=9+\dfrac{\dfrac{3 n}{4}-(2+7)}{7}\cdot(11-9)=\dfrac{75}{7}$.\\
Vậy $T=Q_1-Q_2+2 Q_3=\dfrac{55}{7}-\dfrac{65}{7}+2 \cdot \dfrac{75}{7}=20$.
}
\end{ex}
\Closesolutionfile{ans}
% \indapan{3}{ans/11-CK1-Toan-tu-tam-De-3-Phan-3}
