\setcounter{section}{7}
\section{Mẫu số liệu ghép nhóm}
\subsection{Tóm tắt lý thuyết}
\begin{tomtat}
\subsubsection{Giới thiệu về mẫu số liệu ghép nhóm}
\begin{dn}
	Mẫu số liệu ghép nhóm là mẫu số liệu cho dưới dạng bảng tần số của các nhóm số liệu. Mỗi nhóm số liệu là tập hợp gồm các giá trị của số liệu được ghép nhóm theo một tiêu chí xác định. Nhóm số liệu thường được cho dưới dạng $(a;b]$, trong đó $a$ là đầu mút trái, $b$ là đầu mút phải.
\end{dn}
\textbf{Nhận xét}
\begin{itemize}
	\item Mẫu số liệu ghép nhóm được dùng khi ta không thể thu thập được số liệu chính xác hoặc do yêu cầu của bài toán mà ta phải biểu diễn mẫu số liệu dưới dạng ghép nhóm để thuận lợi cho việc tổ chức, đọc và phân tích số liệu.
	\item Trong một số trường hợp, nhóm số liệu cuối cùng có thể lấy đầu mút bên phải.
\end{itemize}	
\subsubsection{Ghép nhóm mẫu số liệu}
Để chuyển mẫu số liệu không ghép nhóm sang mẫu số liệu ghép nhóm, ta làm như sau:
\begin{itemize}
	\item \textit{Bước 1:} Chia miền giá trị của mẫu số liệu thành một số nhóm theo tiêu chí cho trước.
	\item \textit{Bước 2:} Đếm số giá trị của mẫu số liệu thuộc nhóm (tần số) và lập bảng thống kê cho mẫu số liệu ghép nhóm.
\end{itemize}
\end{tomtat}
\subsection{Các dạng toán thường gặp}
\begin{dang}{Nhận dạng mẫu số liệu ghép nhóm}
	
\end{dang}
\subsubsection{Ví dụ mẫu}
\begin{vd}%[TeX hóa SGK KNTT]%[Ngọc Hiếu]%[1K3B8-1]
	Mẫu số liệu sau cho biết phân bố theo độ tuổi của dân số Việt Nam năm $2019$.
	\begin{center}
		\begin{tabular}{|c|c|c|c|}
			\hline
			Độ tuổi&Dưới $15$ tuổi&Từ $15$ đến dưới $65$ tuổi&Từ $65$ tuổi trở lên\\
			\hline
			Số người& $23\;371\;882$&$65\;420\;451$&$7\;416\;651$\\
			\hline
		\end{tabular}
	\end{center}
	\begin{enumerate}
		\item Mẫu số liệu đã cho có là mẫu số liệu ghép nhóm hay không?
		\item Nêu các nhóm và tần số tương ứng. Dân số Việt Nam năm $2019$ là bao nhiêu?
	\end{enumerate}
	\loigiai{
		\begin{enumerate}
			\item Mẫu số liệu đã cho là mẫu số liệu ghép nhóm.
			\item Có ba nhóm là: Dưới $15$ tuổi, Từ $15$ đến dưới $65$ tuổi, Từ $65$ tuổi trở lên. Có $23\;371\;882$ người dưới $15$ tuổi; $65\;420\;451$ người từ $15$ đến dưới $65$ tuổi và $7\;416\;651$ người từ $65$ tuổi trở lên.\\
			Dân số Việt Nam năm $2019$ là $23\;371\;882+65\;420\;451+7\;416\;651=96\;208\;984$ người. 
		\end{enumerate}
	}
\end{vd}

\begin{vd}%[TeX hóa SGK KNTT]%[Ngọc Hiếu]%[1K3B8-1]
	Bảng thống kê sau cho biết thời gian chạy (phút) của $30$ vận động viên (VĐV) trong một giải chạy Marathon.
	\begin{center}
		\begin{tabular}{|c|c|c|c|c|c|c|c|c|c|c|c|c|}
			\hline
			Thời gian&$129$&$130$&$133$&$134$&$135$&$136$&$138$&$141$&$142$&$143$&$144$&$145$\\
			\hline
			Số VĐV&$1$&$2$&$1$&$1$&$1$&$2$&$3$&$3$&$4$&$5$&$2$&$5$\\
			\hline
		\end{tabular}
	\end{center}
	Hãy chuyển mẫu số liệu trên sang mẫu số liệu ghép nhóm gồm sáu nhóm có độ dài bằng nhau và bằng $3$.
	\loigiai{
		Giá trị nhỏ nhất là $129$, giá trị lớn nhất là $145$ nên khoảng biến thiên là $145-129=16$. Tổng độ dài của sáu nhóm là $18$. Để cho đối xứng, ta chọn đầu mút trái của nhóm đầu tiên là $127{,}5$ và đầu mút phải của nhóm cuối cùng là $145{,}5$ ta được các nhóm là $[127{,}5;130{,}5),\; [130{,5};133{,5}],\ldots , [142{,}5;145{,}5]$. Đếm số giá trị thuộc mỗi nhóm, ta có mẫu số liệu ghép nhóm như sau
		\begin{center}
			\fontsize{9}{1pt}
			{\begin{tabular}{|c|c|c|c|c|c|c|}
					\hline
					Thời gian&$[125{,}5;130{,}5)$&$[130{,}5;133{,}5)$&$[133{,}5;136{,}5)$&$[136{,}5;139{,}5)$&$[139{,}5;142{,}5)$&$[142{,}5;145{,}5)$\\
					\hline
					Số VĐV&$3$&$1$&$4$&$3$&$7$&$12$\\
					\hline
			\end{tabular}}
		\end{center}
	}
\end{vd}	
\begin{vd}%[1C5B1-5]
	Mẫu số liệu sau cho biết kết quả kiểm tra môn Toán của lớp $11$A năm 2022.
	\begin{center}
		\begin{tabular}{|c|c|c|c|c|}
			\hline
			Điểm số & $[3;5)$&$[5;7)$&$[7;9)$&$[9;11)$\\
			\hline
			Số học sinh &5&18&10&7\\
			\hline
		\end{tabular}
	\end{center}
	\begin{enumerate}
		\item Mẫu số liệu đã cho có là mẫu số liệu ghép nhóm hay không?
		\item Nêu các nhóm và tần số tương ứng. Số học sinh của lớp 11A là bao nhiêu?
	\end{enumerate}
	\loigiai{
		\begin{enumerate}
			\item Mẫu số liệu đã cho là mẫu số liệu ghép nhóm.
			\item Có bốn nhóm là: Từ 3 đến dưới 5 điểm, từ 5 đến dưới 7 điểm, từ 7 đến dưới 9 điểm, từ 9 đến dưới 11 điểm.\\	
			Có 5 em từ 3 đến dưới 5 điểm; có 18 em từ 5 đến dưới 7 điểm; có 10 em từ 7 đến dưới 9 điểm; có 7 em từ 9 đến dưới 11 điểm.\\
			Số học sinh của lớp 11A là $5+18+10+7=40$ em. 
		\end{enumerate}
	}
\end{vd}
\subsubsection{Bài tập rèn luyện}
\centerline{\fcolorbox{red}{yellow!50}{\bf {BÀI TẬP TỰ LUẬN }}}
\begin{bt}%[1K3B8-1]
	Trong các mẫu số liệu sau, mẫu nào là mẫu số liệu ghép nhóm? Đọc và giải thích mẫu số liệu ghép nhóm đó.
	\begin{enumerate}
		\item Số tiền lương mà giáo viên nhận được mỗi tháng tại một trường.
		\begin{center}
			\begin{tabular}{|c|c|c|c|c|c|}
				\hline
				Số tiền (triệu đồng)&$[2;4)$&$[4;6)$&$[6;8)$&$[8;10)$&$[10;12)$\\
				\hline
				Số giáo viên&$10$&$50$&$23$&$12$&$5$\\
				\hline
			\end{tabular}
		\end{center}
		\item Thống kê thời gian giải một bài toán khó của 40 học sinh, ta có bảng số liệu sau
		\begin{center}
			\begin{tabular}{|c|c|c|c|c|}
				\hline
				Thời gian (phút)&$[5;9)$&$[9;13)$&$[13;17)$&$[17;21)$\\
				\hline
				Số học sinh &$3$&$12$&$10$&$15$\\
				\hline
			\end{tabular}
		\end{center}
	\end{enumerate}
	\loigiai{
		Cả hai mẫu số liệu trên đều là mẫu số liệu ghép nhóm.
		\begin{enumerate}
			\item Có năm nhóm là
			\begin{enumerate}[\bf --]
				\item Từ $2$ đến dưới $4$ triệu đồng có $10$ giáo viên.
				\item Từ $4$ đến dưới $6$ triệu đồng có $50$ giáo viên.
				\item Từ $6$ đến dưới $8$ triệu đồng có $23$ giáo viên.
				\item Từ $8$ đến dưới $10$ triệu đồng có $12$ giáo viên.
				\item Từ $10$ đến dưới $12$ triệu đồng có $5$ giáo viên.
			\end{enumerate}
			\item Có bốn nhóm là
			\begin{enumerate}[\bf --]
				\item Từ $5$ phút đến dưới $9$ phút có $3$ học sinh.
				\item Từ $9$ phút đến dưới $13$ phút có $12$ học sinh.
				\item Từ $13$ phút đến dưới $17$ phút có $10$ học sinh.
				\item Từ $17$ phút đến dưới $21$ phút có $15$ học sinh.
			\end{enumerate}
		\end{enumerate}
	}
\end{bt}
\begin{bt}
	Cho mẫu số liệu ghép nhóm về thời gian (phút) đi từ nhà đến nơi làm việc của các nhân viên một công ty như sau:
	\begin{center}
		\begin{tabular}{|l|c|c|c|c|c|c|c|}
			\hline Thời gian & {$[15 ; 20)$} & {$[20 ; 25)$} & {$[25 ; 30)$} & {$[30 ; 35)$} & {$[35 ; 40)$} & {$[40 ; 45)$} & {$[45 ; 50)$} \\
			\hline Số nhân viên & 6 & 14 & 25 & 37 & 21 & 13 & 9 \\
			\hline
		\end{tabular}
	\end{center}
	Đọc và giải thích mẫu số liệu này.
	\loigiai{
		Có bảy nhóm là: Từ 15 đến dưới 20 phút, từ 20 đến dưới 25 phút, từ 25 đến dưới 30 phút, từ 30 đến dưới 35 phút, từ 35 đến dưới 40 phút, từ 40 đến dưới 45 phút, từ 45 đến dưới 50 phút\\	
		Có 6 nhân viên đi làm mất từ 15 đến dưới 20 phút;
		có 14 nhân viên đi làm mất từ 20 đến dưới 25 phút;
		có 25 nhân viên đi làm mất từ 25 đến dưới 30 phút;
		có 37 nhân viên đi làm mất từ 30 đến dưới 35 phút;
		có 21 nhân viên đi làm mất từ 35 đến dưới 40 phút;
		có 13 nhân viên đi làm mất từ 40 đến dưới 45 phút;
		có 9 nhân viên đi làm mất từ 45 đến dưới 50 phút.\\
		Số nhân viên của công ty đó là $6+14+25+37+21+13+9=125$ nhân viên. 	
	}
\end{bt}
\centerline{\fcolorbox{red}{yellow!50}{\bf {CÂU HỎI TRẮC NGHIỆM (Tầm 10 - 20 câu theo theo tỉ lệ 4:3:2:1)}}}
\Opensolutionfile{ans}[ans/ans-1K1-3-Dang1]
% bài tập trắc nghiệm
\Closesolutionfile{ans}
\begin{indapan}{10}
	{ans/ans-1K1-3-Dang1}
\end{indapan}
\begin{dang}{Chuyển mẫu số liệu không ghép nhóm sang mẫu số liệu ghép nhóm}
	
\end{dang}
\subsubsection{Ví dụ mẫu}
\begin{vd}%[1K3B8-1]
	Bảng thống kê sau cho biết thời gian chạy (phút) của $30$ vận động viên (VĐV) trong một giải chạy Marathon.
	\begin{center}
		\begin{tabular}{|c|c|c|c|c|c|c|c|c|c|c|c|c|}
			\hline
			Thời gian&$129$&$130$&$133$&$134$&$135$&$136$&$138$&$141$&$142$&$143$&$144$&$145$\\
			\hline
			Số VĐV&$1$&$2$&$1$&$1$&$1$&$2$&$3$&$3$&$4$&$5$&$2$&$5$\\
			\hline
		\end{tabular}
	\end{center}
	Hãy chuyển mẫu số liệu trên sang mẫu số liệu ghép nhóm gồm sáu nhóm có độ dài bằng nhau và bằng $3$.
	\loigiai{
		Giá trị nhỏ nhất là $129$, giá trị lớn nhất là $145$ nên khoảng biến thiên là $145-129=16$. Tổng độ dài của sáu nhóm là $18$. Để cho đối xứng, ta chọn đầu mút trái của nhóm đầu tiên là $127{,}5$ và đầu mút phải của nhóm cuối cùng là $145{,}5$ ta được các nhóm là $[127{,}5;130{,}5),\; [130{,5};133{,5}],\ldots , [142{,}5;145{,}5]$. Đếm số giá trị thuộc mỗi nhóm, ta có mẫu số liệu ghép nhóm như sau
		\begin{center}
			\begin{tabular}{|c|c|c|c|c|c|c|}
				\hline
				Thời gian&$[127{,}5;130{,}5)$&$[130{,}5;133{,}5)$&$[133{,}5;136{,}5)$&$[136{,}5;139{,}5)$&$[139{,}5;142{,}5)$&$[142{,}5;145{,}5)$\\
				\hline
				Số VĐV&$3$&$1$&$4$&$3$&$7$&$12$\\
				\hline
			\end{tabular}
		\end{center}
	}
\end{vd}
\begin{vd}
	Cân nặng (kg) của 35 người trưởng thành tại một khu dân cư được cho như sau:
	\begin{center}
		\begin{tabular}{llllllllllllllllll}
			43 & 51 & 47 & 62 & 48 & 40 & 50 & 62 & 53 & 56 & 40 & 48 & 56 & 53 & 50 & 42 & 55 & \\ 
			52 & 48 & 46 & 45 & 54 & 52 & 50 & 47 & 44 & 54 & 55 & 60 & 63 & 58 & 55 & 60 & 58 & 53.
		\end{tabular}
	\end{center}
	Chuyển mẫu số liệu trên thành dạng ghép nhóm, các nhóm có độ dài bằng nhau, trong đó có nhóm $[40; 45)$.
	\loigiai{
		Vì các nhóm có độ dài bằng nhau, trong đó có nhóm $[40; 45)$ nên độ dài mỗi nhóm là $5$. Ta có mẫu số liệu ghép nhóm
		\begin{center}
			\begin{tabular}{|c|c|c|c|c|c|}
				\hline Cân nặng (kg) &$[40;45)$&$[45;50)$&$[50;55)$&$[55;60)$&$[60;65)$\\
				\hline Số người &5 & 7&11 &7 &5 \\
				\hline
			\end{tabular}
		\end{center}
	}
\end{vd}
\subsubsection{Bài tập rèn luyện}
\centerline{\fcolorbox{red}{yellow!50}{\bf {BÀI TẬP TỰ LUẬN}}}
\begin{bt}
	Một công ty may quần áo đồng phục học sinh cho biết cỡ áo theo chiều cao của học sinh được tính như sau:
	\begin{center}
		\begin{tabular}{|c|c|c|c|c|c|}
			\hline Chiều cao $(\mathrm{cm})$ & {$[150 ; 160)$} & {$[160 ; 167)$} & {$[167 ; 170)$} & {$[170 ; 175)$} & {$[175 ; 180)$} \\
			\hline Cỡ áo & $\mathrm{S}$ & $\mathrm{M}$ & $\mathrm{L}$ & $\mathrm{XL}$ & $\mathrm{XXL}$ \\
			\hline
		\end{tabular}
	\end{center}
	Công ty muốn ước lượng tỉ lệ các cỡ áo khi may cho học sinh lớp 11 đã đo chiều cao của 36 học sinh nam khối 11 của một trường và thu được mẩu số liệu sau (đơn vị là centimét):
	\begin{center}
		\begin{tabular}{lllllllllllll}
			160 & 161 & 161 & 162 & 162 & 162 & 163 & 163 & 163 & 164 & 164 & 164 & 164 \\ 
			165 & 165 & 165 & 165 & 165 & 166 & 166 & 166 & 166 & 167 & 167 & 168 & 168 \\ 
			168 & 168 & 169 & 169 & 170 & 171 & 171 & 172 & 172 & 174 & & & 
		\end{tabular}
	\end{center}
	\begin{enumerate}
		\item Lập bảng tần số ghép nhóm của mẫu số liệu với các nhóm đã cho ở bảng trên.
		\item Công ty may 500 áo đồng phục cho học sinh lớp 11 thì nên may số lượng áo theo mỗi cỡ là bao nhiêu chiếc?
	\end{enumerate}
	\loigiai{
		\begin{enumerate}
			\item Bảng tần số ghép nhóm
			\begin{center}
				\begin{tabular}{|c|c|c|c|c|c|}
					\hline Chiều cao $(\mathrm{cm})$ & {$[150 ; 160)$} & {$[160 ; 167)$} & {$[167 ; 170)$} & {$[170 ; 175)$} & {$[175 ; 180)$} \\
					\hline Số học sinh &0& 22 & 8 & 6 & 0 \\
					\hline
				\end{tabular}
			\end{center}
			\item Công ty may $500$ áo đồng phục cho học sinh lớp $11$ thì nên may số lượng áo theo mỗi cỡ như sau:
			\begin{itemize}
				\item Không nên may áo cỡ S và cỡ XXL;
				\item Số lượng áo cỡ M nên may là $\dfrac{22}{36} \cdot 500 \approx 306$ (chiếc);
				\item Số lượng áo cỡ L nên may là $\dfrac{8}{36} \cdot 500 \approx 111$ (chiếc);
				\item Số lượng áo cỡ XL nên may là 500 - 306 - 111 = 83 (chiếc).
			\end{itemize}
		\end{enumerate}	
	}
\end{bt}
%%=====Ví dụ 2
\begin{bt}%[1T5B1-1]
	Cân nặng của $28$ học sinh nam lớp $11$ được cho như sau:
	\begin{center}
		\begin{tabular}{llllllllllllll}
			$55{,}4$ & $62{,}6$ & $54{,}2$ & $56{,}8$ & $58{,}8$ & $59{,}4$ & $60{,}7$ & $58$ & $59{,}5$ & $63{,}6$ & $61{,}8$ & $52{,}3$ & $63{,}4$ & $57{,}9$\\
			$49{,}7$ & $45{,}1$ & $56{,}2$ & $63{,}2$ & $46{,}1$ & $49{,}6$ & $59{,}1$ & $55{,}3$ & $55{,}8$ & $45{,}5$ & $46{,}8$ & $54$ & $49{,}2$ & $52{,}6$
		\end{tabular}
	\end{center}
	Hãy chuyển mẫu số liệu trên sang mẫu số liệu ghép nhóm gồm $5$ nhóm có độ dài bằng nhau.
	\loigiai{
		Khoảng biến thiên của mẫu số liệu trên là $63{,}6-45{,}1=18{,}5$.
		Độ dài mỗi nhóm là $L>\dfrac{18{,}5}{5}=3{,}7$.
		Ta chọn $L=4$ và chia dữ liệu thành các nhóm $[45; 49)$, $[49; 53)$, $[53; 57)$, $[57; 61)$, $[61; 65)$.
		Khi đó ta có bảng tần số ghép nhóm sau:
		\begin{center}
			\begin{tabular}{|c|c|c|c|c|c|}
				\hline Cân nặng &{$[45; 49)$} &{$[49; 53)$} &{$[53; 57)$} &{$[57; 61)$} &{$[61; 65)$} \\
				\hline Số học sinh & 4 & 5 & 7 & 7 & 5 \\
				\hline
			\end{tabular}
		\end{center}
	}
\end{bt}
\begin{bt}%[1K3B8-1]
	Trong các mẫu số liệu sau, mẫu nào là mẫu số liệu ghép nhóm? Đọc và giải thích mẫu số liệu ghép nhóm đó.
	\begin{enumerate}
		\item Số tiền mà sinh viên chi cho thanh toán cước điện thoại trong tháng.
		\begin{center}
			\begin{tabular}{|c|c|c|c|c|c|}
				\hline
				Số tiền (nghìn đồng)&$[0;50)$&$[50;100)$&$[100;150)$&$[150;200)$&$[200;250)$\\
				\hline
				Số sinh viên&$5$&$12$&$23$&$17$&$3$\\
				\hline
			\end{tabular}
		\end{center}
		\item Thống kê nhiệt độ tại một điểm trong $40$ ngày, ta có bảng số liệu sau
		\begin{center}
			\begin{tabular}{|c|c|c|c|c|}
				\hline
				Nhiệt độ $(^\circ$ C)&$[19;22)$&$[22;25)$&$[25;28)$&$[28;31)$\\
				\hline
				Số ngày&$7$&$15$&$12$&$6$\\
				\hline
			\end{tabular}
		\end{center}
	\end{enumerate}
	\loigiai{
		Cả hai mẫu số liệu trên đều là mẫu số lớp ghép nhóm.
		\begin{enumerate}
			\item Có năm nhóm là
			\begin{enumerate}[\bf --]
				\item Dưới $50$ nghìn đồng có $5$ sinh viên.
				\item Từ $50$ đến dưới $100$ nghìn đồng có $12$ sinh viên.
				\item Từ $100$ đến dưới $150$ nghìn đồng có $23$ sinh viên.
				\item Từ $150$ đến dưới $200$ nghìn đồng có $17$ sinh viên.
				\item Từ $200$ đến dưới $250$ nghìn đồng có $3$ sinh viên.
			\end{enumerate}
			\item Có bốn nhóm là
			\begin{enumerate}[\bf --]
				\item Từ $19^\circ$ C đến dưới $22^\circ$ C có $7$ ngày.
				\item Từ $22^\circ$ C đến dưới $25^\circ$ C có $15$ ngày.
				\item Từ $25^\circ$ C đến dưới $28^\circ$ C có $12$ ngày.
				\item Từ $128^\circ$ C đến dưới $31^\circ$ C có $6$ ngày.
			\end{enumerate}
		\end{enumerate}
	}
\end{bt}
%%=====Bài 1
\begin{bt}%[1T5B1-2]%[1T5B1-3]
	Anh Văn ghi lại cự li 30 lần ném lao của mình ở bảng sau (đơn vị: mét):
	\begin{center}
		\begin{tabular}{|c|c|c|c|c|c|c|c|c|c|}
			\hline $72{,}1$ & $72{,}9$ & $70{,}2$ & $70{,}9$ & $72{,}2$ & $71{,}5$ & $72{,}5$ & $69{,}3$ & $72{,}3$ & $69{,}7$ \\
			\hline $72{,}3$ & $71{,}5$ & $71{,}2$ & $69{,}8$ & $72{,}3$ & $71{,}1$ & $69{,}5$ & $72{,}2$ & $71{,}9$ & $73{,}1$ \\
			\hline $71{,}6$ & $71{,}3$ & $72{,}2$ & $71{,}8$ & $70{,}8$ & $72{,}2$ & $72{,}2$ & $72{,}9$ & $72{,}7$ & $70{,}7$ \\
			\hline
		\end{tabular}
	\end{center}
	Tổng hợp lại kết quả ném của anh Văn vào bảng tần số ghép nhóm theo mẫu sau:
	\begin{center}
		\begin{tabular}{|c|c|c|c|c|c|}
			\hline Cự li $(\mathrm{m})$ &{$[69{,}2; 70)$} &{$[70; 70{,}8)$} &{$[70{,}8; 71{,}6)$} &{$[71{,}6; 72{,}4)$} &{$[72{,}4; 73{,}2)$} \\
			\hline Số lần & $?$ & $?$ & $?$ & $?$ & $?$ \\
			\hline
		\end{tabular}
	\end{center}
	\loigiai{
		Bảng tần số ghép nhóm kết quả ném của anh Văn:
		\begin{center}
			\begin{tabular}{|c|c|c|c|c|c|}
				\hline Cự li $(\mathrm{m})$ &{$[69{,}2; 70)$} &{$[70; 70{,}8)$} &{$[70{,}8; 71{,}6)$} &{$[71{,}6; 72{,}4)$} &{$[72{,}4; 73{,}2)$} \\
				\hline Số lần & $4$ & $2$ & $7$ & $12$ & $5$ \\
				\hline
			\end{tabular}
		\end{center}
	}
\end{bt}	
\begin{bt}%[1K3B8-1]
	Số sản phẩm một công nhân làm được trong một ngày được cho như sau:
	\begin{center}
		\begin{tabular}{c c c c c c c c c c c c c}
			$18$&$25$&$39$&$12$&$54$&$27$&$46$&$25$&$19$&$8$&$36$&$22$&\\
			$20$&$19$&$17$&$44$&$5$&$18$&$23$&$28$&$25$&$34$&$46$&$27$&$16$
		\end{tabular}
	\end{center}
	Hãy chuyển mẫu số liệu sang dạng ghép nhóm với sáu nhóm có độ dài bằng nhau.
	\loigiai{
		Khoảng biến thiên là $54-5=49$.\\
		Ta chia thành các nhóm sau $[4{,}5;13); [13;21{,}5);[21{,}5;30);\ldots ;[47;55{,}5)$.\\
		Đếm số giá trị của mỗi nhóm, ta có bảng ghép nhóm sau:
		\begin{center}
			\begin{tabular}{|c|c|c|c|c|c|c|}
				\hline
				Số sản phẩm &$[4{,}5;13)$&$[13;21{,}5)$&$[21{,}5;30)$&$[30;38{,}5)$&$[38{,}5;47)$&$[47;55{,}5)$\\
				\hline
				Số công nhân&$3$&$7$&$8$&$2$&$4$&$1$\\
				\hline
			\end{tabular}
		\end{center}
	}
\end{bt}
\begin{bt}%[1K3B8-1]
	Thời gian ra sân (giờ) của một số cựu cầu thủ ở giải ngoại hạng Anh qua các thời kì được cho như sau:
	\begin{center}
		\begin{tabular}{c c c c c c c c}
			$653$ & $632$ & $609$ & $572$ & $565$ & $535$ & $516$ & $514$ \\
			$508$ & $505$ & $504$ & $504$ & $503$ & $499$ & $496$ & $492$ 
		\end{tabular}
	\end{center}
	Hãy chuyển mẫu số liệu trên sang dạng ghép nhóm với bảy nhóm có độ dài bằng nhau.
	\loigiai{
		Khoảng biến thiên là $653-492=161$.\\
		Ta chia thành các nhóm sau $[492;515); [515;538);[538;561);\ldots; [47;55{,}5)$.\\
		Đếm số giá trị của mỗi nhóm, ta có bảng ghép nhóm sau:
		\begin{center}
			\begin{tabular}{|c|c|c|c|c|c|c|c|}
				\hline
				Thời gian (giờ) &$[492;515)$&$[515;538)$&$[538;561)$&$[561;584)$&$[584;607)$&$[607;630)$&$[630;653]$\\
				\hline
				Số cầu thủ &$9$&$2$&$0$&$2$&$0$&$1$&$2$\\
				\hline
			\end{tabular}
		\end{center}
	}
\end{bt}
\begin{bt}%[1T5B1-1]
	Một cửa hàng đã thống kê số ba lô bán được mỗi ngày trong tháng 9 với kết quả cho như sau: \begin{center}
		\begin{tabular}{lllllllllllllll}
			$12$ & $29$ & $12$ & $19$ & $15$ & $21$ & $19$ & $29$ & $28$ & $12$ & $15$ & $25$ & $16$ & $20$ & $29$\\
			$21$ & $12$ & $24$ & $14$ & $10$ & $12$ & $10$ & $23$ & $27$ & $28$ & $18$ & $16$ & $10$ & $20$ & $21$
		\end{tabular}
	\end{center}
	Hãy chuyển mẫu số liệu trên sang mẫu số liệu ghép nhóm gồm 5 nhóm có độ dài bằng nhau.
	\loigiai{
		Khoảng biến thiên của mẫu số liệu trên là $29-10=19$.
		Độ dài mỗi nhóm $L>\dfrac{19}{5}=3{,}8$.
		Ta chọn $L=4$ và chia dữ liệu thành các nhóm $[10; 14)$, $[14; 18)$, $[18; 22)$, $[22; 26)$, $[26; 30)$.
		Khi đó ta có bảng tần số ghép nhóm sau:
		\begin{center}
			\begin{tabular}{|c|c|c|c|c|c|}
				\hline Cân nặng &{$[10; 14)$} &{$[14; 18)$} &{$[18; 22)$} &{$[22; 26)$} &{$[26; 30)$} \\
				\hline Số ba lô bán được & $8$ & $5$ & $8$ & $3$ & $6$ \\
				\hline
			\end{tabular}
		\end{center}
	}
\end{bt}
\begin{bt}%[1C5B1-1]
	Một trường trung học phổ thông chọn $36$ học sinh nam của khối $11$, đo chiều cao của các bạn học sinh đó và thu được mẫu số liệu sau (đơn vị: centimét):
	$$
	\begin{array}{llllllllllll}
		160 & 161 & 161 & 162 & 162 & 162 & 163 & 163 & 163 & 164 & 164 & 164 \\
		164 & 165 & 165 & 165 & 165 & 165 & 166 & 166 & 166 & 166 & 167 & 167 \\
		168 & 168 & 168 & 168 & 169 & 169 & 170 & 171 & 171 & 172 & 172 & 174
	\end{array}
	$$
	Lập bảng tần số ghép nhóm cho mẫu số liệu trên có $5$ nhóm ứng với $5$ nửa khoảng:
	$$
	\left[160;163 \right),\ \left[163;169 \right),\ \left[166;169 \right),\ \left[169;172 \right),\ \left[172;175 \right).
	$$
	\loigiai{
		Bảng tần số ghép nhóm như sau:
		\begin{center}
			\begin{tabular}{|c|c|c|c|c|c|}
				\hline Chiều cao& $\left[160;163\right)$ & $\left[163;166\right)$ & $\left[166;169\right)$ & $\left[169;172\right)$ & $\left[172;175\right)$\\
				\hline Số HS nam& $6$& $12$ & $10$ & $5$ & $3$\\
				\hline
			\end{tabular}
		\end{center}
	}
\end{bt}
%%%%%%%%%%%%%%%%%
\begin{bt}%[1C5B1-5]
	Mẫu số liệu dưới đây ghi lại tốc độ của $40$ ô tô khi đi qua một trạm đo tốc độ (đơn vị: km/h):
	\begin{center}
		\begin{tabular}{llllllllll}
			48{,}5 & 43 & 50 & 55 & 45 & 60 & 53 & 55,5 & 44 & 65 \\
			51 & 62,5 & 41 & 44,5 & 57 & 57 & 68 & 49 & 46{,}5 & 53{,}5 \\
			61 & 49{,}5 & 54 & 62 & 59 & 56 & 47 & 50 & 60 & 61 \\
			49{,}5 & 52{,}5 & 57 & 47 & 60 & 55 & 45 & 47,5 & 48 & 61{,}5
		\end{tabular}
	\end{center}
	Hãy chuyển mẫu số liệu trên sang dạng ghép nhóm với 6 nhóm có độ dài bằng nhau.
	\loigiai{
		Ta có bảng tần số ghép nhóm của mẫu số liệu trên như sau:
		\begin{center}
			\begin{tabular}{|c|c|c|c|c|c|c|}
				\hline Tốc độ (km/h) &
				$\left[40;45\right)$ & 
				$\left[45;50\right)$ & 
				$\left[50;55\right)$ & 
				$\left[55;60\right)$ &
				$\left[60;65\right)$ &
				$\left[65;70\right)$ \\
				\hline Số ôtô & $4$& $11$ & $7$ & $8$ & $8$&$2$\\
				\hline
			\end{tabular}
		\end{center}
	}
\end{bt}

\begin{bt}%[1C5B1-5]
	Mẫu số liệu sau ghi lại cân nặng của $30$ bạn học sinh (đơn vị: kilôgam):
	\begin{center}
		\begin{tabular}{llllllllll}
			17 & 40 & 39 & 40{,}5 & 42 & 51 & 41{,}5 & 39 & 41 & 30\\
			40 & 42 & 40{,}5 & 39{,}5 & 41 & 40{,}5 & 37 & 39{,}5 & 40 & 41\\
			38{,}5 & 39{,}5 & 40 & 41 & 39 & 40{,}5 & 40 & 38{,}5 & 39{,}5 & 41{,}5
		\end{tabular}
	\end{center}
	Hãy chuyển mẫu số liệu trên sang dạng ghép nhóm với 8 nhóm có độ dài bằng nhau.
	\loigiai{
		Ta có bảng tần số ghép nhóm của mẫu số liệu trên như sau:
		\begin{center}
			\begin{tabular}{|c|c|c|c|c|c|c|c|c|}
				\hline Tốc độ (km/h) &
				$\left[15;20\right)$ &
				$\left[20;25\right)$ &  
				$\left[25;30\right)$ & 
				$\left[30;35\right)$ & 
				$\left[35;40\right)$ &
				$\left[40;45\right)$ & 
				$\left[45;50\right)$ & 
				$\left[50;55\right)$\\
				\hline Số ôtô & 
				1 &
				0 &
				0 &
				1 &
				10 &
				17 &
				0 &
				1 \\
				\hline
			\end{tabular}
		\end{center}
	}
\end{bt}
%%%%%%%%%%%%%%%%%%
%%=====Bài 2
\begin{bt}%[1T5B1-2]%[1T5B1-3]
	Người ta đếm số xe ô tô đi qua một trạm thu phí mỗi phút trong khoảng thời gian từ $9$ giờ đến $9$ giờ $30$ phút sáng. Kết quả được ghi lại ở bảng sau:
	\begin{center}
		\begin{tabular}{|c|c|c|c|c|c|c|c|c|c|c|c|c|c|c|}
			\hline $15$ & $16$ & $13$ & $21$ & $17$ & $23$ & $15$ & $21$ & $6$ & $11$ & $12$ & $23$ & $19$ & $25$ & $11$ \\
			\hline $25$ & $7$ & $29$ & $10$ & $28$ & $29$ & $24$ & $6$ & $11$ & $23$ & $11$ & $21$ & $9$ & $27$ & $15$ \\
			\hline
		\end{tabular}
	\end{center}
	Hãy chuyển mẫu số liệu trên sang dạng ghép nhóm với 5 nhóm có độ dài bằng nhau.
	\loigiai{
		Bảng tần số ghép nhóm
		\begin{center}
			\begin{tabular}{|c|c|c|c|c|c|}
				\hline Số xe &{$[6; 10]$} &{$[11; 15]$} &{$[16; 20]$} &{$[21; 25]$} &{$[26; 30]$} \\
				\hline Số lần & $5$ & $9$ & $3$ & $9$ & $4$ \\
				\hline
			\end{tabular}
		\end{center}
	}
\end{bt}
%\centerline{\fcolorbox{red}{yellow!50}{\bf {CÂU HỎI TRẮC NGHIỆM (Tầm 10 - 20 câu theo theo tỉ lệ 4:3:2:1)}}}
%\Opensolutionfile{ans}[ans/ans-1K1-3-Dang2]\textsl{}
%
%%bài tập trắc nghiệm
%\Closesolutionfile{ans}
%\begin{indapan}{10}
%	{ans/ans-1K1-3-Dang2}
%\end{indapan}