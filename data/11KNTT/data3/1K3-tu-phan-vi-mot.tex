
\section{Các số đặc trưng đo xu thế trung tâm}
\subsection{Tóm tắt lý thuyết}
\begin{tomtat}
	\subsubsection{Tứ phân vị}
	\begin{itemize}
		\item Để tính tứ phân vị thứ nhất $Q_1$ của mằu số liệu ghép nhóm, trước hết ta xác định nhóm chứa $Q_1$, giả sử đó là nhóm thứ $p$: $\left[a_p ; a_{p-1}\right)$. Khi đó,
		$$
		Q_1=a_p+\dfrac{\frac{n}{4}-\left(m_1+\ldots+m_{p-1}\right)}{m_p}\cdot\left(a_{p+1}-a_p\right),
		$$
		trong đó, $n$ là cỡ mẫu, $m_p$ là tần số nhóm $p$, với $p=1$ ta quy ước $m_1+\ldots+m_{p-1}=0$.
		\item Để tính tứ phân vị thứ ba $Q_1$ của mẫu số liệu ghép nhóm, trước hết ta xác định nhóm chứa $Q_3$. Giả sử đó là nhóm thứ $p$: $\left[a_p ; a_{p-1}\right)$. Khi đó,
		$$
		Q_3=a_p+\dfrac{\frac{3 n}{4}-\left(m_1+\ldots+m_{p-1}\right)}{m_p}\cdot\left(a_{p+1}-a_p\right) .
		$$
		trong đó, $n$ là cỡ mẫu, $m_p$ là tần số nhóm $p$, với $p=1$ ta quy ước $m_1+\ldots+m_{p-1}=0$.
		\item Tứ phân vị thứ hai $Q_2$ chính là trung vị $M_{\text{e}}$.
	\end{itemize}
	\textbf{Ý nghĩa.} Các tứ phân vị của mẫu số liệu ghép nhóm xấp xỉ cho các tứ phân vị của mẫu số liệu gốc, chúng chia mẫu số liệu thành $4$ phần, mối phần chứa $25 \%$ giá trị.\\
	\textbf{Nhận xét.} Ta cũng có thể xác định nhóm chứa tứ phân vị thứ $r$ nhờ tính chất: có khoảng $\left(\dfrac{r \cdot n}{4}\right)$ giá trị nhỏ hơn tứ phân vị này.
	\subsubsection{Mốt}
	Để tìm mốt của mẫu số liệu ghép nhóm, ta thực hiện theo các bước sau:
	\begin{itemize}
		\item Bước 1. Xác định nhóm có tần số lớn nhất (gọi là nhóm chứa mốt), giả sử là nhóm $j$: $\left[a_j;a_{j+1}\right)$.
		\item Bước 2. Mốt được xác định là
		$$M_0=a_j+\frac{m_j-m_{j-1}}{\left(m_j-m_{j-1}\right)+\left(m_j-m_{j+1}\right)}\cdot h$$ trong đó $m_j$ là tần số của nhóm $j$ (quy ước $m_0=m_{k+1}=0$ ) và $h$ là độ dài của nhóm.
	\end{itemize}
	\textbf{Lưu ý.} Người ta chỉ định nghĩa mốt cho mẫu ghép nhóm có độ dài các nhóm bằng nhau. Một mẫu có thể không có mốt hoặc có nhiều hơn một mốt.
	Khi tần số của các nhóm số liệu bằng nhau thì mẫu số liệu ghép nhóm không có mốt. 
\end{tomtat}
\subsection{Các dạng toán thường gặp}
\begin{dang}{Tứ phân vị}
	Tứ phân vị $Q_1$, $Q_2$, $Q_3$ được xác định như sau
	$$
	Q_1=a_p+\dfrac{\frac{n}{4}-\left(m_1+\ldots+m_{p-1}\right)}{m_p}\cdot\left(a_{p+1}-a_p\right);
	$$
	$$
	Q_3=a_p+\dfrac{\frac{3 n}{4}-\left(m_1+\ldots+m_{p-1}\right)}{m_p}\cdot\left(a_{p+1}-a_p\right);
	$$
	$$Q_2=M_{\mathrm{e}}=a_p+\dfrac{\frac{n}{2}-\left(m_1+\ldots+m_{p-1}\right)}{m_p}\cdot\left(a_{p+1}-a_p\right).$$
\end{dang}
\subsubsection{Ví dụ minh hoạ}
\begin{vd}%1%[1K3B9-3]
	Tìm tứ phân vị thứ nhất $Q_1$ và tứ phân vị thứ ba $Q_3$ của mẫu số liệu ghép nhóm cho trong bảng sau
	\begin{center}
		\begin{tabular}{|l|c|c|c|c|c|}
			\hline Thời gian (phút)  &{$[9,5 ; 12,5)$}&{$[12,5 ; 15,5)$}&{$[15,5 ; 18,5)$}&{$[18,5 ; 21,5)$}&{$[21,5 ; 24,5)$}\\
			\hline Số học sinh  & $3$ & $12$ & $15$ & $24$ & $2$ \\
			\hline
		\end{tabular}     
	\end{center}
	\loigiai{
		Cỡ mẫu là $n=56$.\\
		Tứ phân vị thứ nhất $Q_1$ là $\dfrac{x_{14}+x_{15}}{2}$. Do $x_{14}$, $x_{15}$ đều thuộc nhóm $[12,5 ; 15,5)$ nên nhóm này chứa $Q_1$. \\Do đó, $p=2 ; \;a_2=12,5 ;\; m_2=12 ; \;m_1=3, \;a_3-a_2=3$ và ta có
		$$
		Q_1=12,5+\dfrac{\frac{56}{4}-3}{12}\cdot 3=15,25.
		$$
		Với tứ phân vị thứ ba $Q_3$ là $\dfrac{x_{42}+x_{43}}{2}$. Do $x_{42},\; x_{43}$ đều thuộc nhóm $[18,5 ; 21,5)$ nên nhóm này chứa $Q_3$. Do đó, $p=4 ; \;a_4=18,5 ;\; m_4=24 ; \;m_1+m_2+m_3=3+12+15=30 ; \;a_5-a_4=3$ và ta có
		$$
		Q_3=18,5+\dfrac{\frac{3 \cdot 56}{4}-30}{24}\cdot 3=20.
		$$
	}
\end{vd}

\begin{vd}%2%[1K3B9-3]
	Trong phòng thí nghiệm, người ta chia $99$ mẫu vật thành năm nhóm căn cứ trên khối lượng của chúng (đơn vị: gam) và lập bảng tần số ghép nhóm như sau.
	\begin{center}
		\begin{tabular}{|l|c|c|c|c|c|}
			\hline Nhóm (gam)  &{$[27,5 ; 32,5)$}&{$[32,5 ; 37,5)$}&{$[37,5 ; 42,5)$}&{$[42,5 ; 47,5)$}&{$[47,5 ; 52,5)$}\\
			\hline Số mẫu vật  & $16$ & $24$ & $20$ & $30$ & $9$ \\
			\hline
		\end{tabular}     
	\end{center}
	Tìm tứ phân vị của mẫu số liệu.
	\loigiai{
		Cỡ mẫu là $n=99$.\\
		Gọi $x_1, \ldots, x_{99}$ là mẫu số liệu được sắp xếp theo thứ tự không giảm. Khi đó, trung vị là $x_{50}$. Do $x_{50}$ thuộc nhóm $[37,5 ; 42,5)$ nên nhóm này chứa $Q_2$. \\Do đó, $p=3 ; \;a_3=37,5 \;m_3=20;\; m_1+m_2=40 ;  \;a_4-a_3=5$ và ta có
		$$
		Q_2=37,5+\dfrac{\frac{99}{2}-40}{20}\cdot 5=39,875.
		$$
		Tứ phân vị thứ nhất $Q_1$ là $x_{25}$. Do $x_{25}$ đều thuộc nhóm $[32,5 ; 37,5)$ nên nhóm này chứa $Q_1$. \\Do đó, $p=2 ; \;a_2=32,5 ;\; m_2=24 ; \;m_1=16, \;a_3-a_2=5$ và ta có
		$$
		Q_1=32,5+\dfrac{\frac{99}{4}-16}{24}\cdot 5\approx34,323.
		$$
		Với tứ phân vị thứ ba $Q_3$ là $x_{75}$. Do $x_{75}$ đều thuộc nhóm $[42,5 ; 47,5)$ nên nhóm này chứa $Q_3$. Do đó, $p=4 ; \;a_4=42,5 ;\; m_4=30 ; \;m_1+m_2+m_3=3+12+15=60 ; \;a_5-a_4=5$ và ta có
		$$
		Q_3=42,5+\dfrac{\frac{3 \cdot 99}{4}-60}{30}\cdot 5=44,875.
		$$
	}
\end{vd}
\begin{vd}%3%[1K3B9-3]
	Sau khi điều tra về số học sinh trong $100$ lớp học, người ta chia mẫu số liệu đó thành năm nhóm căn cứ vào số lượng học sinh của mỗi lớp (đơn vị: học sinh) và lập bảng lần số ghép nhóm như sau. Tìm tứ phân vị của mẫu số liệu đó.
	\begin{center}
		\begin{tabular}{|l|c|c|c|c|c|}
			\hline Nhóm (học sinh)  &{$[36 ; 38)$}&{$[38 ; 40)$}&{$[40 ; 42)$}&{$[42 ; 44)$}&{$[44 ; 46)$}\\
			\hline Số lớp & $9$ & $15$ & $25$ & $30$ & $21$ \\
			\hline
		\end{tabular}     
	\end{center}
	\loigiai{
		Cỡ mẫu là $n=100$.\\
		Gọi $x_1, \ldots, x_{100}$ là mẫu số liệu được sắp xếp theo thứ tự không giảm. Khi đó, trung vị là $\dfrac{x_{50}+x_{51}}{2}$. Do $x_{50}$, $x_{51}$ thuộc nhóm $[42 ; 44)$ nên nhóm này chứa $Q_2$. \\Do đó, $p=4 ; \;a_4=42 \;m_4=30;\; m_1+m_2+m_3=49 ;  \;a_5-a_4=2$ và ta có
		$$
		Q_2=42+\dfrac{\frac{100}{2}-49}{30}\cdot 2\approx42,06.
		$$
		Tứ phân vị thứ nhất $Q_1$ là $\dfrac{x_{25}+x_{26}}{2}$. Do $x_{25}$, $x_{26}$ đều thuộc nhóm $[40 ; 42)$ nên nhóm này chứa $Q_1$. \\Do đó, $p=3 ; \;a_3=40 ;\; m_3=25 ; \;m_1+m_2=24, \;a_4-a_3=2$ và ta có
		$$
		Q_1=40+\dfrac{\frac{100}{4}-24}{25}\cdot 2\approx40,08.
		$$
		Với tứ phân vị thứ ba $Q_3$ là $\dfrac{x_{75}+x_{76}}{2}$. Do $x_{75}$, $x_{76}$ đều thuộc nhóm $[42 ; 44)$ nên nhóm này chứa $Q_3$. Do đó, $p=4 ; \;a_4=42 ;\; m_4=30 ; \;m_1+m_2+m_3=3+12+15=49 ; \;a_5-a_4=2$ và ta có
		$$
		Q_3=42+\dfrac{\frac{3 \cdot 100}{4}-49}{30}\cdot 2\approx43,7.
		$$
	}
\end{vd}
\begin{vd}%4%[1K3B9-3]
	Giáo viên chủ nhiệm chia thời gian sử dụng Internet trong một ngày của 40 học sinh thành năm nhóm (đơn vị: phút) và lập bảng tẫn số ghép nhóm như sau.
	Tìm tứ phân vị của mẫu số liệu đó.
	\begin{center}
		\begin{tabular}{|l|c|c|c|c|c|}
			\hline Nhóm (phút)  &{$[0 ; 60)$}&{$[60 ; 120)$}&{$[120 ; 180)$}&{$[180 ; 240)$}&{$[240 ; 300)$}\\
			\hline Số học sinh & $6$ & $13$ & $13$ & $6$ & $2$ \\
			\hline
		\end{tabular}     
	\end{center}
	\loigiai{
		Cỡ mẫu là $n=40$.\\
		Gọi $x_1, \ldots, x_{40}$ là mẫu số liệu được sắp xếp theo thứ tự không giảm. Khi đó, trung vị là $\dfrac{x_{20}+x_{21}}{2}$. Do $x_{20}$, $x_{21}$ thuộc nhóm $[120 ; 180)$ nên nhóm này chứa $Q_2$. \\Do đó, $p=3 ; \;a_3=120 \;m_3=13;\; m_1+m_2=19 ;  \;a_4-a_3=60$ và ta có
		$$
		Q_2=120+\dfrac{\frac{40}{2}-19}{13}\cdot 60\approx124,62.
		$$
		Tứ phân vị thứ nhất $Q_1$ là $\dfrac{x_{10}+x_{11}}{2}$. Do $x_{10}$, $x_{11}$ đều thuộc nhóm $[60 ; 120)$ nên nhóm này chứa $Q_1$. \\Do đó, $p=2 ; \;a_2=60 ;\; m_2=13 ; \;m_1=6, \;a_3-a_2=60$ và ta có
		$$
		Q_1=60+\dfrac{\frac{40}{4}-6}{13}\cdot 60\approx78,46.
		$$
		Với tứ phân vị thứ ba $Q_3$ là $\dfrac{x_{30}+x_{31}}{2}$. Do $x_{30}$, $x_{31}$ đều thuộc nhóm $[120 ; 180)$ nên nhóm này chứa $Q_3$. Do đó, $p=3 ; \;a_3=120 ;\; m_3=13 ; \;m_1+m_2=19 ; \;a_4-a_3=60$ và ta có
		$$
		Q_3=120+\dfrac{\frac{3 \cdot 40}{4}-19}{13}\cdot 60\approx170,77.
		$$
	}
\end{vd}
\begin{vd}%[1K3B9-3]
	Thời gian luyện tập trong một ngày (tính theo giờ) của một số vận động viên được ghi lại ở bảng sau
	\begin{center}
		\begin{tabular}{|l|c|c|c|c|c|}
			\hline Thời gian luyện tập (giờ)  &{$[0 ; 2)$}&{$[2 ;4)$}&{$[4 ; 6)$}&{$[6 ; 8)$}&{$[8 ; 10)$}\\
			\hline Số vận động viên & $3$ & $8$ & $12$ & $12$ & $4$ \\
			\hline
		\end{tabular}     
	\end{center}
	Tìm tứ phân vị của mẫu số liệu đó.
	\loigiai{
		Cỡ mẫu là $n=39$.\\
		Gọi $x_1, \ldots, x_{39}$ là mẫu số liệu được sắp xếp theo thứ tự không giảm. Khi đó, trung vị là $x_{20}$. Do $x_{20}$ thuộc nhóm $[4 ; 6)$ nên nhóm này chứa $Q_2$. \\Do đó, $p=3 ; \;a_3=4 \;m_3=12;\; m_1+m_2=11 ;  \;a_4-a_3=2$ và ta có
		$$
		Q_2=4+\dfrac{\frac{39}{2}-11}{12}\cdot 2\approx5,417.
		$$
		Tứ phân vị thứ nhất $Q_1$ là $x_{10}$. Do $x_{10}$  thuộc nhóm $[2;4)$ nên nhóm này chứa $Q_1$. \\Do đó, $p=2 ; \;a_2=2 ;\; m_2=8 ; \;m_1=3, \;a_3-a_2=2$ và ta có
		$$
		Q_1=2+\dfrac{\frac{39}{4}-3}{8}\cdot 2=3,6875.
		$$
		Với tứ phân vị thứ ba $Q_3$ là $x_{30}$. Do $x_{30}$  thuộc nhóm $[6;8)$ nên nhóm này chứa $Q_3$. Do đó, $p=3 ; \;a_3=4 ;\; m_3=12 ; \;m_1+m_2=11 ; \;a_4-a_3=2$ và ta có
		$$
		Q_3=4+\dfrac{\frac{3 \cdot 39}{4}-11}{12}\cdot 2\approx7,042.
		$$ 
	}
\end{vd}

\subsubsection{Bài tập rèn luyện}
\begin{ex}%1%[1K3B9-3]
	Điểm thi môn Toán (thang điểm 100, điểm được làm tròn đến 1) của 60 thí sinh được cho trong bảng sau
	\begin{center}
		\begin{tabular}{|l|c|c|c|c|c|}
			\hline Điểm & $[0-9,5)$ & $[9,5-19,5)$ & $[19,5-29,5)$ & $[29,5-39,5)$ & $[39,5-49,5)$ \\
			\hline Số thí sinh & $1 $& $2$ & $4$ & $6$ & $15$ \\
			\hline Điểm & $[49,5-59,5)$ & $[59,5-69,5)$ & $[69,5-79,5)$ & $[79,5-89,5)$ & $[89,5-99,5)$ \\
			\hline Số thí & $12$ & $10$ & $6$ & $3$ & $1$ \\
			\hline
		\end{tabular}    
	\end{center}
	Tìm  tứ phân vị thứ hai của mẫu số liệu.
	\choice
	{\True $Q_2\approx 51,17$}
	{$Q_2\approx 51,67$}
	{$Q_2\approx 49,5$}
	{$Q_2\approx 41,3$}
	\loigiai{Cỡ mẫu là $n=60$.\\
		Tứ phân vị thứ nhất $Q_2$ là $\dfrac{x_{30}+x_{31}}{2}$. Do $x_{30}$, $x_{31}$ đều thuộc nhóm $[49,5 ; 59,5)$ nên nhóm này chứa $Q_2$. \\Do đó, $p=6 ; \;a_6=49,5 ;\; m_6=12 ; \;m_1+\ldots+m_5=28, \;a_7-a_6=10$ và ta có
		$$
		Q_2=49,5+\dfrac{\frac{60}{2}-28}{12}\cdot 10\approx51,17.
		$$
	}
\end{ex}
%2
\begin{ex}%[1K3B9-3]
	Điểm thi môn Toán (thang điểm 100, điểm được làm tròn đến 1) của $60$ thí sinh được cho trong bảng sau
	\begin{center}
		\begin{tabular}{|l|c|c|c|c|c|}
			\hline Điểm & $[0-9,5)$ & $[9,5-19,5)$ & $[19,5-29,5)$ & $[29,5-39,5)$ & $[39,5-49,5)$ \\
			\hline Số thí sinh & $1 $& $2$ & $4$ & $6$ & $15$ \\
			\hline Điểm & $[49,5-59,5)$ & $[59,5-69,5)$ & $[69,5-79,5)$ & $[79,5-89,5)$ & $[89,5-99,5)$ \\
			\hline Số thí & $12$ & $10$ & $6$ & $3$ & $1$ \\
			\hline
		\end{tabular}    
	\end{center}
	Tìm  tứ phân vị thứ nhất của mẫu số liệu.
	\choice
	{\True $Q_1\approx 41,3$}
	{$Q_1\approx 51,67$}
	{$Q_1\approx 40,83$}
	{$Q_1\approx 51,17$}
	\loigiai{Cỡ mẫu là $n=60$.\\
		Tứ phân vị thứ nhất $Q_1$ là $\dfrac{x_{15}+x_{16}}{2}$. Do $x_{15}$, $x_{16}$ đều thuộc nhóm $[39,5-49,5)$ nên nhóm này chứa $Q_1$. \\Do đó, $p=5 ; \;a_5=39,5 ;\; m_5=15 ; \;m_1+\ldots+m_4=13, \;a_6-a_5=10$ và ta có
		$$
		Q_1=39,5+\dfrac{\frac{60}{4}-13}{15}\cdot 10\approx 40,83.
		$$
	}
\end{ex}
%3
\begin{ex}%[1K3B9-3]
	Phỏng vấn một số học sinh khối 11 vể thời gian (giờ) ngủ của một buổi tối, thu được bảng số liệu như sau.
	\begin{center}
		\begin{tabular}{|l|c|c|c|c|c|}
			\hline Thời gian  (giờ)  &{$[4 ; 5)$}&{$[5 ; 6)$}&{$[6 ; 7)$}&{$[7 ; 8)$}&{$[8 ; 9)$}\\
			\hline Số học sinh & $6$ & $10$ & $13$ & $9$ & $7$ \\
			\hline
		\end{tabular}     
	\end{center}
	Hãy cho biết $75 \%$ học sinh khối 11 ngủ ít nhất bao nhiêu giờ?
	\choice
	{$7,675$}
	{\True $7,53$}
	{$8$}
	{ $7,9$}
	\loigiai{
		Cỡ mẫu là $n=45$.\\
		Gọi $x_1, \ldots, x_{45}$ là mẫu số liệu được sắp xếp theo thứ tự không giảm. Khi đó, trung vị là $x_{23}$. Do đó, tứ phân vị thứ ba $Q_3$ là $x_{34}$. Do $x_{34}$ đều thuộc nhóm $[7;8)$ nên nhóm này chứa $Q_3$. Do đó, $p=4 ; \;a_4=7 ;\; m_4=9 ; \;m_1+m_2+m_3=29 ; \;a_5-a_4=1$ và ta có
		$$
		Q_3=7+\dfrac{\frac{3 \cdot 45}{4}-29}{9}\cdot 1\approx7,53.
		$$ 
		Vậy $75\%$ học sinh khối 11 ngủ ít nhất $7,53$ giờ.
	}
\end{ex}
%4
\begin{ex}%[1K3B9-3]
	Điểm thi môn Toán (thang điểm 100, điểm được làm tròn đến 1) của 60 thí sinh được cho trong bảng sau
	\begin{center}
		\begin{tabular}{|l|c|c|c|c|c|}
			\hline Điểm & $[0-9,5)$ & $[9,5-19,5)$ & $[19,5-29,5)$ & $[29,5-39,5)$ & $[39,5-49,5)$ \\
			\hline Số thí sinh & $1 $& $2$ & $4$ & $6$ & $15$ \\
			\hline Điểm & $[49,5-59,5)$ & $[59,5-69,5)$ & $[69,5-79,5)$ & $[79,5-89,5)$ & $[89,5-99,5)$ \\
			\hline Số thí & $12$ & $10$ & $6$ & $3$ & $1$ \\
			\hline
		\end{tabular}    
	\end{center}
	Tìm  tứ phân vị thứ ba của mẫu số liệu.
	\choice
	{$Q_3=41,3$}
	{$Q_3=51,67$}
	{$Q_3=45$}
	{\True $Q_3=65$}
	\loigiai{Cỡ mẫu là $n=60$.\\
		Với tứ phân vị thứ ba $Q_3$ là $\dfrac{x_{45}+x_{46}}{2}$. Do $x_{45}$, $x_{46}$ đều thuộc nhóm $[60 ; 70)$ nên nhóm này chứa $Q_3$. Do đó, $p=7 ; \;a_7=60 ;\; m_7=10 ; \;m_1+\ldots+m_6=40 ; \;a_8-a_7=10$ và ta có
		$$
		Q_3=59,5+\dfrac{\frac{3 \cdot 60}{4}-40}{10}\cdot 10=64,5.
		$$
	}
\end{ex}
%5
\begin{ex}%[1K3B9-3]
	Một hãng xe ô tô thống kê lại số lần gặp sự cố về động cơ về động cơ của $100$ chiếc xe cùng loại sau 2 năm sử dụng đầu tiên ở dảng sau
	\begin{center}
		\begin{tabular}{|l|c|c|c|c|c|}
			\hline Số lần gặp sự cố  &{$[0,5;2,5)$}&{$[2,5;4,5)$}&{$[4,5;6,5)$}&{$[6,5 ; 8,5)$}&{$[8,5;10,5)$}\\
			\hline Số xe & $17$ & $33$ & $25$ & $20$ & $5$ \\
			\hline
		\end{tabular}     
	\end{center}
	Tìm   tứ phân vị thứ nhất của mẫu số liệu.
	\choice
	{\True $Q_1\approx 4$}
	{$Q_1\approx 2,98$}
	{$Q_1\approx 2,5$}
	{$Q_1\approx 3,5$}
	\loigiai{
		Cỡ mẫu là $n=100$.\\
		Gọi $x_1, \ldots, x_{100}$ là mẫu số liệu được sắp xếp theo thứ tự không giảm. Khi đó, trung vị là $\dfrac{x_{50}+x_{51}}{2}$. 
		Do đó, tứ phân vị thứ nhất $Q_1$ là $\dfrac{x_{25}+x_{26}}{2}$. Do $x_{25}$, $x_{26}$ đều thuộc nhóm $[2,5;4,5)$ nên nhóm này chứa $Q_1$. \\Do đó, $p=2 ; \;a_2=2,5;\; m_2=33 ; \;m_1=17, \;a_3-a_2=2$ và ta có
		$$
		Q_1=2,5+\dfrac{\frac{100}{4}-17}{33}\cdot 2\approx 2,98.
		$$
	}    
\end{ex}
%6
\begin{ex}%[1K3B9-3]
	Một hãng xe ô tô thống kê lại số lần gặp sự cố về động cơ về động cơ của $100$ chiếc xe cùng loại sau 2 năm sử dụng đầu tiên ở dảng sau
	\begin{center}
		\begin{tabular}{|l|c|c|c|c|c|}
			\hline Số lần gặp sự cố  &{$[0,5;2,5)$}&{$[2,5;4,5)$}&{$[4,5;6,5)$}&{$[6,5 ; 8,5)$}&{$[8,5;10,5)$}\\
			\hline Số xe & $17$ & $33$ & $25$ & $20$ & $5$ \\
			\hline
		\end{tabular}     
	\end{center}
	Tìm   tứ phân vị thứ hai của mẫu số liệu.
	\choice
	{\True $Q_2=4,5$}
	{$Q_2\approx 5,12$}
	{$Q_2\approx 4,89$}
	{$Q_2\approx 5,2$}
	\loigiai{
		Cỡ mẫu là $n=100$.\\
		Gọi $x_1, \ldots, x_{100}$ là mẫu số liệu được sắp xếp theo thứ tự không giảm. Khi đó, trung vị là $\dfrac{x_{50}+x_{51}}{2}$. Do $x_{50} \in [2,5;4,5)$, $x_{51} \in [4,5;6,5)$  nên tứ phân vị thứ hai của mẫu số liệu ghép nhóm là  $Q_2=4,5$. 
	}
\end{ex}
%7
\begin{ex}%[1K3B9-3]
	Một hãng xe ô tô thống kê lại số lần gặp sự cố về động cơ về động cơ của $100$ chiếc xe cùng loại sau 2 năm sử dụng đầu tiên ở dảng sau
	\begin{center}
		\begin{tabular}{|l|c|c|c|c|c|}
			\hline Số lần gặp sự cố  &{$[0,5;2,5)$}&{$[2,5;4,5)$}&{$[4,5;6,5)$}&{$[6,5 ; 8,5)$}&{$[8,5;10,5)$}\\
			\hline Số xe & $17$ & $33$ & $25$ & $20$ & $5$ \\
			\hline
		\end{tabular}     
	\end{center}
	Tìm   tứ phân vị thứ ba của mẫu số liệu.  
	\choice
	{$Q_3=6,3$}
	{$Q_3=6,8$}
	{$Q_3=7,2$}
	{\True $Q_3=6,5$}
	\loigiai{ Cỡ mẫu là $n=100$.\\
		Với tứ phân vị thứ ba $Q_3$ là $\dfrac{x_{75}+x_{76}}{2}$. Do $x_{75} \in [4,5;6,5)$, $x_{76} \in [6,5 ; 8,5)$  nên tứ phân vị thứ ba của mẫu số liệu ghép nhóm là $Q_3=6,5$. 
		
	}
\end{ex}
%8
\begin{ex}%[1K3B9-3]
	Lương tháng của một số nhân viên văn phòng được ghi lại như sau (đơn vị: triệu đồng)
	\begin{center}
		\begin{tabular}{|l|c|c|c|c|c|}
			\hline Lương tháng (triệu đồng)  &{$[6;8)$}&{$[8;10)$}&{$[10;12)$}&{$[12;14)$}\\
			\hline Số nhân viên & $3$ & $6$ & $8$ & $7$  \\
			\hline
		\end{tabular}     
	\end{center}  
	Tìm tứ phân vị thứ nhất của mẫu số liệu.
	\choice
	{\True $Q_1= 9$}
	{$Q_1= 8,5$}
	{$Q_1= 9,5$}
	{$Q_1= 8,2$}
	\loigiai{
		Cỡ mẫu là $n=24$.\\
		Gọi $x_1, \ldots, x_{24}$ là mẫu số liệu được sắp xếp theo thứ tự không giảm. Khi đó, trung vị là $\dfrac{x_{12}+x_{13}}{2}$. 
		Do đó, tứ phân vị thứ nhất $Q_1$ là $\dfrac{x_{6}+x_{7}}{2}$. Do $x_{6}$, $x_{7}$ đều thuộc nhóm $[8;10)$ nên nhóm này chứa $Q_1$. \\Do đó, $p=2 ; \;a_2=8;\; m_2=6 ; \;m_1=3, \;a_3-a_2=2$ và ta có
		$$
		Q_1=8+\dfrac{\frac{24}{4}-3}{6}\cdot 2=9.
		$$
	}    
\end{ex}
%9
\begin{ex}%[1K3B9-3]
	Lương tháng của một số nhân viên văn phòng được ghi lại như sau (đơn vị: triệu đồng)
	\begin{center}
		\begin{tabular}{|l|c|c|c|c|c|}
			\hline Lương tháng (triệu đồng)  &{$[6;8)$}&{$[8;10)$}&{$[10;12)$}&{$[12;14)$}\\
			\hline Số nhân viên & $3$ & $6$ & $8$ & $7$  \\
			\hline
		\end{tabular}     
	\end{center}   
	Tìm   tứ phân vị thứ hai của mẫu số liệu.
	\choice
	{\True $Q_2=10,75$}
	{$Q_2= 10,5$}
	{$Q_2= 11$}
	{$Q_2=11,5$}
	\loigiai{
		Cỡ mẫu là $n=24$.\\
		Gọi $x_1, \ldots, x_{24}$ là mẫu số liệu được sắp xếp theo thứ tự không giảm. Khi đó, trung vị là $\dfrac{x_{12}+x_{13}}{2}$. 
		Do đó,  tứ phân vị thứ hai $Q_2$ là $\dfrac{x_{12}+x_{13}}{2}$. Do $x_{12}$, $x_{13}$ đều thuộc nhóm $[10;12)$ nên nhóm này chứa $Q_2$. \\Do đó, $p=3 ; \;a_3=10;\; m_3=8 ; \;m_1+m_2=9, \;a_4-a_3=2$ và ta có
		$$
		Q_2=10+\dfrac{\frac{24}{2}-9}{8}\cdot 2=10,75.
		$$
	}
\end{ex}
%10
\begin{ex}%[1K3B9-3]
	Lương tháng của một số nhân viên văn phòng được ghi lại như sau (đơn vị: triệu đồng)
	\begin{center}
		\begin{tabular}{|l|c|c|c|c|c|}
			\hline Lương tháng (triệu đồng)  &{$[6;8)$}&{$[8;10)$}&{$[10;12)$}&{$[12;14)$}\\
			\hline Số nhân viên & $3$ & $6$ & $8$ & $7$  \\
			\hline
		\end{tabular}     
	\end{center}  
	Tìm   tứ phân vị thứ ba của mẫu số liệu.
	\choice 
	{$Q_3\approx 12,5$}
	{$Q_3\approx 13,2$}
	{$Q_3\approx 13,5$}
	{\True $Q_3\approx 12,3$}
	\loigiai{
		Cỡ mẫu là $n=24$.\\
		Gọi $x_1, \ldots, x_{24}$ là mẫu số liệu được sắp xếp theo thứ tự không giảm. Khi đó, trung vị là $\dfrac{x_{12}+x_{13}}{2}$. 
		Do đó,  tứ phân vị thứ ba $Q_3$ là $\dfrac{x_{18}+x_{19}}{2}$. Do $x_{18}$, $x_{19}$ đều thuộc nhóm $[12;14)$ nên nhóm này chứa $Q_3$. \\Do đó, $p=4 ; \;a_4=12;\; m_4=7 ; \;m_1+m_2+m_3=17, \;a_4-a_3=2$ và ta có
		$$
		Q_2=12+\dfrac{\frac{24\cdot 3}{4}-17}{7}\cdot 2\approx 12,3.
		$$
	}
\end{ex}
%11
\begin{ex}%[1K3B9-3]
	Số điểm một cầu thủ bóng rổ ghi được trong 20 trận đấu được cho ở bảng sau
	\begin{center}
		\begin{tabular}{|l|c|c|c|c|c|}
			\hline Điểm số  &{$[5,5;10,5)$}&{$[10,5;15,5)$}&{$[15,5;20,5)$}&{$[20,5;25,5)$}\\
			\hline Số trận & $3$ & $9$ & $2$ & $6$  \\
			\hline
		\end{tabular}     
	\end{center}  
	Tìm   tứ phân vị thứ ba của mẫu số liệu.
	\choice 
	{$Q_3\approx 23,5$}
	{$Q_3\approx 22,2$}
	{$Q_3\approx 21,6$}
	{\True $Q_3\approx 21,3$}
	\loigiai{
		Cỡ mẫu là $n=20$.\\
		Gọi $x_1, \ldots, x_{20}$ là mẫu số liệu được sắp xếp theo thứ tự không giảm. Khi đó, trung vị là $\dfrac{x_{10}+x_{11}}{2}$. 
		Do đó,  tứ phân vị thứ ba $Q_3$ là $\dfrac{x_{15}+x_{16}}{2}$. Do $x_{15}$, $x_{16}$ đều thuộc nhóm $[20,5;25,5)$ nên nhóm này chứa $Q_3$. \\Do đó, $p=4 ; \;a_4=20,5;\; m_4=6 ; \;m_1+m_2+m_3=14, \;a_5-a_4=25,5-20,5=5$ và ta có
		$$
		Q_2=20,5+\dfrac{\frac{20\cdot 3}{4}-14}{6}\cdot 5 \approx 21,3.
		$$
	}
\end{ex}
%12
\begin{ex}%[1K3B9-3]
	Số điểm một cầu thủ bóng rổ ghi được trong 20 trận đấu được cho ở bảng sau
	\begin{center}
		\begin{tabular}{|l|c|c|c|c|c|}
			\hline Điểm số  &{$[5,5;10,5)$}&{$[10,5;15,5)$}&{$[15,5;20,5)$}&{$[20,5;25,5)$}\\
			\hline Số trận & $3$ & $9$ & $2$ & $6$  \\
			\hline
		\end{tabular}     
	\end{center} 
	Tìm tứ phân vị thứ nhất của mẫu số liệu.
	\choice
	{\True $Q_1\approx 11,6$}
	{$Q_1\approx 11,3$}
	{$Q_1\approx 21,6$}
	{$Q_1\approx 21,3$}
	\loigiai{
		Cỡ mẫu là $n=20$.\\
		Gọi $x_1, \ldots, x_{20}$ là mẫu số liệu được sắp xếp theo thứ tự không giảm. Khi đó, trung vị là $\dfrac{x_{10}+x_{11}}{2}$. 
		Do đó, tứ phân vị thứ nhất $Q_1$ là $\dfrac{x_{5}+x_{6}}{2}$. Do $x_{5}$, $x_{6}$ đều thuộc nhóm $[10,5;15,5)$ nên nhóm này chứa $Q_1$. \\Do đó, $p=2 ; \;a_2=10,5;\; m_2=9 ; \;m_1=3, \;a_3-a_2=5$ và ta có
		$$
		Q_1=10,5+\dfrac{\frac{20}{4}-3}{9}\cdot 5\approx 11,6.
		$$
	}
\end{ex}
%13
\begin{ex}%[1K3B9-3]
	Số điểm một cầu thủ bóng rổ ghi được trong 20 trận đấu được cho ở bảng sau
	\begin{center}
		\begin{tabular}{|l|c|c|c|c|c|}
			\hline Điểm số  &{$[5,5;10,5)$}&{$[10,5;15,5)$}&{$[15,5;20,5)$}&{$[20,5;25,5)$}\\
			\hline Số trận & $3$ & $9$ & $2$ & $6$  \\
			\hline
		\end{tabular}     
	\end{center} 
	Tìm tứ phân vị thứ nhất của mẫu số liệu.
	\choice
	{\True $Q_1\approx 11,6$}
	{$Q_1\approx 14,4$}
	{$Q_1\approx 15,6$}
	{$Q_1\approx 21,3$}
	\loigiai{
		Cỡ mẫu là $n=20$.\\
		Gọi $x_1, \ldots, x_{20}$ là mẫu số liệu được sắp xếp theo thứ tự không giảm. Khi đó, trung vị là $\dfrac{x_{10}+x_{11}}{2}$. 
		Do đó, tứ phân vị thứ nhất $Q_2$ là $\dfrac{x_{10}+x_{11}}{2}$. Do $x_{10}$, $x_{11}$ đều thuộc nhóm $[10,5;15,5)$ nên nhóm này chứa $Q_2$. \\Do đó, $p=2 ; \;a_2=10,5;\; m_2=9 ; \;m_1=3, \;a_3-a_2=5$ và ta có
		$$
		Q_1=10,5+\dfrac{\frac{20}{2}-3}{9}\cdot 5\approx 14,4.
		$$   
	}
\end{ex}
%14
\begin{ex}%[1K3B9-3]
	Một người thống kê lại thời gian thực hiện các cuộc gọi điện thoại của người đó trong một tuần cho trong bảng sau
	\begin{center}
		\begin{tabular}{|l|c|c|c|c|c|}
			\hline Số bệnh nhân &{$[0;60)$}&{$[60;120)$}&{$[120;180)$}&{$[180;240)$}&{$[240;300)$}\\
			\hline Số ngày & $8$ & $10$ & $7$ & $5$ & $2$ \\
			\hline
		\end{tabular}     
	\end{center}
	Tìm   tứ phân vị thứ ba của mẫu số liệu.  
	\choice
	{$Q_3\approx 175,28$}
	{$Q_3\approx 150,32 $}
	{$Q_3=175$}
	{\True $Q_3\approx 171,43$}
	\loigiai{ Cỡ mẫu là $n=32$.\\
		Gọi $x_1, \ldots, x_{32}$ là mẫu số liệu được sắp xếp theo thứ tự không giảm. Khi đó, trung vị là $\dfrac{x_{16}+x_{17}}{2}$.\\
		Do đó, tứ phân vị thứ ba $Q_3$ là $\dfrac{x_{24}+x_{25}}{2}$. Do $x_{24} $, $x_{25} \in [120;180)$   nên nhóm này chứa $Q_3$. \\Do đó, $p= 3; \;a_3=120 ;\; m_3=7 ; \;m_1+m_2=18 ; \;a_4-a_3=60$ và ta có
		$$
		Q_3=120+\dfrac{\frac{3 \cdot 32}{4}-18}{7}\cdot 60\approx 171,43.
		$$
	}
\end{ex}
%15
\begin{ex}%[1K3B9-3]
	Một người thống kê lại thời gian thực hiện các cuộc gọi điện thoại của người đó trong một tuần cho trong bảng sau
	\begin{center}
		\begin{tabular}{|l|c|c|c|c|c|}
			\hline Số bệnh nhân &{$[0;60)$}&{$[60;120)$}&{$[120;180)$}&{$[180;240)$}&{$[240;300)$}\\
			\hline Số ngày & $8$ & $10$ & $7$ & $5$ & $2$ \\
			\hline
		\end{tabular}     
	\end{center}
	Tìm   tứ phân vị thứ hai của mẫu số liệu.  
	\choice
	{$Q_2\approx 80,25$}
	{$Q_2\approx 100,32$}
	{$Q_2=115$}
	{\True $Q_2=108$}
	\loigiai{ Cỡ mẫu là $n=32$.\\
		Gọi $x_1, \ldots, x_{32}$ là mẫu số liệu được sắp xếp theo thứ tự không giảm. Khi đó, trung vị là $\dfrac{x_{16}+x_{17}}{2}$.\\
		Do đó, tứ phân vị thứ hai $Q_2$ là $\dfrac{x_{16}+x_{17}}{2}$. Do $x_{16} $, $x_{17} \in [60;120)$   nên nhóm này chứa $Q_2$. \\Do đó, $p= 2; \;a_2=60 ;\; m_2=10 ; \;m_1=8 ; \;a_3-a_2=60$ và ta có
		$$
		Q_3=60+\dfrac{\frac{ 32}{2}-8}{10}\cdot 60=108.
		$$
	}    
\end{ex}
%%%%%%%%%%%%%%%%%%%%%%%%%%%%%%%%%%%%%%%%%%%%%%%%%%%%%%%%%%%%%%%%%%%%%%%%%%%%%%%%%%%%%%%%%%%%%%%%%%%%%%%%%%%%%%%%%%%%%%%%%%%%%%%%%%%%%%%%%%%%%%%%%%%%%%%%%%%
\begin{dang}{ Mốt}
	Để tìm mốt của mẫu số liệu ghép nhóm, ta thực hiện theo các bước sau:
	\begin{itemize}
		\item Bước 1. Xác định nhóm có tần số lớn nhất (gọi là nhóm chứa mốt), giả sử là nhóm $j$: $\left[a_j;a_{j+1}\right)$.
		\item Bước 2. Mốt được xác định là
		$$M_0=a_j+\frac{m_j-m_{j-1}}{\left(m_j-m_{j-1}\right)+\left(m_j-m_{j+1}\right)}\cdot h$$ trong đó $m_j$ là tần số của nhóm $j$ (quy ước $m_0=m_{k+1}=0$ ) và $h$ là độ dài của nhóm.
	\end{itemize}
\end{dang}
\subsubsection{Ví dụ minh hoạ}
\begin{vd}%[1K3B9-4] 
	Bảng số liệu ghép nhóm sau cho biết chiều cao $(\mathrm{cm})$ của 50 học sinh lớp $11 \mathrm{A}$.
	\begin{center}
		\begin{tabular}{|l|c|c|c|c|c|}
			\hline Khoảng chiều cao $(\mathrm{cm})$ &{$[145 ; 150)$}&{$[150 ; 155)$}&{$[155 ; 160)$}&{$[160 ; 165)$}&{$[165 ; 170)$}\\
			\hline Só học sinh & 7 & 14 & 10 & 10 & 9 \\
			\hline
		\end{tabular}      
	\end{center}
	Tính mốt của mẵu số liệu ghép nhóm này. Có thể kết luận gì từ giá trị tính được?
	\loigiai{
		Tần số lớn nhất là $14$ nên nhóm chứa mốt là nhóm $[150; 155)$. \\
		Ta có, $j=2, a_2=150, m_2=14$, $m_1=7, m_3=10, h=5$. Do đó
		$$
		M_0=150+\frac{14-7}{(14-7)+(14-10)}\cdot 5 \approx 153,18.
		$$
		Số học sinh có chiều cao khoảng $153,18 \mathrm{cm}$ là nhiều nhất.
	}
\end{vd}

\begin{vd}%2%[1K3B9-4]
	Tuồi thọ (năm) của 50 bình ắc quy ô tô được cho như sau:
	\begin{center}
		\begin{tabular}{|l|c|c|c|c|c|c|}
			\hline Tuồi thọ (năm) &{$[2 ; 2,5)$}&{$[2,5 ; 3)$}&{$[3 ; 3,5)$}&{$[3,5 ; 4)$}&{$[4 ; 4,5)$}&{$[4,5 ; 5)$}\\
			\hline Tằn số & $4$ & $9$ & $14$ & $11$ & $7$ & $5$ \\
			\hline
		\end{tabular}
	\end{center}
	Xác định mốt và giải thích ý nghĩa.
	\loigiai{
		Tần số lớn nhất là $14$ nên nhóm chứa mốt là nhóm $[3 ; 3,5)$. \\
		Ta có $j=3, a_3=3, m_3=14$, $m_2=9, m_4=11, h=0,5$. Do đó
		$$
		M_0=3+\frac{14-9}{(14-9)+(14-11)}\cdot 0,5 = 3,3125.
		$$
		Bình ắc quy ô tô có tuổi thọ $3,3125$ năm là nhiều nhất. 
	}
\end{vd}

\begin{vd}%3%[1K3B9-4]
	Một thư viện thống kê số lượng sách được mượn mỗi ngày trong ba tháng ở bảng sau
	\begin{center}
		\begin{tabular}{|l|c|c|c|c|c|c|c|}
			\hline Số sách &{$[16;20]$}&{$[21;25]$}&{$[26;30]$}&{$[31;35]$}&{$[36;40]$}&{$[41;45]$}\\
			\hline Số ngày &$3$ & $6$ & $15$ & $27$ & $22$ & $14$ \\
			\hline
		\end{tabular}
	\end{center}
	Xác định mốt của mẫu số liệu ghép nhóm này.
	\loigiai{Ta hiệu chỉnh bảng số liệu như sau
		\begin{center}
			\begin{tabular}{|l|c|c|c|c|c|c|c|}
				\hline Số sách &{$[15,5;20,5)$}&{$[20,5;25,5)$}&{$[25,5;30,5)$}&{$[30,5;35,5)$}&{$[35,5;40,5)$}&{$[40,5;45,5)$}\\
				\hline Số ngày &$3$ & $6$ & $15$ & $27$ & $22$ & $14$ \\
				\hline
			\end{tabular}
		\end{center} 
		Tần số lớn nhất là $27$ nên nhóm chứa mốt là nhóm $[30,5;35,5)$. \\
		Ta có $j=4, a_4=31, m_4=27$, $m_3=15, m_5=22, h=5$. Do đó
		$$
		M_0=30,5+\frac{27-15}{(27-15)+(27-22)}\cdot 5 \approx 34,03.
		$$
	}
\end{vd}
\begin{vd}%9%[1K3B9-4]
	Thời gian (phút) để học sinh hoàn thành một câu hỏi thi được cho như sau:
	\begin{center}
		\begin{tabular}{|l|c|c|c|c|c|c|}
			\hline Thời gian (phút) &{$[0,5 ; 10,5)$}&{$[10,5 ; 20,5)$}&{$[20,5 ; 30,5)$}&{$[30,5 ; 40,5)$}&{$[40,5 ; 50,5)$}\\
			\hline Số học sinh & $2$ & $10$ & $6$ & $4$ & $3$ \\
			\hline
		\end{tabular}
	\end{center}
	Xác định mốt của mẫu số liệu ghép nhóm này.
	\loigiai{
		Tần số lớn nhất là $10$ nên nhóm chứa mốt là nhóm $[30,5;35,5)$. \\
		Ta có $j=2; a_2=10,5; m_2=10$, $m_3=6, m_1=2, h=10$. Do đó
		$$
		M_0=10,5+\frac{10-2}{(10-2)+(10-6)}\cdot 10 \approx 17,167.
		$$ 
	}
\end{vd}
\begin{vd}%[1K3B9-4]
	Điểm kiểm tra môn Toán của lớp 12A được cho trong bảng sau
	\begin{center}
		\begin{tabular}{|l|c|c|c|c|c|c|}
			\hline Thời gian (phút) &{$[3;5)$}&{$[5;7)$}&{$[7;9)$}&{$[9;11)$}\\
			\hline Số học sinh & $5$ & $18$ & $10$ & $7$  \\
			\hline
		\end{tabular}
	\end{center}
	Xác định mốt của mẫu số liệu ghép nhóm này.
	\loigiai{
		Tần số lớn nhất là $18$ nên nhóm chứa mốt là nhóm $[5;7)$. \\
		Ta có $j=2; a_2=5; m_2=18$, $m_3=10, m_1=5, h=2$. Do đó
		$$
		M_0=5+\frac{18-5}{(18-5)+(18-10)}\cdot 2 \approx 6,24.
		$$
	}
\end{vd}

%%%%%%%%%%%%%%%%%%%%%%%%%%%%%%%%%%%%%%%%%%
\subsubsection{Bài tập rèn luyện}
%5
\begin{ex}%[1K3Y9-4]
	Chọn khẳng định \textbf{sai}.
	\choice
	{ Mốt của mẫu số liệu không ghép nhóm là giá trị có khả năng xuất hiện cao nhất khi lấy mẫu}
	{Mốt của mẫu số liệu sau khi ghép nhóm xấp xỉ với mốt của mẫu số liệu không ghép nhóm}
	{\True Một mẫu số liệu ghép nhóm chỉ có một mốt}
	{Một mẫu số liệu ghép nhóm có thể có nhiều nhóm chứa mốt và nhiều mốt}
	\loigiai{
		Mốt của mẫu số liệu không ghép nhóm là giá trị có khả năng xuất hiện cao nhất khi lấy mẫu.\\ Mốt của mẫu số liệu sau khi ghép nhóm xấp xỉ với mốt của mẫu số liệu không ghép nhóm. \\
		Một mẫu số liệu ghép nhóm có thể có nhiều nhóm chứa mốt và nhiều mốt.\\
		Do đó khẳng định sai là: Một mẫu số liệu ghép nhóm chỉ có một mốt.
	}    
\end{ex}
\begin{ex}%[1K3Y9-4]
	Người ta ghi lại tuổi thọ của một số con ong cho kết quả như sau:
	\begin{center}
		\begin{tabular}{|l|c|c|c|c|c|c|}
			\hline Tuồi thọ (ngày) &{$[0;20)$}&{$[20;40)$}&{$[40;60)$}&{$[60;80)$}&{$[80;100)$}\\
			\hline Số lượng & $5$ & $12$ & $23$ & $31$ & $29$  \\
			\hline
		\end{tabular}
	\end{center}
	Nhóm chứa mốt của mẫu số liệu này là
	\choice
	{ $[20;40)$}
	{$[40;60)$}
	{\True $[60;80)$}
	{$[80;100)$}
	\loigiai{
		Nhóm chứa mốt của mẫu số liệu này là $[60;80)$.
	}    
\end{ex}
%6
\begin{ex}%[1K3B9-4]
	Người ta ghi lại tuổi thọ của một số con ong cho kết quả như sau:
	\begin{center}
		\begin{tabular}{|l|c|c|c|c|c|c|}
			\hline Tuồi thọ (ngày) &{$[0;20)$}&{$[20;40)$}&{$[40;60)$}&{$[60;80)$}&{$[80;100)$}\\
			\hline Số lượng & $5$ & $12$ & $23$ & $31$ & $29$  \\
			\hline
		\end{tabular}
	\end{center}
	Mốt của mẫu số liệu này là
	\choice
	{ $M_0=70$}
	{$M_0=60$}
	{\True $M_0=76$}
	{$M_0=31$}
	\loigiai{
		Tần số lớn nhất là $31$ nên nhóm chứa mốt là nhóm $[60;80)$. \\
		Ta có, $j=4, a_4=60, m_4=31$, $m_3=23, m_5=29, h=20$. Do đó
		$$
		M_0=60+\frac{31-23}{(31-29)+(14-11)}\cdot 20 =76.
		$$
	}   
\end{ex}
%7
\begin{ex}%[1K3Y9-4]
	Doanh thu bán hàng 20 ngày được lựa chọn ngẫu nhiên của một cửa hàng được ghi lại ở bảng sau (đơn vị: triệu đồng)
	\begin{center}
		\begin{tabular}{|l|c|c|c|c|c|c|}
			\hline Doanh thu &{$[5;7)$}&{$[7;9)$}&{$[9;11)$}&{$[11;13)$}&{$[13;15)$}\\
			\hline Số ngày & $2$ & $9$ & $7$ & $3$ & $1$ \\
			\hline
		\end{tabular}
	\end{center}
	Nhóm chứa mốt của mẫu số liệu này là
	\choice
	{ $[9;11)$}
	{$[5;7)$}
	{\True $[7;9)$}
	{$[11;13)$}
	\loigiai{
		Tần số lớn nhất là $9$ nên nhóm chứa mốt là nhóm $[7;9)$. \\
	}   
\end{ex}
%8
\begin{ex}%[1K3B9-4]
	Doanh thu bán hàng 20 ngày được lựa chọn ngẫu nhiên của một cửa hàng được ghi lại ở bảng sau (đơn vị: triệu đồng)
	\begin{center}
		\begin{tabular}{|l|c|c|c|c|c|c|}
			\hline Doanh thu &{$[5;7)$}&{$[7;9)$}&{$[9;11)$}&{$[11;13)$}&{$[13;15)$}\\
			\hline Số ngày & $2$ & $9$ & $7$ & $3$ & $1$ \\
			\hline
		\end{tabular}
	\end{center}
	Xác định mốt của mẫu số liệu.
	\choice
	{ $M_0\approx 8$}
	{$M_0\approx 8,5$}
	{\True $M_0\approx 8,56$}
	{$M_0\approx 9$}
	\loigiai{
		Tần số lớn nhất là $9$ nên nhóm chứa mốt là nhóm $[7;9)$. \\
		Ta có, $j=2, a_2=7, m_2=9$, $m_1=2, m_3=7, h=2$. Do đó
		$$
		M_0=7+\frac{9-2}{(9-2)+(9-7)}\cdot 2\approx 8,56.
		$$
	}    
\end{ex}
%9
\begin{ex}%[1K3B9-4]
	Điểm kiểm tra môn Toán của lớp 12A được cho trong bảng sau
	\begin{center}
		\begin{tabular}{|l|c|c|c|c|c|c|c|c|}
			\hline Khoảng điểm &{$[6,5;7)$}&{$[7;7,5)$}&{$[7,5;8)$}&{$[8;8,5)$}&{$[8,5;9)$}&{$[9;9,5)$}&{$[9,5;10)$}\\
			\hline Tần số & $8$ & $10$ & $16$ & $24$& $13$ & $7$ & $4$ \\
			\hline
		\end{tabular}
	\end{center}
	Xác định mốt của mẫu số liệu ghép nhóm này.    
	\choice
	{ $M_0\approx 8,4$}
	{$M_0\approx 8,5$}
	{\True $M_0\approx 8,21$}
	{$M_0\approx 24$}
	\loigiai{
		Tần số lớn nhất là $24$ nên nhóm chứa mốt là nhóm $[8;8,5)$. \\
		Ta có, $j=4, a_4=8, m_4=24$, $m_3=16, m_5=13, h=0,5$. Do đó
		$$
		M_0=8+\frac{24-16}{(24-16)+(24-13)}\cdot 0,5\approx 8,21.
		$$
	}
\end{ex}
%10
\begin{ex}%[1K3Y9-4]
	Điểm kiểm tra môn Toán của lớp 12A được cho trong bảng sau
	\begin{center}
		\begin{tabular}{|l|c|c|c|c|c|c|c|c|}
			\hline Khoảng điểm &{$[6,5;7)$}&{$[7;7,5)$}&{$[7,5;8)$}&{$[8;8,5)$}&{$[8,5;9)$}&{$[9;9,5)$}&{$[9,5;10)$}\\
			\hline Tần số & $8$ & $10$ & $16$ & $24$& $13$ & $7$ & $4$ \\
			\hline
		\end{tabular}
	\end{center}
	Nhóm chứa mốt của mẫu số liệu này là
	\choice
	{ $[7;7,5)$}
	{$[7,5;8)$}
	{\True $[8;8,5)$}
	{$[8,5;9)$}
	\loigiai{
		Tần số lớn nhất là $24$ nên nhóm chứa mốt là nhóm $[8,5;9)$. \\
	}       
\end{ex}
%11
\begin{ex}%[1K3Y9-4]
	Để kiểm tra thời gian sử dụng pin của   chiếc điện thoại mới, bạn A thống kê thời gian sử dụng điện thoại của mình từ lúc sạc đầy cho đến khi hết pin ở bảng sau
	\begin{center}
		\begin{tabular}{|l|c|c|c|c|c|c|c|c|}
			\hline Thời gian sử dụng (giờ) &{$[7;9)$}&{$[9;11)$}&{$[11;13)$}&{$[13;15)$}&{$[15;17)$}\\
			\hline Số lần & $2$ & $5$ & $7$ & $6$& $3$  \\
			\hline
		\end{tabular}
	\end{center}
	Nhóm chứa mốt của mẫu số liệu này là
	\choice
	{ $[9;11)$}
	{\True $[11;13)$}
	{ $[13;15)$}
	{$[15;17)$}
	\loigiai{
		Tần số lớn nhất là $7$ nên nhóm chứa mốt là nhóm $[11;13)$. \\
	}        
\end{ex}
%12
\begin{ex}%[1K3B9-4]
	Để kiểm tra thời gian sử dụng pin của   chiếc điện thoại mới, bạn A thống kê thời gian sử dụng điện thoại của mình từ lúc sạc đầy cho đến khi hết pin ở bảng sau
	\begin{center}
		\begin{tabular}{|l|c|c|c|c|c|c|c|c|}
			\hline Thời gian sử dụng (giờ) &{$[7;9)$}&{$[9;11)$}&{$[11;13)$}&{$[13;15)$}&{$[15;17)$}\\
			\hline Số lần & $2$ & $5$ & $7$ & $6$& $3$  \\
			\hline
		\end{tabular}
	\end{center}   
	Xác định mốt của mẫu số liệu ghép nhóm này.    
	\choice
	{ $M_0\approx 11,67$}
	{$M_0\approx 12$}
	{\True $M_0\approx 12,33$}
	{$M_0\approx 7$}
	\loigiai{
		Tần số lớn nhất là $7$ nên nhóm chứa mốt là nhóm $[11;13)$. \\
		Ta có, $j=3, a_3=11, m_3=7$, $m_2=5, m_4=6, h=2$. Do đó
		$$
		M_0=11+\frac{7-5}{(7-5)+(7-6)}\cdot 2\approx 12,33.
		$$
	}
\end{ex}
%13
\begin{ex}%[1K3Y9-4]
	Tổng lượng mưa trong tháng 8 đo được tại một trạm quan trắc đặt tại Vũng Tàu từ năm 2002 đến năm 2020 được ghi lại như sau (đơn vị: mm)
	\begin{center}
		\begin{tabular}{|l|c|c|c|c|c|c|c|c|}
			\hline Tổng lượng mưa trong tháng 8 (mm) &{$[120;175)$}&{$[175;230)$}&{$[230;285)$}&{$[285;340)$}\\
			\hline Số năm & $10$ & $5$ & $3$ & $1$  \\
			\hline
		\end{tabular}
	\end{center}  
	Nhóm chứa mốt của mẫu số liệu này là
	\choice
	{ $[175;230)$}
	{ $[230;285)$}
	{\True $[120;175)$}
	{$[285;340)$}
	\loigiai{
		Tần số lớn nhất là $10$ nên nhóm chứa mốt là nhóm $[120;175)$. \\
	}        
\end{ex}
%14
\begin{ex}%[1K3B9-4]
	Tổng lượng mưa trong tháng 8 đo được tại một trạm quan trắc đặt tại Vũng Tàu từ năm 2002 đến năm 2020 được ghi lại như sau (đơn vị: mm)
	\begin{center}
		\begin{tabular}{|l|c|c|c|c|c|c|c|c|}
			\hline Tổng lượng mưa trong tháng 8 (mm) &{$[120;175)$}&{$[175;230)$}&{$[230;285)$}&{$[285;340)$}\\
			\hline Số năm & $10$ & $5$ & $3$ & $1$  \\
			\hline
		\end{tabular}
	\end{center}  
	Xác định mốt của mẫu số liệu ghép nhóm này.    
	\choice
	{ $M_0\approx 172,25$}
	{$M_0\approx 146,125$}
	{\True $M_0\approx 156,67$}
	{$M_0\approx 10$}
	\loigiai{
		Tần số lớn nhất là $10$ nên nhóm chứa mốt là nhóm $[120;175)$. \\
		Ta có, $j=1, a_1=120, m_1=10$, $m_2=5, m_0=0, h=55$. Do đó
		$$
		M_0=120+\frac{10-0}{(10-0)+(10-5)}\cdot 55\approx 156.67.
		$$
	}
\end{ex}
%15
\begin{ex}%[1K3B9-4]
	Một công ty xây dựng khảo sát khách hàng xem họ có nhu cầu mua nhà ở mức giá nào. Kết quả khảo sát được ghi lại ở bảng sau (đơn vị: triệu đồng/$\mathrm{m}^2$
	\begin{center}
		\begin{tabular}{|l|c|c|c|c|c|c|c|c|}
			\hline Mức giá &{$[10;14)$}&{$[14;18)$}&{$[18;22)$}&{$[22;26)$}&{$[26;30)$}\\
			\hline Số khách hàng & $54$ & $78$ & $120$ & $45$& $12$  \\
			\hline
		\end{tabular}
	\end{center}   
	Xác định mốt của mẫu số liệu ghép nhóm này.    
	\choice
	{ $M_0\approx 18$}
	{\True $M_0\approx 19,4$}
	{ $M_0\approx 20$}
	{$M_0\approx 120$}
	\loigiai{
		Tần số lớn nhất là $120$ nên nhóm chứa mốt là nhóm $[18;22)$. \\
		Ta có, $j=3, a_3=18, m_3=120$, $m_2=78, m_4=45, h=4$. Do đó
		$$
		M_0=18+\frac{120-78}{(120-78)+(120-45)}\cdot 4\approx 19,4.
		$$
	}
\end{ex}
\begin{ex}%[1K3K9-4]
	Một công ty xây dựng khảo sát khách hàng xem họ có nhu cầu mua nhà ở mức giá nào. Kết quả khảo sát được ghi lại ở bảng sau (đơn vị: triệu đồng/$\mathrm{m}^2$
	\begin{center}
		\begin{tabular}{|l|c|c|c|c|c|c|c|c|}
			\hline Mức giá &{$[10;14)$}&{$[14;18)$}&{$[18;22)$}&{$[22;26)$}&{$[26;30)$}\\
			\hline Số khách hàng & $54$ & $78$ & $120$ & $45$& $12$  \\
			\hline
		\end{tabular}
	\end{center}   
	Công ty nên xây nhà ở mức giá nào để nhiều người có nhu cầu mua nhất?    
	\choice
	{ $ 18$ triệu đồng/$\mathrm{m}^2$}
	{\True $19,4$ triệu đồng/$\mathrm{m}^2$}
	{ $20$ triệu đồng/$\mathrm{m}^2$}
	{$21$ triệu đồng/$\mathrm{m}^2$}
	\loigiai{
		Tần số lớn nhất là $120$ nên nhóm chứa mốt là nhóm $[18;22)$. \\
		Ta có, $j=3, a_3=18, m_3=120$, $m_2=78, m_4=45, h=4$. Do đó
		$$
		M_0=18+\frac{120-78}{(120-78)+(120-45)}\cdot 4\approx 19,4.
		$$
		Dựa vào kết quả trên ta dự đoán rằng nếu công ty xây nhà ở mức giá $19,4$ triệu đồng/$\mathrm{m}^2$ thì sẽ có nhiều người có nhu cầu mua nhất.
	}
\end{ex}
\begin{ex}%[1K3Y9-4]
	Số cuộc gọi điện thoại một người thực hiện mỗi ngày trong 30 ngày được lựa chọn ngẫu nhiên được thống kê trong bảng sau
	\begin{center}
		\begin{tabular}{|l|c|c|c|c|c|c|c|c|}
			\hline Số cuộc gọi &{$[2,5;5,5)$}&{$[5,5;8,5)$}&{$[8,5;11,5)$}&{$[11,5;14,5)$}&{$[14,5;17,5)$}\\
			\hline Số ngày & $5$ & $13$ & $7$ & $3$& $2$  \\
			\hline
		\end{tabular}
	\end{center}   
	Nhóm chứa mốt của mẫu số liệu này là
	\choice
	{ $[2,5;5,5)$}
	{\True $[5,5;8,5)$}
	{ $[8,5;11,5)$}
	{$[11,5;14,5)$}
	\loigiai{
		Tần số lớn nhất là $13$ nên nhóm chứa mốt là nhóm $[5,5;8,5)$. \\
	}
\end{ex}
\begin{ex}%[1K3K9-4]
	Số cuộc gọi điện thoại một người thực hiện mỗi ngày trong 30 ngày được lựa chọn ngẫu nhiên được thống kê trong bảng sau
	\begin{center}
		\begin{tabular}{|l|c|c|c|c|c|c|c|c|}
			\hline Số cuộc gọi &{$[2,5;5,5)$}&{$[5,5;8,5)$}&{$[8,5;11,5)$}&{$[11,5;14,5)$}&{$[14,5;17,5)$}\\
			\hline Số ngày & $5$ & $13$ & $7$ & $3$& $2$  \\
			\hline
		\end{tabular}
	\end{center}   
	Hãy dự đoán xem khả năng người đó thực hiện bao nhiêu cuộc gọi mỗi ngày là cao nhất?   
	\choice
	{ $5$}
	{\True $7$}
	{ $6$}
	{$8$}
	\loigiai{
		Tần số lớn nhất là $13$ nên nhóm chứa mốt là nhóm $[5,5;8,5)$. \\
		Ta có, $j=2, a_2=5,5, m_2=13$, $m_1=5, m_3=7, h=3$. Do đó
		$$
		M_0=5,5+\frac{13-5}{(13-5)+(13-7)}\cdot 3\approx 7,2.
		$$
		Do đó ta có thể dự đoán khả năng người đó thực hiện $7$ cuộc gọi mỗi ngày là cao nhất.
	}
\end{ex}