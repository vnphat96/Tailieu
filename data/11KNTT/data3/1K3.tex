\setcounter{chapter}{2}
\setcounter{subsubsection}{0}
\setcounter{ex}{0}
\setcounter{bt}{0}
\chap{Một số yếu tố thống kê và xác suất}
% \section{Các số đặc trưng đo xu thế trung tâm cho mẫu số liệu ghép nhóm}

\subsection{Tóm tắt lý thuyết}
\begin{tomtat}
	\subsubsection{Mẫu số liệu ghép nhóm}
		\begin{enumerate}
			\item \textbf{\textit{Mẫu số liệu ghép nhóm}} là mẫu số liệu cho dưới dạng bảng tần số ghép nhóm.
			\item Mỗi số liệu gồm một số giá trị của mẫu số liệu được ghép nhóm theo một tiêu chí xác định có dạng $\left[a;b\right)$, trong đó $a$ là \textit{đầu mút trái}, $b$ là \textit{đầu mút phải}. Độ dài nhóm là $b-a$.
			\item \textbf{\textit{Tần số tích luỹ}} của một nhóm là số số liệu trong mẫu số liệu có giá trị nhỏ hơn giá trị đầu mút phải của nhóm đó. Tần số tích luỹ của nhóm $1$, nhóm $2$, $\ldots$, nhóm $m$ kí hiệu lần lượt là $cf_1$, $cf_2$, $\ldots$, $cf_m$.
		\end{enumerate}
	\subsubsection{Số trung bình cộng (số trung bình)}
		\begin{enumerate}
			\item Trung điểm $x_i$ của nửa khoảng (tính bằng trung bình cộng của hai đầu mút) ứng với nhóm $i$ là \textit{giá trị đại diện} của nhóm đó.
			\item \textit{Số trung bình cộng} của mẫu số liệu ghép nhóm, kí hiệu $\overline{x}$, được tính theo công thức 
				$$\overline{x} = \dfrac{n_1x_1 + n_2x_2 + \ldots + n_mx_m}{n}.$$
		\end{enumerate}
	\subsubsection{Trung vị}
	Để tính trung vị của mẫu số liệu ghép nhóm, ta làm như sau:
\begin{itemize}
    \item \textbf{Bước 1:} Xác định nhóm chứa trung vị. Giả sử đó là nhóm thứ $p : [a_p; a_{p + 1})$.
    \item \textbf{Bước 2:} Trung vị là 
    $$M_e = a_p + \dfrac{\dfrac{n}{2} - (m_1 + \cdots + m_{p-1})}{m_p} \cdot ( a_{p + 1} - a_p)$$
    trong đó $n$ là cỡ mẫu, $m_p$ là tần số nhóm $p$. Với $p=1$ ta quy ước $m_1 + \cdots + m_{p-1} = 0$.
\end{itemize}
		\begin{note}
			Nhóm chứa trung vị là nhóm đầu tiên có tần số tích luỹ $cf_p=m_1 + \cdots + m_{p}$ lớn hơn hoặc bằng $\dfrac{n}{2}$
		\end{note}
	\subsubsection{Tứ phân vị}Để tính tứ phân vị thứ nhất $Q_1$ của mẫu số liệu ghép nhóm, trước hết ta xác định nhóm chứa $Q_1$, giả sử đó là nhóm thứ  $p:\left[a_p;a_{p+1} \right)$. Khi đó 
	$$Q_1=a_p+\dfrac{\dfrac{n}{4}-\left(m_1+\cdots+m_{p-1}\right)}{m_p}\cdot \left(a_{p+1}-a_p\right).$$
	trong đó, $n$ là cỡ mẫu, $m_p$ là tần số nhóm $p$. Với $p=1$, ta quy ước $m_1+\cdots+m_{p-1}=0$.\\
	Để tính tứ phân vị thứ ba $Q_3$ của mẫu số liệu ghép nhóm, trước hết ta xác định nhóm chứa $Q_3$, giả sử đó là nhóm thứ  $p:\left[a_p;a_{p+1} \right)$. Khi đó 
	$$Q_3=a_p+\dfrac{\dfrac{3n}{4}-\left(m_1+\cdots+m_{p-1}\right)}{m_p}\cdot \left(a_{p+1}-a_p\right).$$
	Trong đó $n$ là cỡ mẫu, $m_p$ là tần số nhóm $p$. Với $p=1$, ta quy ước $m_1+\cdots+m_{p-1}=0$.\\
	Tứ phân vị thứ hai $Q_2$ chính là trung vị $M_e$.\\
\subsubsection{Mốt của mẫu số liệu ghép nhóm}
Để tìm mốt của mẫu số liệu ghép nhóm, ta thực hiện theo các bước sau:
\begin{enumerate}
	\item [Bước 1.] Xác định nhóm có tần số lớn nhất (gọi là nhóm chứa mốt), giả sử là nhóm $j:\left[a_j;a_{j+1} \right)$.
	\item [Bước 2.] Mốt được xác định là $M_o=a_j+\dfrac{m_i-m_{j-1}}{\left(m_i-m_{j-1}\right)+\left(m_i-m_{j+1}\right)}\cdot h$.\\
\end{enumerate}
trong đó, $m_j$ là tần số nhóm $j$ (quy ước $m_0=m_{k+1}=0$) và $h$ là độ dài của nhóm.	

\end{tomtat}
%=================================================
\setcounter{subsubsection}{0}
\setcounter{ex}{0}
\setcounter{bt}{0}
\subsection{Các dạng toán thường gặp}
\begin{dang}{Mẫu số liệu ghép nhóm}
\end{dang}
\subsubsection{Ví dụ minh hoạ}
\begin{vd}%[Cánh Diều]%[1C5Y1-1]
	\immini{
		\textbf{Bảng bên} biểu diễn mẫu số liệu ghép nhóm được cho dưới dạng bảng tần số ghép nhóm. Hãy cho biết 
		\begin{enumerate}
			\item Mẫu số liệu có bao nhiêu số liệu; bao nhiêu nhóm?
			\item Tần số của mỗi nhóm.
		\end{enumerate}
	}{
		\begin{tabular}{|c|c|}
			\hline
			\textbf{Nhóm} & \textbf{Tần số}\\ 
			\hline
			$\left[0;5\right)$ & $11$\\
			\hline
			$\left[5;10\right)$ & $31$\\
			\hline
			$\left[10;15\right)$ & $45$\\
			\hline
			$\left[15;20\right)$ & $21$\\
			\hline
			$\left[20;26\right)$ & $12$\\
			\hline
			& $n = 120$ \\
			\hline
		\end{tabular}
	}
	\loigiai{
		\begin{enumerate}
			\item Mẫu số liệu gồm $120$ số liệu và $5$ nhóm.
			\item Tần số lần lượt của các nhóm $1$, $2$, $3$, $4$, $5$ lần lượt là $11$, $31$, $45$, $21$, $12$.
		\end{enumerate}	
	}
\end{vd}
\begin{vd}%[CTST]%[1T5B1-1]
	Một cửa hàng đã thống kê số ba lô bán được mỗi ngày trong tháng 9 với kết quả cho như sau: \begin{center}
		\begin{tabular}{lllllllllllllll}
			$12$ & $29$ & $12$ & $19$ & $15$ & $21$ & $19$ & $29$ & $28$ & $12$ & $15$ & $25$ & $16$ & $20$ & $29$\\
			$21$ & $12$ & $24$ & $14$ & $10$ & $12$ & $10$ & $23$ & $27$ & $28$ & $18$ & $16$ & $10$ & $20$ & $21$
		\end{tabular}
	\end{center}
	Hãy chia mẫu số liệu trên thành 5 nhóm, lập bảng tần số ghép nhóm, hiệu chỉnh bảng tần số ghép nhóm và xác định giá trị đại diện cho mỗi nhóm.
	\loigiai{
		Khoảng biến thiên của mẫu số liệu trên là $R=29-10=19$.\\
		Độ dài mỗi nhóm $L>\dfrac{R}{k}=\dfrac{19}{5}=3{,}8$.\\
		Ta chọn $L=4$ và chia dữ liệu thành các nhóm $[10; 14)$, $[14; 18)$, $[18; 22)$, $[22; 26)$, $[26; 30)$.\\
		Khi đó ta có bảng tần số ghép nhóm sau
		\begin{center}
			\begin{tabular}{|c|c|c|c|c|c|}
				\hline \textbf{Cân nặng} &{$[10; 14)$} &{$[14; 18)$} &{$[18; 22)$} &{$[22; 26)$} &{$[26; 30)$} \\
				\hline \textbf{Giá trị đại diện} & $12$ & $16$ & $20$ & $24$ & $28$ \\
				\hline \textbf{Số ba lô bán được} & $8$ & $5$ & $8$ & $3$ & $6$ \\
				\hline
			\end{tabular}
		\end{center}
	}
\end{vd}
\begin{vd}%[KNTT]%[Ngọc Hiếu]%[1K3B8-1]
	Bảng thống kê sau cho biết thời gian chạy (phút) của $30$ vận động viên (VĐV) trong một giải chạy Marathon.
	\begin{center}
		\begin{tabular}{|c|c|c|c|c|c|c|c|c|c|c|c|c|}
			\hline
			Thời gian&$129$&$130$&$133$&$134$&$135$&$136$&$138$&$141$&$142$&$143$&$144$&$145$\\
			\hline
			Số VĐV&$1$&$2$&$1$&$1$&$1$&$2$&$3$&$3$&$4$&$5$&$2$&$5$\\
			\hline
		\end{tabular}
	\end{center}
	Hãy chuyển mẫu số liệu trên sang mẫu số liệu ghép nhóm gồm sáu nhóm có độ dài bằng nhau và bằng $3$.
	\loigiai{
		Giá trị nhỏ nhất là $129$, giá trị lớn nhất là $145$ nên khoảng biến thiên là $145-129=16$. Tổng độ dài của sáu nhóm là $18$. Để cho đối xứng, ta chọn đầu mút trái của nhóm đầu tiên là $127{,}5$ và đầu mút phải của nhóm cuối cùng là $145{,}5$ ta được các nhóm là $[127{,}5;130{,}5),\; [130{,5};133{,5}],\ldots , [142{,}5;145{,}5]$. Đếm số giá trị thuộc mỗi nhóm, ta có mẫu số liệu ghép nhóm như sau
		\begin{center}
			\fontsize{9}{1pt}
			{\begin{tabular}{|c|c|c|c|c|c|c|}
					\hline
					Thời gian&$[125{,}5;130{,}5)$&$[130{,}5;133{,}5)$&$[133{,}5;136{,}5)$&$[136{,}5;139{,}5)$&$[139{,}5;142{,}5)$&$[142{,}5;145{,}5)$\\
					\hline
					Số VĐV&$3$&$1$&$4$&$3$&$7$&$12$\\
					\hline
			\end{tabular}}
		\end{center}
	}
\end{vd}
\begin{vd}%[Cánh Diều]%[1C5B1-1]
	Một trường trung học phổ thông chọn $36$ học sinh nam của khối $11$, do chiều cao của các bạn học sinh đó và thu được mẫu số liệu sau (đơn vị: centimét):
	$$
	\begin{array}{llllllllllll}
		160 & 161 & 161 & 162 & 162 & 162 & 163 & 163 & 163 & 164 & 164 & 164 \\
		164 & 165 & 165 & 165 & 165 & 165 & 166 & 166 & 166 & 166 & 167 & 167 \\
		168 & 168 & 168 & 168 & 169 & 169 & 170 & 171 & 171 & 172 & 172 & 174
	\end{array}
	$$
	Lập bảng tần số ghép nhóm bao gồm cả tần số tích luỹ cho mẫu số liệu trên có $5$ nhóm ứng với $5$ nửa khoảng:
	$$
	\left[160;163 \right),\ \left[163;169 \right),\ \left[166;169 \right),\ \left[169;172 \right),\ \left[172;175 \right).
	$$
	\loigiai{
		Bảng tần số ghép nhóm bao gồm cả tần số tích luỹ như sau:
		\begin{center}
			\begin{tabular}{|c|c|c|}
				\hline
				\textbf{Nhóm} & \textbf{Tần số} & \textbf{Tần số tích luỹ}\\ 
				\hline
				$\left[169;163\right)$ & $6$ & $6$\\
				\hline
				$\left[163;166\right)$ & $12$ & $18$\\
				\hline
				$\left[166;169\right)$ & $10$ & $28$\\
				\hline
				$\left[169;172\right)$ & $5$ & $33$\\
				\hline
				$\left[172;175\right)$ & $3$ & $36$\\
				\hline
				& $n = 36$ &\\
				\hline
			\end{tabular}
		\end{center}
	}
\end{vd}
\subsubsection{Bài tập rèn luyện}
\begin{bt}%[KNTT]%[1K3B8-1]
	Trong các mẫu số liệu sau, mẫu nào là mẫu số liệu ghép nhóm? Đọc và giải thích mẫu số liệu ghép nhóm đó.
	\begin{enumerate}
		\item Số tiền mà sinh viên chi cho thanh toán cước điện thoại trong tháng.
		\begin{center}
			\begin{tabular}{|c|c|c|c|c|c|}
				\hline
				Số tiền (nghìn đồng)&$[0;50)$&$[50;100)$&$[100;150)$&$[150;200)$&$[200;250)$\\
				\hline
				Số sinh viên&$5$&$12$&$23$&$17$&$3$\\
				\hline
			\end{tabular}
		\end{center}
		\item Thống kê nhiệt độ tại một điểm trong $40$ ngày, ta có bảng số liệu sau
		\begin{center}
			\begin{tabular}{|c|c|c|c|c|}
				\hline
				Nhiệt độ $(^\circ$ C)&$[19;22)$&$[22;25)$&$[25;28)$&$[28;31)$\\
				\hline
				Số ngày&$7$&$15$&$12$&$6$\\
				\hline
			\end{tabular}
		\end{center}
	\end{enumerate}
	\loigiai{
		Cả hai mẫu số liệu trên đều là mẫu số lớp ghép nhóm.
		\begin{enumerate}
			\item Có năm nhóm là
			\begin{itemize}
				\item Dưới $50$ nghìn đồng có $5$ sinh viên.
				\item Từ $50$ đến dưới $100$ nghìn đồng có $12$ sinh viên.
				\item Từ $100$ đến dưới $150$ nghìn đồng có $23$ sinh viên.
				\item Từ $150$ đến dưới $200$ nghìn đồng có $17$ sinh viên.
				\item Từ $200$ đến dưới $250$ nghìn đồng có $3$ sinh viên.
			\end{itemize}
			\item Có bốn nhóm là
			\begin{itemize}
				\item Từ $19^\circ$ C đến dưới $22^\circ$ C có $7$ ngày.
				\item Từ $22^\circ$ C đến dưới $25^\circ$ C có $15$ ngày.
				\item Từ $25^\circ$ C đến dưới $28^\circ$ C có $12$ ngày.
				\item Từ $128^\circ$ C đến dưới $31^\circ$ C có $6$ ngày.
			\end{itemize}
		\end{enumerate}
	}
\end{bt}	
\begin{bt}%[KNTT]%[1K3B8-1]
	Số sản phẩm một công nhân làm được trong một ngày được cho như sau:
	\begin{center}
		\begin{tabular}{c c c c c c c c c c c c c}
			$18$&$25$&$39$&$12$&$54$&$27$&$46$&$25$&$19$&$8$&$36$&$22$&\\
			$20$&$19$&$17$&$44$&$5$&$18$&$23$&$28$&$25$&$34$&$46$&$27$&$16$
		\end{tabular}
	\end{center}
	Hãy chuyển mẫu số liệu sang dạng ghép nhóm với sáu nhóm có độ dài bằng nhau.
	\loigiai{
		Khoảng biến thiên là $54-5=49$.\\
		Ta chia thành các nhóm sau $[4{,}5;13); [13;21{,}5);[21{,}5;30);\ldots ;[47;55{,}5)$.\\
		Đếm số giá trị của mỗi nhóm, ta có bảng ghép nhóm sau:
		\begin{center}
			\begin{tabular}{|c|c|c|c|c|c|c|}
				\hline
				Số sản phẩm &$[4{,}5;13)$&$[13;21{,}5)$&$[21{,}5;30)$&$[30;38{,}5)$&$[38{,}5;47)$&$[47;55{,}5)$\\
				\hline
				Số công nhân&$3$&$7$&$8$&$2$&$4$&$1$\\
				\hline
			\end{tabular}
		\end{center}
	}
\end{bt}
\begin{bt}%[KNTT]%[1K3B8-1]
	Thời gian ra sân (giờ) của một số cựu cầu thủ ở giải ngoại hạng Anh qua các thời kì được cho như sau:
	\begin{center}
		\begin{tabular}{c c c c c c c c}
			$653$ & $632$ & $609$ & $572$ & $565$ & $535$ & $516$ & $514$ \\
			$508$ & $505$ & $504$ & $504$ & $503$ & $499$ & $496$ & $492$ 
		\end{tabular}
	\end{center}
	Hãy chuyển mẫu số liệu trên sang dạng ghép nhóm với bảy nhóm có độ dài bằng nhau.
	\loigiai{
		Khoảng biến thiên là $653-492=161$.\\
		Ta chia thành các nhóm sau $[492;515); [515;538);[538;561);\ldots; [630;653]$.\\
		Đếm số giá trị của mỗi nhóm, ta có bảng ghép nhóm sau:
		\begin{center}
			\begin{tabular}{|c|c|c|c|c|c|c|c|}
				\hline
				Thời gian &$[492;515)$&$[515;538)$&$[538;561)$&$[561;584)$&$[584;607)$&$[607;630)$&$[630;653]$\\
				\hline
				Số cầu thủ &$9$&$2$&$0$&$2$&$0$&$1$&$2$\\
				\hline
			\end{tabular}
		\end{center}
	}
\end{bt}

%===================================
\setcounter{subsubsection}{0}
\setcounter{ex}{0}
\setcounter{bt}{0}
\begin{dang}{Số trung bình cộng (số trung bình)}
\end{dang}
\subsubsection{Ví dụ minh hoạ}
\begin{vd}%[Cánh Diều]%[1C5Y1-2]
	\immini
	{
		Một nhà thực vật học đo chiều dài của $74$ lá cây (đơn vị: milimét) và thu được bảng tần số như bảng bên. Tính chiều dài trung bình của $74$ lá cây trên theo đơn vị milimét (làm tròn kết quả đến hàng phần trăm).
	}
	{
		\begin{tabular}{|c|c|c|}
			\hline
			\textbf{Nhóm} & \textbf{Giá trị đại diện} & \textbf{Tần số}\\ 
			\hline
			$\left[5{,}45;5{,}85\right)$ & $5{,}65$ & $5$\\
			$\left[5{,}85;6{,}25\right)$ & $6{,}05$ & $9$\\
			$\left[6{,}25;6{,}65\right)$ & $6{,}45$ & $15$\\
			$\left[6{,}65;7{,}05\right)$ & $6{,}85$ & $19$\\
			$\left[7{,}05;7{,}45\right)$ & $7{,}25$ & $16$\\
			$\left[7{,}45;7{,}85\right)$ & $7{,}65$ & $8$\\
			$\left[7{,}85;8{,}25\right)$ & $8{,}05$ & $2$\\
			\hline
			&  & $n = 74$\\
			\hline
		\end{tabular}
	}
	\loigiai{
		Chiều dài trung bình của $74$ lá cây mà nhà thực vật học đo xấp xỉ là 
		\[
		\overline{x} = \dfrac{5\cdot 5{,}65 + 9 \cdot 6{,}05 + 15\cdot 6{,}45 + 19\cdot 6{,}85 + 16 \cdot 7{,}25 + 8\cdot 7{,}65 + 2\cdot 8{,}05}{74} \approx 6{,}80\ (\text{mm}).
		\]
	}
\end{vd}
\begin{vd}%[CTST]%[1T5B1-2]
	Kết quả khảo sát cân nặng của $25$ quả cam ở mỗi lô hàng $A$ và $B$ được cho ở bảng sau:
	\begin{center}
		\begin{tabular}{|c|c|c|c|c|c|}
			\hline \multicolumn{1}{|c|}{Cân nặng $(\mathrm{g})$} &{$[150; 155)$} &{$[155; 160)$} &{$[160; 165)$} &{$[165; 170)$} &{$[170; 175)$} \\
			\hline Số quả cam ở lô hàng $A$ & 2 & 6 & 12 & 4 & 1 \\
			\hline Số quả cam ở lô hàng $B$ & 1 & 3 & 7 & 10 & 4 \\
			\hline
		\end{tabular}
	\end{center}
	\begin{enumerate}
		\item Hãy ước lượng cân nặng trung bình của mỗi quả cam ở lô hàng $A$ và lô hàng $B$.
		\item Nếu so sánh theo số trung bình thì cam ở lô hàng nào nặng hơn?
	\end{enumerate}
	\loigiai{
		Ta có bảng thống kê số lượng cam theo giá trị đại diện:
		\begin{center}
			\begin{tabular}{|c|c|c|c|c|c|}
				\hline \multicolumn{1}{|c|}{Cân nặng $(\mathrm{g})$} &{$152{,}5$} &{$157{,}5$} &{$162{,}5$} &{$167{,}5$} &$172{,}5$\\
				\hline Số quả cam ở lô hàng $A$ & 2 & 6 & 12 & 4 & 1 \\
				\hline Số quả cam ở lô hàng $B$ & 1 & 3 & 7 & 10 & 4 \\
				\hline
			\end{tabular}
		\end{center}
		\begin{enumerate}
			\item Cân nặng trung bình của mỗi quả cam ở lô hàng $A$ xấp xỉ bằng
			\[(2\cdot 152{,}5+6\cdot 157{,}5+12\cdot 162{,}5+4\cdot 167{,}5+1\cdot 172{,}5): 25=161{,}7\ (\mathrm{g}). \]
			Cân nặng trung bình của mỗi quả cam ở lô hàng $B$ xấp xỉ bằng
			\[(1\cdot 152{,}5+3\cdot 157{,}5+7\cdot 162{,}5+10\cdot 167{,}5+4\cdot 172{,}5): 25=165{,}1\ (\mathrm{g}). \]
			\item Nếu so sánh theo số trung bình thì cam ở lô hàng $B$ nặng hơn cam ở lô hàng $A$.
		\end{enumerate}
	}
\end{vd}
\begin{vd}%[KNTT]%[1K3B9-1]
	Tìm cân nặng trung bình của học sinh lớp $11D$ cho trong Bảng $3.5$.
	\begin{center}
		\begin{tabular}{|c|c|c|c|c|c|c|}
			\hline
			Cân nặng	& $\left[40{,}5;45{,}5 \right)$ & $\left[45{,}5;50{,}5 \right)$ & $\left[50{,}5;55{,}5 \right)$ & $\left[55{,}5;60{,}5 \right)$ & $\left[60{,}5;65{,}5 \right)$ & $\left[65{,}5;70{,}5 \right)$ \\
			\hline
			Số học sinh&$10$	& $7$ & $16$ &$4$  & $2$ & $3$ \\
			\hline
		\end{tabular}

		Bảng $3.5$. Cân nặng của học sinh lớp $11D$.	
	\end{center}	
	\loigiai{
		Trong mỗi khoảng cân nặng, giá trị đại diện là trung bình cộng của hai giá trị đầu mút nên ta có bảng sau
		\begin{center}
			\begin{tabular}{|c|c|c|c|c|c|c|}
				\hline
				Cân nặng (kg)	& $43$ & $48$ & $53$ & $58$ & $63$ & $68$ \\
				\hline
				Số học sinh &$10$ & $7$ & $16$ &$4$  & $2$ & $3$ \\
				\hline
			\end{tabular}
		\end{center}	
		Tổng số học sinh là $n=42$. Cân nặng trung bình của học sinh lớp $11D$ là $$\overline{x}=\dfrac{10\cdot 43+7\cdot 48+16\cdot 53+4\cdot 58+2\cdot 63+3\cdot 68}{42}\approx51{,}81\,\mathrm{(kg)}.$$
	}
\end{vd}
\subsubsection{Bài tập rèn luyện}
\begin{bt}%[Cánh diều]%[1C5B1-5]
	Mẫu số liệu dưới đây ghi lại tốc độ của $40$ ô tô khi đi qua một trạm đo tốc độ (đơn vị: km/h)
	\[
	\begin{array}{cccccccccc}
		48{,}5 & 43 & 50 & 55 & 45 & 60 & 53 & 55,5 & 44 & 65 \\
		51 & 62,5 & 41 & 44,5 & 57 & 57 & 68 & 49 & 46{,}5 & 53{,}5 \\
		61 & 49{,}5 & 54 & 62 & 59 & 56 & 47 & 50 & 60 & 61 \\
		49{,}5 & 52{,}5 & 57 & 47 & 60 & 55 & 45 & 47,5 & 48 & 61{,}5
	\end{array}
	\]
	\begin{enumerate}
		\item Lập bảng tần số ghép nhóm cho mẫu số liệu trên có sáu nhóm ứng với sáu nửa khoảng:
		\[
		[40 ; 45),[45 ; 50),[50 ; 55),[55 ; 60),[60 ; 65),[65 ; 70).
		\]
		\item Xác định số trung bình cộng của mẫu số liệu ghép nhóm trên.
	\end{enumerate}
	\loigiai{
		\begin{enumerate}
			\item Ta có bảng tần số ghép nhóm của mẫu số liệu trên như sau:
			\begin{center}
				\begin{tabular}{|c|c|c|c|}
					\hline
					\textbf{Nhóm} & \textbf{Giá trị đại diện} & \textbf{Tần số} & \textbf{Tần số tích luỹ}\\ 
					\hline
					$\left[40;45\right)$ & $42{,}5$ & $4$ & $4$\\
					$\left[45;50\right)$ & $47{,}5$ & $11$ & $15$\\
					$\left[50;55\right)$ & $52{,}5$ & $7$ & $22$\\
					$\left[55;60\right)$ & $57{,}5$ & $8$ & $30$\\
					$\left[60;65\right)$ & $62{,}5$ & $8$ & $38$\\
					$\left[65;70\right)$ & $67{,}5$ & $2$ & $40$\\
					\hline
					&  & $n = 40$ &\\
					\hline
				\end{tabular}
			\end{center}
			\item Trung bình cộng của mẫu số liệu trên là
			\[
			\overline{x} = \dfrac{42{,}5 \cdot 4 + 47{,}5 \cdot 11 + 52{,}5 \cdot 7+ 57{,}5 \cdot 8+ 62{,}5 \cdot 8 + 67{,}5 \cdot 2}{40} = 53{,}875\text{ (km/h)}.
			\]
			\item Ta thấy: Nhóm $2$ ứng với nửa khoảng $\left[45;50\right)$ là nhóm có tần số lớn nhất với $u=45$, $g=5$, $n_2 = 11$. Nhóm $1$ có tần số $n_1 = 4$, nhóm $3$ có tần số $n_3 = 7$.
		\end{enumerate}
	}
\end{bt}
\begin{bt}%[KNTT]%[1K3B9-4]
	Tuổi thọ (năm) của 50 bình ắc quy ô tô được cho như sau:
	\begin{center}
		\begin{tabular}{|c|c|c|c|c|c|c|}
			\hline
			Tuổi thọ (năm)	& $\left[2;2{,}5 \right)$ & $\left[2{,}5;3 \right)$ & $\left[3;3{,}5 \right)$&$\left[3{,}5;4 \right)$&$\left[4;4{,}5 \right)$&$\left[4{,}5;5 \right)$  \\
			\hline
			Tần số &$4$	& $9$ & $14$ &$11$  & $7$&$5$ \\
			\hline
		\end{tabular}
	\end{center}
	Tính tuổi thọ trung bình của $50$ bình ắc quy ô tô này.
	\loigiai{
		Ta có bảng sau
		\begin{center}
			\begin{tabular}{|c|c|c|c|c|c|c|}
				\hline
				Tuổi thọ (năm)	& $2{,}25$ & $2{,}75$ & $3{,}25$&$3{,}75$&$4{,}25$&$4{,}75$  \\
				\hline
				Tần số &$4$	& $9$ & $14$ &$11$  & $7$&$5$\\
				\hline
			\end{tabular}	
		\end{center}
		Tuổi thọ trung bình của 50 bình ắc quy ô tô này là
		$$\overline{x}=\dfrac{2{,}25\cdot 4+2{,}75\cdot 9+3{,}25\cdot 14+3{,}75\cdot 11+4{,}25\cdot 7+4{,}75\cdot 5}{50}=3{,}48 \, \text{(năm)}.$$
	}
\end{bt}
\begin{bt}%[KNTT]%[1K3B9-4]
	\immini{
		Phỏng vấn một số học sinh lớp $11$ về thời gian (giờ) ngủ của một buổi tối, thu được bảng số liệu ở bên. So sánh thời gian ngủ trung bình của các bạn học sinh nam và nữ.
	}
	{
		\begin{tabular}{|c|c|c|}
			\hline
			Thời gian	& Số học sinh nam & Số học sinh nữ\\
			\hline
			$\left[4;5 \right)$	& $6$ & $4$ \\
			\hline
			$\left[5;6 \right)$	& $10$ & $8$ \\
			\hline
			$\left[6;7 \right)$	& $13$ & $10$ \\
			\hline
			$\left[7;8 \right)$	& $9$ & $11$ \\
			\hline
			$\left[8;9 \right)$	& $7$ & $8$ \\
			\hline
		\end{tabular}
	}
	\loigiai{
		Trong mỗi khoảng thời gian, giá trị đại diện là trung bình cộng của giá trị hai đầu mút nên ta có bảng sau:
		\begin{center}
			\begin{tabular}{|c|c|c|}
				\hline
				Thời gian	& Số học sinh nam & Số học sinh nữ\\
				\hline
				$4{,}5$	& $6$ & $4$ \\
				\hline
				$5{,}5$	& $10$ & $8$ \\
				\hline
				$6{,}5$	& $13$ & $10$ \\
				\hline
				$7{,}5$	& $9$ & $11$ \\
				\hline
				$8{,}5$	& $7$ & $8$ \\
				\hline
			\end{tabular}	
		\end{center}
		Tổng số học sinh nam là $n_1=6+10+13+9+7=45$.\\ Thời gian ngủ trung bình của học sinh nam là:
		$$\overline{x_1}=\dfrac{4{,}5\cdot 6+5{,}5\cdot10+6{,}5\cdot13+7{,}5\cdot9+8{,}5\cdot7}{45}=\dfrac{587}{90}\approx 6{,}52\,\, \text{(giờ)}.$$
		Tổng số học sinh nữ là $n_2=4+8+10+11+8=41$. Thời gian ngủ trung bình của học sinh nữ là:
		$$\overline{x_2}=\dfrac{4,5\cdot4+5,5\cdot8+6,5\cdot10+7,5\cdot11+8,5\cdot8}{41}=\dfrac{555}{82}\approx 6{,}77 \,\,\text{(giờ)}.$$
		Vì $\overline{x_2}>\overline{x_1}$ nên thời gian ngủ trung bình của các bạn học sinh nữ lớn hơn thời gian ngủ trung bình của các bạn nam.
	}
\end{bt}
\begin{bt}%[KNTT]%[1K3B9-4]
	Quãng đường (km) từ nhà đến nơi làm việc của 40 công nhân một nhà máy được ghi lại như sau:
	\begin{center}
		\begin{tabular}{cccccccccccccccccccc}
			$5$	& $3$ &$10$ & $20$ & $25$ & $11$ & $13$ & $7$ & $12$ & $31$\\
			$19$ &$10$  &$12$  & $17$ & $18$ & $11$ & $32$ & $17$ &$16$  &$2$ \\
			$7$	& $9$ &$7$ & $8$ & $3$ & $5$ & $12$ & $15$ & $18$ & $3$\\
			$12$ &$14$  &$2$  & $9$ & $6$ & $15$ & $15$ & $7$ &$6$  &$12$
		\end{tabular}
	\end{center}
	\begin{enumerate}
		\item [a)] Ghép nhóm dãy số liệu trên thành các khoảng có độ rộng bằng nhau, khoảng đầu tiên là $\left[0;5\right)$. Tìm giá trị đại diện cho mỗi nhóm.
		\item [b)] Tính số trung bình của mẫu số liệu không ghép nhóm và mẫu số liệu ghép nhóm. Giá trị nào chính xác hơn?
	\end{enumerate}
	\loigiai{
		\begin{enumerate}
			\item [a)] Giá trị nhỏ nhất của mẫu số liệu là $2$, giá trị lớn nhất là $32$, khoảng đầu tiên của mẫu số liệu ghép nhóm là $\left[0;5\right)$ nên ta ghép nhóm mẫu số liệu như sau
			\begin{center}
				\begin{tabular}{|c|c|c|c|c|c|c|c|}
					\hline
					Quãng đường		 & $\left[0;5\right)$ & $\left[5;10\right)$ & $\left[10;15\right)$ & $\left[15;20\right)$ & $\left[20;25\right)$& $\left[25;30\right)$& $\left[30;35\right)$\\
					\hline
					Số công nhân		& $5$ & $11$ & $11$ & $9$ & $1$ & $1$ & $2$ \\
					\hline
				\end{tabular}
			\end{center}
			Trong mỗi khoảng, giá trị đại điện là trung bình cộng của hai giá trị đầu mút nên ta có bảng sau
			\begin{center}
				\begin{tabular}{|c|c|c|c|c|c|c|c|}
					\hline
					Quãng đường		 & $2{,}5$ & $7{,}5$ & $12{,}5$ & $17{,}5$ & $22{,}5$& $27{,}5$& $32{,}5$\\
					\hline
					Số công nhân		& $5$ & $11$ & $11$ & $9$ & $1$ & $1$ & $2$ \\
					\hline
				\end{tabular}
			\end{center}
			\item [b)] Số trung bình của mẫu số liệu không ghép nhóm là
			$$\overline{x}=\dfrac{5+3+10+\cdots +12}{40}=11{,}9.$$
			Số trung bình của mẫu số liệu ghép nhóm là
			$$\overline{x}=\dfrac{5\cdot 2{,}5+11\cdot 7{,}5+11\cdot 12{,}5+9\cdot 17{,}5+1\cdot 22{,}5+1\cdot 27{,}5+2\cdot 32{,}5}{40}=12{,}625.$$
			Số trung bình của mẫu số liệu không ghép nhóm sẽ chính xác hơn số trung bình của mẫu số liệu ghép nhóm vì số trung bình của dữ liệu không ghép nhóm sử dụng chính xác các số liệu, còn số trung bình của dữ liệu ghép nhóm sử dụng giá trị đại diện của mỗi khoảng ghép nhóm.
		\end{enumerate}
	}
\end{bt}
\begin{bt}%[CTST]%[1T5B1-2]
	Anh Văn ghi lại cự li 30 lần ném lao của mình ở bảng sau (đơn vị: mét):
	\begin{center}
		\begin{tabular}{|c|c|c|c|c|c|c|c|c|c|}
			\hline $72{,}1$ & $72{,}9$ & $70{,}2$ & $70{,}9$ & $72{,}2$ & $71{,}5$ & $72{,}5$ & $69{,}3$ & $72{,}3$ & $69{,}7$ \\
			\hline $72{,}3$ & $71{,}5$ & $71{,}2$ & $69{,}8$ & $72{,}3$ & $71{,}1$ & $69{,}5$ & $72{,}2$ & $71{,}9$ & $73{,}1$ \\
			\hline $71{,}6$ & $71{,}3$ & $72{,}2$ & $71{,}8$ & $70{,}8$ & $72{,}2$ & $72{,}2$ & $72{,}9$ & $72{,}7$ & $70{,}7$ \\
			\hline
		\end{tabular}
	\end{center}
	\begin{enumerate}
		\item Tính cự li trung bình của mỗi lần ném.
		\item Tổng hợp lại kết quả ném của anh Văn vào bảng tần số ghép nhóm theo mẫu sau:
		\begin{center}
			\begin{tabular}{|c|c|c|c|c|c|}
				\hline Cự li $(\mathrm{m})$ &{$[69{,}2; 70)$} &{$[70; 70{,}8)$} &{$[70{,}8; 71{,}6)$} &{$[71{,}6; 72{,}4)$} &{$[72{,}4; 73{,}2)$} \\
				\hline Số lần & $?$ & $?$ & $?$ & $?$ & $?$ \\
				\hline
			\end{tabular}
		\end{center}
		\item Hãy ước lượng cự li trung bình mỗi lần ném từ bảng tần số ghép nhóm trên.
		\item Khả năng anh Văn ném được khoảng bao nhiêu mét là cao nhất?
	\end{enumerate}
	\loigiai{
		\begin{enumerate}
			\item Điểm tổng của mỗi đợt gồm 10 lần ném
			\begin{center}
				\begin{tabular}{|c|c|c|c|c|c|c|c|c|c|c|}
					\hline Điểm &Điểm &Điểm &Điểm &Điểm &Điểm &Điểm &Điểm &Điểm &Điểm &Tổng \\
					\hline $72{,}1$ & $72{,}9$ & $70{,}2$ & $70{,}9$ & $72{,}2$ & $71{,}5$ & $72{,}5$ & $69{,}3$ & $72{,}3$ & $69{,}7$ &$713{,}6$\\
					\hline $72{,}3$ & $71{,}5$ & $71{,}2$ & $69{,}8$ & $72{,}3$ & $71{,}1$ & $69{,}5$ & $72{,}2$ & $71{,}9$ & $73{,}1$ &$714{,}9$\\
					\hline $71{,}6$ & $71{,}3$ & $72{,}2$ & $71{,}8$ & $70{,}8$ & $72{,}2$ & $72{,}2$ & $72{,}9$ & $72{,}7$ & $70{,}7$ &$718{,}4$\\
					\hline
				\end{tabular}
			\end{center}
			Cự li trung bình của mỗi lần ném của anh Văn
			\[\overline{x}=\dfrac{713{,}6+714{,}9+718{,}4}{30}\approx71{,}56\ (\mathrm{m}). \]
			\item Bảng tần số ghép nhóm kết quả ném của anh Văn:
			\begin{center}
				\begin{tabular}{|c|c|c|c|c|c|}
					\hline Cự li $(\mathrm{m})$ &{$[69{,}2; 70)$} &{$[70; 70{,}8)$} &{$[70{,}8; 71{,}6)$} &{$[71{,}6; 72{,}4)$} &{$[72{,}4; 73{,}2)$} \\
					\hline Số lần & $4$ & $2$ & $7$ & $12$ & $5$ \\
					\hline
				\end{tabular}
			\end{center}
			\item Bảng tần số ghép nhóm kết quả ném của anh Văn (theo giá trị đại diện):
			\begin{center}
				\begin{tabular}{|c|c|c|c|c|c|}
					\hline Cự li $(\mathrm{m})$ &{$[69{,}2; 70)$} &{$[70; 70{,}8)$} &{$[70{,}8; 71{,}6)$} &{$[71{,}6; 72{,}4)$} &{$[72{,}4; 73{,}2)$} \\
					\hline Giá trị đại diện &$69{,}6$ &$70{,}4$ &$71{,}2$ &$72{,}0$ &$72{,}8$\\
					\hline Số lần & $4$ & $2$ & $7$ & $12$ & $5$ \\
					\hline
				\end{tabular}
			\end{center}
			Cự li trung bình mỗi lần ném của anh Văn qua bảng tần số ghép nhóm
			\[(69{,}6\cdot 4+70{,}4\cdot 2+71{,}2\cdot 7+72\cdot 12+72{,}8\cdot 5):30=71{,}52\ (\mathrm{m}).  \]
			\item Nhóm chứa mốt của mẫu số liệu trên là nhóm $[71{,}6; 72{,}4)$.\\
			Do đó $u_m=71{,}6$; $n_{m-1}=7$; $n_m=12$; $n_{m+1}=5$; $u_{m+1}-u_m=72{,}4-71{,}6=0{,}8$.\\
			Mốt của mẫu số liệu ghép nhóm là
			\[M_0=71{,}6+\dfrac{12-7}{(12-7)+(12-5)} \cdot 0{,}8=\dfrac{101}{14} \approx 71{,}93. \]
			Dựa vào kết quả trên thì khả năng anh Văn ném được cao nhất là khoảng $71{,}93$ mét.
		\end{enumerate}
	}
\end{bt}
\begin{bt}%[CTST]%[1T5B1-2]
	Người ta đếm số xe ô tô đi qua một trạm thu phí mỗi phút trong khoảng thời gian từ $9$ giờ đến $9$ giờ $30$ phút sáng. Kết quả được ghi lại ở bảng sau:
	\begin{center}
		\begin{tabular}{|c|c|c|c|c|c|c|c|c|c|c|c|c|c|c|}
			\hline $15$ & $16$ & $13$ & $21$ & $17$ & $23$ & $15$ & $21$ & $6$ & $11$ & $12$ & $23$ & $19$ & $25$ & $11$ \\
			\hline $25$ & $7$ & $29$ & $10$ & $28$ & $29$ & $24$ & $6$ & $11$ & $23$ & $11$ & $21$ & $9$ & $27$ & $15$ \\
			\hline
		\end{tabular}
	\end{center}
	\begin{enumerate}
		\item Tính số xe trung bình đi qua trạm thu phí trong mỗi phút.
		\item Tổng hợp lại số liệu trên vào bảng tần số ghép nhóm theo mẫu sau:
		\begin{center}
			\begin{tabular}{|c|c|c|c|c|c|}
				\hline Số xe &{$[6; 10]$} &{$[11; 15]$} &{$[16; 20]$} &{$[21; 25]$} &{$[26; 30]$} \\
				\hline Số lần & $?$ & $?$ & $?$ & $?$ & $?$ \\
				\hline
			\end{tabular}
		\end{center}
		\item Hãy ước lượng trung bình số xe đi qua trạm thu phí trong mỗi phút từ bảng tần số ghép nhóm trên.
	\end{enumerate}
	\loigiai{
		\begin{enumerate}
			\item 
%			Bảng tần số
%			\begin{center}
%				\begin{tabular}{|c|c|c|c|c|c|c|c|c|c|c|c|c|c|c|c|c|c|c|c|}
%					\hline Giá trị &$6$ & $7$ & $9$ & $10$ & $11$ & $12$ & $13$ & $15$ & $16$ & $17$ & $19$ & $21$ & $23$ & $24$ & $25$ & $27$ & $28$ & $29$ &\\
%					\hline Tần số &$2$ & $1$ & $1$ & $1$ & $4$ & $1$ & $1$ & $3$ & $1$ & $1$ & $1$ & $3$ & $3$ & $1$ & $2$ & $1$ & $1$ & $2$ &$N=30$\\
%					\hline
%				\end{tabular}
%			\end{center}
			Số xe trung bình đi qua trạm thu phí trong mỗi phút là
			\allowdisplaybreaks
			\begin{eqnarray*}
				\overline{x}&=&\dfrac{6\cdot 2+7+9+10+11\cdot 4+12+13+15\cdot 3}{30}\\
				&&+\dfrac{16+17+19+21\cdot 3+23\cdot 3+24+25\cdot 2+27+28+29\cdot 2}{30}\\
				&\approx& 17{,}43\ (\text{xe}).
			\end{eqnarray*}	
			\item Bảng tần số ghép nhóm
			\begin{center}
				\begin{tabular}{|c|c|c|c|c|c|}
					\hline Số xe &{$[6; 10]$} &{$[11; 15]$} &{$[16; 20]$} &{$[21; 25]$} &{$[26; 30]$} \\
					\hline Số lần & $5$ & $9$ & $3$ & $9$ & $4$ \\
					\hline
				\end{tabular}
			\end{center}
			\item Bảng tần số ghép nhóm (theo giá trị đại diện) được hiệu chỉnh lại như sau
			\begin{center}
				\begin{tabular}{|c|c|c|c|c|c|}
					\hline Số xe &{$[5{,}5; 10{,}5)$} &{$[10{,}5; 15{,}5)$} &{$[15{,}5; 20{,}5)$} &{$[20{,}5; 25{,}5)$} &{$[25{,}5; 30{,}5)$} \\
					\hline Giá trị đại diện &{$8$} &{$13$} &{$18$} &{$23$} &{$28$} \\
					\hline Số lần & $5$ & $9$ & $3$ & $9$ & $4$ \\
					\hline
				\end{tabular}
			\end{center}
			Số xe trung bình đi qua trạm qua bảng tần số ghép nhóm là
			\[\overline{x}=\dfrac{8\cdot 5+13\cdot 9+18\cdot 3+23\cdot 9+28\cdot 4}{30}\approx 17{,}67\ (\text{xe}). \]
		\end{enumerate}
	}
\end{bt}
\begin{bt}%[CTST]%[1T5B1-2]
	Một thư viện thống kê số lượng sách được mượn mỗi ngày trong ba tháng ở bảng sau:
	\begin{center}
		\begin{tabular}{|c|c|c|c|c|c|c|c|}
			\hline Số sách &{$[16; 20]$} &{$[21; 25]$} &{$[26; 30]$} &{$[31; 35]$} &{$[36; 40]$} &{$[41; 45]$} &{$[46; 50]$} \\
			\hline Số ngày & 3 & 6 & 15 & 27 & 22 & 14 & 5 \\
			\hline
		\end{tabular}
	\end{center}
	Hãy ước lượng số trung bình của mẫu số liệu ghép nhóm trên.
	\loigiai{
		Vì số lượng sách được mượn là số nguyên nên ta hiệu chỉnh bảng tần số ghép nhóm (theo giá trị đại diện) như sau
		\begin{center}
			{\footnotesize \begin{tabular}{|c|c|c|c|c|c|c|c|}
					\hline Số sách &{$[15{,}5; 20{,}5)$} &{$[20{,}5; 25{,}5)$} &{$[25{,}5; 30{,}5)$} &{$[30{,}5; 35{,}5)$} &{$[35{,}5; 40{,}5]$} &{$[40{,}5; 45{,}5)$} &{$[45{,}5; 50{,}5)$} \\
					\hline Giá trị đại diện &{$18$} &{$23$} &{$28$} &{$33$} &{$38$} &{$43$} &{$48$} \\
					\hline Số ngày & 3 & 6 & 15 & 27 & 22 & 14 & 5 \\
					\hline
			\end{tabular}}
		\end{center}
		Trung bình số lượng sách được mượn mỗi ngày trong 3 tháng của thư viện là
		\[\overline{x}=\dfrac{18\cdot 3+23\cdot 6+28\cdot 15+33\cdot 27+38\cdot 22+43\cdot 14+48\cdot 5}{92}\approx 34{,}58. \]
	}
\end{bt}
\begin{bt}%[CTST]%[1T5B1-2]
	Kết quả đo chiều cao của $200$ cây keo $3$ năm tuổi ở một nông trường được biểu diễn ở biểu đồ dưới đây.
	\begin{center}
		\begin{tikzpicture}[scale=1,font=\scriptsize]
			\def\hoanh{11.5};
			\def\tung{6.5};
			\def\mau{cyan};
			\foreach \x/\n in{1/20,3/35,5/60,7/55,9/30}{\draw[line width=16mm,\mau] (\x,0)--++(0,{\n/10});
				\draw[dashed] (\x,{\n/10})node[above]{$\n$}--(0,{\n/10}) node[left]{$\n$};}
			\foreach \x/\p in {1/[8{,}5;8{,}8),3/[8{,}8;9{,}1),5/[9{,}1;9{,}4),7/[9{,}4;9{,}7),9/[9{,}7;10{,}0)}{\node[below] at (\x,0){\scriptsize $\p$};}
			\draw[->] (0,0)--(\hoanh,0) node[below]{($m$)};
			\draw[->] (0,0)node[below left]{$O$}--(0,\tung) node[left]{(Số cây)};
			\path (current bounding box.north) node[above]		{\textbf{Chiều cao 200 cây keo 3 năm tuổi}};
		\end{tikzpicture}
	\end{center}
	Hãy ước lượng số trung bình của mẫu số liệu ghép nhóm trên.
	\loigiai{
		Bảng tần số ghép nhóm (theo giá trị đại diện)
		\begin{center}
			\begin{tabular}{|c|c|c|c|c|c|}
				\hline Chiều cao &$[8{,}5; 8{,}8)$ &{$[8{,}8; 9{,}1)$} &{$[9{,}1; 9{,}4)$} &{$[9{,}4; 9{,}7)$} &{$[9{,}7; 10{,}0)$} \\
				\hline Giá trị đại diện &$8{,}65$ &$8{,}95$ &$9{,}25$ &$9{,}55$ &$9{,}85$ \\
				\hline Số cây & $20$ & $35$ & $60$ & $55$ & $30$\\
				\hline
			\end{tabular}
		\end{center}
		Chiều cao trung bình của $200$ cây keo 3 năm tuổi là
		\[\overline{x}=\dfrac{8{,}65\cdot 20+8{,}95\cdot 35+9{,}25\cdot 60+9{,}55\cdot 55+9{,}85\cdot 30}{200}\approx 9{,}31. \]
	}
\end{bt}
\begin{bt}%[CTST]%[1T5K2-2]
	Kiểm tra điện lượng của một số viên pin tiểu do một hãng sản xuất thu được kết quả như sau:
	\begin{center}
		\begin{tabular}{|c|c|c|c|c|c|}
			\hline 
			\begin{tabular}{c}
				\textbf{Điện lượng} \\	\textbf{(nghìn mAh)}
			\end{tabular} 
			& $ \left[ 0{,}9 ; 0{,}95\right)  $ & $ \left[ 0{,}95 ; 1{,}0\right)  $ & $ \left[ 1{,}0 ; 1{,}05\right)  $ &$ \left[ 1{,}05 ; 1{,}1\right)  $  &  $ \left[ 1{,}1 ; 1{,}15\right)  $\\ 
			\hline 
			\textbf{Số viên pin}& $ 10 $ & $ 20 $ & $ 35 $ & $ 15 $ & $ 5 $ \\ 
			\hline 
		\end{tabular} 
	\end{center}
	Hãy ước lượng số trung bình của mẫu số liệu ghép nhóm trên.
	\loigiai{
		Tìm số trung bình của mẫu số liệu ghép nhóm.\\
		Ta có bảng thống kê điện lượng của pin theo giá trị đại diện là:
		\begin{center}
			\begin{tabular}{|c|c|c|c|c|c|}
				\hline 
				\begin{tabular}{c}
					\textbf{Điện lượng} \\	\textbf{(nghìn mAh)}
				\end{tabular} 
				& $ \left[ 0{,}9 ; 0{,}95\right)  $ & $ \left[ 0{,}95 ; 1{,}0\right)  $ & $ \left[ 1{,}0 ; 1{,}05\right)  $ &$ \left[ 1{,}05 ; 1{,}1\right)  $  &  $ \left[ 1{,}1 ; 1{,}15\right)  $\\ 
				\hline 
				\textbf{Giá trị đại diện}& $ 0{,}925 $ & $ 0{,}975 $ & $ 1{,}025 $ & $ 1{,}075 $ & $ 1{,}125 $ \\ 
				\hline
				\textbf{Số viên pin}& $ 10 $ & $ 20 $ & $ 35 $ & $ 15 $ & $ 5 $ \\ 
				\hline 
			\end{tabular} 
		\end{center}
		Số trung bình của mẫu số liệu ghép nhóm theo dõi điện lượng của một số viên pin xấp xỉ bằng $$\dfrac{0{,}925\cdot 10 + 0{,}975\cdot 20 +1{,}025 \cdot 35 +1{,}075 \cdot 15+1{,}125 \cdot 5}{10+20+35+15+5}\approx 1{,}016.$$
	}
\end{bt}

%===================================
\setcounter{subsubsection}{0}
\setcounter{ex}{0}
\setcounter{bt}{0}
\begin{dang}{Trung vị}
\end{dang}
\subsubsection{Ví dụ minh hoạ}
\begin{vd}%[Cánh Diều]%[1C5B1-3]
	\immini
	{
		Sau khi kiểm tra về số học sinh trong $100$ lớp học, người ta chia mẫu số liệu đó thành năm nhóm căn cứ vào số lượng học sinh của mỗi lớp (đơn vị: học sinh) và lập bảng tần số ghép nhóm bao gồm tần số tích luỹ như bảng bên. Tìm trung vị của mẫu số liệu đó.
	}
	{
		\begin{tabular}{|c|c|c|}
			\hline
			\textbf{Nhóm} & \textbf{Tần số} & \textbf{Tần số tích luỹ}\\ 
			\hline
			$\left[36;38\right)$ & $9$ & $9$\\
			$\left[38;40\right)$ & $15$ & $24$\\
			$\left[40;42\right)$ & $25$ & $49$\\
			$\left[42;44\right)$ & $30$ & $79$\\
			$\left[44;46\right)$ & $21$ & $100$\\
			\hline
			& $n = 100$ &\\
			\hline
		\end{tabular}
	}
	\loigiai{
		Số phần tử của mẫu là $n=100$. Ta có $\dfrac{n}{2} = \dfrac{100}{2} = 50$.\\
		Do $cf_3 = 49 < 50 < cf_4 = 79$ nên nhóm $4$ là nhóm đầu tiên có tần số tích luỹ lớn hơn hoặc bằng $50$.\\
		Xét nhóm $4$ là nhóm $\left[42;44\right)$ có $r=42$; $d=2$ và $n_4=30$ và nhóm $3$ là nhóm $\left[40;42\right)$ có $cf_3 = 49$.\\
		Khi đó trung vị của mẫu số liệu là 
		\[
		M_e = 42 + \dfrac{50 - 49}{30} \cdot 2 \approx 42\text{ (học sinh)}.
		\]
	}
\end{vd}
\begin{vd}%[KNTT]%[1K3B9-2]
	Thời gian (phút) truy cập internet mỗi buổi tối của một số học sinh được cho trong bảng sau:
	\begin{center}
		\begin{tabular}{|c|c|c|c|c|c|c|}
			\hline
			Thời gian (phút)	& $\left[9{,}5;12{,}5 \right)$ & $\left[12{,}5;15{,}5 \right)$ & $\left[15{,}5;18{,}5 \right)$ & $\left[18{,}5;21{,}5 \right)$ & $\left[21{,}5;24{,}5 \right)$ \\
			\hline
			Số học sinh&$3$	& $12$ & $15$ &$24$  & $2$  \\
			\hline
		\end{tabular}	
	\end{center}
	Tính trung vị của mẫu số liệu ghép nhóm này.
	\loigiai{
		Cỡ mẫu là $n=3+12+15+24+2=56$.\\
		Gọi $x_1,\,\ldots,\,x_{56}$ là thời gian vào internet của $56$ học sinh và giả sử dãy này đã được sắp xếp theo thứ tự tăng dần. Khi đó, trung vị là $\dfrac{x_{28}+x_{29}}{2}$. Do $2$ giá trị $x_{28},\,x_{29}$ thuộc nhóm $\left[15{,}5;18{,}5 \right)$ nên nhóm này chứa trung vị. Do đó, $p=3$; $a_3=15{,}5$; $m_3=15$; $m_1+m_2=3+12=15$; $a_4-a_3=3$ và ta có $$M_e=15{,}5+\dfrac{\dfrac{56}{2}-15}{15}\cdot 3=18{,}1.$$
	}
\end{vd}
\begin{vd}%[CTST]%[1T5B2-1]
	Kết quả khảo sát cân nặng của $ 25 $ quả bơ ở một lô hàng cho trong bảng sau:
	\begin{center}
		\begin{tabular}{|c|c|c|c|c|c|}
			\hline 
			\textbf{Cân nặng}\textbf{ (g)}	& $ \left[150 ; 155 \right) $ & $ \left[ 155 ; 160\right)  $ & $ \left[160 ; 165\right)  $ & $ \left[ 165 ; 170\right)  $ & $ \left[170 ; 175 \right)  $ \\ 
			\hline 
			\textbf{Số quả bơ}	& $ 1 $ & $ 7 $ & $ 12 $ & $ 3 $ & $ 2 $ \\ 
			\hline 
		\end{tabular} 
	\end{center}
	Hãy tìm trung vị của mẫu số liệu ghép nhóm trên.
	\loigiai{
		Gọi $ x_1; x_2; \ldots ; x_{25} $ là cân nặng của $ 25$ quả bơ xếp theo thứ tự không giảm.\\
		Do $ x_1\in \left[150 ; 155 \right) $; $ x_2, \ldots, x_8 \in \left[ 155 ; 160\right) $; $ x_9, \ldots, x_{20} \in \left[ 160 ; 165\right) $ nên trung vị của mẫu số liệu $ x_1; x_2; \ldots; x_{25} $ là $ x_{13}\in \left[ 160 ; 165\right)$.\\
		Ta xác định được $ n=25 $, $ n_m=12 $, $ C=1+7=8 $, $ u_m=160 $, $ u_{m+1}=165 $.\\
		Vậy trung vị của mẫu số liệu ghép nhóm là $$ M_e=160+\dfrac{\dfrac{25}{2}-8}{12}\cdot(165-160) =161{,}875.$$
	}
\end{vd}
\begin{vd}%[CTST]%[1T5K2-1]
	Trong tuần lễ bảo vệ môi trường, các học sinh khối $ 11 $ tiến hành thu nhặt vỏ chai nhựa để tái chế. Nhà trường thống kê kết quả thu nhặt vỏ chai của học sinh khối $ 11 $ ở bảng sau
	\begin{center}
		\begin{tabular}{|c|c|c|c|c|c|}
			\hline 
			\textbf{Số vỏ chai nhựa}	& $ \left[ 11 ; 15\right]  $ & $ \left[ 16 ; 29\right]  $ & $ \left[21 ; 25 \right]  $ & $ \left[ 26 ; 30\right]  $ & $ \left[31 ; 35 \right]  $ \\ 
			\hline 
			\textbf{Số học sinh}	& $ 53 $ & $ 82 $ & $ 48 $ & $ 39 $ & $ 18 $ \\ 
			\hline 
		\end{tabular} 
	\end{center}
	Hãy tìm trung vị của mẫu số liệu ghép nhóm trên.
	\loigiai{
		Do số vỏ chai là số nguyên nên ta hiệu chỉnh lại như sau:
		\begin{center}
			\begin{tabular}{|c|c|c|c|c|c|}
				\hline 
				\textbf{Số vỏ chai nhựa}	& $ \left[ 10{,}5 ; 15{,}5\right) $ & $ \left[ 15{,}5 ; 20{,}5\right) $ & $ \left[ 20{,}5 ; 25{,}5\right) $ & $ \left[ 25{,}5 ; 30{,}5\right) $ & $ \left[30{,}5 ; 35{,}5 \right)  $ \\ 
				\hline 
				\textbf{Số học sinh}& $ 53 $ & $ 82 $ & $ 48 $ & $ 39 $ & $ 18 $ \\ 
				\hline 
			\end{tabular} 
		\end{center}
		Số học sinh tham gia thu nhặt vỏ chai nhựa là $$ n=53+82+48+39+18=240.$$
		Gọi $ x_1; x_2; \ldots ; x_{240} $ lần lượt là số vỏ chai $ 240 $ học sinh khối $ 11 $ thu nhặt được xếp theo thứ tự không giảm.\\
		Do $ x_1, \ldots, x_{53}\in \left[10{,}5 ; 15{,}5 \right) $; $ x_{54}, \ldots, x_{135}\in \left[ 15{,}5 ; 20{,}5\right)$ nên trung vị của mẫu số liệu $ x_1; x_2; \ldots;x_{240} $ là $$ \dfrac{1}{2}\left( x_{120}+x_{121}\right)\in \left[ 15{,}5 ; 20{,}5\right).$$
		Ta xác định được $ n=240$; $ n_m=82 $; $ C=53 $; $ u_m=15{,}5 $; $ u_{m+1}=20{,}5 $.\\
		Trung vị của mẫu số liệu ghép nhóm là $$ M_e=15{,}5+\dfrac{\dfrac{240}{2}-53}{82}\cdot \left( 20{,}5-15{,}5\right)=\dfrac{803}{41}\approx 19{,}59. $$
	}
\end{vd}
\begin{vd}%[CTST]%[1T5K2-1]
	Trong một hội thao, thời gian chạy $200$m của một nhóm các vận động viên được ghi lại ở bảng sau
	\begin{center}
		\begin{tabular}{|c|c|c|c|c|c|}
			\hline 
			\textbf{Thời gian} \textbf{(giây)}& $ \left[21 ; 21{,}5 \right)  $ & $ \left[ 21{,}5 ; 22\right)  $ & $ \left[ 22 ; 22{,}5\right)  $ & $ \left[ 22{,}5 ; 23\right)  $ & $ \left[ 23 ; 23{,}5\right)  $ \\ 
			\hline 
			\textbf{Số vận động viên} & $ 5 $ & $ 12 $ & $ 32 $ & $ 45 $ & $ 30 $ \\ 
			\hline 
		\end{tabular} 
	\end{center}
	Dựa vào bảng số liệu trên, ban tổ chức muốn chọn ra khoảng $ 50 \% $ số vận động viên chạy nhanh nhất để tiếp tục thi vòng $ 2 $. Ban tổ chức nên chọn các vận động viên có thời gian chạy không quá bao nhiêu giây?
	\loigiai{
		Số vận động viên tham gia là $$n=5+12+32+45+30=124.$$
		Gọi $ x_1; x_2; \ldots ; x_{124} $ lần lượt là thời gian chạy $ 200 $ m của $ 124 $ vận động viên được xếp theo thứ tự không giảm.\\
		Do $ x_1, \ldots, x_5 \in \left[ 21 ; 21{,}5 \right)$, $ x_6, \ldots, x_{17} \in \left[ 21{,}5 ; 22\right) $, $ x_{18}, \ldots, x_{49} \in \left[22 ; 22{,}5\right) $, $ x_{50},\ldots, x_{94} \in \left[ 22{,}5 ; 23\right) $ nên trung vị của mẫu số liệu $ x_1; x_2; \ldots ;x_{124} $ là
		$$\dfrac{1}{2}\cdot \left( x_{62}+x_{63}\right) \in  \left[ 22{,}5 ; 23\right).$$
		Ta xác định được $ n=124 $; $ n_m=45$; $ C=5+12+32=49 $; $ u_m= 22{,}5$; $ u_{m+1}=23$.\\
		Trung vị của mẫu số liệu ghép nhóm là $$M_e=22{,}5 +\dfrac{\dfrac{124}{2}-49}{45}\cdot \left( 23-22{,}5\right)= \dfrac{1019}{45}\approx 22{,}64.$$
		Vậy ban tổ chức nên chọn các vận động viên  có thời gian chạy không quá $ 22{,}64$ (giây) được tiếp tục thi vòng hai.
	}
\end{vd}
\subsubsection{Bài tập rèn luyện}
\begin{bt}%[Cánh diều]%[1C5B1-5]
	\immini
	{
		Bảng bên cho ta bảng tần số ghép nhóm số liệu thống kê chiều cao của $40$ mẫu cây ở một vườn thực vật (đơn vị: centimét). Xác định trung vị của mẫu số liệu ghép nhóm trên.
	}
	{
		\begin{tabular}{|c|c|c|}
			\hline
			\textbf{Nhóm} & \textbf{Tần số} & \textbf{Tần số tích luỹ}\\ 
			\hline
			$\left[30;40\right)$ & $4$ & $4$\\
			$\left[40;50\right)$ & $10$ & $14$\\
			$\left[50;60\right)$ & $14$ & $28$\\
			$\left[60;70\right)$ & $6$ & $34$\\
			$\left[70;80\right)$ & $4$ & $38$\\
			$\left[80;90\right)$ & $2$ & $40$\\
			\hline
			& $n = 40$ &\\
			\hline
		\end{tabular}
	}
	\loigiai{
		Ta có $\dfrac{n}{2} = 20$, mà $14<20<28$ nên nhóm $3$ là nhóm đầu tiên có tần số tích luỹ lớn hơn hoặc bằng $20$. \\
		Xét nhóm $3$ là nhóm $\left[50;60\right)$ có $r=50$, $d=10$, $n_3=14$ và nhóm $2$ có $cf_2 = 14$.\\
		Khi đó, tứ phân vị thứ hai (cũng là trung vị) là
		\[
		M_e = 50 + \dfrac{20 - 14}{14} \cdot 10 = 54{,}3\text{ (cm)}.
		\]
	}
\end{bt}
\begin{bt}%[Cánh diều]%[1C5B1-5]
	Mẫu số liệu sau ghi lại cân nặng của $30$ bạn học sinh (đơn vị: kilôgam)
	\[
	\begin{array}{cccccccccc}
		17 & 40 & 39 & 40{,}5 & 42 & 51 & 41{,}5 & 39 & 41 & 30\\
		40 & 42 & 40{,}5 & 39{,}5 & 41 & 40{,}5 & 37 & 39{,}5 & 40 & 41\\
		38{,}5 & 39{,}5 & 40 & 41 & 39 & 40{,}5 & 40 & 38{,}5 & 39{,}5 & 41{,}5
	\end{array}
	\]
	\begin{enumerate}
		\item Lập bảng tần số ghép nhóm cho mẫu số liệu trên có tám nhóm ứng với tám nửa khoảng:
		\[
		[15 ; 20),[20 ; 25),[25 ; 30),[30 ; 35),[35 ; 40),[40 ; 45),[45 ; 50),[50 ; 55).
		\]
		\item Xác định trung vị của mẫu số liệu ghép nhóm trên.
	\end{enumerate}
	\loigiai{
		\begin{enumerate}
			\item Ta có bảng tần số ghép nhóm của mẫu số liệu trên như sau:
			\begin{center}
				\begin{tabular}{|c|c|c|c|}
					\hline
					\textbf{Nhóm} & \textbf{Giá trị đại diện} & \textbf{Tần số} & \textbf{Tần số tích luỹ}\\ 
					\hline
					$\left[15;20\right)$ & $17{,}5$ & $1$ & $1$\\
					$\left[20;25\right)$ & $22{,}5$ & $0$ & $1$\\
					$\left[25;30\right)$ & $27{,}5$ & $0$ & $1$\\
					$\left[30;35\right)$ & $32{,}5$ & $1$ & $2$\\
					$\left[35;40\right)$ & $37{,}5$ & $10$ & $12$\\
					$\left[40;45\right)$ & $42{,}5$ & $17$ & $29$\\
					$\left[45;50\right)$ & $47{,}5$ & $0$ & $29$\\
					$\left[50;55\right)$ & $52{,}5$ & $1$ & $30$\\
					\hline
					&  & $n = 30$ &\\
					\hline
				\end{tabular}
			\end{center}
			\item Ta có $\dfrac{n}{2} = 15$, mà $12<15<29$ nên nhóm $6$ là nhóm đầu tiên có tần số tích luỹ lớn hơn hoặc bằng $15$. \\
			Xét nhóm $6$ là nhóm $\left[40;45\right)$ có $r=40$, $d=5$, $n_6=17$ và nhóm $5$ có $cf_5 = 12$.\\
			Khi đó, trung vị là
			\[
			M_e = 40 + \dfrac{15-12}{17} \cdot 5 = 40{,}9\text{ (kg)}.
			\]
		\end{enumerate}
	}
\end{bt}
\begin{bt}%[CTST]%[1T5G2-2]
	Cân nặng của một số lợn con mới sinh thuộc hai giống $ A $ và $ B $ được cho ở biểu đồ dưới đây (đơn vị: kg).
	\begin{center}
		\begin{tikzpicture}[>=stealth,line join=round,line cap=round,font=\footnotesize,scale=0.85,line width=1pt]
			\draw[->] (0,0)--(0,5)node[left]{(\text{Số con})};
			\foreach \y in {1,2,3,4}
			\draw[shift={(0,\y)}] (0,0)--(-2pt,0) node[left]{\scriptsize ${\y}0$};
			%	\path (4.5,6) node {\normalsize{\textbf{Cân nặng của một số lợn con mới sinh}}};
			\path (4.5,5.5) node {
				$\begin{array}{c}
					\normalsize{\textbf{Cân nặng của một số}}\\
					\normalsize{\textbf{lợn con mới sinh}}
				\end{array}$
			};
			%% nhãn
			\path (2.5,-1.5) node[rectangle,fill=cyan,draw=none]{};
			\path (3.6,-1.5) node {\text{Giống $ A $}};
			\path (5,-1.5) node[rectangle,fill=orange,draw=none]{};
			\path (6.1,-1.5) node {\text{Giống $ B $}};
			% đường gióng
			\foreach \y in {1,2,3,4}{
				\draw[line width=0.2pt] (0,\y)--(8.4,\y);
			}
			%% cột
			\draw[fill=cyan,draw=none] (0,0)--(0,0.8)--(1,0.8)node[midway,above]{$ 8 $}--(1,0)--cycle;
			\draw[fill=orange,draw=none] (1,0)--(1,1.3)--(2,1.3)node[midway,above]{$ 13 $}--(2,0)--cycle;
			\draw[fill=cyan,draw=none] (2,0)--(2,2.8)--(3,2.8)node[midway,above]{$ 28 $}--(3,0)--cycle;
			\draw[fill=orange,draw=none] (3,0)--(3,1.4)--(4,1.4)node[midway,above]{$ 14 $}--(4,0)--cycle;
			\draw[fill=cyan,draw=none] (4,0)--(4,3.2)--(5,3.2)node[midway,above]{$ 32 $}--(5,0)--cycle;
			\draw[fill=orange,draw=none] (5,0)--(5,2.4)--(6,2.4)node[midway,above]{$ 24 $}--(6,0)--cycle;
			\draw[fill=cyan,draw=none] (6,0)--(6,1.7)--(7,1.7)node[midway,above]{$ 17 $}--(7,0)--cycle;
			\draw[fill=orange,draw=none] (7,0)--(7,1.4)--(8,1.4)node[midway,above]{$ 14 $}--(8,0)--cycle;
			%% miền
			\node [below] at (1,0){$ \left[1{,}0 ; 1{,}1 \right)$};
			\node [below] at (3,0){$ \left[1{,}1 ; 1{,}2 \right)$};
			\node [below] at (5,0){$ \left[1{,}2 ; 1{,}3 \right)$};
			\node [below] at (7,0){$ \left[1{,}3 ; 1{,}4 \right)$};
			\draw[->] (0,0)node [below left=-2pt]{$ O $}--(9,0)node[below]{(\text{kg})};
		\end{tikzpicture}
	\end{center}
	Hãy so sánh cân nặng của lợn con mới sinh giống $ A $ và giống $ B $ theo số trung bình và trung vị.
	\loigiai{
		Bảng tần số ghép nhóm thống kê cân nặng của lợn con mới sinh giống $ A $ và giống $ B $ như sau:
		\begin{center}
			\begin{tabular}{|c|c|c|c|c|}
				\hline 
				\textbf{Cân nặng (kg)}	& $ \left[1{,}0 ; 1{,}1 \right)$  &$ \left[1{,}1 ; 1{,}2 \right)$  &$ \left[1{,}2 ; 1{,}3 \right)$  & $ \left[1{,}3 ; 1{,}4 \right)$ \\ 
				\hline 
				\textbf{Giá trị đại diện (kg)}	& $1{,}05 $ & $ 1{,}15 $ & $ 1{,}25 $ & $ 1{,}35 $ \\ 
				\hline 
				\begin{tabular}{c}
					\textbf{Giống A}
					\\ 
					\textbf{(đơn vị: con)}
				\end{tabular} 	& $ 8 $ & $ 28 $ & $ 32 $ & $ 17 $ \\ 
				\hline 
				\begin{tabular}{c}
					\textbf{Giống B}
					\\ 
					\textbf{(đơn vị: con)}
				\end{tabular} 	& $ 13 $ & $ 14 $ & $ 24 $ & $ 14 $ \\ 
				\hline 
			\end{tabular} 
		\end{center}
		Cân nặng trung bình của lợn con mới sinh giống $ A $ là $$ \dfrac{1{,}05\cdot 8 + 1{,}15 \cdot 28 + 1{,}25 \cdot 32 + 1{,}35 \cdot 17}{8+28+32+17}=\dfrac{2071}{1700}\approx 1{,}218.$$
		Cân nặng trung bình của lợn con mới sinh giống $ B $ là $$ \dfrac{1{,}05\cdot 13 + 1{,}15 \cdot 14 + 1{,}25 \cdot 24 + 1{,}35 \cdot 14}{13+14+24+14}=\dfrac{121}{100} \approx 1{,}21.$$
		Suy ra cân nặng trung bình của lợn con mới sinh giống $ A $  lớn hơn cân nặng trung bình của lợn con mới sinh giống $ B $. \\
		Trung vị của mẫu số liệu ghép nhóm cân nặng của lợn con  giống $ A $ là $$M_e=1{,}2+\dfrac{\dfrac{85}{2}-(8+28)}{32}\cdot (1{,}3-1{,}2)=\dfrac{781}{640}\approx 1{,}22.$$
		Trung vị của mẫu số liệu ghép nhóm cân nặng của lợn con  giống  $ B $ là $$M_e=1{,}2+\dfrac{\dfrac{65}{2}-(13+14)}{24}\cdot (1{,}3-1{,}2)=\dfrac{587}{480}\approx 1{,}22.$$
		Suy ra trung vị của mẫu số liệu ghép nhóm cân nặng của của lợn con giống $ A $ bằng trung vị của mẫu số liệu ghép nhóm cân nặng của của lợn con giống $ B $.
	}
\end{bt}

\setcounter{subsubsection}{0}
\setcounter{ex}{0}
\setcounter{bt}{0}
\begin{dang}{Tứ phân vị}
	
\end{dang}
\subsubsection{Ví dụ minh hoạ}
\begin{vd}
	\immini
	{
		Bảng bên cho biết tần số ghép nhóm số liệu thống kê cân nặng của $40$ học sinh lớp $11A$ trong một trường trung học phổ thông (đơn vị: kilôgam). Xác định tứ phân vị của mẫu số liệu ghép nhóm.
	}
	{
		\begin{tabular}{|c|c|c|}
			\hline
			\textbf{Nhóm} & \textbf{Tần số} & \textbf{Tần số tích luỹ}\\ 
			\hline
			$\left[30;40\right)$ & $2$ & $2$\\
			$\left[40;50\right)$ & $10$ & $12$\\
			$\left[50;60\right)$ & $16$ & $28$\\
			$\left[60;70\right)$ & $8$ & $36$\\
			$\left[70;80\right)$ & $2$ & $38$\\
			$\left[80;90\right)$ & $2$ & $40$\\
			\hline
			& $n = 40$ &\\
			\hline
		\end{tabular}
	}
	\loigiai{
		Số phần tử của mẫu là $n=40$.
		\begin{itemize}
			\item Ta có $\dfrac{n}{4} = 10$, mà $2<10<12$ nên nhóm $2$ là nhóm đầu tiên có tần số tích luỹ lớn hơn hoặc bằng $10$. \\
			Xét nhóm $2$ là nhóm $\left[40;50\right)$ có $s=40$, $h=10$, $n_2=10$ và nhóm $1$ có $cf_1 = 2$.\\
			Khi đó, tứ phân vị thứ nhất là
			\[
			Q_1 = 40 + \dfrac{10-2}{10} \cdot 10 = 48\text{ (kg)}.
			\]
			\item Ta có $\dfrac{n}{2} = 20$, mà $12<20<28$ nên nhóm $3$ là nhóm đầu tiên có tần số tích luỹ lớn hơn hoặc bằng $20$. \\
			Xét nhóm $3$ là nhóm $\left[50;60\right)$ có $r=50$, $d=10$, $n_3=16$ và nhóm $2$ có $cf_2 = 12$.\\
			Khi đó, tứ phân vị thứ hai là
			\[
			Q_2 = 50 + \dfrac{20-12}{16} \cdot 10 = 55\text{ (kg)}.
			\]
			\item Ta có $\dfrac{3n}{4} = 30$, mà $28<30<36$ nên nhóm $4$ là nhóm đầu tiên có tần số tích luỹ lớn hơn hoặc bằng $30$. \\
			Xét nhóm $4$ là nhóm $\left[60;70\right)$ có $t=50$, $l=10$, $n_4=8$ và nhóm $3$ có $cf_3 = 28$.\\
			Khi đó, tứ phân vị thứ ba là
			\[
			Q_3 = 60 + \dfrac{30-28}{8} \cdot 10 = 62{,}5\text{ (kg)}.
			\]
		\end{itemize}
		Vậy tứ phân vị của mẫu số liệu trên là $48$, $55$ và $62{,}5$.
	}
\end{vd}
\subsubsection{Bài tập rèn luyện}
\begin{bt}%[1K3B9-3]
	Thời gian (phút) truy cập internet mỗi buổi tối của một số học sinh được cho trong bảng sau:
	\begin{center}
		\begin{tabular}{|c|c|c|c|c|c|c|}
			\hline
			Thời gian (phút)	& $\left[9{,}5;12{,}5 \right)$ & $\left[12{,}5;15{,}5 \right)$ & $\left[15{,}5;18{,}5 \right)$ & $\left[18{,}5;21{,}5 \right)$ & $\left[21{,}5;24{,}5 \right)$ \\
			\hline
			Số học sinh&$3$	& $12$ & $15$ &$24$  & $2$  \\
			\hline
		\end{tabular}	
	\end{center}
	Tìm tứ phân vị thứ nhất $Q_1$ và tứ phân vị thứ ba $Q_3$ của mẫu số liệu ghép nhóm.
	\loigiai{
		Cỡ mẫu là $n=3+12+15+24+2=56$.\\
		Tứ phân vị thứ nhất $Q_1$ là $\dfrac{x_{14}+x_{15}}{2}$. Do $2$ giá trị $x_{28},\,x_{29}$ thuộc nhóm $\left[12{,}5;15{,}5 \right)$ nên nhóm này chứa $Q_1$. Do đó, $p=2$; $a_2=12{,}5$; $m_2=12$; $m_1=3$; $a_3-a_2=3$ và ta có $$Q_1=12{,}5+\dfrac{\dfrac{56}{4}-3}{12}\cdot 3=15{,}25.$$
		Với tứ phân vị thứ ba $Q_3$ là $\dfrac{x_{42}+x_{43}}{2}$. Do $2$ giá trị $x_{42},\,x_{43}$ thuộc nhóm $\left[18{,}5;21{,}5 \right)$ nên nhóm này chứa $Q_3$. Do đó, $p=4$; $a_4=18{,}5$; $m_4=24$; $m_1+m_2+m_3=3+12+15=30$; $a_5-a_4=3$ và ta có $$Q_3=18{,}5+\dfrac{\dfrac{3\cdot 56}{4}-30}{24}\cdot 3=20.$$
	}
\end{bt}
\begin{bt}%[1K3B9-4]
	Điểm thi môn Toán (thang điểm 100, điểm được làm tròn đến 1) của 60 thí sinh được cho trong bảng sau:
	\begin{center}
		\begin{tabular}{|c|c|c|c|c|c|}
			\hline
			Điểm		& $0-9$ & $10-19$ & $20-29$ & $30-39$ & $40-49$ \\
			\hline
			Số thí sinh	& $1$ & $2$ & $4$ & $6$ & $15$ \\
			\hline
			Điểm	& $50-59$ & $60-69$ & $70-79$ & $80-89$ & $90-99$  \\
			\hline
			Số thí sinh	& $12$ & $10$ & $6$ & $3$ & $1$  \\
			\hline
		\end{tabular}
	\end{center}
	\begin{enumerate}
		\item [a)] Hiệu chỉnh để thu được mẫu số liệu ghép nhóm dạng Bảng $3.2$.
		\item [b)] Tìm các tứ phân vị và giải thích ý nghĩa của chúng.
	\end{enumerate}
	\loigiai{
		\begin{enumerate}
			\item [a)] Bảng số liệu ghép nhóm về điểm thi môn Toán của 60 thí sinh
			\begin{center}
				\begin{tabular}{|c|c|c|c|c|c|}
					\hline
					Điểm		& $\left[0;20\right)$ & $\left[20;40\right)$ & $\left[40;60\right)$ & $\left[60;80\right)$ & $\left[80;100\right)$ \\
					\hline
					Số thí sinh	& $3$ & $10$ & $27$ & $16$ & $4$ \\
					\hline
				\end{tabular}
			\end{center}
			
			\item [b)] Cỡ mẫu $n=60$. Gọi $x_1$, $x_2$,$\ldots$, $x_{60}$ là điểm thi môn Toán của 60 học sinh và giả sử dãy này đã được sắp xếp theo thứ tự tăng dần. Khi đó, trung vị là $\dfrac{x_{30}+x_{31}}{2}$.\\
			Do hai giá trị $x_{30}$, $x_{31}$ thuộc nhóm $\left[40;60\right)$ nên nhóm này chứa trung vị. Do đó, $p=3;a_3=40;m_3=27;m_1+m_2=13;a_4-a_3=20$ và ta có
			$$Q_2=M_e=40+\dfrac{\dfrac{60}{2}-13}{27}\cdot 20\approx 52{,}6.$$
			Tứ phân vị thứ nhất $Q_1=\dfrac{x_{15}+x_{16}}{2}$. Do hai giá trị $x_{15}$, $x_{16}$ thuộc nhóm $\left[40;60\right)$ nên nhóm này chứa $Q_1$. Do đó, $p=3;\,a_3=40;\,m_3=27;\,m_1+m_2=13;\,a_4-a_3=20$ và ta có
			$$Q_1=40+\dfrac{\dfrac{60}{4}-13}{27}\cdot 20\approx 41{,}5.$$
			Tứ phân vị thứ ba $Q_3=\dfrac{x_{45}+x_{46}}{2}$. Do hai giá trị $x_{45}$, $x_{46}$ thuộc nhóm $\left[60;80\right)$ nên nhóm này chứa $Q_3$. Do đó, $p=4;\,a_4=60;\,m_4=16;\,m_1+m_2+m_3=40;\,a_5-a_4=20$ và ta có
			$$Q_3=60+\dfrac{\dfrac{3\cdot 60}{4}-40}{16}\cdot 20\approx 66{,}3.$$
			Khoảng cách từ $Q_1$ đến $Q_2$ là $11{,}1$ còn khoảng cách từ $Q_2$ và $Q_3$ là $13{,}7$. Điều này cho thấy mẫu số liệu tập trung với mật độ cao hơn ở bên trái $Q_2$ và mật độ thấp hơn ở bên phải $Q_2$.
		\end{enumerate}	
	}
\end{bt}
\begin{bt}%[1K3B9-4]
	\immini{
		Phỏng vấn một số học sinh lớp $11$ về thời gian (giờ) ngủ của một buổi tối, thu được bảng số liệu ở bên.
		\begin{enumerate}
			\item [a)] So sánh thời gian ngủ trung bình của các bạn học sinh nam và nữ.
			\item [b)] Hãy cho biết $75\%$ học sinh khối $11$ ngủ ít nhất bao nhiêu giờ?
		\end{enumerate}
	}
	{
		\begin{tabular}{|c|c|c|}
			\hline
			Thời gian	& Số học sinh nam & Số học sinh nữ\\
			\hline
			$\left[4;5 \right)$	& $6$ & $4$ \\
			\hline
			$\left[5;6 \right)$	& $10$ & $8$ \\
			\hline
			$\left[6;7 \right)$	& $13$ & $10$ \\
			\hline
			$\left[7;8 \right)$	& $9$ & $11$ \\
			\hline
			$\left[8;9 \right)$	& $7$ & $8$ \\
			\hline
		\end{tabular}
	}
	\loigiai{
		\begin{enumerate}
			\item [a)] Trong mỗi khoảng thời gian, giá trị đại diện là trung bình cộng của giá trị hai đầu mút nên ta có bảng sau:
			\begin{center}
				\begin{tabular}{|c|c|c|}
					\hline
					Thời gian	& Số học sinh nam & Số học sinh nữ\\
					\hline
					$4{,}5$	& $6$ & $4$ \\
					\hline
					$5{,}5$	& $10$ & $8$ \\
					\hline
					$6{,}5$	& $13$ & $10$ \\
					\hline
					$7{,}5$	& $9$ & $11$ \\
					\hline
					$8{,}5$	& $7$ & $8$ \\
					\hline
				\end{tabular}	
			\end{center}
			Tổng số học sinh nam là $n_1=6+10+13+9+7=45$.\\ Thời gian ngủ trung bình của học sinh nam là:
			$$\overline{x_1}=\dfrac{4{,}5\cdot 6+5{,}5\cdot10+6{,}5\cdot13+7{,}5\cdot9+8{,}5\cdot7}{45}=\dfrac{587}{90}\approx 6{,}52\,\, \text{(giờ)}.$$
			Tổng số học sinh nữ là $n_2=4+8+10+11+8=41$. Thời gian ngủ trung bình của học sinh nữ là:
			$$\overline{x_2}=\dfrac{4,5\cdot4+5,5\cdot8+6,5\cdot10+7,5\cdot11+8,5\cdot8}{41}=\dfrac{555}{82}\approx 6{,}77 \,\,\text{(giờ)}.$$
			Vì $\overline{x_2}>\overline{x_1}$ nên thời gian ngủ trung bình của các bạn học sinh nữ lớn hơn thời gian ngủ trung bình của các bạn nam.
			\item [b)] Tổng số học sinh được điều tra là $n=n_1+n_2=45+41=86$.\\
			Giả sử $x_1;x_2;x_3;\cdot \cdot;x_{86}$ là dãy giá trị được sắp xếp theo thứ tự không giảm.\\
			Ta có bảng sau:
			\begin{center}
				\begin{tabular}{|c|c|c|}
					\hline
					Thời gian	& Số học sinh \\
					\hline
					$\left[4;5 \right)$	& $10$  \\
					\hline
					$\left[5;6 \right)$	& $18$  \\
					\hline
					$\left[6;7 \right)$	& $23$  \\
					\hline
					$\left[7;8 \right)$	& $20$  \\
					\hline
					$\left[8;9 \right)$	& $15$  \\
					\hline
				\end{tabular}
			\end{center}
			Tứ phân vị thứ nhất $Q_1$ là $x_{22}$. Do $x_{22}$ thuộc nhóm $\left[5;6\right)$ nên nhóm này chứa $Q_1$.\\ Do đó, $p=2;\,a_2=5;\,m_2=18;\,m_1=10;\,a_3-a_2=1$ và ta có
			$$Q_1=5+\dfrac{\dfrac{86}{4}-10}{18}\cdot 1=\dfrac{203}{36}\approx 5{,}64 \text{(giờ)}.$$
			Nghĩa là có $25\%$ học sinh khối $11$ ngủ ít hơn $5{,}64$ giờ.\\ Vậy $75\%$ học sinh khối $11$ ngủ ít nhất $5{,}64$ giờ.
		\end{enumerate}
	}
\end{bt}
\setcounter{subsubsection}{0}
\setcounter{ex}{0}
\setcounter{bt}{0}
\begin{dang}{Mốt}
	
\end{dang}
\subsubsection{Ví dụ minh hoạ}
\begin{vd}%[Tex hóa SGK CD, TVN-223]%[1C5B1-5]
	Kết quả kiểm tra môn Toán của lớp $11D$ như sau
	\[
	\begin{array}{cccccccccccccccccccc}
	5 & 6 & 7 & 5 & 6 & 9 & 10 & 8 & 5 & 5 & 4 & 5 & 4 & 5 & 7 & 4 & 5 & 8 & 9 & 10 \\
	5 & 3 & 5 & 6 & 5 & 7 & 5 & 8 & 4 & 9 & 5 & 6 & 5 & 6 & 8 & 8 & 7 & 9 & 7 & 9
	\end{array}
	\]
	\begin{enumerate}
		\item Lập bảng tần số ghép nhóm của mẫu số liệu trên có bốn nhóm ứng với bốn nửa khoảng $\left[3;5\right)$, $\left[5;7\right)$, $\left[7;9\right)$, $\left[9;11\right)$.
		\item Mốt của bảng số liệu ghép nhóm trên là bao nhiêu (làm tròn kết quả đến hàng phần mười)?
	\end{enumerate}
	\loigiai{
		\immini
		{
			\begin{enumerate}
				\item Bảng bên là bảng tần số ghép nhóm cho kết quả kiểm tra môn Toán của lớp $11D$.
				\item Ta thấy: Nhóm $2$ ứng với nửa khoảng $\left[5;7\right)$ là nhóm có tần số lớn nhất với $u=5$, $g=2$, $n_2 = 18$. Nhóm $1$ có tần số $n_1 = 5$, nhóm $3$ có tần số $n_3=10$.\\
				Khi đó, mốt của mẫu số liệu là 
				\[
				M_o = 5 + \left( \dfrac{18- 5}{2\cdot 18 - 5 - 10} \right) \cdot 2 \approx 6{,}2.
				\]
			\end{enumerate}
		}
		{
			\begin{tabular}{|c|c|}
				\hline
				\textbf{Nhóm} & \textbf{Tần số}\\ 
				\hline
				$\left[3;5\right)$ & $5$\\
				\hline
				$\left[5;7\right)$ & $18$\\
				\hline
				$\left[7;9\right)$ & $10$\\
				\hline
				$\left[9;11\right)$ & $7$\\
				\hline
				& $n = 40$ \\
				\hline
			\end{tabular}
		}
	}
\end{vd}
\subsubsection{Bài tập rèn luyện}
\begin{bt}%[Tex hóa SGK CD, TVN-223]%[1C5B1-5]
	Mẫu số liệu dưới đây ghi lại tốc độ của $40$ ô tô khi đi qua một trạm đo tốc độ (đơn vị: km/h):
	\[
	\begin{array}{cccccccccc}
	48{,}5 & 43 & 50 & 55 & 45 & 60 & 53 & 55,5 & 44 & 65 \\
	51 & 62,5 & 41 & 44,5 & 57 & 57 & 68 & 49 & 46{,}5 & 53{,}5 \\
	61 & 49{,}5 & 54 & 62 & 59 & 56 & 47 & 50 & 60 & 61 \\
	49{,}5 & 52{,}5 & 57 & 47 & 60 & 55 & 45 & 47,5 & 48 & 61{,}5
	\end{array}
	\]
	\begin{enumerate}
		\item Lập bảng tần số ghép nhóm cho mẫu số liệu trên có sáu nhóm ứng với sáu nửa khoảng:
		\[
		[40 ; 45),[45 ; 50),[50 ; 55),[55 ; 60),[60 ; 65),[65 ; 70).
		\]
		\item Mốt của mẫu số liệu ghép nhóm trên là bao nhiêu?
	\end{enumerate}
	\loigiai{
		\begin{enumerate}
			\item Ta có bảng tần số ghép nhóm của mẫu số liệu trên như sau:
			\begin{center}
				\begin{tabular}{|c|c|c|c|}
					\hline
					\textbf{Nhóm} & \textbf{Giá trị đại diện} & \textbf{Tần số} & \textbf{Tần số tích luỹ}\\ 
					\hline
					$\left[40;45\right)$ & $42{,}5$ & $4$ & $4$\\
					$\left[45;50\right)$ & $47{,}5$ & $11$ & $15$\\
					$\left[50;55\right)$ & $52{,}5$ & $7$ & $22$\\
					$\left[55;60\right)$ & $57{,}5$ & $8$ & $30$\\
					$\left[60;65\right)$ & $62{,}5$ & $8$ & $38$\\
					$\left[65;70\right)$ & $67{,}5$ & $2$ & $40$\\
					\hline
					&  & $n = 40$ &\\
					\hline
				\end{tabular}
			\end{center}
			
			\item Ta thấy: Nhóm $2$ ứng với nửa khoảng $\left[45;50\right)$ là nhóm có tần số lớn nhất với $u=45$, $g=5$, $n_2 = 11$. Nhóm $1$ có tần số $n_1 = 4$, nhóm $3$ có tần số $n_3 = 7$.\\
			Khi đó, mốt của mẫu số liệu là 
			\[
			M_o = 45 + \left( \dfrac{11 - 4}{2\cdot 11 - 4 - 7} \right) \cdot 5 \approx 48{,}2\text{ (km/h)}.
			\]
		\end{enumerate}
	}
\end{bt}
\begin{bt}%[Tex hóa SGK CD, TVN-223]%[1C5B1-5]
	Mẫu số liệu sau ghi lại cân nặng của $30$ bạn học sinh (đơn vị: kilôgam):
	\[
	\begin{array}{cccccccccc}
	17 & 40 & 39 & 40{,}5 & 42 & 51 & 41{,}5 & 39 & 41 & 30\\
	40 & 42 & 40{,}5 & 39{,}5 & 41 & 40{,}5 & 37 & 39{,}5 & 40 & 41\\
	38{,}5 & 39{,}5 & 40 & 41 & 39 & 40{,}5 & 40 & 38{,}5 & 39{,}5 & 41{,}5
	\end{array}
	\]
	\begin{enumerate}
		\item Lập bảng tần số ghép nhóm cho mẫu số liệu trên có tám nhóm ứng với tám nửa khoảng:
		\[
		[15 ; 20),[20 ; 25),[25 ; 30),[30 ; 35),[35 ; 40),[40 ; 45),[45 ; 50),[50 ; 55).
		\]
		
		\item Mốt của mẫu số liệu ghép nhóm trên là bao nhiêu?
	\end{enumerate}
	\loigiai{
		\begin{enumerate}
			\item Ta có bảng tần số ghép nhóm của mẫu số liệu trên như sau:
			\begin{center}
				\begin{tabular}{|c|c|c|c|}
					\hline
					\textbf{Nhóm} & \textbf{Giá trị đại diện} & \textbf{Tần số} & \textbf{Tần số tích luỹ}\\ 
					\hline
					$\left[15;20\right)$ & $17{,}5$ & $1$ & $1$\\
					$\left[20;25\right)$ & $22{,}5$ & $0$ & $1$\\
					$\left[25;30\right)$ & $27{,}5$ & $0$ & $1$\\
					$\left[30;35\right)$ & $32{,}5$ & $1$ & $2$\\
					$\left[35;40\right)$ & $37{,}5$ & $10$ & $12$\\
					$\left[40;45\right)$ & $42{,}5$ & $17$ & $29$\\
					$\left[45;50\right)$ & $47{,}5$ & $0$ & $29$\\
					$\left[50;55\right)$ & $52{,}5$ & $1$ & $30$\\
					\hline
					&  & $n = 30$ &\\
					\hline
				\end{tabular}
			\end{center}
		
			\item Ta thấy: Nhóm $6$ ứng với nửa khoảng $\left[40;45\right)$ là nhóm có tần số lớn nhất với $u=40$, $g=5$, $n_6 = 17$. Nhóm $5$ có tần số $n_5 = 10$, nhóm $7$ có tần số $n_7 = 0$.\\
			Khi đó, mốt của mẫu số liệu là 
			\[
			M_o = 40 + \left( \dfrac{17 - 10}{2\cdot 17 - 10 - 0} \right) \cdot 5 \approx 41{,}5\text{ (kg)}.
			\]
		\end{enumerate}
	}
\end{bt}
\begin{bt}%[Tex hóa SGK CD, TVN-223]%[1C5B1-5]
	\immini
	{
		Bảng bên cho ta bảng tần số ghép nhóm số liệu thống kê chiều cao của $40$ mẫu cây ở một vườn thực vật (đơn vị: centimét).
		
		
			 Mốt của mẫu số liệu ghép nhóm trên là bao nhiêu?
		
	}
	{
		\begin{tabular}{|c|c|c|}
			\hline
			\textbf{Nhóm} & \textbf{Tần số} & \textbf{Tần số tích luỹ}\\ 
			\hline
			$\left[30;40\right)$ & $4$ & $4$\\
			$\left[40;50\right)$ & $10$ & $14$\\
			$\left[50;60\right)$ & $14$ & $28$\\
			$\left[60;70\right)$ & $6$ & $34$\\
			$\left[70;80\right)$ & $4$ & $38$\\
			$\left[80;90\right)$ & $2$ & $40$\\
			\hline
			& $n = 40$ &\\
			\hline
		\end{tabular}
	}
	\loigiai{
	Ta thấy: Nhóm $3$ ứng với nửa khoảng $\left[50;60\right)$ là nhóm có tần số lớn nhất với $u=50$, $g=10$, $n_3 = 14$. Nhóm $2$ có tần số $n_2 = 10$, nhóm $4$ có tần số $n_4 = 6$.\\
			Khi đó, mốt của mẫu số liệu là 
			\[
			M_o = 50 + \left( \dfrac{14 - 10}{2\cdot 14 - 10 - 6} \right) \cdot 10 \approx 53{,}3\text{ (cm)}.
			\]
	
	}
\end{bt}
\subsection{Bài tập trắc nghiệm}
\Opensolutionfile{ans}[ans/ansOC3]
% \begin{ex}%[1C5Y1-1]
% 	Một cuộc khảo sát đã tiến hành xác định tuổi (theo năm) của $120$ chiếc ô-tô. Kết quả điều tra được cho trong bảng sau
% 	\begin{center}
% 		\begin{tabular}{ |c|c|c|c|c|c|c| }
% 			\hline
% 			Nhóm & $[0;4)$ & $[4;8)$ & $[8;12)$ & $[12;16)$ & $[16;20)$ &  \\
% 			\hline
% 			Tần số & $23$ & $25$ & $27$ & $26$ & $19$ & $n=120$ \\
% 			\hline
% 		\end{tabular}
% 	\end{center}
% 	Mẫu số liệu trên có bao nhiêu nhóm?
% 	\choice
% 	{$10$}
% 	{$11$}
% 	{\True $5$}
% 	{$7$}
% 	\loigiai{
% 		Từ bảng, ta thấy mẫu số liệu trên có $5$ nhóm.
% 	}
% \end{ex}
% \begin{ex}%[1C5Y1-1]
% 	Một cuộc khảo sát đã tiến hành xác định tuổi (theo năm) của $120$ chiếc ô-tô. Kết quả điều tra được cho trong bảng sau
% 	\begin{center}
% 		\begin{tabular}{ |c|c|c|c|c|c|c| }
% 			\hline
% 			Nhóm & $[0;4)$ & $[4;8)$ & $[8;12)$ & $[12;16)$ & $[16;20)$ &  \\
% 			\hline
% 			Tần số & $23$ & $25$ & $27$ & $26$ & $19$ & $n=120$ \\
% 			\hline
% 		\end{tabular}
% 	\end{center}
% 	Nhóm có tần số bằng $19$ là
% 	\choice
% 	{$[0;4)$}
% 	{$[8;12)$}
% 	{$[12;16)$}
% 	{\True $[16;20)$}
% 	\loigiai{
% 		Từ bảng, ta thấy nhóm có tần số bằng $19$ là $[16;20)$.
% 	}
% \end{ex}
\begin{ex}%[1C5Y1-1]
	Một cuộc khảo sát đã tiến hành xác định tuổi (theo năm) của $120$ chiếc ô-tô. Kết quả điều tra được cho trong bảng sau
	\begin{center}
		\begin{tabular}{ |c|c|c|c|c|c|c| }
			\hline
			Nhóm & $[0;4)$ & $[4;8)$ & $[8;12)$ & $[12;16)$ & $[16;20)$ &  \\
			\hline
			Tần số & $23$ & $25$ & $27$ & $26$ & $19$ & $n=120$ \\
			\hline
		\end{tabular}
	\end{center}
	Số ô-tô có độ tuổi dưới $12$ là
	\choice
	{\True $75$}
	{$27$}
	{$48$}
	{$26$}
	\loigiai{
		Từ bảng, ta thấy số ô-tô có độ tuổi dưới $12$ là $23+25+27=75$.
	}
\end{ex}
% \begin{ex}%[1C5Y1-1]
% 	Cho mẫu số liệu ghép nhóm sau
% 	\begin{center}
% 		\begin{tabular}{ |c|c|c|c|c|c|c|c| }
% 			\hline
% 			Thời gian & $[15;20)$ & $[20;25)$ & $[25;30)$ & $[30;35)$ & $[35;40)$ & $[40;45)$ & $[45;50)$ \\
% 			\hline
% 			Số nhân viên & $6$ & $14$ & $25$ & $37$ & $21$ & $13$ & $9$ \\
% 			\hline
% 		\end{tabular}
% 	\end{center}
% 	Tần số của nhóm $[15;20)$ là bao nhiêu?
% 	\choice
% 	{\True $6$}
% 	{$7$}
% 	{$14$}
% 	{$25$}
% 	\loigiai{
% 		Ta thấy tần số của nhóm $[15;20)$ là $6$.
% 	}
% \end{ex}
\begin{ex}%[1C5Y1-1]
	Khảo sát thời gian tập thể dục trong ngày của một số học sinh khối $11$ thu được mẫu số liệu ghép nhóm sau
	\begin{center}
		\begin{tabular}{ |c|c|c|c|c|c| }
			\hline
			Thời gian (phút)& $[0;20)$ & $[20;40)$ & $[40;60)$ & $[60;80)$ & $[80;100)$ \\
			\hline
			Số học sinh & $5$ & $9$ & $12$ & $10$ & $6$ \\
			\hline
		\end{tabular}
	\end{center}
	Giá trị đại diện của nhóm $[20;40)$ là
	\choice
	{$10$}
	{\True $30$}
	{$20$}
	{$40$}
	\loigiai{
		Giá trị đại diện của nhóm $[20;40)$ là $\dfrac{20+40}{2}=30$.
	}
\end{ex}
\begin{ex}%[1C5Y1-1]
	Doanh thu bán hàng trong $20$ ngày được lựa chọn ngẫu nhiên của một cửa hàng được ghi lại ở bảng sau (đơn vị: triệu đồng):
	\begin{center}
		\begin{tabular}{ |c|c|c|c|c|c| }
			\hline
			Doanh thu & $[5;7)$ & $[7;9)$ & $[9;11)$ & $[11;13)$ & $[13;15)$ \\
			\hline
			Số ngày & $2$ & $7$ & $7$ & $3$ & $1$ \\
			\hline
		\end{tabular}
	\end{center}
	Doanh thu bán hàng của cửa hàng trong ngày $A$ là $7$ triệu đồng thì được xếp vào nhóm nào?
	\choice
	{$[5;7)$}
	{\True $[7;9)$}
	{$[9;11)$}
	{$[13;15)$}
	\loigiai{
		Doanh thu bán hàng là $7$ triệu đồng thì được xếp vào nhóm $[7;9)$.
	}
\end{ex}
\begin{ex}%[1C5Y1-1]
	Doanh thu bán hàng trong $20$ ngày được lựa chọn ngẫu nhiên của một cửa hàng được ghi lại ở bảng sau (đơn vị: triệu đồng):
	\begin{center}
		\begin{tabular}{ |c|c|c|c|c|c| }
			\hline
			Doanh thu & $[5;7)$ & $[7;9)$ & $[9;11)$ & $[11;13)$ & $[13;15)$ \\
			\hline
			Số ngày & $2$ & $7$ & $7$ & $3$ & $1$ \\
			\hline
		\end{tabular}
	\end{center}
	Các nhóm có độ dài bằng
	\choice
	{\True $2$}
	{$3$}
	{$4$}
	{$5$}
	\loigiai{
		Các nhóm có độ dài bằng nhau, và bằng $2$.
	}
\end{ex}
\begin{ex}%[1C5B1-1]
	Cho bảng số liệu về khối lượng của $30$ củ khoai tây thu hoạch từ một thửa ruộng như hình bên dưới. Tần suất của lớp $[100;110)$ là bao nhiêu?
	\begin{center}
		\begin{tabular}{ |c|c|c|c|c|c| }
			\hline
			Lớp khối lượng (gam) & $[70;80)$ & $[80;90)$ & $[90;100)$ & $[100;110)$ & $[110;120]$ \\
			\hline
			Tần số & $3$ & $6$ & $12$ & $6$ & $3$ \\
			\hline
		\end{tabular}
	\end{center}
	\choice
	{\True $20\%$}
	{$10\%$}
	{$40\%$}
	{$90\%$}
	\loigiai{
		Tần suất ghép lớp $[100;110)$ là $\dfrac{6}{30}\cdot 100\%=20\%$.
	}
\end{ex}
\begin{ex}%[1C5B1-1]
	Cân nặng của $28$ học sinh nam lớp $11$ được cho ở bảng sau
	\begin{center}
		\begin{tabular}{ |c|c|c|c|c|c|c| }
			\hline
			Cân nặng & $[45;49)$ & $[49;53)$ & $[53;57)$ & $[57;61)$ & $[61;65)$\\
			\hline
			Số học sinh & $4$ & $5$ & $7$ & $7$ & $5$ \\
			\hline
		\end{tabular}
	\end{center}
	Tấn số tích lũy của nhóm $[49;53)$ là bao nhiêu?
	\choice
	{$5$}
	{$4$}
	{\True $9$}
	{$20$}
	\loigiai{
		Tần số tích lũy của nhóm $[49;53)$ là $4+5=9$.
	}
\end{ex}
% \begin{ex}%[1C5K1-1]
% 	Một thư viện thống kê sô người đến đọc sách vào buổi tối trong $30$ ngày của tháng vừa qua như sau
% 	\begin{center}
% 		\begin{tabular}{ cccccccccc }
% 			$26$ & $35$ & $68$ & $84$ & $33$ & $84$ & $62$ & $45$ & $57$ & $46$ \\
% 			$35$ & $29$ & $28$ & $50$ & $26$ & $34$ & $75$ & $74$ & $43$ & $49$ \\
% 			$54$ & $55$ & $83$ & $82$ & $81$ & $54$ & $27$ & $36$ & $41$ & $52$ \\
% 		\end{tabular}
% 	\end{center}
% 	Bạn An lập bảng tần số mẫu số liệu trên như sau
% 	\begin{center}
% 		\begin{tabular}{ |c|c|c|c|c|c|c|c| }
% 			\hline
% 			Nhóm & $[25;35)$ & $[35;45)$ & $[45;55)$ & $[55;65)$ & $[65;75]$ & $[75;85)$ & \\
% 			\hline
% 			Tần số & $7$ & $4$ & $7$ & $3$ & $3$ & $6$ & $n=30$ \\
% 			\hline
% 		\end{tabular}
% 	\end{center}
% 	Bạn Khuê lập bảng tần số mẫu số liệu trên như sau
% 	\begin{center}
% 		\begin{tabular}{ |c|c|c|c|c|c|c|c|c| }
% 			\hline
% 			Nhóm & $[23;31)$ & $[31;39)$ & $[39;47)$ & $[47;55)$ & $[55;63]$ & $[71;79)$ & $[79;87)$ & \\
% 			\hline
% 			Tần số & $5$ & $5$ & $4$ & $5$ & $3$ & $1$ & $2$ & $n=30$ \\
% 			\hline
% 		\end{tabular}
% 	\end{center}
% 	Hỏi bảng tần số của bạn nào đúng?
% 	\choice
% 	{Bảng tần số của bạn An}
% 	{\True Bảng tần số của bạn Khuê}
% 	{Cả hai bạn đều đúng}
% 	{Cả hai bạn đều sai}
% 	\loigiai{
% 		Từ các bảng tần số mẫu số liệu trên, ta thấy bảng tần số của bạn An sai ở hai nhóm $[35;45)$ (tần số đúng bằng $5$) và nhóm $[65;75)$ (tần số đúng bằng $2$).
% 	}
% \end{ex}
% \begin{ex}%[0D5Y3-1]
% 	Cho dãy số liệu thống kê: $21$, $23$, $24$, $25$, $22$, $20$. Số trung bình cộng của các số liệu thống kê đã cho là
% 	\choice
% 	{$23{,}5$}
% 	{$22$}
% 	{\True $22{,}5$}
% 	{$14$}
% 	\loigiai{
% 		Ta có $\overline{x}=\dfrac{21+23+24+25+22+20}{6}=22{,}5$.
% 	}
% \end{ex}
% \begin{ex}%[0D5Y3-1]
% 	Điều tra về số con của $30$ gia đình ở khu vực, kết quả thu được như sau
% 	\begin{center}
% 		\begin{tabular}{|c|c|c|c|c|c|c|}
% 			\hline 
% 			Giá trị (số con) & $0$ &$1$ & $2$ &$3$ & $4$ & Tổng \\ 
% 			\hline 
% 			Tần số & $1$ & $7$ & $15$ & $5$ & $2$ & $N=30$ \\ 
% 			\hline 
% 		\end{tabular}
% 	\end{center}
% 	Tìm số trung bình $\overline{x}$ của mẫu số liệu trên.
% 	\choice
% 	{\True $\overline{x}=2$}
% 	{$\overline{x}=1$}
% 	{$\overline{x}=1{,}5$}
% 	{$\overline{x}=3$}
% 	\loigiai{
% 		Ta có $\overline{x}=\dfrac{0\cdot 1+1\cdot 7+2\cdot 15+3\cdot 5+4\cdot 2}{30}=2$.
% 	}
% \end{ex}
% \begin{ex}%[0D5Y3-1]
% 	Điểm môn Toán của lớp $11A$ được cho trong bảng sau
% 	\begin{center}
% 		\begin{tabular}{|l|c|c|c|c|c|c|c|c|c|c|}
% 			\hline
% 			Điểm&$1$&$2$&$3$&$4$&$5$&$6$&$7$&$8$&$9$&$10$\\
% 			\hline
% 			Tần số&$2$&$1$&$4$&$3$&$9$&$7$&$5$&$5$&$3$&$1$\\
% 			\hline
% 		\end{tabular}
% 	\end{center}
% 	Điểm trung bình của các học sinh lớp $10A$ là bao nhiêu?
% 	\choice
% 	{$5$}
% 	{$5{,}5$}
% 	{$5{,}6$}
% 	{\True $5{,}7$}
% 	\loigiai{
% 		Điểm trung bình lớp $11A$ là
% 		$$\overline{x}= \dfrac{1 \cdot 2 + 2 \cdot 1 + \cdots + 10 \cdot 1}{40}= \dfrac{227}{40} \approx 5{,}7.$$
% 	}
% \end{ex}
% \begin{ex}%[0D5Y3-1]
% 	Kết quả điểm kiểm tra môn Toán của $40$ học sinh lớp $10A$ được trình bày ở bảng sau:
% 	\begin{center}
% 		\begin{tabular}{|c|c|c|c|c|c|c|c|c|}
% 			\hline 
% 			Điểm&$4$  &$5$  &$6$  &$7$  &$8$  &$9$  &$10$  &Cộng  \\ 
% 			\hline 
% 			Tần số&$2$  &$8$  &$7$  &$10$  &$8$  &$3$  &$2$  &$40$  \\ 
% 			\hline 
% 		\end{tabular} 
% 	\end{center}
% 	Tính số trung bình cộng của bảng trên. (làm tròn kết quả đến một chữ số thập phân).
% 	\choice
% 	{\True $6{,}8$}
% 	{$6{,}4$}
% 	{$7{,}0$}
% 	{$6{,}7$}
% 	\loigiai{
% 		Ta có $\overline{x}=\dfrac{4\cdot 2+ 5\cdot 8+ 6\cdot 7+ 7\cdot 10+ 8\cdot 8+ 9\cdot 3+ 10\cdot 2}{40}\approx 6{,}8$.
% 	}
% \end{ex}
\begin{ex}%[0D5B3-1]
	Điểm môn Toán của lớp $10$A được cho như bảng sau
	\begin{center}
		\begin{tabular}{|c|c|c|c|c|c|}
			\hline
			Điểm &$[0;2)$& $[2;4)$& $[4;6)$& $[6;8)$& $[8;10)$\\\hline
			Tần số& $3$& $5$& $12$& $12$& 8\\ \hline
		\end{tabular}
	\end{center}
	Điểm trung bình của các học sinh lớp $10$A là bao nhiêu?
	\choice
	{$5$}
	{\True $5{,}85$}
	{$5{,}65$}
	{$5{,}45$}	
	\loigiai{
		Điểm trung bình $\overline{x}=\dfrac{1\cdot 3+3\cdot 5+5\cdot 12+7\cdot 12+9\cdot 8}{40}=5{,}85$.		
	}
\end{ex}
\begin{ex}%[0D5B3-1]
	Cho bảng phân bố tần số ghép lớp
	\begin{center}
		\begin{tabular}{|l|c|c|c|c|}
			\hline
			{Lớp các giá trị $x$}&{[8; 10)}&{[10; 12)}&{[12; 14]}&{Cộng}\\
			\hline
			{Tần số $n_i$}&{15}&{30}&{55}&{100}\\
			\hline
		\end{tabular}	
	\end{center}
	Số trung bình của các giá trị trong bảng trên là
	\choice
	{$9$}
	{$13$}
	{$11$}
	{\True $11{,}8$}	
	\loigiai{
		Giá trị đại diện của lớp $\left[8; 10\right)$: $c_1=\dfrac{8 + 10}{2}=9$.\\ 
		Giá trị đại diện của lớp $\left[10; 12\right)$: $c_2=\dfrac{10 + 12}{2}=11$.\\ 
		Giá trị đại diện của lớp $\left[12; 14\right)$: $c_3=\dfrac{12 + 14}{2}=13$.\\ 
		Vậy số trung bình cộng $\overline{x}=\dfrac{9\cdot 15 + 11\cdot 30 + 13\cdot 55}{15 + 30 + 55}=\dfrac{59}{5}$.
	}
\end{ex}
\begin{ex}%[0D5B3-1]
	Kết quả khảo sát cân nặng của $25$ quả cam ở lô hàng $A$ được cho như sau
	\begin{center}
		\begin{tabular}{|l|c|c|c|c|c|}
			\hline
			Cân nặng (g) & $[150;155)$ & $[155;160)$ & $[160;165)$ & $[165;170)$ & $[170;175)$ \\
			\hline
			Số quả cam & $2$ & $6$ & $12$ & $4$ & $1$ \\
			\hline
		\end{tabular}	
	\end{center}
	Tính cân nặng trung bình của mỗi quả cam ở lô hàng $A$.
	\choice
	{\True $161{,}7$ (g)}
	{$161{,}7$ (kg)}
	{$155$ (g)}
	{$160$ (kg)}
	\loigiai{
		Ta có giá trị đại diện của các nhóm lần lượt là $152{,}5$; $157{,}5$; $162{,}5$; $167{,}5$; $172{,}5$.\\
		Vậy cân nặng trung bình của mỗi quả cam là
		$$\overline{x}=\dfrac{152{,}5\cdot 2+ 157{,}5\cdot 6+162{,}5\cdot 12+167{,}5\cdot 4+172{,}5\cdot 1}{25}=161{,}7 \text{(g).}$$
	}
\end{ex}
% \begin{ex}%[0D5K3-1]
% 	Ba nhóm học sinh gồm $10$ người, $15$ người, $25$ người. Khối lượng trung bình của mỗi nhóm lần lượt là $50$ kg, $38$ kg, $40$ kg. Khối lượng trung bình của cả ba nhóm học sinh là
% 	\choice
% 	{\True $41{,}4$ kg}
% 	{$42{,}4$ kg}
% 	{$26$ kg}
% 	{$37$ kg}
% 	\loigiai{
% 		Tổng khối lượng nhóm thứ nhất là $50\cdot 10=500$ (kg).\\
% 		Tổng khối lượng nhóm thứ hai là $38\cdot 15=570$ (kg).\\
% 		Tổng khối lượng nhóm thứ ba là $40\cdot 25=1000$ (kg).\\
% 		Tổng khối lượng cả ba nhóm là $500+570+1000=2070$ (kg).\\
% 		Tổng số người cả ba nhóm là $10+15+25=50$ (người).\\
% 		Khối lượng trung bình của cả ba nhóm học sinh là $\dfrac{2070}{50}=41{,}4$ (kg).
% 	}
% \end{ex}
\begin{ex}%[0D5K3-1]
	Sau một kì thi học sinh giỏi Toán, người ta thống kê kết quả (thang điểm $20$) và thu được bảng tần số sau
	\begin{center}
		\begin{tabular}{|l|c|c|c|c|}
			\hline
			Lớp điểm & $[6;10]$ & $[11;15]$ & $[16;20]$ & Cộng\\
			\hline
			Tần số & $22$ & $12$ & $6$ & $40$ \\
			\hline
		\end{tabular}
	\end{center}
	Nếu những học sinh chỉ cần đạt điểm trung bình của bảng điểm trên đều được nhận Giấy Khen của ban tổ chức, thì số học sinh được nhận Giấy Khen là bao nhiêu?
	\choice
	{$11$}
	{\True $18$}
	{$12$}
	{$6$}
	\loigiai{
		Ta lập lại bảng với thêm dòng giá trị đại diện
		\begin{center}
			\begin{tabular}{|l|c|c|c|c|}
				\hline
				Lớp điểm & $[6;10]$ & $[11;15]$ & $[16;20]$ & Cộng\\
				\hline
				Giá trị đại diện & $8$ & $13$ & $18$ & \\
				\hline
				Tần số & $22$ & $12$ & $6$ & $40$ \\
				\hline
			\end{tabular}
		\end{center}
		Điểm trung bình là $\overline{x}=\dfrac{22\cdot 8+12\cdot 13+6\cdot 18}{40}=11$.\\
		Vậy số học sinh được nhận thưởng là $12+6=18$ (học sinh).
	}
\end{ex}
% \begin{ex}%[0D5G3-1]
% 	Cho biết tình hình thu hoạch lúa vụ mùa năm $1980$ của ba hợp tác xã ở địa phương $V$ như sau
% 	\begin{center}
% 		\begin{tabular}{|l|c|c|c|}
% 			\hline
% 			Hợp tác xã & $A$ & $B$ & $C$ \\
% 			\hline
% 			Năng suất lúa (tạ/ha) & $40$ & $38$ & $36$ \\
% 			\hline
% 			Diện tích trồng lúa (ha) & $150$ & $130$ & $120$ \\
% 			\hline
% 		\end{tabular}	
% 	\end{center}
% 	Hãy tính năng suất lúa trung bình của vụ mùa năm $1980$ trong toàn bộ ba hợp tác xã kể trên.
% 	\choice
% 	{\True $38{,}15$ tạ/ha}
% 	{$38{,}05$ tạ/ha}
% 	{$38{,}10$ tạ/ha}
% 	{$38{,}20$ tạ/ha}
% 	\loigiai{
% 		Sản lượng lúa của hợp tác xã $A$ là $40\cdot 150=6000$ (tạ).\\
% 		Sản lượng lúa của hợp tác xã $B$ là $38\cdot 130=4940$ (tạ).\\
% 		Sản lượng lúa của hợp tác xã $C$ là $36\cdot 120=4320$ (tạ).\\
% 		Tổng sản lượng lúa của cả ba hợp tác xã là $6000+4940+4320=15260$ (tạ).\\
% 		Tổng diện tích trồng lúa của cả ba hợp tác xã là $150+130+120=400$ (ha).\\
% 		Vậy năng suất lúa trung bình của cả ba hợp tác xã là $\dfrac{15260}{400}=38{,}15$ tạ/ha.\\
% 		\textbf{\underline{Lưu ý}:} Không thể tính năng suất trung bình bằng cách $\dfrac{40+38+36}{3}=38$ (tạ/ha), vì khi chênh lệch diện tích càng lớn thì số trung bình càng không chính xác.
% 	}
% \end{ex}
% \begin{ex}%[0D5Y3-2]
% 	Tiền lương hàng tháng của $7$ nhân viên trong một công ty du lịch là $650$, $840$, $690$, $2500$, $720$, $670$, $3000$ (đơn vị: nghìn đồng). Tìm số trung vị của các số liệu thống kê đã cho.
% 	\choice
% 	{$690$}
% 	{$2500$}
% 	{\True $720$}
% 	{$670$}
% 	\loigiai{ 
% 		Sắp xếp thứ tự các số liệu thống kê, ta thu được dãy tăng các số liệu như sau: $650$, $670$, $690$, $720$, $840$, $2500$, $3000$ (nghìn đồng).\\
% 		Ta có số các số liệu thống kê là $n=7=2\cdot 3+1$ nên số trung vị là $M_e=x_4=720$.
% 	}
% \end{ex}
% \begin{ex}%[0D5Y3-2]
% 	Điểm học kì một của một học sinh được cho bởi bảng số liệu sau (Đơn vị: điểm)
% 	\begin{center}
% 		\begin{tabular}{|c|c|c|c|c|c|c|c|c|}
% 			\hline
% 			5& 6 &6&7& 7 &8 &8& 8,5&9\\
% 			\hline
% 		\end{tabular}
% 	\end{center}
% 	Số trung vị của bảng nói trên là
% 	\choice
% 	{$6$}
% 	{$9$}
% 	{\True $7$}
% 	{$8$}
% 	\loigiai{ Ta có $N=9$ là số lẻ. Số liệu thứ $\dfrac{N+1}{2} = 5$ là số trung vị.\\
% 		Do đó số trung vị là $M_e = 7$ (điểm).
% 	}
% \end{ex}
% \begin{ex}%[0D5Y3-2]
% 	Điều tra số học sinh giỏi khối $10$ của $15$ trường cấp ba trên địa bản tỉnh $A$, ta được bảng số liệu như sau
% 	\begin{center}
% 		\begin{tabular}{|c|c|c|c|c|c|c|c|c|c|c|c|c|c|c|}
% 			\hline
% 			22& 29 &29&29& 30 &31 &32& 32&33 &34 &34 &35 &35 &35 &36  \\
% 			\hline	
% 		\end{tabular}
% 	\end{center}
% 	Số trung vị của bảng nói trên là
% 	\choice
% 	{\True $8$}
% 	{$9$}
% 	{$6$}
% 	{$7$}
% 	\loigiai{ Ta có $N=15$ là số lẻ $\Rightarrow$ số liệu thứ $\dfrac{15+1}{2}=8$ là số trung vị.\\
% 		Vậy số trung vị là $M_e = 8$.
% 	}
% \end{ex}
% \begin{ex}%[0D5Y3-2]
% 	Cho mẫu số liệu thống kê $\{6;4;4;1;9;10;7\}$. Số liệu trung vị của mẫu số liệu thống kê trên là
% 	\choice
% 	{$1$}
% 	{\True $6$}
% 	{$4$}
% 	{$10$}
% 	\loigiai{
% 		Sắp xếp thành dãy không giảm: $1$, $4$, $4$, $6$, $7$, $9$, $10$.\\
% 		Từ dãy trên ta có số trung vị là số $6$ trong dãy trên.
% 	}
% \end{ex}
% \begin{ex}%[0D5B3-2]
% 	Điểm kiểm tra môn Toán của $10$ học sinh được cho như sau: $6$; $7$; $7$; $6$; $7$; $8$; $8$; $7$; $9$; $9$. Số trung vị của mẫu số liệu trên là	
% 	\choice
% 	{$6$}
% 	{\True $7$}
% 	{$8$}
% 	{$9$}
% 	\loigiai{
% 		Ta sắp xếp số liệu theo thứ tự không giảm như sau: $6$; $6$; $7$; $7$; $7$; $7$; $8$; $8$; $9$; $9$.\\
% 		Dãy số trên có tất cả $10$ giá trị, và $2$ giá trị chính giữa bằng $7$.\\
% 		Vậy số trung vị của mẫu số liệu trên là $\dfrac{7+7}{2}=7$.
% 	}
% \end{ex}
% \begin{ex}%[0D5B3-2]
% 	Một cửa hàng dép da đã thống kê cỡ dép của một số khách hàng nam cho kết quả như sau: $39$; $38$; $39$; $40$; $41$; $41$; $43$; $37$; $38$; $40$; $43$; $41$; $42$; $41$; $42$. Tìm trung vị của mẫu số liệu trên.
% 	\choice
% 	{$37$}
% 	{$39$}
% 	{\True $41$}
% 	{$43$}
% 	\loigiai{
% 		Ta sắp xếp số liệu theo thứ tự không giảm: $37$; $38$; $38$; $39$; $39$; $40$; $40$; $41$; $41$; $41$; $41$; $42$; $42$; $42$; $43$.\\
% 		Vì $n=15$ là số lẻ nên số trung vị là số chính giữa của dãy số liệu.\\
% 		Vậy trung vị là $M_e=41$.
% 	}
% \end{ex}
% \begin{ex}%[0D5B3-2]
% 	Điều tra số học sinh của $30$ lớp học, ta được bảng số liệu như sau
% 	\begin{center}
% 		\begin{tabular}{|c|c|c|c|c|c|c|c|c|c|c|c|c|c|c|}
% 			\hline
% 			35& 39 &39&40& 40 &41 &41& 41&41 &44 &44 &45 &45 &45 &46  \\
% 			\hline
% 			48 &48 &48&48& 49 &49 &49&49 &49 &49 &50 &50 &50 &50 &51  \\
% 			\hline	
% 		\end{tabular}
% 	\end{center}
% 	Số trung vị của bảng nói trên là
% 	\choice
% 	{$46$}
% 	{$49$}
% 	{\True $47$}
% 	{$48$}
% 	\loigiai{ Ta có $N=30$ là số chẵn. Số liệu thứ $15$ và $16$ lần lượt là $46$, $48$ là số trung vị.\\
% 		Vậy số trung vị là $M_e = \dfrac{46+48}{2}=47$ (học sinh).
% 	}
% \end{ex}
% \begin{ex}%[0D5B3-2]
% 	Cho bảng phân bố tần số
% 	\begin{center}
% 		\begin{tabular}{|l|c|c|c|c|c|c|}
% 			\hline
% 			Tuổi & $18$ & $19$ & $20$ & $21$ & $22$ & Cộng \\
% 			\hline
% 			Tần số & $10$ & $50$ & $70$ & $29$ & $10$ & $169$ \\
% 			\hline	
% 		\end{tabular}
% 	\end{center}
% 	Số trung vị của bảng phân bố tần số đã cho là
% 	\choice
% 	{$18$ tuổi}
% 	{\True $20$ tuổi}
% 	{$19$ tuổi}
% 	{$21$ tuổi}
% 	\loigiai{
% 		Sau khi sắp xếp các tuổi trên thành dãy không giảm, do có $169$ số nên số trung vị là số thứ $85$ trong dãy trên.\\
% 		Mà số thứ $85$ trong dãy là $20$. Vậy $M_e=20$.
% 	}
% \end{ex}
% \begin{ex}%[0D5B3-2]
% 	Để khảo sát kết quả thi tuyển sinh môn Toán trong kì thi tuyển sinh Đại học năm vừa qua của trường $A$, người điều tra chọn một mẫu gồm $100$ học sinh tham gia kì thi tuyển sinh đó. Điểm môn Toán (thang điểm $10$) của các học sinh này được cho ở bảng phân bố tần số sau đây
% 	\begin{center}
% 		\begin{tabular}{|l|c|c|c|c|c|c|c|c|c|c|c|c|c|}
% 			\hline
% 			Điểm & $0$ & $1$ & $2$ & $3$ & $4$ & $5$ & $6$ & $7$ & $8$ & $9$ & $10$ &  \\
% 			\hline
% 			Tần số & $1$ & $1$ & $3$ & $5$ & $8$ & $13$ & $19$ & $24$ & $14$ & $10$ & $2$ & $n=100$ \\
% 			\hline	
% 		\end{tabular}
% 	\end{center}
% 	Số trung vị của mẫu số liệu trên.
% 	\choice
% 	{\True $M_e=6{,}5$}
% 	{$M_e=7{,}5$}
% 	{$M_e=5{,}5$}
% 	{$M_e=6$}
% 	\loigiai{
% 		Do kích thước mẫu $n=100$ là một số chẵn nên số trung vị là trung bình cộng của hai giá trị đứng thứ $\dfrac{n}{20}=50$ và $\dfrac{n}{2}+1=51$.\\
% 		Do đó $M_e=\dfrac{6+7}{2}=6{,}5$.
% 	}
% \end{ex}
% \begin{ex}%[0D5B3-2]
% 	Số áo bán được trong một quý ở một cửa hàng bán áo sơ-mi nam được cho trong bảng sau
% 	\begin{center}
% 		\begin{tabular}{|l|c|c|c|c|c|c|c|c|}
% 			\hline
% 			Cỡ số & $36$ & $37$ & $38$ & $39$ & $40$ & $41$ & $42$ & Cộng \\
% 			\hline
% 			Số áo bán được & $13$ & $45$ & $126$ & $110$ & $126$ & $40$ & $5$ & $465$ \\
% 			\hline	
% 		\end{tabular}
% 	\end{center}
% 	Hãy tìm số trung vị của các số liệu thống kê trên.
% 	\choice
% 	{$37$}
% 	{$38$}
% 	{\True $39$}
% 	{$40$}
% 	\loigiai{
% 		Ta sắp xếp dãy số áo bán được theo dãy không giảm
% 		$$36, 36, 36, \ldots, 36, 37, 37, \ldots, 37, 38, 38, \ldots, 38, \ldots, 42, 42.$$
% 		Dãy trên gồm $465$ số nên số trung vị là số thứ $233$.\\
% 		Mà số thứ $233$ là số $39$. Vậy $M_e=233$.
% 	}
% \end{ex}
\begin{ex}
	Cho bảng tần số về cân nặng của 180 người dân trong một xã như sau: (đơn vị: kg)
	\begin{center}
	\begin{tabular}{|c|c|c|}
	\hline
	\textbf{Nhóm} & \textbf{Tần số} & \textbf{Tần số tích luỹ}\\ 
	\hline
	$\left[0;10\right)$ & $6$ & $6$\\
	$\left[10;20\right)$ & $15$ & $21$\\
	$\left[20;30\right)$ & $37$ & $58$\\
	$\left[30;40\right)$ & $48$ & $106$\\
	$\left[40;50\right)$ & $22$ & $128$\\
	$\left[50;60\right)$ & $29$ & $157$\\
	$\left[60;70\right)$ & $23$ & $180$\\

	\hline
	& $n = 180$ &\\
	\hline
\end{tabular}	
	\end{center}
	Tứ phân vị thứ nhất của mẫu số liệu trên là
	\choice
	{$56{,}486$ kg}
	{\True $26{,}486$ kg}
	{$25{,}496$ kg}
	{$36{,}486$ kg}
	\loigiai{
	Số phần tử của mẫu là $n=180$  và $\dfrac{n}{4}=\dfrac{\cdot 180}{4}=45$.\\ Ta có  $21<135<58$ nên nhóm $3$ là nhóm  đầu tiên có   tần số tích luỹ  lớn hơn hoặc bằng $45$.\\
	Xét nhóm $3$ là  nhóm $\left[20;30\right)$ có $s=20$, $h=10$, $n_3=37$, nhóm $2$ là nhóm có $cf_2=21$.\\
	Vậy $Q_1=20+\dfrac{45-21}{37}\cdot 10\approx 26{,}49$ (kg).
}
	\end{ex}

\begin{ex}
	Cho bảng tần số chiều cao của 46 học sinh nam của khối lớp $11$ như sau
	\begin{center}
		\begin{tabular}{|c|c|}
			\hline
			\textbf{Nhóm} & \textbf{Tần số} \\
			\hline
			$\left[155;160\right)$ & $3$ \\
			$\left[160;165\right)$ & $18$ \\
			$\left[165;170\right)$ & $10$ \\
			$\left[170;175\right)$ & $15$ \\
			
			\hline
			& $n = 46$ \\
			\hline
		\end{tabular}	
	\end{center}
	Xác định tứ phân vị thứ nhất của mẫu số liệu trên
\choice
{$161{,}36$}
{$161{,}63$}
{\True $162{,}36$}
{$162{,}63$}
\loigiai{
	Số phần tử của mẫu là $n=46$; $\dfrac{n}{4}=11{,}5$.\\
		Ta có $cf_1=3$, $cf_2=3+18=21$ và $6<11{,}5<21$ nên nhóm $2$ là nhóm đầu tiên có   tần số tích luỹ  lớn hơn hoặc bằng $11{,}5$.\\
		Xét nhóm $2$ là nhóm $\left[160;165\right)$, ta có $s=160$, $h=5$, $n_6=21$; nhóm $1$ có tần số tích luỹ bằng $6$.\\
		Vậy $Q_1=160+\dfrac{11{,}5-3}{18}\cdot 5=162{,}36$.
}
\end{ex}
\begin{ex}
	Cho bảng tần số chiều cao của 46 học sinh nam của khối lớp $11$ như sau
	\begin{center}
		\begin{tabular}{|c|c|}
			\hline
			\textbf{Nhóm} & \textbf{Tần số} \\
			\hline
			$\left[155;160\right)$ & $3$ \\
			$\left[160;165\right)$ & $18$ \\
			$\left[165;170\right)$ & $10$ \\
			$\left[170;175\right)$ & $15$ \\
			
			\hline
			& $n = 46$ \\
			\hline
		\end{tabular}	
	\end{center}
	Xác định tứ phân vị thứ ba của mẫu số liệu trên
	\choice
	{$162{,}36$}
	{$166{,}5$}
	{ $166$}
	{\True $171{,}16$}
	\loigiai{
		Số phần tử của mẫu là $n=46$; $\dfrac{3n}{4}=34{,}5$.\\
		Ta có $cf_3=3+18+10=31$, $cf_4=31+15=46$ và $31<34{,}5<46$ nên nhóm $4$ là nhóm đầu tiên có   tần số tích luỹ  lớn hơn hoặc bằng $34{,}5$.\\
		Xét nhóm $4$ là nhóm $\left[170;175\right)$, ta có $t=170$, $l=5$, $n_4=15$; nhóm $3$ có tần số tích luỹ bằng $31$.\\
		Vậy $Q_3=170+\dfrac{34{,}5-31}{15}\cdot 5=171{,}16$.
	}
\end{ex}
\begin{ex}
	\immini{Cho bảng tần số ghép nhóm số liệu thống kê  chiều cao  của $40$ mẫu cây ở  một vườn thực vật 	(đơn vị: centimét).\\
		Xác định tứ phân vị thứ hai của  số liệu ghép nhóm trên
\choice
{\True $56{,}43$}
{$56{,}34$}
{$46{,}43$}
{$36{,}43$}		

}{\begin{tabular}{|c|c|c|}
	\hline
	\textbf{Nhóm} & \textbf{Tần số} & \textbf{Tần số tích luỹ}\\ 
	\hline
	$\left[30;40\right)$ & $4$ & $4$\\
	$\left[40;50\right)$ & $10$ & $14$\\
	$\left[50;60\right)$ & $14$ & $28$\\
	$\left[60;70\right)$ & $6$ & $34$\\
	$\left[70;80\right)$ & $4$ & $38$\\
	$\left[80; 	90\right)$ & $2$ & $40$\\

	
	\hline
	& $n = 40$ &\\
	\hline
\end{tabular}}
\loigiai{
	Số phần tử của mẫu là $n=46$; $\dfrac{n}{2}=23$.\\
	Ta có $cf_2=14<23<cf_3=28$ nên nhóm $3$ là nhóm đầu tiên có tần số tích lũy lớn hơn hoặc bằng $23$.\\
	Xét nhóm $3$ là nhóm $[50;60)$ có $r=50$, $d=10$, $n_3=14$ và nhóm $2$ là nhóm $\left[40;50\right)$ có $cf_2=14$.\\
	Do đó $Q_2=50+\dfrac{23-14}{14}\cdot 10=56{,}43$.
}
\end{ex}
\begin{ex}
Một bảng xếp hạng đã tính điềm chuẩn hoá cho chỉ số nghiên cứu của một số trường đại học ở
Việt Nam và thu được kết quả sau
\begin{center}
\begin{tabular}{|c|c|c|c|c|c|c|}
	\hline
	\textbf{Điểm} & Dưới $20$  &  $[20;30)$&  $[30;40)$ &  $[40;60)$ &  $[60;80)$ &  $[80;100)$\\
	\hline
	Số điểm & $4$ & $19$&$6$&$2$&$3$&$1$\\

	
	
	\hline

\end{tabular}	
\end{center}
Xác định điểm ngưỡng đề đưa ra danh sách $25$\% trường đại học có chỉ số nghiên cứu tốt nhất Việt Nam.	
\choice
{$25{,}26$}
{\True $35{,}42$}
{$45{,}35$}
{$45{,}42$}	
\loigiai{
Điểm ngưỡng để đưa ra danh sách $25$\% trường đại học có chỉ số nghiên cứu tốt nhất Việt Nam là tứ phân
vị thứ ba.\\
Ta có  $n=35$ và $\dfrac{3n}{4}=26{,}25$.\\
Do $cf_2=4+19=23<26{,}25<cf_3=23+6=29$ nên nhóm $[30;40])$ là nhóm đầu tiên có tần số tích lũy lớn hơn hoặc bằng $23{,}25$.\\
Nhóm $[30;40])$ có $r=30$, $d=10$, $n_3=6$; nhóm $2$ có $cf_2=23$. Do đó
$$Q_3=30+\dfrac{26{,}25-23}{6}\cdot 10\approx 35{,}41.$$ 
Vậy để đưa ra danh sách $25$\% trường đại học có chỉ số nghiên cứu tốt nhất Việt Nam ta lấy các trường có
điểm chuẩn hóa trên $35{,}42$.
}
\end{ex}
% \begin{ex}%1%[1K3B9-3]
% 	Điểm thi môn Toán (thang điểm 100, điểm được làm tròn đến 1) của 60 thí sinh được cho trong bảng sau
% 	\begin{center}
% 		\begin{tabular}{|l|c|c|c|c|c|}
% 			\hline Điểm & $[0-9,5)$ & $[9,5-19,5)$ & $[19,5-29,5)$ & $[29,5-39,5)$ & $[39,5-49,5)$ \\
% 			\hline Số thí sinh & $1 $& $2$ & $4$ & $6$ & $15$ \\
% 			\hline Điểm & $[49,5-59,5)$ & $[59,5-69,5)$ & $[69,5-79,5)$ & $[79,5-89,5)$ & $[89,5-99,5)$ \\
% 			\hline Số thí & $12$ & $10$ & $6$ & $3$ & $1$ \\
% 			\hline
% 		\end{tabular}    
% 	\end{center}
% 	Tìm  tứ phân vị thứ hai của mẫu số liệu.
% 	\choice
% 	{\True $Q_2\approx 51,17$}
% 	{$Q_2\approx 51,67$}
% 	{$Q_2\approx 49,5$}
% 	{$Q_2\approx 41,3$}
% 	\loigiai{Cỡ mẫu là $n=60$.\\
% 		Tứ phân vị thứ nhất $Q_2$ là $\dfrac{x_{30}+x_{31}}{2}$. Do $x_{30}$, $x_{31}$ đều thuộc nhóm $[49,5 ; 59,5)$ nên nhóm này chứa $Q_2$. \\Do đó, $p=6 ; \;a_6=49,5 ;\; m_6=12 ; \;m_1+\ldots+m_5=28, \;a_7-a_6=10$ và ta có
% 		$$
% 		Q_2=49,5+\dfrac{\frac{60}{2}-28}{12}\cdot 10\approx51,17.
% 		$$
% 	}
% \end{ex}
% %2
% \begin{ex}%[1K3B9-3]
% 	Điểm thi môn Toán (thang điểm 100, điểm được làm tròn đến 1) của $60$ thí sinh được cho trong bảng sau
% 	\begin{center}
% 		\begin{tabular}{|l|c|c|c|c|c|}
% 			\hline Điểm & $[0-9,5)$ & $[9,5-19,5)$ & $[19,5-29,5)$ & $[29,5-39,5)$ & $[39,5-49,5)$ \\
% 			\hline Số thí sinh & $1 $& $2$ & $4$ & $6$ & $15$ \\
% 			\hline Điểm & $[49,5-59,5)$ & $[59,5-69,5)$ & $[69,5-79,5)$ & $[79,5-89,5)$ & $[89,5-99,5)$ \\
% 			\hline Số thí & $12$ & $10$ & $6$ & $3$ & $1$ \\
% 			\hline
% 		\end{tabular}    
% 	\end{center}
% 	Tìm  tứ phân vị thứ nhất của mẫu số liệu.
% 	\choice
% 	{\True $Q_1\approx 41,3$}
% 	{$Q_1\approx 51,67$}
% 	{$Q_1\approx 40,83$}
% 	{$Q_1\approx 51,17$}
% 	\loigiai{Cỡ mẫu là $n=60$.\\
% 		Tứ phân vị thứ nhất $Q_1$ là $\dfrac{x_{15}+x_{16}}{2}$. Do $x_{15}$, $x_{16}$ đều thuộc nhóm $[39,5-49,5)$ nên nhóm này chứa $Q_1$. \\Do đó, $p=5 ; \;a_5=39,5 ;\; m_5=15 ; \;m_1+\ldots+m_4=13, \;a_6-a_5=10$ và ta có
% 		$$
% 		Q_1=39,5+\dfrac{\frac{60}{4}-13}{15}\cdot 10\approx 40,83.
% 		$$
% 	}
% \end{ex}
%3
\begin{ex}%[1K3B9-3]
	Phỏng vấn một số học sinh khối 11 vể thời gian (giờ) ngủ của một buổi tối, thu được bảng số liệu như sau.
	\begin{center}
		\begin{tabular}{|l|c|c|c|c|c|}
			\hline Thời gian  (giờ)  &{$[4 ; 5)$}&{$[5 ; 6)$}&{$[6 ; 7)$}&{$[7 ; 8)$}&{$[8 ; 9)$}\\
			\hline Số học sinh & $6$ & $10$ & $13$ & $9$ & $7$ \\
			\hline
		\end{tabular}     
	\end{center}
	Hãy cho biết $75 \%$ học sinh khối 11 ngủ ít nhất bao nhiêu giờ?
	\choice
	{$7,675$}
	{\True $7,53$}
	{$8$}
	{ $7,9$}
	\loigiai{
		Cỡ mẫu là $n=45$.\\
		Gọi $x_1, \ldots, x_{45}$ là mẫu số liệu được sắp xếp theo thứ tự không giảm. Khi đó, trung vị là $x_{23}$. Do đó, tứ phân vị thứ ba $Q_3$ là $x_{34}$. Do $x_{34}$ đều thuộc nhóm $[7;8)$ nên nhóm này chứa $Q_3$. Do đó, $p=4 ; \;a_4=7 ;\; m_4=9 ; \;m_1+m_2+m_3=29 ; \;a_5-a_4=1$ và ta có
		$$
		Q_3=7+\dfrac{\frac{3 \cdot 45}{4}-29}{9}\cdot 1\approx7,53.
		$$ 
		Vậy $75\%$ học sinh khối 11 ngủ ít nhất $7,53$ giờ.
	}
\end{ex}
%4
% \begin{ex}%[1K3B9-3]
% 	Điểm thi môn Toán (thang điểm 100, điểm được làm tròn đến 1) của 60 thí sinh được cho trong bảng sau
% 	\begin{center}
% 		\begin{tabular}{|l|c|c|c|c|c|}
% 			\hline Điểm & $[0-9,5)$ & $[9,5-19,5)$ & $[19,5-29,5)$ & $[29,5-39,5)$ & $[39,5-49,5)$ \\
% 			\hline Số thí sinh & $1 $& $2$ & $4$ & $6$ & $15$ \\
% 			\hline Điểm & $[49,5-59,5)$ & $[59,5-69,5)$ & $[69,5-79,5)$ & $[79,5-89,5)$ & $[89,5-99,5)$ \\
% 			\hline Số thí & $12$ & $10$ & $6$ & $3$ & $1$ \\
% 			\hline
% 		\end{tabular}    
% 	\end{center}
% 	Tìm  tứ phân vị thứ ba của mẫu số liệu.
% 	\choice
% 	{$Q_3=41,3$}
% 	{$Q_3=51,67$}
% 	{$Q_3=45$}
% 	{\True $Q_3=65$}
% 	\loigiai{Cỡ mẫu là $n=60$.\\
% 		Với tứ phân vị thứ ba $Q_3$ là $\dfrac{x_{45}+x_{46}}{2}$. Do $x_{45}$, $x_{46}$ đều thuộc nhóm $[60 ; 70)$ nên nhóm này chứa $Q_3$. Do đó, $p=7 ; \;a_7=60 ;\; m_7=10 ; \;m_1+\ldots+m_6=40 ; \;a_8-a_7=10$ và ta có
% 		$$
% 		Q_3=59,5+\dfrac{\frac{3 \cdot 60}{4}-40}{10}\cdot 10=64,5.
% 		$$
% 	}
% \end{ex}
%5
\begin{ex}%[1K3B9-3]
	Một hãng xe ô tô thống kê lại số lần gặp sự cố về động cơ về động cơ của $100$ chiếc xe cùng loại sau 2 năm sử dụng đầu tiên ở dảng sau
	\begin{center}
		\begin{tabular}{|l|c|c|c|c|c|}
			\hline Số lần gặp sự cố  &{$[0,5;2,5)$}&{$[2,5;4,5)$}&{$[4,5;6,5)$}&{$[6,5 ; 8,5)$}&{$[8,5;10,5)$}\\
			\hline Số xe & $17$ & $33$ & $25$ & $20$ & $5$ \\
			\hline
		\end{tabular}     
	\end{center}
	Tìm tứ phân vị thứ nhất của mẫu số liệu.
	\choice
	{\True $Q_1\approx 4$}
	{$Q_1\approx 2,98$}
	{$Q_1\approx 2,5$}
	{$Q_1\approx 3,5$}
	\loigiai{
		Cỡ mẫu là $n=100$.\\
		Gọi $x_1, \ldots, x_{100}$ là mẫu số liệu được sắp xếp theo thứ tự không giảm. Khi đó, trung vị là $\dfrac{x_{50}+x_{51}}{2}$. 
		Do đó, tứ phân vị thứ nhất $Q_1$ là $\dfrac{x_{25}+x_{26}}{2}$. Do $x_{25}$, $x_{26}$ đều thuộc nhóm $[2,5;4,5)$ nên nhóm này chứa $Q_1$. \\Do đó, $p=2 ; \;a_2=2,5;\; m_2=33 ; \;m_1=17, \;a_3-a_2=2$ và ta có
		$$
		Q_1=2,5+\dfrac{\frac{100}{4}-17}{33}\cdot 2\approx 2,98.
		$$
	}    
\end{ex}
%6
% \begin{ex}%[1K3B9-3]
% 	Một hãng xe ô tô thống kê lại số lần gặp sự cố về động cơ về động cơ của $100$ chiếc xe cùng loại sau 2 năm sử dụng đầu tiên ở dảng sau
% 	\begin{center}
% 		\begin{tabular}{|l|c|c|c|c|c|}
% 			\hline Số lần gặp sự cố  &{$[0,5;2,5)$}&{$[2,5;4,5)$}&{$[4,5;6,5)$}&{$[6,5 ; 8,5)$}&{$[8,5;10,5)$}\\
% 			\hline Số xe & $17$ & $33$ & $25$ & $20$ & $5$ \\
% 			\hline
% 		\end{tabular}     
% 	\end{center}
% 	Tìm   tứ phân vị thứ hai của mẫu số liệu.
% 	\choice
% 	{\True $Q_2=4,5$}
% 	{$Q_2\approx 5,12$}
% 	{$Q_2\approx 4,89$}
% 	{$Q_2\approx 5,2$}
% 	\loigiai{
% 		Cỡ mẫu là $n=100$.\\
% 		Gọi $x_1, \ldots, x_{100}$ là mẫu số liệu được sắp xếp theo thứ tự không giảm. Khi đó, trung vị là $\dfrac{x_{50}+x_{51}}{2}$. Do $x_{50} \in [2,5;4,5)$, $x_{51} \in [4,5;6,5)$  nên tứ phân vị thứ hai của mẫu số liệu ghép nhóm là  $Q_2=4,5$. 
% 	}
% \end{ex}
%7
\begin{ex}%[1K3B9-3]
	Một hãng xe ô tô thống kê lại số lần gặp sự cố về động cơ về động cơ của $100$ chiếc xe cùng loại sau 2 năm sử dụng đầu tiên ở dảng sau
	\begin{center}
		\begin{tabular}{|l|c|c|c|c|c|}
			\hline Số lần gặp sự cố  &{$[0,5;2,5)$}&{$[2,5;4,5)$}&{$[4,5;6,5)$}&{$[6,5 ; 8,5)$}&{$[8,5;10,5)$}\\
			\hline Số xe & $17$ & $33$ & $25$ & $20$ & $5$ \\
			\hline
		\end{tabular}     
	\end{center}
	Tìm   tứ phân vị thứ ba của mẫu số liệu.  
	\choice
	{$Q_3=6,3$}
	{$Q_3=6,8$}
	{$Q_3=7,2$}
	{\True $Q_3=6,5$}
	\loigiai{ Cỡ mẫu là $n=100$.\\
		Với tứ phân vị thứ ba $Q_3$ là $\dfrac{x_{75}+x_{76}}{2}$. Do $x_{75} \in [4,5;6,5)$, $x_{76} \in [6,5 ; 8,5)$  nên tứ phân vị thứ ba của mẫu số liệu ghép nhóm là $Q_3=6,5$. 
		
	}
\end{ex}
%8
% \begin{ex}%[1K3B9-3]
% 	Lương tháng của một số nhân viên văn phòng được ghi lại như sau (đơn vị: triệu đồng)
% 	\begin{center}
% 		\begin{tabular}{|l|c|c|c|c|c|}
% 			\hline Lương tháng (triệu đồng)  &{$[6;8)$}&{$[8;10)$}&{$[10;12)$}&{$[12;14)$}\\
% 			\hline Số nhân viên & $3$ & $6$ & $8$ & $7$  \\
% 			\hline
% 		\end{tabular}     
% 	\end{center}  
% 	Tìm tứ phân vị thứ nhất của mẫu số liệu.
% 	\choice
% 	{\True $Q_1= 9$}
% 	{$Q_1= 8,5$}
% 	{$Q_1= 9,5$}
% 	{$Q_1= 8,2$}
% 	\loigiai{
% 		Cỡ mẫu là $n=24$.\\
% 		Gọi $x_1, \ldots, x_{24}$ là mẫu số liệu được sắp xếp theo thứ tự không giảm. Khi đó, trung vị là $\dfrac{x_{12}+x_{13}}{2}$. 
% 		Do đó, tứ phân vị thứ nhất $Q_1$ là $\dfrac{x_{6}+x_{7}}{2}$. Do $x_{6}$, $x_{7}$ đều thuộc nhóm $[8;10)$ nên nhóm này chứa $Q_1$. \\Do đó, $p=2 ; \;a_2=8;\; m_2=6 ; \;m_1=3, \;a_3-a_2=2$ và ta có
% 		$$
% 		Q_1=8+\dfrac{\frac{24}{4}-3}{6}\cdot 2=9.
% 		$$
% 	}    
% \end{ex}
% %9
% \begin{ex}%[1K3B9-3]
% 	Lương tháng của một số nhân viên văn phòng được ghi lại như sau (đơn vị: triệu đồng)
% 	\begin{center}
% 		\begin{tabular}{|l|c|c|c|c|c|}
% 			\hline Lương tháng (triệu đồng)  &{$[6;8)$}&{$[8;10)$}&{$[10;12)$}&{$[12;14)$}\\
% 			\hline Số nhân viên & $3$ & $6$ & $8$ & $7$  \\
% 			\hline
% 		\end{tabular}     
% 	\end{center}   
% 	Tìm   tứ phân vị thứ hai của mẫu số liệu.
% 	\choice
% 	{\True $Q_2=10,75$}
% 	{$Q_2= 10,5$}
% 	{$Q_2= 11$}
% 	{$Q_2=11,5$}
% 	\loigiai{
% 		Cỡ mẫu là $n=24$.\\
% 		Gọi $x_1, \ldots, x_{24}$ là mẫu số liệu được sắp xếp theo thứ tự không giảm. Khi đó, trung vị là $\dfrac{x_{12}+x_{13}}{2}$. 
% 		Do đó,  tứ phân vị thứ hai $Q_2$ là $\dfrac{x_{12}+x_{13}}{2}$. Do $x_{12}$, $x_{13}$ đều thuộc nhóm $[10;12)$ nên nhóm này chứa $Q_2$. \\Do đó, $p=3 ; \;a_3=10;\; m_3=8 ; \;m_1+m_2=9, \;a_4-a_3=2$ và ta có
% 		$$
% 		Q_2=10+\dfrac{\frac{24}{2}-9}{8}\cdot 2=10,75.
% 		$$
% 	}
% \end{ex}
% %10
% \begin{ex}%[1K3B9-3]
% 	Lương tháng của một số nhân viên văn phòng được ghi lại như sau (đơn vị: triệu đồng)
% 	\begin{center}
% 		\begin{tabular}{|l|c|c|c|c|c|}
% 			\hline Lương tháng (triệu đồng)  &{$[6;8)$}&{$[8;10)$}&{$[10;12)$}&{$[12;14)$}\\
% 			\hline Số nhân viên & $3$ & $6$ & $8$ & $7$  \\
% 			\hline
% 		\end{tabular}     
% 	\end{center}  
% 	Tìm   tứ phân vị thứ ba của mẫu số liệu.
% 	\choice 
% 	{$Q_3\approx 12,5$}
% 	{$Q_3\approx 13,2$}
% 	{$Q_3\approx 13,5$}
% 	{\True $Q_3\approx 12,3$}
% 	\loigiai{
% 		Cỡ mẫu là $n=24$.\\
% 		Gọi $x_1, \ldots, x_{24}$ là mẫu số liệu được sắp xếp theo thứ tự không giảm. Khi đó, trung vị là $\dfrac{x_{12}+x_{13}}{2}$. 
% 		Do đó,  tứ phân vị thứ ba $Q_3$ là $\dfrac{x_{18}+x_{19}}{2}$. Do $x_{18}$, $x_{19}$ đều thuộc nhóm $[12;14)$ nên nhóm này chứa $Q_3$. \\Do đó, $p=4 ; \;a_4=12;\; m_4=7 ; \;m_1+m_2+m_3=17, \;a_4-a_3=2$ và ta có
% 		$$
% 		Q_2=12+\dfrac{\frac{24\cdot 3}{4}-17}{7}\cdot 2\approx 12,3.
% 		$$
% 	}
% \end{ex}
%11
% \begin{ex}%[1K3B9-3]
% 	Số điểm một cầu thủ bóng rổ ghi được trong 20 trận đấu được cho ở bảng sau
% 	\begin{center}
% 		\begin{tabular}{|l|c|c|c|c|c|}
% 			\hline Điểm số  &{$[5,5;10,5)$}&{$[10,5;15,5)$}&{$[15,5;20,5)$}&{$[20,5;25,5)$}\\
% 			\hline Số trận & $3$ & $9$ & $2$ & $6$  \\
% 			\hline
% 		\end{tabular}     
% 	\end{center}  
% 	Tìm   tứ phân vị thứ ba của mẫu số liệu.
% 	\choice 
% 	{$Q_3\approx 23,5$}
% 	{$Q_3\approx 22,2$}
% 	{$Q_3\approx 21,6$}
% 	{\True $Q_3\approx 21,3$}
% 	\loigiai{
% 		Cỡ mẫu là $n=20$.\\
% 		Gọi $x_1, \ldots, x_{20}$ là mẫu số liệu được sắp xếp theo thứ tự không giảm. Khi đó, trung vị là $\dfrac{x_{10}+x_{11}}{2}$. 
% 		Do đó,  tứ phân vị thứ ba $Q_3$ là $\dfrac{x_{15}+x_{16}}{2}$. Do $x_{15}$, $x_{16}$ đều thuộc nhóm $[20,5;25,5)$ nên nhóm này chứa $Q_3$. \\Do đó, $p=4 ; \;a_4=20,5;\; m_4=6 ; \;m_1+m_2+m_3=14, \;a_5-a_4=25,5-20,5=5$ và ta có
% 		$$
% 		Q_2=20,5+\dfrac{\frac{20\cdot 3}{4}-14}{6}\cdot 5 \approx 21,3.
% 		$$
% 	}
% \end{ex}
% %12
% \begin{ex}%[1K3B9-3]
% 	Số điểm một cầu thủ bóng rổ ghi được trong 20 trận đấu được cho ở bảng sau
% 	\begin{center}
% 		\begin{tabular}{|l|c|c|c|c|c|}
% 			\hline Điểm số  &{$[5,5;10,5)$}&{$[10,5;15,5)$}&{$[15,5;20,5)$}&{$[20,5;25,5)$}\\
% 			\hline Số trận & $3$ & $9$ & $2$ & $6$  \\
% 			\hline
% 		\end{tabular}     
% 	\end{center} 
% 	Tìm tứ phân vị thứ nhất của mẫu số liệu.
% 	\choice
% 	{\True $Q_1\approx 11,6$}
% 	{$Q_1\approx 11,3$}
% 	{$Q_1\approx 21,6$}
% 	{$Q_1\approx 21,3$}
% 	\loigiai{
% 		Cỡ mẫu là $n=20$.\\
% 		Gọi $x_1, \ldots, x_{20}$ là mẫu số liệu được sắp xếp theo thứ tự không giảm. Khi đó, trung vị là $\dfrac{x_{10}+x_{11}}{2}$. 
% 		Do đó, tứ phân vị thứ nhất $Q_1$ là $\dfrac{x_{5}+x_{6}}{2}$. Do $x_{5}$, $x_{6}$ đều thuộc nhóm $[10,5;15,5)$ nên nhóm này chứa $Q_1$. \\Do đó, $p=2 ; \;a_2=10,5;\; m_2=9 ; \;m_1=3, \;a_3-a_2=5$ và ta có
% 		$$
% 		Q_1=10,5+\dfrac{\frac{20}{4}-3}{9}\cdot 5\approx 11,6.
% 		$$
% 	}
% \end{ex}
%13
\begin{ex}%[1K3B9-3]
	Số điểm một cầu thủ bóng rổ ghi được trong 20 trận đấu được cho ở bảng sau
	\begin{center}
		\begin{tabular}{|l|c|c|c|c|c|}
			\hline Điểm số  &{$[5,5;10,5)$}&{$[10,5;15,5)$}&{$[15,5;20,5)$}&{$[20,5;25,5)$}\\
			\hline Số trận & $3$ & $9$ & $2$ & $6$  \\
			\hline
		\end{tabular}     
	\end{center} 
	Tìm tứ phân vị thứ nhất của mẫu số liệu.
	\choice
	{\True $Q_1\approx 11,6$}
	{$Q_1\approx 14,4$}
	{$Q_1\approx 15,6$}
	{$Q_1\approx 21,3$}
	\loigiai{
		Cỡ mẫu là $n=20$.\\
		Gọi $x_1, \ldots, x_{20}$ là mẫu số liệu được sắp xếp theo thứ tự không giảm. Khi đó, trung vị là $\dfrac{x_{10}+x_{11}}{2}$. 
		Do đó, tứ phân vị thứ nhất $Q_2$ là $\dfrac{x_{10}+x_{11}}{2}$. Do $x_{10}$, $x_{11}$ đều thuộc nhóm $[10,5;15,5)$ nên nhóm này chứa $Q_2$. \\Do đó, $p=2 ; \;a_2=10,5;\; m_2=9 ; \;m_1=3, \;a_3-a_2=5$ và ta có
		$$
		Q_1=10,5+\dfrac{\frac{20}{2}-3}{9}\cdot 5\approx 14,4.
		$$   
	}
\end{ex}
%14
% \begin{ex}%[1K3B9-3]
% 	Một người thống kê lại thời gian thực hiện các cuộc gọi điện thoại của người đó trong một tuần cho trong bảng sau
% 	\begin{center}
% 		\begin{tabular}{|l|c|c|c|c|c|}
% 			\hline Số bệnh nhân &{$[0;60)$}&{$[60;120)$}&{$[120;180)$}&{$[180;240)$}&{$[240;300)$}\\
% 			\hline Số ngày & $8$ & $10$ & $7$ & $5$ & $2$ \\
% 			\hline
% 		\end{tabular}     
% 	\end{center}
% 	Tìm   tứ phân vị thứ ba của mẫu số liệu.  
% 	\choice
% 	{$Q_3\approx 175,28$}
% 	{$Q_3\approx 150,32 $}
% 	{$Q_3=175$}
% 	{\True $Q_3\approx 171,43$}
% 	\loigiai{ Cỡ mẫu là $n=32$.\\
% 		Gọi $x_1, \ldots, x_{32}$ là mẫu số liệu được sắp xếp theo thứ tự không giảm. Khi đó, trung vị là $\dfrac{x_{16}+x_{17}}{2}$.\\
% 		Do đó, tứ phân vị thứ ba $Q_3$ là $\dfrac{x_{24}+x_{25}}{2}$. Do $x_{24} $, $x_{25} \in [120;180)$   nên nhóm này chứa $Q_3$. \\Do đó, $p= 3; \;a_3=120 ;\; m_3=7 ; \;m_1+m_2=18 ; \;a_4-a_3=60$ và ta có
% 		$$
% 		Q_3=120+\dfrac{\frac{3 \cdot 32}{4}-18}{7}\cdot 60\approx 171,43.
% 		$$
% 	}
% \end{ex}
% %15
% \begin{ex}%[1K3B9-3]
% 	Một người thống kê lại thời gian thực hiện các cuộc gọi điện thoại của người đó trong một tuần cho trong bảng sau
% 	\begin{center}
% 		\begin{tabular}{|l|c|c|c|c|c|}
% 			\hline Số bệnh nhân &{$[0;60)$}&{$[60;120)$}&{$[120;180)$}&{$[180;240)$}&{$[240;300)$}\\
% 			\hline Số ngày & $8$ & $10$ & $7$ & $5$ & $2$ \\
% 			\hline
% 		\end{tabular}     
% 	\end{center}
% 	Tìm   tứ phân vị thứ hai của mẫu số liệu.  
% 	\choice
% 	{$Q_2\approx 80,25$}
% 	{$Q_2\approx 100,32$}
% 	{$Q_2=115$}
% 	{\True $Q_2=108$}
% 	\loigiai{ Cỡ mẫu là $n=32$.\\
% 		Gọi $x_1, \ldots, x_{32}$ là mẫu số liệu được sắp xếp theo thứ tự không giảm. Khi đó, trung vị là $\dfrac{x_{16}+x_{17}}{2}$.\\
% 		Do đó, tứ phân vị thứ hai $Q_2$ là $\dfrac{x_{16}+x_{17}}{2}$. Do $x_{16} $, $x_{17} \in [60;120)$   nên nhóm này chứa $Q_2$. \\Do đó, $p= 2; \;a_2=60 ;\; m_2=10 ; \;m_1=8 ; \;a_3-a_2=60$ và ta có
% 		$$
% 		Q_3=60+\dfrac{\frac{ 32}{2}-8}{10}\cdot 60=108.
% 		$$
% 	}    
% \end{ex}
% \begin{ex}
% 	Một công ty xây dựng khảo sát khách hàng xem họ có nhu cầu mua nhà ở mức giá nào. Kết quả khảo sát được ghi lại ở bảng sau
% 	\begin{center}
% 		\begin{tabular}{|c|c|c|c|c|c|}
% 			\hline \begin{tabular}{c}
% 				\textbf{Mức giá} \\
% 				\textbf{(triệu đồng/$\mathrm{m}^2)$}
% 			\end{tabular} &{$[10; 14)$} &{$[14; 18)$} &{$[18; 22)$} &{$[22; 26)$} &{$[26; 30)$} \\
% 			\hline \textbf{Số khách hàng} & 54 & 78 & 120 & 45 & 12 \\
% 			\hline
% 		\end{tabular}
% 	\end{center}
	
% 	 Công ty nên xây nhà ở mức giá nào để nhiều người có nhu cầu mua nhất?
% 	\choice
% 	{\True $19{,}4$ triệu đồng$/ \mathrm{m}^2$ }
% 	{$20{,}4$ triệu đồng$/ \mathrm{m}^2$ }
% 	{$19{,}6$ triệu đồng$/ \mathrm{m}^2$ }
% 	{$20{,}6$ triệu đồng$/ \mathrm{m}^2$ }	
% 	\loigiai{
% 		\ Nhóm chứa mốt của mẫu số liệu trên là nhóm $[18; 22)$.\\ Do đó $u_m=18$, $n_{m-1}=78$, $n_m=120$, $n_{m+1}=45$, $u_{m+1}-u_m=22-18=4$.\\
% 			Mốt của mẫu số liệu ghép nhóm là
% 			\[M_0=18+\dfrac{120-78}{(120-78)+(120-45)} \cdot 4=\dfrac{758}{39} \approx 19{,}4. \]
% 		 Dựa vào kết quả trên ta có thể dự đoán rằng nếu công ty xây nhà ở mức giá $19{,}4$ triệu đồng$/ \mathrm{m}^2$ thì sẽ có nhiều người có nhu cầu mua nhất.
		
% 	}
% \end{ex}
\begin{ex}%[1T5B1-3]
	Số cuộc gọi điện thoại một nguời thực hiện mỗi ngày trong $30$ ngày được lựa chọn ngẫu nhiên được thống kê trong bảng sau:
\begin{center}
	\begin{tabular}{|c|c|c|c|c|c|}
		\hline Số cuộc gọi &{$[3; 5]$} &{$[6; 8]$} &{$[9; 11]$} &{$[12; 14]$} &{$[15; 17]$} \\
		\hline Số ngày & 5 & 13 & 7 & 3 & 2 \\
		\hline
	\end{tabular}
\end{center}
 Tìm mốt của mẫu số liệu ghép nhóm trên.
 Hãy dự đoán xem khả năng người đó thực hiện bao nhiêu cuộc gọi mỗi ngày là cao nhất.
\choice
{$4$}
{$6$}
{$5$}
{\True $7$}	
\loigiai{
	Do số cuộc gọi là số nguyên nên ta hiệu chỉnh lại như sau:
	\begin{center}
		\begin{tabular}{|c|c|c|c|c|c|}
			\hline Số cuộc gọi &{$[2{,}5; 5{,}5)$} &{$[5{,}5; 8{,}5)$} &{$[8{,}5; 11{,}5)$} &{$[11{,}5; 14{,}5)$} &{$[14{,}5; 17{,}5)$} \\
			\hline Số ngày & 5 & 13 & 7 & 3 & 2 \\
			\hline
		\end{tabular}
	\end{center}	
		 Nhóm chứa mốt của mẫu số liệu trên là nhóm $[5{,}5; 8{,}5)$.\\
		Do đó $u_m=5{,}5$; $n_{m-1}=5$; $n_m=13$; $n_{m+1}=7$; $u_{m+1}-u_m=8{,}5-5{,}5=3$.\\
		Mốt của mẫu số liệu ghép nhóm là
		\[M_0=5{,}5+\dfrac{13-5}{(13-5)+(13-7)} \cdot 3=\dfrac{101}{14} \approx 7{,}2. \]
	 Dựa vào kết quả trên ta có thể dự đoán rằng khả năng người đó thực hiện $7$ cuộc gọi mỗi ngày là cao nhất.
	
}
\end{ex}
\begin{ex}
	Một thư viện thống kê số lượng sách được mượn mỗi ngày trong ba tháng ở bảng sau:
	\begin{center}
		\begin{tabular}{|c|c|c|c|c|c|c|c|}
			\hline Số sách &{$[16; 20]$} &{$[21; 25]$} &{$[26; 30]$} &{$[31; 35]$} &{$[36; 40]$} &{$[41; 45]$} &{$[46; 50]$} \\
			\hline Số ngày & 3 & 6 & 15 & 27 & 22 & 14 & 5 \\
			\hline
		\end{tabular}
	\end{center}
	Hãy ước lượng  mốt của mẫu số liệu ghép nhóm trên.
	\choice
	{$34{,}33$}
	{\True $34{,}03$}
	{$35{,}63$}
	{$34{,}13$}	
	\loigiai{
		Vì số lượng sách được mượn là số nguyên nên ta hiệu chỉnh bảng tần số ghép nhóm (theo giá trị đại diện) như sau
		\begin{center}
			{\footnotesize \begin{tabular}{|c|c|c|c|c|c|c|c|}
					\hline Số sách &{$[15{,}5; 20{,}5)$} &{$[20{,}5; 25{,}5)$} &{$[25{,}5; 30{,}5)$} &{$[30{,}5; 35{,}5)$} &{$[35{,}5; 40{,}5]$} &{$[40{,}5; 45{,}5)$} &{$[45{,}5; 50{,}5)$} \\
					\hline Giá trị đại diện &{$18$} &{$23$} &{$28$} &{$33$} &{$38$} &{$43$} &{$48$} \\
					\hline Số ngày & 3 & 6 & 15 & 27 & 22 & 14 & 5 \\
					\hline
			\end{tabular}}
		\end{center}
		Trung bình số lượng sách được mượn mỗi ngày trong 3 tháng của thư viện là
		\[\overline{x}=\dfrac{18\cdot 3+23\cdot 6+28\cdot 15+33\cdot 27+38\cdot 22+43\cdot 14+48\cdot 5}{92}\approx 34{,}58. \]
		Nhóm chứa mốt của mẫu số liệu trên là nhóm $[30{,}5; 35{,}5)$.\\
		Do đó $u_m=30{,}5$; $n_{m-1}=15$; $n_m=27$; $n_{m+1}=22$; $u_{m+1}-u_m=35{,}5-30{,}5=5$.\\
		Mốt của mẫu số liệu ghép nhóm là
		\[M_0=30{,}5+\dfrac{27-15}{(27-15)+(27-22)} \cdot 5\approx 34{,}03. \]
	}
\end{ex}
\begin{ex}
	Kết quả đo chiều cao của $200$ cây keo $3$ năm tuổi ở một nông trường được biểu diễn ở biểu đồ dưới đây.
	\begin{center}
		\begin{tikzpicture}[scale=1,font=\scriptsize]
		\def\hoanh{11.5};
		\def\tung{6.5};
		\def\mau{cyan};
		\foreach \x/\n in{1/20,3/35,5/60,7/55,9/30}{\draw[line width=16mm,\mau] (\x,0)--++(0,{\n/10});
			\draw[dashed] (\x,{\n/10})node[above]{$\n$}--(0,{\n/10}) node[left]{$\n$};}
		\foreach \x/\p in {1/[8{,}5;8{,}8),3/[8{,}8;9{,}1),5/[9{,}1;9{,}4),7/[9{,}4;9{,}7),9/[9{,}7;10{,}0)}{\node[below] at (\x,0){\scriptsize $\p$};}
		\draw[->] (0,0)--(\hoanh,0) node[below]{($m$)};
		\draw[->] (0,0)node[below left]{$O$}--(0,\tung) node[left]{(Số cây)};
		\path (current bounding box.north) node[above]		{\textbf{Chiều cao 200 cây keo 3 năm tuổi}};
		\end{tikzpicture}
	\end{center}
	Mốt của mẫu số liệu ghép nhóm trên là
	\choice
	{$9{,}35$}
	{$10{,}53$}
	{$10{,}35$}
	{$9{,}53$}	
	\loigiai{
		Bảng tần số ghép nhóm (theo giá trị đại diện)
		\begin{center}
			\begin{tabular}{|c|c|c|c|c|c|}
				\hline Chiều cao &$[8{,}5; 8{,}8)$ &{$[8{,}8; 9{,}1)$} &{$[9{,}1; 9{,}4)$} &{$[9{,}4; 9{,}7)$} &{$[9{,}7; 10{,}0)$} \\
				\hline Giá trị đại diện &$8{,}65$ &$8{,}95$ &$9{,}25$ &$9{,}55$ &$9{,}85$ \\
				\hline Số cây & $20$ & $35$ & $60$ & $55$ & $30$\\
				\hline
			\end{tabular}
		\end{center}
		Chiều cao trung bình của $200$ cây keo 3 năm tuổi là
		\[\overline{x}=\dfrac{8{,}65\cdot 20+8{,}95\cdot 35+9{,}25\cdot 60+9{,}55\cdot 55+9{,}85\cdot 30}{200}\approx 9{,}31. \]
		Nhóm chứa mốt của mẫu số liệu trên là nhóm $[9{,}1; 9{,}4)$.\\
		Do đó $u_m=9{,}1$; $n_{m-1}=35$; $n_m=60$; $n_{m+1}=55$; $u_{m+1}-u_m=9{,}4-9{,}1=0{,}3$.\\
		Mốt của mẫu số liệu ghép nhóm là
		\[M_0=9{,}1+\dfrac{60-35}{(60-35)+(60-55)} \cdot 0{,}3= 9{,}35. \]
	}
\end{ex}
\begin{ex}%[1K3B9-4]
	Bảng số liệu ghép nhóm sau cho biết chiều cao (cm) của $50$ học sinh lớp $11A$.
	\begin{center}
		\begin{tabular}{|c|c|c|c|c|c|c|}
			\hline
			Khoảng chiều cao (cm)	& $\left[145;150 \right)$ & $\left[150;155 \right)$ & $\left[155;160 \right)$ & $\left[160;165 \right)$&$\left[165;170 \right)$  \\
			\hline
			Số học sinh&$7$	& $14$ & $10$ &$10$  & $9$ \\
			\hline
		\end{tabular}
	\end{center}
Mốt của mẫu số liệu ghép nhóm là
	\choice
	{$154{,}20$}
	{\True $153{,}18$}
	{$155{,}12$}
	{$158{,}36$}	
	\loigiai{
		Tần số lớn nhất là $14$ nên nhóm chứa mốt là nhóm $\left[150;155 \right)$. Ta có $j=2$, $a_2=150$, $m_2=14$, $m_1=7$, $m_3=10$, $h=5$. Do đó $$M_o=150+\dfrac{14-7}{\left(14-7\right)+\left(14-10\right)\cdot 5}\approx 153{,}18.$$	
		Số học sinh có chiều cao khoảng $153{,}18$ là nhiều nhất.
	}
\end{ex}
\begin{ex}%[1K3Y9-4]
	Chọn khẳng định \textbf{sai}.
	\choice
	{ Mốt của mẫu số liệu không ghép nhóm là giá trị có khả năng xuất hiện cao nhất khi lấy mẫu}
	{Mốt của mẫu số liệu sau khi ghép nhóm xấp xỉ với mốt của mẫu số liệu không ghép nhóm}
	{\True Một mẫu số liệu ghép nhóm chỉ có một mốt}
	{Một mẫu số liệu ghép nhóm có thể có nhiều nhóm chứa mốt và nhiều mốt}
	\loigiai{
		Mốt của mẫu số liệu không ghép nhóm là giá trị có khả năng xuất hiện cao nhất khi lấy mẫu.\\ Mốt của mẫu số liệu sau khi ghép nhóm xấp xỉ với mốt của mẫu số liệu không ghép nhóm. \\
		Một mẫu số liệu ghép nhóm có thể có nhiều nhóm chứa mốt và nhiều mốt.\\
		Do đó khẳng định sai là: Một mẫu số liệu ghép nhóm chỉ có một mốt.
	}    
\end{ex}
% \begin{ex}%[1K3Y9-4]
% 	Người ta ghi lại tuổi thọ của một số con ong cho kết quả như sau:
% 	\begin{center}
% 		\begin{tabular}{|l|c|c|c|c|c|c|}
% 			\hline Tuồi thọ (ngày) &{$[0;20)$}&{$[20;40)$}&{$[40;60)$}&{$[60;80)$}&{$[80;100)$}\\
% 			\hline Số lượng & $5$ & $12$ & $23$ & $31$ & $29$  \\
% 			\hline
% 		\end{tabular}
% 	\end{center}
% 	Nhóm chứa mốt của mẫu số liệu này là
% 	\choice
% 	{ $[20;40)$}
% 	{$[40;60)$}
% 	{\True $[60;80)$}
% 	{$[80;100)$}
% 	\loigiai{
% 		Nhóm chứa mốt của mẫu số liệu này là $[60;80)$.
% 	}    
% \end{ex}
% %6
% \begin{ex}%[1K3B9-4]
% 	Người ta ghi lại tuổi thọ của một số con ong cho kết quả như sau:
% 	\begin{center}
% 		\begin{tabular}{|l|c|c|c|c|c|c|}
% 			\hline Tuồi thọ (ngày) &{$[0;20)$}&{$[20;40)$}&{$[40;60)$}&{$[60;80)$}&{$[80;100)$}\\
% 			\hline Số lượng & $5$ & $12$ & $23$ & $31$ & $29$  \\
% 			\hline
% 		\end{tabular}
% 	\end{center}
% 	Mốt của mẫu số liệu này là
% 	\choice
% 	{ $M_0=70$}
% 	{$M_0=60$}
% 	{\True $M_0=76$}
% 	{$M_0=31$}
% 	\loigiai{
% 		Tần số lớn nhất là $31$ nên nhóm chứa mốt là nhóm $[60;80)$. \\
% 		Ta có, $j=4, a_4=60, m_4=31$, $m_3=23, m_5=29, h=20$. Do đó
% 		$$
% 		M_0=60+\frac{31-23}{(31-29)+(14-11)}\cdot 20 =76.
% 		$$
% 	}   
% \end{ex}
% %7
% \begin{ex}%[1K3Y9-4]
% 	Doanh thu bán hàng 20 ngày được lựa chọn ngẫu nhiên của một cửa hàng được ghi lại ở bảng sau (đơn vị: triệu đồng)
% 	\begin{center}
% 		\begin{tabular}{|l|c|c|c|c|c|c|}
% 			\hline Doanh thu &{$[5;7)$}&{$[7;9)$}&{$[9;11)$}&{$[11;13)$}&{$[13;15)$}\\
% 			\hline Số ngày & $2$ & $9$ & $7$ & $3$ & $1$ \\
% 			\hline
% 		\end{tabular}
% 	\end{center}
% 	Nhóm chứa mốt của mẫu số liệu này là
% 	\choice
% 	{ $[9;11)$}
% 	{$[5;7)$}
% 	{\True $[7;9)$}
% 	{$[11;13)$}
% 	\loigiai{
% 		Tần số lớn nhất là $9$ nên nhóm chứa mốt là nhóm $[7;9)$. \\
% 	}   
% \end{ex}
% %8
% \begin{ex}%[1K3B9-4]
% 	Doanh thu bán hàng 20 ngày được lựa chọn ngẫu nhiên của một cửa hàng được ghi lại ở bảng sau (đơn vị: triệu đồng)
% 	\begin{center}
% 		\begin{tabular}{|l|c|c|c|c|c|c|}
% 			\hline Doanh thu &{$[5;7)$}&{$[7;9)$}&{$[9;11)$}&{$[11;13)$}&{$[13;15)$}\\
% 			\hline Số ngày & $2$ & $9$ & $7$ & $3$ & $1$ \\
% 			\hline
% 		\end{tabular}
% 	\end{center}
% 	Xác định mốt của mẫu số liệu.
% 	\choice
% 	{ $M_0\approx 8$}
% 	{$M_0\approx 8,5$}
% 	{\True $M_0\approx 8,56$}
% 	{$M_0\approx 9$}
% 	\loigiai{
% 		Tần số lớn nhất là $9$ nên nhóm chứa mốt là nhóm $[7;9)$. \\
% 		Ta có, $j=2, a_2=7, m_2=9$, $m_1=2, m_3=7, h=2$. Do đó
% 		$$
% 		M_0=7+\frac{9-2}{(9-2)+(9-7)}\cdot 2\approx 8,56.
% 		$$
% 	}    
% \end{ex}
% %9
% \begin{ex}%[1K3B9-4]
% 	Điểm kiểm tra môn Toán của lớp 12A được cho trong bảng sau
% 	\begin{center}
% 		\begin{tabular}{|l|c|c|c|c|c|c|c|c|}
% 			\hline Khoảng điểm &{$[6,5;7)$}&{$[7;7,5)$}&{$[7,5;8)$}&{$[8;8,5)$}&{$[8,5;9)$}&{$[9;9,5)$}&{$[9,5;10)$}\\
% 			\hline Tần số & $8$ & $10$ & $16$ & $24$& $13$ & $7$ & $4$ \\
% 			\hline
% 		\end{tabular}
% 	\end{center}
% 	Xác định mốt của mẫu số liệu ghép nhóm này.    
% 	\choice
% 	{ $M_0\approx 8,4$}
% 	{$M_0\approx 8,5$}
% 	{\True $M_0\approx 8,21$}
% 	{$M_0\approx 24$}
% 	\loigiai{
% 		Tần số lớn nhất là $24$ nên nhóm chứa mốt là nhóm $[8;8,5)$. \\
% 		Ta có, $j=4, a_4=8, m_4=24$, $m_3=16, m_5=13, h=0,5$. Do đó
% 		$$
% 		M_0=8+\frac{24-16}{(24-16)+(24-13)}\cdot 0,5\approx 8,21.
% 		$$
% 	}
% \end{ex}
% %10
% \begin{ex}%[1K3Y9-4]
% 	Điểm kiểm tra môn Toán của lớp 12A được cho trong bảng sau
% 	\begin{center}
% 		\begin{tabular}{|l|c|c|c|c|c|c|c|c|}
% 			\hline Khoảng điểm &{$[6,5;7)$}&{$[7;7,5)$}&{$[7,5;8)$}&{$[8;8,5)$}&{$[8,5;9)$}&{$[9;9,5)$}&{$[9,5;10)$}\\
% 			\hline Tần số & $8$ & $10$ & $16$ & $24$& $13$ & $7$ & $4$ \\
% 			\hline
% 		\end{tabular}
% 	\end{center}
% 	Nhóm chứa mốt của mẫu số liệu này là
% 	\choice
% 	{ $[7;7,5)$}
% 	{$[7,5;8)$}
% 	{\True $[8;8,5)$}
% 	{$[8,5;9)$}
% 	\loigiai{
% 		Tần số lớn nhất là $24$ nên nhóm chứa mốt là nhóm $[8,5;9)$. \\
% 	}       
% \end{ex}
% %11
% \begin{ex}%[1K3Y9-4]
% 	Để kiểm tra thời gian sử dụng pin của   chiếc điện thoại mới, bạn A thống kê thời gian sử dụng điện thoại của mình từ lúc sạc đầy cho đến khi hết pin ở bảng sau
% 	\begin{center}
% 		\begin{tabular}{|l|c|c|c|c|c|c|c|c|}
% 			\hline Thời gian sử dụng (giờ) &{$[7;9)$}&{$[9;11)$}&{$[11;13)$}&{$[13;15)$}&{$[15;17)$}\\
% 			\hline Số lần & $2$ & $5$ & $7$ & $6$& $3$  \\
% 			\hline
% 		\end{tabular}
% 	\end{center}
% 	Nhóm chứa mốt của mẫu số liệu này là
% 	\choice
% 	{ $[9;11)$}
% 	{\True $[11;13)$}
% 	{ $[13;15)$}
% 	{$[15;17)$}
% 	\loigiai{
% 		Tần số lớn nhất là $7$ nên nhóm chứa mốt là nhóm $[11;13)$. \\
% 	}        
% \end{ex}
% %12
% \begin{ex}%[1K3B9-4]
% 	Để kiểm tra thời gian sử dụng pin của   chiếc điện thoại mới, bạn A thống kê thời gian sử dụng điện thoại của mình từ lúc sạc đầy cho đến khi hết pin ở bảng sau
% 	\begin{center}
% 		\begin{tabular}{|l|c|c|c|c|c|c|c|c|}
% 			\hline Thời gian sử dụng (giờ) &{$[7;9)$}&{$[9;11)$}&{$[11;13)$}&{$[13;15)$}&{$[15;17)$}\\
% 			\hline Số lần & $2$ & $5$ & $7$ & $6$& $3$  \\
% 			\hline
% 		\end{tabular}
% 	\end{center}   
% 	Xác định mốt của mẫu số liệu ghép nhóm này.    
% 	\choice
% 	{ $M_0\approx 11,67$}
% 	{$M_0\approx 12$}
% 	{\True $M_0\approx 12,33$}
% 	{$M_0\approx 7$}
% 	\loigiai{
% 		Tần số lớn nhất là $7$ nên nhóm chứa mốt là nhóm $[11;13)$. \\
% 		Ta có, $j=3, a_3=11, m_3=7$, $m_2=5, m_4=6, h=2$. Do đó
% 		$$
% 		M_0=11+\frac{7-5}{(7-5)+(7-6)}\cdot 2\approx 12,33.
% 		$$
% 	}
% \end{ex}
% %13
% \begin{ex}%[1K3Y9-4]
% 	Tổng lượng mưa trong tháng 8 đo được tại một trạm quan trắc đặt tại Vũng Tàu từ năm 2002 đến năm 2020 được ghi lại như sau (đơn vị: mm)
% 	\begin{center}
% 		\begin{tabular}{|l|c|c|c|c|c|c|c|c|}
% 			\hline Tổng lượng mưa trong tháng 8 (mm) &{$[120;175)$}&{$[175;230)$}&{$[230;285)$}&{$[285;340)$}\\
% 			\hline Số năm & $10$ & $5$ & $3$ & $1$  \\
% 			\hline
% 		\end{tabular}
% 	\end{center}  
% 	Nhóm chứa mốt của mẫu số liệu này là
% 	\choice
% 	{ $[175;230)$}
% 	{ $[230;285)$}
% 	{\True $[120;175)$}
% 	{$[285;340)$}
% 	\loigiai{
% 		Tần số lớn nhất là $10$ nên nhóm chứa mốt là nhóm $[120;175)$. \\
% 	}        
% \end{ex}
% %14
% \begin{ex}%[1K3B9-4]
% 	Tổng lượng mưa trong tháng 8 đo được tại một trạm quan trắc đặt tại Vũng Tàu từ năm 2002 đến năm 2020 được ghi lại như sau (đơn vị: mm)
% 	\begin{center}
% 		\begin{tabular}{|l|c|c|c|c|c|c|c|c|}
% 			\hline Tổng lượng mưa trong tháng 8 (mm) &{$[120;175)$}&{$[175;230)$}&{$[230;285)$}&{$[285;340)$}\\
% 			\hline Số năm & $10$ & $5$ & $3$ & $1$  \\
% 			\hline
% 		\end{tabular}
% 	\end{center}  
% 	Xác định mốt của mẫu số liệu ghép nhóm này.    
% 	\choice
% 	{ $M_0\approx 172,25$}
% 	{$M_0\approx 146,125$}
% 	{\True $M_0\approx 156,67$}
% 	{$M_0\approx 10$}
% 	\loigiai{
% 		Tần số lớn nhất là $10$ nên nhóm chứa mốt là nhóm $[120;175)$. \\
% 		Ta có, $j=1, a_1=120, m_1=10$, $m_2=5, m_0=0, h=55$. Do đó
% 		$$
% 		M_0=120+\frac{10-0}{(10-0)+(10-5)}\cdot 55\approx 156.67.
% 		$$
% 	}
% \end{ex}
% %15
% \begin{ex}%[1K3B9-4]
% 	Một công ty xây dựng khảo sát khách hàng xem họ có nhu cầu mua nhà ở mức giá nào. Kết quả khảo sát được ghi lại ở bảng sau (đơn vị: triệu đồng/$\mathrm{m}^2$
% 	\begin{center}
% 		\begin{tabular}{|l|c|c|c|c|c|c|c|c|}
% 			\hline Mức giá &{$[10;14)$}&{$[14;18)$}&{$[18;22)$}&{$[22;26)$}&{$[26;30)$}\\
% 			\hline Số khách hàng & $54$ & $78$ & $120$ & $45$& $12$  \\
% 			\hline
% 		\end{tabular}
% 	\end{center}   
% 	Xác định mốt của mẫu số liệu ghép nhóm này.    
% 	\choice
% 	{ $M_0\approx 18$}
% 	{\True $M_0\approx 19,4$}
% 	{ $M_0\approx 20$}
% 	{$M_0\approx 120$}
% 	\loigiai{
% 		Tần số lớn nhất là $120$ nên nhóm chứa mốt là nhóm $[18;22)$. \\
% 		Ta có, $j=3, a_3=18, m_3=120$, $m_2=78, m_4=45, h=4$. Do đó
% 		$$
% 		M_0=18+\frac{120-78}{(120-78)+(120-45)}\cdot 4\approx 19,4.
% 		$$
% 	}
% \end{ex}
% \begin{ex}%[1K3K9-4]
% 	Một công ty xây dựng khảo sát khách hàng xem họ có nhu cầu mua nhà ở mức giá nào. Kết quả khảo sát được ghi lại ở bảng sau (đơn vị: triệu đồng/$\mathrm{m}^2$
% 	\begin{center}
% 		\begin{tabular}{|l|c|c|c|c|c|c|c|c|}
% 			\hline Mức giá &{$[10;14)$}&{$[14;18)$}&{$[18;22)$}&{$[22;26)$}&{$[26;30)$}\\
% 			\hline Số khách hàng & $54$ & $78$ & $120$ & $45$& $12$  \\
% 			\hline
% 		\end{tabular}
% 	\end{center}   
% 	Công ty nên xây nhà ở mức giá nào để nhiều người có nhu cầu mua nhất?    
% 	\choice
% 	{ $ 18$ triệu đồng/$\mathrm{m}^2$}
% 	{\True $19,4$ triệu đồng/$\mathrm{m}^2$}
% 	{ $20$ triệu đồng/$\mathrm{m}^2$}
% 	{$21$ triệu đồng/$\mathrm{m}^2$}
% 	\loigiai{
% 		Tần số lớn nhất là $120$ nên nhóm chứa mốt là nhóm $[18;22)$. \\
% 		Ta có, $j=3, a_3=18, m_3=120$, $m_2=78, m_4=45, h=4$. Do đó
% 		$$
% 		M_0=18+\frac{120-78}{(120-78)+(120-45)}\cdot 4\approx 19,4.
% 		$$
% 		Dựa vào kết quả trên ta dự đoán rằng nếu công ty xây nhà ở mức giá $19,4$ triệu đồng/$\mathrm{m}^2$ thì sẽ có nhiều người có nhu cầu mua nhất.
% 	}
% \end{ex}
% \begin{ex}%[1K3Y9-4]
% 	Số cuộc gọi điện thoại một người thực hiện mỗi ngày trong 30 ngày được lựa chọn ngẫu nhiên được thống kê trong bảng sau
% 	\begin{center}
% 		\begin{tabular}{|l|c|c|c|c|c|c|c|c|}
% 			\hline Số cuộc gọi &{$[2,5;5,5)$}&{$[5,5;8,5)$}&{$[8,5;11,5)$}&{$[11,5;14,5)$}&{$[14,5;17,5)$}\\
% 			\hline Số ngày & $5$ & $13$ & $7$ & $3$& $2$  \\
% 			\hline
% 		\end{tabular}
% 	\end{center}   
% 	Nhóm chứa mốt của mẫu số liệu này là
% 	\choice
% 	{ $[2,5;5,5)$}
% 	{\True $[5,5;8,5)$}
% 	{ $[8,5;11,5)$}
% 	{$[11,5;14,5)$}
% 	\loigiai{
% 		Tần số lớn nhất là $13$ nên nhóm chứa mốt là nhóm $[5,5;8,5)$. \\
% 	}
% \end{ex}
% \begin{ex}%[1K3K9-4]
% 	Số cuộc gọi điện thoại một người thực hiện mỗi ngày trong 30 ngày được lựa chọn ngẫu nhiên được thống kê trong bảng sau
% 	\begin{center}
% 		\begin{tabular}{|l|c|c|c|c|c|c|c|c|}
% 			\hline Số cuộc gọi &{$[2,5;5,5)$}&{$[5,5;8,5)$}&{$[8,5;11,5)$}&{$[11,5;14,5)$}&{$[14,5;17,5)$}\\
% 			\hline Số ngày & $5$ & $13$ & $7$ & $3$& $2$  \\
% 			\hline
% 		\end{tabular}
% 	\end{center}   
% 	Hãy dự đoán xem khả năng người đó thực hiện bao nhiêu cuộc gọi mỗi ngày là cao nhất?   
% 	\choice
% 	{ $5$}
% 	{\True $7$}
% 	{ $6$}
% 	{$8$}
% 	\loigiai{
% 		Tần số lớn nhất là $13$ nên nhóm chứa mốt là nhóm $[5,5;8,5)$. \\
% 		Ta có, $j=2, a_2=5,5, m_2=13$, $m_1=5, m_3=7, h=3$. Do đó
% 		$$
% 		M_0=5,5+\frac{13-5}{(13-5)+(13-7)}\cdot 3\approx 7,2.
% 		$$
% 		Do đó ta có thể dự đoán khả năng người đó thực hiện $7$ cuộc gọi mỗi ngày là cao nhất.
% 	}
% \end{ex}

\Closesolutionfile{ans}