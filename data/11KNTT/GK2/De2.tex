\begin{name}
	{\tenchude}{\tendethi}{---vvv---}{\thoigian}
\end{name}
\setcounter{ex}{0}\setcounter{bt}{0}
%Câu 1
\begin{ex}
	Cho số thực dương $a$ và số nguyên dương $n$ tùy ý. Mệnh đề nào dưới đây đúng?
	\choice
	{$\sqrt{a^n}={a^{2+n}}$}
	{$\sqrt{a^n}={a^{2n}}$}
	{$\sqrt{a^n}={a^{\tfrac{2}{n}}}$}
	{$\sqrt{a^n}={a^{\tfrac{n}{2}}}$}
\end{ex}
%Câu 2
\begin{ex}
	Với $a$ là số thực dương tùy ý, $a^4 \cdot {a^{\tfrac{1}{2}}}$ bằng
	\choice
	{$a^8$}
	{$a^2$}
	{${a^{\tfrac{7}{2}}}$}
	{${a^{\tfrac{9}{2}}}$}
\end{ex}
%Câu 3
\begin{ex}
	Với $a$ là số thực dương tùy ý, $\sqrt{\dfrac{1}{a^3}}$ bằng
	\choice
	{${a^{-3}}$}
	{${a^{-\dfrac{3}{2}}}$}
	{${a^{\tfrac{3}{2}}}$}
	{${a^{\tfrac{1}{6}}}$}
\end{ex}
%Câu 4
\begin{ex}
	Cho $a$ là số thực dương. Viết biểu thức $P=a^2 \cdot {a^{\tfrac{2}{5}}}$ dưới dạng lũy thừa mũ hữu tỉ cơ số $a$ ta được kết quả là
	\choice
	{$P={a^{\tfrac{4}{5}}}$}
	{$P={a^{\tfrac{12}{5}}}$}
	{$P={a^{\tfrac{8}{5}}}$}
	{$P=a^5$}
\end{ex}
%Câu 5
\begin{ex}
	Cho $a$ là số thực dương. Viết biểu thức $P=\sqrt[3]{a^7} \cdot \dfrac{1}{\sqrt{a}}$ dưới dạng lũy thừa mũ hữu tỉ cơ số $a$ ta được kết quả là
	\choice
	{$P={a^{\tfrac{11}{6}}}$}
	{$P={a^{\tfrac{6}{11}}}$}
	{$P={a^{\tfrac{17}{6}}}$}
	{$P={a^{\tfrac{14}{3}}}$}
\end{ex}
%Câu 6
\begin{ex}
	Biết $\log _ab=3$. Tính $\log _a{b^{-2}}$.
	\choice
	{$-9$}
	{$6$}
	{$\dfrac{1}{9}$}
	{$-6$}
\end{ex}
%Câu 7
\begin{ex}
	Biết $\log _a7=-2$. Tính $\log _a49a$.
	\choice
	{3}
	{5}
	{$-4$}
	{$-3$}
\end{ex}
%Câu 8
\begin{ex}
	Biểu thức rút gọn của $\log _aM^2+\log _aM^3+\log _aM^4$ (với $a>0,a\ne 1,M>0$) là :
	\choice
	{$2\log _aM$}
	{$3\log _aM$}
	{$4\log _aM$}
	{$9\log _aM$}
\end{ex}
%Câu 9
\begin{ex}
	Rút gọn biểu thức $\log _3\left(x^3-1\right)-\log _3\left(x^2+x+1\right)+\log _3(x+1)$ với $x>1$, ta được:
	\choice
	{$\log _3\left(x^2-1\right)$}
	{$\log _3(x-1)$}
	{$\log _3(x+1)$}
	{$\log _3\left(x^2+x+1\right)$}
\end{ex}
%Câu 10
\begin{ex}
	Cho $a,b$ là các số thực dương thỏa mãn $a^2+b^2=23ab$. Khẳng định nào sau đây là sai?
	\choice
	{$2\log _5(a+b)=1+\log _5a+\log _5b$}
	{$\ln \dfrac{a+b}{5}=\dfrac{\ln a+\ln b}{2}$}
	{$\log _5(a+b)=1+\log_{25}a+{{\log }_{25}}b$}
	{$2\log _5\dfrac{a+b}{5}=\log _5a+\log _5b$}
\end{ex}
%Câu 11
\begin{ex}
	Hàm số nào sau đây đồng biến trên $\left(-\infty ;+\infty\right)$?
	\choice
	{$y=\left(\sqrt{5}-2\right)^x$}
	{$y={{\left(\dfrac{3}{\pi }\right)}^x}$}
	{$y={{\left(0{,}7\right)}^x}$}
	{$y={{\left(\dfrac{e}{2}\right)}^x}$}
\end{ex}
%Câu 12
\begin{ex}
Tập xác định của hàm số $y=\ln (x-2)$ là
\choice
{$\left[2;+\infty\right)$}
{$\left(2;+\infty\right)$}
{$\mathbb{R}$}
{$\mathbb{R}\backslash \{2\}$}
\end{ex}
%Câu 13
\begin{ex}
Hàm số nào sau đây đồng biến trên tập xác định của nó?
\choice
{$y={3^{-x}}$}
{$y={{\log }_{\tfrac{1}{2}}}x$}
{$y=\ln x$}
{$y={{\left(\dfrac{\sqrt{2}}{2}\right)}^x}$}
\end{ex}
%Câu 14
\begin{ex}
Phát biểu nào sau đây sai?
\choice
{Hàm số $y=a^x$ và hàm số $y=\log _ax$ $\left(a>1\right)$ có cùng tính đơn điệu trên tập xác định của nó}
	{Đồ thị hàm số $y=a^x$ $\left(a>0,a\ne 1\right)$ luôn nằm phía trên của trục hoành}
	{Đồ thị hàm số $y=\log _ax$ $\left(a>0,a\ne 1\right)$ luôn nằm phía bên phải của trục tung}
	{Hàm số $y=a^x$ và hàm số $y=\log _ax$ $\left(0<a<1\right)$ đều có đồ thị nằm phía trên của trục hoành}
\end{ex}
%Câu 15
\begin{ex}
	Gọi $M$ và $m$ lần lượt là giá trị lớn nhất và giá trị nhỏ nhất của hàm số $f(x)={e^{3x-2}}$ trên đoạn $[0;2]$. Mối liên hệ giữa $M$ và $m$ là
	\choice
	{$M \cdot m=e^2$}{$\dfrac{M}{m}=e^2$}{$M+m=e^2$}
	{$M-m=e^6$}
\end{ex}
%Câu 16
\begin{ex}
	Tìm tập hợp tất cả các giá trị của tham số $m$ để hàm số $y=\log _2\left(x^2-2mx+m+2\right)$ xác định trên $\mathbb{R}$.
	\choice
	{$m\in (-1;2)$}
	{$m\in (1;2)$}
	{$m\in \left(-\infty ;-1\right)\cup \left(2;+\infty\right)$}
	{$m\in \left(2;+\infty\right)$}
\end{ex}
%Câu 17
\begin{ex}
	Số nghiệm của phương trình $\log _2(x-2)=\log_{\tfrac{1}{2}}(3-x)$ là
	\choice
	{0}
	{1}
	{2}
	{3}
\end{ex}
%Câu 18
\begin{ex}
Nghiệm của phương trình ${3^{x-1}}=27$ là
\choice
{$x=4$}
{$x=3$}
{$x=2$}
{$x=1$}
\end{ex}
%Câu 19
\begin{ex}
Tập nghiệm của bất phương trình ${5^{x+1}}-\dfrac{1}{5}>0$ là
\choice
{$\left(1;+\infty\right)$}
{$\left(-1;+\infty\right)$}
{$\left(-\infty ;-2\right)$}
{$\left(-2;+\infty\right)$}
\end{ex}
%Câu 20
\begin{ex}
Tổng các nghiệm của phương trình ${2^{x^2}}=\dfrac{1}{{4^{x-4}}}$ bằng
\choice
{$-8$}
{$-1$}
{$-2$}
{$2$}
\end{ex}
%Câu 21
\begin{ex}
Tập nghiệm của bất phương trình ${{\left(0{,}2\right)}^{x^2+3}}\le {{\left(\dfrac{1}{5}\right)}^{4x}}$ là
\choice
{$[1;3]$}
{$\left(-\infty ;1\right]\cup \left[3;+\infty\right)$}
{$\left(-\infty ;-3\right]\cup \left[-1;+\infty\right)$}
{$[-3;-1]$}
\end{ex}
%Câu 22
\begin{ex}
Tập nghiệm của phương trình $\log _2\left(x^2-1\right)=3$ là
	\choice
	{$\{-3\}$}
	{$\left\{ -\sqrt{10};\sqrt{10} \right\}$}
	{$\{3\}$}
	{$\left\{ -3;3 \right\}$}
\end{ex}
%Câu 23
\begin{ex}
	Tập nghiệm ${S}$ của bất phương trình $\log_{\tfrac{1}{2}}(x+1)<{{\log }_{\tfrac{1}{2}}}(2x-1)$ là
	\choice
	{$S=\left(2;+\infty\right)$}
	{$S=\left(-\infty ;2\right)$}
	{$S=\left(\dfrac{1}{2};2\right)$}
	{$S=(-1;2)$}
\end{ex}
%Câu 24
\begin{ex}
Chọn mệnh đề đúng?
\choice
{Nếu hai đường thẳng vuông góc với nhau thì hai đường thẳng đó cắt nhau}
{Nếu hai đường thẳng vuông góc với nhau thì hai đường thẳng đó chéo nhau}
{Nếu hai đường thẳng vuông góc với nhau thì hai đường thẳng đó song song với nhau}
{Nếu hai đường thẳng vuông góc với nhau thì chúng hoặc chéo nhau hoặc cắt nhau}
\end{ex}
%Câu 25
\begin{ex}
Cho hình chóp $S \cdot ABCD$ có đáy là hình vuông $ABCD$ cạnh $a$ và các cạnh bên đều bằng $a$. Gọi $M$ và $N$ lần lượt là trung điểm của $AD$ và $SD$. Số đo góc $\left(\widehat{MN,SB}\right)$ bằng
\choice
{$45^\circ $}
{$30^\circ $}
{$90^\circ $}
{$60^\circ $}
\end{ex}
%Câu 26
\begin{ex}
Trong không gian các mệnh đề sau, mệnh đề nào sai?
\choice
{Hai đường thẳng phân biệt cùng vuông góc với một đường thẳng thứ ba thì song song với nhau}
{Hai đường thẳng phân biệt cùng song song với một đường thẳng thì song song với nhau}
{Một đường thẳng vuông góc với hai cạnh của tam giác thì sẽ vuông góc với cạnh thứ ba của tam giác đó}
{Hai đường thẳng vuông góc nếu góc giữa hai véc tơ chỉ phương của chúng bằng $90^\circ $}
\end{ex}
%Câu 27
\begin{ex}
Cho hình lập phương $ABCD.A_1B_1C_1D_1$. Góc giữa $AC$ và $DA_1$ là
	\choice
	{$90^\circ $}{$60^\circ $}{$45^\circ $}
	{$120^\circ $}
\end{ex}
%Câu 28
\begin{ex}
	Cho hình chóp $S \cdot ABC$ có $SA$ vuông góc với mặt đáy $(ABC)$. Mệnh đề nào sau đây là đúng?
	\choice
	{$SA\bot SB$}
	{$SA\bot SC$}
	{$SA\bot BC$}
	{$SB\bot SC$}
\end{ex}
%Câu 29
\begin{ex}
	Mệnh đề nào sau đây sai?
	\choice
	{Hai đường thẳng phân biệt cùng vuông góc với một mặt phẳng thì song song}
	{Hai mặt phẳng phân biệt cùng vuông góc với một đường thẳng thì song song}
	{Cho đường thẳng $\triangle $ song song với mặt phẳng $\left(\alpha\right)$. Đường thẳng $d$ vuông góc với đường thẳng $\triangle $ thì đường thẳng $d$ cũng vuông góc với mặt phẳng $\left(\alpha\right)$}
	{Một đường thẳng và một mặt phẳng (không chứa đường thẳng đã cho) cùng vuông góc với một đường thẳng thì song song nhau}
\end{ex}
%Câu 30
\begin{ex}
	Cho hình chóp $S \cdot ABC$ có $SA\bot \,(ABC)$, tam giác $ABC$ vuông tại $B$. Mệnh đề nào sau đây sai?
	\choice
	{$SB\bot AC$}{$SA\bot AB$}{$SB\bot BC$}
	{$SA\bot BC$}
\end{ex}
%Câu 31
\begin{ex}
	Cho tứ diện $ABCD$ có $AB=AC$ và $DB=DC$. Khẳng định nào sau đây là đúng?
	\choice
	{$AB\bot (ABC)$}
	{$AC\bot BC$}
	{$CD\bot (ABC)$}
	{$BC\bot AD$}
\end{ex}
%Câu 32
\begin{ex}
	Cho hình chóp $S \cdot ABC$ có $SA$ vuông góc với đáy. Góc giữa đường thẳng $SB$ và $(ABC)$ là:
	\choice
	{$\widehat{SBC}$}
	{$\widehat{SCA}$}
	{$\widehat{SAB}$}
	{$\widehat{SBA}$}
\end{ex}
%Câu 33
\begin{ex}
	Cho hình chóp $S \cdot ABCD$ có $ABCD$ là hình vuông, O là giao điểm của AC và BD, $SO$ vuông góc với đáy. Góc giữa đường thẳng $SD$ và $(ABCD)$ là?
	\choice
	{$\widehat{SOC}$}
	{$\widehat{SDO}$}
	{$\widehat{SAB}$}
	{$\widehat{SOA}$}
\end{ex}
%Câu 34
\begin{ex}
	Trong các khẳng định sau, khẳng định nào là đúng?
	\choice
	{Hình lăng trụ tứ giác đều là hình lập phương}
	{Hình lăng trụ đứng là hình lăng trụ đều}
	{Hình lăng trụ có đáy là một đa giác đều là một hình lăng trụ đều}
	{Hình lăng trụ đứng có đáy là một đa giác đều là hình lăng trụ đều}
\end{ex}
%Câu 35
\begin{ex}
	Cho hình chóp $S \cdot ABCD$ có đáy $ABCD$ là hình vuông cạnh $2a$. Đường thẳng $SA$ vuông góc với mặt phẳng đáy. Gọi M là trung điểm của $CD$. Khoảng cách từ $M$ đến mặt phẳng $(SAB)$ nhận giá trị nào sau đây?
	\choice
	{$\dfrac{a\sqrt{2}}{2}$}
	{$a$}
	{$a\sqrt{2}$}
	{$2a$}
\end{ex}
%Câu 36
\begin{ex}
	Cho hình chóp $S \cdot ABCD$ có đáy $ABCD$ là hình vuông cạnh $a$. Hình chiếu vuông góc của $S$ trên mặt phẳng $(ABCD)$ là trung điểm $H$ của cạnh $AB$. Gọi $M$, $N$ lần lượt là trung điểm của cạnh $AD$, $SC$.\\
	a) Chứng minh $CM$ vuông góc với mặt phẳng $(SHD)$.\\
	b) Tính khoảng cách giữa $MN$ và $HC$ biết rằng $SH=2a$
	\end{ex}
%Câu 37
\begin{ex}
Giải phương trình $\log _2\left(2x^2+3\right)=\log _2(3x+2)$
\end{ex}
%Câu 38
\begin{ex}
Cho $a,b$ là các số thực dương và $a$ khác $1$. Rút gọn biểu thức:$P=\sqrt{\log _a^2(ab)-\dfrac{2\ln b}{\ln a}-1}$
\end{ex}
%Câu 39
\begin{ex}
Cho phương trình $\left(\log _3\left(\dfrac{x}{3}\right)\right)^2+3 m \log _3 x+2 m^2-2 m-1=0$, ($m$ là tham số). Tìm tất cả các giá trị nguyên của tham số $m$ lớn hơn $-2024$ sao cho phương trình đã cho có hai nghiệm phân biệt $x_1$, $x_2$ thỏa mãn $x_1+x_2>10$
\end{ex}
