\subsection{BÀI TẬP TRẮC NGHIỆM}
% \ind{PHẦN I.} \inden{Câu trắc nghiệm nhiều phương án lựa chọn. Học sinh trả lời từ câu 1 đến câu 12. Mỗi câu hỏi học sinh chỉ chọn một phương án.}\\
\TN
\setcounter{ex}{0}
\Opensolutionfile{ans}[ans/1D5-B2-1]
\begin{ex}%[1D3N2-1]
	Kết quả của giới hạn 
	$\lim \limits_{x \rightarrow-\infty} \dfrac{1}{x^k}$ (với $k$ nguyên dương) là
	\choice
	{$+\infty$}
	{$-\infty$}
	{$x$}
	{\True $0$ }
	\loigiai{
		Ta có $\lim \limits_{x \rightarrow-\infty} \dfrac{1}{x^k}=0$ (với $k$ nguyên dương). 
	}
\end{ex}

\begin{ex}%[1D3N2-1]
	Giới hạn nào dưới đây có kết quả bằng $3$?
	\choice
	{$\lim \limits_{x \rightarrow 1} \dfrac{3 x}{x-2}$}
	{$\lim \limits_{x \rightarrow 1} \dfrac{-3 x}{2-x}$}
	{\True $\lim \limits_{x \rightarrow 1} \dfrac{-3 x}{x-2}$}
	{$\lim \limits_{x \rightarrow 1} \dfrac{3 x^2}{x-2}$}
	\loigiai{
		Ta có $\lim \limits_{x \rightarrow 1}(-3 x)=-3$ và $\lim \limits_{x \rightarrow 1}(x-2)=-1$ nên $\lim \limits_{x \rightarrow 1} \dfrac{-3 x}{x-2}=\dfrac{\lim \limits_{x \rightarrow 1}(-3 x)}{\lim \limits_{x \rightarrow 1}(x-2)}=\dfrac{-3}{-1}=3$.}
	
\end{ex}

\begin{ex}%[1D3N2-2]
	Giá trị của $I=\lim \limits_{x \rightarrow 1}\left(\dfrac{x^2-4 x+7}{x+1}\right)$
	\choice
	{$I=-4$}
	{$I=5$}
	{$I=4$}
	{\True $I=2$}
	\loigiai{
		Ta có $I=\lim \limits_{x \rightarrow 1}\left(\dfrac{x^2-4 x+7}{x+1}\right)=\dfrac{1^2-4 \cdot 1+7}{1+1}=\dfrac{4}{2}=2$.}
\end{ex}
%
\begin{ex}%[1D3N2-2]
	Giá trị của $\lim \limits_{x \rightarrow-4} \dfrac{x^2+3 x-4}{x^2+4 x}$ bằng
	\choice
	{ $1$ }
	{$-1$ }
	{\True $\dfrac{5}{4}$}
	{$-\dfrac{5}{4}$}
	\loigiai{
		$\lim \limits_{x \rightarrow-4} \dfrac{x^2+3 x-4}{x^2+4 x}=\lim \limits_{x \rightarrow-4} \dfrac{(x-1)(x+4)}{x(x+4)}=\lim \limits_{x \rightarrow-4} \dfrac{(x-1)}{x}=\dfrac{5}{4} $. }
	
\end{ex}
\begin{ex}%[1D3N2-7]
	Cho hàm số $f(x)=\heva{&x^2-1&  \text{ khi } x \geq 1 \\ & 2x+1 & \text{ khi } x < 1 }$. Mệnh đề nào sau đây là \textbf{đúng}?
	\choice
	{$\lim \limits_{x  \rightarrow 1^{-}} f(x)=0$}
	{\True $\lim \limits_{x \rightarrow 1^{-}} f(x)=3$}
	{$\lim \limits_{x \rightarrow 1^{-}} f(x)=-1$}
	{$\lim \limits_{x \rightarrow 1} f(x)=0$}
	\loigiai{
		Do $x  \rightarrow 1^{-} $ nên $x<1$. Ta có: $\lim \limits_{x  \rightarrow 1^{-}} f(x)=\lim \limits_{x  \rightarrow 1^{-}}(2 x+1)=2 \cdot 1+1=3$.\\
		Suy ra phương án $\lim \limits_{x  \rightarrow 1^{-}} f(x)=0$ và $\lim \limits_{x \rightarrow 1^{-}} f(x)=-1$ sai.\\
		Do $x \rightarrow  1^{+}$nên $x>1$. Ta có: $\lim \limits_{x \rightarrow 1^{+}} f(x)=\lim \limits_{x \rightarrow  1^{+}}\left(x^2-1\right)=1^2-1=0$.\\
		$\Rightarrow \lim \limits_{x \rightarrow  1} f(x) \ne  \lim \limits_{x \rightarrow  1^{+}} f(x)$ nên $\lim \limits_{x \rightarrow 1} f(x)$ không tồn tại.\\
		Suy ra phương án $\lim \limits_{x \rightarrow 1} f(x)=0$ sai.}
	
\end{ex}

\begin{ex}%[1D3N2-7]
	Giá trị của $\lim \limits_{x \rightarrow 3^{+}} \dfrac{2 x+7}{x-3}$ là
	\choice
	{\True $+\infty$}
	{$-\infty$}
	{$0$ }
	{$2$} 
	\loigiai{
		Ta có: $\lim \limits_{x \rightarrow 3^{+}}(2 x+7)=13>0, \lim \limits_{x \rightarrow 3^{+}}(x-3)=0, x \rightarrow 3^{+} \Rightarrow x-3>0$.\\
		Vậy $\lim \limits_{x \rightarrow 3^{+}} \dfrac{2 x+7}{x-3}=+\infty$.}
\end{ex}

\begin{ex}%[1D3N2-3]
	Giá trị của $\lim \limits_{x \rightarrow+\infty} \dfrac{1-3 x}{2 x+3}$
	\choice
	{$-3$ }
	{$\dfrac{1}{2}$}
	{\True $-\dfrac{3}{2}$}
	{$-\infty$}
	\loigiai{
		$\lim \limits_{x \rightarrow+\infty} \dfrac{1-3 x}{2 x+3}=\lim \limits_{x \rightarrow+\infty} \dfrac{\dfrac{1}{x}-3}{2+\dfrac{3}{x}}=-\dfrac{3}{2}.$}
\end{ex}

\begin{ex}%[1D3N2-3]
	Giá trị của $\lim \limits_{x \rightarrow-\infty} \dfrac{x}{x^2+1}$ bằng
	\choice
	{$-\infty$}
	{$1$} 
	{$+\infty$}
	{\True $0$ }
	\loigiai{
		Ta có: $\lim \limits_{x \rightarrow-\infty} \dfrac{x}{x^2+1}=\lim \limits_{x \rightarrow-\infty} \dfrac{\dfrac{1}{x}}{1+\dfrac{1}{x^2}}=0$.}
\end{ex}

\begin{ex}%[1D3N2-3]
	Giá trị đúng của $\lim \limits_{x \rightarrow+\infty} \dfrac{x^4+7}{x^4+1}$ là
	\choice 
	{$-1$} 
	{\True $1$} 
	{$7$} 
	{$+\infty$}
	\loigiai{
		$\lim \limits_{x \rightarrow+\infty} \dfrac{x^4+7}{x^4+1}=\lim \limits_{x \rightarrow+\infty} \dfrac{1+\dfrac{7}{x^4}}{1+\dfrac{1}{x^4}}=1.
		$}
\end{ex}

\begin{ex}%[1D3N2-3]
	Giá trị của $\lim \limits_{x \rightarrow-\infty} \dfrac{2 x-1}{\sqrt{x^2+1}-1}$ bằng
	\choice
	{$0$}
	{\True $-2$} 
	{$-\infty$}
	{$2$} 
	\loigiai{
		Ta có: $\lim \limits_{x \rightarrow-\infty} \dfrac{2 x-1}{\sqrt{x^2+1}-1}=\lim \limits_{x \rightarrow-\infty} \dfrac{2 x-1}{-x \sqrt{1+\dfrac{1}{x^2}}-1}=\lim \limits_{x \rightarrow-\infty} \dfrac{2-\dfrac{1}{x}}{-\sqrt{1+\dfrac{1}{x^2}}-\dfrac{1}{x}}=-2$.}
	
\end{ex}

\begin{ex}%[1D3N2-7]
	Giá trị của $\lim \limits_{x \rightarrow 1^{+}} \dfrac{x^2-x+1}{x^2-1}$ bằng
	\choice
	{$-\infty$}
	{$-1$} 
	{$1$} 
	{\True $+\infty$}
	\loigiai{
		$\lim \limits_{x \rightarrow 1^{+}} \dfrac{x^2-x+1}{x^2-1}=+\infty$ \text  vì \\
		$\lim \limits_{x \rightarrow 1^{+}}\left(x^2-x+1\right)=1>0 \text { và } \lim \limits_{x \rightarrow 1^{+}}\left(x^2-1\right)=0 ; x^2-1>0, \forall x>1. $}
	
\end{ex}

\begin{ex}%[1D3N2-7]
	Chọn kết quả đúng của $\lim \limits_{x \rightarrow 0^{+}}\left(\dfrac{1}{x^2}-\dfrac{2}{x^3}\right)$ 
	\choice
	{\True $-\infty$}
	{$0$}
	{$+\infty$}
	{Không tồn tại}
	\loigiai{
		$\lim \limits_{x \rightarrow 0^{+}}\left(\dfrac{1}{x^2}-\dfrac{2}{x^3}\right)=\lim \limits_{x \rightarrow 0^{+}} \dfrac{x-2}{x^3}$
		$\lim \limits_{x \rightarrow 0^{+}} x^3=0, x^3>0$ với mọi $x>0$ và $\lim \limits_{x \rightarrow 0^{+}}(x-2)=-2<0$.
		Do đó, $\lim \limits_{x \rightarrow 0^{+}}\left(\dfrac{1}{x^2}-\dfrac{2}{x^3}\right)=\lim \limits_{x \rightarrow 0^{+}} \dfrac{x-2}{x^3}=-\infty$.}
\end{ex}
\Closesolutionfile{ans}

% \ind{PHẦN II.} \inden{Câu trắc nghiệm đúng sai. Học sinh trả lời từ câu 1 đến câu 4. Trong mỗi ý a), b), c), d) ở mỗi câu, học sinh chọn đúng hoặc sai.}\\
\TNTF
\setcounter{ex}{0}
\Opensolutionfile{ans}[ans/1D5-B2-2]
\begin{ex}%[1D3H2-7]
	Cho hàm số $f(x)=\left\{\begin{array}{ll}x-2 & \text { khi } x<-1 \\ \sqrt{x^2+1} & \text { khi } x \geq-1\end{array}\right.$. Các mệnh đề sau đúng hay sai?
	\choiceTF
	{$\lim \limits_{x \rightarrow-2} f(x)=\sqrt{5}$}
	{\True $\lim \limits_{x \rightarrow-1^{-}} f(x)=-3$}
	{\True $\lim \limits_{x \rightarrow-1^{+}} f(x)=\sqrt{2}$}
	{Hàm số tồn tại giới hạn khi $x \rightarrow-1$}
	\loigiai{
		
		\begin{enumerate}
			\item Ta có: $\lim \limits_{x \rightarrow-2} f(x)=\lim \limits_{x \rightarrow-2}(x-2)=-2-2=-4$
			\item Ta có: $\lim \limits_{x \rightarrow-1^{-}} f(x)=\lim \limits_{x \rightarrow-1^{-}}(x-2)=-1-2=-3$
			\item Khi đó: $\lim \limits_{x \rightarrow-1^{+}} f(x)=\lim \limits_{x \rightarrow-1^{+}} \sqrt{x^2+1}=\sqrt{(-1)^2+1}=\sqrt{2}$.
			\item Vì $\lim \limits_{x \rightarrow-1^{-}} f(x) \neq \lim \limits_{x \rightarrow-1^{+}} f(x)$ (hay $-3 \neq \sqrt{2}$ ) nên không tồn tại $\lim \limits_{x \rightarrow-1} f(x)$.
		\end{enumerate}
	}
\end{ex}

\begin{ex}%[1D3H2-3]
	Các mệnh đề sau đúng hay sai?
	\choiceTF
	{\True $\lim \limits_{x \rightarrow-\infty}\left(x^2-10 x\right)=+\infty$}
	{\True $\lim \limits_{x \rightarrow+\infty} \dfrac{3 x^2-4 x+1}{2 x^2+x+1}=\dfrac{3}{2}$}
	{$\lim \limits_{x \rightarrow-\infty} \dfrac{\sqrt{x^2+x+1}-3 x}{2-3 x}=\dfrac{5}{4}$}
	{\True Để $\lim \limits_{x \rightarrow-\infty}\left(\sqrt{2 x^2+1}+a x\right)=+\infty$ thì $a<\sqrt{2}$}
	\loigiai{
		
		\begin{enumerate}
			\item Ta có:
			$\lim \limits_{x \rightarrow-\infty}\left(x^2-10 x\right)=\lim \limits_{x \rightarrow-\infty}\left[x^2\left(1-\dfrac{10}{x}\right)\right]=\lim \limits_{x \rightarrow-\infty} x^2 \cdot \lim \limits_{x \rightarrow-\infty}\left(1-\dfrac{10}{x}\right)$\\
			Do $\left\{\begin{array}{l}+\lim \limits_{x \rightarrow-\infty} x^2=+\infty \\ +\lim \limits_{x \rightarrow-\infty}\left(1-\dfrac{10}{x}\right)=1>0\end{array} \Rightarrow \lim \limits_{x \rightarrow-\infty}\left(x^2-10 x\right)=+\infty\right.$\\
			Vậy $\lim \limits_{x \rightarrow-\infty}\left(x^2-10 x\right)=+\infty$ đúng.
			\item $\lim \limits_{x \rightarrow+\infty} \dfrac{3 x^2-4 x+1}{2 x^2+x+1}=\lim \limits_{x \rightarrow+\infty} \dfrac{x^2\left(3-\dfrac{4}{x}+\dfrac{1}{x^2}\right)}{x^2\left(2+\dfrac{1}{x}+\dfrac{1}{x^2}\right)}=\lim \limits_{x \rightarrow+\infty} \dfrac{3-\dfrac{4}{x}+\dfrac{1}{x^2}}{2+\dfrac{1}{x}+\dfrac{1}{x^2}}=\dfrac{3}{2}$.\\ Vậy $\lim \limits_{x \rightarrow+\infty} \dfrac{3 x^2-4 x+1}{2 x^2+x+1}=\dfrac{3}{2}$ đúng.
			\item 
			$
			\begin{aligned}
				\lim \limits_{x \rightarrow-\infty} \dfrac{\sqrt{x^2+x+1}-3 x}{2-3 x} & =\lim \limits_{x \rightarrow-\infty} \dfrac{\sqrt{x^2\left(1+\dfrac{1}{x}+\dfrac{1}{x^2}\right)}-3 x}{x\left(\dfrac{2}{x}-3\right)}=\lim \limits_{x \rightarrow-\infty} \dfrac{-x \sqrt{1+\dfrac{1}{x}+\dfrac{1}{x^2}}-3 x}{x\left(\dfrac{2}{x}-3\right)} \\
				& =\lim \limits_{x \rightarrow-\infty} \dfrac{-\sqrt{1+\dfrac{1}{x}+\dfrac{1}{x^2}}-3}{\dfrac{2}{x}-3}=\dfrac{-\sqrt{1}-3}{-3}=\dfrac{4}{3}.
			\end{aligned}
			$
			
			Vậy $\lim \limits_{x \rightarrow-\infty} \dfrac{\sqrt{x^2+x+1}-3 x}{2-3 x}=\dfrac{5}{4}$ sai.
			
			\item Để $\lim \limits_{x \rightarrow-\infty}\left(\sqrt{2 x^2+1}+a x\right)=+\infty$ thì $a<\sqrt{2}$ đúng.
		\end{enumerate}
	}
\end{ex}
\begin{ex}%[1D3H2-5]
	Cho hàm số $f(x)=\dfrac{\sqrt{3 x^2+1}}{x+1}$ và $g(x)=\dfrac{2 x}{x+1}$. Các mệnh đề sau đúng hay sai?
	\choiceTF
	{$\lim \limits_{x \rightarrow 1} f(x)=2$ và $\lim \limits_{x \rightarrow-2} g(x)=-2$}
	{\True $\lim \limits_{x \rightarrow+\infty} g(x)=2 ; \lim \limits_{x \rightarrow+\infty} f(x)=\sqrt{3}$}
	{$\lim \limits_{x \rightarrow-\infty}[f(x)-g(x)]=\sqrt{3}-2$}
	{\True $\lim \limits_{x \rightarrow-1}[f(x)-g(x)]=\dfrac{1}{2}$}
	\loigiai{
		
		\begin{enumerate}
			\item Ta có $\lim \limits_{x \rightarrow 1} f(x)=\lim \limits_{x \rightarrow 1} \dfrac{\sqrt{3 x^2+1}}{x+1}=\dfrac{\sqrt{3 \cdot 1^2+1}}{1+1}=1$.\\
			Và $\lim \limits_{x \rightarrow-2} g(x)=\lim \limits_{x \rightarrow-2} \dfrac{2 x}{x+1}=\lim \limits_{x \rightarrow-2} \dfrac{2 \cdot (-2)}{-2+1}=4$.\\
			Do đó ý $\lim \limits_{x \rightarrow 1} f(x)=2$ và $\lim \limits_{x \rightarrow-2} g(x)=-2$ sai.
			\item Ta có $\lim \limits_{x \rightarrow+\infty} g(x)=\lim \limits_{x \rightarrow+\infty} \dfrac{2 x}{x+1}=\lim \limits_{x \rightarrow+\infty} \dfrac{2}{1+\dfrac{1}{x}}=2$.\\
			$\lim \limits_{x \rightarrow+\infty} f(x)=\lim \limits_{x \rightarrow+\infty} \dfrac{\sqrt{3 x^2+1}}{x+1}=\lim \limits_{x \rightarrow+\infty} \dfrac{x \sqrt{3+\dfrac{1}{x^2}}}{x+1}=\lim \limits_{x \rightarrow-\infty} \dfrac{\sqrt{3+\dfrac{1}{x^2}}}{1+\dfrac{1}{x}}=\sqrt{3}$.\\
			Vậy ý $\lim \limits_{x \rightarrow+\infty} g(x)=2$; $\lim \limits_{x \rightarrow+\infty} f(x)=\sqrt{3}$ đúng.
			\item Ta có $\lim \limits_{x \rightarrow-\infty}[f(x)-g(x)]=\lim \limits_{x \rightarrow-\infty} f(x)-\lim \limits_{x \rightarrow-\infty} g(x)=-\sqrt{3}-2=-\sqrt{3}-2$.
			Vậy ý $\lim \limits_{x \rightarrow-\infty}[f(x)-g(x)]=\sqrt{3}-2$ sai.
			\item Ta có
			$$
			\begin{aligned}
				& \lim \limits_{x \rightarrow-1}[f(x)+g(x)]=\lim \limits_{x \rightarrow-1} \dfrac{\sqrt{3 x^2+1}+2 x}{x+1}=\lim \limits_{x \rightarrow-1} \dfrac{\left(\sqrt{3 x^2+1}+2 x\right)\left(\sqrt{3 x^2+1}-2 x\right)}{(x+1)\left(\sqrt{3 x^2+1}-2 x\right)}= \\
				& =\lim \limits_{x \rightarrow-1} \dfrac{1-x^2}{(x+1)\left(\sqrt{3 x^2+1}-2 x\right)}=\lim \limits_{x \rightarrow-1} \dfrac{(1-x)(1+x)}{(x+1)\left(\sqrt{3 x^2+1}-2 x\right)}=\lim \limits_{x \rightarrow-1} \dfrac{1-x}{\sqrt{3 x^2+1}-2 x}=\dfrac{1}{2}
			\end{aligned}
			$$
			
			Vậy ý $\lim \limits_{x \rightarrow-1}[f(x)-g(x)]=\dfrac{1}{2}$ đúng.
		\end{enumerate}
	}
\end{ex}

\begin{ex}%[1D3H2-7]
	Cho hàm số $f(x)=\heva{&2x+1 \text{ khi } x \leq 1 \\ & \sqrt{x^2+a} \text{ khi } x>1}$. Khẳng định nào đúng, khẳng định nào sai? 
	\choiceTF
	{\True $\lim \limits_{x \in-5} f(x)=-9$}
	{$\lim \limits_{x \in 1} f(x)=1$ và $\lim \limits_{x \in 1^{+}} f(x)=a$}
	{$\lim \limits_{x \in 1^{+}} \dfrac{f(x)}{1-x}=+\infty$}
	{\True Để tồn tại $\lim \limits_{x \to 1} f(x)$ thì $a=8$}
	\loigiai{
		
		\begin{enumerate}
			\item $\lim \limits_{x \in-5} f(x)=-9$ đúng.
			\item Ta có
			$\lim \limits_{x \to 1} f(x)=\lim \limits_{x \to 1}(2 x+1)=3$
			$+\lim \limits_{x \to 1^{+}} f(x)=\lim \limits_{x \to 1^{+}} \sqrt{x^2+a}=\sqrt{1+a}$. Vậy ý $\lim \limits_{x \in 1} f(x)=1$ và $\lim \limits_{x \in 1^{+}} f(x)=a$ sai.
			\item Ta có $\lim \limits_{x \to 1^{+}} \dfrac{f(x)}{1-x}=\lim \limits_{x \to 1^{+}} \dfrac{\sqrt{x^2+a}}{1-x}$
			$\lim \limits_{x \to 1^{+}} \sqrt{x^2+a}=\sqrt{1+a}>0,  a>0$.\\
			$\lim \limits_{x \to 1^{+}}(1-x)=0$.\\
			$x \rightarrow 1^{+} \Rightarrow 1-x<0$.\\
			Suy ra $\lim \limits_{x \to 1^{+}} \dfrac{f(x)}{1-x}=\lim \limits_{x \to 1^{+}} \dfrac{\sqrt{x^2+a}}{1-x}=-\infty $.\\ Vậy $\lim \limits_{x \in 1^{+}} \dfrac{f(x)}{1-x}=+\infty$ sai.
			\item Để tồn tại $\lim \limits_{x \to 1} f(x)$ thì $\lim \limits_{x \to 1} f(x)=f(1)$; $f(1)=3$.
			$$
			\begin{aligned}
				& +\lim \limits_{x \in 1} f(x)=\lim \limits_{x \in 1^1}(2 x+1)=3 \\
				& +\lim \limits_{x \in 1^{+}} f(x)=\lim \limits_{x \in 1^{+}} \sqrt{x^2+a}=\sqrt{1+a} .
			\end{aligned}
			$$
			Suy ra để hàm số tồn tại $\lim \limits_{x \to 1} f(x)$ thì $\sqrt{1+a}=3  \quad a=8$.\\ Vậy ý Để tồn tại $\lim \limits_{x \to 1} f(x)$ thì $a=8$ đúng.
		\end{enumerate}
	}
\end{ex}
\Closesolutionfile{ans}

% \ind{PHẦN III.} \inden{Câu trắc nghiệm trả lời ngắn. Học sinh trả lời từ câu 1 đến câu 6.}\\
\TNSA
\setcounter{ex}{0}
\Opensolutionfile{ans}[ans/1D5-B2-3]

\begin{ex}%[1D3V2-3]
	Cho $\lim \limits_{x \rightarrow+\infty}\left(\dfrac{x^2+3 x+1}{x+1}+a x+b\right)=1$. Tính giá trị của biểu thức $T=2024 a-4049 b$. \\
	\shortans[oly]{$2025$}
	\loigiai{
		Ta  có
		\begin{eqnarray*}
			&& \lim \limits_{x \rightarrow+\infty}\left(\dfrac{x^2+3 x+1}{x+1}+a x+b\right)=1\\ 
			&\Leftrightarrow& \lim \limits_{x \rightarrow+\infty}\left(\dfrac{(a+1) x^2+(a+b+3) x+b+1}{x+1}\right)=1 \\
			& \Leftrightarrow& \lim \limits_{x \rightarrow+\infty}\left(\dfrac{(a+1) x+(a+b+3)+\dfrac{b+1}{x}}{1+\dfrac{1}{x}}\right)=1 \\
			& \Leftrightarrow&\heva{& a + 1 = 0 \\& a + b + 3 = 1 } \Leftrightarrow \heva{&a=-1 \\&	b=-1.}
		\end{eqnarray*}
		Do đó 	$T=2024 a-4049 b=2024\cdot (-1)-4049\cdot (-1)=2025$.
	}
\end{ex}

\begin{ex}%[1D3V2-3]
	Biết $\lim \limits_{x \rightarrow+\infty}\left(\sqrt{3 x^2+2023 x+2}-n \sqrt{3}\right)=\dfrac{a \sqrt{b}}{c}$, ở đó $a, c$ là các số nguyên dương nguyên tố cùng nhau, $b$ là số nguyên tố. Tính $a+b+c$.\\
	\shortans[oly]{$2032$}
	\loigiai{
		Ta có 
		$$\dfrac{a \sqrt{b}}{c}=\lim \limits_{x \rightarrow+\infty} \dfrac{2023 x+2}{\sqrt{3 x^2+2023 x+2}+x \sqrt{3}}=\lim \limits_{x \rightarrow+\infty} \dfrac{2023+\dfrac{2}{x}}{\sqrt{3+\dfrac{2023}{x}+\dfrac{2}{x^2}}+\sqrt{3}}=\dfrac{2023 \sqrt{3}}{6}.$$
		Vậy $a+b+c=2023+3+6=2032$.}
\end{ex}

\begin{ex}%[1D3V2-3]
	Biết $\lim \limits_{x \rightarrow+\infty}\left(\dfrac{x^2+1}{x-2}+a x-b\right)=-5$. Tính giá trị biểu thức $P=\dfrac{1}{2} a+\dfrac{1}{4} b$.\\
	\shortans[oly]{$1{,}25$}
	\loigiai{
		Ta có $\lim \limits_{x \rightarrow+\infty}\left(\dfrac{x^2+1}{x-2}+a x-b\right)=\lim \limits_{x \rightarrow+\infty}\left(\dfrac{(a+1) x^2-(2 a+b) x+2 b+1}{x-2}\right)=-5$.\\
		Vì giới hạn là hữu hạn và bằng $-5$ nên 
		$$\heva{&a+1=0 \\& 2a+b=5} \Leftrightarrow\heva{&a=-1 \\&b=7.}$$
		Vậy $P=\dfrac{1}{2} a+\dfrac{1}{4} b=\dfrac{1}{2} \cdot(-1)+\dfrac{1}{4} \cdot 7=1{,}25$.}
\end{ex}

\begin{ex}%[1D3V2-5]
	Giá trị của $\lim \limits_{x \rightarrow 1} \dfrac{\sqrt{3 x+1}-2}{2 x^2-3 x+1}$ là\\
	\shortans[oly]{$0{,}75$}
	\loigiai{
		$$
		\begin{aligned}
			& \lim \limits_{x \rightarrow 1} \dfrac{\sqrt{3 x+1}-2}{2 x^2-3 x+1} \\
			& =\lim \limits_{x \rightarrow 1} \dfrac{3 x+1-4}{\left(2 x^2-3 x+1\right)(\sqrt{3 x+1}+2)}=\lim \limits_{x \rightarrow 1} \dfrac{3(x-1)}{(x-1)(2 x-1)(\sqrt{3 x+1}+2)} \\
			& =\lim \limits_{x \rightarrow 1} \dfrac{3}{(2 x-1)(\sqrt{3 x+1}+2)}=\dfrac{3}{4}=0,75.
		\end{aligned}
		$$
	}
\end{ex}

\begin{ex}%[1D3H2-3]
	Giá trị của $\lim \limits_{x \rightarrow-\infty} \dfrac{\left(3-2 x^3\right)^2 \cdot(4 x-1)}{\left(x^2-1\right)^2 \cdot(3+2 x)^3}$ là \\
	\shortans[oly]{$2$}
	\loigiai{
		\begin{eqnarray*}
			&&\lim \limits_{x \rightarrow-\infty} \dfrac{\left(3-2 x^3\right)^2 \cdot(4 x-1)}{\left(x^2-1\right)^2 \cdot(3+2 x)^3}=\lim \limits_{x \rightarrow-\infty} \dfrac{x^6\left(\dfrac{3}{x^3}-2\right)^2 \cdot x\left(4-\dfrac{1}{x}\right)}{x^4\left(1-\dfrac{1}{x^2}\right)^2 \cdot x^3\left(\dfrac{3}{x}+2\right)^3}\\
			&=&\lim \limits_{x \rightarrow-\infty} \dfrac{\left(\dfrac{3}{x^3}-2\right)^2 \cdot\left(4-\dfrac{1}{x}\right)}{\left(1-\dfrac{1}{x^2}\right)^2 \cdot\left(\dfrac{3}{x}+2\right)^3}=2.
		\end{eqnarray*}
	}
\end{ex}

\begin{ex}%[1D3H2-8]
	Một cái hồ chứa $600$l nước ngọt. Người ta bơm nước biển có nồng độ muối $30 \mathrm{g} / \mathrm{l}$ vào hồ với tốc độ $15$ l/phút. Nồng độ muối trong hồ khi $t$ dần về dương vô cùng (đơn vị $\mathrm{g} / \mathrm{l})$ là\\
	\shortans[oly]{$30$}
	\loigiai{
		Sau $t$ phút bơm nước vào hồ thì lượng nước là $600+15 t(l)$ và lượng muối có được là $30.15 t(g)$. Nồng độ muối của nước là
		$$C(t)=\dfrac{30.15 t}{600+15 t}=\dfrac{30 t}{40+t}(\mathrm{g} / \mathrm{l}).$$
		Khi $t$ dần về dương vô cùng, ta có:
		$$
		\lim \limits_{t \rightarrow+\infty} C(t)=\lim \limits_{t \rightarrow+\infty} \dfrac{30 t}{40+t}=\lim \limits_{t \rightarrow+\infty} \dfrac{30 t}{t\left(\dfrac{40}{t}+1\right)}=\lim \limits_{t \rightarrow+\infty} \dfrac{30}{\dfrac{40}{t}+1}=30(\mathrm{g} / \mathrm{l}).
		$$
	}
\end{ex}
\Closesolutionfile{ans}
