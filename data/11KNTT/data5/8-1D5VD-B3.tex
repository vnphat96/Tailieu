\subsection{PHÂN LOẠI VÀ PHƯƠNG PHÁP GIẢI TOÁN}
\setcounter{dang}{0}
\begin{dang}{Xét tính liên tục của hàm số tại một điểm}
	Cho hàm số $y=f(x)$ xác định trên tập $D$. Để xét tính liên tục của hàm số $y=f(x)$ tại điểm $x_{0}\in D$, ta thực hiện các bước sau:
	\begin{itemize}
		\item [\iconMT] Bước 1. Tính $f(x_0)$.
		\item [\iconMT] Bước 2. Tìm $\lim\limits_{x\to x_0}f(x)$.
		\item [\iconMT] Bước 3. So sánh và rút ra kết luận.
		\begin{itemize}
			\item Nếu $\lim\limits_{x\to x_0}f(x)=f(x_0)$ thì hàm số $f(x)$ liên tục tại điểm $x_0$.
			\item Nếu $\lim\limits_{x\to x_0}f(x)\ne f(x_0)$ thì hàm số $f(x)$ không liên tục (gián đoạn) tại điểm $x_0$.
		\end{itemize}
	\end{itemize}
\end{dang}

\begin{vd}%[1D4B3-3]
	Xét tính liên tục của hàm số $f(x)=\heva{&\dfrac{x^2-3x+2}{x-2}&\text{khi }x\neq 2\\&4x-7&\text{khi }x=2}$ tại điểm $x_0=2$.\dapso{liên tục}
	\loigiai{
		Ta có $f(x_0)=f(2)=4\cdot 2-7=1$.\\
		$\lim\limits_{x\rightarrow2}f(x)=\lim\limits_{x\rightarrow2}\dfrac{x^2-3x+2}{x-2}=\lim\limits_{x\rightarrow2}\dfrac{(x-1)(x-2)}{x-2}=\lim\limits_{x\rightarrow2}(x-1)=1$.\\
		Suy ra $f(2)=\lim\limits_{x\rightarrow 2}f(x)$ nên hàm số $f(x)$ liên tục tại điểm $x_0=2$.
	}
\end{vd} \dongcham{12}

\begin{vd}
	Xét tính liên tục của hàm số $f(x)=\begin{cases}
		\dfrac{x^2-6x+5}{x^2-1}&\text{ nếu }\: x\ne 1\\
		-2&\text{ nếu }\: x=1
	\end{cases}$ tại điểm $x_0=1$.
	\loigiai{Ta có: $f(1)=-2$.\\
		$\lim\limits_{x\to1}f(x)=\lim\limits_{x\to1}\dfrac{x^2-6x+5}{x^2-1}=\lim\limits_{x\to1}\dfrac{(x-5)(x-1)}{(x-1)(x+1)}=\lim\limits_{x\to1}\dfrac{x-5}{x+1}=-2=f(1)$.\\
		Vậy hàm số $f(x)$ liên tục tại $x=1$.}
\end{vd} \dongcham{18}
\begin{vd}
	Xét tính liên tục của hàm số $f(x)=\begin{cases}
		\dfrac{1-\sqrt{2x-3}}{2-x}&\text{ nếu }\:x\ne2\\
		-1&\text{ nếu }\:x=2
	\end{cases}$
	tại điểm $x_0=2$.
	\loigiai{Ta có:
		\begin{itemize}
			\item $f(2)=-1$.
			\item $\lim\limits_{x\to 2}f(x)=\lim\limits_{x\to 2}\dfrac{1-\sqrt{2x-3}}{2-x}=\lim\limits_{x\to 2}\dfrac{1-(2x-3)}{(2-x)(1+\sqrt{2x-3})}=\lim\limits_{x\to 2}\dfrac{2(2-x)}{(2-x)(1+\sqrt{2x-3})}\\
			=\lim\limits_{x\to 2}\dfrac{2}{1+\sqrt{2x-3}}=1 \ne f(2)$
		\end{itemize}
		Vậy hàm số $f(x)$ gián đoạn tại $x_0=2$.}
\end{vd} \dongcham{18}



\begin{vd}
	Xét tính liên tục của hàm số $f(x)=\begin{cases}
		x^2+1&\text{ nếu }\:x>0\\
		x&\text{ nếu }\:x\leq0
	\end{cases}$ tại điểm $x_0=0$.
	\loigiai{Ta có:
		\begin{itemize}
			\item $f(0)=0$.
			\item $\lim\limits_{x\to0^{+}}f(x)=\lim\limits_{x\to0^{+}}(x^2+1)=1$.
			\item $\lim\limits_{x\to0^{-}}f(x)=\lim\limits_{x\to0^{-}}x=0$.
		\end{itemize}
		Ta có: $f(0)=\lim\limits_{x\to0^{-}}f(x)\ne\lim\limits_{x\to0^{+}}f(x)$. Vậy hàm số $f(x)$ gián đoạn tại điểm $x=0$.
	}
\end{vd} \dongcham{10}

\begin{vd}
	Xét tính liên tục của hàm số $f(x)=\heva{& \dfrac{x-5}{\sqrt{2x-1}-3} &\text{ nếu }& x > 5 \\& (x-5)^{2} + 3 &\text{ nếu }& x \le 5}$ tại điểm $x_0=5$.
	\loigiai{
	}
\end{vd} \dongcham{20}


\begin{dang}{Xét tính liên tục của hàm số trên miền xác định}
	\begin{itemize}
		\item [\iconMT] Hàm đa thức liên tục trên $\mathbb{R}$.
		\item [\iconMT] Hàm phân thức hữu tỉ, hàm lượng giác liên tục trên từng khoảng xác định của chúng.
	\end{itemize}
\end{dang}
\begin{vd}%[Hữu Bình]%[1D4B3]
	Xét tính liên tục của hàm số sau trên tập xác định của chúng.
	\begin{tasks}(1)
		\task $f(x)=\heva{&\dfrac{x^2-x-2}{x+1}&\text{ khi } x\ne -1\\ &-3&\text{ khi } x=-1}$.
		\task $f(x)=\heva{&\dfrac{2x+1}{(x-1)^2}&\text{ khi } x\ne 1\\ &3 &\text{ khi } x=1}$.
	\end{tasks}
	\loigiai{
		\begin{enumerate}
			\item \begin{itemize}
				\item Tập xác định của hàm số là $\mathscr{D}=\mathbb{R}$.
				\item Khi $x \ne -1$, $f(x)=\dfrac{x^2-x-2}{x+1}$ là hàm phân thức hữu tỉ nên liên tục trên $(-\infty;-1)\cup(-1;+\infty)$.
				\item Tại điểm $x=-1$, ta có $f(-1)=-3$.\\
				$\lim\limits_{x\to -1}f(x)=\lim\limits_{x\to -1}\dfrac{x^2-x-2}{x+1}=\lim\limits_{x\to -1}(x-2)=-3=f(-1).$\\
				Do đó hàm số liên tục tại $x=-1$.
				\item Vậy hàm số liên tục trên $\mathbb{R}$.
			\end{itemize}
			\item \begin{itemize}
				\item Tập xác định của hàm số là $\mathscr{D}=\mathbb{R}$.\\
				\item Khi $x \ne 1$, $f(x)=\dfrac{2x+1}{(x-1)^2}$ là hàm phân thức hữu tỉ nên liên tục trên $(-\infty;1)\cup(1;+\infty)$.\\
				\item Tại điểm $x=1$, ta có $f(1)=3$.\\
				$\lim\limits_{x\to 1}f(x)=\lim\limits_{x\to 1}\dfrac{2x+1}{(x-1)^2}=+\infty\ne f(-1).$\\
				Do đó hàm số gián đoạn tại $x=1$.
				\item Vậy hàm số liên tục trên $\mathbb{R}\setminus\{1\}$.
			\end{itemize}
	\end{enumerate}}
\end{vd} \dongcham{16}

\begin{vd}
	Xét tính liên tục của hàm số sau trên tập xác định của chúng.
	\begin{tasks}(1)
		\task $f(x)=\heva{&x^2+3x&\text{ khi } &x\ge 2\\ &6x+1&\text{ khi } &x<2.}$
		\task $f(x)=\heva{&x^2-3x+5&\text{ khi } &x> 1\\ &3 &\text{ khi } &x=1\\&2x+1 &\text{ khi }&x<1.}$
	\end{tasks}
	\loigiai{
		\begin{enumerate}
			\item \begin{itemize}
				\item Tập xác định của hàm số là $\mathscr{D}=\mathbb{R}$.
				\item Khi $x>2$, $f(x)=x^2+3x$ là hàm đa thức nên liên tục trên $(2;+\infty)$.
				\item Khi $x<2$, $f(x)=6x+1$ là hàm đa thức nên liên tục trên $(-\infty;2)$.
				\item Tại điểm $x=2$, ta có $f(2)=10$.\\
				$\lim\limits_{x\to 2^+}f(x)=\lim\limits_{x\to 2^+}(x^2+3x)=10$ và $\lim\limits_{x\to 2^-}f(x)=\lim\limits_{x\to 2^-}(6x+1)=13$.\\
				Vì không tồn tại $\lim\limits_{x\to 2}f(x)$ nên hàm số gián đoạn tại $x=2$.
				\item Vậy hàm số liên tục trên $\mathbb{R}\setminus\{2\}$.
			\end{itemize}
			\item \begin{itemize}
				\item Tập xác định của hàm số là $\mathscr{D}=\mathbb{R}$.
				\item Khi $x>1$, $f(x)=x^2-3x+5$ là hàm đa thức nên liên tục trên $(1;+\infty)$.
				\item Khi $x<1$, $f(x)=2x+1$ là hàm đa thức nên liên tục trên $(-\infty;1)$.
				\item Tại điểm $x=1$, ta có $f(1)=3$.\\
				$\lim\limits_{x\to 1^+}f(x)=\lim\limits_{x\to 1^+}(x^2-3x+5)=3$ và $\lim\limits_{x\to 1^-}f(x)=\lim\limits_{x\to 1^-}(2x+1)=3$.\\
				Vì $\lim\limits_{x\to 1}f(x)=f(1)$ nên hàm số liên tục tại $x=1$.
				\item Vậy hàm số liên tục trên $\mathbb{R}$.
			\end{itemize}
	\end{enumerate}}
\end{vd} \dongcham{16}

\begin{vd} %Bài 5.33. 
	Tìm các giá trị của $a$ để hàm số $f(x)=\heva{&x+1&\text{nếu } x\le a\\&x^2&\text{nếu } x> a}$ liên tục trên $\mathbb{R}.$
	\loigiai{Hàm số liên tục trên các khoảng $(-\infty; a)$ và $(a;+\infty)$.\\
		Ta có 
		\begin{itemize}
			\item $f(a)=a+1. $
			\item $\mathop {\lim }\limits_{x\to a^-}f(x)=\mathop {\lim }\limits_{x\to a^-} (x+1)=a+1.$
			\item $\mathop {\lim }\limits_{x\to a^+}f(x)=\mathop {\lim }\limits_{x\to a^+} x^2=a^2.$
		\end{itemize}
		Để hàm số  liên tục trên $\mathbb{R}$, ta cần $$f(a)=\mathop {\lim }\limits_{x\to a^+}f(x)= \mathop {\lim }\limits_{x\to a^-}f(x)\Leftrightarrow a+1=a^2 \Leftrightarrow \hoac{&a=\dfrac{1+\sqrt{5}}{2}\\&a=\dfrac{1-\sqrt{5}}{2}.}$$}
\end{vd} \dongcham{15}

\begin{dang}{Tìm giá trị của tham số để hàm số liên tục - gián đoạn tại điểm cho trước.}
\end{dang}

\begin{vd}
	Tìm tham số $ m $ để hàm số $ f(x) = \heva{&x^2 + 2x - m   &\text{  khi }  x \neq 2\\&x+m  &\text{  khi } x = 2} $liên tục tại $ x_0 = 2. $
	\loigiai{
		
		Ta có: $ \lim\limits_{x \to 2} f(x) = \lim\limits_{x \to 2} \left(x^2 + 2x - m \right) = 8 - m$\\
		
		và $ f(2) = 2 + m $.\\
		
		Để hàm số liên tục tại $ x_0 = 2 \Leftrightarrow 8 - m = 2 + m \Leftrightarrow m = 3. $
	}
\end{vd} \dongcham{18}
\begin{vd}%	[1D4K3]
	Tìm tham số $ m $ để hàm số $ f(x) = \heva{&\dfrac{x^2 - 2x - 3}{x+1}   &\text{  khi }  x \neq -1\\&m^2 + 5m  &\text{  khi } x = -1} $liên tục tại $ x_0 = -1. $
	\loigiai{
		
		Ta có: $ \lim\limits_{x \to -1} f(x) =  \lim\limits_{x \to -1} \dfrac{x^2 - 2x - 3}{x+1}  = \lim\limits_{x \to -1} \dfrac{(x+1)(x-3)}{x+1} = \lim\limits_{x \to -1} (x-3)  = -4$ \\
		
		và $ f(-1) = m^2 + 5m $.\\
		
		Để hàm số liên tục tại $ x_0 = -1 \Leftrightarrow m^2 + 5m = -4 \Leftrightarrow \hoac{m=-1\\m=-4}. $
	}	
\end{vd} \dongcham{17}
\begin{vd}%	[1D4K3]
	Tìm tham số $ m $ để hàm số $ f(x) = \heva{&\dfrac{\sqrt{4x+5} - 3}{x^2-1}   &\text{  khi }  x > 1\\&2m+3  &\text{  khi } x \leq 1} $gián đoạn tại $ x_0 = 1. $
	\loigiai{ 
		
		Ta có: $ \lim\limits_{x \to 1^+} f(x) =  \lim\limits_{x \to 1^+} \left(\dfrac{\sqrt{4x+5}-3}{x^2-1} \right) =    \lim\limits_{x \to 1^+} \dfrac{4x - 4}{ (x-1)(x+1) \left(\sqrt{4x+5}+3\right)}  \\
		\hspace*{3.3cm} = \lim\limits_{x \to 1^+} \dfrac{4}{(x+1)\left( \sqrt{4x+5} + 3\right)} = \dfrac{4}{2 \cdot 6} = \dfrac{1}{3} \cdot $
		
		Mặt khác: $ \lim\limits_{x \to 1^-} f(x) = f(1) = 2m + 3. $
		
		Để hàm số gián đoạn tại điểm $ x_0 = 1 \Leftrightarrow  \lim\limits_{x \to 1^+} f(x) \neq f(1) \Leftrightarrow 2m + 3 \neq \dfrac{1}{3} \Leftrightarrow m \neq -\dfrac{4}{3} \cdot $
	}
\end{vd} \dongcham{17}

\begin{dang}{Chứng minh phương trình có nghiệm}
	\begin{itemize}
		\item [\iconMT] Để chứng minh phương trình $f(x)=0$ có ít nhất một nghiệm trên $D$, ta chứng minh hàm số $y=f(x)$ liên tục trên $ D $ và có hai số $a,b\in D$ sao cho $f(a).f(b)<0$. 
		\item [\iconMT] Để chứng minh phương trình $f(x)=0$ có $ k $ nghiệm trên $ D $, ta chứng minh hàm số $y=f(x)$ liên tục trên $D$ và tồn tại $k$ khoảng rời nhau $(a_i;{a}_{i+1})\left(i=1,2,\ldots ,k\right)$ nằm trong $D$ sao cho $f(a_i).f({a}_{i+1})<0$.
	\end{itemize}
\end{dang}

\begin{vd}
	Chứng minh rằng phương trình $2x^4-2x^3-3=0$ có ít nhất một nghiệm thuộc khoảng $\left(-1;0\right).$ 
	\loigiai{
		Đặt $f(x)=2x^4-2x^3-3.$ \\
		Vì  $f(x)$ là hàm đa thức xác định trên $\mathbb{R}$ nên $f(x)$ liên tục trên $\mathbb{R}$ $\Rightarrow f(x)$ liên tục trên $\left[-1;0\right].$ \\
		Ta có: $f(0)=-3;f\left(-1\right)=1\Rightarrow f\left(-1\right)f(0)<0.$ \\
		$\Rightarrow f(x)=0$ có ít nhất một nghiệm thuộc khoảng $\left(-1;0\right)$ (đpcm). 
		
	}
\end{vd} \dongcham{12}

\begin{vd}
	Chứng minh rằng  phương trình $6x^3+3x^2-31x+10=0$ có đúng $3$ nghiệm phân biệt.
	\loigiai{
		Đặt $f(x)=6x^3+3x^2-31x+10.$ Hàm số $f(x)$ liên tục trên $\mathbb{R}$ nên liên tục trên $\left[-3;2\right].$ \\
		Ta có:
		\begin{itemize}
			\item [$\bullet$] $ \heva{
				& f(-3)=-32 \\ 
				& f(0)=10 
			}$ $\Rightarrow f(-3)f(0)<0\Rightarrow f(x)=0$ có nghiệm thuộc $\left(-3;0\right).$
			\item [$\bullet$] $ \heva{
				& f(0)=10 \\ 
				& f(1)=-12 
			}$ $\Rightarrow f(0)f(1)<0\Rightarrow f(x)=0$ có nghiệm thuộc $\left(0;1\right).$
			\item [$\bullet$] $\heva{
				& f(1)=-12 \\ 
				& f(2)=8  
			}$ $\Rightarrow f(1)f(2)<0\Rightarrow f(x)=0$ có nghiệm thuộc $\left(1;2\right).$
		\end{itemize}
		Mặt khác vì $f(x)$ là một đa thức bậc ba nên phương trình $f(x)=0$ chỉ có tối đa ba nghiệm. \\
		Vậy phương trình $f(x)=0$ có đúng $3$ nghiệm phân biệt (đpcm).
	}
\end{vd} \dongcham{18}


\subsection{BÀI TẬP TỰ LUYỆN}

\begin{bt}%[Tex hóa SGK CD-CT,T7/22, TVN-001]%[1T3B3-3]
	Xét tính liên tục của hàm số:
	\begin{enumerate}
		\item $f(x)=\heva{&x^2+1&\quad\text{khi }&x\ge0\\&1-x&\quad\text{khi }&x<0}$ tại điểm $x=0$.
		\item $f(x)=\heva{&x^2+2&\quad\text{khi }&x\ge1\\&x&\quad\text{khi }&x<1}$ tại điểm $x=1$.
	\end{enumerate}
	\loigiai{\begin{enumerate}
			\item Ta có $f(0)=0^2+1=1$, $\lim\limits_{x\to0^+}f(x)=\lim\limits_{x\to0^+}\left(x^2+1\right)=0^2+1=1$,\\
			\phantom{Ta có} $\lim\limits_{x\to0^-}f(x)=\lim\limits_{x\to0^-}(1-x)=1-0=1$.\\
			Do $\lim\limits_{x\to0^-}f(x)=\lim\limits_{x\to 0^+}f(x)=f(0)$ nên hàm số $y=f(x)$ liên tục tại điểm $x=0$.
			\item Ta có $f(1)=1^2+2=3$, $\lim\limits_{x\to1^+}f(x)=\lim\limits_{x\to1^+}\left(x^2+2\right)=1^2+2=3$,\\
			\phantom{Ta có} $\lim\limits_{x\to1^-}f(x)=\lim\limits_{x\to1^-}(x)=1=1$.\\
			Do $\lim\limits_{x\to1^+}f(x)\ne\lim\limits_{x\to1^-}f(x)$ nên không tồn tại $\lim\limits_{x\to1}f(x)$.\\
			Vậy hàm số $y=f(x)$ không liên tục tại điểm $x=1$.
	\end{enumerate}}
\end{bt} \dongcham{15}

\begin{bt}%[1T3B3-5]
	Cho hàm số $f(x)=\heva{&\dfrac{x^2-4}{x+2}&\quad\text{khi }&x\ne-2\\&a&\quad\text{khi }&x=-2.}$
	Tìm $a$ để hàm số $y=f(x)$ liên tục trên $\mathbb{R}$.
	\loigiai{Với mọi $x_0\ne-2$, ta có $f(x)=\dfrac{x^2-4}{x+2}$ luôn xác định nên liên tục tại đó.\\
		Mặt khác ta có $f(-2)=a$, $\lim\limits_{x\to-2}f(x)=\lim\limits_{x\to-2}\dfrac{x^2-4}{x+2}=\lim\limits_{x\to-2}(x-2)=-2-2=-4$.\\
		Để hàm số $y=f(x)$ liên tục trên $\mathbb{R}$ thì hàm số $y=f(x)$ phải liên tục tại $x=-2$
		$$\lim\limits_{x\to-2}f(x)=f(-2)\Leftrightarrow a=-4.$$}
\end{bt} \dongcham{13}


\begin{bt}%[1D4B3]
	Xét tính liên tục của hàm số sau trên tập xác định của chúng.
	\begin{tasks}(2)
		\task $f(x)=\heva{&\dfrac{x-2}{x^2-4}&\text{ khi } x\ne 2\\ &1&\text{ khi } x=2}$.
		\task $f(x)=\heva{&\dfrac{x^3-1}{x-1}&\text{ khi } x\ne 1\\ &3 &\text{ khi } x=1}$.
	\end{tasks}
	\loigiai{
		\begin{enumerate}
			\item \begin{itemize}
				\item Tập xác định của hàm số là $\mathscr{D}=\mathbb{R}$.
				\item Khi $x \ne 2$, $f(x)=\dfrac{x-2}{x^2-4}$ là hàm phân thức hữu tỉ nên liên tục trên $(-\infty;2)\cup(2;+\infty)$.
				\item Tại điểm $x=2$, ta có $f(2)=1$.\\
				$\lim\limits_{x\to 2}f(x)=\lim\limits_{x\to 2}\dfrac{x-2}{x^2-4}=\lim\limits_{x\to 2}\dfrac{1}{x+2}=\dfrac{1}{4}\ne f(2).$\\ Do đó hàm số gián đoạn tại $x=2$.
				\item Vậy hàm số liên tục trên $\mathbb{R}\setminus\{2\}$.
			\end{itemize}
			\item \begin{itemize}
				\item Tập xác định của hàm số là $\mathscr{D}=\mathbb{R}$.
				\item Khi $x \ne 1$, $f(x)=\dfrac{x^3-1}{x-1}$ là hàm phân thức hữu tỉ nên liên tục trên $(-\infty;1)\cup(1;+\infty)$.
				\item Tại điểm $x=1$, ta có $f(1)=3$.\\
				$\lim\limits_{x\to 1}f(x)=\lim\limits_{x\to 1}\dfrac{x^3-1}{x-1}=\lim\limits_{x\to 1}(x^2+x+1)=3= f(1).$\\
				Do đó hàm số liên tục tại $x=1$.
				\item Vậy hàm số liên tục trên $\mathbb{R}$.
			\end{itemize}
	\end{enumerate}}
\end{bt} \dongcham{15}

\begin{bt}%[1T3B3-6]
	Một bãi đậu xe ô-tô đưa ra giá $C(x)$ (đồng) khi thời gian đậu xe là $x$ (giờ) như sau: $$C(x)=\heva{&60.000&\quad\text{khi }&0<x\le2\\&100.000&\quad\text{khi }&2<x\le4\\&200.000&\quad\text{khi }&4<x\le24.}$$
	Xét tính liên tục của hàm số $C(x)$.
	\loigiai{\begin{itemize}
			\item Hàm số $C(x)$ là hàm hằng trên từng khoảng $(0;2)$, $(2;4)$, $(4;6)$ nên liên tục trên từng khoảng đó.
			\item Ta có $\heva{&\lim\limits_{x\to2^-}C(x)=60.000\\&\lim\limits_{x\to2^+}C(x)=100.000}\Rightarrow$ không tồn tại $\lim\limits_{x\to2}C(x)$, vậy $C(x)$ không liên tục tại $x_0=2$.
			\item Ta có $\heva{&\lim\limits_{x\to4^-}C(x)=100.000\\&\lim\limits_{x\to4^+}C(x)=200.000}\Rightarrow$ không tồn tại $\lim\limits_{x\to4}C(x)$, vậy $C(x)$ không liên tục tại $x_0=4$.
		\end{itemize}
		Vậy hàm số $C(x)$ liên tục trên từng khoảng $(0;2)$, $(2;4)$, $(4;6)$.}
\end{bt} \dongcham{14}

\begin{bt}%[1T3B3-6]
	Lực hấp dẫn do Trái Đất tác dụng lên một đơn vị khối lượng ở khoảng cách $r$ tính từ tâm của nó là $F(r)=\heva{&\dfrac{GMr}{R^3}&\quad\text{khi }&0<r\le R\\&\dfrac{GM}{r^2}&\quad\text{khi }&r\ge R}$, trong đó $M$ là khối lượng, $R$ là bán kính của Trái Đất, $G$ là hằng số hấp dẫn. Hàm số $F(r)$ có liên tục trên $(0;+\infty)$ không?
	\loigiai{\begin{itemize}
			\item Với mọi $r\in(0;R)$, hàm số $F(r)=\dfrac{GMr}{R^3}$ luôn xác định nên liên tục tại đó.
			\item Với mọi $r\in(R;+\infty)$, hàm số $F(r)=\dfrac{GM}{r^2}$ luôn xác định nên liên tục tại đó.
			\item Ta có $\heva{&\lim\limits_{r\to R^-}F(r)=\lim\limits_{r\to R^-}\dfrac{GMr}{R^3}=\dfrac{GM}{R^2}\\&\lim\limits_{x\to R^+}F(r)=\lim\limits_{x\to R^+}\dfrac{GM}{r^2}=\dfrac{GM}{R^2}\\&F(R)=\dfrac{GM}{R^2}}$ nên hàm số $F(r)$ liên tục tại $r=R$.
		\end{itemize}
		Vậy hàm số $F(r)$ liên tục trên $(0;+\infty)$.}
\end{bt} \dongcham{16}

\begin{bt}%[Thành Lê, dự án 11-EX-DCHT-2]%[1D4B3-6]
	Chứng minh rằng phương trình $x^4 - x^3 - 2x^2 - 15x - 25 = 0$ có ít nhất một nghiệm âm và ít nhất một nghiệm dương.
\end{bt} \dongcham{16}

\begin{bt}%[Thành Lê, dự án 11-EX-DCHT-2]%[1D4B3-6]
	Chứng minh rằng phương trình $x^3 + 4x^2 - 2 = 0$ có ba nghiệm trong khoảng $(-4; 1)$.
\end{bt} \dongcham{18}

\begin{bt}%[Thành Lê, dự án 11-EX-DCHT-2]%[1D4B3-6]
	Chứng minh rằng phương trình $x^5 - 5x^3 + 4x - 1 = 0$ có đúng năm nghiệm.
\end{bt} \dongcham{18}

\begin{bt}
	Chứng minh rằng phương trình $x+1+\cos x=0$  có nghiệm.
	\loigiai{
		Xét hàm số $f(x)=x+1+\cos x$ liên tục trên $\left[-\pi ;0\right]$ và có $\heva{
			& f\left(-\pi \right)=-\pi  \\ 
			& f(0)=2 \\
		}\Rightarrow f\left(-\pi \right).f(0)<0$. Suy ra phương trình $f(x)=0$ có nghiệm $x_0\in \left(-\pi ;0\right)$.\\
		Vậy phương trình $x+1+\cos x=0$  có nghiệm.	
	}
\end{bt} \dongcham{15}

