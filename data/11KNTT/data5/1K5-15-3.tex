\begin{dang}{Giới hạn tại vô cực}
		Ta nói dãy $\{u_n\}$ có giới hạn là $+ \infty$ khi $n \rightarrow + \infty$, nếu $u_n$ có thể lớn hơn một số dương bất kì, kể từ một số hạng nào đó trở đi. \\
	Kí hiệu: $\lim \limits{n \to +\infty}u_n = + \infty$ hay $u_n \rightarrow + \infty$ khi $n \rightarrow + \infty$. \\ 
	Dãy số $\{u_n\}$ có giới hạn là $- \infty$ khi $n \rightarrow + \infty$, nếu $\lim \limits{n \to +\infty}- u_n = + \infty$. \\
	Kí hiệu: $\lim \limits{n \to +\infty}u_n = - \infty$ hay $u_n \rightarrow - \infty$ khi $n \rightarrow + \infty$. \\ 
	\textbf{Một số giới hạn đặc biệt và định lí về giới hạn dãy số} \\
	\textit{Giới hạn đặc biệt}: \\
	$\displaystyle \lim_{n \rightarrow + \infty} \sqrt{n} = + \infty$ \\
	$\displaystyle \lim_{n \rightarrow + \infty} n^k = + \infty$ với $k$ là số nguyên dương. \\
	$\displaystyle \lim_{n \rightarrow + \infty} q^n = + \infty$ nếu $q > 1$ \\
	\textit{Định lý}: \\
	Nếu $\lim \limits{n \to +\infty}u_n = a > 0$ và $\lim \limits{n \to +\infty}v_n = 0$ với $v_n > 0$ thì $\lim \limits{n \to +\infty}\dfrac{u_n}{v_n} = + \infty$ \\
	Nếu $\lim \limits{n \to +\infty}u_n = + \infty$ và $\lim \limits{n \to +\infty}v_n = a > 0$ thì $\lim \limits{n \to +\infty}u_nv_n = + \infty$ \\
\end{dang}

\subsubsection{Ví dụ mẫu}
\begin{vd}
	Tìm giới hạn 
	\begin{enumEX}[a)]{2}
		\item[a)] $\lim \limits{n \to +\infty}(n^3 + n^2 + n + 1)$.
		\item[b)] $\lim \limits{n \to +\infty}\left(n^2 - n\sqrt{n} + 1\right)$.
	\end{enumEX}
	\loigiai{
		\begin{enumEX}[a)]{1}
			\item $\lim \limits{n \to +\infty}(n^3 + n^2 + n + 1) = \lim \limits{n \to +\infty}n^3\left(1 + \dfrac{1}{n} + \dfrac{1}{n^2} + \dfrac{1}{n^3}\right) = + \infty$.
			\item $\lim \limits{n \to +\infty}\left(n^2 - n\sqrt{n} + 1\right) = \lim \limits{n \to +\infty}n^2\left(1 - \dfrac{1}{\sqrt{n}} + \dfrac{1}{n^2}\right) = + \infty.$
		\end{enumEX}
	}
\end{vd}
\begin{vd}
	Tìm giới hạn
	\begin{enumEX}[a)]{3}
		\item[a)] $\lim \limits{n \to +\infty}\dfrac{n^5 + n^4 - n - 2}{4n^3 + 6n^2 + 9}$.
		\item[b)] $\lim \limits{n \to +\infty}\dfrac{\sqrt[3]{n^6 - 7n^3 - 5n + 8}}{n + 12}$.
		\item[c)] $\lim \limits{n \to +\infty}\left(n + \sqrt{n^2 - n + 1}\right)$.
		
	\end{enumEX}
	\loigiai{
		\begin{enumEX}[a)]{1}
			\item $\lim \limits{n \to +\infty}\dfrac{n^5 + n^4 - n - 2}{4n^3 + 6n^2 + 9} = \lim \limits{n \to +\infty}\dfrac{n^2 + n - \frac{1}{n^2} - \frac{2}{n^3}}{4 + \frac{6}{n} + \frac{9}{n^3}} = \lim \limits{n \to +\infty}\dfrac{n^2 + n}{4} = + \infty$.
			\item $\lim \limits{n \to +\infty}\dfrac{\sqrt[3]{n^6 - 7n^3 - 5n + 8}}{n + 12} = \lim \limits{n \to +\infty}\dfrac{n^2\sqrt[3]{1 - \frac{7}{n^3} - \frac{5}{n^5} + \frac{8}{n^6}}}{n + 12} = \lim \limits{n \to +\infty}\dfrac{n\sqrt[3]{1 - \frac{7}{n^3} - \frac{5}{n^5} + \frac{8}{n^6}}}{1 + \frac{12}{n}} = + \infty$.
			\item $\lim \limits{n \to +\infty}\left(n + \sqrt{n^2 - n + 1}\right) = n\left(1 + \sqrt{1 - \frac{1}{n} + \frac{1}{n^2}}\right) = \lim \limits{n \to +\infty}2n = + \infty$
		\end{enumEX}
	}
\end{vd}
\begin{vd}
	Tìm giới hạn
	\begin{enumEX}[a)]{3}
		\item[a)] $\lim \limits{n \to +\infty}\dfrac{1^3 + 2^3 + ... + n^3}{n^2 + 3n\sqrt{n} + 2}$.
		\item[b)] $\lim \limits{n \to +\infty}\left(n + \sqrt[3]{n^3 - 2n + 1}\right)$.
		\item[c)] $\lim \limits{n \to +\infty}\dfrac{n^3 - 3n}{2n + 15}$.
	\end{enumEX}
	\loigiai{
		\begin{enumEX}[a)]{1}
			\item $\lim \limits{n \to +\infty}\dfrac{1^3 + 2^3 + ... + n^3}{n^2 + 3n\sqrt{n} + 2} = \lim \limits{n \to +\infty}\dfrac{\frac{1}{4}n^2(n + 1)^2}{n^2 + 3n\sqrt{n} + 2} = \lim \limits{n \to +\infty}\dfrac{\frac{1}{4}(n + 1)^2}{1 + \frac{3}{\sqrt{n}} + \frac{2}{n^2}} = \lim \limits{n \to +\infty}\dfrac{1}{4}(n + 1)^2 = + \infty$.
			\item $\lim \limits{n \to +\infty}\left(n + \sqrt[3]{n^3 - 2n + 1}\right) = n\left(1 + \sqrt[3]{1 - \frac{2}{n^2} + \frac{1}{n^3}}\right) = \lim \limits{n \to +\infty}2n = + \infty$.
			\item $\lim \limits{n \to +\infty}\dfrac{n^3 - 3n}{2n + 15} = \lim \limits{n \to +\infty}\dfrac{n^2 - 3}{2 + \frac{15}{n}} = + \infty$
		\end{enumEX}
	}
\end{vd}
\subsubsection{Bài tập rèn luyện} 
\centerline{\fcolorbox{red}{yellow!50}{\bf {BÀI TẬP TỰ LUẬN }}}
\begin{bt}%[1K5?E-4]%[1K5?E-4]
	Tìm giới hạn
	\begin{enumEX}[a)]{2}
		\item[a)] $\lim \limits{n \to +\infty}\sqrt{5n^2 - 8n + 7}$.
		\item[b)] $\lim \limits{n \to +\infty}\sqrt{n^3 - 5n + 6}$.
	\end{enumEX}
	\loigiai{
		\begin{enumEX}[a)]{1}
			\item $\lim \limits{n \to +\infty}\sqrt{5n^2 - 8n + 7} = \lim \limits{n \to +\infty}n\sqrt{5 - \dfrac{8}{n} + \dfrac{7}{n^2}} = + \infty$.
			\item $\lim \limits{n \to +\infty}\sqrt{n^3 - 5n + 6} = \lim \limits{n \to +\infty}n\sqrt{n} \sqrt{1 - \dfrac{5}{n^2} + \dfrac{6}{n^3}} = + \infty$.
		\end{enumEX}
	}
\end{bt}
\begin{bt}%[1K5?E-4]%[1K5?E-4]
	Tìm giới hạn
	\begin{enumEX}[a)]{2}
		\item[a)] $\lim \limits{n \to +\infty}\dfrac{\sqrt{5n^4 - 8n^2 + 10}}{4n + 5}$.
		\item[b)] $\lim \limits{n \to +\infty}\dfrac{n^2 - 15n + 11}{\sqrt{n^2 - 8n + 7}}$.
	\end{enumEX}
	\loigiai{
		\begin{enumEX}[a)]{1}
			\item $\lim \limits{n \to +\infty}\dfrac{\sqrt{5n^4 - 8n^2 + 10}}{4n + 5} = \lim \limits{n \to +\infty}\dfrac{n^2\sqrt{5 - \frac{8}{n^2} + \frac{10}{n^4}}}{4n + 5} = \dfrac{n\sqrt{5 - \frac{8}{n^2} + \frac{10}{n^4}}}{4 + \frac{5}{n}} = \lim \limits{n \to +\infty}\dfrac{n\sqrt{5}}{4} = + \infty$.
			\item $\lim \limits{n \to +\infty}\dfrac{n^2 - 15n + 11}{\sqrt{n^2 - 8n + 7}} = \lim \limits{n \to +\infty}\dfrac{n - 15 + \frac{11}{n}}{\sqrt{1 - \frac{8}{n} + \frac{7}{n^2}}} = + \infty$.
		\end{enumEX}
	}
\end{bt}
\begin{bt}%[1K5?E-4]%[1K5?E-4]
	
	Tìm $\lim \limits{n \to +\infty}\left(\frac{1}{n^2}+\frac{2}{n^2}+\ldots+\frac{n}{n^2}\right)$.
\end{bt}

\loigiai{
	$$
	\lim \limits{n \to +\infty}\left(\frac{1}{n^2}+\frac{2}{n^2}+\ldots+\frac{n}{n^2}\right)=\lim \limits{n \to +\infty}\left(\frac{1+2+\ldots+n}{n^2}\right)=\lim \limits{n \to +\infty}\left(\frac{n(n+1)}{2 n^2}\right)=\lim \limits{n \to +\infty}\left(\frac{1+\frac{1}{n}}{2}\right)=\frac{1}{2} .
	$$}
\begin{bt}%[1K5?E-4]%[1K5?E-4]
	Tính giới hạn: $\lim \limits{n \to +\infty}\left[\left(1-\frac{1}{2^2}\right)\left(1-\frac{1}{3^2}\right) \ldots\left(1-\frac{1}{n^2}\right)\right]$.
	Xét dãy số $\left(u_n\right)$, với $u_n=\left(1-\frac{1}{2^2}\right)\left(1-\frac{1}{3^2}\right) \ldots\left(1-\frac{1}{n^2}\right), n \geq 2, n \in \mathbb{N}$.
\end{bt}
\loigiai{
	Ta có:
	$$
	\begin{aligned}
		& u_2=1-\frac{1}{2^2}=\frac{3}{4}=\frac{2+1}{2 \cdot 2} \\
		& u_3=\left(1-\frac{1}{2^2}\right) \cdot\left(1-\frac{1}{3^2}\right)=\frac{3}{4} \cdot \frac{8}{9}=\frac{4}{6}=\frac{3+1}{2 \cdot 3} ; \\
		& u_4=\left(1-\frac{1}{2^2}\right) \cdot\left(1-\frac{1}{3^2}\right)\left(1-\frac{1}{4^2}\right)=\frac{3}{4} \cdot \frac{8}{9} \cdot \frac{15}{16}=\frac{5}{8}=\frac{4+1}{2 \cdot 4} \\
		& \ldots \ldots . \\
		& u_n=\frac{n+1}{2 n} .
	\end{aligned}
	$$
	Dễ dàng chứng minh bằng phương pháp qui nạp để khẳng định $u_n=\frac{n+1}{2 n}, \forall n \geq 2$
	Khi đó $\lim \limits{n \to +\infty}\left[\left(1-\frac{1}{2^2}\right)\left(1-\frac{1}{3^2}\right) \ldots\left(1-\frac{1}{n^2}\right)\right]=\lim \limits{n \to +\infty}\frac{n+1}{2 n}=\frac{1}{2}$.}
\begin{bt}%[1K5?E-4]%[1K5?E-4]
	
	Cho dãy số $\left(u_n\right), n \in \mathbb{N}^*$, thỏa mãn điều kiện $\left\{\begin{array}{c}u_1=3 \\ u_{n+1}=-\frac{u_n}{5}\end{array}\right.$. Gọi $S=u_1+u_2+u_3+\ldots+u_n$ là tồng $n$ số hạng đầu tiên của dãy số đã cho. Khi đó lim $S_n$ bằng
\end{bt}
\loigiai{
	Ta có $\frac{u_{n+1}}{u_n}=\frac{-\frac{u_n}{5}}{u_n}=-\frac{1}{5}$ do đó dãy $\left(u_n\right), n \in \mathbb{N}^*$ là một cấp số nhân lùi vô hạn có $u_1=3, d=-\frac{1}{5}$.
	Suy ra $\lim \limits{n \to +\infty}S_n=\frac{u_1}{1-q}=\frac{3}{1+\frac{1}{5}}=\frac{5}{2}$.}
\begin{bt}%[1K5?E-4]%[1K5?E-4]
	Trong một lần Đoàn trường Lê Văn Hưu tổ chức chơi bóng chuyền hơi, bạn Nam thả một quả bóng chuyền hơi từ tầng ba, độ cao $8 m$ so với mặt đất và thây rằng mỗi lần chạm đất thì quả bóng lại nảy lên một độ cao bằng ba phần tư độ cao lần rơi trước. Biết quả bóng chuyển động vuông góc với mặt đất. Khi đó tồng quãng đường quả bóng đã bay từ lúc thả bơng đến khi quả bóng không máy nữa bằng bao nhiêu ?
\end{bt}
\loigiai{
	Lần đầu rơi xuống, quảng đường quả bóng đã bay đến lúc chạm đất là $8 m$.
	Sau đó quả bóng nảy lên và rơi xuống chạm đất lần thứ 2 thì quảng đường quả bóng đã bay là $8+2.8 \cdot \frac{3}{4}$.
	
	Tương tự, khi quả bóng nảy lên và roi xuống chạm đất lần thứ $n$ thì quảng đường quả bóng đã bay là $8+2.8 \cdot \frac{3}{4}+\ldots \ldots . .+2.8 \cdot\left(\frac{3}{4}\right)^{n-1}=8+\frac{1-\left(\frac{3}{4}\right)^n}{1-\frac{3}{4}}=8+48\left(1-\left(\frac{3}{4}\right)^{n-1}\right)$.
	Quảng đường quả bóng đã bay từ lúc thả đến lúc không máy nữa bằng: $\lim \limits{n \to +\infty}\left[8+48\left(1-\left(\frac{3}{4}\right)^{n-1}\right)\right]=8+48=56$.}
\begin{bt}%[1K5?E-4]%[1K5?E-4]
	Cho hình vuông $A B C D$ có cạnh bằng a. Người ta dựng hình vuông $A_1 B_1 C_1 D_1$ có cạnh bằng $\frac{1}{2}$ đường chéo của hình vuông $A B C D$; dựng hình vuông $A_2 B_2 C_2 D_2$ có cạnh bằng $\frac{1}{2}$ đường chéo của hình vuông $A_1 B_1 C_1 D_1$ và cứ tiếp tục như vậy (tham khảo hình vẽ).
	Giả sử cách dựng trên có thể tiến ra vô hạn. Nếu tổng diện tích $S$ của tất cả các hình vuông $A B C D, A_1 B_1 C_1 \mathrm{D}_1, A_2 B_2 C_2 \mathrm{D}_2 \ldots$ bằng 8 thì $a$ bằng bao nhiêu? 
\end{bt}
\loigiai{
	$$
	\begin{aligned}
		& \text { Ta có } S_{A B C D}=a^2 ; S_{A_1 B_1 C_1 D_1}=\left(\frac{a \sqrt{2}}{2}\right)^2=\frac{a^2}{2} ; S_{A_2 B_2 C_2 D_2}=\left(\frac{a}{2}\right)^2=\frac{a^2}{4}=\frac{a^2}{2^2} \\
		& S=S_{A B C D}+S_{A_1 B_1 C_1 D_1}+S_{A_2 B_2 C_2 D_2}+\ldots=a^2+\frac{a^2}{2}+\frac{a^2}{2^2}+\ldots=a^2\left(1+\frac{1}{2}+\frac{1}{2^2}+\ldots\right)=a^2 \cdot \frac{1}{1-\frac{1}{2}}=2 a^2
	\end{aligned}
	$$
}

\begin{bt}%[1K5?E-4]%[1K5?E-4]
	Cho hình vuông $C_1$ có cạnh bằng $a$. Người ta chia mỗi cạnh của hình vuông thành bốn phần bằng nhau và nối các điểm chia một cách thích hợp để có hình vuông $C_2$ (tham khảo hình vẽ).
	Từ hình vuông $C_2$ lại tiếp tục làm như trên ta nhận được dãy các hình vuông $C_1, C_2, C_3, \ldots, C_n, \ldots$.Gọi $S_i$ là diện tích của hình vuông $C_i(i \in\{1 ; 2 ; 3 ; \ldots\})$. Tính tổng $S=S_1+S_2+S_3+\ldots+S_n+\ldots$
\end{bt} 
\loigiai{
	Ta có $S_1=a^2, S_2=\frac{5}{8} a^2, S_3=\frac{25}{64} a^2, \ldots$
	Nên $S=S_1+S_2+S_3+\ldots+S_n+\ldots$ là tổng của cấp số nhân lùi vô hạn với $\left\{\begin{array}{l}u_1=a^2 \\ q=\frac{5}{8}\end{array}\right.$.
	Khi đó $S=\frac{u_1}{1-q}=\frac{a^2}{1-\frac{5}{8}}=\frac{8}{3} a^2$.}
\centerline{\fcolorbox{red}{yellow!50}{\bf {BÀI TẬP TRẮC NGHIỆM
}}}
\Opensolutionfile{ans}[ans/ans-1K5-1-Dang3]
\begin{ex}%[1K5?E-4]	
	Giá trị của giới hạn $\lim \limits{n \to +\infty}\left(1+\frac{1}{2}+\frac{1}{2^2}+\ldots+\frac{1}{2^n}\right)$ là?
\choice
{1}
{\True 2} 
{$\frac{1}{2}$.}
{$\frac{3}{2}$}
\loigiai{
	Ta có: $\lim \limits{n \to +\infty}\left(1+\frac{1}{2}+\frac{1}{2^2}+\ldots+\frac{1}{2^n}\right)=\frac{1}{1-\frac{1}{2}}=2$.}
\end{ex}
\begin{ex}%[1K5?E-4]
	Tính giới hạn
	$$	I=\lim \limits{n \to +\infty}\frac{5.4^{n+1}+3^{n+2}}{2^{2 n+1}+1}$$
\choice
{$I=+\infty$}
{\True $I=10$}
{$I=0$}
{$I=20$}
\loigiai{
	- Ta có $I=\lim \limits{n \to +\infty}\frac{5.4^{n+1}+3^{n+2}}{2^{2 n+1}+1}=\lim \limits{n \to +\infty}\frac{20.4^n+9.3^n}{2 \cdot 4^n+1}=\lim \limits{n \to +\infty}\frac{20+9 \cdot\left(\frac{3}{4}\right)^n}{2+\left(\frac{1}{4}\right)^n}=\frac{20}{2}=10$.}
\end{ex}

\begin{ex}%[1K5?E-4]
	Tính tồng $S=1+\frac{1}{2}+\frac{1}{4}+\frac{1}{8}+\ldots+\frac{1}{2^n}+\ldots$
\choice
{\True 2.}
{3.}
{1.}
{$\frac{1}{2}$.}
\loigiai{
	 $S=1+\frac{1}{2}+\frac{1}{4}+\frac{1}{8}+\ldots+\frac{1}{2^n}+\ldots=\frac{u_1}{1-q}=\frac{1}{1-\frac{1}{2}}=2$.}
\end{ex}
\begin{ex}%[1K5?E-4]
	Tính $\lim \limits{n \to +\infty}\frac{3 n^3-2}{1-2 n^3}$ được kết quả là
\choice
{$\dfrac{3}{2}$.}
{\True $-\dfrac{3}{2}$.}
{$\dfrac{1}{2}$.}
{$\dfrac{-1}{2}$.}
\loigiai{
	Ta có: $\lim \limits{n \to +\infty}\frac{3 n^3-2}{1-2 n^3}=\lim \limits{n \to +\infty}\frac{n^3\left(3-\frac{2}{n^3}\right)}{n^3\left(\frac{1}{n^3}-2\right)}=\lim \limits{n \to +\infty}\frac{3-\frac{2}{n^3}}{\frac{1}{n^3}-2}=\frac{3-0}{0-2}=-\frac{3}{2}$.}
\end{ex}

\begin{ex}%[1K5?E-4]
	Cho các số $a, b, c \in R ; b+c=5 ; \lim \limits{n \to +\infty}_{x \rightarrow+\infty}\left(\sqrt{a x^2+b x}-c x\right)=2$. Tính $P=a+2 b+c$  
\choice
{$P=12$}
{$P=15$}
{\True $P=10$}
{$P=5$}

\loigiai{
	Ta có: Biện luận \\
	+ Điều kiện cẩn để tồn tại giới hạn đã cho là $a>0$\\
	+ Nếu $c \leq 0 \Rightarrow \lim \limits{n \to +\infty}_{x \rightarrow+\infty}\left(\sqrt{a x^2+b x}-c x\right)=+\infty$ (loại) \\
	+ Nếu $c>0$\\
	$2=\lim \limits{n \to +\infty}_{x \rightarrow+\infty}\left(\sqrt{a x^2+b x}-c x\right)=\lim \limits{n \to +\infty}_{x \rightarrow+\infty} \frac{\left(\sqrt{a x^2+b x}-c x\right)\left(\sqrt{a x^2+b x}+c x\right)}{\sqrt{a x^2+b x}+c x}=\lim \limits{n \to +\infty}_{x \rightarrow+\infty} \frac{\left(a-c^2\right) x^2+b x}{\sqrt{a x^2+b x}+c x}$ là hữu hạn nên: $a-c^2=0 \Leftrightarrow a=c^2$ (1)\\
	Khi đó: $2=\lim \limits{n \to +\infty}_{x \rightarrow+\infty} \frac{b x}{\sqrt{a x^2+b x}+c x}=\lim \limits{n \to +\infty}_{x \rightarrow+\infty} \frac{b}{\sqrt{a+\frac{b}{x}}+c}=\frac{b}{\sqrt{a}+c} \Leftrightarrow 2(\sqrt{a}+c)=b$\\
	Từ ta có hệ: $\left\{\begin{array}{l}a=c^2 \\ 2(\sqrt{a}+c)=b \\ b+c=5 \\ a, c>0\end{array} \Leftrightarrow\left\{\begin{array}{l}a=c^2 \\ 4 c=b \\ b+c=5 \\ a, c>0\end{array} \Leftrightarrow\left\{\begin{array}{l}a=1 \\ b=4 \\ c=1\end{array} \Rightarrow P=a+2 b+c=10\right.\right.\right.$}
\end{ex}
\begin{ex}%[1K5?E-4]
	Tính $I=\lim \limits{n \to +\infty}_{x \rightarrow+\infty} \frac{\sqrt{x^2+x+1}-x}{3}=\frac{a}{b} ; a, b \in \mathbb{N}$ và $\frac{a}{b}$ là phân số tối giản. Khi đó $2 a-b$ bằng kết quả nào sau đây?
\choice
{ 4 }
{\True -4}
{-5}
{5}
\loigiai{
	Ta có, $\lim \limits{n \to +\infty}_{x \rightarrow+\infty} \frac{\sqrt{x^2+x+1}-x}{3}=\lim \limits{n \to +\infty}_{x \rightarrow+\infty} \frac{\left(\sqrt{x^2+x+1}-x\right)\left(\sqrt{x^2+x+1}+x\right)}{3\left(\sqrt{x^2+x+1}+x\right)}$
	$$
	=\lim \limits{n \to +\infty}_{x \rightarrow+\infty} \frac{\left(x^2+x+1\right)-x^2}{3\left(\sqrt{x^2+x+1}+x\right)}=\lim \limits{n \to +\infty}_{x \rightarrow+\infty} \frac{x+1}{3\left(\sqrt{x^2+x+1}+x\right)}=\lim \limits{n \to +\infty}_{x \rightarrow+\infty} \frac{1+\frac{1}{x}}{3\left(\sqrt{1+\frac{1}{x}+\frac{1}{x^2}}+1\right)}=\frac{1}{6}
	$$
	Khi đó, $a=1 ; b=6$. Vậy $2 a-b=-4$}
\end{ex}
\begin{ex}%[1K5?E-4]
	Biết lim $\frac{\sqrt{n^2-4 n}-\sqrt{4 n^2+1}}{\sqrt{3 n^2+1}-n}=\frac{6-\sqrt{3}}{2}-\frac{a}{b}$, trong đó $\frac{a}{b}$ là phân số tối giản, $a$ và $b$ là các số nguyên dương. Chọn khẳng định đúng trong các khẳng định sau:
\choice
{ $a=b$.}
{ $a+b=7$}
{\True $a+b=14$}
{$\frac{b}{a}=\frac{7}{2}$.}
\loigiai{
	$$
	\begin{aligned}
		& \lim \limits{n \to +\infty}\frac{\sqrt{n^2-4 n}-\sqrt{4 n^2+1}}{\sqrt{3 n^2+1}-n}=\lim \limits{n \to +\infty}\frac{\sqrt{1-\frac{4}{n}}-\sqrt{4+\frac{1}{n^2}}}{\sqrt{3+\frac{1}{n^2}}-1}=\frac{-1-\sqrt{3}}{2}=\frac{6-\sqrt{3}}{2}-\frac{7}{2} . \\
		& \text { Suy ra } \frac{a}{b}=\frac{7}{2} \Rightarrow a=7 ; b=2 \Rightarrow a . b=14 .
	\end{aligned}
	$$
}
\end{ex}
\begin{ex}%[1K5?E-4]	
	Tìm giới hạn $I=\lim \limits{n \to +\infty}_{x \rightarrow+\infty}\left(x+1-\sqrt{x^2-x+2}\right)$.
\choice
{$I=\frac{46}{31}$.}
{$I=\frac{17}{11}$.}
{\True $I=\frac{3}{2}$.}
{$I=\frac{1}{2}$}
\loigiai{
	$$
	\text { Ta có } I=\lim \limits{n \to +\infty}_{x \rightarrow+\infty}\left(x+1-\sqrt{x^2-x+2}\right)=\lim \limits{n \to +\infty}_{x \rightarrow+\infty} \frac{3 x-1}{x+1+\sqrt{x^2-x+2}}=\lim \limits{n \to +\infty}_{x \rightarrow+\infty} \frac{3-\frac{1}{x}}{1+\frac{1}{x}+\sqrt{1-\frac{1}{x}+\frac{2}{x^2}}}=\frac{3}{2} \text {. }
	$$
}
\end{ex}
\begin{ex}%[1K5?E-4]
	Cho $a$ là một số thục khác 0thỏa mãn $\lim \limits{n \to +\infty}_{x \rightarrow a} \frac{x^4-a}{x-a}=4$.
	Khi đó $a$ bằng
\choice
{4.}
{-1.}
{\True 1.}
{-4.}
\loigiai{
	Ta có
	$$
	\lim \limits{n \to +\infty}_{x \rightarrow a} \frac{x^4-a}{x-a}=\lim \limits{n \to +\infty}_{x \rightarrow a} \frac{(x-a)(x+a)\left(x^2+a^2\right)}{x-a}=\lim \limits{n \to +\infty}_{x \rightarrow a}\left[(x+a)\left(x^2+a^2\right)\right]=4 a^3
	$$
	Mà theo giả thiết $\lim \limits{n \to +\infty}_{x \rightarrow a} \frac{x^4-a}{x-a}=4$. Do đó $4 a^3=4 \Leftrightarrow a=1$.}
\end{ex}
\begin{ex}%[1K5?E-4]	
	Cho $a, b, c$ là các số thực khác 0 . Tìm hệ thức liên hệ giữa $a, b, c$ để $\lim \limits{n \to +\infty}_{x \rightarrow-\infty} \frac{a x-b \sqrt{9 x^2+2}}{c x+1}=5$.
\choice
{$\frac{a-3 b}{c}=5$}
{$\frac{a+3 b}{c}=-5$}
{$\frac{a-3 b}{c}=-5$}
{\True $\frac{a+3 b}{c}=5$}
\loigiai{
	Ta có: $\lim \limits{n \to +\infty}_{x \rightarrow-\infty} \frac{a x-b \sqrt{9 x^2+2}}{c x+1}=5 \Leftrightarrow \lim \limits{n \to +\infty}_{x \rightarrow-\infty} \frac{a x-b|x| \sqrt{9+\frac{2}{x^2}}}{c x+1}=5 \Leftrightarrow \lim \limits{n \to +\infty}_{x \rightarrow-\infty} \frac{a+b \sqrt{9+\frac{2}{x^2}}}{c+\frac{1}{x}}=5$ $\Leftrightarrow \frac{a+b \sqrt{9+0}}{c+0}=5 \Leftrightarrow \frac{a+3 b}{c}=5$.}
\end{ex}
\begin{ex}%[1K5?E-4]
	Tính $\lim \limits{n \to +\infty}\left(\frac{1}{n^2+4}+\frac{2}{n^2+4}+\frac{3}{n^2+4}+\ldots+\frac{2n+4}{n^2+4}\right)$.
\choice
{$\frac{1}{2}$}
{0}
{1}
{\True 2}
\loigiai{
	Ta có $\lim \limits{n \to +\infty}\left(\frac{1}{n^2+4}+\frac{2}{n^2+4}+\frac{3}{n^2+4}+\ldots+\frac{2 n+4}{n^2+4}\right)$
	$$
	\begin{aligned}
		& =\lim \limits{n \to +\infty}\left(\frac{1+2+3+\ldots+2 n+4}{n^2+4}\right) \\
		& =\lim \limits{n \to +\infty}\frac{(1+2 n+4)(2 n+4)}{2\left(n^2+4\right)} \\
		& =\lim \limits{n \to +\infty}\frac{(2 n+5)(2 n+4)}{2\left(n^2+4\right)} \\
		& =\lim \limits{n \to +\infty}\frac{\left(2+\frac{5}{n}\right)\left(2+\frac{4}{n}\right)}{2\left(1+\frac{4}{n^2}\right)}=2
	\end{aligned}
	$$
}
\end{ex}
\begin{ex}%[1K5?E-4]
	Biết $\lim \limits{n \to +\infty}_{x \rightarrow-\infty}\left(\sqrt{x^2+b x+1}+a x\right)=-\frac{1}{2}$, tính giá trị biều thức $P=a+b$.
\choice
{1}
{3}
{\True 2}
{0}
\loigiai{
	Ta có $\lim \limits{n \to +\infty}_{x \rightarrow-\infty}\left(\sqrt{x^2+b x+1}+a x\right)=\lim \limits{n \to +\infty}_{x \rightarrow-\infty} x\left(-\sqrt{1+\frac{b}{x}+\frac{1}{x^2}}+a\right)$.\\
	Vì $\lim \limits{n \to +\infty}_{x \rightarrow-\infty} x=-\infty$ và $\lim \limits{n \to +\infty}_{x \rightarrow-\infty}\left(-\sqrt{1+\frac{b}{x}+\frac{1}{x^2}}+a\right)=-1+a$ nên để giới hạn đã cho là một số hữu hạn thi điều kiện là $-1+a=0 \Leftrightarrow a=1$.\\
	Với $a=1$ ta có $\lim \limits{n \to +\infty}_{x \rightarrow-\infty}\left(\sqrt{x^2+b x+1}+x\right)=\lim \limits{n \to +\infty}_{x \rightarrow-\infty} \frac{b x+1}{\sqrt{x^2+b x+1}-x}=\lim \limits{n \to +\infty}_{x \rightarrow-\infty} \frac{b+\frac{1}{x}}{-\sqrt{1+\frac{b}{x}+\frac{1}{x^2}}-1}=-\frac{b}{2}$.\\
	Khi đó, theo giả thiết ta có $-\frac{b}{2}=-\frac{1}{2} \Rightarrow b=1$.\\
	Vậy $P=a+b=2$.
}
\end{ex}
\begin{ex}%[1K5?E-4]
	Gọi $S_1$ là diện tích tam giác đều $A_1 B_1 C_1$ cạnh bằng $a$. Gọi $S_2$ là diện tích tam giác $A_2 B_2 C_2$ vói các đỉnh trung điểm các cạnh $A_1 B_1, B_1 C_1, A_1 C_1$, gọi $S_3$ là diện tích tam giác $A_3 B_3 C_3$ với các định trung điểm các cạnh $A_2 B_2, B_2 C_2, A_2 C_2, \ldots$ và gọ $S_n$ là diện tích tam giác $A_n B_n C_n$ với các đính trung điểm các cạnh $A_{n-1} B_{n-1}, B_{n-1} C_{n-1}, A_{n-1} C_{n-1}$. Khi $n$ tiến về dương vô cực tính tổng $S=S_1+S_2+S_3+\ldots+S_n+\ldots$
\choice
{$S=\frac{4 \sqrt{3} a^2}{3}$}
{$S=\frac{\sqrt{3} a^2}{4}$}
{\True $S=\frac{\sqrt{3} a^2}{3}$}
{$S=\frac{\sqrt{3} a^2}{2}$}
\loigiai{
	Ta có: $S_1=\frac{\sqrt{3}}{4}(a)^2, S_2=\frac{\sqrt{3}}{4}\left(\frac{a}{2}\right)^2=\frac{\sqrt{3} a^2}{16}, S_3=\frac{\sqrt{3}}{4}\left(\frac{a}{4}\right)^2=\frac{\sqrt{3} a^2}{64}, \ldots S_n=\frac{\sqrt{3}}{4}\left(\frac{a}{2^{x-1}}\right)^2=\frac{\sqrt{3} a^2}{4^{x-1}}$. \\
	Khi $n \rightarrow+\infty \Rightarrow \frac{1}{4^{n-1}} \rightarrow 0 \Rightarrow S_n \rightarrow 0$. Lúc đó: \\ $S=S_1+S_2+S_3+\ldots+S_n+\ldots$ là tồng cấp số nhân lùi
	vô hạn với $S_1=\frac{\sqrt{3}}{4}(a)^2$ và công bội $q=\frac{1}{4}$. Vậy tổng diện tích các hình là $S=S_1 \cdot \frac{1}{1-q}=\frac{\sqrt{3}}{4}(a)^2 \cdot \frac{4}{3}=\frac{\sqrt{3} a^2}{3}$.}
\end{ex}
\begin{ex}%[1K5?E-4]
	Một quả bóng tenis được thả từ độ cao $81(m)$. Mỗi lần chạm đất, quả bóng lại nảy lên hai phần ba độ cao của lần roi trưóc. Tính tổng các khoảng cách roi và nảy của quả bóng từ lúc thả bóng cho đến lúc bóng không nảy nữa.
\choice
{$243(m)$}
{\True $405(\mathrm{~m})$}
{$486(\mathrm{~m})$}
{$524(m)$}
\loigiai{
	Đặt $h_1=81(m)$. Sau lần chạm đất đầu tien, quả bóng nảy lên một độ cao $h_2=\frac{2}{3} h_1$. Tiếp đó, bóng roi từ độ cao $h_2$, chạm đất và nảy lên độ cao $h_3=\frac{2}{3} h_2$ rồi roi từ độ cao $h_3$ và cứ tiếp tụ như vậy. Sau lần chạm đất thứ $n$ từ độ cao $h_{n^{\prime}}$ quả bóng nảy lên $h_{n+1}=\frac{2}{3} h_{n^{\prime}}, \ldots$ \\
	Vậy tổng các khoảng cách roi và nảy của quả bóng từ líc thả bóng cho đến lúc bóng không nảy nữa là $d=\left(h_1+h_2+\ldots+h_n+\ldots\right)+\left(h_2+\ldots+h_n+\ldots\right) \Rightarrow d$ là tổng của hai cấp số nhân lùi vo hạn có số hạng đầu, theo thứ tự là $h_1, h_2$ và có cùng công bội $q=\frac{2}{3}$. Suy ra: $d=\frac{h_1}{1-\frac{2}{3}}+\frac{h_2}{1-\frac{2}{3}}=405(m)$.}
\end{ex}
\begin{ex}%[1K5?E-4]
	Để trang trí cho một tấm bìa hình vuông có cạnh bằng $1 \mathrm{~m}$, bạn $\mathrm{A}$ quyết định vẽ các hình vuông lên tấm bia bằng cách: hình vuông thứ nhất có các đỉnh là trung điểm của các cạnh tấm bia, hình vuông thứ hai có các đỉnh là trung điểm của các cạnh hinh vuông thứ nhất,hình vuông thứ ba có các đỉnh là trung điểm của các cạnh hình vuông thứ hai,... Giả sử quy trình vẽ hình vuông của bạn $A$ có thể tiến ra vô hạn. Tính độ dài $L$ các nét vẽ hình vuông của bạn $\mathrm{A}$.
\choice
{$1+\sqrt{2}$}
{$2+\sqrt{2}$}
{\True $4+4 \sqrt{2}$}
{$8+4 \sqrt{2}$}
\loigiai{
	Hình vuông thứ nhất có cạnh là $\frac{1}{2} \cdot \sqrt{2}=\frac{\sqrt{2}}{2}$ nên có chu vi $S_1=2 \sqrt{2}$,\\
	Hình vuông thứ hai có cạnh là $\frac{\sqrt{2}}{4} \cdot \sqrt{2}=\frac{1}{2}$ nên có chu vi $S_2=2$,\\
	Hình vuông thứ ba có cạnh là $\frac{1}{4} \cdot \sqrt{2}=\frac{\sqrt{2}}{4}$ nên có chu vi $S_3=\sqrt{2}$,\\
	Hình vuông thứ $n$ có cạnh là $\left(\frac{\sqrt{2}}{2}\right)^n$ nên có chu vi $S_n=4 .\left(\frac{\sqrt{2}}{2}\right)^n, \ldots$\\
	Khi đó độ dài các nét vẽ cạnh hình vuông là $S=S_1+S_2+\ldots+S_n+\ldots=\frac{2 \sqrt{2}}{1-\frac{\sqrt{2}}{2}}=4+4 \sqrt{2}$.}
\end{ex}
\begin{ex}%[1K5?E-4]
	Tính giới hạn của dãy số $u_n=\frac{1}{2 \sqrt{1}+\sqrt{2}}+\frac{1}{3 \sqrt{2}+2 \sqrt{3}}+\ldots+\frac{1}{(n+1) \sqrt{n}+n \sqrt{n+1}}$ :
\choice
{$+\infty$}
{$-\infty$}
{0 }
{ \True 1 }
\loigiai{
	Ta có $: \frac{1}{(k+1) \sqrt{k}+k \sqrt{k+1}}=\frac{1}{\sqrt{k}}-\frac{1}{\sqrt{k+1}}$
	Suy ra $u_n=1-\frac{1}{\sqrt{n+1}} \Rightarrow \lim \limits{n \to +\infty}u_n=1$}
\end{ex}
\begin{ex}%[1K5?E-4]
	Với $n$ là số tự nhiên lớn hơn 2 , đặt $S_n=\frac{1}{C_3^3}+\frac{1}{C_4^3}+\frac{1}{C_5^3}+\ldots+\frac{1}{C_n^3}$. Tính $\lim \limits{n \to +\infty}S_n$
\choice
{1} 
{$\frac{3}{2}$}
{3}
{$\frac{1}{3}$}
\loigiai{
	$$
	\begin{aligned}
		& S_n=\frac{1}{C_3^3}+\frac{1}{C_4^3}+\frac{1}{C_5^3}+\ldots+\frac{1}{C_n^3}=\frac{3 !}{1.2 .3}+\frac{3 !}{2.3 .4}+\frac{3 !}{3.4 .5}+\ldots+\frac{3 !}{n(n-1)(n-2)} \\
		& =6\left[\frac{1}{2}\left(-\frac{1}{3.2}+\frac{1}{2.1}-\frac{1}{4.3}+\frac{1}{3.2}+\ldots-\frac{1}{n(n-1)}+\frac{1}{(n-1)(n-2)}\right)\right]=3\left(\frac{1}{2.1}-\frac{1}{n(n-1)}\right)
	\end{aligned}
	$$
	Vạyy $\lim \limits{n \to +\infty}S_n=\lim \limits{n \to +\infty}3\left(\frac{1}{2}-\frac{1}{n(n-1)}\right)=\frac{3}{2}$}
\end{ex}
\begin{ex}%[1K5?E-4]
	Cho $f(n)=\left(n^2+n+1\right)^2+1$. Xét dãy số $\left(u_n\right)$ sao cho $u_n=\frac{f(1) \cdot f(3) \cdot f(5) \ldots f(2 n-1)}{f(2) \cdot f(4) \cdot f(6) \ldots f(2 n)}$. Tính $\lim \limits{n \to +\infty}n \sqrt{u_n}$
\choice
{\True $\lim \limits{n \to +\infty}n \sqrt{u_n}=\frac{1}{\sqrt{2}}$}
{$\lim \limits{n \to +\infty}n \sqrt{u_n}=\sqrt{2}$}
{$\lim \limits{n \to +\infty}n \sqrt{u_n}=\sqrt{3}$}
{$\lim \limits{n \to +\infty}n \sqrt{u_n}=\frac{1}{\sqrt{3}}$}
\loigiai{
	$g(n)=\frac{f(2 n-1)}{f(2 n)} \Rightarrow g(n)=\frac{\left(4 n^2-2 n+1\right)^2+1}{\left(4 n^2+2 n+1\right)^2+1}$ \\ 
	Đặt $\left\{\begin{array}{l}a=4 n^2+1 \\ b=2 n\end{array} \Rightarrow\left\{\begin{array}{l}a=b^2+1 \\ a-2 b=(2 n-1)^2 \\ a+2 b=(2 n+1)^2\end{array}\right.\right.$.\\ 
	Suy ra $g(n)=\frac{(a-b)^2+1}{(a+b)^2+1}=\frac{a^2-2 a b+b^2+1}{a^2+2 a b+b^2+1}=\frac{a^2-2 a b+a}{a^2+2 a b+b}=\frac{a-2 b+1}{a+2 b+1}=\frac{(2 n-1)^2+1}{(2 n+1)^2+1}$. \\
	$\Rightarrow u_n=g(1) g(2) \ldots \ldots g(n)=\frac{2}{10} \cdot \frac{10}{26} \ldots \ldots \frac{(2 n-1)^2+1}{(2 n+1)^2+1}=\frac{2}{(2 n+1)^2+1}$.
	$\lim \limits{n \to +\infty}n \sqrt{u_n}=\lim \limits{n \to +\infty}n \cdot \sqrt{\frac{2}{(2 n+1)^2+1}}=\frac{1}{\sqrt{2}}$}
\end{ex}

\Closesolutionfile{ans}