
\begin{dang}{Chứng minh phương trình có nghiệm}
		Để chứng minh minh một phương trình có nghiệm, ta thường tiến hành
	\begin{itemize}
		\item Đặt $f(x)$ là vế trái của phương trình (ứng với vế phải bằng $0$).
		\item Lập luận hàm số $f(x)$ liên tục trên $\mathbb{R}$ hoặc trên một đoạn con của $\mathbb{R}$ liên quan tới bài toán.
		\item Chỉ ra tồn tại các số $a$, $b$ ($a<b$) với $a$, $b$ thuộc đoạn con đang xét mà $f(a)\cdot f(b)<0$. Dựa vào tính chất của hàm số liên tục ta suy ra phương trình $f(x)=0$ có nghiệm thuộc khoảng $(a;b)$.		
		\begin{note}
			Nếu bài toán yêu cầu chứng minh phương trình có $k$ nghiệm thì cần lập luận $k$ đoạn con như trên.\\
			Nhiều trường hợp việc chỉ ra các số $a$, $b$ gặp khó khăn, ta có thể khai thác $\lim \limits _{x\to +\infty}f(x)$ hoặc $\lim \limits _{x\to -\infty}f(x)$ để có cơ sở lập luận. 
		\end{note}
	\end{itemize}
\end{dang}
\subsubsection{Ví dụ mẫu}
\begin{vd}[NB]%[DCHT Toán 11 - KNTT -Nguyễn Thành Nhân] %[1K5YG-7]
Chứng minh rằng phương trình $x^5+4x^3-x^2-1=0$ có ít nhất một nghiệm thuộc khoảng $(0;1)$.
\loigiai{
Đặt $f(x) = x^5+4x^3-x^2-1$. Khi đó $f(x)$ liên tục trên $\mathbb{R}$ nên cũng liên tục trên đoạn $[0;1]$.\\
Ta có $f(0) = -1$ và $f(1) = 3$ nên $f(0) \cdot f(1) < 0$.\\
Do đó phương trình $f(x) =0$ có ít nhất một nghiệm thuộc $(0;1)$.
}
\end{vd}
\begin{vd}[TH]%[DCHT Toán 11 - KNTT -Nguyễn Thành Nhân] %[1K5BG-7]
Chứng minh rằng phương trình $x^5-5x^3+4x-1=0$ có đúng $5$ nghiệm phân biệt.
\loigiai{
Đặt $f(x) = x^5-5x^3+4x-1$. Khi đó $f(x)$ liên tục trên $\mathbb{R}$.\\
Ta có $f(-2)=-1<0$; $f\left(-\dfrac{3}{2}\right)=\dfrac{173}{32}>0$; $f(0)=-1<0$; $f\left(\dfrac{1}{2}\right)=\dfrac{18}{32}>0$; $f(1)=-1<0$; $f(3)=119>0$. \\
Dựa vào tính chất liên tục của hàm số $f(x)$ trên $\mathbb{R}$, suy ra trên mỗi khoảng $\left(-2;-\dfrac{3}{2}\right)$; $\left(-\dfrac{3}{2};0\right)$; $\left(0;\dfrac{1}{2}\right)$; $\left(\dfrac{1}{2};1\right)$; $(1;3)$ có ít nhất một nghiệm. \\
Do đó phương trình $f(x)=0$ có ít nhất $5$ nghiệm phân biệt. Vì $f(x)$ là phương trình bậc $5$ nên có tối đa $5$ nghiệm.\\
Vậy phương trình $f(x)=0$ có đúng $5$ nghiệm phân biệt.
}
\end{vd}

\begin{vd}[TH]%[DCHT Toán 11 - KNTT -Nguyễn Thành Nhân] %[1K5BG-7]
	Chứng minh rằng phương trình $x^3-2mx^2-x+m=0$ luôn có nghiệm với mọi $m$ ($m$ là tham số).
	\loigiai{
		Xét hàm số $f(x)=x^3-2mx^2-x+m$. Khi đó $f(x)$ liên tục trên $\mathbb{R}$.\\
		Ta có $f(0)=m$ và $f(1)=-m$ nên $f(0)\cdot f(1)=-m^2\le 0$ với mọi $m$ nên phương trình $f(x)=0$ luôn có nghiệm thuộc đoạn $[0;1]$ với mọi $m$.
	}
\end{vd}

\begin{vd}[TH]%[DCHT Toán 11 - KNTT -Nguyễn Thành Nhân] %[1K5BG-7]
Chứng minh rằng phương trình $\left(1-m^2\right)(x+1)^3+x^2-x-3$ luôn có nghiệm với mọi giá trị của tham số $m$. 
\loigiai{
Đặt $f(x)= \left(1-m^2\right)(x+1)^3+x^2-x-3$ thì $f(x)$ liên tục trên $\mathbb{R}$. Ta có
\[f(0)=-m^2-2<0,\,\forall m.\]
và 
\[f(-2)=m^2+2>0,\,\,\forall m.\]
Vì $f(-2)\cdot f(0)<0$ nên phương trình $f(x)=0$ luôn có ít nhất một nghiệm thuộc khoảng $(-2;0)$ với mọi $m$.
}
\end{vd}
\begin{vd}[TH]%[DCHT Toán 11 - KNTT -Nguyễn Thành Nhân] %[1K5BG-7]
	Chứng minh rằng phương trình $m(x-8)^3(x-9)^4+2x-17=0$ luôn có nghiệm với mọi giá trị của $m$.
	\loigiai{
Xét hàm số $f(x)=	m(x-8)^3(x-9)^4+2x-17$. Hàm số đã cho liên tục trên $\mathbb{R}$.\\
Ta có $f(8)=-1, f(9)=1$. Vậy $f(8)\cdot f(9)<0$. Điều này suy ra phương trình có ít nhất một nghiệm trên $(8;9)$.
}
\end{vd}
\begin{vd}[VD]%[DCHT Toán 11 - KNTT -Nguyễn Thành Nhân]%[1K5KG-7]
	Chứng minh rằng phương trình $m(x+1)^2(x-2)^3+(x+2)(x-3)=0$ luôn có nghiệm với mọi tham số $m$.
	\loigiai{
		Xét hàm số $f(x)=m(x+1)^2(x-2)^3+(x+2)(x-3)$ xác định và liên tục trên $[-2;3]$.\\
		Ta có $f(-2)=-64m$, $f(3)	
		=16m$, $f(-2) \cdot f(3)=-1024m^2 \le 0$.
		\begin{itemize}
			\item Với $m=0$ suy ra $f(-2)=f(3)=0$ suy ra phương trình $f(x)=0$ có hai nghiệm $x=-2$ và $x=3$.
			\item Với $m \ne 0$ suy ra $f(-2) \cdot f(3) <0$, suy ra tồn tại $x_0 \in (-2;3)$ sao cho $f(x_0)=0$.
		\end{itemize}
		Do đó phương trình $f(x)=0$ luôn có nghiệm.\\
		Vậy phương trình ban đầu luôn có nghiệm.
	} 
\end{vd}
\begin{vd}[VDC]%[DCHT Toán 11 - KNTT -Nguyễn Thành Nhân]%[1K5GG-7]
Với mọi giá trị thực của tham số $m$, chứng minh phương trình $\left(m^2+1\right)x^3-2m^2x^2-4x+m^2+1=0$ luôn có ba nghiệm thực.
\loigiai
{
Đặt $f(x)=\left(m^2+1\right)x^3-2m^2x^2-4x+m^2+1$.\\
Hàm số $f(x)=\left(m^2+1\right)x^3-2m^2x^2-4x+m^2+1$ là một hàm số đa thức nên nó liên tục trên $\mathbb{R}$. Suy ra, nó cũng liên tục trên mỗi đoạn $[-3;0]$, $[0;1]$, $[1;2]$.
Ta có
\begin{itemize}
\item $f(-3) =-27m^2-27-18m^2+12+m^2+1=-44m^2-14<0$, với mọi $m\in\mathbb{R}$.
\item $f(0)=m^2+1>0$, với mọi $m\in\mathbb{R}$.
\item $f(1)=m^2+1-2m^2-4+m^2+1=-2<0$.
\item $f(2)=8m^2+8-8m^2-8+m^2+1=m^2+1>0$, với mọi $m\in\mathbb{R}$.
\end{itemize}
Vì $f(-3)\cdot f(0)<0$ nên phương trình đã cho có ít nhất một nghiệm thuộc khoảng $(-3;0)$.\\
Vì $f(0)\cdot f(1)<0$ nên phương trình đã cho có ít nhất một nghiệm thuộc khoảng $(0;1)$.\\
Vì $f(1)\cdot f(2)<0$ nên phương trình đã cho có ít nhất một nghiệm thuộc khoảng $(1;2)$.\\
Phương trình $\left(m^2+1\right)x^3-2m^2x^2-4x+m^2+1=0$ là một phương trình bậc ba $(\text{vì } m^2+1\neq 0,\forall m\in\mathbb{R})$.\\
Vậy phương trình $\left(m^2+1\right)x^3-2m^2x^2-4x+m^2+1=0$ luôn có ba nghiệm thực.
}
\end{vd}
\begin{vd}[VD]%[DCHT Toán 11 - KNTT -Nguyễn Thành Nhân]%[1K5KG-7]
	Chứng minh rằng phương trình sau luôn có nghiệm với mọi giá trị của tham số $m\ge -1$
	\[ (m-1)x^6+\left(m^2-\sqrt{4m+4}\right)x^3+6x-3=0. \]
	\loigiai{
		 Đặt $f(x)=(m-1)x^6+\left(m^2-\sqrt{4m+4}\right)x^3+6x-3$. Khi đó $f(x)$ liên tục trên đoạn $[0;1]$. Ta có 
		 \begin{align*}
		 	f(1)&=(m-1)+\left(m^2-\sqrt{4m+4}\right)+6-3\\
		 	&=m^2+(m+1)-2\sqrt{m+1}+1\\
		 	&=m^2+\left(\sqrt{m+1}-1\right)^2.
		 \end{align*}
	 \begin{itemize}
	 	\item Nếu $m=\sqrt{m+1}-1=0$ hay $m=0$ thì $f(1)=0$.
	 	\item Nếu $m\ne 0$ thì $f(1)>0$, mà $f(0)=-3<0$ nên $f(x)$ có một nghiệm trong khoảng $(0;1)$.
	 \end{itemize}
	 Vậy phương trình $f(x)=0$ luôn có nghiệm với mọi $m\ge -1$. 
	}
\end{vd}
\begin{vd}[VDC]%[DCHT Toán 11 - KNTT -Nguyễn Thành Nhân]%[1K5GG-7]
	Với mọi giá trị thực của tham số $m,$ chứng minh phương trình $x^5+x^2-\left(m^2+2\right)x-1=0$ luôn có ít nhất ba nghiệm thực.
	\loigiai{
		Xét hàm số $ f(x)=x^5+x^2-\left(m^2+2\right)x-1$ liên tục trên $\mathbb R$.\\
		Ta có $ f(0)=-1 < 0$, $f(-1)=m^2+1 > 0$.\\
		Mặt khác, vì $\lim\limits_{x\to-\infty}f(x)=-\infty$ nên tồn tại $a <-1$ sao cho $f(a) < 0$.\\
		Vì $\lim\limits_{x\to+\infty}f(x)=+\infty$ nên tồn tại $b > 0$ sao cho $f(b) > 0$.\\
		Khi đó
		\begin{itemize}
			\item $ f(a)\cdot f\left(-1\right) < 0$ suy ra phương trình $ f(x)=0$ có ít nhất $ 1$ nghiệm thuộc $\left(a;-1\right)$,
			\item $ f\left(-1\right)\cdot f(0) < 0$ suy ra phương trình $ f(x)=0$ có ít nhất $ 1$ nghiệm thuộc $\left(-1;0\right)$,
			\item $ f(0)\cdot f(b) < 0$ suy ra phương trình $ f(x)=0$ có ít nhất $ 1$ nghiệm thuộc $\left(0;b\right)$.
		\end{itemize}
		Vậy phương trình đã cho có ít nhất $ 3$ nghiệm.
	}
\end{vd}
\begin{vd}[VDC]%[DCHT Toán 11 - KNTT -Nguyễn Thành Nhân]%[1K5GG-7]
Cho $a$, $b$ là hai số thực thỏa mãn $9 a+24 b>128$. Chứng minh phương trình $a x^2+b x-2=0$ có ít nhất một nghiệm thuộc khoảng $(0 ; 1)$.	
	\loigiai{
Xét hàm số $ f(x)=x^2+b x-2$ liên tục trên $\mathbb R$.\\		
Ta có $f(0)=-2<0$.\\
Ta có $f\left(\dfrac{1}{2}\right)=\dfrac{a}{4}+\dfrac{b}{2}-2$, $f\left(1\right)=a+b-2$.\\
Khi đó $2f(1)+4f\left(\dfrac{1}{2}\right)=3a+4b-12>\dfrac{128}{3}-12=\dfrac{92}{3}>0$. Suy ra một trong hai số $f(1)$  hoặc $f\left(\dfrac{1}{2}\right)$ là số dương.\\
Do đó $\hoac{&f(0)\cdot f\left(\dfrac{1}{2}\right)<0\\&f(0)\cdot f(1)<0.}$\\
Khi đó phương trình $f(x)=0$ có ít nhất một nghiệm thuộc khoảng $(0;1)$.
	}
\end{vd}
\begin{vd}[VDC]%[DCHT Toán 11 - KNTT -Nguyễn Thành Nhân]%[1K5GG-7]
	Cho phương trình $ax^2+bx+c=0$ với $5a+3b+3c=0$. Chứng minh rằng phương trình luôn có nghiệm.	
	\loigiai{Do $5a+3b+3c=0$ nên $b=-\dfrac{5}{3}a-c$.\\
	Xét hàm số  $f(x)=ax^2+bx+c$ trên $\left[ 0;\dfrac{5}{3}\right] $.\\
	Ta có $f(0)=c$, $f\left( \dfrac{5}{3} \right)=\dfrac{25}{9}\cdot a+\dfrac{5}{3}\cdot b+c =\dfrac{25}{9}\cdot a+\dfrac{5}{3}\left(-\dfrac{5}{3}a-c \right)+c=-\dfrac{2}{3}c  $.
\begin{itemize}
	\item Nếu $c=0$ thì $f(0)=f\left(\dfrac{5}{3} \right)=0 $, phương trình đã cho có hai nghiệm là $x=0$, $x=\dfrac{5}{3}$.
	\item Nếu $c\ne 0$ thì $f(0)\cdot f\left(\dfrac{5}{3} \right)=-\dfrac{2}{3}c^2<0 $. Vì $f(x)$ là hàm đa thức nên liên tục trên $\mathbb{R}$.\\
	 Do đó, nó liên tục trên $\left[ 0;\dfrac{5}{3}\right] $.\\
	Từ đó suy ra phương trình $f(x)=0$  có ít nhất một nghiệm trên $\left( 0;\dfrac{5}{3}\right)$.
\end{itemize}
Vậy phương trình đã cho luôn có nghiệm.
	}
\end{vd}
\subsubsection{Bài tập rèn luyện}
\centerline{\fcolorbox{red}{yellow!50}{\bf {BÀI TẬP TỰ LUẬN }}}
\begin{bt}[NB]%[DCHT Toán 11 - KNTT -Tên GV] %[1K5YG-7]
Chứng minh rằng phương trình $x^5+4x^3-x^2-1=0$ có ít nhất một nghiệm thuộc khoảng $(0;1)$.
\loigiai{
Đặt $f(x) = x^5+4x^3-x^2-1$ liên tục trên $[0;1]$.\\
Ta có $f(0) = -1$ và $f(1) = 3$ nên $f(0) \cdot f(1) < 0$.\\
Do đó phương trình $f(x) =0$ có ít nhất một nghiệm thuộc $(0;1)$.
}
\end{bt}
\begin{bt}[NB]%[DCHT Toán 11 - KNTT -Tên GV] %[1K5BG-7]
		Chứng minh rằng phương trình $ 2x^4-3x^3-5=0 $ có ít nhất một nghiệm.
		\loigiai{
	Đặt $f(x)= 2x^4-3x^3-5 $, $ f(x) $ là hàm đa thức nên liên tục trên $ \mathbb{R} $.\\
	Do đó $ f(x) $ liên tục trên đoạn $ [1;2] $.\\
	Ta có $ \heva{&f(1)=-6\\&f(2)=3} $ suy ra $ f(1)\cdot f(2)=-18<0 $.\\
	Nên phương trình $ f(x)=0 $ có ít nhất một nghiệm nằm trong khoảng $ (1;2) $.\\
	Vậy phương trình đã cho có ít nhất một nghiệm.
			}
	\end{bt}
	\begin{bt}[TH]%[DCHT Toán 11 - KNTT -Nguyễn Thành Nhân]%[1K5BG-7]
Chứng minh rằng phương trình $-3x^5+8x^2-1=0$ có ít nhất một nghiệm.
\loigiai{
Xét hàm số $f(x)=-3x^5+8x^2-1$ liên tục trên $\mathbb{R}$ nên liên tục trên đoạn $[0;1]$. Ta có $f(0)\cdot f(1)=-1\cdot 4=-4<0$.\\
Vậy có ít nhất một số $x_0=c\in \left(0;1\right)$ để $f(x_0)=0$, hay nói cách khác phương trình $f(x)=0\Leftrightarrow -3x^5+8x^2-1=0$ có nghiệm.
}
\end{bt}
\begin{bt}[TH]%[DCHT Toán 11 - KNTT -Nguyễn Thành Nhân] %[1K5BG-7]
Chứng minh phương trình $\left(m^2-2m+3\right)x^4-2x-4=0$ luôn có nghiệm âm với mọi giá trị thực của tham số $m$.
\loigiai
{Hàm số $f(x)=\left(m^2-2m+3\right)x^4-2x-4$ liên tục trên $\mathbb{R}$.\\
Ta có $f(0)=-4$.\\
Vì $m^2-2m+3=(m-1)^2+2>0,\ \forall m$ nên $$\lim\limits_{x\to -\infty} f(x)=\lim\limits_{x\to -\infty} x^4\left(m^2-2m+3-\dfrac{2}{x^3}-\dfrac{4}{x^4}\right)=+\infty.$$
Do đó, tồn tại $a<0$ sao cho $f(a)>0$.\\
Vì hàm số $f(x)$ liên tục trên $\mathbb{R}$ nên nó cũng liên tục trên $[a; 0]$. Hơn nữa $f(a)\cdot f(0)<0$ nên phương trình $f(x)=0$ luôn có ít nhất một nghiệm thuộc khoảng $(a; 0)$.\\
Vậy phương trình đã cho luôn có nghiệm âm với mọi giá trị thực của tham số $m$.
}
\end{bt}

\begin{bt}[TH]%[DCHT Toán 11 - KNTT -Tên GV] %[1K5BG-7]
Chứng minh phương trình $x^4 + x^3 + mx^2 + x\left(2m - 1\right) + m\sin \left(\pi x\right)=1$ có nghiệm với mọi $m$. 
\loigiai{
$x^4 + x^3 + mx^2 + x\left(2m - 1\right) + m\sin \left(\pi x\right)=1\Leftrightarrow x^4 + x^3 - x - 1 + m\left(x^2 + \sin \pi x + 2x\right)=0$.\\
Đặt $f(x)=x^4 + x^3 - x - 1 + m\left(x^2 + \sin \pi x + 2x\right)$.\\
Ta có $f(x)$ liên tục trên $\mathbb{R}$.\\
 $f\left(0\right)= - 1$ và $f\left(- 2\right)=9\Rightarrow f(0)\cdot f(-2)<0$.\\
 Suy ra phương trình đã cho luôn có ít nhất một nghiệm thuộc khoảng $ (-2;0) $.
}
\end{bt}

\begin{bt}[TH]%[DCHT Toán 11 - KNTT -Tên GV] %[1K5BG-7]
Chứng minh phương trình $x^4+mx^2+(3m-1)x-5+2m=0$ luôn có ít nhất một nghiệm với mọi số thực $m$.
\loigiai{
Đặt $f(x)=x^4+mx^2+(3m-1)x-5+2m$.\\
Vì $f(x)$ là hàm đa thức nên $f(x)$ liên tục trên $\mathbb{R}$.\\
Lại có $f(-2)=13, f(-1)=-3$	$\Rightarrow f(-3) \cdot f(-1)<0$, do đó phương trình $f(x)=0$ có ít nhất một nghiệm trên $(-3;-1)$.\\
Do đó phương trình luôn có nghiệm với mọi $m$.
}
\end{bt}

\begin{bt}[TH]%[DCHT Toán 11 - KNTT -Tên GV] %[1K5BG-7]
	Chứng minh rằng phương trình $m(x-8)^3(x-9)^4+2x-17=0$ luôn có nghiệm với mọi giá trị của $m$.
	\loigiai{
Xét hàm số $f(x)=	m(x-8)^3(x-9)^4+2x-17$. Hàm số đã cho liên tục trên $\mathbb{R}$.\\
Ta có $f(8)=-1, f(9)=1$. Vậy $f(8)\cdot f(9)<0$. Điều này suy ra phương trình có ít nhất một nghiệm trên $(8;9)$.
}
\end{bt}
\begin{bt}[TH]%[DCHT Toán 11 - KNTT -Tên GV] %[1K5BG-7]
Chứng minh phương trình $\left(m^2-2m+3\right)x^4-2x-4=0$ luôn có nghiệm âm với mọi giá trị thực của tham số $m$.
\loigiai
{Hàm số $f(x)=\left(m^2-2m+3\right)x^4-2x-4$ có tập xác định là $\mathscr{D}=\mathbb{R}$.\\
Ta có $f(0)=-4$.\\
Vì $m^2-2m+3=(m-1)^2+2>0,\ \forall m$ nên $$\lim\limits_{x\to -\infty} f(x)=\lim\limits_{x\to -\infty} x^4\left(m^2-2m+3-\dfrac{2}{x^3}-\dfrac{4}{x^4}\right)=+\infty.$$
Do đó, tồn tại $a<0$ sao cho $f(a)>0$.\\
Vì hàm số $f(x)$ liên tục trên $\mathbb{R}$ nên nó cũng liên tục trên $[a; 0]$. Hơn nữa $f(a)\cdot f(0)<0$ nên phương trình $f(x)=0$ luôn có ít nhất một nghiệm thuộc khoảng $(a; 0)$.\\
Vậy phương trình đã cho luôn có nghiệm âm với mọi giá trị thực của tham số $m$.
}
\end{bt}
\begin{bt}[TH]%[DCHT Toán 11 - KNTT -Tên GV] %[1K5BG-7]
	Chứng minh rằng phương trình $m\cdot\sin 2x+x^2\cdot\cos x+\left(m^2+1\right)\cdot\cos 2x=0$ luôn có nghiệm thuộc khoảng $\left(0 ;\dfrac{\pi}{2}\right)$ với mọi tham số $m$.
	\loigiai{
Đặt $f(x)=m\cdot\sin 2x+x^2\cdot\cos x+\left(m^2+1\right)\cdot\cos 2x$.
\begin{itemize}
	\item $f(x)$ liên tục trên $\mathbb{R}$.
	\item Ta có $f(0)=m^2+1>0\ \forall m$ và $ f\left(\dfrac{\pi}{2}\right)=-m^2-1<0\ \forall m $.
\end{itemize}
Suy ra phương trình $f(x)=0$ có ít nhất một nghiệm thuộc khoảng $\left(0 ;\dfrac{\pi}{2}\right)$.
	}
\end{bt}
\begin{bt}[TH]%[DCHT Toán 11 - KNTT -Tên GV] %[1K5BG-7]
	Chứng minh phương trình $x^3+m x^2-4m x=19-3m$ có nghiệm với mọi $m$.
	\loigiai{
		Ta có $x^3+m x^2-4m x=19-3m \Leftrightarrow x^3+m x^2-4m x-19+3m=0$.\\
		Đặt $f\left(x\right)=x^3+m x^2-4m x-19+3m$.\\
		$f\left(x\right)$ liên tục trên $\mathbb{R}$.\\
		$f(1)=-18$, 
		$f(3)=8$,\\
		$\Rightarrow f(1)f(3)<0$\\
		$\Rightarrow$ phương trình có nghiệm với mọi $m$.
	}
\end{bt}
\begin{bt}%[DCHT Toán 11 - KNTT -Nguyễn Thành Nhân]%[1K5BG-7]
Tìm các giá trị nguyên của tham số $m$ để phương trình sau vô nghiệm 
\[
(m^2-1)(x-1)^{2020} =2019\sqrt{4-x}. 
\]
\loigiai{
Điều kiện xác định $x \leq 4$. 
\begin{itemize}
\item Dễ thấy với $m^2 -1 <0 \Leftrightarrow -1 <m <1$, phương trình vô nghiệm. 
\item Với $m^2 -1 =0 \Leftrightarrow m = \pm 1$, phương trình có nghiệm $x=4$. 
\item Với $m^2 -1 >0$, do $x=1$ không là nghiệm nên phương trình tương đương với 
\[
m^2 -1 = \dfrac{2019\sqrt{4-x}}{(x-1)^{2020}}. \tag{1}
\]
Xét hàm số $f(x) = \dfrac{2019\sqrt{4-x}}{(x-1)^{2020}}$. \\
Ta có $\displaystyle \lim_{x \to 1^+} f(x) = +\infty$, do đó tồn tại $ 1<\alpha<4$ sao cho $f(\alpha)  > m^2-1$. \\
Dễ thấy hàm số $f(x)$ liên tục trên đoạn $[\alpha ; 4]$ và có $0 = f(0) < m^2-1 < f(\alpha)$, do đó theo định lý giá trị trung gian tồn tại $x_0 \in (\alpha ;4)$ sao cho $f(x_0) = m^2-1$. Điều đó có nghĩa là phương trình (1) có nghiệm $x=x_0$.\\
Vậy $m=0$ là số nguyên duy nhất để phương trình đã cho vô nghiệm. 
\end{itemize}
}
\end{bt}
\begin{bt}%[DCHT Toán 11 - KNTT -Nguyễn Thành Nhân]%%[1K5BG-7]
	Chứng minh phương trình $(1-m^2)(x+1)^3+x^2-x-3=0$ có nghiệm với mọi $m$.
	\loigiai{
		Đặt $f(x)=(1-m^2)(x+1)^3+x^2-x-3$, $f(x)$ là hàm đa thức nên xác định và liên tục trên $\mathbb{R}$. Suy ra $f(x)$ liên tục trên đoạn $[-2;-1]$.\\
		Ta có $\heva{&f(-1)=-1<0\\&f(-2)=(1-m^2)(-1)^3+(-2)^2-(-2)-3=m^2+2>0,\,\forall m\in\mathbb{R}.}$\\
		Vì $f(-2)\cdot f(-1)<0$, $\forall m\in\mathbb{R}$ nên phương trình $f(x)=0$ có ít nhất một nghiệm trong khoảng $(-2;-1)$, $\forall m\in\mathbb{R}$.\\
		Vậy phương trình $(1-m^2)(x+1)^3+x^2-x-3=0$ có nghiệm với mọi $m$.
	}
\end{bt}

\begin{bt}%[DCHT Toán 11 - KNTT -Nguyễn Thành Nhân]%[1K5BG-7]
Chứng minh rằng phương trình $x^5+4x^3-x^2-1=0$ có ít nhất một nghiệm thuộc khoảng $(0;1)$.
\loigiai{
Đặt $f(x) = x^5+4x^3-x^2-1$ liên tục trên $[0;1]$.\\
Ta có $f(0) = -1$ và $f(1) = 3$ nên $f(0) \cdot f(1) < 0$.\\
Do đó phương trình $f(x) =0$ có ít nhất một nghiệm thuộc $(0;1)$.
}
\end{bt}

\begin{bt}%[DCHT Toán 11 - KNTT -Nguyễn Thành Nhân]%[1K5BG-7]
		Chứng minh rằng phương trình $ 2x^4-3x^3-5=0 $ có ít nhất một nghiệm.
		\loigiai{
	Đặt $f(x)= 2x^4-3x^3-5 $, $ f(x) $ là hàm đa thức nên liên tục trên $ \mathbb{R} $.\\
	Do đó $ f(x) $ liên tục trên đoạn $ [1;2] $.\\
	Ta có $ \heva{&f(1)=-6\\&f(2)=3} $ suy ra $ f(1)\cdot f(2)=-18<0 $.\\
	Nên phương trình $ f(x)=0 $ có ít nhất một nghiệm nằm trong khoảng $ (1;2) $.\\
	Vậy phương trình đã cho có ít nhất một nghiệm.
			}
	\end{bt}
\begin{bt}%[DCHT Toán 11 - KNTT -Nguyễn Thành Nhân]%[1K5BG-7]
	Chứng minh rằng phương trình $2x^3 - 5x + 1 = 0$ có đúng ba nghiệm.
	\loigiai{
	Đặt $f(x) = 2x^3 - 5x + 1$. Tập xác định của hàm số là $\mathscr D = \mathbb{R}$.\\
	Ta có $f(-2) = -5$, $f(0) = 1$, $f(1) = -2$, $f(2) = 7$.\\
	Vì $f(x)$ liên tục trên $\mathbb{R}$ nên $f(x)$ liên tục trên $[-2; 0]$ và $f(-2)\cdot f(0) = -5 \cdot 1 = -5 < 0$.\\
	Do đó phương trình $f(x) = 0$ có ít nhất một nghiệm $x_1 \in (-2; 0)$. \hfill (1)\\
		Vì $f(x)$ liên tục trên $\mathbb{R}$ nên $f(x)$ liên tục trên $[0; 1]$ và $f(0)\cdot f(1) = 1 \cdot (-2) = -2 < 0$.\\
	Do đó phương trình $f(x) = 0$ có ít nhất một nghiệm $x_2 \in (0; 1)$. \hfill (2)\\
		Vì $f(x)$ liên tục trên $\mathbb{R}$ nên $f(x)$ liên tục trên $[1; 2]$ và $f(1)\cdot f(2) = -2 \cdot 7 = -14 < 0$.\\
	Do đó phương trình $f(x) = 0$ có ít nhất một nghiệm $x_3 \in (1; 2)$. \hfill (3)\\
	Do các khoảng $(-1; 0)$; $(0; 1)$; $(1; 2)$ không giao nhau
	Từ (1), (2) và (3), suy ra phương trình $2x^3 - 5x + 1 = 0$ có ít nhất ba nghiệm.\\
	Mà phương trình bậc ba có không quá $3$ nghiệm nên phương trình $2x^3 - 5x + 1 = 0$ có đúng ba nghiệm phân biệt. 
	}
\end{bt}
\begin{bt}[VD]%[DCHT Toán 11 - KNTT -Tên GV] %[1K5KG-7]
	Chứng minh phương trình $2x^3-3x^2-1=0$ có nghiệm $x_0\in \left(\sqrt[3]{4};2\right)$.
	\loigiai{
		Đặt $f(x)=2x^3-3x^2-1$ thì $f(x)$ liên tục trên $\mathbb{R}$.\\
		Ta có $f(1)=-2<0$; $f(2)=3>0$. Suy ra phương trình $f(x)=0$ có nghiệm $x_0\in (1;2)$.\\
		Lại áp dụng bất đẳng thức Cauchy, ta có
		\[2x_0^3=3x_0^2+1=x_0^2+x_0^2+x_0^2+1\geq 4\sqrt[4]{x_0^6}=4\sqrt{x_0^3}.\]
		Suy ra $x_0\geq \sqrt[3]{4}$. Nhưng tại $x_0=\sqrt[3]{4}$ thì dấu đẳng thức không xảy ra, suy ra $x_0> \sqrt[3]{4}$.\\
	Vậy, phương trình đã cho có nghiệm $x_0\in \left(\sqrt[3]{4};2\right)$.	
	}
\end{bt}
\begin{bt}[VD]%[DCHT Toán 11 - KNTT -Nguyễn Thành Nhân]%[1K5KG-7]
	Chứng minh phương trình $x^7-3x^6+x^4+x^3-(m^2+3)x+2=0$ có ít nhất một nghiệm dương với mọi tham số $m \in \mathbb{R}$.
	\loigiai{ Đặt $f(x)=x^7-3x^6+x^4+x^3-(m^2+3)x+2$, khi đó hàm số $f(x)$ liên tục trên $\mathbb{R}$.\\
		Ta có $f(0)=2$, $f(1)=-m^2-1$. 
		Suy ra $f(0)\cdot f(1)=2\left(-m^2-1\right)<0$ với mọi $m \in \mathbb{R}$.\\
		Vậy phương trình đã cho có ít nhất một nghiệm dương thuộc khoảng $(0;1)$ với mọi $m \in \mathbb{R}$.
	}
\end{bt}
\begin{bt}[VD]%[DCHT Toán 11 - KNTT -Nguyễn Thành Nhân]%[1K5KG-7]
Cho phương trình $x^4-x^3-(m^2+5)x^2+2(m^2+2)x+4=0$. Chứng minh rằng với mọi số nguyên $m$ thì phương trình sau luôn có đúng $4$ nghiệm phân biệt.
	\loigiai{
	Ta có 
	\allowdisplaybreaks
\begin{eqnarray*}
	&&x^4-x^3-(m^2+5)x^2+2(m^2+2)x+4=0\\
	&\Leftrightarrow&(x-2)\left(x^3+x^2-(m^2+3)x-2\right)=0\\
	&\Leftrightarrow&\hoac{&x-2=0\\&x^3+x^2-(m^2+3)x-2=0.}	
\end{eqnarray*}	
Phương trình ban đầu có đúng $4$ nghiệm khi và chỉ khi phương trình $x^3+x^2-(m^2+3)x-2=0$ có đúng $3$ nghiệm phân biệt khác $2$.\\
Xét $f(x)=x^3+x^2-(m^2+3)x-2$, ta có
\begin{itemize}
	\item $\lim\limits_{x\to -\infty}f(x)=-\infty$ nên $\exists a<-1\colon f(a)<0$.
	\item $f(-1)=m^2+1>0$.
	\item $f(0)=-2<0$.
	\item $f\left(m^2+3\right)=\left(m^2+3\right)^3-2>0$.
\end{itemize}
Hàm số $f(x)$ là hàm số đa thức nên liên tục trên tập xác định $\mathbb{R}$, do đó $f(x)$ cũng liên tục trên các đoạn $[a;-1]$, $[-1;0]$ và $\left[0;m^2+3\right]$.\\
Do $f(a) \cdot f(-1)<0$, $f(-1) \cdot f(0)<0$, $f(0) \cdot f\left(m^{2}+3\right)<0$ và $f(x)=0$ là phương trình bậc 3 nên có đúng 3 nghiệm thuộc các khoảng $\left(a ;-1\right)$, $(-1 ; 0)$, $\left(0 ; m^{2}+3\right)$.\\
Mặt khác $f(2)\ne 0\Leftrightarrow m\ne \pm \sqrt{2}\notin \mathbb{Z}$.\\
Vậy phương trình $x^4-x^3-(m^2+5)x^2+2(m^2+2)x+4=0$ luôn có đúng $4$ nghiệm phân biệt với mọi số nguyên $m$.
	}
\end{bt}

\begin{bt}[VD]%[DCHT Toán 11 - KNTT -Nguyễn Thành Nhân]%[1K5KG-7]
Với $m>2$, chứng minh rằng phương trình $x^3-2mx^2+2=0$ có ba nghiệm phân biệt. 
\loigiai{
Đặt $f(x)=x^3-2mx^2+2$ thì $f(x)$ là hàm đa thức nên liên tục trên $\mathbb{R}$.\\
Ta có \\
$f(-1)=1-2m<0$ do $m>2$, $f(0)=2>0$, $f(1)=3-2m<0$ do $m>2$. \\
Vì $f(-1)\cdot f(0)<0$ nên phương trình có ít nhất một nghiệm thuộc khoảng $(-1;0)$.\\
Vì $f(0)\cdot f(1)<0$ nên phương trình có ít nhất một nghiệm thuộc khoảng $(0;1)$.\\
Vì $\lim\limits_{x\to +\infty}f(x)=\lim\limits_{x\to +\infty}x^3\left(1-\dfrac{2m}{x}+\dfrac{2}{x^3}\right)=+\infty$ nên tồn tại một số $a>1$ để $f(a)>0$.\\
Vì $f(1)\cdot f(a)<0$ nên phương trình có ít nhất một nghiệm thuộc khoảng $(1;a)$.\\
Từ đó suy ra phương trình có ít nhất ba nghiệm phân biệt. Mặt khác $f(x)=0$ là phương trình bậc $3$ nên có không quá ba nghiệm.\\
Vậy phương trình $f(x)=0$ có đúng ba nghiệm phân biệt. 
}
\end{bt}
\begin{bt}[VD]%[DCHT Toán 11 - KNTT -Tên GV] %[1K5KG-7]
Chứng minh phương trình $\left(1-m\right)x^5+9mx^2-16x-m=0$ có ít nhất hai nghiệm thực phân biệt với mọi giá trị thực của tham số $m$.
\loigiai
{Đặt $f(x)=\left(1-m\right)x^5+9mx^2-16x-m$ thì $f(x)$ liên tục trên $\mathbb{R}$.\\ 
Ta biến đổi $f(x)=\left(-x^5+9x^2-1\right)m+x^5-16x$. \\
Ta tính giá trị của $f(x)$ tại các giá trị của $x$ thỏa $x^5-16x=0$, tức là tính tại $0$ và $\pm 2$.\\
Ta có $f(-2)=3m$; $f(0)=-m$; $f(2)=3m$.\\
Nếu $m=0$ thì phương trình tương đương $$x^5-16x=0\Leftrightarrow \hoac{&x=0\\&x=\pm2},$$ nên phương trình có nghiệm khi $m=0$.\\
Với $m\ne 0$ thì $f(-2)\cdot f(0)=-3m^2<0$; $f(0)\cdot f(2)=-3m^2<0$. \\
Theo tính chất của hàm số liên tục, trên mỗi khoảng $(-2;0)$ và $(0;2)$, phương trình có ít nhất một nghiệm.\\
Vậy phương trình đã cho có ít nhất hai nghiệm phân biệt.
}
\end{bt}
\begin{bt}[VD]%[DCHT Toán 11 - KNTT -Tên GV] %[1K5KG-7]
Cho $f(x)$ là hàm số liên tục trên đoạn $\left[a;b\right]$ sao cho với mọi $x\in \left[a;b\right]$ thì $a\le f(x)\le b$. Chứng minh rằng phương trình $f(x)=x$ có ít nhất một nghiệm thuộc đoạn $\left[a;b\right]$.
\loigiai
{Xét hàm số $h(x)=f(x)-x$, khi đó $h(x)$ liên tục trên đoạn $\left[a;b\right]$.\\
Vì $a\le f(x)\le b$ với mọi $x\in \left[a;b\right]$ nên ta có $$h(a)=f(a)-a\geq 0;\,\,h(b)=f(b)-b\le 0.$$
Suy ra $h(a)\cdot h(b)\le 0$. Do đó xảy ra một trong hai khả năng
\begin{itemize}
\item[•] Nếu $h(a)\cdot h(b)=0$ thì $\hoac{&h(a)=0\\&h(b)=0}$, do đó phương trình $h(x)=0$ có nghiệm $x=a$ hoặc $x=b$.\\
 Dẫn đến phương trình $f(x)=x$ có nghiệm $x=a$ hoặc $x=b$.
\item[•] Nếu $h(a)\cdot h(b)<0$ thì do tính liên tục của $h(x)$ nên phương trình $h(x)=0$ có nghiệm thuộc khoảng $(a;b)$.\\
 Dẫn đến phương trình $f(x)=x$ cũng có nghiệm thuộc khoảng $(a;b)$.
\end{itemize}
}
\end{bt}
\begin{bt}[VDC]%[DCHT Toán 11 - KNTT -Nguyễn Thành Nhân]%[1K5GG-7]
	Cho hai số $a$ và $b$ dương, $c\ne 0$ và $m$, $n$ là hai số thực tùy ý. Chứng minh phương trình $\dfrac{a}{x-m}+\dfrac{b}{x-n}=c$ luôn có nghiệm thực.
\loigiai{
Điều kiện xác định của phương trình $x\neq m$ và $x\neq n$.\\
Khi $m=n$, ta có $$\dfrac{a}{x-m}+\dfrac{b}{x-n}=c\Leftrightarrow\dfrac{a}{x-m}+\dfrac{b}{x-m}=c\Leftrightarrow a+b=c\left(x-m\right)\Leftrightarrow x=m+\dfrac{a+b}{c}.$$
Như vậy, khi $m=n$, phương trình đã cho luôn có nghiệm.\\
Khi $m\ne n$, ta có
$$\dfrac{a}{x-m}+\dfrac{b}{x-n}=c\Leftrightarrow a(x-n)+b(x-m)-c(x-m)(x-n)=0.$$
Xét hàm số $f(x)=a(x-n)+b(x-m)-c(x-m)(x-n)$.\\
Ta có hàm số liên tục trên $\mathbb{R}$.\\
Dễ thấy $f(m)=a(m-n)$, $f(n)=b(n-m)=-b(m-n)$.\\
Do đó $f(m)\cdot f(n)=-ab(m-n)^2< 0,\forall a > 0$, $b > 0$, $m\neq n$ .\\
Do đó phương trình đã cho có ít nhất một nghiệm thuộc $\left(m;,n\right)$ nếu $m < n$, hoặc ít nhất một nghiệm thuộc $\left(n;m\right)$ nếu $m > n$.\\
Vậy phương trình đã cho luôn có nghiệm thực.
}
\end{bt}
\begin{bt}[VDC]%[DCHT Toán 11 - KNTT -Tên GV] %[1K5GG-7]
Cho $a$, $b$, $c$ là ba số thực tùy ý. Chứng minh rằng phương trình
\[ab(x-a)(x-b)+bc(x-b)(x-c)+ca(x-c)(x-a)=0\] luôn có nghiệm.
\loigiai
{Đặt $f(x)=ab(x-a)(x-b)+bc(x-b)(x-c)+ca(x-c)(x-a)$ thì $f(x)$ liên tục trên $\mathbb{R}$.\\ 
Ta có 
\begin{eqnarray*}
f(a)\cdot f(b)\cdot f(c)\cdot f(0)&=&\left[bc(a-b)(a-c)+ca(b-c)(b-a)+ab(c-a)(c-b)\right]\\
&=&-a^2 b^2c^2\cdot (a-b)^2(b-c)^2(c-a)^2\cdot \left(a^2b^2+b^2c^2+c^2a^2\right)\\
&\le & 0.\quad (1)
\end{eqnarray*}
Xảy ra một trong hai khả năng
\begin{itemize}
\item[•] Nếu $f(a)\cdot f(b)\cdot f(c)\cdot f(0)=0$ thì phương trình $f(x)=0$ có ít nhất một nghiệm là một trong các số $a$, $b$, $c$, $0$.
\item[•] Nếu $f(a)\cdot f(b)\cdot f(c)\cdot f(0)<0$ thì từ $(1)$ suy ra $a$, $b$, $c$ khác $0$.\\
Vì tích bốn số nhỏ hơn $0$ nên phải tồn tại hai số trong các số $f(a)$, $ f(b)$, $f(c)$, $f(0)$ trái dấu. Dù là hai số nào cũng đều dẫn đến phương trình $f(x)=0$ có nghiệm.
\end{itemize}
}
\end{bt}

\begin{bt}[VDC]%[DCHT Toán 11 - KNTT -Tên GV] %[1K5GG-7]
Cho phương trình $x^3-3x^2+\left(2m-2\right)x+m-3=0$. Tìm tất cả các giá trị của tham số $m$ để phương trình có ba nghiệm phân biệt $x_1$, $x_2$, $x_3$ thỏa mãn $x_1<-1<x_2<x_3$.
\loigiai
{ Ta giải bài toán bằng điều kiện cần và điều kiện đủ.
\begin{itemize}
\item[•] \textit{Điều kiện cần}.\\
Đặt $f(x)=x^3-3x^2+\left(2m-2\right)x+m-3$ thì $f(x)$ liên tục trên $\mathbb{R}$.\\
Giả sử phương trình có ba nghiệm phân biệt $x_1$, $x_2$, $x_3$ thỏa mãn $x_1<-1<x_2<x_3$. \\
Ta có $f(x)=(x-x_1)\cdot (x-x_2)\cdot (x-x_3)$.\\
Từ giả thiết $x_1<-1<x_2$ suy ra 
\[f(-1)>0\Leftrightarrow -m-5>0\Rightarrow m<-5.\]
Đó là điều kiện cần của bài toán. Ta chứng minh đó cũng là điều kiện đủ.
\item[•] \textit{Điều kiện đủ}.\\
Với $m<-5$. Ta có $f(-1)=-m-5>0$.\\
Lai có $\lim \limits _{x\to -\infty}f(x)=-\infty$ nên tồn tại $a<-1$ để $f(a)<0$.\\
Ta có $f(0)=m-3<0$. \\
Lại có có $\lim \limits _{x\to +\infty}f(x)=+\infty$ nên tồn tại $b>0$ để $f(b)>0$.\\
Vì $f(a)\cdot f(-1)<0$; $f(-1)\cdot f(0)<0$; $f(0)\cdot f(b)<0$ nên theo tính chất của hàm số liên tục, phương trình $f(x)=0$ có các nghiệm $x_1\in (a;-1)$, $x_2\in (-1;0)$; $x_3\in (0;b)$.\\
Do đó phương trình $f(x)=0$ có ba nghiệm phân biệt $x_1$, $x_2$, $x_3$ thỏa mãn $x_1<-1<x_2<x_3$.
\end{itemize}
Vậy $m<-5$ là điều kiện cần và đủ của bài toán.
}
\end{bt}
\begin{bt}[VDC]%[DCHT Toán 11 - KNTT -Tên GV] %[1K5GG-7]
Cho $2a+6b+19c=0$. Chứng minh rằng phương trình $ax^2+bx+c=0$ có nghiệm $\break x_0\in \left[0;\dfrac{1}{3}\right]$.
\loigiai
{ Đặt $f(x)=ax^2+bx+c$ thì $f(x)$ liên tục trên $\mathbb{R}$.\\
Xét các giá trị sau $$f(0)=c;\,\,18\cdot f\left(\dfrac{1}{3}\right)=2a+6b+18c=(2a+6b+19c)-c=-c.$$
Do đó $18\cdot f(0)\cdot f\left(\dfrac{1}{3}\right)=-c^2\le 0$. Xét hai khả năng
\begin{itemize}
\item[•] Nếu $c=0$ thì $18\cdot f(0)\cdot f\left(\dfrac{1}{3}\right)\Leftrightarrow \hoac{&f(0)=0\\&f\left(\dfrac{1}{3}\right)=0}$, suy ra phương trình $f(x)=0$ có nghiệm $x=0$ hoặc $x=\dfrac{1}{3}$.
\item[•] Nếu $c\ne 0$ thì $18\cdot f(0)\cdot f\left(\dfrac{1}{3}\right)=-c^2<0$ nên phương trình luôn có nghiệm thuộc khoảng $\left(0;\dfrac{1}{3}\right)$.
\end{itemize} 
Vậy phương trình luôn có nghiệm thuộc đoạn $\left[0;\dfrac{1}{3}\right]$.
}
\end{bt}
\begin{bt}[VDC]%[DCHT Toán 11 - KNTT -Nguyễn Thành Nhân]%[1K5GG-7]
Cho phương trình $a\cos{2x}+b\cos{x}+c=0$, với $a$, $b$, $c$ là các số thực thỏa mãn $3b+7c=3a$. Chứng minh phương trình đã cho luôn có nghiệm.
\loigiai{Đặt $t=\cos{x}$, điều kiện $|t|\le 1$.\\
Khi đó, ta có $\cos 2x=2\cos^2x-1=2t^2-1$. Phương trình đã cho trở thành 
\[a(2t^2-1)+bt+c=0 \Leftrightarrow 2at^2+bt+(c-a)=0. \qquad (1)\]
Phương trình đã cho có nghiệm khi và chỉ khi phương trình $(1)$ có nghiệm $t \in [-1;1]$.\\
Xét $f(t)=2at^2+bt+(c-a)$, liên tục trên $\mathbb{R}$.\\
Ta có $f(0)=c-a$, và
\[f\left(\dfrac{2}{3}\right)=\dfrac{2}{3}\left(\dfrac{4}{3}a+b\right)+(c-a)=\dfrac{2}{3}\left(\dfrac{4a+3b}{3}\right)+(c-a).\]
Theo giả thiết, ta có 
\[3b+7c=3a \Rightarrow 3b=3a-7c \Rightarrow 4a+3b=7(a-c).\]
Do đó $f\left(\dfrac{2}{3}\right)=\dfrac{2}{3}\cdot\dfrac{7(a-c)}{3}+(c-a)=\dfrac{5}{9}(a-c)$.\\
Suy ra $f(0)f\left(\dfrac{2}{3}\right)=-\dfrac{5}{9}(c-a)^2 \le 0$.\\
Vì $f$ liên tục trên $\mathbb{R}$ nên $f$ liên tục trên $\left[0;\dfrac{2}{3}\right]$. Do đó tồn tại $t_0 \in \left[0;\dfrac{2}{3}\right]$ sao cho $f(t_0)=0$.\\
Suy ra phương trình $(1)$ có nghiệm thuộc $\left[0;\dfrac{2}{3}\right]$, cũng thuộc $[-1;1]$.\\
Vậy, phương trình $a\cos{2x}+b\cos{x}+c=0$ luôn có nghiệm.}
\end{bt}

\begin{bt}[VDC]%[DCHT Toán 11 - KNTT -Nguyễn Thành Nhân]%[1K5GG-7]
	Giả sử hai hàm số $y=f(x)$ và $y=f(x+1)$ đều liên tục trên đoạn $[0;2]$ và $f(0)=f(2)$. Chứng minh phương trình $f(x)-f(x+1)=0$ luôn có nghiệm thuộc đoạn $[0;1]$.
	\loigiai{
		Xét hàm số $g(x)=f(x)-f(x+1)$ trên đoạn $[0;1]$.\\
		Vì $y=f(x)$ và $y=f(x+1)$ đều liên tục trên đoạn $[0;2]$ nên hàm số $g(x)$ liên tục trên đoạn $[0;1]$.\\
		Ta có $\heva{&g(0)=f(0)-f(1)\\&g(1)=f(1)-f(2)=f(1)-f(0).}$\\
		Ta xét các trường hợp sau
		\begin{itemize}
			\item Nếu $f(0)-f(1)=0\Rightarrow \heva{&g(0)=0\\&g(1)=0}$. Suy ra phương trình $g(x)=0$ có nghiệm $x=0$, $x=1$. \quad $(1)$
			\item Nếu $f(0)-f(1)\ne 0$ thì $g(0)\cdot g(1)=[f(0)-f(1)]\cdot [f(1)-f(0)]<0$. Suy ra phương trình $g(x)=0$ luôn có ít nhất một nghiệm thuộc đoạn $[0;1]$. \quad $(2)$
		\end{itemize}
		Từ $(1)$ và $(2)$, suy ra phương trình $f(x)-f(x+1)=0$ luôn có nghiệm thuộc đoạn $[0;1]$.
	}
\end{bt}
\subsubsection{Bài tập trắc nghiệm}

\setcounter{ex}{0}
\Opensolutionfile{ans}[ans/ans-1K5-3-Dang6]
\begin{ex}%[DCHT Toán 11 - KNTT -Nguyễn Thành Nhân]%[1K5YG-7]
	Cho hàm số $y=f(x)$ liên tục trên đoạn $[a;b]$. Mệnh đề nào sau đây đúng?
	\choice
	{Nếu $f(a)\cdot f(b)>0$ thì phương trình $f(x)=0$ không có nghiệm thuộc $(a ; b)$}
	{\True Nếu $f(a)\cdot f(b)<0$ thì phương trình $f(x)=0$ có nghiệm thuộc $(a ; b)$}
	{Nếu $f(a)\cdot f(b)<0$ thì phương trình $f(x)=0$ không có nghiệm thuộc $(a ; b)$}
	{Nếu $f(a)\cdot f(b)>0$ thì phương trình $f(x)=0$ có nghiệm thuộc $(a ; b)$}
	\loigiai{
		"Nếu $f(a)\cdot f(b)<0$ thì phương trình $f(x)=0$ có nghiệm thuộc $(a ; b)$" là mệnh đề đúng.	
	} 
\end{ex}

\begin{ex}%[DCHT Toán 11 - KNTT -Nguyễn Thành Nhân]%[1K5YG-7]
	Phương trình $x^5+2x^3+16=0$ có nghiệm thuộc khoảng nào sau đây?
	\choice
	{$(0;1)$}
	{$(-10;-2)$}
	{$(-1;0)$}
	{\True $(-2;-1)$}
	\loigiai{
		Hàm số $x^5+2 x^3+16=0$ liên tục trên $\mathbb{R}$.\\
		Do $f(-2)\cdot f(-1)=-32\cdot 13 <0$ nên phương trình $x^5+2 x^3+16=0$ có nghiệm thuộc khoảng $(-2;-1)$. 	
	} 
\end{ex}

\begin{ex}%[DCHT Toán 11 - KNTT -Nguyễn Thành Nhân]%[1K5YG-7]
	Phương trình $ 3x^5+5x^3+10=0 $ có nghiệm thuộc khoảng nào sau đây?
	\choice
	{$ (0;1) $}
	{$ (-1;0) $}
	{$ (-10;-2) $}
	{\True $ (-2;-1) $}
	\loigiai{
	Đặt $ f(x)=3x^5+5x^3+10 $.\\
	Tập xác định của hàm số là $ \mathscr{D}=\mathbb{R} \Rightarrow f(x)$	liên tục trên $ \mathbb{R} $.\\
	Suy ra $ f(x) $ liên tục trên $ [-2;-1] $.\\
	Ta có $ f(-2)=-126 $; $ f(-1)=2\Rightarrow f(-2)\cdot f(-10)=-252<0 $.\\
	Phương trình $ 3x^5+5x^3+10=0 $ có nghiệm thuộc khoảng $ (-2;-1) $.
	}
\end{ex}

\begin{ex}%[DCHT Toán 11 - KNTT -Nguyễn Thành Nhân]%[1K5BG-7]
Cho hàm số $f(x)$ liên tục trên đoạn $[-1 ; 4]$, biết $f(-1)=2$, $f(4)=7$. Có thể nói gì về số nghiệm của phương trình $f(x)=5$ trên đoạn $[-1 ; 4]$.	
	\choice
	{Có hai nghiệm phân biệt }
	{Có đúng một nghiệm}
	{\True Có ít nhất một nghiệm}
	{Vô nghiệm}
	\loigiai{
Xét hàm số $g(x)=f(x)-5$ liên tục trên 	đoạn $[-1 ; 4]$ có $g(-1)=f(-1)-5=2-5=-3<0$ và $g(4)=f(4)-5=7-5=2>0$.\\
Suy ra $g(-1)\cdot g(4)<0$.\\
Do đó phương trình $g(x)=0$ có ít nhất một nghiệm thuộc $(-1;4)$ hay phương trình $f(x)=5$ trên đoạn $[-1 ; 4]$ 	Có ít nhất một nghiệm.
	}
\end{ex}

\begin{ex}%[DCHT Toán 11 - KNTT -Nguyễn Thành Nhân]%[1K5BG-7]
Cho hàm số $y=f(x)$ liên tục trên $[2022;2023]$, biết rằng $f(2022)=-2023$, $f(2023)=2022$. Kết luận nào sau đây chắc chắn đúng?
\choice
{\True Phương trình $f(x)=0$ có nghiệm}
{Phương trình $f(x)=0$ vô nghiệm}
{Phương trình $f(x)=0$ có $3$ nghiệm}
{Phương trình $f(x)=0$ có $2$ nghiệm}
\loigiai{Ta có $f(x)$ liên tục trên $[2022;2023]$ và $f(2022)\cdot f(2023)<0$ nên $\exists x_0\in(2022;2023)\colon f(x_0)=0$.\\
Vậy mệnh đề chắc chắc đúng là ``Phương trình $f(x)=0$ có nghiệm''.}
\end{ex}

\begin{ex}%[DCHT Toán 11 - KNTT -Nguyễn Thành Nhân]%[1K5BG-7]
	Cho các câu
	\begin{enumerate}
	\item Nếu hàm số $y=f(x)$ liên tục trên $(a;b)$ và $f(a) \cdot f(b)<0$ thì tồn tại $x_0 \in (a;b)$ sao cho $f\left(x_0 \right) = 0$.
	\item Nếu $y=f(x)$ liên tục trên $[a;b]$ và $f(a)\cdot f(b) <0$ thì phương trình $f(x) = 0$ có nghiệm thuộc khoảng $(a;b)$.
	\item Nếu hàm số $y=f(x)$ liên tục, đơn điệu trên $[a;b]$ và $f(a) \cdot f(b) <0$ thì phương trình $f\left( x \right)=0$ có nghiệm duy nhất thuộc $(a;b)$.
	\end{enumerate}
	Trong ba câu trên
\choice
{\True có đúng một câu {\bf sai}}
{cả ba câu đều đúng}
{có đúng hai câu {\bf sai}}
{cả ba câu đều {\bf sai}}
\loigiai{
	Trong ba câu trên có hai câu đúng
	\begin{itemize}
	\item Nếu $y=f(x)$ liên tục trên $[a;b]$ và $f(a)\cdot f(b) <0$ thì phương trình $f(x) = 0$ có nghiệm thuộc khoảng $(a;b)$.
	\item Nếu hàm số $y=f(x)$ liên tục, đơn điệu trên $[a;b]$ và $f(a) \cdot f(b) <0$ thì phương trình $f\left( x \right)=0$ có nghiệm duy nhất thuộc $(a;b)$.
	\end{itemize}
	và một câu sai
	\begin{itemize}
	\item Nếu hàm số $y=f(x)$ liên tục trên $(a;b)$ và $f(a) \cdot f(b)<0$ thì tồn tại $x_0 \in (a;b)$ sao cho $f\left(x_0 \right) = 0$.
	\end{itemize}
}
\end{ex}

\begin{ex}%[DCHT Toán 11 - KNTT -Nguyễn Thành Nhân]%[1K5BG-7]
	Cho hàm số $f(x)$ xác định trên $[a;b]$. Trong các mệnh đề sau, mệnh đề nào đúng?
	\choice
	{Nếu phương trình $f(x)=0$ có nghiệm trong khoảng $(a;b)$ thì hàm số $f(x)$ phải liên tục trên $(a;b)$}
	{Nếu hàm số $f(x)$ liên tục trên $[a;b]$ và $f(a)\cdot f(b) >0$ thì phương trình $f(x)=0$ không có nghiệm trong khoảng $(a;b)$}
	{Nếu $f(a) \cdot f(b)<0$ thì phương trình $f(x)=0$ có ít nhất một nghiệm trong khoảng $(a;b)$}
	{\True Nếu hàm số $f(x)$ liên tục, tăng trên $[a;b]$ và $f(a) \cdot f(b)>0$ thì phương trình $f(x) =0$ không có nghiệm trong khoảng $(a;b)$}
	\loigiai{
	Mệnh đề đúng là ``Nếu hàm số $f(x)$ liên tục, tăng trên $[a;b]$ và $f(a) \cdot f(b)>0$ thì phương trình $f(x) =0$ không có nghiệm trong khoảng $(a;b)$.''
	}
\end{ex}

\begin{ex}%[DCHT Toán 11 - KNTT -Nguyễn Thành Nhân]%[1K5BG-7]
Chứng minh rằng phương trình $x^3-x+3=0$ có ít nhất một nghiệm. Một bạn học sinh trình bày lời giải như sau:
\begin{enumerate}[Bước 1.]
\item Xét hàm số $y=f(x)=x^3-3x+3$ liên tục trên $\Bbb{R}$.
\item Ta có $f(0)=3$ và $f(-2)=-3$.
\item Suy ra $f(0)\cdot f(-2)>0$.
\item Vậy phương trình đã cho có ít nhất một nghiệm. Hãy tìm bước giải {\bf sai} của bạn học sinh trên?
\end{enumerate}
	\choice
	{\True Bước $3$}
	{Bước $4$ và Bước $3$}
	{Bước $1$}
	{Bước $1$ và Bước $3$}
	\loigiai{
Học sinh sai ở Bước $3$ vì $f(0)\cdot f(-2)<0$.
	}
\end{ex}


\begin{ex}%[DCHT Toán 11 - KNTT -Nguyễn Thành Nhân]%[1K5BG-7]
	Cho hàm số $f(x)=3x^4+3x-2$. Khẳng định nào sau đây là {\bf sai}?
	\choice
	{\True Phương trình $f(x)=0$ vô nghiệm}
	{Phương trình $f(x)=0$ có nghiệm trong khoảng $(0;1)$}
	{Phương trình $f(x)=0$ có 1 it nhất một nghiệm trong khoảng $(-1;1)$}
	{Phương trình $f(x)=0$ có nghiệm trên $\mathbb{R}$}
	\loigiai{
		Hàm số $f(x)=3x^4+3x-2$ liên tục trên $\mathscr{D}=\mathbb{R}$.\\
		Ta có $\heva{&f(0)=-2\\&f(1)=4}\Rightarrow f(0)\cdot f(1)=-8<0$ nên phương trình $f(x)=0$ có ít nhất $1$ nghiệm thuộc khoảng $(0;1)$.\\
		Suy ra phương trình $f(x)=0$ có ít nhất $1$ nghiệm thuộc khoảng $(-1;1)$ cũng như có nghiệm trên $\mathbb{R}$.\\
		Vậy khẳng định {\bf sai} là \lq\lq Phương trình $f(x)=0$ vô nghiệm \rq\rq.
	}
\end{ex}

\begin{ex}%[DCHT Toán 11 - KNTT -Nguyễn Thành Nhân]%[1K5BG-7]
	Cho hàm số $y=f(x)$ liên tục trên đoạn $[1;5]$ và $f(1)=2$, $f(5)=10$. Khẳng định nào sau đây \textbf{đúng}?
	\choice
	{Phương trình $f(x)=6$ vô nghiệm}
	{\True Phương trình $f(x)=7$ có ít nhất một nghiệm trên khoảng $(1;5)$}
	{Phương trình $f(x)=2$ có hai nghiệm $x=1$, $x=5$}
	{Phương trình $f(x)=7$ vô nghiệm}
	\loigiai{
	\immini{Đồ thị hàm số $f(x)$ là một đường liền nét trên khoảng $(1;5)$.\\
	Đường thẳng $y=7$ song song với trục $Ox$ và sẽ cắt đồ thị hàm $y=f(x)$ tại ít nhất một điểm có hoành độ thuộc khoảng $(1;5)$. Do đó phương trình $f(x)=7$ có ít nhất một nghiệm thuộc khoảng $(1;5)$.\\
	Ta có hình vẽ minh hoạ như hình vẽ bên.}{\begin{tikzpicture}[scale=0.3, font=\footnotesize, line join=round, line cap=round, >=stealth]
	\tikzset{label style/.style={font=\footnotesize}}
	%%Nhập giới hạn đồ thị và hàm số cần vẽ
	\def \xmin{-1}
	\def \xmax{6}
	\def \ymin{-1}
	\def \ymax{11}
	%\def \hamso{sin(\x r)}
	%\def \tiemcanxien{\x+1}
	%%Tự động
	\draw[->] (\xmin,0)--(\xmax,0) node[below right] {$x$};
	\draw[->] (0,\ymin)--(0,\ymax) node[above] {$y$};
	\draw (0,0) node [below left] {$O$};
	%%Vẽ các điểm trên 2 hệ trục
	\foreach \x in {1,5}
	\draw[thin] (\x,1pt)--(\x,-1pt) node [below] {$\x$};
	\foreach \y in {2,10}
	\draw[thin] (1pt,\y)--(-1pt,\y) node [left] {$\y$};
	\draw (0,7) node[above left]{$y=7$};
	%%Vẽ thêm mấy cái râu ria
	\draw[name path=duongthang](-1,7)--(5.5,7);
	\draw[name path=duongcong1] (1,2) .. controls (2,-1) and (2.5,15) .. (3,7);
	\draw[name path=duongcong2] (3,7) .. controls (3.5,-3) and (4.5,9) .. (5,10);
	\clip (\xmin+0.01,\ymin+0.01) rectangle (\xmax-0.01,\ymax-0.01);
	\draw[dashed,thin](1,0)--(1,2)--(0,2);
	\draw[dashed,thin](5,0)--(5,10)--(0,10);
	\end{tikzpicture}}}
\end{ex}

\begin{ex}%[DCHT Toán 11 - KNTT -Nguyễn Thành Nhân]%[1K5BG-7]
	Cho phương trình $882x^5-441x^4-116x^3+58x^2+2x-1=0$. Mệnh đề nào sau đây \textbf{sai}?
	\choice
	{Phương trình có nghiệm trong khoảng $(0;1)$}
	{Phương trình có nghiệm trong khoảng $(-1;0)$}
	{Phương trình có $5$ nghiệm phân biệt}
	{\True Phương trình có đúng $4$ nghiệm}
	\loigiai{Hàm số $f(x)=882x^5-441x^4-116x^3+58x^2+2x-1$ liên tục trên $\mathbb{R}$ nên nó liên tục trên $\left(-1;1\right)$.
	\begin{itemize}
	\item $\heva{&f(-0{,}4)=-5{,}417\\&f(-0{,}3)=1{,}0366}\Rightarrow f(-0{,}4)\cdot f(-0{,}3)<0$ nên phương trình có ít nhất một nghiệm trên khoảng $\left(-0{,}4; -0{,}3\right)$.
	\item $\heva{&f(-0{,}2)=-0{,}8601\\&f(-0{,}1)=-0{,}556}\Rightarrow f(-0{,}2)\cdot f(-0{,}1)<0$ nên phương trình có ít nhất một nghiệm trên khoảng $\left(-0{,}2; -0{,}1\right)$.
	\item $\heva{&f(0{,}1)=-0{,}371\\&f(0{,}2)=0{,}3686}\Rightarrow f(0{,}1)\cdot f(0{,}2)<0$ nên phương trình có ít nhất một nghiệm trên khoảng $\left(0{,}1; 0{,}2\right)$.
	\item $\heva{&f(0{,}3)=0{,}2591\\&f(0{,}4)=-0{,}601}\Rightarrow f(0{,}3)\cdot f(0{,}4)<0$ nên phương trình có ít nhất một nghiệm trên khoảng $\left(0{,}3; 0{,}4\right)$.
	\item $\heva{&f(0{,}4)=-0{,}601\\&f(0{,}6)=7{,}4547}\Rightarrow f(0{,}4)\cdot f(0{,}6)<0$ nên phương trình có ít nhất một nghiệm trên khoảng $\left(0{,}4; 0{,}6\right)$.
	\end{itemize}
	Phương trình đã cho là phương trình bậc $5$ nên nó có đúng $5$ nghiệm.	
	}
\end{ex}



\begin{ex}%[DCHT Toán 11 - KNTT -Nguyễn Thành Nhân]%[1K5BG-7]
Phương trình $x^3-3x^2+5x+1=0$ có ít nhất một nghiệm thuộc khoảng nào sau đây?
	\choice
	{$(0;1)$}
	{$(2;3)$}
	{$(-2;-1)$}
	{\True $(-1;0)$}
	\loigiai{Xét hàm số $f(x)=x^3-3x^2+5x+1$ là hàm đa thức nên liên tục trên $\mathbb{R}$.\\
	Ta có $f(x)$ liên tục trên $[-1;0]$ và $f(-1)\cdot f(0)=(-8)\cdot1=-8<0$.\\
Từ đó ta suy ra tồn tại $x_0\in(-1;0)$ sao cho $f\left(x_0\right)=0$, hay phương trình $f(x)=0$ có ít nhất một nghiệm thuộc khoảng $(-1;0)$.}
\end{ex}

\begin{ex}%[DCHT Toán 11 - KNTT -Nguyễn Thành Nhân]%[1K5BG-7]
Hàm số $f(x)$ liên tục trên đoạn $[2 ; 4]$ và $f(2) \cdot f(4)<0$. Khẳng định nào sau đây đúng?
\choice
{\True Phương trình $f(x)=0$ có nghiệm}
{Phương trình $f(x)=0$ vô số nghiệm}
{Phương trình $f(x)=0$ có ít nhất $6$ nghiệm}
{Phương trình $f(x)=0$ vô nghiệm}
\loigiai{
Hàm số $f(x)$ liên tục trên đoạn $[2 ; 4]$ và $f(2) \cdot f(4)<0$ nên phương trình $f(x)=0$ có ít nhất một nghiệm thuộc $(2;4)$.
}
\end{ex}

\begin{ex}%[DCHT Toán 11 - KNTT -Nguyễn Thành Nhân]%[1K5BG-7]
Cho phương trình $2x^3-10x-7=0\quad (1)$. Mệnh đề nào sau đây là mệnh đề \textbf{sai}?
\choice
{\True Phương trình $(1)$ không có nghiệm trên khoảng $(0;+\infty)$}
{Hàm số $f(x)=2x^3-10x-7$ liên tục trên $\mathbb{R}$}
{Phương trình $(1)$ có ít nhất hai nghiệm trên khoảng $(-1;3)$}
{Phương trình $(1)$ có ít nhất một nghiệm trên khoảng $(0;3)$}
\loigiai
{
Hàm số $f(x)=2x^3-10x-7$ là hàm số đa thức nên nó liên tục trên $\mathbb{R}$. Vì thế, nó cũng liên tục trên mỗi đoạn $[0;3]$, $[-1;3]$.\\
Vì $f(-1)=1$ và $f(0)=-7$ nên $f(-1)\cdot f(0)<0$. Vì thế, phương trình $2x^3-10-7=0$ có ít nhất một nghiệm thuộc khoảng $(-1;0)$.\\
Vì $f(0)=-7$ và $f(3)=17$ nên $f(0)\cdot f(3)<0$. Vì thế, phương trình $2x^3-10-7=0$ có ít nhất một nghiệm thuộc khoảng $(0;3)$.\\
Do đó, phương trình $2x^3-10x-7=0$ có ít nhất hai nghiệm thuộc khoảng $(-1;3)$.
}
\end{ex}

\begin{ex}%[DCHT Toán 11 - KNTT -Nguyễn Thành Nhân]%[1K5BG-7]
Phương trình $2x^3-6x+1=0$ có bao nhiêu nghiệm phân biệt thuộc $(-2;2)$?
\choice 
{\True $3$}
{$1$}
{$0$}
{$2$}
\loigiai{
Xét hàm số $f(x)=2x^3-6x+1$ liên tục trên đoạn $[-2;2]$.\\
Ta có
\[f(-2)=-3,\quad f(0)=1,\quad f(1)=-3,\quad f(2)=5.\]
Suy ra $f(-2)f(0)<0$, $f(0)f(1)<0$, $f(0)f(2)<0$.\\ 
Từ đó suy ra phương trình đã cho có ít nhất $3$ nghiệm phân biệt thuộc khoảng $(-2;2)$. Mặt khác, do phương trình đã cho là phương trình bậc ba nên nó có đúng $3$ nghiệm phân biệt thuộc khoảng $(-2;2)$.
}
\end{ex}


\begin{ex}%[DCHT Toán 11 - KNTT -Nguyễn Thành Nhân]%[1K5BG-7]
	Cho hàm số $f(x)=-4x^3+4 x-1$. Mệnh đề nào sau đây \textbf{sai}?
	\choice
	{Hàm số đã cho liên tục trên $\mathbb{R}$}
	{\True Phương trình $f(x)=0$ không có nghiệm trên khoảng $(-\infty ; 1)$}
	{Phương trình $f(x)=0$ có nghiệm trên khoảng $(-2 ; 0)$}
	{Phương trình $f(x)=0$ có ít nhất hai nghiệm trên khoảng $\left(-3 ; \dfrac{1}{2}\right)$}
	\loigiai{
	\begin{itemize}
		\item Hàm $f$ là hàm đa thức nên liên tục trên $\mathbb{R}$. Nên mệnh đề \lq\lq Hàm số đã cho liên tục trên $\mathbb{R}$\rq\rq\; đúng.
		\item Ta có $\heva{&f(-1)=-1<0 \\& f(-2)=23>0} \Rightarrow f(x)=0$ có nghiệm $x_1$ trên $(-2 ; -1)$, mà
		$(-2 ;-1) \subset(-2 ; 0) \subset(-\infty ; 1)$.\\
		Nên mệnh đề \lq\lq Phương trình $f(x)=0$ không có nghiệm trên khoảng $(-\infty ; 1)$\rq\rq\; sai và mệnh đề \lq\lq Phương trình $f(x)=0$ có nghiệm trên khoảng $(-2 ; 0)$\rq\rq\; đúng.
		\item Ta có $\heva{&f(-1)=-1<0 \\& f(-2)=23>0} \Rightarrow f(x)=0$ có nghiệm $x_1$ trên $(-2 ; -1)$, và
		$\heva{&f(-1)=-1 \\& f\left(\dfrac{1}{2}\right)=\dfrac{1}{2}} \Rightarrow f(-1) \cdot f\left(\dfrac{1}{2}\right)<0$\\
		$\Rightarrow f(x)=0 \text { có nghiệm } x_2 \text { trên }\left(-1 ; \dfrac{1}{2}\right)$.\\
		Nên mệnh đề \lq\lq Phương trình $f(x)=0$ có ít nhất hai nghiệm trên khoảng $\left(-3 ; \dfrac{1}{2}\right)$\rq\rq\; đúng.
	\end{itemize}
	}
\end{ex}

\begin{ex}%[DCHT Toán 11 - KNTT -Nguyễn Thành Nhân]%[1K5BG-7]
    Tìm khẳng định \textbf{sai} trong các khẳng định sau.
    \choice
    {\True Hàm số $f(x)=\dfrac{1}{x}$ có $f(-1)\cdot f(1)<0$ nên phương trình $f(x)=0$ có nghiệm trong $(-1;1)$}
    {Phương trình $-x^3-2x^2-3x+4=0$ có ít nhất một nghiệm}
    {Hàm số $f(x)=-x^3-2x^2-3x+4$ liên tục trên $\mathbb{R}$}
    {Hàm số $f(x)=-x^3-2x^2-3x+4$ có $f(0)\cdot f(1)<0$ nên phương trình $f(x)=0$ có ít nhất một nghiệm trong khoảng $(0;1)$}
    \loigiai{
        Hàm số $f(x)=\dfrac{1}{x}$ không xác định tại $x=0$ nên không liên tục trên $[0;1]$, do đó khẳng định \lq\lq$f(-1)\cdot f(1)<0$ nên phương trình $f(x)=0$ có nghiệm trong $(-1;1)$\rq\rq\, là sai.\\
        Hàm số $f(x)=-x^3-2x^2-3x+4$ liên tục trên $\mathbb{R}$ nên liên tục trên $[0;1]$, có $f(0)\cdot f(1)=-8<0$ nên phương trình $f(x)=0$ có ít nhất một nghiệm trên $(0;1)$, suy ra phương trình $f(x)=0$ có ít nhất một nghiệm.}
\end{ex}
\begin{ex}%[DCHT Toán 11 - KNTT -Nguyễn Thành Nhân]%[1K5BG-7]
Cho phương trình $2x^3+x^2-1=0$ $(1)$. Mệnh đề nào dưới đây đúng?
\choice
{Phương trình $(1)$ vô nghiệm trên khoảng $(0 ; 2)$}
{\True Phương trình $(1)$ có ít nhất một nghiệm trên khoảng $(0 ; 2)$}
{Phương trình $(1)$ có đúng 4 nghiệm trên khoảng $(0 ; 2)$}
{Phương trình $(1)$ có vô số nghiệm trên khoảng $(0 ; 2)$}
\loigiai{
Đặt $f(x)=2x^3+x^2-1=0$ là hàm liên tục trên $[0;2]$.\\
Có $f(0)=-1$, $f(2)=19$ nên $f(0)\cdot f(2)<0$, vậy phương trình  $(1)$ có ít nhất một nghiệm trên khoảng $(0 ; 2)$.
}
\end{ex}

\begin{ex}%[DCHT Toán 11 - KNTT -Nguyễn Thành Nhân]%[1K5KG-7]
	Xét phương trình sau trên tập số thực $x^3+x=a \quad$(1). Chọn khẳng định đúng trong các khẳng định dưới đây?
	\choice
	{Phương trình $(1)$ chỉ có nghiệm khi $x>a$}
	{Phương trình $(1)$ vô nghiệm khi $x \geq a$}
	{Phương trình $(1)$ chỉ có nghiệm khi $x \geq a$}
	{\True Phương trình $(1)$ có nghiệm $\forall a \in \mathbb{R}$}
	\loigiai
	{
		Xét $f(x)=x^3+x-a$ liên tục trên $\mathbb{R}$.\\
		Với mọi $a$ ta có $\lim \limits_{x \to -\infty} f(x)=-\infty$ và $\lim \limits_{x \to +\infty} f(x)=+\infty$.\\
		Do đó tồn tại các số $m$, $n$ sao cho $f(m)>0$ và $f(n)<0$, tức là $f(m)\cdot f(n)<0$.\\
		Vậy $f(x)$ luôn có nghiệm với mọi $a\in\mathbb{R}$.
	}
\end{ex}

\begin{ex}%[DCHT Toán 11 - KNTT -Nguyễn Thành Nhân]%[1K5KG-7]
	Cho phương trình $-4x^3+4x-1=0$. Tìm khẳng định \textbf{sai} trong các khẳng định sau.
	\choice
	{Phương trình đã cho có ba nghiệm phân biệt}
	{\True Phương trình đã cho chỉ có một nghiệm trong khoảng $(0;1)$}
	{Phương trình đã cho có ít nhất một nghiệm trong khoảng $(-2;0)$}
	{Phương trình đã cho có ít nhất một nghiệm trong khoảng $\left(-\dfrac{1}{2};\dfrac{1}{2}\right)$}
	\loigiai{
		Đặt $f(x)=-4x^3+4x-1$, $f(x)$ là hàm đa thức nên xác định và liên tục trên $\mathbb{R}$.\\
		Ta có $f(-2)=23$, $f(0)=-1$, $f\left(\dfrac{1}{2}\right)=\dfrac{1}{2}$, $f(1)=-1$, $f\left(-\dfrac{1}{2}\right)=-\dfrac{5}{2}$.\\
		Suy ra $\heva{&f(-2)\cdot f(0)<0\\&f(0)\cdot f\left(\dfrac{1}{2}\right)<0\\&f\left(\dfrac{1}{2}\right)\cdot f(1)<0.}$\\
		Do đó phương trình $f(x)=0$ có ít nhất một nghiệm trong khoảng $(-2;0)$, có ít nhất một nghiệm trong khoảng $\left(0;\dfrac{1}{2}\right)$, có ít nhất một nghiệm trong khoảng $\left(\dfrac{1}{2};1\right)$.\\
		Khi đó phương trình $f(x)=0$ có ít nhất hai nghiệm trong khoảng $(0;1)$ và phương trình $f(x)=0$ có $3$ nghiệm phân biệt.\\
		Mặt khác $f\left(-\dfrac{1}{2}\right)\cdot f\left(\dfrac{1}{2}\right)<0$ nên phương trình $f(x)=0$ có ít nhất một nghiệm trong khoảng $\left(-\dfrac{1}{2};\dfrac{1}{2}\right)$.\\
		Vậy khẳng định ``Phương trình đã cho chỉ có một nghiệm trong khoảng $(0;1)$'' là sai.
	}
\end{ex}


\begin{ex}%[DCHT Toán 11 - KNTT -Nguyễn Thành Nhân]%[1K5KG-7]
	Phương trình nào sau đây có nghiệm trong khoảng $(0;1)$?
	\choice
	{$(x-1)^5-x^9-2=0$}
	{$3x^4-4x^2+5=0$}
	{\True $3x^{2019}-8x+4=0$}
	{$2x^2-3x+4=0$}
	\loigiai{
		\begin{itemize}
			\item Hàm số $f(x)=3x^{2019}-8x+4$ liên tục trên đoạn $[0;1]$ và có $f(0)\cdot f(1)=4\cdot(-1)<0$ nên phương trình $f(x)=0$ có nghiệm trên khoảng $(0;1)$.
			\item Hàm số $g(x)=3x^4-4x^2+5=3\left(x^2-\dfrac{2}{3}\right)^2+\dfrac{11}{3} \geq \dfrac{11}{3}>0,\forall x$ nên phương trình $g(x)=0$ vô nghiệm.
			\item Hàm số $h(x)=2x^2-3x+4=2\left(x-\dfrac{3}{4}\right)^2+\dfrac{23}{8}>0,\forall x$ nên phương trình $h(x)=0$ vô nghiệm.
			\item Hàm số $p(x)=(x-1)^5-x^9-2$ trên $(0;1)$ có $-1<(x-1)^5<0 ;-1<-x^9<0$
			$ \Rightarrow p(x)<0,\forall x \in (0;1)$ nên trên $(0;1)$ phương trình $p(x)=0$ vô nghiệm.
		\end{itemize}		
	}
\end{ex}
\begin{ex}%[DCHT Toán 11 - KNTT -Nguyễn Thành Nhân]%[1K5KG-7]
	Cho phương trình $\left(3m^2-m-2\right)x^{2024}\cdot \left(x^{2023}+1\right)+2 x-1=0$. Tìm tất cả các giá trị của $m$ để phương trình có nghiệm.
	\choice
	{$m\in\mathbb{R}\setminus\left\{1;-\dfrac{2}{3}\right\}$}
	{\True $\forall m\in\mathbb{R}$}
	{$m=1; m=-\dfrac{2}{3}$}
	{$\hoac{&m<0\\& m>1}$}
	\loigiai{
		Xét hàm số $y=\left(3m^2-m-2\right)x^{2024}\cdot \left(x^{2023}+1\right)+2 x-1$.
		\begin{itemize}
			\item \textbf{Trường hợp $1$:} $3m^2-m-2>0$, ta có $\lim\limits_{x\rightarrow +\infty} y=+\infty$, $\lim\limits_{x\rightarrow -\infty} y=-\infty$ suy ra phương trình $y=0$ có nghiệm.
			\item \textbf{Trường hợp $2$:} $3m^2-m-2<0$, ta có $\lim\limits_{x\rightarrow +\infty} y=-\infty$, $\lim\limits_{x\rightarrow -\infty} y=+\infty$ suy ra phương trình $y=0$ có nghiệm.
			\item \textbf{Trường hợp $3$}: $3m^2-m-2=0$, khi đó phương trình đã cho có nghiệm $x=\dfrac{1}{2}$.
		\end{itemize}
		Vậy phương trình đã cho luôn có nghiệm $\forall m\in\mathbb{R}$.
	}
\end{ex}
\begin{ex}%[DCHT Toán 11 - KNTT -Nguyễn Thành Nhân]%[1K5KG-7]
Cho phương trình $x^4-3x^3+x-\dfrac{1}{8}=0\quad(1)$. Chọn khẳng định đúng.
\choice
{Phương trình $(1)$ có đúng ba nghiệm trên khoảng $(-1;3)$}
{\True Phương trình $(1)$ có đúng bốn nghiệm trên khoảng $(-1;3)$}
{Phương trình $(1)$ có đúng hai nghiệm trên khoảng $(-1;3)$}
{Phương trình $(1)$ có đúng một nghiệm trên khoảng $(-1;3)$}
\loigiai{
Hàm số $f(x)=x^4-3x^3+x-\dfrac{1}{8}$ liên tục trên đoạn $[-1;3]$ và có $f(-1)=\dfrac{23}{8}$;$f\left(-\dfrac{1}{2}\right)=-\dfrac{3}{16}$; $f\left(\dfrac{1}{2}\right)=\dfrac{1}{16}$;  $f\left(1\right)=-\dfrac{9}{8}$; $f\left(3\right)=\dfrac{23}{8}$.\\
Suy ra trên mỗi khoảng $\left(-1;-\dfrac{1}{2}\right)$; $\left(-\dfrac{1}{2};\dfrac{1}{2}\right)$; $\left(\dfrac{1}{2};1\right)$ ; $\left(1;3\right)$ phương trình có ít nhất một nghiệm.\\ Mặt khác phương trình là bậc $4$ nên có không quá $4$ nghiệm.\\
Vậy phương trình $(1)$ có đúng bốn nghiệm trên khoảng $(-1;3)$.
}
\end{ex}
\begin{ex}%[DCHT Toán 11 - KNTT -Nguyễn Thành Nhân]%[1K5KG-7]
Cho phương trình $x^3+ax^2+bx+c=0$\qquad $(1)$ trong đó $a$, $b$, $c$ là các tham số thực. Chọn khẳng định đúng trong các khẳng định sau.
\choice
{\True Phương trình $(1)$ có ít nhất một nghiệm với mọi $a$, $b$, $c$}
{Phương trình $(1)$ có ít nhất hai nghiệm với mọi $a$, $b$, $c$}
{Phương trình $(1)$ có ít nhất ba nghiệm với mọi $a$, $b$, $c$}
{Phương trình $(1)$ vô nghiệm với mọi $a$, $b$, $c$}
\loigiai{
Xét hàm số $f(x)=x^3+ax^2+bx+c$ liên tục trên $\mathbb{R}$.\\
Ta có $\lim\limits_{x\to -\infty} f(x)=-\infty$ và $\lim\limits_{x\to +\infty} f(x)=+\infty$.\\
Do đó phương trinh $f(x)=0$ có ít nhất một nghiệm với mọi $a$, $b$, $c$.
}
\end{ex}

\begin{ex}%[DCHT Toán 11 - KNTT -Nguyễn Thành Nhân]%[1K5KG-7]
    Khẳng định nào trong các khẳng định sau đây là \textbf{sai}?
    \choice
    {Phương trình $x^4+mx^2-2mx-2=0$ luôn có nghiệm với mọi giá trị của tham số $m$}
    {\True Phương trình $3x^6-3x^3+5x-2=0$ \textbf{không} có nghiệm thuộc khoảng $(-2;2)$}
    {Phương trình $x^3-3x+1=0$ có ba nghiệm phân biệt}
    {Phương trình $m(x-1)^2(x-2)+2x-3=0$ luôn có nghiệm với mọi giá trị của tham số $m$}
    \loigiai{
        \begin{itemize}
            \item Hàm số $f(x)=x^4+mx^2-2mx-2$ liên tục trên đoạn $[0;2]$, có $f(0)\cdot f(2) = -2\cdot 14 = -28<0$ nên phương trình $f(x)=0$ có ít nhất một nghiệm thuộc khoảng $(0;2)$ với mọi $m$ hay phương trình $x^4+mx^2-2mx-2=0$ luôn có nghiệm với mọi giá trị của tham số $m$.
            \item Hàm số $g(x)=3x^6-3x^3+5x-2$ liên tục trên đoạn $[0;1]$, có $g(0)\cdot g(1) = -2\cdot 3 =-6<0$ nên phương trình $g(x)=0$ có ít nhất một nghiệm thuộc khoảng $(0;1)$ hay phương trình $g(x)=0$ có nghiệm thuộc khoảng $(-2;2)$.
            \item Hàm số $h(x)=x^3-3x+1$ liên tục trên các đoạn $[-2;0]$, $[0;1]$, $[1;2]$ và có $h(-2)=-1$, $h(0)=1$, $h(1)=-1$, $h(2)=3$.\\
            Do $h(-2)\cdot h(0)<0$, $h(0)\cdot h(1)<0$, $h(1)\cdot h(2)<0$ nên phương trình $h(x)=0$ có ba nghiệm $x_1\in (-2;0)$, $x_2\in (0;1)$, $x_3\in (1;2)$ là ba nghiệm phân biệt.\\
            Mặt khác phương trình $h(x)=0$ là phương trình bậc ba nên có không quá ba nghiệm.\\
            Vậy phương trình $h(x)=0$ có ba nghiệm phân biệt.
            \item Hàm số $p(x)=m(x-1)^2(x-2)+2x-3$ liên tục trên đoạn $[1;2]$, có $p(1)\cdot p(2) = (-1)\cdot 1 = -1<0$ nên phương trình $p(x)=0$ luôn có nghiệm thuộc khoảng $(1;2)$ với mọi $m$ hay phương trình $m(x-1)^2(x-2)+2x-3=0$ luôn có nghiệm với mọi giá trị của tham số $m$.
        \end{itemize}
    }
\end{ex}

\begin{ex}%[DCHT Toán 11 - KNTT -Nguyễn Thành Nhân]%[1K5KG-7]
	Cho phương trình $x^4-4x^3+1=0$. Tìm mệnh đề \textbf{sai} trong các mệnh đề sau
	\choice
	{Phương trình có đúng một nghiệm $x>3$}
	{\True Phương trình vô nghiệm trên khoảng $(0;1)$}
	{Phương trình có ít nhất hai nghiệm}
	{Phương trình vô nghiệm trên khoảng $(-1;0)$}
	\loigiai{
		Xét hàm số $f(x)=x^4-4x^3+1$.\\
	Hàm số liên tục trên $\mathbb{R}$.\\
	Ta có $f(0)=1>0$ và $f(1)=-2<0$.\\
	Do $f(0)\cdot f(1)=-2<0$ và hàm số liên tục trên đoạn $[0;1]$ nên phương trình $f(x)=0$ có ít nhất một nghiệm $x_0\in (0;1)$.\\
	Vậy mệnh đề \lq\lq Phương trình vô nghiệm trên khoảng $(0;1)$\rq\rq\, là sai.	
	}
\end{ex}

\begin{ex}%[DCHT Toán 11 - KNTT -Nguyễn Thành Nhân]%[1K5KG-7]
	Xét phương trình sau trên tập số thực $x^{2023}+x=a \quad(1)$. Chọn khẳng định đúng trong các khẳng định dưới đây. 
	\choice
	{Phương trình $(1)$ chỉ có nghiệm khi $a>0$}
	{Phương trình $(1)$ chỉ có nghiệm khi $a<0$}
	{Phương trình $(1)$ vô nghiệm khi $a\geq 0$}
	{\True Phương trình $(1)$ có nghiệm $\forall a\in \mathbb{R}$}
	\loigiai
	{Phương trình $x^{2023}+x=a \Leftrightarrow x^{2023}+x-a=0 \quad(2)$.\\
	Xét hàm số $f(x)=x^{2023}+x-1$ trên $\mathbb{R}$, ta có
	\begin{itemize}
		\item $\lim\limits_{x\to -\infty}f(x)=\lim\limits_{x\to -\infty}\left(x^{2023}+x-a\right)=-\infty$ nên tồn tại số $\alpha<0$ để $f(\alpha)<0$.
		\item $\lim\limits_{x\to +\infty}f(x)=\lim\limits_{x\to +\infty}\left(x^{2023}+x-a\right)=+\infty$ nên tồn tại số $\beta>0$ để $f(\beta)>0$.
	\end{itemize}
	Hàm số $f(x)$ liên tục trên $[\alpha;\beta]$, có $f(\alpha)\cdot f(\beta)<0$ nên luôn tồn tại $c\in(\alpha;\beta)$ sao cho $f(c)=0$.\\
	Vậy phương trình $f(x)=0$ luôn có nghiệm với mọi $a\in\mathbb{R}$.}
\end{ex}

\begin{ex}%[DCHT Toán 11 - KNTT -Nguyễn Thành Nhân]%[1K5KG-7]
	Tìm tất cả các giá trị của tham số $m$ sao cho phương trình $(m^2-5m+3)x^5-2x^2+1=0$ có ít nhất một nghiệm thuộc khoảng $(-1;0)$.
	\choice
	{$m\in\left(-\infty;1\right)\cup\left(4;+\infty\right)$}
	{$m\in\mathbb{R}$}
	{\True $m\in\left(1;4\right)$}
	{$m\in\left(-\infty;1 \right]\cup\left[4;+\infty \right) $}
	\loigiai
	{
	Đặt $f(x)=(m^2-5m+3)x^5-2x^2+1$.\\
	Phương trình $(m^2-5m+3)x^5-2x^2+1=0$ có ít nhất một nghiệm thuộc khoảng $(-1;0)$
	\begin{eqnarray*}
		\Leftrightarrow f(-1)\cdot f(0)<0\Leftrightarrow (-m^2+5m-4)\cdot 1<0\Leftrightarrow 1<m<4.		
	\end{eqnarray*}
	Vậy $1<m<4$ thỏa yêu cầu bài toán.
	}
\end{ex}


\begin{ex}%[DCHT Toán 11 - KNTT -Nguyễn Thành Nhân]%[1K5KG-7]
	Cho phương trình $2x^4-5x^2+x+1=0$\quad$(1)$. Tìm mệnh đề đúng trong các mệnh đề sau:
	\choice
	{Phương trình $(1)$ không có nghiệm trong khoảng $(-2;0)$}
	{\True Phương trình $(1)$ có ít nhất $2$ nghiệm trong khoảng $(0;2)$}
	{Phương trình $(1)$ không có nghiệm trong khoảng $(-1;1)$}
	{Phương trình $(1)$ chỉ có $1$ nghiệm trong khoảng $(-2;1)$}
	\loigiai{
		Xét hàm số $f(x)=2x^4-5x^2+x+1$. Đây là hàm đa thức nên liên tục trên $\mathbb{R}$, do đó nó liên tục trên các đoạn $\left[0;1\right]$ và $\left[1;2\right]$.\\
		Ta có:
		$f(0)=1$; $f(1)=-1$; $f(2)=15$.\\
		$f(0)\cdot f(1)<0$ nên phương trình $f(x)=0$ có ít nhất một nghiệm $x_1\in(0;1)$\hfill$(1)$\\
		$f(1)\cdot f(2)<0$ nên phương trình $f(x)=0$ có ít nhất một nghiệm $x_2\in(1;2)$\hfill$(2)$\\
		Từ $(1)$ và $(2)$ suy ra phương trình $f(x)=0$ có ít nhất hai nghiệm trong khoảng $(0;2)$.
	}
\end{ex}
\begin{ex}%[DCHT Toán 11 - KNTT -Nguyễn Thành Nhân]%[1K5GG-7]
	Tập tất cả các giá trị của tham số thực $m$ để phương trình\\ $\left(2m^2-5m+2\right)\left(x-1\right)^{2021}\left(x^{2020}-2\right)+2x+3=0$ có nghiệm là
	\choice
	{$m\in \mathbb{R}\setminus \left\lbrace \dfrac{1}{2};2\right\rbrace$}
	{$m\in \left\lbrace \dfrac{1}{2};2\right\rbrace$}
	{$m\in \left(-\infty;\dfrac{1}{2}\right)\cup(2;+\infty)$}
	{\True $m\in \mathbb{R}$}
\loigiai{
	Đặt $f(x)=\left(2m^2-5m+2\right)\left(x-1\right)^{2021}\left(x^{2020}-2\right)+2x+3$.
	\begin{itemize}
		\item Với $2m^2-5m+2=0\Leftrightarrow\hoac{&m=\dfrac{1}{2}\\&m=2}$ phương trình trên trở thành $2x+3=0\Leftrightarrow x=\dfrac{-3}{2}$.\\
		Vậy với $m\in \left\lbrace \dfrac{1}{2};2\right\rbrace$ phương trình có nghiệm $\qquad(*)$
		\item Với $2m^2-5m+2\ne0\Leftrightarrow m\in \mathbb{R}\setminus \left\lbrace \dfrac{1}{2};2\right\rbrace$, $f(x)$ là đa thức bậc lẻ có TXĐ $\mathscr{D}=\mathbb{R}$.\\
		Khi đó $f(x)$ liên tục trên $\mathbb{R}$.		
		\begin{itemize}
			\item Với $m\in\left(-\infty;\dfrac{1}{2}\right)\cap\left(2;+\infty\right)\Rightarrow 2m^2-5m+2>0$\\
			$\lim\limits_{x\rightarrow -\infty}f(x)=-\infty \Rightarrow \exists x_1<0\colon f(x_1)<0\qquad(1)$\\
			$\lim\limits_{x\rightarrow +\infty}f(x)=+\infty \Rightarrow \exists x_2>0\colon f(x_2)>0\qquad(2)$\\
			Từ $(1)$ và $(2)$ suy ra $f(x)$ luôn có ít nhất $1$ nghiệm $x_0\in\left(-\infty;+\infty\right)\qquad(**)$.
			\item Với $m\in\left(\dfrac{1}{2};2\right)\Rightarrow 2m^2-5m+2<0$\\
			$\lim\limits_{x\rightarrow -\infty}f(x)=+\infty \Rightarrow \exists x_3<0\colon f(x_3)>0\qquad(3)$\\
			$\lim\limits_{x\rightarrow +\infty}f(x)=-\infty \Rightarrow \exists x_4>0\colon f(x_4)<0\qquad(4)$\\
			Từ $(3)$ và $(4)$ suy ra $f(x)$ luôn có ít nhất $1$ nghiệm $x_0\in\left(-\infty;+\infty\right)\qquad(***)$.
		\end{itemize}
	\end{itemize}	
	Từ $(*),\;(**)$ và $(***)$ suy ra phương trình luôn có nghiệm với $m\in\mathbb{R}$.
}
\end{ex}


\Closesolutionfile{ans}