%\chapter{Hàm số  lượng giác và phương trình lượng giác}
%\setcounter{section}{2}
%\section{Giới hạn hàm số}
%\subsection{Tóm tắt lý thuyết}
%\begin{tomtat}
%	\subsubsection{mục 1}
%	\subsubsection{mục 2}
%\end{tomtat}
%\subsection{Các dạng toán thường gặp}
\begin{dang}{Phương pháp đặt thừa số chung - kết quả hữu hạn}
		\begin{itemize}
		\item Nếu tam thức bậc hai $ax^2+bx+c$ có hai nghiệm $x_1$, $x_2$ thì $ ax^2+bx+c=a(x-x_1)(x-x_2)$.
		\item $a^n-b^n=(a-b)\left(a^{n-1}+a^{n-2}b+\cdots +ab^{n-2}+b^{n-1}\right)$.
		\item $\lim\limits_{x\to \pm \infty } c=c;\lim\limits_{x\to \pm \infty } \dfrac {c}{x^k}=0$ với $c$ là hằng số và $ k\in \mathbb{N}$.
		\item $a\sqrt{b}=\heva{&\sqrt{a^2b}\quad a\ge 0\\&-\sqrt{a^2b}\quad a<0.}$
		\end{itemize}
\end{dang}
\subsubsection{Ví dụ minh hoạ}
%VD1
\begin{vd}[NB]%[DCHT Toán 11 - KNTT -Nguyễn Văn Hiệp]%[1K5YF-3]
	Tính giới hạn $\lim\limits_{x\to 3} \dfrac {x^2-9}{x-3}$.\dapso{$I=6$.}
	\loigiai{Ta có $\lim\limits_{x\to 3} \dfrac {x^2-9}{x-3}=\lim\limits_{x\to 3} \dfrac {(x-3)(x+3)}{x-3}=\lim\limits_{x\to 3} (x+3)=6$.}
\end{vd}
%VD2
\begin{vd}[TH]%[DCHT Toán 11 - KNTT -Nguyễn Văn Hiệp]%[1K5BF-3]
	Tính giới hạn $I=\lim\limits_{x\to 2} \dfrac {x^2-5x+6}{x-2}$.
	\dapso{$I=-1$.}
	\loigiai{
		$I=\lim\limits_{x\to 2} \dfrac {x^2-5x+6}{x-2}=\lim\limits_{x\to 2} \dfrac {(x-2)(x-3)}{x-2}=\lim\limits_{x\to 2} (x-3)=-1$.
	}
\end{vd}
%VD3
\begin{vd}[TH]%[DCHT Toán 11 - KNTT -Nguyễn Văn Hiệp]%[1K5BF-3]
	Tính giới hạn $\lim\limits_{x\to +\infty } \dfrac {x^4+7}{x^4+1}$.
	\dapso{$1$.}
	\loigiai{Ta có
		$\lim\limits_{x\to +\infty} \dfrac {x^4+7}{x^4+1}=\lim\limits_{x\to +\infty } \dfrac {x^4\left(1+\dfrac {7}{x^4}\right)}{x^4\left(1+\dfrac {1}{x^4}\right)}=\lim\limits_{x\to +\infty } \dfrac {1+\dfrac {7}{x^4}}{1+\dfrac {1}{x^4}}=1$.
	}
\end{vd}
%VD4
\begin{vd}[TH]%[DCHT Toán 11 - KNTT -Nguyễn Văn Hiệp]%[1K5BF-3]
	Tìm giới hạn $\lim\limits_{x\to +\infty } \sqrt {\dfrac {x^2+1}{2x^4+x^2-3}}$.
	\dapso{$0$.}
	\loigiai{
		Ta có $\lim\limits_{x\to +\infty } \sqrt {\dfrac {x^2+1}{2x^4+x^2-3}}=\lim\limits_{x\to +\infty} \dfrac{x}{x^2}\cdot \sqrt {\dfrac {\dfrac {1}{x^2}+\dfrac {1}{x^4}}{2+\dfrac {1}{x^2}-\dfrac {3}{x^4}}}=\lim\limits_{x\to +\infty} \dfrac{1}{x}\cdot \sqrt {\dfrac {\dfrac {1}{x^2}+\dfrac {1}{x^4}}{2+\dfrac {1}{x^2}-\dfrac {3}{x^4}}}=0$.}
\end{vd}
%VD5
\begin{vd}[TH]%[DCHT Toán 11 - KNTT -Nguyễn Văn Hiệp]%[1K5BF-3]
	Tính giới hạn $\lim\limits_{x\to 1} \left(\dfrac {1}{1-x}-\dfrac {3}{1-x^3} \right)$. 
	\dapso{$-1$.}
	\loigiai{$\lim\limits_{x\to 1} \left(\dfrac {1}{1-x}-\dfrac {3}{1-x^3} \right)=\lim\limits_{x\to 1} \dfrac {1+x+x^2-3}{1-x^3}=\lim\limits_{x\to 1} \dfrac {(x-1 )(x+2)}{(1-x)\left( 1+x+x^2 \right)}=\lim\limits_{x\to 1} \dfrac {-(x+2)}{1+x+x^2}=-1$. 
	}
\end{vd}
%VD6
\begin{vd}[VDT]%[DCHT Toán 11 - KNTT -Nguyễn Văn Hiệp]%[1K5KF-3]
	Cho $m,n$ là các số thực khác $0$. Nếu giới hạn $\lim\limits_{x\to -5} \dfrac {x^2+mx+n}{x+5}=3$, hãy tìm $mn$.
	\dapso{$mn=520$.}
	\loigiai{
		Vì $\lim\limits_{x\to -5} \dfrac {x^2+mx+n}{x-5}=3$ nên $ x=-5$ là nghiệm của phương trình $ x^2+mx+n=0$.\\
		$\Rightarrow -5m+n+25=0\Leftrightarrow n=5m-25$.\\
		Khi đó 
		\allowdisplaybreaks
		$\begin{aligned}[t]
		\lim\limits_{x\to -5} \dfrac {x^2+mx+n}{x-1}&=\lim\limits_{x\to -5} \dfrac {x^2+mx+5m-25}{x+5}\\
		&=\lim\limits_{x\to -5} \dfrac {(x+5)(x-5+m)}{x+5}\\
		&=\lim\limits_{x\to -5} (x-5+m)=m-10.
		\end{aligned}$\\
		Ta có $m-10=3\Leftrightarrow m=13\Rightarrow n=40$.\\
		Vậy $mn=13\cdot 40=520$.
	}
\end{vd}
%VD7
\begin{vd}[VDT]%[DCHT Toán 11 - KNTT -Nguyễn Văn Hiệp]%[1K5KF-3]
Tìm số thực $a$ thỏa mãn $\lim\limits_{x\to +\infty} \dfrac {a\sqrt {2x^2+3}+2024}{2x+2023}=\dfrac {1}{2}$.\dapso{$a=\dfrac {\sqrt {2}}{2}$.}
\loigiai{Ta có $\lim\limits_{x\to +\infty } \dfrac {a\sqrt {2x^2+3}+2024}{2x+2023}=\dfrac {1}{2}\Leftrightarrow \lim\limits_{x\to +\infty } \dfrac {a\sqrt {2+\dfrac {3}{x^2}}+\dfrac {2024}{x}}{2+\dfrac {2023}{x}}=\dfrac {1}{2}\Leftrightarrow \dfrac {a\sqrt {2}}{2}=\dfrac {1}{2}\Leftrightarrow a=\dfrac {\sqrt {2}}{2}$.}
\end{vd}
\subsubsection{Bài tập rèn luyện}
\subsubsection{Bài tập tự luận}
%BT1
\begin{bt}[NB]%[DCHT Toán 11 - KNTT -Nguyễn Văn Hiệp]%[1K5YF-3]
	Tính $\lim\limits_{x\to 2} \dfrac {x^2-4}{x-2}$.
	\dapso{$4$}
	\loigiai{$\lim\limits_{x\to 2} \dfrac {x^2-4}{x-2}=\lim\limits_{x\to 2} \dfrac {(x-2)(x+2)}{x-2}=\lim\limits_{x\to 2} (x+2)=2+2=4$.}
\end{bt}
%BT2
\begin{bt}[NB]%[DCHT Toán 11 - KNTT -Nguyễn Văn Hiệp]%[1K5YF-3]
	Tính $\lim\limits_{x\to 5} \dfrac {x^2-12x+35}{25-5x}$.
	\dapso{$\dfrac{2}{5}$.}
	\loigiai{Ta có $\lim\limits_{x\to 5} \dfrac {x^2-12x+35}{25-5x}=\lim\limits_{x\to 5} \dfrac {(x-7)(x-5)}{5(5-x)}=\lim\limits_{x\to 5} \dfrac {7-x}{5}=\dfrac {2}{5}$.\\
	Vậy $\lim\limits_{x\to 5} \dfrac {x^2-12x+35}{25-5x}=\dfrac {2}{5}$.
}
\end{bt}
%BT3
\begin{bt}[TH]%[DCHT Toán 11 - KNTT -Nguyễn Văn Hiệp]%[1K5BF-3]
	Tính giới hạn $I=\lim\limits_{x\to 2} \dfrac {x^3-8}{x^2-4}$.
	\dapso{$I=3$.}
	\loigiai{
		Ta có $I=\lim\limits_{x\to 2} \dfrac {x^3-8}{x^2-4}=\lim\limits_{x\to 2} \dfrac {(x-2)\left(x^2+2x+4\right)}{(x-2)(x+2)}=\lim\limits_{x\to 2} \dfrac {x^2+2x+4}{x+2}=\dfrac {12}{4}=3$.
	}
\end{bt}
%BT4
\begin{bt}[TH]%[DCHT Toán 11 - KNTT -Nguyễn Văn Hiệp]%[1K5BF-3]
	Tìm giới hạn $A=\lim\limits_{x\to 2} \dfrac {x^4-5x^2+4}{x^3-8}$.
	\dapso{$A=1$.}
	\loigiai{Ta có 
		\allowdisplaybreaks
		\begin{eqnarray*}
		A&=&\lim\limits_{x\to 2} \dfrac {x^4-5x^2+4}{x^3-8}=\lim\limits_{x\to 2} \dfrac {\left(x^2-1\right) \left(x^2-4\right)}{x^3-2^3}\\
		&=&\lim\limits_{x\to 2} \dfrac {\left(x^2-1\right)\left(x-2 \right)\left(x+2\right)}{\left(x-2\right)\left(x^2+2x+4 \right)}\\
		&=&\lim\limits_{x\to 2} \dfrac {\left(x^2-1\right)(x+2)}{x^2+2x+4}\\
		&=&1.
		\end{eqnarray*}}
\end{bt}
%BT5
\begin{bt}[TH]%[DCHT Toán 11 - KNTT -Nguyễn Văn Hiệp]%[1K5BF-3]
	Tìm giới hạn $\lim\limits_{x\to -\infty }\dfrac {1+3x}{\sqrt {2x^2+3}}$.
	\dapso{$-\dfrac {3\sqrt {2}}{2}$.}
	\loigiai{
		Ta có $\lim\limits_{x\to -\infty } \dfrac {1+3x}{\sqrt {2x^2+3}}=\lim\limits_{x\to -\infty } \dfrac {x\cdot \left(\dfrac {1}{x}+3\right)}{-x\cdot \left(\sqrt {2+\dfrac {3}{x}}\right)}=\lim\limits_{x\to -\infty } \dfrac {\dfrac {1}{x}+3}{-\sqrt {2+\dfrac {3}{x}}}=-\dfrac {3\sqrt {2}}{2}$.
	}
\end{bt}
%BT6
\begin{bt}[TH]%[DCHT Toán 11 - KNTT -Nguyễn Văn Hiệp]%[1K5BF-3]
	Tìm giới hạn $\lim\limits_{x\to +\infty } \dfrac {2x-\sqrt {3x^2+2}}{5x+\sqrt {x^2+2}}$.
	\dapso{$\dfrac {2-\sqrt {3}}{6}$.}
	\loigiai{
		Ta có $\lim\limits_{x\to +\infty } \dfrac {2x-\sqrt {3x^2+2}}{5x+\sqrt {x^2+2}}=\lim\limits_{x\to +\infty }\dfrac{x}{x}\cdot \dfrac {2-\sqrt {3+\dfrac {2}{x^2}}}{5+\sqrt {1+\dfrac {2}{x^2}}}=\lim\limits_{x\to +\infty } \dfrac {2-\sqrt {3+\dfrac {2}{x^2}}}{5+\sqrt {1+\dfrac {2}{x^2}}}=\dfrac {2-\sqrt {3}}{6}$.
	}
\end{bt}
%BT7
\begin{bt}[VDT]%[DCHT Toán 11 - KNTT -Nguyễn Văn Hiệp]%[1K5KF-3]
	Giá trị của $\lim\limits_{x\to 1} \dfrac {x^{2024}+x-2}{x^{2023}+x-2}$ bằng $\dfrac {a}{b}$, với $\dfrac {a}{b}$ là phân số tối giản. Tính giá trị của $a^2-b^2$.
	\dapso{$4049$.}
	\loigiai{Ta có
		\allowdisplaybreaks
		\begin{eqnarray*}
			&&\lim\limits_{x\to 1} \dfrac {x^{2024}+x-2}{x^{2023}+x-2}=\lim\limits_{x\to 1} \dfrac {x^{2024}-1+x-1}{x^{2023}-1+x-1}\\
			&=&\lim\limits_{x\to 1} \dfrac {(x-1)(x^{2023}+x^{2022}\cdots +x+1)+x-1}{(x-1)(x^{2022}+x^{2021}+\cdots+x+1)+x-1}\\
			&=&\lim\limits_{x\to 1} \dfrac {x^{2023}+x^{2022}\cdots +x+2}{x^{2022}+x^{2021}+\cdots+x+2}\\
			&=&\dfrac {1+1+\cdots +1+2}{1+1+\cdots +1+2}=\dfrac {2025}{2024}.
		\end{eqnarray*}
		Vậy $a^2-b^2=2025^2-2024^2=4049$.}
\end{bt}
%BT8
\begin{bt}[VDT]%[DCHT Toán 11 - KNTT -Nguyễn Văn Hiệp]%[1K5KF-3]
	Cho giới hạn $\lim\limits_{x\to 3} \dfrac {x^2+ax+b}{x-3}=3$. Tìm $a$, $b$.
	\dapso{$a=-3$, $b=0$.}
	\loigiai{
		Để $\lim\limits_{x\to 3} \dfrac {x^2+ax+b}{x-3}=3$ thì ta phải có $x^2+ax+b=(x-3)(x-m)$.\\
		Khi đó $3-m=3\Leftrightarrow m=0$. Vậy $x^2+ax+b=(x-3 )x=x^2-3x$.\\
		Suy ra $a=-3$ và $b=0$.
	}
\end{bt}
%BT9
\begin{bt}[VDT]%[DCHT Toán 11 - KNTT -Nguyễn Văn Hiệp]%[1K5KF-3]
Tìm $m$ để $\lim\limits_{x\to -\infty } \dfrac {\sqrt {4x^2+x+1}+4}{mx-2}=\dfrac {1}{2}$.\dapso{$m=-4$.}
\loigiai{Ta có $\lim\limits_{x\to -\infty } \dfrac {\sqrt {4x^2+x+1}+4}{mx-2}=\lim\limits_{x\to -\infty }\dfrac{-x}{x}\cdot  \dfrac {\sqrt {4+\dfrac {1}{x}+\dfrac {1}{x^2}}-\dfrac {4}{x}}{m-\dfrac {2}{x}}=\lim\limits_{x\to -\infty } \dfrac {-\sqrt {4+\dfrac {1}{x}+\dfrac {1}{x^2}}+\dfrac {4}{x}}{m-\dfrac {2}{x}}=-\dfrac {2}{m}$.\\
Theo bài ra ta có $-\dfrac {2}{m}=\dfrac {1}{2}\Leftrightarrow m=-4$.
	}
\end{bt}
%BT10
\begin{bt}[VDC]%[DCHT Toán 11 - KNTT -Nguyễn Văn Hiệp]%[1K5GF-3]
	Tính giới hạn $\lim\limits_{x\to 1} \left( \dfrac {m}{1-x^m}-\dfrac {n}{1-x^n} \right)$, $m,n\in \mathbb{N^*}$.
	\dapso{ $\dfrac{m-n}{2}$.}
	\loigiai{
		\allowdisplaybreaks
		\begin{eqnarray*}
		\lim\limits_{x\to 1} \left(\dfrac {m}{1-x^m}-\dfrac {n}{1-x^n} \right)&=&\lim\limits_{x\to 1} \left[ \left( \dfrac {m}{1-x^m}-\dfrac {1}{1-x} \right)-\left( \dfrac {n}{1-x^n}-\dfrac {1}{1-x} \right)\right]\\
		&=&\lim\limits_{x\to 1} \left( \dfrac {m}{1-x^m}-\dfrac {1}{1-x} \right)-\lim\limits_{x\to 1} \left( \dfrac {n}{1-x^n}-\dfrac {1}{1-x} \right)=A-B.
		\end{eqnarray*}
		\allowdisplaybreaks
		\begin{eqnarray*}
		A&=&\lim\limits_{x\to 1} \left( \dfrac {m}{1-x^m}-\dfrac {1}{1-x} \right)\\
		&=&\lim\limits_{x\to 1} \dfrac {m-\left(1+x+x^2+\cdots +x^{m-1}\right)}{1-x^m}\\
		&=&\lim\limits_{x\to 1} \dfrac {(1-x)+\left(1-x^2\right)+\cdots +\left(1-x^{m-1}\right)}{1-x^m}\\
		&=&\lim\limits_{x\to 1} \dfrac {(1-x)\left[1+(1+x)+\cdots +\left( 1+x+\cdots +x^{m-2} \right) \right]}{(1-x)\left(1+x+\cdots +x^{m-1}\right)}\\
		&=&\lim\limits_{x\to 1} \dfrac {1+(1+x)+\cdots +\left(1+x+\cdots +x^{m-2} \right)}{1+x+\cdots +x^{m-1}}\\
		&=&\lim\limits_{x\to 1} \dfrac {1+2+\cdots +m-1}{m}\\
		&=&\dfrac {m-1}{2}.
		\end{eqnarray*}
		Tương tự, ta tính được $B=\dfrac {n-1}{2}$.\\
		Vậy $\lim\limits_{x\to 1} \left( \dfrac {m}{1-x^m}-\dfrac {n}{1-x^n} \right)=A-B=\dfrac {m-n}{2}$.
	}
\end{bt}
\subsubsection{Bài tập trắc nghiệm}
\Opensolutionfile{ans}[ans/ans-1K5-2-Dang2]

\begin{ex}%[DCHT Toán 11 - KNTT -Nguyễn Văn Hiệp]%[1K5BF-3]
	Tìm $\lim\limits_{x\to 3} \dfrac {9-x^2}{x^2-4x+3}$. Kết quả là
	\choice
	{\True $-3$}
	{$4$}
	{$-4$}
	{$3$}
	\loigiai
	{Ta có $\lim\limits_{x\to 3} \dfrac {9-x^2}{x^2-4x+3}=\lim\limits_{x\to 3} \dfrac {(3-x)(3+x)}{(x-3)(x-1)}=\lim\limits_{x\to 3} \dfrac {-(x+3)}{x-1}=-3$.
	}
\end{ex}
%Cau2
\begin{ex}%[DCHT Toán 11 - KNTT -Nguyễn Văn Hiệp]%[1K5YF-3]
Tìm $\lim\limits_{x\to 4} \dfrac {x^2-16}{x-4}$. Kết quả là
	\choice
	{$7$}
	{\True $8$}
	{$5$}
	{$6$}
	\loigiai
	{Ta có $\lim\limits_{x\to 4} \dfrac {x^2-16}{x-4}=\lim\limits_{x\to 4} \dfrac {(x-4)(x+4)}{x-4}=\lim\limits_{x\to 4} (x+4)=8$.
	}
\end{ex}
%Cau3
\begin{ex}%[DCHT Toán 11 - KNTT -Nguyễn Văn Hiệp]%[1K5YF-3]
Tính giới hạn $A=\lim\limits_{x\to 1} \dfrac {x^3-1}{x-1}$.
\choice
{$A=-\infty $}
{$A=0$}
{\True $A=3$}
{$A=+\infty $}
\loigiai{Ta có $A=\lim\limits_{x\to 1} \dfrac {x^3-1}{x-1}=\lim\limits_{x\to 1} \dfrac {( x-1 )\left(x^2+x+1 \right)}{x-1}=\lim\limits_{x\to 1} \left(x^2+x+1 \right)=3$.}
\end{ex}
%Cau4
\begin{ex}%[DCHT Toán 11 - KNTT -Nguyễn Văn Hiệp]%[1K5BF-3]
Chọn kết quả đúng trong các kết quả sau của $\lim\limits_{x\to -1} \dfrac {x^2+2x+1}{2x+2}$ là
	\choice
	{$-\infty $}
	{\True $0$}
	{$\dfrac {1}{2}$}
	{$+\infty $}
	\loigiai
	{$\lim\limits_{x\to -1} \dfrac {x^2+2x+1}{2x+2}=\lim\limits_{x\to -1} \dfrac {(x+1)^2}{2(x+1)}=\lim\limits_{x\to -1} \dfrac {x+1}{2}=0$.
	}
\end{ex}
%Cau5
\begin{ex}%[DCHT Toán 11 - KNTT -Nguyễn Văn Hiệp]%[1K5BF-3]
Giới hạn $\lim\limits_{x\to 4} \dfrac {x^2+2x-15}{x-3}$ bằng
	\choice
	{$\dfrac {1}{8}$}
	{\True $9$}
	{$+\infty $}
	{$8$}
	\loigiai{$\lim\limits_{x\to 4} \dfrac {x^2+2x-15}{x-3}=\lim\limits_{x\to 4} \dfrac {( x-3 )( x+5 )}{x-3}=9 $.}
\end{ex}
%Cau6
\begin{ex}%[DCHT Toán 11 - KNTT -Nguyễn Văn Hiệp]%[1K5BF-3]
	 Tính giới hạn $I=\lim\limits_{x\to -\infty } \dfrac {3x-2}{2x+1}$.
	 \choice
	 {$I=-2$}
	 {$I=-\dfrac {3}{2}$}
	 {$I=2$}
	 {\True $I=\dfrac {3}{2}$}
	 \loigiai{Ta có $I=\lim\limits_{x\to -\infty } \dfrac {3x-2}{2x+1}=\lim\limits_{x\to -\infty } \dfrac {x\left(3-\dfrac {2}{x}\right)}{x\left(2+\dfrac {1}{x}\right)}=\lim\limits_{x\to -\infty } \dfrac {3-\dfrac {2}{x}}{2+\dfrac {1}{x}}=\dfrac {3}{2}$.}
\end{ex}
%Cau7
\begin{ex}%[DCHT Toán 11 - KNTT -Nguyễn Văn Hiệp]%[1K5BF-3]
	 $\lim\limits_{x\to +\infty } \dfrac {x-2}{x+3}$ bằng
	 \choice
	 {$-\dfrac {2}{3}$}
	 {\True $1$}
	 {$2$}
	 {$-3$}
	 \loigiai{Ta có $\lim\limits_{x\to +\infty} \dfrac {x-2}{x+3}=\lim\limits_{x\to +\infty } \dfrac {x\left(1-\dfrac {2}{x}\right)}{x\left(1+\dfrac {3}{x}\right)}=\lim\limits_{x\to +\infty } \dfrac {1-\dfrac {2}{x}}{1+\dfrac {3}{x}}=\dfrac {1}{1}=1$.}
\end{ex}
%Cau8
\begin{ex}%[DCHT Toán 11 - KNTT -Nguyễn Văn Hiệp]%[1K5BF-3]
	 Giới hạn $\lim\limits_{x\to 1} \dfrac {x^2-3x+2}{x^3-x^2+x-1}$ bằng
	 \choice
	 {$-2$}
	 {$-1$}
	 {\True $-\dfrac {1}{2}$}
	 {$\dfrac {1}{2}$}
	 \loigiai{Ta có $\lim\limits_{x\to 1} \dfrac {x^2-3x+2}{x^3-x^2+x-1}$$=\lim\limits_{x\to 1} \dfrac {( x-1 )( x-2 )}{( x-1 )( x^2+1 )}$$=\lim\limits_{x\to 1} \dfrac {x-2}{x^2+1}$$=-\dfrac {1}{2}$.}
\end{ex}
%Cau9
\begin{ex}%[DCHT Toán 11 - KNTT -Nguyễn Văn Hiệp]%[1K5BF-3]
	 Giới hạn $T=\lim\limits_{x\to 1} \dfrac {x^4-3x+2}{x^3+2x-3}$ bằng
	 \choice
	 {$\dfrac {2}{9}$}
	 {$\dfrac {2}{5}$}
	 {\True $\dfrac {1}{5}$}
	 {$+\infty $}
	 \loigiai
	 {$T=\lim\limits_{x\to 1} \dfrac {x^4-3x+2}{x^3+2x-3}=\lim\limits_{x\to 1} \dfrac {(x-1)\left(x^3+x^2+x-2\right)}{(x-1)\left(x^2+x+3 \right)}=\lim\limits_{x\to 1} \dfrac {x^3+x^2+x-2}{x^2+x+3}=\dfrac {1^3+1^2+1-2}{1^2+1+3}=\dfrac {1}{5}$.
	 }
\end{ex}
%Cau10
\begin{ex}%[DCHT Toán 11 - KNTT -Nguyễn Văn Hiệp]%[1K5BF-3]
	Giới hạn $\lim\limits_{x\to 2} \dfrac {x^2-5x+6}{x^3-x^2-x-2}$ bằng
	\choice
	{$0$}
	{\True $-\dfrac {1}{7}$}
	{$-7$}
	{$+\infty $}
	\loigiai{Ta có $\lim\limits_{x\to 2} \dfrac {x^2-5x+6}{x^3-x^2-x-2}=\lim\limits_{x\to 2} \dfrac {(x-2)(x-3 )}{(x-2)\left(x^2+x+1\right)}=\lim\limits_{x\to 2} \dfrac {x-3}{x^2+x+1}=\dfrac {-1}{7}$.}
\end{ex}
%Cau11
\begin{ex}%[DCHT Toán 11 - KNTT -Nguyễn Văn Hiệp]%[1K5BF-3]
	Tìm $\lim\limits_{x\to 1} \dfrac {x^4-3x^2+2}{x^3+2x-3}$.
	\choice
	{$-\dfrac {5}{2}$}
	{\True $-\dfrac {2}{5}$}
	{$\dfrac {1}{5}$}
	{$+\infty $}
	\loigiai{$\lim\limits_{x\to 1} \dfrac {x^4-3x^2+2}{x^3+2x-3}=\lim\limits_{x\to 1} \dfrac {(x-1)(x+1)\left(x^2-2\right)}{(x-1)\left(x^2+x+3 \right)}=\lim\limits_{x\to 1} \dfrac {(x+1)\left(x^2-2 \right)}{x^2+x+3}=-\dfrac {2}{5}$.}
\end{ex}
%Câu 12
\begin{ex}%[DCHT Toán 11 - KNTT -Nguyễn Văn Hiệp]%[1K5BF-3]
	Tính $\lim\limits_{x\to 2} \left( \dfrac {1}{x^2-3x+2}+\dfrac {1}{x^2-5x+6} \right)$.
	\choice
	{$2 $}
	{$+\infty$}
	{\True $-2$}
	{$0$}
	\loigiai
	{\allowdisplaybreaks
	$\begin{aligned}[t]
	&\lim\limits_{x\to 2} \left( \dfrac {1}{x^2-3x+2}+\dfrac {1}{x^2-5x+6} \right)=\lim\limits_{x\to 2} \dfrac {2x^2-8x+8}{\left(x^2-3x+2 \right) \left(x^2-5x+6\right)}\\
	=&\lim\limits_{x\to 2} \dfrac {2(x-2)^2}{(x-1)(x-2)(x-2)(x-3)}=\lim\limits_{x\to 2} \dfrac {2}{(x-1)(x-3)}=-2.
	\end{aligned}$
}
\end{ex}
%Cau 13
\begin{ex}%[DCHT Toán 11 - KNTT -Nguyễn Văn Hiệp]%[1K5BF-3]
	Giới hạn $\lim \limits_{ x \to + \infty} \dfrac {\sqrt {x ^2  + 2} - 2} {x - 2}$ bằng
	\choice
	{$- \infty$}
	{\True $1$}
	{$+\infty$}
	{$-1$}
	\loigiai{$\lim\limits_{x\to +\infty } \dfrac {\sqrt {x^2+2}-2}{x-2}=\lim\limits_{x\to +\infty } \dfrac {x\sqrt {1+\dfrac {2}{x^2}}-2}{x-2}=\lim\limits_{x\to +\infty } \dfrac {\sqrt {1+\dfrac {2}{x^2}}-\dfrac {2}{x}}{1-\dfrac {2}{x}}=1$.}
\end{ex}
%Cau 14
\begin{ex}%[DCHT Toán 11 - KNTT -Nguyễn Văn Hiệp]%[1K5BF-3]
Cho hàm số $ f(x)=\dfrac {(4x+1)^3(2x+1)^4}{(3+2x)^7}$. Tính $\lim\limits_{x\to -\infty}f(x)$.
\choice
{$2$}
{\True $8$}
{$4$}
{$0$}
\loigiai{$\lim\limits_{x\to -\infty} f(x)=\lim\limits_{x\to -\infty } \dfrac {(4x+1)^3(2x+1)^4}{(3+2x)^7}=\lim\limits_{x\to -\infty} \dfrac {\left(4+\dfrac {1}{x} \right)^3\left( 2+\dfrac {1}{x} \right)^4}{\left( \dfrac {3}{x}+2 \right)^7}=2^3=8$.}
\end{ex}
%Cau 15
\begin{ex}%[DCHT Toán 11 - KNTT -Nguyễn Văn Hiệp]%[1K5KF-3]
	Biết $\lim\limits_{x\to 3} \dfrac {x^2+bx+c}{x-3}=8$, $(b,c\in \mathbb{R})$. Tính $P=b+c$.
	\choice
	{\True $P=-13$}
	{$P=-11$}
	{$P=5$}
	{$P=-12 $}
	\loigiai{Vì $\lim\limits_{x\to 3} \dfrac {x^2+bx+c}{x-3}=8$ là hữu hạn nên tam thức $ x^2+bx+c$ có nghiệm $x=3$.\\
	$\Rightarrow 3b+c+9=0\Leftrightarrow c=-9-3b$.\\
		Khi đó
		$\begin{aligned}[t]
		\lim\limits_{x\to 3} \dfrac {x^2+bx+c}{x-3}&=\lim\limits_{x\to 3} \dfrac {x^2+bx-9-3b}{x-3}=\lim\limits_{x\to 3} \dfrac {(x-3)(x+3+b)}{x-3} \\
		&=\lim\limits_{x\to 3} (x+3+b)=8\Leftrightarrow 6+b=8\Leftrightarrow b=2\Rightarrow c=-15.
		\end{aligned}$\\
		Vậy $P=b+c=-13$.}
\end{ex}
%Cau 16
\begin{ex}%[DCHT Toán 11 - KNTT -Nguyễn Văn Hiệp]%[1K5KF-3]
Cho $a,b$ là số nguyên và $\lim\limits_{x\to 1} \dfrac {ax^2+bx-5}{x-1}=7$. Tính $a^2+b^2+a+b$.
	\choice
	{\True $18$}
	{$1$}
	{$15$}
	{$5$}
\loigiai{Vì $\lim\limits_{x\to 1} \dfrac {ax^2+bx-5}{x-1}=7$ hữu hạn nên $ x=1$ phải là nghiệm của phương trình $ ax^2+bx-5=0$ suy ra $ a+b-5=0\Rightarrow b=5-a$.\\
	Khi đó $\lim\limits_{x\to 1} \dfrac {ax^2+\left( 5-a \right)x-5}{x-1}=\lim\limits_{x\to 1} \dfrac {\left( x-1 \right)( ax+5 )}{x-1}=a+5=7\Rightarrow a=2$ nên $b=3$.\\
	Suy ra $a^2+b^2+a+b=18$.}
\end{ex}
%Cau 17
\begin{ex}%[DCHT Toán 11 - KNTT -Nguyễn Văn Hiệp]%[1K5KF-3]
	Biết rằng $\lim\limits_{x\to +\infty } \left( \dfrac {x^2+1}{x-2}+ax-b \right)=-5$. Tính tổng $a+b$.
	\choice
	{\True $6$}
	{$7$}
	{$8$}
	{$5$}
	\loigiai{
		\allowdisplaybreaks
		$\begin{aligned}[t]
		&\lim\limits_{x\to +\infty } \left(\dfrac {x^2+1}{x-2}+ax-b\right)=\lim\limits_{x\to +\infty}\left( \dfrac {(a+1)x^2-(2a+b)x+2b+1}{x-2}\right)=-5\\
		\Leftrightarrow& \heva{&a+1=0 \\ &2a+b=5}\Leftrightarrow \heva{&a=-1\\&b=7.}
		\end{aligned}$\\
		Vậy $ a+b=6$.}
\end{ex}
%Cau 18
\begin{ex}%[DCHT Toán 11 - KNTT -Nguyễn Văn Hiệp]%[1K5KF-3]
	Cho hai số thực $ a$ và $ b$ thỏa mãn $\lim\limits_{x\to +\infty} \left( \dfrac {4x^2-3x+1}{x+2}-ax-b\right)=0$. Khi đó $a+b$ bằng
	\choice
	{$-4$}
	{$4$}
	{$7$}
	{\True $-7$}
	\loigiai{$\lim\limits_{x\to +\infty}\left(\dfrac{4x^2-3x+1}{x+2}-ax-b \right)=0\Leftrightarrow \lim\limits_{x\to +\infty } \left( \left( 4-a \right)x-b-11+\dfrac {23}{x+2} \right)=0$.\\
	$\Rightarrow \heva{&4-a=0\\&-11-b=0 }\Leftrightarrow \heva{&a=4\\&b=-11}\Rightarrow a+b=-7$.}
\end{ex}
%Cau 19
\begin{ex}%[DCHT Toán 11 - KNTT -Nguyễn Văn Hiệp]%[1K5KF-3]
	Cho $\lim\limits_{x\to 1} \dfrac {f(x)+1}{x-1}=-1$. Tính $\lim\limits_{x\to 1} \dfrac {\left(x^2+x\right)f(x)+2}{x-1}$.
	\choice
	{$I=5$}
	{$I=-4$}
	{$I=4$}
	{\True $I=-5$}
	\loigiai{$\lim\limits_{x\to 1} \dfrac {\left(x^2+x \right)f(x)+2}{x-1}=\lim\limits_{x\to 1} \dfrac {\left(x^2+x \right)\left(f(x)+1 \right)-x^2-x+2}{x-1}=\lim\limits_{x\to 1} \left( \dfrac {\left(x^2+x\right)(f(x)+1)}{x-1}-x-2\right)=-5$.}
\end{ex}
%Cau 20
\begin{ex}%[DCHT Toán 11 - KNTT -Nguyễn Văn Hiệp]%[1K5GF-3]
Gọi $A$ là giới hạn của hàm số $f(x)=\dfrac {x+x^2+x^3+\cdots +x^{50}-50}{x-1}$ khi $x$ tiến đến 1. Tính giá trị của $A$.
	\choice
	{$A$ không tồn tại}
	{$A=1725$}
	{$A=1527$}
	{\True $A=1275$}
	\loigiai
	{Ta có \allowdisplaybreaks
		\begin{eqnarray*}
		\lim\limits_{x\to 1} f(x)&=&\lim\limits_{x\to 1} \dfrac {x+x^2+x^3+\cdots +x^{50}-50}{x-1}\\
		&=&\lim\limits_{x\to 1} \left[ 1+(x+1)+\left(x^2+x+1\right)+\cdots +\left(x^{49}+x^{48}+\cdots +1\right) \right]\\
		&=&1+2+3+\cdots +50=25(1+50)=1275.
		\end{eqnarray*}
		Vậy $A=\lim\limits_{x\to 1} f(x)=1275$.}
\end{ex}
\Closesolutionfile{ans}




