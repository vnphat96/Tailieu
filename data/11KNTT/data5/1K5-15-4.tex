

\begin{dang}{Tính tổng của dãy cấp số nhân lùi vô hạn}
		\begin{dn}
		Cấp số nhân vô hạn $u_1, u_1q,...,u_1q^{n-1},...$ có công bội $q$ thỏa mãn $|q|<1$ được gọi là cấp số nhân lùi vô hạn. 
		Tổng của cấp số nhân lùi vô hạn đã cho là $$S=u_1+u_1q+u_1q^2+...=\dfrac{u_1}{1-q}.$$
	\end{dn}
\end{dang}
\subsection*{Ví dụ mẫu}
\begin{vd}%[1C3B1-5]
	Cho cấp số nhân $(u_n)$, với $u_1=1$ và công bội $q=\dfrac{1}{2}$.
	\begin{enumEX}{1}
		\item So sánh $\left|q\right|$ với $1$.
		\item Tính $S_n=u_1+u_2+\cdots+u_n$ từ đó hãy tính $\lim \limits{n \to +\infty}S_n$.
	\end{enumEX}
	\loigiai{
		\begin{enumerate}
			\item Ta có $\left|q\right|=\left|\dfrac{1}{2}\right|=\dfrac{1}{2}<1$.
			\item Ta có $S_n=\dfrac{u_1\left(1-q^n\right)}{1-q}=\dfrac{1\cdot\left[1-\left(\dfrac{1}{2}\right)^n\right]}{1-\dfrac{1}{2}}=2\cdot\left(1-\dfrac{1}{2^n}\right)=2-\dfrac{1}{2^{n-1}}$.\\
			Khi đó $\lim \limits{n \to +\infty}S_n=\lim \limits{n \to +\infty}\left(2-\dfrac{1}{2^{n-1}}\right)=2$.
		\end{enumerate}
	}
\end{vd}
\begin{vd}%[1C3Y1-5]
	Tính tổng $T=1+\dfrac{1}{3}+\dfrac{1}{3^2}+\ldots+\dfrac{1}{3^n}+\ldots$
	\loigiai{
		Các số hạn của tổng lập thành câp số nhân $(u_n)$, có $u_1=1$, $q=\dfrac{1}{3}$ nên\\ $$T=1+\dfrac{1}{3}+\dfrac{1}{3^2}+\ldots+\dfrac{1}{3^n}+\ldots=\dfrac{1}{1-\dfrac{1}{3}}=\dfrac{2}{3}\cdot$$
	}
\end{vd}
\begin{vd}
	Tính tổng $S=1-\dfrac{1}{2}+\dfrac{1}{4}-\dfrac{1}{8}+\ldots+\left(-\dfrac{1}{2}\right)^{n-1}+\ldots$.
	\loigiai{
		Đây là tổng của cấp số nhân lùi vô hạng với $u_{1}=1$ và $q=-\dfrac{1}{2}$. Do đó
		$
		S=\dfrac{u_{1}}{1-q}=\dfrac{1}{1-\left(-\dfrac{1}{2}\right)}=\dfrac{2}{3}.
		$	
	}
\end{vd}

\begin{vd}%[1C3B1-5]
	Biểu diễn số thập phân vô hạn tuần hoàn $2{,}222 \ldots$ dưới dạng phân số.
	\loigiai{
		Ta có $2{,}222 \ldots=2+0{,}2+0{,}02+0{,}002+\ldots=2+2 \cdot 10^{-1}+2 \cdot 10^{-2}+2 \cdot 10^{-3}+\ldots$.\\
		Đây là tổng của cấp số nhân lùi vô hạn với $u_{1}=2, q=10^{-1}$ nên
		$$
		2{,}222 \ldots=\dfrac{u_{1}}{1-q}=\dfrac{2}{1-\dfrac{1}{10}}=\dfrac{20}{9}.
		$$
	}
\end{vd}
\begin{vd}%[1C3B1-5]
	Biểu diễn số thập phân vô hạn tuần hoàn $0,(3)$ dưới dạng phân số.
	\loigiai{
		Ta có  $0,(3)=\dfrac{3}{10}+\dfrac{3}{10^2}+\ldots+\dfrac{3}{10^n}+\ldots=\dfrac{\dfrac{3}{10}}{1-\dfrac{1}{10}}=\dfrac{1}{3}\cdot$
	}
\end{vd}
\subsection*{Bài tập rèn luyện}
\begin{bt}%[1C3B1-5]
	\begin{enumEX}{1}
		\item[] 
		\item Tính tổng cấp số nhân lùi vô hạn $(u_n)$ với $u_1=\dfrac{2}{3}, q=-\dfrac{1}{4}$.
		\item Biểu diễn số thập phân vô hạn tuần hoàn $1,(6)$ dưới dạng phân số.
	\end{enumEX}
	\loigiai{
		\begin{enumerate}
			\item Ta có  $S=\dfrac{u_1}{1-q}=\dfrac{\dfrac{2}{3}}{1-\left(-\dfrac{1}{4}\right)}=\dfrac{8}{15}$.
			\item Ta có  $1,(6)=1+0,(6)=1+\dfrac{6}{10}+\dfrac{6}{10^2}+\cdots+\dfrac{6}{10^n}+\cdots=1+\dfrac{\dfrac{6}{10}}{1-\dfrac{1}{10}}=\dfrac{5}{3}$.
		\end{enumerate}
	}
\end{bt}
\begin{bt}%[1T3B1-6]
	Tính tổng của các cấp số nhân lùi vô hạn sau
	\begin{listEX}[2]
		\item $-\dfrac{1}{2}+\dfrac{1}{4}-\dfrac{1}{8}+\cdots+\left(-\dfrac{1}{2}\right)^n+\cdots$.
		\item $\dfrac{1}{4}+\dfrac{1}{16}+\dfrac{1}{64}+\cdots+\left(\dfrac{1}{4}\right)^n+\cdots$.
	\end{listEX}
	\loigiai{
		\begin{enumerate}
			\item Tổng trên là tổng của cấp số nhân lùi vô hạn có số hạng đầu $u_1=-\dfrac{1}{2}$ và công bội $q=-\dfrac{1}{2}$ nên 
			\[-\dfrac{1}{2}+\dfrac{1}{4}-\dfrac{1}{8}+\cdots+\left(-\dfrac{1}{2}\right)^n+\cdots=\dfrac{-\dfrac{1}{2}}{1-\left(-\dfrac{1}{2}\right)}=-\dfrac{1}{3}. \]
			\item Tổng trên là tổng của cấp số nhân lùi vô hạn có số hạng đầu $u_1=\dfrac{1}{4}$ và công bội $q=\dfrac{1}{4}$ nên 
			\[\dfrac{1}{4}+\dfrac{1}{16}+\dfrac{1}{64}+\cdots+\left(\dfrac{1}{4}\right)^n+\cdots=\dfrac{\dfrac{1}{4}}{1-\dfrac{1}{4}}=\dfrac{1}{3}. \]
		\end{enumerate}
	}
\end{bt}

\begin{bt}%[1T3B1-5]
	Tính tổng của cấp số nhân lùi vô hạn: $1-\dfrac{1}{4}+\dfrac{1}{16}-\dfrac{1}{64}+\cdots+\left(-\dfrac{1}{4}\right)^n+\cdots$.
	\loigiai{
		Tổng trên là tổng của cấp số nhân lùi vô hạn có số hạng đầu $u_1=1$ và công bội $q=-\dfrac{1}{4}$ nên
		\[ 1-\dfrac{1}{4}+\dfrac{1}{16}-\dfrac{1}{64}+\cdots+\left(-\dfrac{1}{4}\right)^n+\cdots=\dfrac{1}{1-\left(-\dfrac{1}{4}\right)}=\dfrac{4}{5}.\]
	}
\end{bt}

\begin{bt}%[1T3B1-5]
	Biết rằng có thể coi số thập phân vô hạn tuần hoàn $0{,}666 \ldots$ là tổng của một cấp số nhân lùi vô hạn:
	\[ 0{,}666 \ldots=0{,}6+0{,}06+0{,}006+\cdots=0{,}6+0{,}6 \cdot \dfrac{1}{10}+0{,}6 \cdot \dfrac{1}{10^2}+\cdots.
	\]
	Hãy viết $0{,}666 \ldots$ dưới dạng phân số.
	\loigiai{
		Số $0{,}666 \ldots$ là tổng của cấp số nhân lùi vô hạn có số hạng đầu bằng $0{,}6$ và công bội bằng $\dfrac{1}{10}$.\\
		Do đó $0{,}666\ldots=\dfrac{0{,}6}{1-\dfrac{1}{10}}=\dfrac{6}{9}=\dfrac{2}{3}$.
	}
\end{bt}

\begin{bt}%[1T3B1-5]
	Tính tổng của cấp số nhân lùi vô hạn: $1+\dfrac{1}{3}+\left(\dfrac{1}{3}\right)^2+\cdots+\left(\dfrac{1}{3}\right)^n+\cdots$.	
	\loigiai{
		Tổng trên là tổng của cấp số nhân lùi vô hạn có số hạng đầu $u_1=1$ và công bội $q=\dfrac{1}{3}$ nên
		\[ 1+\dfrac{1}{3}+\left(\dfrac{1}{3}\right)^2+\cdots+\left(\dfrac{1}{3}\right)^n+\cdots=\dfrac{1}{1-\dfrac{1}{3}}= \dfrac{3}{2}.\]
	}
\end{bt}
\subsection*{Bài tập trắc nghiệm}
%\paragraph{Câu hỏi trắc nghiệm}
\Opensolutionfile{ans}[ans/ans-1K5-1-Dang4]
\begin{ex}%[1D4B1-5]
	Cho cấp số nhân $u_1,u_2,\ldots$ với công bội $q$ thỏa điều kiện $|q|<1$. Lúc đó, ta nói cấp số nhân đã cho là lùi vô hạn. Tổng của cấp số nhân đã cho là $S=u_1+u_2+u_3+\cdots +u_n+\cdots$ bằng 
	\choice
	{$\dfrac{u_1}{q-1}$}
	{$\dfrac{u_1\left(q^n-1\right)}{q-1}$}
	{$\dfrac{u_1}{1+q}$}
	{\True $\dfrac{u_1}{1-q}$}
	\loigiai{
		Theo định nghĩa cấp số nhân lùi vô hạn ta chứng minh được.\\
		$S=u_1+u_2+u_3+\cdots +u_n+\cdots =u_1+u_1q^1+u_1q^2+\cdots +u_1q^{n-1}+\cdots =\dfrac{u_1}{1-q}$.}
\end{ex}
\begin{ex}%[1D4B1-5]
	Gọi $S=\dfrac{1}{3}-\dfrac{1}{9}+\cdots +\dfrac{(-1)^{n+1}}{3^n}$. Khi đó, $\lim \limits{n \to +\infty}S$ bằng 
	\choice
	{$\dfrac{3}{4}$}
	{\True $\dfrac{1}{4}$}
	{$\dfrac{1}{2}$}
	{$1$}
	\loigiai{
		Ta có 
		\allowdisplaybreaks
		\begin{eqnarray*}		
			&&S=\dfrac{1}{3}-\dfrac{1}{9}+\cdots +\dfrac{(-1)^{n+1}}{3^n} \\
			&\Leftrightarrow& S=-\left(-\dfrac{1}{3}+\dfrac{1}{9}+\cdots +\dfrac{(-1)^n}{3^n}\right) \\
			&\Leftrightarrow& S=\dfrac{1}{3}\cdot\dfrac{1-\left(\dfrac{-1}{3}\right)^n}{1-\dfrac{-1}{3}}\\
			&\Leftrightarrow& S=\dfrac{1}{4}\cdot\left(1-\left(\dfrac{-1}{3}\right)^n\right).
		\end{eqnarray*}
		Suy ra $\lim \limits{n \to +\infty}S=\lim \limits{n \to +\infty}\dfrac{1}{4}\cdot\left(1-\left(\dfrac{-1}{3}\right)^n\right)=\dfrac{1}{4}$.	
	}
\end{ex}
\begin{ex}%[1D4B1-5]
	Tổng $S=\dfrac{1}{3}+\dfrac{1}{3^2}+\cdots +\dfrac{1}{3^n}+\cdots$ có giá trị là 
	\choice
	{$\dfrac{1}{3}$}
	{\True $\dfrac{1}{2}$}
	{$\dfrac{1}{9}$}
	{$\dfrac{1}{4}$}
	\loigiai{
		Ta có $S=\dfrac{1}{3}+\dfrac{1}{3^2}+\cdots +\dfrac{1}{3^n}+\cdots=\dfrac{1}{3}\cdot\dfrac{1}{1-\frac{1}{3}}=\dfrac{1}{2}$. 
	}
\end{ex}
\begin{ex}%[1D4B1-5]
	Tính $S=9+3+1+\dfrac{1}{3}+\dfrac{1}{9}+\cdots +\dfrac{1}{3^{n-3}}+\cdots$. Kết quả là 
	\choice
	{\True $\dfrac{27}{2}$}
	{$14$}
	{$16$}
	{$15$}
	\loigiai{
		Ta có $S=9+3+1+\dfrac{1}{3}+\dfrac{1}{9}+\cdots +\dfrac{1}{3^{n-3}}+\cdots=13+\dfrac{1}{3}\cdot\dfrac{1}{1-\frac{1}{3}}=13+\dfrac{1}{2}=\dfrac{27}{2}$. 
	}
\end{ex}
\begin{ex}%[1D4B1-5]
	Tổng các cấp số nhân vô hạn: $1,-\dfrac{1}{2},\dfrac{1}{4},-\dfrac{1}{8},\ldots,\dfrac{(-1)^{n+1}}{2^{n-1}},\ldots$ là
	\choice
	{$\dfrac{3}{2}$}
	{\True $\dfrac{2}{3}$}
	{$-\dfrac{2}{3}$}
	{$2$}
	\loigiai{
		Ta có $S=1-\dfrac{1}{2}+\dfrac{1}{4}-\dfrac{1}{8}+\cdots +\dfrac{(-1)^{n+1}}{2^{n-1}}+\cdots=1-\dfrac{1}{2}\cdot\dfrac{1}{1+\frac{1}{2}}=1-\dfrac{1}{3}=\dfrac{2}{3}$.
	}
\end{ex}
\begin{ex}%[1D4B1-5]
	Gọi $S=1+\dfrac{2}{3}+\dfrac{4}{9}+\cdots +\dfrac{2^n}{3^n}+\cdots$. Giá trị của $S$ bằng
	\choice
	{\True $3$}
	{$5$}
	{$6$}
	{$4$}
	\loigiai{
		Ta có $S=1+\dfrac{2}{3}+\dfrac{4}{9}+\cdots +\dfrac{2^n}{3^n}+\cdots=1+\dfrac{2}{3}\cdot\dfrac{1}{1-\frac{2}{3}}=1+2=3$.
	}
\end{ex}
\begin{ex}%[1D4B1-5]%
	Số thập phân vô hạn tuần hoàn $0{,}233333\ldots$ biểu diễn dưới dạng số là 
	\choice
	{$\dfrac{1}{23}$}
	{$\dfrac{2333}{10000}$}
	{$\dfrac{23333}{10^5}$}
	{\True $\dfrac{7}{30}$}
	\loigiai{
		$0{,}233333\ldots=0{,}2+3\left(\dfrac{1}{10^2}+\dfrac{1}{10^3}+\ldots\right)=0{,}2+3\cdot\dfrac{1}{100}\cdot\dfrac{1}{1-\frac{1}{10}}=\dfrac{1}{5}+\dfrac{1}{30}=\dfrac{7}{30}$.
	}
\end{ex}
\begin{ex}%[1D4B1-5]%
	Số thập phân vô hạn tuần hoàn $0{,}212121\ldots$ biểu diễn dưới dạng phân số là 
	\choice
	{$\dfrac{2121}{10^4}$}
	{$\dfrac{1}{21}$}
	{\True $\dfrac{7}{33}$}
	{$\dfrac{212121}{10^6}$}
	\loigiai{
		$0{,}212121\ldots=21\left(\dfrac{1}{10^2}+\dfrac{1}{10^4}+\ldots\right)=21\cdot\dfrac{1}{10^2}\cdot\dfrac{1}{1-\frac{1}{100}}=\dfrac{7}{33}$.
	}
\end{ex}

\begin{ex}%[1D4B1-5]
	Số thập phân vô hạn tuần hoàn $0{,}271414\ldots$ được biểu diễn bằng phân số: 
	\choice
	{$\dfrac{2714}{9900}$}
	{$\dfrac{2617}{9900}$}
	{$\dfrac{2786}{9900}$}
	{\True $\dfrac{2687}{9900}$}
	\loigiai{
		$0{,}271414\cdots=0{,}27+14\left(\dfrac{1}{10^4}+\dfrac{1}{10^6}\ldots\right)=0{,}27+14\cdot\dfrac{1}{10^4}\cdot\dfrac{1}{1-\frac{1}{100}}=\dfrac{27}{100}+\dfrac{7}{4950}=\dfrac{2687}{9900}$.
	}
\end{ex}

\begin{ex}%[1D4B1-5]
	Tổng của cấp số nhân lùi vô hạn: $-\dfrac{1}{2},\dfrac{1}{4},\dfrac{1}{8},\ldots,\dfrac{(-1)^n}{2^n},\ldots$ là 
	\choice
	{\True $-\dfrac{1}{3}$}
	{$-\dfrac{1}{4}$}
	{$-1$}
	{$\dfrac{1}{2}$}
	\loigiai{
		Từ $-\dfrac{1}{2},\dfrac{1}{4},\dfrac{1}{8},\ldots,\dfrac{(-1)^n}{2^n},\ldots$ có $u_1=-\dfrac{1}{2}$ và $q=-\dfrac{1}{2}$.\\
		Có $S=-\dfrac{1}{2}+\dfrac{1}{4}+\dfrac{1}{8}+\cdots +\dfrac{(-1)^n}{2^n}+\cdots =\dfrac{\left(-\frac{1}{2}\right)}{1-\left(-\frac{1}{2}\right)}=-\dfrac{1}{3}$.}
\end{ex}
\begin{ex}%[1D4B1-5]
	Số thập phân vô hạn tuần hoàn $0{,}511111\cdots$ được biểu diễn bởi phân số
	\choice
	{$\dfrac{47}{90}$}
	{\True $\dfrac{46}{90}$}
	{$\dfrac{6}{11}$}
	{$\dfrac{43}{90}$}
	\loigiai{
		Ta có\\
		$0{,}511111\cdots=0{,}5+\dfrac{1}{10^2}+\dfrac{1}{10^3}\ldots=\dfrac{1}{2}+\dfrac{1}{10^2}\cdot\dfrac{1}{1-\frac{1}{10}}=\dfrac{1}{2}+\dfrac{1}{90}=\dfrac{23}{45}$.
	}
\end{ex}
\begin{ex}%[1D4B1-5]
	Tổng của cấp số nhân vô hạn $\dfrac{1}{2}$, $-\dfrac{1}{4}$, $\dfrac{1}{8},\ldots\dfrac{(-1)^{n+1}}{2^n},\ldots$ là
	\choice
	{$-\dfrac{2}{3}$}
	{$1$}
	{$-\dfrac{1}{3}$}
	{\True $\dfrac{1}{3}$}
	\loigiai{
		Ta có $S=\dfrac{1}{2}-\dfrac{1}{4}+\dfrac{1}{8}+\cdots +\dfrac{(-1)^{n+1}}{2^n}+\cdots =\dfrac{1}{2}\cdot\dfrac{1}{1+\frac{1}{2}}=\dfrac{1}{3}$.}
\end{ex}
\begin{ex}%[1D4B1-5]
	Tổng của cấp số nhân lùi vô hạn $\dfrac{1}{2},-\dfrac{1}{6},\dfrac{1}{18},\ldots,\dfrac{(-1)^{n+1}}{2\cdot 3^{n-1}},\ldots$ là
	\choice
	{$\dfrac{3}{4}$}
	{$\dfrac{8}{3}$}
	{$\dfrac{2}{3}$}
	{\True $\dfrac{3}{8}$}
	\loigiai{
		Cấp số nhân có $u_1=\dfrac{1}{2}, q=-\dfrac{1}{3}$. Do đó tổng cần tìm là
		$$S=\dfrac{u_1}{1-q}=\dfrac{\dfrac{1}{2}}{1+\frac{1}{3}}=\dfrac{1}{2}\cdot\dfrac{3}{4}=\dfrac{3}{8}.$$}
\end{ex}
\begin{ex}%[1D4B1-5]
	Tổng của cấp số nhân lùi vô hạn $\dfrac{1}{3},-\dfrac{1}{9},\dfrac{1}{27},\ldots\cdot,\dfrac{(-1)^{n+1}}{3^n},\ldots$ là
	\choice
	{$4$}
	{$\dfrac{1}{2}$}
	{$\dfrac{3}{4}$}
	{\True $\dfrac{1}{4}$}
	\loigiai{
		Cấp số nhân có $u_1=\dfrac{1}{3}, q=-\dfrac{1}{3}$. Do đó tổng cần tìm là\\
		$$S=\dfrac{u_1}{1-q}=\dfrac{\dfrac{1}{3}}{1+\frac{1}{3}}=\dfrac{1}{3}\cdot\dfrac{3}{4}=\dfrac{1}{4}.$$}
\end{ex}
\begin{ex}%[1D4B1-5]
	Số thập phân vô hạn tuần hoàn $0{,}17232323\ldots$ được biểu diễn bởi phân số?
	\choice
	{$\dfrac{1706}{9900}$}
	{$\dfrac{153}{990}$}
	{$\dfrac{164}{990}$}
	{\True $\dfrac{853}{4950}$}
	\loigiai{
		$0{,}17232323\ldots=0{,}17+23\left(\dfrac{1}{10^4}+\dfrac{1}{10^6}+\ldots\right)=\dfrac{17}{100}+23\cdot\dfrac{1}{10^4}\cdot\dfrac{1}{1-\frac{1}{100}}=\dfrac{17}{100}+\dfrac{23}{9900}=\dfrac{853}{4950}$.
	}
\end{ex}
\Closesolutionfile{ans}

