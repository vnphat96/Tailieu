
\begin{dang}{Nguyên lý kẹp}
		Để tìm giới hạn của dãy số theo nguyên lý kẹp ta cần nhớ:
	\begin{itemize}
		\item Cho hai dãy số $(u_n)$ và $(v_n)$. Nếu $|u_n| \leq v_n$ với mọi $n$ và $\lim \limits_{n \to +\infty}v_n = 0$ thì $\lim \limits_{n \to +\infty}u_n =0$.
		\item Cho ba dãy số $(u_n)$, $(v_n)$ và $(w_n)$. Nếu $ u_n \leq v_n \leq w_n$ với mọi $n$ và $\lim \limits_{n \to +\infty}u_n = \lim \limits_{n \to +\infty}w_n = L$ thì $\lim \limits_{n \to +\infty}v_n = L$.
	\end{itemize}
\end{dang}
\subsubsection{Ví dụ minh hoạ}
%bai1
\begin{vd}%[1D4B1-2]
	Chứng minh rằng các dãy số với số hạng tổng quát sau đây có giới hạn $0$.
	\begin{enumEX}[a)]{2}
		\item $u_n=\dfrac{(-1)^n}{3n+2}$.
		\item $u_n=\dfrac{n\sin 2n}{n^3+2}$.
		\item $u_n=\dfrac{(-1)^n\cos n}{\sqrt{n}}$.
		\item $u_n=\dfrac{3\sin n-4\cos n}{2n^2+1}$.
	\end{enumEX}
	\loigiai{
		\begin{enumerate}[a)]
			\item Ta có: $0 \leqslant \left|u_n\right|= \dfrac{1}{3n+2}<\dfrac{1}{3n}<\dfrac{1}{n}$, $\forall n\in \mathbb{N^*}$.\\
			Mà $\lim \limits_{n \to +\infty}\dfrac{1}{n}=0$ nên suy ra $\lim \limits_{n \to +\infty}\dfrac{(-1)^n}{3n+2}=0$.
			\item Ta có $0\leqslant \left|u_n\right|=\dfrac{n\left|\sin 2n\right|}{n^3+2}\leqslant \dfrac{n}{n^3+2}<\dfrac{n}{n^3}\leqslant \dfrac{1}{n^2}$, $\forall n\in \mathbb{N^*}$.\\
			Mà $\lim \limits_{n \to +\infty}\dfrac{1}{n^2}=0$ nên suy ra $\lim \limits_{n \to +\infty}\dfrac{n\sin 2n}{n^3+2}=0$.
			\item Ta có $0\leqslant \left|\dfrac{(-1)^n\cos n}{\sqrt{n}}\right|\leqslant \dfrac{1}{\sqrt{n}}$, $\forall n\in \mathbb{N^*}$.\\
			Mà $\lim \limits_{n \to +\infty}\dfrac{1}{\sqrt{n}}=0$ nên suy ra $\lim \limits_{n \to +\infty}\dfrac{(-1)^n\cos n}{\sqrt{n}}=0$.
			\item Theo bất đẳng thức Bunhiacopxki, ta có\\
			$\left|3\sin n-4\cos n\right|\leqslant \sqrt{(3^2+4^2)\left(\sin^2n+\cos^2n\right)}=5$.\\
			Do đó $0\leqslant \left|\dfrac{3\sin n-4\cos n}{2n^2+1}\right|\leqslant \dfrac{5}{2n^2+1}<\dfrac{5}{2n^2}$, $\forall n\in \mathbb{N^*}$.\\
			Mà $\lim \limits_{n \to +\infty}\dfrac{5}{2n^2+1}=0$ nên suy ra $\lim \limits_{n \to +\infty}\dfrac{3\sin n-4\cos n}{2n^2+1}=0$.
		\end{enumerate}
	}
\end{vd}
%bai2
\begin{vd}%[1D4K1-2]
	Chứng minh rằng các dãy số với số hạng tổng quát sau đây có giới hạn $0$. 
	\begin{enumEX}[a)]{2}
		\item $u_n=\sqrt{n^3+2}-\sqrt{n^3+1}$.
		\item $u_n=\dfrac{3^n\sin 2n+4^n}{2^n+4\cdot 5^n}$.
		\item $u_n=\dfrac{n+\sin 2n}{n^2+n}$.
		\item $\dfrac{n+\cos \dfrac{n\pi}{5}}{n\sqrt{n}+\sqrt{n}}$.
	\end{enumEX}
	\loigiai{
		\begin{enumerate}[a)]
			\item Ta có $u_n=\sqrt{n^3+2}-\sqrt{n^3+1}=\dfrac{n^3+2-(n^3+1)}{\sqrt{n^3+2}+\sqrt{n^3+1}}=\dfrac{1}{\sqrt{n^3+2}+\sqrt{n^3+1}}$.\\
			Do đó $0\leqslant \left|u_n\right|=\dfrac{1}{\sqrt{n^3+2}+\sqrt{n^3+1}}<\dfrac{2}{\sqrt{n^2}+\sqrt{n^2}}=\dfrac{2}{2n}=\dfrac{1}{n}$, $\forall n\in \mathbb{N^*}$.\\
			Mà $\lim \limits_{n \to +\infty}\dfrac{1}{n}=0$ nên suy ra $\lim \limits_{n \to +\infty}\left(\sqrt{n^3+2}-\sqrt{n^3+1}\right)=0$.
			\item Ta có $0\leqslant \left|\dfrac{3^n\sin 2n+4^n}{2^n+4.5^n}\right|\leqslant \dfrac{3^n\left|\sin 2n\right|+4^n}{2^n+4\cdot 5^n}\leqslant \dfrac{3^n+4^n}{2^n+4\cdot 5^n}=\dfrac{{\left(\dfrac{3}{5}\right)}^n+{\left(\dfrac{4}{5}\right)}^n}{{\left(\dfrac{2}{5}\right)}^n+4}$, $\forall n\in \mathbb{N^*}$.\\
			Mà $\lim \limits_{n \to +\infty}\left[{\left(\dfrac{3}{5}\right)}^n+{\left(\dfrac{4}{5}\right)}^n\right]=\lim \limits_{n \to +\infty}{\left(\dfrac{3}{5}\right)}^n+\lim \limits_{n \to +\infty}{\left(\dfrac{4}{5}\right)}^n=0$ và $$\lim \limits_{n \to +\infty}\left[{\left(\dfrac{2}{5}\right)}^n+4\right]=\lim \limits_{n \to +\infty}{\left(\dfrac{2}{5}\right)}^n+4=0+4=4$$ nên $\lim \limits_{n \to +\infty}\dfrac{3^n\sin 2n+4^n}{2^n+4\cdot 5^n}=0$.
			\item Ta có $0\leqslant \left|\dfrac{n+\sin 2n}{n^2+n}\right|\leqslant \dfrac{n+\left|\sin 2n\right|}{n^2+n}\leqslant \dfrac{n+1}{n^2+n}=\dfrac{1}{n}$, $\forall n\in \mathbb{N^*}$. \\
			Mà $\lim \limits_{n \to +\infty}\dfrac{1}{n}=0$ nên suy ra $\lim \limits_{n \to +\infty}\dfrac{n+\sin 2n}{n^2+n}=0$.
			\item Ta có $0\leqslant \left|\dfrac{n+\cos \dfrac{n\pi}{5}}{n\sqrt{n}+\sqrt{n}}\right|\leqslant \dfrac{n+\left|\cos \dfrac{n\pi}{5}\right|}{n\sqrt{n}+\sqrt{n}}\leqslant \dfrac{n+1}{n\sqrt{n}+\sqrt{n}}=\dfrac{1}{\sqrt{n}}$, $\forall n\in \mathbb{N^*}$. \\
			Mà $\lim \limits_{n \to +\infty}\dfrac{1}{\sqrt{n}}=0$ nên suy ra $\lim \limits_{n \to +\infty}\dfrac{n+\cos \dfrac{n\pi}{5}}{n\sqrt{n}+\sqrt{n}}=0$.
		\end{enumerate}
	}
\end{vd}
%bai3
\begin{vd}%[1D4B1-1]
	Cho dãy số $(u_n)$ với $u_n=\dfrac{n}{3^n}$.
	\begin{enumerate}[a)]
		\item Chứng minh rằng $\dfrac{u_{n+1}}{u_n}\leqslant \dfrac{2}{3}$ với mọi $n\in \mathbb{N^*}$.
		\item Bằng phương pháp quy nạp chứng minh rằng $0<u_n<{\left(\dfrac{2}{3}\right)}^n$ với mọi $n\in \mathbb{N^*}$.
		\item Dãy $(u_n)$ có giới hạn $0$.
	\end{enumerate}
	\loigiai{
		\begin{enumerate}[a)]
			\item Với mọi $n\in \mathbb{N}$, ta có $\dfrac{u_{n+1}}{u_n}=\dfrac{\dfrac{n+1}{3^{n+1}}}{\dfrac{n}{3^n}}=\dfrac{n+1}{n}\cdot \dfrac{1}{3}$. \\
			Mặt khác, $n+1\leqslant n+n\leqslant 2n$. Suy ra $\dfrac{n+1}{n}\leqslant 2$. \\
			Do đó $\dfrac{u_{n+1}}{u_n}\leqslant \dfrac{2}{3}$ với mọi $n\in \mathbb{N^*}$.
			\item Rõ ràng với mọi $n\in \mathbb{N^*}$, ta có $u_n>0$. Do đó ta chỉ cần chứng minh $u_n<{\left(\dfrac{2}{3}\right)}^n$.
			\begin{itemize}
				\item Với $n=1$, ta có $u_1=\dfrac{1}{3^1}=\dfrac{1}{3}<{\left(\dfrac{2}{3}\right)}^1$. Nghĩa là mệnh đề đúng với $n=1$. \\
				\item Giả sử mệnh đề đúng với $n=k\geqslant 1$, tức là $u_k<{\left(\dfrac{2}{3}\right)}^k$. \\
				\item Bây giờ ta cần chứng minh mệnh đề đúng với $n=k+1$, tức là cần chứng minh $u_{k+1}<{\left(\dfrac{2}{3}\right)}^{k+1}$. \\
				Theo chứng minh câu a) ta có $\dfrac{u_{k+1}}{u_k}\leqslant \dfrac{2}{3}$ suy ra $u_{k+1}\leqslant \dfrac{2}{3}\cdot u_k<\dfrac{2}{3}\cdot {\left(\dfrac{2}{3}\right)}^k={\left(\dfrac{2}{3}\right)}^{k+1}$ hay $u_{k+1}<{\left(\dfrac{2}{3}\right)}^{k+1}$. \\
				Nghĩa là mệnh đề cũng đúng với $n=k+1$. Vậy $0<u_n<{\left(\dfrac{2}{3}\right)}^n$ với mọi $n\in \mathbb{N^*}$.
			\end{itemize}
			\item Theo câu b), ta có $0<u_n<{\left(\dfrac{2}{3}\right)}^n$. Mà $\lim \limits_{n \to +\infty}{\left(\dfrac{2}{3}\right)}^n=0$. Do đó $\lim \limits_{n \to +\infty}u_n=0$.
		\end{enumerate}
	}
\end{vd}
%bai4
\begin{vd}%[1D4B1-2]
	Chứng minh rằng
	\begin{enumEX}[a)]{2}
		\item $\lim \limits_{n \to +\infty}\left(\dfrac{-n^3}{n^3+1}\right)=-1$.
		\item $\lim \limits_{n \to +\infty}\left(\dfrac{n^2+3n+2}{2n^2+n}\right)=\dfrac{1}{2}$.
	\end{enumEX}
	\loigiai{
		\begin{enumerate}[a)]
			\item Ta có $\lim \limits_{n \to +\infty}\left(\dfrac{-n^3}{n^3+1}-(-1)\right)=\lim \limits_{n \to +\infty}\left(\dfrac{1}{n^3+1}\right)$. \\
			Vì $0\leqslant \left|\dfrac{1}{n^3+1}\right|<\dfrac{1}{n^3}$, $\forall n\in \mathbb{N^*}$. Mà $\lim \limits_{n \to +\infty}\dfrac{1}{n^3}=0$ nên suy ra $\lim \limits_{n \to +\infty}\left(\dfrac{1}{n^3+1}\right)=0$. \\
			Do đó $\lim \limits_{n \to +\infty}\left(\dfrac{-n^3}{n^3+1}\right)=-1$.
			\item Ta có $\lim \limits_{n \to +\infty}\left(\dfrac{n^2+3n+2}{2n^2+n}-\dfrac{1}{2}\right)=\lim \limits_{n \to +\infty}\dfrac{5n+4}{2(2n^2+n)}$. \\
			Vì $0<\left|\dfrac{5n+4}{2(2n^2+n)}\right|<\dfrac{5n+5}{2n(n+1)}=\dfrac{5}{2}\cdot \dfrac{1}{n}$, $\forall n\in \mathbb{N^*}$. Mà $\lim \limits_{n \to +\infty}\left(\dfrac{5}{2}\cdot \dfrac{1}{n}\right)=\dfrac{5}{2}\cdot \lim \limits_{n \to +\infty}\dfrac{1}{n}=0$ nên suy ra $\lim \limits_{n \to +\infty}\dfrac{5n+4}{2(2n^2+n)}=0$. \\
			Do đó $\lim \limits_{n \to +\infty}\left(\dfrac{n^2+3n+2}{2n^2+n}\right)=\dfrac{1}{2}$.
		\end{enumerate}
	}
\end{vd}
%bai5
\begin{vd}%[1D4K1-2]
	Chứng minh rằng
	\begin{enumEX}[a)]{2}
		\item $\lim \limits_{n \to +\infty}\left(\dfrac{3\cdot 3^n-\sin 3n}{3^n}\right)=3$.
		\item $\lim \limits_{n \to +\infty}\left(\sqrt{n^2+n}-n\right)=\dfrac{1}{2}$.
	\end{enumEX}
	\loigiai{
		\begin{enumerate}[a)]
			\item Ta có $\lim \limits_{n \to +\infty}\left(\dfrac{3.3^n-\sin 3n}{3^n}-3\right)=\lim \limits_{n \to +\infty}\left(\dfrac{-\sin 3n}{3^n}\right)$. \\
			Vì $0\leqslant \left|\dfrac{-\sin 3n}{3^n}\right|=\dfrac{\left|-\sin 3n\right|}{3^n}\leqslant \dfrac{1}{3^n}={\left(\dfrac{1}{3}\right)}^n$, $\forall n\in \mathbb{N^*}$. Mà $\lim \limits_{n \to +\infty}{\left(\dfrac{1}{3}\right)}^n=0$ nên suy ra $\lim \limits_{n \to +\infty}\left(\dfrac{-\sin 3n}{3^n}\right)=0$. \\
			Do đó $\lim \limits_{n \to +\infty}\left(\dfrac{3.3^n-\sin 3n}{3^n}\right)=3$.
			\item Ta có $\lim \limits_{n \to +\infty}\left(\sqrt{n^2+n}-n-\dfrac{1}{2}\right)=\lim \limits_{n \to +\infty}\dfrac{2\sqrt{n^2+n}-(2n+1)}{2}=\lim \limits_{n \to +\infty}\dfrac{-1}{2\left(2\sqrt{n^2+n}+(2n+1)\right)}$. \\
			Vì $0\leqslant \left|\dfrac{-1}{2\left(2\sqrt{n^2+n}+(2n+1)\right)}\right| \leqslant \dfrac{1}{2\left(2\sqrt{n^2+n}+(2n+1)\right)}\leqslant \dfrac{1}{2\left(2\sqrt{n^2}+2n\right)}=\dfrac{1}{8}\cdot \dfrac{1}{n}$, $\forall n\in \mathbb{N^*}$. \\
			Mà $\lim \limits_{n \to +\infty}\dfrac{1}{8}\cdot \dfrac{1}{n}=\dfrac{1}{8}\lim \limits_{n \to +\infty}\dfrac{1}{n}=0$ nên suy ra $\lim \limits_{n \to +\infty}\left(\sqrt{n^2+n}-n-\dfrac{1}{2}\right)$. \\
			Do đó $\lim \limits_{n \to +\infty}\left(\sqrt{n^2+n}-n\right)=\dfrac{1}{2}$.
		\end{enumerate}
	}
\end{vd}
%Bài 6
\begin{vd}%[1D4K1-5]
	Tìm các giới hạn sau
	\begin{enumEX}[a)]{1}
		\item $\lim \limits_{n \to +\infty}\left(\dfrac{1}{\sqrt{4n^2+1}}+\dfrac{1}{\sqrt{4n^2+2}}+\cdots +\dfrac{1}{\sqrt{4n^2+n}}\right)$.
		\item $\lim \limits_{n \to +\infty}\dfrac{{1\cdot 3\cdot 5\cdot 7}\cdots (2n-1)}{{2\cdot 4\cdot 6}\cdots (2n)}$.
	\end{enumEX}
	\loigiai{
		\begin{enumerate}[a)]
			\item Ta có
			\begin{align*}
				\dfrac{1}{\sqrt{4n^2}}+\dfrac{1}{\sqrt{4n^2}}+\cdots +\dfrac{1}{\sqrt{4n^2}} &\leqslant \dfrac{1}{\sqrt{4n^2+1}}+\dfrac{1}{\sqrt{4n^2+2}}+\cdots +\dfrac{1}{\sqrt{4n^2+n}}\\
				&\leqslant \dfrac{1}{\sqrt{4n^2+n}}+\dfrac{1}{\sqrt{4n^2+n}}+\cdots +\dfrac{1}{\sqrt{4n^2+n}}.
			\end{align*}
			hay
			\begin{center}
				$\dfrac{n}{\sqrt{4n^2}}\leqslant \dfrac{1}{\sqrt{4n^2+1}}+\dfrac{1}{\sqrt{4n^2+2}}+\cdots +\dfrac{1}{\sqrt{4n^2+n}}\leqslant \dfrac{n}{\sqrt{4n^2+n}}$ với mọi $n\in \mathbb{N^*}$.
			\end{center}
			Mà $\lim \limits_{n \to +\infty}\dfrac{n}{\sqrt{4n^2}}=\lim \limits_{n \to +\infty}\dfrac{1}{2}=\dfrac{1}{2}$; $\lim \limits_{n \to +\infty}\dfrac{n}{\sqrt{4n^2+n}}=\lim \limits_{n \to +\infty}\dfrac{1}{\sqrt{4+\dfrac{1}{n}}}=\dfrac{1}{\sqrt{4+0}}=\dfrac{1}{2}$. \\
			Do đó $\lim \limits_{n \to +\infty}\left(\dfrac{1}{\sqrt{4n^2+1}}+\dfrac{1}{\sqrt{4n^2+2}}+\cdots +\dfrac{1}{\sqrt{4n^2+n}}\right)=\dfrac{1}{2}$.
			\item Ta có $u_n=\dfrac{{1\cdot 3\cdot 5\cdot 7}\cdots (2n-1)}{{2\cdot 4\cdot 6}\cdots (2n)}$, suy ra $$u_n^2=\dfrac{1^2\cdot 3^2\cdot 5^2\cdot 7^2\cdots (2n-1)^2}{2^2\cdot 4^2\cdot 6^2\cdots (2n)^2}=\dfrac{1\cdot 3}{2^2}\cdot \dfrac{3\cdot 5}{4^2}\cdots \dfrac{(2n-1)(2n+1)}{(2n)^2}\cdot \dfrac{1}{2n+1}<\dfrac{1}{2n+1}.$$
			(do $\dfrac{1\cdot 3}{2^2}\cdot \dfrac{3\cdot 5}{4^2}\cdots \dfrac{(2n-1)(2n+1)}{(2n)^2}<\dfrac{2^2}{2^2}\cdot \dfrac{4^2}{4^2}\cdots \dfrac{(2n)^2}{(2n)^2}=1$ ) \\
			Vậy ta có $0<u_n<\dfrac{1}{\sqrt{2n+1}}$, $\forall n\in \mathbb{N^*}$. Mà $\lim \limits_{n \to +\infty}\dfrac{1}{\sqrt{2n+1}}=0$ nên suy ra $$\lim \limits_{n \to +\infty}\dfrac{{1\cdot 3\cdot 5\cdot 7}\cdots (2n-1)}{{2\cdot 4\cdot 6}\cdots (2n)}=0.$$
		\end{enumerate}
	}
\end{vd}
\subsubsection{Bài tập rèn luyện}
\subsubsection{Bài tập trắc nghiệm}
\Opensolutionfile{ans}[ans/ans-1K5-1-Dang6]

%%==========Câu 1
\begin{ex}%[1D4B1-2]
	Giới hạn $\displaystyle\lim\dfrac{\sin n+1}{n}$ bằng
	\choice
	{$+\infty$}
	{$1$}
	{$-\infty$}
	{\True $0$}
	\loigiai{
		Với mọi $n>0$ thì $|\sin n+1|\leq 2$. Do đó, với mọi $n>0$, ta có
		$$0\leq \left|\dfrac{\sin n+1}{n}\right|\leq \dfrac{2}{n}.$$
		Từ đó $$0\leq \lim\limits\left|\dfrac{\sin n+1}{n}\right|\leq \lim\limits\dfrac{2}{n}=0\Rightarrow \lim\limits\left|\dfrac{\sin n+1}{n}\right|=0\Rightarrow \lim\limits\dfrac{\sin n+1}{n}=0.$$
	}
\end{ex}
%%==========Câu 2
\begin{ex}%[1D4B1-2]
	Giới hạn $\displaystyle\lim\dfrac{\sin n+1}{n}$ bằng
	\choice
	{$+\infty$}
	{$1$}
	{$-\infty$}
	{\True $0$}
	\loigiai{
		Với mọi $n>0$ thì $|\sin n+1|\leq 2$. Do đó, với mọi $n>0$, ta có
		$$0\leq \left|\dfrac{\sin n+1}{n}\right|\leq \dfrac{2}{n}.$$
		Từ đó $$0\leq \lim\limits\left|\dfrac{\sin n+1}{n}\right|\leq \lim\limits\dfrac{2}{n}=0\Rightarrow \lim\limits\left|\dfrac{\sin n+1}{n}\right|=0\Rightarrow \lim\limits\dfrac{\sin n+1}{n}=0.$$
	}
\end{ex}
%%==========Câu 3
\begin{ex}%[1D4B1-2]
	Giới hạn $\lim \limits_{n \to +\infty}\dfrac{\cos n}{n}$ bằng
	\choice
	{$1$}
	{\True $0$}
	{$-1$}
	{$+\infty$}
	\loigiai{
		Ta có: $\left| \dfrac{\cos n}{n} \right|\le \dfrac{1}{n}$ và $\lim \limits_{n \to +\infty}\dfrac{1}{n}=0$ nên $\lim \limits_{n \to +\infty}\dfrac{\cos n}{n}=0$.
	}
\end{ex}
%%==========Câu 4
\begin{ex}%[1D4B1-2]
	Tính $\lim \limits_{n \to +\infty}\dfrac{\sin n}{n^3+1}$.
	\choice
	{$1$}
	{\True $0$}
	{$-\infty$}
	{$+\infty$}
	\loigiai{
		Ta có
		$ \left|\dfrac{\sin n}{n^3+1}\right|\le \dfrac{1}{n^3+1}$ mà
		$\lim \limits_{n \to +\infty}\dfrac{1}{n^3+1}=\lim \limits_{n \to +\infty}\dfrac{1}{n^3\left(1+\dfrac{1}{n^3}\right)}=0$.\\
		Vậy $\lim \limits_{n \to +\infty}\dfrac{\sin n}{n^3+1}=0$
	}
\end{ex}
%%==========Câu 5
\begin{ex}%[1D4B1-2]
	Tính $\lim \limits_{n \to +\infty}\dfrac{\sin 2024n}{n}$.
	\choice
	{\True $0$}
	{$1$}
	{$+ \infty$}
	{$2024$}
	\loigiai{
		Ta có $-1 \leqslant \sin 2024n \leqslant 1  \Leftrightarrow - \dfrac{1}{n} \leqslant \dfrac{\sin 2024n}{n} \leqslant \dfrac{1}{n}$.\\
		Vì $\lim \limits_{n \to +\infty}\left( - \dfrac{1}{n} \right) = \lim \limits_{n \to +\infty} \dfrac{1}{n} = 0$ nên $\lim \limits_{n \to +\infty}\dfrac{\sin 2024n}{n} = 0$.
	}
\end{ex}
%%==========Câu 6
\begin{ex}%[1D4K1-2]
	Tính $I=\lim \limits_{n \to +\infty}\left(\dfrac{1}{\sqrt{n^2+n+1}}+\dfrac{1}{\sqrt{n^2+n+2}}+...+\dfrac{1}{\sqrt{n^2+2n}}\right)$.
	\choice
	{$I=+\infty$}
	{$I=3$}
	{$I=2$}
	{\True $I=1$}
	\loigiai{
		Ta có $\dfrac{1}{\sqrt{n^2+2n}}<\dfrac{1}{\sqrt{n^2+n+k}}<\dfrac{1}{\sqrt{n^2+n+1}},\forall k=2,3,...,n-1$.\\
		$\Rightarrow \dfrac{n}{\sqrt{n^2+2n}}<\dfrac{1}{\sqrt{n^2+n+1}}+\dfrac{1}{\sqrt{n^2+n+2}}+\cdots+\dfrac{1}{\sqrt{n^2+2n}}<\dfrac{n}{\sqrt{n^2+n+1}}$.\\
		Mà $\lim \limits_{n \to +\infty}\dfrac{n}{\sqrt{n^2+2n}}=\lim \limits_{n \to +\infty}\dfrac{1}{\sqrt{1+\dfrac{2}{n}}}=1$; $\lim \limits_{n \to +\infty}\dfrac{n}{\sqrt{n^2+n+1}}=\lim \limits_{n \to +\infty}\dfrac{1}{\sqrt{1+\dfrac{1}{n}+\dfrac{1}{n^2}}}=1$.\\
		Vậy $I=1$.
	}
\end{ex}
%%==========Câu 7
\begin{ex}%[1D4K1-2]
	Tính $T=\lim\dfrac{n\sin n-3n^2}{n^2}$.
	\choice
	{$T=+\infty$}
	{$T=-\infty$}
	{$T=1$}
	{\True $T=-3$}
	\loigiai{
		Ta có $T=\lim\dfrac{n\sin n-3n^2}{n^2}=\lim\left(\dfrac{\sin n}{n}-3\right)$.\\
		Do $-1\le \sin n\le 1$ suy ra $-\dfrac{1}{n}\le \dfrac{\sin n}{n}\le \dfrac{1}{n}$, mà $\lim\left(-\dfrac{1}{n}\right)=0$ và $\lim\dfrac{1}{n}=0$ nên $\lim\dfrac{\sin n}{n}=0$.\\
		Từ đó suy ra $T=\lim\left(\dfrac{\sin n}{n}-3\right)=-3$.
	}
\end{ex}
%%==========Câu 8
\begin{ex}%[1D4K1-2]
	Tính giá trị của $I=\lim\dfrac{n^3+n\cdot\sin^2n}{10000n^3-n+2}$.
	\choice
	{\True $I=0{,}0001$}
	{$I=\dfrac{1}{1000}$}
	{$I=0$}
	{$I=0{,}00001$}
	\loigiai{Ta có: $I=\lim\dfrac{n^3+n\cdot\sin^2n}{10000n^3-n+2}=\lim\dfrac{1+\dfrac{\sin^2 n}{n^2}}{10000-\dfrac{1}{n^2}+\dfrac{2}{n^3}}=\dfrac{1}{10000}=0{,}0001$.\\
		Chú ý rằng: $0\leq\dfrac{\sin^2n}{n^2}\leq\dfrac{1}{n^2}$. Mà $\lim\dfrac{1}{n^2}=0\Rightarrow\lim\dfrac{\sin^2n}{n^2}=0$.}
\end{ex}
%%==========Câu 9
\begin{ex}%[1D4G1-2]
	Tính $I=\lim \limits_{n \to +\infty}\left(\dfrac{1}{2}+\dfrac{3}{2^2}+\dfrac{5}{2^3}+\cdots +\dfrac{2n-1}{2^n}\right)$.
	\choice
	{\True $I=3$}
	{$I=0$}
	{$I=\dfrac{1}{2}$}
	{$I=+\infty $}
	\loigiai{
		Đặt $S_n=\dfrac{1}{2}+\dfrac{3}{2^2}+\dfrac{5}{2^3}+\cdots +\dfrac{2n-1}{2^n}$. \\
		Khi đó $\dfrac{1}{2}S_n=\dfrac{1}{2^2}+\dfrac{3}{2^3}+\dfrac{5}{2^4}+\cdots +\dfrac{2n-3}{2^n}+\dfrac{2n-1}{2^{n+1}}$. \\
		Trừ vế theo vế ta được \\
		$S_n-\dfrac{1}{2}S_n=\dfrac{1}{2}+\left(\dfrac{2}{2^2}+\dfrac{2}{2^3}+\cdots +\dfrac{2}{2^n}\right)-\dfrac{2n-1}{2^{n+1}}$. \\
		Từ đó $S_n=1+\left(1+\dfrac{1}{2}+\dfrac{1}{4}+\cdots +\dfrac{1}{2^{n-2}}\right)-\dfrac{2n-1}{2^n}=1+2\left[1-{\left(\dfrac{1}{2}\right)}^{n-1}\right]-\dfrac{2n-1}{2^n}$. \\
		Với mọi $n\geqslant 4$ ta có $2^n\geqslant n^2$. Thật vậy,
		\begin{itemize}
			\item Ta có $2^4 \geqslant 4^2$.
			\item Nếu $2^k \geqslant k^2~(k \geqslant 4)$ thì $2^{k+1}=2\cdot 2^k \geqslant 2\cdot k^2>k^2+(2k+1)=(k+1)^2$ (do $k \geqslant 4$).
		\end{itemize}
		Từ đó $0<\dfrac{2n-1}{2^n}\leqslant \dfrac{2n-1}{n^2}$. Mà $\lim \limits_{n \to +\infty}\dfrac{2n-1}{n^2}=0$ nên $\lim \limits_{n \to +\infty}\dfrac{2n-1}{2^n}=0$. \\
		Vậy $I=\lim \limits_{n \to +\infty}S_n=\lim \limits_{n \to +\infty}\left[1+2\left[1-{\left(\dfrac{1}{2}\right)}^{n-1}\right]-\dfrac{2n-1}{2^n}\right]=1+2=3$.}
\end{ex}
%%==========Câu 10
\begin{ex}%[1D4G1-2]
	Cho dãy số $(u_n)$ được xác định bởi $\heva{& u_1=3 \\ & 2(n+1)u_{n+1}=nu_n+n+2}$. Tính $\lim \limits_{n \to +\infty}u_n$.
	\choice
	{\True $\lim \limits_{n \to +\infty}u_n=1$}
	{$\lim \limits_{n \to +\infty}u_n=4$}
	{$\lim \limits_{n \to +\infty}u_n=3$}
	{$\lim \limits_{n \to +\infty}u_n=0$}
	\loigiai{
		Ta chứng minh $1\le u_{n+1}\le 1+\dfrac{1}{2n}$, $\forall n\ge 1$. Thật vậy
		\begin{itemize}
			\item $u_{n+1}\ge 1$, $\forall n\ge 1$ $(1)$.\\
			Với $n=1\Rightarrow u_2=\dfrac{3}{2}\ge 1\Rightarrow (1)$ đúng với $n=1$.\\
			Giả sử $(1)$ đúng với $n=k\ge 1$, tức là $u_{k+1}\ge 1$.\\
			Ta cần chứng minh $(1)$ đúng với $n=k+1$, tức là chứng minh $u_{k+2}\ge 1$. Thật vậy\\
			$u_{k+2}=\dfrac{(k+1) u_{k+1}+1}{2(k+2)}+\dfrac{1}{2}\ge \dfrac{k+2}{2(k+2)}+\dfrac{1}{2}=1$.
			\item $u_{n+1}\le 1+\dfrac{1}{2n}$, $\forall n\ge 1$ $(2)$.\\
			Với $n=1\Rightarrow u_2=\dfrac{3}{2}\le 1+\dfrac{3}{2}\Rightarrow (2)$ đúng với $n=1$.\\
			Giả sử $(2)$ đúng với $n=k\ge 1$, tức là $u_{k+1}\le 1+\dfrac{1}{2k}$.\\
			Ta cần chứng minh $(2)$ đúng với $n=k+1$, tức là chứng minh $u_{k+2}\le 1+\dfrac{1}{2(k+1)}$. Thật vậy
			$$u_{k+2}=\dfrac{(k+1)u_{k+1}+1}{2(k+2)}+\dfrac{1}{2}\le \dfrac{(k+1)\left(1+\dfrac{1}{2(k+1)}\right)}{2(k+2)}+\dfrac{1}{2}\le 1+\dfrac{1}{4(k+2)}\le 1+\dfrac{1}{2(k+1)}.$$
			Vậy $\lim \limits_{n \to +\infty}u_n=1$.
		\end{itemize}
	}
\end{ex}
%%==========Câu 11
\begin{ex}%[1D4G1-2]
	Cho dãy số $\left( u_n\right) $ thỏa mãn $\heva{&u_1=\dfrac{1}{3}\\&u_{n+1}=\dfrac{(n+1)u_n}{3n},\,\, \forall n\geq 1}$. Có bao nhiêu số nguyên dương $n$ thỏa mãn $u_n<\dfrac{1}{2020}.$
	\choice
	{$0$}
	{$9$}
	{\True vô số}
	{ $5$}
	\loigiai
	{
		\begin{itemize}
			\item Đặt $v_n=\dfrac{u_n}{n}$, ta có $v_{n+1}=\dfrac{v_n}{3}$.
			\item Do đó $v_n$ là cấp số nhân với công bội là $\dfrac{1}{3}$, mà $v_1=\dfrac{u_1}{1}=\dfrac{1}{3}$ nên
			$v_n=v_1\cdot\left(\dfrac{1}{3}\right)^{n-1}=\left(\dfrac{1}{3}\right)^n$.
			\item Từ đó $u_n=n\left(\dfrac{1}{3}\right)^n=\dfrac{n}{3^n}$.
			\item Bằng quy nạp, ta chứng minh được $3^n>n^2$, $\forall\, n\ge 1$. Khi đó $|u_n|=\dfrac{n}{3^n}<\dfrac{n}{n^2}=\dfrac{1}{n}$.\\
			Mà $\lim\dfrac{1}{n}=0\Rightarrow\lim \limits_{n \to +\infty}u_n=0$. Suy ra có vô số $n$ để $u_n<\dfrac{1}{2020}$.
		\end{itemize}
	}
\end{ex}
%%==========Câu 12
\begin{ex}%[1D4K1-2]
	Tìm giới hạn của $(u_n)$ với $u_n = \dfrac{1}{\sqrt{n^2 + 1}} + \dfrac{1}{\sqrt{n^2 + }} +  \cdots + \dfrac{1}{\sqrt{n^2 + n}}$.
	\choice
	{\True $1$}
	{$0$}
	{$+\infty$}
	{$-\infty$}
	\loigiai{Với mỗi số nguyên $k$ mà $1 \leq k \leq n$, ta có $\dfrac{1}{\sqrt{n^2 + n}} \leq \dfrac{1}{\sqrt{n^2 + k}} \leq \dfrac{1}{\sqrt{n^2 + 1}}$.\\
	Do đó $\dfrac{n}{\sqrt{n^2 + n}} \leq u_n \leq \dfrac{n}{\sqrt{n^2 + 1}}$ với mọi $n$.\\
	Mặt khác $\lim \limits_{n \to +\infty}\dfrac{n}{\sqrt{n^2 + n}} = \lim \limits_{n \to +\infty}\dfrac{n}{\sqrt{n^2 + 1}} = 1$.\\
	Do đó $\lim \limits_{n \to +\infty}u_n = 1$.
	}
\end{ex}
%Câu 13
\begin{ex}%[1D4K1-2]
	Tìm giới hạn của $(u_n)$ với $u_n = \dfrac{1}{\sqrt{n^2 + 1}} + \dfrac{1}{\sqrt{n^2 + }} +  \cdots + \dfrac{1}{\sqrt{n^2 + n}}$.
	\choice
	{\True $1$}
	{$0$}
	{$+\infty$}
	{$-\infty$}
	\loigiai{Với mỗi số nguyên $k$ mà $1 \leq k \leq n$, ta có $\dfrac{1}{\sqrt{n^2 + n}} \leq \dfrac{1}{\sqrt{n^2 + k}} \leq \dfrac{1}{\sqrt{n^2 + 1}}$.\\
		Do đó $\dfrac{n}{\sqrt{n^2 + n}} \leq u_n \leq \dfrac{n}{\sqrt{n^2 + 1}}$ với mọi $n$.\\
		Mặt khác $\lim \limits_{n \to +\infty}\dfrac{n}{\sqrt{n^2 + n}} = \lim \limits_{n \to +\infty}\dfrac{n}{\sqrt{n^2 + 1}} = 1$.\\
		Do đó $\lim \limits_{n \to +\infty}u_n = 1$.
	}
\end{ex}
%Câu 14
\begin{ex}%[1D4K1-2]
	Kết quả đúng của $\lim \limits_{n \to +\infty}\left(5 - \dfrac{n\cos 2n}{n^2 + 1} \right)$ là
	\choice
	{$4$}
	{\True $5$}
	{$–4$}
	{$\dfrac{1}{4}$}
	\loigiai{
		Với mọi $n \in \mathbb{N}$ ta có $-\dfrac{n}{n^2 + 1} \le \dfrac{n\cos 2n}{n^2 + 1} \le \dfrac{n}{n^2 + 1}$.\\
		Ta có $\lim \limits_{n \to +\infty}\left(-\dfrac{n}{n^2 + 1}\right) = \lim \limits_{n \to +\infty}\dfrac{-\dfrac{1}{n}}{1 + \dfrac{1}{n^2}} = 0$; $\lim \limits_{n \to +\infty}\dfrac{n}{n^2 + 1} = \lim \limits_{n \to +\infty}\dfrac{\dfrac{1}{n}}{1 + \dfrac{1}{n^2}} = 0$.\\
		Suy ra $\lim \limits_{n \to +\infty}\left(\dfrac{n\cos 2n}{n^2 + 1}\right) = 0 \Rightarrow \lim \limits_{n \to +\infty}\left(5 - \dfrac{n\cos 2n}{n^2 + 1}\right) = 5$.
	}
\end{ex}
%Câu 15
\begin{ex}%[1D4K1-2]
	Kết quả của $\lim \limits_{n \to +\infty}\left(n^2\sin \dfrac{n\pi }{5} - 2n^3\right)$ bằng
	\choice
	{\True $-\infty$}
	{$0$}
	{$+\infty$}
	{$-2$}
	\loigiai{
	Ta có $\lim \limits_{n \to +\infty}\left(n^2\sin \dfrac{n\pi}{5} - 2n^3\right) = \lim \limits_{n \to +\infty}n^3\left(\dfrac{1}{n}\sin \dfrac{n\pi}{5} - 2\right) = -\infty $.\\
	Vì $\sin \dfrac{n\pi}{5} \le 1 \Rightarrow \dfrac{1}{n} \sin \dfrac{n\pi }{5} \le \dfrac{1}{n}$.\\
	Mà $\lim \limits_{n \to +\infty}\dfrac{1}{n} = 0$ nên $\lim \limits_{n \to +\infty}\left(\dfrac{1}{n}\sin \dfrac{n\pi}{5} - 2\right) = -2$.\\
	Mặt khác $\lim \limits_{n \to +\infty}n^3 = +\infty$.\\
	Vậy $\lim \limits_{n \to +\infty}\left(n^2\sin \dfrac{n\pi }{5} - 2n^3\right) = -\infty$.
	}
\end{ex}
%Câu 16
\begin{ex}%[1D4K1-2]
	Tính $I = \lim \limits_{n \to +\infty}\left( \dfrac{1}{\sqrt{n^2 + n + 1}} + \dfrac{1}{\sqrt {n^2 + n + 2}} + \cdots + \dfrac{1}{\sqrt{n^2 + 2n}}\right)$.
	\choice
	{ $I = +\infty$}
	{ $I = 3$}
	{ $I = 2$}
	{\True $I = 1$}
	\loigiai{
	Ta có $\dfrac{1}{\sqrt{n^2 + 2n}} < \dfrac{1}{\sqrt{n^2 + n + k}} < \dfrac{1}{\sqrt{n^2 + n + 1}},\forall k = 2{,}3,\cdots ,n-1$.\\
	Suy ra $\dfrac{n}{\sqrt{n^2 + 2n}} < \dfrac{1}{\sqrt{n^2 + n + 1}} + \dfrac{1}{\sqrt{n^2 + n + 2}} + \cdots + \dfrac{1}{\sqrt{n^2 + 2n}} < \dfrac{n}{\sqrt{n^2 + n + 1}}$.\\
	Mà $\lim \limits_{n \to +\infty}\dfrac{n}{\sqrt{n^2 + 2n}} = \lim \limits_{n \to +\infty}\dfrac{1}{\sqrt {1 + \dfrac{2}{n}}} = 1$; $ \lim \limits_{n \to +\infty}\dfrac{n}{\sqrt{n^2 + n + 1}} = \lim \limits_{n \to +\infty}\dfrac{1}{\sqrt {1 + \dfrac{1}{n} + \dfrac{1}{n^2}}} = 1$.\\
	Vậy $I = 1$.
	}
\end{ex}

\Closesolutionfile{ans}
