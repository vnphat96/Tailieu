\subsection{PHÂN LOẠI VÀ PHƯƠNG PHÁP GIẢI TOÁN}
\setcounter{dang}{0}
\begin{dang}{Giới hạn của hàm số khi $x \to x_0$. Khử dạng vô định $\displaystyle\frac{0}{0}$ }
	Xét giới hạn $\lim\limits_{x\to x_0}\dfrac{f(x)}{g(x)}$.
	\begin{enumerate}[\iconCV]
		\item \indamm{Phương pháp giải:} Thay $x_0$ vào $\dfrac{f(x)}{g(x)}$ để kiểm tra, sẽ có một trong các trường hợp:
		\begin{itemize}
			\item [\ding{172}] Tử số $f(x_0)=a$ và mẫu số $g(x_0)=b \ne 0$,  ta suy ra luôn kết quả $$\lim\limits_{x\to x_0}\dfrac{f(x)}{g(x)}=\dfrac{f(x_0)}{g(x_0)}=\dfrac{a}{b}.$$
			\item [\ding{173}]  Cả tử số và mẫu số đều bằng 0 hay $f(x_0)=g(x_0)=0$, ta xem đây là dạng vô định $\displaystyle\frac{0}{0}$. Khử dạng vô định này bằng cách phân tích nhân tử $x-x_0$.\\
			Phân tích $f(x)=(x-x_0) \cdot f_1(x)$ và $g(x)=(x-x_0) \cdot g_1(x)$. Khi đó
			$$\lim\limits_{x\to x_0}\frac{f(x)}{g(x)}=\lim\limits_{x\to x_0}\frac{(x-x_0)f_1(x)}{(x-x_0)g_1(x)}=\lim\limits_{x\to x_0}\frac{f_1(x)}{g_1(x)} \quad (1)$$
			Ta tiếp tục tính giới hạn (1).
			\item [\ding{174}] Tử số $f(x_0) \ne 0$ và mẫu số $g(x_0)=0$. Ta áp dụng các định lý liên quan đến giới hạn vô cực để tìm kết quả.
		\end{itemize}
		\end{enumerate}
	\begin{note}
		Một số cách phân tích nhân tử thường dùng:
		\begin{itemize}
			\item [$\bullet$] Nếu $f(x)=ax^2+bx+c$ có hai nghiệm $x_1$, $x_2$ thì $f(x)=a(x-x_1)(x-x_2)$.
			\item [$\bullet$] Nếu $f(x)$ là một đa thức bậc ba, bậc bốn,...ta có thể dùng phương pháp chia đa thức.
			\item [$\bullet$] Nếu $f(x)$ là biểu thức chứa căn, ta dùng cách nhân lượng liên hợp.	
		\end{itemize}
	\end{note}
\end{dang}

\begin{vd}
	Tính các giới hạn sau
	\begin{tasks}(2)
		\task $\lim\limits_{x\to 2}\dfrac{x^2+2x}{4}$.
		\task $\lim\limits_{x\to 0}\dfrac{\sqrt{x+4}+1}{x^2+2}$.
	\end{tasks}
	
\end{vd}\dongcham{11}

\begin{vd}
	Tính các giới hạn sau:
	\begin{tasks}(3)
		\task $\lim\limits_{x\to -4}\dfrac{x^2+2x-8}{x^2+4x}$.
		\task $\lim\limits_{x\to \frac{1}{2}}\dfrac{2x^2-5x+2}{1-2x}$.
		\task $\lim\limits_{x\to 2}\dfrac{2x^2-5x+2}{x^2+x-6}$.
		\task $\lim\limits_{x\to 1} \dfrac{3x^3-5x^2+2}{3x^2-5x+2} $.
		\task $\lim\limits_{x\to 2} \dfrac{x^3+3x^2-9x-2}{x^3-x-6}$.
		\task $\lim\limits_{x\to -1} \dfrac{4x^2-3x-7}{x^3+1}$.
	\end{tasks}	
\end{vd}\dongcham{35}

\begin{vd}
	Tính các giới hạn sau
	\begin{tasks}(3)
		\task $\lim\limits_{x\to 1} \dfrac{\sqrt{x+3}-2}{2x-2} $.
		\task $\lim\limits_{x\to 2} \dfrac{x-2}{\sqrt{x+2}-2} $.
		\task $\lim\limits_{x\to -1} \dfrac{2-\sqrt{5-x^2}}{x+1} $.
		\task $\lim\limits_{x\to 0} \dfrac{1-\sqrt{4x+1}}{x^2+3x} $.
		\task $\lim\limits_{x\to 2} \dfrac{x^2-4}{\sqrt{x^2+3x-1}-3} $.
		\task $\lim\limits_{x\to - 1} \dfrac{x^3+1}{\sqrt{x+5}-2} $.
		\end{tasks}
\end{vd}\dongcham{35}

\begin{dang}{Giới hạn của hàm số khi $x \to \pm \infty$. Khử dạng vô định $\frac{\infty }{\infty };\infty -\infty ;0\cdot\infty$}
	\iconCV\indamm{Phương pháp giải:} Tương tự như bài toán giới hạn dãy số
\end{dang}

\begin{vd}
	Tính các giới hạn sau:
	\begin{tasks}(2)
		\task $\lim\limits_{x\to -\infty}\dfrac{5x-2}{3x+1}$. 	
		\task $\lim\limits_{x\to +\infty}\dfrac{2x^3-x+10}{x^3+3x-3}$. 	
		\task $\lim\limits_{x\to +\infty}\dfrac{x^{4}-x^{3}+3}{2x^{6}-7}$. 	
		\task $\lim\limits_{x\to +\infty}\dfrac{1}{x}\left(\dfrac{2x^2}{x+1}-1\right)$. 	
		\task $\lim\limits_{x\to +\infty}\dfrac{x^3-3x}{(x+1)(2x^2-3)}$. 	
		\task $\lim\limits_{x\to +\infty}\dfrac{(x+1)^{2}(2x+1)^{2}}{(2x^{3}+1)(x-2)^{3}}$. 
	\end{tasks}
\end{vd}\dongcham{30}


\begin{vd}
	Tính các giới hạn sau:
	\begin{tasks}(3)
		\task	 $\lim\limits_{x\to +\infty}\dfrac{\sqrt{x^2-x+5}}{5x-1}$. 	
		\task	 $\lim\limits_{x\to +\infty}\dfrac{\sqrt{x^2+x}+2x}{2x+3}$. 	
		\task	 $\lim\limits_{x\to -\infty}\dfrac{2x-3}{\sqrt{4x^2+2x}+x}$. 	
		\task	 $\lim\limits_{x\to -\infty}\dfrac{\sqrt{2x^{2}-7x+1}}{3\left| x \right|-7}$. 	
		\task	 $\lim\limits_{x\to -\infty}\sqrt[3]{\dfrac{x^{2}+2x}{8x^{2}-x+5}}$. 	
		\task	 $\lim\limits_{x\to -\infty}\dfrac{x+\sqrt{x^2+2}}{\sqrt[3]{8x^3+x^2+1}}$.
\end{tasks}
\end{vd}\dongcham{36}


\begin{vd}
	Tính các giới hạn sau:
	\begin{tasks}(3)
		\task $\lim\limits_{x\to +\infty}\left(\sqrt{x^2+x}-x \right) $. 	
		\task $\lim\limits_{x\to +\infty}\left(\sqrt{x^2+x}- \sqrt{x^2+2x}\right) $. 
		\task $\lim\limits_{x\to -\infty}x\left(\sqrt{x^2+1}+x\right)$.
	\end{tasks}
\end{vd}\dongcham{25}

\begin{vd}
	Tính các giới hạn sau:
	\begin{tasks}(2)
		\task $\lim\limits_{x\to +\infty}\left(\sqrt[3]{x^3+x}-x \right) $. 	
		\task $\lim\limits_{x\to +\infty}\left(\sqrt[3]{x^3+x}- \sqrt{x^2+x}\right) $. 		
	\end{tasks}
\end{vd}\dongcham{20}

 \begin{vd}
	Tính giới hạn của các hàm số sau:
	\begin{tasks}(2)
		\task $\lim\limits_{x\to +\infty}\left(2x^5-x^4+4x^3-3\right)$.
		\task $\lim\limits_{x\to -\infty}\left(2x^5-x^4+4x^3-3\right)$.
		\task $\lim\limits_{x\to +\infty}\left(-x^3-x^2+4x+2\right)$.
		\task $\lim\limits_{x\to -\infty}\left(-x^3-x^2+4x+2\right)$.
		\task $\lim\limits_{x\to +\infty}\left(\sqrt{x^2+x}+x \right) $. 	
		\task $\lim\limits_{x\to -\infty}\left(2x-\sqrt{x^2+x}\right) $. 	
	\end{tasks}
\end{vd}\dongcham{25}

 \begin{vd}
 	Tính các giới hạn sau:
 	\begin{tasks}(3)
 		\task $\lim\limits_{x\to +\infty}\dfrac{5x^2-2x+3}{x+1}$. 	
 		\task $\lim\limits_{x\to +\infty}\dfrac{2x^3-x+10}{x^2+3x-3}$. 	
 		\task $\lim\limits_{x\to -\infty}\dfrac{3x^{4}+5x^{2}+7}{x^{3}-15x}$.
 		\task $\lim\limits_{x\to +\infty}\dfrac{x^2+\sqrt{x+1}}{2x-7}$. 	
 		\task  $\lim\limits_{x\to +\infty}\dfrac{x+3}{\sqrt{x^2+x}-x}$. 	
 		\task  $\lim\limits_{x\to -\infty}\dfrac{x+3}{\sqrt{x^2+x}+x}$. 	 
 	\end{tasks}
 \end{vd}\dongcham{35}
 
\begin{dang}{Giới hạn một bên. Sự tồn tại giới hạn}
	Phương pháp tính $\lim\limits_{x\to x_0^-}f(x)$ và $\lim\limits_{x\to x_0^+}f(x)$ hoàn toàn tương tự như bài toán tính $\lim\limits_{x\to x_0} f(x)$.
	\begin{note}
		Lưu ý: $\lim\limits_{x\to x_0} f(x)=L$ khi và chỉ khi $\lim\limits_{x\to x_0^-}f(x)=\lim\limits_{x\to x_0^+}f(x)=L$.
	\end{note}
\end{dang}

\begin{vd}%[1T3B2-7]
	Cho hàm số $ f(x)=\heva{& -x^2 && \text{khi } x<1\\ & \quad x&& \text{khi }x\geq 1.} $
	Tìm các giới hạn $ \lim \limits_{x \to 1^+} f(x) $, $ \lim \limits_{x \to 1^-} f(x) $, $ \lim \limits_{x \to 1} f(x) $ (nếu có).
	\loigiai{
		\begin{enumerate}
			\item $ \lim \limits_{x \to 1^+} f(x) =\lim \limits_{x \to 1^+} x=1$.
			\item $ \lim \limits_{x \to 1^-} f(x) =\lim \limits_{x \to 1^-} (-x^2)=-1^2=-1$.\\
			Vì $ \lim \limits_{x \to 1^+} f(x)=1\neq -1=\lim \limits_{x \to 1^-} f(x) $ nên không tồn tại giới hạn $ \lim \limits_{x \to 1} f(x) $. 
		\end{enumerate}
	}
\end{vd}\dongcham{10}

\begin{vd}
	Tính giới hạn của các hàm số sau:
	\begin{tasks}(2)
		\task $\lim\limits_{x\to +\infty}\dfrac{3}{x^2-2x+6}$;
		\task $\lim\limits_{x\to 3^+}\dfrac{-x^2+5}{x-3}$;
		\task $\lim\limits_{x\to 3^-}\dfrac{2x^2+\sqrt{3-x}}{x-3}$;
		\task $\lim\limits_{x\to -2^+}\dfrac{|x^2-4|}{x+2}$.
	\end{tasks}
	\loigiai{
		\begin{enumerate}[a)]
			\item $I_1=\lim\limits_{x\to +\infty}\dfrac{3}{x^2-2x+6}=0$ vì $\lim\limits_{x\to +\infty}(x^2-2x+6)=+\infty$;
			\item Ta có $\lim\limits_{x\to 3^+}(-x^2+5)=-4<0,\ \lim\limits_{x\to 3^+}(x-3)=0$ và $x-3>0,\forall x>3$.\\ Do đó $I_2=\lim\limits_{x\to 3^+}\dfrac{-x^2+5}{x-3}=-\infty$.
			\item $I_3=\lim\limits_{x\to 3^-}\dfrac{2x^2+\sqrt{3-x}}{x-3}=-\infty$.
			\item Ta có $\lim\limits_{x\to -2^+}\dfrac{|x^2-4|}{x+2}=\lim\limits_{x\to -2^+}\dfrac{4-x^2}{x+2}=\lim\limits_{x\to -2^+}(2-x)=4$.
		\end{enumerate}
	}
\end{vd}\dongcham{17}


\begin{dang}{Vận dụng thực tiễn}
\end{dang}

\begin{vd}%[1C3B2-8]
	Một công ty sản xuất máy tính đã xác định được rằng, tính trung bình một nhân viên có thể lắp ráp được $N(t)=\dfrac{50 t}{t+4}\, (t \geq 0)$ bộ phận mỗi ngày sau $t$ ngày đào tạo. Tính $\lim\limits_{t \to+\infty} N(t)$ và cho biết ý nghĩa của kết quả.
	\loigiai{
		Ta có
		\allowdisplaybreaks
		\begin{eqnarray*}
			\lim\limits_{t \to+\infty} N(t)=\lim\limits_{t \to+\infty}\dfrac{50 t}{t+4}&=&\lim\limits_{t \to+\infty}\dfrac{50t}{t\left(1+\dfrac{4}{t}\right)}\\
			&=&\lim\limits_{t \to+\infty}\dfrac{50}{1+\dfrac{4}{t}}\\
			&=&\dfrac{\lim\limits_{t \to+\infty}50}{\lim\limits_{t \to+\infty}1+\lim\limits_{t \to+\infty}\dfrac{4}{t}}\\
			&=&\dfrac{50}{1+0}\\
			&=&50.
		\end{eqnarray*}
		Ý nghĩa của kết quả: năng suất lao động cao nhất trong một ngày của một nhân viên là $50$.
	}
\end{vd}\dongcham{15}

\begin{vd}%[Tex hóa SGK CTST,T12/22, TVN-001]%[1T3K2-8]
	Một cái hồ đang chứa $ 200 $ m$^3$ nước mặn với nồng độ muối $ 10 \mathrm{\;kg/m}^3$. Người ta ngọt hóa nước trong hồ bằng cách bơm nước ngọt vào hồ với vận tốc $ 2 $ m$^3$/phút.
	\begin{enumerate}
		\item Viết biểu thức $ C(t) $ biểu thị nồng độ muối trong hồ sau $ t $ phút kể từ khi bắt đầu bơm.
		\item Tìm giới hạn $ \lim \limits_{t \to +\infty} C(t) $ và giải thích ý nghĩa.
	\end{enumerate}
	\loigiai{
		\begin{enumerate}
			\item Sau thời gian $ t $ phút, số m$^3$ nước trong hồ là $200+2t  $ (m$^3$).\\
			Số kilôgam muối là $ 200\cdot 10=2000 $ (kg).\\
			Nồng độ muối của nước trong hồ sau $ t $ phút khi bắt đầu bơm là 
			$$C(t)=\dfrac{2000}{200+2t}=\dfrac{1000}{100+t} \;  (\mathrm{kg/m}^3).$$
			\item Khi $ t\to +\infty $, ta xét giới hạn
			$$\lim \limits_{t \to +\infty} C(t)=\lim \limits_{t \to +\infty} \dfrac{1000}{100+t}=0.$$
		\end{enumerate}
	}
\end{vd}\dongcham{15}

\begin{vd}%[1K5KF-8]%[Tex SGK toán 11 - Nguyễn Sĩ Đạt]
	Trong Thuyết tương đối của Einstein, khối lượng của vật chuyển động với vận tốc v cho bởi công thức
	$$m=\dfrac{m_0}{\sqrt{1-\dfrac{v^2}{c^2}}}.$$
	trong đó $m_0$ là khối lượng của vật khi nó đứng yên, $c$ là vận tốc ánh sáng. Chuyện gì xảy ra với khối lượng của vật khi vận tốc của vật gần với vận tốc ánh sáng?
	\loigiai{
		Từ công thức khối lượng
		$$
		m=\dfrac{m_{0}}{\sqrt{1-\dfrac{v^{2}}{c^{2}}}}
		$$
		ta thấy $m$ là một hàm số của $v$, với tập xác định là nửa khoảng $[0 ; c)$. Rõ ràng khi $v$ tiến gần tới vận tốc ánh sáng, tức là $v \rightarrow c^{-}$, ta có $\sqrt{1-\dfrac{v^{2}}{c^{2}}} \rightarrow 0$. Do đó $\lim \limits_{v \rightarrow c^{-}} m(v)=+\infty$, nghĩa là khối lượng $m$ của vật trở nên vô cùng lớn khi vận tốc của vật gần với vận tốc ánh sáng.
	}
\end{vd}\dongcham{15}

\subsection{BÀI TẬP TỰ LUYỆN}

\begin{bt}
	Cho hàm số $ f(x)=\heva{& 0 \quad \text{khi } x<0\\ & 1\quad \text{khi } x>0.} $
	\begin{listEX}[1]
		\item Tìm các giới hạn $ \lim \limits_{x \to 0^+} f(x)$ và $ \lim \limits_{x \to 0^-} f(x) $.
		\item Có tồn tại $ \lim \limits_{x \to 0} f(x)$?
	\end{listEX}
	\loigiai{
		\begin{enumerate}
			\item Giả sử $ (x_n) $ là  dãy số bất kì, $ x_n>0 $ và $ x_n \to 0$. Khi đó $ f(x_n)=1 $ nên $ \lim f(x_n)=\lim 1=1 $.\\
			Vậy $ \lim \limits_{x \to 0^+} f(x)=1$.\\
			Giả sử $ (x_n) $ là  dãy số bất kì, $ x_n<0 $ và $ x_n \to 0$. Khi đó $ f(x_n)=0 $ nên $ \lim f(x_n)=\lim 0= 0$.\\
			Vậy $ \lim \limits_{x \to 0^-} f(x)=0$.
			\item  Vì $ \lim \limits_{x \to 0^+} f(x) \neq \lim \limits_{x \to 0^-} f(x) $ nên không tồn tại $ \lim \limits_{x \to 0} f(x) $.
		\end{enumerate}
	}
\end{bt}\dongcham{12}

\begin{bt}
	Cho hàm số $ f(x)=\heva{& 1-2x && \text{khi } x\leq -1\\ & x^2+2&& \text{khi } x>-1.} $\\
	Tìm các giới hạn $ \lim \limits_{x \to {-1}^+} f(x)$ và $ \lim \limits_{x \to {-1}^-} f(x) $ và $ \lim \limits_{x \to -1} f(x) $ (nếu có).
	\loigiai{
		\begin{enumerate}[+]
			\item Giả sử $ (x_n) $ là  dãy số bất kì, $ x_n<-1 $ và $ x_n \to -1$. Khi đó $ \lim f(x_n)=\lim (1-2x_n)=3 $.\\
			Vậy $ \lim \limits_{x \to -1^-} f(x)=3$.\\
			Giả sử $ (x_n) $ là  dãy số bất kì, $ x_n>-1 $ và $ x_n \to -1$. Khi đó  $ \lim  f(x_n)=\lim (x_n^2+2)=3 $.\\
			Vậy $ \lim \limits_{x \to -1^+} f(x)=3$.
			\item  Vì $ \lim \limits_{x \to -1^+} f(x) = \lim \limits_{x \to -1^-} f(x) $ nên tồn tại $ \lim \limits_{x \to -1} f(x) $ và $ \lim \limits_{x \to -1} f(x)=3 $.
		\end{enumerate}
	}
\end{bt}\dongcham{11}

\begin{bt}
	Tính các giới hạn sau:
	\begin{tasks}(2)
		\task $\lim\limits_{x \to -1}\dfrac{4x^2-x-5}{7x^2+5x-2}$.
		\task $\lim\limits_{x \to -2}\dfrac{4-x^2}{x+2}$.
		\task $\lim\limits_{x \to 3}\dfrac{x^2+2x-15}{x-3}$.
		\task $\lim\limits_{x \to 2}\dfrac{2x^2-5x+2}{x^2-4}$.
	\end{tasks}
	\loigiai
	{
		\begin{enumerate}[a)]
			\item $\lim\limits_{x \to -1}\dfrac{4x^2-x-5}{7x^2+5x-2} = \lim\limits_{x\to -1}\dfrac{(x+1)(4x-5)}{(x+1)(7x-2)} = \lim\limits_{x\to -1}\dfrac{4x-5}{7x-2} = \dfrac{4 \cdot (-1)-5}{7 \cdot (-1)-2} = 1$.
			
			\item $\lim\limits_{x \to -2}\dfrac{4-x^2}{x+2} = \lim\limits_{x \to -2} \dfrac{(2-x)(2+x)}{x+2} = \lim\limits_{x \to -2} (2-x) = 4$.
			
			\item $\lim\limits_{x \to 3}\dfrac{x^2+2x-15}{x-3} = \lim\limits_{x \to 3} \dfrac{(x-3)(x+5)}{x-3} = \lim\limits_{x \to 3} (x + 5) = 8$.
			
			\item $\lim\limits_{x \to 2}\dfrac{2x^2-5x+2}{x^2-4} = \lim\limits_{x \to 2}\dfrac{(x-2)(2x-1)}{(x-2)(x+2)} = \lim\limits_{x \to 2}\dfrac{2x-1}{x+2} = \dfrac{3}{4}$.
			
		\end{enumerate}
	}
\end{bt}\dongcham{19}

\begin{bt}
	Tính các giới hạn sau
	\begin{tasks}(2)
		\task $\lim\limits_{x \to 0}\dfrac{\sqrt{1+2x} - 1}{2x}$.
		\task $\lim\limits_{x \to 2} \dfrac{x - \sqrt{3x-2}}{x^2 - 4}$.
		\task $\lim\limits_{x \to 0}\dfrac{\sqrt{1+x^2} - 1}{2x^3 - 3x^2}$.
		\task $\lim\limits_{x \to 1}\dfrac{\sqrt{2x+7} - x - 2}{x^3 - 4x + 3}$.
	\end{tasks}
	\loigiai{
		\begin{enumerate}[a)]
			\item $\lim\limits_{x \to 0}\dfrac{\sqrt{1+2x} - 1}{2x} = \lim\limits_{x \to 0}\dfrac{2x}{2x\left(\sqrt{1+2x}+1\right)} = \lim\limits_{x \to 0}\dfrac{1}{\sqrt{1+2x}+1} = \dfrac{1}{2}$.
			\item $\lim\limits_{x \to 2} \dfrac{x - \sqrt{3x-2}}{x^2 - 4} = \lim\limits_{x \to 2}\dfrac{x^2 - 3x + 2}{(x^2 - 4) \left(x + \sqrt{3x-2}\right)} = \lim\limits_{x \to 2}\dfrac{(x-2)(x-1)}{(x-2)(x+2)\left(x+\sqrt{3x-2}\right)}$\\
			$= \lim\limits_{x \to 2}\dfrac{x-1}{(x+2)\left(x + \sqrt{3x-2}\right)} = \dfrac{1}{16}$.
			\item $\lim\limits_{x \to 0}\dfrac{\sqrt{1+x^2} - 1}{2x^3 - 3x^2} = \lim\limits_{x \to 0}\dfrac{x^2}{(2x^3 - 3x^2) \left(\sqrt{1+x^2} + 1\right)} = \lim\limits_{x \to 0}\dfrac{1}{(2x - 3) \left(\sqrt{1+x^2} + 1\right)} = -\dfrac{1}{6}$.
			\item $\lim\limits_{x \to 1}\dfrac{\sqrt{2x+7} - x - 2}{x^3 - 4x + 3} = \lim\limits_{x \to 1}\dfrac{2x+7-(x+2)^2}{(x^3-4x+3)\left(\sqrt{2x+7}+x+2\right)} = \lim\limits_{x \to 1}\dfrac{-x^2 - 2x + 3}{(x^3-4x+3)\left(\sqrt{2x+7}+x+2\right)}$\\
			$= \lim\limits_{x \to 1}\dfrac{-(x-1)(x+3)}{(x-1)(x^2+x-3)\left(\sqrt{2x+7}+x+2\right)} = \lim\limits_{x \to 1}\dfrac{-(x+3)}{(x^2+x-3)\left(\sqrt{2x+7}+x+2\right)} = \dfrac{2}{3}$.
	\end{enumerate}}
\end{bt}\dongcham{33}

\begin{bt}%[1K5BF-7]%
	Tính các giới hạn một bên:
	\begin{tasks}(2)
		\task $\lim \limits_{x \rightarrow 1^{+}} \dfrac{x-2}{x-1}$.
		\task $\lim \limits_{x \rightarrow 4^{-}} \dfrac{x^{2}-x+1}{4-x}$.
	\end{tasks}
	\loigiai{
		\begin{enumerate}
			\item Ta có $\lim \limits_{x \rightarrow 1^{+}} (x-2)=-1<0$. \\
			Hơn nữa $\lim \limits_{x \rightarrow 1^{+}} (x-1)=0$, và $x-1>0$ khi $x>1$.\\	
			Áp dụng quy tắc tìm giới hạn của thương, ta được $\lim \limits_{x \rightarrow 1^{+}} \dfrac{x-2}{x-1}=-\infty$.
			\item Ta có $\lim \limits_{x \rightarrow 4^{-}} (x^{2}-x+1)=13>0$. \\
			Hơn nữa $\lim \limits_{x \rightarrow 4^{-}} (4-x)=0$, và $4-x>0$ khi $x<4$.\\	
			Áp dụng quy tắc tìm giới hạn của thương, ta được $\lim \limits_{x \rightarrow 4^{-}} \dfrac{x^{2}-x+1}{4-x}=+\infty$.
		\end{enumerate}	
	}
\end{bt}\dongcham{10}

\begin{bt}%[1K5KF-7]
	Cho hàm số $g(x)=\dfrac{x^{2}-5 x+6}{|x-2|}$. Tìm $\lim \limits_{x \rightarrow 2^{+}} g(x)$ và $\lim \limits_{x \rightarrow 2^{-}} g(x)$.
	\loigiai{
		Ta có $\lim \limits_{x \rightarrow 2^{+}} g(x)=\lim \limits_{x \rightarrow 2^{+}}\dfrac{(x-2)(x-3)}{x-2}\,\,(\text{do } x>2)=\lim \limits_{x \rightarrow 2^{+}} (x-3)=-1$.\\
		Tương tự $\lim \limits_{x \rightarrow 2^{-}} g(x)=\lim \limits_{x \rightarrow 2^{-}}-\dfrac{(x-2)(x-3)}{x-2}\,\,(\text{do } x<2)=-\lim \limits_{x \rightarrow 2^{-}} (x-3)=1$.
	}
\end{bt}\dongcham{10}

\begin{bt}%[1K5KF-5]
	Tính các giới hạn sau:
	\begin{tasks}(2)
		\task $\lim \limits_{x \rightarrow+\infty} \dfrac{1-2 x}{\sqrt{x^{2}+1}}$.
		\task $\lim \limits_{x \rightarrow+\infty}\left(\sqrt{x^{2}+x+2}-x\right)$.
	\end{tasks}
	\loigiai{
		\begin{enumerate}
			\item Ta có
			\allowdisplaybreaks
			\begin{eqnarray*}
				\lim \limits_{x \rightarrow+\infty} \dfrac{1-2 x}{\sqrt{x^{2}+1}}&=& -\lim \limits_{x \rightarrow+\infty}\sqrt{\dfrac{4x^2-4x+1}{x^2+1}}\\
				&=& -\lim \limits_{x \rightarrow+\infty}\sqrt{4-\dfrac{4x}{x^2+1}+\dfrac{1}{x^2+1}}\\
				&=& -\sqrt{4-\lim \limits_{x \rightarrow+\infty} \dfrac{4x}{x^2+1}+ \lim \limits_{x \rightarrow+\infty}\dfrac{1}{x^2+1}}\\
				&=& -\sqrt{4-\lim \limits_{x \rightarrow+\infty} \dfrac{\dfrac{4}{x}}{1+\dfrac{1}{x^2}}+ \lim \limits_{x \rightarrow+\infty}\dfrac{\dfrac{1}{x^2}}{1+\dfrac{1}{x^2}}}\\
				&=& -2.
			\end{eqnarray*}
			\item Ta có 
			\allowdisplaybreaks
			\begin{eqnarray*}
				\sqrt{x^{2}+x+2}-x&=& \dfrac{\left(\sqrt{x^{2}+x+2} \right)^2-x^2 }{\sqrt{x^{2}+x+2}+x}=\dfrac{x+2 }{\sqrt{x^{2}+x+2}+x}\\
				&=& \dfrac{x\cdot\left(1+\dfrac{2}{x} \right) }{x\cdot \left(\sqrt{1+\dfrac{1}{x}+\dfrac{2}{x^2}}+1 \right) }=\dfrac{1+\dfrac{2}{x}}{\sqrt{1+\dfrac{1}{x}+\dfrac{2}{x^2}}+1 }.
			\end{eqnarray*}
			Khi đó $\lim \limits_{x \rightarrow+\infty}\left(\sqrt{x^{2}+x+2}-x\right)=\lim \limits_{x \rightarrow+\infty} \dfrac{1+\dfrac{2}{x}}{\sqrt{1+\dfrac{1}{x}+\dfrac{2}{x^2}}+1 }=\dfrac{1}{1+1}=\dfrac{1}{2}$.
		\end{enumerate}	
	}
\end{bt}\dongcham{11}

\begin{bt}%[1T3K2-8]
	Trong hồ có chứa $ 6000 $ lít nước ngọt. Người ta bơm nước biển có nồng độ muối là  $ 30 $ gam/lít vào hồ với tốc độ $ 15 $ lít/phút.
	\begin{enumerate}
		\item Chứng tỏ rằng nồng độ muối của nước trong hồ sau $ t $ phút kể từ khi bắt đầu bơm là \break $ C(t)=\dfrac{30t}{400+t} $ (gam/lít).
		\item Nồng độ muối trong hồ như thế nào nếu $ t\to +\infty $.
	\end{enumerate}
	\loigiai{
		\begin{enumerate}
			\item Sau thời gian $ t $ phút, số lít nước trong hồ là $6000+15t  $ (lít).\\
			Số gam muối trong số lít nước bơm vào là $ 30\cdot 15t=450t $ (gam).\\
			Nồng độ muối của nước trong hồ sau $ t $ phút khi bắt đầu bơm là 
			$$C(t)=\dfrac{450t}{6000+15t}=\dfrac{30t}{400+t} \, \text{(gam/lít)}.$$
			\item Khi $ t\to +\infty $, ta xét giới hạn
			$$\lim \limits_{t \to +\infty} C(t)=\lim \limits_{t \to +\infty} \dfrac{30t}{400+t}=\lim \limits_{t \to +\infty} \dfrac{30}{\dfrac{400}{t}+1}=30.$$
			Vậy khi bơm nước biểm vào hồ chứa, không giới hạn thời gian thì nồng độ muối trong hồ chứa chính là nồng độ muối của nước biển.
		\end{enumerate}
	}
\end{bt}\dongcham{15}

\begin{bt}%[1K5YF-8]%[Tex SGK toán 11 - Nguyễn Sĩ Đạt]
	Cho hàm số $H(t)=\begin{cases}
		0 & \text{nếu }t<0\\ 
		1 & \text{nếu }t \geq 0
	\end{cases}$ (hàm Heaviside, thường được dùng để mô tả việc chuyển trạng thái tắt/mở của dòng điện tại thời điểm $t=0$ ). Tính $\lim \limits_{t \rightarrow 0^{+}} H(t)$ và $\lim \limits_{t \rightarrow 0^{-}} H(t)$.
	\loigiai{
		Ta có $\lim \limits_{t \rightarrow 0^{+}} H(t)=1$ và $\lim \limits_{t \rightarrow 0^{-}} H(t)=0$.	
	}
\end{bt}\dongcham{15}

\begin{bt}%[1C3B2-8]
	Chi phí (đơn vị: nghìn đồng) để sản xuất $x$ sản phẩm của một công ty được xác định bởi hàm số $C(x)=50000+105 x$.
	\begin{enumerate}
		\item  Tính chi phí trung bình $\overline{C}(x)$ để sản xuất một sản phẩm.
		\item  Tính $\lim\limits_{x \to+\infty} \overline{C}(x)$ và cho biết ý nghĩa của kết quả.		
	\end{enumerate}
	\loigiai{
		\begin{enumerate}
			\item Chi phí trung bình để sản xuất một sản phẩm là $\overline{C}(x)=\dfrac{50000+105 x}{x}$.
			\item Ta có 
			\allowdisplaybreaks
			\begin{eqnarray*}
				\lim\limits_{x \to+\infty} \overline{C}(x)&=&\lim\limits_{x \to+\infty}\dfrac{50000+105 x}{x}\\
				&=&\lim\limits_{x \to+\infty}\left(105+\dfrac{50000}{x}\right)\\
				&=&\lim\limits_{x \to+\infty}105+\lim\limits_{x \to+\infty}\dfrac{50000}{x}\\
				&=&105.
			\end{eqnarray*}
		\end{enumerate}
		Ý nghĩa của kết quả: số lượng sản phẩm càng nhiều thì chi phí sản xuất sẽ càng giảm, chi phí thấp nhất là $105$ nghìn đồng.
	}
\end{bt}\dongcham{13}


\begin{bt}
	Cho hàm số $f(x)=\heva{&\dfrac{ax^2+3ax-4a}{x-1}&\textrm{ khi }&x<1\\&2bx+1&\textrm{ khi }&x\ge 1}$. Biết rằng $a,b$ là các số thực thỏa mãn hàm số $f(x)$ có giới hạn tại $x=1$.
	\begin{tasks}(1)
		\task Tìm mối quan hệ giữa $a$ và $b$.
		\task Tìm giá trị nhỏ nhất của biểu thức $P=a^2+b^2$.
	\end{tasks}
	\loigiai{
		\begin{enumerate}[a)]
			\item Ta có $\lim\limits_{x\to 1^-}f(x)=\lim\limits_{x\to 1^-}a(x-4)=-3a$, $\lim\limits_{x\to 1^+}f(x)=2b+1$.\\
			Hàm số $f(x)$ có giới hạn tại $x=1$ khi và chỉ khi $-3a=2b+1$.
			\item Từ câu a) ta có $1=(3a+2b)^2\le (9+4)(a^2+b^2)\Rightarrow P=a^2+b^2\ge \dfrac{1}{13}$. Đẳng thức có được khi và chỉ khi $a=-\dfrac{3}{13}$ và $b=-\dfrac{2}{13}$. Vậy $\min P=\dfrac{1}{13}$.
		\end{enumerate}
	}
\end{bt}\dongcham{15}