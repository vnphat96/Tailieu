\def\tenchude{GIỚI HẠN DÃY SỐ}
\setcounter{section}{14}
\section{GIỚI HẠN CỦA DÃY SỐ}
\subsection{TÓM TẮT LÝ THUYẾT}
\subsubsection{GIỚI HẠN HỮU HẠN CỦA DÃY SỐ}
\begin{itemize}
	\item [\iconMT] \indam{Định nghĩa 1:} Dãy số $(u_n)$ có giới hạn là $0$ khi $n$ dần tới dương vô cực nếu $|u_n|$ có thể nhỏ hơn một số dương bé tuỳ ý, kể từ một số hạng nào đó trở đi.
	Kí hiệu
	\boxmini{$\lim\limits_{n \to +\infty}u_n=0$}
	\item [\iconMT]  \indam{Định nghĩa 2:} Dãy số $(u_n)$ có giới hạn là $a$ nếu $\lim\limits_{n \to +\infty}(u_n-a)=0$. Kí hiệu
	\boxmini{$\lim\limits_{n \to +\infty}u_n=a$}	
	% \begin{note}
	% 	Ta có thể viết $\lim\limits_{n \to +\infty}u_n=a$ thay cho cách viết $\lim\limits_{n \to +\infty}u_n=a$ (không cần viết chỉ số $n \to +\infty$)
	% \end{note}
	\item [\iconMT] \indam{Một vài giới hạn đặc biệt:} (\textit{có thể xem như công thức})
	\begin{khung}
		\begin{itemize}
			\begin{multicols}{2}
				\item $\lim\limits_{n \to +\infty}\dfrac{1}{n}=0$;
				\item $\lim\limits_{n \to +\infty}\dfrac{1}{n^k}=0$, với $k\in \mathbb{N}^*$;
				\item $\lim\limits_{n \to +\infty}\dfrac{1}{\sqrt{n}}=0$;
				\item $\lim\limits_{n \to +\infty}C=C,\, \forall C\in \mathbb{R}$;
				\item $\lim\limits_{n \to +\infty}\dfrac{1}{q^n}=0$, với $|q|>1$;
				\item $\lim\limits_{n \to +\infty}q^n=0$, nếu $|q|<1$.
			\end{multicols}
		\end{itemize}
	\end{khung}
\end{itemize}
\subsubsection{ĐỊNH LÝ VỀ GIỚI HẠN HỮU HẠN CỦA DÃY SỐ}
\begin{itemize}
	\item [\iconMT] Nếu $\lim\limits_{n \to +\infty}u_n=a$ và $\lim\limits_{n \to +\infty}v_n=b$ thì ta có:
	\begin{khung}
		\begin{itemize}
			\begin{multicols}{2}
				\item $\lim\limits_{n \to +\infty}\left(u_n \pm v_n\right)=a+b$;
				\item $\lim\limits_{n \to +\infty}\left( u_n.v_n\right)=a.b$;
				\item $\lim\limits_{n \to +\infty}\left(\dfrac{u_n}{v_n}\right)=\dfrac{a}{b}$, với $b\ne 0$;
				\item $\lim\limits_{n \to +\infty}\sqrt{u_n}=\sqrt{a}$, với $a\ge 0$;
				\item $\lim\limits_{n \to +\infty}|u_n|=|a|$;
				\item $\lim\limits_{n \to +\infty}\left(k. u_n\right)=k. a$ $(k\in \mathbb{R})$.
			\end{multicols}
		\end{itemize}
	\end{khung}
	\item [\iconMT] Định lý "kẹp":
	\begin{khung}
		\begin{itemize}
			\item Nếu $0\le \left|u_n\right|\le v_n$, $\forall n\in \mathbb{N}^*$ và $\lim\limits_{n \to +\infty}v_n=0$ thì $\lim\limits_{n \to +\infty}u_n=0$.
			\item Nếu $w_n\le u_n\le v_n$, $\forall n\in \mathbb{N}^*$ và $\lim\limits_{n \to +\infty}w_n=\lim\limits_{n \to +\infty}v_n=a$ thì $\lim\limits_{n \to +\infty}u_n=a$.
		\end{itemize}
	\end{khung}
\end{itemize}

\subsubsection{TỔNG CỦA CẤP SỐ NHÂN LÙI VÔ HẠN}
\begin{itemize}
	\item [\iconMT] Cấp số nhân vô hạn $(u_n)$ có công bội $q$ thoả mãn $|q|<1$ được gọi là \textit{cấp số nhân lùi vô hạn}.
	\item [\iconMT] Cho cấp số nhân lùi vô hạn $(u_n)$, ta có tổng của cấp số nhân lùi vô hạn đó là 
	\begin{khung}
		$$S=u_1+u_2+u_3+...+u_n+...=\dfrac{u_1}{1-q}, (|q|<1)$$
	\end{khung}
\end{itemize}

\subsubsection{GIỚI HẠN VÔ CỰC CỦA DÃY SỐ}
\begin{itemize}
	\item [\iconMT] \indam{Định nghĩa 1:} Ta nói dãy số $(u_n)$ có giới hạn $+\infty$ khi $n\to +\infty$, nếu $u_n$ có thể lớn hơn một số dương bất kì, kể từ một số hạng nào đó trở đi.	Kí hiệu
	\boxmini{$\lim\limits_{n \to +\infty}u_n=+\infty$}
	\item [\iconMT] \indam{Định nghĩa 2:} Ta nói dãy số $(u_n)$ có giới hạn $-\infty$ khi $n\to +\infty$, nếu $\lim\limits_{n \to +\infty}(-u_n)=+\infty$. Kí hiệu
	\boxmini{$\lim\limits_{n \to +\infty}u_n=-\infty$}
	\item [\iconMT] \indam{Một số giới hạn đặc biệt:}
	\begin{khung}
		\begin{listEX}[2]
			\item $\lim\limits_{n \to +\infty}n^k=+\infty$, với $k \in \mathbb{N^*}$.
			\item $\lim\limits_{n \to +\infty}q^n=+\infty$, với $q>1$.
		\end{listEX}
	\end{khung}
		 
	\item [\iconMT] \indam{Một số quy tắc tính giới hạn vô cực:}
	\begin{itemize}
		\item [\ding{172}] Quy tắc tìm giới hạn của tích $u_n\cdot v_n$
		\begin{center}
			\begin{tabular}{|c|c|c|}
				\hline
				$\lim\limits_{n \to +\infty}u_n=L$ & $\lim\limits_{n \to +\infty}v_n = \infty$   & $\lim\limits_{n \to +\infty}\left[{u_n\cdot v_n}\right]$ \\
				\hline
				$L>0$   & $+\infty $   & $+\infty $ \\
				\hline
				$L>0$    & $-\infty $    & $-\infty $ \\
				\hline
				$L<0$   & $+\infty $   & $-\infty $ \\
				\hline
				$L<0$    & $-\infty $    & $+\infty $ \\
				\hline
			\end{tabular}
		\end{center}
		\item [\ding{173}]Quy tắc tìm giới hạn của thương $\dfrac{u_n}{v_n}$
		\begin{center}
			\begin{tabular}{|c|c|c|c|}
				\hline
				$\lim\limits_{n \to +\infty}u_n=L$ & $\lim\limits_{n \to +\infty}v_n$   & Dấu của $v_n$ & $\lim\limits_{n \to +\infty}\dfrac{u_n}{v_n}$\\
				\hline
				$L$   & $\pm \infty $   & Tùy ý & $0$\\
				\hline
				$L>0$   & $0$   & $+$ & $+\infty $\\
				\hline
				$L>0$    & $0$   & $-$ & $-\infty $\\
				\hline
				$L<0$   & $0$   & $+$ & $-\infty $\\
				\hline
				$L<0$    & $0$    & $-$ & $+\infty $\\
				\hline
			\end{tabular}
		\end{center}  
	\end{itemize}
\end{itemize}


