
\subsection{Giới hạn một bên}
\subsubsection{Ví dụ}

\begin{vd}%[DCHT Toán 11 - KNTT- Phạm Tuấn]%[1K5BF-7] 
Tính giới hạn $\lim\limits _{x \rightarrow 2^{-}} \dfrac{x^2-3 x+2}{\sqrt{2-x}}$.
\dapso{$\lim\limits _{x \rightarrow 2^{-}} \dfrac{x^2-3 x+2}{\sqrt{2-x}} =0$}
\loigiai{
Ta có 
\[
\lim\limits _{x \rightarrow 2^{-}} \dfrac{x^2-3 x+2}{\sqrt{2-x}} = \lim\limits _{x \rightarrow 2^{-}} \dfrac{(2-x)(1-x)}{\sqrt{2-x}} = \lim\limits _{x \rightarrow 2^{-}} (1-x)\sqrt{2-x}  = 0.
\]
}
\end{vd}


\begin{vd}%[DCHT Toán 11 - KNTT- Phạm Tuấn]%[1K5BF-7] 
Tính giới hạn $\lim\limits _{x \rightarrow (-1)^{+}} \dfrac{x^3+1}{x^3+2x^2+x}$. 
\dapso{$\lim\limits _{x \rightarrow (-1)^{+}} \dfrac{x^3+1}{x^3+2x^2+x}= -\infty$}
\loigiai{
Ta có 
\[
\lim\limits _{x \rightarrow (-1)^{+}} \dfrac{x^3+1}{x^3+2x^2+x} = \lim\limits _{x \rightarrow (-1)^{+}} \dfrac{(x+1)(x^2-x+1)}{x(x+1)^2} = \lim\limits _{x \rightarrow (-1)^{+}} \dfrac{x^2-x+1}{x(x+1)}.
\]
Khi $x \to (-1)^+$ thì $\heva{&x+1 \to 0\\&x+1 >0\\& \dfrac{x^2-x+1}{x} \to -3}$ suy ra $\lim\limits _{x \rightarrow (-1)^{+}} \dfrac{x^2-x+1}{x(x+1)} = -\infty.$ \\
Vậy $\lim\limits _{x \rightarrow (-1)^{+}} \dfrac{x^3+1}{x^3+2x^2+x}= -\infty$. 
}
\end{vd}


\begin{vd}%[DCHT Toán 11 - KNTT- Phạm Tuấn]%[1K5BF-7] 
Cho hàm số $f(x) = \heva{&\sqrt{9-x^2} && \text{ khi } -3 \leq x < 3\\& 1 && \text{ khi } x=3\\& \sqrt{x^2-9} && \text{ khi } x>3.}$ \\
Hàm số $f(x)$ có giới hạn khi $x \to 3$ hay không?
\dapso{$\lim\limits _{x \rightarrow 3} f(x) =0$}
\loigiai{
Ta có $\lim\limits _{x \rightarrow 3^{-}} f(x) = \lim\limits _{x \rightarrow 3^{-}} \sqrt{9-x^2} =0$; $\lim\limits _{x \rightarrow 3^{+}} f(x) = \lim\limits _{x \rightarrow 3^{+}} \sqrt{x^2-9} =0$. \\
Suy ra $\lim\limits _{x \rightarrow 3^{-}} f(x) = \lim\limits _{x \rightarrow 3^{+}} f(x) =0$. \\
Vậy $\lim\limits _{x \rightarrow 3} f(x) =0$.
}
\end{vd}


\begin{vd}%[DCHT Toán 11 - KNTT- Phạm Tuấn]%[1K5BF-7] 
Ta gọi phần nguyên của số thực $x$ là số nguyên lớn nhất không lớn hơn $x$ và kí hiệu nó là $[x]$. 
Ví dụ $[5]=5 $; $[3,12]=3 $; $[-2{,}725]=-3$. \\
Tìm $\lim\limits _{x \rightarrow 1^{-}} [x]$ và  $\lim\limits _{x \rightarrow 1^{+}} [x]$. Giới hạn $\lim\limits _{x \rightarrow 1} [x]$ có tồn tại hay không?
\dapso{$\lim\limits _{x \rightarrow 1^{-}} [x] =0$; $\lim\limits _{x \rightarrow 1^{+}} [x] =1$}
\loigiai{
Ta có $\lim\limits _{x \rightarrow 1^{-}} [x] =0$; $\lim\limits _{x \rightarrow 1^{+}} [x] =1$. \\
Suy ra $\lim\limits _{x \rightarrow 1^{+}} [x] \neq  \lim\limits _{x \rightarrow 1^{-}} [x]$. \\
Vậy giới hạn $\lim\limits _{x \rightarrow 1} [x]$ không tồn tại.
}
\end{vd}


\begin{vd}%[DCHT Toán 11 - KNTT- Phạm Tuấn]%[1K5BF-7] 
Cho hàm số $f(x) = \heva{& \dfrac{x-\sqrt{2x}}{4-x^2} && \text{ khi } x < 2\\& x^2-x+m && \text{ khi } x \geq  2}$  ($m$ là tham số). \\
Tìm $m$ để hàm số $f(x)$ có giới hạn khi $x \to 2$.
\dapso{$m= -\dfrac{17}{8}$}
\loigiai{
Ta có 
\begin{align*}
&\lim\limits _{x \rightarrow 2^{-}}  f(x) = \lim\limits _{x \rightarrow 2^{-}} \dfrac{x-\sqrt{2x}}{4-x^2} = \lim\limits _{x \rightarrow 2^{-}}  \dfrac{x(x-2)}{-(x-2)(x+2)(x+\sqrt{2x})} = -\dfrac{1}{8}; \\
& \lim\limits _{x \rightarrow 2^{+}}  f(x)  = \lim\limits _{x \rightarrow 2^{+}}  (x^2-x+m) = 2+m.
\end{align*}
Hàm số $f(x)$ có giới hạn khi $x \to 2$ khi và chỉ khi 
$$\lim\limits _{x \rightarrow 2^{-}}  f(x)  = \lim\limits _{x \rightarrow 2^{+}}  f(x) \Leftrightarrow -\dfrac{1}{8}=2+m \Leftrightarrow m= -\dfrac{17}{8}. $$
}
\end{vd}




\subsubsection{Bài tập rèn luyện}
\Opensolutionfile{ans}[ans1]

\noindent \textbf{Bài tập tự luận}

\begin{bt}%[DCHT Toán 11 - KNTT- Phạm Tuấn]%[1K5BF-7] 
Tính giới hạn $\lim\limits _{x \rightarrow 1^{-}} \dfrac{-x^2-x+2}{x^2-3x^2+3x-1}$. 
\dapso{$\lim\limits _{x \rightarrow 1^{-}} \dfrac{-x^2-x+2}{x^2-3x^2+3x-1} = -\infty$}
\loigiai{
Ta có $\lim\limits _{x \rightarrow 1^{-}} \dfrac{-x^2-x+2}{x^2-3x^2+3x-1} = \lim\limits _{x \rightarrow 1^{-}}  \dfrac{-(x-1)(x+2)}{(x-1)^3} = \lim\limits _{x \rightarrow 1^{-}}   \dfrac{-x-2}{(x-1)^2}$.  \\
Khi $x \to 1^-$ thì $\heva{& (x-1)^2  \to 0\\& (x-1)^2 >0\\& -x-2\to -3}$ suy ra $\lim \limits _{x \rightarrow 1^{-}}   \dfrac{-x-2}{(x-1)^2}= -\infty.$\\
Vậy $\lim\limits _{x \rightarrow 1^{-}} \dfrac{-x^2-x+2}{x^2-3x^2+3x-1} = -\infty$. 
}
\end{bt}

\begin{bt}%[DCHT Toán 11 - KNTT- Phạm Tuấn]%[1K5BF-7] 
Cho hàm số $f(x)=\heva{& \dfrac{x^2-1}{1-x} \,&\text{ khi }x < 1\\& x^3-2x^2+3\,&\text{ khi }x \geq  1}$. Tính $\lim\limits_{x\to 1^-}f(x)$ và $\lim\limits_{x\to 1^+}f(x)$.
\dapso{$\lim\limits_{x\to 1^-}f(x)=-2$; $\lim\limits_{x\to 1^+}f(x)=2$}
\loigiai{
Ta có $\lim\limits_{x\to 1^-}f(x) = \lim\limits_{x\to 1^-}  \dfrac{x^2-1}{1-x} =   \lim\limits_{x\to 1^-}  -(x+1) = -2$; 
$\lim\limits_{x\to 1^+} f(x) = \lim\limits_{x\to 1^+} (x^3-2x^2+3) = 2 $.
}
\end{bt}

\begin{bt}%[DCHT Toán 11 - KNTT- Phạm Tuấn]%[1K5BF-7] 
Tính giới hạn $\lim\limits _{x \rightarrow 2^{-}} \dfrac{|x^2-3x+2|}{x^2-4}$. 
\dapso{$\lim\limits _{x \rightarrow 2^{-}} \dfrac{|x^2-3x+2|}{x^2-4} =  -\dfrac{1}{4}$}
\loigiai{
Khi $x \to 2^-$ thì $x^2-3x+2 <0$ nên 
\[
\lim\limits _{x \rightarrow 2^{-}} \dfrac{|x^2-3x+2|}{x^2-4} = \lim\limits _{x \rightarrow 2^{-}} \dfrac{-x^2+3x-2}{x^2-4} = \lim\limits _{x \rightarrow 2^{-}} \dfrac{1-x}{x+2} = -\dfrac{1}{4}.
\]
}
\end{bt}



\begin{bt}%[DCHT Toán 11 - KNTT- Phạm Tuấn]%[1K5BF-7] 
Cho hàm số $f(x) = \heva{&\dfrac{1-\sqrt{x}}{x^2-2x+1} \text{ khi  } x >1\\& \dfrac{2x}{x^3-2x+1}  \text{ khi  } x <1}$. Tính $\lim\limits _{x \rightarrow 1} f(x)$.
\dapso{$\lim\limits _{x \rightarrow 1} f(x)=-\infty$}
\loigiai{
Xét $\lim\limits _{x \rightarrow 1^{+}}  f(x) = \lim\limits _{x \rightarrow 1^{+}} \dfrac{1-\sqrt{x}}{x^2-2x+1} =\lim\limits _{x \rightarrow 1^{+}} \dfrac{1-x}{(x-1)^2(\sqrt{x}+1)} = \lim\limits _{x \rightarrow 1^{+}} \dfrac{1}{(1-x)(\sqrt{x}+1)}$. \\
Khi $x \to 1^+$ thì $\heva{&1-x <0\\&1-x \to 0\\&\sqrt{x}+1 \to 2}$, suy ra $\lim\limits _{x \rightarrow 1^{+}}  f(x) = -\infty$. \\
Xét $\lim\limits _{x \rightarrow 1^{-}}  f(x) = \lim\limits _{x \rightarrow 1^{-}}  \dfrac{2x}{x^3-2x+1} = \lim\limits _{x \rightarrow 1^{-}} \dfrac{2x}{(x-1)(x^2+x-1)}$. \\
Khi $x \to 1^-$ thì $\heva{&x-1 <0\\&x-1 \to 0\\&x^2+x-1 \to 1}$, suy ra $\lim\limits _{x \rightarrow 1^{-}}  f(x) = -\infty$. \\
Suy ra $\lim\limits _{x \rightarrow 1^{+}}  f(x) =\lim\limits _{x \rightarrow 1^{-}}  f(x) = -\infty$. Vậy $\lim\limits _{x \rightarrow 1} f(x)=-\infty$.
}
\end{bt}

\begin{bt}%[DCHT Toán 11 - KNTT- Phạm Tuấn]%[1K5BF-7] 
Cho hàm số $f(x) = |x^2-2x-3|$. Tính các giới hạn $\lim\limits _{x \rightarrow 0^{-}} \dfrac{f(x+3)- f(3)}{x}$ và $\lim\limits _{x \rightarrow 0^{+}} \dfrac{f(x+3)- f(3)}{x}$. 
\dapso{$\lim\limits _{x \rightarrow 0^{-}} \dfrac{f(x+3)- f(3)}{x}=-4$;  $\lim\limits _{x \rightarrow 0^{+}} \dfrac{f(x+3)- f(3)}{x}=4$}
\loigiai{
Ta có $\lim\limits _{x \rightarrow 0^{-}} \dfrac{f(x+3)- f(3)}{x} = \lim\limits _{x \rightarrow 0^{-}} \dfrac{|(x+3)^2-2(x+3)-3|-0}{x} = \lim\limits _{x \rightarrow 0^{-}} \dfrac{|x(x+4)|}{x}$. \\
Khi $x \to 0^-$ thì $x<0$, suy ra  $\lim\limits _{x \rightarrow 0^{-}} \dfrac{|x(x+4)|}{x} = \lim\limits _{x \rightarrow 0^{-}} -(x+4) = -4$. \\
Ta có $\lim\limits _{x \rightarrow 0^{+}} \dfrac{f(x+3)- f(3)}{x} = \lim\limits _{x \rightarrow 0^{+}} \dfrac{|(x+3)^2-2(x+3)-3|-0}{x} = \lim\limits _{x \rightarrow 0^{+}} \dfrac{|x(x+4)|}{x}$. \\
Khi $x \to 0^+$ thì $x>0$, suy ra  $\lim\limits _{x \rightarrow 0^{+}} \dfrac{|x(x+4)|}{x} = \lim\limits _{x \rightarrow 0^{+}} (x+4) = 4$.
}
\end{bt}

\begin{bt}%[DCHT Toán 11 - KNTT- Phạm Tuấn]%[1K5KF-7] 
Tìm $m$ để hàm số $f(x) = \heva{&\sin \dfrac{1}{2x} && \text{ khi } x <0\\& x^2+m && \text{ khi } x \geq 0}$ có giới hạn khi $x\to 0$.
\dapso{Không tồn tại $m$}
\loigiai{
Ta có $\lim\limits _{x \rightarrow 0^{+}} f(x) = \lim\limits _{x \rightarrow 0^{+}} (x^2+m)=m$. \\
Xét $\lim\limits _{x \rightarrow 0^{-}} f(x) = \lim\limits _{x \rightarrow 0^{+}} \sin \dfrac{1}{2x}$. \\
Chọn dãy số $x_n = -\dfrac{2}{n\pi }$. Dễ thấy $x_n<0$ và $\lim \limits{n \to +\infty}x_n =0$. \\
Ta có $\lim \limits{n \to +\infty}\sin \dfrac{1}{2x} = \lim \limits{n \to +\infty}\sin (-n\pi ) =0$. \\
Chọn dãy số $x_n = -\dfrac{2}{\frac{\pi}{2}+ n2\pi} $. Dễ thấy $x_n<0$ và $\lim \limits{n \to +\infty}x_n =0$. \\
Ta có $\lim \limits{n \to +\infty}\sin \dfrac{1}{2x} = \lim \limits{n \to +\infty}\sin (-\frac{\pi}{2}- n2\pi ) =-1$.  \\
Suy ra $\lim\limits _{x \rightarrow 0^{-}} f(x)$ không tồn tại. \\
Vậy không tồn tại $m$ để $f(x)$ có giới hạn khi $x\to 0$.
}
\end{bt}


\begin{bt}%[DCHT Toán 11 - KNTT- Phạm Tuấn]%[1K5BF-7] 
Cho hàm số $f(x)=\heva{& \dfrac{1}{x-1} - \dfrac{3}{x^3-1} \,&\text{ nếu }x > 1\\& mx+2\,&\text{ nếu }x \geq  1}$.  \\
Với giá trị nào của tham số $m$ thì hàm số $f(x)$ có giới hạn khi $x \rightarrow 1$? Tìm giới hạn này.
\dapso{$m=-1$; $\lim\limits _{x \rightarrow 1} f(x)=1$}
\loigiai{
Ta có
\begin{align*}
\lim\limits  _{x \rightarrow 1^{+}} f(x) &=\lim\limits  _{x \rightarrow 1^{+}}\left(\frac{1}{x-1}-\frac{3}{x^3-1}\right)=\lim\limits  _{x \rightarrow 1^{+}} \frac{x^2+x-2}{(x-1)\left(x^2+x+1\right)} \\
&=\lim\limits  _{x \rightarrow 1^{+}} \frac{(x-1)(x+2)}{(x-1)\left(x^2+x+1\right)}=\lim\limits  _{x \rightarrow 1^{+}} \frac{x+2}{x^2+x+1}=1 .
\end{align*}
$\lim\limits _{x \rightarrow 1^{-}} f(x)=\lim\limits _{x \rightarrow 1^{-}}(m x+2)=m+2$. \\
$f(x)$ có giới hạn khi $x \rightarrow 1 \Leftrightarrow m+2=1 \Leftrightarrow m=-1$. Khi đó $\lim\limits _{x \rightarrow 1} f(x)=1$.
}
\end{bt}


\begin{bt}%[DCHT Toán 11 - KNTT- Phạm Tuấn]%[1K5BF-7] 
Cho hàm số $f(x) = \heva{&x\cos \dfrac{1}{x} && \text{ khi } x <0\\& \sin x^2 + m  && \text{ khi } x \geq 0.}$ \\
Tìm $m$ để hàm số $f(x)$ có giới hạn khi $x \to 0$.
\dapso{$m=0$}
\loigiai{
Xét $\lim\limits _{x \rightarrow 0^{-}} f(x) = \lim\limits _{x \rightarrow 0^{-}}  x\cos \dfrac{1}{x}$. \\
Ta có $0\leq  |x\cos \dfrac{1}{x}| \leq |x|$ và $\lim\limits _{x \rightarrow 0^{-}} |x| =0$. Suy ra $\lim\limits _{x \rightarrow 0^{-}}  x\cos \dfrac{1}{x} =0$. \\
Ta lại có $\lim\limits _{x \rightarrow 0^{+}} f(x) = \lim\limits _{x \rightarrow 0^{-}}  (\sin x^2 + m) = m$. \\
$f(x)$ có giới hạn khi $x \rightarrow 0$ khi và chỉ khi 
\[
\lim\limits _{x \rightarrow 0^{-}}  f(x) = \lim\limits _{x \rightarrow 0^{+}}  f(x)  \Leftrightarrow m=0.
\]
}
\end{bt}
\noindent \textbf{Bài tập trắc nghiệm}
\Opensolutionfile{ans}[ans/ans-1K5-2-Dang6]
\begin{ex}%[DCHT Toán 11 - KNTT- Phạm Tuấn]%[1K5BF-7]
	Tính giới hạn $\lim\limits_{x \to(-2)^{-}} \dfrac{3+2 x}{x+2}$.
	\choice
	{$-\infty$}
	{$2$}
	{\True $+\infty$}
	{$\dfrac{3}{2}$}
	\loigiai
	{
		Khi $x \to (-2)^{-}$ thì $\heva{& 3+3x\to -1\\&x+2\to 0 \\&x+2<0.}$ \\
		Suy ra  $\lim\limits_{x \to(-2)^{-}} \dfrac{3+2 x}{x+2}=+\infty$.
	}
\end{ex}

\begin{ex}%[DCHT Toán 11 - KNTT- Phạm Tuấn]%[1K5BF-7]
Cho hàm số $f(x)=\heva{&2x^2-2\,&\text{ khi }x\ge 6\\&x-2\,&\text{ khi }x<6}$. Tính $\lim\limits_{x\to 6^-}f(x)$ bằng
\choice
{$2$}
{$5$}
{$1$}
{\True $4$}
\loigiai{
Ta có $\lim\limits_{x\to 6^-}f(x)=\lim\limits_{x\to 6^-}(x-2)=6-2=4$.
}
\end{ex}


\begin{ex}%[DCHT Toán 11 - KNTT- Phạm Tuấn]%[1K5BF-7]
$\displaystyle \lim \limits_{x \rightarrow 5^+} \dfrac{|10-2x|}{x^2-6x+5}$ là
\choice
{\True $\dfrac{1}{2}$}
{$0$}
{$+\infty$}
{$- \dfrac{1}{2}$}
\loigiai{
Ta có $\displaystyle \lim \limits_{x \rightarrow 5^+} \dfrac{|10-2x|}{x^2-6x+5} 
	= \lim \limits_{x \rightarrow 5^+} \dfrac{2x-10}{(x-1)(x-5)} 
	= \lim \limits_{x \rightarrow 5^+} \dfrac{2}{x-1} = \dfrac{1}{2}$.
}
\end{ex}

\begin{ex}%[DCHT Toán 11 - KNTT- Phạm Tuấn]%[1K5BF-7]
Tính $\lim\limits _{x \to 2^{+}}\dfrac{|2-x|}{x^{2}-x-2}$. 
\choice
{$+\infty$}
{$0$}
{$-\dfrac{1}{3}$}
{\True $\dfrac{1}{3}$}
\loigiai{
Vì $x \to 2^+$ nên $x>2$. Do đó $\lim\limits _{x \to 2^{+}}\dfrac{|2-x|}{x^{2}-x-2} = \lim\limits _{x \to 2^{+}}\dfrac{x-2}{(x-2)(x+1)} = \lim\limits _{x \to 2^{+}}\dfrac{1}{x+1}=\dfrac{1}{3}$.
}
\end{ex}

\begin{ex}%[DCHT Toán 11 - KNTT- Phạm Tuấn]%[1K5BF-7]
Trong các giới hạn sau, giới hạn nào không tồn tại?
\choice
{$\lim\limits_{x \to \infty} \dfrac{2x + 1}{x^2 + 1}$}
{$\lim\limits_{x \to 0} \dfrac{x}{\sqrt{x} + 1}$}
{$\lim\limits_{x \to 1}\dfrac{x}{(x + 1)^2}$}
{\True $\lim\limits_{x \to 0} \dfrac{1}{x}$}	
\loigiai{
$\lim\limits_{x \to 0^+} = \dfrac{1}{x} = +\infty; \lim\limits_{x \to 0^-} = \dfrac{1}{x} = -\infty $ nên giới hạn không tồn tại.
}
\end{ex}

\begin{ex}%[DCHT Toán 11 - KNTT- Phạm Tuấn]%[1K5BF-7]
Gọi $a$ là số thực để hàm số $f(x)=\heva{ & x^2+ax+2 & & \text{khi} \ x>2 \\ & 2x^2-x+1 & & \text{khi} \ x\leqslant2}$ có giới hạn khi $x\to2$. Hãy chọn hệ thức đúng.
\choice
{$2a^2+3a+1=0$}
{$a^2-3a+2=0$}
{\True $4a^2-1=0$}
{$a^2-4=0$}
\loigiai
{
Ta có 
\[\heva{&\lim\limits_{x\to2^+}f(x) = \lim\limits_{x\to2^+}\left(x^2+ax+2\right) = 2a+6 \\&\lim\limits_{x\to2^-}f(x) = \lim\limits_{x\to2^-} \left(2x^2 - x  + 1\right)=7.}\]
Để hàm số có giới hạn khi $x\to 2$ thì \[\lim\limits_{x\to2^+}f(x) = \lim\limits_{x\to2^-}f(x) \Leftrightarrow 2a + 6 = 7 \Leftrightarrow a = \dfrac{1}{2}.\]
Khi đó $4a^2 - 1=0$ là hệ thức đúng.
}
\end{ex}

\begin{ex}%[DCHT Toán 11 - KNTT- Phạm Tuấn]%[1K5BF-7]
	Cho hàm số $f(x)=\heva{&\dfrac{x^3-3 x^2+2}{x-1}\,\, \text { nếu } \,\, x>1 \\& a x+3 \,\, \text { nếu } \,\,x \leq 1}$. Tìm $a$ để $\lim\limits_{x \to 1} f(x)$ tồn tại.
	\choice
	{$a=6$}
	{$a=1$}
	{$a=0$}
	{\True $a=-6$}
	\loigiai{
		$\lim\limits_{x \to 1^+} f(x)=\lim\limits_{x \to 1^+} \dfrac{x^3-3 x^2+2}{x-1}=\lim\limits_{x \to 1+} \dfrac{(x-1)(x^2-2x-2)}{x-1}=\lim\limits_{x \to 1^+}(x^2-2x-2)=-3$.\\
		$\lim\limits_{x \to 1^-} f(x)=\lim\limits_{x \to 1^-} (ax+3)=3+a$.\\
		Giới hạn $\lim\limits_{x \to 1} f(x)$ tồn tại  khi $3+a=-3\Rightarrow a=-6$. 	
	} 
\end{ex}

\begin{ex}%[DCHT Toán 11 - KNTT- Phạm Tuấn]%[1K5BF-7]
	Cho hàm số $f(x)=\heva{&\dfrac{\left|x-1\right|}{x-1}&\text{ khi }& x<1\\&x+2+a &\text{ khi } &1\leq x\leq 3\\ & \dfrac{x^2-81}{\sqrt{x}-3} &\text{ khi }& x>3}$. Tìm tất cả giá trị của tham số $a$ để hàm số có giới hạn tại $x=3$.
	\choice
	{$a=12\left(3+\sqrt{3}\right)$}
	{$a=12\left(3-\sqrt{3}\right)$}
	{\True $a=12\left(3+\sqrt{3}\right)-5$}
	{$a=12\left(3-\sqrt{3}\right)+5$}
	\loigiai{Với $1\leq x\leq 3$ thì $f(x)=x+2+a$ nên $\lim\limits_{x \to 3^-}f(x)=\lim\limits_{x \to 3^-} \left(x+2-a\right)=5+a$.\\
		Với $x>3$ thì $f(x)=\dfrac{x^2-81}{\sqrt{x}-3}=\left(\sqrt{x}+3\right)\left(x+9\right)$ nên $\lim\limits_{x \to 3^+} f(x)=\lim\limits_{x \to 3^+}\left(\sqrt{x}+3\right)\left(x+9\right) =12\left(3+\sqrt{3}\right)$.\\
		Do đó, để hàm số có giới hạn tại $x=3$ thì $\lim\limits_{x \to 3^-}f(x)=\lim\limits_{x \to 3^+}f(x) \Leftrightarrow a=12\left(3+\sqrt{3}\right)-5$.
		}
\end{ex}
\Closesolutionfile{ans}


