
\subsection{Toán thực tế, liên môn về giới hạn hàm số}
\subsubsection{Ví dụ}
\begin{vd}%[DCHT Toán 11 - KNTT- Phạm Tuấn]%[1K5YF-8]
Chiều dài một loài động vật nhỏ được tính theo công công thức $h(t)=\dfrac{300}{1+9 \cdot  (0{,}8)^t}$ mm, trong đó $t$ số ngày sau khi sinh của loài động vật đó. Tính chiều dài cuối cùng của nó (chiều dài khi $t \to +\infty$).
\dapso{$300$ mm}
\loigiai{
Ta có $\displaystyle \lim \limits_{t \to +\infty } \dfrac{300}{1+9 \cdot  (0{,}8)^t} = 300$. \\
Vậy chiều dài cuối cùng của loài động vật  là $300$ mm. 
}
\end{vd}


\begin{vd}%[DCHT Toán 11 - KNTT- Phạm Tuấn]%[1K5BF-8] 
Theo thuyết tương đối, khối lượng $m$ của một hạt phụ thuộc vào vận tốc $v$ của nó, theo công thức
$$
m=\frac{m_0}{\sqrt{1-\dfrac{v^2}{c^2}}}
$$
trong đó $m_0$ là khối lượng khi hạt đứng yên và $c$ là tốc độ ánh sáng. Tìm giới hạn của khối lượng khi $v$ tiến đến $c^{-}$.
\dapso{$\lim\limits _{v \to c^-} \frac{m_0}{\sqrt{1-\dfrac{v^2}{c^2}}} = +\infty$}
\loigiai{
Với $m_0 =0$ thì $\displaystyle \lim_{v \to c^-} =0$. \\
Với $m_0 \neq 0$. \\
Khi $c \to c^-$ thì $\heva{&\sqrt{1-\dfrac{v^2}{c^2}} \to 0\\&\sqrt{1-\dfrac{v^2}{c^2}} >0}$ suy ra $\displaystyle \lim_{v \to c^-} \frac{m_0}{\sqrt{1-\dfrac{v^2}{c^2}}} = +\infty$. \\
Vậy nếu một hạt có khối lượng nghỉ khác $0$ thì khối lượng của hạt sẽ lớn vô cùng khi vận tốc tiến gần vận tốc ánh sáng.
}
\end{vd}

\begin{vd}%[DCHT Toán 11 - KNTT- Phạm Tuấn]%[1K5BF-8] 
Một chất điểm chuyển động thẳng với phương trình $s(t)$. Khi đó vận tốc tức thời tại thời điểm $t_0$ được định nghĩa là $\displaystyle \lim \limits_{\Delta t} \dfrac{s(t_0+ \Delta t) - s(t_0)}{\Delta t}$. Tính vận tốc tức thời của chất điểm với phương trình chuyển động $s(t) = 3t^2-2t+3$ ($s(t)$ có đơn vị là m, $t$ đơn vị là giây), tại thời điểm $t=4$ giây. 
\dapso{$v=22  \mathrm{~m/s}$}
\loigiai{
Vận tốc tức thời của chất điểm tại thời điểm $t=4$ giây là
\begin{align*}
\lim \limits_{\Delta t \to 0} \dfrac{s(4+\Delta t) - s(4)}{\Delta t} &=  \lim \limits_{\Delta t \to 0} \dfrac{3(4+\Delta t)^2-2(4+\Delta t)+3 - 43}{\Delta t} \\
& = \lim \limits_{\Delta t \to 0} \dfrac{3 (\Delta t)^2+ 22 \Delta t}{\Delta t} = 22  \mathrm{~m/s}.
\end{align*}
}
\end{vd}

\begin{vd}%[DCHT Toán 11 - KNTT- Phạm Tuấn]%[1K5BF-8] 
Số lượng đơn vị hàng tồn kho trong một công ty  được cho bởi
$$
N(t)=200\left(3 \left [\frac{t+3}{3}  \right ]-t\right)
$$
trong đó $t$ là thời gian tính bằng ngày, $[x]$ là số nguyên lớn nhất không vượt quá $x$ (ví dụ $[-1{,}5]=-2$, $[8{,}8] = 8$).
\begin{enumerate}
\item Tính $\displaystyle \lim_{t \to 55^+} N(t)$.
\item  Tính $\displaystyle \lim_{t \to 201^-} N(t)$.
\end{enumerate}
\dapso{$\displaystyle \lim_{t \to 55^+} N(t) = 400$; $\displaystyle \lim_{t \to 201^-} N(t) =0$}
\loigiai{
\begin{enumerate}
\item 
Khi $t \to 55^+$, ta có $\left [\dfrac{t+3}{3}  \right ] = 19$. \\
Suy ra  $\displaystyle \lim_{t \to 55^+} N(t) =\lim_{t \to 55^+}  200\left(3 \left [\frac{t+3}{3}  \right ]-t\right) =200(3 \cdot 19 - 55) = 400$.
\item  
Khi $t \to 201^-$, ta có $\left [\dfrac{t+3}{3}  \right ] = 201$. \\
Suy ra  $\displaystyle \lim_{t \to 201^-} N(t) = \lim_{t \to 201^-}  200\left(3 \left [\frac{t+3}{3}  \right ]-t\right)= 200 \cdot 0 = 0$.
\end{enumerate}
}
\end{vd}


\begin{vd}%[DCHT Toán 11 - KNTT- Phạm Tuấn]%[1K5BF-8] 
Một chất điểm chuyển động thẳng với vận tốc $v(t)$. Khi đó gia tốc tức thời tại thời điểm $t_0$ được định nghĩa là $\displaystyle \lim \limits_{\Delta t \to 0} \dfrac{v(t_0+ \Delta t) - v(t_0)}{\Delta t}$. Một chất điểm chuyển động với vận tốc $v(t) = 0{,}1t^2-0{,}4t+1$ (m/s), tính gia tốc tức thời tại thời điểm $t=8$ giây. 
\dapso{$1{,}2$ $\mathrm{m/s^2}$}
\loigiai{
Gia tốc tức thời của chất điểm tại thời điểm $t=8$ giây là
\begin{align*}
\lim \limits_{\Delta t \to 0} \dfrac{v(8+\Delta t) - v(8)}{\Delta t} &=  \lim \limits_{\Delta t \to 0} \dfrac{0{,}1(8+\Delta t)^2-0{,}4(8+\Delta t)+1- 4{,}2}{\Delta t} \\
& = \lim \limits_{\Delta t \to 0} \dfrac{0{,}1 (\Delta t)^2 + 1{,}2 \Delta t}{\Delta t}   \\
& = \lim \limits_{\Delta t \to 0} (0{,}1\Delta t + 1{,}2)  \\
& = 1{,}2.
\end{align*}
Vậy gia tốc tức thời của chất điểm tại thời điểm $t=8$ giây là $1{,}2$ ($\mathrm{m/s^2}$).
}
\end{vd}

\begin{vd}%[DCHT Toán 11 - KNTT- Phạm Tuấn]%[1K5BF-8] 
Một người lái xe từ thành phố $A$ đến thành phố $B$ với vận tốc trung bình  là $x$ km/h. Trên chuyến trở về, vận tốc trung bình là $y$ km/h. Vận tốc trung bình của cả đi và về là $60$ km/h. (Giả sử người lái xe đi trên cùng  một con đường trên cả chuyến đi và về).
\begin{enumerate}
\item Chứng minh rằng $y= \dfrac{30x}{x-30}$.
\item Tìm giới hạn của $y$ khi $x \rightarrow 30^{+}$.
\end{enumerate}
\dapso{$\displaystyle \lim\limits_{x\to 30^+}  y  = +\infty$}
\loigiai{
\begin{enumerate}
\item  
Gọi  khoảng cách giữa $A$ và $B$ là $s$ km. \\
Thời gian  chuyến đi là $\dfrac{s}{x}$, thời gian  chuyến trở về là $\dfrac{s}{y}$. \\
Suy ra 
$$\dfrac{2s}{60} = \dfrac{s}{x} + \dfrac{s}{y} \Leftrightarrow \dfrac{1}{y} = \dfrac{1}{30} - \dfrac{1}{x} \Leftrightarrow y= \dfrac{30x}{x-30}.$$
\item  Ta có $\displaystyle \lim\limits_{x\to 30^+}  y = \lim\limits_{x\to 30^+}  \dfrac{30x}{x-30} = +\infty$. 
\end{enumerate}
}
\end{vd}


\begin{vd}%[DCHT Toán 11 - KNTT- Phạm Tuấn]%[1K5BF-8] 
Một hình elip với bán trục lớn $a$ và bán trục nhỏ $b$ thì diện tích được tính theo công thức $S=\pi ab$. Tính giới hạn diện tích của elip khi tiêu cự gần tới $0$.
\dapso{$\pi a^2$}
\loigiai{
Ta có $S= \pi ab = \pi a \sqrt{a^2-c^2}$. \\
Vậy $\lim\limits_{c \to 0} S= \lim\limits_{c \to 0} \pi a \sqrt{a^2-c^2} = \pi a^2$. \\
Ta thấy khi $c\to 0$, thì giới hạn diện tích của elip là diện tích hình tròn bán kính $R=a$.
}
\end{vd}

\begin{vd}%[DCHT Toán 11 - KNTT- Phạm Tuấn]%[1K5BF-8]  
Các nhà vật lý  thấy rằng thuyết tương đối hẹp của Einstein quy về cơ học Newton khi $c \rightarrow +\infty$, trong đó $c$ là tốc độ ánh sáng. Điều này được minh họa bởi ví dụ: Một hòn đá được ném thẳng đứng từ mặt đất để nó quay trở lại trái đất một giây sau đó. Sử dụng các định luật Newton, chúng ta thấy rằng chiều cao tối đa của hòn đá là $h=\dfrac{g}{8}$ mét ($g = 9{,}8 \mathrm{m/ s ^2}$). Theo thuyết tương đối hẹp, khối lượng của hòn đá phụ thuộc vào vận tốc của nó chia cho $c$, và có chiều cao cực đại là 
\[
h(c)=c \sqrt{\dfrac{c^2}{g^2}+\dfrac{1}{4}}- \dfrac{c^2}{g}.
\]
Tính $\lim\limits _{c \rightarrow +\infty} h(c)$.
\dapso{$\lim\limits _{c \rightarrow +\infty} h(c)= \dfrac{g}{8}$}
\loigiai{
Ta có 
\[
\lim\limits _{c \rightarrow +\infty} c \sqrt{\dfrac{c^2}{g^2}+\dfrac{1}{4}}- \dfrac{c^2}{g} = \lim\limits _{c \rightarrow +\infty} \dfrac{c\left (\dfrac{c^2}{g^2}+\dfrac{1}{4} - \dfrac{c^2}{g^2}\right )}{\sqrt{\dfrac{c^2}{g^2}+\dfrac{1}{4}} + \dfrac{c}{g}} = \lim\limits _{c \rightarrow +\infty} \dfrac{\dfrac{1}{4}}{\sqrt{\dfrac{1}{g^2} + \dfrac{1}{4c^2}}+ \dfrac{1}{g}} = \dfrac{g}{8}.
\]
}
\end{vd}



\subsubsection{Bài tập rèn luyện}
\begin{bt}%[DCHT Toán 11 - KNTT- Phạm Tuấn]%[1K5YF-8] 
Thế Lennard-Jones có dạng $$U(r) = \dfrac{B}{r^{12}} - \dfrac{A}{r^6}$$ trong đó $A$, $B$ là các hằng số và $r$ là khoảng cách giữa các hạt. 
Tính $\lim\limits _{r \rightarrow +\infty} U(r)$.
\dapso{$\lim\limits _{r \rightarrow +\infty} U(r) =0$}
\loigiai{
Ta có 
$$\lim\limits _{r \rightarrow +\infty} U(r) =\lim\limits _{r \rightarrow +\infty} \left (\dfrac{B}{r^{12}} - \dfrac{A}{r^6} \right )  =0.$$
}
\end{bt}

\begin{bt}%[DCHT Toán 11 - KNTT- Phạm Tuấn]%[1K5YF-8] 
Trong thuyết tương đối, chiều dài của một vật thể đối với người quan sát phụ thuộc vào tốc độ mà vật thể đang chuyển động đối với người quan sát. Nếu người quan sát đo chiều dài của vật thể là $L_0$ khi đứng yên, thì ở tốc độ $v$ chiều dài  là
$$
L=L_0 \sqrt{1-\frac{v^2}{c^2}}
$$
trong đó $c$ là tốc độ ánh sáng trong chân không. Tìm $\displaystyle \lim _{v \rightarrow c^{-}} L$. 
\dapso{$\displaystyle \lim _{v \rightarrow c^{-}} L =0$}
\loigiai{
Ta có $\displaystyle \lim _{v \rightarrow c^{-}} L = \lim _{v \rightarrow c^{-}} L_0 \sqrt{1-\frac{v^2}{c^2}} =  \lim _{v \rightarrow c^{-}} L_0 \sqrt{1-\frac{c^2}{c^2}} =0$. 
}
\end{bt}

\begin{bt}%[DCHT Toán 11 - KNTT- Phạm Tuấn]%[1K5BF-8] 
Trong kỹ thuật ứng dụng, chúng ta thường xuyên ghi nhận được các hàm số mà giá trị của nó thay đổi đột ngột tại một thời điểm $t$ xác định. Ví dụ:  Sự thay đổi điện áp của một mạch điện tại thời điểm t khi đóng hoặc ngắt mạch. Thông thường, giá trị t = 0 luôn được chọn là thời điểm bắt đầu cho việc đóng hoặc ngắt điện áp. Quá trình đóng, ngắt mạch trên có thể mô tả bằng mô hình toán học bởi hàm Heaviside
\[
u(t) = \heva{&0 && \text{ nếu } t <0\\& 1 && \text{ nếu } t \geq 0.}
\]
Có tồn tại giới hạn $\displaystyle \lim \limits_{t\to 0} u(t)$ hay không?
\dapso{$\displaystyle \lim \limits_{t\to 0} u(t)$ không tồn tại}
\loigiai{
Ta có $\displaystyle \lim \limits_{t\to 0^+} u(t) = \lim \limits_{t\to 0^+} 1 =1$; $\displaystyle \lim \limits_{t\to 0^-} u(t) = \lim \limits_{t\to 0^-} 0 =0$. \\
Vậy giới hạn $\displaystyle \lim \limits_{t\to 0} u(t)$ không tồn tại.
}
\end{bt}

\begin{bt}%[DCHT Toán 11 - KNTT- Phạm Tuấn]%[1K5BF-8] 
Trong một cuộc thi các môn thể thao trên tuyết, người ta muốn thiết kế một đường trượt bằng băng cho nội dung đổ dốc tốc độ đường dài.
\begin{center}
\begin{tikzpicture}[scale=0.9, font=\footnotesize, line join=round, line cap=round, >=stealth]
\draw[->] (0,0)--(10,0) node[below]{$x$} ;
\draw[->] (0,0)--(0,4) node[left]{$y$} ;
\foreach \x in {5,10,15,20,25,30,35,40,45}
\draw[shift={({\x/5},0)},color=black] (0,0) -- (0pt,-2pt) node[below] {$\x$};
\draw (0,3) node[left]{$15$} (0,0) node[below left]{$O$};
\clip (0,0) rectangle (9,4) ;
\draw[thick,smooth,samples=100,domain=0:9] plot(\x,{9/(2*(\x)+3)}) ;
\end{tikzpicture}
\end{center}
Vận động viên sẽ xuất phát từ vị trí $(0 ; 15)$ cao $15$ m so với mặt đất (trục $Ox$). Đường trượt phải thoả mãn yêu cầu là càng ra xa thì càng gần mặt đất để tiết kiệm lượng tuyết nhân tạo. Một nhà thiết kế đề nghị sử dụng đường cong là đồ thị hàm số $y=f(x)=\dfrac{150}{x+10}$, với $x \geq 0$. Hãy kiểm tra xem hàm số $y=f(x)$ có thoả mãn các điều kiện dưới đây hay không:
\begin{enumerate}
\item    Có đồ thị qua điểm $(0; 15)$;
\item    Giảm trên $[0 ;+\infty)$;
\item    Càng ra xa ($x$ càng lớn), đồ thị của hàm số càng gần trục $O x$ với khoảng cách nhỏ tuỳ ý.
\end{enumerate}
\dapso{Đồ thị qua điểm $(0; 15)$; Hàm số giảm trên $[0 ;+\infty)$; $\lim\limits _{x \rightarrow +\infty} f(x) = 0$}
\loigiai{
\begin{enumerate}
\item    Ta có $f(0)= \dfrac{150}{10}$ nên đồ thị hàm số $f(x)$ đi qua điểm $(0; 15)$.
\item    Chọn bất kì $x_1,x_2 \in [0;+\infty]$ và $x_1 \ne x_2$. \\
Ta có $\dfrac{f(x_2)-f(x_1)}{x_2-x_1} =  \dfrac{\dfrac{150}{x_2+10} - \dfrac{150}{x_1+10}}{x_2-x_1} = \dfrac{x_1-x_2}{(x_2-x_1)(x_1+10)(x_2+10)} = -\dfrac{1}{(x_1+10)(x_2+10)} <0$. \\
Suy ra hàm số nghịch biến trên  $[0 ;+\infty)$ hay hàm số giảm trên $[0 ;+\infty)$.
\item   Ta có $\lim\limits _{x \rightarrow +\infty}  f(x) = \lim\limits _{x \rightarrow +\infty} \dfrac{150}{x+10} = 0$. \\
Vậy khi $x$ càng lớn, đồ thị của hàm số càng gần trục $O x$ với khoảng cách nhỏ tuỳ ý.
\end{enumerate}
}
\end{bt}

\begin{bt}%[DCHT Toán 11 - KNTT- Phạm Tuấn]%[1K5YF-8] 
Chiều dài một loài động vật nhỏ được tính theo công công thức $h(t)=\dfrac{100}{2+3 \cdot  (0{,}4)^t}$ mm, trong đó $t$ số ngày sau khi sinh của loài động vật đó. Tính chiều dài  cuối cùng của nó (chiều dài khi $t \to +\infty$).
\dapso{$100$ mm}
\loigiai{
Ta có $\displaystyle \lim \limits_{t \to +\infty } h(t)=\dfrac{100}{2+3 \cdot  (0{,}4)^t} = 100$. \\
Vậy chiều dài của loài động vật khi trưởng thành là $100$ mm. 
}
\end{bt}

\begin{bt}%[DCHT Toán 11 - KNTT- Phạm Tuấn]%[1K5BF-8] 
Một chất điểm chuyển động thẳng với phương trình $s(t)$. Khi đó vận tốc tức thời tại thời điểm $t_0$ được định nghĩa là $\displaystyle \lim \limits_{\Delta t} \dfrac{s(t_0+ \Delta t) - s(t_0)}{\Delta t}$. Tính vận tốc tức thời của chất điểm với phương trình chuyển động $s(t) = 4t^2-3t+1$ ($s(t)$ có đơn vị là m, $t$ đơn vị là giây), tại thời điểm $t=8$ giây. 
\dapso{$61  \mathrm{~m/s}$}
\loigiai{
Vận tốc tức thời của chất điểm tại thời điểm $t=8$ giây là
\begin{align*}
\lim \limits_{\Delta t \to 0} \dfrac{s(8+\Delta t) - s(8)}{\Delta t} &=  \lim \limits_{\Delta t \to 0} \dfrac{4(8+\Delta t)^2-3(8+\Delta t)+1 - 233}{\Delta t} \\
& = \lim \limits_{\Delta t \to 0} \dfrac{4 (\Delta t)^2+ 61 \Delta t}{\Delta t} = 61  \mathrm{~m/s}.
\end{align*}
}
\end{bt}

\begin{bt}%[DCHT Toán 11 - KNTT- Phạm Tuấn]%[1K5YF-8] 
Bỏ qua lực cản của không khí, độ cao tối đa mà tên lửa đạt được khi phóng với vận tốc ban đầu $v_0$ là $h=\dfrac{v_0^2 R}{19{,}6 R-v_0^2}$, trong đó $R$ là bán kính của trái đất. Tính  $\displaystyle \lim \limits_{R \to +\infty} h$. 
\dapso{$\lim \limits_{R \to +\infty} h =  \dfrac{v_0^2}{19{,}6}$}
\loigiai{
Ta có 
\begin{align*}
\lim \limits_{R \to +\infty} h &= \displaystyle \lim \limits_{R \to +\infty} \dfrac{v_0^2 R}{19{,}6 R-v_0^2} \\
&= \lim \limits_{R \to +\infty} \dfrac{v_0^2}{19{,}6 - \dfrac{v_0^2}{R}} = \dfrac{v_0^2}{19{,}6}.
\end{align*}
}
\end{bt}


\begin{bt}%[DCHT Toán 11 - KNTT- Phạm Tuấn]%[1K5BF-8] 
Một hình elip với bán trục lớn $a$ và bán trục nhỏ $b$ thì diện tích được tính theo công thức $S=\pi ab$. Cho elip có bán trục nhỏ bằng $30$ cm, tính giới hạn diện tích của elip khi tiêu cự gần tới $0$.
\dapso{$900\pi \mathrm{~cm^2}$}
\loigiai{
Ta có $S= \pi ab = \pi b\sqrt{b^2+c^2}$. \\
Vậy $\lim\limits_{c \to 0} S= \lim\limits_{c \to 0} \pi b\sqrt{b^2+c^2} = \pi b^2 = 900\pi \mathrm{~cm^2}$. 
}
\end{bt}





\begin{bt}%[DCHT Toán 11 - KNTT- Phạm Tuấn]%[1K5BF-8] 
Số lượng đơn vị hàng tồn kho trong một công ty nhỏ được cho bởi
$$
N(t)=25\left(2 \left [\frac{t+2}{2}  \right ]-t\right)
$$
trong đó $t$ là thời gian tính bằng tháng, $[x]$ là số nguyên lớn nhất không vượt quá $x$ (ví dụ $[2{,}4]=2$, $[-2{,}7] = -3$).
\begin{enumerate}
\item Tính $\lim\limits_{t \to 8^+} N(t)$.
\item  Tính $\lim\limits_{t \to 16^-} N(t)$.
\end{enumerate}
\dapso{$\lim\limits _{t \to 8^+} N(t)=50$; $\lim\limits _{t \to 16^-} N(t) =0$}
\loigiai{
\begin{enumerate}
\item 
Khi $t \to 8^+$, ta có $\left [\dfrac{t+2}{2}  \right ] = 5$. \\
Suy ra  $\lim\limits _{t \to 8^+} N(t) =\lim_{t \to 8^+}  25\left(2 \left [\frac{t+2}{2}  \right ]-t\right) = 50$.
\item  
Khi $t \to 16^-$, ta có $\left [\dfrac{t+2}{2}  \right ] = 8$. \\
Suy ra  $\lim\limits _{t \to 16^-} N(t) = \lim_{t \to 16^-} 25\left(2 \left [\frac{t+2}{2}  \right ]-t\right) = 0$.
\end{enumerate}
}
\end{bt}


\begin{bt}%[DCHT Toán 11 - KNTT- Phạm Tuấn]%[1K5BF-8] 
Định luật Boyle được phát biểu:  ``Đối với một lượng khí ở nhiệt độ không đổi, áp suất $P$ tỷ lệ nghịch với thể tích $V$''. Tìm giới hạn của $P$ là $V \rightarrow 0^{+}$.
\dapso{$\lim \limits_{V \to 0^+} P = +\infty$}
\loigiai{
Ta có $P= \dfrac{k}{V}$ với $k$ là số thực dương không đổi. \\
Khi đó $\lim \limits_{V \to 0^+} P= \dfrac{k}{V} = +\infty$.
}
\end{bt}

\begin{bt}%[DCHT Toán 11 - KNTT- Phạm Tuấn]%[1K5BF-8] 
Một vật khối lượng $m$ (không đổi) bắt đầu chuyển động với vận tốc $v_0=0$, được gia tốc bởi một lực $F$ không đổi trong $t$ giây. Theo định luật Newton về chuyển động, vật tốc của vật là $v_N = \dfrac{Ft}{m}$. Theo thuyết tương đối Einstein, vật có vận tốc $v_E = \dfrac{Fct}{\sqrt{m^2c^2+F^2t^2}}$, với $c$ là vận tốc ánh sáng. Tính $\displaystyle \lim \limits_{t \to +\infty} v_N$ và $\displaystyle \lim \limits_{t \to +\infty} v_E$.
\dapso{$\lim \limits_{t \to +\infty} v_N  = +\infty$; $\lim \limits_{t \to +\infty} v_E  = \dfrac{c}{F}$}
\loigiai{
Ta có
\begin{align*}
& \lim_{t \to +\infty} v_N = \lim_{t \to +\infty}\dfrac{Ft}{m} = +\infty; \\
& \lim_{t \to +\infty} v_E = \lim_{t \to +\infty}\dfrac{Fct}{\sqrt{m^2c^2+F^2t^2}} =  \lim_{t \to +\infty} \dfrac{Fc}{\sqrt{\dfrac{m^2c^2}{t^2}}+F^2} = \dfrac{c}{F}.
\end{align*}
}
\end{bt}


\begin{bt}%[DCHT Toán 11 - KNTT- Phạm Tuấn]%[1K5BF-8] 
\immini{
Gọi $S$ là diện tích hình phẳng giới hạn bởi đường tròn bán kính $10$ và tam giác vuông (hình vẽ bên).
\begin{enumerate}
\item Đặt $S= f(\varphi)$, với $f(\varphi)$ là hàm số của $\varphi$ (Đơn vị rad). Tìm công thức của $f(\varphi)$ với $0< \varphi < \dfrac{\pi}{2}$.
\item Tính giới hạn của $f(\varphi)$ khi $\varphi \to \dfrac{\pi}{2}^-$.
\end{enumerate}
}
{
\begin{tikzpicture}[scale=1, font=\footnotesize, line join=round, line cap=round, >=stealth]
\path
(0,0) coordinate (A)
(4,0) coordinate (B)
(0,2.5) coordinate (C)
($(B)!{4/sqrt(4^2+2.5^2)}!(C)$)  coordinate (D)
;
\fill[cyan!30] (A) arc (180:{180-atan(2.5/4)}:4) -- (C)--cycle;
\draw (A) arc (180:{180-atan(2.5/4)}:4)  ;
\draw (A)--(B)--(C)--(A);
\draw pic["$\varphi$", draw=black, angle eccentricity=1.3, angle radius=0.7cm]{angle=C--B--A} ;
\foreach \x/\g in {A/-120,B/-30,C/90,D/50} 
\fill[black](\x) circle (1pt)+(\g:2.5mm) node{$\x$};
\end{tikzpicture}
}
\dapso{$S=8(\tan \varphi - \varphi)$; $\lim \limits_{\varphi \to \tfrac{\pi}{2}^-}  f(\varphi) =+\infty $}
\loigiai{
\begin{enumerate}
\item Diện tích tam giác $ABC$ là $S_{ABC} = \dfrac{1}{2} AB \cdot AC = \dfrac{1}{2} \cdot 4 \cdot 4 \tan \varphi = 8  \tan \varphi$. \\
Diện tích hình quạt $ABD$ là $S_q = \dfrac{4^2 \varphi }{2} = 8 \varphi$. \\
Diện tích hình phẳng $S$ là $S=f(\varphi) = S_{ABC} - S_q = 8(\tan \varphi - \varphi)$.
\item 
Khi $\varphi \to \tfrac{\pi}{2}^-$ thì $\heva{&\cos \varphi \to 0\\& \cos \varphi >0.}$\\
Suy ra $\displaystyle \lim \limits_{\varphi \to \tfrac{\pi}{2}^-}  f(\varphi) = \lim \limits_{\varphi \to \tfrac{\pi}{2}^-}  8\left (\dfrac{\sin \varphi}{\cos \varphi} - \varphi \right )= +\infty$.
\end{enumerate}
}
\end{bt}

\begin{bt}%[DCHT Toán 11 - KNTT- Phạm Tuấn]%[1K5BF-8] 
Trên một chuyến đi dài $d$ km đến một thành phố khác, vận tốc trung bình của một tài xế xe tải là $x$ km/h. Trên chuyến trở về, vận tốc trung bình là $y$ km/h. Vận tốc trung bình của cả đi và về là 50 km/h.
\begin{enumerate}
\item Chứng minh rằng $y= \dfrac{25x}{x-25}$.
\item Tìm giới hạn của $y$ khi $x \rightarrow 25^{+}$ và giải thích ý nghĩa của nó.
\end{enumerate}
\dapso{$\displaystyle \lim\limits_{x\to 25^+}  y = +\infty$}
\loigiai{
\begin{enumerate}
\item  Thời gian  chuyến đi là $\dfrac{d}{x}$, thời gian  chuyến trở về là $\dfrac{d}{y}$. \\
Suy ra 
$$\dfrac{2d}{50} = \dfrac{d}{x} + \dfrac{d}{y} \Leftrightarrow \dfrac{1}{y} = \dfrac{1}{25} - \dfrac{1}{x} \Leftrightarrow y= \dfrac{25x}{x-25}.$$
\item  Ta có $\displaystyle \lim\limits_{x\to 25^+}  y = \lim\limits_{x\to 25^+}  \dfrac{25x}{x-25} = +\infty$. \\
Khi vận tốc trung bình chuyến đi bằng 25 km/h,  thì  vận tốc trung bình chuyến của cả chuyến đi và về không thể là 50 km/h.
\end{enumerate}
}
\end{bt}

\begin{bt}%[DCHT Toán 11 - KNTT- Phạm Tuấn]%[1K5BF-8] 
Một chất điểm chuyển động thẳng với vận tốc $v(t)$. Khi đó gia tốc tức thời tại thời điểm $t_0$ được định nghĩa là $\displaystyle \lim \limits_{\Delta t \to 0} \dfrac{v(t_0+ \Delta t) - v(t_0)}{\Delta t}$. Một chất điểm chuyển động với vận tốc $v(t) = 5 \sin \left (4\pi t\right )$ (m/s), tính gia tốc tức thời tại thời điểm $t=5$ giây.  (Biết $\displaystyle \lim_{x \to 0} \dfrac{\sin x}{x} =1$). 
\dapso{$20\pi$ $\mathrm{m/s^2}$}
\loigiai{
Gia tốc tức thời của chất điểm tại thời điểm $t=5$ giây là
\begin{align*}
\lim \limits_{\Delta t \to 0} \dfrac{v(5+\Delta t) - s(5)}{\Delta t} &=  \lim \limits_{\Delta t \to 0} \dfrac{5\sin (20\pi+4\pi\Delta t) - 5\sin (20\pi)}{\Delta t} \\
& = \lim \limits_{\Delta t \to 0} \dfrac{5\sin (4\pi\Delta t)}{\Delta t}   \\
& = \lim \limits_{\Delta t \to 0} \dfrac{20\pi\sin (4\pi\Delta t)}{4\pi\Delta t}   \\
& = 20\pi.
\end{align*}
Vậy gia tốc tức thời của chất điểm tại thời điểm $t=5$ giây là $20\pi$ ($\mathrm{m/s^2}$).
}
\end{bt}


\begin{bt}%[DCHT Toán 11 - KNTT- Phạm Tuấn]%[1K5BF-8] 
Một bể chứa $5000$ lít nước tinh khiết. Nước muối chứa $30$ g muối trên một lít nước được bơm vào bể với tốc độ $25$ lít/phút. Gọi nồng độ của muối sau $t$ phút (tính bằng gam trên lít) là $C(t)$. Tính $\displaystyle \lim \limits_{t \to +\infty} C(t)$.  Giải thích ý nghĩa của giới hạn này.
\dapso{$\lim \limits_{t \to +\infty} C(t) =30$}
\loigiai{
Số lít nước muối được bơm vào bể sau $t$ phút là $25t$ lít. \\
Số g muối có trong $25t$ lít nước muối là $30 \cdot 25 t = 750t$ gam. \\
Nồng độ của muối trong bể sau $t$ phút  là
\[
\dfrac{750t}{25t + 5000} = \dfrac{30t}{t+200} \text{~gam/lít}.
\]
Ta có $\displaystyle \lim \limits_{t \to +\infty} C(t) = \lim \limits_{t \to +\infty} \dfrac{30t}{t+200} =  30  \text{~gam/lít}$. \\
Khi thời gian tiến tới vô hạn thì nồng độ của muối trong bể bằng nồng độ của nước muối bơm vào bể.
}
\end{bt}



\begin{bt}%[DCHT Toán 11 - KNTT- Phạm Tuấn]%[1K5BF-8] 
Một thấu kính hội tụ có tiêu cự $f=30 \mathrm{~cm}$. Trong Vật lí, ta biết rằng nếu đặt vật thật $A B$ cách quang tâm của thấu kính một khoảng $d>30$ ($\mathrm{cm}$) thì được ảnh thật $A' B'$ cách quang tâm của thấu kính một khoảng $d'$ (cm) (Hình vẽ dưới). Ngược lại, nếu $0<d<30$, ta có ảnh ảo. Công thức của thấu kính là $\dfrac{1}{d}+\dfrac{1}{d'}=\dfrac{1}{30}$.
\begin{center}
\begin{tikzpicture}[scale=1, font=\footnotesize, line join=round, line cap=round, >=stealth]
\path
(0,0) coordinate (O)
(-4,0) coordinate (A)
(8,0) coordinate (A')
(-4,1) coordinate (B)
(8,-2) coordinate (B')
(0,1) coordinate (B1)
({-8/3},1.4) coordinate (P)
(0,1.4) coordinate (Q)
({8/3},1.4) coordinate (R)
(-4,-1.4) coordinate (X)
(0,-1.4) coordinate (Y)
(8,-1.4) coordinate (Z)
(intersection of O--A' and B1--B')  coordinate (F')
($(F')!2!(O)$) coordinate (F)
;
\draw (-5,0)--(9,0) ;
\draw[<->,>=triangle 45,very thick] (0,-2.2)--(0,2.2);  
\draw[->,>=triangle 45,very thick] (A)--(B);  
\draw[->,>=triangle 45,very thick] (A')--(B');  
\draw[<->,dashed] (P)--(Q) ;
\draw[<->,dashed] (Q)--(R) ;
\draw[<->,dashed] (X)--(Y) ;
\draw[<->,dashed] (Y)--(Z) ;
\draw[dashed] (F)--(P) (F')--(R) (A)--(X) (A')--(Z);
\draw[fill] ($(P)!0.5!(Q)$) node[above]{$30$} ($(Q)!0.5!(R)$) node[above]{$30$}
($(X)!0.5!(Y)$) node[above]{$d$} ($(Y)!0.4!(Z)$) node[above]{$d'$} (F) circle(1.2pt) (F') circle(1.2pt);
\draw (B)--(B') (B)--(B1)--(B');
\foreach \x/\g in{A/-50,B/90,A'/90,B'/-90,O/-130,F/-90,F'/-90}
\fill[black](\x)  ($(\x)+(\g:3mm)$) node{$\x$}; 
\end{tikzpicture}
\end{center} 
\begin{enumerate}
\item Từ công thức của thấu kính, hãy tìm biểu thức xác định hàm số $d'=h(d)$.
\item Tìm các giới hạn $\lim\limits  _{d \rightarrow 30^{+}} h(d) ; \lim\limits _{d \rightarrow 30^{-}} h(d)$ và $\lim\limits _{d \rightarrow+\infty} h(d)$. Sử dụng các kết quả này để giải thích ý nghĩa đã biết trong Vật lí.
\end{enumerate}
\dapso{$d'=h(d) = \dfrac{30d}{d-30}$; $\lim\limits  _{d \rightarrow 30^{+}} h(d) = +\infty$; $\lim\limits  _{d \rightarrow 30^{-}} h(d) = -\infty$}
\loigiai{
\begin{enumerate}
\item Ta có $\dfrac{1}{d}+\dfrac{1}{d'}=\dfrac{1}{30} \Leftrightarrow \dfrac{1}{d'} = \dfrac{1}{30} - \dfrac{1}{d} = \dfrac{d-30}{30d} \Leftrightarrow d'= \dfrac{30d}{d-30}$. \\
Vậy $d'=h(d) = \dfrac{30d}{d-30}$.
\item 
Khi $d\to 30^+$, ta có $d-30 \to 0$, $d-30 >0$ và $30d \to 900$.\\
Suy ra $\lim\limits _{d \rightarrow 30^{+}} h(d) = \lim\limits _{d \rightarrow 30^{+}} \dfrac{30d}{d-30} = +\infty$. \\
Khi $d\to 30^-$, ta có $d-30 \to 0$, $d-30 <0$ và $30d \to 900$.\\
Suy ra $\lim\limits _{d \rightarrow 30^{-}} h(d) = \lim\limits _{d \rightarrow 30^{-}} \dfrac{30d}{d-30} = -\infty$. \\
Ta có $\lim\limits _{d \rightarrow +\infty} h(d) =  \lim\limits _{d \rightarrow +\infty} \dfrac{30d}{d-30} = \lim\limits _{d \rightarrow +\infty}  \dfrac{30}{1-\dfrac{30}{d}} =30$. \\
Vậy 
\begin{itemize}
\item  Khi vị trí của vật tiến gần  tiêu điểm $F$ ($d >f$) thì vị trí ảnh thật của vật dần ra xa vô cực. 
\item  Khi vị trí của vật tiến gần  tiêu điểm $F$ ($d <f$) thì vị trí ảnh ảo của vật dần ra xa vô cực. 
\item  Khi vị trí của vật tiến ra xa vô cực thì  ảnh thật của vật dần tới tiêu điểm. 
\end{itemize}
\end{enumerate}
}
\end{bt}


