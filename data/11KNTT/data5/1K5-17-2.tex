%\subsection{CÁC DẠNG TOÁN THƯỜNG GẶP}
\begin{dang}{Hàm số liên tục tại một điểm}
	Để kiểm tra tính liên tục của hàm số $y=f(x)$ tại điểm $x=x_0$ ta cần làm như sau:
	\begin{itemize}
		\item Bước 1: Tính $\mathop {\lim \limits_{n \to +\infty}}\limits_{x \to {x_0}} f\left( x \right).$
		\item Bước 2: Tính $= f\left( {{x_0}} \right).$
		Nếu $\mathop {\lim \limits_{n \to +\infty}}\limits_{x \to {x_0}} f\left( x \right) = f\left( {{x_0}} \right)$ thì kết luận hàm số $f(x)$ liên tục tại $x=x_0.$
		Nếu $\mathop {\lim \limits_{n \to +\infty}}\limits_{x \to {x_0}} f\left( x \right) \ne f\left( {{x_0}} \right)$ thì kết luận hàm số $f(x)$ liên tục tại $x=x_0.$
	\end{itemize}
\end{dang}
\subsubsection{Ví dụ minh hoạ}
\begin{vd}[NB]%[DCHT Toán 11 - KNTT -Hứa Chí Ninh] %[1K5BG-3]
	Biết rằng $\lim\limits_{x\to 0} \dfrac{\sin x}{x}=1.$ Hàm số $f\left( x \right)=\left\{ \begin{array}{*{35}{l}}
		\dfrac{\tan x}{x} & \text{khi }x\ne 0 \\
		0 & \text{khi }x=0 \\
	\end{array} \right.$. Xét tính liên tục của $y=f(x)$ tại $x=0$? \dapso{ $f\left( x \right)$ không liên tục tại $x=0$.}
	\loigiai{	Tập xác định $\mathscr{D}=\mathbb{R}\setminus \left\{ \dfrac{\pi }{2}+k\pi |k\in \mathbb{Z} \right\}$.\\
		Ta có $\lim\limits_{x\to 0} f\left( x \right)=\lim\limits_{x\to 0} \dfrac{\tan x}{x}=\lim\limits_{x\to 0} \dfrac{\sin x}{x}\cdot\dfrac{1}{\cos x}=1\cdot\dfrac{1}{\cos 0}=1\ne f\left( 0 \right)\Rightarrow $ $f\left( x \right)$ không liên tục tại $x=0$.}
	\end{vd}
	\begin{vd}[TH]%[DCHT Toán 11 - KNTT -Hứa Chí Ninh] %[1K5BG-3]
		Hàm số $f\left( x \right)=\left\{ \begin{array}{*{35}{l}}
		3 & \text{khi}\,\,x=-1 \\
		\dfrac{x^4+x}{x^2+x} & \text{khi}\,\,x\ne -1,\,\,x\ne 0 \\
		1 & \text{khi}\,\,x=0 \\
	\end{array} \right.$. Xét tính liên tục của hàm số tại $x=-1, x=0.$
		\dapso{Hàm số liên tục tại $x=-1, x=0.$}
		\loigiai{
			Hàm số $y=f\left( x \right)$ có tập xác định $\mathscr{D}=\mathbb{R}$.\\
		Dễ thấy hàm số $y=f\left( x \right)$ liên tục trên mỗi khoảng $\left( -\infty ;-1 \right),\left( -1;0 \right)$ và $\left( 0;+\infty \right)$.\\
		(i) Xét tại $x=-1$, ta có\\
		$\lim\limits_{x\to -1} f\left( x \right)=\lim\limits_{x\to -1} \dfrac{x^4+x}{x^2+x}=\lim\limits_{x\to -1} \dfrac{x\left( x+1 \right)\left( {x^2}-x+1 \right)}{x\left( x+1 \right)}=\lim\limits_{x\to -1} \left( {x^2}-x+1 \right)=3=f\left( -1 \right).$ Vậy hàm số $y=f\left( x \right)$ liên tục tại $x=-1$.\\
		(ii) Xét tại $x=0$, ta có\\
		$\lim\limits_{x\to 0} f\left( x \right)=\lim\limits_{x\to 0} \dfrac{x^4+x}{x^2+x}=\lim\limits_{x\to 0} \dfrac{x\left( x+1 \right)\left( {x^2}-x+1 \right)}{x\left( x+1 \right)}=\lim\limits_{x\to 0} \left( {x^2}-x+1 \right)=1=f\left( 0 \right).$ Vậy hàm số $y=f\left( x \right)$ liên tục tại $x=0$.
		}
	\end{vd}
		\begin{vd}[TH]%[DCHT Toán 11 - KNTT -Hứa Chí Ninh] %[1K5BG-3]
		Tìm số điểm gián đoạn của hàm số $f\left( x \right)=\left\{ \begin{array}{*{35}{l}}
		0,5 & \text{khi}\,\,x=-1 \\
		\dfrac{x\left( x+1 \right)}{x^2-1} & \text{khi}\,\,x\ne -1,\,\,x\ne 1 \\
		1 & \text{khi}\,\,x=1 \\
	\end{array} \right.$?
		\dapso{Hàm số $y=f\left( x \right)$ gián đoạn tại $x=1$.}
		\loigiai{
			Hàm số $y=f\left( x \right)$ có tập xác định $\mathscr{D}=\mathbb{R}$.\\
			Hàm số $f\left( x \right)=\dfrac{x\left( x+1 \right)}{x^2-1}$ liên tục trên mỗi khoảng $\left( -\infty ;-1 \right)$, $\left( -1;1 \right)$ và $\left( 1;+\infty \right)$.\\
			(i) Xét tại $x=-1$, ta có $\lim\limits_{x\to -1} f\left( x \right)=\lim\limits_{x\to -1} \dfrac{x\left( x+1 \right)}{x^2-1}=\lim\limits_{x\to -1} \dfrac{x}{x-1}=\dfrac{1}{2}=f\left( -1 \right)\Rightarrow $ Hàm số liên tục tại $x=-1.$\\
			(ii) Xét tại $x=1$, ta có $\left\{ \begin{aligned}
				& \lim\limits_{x\to {1^{+}}} f\left( x \right)=\lim\limits_{x\to {1^{+}}} \dfrac{x\left( x+1 \right)}{x^2-1}=\lim\limits_{x\to {1^{+}}} \dfrac{x}{x-1}=+\infty \\
				& \lim\limits_{x\to {1^{-}}} f\left( x \right)=\lim\limits_{x\to {1^{-}}} \dfrac{x\left( x+1 \right)}{x^2-1}=\lim\limits_{x\to {1^{-}}} \dfrac{x}{x-1}=-\infty \\
			\end{aligned} \right.\Rightarrow $Hàm số $y=f\left( x \right)$ gián đoạn tại $x=1$.
		}
	\end{vd}
		\begin{vd}[TH]%[DCHT Toán 11 - KNTT -Hứa Chí Ninh] %[1K5BG-3]
		Xét tính liên tục của hàm số $f\left( x \right)=\left\{ \begin{array}{*{35}{l}}
		1-\cos x & \text{khi }x\le 0 \\
		\sqrt{x+1} & \text{khi }x>0 \\
	\end{array} \right.$ tại $x=0?$
	\dapso{Hàm số $y=f\left( x \right)$ gián đoạn tại $x=0$.}
	\loigiai{
			Hàm số xác định với mọi $x\in \mathbb{R}$.\\
		Ta có $f\left( x \right)$ liên tục trên $\left( -\infty ;0 \right)$ và $\left( 0;+\infty \right).$\\
		Mặt khác $\left\{ \begin{aligned}
			& f\left( 0 \right)=1 \\
			& \lim\limits_{x\to {0^{-}}} f\left( x \right)=\lim\limits_{x\to {0^{-}}} \left( 1-\cos x \right)=1-\cos 0=0 \\
			& \lim\limits_{x\to {0^{+}}} f\left( x \right)=\lim\limits_{x\to {0^{+}}} \sqrt{x+1}=\sqrt{0+1}=1 \\
		\end{aligned} \right.\Rightarrow f\left( x \right)$ gián đoạn tại $x=0$.
	}
\end{vd}
		\begin{vd}[TH]%[DCHT Toán 11 - KNTT -Hứa Chí Ninh] %[1K5BG-3]
	Cho hàm số $f\left( x \right)=\left\{ \begin{aligned}
	& \dfrac{x^2}{x}\text{ khi }x<1,x\ne 0 \\
	& 0\text{ khi }x=0 \\
	& \sqrt{x}\text{ khi }x\ge 1 \\
\end{aligned} \right..$ Xét tính liên tục của hàm số $f\left( x \right)$ tại $x=0, x=1?$
	\dapso{Hàm số $y=f\left( x \right)$ liên tục tại $x=0$ và $x=1$.}
	\loigiai{
		Hàm số $y=f\left( x \right)$ có tập xác định $\mathscr{D}=\mathbb{R}$.\\
	Dễ thấy hàm số $y=f\left( x \right)$ liên tục trên mỗi khoảng $\left( -\infty ;0 \right),\left( 0;1 \right)$ và $\left( 1;+\infty \right)$.\\
	Ta có $\left\{ \begin{aligned}
		& f\left( 0 \right)=0 \\
		& \lim\limits_{x\to {0^{-}}} f\left( x \right)=\lim\limits_{x\to {0^{-}}} \dfrac{x^2}{x}=\lim\limits_{x\to {0^{-}}} x=0 \\
		& \lim\limits_{x\to {0^{+}}} f\left( x \right)=\lim\limits_{x\to {0^{+}}} \dfrac{x^2}{x}=\lim\limits_{x\to {0^{+}}} x=0 \\
	\end{aligned} \right.\Rightarrow f\left( x \right)$ liên tục tại $x=0.$\\
	Ta có $\left\{ \begin{aligned}
		& f\left( 1 \right)=1 \\
		& \lim\limits_{x\to {1^{-}}} f\left( x \right)=\lim\limits_{x\to {1^{-}}} \dfrac{x^2}{x}=\lim\limits_{x\to {1^{-}}} x=1 \\
		& \lim\limits_{x\to {1^{+}}} f\left( x \right)=\lim\limits_{x\to {1^{+}}} \sqrt{x}=1 \\
	\end{aligned} \right.\Rightarrow f\left( x \right)$ liên tục tại $x=1.$
	}
\end{vd}
\subsubsection{Bài tập rèn luyện}
	\centerline{\fcolorbox{red}{yellow!50}{\bf {BÀI TẬP TỰ LUẬN}}}
		\begin{bt}[TH]%[DCHT Toán 11 - KNTT -Hứa Chí Ninh] %[1K5BG-3]
		
		Cho hàm số $f\left( x \right)=\left\{ \begin{aligned}
			& \dfrac{x^2-1}{x-1}\text{ khi}\,\,x<3,\,\,x\ne 1 \\
			& 4\text{ khi}\,\,x=1 \\
			& \sqrt{x+1}\text{ khi}\,\,x\ge 3 \\
		\end{aligned} \right.$. Xét tính liên tục của hàm số $f\left( x \right)$?
		
			\dapso{Hàm số $f(x)$ gián đoạn tại 2 điểm $x=1$ và $x=3$.}
		\loigiai{
			Hàm số $y=f\left( x \right)$ có tập xác định $\mathscr{D}=\mathbb{R}$.\\
			Dễ thấy hàm số $y=f\left( x \right)$ liên tục trên mỗi khoảng $\left( -\infty ;1 \right),\left( 1;3 \right)$ và $\left( 3;+\infty \right)$.\\
			Ta có $\left\{ \begin{aligned}
				& f\left( 1 \right)=4 \\
				& \lim\limits_{x\to 1} f\left( x \right)=\lim\limits_{x\to 1} \dfrac{x^2-1}{x-1}=\lim\limits_{x\to 1} \left( x+1 \right)=2 \\
			\end{aligned} \right.\Rightarrow f\left( x \right)$ gián đoạn tại $x=1.$\\
			Ta có $\left\{ \begin{aligned}
				& f\left( 3 \right)=2 \\
				& \lim\limits_{x\to {3^{-}}} f\left( x \right)=\lim\limits_{x\to {3^{-}}} \dfrac{x^2-1}{x-1}=\lim\limits_{x\to {3^{-}}} \left( x+1 \right)=4 \\
			\end{aligned} \right.\Rightarrow f\left( x \right)$ gián đoạn tại $x=3$.}
	\end{bt}
	\begin{bt}[TH]%[DCHT Toán 11 - KNTT -Hứa Chí Ninh] %[1K5BG-3]
		
		Tìm điểm gián đoạn của hàm số $h\left( x \right)=\left\{ \begin{aligned}
			& 2x\text{    khi }x<0 \\
			& {x^2}+1\text{  khi }0\le x\le 2 \\
			& 3x-1\text{  khi }x>2 \\
		\end{aligned} \right.$?
	\dapso{Hàm số gián đoạn tại điểm $x=0$}
		\loigiai{
			Hàm số $y=h\left( x \right)$ có tập xác định $\mathscr{D}=\mathbb{R}$.\\
			Dễ thấy hàm số $y=h\left( x \right)$ liên tục trên mỗi khoảng $\left( -\infty ;0 \right),\left( 0;2 \right)$ và $\left( 2;+\infty \right)$.\\
			Ta có $\left\{ \begin{aligned}
				& h\left( 0 \right)=1 \\
				& \lim\limits_{x\to {0^{-}}} h\left( x \right)=\lim\limits_{x\to {0^{-}}} 2x=0 \\
			\end{aligned} \right.\Rightarrow f\left( x \right)$ không liên tục tại $x=0$.\\
			Ta có $\left\{ \begin{aligned}
				& h\left( 2 \right)=5 \\
				& \lim\limits_{x\to {2^{-}}} h\left( x \right)=\lim\limits_{x\to {2^{-}}} \left( {x^2}+1 \right)=5 \\
				& \lim\limits_{x\to {2^{+}}} h\left( x \right)=\lim\limits_{x\to {2^{+}}} \left( 3x-1 \right)=5 \\
			\end{aligned} \right.\Rightarrow f\left( x \right)$ liên tục tại $x=2$.}
	\end{bt}
	\begin{bt}[TH]%[DCHT Toán 11 - KNTT -Hứa Chí Ninh] %[1K5BG-3]
		
		Cho hàm số $f\left( x \right)=\left\{ \begin{aligned}
			& -x\cos x\text{ khi}\,\,x<0 \\
			& \dfrac{x^2}{1+x}\text{ khi}\,\,0\le x<1 \\
			& {x^3}\text{ khi}\,\,x\ge 1 \\
		\end{aligned} \right..$ Hàm số $f\left( x \right)$ gián đoạn tại điểm nào?
	\dapso{Hàm số gián đoạn tại điểm $x=1.$}
		\loigiai{
			Hàm số $y=f\left( x \right)$ có tập xác định $\mathscr{D}=\mathbb{R}$.\\
			Dễ thấy $f\left( x \right)$ liên tục trên mỗi khoảng $\left( -\infty ;0 \right),\left( 0;1 \right)$ và $\left( 1;+\infty \right)$.\\
			Ta có $\left\{ \begin{aligned}
				& f\left( 0 \right)=0 \\
				& \lim\limits_{x\to {0^{-}}} f\left( x \right)=\lim\limits_{x\to {0^{-}}} \left( -x\cos x \right)=0 \\
				& \lim\limits_{x\to {0^{+}}} f\left( x \right)=\lim\limits_{x\to {0^{+}}} \dfrac{x^2}{1+x}=0 \\
			\end{aligned} \right. \Rightarrow f\left( x \right)$ liên tục tại $x=0$.\\
			Ta có $\left\{ \begin{aligned}
				& f\left( 1 \right)=1 \\
				& \lim\limits_{x\to {1^{-}}} f\left( x \right)=\lim\limits_{x\to {1^{-}}} \dfrac{x^2}{1+x}=\dfrac{1}{2} \\
				& \lim\limits_{x\to {1^{+}}}{\mathop{\,\,\lim \limits_{n \to +\infty}}}\,f\left( x \right)=\,\lim\limits_{x\to {1^{+}}}{\mathop{\lim \limits_{n \to +\infty}{x^3}=1}}\, \\
			\end{aligned} \right. \Rightarrow f\left( x \right)$ không liên tục tại $x=1$.\\
		}
	\end{bt}
\begin{bt}[VD]%[1K5BG-3]% 
	
	Cho hàm số $y=\left\{ \begin{aligned}
		& \dfrac{1-x^3}{1-x},\text{khi  }x<1 \\
		& 1\text{  },\text{khi  }x\ge 1 \\
	\end{aligned} \right.$. Xét tính liên tục phải của hàm số tại $x=1$?
\dapso{Hàm số liên tục phải tại $x=1.$}
	\loigiai{
		Ta có   $y\left( 1 \right)=1$.\\
		Ta có   $\lim\limits_{x\to {1^{+}}} y=1$; $\lim\limits_{x\to {1^{-}}} y=\lim\limits_{x\to {1^{-}}} \dfrac{1-x^3}{1-x}=\lim\limits_{x\to {1^{-}}} \dfrac{\left( 1-x \right)\left( 1+x+x^2 \right)}{1-x}=\lim\limits_{x\to {1^{-}}} \left( 1+x+x^2 \right)=4$.\\
		Nhận thấy $\lim\limits_{x\to {1^{+}}} y=y\left( 1 \right)$, suy ra $y$ liên tục phải tại $x=1$.}
\end{bt}
\begin{bt}[VD]%[1K5BG-3]% 
	
	Cho hàm số $y=\left\{ \begin{aligned}
		& \dfrac{x^2-7x+12}{x-3}\,\,\,\,\text{ khi}\,\,\,x\ne 3 \\
		& -1\,\,\,\,\,\,\,\,\,\,\,\,\,\,\,\,\,\,\,\,\,\,\,\,\,\text{ khi}\,\,x=3 \\
	\end{aligned} \right.$. Hàm số đã cho có đạo hàm tại $x=3$ không?
	\dapso{Hàm số đã cho có đạo hàm tại $x=3$}
	\loigiai{
		$\lim\limits_{x\to 3} \dfrac{x^2-7x+12}{x-3}=\lim\limits_{x\to 3} \left( x-4 \right)=-1=y\left( 3 \right)$ nên hàm số liên tục tại $x_0=3$.\\
		$\lim\limits_{x\to 3} \dfrac{\left( {x^2}-7x+12 \right)-\left( {3^2}-7.3+12 \right)}{x-3}=\lim\limits_{x\to 3} \dfrac{\left( {x^2}-7x+12 \right)}{x-3}=\lim\limits_{x\to 3} \left( x-4 \right)=-1\Rightarrow y'\left( 3 \right)=-1$.}
\end{bt}
\begin{bt}[VD]%[1K5KG-3]% 
	
	Cho hàm số $f\left( x \right)=\left\{ \begin{aligned}
		& \dfrac{x-2}{\sqrt{x+2}-2}\text{ khi }x\ne 2 \\
		& 4\text{         khi }x=2 \\
	\end{aligned} \right.$. Xét tính liên tục của hàm số tại $x=2$?
\dapso{Hàm số liên tục tại $x=2.$}
	\loigiai{
		Tập xác định  $\mathscr{D}=\mathbb{R}$.\\
		$\lim\limits_{x\to 2} f\left( x \right)=\lim\limits_{x\to 2} \dfrac{x-2}{\sqrt{x+2}-2}=\lim\limits_{x\to 2} \dfrac{\left( x-2 \right)\left( \sqrt{x+2}+2 \right)}{x-2}=\lim\limits_{x\to 2} \left( \sqrt{x+2}+2 \right)=4$.\\
		$f\left( 2 \right)=4$
		$\Rightarrow \lim\limits_{x\to 2} f\left( x \right)=f\left( 2 \right)$.
		Vậy hàm số liên tục tại $x=2$.}
\end{bt}
\begin{bt}[VD]%[1K5KG-3]% 
	
	Cho hàm số $f\left( x \right)=\left\{ \begin{aligned}
		& \dfrac{1-\cos x}{x^2}\,\,\,\text{khi }x\ne 0 \\
		& 1\,\,\,\,\,\,\,\,\,\,\,\,\,\,\,\,\,\,\,\text{khi}\,\,x=0 \\
	\end{aligned} \right.\,\,$. Xét tính liên tục của hàm số tại $x=0?$
\dapso{Hàm số gián đoạn tại $x=0.$}
	\loigiai{
		Hàm số xác định trên $\mathbb{R}$.\\
		Ta có $f\left( 0 \right)=1$ và $\lim\limits_{x\to 0} f\left( x \right)=\lim\limits_{x\to 0} \dfrac{1-\cos x}{x^2}=\lim\limits_{x\to 0} \dfrac{2{\sin ^2}\dfrac{x}{2}}{4\cdot{{\left( \dfrac{x}{2} \right)}^2}}=\dfrac{1}{2}.$\\
		Vì $f\left( 0 \right)\ne \lim\limits_{x\to 0} f\left( x \right)$ nên $f\left( x \right)$ gián đoạn tại $x=0$. Do đó $f\left( x \right)$không có đạo hàm tại $x=0$.\\
		Vì $\forall x\ne 0$, $f\left( x \right)=\dfrac{1-\cos x}{x^2}\ge 0$ nên $f\left( \sqrt{2} \right)>0.$ Vậy $f\left( x \right)$ gián đoạn tại $x=0$.}
\end{bt}
\begin{bt}[VD]%[1K5BG-4]%  
	
	Cho hàm số $f\left( x \right)=\left\{ \begin{aligned}
		& -x\cos x,x<0 \\
		& \dfrac{x^2}{1+x},0\le x<1 \\
		& {x^3},x\ge 1 \\
	\end{aligned} \right.$. Xét tính liên tục của hàm số tại $x=0?$
\dapso{Hàm số gián đoạn tại $x=1.$}
	\loigiai{
		$\bullet$ $f\left( x \right)$ liên tục tại $x\ne 0$ và $x\ne 1$.\\
		$\bullet$ Tại $x=0$\\
		$\lim\limits_{x\to {0^{-}}} f\left( x \right)=\lim\limits_{x\to {0^{-}}} \left( -x\cos x \right)=0$, $\lim\limits_{x\to {0^{+}}} f\left( x \right)=\lim\limits_{x\to {0^{+}}} \dfrac{x^2}{1+x}=0$, $f\left( 0 \right)=0$.\\
		Suy ra $\lim\limits_{x\to {0^{-}}} f\left( x \right)=\lim\limits_{x\to {0^{+}}} f\left( x \right)=f\left( 0 \right)$. Hàm số liên tục tại $x=0$.\\
		$\bullet$ Tại $x=1$\\
		$\lim\limits_{x\to {1^{-}}} f\left( x \right)=\lim\limits_{x\to {1^{-}}} \dfrac{x^2}{1+x}=\dfrac{1}{2}$, $\lim\limits_{x\to {1^{+}}} f\left( x \right)=\lim\limits_{x\to {1^{+}}} x^3=1$.\\
		Suy ra $\lim\limits_{x\to {1^{-}}} f\left( x \right)\ne \lim\limits_{x\to {1^{+}}} f\left( x \right)$. Hàm số gián đoạn tại $x=1$.}
\end{bt}
\subsubsection{Bài tập trắc nghiệm}
\Opensolutionfile{ans}[ans/ans-1K5-3-Dang2]
\begin{ex}%[1K5BG-3]% 
	
	Cho hàm số $f\left( x \right)=\dfrac{2x-1}{x^3-x}$. Kết luận nào sau đây đúng?
	\choice
	{ Hàm số liên tục tại $x=-1$}
	{ Hàm số liên tục tại $x=0$}
	{ Hàm số liên tục tại $x=1$}
	{\True Hàm số liên tục tại $x=\dfrac{1}{2}$}
	\loigiai{
		Tại $x=\dfrac{1}{2}$, ta có   $\lim\limits_{x\to \dfrac{1}{2}} f\left( x \right)=\lim\limits_{x\to \dfrac{1}{2}} \dfrac{2x-1}{x^3-1}=0=f\left( \dfrac{1}{2} \right)$. Vậy hàm số liên tục tại $x=2$.}
\end{ex}
\begin{ex}%[1K5BG-3]% 
	
	Hàm số nào sau đây liên tục tại $x=1$$\colon$ 
	\choice
	{ $f\left( x \right)=\dfrac{x^2+x+1}{x-1}$}
	{ $f\left( x \right)=\dfrac{x^2-x-2}{x^2-1}$}
	{\True $f\left( x \right)=\dfrac{x^2+x+1}{x}$}
	{ $f\left( x \right)=\dfrac{x+1}{x-1}$}
	\loigiai{
		$\bullet$ $f\left( x \right)=\dfrac{x^2+x+1}{x-1}$\\
		$\lim\limits_{x\to {1^{+}}} f\left( x \right)=+\infty $ suy ra $f\left( x \right)$ không liên tục tại $x=1$.\\
		$\bullet$ $f\left( x \right)=\dfrac{x^2-x-2}{x^2-1}$\\
		$\lim\limits_{x\to {1^{+}}} f\left( x \right)=\lim\limits_{x\to {1^{+}}} \dfrac{x-2}{x-1}=-\infty $ suy ra $f\left( x \right)$ không liên tục tại $x=1$.\\
		$\bullet$ $f\left( x \right)=\dfrac{x^2+x+1}{x}$\\
		$\lim\limits_{x\to 1} f\left( x \right)=\lim\limits_{x\to 1} \dfrac{x^2+x+1}{x}=3=f\left( 1 \right)$ suy ra $f\left( x \right)$ liên tục tại $x=1$.\\
		$\bullet$ $f\left( x \right)=\dfrac{x+1}{x-1}$\\
		$\lim\limits_{x\to {1^{+}}} f\left( x \right)=\lim\limits_{x\to {1^{+}}} \dfrac{x+1}{x-1}=+\infty $ suy ra $f\left( x \right)$ không liên tục tại $x=1$.}
\end{ex}
\begin{ex}%[1K5BG-3]% 
	
	Hàm số nào dưới đây gián đoạn tại điểm $x_0=-1$.
	\choice
	{ $y=\left( x+1 \right)\left( {x^2}+2 \right)$}
	{\True $y=\dfrac{2x-1}{x+1}$}
	{ $y=\dfrac{x}{x-1}$}
	{ $y=\dfrac{x+1}{x^2+1}$}
	\loigiai{
		Ta có $y=\dfrac{2x-1}{x+1}$ không xác định tại $x_0=-1$ nên gián đoạn tại $x_0=-1$.}
\end{ex}
\begin{ex}%[1K5BG-3]% 
	
	Hàm số nào sau đây gián đoạn tại $x=2$?
	\choice
	{\True $y=\dfrac{3x-4}{x-2}$}
	{ $y=\sin x$}
	{ $y=x^4-2x^2+1$}
	{ $y=\tan x$}
	\loigiai{
		Ta có   $y=\dfrac{3x-4}{x-2}$ có tập xác định $\mathscr{D}=\mathbb{R}\setminus \left\{ 2 \right\}$, do đó gián đoạn tại $x=2$.}
\end{ex}
\begin{ex}%[1K5BG-3]% 
	
	Hàm số $y=\dfrac{x}{x+1}$ gián đoạn tại điểm $x_0$ bằng?
	\choice
	{ $x_0=2018$}
	{ $x_0=1$}
	{ $x_0=0$}
	{\True $x_0=-1$}
	\loigiai{
		Vì hàm số $y=\dfrac{x}{x+1}$ có tập xác định $\mathscr{D}=\mathbb{R}\setminus \left\{ -1 \right\}$ nên hàm số gián đoạn tại điểm $x_0=-1$.}
\end{ex}
\begin{ex}%[1K5BG-3]%  
	
	Cho hàm số $y=\dfrac{x-3}{x^2-1}$. Mệnh đề nào sau đây đúng?
	\choice
	{\True Hàm số không liên tục tại các điểm $x=\pm 1$}
	{ Hàm số liên tục tại mọi $x\in \mathbb{R}$}
	{ Hàm số liên tục tại các điểm $x=-1$}
	{ Hàm số liên tục tại các điểm $x=1$}
	\loigiai{
		Hàm số $y=\dfrac{x-3}{x^2-1}$ có tập xác định $\mathbb{R}\setminus \left\{ \pm 1 \right\}$. Do đó hàm số không liên tục tại các điểm $x=\pm 1$.}
\end{ex}
\begin{ex}%[1K5BG-5]% 
	
	Để hàm số $y=\left\{ \begin{aligned}
		& {x^2}+3x+2\begin{matrix}
			{} & \text{khi}\begin{matrix}
				{} & x\le -1 \\
			\end{matrix} \\
		\end{matrix} \\
		& 4x+a\begin{matrix}
			{} & {} & \,\,\text{khi}\begin{matrix}
				{} & x>-1 \\
			\end{matrix} \\
		\end{matrix} \\
	\end{aligned} \right.$ liên tục tại điểm $x=-1$ thì giá trị của $a$ là
	\choice
	{ $-4$}
	{\True 4}
	{ 1}
	{ $-1$}
	\loigiai{
		Hàm số liên tục tại $x=-1$ khi và chỉ khi $\lim\limits_{x\to -{1^{+}}} y=\lim\limits_{x\to -{1^{-}}} y=y\left( -1 \right).$\\
		$\Leftrightarrow \lim\limits_{x\to -{1^{+}}} \left( 4x+a \right)=\lim\limits_{x\to -{1^{-}}} \left( {x^2}+3x+2 \right)=y\left( -1 \right)$ $\Leftrightarrow a-4=0\Leftrightarrow a=4$.}
\end{ex}
\begin{ex}%[1K5BG-5]% 
	
	Tìm giá trị thực của tham số $m$ để hàm số $f\left( x \right)=\left\{ \begin{aligned}
		& \dfrac{x^3-x^2+2x-2}{x-1}\ \ \ \ khi\ x\ne 1 \\
		& 3x+m\ \quad \quad \quad \quad \ \ khi\ x=1 \\
	\end{aligned} \right.$ liên tục tại $x=1$.
	\choice
	{\True $m=0$}
	{ $m=6$}
	{ $m=4$}
	{ $m=2$}
	\loigiai{
		Ta có   $f\left( 1 \right)=m+3$.\\
		$\lim\limits_{x\to 1} f\left( x \right)=\lim\limits_{x\to 1} \dfrac{x^3-x^2+2x-2}{x-1}=\lim\limits_{x\to 1} \dfrac{\left( x-1 \right)\left( {x^2}+2 \right)}{x-1}=\lim\limits_{x\to 1} \left( {x^2}+2 \right)=3$.\\
		Để hàm số $f\left( x \right)$ liên tục tại $x=1$ thì $\lim\limits_{x\to 1} f\left( x \right)=f\left( 1 \right)\Leftrightarrow 3=m+3\Leftrightarrow m=0$.}
\end{ex}
\begin{ex}%[1K5GG-5]%
	
	Cho hàm số $f\left( x \right)=\left\{ \begin{aligned}
		& \dfrac{{x^{2016}}+x-2}{\sqrt{2018\text{x}+1}-\sqrt{x+2018}}\,\,khi\,\,x\ne 1 \\
		& k\,\,\,\,\,\,\,\,\,\,\,\,\,\,\,\,\,\,\,\,\,\,\,\,\,\,\,\,\,\,\,\,\,\,\,\,\,\,\,\,\,\,\,\,\,\,\,\,\,\,khi\,\,x=1 \\
	\end{aligned} \right.$. Tìm $k$ để hàm số $f\left( x \right)$ liên tục tại $x=1$.
	\choice
	{\True $k=2\sqrt{2019}$}
	{ $k=\dfrac{2017.\sqrt{2018}}{2}$}
	{ $k=1$}
	{ $k=\dfrac{20016}{2017}\sqrt{2019}$}
	\loigiai{
		Ta có   $\lim\limits_{x\to 1} \dfrac{{x^{2016}}+x-2}{\sqrt{2018\text{x}+1}-\sqrt{x+2018}}=\lim\limits_{x\to 1} \dfrac{\left( {x^{2016}}-1+x-1 \right)\left( \sqrt{2018\text{x}+1}+\sqrt{x+2018} \right)}{2017\text{x}-2017}$\\
		$=\lim\limits_{x\to 1} \dfrac{\left( x-1 \right)\left( {x^{2015}}+{x^{2014}}+...+x+1+1 \right)\left( \sqrt{2018\text{x}+1}+\sqrt{x+2018} \right)}{2017\left( \text{x}-1 \right)}=2\sqrt{2019}$.\\
		Để hàm số liên tục tại $x=1$ $\Leftrightarrow \lim\limits_{x\to 1} f\left( x \right)=f\left( 1 \right)$ $\Leftrightarrow k=2\sqrt{2019}$.}
\end{ex}
\begin{ex}%[1K5KG-5]% 
	
	Cho hàm số $f\left( x \right)=\left\{ \begin{aligned}
		& \dfrac{\sqrt{x}-1}{x-1}\,\,khi\,\,x\ne 1 \\
		& a\,\,\,\,\,\,\,\,\,\,\,\,\,khi\,\,x=1 \\
	\end{aligned} \right.$. Tìm $a$ để hàm số liên tục tại $x_0=1$.
	\choice
	{ $a=0$}
	{ $a=-\dfrac{1}{2}$}
	{\True $a=\dfrac{1}{2}$}
	{ $a=1$}
	\loigiai{
		Ta có $\lim\limits_{x\to 1} f\left( x \right)=\lim\limits_{x\to 1} \dfrac{\sqrt{x}-1}{x-1}=\lim\limits_{x\to 1} \dfrac{\sqrt{x}-1}{\left( \sqrt{x}-1 \right)\left( \sqrt{x}+1 \right)}=\lim\limits_{x\to 1} \dfrac{1}{\sqrt{x}+1}=\dfrac{1}{2}$.\\
		Để hàm số liên tục tại $x_0=1$ khi $\lim\limits_{x\to 1} f\left( x \right)=f\left( 1 \right)\Leftrightarrow a=\dfrac{1}{2}$.}
\end{ex}
\begin{ex}%[1K5BG-5]% 
	
	Biết hàm số $f\left( x \right)=\left\{ \begin{aligned}
		& 3x+b\,\,\,khi\,\,x\le -1 \\
		& x+a\,\,\,\,khi\,\,x>-1 \\
	\end{aligned} \right.$ liên tục tại $x=-1$. Mệnh đề nào dưới đây đúng?
	\choice
	{\True $a=b-2$}
	{ $a=-2-b$}
	{ $a=2-b$}
	{ $a=b+2$}
	\loigiai{
		$\lim\limits_{x\,\to \,-{1^{-}}} f\left( x \right)=f\left( -1 \right)=b-3$; $\lim\limits_{x\,\to \,-{1^{+}}} f\left( x \right)=a-1$. Để hàm số liên tục tại $x=-1$ thì $b-3=a-1\Leftrightarrow a=b-2$.}
\end{ex}
\begin{ex}%[1K5KG-5]%
	
	Cho hàm số $f\left( x \right)=\left\{ \begin{aligned}
		& \dfrac{3-x}{\sqrt{x+1}-2}\text{ khi }x\ne 3 \\
		& m\text{        khi $x=3$} \\
	\end{aligned} \right.$. Hàm số đã cho liên tục tại $x=3$ khi $m$ bằng bao nhiêu?
	\choice
	{ $-1$}
	{ $1$}
	{ $4$}
	{\True $-4$}
	\loigiai{
		$f\left( 3 \right)=m$
		$\lim\limits_{x\to 3} f\left( x \right)=\lim\limits_{x\to 3} \dfrac{3-x}{\sqrt{x+1}-2}=\lim\limits_{x\to 3} \dfrac{\left( 3-x \right)\left( \sqrt{x+1}+2 \right)}{x-3}$ $=\lim\limits_{x\to 3} \left( -\sqrt{x+1}-2 \right)=-4.$\\
		Để hàm số liên tục tại $x=3$ thì $\lim\limits_{x\to 3} f\left( x \right)=f\left( 3 \right)$.\\
		Suy ra $m=-4$.}
\end{ex}

\begin{ex}%[1K5BG-5]%  
	
	Biết hàm số $f\left( x \right)=\left\{ \begin{matrix}
		a{x^2}+bx-5 & \text{khi} & x\le 1 \\
		2ax-3b & \text{khi} & x>1 \\
	\end{matrix} \right.$ liên tục tại $x=1$ Tính giá trị của biểu thức $P=a-4b$.
	\choice
	{ $P=-4$}
	{\True $P=-5$}
	{ $P=5$}
	{ $P=4$}
	\loigiai{
		Ta có $\lim\limits_{x\to {1^{-}}} f\left( x \right)=\lim\limits_{x\to {1^{-}}} \left( a{x^2}+bx-5 \right)=a+b-5=f\left( 1 \right)$.\\
		$\lim\limits_{x\to {1^{+}}} f\left( x \right)=\lim\limits_{x\to {1^{+}}} \left( 2ax-3b \right)=2a-3b$.
		Vì hàm số liên tục tại $x=1$ nên $a+b-5=2a-3b\Rightarrow a-4b=-5$.}
\end{ex}
\begin{ex}%[1K5BG-5]% 
	
	Tìm $m$ để hàm số $f(x)=\left\{ \begin{aligned}
		& {{\dfrac{x^2-x}{x-1}}} \text{ khi } x\ne 1 \\
		& m-1 \text { khi } \mathop x=1 \\
	\end{aligned} \right.$ liên tục tại $x=1$?
	\choice
	{ $m=0$}
	{ $m=-1$}
	{ $m=1$}
	{\True $m=2$}
	\loigiai{
		Tập xác định $\mathscr{D}=R$\\
		Ta có $\lim\limits_{x\to 1} f(x)=\lim\limits_{x\to 1} \dfrac{x^2-x}{x-1}=\lim\limits_{x\to 1} x=1$ và $f(1)=m-1$.\\
		Hàm số liên tục tại $x=1\Leftrightarrow m-1=1\Leftrightarrow m=2$.}
\end{ex}
\begin{ex}%[1K5BG-5]% 
	
	Có bao nhiêu số tự nhiên $m$ để hàm số $f\left( x \right)=\left\{ \begin{aligned}
		& \dfrac{x^2-3x+2}{x-1} \text{ khi } x\ne 1 \\
		& {m^2}+m-1 \text{ khi } x=1 \\
	\end{aligned} \right.$ liên tục tại điểm $x=1$?
	\choice
	{ 0}
	{ $3$}
	{ $2$}
	{\True $1$}
	\loigiai{
		$\lim\limits_{x\to 1} \dfrac{x^2-3x+2}{x-1}=\lim\limits_{x\to 1} \dfrac{\left( x-1 \right)\left( x-2 \right)}{x-1}=\lim\limits_{x\to 1} \left( x-2 \right)=-1$.\\
		Vì hàm số $f\left( x \right)$ liên tục tại điểm $x=1$ nên $\lim\limits_{x\to 1} f\left( x \right)=f\left( 1 \right)$\\
		$\Leftrightarrow {m^2}+m-1=-1$
		$\Leftrightarrow {m^2}+m=0\Leftrightarrow \left[ \begin{aligned}
			& m=0 \text{ (Thoả mãn)} \\
			& m=-1 \text{ (Loại)}. \\
		\end{aligned} \right.$}
\end{ex}
\begin{ex}%[1K5KG-5]%
	
	Tìm $a$ để hàm số $f\left( x \right)=\left\{ \begin{aligned}
		& \dfrac{\sqrt{x+2}-2}{x-2}\,\,\,\,\,\,\,\text{khi }x\ne 2 \\
		& 2x+a\,\,\,\,\,\,\,\,\,\,\,\,\,\,\,\,\,\text{khi}\,x=2 \\
	\end{aligned} \right.$ liên tục tại $x=2$?
	\choice
	{ $\dfrac{15}{4}$}
	{\True $-\dfrac{15}{4}$}
	{ $\dfrac{1}{4}$}
	{ $1$}
	\loigiai{
		Ta có $f\left( 2 \right)=4+a$.\\
		Ta tính được $\lim\limits_{x\to 2} f\left( x \right)=\lim\limits_{x\to 2} \dfrac{x+2-4}{\left( x-2 \right)\left( \sqrt{x+2}+2 \right)}=\lim\limits_{x\to 2} \dfrac{1}{\sqrt{x+2}+2}=\dfrac{1}{4}$.\\
		Hàm số đã cho liên tục tại $x=2$ khi và chỉ khi $f\left( 2 \right)=\lim\limits_{x\to 2} f\left( x \right)\Leftrightarrow 4+a=\dfrac{1}{4}\Leftrightarrow a=-\dfrac{15}{4}$.\\
		Vậy hàm số liên tục tại $x=2$ khi $a=-\dfrac{15}{4}$.}
\end{ex}
\begin{ex}%[1K5KG-5]% 
	
	Cho hàm số $f\left( x \right)=\left\{ \begin{aligned}
		& \dfrac{x^2-3x+2}{\sqrt{x+2}-2}\,\,\,\,\,\,\,\,\,khi\,\,x>2 \\
		& {m^2}x-4m+6\,\,\,\,khi\,\,\,x\le 2 \\
	\end{aligned} \right.$, $m$ là tham số. Có bao nhiêu giá trị của $m$ để hàm số đã cho liên tục tại $x=2$?
	\choice
	{ $3$}
	{ $0$}
	{ $2$}
	{\True $1$}
	\loigiai{
		Ta có\\
		$\lim\limits_{x\to {2^{+}}} f(x)=\lim\limits_{x\to {2^{+}}} \dfrac{x^2-3x+2}{\sqrt{x+2}-2}=\lim\limits_{x\to {2^{+}}} \dfrac{\left( x-2 \right)\left( x-1 \right)\left( \sqrt{x+2}+2 \right)}{x-2}=\lim\limits_{x\to {2^{+}}} \left( x-1 \right)\left( \sqrt{x+2}+2 \right)=4$.\\
		$\lim\limits_{x\to {2^{-}}} f(x)=\lim\limits_{x\to {2^{-}}} \left( {m^2}x-4m+6 \right)=2m^2-4m+6$.\\
		$f(2)=2m^2-4m+6$.\\
		Để hàm số liên tục tại $x=2$ thì $\lim\limits_{x\to {2^{+}}} f(x)=\lim\limits_{x\to {2^{-}}} f(x)=f(2)\Leftrightarrow 2m^2-4m+6=4\Leftrightarrow 2m^2-4m+2=0\Leftrightarrow m=1$.\\
		Vậy có một giá trị của $m$ thỏa mãn hàm số đã cho liên tục tại $x=2$.}
\end{ex}
\begin{ex}%[1K5KG-5]%
	
	Cho hàm số $f\left( x \right)=\left\{ \begin{aligned}
		& \dfrac{\sqrt{3x^2+2x-1}-2}{x^2-1},\ x\ne 1 \\
		& 4-m,\ \quad \quad \quad \quad \quad x=1 \\
	\end{aligned} \right.$. Hàm số $f\left( x \right)$ liên tục tại $x_0=1$ khi
	\choice
	{\True $m=3$}
	{ $m=-3$}
	{ $m=7$}
	{ $m=-7$}
	\loigiai{
		Tập xác định $\mathscr{D}=\mathbb{R}$, $x_0=1\in \mathbb{R}$.\\
		Ta có $f\left( 1 \right)=4-m$.\\
		$\lim\limits_{x\to 1} f\left( x \right)=\lim\limits_{x\to 1} \dfrac{\sqrt{3x^2+2x-1}-2}{\left( x+1 \right)\left( x-1 \right)}$ $=\lim\limits_{x\to 1} \dfrac{\left( x-1 \right)\left( 3x+5 \right)}{\left( x+1 \right)\left( x-1 \right)\left( \sqrt{3x^2+2x-1}+2 \right)}$\\
		$=\lim\limits_{x\to 1} \dfrac{3x+5}{\left( x+1 \right)\left( \sqrt{3x^2+2x-1}+2 \right)}=1$.\\
		Hàm số $f\left( x \right)$ liên tục tại $x_0=1$ khi và chỉ khi $\lim\limits_{x\to 1} \left( x \right)=f\left( 1 \right)\Leftrightarrow 4-m=1\Leftrightarrow m=3$.}
\end{ex}
\begin{ex}%[1K5KG-5]% 
	
	Tìm giá trị của tham số $m$ để hàm số $f\left( x \right)=\left\{ \begin{aligned}
		& \dfrac{x^2+3x+2}{x^2-1}\,\,\,\,\,\text{khi}\,\,\,\,\,\,x\,<\,-1 \\
		& mx+2\,\,\,\,\,\,\,\,\,\,\,\,\,\,\,\text{khi}\,\,\,\,\,\,x\,\ge \,-1 \\
	\end{aligned} \right.$ liên tục tại $x=-1$.
	\choice
	{ $m=\dfrac{-3}{2}$}
	{ $m=\dfrac{-5}{2}$}
	{ $m=\dfrac{3}{2}$}
	{\True $m=\dfrac{5}{2}$}
	\loigiai{
		Ta có  \\
		$\bullet$ $f\left( -1 \right)=-m+2$.\\
		$\bullet$ $\lim\limits_{x\to {{\left( -1 \right)}^{+}}} f\left( x \right)=-m+2$.\\
		$\bullet$$\lim\limits_{x\to {{\left( -1 \right)}^{-}}} f\left( x \right)=\lim\limits_{x\to {{\left( -1 \right)}^{-}}} \dfrac{x^2+3x+2}{x^2-1}=\lim\limits_{x\to {{\left( -1 \right)}^{-}}} \dfrac{\left( x+1 \right)\left( x+2 \right)}{\left( x-1 \right)\left( x+1 \right)}=\lim\limits_{x\to {{\left( -1 \right)}^{-}}} \dfrac{x+2}{x-1}=\dfrac{-1}{2}$.\\
		Hàm số liên tục tại $x=-1\Leftrightarrow f\left( -1 \right)=\lim\limits_{x\to {{\left( -1 \right)}^{+}}} f\left( x \right)=\lim\limits_{x\to {{\left( -1 \right)}^{-}}} f\left( x \right)\Leftrightarrow -m+2=\dfrac{-1}{2}\Leftrightarrow m=\dfrac{5}{2}$.}
\end{ex}
\begin{ex}%[1K5KG-5]% 
	
	Cho hàm số $f(x)=\left\{ \begin{matrix}
		\dfrac{\sqrt{x^2+4}-2}{x^2}\ \ \ \,\,\text{khi }\ x\ne 0 \\
		2a-\dfrac{5}{4}\ \ \ \ \ \ \ \ \ \ \ \,\text{khi }\ x=0 \\
	\end{matrix} \right.$. Tìm giá trị thực của tham số $a$ để hàm số $f(x)$ liên tục tại $x=0$.
	\choice
	{ $a=-\dfrac{3}{4}$}
	{ $a=\dfrac{4}{3}$}
	{ $a=-\dfrac{4}{3}$}
	{\True $a=\dfrac{3}{4}$}
	\loigiai{
		Tập xác định $\mathscr{D}=\mathbb{R}$.\\
		$\lim\limits_{x\to 0} f(x)=\lim\limits_{x\to 0} \dfrac{\sqrt{x^2+4}-2}{x^2}=\lim\limits_{x\to 0} \dfrac{\left( \sqrt{x^2+4}-2 \right)\left( \sqrt{x^2+4}+2 \right)}{x^2\left( \sqrt{x^2+4}+2 \right)}$
		$=\lim\limits_{x\to 0} \dfrac{x^2+4-4}{x^2(\sqrt{x^2+4}+2)}=\lim\limits_{x\to 0} \dfrac{1}{\sqrt{x^2+4}+2}=\dfrac{1}{4}$.\\
		$f(0)=2a-\dfrac{5}{4}$.\\
		Hàm số $f(x)$ liên tục tại $x=0\Leftrightarrow \lim\limits_{x\to 0} f(x)=f(0)\Leftrightarrow 2a-\dfrac{5}{4}=\dfrac{1}{4}\Leftrightarrow a=\dfrac{3}{4}$.\\
		Vậy $a=\dfrac{3}{4}$.}
\end{ex}
\begin{ex}%[1K5BG-5]% 
	
	Cho hàm số $f\left( x \right)=\left\{ \begin{aligned}
		& {x^2}-2x+3\text{  khi }x\ne 1 \\
		& 3x+m-1\text{   khi }x=1 \\
	\end{aligned} \right.$. Tìm $m$ để hàm số liên tục tại $x_0=1$.
	\choice
	{ $m=1$}
	{ $m=3$}
	{\True $m=0$}
	{ $m=2$}
	\loigiai{
		Tập xác định $\mathscr{D}=\mathbb{R}$.\\
		Ta có $f\left( 1 \right)=2+m$.\\
		$\lim\limits_{x\to 1} f\left( x \right)=\lim\limits_{x\to 1}{\mathop{ }}\left( {x^2}-2x+3 \right)=2$.\\
		Hàm số liên tục tại $x_0=1\Leftrightarrow \lim\limits_{x\to 1} f\left( x \right)=f\left( 1 \right)\Leftrightarrow 2=m+2\Leftrightarrow m=0$.}
\end{ex}
\begin{ex}%[1K5BG-5]% 
	
	Cho hàm số $f(x)=\left\{ \begin{aligned}
		& \dfrac{x^2-3x+2}{x-2} \text{ khi } x\ne 2 \\
		& a \text{ khi } x=2 \\
	\end{aligned} \right.$. Hàm số liên tục tại $x=2$ khi $a$ bằng
	\choice
	{\True $1$}
	{ $0$}
	{ $2$}
	{ $-1$}
	\loigiai{
		Hàm số liên tục tại $x=2$ khi và chỉ khi $\lim\limits_{x\to 2} f(x)=f(2)$.\\
		Ta có $f(2)=a,\lim\limits_{x\to 2} f(x)=\lim\limits_{x\to 2} \dfrac{x^2-3x+2}{x-2}=\lim\limits_{x\to 2} (x-1)=1$. Do đó $a=1$.}
\end{ex}

\Closesolutionfile{ans}
