\begin{dang}{Phương pháp lượng liên hợp kết quả hữu hạn}
	Nội dung và phương pháp giải
\end{dang}
\subsubsection{Ví dụ minh hoạ}
\Opensolutionfile{ans}[ans/ans-1K5-16-4]
\setcounter{vd}{0}
	\begin{vd}%[1K5BF-5]
	Cho $P = \lim \limits_{x \to 2} \dfrac{\sqrt{x+2}-2}{x-2}$. Tính $P$.
	\choice
	{\True $P = \dfrac{1}{4}$}
	{$P = \dfrac{1}{2}$}
	{$P = 1$}
	{$P = 0$}
	\loigiai{
		Ta có: $\lim \limits_{x \to 2} \dfrac{\sqrt{x+2}-2}{x-2} = \lim \limits_{x \to 2} \dfrac{x-2}{(x-2) \left(\sqrt{x + 2}+2 \right)} = \lim \limits_{x \to 2} \dfrac{1}{\sqrt{x + 2}+2} = \dfrac{1}{4}$. \\
		Vậy $P = \dfrac{1}{4}$.
	}
\end{vd}
\begin{vd}%[1K5BF-5]
	Cho $m$ là hằng số. Tính $\lim\limits_{x\to 1}\dfrac{\sqrt{x+3}-2}{x^2+mx-x-m}$.
	\choice
	{$\dfrac{1}{m}$}
	{$1$}
	{$\dfrac{1}{4}$}
	{\True $\dfrac{1}{4(m+1)}$}
	\loigiai{
		$\lim\limits_{x\to 1}\dfrac{\sqrt{x+3}-2}{x^2+mx-x-m}=\lim\limits_{x\to 1}\dfrac{x-1}{(x-1)(x+m)\left(\sqrt{x+3}+2\right)}=\dfrac{1}{4(m+1)}$.
	}
\end{vd}
\begin{vd}%[1K5BF-5]
	Biết $\lim\limits_{x\to -\infty}\left(\sqrt{x^2+1}+x+1\right)=a$. Tính $2a+1.$
	\choice
	{$-1$}
	{$-3$}
	{$0$}
	{\True $3$}
	\loigiai{
		{\allowdisplaybreaks
			\begin{eqnarray*}
				\lim\limits_{x\to -\infty}\left(\sqrt{x^2+1}+x+1\right)
				&=&\lim\limits_{x\to -\infty}\dfrac{-2x}{\sqrt{x^2+1}-(x+1)}\\
				&=&\lim\limits_{x\to -\infty}\dfrac{-2}{-\sqrt{1+\dfrac{1}{x^2}}-\left(1+\dfrac{1}{x}\right)}\\
				&=&1\\
				&\Rightarrow& a=1.	
			\end{eqnarray*}
			Vậy $2a+1=3$.
		}		
	}
\end{vd}


\begin{vd}%[1K5KF-5]
	Biết $\lim\limits_{x\rightarrow +\infty} \left(\sqrt{4x^2-3x+1}-(ax+b)\right) = 0.$ Tính giá trị biểu thức $T=a-4b$.
	\choice
	{$T=-2$}
	{\True $T=5$}
	{$T=-1$}
	{$T=3$}
	\loigiai{
		Từ giả thiết, đường thẳng $y=ax+b$ là tiệm cận xiên của đồ thị hàm số $y=\sqrt{4x^2-3x+1}$, khi $x\to +\infty.$ Từ đó,
		\begin{eqnarray*}\begin{array}{ccl} a&=&\lim\limits_{x\rightarrow +\infty}\dfrac{\sqrt{4x^2-3x+1}}x=2,\\
				b&=&\lim\limits_{x\rightarrow +\infty}\left(\sqrt{4x^2-3x+1}-2x\right)\\
				&=&\lim\limits_{x\rightarrow +\infty}\dfrac{-3x+1}{\sqrt{4x^2-3x+1}+2x}\\
				&=&\lim\limits_{x\rightarrow +\infty}\dfrac{-3+\frac1x}{\sqrt{4-\frac3x+\frac1{x^2}}+2}=-\dfrac34.
			\end{array}
		\end{eqnarray*}
		Suy ra $a-4b=5.$}
\end{vd}
\begin{vd}%[1K5GF-5]
	Cho $f(x)$ là hàm đa thức thỏa $\lim\limits_{x \to 2}\dfrac{f(x)+1}{x-2}=a$ và tồn tại $\lim\limits_{x \to 2}\dfrac{\sqrt{f(x)+2x+1}-x}{x^2-4}=T$. Chọn đẳng thức đúng
	\choice
	{$T=\dfrac{a+2}{16}$}
	{$T=\dfrac{a+2}{8}$}
	{$T=\dfrac{a-2}{8}$}
	{\True $T=\dfrac{a-2}{16}$}
	\loigiai{
		Vì $f(x)$ là đa thức và $\lim\limits_{x \to 2}\dfrac{f(x)+1}{x-2}=a$ nên suy ra $f(x)+1=(x-2)g(x), g(2)=a$.\\
		Do đó
		\begin{eqnarray*}
			&T&=\lim\limits_{x \to 2}\dfrac{\sqrt{(x-2)g(x)+2x}-x}{x^2-4}\\
			& &=\lim\limits_{x \to 2}\dfrac{(x-2)g(x)+2x-x^2}{(x-2)(x+2)\left[\sqrt{(x-2)g(x)+2x}+x\right]}\\
			& &=\lim\limits_{x \to 2}\dfrac{g(x)-x}{(x+2)\left[\sqrt{(x-2)g(x)+2x}+x\right]} \\
			& &=\dfrac{a-2}{16}.
		\end{eqnarray*}
	}
\end{vd}
\subsubsection{Bài tập rèn luyện}
\Opensolutionfile{ans}[ans/ans-1K5-2-Dang4]
%\setcounter{ex}{0}
\begin{ex}%[1K5BF-5]
	Biết $\displaystyle\lim_{x\rightarrow +\infty}\left(\sqrt{x^2+ax-1}-x\right)=5$. Khi đó giá trị của tham số $a$ là
	\choice
	{\True $10$}
	{$-6$}
	{$6$}
	{$-10$}
	\loigiai{
		$\displaystyle\lim_{x\rightarrow +\infty}\left(\sqrt{x^2+ax-1}-x\right)=\lim_{x\rightarrow +\infty}\dfrac{x^2+ax-1-x^2}{\sqrt{x^2+ax-1}+x}=\lim_{x\rightarrow +\infty}\dfrac{a-\dfrac{1}{x}}{\sqrt{1+\dfrac{a}{x}-\dfrac{1}{x^2}}+1}=\dfrac{a}{2}=5$.
		\\ Suy ra $a=10$. 
	}
\end{ex}
\begin{ex}%[1K5KF-5]
	Có tất cả bao nhiêu giá trị nguyên của tham số m để $\displaystyle\lim\limits_{x\to +\infty}(\sqrt{{x^2+m^2x}}-x)=\dfrac{1}{2}$?
	\choice
	{$0$}
	{$1$}
	{\True $2$}
	{$4$}
	\loigiai{
		Ta có \begin{align*}
		\displaystyle \lim \limits_{x\to +\infty}\left (\sqrt{x^2+m^2x}-x\right ) =&\displaystyle \lim \limits_{x\to +\infty}\dfrac{m^2x}{\sqrt{x^2+m^2x}+x}\\
		=& \displaystyle \lim \limits_{x\to +\infty}\dfrac{m^2}{\sqrt{1+\frac{m^2}{x}} +1}\\
		=&\dfrac{m^2}{2}.  
		\end{align*}	
		Do đó $\displaystyle\lim\limits_{x\to +\infty}(\sqrt{{x^2+m^2x}}-x)=\dfrac{1}{2}\Leftrightarrow \dfrac{m^2}{2}=\dfrac{1}{2}\Leftrightarrow m=\pm 1.$
		Vậy có hai giá trị nguyên của tham số $m$ thỏa mãn yêu cầu của bài toán.
	}
\end{ex}
\begin{ex}%[1K5BF-5]
	Tính $L=\lim\limits_{x \to - \infty} \left( \sqrt{x^2 -7x+1}- \sqrt{x^2-3x+2}\right)$.
	\choice
	{$L= + \infty$}
	{$L= - \infty$}
	{\True $L= 2$}
	{$L= -2$}
	\loigiai{
		\begin{align*}
		L &= \lim\limits_{x \to - \infty} \dfrac{-4x-1}{\sqrt{x^2 - 7x +1}+\sqrt{x^2-3x+2}}\\
		&=\lim\limits_{x \to - \infty} \dfrac{-4x-1}{-x \sqrt{1-\dfrac{7}{x}+\dfrac{1}{x^2}}-x \sqrt{1-\dfrac{3}{x}+\dfrac{2}{x^2}}}\\
		&= \lim\limits_{x \to - \infty} \dfrac{-4-\dfrac{1}{x}}{- \sqrt{1-\dfrac{7}{x}+\dfrac{1}{x^2}}-\sqrt{1-\dfrac{3}{x}+\dfrac{2}{x^2}}}\\
		&=2.
		\end{align*}
	}
\end{ex}


\begin{ex}%[1K5BF-5]
	Giá trị của $\lim\limits_{x \to 1} \dfrac{\sqrt {2x + 1}  - \sqrt {x + 2} }{x - 1}$  là 
	\choice
	{$- \dfrac{ \sqrt{3} }{5}$}
	{$-\dfrac{\sqrt{3} }{6}$}
	{\True $\dfrac{\sqrt{3}}{6}$}
	{$\dfrac{\sqrt{3}}{5}$}
	\loigiai{
		Ta có $$\lim\limits_{x \to 1} \dfrac{\sqrt{2x+1}+\sqrt{x+2}}{x-1}=\lim\limits_{x \to 1}\dfrac{x-1}{(x-1)\left(\sqrt{2x+1}-\sqrt{x+2} \right)}=\lim\limits_{x \to 1}\dfrac{1}{\sqrt{2x+1}+\sqrt{x+2}}=\dfrac{1}{2\sqrt{3}}.$$
	}
\end{ex}

\begin{ex}%[1K5BF-5]
	Cho $\displaystyle\lim_{x \to 0}\dfrac{1-\sqrt[3]{1-x}}{x}=\dfrac{m}{n}$, trong đó $m,n$ là các số nguyên và $\dfrac{m}{n}$ tối giản.\\ Tính $A=2m-n$.
	\choice
	{$A=1$}
	{\True $A=-1$}
	{$A=0$}
	{$A=-2$}
	\loigiai
	{Ta có $\displaystyle\lim_{x \to 0}\dfrac{1-\sqrt[3]{1-x}}{x}=\displaystyle\lim_{x \to 0}\dfrac{x}{x\cdot\left(1+\sqrt[3]{1-x}+\sqrt[3]{(1-x)^2}\right)}=\dfrac{1}{3}.$\\
		Vậy $A=2m-n=2\cdot1-3=-1.$}
\end{ex}
\begin{ex}%[1K5BF-5]
	Giới hạn $\displaystyle\lim\limits_{x\to 3}\dfrac{x+1-\sqrt{5x+1}}{x-\sqrt{4x-3}}=\dfrac{a}{b}$, với $a,b\in\mathbb{Z},b>0$ và $\dfrac{a}{b}$ là phân số tối giản. Giá trị của $a-b$ là
	\choice
	{$\dfrac{1}{9}$}
	{$-1$}
	{$\dfrac{9}{8}$}
	{\True $1$}
	\loigiai{
		Ta có
		\begin{eqnarray*}
			\displaystyle\lim\limits_{x\to 3}\dfrac{x+1-\sqrt{5x+1}}{x-\sqrt{4x-3}}&=& \lim\limits_{x\to 3}\dfrac{\dfrac{(x+1)^2-(5x+1)}{x+1+\sqrt{5x+1}}}{\dfrac{x^2-(4x-3)}{x+\sqrt{4x-3}}} \\
			&=& \lim\limits_{x\to 3}\dfrac{\left(x+\sqrt{4x-3}\right)(x-3)x}{\left(x+1+\sqrt{5x+1}\right)(x-3)(x-1)} \\
			&=& \lim\limits_{x\to 3}\dfrac{x\left(x+\sqrt{4x-3}\right)}{\left(x+1+\sqrt{5x+1}\right)(x-1)}=\dfrac{9}{8}.
		\end{eqnarray*}
		Vậy $a=9,b=8$, suy ra $a-b=1$.
	}
\end{ex}
\begin{ex}%[1K5BF-5]
	Cho $\underset{x\to 4}{\mathop{\lim}}\,\dfrac{\sqrt{3x+4}-4}{x-4}=\dfrac{a}{b}$, với $\dfrac{a}{b}$ là phân số tối giản. Tính $2a+b^2$.
	\choice
	{$22$}
	{$66$}
	{$14$}
	{\True $70$}
	\loigiai{
		Có $\underset{x\to 4}{\mathop{\lim}}\,\dfrac{\sqrt{3x+4}-4}{x-4}=\underset{x\to 4}{\mathop{\lim}}\,\dfrac{3\left( x-4 \right)}{\left( x-4 \right)\left( \sqrt{3x+4}+4 \right)}=\underset{x\to 4}{\mathop{\lim}}\,\dfrac{3}{\sqrt{3x+4}+4}=\dfrac{3}{8}$.\\
		$\Rightarrow 2a+b^2=6+64=70$.}
\end{ex}

\begin{ex}%[1K5BF-5]
	Tính $\lim\limits_{x \rightarrow + \infty} \left(\sqrt{x^2 + 3x + 2} - x\right)$.
	\choice
	{$- \dfrac{3}{2}$}
	{\True $\dfrac{3}{2}$}
	{$\dfrac{7}{2}$}
	{$- \dfrac{7}{2}$}
	\loigiai{
		\begin{eqnarray*}
			\lim\limits_{x \rightarrow + \infty} \left(\sqrt{x^2 + 3x + 2} - x\right) & = &\lim\limits_{x \rightarrow + \infty} \dfrac{3x + 2}{\sqrt{x^2 + 3x + 2} + x}\\
			& = &\lim\limits_{x \rightarrow + \infty} \dfrac{x\left(3 + \dfrac{2}{x}\right)}{|x|\sqrt{1 + \dfrac{3}{x} + \dfrac{2}{x^2}} + x}\\
			& = &\lim\limits_{x \rightarrow + \infty} \dfrac{x\left(3 + \dfrac{2}{x}\right)}{x\left(\sqrt{1 + \dfrac{3}{x} + \dfrac{2}{x^2}} + 1\right)}\\
			& = &\lim\limits_{x \rightarrow + \infty} \dfrac{3 + \dfrac{2}{x}}{\sqrt{1 + \dfrac{3}{x} + \dfrac{2}{x^2}} + 1} = \dfrac{3}{2}.
		\end{eqnarray*}
	}
\end{ex}
\begin{ex}%[1K5BF-5]
	Tìm giới hạn $M=\underset{x\to -\infty}{\lim}\left(\sqrt{x^2-4x}-\sqrt{x^2-x}\right)$.
	\choice
	{$M=-\dfrac{1}{2}$}
	{\True $M=\dfrac{3}{2}$}
	{$M=-\dfrac{3}{2}$}
	{$M=\dfrac{1}{2}$}
	\loigiai{
		Ta có
		\begin{align*}
		M&=\underset{x\to -\infty}{\lim}\left(\sqrt{x^2-4x}-\sqrt{x^2-x}\right)\\
		&=\underset{x\to -\infty}{\lim}\dfrac{-3x}{\sqrt{x^2-4x}+\sqrt{x^2-x}}\\
		&=\underset{x\to -\infty}{\lim}\dfrac{-3x}{|x|\left(\sqrt{1-\dfrac{4}{x}}+\sqrt{1-\dfrac{1}{x}}\right)}\\
		&=\underset{x\to -\infty}{\lim}\dfrac{3}{\sqrt{1-\dfrac{4}{x}}+\sqrt{1-\dfrac{1}{x}}}=\dfrac{3}{2}.
		\end{align*}
	}
\end{ex}





\begin{ex}%[1K5BF-5]
	Biết rằng $\lim\limits_{x\to -\infty} \left(\sqrt{2x^2-3x+1}+x\sqrt{2}\right)=\dfrac{a}{b}\sqrt{2}$,\quad ($a,b\in\mathbb{Z}, b>0, \dfrac{a}{b}$ tối giản). Tổng $a+b$ có giá trị là
	\choice
	{$5$}
	{$4$}
	{\True $7$}
	{$1$}
	\loigiai{
		Ta có
		\begin{eqnarray*}
			\lim\limits_{x\to -\infty} \left(\sqrt{2x^2-3x+1}+x\sqrt{2}\right) & = & \lim\limits_{x\to -\infty} \dfrac{2x^2-3x+1-2x^2}{\sqrt{2x^2-3x+1}-x\sqrt{2}}\\
			&=& \lim\limits_{x\to -\infty} \dfrac{-3x+1}{\sqrt{2x^2-3x+1}-x\sqrt{2}}\\
			&=&\lim\limits_{x\to -\infty} \dfrac{-3+\dfrac{1}{x}}{-\sqrt{2-\dfrac{3}{x}+\dfrac{1}{x^2}}-\sqrt{2}}\\
			&=& \dfrac{3}{4}\cdot \sqrt{2}.
		\end{eqnarray*}
		Vậy $a=3, b=4$. Tổng $a+b=7$.
	}
\end{ex}




\begin{ex}%[1K5BF-5]
	Tìm giới hạn $I=\underset{x\to +\infty}{\mathop{\lim}}\,\left( x+1-\sqrt{x^2-x+2} \right)$.
	\choice
	{$I=\dfrac{1}{2}$}
	{$I=\dfrac{46}{31}$}
	{$I=\dfrac{17}{11}$}
	{\True $I=\dfrac{3}{2}$}
	\loigiai{
		Ta có: $I=\underset{x\to +\infty}{\mathop{\lim}}\,\left( x+1-\sqrt{x^2-x+2} \right)=\underset{x\to +\infty}{\mathop{\lim}}\,\left( \dfrac{x^2-x^2+x-2}{x+\sqrt{x^2-x+2}}+1 \right)\\
		=\underset{x\to +\infty}{\mathop{\lim}}\,\left( \dfrac{x-2}{x+\sqrt{x^2-x+2}}+1 \right)=\underset{x\to +\infty}{\mathop{\lim}}\,\left( \dfrac{1-\dfrac{2}{x}}{1+\sqrt{1-\dfrac{1}{x}+\dfrac{2}{x^2}}}+1 \right)=\dfrac{3}{2}$.}
\end{ex}

\begin{ex}%[1K5KF-5]
	Cho $f(x)$ là một đa thức thỏa mãn $\lim \limits_{x\to 2} \dfrac{f(x)-15}{x-2}=3$. Tính \[\lim \limits_{x\to 2} \dfrac{f(x)-15}{(x^2-4)\left( \sqrt{2f(x)+6}+3\right)}.\]
	\choice
	{$\dfrac{1}{10}$}
	{$\dfrac{1}{6}$}
	{\True $\dfrac{1}{12}$}
	{$\dfrac{1}{8}$}
	\loigiai{
		Do $\lim \limits_{x\to 2} \dfrac{f(x)-15}{x-2}=3$ và $\lim \limits_{x\to 2} (x-2) = 0$ nên $\lim \limits_{x\to 2} \left( f(x) - 15 \right) = 0 \Rightarrow f(2) = 15$.\\
		Ta có
		\begin{eqnarray*} 
			\lim \limits_{x\to 2} \dfrac{f(x)-15}{(x^2-4)\left( \sqrt{2f(x)+6}+3\right)}
			& = & \lim \limits_{x\to 2} \left[ \dfrac{f(x)-15}{x-2} \cdot \dfrac{1}{(x+2)\left( \sqrt{2f(x)+6}+3\right)}\right] \\
			& = & 3 \cdot \dfrac{1}{4\cdot (\sqrt{2\cdot 15 +6}+3)}=\dfrac{1}{12}.
		\end{eqnarray*}
	}
\end{ex}
\begin{ex}%[1K5GF-5]
	Biết rằng $b>0$, $a+b=5$ và $\lim\limits_{x\to 0}\dfrac{\sqrt[3]{ax+1}-\sqrt{1-bx}}{x}=2$. Khẳng định nào dưới đây là \textbf{sai}?
	\choice
	{$a^2+b^2>10$}
	{ \True  $a^2-b^2>6$}
	{ $a-b\geq 0$}
	{$1\le a\le 3$}
	\loigiai{ Ta có
		$$\lim\limits_{x\to 0}\dfrac{\sqrt[3]{ax+1}-\sqrt{1-bx}}{x}=\lim\limits_{x\to 0}\dfrac{\sqrt[3]{ax+1}-1}{x}-\lim\limits_{x\to 0}\dfrac{\sqrt{1-bx}-1}{x}.$$
		Với $L_1=\lim\limits_{x\to 0}\dfrac{\sqrt[3]{ax+1}-1}{x}=\lim\limits_{x\to 0}\dfrac{ax}{x\left(\sqrt[3]{(ax+1)^2}+\sqrt[3]{ax+1}+1\right)}=\dfrac{a}{3}$.\\
		Với $\lim\limits_{x\to 0}\dfrac{\sqrt{1-bx}-1}{x}=\lim\limits_{x\to 0}\dfrac{-bx}{x\left(\sqrt{1-bx}+1\right)}=-\dfrac{b}{2}$.\\
		Từ giả thiết bài toán ta có
		$$L=L_1-L_2=2\Leftrightarrow \dfrac{a}{3}+\dfrac{b}{2}=2\Leftrightarrow 2a+3b=12.$$
		Ta có hệ phương trình $\heva{&a+b=5\\&2a+3b=12}\Leftrightarrow \heva{&a=3\\&b=2.}$\\
		Kiểm tra trực tiếp từng đáp án ta thấy $a^2-b^2>6$ là sai.
	}
\end{ex}

\Closesolutionfile{ans}
%\inputans{10}{ans/ans-1K5-16-4}