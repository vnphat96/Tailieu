\section{Giới hạn hàm số}
\subsection{Tóm tắt lý thuyết}
\subsubsection{Giới hạn hữu hạn của hàm số tại một điểm}
\paragraph{Định nghĩa}
\begin{dn}
Cho khoảng $\mathscr{K}$ chứa điểm $x_0$ và hàm số $y=f(x)$ xác định trên $\mathscr{K}$ hoặc trên $\mathscr{K}\setminus \{x_0\}$.\\
Ta nói hàm số $y = f(x)$ có giới hạn là số $L$ khi $x$ dần tới $x_0$ nếu với dãy số $(x_n)$ bất kỳ, $x_n\in K\setminus \{x_0\}$ và $x_n\to x_0$, ta có $\displaystyle\lim f(x_n)=L$.\\
Kí hiệu $\displaystyle\lim_{x\to x_0} f(x)=L$ hay $f(x)\rightarrow L$ khi $x \rightarrow x_0$.
\end{dn}
\begin{vd}%[Nguyễn Thành Sơn]%[1D4B2]
Cho hàm số $f(x) = \dfrac{x^2 - 4}{x + 2}$. Chứng minh rằng $\displaystyle\lim_{x\to -2} f(x)= -4$.
\end{vd}
\loigiai{
Tập xác định: $\mathscr{D} = \mathbb{R} \setminus \{-2 \}$. \\
Giả sử $\left(x_n \right)$ là một dãy số bất kỳ, thõa mãn $ x_n \neq -2 $ và $x_n \rightarrow -2$ khi $n \rightarrow + \infty$. \\
Ta có $\displaystyle \lim f \left(x_n\right) = \displaystyle \lim \dfrac{x_n^2 - 4}{x_n + 2} = \displaystyle \lim \dfrac{\left(x_n + 2 \right) \cdot \left(x_n - 2 \right)}{\left(x_n + 2 \right)} = \displaystyle \lim \left(x_n - 2 \right) = -4$.
Do đó $\displaystyle \lim_{x \to -2} f(x) = -4$.}
\begin{note}
$\displaystyle \lim_{x \to x_0}x = x_0; \text{ }\displaystyle \lim_{x \to x_0} c = c$, với $c$ là hằng số. 
\end{note}
\paragraph{Định lí về giới hạn hữu hạn}
\begin{dl} 
\textbf{a)} Giả sử $\displaystyle \lim_{x \to x_0}f(x) = L$ và $\displaystyle \lim_{x \to x_0}g(x) = M$. Khi đó
\begin{itemize}
\item	$\displaystyle \lim_{x \to x_0} \left[ f(x) + g(x) \right] = L + M$.
\item$\displaystyle \lim_{x \to x_0} \left[ f(x) - g(x) \right] = L - M$.
\item	$\displaystyle \lim_{x \to x_0} \left[ f(x) \cdot g(x) \right] = L \cdot M$.
\item	$\displaystyle \lim_{x \to x_0} \dfrac{f(x)}{g(x)} = \dfrac{L}{M}$ (nếu $M \neq 0$).
\end{itemize}
\textbf{b)} Nếu $f(x) \geq 0$ và $\displaystyle \lim_{x \to x_0}f(x) = L$, thì $$ L \geq 0 \text{ và }\displaystyle \lim_{x \to x_0} \sqrt{f(x)} = \sqrt{L}.$$
( Dấu của $f(x)$ được xét trên khoảng đang tìm giới hạn, với $x \neq x_0)$.
\end{dl}
\begin{vd}%[Nguyễn Thành Sơn]%[1D4B2]
Tính $\displaystyle \lim_{x \to 1} \dfrac{x^2 + x - 2}{x - 1}$.
\end{vd}
\loigiai{
$\displaystyle \lim_{x \to 1} \dfrac{x^2 + x - 2}{x - 1} = \displaystyle \lim_{x \to 1} \dfrac{(x - 1) \cdot (x + 2)}{x - 1} = \displaystyle \lim_{x \to 1} (x + 2) = 3$.
}
\paragraph{Giới hạn một bên}
\begin{dn}
\begin{itemize}
\item	Cho hàm số $y = f(x)$ xác định trên khoảng $\left(x_0; b \right)$. \\
Số $L$ được gọi là \textbf{\textit{giới hạn bên phải}} của hàm số $y = f(x)$ khi $x \rightarrow x_0$ nếu với dãy số $\left(x_n \right)$ bất kì, $x_0 < x_n < b$ và $x_n \rightarrow x_0$, ta có $f\left(x_n \right) \rightarrow L$. \\
Kí hiêu: $\displaystyle \lim_{x \to x_0^+} f(x) = L$.
\item	Cho hàm số $y = f(x)$ xác định trên khoảng $\left(a; x_0 \right)$. \\
Số $L$ được gọi là \textbf{\textit{giới hạn bên trái}} của hàm số $y = f(x)$ khi $x \rightarrow x_0$ nếu với dãy số $\left(x_n \right)$ bất kì, $a < x_n < x_0$ và $x_n \rightarrow x_0$, ta có $f\left(x_n \right) \rightarrow L$. \\
Kí hiêu: $\displaystyle \lim_{x \to x_0^-} f(x) = L$.
\end{itemize}
\end{dn}
\begin{dl}
$\displaystyle \lim_{x \to x_0} f(x) = L$ khi và chỉ khi $\displaystyle \lim_{x \to x_0^-} f(x) = \displaystyle \lim_{x \to x_0^+} f(x) = L$.
\end{dl}
\begin{vd}%[Nguyễn Thành Sơn]%[1D4K2]
Cho hàm số $f(x)=\left\{\begin{aligned}   & {5x+2} \text{ nếu } x \neq 1 \\ 
& {x^2-3} \text{ nếu } x <1 \\ \end{aligned}\right.$. \\
Tìm $\displaystyle \lim_{x \to 1^-}f(x), \displaystyle \lim_{x \to 1^+}f(x)$, và $\displaystyle \lim_{x \to 1}f(x)$ (nếu có).
\end{vd}
\loigiai{
Ta có: $\displaystyle \lim_{x \to 1^-}f(x) = \displaystyle \lim_{x \to 1^-} \left(x^2 - 3 \right) = 1^2 - 3 = - 2$;\\
$\displaystyle \lim_{x \to 1^+}f(x) = \displaystyle \lim_{x \to 1^+}(5x + 2) = 5 \cdot 1 + 2 = 7$. \\
Theo đinh lí $2$, $\displaystyle \lim_{x \to 1}f(x)$ không tồn tại.
}
\subsubsection{Giới hạn hữu hạn của hàm số tại vô cực}
\begin{dn}
\textbf{a)}	Cho hàm số $y = f(x)$ xác định trên khoảng $\left(a; + \infty \right)$. \\
Ta nói hàm số $y = f(x)$ có giới hạn là số $L$ khi $x \rightarrow +\infty$ nếu với dãy số $\left(x_n \right)$ bất kì, $x_n > a$ và $x_n \rightarrow +\infty$, ta có $f\left(x_n \right) \rightarrow L$. \\
Kí hiệu: $\displaystyle \lim_{x \to + \infty} = L$ hay $f(x) \rightarrow L$ khi $x \rightarrow +\infty$. \\
\textbf{b)}	Cho hàm số $y = f(x)$ xác định trên khoảng $\left(-\infty; a \right)$. \\
Ta nói hàm số $y = f(x)$ có giới hạn là số $L$ khi $x \rightarrow - \infty$ nếu với dãy số $\left(x_n \right)$ bất kì, $x_n < a$ và $x_n \rightarrow - \infty$, ta có $ f\left(x_n \right) \rightarrow L$.\\
Kí hiệu: $\displaystyle \lim_{x \to - \infty} = L$ hay $f(x) \rightarrow L$ khi $x \rightarrow -\infty$.
\end{dn}
\begin{vd}%[Nguyễn Thành Sơn]%[1D4B2]
Cho hàm số $y = f(x) = \dfrac{2x + 3}{x - 1}$. Tìm $\displaystyle \lim_{x \to - \infty}f(x)$ và $\displaystyle \lim_{x \to + \infty}f(x)$.
\end{vd}
\loigiai{ Hàm số đã cho xác định trên $\left( - \infty;1 \right)$ và trên $\left(1; + \infty\right)$.\\
Giả sử $\left(x_n \right)$ là một dãy số bất kì, thỏa mãn $x_n < 1$ và $x_n \rightarrow - \infty$.\\
Ta có $\displaystyle \lim f\left(x_n \right) = \displaystyle \lim \dfrac{2x_n + 3}{x_n - 1} = \displaystyle \lim \dfrac{2 + \cfrac{3}{x_n}}{1 - \cfrac{1}{x_n}} = 2.$ \\
Vậy $\displaystyle \lim_{x \to - \infty} = \displaystyle \lim_{x \to - \infty} \dfrac{2x + 3}{x - 1} = 2$.\\
Giả sử $\left(x_n \right)$ là một dãy số bất kì, thỏa mãn $x_n > 1$ và $x_n \rightarrow + \infty$.\\
Ta có $\displaystyle \lim f\left(x_n \right) = \displaystyle \lim \dfrac{2x_n + 3}{x_n - 1} = \displaystyle \lim \dfrac{2 + \cfrac{3}{x_n}}{1 - \cfrac{1}{x_n}} = 2.$ \\
Vậy $\displaystyle \lim_{x \to + \infty} = \displaystyle \lim_{x \to + \infty} \dfrac{2x + 3}{x - 1} = 2$.
}
\begin{note}
\begin{itemize}
\item	Với $c,k$ là các hằng số và $k$ nguyên dương, ta luôn có: 
$$ \displaystyle \lim_{x \to + \infty}c = c; \displaystyle \lim_{x \to - \infty} c = c; \displaystyle \lim_{x \to + \infty} \dfrac{c}{x^k}= 0;\displaystyle \lim_{x \to - \infty} \dfrac{c}{x^k}= 0.$$
\item	Định lí $1$ về giới hạn hữu hạn của hàm số khi $x \rightarrow x_0$ còn đúng khi $x \rightarrow +\infty$ hoặc $x \rightarrow -\infty$.
\end{itemize}
\end{note}
\begin{vd}%[Nguyễn Thành Sơn]%[1D4B2]
Tìm $\displaystyle \lim_{x \to +\infty} \dfrac{3x^2 - 2x}{x^2 + 1}$.
\end{vd}
\loigiai{
$\displaystyle \lim_{x \to +\infty} \dfrac{3x^2 - 2x}{x^2 + 1} = \displaystyle \lim_{x \to +\infty} \dfrac{3 - \cfrac{2}{x}}{1 + \cfrac{1}{x^2}} = \dfrac{3 - 0}{1 + 0} = 3.$
}
\subsubsection{Giới hạn vô cực của hàm số}
\paragraph{Giới hạn vô cực}
\begin{dn}
Cho hàm số $y = f(x)$ xác định trên khoảng $\left(a; +\infty \right)$. \\
Ta nói hàm số $y = f(x)$ có giới hạn là $- \infty$ khi $x \rightarrow + \infty$ nếu với dãy số $\left(x_n \right)$ bất kì, $x_n > a$ và $x_n \rightarrow +\infty$, ta có $f\left(x_n \right) \rightarrow - \infty$.\\
Kí hiệu: $\displaystyle \lim_{x \to + \infty}f(x) = - \infty$ hay $f(x) \rightarrow - \infty$ khi $x \rightarrow + \infty$.
\end{dn}
\textbf{Nhận xét:}$\displaystyle \lim_{x \to + \infty}f(x) = + \infty \Leftrightarrow \displaystyle \lim_{x \to + \infty} (-f(x)) = - \infty$.
\paragraph{Một vài giới hạn đặc biệt}
\begin{enumerate}
\item	$\displaystyle \lim_{x \to +\infty} x^k = + \infty$ với $k$ nguyên dương.
\item	$\displaystyle \lim_{x \to -\infty} x^k = - \infty$ nếu $k$ là số lẻ.
\item 	$\displaystyle \lim_{x \to +\infty} x^k = + \infty$ nếu $k$ là số chẵn.
\end{enumerate}
\paragraph{Một vài quy tắc về giới hạn vô cực}
\begin{enumerate}
\item Quy tắc tìm giới hạn của tích $f(x).g(x)$
\begin{center}
% {\renewcommand\arraystretch{1.2}
% \begin{tabular}{|c|c|c|}
% \hline 
% $\displaystyle\lim_{x\to x_0} f(x)$ & $\displaystyle\lim_{x\to x_0} g(x)$ & $\displaystyle\lim_{x\to x_0} f(x)g(x)$\\
% \hline 
% \multirowcell{1}[-1.75ex]{$\L>0$} & \makecell[c]{$+\infty$} & $+\infty$ \\
% \cline{2-3}
% ~& \makecell[c]{$-\infty$} & $-\infty$\\
% \hline
% \multirowcell{1}[-1.75ex]{$\L<0$} & \makecell[c]{$+\infty$} & $-\infty$ \\
% \cline{2-3}
% ~& \makecell[c]{$-\infty$} & $+\infty$\\
% \hline 
% \end{tabular} 
% }
% \end{center}
% \item Quy tắc tìm giới hạn của thương $\dfrac{f(x)}{g(x)}$
% \begin{center}
{\renewcommand\arraystretch{1.35}
\begin{tabular}{|c|c|c|c|}
\hline 
$\displaystyle\lim_{x\to x_0} f(x)$ & $\displaystyle\lim_{x\to x_0} g(x)$ & Dấu của $g(x)$ & $\displaystyle\lim_{x\to x_0} \dfrac{f(x)}{g(x)}$\\
\hline
$\alpha$ & $\pm\infty$ & Tùy ý & 0\\ 
\hline
\multirowcell{1}[-1.75ex]{$\L>0$} & \multirowcell{1}[-1.75ex]{0}& \makecell[c]{$+$} & $+\infty$ \\
\cline{3-4}
~& ~&\makecell[c]{$-$} & $-\infty$\\
\hline
\multirowcell{1}[-1.75ex]{$\L<0$} & \multirowcell{1}[-1.75ex]{0}&  \makecell[c]{$+$} & $-\infty$ \\
\cline{3-4}
~& ~& \makecell[c]{$-$} & $+\infty$\\
\hline 
\end{tabular} 
}
\end{center}
\end{enumerate}
Các quy tắc trên vẫn đúng cho các trường hợp $ x \rightarrow x_0^+$, $ x \rightarrow x_0^-$, $ x \rightarrow + \infty$, và $ x \rightarrow -\infty$.
\begin{vd}%[Nguyễn Thành Sơn]%[1D4B2]
Tìm $\displaystyle \lim_{x \to - \infty} \left(x^3 - 2x \right)$.
\end{vd}
\loigiai{Ta có:  $\displaystyle \lim_{x \to - \infty} \left(x^3 - 2x \right) = \displaystyle \lim_{x \to - \infty}x^3 \left(1 - \dfrac{2}{x^2} \right) = - \infty$, vì \\
$\displaystyle \lim_{x \to - \infty}x^3 = - \infty$ và $\displaystyle \lim_{x \to - \infty} \left(1 - \dfrac{2}{x^2} \right) = 1 > 0$.
}
\begin{vd}%[Nguyễn Thành Sơn]%[1D4B2]
Tính $\displaystyle \lim_{x \to 1^-1}\dfrac{2x - 3}{x - 1}$.
\end{vd}
\loigiai{
Ta có:$\displaystyle \lim_{x \to 1^-}\dfrac{2x - 3}{x - 1} = + \infty$, vì \\
$\displaystyle \lim_{x \to 1^-}(2x - 3)= 2 \cdot 1 - 3 = - 1 < 0$, và $\displaystyle \lim_{x \to 1^-}(x - 1) = 0$, $x - 1 < 0 \text{ }\forall x < 1$.
}
\subsection{Các dạng toán}
\begin{dang}{Giới hạn của hàm số dạng vô định 0/0}
* Biểu thức có dạng $\lim\limits_{x\to x_0}\dfrac{f(x)}{g(x)}$ trong đó $f(x), g(x)$ là các đa thức và $f(x_0)=g(x_0)=0$.\\
Khử dạng vô định bằng cách phân tích cả tử và mẫu thành nhân tử với nhân tử chung là $x-x_0$.\\
Giả sử $f(x)=(x-x_0) \cdot f_1(x)$ và $g(x)=(x-x_0) \cdot g_1(x)$. Khi đó:
$$\lim\limits_{x\to x_0}\frac{f(x)}{g(x)}=\lim\limits_{x\to x_0}\frac{f_1(x)}{g_1(x)}$$
Nếu giới hạn $\lim\limits_{x\to x_0}\dfrac{f_1(x)}{g_1(x)}$ vẫn ở dạng vô định $\displaystyle\frac{0}{0}$ thì ta lặp lại quá trình khử đến khi không còn dạng vô định.\\
Việc phân tích thành nhân tử ở trên được thực hiện bằng phương pháp chia Horner.\\

* Biểu thức có dạng $\lim\limits_{x\to x_0}\dfrac{f(x)}{g(x)}$ trong đó $f(x)$, $g(x)$ là các biểu thức có chứa căn thức và $f(x_0)=g(x_0)=0$.\\
Khử dạng vô định bằng cách nhân cả tử và mẫu với biểu thức liên hợp tương ứng của biểu thức chứa căn thức để trục các nhân tử $x-x_0$ ra khỏi các căn thức, nhằm khử các thành phần có giới hạn bằng 0. Lưu ý có thể nhân liên hợp một hay nhiều lần để khử dạng vô định.\\
\textit{Chú ý:} Các hằng đẳng thức\\
\hspace{1.5cm} $A^2-B^2=(A-B)(A+B)$.\\
\hspace{1.5cm} $A^3-B^3=(A-B)(A^2+AB+B^2)$.\\
\hspace{1.5cm} $A^3+B^3=(A+B)(A^2-AB+B^2)$.
\end{dang}

\begin{vd}%[Cao Thành Thái]%[1D4B2]
Tính các giới hạn sau:
\begin{multicols}{2}
\begin{enumerate}
\item $\lim\limits_{x\to -4}\dfrac{x^2+2x-8}{x^2+4x}$.
\item $\lim\limits_{x\to \frac{1}{2}}\dfrac{2x^2-5x+2}{1-2x}$.
\item $\lim\limits_{x\to 2}\dfrac{2x^2-5x+2}{x^2+x-6}$.
\item $\lim\limits_{x\to -1}\dfrac{1+x^3}{1-x^2}$.
\end{enumerate}
\end{multicols}
\loigiai
{
\begin{enumerate}
\item $\lim\limits_{x\to -4}\dfrac{x^2+2x-8}{x^2+4x} = \lim\limits_{x\to -4}\dfrac{(x+4)(x-2)}{x(x+4)} = \lim\limits_{x\to -4}\dfrac{x-2}{x} = \dfrac{-4-2}{-4} = \dfrac{3}{2}$.
\item $\lim\limits_{x\to \frac{1}{2}}\dfrac{2x^2-5x+2}{1-2x} = \lim\limits_{x\to \frac{1}{2}}\dfrac{(2x-1)(x-2)}{1-2x} = \lim\limits_{x\to \frac{1}{2}}(2-x)= 2 - \dfrac{1}{2} = \dfrac{3}{2}$.
\item $\lim\limits_{x\to 2}\dfrac{2x^2-5x+2}{x^2+x-6} = \lim\limits_{x\to 2}\dfrac{(x-2)(2x-1)}{(x-2)(x+3)} = \lim\limits_{x\to 2}\dfrac{2x-1}{x+3} = \dfrac{2 \cdot 2-1}{2+3} = \dfrac{3}{5}$.
\item $\lim\limits_{x\to -1}\dfrac{1+x^3}{1-x^2} = \lim\limits_{x\to -1}\dfrac{(1+x)(1-x+x^2)}{(1+x)(1-x)} = \lim\limits_{x\to -1}\dfrac{1-x+x^2}{1-x} = \dfrac{1-(-1)+(-1)^2}{1-(-1)} = \dfrac{3}{2}$.
\end{enumerate}
}
\end{vd}


\begin{vd}%[Cao Thành Thái]%[1D4B2]
Tính giới hạn $\lim\limits_{x\to -1}\dfrac{x^2-1}{2x+\sqrt{3x^2+1}}$.
\loigiai
{
$\lim\limits_{x\to -1}\dfrac{x^2-1}{2x+\sqrt{3x^2+1}} = \lim\limits_{x\to -1}\dfrac{(x^2-1)\left(2x-\sqrt{3x^2+1}\right)}{4x^2-(3x^2+1)} = \lim\limits_{x\to -1}\dfrac{(x^2-1)\left(2x-\sqrt{3x^2+1}\right)}{x^2-1}$\\
$= \lim\limits_{x\to -1}\left(2x-\sqrt{3x^2+1}\right) = 2 \cdot (-1)-\sqrt{3 \cdot (-1)^2+1} = -4$.
}
\end{vd}


\begin{vd}%[Cao Thành Thái]%[1D4B2]
Tính giới hạn $\lim\limits_{x\to 5}\dfrac{2x-5\sqrt{x-1}}{3-\sqrt{x+4}}$.
\loigiai
{
$\lim\limits_{x\to 5}\dfrac{2x-5\sqrt{x-1}}{3-\sqrt{x+4}} = \lim\limits_{x\to 5}\dfrac{\left[4x^2-25(x-1)\right]\left(3+\sqrt{x+4}\right)}{\left[9-(x+4)\right]\left(2x+5\sqrt{x-1}\right)} = \lim\limits_{x\to 5}\dfrac{(4x^2-25x+25)\left(3+\sqrt{x+4}\right)}{(5-x)\left(2x+5\sqrt{x-1}\right)}$\\
$= \lim\limits_{x\to 5}\dfrac{(x-5)(4x-5)\left(3+\sqrt{x+4}\right)}{(5-x)\left(2x+5\sqrt{x-1}\right)} = \lim\limits_{x\to 5}\dfrac{(5-4x)\left(3+\sqrt{x+4}\right)}{2x+5\sqrt{x-1}}$\\
$= \dfrac{(5-4 \cdot 5)(3+\sqrt{5+4})}{2 \cdot 5+5\sqrt{5-1}} = -\dfrac{9}{2}$.
}
\end{vd}


\begin{vd}%[Cao Thành Thái]%[1D4B2]
Tính giới hạn $\lim\limits_{x\to 0}\dfrac{1-\sqrt[3]{12x+1}}{4x}$.
\loigiai
{
$\lim\limits_{x\to 0}\dfrac{1-\sqrt[3]{12x+1}}{4x} = \lim\limits_{x\to 0}\dfrac{1-(12x+1)}{4x\left[1+\sqrt[3]{12x+1}+\sqrt[3]{(12x+1)^2}\right]} = \lim\limits_{x\to 0}\dfrac{-12x}{4x\left[1+\sqrt[3]{12x+1}+\sqrt[3]{(12x+1)^2}\right]}$\\
$= \lim\limits_{x\to 0}\dfrac{-3}{1+\sqrt[3]{12x+1} + \sqrt[3]{(12x+1)^2}} = \dfrac{-3}{1+\sqrt[3]{12 \cdot 0+1}+\sqrt[2]{(12 \cdot 0+1)^2}} = -1$.
}
\end{vd}


\begin{vd}%[Cao Thành Thái]%[1D4K2]
Tính giới hạn $\lim\limits_{x\to -4}\dfrac{\sqrt{2x+9}-x-5}{\sqrt[3]{x+5}+\sqrt[3]{x+3}}$.
\loigiai
{
$\lim\limits_{x\to -4}\dfrac{\sqrt{2x+9}-x-5}{\sqrt[3]{x+5}+\sqrt[3]{x+3}} = \lim\limits_{x\to -4}\dfrac{\left[2x+9-(x+5)^2\right]\left[\sqrt[3]{(x+5)^2}-\sqrt[3]{(x+5)(x+3)}+\sqrt[3]{(x+3)^2}\right]}{(x+5+x+3)\left(\sqrt{2x+9}+x+5\right)}$\\
$= \lim\limits_{x\to -4}\dfrac{\left(-x^2-8x-16\right)\left[\sqrt[3]{(x+5)^2}-\sqrt[3]{(x+5)(x+3)}+\sqrt[3]{(x+3)^2}\right]}{(2x+8)\left(\sqrt{2x+9}+x+5\right)}$\\
$= \lim\limits_{x\to -4}\dfrac{-(x+4)^2\left[\sqrt[3]{(x+5)^2}-\sqrt[3]{(x+5)(x+3)}+\sqrt[3]{(x+3)^2}\right]}{2(x+4)\left(\sqrt{2x+9}+x+5\right)}$\\
$= \lim\limits_{x\to -4}\dfrac{-(x+4)\left[\sqrt[3]{(x+5)^2}-\sqrt[3]{(x+5)(x+3)}+\sqrt[3]{(x+3)^2}\right]}{2\left(\sqrt{2x+9}+x+5\right)}=0$.
}
\end{vd}


\begin{vd}%[Cao Thành Thái]%[1D4K2]
Tính giới hạn $I = \lim\limits_{x \to 0} \dfrac{(1+x)^{n} - 1}{x}$ với $n$ là số nguyên dương.
\loigiai
{
Đặt $t = 1 + x$. Suy ra $x = t - 1$. Khi $x \to 0$ thì $t \to 1$.\\
Do đó:\\
$I = \lim\limits_{t \to 1} \dfrac{t^n - 1}{t - 1} = \lim\limits_{t \to 1} \dfrac{(t -1)\left(t^{n-1} + t^{n-2} + t^{n-3} + \cdots + t + 1\right)}{t - 1} $\\
$= \lim\limits_{t \to 1} \left(t^{n-1} + t^{n-2} + t^{n-3} + \cdots + t + 1\right) = n$.
}
\end{vd}


\begin{vd}%[Cao Thành Thái]%[1D4B2]
Tính giới hạn $\lim\limits_{x \to 0} \dfrac{\sqrt{1 + ax} - 1}{x}$ với $a \neq 0$.
\loigiai
{
$\lim\limits_{x \to 0} \dfrac{\sqrt{1 + ax} - 1}{x} = \lim\limits_{x \to 0} \dfrac{ax}{x \left(\sqrt{1 + ax} + 1\right)} = \lim\limits_{x \to 0} \dfrac{a}{\sqrt{1 + ax} + 1} = \dfrac{a}{2}$.
}
\end{vd}


\begin{vd}%[Cao Thành Thái]%[1D4K2]
Tính giới hạn $\lim\limits_{x \to 0} \dfrac{\sqrt[3]{1 + ax} - 1}{x}$ với $a \neq 0$.
\loigiai
{
$\lim\limits_{x \to 0} \dfrac{\sqrt[3]{1 + ax} - 1}{x} = \lim\limits_{x \to 0} \dfrac{ax}{x\left[\sqrt[3]{(1 + ax)^2} + \sqrt[3]{1 + ax} +1\right]} = \lim\limits_{x \to 0} \dfrac{a}{\sqrt[3]{(1 + ax)^2} + \sqrt[3]{1 + ax} + 1} = \dfrac{a}{3}$.
}
\end{vd}


\begin{vd}%[Cao Thành Thái]%[1D4G2]
Tính giới hạn $J = \lim\limits_{x \to 0} \dfrac{\sqrt[n]{1 + ax} - 1}{x}$ với $a\neq 0$, $n$ là số nguyên và $n \geq 2$.
\loigiai
{
Đặt $t = \sqrt[n]{1 + ax}$. Suy ra $t^n = 1 + ax \Leftrightarrow x = \dfrac{t^n - 1}{a}$. Khi $x \to 0$ thì $t \to 1$. Do đó:\\
$J = \lim\limits_{t \to 1} \dfrac{t - 1}{\dfrac{t^n - 1}{a}} = \lim\limits_{t \to 1} \dfrac{a(t-1)}{t^n - 1} = \lim\limits_{t \to 1} \dfrac{a(t - 1)}{(t - 1)\left(t^{n-1} + t^{n-2} + t^{n-3} + \cdots + t + 1\right)}$\\
$= \lim\limits_{t \to 1} \dfrac{a}{t^{n-1} + t^{n-2} + t^{n-3} + \cdots + t + 1} = \dfrac{a}{n}$.
}
\end{vd}

\begin{note}
\textbf{\textit{Chú ý:}} Các giới hạn $I = \lim\limits_{x \to 0} \dfrac{(1+x)^{n} - 1}{x} = n$ với $n \in \mathbb{N}$; và $J = \lim\limits_{x \to 0} \dfrac{\sqrt[n]{1 + ax} - 1}{x} = \dfrac{a}{n}$ với $a\neq 0$, $n$ là số nguyên và $n \geq 2$ được gọi là các ``giới hạn cơ bản''.
\end{note}


\begin{vd}%[Cao Thành Thái]%[1D4K2]
Tính giới hạn $\lim\limits_{x\to 1}\dfrac{\sqrt{5-x^3}-\sqrt[3]{x^2+7}}{x^2-1}$.
\loigiai
{
$\lim\limits_{x\to 1}\dfrac{\sqrt{5-x^3}-\sqrt[3]{x^2+7}}{x^2-1} = \lim\limits_{x\to 1}\dfrac{\sqrt{5-x^3}-2+2-\sqrt[3]{x^2+7}}{x^2-1} = \lim\limits_{x\to 1}\left(\dfrac{\sqrt{5-x^3}-2}{x^2-1}+\dfrac{2-\sqrt[3]{x^2+7}}{x^2-1}\right)$\\
$= \lim\limits_{x\to 1}\left\{\dfrac{1-x^3}{(x^2-1)\left(\sqrt{5-x^3}+2\right)}+\dfrac{1-x^2}{(x^2-1)\left[\sqrt[3]{(x^2+7)^2}+2\sqrt[3]{x^2+7}+4\right]}\right\}$\\
$= \lim\limits_{x\to 1}\left[\dfrac{-(x^2+x+1)}{(x+1)\left(\sqrt{5-x^3}+2\right)}-\dfrac{1}{\sqrt[3]{(x^2+7)^2}+2\sqrt[3]{x^2+7}+4}\right]$\\
$= \dfrac{-(1^2+1+1)}{(1+1) \cdot \left(\sqrt{5-1^3}+2\right)}-\dfrac{1}{\sqrt[3]{(1^2+7)^2}+2\sqrt[3]{1^2+7}+4} = -\dfrac{11}{24}$.
}
\end{vd}
\begin{vd}%[Vũ Văn Trường]%[1D4B2]
Tính các giới hạn sau:
\begin{enumerate}
\begin{multicols}{2}
\item $\lim\limits_{x\to 3}\dfrac{x^3 - 4x^2 + 4x -3}{x^2 - 3x}$.
\item $\lim\limits_{x\to \frac1{2}}\dfrac{8x^3 -1}{6x^2 - 5x +1}$.
\item $\lim\limits_{x\to 0}\dfrac{(1+x)^3 - (1+3x)}{x^2 + x^3}$.
\item $\lim\limits_{x\to -1}\dfrac{x^{2017}+1}{x^{2018}+1}$.
\end{multicols}
\end{enumerate}
\loigiai{
\begin{enumerate}
\item $\lim\limits_{x\to 3}\dfrac{x^3 - 4x^2 + 4x -3}{x^2 - 3x}=\lim\limits_{x\to 3}\dfrac{(x-3)(x^2-x+1)}{x(x-3)}=\lim\limits_{x\to 3}\dfrac{x^2-x+1}{x}=\dfrac{7}{3}$.
\item $\lim\limits_{x\to \frac1{2}}\dfrac{8x^3 -1}{6x^2 - 5x +1}=\lim\limits_{x\to \frac1{2}}\dfrac{\left(x-\dfrac{1}{2}\right)\left(8x^2+4x+2\right)}{\left(x-\dfrac{1}{2}\right)\left(6x-2\right)}=\lim\limits_{x\to \frac1{2}}\dfrac{8x^2+4x+2}{6x-2}=6$.
\item $\lim\limits_{x\to 0}\dfrac{(1+x)^3 - (1+3x)}{x^2 + x^3}=\lim\limits_{x\to 0}\dfrac{x^3+3x^2}{x^2 + x^3}=\lim\limits_{x\to 0}\dfrac{x+3}{x+1}=3$.
\item $\lim\limits_{x\to -1}\dfrac{x^{2017}+1}{x^{2018}+1}=\lim\limits_{x\to 1}\dfrac{1-x^{2017}}{1-x^{2018}}=\lim\limits_{x\to 1}\dfrac{1+x+x^2+\cdots+x^{2016}}{1+x+x^2+\cdots+x^{2017}}=\dfrac{2017}{2018}$.
\end{enumerate}
}
\end{vd}

\begin{vd}%[Vũ Văn Trường]%[1D4B2]
Tính giới hạn $\lim\limits_{x\to 1}\dfrac{2x - \sqrt{3x+1}}{x^2 -1}$.
\loigiai{
\begin{align*}
\lim\limits_{x\to 1}\dfrac{2x - \sqrt{3x+1}}{x^2 -1}&=\lim\limits_{x\to 1}\dfrac{4x^2 - (3x+1)}{\left(x^2 -1\right)\left(2x+ \sqrt{3x+1}\right)}\\
&=\lim\limits_{x\to 1}\dfrac{(x-1)(4x+1)}{\left(x -1\right)\left(x +1\right)\left(2x+ \sqrt{3x+1}\right)}\\
&=\lim\limits_{x\to 1}\dfrac{4x+1}{\left(x +1\right)\left(2x+ \sqrt{3x+1}\right)}\\
&=\dfrac{5}{8}.
\end{align*}
}
\end{vd}

\begin{vd}%[Vũ Văn Trường]%[1D4B2]
Tính giới hạn $\lim\limits_{x\to 2}\dfrac{\sqrt{x^2-x}-\sqrt{2x-2}}{x^2-2x}$.
\loigiai{
\begin{align*}
\lim\limits_{x\to 2}\dfrac{\sqrt{x^2-x}-\sqrt{2x-2}}{x^2-2x}&=\lim\limits_{x\to 2}\dfrac{(x^2-x) - (2x-2)}{\left(x^2 -2x\right)\left(\sqrt{x^2-x}+\sqrt{2x-2}\right)}\\
&=\lim\limits_{x\to 2}\dfrac{(x-1)(x-2)}{x\left(x-2\right)\left(\sqrt{x^2-x}+\sqrt{2x-2}\right)}\\
&=\lim\limits_{x\to 2}\dfrac{x-1}{x\left(\sqrt{x^2-x}+\sqrt{2x-2}\right)}\\
&=\dfrac{1}{4\sqrt{2}}\\
&=\dfrac{\sqrt{2}}{8}.
\end{align*}
}
\end{vd}

\begin{vd}%[Vũ Văn Trường]%[1D4B2]
Tính giới hạn $\lim\limits_{x\to 1}\dfrac{\sqrt[3]{2x-1} -\sqrt[3]{x}}{\sqrt{x} -1}$.
\loigiai{
\begin{align*}
\lim\limits_{x\to 1}\dfrac{\sqrt[3]{2x-1} -\sqrt[3]{x}}{\sqrt{x} -1}&=\lim\limits_{x\to 1}\dfrac{\left[(2x-1) - x\right]\left(\sqrt{x}+1\right)}{\left(x-1\right)\left[\sqrt[3]{(2x-1)^2}+ \sqrt[3]{2x-1}\cdot \sqrt[3]{x}+\sqrt[3]{x^2}\right]}\\
&=\lim\limits_{x\to 1}\dfrac{\sqrt{x}+1}{\sqrt[3]{(2x-1)^2}+ \sqrt[3]{2x-1}\cdot \sqrt[3]{x}+\sqrt[3]{x^2}}\\
&=\dfrac{2}{3}.
\end{align*}
}
\end{vd}

\begin{vd}%[Vũ Văn Trường]%[1D4K2]
Tính giới hạn $\lim\limits_{x\to -2}\dfrac{\sqrt[3]{x^2-2x}-\sqrt{2-x}}{x^2+5x+6}$.
\loigiai{
\begin{align*}
&\lim\limits_{x\to -2}\dfrac{\sqrt[3]{x^2-2x}-\sqrt{2-x}}{x^2+5x+6}\\
=&\lim\limits_{x\to -2}\dfrac{(\sqrt[3]{x^2-2x}-2)+(2-\sqrt{2-x})}{(x+2)(x+3)}&&\\
=&\lim\limits_{x\to -2}\left[\dfrac{\sqrt[3]{x^2-2x}-2}{(x+2)(x+3)}+ \dfrac{2-\sqrt{2-x}}{(x+2)(x+3)}\right]&&\\
=&\lim\limits_{x\to -2}\left[\dfrac{x^2-2x-8}{(x+2)(x+3)(\sqrt[3]{(x^2-2x)^2}+2\sqrt[3]{x^2-2x}+4)}+ \dfrac{2+x}{(x+2)(x+3)( 2+\sqrt{2-x})}\right]
&&\\
=&\lim\limits_{x\to -2}\left[\dfrac{x-4}{(x+3)(\sqrt[3]{(x^2-2x)^2}+2\sqrt[3]{x^2-2x}+4)}+ \dfrac{1}{(x+3)( 2+\sqrt{2-x})}\right]&&\\
=&-\dfrac{1}{2}+\dfrac{1}{4}=-\dfrac{1}{4}.
\end{align*}
}
\end{vd}
\begin{vd}%[Vũ Văn Trường]%[1D4G2]
Tính giới hạn $\lim \limits_{x \to 1} \dfrac{\left( 1 - \sqrt x \right)\left( 1 - \sqrt[3]{x} \right)\cdots\left( 1 - \sqrt[n]{x} \right)}{\left( {1 - x} \right)^{n - 1}}$.
\loigiai{
Ta có $\lim \limits_{x \to 1} \dfrac{\left( 1 - \sqrt x \right)\left( 1 - \sqrt[3]{x} \right)\cdots\left( 1 - \sqrt[n]{x} \right)}{\left( {1 - x} \right)^{n - 1}}=\lim \limits_{x \to 1}\left[\dfrac{1 - \sqrt{x}}{1 - x}\cdot\dfrac{1 - \sqrt[3]{x}}{1 - x}\cdots\dfrac{1 - \sqrt[n]{x}}{1 - x}\right]$.\\
Với $n$ là số tự nhiên không bé hơn $2$, ta sẽ chứng minh $\lim \limits_{x \to 1} \dfrac{1 - \sqrt[n]{x}}{1 - x} = \dfrac{1}{n}$. Thật vậy, đặt $t=\sqrt[n]{x} \Rightarrow x=t^n$ và khi $x \to 1$ thì $t \to 1$.\\
Khi đó ta có
\begin{align*}
\lim \limits_{x \to 1} \dfrac{1 - \sqrt[n]{x}}{1 - x}&=\lim \limits_{t \to 1} \dfrac{1 - t}{1 - t^n}\\
&=\lim \limits_{t \to 1} \dfrac{1 - t}{(1 - t)\left(1+t+t^2+\cdots+t^{n-1}\right)}\\
&=\lim \limits_{t \to 1} \dfrac{1}{1+t+t^2+\cdots+t^{n-1}}\\
&=\dfrac{1}{n}
\end{align*}
Từ đó suy ra  $\lim\limits_{x \to 1}\left[\dfrac{1 - \sqrt{x}}{1 - x}\cdot\dfrac{1 - \sqrt[3]{x}}{1 - x}\cdots\dfrac{1 - \sqrt[n]{x}}{1 - x}\right]=\dfrac{1}{2}\cdot\dfrac{1}{3}\cdots\dfrac{1}{n}=\dfrac{1}{n!}$.
}
\end{vd}


\begin{vd}%[Cao Thành Thái]%[1D4G2]
Tính giới hạn $\lim\limits_{x\to 0}\dfrac{(x^2+1998)\sqrt[7]{1-2x}-1998}{x}$.
\loigiai
{
$\lim\limits_{x\to 0}\dfrac{(x^2+1998)\sqrt[7]{1-2x}-1998}{x}$\\
$= \lim\limits_{x\to 0}\dfrac{(x^2+1998)\sqrt[7]{1-2x}-(x^2+1998)+(x^2+1998)-1998}{x}$\\
$= \lim\limits_{x\to 0}\left[\dfrac{(x^2+1998)\sqrt[7]{1-2x}-(x^2+1998)}{x}+\dfrac{x^2}{x}\right] = \lim\limits_{x\to 0}\left[(x^2+1998) \cdot \dfrac{\sqrt[7]{1-2x}-1}{x}+x\right]$\\
$= (0^2+1998) \cdot \left(-\dfrac{2}{7}\right)+0=-\dfrac{3996}{7}$.
}
\end{vd}


\begin{center}
\textbf{BÀI TẬP TỰ LUYỆN}
\end{center}

\begin{bt}%[Cao Thành Thái]%[1D4B2]
Tính các giới hạn sau:
\begin{multicols}{2}
\begin{enumerate}
\item $\lim\limits_{x \to -1}\dfrac{4x^2-x-5}{7x^2+5x-2}$.
\item $\lim\limits_{x \to -2}\dfrac{4-x^2}{x+2}$.
\item $\lim\limits_{x \to 3}\dfrac{x^2+2x-15}{x-3}$.
\item $\lim\limits_{x \to 2}\dfrac{2x^2-5x+2}{x^2-4}$.
\end{enumerate}
\end{multicols}
\loigiai
{
\begin{enumerate}
\item $\lim\limits_{x \to -1}\dfrac{4x^2-x-5}{7x^2+5x-2} = \lim\limits_{x\to -1}\dfrac{(x+1)(4x-5)}{(x+1)(7x-2)} = \lim\limits_{x\to -1}\dfrac{4x-5}{7x-2} = \dfrac{4 \cdot (-1)-5}{7 \cdot (-1)-2} = 1$.

\item $\lim\limits_{x \to -2}\dfrac{4-x^2}{x+2} = \lim\limits_{x \to -2} \dfrac{(2-x)(2+x)}{x+2} = \lim\limits_{x \to -2} (2-x) = 4$.

\item $\lim\limits_{x \to 3}\dfrac{x^2+2x-15}{x-3} = \lim\limits_{x \to 3} \dfrac{(x-3)(x+5)}{x-3} = \lim\limits_{x \to 3} (x + 5) = 8$.

\item $\lim\limits_{x \to 2}\dfrac{2x^2-5x+2}{x^2-4} = \lim\limits_{x \to 2}\dfrac{(x-2)(2x-1)}{(x-2)(x+2)} = \lim\limits_{x \to 2}\dfrac{2x-1}{x+2} = \dfrac{3}{4}$.

\end{enumerate}
}
\end{bt}


\begin{bt}%[Cao Thành Thái]%[1D4K2]
Tính các giới hạn sau:
\begin{multicols}{2}
\begin{enumerate}
\item $\lim\limits_{x \to 1}\dfrac{x^3-x^2-x+1}{x^2-3x+2}$.
\item $\lim\limits_{x \to 1}\dfrac{x^4-1}{x^3-2x^2+1}$.
\item $\lim\limits_{x \to -1}\dfrac{x^5+1}{x^3+1}$.
\item $\lim\limits_{x \to 3}\dfrac{x^3-5x^2+3x+9}{x^4-8x^2-9}$.
\end{enumerate}
\end{multicols}
\loigiai
{
\begin{enumerate}
\item $\lim\limits_{x \to 1}\dfrac{x^3-x^2-x+1}{x^2-3x+2} = \lim\limits_{x \to 1}\dfrac{(x-1)(x^2-1)}{(x-1)(x-2)} = \lim\limits_{x \to 1}\dfrac{x^2-1}{x-2} = 0$.

\item $\lim\limits_{x \to 1}\dfrac{x^4-1}{x^3-2x^2+1} = \lim\limits_{x \to 1}\dfrac{(x-1)(x^3+x^2+x+1)}{(x-1)(x^2-x-1} = \lim\limits_{x \to 1}\dfrac{x^3+x^2+x+1}{x^2-x-1} = -4$.

\item $\lim\limits_{x \to -1}\dfrac{x^5+1}{x^3+1} = \lim\limits_{x \to -1}\dfrac{(x+1)(x^4-x^3+x^2-x+1)}{(x+1)(x^2-x+1)} = \lim\limits_{x \to -1}\dfrac{x^4-x^3+x^2-x+1}{x^2-x+1} = \dfrac{5}{3}$.

\item $\lim\limits_{x \to 3}\dfrac{x^3-5x^2+3x+9}{x^4-8x^2-9} = \lim\limits_{x \to 3}\dfrac{(x-3)(x^2-2x-3)}{(x-3)(x^3+3x^2+x+3)} = \lim\limits_{x \to 3}\dfrac{x^2-2x-3}{x^3+3x^2+x+3} = 0$.

\end{enumerate}
}
\end{bt}


\begin{bt}%[Cao Thành Thái]%[1D4B2]
Tính giới hạn $\lim\limits_{x \to 0}\dfrac{\sqrt{1+2x} - 1}{2x}$.
\loigiai
{
$\lim\limits_{x \to 0}\dfrac{\sqrt{1+2x} - 1}{2x} = \lim\limits_{x \to 0}\dfrac{2x}{2x\left(\sqrt{1+2x}+1\right)} = \lim\limits_{x \to 0}\dfrac{1}{\sqrt{1+2x}+1} = \dfrac{1}{2}$.
}
\end{bt}


\begin{bt}%[Cao Thành Thái]%[1D4B2]
Tính giới hạn $\lim\limits_{x \to 2} \dfrac{x - \sqrt{3x-2}}{x^2 - 4}$.
\loigiai
{
$\lim\limits_{x \to 2} \dfrac{x - \sqrt{3x-2}}{x^2 - 4} = \lim\limits_{x \to 2}\dfrac{x^2 - 3x + 2}{(x^2 - 4) \left(x + \sqrt{3x-2}\right)} = \lim\limits_{x \to 2}\dfrac{(x-2)(x-1)}{(x-2)(x+2)\left(x+\sqrt{3x-2}\right)}$\\
$= \lim\limits_{x \to 2}\dfrac{x-1}{(x+2)\left(x + \sqrt{3x-2}\right)} = \dfrac{1}{16}$.
}
\end{bt}


\begin{bt}%[Cao Thành Thái]%[1D4B2]
Tính giới hạn $\lim\limits_{x \to 0}\dfrac{\sqrt{1+x^2} - 1}{2x^3 - 3x^2}$.
\loigiai
{
$\lim\limits_{x \to 0}\dfrac{\sqrt{1+x^2} - 1}{2x^3 - 3x^2} = \lim\limits_{x \to 0}\dfrac{x^2}{(2x^3 - 3x^2) \left(\sqrt{1+x^2} + 1\right)} = \lim\limits_{x \to 0}\dfrac{1}{(2x - 3) \left(\sqrt{1+x^2} + 1\right)} = -\dfrac{1}{6}$.
}
\end{bt}


\begin{bt}%[Cao Thành Thái]%[1D4K2]
Tính giới hạn $\lim\limits_{x \to 1}\dfrac{\sqrt{2x+7} - x - 2}{x^3 - 4x + 3}$.
\loigiai
{
$\lim\limits_{x \to 1}\dfrac{\sqrt{2x+7} - x - 2}{x^3 - 4x + 3} = \lim\limits_{x \to 1}\dfrac{2x+7-(x+2)^2}{(x^3-4x+3)\left(\sqrt{2x+7}+x+2\right)} = \lim\limits_{x \to 1}\dfrac{-x^2 - 2x + 3}{(x^3-4x+3)\left(\sqrt{2x+7}+x+2\right)}$\\
$= \lim\limits_{x \to 1}\dfrac{-(x-1)(x+3)}{(x-1)(x^2+x-3)\left(\sqrt{2x+7}+x+2\right)} = \lim\limits_{x \to 1}\dfrac{-(x+3)}{(x^2+x-3)\left(\sqrt{2x+7}+x+2\right)} = \dfrac{2}{3}$.
}
\end{bt}


\begin{bt}%[Cao Thành Thái]%[1D42]
Tính giới hạn $\lim\limits_{x \to -1}\dfrac{x^2 - 8x - 9}{\sqrt{4-3x^2} - 2x - 3}$.
\loigiai
{
$\lim\limits_{x \to -1}\dfrac{x^2 - 8x - 9}{\sqrt{4-3x^2} - 2x - 3} = \lim\limits_{x \to -1}\dfrac{(x^2 - 8x - 9)\left(\sqrt{4-3x^2} + 2x + 3\right)}{4 - 3x^2 - (2x + 3)^2}$\\
$= \lim\limits_{x \to -1}\dfrac{(x^2 - 8x - 9)\left(\sqrt{4 - 3x^2} + 2x + 3\right)}{-7x^2 - 12x - 5} = \lim\limits_{x \to -1}\dfrac{(x+1)(x-9)\left(\sqrt{4-3x^2} + 2x + 3\right)}{(x+1)(-7x - 5)}$\\
$= \lim\limits_{x \to -1}\dfrac{(x - 9) \left(\sqrt{4 - 3x^2} + 2x + 3\right)}{-7x - 5} = -10$.
}
\end{bt}


\begin{bt}%[Cao Thành Thái]%[1D4B2]
Tính giới hạn $\lim\limits_{x \to 0}\dfrac{1 - \sqrt[3]{x+1}}{3x}$.
\loigiai
{
$\lim\limits_{x \to 0}\dfrac{1 - \sqrt[3]{x+1}}{3x} = \lim\limits_{x \to 0}\dfrac{-x}{3x\left[1 + \sqrt[3]{x+1} + \sqrt[3]{(x+1)^2}\right]} = \lim\limits_{x \to 0}\dfrac{-1}{3 + 3\sqrt[3]{x+1} + 3\sqrt[3]{(x+1)^2}} = -\dfrac{1}{9}$.
}
\end{bt}


\begin{bt}%[Cao Thành Thái]%[1D4B2]
Tính giới hạn $\lim\limits_{x \to 1}\dfrac{\sqrt[3]{x-2} + \sqrt[3]{1-x+x^2}}{x^2-1}$.
\loigiai
{
$\lim\limits_{x \to 1}\dfrac{\sqrt[3]{x-2} + \sqrt[3]{1-x+x^2}}{x^2-1} = \lim\limits_{x \to 1}\dfrac{x^2 - 1}{(x^2 - 1)\left[\sqrt[3]{(x-2)^2} - \sqrt[3]{(x-2)(1-x+x^2)} + \sqrt[3]{(1-x+x^2)^2}\right]}$\\
$= \lim\limits_{x \to 1}\dfrac{1}{\sqrt[3]{(x-2)^2} - \sqrt[3]{(x-2)(1-x+x^2)} + \sqrt[3]{(1-x+x^2)^2}} = \dfrac{1}{3}$.
}
\end{bt}


\begin{bt}%[Cao Thành Thái]%[1D4B2]
Tính giới hạn $\lim\limits_{x \to 1}\dfrac{\sqrt[3]{3x-2} - \sqrt[3]{4x^2-x-2}}{x^2 - 3x +2}$.
\loigiai
{
$\lim\limits_{x \to 1}\dfrac{\sqrt[3]{3x-2} - \sqrt[3]{4x^2-x-2}}{x^2 - 3x +2}$\\
$= \lim\limits_{x \to 1}\dfrac{-4x^2+4x}{(x^2-3x+2)\left[\sqrt[3]{(3x-2)^2} + \sqrt[3]{(3x-2)(4x^2-x-2)} + \sqrt[3]{(4x^2-x-2)^2}\right]}$\\
$= \lim\limits_{x \to 1}\dfrac{-4x(x-1)}{(x-1)(x-2)\left[\sqrt[3]{(3x-2)^2} + \sqrt[3]{(3x-2)(4x^2-x-2)} + \sqrt[3]{(4x^2-x-2)^2}\right]}$\\
$= \lim\limits_{x \to 1}\dfrac{-4x}{(x-2)\left[\sqrt[3]{(3x-2)^2} + \sqrt[3]{(3x-2)(4x^2-x-2)} + \sqrt[3]{(4x^2-x-2)^2}\right]}$\\
$= \dfrac{4}{3}$.
}
\end{bt}


\begin{bt}%[Cao Thành Thái]%[1D4B2]
Tính giới hạn $\lim\limits_{x \to 2}\dfrac{\sqrt[3]{3x+2} + x - 4}{x^2 - 3x + 2}$.
\loigiai
{
$\lim\limits_{x \to 2}\dfrac{\sqrt[3]{3x+2} + x - 4}{x^2 - 3x + 2} = \lim\limits_{x \to 2}\dfrac{3x + 2 + (x-4)^3}{(x^2 - 3x + 2)\left[\sqrt[3]{(3x+2)^2} - (x-4)\sqrt[3]{3x+2} + (x-4)^2\right]}$\\
$= \lim\limits_{x \to 2}\dfrac{x^3 - 12x^2 + 51x - 62}{(x^2 - 3x + 2)\left[\sqrt[3]{(3x+2)^2} - (x-4)\sqrt[3]{3x+2} + (x-4)^2\right]}$\\
$= \lim\limits_{x \to 2}\dfrac{(x-2)(x^2 - 10x + 31)}{(x-2)(x-1)\left[\sqrt[3]{(3x+2)^2} - (x-4)\sqrt[3]{3x+2} + (x-4)^2\right]}$\\
$= \lim\limits_{x \to 2}\dfrac{x^2 - 10x + 31}{(x-1)\left[\sqrt[3]{(3x+2)^2} - (x-4)\sqrt[3]{3x+2} + (x-4)^2\right]}$\\
$= \dfrac{5}{4}$.
}
\end{bt}


\begin{bt}%[Cao Thành Thái]%[1D4K2]
Tính giới hạn $\lim\limits_{x \to 4}\dfrac{\sqrt[3]{x+4} + \sqrt[3]{4-3x}}{\sqrt{x^2+9} - \sqrt{x + 21}}$.
\loigiai
{
$\lim\limits_{x \to 4}\dfrac{\sqrt[3]{x+4} + \sqrt[3]{4-3x}}{\sqrt{x^2+9} - \sqrt{x + 21}}$
$= \lim\limits_{x \to 4}\dfrac{(8-2x)\left(\sqrt{x^2+9} + \sqrt{x + 21}\right)}{(x^2-x-12)\left[\sqrt[3]{(x+4)^2} - \sqrt[3]{(x+4)(4-3x)} + \sqrt[3]{(4-3x)^2}\right]}$\\
$= \lim\limits_{x \to 4}\dfrac{-2(x-4)\left(\sqrt{x^2+9} + \sqrt{x + 21}\right)}{(x-4)(x+3)\left[\sqrt[3]{(x+4)^2} - \sqrt[3]{(x+4)(4-3x)} + \sqrt[3]{(4-3x)^2}\right]}$\\
$= \lim\limits_{x \to 4}\dfrac{-2\left(\sqrt{x^2+9} + \sqrt{x + 21}\right)}{(x+3)\left[\sqrt[3]{(x+4)^2} - \sqrt[3]{(x+4)(4-3x)} + \sqrt[3]{(4-3x)^2}\right]}$\\
$= -\dfrac{5}{3}$.
}
\end{bt}


\begin{bt}%[Cao Thành Thái]%[1D4K2]
Tính giới hạn $\lim\limits_{x\to 0}\displaystyle\frac{\sqrt{8x^3+x^2+6x+9}-\sqrt[3]{9x^2+27x+27}}{x^3}$.
\loigiai
{
Ta có: $\dfrac{\sqrt{8x^3+x^2+6x+9}-\sqrt[3]{9x^2+27x+27}}{x^3}$\\
$= \dfrac{\sqrt{8x^3+x^2+6x+9}-(x+3)+(x+3)-\sqrt[3]{9x^2+27x+27}}{x^3}$\\
$= \dfrac{\sqrt{8x^3+x^2+6x+9}-(x+3)}{x^3}+\dfrac{(x+3)-\sqrt[3]{9x^2+27x+27}}{x^3}$\\
$= \dfrac{8x^3}{x^3\left(\sqrt{8x^3+x^2+6x+9}+x+3\right)}+\dfrac{x^3}{x^3\left[(x+3)^2+(x+3)\sqrt[3]{9x^2+27x+27}+\sqrt[3]{(9x^2+27x+27)^2}\right]}$\\
$= \dfrac{8}{\sqrt{8x^3+x^2+6x+9}+x+3}+\dfrac{1}{(x+3)^2+(x+3)\sqrt[3]{9x^2+27x+27}+\sqrt[3]{(9x^2+27x+27)^2}}$\\
Do đó:\\
$\lim\limits_{x\to 0}\displaystyle\frac{\sqrt{8x^3+x^2+6x+9}-\sqrt[3]{9x^2+27x+27}}{x^3}$\\
$= \dfrac{8}{\sqrt{8 \cdot 0^3+0^2+6 \cdot 0+9}+0+3}+\dfrac{1}{(0+3)^2+(0+3)\sqrt[3]{9 \cdot 0^2+27 \cdot 0+27}+\sqrt[3]{(9 \cdot 0^2+27 \cdot 0+27)^2}}$\\
$= \dfrac{37}{27}$.
}
\end{bt}


\begin{bt}%[Cao Thành Thái]%[1D4K2]
Tính giới hạn $\lim\limits_{x \to 1}\dfrac{\sqrt{5 - x^3} - \sqrt[3]{x^2 + 7}}{x^2 -1}$.
\loigiai
{
Ta có:\\
$\dfrac{\sqrt{5 - x^3} - \sqrt[3]{x^2 + 7}}{x^2 -1} = \dfrac{\sqrt{5-x^3}-2}{x^2-1} + \dfrac{2 - \sqrt[3]{x^2+7}}{x^2-1}$\\
$= \dfrac{-(x^3-1)}{(x^2-1)\left(\sqrt{5-x^3} + 2\right)} + \dfrac{1-x^2}{(x^2-1)\left[4 + 2 \sqrt[3]{x^2+7} + \sqrt[3]{(x^2+7)^2}\right]}$\\
$= \dfrac{-(x^2+x+1)}{(x+1)\left(\sqrt{5-x^3} + 2\right)} - \dfrac{1}{4 + 2 \sqrt[3]{x^2+7} + \sqrt[3]{(x^2+7)^2}}$.\\
Do đó: $\lim\limits_{x \to 1}\dfrac{\sqrt{5 - x^3} - \sqrt[3]{x^2 + 7}}{x^2 -1} = -\dfrac{3}{8} - \dfrac{1}{12} = -\dfrac{11}{24}$.
}
\end{bt}


\begin{bt}%[Cao Thành Thái]%[1D4K2]
Tính giới hạn $\lim\limits_{x \to 2}\dfrac{\sqrt[3]{8x+11} - \sqrt{x+7}}{x^2 - 3x +2}$.
\loigiai
{
Ta có:\\
$\dfrac{\sqrt[3]{8x+11} - \sqrt{x+7}}{x^2 - 3x +2} = \dfrac{\sqrt[3]{5x+11}-3}{x^2 - 3x +2} + \dfrac{3-\sqrt{x+7}}{x^2 - 3x + 2}$\\
$= \dfrac{8x-16}{(x-2)(x-1)\left[\sqrt[3]{(8x+11)^2}+3\sqrt[3]{8x+11}+9\right]} - \dfrac{x-2}{(x-2)(x-1)\left(3+\sqrt{x+7}\right)}$\\
$= \dfrac{8}{(x-1)\left[\sqrt[3]{(8x+11)^2}+3\sqrt[3]{8x+11}+9\right]} - \dfrac{1}{(x-1)\left(3+\sqrt{x+7}\right)}$.\\
Do đó: $\lim\limits_{x \to 2}\dfrac{\sqrt[3]{8x+11} - \sqrt{x+7}}{x^2 - 3x +2} = \dfrac{8}{27} - \dfrac{1}{6} = \dfrac{7}{54}$.
}
\end{bt}


\begin{bt}%[Cao Thành Thái]%[1D4K2]
Tính giới hạn $\lim\limits_{x \to 1}\dfrac{\sqrt{3x+1} + \sqrt{x^2+8} - 5}{x^2 - 3x + 2}$.
\loigiai
{
Ta có:\\
$\dfrac{\sqrt{3x+1} + \sqrt{x^2+8} - 5}{x^2 - 3x + 2} = \dfrac{\sqrt{3x+1}-2}{x^2-3x+2} + \dfrac{\sqrt{x^2+8}-3}{x^2-3x+2}$\\
$= \dfrac{3x-3}{(x-1)(x-2)\left(\sqrt{3x+1}+2\right)} + \dfrac{(x-1)(x+1)}{(x-1)(x-2)\left(\sqrt{x^2+8}+3\right)}$\\
$= \dfrac{3}{(x-2)\left(\sqrt{3x+1}+2\right)} + \dfrac{x+1}{(x-2)\left(\sqrt{x^2+8}+3\right)}$.\\
Do đó: $\lim\limits_{x \to 1}\dfrac{\sqrt{3x+1} + \sqrt{x^2+8} - 5}{x^2 - 3x + 2} = -\dfrac{3}{4} - \dfrac{2}{6} = -\dfrac{13}{12}$.
}
\end{bt}


\begin{bt}%[Cao Thành Thái]%[1D4K2]
Tính giới hạn $\lim\limits_{x \to 2}\dfrac{4x - \sqrt{x+2} - \sqrt{5x+26}}{x - 2}$.
\loigiai
{
Ta có:\\
$\dfrac{4x - \sqrt{x+2} - \sqrt{5x+26}}{x - 2} = \dfrac{x - \sqrt{x+2}}{x-2} + \dfrac{3x - \sqrt{5x+26}}{x-2}$\\
$= \dfrac{x^2-x-2}{(x-2)\left(x+\sqrt{x+2}\right)} + \dfrac{9x^2-5x-26}{(x-2)\left(3x+\sqrt{5x+26}\right)}$\\
$= \dfrac{x+1}{x+\sqrt{x+2}} + \dfrac{9x+13}{3x+\sqrt{5x+26}}$.\\
Do đó: $\lim\limits_{x \to 2}\dfrac{4x - \sqrt{x+2} - \sqrt{5x+26}}{x - 2} = \dfrac{3}{4} + \dfrac{31}{12} = \dfrac{10}{3}$.
}
\end{bt}


\begin{bt}%[Cao Thành Thái]%[1D4K2]
Tính giới hạn $\lim\limits_{x \to -2}\dfrac{\sqrt[3]{x^2 - x + 2} + \sqrt{x + 3} - 3}{2x^2 + 5x + 2}$.
\loigiai
{
Ta có:\\
$\dfrac{\sqrt[3]{x^2 - x + 2} + \sqrt{x + 3} - 3}{2x^2 + 5x + 2} = \dfrac{\sqrt[3]{x^2-x+2}-2}{2x^2+5x+2} + \dfrac{\sqrt{x+3}-1}{2x^2+5x+2}$\\
$= \dfrac{x^2-x-6}{(x+2)(2x+1)\left[\sqrt[3]{(x^2-x+2)^2} + 2\sqrt[3]{x^2-x+2} + 4\right]} + \dfrac{x+2}{(x+2)(2x+1)\left(\sqrt{x+3} + 1\right)}$\\
$= \dfrac{x-3}{(2x+1)\left[\sqrt[3]{(x^2-x+2)^2} + 2\sqrt[3]{x^2-x+2} + 4\right]} + \dfrac{1}{(2x+1)\left(\sqrt{x+3} + 1\right)}$.\\
Do đó: $\lim\limits_{x \to -2}\dfrac{\sqrt[3]{x^2 - x + 2} + \sqrt{x + 3} - 3}{2x^2 + 5x + 2} = \dfrac{5}{36} - \dfrac{1}{6} = -\dfrac{1}{36}$.
}
\end{bt}

\begin{center}
\textbf{BÀI TẬP TỔNG HỢP}
\end{center}

\begin{bt}%[Cao Thành Thái]%[1D4K2]
Tính giới hạn $\lim\limits_{x \to 2}\dfrac{(x^2 - x - 2)^{20}}{(x^3 - 12x + 16)^{10}}$.
\loigiai
{
$\lim\limits_{x \to 2}\dfrac{(x^2 - x - 2)^{20}}{(x^3 - 12x + 16)^{10}} = \lim\limits_{x \to 2}\dfrac{(x+1)^{20} \cdot (x-2)^{20}}{(x-2)^{20} \cdot (x+4)^{10}} = \lim\limits_{x \to 2}\dfrac{(x+1)^{20}}{(x+4)^{10}} = \dfrac{3^{20}}{6^{10}} = \left(\dfrac{3}{2}\right)^{10}$.
}
\end{bt}


\begin{bt}%[Cao Thành Thái]%[1D4K2]
Tính giới hạn $\lim\limits_{x \to 1}\dfrac{x^{100} - 2x + 1}{x^{50} - 2x + 1}$.
\loigiai
{
$\lim\limits_{x \to 1}\dfrac{x^{100} - 2x + 1}{x^{50} - 2x + 1} = \lim\limits_{x \to 1}\dfrac{(x^{100}-1)-2(x-1)}{(x^{50}-1)-2(x-1)} = \lim\limits_{x \to 1}\dfrac{(x-1)(x^{99}+x^{98}+ \cdots + x +1 - 2)}{(x-1)(x^{49}+x^{48}+ \cdots + x +1 - 2)}$\\
$= \lim\limits_{x \to 1}\dfrac{x^{99}+x^{98}+ \cdots + x - 1}{x^{49}+x^{48}+ \cdots + x - 1} = \dfrac{98}{48} = \dfrac{49}{24}$.
}
\end{bt}


\begin{bt}%[Cao Thành Thái]%[1D4K2]
Tính giới hạn $\lim\limits_{x \to 1} \dfrac{\sqrt{x^5} - 1}{1 - x^4}$.
\loigiai
{
Ta có:\\
$\dfrac{\sqrt{x^5} - 1}{1 - x^4} = \dfrac{x^5 - 1}{-(x^4 - 1)\left(\sqrt{x^5} + 1\right)} = \dfrac{x^4 + x^3 + x^2 + x + 1}{-(x^3 + x^2 + x + 1)\left(\sqrt{x^5} + 1\right)}$.\\
Do đó: $\lim\limits_{x \to 1} \dfrac{\sqrt{x^5} - 1}{1 - x^4} = -\dfrac{5}{8}$.
}
\end{bt}


\begin{bt}%[Cao Thành Thái]%[1D4K2]
Tính giới hạn $\lim\limits_{x \to 1}\dfrac{3\sqrt[3]{x^2} + 2\sqrt{x} - 5}{x - 1}$.
\loigiai
{
Ta có:\\
$\dfrac{3\sqrt[3]{x^2} + 2\sqrt{x} - 5}{x - 1} = \dfrac{3\sqrt[3]{x^2} - 3}{x - 1} + \dfrac{2\sqrt{x} - 2}{x - 1} = \dfrac{3(x^2 - 1)}{(x - 1)\left(\sqrt[3]{x^4} + \sqrt[3]{x^2} + 1\right)} + \dfrac{2(x-1)}{(x - 1)\left(\sqrt{x} - 1\right)}$\\
$= \dfrac{3(x+1)}{\sqrt[3]{x^4} + \sqrt[3]{x^2} + 1} + \dfrac{2}{\sqrt{x} + 1}$.\\
Do đó: $\lim\limits_{x \to 1}\dfrac{3\sqrt[3]{x^2} + 2\sqrt{x} - 5}{x - 1} = \dfrac{6}{3} + 1 = 3$.
}
\end{bt}


\begin{bt}%[Cao Thành Thái]%[1D4K2]
Tính giới hạn $\lim\limits_{x \to -1}\dfrac{\sqrt[3]{x} + x^2 + x + 1}{x + 1}$.
\loigiai
{
Ta có:\\
$\dfrac{\sqrt[3]{x} + x^2 + x + 1}{x + 1} = \dfrac{\sqrt[3]{x} + 1}{x + 1} + \dfrac{x^2 + x}{x + 1} = \dfrac{x + 1}{(x + 1)\left(\sqrt[3]{x^2} - \sqrt[3]{x} + 1\right)} + \dfrac{x(x+1)}{x+1}$\\
$= \dfrac{1}{\sqrt[3]{x^2} - \sqrt[3]{x} + 1} + x$.\\
Do đó: $\lim\limits_{x \to -1}\dfrac{\sqrt[3]{x} + x^2 + x + 1}{x + 1} = \dfrac{1}{3} - 1 = -\dfrac{2}{3}$.
}
\end{bt}


\begin{bt}%[Cao Thành Thái]%[1D4K2]
Tính giới hạn $\lim\limits_{x \to 2}\dfrac{\sqrt{x-1}+x^4 - 3x^3 + x^2 + 3}{\sqrt{2x} - 2}$.
\loigiai
{
$\dfrac{\sqrt{x-1}+x^4 - 3x^3 + x^2 + 3}{\sqrt{2x} - 2} = \dfrac{\sqrt{x-1} - 1}{\sqrt{2x} - 2} + \dfrac{x^4-3x^3+x^2 +4}{\sqrt{2x}-2} $\\
$= \dfrac{(x-2)\left(\sqrt{2x}+2\right)}{(2x-4)(\sqrt{x-1}+1)} + \dfrac{(x-2)(x^3-x^2-x-2)\left(\sqrt{2x}+2\right)}{2x-4}$\\
$= \dfrac{\sqrt{2x}+2}{2(\sqrt{x-1}+1)} + \dfrac{(x^3-x^2-x-2)\left(\sqrt{2x}+2\right)}{2}$.\\
Do đó: $\lim\limits_{x \to -1}\dfrac{\sqrt[3]{x} + x^2 + x + 1}{x + 1} = 1 + 0 = 1$.
}
\end{bt}


\begin{bt}%[Cao Thành Thái]%[1D4K2]
Tính giới hạn $\lim\limits_{x \to 0}\dfrac{\sqrt{1+4x} \cdot \sqrt{1 + 6x} - 1}{x}$.
\loigiai
{
Ta có:\\
$\dfrac{\sqrt{1+4x} \cdot \sqrt{1 + 6x} - 1}{x} = \dfrac{\sqrt{1+4x} \cdot \sqrt{1+6x} - \sqrt{1+4x}}{x} + \dfrac{\sqrt{1+4x}-1}{x}$\\
$= \sqrt{1+4x} \cdot \dfrac{\sqrt{1+6x} - 1}{x} + \dfrac{\sqrt{1+4x} - 1}{x}$.\\
Do đó: $\lim\limits_{x \to 0}\dfrac{\sqrt{1+4x} \cdot \sqrt{1 + 6x} - 1}{x} = 1 \cdot \dfrac{6}{2} + \dfrac{4}{2} = 5$.
}
\end{bt}


\begin{bt}%[Cao Thành Thái]%[1D4K2]
Tính giới hạn $\lim\limits_{x \to 0}\dfrac{\sqrt{1+2x} \cdot \sqrt[3]{1+4x} - 1}{x}$.
\loigiai
{
Ta có:\\
$\dfrac{\sqrt{1+2x} \cdot \sqrt[3]{1 + 4x} - 1}{x} = \dfrac{\sqrt{1+2x} \cdot \sqrt[3]{1+4x} - \sqrt{1+2x}}{x} + \dfrac{\sqrt{1+2x}-1}{x}$\\
$= \sqrt{1+2x} \cdot \dfrac{\sqrt[3]{1+4x} - 1}{x} + \dfrac{\sqrt{1+2x} - 1}{x}$.\\
Do đó: $\lim\limits_{x \to 0}\dfrac{\sqrt{1+2x} \cdot \sqrt[3]{1 + 4x} - 1}{x} = 1 \cdot \dfrac{4}{3} + \dfrac{2}{2} = \dfrac{7}{3}$.
}
\end{bt}
\begin{bt}%[Vũ Văn Trường]%[1D4B2]
Cho $I=\lim\limits_{x\to 0}\dfrac{\sqrt{2x + 1} - 1}{x}$ và $J=\lim\limits_{x\to 1}\dfrac{x^2+x-2}{x-1}$. Tính $I+J$.
\loigiai{
Ta có
\begin{align*}
I &=\lim\limits_{x\to 0}\dfrac{\sqrt{2x + 1} - 1}{x}\\
& =\lim\limits_{x\to 0}\dfrac{\left(\sqrt{2x + 1} - 1\right)\left(\sqrt{2x + 1} + 1\right)}{x\left(\sqrt{2x + 1} + 1\right)} \\
& =\lim\limits_{x\to 0}\dfrac{2 x}{x\left(\sqrt{2x + 1} + 1\right)} \\
& =\lim\limits_{x\to 0}\dfrac{2}{\sqrt{2x + 1} + 1} = 1
\end{align*}
$$J =\lim\limits_{x\to 1}\dfrac{x^2 + x - 2}{x - 1} =\lim\limits_{x\to 1}\dfrac{(x - 1)(x + 2)}{x - 1} =\lim\limits_{x\to 1}(x + 2) = 3$$
Vậy $I + J = 4$.
}
\end{bt}
\begin{bt}%[Vũ Văn Trường]%[1D4B2]
Tính giới hạn $\lim \limits_{x \to 0} \dfrac{\sqrt {x + 9}  + \sqrt {x + 16}  - 7}{x}$.
\loigiai{
Ta có
\begin{align*}
\lim \limits_{x \to 0} \dfrac{\sqrt {x + 9}  + \sqrt {x + 16}  - 7}{x}&=\lim \limits_{x \to 0} \dfrac{\left(\sqrt {x + 9}-3\right)+\left(\sqrt {x + 16}  - 4\right)}{x} \\
&=\lim \limits_{x \to 0}\left[\dfrac{\sqrt {x + 9}-3}{x}+\dfrac{\sqrt{x+16}-4}{x}\right]\\
&=\lim \limits_{x \to 0}\left[\dfrac{x}{x\left(\sqrt {x + 9}+3\right)}+\dfrac{x}{x\left(\sqrt{x+16}+4\right)}\right]\\
&=\lim \limits_{x \to 0}\left[\dfrac{1}{\sqrt {x + 9}+3}+\dfrac{1}{\sqrt{x+16}+4}\right]\\
&=\dfrac{1}{6}+\dfrac{1}{8}\\
&=\dfrac{7}{24}.
\end{align*}
}
\end{bt}

\begin{bt}%[Vũ Văn Trường]%[1D4B2]
Tìm giới hạn $\lim \limits_{x \to 7} \dfrac{\sqrt[4]{x+9}-2}{x-7}$.
\loigiai{
Đặt $t=\sqrt[4]{x+9} \Rightarrow x=t^4-9,$ và khi $x\rightarrow 7$ thì $t \rightarrow 2$. Khi đó:
$$\lim \limits_{x \to 7} \dfrac{\sqrt[4]{x+9}-2}{x-7}=\lim \limits_{t \to 2} \dfrac{t-2}{t^4-16}=\lim \limits_{t \to 2} \dfrac{t-2}{(t-2)(t^3+2t^2+4t+8)}=\lim \limits_{t \to 2} \dfrac{1}{t^3+2t^2+4t+8}=\dfrac{1}{32}.$$
}
\end{bt}

\begin{bt}%[Vũ Văn Trường]%[1D4B2]
Tính giới hạn $\lim\limits_{x\to 0}\dfrac{2\sqrt{x+1}-\sqrt[3]{8-x}}{x}$.
\loigiai{
Ta có
\begin{align*}
\lim\limits_{x\to 0}\dfrac{2\sqrt{x+1}-\sqrt[3]{8-x}}{x}&=\lim\limits_{x\to 0}\left[\dfrac{2\sqrt{x+1}-2}{x}+\dfrac{2-\sqrt[3]{8-x}}{x}\right]\\
&=\lim\limits_{x\to 0}\left[\dfrac{2(1+x-1)}{x\left(\sqrt{1+x}+1\right)}+\dfrac{8-(8-x)}{x\left(4+2\sqrt[3]{8-x}+\sqrt[3]{(8-x)^2}\right)}\right]\\
&=\lim\limits_{x\to 0}\left[\dfrac{2}{\sqrt{1+x}+1}+\dfrac{1}{4+2\sqrt[3]{8-x}+\sqrt[3]{(8-x)^2}}\right] \\
&=1+\dfrac{1}{12}=\dfrac{13}{12}.
\end{align*}
}
\end{bt}

\begin{bt}%[Vũ Văn Trường]%[1D4B2]
Tính giới hạn $\lim\limits_{x\to 1}\dfrac{\sqrt[5]{2x-1}-\sqrt[6]{3x-2}}{x-1}$.
\loigiai{
Ta có $\lim\limits_{x\to 1}\dfrac{\sqrt[5]{2x-1}-\sqrt[6]{3x-2}}{x-1}=\lim\limits_{x\to 1}\left[\dfrac{\sqrt[5]{2x-1}-1}{x-1}+\dfrac{1-\sqrt[6]{3x-2}}{x-1}\right]=\dfrac{2}{5}-\dfrac{1}{2}=-\dfrac{1}{10}$.
}
\end{bt}

\begin{bt}%[Vũ Văn Trường]%[1D4K2]
Tính giới hạn $\lim\limits_{x\to 0}\dfrac{\sqrt{1+2x}\sqrt[3]{1+3x}\sqrt[4]{1+4x}-1}{x}$.
\loigiai{
Ta có
\begin{align*}
&\lim\limits_{x\to 0}\dfrac{\sqrt{1+2x}\sqrt[3]{1+3x}\sqrt[4]{1+4x}-1}{x}\\
=&\lim\limits_{x\to 0}\dfrac{\sqrt{1+2x}\sqrt[3]{1+3x}(\sqrt[4]{1+4x}-1)+\sqrt{1+2x}(\sqrt[3]{1+3x}-1)+\sqrt{1+4x}-1}{x}
\\
= &\lim\limits_{x\to 0}\dfrac{\sqrt{1+2x}\sqrt[3]{1+3x}(\sqrt[4]{1+4x}-1)}{x}+ \lim\limits_{x\to 0}\dfrac{\sqrt{1+2x}(\sqrt[3]{1+3x}-1)}{x}+ \lim\limits_{x\to 0}\dfrac{\sqrt{1+2x}-1}{x}\\
=&3.
\end{align*}
}
\end{bt}
\begin{bt}%[Vũ Văn Trường]%[1D4G2]
Tính giới hạn $\lim\limits_{x\to 0}\dfrac{\sqrt{2x +1} -\sqrt[3]{3x+1}}{x^2}$.
\loigiai{
Ta có
\begin{align*}
&\lim\limits_{x\to 0}\dfrac{\sqrt{2x +1} -\sqrt[3]{3x+1}}{x^2}\\
=&\lim\limits_{x\to 0}\left[\dfrac{\sqrt{2x +1}-(1+x)}{x^2} -\dfrac{\sqrt[3]{3x+1}-(1+x)}{x^2}\right]\\
=&\lim\limits_{x\to 0}\left[\dfrac{2x +1-x^2-2x-1}{x^2\left(\sqrt{2x +1}+(1+x)\right)} -\dfrac{{3x+1}-x^3-3x^2-3x-1}{x^2\left(\sqrt[3]{(3x+1)^2}+(1+x)\sqrt[3]{3x+1}+(x+1)^2\right)}\right]\\
=&\lim\limits_{x\to 0}\left[\dfrac{-x^2}{x^2\left(\sqrt{2x +1}+(1+x)\right)} +\dfrac{x^3+3x^2}{x^2\left(\sqrt[3]{(3x+1)^2}+(1+x)\sqrt[3]{3x+1}+(x+1)^2\right)}\right]\\
=&\lim\limits_{x\to 0}\left[\dfrac{-1}{\sqrt{2x +1}+(1+x)} +\dfrac{x+3}{\sqrt[3]{(3x+1)^2}+(1+x)\sqrt[3]{3x+1}+(x+1)^2}\right]\\
=&1-\dfrac{1}{2}=\dfrac{1}{2}.
\end{align*}
}
\end{bt}

\begin{bt}%[Cao Thành Thái]%[1D4K2]
Tính giới hạn $\lim\limits_{x \to 0}\dfrac{\sqrt[m]{1 + \alpha x} \cdot \sqrt[n]{1 + \beta x} - 1}{x}$ với $\alpha \cdot \beta \neq 0$ và $m$, $n$ là các số nguyên dương.
\loigiai
{
Ta có:\\
$\dfrac{\sqrt[m]{1 + \alpha x} \cdot \sqrt[n]{1 + \beta x} - 1}{x} = \dfrac{\sqrt[m]{1 + \alpha x} \cdot \sqrt[n]{1 + \beta x} - \sqrt[m]{1 + \alpha x}}{x} + \dfrac{\sqrt[m]{1 + \alpha x}-1}{x}$\\
$= \sqrt[m]{1 + \alpha x} \cdot \dfrac{\sqrt[n]{1 + \beta x }- 1}{x} + \dfrac{\sqrt[m]{1 + \alpha x} - 1}{x}$.\\
Do đó: $\lim\limits_{x \to 0}\dfrac{\sqrt[m]{1 + \alpha x} \cdot \sqrt[n]{1 + \beta x} - 1}{x} = 1 \cdot \dfrac{\beta}{n} + \dfrac{\alpha}{m} = \dfrac{\alpha}{m} + \dfrac{\beta}{n}$.
}
\end{bt}


\begin{bt}%[Cao Thành Thái]%[1D4G2]
Tính giới hạn $\lim\limits_{x\to a}\dfrac{x^{\alpha}- a^{\alpha}}{x^{\beta}-a^{\beta}}$ với $a \neq 0$ và $\alpha$, $\beta$ là các số nguyên dương.
\loigiai
{
$\lim\limits_{x\to a}\dfrac{x^{\alpha}-a^{\alpha}}{x^{\beta}-a^{\beta}} = \lim\limits_{x\to a}\dfrac{a^{\alpha}\left[\left(\dfrac{x}{a}\right)^{\alpha}-1\right]}{a^{\beta}\left[\left(\dfrac{x}{a}\right)^{\beta}-1\right]}=\lim\limits_{x\to a}\left[a^{\alpha-\beta} \cdot \dfrac{\left(1+\dfrac{x}{a}-1\right)^{\alpha}-1}{\dfrac{x}{a}-1} \cdot \dfrac{\dfrac{x}{a}-1}{\left(1+\dfrac{x}{a}-1\right)^{\beta}-1}\right]$\\
$=a^{\alpha-\beta} \cdot \dfrac{\alpha}{\beta}$.
}
\end{bt}


\begin{bt}%[Cao Thành Thái]%[1D4G2]
Tính giới hạn $\lim\limits_{x \to 1}\dfrac{x + x^2 + \cdots + x^n - n}{x - 1}$ với $n$ là số nguyên dương.
\loigiai
{
Ta có:\\
$\dfrac{x + x^2 + \cdots + x^n - n}{x - 1} = \dfrac{x-1}{x-1} + \dfrac{x^2-1}{x-1} + \cdots + \dfrac{x^n - 1}{x-1}$\\
$= 1 + (x+1) + \cdots + \left(x^{n-1} + x^{n-2} + \cdots + x + 1\right)$.\\
Do đó: $\lim\limits_{x \to 1}\dfrac{x + x^2 + \cdots + x^n - n}{x - 1} = 1 + 2 + \cdots + n = \dfrac{n(n+1)}{2}$.
}
\end{bt}


\begin{bt}%[Cao Thành Thái]%[1D4G2]
Tính giới hạn $\lim\limits_{x \to 1}\dfrac{x^{n+1} - (n+1)x + n}{(x-1)^2}$.
\loigiai
{
Ta có:\\
$\dfrac{x^{n+1} - (n+1)x + n}{(x-1)^2} = \dfrac{x^{n+1} - nx - x + n}{(x-1)^2} = \dfrac{\left(x^{n+1} - x\right) - n(x - 1)}{(x-1)^2}$\\
$= \dfrac{x(x^n - n) - n(x - 1)}{(x-1)^2} = \dfrac{x(x-1)(x^{n-1}+x^{n-2}+ \cdots + 1) - n(x-1)}{(x-1)^2}$\\
$= \dfrac{x^n - x^{n-1} + \cdots + x - n}{x-1} = = 1 + (x+1) + \cdots + \left(x^{n-1} + x^{n-2} + \cdots + x + 1\right)$.\\
Do đó: $\lim\limits_{x \to 1}\dfrac{x^{n+1} - (n+1)x + n}{(x-1)^2} = 1 + 2 + \cdots + n = \dfrac{n(n+1)}{2}$.
}
\end{bt}


\begin{bt}%[Cao Thành Thái]%[1D4G2]
Tính giới hạn $\lim\limits_{x \to a}\dfrac{(x^n - a^n) - na^{n-1}(x-a)}{(x-a)^2}$.
\loigiai
{
Ta có: $\dfrac{(x^n - a^n) - na^{n-1}(x-a)}{(x-a)^2} $\\
$= \dfrac{(x-a)\left(x^{n-1} + ax^{n-2} + a^2x^{n-3} + \cdots + a^{n-2}x + a^{n-1}\right) - na^{n-1}(x-a)}{(x-a)^2}$\\
$= \dfrac{x^{n-1} + ax^{n-2} + a^2x^{n-3} + \cdots + a^{n-2}x + a^{n-1} - na^{n-1}}{x-a}$\\
$= \dfrac{x^{n-1} + ax^{n-2} + a^2x^{n-3} + \cdots + a^{n-2}x - (n-1)a^{n-1}}{x-a}$\\
$= \dfrac{x^{n-1} - a^{n-1} + ax^{n-2} - a^{n-1} + a^2x^{n-3} - a^{n-1} + \cdots + a^{n-2}x - a^{n-1}}{x-a}$\\
$= \dfrac{(x-a)\left(x^{n-2}+ax^{n-3}+ \cdots + a^{n-3}x + a^{n-2}\right)}{x-a} + \dfrac{a(x-a)\left(x^{n-3}+ax^{n-4}+ \cdots + a^{n-4}x + a^{n-3}\right)}{x-a}$\\
$+ \dfrac{a^2(x-a)\left(x^{n-4}+ax^{n-5}+ \cdots + a^{n-5}x + a^{n-4}\right)}{x-a} + \cdots + \dfrac{a^{n-2}(x-a)}{x-a}$\\
$= \left(x^{n-2}+ax^{n-3}+ \cdots + a^{n-3}x + a^{n-2}\right) + a\left(x^{n-3}+ax^{n-4}+ \cdots + a^{n-4}x + a^{n-3}\right)$\\
$+ a^2\left(x^{n-4}+ax^{n-5}+ \cdots + a^{n-5}x + a^{n-4}\right) + \cdots + a^{n-2}$.\\
Do đó: $\lim\limits_{x \to a}\dfrac{(x^n - a^n) - na^{n-1}(x-a)}{(x-a)^2} = (n-1)a^{n-2} + (n-2)a^{n-2} + (n-3)a^{n-2} + \cdots + a^{n-2} = a^{n-2}\left[1+ 2 + \cdots + (n-1)\right] = \dfrac{n(n-1)a^{n-2}}{2}$.
}
\end{bt}


\begin{bt}%[Cao Thành Thái]%[1D4K2]
Tính giới hạn $\lim\limits_{x \to a} \dfrac{\sqrt{x} - \sqrt{a} + \sqrt{x-a}}{\sqrt{x^2-a^2}}$.
\loigiai
{
Ta có:\\
$\dfrac{\sqrt{x} - \sqrt{a} + \sqrt{x-a}}{\sqrt{x^2-a^2}} = \dfrac{\sqrt{x} - \sqrt{a}}{\sqrt{x^2 - a^2}} + \dfrac{\sqrt{x - a}}{\sqrt{x^2 - a^2}} = \dfrac{x - a}{\sqrt{(x-a)(x+a)}} + \dfrac{\sqrt{x-a}}{\sqrt{(x-a)(x+a)}}$\\
$= \dfrac{\sqrt{x - a}}{\sqrt{x + a}} + \dfrac{1}{\sqrt{x + a}}$.\\
Do đó: $\lim\limits_{x \to a} \dfrac{\sqrt{x} - \sqrt{a} + \sqrt{x-a}}{\sqrt{x^2-a^2}} = \dfrac{1}{\sqrt{2a}}$.
}
\end{bt}


\begin{bt}%[Cao Thành Thái]%[1D4G2]
Tính giới hạn $\lim\limits_{x \to 0}\dfrac{\sqrt[m]{1 + \alpha x} - \sqrt[n]{1 + \beta x}}{x}$.
\loigiai
{
$\lim\limits_{x \to 0}\dfrac{\sqrt[m]{1 + \alpha x} - \sqrt[n]{1 + \beta x}}{x} = \lim\limits_{x \to 0}\left(\dfrac{\sqrt[m]{1 + \alpha x} - 1}{x} - \dfrac{\sqrt[n]{1 + \beta x} - 1}{x}\right) = \dfrac{\alpha}{m} - \dfrac{\beta}{n}$.
}
\end{bt}


\begin{bt}%[Cao Thành Thái]%[1D4K2]
Tính giới hạn $\lim\limits_{x \to 0}\dfrac{\sqrt[3]{1 + \dfrac{x}{3}} - \sqrt[4]{1 + \dfrac{x}{4}}}{1 - \sqrt{1 - \dfrac{x}{2}}}$.
\loigiai
{
$\dfrac{\sqrt[3]{1 + \dfrac{x}{3}} - \sqrt[4]{1 + \dfrac{x}{4}}}{1 - \sqrt{1 - \dfrac{x}{2}}} = \dfrac{\sqrt[3]{1 + \dfrac{x}{3}} - 1}{1 - \sqrt{1 - \dfrac{x}{2}}} + \dfrac{1 - \sqrt[4]{1 + \dfrac{x}{4}}}{1 - \sqrt{1 - \dfrac{x}{2}}}$\\
$= \dfrac{\dfrac{x}{3} \cdot \left(1 + \sqrt{1 - \dfrac{x}{2}}\right)}{\dfrac{x}{2} \cdot \left[\sqrt[3]{\left(1 + \dfrac{x}{3}\right)^2} + \sqrt[3]{1 + \dfrac{x}{3}} + 1\right]} - \dfrac{\dfrac{x}{4} \cdot \left(1 + \sqrt{1 - \dfrac{x}{2}}\right)}{\dfrac{x}{2} \cdot \left[\sqrt[4]{\left(1 + \dfrac{x}{4}\right)^3} + \sqrt[4]{\left(1 + \dfrac{x}{4}\right)^2} + \sqrt[4]{1 + \dfrac{x}{4}} + 1\right]}$\\
$= \dfrac{2}{3} \cdot \dfrac{1 + \sqrt{1 - \dfrac{x}{2}}}{\sqrt[3]{\left(1 + \dfrac{x}{3}\right)^2} + \sqrt[3]{1 + \dfrac{x}{3}} + 1} - \dfrac{1}{2} \cdot \dfrac{1 + \sqrt{1 - \dfrac{x}{2}}}{\sqrt[4]{\left(1 + \dfrac{x}{4}\right)^3} + \sqrt[4]{\left(1 + \dfrac{x}{4}\right)^2} + \sqrt[4]{1 + \dfrac{x}{4}} + 1}$\\
Do đó: $\lim\limits_{x \to 0}\dfrac{\sqrt[3]{1 + \dfrac{x}{3}} - \sqrt[4]{1 + \dfrac{x}{4}}}{1 - \sqrt{1 - \dfrac{x}{2}}} = \dfrac{2}{3} \cdot \dfrac{2}{3} - \dfrac{1}{2} \cdot \dfrac{1}{2} = \dfrac{7}{36}$.
}
\end{bt}


\begin{bt}%[Cao Thành Thái]%[1D4G2]
Tính giới hạn $\lim\limits_{x \to 1}\dfrac{(1 - \sqrt{x})(1 - \sqrt[3]{x}) \cdots (1 - \sqrt[n]{x})}{(1-x)^{n-1}}$.
\loigiai
{
Nhận xét: $\dfrac{1 - \sqrt[n]{x}}{1-x} = \dfrac{1-x}{(1-x)\left(1 + \sqrt[n]{x} + \cdots + \sqrt[n]{x^{n-1}}\right)}$.\\
Khi đó: $\dfrac{(1 - \sqrt{x})(1 - \sqrt[3]{x}) \cdots (1 - \sqrt[n]{x})}{(1-x)^{n-1}} = \dfrac{1 - \sqrt{x}}{1-x} \cdot \dfrac{1 - \sqrt[3]{x}}{1-x} \cdots \dfrac{1 - \sqrt[n]{x}}{1-x}$\\
$= \dfrac{1}{1 + \sqrt{x}} \cdot \dfrac{1}{1 + \sqrt[3]{x} + \sqrt[3]{x^2}} \cdots \dfrac{1}{1 + \sqrt[n]{x} + \cdots + \sqrt[n]{x^{n-1}}}$.\\
Do đó: $\lim\limits_{x \to 1}\dfrac{(1 - \sqrt{x})(1 - \sqrt[3]{x}) \cdots (1 - \sqrt[n]{x})}{(1-x)^{n-1}} = \dfrac{1}{2} \cdot \dfrac{1}{3} \cdots \dfrac{1}{n} = \dfrac{1}{n!}$.
}
\end{bt}


\begin{bt}%[Cao Thành Thái]%[1D4G2]
Tính giới hạn $\lim\limits_{x \to 0}\dfrac{(\sqrt{1+x^2}+x)^n - (\sqrt{1+x^2}-x)^n}{x}$.
\loigiai
{
Ta có:\\ 
$\dfrac{(\sqrt{1+x^2}+x)^n - (\sqrt{1+x^2}-x)^n}{x} $\\
$= \dfrac{2x\left[\left(\sqrt{1+x^2}+x\right)^{n-1}+\left(\sqrt{1+x^2}+x\right)^{n-2}\left(\sqrt{1+x^2}-x\right)+\cdots + \left(\sqrt{1+x^2}-x\right)^{n-1}\right]}{x}$\\
$= 2\left[\left(\sqrt{1+x^2}+x\right)^{n-1}+\left(\sqrt{1+x^2}+x\right)^{n-2}\left(\sqrt{1+x^2}-x\right)+\cdots + \left(\sqrt{1+x^2}-x\right)^{n-1}\right]$.\\
Do đó: $\lim\limits_{x \to 0}\dfrac{(\sqrt{1+x^2}+x)^n - (\sqrt{1+x^2}-x)^n}{x} = 2n$.
}
\end{bt}

\begin{dang}{Giới hạn dạng vô định $\infty/\infty$; $\infty -\infty$;$0\cdot\infty$}
\begin{enumerate}[Dạng 1:]
\item $I=\lim\limits_{x\to \infty}\dfrac{P(x)}{Q(x)}$ với $P(x),Q(x)$ là đa thức hoặc các hàm đại số .\\
Phương pháp: Gọi $p=\deg P(x),q=\deg Q(x)$ và $m=\min(p,q)$. Chia cả tử và mẫu cho $x^m$ ta có kết luận. ($\deg P(x)$ là bậc cao nhất của đa thức $P(x)$).
\begin{enumerate}[+]
\item Nếu $p\leq q$ thì tồn tại giới hạn.
\item Nếu $p>q$ thì không tồn tại giới hạn.
\end{enumerate}
\item Giới hạn $\infty -\infty$.\\
Phương pháp sử dụng các biểu thức liên hợp đưa về dạng $\dfrac{\infty}{\infty}$
\item Giới hạn $0.\infty $.\\
Phương pháp sử dụng các biểu thức liên hợp đưa về dạng $\dfrac{\infty }{\infty }$.
\end{enumerate}
\end{dang}

\begin{vd}%[Dũng Lê]%[1H3K4]
Tính $D=\lim\limits_{x\to +\infty}\dfrac{2x^3-3x^2+4x+1}{x^4-5x^3+2x^2-x+3}$
\loigiai{
Ta có $D=\lim\limits_{x\to +\infty}\dfrac{2x^3-3x^2+4x+1}{x^4-5x^3+2x^2-x+3}\\
=\lim\limits_{x\to +\infty}\dfrac{x^4\left(\dfrac{2}{x}-\dfrac{3}{x^2}+\dfrac{4}{x^3}+\dfrac{1}{x^4}\right)}{x^4\left(1-\dfrac{5}{x}+\dfrac{2}{x^2}-\dfrac{1}{x^3}+\dfrac{3}{x^4}\right)}=\lim\limits_{x\to +\infty}\dfrac{\dfrac{2}{x}-\dfrac{3}{x^2}+\dfrac{4}{x^3}+\dfrac{1}{x^4}}{1-\dfrac{5}{x}+\dfrac{2}{x^2}-\dfrac{1}{x^3}+\dfrac{3}{x^4}}=\dfrac{0}{1}=0$
}
\end{vd}

\begin{vd}%[Dũng Lê]%[1D4Y2]
Tính $D=\lim\limits_{x\to -\infty}\dfrac{x+\sqrt{x^2+2}}{\sqrt[3]{8x^3+x^2+1}}$
\loigiai{
Ta có:\\
$D=\lim\limits_{x\to -\infty}\dfrac{x+\sqrt{x^2+2}}{\sqrt[3]{8x^3+x^2+1}}=\lim\limits_{x\to -\infty}\dfrac{x+|x|\sqrt{1+\dfrac{2}{x^2}}}{x\sqrt[3]{8+\dfrac{1}{x}+\dfrac{1}{x^3}}}=\lim\limits_{x\to -\infty}\dfrac{1-\sqrt{1+\dfrac{2}{x^2}}}{\sqrt[3]{8+\dfrac{1}{x}+\dfrac{1}{x^3}}}=\dfrac{0}{\sqrt[3]{8}}=0$.
}
\end{vd}

\begin{vd}%[Dũng Lê]%[1D4B2]
Tìm giới hạn $D=\lim\limits_{x\to +\infty}\left(\sqrt{x+\sqrt{x}}-\sqrt{x}\right)$.
\loigiai{
Ta có\\
$D=\lim\limits_{x\to +\infty}\dfrac{x+\sqrt{x}-x}{\sqrt{x+\sqrt{x}}+\sqrt{x}}=\lim\limits_{x\to +\infty}\dfrac{\sqrt{x}}{\sqrt{x}\left(\sqrt{1+\dfrac{1}{\sqrt{x}}}+1\right)}=\lim\limits_{x\to +\infty}\dfrac{1}{\sqrt{1+\dfrac{1}{\sqrt{x}}}+1}=\dfrac{1}{2}$.
}
\end{vd}

\begin{vd}%[Dũng Lê]%[1D4B2]
Tìm giới hạn $D=\lim\limits_{x\to +\infty}x\left(\sqrt{x^2+1}-x\right)$.
\loigiai{
Ta có:\\
$D=\lim\limits_{x\to +\infty}\dfrac{x(x^2+1-x^2)}{\sqrt{x^2+1}+x}=\lim\limits_{x\to +\infty}\dfrac{x}{x\left(\sqrt{1+\dfrac{1}{x^2}}+1\right)}=\lim\limits_{x\to +\infty}\dfrac{1}{\sqrt{1+\dfrac{1}{x^2}}+1}=\dfrac{1}{2}$.
}
\end{vd}

\begin{vd}%[Dũng Lê]%[1D4K2]
Tìm giới hạn $D=\lim\limits_{x\to \infty}x^2\left(\sqrt{9x^4+7}-\sqrt[3]{27x^6-5}\right)$.
\loigiai{
Ta có\\
$D=\lim\limits_{x\to \infty}\left[x^2\left(\sqrt{9x^4+7}-3x^2\right)+x^2\left(3x^2-\sqrt[3]{27x^6-5}\right)\right]\\
=\lim\limits_{x\to \infty}\left[\dfrac{x^2(9x^4+7-9x^4)}{\sqrt{9x^4+7}+3x^2}+\dfrac{x^2(27x^6+5-27x^6)}{\sqrt[3]{(27x^6-5)^2}+3x^2\sqrt[3]{27x^6-5}+9x^4}\right]\\
=\lim\limits_{x\to \infty}\left[\dfrac{7x^2}{\sqrt{9x^4+7}+3x^2}+\dfrac{5x^2}{\sqrt[3]{(27x^6-5)^2}+3x^2\sqrt[3]{27x^6-5}+9x^4}\right]\\
=\lim\limits_{x\to \infty}\left[\dfrac{7}{\sqrt{9+\dfrac{7}{x^4}}+3}+\dfrac{\dfrac{5}{x^2}}{\sqrt[3]{\left(27-\dfrac{5}{x^6}\right)^2}+3\sqrt[3]{27-\dfrac{5}{x^6}}+9}\right]=\dfrac{7}{6}$.
}
\end{vd}

\begin{center}
\textbf{BÀI TẬP TỰ LUYỆN}
\end{center}
\setcounter{bt}{0}
\begin{bt}%[Dũng Lê]%[1H3K4]
Tính $D=\lim\limits_{x\to -\infty}\dfrac{2x^3-3x^2+4x+1}{x^4-5x^3+2x^2-x+3}$
\loigiai{
Ta có $D=\lim\limits_{x\to -\infty}\dfrac{2x^3-3x^2+4x+1}{x^4-5x^3+2x^2-x+3}\\
=\lim\limits_{x\to -\infty}\dfrac{x^4\left(\dfrac{2}{x}-\dfrac{3}{x^2}+\dfrac{4}{x^3}+\dfrac{1}{x^4}\right)}{x^4\left(1-\dfrac{5}{x}+\dfrac{2}{x^2}-\dfrac{1}{x^3}+\dfrac{3}{x^4}\right)}=\lim\limits_{x\to -\infty}\dfrac{\dfrac{2}{x}-\dfrac{3}{x^2}+\dfrac{4}{x^3}+\dfrac{1}{x^4}}{1-\dfrac{5}{x}+\dfrac{2}{x^2}-\dfrac{1}{x^3}+\dfrac{3}{x^4}}=\dfrac{0}{1}=0$
}
\end{bt}

\begin{bt}%[Dũng Lê]%[1D4Y2]
Tính $D=\lim\limits_{x\to +\infty}\dfrac{x+\sqrt{x^2+2}}{\sqrt[3]{8x^3+x^2+1}}$
\loigiai{
Ta có:\\
$D=\lim\limits_{x\to +\infty}\dfrac{x+\sqrt{x^2+2}}{\sqrt[3]{8x^3+x^2+1}}=\lim\limits_{x\to +\infty}\dfrac{x+|x|\sqrt{1+\dfrac{2}{x^2}}}{x\sqrt[3]{8+\dfrac{1}{x}+\dfrac{1}{x^3}}}=\lim\limits_{x\to +\infty}\dfrac{1+\sqrt{1+\dfrac{2}{x^2}}}{\sqrt[3]{8+\dfrac{1}{x}+\dfrac{1}{x^3}}}=\dfrac{2}{\sqrt[3]{8}}=1$.
}
\end{bt}

\begin{bt}%[Dũng Lê]%[1D4B2]
Tính $D=\lim\limits_{x\to -\infty}\dfrac{-6x^5+7x^3-4x+3}{8x^5-5x^4+2x^2-1}$.
\loigiai{
Ta có\\
$D=\lim\limits_{x\to -\infty}\dfrac{-6+\dfrac{7}{x^2}-\dfrac{4}{x^4}+\dfrac{3}{x^5}}{8-\dfrac{5}{x}+\dfrac{2}{x^3}-\dfrac{1}{x^5}}=\dfrac{-6}{8}=-\dfrac{3}{4}$.
}
\end{bt}

\begin{bt}%[Dũng Lê]%[1D4B2]
Tính $D=\lim\limits_{x\to +\infty}\dfrac{-6x^5+7x^3-4x+3}{8x^5-5x^4+2x^2-1}$.
\loigiai{
Ta có\\
$D=\lim\limits_{x\to +\infty}\dfrac{-6+\dfrac{7}{x^2}-\dfrac{4}{x^4}+\dfrac{3}{x^5}}{8-\dfrac{5}{x}+\dfrac{2}{x^3}-\dfrac{1}{x^5}}=\dfrac{-6}{8}=-\dfrac{3}{4}$.
}
\end{bt}

\begin{bt}%[Dũng Lê]%[1D4K2]
Tính $D=\lim\limits_{x\to +\infty}\dfrac{\sqrt{9x^2+2}-\sqrt[3]{6x^2+5}}{\sqrt[4]{16x^4+3}-\sqrt[5]{8x^4+7}}$.
\loigiai{
Ta có\\
$D=\lim\limits_{x\to +\infty}\dfrac{|x|\sqrt{9+\dfrac{2}{x^2}}-x\sqrt[3]{\dfrac{6}{x}+\dfrac{5}{x^3}}}{|x|\sqrt[4]{16+\dfrac{3}{x^4}}-x\sqrt[5]{\dfrac{8}{x}+\dfrac{7}{x^5}}}=\lim\limits_{x\to +\infty}\dfrac{x\sqrt{9+\dfrac{2}{x^2}}-x\sqrt[3]{\dfrac{6}{x}+\dfrac{5}{x^3}}}{x\sqrt[4]{16+\dfrac{3}{x^4}}-x\sqrt[5]{\dfrac{8}{x}+\dfrac{7}{x^5}}}\\
=\lim\limits_{x\to +\infty}\dfrac{\sqrt{9+\dfrac{2}{x^2}}-\sqrt[3]{\dfrac{6}{x}+\dfrac{5}{x^3}}}{\sqrt[4]{16+\dfrac{3}{x^4}}-\sqrt[5]{\dfrac{8}{x}+\dfrac{7}{x^5}}}=\dfrac{3}{2}$.\\
Suy ra $D=\dfrac{3}{2}$.
}
\end{bt}

\begin{bt}%[Dũng Lê]%[1D4K2]
Tính $D=\lim\limits_{x\to -\infty}\dfrac{\sqrt{9x^2+2}-\sqrt[3]{6x^2+5}}{\sqrt[4]{16x^4+3}-\sqrt[5]{8x^4+7}}$.
\loigiai{
Ta có\\
$D=\lim\limits_{x\to -\infty}\dfrac{|x|\sqrt{9+\dfrac{2}{x^2}}-x\sqrt[3]{\dfrac{6}{x}+\dfrac{5}{x^3}}}{|x|\sqrt[4]{16+\dfrac{3}{x^4}}-x\sqrt[5]{\dfrac{8}{x}+\dfrac{7}{x^5}}}=\lim\limits_{x\to -\infty}\dfrac{-x\sqrt{9+\dfrac{2}{x^2}}-x\sqrt[3]{\dfrac{6}{x}+\dfrac{5}{x^3}}}{-x\sqrt[4]{16+\dfrac{3}{x^4}}-x\sqrt[5]{\dfrac{8}{x}+\dfrac{7}{x^5}}}\\
=\lim\limits_{x\to -\infty}\dfrac{-\sqrt{9+\dfrac{2}{x^2}}-\sqrt[3]{\dfrac{6}{x}+\dfrac{5}{x^3}}}{-\sqrt[4]{16+\dfrac{3}{x^4}}-\sqrt[5]{\dfrac{8}{x}+\dfrac{7}{x^5}}}=\dfrac{3}{2}$.
}
\end{bt}

\begin{bt}%[Dũng Lê]%[1D4B2]
Tính giới hạn $D=\lim\limits_{x\to -\infty}\dfrac{(2x-3)^{20}(3x+2)^{30}}{(2x+1)^{50}}$.
\loigiai{
Ta có\\
$D=\lim\limits_{x\to -\infty}\dfrac{x^{50}\left(2-\dfrac{3}{x}\right)^{20}\left(3+\dfrac{2}{x}\right)^{30}}{x^{50}\left(2+\dfrac{1}{x}\right)^{50}}=\lim\limits_{x\to -\infty}\dfrac{\left(2-\dfrac{3}{x}\right)^{20}\left(3+\dfrac{2}{x}\right)^{30}}{\left(2+\dfrac{1}{x}\right)^{50}}=\left(\dfrac{3}{2}\right)^{30}$.
}
\end{bt}

\begin{bt}%[Dũng Lê]%[1D4K2]
Tính giới hạn $D=\lim\limits_{x\to +\infty}\dfrac{\sqrt{x^2+2x}+3x}{\sqrt{4x^2+1}-x+2}$.
\loigiai{
Ta có\\
$D=\lim\limits_{x\to \infty}\dfrac{|x|\sqrt{1+\dfrac{2}{x}}+3x}{|x|\sqrt{4+\dfrac{1}{x^2}}-x+2}=\lim\limits_{x\to +\infty}\dfrac{x\sqrt{1+\dfrac{2}{x}}+3x}{x\sqrt{4+\dfrac{1}{x^2}}-x+2}=\lim\limits_{x\to +\infty}\dfrac{\sqrt{1+\dfrac{2}{x}}+3}{\sqrt{4+\dfrac{1}{x^2}}-1+\dfrac{2}{x}}=4$.
}
\end{bt}

\begin{bt}%[Dũng Lê]%[1D4K2]
Tính giới hạn $D=\lim\limits_{x\to -\infty}\dfrac{\sqrt{x^2+2x}+3x}{\sqrt{4x^2+1}-x+2}$.
\loigiai{
Ta có\\
$D=\lim\limits_{x\to -\infty}\dfrac{|x|\sqrt{1+\dfrac{2}{x}}+3x}{|x|\sqrt{4+\dfrac{1}{x^2}}-x+2}=\lim\limits_{x\to -\infty}\dfrac{-x\sqrt{1+\dfrac{2}{x}}+3x}{-x\sqrt{4+\dfrac{1}{x^2}}-x+2}=\lim\limits_{x\to -\infty}\dfrac{-\sqrt{1+\dfrac{2}{x}}+3}{-\sqrt{4+\dfrac{1}{x^2}}-1+\dfrac{2}{x}}=-\dfrac{2}{3}$.
}
\end{bt}

\begin{bt}
Tính giới hạn $D=\lim\limits_{x\to +\infty}\left(\sqrt{(x+a)(x+b)}-x\right)$.
\loigiai{
Ta có\\
$D=\lim\limits_{x\to +\infty}\dfrac{(x+a)(x+b)-x^2}{\sqrt{(x+a)(x+b)}+x}=\lim\limits_{x\to +\infty}\dfrac{(a+b)x+ab}{x\sqrt{\left(1+\dfrac{a}{x}\right)\left(1+\dfrac{b}{x}\right)}+x}\\
=\lim\limits_{x\to +\infty}\dfrac{a+b+\dfrac{ab}{x}}{\sqrt{\left(1+\dfrac{a}{x}\right)\left(1+\dfrac{b}{x}\right)}+1}=\dfrac{a+b}{2}$.
}
\end{bt}

\begin{bt}
Tính giới hạn $D=\lim\limits_{x\to +\infty}\left(2x-5-\sqrt{4x^2-4x-1}\right)$.
\loigiai{
Ta có\\
$D=\lim\limits_{x\to +\infty}\dfrac{(2x-5)^2-(4x^2-4x-1)}{2x-5+\sqrt{4x^2-4x-1}}=\dfrac{-16x+26}{2x-5+x\sqrt{4-\dfrac{4}{x}-\dfrac{1}{x^2}}}\\
=\dfrac{-16+\dfrac{26}{x}}{2-\dfrac{5}{x}+\sqrt{4-\dfrac{4}{x}-\dfrac{1}{x^2}}}=-4$.
}
\end{bt}

\begin{bt}
Tính giới hạn $D=\lim\limits_{x\to +\infty}\left(\sqrt[3]{x^3+2}-\sqrt{x^2+1} \right)$.
\loigiai{
Ta có\\
$D=\lim\limits_{x\to +\infty}\left(\sqrt[3]{x^3+2}-x+x-\sqrt{x^2+1}\right)=\lim\limits_{x\to +\infty}\left(\dfrac{x^3+2-x^3}{\sqrt[3]{(x^3+2)^2}+x\sqrt[3]{x^3+2}+x^2}+\dfrac{x^2-(x^2+1)}{x+\sqrt{x^2+1}} \right)\\
=\lim\limits_{x\to +\infty}\left(\dfrac{\dfrac{2}{x^2}}{\sqrt[3]{\left(1+\dfrac{2}{x^3}\right)^2}+\sqrt[3]{1+\dfrac{2}{x^3}}+1}-\dfrac{\dfrac{1}{x}}{1+\sqrt{1+\dfrac{1}{x^2}}} \right)=0$.
}
\end{bt}

\begin{bt}
Tính giói hạn $D=\lim\limits_{x\to +\infty}x^{\frac{3}{2}}\left(\sqrt{x^3+1}-\sqrt{x^3-1}\right)$.
\loigiai{
Ta có:\\
$D=\lim\limits_{x\to +\infty}\dfrac{x^{\frac{3}{2}}\left(x^3+1-(x^3-1)\right)}{\sqrt{x^3+1}+\sqrt{x^3-1}}=\lim\limits_{x\to +\infty}\dfrac{2x^{\frac{3}{2}}}{x^{\frac{3}{2}}\left(\sqrt{1+\dfrac{1}{x^3}}+\sqrt{1-\dfrac{1}{x^3}}\right)}\\
=\lim\limits_{x\to +\infty}\dfrac{2}{\sqrt{1+\dfrac{1}{x^3}}+\sqrt{1-\dfrac{1}{x^3}}}=1$.
}
\end{bt}

\begin{bt}
Tìm giới hạn $D=\lim\limits_{x\to +\infty}x\left( \sqrt{4x^2+5}-\sqrt[3]{8x^3-1}\right)$.
\loigiai{
Ta có:\\
$D=\lim\limits_{x\to +\infty}x\left( \sqrt{4x^2+5}-2x+2x-\sqrt[3]{8x^3-1}\right)\\
=\lim\limits_{x\to +\infty}x\left(\dfrac{4x^2+5-4x^2}{\sqrt{4x^2+5}+2x}+\dfrac{8x^3-(8x^3-1)}{\sqrt[3]{(8x^3-1)^2}+2x\sqrt[3]{8x^3-1}+4x^2} \right)\\
=\lim\limits_{x\to +\infty}\left(\dfrac{5x}{|x|\sqrt{4+\dfrac{5}{x^2}}+2x}+\dfrac{x}{x\sqrt[3]{\left(8-\dfrac{1}{x^3}\right)^2}+2x^2\sqrt[3]{8-\dfrac{1}{x^3}}+4x^2} \right)\\
=\lim\limits_{x\to +\infty}\left(\dfrac{5}{\sqrt{4+\dfrac{5}{x^2}}+2}+\dfrac{\dfrac{1}{x}}{\sqrt[3]{\left(8-\dfrac{1}{x^3}\right)^2}+2\sqrt[3]{8-\dfrac{1}{x^3}}+4}\right)=\dfrac{5}{4}+\dfrac{0}{12}=\dfrac{5}{4}$.
}
\end{bt}
\begin{dang}{Tính giới hạn hàm đa thức, hàm phân thức và giới hạn một bên.}
$\bullet$ Nếu $\lim\limits_{x\to x_0}f(x)=L\ne 0$ và $\lim\limits_{x\to x_0}g(x)=\pm\infty$ thì:
\begin{enumerate}
\item $\lim\limits_{x\to x_0}f(x)\cdot g(x)=\heva{&+\infty\textrm{ nếu }L\textrm{ và }\lim\limits_{x\to x_0}g(x)\textrm{ cùng dấu}\\ &-\infty\textrm{ nếu }L\textrm{ và }\lim\limits_{x\to x_0}g(x)\textrm{ trái dấu}.}$
\item $\lim\limits_{x\to x_0}\dfrac{f(x)}{g(x)}=\heva{&0&\textrm{ nếu }&\lim\limits_{x\to x_0} g(x)=\pm\infty\\ &+\infty&\textrm{ nếu }&\lim\limits_{x\to x_0} g(x)=0\textrm{ và }L\cdot g(x)>0\\& -\infty&\textrm{ nếu }&\lim\limits_{x\to x_0} g(x)=0\textrm{ và }L\cdot g(x)<0.}$
\end{enumerate}
$\bullet$ $\lim\limits_{x\to x_0} f(x)=L\Leftrightarrow \lim\limits_{x\to x_0^-}f(x)=\lim\limits_{x\to x_0^+}f(x)=L$.
\end{dang}
\setcounter{vd}{0}
\begin{vd}%[Lê Đình Mẫn]%[1D4B2]
Tính giới hạn của các hàm số sau:
\begin{multicols}{2}
\begin{enumerate}
\item $I_1=\lim\limits_{x\to \sqrt[3]{2}}\left(x^3-2x^6+1\right)$;
\item $I_2=\lim\limits_{x\to +\infty}\left(2x^5-x^4+4x^3-3\right)$;
\item $I_3=\lim\limits_{x\to -\infty}\left(2x^5-x^4+4x^3-3\right)$;
\item $I_4=\lim\limits_{x\to +\infty}\left(-x^3-x^2+4x+2\right)$;
\item $I_5=\lim\limits_{x\to -\infty}\left(-x^3-x^2+4x+2\right)$;
\item $I_6=\lim\limits_{x\to -\infty}\left(x^6+2x^3-4x^2+4x\right)$.
\end{enumerate}
\end{multicols}
\loigiai{
\begin{enumerate}
\item $I_1=\lim\limits_{x\to \sqrt[3]{2}}\left(x^3-2x^6+1\right)=(\sqrt[3]{2})^3-2(\sqrt[3]{2})^6=2-2\cdot 2^2+1=-5$;
\item $I_2=\lim\limits_{x\to +\infty}\left(2x^5-x^4+4x^3-3\right)=\lim\limits_{x\to +\infty}x^5\left(2-\dfrac{1}{x}+\dfrac{4}{x^2}-\dfrac{3}{x^5}\right)$.\\
Do $\lim\limits_{x\to +\infty}x^5=+\infty$ và $\lim\limits_{x\to +\infty} \left(2-\dfrac{1}{x}+\dfrac{4}{x^2}-\dfrac{3}{x^5}\right)=2>0$ nên
$$I_2=\lim\limits_{x\to +\infty}\left(2x^5-x^4+4x^3-3\right)=+\infty.$$
\item $I_3=\lim\limits_{x\to -\infty}\left(2x^5-x^4+4x^3-3\right)=-\infty$;
\item $I_4=\lim\limits_{x\to +\infty}\left(-x^3-x^2+4x+2\right)=-\infty$;
\item $I_5=\lim\limits_{x\to -\infty}\left(-x^3-x^2+4x+2\right)=+\infty$;
\item $I_6=\lim\limits_{x\to -\infty}\left(x^6+2x^3-4x^2+4x\right)=+\infty$.
\end{enumerate}
}
\end{vd}

\begin{vd}%[Lê Đình Mẫn]%[1D4B2]
Tính giới hạn của các hàm số sau:
\begin{multicols}{2}
\begin{enumerate}
\item $I_1=\lim\limits_{x\to +\infty}\dfrac{3}{x^2-2x+6}$;
\item $I_2=\lim\limits_{x\to 3^+}\dfrac{-x^2+5}{x-3}$;
\item $I_3=\lim\limits_{x\to 3^-}\dfrac{2x^2+\sqrt{3-x}}{x-3}$;
\item $I_4=\lim\limits_{x\to -2^+}\dfrac{|x^2-4|}{x+2}$.
\end{enumerate}
\end{multicols}
\loigiai{
\begin{enumerate}
\item $I_1=\lim\limits_{x\to +\infty}\dfrac{3}{x^2-2x+6}=0$ vì $\lim\limits_{x\to +\infty}(x^2-2x+6)=+\infty$;
\item Ta có $\lim\limits_{x\to 3^+}(-x^2+5)=-4<0,\ \lim\limits_{x\to 3^+}(x-3)=0$ và $x-3>0,\forall x>3$.\\ Do đó $I_2=\lim\limits_{x\to 3^+}\dfrac{-x^2+5}{x-3}=-\infty$.
\item $I_3=\lim\limits_{x\to 3^-}\dfrac{2x^2+\sqrt{3-x}}{x-3}=-\infty$.
\item Ta có $\lim\limits_{x\to -2^+}\dfrac{|x^2-4|}{x+2}=\lim\limits_{x\to -2^+}\dfrac{4-x^2}{x+2}=\lim\limits_{x\to -2^+}(2-x)=4$.
\end{enumerate}
}
\end{vd}

\begin{vd}%[Lê Đình Mẫn]%[1D4B2]
Tính giới hạn một bên của các hàm số sau tại điểm được chỉ ra:
\begin{enumerate}
\item $f(x)=\heva{&\dfrac{x^2-3x+2}{x-1}&\textrm{ khi }&x<1\\&x&\textrm{ khi }&x\ge 1}\textrm{ tại }x=1$;
\item $g(x)=\heva{&\dfrac{\sqrt{x+7}-3}{x-2}&\textrm{ khi }&x>2\\&\dfrac{x-1}{6}&\textrm{ khi }&x\le 2}\textrm{ tại }x=2$.
\end{enumerate}
\loigiai{
\begin{enumerate}
\item Ta có $\lim\limits_{x\to 1^-}f(x)=\lim\limits_{x\to 1^-}\dfrac{x^2-3x+2}{x-1}=\lim\limits_{x\to 1^-}(x-2)=-1$ và $\lim\limits_{x\to 1^+}f(x)=\lim\limits_{x\to 1^+}x=1$.
\item Ta có $\lim\limits_{x\to 2^+}g(x)=\lim\limits_{x\to 2^+}\dfrac{\sqrt{x+7}-3}{x-2}=\lim\limits_{x\to 2^+}\dfrac{1}{\sqrt{x+7}+3}=\dfrac{1}{6}$ và $\lim\limits_{x\to 2^-}g(x)=\lim\limits_{x\to 2^-}\dfrac{x-1}{6}=\dfrac{1}{6}$.\\
Từ đó suy ra $\lim\limits_{x\to 2}g(x)=\dfrac{1}{6}$.
\end{enumerate}
}
\end{vd}
\begin{center}
\textbf{BÀI TẬP TỰ LUYỆN}
\end{center}
\setcounter{bt}{0}
\begin{bt}%[Lê Đình Mẫn]%[1D4B2]
Tính các giới hạn sau:
\begin{multicols}{2}
\begin{enumerate}
\item $I_1=\lim\limits_{x\to +\infty} (-6x^4+2x^3-x+5)$;
\item $I_2=\lim\limits_{x\to +\infty}\left(\sqrt{4x^2-3}+2x\right)$;
\item $I_3=\lim\limits_{x\to -\infty}\left(\sqrt{4x^2-3}-2x\right)$;
\item $I_4=\lim\limits_{x\to -\infty}\left(x+\sqrt[3]{x^3-1}\right)$.
\end{enumerate}
\end{multicols}
\loigiai{
\begin{multicols}{2}
\begin{enumerate}
\item $I_1=\lim\limits_{x\to +\infty}x^4\left(-6+\dfrac{2}{x}-\dfrac{1}{x^3}+\dfrac{5}{x^4}\right)$\\
$=-\infty$.
\item $I_2=\lim\limits_{x\to +\infty}x\left(\sqrt{4-\dfrac{3}{x^2}}+2\right)=+\infty$.
\item $I_3=\lim\limits_{x\to -\infty}x\left(-\sqrt{4-\dfrac{3}{x^2}}-2\right)=-\infty$.
\item $I_4=\lim\limits_{x\to -\infty}x\left(1+\sqrt[3]{1-\dfrac{1}{x^3}}\right)=-\infty$.
\end{enumerate}
\end{multicols}
}
\end{bt}

\begin{bt}%[Lê Đình Mẫn]%[1D4B2]
Tính các giới hạn sau:
\begin{multicols}{2}
\begin{enumerate}
\item $I_1=\lim\limits_{x\to -1^-} \dfrac{\sqrt{-4-4x}+3x^2}{x+1}$;
\item $I_2=\lim\limits_{x\to 2^-}\dfrac{3x+1}{2-x}$;
\item $I_3=\lim\limits_{x\to 2^+}\dfrac{2x^2-5x+2}{(x-2)^2}$;
\item $I_4=\lim\limits_{x\to -3^+}\dfrac{\sqrt{x+7}-2}{|x^2-9|}$.
\end{enumerate}
\end{multicols}
\loigiai{
\begin{enumerate}
\item Ta có $\lim\limits_{x\to -1^-}\left(\sqrt{-4-4x}+3x^2\right)=3>0, \lim\limits_{x\to -1^-}(x+1)=$ và $x+1<0,\ \forall x<-1$.\\
Do đó $I_1=-\infty$.
\item $I_2=\lim\limits_{x\to 2^-}\dfrac{3x+1}{2-x}=+\infty$.
\item $I_3=\lim\limits_{x\to 2^+}\dfrac{2x-1}{x-2}=+\infty$.
\item $I_4=\lim\limits_{x\to -3^+}\dfrac{x+3}{(9-x^2)(\sqrt{x+7}+2)}=\lim\limits_{x\to -3^+}\dfrac{1}{(3-x)(\sqrt{x+7}+2)}=\dfrac{1}{24}$.
\end{enumerate}}
\end{bt}

\begin{bt}%[Lê Đình Mẫn]%[1D4K2]
Tính giới hạn $\lim\limits_{x\to 3} \dfrac{x^2-4x+3}{(x-3)^2}$.
\loigiai{
Xét các giới hạn một bên:\\
$\lim\limits_{x\to 3^-} \dfrac{x^2-4x+3}{(x-3)^2}=\lim\limits_{x\to 3^-} \dfrac{x-1}{x-3}=-\infty$ và $\lim\limits_{x\to 3^+} \dfrac{x^2-4x+3}{(x-3)^2}=\lim\limits_{x\to 3^+} \dfrac{x-1}{x-3}=+\infty$.\\
Từ đó suy ra $\lim\limits_{x\to 3} \dfrac{x^2-4x+3}{(x-3)^2}$ không tồn tại.}
\end{bt}

\begin{bt}%[Lê Đình Mẫn]%[1D4K2]
Cho hàm số $f(x)=\heva{&\dfrac{2-\sqrt{x+3}}{x^2-1}&\textrm{ khi }&x>1\\&m-2x&\textrm{ khi }&x\le 1}$. Xác định các giá trị của tham số $m$ để $f(x)$ có giới hạn tại điểm $x=1$.
\loigiai{
Ta có $\lim\limits_{x\to 1^+}f(x)=\lim\limits_{x\to 1^+}\dfrac{2-\sqrt{x+3}}{x^2-1}=\lim\limits_{x\to 1^+}\dfrac{-1}{x+1}=-\dfrac{1}{2}$. Để tồn tại $\lim\limits_{x\to 1}f(x)$ thì điều kiện cần và đủ là $\lim\limits_{x\to 1^-}f(x)=-\dfrac{1}{2}\Leftrightarrow m-2=-\dfrac{1}{2}\Leftrightarrow m=\dfrac{3}{2}$.}
\end{bt}
\begin{center}
\textbf{BÀI TẬP TỔNG HỢP}
\end{center}
\begin{bt}%[Lê Đình Mẫn]%[1D4K2]
Tính các giới hạn sau:
\begin{multicols}{2}
\begin{enumerate}
\item $I_1=\lim\limits_{x\to +\infty} (4x^3-\sqrt{x^2+2})$;
\item $I_2=\lim\limits_{x\to -\infty} \dfrac{2x-\sqrt[3]{2x^6+x^4-1}}{x^2+\sqrt{x}}$;
\item $I_3=\lim\limits_{x\to +\infty} \dfrac{\sqrt[3]{2x^6+x^4-1}}{1-x^2}$;
\item $I_4=\lim\limits_{x\to +\infty} \dfrac{\sqrt{16x^8+3}-x^2}{x(x+2)(x+4)(x+6)}$.
\end{enumerate}
\end{multicols}
\loigiai{
\begin{multicols}{4}
\begin{enumerate}
\item $I_1=+\infty$.
\item $I_2=-\sqrt[3]{2}$.
\item $I_3=\sqrt[3]{2}$.
\item $I_4=4$.
\end{enumerate}
\end{multicols}
}
\end{bt}

\begin{bt}%[Lê Đình Mẫn]%[1D4K2]
Tính các giới hạn sau:
\begin{multicols}{2}
\begin{enumerate}
\item $I_1=\lim\limits_{x\to -4^+} \dfrac{x^3-16x}{|x+4|}$;
\item $I_2=\lim\limits_{x\to -4^-} \dfrac{\sqrt{x^2-16}}{|x+4|}$.
\end{enumerate}
\end{multicols}
\loigiai{
\begin{enumerate}
\item $I_1=\lim\limits_{x\to -4^+} \dfrac{x(x^2-16)}{x+4}=\lim\limits_{x\to -4^+} x(x-4)=32$.
\item $I_2=\lim\limits_{x\to -4^-} \dfrac{\sqrt{x^2-16}}{-(x+4)}=\lim\limits_{x\to -4^-} \dfrac{\sqrt{4-x}}{\sqrt{-x-4}}=+\infty$.
\end{enumerate}
}
\end{bt}

\begin{bt}%[Lê Đình Mẫn]%[1D4K2]
Cho hàm số $f(x)=\heva{&\dfrac{ax^2+3ax-4a}{x-1}&\textrm{ khi }&x<1\\&2bx+1&\textrm{ khi }&x\ge 1}$. Biết rằng $a,b$ là các số thực thỏa mãn hàm số $f(x)$ có giới hạn tại $x=1$.
\begin{enumerate}
\item Tìm mối quan hệ giữa $a$ và $b$.
\item Tìm giá trị nhỏ nhất của biểu thức $P=a^2+b^2$.
\end{enumerate}
\loigiai{
\begin{enumerate}
\item Ta có $\lim\limits_{x\to 1^-}f(x)=\lim\limits_{x\to 1^-}a(x-4)=-3a$, $\lim\limits_{x\to 1^+}f(x)=2b+1$.\\
Hàm số $f(x)$ có giới hạn tại $x=1$ khi và chỉ khi $-3a=2b+1$.
\item Từ câu a) ta có $1=(3a+2b)^2\le (9+4)(a^2+b^2)\Rightarrow P=a^2+b^2\ge \dfrac{1}{13}$. Đẳng thức có được khi và chỉ khi $a=-\dfrac{3}{13}$ và $b=-\dfrac{2}{13}$. Vậy $\min P=\dfrac{1}{13}$.
\end{enumerate}
}
\end{bt}

\begin{bt}%[Lê Đình Mẫn]%[1D4K2]
Tính các giới hạn sau:
\begin{multicols}{2}
\begin{enumerate}
\item $I_1=\lim\limits_{x\to 1} \dfrac{2x^5+x^4-4x^2+1}{x^3-1}$;
\item $I_2=\lim\limits_{x\to -2} \dfrac{2x^4+9x^3+11x^2-4}{(x+2)^2}$;
\item $I_3=\lim\limits_{x\to -1} \dfrac{x^{11}+1}{x^7+1}$;
\item $I_4=\lim\limits_{x\to 1} \dfrac{x+x^2+\cdots+x^{2018}-2018}{x^2-1}$.
\end{enumerate}
\end{multicols}
\loigiai{
\begin{enumerate}
\item Ta có $I_1=\lim\limits_{x\to 1} \dfrac{(x-1)(2x^4+3x^3+3x^2-x-1)}{(x-1)(x^2+x+1)}=\lim\limits_{x\to 1} \dfrac{2x^4+3x^3+3x^2-x-1}{x^2+x+1}=2$.
\item Ta có $2x^4+9x^3+11x^2-4=(x+2)^2(2x^2+x-1)$, suy ra $I_2=\lim\limits_{x\to -2}(2x^2+x-1)=5$.
\item $I_3=\lim\limits_{x\to -1} \dfrac{(x+1)(x^{10}-x^9+x^8-\cdots-x+1)}{(x+1)(x^6-x^5+\cdots-x+1)}=\lim\limits_{x\to -1}\dfrac{x^{10}-x^9+x^8-\cdots-x+1}{x^6-x^5+\cdots-x+1}=\dfrac{11}{7}$.
\item Ta có\\
$\begin{aligned}
x+x^2+\cdots+x^{2018}-2018&=(x-1)+(x^2-1)+\cdots+(x^{2018}-1)\\
&=(x-1)\left[1+(1+x)+\cdots+(1+x+x^2+\cdots+x^{2017})\right].
\end{aligned}$\\
Do đó\\
$\begin{aligned}I_4&=\lim\limits_{x\to 1}\dfrac{1+(1+x)+\cdots+(1+x+x^2+\cdots+x^{2017})}{x+1}\\
&=\dfrac{1+2+\cdots+2018}{2}=\dfrac{2037171}{2}.\end{aligned}$
\end{enumerate}
}
\end{bt}

\begin{bt}%[Lê Đình Mẫn]%[1D4G2]
Tìm các giá trị của $a,b$ sao cho $\lim\limits_{x\to +\infty}(\sqrt{x^2+x+1}-ax-b)=0$.
\loigiai{
Nếu $a\le 0$ thì $\lim\limits_{x\to +\infty}(\sqrt{x^2+x+1}-ax-b)=+\infty$. Do đó, ta chỉ xét với $a>0$. Khi đó, ta có\\
$\lim\limits_{x\to +\infty}(\sqrt{x^2+x+1}-ax-b)=\lim\limits_{x\to +\infty}\dfrac{(1-a^2)x^2+(1-2ab)x+1-b^2}{\sqrt{x^2+x+1}+ax+b}$.\\
Suy ra $1-a^2=0\Leftrightarrow a=\pm 1$.\\
$\bullet$ Với $a=1$ thì $\lim\limits_{x\to +\infty}(\sqrt{x^2+x+1}-ax-b)=\lim\limits_{x\to +\infty}\dfrac{1-2b+\dfrac{1-b^2}{x}}{\sqrt{1+\dfrac{1}{x}+\dfrac{1}{x^2}}+1+\dfrac{b}{x}}=0$ khi $b=\dfrac{1}{2}$.\\
$\bullet$ Với $a=-1$ tương tự ta tìm được $b=-\dfrac{1}{2}$.
}
\end{bt}

\begin{bt}%[Lê Đình Mẫn]%[1D4G2]
Tính các giới hạn sau:
\begin{multicols}{2}
\begin{enumerate}
\item $I_1=\lim\limits_{x\to -2} \dfrac{x-1+\sqrt{5-2x}}{x^2+x-2}$;
\item $I_2=\lim\limits_{x\to 1} \dfrac{2\sqrt{2-x}-\sqrt[3]{9-x}}{1-x}$;
\item $I_3=\lim\limits_{x\to -1} \dfrac{\sqrt[3]{7+6x}-\sqrt{5+4x}}{(x+1)^2}$;
\item $I_4=\lim\limits_{x\to 0} \dfrac{\sqrt{1+2017x}\cdot\sqrt[3]{1+2018x}-1}{x}$.
\end{enumerate}
\end{multicols}
\loigiai{
\begin{enumerate}
\item Ta có $I_1=\lim\limits_{x\to -2}\dfrac{x^2-4}{(x+2)(x-1)(x-1-\sqrt{5-2x})}=\lim\limits_{x\to -2}\dfrac{x-2}{(x-1)(x-1-\sqrt{5-2x})}=-\dfrac{2}{9}$.
\item Ta có\\
$\begin{aligned}I_2&=\lim\limits_{x\to 1} \dfrac{2(\sqrt{2-x}-1)+(2-\sqrt[3]{9-x})}{1-x}\\
&=\lim\limits_{x\to 1} \dfrac{2}{\sqrt{2-x}+1}+\lim\limits_{x\to 1} \dfrac{-1}{4+2\sqrt[3]{9-x}+\sqrt[3]{(9-x)^2}}=1-\dfrac{1}{12}=\dfrac{11}{12}.\end{aligned}$
\item Ta có\\
$\begin{aligned}
I_3&=\lim\limits_{x\to -1}\dfrac{\sqrt[3]{7+6x}-(2x+3)+[(2x+3)-\sqrt{5+4x}]}{(x+1)^2}\\
&=\lim\limits_{x\to -1}\dfrac{\sqrt[3]{7+6x}-(2x+3)}{(x+1)^2}
+\lim\limits_{x\to -1}\dfrac{(2x+3)-\sqrt{5+4x}}{(x+1)^2}=-4+2=-2.\end{aligned}$
\item Ta có\\
$\begin{aligned}I_4&=\lim\limits_{x\to 0} \dfrac{\sqrt{1+2017x}(\sqrt[3]{1+2018x}-1)+\sqrt{1+2017x}-1}{x}\\
&=\lim\limits_{x\to 0}\dfrac{2018\sqrt{1+2017x}}{\sqrt[3]{(1+2018x)^2}+\sqrt[3]{1+2018x}+1}+\lim\limits_{x\to 0}\dfrac{2017}{\sqrt{1+2017x}+1}=\dfrac{10087}{6}.\end{aligned}$
\end{enumerate}
}
\end{bt}

\begin{bt}%[Lê Đình Mẫn]%[1D4G2]
Tính các giới hạn sau:
\begin{multicols}{2}
\begin{enumerate}
\item $I_1=\lim\limits_{x\to +\infty} (\sqrt{x^2+2x-1}-x-1)$;
\item $I_2=\lim\limits_{x\to -\infty} (\sqrt{x^2-2x-1}+x-1)$;
\item $I_3=\lim\limits_{x\to +\infty} (\sqrt{4x^2-x}-\sqrt[3]{8x^3+3x^2})$;
\item $I_4=\lim\limits_{x\to 1} \left(\dfrac{2017}{1-x^{2017}}-\dfrac{2018}{1-x^{2018}}\right)$.

\end{enumerate}
\end{multicols}
\loigiai{
\begin{enumerate}
\item Ta có $I_1=\lim\limits_{x\to +\infty} \dfrac{-2}{\sqrt{x^2+2x-1}+x+1}=0$.
\item $I_2=\lim\limits_{x\to -\infty}\dfrac{-2}{\sqrt{x^2-2x-1}-x+1}=0$.
\item $I_3=\lim\limits_{x\to +\infty} (\sqrt{4x^2-x}-2x)+\lim\limits_{x\to +\infty} (2x-\sqrt[3]{8x^3+3x^2})=-\dfrac{1}{4}-\dfrac{1}{4}=-\dfrac{1}{2}$.
\item Ta có\\
$\begin{aligned}\lim\limits_{x\to 1}\left(\dfrac{2017}{1-x^{2017}}-\dfrac{1}{1-x}\right)&=\lim\limits_{x\to 1}\dfrac{(1-x^{2016})+(1-x^{2015})+\cdots+(1-x)}{1-x^{2017}}\\
&=\dfrac{2016+2015+\cdots+1}{2017}=1008\end{aligned}$\\
và\\ $\begin{aligned}\lim\limits_{x\to 1}\left(\dfrac{2018}{1-x^{2018}}-\dfrac{1}{1-x}\right)&=\lim\limits_{x\to 1}\dfrac{(1-x^{2017})+(1-x^{2016}+\cdots+(1-x))}{1-x^{2018}}\\
&=\dfrac{2017+2016+\cdots+1}{2018}=\dfrac{2017}{2}.\end{aligned}$\\
Vậy $I_4=1008-\dfrac{2017}{2}=-\dfrac{1}{2}$.
\end{enumerate}
}
\end{bt}