\chapter{Giới hạn. Hàm số liên tục}
\section{Giới hạn dãy số}
\subsection{Tóm tắt lý thuyết}
\begin{tomtat}
	\subsubsection{Dãy số có giới hạn $0$}
	\begin{dn}
		Ta nói dãy số $\left(u_{n}\right)$ có giới hạn là $0$ khi $n$ dần tới dương vô cực, nếu $\left|u_{n}\right|$ có thể nhỏ hơn một số dương bé tuỳ ý, kể từ một số hạng nào đó trở đi, kí hiệu $\lim \limits_{n \rightarrow+\infty} u_{n}=0$ hay $u_{n} \rightarrow 0$ khi $n \rightarrow+\infty$.
	\end{dn}
Từ định nghĩa dãy số có giới hạn $0$, ta có các kết quả sau:
\begin{itemize}
	\item $\lim \limits_{n \rightarrow+\infty} \dfrac{1}{n^{k}}=0$ với $k$ là một số nguyên dương;	
	\item $\lim \limits_{n \rightarrow+\infty} q^{n}=0$ nếu $|q|<1$;	
	\item Nếu $\left|u_{n}\right| \leq v_{n}$ với mọi $n \geq 1$ và $\lim \limits_{n \rightarrow+\infty} v_{n}=0$ thì $\lim \limits_{n \rightarrow+\infty} u_{n}=0$.	
\end{itemize}
	\subsubsection{Dãy số có giới hạn hữu hạn}
	\begin{dn}
		Ta nói dãy số $\left(u_{n}\right)$ có giới hạn là số thực a khi $n$ dần tới dương vô cực nếu $$\lim \limits_{n \rightarrow+\infty}\left(u_{n}-a\right)=0,$$ kí hiệu $\lim \limits_{n \rightarrow+\infty} u_{n}=a$ hay $u_{n} \rightarrow a$ khi $n \rightarrow+\infty$. 
	\end{dn}
\begin{itemize}
	\item Nếu $u_{n}=c$ (c là hằng số) thì $\lim \limits_{n \rightarrow+\infty} u_{n}=c$. 
	\item $\lim \limits_{n \rightarrow+\infty} u_{n}=a$ khi và chỉ khi $\lim \limits_{n \rightarrow+\infty}\left(u_{n}-a\right)=0$.
\end{itemize}
\subsubsection{Các quy tắc tính giới hạn}
\begin{tc} 
	\begin{enumEX}[a)]{1}
		\item Nếu $\lim \limits_{n \rightarrow+\infty} u_{n}=a$ và $\lim \limits_{n \rightarrow+\infty} v_{n}=b$ thì 
		\begin{enumEX}[-)]{2}
			\item 	$\lim \limits_{n \rightarrow+\infty}\left(u_{n}+v_{n}\right)=a+b$.
			\item   $\lim \limits_{n \rightarrow+\infty}\left(u_{n}-v_{n}\right)=a-b$. 
			\item $\lim \limits_{n \rightarrow+\infty}\left(u_{n} \cdot v_{n}\right)=a \cdot b$.
			\item $\lim \limits_{n \rightarrow+\infty} \dfrac{u_{n}}{v_{n}}=\dfrac{a}{b}$ (nếu $b \neq 0$).
		\end{enumEX}
		\item Nếu $u_{n} \geq 0$ với mọi $n$ và $\lim \limits_{n \rightarrow+\infty} u_{n}=a$ thì $
		a \geq 0 \text { và } \lim \limits_{n \rightarrow+\infty} \sqrt{u_{n}}=\sqrt{a}$.
	\end{enumEX}
	
\end{tc}
\end{tomtat}
\subsection{Các dạng toán thường gặp}
\begin{dang}{Phương pháp đặt thừa số chung (lim hữu hạn)}
	
\end{dang}
\subsubsection{Ví dụ mẫu}
\begin{vd}%[1C3Y1-2]%[Anh Duy]%Ví dụ 1.
	Tìm giới hạn sau $\lim\dfrac{2n^3-2n+3}{1-4n^3}$.
	\loigiai{
		\[\lim\dfrac{2n^3-2n+3}{1-4n^3}=\lim\dfrac{2-\dfrac{2}{n^2}+\dfrac{3}{n^3}}{\dfrac{1}{n^3}-4}=-\dfrac{1}{2}.\]}
\end{vd}
\begin{vd}%[1C3Y1-2]%[Anh Duy]%Ví dụ 2.
	Tìm giới hạn sau $\lim\dfrac{\sqrt{n^4+2n+2}}{n^2+1}$.
	\loigiai{
		\[\lim\dfrac{\sqrt{n^4+2n+2}}{n^2+1} = \lim\dfrac{\sqrt{1+\dfrac{2}{n^3}+\dfrac{2}{n^4}}}{1+\dfrac{1}{n^2}}=1.\]}
\end{vd}
\begin{vd}%[1C3Y1-2]%[Anh Duy]%Ví dụ 3.
	Tìm giới hạn sau $\lim\dfrac{3^{n+1}-4^n}{4^{n-1}+3}$.
	\loigiai{
		\[\lim\dfrac{3^{n+1}-4^n}{4^{n-1}+3} = \lim\dfrac{9\cdot 3^{n-1}-4\cdot 4^{n-1}}{4^{n-1}+3}=\lim\dfrac{9\cdot\left(\dfrac{3}{4}\right)^{n-1}-4}{1+3\cdot\left(\dfrac{1}{4}\right)^{n-1}}=-4.\]}
\end{vd}

\begin{vd}%[1C3B1-2]%[Anh Duy]%Ví dụ 4.
	Tìm giới hạn sau $\lim\dfrac{1+2+2^2+\cdots +2^n}{1+3+3^2+\cdots +3^n}$.
	\loigiai{
		\[\lim\dfrac{1+2+2^2+\cdots +2^n}{1+3+3^2+\cdots +3^n} = \lim\dfrac{\dfrac{1-2^{n+1}}{-1}}{\dfrac{1-3^{n+1}}{-2}} = \lim\dfrac{\left(1-2^{n+1}\right)\cdot 2}{1-3^{n+1}} = \lim\dfrac{\left(\left(\dfrac{1}{3}\right)^{n+1}-\left(\dfrac{2}{3}\right)^{n+1}\right)\cdot 2}{\left(\dfrac{1}{3}\right)^{n+1}-1}=0.\]}
\end{vd}
\subsubsection{Bài tập rèn luyện}
\centerline{\fcolorbox{red}{yellow!50}{\bf {BÀI TẬP TỰ LUẬN }}}
\begin{bt}%[1C3B1-2]%[Anh Duy]
	Tìm các giới hạn sau
	\begin{enumEX}[a)]{2}
		\item[a)] $\lim \limits_{n \rightarrow+\infty} \dfrac{n^{2}+n+1}{2 n^{2}+1}$.
		\item[b)] $\lim \limits_{n \rightarrow+\infty}\left(\sqrt{n^{2}+2 n}-n\right)$.
	\end{enumEX}
	\loigiai{
		\begin{enumEX}[a)]{1}
			\item $\lim \limits_{n \rightarrow+\infty} \dfrac{n^{2}+n+1}{2 n^{2}+1}=\lim \limits_{n \rightarrow+\infty} \dfrac{1+\dfrac{1}{n}+\dfrac{1}{n^2}}{2+\dfrac{1}{n^2}}=\dfrac{\lim \limits_{n \rightarrow+\infty} \left(1+\dfrac{1}{n}+\dfrac{1}{n^2}\right)}{\lim \limits_{n \rightarrow+\infty} \left(2+\dfrac{1}{n^2}\right)}=\dfrac{1}{2}$.
			\item $\lim \limits_{n \rightarrow+\infty}\left(\sqrt{n^{2}+2 n}-n\right)=\lim \limits_{n \rightarrow+\infty}\dfrac{n^2+2n-n^2}{\sqrt{n^{2}+2 n}+n}=\lim \limits_{n \rightarrow+\infty} \dfrac{2}{\sqrt{1+\dfrac{2}{n}}+1}=\dfrac{2}{\lim \limits_{n \rightarrow+\infty} \left(\sqrt{1+\dfrac{2}{n}}+1\right)}=1$.
		\end{enumEX}
	}
\end{bt}

\begin{bt}%[1C3B1-2]%[Anh Duy]
	Cho hai dãy số không âm $\left(u_{n}\right)$ và $\left(v_{n}\right)$ với $\lim \limits_{n \rightarrow+\infty} u_{n}=2$ và $\lim \limits_{n \rightarrow+\infty} v_{n}=3$. Tìm các giới hạn sau
	\begin{enumEX}[a)]{2}
		\item[a)] $\lim \limits_{n \rightarrow+\infty} \dfrac{u_{n}^{2}}{v_{n}-u_{n}}$;
		\item[b)] $\lim \limits_{n \rightarrow+\infty} \sqrt{u_{n}+2 v_{n}}$.
	\end{enumEX}
	\loigiai{
		\begin{enumEX}[a)]{1}
			\item $\lim \limits_{n \rightarrow+\infty} \dfrac{u_{n}^{2}}{v_{n}-u_{n}} = \dfrac{\lim \limits_{n \rightarrow+\infty} u_{n}^{2}}{\lim \limits_{n \rightarrow+\infty} v_{n}-\lim \limits_{n \rightarrow+\infty} u_{n}} = \dfrac{\left(\lim \limits_{n \rightarrow+\infty} u_{n}\right)^{2}}{\lim \limits_{n \rightarrow+\infty} v_{n}-\lim \limits_{n \rightarrow+\infty} u_{n}}=\dfrac{2^2}{3-2}=4$ ;
			\item $\lim \limits_{n \rightarrow+\infty} \sqrt{u_{n}+2 v_{n}} =  \sqrt{\lim \limits_{n \rightarrow+\infty}  u_{n}+\lim \limits_{n \rightarrow+\infty} 2 v_{n}} =\sqrt{\lim \limits_{n \rightarrow+\infty}  u_{n}+2\lim \limits_{n \rightarrow+\infty}  v_{n}} =\sqrt{2+2\cdot 3}=2\sqrt{2}$.
		\end{enumEX}
	}
\end{bt}
\begin{bt}%[1C3B1-2]%[Anh Duy]
	Tính các giới hạn sau:
	\begin{enumEX}{2}
	\item $\lim \limits{n \to +\infty}\dfrac{2^n+3\cdot 4^n}{4^n-5\cdot 3^n}$.
	\item $T=\lim\dfrac{3\cdot7^n+2\cdot 4^n}{4\cdot 5^n+7^n}$.
	\end{enumEX}
\loigiai{
\begin{enumEX}{1}
	\item $\lim \limits{n \to +\infty}\dfrac{2^n+3\cdot 4^n}{4^n-5\cdot 3^n}=\lim \limits{n \to +\infty}\dfrac{\left(\dfrac {1} {2}\right)^n+3}{1-5\left(\dfrac {3} {4}\right)^n}=3$.
	\item Ta có $T=\lim\dfrac{3\cdot7^n+2\cdot 4^n}{4\cdot 5^n+7^n}=\lim\dfrac{3+2\cdot\left(\dfrac{4}{7}\right)^n}{4\cdot\left(\dfrac{5}{7}\right)^n+1}=3$.
\end{enumEX}
}
\end{bt}
\centerline{\fcolorbox{red}{yellow!50}{\bf {CÂU HỎI TRẮC NGHIỆM}}}
\Opensolutionfile{ans}[ans/ans-1K5-15-Dang1]
\begin{ex}%[1C3Y1-2]%[Anh Duy]%Câu 1.
	Tính giới hạn $I=\lim\dfrac{2n+2023}{3n+2024}$. 
	\choice
	{\True $I=\dfrac{2}{3}$}
	{$I=\dfrac{3}{2}$}
	{$I=\dfrac{2023}{2024}$}
	{$I=1$}
	\loigiai{
		Ta có $I=\lim\dfrac{2n+2023}{3n+2024} =\lim\dfrac{2+\dfrac{2017}{n}}{3+\dfrac{2018}{n}} =\dfrac{2}{3}$.}
\end{ex}
\begin{ex}%[1C3Y1-1]%[Anh Duy]%Câu 2.
	Phát biểu nào sau đây là \textbf{sai}?
	\choice
	{$\lim \limits{n \to +\infty}u_n=c$ ($u_n=c$ là hằng số)}
	{\True $\lim \limits{n \to +\infty}q^n=0 \;(|q|>1)$}
	{$\lim\dfrac{1}{n}=0$}
	{$\lim\dfrac{1}{n^k}=0 \; (k>1)$}
	\loigiai{
		Theo định nghĩa giới hạn hữu hạn của dãy số thì $\lim \limits{n \to +\infty}q^n=0 \; (|q|<1)$.}
\end{ex}
\begin{ex}%[1C3Y1-2]%[Anh Duy]%Câu 3.
	Giá trị của $\lim\dfrac{2-n}{n+1}$ bằng
	\choice
	{$1$}
	{$2$}
	{\True $-1$}
	{$0$}
	\loigiai{
		Ta có $\lim\dfrac{2-n}{n+1} =\lim\dfrac{\dfrac{2}{n}-1}{1+\dfrac{1}{n}} =\dfrac{0-1}{1+0} =-1$.}
\end{ex}
\begin{ex}%[1C3Y1-2]%[Anh Duy]%Câu 4.
	Tính giới hạn $\lim\dfrac{4n+2024}{2n+1}$. 
	\choice
	{$\dfrac{1}{2}$}
	{$4$}
	{\True $2$}
	{$2024$}
	\loigiai{
		Ta có $\lim\dfrac{4n+2024}{2n+1}=\lim\dfrac{4+\dfrac{2024}{n}}{2+\dfrac{1}{n}}=2$.}
\end{ex}
\begin{ex}%[1C3Y1-2]%[Anh Duy]%Câu 5.
	$\lim\dfrac{2n^2-3}{n^6+5n^5}$ bằng 
	\choice
	{$2$}
	{\True $0$}
	{$\dfrac{-3}{5}$}
	{$-3$}
	\loigiai{
		Ta có $\lim\dfrac{2n^2-3}{n^6+5n^5} =\lim\dfrac{\dfrac{2}{n^4}-\dfrac{3}{n^6}}{1+\dfrac{5}{n}} =0$.}
\end{ex}
\begin{ex}%[1C3Y1-2]%[Anh Duy]%Câu 6.
	Tính $\lim\dfrac{2n+1}{1+n}$ được kết quả là
	\choice
	{\True $2$}
	{$0$}
	{$\dfrac{1}{2}$}
	{$1$}
	\loigiai{
		Ta có $\lim\dfrac{2n+1}{1+n}=\lim\dfrac{n\left(2+\dfrac{1}{n}\right)}{n\left(\dfrac{1}{n}+1\right)}=\lim\dfrac{2+\dfrac{1}{n}}{\dfrac{1}{n}+1}=\dfrac{2+0}{0+1}=2$.}
\end{ex}

\begin{ex}%[1C3Y1-2]%[Anh Duy]%Câu 7.
	Dãy số nào sau đây có giới hạn khác $0$?
	\choice
	{$\dfrac{1}{n}$}
	{$\dfrac{1}{\sqrt{n}}$}
	{\True $\dfrac{n+1}{n}$}
	{$\dfrac{\sin n}{\sqrt{n}}$}
	\loigiai{
		Có $\lim\dfrac{n+1}{n}=\lim \limits{n \to +\infty}1+\lim\dfrac{1}{n}=1$.}
\end{ex}

\begin{ex}%[1C3B1-2]%[Anh Duy]%Câu 8.
	Giới hạn $\lim\dfrac{\sqrt{n}}{2n^2+3}$ có kết quả là 
	\choice
	{$2$}
	{\True $0$}
	{$+\infty$}
	{$4$}
	\loigiai{
		$\lim\dfrac{\sqrt{n}}{2n^2+3}=\lim\dfrac{\sqrt{\dfrac{1}{n^3}}}{2+\dfrac{3}{n^2}}=\dfrac{0}{2+0}=0$.}
\end{ex}
\begin{ex}%[1C3Y1-1]%[Anh Duy]%Câu 9.
	Dãy số $(u_n)$ với $u_n=\dfrac{1}{2n}$, chọn $M=\dfrac{1}{100}$, để $\dfrac{1}{2n}<\dfrac{1}{100}$ thì $n$ phải lấy từ số hạng thứ bao nhiêu trở đi?
	\choice
	{\True $51$}
	{$49$}
	{$48$}
	{$50$}
	\loigiai{Ta có $\dfrac{1}{2n}<\dfrac{1}{100}\Leftrightarrow 2n>100\Leftrightarrow n>50$.\\
	Vậy $n$ phải lấy từ số hạng thứ $51$ trở đi.}
\end{ex}
\begin{ex}%[1C3B1-2]%[Anh Duy]%Câu 10.
	Giới hạn $\lim\dfrac{3^n+2^n}{4^n}$ có kết quả là 
	\choice
	{\True $0$}
	{$\dfrac{5}{4}$}
	{$\dfrac{3}{4}$}
	{$+\infty$}
	\loigiai{
		Ta có $\lim\dfrac{3^n+2^n}{4^n}=\lim\dfrac{\left(\dfrac{3}{4}\right)^n+\left(\dfrac{2}{4}\right)^n}{1}=0$.}
\end{ex}
\begin{ex}%[1C3B1-2]%[Anh Duy]%Câu 11.
	Tính giới hạn $\lim\left[\dfrac{1}{1\cdot 2}+\dfrac{1}{2\cdot 3}+\dfrac{1}{3\cdot 4}+\cdots +\dfrac{1}{n(n+1)}\right]$. 
	\choice
	{$0$}
	{$2$}
	{\True $1$}
	{$\dfrac{3}{2}$}
	\loigiai{
		Ta có $\dfrac{1}{1\cdot 2}+\dfrac{1}{2\cdot 3}+\dfrac{1}{3\cdot 4}+\cdots +\dfrac{1}{n(n+1)} =\dfrac{1}{1}-\dfrac{1}{2}+\dfrac{1}{2}-\dfrac{1}{3}+\cdots+\dfrac{1}{n-1}-\dfrac{1}{n}+\dfrac{1}{n}-\dfrac{1}{n+1} =1-\dfrac{1}{n+1}$.\\
		Vậy $\lim\left[\dfrac{1}{1\cdot 2}+\dfrac{1}{2\cdot 3}+\dfrac{1}{3\cdot 4}+\cdots +\dfrac{1}{n(n+1)}\right] =\lim\left(1-\dfrac{1}{n+1}\right)=1$.}
\end{ex}

\begin{ex}%[1C3K1-2]%[Anh Duy]%Câu 12.
	Tính $\lim\sqrt{\dfrac{1^2+2^2+3^2+\cdots +n^2}{2n(n+7)(6n+5)}}$. 
	\choice
	{\True $\dfrac{1}{6}$}
	{$\dfrac{1}{2\sqrt{6}}$}
	{$\dfrac{1}{2}$}
	{$+\infty$}
	\loigiai{
		Ta có $1^2+2^2+3^2+\cdots +n^2=\dfrac{n(n+1)(2n+1)}{6}$.\\
		Khi đó $\lim\sqrt{\dfrac{1^2+2^2+3^3+\cdots +n^2}{2n(n+7)(6n+5)}}=\lim\sqrt{\dfrac{n(n+1)(2n+1)}{12n(n+7)(6n+5)}} =\lim\sqrt{\dfrac{\left(1+\dfrac{1}{n}\right)\left(2+\dfrac{1}{n}\right)}{12\left(1+\dfrac{7}{n}\right)\left(6+\dfrac{5}{n}\right)}} =\dfrac{1}{6}$.}
\end{ex}
\begin{ex}%[1C3K1-2]%[Anh Duy]%Câu 13.
	Giới hạn $\lim\dfrac{(2n-1)(3-n)^2}{(4n-5)^3}$ có kết quả bằng 
	\choice
	{$0$}
	{\True $\dfrac{1}{32}$}
	{$\dfrac{3}{2}$}
	{$\dfrac{1}{2}$}
	\loigiai{
		$\lim\dfrac{(2n-1)(3-n)^2}{(4n-5)^3}=\lim\dfrac{\left(2-\dfrac{1}{n}\right)\left(\dfrac{3}{n}-1\right)^2}{\left(4-\dfrac{5}{n}\right)^3}=\dfrac{2}{4^3}=\dfrac{1}{32}$.}
\end{ex}
\begin{ex}%[1C3K1-2]%[Anh Duy]%Câu 14.
	Tìm $L=\lim\left(\dfrac{1}{1}+\dfrac{1}{1+2}+\cdots +\dfrac{1}{1+2+\cdots +n}\right)$.
	\choice
	{$L=\dfrac{5}{2}$}
	{$L=+\infty$}
	{\True $L=2$}
	{$L=\dfrac{3}{2}$}
	\loigiai{
		Ta có $1+2+3+\cdots +k$ là tổng của cấp số cộng có $u_1=1$, $d=1$ nên $1+2+3+\cdots +k=\dfrac{(1+k)k}{2}$. Khi đó
		\[\dfrac{1}{1+2+\cdots +k}=\dfrac{2}{k(k+1)} =\dfrac{2}{k}-\dfrac{2}{k+1},\; \forall k\in\mathbb{N}^*.\]
	Suy ra	\[L=\lim\left(\dfrac{2}{1}-\dfrac{2}{2}+\dfrac{2}{2}-\dfrac{2}{3}+\dfrac{2}{3}-\dfrac{2}{4}+\cdots +\dfrac{2}{n}-\dfrac{2}{n+1}\right) =\lim\left(\dfrac{2}{1}-\dfrac{2}{n+1}\right) =2.\]}
\end{ex}
%--------------------------------------------------------------------------------------------------

\begin{ex}%[1C3G1-2]%[Anh Duy]%Câu 15.
	Đặt $f(n)=\left(n^2+n+1\right)^2+1$.
	Xét dãy số $(u_n)$ sao cho $u_n=\dfrac{f(1)\cdot f(3)\cdot f(5)\cdots f(2n-1)}{f(2)\cdot f(4)\cdot f(6)\cdots f(2n)}$. Tính $\lim \limits{n \to +\infty}n\sqrt{u_n}$. 
	\choice
	{$\lim \limits{n \to +\infty}n\sqrt{u_n}=\sqrt{2}$}
	{$\lim \limits{n \to +\infty}n\sqrt{u_n}=\dfrac{1}{\sqrt{3}}$}
	{$\lim \limits{n \to +\infty}n\sqrt{u_n}=\sqrt{3}$}
	{\True $\lim \limits{n \to +\infty}n\sqrt{u_n}=\dfrac{1}{\sqrt{2}}$}
	\loigiai{
		Xét $g(n)=\dfrac{f(2n-1)}{f(2n)}\Rightarrow g(n)=\dfrac{\left(4n^2-2n+1\right)^2+1}{\left(4n^2+2n+1\right)^2+1}$.\\
		$g(n)=\dfrac{\left(4n^2+1\right)^2-4n\left(4n^2+1\right)+\left(4n^2+1\right)}{\left(4n^2+1\right)^2+4n\left(4n^2+1\right)+\left(4n^2+1\right)}=\dfrac{4n^2+1-4n+1}{4n^2+1+4n+1}=\dfrac{(2n-1)^2+1}{(2n+1)^2+1}$ \\
		$ \Rightarrow u_n=\dfrac{2}{10}\cdot\dfrac{10}{26}\cdot\dfrac{26}{50}\cdots\cdot\dfrac{(2n-3)^2+1}{(2n-1)^2+1}\cdot\dfrac{(2n-1)^2+1}{(2n+1)^2+1}=\dfrac{2}{(2n+1)^2+1} $ \\
		$ \Rightarrow\lim \limits{n \to +\infty}n\sqrt{u_n}=\lim\sqrt{\dfrac{2n^2}{4n^2+4n+2}}=\dfrac{1}{\sqrt{2}} $.}
\end{ex}
\begin{ex}%[1C3G1-2]%[Anh Duy]%Câu 16.
	Có bao nhiêu giá trị nguyên của tham số $a$ thuộc khoảng $(0;2024)$ để có
	\[ \lim\sqrt{\dfrac{9^n+3^{n+1}}{5^n+9^{n+a}}}\leq\dfrac{1}{2187}\,? \]
	\choice
	{\True $2017$}
	{$2016$}
	{$2023$}
	{$2024$}
	\loigiai{
		Do $\dfrac{9^n+3^{n+1}}{5^n+9^{n+a}}>0$ với $\forall n$ nên $\lim\sqrt{\dfrac{9^n+3^{n+1}}{5^n+9^{n+a}}}=\sqrt{\lim\dfrac{9^n+3^{n+1}}{5^n+9^{n+a}}} =\sqrt{\lim\dfrac{1+3\cdot\left(\dfrac{1}{3}\right)^n}{\left(\dfrac{5}{9}\right)^n+9^a}} =\sqrt{\dfrac{1}{9^a}} =\dfrac{1}{3^a}$.\\
		Theo đề bài ta có $\lim\sqrt{\dfrac{9^n+3^{n+1}}{5^n+9^{n+a}}}\leq\dfrac{1}{2187}\Leftrightarrow\dfrac{1}{3^a}\leq\dfrac{1}{2187}\Leftrightarrow a\geq 7$.\\
		Do $a$ là số nguyên thuộc khoảng $(0;2024)$ nên có $a\in\left\{7;8;9;\ldots;2023\right\}\Rightarrow$ có $2017$ giá trị của $a$.}
\end{ex}



\Closesolutionfile{ans}
