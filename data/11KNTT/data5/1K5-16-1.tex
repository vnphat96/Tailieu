
\section{Giới hạn của hàm số}

\subsection{Tóm tắt lý thuyết}
\begin{tomtat}
	\subsubsection{Giới hạn hữu hạn của hàm số tại một điểm}

\begin{dn}
	Cho điểm $ x_0 $ thuộc khoảng $ K $ và hàm số $ y=f(x) $ xác định trên $ K $ hoặc $ K\setminus\{x_0\} $.\\
	Ta nói hàm số $ y=f(x) $ \textbf{\textit{có giới hạn hữu hạn}} là số $ L $ khi $ x $ dần tới $ x_0 $ nếu với dãy số $ (x_n) $ bất kì, $ x_n\in K\setminus\{x_0\} $ và $ x_n \to x_0 $ thì $ f(x_n)\to L $, kí hiệu $ \lim \limits_{x \to x_0} f(x) =L$ hay $ f(x)\to L $ khi $ x\to x_0 $.
\end{dn}

\begin{note} 
	$ \lim \limits_{x \to x_0} x=x_0 $; \quad $ \lim \limits_{x \to x_0} c=c $ ($ c $ là hằng số).
\end{note} 
	\subsubsection{Các phép toán về giới hạn hữu hạn của hàm số}
		\begin{enumerate}
			\item Cho $ \lim \limits_{x \to x_0} f(x) =L$ và $ \lim \limits_{x \to x_0} g(x)=M $. Khi đó:
			\begin{itemize}
				\item $ \lim \limits_{x \to x_0} [f(x)+g(x)]=L+M $;
				\item $ \lim \limits_{x \to x_0} [f(x)-g(x)]=L-M $;
				\item $ \lim \limits_{x \to x_0} [f(x)\cdot g(x)]=L\cdot M $;
				\item $ \lim \limits_{x \to x_0} \dfrac{f(x)}{g(x)}=\dfrac{L}{M} $ (với $ M\neq 0 $).			
			\end{itemize}
			\item Nếu $ f(x)\geq 0 $ và $ \lim \limits_{x \to x_0} f(x) =L$ thì $ L\geq 0 $ và $ \lim \limits_{x \to x_0} \sqrt{f(x)}=\sqrt{L} $.\\
			(Dấu của $ f(x) $ được xét trên khoảng tìm giới hạn, $ x\neq x_0 $).
		\end{enumerate}
\begin{note} 
	\indent
	\begin{enumerate}
		\item $ \lim \limits_{x \to x_0} x^k=x_0^k $, $ k $ là số nguyên dương;
		\item $ \lim \limits_{x \to x_0} [cf(x)]=c\lim \limits_{x \to x_0} f(x) $ ($ c\in \mathbb{R} $, nếu tồn tại $ \lim \limits_{x \to x_0} f(x) \in \mathbb{R}$).
	\end{enumerate}
\end{note} 

	\subsubsection{Giới hạn một phía}
\begin{dn}
	\
	\begin{itemize}
		\item Cho hàm số $ y=f(x) $ xác định trên khoảng $ (x_0;b) $.\\
		Ta nói hàm số $ y=f(x) $ \textbf{\textit{có giới hạn bên phải}} là số $ L $ khi $ x $ dần tới $ x_0 $ nếu với dãy số $ (x_n) $ bất kì, $ x_0<x_n<b $ và $ x_n\to x_0 $ thì $ f(x_n)\to L $, kí hiệu $ \lim \limits_{x \to x_0^+} f(x) =L$.
		\item Cho hàm số $ y=f(x) $ xác định trên khoảng $ (a;x_0) $.\\
		Ta nói hàm số $ y=f(x) $ \textbf{\textit{có giới hạn bên trái}} là số $ L $ khi $ x $ dần tới $ x_0 $ nếu với dãy số $ (x_n) $ bất kì, $ a<x_n<x_0 $ và $ x_n\to x_0 $ thì $ f(x_n)\to L $, kí hiệu $ \lim \limits_{x \to x_0^-} f(x) =L$.
	\end{itemize}
\end{dn}
\begin{note} 
	\begin{enumerate}
		\item Ta thừa nhận các kết quả sau:
		\begin{itemize}
			\item $ \lim \limits_{x \to x_0^+} f(x)=L$ và $ \lim \limits_{x \to x_0^-} f(x)=L $ khi và chỉ khi $ \lim \limits_{x \to x_0} f(x) =L$;
			\item Nếu $ \lim \limits_{x \to x_0^+} f(x)\neq \lim \limits_{x \to x_0^-} f(x)$ thì không tồn tại $ \lim \limits_{x \to x_0} f(x) $.
		\end{itemize}
		\item Các phép toán về giới hạn hữu hạn của hàm số ở Mục 2 vẫn đúng khi ta thay $ x\to x_0 $ bằng $ x\to x_0^+ $ hoặc $ x\to x_0^- $.
	\end{enumerate}
\end{note}

    \subsubsection{Giới hạn hữu hạn của hàm số tại vô cực}
\begin{dn}
	\
	\begin{itemize}
		\item Cho hàm số $ y=f(x) $ xác định trên khoảng $ (a;+\infty) $.\\
		Ta nói hàm số $ y=f(x) $ \textbf{\textit{có giới hạn hữu hạn}} là số $ L $ khi $ x \to +\infty  $ nếu với dãy số $ (x_n) $ bất kì, $ x_n>a $ và $ x_n\to +\infty $ thì $ f(x_n)\to L $, kí hiệu $ \lim \limits_{x \to +\infty} f(x) =L$ hay $ f(x) \to L $ khi $ x\to +\infty $.
		\item Cho hàm số $ y=f(x) $ xác định trên khoảng $ (-\infty;a) $.\\
		Ta nói hàm số $ y=f(x) $ \textbf{\textit{có giới hạn hữu hạn}} là số $ L $ khi $ x \to -\infty $ nếu với dãy số $ (x_n) $ bất kì, $ x_n<a $ và $ x_n\to -\infty $ thì $ f(x_n)\to L $, kí hiệu $ \lim \limits_{x \to -\infty} f(x) =L$ hay $ f(x)\to L $ khi $ x\to -\infty $.
	\end{itemize}
\end{dn}
\begin{note}
	\begin{enumerate}
		\item Với $ c $ là hằng số và $ k $ là  số nguyên dương, ta luôn có:
		$$\lim \limits_{x \to \pm \infty} c=c\quad \text{và} \quad \lim \limits_{x \to \pm \infty} \dfrac{c}{x^k}=0.$$
		\item Các phép toán trên giới hạn hàm số ở Mục 2 vẫn đúng khi thay $ x\to x_0 $ bằng $ x\to +\infty $ hoặc $ x\to -\infty $.
	\end{enumerate}
\end{note}

\subsubsection{Giới hạn vô cực của hàm số tại một điểm}

\begin{dn}
	Cho hàm số $ y=f(x) $ xác định trên khoảng $ (x_0;b) $.
	\begin{itemize}
		\item 
		Ta nói hàm số $ y=f(x) $ \textbf{\textit{có giới hạn bên phải}} là $ +\infty $ khi $ x $ dần tới $ x_0 $ về bên phải nếu với dãy số $ (x_n) $ bất kì, $ x_0<x_n<b $ và $ x_n\to x_0 $ thì $ f(x_n)\to +\infty $, kí hiệu $ \lim \limits_{x \to x_0^+} f(x) =+\infty$ hay $ f(x)\to +\infty $ khi $ x \to x_0^+ $.
		\item Ta nói hàm số $ y=f(x) $ \textbf{\textit{có giới hạn bên phải}} là  $ -\infty $ khi $ x $ dần tới $ x_0 $ về bên phải nếu với dãy số $ (x_n) $ bất kì, $ x_0<x_n<b $ và $ x_n\to x_0 $ thì $ f(x_n)\to -\infty $, kí hiệu $ \lim \limits_{x \to x_0^+} f(x) =-\infty$ hay $ f(x)\to -\infty $ khi $ x\to x_0^+ $.
	\end{itemize}
\end{dn}

\begin{note}
	\begin{enumerate}
		\item Các giới hạn $ \lim \limits_{x \to x_0^-} f(x)=+\infty $, $ \lim \limits_{x \to x_0^-} f(x)=-\infty $, $ \lim \limits_{x \to +\infty} f(x)=+\infty $, $ \lim \limits_{x \to +\infty} f(x)=-\infty $, $ \lim \limits_{x \to -\infty} f(x)=+\infty $, $ \lim \limits_{x \to -\infty} f(x)=-\infty $ được định nghĩa như trên.
		\item Ta có các giới hạn thường dùng như sau:
		\begin{itemize}
			\item $ \lim \limits_{x \to a^+} \dfrac{1}{x-a}=+\infty $ và $ \lim \limits_{x \to a^-}\dfrac{1}{x-a}=-\infty $ ($ a\in \mathbb{R} $);
			\item $ \lim \limits_{x \to +\infty}x^k=+\infty $ với $ k $ nguyên dương;
			\item $ \lim \limits_{x \to -\infty}x^k=+\infty $ với $ k $ là số chẵn;
			\item $ \lim \limits_{x \to -\infty}x^k=-\infty $ với $ k $ là số lẻ.
		\end{itemize}
		\item Các phép toán trên giới hạn hàm số của Mục 2 chỉ áp dụng được khi tất cả các hàm số được xét có giới hạn hữu hạn. Với giới hạn vô cực, ta có một số quy tắc sau đây.\\
		Nếu $ \lim \limits_{x \to x_0^+} f(x)=L\neq 0 $ và $ \lim \limits_{x \to x_0^+} g(x)=+\infty $ (hoặc $ \lim \limits_{x \to x_0^+} g(x)=-\infty $) thì $ \lim \limits_{x \to x_0^+}[f(x)\cdot g(x)] $ được tính theo quy tắc cho bởi bảng sau:
		\begin{center}
			\begin{tabular}{|c|c|c|}
				\hline 
				$ \lim \limits_{x \to x_0^+} f(x) $	& $  \lim \limits_{x \to x_0^+} g(x) $&$ \lim \limits_{x \to x_0^+} [f(x)\cdot g(x)] $  \\ 
				\hline 
				$ L>0 $	&$ +\infty $  &$ +\infty $  \\ 
				\hline 
				$ L>0 $	&$ -\infty $ &$ -\infty $  \\ 
				\hline 
				$ L<0 $&  $ +\infty $&  $ -\infty $\\ 
				\hline 
				$ L<0 $& $ -\infty $ &$ +\infty $  \\ 
				\hline 
			\end{tabular} 
		\end{center}
		Các quy tắc trên vẫn đúng khi thay $ x_0^+ $ thành $ x_0^- $ (hoặc $ +\infty $, $ -\infty $).
	\end{enumerate}
\end{note}
\end{tomtat}

\subsection{Các dạng toán thường gặp}
\begin{dang}{Thay số trực tiếp}
	Nội dung và phương pháp giải
\end{dang}
\subsubsection{Ví dụ minh hoạ}
\begin{vd}%[1K5YF-2]
	Tính các giới hạn sau
	\begin{listEX}[2]
		\item $ \lim \limits_{x \to 1} (x^2-4x+2)$;
		\item $ \lim \limits_{x \to 2} \dfrac{3x-2}{2x+1}$.
	\end{listEX}
	\loigiai{
		\begin{enumerate}[a)]
			\item $ \lim \limits_{x \to 1} (x^2-4x+2)=\lim \limits_{x \to 1} x^2-\lim \limits_{x \to 1} (4x)+\lim \limits_{x \to 1} 2=1^2-4\lim \limits_{x \to 1} x+2=1-4\cdot 1+2=-1$;
			\item $ \lim \limits_{x \to 2} \dfrac{3x-2}{2x+1}=\dfrac{\lim \limits_{x \to 2} (3x-2)}{\lim \limits_{x \to 2} (2x+1)}=\dfrac{3\lim \limits_{x \to 2} x-2}{2\lim \limits_{x \to 2} x+1}=\dfrac{3 \cdot 2-2}{2\cdot 2+1}=\dfrac{4}{5}$.
		\end{enumerate}
	}
\end{vd}

\begin{vd}%[1K5YF-2]
	Tìm các giới hạn sau
	\begin{enumEX}[a)]{2}
		\item $\lim\limits_{x\to 3} \sqrt{\dfrac{x^2}{x^3-x-6}}$.
		\item $\lim\limits_{x\to -2} \sqrt[3]{\dfrac{2x^4+3x+2}{x^2-x+2}}$.
	\end{enumEX}
	\loigiai{
		\begin{enumerate}[a)]
			\item $\lim\limits_{x\to 3} \sqrt{\dfrac{x^2}{x^3-x-6}}$; do $\lim\limits_{x\to 3} \dfrac{x^2}{x^3-x-6}=\dfrac{3^2}{3^3-3-6}=\dfrac{1}{2}>0$ \\
			$ \Rightarrow \lim\limits_{x\to 3} \sqrt{\dfrac{x^2}{x^3-x-6}}=\sqrt{\dfrac{1}{2}}=\dfrac{\sqrt{2}}{2}$.
			\item $\lim\limits_{x\to -2} \sqrt[3]{\dfrac{2x^4+3x+2}{x^2-x+2}}$; do $\lim\limits_{x\to -2} \dfrac{2x^4+3x+2}{x^2-x+2}=\dfrac{7}{2} \Rightarrow \lim\limits_{x\to -2} \sqrt[3]{\dfrac{2x^4+3x+2}{x^2-x+2}}=\sqrt[3]{\dfrac{7}{2}}=\dfrac{\sqrt[3]{28}}{2}$. 
		\end{enumerate}
	}
\end{vd}


\begin{vd}%[1K5YF-2]
	Cho $f(x)$ là một đa thức thỏa mãn $\displaystyle\lim\limits_{x\to 1}\dfrac{f(x)-16}{x-1}=24$. Tính giới hạn sau $$\displaystyle\lim\limits_{x\to 1}\dfrac{f(x)-16}{\left({x-1}\right)\left(\sqrt{2f(x)+4}+6\right)}.$$
	\loigiai{
		Vì $\displaystyle\lim\limits_{x\to 1}\dfrac{f(x)-16}{x-1}=24$ nên $f(1)=16.$ Khi đó
		$$ \lim\limits_{x\to 1}\dfrac{f(x)-16}{\left({x-1}\right)\left(\sqrt{2f(x)+4}+6\right)} =\frac{1}{12}\cdot \lim\limits_{x\to 1}\dfrac{f(x)-16}{x-1}=2.$$
	}
\end{vd}
\subsubsection{Bài tập rèn luyện}
\subsubsection{Bài tập tự luận}
\begin{bt}%[1K5YF-2]
	Tính các giới hạn sau:
	\begin{listEX}[2]
		\item $ \lim \limits_{x \to -2} (x^2+5x-2)$;
		\item $ \lim \limits_{x \to 1} \dfrac{x^2-1}{x-1}$.
	\end{listEX}
	\loigiai{
		\begin{enumerate}
			\item $ \lim \limits_{x \to -2} (x^2+5x-2)=(\lim \limits_{x \to -2} x)^2+\lim \limits_{x \to -2} (5x)-\lim \limits_{x \to -2} 2=(-2)^2+5\cdot (-2)-2=-8$.
			\item $ \lim \limits_{x \to 1} \dfrac{x^2-1}{x-1}=\lim \limits_{x \to 1} \dfrac{(x-1)(x+1)}{(x-1)}=\lim \limits_{x \to 1} (x+1)=\lim \limits_{x \to 1} x+1=1+1=2$.
		\end{enumerate}
	}
\end{bt}

\begin{bt}%[1K5YF-2]
	Tính các giới hạn sau
	\begin{enumEX}[a)]{2}
		\item $\lim\limits_{x\to -1} (3x^2-2x+1)$.
		\item $\lim\limits_{x\to 2} \dfrac{(x^3-3x)(x+1)}{x^2+3}$.
	\end{enumEX}
	\loigiai{
		\begin{enumerate}[a)]
			\item $\lim\limits_{x\to -1} (3x^2-2x+1)=3\lim\limits_{x\to -1} x^2-2\lim\limits_{x\to -1} x+\lim\limits_{x\to -1} 1=3{(1)}^2-2\cdot 1+1=2$.
			\item Do $\lim\limits_{x\to 2} (x^2+3)=2^2+3=7\ne 0$ và\\ $$\lim\limits_{x\to 2} (x^3-3x)(x+1)=\lim\limits_{x\to 2} (x^3-3x)\cdot \lim\limits_{x\to 2} (x+1)=(2^3-3\cdot 2)\cdot (2+1)=6$$
			Nên $\lim\limits_{x\to 2} \dfrac{(x^3-3x)(x+1)}{x^2+3}=\dfrac{6}{7}$.
		\end{enumerate}
	}
\end{bt}

\begin{bt}%[1K5YF-2]
	Tìm các giới hạn sau
	\begin{enumEX}[a)]{2}
		\item $\lim\limits_{x\to 2} \sqrt{\dfrac{2}{x^2-x+3}}$.
		\item $\lim\limits_{x\to -3} \sqrt[3]{\dfrac{-5}{x^2+x-12}}$.
	\end{enumEX}
	\loigiai{
		\begin{enumerate}[a)]
			\item $\lim\limits_{x\to 2} \sqrt{\dfrac{x}{x^2-x+3}}$; do $\lim\limits_{x\to 2} \dfrac{2}{x^2-x+3}=\dfrac{2}{2^2-2+3}=\dfrac{2}{5}>0$ \\
			$ \Rightarrow \lim\limits_{x\to 2} \sqrt{\dfrac{2}{x^2-x+3}}=\sqrt{\dfrac{2}{5}}=\dfrac{\sqrt{10}}{5}$.
			\item $\lim\limits_{x\to -3} \sqrt[3]{\dfrac{-5}{x^2+x-12}}$; do $\lim\limits_{x\to -3} \dfrac{-5}{x^2+x-12}=\dfrac{5}{6} \Rightarrow \lim\limits_{x\to -3} \sqrt[3]{\dfrac{-5}{x^2+x-12}}=\sqrt[3]{\dfrac{5}{6}}=\dfrac{\sqrt[3]{180}}{6}$. 
		\end{enumerate}
	}
\end{bt} 

\begin{bt}%[1K5YF-2]
	Cho $f(x)=x-1$ và $g(x)=x^{3}$. Tính các giới hạn sau:
	\begin{enumerate}
		\item $\lim \limits_{x \rightarrow 1}[3 f(x)-g(x)]$.
		\item $\lim \limits_{x \rightarrow 1} \dfrac{[f(x)]^{2}}{g(x)}$.
	\end{enumerate}
	\loigiai{Ta có $\lim \limits_{x \rightarrow 1} f(x)=\lim \limits_{x \rightarrow 1}(x-1)=\lim \limits_{x \rightarrow 1} x-\lim \limits_{x \rightarrow 1} 1=1-1=0$. Mặt khác, ta thấy $\lim \limits_{x \rightarrow 1} g(x)=\lim \limits_{x \rightarrow 1} x^{3}=1$.
		\begin{enumerate}
			\item Ta có
			$$
			\lim \limits_{x \rightarrow 1}[3 f(x)-g(x)]=\lim \limits_{x \rightarrow 1}[3 f(x)]-\lim \limits_{x \rightarrow 1} g(x)=\lim \limits_{x \rightarrow 1} 3 \cdot \lim \limits_{x \rightarrow 1} f(x)-\lim \limits_{x \rightarrow 1} g(x)=3 \cdot 0-1=-1 .
			$$
			\item  Ta có
			$$
			\lim \limits_{x \rightarrow 1} \dfrac{[f(x)]^{2}}{g(x)}=\dfrac{\lim \limits_{x \rightarrow 1}[f(x)]^{2}}{\lim \limits_{x \rightarrow 1} g(x)}=\dfrac{\lim \limits_{x \rightarrow 1} f(x) \cdot \lim \limits_{x \rightarrow 1} f(x)}{\lim \limits_{x \rightarrow 1} g(x)}=\dfrac{0}{1}=0.
			$$
	\end{enumerate}}
\centerline{\fcolorbox{red}{yellow!50}{\bf {BÀI TẬP TRẮC NGHIỆM }}}
\end{bt}
\Opensolutionfile{ans}[ans/ans-1K5-2-Dang1]
\setcounter{vd}{0}
\begin{ex}%[1T3Y2-1]
	Biết $\lim\limits_{x \to +\infty}f(x)=m$, $\lim\limits_{x \to +\infty}g(x)=n$. Tính $\lim\limits_{x \to +\infty}\left[f(x)+g(x)\right]$.
	\choice
	{\True $m+n$}
	{$m-n$}
	{$mn$}
	{$\dfrac{m}{n}$}
	\loigiai
	{
		Ta có $\lim\limits_{x \to +\infty}\left[f(x)+g(x)\right] = \lim\limits_{x \to +\infty}f(x) + \lim\limits_{x \to +\infty}g(x) = m+n$.
	}
\end{ex}


\begin{ex}%[1K5YF-2]
	Khẳng định nào sau đây là đúng?
	\choice
	{$\lim\limits_{x\to x_0} \sqrt[3]{f(x)+g(x)} =\sqrt[3]{\lim\limits_{x \to x_0} f(x)}+\sqrt[3]{\lim\limits_{x \to x_0} g(x)}$}
	{$\lim\limits_{x\to x_0} \sqrt[3]{f(x)+g(x)} =\lim\limits_{x \to x_0}\sqrt[3]{f(x)}+\lim\limits_{x \to x_0}\sqrt[3]{g(x)}$}
	{\True $\lim\limits_{x\to x_0} \sqrt[3]{f(x)+g(x)} =\sqrt[3]{\lim\limits_{x \to x_0} [f(x)+g(x)]}$}
	{$\lim\limits_{x\to x_0} \sqrt[3]{f(x)+g(x)} =\lim\limits_{x \to x_0}\left[\sqrt[3]{f(x)}+\sqrt[3]{g(x)}\right]$}
	\loigiai{
		Theo định lý về giới hạn của thì $\lim\limits_{x\to x_0} \sqrt[3]{f(x)+g(x)} =\sqrt[3]{\lim\limits_{x \to x_0} [f(x)+g(x)]}$.
	}
\end{ex}


\begin{ex}%[1K5YF-2]
	Cho các giới hạn $\lim \limits_{x\rightarrow x_0}f(x)=2$, $\lim \limits_{x\rightarrow x_0}g(x)=3$. Tính $M=\lim \limits_{x\rightarrow x_0}[3f(x)-4g(x)]$.
	\choice
	{$M=5$}
	{$M=2$}
	{\True $M=-6$}
	{$M=3$}
	\loigiai{
		Ta có $M=\lim \limits_{x\rightarrow x_0}[3f(x)-4g(x)]=3\lim \limits_{x\rightarrow x_0}f(x)-4\lim \limits_{x\rightarrow x_0}g(x)=6-12=-6.$
	}
\end{ex}


\begin{ex}%[1K5YF-2]
	Biết $\lim\limits_{x \to +\infty}f(x)=m$, $\lim\limits_{x \to +\infty}g(x)=n$. Tính $\lim\limits_{x \to +\infty}\left[f(x)-g(x)\right]$.
	\choice
	{ $m+n$}
	{\True $m-n$}
	{$mn$}
	{$\dfrac{m}{n}$}
	\loigiai
	{
		Ta có $\lim\limits_{x \to +\infty}\left[f(x)-g(x)\right] = \lim\limits_{x \to +\infty}f(x) - \lim\limits_{x \to +\infty}g(x) = m-n$.
	}
\end{ex}


\begin{ex}%[1K5YF-2]
	Cho $\lim\limits_{x \to a} f(x)=-\infty$, kết quả của $\lim\limits_{x \to a} [-3\cdot f(x)]$ bằng
	\choice
	{\True $+ \infty$}
	{$0$}
	{$3$}
	{$-\infty$}
	\loigiai{
		Có $\lim\limits_{x \to a} [-3\cdot f(x)]=-3\cdot\lim\limits_{x \to a}f(x)=+ \infty$.
		
	}
\end{ex}


\begin{ex}%[1K5YF-2]
	Cho $k\in \mathbb{Z}$, kết quả của $\lim\limits_{x \to -\infty} x^{2k+1}$ bằng
	\choice
	{$0$}
	{\True $-\infty$}
	{$+\infty$}
	{$5$}
	\loigiai{
		Theo tính chất của giới hạn hàm số, ta có $\lim\limits_{x \to -\infty} x^{2k+1}=-\infty$.
		
	}
\end{ex}


\begin{ex}%[1K5YF-2]
	Cho $\displaystyle\lim\limits_{x \to 3} f(x) =-2$. Giá trị $\displaystyle\lim\limits_{x \to 3} \left[f(x)+4x-1\right]$ bằng
	\choice
	{$5$}
	{$6$}
	{$-11$}
	{\True $9$}
	\loigiai{
		$\displaystyle\lim\limits_{x \to 3} \left[f(x)+4x-1\right] = \lim\limits_{x \to 3} f(x) + \lim\limits_{x \to 3} 4x -1 = -2 + 4 \cdot 3 -1 =9$.
	}
\end{ex}


\begin{ex}%[1K5YF-2]
	Cho $\lim\limits_{x\to 2}f(x)=3$. Giá trị của $\lim\limits_{x \to 2}\left[f(x)+x\right]$ bằng
	\choice
	{\True $5$}
	{$6$}
	{$1$}
	{$4$}
	\loigiai
	{
		Ta có $\lim\limits_{x \to 2}\left[f(x)+x\right] = \lim\limits_{x \to 2}f(x) + \lim\limits_{x \to 2}x = 3+2=5$.
	}
\end{ex}


\begin{ex}%[1K5YF-2]
	Với $k$ là số nguyên dương. Kết quả của giới hạn $\lim\limits_{x\to+\infty}x^{2k}$ là
	\choice
	{\True $+\infty$}
	{$0$}
	{$-\infty$}
	{$1$}
	\loigiai{
		Với $k$ nguyên dương thì $\lim\limits_{x\to +\infty}x^k = +\infty \Rightarrow \lim\limits_{x\to+\infty}x^{2k}=+\infty$.
	}
\end{ex}


\begin{ex}%[1K5YF-2]
	Cho $c$ là hằng số, $k$ là số nguyên dương. Khẳng định nào sau đây \textbf{sai}?	
	\choice
	{\True $\lim\limits_{ x \rightarrow +\infty}c=+\infty$}
	{$\lim\limits_{ x \rightarrow +\infty}\dfrac{c}{x^k}=0$}
	{$\lim\limits_{ x \rightarrow x_0}c=c$}
	{$\lim\limits_{ x \rightarrow x_0}x=x_0$}
	\loigiai{Theo định lý về giới hạn, khẳng định sai là $\lim\limits_{ x \rightarrow +\infty}c=+\infty $.
		
	}
\end{ex}


\begin{ex}%[1K5YF-2]
	Cho $\lim\limits_{x \to +\infty} f(x)=a$,$\lim\limits_{x \to +\infty} g(x)=b$. Hỏi mệnh đề nào sau đây là mệnh đề $\textbf{sai}$?
	\choice
	{$\lim\limits_{x \to +\infty} \left[f(x)\cdot g(x)\right]=ab$}
	{$\lim\limits_{x \to +\infty} \left[f(x)- g(x)\right]=a-b$}
	{$\lim\limits_{x \to +\infty} \left[f(x)+ g(x)\right]=a+b$}
	{$\lim\limits_{x \to +\infty} \dfrac{f(x)}{g(x)}=\dfrac{a}{b}$}
	\loigiai{
		Khi $\lim\limits_{x \to +\infty} g(x)=b=0$ thì $\lim\limits_{x \to +\infty} \dfrac{f(x)}{g(x)}=\dfrac{a}{b}$ không đúng.
	}
\end{ex}


\begin{ex}%[1K5YF-2]
	Với $k$ là số nguyên dương thì $\lim\limits_{x \to -\infty} \dfrac{1}{x^k}$ bằng
	\choice
	{$+\infty$}
	{$-\infty$}
	{$x$}
	{\True $0$}
	\loigiai{
		Vì $\left\{\begin{aligned}
			&\lim\limits_{x \to -\infty} 1=1\\
			&\lim\limits_{x \to -\infty} x^k=\pm \infty\\
		\end{aligned}\right. $ nên $\lim\limits_{x \to -\infty} \dfrac{1}{x^k}=0$.
	}
\end{ex}


\begin{ex}%[1K5YF-2]
	Tính $I=\lim\limits_{x \to 2}\left(x^2+x-6\right)$.
	\choice
	{\True $0$}
	{$1$}
	{$2$}
	{$3$}
	\loigiai{
		\begin{eqnarray*}
			\lim\limits_{x \to 2}\left(x^2+x-6\right)&=&\lim\limits_{x \to 2} x^2+\lim\limits_{x \to 2} x-\lim\limits_{x \to 2} 6\\
			&=&4+2-6\\
			&=&0.
		\end{eqnarray*}	
	}
\end{ex}


\begin{ex}%[1K5YF-2]
	Tính $I=\lim\limits_{x \to 1} \dfrac{x^2+2 x+3}{2 x-1}$.	
	\choice
	{ $4$}
	{$5$}
	{\True $6$}
	{$7$}
	\loigiai{
		\begin{eqnarray*}
			\lim\limits_{x \to 1} \dfrac{x^2+2 x+3}{2 x-1}&=&\dfrac{\lim\limits_{x \to 1}\left(x^2+2 x+3\right)}{\lim\limits_{x \to 1}(2 x-1)}\\
			&=&\dfrac{\lim\limits_{x \to 1} x^2+\lim\limits_{x \to 1}(2 x)+\lim\limits_{x \to 1} 3}{\lim\limits_{x \to 1}(2 x)-\lim\limits_{x \to 1} 1}\\
			&=&\dfrac{1+2+3}{2-1}\\
			&=&6.
		\end{eqnarray*}		
	}
\end{ex}


\begin{ex}%[1K5YF-2]
	Tính  $I=\lim\limits_{x \to 0} \dfrac{\left|x\right|}{x}$. 
	\choice
	{$1$}
	{$-1$}
	{\True Không tồn tại }
	{$0$}
	\loigiai{
		Ta có :\\
		$\lim\limits_{x \to 0^{+}} \dfrac{\left|x\right|}{x}=\lim\limits_{x \to 0^{+}} \dfrac{x}{x}=\lim\limits_{x \to 0^{+}} 1=1$.\\
		$\lim\limits_{x \to 0^{-}} \dfrac{\left|x\right|}{x}=\lim\limits_{x \to 0^{-}} \dfrac{(-x)}{x}=\lim\limits_{x \to 0^{-}} (-1)=-1$.\\
		Vậy không tồn tại $\lim\limits_{x \to 0} \dfrac{\left|x\right|}{x}$. 
		
	}
\end{ex}
\Closesolutionfile{ans}

