\section{Giới hạn của dãy số}
\subsection{Tóm tắt lí thuyết}
\subsubsection{Giới hạn của dãy số}
\begin{dn}
	Dãy số $(u_n)$ có giới hạn là $0$ khi $n$ dần tới dương vô cực nếu $|u_n|$ có thể nhỏ hơn một số dương bé tuỳ ý, kể từ một số hạng nào đó trở đi.\\
	Kí hiệu: $\lim\limits_{n \to +\infty}u_n=0$ hay $\lim u_n=0$.
\end{dn}

\begin{vd}
	$\lim\limits_{n \to +\infty}\dfrac{1}{n^2}=0$.
\end{vd}

\begin{dn}
	Dãy số $(u_n)$ có giới hạn là $a$ nếu $|u_n-a|$ có giới hạn bằng $0$.\\
	Nghĩa là: $\lim\limits_{n \to +\infty}u_n=a$  $\Leftrightarrow \lim\limits_{n \to +\infty}(u_n-a)=0$.
\end{dn}

\begin{vd}
	$\lim\limits_{n\to +\infty}{\dfrac{2n+1}{n+3}}=2$.
\end{vd}
\subsubsection{Các định lý về giới hạn hữu hạn}
\begin{dl}$\textrm{ }$\\
	\begin{itemize}
\item  $\lim \dfrac{1}{n}=0; \lim \dfrac{1}{n^k}=0$ với $k$ là số nguyên dương.\\
\item  $\lim q^n=0$ nếu $|q|<1$.
\end{itemize}
\end{dl}

\begin{dl}$\textrm{ }$\\
	\begin{itemize}
		\item  Nếu $\lim u_n=a$ và $\lim v_n=b$  thì $\lim \left(u_n\pm v_n\right)=a\pm b$, $lim \left(u_n.v_n\right)=a.b$, $\lim\left(\dfrac{u_n}{v_n}\right)=\dfrac{a}{b}$ (nếu $b\neq 0$).
		\item  Nếu $u_n\geq 0$ với mọi $n$ và $\lim u_n=a$ thì $a\geq 0$ và $\lim\sqrt{u_n}=\sqrt{a}$.
	\end{itemize}
\end{dl}

\subsubsection{Tổng của cấp số nhân lùi vô hạn}
\begin{dn}
	Cấp số nhân vô hạn $(u_n)$ có công bội $q$ thoả mãn $|q|<1$ được gọi là \textbf{cấp số nhân lùi vô hạn}.
\end{dn}

\begin{dl}
Cho cấp số nhân lùi vô hạn $(u_n)$, ta có tổng của cấp số nhân lùi vô hạn đó là $$S=u_1+u_2+u_3+...+u_n+...=\dfrac{u_1}{1-q}, (|q|<1)$$
\end{dl}

\subsubsection{Giới hạn vô cực}
\begin{dn} $\textrm{ }$\\	
\begin{itemize}
\item  Ta nói dãy số $(u_n)$ có giới hạn $+\infty$ khi $n\to +\infty$, nếu $u_n$ có thể lớn hơn một số dương bất kì, kể từ một số hạng nào đó trở đi.\\
Kí hiệu: $\lim u_n=+\infty$.
\item  Ta nói dãy số $(u_n)$ có giới hạn $-\infty$ khi $n\to +\infty$, nếu $\lim (-u_n)=+\infty$.\\
Kí hiệu: $\lim u_n=-\infty$.
\end{itemize}
\end{dn}

\begin{dl} $\textrm{ }$\\
	a) Nếu $\lim u_n=a$ và $\lim v_n=\pm \infty$ thì $\lim\dfrac{u_n}{v_n}=0$.\\
	b) Nếu $\lim u_n=a>0, \lim v_n=0$ và $v_n>0$ với mọi $n$ thì $\lim\dfrac{u_n}{v_n}=+\infty$.\\
	c) Nếu $\lim u_n=+\infty$ và $\lim v_n=a>0$ thì $\lim u_nv_n=+\infty$.
\end{dl}

\subsection{Các dạng toán}
\begin{dang}{Dùng định nghĩa chứng minh giới hạn}
	Để chứng minh $\lim u_n = L$ ta chứng minh $\lim  \left( u_n- L \right) = 0$.
\end{dang}
\begin{vd}%[Phan Quốc Trí]%[1D4B1]
	Chứng minh rằng
	\begin{multicols}{2}
		\begin{enumerate}[a.]
			\item $\lim \left( \dfrac{-n^3}{n^3+1} \right) = -1$
			\item $\lim \left( \dfrac{n^2+3n+2}{2n^2+n}  \right)=\dfrac{1}{2}$.
		\end{enumerate}
	\end{multicols}
	\loigiai{
		\begin{enumerate}[a.]
			\item Ta có $\lim \left( \dfrac{-n^3}{n^3+1}  - (-1) \right)= \lim \left( \dfrac{1}{n^3+1} \right)$. Vì $0\le \left| \dfrac{1}{n^3+1} \right|<\dfrac{1}{n^3}, \; \forall n \in \mathbb{N^*}$.\\
			Mà $\lim \dfrac{1}{n^3}=0$ nên suy ra $\lim\left( \dfrac{1}{n^3+1} \right) =0$. Do đó $\lim \left( \dfrac{-n^3}{n^3+1} \right) = -1$.
			\item Ta có $\lim \left( \dfrac{n^2+3n+2}{2n^2+n} -\dfrac{1}{2} \right)=\lim  \dfrac{5n+4}{2\left(2n^2+n\right)}$\\
			Vì $0<\left|\dfrac{5n+4}{2\left(2n^2+n\right)} \right|<\dfrac{5n+5}{2n\left(n+1\right)}=\dfrac{5}{2}.\dfrac{1}{n},\; \forall n \in \mathbb{N^*}$. Mà  $\lim \left( \dfrac{5}{2}.\dfrac{1}{n} \right) =\dfrac{5}{2}.\lim \dfrac{1}{n}=0$\\
			Nên suy ra $\lim  \dfrac{5n+4}{2\left(2n^2+n\right)}=0$. Do đó $\lim \left( \dfrac{n^2+3n+2}{2n^2+n}  \right)=\dfrac{1}{2}$.
		\end{enumerate}
	}
\end{vd}
\begin{vd}%[Phan Quốc Trí]%[1D4B1]
	Chứng minh rằng 
	\begin{multicols}{2}
		\begin{enumerate}[a.]
			\item $\lim \left(\dfrac{3.3^n-\sin 3n}{3^n} \right)=3$
			\item $\lim \left( \sqrt{n^2+n}-n \right) = \dfrac{1}{2}$.
		\end{enumerate}
	\end{multicols}
	\loigiai{
		\begin{enumerate}[a.]
			\item Ta có $\lim \left(\dfrac{3.3^n-\sin 3n}{3^n}-3 \right)=\lim \left( \dfrac{-\sin 3n}{3^n} \right)$.\\
			Vì $0\le \left|\dfrac{-\sin 3n}{3^n}\right|=\dfrac{\left|-\sin 3n \right|}{3^n} \le \dfrac{1}{3^n}=\left(\dfrac{1}{3}\right)^n, \; \forall n \in \mathbb{N^*} $.\\
			Mà $\lim \left(\dfrac{1}{3}\right)^n =0$ nên suy ra $\lim \left( \dfrac{-\sin 3n}{3^n} \right)=0$. Do đó $\lim \left(\dfrac{3.3^n-\sin 3n}{3^n} \right)=3$.
			\item Ta có $\lim \left( \sqrt{n^2+n}-n -\dfrac{1}{2} \right)=\lim \dfrac{2\sqrt{n^2+n}-(2n+1)}{2}=\lim \dfrac{-1}{2\left(2\sqrt{n^2+n}+(2n+1)\right)}$\\
			\begin{eqnarray*}
				\text{Vì }0\le \left|\dfrac{-1}{2\left(2\sqrt{n^2+n}+(2n+1)\right)} \right| \le \dfrac{1}{2\left( 2\sqrt{n^2+n}+(2n+1) \right)}&\le& \dfrac{1}{2\left( 2\sqrt{n^2}+2n \right)}\\
				&=&\dfrac{1}{8}.\dfrac{1}{n}, \; \forall n \in \mathbb{N^*}
			\end{eqnarray*}
			Mà $\lim \dfrac{1}{8}.\dfrac{1}{n}=\dfrac{1}{8}\lim \dfrac{1}{n}=0$ nên suy ra $\lim \dfrac{-1}{2\left(2\sqrt{n^2+n}+(2n+1)\right)}=0$.\\
			Do đó  $\lim \left( \sqrt{n^2+n}-n \right) = \dfrac{1}{2}$.
		\end{enumerate}
	}
\end{vd}
\begin{center}
	\textbf{BÀI TẬP TỰ LUYỆN}
\end{center}
\begin{bt}%[Phan Quốc Trí]%[1D4B1]
	Chứng minh rằng 
	\begin{multicols}{2}
		\begin{enumerate}[a.]
			\item $\lim \dfrac{2n^2+n}{n^2+4}=2$
			\item $\lim \dfrac{6n+2}{n+5}=6$
			\item $\lim \dfrac{7^n-2.8^n}{8^n+3^n}=-2$
			\item $\lim \dfrac{2.3^n +5^n}{5^n+3^n}=1$.
		\end{enumerate}
	\end{multicols}
	\loigiai{
		\begin{enumerate}[a.]
			\item Ta có  $\lim \left( \dfrac{2n^2+n}{n^2+4} -2 \right)= \lim \dfrac{n-8}{n^2+4}$. Vì $0\le \left|\dfrac{n-8}{n^2+4} \right|\le \dfrac{n}{n^2}=\dfrac{1}{n}$.\\
			Mà $\lim \dfrac{1}{n} = 0$ nên suy ra $\lim \left( \dfrac{2n^2+n}{n^2+4} -2 \right)=0$. Do đó $\lim \dfrac{2n^2+n}{n^2+4}=2$.
			\item Ta có $\lim \left( \dfrac{6n+2}{n+5}-6 \right) = \lim \dfrac{-28}{n+5}$\\
			Vì $\left| \dfrac{-28}{n+5} \right| < \dfrac{28}{n}$. Mà $\lim \dfrac{28}{n}=0$ nên $\lim \left( \dfrac{6n+2}{n+5}-6 \right) =0$. Do đó  $\lim \dfrac{6n+2}{n+5}=6$.
			\item Ta có $\lim \left( \dfrac{7^n-2.8^n}{8^n+3^n} +2 \right) = \lim \dfrac{7^n+2.3^n}{8^n+3^n}$\\
			Vì $0<\left| \dfrac{7^n+2.3^n}{8^n+3^n} \right| < \dfrac{7^n+2.3^n}{8^n+3^n} < \dfrac{3.7^n}{8^n}=3\left(\dfrac{7}{8} \right)^n$.\\
			Mà $\lim 3\left(\dfrac{7}{8}\right)^n=0$ nên $\lim \left( \dfrac{7^n-2.8^n}{8^n+3^n} +2 \right)=0$. Do đó $\lim \dfrac{7^n-2.8^n}{8^n+3^n}=-2$.
			\item Ta có $\lim \left( \dfrac{2.3^n +5^n}{5^n+3^n}-1\right) = \lim \dfrac{3^n}{5^n+3^n}$.\\
			vì $0< \left| \dfrac{3^n}{5^n+3^n} \right|< \dfrac{3^n}{5^n+3^n} <\left( \dfrac{3}{5} \right)^n$\\
			Mà $\lim \left( \dfrac{3}{5} \right)^n =0$ nên $\lim \left( \dfrac{2.3^n +5^n}{5^n+3^n}-1\right)=0$. Do đó $\lim \dfrac{2.3^n +5^n}{5^n+3^n}=1$.
		\end{enumerate}
	}
\end{bt}

\begin{bt}%[Phan Quốc Trí]%[1D4B1]
	Chứng minh rằng 
	\begin{multicols}{2}
		\begin{enumerate}[a.]
			\item $\lim \left( \sqrt{4n^2+4n} -2n \right)=1$
			\item $\lim \dfrac{\sqrt{n}+\sin^n n}{\sqrt{n}+1}=1$
			\item $\lim \dfrac{\sqrt{n^2+2n}-n}{n}=0$
			\item $\lim \left( \sqrt[3]{n^3+2n}-n \right)=0$.
		\end{enumerate}
	\end{multicols}
	\loigiai{
		\begin{enumerate}[a.]
			\item Ta có $\lim \left( \sqrt{4n^2+4n} -2n-1 \right)=\lim \dfrac{-1}{\sqrt{4n^2+4n} +2n+1}$\\
			Vì $0\le \left| \dfrac{-1}{\sqrt{4n^2+4n} +2n+1} \right| \le \dfrac{1}{\sqrt{4n^2+4n} +2n+1} < \dfrac{1}{2n+2n}=\dfrac{1}{4n}$\\
			Mà $\lim \dfrac{1}{4n}=0$ nên $\lim \left( \sqrt{4n^2+4n} -2n-1 \right)=0$. Do đó $\lim \left( \sqrt{4n^2+4n} -2n \right)=1$
			\item Ta có $\lim \left(  \dfrac{\sqrt{n}+\sin^n n}{\sqrt{n}+1} -1\right)=\lim \dfrac{\sin^n n -1}{\sqrt{n}+1}$\\
			Vì $0 \le \left| \dfrac{\sin^n n -1}{\sqrt{n}+1} \right| < \dfrac{2}{\sqrt{n}}$.\\
			Mà $\lim \dfrac{2}{\sqrt{n}} =0$ nên $\lim \left(  \dfrac{\sqrt{n}+\sin^n n}{\sqrt{n}+1} -1\right)=0$. Do đó $\lim \dfrac{\sqrt{n}+\sin^n n}{\sqrt{n}+1}=1$.
			\item Ta có $\left|\dfrac{\sqrt{n^2+2n}-n}{n}\right|= \left| \dfrac{n^2+2n-n^2}{n\left( \sqrt{n^2+2n}+n \right)} \right| =\left| \dfrac{2}{\sqrt{n^2+2n}+n} \right|< \dfrac{2}{\sqrt{n^2}+n}=\dfrac{1}{n}$.\\
			Mà $\lim \dfrac{1}{n} =0$ nên $\lim \dfrac{\sqrt{n^2+2n}-n}{n}=0$.
			\item Ta có
			\begin{eqnarray*}
				\left| \sqrt[3]{n^3+2n}-n \right|&=& \left| \dfrac{n^3+2n-n^3}{\sqrt[3]{(n^3+2n)^2} +n\sqrt[3]{n^3+2n}+n^2} \right|\\
				&=& \left|\dfrac{2n}{\sqrt[3]{(n^3+2n)^2} +n\sqrt[3]{n^3+2n}+n^2} \right|<\dfrac{2n}{3n^2}<\dfrac{1}{n}.
			\end{eqnarray*}
			Mà $\lim \dfrac{1}{n} =0$. Do đó $\lim \left( \sqrt[3]{n^3+2n}-n \right)=0$
		\end{enumerate}
	}
\end{bt}
\begin{bt}%[Phan Quốc Trí]%[1D4B1]
	Chứng minh rằng 
	\begin{multicols}{2}
		\begin{enumerate}[a.]
			\item $\lim \dfrac{6^n \cos 3n +5^n}{2^n+2.7^n}=0$
			\item $\lim \dfrac{4n \sin^n 2n + \cos^n 2n}{4n^2+8n}=0$
		\end{enumerate}
	\end{multicols}
	\loigiai{
		\begin{enumerate}[a.]
			\item Ta có $\left| \dfrac{6^n \cos 3n +5^n}{2^n+2.7^n} \right|\le \dfrac{6^n+5^n}{2.7^n}\le \dfrac{2.6^n}{2.7^n}=\left(\dfrac{6}{7} \right)^n$.\\
			Mà $\lim \left(\dfrac{6}{7} \right)^n = 0$ nên $\lim \dfrac{6^n \cos 3n +5^n}{2^n+2.7^n}=0$.
			\item Ta có $\left| \dfrac{4n \sin^n 2n + \cos^n 2n}{4n^2+8n} \right|\le \left| \dfrac{4n+1}{4n(n+2)} \right| \le \dfrac{4(n+2)}{4n(n+2)}=\dfrac{1}{n}$\\
			Mà $\lim \dfrac{1}{n}=0$ nên $\lim \dfrac{4n \sin^n 2n + \cos^n 2n}{4n^2+8n}=0$.
		\end{enumerate}
	}
\end{bt}

\begin{dang}{Tính giới hạn dãy số dạng phân thức}
	Tính giới hạn	$\lim \dfrac{f\left(n\right)}{g\left(n\right)}$ trong đó $f\left(n\right)$ và $g\left(n\right)$ là các đa thức bậc $n$.\\
	\begin{itemize}
		\item Bước 1: Đặt $n^k$, $n^i$  với $k$ là số mũ cao nhất của đa thức $f\left(n\right)$ và $i$ là số mũ cao nhất của đa thức $g\left(n\right)$ ra làm nhân tử chung.
		\item Đơn giản. Sau đó áp dụng kết quả 	$\lim \dfrac{1}{n^k}=0$.
	\end{itemize}
\end{dang}
\begin{dang}{Tính giới hạn dãy số dạng phân thức chứa $a^n$}
	\begin{itemize}
		\item Bước 1: Đưa biểu thức về cùng một số mũ $n$.
		\item Bước 2: Chia tử và mẫu số cho $a^n$ trong đó $a$ là số có trị tuyệt đối lớn nhất.
		\item Bước 3: Áp dụng kết quả "Nếu $|q| <1$ thì $\lim q^n =1$".
	\end{itemize}
\end{dang}
\begin{vd}%[Nguyễn Bình Nguyên]%[1D4B1]
	Tính $\lim \dfrac{{{n}^2}-4{{n}^3}}{2{{n}^3}+5n-2}$.
	\loigiai{
		Ta có: $\lim \dfrac{{{n}^2}-4{{n}^3}}{2{{n}^3}+5n-2}=\lim \dfrac{{{n}^3}(\dfrac{1}{n}-4 )}{{{n}^3}(2+\dfrac{5}{{{n}^2}}-\dfrac{2}{{{n}^3}} )}=\lim \dfrac{\dfrac{1}{n}-4}{2+\dfrac{5}{{{n}^2}}-\dfrac{2}{{{n}^3}}}=-\dfrac{4}{2}=-2$.
	}
\end{vd}
\begin{vd}%[Nguyễn Bình Nguyên]%[1D4B1]
	Tính $\lim \dfrac{{{n}^3}-7n}{1-2{{n}^2}}$.
	\loigiai{
		Ta có: $\lim \dfrac{{{n}^3}-7n}{1+2{{n}^2}}=\lim \dfrac{{{n}^3}(1-\dfrac{7}{{{n}^2}} )}{{{n}^2}(\dfrac{1}{{{n}^2}}+2 )}=\lim (n.\dfrac{1-\dfrac{7}{{{n}^2}}}{\dfrac{1}{{{n}^2}}+2} )=+\infty $.\\
		Vì $\lim (\dfrac{1-\dfrac{7}{{{n}^2}}}{\dfrac{1}{{{n}^2}}+2} )=\dfrac{1}{2}>0;\,\,\lim n=+\infty $.
	}
\end{vd}
\begin{vd}%[Nguyễn Bình Nguyên]%[1D4B1]
	Tính $\lim \dfrac{n+2}{{{n}^2}+n+1}$
	\loigiai{
		Ta có: $\lim \dfrac{n+3}{{{n}^2}+n+2}=\lim \dfrac{n(1+\dfrac{3}{n} )}{{{n}^2}(1+\dfrac{1}{n}+\dfrac{2}{{{n}^2}} )}=\lim \dfrac{1}{n}.\dfrac{1+\dfrac{3}{n}}{1+\dfrac{1}{n}+\dfrac{3}{{{n}^2}}}=0$.
	}
\end{vd}
\begin{vd}%[Nguyễn Bình Nguyên]%[1D4B1]
	Tính $\lim \dfrac{5^{n+1}-4^{n}+1}{2.5^n-6^n}$.
	\loigiai{
		Ta có : $\lim \dfrac{5^{n+1}-4^{n}+1}{2.5^n-6^n}=\lim \dfrac{\dfrac{5^{n+1}-4^n+1}{6^n}}{\dfrac{2.5^n-6^n}{6^n}}=\lim\dfrac{5.\left(\dfrac{5}{6}\right)^n-\left(\dfrac{2}{3}\right)^n+\left(\dfrac{1}{6}\right)^n}{2\left(\dfrac{5}{6}\right)^n-1}=0$.
	}
\end{vd}
\begin{center}
	\textbf{BÀI TẬP TỰ LUYỆN (Cho mỗi dạng)}
\end{center}
\begin{bt}%[Nguyễn Bình Nguyên]%[1D4Y1]
	Tính các giới hạn
	\begin{enumerate}
		\begin{multicols}{2}
			\item $\lim \dfrac{3n+2}{2n+3}$.
			\item $\lim \dfrac{4n^2-1}{2n^2+n}$.
		\end{multicols}
	\end{enumerate}
	\loigiai{
		\begin{enumerate}
			\item Chia cả tử và mẫu cho $n$ có bậc lớn nhất. Ta có : $\lim \dfrac{3n+2}{2n+3}=\lim\dfrac{3+\dfrac{2}{n}}{2+\dfrac{3}{n}}=\dfrac{3}{2}$.
			\item Tương tự: $\lim \dfrac{4n^2-1}{2n^2+n}=\lim \dfrac{4-\dfrac{1}{n^2}}{2+\dfrac{1}{n}}=2$.
		\end{enumerate}
	}
\end{bt}
\begin{bt}%[Nguyễn Bình Nguyên]%[1D4B1]
	Tính các giới hạn
	\begin{enumerate}
		\begin{multicols}{2}
			\item $\lim \dfrac{\sqrt{n^2+2n}-3}{n+2}$.
			\item $\lim \dfrac{\sqrt{n^2+2n}-n-1}{\sqrt{n^2+n}+n}$.
		\end{multicols}
	\end{enumerate}
	\loigiai{
		\begin{enumerate}
			\item Ta có : $\lim \dfrac{\sqrt{1+\dfrac{2}{n}}-\dfrac{3}{n}}{1+\dfrac{2}{n}}=1$.
			\item Tương tự: $\lim \dfrac{\sqrt{1+\dfrac{2}{n}}-1-\dfrac{1}{n}}{\sqrt{1+\dfrac{1}{n}}+1}=0$.
		\end{enumerate}
	}
\end{bt}
\begin{bt}%[Nguyễn Bình Nguyên]%[1D4K1]
	Tính giới hạn $\lim \dfrac{\sqrt{4n^4+2n}-3n^2}{\sqrt{n^3+2n}-n}$.
	\loigiai{
		Ta có : \\
		$\lim \dfrac{\sqrt{4n^4+2n}-3n^2}{\sqrt{n^3+2n}-n}
		=\lim \dfrac{\sqrt{n^4\left(4+\dfrac{2}{n^3}\right)}-3n^2}{\sqrt{n^3\left(1+\dfrac{2}{n^2}\right)}-n}$\\
		$=\lim \dfrac{n^2\left(\sqrt{4+\dfrac{2}{n^3}}-3\right)}{\sqrt{n^3}\left(\sqrt{1+\dfrac{2}{n^2}}-\dfrac{1}{\sqrt n}\right)}
		=\lim\dfrac{\sqrt n\left(\sqrt{4+\dfrac{2}{n^3}-3}\right)}{\sqrt{1+\dfrac{2}{n^2}}-\dfrac{1}{\sqrt n}}$.\\
		Vì $\lim\sqrt n=+\infty$ và $\lim \dfrac{\sqrt{4+\dfrac{2}{n^3}}-3}{\sqrt{1+\dfrac{2}{n^2}}-\dfrac{1}{\sqrt n}}=\dfrac{2-3}{1}=-1$.\\
		Do đó : $\lim \dfrac{\sqrt{4n^4+2n}-3n^2}{\sqrt{n^3+2n}-n}=-\infty$.
	}
\end{bt}
\begin{bt}%[Nguyễn Bình Nguyên]%[1D4B1]
	Tính các giới hạn
	\begin{enumerate}
		\begin{multicols}{2}
			\item $\lim \dfrac{7. 5^n-2. 7^n}{5^n-5. 7^n}$.
			\item $\lim \dfrac{4. 3^n+7^{n+1}}{2. 5^n+7^n}$.
			\item $\lim\dfrac{4^{n+1}+6^{n+2}}{5^n+8^n}$.
		\end{multicols}
	\end{enumerate}
	\loigiai{
		\begin{enumerate}
			\item Ta có : $\lim \dfrac{7. 5^n-2. 7^n}{5^n-5. 7^n}=\lim \dfrac{7. \dfrac{5^n}{7^n}-2}{\dfrac{5^n}{7^n}-5}=\dfrac{2}{5}$.
			\item Tương tự: $\lim \dfrac{4. 3^n+7^{n+1}}{2. 5^n+7^n}=\lim\dfrac{4.\dfrac{3^n}{7^n}+7}{2.\dfrac{5^n}{7^n}+1}=7$.
			\item $\lim\dfrac{4^{n+1}+6^{n+2}}{5^n+8^n}=\lim\dfrac{4.\left(\dfrac{1}{2}\right)^n+36\left(\dfrac{3}{4}\right)^n}{\left(\dfrac{5}{8}\right)^n+1}=0$.
		\end{enumerate}
	}
\end{bt}




\begin{bt}%[Nguyễn Bình Nguyên]%[1D4K1]
	Tính giới hạn của 
	\begin{enumerate}[a)]
		\begin{multicols}{2}
			\item $\lim\dfrac{\sin{10n}+\cos{10n}}{n^2+1}$.
			\item $\lim\dfrac{1-\sin{n\pi}}{n+1}$.
		\end{multicols}
	\end{enumerate}
	\loigiai{
		\begin{enumerate}[a)]
			\item Vì $\bigg|\dfrac{\sin{10n}+\cos{10n}}{n^2+1}\bigg|<\dfrac{\sqrt{2}}{n^2}\ \ $mà $\lim\dfrac{\sqrt{2}}{n^2}=0\Rightarrow \lim\dfrac{\sin{10n}+\cos{10n}}{n^2+1}=0$.
			\item Vì $\bigg|\dfrac{1-\sin{n\pi}}{n+1}\bigg|\leq\dfrac{2}{n}\ \ $mà $\lim\dfrac{2}{n}=0\Rightarrow \lim\dfrac{1-\sin{n\pi}}{n+1}=0$.
	\end{enumerate}}
\end{bt}

\begin{bt}%[Nguyễn Bình Nguyên]%[1D4K1]
	Tính giới hạn của 
	\begin{enumerate}[a)]
		\item $A=\lim \left[\dfrac{1}{1.3}+\dfrac{1}{3.5}+...+\dfrac{1}{(2n-1)(2n+1)}\right]$.
		\item $B=\lim \left[\dfrac{1}{2\sqrt{1}+1\sqrt{2}}+\dfrac{1}{3\sqrt{2}+2\sqrt{3}}+...+\dfrac{1}{(n+1)\sqrt{n}+n\sqrt{n+1}}\right]$.
	\end{enumerate}
	\loigiai{
		\begin{enumerate}
			\item 	$A=\lim \left[\dfrac{1}{1.3}+\dfrac{1}{3.5}+...+\dfrac{1}{(2n-1)(2n+1)}\right]$\\
			$=\lim \left[\left(1-\dfrac{1}{3}\right)+\left(\dfrac{1}{3}-\dfrac{1}{5}\right)+...+\left(\dfrac{1}{2n-1}-\dfrac{1}{2n+1}\right)\right]$\\
			$=\lim \left[1-\dfrac{1}{2n+1}\right]=1$.
			\item $B=\lim \left[\dfrac{1}{2\sqrt{1}+1\sqrt{2}}+\dfrac{1}{3\sqrt{2}+2\sqrt{3}}+...+\dfrac{1}{(n+1)\sqrt{n}+n\sqrt{n+1}}\right]$\\
			$=\lim \left[\left(\dfrac{2\sqrt{1}-1\sqrt{2}}{2.1}\right)+\left(\dfrac{3\sqrt{2}-2\sqrt{3}}{3.2}\right)+...+\left(\dfrac{(n+1)\sqrt{n}-n\sqrt{n+1}}{n(n+1)}\right)\right]$\\
			$=\lim \left[\left(\sqrt{1}-\dfrac{1}{\sqrt{2}}\right)+\left(\dfrac{1}{\sqrt{2}}-\dfrac{1}{\sqrt{3}}\right)+...+\left(\dfrac{1}{\sqrt{n}}-\dfrac{1}{\sqrt{n+1}}\right)\right]\\
			$\\
			$=\lim \left[1-\dfrac{1}{\sqrt{n+1}}\right]=1$.
		\end{enumerate}
	}
\end{bt}






\begin{bt}%[Nguyễn Bình Nguyên]%[1D4G1]
	Cho dãy số $(u_n)$ xác định bởi 
	$$\left\{ \begin{array}{l}
	u_1=\dfrac{2}{3} \\ 
	u_{n+1}=\dfrac{u_n}{2\left(2n+1\right)u_n+1}, \forall n \ge 1 \\ 
	\end{array} \right.$$
	Tìm số hạng tổng quát $u_n$ của dãy. Tính $\lim u_n.$
	\loigiai{
		$u_n \ne 0, \forall n \ge 1$ nên 
		$$u_{n+1}=\dfrac{u_n}{2\left(2n+1\right)u_n+1} \Leftrightarrow \dfrac{1}{u_{n+1}}=2(2n+1)+\dfrac{1}{u_n}.$$
		Đặt $a_n=\dfrac{1}{u_n}$ ta thu được dãy $(a_n)$: $\left\{ \begin{array}{l}
		a_1=\dfrac{3}{2} \\ 
		a_{n+1}=2\left(2n+1\right)+a_n, \forall n \ge 1 \\ 
		\end{array} \right.$\\
		Từ đó ta có
		$$a_{n+1}=2\left(2n+1\right)+a_n = 2\left(2n+1\right)+2\left[2(n-1)+1\right]+a_{n-1}=a_1+4(1+2+...+n)+2n$$
		Suy ra $a_{n+1}=\dfrac{3}{2}+4\cdot \dfrac{n(n+1)}{2}+2n=\dfrac{4n^2+8n+3}{2} \Rightarrow a_n=\dfrac{4n^2-5}{2} \Rightarrow u_n=\dfrac{2}{4n^2-5}.$\\
		Vậy $\lim u_n =\lim \dfrac{2}{4n^2-5}=0$.}
\end{bt}

\begin{bt}%[Nguyễn Bình Nguyên]%[1D4G1]
	Cho dãy số $\left(a_n \right)$ thỏa mãn:
	$$\left\{ \begin{array}{l}
	a_1=\dfrac{4}{3} \\ 
	\dfrac{\left( n+2 \right)^2}{a_{n+1}}=\dfrac{n^2}{a_n}-\left( n+1 \right) \\ 
	\end{array} \right.; \forall n\ge 1,\; n\in \mathbb{N}$$.
	Tìm $\lim a_n$.
	\loigiai{
		Với mỗi $n\in \mathbb{N}^*$, đặt $y_n=\dfrac{1}{a_n}+\dfrac{1}{4}$ ta có $y_1=1$ và
		$$\left( n+2 \right)^2\left(y_{n+1}-\dfrac{1}{4} \right)=n^2\left( y_n-\dfrac{1}{4} \right)-\left( n+1 \right)\,\Rightarrow \,\left( n+2 \right)^2y_{n+1}={{n}^{2}}{{y}_{n}}\,\Rightarrow \,{{y}_{n+1}}=\dfrac{n^2}{\left( n+2 \right)^2}y_n$$
		Do đó $$y_n=\left( \dfrac{n-1}{n+1} \right)^2\left( \dfrac{n-2}{n} \right)^2...\left( \dfrac{1}{3} \right)^2y_1=\dfrac{4}{\left( n+1 \right)^2 n^2}\Rightarrow \,a_n=\dfrac{4n^2\left( n+1 \right)^2}{16-n^2\left( n+1 \right)^2}$$
		Vậy $\lim a_n=-4$.
	}
\end{bt}




\begin{bt}%[Nguyễn Bình Nguyên]%[1D4G1]
	Cho dãy số $(u_n)$ xác định như sau: 
	$ \begin{cases}
	u_1=\dfrac{1}{3}\\ u_{n+1}=\dfrac{u_n^2}{2}-1
	\end{cases}$.
	Tìm $\mathop{\lim} u_n.$
	
	\loigiai{
		Trước hết ta dễ thấy $-1<u_n<0$ với mọi $n\geq 2.$ Ta lại có
		\begin{align*}
		|u_{n+1}-(1-\sqrt{3})|&=\left|\left(\frac{u_n^2}{2}-1\right)-\left(\frac{(1-\sqrt{3})^2}{2}-1\right)\right|\\
		&=\frac{1}{2}|u_n-(1-\sqrt{3})|\cdot |u_n-(1-\sqrt{3})| \\
		&\leq \frac{\sqrt{3}}{2}|u_n-(1-\sqrt{3})|.
		\end{align*}
		Lập luận tương tự như thế ta được $$|u_{n+1}-(1-\sqrt{3})|\leq \left(\frac{\sqrt{3}}{2}\right)^n,\forall n.  $$
		Mà $\mathop{\lim} \left(\dfrac{\sqrt{3}}{2}\right)^n=0$ nên $\mathop{\lim} u_n=1-\sqrt{3}.$
	}
\end{bt}

\begin{bt}%[Nguyễn Bình Nguyên]%[1D4G1]
	Cho dãy số $(u_n)$ xác định như sau: 
	$ \begin{cases}
	u_1=1\\ u_{n+1}=u_n+n
	\end{cases}$.
	Tìm $\mathop{\lim} \dfrac{u_n}{u_{n+1}}.$
	\loigiai{
		Ta có 
		\begin{align*}
		&u_1=u_1+0\\ &u_2=u_1+1\\& u_3=u_2+2\\ &\dotsb\\ &u_n=u_{n-1}+n-1.
		\end{align*}
		Cộng các đẳng thức trên vế theo vế ta được 
		$$ u_n=u_1+1+2+\dotsb+(n-1)=\frac{n^2-n+2}{2} .$$
		Từ đó $\dfrac{u_n}{u_{n+1}}=\dfrac{n^2-n+2}{n^2+n+2}$ nên $\mathop{\lim} \dfrac{u_n}{u_{n+1}}= \mathop{\lim}\dfrac{n^2-n+2}{n^2+n+2} =1.$
	}
\end{bt}
\begin{bt}%[Nguyễn Bình Nguyên]%[1D4G1]
	Cho dãy số $(x_n)$ xác định bởi $\left\{ \begin{array}{l}
	x_1 = 2017\\
	x_{n + 1} = \dfrac{x_n^4 + 3}{4} \text{\;\;với mọi\;\;} n \ge 1
	\end{array} \right.$\\
	Với mỗi số nguyên dương $n$ đặt $y_n = \sum\limits_{i = 1}^n \left(\dfrac{1}{x_i + 1} + \dfrac{2}{x_i^2 + 1} \right) .$\\
	Chứng minh dãy số $(y_n)$ có giới hạn hữu hạn và tìm giới hạn đó.
	
	\loigiai{
		Ta có $x_{n_+1}-1=\dfrac{x_n^4-1}{4}=\dfrac{\left( x_n-1 \right)\left( x_n+1 \right)\left(x_n^2+1 \right) }{4},  \forall n \ge 1.$\\
		Kết hợp $x_1=2017$ ta có $x_n>2017,\forall n \ge 2.$\\
		Ta có $x_{n+1}-x_n=\dfrac{x_n^4-4x_n+3}{4}=\dfrac{\left(x_n-1 \right)^2 \left(x_n^2+2x_n+3 \right)}{4}>0, \forall n \ge 1$.\\
		Suy ra $(x_n)$ là dãy tăng ngặt. Giả sử $(x_n)$ bị chặn trên suy ra $(x_n)$ có giới hạn hữu hạn.\\
		Đặt $\lim x_n=L$  suy ra $L \ge 2017$. Khi đó ta có:
		$$L=\dfrac{L^4+3}{4} \Leftrightarrow L^4-4L+3 =0 \Leftrightarrow \left(L-1 \right)^2\left(L^2+2L+3 \right)=0 \Leftrightarrow L=1,\;\text{vô lý.}$$\\
		Vậy $\lim x_n=+\infty$.\\
		Ta có $\dfrac{x_{n+1}-x_n}{x_{n+1}-1}=\dfrac{\left(x_n-1 \right)\left(x_n^2+2x_n+3 \right)}{\left(x_n+1 \right)\left(x_n^2+1 \right)}, \forall n \ge 1.$\\
		Do đó:
		$$\dfrac{1}{x_n + 1} + \dfrac{2}{x_n^2 + 1}=\dfrac{\left(x_n^2+2x_n+3 \right)}{\left(x_n+1 \right)\left(x_n^2+1 \right)}=\dfrac{x_{n+1}-x_n}{\left(x_{n+1}-1\right)\left(x_n-1\right)}=\dfrac{1}{x_n-1}-\dfrac{1}{x_{n+1}-1}, \forall n \ge 1$$
		Suy ra $$y_n = \sum\limits_{i = 1}^n \left(\dfrac{1}{x_i + 1} + \dfrac{2}{x_i^2 + 1} \right) = \dfrac{1}{2016}-\dfrac{1}{x_{n+1}-1}, \forall n \ge 1.$$
		Do $\lim \dfrac{1}{x_{n+1}-1}=0$ nên dãy $(y_n)$ có giới hạn hữu hạn và $\lim y_n=\dfrac{1}{2016}.$
	}
\end{bt}

\begin{dang}{Dãy số dạng Lũy thừa - Mũ}
	\begin{itemize}
		\begin{multicols}{2}
			\item $\lim n^k=+\infty$, $k>0$.
			\item $\lim \dfrac{1}{n^{k}}=0$, $k>0$.
			\item $\lim a^n=0,-1<a<1$.
			\item $\lim a^n=+\infty,a>1$.
			\item Nếu $\left(u_n\right)$ là CSN lùi vô hạn với công bội $q$, ta có $S=u_1+u_2+\cdots +u_n=\dfrac{u_1}{1-q}$.
		\end{multicols}
	\end{itemize}
\end{dang}
\begin{note}
	\begin{itemize}
		\item $\lim u_n=+\infty, \lim v_{n}=a>0\Rightarrow \lim u_nv_n=+\infty$;
		\item $\lim u_n=+\infty, \lim v_{n}=a<0\Rightarrow \lim u_nv_n=-\infty$;
		\item $\lim u_n=-\infty, \lim v_{n}=a>0\Rightarrow \lim u_nv_n=-\infty$;
		\item $\lim u_n=-\infty, \lim v_{n}=a<0\Rightarrow \lim u_nv_n=+\infty$.
	\end{itemize}
\end{note}
\begin{vd}%[P[han Chiến Thắng]%[1D4Y1]
	Tìm các giới hạn sau
	\begin{enumerate}[a)]
		\begin{multicols}{2}
			\item $\lim (2^n+3^n)$;
			\item $\lim\left[-4^n+(-2)^n\right]$.
		\end{multicols}
		
	\end{enumerate}
	\loigiai{
		\begin{enumerate}[a)]
			\item $\lim (2^n+3^n)=\lim 3^n\left[\left(\dfrac{2}{3}\right)^n+1\right]=+\infty$.
			\item $\lim\left[-4^n+(-2)^n\right]=\lim 4^n\left[-1+\left(\dfrac{-2}{4}\right)^n\right]=-\infty$.
		\end{enumerate}
	}
\end{vd}

\begin{vd}%[P[han Chiến Thắng]%[1D4B1]
	Tìm các giới hạn sau
	\begin{enumerate}[a)]
		\begin{multicols}{3}
			\item $\lim\left(\dfrac{1+3^n}{3\cdot 3^n+2^n}\right)$;
			\item $\lim\left(\dfrac{4\cdot 3^n-2^n}{2\cdot 5^n+4^n}\right)$;
			\item $\lim\left(\dfrac{7^n+1}{-2\cdot 3^n-3\cdot 6^n}\right)$.
		\end{multicols}
	\end{enumerate}
	\loigiai{
		\begin{enumerate}[a)]
			\item $\lim\left(\dfrac{1+3^n}{3\cdot 3^n+2^n}\right)=\lim\left(\dfrac{\dfrac{1}{3^n}+1}{3+\dfrac{2^n}{3^n}}\right)=\dfrac{1}{3}$.
			\item $\lim\left(\dfrac{4\cdot 3^n-2^n}{3\cdot 5^n+4^n}\right)=\lim\left(\dfrac{4\cdot\dfrac{3^n}{5^n}-\dfrac{2^n}{5^n}}{2+\dfrac{4^n}{5^n}}\right)=0$.
			\item $\lim\left(\dfrac{7^n+1}{-2\cdot 3^n-3\cdot 6^n}\right)=\lim\left(\dfrac{1+\dfrac{1}{7^n}}{-2\cdot\dfrac{3^n}{7^n}-3\cdot\dfrac{6^n}{7^n}}\right)=-\infty$.
		\end{enumerate}
	}
\end{vd}

\begin{center}
	\textbf{BÀI TẬP TỰ LUYỆN (Cho mỗi dạng)}
\end{center}
\begin{bt}%[P[han Chiến Thắng]%[1D4B1]
	Tìm các giới hạn sau
	\begin{enumerate}[a)]
		\begin{multicols}{2}
			\item  $\lim\dfrac{2^{3n}+3^{2n+1}}{2\cdot 9^n+4^n}$;
			\item $\lim(2\cdot 3^n-4^{n+1}+7)$.
		\end{multicols}
	\end{enumerate}
	\loigiai{
		\begin{enumerate}[a)]
			\item $\lim\dfrac{2^{3n}+3^{2n+1}}{2\cdot 9^n+4^n}=\lim\dfrac{8^n+3\cdot 9^n}{2\cdot 9^n+4^n}=\lim\left(\dfrac{\dfrac{8^n}{9^n}+3}{2+\dfrac{4^n}{9^n}}\right)=\dfrac{3}{2}$.
			\item \item $\lim(2\cdot 3^n-4^{n+1}+7)=\lim 4^n\left(2\cdot\dfrac{3^n}{4^n}-4+\dfrac{7}{4^n}\right)=-\infty$.
		\end{enumerate}
	}
\end{bt}
\begin{bt}%[P[han Chiến Thắng]%[1D4K1] 
	Tính giới hạn sau $\lim (2\cdot 3^n-n+1)$.
	\loigiai{Ta có: $3^n-n>0$ với $\forall n\in\mathbb{N}$. Do đó, $\lim (2\cdot 3^n-n+1)\geq \lim (3^n+1)=+\infty$. 
		
		Vậy $\lim (2\cdot 3^n-n+1)=+\infty$.}
\end{bt}
\begin{bt}%[P[han Chiến Thắng]%[1D4K1]
	Tìm giới hạn sau $\lim\dfrac{1+\dfrac{1}{3}+\left(\dfrac{1}{3}\right)^2+\cdots +\left(\dfrac{1}{3}\right)^n}{1+\dfrac{2}{5}+\left(\dfrac{2}{5}\right)^2+\cdots +\left(\dfrac{2}{5}\right)^n}$
	\loigiai{
		Đặt $u_n=1+\dfrac{1}{3}+\left(\dfrac{1}{3}\right)^2+\cdots +\left(\dfrac{1}{3}\right)^n$; $v_n=1+\dfrac{2}{5}+\left(\dfrac{2}{5}\right)^2+\cdots +\left(\dfrac{2}{5}\right)^n$.
		
		Ta có: $u_n=1+\dfrac{1}{3}\cdot\dfrac{1-\left(\dfrac{1}{3}\right)^n}{1-\dfrac{1}{3}}=1+\dfrac{1}{2}\left(1-\dfrac{1}{3^n}\right)$. Tương tự, $v_n=1+\dfrac{2}{3}\left(1-\dfrac{2^n}{5^n}\right)$.
		
		Từ đó, $\lim u_n=\dfrac{3}{2}$, $\lim v_n=\dfrac{5}{3}$. Vậy $\lim\dfrac{1+\dfrac{1}{3}+\left(\dfrac{1}{3}\right)^2+\cdots +\left(\dfrac{1}{3}\right)^n}{1+\dfrac{2}{5}+\left(\dfrac{2}{5}\right)^2+\cdots +\left(\dfrac{2}{5}\right)^n}=\dfrac{9}{10}$.
	}
\end{bt}

\begin{bt}%[P[han Chiến Thắng]%[1D4K1]
	Tìm giới hạn sau $\lim\dfrac{1+3+3^2+\cdots +3^n}{2\cdot 3^{n+1}+2^n}$
	\loigiai{Ta có: 
		$\lim\dfrac{1+3+3^2+\cdots +3^n}{2\cdot 3^{n+1}+2^n}=\lim\dfrac{1-\dfrac{3}{2}(1-3^n)}{2\cdot 3^{n+1}+2^n}=\dfrac{1}{4}$
	}
\end{bt}
\begin{bt}%[P[han Chiến Thắng]%[1D4G1]
	Cho dãy số $(u_n)$ xác định bởi $u_1=1$, $u_{n+1}=\dfrac{u_n-4}{u_n+6}, \forall n\geq 1$. Tính giới hạn $\lim\dfrac{u_n+1}{u_n+4}$.
	\loigiai{Đặt $v_n=\dfrac{u_n+1}{u_n+4}$. Ta có: $v_{n+1}=\dfrac{u_{n+1}+1}{u_{n+1}+4}=\dfrac{2(u_n+1)}{5(u_n+4)}=\dfrac{2}{5}v_n=\cdots=\left(\dfrac{2}{5}\right)^{n+1}$. 
		
		Vậy, ta có $v_n=\left(\dfrac{2}{5}\right)^{n}$, do đó $\lim\dfrac{u_n+1}{u_n+4}=\lim v_n=0$.
	}
\end{bt}

\begin{bt}%[P[han Chiến Thắng]%[1D4G1]
	Cho dãy số $(u_n)$ xác định bởi $u_1=3$, $u_{n+1}=\dfrac{u_n+1}{2}, \forall n\geq 1$. Tính giới hạn $\lim u_n$.
	\loigiai{ Ta có: $u_{n+1}-1=\dfrac{u_n-1}{2}=\dfrac{1}{2^2}(u_{n-1}-1)=\cdots =\dfrac{1}{2^{n}}(u_1-1)=\dfrac{1}{2^{n-1}}$. 
		
		Do đó, $u_n=\dfrac{1}{2^{n-2}}+1$. Vậy, $\lim u_n=\lim\left(\dfrac{1}{2^{n-2}}+1\right)=1.$
	}
\end{bt}

\begin{dang}{Giới hạn dãy số chứa căn thức}
	Ta thường gặp hai dạng sau:
	\begin{enumerate}[Dạng 1.]
		\item Sử dụng các tính chất giới hạn để tính.
		\item Dạng vô định, cần nhân lượng liên hợp hoặc thêm bớt hạng tử.
	\end{enumerate}
\end{dang}
\begin{vd}%[Lê Minh Cường][1D4Y1]
	Tìm giới hạn $$\lim \sqrt{\dfrac{8n+2}{2n-1}} $$
	\loigiai{
		Ta có
		$$\lim \sqrt{\dfrac{8n+2}{2n-1}} = \lim \sqrt{\dfrac{8+\dfrac{2}{n}}{2-\dfrac{1}{n}}} = \sqrt{\dfrac{8+0}{2-0}} = 2.  $$
	}
\end{vd}

\begin{vd}%[Lê Minh Cường][1D4Y1]
	Tính giới hạn của dãy số sau: $u_n=\sqrt{\dfrac{2n+9}{n+2}},n\in\mathbb{N^*}$.
	\loigiai{Ta có:$\lim\sqrt{\dfrac{2n+9}{n+2}}=\lim\limits_{n\to+\infty}\sqrt{\dfrac{2+\dfrac{9}{n}}{1+\dfrac{2}{n}}}=\sqrt{\dfrac{2}{1}}=\sqrt{2}$.}
\end{vd}
\begin{vd}%[Lê Minh Cường][1D4B1]
	Tính giới hạn:
	$$\lim \left(\sqrt{4n^2+3n+1}-2n\right)$$
	\loigiai{
		\begin{align*}
		&\lim \left(\sqrt{4n^2+3n+1}-2n\right)
		=\lim \dfrac{4n^2+3n+1-4n^2}{\sqrt{4n^2+3n+1}+2n}\quad (*)\\
		=&\lim \dfrac{3n+1}{\sqrt{4n^2+3n+1}+2n}
		=\lim \dfrac{n\left(3+\dfrac{1}{n}\right)}{\sqrt{n^2\left(4+\dfrac{3}{n}+\dfrac{1}{n^2}\right)}+2n}&\\
		=&\lim \dfrac{n\left(3+\dfrac{1}{n}\right)}{n\left(\sqrt{4+\dfrac{3}{n}+\dfrac{1}{n^2}}+2\right)}
		=\lim \dfrac{3+\dfrac{1}{n}}{\sqrt{4+\dfrac{3}{n}+\dfrac{1}{n^2}}+2}
		=\dfrac{3}{4}.
		\end{align*}
		\textbf{Nhận xét.} 
		\begin{itemize}
			\item Ở bước $(*)$ ta đã \textbf{\textit{nhân biểu thức liên hợp}} của $\left(\sqrt{4n^2+3n+1}-2n\right)$ để \textbf{\textit{khử dạng vô định}} $\infty - \infty$.
			\item Giới hạn $\lim \dfrac{a}{n^k}=0$, với $a=\textrm{const}$ lại một lần nữa được sử dụng.
		\end{itemize} 
	}
\end{vd}
\begin{vd}%[Lê Minh Cường][1D4B1]
	Tính các giới hạn sau
	\begin{enumerate}[a)]
		\item $\lim \dfrac{\sqrt{4n^2+1}+2n-1}{\sqrt{n^2+4n+1}+n}$.
		\item $\lim \dfrac{n^2 +\sqrt[3]{1-n^6}}{\sqrt{n^4+1}+n^2}$.
	\end{enumerate}
	\loigiai{
		\begin{enumerate}[a)]
			\item $\lim  \dfrac{\sqrt{4n^2+1}+2n-1}{\sqrt{n^2+4n+1}+n}=\lim \dfrac{\sqrt{4+\dfrac{1}{n^2}}+2-\dfrac{1}{n}}{\sqrt{1+\dfrac{4}{n}+\dfrac{1}{n^2}}+1}=\dfrac{\sqrt{4}+2}{\sqrt{1}+1}=2$.
			\item $\lim \dfrac{n^2 +\sqrt[3]{1-n^6}}{\sqrt{n^4+1}+n^2}=\lim \dfrac{1+\sqrt[3]{\dfrac{1}{n^6}-1}}{\sqrt{1+\dfrac{1}{n^4}}+1}=\dfrac{1+\sqrt[3]{-1}}{\sqrt{1}+1}=0$.
		\end{enumerate}
	}
\end{vd}
\begin{vd}%[Lê Minh Cường][1D4B1]
	Tính giới hạn:
	$$\lim \dfrac{\sqrt{4n^2+1}-\sqrt{9n^2+2}}{2-n}.$$
	\loigiai{
		\begin{align*}
		&\lim \dfrac{\sqrt{4n^2+1}-\sqrt{9n^2+2}}{2-n}
		=\lim \dfrac{\sqrt{n^2\left(4+\dfrac{1}{n^2}\right)}-\sqrt{n^2\left(9+\dfrac{2}{n^2}\right)}}{n\left(\dfrac{2}{n}-1\right)}\\
		=&\lim \dfrac{n\left(\sqrt{4+\dfrac{1}{n^2}}-\sqrt{9+\dfrac{2}{n^2}}\right)}{n\left(\dfrac{2}{n}-1\right)}
		=\lim \dfrac{\sqrt{4+\dfrac{1}{n^2}}-\sqrt{9+\dfrac{2}{n^2}}}{\dfrac{2}{n}-1}
		=1.
		\end{align*}
		\textbf{Nhận xét.} 
		\begin{itemize}
			\item Trong ví dụ này, ta đã \textbf{\textit{rút $n^k$}} (ở cả tử và mẫu) làm nhân tử chung với \textbf{\textit{$k$ là bậc cao nhất của $n$ ở tử số và mẫu số}}.
			\item Cần chú ý giới hạn quan trọng $\lim \dfrac{a}{n^k}=0$, với $a=\textrm{const}$.
		\end{itemize} 
	}
\end{vd}


\begin{vd}%[Lê Minh Cường][1D4B1]
	Tính giới hạn:
	$$\lim \left(\sqrt{n+3}-\sqrt{n-5}\right)n$$
	\loigiai{
		\begin{align*}
		&\lim \left(\sqrt{n+3}-\sqrt{n-5}\right)n\\
		=&\lim \dfrac{(n+3-n+5)n}{\sqrt{n+3}+\sqrt{n-5}}\\
		=&\lim \dfrac{8n}{\sqrt{n}\left(\sqrt{1+\dfrac{3}{n}}+\sqrt{1-\dfrac{5}{n}}\right)}\\
		=&\lim \sqrt{n}\dfrac{8}{\sqrt{1+\dfrac{3}{n}}+\sqrt{1-\dfrac{5}{n}}}\\
		=&+\infty. \left(\text{vì }  \lim \sqrt{n}=+\infty  \text{ và } \lim \dfrac{8}{\sqrt{1+\dfrac{3}{n}}+\sqrt{1-\dfrac{5}{n}}}=\dfrac{8}{2}=4=\textrm{const} \right).
		\end{align*}
		\textbf{Nhận xét.} Cần chú ý giới hạn sau:
		\begin{center}
			Nếu $\left\{\begin{array}{l}
			u_n \longrightarrow +\infty \\
			v_n \longrightarrow c=\textrm{const} \neq 0
			\end{array}\right.$ thì $\lim {u_n}.{v_n}=\left\{\begin{array}{cl}
			+\infty & \text{(nếu $ c>0 $)}\\
			-\infty & \text{(nếu $ c<0 $)}
			\end{array}\right.$.
		\end{center}
	}
\end{vd}



\begin{center}
	\textbf{BÀI TẬP TỰ LUYỆN (Cho mỗi dạng)}
\end{center}
\begin{bt}%[Lê Minh Cường][1D4Y1]
	Tính giới hạn của các dãy số sau:
	\begin{enumerate}
		\item [a)]
		$u_n=\sqrt{n^2+1},n\in\mathbb{N^*}$;
		\item [b)] $v_n=\sqrt{\dfrac{n^2+2n+4}{2n-3}},n\ge 2$.
	\end{enumerate}
	\loigiai{\begin{enumerate} \item [a)] Ta có: $\lim\sqrt{n^2+1}=\lim\sqrt{n^2(1+\dfrac{1}{n^2})}$;\\
			Vì $\begin{cases}
			\lim\sqrt{n^2}=+\infty\\
			\lim\sqrt{1+\dfrac{1}{n^2}}=1;
			\end{cases}$
			$\Rightarrow \lim\sqrt{n^2(1+\dfrac{1}{n^2})}=+\infty$.,\\
			Vậy $\lim u_n=+\infty.$
			
			\item [b)] Ta có:
			$\lim\sqrt{\dfrac{n^2+2n+4}{2n-3}}=\lim\sqrt{\dfrac{1+\dfrac{2}{n}+\dfrac{4}{n^2}}{\dfrac{2}{n}-\dfrac{3}{n^2}}}$
			\\Vì $\begin{cases}
			\lim\sqrt{1+\dfrac{2}{n}+\dfrac{4}{n^2}}=1\\
			\lim\sqrt{\dfrac{2}{n}-\dfrac{3}{n^2}}=0;
			\end{cases}
			\Rightarrow \lim\sqrt{\dfrac{n^2+2n+4}{2n-3}}=+\infty$.\\
			Vậy $\lim v_n=+\infty.$
			
		\end{enumerate}
	}
\end{bt}



\begin{bt}%[Lê Minh Cường][1D4Y1]
	Tính giới hạn:
	$$\lim \left(\sqrt{3}n-\sqrt{3n^2-2n-1}\right)$$
	\loigiai{
		\begin{align*}
		&\lim \left(\sqrt{3}n-\sqrt{3n^2-2n-1}\right)\\
		=&\lim \dfrac{3n^2-3n^2+2n+1}{\sqrt{3}n+\sqrt{3n^2-2n-1}}\\
		=&\lim \dfrac{2n+1}{\sqrt{3}n+\sqrt{3n^2-2n-1}}\\
		=&\lim \dfrac{n\left(2+\dfrac{1}{n}\right)}{n\left(\sqrt{3}+\sqrt{3-\dfrac{2}{n}-\dfrac{1}{n^2}}\right)}\\
		=&\lim \dfrac{2+\dfrac{1}{n}}{\sqrt{3}+\sqrt{3-\dfrac{2}{n}-\dfrac{1}{n^2}}}\\
		=&\dfrac{1}{\sqrt{3}}.
		\end{align*}
	}
\end{bt}

\begin{bt}%[Lê Minh Cường][1D4B1]
	Tìm giới hạn $$\lim \left( \sqrt{n^2 + 2n} - n \right) $$
	\loigiai{
		Ta có
		\begin{align*}
		\lim \left( \sqrt{n^2 + 2n} - n \right) &= \lim \dfrac{\left( \sqrt{n^2 + 2n} - n \right) \left( \sqrt{n^2 + 2n} + n \right)}{\left( \sqrt{n^2 + 2n} + n \right)} = \lim \dfrac{(n^2 + 2n) - n^2}{\left( \sqrt{n^2 + 2n} + n \right)} \\
		&=\lim \dfrac{2n}{n \left( \sqrt{1 + \dfrac{2}{n}} + 1 \right)} = \lim \dfrac{2}{\sqrt{1 + \dfrac{2}{n}} + 1} \\
		&= \dfrac{2}{\sqrt{1 - 0} + 1} = 1
		\end{align*}
	}
\end{bt}

\begin{bt}%[Lê Minh Cường][1D4B1]
	Tìm giới hạn $$\lim \left( \sqrt{n^3 + 2n} - n^2 \right) $$
	\loigiai{
		Ta có
		$$ \lim \left( \sqrt{n^3 + 2n} - n^2 \right) = \lim \left[ n^2 \left( \sqrt{\dfrac{1}{n} + \dfrac{2}{n^3}} - 1 \right) \right] $$
		Mà $\lim n^2 = +\infty $, $\lim \left( \sqrt{\dfrac{1}{n} + \dfrac{2}{n^3}} - 1 \right) = ( \sqrt{0 + 0} - 1) = -1 < 0$ nên
		$$ \lim \left[ n^2 \left( \sqrt{\dfrac{1}{n} + \dfrac{2}{n^3}} - 1 \right) \right] = -\infty $$
		Vậy $\lim \left( \sqrt{n^3 + 2n} - n^2 \right) = -\infty$.
	}
\end{bt}

\begin{bt}%[Lê Minh Cường][1D4B1]Tìm giới hạn
	$$\lim(\sqrt{n^2+3n+2}-n+1)$$	
	\loigiai{$\lim(\sqrt{n^2+3n+2}-(n-1))=\lim\dfrac{(\sqrt{n^2+3n+2}-(n-1))(\sqrt{n^2+3n+2}+n-1)}{\sqrt{n^2+3n+2}+n-1}\\=\lim\dfrac{(\sqrt{n^2+3n+2})^2-(n-1)^2}{\sqrt{n^2+3n+2}+n-1}=\lim\dfrac{5n+1}{\sqrt{n^2+3n+2}+n-1}\\=\lim\dfrac{5+\dfrac{1}{n}}{\sqrt{1+\dfrac{3}{n}+\dfrac{2}{n^2}}+1-\dfrac{1}{n}}=\dfrac{5}{2}$.}
\end{bt}



\begin{bt}%[Lê Minh Cường][1D4B1] Tìm giới hạn
	$$\lim(\sqrt{n^2+2n+3}-n)$$
	\loigiai{$\lim(\sqrt{n^2+2n+3}-n)=\lim\dfrac{(\sqrt{n^2+2n+3}-n)(\sqrt{n^2+2n+3}+n)}{\sqrt{n^2+2n+3}+n}\\=\lim\dfrac{2n+3}{\sqrt{n^2+2n+3}+n}=\lim\dfrac{2+\dfrac{3}{n}}{\sqrt{1+\dfrac{2}{n}+\dfrac{3}{n^2}}+1}=1$.	}
	
\end{bt}

\begin{bt}%[Lê Minh Cường][1D4B1]Tìm giới hạn
	$$\lim\dfrac{1}{\sqrt{n+1}-\sqrt{n+3}}$$
	\loigiai{	$\lim\dfrac{1}{\sqrt{n+1}-\sqrt{n+3}}=\lim\dfrac{\sqrt{n+1}+\sqrt{n+3}}{(\sqrt{n+1}-\sqrt{n+3})(\sqrt{n+1}+\sqrt{n+3})}\\
		=\lim\dfrac{\sqrt{n+1}+\sqrt{n+3}}{-2}=-\infty$.}
\end{bt}

\begin{bt}%[Lê Minh Cường][1D4B1] Tìm giới hạn
	$$\lim(\sqrt{n^2+3n-1}-\sqrt{n+1})$$
	\loigiai{$\lim(\sqrt{n^2+3n-1}-\sqrt{n+1})=\lim\dfrac{(\sqrt{n^2+3n-1}-\sqrt{n+1})(\sqrt{n^2+3n-1}+\sqrt{n+1})}{\sqrt{n^2+3n-1}+\sqrt{n+1}}\\
		=\lim\dfrac{n^2+2n-2}{\sqrt{n^2+3n-1}+\sqrt{n+1}}=\lim \dfrac{n\left(1+\dfrac{2}{n}-\dfrac{2}{n^2}\right)}{\sqrt{1+\dfrac{3}{n}-\dfrac{1}{n^2}}+\sqrt{1+\dfrac{1}{n}}}=+\infty$.}
\end{bt}

\begin{bt}%[Lê Minh Cường][1D4G1]
	Tìm giới hạn của dãy $(u_n)$, với
	$$\heva{&u_1 = 1 \\&u_{n+1} = \sqrt{u_n^3 + 2}} $$
	\loigiai{
		Ta chứng minh bằng quy nạp rằng $u_n \ge \sqrt{n}, \forall n \in \mathbb{N}^*$ (*)\\
		Rõ ràng (*) đúng khi $n = 1$.\\
		Giả sử (*) đúng khi $n = k, k \in \mathbb{N}^*$, tức là $u_k \ge \sqrt{k}$ \\
		Khi đó ta có
		\begin{align*}
		u_{k+1} &= \sqrt{u_k^3 + 2} = \sqrt{u_k^2.u_k + 2} \ge \sqrt{u_k^2.\sqrt{k} + 2}
		&> \sqrt{u_k^2.1 + 1} = \sqrt{u_k^2 + 1} \ge \sqrt{(\sqrt{k})^2 + 1} = \sqrt{k+1}
		\end{align*}
		Theo nguyên lý quy nạp, ta có điều phải chứng minh. \\
		
		Trở lại bài toán. Lấy $M > 0$ tùy ý. Khi đó có số $m \in \mathbb{N}^*$ sao cho $m > M$. \\ Hơn nữa, từ (*) ta có $$\forall k \in \mathbb{N}, k > m^2: u_k \ge \sqrt{k} > \sqrt{m^2} = m > M$$
		Như vậy, các số hạng của dãy $u_n$ kể từ số hạng thứ $m^2 + 1$ trở đi đều lớn hơn $M$. Do đó $\lim u_n = +\infty$.}
	
\end{bt}

\begin{bt}%[Lê Minh Cường][1D4K1]
	Tính $\lim \dfrac{\sqrt{n^2+2}-\sqrt{n+5}}{3n+3}.$
	\loigiai{
		$\lim \dfrac{\sqrt{n^2+2}-\sqrt{n+5}}{3n+3}=\lim \dfrac{\sqrt{n^2\left(1+\dfrac{2}{n^2}\right)}-\sqrt{n^2\left(\dfrac{1}{n}+\dfrac{5}{n^2}\right)}}{n\left(3+\dfrac{3}{n}\right)}=\lim \dfrac{n\sqrt{1+\dfrac{2}{n^2}}-n\sqrt{\dfrac{1}{n}+\dfrac{5}{n^2}}}{n\left(3+\dfrac{3}{n}\right)}=\newline \lim \dfrac{\sqrt{1+\dfrac{2}{n^2}}-\sqrt{\dfrac{1}{n}+\dfrac{5}{n^2}}}{\left(3+\dfrac{3}{n}\right)}=\dfrac{1}{3}.$}	     
\end{bt}


\begin{bt}%[Lê Minh Cường][1D4K1]
	Tính giới hạn của dãy số sau $u_n=\dfrac{\sqrt{n^2+1}-\sqrt{2n^2+4n-4}}{3n+15},n\in\mathbb{N^*}.$
	\loigiai{\begin{align*}
		\text{Ta có}: \lim u_n&=\lim\dfrac{\sqrt{n^2+1}-\sqrt{2n^2+4n-4}}{3n+15}\\
		&=\lim\dfrac{(n^2+1)-(2n^2+4n-4)}{3(n+5)(\sqrt{n^2+1}+\sqrt{2n^2+4n-4})}\\
		&=\lim\dfrac{(n+5)(1-n)}{3(n+5)(\sqrt{n^2+1}+\sqrt{2n^2+4n-4})}\\
		&=\lim\dfrac{1-n}{3(\sqrt{n^2+1}+\sqrt{2n^2+4n-4})}\\
		&=\lim\dfrac{\dfrac{1}{n}-1}{3(\sqrt{1+\dfrac{1}{n^2}}+\sqrt{2+\dfrac{4}{n}-\dfrac{4}{n^2}})}\\
		&=\dfrac{-1}{3(\sqrt{1}+\sqrt{2})} = \dfrac{1-\sqrt{2}}{3}
		\end{align*}\\
		Vậy $\lim u_n		=\dfrac{1-\sqrt{2}}{3}.$
	}
\end{bt}
\begin{bt}%[Lê Minh Cường][1D4K1]
	Tính giới hạn của dãy số $(u_n)$ với $u_n=(\sqrt{n^2-n+2}-n).$
	\loigiai{
		$\lim u_n= \lim (\sqrt{n^2-n+2}-n)=\lim \dfrac{n^2-n+2-n^2}{\sqrt{n^2-n}+n}=\lim \dfrac{-n+2}{\sqrt{n^2-n}+n}=\lim \dfrac{n\left(-1+\frac{2}{n}\right)}{\sqrt{n^2\left(1-\frac{1}{n}\right)}+n}=\newline 
		\lim \dfrac{n\left(-1+\frac{2}{n}\right)}{n\sqrt{1-\frac{1}{n}}+n}=\lim \dfrac{-1+\frac{2}{n}}{\sqrt{1-\frac{1}{n}}+1}=-\dfrac{1}{2}.$}
\end{bt}

\begin{bt}%[Lê Minh Cường][1D4K1]
	Tính $\lim \dfrac{\sqrt{n^3+3n^2-2n+1}}{n-1}.$
	\loigiai{
		$\lim \dfrac{\sqrt{n^3+3n^2-2n+1}}{n-1}=\lim \dfrac{\sqrt{n^2\left(n+3-\dfrac{2}{n}+\dfrac{1}{n^2}\right)}}{n-1}= \lim \dfrac{n\sqrt{n+3-\dfrac{2}{n}+\dfrac{1}{n^2}}}{n\left(1-\dfrac{1}{n}\right)}=\newline \lim \dfrac{\sqrt{n+3-\dfrac{2}{n}+\dfrac{1}{n^2}}}{\left(1-\dfrac{1}{n}\right)}=+\infty.$}
\end{bt}	


\begin{bt}%[Lê Minh Cường][1D4K1]
	Tính các giới hạn sau
	\begin{enumerate}[a)]
		\item $\lim \left(\sqrt{n^2+2n}-n-1\right)$.
		\item $\lim \dfrac{\sqrt{4n^2+1}-2n-1}{\sqrt{n^2+4n+1}-n}$.
	\end{enumerate}
	\loigiai{
		\begin{enumerate}[a)]
			\item \begin{align*}
			\lim \left(\sqrt{n^2+2n}-n-1\right)&=\lim\dfrac{\left(\sqrt{n^2+2n}-(n+1)\right)\left(\sqrt{n^2+2n}+(n+1)\right)}{\sqrt{n^2+2n}+n+1}\\
			&=\lim \dfrac{-1}{\sqrt{n^2+2n}+n+1}=0.
			\end{align*}
			\item \begin{align*}
			\lim \dfrac{\sqrt{4n^2+1}-2n-1}{\sqrt{n^2+4n+1}-n}&=\lim \dfrac{\left(\sqrt{4n^2+1}-(2n+1)\right)\left(\sqrt{4n^2+1}+2n+1\right)\left(\sqrt{n^2+4n+1}+n\right)}{\left(\sqrt{n^2+4n+1}-n\right)\left(\sqrt{n^2+4n+1}+n\right)\left(\sqrt{4n^2+1}+2n+1\right)}\\
			&=\lim \dfrac{-4n\left(\sqrt{n^2+4n+1}+n\right)}{\left(4n+1\right)\left(\sqrt{4n^2+1}+2n+1\right)}\\
			&=\lim \dfrac{-4\left(\sqrt{1+\dfrac{4}{n}+\dfrac{1}{n^2}}+1\right)}{\left(4+\dfrac{1}{n}\right)\left(\sqrt{4+\dfrac{1}{n^2}}+2+\dfrac{1}{n}\right)}\\
			&=-\dfrac{4\left(\sqrt{1}+1\right)}{4\left(\sqrt{4}+2\right)}=-\dfrac{1}{2}.
			\end{align*}
		\end{enumerate}
	}
\end{bt}
\begin{bt}%[Lê Minh Cường][1D4K1]
	Tính giới hạn $\lim(\sqrt{n^2+2n+3}-1+n)$.
	\loigiai{
		\begin{align*}
		\lim\left(\sqrt{n^2+2n+3}-1+n\right)=\lim\left[\sqrt{n^2+2n+3}-(1-n)\right]&=\lim \dfrac{n^2+2n+3-(1-n)^2}{\sqrt{n^2+2n+3}+n-1}\\
		&=\lim \dfrac{4n+2}{\sqrt{n^2+2n+3}+n-1}\\
		&=\lim \dfrac{4+\dfrac{2}{n}}{\sqrt{1+\dfrac{2}{n}+\dfrac{3}{n}}+1-\dfrac{1}{n}}=2.
		\end{align*}}
\end{bt}
\begin{bt}%%[Lê Minh Cương]%[1D4K1]
	Tính giới hạn $\lim\sqrt[n]{a}$ với $a>0$.
	\loigiai{
		Giả sử $a>1$. Khi đó $a=\left[1+\left(\sqrt[n]{a}-1\right)\right]^n>n\left(\sqrt[n]{a}\right) $.
		\newline Suy ra $0<\sqrt[n]{a}-1<\dfrac{a}{n}\rightarrow 0$ nên $\lim \sqrt[n]{a}=1$.
		\newline Với $0<a<1$ thì $\dfrac{1}{a}>1\Rightarrow \lim \sqrt[n]{\dfrac{1}{a}}=1\Rightarrow \lim \sqrt[n]{a}=1$\\
		Tóm lại ta luôn có : $\lim\sqrt[n]{a}=1$ với $a>0$.
		
	}
\end{bt}
\begin{bt}%[Lê Minh Cường][1D4G1]
	Tính giới hạn $$\lim(\sqrt[3]{n^3-3}-\sqrt{n^2+n-2})$$.
	\loigiai{
		\begin{align*}
		\lim \left(\sqrt[3]{n^3-3}-\sqrt{n^2+n-2}\right)&= \lim\left[\left(\sqrt[3]{n^3-3}-n\right)+\left( n-\sqrt{n^2+n-2}\right)\right]\\
		&=\lim \left[\dfrac{\left(\sqrt[3]{n^3-3}-n\right)\left(\sqrt[3]{(n^3-3)^2}+n\sqrt[3]{n^3-3}+n^2\right)}{\sqrt[3]{(n^3-3)^2}+n\sqrt[3]{n^3-3}+n^2}\right.\\
		&\hspace*{60pt}\left.+\dfrac{\left(n-\sqrt{n^2+n-2}\right)\left(n+\sqrt{n^2+n-2}\right)}{n+\sqrt{n^2+n-2}}\right]\\
		&=\lim \left[\dfrac{-3}{\sqrt[3]{(n^3-3)^2}+n\sqrt[3]{n^3-3}+n^2}+ \dfrac{2-n}{n+\sqrt{n^2+n-2}}\right]\\
		&=\lim \left[\dfrac{\dfrac{-3}{n^2}}{\sqrt[3]{\left(1-\dfrac{3}{n^3}\right)^2}+\sqrt[3]{1-\dfrac{3}{n^3}}+1}+\dfrac{\dfrac{2}{n}-1}{1+\sqrt{1+\dfrac{1}{n}-\dfrac{2}{n^2}}}\right]\\
		&=0-\dfrac{1}{2}=-\dfrac{1}{2}.
		\end{align*}}
\end{bt}
\begin{bt}%[Lê Minh Cường][1D4G1]
	Tìm $\lim u_n$ biết $u_n=\dfrac{1}{2\sqrt{1}+1\sqrt{2}}
	+\dfrac{1}{3\sqrt{2}+2\sqrt{3}}
	+\ldots
	+\dfrac{1}{(n+1)\sqrt{n}+n\sqrt{n+1}}.
	$
	\loigiai{
		Ta có $\dfrac{1}{(k+1)\sqrt{k}+k\sqrt{k+1}}=\dfrac{\sqrt{k+1}-\sqrt{k}}{\sqrt{k(k+1)}}=\dfrac{1}{\sqrt{k}}-\dfrac{1}{\sqrt{k+1}}$.\\
		Suy ra $u_n=\dfrac{1}{\sqrt{1}}-\dfrac{1}{\sqrt{2}}+\dfrac{1}{\sqrt{2}}-\dfrac{1}{\sqrt{3}}+\ldots+\dfrac{1}{\sqrt{n}}-\dfrac{1}{\sqrt{n+1}}=\dfrac{1}{\sqrt{1}}-\dfrac{1}{\sqrt{n+1}}$ từ đó ta có $\lim u_n=1$.
	}
\end{bt}

\begin{bt}%%[Lê Minh Cương]%[1D4G1]
	Tính giới hạn $\lim \left(\dfrac{1}{\sqrt{n^2+n}}+\dfrac{1}{\sqrt{n^2+n+1}}+\ldots+\dfrac{1}{\sqrt{n^2+2n}}\right)$.
	\loigiai{
		Sử dụng đánh giá $1<\dfrac{1}{\sqrt{n^2+n}}+\dfrac{1}{\sqrt{n^2+n+1}}+\ldots+\dfrac{1}{\sqrt{n^2+2n}}<\dfrac{n+1}{\sqrt{n^2+n}}$ và 
		$\lim \dfrac{n+1}{\sqrt{n^2+n}}=1$.
		
		Ta được $\lim \left(\dfrac{1}{\sqrt{n^2+n}}+\dfrac{1}{\sqrt{n^2+n+1}}+\ldots+\dfrac{1}{\sqrt{n^2+2n}}\right)=1$
	}
\end{bt}


\begin{bt}%[Lê Minh Cường][1D4G1]
	Cho dãy số $u_n$ thỏa:\\
	$\begin{cases}
	u_1=3,u_2=6\\
	2u_n=u_{n-1}+u_{n+1}-2;
	\end{cases}
	\forall n\in\mathbb{N^*},n\geq 3.$\\
	Biết rằng $u_n$ có duy nhất một công thức, tính:$\lim\limits_{n\to+\infty}\dfrac{n+2-\sqrt{u_n}}{n+1-\sqrt{u_n+3n-2}}$.
	\loigiai{
		Dựa vào biểu thức $u_n$ ta tính:
		\begin{align*}
		&u_1=3=1+2=1^2+2;\\
		&u_2=6=4+2=2^2+2;\\
		&u_3=11=9+2=3^2+2;\\
		&...\\
		&u_n=n^2+2;\\
		&...
		\end{align*}
		\\Ta dự đoán công thức $u_n=n^2+2$, thật vậy:\\
		$\begin{cases}
		2u_n=2n^2+4\\
		u_{n-1}+u_{n+1}-2=[(n-1)^2+2]+[(n+1)^2+2]-2=2n^2+4;
		\end{cases}$
		\\Suy ra $u_n=n^2+2,n\in\mathbb{N^*},n\geq 3$;\\
		Ta có:
		\begin{align*}
		\lim\limits_{n\to+\infty}\dfrac{n+2-\sqrt{n^2+2}}{n+1-\sqrt{n^2+3n}}
		&=\lim\limits_{n\to+\infty}\dfrac{[(n+2)^2-(n^2+2)](n+1+\sqrt{n^2+3n})}{[(n+1)^2-(n^2+3n)](n+2+\sqrt{n^2+2})}\\
		&=\lim\limits_{n\to+\infty}\dfrac{(4n+2)(n+1+\sqrt{n^2+3n})}{(-n+1)(n+2+\sqrt{n^2+2})}\\
		&=-4.
		\end{align*}
		\\Vậy $\lim\limits_{n\to+\infty} u_n=-4$.
	}
\end{bt}
\begin{bt}%[Lê Minh Cường][1D4G1]
	Tính giới hạn $L=\lim\limits_{n\to \infty}\left(\dfrac{1-2n}{\sqrt{n^2+1}} \right). $
	\loigiai{Với $a$ nhỏ tùy ý, ta chọn $n_{a}>\sqrt{\dfrac{9}{a^2}-1}$, ta có: 
		$\left|\dfrac{1-2n}{\sqrt{n^2+1}}+2 \right|=\left|\dfrac{1-2n+2\sqrt{n^2+1}}{\sqrt{n^2+1}} \right| $
		\newline $<\left|\dfrac{1-2n+2(n+1)}{\sqrt{n^2+1}} \right|=\dfrac{3}{\sqrt{n^2+1}}<\dfrac{3}{\sqrt{{n_a}^2+1}} <a. $
		\newline Suy ra $\lim \left|\dfrac{1-2n}{\sqrt{n^2+1}}+2\right| =0\Rightarrow \lim\limits_{n\to \infty}\left(\dfrac{1-2n}{\sqrt{n^2+1}} \right)=-2. $		
	}
\end{bt}

\begin{bt}%[Lê Minh Cường][1D4G1]
	Tính giới hạn của $B=\lim \dfrac{\sqrt{1+2+...+n}-n}{\sqrt[3]{1^2+2^2+...+n^2}+2n}$.
	\loigiai{
		Việc đầu tiên ta phải tính tổng của hai dãy số dưới dấu căn
		\newline $1+2+3+...+n=\dfrac{n(n+1)}{2}$.
		\newline $1^2+2^2+...+n^2=\dfrac{n(n+1)(2n+1)}{6}$.
		\newline Lúc này: $B=\lim\dfrac{\sqrt{\dfrac{n(n+1)}{2}}-n}{\sqrt[3]{\dfrac{n(n+1)(2n+1)}{6}}+2n} =\lim\dfrac{n\sqrt{\dfrac{1}{2}+\dfrac{1}{2n}}-n}{n\sqrt[3]{\dfrac{1}{3}+\dfrac{1}{2n}+\dfrac{1}{6n^2}}+2n}=\dfrac{\dfrac{1}{\sqrt{2}}-1}{\sqrt[3]{\dfrac{1}{3}}+2}=\dfrac{(1-\sqrt{2})\sqrt[3]{3}}{\sqrt{2}(1+2\sqrt[3]{3})}. $
	}
\end{bt}