\begin{dang}{Phương pháp đặt thừa số chung - kết quả vô cực}
	Để tìm giới hạn của hàm số ta cần nhớ
	\begin{itemize}
		\item $\lim\limits_{x\to +\infty} x^k=+\infty$; $\lim\limits_{x\to -\infty} x^k=\heva{& +\infty,k=2n\\& -\infty,k=2n+1.}$
		\item $\lim\limits_{x\to \pm \infty}c=c$; $\lim\limits_{x\to \pm \infty} \dfrac {c}{x^k}=0$; $\lim\limits_{x\to 0} \dfrac {1}{x}=\infty$.
	\end{itemize}
\end{dang}
\subsubsection{Ví dụ minh hoạ}
%VD1
\begin{vd}%[NB]%[DCHT Toán 11 - KNTT -Nguyễn Văn Hiệp]%[1K5YF-4]
	Tính $\lim\limits_{x\to +\infty} x^3$.\dapso{$+\infty$.}
	\loigiai{Ta có $\lim\limits_{x\to +\infty} x^3=+\infty$.}
\end{vd}
%VD2
\begin{vd}%[TH]%[DCHT Toán 11 - KNTT -Nguyễn Văn Hiệp]%[1K5BF-4]
	Tính $\lim\limits_{x\to -\infty }\left(x^3+3x+1\right)$.
	\dapso{$-\infty$}
	\loigiai{Ta có $\lim\limits_{x\to -\infty } \left( x^3+3x+1 \right)=\lim\limits_{x\to -\infty } \left[ x^3\left( 1+\dfrac {3}{x^2}+\dfrac {1}{x^3} \right) \right]=-\infty $.\\
		Vì $\lim\limits_{x\to -\infty } x^3=-\infty$; $\lim\limits_{x\to -\infty } \left( 1+\dfrac {3}{x^2}+\dfrac {1}{x^3} \right)=1>0$.}
\end{vd}
%VD3
\begin{vd}%[TH]%[DCHT Toán 11 - KNTT -Nguyễn Văn Hiệp]%[1K5BF-4]
	Tính $\lim\limits_{x\to -\infty}\left(-4x^5-3x^3+x+1\right)$.
	\dapso{$+\infty$.}
	\loigiai{
		Ta có $\lim\limits_{x\to -\infty } \left( -4x^5-3x^3+x+1 \right)=\lim\limits_{x\to -\infty } x^5\left( -4-\dfrac {3}{x^2}+\dfrac {1}{x^4}+\dfrac {1}{x^5} \right)=+\infty $.\\
		Vì $\heva{&\lim\limits_{x\to -\infty } \left( -4-\dfrac {3}{x^2}+\dfrac {1}{x^4}+\dfrac {1}{x^5} \right)=-4<0 \\& \lim\limits_{x\to -\infty } x^5=-\infty.}$
	}
\end{vd}
%VD4
\begin{vd}%[TH]%[DCHT Toán 11 - KNTT -Nguyễn Văn Hiệp]%[1K5BF-4]
	Tính giới hạn $\lim\limits_{x\to -3} \dfrac {x+2}{(x+3)^2}$. 
	\dapso{$-\infty$.}
	\loigiai{Ta có $\lim\limits_{x\to -3} \dfrac {x+2}{(x+3)^2}=-\infty $.\\
		Vì $\lim\limits_{x\to -3} (x+2)=-3+2=-1<0$, $\lim\limits_{x\to -3} (x+3)^2=0$ và $(x+3)^2>0$ khi $ x\ne -3$.}
\end{vd}
%VD5
\begin{vd}[VDT]%[DCHT Toán 11 - KNTT -Nguyễn Văn Hiệp]%[1K5KF-4]
	Tìm tất cả các giá trị nguyên của tham số $m$ để $I=\lim\limits_{x\to +\infty}\left[(m^2-1)x^3+2x\right]=-\infty$.
	\dapso{$m=0$}
	\loigiai{
		Ta có $\lim\limits_{x\to +\infty } \left[ \left(m^2-1 \right)x^3+2x \right]=\lim\limits_{x\to +\infty} x^3\left[ m^2-1+\dfrac {2}{x^2} \right]$.\\
		Vì $\lim\limits_{x\to +\infty} x^3=+\infty $ nên $I=-\infty \Leftrightarrow \lim\limits_{x\to +\infty } \left[ m^2-1+\dfrac {2}{x^2} \right]<0\Leftrightarrow m^2-1<0\Leftrightarrow -1<m<1$.\\
		Do $ m\in \mathbb{Z}$ nên $m=0$.}
\end{vd}
\subsubsection{Bài tập rèn luyện}
\centerline{\fcolorbox{red}{yellow!50}{\bf {BÀI TẬP TỰ LUẬN}}}
%bt1
\begin{bt}%[NB]%[DCHT Toán 11 - KNTT -Nguyễn Văn Hiệp]%[1K5YF-4]
	Tính $\lim\limits_{x\to -\infty} x^2$.\dapso{$+\infty$.}
	\loigiai{Ta có $\lim\limits_{x\to -\infty} x^2=+\infty$.}
\end{bt}
%bt2
\begin{bt}%[NB]%[DCHT Toán 11 - KNTT -Nguyễn Văn Hiệp]%[1K5YF-4]
	Tính $\lim\limits_{x\to -\infty} \left(-x^4-\dfrac{1}{x}\right)$.\dapso{$-\infty$.}
	\loigiai{Ta có $\lim\limits_{x\to -\infty} -x^4=-\infty$ và $\lim\limits_{x\to -\infty} \dfrac{1}{x}=0$. Suy ra $\lim\limits_{x\to -\infty} \left(-x^4-\dfrac{1}{x}\right)=-\infty$.}
\end{bt}
%bt3
\begin{bt}%[TH]%[DCHT Toán 11 - KNTT -Nguyễn Văn Hiệp]%[1K5BF-4]
	Tính giới hạn $\lim\limits_{x\to +\infty }\left(-x^3+5x^2+2x+1\right)$.
	\dapso{$-\infty$.}
	\loigiai{
		Ta có$\lim\limits_{x\to +\infty} \left(-x^3+5x^2+2x+1 \right)=\lim\limits_{x\to +\infty } \left[ x^3\left(-1+\dfrac {5}{x}+\dfrac {2}{x^2}+\dfrac {1}{x^3} \right) \right]$.\\
		Do $\lim\limits_{x\to +\infty}x^3=+\infty$; $\lim\limits_{x\to +\infty}\left(-1+\dfrac {5}{x}+\dfrac {2}{x^2}+\dfrac {1}{x^3} \right)=-1<0$ nên $\lim\limits_{x\to +\infty} \left(-x^3+5x^2+2x+1\right)=-\infty$.}
\end{bt}
%bt4
\begin{bt}%[TH]%[DCHT Toán 11 - KNTT -Nguyễn Văn Hiệp]%[1K5BF-4]
	Tính $\lim\limits_{x\to +\infty} \dfrac {3x^2-x}{x+1}$.
	\dapso{$+\infty$.}
	\loigiai{
		Ta có $\lim\limits_{x\to +\infty} \dfrac {3x^2-x}{x+1}=\lim\limits_{x\to +\infty} \dfrac{x^2}{x}\cdot \left(\dfrac{3-\dfrac{1}{x}}{1+\dfrac{1}{x}} \right)=\lim\limits_{x\to +\infty} x\cdot \left(\dfrac {3-\dfrac {1}{x}}{1+\dfrac {1}{x}} \right)=+\infty $.\\
		Vì $\lim\limits_{x\to +\infty} x=+\infty $ và $\lim\limits_{x\to +\infty} \dfrac {3-\dfrac {1}{x}}{1+\dfrac {1}{x}}=3$.}
\end{bt}
%bt5
\begin{bt}%[TH]%[DCHT Toán 11 - KNTT -Nguyễn Văn Hiệp]%[1K5BF-4]
	Giá trị của giới hạn $\lim\limits_{x\to +\infty} \left(\sqrt {1+2x^2}-x \right)$ là bao nhiêu?
	\dapso{$+\infty$.}
	\loigiai{Ta có $\lim\limits_{x\to +\infty} \left(\sqrt {1+2x^2}-x\right)=\lim\limits_{x\to +\infty } x\left(\sqrt {\dfrac {1}{x^2}+2}-1 \right)=+\infty$.\\
		Vì $\lim\limits_{x\to +\infty}x=+\infty$; $\lim\limits_{x\to +\infty}\left(\sqrt {\dfrac {1}{x^2}+2}-1 \right)=\sqrt {2}-1>0$.
	}
\end{bt}
%bt6
\begin{bt}%[TH]%[DCHT Toán 11 - KNTT -Nguyễn Văn Hiệp]%[1K5BF-4]
	Tính $\lim\limits_{x\to 3} \left( \dfrac {1}{x}-\dfrac {1}{3} \right)\dfrac {1}{(x-3)^3}$.
	\dapso{$-\infty$}
	\loigiai{$\lim\limits_{x\to 3} \left(\dfrac {1}{x}-\dfrac {1}{3} \right)\dfrac {1}{(x-3)^3}=\lim\limits_{x\to 3} \dfrac {3-x}{3x}\cdot \dfrac {1}{(x-3)^3}=\lim\limits_{x\to 3} \dfrac {-1}{3x(x-3)^2}=-\infty$.}
\end{bt}
%bt7
\begin{bt}[VDT]%[DCHT Toán 11 - KNTT -Nguyễn Văn Hiệp]%[1K5KF-4]
	Có bao nhiêu giá trị $ m$ nguyên thuộc đoạn $[-20;20]$ để $\lim\limits_{x\to +\infty } \left( \sqrt {4x^2-3x+2}+mx-1 \right)=-\infty$?
	\dapso{$18$.}
	\loigiai{Ta có
		\allowdisplaybreaks
		$\begin{aligned}[t]
			\lim\limits_{x\to +\infty} \left(\sqrt {4x^2-3x+2}+mx-1 \right)&=\lim\limits_{x\to +\infty} \left(x\sqrt {4-\dfrac {3}{x}+\dfrac {2}{x^2}}+mx-1\right)\\
			&=\lim\limits_{x\to +\infty} x\left(\sqrt {4-\dfrac {3}{x}+\dfrac {2}{x^2}}+m-\dfrac {1}{x}\right).
		\end{aligned}$\\
		Mà $\lim\limits_{x\to +\infty } x=+\infty $ và $\lim\limits_{x\to +\infty } \left(\sqrt {4-\dfrac {3}{x}+\dfrac {2}{x^2}}+m-\dfrac {1}{x} \right)=2+m$ nên $\lim\limits_{x\to +\infty } \left( \sqrt {4x^2-3x+2}+mx-1 \right)=-\infty $ khi $2+m<0\Leftrightarrow m<-2$.\\
		Do $ m$ nguyên thuộc đoạn $[-20;20]$ nên $m\in \{ -20;-19;-18;\ldots;-3 \}$.\\
		Vậy có $18$ giá trị $ m$ nguyên thuộc đoạn $[-20;20]$ thỏa bài toán.}
\end{bt}

\subsubsection{Bài tập trắc nghiệm}
\Opensolutionfile{ans}[ans/ans-1K5-2-Dang3]
%Câu 1
\begin{ex}%[DCHT Toán 11 - KNTT -Nguyễn Văn Hiệp]%[1K5YF-4]
	Giá trị của $\lim\limits_{x\to -\infty}(-x^3)$ bằng
	\choice
	{\True $+\infty$}
	{$-\infty$}
	{$1$}
	{$-1$}
	\loigiai{Ta có $\lim\limits_{x\to -\infty}(-x^3)=+\infty $.}
\end{ex}
%Cau2
\begin{ex}%[DCHT Toán 11 - KNTT -Nguyễn Văn Hiệp]%[1K5BF-4]
	Giới hạn$\lim\limits_{x\to -\infty} \left(3x^3+5x^2-9\sqrt {3}x-2022 \right)$ bằng
	\choice
	{\True $-\infty $}
	{$3$}
	{$-3$}
	{$+\infty $}
	\loigiai{Ta có $\lim\limits_{x\to -\infty } \left( 3x^3+5x^2-9\sqrt {3}x-2022 \right)=\lim\limits_{x\to -\infty } x^3\left( 3+5\cdot \dfrac {1}{x}-9\sqrt {3}\cdot \dfrac {1}{x^2}-2022\cdot \dfrac {1}{x^3} \right)=-\infty$.}
\end{ex}
%Cau3
\begin{ex}%[DCHT Toán 11 - KNTT -Nguyễn Văn Hiệp]%[1K5BF-4]
	Tính $\lim\limits_{x\to -\infty } ( x^3+3x-3 )$.
	\choice
	{$2$}
	{$1$}
	{\True $-\infty $}
	{$+\infty $}
	\loigiai{Ta có $\lim\limits_{x\to -\infty} \left( x^3+3x-3 \right)=\lim\limits_{x\to -\infty } \left[ x^3\left( 1+\dfrac {3}{x^2}-\dfrac {3}{x^3} \right) \right]=-\infty $.\\
		Vì $\lim\limits_{x\to -\infty}x^3=-\infty$; $\lim\limits_{x\to -\infty}\left(1+\dfrac {3}{x^2}-\dfrac {3}{x^3}\right)=1>0$.}
\end{ex}
%Cau4
\begin{ex}%[DCHT Toán 11 - KNTT -Nguyễn Văn Hiệp]%[1K5BF-4]
	Với $ k$ là số nguyên dương chẵn. Kết quả của $\lim\limits_{x\to -\infty } \left(-3x^k \right)$ là
	\choice
	{$0$}
	{\True $-\infty $}
	{$-3x_0^k$}
	{$+\infty $}
	\loigiai{Ta có $\lim\limits_{x\to -\infty } x^k=+\infty $ khi $ k$ là số nguyên dương chẵn.\\
		Suy ra $\lim\limits_{x\to -\infty } \left(-3x^k\right)=-\infty $.}
\end{ex}
%Cau5
\begin{ex}%[DCHT Toán 11 - KNTT -Nguyễn Văn Hiệp]%[1K5BF-4]
	Cho hai hàm số $ f(x)$, $g(x)$ thỏa mãn $\lim\limits_{x\to 1} f(x)=2$ và $\lim\limits_{x\to 1} g(x)=+\infty $. Giá trị của $\lim\limits_{x\to 1} [ f(x)\cdot g(x)]$ bằng
	\choice
	{\True $+\infty $}
	{$-\infty $}
	{$2$}
	{$-2$}
	\loigiai{Theo quy tắc giới hạn vô cực ta có $\lim\limits_{x\to 1} f(x)=2>0$ và $\lim\limits_{x\to 1} g( x )=+\infty $ thì $\lim\limits_{x\to 1} [f(x)\cdot g(x)]=+\infty $.}
\end{ex}
%Cau6
\begin{ex}%[DCHT Toán 11 - KNTT -Nguyễn Văn Hiệp]%[1K5BF-4]
	Giới hạn $\lim\limits_{x\to -2} \dfrac {x+1}{(x+2)^2}$ bằng
	\choice
	{\True $-\infty $}
	{$\dfrac {3}{16}$}
	{$0$}
	{$+\infty $}
	\loigiai{Ta có $\lim\limits_{x\to -2} \dfrac {x+1}{(x+2)^2}=-\infty$.\\
		Vì $\lim\limits_{x\to -2} (x+1)=-2+1=-1<0$, $\lim\limits_{x\to -2} (x+2)^2=0$ và $(x+2)^2>0$ khi $ x\ne -2$.}
\end{ex}
%Cau7
\begin{ex}%[DCHT Toán 11 - KNTT -Nguyễn Văn Hiệp]%[1K5BF-4]
	$\lim\limits_{x\to -\infty }\left(-3x^3+2x\right)$ bằng
	\choice
	{$-\infty$}
	{\True $+\infty$}
	{$1$}
	{$-1$}
	\loigiai{Ta có $\lim\limits_{x\to -\infty }\left( -3x^3+2x \right)=\lim\limits_{x\to -\infty}x^3\left(-3+\dfrac {2}{x^2} \right)=+\infty $.\\
		Vì $\lim\limits_{x\to -\infty } x^3=-\infty $ và $\lim\limits_{x\to -\infty } \left( -3+\dfrac {2}{x^2} \right)=-3<0$.}
\end{ex}
%Cau8
\begin{ex}%[DCHT Toán 11 - KNTT -Nguyễn Văn Hiệp]%[1K5BF-4]
	Tìm $L=\lim\limits_{x\to -1} \dfrac {2x^2+x-3}{(x+1)^2}$.
	\choice
	{$L=+\infty $}	
	{$L=2$}
	{Không tồn tại $\lim\limits_{x\to -1} \dfrac {2x^2+x-3}{(x+1)^2}$}
	{\True $L=-\infty $}
	\loigiai{$\lim\limits_{x\to -1} \dfrac {2x^2+x-3}{(x+1)^2}=-\infty $ vì $\heva{& \lim\limits_{x\to -1} (2x^2+x-3)=-2 \\ & \lim\limits_{x\to -1} (x+1)^2=0 \\ & x\to -1\Rightarrow (x+1)^2>0.}$}
\end{ex}
%Cau9
\begin{ex}%[DCHT Toán 11 - KNTT -Nguyễn Văn Hiệp]%[1K5BF-4]
	$\lim\limits_{x\to +\infty } \left(-2x^3-2x\right)$ bằng
	\choice
	{\True $-\infty$}
	{$+\infty$}
	{$2$}
	{$-2$}
	\loigiai{Ta có $\lim\limits_{x\to +\infty} \left(-2x^3-2x\right)=\lim\limits_{x\to +\infty}x^3\left(-2-\dfrac {2}{x^2}\right)$.\\
		Mà $\lim\limits_{x\to +\infty}x^3=+\infty$; $\lim\limits_{x\to +\infty}(-2-\dfrac {2}{x^2})=-2<0$ nên $\lim\limits_{x\to +\infty}x^3\left(-2-\dfrac {2}{x^2}\right)=-\infty$.\\
		Vậy $\lim\limits_{x\to +\infty }\left(-2x^3-2x\right)=-\infty$.}
\end{ex}
%Cau10
\begin{ex}%[DCHT Toán 11 - KNTT -Nguyễn Văn Hiệp]%[1K5BF-4]
	Tính giới hạn $\lim\limits_{x\to -\infty } \dfrac {x^2+1}{x-2}$.
	\choice
	{$1$}
	{$-\dfrac {1}{2}$}
	{$+\infty $}
	{\True $-\infty $}
	\loigiai{$\lim\limits_{x\to -\infty} \dfrac {x^2+1}{x-2}=\lim\limits_{x\to -\infty } \dfrac {x^2}{x}\cdot \dfrac {1+\dfrac {1}{x^2}}{1-\dfrac {2}{x}}=\lim\limits_{x\to -\infty } x\cdot \dfrac {1+\dfrac {1}{x^2}}{1-\dfrac {2}{x}}$.\\
		Do $\lim\limits_{x\to -\infty } x=-\infty $ và $\lim\limits_{x\to -\infty } \dfrac {1+\dfrac {1}{x^2}}{1-\dfrac {2}{x}}=1$ nên $\lim\limits_{x\to -\infty } \dfrac {x^2+1}{x-2}=-\infty $.}
\end{ex}
%Câu 11
\begin{ex}%[DCHT Toán 11 - KNTT -Nguyễn Văn Hiệp]%[1K5KF-4]
	Trong các mệnh đề sau, mệnh đề nào đúng?
	\choice
	{\True $\lim\limits_{x\to -\infty } \dfrac {\sqrt {x^4-x}}{1-2x}=+\infty $}
	{ $\lim\limits_{x\to -\infty } \dfrac {\sqrt {x^4-x}}{1-2x}=1$}
	{ $\lim\limits_{x\to -\infty } \dfrac {\sqrt {x^4-x}}{1-2x}=-\infty $}
	{ $\lim\limits_{x\to -\infty } \dfrac {\sqrt {x^4-x}}{1-2x}=0$}
	\loigiai{
		Vì $\lim\limits_{x\to -\infty } \dfrac {\sqrt {x^4-x}}{1-2x}=\lim\limits_{x\to -\infty} \dfrac {x^2 \cdot \sqrt {1-\dfrac {1}{x^3}}}{x\cdot \left( \dfrac {1}{x}-2 \right)}=\lim\limits_{x\to -\infty } x\cdot \dfrac {\sqrt {1-\dfrac {1}{x^3}}}{\dfrac {1}{x}-2}=+\infty $. }
\end{ex}
%Cau 12
\begin{ex}%[DCHT Toán 11 - KNTT -Nguyễn Văn Hiệp]%[1K5KF-4]
	Biết $\lim\limits_{x\to -2} f(x)=-1$. Khi đó $\lim\limits_{x\to -2} \dfrac {f(x)}{(x+2)^4}$ bằng
	\choice
	{$-1$}
	{$+\infty $}
	{\True $-\infty $}
	{$0$}
	\loigiai{Ta có $\lim\limits_{x\to -2} f(x)=-1<0$; $\lim\limits_{x\to -2} (x+2)^4=0$ và $\forall x\ne -2$ thì $(x+2)^4>0$.\\
		Suy ra $\lim\limits_{x\to -2} \dfrac {f(x)}{(x+2)^4}=-\infty $.}
\end{ex}
%Cau 13
\begin{ex}%[DCHT Toán 11 - KNTT -Nguyễn Văn Hiệp]%[1K5KF-4]
	Biết $\lim\limits_{x\to 1} f(x)=-2$. Khi đó $\lim\limits_{x\to 1} \dfrac {f(x)}{(x-1)^2}$ bằng
	\choice
	{\True $-\infty $}
	{$0$}
	{$+\infty$}
	{$-2$}
	\loigiai{Có $\lim\limits_{x\to 1} f(x)=-2<0$, $\lim\limits_{x\to 1} (x-1)^2=0$ và $(x-1)^2>0$, $\forall x\ne 1$ nên $\lim\limits_{x\to 1} \dfrac {f(x)}{(x-1)^2}=-\infty $.}
\end{ex}
%Cau 14
\begin{ex}%[DCHT Toán 11 - KNTT -Nguyễn Văn Hiệp]%[1K5KF-4]
	Có bao nhiêu giá trị nguyên của tham số $m$ thuộc $[-5;5]$ để $L=\lim\limits_{x\to +\infty}\left[x-2(m^2-4)x^3\right]=-\infty$?
	\choice
	{$5$}
	{$10$}
	{$3$}
	{\True $6$}
	\loigiai{Ta có $\lim\limits_{x\to +\infty }\left[ x-2\left( m^2-4 \right)x^3 \right]=\lim\limits_{x\to +\infty } x^3\left[\dfrac {1}{x^2}-2\left( m^2-4 \right) \right]$.\\
		Ta có
		\allowdisplaybreaks
		\begin{eqnarray*}
			&&\lim\limits_{x\to +\infty}x^3=+\infty \Rightarrow L=-\infty \Leftrightarrow \lim\limits_{x\to +\infty} \left[\dfrac {1}{x^2}-2\left(m^2-4\right) \right]<0\\
			&\Leftrightarrow& -2(m^2-4)<0\Leftrightarrow m^2-4>0\Leftrightarrow \hoac{&m>2\\& m<-2.}
		\end{eqnarray*}
		Lại có $m$ thuộc đoạn $[-5;5]$ nên các giá trị nguyên thỏa mãn bài toán của $m$ là $\{-5;-4;-3;3;4;5\}$.\\
		Vậy có $6$ số nguyên thỏa mãn bài toán.}
\end{ex}
%Câu 15
\begin{ex}%[DCHT Toán 11 - KNTT -Nguyễn Văn Hiệp]%[1K5GF-4]
	Có bao nhiêu giá trị $m$ nguyên thuộc đoạn $[-20;20 ]$ để $\lim\limits_{x\to -\infty } \left( \sqrt {4x^2-3x+2}+mx-1 \right)=-\infty$?
	\choice
	{$21$}
	{$22$}
	{\True $18$}
	{$41$}
	\loigiai{Ta có $\lim\limits_{x\to -\infty } \left( \sqrt {4x^2-3x+2}+mx-1 \right)=\lim\limits_{x\to -\infty } x\left( -\sqrt {4-\dfrac {3}{x}+\dfrac {2}{x^2}}+m-\dfrac {1}{x} \right)$.\\
		Có $\lim\limits_{x\to -\infty } x=-\infty $ và $\lim\limits_{x\to -\infty } \left( -\sqrt {4-\dfrac {3}{x}+\dfrac {2}{x^2}}+m-\dfrac {1}{x} \right)=m-2$\\
		Để $\lim\limits_{x\to -\infty } \left( \sqrt {4x^2-3x+2}+mx-1 \right)=-\infty $ suy ra $ m-2>0\Leftrightarrow m>2$.\\
		Với $ m\in \mathbb{Z}$ và $ m\in [ -20;20 ]$ có $ m\in \{ 3;4;5;\ldots ;20 \}$ thỏa mãn yêu cầu bài toán.\\
		Kết luận: Vậy có $18$ giá trị nguyên của $ m$ thỏa mãn yêu cầu bài toán.}
\end{ex}
\Closesolutionfile{ans}
