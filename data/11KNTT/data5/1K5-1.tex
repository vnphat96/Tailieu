\chap{Giới hạn. Hàm số liên tục}
\section{Giới hạn dãy số}
\subsection{Tóm tắt lý thuyết}
\subsubsection{Dãy số có giới hạn $0$}
	\begin{dn}
		Ta nói dãy số $\left(u_{n}\right)$ có giới hạn là $0$ khi $n$ dần tới dương vô cực, nếu $\left|u_{n}\right|$ có thể nhỏ hơn một số dương bé tuỳ ý, kể từ một số hạng nào đó trở đi, kí hiệu $\lim \limits_{n \rightarrow+\infty} u_{n}=0$ hay $u_{n} \rightarrow 0$ khi $n \rightarrow+\infty$.
	\end{dn}
	Từ định nghĩa dãy số có giới hạn $0$, ta có các kết quả sau:
	\begin{itemize}
		\item $\lim \limits_{n \rightarrow+\infty} \dfrac{1}{n^{k}}=0$ với $k$ là một số nguyên dương;	
		\item $\lim \limits_{n \rightarrow+\infty} q^{n}=0$ nếu $|q|<1$;	
		\item Nếu $\left|u_{n}\right| \leq v_{n}$ với mọi $n \geq 1$ và $\lim \limits_{n \rightarrow+\infty} v_{n}=0$ thì $\lim \limits_{n \rightarrow+\infty} u_{n}=0$.	
	\end{itemize}
	\subsubsection{Dãy số có giới hạn hữu hạn}
	\begin{dn}
		Ta nói dãy số $\left(u_{n}\right)$ có giới hạn là số thực a khi $n$ dần tới dương vô cực nếu $$\lim \limits_{n \rightarrow+\infty}\left(u_{n}-a\right)=0,$$ kí hiệu $\lim \limits_{n \rightarrow+\infty} u_{n}=a$ hay $u_{n} \rightarrow a$ khi $n \rightarrow+\infty$. 
	\end{dn}
	\begin{itemize}
		\item Nếu $u_{n}=c$ (c là hằng số) thì $\lim \limits_{n \rightarrow+\infty} u_{n}=c$. 
		\item $\lim \limits_{n \rightarrow+\infty} u_{n}=a$ khi và chỉ khi $\lim \limits_{n \rightarrow+\infty}\left(u_{n}-a\right)=0$.
	\end{itemize}
	\subsubsection{Các quy tắc tính giới hạn}
	\begin{tc} 
		\begin{enumEX}[a)]{1}
			\item Nếu $\lim \limits_{n \rightarrow+\infty} u_{n}=a$ và $\lim \limits_{n \rightarrow+\infty} v_{n}=b$ thì 
			\begin{enumEX}[-)]{2}
				\item 	$\lim \limits_{n \rightarrow+\infty}\left(u_{n}+v_{n}\right)=a+b$.
				\item   $\lim \limits_{n \rightarrow+\infty}\left(u_{n}-v_{n}\right)=a-b$. 
				\item $\lim \limits_{n \rightarrow+\infty}\left(u_{n} \cdot v_{n}\right)=a \cdot b$.
				\item $\lim \limits_{n \rightarrow+\infty} \dfrac{u_{n}}{v_{n}}=\dfrac{a}{b}$ (nếu $b \neq 0$).
			\end{enumEX}
			\item Nếu $u_{n} \geq 0$ với mọi $n$ và $\lim \limits_{n \rightarrow+\infty} u_{n}=a$ thì $
			a \geq 0 \text { và } \lim \limits_{n \rightarrow+\infty} \sqrt{u_{n}}=\sqrt{a}$.
		\end{enumEX}
		
	\end{tc}
\subsection{Các dạng toán thường gặp}
\begin{dang}{Phương pháp đặt thừa số chung (lim hữu hạn)}
	
\end{dang}
\subsubsection{Ví dụ minh hoạ}
\begin{vd}%[1C3Y1-2]%[Anh Duy]%Ví dụ 1.
	Tìm giới hạn sau $\lim\dfrac{2n^3-2n+3}{1-4n^3}$.
	\loigiai{
		\[\lim\dfrac{2n^3-2n+3}{1-4n^3}=\lim\dfrac{2-\dfrac{2}{n^2}+\dfrac{3}{n^3}}{\dfrac{1}{n^3}-4}=-\dfrac{1}{2}.\]}
\end{vd}
\begin{vd}%[1C3Y1-2]%[Anh Duy]%Ví dụ 2.
	Tìm giới hạn sau $\lim\dfrac{\sqrt{n^4+2n+2}}{n^2+1}$.
	\loigiai{
		\[\lim\dfrac{\sqrt{n^4+2n+2}}{n^2+1} = \lim\dfrac{\sqrt{1+\dfrac{2}{n^3}+\dfrac{2}{n^4}}}{1+\dfrac{1}{n^2}}=1.\]}
\end{vd}
\begin{vd}%[1C3Y1-2]%[Anh Duy]%Ví dụ 3.
	Tìm giới hạn sau $\lim\dfrac{3^{n+1}-4^n}{4^{n-1}+3}$.
	\loigiai{
		\[\lim\dfrac{3^{n+1}-4^n}{4^{n-1}+3} = \lim\dfrac{9\cdot 3^{n-1}-4\cdot 4^{n-1}}{4^{n-1}+3}=\lim\dfrac{9\cdot\left(\dfrac{3}{4}\right)^{n-1}-4}{1+3\cdot\left(\dfrac{1}{4}\right)^{n-1}}=-4.\]}
\end{vd}

\begin{vd}%[1C3B1-2]%[Anh Duy]%Ví dụ 4.
	Tìm giới hạn sau $\lim\dfrac{1+2+2^2+\cdots +2^n}{1+3+3^2+\cdots +3^n}$.
	\loigiai{
		\[\lim\dfrac{1+2+2^2+\cdots +2^n}{1+3+3^2+\cdots +3^n} = \lim\dfrac{\dfrac{1-2^{n+1}}{-1}}{\dfrac{1-3^{n+1}}{-2}} = \lim\dfrac{\left(1-2^{n+1}\right)\cdot 2}{1-3^{n+1}} = \lim\dfrac{\left(\left(\dfrac{1}{3}\right)^{n+1}-\left(\dfrac{2}{3}\right)^{n+1}\right)\cdot 2}{\left(\dfrac{1}{3}\right)^{n+1}-1}=0.\]}
\end{vd}
% \subsubsection{Bài tập rèn luyện}
% % \subsubsection{Bài tập tự luận}
% \begin{bt}%[1C3B1-2]%[Anh Duy]
% 	Tìm các giới hạn sau
% 	\begin{enumEX}[a)]{2}
% 		\item[a)] $\lim \limits_{n \rightarrow+\infty} \dfrac{n^{2}+n+1}{2 n^{2}+1}$.
% 		\item[b)] $\lim \limits_{n \rightarrow+\infty}\left(\sqrt{n^{2}+2 n}-n\right)$.
% 	\end{enumEX}
% 	\loigiai{
% 		\begin{enumEX}[a)]{1}
% 			\item $\lim \limits_{n \rightarrow+\infty} \dfrac{n^{2}+n+1}{2 n^{2}+1}=\lim \limits_{n \rightarrow+\infty} \dfrac{1+\dfrac{1}{n}+\dfrac{1}{n^2}}{2+\dfrac{1}{n^2}}=\dfrac{\lim \limits_{n \rightarrow+\infty} \left(1+\dfrac{1}{n}+\dfrac{1}{n^2}\right)}{\lim \limits_{n \rightarrow+\infty} \left(2+\dfrac{1}{n^2}\right)}=\dfrac{1}{2}$.
% 			\item $\lim \limits_{n \rightarrow+\infty}\left(\sqrt{n^{2}+2 n}-n\right)=\lim \limits_{n \rightarrow+\infty}\dfrac{n^2+2n-n^2}{\sqrt{n^{2}+2 n}+n}=\lim \limits_{n \rightarrow+\infty} \dfrac{2}{\sqrt{1+\dfrac{2}{n}}+1}=\dfrac{2}{\lim \limits_{n \rightarrow+\infty} \left(\sqrt{1+\dfrac{2}{n}}+1\right)}=1$.
% 		\end{enumEX}
% 	}
% \end{bt}

% \begin{bt}%[1C3B1-2]%[Anh Duy]
% 	Cho hai dãy số không âm $\left(u_{n}\right)$ và $\left(v_{n}\right)$ với $\lim \limits_{n \rightarrow+\infty} u_{n}=2$ và $\lim \limits_{n \rightarrow+\infty} v_{n}=3$. Tìm các giới hạn sau
% 	\begin{enumEX}[a)]{2}
% 		\item[a)] $\lim \limits_{n \rightarrow+\infty} \dfrac{u_{n}^{2}}{v_{n}-u_{n}}$;
% 		\item[b)] $\lim \limits_{n \rightarrow+\infty} \sqrt{u_{n}+2 v_{n}}$.
% 	\end{enumEX}
% 	\loigiai{
% 		\begin{enumEX}[a)]{1}
% 			\item $\lim \limits_{n \rightarrow+\infty} \dfrac{u_{n}^{2}}{v_{n}-u_{n}} = \dfrac{\lim \limits_{n \rightarrow+\infty} u_{n}^{2}}{\lim \limits_{n \rightarrow+\infty} v_{n}-\lim \limits_{n \rightarrow+\infty} u_{n}} = \dfrac{\left(\lim \limits_{n \rightarrow+\infty} u_{n}\right)^{2}}{\lim \limits_{n \rightarrow+\infty} v_{n}-\lim \limits_{n \rightarrow+\infty} u_{n}}=\dfrac{2^2}{3-2}=4$ ;
% 			\item $\lim \limits_{n \rightarrow+\infty} \sqrt{u_{n}+2 v_{n}} =  \sqrt{\lim \limits_{n \rightarrow+\infty}  u_{n}+\lim \limits_{n \rightarrow+\infty} 2 v_{n}} =\sqrt{\lim \limits_{n \rightarrow+\infty}  u_{n}+2\lim \limits_{n \rightarrow+\infty}  v_{n}} =\sqrt{2+2\cdot 3}=2\sqrt{2}$.
% 		\end{enumEX}
% 	}
% \end{bt}
% \begin{bt}%[1C3B1-2]%[Anh Duy]
% 	Tính các giới hạn sau:
% 	\begin{enumEX}{2}
% 		\item $\lim \limits_{n \to +\infty}\dfrac{2^n+3\cdot 4^n}{4^n-5\cdot 3^n}$.
% 		\item $T=\lim\dfrac{3\cdot7^n+2\cdot 4^n}{4\cdot 5^n+7^n}$.
% 	\end{enumEX}
% 	\loigiai{
% 		\begin{enumEX}{1}
% 			\item $\lim \limits_{n \to +\infty}\dfrac{2^n+3\cdot 4^n}{4^n-5\cdot 3^n}=\lim \limits_{n \to +\infty}\dfrac{\left(\dfrac {1} {2}\right)^n+3}{1-5\left(\dfrac {3} {4}\right)^n}=3$.
% 			\item Ta có $T=\lim\dfrac{3\cdot7^n+2\cdot 4^n}{4\cdot 5^n+7^n}=\lim\dfrac{3+2\cdot\left(\dfrac{4}{7}\right)^n}{4\cdot\left(\dfrac{5}{7}\right)^n+1}=3$.
% 		\end{enumEX}
% 	}
% \end{bt}
\subsubsection{Câu hỏi trắc nghiệm}
\Opensolutionfile{ans}[ans/ans-1K5-1-Dang1]
\begin{ex}%[1C3Y1-2]%[Anh Duy]%Câu 1.
	Tính giới hạn $I=\lim\dfrac{2n+2023}{3n+2024}$. 
	\choice
	{\True $I=\dfrac{2}{3}$}
	{$I=\dfrac{3}{2}$}
	{$I=\dfrac{2023}{2024}$}
	{$I=1$}
	\loigiai{
		Ta có $I=\lim\dfrac{2n+2023}{3n+2024} =\lim\dfrac{2+\dfrac{2017}{n}}{3+\dfrac{2018}{n}} =\dfrac{2}{3}$.}
\end{ex}
\begin{ex}%[1C3Y1-1]%[Anh Duy]%Câu 2.
	Phát biểu nào sau đây là \textbf{sai}?
	\choice
	{$\lim \limits_{n \to +\infty}u_n=c$ ($u_n=c$ là hằng số)}
	{\True $\lim \limits_{n \to +\infty}q^n=0 \;(|q|>1)$}
	{$\lim\dfrac{1}{n}=0$}
	{$\lim\dfrac{1}{n^k}=0 \; (k>1)$}
	\loigiai{
		Theo định nghĩa giới hạn hữu hạn của dãy số thì $\lim \limits_{n \to +\infty}q^n=0 \; (|q|<1)$.}
\end{ex}
\begin{ex}%[1C3Y1-2]%[Anh Duy]%Câu 3.
	Giá trị của $\lim\dfrac{2-n}{n+1}$ bằng
	\choice
	{$1$}
	{$2$}
	{\True $-1$}
	{$0$}
	\loigiai{
		Ta có $\lim\dfrac{2-n}{n+1} =\lim\dfrac{\dfrac{2}{n}-1}{1+\dfrac{1}{n}} =\dfrac{0-1}{1+0} =-1$.}
\end{ex}
\begin{ex}%[1C3Y1-2]%[Anh Duy]%Câu 4.
	Tính giới hạn $\lim\dfrac{4n+2024}{2n+1}$. 
	\choice
	{$\dfrac{1}{2}$}
	{$4$}
	{\True $2$}
	{$2024$}
	\loigiai{
		Ta có $\lim\dfrac{4n+2024}{2n+1}=\lim\dfrac{4+\dfrac{2024}{n}}{2+\dfrac{1}{n}}=2$.}
\end{ex}
\begin{ex}%[1C3Y1-2]%[Anh Duy]%Câu 5.
	$\lim\dfrac{2n^2-3}{n^6+5n^5}$ bằng 
	\choice
	{$2$}
	{\True $0$}
	{$\dfrac{-3}{5}$}
	{$-3$}
	\loigiai{
		Ta có $\lim\dfrac{2n^2-3}{n^6+5n^5} =\lim\dfrac{\dfrac{2}{n^4}-\dfrac{3}{n^6}}{1+\dfrac{5}{n}} =0$.}
\end{ex}
\begin{ex}%[1C3Y1-2]%[Anh Duy]%Câu 6.
	Tính $\lim\dfrac{2n+1}{1+n}$ được kết quả là
	\choice
	{\True $2$}
	{$0$}
	{$\dfrac{1}{2}$}
	{$1$}
	\loigiai{
		Ta có $\lim\dfrac{2n+1}{1+n}=\lim\dfrac{n\left(2+\dfrac{1}{n}\right)}{n\left(\dfrac{1}{n}+1\right)}=\lim\dfrac{2+\dfrac{1}{n}}{\dfrac{1}{n}+1}=\dfrac{2+0}{0+1}=2$.}
\end{ex}

\begin{ex}%[1C3Y1-2]%[Anh Duy]%Câu 7.
	Dãy số nào sau đây có giới hạn khác $0$?
	\choice
	{$\dfrac{1}{n}$}
	{$\dfrac{1}{\sqrt{n}}$}
	{\True $\dfrac{n+1}{n}$}
	{$\dfrac{\sin n}{\sqrt{n}}$}
	\loigiai{
		Có $\lim\dfrac{n+1}{n}=\lim \limits_{n \to +\infty}1+\lim\dfrac{1}{n}=1$.}
\end{ex}

\begin{ex}%[1C3B1-2]%[Anh Duy]%Câu 8.
	Giới hạn $\lim\dfrac{\sqrt{n}}{2n^2+3}$ có kết quả là 
	\choice
	{$2$}
	{\True $0$}
	{$+\infty$}
	{$4$}
	\loigiai{
		$\lim\dfrac{\sqrt{n}}{2n^2+3}=\lim\dfrac{\sqrt{\dfrac{1}{n^3}}}{2+\dfrac{3}{n^2}}=\dfrac{0}{2+0}=0$.}
\end{ex}
\begin{ex}%[1C3Y1-1]%[Anh Duy]%Câu 9.
	Dãy số $(u_n)$ với $u_n=\dfrac{1}{2n}$, chọn $M=\dfrac{1}{100}$, để $\dfrac{1}{2n}<\dfrac{1}{100}$ thì $n$ phải lấy từ số hạng thứ bao nhiêu trở đi?
	\choice
	{\True $51$}
	{$49$}
	{$48$}
	{$50$}
	\loigiai{Ta có $\dfrac{1}{2n}<\dfrac{1}{100}\Leftrightarrow 2n>100\Leftrightarrow n>50$.\\
		Vậy $n$ phải lấy từ số hạng thứ $51$ trở đi.}
\end{ex}
\begin{ex}%[1C3B1-2]%[Anh Duy]%Câu 10.
	Giới hạn $\lim\dfrac{3^n+2^n}{4^n}$ có kết quả là 
	\choice
	{\True $0$}
	{$\dfrac{5}{4}$}
	{$\dfrac{3}{4}$}
	{$+\infty$}
	\loigiai{
		Ta có $\lim\dfrac{3^n+2^n}{4^n}=\lim\dfrac{\left(\dfrac{3}{4}\right)^n+\left(\dfrac{2}{4}\right)^n}{1}=0$.}
\end{ex}
\begin{ex}%[1C3B1-2]%[Anh Duy]%Câu 11.
	Tính giới hạn $\lim\left[\dfrac{1}{1\cdot 2}+\dfrac{1}{2\cdot 3}+\dfrac{1}{3\cdot 4}+\cdots +\dfrac{1}{n(n+1)}\right]$. 
	\choice
	{$0$}
	{$2$}
	{\True $1$}
	{$\dfrac{3}{2}$}
	\loigiai{
		Ta có $\dfrac{1}{1\cdot 2}+\dfrac{1}{2\cdot 3}+\dfrac{1}{3\cdot 4}+\cdots +\dfrac{1}{n(n+1)} =\dfrac{1}{1}-\dfrac{1}{2}+\dfrac{1}{2}-\dfrac{1}{3}+\cdots+\dfrac{1}{n-1}-\dfrac{1}{n}+\dfrac{1}{n}-\dfrac{1}{n+1} =1-\dfrac{1}{n+1}$.\\
		Vậy $\lim\left[\dfrac{1}{1\cdot 2}+\dfrac{1}{2\cdot 3}+\dfrac{1}{3\cdot 4}+\cdots +\dfrac{1}{n(n+1)}\right] =\lim\left(1-\dfrac{1}{n+1}\right)=1$.}
\end{ex}

\begin{ex}%[1C3K1-2]%[Anh Duy]%Câu 12.
	Tính $\lim\sqrt{\dfrac{1^2+2^2+3^2+\cdots +n^2}{2n(n+7)(6n+5)}}$. 
	\choice
	{\True $\dfrac{1}{6}$}
	{$\dfrac{1}{2\sqrt{6}}$}
	{$\dfrac{1}{2}$}
	{$+\infty$}
	\loigiai{
		Ta có $1^2+2^2+3^2+\cdots +n^2=\dfrac{n(n+1)(2n+1)}{6}$.\\
		Khi đó $\lim\sqrt{\dfrac{1^2+2^2+3^3+\cdots +n^2}{2n(n+7)(6n+5)}}=\lim\sqrt{\dfrac{n(n+1)(2n+1)}{12n(n+7)(6n+5)}} =\lim\sqrt{\dfrac{\left(1+\dfrac{1}{n}\right)\left(2+\dfrac{1}{n}\right)}{12\left(1+\dfrac{7}{n}\right)\left(6+\dfrac{5}{n}\right)}} =\dfrac{1}{6}$.}
\end{ex}
\begin{ex}%[1C3K1-2]%[Anh Duy]%Câu 13.
	Giới hạn $\lim\dfrac{(2n-1)(3-n)^2}{(4n-5)^3}$ có kết quả bằng 
	\choice
	{$0$}
	{\True $\dfrac{1}{32}$}
	{$\dfrac{3}{2}$}
	{$\dfrac{1}{2}$}
	\loigiai{
		$\lim\dfrac{(2n-1)(3-n)^2}{(4n-5)^3}=\lim\dfrac{\left(2-\dfrac{1}{n}\right)\left(\dfrac{3}{n}-1\right)^2}{\left(4-\dfrac{5}{n}\right)^3}=\dfrac{2}{4^3}=\dfrac{1}{32}$.}
\end{ex}
\begin{ex}%[1C3K1-2]%[Anh Duy]%Câu 14.
	Tìm $L=\lim\left(\dfrac{1}{1}+\dfrac{1}{1+2}+\cdots +\dfrac{1}{1+2+\cdots +n}\right)$.
	\choice
	{$L=\dfrac{5}{2}$}
	{$L=+\infty$}
	{\True $L=2$}
	{$L=\dfrac{3}{2}$}
	\loigiai{
		Ta có $1+2+3+\cdots +k$ là tổng của cấp số cộng có $u_1=1$, $d=1$ nên $1+2+3+\cdots +k=\dfrac{(1+k)k}{2}$. Khi đó
		\[\dfrac{1}{1+2+\cdots +k}=\dfrac{2}{k(k+1)} =\dfrac{2}{k}-\dfrac{2}{k+1},\; \forall k\in\mathbb{N}^*.\]
		Suy ra	\[L=\lim\left(\dfrac{2}{1}-\dfrac{2}{2}+\dfrac{2}{2}-\dfrac{2}{3}+\dfrac{2}{3}-\dfrac{2}{4}+\cdots +\dfrac{2}{n}-\dfrac{2}{n+1}\right) =\lim\left(\dfrac{2}{1}-\dfrac{2}{n+1}\right) =2.\]}
\end{ex}
%--------------------------------------------------------------------------------------------------

\begin{ex}%[1C3G1-2]%[Anh Duy]%Câu 15.
	Đặt $f(n)=\left(n^2+n+1\right)^2+1$.
	Xét dãy số $(u_n)$ sao cho $u_n=\dfrac{f(1)\cdot f(3)\cdot f(5)\cdots f(2n-1)}{f(2)\cdot f(4)\cdot f(6)\cdots f(2n)}$. Tính $\lim \limits_{n \to +\infty}n\sqrt{u_n}$. 
	\choice
	{$\lim \limits_{n \to +\infty}n\sqrt{u_n}=\sqrt{2}$}
	{$\lim \limits_{n \to +\infty}n\sqrt{u_n}=\dfrac{1}{\sqrt{3}}$}
	{$\lim \limits_{n \to +\infty}n\sqrt{u_n}=\sqrt{3}$}
	{\True $\lim \limits_{n \to +\infty}n\sqrt{u_n}=\dfrac{1}{\sqrt{2}}$}
	\loigiai{
		Xét $g(n)=\dfrac{f(2n-1)}{f(2n)}\Rightarrow g(n)=\dfrac{\left(4n^2-2n+1\right)^2+1}{\left(4n^2+2n+1\right)^2+1}$.\\
		$g(n)=\dfrac{\left(4n^2+1\right)^2-4n\left(4n^2+1\right)+\left(4n^2+1\right)}{\left(4n^2+1\right)^2+4n\left(4n^2+1\right)+\left(4n^2+1\right)}=\dfrac{4n^2+1-4n+1}{4n^2+1+4n+1}=\dfrac{(2n-1)^2+1}{(2n+1)^2+1}$ \\
		$ \Rightarrow u_n=\dfrac{2}{10}\cdot\dfrac{10}{26}\cdot\dfrac{26}{50}\cdots\cdot\dfrac{(2n-3)^2+1}{(2n-1)^2+1}\cdot\dfrac{(2n-1)^2+1}{(2n+1)^2+1}=\dfrac{2}{(2n+1)^2+1} $ \\
		$ \Rightarrow\lim \limits_{n \to +\infty}n\sqrt{u_n}=\lim\sqrt{\dfrac{2n^2}{4n^2+4n+2}}=\dfrac{1}{\sqrt{2}} $.}
\end{ex}
\begin{ex}%[1C3G1-2]%[Anh Duy]%Câu 16.
	Có bao nhiêu giá trị nguyên của tham số $a$ thuộc khoảng $(0;2024)$ để có
	\[ \lim\sqrt{\dfrac{9^n+3^{n+1}}{5^n+9^{n+a}}}\leq\dfrac{1}{2187}\,? \]
	\choice
	{\True $2017$}
	{$2016$}
	{$2023$}
	{$2024$}
	\loigiai{
		Do $\dfrac{9^n+3^{n+1}}{5^n+9^{n+a}}>0$ với $\forall n$ nên $\lim\sqrt{\dfrac{9^n+3^{n+1}}{5^n+9^{n+a}}}=\sqrt{\lim\dfrac{9^n+3^{n+1}}{5^n+9^{n+a}}} =\sqrt{\lim\dfrac{1+3\cdot\left(\dfrac{1}{3}\right)^n}{\left(\dfrac{5}{9}\right)^n+9^a}} =\sqrt{\dfrac{1}{9^a}} =\dfrac{1}{3^a}$.\\
		Theo đề bài ta có $\lim\sqrt{\dfrac{9^n+3^{n+1}}{5^n+9^{n+a}}}\leq\dfrac{1}{2187}\Leftrightarrow\dfrac{1}{3^a}\leq\dfrac{1}{2187}\Leftrightarrow a\geq 7$.\\
		Do $a$ là số nguyên thuộc khoảng $(0;2024)$ nên có $a\in\left\{7;8;9;\ldots;2023\right\}\Rightarrow$ có $2017$ giá trị của $a$.}
\end{ex}
\Closesolutionfile{ans}
% \begin{indapan}{10}
% 	{ans/ans-1K5-1-Dang1}
% \end{indapan}
\begin{dang}{Phương pháp lượng liên hợp (lim hữu hạn)}
	Nếu giới hạn của dãy số ở dạng vô định thì ta sử dụng các phép biến đổi để đưa về dạng cơ bản. \\
	Một số phép biến đổi liên hợp: \\
	\begin{align*}
		f(n) - g(n) &= \dfrac{(f(n))^2 - (g(n))^2}{f(n) + g(n)} \\
		\sqrt{f(n)} - \sqrt{g(n)} &= \dfrac{f(n) - g(n)}{\sqrt{f(n)} + \sqrt{g(n)}} \\
		\sqrt{f(n)} - g(n) &= \dfrac{f(n) - (g(n))^2}{\sqrt{f(n)} + g(n)} \\
		\sqrt[3]{f(n)} - \sqrt[3]{g(n)} &= \dfrac{f(n) - g(n)}{\sqrt[3]{(f(n))^2} + \sqrt[3]{f(n)g(n)} + \sqrt[3]{(g(n))^2}}
	\end{align*}
	
\end{dang}
\subsubsection{Ví dụ minh hoạ}
\begin{vd}%[Dự án soạn đề cương toán 11 - KNTT, Minh Trí]%[1K5BE-3]
	Tính giới hạn $I = \lim \limits_{n \to +\infty}\left(\sqrt{n^2 - 2n + 3} - n\right)$. 
	\loigiai{
		Ta có 
		\begin{align*}
			I &= \lim \limits_{n \to +\infty}\left(\sqrt{n^2 - 2n + 3} - n\right) \\
			&= \lim \limits_{n \to +\infty}\dfrac{n^2 - 2n + 3 - n^2}{\sqrt{n^2 - 2n + 3} + n} \\
			&= \lim \limits_{n \to +\infty}\dfrac{- 2n + 3}{\sqrt{n^2 - 2n + 3} + n} \\
			&= \lim \limits_{n \to +\infty}\dfrac{- 2 + \dfrac{3}{n}}{\sqrt{1 - \dfrac{2}{n} + \dfrac{3}{n^2}} + 1} \\
			&= \dfrac{- 2}{\sqrt{1} + 1} = - 1
		\end{align*}
	}
\end{vd} 
\begin{vd}%[Dự án soạn đề cương toán 11 - KNTT, Minh Trí]%[1K5BE-3]
	Tính giới hạn $I = \lim \limits_{n \to +\infty}\left(\sqrt{n^2 + 7} - \sqrt{n^2 + 5}\right)$. 
	\loigiai{
		Ta có 
		\begin{align*}
			I &= \lim \limits_{n \to +\infty}\left(\sqrt{n^2 + 7} - \sqrt{n^2 + 5}\right) \\
			&= \lim \limits_{n \to +\infty}\dfrac{n^2 + 7 - (n^2 + 5)}{\sqrt{n^2 + 7} + \sqrt{n^2 + 5}} \\
			&= \lim \limits_{n \to +\infty}\dfrac{2}{\sqrt{n^2 + 7} + \sqrt{n^2 + 5}} \\
			&= 0
		\end{align*}
	}
\end{vd}
\begin{vd}%[Dự án soạn đề cương toán 11 - KNTT, Minh Trí]%[1K5KE-3]
	Tính giới hạn $I = \lim \limits_{n \to +\infty}\left(\sqrt{n^2 + 2n} - \sqrt{n^2 - 2n}\right)$. 
	\loigiai{
		Ta có 
		\begin{align*}
			I &= \lim \limits_{n \to +\infty}\left(\sqrt{n^2 + 2n} - \sqrt{n^2 - 2n}\right) \\
			&= \lim \limits_{n \to +\infty}\dfrac{n^2 + 2n - (n^2 - 2n)}{\sqrt{n^2 + 2n} + \sqrt{n^2 - 2n}} \\
			&= \lim \limits_{n \to +\infty}\dfrac{4n}{\sqrt{n^2 + 2n} + \sqrt{n^2 - 2n}} \\
			&= \lim \limits_{n \to +\infty}\dfrac{4}{\sqrt{1 + \dfrac{2}{n}} + \sqrt{1 - \dfrac{2}{n}}} \\
			&= \dfrac{4}{\sqrt{1} + \sqrt{1}} = 2
		\end{align*}
	}
\end{vd}

\begin{vd}%[Dự án soạn đề cương toán 11 - KNTT, Minh Trí]%[1K5KE-3]
	Tính giới hạn $I = \lim \limits_{n \to +\infty}\left(\sqrt{2n^2 - n + 1} - \sqrt{2n^2 - 3n + 2}\right)$. 
	\loigiai{
		Ta có 
		\begin{align*}
			I &= \lim \limits_{n \to +\infty}\left(\sqrt{2n^2 - n + 1} - \sqrt{2n^2 - 3n + 2}\right) \\
			&= \lim \limits_{n \to +\infty}\dfrac{2n^2 - n + 1 - (2n^2 - 3n + 2)}{\sqrt{2n^2 - n + 1} + \sqrt{2n^2 - 3n + 2}} \\
			&= \lim \limits_{n \to +\infty}\dfrac{2n - 1}{\sqrt{2n^2 - n + 1} + \sqrt{2n^2 - 3n + 2}} \\
			&= \lim \limits_{n \to +\infty}\dfrac{2 - \dfrac{1}{n}}{\sqrt{2 - \dfrac{1}{n} + \dfrac{1}{n^2}} + \sqrt{2 - \dfrac{3}{n} + \dfrac{2}{n^2}}} \\
			&= \dfrac{2}{\sqrt{2} + \sqrt{2}} = \dfrac{1}{\sqrt{2}}
		\end{align*}
	}
\end{vd} 
\begin{vd}%[Dự án soạn đề cương toán 11 - KNTT, Minh Trí]%[1K5GE-3]
	Tính giới hạn $I = \lim \limits_{n \to +\infty}\left(n - \sqrt[3]{n^3 + 3n^2 + 1}\right)$.
	\loigiai{
		Ta có 
		\begin{align*} 
			I &= \lim \limits_{n \to +\infty}\left(n - \sqrt[3]{n^3 + 3n^2 + 1}\right) \\
			&= \lim \limits_{n \to +\infty}\dfrac{n^3 - (n^3 + 3n^2 + 1)}{n^2 + \sqrt[3]{n^3 + 3n^2 + 1} + \sqrt[3]{\left(n^3 + 3n^2 + 1\right)^2}} \\
			&= \lim \limits_{n \to +\infty}\dfrac{- 3n^2 - 1}{n^2 + \sqrt[3]{n^3 + 3n^2 + 1} + \sqrt[3]{\left(n^3 + 3n^2 + 1\right)^2}} \\
			&= \lim \limits_{n \to +\infty}\dfrac{- 3 - \dfrac{1}{n^2}}{1 + \sqrt[3]{1 + \dfrac{3}{n} + \dfrac{1}{n^3}} + \sqrt[3]{\left(1 + \dfrac{3}{n} + \dfrac{1}{n^3}\right)^2}} \\ 
			&= \dfrac{- 3}{1 + \sqrt[3]{1} + \sqrt[3]{1}} = - 1
		\end{align*}
	}
\end{vd}
% \subsubsection{Bài tập rèn luyện}
% % \subsubsection{Bài tập tự luận}
% \begin{bt}%[Dự án soạn đề cương toán 11 - KNTT, Minh Trí]%[1K5KE-3]
% 	Tính giới hạn  $I = \lim \limits_{n \to +\infty}\left(\sqrt[3]{n^3 - 2n} - n\right)$. 
% 	\loigiai{
% 		Ta có
% 		\begin{align*} 
% 			I &= \lim \limits_{n \to +\infty}\left(\sqrt[3]{n^3 - 2n} - n\right) \\
% 			&= \lim \limits_{n \to +\infty}\dfrac{n^3 - 2n - n^3}{\sqrt[3]{\left(n^3 - 2n\right)^2} + n \sqrt[3]{n^3 - 2n} + n^2} \\ 
% 			&= \lim \limits_{n \to +\infty}\dfrac{-2n}{\sqrt[3]{\left(n^3 - 2n\right)^2} + n \sqrt[3]{n^3 - 2n} + n^2} \\
% 			&= \lim \limits_{n \to +\infty}\dfrac{\dfrac{-2}{n}}{\sqrt[3]{\left(1 - \dfrac{2}{n^2}\right)^2} + \sqrt[3]{1 - \dfrac{2}{n^2}} + 1} \\
% 			&= 0 
% 		\end{align*}
% 	}
% \end{bt}
% \begin{bt}%[Dự án soạn đề cương toán 11 - KNTT, Minh Trí]%[1K5KE-3]
% 	Tính giới hạn $I = \lim \limits_{n \to +\infty}\left(\sqrt{4n^2 + 5n} - 2n\right)$.
% 	\loigiai{
% 		Ta có
% 		\begin{align*} 
% 			I &= \lim \limits_{n \to +\infty}\left(\sqrt{4n^2 + 5n} - 2n\right) \\
% 			&= \lim \limits_{n \to +\infty}\dfrac{4n^2 + 5n - 4n^2}{\sqrt{4n^2 + 5n} - 2n} \\ 
% 			&= \lim \limits_{n \to +\infty}\dfrac{5n}{\sqrt{4n^2 + 5n} + 2n} \\
% 			&= \lim \limits_{n \to +\infty}\dfrac{5}{\sqrt{4 + \dfrac{5}{n}} + 2} \\
% 			&= \dfrac{5}{\sqrt{4} + 2} = \dfrac{5}{4}
% 		\end{align*}
% 	}
% \end{bt}
% \begin{bt}%[Dự án soạn đề cương toán 11 - KNTT, Minh Trí]%[1K5KE-3]
% 	Tính giới hạn $I = \lim \limits_{n \to +\infty}\left(3n - \sqrt{9n^2 + 1}\right)$.
% 	\loigiai{
% 		Ta có
% 		\begin{align*} 
% 			I &= \lim \limits_{n \to +\infty}\left(3n - \sqrt{9n^2 + 1}\right) \\
% 			&= \lim \limits_{n \to +\infty}\dfrac{9n^2 - (9n^2 + 1)}{3n + \sqrt{9n^2 + 1}} \\ 
% 			&= \lim \limits_{n \to +\infty}\dfrac{- 1}{3n + \sqrt{9n^2 + 1}} \\
% 			&= 0
% 		\end{align*}
% 	}
% \end{bt}  
% \begin{bt}%[Dự án soạn đề cương toán 11 - KNTT, Minh Trí]%[1K5KE-3]
% 	Tính giới hạn $I = \lim \limits_{n \to +\infty}\left(3n - 5 - \sqrt{9n^2 + 1}\right)$.
% 	\loigiai{
% 		Ta có
% 		\begin{align*} 
% 			I &= \lim \limits_{n \to +\infty}\left(3n - \sqrt{9n^2 + 1}\right) \\
% 			&= \lim \limits_{n \to +\infty}\dfrac{(3n - 5)^2 - (9n^2 + 1)}{3n - 5 + \sqrt{9n^2 + 1}} \\ 
% 			&= \lim \limits_{n \to +\infty}\dfrac{- 30n + 24}{3n - 5 + \sqrt{9n^2 + 1}} \\
% 			&= \lim \limits_{n \to +\infty}\dfrac{- 30 + \dfrac{24}{n}}{3 - \dfrac{5}{n} + \sqrt{\dfrac{1}{n^2}}} \\
% 			&= \dfrac{- 30}{3 + \sqrt{9}} = - 5
% 		\end{align*}
% 	}
% \end{bt}  
% \begin{bt}%[Dự án soạn đề cương toán 11 - KNTT, Minh Trí]%[1K5KE-3]
% 	Tính giới hạn $I = \lim \limits_{n \to +\infty}\left(\sqrt[3]{n + 2} - \sqrt[3]{n}\right)$.
% 	\loigiai{
% 		Ta có
% 		\begin{align*} 
% 			I &= \lim \limits_{n \to +\infty}\left(\sqrt[3]{n + 2} - \sqrt[3]{n}\right) \\
% 			&= \lim \limits_{n \to +\infty}\dfrac{n + 2 - n}{\sqrt[3]{(n + 2)^2} + \sqrt[3]{n(n + 2)} + \sqrt[3]{n^2}} \\ 
% 			&= \lim \limits_{n \to +\infty}\dfrac{2}{\sqrt[3]{(n + 2)^2} + \sqrt[3]{n(n + 2)} + \sqrt[3]{n^2}} \\  
% 			&= 0
% 		\end{align*}
% 	}
% \end{bt} 
% \begin{bt}%[Dự án soạn đề cương toán 11 - KNTT, Minh Trí]%[1K5KE-3]
% 	Tính giới hạn $I = \lim \limits_{n \to +\infty}\dfrac{\sqrt{4n^2 + 2n} - n + 1}{\sqrt{9n^2 + n} - 2n}$.
% 	\loigiai{
% 		Ta có
% 		\begin{align*} 
% 			I &= \lim \limits_{n \to +\infty}\dfrac{\sqrt{4n^2 + 2n} - n + 1}{\sqrt{9n^2 + n} - 2n} \\
% 			&= \lim \limits_{n \to +\infty}\dfrac{\left[4n^2 + 2n - (n - 1)^2\right]\left(\sqrt{9n^2 + n} + 2n\right)}{\left(\sqrt{4n^2 + 2n} + n - 1\right)(9n^2 + n - 4n^2)} \\ 
% 			&= \lim \limits_{n \to +\infty}\dfrac{(3n^2 + 4n - 1)\left(\sqrt{9n^2 + n} + 2n\right)}{\left(\sqrt{4n^2 + 2n} + n - 1)(5n^2 + n\right)} \\
% 			&= \lim \limits_{n \to +\infty}\dfrac{\left(3 + \dfrac{4}{n} - \dfrac{1}{n^2}\right)\left(\sqrt{9 + \dfrac{1}{n}} + 2\right)}{\left(\sqrt{4 + \dfrac{2}{n}} + 1 - \dfrac{1}{n}\right)\left(5 + \dfrac{1}{n}\right)} \\
% 			&= \dfrac{3\left(\sqrt{9} + 2\right)}{5\left(\sqrt{4} + 1\right)} = 1
% 		\end{align*}
% 	}
% \end{bt}  
% \begin{bt}%[Dự án soạn đề cương toán 11 - KNTT, Minh Trí]%[1K5KE-3]
% 	Tính giới hạn $I = \lim \limits_{n \to +\infty}\left(\sqrt[3]{8n^3 + 1} - \sqrt{4n^2 - n + 5}\right)$.
% 	\loigiai{
% 		Ta có
% 		\begin{align*} 
% 			I &= \lim \limits_{n \to +\infty}\left(\sqrt[3]{8n^3 + 1} - \sqrt{4n^2 - n + 5}\right) \\
% 			&= \lim \limits_{n \to +\infty}\left(\sqrt[3]{8n^3 + 1} - 2n\right) + \lim \limits_{n \to +\infty}\left(2n - \sqrt{4n^2 - n + 5}\right) \\ 
% 			&= \lim \limits_{n \to +\infty}\dfrac{8n^3 + 1 - 8n^3}{\sqrt[3]{(8n^3 + 1)^2} + 2n\sqrt[3]{8n^3 + 1} + 4n^2} + \lim \limits_{n \to +\infty}\dfrac{4n^2 - (4n^2 - n + 5)}{2n + \sqrt{4n^2 - n + 5}} \\
% 			&= \lim \limits_{n \to +\infty}\dfrac{1}{\sqrt[3]{(8n^3 + 1)^2} + 2n\sqrt[3]{8n^3 + 1} + 4n^2} + \lim \limits_{n \to +\infty}\dfrac{n - 5}{2n + \sqrt{4n^2 - n + 5}} \\
% 			&= 0 + \lim \limits_{n \to +\infty}\dfrac{1 - \dfrac{5}{n}}{2 + \sqrt{4 - \dfrac{1}{n} + \dfrac{5}{n^2}}} \\
% 			&= \dfrac{1}{2 + \sqrt{4}} = \dfrac{1}{4}
% 		\end{align*}
% 	}
% \end{bt}  
% \begin{bt}%[Dự án soạn đề cương toán 11 - KNTT, Minh Trí]%[1K5KE-3]
% 	Tính giới hạn $I = \lim \limits_{n \to +\infty}\dfrac{\sqrt{3n^2 + 1} - \sqrt{n - 1}}{n}$.
% 	\loigiai{
% 		Ta có
% 		\begin{align*} 
% 			I &= \lim \limits_{n \to +\infty}\dfrac{\sqrt{3n^2 + 1} - \sqrt{n - 1}}{n} \\
% 			&= \lim \limits_{n \to +\infty}\dfrac{3n^2 + 1 - (n - 1)}{n \left(\sqrt{3n^2 + 1} + \sqrt{n - 1}\right)} \\ 
% 			&= \lim \limits_{n \to +\infty}\dfrac{3n^2 - n + 2}{n \left(\sqrt{3n^2 + 1} + \sqrt{n - 1}\right)} \\
% 			&= \lim \limits_{n \to +\infty}\dfrac{3 - \dfrac{1}{n} + \dfrac{2}{n^2}}{\sqrt{3 + \dfrac{1}{n^2}} + \sqrt{\dfrac{1}{n} - \dfrac{1}{n^2}}} \\
% 			&= \sqrt{3}
% 		\end{align*}
% 	}
% \end{bt}  
\subsubsection{Câu hỏi trắc nghiệm}
\Opensolutionfile{ans}[ans/ans-1K5-1-Dang2]
\begin{ex}%[Dự án soạn đề cương toán 11 - KNTT, Minh Trí]%[1K5TE-3]
	Tính giới hạn $I = \lim \limits_{n \to +\infty}\left(\sqrt{n^2 + 2n + 3} - n\right)$
	\choice
	{\True $1$}
	{$0$}
	{$2$}
	{$3$}
	\loigiai{
		Ta có $I = \lim \limits_{n \to +\infty}\left(\sqrt{n^2 + 2n + 3} - n\right) = \lim \limits_{n \to +\infty}\dfrac{2n + 3}{\sqrt{n^2 + 2n + 3} + n} = \lim \limits_{n \to +\infty}\dfrac{2 + \dfrac{3}{n}}{\sqrt{1 + \dfrac{2}{n} + \dfrac{3}{n^2}} + 1} = 1$
	}
\end{ex}
\begin{ex}%[Dự án soạn đề cương toán 11 - KNTT, Minh Trí]%[1K5KE-3]
	Tính giới hạn $I = \lim \limits_{n \to +\infty}\left(\sqrt{n^2 + 1} - \sqrt{n^2 - 2}\right)$
	\choice
	{$3$}
	{\True $0$}
	{$\sqrt{3}$}
	{$\dfrac{3}{2}$}
	\loigiai{
		Ta có $I = \lim \limits_{n \to +\infty}\left(\sqrt{n^2 + 1} - \sqrt{n^2 - 2}\right) = \lim \limits_{n \to +\infty}\dfrac{3}{\sqrt{n^2 + 1} + \sqrt{n^2 - 2}} = 0$
	}
\end{ex}

\begin{ex}%[Dự án soạn đề cương toán 11 - KNTT, Minh Trí]%[1K5BE-3]
	Tính giới hạn $I = \lim \limits_{n \to +\infty}\left(n - \sqrt{n^2 + 2n - 3}\right)$
	\choice
	{$0$}
	{$-2$}
	{\True $-1$}
	{$2$}
	\loigiai{
		Ta có $I = \lim \limits_{n \to +\infty}\left(n - \sqrt{n^2 + 2n - 3}\right) = \lim \limits_{n \to +\infty}\dfrac{- 2n + 3}{n + \sqrt{n^2 + 2n - 3}} = \lim \limits_{n \to +\infty}\dfrac{- 2 + \dfrac{3}{n}}{1 + \sqrt{1 + \dfrac{2}{n} - \dfrac{3}{n^2}}} = - 1$
	}
\end{ex}
\begin{ex}%[Dự án soạn đề cương toán 11 - KNTT, Minh Trí]%[1K5BE-3]
	Tính giới hạn $I = \lim \limits_{n \to +\infty}\left(\sqrt{n^2 - n + 1} - n\right)$
	\choice
	{$\dfrac{1}{2}$}
	{$1$}
	{\True $- \dfrac{1}{2}$}
	{$0$}
	\loigiai{
		Ta có $I = \lim \limits_{n \to +\infty}\left(\sqrt{n^2 - n + 1} - n\right) = \lim \limits_{n \to +\infty}\dfrac{- n + 1}{\sqrt{n^2 - n + 1} + n} = \lim \limits_{n \to +\infty}\dfrac{- 1 + \dfrac{1}{n}}{\sqrt{1 - \dfrac{1}{n} + \dfrac{1}{n^2}} + 1} = - \dfrac{1}{2}$
	}
\end{ex}
\begin{ex}%[Dự án soạn đề cương toán 11 - KNTT, Minh Trí]%[1K5BE-3]
	Tính giới hạn $I = \lim \limits_{n \to +\infty}\left(\sqrt[3]{n^3 - n^2} - n\right)$
	\choice
	{\True $- \dfrac{1}{3}$}
	{$\dfrac{1}{3}$}
	{$1$}
	{$0$}
	\loigiai{
		Ta có $I = \lim \limits_{n \to +\infty}\left(\sqrt[3]{n^3 - n^2} - n\right)) = \lim \limits_{n \to +\infty}\dfrac{- n^2}{\sqrt[3]{(n^3 - n^2)^2} + n\sqrt[3]{(n^3 - n^2)} + n^2} = \lim \limits_{n \to +\infty}\dfrac{- 1}{\sqrt[3]{\left(1 - \dfrac{1}{n}\right)^2} + \sqrt[3]{1 - \dfrac{1}{n}} + 1} = - \dfrac{1}{3}$
	}
\end{ex}

\begin{ex}%[Dự án soạn đề cương toán 11 - KNTT, Minh Trí]%[1K5BE-3]
	Tính giới hạn $I = \lim \limits_{n \to +\infty}\left(\sqrt{2n^2 + 2n - 1} - \sqrt{2n^2 + n}\right)$
	\choice
	{$0$}
	{$\dfrac{1}{4}$}
	{$\dfrac{1}{2}$}
	{\True $\dfrac{1}{2\sqrt{2}}$}
	\loigiai{
		Ta có $I = \lim \limits_{n \to +\infty}\left(\sqrt{2n^2 + 2n - 1} - \sqrt{2n^2 + n}\right) = \lim \limits_{n \to +\infty}\dfrac{n}{\sqrt{2n^2 + 2n - 1} + \sqrt{2n^2 + n}} = \lim \limits_{n \to +\infty}\dfrac{1}{\sqrt{2 + \dfrac{2}{n} - \dfrac{1}{n^2}} + \sqrt{2 + \dfrac{2}{n}}} = \dfrac{1}{2\sqrt{2}}$
	}
\end{ex}

\begin{ex}%[Dự án soạn đề cương toán 11 - KNTT, Minh Trí]%[1K5BE-3]
	Tính giới hạn $I = \lim \limits_{n \to +\infty}\left(\sqrt[3]{n^3 + 2} - \sqrt[3]{n^3 + 1}\right)$
	\choice
	{\True $0$}
	{$-\dfrac{1}{3}$}
	{$1$}
	{$\dfrac{1}{3}$}
	\loigiai{
		Ta có $I = \lim \limits_{n \to +\infty}\left(\sqrt[3]{n^3 + 2} - \sqrt[3]{n^3 + 1}\right) = \lim \limits_{n \to +\infty}\dfrac{1}{\sqrt[3]{(n^3 + 2)^2} + \sqrt[3]{(n^3 + 2)(n^3 + 1)} + \sqrt[3]{(n^3 + 1)^2}} = 0$
	}
\end{ex}
\begin{ex}%[Dự án soạn đề cương toán 11 - KNTT, Minh Trí]%[1K5BE-3]
	Tính giới hạn $I = \lim \limits_{n \to +\infty}n\left(\sqrt{n^2 + n + 1} - \sqrt{n^2 + n - 8}\right)$
	\choice
	{$0$}
	{$\infty$}
	{$2$}
	{\True $\dfrac{9}{2}$}
	\loigiai{
		Ta có $I = \lim \limits_{n \to +\infty}n\left(\sqrt{n^2 + n + 1} - \sqrt{n^2 + n - 8}\right) = \lim \limits_{n \to +\infty}\dfrac{9n}{\sqrt{n^2 + n + 1} + \sqrt{n^2 + n - 8}}\\ = \lim \limits_{n \to +\infty}\dfrac{9}{\sqrt{1 + \dfrac{1}{n} + \dfrac{1}{n^2}} + \sqrt{1 + \dfrac{1}{n} - \dfrac{8}{n^2}}} = \dfrac{9}{2}$ 
	}
\end{ex}
\Closesolutionfile{ans}
% \begin{indapan}{10}
% 	{ans/ans-1K5-1-Dang2}
% \end{indapan}
\begin{dang}{Giới hạn vô cực}
	Ta nói dãy $\{u_n\}$ có giới hạn là $+ \infty$ khi $n \rightarrow + \infty$, nếu $u_n$ có thể lớn hơn một số dương bất kì, kể từ một số hạng nào đó trở đi. \\
	Kí hiệu: $\lim \limits_{n \to +\infty}u_n = + \infty$ hay $u_n \rightarrow + \infty$ khi $n \rightarrow + \infty$. \\ 
	Dãy số $\{u_n\}$ có giới hạn là $- \infty$ khi $n \rightarrow + \infty$, nếu $\lim \limits_{n \to +\infty}- u_n = + \infty$. \\
	Kí hiệu: $\lim \limits_{n \to +\infty}u_n = - \infty$ hay $u_n \rightarrow - \infty$ khi $n \rightarrow + \infty$. \\ 
	\textbf{Một số giới hạn đặc biệt và định lí về giới hạn dãy số} \\
	\textit{Giới hạn đặc biệt}: \\
	$\displaystyle \lim_{n \rightarrow + \infty} \sqrt{n} = + \infty$ \\
	$\displaystyle \lim_{n \rightarrow + \infty} n^k = + \infty$ với $k$ là số nguyên dương. \\
	$\displaystyle \lim_{n \rightarrow + \infty} q^n = + \infty$ nếu $q > 1$ \\
	\textit{Định lý}: \\
	Nếu $\lim \limits_{n \to +\infty}u_n = a > 0$ và $\lim \limits_{n \to +\infty}v_n = 0$ với $v_n > 0$ thì $\lim \limits_{n \to +\infty}\dfrac{u_n}{v_n} = + \infty$. \\
	Nếu $\lim \limits_{n \to +\infty}u_n = + \infty$ và $\lim \limits_{n \to +\infty}v_n = a > 0$ thì $\lim \limits_{n \to +\infty}u_nv_n = + \infty$.
\end{dang}
\subsubsection{Ví dụ minh hoạ}
\begin{vd}%[Dự án soạn đề cương toán 11 - KNTT, Minh Trí]%[1K5YE-4]
	Tìm giới hạn 
	\begin{enumEX}[a)]{2}
		\item[a)] $\lim \limits_{n \to +\infty}(n^3 + n^2 + n + 1)$.
		\item[b)] $\lim \limits_{n \to +\infty}\left(n^2 - n\sqrt{n} + 1\right)$.
	\end{enumEX}
	\loigiai{
		\begin{enumEX}[a)]{1}
			\item $\lim \limits_{n \to +\infty}(n^3 + n^2 + n + 1) = \lim \limits_{n \to +\infty}n^3\left(1 + \dfrac{1}{n} + \dfrac{1}{n^2} + \dfrac{1}{n^3}\right) = + \infty$.
			\item $\lim \limits_{n \to +\infty}\left(n^2 - n\sqrt{n} + 1\right) = \lim \limits_{n \to +\infty}n^2\left(1 - \dfrac{1}{\sqrt{n}} + \dfrac{1}{n^2}\right) = + \infty.$
		\end{enumEX}
	}
\end{vd}
\begin{vd}%[Dự án soạn đề cương toán 11 - KNTT, Minh Trí]%[1K5BE-4]
	Tìm giới hạn
	\begin{enumEX}[a)]{3}
		\item[a)] $\lim \limits_{n \to +\infty}\dfrac{n^5 + n^4 - n - 2}{4n^3 + 6n^2 + 9}$.
		\item[b)] $\lim \limits_{n \to +\infty}\dfrac{\sqrt[3]{n^6 - 7n^3 - 5n + 8}}{n + 12}$.
		\item[c)] $\lim \limits_{n \to +\infty}\left(n + \sqrt{n^2 - n + 1}\right)$.
	\end{enumEX}
	\loigiai{
		\begin{enumEX}[a)]{1}
			\item $\lim \limits_{n \to +\infty}\dfrac{n^5 + n^4 - n - 2}{4n^3 + 6n^2 + 9} = \lim \limits_{n \to +\infty}\dfrac{n^2 + n - \dfrac{1}{n^2} - \dfrac{2}{n^3}}{4 + \dfrac{6}{n} + \dfrac{9}{n^3}} = \lim \limits_{n \to +\infty}\dfrac{n^2 + n}{4} = + \infty$.
			\item $\lim \limits_{n \to +\infty}\dfrac{\sqrt[3]{n^6 - 7n^3 - 5n + 8}}{n + 12} = \lim \limits_{n \to +\infty}\dfrac{n^2\sqrt[3]{1 - \dfrac{7}{n^3} - \dfrac{5}{n^5} + \dfrac{8}{n^6}}}{n + 12} = \lim \limits_{n \to +\infty}\dfrac{n\sqrt[3]{1 - \dfrac{7}{n^3} - \dfrac{5}{n^5} + \dfrac{8}{n^6}}}{1 + \dfrac{12}{n}} = + \infty$.
			\item $\lim \limits_{n \to +\infty}\left(n + \sqrt{n^2 - n + 1}\right) = n\left(1 + \sqrt{1 - \dfrac{1}{n} + \dfrac{1}{n^2}}\right) = \lim \limits_{n \to +\infty}2n = + \infty$
		\end{enumEX}
	}
\end{vd}
\begin{vd}%[Dự án soạn đề cương toán 11 - KNTT, Minh Trí]%[1K5KE-4]
	Tìm giới hạn
	\begin{enumEX}[a)]{3}
		\item[a)] $\lim \limits_{n \to +\infty}\dfrac{1^3 + 2^3 + ... + n^3}{n^2 + 3n\sqrt{n} + 2}$.
		\item[b)] $\lim \limits_{n \to +\infty}\left(n + \sqrt[3]{n^3 - 2n + 1}\right)$.
		\item[c)] $\lim \limits_{n \to +\infty}\dfrac{n^3 - 3n}{2n + 15}$.
	\end{enumEX}
	\loigiai{
		\begin{enumEX}[a)]{1}
			\item $\lim \limits_{n \to +\infty}\dfrac{1^3 + 2^3 + ... + n^3}{n^2 + 3n\sqrt{n} + 2} = \lim \limits_{n \to +\infty}\dfrac{\dfrac{1}{4}n^2(n + 1)^2}{n^2 + 3n\sqrt{n} + 2} = \lim \limits_{n \to +\infty}\dfrac{\dfrac{1}{4}(n + 1)^2}{1 + \dfrac{3}{\sqrt{n}} + \dfrac{2}{n^2}} = \lim \limits_{n \to +\infty}\dfrac{1}{4}(n + 1)^2 = + \infty$.
			\item $\lim \limits_{n \to +\infty}\left(n + \sqrt[3]{n^3 - 2n + 1}\right) = n\left(1 + \sqrt[3]{1 - \dfrac{2}{n^2} + \dfrac{1}{n^3}}\right) = \lim \limits_{n \to +\infty}2n = + \infty$.
			\item $\lim \limits_{n \to +\infty}\dfrac{n^3 - 3n}{2n + 15} = \lim \limits_{n \to +\infty}\dfrac{n^2 - 3}{2 + \dfrac{15}{n}} = + \infty$
		\end{enumEX}
	}
\end{vd}
% \subsubsection{Bài tập rèn luyện} 
% \subsubsection{Bài tập tự luận}
% \begin{bt}%[Dự án soạn đề cương toán 11 - KNTT, Minh Trí]%[1K5BE-4]
% 	Tìm giới hạn
% 	\begin{enumEX}[a)]{2}
% 		\item[a)] $\lim \limits_{n \to +\infty}\sqrt{5n^2 - 8n + 7}$.
% 		\item[b)] $\lim \limits_{n \to +\infty}\sqrt{n^3 - 5n + 6}$.
% 	\end{enumEX}
% 	\loigiai{
% 		\begin{enumEX}[a)]{1}
% 			\item $\lim \limits_{n \to +\infty}\sqrt{5n^2 - 8n + 7} = \lim \limits_{n \to +\infty}n\sqrt{5 - \dfrac{8}{n} + \dfrac{7}{n^2}} = + \infty$.
% 			\item $\lim \limits_{n \to +\infty}\sqrt{n^3 - 5n + 6} = \lim \limits_{n \to +\infty}n\sqrt{n} \sqrt{1 - \dfrac{5}{n^2} + \dfrac{6}{n^3}} = + \infty$.
% 		\end{enumEX}
% 	}
% \end{bt}
% \begin{bt}%[Dự án soạn đề cương toán 11 - KNTT, Minh Trí]%[1K5KE-4]
% 	Tìm giới hạn
% 	\begin{enumEX}[a)]{2}
% 		\item[a)] $\lim \limits_{n \to +\infty}\dfrac{\sqrt{5n^4 - 8n^2 + 10}}{4n + 5}$.
% 		\item[b)] $\lim \limits_{n \to +\infty}\dfrac{n^2 - 15n + 11}{\sqrt{n^2 - 8n + 7}}$.
% 	\end{enumEX}
% 	\loigiai{
% 		\begin{enumEX}[a)]{1}
% 			\item $\lim \limits_{n \to +\infty}\dfrac{\sqrt{5n^4 - 8n^2 + 10}}{4n + 5} = \lim \limits_{n \to +\infty}\dfrac{n^2\sqrt{5 - \dfrac{8}{n^2} + \dfrac{10}{n^4}}}{4n + 5} = \dfrac{n\sqrt{5 - \dfrac{8}{n^2} + \dfrac{10}{n^4}}}{4 + \dfrac{5}{n}} = \lim \limits_{n \to +\infty}\dfrac{n\sqrt{5}}{4} = + \infty$.
% 			\item $\lim \limits_{n \to +\infty}\dfrac{n^2 - 15n + 11}{\sqrt{n^2 - 8n + 7}} = \lim \limits_{n \to +\infty}\dfrac{n - 15 + \dfrac{11}{n}}{\sqrt{1 - \dfrac{8}{n} + \dfrac{7}{n^2}}} = + \infty$.
% 		\end{enumEX}
% 	}
% \end{bt}
% \begin{bt}%[Dự án soạn đề cương toán 11 - KNTT, Minh Trí]%[1K5BE-4]
% 	Tìm $\lim \limits_{n \to +\infty}\left(\dfrac{1}{n^2}+\dfrac{2}{n^2}+\ldots+\dfrac{n}{n^2}\right)$.
% 	\loigiai{
% 		$$
% 		\lim \limits_{n \to +\infty}\left(\dfrac{1}{n^2}+\dfrac{2}{n^2}+\ldots+\dfrac{n}{n^2}\right)=\lim \limits_{n \to +\infty}\left(\dfrac{1+2+\ldots+n}{n^2}\right)=\lim \limits_{n \to +\infty}\left(\dfrac{n(n+1)}{2 n^2}\right)=\lim \limits_{n \to +\infty}\left(\dfrac{1+\dfrac{1}{n}}{2}\right)=\dfrac{1}{2} .
% 		$$}
% \end{bt}
% \begin{bt}%[Dự án soạn đề cương toán 11 - KNTT, Minh Trí]%[1K5GE-4]
% 	Tính giới hạn: $\lim \limits_{n \to +\infty}\left[\left(1-\dfrac{1}{2^2}\right)\left(1-\dfrac{1}{3^2}\right) \ldots\left(1-\dfrac{1}{n^2}\right)\right]$.\\
% 	Xét dãy số $\left(u_n\right)$, với $u_n=\left(1-\dfrac{1}{2^2}\right)\left(1-\dfrac{1}{3^2}\right) \ldots\left(1-\dfrac{1}{n^2}\right), n \geq 2, n \in \mathbb{N}$.
% 	\loigiai{
% 		Ta có:
% 		$$
% 		\begin{aligned}
% 			& u_2=1-\dfrac{1}{2^2}=\dfrac{3}{4}=\dfrac{2+1}{2 \cdot 2} \\
% 			& u_3=\left(1-\dfrac{1}{2^2}\right) \cdot\left(1-\dfrac{1}{3^2}\right)=\dfrac{3}{4} \cdot \dfrac{8}{9}=\dfrac{4}{6}=\dfrac{3+1}{2 \cdot 3} ; \\
% 			& u_4=\left(1-\dfrac{1}{2^2}\right) \cdot\left(1-\dfrac{1}{3^2}\right)\left(1-\dfrac{1}{4^2}\right)=\dfrac{3}{4} \cdot \dfrac{8}{9} \cdot \dfrac{15}{16}=\dfrac{5}{8}=\dfrac{4+1}{2 \cdot 4} \\
% 			& \ldots \ldots . \\
% 			& u_n=\dfrac{n+1}{2 n} .
% 		\end{aligned}
% 		$$
% 		Dễ dàng chứng minh bằng phương pháp qui nạp để khẳng định $u_n=\dfrac{n+1}{2 n}, \forall n \geq 2$
% 		\\ Khi đó $\lim \limits_{n \to +\infty}\left[\left(1-\dfrac{1}{2^2}\right)\left(1-\dfrac{1}{3^2}\right) \ldots\left(1-\dfrac{1}{n^2}\right)\right]=\lim \limits_{n \to +\infty}\dfrac{n+1}{2 n}=\dfrac{1}{2}$.}
% \end{bt}
% \begin{bt}%[Dự án soạn đề cương toán 11 - KNTT, Minh Trí]%[1K5BE-4]
% 	Cho dãy số $\left(u_n\right), n \in \mathbb{N}^*$, thỏa mãn điều kiện $\left\{\begin{array}{c}u_1=3 \\ u_{n+1}=-\dfrac{u_n}{5}\end{array}\right.$. Gọi $S=u_1+u_2+u_3+\ldots+u_n$ là tồng $n$ số hạng đầu tiên của dãy số đã cho. Khi đó lim $S_n$ bằng
% 	\loigiai{
% 		Ta có $\dfrac{u_{n+1}}{u_n}=\dfrac{-\dfrac{u_n}{5}}{u_n}=-\dfrac{1}{5}$ do đó dãy $\left(u_n\right), n \in \mathbb{N}^*$ là một cấp số nhân lùi vô hạn có $u_1=3, d=-\dfrac{1}{5}$.
% 		Suy ra $\lim \limits_{n \to +\infty}S_n=\dfrac{u_1}{1-q}=\dfrac{3}{1+\dfrac{1}{5}}=\dfrac{5}{2}$.}
% \end{bt}
% \begin{bt}%[Dự án soạn đề cương toán 11 - KNTT, Minh Trí]%[1K5TE-4]
% 	Trong một lần Đoàn trường Lê Văn Hưu tổ chức chơi bóng chuyền hơi, bạn Nam thả một quả bóng chuyền hơi từ tầng ba, độ cao $8 m$ so với mặt đất và thấy rằng mỗi lần chạm đất thì quả bóng lại nảy lên một độ cao bằng ba phần tư độ cao lần rơi trước. Biết quả bóng chuyển động vuông góc với mặt đất. Khi đó tổng quãng đường quả bóng đã bay từ lúc thả bóng đến khi quả bóng không nảy nữa bằng bao nhiêu ?
% 	\loigiai{
% 		Lần đầu rơi xuống, quảng đường quả bóng đã bay đến lúc chạm đất là $8 m$.\\
% 		Sau đó quả bóng nảy lên và rơi xuống chạm đất lần thứ 2 thì quảng đường quả bóng đã bay là $8+2.8 \cdot \dfrac{3}{4}$.\\
% 		Tương tự, khi quả bóng nảy lên và rơi xuống chạm đất lần thứ $n$ thì quảng đường quả bóng đã bay là $8+2\cdot 8 \cdot \dfrac{3}{4}+\ldots \ldots +2.8 \cdot\left(\dfrac{3}{4}\right)^{n-1}=8+\dfrac{1-\left(\dfrac{3}{4}\right)^n}{1-\dfrac{3}{4}}=8+48\left(1-\left(\dfrac{3}{4}\right)^{n-1}\right)$.\\
% 		Quảng đường quả bóng đã bay từ lúc thả đến lúc không máy nữa bằng: $\lim \limits_{n \to +\infty}\left[8+48\left(1-\left(\dfrac{3}{4}\right)^{n-1}\right)\right]=8+48=56$.}
% \end{bt}
% \begin{bt}%[Dự án soạn đề cương toán 11 - KNTT, Minh Trí]%[1K5GE-4]
% 	Cho hình vuông $ABCD$ có cạnh bằng $a$. Người ta dựng hình vuông $A_1B_1C_1D_1$ có cạnh bằng $\dfrac{1}{2}$ đường chéo của hình vuông $ABCD$; dựng hình vuông $A_2 B_2 C_2 D_2$ có cạnh bằng $\dfrac{1}{2}$ đường chéo của hình vuông $A_1B_1C_1D_1$ và cứ tiếp tục như vậy (tham khảo hình vẽ).
% 	Giả sử cách dựng trên có thể tiến ra vô hạn. Nếu tổng diện tích $S$ của tất cả các hình vuông $ABCD$, $A_1B_1C_1D_1$, $A_2B_2C_2D_2$, $\ldots$ bằng $8$ thì $a$ bằng bao nhiêu? 
% 	\begin{center}
% 		\begin{tikzpicture}[scale=1, font=\footnotesize, line join=round, line cap=round, >=stealth]
% 			\coordinate (A) at (0,0);
% 			\coordinate (B) at (6,0);
% 			\coordinate (D) at (0,6);
% 			\coordinate (C) at ($(B)+(D)-(A)$);
% 			\draw (A)--(B)--(C)--(D)--(A);
			
% 			\coordinate (A_1) at ($(A)!1/2!(B)$);
% 			\coordinate (B_1) at ($(C)!1/2!(B)$);
% 			\coordinate (C_1) at ($(C)!1/2!(D)$);
% 			\coordinate (D_1) at ($(A)!1/2!(D)$);
% 			\draw (A_1)--(B_1)--(C_1)--(D_1)--(A_1);
% 			\foreach \x in {2,3,4,5,6,7}
% 			{ \pgfmathsetmacro{\j}{\x-1}
% 				\coordinate (A_\x) at ($(A_\j)!1/2!(D_\j)$);
% 				\coordinate (B_\x) at ($(A_\j)!1/2!(B_\j)$);
% 				\coordinate (C_\x) at ($(C_\j)!1/2!(B_\j)$);
% 				\coordinate (D_\x) at ($(C_\j)!1/2!(D_\j)$);
% 				\draw (A_\x)--(B_\x)--(C_\x)--(D_\x)--(A_\x);
% 			}
% 			\foreach \x/\y in {A/225,B/-45,C/45,D/135,A_1/-90,B_1/0,C_1/90,D_1/180}{\fill (\x) circle (1pt) ($(\x)+(\y:0.3cm)$) node{$\x$};}
% 			\def \goc{-90};
			
% 			\foreach \t in {2,3,4,5}
% 			{\pgfmathsetmacro{\goca}{\goc - \t*45+45}
% 				\pgfmathsetmacro{\gocb}{\goca + 90}
% 				\pgfmathsetmacro{\gocc}{\gocb + 90}
% 				\pgfmathsetmacro{\gocd}{\gocc + 90}
% 				\foreach \x/\y in {A_\t/\goca,B_\t/\gocb,C_\t/\gocc,D_\t/\gocd}{\fill (\x) circle (1pt) ($(\x)+(\y:0.3cm)$) node{$\x$};}		
% 			}
			
% 		\end{tikzpicture}
% 	\end{center}
% 	\loigiai{
% 		$$
% 		\begin{aligned}
% 			& \text { Ta có } S_{A B C D}=a^2 ; S_{A_1 B_1 C_1 D_1}=\left(\dfrac{a \sqrt{2}}{2}\right)^2=\dfrac{a^2}{2} ; S_{A_2 B_2 C_2 D_2}=\left(\dfrac{a}{2}\right)^2=\dfrac{a^2}{4}=\dfrac{a^2}{2^2} \\
% 			& S=S_{A B C D}+S_{A_1 B_1 C_1 D_1}+S_{A_2 B_2 C_2 D_2}+\ldots=a^2+\dfrac{a^2}{2}+\dfrac{a^2}{2^2}+\ldots=a^2\left(1+\dfrac{1}{2}+\dfrac{1}{2^2}+\ldots\right)=a^2 \cdot \dfrac{1}{1-\dfrac{1}{2}}=2 a^2
% 		\end{aligned}
% 		$$
% 	}
% \end{bt}
% \begin{bt}%[Dự án soạn đề cương toán 11 - KNTT, Minh Trí]%[1K5GE-4]
% 	Cho hình vuông $C_1$ có cạnh bằng $a$. Người ta chia mỗi cạnh của hình vuông thành bốn phần bằng nhau và nối các điểm chia một cách thích hợp để có hình vuông $C_2$ (tham khảo hình vẽ).
% 	Từ hình vuông $C_2$ lại tiếp tục làm như trên ta nhận được dãy các hình vuông $C_1, C_2, C_3, \ldots, C_n, \ldots$.Gọi $S_i$ là diện tích của hình vuông $C_i(i \in\{1 ; 2 ; 3 ; \ldots\})$. Tính tổng $S=S_1+S_2+S_3+\ldots+S_n+\ldots$\\
% 	\begin{center}
% 		\begin{tikzpicture}[scale=1, font=\footnotesize, line join=round, line cap=round, >=stealth]
% 			\coordinate (A1) at (0,0);
% 			\coordinate (B1) at (4,0);
% 			\coordinate (D1) at (0,4);
% 			\coordinate (C1) at ($(B1)+(D1)-(A1)$);
% 			\draw (A1)--(B1)--(C1)--(D1)--(A1);
			
% 			\foreach \x in {2,3,4}
% 			{ \pgfmathsetmacro{\j}{\x-1}
% 				\coordinate (A\x) at ($(A\j)!1/4!(B\j)$);
% 				\coordinate (B\x) at ($(B\j)!1/4!(C\j)$);
% 				\coordinate (C\x) at ($(C\j)!1/4!(D\j)$);
% 				\coordinate (D\x) at ($(D\j)!1/4!(A\j)$);
% 				\draw (A\x)--(B\x)--(C\x)--(D\x)--(A\x);
% 			}
% 		\end{tikzpicture}
% 	\end{center}
% 	\loigiai{
% 		Ta có $S_1=a^2, S_2=\dfrac{5}{8} a^2, S_3=\dfrac{25}{64} a^2, \ldots$
% 		Nên $S=S_1+S_2+S_3+\ldots+S_n+\ldots$ là tổng của cấp số nhân lùi vô hạn với $\left\{\begin{array}{l}u_1=a^2 \\ q=\dfrac{5}{8}\end{array}\right.$.
% 		Khi đó $S=\dfrac{u_1}{1-q}=\dfrac{a^2}{1-\dfrac{5}{8}}=\dfrac{8}{3} a^2$.}
% \end{bt}
% \subsubsection{Câu hỏi trắc nghiệm}
% \Opensolutionfile{ans}[ans/ans-1K5-1-Dang3]
% \begin{ex}%[Dự án soạn đề cương toán 11 - KNTT, Minh Trí]%[1K5KE-4]
% 	Giá trị của giới hạn $\lim \limits_{n \to +\infty}\left(1+\dfrac{1}{2}+\dfrac{1}{2^2}+\ldots+\dfrac{1}{2^n}\right)$ là
% 	\choice
% 	{$1$}
% 	{\True $2$} 
% 	{$\dfrac{1}{2}$}
% 	{$\dfrac{3}{2}$}
% 	\loigiai{
% 		Ta có: $\lim \limits_{n \to +\infty}\left(1+\dfrac{1}{2}+\dfrac{1}{2^2}+\ldots+\dfrac{1}{2^n}\right)=\dfrac{1}{1-\dfrac{1}{2}}=2$.}
% \end{ex}
% \begin{ex}%[Dự án soạn đề cương toán 11 - KNTT, Minh Trí]%[1K5BE-4]
% 	Tính giới hạn $I=\lim \limits_{n \to +\infty}\dfrac{5\cdot4^{n+1}+3^{n+2}}{2^{2 n+1}+1}$.
% 	\choice
% 	{$I=+\infty$}
% 	{\True $I=10$}
% 	{$I=0$}
% 	{$I=20$}
% 	\loigiai{
% 		Ta có $I=\lim \limits_{n \to +\infty}\dfrac{5.4^{n+1}+3^{n+2}}{2^{2 n+1}+1}=\lim \limits_{n \to +\infty}\dfrac{20.4^n+9.3^n}{2 \cdot 4^n+1}=\lim \limits_{n \to +\infty}\dfrac{20+9 \cdot\left(\dfrac{3}{4}\right)^n}{2+\left(\dfrac{1}{4}\right)^n}=\dfrac{20}{2}=10$.}
% \end{ex}

% \begin{ex}%[Dự án soạn đề cương toán 11 - KNTT, Minh Trí]%[1K5BE-4]
% 	Tính tồng $S=1+\dfrac{1}{2}+\dfrac{1}{4}+\dfrac{1}{8}+\cdots+\dfrac{1}{2^n}+\cdots$
% 	\choice
% 	{\True $2$}
% 	{$3$}
% 	{$1$}
% 	{$\dfrac{1}{2}$}
% 	\loigiai{
% 		$S=1+\dfrac{1}{2}+\dfrac{1}{4}+\dfrac{1}{8}+\ldots+\dfrac{1}{2^n}+\ldots=\dfrac{u_1}{1-q}=\dfrac{1}{1-\dfrac{1}{2}}=2$.}
% \end{ex}
% \begin{ex}%[Dự án soạn đề cương toán 11 - KNTT, Minh Trí]%[1K5BE-4]
% 	Tính $\lim \limits_{n \to +\infty}\dfrac{3 n^3-2}{1-2 n^3}$ được kết quả là
% 	\choice
% 	{$\dfrac{3}{2}$}
% 	{\True $-\dfrac{3}{2}$}
% 	{$\dfrac{1}{2}$}
% 	{$\dfrac{-1}{2}$}
% 	\loigiai{
% 		Ta có: $\lim \limits_{n \to +\infty}\dfrac{3 n^3-2}{1-2 n^3}=\lim \limits_{n \to +\infty}\dfrac{n^3\left(3-\dfrac{2}{n^3}\right)}{n^3\left(\dfrac{1}{n^3}-2\right)}=\lim \limits_{n \to +\infty}\dfrac{3-\dfrac{2}{n^3}}{\dfrac{1}{n^3}-2}=\dfrac{3-0}{0-2}=-\dfrac{3}{2}$.}
	
% \end{ex}
% \begin{ex}%[Dự án soạn đề cương toán 11 - KNTT, Minh Trí]%[1K5GE-4]
% 	Cho các số $a, b, c \in R ; b+c=5 ; \lim\limits_{x \rightarrow+\infty}\left(\sqrt{a x^2+b x}-c x\right)=2$. Tính $P=a+2 b+c$
% 	\choice
% 	{$P=12$}
% 	{$P=15$}
% 	{\True $P=10$}
% 	{$P=5$}
% 	\loigiai{
% 		Ta có: Biện luận \\
% 		+ Điều kiện cần để tồn tại giới hạn đã cho là $a>0$\\
% 		+ Nếu $c \leq 0 \Rightarrow \lim\limits_{x \rightarrow+\infty}\left(\sqrt{a x^2+b x}-c x\right)=+\infty$ (loại) \\
% 		+ Nếu $c>0$\\
% 		$2=\lim\limits_{x \rightarrow+\infty}\left(\sqrt{a x^2+b x}-c x\right)=\lim\limits_{x \rightarrow+\infty} \dfrac{\left(\sqrt{a x^2+b x}-c x\right)\left(\sqrt{a x^2+b x}+c x\right)}{\sqrt{a x^2+b x}+c x}=\lim\limits_{x \rightarrow+\infty} \dfrac{\left(a-c^2\right) x^2+b x}{\sqrt{a x^2+b x}+c x}$ là hữu hạn nên: $a-c^2=0 \Leftrightarrow a=c^2$ (1)\\
% 		Khi đó: $2=\lim\limits_{x \rightarrow+\infty} \dfrac{b x}{\sqrt{a x^2+b x}+c x}=\lim\limits_{x \rightarrow+\infty} \dfrac{b}{\sqrt{a+\dfrac{b}{x}}+c}=\dfrac{b}{\sqrt{a}+c} \Leftrightarrow 2(\sqrt{a}+c)=b$\\
% 		Từ ta có hệ: $\left\{\begin{array}{l}a=c^2 \\ 2(\sqrt{a}+c)=b \\ b+c=5 \\ a, c>0\end{array} \Leftrightarrow\left\{\begin{array}{l}a=c^2 \\ 4 c=b \\ b+c=5 \\ a, c>0\end{array} \Leftrightarrow\left\{\begin{array}{l}a=1 \\ b=4 \\ c=1\end{array} \Rightarrow P=a+2 b+c=10\right.\right.\right.$
% 	}
% \end{ex} 
% \begin{ex}%[Dự án soạn đề cương toán 11 - KNTT, Minh Trí]%[1K5KE-4]
% 	Tính $I=\lim\limits_{x \rightarrow+\infty} \dfrac{\sqrt{x^2+x+1}-x}{3}=\dfrac{a}{b} ; a, b \in \mathbb{N}$ và $\dfrac{a}{b}$ là phân số tối giản. Khi đó $2 a-b$ bằng kết quả nào sau đây?
% 	\choice
% 	{$4$}
% 	{\True $-4$}
% 	{$-5$}
% 	{$5$}
% 	\loigiai{
% 		Ta có, $\lim\limits_{x \rightarrow+\infty} \dfrac{\sqrt{x^2+x+1}-x}{3}=\lim\limits_{x \rightarrow+\infty} \dfrac{\left(\sqrt{x^2+x+1}-x\right)\left(\sqrt{x^2+x+1}+x\right)}{3\left(\sqrt{x^2+x+1}+x\right)}$
% 		$$
% 		=\lim\limits_{x \rightarrow+\infty} \dfrac{\left(x^2+x+1\right)-x^2}{3\left(\sqrt{x^2+x+1}+x\right)}=\lim\limits_{x \rightarrow+\infty} \dfrac{x+1}{3\left(\sqrt{x^2+x+1}+x\right)}=\lim\limits_{x \rightarrow+\infty} \dfrac{1+\dfrac{1}{x}}{3\left(\sqrt{1+\dfrac{1}{x}+\dfrac{1}{x^2}}+1\right)}=\dfrac{1}{6}
% 		$$
% 		Khi đó, $a=1 ; b=6$. Vậy $2 a-b=-4$}
% \end{ex}
% \begin{ex}%[Dự án soạn đề cương toán 11 - KNTT, Minh Trí]%[1K5GE-4]
% 	Biết lim $\dfrac{\sqrt{n^2-4 n}-\sqrt{4 n^2+1}}{\sqrt{3 n^2+1}-n}=\dfrac{6-\sqrt{3}}{2}-\dfrac{a}{b}$, trong đó $\dfrac{a}{b}$ là phân số tối giản, $a$ và $b$ là các số nguyên dương. Chọn khẳng định đúng trong các khẳng định sau:
% 	\choice
% 	{ $a=b$}
% 	{ $a+b=7$}
% 	{\True $a+b=14$}
% 	{$\dfrac{b}{a}=\dfrac{7}{2}$}
% 	\loigiai{
		
% 		$$
% 		\begin{aligned}
% 			& \lim \limits_{n \to +\infty}\dfrac{\sqrt{n^2-4 n}-\sqrt{4 n^2+1}}{\sqrt{3 n^2+1}-n}=\lim \limits_{n \to +\infty}\dfrac{\sqrt{1-\dfrac{4}{n}}-\sqrt{4+\dfrac{1}{n^2}}}{\sqrt{3+\dfrac{1}{n^2}}-1}=\dfrac{-1-\sqrt{3}}{2}=\dfrac{6-\sqrt{3}}{2}-\dfrac{7}{2} . \\
% 			& \text { Suy ra } \dfrac{a}{b}=\dfrac{7}{2} \Rightarrow a=7 ; b=2 \Rightarrow a . b=14 .
% 		\end{aligned}
% 		$$
% 	}
% \end{ex}
% \begin{ex}%[Dự án soạn đề cương toán 11 - KNTT, Minh Trí]%[1K5KE-4]
% 	Tìm giới hạn $I=\lim\limits_{x \rightarrow+\infty}\left(x+1-\sqrt{x^2-x+2}\right)$.
% 	\choice
% 	{$I=\dfrac{46}{31}$}
% 	{$I=\dfrac{17}{11}$}
% 	{\True $I=\dfrac{3}{2}$}
% 	{$I=\dfrac{1}{2}$}
% 	\loigiai{
% 		$$
% 		\text { Ta có } I=\lim\limits_{x \rightarrow+\infty}\left(x+1-\sqrt{x^2-x+2}\right)=\lim\limits_{x \rightarrow+\infty} \dfrac{3 x-1}{x+1+\sqrt{x^2-x+2}}=\lim\limits_{x \rightarrow+\infty} \dfrac{3-\dfrac{1}{x}}{1+\dfrac{1}{x}+\sqrt{1-\dfrac{1}{x}+\dfrac{2}{x^2}}}=\dfrac{3}{2} \text {. }
% 		$$
% 	}
% \end{ex}
% \begin{ex}%[Dự án soạn đề cương toán 11 - KNTT, Minh Trí]%[1K5BE-4]
% 	Cho $a$ là một số thực khác $0$ thỏa mãn $\lim\limits_{x \rightarrow a} \dfrac{x^4-a}{x-a}=4$.
% 	Khi đó $a$ bằng
% 	\choice
% 	{$4$}
% 	{$-1$}
% 	{\True $1$}
% 	{$-4$}
% 	\loigiai{
% 		Ta có
% 		$$
% 		\lim\limits_{x \rightarrow a} \dfrac{x^4-a}{x-a}=\lim\limits_{x \rightarrow a} \dfrac{(x-a)(x+a)\left(x^2+a^2\right)}{x-a}=\lim\limits_{x \rightarrow a}\left[(x+a)\left(x^2+a^2\right)\right]=4 a^3
% 		$$
% 		Mà theo giả thiết $\lim\limits_{x \rightarrow a} \dfrac{x^4-a}{x-a}=4$. Do đó $4 a^3=4 \Leftrightarrow a=1$.}
% \end{ex}
% \begin{ex}%[Dự án soạn đề cương toán 11 - KNTT, Minh Trí]%[1K5KE-4]
% 	Cho $a, b, c$ là các số thực khác 0 . Tìm hệ thức liên hệ giữa $a, b, c$ để $\lim\limits_{x \rightarrow-\infty} \dfrac{a x-b \sqrt{9 x^2+2}}{c x+1}=5$.
% 	\choice
% 	{$\dfrac{a-3 b}{c}=5$}
% 	{$\dfrac{a+3 b}{c}=-5$}
% 	{$\dfrac{a-3 b}{c}=-5$}
% 	{\True $\dfrac{a+3 b}{c}=5$}
% 	\loigiai{
% 		Ta có: $\lim\limits_{x \rightarrow-\infty} \dfrac{a x-b \sqrt{9 x^2+2}}{c x+1}=5 \Leftrightarrow \lim\limits_{x \rightarrow-\infty} \dfrac{a x-b|x| \sqrt{9+\dfrac{2}{x^2}}}{c x+1}=5 \Leftrightarrow \lim\limits_{x \rightarrow-\infty} \dfrac{a+b \sqrt{9+\dfrac{2}{x^2}}}{c+\dfrac{1}{x}}=5$ $\Leftrightarrow \dfrac{a+b \sqrt{9+0}}{c+0}=5 \Leftrightarrow \dfrac{a+3 b}{c}=5$.}
% \end{ex}
% \begin{ex}%[Dự án soạn đề cương toán 11 - KNTT, Minh Trí]%[1K5KE-4]
% 	Tính $\lim \limits_{n \to +\infty}\left(\dfrac{1}{n^2+4}+\dfrac{2}{n^2+4}+\dfrac{3}{n^2+4}+\ldots+\dfrac{2n+4}{n^2+4}\right)$.
% 	\choice
% 	{$\dfrac{1}{2}$}
% 	{$0$}
% 	{$1$}
% 	{\True $2$}
% 	\loigiai{
% 		Ta có $\lim \limits_{n \to +\infty}\left(\dfrac{1}{n^2+4}+\dfrac{2}{n^2+4}+\dfrac{3}{n^2+4}+\ldots+\dfrac{2 n+4}{n^2+4}\right)$
% 		$$
% 		\begin{aligned}
% 			& =\lim \limits_{n \to +\infty}\left(\dfrac{1+2+3+\ldots+2 n+4}{n^2+4}\right) \\
% 			& =\lim \limits_{n \to +\infty}\dfrac{(1+2 n+4)(2 n+4)}{2\left(n^2+4\right)} \\
% 			& =\lim \limits_{n \to +\infty}\dfrac{(2 n+5)(2 n+4)}{2\left(n^2+4\right)} \\
% 			& =\lim \limits_{n \to +\infty}\dfrac{\left(2+\dfrac{5}{n}\right)\left(2+\dfrac{4}{n}\right)}{2\left(1+\dfrac{4}{n^2}\right)}=2
% 		\end{aligned}
% 		$$
% 	}
% \end{ex}

% \begin{ex}%[Dự án soạn đề cương toán 11 - KNTT, Minh Trí]%[1K5GE-4]
% 	Gọi $S_1$ là diện tích tam giác đều $A_1 B_1 C_1$ cạnh bằng $a$. Gọi $S_2$ là diện tích tam giác $A_2 B_2 C_2$ vói các đỉnh trung điểm các cạnh $A_1 B_1, B_1 C_1, A_1 C_1$, gọi $S_3$ là diện tích tam giác $A_3 B_3 C_3$ với các định trung điểm các cạnh $A_2 B_2, B_2 C_2, A_2 C_2, \ldots$ và gọi $S_n$ là diện tích tam giác $A_n B_n C_n$ với các đính trung điểm các cạnh $A_{n-1} B_{n-1}, B_{n-1} C_{n-1}, A_{n-1} C_{n-1}$. Khi $n$ tiến về dương vô cực tính tổng $S=S_1+S_2+S_3+\ldots+S_n+\ldots$
% 	\begin{center}
% 		\begin{tikzpicture}[scale=1, font=\footnotesize, line join=round, line cap=round, >=stealth]
% 			\coordinate (A_1) at (90:4);
% 			\coordinate (B_1) at (210:4);
% 			\coordinate (C_1) at (-30:4);
% 			\draw (A_1)--(B_1)--(C_1)--(A_1);
% 			\def \goc{-90};
% 			\foreach \t in {2,3,4}
% 			{ \pgfmathsetmacro{\j}{\t-1}
% 				\coordinate (A_\t) at ($(C_\j)!1/2!(B_\j)$);
% 				\coordinate (B_\t) at ($(A_\j)!1/2!(C_\j)$);
% 				\coordinate (C_\t) at ($(A_\j)!1/2!(B_\j)$);
% 				\draw (A_\t)--(B_\t)--(C_\t)--(A_\t);
% 			}
% 			\foreach \t in {1,2,3,4}
% 			{\pgfmathsetmacro{\goca}{\goc + \t*180}
% 				\pgfmathsetmacro{\gocb}{\goca + 120}
% 				\pgfmathsetmacro{\gocc}{\gocb + 120}
% 				\foreach \x/\y in {A_\t/\goca,B_\t/\gocb,C_\t/\gocc}{\fill (\x) circle (1pt) ($(\x)+(\y:0.3cm)$) node{$\x$};}
% 			}
% 		\end{tikzpicture}
% 	\end{center}
% 	\choice
% 	{$S=\dfrac{4 \sqrt{3} a^2}{3}$}
% 	{$S=\dfrac{\sqrt{3} a^2}{4}$}
% 	{\True $S=\dfrac{\sqrt{3} a^2}{3}$}
% 	{$S=\dfrac{\sqrt{3} a^2}{2}$}	
% 	\loigiai{
% 		Ta có: $S_1=\dfrac{\sqrt{3}}{4}(a)^2, S_2=\dfrac{\sqrt{3}}{4}\left(\dfrac{a}{2}\right)^2=\dfrac{\sqrt{3} a^2}{16}, S_3=\dfrac{\sqrt{3}}{4}\left(\dfrac{a}{4}\right)^2=\dfrac{\sqrt{3} a^2}{64}, \ldots S_n=\dfrac{\sqrt{3}}{4}\left(\dfrac{a}{2^{x-1}}\right)^2=\dfrac{\sqrt{3} a^2}{4^{x-1}}$. \\
% 		Khi $n \rightarrow+\infty \Rightarrow \dfrac{1}{4^{n-1}} \rightarrow 0 \Rightarrow S_n \rightarrow 0$. Lúc đó: \\ $S=S_1+S_2+S_3+\ldots+S_n+\ldots$ là tồng cấp số nhân lùi
% 		vô hạn với $S_1=\dfrac{\sqrt{3}}{4}(a)^2$ và công bội $q=\dfrac{1}{4}$. Vậy tổng diện tích các hình là $S=S_1 \cdot \dfrac{1}{1-q}=\dfrac{\sqrt{3}}{4}(a)^2 \cdot \dfrac{4}{3}=\dfrac{\sqrt{3} a^2}{3}$.}
% \end{ex}
% \begin{ex}%[Dự án soạn đề cương toán 11 - KNTT, Minh Trí]%[1K5TE-4]
% 	Một quả bóng tenis được thả từ độ cao $81(m)$. Mỗi lần chạm đất, quả bóng lại nảy lên hai phần ba độ cao của lần rơi trước. Tính tổng các khoảng cách rơi và nảy của quả bóng từ lúc thả bóng cho đến lúc bóng không nảy nữa.
% 	\choice
% 	{$243(m)$}
% 	{\True $405(\mathrm{~m})$}
% 	{$486(\mathrm{~m})$}
% 	{$524(m)$}
% 	\loigiai{
% 		Đặt $h_1=81(m)$. Sau lần chạm đất đầu tiên, quả bóng nảy lên một độ cao $h_2=\dfrac{2}{3} h_1$. Tiếp đó, bóng roi từ độ cao $h_2$, chạm đất và nảy lên độ cao $h_3=\dfrac{2}{3} h_2$ rồi roi từ độ cao $h_3$ và cứ tiếp tụ như vậy. Sau lần chạm đất thứ $n$ từ độ cao $h_{n'}$ quả bóng nảy lên $h_{n+1}=\dfrac{2}{3} h_{n'}, \ldots$ \\
% 		Vậy tổng các khoảng cách rơi và nảy của quả bóng từ lúc thả bóng cho đến lúc bóng không nảy nữa là $d=\left(h_1+h_2+\ldots+h_n+\ldots\right)+\left(h_2+\ldots+h_n+\ldots\right) \Rightarrow d$ là tổng của hai cấp số nhân lùi vo hạn có số hạng đầu, theo thứ tự là $h_1, h_2$ và có cùng công bội $q=\dfrac{2}{3}$. Suy ra: $d=\dfrac{h_1}{1-\dfrac{2}{3}}+\dfrac{h_2}{1-\dfrac{2}{3}}=405(m)$.}
% \end{ex}
% \begin{ex}%[Dự án soạn đề cương toán 11 - KNTT, Minh Trí]%[1K5TE-4]
% 	Để trang trí cho một tấm bìa hình vuông có cạnh bằng $1$m, bạn $\mathrm{A}$ quyết định vẽ các hình vuông lên tấm bìa bằng cách: hình vuông thứ nhất có các đỉnh là trung điểm của các cạnh tấm bìa, hình vuông thứ hai có các đỉnh là trung điểm của các cạnh hình vuông thứ nhất,hình vuông thứ ba có các đỉnh là trung điểm của các cạnh hình vuông thứ hai,... Giả sử quy trình vẽ hình vuông của bạn $A$ có thể tiến ra vô hạn. Tính độ dài $L$ các nét vẽ hình vuông của bạn $A$.
% 	\begin{center}
% 		\begin{tikzpicture}[scale=0.8, font=\footnotesize, line join=round, line cap=round, >=stealth]
% 			\coordinate (A) at (0,0);
% 			\coordinate (B) at (6,0);
% 			\coordinate (D) at (0,6);
% 			\coordinate (C) at ($(B)+(D)-(A)$);
% 			\draw (A)--(B)--(C)--(D)--(A);
			
% 			\coordinate (A_1) at ($(A)!1/2!(B)$);
% 			\coordinate (B_1) at ($(C)!1/2!(B)$);
% 			\coordinate (C_1) at ($(C)!1/2!(D)$);
% 			\coordinate (D_1) at ($(A)!1/2!(D)$);
% 			\draw (A_1)--(B_1)--(C_1)--(D_1)--(A_1);
% 			\foreach \x in {2,3,4,5,6,7}
% 			{ \pgfmathsetmacro{\j}{\x-1}
% 				\coordinate (A_\x) at ($(A_\j)!1/2!(D_\j)$);
% 				\coordinate (B_\x) at ($(A_\j)!1/2!(B_\j)$);
% 				\coordinate (C_\x) at ($(C_\j)!1/2!(B_\j)$);
% 				\coordinate (D_\x) at ($(C_\j)!1/2!(D_\j)$);
% 				\draw (A_\x)--(B_\x)--(C_\x)--(D_\x)--(A_\x);
% 			}			
% 		\end{tikzpicture}
% 	\end{center}
% 	\choice
% 	{$1+\sqrt{2}$}
% 	{$2+\sqrt{2}$}
% 	{\True $4+4 \sqrt{2}$}
% 	{$8+4 \sqrt{2}$}
% 	\loigiai{
% 		Hình vuông thứ nhất có cạnh là $\dfrac{1}{2} \cdot \sqrt{2}=\dfrac{\sqrt{2}}{2}$ nên có chu vi $S_1=2 \sqrt{2}$,\\
% 		Hình vuông thứ hai có cạnh là $\dfrac{\sqrt{2}}{4} \cdot \sqrt{2}=\dfrac{1}{2}$ nên có chu vi $S_2=2$,\\
% 		Hình vuông thứ ba có cạnh là $\dfrac{1}{4} \cdot \sqrt{2}=\dfrac{\sqrt{2}}{4}$ nên có chu vi $S_3=\sqrt{2}$,\\
% 		Hình vuông thứ $n$ có cạnh là $\left(\dfrac{\sqrt{2}}{2}\right)^n$ nên có chu vi $S_n=4 .\left(\dfrac{\sqrt{2}}{2}\right)^n, \ldots$\\
% 		Khi đó độ dài các nét vẽ cạnh hình vuông là $S=S_1+S_2+\ldots+S_n+\ldots=\dfrac{2 \sqrt{2}}{1-\dfrac{\sqrt{2}}{2}}=4+4 \sqrt{2}$.}
% \end{ex}
% \begin{ex}%[Dự án soạn đề cương toán 11 - KNTT, Minh Trí]%[1K5GE-4]
% 	Tính giới hạn của dãy số $u_n=\dfrac{1}{2 \sqrt{1}+\sqrt{2}}+\dfrac{1}{3 \sqrt{2}+2 \sqrt{3}}+\ldots+\dfrac{1}{(n+1) \sqrt{n}+n \sqrt{n+1}}$ :
% 	\choice
% 	{$+\infty$}
% 	{$-\infty$}
% 	{$0$}
% 	{\True $1$}
% 	\loigiai{
% 		Ta có $: \dfrac{1}{(k+1) \sqrt{k}+k \sqrt{k+1}}=\dfrac{1}{\sqrt{k}}-\dfrac{1}{\sqrt{k+1}}$
% 		Suy ra $u_n=1-\dfrac{1}{\sqrt{n+1}} \Rightarrow \lim \limits_{n \to +\infty}u_n=1$}
% \end{ex}

% \begin{ex}%[Dự án soạn đề cương toán 11 - KNTT, Minh Trí]%[1K5GE-4]
% 	Với $n$ là số tự nhiên lớn hơn $2$ , đặt $S_n=\dfrac{1}{\mathrm{C}_3^3}+\dfrac{1}{\mathrm{C}_4^3}+\dfrac{1}{\mathrm{C}_5^3}+\ldots+\dfrac{1}{\mathrm{C}_n^3}$. Tính $\lim \limits_{n \to +\infty}S_n$
% 	\choice
% 	{$1$} 
% 	{$\dfrac{3}{2}$}
% 	{$3$}
% 	{$\dfrac{1}{3}$}
% 	\loigiai{
% 		$$
% 		\begin{aligned}
% 			& S_n=\dfrac{1}{\mathrm{C}_3^3}+\dfrac{1}{\mathrm{C}_4^3}+\dfrac{1}{\mathrm{C}_5^3}+\ldots+\dfrac{1}{\mathrm{C}_n^3}=\dfrac{3!}{1\cdot2\cdot3}+\dfrac{3 !}{2\cdot3\cdot 4}+\dfrac{3!}{3\cdot 4\cdot 5}+\ldots+\dfrac{3!}{n(n-1)(n-2)} \\
% 			& =6\left[\dfrac{1}{2}\left(-\dfrac{1}{3.2}+\dfrac{1}{2.1}-\dfrac{1}{4.3}+\dfrac{1}{3\cdot 2}+\ldots-\dfrac{1}{n(n-1)}+\dfrac{1}{(n-1)(n-2)}\right)\right]=3\left(\dfrac{1}{2\cdot 1}-\dfrac{1}{n(n-1)}\right)
% 		\end{aligned}
% 		$$
% 		Vậy $\lim \limits_{n \to +\infty}S_n=\lim \limits_{n \to +\infty}3\left(\dfrac{1}{2}-\dfrac{1}{n(n-1)}\right)=\dfrac{3}{2}$}
% \end{ex} 
% \begin{ex}%[Dự án soạn đề cương toán 11 - KNTT, Minh Trí]%[1K5GE-4]
% 	Cho $f(n)=\left(n^2+n+1\right)^2+1$. Xét dãy số $\left(u_n\right)$ sao cho $u_n=\dfrac{f(1) \cdot f(3) \cdot f(5) \ldots f(2 n-1)}{f(2) \cdot f(4) \cdot f(6) \ldots f(2 n)}$. Tính $\lim \limits_{n \to +\infty}n \sqrt{u_n}$
% 	\choice
% 	{\True $\lim \limits_{n \to +\infty}n \sqrt{u_n}=\dfrac{1}{\sqrt{2}}$}
% 	{$\lim \limits_{n \to +\infty}n \sqrt{u_n}=\sqrt{2}$}
% 	{$\lim \limits_{n \to +\infty}n \sqrt{u_n}=\sqrt{3}$}
% 	{$\lim \limits_{n \to +\infty}n \sqrt{u_n}=\dfrac{1}{\sqrt{3}}$}
% 	\loigiai{
% 		$g(n)=\dfrac{f(2 n-1)}{f(2 n)} \Rightarrow g(n)=\dfrac{\left(4 n^2-2 n+1\right)^2+1}{\left(4 n^2+2 n+1\right)^2+1}$ \\ 
% 		Đặt $\left\{\begin{array}{l}a=4 n^2+1 \\ b=2 n\end{array} \Rightarrow\left\{\begin{array}{l}a=b^2+1 \\ a-2 b=(2 n-1)^2 \\ a+2 b=(2 n+1)^2\end{array}\right.\right.$.\\ 
% 		Suy ra $g(n)=\dfrac{(a-b)^2+1}{(a+b)^2+1}=\dfrac{a^2-2 a b+b^2+1}{a^2+2 a b+b^2+1}=\dfrac{a^2-2 a b+a}{a^2+2 a b+b}=\dfrac{a-2 b+1}{a+2 b+1}=\dfrac{(2 n-1)^2+1}{(2 n+1)^2+1}$. \\
% 		$\Rightarrow u_n=g(1) g(2) \ldots \ldots g(n)=\dfrac{2}{10} \cdot \dfrac{10}{26} \ldots \ldots \dfrac{(2 n-1)^2+1}{(2 n+1)^2+1}=\dfrac{2}{(2 n+1)^2+1}$.
% 		$\lim \limits_{n \to +\infty}n \sqrt{u_n}=\lim \limits_{n \to +\infty}n \cdot \sqrt{\dfrac{2}{(2 n+1)^2+1}}=\dfrac{1}{\sqrt{2}}$}
% \end{ex}
\Closesolutionfile{ans}
% \begin{indapan}{10}
% 	{ans/ans-1K5-1-Dang3}
% \end{indapan}
\begin{dang}{Tính tổng của dãy cấp số nhân lùi vô hạn}
	\begin{dn}
		Cấp số nhân vô hạn $u_1, u_1q,...,u_1q^{n-1},...$ có công bội $q$ thỏa mãn $|q|<1$ được gọi là cấp số nhân lùi vô hạn. 
		Tổng của cấp số nhân lùi vô hạn đã cho là $$S=u_1+u_1q+u_1q^2+...=\dfrac{u_1}{1-q}.$$
	\end{dn}
\end{dang}
\subsubsection{Ví dụ minh hoạ}
\begin{vd}%[1C3B1-5]
	Cho cấp số nhân $(u_n)$, với $u_1=1$ và công bội $q=\dfrac{1}{2}$.
	\begin{enumEX}{1}
		\item So sánh $\left|q\right|$ với $1$.
		\item Tính $S_n=u_1+u_2+\cdots+u_n$ từ đó hãy tính $\lim \limits_{n \to +\infty}S_n$.
	\end{enumEX}
	\loigiai{
		\begin{enumerate}
			\item Ta có $\left|q\right|=\left|\dfrac{1}{2}\right|=\dfrac{1}{2}<1$.
			\item Ta có $S_n=\dfrac{u_1\left(1-q^n\right)}{1-q}=\dfrac{1\cdot\left[1-\left(\dfrac{1}{2}\right)^n\right]}{1-\dfrac{1}{2}}=2\cdot\left(1-\dfrac{1}{2^n}\right)=2-\dfrac{1}{2^{n-1}}$.\\
			Khi đó $\lim \limits_{n \to +\infty}S_n=\lim \limits_{n \to +\infty}\left(2-\dfrac{1}{2^{n-1}}\right)=2$.
		\end{enumerate}
	}
\end{vd}
\begin{vd}%[1C3Y1-5]
	Tính tổng $T=1+\dfrac{1}{3}+\dfrac{1}{3^2}+\ldots+\dfrac{1}{3^n}+\ldots$
	\loigiai{
		Các số hạn của tổng lập thành câp số nhân $(u_n)$, có $u_1=1$, $q=\dfrac{1}{3}$ nên\\ $$T=1+\dfrac{1}{3}+\dfrac{1}{3^2}+\ldots+\dfrac{1}{3^n}+\ldots=\dfrac{1}{1-\dfrac{1}{3}}=\dfrac{2}{3}\cdot$$
	}
\end{vd}
\begin{vd}
	Tính tổng $S=1-\dfrac{1}{2}+\dfrac{1}{4}-\dfrac{1}{8}+\ldots+\left(-\dfrac{1}{2}\right)^{n-1}+\ldots$.
	\loigiai{
		Đây là tổng của cấp số nhân lùi vô hạng với $u_{1}=1$ và $q=-\dfrac{1}{2}$. Do đó
		$
		S=\dfrac{u_{1}}{1-q}=\dfrac{1}{1-\left(-\dfrac{1}{2}\right)}=\dfrac{2}{3}.
		$	
	}
\end{vd}

\begin{vd}%[1C3B1-5]
	Biểu diễn số thập phân vô hạn tuần hoàn $2{,}222 \ldots$ dưới dạng phân số.
	\loigiai{
		Ta có $2{,}222 \ldots=2+0{,}2+0{,}02+0{,}002+\ldots=2+2 \cdot 10^{-1}+2 \cdot 10^{-2}+2 \cdot 10^{-3}+\ldots$.\\
		Đây là tổng của cấp số nhân lùi vô hạn với $u_{1}=2, q=10^{-1}$ nên
		$$
		2{,}222 \ldots=\dfrac{u_{1}}{1-q}=\dfrac{2}{1-\dfrac{1}{10}}=\dfrac{20}{9}.
		$$
	}
\end{vd}
\begin{vd}%[1C3B1-5]
	Biểu diễn số thập phân vô hạn tuần hoàn $0,(3)$ dưới dạng phân số.
	\loigiai{
		Ta có  $0,(3)=\dfrac{3}{10}+\dfrac{3}{10^2}+\ldots+\dfrac{3}{10^n}+\ldots=\dfrac{\dfrac{3}{10}}{1-\dfrac{1}{10}}=\dfrac{1}{3}\cdot$
	}
\end{vd}
% \subsubsection{Bài tập rèn luyện} 
% % \subsubsection{Bài tập tự luận}
% \begin{bt}%[1C3B1-5]
% 	\begin{enumEX}{1}
% 		\item[] 
% 		\item Tính tổng cấp số nhân lùi vô hạn $(u_n)$ với $u_1=\dfrac{2}{3}, q=-\dfrac{1}{4}$.
% 		\item Biểu diễn số thập phân vô hạn tuần hoàn $1,(6)$ dưới dạng phân số.
% 	\end{enumEX}
% 	\loigiai{
% 		\begin{enumerate}
% 			\item Ta có  $S=\dfrac{u_1}{1-q}=\dfrac{\dfrac{2}{3}}{1-\left(-\dfrac{1}{4}\right)}=\dfrac{8}{15}$.
% 			\item Ta có  $1,(6)=1+0,(6)=1+\dfrac{6}{10}+\dfrac{6}{10^2}+\cdots+\dfrac{6}{10^n}+\cdots=1+\dfrac{\dfrac{6}{10}}{1-\dfrac{1}{10}}=\dfrac{5}{3}$.
% 		\end{enumerate}
% 	}
% \end{bt}
% \begin{bt}%[1T3B1-6]
% 	Tính tổng của các cấp số nhân lùi vô hạn sau
% 	\begin{listEX}[2]
% 		\item $-\dfrac{1}{2}+\dfrac{1}{4}-\dfrac{1}{8}+\cdots+\left(-\dfrac{1}{2}\right)^n+\cdots$.
% 		\item $\dfrac{1}{4}+\dfrac{1}{16}+\dfrac{1}{64}+\cdots+\left(\dfrac{1}{4}\right)^n+\cdots$.
% 	\end{listEX}
% 	\loigiai{
% 		\begin{enumerate}
% 			\item Tổng trên là tổng của cấp số nhân lùi vô hạn có số hạng đầu $u_1=-\dfrac{1}{2}$ và công bội $q=-\dfrac{1}{2}$ nên 
% 			\[-\dfrac{1}{2}+\dfrac{1}{4}-\dfrac{1}{8}+\cdots+\left(-\dfrac{1}{2}\right)^n+\cdots=\dfrac{-\dfrac{1}{2}}{1-\left(-\dfrac{1}{2}\right)}=-\dfrac{1}{3}. \]
% 			\item Tổng trên là tổng của cấp số nhân lùi vô hạn có số hạng đầu $u_1=\dfrac{1}{4}$ và công bội $q=\dfrac{1}{4}$ nên 
% 			\[\dfrac{1}{4}+\dfrac{1}{16}+\dfrac{1}{64}+\cdots+\left(\dfrac{1}{4}\right)^n+\cdots=\dfrac{\dfrac{1}{4}}{1-\dfrac{1}{4}}=\dfrac{1}{3}. \]
% 		\end{enumerate}
% 	}
% \end{bt}

% \begin{bt}%[1T3B1-5]
% 	Tính tổng của cấp số nhân lùi vô hạn: $1-\dfrac{1}{4}+\dfrac{1}{16}-\dfrac{1}{64}+\cdots+\left(-\dfrac{1}{4}\right)^n+\cdots$.
% 	\loigiai{
% 		Tổng trên là tổng của cấp số nhân lùi vô hạn có số hạng đầu $u_1=1$ và công bội $q=-\dfrac{1}{4}$ nên
% 		\[ 1-\dfrac{1}{4}+\dfrac{1}{16}-\dfrac{1}{64}+\cdots+\left(-\dfrac{1}{4}\right)^n+\cdots=\dfrac{1}{1-\left(-\dfrac{1}{4}\right)}=\dfrac{4}{5}.\]
% 	}
% \end{bt}

% \begin{bt}%[1T3B1-5]
% 	Biết rằng có thể coi số thập phân vô hạn tuần hoàn $0{,}666 \ldots$ là tổng của một cấp số nhân lùi vô hạn:
% 	\[ 0{,}666 \ldots=0{,}6+0{,}06+0{,}006+\cdots=0{,}6+0{,}6 \cdot \dfrac{1}{10}+0{,}6 \cdot \dfrac{1}{10^2}+\cdots.
% 	\]
% 	Hãy viết $0{,}666 \ldots$ dưới dạng phân số.
% 	\loigiai{
% 		Số $0{,}666 \ldots$ là tổng của cấp số nhân lùi vô hạn có số hạng đầu bằng $0{,}6$ và công bội bằng $\dfrac{1}{10}$.\\
% 		Do đó $0{,}666\ldots=\dfrac{0{,}6}{1-\dfrac{1}{10}}=\dfrac{6}{9}=\dfrac{2}{3}$.
% 	}
% \end{bt}

% \begin{bt}%[1T3B1-5]
% 	Tính tổng của cấp số nhân lùi vô hạn: $1+\dfrac{1}{3}+\left(\dfrac{1}{3}\right)^2+\cdots+\left(\dfrac{1}{3}\right)^n+\cdots$.	
% 	\loigiai{
% 		Tổng trên là tổng của cấp số nhân lùi vô hạn có số hạng đầu $u_1=1$ và công bội $q=\dfrac{1}{3}$ nên
% 		\[ 1+\dfrac{1}{3}+\left(\dfrac{1}{3}\right)^2+\cdots+\left(\dfrac{1}{3}\right)^n+\cdots=\dfrac{1}{1-\dfrac{1}{3}}= \dfrac{3}{2}.\]
% 	}
% \end{bt}
\subsubsection{Câu hỏi trắc nghiệm}
\Opensolutionfile{ans}[ans/ans-1K5-1-Dang4]
\begin{ex}%[1D4B1-5]
	Cho cấp số nhân $u_1,u_2,\ldots$ với công bội $q$ thỏa điều kiện $|q|<1$. Lúc đó, ta nói cấp số nhân đã cho là lùi vô hạn. Tổng của cấp số nhân đã cho là $S=u_1+u_2+u_3+\cdots +u_n+\cdots$ bằng 
	\choice
	{$\dfrac{u_1}{q-1}$}
	{$\dfrac{u_1\left(q^n-1\right)}{q-1}$}
	{$\dfrac{u_1}{1+q}$}
	{\True $\dfrac{u_1}{1-q}$}
	\loigiai{
		Theo định nghĩa cấp số nhân lùi vô hạn ta chứng minh được.\\
		$S=u_1+u_2+u_3+\cdots +u_n+\cdots =u_1+u_1q^1+u_1q^2+\cdots +u_1q^{n-1}+\cdots =\dfrac{u_1}{1-q}$.}
\end{ex}
\begin{ex}%[1D4B1-5]
	Gọi $S=\dfrac{1}{3}-\dfrac{1}{9}+\cdots +\dfrac{(-1)^{n+1}}{3^n}$. Khi đó, $\lim \limits_{n \to +\infty}S$ bằng 
	\choice
	{$\dfrac{3}{4}$}
	{\True $\dfrac{1}{4}$}
	{$\dfrac{1}{2}$}
	{$1$}
	\loigiai{
		Ta có 
		\allowdisplaybreaks
		\begin{eqnarray*}		
			&&S=\dfrac{1}{3}-\dfrac{1}{9}+\cdots +\dfrac{(-1)^{n+1}}{3^n} \\
			&\Leftrightarrow& S=-\left(-\dfrac{1}{3}+\dfrac{1}{9}+\cdots +\dfrac{(-1)^n}{3^n}\right) \\
			&\Leftrightarrow& S=\dfrac{1}{3}\cdot\dfrac{1-\left(\dfrac{-1}{3}\right)^n}{1-\dfrac{-1}{3}}\\
			&\Leftrightarrow& S=\dfrac{1}{4}\cdot\left(1-\left(\dfrac{-1}{3}\right)^n\right).
		\end{eqnarray*}
		Suy ra $\lim \limits_{n \to +\infty}S=\lim \limits_{n \to +\infty}\dfrac{1}{4}\cdot\left(1-\left(\dfrac{-1}{3}\right)^n\right)=\dfrac{1}{4}$.	
	}
\end{ex}
\begin{ex}%[1D4B1-5]
	Tổng $S=\dfrac{1}{3}+\dfrac{1}{3^2}+\cdots +\dfrac{1}{3^n}+\cdots$ có giá trị là 
	\choice
	{$\dfrac{1}{3}$}
	{\True $\dfrac{1}{2}$}
	{$\dfrac{1}{9}$}
	{$\dfrac{1}{4}$}
	\loigiai{
		Ta có $S=\dfrac{1}{3}+\dfrac{1}{3^2}+\cdots +\dfrac{1}{3^n}+\cdots=\dfrac{1}{3}\cdot\dfrac{1}{1-\frac{1}{3}}=\dfrac{1}{2}$. 
	}
\end{ex}
\begin{ex}%[1D4B1-5]
	Tính $S=9+3+1+\dfrac{1}{3}+\dfrac{1}{9}+\cdots +\dfrac{1}{3^{n-3}}+\cdots$. Kết quả là 
	\choice
	{\True $\dfrac{27}{2}$}
	{$14$}
	{$16$}
	{$15$}
	\loigiai{
		Ta có $S=9+3+1+\dfrac{1}{3}+\dfrac{1}{9}+\cdots +\dfrac{1}{3^{n-3}}+\cdots=13+\dfrac{1}{3}\cdot\dfrac{1}{1-\frac{1}{3}}=13+\dfrac{1}{2}=\dfrac{27}{2}$. 
	}
\end{ex}
\begin{ex}%[1D4B1-5]
	Tổng các cấp số nhân vô hạn: $1,-\dfrac{1}{2},\dfrac{1}{4},-\dfrac{1}{8},\ldots,\dfrac{(-1)^{n+1}}{2^{n-1}},\ldots$ là
	\choice
	{$\dfrac{3}{2}$}
	{\True $\dfrac{2}{3}$}
	{$-\dfrac{2}{3}$}
	{$2$}
	\loigiai{
		Ta có $S=1-\dfrac{1}{2}+\dfrac{1}{4}-\dfrac{1}{8}+\cdots +\dfrac{(-1)^{n+1}}{2^{n-1}}+\cdots=1-\dfrac{1}{2}\cdot\dfrac{1}{1+\frac{1}{2}}=1-\dfrac{1}{3}=\dfrac{2}{3}$.
	}
\end{ex}
\begin{ex}%[1D4B1-5]
	Gọi $S=1+\dfrac{2}{3}+\dfrac{4}{9}+\cdots +\dfrac{2^n}{3^n}+\cdots$. Giá trị của $S$ bằng
	\choice
	{\True $3$}
	{$5$}
	{$6$}
	{$4$}
	\loigiai{
		Ta có $S=1+\dfrac{2}{3}+\dfrac{4}{9}+\cdots +\dfrac{2^n}{3^n}+\cdots=1+\dfrac{2}{3}\cdot\dfrac{1}{1-\frac{2}{3}}=1+2=3$.
	}
\end{ex}
\begin{ex}%[1D4B1-5]%
	Số thập phân vô hạn tuần hoàn $0{,}233333\ldots$ biểu diễn dưới dạng số là 
	\choice
	{$\dfrac{1}{23}$}
	{$\dfrac{2333}{10000}$}
	{$\dfrac{23333}{10^5}$}
	{\True $\dfrac{7}{30}$}
	\loigiai{
		$0{,}233333\ldots=0{,}2+3\left(\dfrac{1}{10^2}+\dfrac{1}{10^3}+\ldots\right)=0{,}2+3\cdot\dfrac{1}{100}\cdot\dfrac{1}{1-\frac{1}{10}}=\dfrac{1}{5}+\dfrac{1}{30}=\dfrac{7}{30}$.
	}
\end{ex}
\begin{ex}%[1D4B1-5]%
	Số thập phân vô hạn tuần hoàn $0{,}212121\ldots$ biểu diễn dưới dạng phân số là 
	\choice
	{$\dfrac{2121}{10^4}$}
	{$\dfrac{1}{21}$}
	{\True $\dfrac{7}{33}$}
	{$\dfrac{212121}{10^6}$}
	\loigiai{
		$0{,}212121\ldots=21\left(\dfrac{1}{10^2}+\dfrac{1}{10^4}+\ldots\right)=21\cdot\dfrac{1}{10^2}\cdot\dfrac{1}{1-\frac{1}{100}}=\dfrac{7}{33}$.
	}
\end{ex}

\begin{ex}%[1D4B1-5]
	Số thập phân vô hạn tuần hoàn $0{,}271414\ldots$ được biểu diễn bằng phân số: 
	\choice
	{$\dfrac{2714}{9900}$}
	{$\dfrac{2617}{9900}$}
	{$\dfrac{2786}{9900}$}
	{\True $\dfrac{2687}{9900}$}
	\loigiai{
		$0{,}271414\cdots=0{,}27+14\left(\dfrac{1}{10^4}+\dfrac{1}{10^6}\ldots\right)=0{,}27+14\cdot\dfrac{1}{10^4}\cdot\dfrac{1}{1-\frac{1}{100}}=\dfrac{27}{100}+\dfrac{7}{4950}=\dfrac{2687}{9900}$.
	}
\end{ex}

\begin{ex}%[1D4B1-5]
	Tổng của cấp số nhân lùi vô hạn: $-\dfrac{1}{2},\dfrac{1}{4},\dfrac{1}{8},\ldots,\dfrac{(-1)^n}{2^n},\ldots$ là 
	\choice
	{\True $-\dfrac{1}{3}$}
	{$-\dfrac{1}{4}$}
	{$-1$}
	{$\dfrac{1}{2}$}
	\loigiai{
		Từ $-\dfrac{1}{2},\dfrac{1}{4},\dfrac{1}{8},\ldots,\dfrac{(-1)^n}{2^n},\ldots$ có $u_1=-\dfrac{1}{2}$ và $q=-\dfrac{1}{2}$.\\
		Có $S=-\dfrac{1}{2}+\dfrac{1}{4}+\dfrac{1}{8}+\cdots +\dfrac{(-1)^n}{2^n}+\cdots =\dfrac{\left(-\frac{1}{2}\right)}{1-\left(-\frac{1}{2}\right)}=-\dfrac{1}{3}$.}
\end{ex}
\begin{ex}%[1D4B1-5]
	Số thập phân vô hạn tuần hoàn $0{,}511111\cdots$ được biểu diễn bởi phân số
	\choice
	{$\dfrac{47}{90}$}
	{\True $\dfrac{46}{90}$}
	{$\dfrac{6}{11}$}
	{$\dfrac{43}{90}$}
	\loigiai{
		Ta có\\
		$0{,}511111\cdots=0{,}5+\dfrac{1}{10^2}+\dfrac{1}{10^3}\ldots=\dfrac{1}{2}+\dfrac{1}{10^2}\cdot\dfrac{1}{1-\frac{1}{10}}=\dfrac{1}{2}+\dfrac{1}{90}=\dfrac{23}{45}$.
	}
\end{ex}
\begin{ex}%[1D4B1-5]
	Tổng của cấp số nhân vô hạn $\dfrac{1}{2}$, $-\dfrac{1}{4}$, $\dfrac{1}{8},\ldots\dfrac{(-1)^{n+1}}{2^n},\ldots$ là
	\choice
	{$-\dfrac{2}{3}$}
	{$1$}
	{$-\dfrac{1}{3}$}
	{\True $\dfrac{1}{3}$}
	\loigiai{
		Ta có $S=\dfrac{1}{2}-\dfrac{1}{4}+\dfrac{1}{8}+\cdots +\dfrac{(-1)^{n+1}}{2^n}+\cdots =\dfrac{1}{2}\cdot\dfrac{1}{1+\frac{1}{2}}=\dfrac{1}{3}$.}
\end{ex}
\begin{ex}%[1D4B1-5]
	Tổng của cấp số nhân lùi vô hạn $\dfrac{1}{2},-\dfrac{1}{6},\dfrac{1}{18},\ldots,\dfrac{(-1)^{n+1}}{2\cdot 3^{n-1}},\ldots$ là
	\choice
	{$\dfrac{3}{4}$}
	{$\dfrac{8}{3}$}
	{$\dfrac{2}{3}$}
	{\True $\dfrac{3}{8}$}
	\loigiai{
		Cấp số nhân có $u_1=\dfrac{1}{2}, q=-\dfrac{1}{3}$. Do đó tổng cần tìm là
		$$S=\dfrac{u_1}{1-q}=\dfrac{\dfrac{1}{2}}{1+\frac{1}{3}}=\dfrac{1}{2}\cdot\dfrac{3}{4}=\dfrac{3}{8}.$$}
\end{ex}
\begin{ex}%[1D4B1-5]
	Tổng của cấp số nhân lùi vô hạn $\dfrac{1}{3},-\dfrac{1}{9},\dfrac{1}{27},\ldots\cdot,\dfrac{(-1)^{n+1}}{3^n},\ldots$ là
	\choice
	{$4$}
	{$\dfrac{1}{2}$}
	{$\dfrac{3}{4}$}
	{\True $\dfrac{1}{4}$}
	\loigiai{
		Cấp số nhân có $u_1=\dfrac{1}{3}, q=-\dfrac{1}{3}$. Do đó tổng cần tìm là\\
		$$S=\dfrac{u_1}{1-q}=\dfrac{\dfrac{1}{3}}{1+\frac{1}{3}}=\dfrac{1}{3}\cdot\dfrac{3}{4}=\dfrac{1}{4}.$$}
\end{ex}
\begin{ex}%[1D4B1-5]
	Số thập phân vô hạn tuần hoàn $0{,}17232323\ldots$ được biểu diễn bởi phân số?
	\choice
	{$\dfrac{1706}{9900}$}
	{$\dfrac{153}{990}$}
	{$\dfrac{164}{990}$}
	{\True $\dfrac{853}{4950}$}
	\loigiai{
		$0{,}17232323\ldots=0{,}17+23\left(\dfrac{1}{10^4}+\dfrac{1}{10^6}+\ldots\right)=\dfrac{17}{100}+23\cdot\dfrac{1}{10^4}\cdot\dfrac{1}{1-\frac{1}{100}}=\dfrac{17}{100}+\dfrac{23}{9900}=\dfrac{853}{4950}$.
	}
\end{ex}
\Closesolutionfile{ans}
% \begin{indapan}{10}
% 	{ans/ans-1K5-1-Dang4}
% \end{indapan}

\begin{dang}{Toán thực tế, liên môn liên quan đến giới hạn dãy số}
	$$S=u_1+u_1q+u_1q^2+...=\dfrac{u_1}{1-q}.$$
\end{dang}
\subsubsection{Ví dụ minh hoạ}
\begin{vd}%[1C3K1-6]
	\immini{Từ hình vuông có độ dài cạnh bằng $1$, người ta nối các trung điểm của cạnh hình vuông để tạo ra hình vuông mới như hình bên. Tiếp tục quá trình này đến vô hạn.
		\begin{enumEX}{1}
			\item Tính diện tích $S_n$ của hình vuông được tạo thành từ bước thứ $n$.
			\item Tính tổng diện tích của tất cả các hình vuông được tạo thành.
	\end{enumEX}}	
	{
		\begin{tikzpicture}[scale=0.6, font=\footnotesize,>=stealth]
			\def\canhAD{6};
			\coordinate (D) at (0,0);
			\coordinate (A) at ($(D)+(90:\canhAD)$);
			\coordinate (C) at ($(D)+(0:\canhAD)$);
			\coordinate (B) at ($(A)+(0:\canhAD)$);
			\coordinate (E) at ($(A)!0.5!(B)$);
			\coordinate (F) at ($(B)!0.5!(C)$);
			\coordinate (G) at ($(C)!0.5!(D)$);
			\coordinate (H) at ($(D)!0.5!(A)$);
			\coordinate (K) at ($(E)!0.5!(F)$);
			\coordinate (L) at ($(F)!0.5!(G)$);
			\coordinate (M) at ($(G)!0.5!(H)$);
			\coordinate (J) at ($(H)!0.5!(E)$);
			\coordinate (I) at ($(E)!0.5!(G)$);
			\coordinate (X) at ($(J)!0.5!(K)$);
			\coordinate (Y) at ($(K)!0.5!(L)$);
			\coordinate (T) at ($(L)!0.5!(M)$);
			\coordinate (Q) at ($(M)!0.5!(J)$);
			\draw(A) rectangle (C)(E)--(F)--(G)--(H)--cycle (J)--(K)--(L)--(M)--cycle (X)--(Y)--(T)--(Q)--cycle;
			\draw [>=stealth,|<->|] ([shift=({0,-0.4})]D)--([shift=({0,-0.4})]C) node [midway,below]{$1$};
			
			%			\foreach \x/\y in {A/90,D/-90,C/-90,B/90,E/90,G/-90,H/180,F/0,K/90,L/-90,M/-90,J/90,I/-110}{\fill (\x) circle(1pt) ($(\x)+(\y:0.3cm)$) node{$\x$};}
		\end{tikzpicture}
	}
	
	\loigiai{
		\begin{enumerate}
			\item Từ giả thiết suy ra diện tích hình vuông sau bằng $\dfrac{1}{2}$ diện tích hình vuông trước.\\ 
			Khi đó diện tích của các hình vuông tạo thành một cấp số nhân lùi vô hạn với số hạng đầu $S_1=1$ và công bội $q=\dfrac{1}{2}$.\\
			Diện tích $S_n$ của hình vuông được tạo thành từ bước thứ $n$ là $S_n=S_1\cdot q^{n-1}=\left(\dfrac{1}{2}\right)^{n-1}$.
			\item Tổng diện tích của tất cả các hình vuông được tạo thành là:\\
			$$S=\dfrac{u_1}{1-q}=\dfrac{1}{1-\dfrac{1}{2}}=2.$$
		\end{enumerate}
	}
\end{vd}
\begin{vd}%[1C3K1-6]
	Có $1$ kg chất phóng xạ độc hại. Biết rằng, cứ sau một khoảng thời gian $T=24000$ năm thì một nửa số chất phóng xạ này bị phân rã thành chất khác không độc hại đối với sức khỏe của con người ($T$ được gọi là \textit{chu kì bán rã}).
	\begin{flushright}
		(\textit{Nguồn: Đại số và giải tích 11, NXB GD Việt Nam, 2021})
	\end{flushright}
	Gọi $u_n$ là khối lượng chất phóng xạ còn lại sau chu kì thứ $n$.
	\begin{enumerate}
		\item  Tìm số hạng tổng quát $u_n$ của dãy số $(u_n)$.
		\item  Chứng minh rằng $(u_n)$ có giới hạn là $0$.
		\item  Từ kết quả câu $2$, chứng tỏ rằng sau một số năm nào đó khối lượng phóng xạ đã cho ban đầu không còn độc hại với con người, biết rằng chất phóng xạ này sẽ không độc hại nữa nếu khối lượng chất phóng xạ còn lại bé hơn $10^{-6}$ g.
	\end{enumerate}
	\loigiai{
		\begin{enumerate}
			\item  Khối lượng chất phóng xạ còn lại sau chu kì bán rã thứ $1$ là $u_1=\dfrac{1}{2}\cdot 1=1$ kg.\\
			Khối lượng chất phóng xạ còn lại sau chu kì bán rã thứ $2$ là $u_2=\dfrac{1}{2}\cdot u_1=\dfrac{1}{2}\cdot \dfrac{1}{2}=\dfrac{1}{2^2}$ kg.\\
			Khối lượng chất phóng xạ còn lại sau chu kì bán rã thứ $3$ là $u_3=\dfrac{1}{2}\cdot u_2=\dfrac{1}{2}\cdot \dfrac{1}{4}=\dfrac{1}{2^3}$ kg.\\
			Khối lượng chất phóng xạ còn lại sau chu kì bán rã thứ $n$ là $u_n=\dfrac{1}{2^n}$ kg.\\
			\item $\lim \limits_{n \to +\infty}u_n=\lim \limits_{n \to +\infty}\dfrac{1}{2^n}=\lim\left(\dfrac{1}{2}\right)^n=0$.
			\item Chất phóng xạ sẽ không độc hại nữa nếu khối lượng chất phóng xạ còn lại bé hơn $10^{-6}~\mathrm{g}=10^{-9}$ kg
			$$\Leftrightarrow u_n<10^{-9}\Leftrightarrow\dfrac{1}{2^n}<10^{-9}\Leftrightarrow 2^n>10^9\Leftrightarrow n\geq 30.$$
			Vậy sau ít nhất $30$ chu kì bằng $30\cdot 24000=720000$ năm thì khối lượng phóng xạ đã cho ban đầu không còn độc hại với con người nữa.
		\end{enumerate}
	}
\end{vd}
\begin{vd}%[1C3G1-6]
	Gọi $C$ là nữa đường tròn đường kính $AB=2R$.\\
	$C_1$ là đường gồm hai nửa đường tròn đường kính $\dfrac{AB}{2}$,\\
	$C_2$ là đường gồm bốn nửa đường tròn đường kính $\dfrac{AB}{4},\cdots$\\
	$C_n$ là đường gồm $2^n$ nửa đường tròn đường kính $\dfrac{AB}{2^n},\cdots$
	\immini{	Gọi $p_n$ là độ dài của $C_n$, $S_n$ là diện tích hình phẳng giới hạn bởi $C_n$ và đoạn thẳng $AB$.
		\begin{enumEX}{1}
			\item Tính $p_n$, $S_n$.
			\item Tính giới hạn của các dãy số $(p_n)$ và $(S_n)$.
		\end{enumEX}
	}{
		\begin{tikzpicture}[declare function={r=3;}]
			\path
			(0,0) coordinate (O)
			(0:r) coordinate (A)
			(180:r) coordinate (B)
			($(A)!.5!(O)$) coordinate (M)
			($(B)!.5!(O)$) coordinate (N)
			($(B)!.5!(N)$) coordinate (I)
			($(N)!.5!(O)$) coordinate (J)
			($(O)!.5!(M)$) coordinate (K)
			($(M)!.5!(A)$) coordinate (H)
			($(B)!.5!(I)$) coordinate (Q);
			\draw (A) arc(0:180:r) (A) arc(0:180:r/2) (O) arc(0:180:r/2) (A) arc(0:180:r/4) (A) arc(0:180:r/4)(M) arc(0:180:r/4) (O) arc(0:180:r/4) (N) arc(0:180:r/4)
			(A) arc(0:180:r/8) (H) arc(0:180:r/8) (M) arc(0:180:r/8) (K) arc(0:180:r/8) (O) arc(0:180:r/8) (J) arc(0:180:r/8) (N) arc(0:180:r/8) (I) arc(0:180:r/8);
			\draw (A)--(B);
			\path 
			($(O)+(90:0.9*r)$) node{$C$}
			($(N)+(90:0.4*r)$) node[scale=0.8]{$C_1$}
			($(I)+(90:0.18*r)$) node[scale=0.7]{$C_2$}
			($(Q)+(90:0.06*r)$) node[scale=0.6]{$C_3$};
			\foreach \x/\y in {A/-90, B/-90}{\fill (\x) circle(1pt) ($(\x)+(\y:0.3cm)$) node{$\x$};}
		\end{tikzpicture}
	}
	
	\loigiai{
		\begin{enumerate}
			\item Ta có  
			\begin{eqnarray*}
				p_n&=&2^n \cdot \pi r=2^n\cdot\pi \cdot \dfrac{AB}{2\cdot2^n}\\
				&=&\dfrac{\pi AB}{2}\\
				&=&\dfrac{\pi \cdot 2R}{2}\\
				&=&\pi R.
			\end{eqnarray*}
			\begin{eqnarray*}
				S_n&=&2^n \cdot \dfrac{1}{2}\pi r^2\\
				&=&2^n \cdot \dfrac{1}{2}\pi \left(\dfrac{AB}{2\cdot2^n}\right)^2\\
				&=&2^n \cdot \dfrac{1}{2}\pi \left(\dfrac{2R}{2\cdot2^n}\right)^2\\
				&=&2^n \cdot \dfrac{1}{2}\pi \dfrac{R^2}{(2^n)^2}\\
				&=&\dfrac{\pi R^2}{2^{n+1}}.	
			\end{eqnarray*}			
			\item $\lim \limits_{n \to +\infty}p_n=\lim \limits_{n \to +\infty}\left(\pi R\right) = \pi R$.\\
			$\lim \limits_{n \to +\infty}S_n=\lim \limits_{n \to +\infty}\dfrac{\pi R^2}{2^{n+1}}=0$ (Vì $\lim \limits_{n \to +\infty}\left(\pi R^2\right)=\pi R^2$ và $\lim \limits_{n \to +\infty}2^{n+1}=+\infty$).
		\end{enumerate}
	}
\end{vd}
\begin{vd}%[1D4K1-6]
	\immini{Từ độ cao $55,8 \mathrm{~m}$ của tháp nghiêng Pisa nước Ý, người ta thả một quả bóng cao su chạm xuống đất hình bên dưới. Giả sử mỗi lần chạm đất quả bóng lại nảy lên độ cao bằng $\dfrac{1}{10}$ độ cao mà quả bóng đạt được trước đó. Gọi $S_n$ là tổng độ dài quãng đường di chuyển của quả bóng tính từ lúc thả ban đầu cho đến khi quả bóng đó chạm đất $n$ lần. Tính $\lim \limits_{n \to +\infty}S_n$.}{
		
		\definecolor{lightcornflowerblue}{rgb}{0.6, 0.81, 0.93}
		\definecolor{cadmiumgreen}{rgb}{0.0, 0.42, 0.24}
		\definecolor{trueblue}{rgb}{0.0, 0.45, 0.81}
		\definecolor{tumbleweed}{rgb}{0.87, 0.67, 0.53}%màu cát
		\begin{tikzpicture}[line join=round, line cap=round,scale=1,transform shape]
			\clip (-4,-3.5) rectangle (4,3.5);
			\tikzset{thap/.pic={%Xây từ dưới lên
					\def\T{ 
						(-.88,-1.78)%1
						..controls +(40:.17) and +(120:.1) ..  (.686,-1.91)--(.68,-1.94)
						..controls +(120:.1) and +(40:.12) ..  (-.87,-1.81)
						--cycle
						
						(-.8,-1.2)%2
						..controls +(40:.14) and +(120:.14) ..  (.72,-1.34)--(.7,-1.36)
						..controls +(120:.1) and +(40:.12) ..  (-.78,-1.23)
						--cycle
						
						(-.74,-.65)%3
						..controls +(40:.22) and +(120:.17) ..  (.74,-.78)--(.72,-.8)
						..controls +(120:.15) and +(40:.15) ..  (-.72,-.67)
						--cycle
						
						(-.68,-.13)%4
						..controls +(40:.34) and +(110:.2) ..  (.8,-.26)--(.78,-.28)
						..controls +(110:.17) and +(40:.27) ..  (-.66,-.15)
						--cycle
						
						(-.65,.36)%5
						..controls +(40:.44) and +(120:.3) ..  (.85,.24)--(.83,.22)
						..controls +(130:.35) and +(40:.35) ..  (-.63,.34)
						--cycle
						
						(-.58,.88)%6
						..controls +(40:.44) and +(120:.3) ..  (.87,.78)--(.85,.76)
						..controls +(130:.35) and +(40:.35) ..  (-.56,.86)
						--cycle
						
						(-.52,1.35)%7
						..controls +(120:.32) and +(110:.47) ..  (.93,1.26)--(.9,1.25)
						..controls +(110:.43) and +(110:.27) ..  (-.49,1.35)
						--cycle
						
						(-.31,2)%8
						..controls +(120:.47) and +(55:.46) ..  (.76,1.94)--(.74,1.94)
						..controls +(75:.31) and +(115:.28) ..  (-.28,2)
						--cycle
						
						(-.48,1.5)
						..controls +(40:.33) and +(110:.2) ..  (.86,1.4)
						(-.476,1.52)
						..controls +(40:.33) and +(110:.2) ..  (.86,1.42)
						(-.472,1.54)
						..controls +(40:.33) and +(110:.2) ..  (.86,1.44)
						(-.468,1.56)
						..controls +(40:.33) and +(110:.2) ..  (.86,1.46)
						
						(-.28,2.2)
						..controls +(40:.27) and +(140:.2) ..  (.74,2.16)
						(-.276,2.22)
						..controls +(40:.27) and +(140:.2) ..  (.74,2.18)
						(-.272,2.24)
						..controls +(40:.27) and +(140:.2) ..  (.74,2.2)
						(-.268,2.26)
						..controls +(40:.27) and +(140:.2) ..  (.74,2.22)
						;}
					\draw \T;
					%\fill[tumbleweed] \T;
			}}
			
			\tikzset{rao/.pic={%Rào
					\def\R{ 
						(-.48,1.5)
						..controls +(40:.33) and +(110:.2) ..  (.86,1.4)
						(-.476,1.52)
						..controls +(40:.33) and +(110:.2) ..  (.86,1.42)
						(-.472,1.54)
						..controls +(40:.33) and +(110:.2) ..  (.86,1.44)
						(-.468,1.56)
						..controls +(40:.33) and +(110:.2) ..  (.86,1.46)
						
						(-.28,2.2)
						..controls +(40:.27) and +(140:.2) ..  (.74,2.16)
						(-.276,2.22)
						..controls +(40:.27) and +(140:.2) ..  (.74,2.18)
						(-.272,2.24)
						..controls +(40:.27) and +(140:.2) ..  (.74,2.2)
						(-.268,2.26)
						..controls +(40:.27) and +(140:.2) ..  (.74,2.22)
						;}
					\draw \R;
					%\fill[tumbleweed] \R;
			}}
			
			\tikzset{co/.pic={%Cờ
					\def\C{ 
						(0.1,2.3)--(.1,2.9)
						(.1,2.82)
						..controls +(-40:.2) and +(140:.22) ..  (.2,2.6)
						..controls +(-60:0) and +(100:.1) ..(.16,2.5)
						..controls +(140:.1) and +(-30:0) ..  (.1,2.53)--cycle
						;}
					\draw \C;
					\fill[red] \C;
			}}
			
			\tikzset{cong0/.pic={%Đường cong trệt
					\def\D0{ 
						(.68,-1.96)
						..controls +(120:.1) and +(40:.12) ..  (-.87,-1.81)--
						(-.87,-1.94)
						..controls +(80:.12) and +(95:0.12) ..  (-.68,-2.06)
						..controls +(80:.12) and +(85:0.3) ..  (-.4,-2.12)
						..controls +(80:.25) and +(95:0.25) ..  (-.02,-2.12)
						..controls +(80:.25) and +(95:0.25) ..  (.27,-2.14)
						..controls +(80:.12) and +(95:0.12) ..  (.5,-2.14)
						..controls +(70:.12) and +(95:0.02) ..  (.66,-2.1)--cycle
						;}
					\draw \D0;
					\fill[tumbleweed] \D0;
			}}
			
			\tikzset{cong1/.pic={%Đường cong 1
					\def\D1{ 
						(.7,-1.36)
						..controls +(120:.1) and +(40:.12) ..  (-.78,-1.23)--
						(-.78,-1.35)
						..controls +(80:.12) and +(95:0.12) ..  (-.72,-1.35)
						..controls +(80:.12) and +(95:0.12) ..  (-.64,-1.35)
						..controls +(80:.12) and +(95:0.12) ..  (-.55,-1.35)
						..controls +(80:.12) and +(95:0.12) ..  (-.44,-1.36)
						..controls +(80:.12) and +(95:0.12) ..  (-.3,-1.37)
						..controls +(80:.12) and +(95:0.12) ..  (-.14,-1.38)
						..controls +(80:.12) and +(95:0.12) ..  (0.02,-1.39)
						..controls +(80:.12) and +(95:0.12) ..  (0.18,-1.4)
						..controls +(80:.12) and +(95:0.12) ..  (0.32,-1.42)
						..controls +(70:.12) and +(95:0.12) ..  (0.42,-1.44)
						..controls +(70:.12) and +(95:0.12) ..  (0.52,-1.46)
						..controls +(70:.12) and +(95:0.12) ..  (0.6,-1.47)
						..controls +(70:.12) and +(95:0.12) ..  (0.68,-1.5)--cycle
						;}
					\draw \D1;
					\fill[tumbleweed] \D1;
			}}
			
			\tikzset{cong2/.pic={%Đường cong 2
					\def\D2{ 
						(.72,-.8)
						..controls +(120:.15) and +(40:.15) ..  (-.72,-.67)--
						(-.72,-.79)
						..controls +(80:.12) and +(95:0.12) ..  (-.66,-.79)
						..controls +(80:.12) and +(95:0.12) ..  (-.58,-.79)
						..controls +(80:.12) and +(95:0.12) ..  (-.48,-.79)
						..controls +(80:.12) and +(95:0.12) ..  (-.38,-.8)
						..controls +(80:.12) and +(95:0.12) ..  (-.23,-.8)
						..controls +(80:.12) and +(95:0.12) ..  (-0.08,-.81)
						..controls +(80:.12) and +(95:0.12) ..  (0.08,-.82)
						..controls +(80:.12) and +(95:0.12) ..  (0.22,-.84)
						..controls +(80:.12) and +(95:0.12) ..  (0.36,-.85)
						..controls +(70:.12) and +(95:0.12) ..  (0.48,-.87)
						..controls +(70:.12) and +(95:0.12) ..  (0.58,-.89)
						..controls +(70:.12) and +(95:0.12) ..  (0.66,-.9)
						..controls +(70:.12) and +(95:0.12) ..  (0.71,-.92)--cycle
						;}
					\draw \D2;
					\fill[tumbleweed] \D2;
			}}
			
			\tikzset{cong3/.pic={%Đường cong 3
					\def\D3{ 
						(.78,-.28)
						..controls +(110:.17) and +(40:.27) ..  (-.66,-.15)--
						(-.66,-.26)
						..controls +(80:.12) and +(95:0.12) ..  (-.6,-.25)
						..controls +(80:.12) and +(95:0.12) ..  (-.52,-.24)
						..controls +(80:.12) and +(95:0.12) ..  (-.44,-.23)
						..controls +(80:.12) and +(95:0.12) ..  (-.32,-.23)
						..controls +(80:.12) and +(95:0.12) ..  (-.18,-.23)
						..controls +(80:.12) and +(95:0.12) ..  (-.04,-.24)
						..controls +(80:.12) and +(95:0.12) ..  (0.13,-.25)
						..controls +(80:.12) and +(95:0.12) ..  (0.26,-.26)
						..controls +(70:.12) and +(95:0.12) ..  (0.42,-.29)
						..controls +(70:.12) and +(95:0.12) ..  (0.54,-.31)
						..controls +(70:.12) and +(95:0.12) ..  (0.63,-.33)
						..controls +(70:.12) and +(95:0.12) ..  (0.7,-.35)
						..controls +(70:.12) and +(95:0.12) ..  (0.765,-.38)--cycle
						;}
					\draw \D3;
					\fill[tumbleweed] \D3;
			}}
			
			\tikzset{cong4/.pic={%Đường cong 4
					\def\D4{ 
						(.83,.22)
						..controls +(130:.35) and +(40:.35) ..  (-.63,.34)--
						(-.63,.23)
						..controls +(80:.12) and +(95:0.12) ..  (-.57,.25)
						..controls +(80:.12) and +(95:0.12) ..  (-.49,.26)
						..controls +(80:.12) and +(95:0.12) ..  (-.4,.28)
						..controls +(80:.12) and +(95:0.12) ..  (-.27,.3)
						..controls +(80:.12) and +(95:0.12) ..  (-.15,.3)
						..controls +(80:.12) and +(95:0.12) ..  (0.02,.3)
						..controls +(80:.12) and +(95:0.12) ..  (0.16,.29)
						..controls +(80:.12) and +(95:0.12) ..  (0.31,.27)
						..controls +(70:.12) and +(95:0.12) ..  (0.45,.25)
						..controls +(70:.12) and +(95:0.12) ..  (0.58,.23)
						..controls +(70:.12) and +(95:0.12) ..  (0.67,.2)
						..controls +(70:.12) and +(95:0.12) ..  (0.75,.17)
						..controls +(70:.12) and +(95:0.12) ..  (0.8,.14)--cycle
						;}
					\draw \D4;
					\fill[tumbleweed] \D4;
			}}
			
			\tikzset{cong5/.pic={%Đường cong 5
					\def\D5{ 
						(.85,.76)
						..controls +(130:.35) and +(40:.35) ..  (-.56,.86)--
						(-.56,.74)
						..controls +(80:.12) and +(95:0.12) ..  (-.5,.77)
						..controls +(80:.12) and +(95:0.12) ..  (-.44,.79)
						..controls +(80:.12) and +(95:0.12) ..  (-.34,.81)
						..controls +(80:.12) and +(95:0.12) ..  (-.22,.83)
						..controls +(80:.12) and +(95:0.12) ..  (-0.08,.85)
						..controls +(80:.12) and +(95:0.12) ..  (0.06,.85)
						..controls +(80:.12) and +(95:0.12) ..  (0.22,.83)
						..controls +(70:.12) and +(95:0.12) ..  (0.36,.81)
						..controls +(70:.12) and +(95:0.12) ..  (0.5,.78)
						..controls +(70:.12) and +(95:0.12) ..  (0.62,.74)
						..controls +(70:.12) and +(95:0.12) ..  (0.73,.72)
						..controls +(70:.12) and +(95:0.12) ..  (0.82,.68)--cycle
						;}
					\draw \D5;
					\fill[tumbleweed] \D5;
			}}
			
			\tikzset{cong6/.pic={%Đường cong 6
					\def\D6{ 
						(.9,1.25)
						..controls +(110:.43) and +(110:.27) ..  (-.49,1.35)--
						(-.49,1.3)
						..controls +(80:.12) and +(95:0.12) ..  (-.45,1.33)
						..controls +(80:.12) and +(95:0.12) ..  (-.38,1.33)
						..controls +(80:.12) and +(95:0.12) ..  (-.28,1.35)
						..controls +(80:.12) and +(95:0.12) ..  (-.16,1.37)
						..controls +(80:.12) and +(95:0.12) ..  (-0.04,1.38)
						..controls +(80:.12) and +(95:0.12) ..  (0.12,1.38)
						..controls +(80:.12) and +(95:0.12) ..  (0.26,1.38)
						..controls +(70:.12) and +(95:0.12) ..  (0.42,1.37)
						..controls +(70:.12) and +(95:0.12) ..  (0.55,1.34)
						..controls +(70:.12) and +(95:0.12) ..  (0.67,1.3)
						..controls +(70:.12) and +(95:0.12) ..  (0.78,1.25)
						..controls +(70:.12) and +(95:0.12) ..  (0.84,1.21)
						..controls +(70:.12) and +(95:0.12) ..  (0.88,1.16)--cycle
						;}
					\draw \D6;
					\fill[tumbleweed] \D6;
			}}
			
			\tikzset{cong7/.pic={%Đường cong 7
					\def\D7{ 
						(.74,1.94)
						..controls +(75:.31) and +(115:.28) ..  (-.28,2)--
						(-.28,1.94)
						..controls +(80:.12) and +(95:0.12) ..  (-.24,1.98)
						..controls +(80:.12) and +(95:0.12) ..  (-.14,2.04)
						..controls +(80:.12) and +(95:0.12) ..  (0.02,2.08)
						..controls +(80:.12) and +(95:0.12) ..  (0.2,2.08)
						..controls +(80:.12) and +(95:0.12) ..  (0.4,2.05)
						..controls +(80:.12) and +(95:0.12) ..  (0.56,2)
						..controls +(80:.12) and +(95:0.12) ..  (0.66,1.96)
						..controls +(70:.12) and +(95:0.12) ..  (0.74,1.9)--cycle
						;}
					\draw \D7;
					\fill[tumbleweed] \D7;
			}}
			
			\tikzset{cua/.pic={%Cửa
					\def\W{ 
						(-.7,-2.85)--(-.63,-2.25)
						..controls +(70:.1) and +(100:.1) ..  (-.46,-2.25)--(-.52,-2.85)
						--cycle
						;}
					\draw \W;
					\fill[tumbleweed] \W;
			}}
			
			\tikzset{vien/.pic={%viền ngoài
					\def\V{ 
						(.9,1.25)
						..controls +(110:.43) and +(110:.27) ..  (-.49,1.35)--(-.98,-2.85)--(.59,-2.85)--cycle
						
						(.74,1.94)
						..controls +(75:.31) and +(115:.28) ..  (-.28,2)--(-.32,1.54)
						..controls +(115:.2) and +(75:.13) ..  (.72,1.48)--cycle
						;}
					\draw \V;
					\fill[gray!50!] \V;
			}}
			
			\tikzset{soc/.pic={%Thanh sọc
					\def\S{ 
						(.82,1.36)--(.54,-1.85)--(.58,-1.85)--(.86,1.36)
						--cycle
						;}
					\draw \S;
					\fill[tumbleweed!40!] \S;
			}}
			
			\tikzset{soc2/.pic={%Thanh sọc lớn
					\def\S{ 
						(.54,-1.95)--(.44,-2.85)--(.48,-2.85)--(.58,-1.95)
						--cycle
						;}
					\draw \S;
					\fill[tumbleweed!40!] \S;
			}}
			
			\tikzset{soc3/.pic={%Thanh sọc
					\def\S{ 
						(.54,2)--(.57,2)--(.54,1.5)--(.51,1.5)
						--cycle
						(.66,2)--(.69,2)--(.66,1.5)--(.63,1.5)
						--cycle
						(-.26,2)--(-.23,2)--(-.26,1.5)--(-.29,1.5)
						--cycle
						;}
					\draw \S;
					\fill[tumbleweed!40!] \S;
			}}
			
			\tikzset{thoi/.pic={%Hình thoi
					\def\T{ 
						(0.15,-1.98)--(.24,-2.12)--(0.13,-2.22)--(0.04,-2.1)
						--cycle
						;}
					\draw \T;
					\fill[tumbleweed!40!] \T;
			}}
			
			\tikzset{cay/.pic={%Cây
					\def\T{ 
						(-1.73,-2.85)
						..controls +(40:.03) and +(40:.01) ..  (-1.74,-2.7)
						..controls +(140:.04) and +(60:.02) ..  (-1.78,-2.6)
						..controls +(140:.04) and +(60:.02) ..  (-1.8,-2.55)
						..controls +(100:.03) and +(70:.02) ..  (-1.82,-2.5)
						..controls +(140:.04) and +(60:.02) ..  (-1.815,-2.4)
						..controls +(140:.04) and +(60:.02) ..  (-1.818,-2.3)
						..controls +(85:.03) and +(-30:.03) ..  (-1.8,-2.2)
						..controls +(80:.04) and +(50:.02) ..  (-1.78,-2)
						..controls +(60:.04) and +(120:.02) ..  (-1.76,-1.9)
						..controls +(80:.03) and +(-100:.02) ..  (-1.74,-1.8)
						..controls +(80:.01) and +(-100:.02) ..  (-1.72,-1.9)
						..controls +(60:.02) and +(-120:.02) ..  (-1.7,-2)
						..controls +(-80:.02) and +(110:.03) ..  (-1.66,-2.2)
						..controls +(-100:.02) and +(60:.03) ..  (-1.64,-2.3)
						..controls +(-30:.02) and +(70:.02) ..  (-1.6,-2.44)
						..controls +(-100:.02) and +(70:.02) ..  (-1.63,-2.52)
						..controls +(-60:.02) and +(70:.01) ..  (-1.66,-2.6)
						..controls +(60:.01) and +(70:.02) ..  (-1.7,-2.7)
						..controls +(-120:.02) and +(120:.03) ..  (-1.68,-2.85)
						;}
					\draw \T;
					\fill[cadmiumgreen!90!] \T;
			}}
			\fill[cadmiumgreen!80!] (-4,-3.5) rectangle (8,-2.2);
			\fill[lightcornflowerblue] (-4,3.5) rectangle (8,-2.2);
			\path 
			(0,0)pic[scale=1]{cay}(.35,0)pic[scale=.9]{cay}(1.05,0.6)pic[scale=1.2]{cay}(-.5,-1)pic[scale=.6]{cay}
			(3,0)pic[scale=1]{cay}(2.8,0)pic[scale=1]{cay}
			(0,0)pic[scale=1]{vien}
			(0,0)pic[scale=1]{soc} (0,0)pic[scale=1]{soc}(-.15,0)pic[scale=1]{soc}(-.3,0)pic[scale=1]{soc}(-.45,0)pic[scale=1]{soc}(-.6,0)pic[scale=1]{soc}(-.75,0)pic[scale=1]{soc}(-.9,0)pic[scale=1]{soc}(-1.02,0)pic[scale=1]{soc}(-1.12,0)pic[scale=1]{soc}(-1.22,0)pic[scale=1]{soc}(-1.32,0)pic[scale=1]{soc}
			
			(0,0)pic[scale=1]{soc3}
			(0,0)pic[scale=1]{thap}(0,0)pic[scale=1]{co}
			(.4,3.6)pic[scale=.7,rotate=3]{cua}
			(.8,3.6)pic[scale=1.2,rotate=7,yscale=.6]{cua}
			(.3,3.4)pic[scale=1.1,rotate=7,yscale=.6]{cua}
			(0,-.04)pic[scale=1]{thoi}(0.25,-.05)pic[scale=1]{thoi}(-0.35,0)pic[scale=1]{thoi}(-.67,0)pic[scale=1]{thoi}(-.9,0)pic[scale=1]{thoi}
			(-1.22,0)pic[scale=1]{soc2}(-1.36,0)pic[scale=1]{soc2}
			(-.94,0)pic[scale=1]{soc2}(-.56,0)pic[scale=1]{soc2}
			(-.56,0)pic[scale=1]{soc2}(.08,0)pic[scale=1]{soc2}(-.04,0)pic[scale=1]{soc2}(-.28,0)pic[scale=1]{soc2}
			(0,0)pic[scale=1]{cong0} (0,0.02)pic[scale=1]{cong0}%
			(0,0)pic[scale=1]{cong1} (0,0.02)pic[scale=1]{cong1}%
			(0,0)pic[scale=1]{cong2} (0,0.02)pic[scale=1]{cong2}%
			(0,0)pic[scale=1]{cong3}(0,0.02)pic[scale=1]{cong3}
			(0,0)pic[scale=1]{cong4}(0,0.02)pic[scale=1]{cong4}
			(0,0)pic[scale=1]{cong5}(0,0.02)pic[scale=1]{cong5}
			(0,0)pic[scale=1]{cong6}(0,0.02)pic[scale=1]{cong6}
			(0,0)pic[scale=1]{cong7}(0,0.02)pic[scale=1]{cong7}
			(0,0)pic[scale=1]{cua}
			
			(0,0)pic[scale=1]{rao}
			;		
		\end{tikzpicture}
	}
	\loigiai{Mỗi khi chạm đất quả bóng lại nảy lên một độ cao bằng $\dfrac{1}{10}$ độ cao của lần rơi ngay trước đó và sau đó lại rơi xuống từ độ cao thứ hai này. Do đó, độ dài hành trình của quả bóng kể từ thời điểm rơi ban đầu đến:\\    
		Thời điểm chạm đất lần thứ nhất là $d_1=55{,}8$.\\
		Thời điềm chạm đất lần thứ hai là $d_2=55{,}8+2\cdot \dfrac{55{,}8}{10}$.\\
		Thời điểm chạm đất lần thứ ba là $d_3=55{,}8+2 \cdot\dfrac{55{,}8}{10}+2\cdot \dfrac{55{,}8}{10^2}$.\\
		Thời điểm chạm đất lần thứ tư là $d_4=55{,}8+2 \cdot\dfrac{55{,}8}{10}+2\cdot \dfrac{55,8}{10^2}+2\cdot \dfrac{55{,}8}{10^3}$.\\
		$\ldots$\\
		Thời điểm chạm đất lần thứ $n~(n>1)$ là
		$$d_n=55{,}8+2\cdot55{,}8+2\cdot \frac{55{,}8}{10^2}+2\cdot \frac{55{,}8}{10^3}+\ldots+2\cdot \frac{55{,}8}{10^{n-1}}.$$
		Do đó, quãng đường mà quả bóng đi được kể từ thời điềm rơi đến khi nằm yên trên mặt đất là:
		$$ d=55{,}8+2.55{,}8+2\cdot \frac{55{,}8}{10^2}+2\cdot \frac{55{,}8}{10^3}+\ldots+2\cdot \frac{55{,}8}{10^{n-1}}+\ldots=\lim \limits_{n \to +\infty}d_n.$$
		Vì $2\cdot \dfrac{55{,}8}{10} ; 2\cdot \dfrac{55{,}8}{10^2} ; 2\cdot \dfrac{55{,}8}{10^3}; \ldots ; 2\cdot \dfrac{55{,}8}{10^{n-1}}; \ldots$ là một cấp số nhân lùi vô hạn với công bội $q=\dfrac{1}{10}$ nên ta có:
		$$ 2 \cdot\dfrac{55,8}{10}+2\cdot \dfrac{55{,}8}{10^2}+2\cdot \dfrac{55{,}8}{10^3}+\ldots+2\cdot \dfrac{55{,}8}{10^{n-1}}+\ldots=\dfrac{2\cdot \dfrac{55{,}8}{10}}{1-\dfrac{1}{10}}=12{,}4.$$
		Vậy $d=55{,}8+12{,}4=68{,}2$ m.
	}
\end{vd}
\begin{vd}%[0D1Y1-1]
	Cho một tam giác đều $A B C$ cạnh $a$. Tam giác $A_1 B_1 C_1$ có các đỉnh là trung điểm các cạnh của tam giác $A B C$, tam giác $A_2 B_2 C_2$ có các đỉnh là trung điểm các cạnh của tam giác $A_1 B_1 C_1, \ldots$, tam giác $A_{n+1} B_{n+1} C_{n+1}$ có các đỉnh là trung điểm các cạnh của tam giác $A_n B_n C_n, \ldots$ Gọi $p_1, p_2, \ldots, p_n, \ldots$ và $S_1, S_2, \ldots, S_n, \ldots$ theo thứ tự là chu vi và diện tích của các tam giác $A_1 B_1 C_1, A_2 B_2 C_2, \ldots, A_n B_n C_n, \ldots$.
	\begin{listEX}
		\item[a)] Tìm giới hạn của các dãy số $\left(p_n\right)$ và $\left(S_n\right)$.
		\item[b)] Tìm các tổng $p_1+p_2+\ldots+p_n+\ldots$ và $S_1+S_2+\ldots+S_n+\ldots$.
	\end{listEX}
	\loigiai{
		\begin{listEX}
			\item[a)] Ta có $p_1, p_2, \ldots, p_n, \ldots$ lần lượt là chu vi của các tam giác $A_1 B_1 C_1, A_2 B_2 C_2, \ldots, A_n B_n C_n, \ldots$
			$$\begin{aligned}
				& p_1=3 a \\
				& p_2=3 \cdot \frac{1}{2} a \\
				& \ldots \\
				& p_n=3 \cdot \frac{1}{2^{n-1}} a 
			\end{aligned} $$
			suy ra $\lim \limits_{n \to +\infty}p_n=\lim \limits_{n \to +\infty}3 \cdot \dfrac{1}{2^{n-1}} a=0$.
			$$ \begin{aligned}
				& S_1=\frac{a^2 \sqrt{3}}{4} \\
				& S_2=\frac{1}{4} \frac{a^2 \sqrt{3}}{4} \\
				& \ldots \\
				& S_n=\frac{1}{4^{n-1}} \cdot \frac{a^2 \sqrt{3}}{4}
			\end{aligned}$$
			suy ra $\lim \limits_{n \to +\infty}S_n=\lim \limits_{n \to +\infty}\dfrac{1}{4^{n-1}} \cdot \dfrac{a^2 \sqrt{3}}{4}=0$.
			\item[b)] Dựa vào dữ kiện đề bài suy ra tổng $\left(p_n\right)$ là tổng của cấp số nhân lùi vô hạn với công bội $q=\dfrac{1}{2}$ và
			$ p_1+p_2+\ldots+p_n+\ldots=\lim \limits_{n \to +\infty}\left(p_n\right)=\dfrac{p_1}{1-q}=\dfrac{3 a}{1-\frac{1}{2}}=6a.$\\
			Dựa vào dữ kiện đề bài suy ra tổng $\left(S_n\right)$ là tổng của cấp số nhân lùi vô hạn với công bội $q=\dfrac{1}{4}$ và
			$S_1+S_2+\ldots+S_n+\ldots=\lim \limits_{n \to +\infty}\left(S_n\right)=\dfrac{S_1}{1-q}=\dfrac{\frac{a^2 \sqrt{3}}{4}}{1-\frac{1}{4}}=\dfrac{a^2 \sqrt{3}}{12}$.
		\end{listEX}
	}
\end{vd}
% \subsubsection{Bài tập rèn luyện} 
% % \subsubsection{Bài tập tự luận}
% \begin{bt}%[1T3K1-5]
% 	Từ tờ giấy, cắt một hình tròn bán kính $R$ (cm) như Hình $3a$. Tiếp theo, cắt hai hình~tròn
% 	\immini{
% 		bán kính $\dfrac{R}{2}$ rồi chồng lên hình tròn đầu tiên như Hình $3b$. Tiếp theo, cắt bốn hình tròn bán  kính $\dfrac{R}{4}$ 
% 		rồi chồng lên các hình trước như Hình $3c$. Cứ thế tiếp tục mãi. Tính tổng diện tích của các hình tròn.
% 	}{
% 		\begin{tikzpicture}[>=stealth,line join=round,line cap=round,font=\footnotesize,scale=1,declare function={r=1.5;}]
% 			\begin{scope}
% 				\draw[fill=blue] (0,0)circle(r);
% 				\path (0,-r) node[below]{$a)$};
% 			\end{scope}
% 			\begin{scope}[xshift={3.5cm}]
% 				\draw[fill=blue] (0,0)circle(r);
% 				\draw[fill=yellow] (r/2,0)circle(r/2);
% 				\draw[fill=yellow] (-r/2,0)circle(r/2);
% 				\path (0,-r) node[below]{$b)$};
% 			\end{scope}
% 			\begin{scope}[xshift={7cm}]
% 				\draw[fill=blue] (0,0)circle(r);
% 				\draw[fill=yellow] (r/2,0)circle(r/2);
% 				\draw[fill=yellow] (-r/2,0)circle(r/2);
% 				\foreach \i in {0,1,2,3}
% 				\draw[fill=green,shift={(r/2*\i,0)}] (-3*r/4,0)circle(r/4);
% 				\path (0,-r) node[below]{$c)$};
% 			\end{scope}
% 			\path (current bounding box.south) node[below]{Hình $3$};
% 		\end{tikzpicture}
% 	}
% 	\loigiai{
% 		Diện tích của các hình tròn trong các lần cắt là
% 		\begin{enumerate}
% 			\item Lần thứ 1: $S_1=\pi R^2$.
% 			\item  Lần thứ 2: $S_2=2\cdot \pi \left(\dfrac{R}{2}\right)^2= \dfrac{\pi R^2}{2}$.
% 			\item  Lần thứ 3: $S_2=4\cdot \pi \left(\dfrac{R}{4}\right)^2= \dfrac{\pi R^2}{2^2}$.	
% 			\item Lần thứ $n$: $S_n= \dfrac{\pi R^2}{2^{n-1}}$.
% 		\end{enumerate}
% 		Do đó  diện tích các hình tròn lập thành một cấp số nhân lùi vô hạn có số hạng đầu $S_1=\pi R^2$ và công bội $q=\dfrac{1}{2}$ nên tổng diện tích các hình tròn là 
% 		\[ S_1+S_2+\cdots=\dfrac{\pi R^2}{1-\dfrac{1}{2}}=2\pi R^2. \]
% 	}
% \end{bt}
% \begin{bt}%[1T3K1-5]
% 	\immini{
% 		Từ hình vuông đầu tiên có cạnh bằng $1$ (đơn vị độ dài), nối các trung điểm của bốn cạnh để có hình vuông thứ hai. Tiếp tục nối các trung điểm của bốn cạnh của hình vuông thứ hai để được hình vuông thứ ba. Cứ tiếp tục làm như thế, nhận được một dãy hình vuông (xem Hình $5$).
% 	}{\hspace*{.5cm}
% 		\begin{tikzpicture}[>=stealth,line join=round,line cap=round,font=\footnotesize,scale=1,declare function={a=3;}]
% 			\foreach \i in {1,2,...,7}{
% 				\pgfmathsetmacro\r{a*(sin(45))^(\i-1)}
% 				\pgfmathsetmacro{\j}{int(mod(\i,2))}
% 				\ifnum \j=1
% 				\draw (-\r/2,-\r/2) rectangle (\r/2,\r/2);
% 				\else
% 				\draw[rotate=45] (-\r/2,-\r/2) rectangle (\r/2,\r/2);
% 				\fi
% 			}
% 			\path (current bounding box.south) node[below]{Hình $5$};
% 		\end{tikzpicture}\hspace*{1cm}
% 	}
% 	\begin{enumerate}
% 		\item Kí hiệu $a_n$ là diện tích của hình vuông thứ $n$ và $S_n$ là tổng diện tích của $n$ hình vuông đầu tiên. Viết công thức tính $a_n$, $S_n$ ($n=1,2,3, \ldots$) và tìm $\lim \limits_{n \to +\infty}S_n$ (giới hạn này nếu có được gọi là tổng diện tích của các hình vuông).
% 		\item Kí hiệu $p_n$ là chu vi của hình vuông thứ $n$ và $Q_n$ là tổng chu vi của $n$ hình vuông đầu tiên. Viết công thức tính $p_n$ và $Q_n$ $(n=1,2,3, \ldots)$ và tìm $\lim \limits_{n \to +\infty}Q_n$ (giới hạn này nếu có được gọi là tổng chu vi của các hình vuông).
% 	\end{enumerate}
% 	\loigiai{
% 		\begin{enumerate}
% 			\item Ta có hình vuông thứ nhất có cạnh bằng $1$,
% 			hình vuông thứ hai có cạnh bằng $\dfrac{\sqrt{2}}{2}$.\\
% 			Hình vuông thứ ba có cạnh bằng $\dfrac{1}{2}$.\\
% 			Suy ra	hình vuông thứ $n$ có cạnh bằng $\left(\dfrac{\sqrt{2}}{2}\right)^{n-1}$.\\
% 			Diện tích của hình vuông thứ $n$ là $a_n=\left(\dfrac{\sqrt{2}}{2}\right)^{n-1}\cdot \left(\dfrac{\sqrt{2}}{2}\right)^{n-1}=\left(\dfrac{1}{2}\right)^{n-1}$.\\
% 			Tổng diện tích của $n$ hình vuông đầu tiên là tổng của cấp số nhân có  số hạng đầu $a_1=1$ và công bội $q=\dfrac{1}{2}$ nên
% 			\[ S_n=\dfrac{a_1\left(1-q^n\right)}{1-q}=\dfrac{1-\left(\dfrac{1}{2}\right)^n}{1-\dfrac{1}{2}}=2\left[1-\left(\dfrac{1}{2}\right)^n\right].\]
% 			$\lim \limits_{n \to +\infty}S_n=\lim \limits_{n \to +\infty}2\left[1-\left(\dfrac{1}{2}\right)^n\right]=2 \left[\lim \limits_{n \to +\infty}1-\lim \limits_{n \to +\infty}\left(\dfrac{1}{2}\right)^n\right]=2 $.
% 			\item Hình vuông thứ nhất có chu vi bằng $4$, hình vuông thứ $2$ có chu vi là $2\sqrt{2}$, hình vuông thứ $3$ có chu vi là $2$.\\
% 			Suy ra hình vuông thứ $n$ có chu vi bằng $p_n=4\cdot \left(\dfrac{\sqrt{2}}{2}\right)^{n-1}$.
% 			Tổng chu vi của $n$ hình vuông đầu tiên là tổng của cấp số nhân có  số hạng đầu $p_1=4$ và công bội $q=\dfrac{\sqrt{2}}{2}$ nên 
% 			\[ Q_n=\dfrac{p_1\left(1-q^n\right)}{1-q}=\dfrac{4\left(1-\left(\dfrac{\sqrt{2}}{2}\right)^n\right)}{1-\dfrac{\sqrt{2}}{2}}=\left(8+4\sqrt{2}\right)\left[1-\left(\dfrac{\sqrt{2}}{2}\right)^n\right].\]
% 			$\lim \limits_{n \to +\infty}Q_n=\left(8+4\sqrt{2}\right)\lim \limits_{n \to +\infty}\left[1-\left(\dfrac{\sqrt{2}}{2}\right)^n\right]=\left(8+4\sqrt{2}\right)\left(1-0\right)=8+4\sqrt{2}$.
% 		\end{enumerate}	
% 	}
% \end{bt}

% \begin{bt}%[1T3K1-4]
% 	Xét quá trình tạo ra hình có chu vi vô cực và diện tích bằng $0$ như sau:\\ Bắt đầu bằng một hình vuông $H_0$ cạnh bằng 1 đơn vị độ dài (xem Hình $6a$). Chia hình vuông $H_0$ thành chín hình vuông bằng nhau, bỏ đi bốn hình vuông, nhận được hình $H_1$ (xem Hình $6b$). Tiếp theo, chia mỗi hình vuông của $H_1$ thành chín hình vuông, rồi bỏ đi bốn hình vuông, nhận được hình $H_2$ (xem Hình $6c$). Tiếp tục quá trình này, ta nhận được một dãy hình $H_n$ $(n=1,2,3,\ldots)$.	
% 	\\[1mm]
% 	\centerline{
% 		\begin{tikzpicture}% Muốn vẽ hình Hn thì dùng \hv{n}
% 			\def\a{2}
% 			\pgfmathsetmacro\sh{2*\a *sqrt(2)/3}
% 			\def\hv#1{
% 				\ifnum#1>0
% 				\draw[white,fill=white] 
% 				(-\a/3,\a/3) rectangle (\a/3,\a)
% 				(-\a/3,-\a/3) rectangle (-\a,\a/3)
% 				(-\a/3,-\a/3) rectangle (\a/3,-\a)
% 				(\a/3,-\a/3) rectangle (\a,\a/3)
% 				;
% 				\pgfmathtruncatemacro{\k}{#1-1}
% 				\begin{scope}[scale=1/3]\hv{\k}\end{scope}
% 				\begin{scope}[shift={(45:\sh)},scale=1/3]\hv{\k}\end{scope}
% 				\begin{scope}[shift={(135:\sh)},scale=1/3]\hv{\k}\end{scope}
% 				\begin{scope}[shift={(225:\sh)},scale=1/3]\hv{\k}\end{scope}
% 				\begin{scope}[shift={(315:\sh)},scale=1/3]\hv{\k}\end{scope}
% 				\fi
% 			}
% 			\begin{scope}
% 				\fill[green] (-\a,-\a) rectangle (\a,\a);
% 				\hv{0}
% 				\path (0,-\a)node[below]{$H_0$}
% 				node[below=.5cm]{$a)$};
% 			\end{scope}
% 			\begin{scope}[xshift=4.5cm]
% 				\fill[green] (-\a,-\a) rectangle (\a,\a);
% 				\hv{1}
% 				\path (0,-\a)node[below]{$H_1$}
% 				node[below=.5cm]{$b)$};
% 			\end{scope}
% 			\begin{scope}[xshift=9cm]
% 				\fill[green] (-\a,-\a) rectangle (\a,\a);
% 				\hv{2}
% 				\path (0,-\a)node[below]{$H_2$}
% 				node[below=.5cm]{$c)$};
% 			\end{scope}
% 			\begin{scope}[xshift=13.5cm]
% 				\fill[green] (-\a,-\a) rectangle (\a,\a);
% 				\hv{3}
% 				\path (0,-\a)node[below]{$H_3$}
% 				node[below=.5cm]{$d)$};
% 			\end{scope}
% 			\path (current bounding box.south) node[below]{Hình $6$};
% 		\end{tikzpicture}
% 	}
% 	Ta có: $H_1$ có $5$ hình vuông, mỗi hình vuông có cạnh bằng $\dfrac{1}{3}$;\\
% 	{\color{white}{Ta có: }}$H_2$ có $5\cdot5=5^2$ hình vuông, mỗi hình vuông có cạnh bằng $\dfrac{1}{3} \cdot \dfrac{1}{3}=\dfrac{1}{3^2}; \ldots$.\\
% 	Từ đó, nhận được $H_n$ có $5^n$ hình vuông, mỗi hình vuông có cạnh bằng $\dfrac{1}{3^n}$.
% 	\begin{enumerate}
% 		\item Tính diện tích $S_n$ của $H_n$ và tính $\lim \limits_{n \to +\infty}S_n$.
% 		\item Tính chu vi $p_n$ của $H_n$ và tính $\lim \limits_{n \to +\infty}p_n$.
% 	\end{enumerate}
% 	(Quá trình trên tạo nên một hình, gọi là một fractal, được coi là có diện tích $\lim \limits_{n \to +\infty}S_n$ và chu vi $\lim \limits_{n \to +\infty}p_n$).
% 	\loigiai{
% 		\begin{enumerate}
% 			\item Hình vuông $H_1$ có diện tích $S_1=5\cdot \left(\dfrac{1}{3}\right)^2=\dfrac{5}{9}$.\\
% 			Hình vuông $H_2$ có diện tích $S_2=5^2\cdot \left(\dfrac{1}{3^2}\right)^2=\left(\dfrac{5}{9}\right)^2$.\\
% 			Hình vuông $H_n$ có diện tích $S_n=5^n\cdot \left(\dfrac{1}{3^n}\right)^2=\left(\dfrac{5}{9}\right)^n$.\\
% 			$\lim \limits_{n \to +\infty}S_n=\lim \limits_{n \to +\infty}\left(\dfrac{5}{9}\right)^n=0$.
% 			\item Hình vuông $H_1$ có chu vi $p_1=5\cdot 4\cdot  \dfrac{1}{3}=4\cdot \dfrac{5}{3}$.\\
% 			Hình vuông $H_2$ có chu vi $p_2=5^2\cdot4\cdot \dfrac{1}{3^2}=4\cdot \left(\dfrac{5}{3}\right)^2$.\\
% 			Hình vuông $H_n$ có diện tích $p_n=5^n\cdot4\cdot  \dfrac{1}{3^n}=4\cdot \left(\dfrac{5}{3}\right)^n$.\\
% 			$\lim \limits_{n \to +\infty}p_n=\lim \limits_{n \to +\infty}4\cdot \left(\dfrac{5}{3}\right)^n=+\infty$.
% 		\end{enumerate}
% 	}
% \end{bt}
% \begin{bt}%[1T3K1-5]
% 	\immini{Cho tam giác đều có cạnh bằng $a$, gọi là tam giác $H_1$. Nối các trung điểm của $H_1$ để tạo thành tam giác $\mathrm{H}_2$. Tiếp theo, nối các trung điểm của $\mathrm{H}_2$ để tạo thành tam giác $\mathrm{H}_3$ (Hình bên). Cứ tiếp tục như vậy, nhận được dãy tam giác $H_1, H_2, H_3, \ldots$\\
% 		Tính tổng chu vi và tổng diện tích các tam giác của dãy.}{
% 		\begin{tikzpicture}[scale=1, font=\footnotesize, line join=round, line cap=round, >=stealth]
% 			(0,0) coordinate (A)	
% 			(4,0) coordinate (B)	
% 			(2,2.8284) coordinate (C)
% 			\draw (A)--(B)--(C)--cycle;
% 			\coordinate (I) at ($(A)!0.5!(B)$);
% 			\coordinate (J) at ($(B)!0.5!(C)$);
% 			\coordinate (K) at ($(C)!0.5!(A)$);
% 			\draw (I)--(J)--(K)--cycle;
% 			\coordinate (H) at ($(I)!0.5!(J)$);
% 			\coordinate (G) at ($(J)!0.5!(K)$);
% 			\coordinate (F) at ($(K)!0.5!(I)$);
% 			\draw (H)--(G)--(F)--cycle;
% 			\coordinate (X) at ($(H)!0.5!(G)$);
% 			\coordinate (Y) at ($(G)!0.5!(F)$);
% 			\coordinate (Z) at ($(F)!0.5!(H)$);
% 			\draw (X)--(Y)--(Z)--cycle;
% 			\draw[dashed] (2,2.8284)--(3.3,1.45)node[above right]{$a$}--(4,0);
% 		\end{tikzpicture}
% 	}
% 	\loigiai{
% 		Gọi $S_i$ và $C_i$ ($i=1,2,\ldots$) lần lượt là diện tích và chu vi của tam giác $H_i$, $i=1,2,\ldots $.\\
% 		Khi đó ta có
% 		\begin{itemize}
% 			\item[$\bullet$] $S_1=\dfrac{a^2\sqrt{3}}{4};S_2=\left(\dfrac{a}{2}\right)^2\dfrac{\sqrt{3}}{4}=\dfrac{a^2\sqrt{3}}{16}=\dfrac{S_1}{4};S_3=\dfrac{1}{4}S_2,\ldots $.\\
% 			Do đó $(S_n)$ là một cấp số nhân lùi vô hạn với $S_1=\dfrac{a^2\sqrt{3}}{4}$ và $q_1=\dfrac{1}{4}$.\\
% 			Tổng diện tích $S=S_1+S_2+\cdots =\dfrac{S_1}{1-q_1}=\dfrac{a^2\sqrt{3}}{3}$.
% 			\item[$\bullet$] $C_1=3a; C_2=\dfrac{3a}{2}=\dfrac{1}{2}C_1,\ldots$\\
% 			Do đó $(C_n)$ là một cấp số nhân lùi vô hạn với $C_1=3a;q_2=\dfrac{1}{2}$.\\
% 			Tổng chu vi là $C=C_1+C_2+\cdots =\dfrac{C_1}{1-q_2}=6a$.
% 		\end{itemize}
% 	}	
% \end{bt}
% \begin{bt}%[1D4K1-6]
% 	Một thấu kính hội tụ có tiêu cự là $f$. Gọi $d$ và $d'$ lần lượt là khoảng cách từ một vật thật $A B$ và từ ảnh $A' B'$ của nó tới quang tâm $O$ của thấu kính như hình vẽ bên dưới. Công thức thấu kính là $\dfrac{1}{d}+\dfrac{1}{d}=\dfrac{1}{f}$.
% 	\begin{center}
% 		\begin{tikzpicture}[scale=0.6, font=\footnotesize, line join=round, line cap=round, >=stealth]		
% 			%\draw[domain=-3.65:3.65, blue,] plot({3/10*(\x)^2},\x);
% 			%\draw[gray!10] (-1,-1) grid (5,5);
% 			\draw [red] (-10,0)--(9,0);
% 			\draw [dashed] (0,-3)--(0,2.5);
% 			\coordinate (A) at (-10,0); 
% 			\coordinate (A') at (7,0); 
% 			\coordinate (B) at (-10,1); 
% 			\coordinate (L) at ( 0,1);
% 			\coordinate (O) at (0,0); 
% 			\coordinate (B') at (7,-1); 
% 			\coordinate (F') at (5,0); 
% 			\coordinate (F) at (-5,0); 
% 			\coordinate (H) at (-5,1); 
% 			\coordinate (M) at (5,2); 
% 			\coordinate (M') at (-5,2); 
% 			\coordinate (S) at (-10,-3); 
% 			\coordinate (K) at (7,-3);
% 			\coordinate (K') at (-5.5,0.55); 
% 			\coordinate (N) at (5,-0.74); 
% 			\coordinate (G) at (2,0.45); 
% 			\coordinate (Q) at (-2.5,0.9); 
% 			\coordinate (Q') at (2.5,0.9);
% 			\coordinate (J) at (-5,-2.9); 
% 			\coordinate (J') at (5,-2.9);
% 			\coordinate (X) at (-8.5,0);
% 			\coordinate (V) at (0,-3); \coordinate (V') at (0,2);
% 			%\draw  (F)--(D) (G)--(C) (F')--(D') (G')--(C') ;
% 			\draw (B)--(L)--(B');
% 			\draw (B)--(O)--(B');\draw[red,->] (A)--(X);
% 			\draw[blue,->] (B)--(H); \draw[blue,->] (B)--(K');
% 			\draw[blue,->] (A)--(B); \draw[blue,->] (O)--(N);
% 			\draw[blue,->] (L)--(G); \draw[red,->] (A')--(B');
% 			\draw[dashed] (F)--(M')--(M)--(F');
% 			\draw[dashed] (A)--(S)--(K)--(B');
% 			\draw[dashed,<->] (S)--(V);\draw[dashed,<->] (V)--(K) ;
% 			\draw[dashed,<->] (M')--(V');\draw[dashed,<->] (V')--(M) ;
% 			%\draw[red,->] (E)--(H3); 
% 			%\draw[->] (C)--(K3) ;
% 			%\draw[red,->] (E)--(H4); 
% 			%\draw[->] (C')--(K4) ;
% 			%\draw[red,->] (E)--(H2); 
% 			%\draw[->] (D')--(K2) ;
			
% 			%\draw[red] (E)--(C) (E)--(D) (E)--(C') (E)--(D');
% 			\draw[fill=black] (A') circle (1pt) node [above] {$A'$};
% 			\draw (Q) node [above] {$f$}; \draw (Q') node [above] {$f$}; \draw (J) node [above] {$d$}; \draw (J') node [above] {$d'$};
% 			\draw[fill=black] (B') circle (1pt) node [right] {$B'$};
% 			\draw[fill=black] (F') circle (1pt) node [above right] {$F'$};
% 			\draw[fill=black] (F) circle (1pt) node [below] {$F$};
% 			\draw[fill=black] (A) circle (1pt) node [left] {$A$};
% 			\draw[fill=black] (B) circle (1pt) node [left] {$B$};
% 			%\draw[fill=black] (E) circle (1pt) node [below right] {$S$};
% 			\draw[fill=black] (O) circle (1pt) node [below left] {$O$};
% 		\end{tikzpicture}
% 	\end{center}
% 	\begin{listEX}
% 		\item[a)] Tìm biểu thức xác định hàm số $d'=\varphi(d)$.
% 		\item[b)] Tìm $\underset{d \rightarrow f^{+}}\lim \limits_{n \to +\infty}\varphi(d), \underset{d \rightarrow f^{-}}\lim \limits_{n \to +\infty}\varphi(d)$ và $\underset{d \rightarrow f}\lim \limits_{n \to +\infty}\varphi(d)$. Giải thích ý nghĩa của các kết quả tìm được.
% 	\end{listEX}
% 	\loigiai{
% 		\begin{listEX}
% 			\item[a)] Ta có $$\dfrac{1}{d}+\dfrac{1}{d'}=\dfrac{1}{f} \Leftrightarrow d'=\dfrac{d f}{d-f}.$$
% 			Vậy $\varphi(d)=\dfrac{d f}{d-f}$.		
% 			\item[b)] Vì $\underset{d \rightarrow f^{+}}\lim \limits_{n \to +\infty}df=f^2; \underset{d \rightarrow f^{+}}\lim \limits_{n \to +\infty}(d-f)=0 ; d \rightarrow f^{+} \Rightarrow d-f>0 $ nên $ \underset{d \rightarrow f^{+}}\lim \limits_{n \to +\infty} \dfrac{d f}{d-f}=+\infty$.\\
% 			Vậy $\underset{d \rightarrow f^{+}}\lim \limits_{n \to +\infty}\varphi(d)=\underset{d \rightarrow f^{+}}\lim \limits_{n \to +\infty} \dfrac{d f}{d-f}=+\infty$.\\
% 			\textbf{Ý nghĩa}: Khi đặt vật nằm ngoài tiêu cự và tiến dần đến tiêu điểm thì cho ảnh thật ngược chiều với vật ở vô cùng.\\
% 			Vì $\underset{d \rightarrow f^{-}}\lim \limits_{n \to +\infty}df=f^2; \underset{d \rightarrow f^{+}}\lim \limits_{n \to +\infty}(d-f)=0 ; d \rightarrow f^{-} \Rightarrow d-f<0 $ nên $ \underset{d \rightarrow f^{+}}\lim \limits_{n \to +\infty} \dfrac{d f}{d-f}=-\infty$.\\
% 			Vậy $\underset{d \rightarrow f^{+}}\lim \limits_{n \to +\infty}\varphi(d)=\underset{d \rightarrow f^{+}}\lim \limits_{n \to +\infty} \dfrac{d f}{d-f}=-\infty$.\\
% 			\textbf{Ý nghĩa}: Khi đặt vật nằm trong tiêu cự và tiến dần đến tiêu điểm thì cho ảnh ảo cùng chiều với vật và nằm ở vô cùng.\\
% 			Vì không tồn tại $\underset{d \rightarrow f^{+}}\lim \limits_{n \to +\infty}\varphi(d)$ và $\underset{d \rightarrow f^{-}}\lim \limits_{n \to +\infty}\varphi(d)$ nên không tồn tại $\underset{d \rightarrow f}\lim \limits_{n \to +\infty}\varphi(d)$.
% 		\end{listEX}
% 	}
% \end{bt}
\subsubsection{Câu hỏi trắc nghiệm}
\Opensolutionfile{ans}[ans/ans-1K5-1-Dang5]
\begin{ex}%[1C3K1-6]
	Có $1$ kg chất phóng xạ độc hại. Biết rằng, cứ sau một khoảng thời gian $T=24000$ năm thì một nửa số chất phóng xạ này bị phân rã thành chất khác không độc hại đối với sức khỏe của con người ($T$ được gọi là \textit{chu kì bán rã}). Gọi $u_n$ là khối lượng chất phóng xạ còn lại sau chu kì thứ $n$.
	Sau ít nhất bao nhiêu chu kì bán rã thì khối lượng phóng xạ đã cho ban đầu không còn độc hại với con người, biết rằng chất phóng xạ này sẽ không độc hại nữa nếu khối lượng chất phóng xạ còn lại bé hơn $10^{-6}$ g.
	\choice
	{$24$}
	{\True $30$}
	{$100$}
	{$15$}
	\loigiai{
		\item Chất phóng xạ sẽ không độc hại nữa nếu khối lượng chất phóng xạ còn lại bé hơn $10^{-6}~\mathrm{g}=10^{-9}$ kg
		$$\Leftrightarrow u_n<10^{-9}\Leftrightarrow\dfrac{1}{2^n}<10^{-9}\Leftrightarrow 2^n>10^9\Leftrightarrow n\geq 30.$$
		Vậy sau ít nhất $30$ chu kì thì khối lượng phóng xạ đã cho ban đầu không còn độc hại với con người nữa.
	}
\end{ex}
\begin{ex}%[1C3K1-6]
	Có $1$ kg chất phóng xạ độc hại. Biết rằng, cứ sau một khoảng thời gian $T=24000$ năm thì một nửa số chất phóng xạ này bị phân rã thành chất khác không độc hại đối với sức khỏe của con người ($T$ được gọi là \textit{chu kì bán rã}). Gọi $u_n$ là khối lượng chất phóng xạ còn lại sau chu kì thứ $n$.
	Sau ít nhất bao nhiêu năm thì khối lượng phóng xạ đã cho ban đầu không còn độc hại với con người, biết rằng chất phóng xạ này sẽ không độc hại nữa nếu khối lượng chất phóng xạ còn lại bé hơn $10^{-6}$ g.
	\choice
	{$30$}
	{$2400$}
	{\True $720000$}
	{$10000$}
	\loigiai{
		\item Chất phóng xạ sẽ không độc hại nữa nếu khối lượng chất phóng xạ còn lại bé hơn $10^{-6}~\mathrm{g}=10^{-9}$ kg
		$$\Leftrightarrow u_n<10^{-9}\Leftrightarrow\dfrac{1}{2^n}<10^{-9}\Leftrightarrow 2^n>10^9\Leftrightarrow n\geq 30.$$
		Vậy sau ít nhất $30$ chu kì bằng $30\cdot 24000=720000$ năm thì khối lượng phóng xạ đã cho ban đầu không còn độc hại với con người nữa.
	}
\end{ex}
\begin{ex}%[1C3K1-6]
	Từ hình vuông có độ dài cạnh bằng $1$, người ta nối các trung điểm của cạnh hình vuông để tạo ra hình vuông mới như hình bên. Tiếp tục quá trình này đến vô hạn. Tính tổng diện tích của tất cả các hình vuông được tạo thành.
	\choice
	{$1$}
	{\True $2$}
	{$3$}
	{$4$}	
	\loigiai{
		Từ giả thiết suy ra diện tích hình vuông sau bằng $\dfrac{1}{2}$ diện tích hình vuông trước.\\ 
		Khi đó diện tích của các hình vuông tạo thành một cấp số nhân lùi vô hạn với số hạng đầu $S_1=1$ và công bội $q=\dfrac{1}{2}$.\\
		Diện tích $S_n$ của hình vuông được tạo thành từ bước thứ $n$ là $S_n=S_1\cdot q^{n-1}=\left(\dfrac{1}{2}\right)^{n-1}$.\\
		Tổng diện tích của tất cả các hình vuông được tạo thành là:\\
		$$S=\dfrac{u_1}{1-q}=\dfrac{1}{1-\dfrac{1}{2}}=2.$$		
	}
\end{ex}
\begin{ex}%[1T3K1-5]
	Từ hình vuông đầu tiên có cạnh bằng $1$ (đơn vị độ dài), nối các trung điểm của bốn cạnh để có hình vuông thứ hai. Tiếp tục nối các trung điểm của bốn cạnh của hình vuông thứ hai để được hình vuông thứ ba. Cứ tiếp tục làm như thế, nhận được một dãy hình vuông. Tính tổng chu vi của dãy các hình vuông trên. 
	\choice
	{$8+\sqrt{2}$}
	{$2+\sqrt{2}$}
	{\True $8+4\sqrt{2}$}
	{$4+4\sqrt{2}$}
	\loigiai{
		Hình vuông thứ nhất có chu vi bằng $4$, hình vuông thứ $2$ có chu vi là $2\sqrt{2}$, hình vuông thứ $3$ có chu vi là $2$.\\
		Suy ra hình vuông thứ $n$ có chu vi bằng $p_n=4\cdot \left(\dfrac{\sqrt{2}}{2}\right)^{n-1}$.
		Tổng chu vi của $n$ hình vuông đầu tiên là tổng của cấp số nhân có  số hạng đầu $p_1=4$ và công bội $q=\dfrac{\sqrt{2}}{2}$ nên 
		\[ Q_n=\dfrac{p_1\left(1-q^n\right)}{1-q}=\dfrac{4\left(1-\left(\dfrac{\sqrt{2}}{2}\right)^n\right)}{1-\dfrac{\sqrt{2}}{2}}=\left(8+4\sqrt{2}\right)\left[1-\left(\dfrac{\sqrt{2}}{2}\right)^n\right].\]
		$\lim \limits_{n \to +\infty}Q_n=\left(8+4\sqrt{2}\right)\lim \limits_{n \to +\infty}\left[1-\left(\dfrac{\sqrt{2}}{2}\right)^n\right]=\left(8+4\sqrt{2}\right)\left(1-0\right)=8+4\sqrt{2}$.
	}
\end{ex}
\begin{ex}%[1D4K1-6]
	Từ độ cao $55,8 \mathrm{~m}$ của tháp nghiêng Pisa nước Ý, người ta thả một quả bóng cao su chạm xuống đất hình bên dưới. Giả sử mỗi lần chạm đất quả bóng lại nảy lên độ cao bằng $\dfrac{1}{10}$ độ cao mà quả bóng đạt được trước đó. Gọi $S_n$ là tổng độ dài quãng đường di chuyển của quả bóng tính từ lúc thả ban đầu cho đến khi quả bóng đó chạm đất $n$ lần. Tính $\lim \limits_{n \to +\infty}S_n$.
	\choice
	{$58{,}8$}
	{$67{,}2$}
	{$68$}
	{\True $68{,}2$}
	\loigiai{Mỗi khi chạm đất quả bóng lại nảy lên một độ cao bằng $\dfrac{1}{10}$ độ cao của lần rơi ngay trước đó và sau đó lại rơi xuống từ độ cao thứ hai này. Do đó, độ dài hành trình của quả bóng kể từ thời điểm rơi ban đầu đến:\\    
		Thời điểm chạm đất lần thứ nhất là $d_1=55{,}8$.\\
		Thời điềm chạm đất lần thứ hai là $d_2=55{,}8+2\cdot \dfrac{55{,}8}{10}$.\\
		Thời điểm chạm đất lần thứ ba là $d_3=55{,}8+2 \cdot\dfrac{55{,}8}{10}+2\cdot \dfrac{55{,}8}{10^2}$.\\
		Thời điểm chạm đất lần thứ tư là $d_4=55{,}8+2 \cdot\dfrac{55{,}8}{10}+2\cdot \dfrac{55,8}{10^2}+2\cdot \dfrac{55{,}8}{10^3}$.\\
		$\ldots$\\
		Thời điểm chạm đất lần thứ $n~(n>1)$ là
		$$d_n=55{,}8+2\cdot55{,}8+2\cdot \frac{55{,}8}{10^2}+2\cdot \frac{55{,}8}{10^3}+\ldots+2\cdot \frac{55{,}8}{10^{n-1}}.$$
		Do đó, quãng đường mà quả bóng đi được kể từ thời điềm rơi đến khi nằm yên trên mặt đất là:
		$$ d=55{,}8+2.55{,}8+2\cdot \frac{55{,}8}{10^2}+2\cdot \frac{55{,}8}{10^3}+\ldots+2\cdot \frac{55{,}8}{10^{n-1}}+\ldots=\lim \limits_{n \to +\infty}d_n.$$
		Vì $2\cdot \dfrac{55{,}8}{10} ; 2\cdot \dfrac{55{,}8}{10^2} ; 2\cdot \dfrac{55{,}8}{10^3}; \ldots ; 2\cdot \dfrac{55{,}8}{10^{n-1}}; \ldots$ là một cấp số nhân lùi vô hạn với công bội $q=\dfrac{1}{10}$ nên ta có:
		$$ 2 \cdot\dfrac{55,8}{10}+2\cdot \dfrac{55{,}8}{10^2}+2\cdot \dfrac{55{,}8}{10^3}+\ldots+2\cdot \dfrac{55{,}8}{10^{n-1}}+\ldots=\dfrac{2\cdot \dfrac{55{,}8}{10}}{1-\dfrac{1}{10}}=12{,}4.$$
		Vậy $d=55{,}8+12{,}4=68{,}2$ m.
	}
\end{ex}
\Closesolutionfile{ans}
% \begin{indapan}{10}
% 	{ans/ans-1K5-1-Dang5}
% \end{indapan}
\begin{dang}{Nguyên lý kẹp}
	Để tìm giới hạn của dãy số theo nguyên lý kẹp ta cần nhớ:
	\begin{itemize}
		\item Cho hai dãy số $(u_n)$ và $(v_n)$. Nếu $|u_n| \leq v_n$ với mọi $n$ và $\lim \limits_{n \to +\infty}v_n = 0$ thì $\lim \limits_{n \to +\infty}u_n =0$.
		\item Cho ba dãy số $(u_n)$, $(v_n)$ và $(w_n)$. Nếu $ u_n \leq v_n \leq w_n$ với mọi $n$ và $\lim \limits_{n \to +\infty}u_n = \lim \limits_{n \to +\infty}w_n = L$ thì $\lim \limits_{n \to +\infty}v_n = L$.
	\end{itemize}
\end{dang}
\subsubsection{Ví dụ minh hoạ}
%bai1
\begin{vd}%[1D4B1-2]
	Chứng minh rằng các dãy số với số hạng tổng quát sau đây có giới hạn $0$.
	\begin{enumEX}[a)]{2}
		\item $u_n=\dfrac{(-1)^n}{3n+2}$.
		\item $u_n=\dfrac{n\sin 2n}{n^3+2}$.
		\item $u_n=\dfrac{(-1)^n\cos n}{\sqrt{n}}$.
		\item $u_n=\dfrac{3\sin n-4\cos n}{2n^2+1}$.
	\end{enumEX}
	\loigiai{
		\begin{enumerate}[a)]
			\item Ta có: $0 \leqslant \left|u_n\right|= \dfrac{1}{3n+2}<\dfrac{1}{3n}<\dfrac{1}{n}$, $\forall n\in \mathbb{N^*}$.\\
			Mà $\lim \limits_{n \to +\infty}\dfrac{1}{n}=0$ nên suy ra $\lim \limits_{n \to +\infty}\dfrac{(-1)^n}{3n+2}=0$.
			\item Ta có $0\leqslant \left|u_n\right|=\dfrac{n\left|\sin 2n\right|}{n^3+2}\leqslant \dfrac{n}{n^3+2}<\dfrac{n}{n^3}\leqslant \dfrac{1}{n^2}$, $\forall n\in \mathbb{N^*}$.\\
			Mà $\lim \limits_{n \to +\infty}\dfrac{1}{n^2}=0$ nên suy ra $\lim \limits_{n \to +\infty}\dfrac{n\sin 2n}{n^3+2}=0$.
			\item Ta có $0\leqslant \left|\dfrac{(-1)^n\cos n}{\sqrt{n}}\right|\leqslant \dfrac{1}{\sqrt{n}}$, $\forall n\in \mathbb{N^*}$.\\
			Mà $\lim \limits_{n \to +\infty}\dfrac{1}{\sqrt{n}}=0$ nên suy ra $\lim \limits_{n \to +\infty}\dfrac{(-1)^n\cos n}{\sqrt{n}}=0$.
			\item Theo bất đẳng thức Bunhiacopxki, ta có\\
			$\left|3\sin n-4\cos n\right|\leqslant \sqrt{(3^2+4^2)\left(\sin^2n+\cos^2n\right)}=5$.\\
			Do đó $0\leqslant \left|\dfrac{3\sin n-4\cos n}{2n^2+1}\right|\leqslant \dfrac{5}{2n^2+1}<\dfrac{5}{2n^2}$, $\forall n\in \mathbb{N^*}$.\\
			Mà $\lim \limits_{n \to +\infty}\dfrac{5}{2n^2+1}=0$ nên suy ra $\lim \limits_{n \to +\infty}\dfrac{3\sin n-4\cos n}{2n^2+1}=0$.
		\end{enumerate}
	}
\end{vd}
%bai2
\begin{vd}%[1D4K1-2]
	Chứng minh rằng các dãy số với số hạng tổng quát sau đây có giới hạn $0$. 
	\begin{enumEX}{2}
		\item $u_n=\sqrt{n^3+2}-\sqrt{n^3+1}$.
		\item $u_n=\dfrac{3^n\sin 2n+4^n}{2^n+4\cdot 5^n}$.
		\item $u_n=\dfrac{n+\sin 2n}{n^2+n}$.
		\item $\dfrac{n+\cos \dfrac{n\pi}{5}}{n\sqrt{n}+\sqrt{n}}$.
	\end{enumEX}
	\loigiai{
		\begin{enumerate}
			\item Ta có $u_n=\sqrt{n^3+2}-\sqrt{n^3+1}=\dfrac{n^3+2-(n^3+1)}{\sqrt{n^3+2}+\sqrt{n^3+1}}=\dfrac{1}{\sqrt{n^3+2}+\sqrt{n^3+1}}$.\\
			Do đó $0\leqslant \left|u_n\right|=\dfrac{1}{\sqrt{n^3+2}+\sqrt{n^3+1}}<\dfrac{2}{\sqrt{n^2}+\sqrt{n^2}}=\dfrac{2}{2n}=\dfrac{1}{n}$, $\forall n\in \mathbb{N^*}$.\\
			Mà $\lim \limits_{n \to +\infty}\dfrac{1}{n}=0$ nên suy ra $\lim \limits_{n \to +\infty}\left(\sqrt{n^3+2}-\sqrt{n^3+1}\right)=0$.
			\item Ta có $0\leqslant \left|\dfrac{3^n\sin 2n+4^n}{2^n+4.5^n}\right|\leqslant \dfrac{3^n\left|\sin 2n\right|+4^n}{2^n+4\cdot 5^n}\leqslant \dfrac{3^n+4^n}{2^n+4\cdot 5^n}=\dfrac{{\left(\dfrac{3}{5}\right)}^n+{\left(\dfrac{4}{5}\right)}^n}{{\left(\dfrac{2}{5}\right)}^n+4}$, $\forall n\in \mathbb{N^*}$.\\
			Mà $\lim \limits_{n \to +\infty}\left[{\left(\dfrac{3}{5}\right)}^n+{\left(\dfrac{4}{5}\right)}^n\right]=\lim \limits_{n \to +\infty}{\left(\dfrac{3}{5}\right)}^n+\lim \limits_{n \to +\infty}{\left(\dfrac{4}{5}\right)}^n=0$ và $$\lim \limits_{n \to +\infty}\left[{\left(\dfrac{2}{5}\right)}^n+4\right]=\lim \limits_{n \to +\infty}{\left(\dfrac{2}{5}\right)}^n+4=0+4=4$$ nên $\lim \limits_{n \to +\infty}\dfrac{3^n\sin 2n+4^n}{2^n+4\cdot 5^n}=0$.
			\item Ta có $0\leqslant \left|\dfrac{n+\sin 2n}{n^2+n}\right|\leqslant \dfrac{n+\left|\sin 2n\right|}{n^2+n}\leqslant \dfrac{n+1}{n^2+n}=\dfrac{1}{n}$, $\forall n\in \mathbb{N^*}$. \\
			Mà $\lim \limits_{n \to +\infty}\dfrac{1}{n}=0$ nên suy ra $\lim \limits_{n \to +\infty}\dfrac{n+\sin 2n}{n^2+n}=0$.
			\item Ta có $0\leqslant \left|\dfrac{n+\cos \dfrac{n\pi}{5}}{n\sqrt{n}+\sqrt{n}}\right|\leqslant \dfrac{n+\left|\cos \dfrac{n\pi}{5}\right|}{n\sqrt{n}+\sqrt{n}}\leqslant \dfrac{n+1}{n\sqrt{n}+\sqrt{n}}=\dfrac{1}{\sqrt{n}}$, $\forall n\in \mathbb{N^*}$. \\
			Mà $\lim \limits_{n \to +\infty}\dfrac{1}{\sqrt{n}}=0$ nên suy ra $\lim \limits_{n \to +\infty}\dfrac{n+\cos \dfrac{n\pi}{5}}{n\sqrt{n}+\sqrt{n}}=0$.
		\end{enumerate}
	}
\end{vd}
%bai3
\begin{vd}%[1D4B1-1]
	Cho dãy số $(u_n)$ với $u_n=\dfrac{n}{3^n}$.
	\begin{enumerate}[a)]
		\item Chứng minh rằng $\dfrac{u_{n+1}}{u_n}\leqslant \dfrac{2}{3}$ với mọi $n\in \mathbb{N^*}$.
		\item Bằng phương pháp quy nạp chứng minh rằng $0<u_n<{\left(\dfrac{2}{3}\right)}^n$ với mọi $n\in \mathbb{N^*}$.
		\item Dãy $(u_n)$ có giới hạn $0$.
	\end{enumerate}
	\loigiai{
		\begin{enumerate}[a)]
			\item Với mọi $n\in \mathbb{N}$, ta có $\dfrac{u_{n+1}}{u_n}=\dfrac{\dfrac{n+1}{3^{n+1}}}{\dfrac{n}{3^n}}=\dfrac{n+1}{n}\cdot \dfrac{1}{3}$. \\
			Mặt khác, $n+1\leqslant n+n\leqslant 2n$. Suy ra $\dfrac{n+1}{n}\leqslant 2$. \\
			Do đó $\dfrac{u_{n+1}}{u_n}\leqslant \dfrac{2}{3}$ với mọi $n\in \mathbb{N^*}$.
			\item Rõ ràng với mọi $n\in \mathbb{N^*}$, ta có $u_n>0$. Do đó ta chỉ cần chứng minh $u_n<{\left(\dfrac{2}{3}\right)}^n$.
			\begin{itemize}
				\item Với $n=1$, ta có $u_1=\dfrac{1}{3^1}=\dfrac{1}{3}<{\left(\dfrac{2}{3}\right)}^1$. Nghĩa là mệnh đề đúng với $n=1$. \\
				\item Giả sử mệnh đề đúng với $n=k\geqslant 1$, tức là $u_k<{\left(\dfrac{2}{3}\right)}^k$. \\
				\item Bây giờ ta cần chứng minh mệnh đề đúng với $n=k+1$, tức là cần chứng minh $u_{k+1}<{\left(\dfrac{2}{3}\right)}^{k+1}$. \\
				Theo chứng minh câu a) ta có $\dfrac{u_{k+1}}{u_k}\leqslant \dfrac{2}{3}$ suy ra $u_{k+1}\leqslant \dfrac{2}{3}\cdot u_k<\dfrac{2}{3}\cdot {\left(\dfrac{2}{3}\right)}^k={\left(\dfrac{2}{3}\right)}^{k+1}$ hay $u_{k+1}<{\left(\dfrac{2}{3}\right)}^{k+1}$. \\
				Nghĩa là mệnh đề cũng đúng với $n=k+1$. Vậy $0<u_n<{\left(\dfrac{2}{3}\right)}^n$ với mọi $n\in \mathbb{N^*}$.
			\end{itemize}
			\item Theo câu b), ta có $0<u_n<{\left(\dfrac{2}{3}\right)}^n$. Mà $\lim \limits_{n \to +\infty}{\left(\dfrac{2}{3}\right)}^n=0$. Do đó $\lim \limits_{n \to +\infty}u_n=0$.
		\end{enumerate}
	}
\end{vd}
%bai4
\begin{vd}%[1D4B1-2]
	Chứng minh rằng
	\begin{enumEX}[a)]{2}
		\item $\lim \limits_{n \to +\infty}\left(\dfrac{-n^3}{n^3+1}\right)=-1$.
		\item $\lim \limits_{n \to +\infty}\left(\dfrac{n^2+3n+2}{2n^2+n}\right)=\dfrac{1}{2}$.
	\end{enumEX}
	\loigiai{
		\begin{enumerate}
			\item Ta có $\lim \limits_{n \to +\infty}\left(\dfrac{-n^3}{n^3+1}-(-1)\right)=\lim \limits_{n \to +\infty}\left(\dfrac{1}{n^3+1}\right)$. \\
			Vì $0\leqslant \left|\dfrac{1}{n^3+1}\right|<\dfrac{1}{n^3}$, $\forall n\in \mathbb{N^*}$. Mà $\lim \limits_{n \to +\infty}\dfrac{1}{n^3}=0$ nên suy ra $\lim \limits_{n \to +\infty}\left(\dfrac{1}{n^3+1}\right)=0$. \\
			Do đó $\lim \limits_{n \to +\infty}\left(\dfrac{-n^3}{n^3+1}\right)=-1$.
			\item Ta có $\lim \limits_{n \to +\infty}\left(\dfrac{n^2+3n+2}{2n^2+n}-\dfrac{1}{2}\right)=\lim \limits_{n \to +\infty}\dfrac{5n+4}{2(2n^2+n)}$. \\
			Vì $0<\left|\dfrac{5n+4}{2(2n^2+n)}\right|<\dfrac{5n+5}{2n(n+1)}=\dfrac{5}{2}\cdot \dfrac{1}{n}$, $\forall n\in \mathbb{N^*}$. Mà $\lim \limits_{n \to +\infty}\left(\dfrac{5}{2}\cdot \dfrac{1}{n}\right)=\dfrac{5}{2}\cdot \lim \limits_{n \to +\infty}\dfrac{1}{n}=0$ nên suy ra $\lim \limits_{n \to +\infty}\dfrac{5n+4}{2(2n^2+n)}=0$. \\
			Do đó $\lim \limits_{n \to +\infty}\left(\dfrac{n^2+3n+2}{2n^2+n}\right)=\dfrac{1}{2}$.
		\end{enumerate}
	}
\end{vd}
%bai5
\begin{vd}%[1D4K1-2]
	Chứng minh rằng
	\begin{enumEX}[a)]{2}
		\item $\lim \limits_{n \to +\infty}\left(\dfrac{3\cdot 3^n-\sin 3n}{3^n}\right)=3$.
		\item $\lim \limits_{n \to +\infty}\left(\sqrt{n^2+n}-n\right)=\dfrac{1}{2}$.
	\end{enumEX}
	\loigiai{
		\begin{enumerate}[a)]
			\item Ta có $\lim \limits_{n \to +\infty}\left(\dfrac{3.3^n-\sin 3n}{3^n}-3\right)=\lim \limits_{n \to +\infty}\left(\dfrac{-\sin 3n}{3^n}\right)$. \\
			Vì $0\leqslant \left|\dfrac{-\sin 3n}{3^n}\right|=\dfrac{\left|-\sin 3n\right|}{3^n}\leqslant \dfrac{1}{3^n}={\left(\dfrac{1}{3}\right)}^n$, $\forall n\in \mathbb{N^*}$. Mà $\lim \limits_{n \to +\infty}{\left(\dfrac{1}{3}\right)}^n=0$ nên suy ra $\lim \limits_{n \to +\infty}\left(\dfrac{-\sin 3n}{3^n}\right)=0$. \\
			Do đó $\lim \limits_{n \to +\infty}\left(\dfrac{3.3^n-\sin 3n}{3^n}\right)=3$.
			\item Ta có $\lim \limits_{n \to +\infty}\left(\sqrt{n^2+n}-n-\dfrac{1}{2}\right)=\lim \limits_{n \to +\infty}\dfrac{2\sqrt{n^2+n}-(2n+1)}{2}=\lim \limits_{n \to +\infty}\dfrac{-1}{2\left(2\sqrt{n^2+n}+(2n+1)\right)}$. \\
			Vì $0\leqslant \left|\dfrac{-1}{2\left(2\sqrt{n^2+n}+(2n+1)\right)}\right| \leqslant \dfrac{1}{2\left(2\sqrt{n^2+n}+(2n+1)\right)}\leqslant \dfrac{1}{2\left(2\sqrt{n^2}+2n\right)}=\dfrac{1}{8}\cdot \dfrac{1}{n}$, $\forall n\in \mathbb{N^*}$. \\
			Mà $\lim \limits_{n \to +\infty}\dfrac{1}{8}\cdot \dfrac{1}{n}=\dfrac{1}{8}\lim \limits_{n \to +\infty}\dfrac{1}{n}=0$ nên suy ra $\lim \limits_{n \to +\infty}\left(\sqrt{n^2+n}-n-\dfrac{1}{2}\right)$. \\
			Do đó $\lim \limits_{n \to +\infty}\left(\sqrt{n^2+n}-n\right)=\dfrac{1}{2}$.
		\end{enumerate}
	}
\end{vd}
%Bài 6
\begin{vd}%[1D4K1-5]
	Tìm các giới hạn sau
	\begin{enumEX}[a)]{1}
		\item $\lim \limits_{n \to +\infty}\left(\dfrac{1}{\sqrt{4n^2+1}}+\dfrac{1}{\sqrt{4n^2+2}}+\cdots +\dfrac{1}{\sqrt{4n^2+n}}\right)$.
		\item $\lim \limits_{n \to +\infty}\dfrac{{1\cdot 3\cdot 5\cdot 7}\cdots (2n-1)}{{2\cdot 4\cdot 6}\cdots (2n)}$.
	\end{enumEX}
	\loigiai{
		\begin{enumerate}[a)]
			\item Ta có
			\begin{align*}
				\dfrac{1}{\sqrt{4n^2}}+\dfrac{1}{\sqrt{4n^2}}+\cdots +\dfrac{1}{\sqrt{4n^2}} &\leqslant \dfrac{1}{\sqrt{4n^2+1}}+\dfrac{1}{\sqrt{4n^2+2}}+\cdots +\dfrac{1}{\sqrt{4n^2+n}}\\
				&\leqslant \dfrac{1}{\sqrt{4n^2+n}}+\dfrac{1}{\sqrt{4n^2+n}}+\cdots +\dfrac{1}{\sqrt{4n^2+n}}.
			\end{align*}
			hay
			\begin{center}
				$\dfrac{n}{\sqrt{4n^2}}\leqslant \dfrac{1}{\sqrt{4n^2+1}}+\dfrac{1}{\sqrt{4n^2+2}}+\cdots +\dfrac{1}{\sqrt{4n^2+n}}\leqslant \dfrac{n}{\sqrt{4n^2+n}}$ với mọi $n\in \mathbb{N^*}$.
			\end{center}
			Mà $\lim \limits_{n \to +\infty}\dfrac{n}{\sqrt{4n^2}}=\lim \limits_{n \to +\infty}\dfrac{1}{2}=\dfrac{1}{2}$; $\lim \limits_{n \to +\infty}\dfrac{n}{\sqrt{4n^2+n}}=\lim \limits_{n \to +\infty}\dfrac{1}{\sqrt{4+\dfrac{1}{n}}}=\dfrac{1}{\sqrt{4+0}}=\dfrac{1}{2}$. \\
			Do đó $\lim \limits_{n \to +\infty}\left(\dfrac{1}{\sqrt{4n^2+1}}+\dfrac{1}{\sqrt{4n^2+2}}+\cdots +\dfrac{1}{\sqrt{4n^2+n}}\right)=\dfrac{1}{2}$.
			\item Ta có $u_n=\dfrac{{1\cdot 3\cdot 5\cdot 7}\cdots (2n-1)}{{2\cdot 4\cdot 6}\cdots (2n)}$, suy ra $$u_n^2=\dfrac{1^2\cdot 3^2\cdot 5^2\cdot 7^2\cdots (2n-1)^2}{2^2\cdot 4^2\cdot 6^2\cdots (2n)^2}=\dfrac{1\cdot 3}{2^2}\cdot \dfrac{3\cdot 5}{4^2}\cdots \dfrac{(2n-1)(2n+1)}{(2n)^2}\cdot \dfrac{1}{2n+1}<\dfrac{1}{2n+1}.$$
			(do $\dfrac{1\cdot 3}{2^2}\cdot \dfrac{3\cdot 5}{4^2}\cdots \dfrac{(2n-1)(2n+1)}{(2n)^2}<\dfrac{2^2}{2^2}\cdot \dfrac{4^2}{4^2}\cdots \dfrac{(2n)^2}{(2n)^2}=1$ ) \\
			Vậy ta có $0<u_n<\dfrac{1}{\sqrt{2n+1}}$, $\forall n\in \mathbb{N^*}$. Mà $\lim \limits_{n \to +\infty}\dfrac{1}{\sqrt{2n+1}}=0$ nên suy ra $$\lim \limits_{n \to +\infty}\dfrac{{1\cdot 3\cdot 5\cdot 7}\cdots (2n-1)}{{2\cdot 4\cdot 6}\cdots (2n)}=0.$$
		\end{enumerate}
	}
\end{vd}
% \subsubsection{Bài tập rèn luyện}
% \subsubsection{Câu hỏi trắc nghiệm}
% \Opensolutionfile{ans}[ans/ans-1K5-1-Dang6]

% %%==========Câu 1
% \begin{ex}%[1D4B1-2]
% 	Giới hạn $\displaystyle\lim\dfrac{\sin n+1}{n}$ bằng
% 	\choice
% 	{$+\infty$}
% 	{$1$}
% 	{$-\infty$}
% 	{\True $0$}
% 	\loigiai{
% 		Với mọi $n>0$ thì $|\sin n+1|\leq 2$. Do đó, với mọi $n>0$, ta có
% 		$$0\leq \left|\dfrac{\sin n+1}{n}\right|\leq \dfrac{2}{n}.$$
% 		Từ đó $$0\leq \lim\limits\left|\dfrac{\sin n+1}{n}\right|\leq \lim\limits\dfrac{2}{n}=0\Rightarrow \lim\limits\left|\dfrac{\sin n+1}{n}\right|=0\Rightarrow \lim\limits\dfrac{\sin n+1}{n}=0.$$
% 	}
% \end{ex}
% %%==========Câu 2
% \begin{ex}%[1D4B1-2]
% 	Giới hạn $\displaystyle\lim\dfrac{\sin n+1}{n}$ bằng
% 	\choice
% 	{$+\infty$}
% 	{$1$}
% 	{$-\infty$}
% 	{\True $0$}
% 	\loigiai{
% 		Với mọi $n>0$ thì $|\sin n+1|\leq 2$. Do đó, với mọi $n>0$, ta có
% 		$$0\leq \left|\dfrac{\sin n+1}{n}\right|\leq \dfrac{2}{n}.$$
% 		Từ đó $$0\leq \lim\limits\left|\dfrac{\sin n+1}{n}\right|\leq \lim\limits\dfrac{2}{n}=0\Rightarrow \lim\limits\left|\dfrac{\sin n+1}{n}\right|=0\Rightarrow \lim\limits\dfrac{\sin n+1}{n}=0.$$
% 	}
% \end{ex}
% %%==========Câu 3
% \begin{ex}%[1D4B1-2]
% 	Giới hạn $\lim \limits_{n \to +\infty}\dfrac{\cos n}{n}$ bằng
% 	\choice
% 	{$1$}
% 	{\True $0$}
% 	{$-1$}
% 	{$+\infty$}
% 	\loigiai{
% 		Ta có: $\left| \dfrac{\cos n}{n} \right|\le \dfrac{1}{n}$ và $\lim \limits_{n \to +\infty}\dfrac{1}{n}=0$ nên $\lim \limits_{n \to +\infty}\dfrac{\cos n}{n}=0$.
% 	}
% \end{ex}
% %%==========Câu 4
% \begin{ex}%[1D4B1-2]
% 	Tính $\lim \limits_{n \to +\infty}\dfrac{\sin n}{n^3+1}$.
% 	\choice
% 	{$1$}
% 	{\True $0$}
% 	{$-\infty$}
% 	{$+\infty$}
% 	\loigiai{
% 		Ta có
% 		$ \left|\dfrac{\sin n}{n^3+1}\right|\le \dfrac{1}{n^3+1}$ mà
% 		$\lim \limits_{n \to +\infty}\dfrac{1}{n^3+1}=\lim \limits_{n \to +\infty}\dfrac{1}{n^3\left(1+\dfrac{1}{n^3}\right)}=0$.\\
% 		Vậy $\lim \limits_{n \to +\infty}\dfrac{\sin n}{n^3+1}=0$
% 	}
% \end{ex}
% %%==========Câu 5
% \begin{ex}%[1D4B1-2]
% 	Tính $\lim \limits_{n \to +\infty}\dfrac{\sin 2024n}{n}$.
% 	\choice
% 	{\True $0$}
% 	{$1$}
% 	{$+ \infty$}
% 	{$2024$}
% 	\loigiai{
% 		Ta có $-1 \leqslant \sin 2024n \leqslant 1  \Leftrightarrow - \dfrac{1}{n} \leqslant \dfrac{\sin 2024n}{n} \leqslant \dfrac{1}{n}$.\\
% 		Vì $\lim \limits_{n \to +\infty}\left( - \dfrac{1}{n} \right) = \lim \limits_{n \to +\infty} \dfrac{1}{n} = 0$ nên $\lim \limits_{n \to +\infty}\dfrac{\sin 2024n}{n} = 0$.
% 	}
% \end{ex}
% %%==========Câu 6
% \begin{ex}%[1D4K1-2]
% 	Tính $I=\lim \limits_{n \to +\infty}\left(\dfrac{1}{\sqrt{n^2+n+1}}+\dfrac{1}{\sqrt{n^2+n+2}}+...+\dfrac{1}{\sqrt{n^2+2n}}\right)$.
% 	\choice
% 	{$I=+\infty$}
% 	{$I=3$}
% 	{$I=2$}
% 	{\True $I=1$}
% 	\loigiai{
% 		Ta có $\dfrac{1}{\sqrt{n^2+2n}}<\dfrac{1}{\sqrt{n^2+n+k}}<\dfrac{1}{\sqrt{n^2+n+1}},\forall k=2,3,...,n-1$.\\
% 		$\Rightarrow \dfrac{n}{\sqrt{n^2+2n}}<\dfrac{1}{\sqrt{n^2+n+1}}+\dfrac{1}{\sqrt{n^2+n+2}}+\cdots+\dfrac{1}{\sqrt{n^2+2n}}<\dfrac{n}{\sqrt{n^2+n+1}}$.\\
% 		Mà $\lim \limits_{n \to +\infty}\dfrac{n}{\sqrt{n^2+2n}}=\lim \limits_{n \to +\infty}\dfrac{1}{\sqrt{1+\dfrac{2}{n}}}=1$; $\lim \limits_{n \to +\infty}\dfrac{n}{\sqrt{n^2+n+1}}=\lim \limits_{n \to +\infty}\dfrac{1}{\sqrt{1+\dfrac{1}{n}+\dfrac{1}{n^2}}}=1$.\\
% 		Vậy $I=1$.
% 	}
% \end{ex}
% %%==========Câu 7
% \begin{ex}%[1D4K1-2]
% 	Tính $T=\lim\dfrac{n\sin n-3n^2}{n^2}$.
% 	\choice
% 	{$T=+\infty$}
% 	{$T=-\infty$}
% 	{$T=1$}
% 	{\True $T=-3$}
% 	\loigiai{
% 		Ta có $T=\lim\dfrac{n\sin n-3n^2}{n^2}=\lim\left(\dfrac{\sin n}{n}-3\right)$.\\
% 		Do $-1\le \sin n\le 1$ suy ra $-\dfrac{1}{n}\le \dfrac{\sin n}{n}\le \dfrac{1}{n}$, mà $\lim\left(-\dfrac{1}{n}\right)=0$ và $\lim\dfrac{1}{n}=0$ nên $\lim\dfrac{\sin n}{n}=0$.\\
% 		Từ đó suy ra $T=\lim\left(\dfrac{\sin n}{n}-3\right)=-3$.
% 	}
% \end{ex}
% %%==========Câu 8
% \begin{ex}%[1D4K1-2]
% 	Tính giá trị của $I=\lim\dfrac{n^3+n\cdot\sin^2n}{10000n^3-n+2}$.
% 	\choice
% 	{\True $I=0{,}0001$}
% 	{$I=\dfrac{1}{1000}$}
% 	{$I=0$}
% 	{$I=0{,}00001$}
% 	\loigiai{Ta có: $I=\lim\dfrac{n^3+n\cdot\sin^2n}{10000n^3-n+2}=\lim\dfrac{1+\dfrac{\sin^2 n}{n^2}}{10000-\dfrac{1}{n^2}+\dfrac{2}{n^3}}=\dfrac{1}{10000}=0{,}0001$.\\
% 		Chú ý rằng: $0\leq\dfrac{\sin^2n}{n^2}\leq\dfrac{1}{n^2}$. Mà $\lim\dfrac{1}{n^2}=0\Rightarrow\lim\dfrac{\sin^2n}{n^2}=0$.}
% \end{ex}
% %%==========Câu 9
% \begin{ex}%[1D4G1-2]
% 	Tính $I=\lim \limits_{n \to +\infty}\left(\dfrac{1}{2}+\dfrac{3}{2^2}+\dfrac{5}{2^3}+\cdots +\dfrac{2n-1}{2^n}\right)$.
% 	\choice
% 	{\True $I=3$}
% 	{$I=0$}
% 	{$I=\dfrac{1}{2}$}
% 	{$I=+\infty $}
% 	\loigiai{
% 		Đặt $S_n=\dfrac{1}{2}+\dfrac{3}{2^2}+\dfrac{5}{2^3}+\cdots +\dfrac{2n-1}{2^n}$. \\
% 		Khi đó $\dfrac{1}{2}S_n=\dfrac{1}{2^2}+\dfrac{3}{2^3}+\dfrac{5}{2^4}+\cdots +\dfrac{2n-3}{2^n}+\dfrac{2n-1}{2^{n+1}}$. \\
% 		Trừ vế theo vế ta được \\
% 		$S_n-\dfrac{1}{2}S_n=\dfrac{1}{2}+\left(\dfrac{2}{2^2}+\dfrac{2}{2^3}+\cdots +\dfrac{2}{2^n}\right)-\dfrac{2n-1}{2^{n+1}}$. \\
% 		Từ đó $S_n=1+\left(1+\dfrac{1}{2}+\dfrac{1}{4}+\cdots +\dfrac{1}{2^{n-2}}\right)-\dfrac{2n-1}{2^n}=1+2\left[1-{\left(\dfrac{1}{2}\right)}^{n-1}\right]-\dfrac{2n-1}{2^n}$. \\
% 		Với mọi $n\geqslant 4$ ta có $2^n\geqslant n^2$. Thật vậy,
% 		\begin{itemize}
% 			\item Ta có $2^4 \geqslant 4^2$.
% 			\item Nếu $2^k \geqslant k^2~(k \geqslant 4)$ thì $2^{k+1}=2\cdot 2^k \geqslant 2\cdot k^2>k^2+(2k+1)=(k+1)^2$ (do $k \geqslant 4$).
% 		\end{itemize}
% 		Từ đó $0<\dfrac{2n-1}{2^n}\leqslant \dfrac{2n-1}{n^2}$. Mà $\lim \limits_{n \to +\infty}\dfrac{2n-1}{n^2}=0$ nên $\lim \limits_{n \to +\infty}\dfrac{2n-1}{2^n}=0$. \\
% 		Vậy $I=\lim \limits_{n \to +\infty}S_n=\lim \limits_{n \to +\infty}\left[1+2\left[1-{\left(\dfrac{1}{2}\right)}^{n-1}\right]-\dfrac{2n-1}{2^n}\right]=1+2=3$.}
% \end{ex}
% %%==========Câu 10
% \begin{ex}%[1D4G1-2]
% 	Cho dãy số $(u_n)$ được xác định bởi $\heva{& u_1=3 \\ & 2(n+1)u_{n+1}=nu_n+n+2}$. Tính $\lim \limits_{n \to +\infty}u_n$.
% 	\choice
% 	{\True $\lim \limits_{n \to +\infty}u_n=1$}
% 	{$\lim \limits_{n \to +\infty}u_n=4$}
% 	{$\lim \limits_{n \to +\infty}u_n=3$}
% 	{$\lim \limits_{n \to +\infty}u_n=0$}
% 	\loigiai{
% 		Ta chứng minh $1\le u_{n+1}\le 1+\dfrac{1}{2n}$, $\forall n\ge 1$. Thật vậy
% 		\begin{itemize}
% 			\item $u_{n+1}\ge 1$, $\forall n\ge 1$ $(1)$.\\
% 			Với $n=1\Rightarrow u_2=\dfrac{3}{2}\ge 1\Rightarrow (1)$ đúng với $n=1$.\\
% 			Giả sử $(1)$ đúng với $n=k\ge 1$, tức là $u_{k+1}\ge 1$.\\
% 			Ta cần chứng minh $(1)$ đúng với $n=k+1$, tức là chứng minh $u_{k+2}\ge 1$. Thật vậy\\
% 			$u_{k+2}=\dfrac{(k+1) u_{k+1}+1}{2(k+2)}+\dfrac{1}{2}\ge \dfrac{k+2}{2(k+2)}+\dfrac{1}{2}=1$.
% 			\item $u_{n+1}\le 1+\dfrac{1}{2n}$, $\forall n\ge 1$ $(2)$.\\
% 			Với $n=1\Rightarrow u_2=\dfrac{3}{2}\le 1+\dfrac{3}{2}\Rightarrow (2)$ đúng với $n=1$.\\
% 			Giả sử $(2)$ đúng với $n=k\ge 1$, tức là $u_{k+1}\le 1+\dfrac{1}{2k}$.\\
% 			Ta cần chứng minh $(2)$ đúng với $n=k+1$, tức là chứng minh $u_{k+2}\le 1+\dfrac{1}{2(k+1)}$. Thật vậy
% 			$$u_{k+2}=\dfrac{(k+1)u_{k+1}+1}{2(k+2)}+\dfrac{1}{2}\le \dfrac{(k+1)\left(1+\dfrac{1}{2(k+1)}\right)}{2(k+2)}+\dfrac{1}{2}\le 1+\dfrac{1}{4(k+2)}\le 1+\dfrac{1}{2(k+1)}.$$
% 			Vậy $\lim \limits_{n \to +\infty}u_n=1$.
% 		\end{itemize}
% 	}
% \end{ex}
% %%==========Câu 11
% \begin{ex}%[1D4G1-2]
% 	Cho dãy số $\left( u_n\right) $ thỏa mãn $\heva{&u_1=\dfrac{1}{3}\\&u_{n+1}=\dfrac{(n+1)u_n}{3n},\,\, \forall n\geq 1}$. Có bao nhiêu số nguyên dương $n$ thỏa mãn $u_n<\dfrac{1}{2020}.$
% 	\choice
% 	{$0$}
% 	{$9$}
% 	{\True vô số}
% 	{$5$}
% 	\loigiai
% 	{
% 		\begin{itemize}
% 			\item Đặt $v_n=\dfrac{u_n}{n}$, ta có $v_{n+1}=\dfrac{v_n}{3}$.
% 			\item Do đó $v_n$ là cấp số nhân với công bội là $\dfrac{1}{3}$, mà $v_1=\dfrac{u_1}{1}=\dfrac{1}{3}$ nên
% 			$v_n=v_1\cdot\left(\dfrac{1}{3}\right)^{n-1}=\left(\dfrac{1}{3}\right)^n$.
% 			\item Từ đó $u_n=n\left(\dfrac{1}{3}\right)^n=\dfrac{n}{3^n}$.
% 			\item Bằng quy nạp, ta chứng minh được $3^n>n^2$, $\forall\, n\ge 1$. Khi đó $|u_n|=\dfrac{n}{3^n}<\dfrac{n}{n^2}=\dfrac{1}{n}$.\\
% 			Mà $\lim\dfrac{1}{n}=0\Rightarrow\lim \limits_{n \to +\infty}u_n=0$. Suy ra có vô số $n$ để $u_n<\dfrac{1}{2020}$.
% 		\end{itemize}
% 	}
% \end{ex}
% %%==========Câu 12
% \begin{ex}%[1D4K1-2]
% 	Tìm giới hạn của $(u_n)$ với $u_n = \dfrac{1}{\sqrt{n^2 + 1}} + \dfrac{1}{\sqrt{n^2 + }} +  \cdots + \dfrac{1}{\sqrt{n^2 + n}}$.
% 	\choice
% 	{\True $1$}
% 	{$0$}
% 	{$+\infty$}
% 	{$-\infty$}
% 	\loigiai{Với mỗi số nguyên $k$ mà $1 \leq k \leq n$, ta có $\dfrac{1}{\sqrt{n^2 + n}} \leq \dfrac{1}{\sqrt{n^2 + k}} \leq \dfrac{1}{\sqrt{n^2 + 1}}$.\\
% 		Do đó $\dfrac{n}{\sqrt{n^2 + n}} \leq u_n \leq \dfrac{n}{\sqrt{n^2 + 1}}$ với mọi $n$.\\
% 		Mặt khác $\lim \limits_{n \to +\infty}\dfrac{n}{\sqrt{n^2 + n}} = \lim \limits_{n \to +\infty}\dfrac{n}{\sqrt{n^2 + 1}} = 1$.\\
% 		Do đó $\lim \limits_{n \to +\infty}u_n = 1$.
% 	}
% \end{ex}
% %Câu 13
% \begin{ex}%[1D4K1-2]
% 	Tìm giới hạn của $(u_n)$ với $u_n = \dfrac{1}{\sqrt{n^2 + 1}} + \dfrac{1}{\sqrt{n^2 + }} +  \cdots + \dfrac{1}{\sqrt{n^2 + n}}$.
% 	\choice
% 	{\True $1$}
% 	{$0$}
% 	{$+\infty$}
% 	{$-\infty$}
% 	\loigiai{Với mỗi số nguyên $k$ mà $1 \leq k \leq n$, ta có $\dfrac{1}{\sqrt{n^2 + n}} \leq \dfrac{1}{\sqrt{n^2 + k}} \leq \dfrac{1}{\sqrt{n^2 + 1}}$.\\
% 		Do đó $\dfrac{n}{\sqrt{n^2 + n}} \leq u_n \leq \dfrac{n}{\sqrt{n^2 + 1}}$ với mọi $n$.\\
% 		Mặt khác $\lim \limits_{n \to +\infty}\dfrac{n}{\sqrt{n^2 + n}} = \lim \limits_{n \to +\infty}\dfrac{n}{\sqrt{n^2 + 1}} = 1$.\\
% 		Do đó $\lim \limits_{n \to +\infty}u_n = 1$.
% 	}
% \end{ex}
% %Câu 14
% \begin{ex}%[1D4K1-2]
% 	Kết quả đúng của $\lim \limits_{n \to +\infty}\left(5 - \dfrac{n\cos 2n}{n^2 + 1} \right)$ là
% 	\choice
% 	{$4$}
% 	{\True $5$}
% 	{$-4$}
% 	{$\dfrac{1}{4}$}
% 	\loigiai{
% 		Với mọi $n \in \mathbb{N}$ ta có $-\dfrac{n}{n^2 + 1} \le \dfrac{n\cos 2n}{n^2 + 1} \le \dfrac{n}{n^2 + 1}$.\\
% 		Ta có $\lim \limits_{n \to +\infty}\left(-\dfrac{n}{n^2 + 1}\right) = \lim \limits_{n \to +\infty}\dfrac{-\dfrac{1}{n}}{1 + \dfrac{1}{n^2}} = 0$; $\lim \limits_{n \to +\infty}\dfrac{n}{n^2 + 1} = \lim \limits_{n \to +\infty}\dfrac{\dfrac{1}{n}}{1 + \dfrac{1}{n^2}} = 0$.\\
% 		Suy ra $\lim \limits_{n \to +\infty}\left(\dfrac{n\cos 2n}{n^2 + 1}\right) = 0 \Rightarrow \lim \limits_{n \to +\infty}\left(5 - \dfrac{n\cos 2n}{n^2 + 1}\right) = 5$.
% 	}
% \end{ex}
% %Câu 15
% \begin{ex}%[1D4K1-2]
% 	Kết quả của $\lim \limits_{n \to +\infty}\left(n^2\sin \dfrac{n\pi }{5} - 2n^3\right)$ bằng
% 	\choice
% 	{\True $-\infty$}
% 	{$0$}
% 	{$+\infty$}
% 	{$-2$}
% 	\loigiai{
% 		Ta có $\lim \limits_{n \to +\infty}\left(n^2\sin \dfrac{n\pi}{5} - 2n^3\right) = \lim \limits_{n \to +\infty}n^3\left(\dfrac{1}{n}\sin \dfrac{n\pi}{5} - 2\right) = -\infty $.\\
% 		Vì $\sin \dfrac{n\pi}{5} \le 1 \Rightarrow \dfrac{1}{n} \sin \dfrac{n\pi }{5} \le \dfrac{1}{n}$.\\
% 		Mà $\lim \limits_{n \to +\infty}\dfrac{1}{n} = 0$ nên $\lim \limits_{n \to +\infty}\left(\dfrac{1}{n}\sin \dfrac{n\pi}{5} - 2\right) = -2$.\\
% 		Mặt khác $\lim \limits_{n \to +\infty}n^3 = +\infty$.\\
% 		Vậy $\lim \limits_{n \to +\infty}\left(n^2\sin \dfrac{n\pi }{5} - 2n^3\right) = -\infty$.
% 	}
% \end{ex}
% %Câu 16
% \begin{ex}%[1D4K1-2]
% 	Tính $I = \lim \limits_{n \to +\infty}\left( \dfrac{1}{\sqrt{n^2 + n + 1}} + \dfrac{1}{\sqrt {n^2 + n + 2}} + \cdots + \dfrac{1}{\sqrt{n^2 + 2n}}\right)$.
% 	\choice
% 	{$I = +\infty$}
% 	{$I = 3$}
% 	{$I = 2$}
% 	{\True $I = 1$}
% 	\loigiai{
% 		Ta có $\dfrac{1}{\sqrt{n^2 + 2n}} < \dfrac{1}{\sqrt{n^2 + n + k}} < \dfrac{1}{\sqrt{n^2 + n + 1}},\forall k = 2{,}3,\cdots ,n-1$.\\
% 		Suy ra $\dfrac{n}{\sqrt{n^2 + 2n}} < \dfrac{1}{\sqrt{n^2 + n + 1}} + \dfrac{1}{\sqrt{n^2 + n + 2}} + \cdots + \dfrac{1}{\sqrt{n^2 + 2n}} < \dfrac{n}{\sqrt{n^2 + n + 1}}$.\\
% 		Mà $\lim \limits_{n \to +\infty}\dfrac{n}{\sqrt{n^2 + 2n}} = \lim \limits_{n \to +\infty}\dfrac{1}{\sqrt {1 + \dfrac{2}{n}}} = 1$; $ \lim \limits_{n \to +\infty}\dfrac{n}{\sqrt{n^2 + n + 1}} = \lim \limits_{n \to +\infty}\dfrac{1}{\sqrt {1 + \dfrac{1}{n} + \dfrac{1}{n^2}}} = 1$.\\
% 		Vậy $I = 1$.
% 	}
% \end{ex}	
% \Closesolutionfile{ans}
% \begin{indapan}{10}
% 	{ans/ans-1K5-1-Dang6}
% \end{indapan}