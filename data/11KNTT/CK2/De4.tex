\section*{ÔN TẬP KIỂM TRA CUỐI KÌ 2 - ĐỀ 04}
\setcounter{ex}{0}\setcounter{bt}{0}
\Opensolutionfile{ans}[ans/ansBTTeX4]

\noindent\textbf{I. PHẦN TRẮC NGHIỆM:}
%Câu 1
\begin{ex}
Với $a>0$, viết biểu thức $\sqrt[5]{a^4}$ dưới dạng luỹ thừa.
\choice
{${a^{\tfrac{4}{5}}}$}
{${a^{\tfrac{5}{4}}}$}
{$a^4$}
{$a^5$}
\end{ex}
%Câu 2
\begin{ex}
Cho $a,b>0\,\left(a\ne 1\right)$ và $\log _ab=3$. Tính giá trị biểu thức $P=\log_{a^2}\left(a^6 \cdot b^5\right)$.
\choice
{$\dfrac{15}{2}$}
{$\dfrac{11}{2}$}
{$21$}
{$\dfrac{21}{2}$}
\end{ex}
%Câu 3
\begin{ex}
Cho $0<a\ne 1$, tính giá trị biểu thức $P=a^{\log _a3}+\log _a(2a)$.
\choice
{$4+\log _a2$}
{$3+\log _a2$}
{$1+\log _a2$}
{$4\log _a2$}
\end{ex}
%Câu 4
\begin{ex}
Tìm số giá trị nguyên của tham số $m\in [-10;10]$ để hàm số $y=(3m-4)^x$ đồng biến trên tập xác định:
\choice
{$13$}
{$9$}
{$11$}
{$14$}
\end{ex}
%Câu 5
\begin{ex}
Tập xác định của hàm số $y=\log _3(x+1)-2$ là
\choice
{$\left(-1;+\infty\right)$}
{$\left[-1;+\infty\right)$}
{$\left(0;+\infty\right)$}
{$\mathbb{R}$}
\end{ex}
%Câu 6
\begin{ex}
Cho phương trình $\log _2^2x^2-\log_{\sqrt{2}}(2x)-4=0$. Đặt $t=\log _2x$ ta được phương trình nào sau đây:
\choice
{$2t^2-2t-6=0$}
{$4t^2-2t-6=0$}
{$2t^2-2t-4=0$}
{$4t^2-2t-5=0$}
\end{ex}
%Câu 7
\begin{ex}
Cho hình chóp $S.ABCD$ có đáy $ABCD$ là hình bình hành. Góc giữa hai đường thẳng $SD$ và $BC$ bằng
\choice
{Góc giữa hai đường thẳng $SD$ và$AD$}
{Góc giữa hai đường thẳng $SD$ và $DC$}
{Góc giữa hai đường thẳng $SD$ và $BD$}
{Góc giữa hai đường thẳng $SD$ và $SC$}
\end{ex}
%Câu 8
\begin{ex}
Cho hình chóp $S.ABCD$ có $SA$ vuông góc với mặt phẳng $(ABCD)$. Mệnh đề nào sau đây là đúng?
\choice
{$SA\perp SB$}
{$SA\perp SC$}
{$SB\perp SD$}
{$SA\perp AC$}
\end{ex}
%Câu 9
\begin{ex}
Cho hình lập phương $ABCD \cdot A'B'C'D'$. Mặt phẳng $\left(AB'D'\right)$ vuông góc với đường thẳng nào sau đây?
\choice
{$A'D$}
{$A'C'$}
{$A'C$}
{$A'B$}
\end{ex}
%Câu 10
\begin{ex}
Cho hình chóp $S.ABCD$ có đáy $ABCD$ là hình vuông, $SA\perp (ABCD)$. Khẳng định nào sau đây sai?
\choice
{$BC\perp (SAB)$}
{$AC\perp (SBD)$}
{$CD\perp (SAD)$}
{$BD\perp (SAC)$}
\end{ex}
%Câu 11
\begin{ex}
Cho hình chóp $S.ABCD$ có $ABCD$ là hình chữ nhật, $SA\perp (ABCD)$. Khẳng định nào sau đây sai?
\choice
{$(SBC)\perp (SCD)$}
{$(SAB)\perp (ABCD)$}
{$(SCD)\perp (SAD)$}
{$(SAD)\perp (ABCD)$}
\end{ex}
%Câu 12
\begin{ex}
Cho hình chóp $S.ABC$ có $ABC$ là tam giác vuông tại $A$, $SA\perp (ABC)$. Khẳng định nào sau đây sai?
\choice
{$(SBC)\perp (ABC)$}
{$(SAB)\perp (ABC)$}
{$(SAC)\perp (ABC)$}
{$(SAB)\perp (SAC)$}
\end{ex}
%Câu 13
\begin{ex}
Cho lăng trụ đứng $ABC \cdot A'B'C'$ có đáy là tam giác $ABC$ vuông cân tại $A$. Gọi $M$ là trung điểm của $BC$, mệnh đề nào sau đây sai.
\choice
{$\left(ABB'\right)\perp \left(ACC'\right)$}
{$\left(AC'M\right)\perp (ABC)$}
{$\left(AMC'\right)\perp \left(BCC'\right)$}
{$(ABC)\perp \left(ABA'\right)$}
\end{ex}
%Câu 14
\begin{ex}
Cho hình chóp $S.ABC$ có đáy $ABC$ là tam giác vuông cân tại $B$, $SA$ vuông góc với đáy. Gọi $M$ là trung điểm $AC$. Khẳng định nào sau đây sai?
\choice
{$BM\perp SC$}
{$(SBM)\perp (SAC)$}
{$(SAB)\perp (SBC)$}
{$(SAB)\perp (SAC)$}
\end{ex}
%Câu 15
\begin{ex}
Cho hình chóp $S.ABC$ có đáy là tam giác đều cạnh bằng$a$ Biết $SA$ vuông góc với mặt phẳng $(ABC)$. Khoảng cách từ $C$ đến mặt phẳng $(SAB)$ bằng
\choice
{$\dfrac{a\sqrt{5}}{15}$}
{$\dfrac{a\sqrt{5}}{5}$}
{$a\sqrt{3}$}
{$\dfrac{a\sqrt{3}}{2}$}
\end{ex}
%Câu 16
\begin{ex}
Cho hình lập phương $ABCD \cdot A'B'C'D'$, biết $AB=6$. Khoảng cách từ điểm $D$ đến mặt phẳng $\left(ACC'A'\right)$ bằng
\choice
{$6\sqrt{2}$}
{$6$}
{$3$}
{$3\sqrt{2}$}
\end{ex}
%Câu 17
\begin{ex}
Cho hình chóp $S.ABCD$ có đáy $ABCD$ là hình bình hành, $SA\perp (ABCD)$. Khi đó góc giữa $SB$ với mặt đáy là
\choice
{$\widehat{SAB}$}
{$\widehat{SBD}$}
{$\widehat{SBC}$}
{$\widehat{SBA}$}
\end{ex}
%Câu 18
\begin{ex}
Cho hình chóp $S.ABC$ có đáy $ABC$ là tam giác đều cạnh $a$, cạnh bên $SA$ vuông góc với đáy. Khi đó số đo góc phẳng nhị diện $\left[B, SA, C\right]$ bằng
\choice
{$60^\circ$}
{$45^\circ$}
{${{90}^\circ}$}
{${{30}^\circ}$}
\end{ex}
%Câu 19
\begin{ex}
Cho hình chóp $S.ABCD$ có đáy $ABCD$ là hình vuông, cạnh bên $SA$ vuông góc với mặt phẳng đáy. Góc phẳng nhị diện $[S$, $BC$, $A]$ là
\choice
{$\widehat{S B A}$}
{$\widehat{SCD}$}
{$\widehat{A S C}$}
{$\widehat{A S B}$}
\end{ex}
%Câu 20
\begin{ex}
Cho khối chóp cụt tam giác đều $ABC \cdot A'B'C'$ có chiều cao bằng $3a$, $A'B'=4a$, $AB=a$. Thể tích khối chóp cụt tam giác đều $ABC.A'B'C'$ bằng
\choice
{$\dfrac{63\sqrt{3}a^3}{4}$}
{$\dfrac{21\sqrt{3}a^3}{4}$}
{$\dfrac{21\sqrt{3}a^3}{12}$}
{$\dfrac{12\sqrt{3}a^3}{4}$}
\end{ex}
%Câu 21
\begin{ex}
Cho hình chóp $S.ABC$ có đáy $ABC$ là tam giác vuông tại $B$, $AB=2a,\widehat{BAC}=60^\circ $. Cạnh bên $SA$ vuông góc với mặt phẳng $(ABC)$ và $SA=a\sqrt{3}$. Thể tích khối chóp $S.ABC$ theo $a$ bằng
\choice
{${{V}_{SABC}}=2a^3$}
{${{V}_{SABC}}=a^3$}
{${{V}_{SABC}}=\dfrac{2a^3}{3}$}
{${{V}_{SABC}}=\dfrac{a^3}{3}$}
\end{ex}
%Câu 22
\begin{ex}
Xét một phép thử có một số hữu hạn kết quả đồng khả năng xuất hiện (có không gian mẫu là $\Omega $) và $A$ là một biến cố của phép thử đó. Phát biểu nào dưới đây là sai?
\choice
{$P(A)=1+P\left(\overline{A}\right)$}
{$0\le P(A)\le 1$}
{Xác suất của biến cố$A$ là $P(A)=\dfrac{n(A)}{n\left(\Omega\right)}$}
{$P\left(\Omega\right)=1$}
\end{ex}
%Câu 23
\begin{ex}
Một chiếc máy bay có hai động cơ I và II hoạt động độc lập với nhau. Xác suất để động cơ I chạy tốt là $0{,}8$ và xác suất để động cơ II chạy tốt là $0{,}7$. Tính xác suất để cả hai động cơ cùng chạy tốt?
\choice
{$0{,}06$}
{$0{,}56$}
{$0{,}44$}
{$0{,}94$}
\end{ex}
%Câu 24
\begin{ex}
Một tổ học sinh có $6$ nam và $4$ nữ. Chọn ngẫu nhiên $2$ học sinh. Tính xác suất sao cho hai học sinh được chọn đều là nữ.
\choice
{$\dfrac{2}{15}$}
{$\dfrac{7}{15}$}
{$\dfrac{8}{15}$}
{$\dfrac{1}{3}$}
\end{ex}
%Câu 25
\begin{ex}
Một lớp có 38 học sinh. Trong đó có 17 học sinh khá môn Toán, 15 học sinh khá môn Ngữ Văn, 8 học sinh khá cả môn Toán và môn Ngữ Văn. Chọn ngẫu nhiên một học sinh trong lớp. Xác suất để chọn được học sinh hoặc khá môn toán hoặc khá môn văn hoặc khá cả 2 môn?
\choice
{$\dfrac{32}{38}$}
{$\dfrac{8}{38}$}
{$\dfrac{30}{38}$}
{$\dfrac{24}{38}$}
\end{ex}
%Câu 26
\begin{ex}
Cho tập $E=\left\{ 1,2,3,4,5,6,7 \right\}$. Viết ngẫu nhiên lên bảng hai số tự nhiên, mỗi số gồm $3$ chữ số đôi một khác nhau thuộc tập $E$. Tính xác suất để trong hai số đó có đúng một số có chữ số $5$.
\choice
{$P=\dfrac{3}{4}$}
{$P=\dfrac{24}{49}$}
{$P=\dfrac{25}{49}$}
{$P=\dfrac{1}{4}$}
\end{ex}
%Câu 27
\begin{ex}
Có hai hộp chứa các quả cầu. Hộp thứ nhất chứa $3$ quả cầu đỏ và $5$ quả cầu xanh. Hộp thứ hai chứa $7$ quả cầu đỏ và $4$ quả cầu xanh. Lấy từ mỗi hộp lấy ngẫu nhiên một quả cầu. Tính xác suất để hai quả cầu lấy ra cùng màu xanh.
\choice
{$\dfrac{67}{88}$}
{$\dfrac{5}{22}$}
{$\dfrac{17}{22}$}
{$\dfrac{21}{88}$}
\end{ex}
%Câu 28
\begin{ex}
Một hộp đựng $11$ tấm thẻ được đánh số từ $1$ đến $11$. Chọn ngẫu nhiên $6$ tấm thẻ. Gọi $P$ là xác suất để tổng số ghi trên $6$ tấm thẻ ấy là một số lẻ. Khi đó $P$ bằng
\choice
{$\dfrac{100}{231}$}
{$\dfrac{115}{231}$}
{$\dfrac{118}{231}$}
{$\dfrac{1}{2}$}
\end{ex}
%Câu 29
\begin{ex}
Cho hàm số $f(x)$ liên tục tại $x_0$. Đạo hàm của $f(x)$ tại $x_0$ là
\choice
{$f\left(x_0\right)$}
{$ \lim \limits_{x\to x_0} \,\dfrac{f(x)-f(x_0)}{x-x_0}$ (nếu tồn tại giới hạn)}
{$ \lim \limits_{x\to 0} \,\dfrac{f(x)-f(x_0)}{x-x_0}$ (nếu tồn tại giới hạn)}
{$\dfrac{f(x)-f(x_0)}{x-x_0}$}
\end{ex}
%Câu 30
\begin{ex}
Phương trình tiếp tuyến của đồ thị hàm số $y=x^3-3x+2$ tại điểm có hoành độ $x=2$ là
\choice
{$y=8x+4$}
{$y=9x-18$}
{$y=9x+18$}
{$y=4x+4$}
\end{ex}
%Câu 31
\begin{ex}
Cho hàm số $y=-x^3+3x-2$ có đồ thị $(C)$. Viết phương trình tiếp tuyến của $(C)$ tại giao điểm của $(C)$ với trục tung.
\choice
{$y=-3x+2$}
{$y=-3x-2$}
{$y=3x-2$}
{$y=-2x+3$}
\end{ex}
%Câu 32
\begin{ex}
Đạo hàm của hàm số $y=\dfrac{x^4}{2}+\dfrac{5x^3}{3}-\sqrt{2x}+a^2$($a$ là hằng số) bằng
\choice
{$2x^3+5x^2-\dfrac{1}{\sqrt{2x}}+2a$}
{$2x^3+5x^2+\dfrac{1}{2\sqrt{2x}}$}
{$2x^3+5x^2-\dfrac{1}{\sqrt{2x}}$}
{$2x^3+5x^2-\sqrt{2}$}
\end{ex}
%Câu 33
\begin{ex}
Cho hàm số $y=x^3-3x+2017$. Bất phương trình $y'<0$ có tập nghiệm là
\choice
{$S=(-1;1)$}
{$S=\left(-\infty ;-1\right)\cup \left(1;+\infty\right)$}
{$S=\left(1;+\infty\right)$}
{$S=\left(-\infty ;-1\right)$}
\end{ex}
%Câu 34
\begin{ex}
Đạo hàm của ${e^{x^2+1}}$ bằng
\choice
{$2x \cdot {e^{x^2+1}}$}
{$x \cdot {e^{x^2+1}}$}
{$2x \cdot {e^{x^2}}$}
{$\left(x^2+1\right) \cdot {e^{x^2}}$}
\end{ex}
%Câu 35
\begin{ex}
Một chuyển động thẳng xác định bởi phương trình $s(t)=t^3-3t^2-9t+2$, trong đó $t>0$, $t$ tính bằng giây (s) và $s$ tính bằng mét (m). Tính gia tốc tại thời điểm vận tốc triệt tiêu.
\choice
{$12\mathrm{(m/s^2)}$}
{$-12\mathrm{(m/s^2)}$}
{$6\mathrm{(m/s^2)}$}
{$18\mathrm{(m/s^2)}$}
\end{ex}


\noindent\textbf{II. PHẦN TỰ LUẬN}
%Câu 36 
\begin{ex}
Giải phương trình ${9^{|3x-1|}}={3^{8x-2}}$
\end{ex}
%Câu 37
\begin{ex}
Viết phương trình tiếp tuyến đồ thị hàm số $y=x^3-x^2+3x-3$ tại giao điểm với trục hoành.
\end{ex}
%Câu 38
\begin{ex}
Trong một hộp có  8 viên bi xanh và  5 viên bi đỏ. Hạnh lấy ngẫu nhiên một viên bi . Tiếp đó đến lượt Phúc lấy ngẫu nhiên một viên bi. Xác suất để Phúc lấy được viên bi xanh là.
\end{ex}
%Câu 39
\begin{ex}
Cho hình chóp $S \cdot ABC$ có đáy $ABC$ là tam giác vuông tại $B$, $AB=a$, $AC=a\sqrt{3}$, $SB<2a$ và $\widehat{BAS}=\widehat{BCS}=90^\circ $. Biết sin của góc giữa đường thẳng $SB$ và mặt phẳng $(SAC)$ bằng $\dfrac{\sqrt{11}}{11}$. Tính thể tích của khối chóp $S.ABC$
\end{ex}


\Closesolutionfile{ans}