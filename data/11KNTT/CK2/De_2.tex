\begin{name}
	{\tenchude}
	{\tendethi}
	{\tentruong}
	{\thoigian}
\end{name}
\Opensolutionfile{ansbook}[ans/ansbookDe2]
\TN
\Opensolutionfile{ans}[ans/ansDe2-TN1]
\begin{ex}%[1D6N1-1]::Cau 4::
Biểu thức $K=\sqrt[3]{\dfrac{2}{3}\sqrt[3]{\dfrac{2}{3}\sqrt{\dfrac{2}{3}}}}$ viết dưới dạng luỹ thừa với số mũ hữu tỉ là
\choice
{$\left(\dfrac{2}{3}\right)^{\frac{5}{18}}$}
{\True $\left(\dfrac{2}{3}\right)^{\frac{1}{2}}$}
{$\left(\dfrac{2}{3}\right)^{\frac{1}{8}}$}
{$\left(\dfrac{2}{3}\right)^{\frac{1}{6}}$}
\loigiai{
Coi $X=\dfrac{2}{3}$. Theo nguyên tắc \lq\lq Chia cộng\rq\rq  ta có\\
$K=\sqrt[3]{X\sqrt[3]{X\sqrt{X}}}=\sqrt[3]{X\sqrt[3]{X\cdot X^{\frac{1}{2}}}}=\sqrt[3]{X\sqrt[3]{X^{\frac{3}{2}}}}=\sqrt[3]{X\cdot X^{\frac{1}{2}}}=\sqrt[3]{X^{\frac{3}{2}}}=X^{\frac{1}{2}}$.
}
\end{ex}

\begin{ex}%[1D6N2-2]%[Dự án đề kiểm tra Toán 11 GHKII NH23-24 - Nguyễn Ngọc Dũng]%[De-3]
Với số thực dương $a$ bất kì, giá trị của $\log_2{2a}$ bằng
\choice
{\True $ 1+ \log_2{a}$}
{$2+ \log_2{a}$}
{$4+ \log_2{a} $}
{$2\log_2{a}$}
\loigiai{
Ta có $\log_2{2a}=\log_2{2}+\log_2{a}=1+ \log_2{a}$.
}
\end{ex}

\begin{ex}%[1D6N3-1]
Tìm tập xác định $\mathscr{D}$ của hàm số $y=\left(1-x^2\right)^{\sqrt{3}}+x^{-3}$.
\choice
{\True$\mathscr{D}=(-1 ; 1) \setminus\{0\}$}
{$\mathscr{D}=(0 ; 1)$}	{$\mathscr{D}=(-1 ; 1)$} {$\mathscr{D}=\mathbb{R}\setminus\{-1 ; 1\}$}
\loigiai{
Điều kiện xác định là $\heva{&1-x^2>0\\&x\ne 0}\Leftrightarrow \heva{&-1<x<1\\&x\ne 0.}$\\
Vậy $\mathscr{D}=(-1 ; 1) \setminus\{0\}$.
}
\end{ex}

\begin{ex}%[Tex hóa đề CK2 - form 2025 - đợt 1 - Phạm Hoài]%[1H8N1-3]
	\immini{Cho hình chóp $S.ABCD$ có đáy $ABCD$ là hình bình hành (hình vẽ minh hoạ). Góc giữa hai đường thẳng $SD$ và $BC$ bằng
	\choice
	{Góc giữa hai đường thẳng $SD$ và $DC$}
	{\True Góc giữa hai đường thẳng $SD$ và $AD$}
	{Góc giữa hai đường thẳng $SD$ và $BD$}
	{Góc giữa hai đường thẳng $SD$ và $SC$}}
	{\begin{tikzpicture}[>=stealth,line join=round,line cap=round,font=\footnotesize,scale=1]
	\path
	(0,0) coordinate (A)
	(-1,-1.2) coordinate (B)
	(3,0) coordinate (D)
	($(B)+(D)-(A)$) coordinate (C)
	($(A)!0.7!90:(D)$) coordinate (S)
	;
	\draw (S)--(B)--(C)--(S)--(D)--(C);
	\draw[dashed](S)--(A)--(B)(A)--(D);
	\foreach \x/\y in {A/180,B/180,C/-20,S/90,D/10}
	\draw[fill=black] (\x) circle (1.1pt) + (\y:0.3cm) node{$\x$};
	\draw pic[draw, angle radius=2mm, angle eccentricity=1.5]{right angle = D--A--S};
	\end{tikzpicture}
	}
	\loigiai{
	Do $BC\parallel AD$ nên góc giữa hai đường thẳng $SD$ và $BC$ bằng góc giữa hai đường thẳng $SD$ và $AD$.}
	\end{ex}

\begin{ex}%[10-11-12EX-HK1-2425]%[Tran Quoc]%[1H8N2-1]
Trong không gian, cho đường thẳng $\Delta$ và mặt phẳng $(\alpha)$. Khẳng định nào sau đây {\bf sai}?
\choice
{$\Delta$ vuông góc với $(\alpha)$ nếu $\Delta$ vuông góc với mọi đường thẳng nằm trong $(\alpha)$}
{$\Delta$ vuông góc với $(\alpha)$ thì $\Delta$ cắt $(\alpha)$ tại một điểm}
{$\Delta$ vuông góc với $(\alpha)$ nếu $\Delta$ vuông góc với hai đường thẳng cắt nhau nằm trong $(\alpha)$}
{\True $\Delta$ vuông góc với $(\alpha)$ nếu $\Delta$ vuông góc với hai đường thẳng bất kỳ nằm trong $(\alpha)$}
\loigiai{Khẳng định \lq\lq $\Delta$ vuông góc với $(\alpha)$ nếu $\Delta$ vuông góc với hai đường thẳng bất kỳ nằm trong $(\alpha)$\rq\rq~là khẳng định sai.}
\end{ex}

\begin{ex}%[Cường Lee Minh]%[Tài liệu DT Toán 11 -đợt 3]%[1H8N4-1]
	Cho hình chóp đều $S.ABC$. Gọi $d$ và $\Delta$ là hai đường lần lượt vuông góc với mặt phẳng $\left(ABC\right)$ và mặt phẳng $\left(SBC\right)$. Mệnh đề nào sau đây đúng?
	\choice
	{\True $\left(\left(SBC\right),\left(ABC\right)\right)=\left(\Delta, d\right)$}
	{$\left(\left(SBC\right),\left(ABC\right)\right)=60^\circ$}
	{$\left(\left(SBC\right),\left(ABC\right)\right)=\left(\Delta, \left(ABC\right)\right)$}
	{$\left(\left(SBC\right),\left(ABC\right)\right)=\left(\Delta, \left(SBC\right)\right)$}
	\loigiai{
	Từ định nghĩa góc giữa hai mặt phẳng nên $\left(\left(SBC\right),\left(ABC\right)\right)=\left(\Delta, d\right)$.
	}
	\end{ex}

\begin{ex}%[1H8N6-1]
Cho hình chóp $S.ABCD$ có đáy $ABCD$ là hình vuông tâm $O$ và $SB \perp (ABCD)$. Khi đó góc giữa đường thẳng $SO$ và mặt phẳng $(ABCD)$ là góc
\choice
{$\widehat{SOD}$}
{$\widehat{SOA}$}
{$\widehat{SOC}$}
{\True $\widehat{SOB}$}
\loigiai{
Ta có $SB \perp (ABCD)$ nên $BO$ là hình chiếu vuông góc của $SO$ lên mặt phẳng $(ABCD)$ hay
\[\left(SO,(ABCD)\right)=\left(SO,BO\right)=\widehat{SOB}.\]
}
\end{ex}

\begin{ex}%[1H8N7-1]%[Dự án đề kiểm tra Toán 11 HK II NH23-24 - Nguyễn Thành Sơn]%[THPT Nguyễn Thái Bình - Tp.HCM]
Thể tích của khối lăng trụ có diện tích đáy bằng $9$, chiều cao bằng $2$ là
\choice
{$9$}
{\True $18$}
{$6$}
{$54$}
\loigiai{
Thể tích của khối lăng trụ có diện tích đáy bằng $9$, chiều cao bằng $2$ là $9\cdot 2=18$.
}
\end{ex}

\begin{ex}%[1D9N1-1]%[Dự án đề kiểm tra Toán 11 CHKII NH23-24- Đoàn Thanh Phong]%[THPT Nguyễn Quốc Trinh - Hà Nội]
Một hộp có $30$ tấm thẻ được đánh số từ $1$ đến $30$. Lấy ngẫu nhiên một tấm thẻ từ hộp. Xét các biến cố sau:\\
$P$ : \lq\lq Số ghi trên thẻ được lấy là số chia hết cho $2$ \rq\rq.\\
$Q$ : \lq\lq Số ghi trên thẻ được lấy là số chia hết cho $4$ \rq\rq.\\
Khi đó biến cố $PQ$ là
\choice
{\lq\lq Số ghi trên thẻ được lấy là số chia hết cho $6$ \rq\rq}
{\lq\lq Số ghi trên thẻ được lấy là số chia hết cho $3$ \rq\rq}
{\True \lq\lq Số ghi trên thẻ được lấy là số chia hết cho $4$ \rq\rq}
{\lq\lq Số ghi trên thẻ được lấy là số chia hết cho $10$ \rq\rq}
\loigiai{
Biến cố $PQ$ là \lq\lq Số ghi trên thẻ được lấy là số chia hết cho $4$ \rq\rq.
}
\end{ex}

\begin{ex}%[1D9N2-1]%[Dự án đề kiểm tra Toán 11 HKII NH23-24- Hector Tran]%[THPT Trần Phú- Tp HCM]
Cho $A$, $B$ là hai biến cố xung khắc. Đẳng thức nào sau đây đúng?
\choice
{$\mathrm {P}(A \cup B)=\mathrm {P}(A)\cdot \mathrm {P}(B)$}
{$\mathrm {P}(A \cap B)=\mathrm {P}(A)+\mathrm {P}(B)$}
{\True $\mathrm {P}(A \cup B)=\mathrm {P}(A)+\mathrm {P}(B)$}
{$\mathrm {P}(A \cup B)=\mathrm {P}(A)-\mathrm {P}(B)$}
\loigiai{
Vì  $A$, $B$ là hai biến cố xung khắc nên $\mathrm {P}(A \cup B)=\mathrm {P}(A)+\mathrm {P}(B)$.
}
\end{ex}

\begin{ex}%[1D7N1-1]%[Dự án đề kiểm tra Toán 11 HKII NH23-24- PHẠM CHÍ DŨNG]%[THPT HÙNG VƯƠNG TPHCM]
Cho hàm số $y=f(x)$ xác định trên $\mathbb{R}$ thỏa mãn $\lim\limits_{x \to 3} \dfrac{f(x)-f(3)}{x-3}=2$. Khẳng định nào sau đây là \textbf{đúng}?
\choice
{$f'(x)=3$}
{\True $f'(3)=2$}
{$f'(x)=2$}
{$f'(2)=3$}
\loigiai{
Theo định nghĩa đạo hàm của hàm số, ta có $\lim\limits_{x \to 3} \dfrac{f(x)-f(3)}{x-3}=f'(3)$ $\Rightarrow$ $f'(3)=2$.
}
\end{ex}

\begin{ex}%[1D7N1-4]%[Tex hóa đề CK2 - form 2025 - đợt 2 - Nguyễn Tiến]
	Nếu hàm số $y=f(x)$ biểu thị quãng đường di chuyển của vật theo thời gian $t$ thì $f'(t_0)$ biểu thị điều gì?
	\choice
	{Gia tốc của chuyển động tại thời điểm $t_0$}
	{Vị trí của chuyển động tại thời điểm $t_0$}
	{\True Vận tốc tức thời của chuyển động tại thời điểm $t_0$}
	{Quãng đường đã di chuyển của vật tại thời điểm $t_0$}
	\loigiai{
	Ta có $f'(t_0)$ biểu thị vận tốc tức thời của chuyển động tại thời điểm $t_0$.
	}
	\end{ex}
\Closesolutionfile{ans}

\TNTF
\Opensolutionfile{ans}[ans/ansDe2-TN2]
\begin{ex}%[1D7H2-3]%[Dự án đề kiểm tra Toán 11 HK2 NH23-24- Nguyễn Hữu Duy]%[THPT Chuyên Hùng vương - Phú Thọ]
	Cho hàm số $y = f(x) = \sqrt{x}$, có đồ thị $(C)$
	\choiceTF
	{\True Hàm số có đạo hàm trên $(0;+\infty)$}
	{\True $f'(9) = \dfrac{1}{6}$}
	{Hàm số $y = f(x^2 + 1)$ có đạo hàm là $y' = \dfrac{1}{2\sqrt{x^2 + 1}}$ trên $\mathbb{R}$}
	{Gọi $M$ là điểm thuộc $(C)$ có hoành độ bằng $4$, tiếp tuyến của $(C)$ tại $M$ có hệ số góc bằng $\dfrac{1}{2}$}
	\loigiai{
	\begin{itemchoice}
	\itemch \textbf{Đúng.} Tập xác định $\mathscr{D} = [0;+\infty)$ và hàm số có đạo hàm trên $(0;+\infty)$.
	\itemch \textbf{Đúng.} Ta có $f'(x) = \dfrac{1}{2\sqrt{x}} \Rightarrow f'(9) = \dfrac{1}{2\sqrt{9}} = \dfrac{1}{6}$.
	\itemch \textbf{Sai.} Ta có $y' = \left(f(x^2 + 1)\right)'= \dfrac{(x^2 + 1)'}{2 \sqrt{x^2 + 1}} = \dfrac{2x}{2 \sqrt{x^2 + 1}} = \dfrac{x}{\sqrt{x^2 + 1}}.$
	\itemch \textbf{Sai.} Hệ số góc của tiếp tuyến tại điểm $M$ là $k = f'(4) = \dfrac{1}{2\sqrt{4}} = \dfrac{1}{4}$.
	\end{itemchoice}
	}
	\end{ex}

\begin{ex}%[1D7H2-1]%[tex hóa ck2-form 2025-dot 2-Nguyễn Chín Em]
Cho hai hàm số $ f(x)=\dfrac{3}{x+1}$ và $ g(x)=\dfrac{x^2}{x+2}$. Xét tính đúng sai của các mệnh đề sau.
\choiceTF
{\True $f'(x)=-\dfrac{3}{\left(x+1\right)^2},\forall x\ne-1$}
{\True $\left[g(x)\right]'=\dfrac{x^2+4x}{\left(x+2\right)^2},\forall x\ne-2$}
{ $\left[f(x)\cdot g(x)\right]'=f'(x)\cdot g'(x)\,\forall x\in \mathbb{R} \setminus \left\{-1,-2\right\}$}
{$g'(1)=-1$}
\loigiai{
\begin{itemchoice}
\itemch $f'(x)=-\dfrac{3}{\left(x+1\right)^2},\forall x\ne-1$.
\itemch  $\left[g(x)\right]'=\dfrac{2x\cdot \left(x+2\right)-x^2}{\left(x+2\right)^2}=\dfrac{x^2+4x}{\left(x+2\right)^2}$, $\forall x\ne-2$.
\itemch $\left[f(x).g(x)\right]'=f'(x)\cdot g'(x)+f(x)\cdot g'(x)\,\forall x\in \mathbb{R} \setminus \left\{-1.-2\right\}$.
\itemch $g'(1)=\dfrac{5}{9}$.
\end{itemchoice}}
\end{ex}
\Closesolutionfile{ans}

\TNSA
\Opensolutionfile{ans}[ans/ansDe2-TN3]
\begin{ex}%[10-11EX-GK2-2425]%[Phan Anh]%[1D6H4-3]
Biết rằng tập nghiệm $S$ của bất phương trình $3^{x^2-2x}<27$ có dạng $S=(a;b)$. Tính giá trị $b^2-a^2$.
\par\shortans[oly]{8}
\loigiai{Ta có $3^{x^2-2x}<27\Leftrightarrow x^2-2x<3\Leftrightarrow x^2-2x-3<0\Leftrightarrow-1<x<3$.\\
Vậy $a=-1$, $b=3$ nên $b^2-a^2=8$.}
\end{ex}

\begin{ex}%[1H8H5-2]
Cho hình chóp $S.ABCD$ có đáy $ABCD$ là hình vuông cạnh $a$, $SA \perp (ABCD)$ và $SA=\sqrt{6}$. Tính khoảng cách từ $A$ đến đường thẳng $SC$.
\shortans{$2$}
\loigiai{
\begin{center}
\begin{tikzpicture}[scale=1.0,font=\footnotesize, line join=round, line cap=round, >=stealth]
\coordinate (A) at (0,0);
\coordinate (D) at (5,0);
\coordinate (B) at (-2,-2);
\coordinate (C) at ($(B)+(D)-(A)$);
\coordinate (S) at ($(A)+(0,4)$);
\coordinate (H) at ($(S)!0.4!(C)$);
\foreach \x/\g in {A/160,B/-100,C/-60,D/20,S/90,H/0} \fill[black](\x) circle (1.5pt) ($(\x)+(\g:3mm)$) node{$\x$};
\draw (S)--(B)--(C)--(D)--(S)--(C) ;
\draw[dashed] (D)--(B)--(A)--(D) 	(S)--(A) (A)--(C) (A)--(H);
\draw ($ (H)!5pt!(S)$)--($(H)!2!($($(H)!5pt!(S)$)!.5!($(H)!5pt!(A)$)$)$)--($(H)!5pt!(A)$);
\end{tikzpicture}
\end{center}
Ta có $SA \perp (ABCD) \Rightarrow SA \perp AC$.\\
Kẻ $AH \perp SC$, suy ra  $\mathrm{d}(A,SC)=AH$.\\
Ta có $\triangle SAC$ vuông tại $A$ nên $ \dfrac{1}{AH^2}=\dfrac{1}{SA^2}+\dfrac{1}{AC^2} \Rightarrow AH=2$.\\
}
\end{ex}

\begin{ex}%[1D9V2-4]%[Dự án - 2025_TLDT Toán 11]%[Nguyễn Tiến Liên]
Hai bạn Chiến và Công cùng chơi cờ với nhau. Trong một ván cờ, xác suất Chiến thắng Công là $0{,}3$ và xác suất để Công thắng Chiến là $0{,}4$. Hai bạn dừng chơi khi có người thắng, người thua. Tính xác suất để hai bạn dừng chơi sau hai ván cờ.
\shortans{$0{,}21$}
\loigiai{
Gọi $A$ là biến cố: \lq\lq Chiến thắng Công trong ván cờ\rq\rq, $B$ là biến cố: \lq\lq Công thắng Chiến trong ván cờ\rq\rq, $C$: \lq\lq Công và Chiến hoà nhau trong ván cờ\rq\rq.\\
Dễ thấy $A$, $B$, $C$ là các biến cố xung khắc.\\
Theo giả thiết thì ván đấu thứ nhất hai bạn hoà nhau, ván đấu thứ hai sẽ có thắng thua.\\
Xét ván thứ nhất: $\mathrm{P}(C)=1-\mathrm{P}(A)-\mathrm{P}(B)=1-0{,}3-0{,}4=0{,}3$.\\
Xét ván thứ hai: $\mathrm{P}(A\cup B)=\mathrm{P}(A)+\mathrm{P}(B)=0{,}3+0{,}4=0{,}7$.\\
Xác suất để hai bạn dừng chơi sau hai ván đấu là $\mathrm{P}=0{,}3\cdot 0{,}7=0{,}21$.
}
\end{ex}

\begin{ex} 	%[1D7V2-8] -TLN
	Chuyển động của một hạt trên một dây rung được cho bởi công thức $ s\left( t \right)=10+\sqrt {2}\sin \left( 4\pi t+\dfrac{\pi }{6} \right)$, trong đó $ s$ tính bằng centimét và $ t$ tính bằng giây. Vận tốc cực đại của hạt là bao nhiêu (làm tròn kết quả đến chữ số thập phân thứ nhất).\\
	\shortans{$17{,}8$}
	\loigiai{
	Vận tốc của hạt sau $ t$ giây là: $ v\left( t \right)={s}'\left( t \right)=4\pi \sqrt {2}\text{cos}\left( 4\pi t+\dfrac{\pi }{6} \right)$.\\
	Vận tốc cực đại của hạt là: ${v_{\text{max}}}=4\pi \sqrt {2}\approx 17{,}8$m/s, đạt được khi $| \text{cos}( 4\pi t+\dfrac{\pi }{6} ) |=1$ hay $ t=\dfrac{5}{24}+\dfrac{k}{4},k\in \mathbb{N}$.\\
	}
	\end{ex}
\Closesolutionfile{ans}

\TL
\begin{ex}%[1D7H2-1]
Cho hàm số $y=-4x^3+\dfrac{x^2}{2}-2x+3$. Xác định giao điểm giữa đồ thị hàm số $y'$ và đường thẳng $y=3$.
\loigiai{
Ta có $y'=-4\cdot 3\cdot x^2+\dfrac{1}{2} \cdot 2\cdot x-2+0=-12x^2+x-2$.\\
Xét phương trình hoành độ giao điểm $y'=3\Leftrightarrow -12x^2+x-5=0$ vô nghiệm.\\
Vậy không tồn tại giao điểm giữa đồ thị hàm số $y'$ và đường thẳng $y=3$.
}
\end{ex}

\begin{ex}%[Tex hóa Dự án Toán từ tâm K11 Đợt 3,Nhật Thiện]%[1H8V5-4]
Cho hình chóp $S.ABCD$ có đáy là hình vuông cạnh $a$, $SA=SB=SC=SD=a\sqrt{2}$. Gọi $I$, $K$ lần lượt là trung điểm của $AD$, $BC$. Tính khoảng cách giữa hai đường thẳng $SB$ và $AD$ theo $a$.
\loigiai{
\immini{Gọi $O$ là giao điểm của $AC$ và $BD$. \\
Ta có $AD\parallel BC\Rightarrow AD\parallel (SBC)$ .\\
Khi đó \[\mathrm{d}(AD, SB)=\mathrm{d}\left(AD, (SBC)\right)=\mathrm{d}\left(I, (SBC)\right).\]
Gọi $H$ là hình chiếu vuông góc của $O$ trên $SK$ .\\
Ta có $\heva{
& OH\perp SK \\
& OH\perp BC} \Rightarrow OH\perp (SBC)\Rightarrow \mathrm{d}(O, \left(SBC)\right)=OH$. \\
Trong $\Delta SAO$, ta tính được \[SO=\sqrt{SA^2-AO^2}=\sqrt{2a^2-\dfrac{a^2}{2}}=\dfrac{a\sqrt{6}}{2}.\]
Trong $\Delta SOK$ thì \[\dfrac{1}{OH^2}=\dfrac{1}{OS^2}+\dfrac{1}{OK^2}=\dfrac{4}{6a^2}+\dfrac{4}{a^2}=\dfrac{14a^2}{3}\Rightarrow OH=\dfrac{a\sqrt{42}}{14}.\]}{
\begin{tikzpicture}[scale=1, font=\footnotesize, line join=round, line cap=round, >=stealth]
\path
(0,0) coordinate (A)
(4,0) coordinate (B)
(6,1) coordinate (C)
($(A)+(C)-(B)$) coordinate (D)
($(B)!.5!(C)$) coordinate (K)
($(A)!.5!(D)$) coordinate (I)
($(A)!.5!(C)$) coordinate (O)
(O)+(0,4) coordinate (S)
($(S)!.6!(K)$) coordinate (H)
;

\draw (S)--(A)--(B)--(C)--cycle (S)--(B) (S)--(K);
\draw[dashed] (S)--(O) (I)--(K) (O)--(H) (S)--(D) (A)--(D)--(C) (D)--(B);
\foreach \x/\dinh/\y in {S/O/K,O/K/B,O/H/K} \draw[fill = gray!50] ($(\dinh)!3pt!(\x)$)--($(\dinh)!3pt!(\x)+(\dinh)!3pt!(\y)-(\dinh)$)--($(\dinh)!3pt!(\y)$)--(\dinh)--cycle;
\foreach \p/\r in {A/-120,B/-60,C/0,D/120,S/90,O/-90,H/45,K/-45,I/-120}
\fill (\p) circle (1pt) node[shift={(\r:3mm)}]{$\p$};
\end{tikzpicture}
}
\noindent
Vì $OI\cap (SBC)=\left\{K\right\}$ nên $\dfrac{\mathrm{d}(I, \left(SBC)\right)}{\mathrm{d}(O, \left(SBC)\right)}=\dfrac{IK}{OK}=2$. \\
Vậy $\mathrm{d}(AD, SB)=\mathrm{d}(I, \left(SBC)\right)=2\mathrm{d}(O, \left(SBC)\right)=2OH=\dfrac{a\sqrt{42}}{7}$.
}
\end{ex}
\begin{ex}%[1D9C1-3]
	Gieo ngẫu nhiên $n$ đồng xu. Tìm $n$ lớn nhất để xác suất tất cả các đồng xu không cùng ngửa hoặc không cùng sấp luôn nhỏ hơn $\dfrac{63}{64}$.
	\shortans{$6$}
	\loigiai{
	Không gian mẫu có $2^n$ phần tử.\\
	Gọi $A$ là biến cố các mặt không cùng sấp hoặc không cùng ngửa. Suy ra $\overline{A}$ có $2$ phần tử là $SS\ldots S$ và $NNN\ldots N $.\\
	Suy ra $P(A)=1-P(\overline{A})=1-\dfrac{2}{2^n}=1-\dfrac{1}{2^{n-1}}$.\\
	$P(A)<\dfrac{63}{64}\Leftrightarrow \dfrac{1}{2^{n-1}}>\dfrac{1}{64}\Leftrightarrow n<7$.\\
	Vậy $n$ lớn nhất là $n=6$.
	}
	\end{ex}
\Closesolutionfile{ansbook}
% \HetDe
% \label{De2}
% %
% \cleardoublepage
% \setcounter{page}{1}
% \rfoot{Trang \thepage/\pageref{DA2} - Đáp án trắc nghiệm Đề 2}
% \begin{center}
% 	\bfseries ĐÁP ÁN TRẮC NGHIỆM ĐỀ 2
% \end{center}

% \inputansbox{10}{ans/ansDe2-TN1}
% \inputansbox[3]{2}{ans/ansDe2-TN2}
% \inputansbox{3}{ans/ansDe2-TN3}
% \label{DA2}
% %
