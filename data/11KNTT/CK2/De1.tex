\section*{ÔN TẬP KIỂM TRA CUỐI KÌ 2 - ĐỀ 01}
\setcounter{ex}{0}\setcounter{bt}{0}
\Opensolutionfile{ans}[ans/ansBTTeX1]

\noindent\textbf{I. PHẦN TRẮC NGHIỆM:}
%Câu 1
\begin{ex}
Với $a$ là số thực dương tùy ý, tích $a^2 \cdot {a^{\tfrac{1}{2}}}$ bằng
\choice
{${a^{\tfrac{5}{2}}}$}
{$a$}
{${a^{\tfrac{3}{2}}}$}
{${a^{\tfrac{1}{4}}}$}
\end{ex}
%Câu 2
\begin{ex}
Hàm số nào dưới đây đồng biến trên tập xác định của nó
\choice
{$y=\left(\dfrac{1}{e}\right)^x$}
{$y={{\left(\sqrt{\dfrac{1}{\pi }}\right)}^x}$}
{$y={{\left(\dfrac{1}{3}\right)}^x}$}
{$y={{\left(\sqrt[2024]{\pi }\right)}^x}$}

\end{ex}
%Câu 3
\begin{ex}
Cho hình chóp $S \cdot ABCD$ có đáy $ABCD$ là hình bình hành (hình vẽ minh hoạ). Góc giữa hai đường thẳng $SD$ à $BC$ bằng
\choice
{Góc giữa hai đường thẳng $SD$ và $DC$}
{Góc giữa hai đường thẳng $SD$ và.$AD$}
{Góc giữa hai đường thẳng $SD$ và $BD$}
{Góc giữa hai đường thẳng $SD$ và $SC$}
\end{ex}
%Câu 4
\begin{ex}
Cho hình chóp $S \cdot ABC$ có $SA$ vuông góc với mặt phẳng đáy. Góc giữa $SB$ và mặt phẳng đáy là góc nào?
\choice
{$\widehat{ASB}$}
{$\widehat{ABC}$}
{$\widehat{SBA}$}
{$\widehat{SBC}$}
\end{ex}
%Câu 5
\begin{ex}
Cho hình chóp $S \cdot ABCD$ có đáy $ABCD$ là hình vuông tâm $O$, $SA$ vuông góc với mặt phẳng đáy. Mặt phẳng vuông góc với $(SAC)$ là
\choice
{$(SAB)$}
{$(SBD)$}
{$(SBC)$}
{$(SAD)$}
\end{ex}
%Câu 6
\begin{ex}
Cho hình chóp $S \cdot ABC$, cạnh bên $SA$ vuông góc với mặt phẳng đáy, $SA=\dfrac{3a}{2}$, diện tích đáy bằng $\dfrac{a^2\sqrt{3}}{4}$. Thể tích khối chóp $S \cdot ABC$ bằng
\choice
{$\dfrac{a^3\sqrt{3}}{12}$}
{$\dfrac{a^3}{8}$}
{$\dfrac{3a^3\sqrt{3}}{8}$}
{$\dfrac{a^3\sqrt{3}}{8}$}
\end{ex}
%Câu 7
\begin{ex}
Cho hình chóp $S \cdot ABCD$ có đáy $ABCD$ là hình vuông cạnh $a$, cạnh bên $SA$ vuông góc với mặt phẳng đáy và $SA=a\sqrt{2}$. Tính thể tích $V$ của khối chóp $S \cdot ABCD$.
\choice
{$V=\dfrac{a^3\sqrt{2}}{6}$}
{$V=\dfrac{a^3\sqrt{2}}{4}$}
{$V=a^3\sqrt{2}$}
{$V=\dfrac{a^3\sqrt{2}}{3}$}
\end{ex}
%Câu 8
\begin{ex}
Chọn ngẫu nhiên một số tự nhiên từ 1 đến 20. Xét các biến cố $A\colon $“Số được chọn chia hết cho 3”; $B\colon $“Số được chọn chia hết cho 4”. Khi đó biến cố $A\cap B$ là
\choice
{$\left\{ 3;4;12 \right\}$}
{$\left\{ 3;4;6;8;9;12;15;16;18;20 \right\}$}
{$\{12\}$}
{$\left\{ 3;6;9;12;15;18 \right\}$}
\end{ex}
%Câu 9
\begin{ex}
Cho hai biến cố $A$ và $B$. Nếu việc xảy ra hay không xảy ra của biến cố này không ảnh hưởng đến xác suất xảy ra của biến cố kia thì hai biến cố $A$ và $B$ được gọi là
\choice
{Xung khắc với nhau}
{Biến cố đối của nhau}
{Độc lập với nhau}
{Không giao với nhau}
\end{ex}
%Câu 10
\begin{ex}
Cho hàm số $y=f(x)$ có đạo hàm tại $x_0$ là $f'\left(x_0\right)$. Khẳng định nào sau đây đúng?
\choice
{$f'\left(x_0\right)= \lim \limits_{x\to x_0} \dfrac{f(x)-f\left(x_0\right)}{x-x_0}$}
{$f'\left(x_0\right)= \lim \limits_{x\to x_0} \dfrac{f\left(x_0\right)+f(x)}{x-x_0}$}
{$f'\left(x_0\right)= \lim \limits_{h \to 0} \,\dfrac{f\left(x_0h\right)-f\left(x_0\right)}{h}$}
{$f'\left(x_0\right)= \lim \limits_{h\to 0} \,\dfrac{f\left(h-x_0\right)-f\left(x_0\right)}{h}$}
\end{ex}
%Câu 11
\begin{ex}
Số gia của hàm số $f(x)=x^2+1$ ứng với $x_0=3$ và $\triangle x=2$ bằng
\choice
{$18$}
{$35$}
{$-5$}
{$16$}
\end{ex}
%Câu 12
\begin{ex}
Hàm số $y=\sqrt{x}$ có đạo hàm là
\choice
{$y'=\dfrac{1}{\sqrt{x}}$}
{$y'=\dfrac{1}{2\sqrt{x}}$}
{$y'=2\sqrt{x}$}
{$y'=\dfrac{-1}{\sqrt{x}}$}
\end{ex}
%Câu 13
\begin{ex}
Đạo hàm của hàm số $y=x^3-2x^2+4x-5$ là
\choice
{$y'=3x^2-4x+4$}
{$y'=x^2-4x+4$}
{$y'=3x^2-2x+4$}
{$y'=3x^2-4x-1$}
\end{ex}
%Câu 14
\begin{ex}
Đạo hàm của hàm số $y=\ln \left(x^2+2024\right)$ là
\choice
{$y'=\dfrac{1}{x^2+2024}$}
{$y'=2x \cdot \left(x^2+2024\right)$}
{$y'=\dfrac{2x}{x^2+2024}$}
{$y'=\dfrac{x}{x^2+2024}$}
\end{ex}
%Câu 15
\begin{ex}
Đạo hàm cấp hai của hàm số $y=\cos 3x$ là
\choice
{${y'}'=9\cos 3x$}
{${y'}'=-3\cos 3x$}
{${y'}'=-3\sin 3x$}
{${y'}'=-9\cos 3x$}
\end{ex}
%Câu 16
\begin{ex}
Cho $\log _ax=2$, $log _bx=8$ với $a,\ b$ là các số thực lớn hơn $1$. Giá trị của
$P=\log_{\tfrac{a}{b^2}}x$ là
\choice
{$P=-6$}
{$P=4$}
{$P=\dfrac{1}{4}$}
{$P=-\dfrac{1}{4}$}
\end{ex}
%Câu 17
\begin{ex}
Tập nghiệm $S$ của bất phương trình ${5^{x+2}}<{{\left(\dfrac{1}{125}\right)}^{-x}}$ là
\choice
{$S=\left(-\infty ;1\right)$}
{$S=\left(-\infty ;2\right)$}
{$S=\left(2;+\infty\right)$}
{$S=\left(1;+\infty\right)$}
\end{ex}
%Câu 18
\begin{ex}
Tập nghiệm của bất phương trình ${{\log }_{\tfrac{1}{6}}}(x-2)>-1$ là
\choice
{$\left(\dfrac{13}{6};+\infty\right)$}
{$\left(2;\dfrac{13}{6}\right)$}
{$\left(-\infty ;2\right)$}
{$(2;8)$}
\end{ex}
%Câu 19
\begin{ex}
Cho hình chóp $S \cdot ABCD$ có tất cả các cạnh đều bằng nhau. Gọi $I$ và $J$ lần lượt là trung điểm của $SC$ và $BC$. Số đo của góc giữa $IJ$ và $CD$ bằng
\choice
{$90^\circ $}
{$45^\circ $}
{$60^\circ $}
{$30^\circ $}
\end{ex}
%Câu 20
\begin{ex}
Cho hình chóp $\text{S} \cdot ABCD$ có đáy $ABCD$ là hình chữ nhật, $SA\bot (ABCD)$. Gọi $H$, $K$ lần lượt là hình chiếu của $A$ lên $SC$, $SD$. Khẳng định nào sau đây đúng?
\choice
{$BC\bot (SAC)$}
{$BD\bot (SAC)$}
{$AH\bot (SCD)$}
{$AK\bot (SCD)$}
\end{ex}
%Câu 21
\begin{ex}
Cho hình chóp $S \cdot ABC$ có đáy $ABC$ làSy3| tam giác vuông tại $B$, cạnh bên $SA$ vuông góc với đáy. Khẳng định nào sau đây đúng?
\choice
{$AC\bot (SBC)$}
{$BC\bot (SAC)$}
{$BC\bot (SAB)$}
{$AB\bot (SBC)$}
\end{ex}
%Câu 22
\begin{ex}
Cho hình hộp chữ nhật $ABCD.A’B’C’D’$ có $AB=AD=2$, $AA’=2\sqrt{2}$. Góc giữa đường thẳng $A’C$ và mặt phẳng $(ABCD)$ bằng:
\choice
{$30^\circ$}
{$45^\circ$}
{$60^\circ$}
{$90^\circ$}
\end{ex}
%Câu 23
\begin{ex}
Cho hình chóp $S \cdot ABC$ có đáy $ABC$ tam giác vuông tại $A$, cạnh bên $SA$ vuông góc với đáy. Khẳng định nào sau đây đúng?
\choice
{$(SBC)\bot (SAB)$}
{$(SAC)\bot (SAB)$}
{$(SAC)\bot (SBC)$}
{$(ABC)\bot (SBC)$}
\end{ex}
%Câu 24
\begin{ex}
Cho tứ diện $OABC$, trong đó $OA$, $OB$, $OC$ đôi một vuông góc với nhau và $OA=OB=OC=a$. Khoảng cách giữa $OA$ và $BC$ bằng bao nhiêu?
\choice
{$\dfrac{a}{\sqrt{2}}$}
{$\dfrac{a\sqrt{3}}{2}$}
{$a$}
{$\dfrac{a}{2}$}
\end{ex}
%Câu 25
\begin{ex}
Cho hình chóp $S \cdot ABCD$ có $SA\bot (ABCD)$, tứ giác $ABCD$ là hình vuông cạnh $a$, góc giữa $SB$ và $(ABCD)$ là $45^\circ $. Tính thể tích khối chóp $S \cdot ABCD$.
\choice
{$V=\dfrac{a^3}{3}$}
{$V=\dfrac{a^3}{6}$}
{$V=a^3$}
{$V=3a^3$}
\end{ex}
%Câu 26
\begin{ex}
Cho hai biến cố $A$ và $B$ có $P(A)=\dfrac{1}{3},P(B)=\dfrac{1}{4},P(A\cup B)=\dfrac{1}{2}$. Khẳng định nào sau đây đúng?
\choice
{$P\left(A\cap B\right)=\dfrac{1}{12}$}
{$P\left(A\cap B\right)=\dfrac{1}{6}$}
{$A$ và $B$ xung khắc với nhau}
{$P\left(A\cap B\right)=P\left(A\cup B\right)$}
\end{ex}
%Câu 27
\begin{ex}
Một xưởng sản xuất có hai máy chạy độc lập với nhau . Xác suất để máy I và máy II chạy tốt lần lượt là $0{,}7$ và $0{,}6$. Tính xác suất của biến cố $C$ : "Cả hai máy của xưởng sản xuất đều chạy không tốt”.
\choice
{$P(C)=0{,}42$}
{$P(C)=0{,}12$}
{$P(C)=0{,}3$}
{$P(C)=0{,}28$}
\end{ex}
%Câu 28
\begin{ex}
Thực hiện hai thí nghiệm. Gọi $T_1$ và $T_2$ lần lượt là các biến cố “Thí nghiệm thứ nhất thành công” và “Thí nghiệm thứ hai thành công”. Hãy biểu diễn biến cố $A$: “Có đúng một trong hai thí nghiệm thành công” theo hai biến cố $T_1$ và $T_2$.
\choice
{$A=T_1\cup T_2$}
{$A=T_1T_2$}
{$A=T_1T_2\cup \overline{T_1}T_2\cup T_1\overline{T_2}$}
{$A=T_1\overline{T_2}\cup \overline{T_1}T_2$}
\end{ex}
%Câu 29
\begin{ex}
Cho $A$ và $B$ là hai biến cố độc lập, biết $P(A)=0{,}3$ và $P(B)=0{,}4$. Tính xác suất của biến cố $A\cup B$.
\choice
{$P\left(A\cup B\right)=0{,}7$}
{$P\left(A\cup B\right)=0{,}58$}
{$P\left(A\cup B\right)=0{,}3$}
{$P\left(A\cup B\right)=0{,}12$}
\end{ex}
%Câu 30
\begin{ex}
Hai vận động viên $A$ và $B$ cùng ném bóng vào rổ một cách độc lập với nhau. Xác suất ném bóng trúng vào rổ của hai vận động viên $A$ và $B$ lần lượt là $0{,}2$ và $0{,}5$. Xác suất của biến $C$: “Cả hai cùng không ném bóng trúng vào rổ” bằng
\choice
{$0{,}4$}
{$0{,}1$}
{$0{,}7$}
{$0{,}3$}
\end{ex}
%Câu 31
\begin{ex}
Một hộp có 6 viên bi xanh và 4 viên bi đỏ. Lây ngẫu nhiên đồng thời hai viên bi từ hộp. Gọi ${A}$ là biến cố: "Hai viên bi được chọn có cùng màu".Xác suất của biến cố A bằng
\choice
{$\dfrac{8}{15}$}
{$\dfrac{7}{15}$}
{$\dfrac{3}{15}$}
{$\dfrac{14}{15}$}
\end{ex}
%Câu 32
\begin{ex}
Hai xạ thủ ${A}$ và ${B}$ thi bắn súng một cách độc lập với nhau. Xác suất để xạ thủ ${A}$ và xạ thủ ${B}$ bắn trúng bia tương ứng là 0{,}8 và 0{,}9. Xác suất để có ít nhất một xạ thủ bắn trúng bia là
\choice
{$0{,}98$}
{$0{,}72$}
{$0{,}28$}
{$0{,}26$}
\end{ex}
%Câu 33
\begin{ex}
Cho hàm số $f(x)=\sqrt{1+3g(x)}$ với $g(0)=1,g'(0)=-4$. Đạo hàm $f'(0)$ bằng
\choice
{$-3$}
{$3$}
{$-6$}
{$6$}
\end{ex}
%Câu 34
\begin{ex}
Biết đạo hàm của hàm số $y=\dfrac{1}{3}x^3-\dfrac{1}{x}-2^x$ có dạng $y'=ax^2+\dfrac{b}{x^2}-2^x \cdot \ln c$, với $a,b,c\in \mathbb{Z}$. Khi đó $a+b+c$ bằng
\choice
{$2$}
{$-2$}
{$4$}
{$0$}
\end{ex}
%Câu 35
\begin{ex}
Cho hàm số $f(x)=\dfrac{a}{3}x^3+bx+5$ ($a,b$ là hằng số). Tìm $a,b$ biết $f'(1)=1\,,f''(1)=2$
\choice
{$a=1,b=0$}
{$a=2,b=1$}
{$a=1,b=1$}
{$a=1,b=2$}
\end{ex}

\noindent\textbf{II. PHẦN TỰ LUẬN}

%Câu 36
\begin{ex}
Viết phương trình tiếp tuyến đồ thị hàm số $y=x^3-4x+1$ tại giao điểm với trục hoành
\end{ex}
%Câu 37
\begin{ex}
Tìm tập nghiệm của bất phương trình sau: $\left(7+4\sqrt{3}\right)^x\ge \left(2-\sqrt{3}\right)^{3-x^2}$
\end{ex}
%Câu 38
\begin{ex}
Cho hình chóp đều $S.ABC$ có đáy là tam giác đều cạnh $a$. Gọi $M$, $N$ lần lượt là trung điểm của $SB$, $SC$. Biết mặt phẳng $(AMN)$ vuông góc với mặt phẳng $(SBC)$, hãy tính thể tích khối chóp $S.ABC$
\end{ex}
%Câu 39
\begin{ex}
Hộp $I$ chứa $4$ viên bi trắng, $5$ viên bi đỏ và $6$ viên bi xanh. Hộp $II$ có $7$ viên bi trắng, $6$ viên bi đỏ và $5$ viên bi xanh. Lấy ngẫu nhiên từ mỗi hộp $1$ viên bi. Tính xác suất để hai viên bi được lấy ra có cùng màu
\end{ex}

\Closesolutionfile{ans}