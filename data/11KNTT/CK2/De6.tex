\section*{ÔN TẬP KIỂM TRA CUỐI KÌ 2 - ĐỀ 06}
\setcounter{ex}{0}\setcounter{bt}{0}
\Opensolutionfile{ans}[ans/ansBTTeX6]

\noindent\textbf{I. PHẦN TRẮC NGHIỆM:}
%Câu 1
\begin{ex}
Với $x$ là số thực dương, giá trị của biểu thức $P={x^{4-2\sqrt{5}}} \cdot {x^{2\sqrt{5}-3}}$ là
\choice
{$P={x^{-32+14\sqrt{5}}}$}
{$P={x^{7-4\sqrt{5}}}$}
{\True $P=x$}
{$P=x^7$}
\end{ex}
%Câu 2
\begin{ex}
Cho các số thực dương $a,b$ thỏa mãn $\ln a=x;\ln b=y$. Tính $\ln \left(a^2b^5\right)$.
\choice
{$P=x^2y^5$}
{$P=10xy$}
{\True $P=2x+5y$}
{$P=x^2+y^5$}
\end{ex}
%Câu 3
\begin{ex}
\immini{Hàm số nào có đồ thị như hình bên dưới đây?
\choice
{$y=\log _2x$}
{$y=\log_{\tfrac{1}{2}}x$}
{\True $y=2^x$}
{$y={{\left(\dfrac{1}{2}\right)}^x}$}
}{\begin{tikzpicture}[line join = round, line cap = round, >=stealth, scale = .9]
%Hệ trục Oxy và hàm số cần vẽ
\def\xmin{-3}     \def\xmax{3}
\def\ymin{-.5}       \def\ymax{4}
% \def\f(#1){(1/3)^(#1)}
% \def\g(#1){(2/3)^(#1)}
\def\h(#1){2^(#1)}
%Vẽ hệ trục
\draw[->] (\xmin,0)--(0,0) node[below right]{$O$}--(\xmax,0) node[below]{$x$};
\draw[->] (0,\ymin)--(0,\ymax) node[right]{$y$};
\draw (0,1)node[right]{$1$};% (-2.5,2)node{$y=a^x$} (-1.7,3)node{$y=b^x$} (2,2.5)node{$y=c^x$};
%Vẽ hàm số
\begin{scope}
\clip (\xmin,\ymin) rectangle (\xmax,\ymax);
% \draw[smooth, thick, black] plot[domain = \xmin:\xmax, samples = 200, variable=\x]({\x},{\f(\x)});
% \draw[smooth, thick, black] plot[domain = \xmin:\xmax, samples = 200, variable=\x]({\x},{\g(\x)});
\draw[smooth, thick, black] plot[domain = \xmin:\xmax, samples = 200, variable=\x]({\x},{\h(\x)});
\end{scope}
\end{tikzpicture}
}
\end{ex}
%Câu 4
\begin{ex}
Phương trình ${2^{x+3}}=1$ có nghiệm thuộc khoảng nào dưới đây
\choice
{$\left(-\infty ;-4\right)$}
{$(-1;0)$}
{\True $(-4;0)$}
{$\left(1;+\infty\right)$}
\end{ex}
%Câu 5
\begin{ex}
Tìm tập nghiệm $S$ của bất phương trình$\log _5(2x+1)<2$.
\choice
{$S=\left(-\infty ;-\dfrac{1}{2}\right)$}
{$S=\left[-\dfrac{1}{2};12\right)$}
{$S=\left(12;+\infty\right)$}
{\True $S=\left(-\dfrac{1}{2};12\right)$}
\end{ex}
%Câu 6
\begin{ex}
Bác Bảo gửi một số tiền tiết kiệm ban đầu là 100 triệu đồng với lãi suất 0{,}58%/tháng (không kỳ hạn). Hỏi bạn Bảo phải gửi bao nhiêu tháng thì được cả vốn lẫn lãi bằng hoặc vượt quá 120 triệu đồng?
\choice
{30 tháng}
{31 tháng}
{\True 32 tháng}
{33 tháng}
\end{ex}
%Câu 7
\begin{ex}
Phát biểu nào trong các phát biểu sau là đúng?
\choice
{Nếu hàm số $y=f(x)$ có đạo hàm trái tại $x_0$ thì nó liên tục tại điểm đó}
{Nếu hàm số $y=f(x)$ có đạo hàm phải tại $x_0$ thì nó liên tục tại điểm đó}
{Nếu hàm số $y=f(x)$ có đạo hàm tại $x_0$ thì nó liên tục tại điểm $-x_0$}
{\True Nếu hàm số $y=f(x)$ có đạo hàm tại $x_0$ thì nó liên tục tại điểm đó}
\end{ex}
%Câu 8
\begin{ex}
Cho hàm số $y=\dfrac{4}{x-1}$. Khi đó $y'(-1)$ bằng:
\choice
{\True $-1$}
{$-2$}
{$2$}
{$1$}
\end{ex}
%Câu 9
\begin{ex}
Tính đạo hàm của hàm số $y=x^3+2x+1$.
\choice
{$y'=3x^2+2x$}
{\True $y'=3x^2+2$}
{$y'=3x^2+2x+1$}
{$y'=x^2+2$}
\end{ex}
%Câu 10
\begin{ex}
Cho hàm số $f(x)={{(3x-7)}^5}$. Tính $f''(1)$, .
\choice
{$f''(1) = 0$}
{$f''(1) = 20$}
{\True $f''(1) =-180$}
{$f''(1) =30$}
\end{ex}
%Câu 11
\begin{ex}
Tính đạo hàm của hàm số: $y={x^{2024}}$?
\choice
{${y'}=\dfrac{{x^{2023}}}{2024}$}
{\True ${y'}=2024{x^{2023}}$}
{${y'}=\dfrac{2024}{{x^{2023}}}$}
{${y'}={x^{2023}}$}
\end{ex}
%Câu 12
\begin{ex}
Tính đạo hàm của hàm số: $y=2x^3-3x^2+1$.
\choice
{${y'}=6x^2-6x+1$}
{${y'}=3x^2-2x$}
{${y'}=x^2-x$}
{\True ${y'}=6x^2-6x$}
\end{ex}
%Câu 13
\begin{ex}
Tính đạo hàm cấp 2 của hàm số: $y=2x-\sin x$.
\choice
{\True $y''=\sin x$}
{$y''=\cos x$}
{$y''=2-\sin x$}
{$y''=2+\sin x$}
\end{ex}
%Câu 14
\begin{ex}
Cho hàm số $y=x^3-3x+2024\pi $. Bất phương trình $y'<0$ có tập nghiệm là
\choice
{\True $S=(-1;1)$}
{$S=\left(-\infty ;-1\right)\cup \left(1;+\infty\right)$}
{$\left(1;+\infty\right)$}
{$\left(-\infty ;-1\right)$}
\end{ex}
%Câu 15
\begin{ex}
Cho hàm số $y=\dfrac{-2x^2+x-7}{x^2+3}$. Tập nghiệm của phương trình $y'=0$ là
\choice
{\True $\left\{ -1;3 \right\}$}
{$\left\{ 1;3 \right\}$}
{$\left\{ -3;1 \right\}$}
{$\left\{ -3;-1 \right\}$}
\end{ex}
%Câu 16
\begin{ex}
Cho phép thử chọn ngẫu nhiên 3 học sinh từ 4 học sinh nam và 5 học sinh nữ của lớp 11A. Xét 2 biến cố sau:
A: “Trong ba học sinh được chọn có ít nhất một học sinh nam”
B: “Trong ba học sinh được chọn có ít nhất hai học sinh nữ”
Biến cố giao của hai biến cố trên có bao nhiêu phần tử
\choice
{$45$}
{\True $30$}
{$90$}
{$60$}
\end{ex}
%Câu 17
\begin{ex}
Gieo đồng thời một con súc sắc và một đồng xu cân đối đồng chất. Xác suất đồng xu xuất hiện mặt ngửa và con súc sắc xuất hiện mặt $6$ chấm là
\choice
{$\dfrac{1}{2}$}
{$\dfrac{1}{6}$}
{$\dfrac{2}{3}$}
{\True $\dfrac{1}{12}$}
\end{ex}
%Câu 18
\begin{ex}
Ba xạ thủ $A$, $B,C$ cùng nổ súng bắn vào một mục tiêu. Xác suất bắn trúng mục tiêu của ba xạ thủ lần lượt là $0{,}5 ;0{,}6$ và $0{,}8$. Tính xác suất để có ít nhất một người bắn trúng mục tiêu.
\choice
{\True $0{,}96$}
{$0{,}9$}
{$0{,}4$}
{$0{,}84$}
\end{ex}
%Câu 19
\begin{ex}
Hai xạ thủ cùng bắn, mỗi người một viên đạn vào bia một cách độc lập với nhau.
Xác suất bắn trúng bia của hai xạ thủ lần lượt là $\dfrac{1}{3}$ và $\dfrac{1}{6}$. Tính xác suất của biến cố cả hai
xạ thủ đều không bắn trúng bia.
\choice
{\True $\dfrac{5}{9}$}
{$\dfrac{1}{18}$}
{$\dfrac{4}{9}$}
{$\dfrac{5}{18}$}
\end{ex}
%Câu 20
\begin{ex}
Cho hai biến cố $A$ và $B$ xung khắc, biết $P(A)=\dfrac{1}{3},P\left(A\cup B\right)=\dfrac{5}{6}$. Xác suất của biến cố $B$ bằng
\choice
{$\dfrac{1}{6}$}
{$\dfrac{1}{4}$}
{$\dfrac{1}{12}$}
{\True $\dfrac{1}{2}$}
\end{ex}
%Câu 21
\begin{ex}
Cho ${A}$ và ${B}$ là hai biến cố độc lập. Biết $P(A)=0{,}3$ và $P(B)=0{,}18$. Tính xác suất của biến cố $P\left(A\overline{{B}}\right)$.
\choice
{${0{,}13}$}
{${0{,}57}$}
{${0{,}05}$}
{\True ${0{,}25}$}
\end{ex}
%Câu 22
\begin{ex}
Lớp 11A8 trường THPT X có 25 học sinh nam và 20 học sinh nữ. Chọn ngẫu nhiên đồng thời hai bạn từ lớp này để tham dự cuộc họp của trường. Tính xác suất chọn được hai bạn có cùng giới tính để đi dự cuộc họp.
\choice
{$\dfrac{19}{99}$}
{$\dfrac{10}{33}$}
{\True $\dfrac{49}{99}$}
{$\dfrac{29}{99}$}
\end{ex}
%Câu 23
\begin{ex}
Hai xạ thủ bắn mỗi người một viên đạn vào bia, biết xác suất bắn trúng vòng 10 của xạ thủ thứ nhất là $0{,}75$ và của xạ thủ thứ hai là $0{,}85$. Tính xác suất để có ít nhất một xạ thủ bắn trúng vòng 10.
\choice
{$0{,}325$}
{$0{,}6375$}
{$0{,}0375$}
{\True $0{,}9625$}
\end{ex}
%Câu 24
\begin{ex}
Cho hình lập phương $ABCD.A'B'C'D'$. Góc giữa hai đường thẳng $BA'$ và $AB$ bằng:
\choice
{\True $45^\circ $}
{$60^\circ $}
{$30^\circ $}
{$90^\circ $}
\end{ex}
%Câu 25
\begin{ex}
Khẳng định nào sau đây sai?
\choice
{Nếu đường thẳng $d\perp \left(\alpha\right)$ thì $d$ vuông góc với hai đường thẳng trong $\left(\alpha\right)$}
{\True Nếu đường thẳng $d$ vuông góc với hai đường thẳng nằm trong $\left(\alpha\right)$ thì $d\perp \left(\alpha\right)$}
{Nếu đường thẳng $d$ vuông góc với hai đường thẳng cắt nhau nằm trong $\left(\alpha\right)$ thì $d$ vuông góc với bất kì đường thẳng nào nằm trong $\left(\alpha\right)$}
{Nếu $d\perp \left(\alpha\right)$ và đường thẳng $a  \parallel \left(\alpha\right)$ thì $d\perp a$}
\end{ex}
%Câu 26
\begin{ex}
Cho hình chóp $S.ABCD$ có $SA\perp (ABCD)$ và đáy $ABCD$ là hình vuông. Mệnh đề nào dưới đây đúng?
\choice
{\True $(SAC)\perp (SBD)$}
{$(SAD)\perp (SBC)$}
{$AC\perp (SAB)$}
{$BD\perp (SAD)$}
\end{ex}
%Câu 27
\begin{ex}
Cho hình chóp $S.ABCD$ có đáy $ABCD$ là hình chữ nhật, $AB=a\sqrt{3}$, $BC=a\sqrt{2}$. Cạnh bên $SA=a$ và $SA$ vuông góc với mặt phẳng đáy. Khoảng cách giữa $SA$ và $DC$ bằng
\choice
{\True $a\sqrt{2}$}
{$\dfrac{2a}{3}$}
{$a\sqrt{3}$}
{$\dfrac{a\sqrt{3}}{2}$}
\end{ex}
%Câu 28
\begin{ex}
Cho hình chóp $S.ABCD$ có đáy $ABCD$ là hình vuông, $SA\perp (ABCD)$. Hình chiếu của $SC$ lên mặt phẳng $(SAD)$ là
\choice
{$CD$}
{\True $SD$}
{$AC$}
{$SA$}
\end{ex}
%Câu 29
\begin{ex}
Cho hình chóp $S.ABC$ có đáy $ABC$ là tam giác vuông cân tại $B$ và $SA\perp (ABC)$. Gọi $M$ là trung điểm của $AC$. Mặt phẳng vuông góc với $(SBM)$ là:
\choice
{$(SAB)$}
{$(ABC)$}
{$(SBC)$}
{\True $(SAC)$}
\end{ex}
%Câu 30
\begin{ex}
Cho hình chóp $S.ABCD$ có đáy là hình vuông cạnh $2a$, $SA\perp (ABCD)$, $SA=2\sqrt{2}a$. Khi đó khoảng cách từ $A$ đến đường thẳng $SB$ bằng
\choice
{\True $\dfrac{2\sqrt{6}a}{3}$}
{$\dfrac{2\sqrt{3}a}{3}$}
{$\sqrt{3}a$}
{$\dfrac{\sqrt{6}a}{3}$}
\end{ex}
%Câu 31
\begin{ex}
Cho hình lập phương $ABCD.A'B'C'D'$ có cạnh bằng $3$. Khoảng cách từ điểm $A'$ đến mặt phẳng $(ABCD)$ bằng
\choice
{$\sqrt{3}$}
{$3\sqrt{2}$}
{$\dfrac{3}{2}$}
{\True $3$}
\end{ex}
%Câu 32
\begin{ex}
Cho hình chóp $S.ABC$ có $SA$ vuông góc $(ABC)$. Góc giữa $SB$ với $(ABC)$ là góc giữa:
\choice
{\True $SB$ và $AB$}
{$SB$ và $AC$}
{$SB$ và $BC$}
{$SB$ và $SC$}
\end{ex}
%Câu 33
\begin{ex}
Cho hình chóp $S.ABCD$ có đáy là hình vuông, $SA\perp (ABCD)$. Góc giữa $SC$ và mặt phẳng $(ABCD)$ là
\choice
{$\widehat{SCB}$}
{$\widehat{CAS}$}
{\True $\widehat{SCA}$}
{$\widehat{ASC}$}
\end{ex}
%Câu 34
\begin{ex}
Cho hình chóp tam giác đều $S.ABC$ có cạnh đáy bằng $a$, cạnh bên bằng $\dfrac{2\sqrt{3}a}{3}$. Tính góc giữa cạnh bên và mặt đáy.
\choice
{\True ${{60}^\circ}$}
{${{30}^\circ}$}
{${{45}^\circ}$}
{${{73}^\circ}$}
\end{ex}
%Câu 35
\begin{ex}
Cho hình chóp tam $S.ABC$ có $SA\perp  (ABC)$, $SA=a\sqrt{3}$, đáy là tam giác đều cạnh $2a$. Tính góc phẳng nhị diện $\left[S,BC,A\right]$
\choice
{${{60}^\circ}$}
{${{30}^\circ}$}
{\True ${{45}^\circ}$}
{${{73}^\circ}$}
\end{ex}


\noindent\textbf{II. PHẦN TỰ LUẬN}
%Câu 36 
\begin{ex}
Giải bất phương trình $\log_{\tfrac{1}{2}}\left(x^2-5x+7\right)>0$
\end{ex}
%Câu 37
\begin{ex}
Viết phương trình tiếp tuyến đồ thị hàm số $y=x^3-x^2+1$ tại giao điểm với trục tung.
\end{ex}
%Câu 38
\begin{ex}
Một hộp đựng 9 viên bi xanh và 5 viên bi đỏ, có cùng kích thước và khối lượng. Bạn An lấy ngẫu nhiên một viên bi từ hộp (lấy xong không trả lại vào hộp). Tiếp đó đến lượt bạn Tú lấy ngẫu nhiên một viên bi từ hộp đó. Tính xác suất để bạn Tú lấy được viên bi màu xanh.
\end{ex}
%Câu 39
\begin{ex}%[Đề Khảo sát chất lượng Chuyên Hưng Yên 2020-2021]%[Nguyễn Tấn Linh, dự án 12-EX-3-2021]%[2H1G3-2]
	Cho hình chóp $S.ABC$ có đáy là tam giác $ABC$ vuông cân tại $A,$ $SB=12$, $SB$ vuông góc với mặt phẳng $\left(ABC\right)$. Gọi $D$, $E$ lần lượt là các điểm thuộc các đoạn $SA$, $SC$ sao cho $SD=2DA$, $ES=EC$. Biết $DE=2\sqrt{3}$, hãy tính thể tích khối chóp $B.ACED$.
	\loigiai{
		\begin{center}
			\begin{tikzpicture}[line join=round,line cap=round,font=\footnotesize,scale=1]
				\coordinate[label=left:$B$] (B) at (0,0);
				\coordinate[label=below left:$A$] (A) at (2.7,-1);
				\coordinate[label=right:$C$] (C) at (4,0);
				\coordinate[label=above left:$S$] (S) at ($(B)+(90:3)$);
				\coordinate[label=right:$E$] (E) at ($(S)!0.5!(C)$);
				\coordinate[label=right:$D$] (D) at ($(S)!2/3!(A)$);
				\draw (B)--(A)--(C)--(S)--cycle (S)--(A) (B)--(D)--(E);
				\draw[dashed] (E)--(B)--(C);
				\draw ($ (B)!5pt!(C)$)--($(B)!2!($($(B)!5pt!(C)$)!.5!($(B)!5pt!(S)$)$)$)--($(B)!5pt!(S)$);
				\foreach \diem in {B,A,C,S,D,E} \fill (\diem) circle(1pt);
			\end{tikzpicture}
		\end{center}
		Đặt $AB=AC=x>0$. Ta có $SA=\sqrt{x^2+144}$; $BC=x\sqrt{2}\Rightarrow SC=\sqrt{2x^2+144}$.\\
		Định lý hàm số cos, ta có
		\allowdisplaybreaks
		\begin{eqnarray*}
			&&\cos \widehat{ASC}=\dfrac{SA^2+SC^2-AC^2}{2SA\cdot SC}=\dfrac{SD^2+SE^2-ED^2}{2SD\cdot SE}\\
			&\Rightarrow& \dfrac{x^2+144+2x^2+144-x^2}{SA\cdot SC}=\dfrac{\dfrac{4}{9}\left(x^2+144\right)+\dfrac{1}{4}\left(2x^2+144\right)-12}{\dfrac{2}{3}SA\cdot \dfrac{1}{2}SC}\\
			&\Rightarrow& 2x^2+288=\dfrac{4}{3}\left(x^2+144\right)+\dfrac{3}{4}\left(2x^2+144\right)-36\Rightarrow x=\dfrac{12\sqrt{5}}{5}.
		\end{eqnarray*}
		Vậy $V_{B.ACED}=\left(1-\dfrac{SD}{SA}\cdot \dfrac{SE}{SC}\right)V_{S.ABC}=\dfrac{2}{3}V_{S.ABC}=\dfrac{2}{3}\cdot \dfrac{1}{3}S_{ABC}\cdot SB=\dfrac{2}{9}\cdot \dfrac{1}{2}\left(\dfrac{12\sqrt{5}}{5}\right)^{2}\cdot 12=\dfrac{192}{5}$.
	}
\end{ex}


\Closesolutionfile{ans}