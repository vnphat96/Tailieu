\section*{ÔN TẬP KIỂM TRA CUỐI KÌ 2 - ĐỀ 07}
\setcounter{ex}{0}\setcounter{bt}{0}
\Opensolutionfile{ans}[ans/ansBTTeX7]

\noindent\textbf{I. PHẦN TRẮC NGHIỆM:}
%Câu 1
\begin{ex}
Với $a$ là số thực dương tuỳ ý, $\log _5a^2$ bằng
\choice
{$2\log _5a$}
{$2+\log _5a$}
{$\dfrac{1}{2}+\log _5a$}
{$\dfrac{1}{2}\log _5a$}
\end{ex}
%Câu 2
\begin{ex}
Hàm số nào sau đây có đồ thị như hình bên
{\centering\color{red} HINH O DAY}
\choice
{$y=\log_{\tfrac{1}{2}}x$}
{$y=2^x$}
{$y=\log _2x$}
{$y=\left(\dfrac{1}{2}\right)^x$}
\end{ex}
%Câu 3
\begin{ex}
Trong các bất phương trình sau bất phương trình nào là bất phương trình mũ cơ bản
\choice
{$\log \left(2x^2-1\right)<\log x$}
{${{\left(\dfrac{1}{3}\right)}^x}\ge {3^{2x}}$}
{$\log \left(2x^2-1\right)<4$}
{$2^x<3$}
\end{ex}
%Câu 4
\begin{ex}
Trong các mệnh đề sau mệnh đề nào sai
\choice
{Trong không gian hai đường thẳng vuông góc có thể cắt nhau hoặc chéo nhau}
{Góc giữa hai đường thẳng nhận giá trị từ ${0^{\circ }}$ đến ${{90}^{\circ }}$}
{Trong không gian hai đường thẳng cùng vuông góc với đường thẳng thứ ba thì song song}
{Hai đường thẳng song song thì góc giữa chúng bằng ${0^{\circ }}$}
\end{ex}
%Câu 5
\begin{ex}
Cho hình chóp $S.ABC$ có $SA$ vuông góc với mặt đáy $(ABC)$. Mệnh đề nào sau đây là đúng?
\choice
{$SA\perp SB$}
{$SA\perp SC$}
{$SA\perp BC$}
{$SB\perp SC$}
\end{ex}
%Câu 6
\begin{ex}
Cho hình chóp $S.ABCD$ có đáy $ABCD$ là hình thoi tâm $O$. Biết $SA=SC$, và $SB=SD$. Khẳng định nào dưới đây sai?
\choice
{$AC\perp SD$}
{$SO\perp (ABCD)$}
{$CD\perp (SBD)$}
{$BD\perp SA$}
\end{ex}
%Câu 7
\begin{ex}
Cho hình chóp $S.ABC$ có cạnh $SA$ vuông góc với đáy. Góc giữa đường thẳng $SB$ và mặt phẳng đáy là góc giữa hai đường thẳng nào dưới đây?
\choice
{$SB$ và $AB$}
{$SB$ và $SC$}
{$SA$ và $SB$}
{$SB$ và $BC$}
\end{ex}
%Câu 8
\begin{ex}
Cho hình chóp $S.ABCD$ có đáy $ABCD$ là hình thoi và $SB$ vuông góc với mặt phẳng $(ABCD)$. Mặt phẳng nào sau đây vuông góc với mặt phẳng $(SBD)$?
\choice
{$(SBC)$}
{$(SAD)$}
{$(SCD)$}
{$(SAC)$}
\end{ex}
%Câu 9
\begin{ex}
Cho hình chóp $S.ABCD$ có $SA\perp (ABCD)$ và đáy $ABCD$ là hình vuông. Mệnh đề nào dưới đây đúng?
\choice
{$AC\perp (SAB)$}
{$(SAD)\perp (SBC)$}
{$(SAC)\perp (SBD)$}
{$BD\perp (SAD)$}
\end{ex}
%Câu 10
\begin{ex}
Cho hình lập phương $ABCD.A'B'C'D'$ cạnh $a$. Khi đó số đo góc giữa hai mặt phẳng $\left(ADC'B'\right)$ và $(ABCD)$ là
\choice
{$45^\circ $}
{$60^\circ $}
{$90^\circ $}
{$30^\circ $}
\end{ex}
%Câu 11
\begin{ex}
Cho hình chóp $S.ABC$ có đáy $ABC$ là tam giác vuông tại $B$, $M$ là trung điểm của $BC$ và $SA$ vuông góc với mặt đáy. Đường vuông góc chung của $SA$ và $BC$ là
\choice
{$SB$}
{$AC$}
{$AM$}
{$AB$}
\end{ex}
%Câu 12
\begin{ex}
Cho hình chóp $S.ABCD$ có đáy là hình vuông tâm $O$ cạnh là $a$, $SA$ vuông góc với đáy $ABCD$ và $SC=2a$. Xác định khoảng cách từ điểm $S$ đến mặt phẳng$(ABCD)$?
\choice
{$\dfrac{a\sqrt{10}}{2}$}
{$a\sqrt{2}$}
{$\dfrac{a\sqrt{3}}{2}$}
{$a\sqrt{5}$}
\end{ex}
%Câu 13
\begin{ex}
Cho hai biến cố xung khắc $A,\ B$ biết $P(A)=\dfrac{1}{3}$, $P(B)=\dfrac{3}{5}$. Tính $P\left(A\cup B\right)$?
\choice
{$\dfrac{14}{15}$}
{$\dfrac{4}{5}$}
{$\dfrac{1}{5}$}
{$\dfrac{13}{15}$}
\end{ex}
%Câu 14
\begin{ex}
Cho hai biến cố độc lập $A,\ B$ biết $P(A)=\dfrac{1}{7}$, $P(B)=\dfrac{2}{5}$. Tính $P\left(AB\right)$?
\choice
{$\dfrac{19}{35}$}
{$\dfrac{3}{35}$}
{$\dfrac{1}{35}$}
{$\dfrac{2}{35}$}
\end{ex}
%Câu 15
\begin{ex}
Gieo một con xúc xắc đồng chất đồng chất. Gọi $A$ là biến cố: “số chấm gieo được là số chẵn” và $B$ là biến cố: “số chấm gieo được là số chia hết cho 3”, khi đó biến cố $A\cap B$ là
\choice
{$\left\{ 2;4;6 \right\}$}
{$\left\{ 3;6 \right\}$}
{$\left\{ 2;4 \right\}$}
{$\{6\}$}
\end{ex}
%Câu 16
\begin{ex}
Chọn ngẫu nhiên một bạn trong 30 học sinh giỏi toán hoặc văn của một lớp. Gọi $A$ là biến cố: “Học sinh được chọn là học sinh giỏi toán” và $B$ là biến cố: “Học sinh được chọn là học sinh giỏi văn”. Biết rằng $n(A)=20$ và $n(B)=16$, tính $n\left(A\cap B\right)$.
\choice
{$36$}
{$4$}
{$10$}
{$6$}
\end{ex}
%Câu 17
\begin{ex}
Chọn ngẫu nhiên một học sinh trong trường THPT Lê Quý Đôn. Gọi biến cố $A$: “Học sinh được chọn bị cận thị” và biến cố $B$: “Học sinh được chọn học giỏi môn Toán”. Xác định biến cố $A\cup B$.
\choice
{Học sinh được chọn vừa bị cận thị vừa học giỏi môn Toán}
{Học sinh được chọn học giỏi môn Toán nhưng không bị cận thị}
{Học sinh được chọn bị cận thị nhưng không học giỏi môn Toán}
{Học sinh được chọn bị cận thị hoặc học giỏi môn Toán}
\end{ex}
%Câu 18
\begin{ex}
Một hộp chứa $48$ chiếc thẻ được đánh số từ $1{,}2,3,\ldots,48$(mỗi thẻ đánh một số). Chọn ngẫu nhiên một chiếc thẻ trong hộp. Gọi biến cố $A$: “ Chọn được thẻ có đánh số chia hết cho $3$” và biến cố $B$: “ Chọn được thẻ có đánh số chia hết cho $4$”. Biến cố $A\cap B$ có bao nhiêu phần tử?
\choice
{$3$}
{$4$}
{$2$}
{$12$}
\end{ex}
%Câu 19
\begin{ex}
Cho hai biến cố $A$ và $B$ là hai biến cố xung khắc. Biết $P(A)=\dfrac{1}{3}$, $P\left(A\cup B\right)=\dfrac{1}{2}$. Tính $P(B)$
\choice
{$\dfrac{5}{6}$}
{$\dfrac{1}{6}$}
{$\dfrac{1}{3}$}
{$\dfrac{2}{3}$}
\end{ex}
%Câu 20
\begin{ex}
Một hộp có 6 bi xanh, 4 bi đỏ, 5 bi vàng. Lấy ngẫu nhiên từ hộp 1 viên bi. Tính xác suất lấy được một viên bi màu đỏ hoặc màu vàng
\choice
{$\dfrac{7}{15}$}
{$\dfrac{1}{3}$}
{$\dfrac{1}{15}$}
{$\dfrac{3}{5}$}
\end{ex}
%Câu 21
\begin{ex}
Cho hai biến cố $A$ và $B$ độc lập với nhau. Biết $P(A)=0{,}9$ và $P(B)=0{,}7$. Hãy tính xác suất của biến cố $A \cup B$.
\choice
{$0{,}97$}
{$0{,}93$}
{$0{,}63$}
{$0{,}8$}
\end{ex}
%Câu 22
\begin{ex}
Cho hai biến cố $A$ và $B$ độc lập với nhau. Biết $P(B)=0{,}5$ và $P(A \cup B)=0{,}7$. Tính xác suất của biến cố $A$.
\choice
{$0{,}4$}
{$0{,}6$}
{$0{,}75$}
{$0{,}35$}
\end{ex}
%Câu 23
\begin{ex}
Chọn ngẫu nhiên một vé xổ số có 5 chữ số được lập từ các chữ số từ 0 đến 9. Tính xác suất của biến cố $X$: \lq\lq  lấy được vé không có chữ số 2 hoặc không có chữ số $7$ \rq\rq
\choice
{$0{,}8533$}
{$0{,}85314$}
{$0{,}8545$}
{$0{,}853124$}
\end{ex}
%Câu 24
\begin{ex}
Chọn ngẫu nhiên một số tự nhiên có hai chữ số. Tính xác suất để số này hoặc chia hết cho 4 hoặc chia hết cho 6?
\choice
{$\dfrac{29}{90}$}
{$\dfrac{37}{90}$}
{$\dfrac{23}{90}$}
{$\dfrac{41}{90}$}
\end{ex}
%Câu 25
\begin{ex}
Có hai hộp chứa các quả cầu. Hộp thứ nhất chứa $3$ quả cầu đỏ và $5$ quả cầu xanh. Hộp thứ hai chứa $7$ quả cầu đỏ và $4$ quả cầu xanh. Lấy từ mỗi hộp lấy ngẫu nhiên một quả cầu. Tính xác suất để hai quả cầu lấy ra cùng màu xanh.
\choice
{$\dfrac{67}{88}$}
{$\dfrac{5}{22}$}
{$\dfrac{17}{22}$}
{$\dfrac{21}{88}$}
\end{ex}
%Câu 26
\begin{ex}
Hai người đi săn cùng bắn vào hai mục tiêu riêng biệt. Xác suất người thứ nhất bắn trúng mục tiêu là $0{,}9$, xác suất người thứ hai bắn trúng mục tiêu là $0{,}8$. Tính xác suất để không mục tiêu nào bị bắn trúng.
\choice
{$0{,}08$}
{$0{,}8$}
{$0{,}2$}
{$0{,}02$}
\end{ex}
%Câu 27
\begin{ex}
Tiếp tuyến của đồ thị hàm số $y=x^3-2x+1$ tại điểm có hoành độ $x_0=-1$ có hệ số góc bằng
\choice
{$1$}
{$-5$}
{$2$}
{$0$}
\end{ex}
%Câu 28
\begin{ex}
Một vật chuyển động theo quy luật $s(t)=-\dfrac{1}{2}t^3+12t^2$, trong đó $t(s)$ là khoảng thời gian tính từ lúc vật bắt đầu chuyển động, $s(m)$ là quãng đường vật chuyển động trong $t$ giây. Tính vận tốc tức thời của vật tại thời điểm $t=10$ $(s)$.
\choice
{$80\,\left(\text{m/s}\right)$}
{$70\,\left(\text{m/s}\right)$}
{$90\,\left(\text{m/s}\right)$}
{$100\,\left(\text{m/s}\right)$}
\end{ex}
%Câu 29
\begin{ex}
Tính đạo hàm của hàm số $f(x)=\sin^2 2x-\cos 3x$.
\choice
{$f'(x)=2\sin 4x-3\sin 3x$}
{$f'(x)=2\sin 4x+3\sin 3x$}
{$f'(x)=\sin 4x+3\sin 3x$}
{$f'(x)=2\sin 2x+3\sin 3x$}
\end{ex}
%Câu 30
\begin{ex}
Tiếp tuyến của đồ thị hàm số $y=2x^3+3x^2$ tại điểm $M$ có tung độ bằng $5$ có phương trình là:
\choice
{$y=-12x-7$}
{$y=12x-7$}
{$y=-12x+17$}
{$y=12x+17$}
\end{ex}
%Câu 31
\begin{ex}
Đạo hàm của hàm số $y=\sin \left(x^2\right)+\ln \left(\sqrt{x}\right)$
\choice
{$y'=\cos \left(x^2\right)+\dfrac{1}{\sqrt{x}}$}
{$y'=2x \cdot \sin \left(x^2\right)+\dfrac{1}{2\sqrt{x}}$}
{$y'=-2x\cos \left(x^2\right)+\dfrac{1}{2x}$}
{$y'=2x \cdot \cos \left(x^2\right)+\dfrac{1}{2x}$}
\end{ex}
%Câu 32
\begin{ex}
Phương trình tiếp tuyến của đồ thị hàm số $(C)\colon \,y=\dfrac{x-1}{x+2}$ tại điểm có hoành độ $x_0=2$ là
\choice
{$y=\dfrac{3}{4}x-\dfrac{5}{4}$}
{$y=\dfrac{3}{16}x-\dfrac{1}{8}$}
{$y=\dfrac{3}{4}x-\dfrac{7}{4}$}
{$y=\dfrac{3}{16}x-\dfrac{5}{8}$}
\end{ex}
%Câu 33
\begin{ex}
Tìm $m\in \mathbb{R}$ để tiếp tuyến có hệ số góc nhỏ nhất của $\left(C_m\right) \colon y=x^3-2x^2+(m-1)x+2m$ vuông góc với đường thẳng $y=-x$?
\choice
{$m=\dfrac{10}{3}$}
{$m=\dfrac{1}{3}$}
{$m=\dfrac{10}{13}$}
{$m=1$}
\end{ex}
%Câu 34
\begin{ex}
Đạo hàm cấp hai của hàm số $y=f(x)=x\sin x-3$ là biểu thức nào trong các biểu thức sau?
\choice
{$f''(x)=\sin x-x\cos x$}
{$f''(x)=-x\sin x$}
{$f''(x)=1+\cos x$}
{$f''(x)=2\cos x-x\sin x$}
\end{ex}
%Câu 35
\begin{ex}
Tính đạo hàm cấp hai của hàm số $y=x^3+\sin 2x-e^x$
\choice
{$y''=6x+4\sin 2x-e^x$}
{$y''=6x-4\sin 2x-e^x$}
{$y''=6x-4\sin 2x+e^x$}
{$y''=6-4\sin 2x-e^x$}
\end{ex}


\noindent\textbf{II. PHẦN TỰ LUẬN}
%Câu 36 
\begin{ex}
Giải bất phương trình $\log_{2} (x-2) \ge 1- \log_2(x-3)$
\end{ex}
%Câu 37
\begin{ex}
Viết phương trình tiếp tuyến đồ thị hàm số $y=2x^2-x^2-1$ tại giao điểm với trục hoành.
\end{ex}
%Câu 38
\begin{ex}%[1D2K5-2]%
	Một hộp chứa $11$ viên bi được đánh số từ $1$ đến $11$. Chọn ngẫu nhiên $6$ viên bi từ hộp. Tính xác suất để tổng các số trên các viên bi là một số lẻ?
	\loigiai{
		Phép thử là chọn $6$ viên bi từ $11$ viên bi nên số phần tử của không gian mẫu là: $n(\Omega)=\mathrm{C}_{11}^6$.\\
		Gọi biến cố $A$:\lq\lq  Tổng các số trên các viên bi là một số lẻ\rq\rq.
		\begin{itemize}
			\item 	TH1: Chọn được 5 viên bi đánh số lẻ và 1 viên bi số chẵn: có $\mathrm{C}_6^5\mathrm{C}_5^1$ cách chọn.
			\item 	TH2: Chọn được 3 viên bi đánh số lẻ và 2 viên bi số chẵn: có $\mathrm{C}_6^3\mathrm{C}_5^2$ cách chọn.
			\item TH3: Chọn được 1 viên bi đánh số lẻ và 5 viên bi số chẵn: có $\mathrm{C}_6^1\mathrm{C}_5^5$ cách chọn.\\
			$n(A)=\mathrm{C}_6^5\mathrm{C}_5^1+\mathrm{C}_6^3\mathrm{C}_5^2+\mathrm{C}_6^1\mathrm{C}_5^5$.
		\end{itemize}
	Vậy	xác suất của biến cố $A$ là $\mathrm {P}(A)=\dfrac{n(A)}{n(\Omega)} =\dfrac{118}{231}$.}
\end{ex}
%Câu 39
\begin{ex}%[Đề thi thử THPT Yên Lạc 2 Vĩnh Phúc]%[Nguyễn Thành Nhân, 12EX-5-2223]%[2H1G3-2]
	Cho hình chóp $S.ABC$ có đáy $ABC$ là tam giác đều cạnh $3$. Các mặt bên $(SAB)$, $(SAC)$, $(SBC)$ lần lượt tạo với đáy các góc $30^\circ$, $45^\circ$, $60^\circ$. Biết hình chiếu vuông góc của $S$ trên mặt phẳng $(ABC)$ nằm bên trong tam giác $ABC$. Thể tích $V$ của khối chóp $S.ABC$ là
	\loigiai{
		\immini{
			Kẻ $SH \perp (ABC)$ tại $H$.\\
			Kẻ $HM \perp AB$; $HN \perp AC$, $HP \perp BC$. Ta có $\widehat{SMH} = 30^\circ$, $\widehat{SNH} = 45^\circ$, $\widehat{SPH} = 60^\circ$.\\
			Đặt $SH = x$, ta có:
			\begin{itemize}
				\item Tam giác $SHM$ vuông tại $H$ có $\widehat{SMH} = 30^\circ$\\ $\Rightarrow HM = \dfrac{SH}{\tan 30^\circ} = x\sqrt{3}$.
				\item Tam giác $SHN$ vuông tại $H$ có $\widehat{SMN} = 45^\circ$\\ $\Rightarrow HN = SH = x$.
				\item Tam giác $SHP$ vuông tại $H$ có $\widehat{SMP} = 60^\circ$\\ $\Rightarrow HP = \dfrac{SH}{\tan 60^\circ} = \dfrac{x\sqrt{3}}{3}$.
			\end{itemize}		
		}{
			\begin{tikzpicture}[font=\footnotesize,line join=round, line cap=round, >=stealth,scale=1]
			\path
			(0,0) coordinate (A)
			(0.8,-1.5) coordinate (B)
			(4,0) coordinate (C)
			(2,3) coordinate (S)
			(2,-0.5) coordinate (H)
			($(A)!0.4!(B)$) coordinate (M)
			($(A)!0.43!(C)$) coordinate (N)
			($(C)!0.45!(B)$) coordinate (P)
			;
			\draw (A)--(S)--(B)--(M)--(A) (C)--(B)--(A)--(M)--(S)--(P) (S)--(C);
			\draw[dashed] (A)--(C) (P)--(H)--(S)--(N)--(H)--(M) ;
			\foreach \x/\g in
			{S/120,A/180,P/300,B/200,M/180,N/145,C/0,H/45}\fill[black](\x) circle (1pt)($(\x)+(\g:3mm)$) node{\x};
			\end{tikzpicture}	
		}
		Mà $S_{ABC} = S_{HAB} + S_{HBC} + S_{HCA} \Leftrightarrow x\sqrt{3} + x + \dfrac{x\sqrt{3}}{3} = \dfrac{3\sqrt{3}}{2} \Leftrightarrow x = \dfrac{9\sqrt{3}}{2\left(3 + 4\sqrt{3}\right)}\cdot$\\
		Vậy thể tích khối chóp $S.ABC$ là
		$$V = \dfrac{1}{3}\cdot\dfrac{9\sqrt{3}}{2\left(3 + 4\sqrt{3}\right)}\cdot\dfrac{9\sqrt{3}}{4} = \dfrac{81}{8\left(3 + 4\sqrt{3}\right)} = \dfrac{27\sqrt{3}}{8\left(4 + \sqrt{3}\right)}\cdot$$
	}
\end{ex}


\Closesolutionfile{ans}