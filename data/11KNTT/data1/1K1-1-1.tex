\subsection{Các dạng toán thường gặp}
\begin{dang}{Chuyển đổi đơn vị độ - rađian}
	Để chuyển đổi đơn vị độ - rađian cần nhớ:
	\begin{itemize}
		\item $1^\circ=\dfrac{\pi}{180}\,\mathrm{rad}\Rightarrow a^\circ=\dfrac{a\cdot\pi}{180}\,\mathrm{rad}$
		\item $1\,\mathrm{rad}=\left(\dfrac{180}{\pi}\right)^\circ\Rightarrow n\,  \mathrm{rad}=\left(\dfrac{n\cdot180}{\pi}\right)^\circ$
	\end{itemize}
\end{dang}
\subsubsection{Ví dụ mẫu}
\begin{vd}%[DCHT Toán 11 - KNTT - Dương Quang]%[1K1Y1-1]
	Đổi $50^\circ$ sang rađian.\dapso{$50^{\circ}=\dfrac{5\pi}{18}\,\mathrm{rad}$.}
	\loigiai{
		Ta có $1^{\circ}=\dfrac{\pi}{180}\,\mathrm{rad}$.\\
		Nên $50^{\circ}=\cdot\dfrac{\pi}{180}=\dfrac{5\pi}{18}\,\mathrm{rad}$.}
\end{vd}
\begin{vd}%[DCHT Toán 11 - KNTT - Dương Quang]%[1K1Y1-1]
	Đổi $\dfrac{3\pi}{4}\,\mathrm{rad}$ sang độ.\dapso{$\dfrac{3\pi}{4}\,\mathrm{rad}=135^\circ$.}
	\loigiai{
		Ta có $1\,\mathrm{rad}=\left(\dfrac{180}{\pi}\right)^\circ$.\\
		Nên $\dfrac{3\pi}{4}\,\mathrm{rad}=\left(\dfrac{3\pi}{4}\cdot\dfrac{180}{\pi}\right)^\circ=135^\circ$.}
\end{vd}
\begin{vd}%[DCHT Toán 11 - KNTT - Dương Quang]%[1K1Y1-1]
	\begin{enumerate}
		\item Đổi từ độ sang rađian các số đo sau: $45^{\circ} ; 150^{\circ}$.
		\item Đổi từ rađian sang độ các số đo sau: $\dfrac{\pi}{3} ; \dfrac{5 \pi}{4}$.
	\end{enumerate}
	\loigiai{
		\begin{enumerate}
			\item Ta có:
			$$
			\begin{aligned}
				& 45^{\circ}=45 \cdot \frac{\pi}{180}=\frac{\pi}{4}, \\
				& 150^{\circ}=150 \cdot \frac{\pi}{180}=\frac{5 \pi}{6} .
			\end{aligned}
			$$
			\item Ta có
			$$
			\begin{aligned}
				& \frac{\pi}{3}=\frac{\pi}{3} \cdot\left(\frac{180}{\pi}\right)^\circ=60^{\circ}, \\
				& \frac{5 \pi}{4}=\frac{5 \pi}{4} \cdot\left(\frac{180}{\pi}\right)^{\circ}=225^{\circ} .
			\end{aligned}
			$$
		\end{enumerate}
	}
\end{vd}
\begin{vd}%[DCHT Toán 11 - KNTT - Dương Quang]%[1K1B1-1]
	Đổi số đo của các góc sau ra rađian: $72^\circ; 600^\circ; -37^\circ45'30''$.
	\loigiai{
		Vì $1^\circ=\dfrac{\pi}{180}\, \mathrm{rad}$ nên\\
		$72^\circ=72\cdot\dfrac{\pi}{180}=\dfrac{2\pi}{5}$;\\
		$600^\circ=600\cdot\dfrac{\pi}{180}=\dfrac{10\pi}{3}$;\\
		$-37^\circ45'30''=-37^\circ-\left(\dfrac{45}{60}\right)^\circ-\left(\dfrac{30}{60 \cdot 60}\right) ^\circ=\left(\dfrac{4531}{120}\right)^\circ =\dfrac{4531}{120}\cdot\dfrac{\pi}{180}\approx 0,6587$.
	}
\end{vd}

\begin{vd}%[DCHT Toán 11 - KNTT - Dương Quang]%[1K1B1-1]
	Đổi số đo của các góc sau ra độ: $\dfrac{5\pi}{18}; \dfrac{3\pi}{5}; -4$.
	\loigiai{
		Vì $1\, \mathrm{rad}=\left(\dfrac{180}{\pi}\right)^\circ$ nên\\
		$\dfrac{5\pi}{18}=\left(\dfrac{5\pi}{18}\cdot\dfrac{180}{\pi}\right)^\circ =50^\circ$;\\
		$\dfrac{3\pi}{5}=\left(\dfrac{3\pi}{5}\cdot \dfrac{180}{\pi}\right)^\circ =108^\circ$;\\
		$-4=-\left(4\cdot\dfrac{180}{\pi}\right)^\circ \approx -2260^\circ 48'$.
	}
\end{vd}
\begin{vd}%[DCHT Toán 11 - KNTT - Dương Quang]%[1K1B1-1]
	Hoàn thành bảng chuyển đổi số đo độ và số đo rađian của một số góc đặc biệt sau
	\begin{center}
		\begin{longtable}{|c|c|c|c|c|c|c|c|}
			\hline
			Độ & $30^\circ$ & ? & $60^\circ$ & ? & $120^\circ$ & ? & $180^\circ$\\
			\hline
			Radian & ? & $\dfrac{\pi}{4}$ & ? & $\dfrac{\pi}{2}$ & ? & $\dfrac{3\pi}{4}$ & ? \\
			\hline
		\end{longtable}
	\end{center}
	\loigiai{
		Ta có bảng chuyển đổi số đo độ và số đo rađian của một số góc đặc biệt.
		\begin{center}
			\begin{longtable}{|c|c|c|c|c|c|c|c|}
				\hline
				Độ & $30^\circ$ & $45^\circ$ & $60^\circ$ & $90^\circ$ & $120^\circ$ & $135^\circ$ & $180^\circ$\\
				\hline
				Radian & $\dfrac{\pi}{6}$ & $\dfrac{\pi}{4}$ & $\dfrac{\pi}{3}$ & $\dfrac{\pi}{2}$ & $\dfrac{2\pi}{3}$ & $\dfrac{3\pi}{4}$ & $\pi$ \\
				\hline
			\end{longtable}
		\end{center}
	}
\end{vd}
\subsubsection{Bài tập rèn luyện}
\centerline{\fcolorbox{red}{yellow!50}{\bf {BÀI TẬP TỰ LUẬN }}}
\begin{bt}%[DCHT Toán 11 - KNTT - Dương Quang]%[1K1Y1-1]
	Đổi $60^\circ$ sang rađian.\dapso{$60^{\circ}=\dfrac{\pi}{3}\,\mathrm{rad}$.}
	\loigiai{
		Ta có $1^{\circ}=\dfrac{\pi}{180}\,\mathrm{rad}$.\\
		Nên $60^{\circ}=60\cdot\dfrac{\pi}{180}=\dfrac{\pi}{3}\,\mathrm{rad}$.}
\end{bt}
\begin{bt}%[DCHT Toán 11 - KNTT - Dương Quang]%[1K1Y1-1]
	Đổi $\dfrac{2\pi}{3}\,\mathrm{rad}$ sang độ.\dapso{$\dfrac{2\pi}{3}\,\mathrm{rad}=120^\circ$.}
	\loigiai{
		Ta có $1\,\mathrm{rad}=\left(\dfrac{180}{\pi}\right)^\circ$.\\
		Nên $\dfrac{2\pi}{3}\,\mathrm{rad}=\left(\dfrac{2\pi}{3}\cdot\dfrac{180}{\pi}\right)^\circ=120^\circ$.}
\end{bt}
\begin{bt}%[DCHT Toán 11 - KNTT - Dương Quang]%[1K1Y1-1]
	Hãy hoàn thành bảng chuyển đổi số đo độ và số đo rađian của một số góc sau.
	\begin{center}
		\begin{longtable}{|c|c|c|c|c|}
			\hline
			Độ & $18^\circ$ & ? & $72^\circ$ & ? \\
			\hline
			Radian & ? & $\dfrac{2\pi}{9}$ & ? & $\dfrac{5\pi}{6}$\\
			\hline
		\end{longtable}
	\end{center}
	\loigiai{
		Ta có bảng chuyển đổi số đo độ và số đo rađian của một số góc.
		\begin{center}
			\begin{longtable}{|c|c|c|c|c|}
				\hline
				Độ & $18^\circ$ & $40^\circ$ & $72^\circ$ & $150^\circ$ \\
				\hline
				Radian & $\dfrac{\pi}{10}$ & $\dfrac{2\pi}{9}$ & $\dfrac{2\pi}{5}$ & $\dfrac{5\pi}{6}$\\
				\hline
			\end{longtable}
		\end{center}
	}
\end{bt}
\begin{bt}%[DCHT Toán 11 - KNTT - Dương Quang]%[1K1B1-1]
	Đổi các số đo góc sau đây từ rađian sang độ hoặc ngược lại
	\begin{listEX}[3]
		\item $-60^{\circ}$.
		\item $\dfrac{2 \pi}{5}\, \mathrm{rad}$.
		\item $3\, \mathrm{rad}$.
	\end{listEX}
	\loigiai{
		\begin{listEX}[1]
			\item $-60^{\circ}=-\dfrac{60 \pi}{180}\, \mathrm{rad}=-\dfrac{\pi}{3}\, \mathrm{rad}$.
			\item $\dfrac{2 \pi}{5}\, \mathrm{rad}=\left(\dfrac{2 \pi}{5}\cdot \dfrac{180}{\pi}\right)^{\circ}=72^{\circ}$.
			\item $3\, \mathrm{rad}=\left(3 \cdot \dfrac{180}{\pi}\right)^{\circ}=\left(\dfrac{540}{\pi}\right)^{\circ} \approx 171{,}89^{\circ}$.
		\end{listEX}	
	}
\end{bt}
\begin{bt}%[DCHT Toán 11 - KNTT - Dương Quang]%[1K1B1-1]
	Đổi số đo của các góc sau ra rađian: $54^\circ; 30^\circ 45'; -60^\circ; -210^\circ$.
	\loigiai{
		$54^\circ=54\cdot\dfrac{\pi}{180}=\dfrac{3\pi}{10}$;\\
		$30^\circ 45'=30^\circ+\left(\dfrac{45}{60}\right)^\circ=\left(\dfrac{123}{4}\right)^\circ=\dfrac{123}{4}\cdot\dfrac{\pi}{180}=\dfrac{41\pi}{240}\approx 0,5367$;\\
		$-60^\circ=-60\cdot\dfrac{\pi}{180}=-\dfrac{\pi}{3}$;\\
		$-210^\circ=-210\cdot\dfrac{\pi}{180}=-\dfrac{7\pi}{6}$.
	}
\end{bt}
\begin{bt}%[DCHT Toán 11 - KNTT - Dương Quang]%[1K1B1-1]
	Đổi số đo của các góc sau ra độ: $\dfrac{\pi}{5}; -\dfrac{5\pi}{6}; \dfrac{4\pi}{3}; 3,56\pi$.
	\loigiai{
		$\dfrac{\pi}{5}=\left(\dfrac{\pi}{5}\cdot\dfrac{180}{\pi}\right)^\circ=36^\circ$;\\
		$-\dfrac{5\pi}{6}=-\left(\dfrac{5\pi}{6}\cdot\dfrac{180}{\pi}\right)^\circ= 150^\circ$;\\
		$\dfrac{4\pi}{3}=\left(\dfrac{4\pi}{3}\cdot\dfrac{180}{\pi}\right)^\circ=240^\circ$;\\
		$3,56\pi=\left(3,56\pi\cdot\dfrac{180}{\pi}\right)^\circ\approx 640^\circ48'$.
	}
\end{bt}
\centerline{\fcolorbox{red}{yellow!50}{\bf {CÂU HỎI TRẮC NGHIỆM}}}
\Opensolutionfile{ans}[ans/ans-1K1-1-Dang1]
\begin{ex}%[DCHT Toán 11 - KNTT - Dương Quang]%[1K1Y1-1]
	Chọn khẳng định đúng.
	\choice
	{\True $1\,\mathrm{rad}=\left(\dfrac{180}{\pi }\right)^\circ$}
	{$1\,\mathrm{rad}=60^\circ$}
	{$1\,\mathrm{rad}=180^\circ$}
	{$1\,\mathrm{rad}=1^\circ$}
	\loigiai{
		Ta có công thức $1\,\mathrm{rad}=\left(\dfrac{180}{\pi }\right)^\circ$.
	}
\end{ex}
\begin{ex}%[DCHT Toán 11 - KNTT - Dương Quang]%[1K1Y1-1]
	Cung tròn có số đo là $\pi$. Hãy chọn số đo độ của cung tròn đó trong các cung tròn sau đây.
	\choice
	{$30^{\circ}$}
	{$45^{\circ}$}
	{$90^{\circ}$}
	{\True $180^{\circ}$}
	\loigiai{
		Ta có $\pi\,\mathrm{rad}=\left(\dfrac{180\cdot\pi}{\pi }\right)^\circ=180^\circ$.
	}
\end{ex}
\begin{ex}%[DCHT Toán 11 - KNTT - Dương Quang]%[1K1Y1-1]
	Góc có số đo $135^{\circ}$ đổi sang rađian là
	\choice
	{$\dfrac{4\pi }{3}$}
	{\True $\dfrac{3\pi }{4}$}
	{$\dfrac{5\pi }{6}$}
	{$\dfrac{3\pi }{5}$}
	\loigiai{
		Ta có $1^{\circ}=\dfrac{\pi}{180}\,\mathrm{rad}$.\\
		Vậy $135^{\circ}=135\cdot\dfrac{\pi}{180}=\dfrac{3\pi}{4}\,\mathrm{rad}$.
	}
\end{ex}
\begin{ex}%[DCHT Toán 11 - KNTT - Dương Quang]%[1K1Y1-1]
	Đổi sang rađian góc có số đó $108^\circ$ ta được
	\choice
	{$\dfrac{\pi}{4}$}
	{$\dfrac{\pi}{10}$}
	{$\dfrac{3\pi}{2}$}
	{\True $\dfrac{3\pi}{5}$}
	\loigiai{
		Ta có $108^\circ=108^\circ \cdot \dfrac{\pi}{180^\circ}$ $=\dfrac{3\pi}{5}$.}
\end{ex}
\begin{ex}%[DCHT Toán 11 - KNTT - Dương Quang]%[1K1B1-1]
	Đổi sang rađian góc có số đó ${960}^{\circ}$ ta được
	\choice
	{$\dfrac{8}{3}\pi $}
	{\True $\dfrac{16}{3}\pi $}
	{$\dfrac{16}{3}$}
	{$\dfrac{3}{16}\pi $}
	\loigiai{
		Số đo góc ${960}^{\circ}$ theo đơn vị rađian là $\dfrac{960}{180}\pi =\dfrac{16}{3}\pi.$}
\end{ex}
\begin{ex}%[DCHT Toán 11 - KNTT - Dương Quang]%[1K1B1-1]
	Đổi sang rađian góc có số đó $250^\circ$ ta được
	\choice
	{\True $\dfrac{25\pi}{12}$}
	{$\dfrac{25\pi}{18}$}
	{$\dfrac{25\pi}{9}$}
	{$\dfrac{35\pi}{18}$}
	\loigiai{
		Ta có $250^\circ=\dfrac{\pi}{180}\cdot 250=\dfrac{25\pi}{18}$.}
\end{ex}
\begin{ex}%[DCHT Toán 11 - KNTT - Dương Quang]%[1K1B1-1]
	Đổi sang rađian góc có số đó $\dfrac{5\pi}{4}$ ta được
	\choice
	{$172^\circ$}
	{$15^\circ$}
	{\True $225^\circ$}
	{$5^\circ$}
	\loigiai{
		Ta có $\dfrac{5\pi}{4}=\left(\dfrac{180}{\pi}\cdot\dfrac{5\pi}{4}\right)^\circ=225^\circ$.}
\end{ex}
\begin{ex}%[DCHT Toán 11 - KNTT - Dương Quang]%[1K1B1-1]
	Góc có số đo $\dfrac{\pi}{12}$ đổi sang độ là
	\choice
	{\True  $15^\circ$}
	{$16^\circ$}
	{$17^\circ 30'$}
	{$14^\circ$}
	\loigiai{
		Ta có $\dfrac{\pi}{12}\,\mathrm{rad}=\left(\dfrac{\pi}{12}\cdot \dfrac{180}{\pi}\right)^\circ = 15^\circ$.
	}
\end{ex}
\begin{ex}%[DCHT Toán 11 - KNTT - Dương Quang]%[1K1B1-1]
	Đổi góc $\alpha=\dfrac{\pi}{9}$ ra đơn vị độ ta được
	\choice
	{\True $\alpha=20^\circ$}
	{$\alpha=10^\circ$}
	{$\alpha=15^\circ$}
	{$\alpha=25^\circ$}
	\loigiai{
		Ta có $\dfrac{\pi}{9}=\dfrac{180^\circ}{9}=20^\circ$.
	}
\end{ex}
\begin{ex}%[DCHT Toán 11 - KNTT - Dương Quang]%[1K1B1-1]
	Nếu một cung tròn có số đo bằng rađian là $\dfrac{17\pi }{6}$ thì số đo bằng độ của cung tròn đó là
	\choice
	{$30^{\circ}$}
	{$390^{\circ}$}
	{\True $510^{\circ}$}
	{$520^{\circ}$}
	\loigiai{
		Ta có $\dfrac{17\pi }{6}\text{(rad)}=\left(\dfrac{17\pi }{6}\cdot \dfrac{180}{\pi }\right)^{\circ}=510^{\circ}$.
	}
\end{ex}
\begin{ex}%[DCHT Toán 11 - KNTT - Dương Quang]%[1K1G1-1]
	Bánh xe của người đi xe đạp quay được $11$ vòng trong $5$ giây. Tìm góc theo rađian mà bánh xe quay được trong $1$ giây.
	\choice
	{\True$\dfrac{22\pi}{5}\,\mathrm{rad}$}
	{$\dfrac{11\pi}{5}\,\mathrm{rad}$}
	{$22\pi\,\mathrm{rad}$}
	{$11\pi\,\mathrm{rad}$}
	\loigiai{
		Trong $1$ giây, bánh xe quay được $\dfrac{11}{5}$ vòng.\\
		Một vòng ứng với số đo là $2\pi\,\mathrm{rad}$.\\
		Vậy trong $1$ giây bánh xe quay được một góc $\dfrac{11}{5}\cdot2\pi=\dfrac{22\pi}{5}.$
	}
\end{ex}
\Closesolutionfile{ans}
\begin{indapan}{10}
	{ans/ans-1K1-1-Dang1}
\end{indapan}