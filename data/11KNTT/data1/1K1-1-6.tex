
%\subsection{Các dạng toán thường gặp}
\begin{dang}{Tính giá trị lượng giác của một góc}	
\begin{itemize}
	\item $\sin^2a + \cos^2 a = 1$.
	\item $\tan a \cdot \cot a =1$.
	\item $\tan^2 a+1 = \dfrac{1}{\cos^2 a}$.
	\item $\cot^2 a+1 = \dfrac{1}{\sin^2 a}$.
\end{itemize}
\end{dang}
\subsubsection{Ví dụ minh hoạ}
\begin{vd}%[DCHT Toán 11 - KNTT - Nguyễn Hoài Nam]%[1K1Y1-6] 
	Cho góc lượng giác có số đo bằng $-\dfrac{\pi}{3}$.
	\begin{enumerate}[a)]
		\item Xác định điểm $M$ trên đường tròn lượng giác biểu diễn góc lượng giác đã cho.
		\item Tính các giá trị lượng giác của góc lượng giác đã cho.
	\end{enumerate}
	\loigiai{
		\immini{\begin{enumerate}[a)]
				\item Điểm $M$ trên đường tròn lượng giác biểu diễn góc lượng giác có số đo là $-\dfrac{\pi}{3}$ được xác định trong hình bên.
				\item Ta có $\cos\left(-\dfrac{\pi}{3}\right)=\dfrac{1}{2}$; $\sin\left(-\dfrac{\pi}{3}\right)=-\dfrac{\sqrt{3}}{2}$;\\
				$\tan\left(-\dfrac{\pi}{3}\right)=\dfrac{\sin\left(-\dfrac{\pi}{3}\right)}{\cos\left(-\dfrac{\pi}{3}\right)}=-\sqrt{3}$;\\ $\cot\left(-\dfrac{\pi}{3}\right)=\dfrac{\cos\left(-\dfrac{\pi}{3}\right)}{\sin\left(-\dfrac{\pi}{3}\right)}=-\dfrac{1}{\sqrt{3}}$.
		\end{enumerate}}
		{\begin{tikzpicture}[line join = round, line cap = round, >=stealth, font=\footnotesize, scale=0.6]
				\tikzset{label style/.style={font=\footnotesize}}
				\path (0,0) coordinate (O)
				(3,0) coordinate (A)
				(0,-3) coordinate (B)
				(0:0)++(-60:3) coordinate (M)
				($(O)!(M)!(A)$) coordinate (H)
				($(O)!(M)!(B)$) coordinate (K)
				;
				\draw[->] (-4,0) -- (4,0) node[above,blue]{$x$};
				\draw[->] (0,-4) -- (0.,4) node[left,blue]{$y$};
				\draw[orange] (O) circle (3cm);
				\draw[rotate=0,->,green!50!black] (0.5,0) arc (0:-60:0.6cm);
				\draw (0.85,-0.45) node[blue] {$-\frac{\pi}{3}$};
				\draw[dashed] (H)--(M)--(K);
				\draw[green!50!black] (M)--(O);
				\draw[blue] (1.5,0) node[above]{$\frac{1}{2}$}(0,-2.57) node[left]{$-\frac{\sqrt{3}}{2}$};
				\foreach \p/\r in {A/-45,M/-60,O/-150}
				\fill (\p) circle (1pt) node[shift={(\r:3mm)},blue]{$\p$};
		\end{tikzpicture}}
	}
\end{vd}
\begin{vd}%[DCHT Toán 11 - KNTT - Nguyễn Hoài Nam]%[1K1B1-6]
	Tính các giá trị lượng giác của góc $\alpha$, biết
		\begin{enumEX}{2}
			\item $\cos\alpha=\dfrac{1}{5}$ và $0<\alpha<\dfrac{\pi}{2}$;
			\item $\sin\alpha=\dfrac{2}{5}$ và $\dfrac{\pi}{2}<\alpha<\pi$;
			\item $\tan\alpha=\sqrt{5}$ và $\pi<\alpha<\dfrac{3\pi}{2}$;
			\item $\cot\alpha=-\dfrac{1}{\sqrt{2}}$ và $\dfrac{3\pi}{2}<\alpha<2\pi$.
		\end{enumEX}
	\loigiai{
		\begin{enumerate}[a)]
			\item Ta có $\sin^2\alpha+\cos^2\alpha=1\Rightarrow \sin^2\alpha=1-\cos^2\alpha\Leftrightarrow \hoac{& \sin\alpha=\sqrt{1-\cos^2\alpha}\\&\sin\alpha=-\sqrt{1-\cos^2\alpha}.}$\\
			Vì $0<\alpha<\dfrac{\pi}{2}$ nên $\sin\alpha>0$, suy ra $\sin\alpha=\sqrt{1-\cos^2\alpha}$.\\
			$\Rightarrow\sin\alpha=\sqrt{1-\left(\dfrac{1}{5}\right)^2}=\sqrt{\dfrac{24}{25}}=\dfrac{2\sqrt{6}}{5}$.\\
			Mà $\tan\alpha=\dfrac{\sin\alpha}{\cos\alpha}$ nên $\tan\alpha=\dfrac{2\sqrt{6}}{5}:\dfrac{1}{5}=2\sqrt{6}$ và $\cot\alpha=\dfrac{\sqrt{6}}{12}$.
			\item Ta có $\sin^2\alpha+\cos^2\alpha=1\Rightarrow \cos^2\alpha=1-\sin^2\alpha\Leftrightarrow \hoac{& \cos\alpha=\sqrt{1-\sin^2\alpha}\\&\cos\alpha=-\sqrt{1-\sin^2\alpha}.}$\\
			Vì $\dfrac{\pi}{2}<\alpha<\pi$ nên $\cos\alpha<0$, suy ra $\cos\alpha=-\sqrt{1-\sin^2\alpha}$.\\
			$\Rightarrow\cos\alpha=-\sqrt{1-\left(\dfrac{2}{5}\right)^2}=-\sqrt{\dfrac{21}{25}}=-\dfrac{\sqrt{21}}{5}$.\\
			Mà $\tan\alpha=\dfrac{\sin\alpha}{\cos\alpha}$ nên $\tan\alpha=\dfrac{2}{5}:\left(-\dfrac{\sqrt{21}}{5}\right)=-\dfrac{2\sqrt{21}}{21}$ và $\cot\alpha=-\dfrac{\sqrt{21}}{2}$.
			\item Vì $\pi<\alpha<\dfrac{3\pi}{2}$ nên $\cos\alpha<0$, do đó từ công thức:\\
			$1+\tan^2\alpha=\dfrac{1}{\cos^2\alpha}$, suy ra $\cos^2\alpha=\dfrac{1}{1+\tan^2\alpha}$.\\
			$\Rightarrow\cos\alpha=-\dfrac{1}{\sqrt{1+\tan^2\alpha}}=-\dfrac{1}{\sqrt{1+(\sqrt{5})^2}}=-\dfrac{\sqrt{6}}{6}$;\\ $\cot\alpha=\dfrac{1}{\tan\alpha}=\dfrac{\sqrt{5}}{5}$ và $\sin\alpha=\tan\alpha\cdot\cos\alpha=-\dfrac{\sqrt{30}}{6}$.
			\item 
			Vì $\dfrac{3\pi}{2}<\alpha<2\pi$ nên $\sin\alpha<0$, do đó từ công thức:\\
			$1+\cot^2\alpha=\dfrac{1}{\sin^2\alpha}$, suy ra $\sin^2\alpha=\dfrac{1}{1+\cot^2\alpha}$.\\
			$\Rightarrow\sin\alpha=-\dfrac{1}{\sqrt{1+\cot^2\alpha}}=-\dfrac{1}{\sqrt{1+\left(-\dfrac{1}{\sqrt{2}}\right)^2}}=-\dfrac{\sqrt{6}}{3}$;\\ $\tan\alpha=\dfrac{1}{\cot\alpha}=-\sqrt{2}$ và $\cos\alpha=\cot\alpha\cdot\sin\alpha=\dfrac{\sqrt{3}}{3}$.
		\end{enumerate}
	}
\end{vd}
% \subsubsection{Bài tập rèn luyện}
\subsubsection{Bài tập tự luận}
\begin{bt}%[DCHT Toán 11 - KNTT - Nguyễn Hoài Nam]%[1K1Y1-6]
	Biết $\sin{\alpha}=\dfrac{1}{3}\ \left(90^{\circ}<\alpha<180^{\circ}\right)$. Hỏi giá trị của $\cos{\alpha}$ là bao nhiêu?
	\loigiai{
		Vì $90^{\circ}<\alpha<180^{\circ}$ nên $\cos{\alpha}<0$.
		Ta có $\sin^2{\alpha}+\cos^2{\alpha}=1\Leftrightarrow \cos^2{\alpha}=1-\dfrac{1}{9}\Leftrightarrow\cos{\alpha}=-\dfrac{2\sqrt{2}}{3}$.}
\end{bt}
\begin{bt}%[DCHT Toán 11 - KNTT - Nguyễn Hoài Nam]%[1K1Y1-6]
	Biết $\tan{\alpha}=2$, tính $\cot{\alpha}$.
	\loigiai{
		Áp dụng hệ thức $\tan{\alpha}\cdot\cot{\alpha}=1\Leftrightarrow \cot{\alpha}=\dfrac{1}{2}$.}
\end{bt}
\begin{bt}%[DCHT Toán 11 - KNTT - Nguyễn Hoài Nam]%[1K1B1-6]
	Cho $\sin \alpha=-\dfrac{1}{3}$ với $\alpha \in\left(\dfrac{3 \pi}{2};2 \pi\right)$. Khi đó, giá trị của $\cos \alpha$ bằng
	\loigiai{
		Vì 	$\alpha \in\left(\dfrac{3 \pi}{2};2 \pi\right)$ nên $ \cos \alpha >0 $.\\
		Ta có $ \cos^2 \alpha=1-\sin^2 \alpha =\dfrac{8}{9}\Rightarrow \cos \alpha =\dfrac{2\sqrt{2}}{3}$.
	}
\end{bt}
\begin{bt}%[DCHT Toán 11 - KNTT - Nguyễn Hoài Nam]%[1K1K1-6]
	Cho $\tan a=2,\left(\pi<a<\dfrac{3\pi}{2}\right).$ Tính $A=\sin a+\cos a.$
	\loigiai{Ta có $1+\tan^2 a=\dfrac{1}{\cos^2 a}\Leftrightarrow \cos^2 a=\dfrac{1}{5}\Rightarrow \cos a=-\dfrac{\sqrt{5}}{5}$ (vì $\pi<a<\dfrac{3\pi}{2}$).\\
		$\sin^2 a=1-\cos^2 a=\dfrac{4}{5}\ \Leftrightarrow \hoac{&\sin a=-\dfrac{2\sqrt{5}}{5}\\&\sin a=\dfrac{2\sqrt{5}}{5}.}$\\
		Vì $\pi<a<\dfrac{3\pi}{2}$ nên $\sin a=-\dfrac{2\sqrt{5}}{5}$.\\
		Do đó $A=-\dfrac{2\sqrt{5}}{5}-\dfrac{\sqrt{5}}{5}=-\dfrac{3\sqrt{5}}{5}.$
	}
\end{bt}
\begin{bt}%[DCHT Toán 11 - KNTT - Nguyễn Hoài Nam]%[1K1G1-6]
	Cho $\cot \alpha =2$. Tính giá giá của biểu thức $A=\dfrac{\cos^2 \alpha+\tan^2 \alpha-1}{\sin^2 \alpha }$.
	\loigiai{
		\begin{eqnarray*}
			A&=&\dfrac{\cos^2 \alpha+\tan^2 \alpha-1}{\sin^2 \alpha }\\
			&=& \dfrac{-\sin^2 \alpha +\tan^2 \alpha}{\sin^2 \alpha}\\
			&=& -\dfrac{\sin^2 \alpha}{\sin^2\alpha}+\dfrac{\tan^2 \alpha}{\sin^2 \alpha}\\
			&=& -1+\dfrac{1}{\cos^2 \alpha}\\
			&=& -1+1+\tan^2 \alpha\\
			&=& \dfrac{1}{\cot^2 \alpha}=\dfrac{1}{4}.
		\end{eqnarray*}
	}
\end{bt}
\subsubsection{Bài tập trắc nghiệm}
\Opensolutionfile{ans}[ans/ans-1K1-1-Dang6]
\begin{ex}%[DCHT Toán 11 - KNTT - Nguyễn Hoài Nam]%[1K1Y1-6]
	Cho $\alpha$ là góc tù. Khẳng định nào sau đây là đúng?
	\choice
	{$\sin{\alpha}<0$}
	{$\cos{\alpha}>0$}
	{\True $\tan{\alpha}<0$}
	{$\cot{\alpha}>0$}
	\loigiai{
		Vì $\alpha$ là góc tù, nên $\sin{\alpha}>0,\ \cos{\alpha}<0$ nên $\tan{\alpha}<0$.}
\end{ex}
\begin{ex}%[DCHT Toán 11 - KNTT - Nguyễn Hoài Nam]%[1K1Y1-6]
	Giá trị của $\sin{(-240^{\circ})}$ là
	\choice
	{\True$\dfrac{\sqrt{3}}{2}$}
	{$\dfrac{1}{2}$}
	{$\dfrac{-\sqrt{3}}{2}$}
	{$-\dfrac{1}{2}$}
	\loigiai{
		Ta có $\sin{(-240^{\circ})}=\sin{\left(120^{\circ}-360^{\circ}\right)}=\sin{120^{\circ}}=\dfrac{\sqrt{3}}{2}$.}
\end{ex}
\begin{ex}%[DCHT Toán 11 - KNTT - Nguyễn Hoài Nam]%[1K1Y1-6]
	Với điều kiện của $\alpha$ đã được thỏa mãn. Chọn khẳng định \textbf{sai} trong các khẳng định sau?
	\choice
	{$1+\tan^2{\alpha}=\dfrac{1}{\cos^2{\alpha}}$}
	{\True$\tan{\alpha}\cdot\cot{\alpha}=-1$}
	{$1+\cot^2{\alpha}=\dfrac{1}{\sin^2{\alpha}}$}
	{$\sin^2{\alpha}+\cos^2{\alpha}=1$}
	\loigiai{
		Ta có $\tan{\alpha}\cdot \cot{\alpha}=1$.}
\end{ex}
\begin{ex}%[DCHT Toán 11 - KNTT - Nguyễn Hoài Nam]%[1K1Y1-6]
	Giá trị của $\tan{180^{\circ}}$ là
	\choice
	{$1$}
	{\True$0$}
	{$-1$}
	{Không xác định}
	\loigiai{
		Ta có $\tan{180^{\circ}}=0$.}
\end{ex}
\begin{ex}%[DCHT Toán 11 - KNTT - Nguyễn Hoài Nam]%[1K1Y1-6]
	Cho $\dfrac{\pi}{2}<\alpha<\pi$. Chọn khẳng định đúng.
	\choice
	{$\sin{\alpha}>0,\ \cos{\alpha}>0$}
	{$\sin{\alpha}<0,\ \cos{\alpha}<0$}
	{\True $\sin{\alpha}>0,\ \cos{\alpha}<0$}
	{$\sin{\alpha}<0,\ \cos{\alpha}>0$}
	\loigiai{
		Với $\dfrac{\pi}{2}<\alpha<\pi$ là góc phần tư thứ II, nên $\sin{\alpha}>0,\ \cos{\alpha}<0$.}
\end{ex}
\begin{ex}%[DCHT Toán 11 - KNTT - Nguyễn Hoài Nam]%[1K1Y1-6]
	Biểu thức $f(x)=\cos^4{x}+\cos^2{x}\sin^2{x}+\sin^2{x}$ có giá trị bằng
	\choice
	{\True $1$}
	{$2$}
	{$-2$}
	{$-1$}
	\loigiai{
		Ta có $f(x)=\cos^4{x}+\cos^2{x}\sin^2{x}+\sin^2{x}=\cos^2{x}\left(\cos^2{x}+\sin^2{x}\right)+\sin^2{x}=\cos^2{x}+\sin^2{x}=1$. }
\end{ex}
%------------TH
\begin{ex}%[DCHT Toán 11 - KNTT - Nguyễn Hoài Nam]%[1K1B1-6]
	Cho $\cot x=-5$. Giá trị của $P=\dfrac{\sin x-2\cos x}{3\sin x+4\cos x}$ là
	\choice
	{$-\dfrac{7 }{11}$}
	{$\dfrac{ 7}{11}$}
	{\True $-\dfrac{11 }{17}$}
	{$\dfrac{11}{17}$}
	\loigiai{
		Biến đổi được $P=\dfrac{1-2\cot x}{3+4\cot x} =\dfrac{1-2(-5)}{3+4(-5)} =-\dfrac{11}{17}$.
	}
\end{ex}

\begin{ex}%[DCHT Toán 11 - KNTT - Nguyễn Hoài Nam]%[1K1B1-6] 
	Cho $ \cos 2\alpha=\dfrac{3}{5}\,\left(\dfrac{3\pi}{4}<\alpha<\pi\right) $.  Giá trị của $ \sin \alpha $ bằng
	\choice
	{\True $ \dfrac{\sqrt{5}}{5} $}
	{$ \dfrac{2\sqrt{5}}{5} $}
	{$ -\dfrac{2\sqrt{5}}{5} $}
	{$ -\dfrac{\sqrt{5}}{5} $}
	\loigiai{
		Ta có \[\sin^2\alpha=\dfrac{1-\cos2\alpha}{2}=\dfrac{1-\dfrac{3}{5}}{2}=\dfrac{1}{5}\Leftrightarrow \sin\alpha=\pm \dfrac{\sqrt{5}}{5}.\]
		Do $\dfrac{3\pi}{4}<\alpha<\pi$ nên $\sin\alpha>0$. Suy ra $\sin\alpha=\dfrac{\sqrt{5}}{5}$.
	}
\end{ex}

\begin{ex}%[DCHT Toán 11 - KNTT - Nguyễn Hoài Nam]%[1K1B1-6]
	Cho $\alpha$ là một góc lượng giác thỏa mãn $\tan \alpha=-2$, với $\dfrac{\pi}{2}<\alpha<\pi$. Khi đó, giá trị $\cos \alpha$ bằng
	\choice
	{$\cos \alpha=\dfrac{\sqrt{5}}{5}$}
	{\True $\cos \alpha=\dfrac{-\sqrt{5}}{5}$}
	{$\cos \alpha=\dfrac{-1}{5}$}
	{$\cos \alpha=\dfrac{1}{5}$}
	\loigiai{
		Ta có $1+\tan^2\alpha=\dfrac{1}{\cos^2\alpha}\Leftrightarrow 1+(-2)^2=\dfrac{1}{\cos^2\alpha}\Leftrightarrow \cos \alpha=\pm\dfrac{\sqrt{5}}{5}$.\\
		Vì $\dfrac{\pi}{2}<\alpha<\pi$ nên $\cos \alpha=\dfrac{-\sqrt{5}}{5}$.
	}
\end{ex}

\begin{ex}%[DCHT Toán 11 - KNTT - Nguyễn Hoài Nam]%[1K1B1-6]
	Giá trị của $\sin \alpha$ biết $\cos \alpha=-\dfrac{4}{5}$ và $\pi<\alpha<\dfrac{3 \pi}{2}$ bằng
	\choice
	{\True $\sin \alpha=-\dfrac{3}{5}$}
	{$\sin \alpha=\dfrac{1}{5}$}
	{$\sin \alpha=\dfrac{3}{5}$}
	{$\sin \alpha=-\dfrac{1}{5}$}
	\loigiai{
		Vì	$\pi<\alpha<\dfrac{3 \pi}{2}$ nên $\sin\alpha=-\sqrt{1-\cos^2\alpha}=-\sqrt{1-\left(-\dfrac{4}{5}\right) }=\sin \alpha=-\dfrac{3}{5}$.
	}
\end{ex}

\begin{ex}%[DCHT Toán 11 - KNTT - Nguyễn Hoài Nam]%[1K1B1-6]
	Cho $\tan a=2$. Khi đó, giá trị $A=\dfrac{1}{\cos^2a}+\dfrac{\cos a+\sin a}{\cos a-\sin a}-5$ bằng
	\choice
	{$A=-5$}
	{$A=-4$}
	{\True $A=-3$}
	{$A=-2$}
	\loigiai{
		Ta có 
		\allowdisplaybreaks
		\begin{eqnarray*}
			A&=&\dfrac{1}{\cos^2a}+\dfrac{\cos a+\sin a}{\cos a-\sin a}-5\\
			&=&1+\tan^2 a+\dfrac{\dfrac{\cos a+\sin a}{\cos a}}{\dfrac{\cos a-\sin a}{\cos a}}-5\\
			&=&1+\tan^2 a+\dfrac{1+\tan a}{1-\tan a}-5\\
			&=&1+4+\dfrac{1+2}{1-2}-5=-3.
		\end{eqnarray*}
	}
\end{ex}
%------------------VD-VDC


\begin{ex}%[DCHT Toán 11 - KNTT - Nguyễn Hoài Nam]%[1K1K1-6]
	Cho $\cot a = 4\tan a$  và $a\in\left(\dfrac{\pi}{2};\pi\right)$. Khi đó $\sin a$  bằng
	\choice{$ - \dfrac{{\sqrt 5 }}{5}$}{$\dfrac{1}{2}$}{$\dfrac{{2\sqrt 5 }}{5}$}{\True $\dfrac{{\sqrt 5 }}{5}$}
	\loigiai{
		Từ giả thiết ta có $$\cot^2a =4\tan a\cdot \cot a=4 \Rightarrow\sin^2 a=\dfrac{1}{1+\cot^2a}=\dfrac{1}{5}\Rightarrow \sin a=\dfrac{1}{\sqrt{5}}.$$
	}
\end{ex}
\begin{ex}%[DCHT Toán 11 - KNTT - Nguyễn Hoài Nam]%[1K1K1-6]
	Cho $\tan \alpha = - 2$. Giá trị của biểu thức $P = \dfrac{- \sin \alpha + 4 \cos \alpha}{\sin \alpha + 3 \cos \alpha}$ bằng
	\choice
	{\True $6$}
	{$2$}
	{$3$}
	{$12$}
	\loigiai{
		$$P = \dfrac{- \dfrac{\sin \alpha}{\cos \alpha} + 4}{\dfrac{\sin \alpha}{\cos \alpha} + 3} = \dfrac{- \tan \alpha + 4}{\tan \alpha + 3} = \dfrac{2 + 4}{-2 + 3} = 6.$$
	}
\end{ex}

\begin{ex}%[DCHT Toán 11 - KNTT - Nguyễn Hoài Nam]%[1K1K1-6]
	Cho $\sin a +2\cos a =0$  và $a\in\left(\dfrac{3\pi}{2};2\pi\right)$. Khi đó $\cos a$  bằng
	\choice
	{\True $ - \dfrac{{\sqrt 5 }}{5}$}
	{$\dfrac{1}{2}$}
	{$\dfrac{{2\sqrt 5 }}{5}$}
	{ $\dfrac{{\sqrt 5 }}{5}$}
	\loigiai{
		Ta có	 $\sin a +2\cos a =0 \Leftrightarrow \sin a=-2\cos a \Leftrightarrow \tan a =-2 .$
		\begin{eqnarray*}
			& & \cos^2 a=\dfrac{1}{1+\tan^2 a}=\dfrac{1}{1+(-2)^2}=\dfrac{1}{5}\\
			& \Rightarrow & \cos a =-\dfrac{\sqrt{5}}{5} \ (\text{vì} \ \dfrac{3\pi}{2}<a<2\pi).
		\end{eqnarray*}
		
	}
\end{ex}
\begin{ex}%[DCHT Toán 11 - KNTT - Nguyễn Hoài Nam]%[1K1G1-6]
	Cho $ \tan x = 2 $. Tính $ A = \dfrac{{\sin^2}x - 2\sin x\cdot\cos x}{{\cos^2}x + 3\cdot{\sin^2}x} $.
	\choice
	{$A = 4$}
	{\True$A = 0$}
	{$A = 1$}
	{$A = 2$}
	\loigiai{
		Chia cả tử và mẫu của $A$ cho $ {\cos^2}x $ ta được \\
		$$ A = \dfrac{{\tan^2}x - 2\tan x}{1+3\tan x} = \dfrac{2^2 - 2\cdot 2}{1+3\cdot2} = 0.$$}
\end{ex}

\Closesolutionfile{ans}
% \begin{indapan}{10}
% 	{ans/ans-1K1-1-Dang6}
% \end{indapan}

