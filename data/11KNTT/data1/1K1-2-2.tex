
% \subsection{Các dạng toán thường gặp}
\begin{dang}{Áp dụng công thức nhân đôi, hạ bậc}
\end{dang}
\subsubsection{Ví dụ mẫu}
\begin{vd}%[DCHT Toán 11 - KNTT -Nguyễn Thành Nhân]%[1K1B2-2]
 Biến đổi thành tích biểu thức sau
\[A=\sin 2x-\sin x+2\cos x-1.\]
	\loigiai{
	Ta có
	\begin{eqnarray*}
	A&=& \sin 2x-\sin x+2\cos x-1=2\sin x\cos x-\sin x+2\cos x-1\\
	&=&\sin x(2\cos x-1)+(2\cos x-1)\\
	&=& (2\cos x-1)(\sin x+1).
	\end{eqnarray*}
	}
	\end{vd}
\begin{vd} %[DCHT Toán 11 - KNTT -Nguyễn Thành Nhân]%[1K1B2-2]
Rút gọn các biểu thức (giả sử các góc làm cho biểu thức có nghĩa).
\begin{enumerate}
\item $A=\dfrac{\left(1+\sin 2a\right)(\cos a-\sin a)}{\cos 2a(\cos a+\sin a)}$.
\item $B=\dfrac{\sin a+\sin 2a}{\cos a+\cos 2a+1}$.
\end{enumerate} 
	\loigiai{
	\begin{enumerate}
\item Ta có 
\begin{eqnarray*}
A&=&\dfrac{\left(1+\sin 2a\right)(\cos a-\sin a)}{\cos 2a(\cos a+\sin a)}=\dfrac{\left(\sin a+\cos a\right)^2(\cos a-\sin a)}{(\cos^2 a-\sin^2 a)\left(\sin a+\cos a\right)}=1.
\end{eqnarray*}
\item Ta có 
\begin{eqnarray*}
B&=&\dfrac{\sin a+\sin 2a}{\cos a+\cos 2a+1}=\dfrac{\sin a(1+2\cos a)}{\cos a +2\cos^2 a}=\dfrac{\sin a(1+2\cos a)}{\cos a(1+2\cos a)}=\tan a.
\end{eqnarray*}
\end{enumerate}
	}
	\end{vd}
\begin{vd}%[DCHT Toán 11 - KNTT -Nguyễn Thành Nhân]%[1K1B2-2]
 Cho $\cos a=\dfrac{5}{13}$ với $0<a<\dfrac{\pi}{2}$. Tính $\sin 2a$, $\cos 2a$, $\tan 2a$, $\sin \left(2a+\dfrac{\pi}{3}\right)$, $\tan\left(2a-\dfrac{\pi}{6}\right)$. 
	\loigiai{
	Ta có $\sin^2 a=1-\cos^2 a=1-\dfrac{25}{169}=\dfrac{144}{169}\Leftrightarrow \hoac{&\sin a=\dfrac{12}{13}\\&\sin a=\dfrac{-12}{13}.}$\\
	Vì $0<a<\dfrac{\pi}{2}$ nên $\sin a>0$, suy ra $\sin a=\dfrac{12}{13}$.\\
	Từ đó ta có
	\begin{eqnarray*}
	&& \sin 2a=2\sin a\cos a=2\cdot \dfrac{5}{13}\cdot \dfrac{12}{13}=\dfrac{120}{169};\\
	&& \cos 2a=2\cos ^2 a-1=2\cdot \dfrac{25}{169}-1=-\dfrac{119}{169};\\
	&& \tan 2a=\dfrac{\sin 2a}{\cos 2a}=-\dfrac{120}{119};\\
	\sin \left(2a+\dfrac{\pi}{3}\right)&=&\sin 2a\cdot \cos \dfrac{\pi}{3}+ \cos 2a\cdot \sin \dfrac{\pi}{3}\\
	&=&\dfrac{120}{169}\cdot \dfrac{1}{2}-\dfrac{120}{119}\cdot \dfrac{\sqrt{3}}{2}=\dfrac{60(1-\sqrt{3})}{119}.\\
	 \tan\left(2a-\dfrac{\pi}{6}\right)&=&\dfrac{\tan 2a-\tan \dfrac{\pi}{6}}{1+\tan 2a\tan \dfrac{\pi}{6}}=\dfrac{-\dfrac{120}{119}-\dfrac{\sqrt{3}}{3}}{1-\dfrac{120}{119}\cdot \dfrac{\sqrt{3}}{3}}\\
	 &=&-\dfrac{360+119\sqrt{3}}{357-120\sqrt{3}}.
	\end{eqnarray*}
	}
	\end{vd}
	\begin{vd} %[DCHT Toán 11 - KNTT -Nguyễn Thành Nhân]%[1K1B2-2]
	Cho $\sin 2 a=\dfrac{3}{5}$ với $\dfrac{\pi}{2}<a<\pi$. Tính $\tan a+\cot a$, $\tan a-\cot a$.
	\loigiai{
	Ta có 
	\begin{eqnarray*}
	\tan a+\cot a&=&\dfrac{\sin a}{\cos a}+\dfrac{\cos a}{\sin a}=\dfrac{\sin^2 a+\cos^2 a}{\sin a\cos a} \\
	&=&\dfrac{1}{\dfrac{1}{2}\sin 2a}=\dfrac{2}{\sin 2a}.
	\end{eqnarray*}
	Thay $\sin 2a=\dfrac{3}{5}$, ta được $\tan a+\cot a=\dfrac{2}{\dfrac{3}{5}}=\dfrac{10}{3}$.\\
	Ta cũng có 
	\begin{eqnarray*}
	\tan a-\cot a&=&\dfrac{\sin a}{\cos a}-\dfrac{\cos a}{\sin a}=\dfrac{\sin^2 a-\cos^2 a}{\sin a\cos a} \\
	&=&\dfrac{-\cos 2a}{\dfrac{1}{2}\sin 2a}=-2\cot 2a.
	\end{eqnarray*}
	Vì $\sin 2a=\dfrac{3}{5}$ và $\dfrac{\pi}{2}<a<\pi$ nên $\cos 2a<0$, suy ra $\cos 2a=-\sqrt{1-\sin^2 2a} =-\dfrac{4}{5}$. \\
	Do đó $\tan a-\cot a=-\cot 2a=-\dfrac{-\dfrac{4}{5}}{\dfrac{3}{5}}=\dfrac{4}{3}$.
	}
	\end{vd}
	\begin{vd}%[DCHT Toán 11 - KNTT -Nguyễn Thành Nhân]%[1K1B2-2]
	 Cho $\sin a+\cos a=m,\,(-\sqrt{2}\le m \le \sqrt{2})$. Tính $\left|\sin a-\cos a\right|$.
	\loigiai{
	Ta có $m^2=\left(\sin a+\cos a\right)^2=1+2\sin a\cos a=1+\sin 2a$. Suy ra $\sin 2a=m^2-1$. Do đó
	\begin{eqnarray*}
	\left|\sin a-\cos a\right|&=& \sqrt{\left(\sin a-\cos a\right)^2}=\sqrt{1-\sin 2a}=\sqrt{2-m^2}.
	\end{eqnarray*}
	
	}
	\end{vd}
	
	
	\begin{vd} %[DCHT Toán 11 - KNTT -Nguyễn Thành Nhân]%[1K1B2-2]
	Rút gọn biểu thức $P=\dfrac{3-4\cos 2a+\cos 4a}{3+4\cos 2a+\cos 4a}$.
	\loigiai{
Ta có
\begin{eqnarray*}
P&=& \dfrac{3-4\cos 2a+\cos 4a}{3+4\cos 2a+\cos 4a}\\
&=& \dfrac{3-4\cos 2a+2\cos^2 2a-1}{3+4\cos 2a+2\cos^2 2a-1}\\
&=& \dfrac{2\cos^2 2a-4\cos 2a+2}{2\cos^2 2a+4\cos 2a+2}\\
&=& \dfrac{2\left(\cos 2a-1\right)^2}{2\left(\cos 2a+1\right)^2}= \dfrac{\left(-2\sin^2 a\right)^2}{\left(2\cos^2 a\right)^2}=\tan^4 a.
\end{eqnarray*}
	}
	\end{vd}
	\begin{vd}\label{vd1}%[DCHT Toán 11 - KNTT -Nguyễn Thành Nhân]%[1K1K2-2]
	Chứng minh các đẳng thức
	\begin{enumerate}
	\item $\sin^4 x+\cos^4 x=\dfrac{1}{4}\cos 4x+\dfrac{3}{4}$;
	\item $\sin^6 x+\cos^6 x=\dfrac{3}{8}\cos 4x+\dfrac{5}{8}$.
	\end{enumerate}
	\loigiai{
	\begin{enumerate}
	\item Ta có 
	\begin{eqnarray*}
	 \sin^4 x+\cos^4 x&=&\left(\sin^2 x+\cos^2 x\right)^2-2\sin^2 x\cos^2 x\\
	&=& 1-\dfrac{1}{2}\sin^2 2x\\
	&=& 1-\dfrac{1}{2}\cdot \dfrac{1-\cos 4x}{2}\\
	&=&\dfrac{1}{4}\cos 4x+\dfrac{3}{4}.
	\end{eqnarray*}
	\item  Ta có 
	\begin{eqnarray*}
	 \sin^6 x+\cos^6 x&=&\left(\sin^2 x+\cos^2 x\right)^3-3\sin^2 x\cos^2 x\left(\sin^2 x+\cos^2 x\right)\\
	&=& 1-\dfrac{3}{4}\sin^2 2x\\
	&=& 1-\dfrac{3}{4}\cdot \dfrac{1-\cos 4x}{2}\\
	&=&\dfrac{3}{8}\cos 4x+\dfrac{5}{8}.
	\end{eqnarray*}
	\end{enumerate}
	}
	\end{vd}
\subsubsection{Bài tập rèn luyện}
\begin{bt} %[DCHT Toán 11 - KNTT -Nguyễn Thành Nhân]%[1K1B2-2]
Biến đổi thành tích biểu thức $B=\cos 2x+\cos x-\sin x$.
	\loigiai{
Ta có 
\begin{eqnarray*}
B&=&\cos^2 x-\sin^2 x+\cos x-\sin x\\
&=& (\cos x-\sin x)(\cos x+\sin x)+\cos x-\sin x\\
&=& (\cos x-\sin x)(\cos x+\sin x+1).
\end{eqnarray*}
	}
	\end{bt}
	\begin{bt} %[DCHT Toán 11 - KNTT -Nguyễn Thành Nhân]%[1K1B2-2]
	Rút gọn biểu thức (giả sử các góc làm cho biểu thức có nghĩa)
	\[P=\dfrac{\cos 2x+\cos x+\sin x}{\cos x-\sin x+1}-\cos x-\sin x+2023\]
	\loigiai{
Ta có 
\begin{eqnarray*}
P&=&\dfrac{\cos 2x+\cos x+\sin x}{\cos x-\sin x+1}-\cos x-\sin x+2023\\
&=& \dfrac{\cos^2 x-\sin^2 x+\cos x+\sin x}{\cos x-\sin x+1}-\cos x-\sin x+2023\\
&=&\dfrac{(\cos x+\sin x)(\cos x-\sin x+1)}{\cos x-\sin x+1}-\cos x-\sin x+2023\\
&=&\cos x+\sin x-\cos x-\sin x+2023\\
&=& 2023.
\end{eqnarray*}
	}
	\end{bt}
\begin{bt} %[DCHT Toán 11 - KNTT -Nguyễn Thành Nhân]%[1K1B2-2]
Chon $\sin 4x=\dfrac{1}{2}$. Tính giá trị biểu thức $A=\sin x \cos^3 x-\cos x\sin^3 x$.
	\loigiai{
Ta có 
\begin{eqnarray*}
A&=&\sin x \cos^3 x-\cos x\sin^3 x\\
&=& \sin x\cos x\left(\cos^2 x-\sin^2 x\right)\\
&=& \sin x\cdot \cos x\cdot \cos 2x=\dfrac{1}{2}\cdot \sin 2x 
\cdot \cos 2x\\
&=& \dfrac{1}{4}\cdot \sin 4x.
\end{eqnarray*}
Vậy $A=\dfrac{1}{4}\cdot \dfrac{1}{2}=\dfrac{1}{8}$.
	}
	\end{bt}
	\begin{bt}%[DCHT Toán 11 - KNTT -Nguyễn Thành Nhân]%[1K1B2-2]
	 Biết $\tan^2 x+\cot^2 x+\dfrac{1}{\sin^2 x}+\dfrac{1}{\cos^2 x}=7$. Tính $\sin^2 2x$.
	\loigiai{
Ta có 
\begin{eqnarray*}
&& \tan^2 x+\cot^2 x+\dfrac{1}{\sin^2 x}+\dfrac{1}{\cos^2 x}=7\\
&\Leftrightarrow & \dfrac{\sin^2 x}{\cos^2 x}+ \dfrac{\cos^2 x}{\sin^2 x}+\dfrac{\sin^2 x+\cos^2 x}{\sin^2 x\cdot \cos^2 x}=7\\
&\Leftrightarrow & \dfrac{\sin^4 x+\cos^4 x+1}{\sin^2 x\cdot \cos^2 x}=7\\
&\Leftrightarrow & \dfrac{\left(\sin^2 x+\cos^2 x\right)^2-2\sin^2 x\cdot \cos^2 x+1}{\dfrac{1}{4}\sin^2 2x}=7\\
&\Leftrightarrow & 2-\dfrac{2}{4}\sin^2 2x=\dfrac{7}{4}\sin^2 2x\\
&\Leftrightarrow &\dfrac{9}{4}\sin^2 2x=2\Leftrightarrow \sin^2 2x=\dfrac{8}{9}.
\end{eqnarray*}
Vậy $\sin^2 2x=\dfrac{8}{9}$.
	}
	\end{bt}
	\begin{bt}%[DCHT Toán 11 - KNTT -Nguyễn Thành Nhân]%[1K1B2-2]
	 Cho $\cos a=-\dfrac{2}{3}$ với $\dfrac{\pi}{2}<a<\pi$. Biết $S=\cos 2a+\sin 2a=m+n\sqrt{5}$ với $m$, $n\in \mathbb{Q}$ và $\dfrac{m}{n}=\dfrac{p}{q}$ là phân số tối giản. Tính $p-q$.
	\loigiai{
Vì $\dfrac{\pi}{2}<a<\pi$ nên $\sin a>0$. Do đó 
\[\sin a=\sqrt{1-\cos^2 a}=\sqrt{1-\dfrac{4}{9}}=\dfrac{\sqrt{5}}{3}.\]
Do đó
\begin{eqnarray*}
S&=& \cos^2 a-\sin^2 a+2\sin a\cdot \cos a\\
&=& \dfrac{4}{9}-\dfrac{5}{9}+2\cdot \dfrac{-2}{3}\cdot \dfrac{\sqrt{5}}{3}\\
&=&-\dfrac{1}{9}-\dfrac{4}{9}\cdot \sqrt{5}.
\end{eqnarray*}
Suy ra $m=-\dfrac{1}{9}$ và $n=-\dfrac{4}{9}$. Nên $\dfrac{m}{n}=\dfrac{1}{4}$. Suy ra $p=1$, $q=4$.\\
Vậy $p-q=-3$.
	}
	\end{bt}
	\begin{bt}%[DCHT Toán 11 - KNTT -Nguyễn Thành Nhân]%[1K1B2-2]
Rút gọn các biểu thức sau
\begin{listEX}[2]
\item $A=\sin x\cos x\cos 2x$.
\item $B=\cos^42x-\sin^42x$.
\item $C=4\sin x\cdot\sin\left(x+\dfrac{\pi}{2}\right)\cdot\sin\left(2x+\dfrac{\pi}{2}\right)$.
\item $D=\sin 2x+\cos 2x-2\cos x\left(\sin x+\cos x\right)+1$.
\end{listEX}
\loigiai{
\begin{enumerate}
\item $A=\dfrac{1}{2}\cdot 2\sin x\cos x\cos 2x=\dfrac{1}{2}\sin 2x\cos 2x=\dfrac{1}{4}\cdot 2\sin 2x\cos 2x=\dfrac{1}{4}\sin 4x$.
\item $B=\left(\cos^22x+\sin^22x\right)\left(\cos^22x-\sin^22x\right)=\cos 4x$.
\item $C=4\sin x\cdot\cos x\cdot\cos 2x=2\sin 2x\cdot\cos 2x=\sin 4x$.
\item $D=\sin 2x+\cos 2x-\sin 2x-2\cos^2x+1=\cos 2x-\cos 2x=0$.
\end{enumerate}
}
\end{bt}
\begin{bt}%[DCHT Toán 11 - KNTT -Nguyễn Thành Nhân]%[1K1B2-2]
Cho $\cos 2x=\dfrac{1}{3}$ Tính giá trị các biểu thức sau
\begin{listEX}[2]
\item $A=\sin^2x\cdot\cos^2x$.
\item $B=\dfrac{1+\sin^2x}{\cos^2x}$.
\item $C=\dfrac{1+\cot^2x}{1-\cot^2x}$.
\item $D=\sin^6x+\cos^6x$.
\end{listEX}
\loigiai{
\begin{enumerate}
\item $A=\dfrac{1}{2}\left(1-\cos 2x\right)\cdot\dfrac{1}{2}\left(1+\cos 2x\right)=\dfrac{1}{4}\left(1-\cos^22x\right)=\dfrac{1}{4}\left(1-\dfrac{1}{9}\right)=\dfrac{2}{9}$.
\item $B=\dfrac{1+\dfrac{1}{2}\left(1-\cos 2x\right)}{\dfrac{1}{2}\left(1+\cos 2x\right)}=\dfrac{3-\cos 2x}{1+\cos 2x}=\dfrac{3-\dfrac{1}{3}}{1+\dfrac{1}{3}}=2$.
\item $C=\dfrac{\sin^2x+\cos^2x}{\sin^2x-\cos^2x}=\dfrac{-1}{\cos 2x}=\dfrac{-1}{\dfrac{1}{3}}=-3$.
\item \begin{eqnarray*}
D&=&\left(\sin^2x+\cos^2x\right)^3-3\sin^2x\cos^2x\left(\sin^2x+\cos^2x\right)\\
&=&1-\dfrac{3}{4}\left(1-\cos^22x\right)=\dfrac{1}{4}+\dfrac{3}{4}\cos^22x\\
&=& \dfrac{1}{4}+\dfrac{3}{4}\cdot \dfrac{1}{9}=\dfrac{1}{3}.
\end{eqnarray*}
\end{enumerate}
}
\end{bt}
\begin{bt}%[DCHT Toán 11 - KNTT -Nguyễn Thành Nhân]%[1K1K2-2]
Chứng minh đẳng thức $\sin^6 x \cos^2 x+\sin^2 x\cos^6 x=\dfrac{1}{8}\left(1-\cos^4 2x\right)$.
\loigiai{Biến đổi vế trái, ta có
\begin{eqnarray*}
VT&=& \sin^6 x \cos^2 x+\sin^2 x\cos^6 x\\
&=& \sin^2 x\cos^2 x\left(\sin^4 x+\cos^4 x\right)\\
&=& \dfrac{1}{4}\cdot \sin^2 2x\left(1-2\sin^2 x\cos ^2 x\right)\\
&=& \dfrac{1}{4}\cdot \sin^2 2x\left(1-\dfrac{1}{2}\sin^2 2x\right)\\
&=& \dfrac{1}{8}\cdot \sin^2 2x\left(2-\sin^2 2x\right)\\
&=& \dfrac{1}{8}\cdot \left(1-\cos^2 2x\right)\cdot \left(1+\cos^2 2x\right)\\
&=&\dfrac{1}{8}\left(1-\cos^4 2x\right)=VP.
\end{eqnarray*}

}
\end{bt}

\begin{bt}%[DCHT Toán 11 - KNTT -Nguyễn Thành Nhân]%[1K1B2-2]
Chứng minh các đẳng thức sau
\begin{listEX}[1]
\item $8\sin^4x=3-4\cos 2x+\cos 4x$.
\item $\sin 4x=4\sin x\cdot\cos x\left(1-2\sin^2x\right)$.
\end{listEX}
\loigiai{
\begin{enumerate}
\item $VP=3-4\left(1-2\sin^2x\right)+2\cos^22x-1=-2+8\sin^2x+2\left(1-2\sin^2x\right)^2=8\sin^4x=VT$.
\item $VP=2\sin 2x\cdot\cos 2x=\sin 4x=VT$.
\end{enumerate}
}
\end{bt}
\begin{bt}%[DCHT Toán 11 - KNTT -Nguyễn Thành Nhân]%[1K1B2-2]
Chứng minh đẳng thức $\dfrac{\sin x+\cos x-1}{\sin x-\cos x+1}=\dfrac{\cos x}{1+\sin x}$.
\loigiai{
Biến đổi vế trái, ta được
\begin{eqnarray*}
VT&=& \dfrac{\sin x+\cos x-1}{\sin x-\cos x+1}=\dfrac{2\sin \dfrac{x}{2}\cos \dfrac{x}{2}-2\sin^2 \dfrac{x}{2}}{2\sin \dfrac{x}{2}\cos \dfrac{x}{2}+2\sin^2 \dfrac{x}{2}}\\
&=& \dfrac{2\sin \dfrac{x}{2}\left(\cos \dfrac{x}{2}-\sin \dfrac{x}{2}\right)}{2\sin \dfrac{x}{2}\left(\cos \dfrac{x}{2}+\sin \dfrac{x}{2}\right)}=\dfrac{ \cos \dfrac{x}{2}-\sin \dfrac{x}{2}}{\cos \dfrac{x}{2}+\sin \dfrac{x}{2}}\\
&=& \dfrac{\cos^2 \dfrac{x}{2}-\sin^2 \dfrac{x}{2}}{\left(\cos \dfrac{x}{2}+\sin \dfrac{x}{2}\right)^2}=\dfrac{\cos x}{1+\sin x}=VP.
\end{eqnarray*}
}
\end{bt}
\begin{bt}%[DCHT Toán 11 - KNTT -Nguyễn Thành Nhân]%[1K1K2-2]
Chứng minh đẳng thức $\sin^2 \left(\dfrac{\pi}{8}+x\right)-\sin^2 \left(\dfrac{\pi}{8}-x\right)=\dfrac{\sqrt{2}}{2}\sin 2x$.
\loigiai{
Áp dụng công thức hạ bậc, biến đổi vế trái, ta được
\begin{eqnarray*}
VT&=& \sin^2 \left(\dfrac{\pi}{8}+x\right)-\sin^2 \left(\dfrac{\pi}{8}-x\right)\\
&=& \dfrac{1-\cos \left(\dfrac{\pi}{4}+2x\right)}{2}-\dfrac{1-\cos \left(\dfrac{\pi}{4}-2x\right)}{2}\\
&=& \dfrac{\cos \left(\dfrac{\pi}{4}-2x\right)-\cos \left(\dfrac{\pi}{4}+2x\right)}{2}\\
&=& \dfrac{-2 \sin \dfrac{\pi}{4}\sin (-2x)}{2}=\dfrac{\sqrt{2}}{2}\sin 2x=VP.
\end{eqnarray*}
}
\end{bt}

\begin{bt}%[DCHT Toán 11 - KNTT -Nguyễn Thành Nhân]%[1K1K2-2]
Chứng minh đẳng thức $4\cos^4 x-2\cos 2x-\dfrac{1}{4}\cos 4x=\dfrac{3}{2}$.
\loigiai{
Điều phải chứng minh tương đương
\begin{eqnarray*}
&& 4\cos^4 x-2\cos 2x-\dfrac{1}{4}\cos 4x-\dfrac{1}{2}=1\\
&& \Leftrightarrow 4\left(\cos^2 x\right)^2-2\cos 2x-\dfrac{1}{2}\left(1+\cos 4x\right)=1\\
&& \Leftrightarrow 4\left(\dfrac{1+\cos 2x}{2}\right)^2-2\cos 2x-\dfrac{1}{2}\cdot 2\cos^2 2x=1\\
&& \Leftrightarrow 1+\cos^2 2x+2\cos 2x-2\cos 2x-\cos^2 2x=1\\
&& \Leftrightarrow 1=1.
\end{eqnarray*}
Đẳng thức cuối hiến nhiên đúng. Ta có điều cần chứng minh.
}
\end{bt}
\begin{bt}%[DCHT Toán 11 - KNTT -Nguyễn Thành Nhân]%[1K1B2-2]
Chứng minh đẳng thức $\dfrac{\cos x+\sin x}{\cos x-\sin x}-\dfrac{\cos x-\sin x}{\cos x+\sin x}=2\tan 2x$ với $x$ mà biểu thức có nghĩa.
\loigiai{
Biến đổi vế trái, ta được
\begin{eqnarray*}
VT&=& \dfrac{\cos x+\sin x}{\cos x-\sin x}-\dfrac{\cos x-\sin x}{\cos x+\sin x}\\
&=& \dfrac{\left(\cos x+\sin x\right)^2-\left(\cos x-\sin x\right)^2}{\cos^2 x-\sin^2 x}\\
&=&\dfrac{1+\sin 2x-(1-\sin 2x)}{\cos 2x}\\
&=& \dfrac{2\sin 2x}{\cos 2x}=2\tan 2x\\
&=& VP.
\end{eqnarray*}
}
\end{bt}
\begin{bt}%[DCHT Toán 11 - KNTT -Nguyễn Thành Nhân]%[1K1B2-2]
Chứng minh biểu thức 
\[A=\cos^2 x+\cos^2 \left(x+\dfrac{\pi}{3}\right)+\cos^2 \left(\dfrac{\pi}{3}-x\right)\]
 có giá trị không phụ thuộc vào biến số $x$.
\loigiai{
Áp dụng công thức hạ bậc, ta được
\begin{eqnarray*}
A&=&\cos^2 x+\cos^2 \left(x+\dfrac{\pi}{3}\right)+\cos^2 \left(\dfrac{\pi}{3}-x\right)\\
&=& \dfrac{1+\cos 2x}{2}+\dfrac{1+\cos\left(2x+\dfrac{2\pi}{3}\right)}{2}+\dfrac{1+\cos\left(2x-\dfrac{2\pi}{3}\right)}{2}\\
&=&\dfrac{3}{2}+\dfrac{1}{2}\cos 2x+\dfrac{1}{2}\left[\cos\left(2x+\dfrac{2\pi}{3}\right)+\cos\left(2x-\dfrac{2\pi}{3}\right)\right]\\
&=& \dfrac{3}{2}+\dfrac{1}{2}\cos 2x+\dfrac{1}{2}\cdot 2\cdot \cos 2x\cdot \cos \dfrac{2\pi}{3}\\
&=&\dfrac{3}{2}+\dfrac{1}{2}\cos 2x -\dfrac{1}{2}\cos 2x \\
&=& \dfrac{3}{2}.
\end{eqnarray*}
}
\end{bt}
\begin{bt}%[DCHT Toán 11 - KNTT -Nguyễn Thành Nhân]%[1K1G2-2]
Cho tam giác $ABC$ không tù, thỏa mãn điều kiện
\[\cos 2A+2\sqrt{2}\cos B+2\sqrt{2}\cos C=3.\]
Xác định ba góc của tam giác.
\loigiai{
Xét $M=\cos 2A+2\sqrt{2}\cos B+2\sqrt{2}\cos C-3$. Ta có
\begin{eqnarray*}
M&=&2\cos^2 A-1+4\sqrt{2}\cos \dfrac{B+C}{2}\cdot \cos \dfrac{B-C}{2}-3\\
&=&2\cos^2 A+4\sqrt{2}\sin \dfrac{A}{2}\cdot \cos \dfrac{B-C}{2}-4.
\end{eqnarray*}
Do $\sin \dfrac{A}{2}>0$ và $0<\cos \dfrac{B-C}{2}\le 1$ nên 
\[M\le 2\cos^2 A +4\sqrt{2}\sin \dfrac{A}{2}-4.\]
Mặt khác, tam giác $ABC$ không tù nên $\cos A\geq 0\Rightarrow \cos^2 A\le \cos A$. Do đó
\begin{eqnarray*}
M&\le & 2\cos A+4\sqrt{2}\sin \dfrac{A}{2}-4\\
&=& 2\left(1-2\sin^2 \dfrac{A}{2}\right)+4\sqrt{2}\sin \dfrac{A}{2}-4\\
&=& -2\left(\sqrt{2}\sin \dfrac{A}{2}-1\right)^2\le 0.
\end{eqnarray*}
Do đó $M\le 0$ hay $\cos 2A+2\sqrt{2}\cos B+2\sqrt{2}\cos C\le 3$. Dấu đẳng thức xảy ra khi và chỉ khi
\[\heva{&\cos^2 A=\cos A\\&\cos \dfrac{B-C}{2}=1\\&\sin \dfrac{A}{2}=\dfrac{\sqrt{2}}{2}}\Leftrightarrow \heva{&A=90^{\circ}\\&B=C=45^{\circ}.}\]
Vậy tam giác $ABC$ vuông cân tại $A$.
}
\end{bt}
\begin{bt}%[DCHT Toán 11 - KNTT -Nguyễn Thành Nhân]%[1K1K2-2]
Chứng minh rằng $\dfrac{\sin^4 x+\cos^4 x-1}{\sin^6 x+\cos^6 x-1}=\dfrac{2}{3}$.
\loigiai{
Ta đã chứng minh được (xem Ví dụ \ref{vd1}) các kết quả \\
$\sin^4 x+\cos^4 x=\dfrac{1}{4}\cos 4x+\dfrac{3}{4}$; $\sin^6 x+\cos^6 x=\dfrac{3}{8}\cos 4x+\dfrac{5}{8}$. Do đó
\begin{eqnarray*}
VT&=& \dfrac{\sin^4 x+\cos^4 x-1}{\sin^6 x+\cos^6 x-1}= \dfrac{\dfrac{1}{4}\cos 4x-\dfrac{1}{4}}{\dfrac{3}{8}\cos 4x-\dfrac{3}{8}}\\
&=& \dfrac{\dfrac{1}{4}\left(\cos 4x-1\right)}{\dfrac{3}{8}\left(\cos 4x-1\right)}=\dfrac{2}{3}\\
&=& VP.
\end{eqnarray*}
}
\end{bt}

\begin{bt}%[DCHT Toán 11 - KNTT -Nguyễn Thành Nhân]%[1K1G2-2]
Chứng minh với mọi $x$, $y$, $z$, ta có
\[\cos^2 x+\cos^2 y-\cos^2 z-\cos^2 \left(x+y+z\right)=2\cos\left(x+y\right)\sin\left(y+z\right)\sin\left(z+x\right).\]
\loigiai{
Biến đổi vế trái, ta được
\begin{eqnarray*}
VT&=& \cos^2 x+\cos^2 y-\cos^2 z-\cos^2 \left(x+y+z\right)\\
&=& \dfrac{1+\cos 2x}{2}+\dfrac{1+\cos 2y}{2}-\dfrac{1+\cos 2z}{2}-\dfrac{1+\cos 2(x+y+z)}{2}\\
&=& \dfrac{1}{2}\left(\cos 2x+\cos 2y\right)-\dfrac{1}{2}\left(\cos 2(x+y+z)+\cos 2z\right)\\
&=& \cos (x+y)\cdot \cos (x-y)-\cos (x+y+2z)\cdot \cos (x+y)\\
&=&-\cos (x+y)\left[\cos (x+y+2z)-\cos (x-y)\right]\\
&=& 2\cos\left(x+y\right)\sin\left(y+z\right)\sin\left(z+x\right)\\
&=& VP.
\end{eqnarray*}

}
\end{bt}
