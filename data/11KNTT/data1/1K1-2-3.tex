\begin{dang}{Áp dụng công thức biến đổi tích thành tổng, tổng thành tích}
	
\end{dang}
\subsubsection{Ví dụ mẫu}
\setcounter{vd}{0}
\begin{vd}%%[DCHT Toán 11 - KNTT -Nguyễn Thành Nhân]%[1K1B2-3]
	Biến đổi các tổng sau thành tích
	\begin{listEX}[2]
		\item $A=\sin 5x+\sin 6x+\sin 7x+\sin 8x$.
		\item $B=\sin x-\sin 3x+\sin 7x-\sin 5x$.
		\item $C=\cos 7x+\sin 3x+\sin 2x-\cos 3x$.
		\item $D=\sin 35^\circ +\cos 40^\circ +\sin 55^\circ +\cos 20^\circ$.
	\end{listEX}
	\loigiai{
		\begin{enumerate}
			\item $A=\left(\sin 8x+\sin 5x\right)+\left(\sin 7x+\sin 6x\right)=2\sin \dfrac{13x}{2}\cos \dfrac{3x}{2}+2\sin \dfrac{13x}{2}\cos \dfrac{x}{2}=2\sin \dfrac{13x}{2}\left(\cos \dfrac{3x}{2}+\cos \dfrac{x}{2}\right)$.
			\item $B=\left(\sin 7x+\sin x\right)-\left(\sin 5x+\sin 3x\right)=2\sin 4x\cos 3x-2\sin 4x\cos x=2\sin 4x\left(\cos 3x-\cos x\right)$.
			\item $C=\left(\cos 7x-\cos 3x\right)+\left(\sin 3x+\sin 2x\right)=-2\sin 5x\sin 2x+2\sin \dfrac{5x}{2}\cos \dfrac{x}{2}=2\sin \dfrac{5x}{2}\left(-2\cos \dfrac{5x}{2}\sin 2x+\cos \dfrac{x}{2}\right)$.
			\item $D=\left(\sin 55^\circ +\sin 35^\circ\right)+\left(\cos 40^\circ +\cos 20^\circ\right)=2\sin 45^\circ\cos 10^\circ +2\cos 30^\circ\cos 10^\circ =2\cos 10^\circ\left(\sin 45^\circ +\cos 30^\circ\right)$.
		\end{enumerate}
	}
\end{vd}
\begin{vd}%%[DCHT Toán 11 - KNTT -Nguyễn Thành Nhân]%[1K1B2-3]
	Chứng minh đẳng thức $\cos^3 a\cos 3a-\sin^3 a\sin 3a=\dfrac{3}{4}\cos 4a+\dfrac{1}{4}$.
	\loigiai{
		Biến đổi vế trái, ta có
		\begin{eqnarray*}
			VT&=& \cos^3 a\cdot\cos 3a-\sin^3 a\cdot\sin 3a\\
			&=& \left(\cos a\cdot \cos 3a\right)\cdot \cos^2 a-\left(\sin a\cdot \sin 3a\right)\cdot \sin^2 a\\
			&=& \dfrac{1}{2}\left[\cos 2a+\cos 4a\right]\cdot \cos^2 a-\dfrac{1}{2}\left[\cos 2a-\cos 4a\right]\cdot \sin^2 a\\
			&=& \dfrac{1}{2}\cdot \cos 2a\left(\cos^2 a-\sin^2 a\right)+\dfrac{1}{2}\cdot \cos 4a\left(\cos^2 a+\sin^2 a\right)\\
			&=&\dfrac{1}{2}\cdot \cos^2 2a+\dfrac{1}{2}\cdot \cos 4a=\dfrac{1}{2}\cdot \dfrac{1+\cos 4a}{2}+\dfrac{1}{2}\cdot \cos 4a\\
			&=&\dfrac{3}{4}\cos 4a+\dfrac{1}{4}\\
			&=&VP.
		\end{eqnarray*}
	}
\end{vd}
\begin{vd}%%[DCHT Toán 11 - KNTT -Nguyễn Thành Nhân]%[1K1B2-3]
	Rút gọn các biểu thức sau
	\begin{listEX}[2]
		\item $A=\cos 11x\cos 3x-\cos 17x\cos 9x$.
		\item $B=\sin 18x\cos 3x-\sin 19x\cos 4x$.
		\item $C=\sin x\sin 3x+\sin 4x\sin 8x$.
		\item $D=\sin 2x\sin 6x-\cos x\cos 3x$.
		\item $E=\sin x\sin\left(\dfrac{\pi}{3}-x\right)\sin\left(\dfrac{\pi}{3}+x\right)$.
		\item $F=\cos \dfrac{x}{2}\cos \dfrac{3x}{2}-\sin x\sin 3x-\sin 2x\sin 3x$.
	\end{listEX}
	\loigiai{
		\begin{enumerate}
			\item $A=\dfrac{1}{2}\cos 14x+\dfrac{1}{2}\cos 8x-\dfrac{1}{2}\cos 26x-\dfrac{1}{2}\cos 8x=\dfrac{1}{2}\left(\cos 14x-\cos 26x\right)=\sin 15x\sin 12x$.
			\item $B=\dfrac{1}{2}\sin 21x+\dfrac{1}{2}\sin 15x-\dfrac{1}{2}\sin 23x-\dfrac{1}{2}\sin 15x=\dfrac{1}{2}\left(\sin 21x-\sin 23x\right)=-\cos 22x\sin x$.
			\item $C=-\dfrac{1}{2}\cos 4x+\dfrac{1}{2}\cos 2x-\dfrac{1}{2}\cos 12x+\dfrac{1}{2}\cos 4x=\dfrac{1}{2}\left(\cos 12x+\cos 2x\right)=\cos 7x\cos 5x$.
			\item $D=-\dfrac{1}{2}\cos 8x+\dfrac{1}{2}\cos 4x-\dfrac{1}{2}\cos 4x-\dfrac{1}{2}\cos 2x=-\dfrac{1}{2}\left(\cos 8x+\cos 2x\right)=-\cos 5x\cos 3x$.
			\item $E=\sin x\cdot\dfrac{1}{2}\left(\cos 2x+\dfrac{1}{2}\right)=\dfrac{1}{4}\sin 3x-\dfrac{1}{4}\sin x+\dfrac{1}{4}\sin x=\dfrac{1}{4}\sin 3x$.
			\item $F=\dfrac{1}{2}\cos 2x+\dfrac{1}{2}\cos x+\dfrac{1}{2}\cos 4x-\dfrac{1}{2}\cos 2x+\dfrac{1}{2}\cos 5x-\dfrac{1}{2}\cos x=\dfrac{1}{2}\left(\cos 4x-\cos 5x\right)=\sin \dfrac{9x}{2}\sin \dfrac{x}{2}$.
		\end{enumerate}
	}
\end{vd}
\begin{vd}%[DCHT Toán 11 - KNTT -Nguyễn Thành Nhân]%[1K1B2-3]
	Cho $\tan 3a=2023$. Tính giá trị biểu thức $P=\dfrac{\sin 2a-\sin 3a+\sin 4a}{\cos 2a-\cos 3a+\cos 4a}$.
	\loigiai{
		Ta có 
		\begin{eqnarray*}
			P&=&\dfrac{\sin 2a-\sin 3a+\sin 4a}{\cos 2a-\cos 3a+\cos 4a}\\
			&=&\dfrac{\left(\sin 4a+\sin 2a\right)-\sin 3a}{\left(\cos 4a+\cos 2a\right)-\cos 3a}\\
			&=& \dfrac{2\sin 3a\cdot \cos a-\sin 3a}{2\cos 3a\cdot \cos a-\cos 3a}\\
			&=& \dfrac{\sin 3a \cdot\left(2\cos a-1\right)}{\cos 3a \cdot \left(2\cos a-1\right)}=\tan 3a.
		\end{eqnarray*}
		Vậy $P=\tan 3a=2023$.
	}
\end{vd}
\begin{vd}%%[DCHT Toán 11 - KNTT -Nguyễn Thành Nhân]%[1K1B2-3]
	Rút gọn biểu thức $S=2\sin x\left(\cos x+\cos 3x+\cos 5x\right)$. Từ đó tính giá trị biểu thức $$P=\cos \dfrac{\pi}{7}+\cos \dfrac{3\pi}{7}+\cos \dfrac{5\pi}{7}.$$
	\loigiai{
		Ta có 
		\begin{eqnarray*}
			S&=&2\sin x\left(\cos x+\cos 3x+\cos 5x\right)\\
			&=& 2\sin x\cdot \cos x+2\sin x\cdot \cos 3x+2\sin x\cdot \cos 5x\\
			&=& \sin 2x+\sin 4x-\sin 2x+\sin 6x-\sin 4x\\
			&=& \sin 6x.
		\end{eqnarray*}
		Vậy $S=\sin 6x$.\\
		Áp dụng kết quả trên để tính $P=\cos \dfrac{\pi}{7}+\cos \dfrac{3\pi}{7}+\cos \dfrac{5\pi}{7}$.\\
		Vì $\sin \dfrac{\pi}{7}\ne 0$ nên
		\begin{eqnarray*}
			P&=&\cos \dfrac{\pi}{7}+\cos \dfrac{3\pi}{7}+\cos \dfrac{5\pi}{7}\\
			\Rightarrow  2P\cdot \sin \dfrac{\pi}{7}&=&2 \sin \dfrac{\pi}{7}\left(\cos \dfrac{\pi}{7}+\cos \dfrac{3\pi}{7}+\cos \dfrac{5\pi}{7}\right)\\
			&=& \sin \dfrac{6\pi}{7}=\sin \left(\pi-\dfrac{\pi}{7}\right)=\sin \dfrac{\pi}{7}\\
			\Rightarrow P&=& \dfrac{1}{2}.
		\end{eqnarray*}
		Vậy $P=\dfrac{1}{2}$.
	}
\end{vd}
\subsubsection{Bài tập rèn luyện}
\setcounter{bt}{0}
\begin{bt}%%[DCHT Toán 11 - KNTT -Nguyễn Thành Nhân]%[1K1B2-3]
	Cho biểu thức $A=\cos^2 a-\cos^2 3a-\sin 4a\cdot \sin 2a$. Chứng minh $A=0$.
	\loigiai{Ta có
		\begin{align*}
			\cos^2 a-\cos^2 3a=&(\cos a-\cos 3a)(\cos a+\cos 3a)\\
			=&-2\sin \dfrac{a+3a}{2}\sin\dfrac{a-3a}{2}\cdot 2\cos \dfrac{a+3a}{2}\cos\dfrac{a-3a}{2}\\
			=&-2\sin 2a\cdot\sin(-a)\cdot 2\cos 2a\cdot\cos (-a)\\
			=& 2\sin 2a\cdot\cos 2a\cdot 2\sin a\cos a\\
			=&\sin 4a\cdot \sin 2a. 
		\end{align*}
		Từ đó suy ra $A=0$.
	}
\end{bt}
\begin{bt}%%[DCHT Toán 11 - KNTT -Nguyễn Thành Nhân]%[1K1B2-3]
	Cho $\cos^2 x+\cos^2y=m$. Tính giá trị biểu thức $P=\cos(x+y)\cdot\cos(x-y)$.
	\loigiai{
		Ta có \begin{eqnarray*}
			P&=&\cos(x+y)\cdot\cos(x-y)\\
			&=&\dfrac12[\cos(x+y+x-y)+\cos (x+y-x+y)]\\
			&=&\dfrac12(\cos 2x+\cos 2y)\\
			&=&\dfrac 12(2\cos^2 x-1+2\cos^2y-1)\\
			&=&\dfrac 12(2\cos^2 x+2\cos^2y-2)\\
			&=&\cos^2x+\cos^2y-1=m-1.
		\end{eqnarray*}
	}
\end{bt}
\begin{bt}%%[DCHT Toán 11 - KNTT -Nguyễn Thành Nhân]%[1K1K2-3]
	Biểu thức $A=5+4\sin 2x\cos 2x$ nhận tất cả bao nhiêu giá trị nguyên?
	\loigiai{
		Ta có $A=5+4\sin 2x\cos 2x=5+2\sin 4x$.\\
		Mà $-1\le \sin 4x\le 1\Rightarrow -2\le 2\sin 4x\le 2\Rightarrow 3\le 5+2\sin 4x\le 7$\\
		$\Rightarrow 3\le A\le 7$ mà $A\in \mathbb{Z}\Rightarrow A\in \left\{3;4;5;6;7\right\}$ nên $A$ nhận tất cả $5$ giá trị nguyên.}
\end{bt}

\begin{bt}%%[DCHT Toán 11 - KNTT -Nguyễn Thành Nhân]%[1C1K2-3] 
	Chứng minh rằng với mọi tam giác $ABC$ ta luôn có
	$$\sin A+\sin B-\sin C=4\sin \dfrac{A}{2}\sin \dfrac{B}{2}\cos \dfrac{C}{2}.$$
	\loigiai{\\
		Ta có $\sin A+\sin B-\sin C=2\sin \dfrac{A+B}{2} \cos \dfrac{A-B}{2}-\sin \left(2\cdot \dfrac{C}{2}\right)$ \\
		Vì $\dfrac{A+B}{2}+\dfrac{C}{2}=90^\circ$ nên $\sin \dfrac{A+B}{2}=\cos \dfrac{C}{2}$ và $\sin \dfrac{C}{2}=\cos \dfrac{A+B}{2}$.\\
		Từ đó $\sin A+\sin B-\sin C=2\cos\dfrac{C}{2}\cos\dfrac{A-B}{2}-2\sin\dfrac{C}{2}\cos\dfrac{C}{2}$\\
		$=2\cos\dfrac{C}{2}\left[\cos\dfrac{A-B}{2}-\cos\dfrac{A+B}{2} \right]$\\
		$=2\cos \dfrac{C}{2}\cdot (-2)\sin \dfrac{A}{2} \sin \left(-\dfrac{B}{2}\right)=4\sin \dfrac{A}{2} \sin \dfrac{B}{2} \cos \dfrac{C}{2}$.\\
		Vậy $\sin A+\sin B-\sin C=4\sin \dfrac{A}{2}\sin \dfrac{B}{2}\cos \dfrac{C}{2}$.
	}
\end{bt}
\begin{bt}%%[DCHT Toán 11 - KNTT -Nguyễn Thành Nhân]%[1K1K2-3] 
	Chứng minh rằng với mọi tam giác nhọn $ABC$ ta luôn có
	$$\dfrac{\sin A+\sin B-\sin C}{\cos A+\cos B-\cos C+1}=\tan \dfrac{A}{2} \tan \dfrac{B}{2} \cot \dfrac{C}{2}.$$
	\loigiai{\\
		Ta có $\dfrac{\sin A+\sin B-\sin C}{\cos A+\cos B-\cos C+1}=\dfrac{2\sin \dfrac{A+B}{2} \cos \dfrac{A-B}{2}-\sin \left(2\cdot \dfrac{C}{2}\right)}{2\cos \dfrac{A+B}{2} \cos \dfrac{A-B}{2}+1-\cos \left(2 \dfrac{C}{2}\right)}$ \\
		$\dfrac{2\cos \dfrac{C}{2} \cos \dfrac{A-B}{2}-2\sin \dfrac{C}{2} \cos \dfrac{C}{2}}{2\sin \dfrac{C}{2} \cos \dfrac{A-B}{2}+2\sin^2\dfrac{C}{2}}=\dfrac{2\cos \dfrac{C}{2} \left(\cos \dfrac{A-B}{2}-\cos \dfrac{A+B}{2}\right)}{2\sin \dfrac{C}{2} \left(\cos \dfrac{A-B}{2}+\cos \dfrac{A+B}{2}\right)}$ \\
		$=\cot \dfrac{C}{2} \dfrac{-2\sin \dfrac{A}{2} \sin \left(-\dfrac{B}{2}\right)}{2\cos \dfrac{A}{2} \cos \left(-\dfrac{B}{2}\right)}=\cot \dfrac{C}{2} \tan \dfrac{A}{2} \tan \dfrac{B}{2}=\tan \dfrac{A}{2} \tan \dfrac{B}{2} \cot \dfrac{C}{2}$.\\
		Vậy $\dfrac{\sin A+\sin B-\sin C}{\cos A+\cos B-\cos C+1}=\tan \dfrac{A}{2} \tan \dfrac{B}{2} \cot \dfrac{C}{2}$.
	}
\end{bt}

\begin{bt}%%[DCHT Toán 11 - KNTT -Nguyễn Thành Nhân]%[1K1B2-3] 
	Chứng minh rằng đẳng thức $4\cos x\cos \left(\dfrac{\pi}{3}-x\right)\cos \left(\dfrac{\pi}{3}+x\right)=\cos 3x$, với mọi $x \in \mathbb{R}$.
	\loigiai{
		Ta có 
		\begin{eqnarray*}
			4\cos x\cos \left(\dfrac{\pi}{3}-x\right)\cos \left(\dfrac{\pi}{3}+x\right)&=&4\cos x\cdot \dfrac{1}{2}\left[\cos (-2x)+\cos \dfrac{2\pi}{3}\right]\\
			&=&2\cos x\cos 2x-\cos x\\
			&=&\cos 3x+\cos (-x)-\cos x\\
			&=&\cos 3x, \forall x \in \mathbb{R}.
		\end{eqnarray*}
	}        
\end{bt}
\begin{bt}%%[DCHT Toán 11 - KNTT -Nguyễn Thành Nhân]%[1K1B2-3] 
	Tính giá trị của biểu thức $S=\cos^2x+\cos^2\left(\dfrac{2\pi}{3}+x\right)+\cos^2\left(\dfrac{2\pi}{3}-x\right)$.
	\loigiai{
		Ta có
		\allowdisplaybreaks
		\begin{eqnarray*}
			S&=&\cos^2x+\cos^2\left(\dfrac{2\pi}{3}+x\right)+\cos^2\left(\dfrac{2\pi}{3}-x\right)\\
			&=&\dfrac{1+\cos 2x}{2}+\dfrac{1+\cos \left(\dfrac{4\pi}{3}+2x\right)}{2}+\dfrac{1+\cos \left(\dfrac{4\pi}{3}-2x\right)}{2}\\
			&=&\dfrac{3}{2}+\dfrac{1}{2}\cos 2x+\dfrac{1}{2}\left[\cos \left(\dfrac{4\pi}{3}+2x\right)+\cos \left(\dfrac{4\pi}{3}-2x\right)\right]\\
			&=&\dfrac{3}{2}+\dfrac{1}{2}\cos 2x+\dfrac{1}{2}\cdot 2\cos \dfrac{4\pi}{3}\cos 2x\\
			&=&\dfrac{3}{2}+\dfrac{1}{2}\cos 2x+\dfrac{1}{2}\cdot 2\cdot \left(-\dfrac{1}{2}\right)\cos 2x\\
			&=&\dfrac{3}{2}.
		\end{eqnarray*}
	}        
\end{bt}

\begin{bt}%%[DCHT Toán 11 - KNTT -Nguyễn Thành Nhân]%[1K1K2-3] 
	Chứng minh giá trị của biểu thức $$A=\cos \left(\dfrac{\pi}{3}-x\right)\cos \left(\dfrac{\pi}{4}+x\right)+\cos \left(\dfrac{\pi}{6}+x\right)\cos \left(\dfrac{3\pi}{4}+x\right)$$ không phụ thuộc vào giá trị của biến $x$.
	\loigiai{
		Ta có $\cos \left(\dfrac{\pi}{6}+x\right)=\sin\left(\dfrac{\pi}{2}-\dfrac{\pi}{6}-x\right)=\sin\left(\dfrac{\pi}{3}-x\right)$; $\cos \left(\dfrac{3\pi}{4}+x\right)=\sin\left(-\dfrac{\pi}{4}-x\right)=-\sin\left(\dfrac{\pi}{4}+x\right)$.\\
		Do đó $A=\cos \left(\dfrac{\pi}{3}-x\right)\cos \left(\dfrac{\pi}{4}+x\right)-\sin \left(\dfrac{\pi}{3}-x\right)\sin \left(\dfrac{\pi}{4}+x\right)=\cos\left(\dfrac{\pi}{3}+\dfrac{\pi}{4}\right)=\cos\dfrac{7\pi}{12}$.\\
		Vậy giá trị của biểu thức $A$ không phụ thuộc vào giá trị của biến $x$.
	}        
\end{bt}
\begin{bt}%%[DCHT Toán 11 - KNTT -Nguyễn Thành Nhân]%[1K1K2-3] 
	Chon $\sin 2x=m,\,(-1\le m\le 1)$. Tính theo $m$ giá trị của biểu thức $$S=\dfrac{1}{2}\left(\cos \left(\dfrac{\pi}{3}-2x\right)-\cos \left(\dfrac{\pi}{2}+2x\right)\right)-\sin \dfrac{\pi}{12} \cdot \cos \left(\dfrac{\pi}{12}+2x\right).$$
	\loigiai{
		Ta có
		\allowdisplaybreaks
		\begin{align*}
			& \dfrac{1}{2}\left(\cos \left(\dfrac{\pi}{3}-2x\right)-\cos \left(\dfrac{\pi}{2}+2x\right)\right)-\sin \dfrac{\pi}{12} \cdot \cos \left(\dfrac{\pi}{12}+2x\right) \\
			=&-\dfrac{1}{2}\left(\cos \left(\dfrac{\pi}{2}+2x\right)-\cos \left(\dfrac{\pi}{3}-2x\right)\right)-\sin \dfrac{\pi}{12} \cdot \cos \left(\dfrac{\pi}{12}+2x\right) \\
			=&\sin \dfrac{5\pi}{12}\sin \left(\dfrac{\pi}{12}+2x\right)-\sin \dfrac{\pi}{12} \cdot \cos \left(\dfrac{\pi}{12}+2x\right) \\
			=&\sin \left(\dfrac{\pi}{12}+2x\right)\cos \dfrac{\pi}{12}-\cos \left(\dfrac{\pi}{12}+2x\right)\sin \dfrac{\pi}{12}=\sin 2x.
	\end{align*}}
	Vậy $S=m$.
\end{bt}

\begin{bt}%%[DCHT Toán 11 - KNTT -Nguyễn Thành Nhân]%[1K1K2-3] 
	Cho $a$, $b$ thỏa mãn $\sin(2a+b)=3\sin b$. Chứng minh rằng $\tan(a+b)=2\tan a$.
	\loigiai{
		\allowdisplaybreaks
		\begin{eqnarray*}
			&&\sin(2a+b)=3\sin b \Leftrightarrow  \sin\left[(a+b)+a\right]=3\sin\left[(a+b)-a\right]\\
			&\Leftrightarrow & \sin(a+b)\cos a+\cos(a+b)\sin a=3\left[ \sin(a+b)\cos a -\cos(a+b)\sin a\right]\\
			&\Leftrightarrow & \sin(a+b)\cos a=2\cos(a+b)\sin a\\
			&\Leftrightarrow & \dfrac{\sin(a+b)\cos a}{\cos(a+b)\cos a}=2\cdot\dfrac{\cos(a+b)\sin a}{\cos(a+b)\cos a}\\
			&\Leftrightarrow & \tan(a+b)=2\tan a.
	\end{eqnarray*}}
\end{bt}

\begin{bt}%%[DCHT Toán 11 - KNTT -Nguyễn Thành Nhân]%[1K1K2-3] 
	Chứng minh trong tam giác $ABC$, ta luôn co $$\sin 2A+\sin 2B+\sin 2C = 4\sin A\sin B\sin C.$$ 
	\loigiai{
		Vì $A$, $B$, $C$ là ba góc trong $\triangle ABC$ nên $A+B+C=\pi$.\\
		$\begin{aligned}
			\sin 2A+\sin 2B+\sin 2C &=2\sin (A+B)\cos (A-B)+2\sin C\cos C\\
			&=2\sin C\cos (A-B)+2\sin C\cos C\\
			&=2\sin C\left[ \cos (A-B)+\cos C\right]\\
			&=4\sin C\cos \dfrac{A-B+C}{2}\cos \dfrac{A-B-C}{2}.
		\end{aligned}$\\
		Ta lại có
		\begin{align*}		
			\dfrac{A-B+C}{2}+B&=\dfrac{A+B+C}{2}=\dfrac{\pi}{2}\Rightarrow\dfrac{A-B+C}{2}=\dfrac{\pi}{2}-B\Rightarrow \cos \dfrac{A-B+C}{2}=\sin B. \\ 
			\dfrac{A-B-C}{2}-A&=-\dfrac{A+B+C}{2}=-\dfrac{\pi}{2}\Rightarrow \dfrac{A-B-C}{2}=-\dfrac{\pi}{2}+A\Rightarrow \cos \dfrac{A-B-C}{2}=\sin A.
		\end{align*}
		Vậy suy ra $\sin 2A+\sin 2B+\sin 2C = 4\sin A\sin B\sin C$.	
	}
\end{bt}

\begin{bt}%%[DCHT Toán 11 - KNTT -Nguyễn Thành Nhân]%[1K1K2-3] 
	Cho tam giác $ABC$ có ba góc $A$, $B$, $C$ thỏa mãn hệ thức $\sin A=\cos B+\cos C$. Chứng minh rằng tam giác $ABC$ là tam giác vuông.
	\loigiai{
		Ta có
		\allowdisplaybreaks
		\begin{eqnarray*}
			& &\sin A=\cos B+\cos C\\
			&\Leftrightarrow & 2\sin \dfrac{A}{2}\cos \dfrac{A}{2}=2\cos \dfrac{B+C}{2}\cos \dfrac{B-C}{2}\\
			&\Leftrightarrow & 2\sin \dfrac{A}{2}\cos \dfrac{A}{2}=2\cos \left(\dfrac{\pi}{2}-\dfrac{A}{2}\right)\cos \dfrac{B-C}{2} \\
			&\Leftrightarrow & \cos \dfrac{A}{2}=\cos \dfrac{B-C}{2}.
		\end{eqnarray*}
		Suy ra $\hoac{&\dfrac{A}{2}=\dfrac{B-C}{2} \\& \dfrac{A}{2}=-\dfrac{B-C}{2}}\Rightarrow \hoac{&B=A+C \\& C=A+B.}$\\
		Vậy tam  giác $ABC$ vuông tại $B$ hoặc tại $C$.}
\end{bt}

\begin{bt}%%[DCHT Toán 11 - KNTT -Nguyễn Thành Nhân]%[1K1K2-3] 
	Cho biểu thức $T=\cos 2x\cdot\cos x+\sin x\cdot\cos x\cdot \sin 3x-\sin^2x\cdot\cos 3x$. Gọi $S$ là tập các giá trị nguyên mà $T$ nhận. Tìm $S$.
	\loigiai{
		\allowdisplaybreaks
		\begin{eqnarray*}
			T&=&\cos x\left(\cos 2x-\sin x\cdot\sin 3x\right)-\sin^2x\cdot\cos 3x\\
			&=&\cos x\left(\cos 2x-\dfrac{1}{2}\left(\cos 2x-\cos 4x\right)\right)-\sin^2x\cdot\cos 3x\\
			&=&\cos x\cdot\dfrac{1}{2}\left(\cos 2x+\cos 4x\right)-\sin^2x\cdot\cos 3x\\
			&=&\left(\cos^2x-\sin^2x\right)\cos 3x\\
			&=&\cos 2x\cdot\cos 3x.
		\end{eqnarray*}
		Vì $-1\leq \cos 2x\leq 1, -1\leq \cos 3x\leq 1$ với $\forall x \in \mathbb{R}$ nên $S=\left\{-1;0;1\right\}$.}
\end{bt}
\begin{bt}%%[DCHT Toán 11 - KNTT -Nguyễn Thành Nhân]%[1K1G2-3] 
	Chứng minh đẳng thức 
	\[\dfrac{\sin a+\sin 3a+\sin 5a+\cdots+\sin (2n-1)a}{\cos a+\cos 3a+\cos5a+\cdots+\cos (2n-1)a}=\tan na.\]
	\loigiai{
		Biến đổi vế trái
		\begin{eqnarray*}
			VT&=& \dfrac{\sin a+\sin 3a+\sin 5a+\cdots+\sin (2n-1)a}{\cos a+\cos 3a+\cos+\cdots+\cos (2n-1)a}\\
			&=& \dfrac{\left[\sin a+\sin (2n-1)a\right]+\left[\sin 3a+\sin (2n-3)a\right]+\cdots+\left[\sin (n-1)a+\sin (n+1)a\right]}{\left[\cos a+\cos(2n-1)a\right]+\left[\cos 3a+\cos (2n-3)a\right]+\cdots+\left[\cos(n-1)a+\cos (n+1)a\right]}\\
			&=& \dfrac{2\sin na\cdot \cos (n-1)a+2\sin na\cdot \cos (n-2)a+\cdots+2\sin na\cdot \cos a}{2\cos na\cdot \cos (n-1)a+2\cos na\cdot \cos (n-2)a+\cdots+2\cos na\cdot \cos a}\\
			&=& \dfrac{2\sin na\left[\cos a+\cos 2a+\cdots+\cos (n-1)a\right]}{2\cos na\left[\cos a+\cos 2a+\cdots+\cos (n-1)a\right]}\\
			&=& \tan na\\
			&=&VP.
		\end{eqnarray*}
	}
\end{bt}

\begin{bt}%%[DCHT Toán 11 - KNTT -Nguyễn Thành Nhân]%[1K1G2-3] 
	Cho $a$, $b$ là các góc thỏa mãn $\heva{&\cos a+\cos b=m\\&\sin a+\sin b=n}$ với $m$, $n$ khác $0$. Tính $\sin (a+b)$.
	\loigiai{
		Nhân vế theo vế các phương trình trong giả thiết, ta được
		\begin{eqnarray*}
			m\cdot n&=& \left(\cos a+\cos b\right)\cdot \left(\sin a+\sin b\right)\\
			&=& \sin a\cdot \cos a+\sin b\cdot \cos b+\sin a\cdot \cos b+\cos a\cdot \sin b\\
			&=&\dfrac{1}{2}\cdot \sin 2a+\dfrac{1}{2}\cdot \sin 2b+\sin (a+b)\\
			&=& \dfrac{1}{2}\left(\sin 2a+\sin 2b\right)+\sin (a+b)\\
			&=&\sin (a+b)\cdot \cos (a-b)+\sin (a+b)\\
			&=& \sin (a+b)\left[\cos (a-b)+1\right].\quad (1)
		\end{eqnarray*}
		Lại có 
		\begin{eqnarray*}
			m^2+n^2&=& \left(\cos a+\cos b\right)^2+\left(\sin a+\sin b\right)^2\\
			&=& 2+2\left(\cos a\cdot \cos b+\sin a\cdot \sin b\right)\\
			&=& 2+2\cos (a-b)\\
			\Rightarrow \cos (a-b)&=&\dfrac{m^2+n^2-2}{2}.\quad (2)
		\end{eqnarray*}
		Thay $(2)$ vào $(1)$ ta được
		\begin{eqnarray*}
			\sin (a+b)=\dfrac{m\cdot n}{\dfrac{m^2+n^2-2}{2}+1}=\dfrac{2m\cdot n}{m^2+n^2}.
		\end{eqnarray*}
		Vậy $\sin (a+b)=\dfrac{2m\cdot n}{m^2+n^2}$.
	}
\end{bt}
\subsubsection{Bài tập trắc nghiệm}
\Opensolutionfile{ans}[ans/ans-1K1-2-Dang3]
\begin{ex}%[DCHT Toán 11 - KNTT -Nguyễn Thành Nhân]%[1K1Y2-2] 
	Đẳng thức nào sau đây \textbf{đúng}?
	\choice
	{$ \cos2x=1-2\cos^2x $}
	{\True $ \cos x \sin y=\dfrac{1}{2}\left[ \sin(x+y)-\sin(x-y)\right]  $}
	{$ \sin^2x=\dfrac{1-2\cos x}{2} $}
	{$ 1+\cot^2x=\dfrac{1}{\cos^2x}$ với $x\ne\dfrac{k\pi}{2},k\in\mathbb{Z} $}
	\loigiai{
		Theo công thức lượng giác.	
	}
\end{ex}
\begin{ex}%[DCHT Toán 11 - KNTT -Nguyễn Thành Nhân]%[1K1B2-2] 
	Cho $\sin\alpha=\dfrac{5}{13},\ 0<\alpha<\dfrac{\pi}{4}$. Giá trị của $\sin2\alpha$ bằng
	\choice
	{\True $\sin2\alpha=\dfrac{120}{169}$}
	{$\sin2\alpha=-\dfrac{120}{169}$}
	{$\sin2\alpha=\dfrac{60}{169}$}
	{$\sin2\alpha=-\dfrac{60}{169}$}
	\loigiai{
		Vì $0<\alpha<\dfrac{\pi}{4}$ nên $\cos\alpha>0$, suy ra $\cos\alpha=\sqrt{1-\sin^2\alpha}=\sqrt{1-\dfrac{25}{169}}=\dfrac{12}{13}$.\\
		Vậy $\sin2\alpha=2\sin\alpha\cos\alpha=2\cdot\dfrac{5}{13}\cdot\dfrac{12}{13}=\dfrac{120}{169}$.
	}
\end{ex}
\begin{ex}%[DCHT Toán 11 - KNTT -Nguyễn Thành Nhân]%[1K1B2-2] 
	Cho $\sin\alpha =\dfrac{3}{4}$. Khi đó $\cos 2\alpha$ bằng
	\choice
	{\True $\dfrac{1}{8}$}
	{$\dfrac{\sqrt{7}}{4}$}
	{$-\dfrac{1}{8}$}
	{$-\dfrac{\sqrt{7}}{4}$}
	\loigiai{
		Ta có $\cos 2\alpha=1-2\sin^2\alpha =1-2\cdot \dfrac{9}{16}=-\dfrac{1}{8}$.
	}
\end{ex}

\begin{ex}%[DCHT Toán 11 - KNTT -Nguyễn Thành Nhân]%[1K1B2-2] 
	Cho $\sin\alpha+\cos\alpha=\dfrac{5}{4}$. Khi đó $\sin 2\alpha$ có giá trị bằng
	\choice
	{$\dfrac{5}{2}$}
	{$2$}
	{$\dfrac{3}{32}$}
	{\True $\dfrac{9}{16}$}
	\loigiai{
		Từ giả thiết ta có
		$$ \left(\dfrac{5}{4}\right)^2=(\sin\alpha+\cos\alpha)^2=\sin^2\alpha+\cos^2\alpha+2\sin\alpha\cdot\cos\alpha=1+\sin 2\alpha. $$
		Vậy $\sin 2\alpha=\dfrac{25}{16}-1=\dfrac{9}{16}.$
	}
\end{ex}

\begin{ex}%[DCHT Toán 11 - KNTT -Nguyễn Thành Nhân]%[1K1B2-3] 
	Rút gọn biểu thức $P=\dfrac{\cos a-\cos 5a}{\sin 4a+\sin 2a}$ (với $\sin 4a+\sin 2a\ne 0$ ) ta được
	\choice
	{$P=2\cot a$}
	{$P=2\cos a$}
	{$P=2\tan a$}
	{\True $P=2\sin a$}
	\loigiai{
		Ta có
		\begin{align*}
			\cos a-\cos 5a&=2\sin 3a\sin 2a\\ 
			\sin 4a+\sin 2a&=2\sin 3a\cos a 
		\end{align*}
		Do đó $P=\dfrac{\sin 2a}{\cos a}=2\sin a$.
	}
\end{ex}
\begin{ex}%[DCHT Toán 11 - KNTT -Nguyễn Thành Nhân]%[1K1B2-2] 
	Cho $\cos x=-\dfrac{3}{5}$. Tính $\cos 2x$.
	\choice 
	{\True $\cos 2x=-\dfrac{7}{25}$}
	{ $\cos 2x=-\dfrac{3}{10}$}
	{ $\cos 2x=-\dfrac{8}{9}$}
	{ $\cos 2x=\dfrac{7}{25}$} 
	\loigiai{
		Ta có $\cos 2x=2\cos^2 x-1=-\dfrac{7}{25}$.	
	}
\end{ex} 

\begin{ex}%[DCHT Toán 11 - KNTT -Nguyễn Thành Nhân]%[1K1B2-2] 
	Cho $\sin 2 \alpha=-\dfrac{1}{2}$, thì $\tan ^2 \alpha+\cot ^2 \alpha$ có giá trị bằng	
	\choice
	{$18$}
	{$12$}
	{\True $14$}
	{$16$}
	\loigiai{
		Ta có \[\tan ^2 \alpha+\cot ^2 \alpha = (\tan \alpha+\cot \alpha)^2 - 2\tan \alpha\cot \alpha = \left(\dfrac{1}{\sin \alpha\cos \alpha}\right)^2 - 2 = \left(\dfrac{2}{\sin 2 \alpha}\right)^2 - 2 = 14.\]
	}
\end{ex}

\begin{ex}%[DCHT Toán 11 - KNTT -Nguyễn Thành Nhân]%[1K1B2-2] 
	Cho $\cot \alpha=15$ thì $\sin 2 \alpha$ bằng
	\choice
	{$\dfrac{11}{113}$}
	{\True $\dfrac{15}{113}$}
	{$\dfrac{17}{113}$}
	{$\dfrac{13}{113}$}
	\loigiai{
		Ta có \[\cot \alpha=15\Rightarrow \tan \alpha=\dfrac{1}{15} \Rightarrow \dfrac{226}{15} = \cot \alpha + \tan \alpha = \dfrac{1}{\sin \alpha \cos \alpha} = \dfrac{2}{\sin 2 \alpha} \Rightarrow \sin 2 \alpha = \dfrac{15}{113}.\]
	}
\end{ex}

\begin{ex}%[DCHT Toán 11 - KNTT -Nguyễn Thành Nhân]%[1K1K2-2] 
	Khi $\cos \alpha=\dfrac{3}{4}$ thì tích số $16 \cdot \sin \dfrac{\alpha}{2}\cdot \sin \dfrac{3 \alpha}{2}$ là một số nguyên. Số nguyên này bằng
	\choice
	{$6$}
	{$7$}
	{\True $5$}
	{$8$}
	\loigiai{
		Ta có $16 \cdot \sin \dfrac{\alpha}{2}\cdot \sin \dfrac{3 \alpha}{2} = 8\left(\cos \alpha - \cos 2\alpha\right)=8\left(\cos\alpha - 2\cos ^2\alpha + 1 \right) = 5$.
	}
\end{ex}

\begin{ex}%[DCHT Toán 11 - KNTT -Nguyễn Thành Nhân]%[1K1K2-2] 
	Tìm khẳng định \textbf{sai}.
	\choice
	{$\sin^4 x+\cos^4 x=\dfrac{3}{4}+\dfrac{1}{4}\cos 4x$}
	{\True $\sin^4 x+\cos^4 x=\dfrac{3}{4}-\dfrac{1}{4}\cos 4x$}
	{$\sin^4 x-\cos^4 x=-\cos 2x$}
	{$\sin^6 x+\cos^6 x=\dfrac{5}{8}+\dfrac{3}{8}\cos 4x$}
	\loigiai{
		Ta có \begin{align*}
			\sin^4x+\cos^4x=&(\sin^2x+\cos^2x)^2-2\sin^2x\cos^2x\\
			=&1-2\sin^2x\cos^2x=1-\dfrac{1}{2}\sin^2 2x\\
			=&1-\dfrac{1}{4}(1-\cos 4x)=1-\dfrac{1}{4}+\dfrac{1}{4}\cos 4x\\
			=&\dfrac{3}{4}+\dfrac{1}{4}\cos 4x.
		\end{align*}
		Suy ra đẳng thức $\sin^4 x+\cos^4 x=\dfrac{3}{4}-\dfrac{1}{4}\cos 4x$ là sai.	
	}
\end{ex}
\begin{ex}%[DCHT Toán 11 - KNTT -Nguyễn Thành Nhân]%[1K1B2-3] 
	Cho $\sin x\cdot\cos^5 x-\cos x\cdot\sin^5 x=\dfrac{1}{4}$. Khi đó $\cos 4x$ bằng
	\choice
	{$\dfrac{1}{2}$}
	{$-\dfrac{1}{2}$}
	{\True $0$}
	{$1$}
	\loigiai{
		Ta có \begin{align*}
			\sin x\cdot\cos^5 x-\cos x\cdot\sin^5 x=&\sin x\cdot\cos x\cdot(\cos^4 x-\sin^4x)\\
			=&\sin x\cdot\cos x\cdot(\cos^2 x-\sin^2x)(\cos^2x+\sin^2x)\\
			=&\sin x\cdot\cos x\cdot(\cos^2 x-\sin^2x)\\
			=&\dfrac{\sin 2x}{2}\cdot\cos 2x=\dfrac{\sin 4x}{4}.
		\end{align*}	
		Suy ra $\dfrac{\sin 4x}{4}=\dfrac 14\Rightarrow \sin 4x=1\Rightarrow\cos 4x=0$.
	}
\end{ex}

\begin{ex}%[DCHT Toán 11 - KNTT -Nguyễn Thành Nhân]% [1K1B2-3] 
	Cho $\cos a=\dfrac{3}{5},\cos b=\dfrac{2}{5}$. Tính $M=\cos(a+b)\cdot\cos(a-b)$.
	\choice
	{\True $M=-\dfrac{12}{25}$}
	{$M=\dfrac{12}{25}$}
	{$M=-\dfrac{13}{25}$}
	{$M=\dfrac{13}{25}$}
	\loigiai{
		Dùng công thức biến đổi tích thành tổng \begin{align*}
			M=&\dfrac{1}{2}\left[\cos(a+b+a-b)+\cos(a+b-a+b) \right]\\
			=&\dfrac{1}{2}(\cos 2a+\cos 2b)\\
			=&\dfrac{1}{2}(2\cos^2a-1+2\cos^2b-1)\\
			=&\dfrac{1}{2}\left(2\cdot\dfrac{9}{25}-1+2\cdot\dfrac{4}{25}-1 \right) \\
			=&-\dfrac{12}{25}.
		\end{align*}
	}
\end{ex}

\begin{ex}%[DCHT Toán 11 - KNTT -Nguyễn Thành Nhân]%[1K1B2-2] 
	Cho $\sin\alpha=m$. Tính \begin{center}
		$P=\cos\left( \dfrac{\pi}{2}-\alpha \right) \sin(\pi-\alpha)-\sin\left( \dfrac{\pi}{2}-\alpha \right) \cos(\pi-\alpha)+\sin^2(\alpha+2018\pi).$
	\end{center}
	\choice
	{$P=m^2+2$}
	{$P=m^2-2$}
	{\True $P=m^2+1$}
	{$P=m+1$}
	\loigiai{
		Ta có $P=\sin\alpha\cdot\sin\alpha-\cos\alpha\cdot(-\cos\alpha)+\sin^2\alpha=\sin^2\alpha+\cos^2\alpha+\sin^2\alpha=1+\sin^2\alpha=1+m^2$.
	}
\end{ex}

\begin{ex}%[DCHT Toán 11 - KNTT -Nguyễn Thành Nhân]%[1K1B2-2] 
	Cho $\tan\dfrac{\alpha}{2}=2$. Giá trị của biểu thức $P=\dfrac{1+2020\sin\alpha}{1-2015\sin\alpha}$ là
	\choice
	{$\dfrac{1616}{1612}$}
	{\True $-\dfrac{1617}{1611}$}
	{$-\dfrac{1615}{1611}$}
	{$-\dfrac{1616}{1612}$}
	\loigiai{
		Ta có $P=\dfrac{1+2020\cdot2\sin\dfrac{\alpha}{2}\cos\dfrac{\alpha}{2}}{1-2015\cdot2\sin\dfrac{\alpha}{2}\cos\dfrac{\alpha}{2}}$. Chia cả tử và mẫu cho $\cos^2\dfrac{\alpha}{2}$, ta được
		\begin{align*}
			P=&\dfrac{\dfrac{1}{\cos^2\dfrac{\alpha}{2}}+4040\tan\dfrac{\alpha}{2}}{\dfrac{1}{\cos^2\dfrac{\alpha}{2}}-4030\tan\dfrac{\alpha}{2}}
			=\dfrac{1+\tan^2\dfrac{\alpha}{2}+4040\tan\dfrac{\alpha}{2}}{1+\tan^2\dfrac{\alpha}{2}-4030\tan\dfrac{\alpha}{2}}\\
			=&\dfrac{1+2^2+4040\cdot2}{1+2^2-4030\cdot 2}=-\dfrac{8085}{8055}=-\dfrac{1617}{1611}.
		\end{align*}	
	}
\end{ex}
\begin{ex}%[DCHT Toán 11 - KNTT -Nguyễn Thành Nhân]%[1K1K2-2] 
	Cho $\sin x\cdot\cos^5 x-\cos x\cdot\sin^5 x=\dfrac{1}{4}$. Khi đó $\cos 4x$ bằng
	\choice
	{$\dfrac{1}{2}$}
	{$-\dfrac{1}{2}$}
	{\True $0$}
	{$1$}
	\loigiai{
		Ta có \begin{align*}
			\sin x\cdot\cos^5 x-\cos x\cdot\sin^5 x=&\sin x\cdot\cos x\cdot(\cos^4 x-\sin^4x)\\
			=&\sin x\cdot\cos x\cdot(\cos^2 x-\sin^2x)(\cos^2x+\sin^2x)\\
			=&\sin x\cdot\cos x\cdot(\cos^2 x-\sin^2x)\\
			=&\dfrac{\sin 2x}{2}\cdot\cos 2x=\dfrac{\sin 4x}{4}.
		\end{align*}	
		Suy ra $\dfrac{\sin 4x}{4}=\dfrac 14\Rightarrow \sin 4x=1\Rightarrow\cos 4x=0$.
	}
\end{ex}

\begin{ex}%[DCHT Toán 11 - KNTT -Nguyễn Thành Nhân]%[1K1K2-2] 
	Cho $\cos a=\dfrac{3}{5},\cos b=\dfrac{2}{5}$. Tính $M=\cos(a+b)\cdot\cos(a-b)$.
	\choice
	{\True $M=-\dfrac{12}{25}$}
	{$M=\dfrac{12}{25}$}
	{$M=-\dfrac{13}{25}$}
	{$M=\dfrac{13}{25}$}
	\loigiai{
		Dùng công thức biến đổi tích thành tổng \begin{align*}
			M=&\dfrac{1}{2}\left[\cos(a+b+a-b)+\cos(a+b-a+b) \right]\\
			=&\dfrac{1}{2}(\cos 2a+\cos 2b)\\
			=&\dfrac{1}{2}(2\cos^2a-1+2\cos^2b-1)\\
			=&\dfrac{1}{2}\left(2\cdot\dfrac{9}{25}-1+2\cdot\dfrac{4}{25}-1 \right) \\
			=&-\dfrac{12}{25}.
		\end{align*}
	}
\end{ex}
\begin{ex}%[DCHT Toán 11 - KNTT -Nguyễn Thành Nhân]%[1K1B2-3] 
	Giá trị của biểu thức $I=\dfrac{\cos5x+\cos3x}{\sin5x-\sin3x}$, biết $\tan x=\dfrac{1}{3}$ là
	\choice
	{$I=\dfrac{1}{3}$}
	{$I=-\dfrac{1}{3}$}
	{\True $I=3$}
	{$I=-3$}
	\loigiai{
		Ta có
		$$I=\dfrac{\cos5x+\cos3x}{\sin5x-\sin3x}=\dfrac{2\cos4x\cos x}{2\cos4x\sin x}=\dfrac{1}{\tan x}=3.$$
	}
\end{ex}
\begin{ex}%[DCHT Toán 11 - KNTT -Nguyễn Thành Nhân]%[1K1K2-2] 
	Cho góc $\alpha$ thỏa mãn $\dfrac{\pi}{2}<\alpha<\pi$. Biết $\sin\alpha +2\cos\alpha=-1$. Tính giá trị $\sin 2\alpha$.
	\choice
	{$\dfrac{2\sqrt{6}}{5}$}
	{$\dfrac{24}{25}$}
	{$-\dfrac{2\sqrt{6}}{5}$}
	{\True $-\dfrac{24}{25}$}
	\loigiai{
		Xét hệ phương trình $\heva{&\sin\alpha+2\cos\alpha=-1\\&\sin^2\alpha+\cos^2\alpha=1}\Leftrightarrow\heva{&\sin\alpha=-1-2\cos\alpha\\&5\cos^2\alpha+4\cos\alpha=0}\Leftrightarrow\hoac{&\cos\alpha=0\\&\cos\alpha=-\dfrac{4}{5}.}$\\
		Vì $\dfrac{\pi}{2}<\alpha<\pi\Rightarrow \cos\alpha=-\dfrac{4}{5}\Rightarrow\sin\alpha =\dfrac{3}{5}$ suy ra $\sin2\alpha =2\sin\alpha\cdot\cos\alpha=-\dfrac{24}{25}$.}
\end{ex}
\begin{ex}%[DCHT Toán 11 - KNTT -Nguyễn Thành Nhân]%[1K1B2-2] 
	Cho $\cos\left( \dfrac{\pi}{2}+x\right) =-\dfrac{1}{5}$ với $  2\pi<x<\dfrac{5\pi}{2} $. Giá trị của $ \sin2x $ bằng
	\choice
	{\True $ \dfrac{4\sqrt{6}}{25} $}
	{$ \dfrac{2\sqrt{6}}{5} $}
	{$ -\dfrac{4\sqrt{6}}{25} $}
	{$ -\dfrac{2\sqrt{6}}{5} $}
	\loigiai{
		Ta có $ \cos\left( \dfrac{\pi}{2}+x\right) =-\dfrac{1}{5}\Rightarrow \sin x=\dfrac{1}{5}\Rightarrow \cos x=\pm \sqrt{1-\sin^2x}=\pm \sqrt{1-\left(\dfrac{1}{5}\right)^2}=\pm \dfrac{2\sqrt{6}}{5} $.\\
		Vì $ 2\pi<x<\dfrac{5\pi}{2} $ nên $ \cos x= \dfrac{2\sqrt{6}}{5}$. Do đó $ \sin 2x=2\sin x \cos x=2\cdot \dfrac{1}{5}\cdot\dfrac{2\sqrt{6}}{5}=\dfrac{4\sqrt{6}}{25} $.
	}
\end{ex}
\begin{ex}%[DCHT Toán 11 - KNTT -Nguyễn Thành Nhân]%[1K1K2-2] 
	Nếu biết $\sin\alpha=\dfrac{5}{13}\left(\dfrac{\pi}{2}<\alpha<\pi\right)$, $\cos\beta=\dfrac{3}{5}\left(0<\beta<\dfrac{\pi}{2}\right)$ thì giá trị đúng của $\cos(\alpha-\beta)$ là
	\choice
	{$\dfrac{16}{65}$}
	{$-\dfrac{18}{65}$}
	{\True $-\dfrac{16}{65}$}
	{$\dfrac{56}{65}$}
	\loigiai{
		Có $\sin\alpha=\dfrac{5}{13}\Rightarrow \cos\alpha =\sqrt{1-\sin^2\alpha}=\pm \dfrac{12}{13}$. Vì $\dfrac{\pi}{2}<\alpha<\pi$ nên $\cos \alpha =-\dfrac{12}{13}$.\\
		Tương tự có $\sin \beta =\pm \dfrac{4}{5}$. Vì $0<\beta<\dfrac{\pi}{2}$ nên $\sin \beta =\dfrac{4}{5}$.\\
		Có $\cos (\alpha -\beta)=\cos \alpha\cos\beta +\sin \alpha \sin\beta=-\dfrac{12}{13}\cdot \dfrac{3}{5}+\dfrac{5}{13}\cdot \dfrac{4}{5}=-\dfrac{16}{65}$.
	}
\end{ex}
\begin{ex}%[DCHT Toán 11 - KNTT -Nguyễn Thành Nhân]%[1K1B2-2] 
	Nếu $\tan\alpha+\cot\alpha=2$ $\left(0<\alpha<\dfrac{\pi}{2}\right)$ thì $\sin 2\alpha$ bằng	
	\choice
	{$\dfrac{\pi}{2}$}
	{\True $1$}
	{$-\dfrac{1}{3}$}
	{$\dfrac{\sqrt{2}}{2}$}
	\loigiai{
		Ta có $\tan\alpha+\cot\alpha=2\Leftrightarrow\tan^2\alpha-2\tan\alpha+1=0\Leftrightarrow\tan\alpha=1\Rightarrow\heva{& \sin\alpha=\dfrac{\sqrt{2}}{2} \\ & \cos\alpha=\dfrac{\sqrt{2}}{2}.}$\\
		Suy ra $\sin 2\alpha=2\sin\alpha\cdot\cos\alpha=2\cdot\dfrac{\sqrt{2}}{2}\cdot\dfrac{\sqrt{2}}{2}=1$.
	}
\end{ex}
\begin{ex}%[DCHT Toán 11 - KNTT -Nguyễn Thành Nhân]% [1K1B2-2] 
	Cho $\cos a=\dfrac{3}{5},\cos b=\dfrac{2}{5}$. Tính $M=\cos(a+b)\cdot\cos(a-b)$.
	\choice
	{\True $M=-\dfrac{12}{25}$}
	{$M=\dfrac{12}{25}$}
	{$M=-\dfrac{13}{25}$}
	{$M=\dfrac{13}{25}$}
	\loigiai{
		Dùng công thức biến đổi tích thành tổng \begin{align*}
			M=&\dfrac{1}{2}\left[\cos(a+b+a-b)+\cos(a+b-a+b) \right]\\
			=&\dfrac{1}{2}(\cos 2a+\cos 2b)\\
			=&\dfrac{1}{2}(2\cos^2a-1+2\cos^2b-1)\\
			=&\dfrac{1}{2}\left(2\cdot\dfrac{9}{25}-1+2\cdot\dfrac{4}{25}-1 \right) \\
			=&-\dfrac{12}{25}.
		\end{align*}
	}
\end{ex}
\begin{ex}%[DCHT Toán 11 - KNTT -Nguyễn Thành Nhân]%[1K1G2-2] 
	Cho hai góc nhọn $ x $ và $ y $ thỏa mãn $ \heva{&{3\sin2x-\sin2y=0} \\&{6\cos^2x-2\sin^2y=5} } $. Khi đó số đo góc $ 2x+y $ gần bằng giá trị nào nhất trong các giá trị sau
	\choice
	{\True $ 60^\circ $}
	{$ 90^\circ $}
	{$ 75^\circ $}
	{$ 180^\circ $}
	\loigiai{
		$$ \heva{&{3\sin2x-\sin2y=0} \\&{6\cos^2x-2\sin^2y=5} }\Leftrightarrow \heva{&{3\sin2x=\sin2y} \\&{3\cos2x +\cos2y=3.} } $$
		Ta có $ (3\sin2x)^2+(3\cos2x)^2 =9$.\\Suy ra $\sin^22y+(3-\cos2y)^2=9 \Leftrightarrow \cos2y=\dfrac{1}{6}\Leftrightarrow 2y\approx 80{,}4^\circ \Rightarrow y\approx 40{,}2^\circ$.\\
		Do $ \cos2y=\dfrac{1}{6} $ nên ta có $ 3\cos2x=\dfrac{17}{18}\Rightarrow 2x\approx 19{,}2^\circ $.\\
		Vậy $ 2x+y\approx 59{,}4^\circ $.
	}
\end{ex}

\begin{ex}%[DCHT Toán 11 - KNTT -Nguyễn Thành Nhân]%[1K1G2-2] 
	Nếu $\sin x+\cos x=\dfrac{1}{2}$ và $0<x<\pi$ thì $\tan x=-\dfrac{a+\sqrt{b}}{3},\ (a; b \in \mathbb{Z})$. Tính $S=a+b$ 
	\choice
	{$S=3$}
	{$S=-11$}
	{\True $S=11$}
	{$S=-3$}
	\loigiai{
		Từ giả thiết ta có $$\dfrac{1}{4}=(\sin x+\cos x)^2=1+2\sin x\cos x,$$ suy ra $\sin x\cos x=-\dfrac{3}{8}$. Do đó 
		$$\tan x+\cot x=\dfrac{1}{\sin x\cos x}=-\dfrac{8}{3},$$ hay $$3\tan^2x+8\tan x+3=0\Leftrightarrow \tan x=\dfrac{-4\pm \sqrt{7}}{3}.$$
		Từ đó, ta có $\tan x=-\dfrac{4+\sqrt{7}}{3}$. 
		Suy ra $S=a+b=4+7=11.$	
	}
\end{ex}

\begin{ex}%[DCHT Toán 11 - KNTT -Nguyễn Thành Nhân][1K1G2-2] 
	Biết rằng $\tan\alpha$, $\tan\beta$ là các nghiệm của phương trình $x^2-p x+q=0$. Giá trị của biểu thức $A=\cos^2(\alpha+\beta)+p\sin(\alpha+\beta)\cdot\cos(\alpha+\beta)+q\sin^2(\alpha+\beta)$ bằng 
	\choice
	{$q$}
	{$p$}
	{$\dfrac{p}{q}$}
	{\True $1$}
	\loigiai{
		Vì $\tan\alpha$, $\tan\beta$ là các nghiệm của phương trình $x^2-p x+q=0$ nên theo hệ thức Vi-ét ta có $\heva{&\tan \alpha +\tan \beta =p\\&\tan \alpha \cdot \tan \beta =q.}$\\
		Ta có
		\begin{eqnarray*}
			A& =& \cos^2(\alpha+\beta)+p\sin(\alpha+\beta)\cdot\cos(\alpha+\beta)+q\sin^2(\alpha+\beta)\\
			&=& \cos^2(\alpha+\beta)+(\tan \alpha +\tan \beta)\cdot\sin(\alpha+\beta)\cdot\cos(\alpha+\beta)+(\tan \alpha \cdot \tan \beta)\cdot\sin^2(\alpha+\beta)\\
			&=& \cos^2(\alpha+\beta)+\dfrac{\sin^2 (\alpha+\beta)}{\cos\alpha \cdot \cos \beta}\cdot(\cos\alpha \cos\beta-\sin\alpha \sin\beta)+\sin^2(\alpha+\beta)\cdot(\tan \alpha \cdot \tan \beta)\\
			&=&\cos^2(\alpha+\beta)+\sin^2(\alpha+\beta)\cdot\left( 1-\tan \alpha \cdot \tan \beta+\tan \alpha \cdot \tan \beta \right)\\
			&=&\cos^2(\alpha+\beta)+\sin^2(\alpha+\beta)=1.
		\end{eqnarray*}
	}
\end{ex}
\Closesolutionfile{ans}