\setcounter{deso}{0}
\begin{name}
	{\tenchude}
	{ĐỀ ÔN TẬP CHƯƠNG I}
	{LỚP TOÁN THẦY PHÁT}
	{\thoigian}
\end{name}
\TN
%Câu 1
\begin{ex}
	Cho góc lượng giác $\alpha $. Mệnh đề nào sau đây đúng?
	\choice
	{$\sin \left(-\alpha\right)=\sin \alpha $}
	{$\cos \left(-\alpha\right)=-\cos \alpha $}
	{$\tan \left(-\alpha\right)=\tan \alpha $}
	{\True $\cot \left(-\alpha\right)=-\cot \alpha $}
	\loigiai{
		Dựa vào tính chất của hai góc đối nhau nên $\cot \left(-\alpha\right)=-\cot \alpha $
	}
\end{ex}
%Câu 2
\begin{ex}
	Giá trị $\cos 75^\circ $ là :
	\choice
	{$\dfrac{\sqrt{6}+\sqrt{2}}{4}$}
	{$\dfrac{\sqrt{6}-\sqrt{2}}{2}$}
	{\True $\dfrac{\sqrt{6}-\sqrt{2}}{4}$}
	{$\dfrac{\sqrt{6}+\sqrt{2}}{2}$}
	\loigiai{
		Ta có $\cos 75^\circ =\cos \left(30^\circ+45^\circ \right)=\cos 30^\circ \cos 45^\circ -\sin 30^\circ \sin 45^\circ=\dfrac{\sqrt{6}-\sqrt{2}}{4}$
	}
\end{ex}
%Câu 3
\begin{ex}
	Cho $\sin \alpha =\dfrac{5}{13}$ với $\dfrac{\pi }{2}<\alpha <\pi $. Mệnh đề nào sau đây đúng?
	\choice
	{$\cos \alpha =\dfrac{12}{13}$}
	{$\cos \alpha =\dfrac{8}{13}$}
	{$\cos \alpha =-\dfrac{8}{13}$}
	{\True $\cos \alpha =-\dfrac{12}{13}$}
	\loigiai{
	Ta có $\cos \alpha =\pm \sqrt{1-{{\sin }^2}\alpha }=\pm \dfrac{12}{13}$. Do $\dfrac{\pi }{2}<\alpha <\pi $ nên $\cos \alpha =-\dfrac{12}{13}$
	}
\end{ex}
%Câu 4
\begin{ex}
	Cho các góc $\alpha $, $\beta $ thỏa mãn $\alpha ,\beta \in \left(\dfrac{\pi }{2};\pi\right)$ và $\sin \alpha =\dfrac{1}{3}$, $\cos \beta =-\dfrac{2}{3}$. Tính $\sin \left(\alpha +\beta\right)$.
	\choice
	{\True $\sin \left(\alpha +\beta\right)=-\dfrac{2+2\sqrt{10}}{9}$}
	{$\sin \left(\alpha +\beta\right)=\dfrac{2\sqrt{10}-2}{9}$}
	{$\sin \left(\alpha +\beta\right)=\dfrac{\sqrt{5}-4\sqrt{2}}{9}$}
	{$\sin \left(\alpha +\beta\right)=\dfrac{\sqrt{5}+4\sqrt{2}}{9}$}
	\loigiai{
	Do $\alpha ,\beta \in \left(\dfrac{\pi }{2};\pi\right)$ nên có: $\heva{& \cos \alpha <0 \\& \sin \beta >0}$.\\
	Ta có $\cos \alpha =-\ \sqrt{1-{{\sin }^2}\alpha }=-\ \sqrt{1-\dfrac{1}{9}}=-\ \dfrac{2\sqrt{2}}{3}$ và $\sin \beta =\sqrt{1-{{\cos }^2}\beta }=\sqrt{1-\dfrac{4}{9}}=\dfrac{\sqrt{5}}{3}$.\\
	Suy ra $\sin \left(\alpha +\beta\right)=\sin \alpha \cdot \cos \beta +\cos \alpha \cdot \sin \beta =\dfrac{1}{3} \cdot \left(-\dfrac{2}{3}\right)+\left(-\dfrac{2\sqrt{2}}{3}\right) \cdot \dfrac{\sqrt{5}}{3}=-\ \dfrac{2+2\sqrt{10}}{9}$.\\
	Vậy $\sin \left(\alpha +\beta\right)=-\ \dfrac{2+2\sqrt{10}}{9}$
	}
\end{ex}
%Câu 5
\begin{ex}
	Biết $\sin \alpha +\text{cos}\alpha =m$. Tính $P=\text{cos}\left(\alpha -\dfrac{\pi }{4}\right)$ theo $m$.
	\choice
	{$P=2m$}
	{$P=\dfrac{m}{2}$}
	{\True $P=\dfrac{m}{\sqrt{2}}$}
	{$P=m\sqrt{2}$}
	\loigiai{
	Ta có $P=\text{cos}\left(\alpha -\dfrac{\pi }{4}\right)=\text{cos}\alpha \cdot \cos \dfrac{\pi }{4}+\sin \alpha \sin \dfrac{\pi }{4}=\dfrac{1}{\sqrt{2}}\text{cos}\alpha +\dfrac{1}{\sqrt{2}}\sin \alpha $\\
	$\Rightarrow P=\dfrac{1}{\sqrt{2}}\left(\sin \alpha +\text{cos}\alpha\right)=\dfrac{m}{\sqrt{2}}$
	}
\end{ex}
%Câu 6
\begin{ex}
	Cho $x=\tan \alpha $. Tính $\sin 2\alpha $ theo $x$.
	\choice
	{$2x\sqrt{1+x^2}$}
	{$\dfrac{1-x^2}{1+x^2}$}
	{$\dfrac{2x}{1-x^2}$}
	{\True $\dfrac{2x}{1+x^2}$}
	\loigiai{
	Ta có $\sin 2\alpha =2\sin \alpha \cdot \cos\alpha =2\dfrac{\sin \alpha }{\cos\alpha }\cdot \cos^2\alpha =2\tan \alpha \cdot \dfrac{1}{1+{{\tan }^2}\alpha }=\dfrac{2x}{1+x^2}$
	}
\end{ex}
%Câu 7
\begin{ex}
	Tập xác định của hàm số $y=\cot x$ là
	\choice
	{$D=\mathbb{R}$}
	{$D=\mathbb{R}\backslash \left\{ k\dfrac{\pi }{2}\left| k\in \mathbb{Z} \right. \right\}$}
	{$D=\mathbb{R}\backslash \left\{ \pi +k\dfrac{\pi }{2}\left| k\in \mathbb{Z} \right. \right\}$}
	{\True $D=\mathbb{R}\backslash \left\{ k\pi \left| k\in \mathbb{Z} \right. \right\}$}
	\loigiai{
		Điều kiện: $\sin x\ne 0\Leftrightarrow x\ne k\pi \left(k\in \mathbb{Z}\right)$.\\
		Do đó, tập xác định của hàm số $y=\cot x$ là $D=\mathbb{R}\backslash \left\{ k\pi \left| k\in \mathbb{Z} \right. \right\}$
	}
\end{ex}
%Câu 8
\begin{ex}
	Trên khoảng $\left(-\pi ;\pi\right)$, hàm số $y=\sin x$ nghịch biến trên khoảng nào sau đây?
	\choice
	{$\left(-\pi ;0\right)$}
	{$\left(-\dfrac{\pi }{2};\dfrac{\pi }{2}\right)$}
	{$\left(0;\pi\right)$}
	{\True $\left(\dfrac{\pi }{2};\pi\right)$}
	\loigiai{
		Hàm số $y=\sin x$ nghịch biến trong khoảng $\left(\dfrac{\pi }{2};\pi\right)$
	}
\end{ex}
%Câu 9
\begin{ex}
	Hàm số $y={{\sin }^2}2x-{{\cos }^2}2x$ tuần hoàn với chu kỳ bằng
	\choice
	{$2\pi $}
	{$\pi $}
	{\True $\dfrac{\pi }{2}$}
	{$\dfrac{\pi }{4}$}
	\loigiai{
	Ta có $y={{\sin }^2}2x-{{\cos }^2}2x=-\cos 4x$. Vậy hàm số đã cho tuần hoàn với chu kỳ $\dfrac{2\pi }{4}=\dfrac{\pi }{2}$
	}
\end{ex}
%Câu 10
\begin{ex}
	Nghiệm của phương trình $2\sin x+1=0$ là
	\choice
	{$x=\dfrac{\pm \pi }{6}+k2\pi ,k\in \mathbb{Z}$}
	{$x=\dfrac{\pi }{6}+k2\pi ,k\in \mathbb{Z}$}
	{$x=\dfrac{7\pi }{6}+k2\pi ,k\in \mathbb{Z}$}
	{\True $\hoac{& x=\dfrac{-\pi }{6}+k2\pi \\& x=\dfrac{7\pi }{6}+k2\pi},k\in \mathbb{Z}$}
	\loigiai{
		Ta có: $2\sin x+1=0\Leftrightarrow \sin x=\dfrac{-1}{2}\Leftrightarrow \hoac{& x=\dfrac{-\pi }{6}+k2\pi \\& x=\dfrac{7\pi }{6}+k2\pi},k\in \mathbb{Z}$
	}
\end{ex}
%Câu 11
\begin{ex}
	Phương trình nào dưới đây vô nghiệm.
	\choice
	{$\cos x=\dfrac{1}{2}$}
	{\True $\sin x-\cos x=2$}
	{$\sin (5x+1)=1$}
	{$\sin x+\sqrt{3}\cos x=1$}
	\loigiai{
		Chú ý\\
		- $\left| \sin \alpha \right|\le 1,\forall \alpha \in \mathbb{R}$ và $\left| \cos \alpha \right|\le 1,\forall \alpha \in \mathbb{R}$ nên các phương trình ở đáp án A, C có nghiệm.\\
		- Phương trình $a\sin x+b\cos x=c$ có nghiệm khi $a^2+b^2\ge c^2$, ta kiểm tra được phương trình đáp án B vô nghiệm, đáp án D có nghiệm
	}
\end{ex}
%Câu 12
\begin{ex}
	Cho phương trình $2\tan x-3=\dfrac{-2}{\tan x+1}$. Gọi $S$ là tập hợp các nghiệm của phương trình thuộc khoảng $\left(0;\dfrac{\pi }{2}\right)$. Tổng các phần tử của $S$ là
	\choice
	{$0$}
	{$\dfrac{\pi }{3}$}
	{\True $\dfrac{\pi }{4}$}
	{$1$}
	\loigiai{
		Điều kiện : $\cos x\ne 0,\tan x\ne -1$.\\
		Vì $x\in \left(0;\dfrac{\pi }{2}\right)\Rightarrow \tan x>0$.\\
		Phương trình ban đầu tương đương\\
		$\begin{aligned}
				& \Leftrightarrow \left(2\tan x-3\right)\left(\tan x+1\right)=-2\Leftrightarrow 2{{\tan }^2}x-\tan x-3=-2 \\& \Leftrightarrow 2{{\tan }^2}x-\tan x-1=0 \end{aligned}$\\
		$\Leftrightarrow \hoac{& \tan x=1\begin{matrix}\\
					{} & (TM) \\\\
				\end{matrix} \\& \tan x=\dfrac{-1}{2}(L)}$\\
		+ Với $\tan x=1\Leftrightarrow x=\dfrac{\pi }{4}+k\pi ,k\in \mathbb{Z}$. Vì $x\in \left(0;\dfrac{\pi }{2}\right)$ nên $x=\dfrac{\pi }{4}$.\\
		Vậy $S=\left\{ \dfrac{\pi }{4} \right\}$ và tổng các phần tử của $S$ là $\dfrac{\pi }{4}$
	}
\end{ex}
\TNTF
%Câu 13
\begin{ex}
	Xét tính đúng sai của các mệnh đề sau:
	\choiceTF
	{${{\sin }^2}x=\dfrac{1+\sin 2x}{2}$}
	{\True Nếu $\cos \alpha =\dfrac{1}{3}$ thì $\cos 2\alpha =-\dfrac{7}{9}$}
	{\True Nếu $\sin x=\dfrac{3}{4}$ với $x\in \left(0;\dfrac{\pi }{2}\right)$ thì $\sin 2x=\dfrac{3\sqrt{7}}{8}$}
	{\True Cho $\cos \alpha =\dfrac{2}{3}$ với $\alpha \in \left(-\dfrac{\pi }{2};0\right)$ biết $\tan \left(\alpha +\dfrac{\pi }{4}\right)=a+b\sqrt{c}$, $c$ là số nguyên tố $\left(a,b,c\in \mathbb{Z},c\ge 0\right)$ Khi đó $a+b+c=0$}
	\loigiai{
	a) ${{\sin }^2}x=\dfrac{1-\cos 2x}{2}$\\
	b) $\cos 2\alpha =2{{\cos }^2}\alpha -1=2{{\left(\dfrac{1}{3}\right)}^2}-1=\dfrac{-7}{9}$\\
	c) Ta có ${{\cos }^2}x=1-{{\sin }^2}x=1-{{\left(\dfrac{3}{4}\right)}^2}=\dfrac{7}{16}$.\\
	Vì $x\in \left(0;\dfrac{\pi }{2}\right)$ nên $\cos x>0\Rightarrow \cos x=\dfrac{\sqrt{7}}{4}$ suy ra $\sin 2x=2\sin x \cdot \cos x=2\cdot \dfrac{\sqrt{7}}{4}\cdot \dfrac{3}{4}=\dfrac{3\sqrt{7}}{8}$\\
	d) Ta có ${{\tan }^2}\alpha =\dfrac{1}{{{\cos }^2}\alpha }-1=\dfrac{1}{{{\left(\dfrac{2}{3}\right)}^2}}-1=\dfrac{5}{4}$\\
	Vì $\alpha \in \left(-\dfrac{\pi }{2};0\right)$ nên $\tan \alpha <0\Rightarrow \tan \alpha =\dfrac{-\sqrt{5}}{2}$\\
	$\tan \left(\alpha +\dfrac{\pi }{4}\right)=\dfrac{\tan \alpha +\tan \dfrac{\pi }{4}}{1-\tan \alpha \cdot \tan \dfrac{\pi }{4}}=\dfrac{\dfrac{-\sqrt{5}}{2}+1}{1-\left(\dfrac{-\sqrt{5}}{2}\right) \cdot 1}=-9+4\sqrt{5}$\\
	Vậy $a=-9,b=4,c=5$ nên mệnh đề đúng
	}
\end{ex}
%Câu 14
\begin{ex}
	Biết $\cos x=\dfrac{1}{3}$ và $-\dfrac{\pi }{2}<x<0$. Khi đó: Các mệnh đề sau đúng hay sai?
	\choiceTF
	{\True $\sin \left(\dfrac{\pi }{2}-x\right)>0$}
    {$\sin 2x=\dfrac{4\sqrt{2}}{9}$}
	{\True $\cos \left(x+\dfrac{4\pi }{3}\right)=-\dfrac{1+3\sqrt{6}}{6}$}
	{\True $\sin x+\sin 3x=-\dfrac{8\sqrt{2}}{27}$}
	\loigiai{
	a) Ta có $\sin \left(\dfrac{\pi }{2}-x\right)=\cos x=\dfrac{1}{3}>0$\\
	b) Ta có ${{\sin }^2}x=1-{{\cos }^2}x=1-{{\left(\dfrac{1}{3}\right)}^2}=\dfrac{8}{9}\Rightarrow \sin x=\pm \dfrac{2\sqrt{2}}{3}$.\\
	Vì $-\dfrac{\pi }{2}<x<0$ nên $\sin x=-\dfrac{2\sqrt{2}}{3}$.\\
	Áp dụng công thức nhân đôi ta có: $\sin 2x=2\sin x\cos x=2 \cdot \left(-\dfrac{2\sqrt{2}}{3}\right) \cdot \dfrac{1}{3}=-\dfrac{4\sqrt{2}}{9}$\\
	c) $\cos \left(x+\dfrac{4\pi }{3}\right)=\cos x \cdot \cos \dfrac{4\pi }{3}-\sin x \cdot \sin \dfrac{4\pi }{3}=\dfrac{1}{3} \cdot \left(-\dfrac{1}{2}\right)-\left(-\dfrac{2\sqrt{2}}{2}\right) \cdot \left(-\dfrac{\sqrt{3}}{2}\right)=-\dfrac{1+3\sqrt{6}}{6}$\\
	d) Áp dụng công thức ta có:\\
	$\sin x+\sin 3x=2\sin 2x \cdot \cos x=2 \cdot \left(-\dfrac{4\sqrt{2}}{9}\right) \cdot \dfrac{1}{3}=-\dfrac{8\sqrt{2}}{27}$
	}
\end{ex}
%Câu 15
\begin{ex}
	Cho hàm số $f(x)=-2\sin \left(2x-\dfrac{\pi }{2}\right)+2025$. Các mệnh đề sau đúng hay sai?
	\choiceTF
	{\True Hàm số $f(x)$ có tập xác định là $\mathbb{R}$}
	{Hàm số $f(x)$ tuần hoàn với chu kì $T=2\pi $}
	{Hàm số $f(x)$ không chẵn, không lẻ}
	{\True Hàm số $f(x)$ đạt giá trị lớn nhất tại $x=k\pi ,k\in \mathbb{Z}$}
	\loigiai{
		a). Vì tập xác định của hàm $\sin $ là $\mathbb{R}$ nên hàm số $f(x)$ có tập xác định là $\mathbb{R}$.\\
		b). Ta có $-2\sin \left(2x-\dfrac{\pi }{2}\right)+2025=2\sin \left(\dfrac{\pi }{2}-2x\right)+2025=2\cos 2x+2025$.\\
		Do đó $f(x)=2\cos 2x+2025$ nên hàm số $f(x)$ tuần hoàn với chu kì $T=\dfrac{2\pi }{2}=\pi $.\\
		c) Ta có $\forall x\in \mathbb{R},-x\in \mathbb{R}$ và $f(-x)=2\cos (-2x)+2025=2\cos 2x+2025=f(x)$ nên hàm số $f(x)$ là hàm số chẵn.\\
		d) Ta có $-2\le 2\cos 2x\le 2,\forall x\in \mathbb{R}$ hay $2023\le 2\cos 2x+2025\le 2027,\forall x\in \mathbb{R}$.\\
		Do đó $f(x)=2027\Leftrightarrow \cos 2x=1\Leftrightarrow x=k\pi ,k\in \mathbb{Z}$.\\
		Vậy hàm số $f(x)$ đạt giá trị lớn nhất tại $x=k\pi ,k\in \mathbb{Z}$
	}
\end{ex}
%Câu 16
\begin{ex}
	Cho hàm số $f(x)=\dfrac{1}{{{\cos }^2}x}+\dfrac{1}{{{\sin }^2}x}$. Xét tính đúng sai của các mệnh đề sau
	\choiceTF
	{\True Hàm số đã cho là hàm số tuần hoàn}
	{\True Hàm số đã cho là hàm số chẵn}
	{Tập xác định của hàm số là $D=\mathbb{R}\backslash \left\{ \dfrac{\pi }{2}+k\pi ,k\in \mathbb{Z} \right\}$}
	{\True Giá trị nhỏ nhất của hàm số là 4}
	\loigiai{
		a) Hàm số tuần hoàn do hai hàm $y=\operatorname{sinx}$ và $y=\cos x$ cùng tuần hoàn với chu kì $2\pi $.\\
		b) Ta có $f(-x)=\dfrac{1}{{{\cos }^2}(-x)}+\dfrac{1}{{{\sin }^2}(-x)}=\dfrac{1}{{{\left(\operatorname{cosx}\right)}^2}}+\dfrac{1}{{{\left(-\sin x\right)}^2}}=\dfrac{1}{{{\cos }^2}x}+\dfrac{1}{{{\sin }^2}x}=f(x)$.\\
		Do đó hàm số đã cho là hàm số chẵn\\
		c) Hàm số xác định khi $\heva{& \operatorname{sinx}\ne 0 \\& \operatorname{cosx}\ne 0}\Leftrightarrow \sin 2x\ne 0\Leftrightarrow 2x\ne k\pi \Leftrightarrow x\ne \dfrac{k\pi }{2},k\in \mathbb{Z}$\\
		Tập xác định của hàm số là $D=\mathbb{R}\backslash \left\{ \dfrac{k\pi }{2},k\in \mathbb{Z} \right\}$.\\
		d) Khi $x\ne \dfrac{k\pi }{2},k\in \mathbb{Z}$ ta có\\
		$f(x)=\dfrac{1}{{{\cos }^2}x}+\dfrac{1}{{{\sin }^2}x}\ge 2\sqrt{\dfrac{1}{{{\cos }^2}x} \cdot \dfrac{1}{{{\sin }^2}x}}=2\sqrt{\dfrac{4}{{{\sin }^2}2x}}=\dfrac{4}{\left| \sin 2x \right|}\ge \dfrac{4}{1}=4$.\\
		Nên giá trị nhỏ nhất của hàm số là 4
	}
\end{ex}
\TNSA
%Câu 19
\begin{ex}
    Tìm tập giá trị của các hàm số $y=\sqrt{2+\cos x}-5$ là đoạn $[a;b]$. Giá trị $a+b$ (làm tròn đến hàng phần chục) là
    \shortans{-7,3}
    \loigiai{
    Vì $\cos x\ge -1\Leftrightarrow 2+\cos x\ge 1>0,\forall x\in \mathbb{R}$ nên tập xác định của hàm số là $D=\mathbb{R}$.\\
    $\forall x\in \mathbb{R}$, ta có:
    \begin{eqnarray*}
        & & -1\le \cos x\le 1 \\
        & \Leftrightarrow & 1\le 2+\cos x\le 3 \\
        & \Leftrightarrow & 1\le \sqrt{2+\cos x}\le \sqrt{3} \\
        & \Leftrightarrow & -4\le \sqrt{2+\cos x}\,-5\le \sqrt{3}-5
    \end{eqnarray*}
    Vậy tập giá trị của hàm số là $T=\left[-4;\sqrt{3}-5\right]$. Suy ra $a+b \approx -7,3$.
    }    
\end{ex}
%Câu 18
\begin{ex}
	Tổng số giờ ban ngày của ngày thứ $x$ trong một năm không nhuận được tính bởi công thức
	$g(x)=3\sin (0,0172x-1,376)+12$.
	Trong đó $x$ đại diện cho ngày trong năm, $1\le x\le 365$. Ngày $\overline{ab}$ tháng $\overline{cd}$ có số giờ ban ngày dài nhất. Số $\overline{abcd}$ bằng
	\shortans{2006}
	\loigiai{
		Ta có $-1\le \sin (0,0172x-1,376)\le 1$\\
		$-3\le 3\sin (0,0172x-1,376)\le 3$\\
		$9\le 3\sin (0,0172x-1,376)+12\le 15$\\
		Suy ra $9\le g(x)\le 15$\\
		Do đó, số giờ ban ngày dài nhất trong một ngày là 15 giờ.\\
		Ta có phương trình $3\sin (0,0172x-1,376)+12=15$\\
		$\sin (0,0172x-1,376)=1$\\
		$x\approx 171{,}3$\\
		Vậy vào khoảng ngày thứ 171 trong năm (ngày 20 tháng 6) thì số giờ ban ngày dài nhất
	}
\end{ex}
%Câu 19
\begin{ex}
	Hai thành phố có cùng kinh độ. Vĩ tuyến của thành phố A là $10^\circ $ Bắc và vĩ tuyến của thành phố B là $40^\circ $ Bắc. Giả sử bán kính trái đất là 3960 dặm. Tìm khoảng cách giữa hai thành phố (làm tròn đến chữ số hàng đơn vị)
	\shortans{2073}
	\loigiai{
		Khoảng cách từ điểm trên đường xích đạo đến thành phố B ở cùng kinh độ là $3960 \cdot \dfrac{40}{180} \cdot \pi =880\pi $ (dặm)\\
		Khoảng cách từ điểm trên đường xích đạo đến thành phố A ở cùng kinh độ là $3960 \cdot \dfrac{10}{180}\pi =220\pi $ (dặm)\\
		Khoảng cách giữa hai thành phố A và B là $880\pi -220\pi =660\pi \approx 2073$ (dặm)
	}
\end{ex}
%Câu 20
\begin{ex}
	Giả sử vận tốc $v$ (tính bằng lít/ giây) của luồng khí trong một chu kì hô hấp (tức là thời gian từ lúc bắt đầu của một nhịp thở đến khi bắt đầu của nhịp thở tiếp theo) của một người nào đó ở trạng thái nghỉ ngơi được cho bởi công thức $v=0{,}85\sin \dfrac{\pi t}{3}$, trong đó $t$ là thời gian (tính bằng giây). Biết rằng quá trình hít vào xảy ra khi $v>0$ và quá trình thở ra xảy ra khi $v<0$. Trong khoảng thời gian từ 5 đến 10 giây, khoảng thời điểm sau $a$ giây đến trước $b$ giây thì người đó hít vào. Tính $\sqrt{a+b}$ (làm tròn đến hàng phần trăm).
	\shortans{3,87}
	\loigiai{
		+) Vì quá trình hít vào xảy ra khi $v>0$ nên ta có\\
		$0{,}85\sin \dfrac{\pi t}{3}>0\Leftrightarrow \sin \dfrac{\pi t}{3}>0\Leftrightarrow \dfrac{\pi t}{3}\in \left(k2\pi ;\pi +k2\pi\right)(k\in \mathbb{Z})$\\
		$\Leftrightarrow t\in \left(6k;3+6k\right)\,\left(k\in \mathbb{Z}\right)$\\
		+) Vì $t\in [5;10]$ nên $k=1$ suy ra $t\in \left(6;9\right)$.\\
		Trong khoảng thời gian từ 5 đến 10 giây, khoảng thời điểm sau $6$ giây đến trước $9$ giây thì người đó hít vào nên $\sqrt{a+b}=\sqrt{15}\approx 3,87$.
	}
\end{ex}
%Câu 21
\begin{ex}
	Nghiệm phương trình lượng giác $\sqrt{3}\sin x-\cos x=0$ có dạng $x=\dfrac{\pi }{a}+k \cdot b\pi $ ($a$, $b$, $k\in \mathbb{Z}$, $a\ne 0$). Tính $(a+b)^4$.
	\shortans{2401}
	\loigiai{
		Phương trình tương đương\\
		$\dfrac{\sqrt{3}}{2}\sin x-\dfrac{1}{2}\cos x=0$\\
		$\Leftrightarrow \sin x\cos \dfrac{\pi }{6}-\cos x\sin \dfrac{\pi }{6}=0$\\
		$\Leftrightarrow \sin \left(x-\dfrac{\pi }{6}\right)=0$\\
		$\Leftrightarrow x-\dfrac{\pi }{6}=k\pi $ ($k\in \mathbb{Z}$)\\
		$\Leftrightarrow x=\dfrac{\pi }{6}+k\pi $.\\
		Phương trình có nghiệm là: $x=\dfrac{\pi }{6}+k\pi $ ($k\in \mathbb{Z}$).\\
		Suy ra $a=6$; $b=1$. Vậy $(a+b)^4=7^4=2401$.
	}
\end{ex}
%Câu 22
\begin{ex}
	Một vật $M$ được gắn vào đầu lò xo và dao động quanh vị trí cân bằng, toạ độ $x$ (đơn vị: cm) tại thời điểm $t$ (giây) được tính bởi công thức $x=8{,}6\sin \left(8t+\dfrac{\pi }{2}\right)$. Có $n$ thời điểm trong khoảng 2 giây đầu tiên thì $s=4{,}3$ cm. Giá trị $\sqrt[3]{n}$ (làm tròn đến hàng phần trăm)
	\shortans{1,71}
	\loigiai{
	Khi $x=4{,}3$ thì $8{,}6\sin \left(8t+\dfrac{\pi }{2}\right)=4{,}3\Rightarrow \sin \left(8t+\dfrac{\pi }{2}\right)=\dfrac{1}{2}$\\
	$\Leftrightarrow \hoac{&8t+\dfrac{\pi }{2}=\dfrac{\pi }{6}+k2\pi  \\
			&8t+\dfrac{\pi }{2}=\dfrac{5\pi }{6}+l2\pi}(k,l\in \mathbb{Z})
            \Leftrightarrow \hoac{& t=-\dfrac{\pi }{24}+k\dfrac{\pi }{4} \\
            & t=\dfrac{\pi }{24}+l\dfrac{\pi }{4} }(k,l\in \mathbb{Z})$.\\
	Vì $t\in (0;2)$ nên $\heva{& 0<-\dfrac{\pi }{24}+k\dfrac{\pi }{4}<2 \\ & 0<\dfrac{\pi }{24}+l\dfrac{\pi }{4}<2} \Leftrightarrow \heva{& \dfrac{1}{6}<k<\dfrac{8}{\pi }+\dfrac{1}{6}  \\ & -\dfrac{1}{6}<l<\dfrac{8}{\pi }-\dfrac{1}{6}}$\\
	Mà $k,l\in \mathbb{Z}$ nên $k\in \left\{ 1;2 \right\}$; $l\in \left\{ 0;1;2 \right\}$.\\
	Vậy có $5$ thời điểm thỏa mãn đề bài nên $\sqrt[3]{n}\approx 1,71$
	}
\end{ex}
