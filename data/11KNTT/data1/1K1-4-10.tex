
\begin{dang}{Phương trình đối xứng, phản đối xứng $a\cdot \left(\sin x \pm \cos x\right)+b\cdot \sin x\cos x +c=0$}
	Đặt $t=\sin x \pm \cos x=\sqrt{2}\sin\left(x\pm \dfrac{\pi}{4}\right) \Rightarrow t^2=1\pm 2\sin x \cos x$  $\Rightarrow \sin x\cos x=\dfrac{t^2-1}{\pm 2}$.\\
	Đưa về phương trình ẩn $t$, sau đó tìm $t$ và giải phương trình  $\sqrt{2}\sin\left(x\pm \dfrac{\pi}{4}\right)=t$.
	\begin{note}
		Khi đặt $t=\left | \sin x \pm \cos x \right |$ thì điểu kiện là: $0\leq t\leq \sqrt{2}$.
	\end{note}
\end{dang}
\subsubsection{Ví dụ}
\begin{vd}%[1K1K4-0]%[DCHT-11-KNTT]%[Vĩ Lê Văn]
	Giải phương trình $2(\sin x+\cos x)+\sin 2 x+1=0$.
	\dapso{$x=-\dfrac{\pi}{4}+k \pi, \quad k \in \mathbb{Z}$.}
	\loigiai{
		Ta có $2(\sin x+\cos x)+\sin 2 x+1=0
		\Leftrightarrow 2(\sin x+\cos x)+2\sin x \cos x+1=0$.\\
		Đặt $t=\sin x+\cos x=\sqrt{2} \sin \left(x+\dfrac{\pi}{4}\right) \Rightarrow \sin x \cos x=\dfrac{t^2-1}{2}$, điều kiện $-\sqrt{2} \leq t \leq \sqrt{2}$.\\
		Phương trình trở thành: $2 t+\left(t^2-1\right)+1=0\Leftrightarrow t^2+2 t=0\Leftrightarrow\hoac{&t=0\quad (\text{nhận}) \\& t=-2\quad(\text {loại }).}$\\
		Khi đó $\sin x+\cos x=0\Leftrightarrow \sin \left(x+\dfrac{\pi}{4}\right)=0\Leftrightarrow x=-\dfrac{\pi}{4}+k \pi, \quad k \in \mathbb{Z}$.
		\\
		Vậy phương trình có nghiệm $x=-\dfrac{\pi}{4}+k \pi, \quad k \in \mathbb{Z}$.}
\end{vd}

\begin{vd}%[1K1K4-0]%[DCHT-11-KNTT]%[Vĩ Lê Văn]
	Giải phương trình	$
	\sin x \cos x=6(\sin x-\cos x-1)$.
		\dapso{$\hoac{&x=\dfrac{\pi}{2}+k \pi \\& x=\pi+k 2\pi},\quad k \in \mathbb{Z}$.}
		\loigiai{
		Đặt $t=\sin x-\cos x=\sqrt{2} \sin \left(x-\dfrac{\pi}{4}\right) \Rightarrow \sin x \cos x=\dfrac{1-t^2}{2}$, điều kiện $-\sqrt{2} \leq t \leq \sqrt{2}$.\\
		Phương trình trở thành: $1-t^2=12(t-1) \Leftrightarrow t^2+12 t-13=0\Leftrightarrow\hoac{&t=1 \quad (\text{nhận}) \\& t=-13\quad(\text {loại}).}$\\
		Vớt $t=1$, ta có $\sin x-\cos x=1\Leftrightarrow \sin \left(x-\dfrac{\pi}{4}\right)=\dfrac{1}{\sqrt{2}}\Leftrightarrow \hoac{&x=\dfrac{\pi}{2}+k \pi \\& x=\pi+k 2\pi}, \quad k \in \mathbb{Z}$.\\
		Vậy phương trình có nghiệm: $\hoac{&x=\dfrac{\pi}{2}+k \pi \\& x=\pi+k 2\pi},\quad k \in \mathbb{Z}$.}
	
\end{vd}
\begin{vd}%[1K1K4-0]%[DCHT-11-KNTT]%[Vĩ Lê Văn]
	Giải phương trình $\tan x-2\sqrt{2} \sin x=1$.
		\dapso{$\hoac{&x=\dfrac{5\pi}{12}+k 2\pi \\& x=\dfrac{13\pi}{12}+k 2\pi\\&x=-\dfrac{\pi}{4}+k 2\pi}, \quad k \in \mathbb{Z}$.}
		\loigiai{
		Điều kiện $\cos x \neq 0$.\\
		Ta có $\tan x-2\sqrt{2} \sin x=1\Leftrightarrow \sin x-\cos x-2\sqrt{2} \sin x \cos x=0$.\\
		Đặt $t=\sin x-\cos x=\sqrt{2} \sin \left(x-\dfrac{\pi}{4}\right) \Rightarrow \sin x \cos x=\dfrac{1-t^2}{2}$, điều kiện $-\sqrt{2} \leq t \leq \sqrt{2}$.\\
		PT trở thành: $t-\sqrt{2}\left(1-t^2\right)=0\Leftrightarrow \sqrt{2} t^2+t-\sqrt{2}=0\Leftrightarrow\hoac{&t=\dfrac{\sqrt{2}}{2}\quad (\text{nhận}) \\& t=-\sqrt{2}\quad (\text{nhận}).}$\\
		$\Leftrightarrow \hoac{&\sqrt{2}\sin \left(x-\dfrac{\pi}{4}\right)=\dfrac{\sqrt{2}}{2}\\&\sqrt{2}\sin \left(x-\dfrac{\pi}{4}\right)=-\sqrt{2}} \Leftrightarrow \hoac{&\sin \left(x-\dfrac{\pi}{4}\right)=\dfrac{1}{2}\\&\sin \left(x-\dfrac{\pi}{4}\right)=-1} \Leftrightarrow\hoac{&x=\dfrac{5\pi}{12}+k 2\pi \\& x=\dfrac{13\pi}{12}+k 2\pi\\&x=-\dfrac{\pi}{4}+k 2\pi}, \quad k \in \mathbb{Z}$.
		\\
		Kết hợp với điều kiện, PT có nghiệm: $\hoac{&x=\dfrac{5\pi}{12}+k 2\pi \\& x=\dfrac{13\pi}{12}+k 2\pi\\&x=-\dfrac{\pi}{4}+k 2\pi}, \quad k \in \mathbb{Z}$.}
\end{vd}
\begin{vd}%[1K1K4-0]%[DCHT-11-KNTT]%[Vĩ Lê Văn]
	Giải phương trình $\sin 2 x+\sqrt{2} \sin \left(x-\dfrac{\pi}{4}\right)=1$.
		\dapso{$\hoac{& x=\dfrac{\pi}{4}+k \pi\\&x=\dfrac{\pi}{2}+k 2\pi \\& x=\pi+k 2\pi}, \quad k \in \mathbb{Z}$.}
	\loigiai{
		Ta có
		$\sin 2 x+\sqrt{2} \sin \left(x-\dfrac{\pi}{4}\right)=1\Leftrightarrow 2\sin x \cos x+\sin x-\cos x=1$.\\
		Đặt $t=\sin x-\cos x=\sqrt{2} \sin \left(x-\dfrac{\pi}{4}\right) \Rightarrow \sin x \cos x=\dfrac{1-t^2}{2}$, điều kiện $-\sqrt{2} \leq t \leq \sqrt{2}$.\\
		Phương trình trở thành: $1-t^2+t=1\Leftrightarrow t^2-t=0\Leftrightarrow\hoac{&t=0\quad (\text{nhận})\\& t=1\quad (\text{nhận})}$\\
		$\Leftrightarrow \hoac{&\sqrt{2}sin \left(x-\dfrac{\pi}{4}\right)=0\\&\sqrt{2}sin \left(x-\dfrac{\pi}{4}\right)=1}\Leftrightarrow \hoac{&sin \left(x-\dfrac{\pi}{4}\right)=0\\&\sin \left(x-\dfrac{\pi}{4}\right)=\dfrac{1}{\sqrt{2}}}\Leftrightarrow \hoac{& x=\dfrac{\pi}{4}+k \pi\\&x=\dfrac{\pi}{2}+k 2\pi \\& x=\pi+k 2\pi}, \quad k \in \mathbb{Z}$.\\
		Vậy phương trình có nghiệm: $\hoac{& x=\dfrac{\pi}{4}+k \pi\\&x=\dfrac{\pi}{2}+k 2\pi \\& x=\pi+k 2\pi}, \quad k \in \mathbb{Z}$.}
\end{vd}

\subsubsection{Bài tập tự luận}
\begin{bt}%[1K1K4-0]%[DCHT-11-KNTT]%[Vĩ Lê Văn]
	Giải phương trình  $\sin x-\cos x+2\sin 2x+1=0$.
		\dapso{$\hoac{&x=k2\pi \\&x=\dfrac{3\pi}{2}+k2\pi}, \quad k \in \mathbb{Z}$.}
	\loigiai{
		Đặt $t=\sin x-\cos x=\sqrt{2}\sin\left(x- \dfrac{\pi}{4}\right)$ thì $t \in \left[-\sqrt{2};\sqrt{2}\right]$.\\
		$\Rightarrow \left(\sin x-\cos x\right)^2=t^2 \Leftrightarrow \sin^2 x -2\sin x\cos x+ \cos^2 x =t^2 \Leftrightarrow 1-\sin 2x=t^2 \Leftrightarrow \sin 2x=1-t^2$.\\
		Phương trình trở thành: $t+2(1-t^2)+1=0 \Leftrightarrow -2t^2+t+3=0 \Leftrightarrow \hoac{&t=-1\\&t=\dfrac{3}{2} \quad (\text{loại}).}$\\
		Với $t=-1 \Leftrightarrow \sqrt{2}\sin\left(x- \dfrac{\pi}{4}\right)=-1 \Leftrightarrow \sin \left(x-\dfrac{\pi}{4}\right) =\dfrac{-1}{\sqrt{2}} \Leftrightarrow \hoac{&x=k2\pi \\&x=\dfrac{3\pi}{2}+k2\pi}, \quad k \in \mathbb{Z}.$\\
		Vậy nghiệm của phương trình là $\hoac{&x=k2\pi \\&x=\dfrac{3\pi}{2}+k2\pi}, \quad k \in \mathbb{Z}$.}

\end{bt}
\begin{bt}%[1K1K4-0]%[DCHT-11-KNTT]%[Vĩ Lê Văn]
	Giải phương trình  $\sin x+\cos x+\sin x\cos x=1$.
		\dapso{$\hoac{&x=k2\pi \\&x=\dfrac{\pi}{2}+k2\pi}, \quad k \in \mathbb{Z}$.}
	\loigiai{
		Đặt $t=\sin x+\cos x=\sqrt{2}\sin\left(x+ \dfrac{\pi}{4}\right)$ thì $t \in \left[-\sqrt{2};\sqrt{2}\right]$.\\
		$\Rightarrow \left(\sin x+\cos x\right)^2=t^2 \Leftrightarrow 1+2\sin x\cos x=t^2 \Leftrightarrow \sin x\cos x=\dfrac{t^2-1}{2}$.\\
		Phương trình trở thành: $t+\dfrac{t^2-1}{2}=1 \Leftrightarrow t^2+2t-3=0 \Leftrightarrow \hoac{&t=1\\&t=-3 \quad (\text{loại}).}$\\
		Với $t=1 \Leftrightarrow \sqrt{2}\sin\left(x+ \dfrac{\pi}{4}\right)=1 \Leftrightarrow \sin \left(x+\dfrac{\pi}{4}\right) =\dfrac{1}{\sqrt{2}} \Leftrightarrow \hoac{&x=k2\pi \\&x=\dfrac{\pi}{2}+k2\pi}, \quad k \in \mathbb{Z}.$\\
		Vậy nghiệm của phương trình là $\hoac{&x=k2\pi \\&x=\dfrac{\pi}{2}+k2\pi}, \quad k \in \mathbb{Z}$.}

\end{bt}
\begin{bt}%[1K1K4-0]%[DCHT-11-KNTT]%[Vĩ Lê Văn]
	Giải phương trình  $\sin x+\cos x-\sin x\cos x=1$.
		\dapso{$\hoac{&x=k2\pi \\&x=\dfrac{\pi}{2}+k2\pi}, \quad k \in \mathbb{Z}$.}
	\loigiai{
		Đặt $t=\sin x+\cos x=\sqrt{2}\sin\left(x+ \dfrac{\pi}{4}\right)$ thì $t \in \left[-\sqrt{2};\sqrt{2}\right]$.\\
		$\Rightarrow \left(\sin x+\cos x\right)^2=t^2 \Leftrightarrow 1+2\sin x\cos x=t^2 \Leftrightarrow \sin x\cos x=\dfrac{t^2-1}{2}$.\\
		Phương trình trở thành: $t-\dfrac{t^2-1}{2}=1 \Leftrightarrow -t^2+2t-1=0 \Leftrightarrow t=1.$\\
		Với $t=1 \Leftrightarrow \sqrt{2}\sin\left(x+ \dfrac{\pi}{4}\right)=1 \Leftrightarrow \sin \left(x+\dfrac{\pi}{4}\right) =\dfrac{1}{\sqrt{2}} \Leftrightarrow \hoac{&x=k2\pi \\&x=\dfrac{\pi}{2}+k2\pi}, \quad k \in \mathbb{Z}.$\\
		Vậy nghiệm của phương trình là $\hoac{&x=k2\pi \\&x=\dfrac{\pi}{2}+k2\pi}, \quad k \in \mathbb{Z}$.}

\end{bt}
\begin{bt}%[1K1K4-0]%[DCHT-11-KNTT]%[Vĩ Lê Văn]
	Giải phương trình  $\sin 2x-2\sqrt{2}\left( \sin x+\cos x\right)=5$.
		\dapso{$x=\dfrac{-3\pi}{4}+k2\pi, \quad k \in \mathbb{Z}$.}
	\loigiai{
		Đặt $t=\sin x+\cos x=\sqrt{2}\sin\left(x+ \dfrac{\pi}{4}\right)$ thì $t \in \left[-\sqrt{2};\sqrt{2}\right]$.\\
		$\Rightarrow \left(\sin x+\cos x\right)^2=t^2 \Leftrightarrow 1+2\sin x\cos x=t^2 \Leftrightarrow \sin 2x=t^2-1$.\\
		Phương trình trở thành: $t^2-1-2\sqrt{2}t=5 \Leftrightarrow t^2-2\sqrt{2}t-6=0 \Leftrightarrow \hoac{&t=-\sqrt{2}\quad (\text{nhận})\\&t=3\sqrt{2} \quad (\text{loại}).}$\\
		Với $t=-\sqrt{2} \Leftrightarrow \sqrt{2}\sin\left(x+ \dfrac{\pi}{4}\right)=-\sqrt{2} \Leftrightarrow \sin \left(x+\dfrac{\pi}{4}\right) =-1 \Leftrightarrow x=\dfrac{-3\pi}{4}+k2\pi, \quad k \in \mathbb{Z}$.\\
		Vậy nghiệm của phương trình là $x=\dfrac{-3\pi}{4}+k2\pi, \quad k \in \mathbb{Z}$.}

\end{bt}
\begin{bt}%[1K1K4-0]%[DCHT-11-KNTT]%[Vĩ Lê Văn]
	Giải phương trình  $5\sin 2x+12=12\left( \sin x-\cos x\right) $.
		\dapso{$\hoac{&x=\pi+k2\pi \\&x=\dfrac{\pi}{2}+k2\pi}, \quad k \in \mathbb{Z}$.}
	\loigiai{
		Đặt $t=\sin x-\cos x=\sqrt{2}\sin\left(x- \dfrac{\pi}{4}\right)$ thì $t \in \left[-\sqrt{2};\sqrt{2}\right]$.\\
		$\Rightarrow \left(\sin x-\cos x\right)^2=t^2 \Leftrightarrow \sin^2 x -2\sin x\cos x \cos^2 x =t^2 \Leftrightarrow 1-\sin 2x=t^2 \Leftrightarrow \sin 2x=1-t^2$.\\
		Phương trình trở thành: $5(1-t^2)+12=12t \Leftrightarrow -5t^2-12t+17=0 \Leftrightarrow \hoac{&t=1\\&t=-\dfrac{17}{5} \quad (\text{loại}).}$\\
		Với $t=1 \Leftrightarrow \sin x -\cos x=1 \Leftrightarrow \sin \left(x-\dfrac{\pi}{4}\right) =\dfrac{1}{\sqrt{2}} \Leftrightarrow \hoac{&x=\pi+k2\pi \\&x=\dfrac{\pi}{2}+k2\pi}, \quad k \in \mathbb{Z}.$\\
		Vậy nghiệm của phương trình là $\hoac{&x=\pi+k2\pi \\&x=\dfrac{\pi}{2}+k2\pi}, \quad k \in \mathbb{Z}$.}
	
\end{bt}
\begin{bt}%[1K1K4-0]%[DCHT-11-KNTT]%[Vĩ Lê Văn]
	Giải phương trình  $\sin x\cos x=6\left( \sin x-\cos x+1\right) $.
		\dapso{$\hoac{&x=k2\pi \\&x=\dfrac{3\pi}{2}+k2\pi}, \quad k \in \mathbb{Z}$.}
	\loigiai{
		Đặt $t=\sin x-\cos x=\sqrt{2}\sin\left(x- \dfrac{\pi}{4}\right)$ thì $t \in \left[-\sqrt{2};\sqrt{2}\right]$.\\
		$\Rightarrow \left(\sin x-\cos x\right)^2=t^2 \Leftrightarrow 1-2\sin x\cos x=t^2 \Leftrightarrow \sin x\cos x=\dfrac{1-t^2}{2}$.\\
		Phương trình trở thành: $\dfrac{1-t^2}{2}=6(t+1) \Leftrightarrow t^2+12t+11=0 \Leftrightarrow \hoac{&t=1\\&t=-11 \quad (\text{loại}).}$\\
		Với $t=-1 \Leftrightarrow \sqrt{2}\sin\left(x- \dfrac{\pi}{4}\right)=-1 \Leftrightarrow \sin \left(x-\dfrac{\pi}{4}\right) =-\dfrac{1}{\sqrt{2}} \Leftrightarrow \hoac{&x=k2\pi \\&x=\dfrac{3\pi}{2}+k2\pi}, \quad k \in \mathbb{Z}$.\\
		Vậy nghiệm của phương trình là $\hoac{&x=k2\pi \\&x=\dfrac{3\pi}{2}+k2\pi}, \quad k \in \mathbb{Z}$.}

\end{bt}
\begin{bt}%[1K1K4-0]%[DCHT-11-KNTT]%[Vĩ Lê Văn]
	Giải phương trình  $\sin x\cos x+\left| \sin x+\cos x\right| =1$.
		\dapso{$x=k\dfrac{\pi}{2}, \quad k \in \mathbb{Z}$.}
	\loigiai{
		Đặt $t=\left| \sin x+\cos x\right|=\left| \sqrt{2}\sin\left(x+ \dfrac{\pi}{4}\right)\right|$ thì $t \in \left[0;\sqrt{2}\right]$.\\
		$\Rightarrow \left|\sin x+\cos x\right|^2=t^2 \Leftrightarrow 1+2\sin x\cos x=t^2 \Leftrightarrow \sin x\cos x=\dfrac{t^2-1}{2}$.\\
		Phương trình trở thành: $\dfrac{t^2-1}{2}+t=1 \Leftrightarrow t^2+2t-3=0 \Leftrightarrow \hoac{&t=1\\&t=-3 \quad (\text{loại}).}$\\
		Với $t=1 \Leftrightarrow \left| \sqrt{2}\sin\left(x+ \dfrac{\pi}{4}\right)\right|=1 \Leftrightarrow \left| \sin \left(x+\dfrac{\pi}{4}\right)\right| =\dfrac{1}{\sqrt{2}} \Leftrightarrow \hoac{&\sin\left(x+\dfrac{\pi}{4}\right)=\dfrac{\sqrt{2}}{2}\\&\sin\left(x+\dfrac{\pi}{4}\right)=-\dfrac{\sqrt{2}}{2}} $\\
		$\Leftrightarrow \hoac{&x=k2\pi \\&x=\dfrac{\pi}{2}+k2\pi\\&x=-\dfrac{\pi}{2}+k2\pi\\&x=\pi+k2\pi} \Leftrightarrow x=k\dfrac{\pi}{2}, \quad k \in \mathbb{Z}.$\\
		Vậy nghiệm của phương trình là $x=k\dfrac{\pi}{2}, \quad k \in \mathbb{Z}$.}

\end{bt}
\begin{bt}%[1K1K4-0]%[DCHT-11-KNTT]%[Vĩ Lê Văn]
	Giải phương trình  $\left| \cos x-\sin x\right|+3\sin 2x =1$.
		\dapso{$x=k\dfrac{\pi}{2}, \quad k \in \mathbb{Z}$.}
	\loigiai{
		Đặt $t=\left| \cos x-\sin x\right|=\left|\sqrt{2}\sin\left(x- \dfrac{\pi}{4}\right)\right|$ thì $t \in \left[0;\sqrt{2}\right]$.\\
		$\Rightarrow \left| \cos x-\sin x\right|^2=t^2 \Leftrightarrow 1-2\sin x\cos x=t^2 \Leftrightarrow \sin 2x=1-t^2$.\\
		Phương trình trở thành: $t+3(1-t^2)=1 \Leftrightarrow -3t^2+t+2=0 \Leftrightarrow \hoac{&t=1\\&t=-\dfrac{2}{3} \quad (\text{loại}).}$\\
		Với $t=1 \Leftrightarrow \left| \sqrt{2}\sin\left(x- \dfrac{\pi}{4}\right)\right|=1 \Leftrightarrow \left| \sin \left(x-\dfrac{\pi}{4}\right)\right| =\dfrac{1}{\sqrt{2}} \Leftrightarrow \hoac{&\sin\left(x-\dfrac{\pi}{4}\right)=\dfrac{\sqrt{2}}{2}\\&\sin\left(x-\dfrac{\pi}{4}\right)=-\dfrac{\sqrt{2}}{2}}$\\
		$\Leftrightarrow \hoac{&x=k2\pi \\&x=\dfrac{\pi}{2}+k2\pi\\&x=-\dfrac{\pi}{2}+k2\pi\\&x=-\pi+k2\pi}, \quad k \in \mathbb{Z} \Leftrightarrow x=k\dfrac{\pi}{2}, \quad k \in \mathbb{Z}$.\\
		Vậy nghiệm của phương trình là $x=k\dfrac{\pi}{2}, \quad k \in \mathbb{Z}$.}
\end{bt}
\begin{bt}%[1K1K4-0]%[DCHT-11-KNTT]%[Vĩ Lê Văn]
	Giải phương trình: $\tan^2x+\cot^2x-(\tan x-\cot x)-2=0$.
	\dapso{$\hoac{&x=\dfrac{\pi}{4}+k\dfrac{\pi}{2}\\&x=\arctan\dfrac{1+\sqrt{5}}{2}+k\pi\\&x=\arctan\dfrac{1-\sqrt{5}}{2}+k\pi}, \quad k \in \mathbb{Z}$.}
	\loigiai{
		Điều kiện: $x \neq \dfrac{k\pi}{2} \; (k \in \mathbb{Z})$. Đặt $t=\tan x-\cot x$.\\
		$\Rightarrow \left( \tan x-\cot x\right) ^2=t^2 \Leftrightarrow \tan^2x-2\tan x\cot x+\cot^2x=t^2 \Leftrightarrow \tan^2x+\cot^2x=t^2+2$.\\
		Phương trình trở thành: $t^2+2-t-2=0 \Leftrightarrow t^2-t=0 \Leftrightarrow \hoac{&t=0\\&t=1.}$\\
		Với $t=0 \Leftrightarrow \tan x-\cot x =0 \Leftrightarrow \tan x =\tan \left(\dfrac{\pi}{2}-x\right) \Leftrightarrow x=\dfrac{\pi}{2}-x+k\pi \Leftrightarrow x=\dfrac{\pi}{4}+k\dfrac{\pi}{2}, \quad k \in \mathbb{Z}$.\\
		Với  $t=1 \Leftrightarrow \tan x-\cot x =1 \Leftrightarrow \tan x -\dfrac{1}{\tan x}=1 \Leftrightarrow \tan^2x-\tan x -1 =0$\\
		$\Leftrightarrow \hoac{&\tan x=\dfrac{1+\sqrt{5}}{2}\\&\tan x=\dfrac{1-\sqrt{5}}{2}} \Leftrightarrow \hoac{&x=\arctan\dfrac{1+\sqrt{5}}{2}+k\pi \\&x=\arctan\dfrac{1-\sqrt{5}}{2}+k\pi}, \quad k \in \mathbb{Z}$.\\
		Vậy nghiệm của phương trình là $\hoac{&x=\dfrac{\pi}{4}+k\dfrac{\pi}{2}\\&x=\arctan\dfrac{1+\sqrt{5}}{2}+k\pi\\&x=\arctan\dfrac{1-\sqrt{5}}{2}+k\pi}, \quad k \in \mathbb{Z}$.}
	
\end{bt}
\begin{bt}%[1K1K4-0]%[DCHT-11-KNTT]%[Vĩ Lê Văn]
	Giải phương trình: $3\tan^2x+4\tan x+4\cot x+3\cot^2x+2=0$.
	\dapso{$x=-\dfrac{\pi}{4}+k\pi, \quad k \in \mathbb{Z}$.}
	\loigiai{
		Điều kiện: $x \neq \dfrac{k\pi}{2} \; (k \in \mathbb{Z})$. Đặt $t=\tan x+\cot x$.\\
		$\Rightarrow \left( \tan x+\cot x\right) ^2=t^2 \Leftrightarrow \tan^2x+2\tan x\cot x+\cot^2x=t^2 \Leftrightarrow \tan^2x+\cot^2x=t^2-2$\\
		Phương trình trở thành: $3(t^2-2)+4t+2=0 \Leftrightarrow 3t^2+4t-4=0 \Leftrightarrow \hoac{&t=-2\\&t=\dfrac{2}{3}.}$\\
		Với  $t=-2 \Leftrightarrow \tan x+\cot x =-2 \Leftrightarrow \tan x +\dfrac{1}{\tan x}=-2 \Leftrightarrow \tan^2x+2\tan x +1 =0$\\
		$\Leftrightarrow \tan x=-1 \Leftrightarrow x=-\dfrac{\pi}{4}+k\pi, \quad k \in \mathbb{Z}$.\\
		Với $t=\dfrac{2}{3} \Leftrightarrow \tan x+\cot x =\dfrac{2}{3} \Leftrightarrow \tan x +\dfrac{1}{\tan x}=\dfrac{2}{3} \Leftrightarrow \tan^2x-\dfrac{2}{3}\tan x +1 =0$.\\
		Phương trình này vô nghiệm.\\
		Vậy nghiệm của phương trình là $x=-\dfrac{\pi}{4}+k\pi, \quad k \in \mathbb{Z}$.}

\end{bt}


\subsubsection{Bài tập trắc nghiệm}
\Opensolutionfile{ans}[ans/ans-1K1-4-Dang10]
\begin{ex}%[1K1K4-0]%[DCHT-11-KNTT]%[Vĩ Lê Văn]
	Tổng nghiệm âm lớn nhất và nghiệm dương bé nhất của phương trình 
	$\sin x+\sin ^2 x+\sin ^3 x+\sin ^4 x=\cos x+\cos ^2 x+\cos ^3 x+\cos ^4 x$ là
	\choice
	{\True $-\dfrac{\pi}{4}$}
	{$\dfrac{\pi}{4}$}
	{$-\dfrac{\pi}{6}$}
	{$\dfrac{\pi}{6}$}
	\loigiai{
		Ta có:$$
		\begin{aligned}
			 \sin ^3 x-\cos ^3 x&=(\sin x-\cos x)\left(\sin ^2 x+\cos ^2 x+\sin x \cos x\right)\\&=(\sin x-\cos x)=\left(1+\sin x \cos x\right). \\
			 \sin ^4 x-\cos ^4 x&=\left(\sin ^2 x-\cos ^2 x\right)\left(\sin ^2 x+\cos ^2 x\right)\\
			&=\sin ^2 x-\cos ^2 x=(\sin x-\cos x)(\sin x+\cos x).
		\end{aligned}$$
		Phương trình được viết lại dưới dạng:$$
		\begin{aligned}
			& \sin x-\cos x+\sin ^2 x-\cos ^2 x+\sin ^3 x-\cos ^3 x+\sin ^4 x-\cos ^4 x=0.  \\
			& \Leftrightarrow(\sin x-\cos x)[1+2(\sin x+\cos x)+1+\sin x \cos x]=0. \\
			& \Leftrightarrow\hoac{&\sin x-\cos x=0\quad (1) \\&2(\sin x+\cos x)+\sin x \cos x+2=0\quad (2).}
		\end{aligned}$$
		+ Giải (1): Ta được $\tan  x=1\Leftrightarrow x=\dfrac{\pi}{4}+k \pi, \quad k \in \mathbb{Z}.$\\
		+ Giải (2): Đặt $\sin x+\cos x=t$,  điều kiện $|t| \leq \sqrt{2}$, suy ra $\sin x \cos x=\dfrac{t^2-1}{2}$.\\
		Khi đó (2) có dạng:$$
		\begin{aligned}
			& 2 t+\dfrac{t^2-1}{2}+2=0\Leftrightarrow t^2+4 t+3=0\Leftrightarrow\left[\begin{aligned}
				&
				t=-1\\&
				t=-3
			\end{aligned} \quad (\text{loại})\right. \\
			& \Leftrightarrow \sqrt{2}\sin\left(x+ \dfrac{\pi}{4}\right)=-1\Leftrightarrow\sin\left(x+ \dfrac{\pi}{4}\right)=-\dfrac{1}{\sqrt{2}}\Leftrightarrow\left[\begin{aligned}
				&
				x=-\dfrac{\pi}{2}+2 k \pi \\&
				x=\pi+2 k \pi
			\end{aligned}, \quad k \in \mathbb{Z}\right.
		\end{aligned}$$
		Vậy phương trình có ba họ nghiệm $\left[\begin{aligned}
			&
			x=-\dfrac{\pi}{2}+2 k \pi \\&
			x=\pi+2 k \pi\\
			&x=\dfrac{\pi}{4}+k \pi
		\end{aligned}, \quad k \in \mathbb{Z}\right.$.}
\end{ex}
\begin{ex}%[1K1K4-0]%[DCHT-11-KNTT]%[Vĩ Lê Văn]
	Hỏi trên đoạn $[0;2024\pi]$, phương trình $\left|\sin x-\cos x\right|+4\sin 2x=1$ có bao nhiêu nghiệm?
	\choice
	{\True $4049$}
	{$4048$}
	{$2024$}
	{$2025$}
	\loigiai{
		Đặt $t=\left|\sin x-\cos x\right|=\sqrt{2}\left|\sin\left(x-\dfrac{\pi}{4}\right)\right|$. Vì $\sin\left(x-\dfrac{\pi}{4}\right)\in[-1;1]\Rightarrow t\in[0;\sqrt{2}]$.\\
		Ta có $t^2=\left(\sin x-\cos x\right)^2=\sin^2x+\cos^2x-2\sin x\cos x\Rightarrow\sin 2x=1-t^2$.\\
		Phương trình đã cho trở thành
		$$t+4\left(1-t^2\right)=1\Leftrightarrow -4t^2+t+3=0 \Leftrightarrow\hoac{&t=1\\&t=-\dfrac{3}{4}\quad (\text{loại}).}$$
		Với $t=1$, ta được 
		$$\sin 2x=0\Leftrightarrow 2x=k\pi\Leftrightarrow x=\dfrac{k\pi}{2}, \quad k \in \mathbb{Z}.$$
		Theo giả thiết $x\in[0;2024\pi]\Rightarrow 0\leq\dfrac{k\pi}{2}\leq 2024\pi\Leftrightarrow 0\leq k\leq 4048\xrightarrow{{k\in\mathbb{Z}}}k\in\left\{0;1;2;3;\ldots;4048\right\}\Rightarrow$ có $4049$ giá trị của $k$ nên có $4049$ nghiệm.}
\end{ex}
\begin{ex}%[1K1K4-0]%[DCHT-11-KNTT]%[Vĩ Lê Văn]
	Từ phương trình $\sqrt{2}\left(\sin x+\cos x\right)=\tan x+\cot x$, ta tìm được $\cos x$ có giá trị bằng
	\choice
	{$1$}
	{$-\dfrac{\sqrt{2}}{2}$}
	{\True $\dfrac{\sqrt{2}}{2}$}
	{$-1$}
	\loigiai{
		Điều kiện $\heva{&\sin x\neq 0\\&\cos x\neq 0}\Leftrightarrow\sin 2x\neq 0$.\\
		Ta có 
		\begin{eqnarray*}
			&&\sqrt{2}\left(\sin x+\cos x\right)=\tan x+\cot x\Leftrightarrow\sqrt{2}\left(\sin x+\cos x\right)=\dfrac{\sin x}{\cos x}+\dfrac{\cos x}{\sin x}\\
			&\Leftrightarrow&\sqrt{2}\left(\sin x+\cos x\right)=\dfrac{\sin^2x+\cos^2x}{\sin x\cos x}\Leftrightarrow 2\sin x\cos x\cdot\sqrt{2}\left(\sin x+\cos x\right)=2.
		\end{eqnarray*}
		Đặt $t=\sin x+\cos x=\sqrt{2}\sin\left(x+ \dfrac{\pi}{4}\right)$, điều kiện $-\sqrt{2}\leq t\leq\sqrt{2}$. \\
		Khi đó $\sin x\cos x=\dfrac{t^2-1}{2}$.\\
		Phương trình trở thành 
		$$\sqrt{2} t\left(t^2-1\right)=2\Leftrightarrow t^3-t-\sqrt{2}=0\Leftrightarrow t=\sqrt{2}.$$
		Khi đó $\sin x+\cos x=\sqrt{2}\Leftrightarrow\sqrt{2}\sin\left(x+ \dfrac{\pi}{4}\right)=\sqrt{2}$\\
		$\Leftrightarrow \sin\left(x+ \dfrac{\pi}{4}\right)=1\Leftrightarrow x=\dfrac{\pi}{4}+k2\pi\Rightarrow \cos x=\dfrac{\sqrt{2}}{2}$.
	}
\end{ex}
\begin{ex}%[1K1K4-0]%[DCHT-11-KNTT]%[Vĩ Lê Văn]
	Từ phương trình $1+\sin^3x+\cos^3x=\dfrac{3}{2}\sin 2x$, ta tìm được $\cos\left(x+\dfrac{\pi}{4}\right)$ có giá trị bằng
	\choice
	{$1$}
	{$-\dfrac{\sqrt{2}}{2}$}
	{$\dfrac{\sqrt{2}}{2}$}
	{\True $\pm\dfrac{\sqrt{2}}{2}$}
	\loigiai{
		Ta có
		\begin{eqnarray*}
			&&1+\sin^3x+\cos^3x=\dfrac{3}{2}\sin 2x\\
			&\Leftrightarrow& 1+\left(\sin x+\cos x\right)\left(1-\sin x\cos x\right)=\dfrac{3}{2}\sin 2x\\
			& \Leftrightarrow& 2+\left(\sin x+\cos x\right)(2-\sin 2x)=3\sin 2x.
		\end{eqnarray*}
		Đặt $t=\sin x+\cos x$,  điều kiện $-\sqrt{2}\leq t\leq\sqrt{2}$.
		Khi đó $\sin x\cos x=\dfrac{t^2-1}{2}\Rightarrow \sin 2x=t^2-1$.\\
		Phương trình trở thành
		\begin{eqnarray*}
			&&2+t\left(2-t^2+1\right)=3\left(t^2-1\right)\\
			&\Leftrightarrow& t^3+3t^2-3t-5=0\\
			&\Leftrightarrow&\hoac{&t=-1\\&t=-1\pm\sqrt{6}\quad(\text{loại}).}
		\end{eqnarray*}
		Với $t=-1$, ta được $\sin x+\cos x=-1\Leftrightarrow\sin\left(x+\dfrac{\pi}{4}\right)=-\dfrac{1}{\sqrt{2}}$.\\
		Mà $\sin^2\left(x+\dfrac{\pi}{4}\right)+\cos^2\left(x+\dfrac{\pi}{4}\right)=1\Rightarrow\cos^2\left(x+\dfrac{\pi}{4}\right)=\dfrac{1}{2}\Leftrightarrow\cos\left(x+\dfrac{\pi}{4}\right)=\pm\dfrac{\sqrt{2}}{2}$.}
\end{ex}
\begin{ex}%[1K1K4-0]%[DCHT-11-KNTT]%[Vĩ Lê Văn]
	Tổng các nghiệm của phương trình $\sin x\cos x+\left|\sin x+\cos x\right|=1$ trên $(0;2\pi)$ bằng
	\choice
	{$\pi$}
	{$2\pi$}
	{\True $3\pi$}
	{$4\pi$}
	\loigiai{
		Đặt $t=\left|\sin x+\cos x\right|\;\left(0\leq t\leq\sqrt{2}\right)$, suy ra $\sin x\cos x=\dfrac{t^2-1}{2}\Leftrightarrow \sin2x=t^2-1$.\\
		Phương trình trở thành
		$$\dfrac{t^2-1}{2}+t=1\Leftrightarrow t^2+2t-3=0\Leftrightarrow\hoac{&t=1\\&t=-3\;(\text{loại}).}$$
		Với $t=1,$ ta được $\sin2x=0\Leftrightarrow 2x=k\pi\Leftrightarrow x=k\dfrac{\pi}{2},\quad  k\in\mathbb{Z}\xrightarrow{x\in(0;2\pi)}x\in\left\{\dfrac{\pi}{2};\pi;\dfrac{3\pi}{2}\right\}$.
		\\
		Vậy tổng các nghiệm của phương trình trên $(0;2\pi)$ bằng $\dfrac{\pi}{2}+\pi+\dfrac{3\pi}{2}=3\pi$.
	}
\end{ex}
\begin{ex}%[1K1K4-0]%[DCHT-11-KNTT]%[Vĩ Lê Văn]
	Cho $x_0$ là nghiệm của phương trình $\sin x\cos x+2\left(\sin x+\cos x\right)=2$ thì giá trị của biểu thức $P=\sin\left(x_0+\dfrac{\pi}{4}\right)$ bằng
	\choice
	{\True $\dfrac{\sqrt{2}}{2}$}
	{$1$}
	{$\dfrac{1}{2}$}
	{$-\dfrac{\sqrt{2}}{2}$}
	\loigiai{
		Đặt $t=\sin x+\cos x=\sqrt{2}\sin \left(x+\dfrac{\pi}{4}\right)\in \left[-\sqrt{2};\sqrt{2}\right]$.\\
		Khi đó $t^2=1+2\sin x\cos x\Rightarrow \sin x\cos x=\dfrac{t^2-1}{2}$.\\
		Phương trình đã cho trở thành $\dfrac{t^2-1}{2}+2t=2\Leftrightarrow t^2+4t-5=0\Leftrightarrow \hoac{&t=1\\&t=-5.}$\\
		Kết hợp điều kiện ta được $t=1$, hay $\sin x+\cos x=1\Rightarrow \sqrt{2}\sin \left(x+\dfrac{\pi}{4}\right)=1$.\\
		Vì $x_0$ là nghiệm của phương trình nên $P=\sin \left(x+\dfrac{\pi}{4}\right)=\dfrac{\sqrt{2}}{2}$.	
	}
\end{ex}

\begin{ex}%[1K1K4-0]%[DCHT-11-KNTT]%[Vĩ Lê Văn]
		Số điểm biểu diễn của nghiệm  phương trình $\sin ^3 x+\cos ^3 x=2\sin x \cos x+\sin x+\cos x$ trên đường tròn lượng giác là 
		\choice
		{\True $4$}
		{$1$}
		{$2$}
		{vô số}
		\loigiai{ 
			$\sin ^3 x+\cos ^3 x=2\sin x \cos x+\sin x+\cos x \Leftrightarrow(\sin x+\cos x)(1-\sin x \cos x)=2\sin x \cos x+\sin x+\cos x$
	
	Đặt $t=\sin x+\cos x=\sqrt{2} \sin \left(x+\dfrac{\pi}{4}\right) \Rightarrow \sin x \cos x=\dfrac{t^2-1}{2}$, điều kiện $-\sqrt{2} \leq t \leq \sqrt{2}$.\\	
	PT trở thành: $t\left(2-t^2+1\right)=2\left(t^2-1\right)+2 t \Leftrightarrow t^3+2t^2-t-2=0\Leftrightarrow\hoac{&t=1 \quad (\text{thoả mãn}) \\& t=-1 \quad (\text{thoả mãn}) \\& t=-2 \quad (\text{loại})}$\\
	$\Leftrightarrow\hoac{&\sqrt{2}\sin \left(x+\dfrac{\pi}{4}\right)=1\\ &\sqrt{2}\sin \left(x+\dfrac{\pi}{4}\right)=-1}\Leftrightarrow \hoac{&\sin \left(x+\dfrac{\pi}{4}\right)=\dfrac{1}{\sqrt{2}} \\&\sin \left(x+\dfrac{\pi}{4}\right)=-\dfrac{1}{\sqrt{2}}}\Leftrightarrow\hoac{&x=k 2\pi \\& x=\dfrac{\pi}{2}+k 2\pi\\&x=-\dfrac{\pi}{2}+k 2\pi \\& x=\pi+k 2\pi}\Leftrightarrow x=k\dfrac{\pi}{2},\quad k \in \mathbb{Z}$.\\	
	Do đó  $x=k\dfrac{\pi}{2},\quad k \in \mathbb{Z}$.\\
Vậy số điểm biểu diễn của nghiệm  phương trình đã cho là $4$.}
\end{ex}
\begin{ex}%[1K1K4-0]%[DCHT-11-KNTT]%[Vĩ Lê Văn]
		Tổng nghiệm âm lớn nhất và nghiệm dương bé nhất của phương trình $1-\sin ^3 x+\cos ^3 x=\sin 2 x$ là 
		\choice
		{\True $-\dfrac{\pi}{2}$}
		{$\dfrac{\pi}{2}$}
		{$\dfrac{\pi}{6}$}
		{$-\dfrac{\pi}{6}$}
		\loigiai{ Ta có 
			  $1-\sin ^3 x+\cos ^3 x=\sin 2 x$
	$PT \Leftrightarrow 1-(\sin x-\cos x)(1+\sin x \cos x)=2\sin x \cos x$.\\	
	Đặt $t=\sin x-\cos x=\sqrt{2} \sin \left(x-\dfrac{\pi}{4}\right) \Rightarrow \sin x \cos x=\dfrac{-t^2+1}{2}$, điều kiện $-\sqrt{2} \leq t \leq \sqrt{2}$.\\	
	PT trở thành: $2-t\left(2+1-t^2\right)=2\left(1-t^2\right) \Leftrightarrow t^3+2t^2-3t=0\Leftrightarrow\hoac{&t=1 \quad (\text{thoả mãn}) \\& t=0 \quad (\text{thoả mãn}) \\& t=-3 \quad (\text{loại})}$\\
	$\Leftrightarrow \hoac{&\sqrt{2}\sin \left(x-\dfrac{\pi}{4}\right)=1\\&\sqrt{2}\sin \left(x-\dfrac{\pi}{4}\right)=0}\Leftrightarrow\hoac{&\sin \left(x-\dfrac{\pi}{4}\right)=\dfrac{1}{\sqrt{2}}\\&\sin \left(x-\dfrac{\pi}{4}\right)=0} \Leftrightarrow\hoac{&x=\dfrac{\pi}{2}+k 2\pi \\& x=\pi+k 2\pi\\&x=\dfrac{\pi}{4}+k \pi}, \quad k \in \mathbb{Z}.$\\
	Do đó $x=-\dfrac{3\pi}{4}$ là nghiệm âm lớn nhất và $x=\dfrac{\pi}{4}$ là nghiệm dương bé nhất của phương trình.\\
Vậy tổng nghiệm âm lớn nhất và nghiệm dương bé nhất của phương trình là $-\dfrac{\pi}{2}$.	}
\end{ex}
\begin{ex}%[1K1K4-0]%[DCHT-11-KNTT]%[Vĩ Lê Văn]
Số điểm biểu diễn của nghiệm  của phương trình 
	$(1+\sin x)(1+\cos x)=2$ trên đường tròn lượng giác là 
		
		\choice
		{$3$}
		{$1$}
		{$0$}
		{\True $2$}
		\loigiai{
	Ta có
	$(1+\sin x)(1+\cos x)=2\Leftrightarrow \sin x \cos x+\sin x+\cos x=1$.\\	
	Đặt $t=\sin x+\cos x=\sqrt{2} \sin \left(x+\dfrac{\pi}{4}\right) \Rightarrow \sin x \cos x=\dfrac{t^2-1}{2}$, điều kiện $-\sqrt{2} \leq t \leq \sqrt{2}$.\\	
	PT trở thành: $t^2-1+2 t=2\Leftrightarrow t^2+2 t-3=0\Leftrightarrow\hoac{&t=1 \quad (\text{thoả mãn}) \\& t=-3\quad(\text {loại.})}$\\	
	Khi đó $\sin x+\cos x=1\Leftrightarrow \sin \left(x+\dfrac{\pi}{4}\right)=\dfrac{1}{\sqrt{2}} \Leftrightarrow\hoac{&x=k 2\pi \\& x=\dfrac{\pi}{2}+k 2\pi}$.\\
Vậy số điểm biểu diễn của nghiệm  của phương trình là $2$.
	}

\end{ex}
\begin{ex}%[1K1K4-A]%[DCHT-11-KNTT]%[Vĩ Lê Văn]
	Giải phương trình $\cos^3x-\sin^3x=\cos 2x$. 
	\choice
	{$\hoac{&x=k2\pi\\&x=\dfrac{\pi}{2}+k\pi\\&x=\dfrac{\pi}{4}+k\pi}, \quad k \in \mathbb{Z}$}
	{$\hoac{&x=k2\pi\\&x=\dfrac{\pi}{2}+k2\pi\\&x=\dfrac{\pi}{4}+k2\pi}, \quad k \in \mathbb{Z}$}
	{\True $\hoac{&x=k2\pi\\&x=\dfrac{\pi}{2}+k2\pi\\&x=\dfrac{\pi}{4}+k\pi}, \quad k \in \mathbb{Z}$}
	{$\hoac{&x=k\pi\\&x=\dfrac{\pi}{2}+k\pi\\&x=\dfrac{\pi}{4}+k\pi}, \quad k \in \mathbb{Z}$}
	\loigiai{
		Ta có\allowdisplaybreaks\begin{eqnarray*}
			&&\cos^3x-\sin^3x=\cos 2x\\
			&\Leftrightarrow&\left(\cos x-\sin x\right)\left(1+\sin x\cos x\right)=\left(\cos x-\sin x\right)\left(\cos x+\sin x\right)\\
			&\Leftrightarrow&\left(\cos x-\sin x\right)\left(1+\sin x\cos x-\cos x-\sin x\right)=0  \\
			& \Leftrightarrow& \hoac{&\cos x-\sin x=0\quad (1)\\&\sin x\cos x-\sin x-\cos x+1=0\quad (2).}
		 \end{eqnarray*}
	 $\sin x-\cos x=0\Leftrightarrow \sqrt{2}\sin\left(x-\dfrac{\pi}{4}\right)=0\Leftrightarrow x=\dfrac{\pi}{4}+k\pi$.\\
	 	Đặt $t=\sin x+\cos x=\sqrt{2} \sin \left(x+\dfrac{\pi}{4}\right) \Rightarrow \sin x \cos x=\dfrac{t^2-1}{2}$, điều kiện $-\sqrt{2} \leq t \leq \sqrt{2}$.\\	
	 PT trở thành: $t^2-1-2 t+2=0\Leftrightarrow t^2-2 t+1=0\Leftrightarrow t=1$\\	
	 Khi đó $\sin x+\cos x=1\Leftrightarrow \sin \left(x+\dfrac{\pi}{4}\right)=\dfrac{1}{\sqrt{2}} \Leftrightarrow\hoac{&x=k 2\pi \\& x=\dfrac{\pi}{2}+k 2\pi}$.\\
	 Vậy phương trình có nghiệm $\hoac{&x=\dfrac{\pi}{4}+k\pi\\&x=k2\pi\\&x=\dfrac{\pi}{2}+k2\pi}, \quad k \in \mathbb{Z}$.  }
\end{ex}
\Closesolutionfile{ans}
\begin{indapan}{10}
	{ans/ans-1K1-4-Dang10}
\end{indapan}

