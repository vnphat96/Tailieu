\begin{dang}{Xét tính chẵn lẻ của hàm số lượng giác}
	\begin{enumerate}[\bf Bước 1.]
		\item Tìm tập xác định $\mathscr{D}$ của hàm số lượng giác.\\
		Nếu $\forall x\in\mathscr{D}$ thì $-x\in\mathscr{D}$, suy ra $\mathscr{D}$ là tập đối xứng và chuyển sang bước tiếp theo.
		\item Tính $f(-x)$, nghĩa là ta sẽ thay $x$ bằng $-x$, sẽ có hai kết quả thường gặp sau:
		\begin{itemize}
			\item Nếu $f(-x)=f(x)$ thì $f(x)$ là hàm số chẵn.
			\item Nếu $f(-x)=-f(x)$ thì $f(x)$ là hàm số lẻ.
		\end{itemize}
	\end{enumerate}
	\begin{note}
		\begin{itemize}
			\item Nếu $\mathscr{D}$ không là tập đối xứng ($\exists x\in\mathscr{D}\Rightarrow -x\not\in\mathscr{D}$) hoặc ($f(-x)\ne f(x)$ và $f(-x)\ne -f(x)$) ta sẽ kết luận hàm số $f(x)$ không chẵn, không lẻ.
			% \item Ta thường sử dụng cung góc liên kết trong dạng toán này, cụ thể
			% \[\cos(-a)=\cos a,\,\sin(-a)=-\sin a,\,\tan(-a)=-\tan a,\,\cot(-a)=-\cot a.\]
			% \item Lũy thừa: $\sin^{2n}(-\alpha)=\sin^{2n}\alpha$, $\cos^{2n}(-\alpha)=\cos^{2n}\alpha$, $\tan^{2n}(-\alpha)=\tan^{2n}\alpha$, $\dots$
			\item Đồ thị của hàm số chẵn nhận trục tung làm trục đối xứng, đồ thị hàm số lẻ nhận gốc tọa độ $O$ làm tâm đối xứng.
		\end{itemize}
	\end{note}
\end{dang}
\setcounter{bt}{0}
\begin{bt}%[Danh Trần - DA2.1]%[1D1B1-3]
	Xét tính chẵn, lẻ của hàm số $f(x)=\sin^2 2x+\cos 3x$.
	\loigiai
	{
		Tập xác định của hàm số $\mathscr{D}=\mathbb{R}$.\\
		Với mọi $x\in\mathscr{D}$ thì $-x\in\mathscr{D}$ nên $\mathscr{D}$ là tập đối xứng.\\
		Ta có $f(-x)=\sin^2(-2x)+\cos(-3x)=\sin^2 2x+\cos 3x=f(x)$, $\forall x\in\mathscr{D}$.\\
		Do đó hàm số $f(x)$ đã cho là hàm số chẵn.
	}
\end{bt}
\begin{bt}%[Danh Trần - DA2.1]%[1D1B1-3]
	Xét tính chẵn, lẻ của hàm số $f(x)=\cos^2 3x+\cos x$.
	\loigiai
	{
		Tập xác định của hàm số $\mathscr{D}=\mathbb{R}$.\\
		Với mọi $x\in\mathscr{D}$ thì $-x\in\mathscr{D}$ nên $\mathscr{D}$ là tập đối xứng.\\
		Ta có $f(-x)=\cos^2(-3x)+\cos(-x)=\cos^2 3x+\cos x=f(x)$, $\forall x\in\mathscr{D}$.\\
		Do đó hàm số $f(x)$ đã cho là hàm số chẵn.
	}
\end{bt}
\begin{bt}%[Danh Trần - DA2.1]%[1D1K1-3]
	Xét tính chẵn, lẻ của hàm số $f(x)=\dfrac{\sin^2 x-\cos x}{\sin 3x}$.
	\loigiai
	{
		Điều kiện xác định $\sin 3x\ne0\Leftrightarrow x\ne\dfrac{k\pi}{3}$, $k\in\mathbb{Z}$.\\
		Tập xác định của hàm số $\mathscr{D}=\mathbb{R}\setminus\left\{\dfrac{k\pi}{3},\,k\in\mathbb{Z}\right\}$.\\
		Với mọi $x\in\mathscr{D}$ thì $-x\in\mathscr{D}$ nên $\mathscr{D}$ là tập đối xứng.\\
		Ta có $f(-x)=\dfrac{\sin^2(-x)-\cos(-x)}{\sin(-3x)}=\dfrac{\sin^2 x-\cos x}{-\sin 3x}=-f(x)$, $\forall x\in\mathscr{D}$.\\
		Do đó hàm số $f(x)$ đã cho là hàm số lẻ.
	}
\end{bt}
\begin{bt}%[Danh Trần - DA2.1]%[1D1K1-3]
	Xét tính chẵn, lẻ của hàm số $f(x)=1+\cos x\cdot\sin\left(\dfrac{3\pi}{2}-2x\right)$.
	\loigiai
	{
		Tập xác định của hàm số $\mathscr{D}=\mathbb{R}$.\\
		Với mọi $x\in\mathscr{D}$ thì $-x\in\mathscr{D}$ nên $\mathscr{D}$ là tập đối xứng.\\
		Ta có $f(x)=1+\cos x\cdot\sin\left(\dfrac{3\pi}{2}-2x\right)=1-\cos x\cdot\sin\left(\dfrac{\pi}{2}-2x\right)=1-\cos x\cdot\cos 2x$.\\
		Khi đó $f(-x)=1-\cos(-x)\cdot\cos(-2x)=1-\cos x\cdot\cos 2x=f(x)$, $\forall x\in\mathscr{D}$.\\
		Do đó hàm số $f(x)$ đã cho là hàm số chẵn.
	}
\end{bt}
% \begin{bt}%[Danh Trần - DA2.1]%[1D1B1-3]
% 	Xét tính chẵn, lẻ của hàm số $f(x)=\cos\sqrt{x^2-16}$.
% 	\loigiai
% 	{
% 		Điều kiện xác định của hàm số là $x^2-16\ge0\Leftrightarrow\hoac{&x\ge4\\&x\le-4.}$\\
% 		Tập xác định của hàm số $\mathscr{D}=\mathbb{R}\setminus(-4;4)$.\\
% 		Với mọi $x\in\mathscr{D}$ thì $-x\in\mathscr{D}$ nên $\mathscr{D}$ là tập đối xứng.\\
% 		Ta có $f(-x)=\cos\sqrt{(-x)^2-16}=\cos\sqrt{x^2-16}=f(x)$, $\forall x\in\mathscr{D}$.\\
% 		Do đó hàm số $f(x)$ đã cho là hàm số chẵn.
% 	}
% \end{bt}
\begin{bt}%[Danh Trần - DA2.1]%[1D1B1-3]
	Xét tính chẵn, lẻ của hàm số $f(x)=\tan x+\cot x$.
	\loigiai
	{
		Điều kiện xác định của hàm số là $\heva{&\sin x\ne0\\&\cos x\ne0}\Leftrightarrow x\ne\dfrac{k\pi}{2}$, $k\in\mathbb{Z}$.\\
		Tập xác định của hàm số $\mathscr{D}=\mathbb{R}\setminus\left\{\dfrac{k\pi}{2},\,k\in\mathbb{Z}\right\}$.\\
		Với mọi $x\in\mathscr{D}$ thì $-x\in\mathscr{D}$ nên $\mathscr{D}$ là tập đối xứng.\\
		Ta có $f(-x)=\tan(-x)+\cot(-x)=-\tan x-\cot x=-f(x)$, $\forall x\in\mathscr{D}$.\\
		Do đó hàm số $f(x)$ đã cho là hàm số lẻ.
	}
\end{bt}
% \begin{bt}%[Danh Trần - DA2.1]%[1D1K1-3]
% 	Xét tính chẵn, lẻ của hàm số $f(x)=\cot(4x+5\pi)\cdot\tan(2x-3\pi)$.
% 	\loigiai
% 	{
% 		Ta có $f(x)=\cot(4x+5\pi)\cdot\tan(2x-3\pi)=\cot 4x\cdot\tan 2x$.\\
% 		Điều kiện xác định của hàm số là $\heva{&\sin 4x\ne0\\&\cos 2x\ne0}\Leftrightarrow\heva{&x\ne\dfrac{k\pi}{4}\\&x\ne\dfrac{\pi}{4}+\dfrac{k\pi}{2}}\Leftrightarrow x\ne\dfrac{k\pi}{4}$, $k\in\mathbb{Z}$.\\
% 		Tập xác định của hàm số $\mathscr{D}=\mathbb{R}\setminus\left\{\dfrac{k\pi}{4},\,k\in\mathbb{Z}\right\}$.\\
% 		Với mọi $x\in\mathscr{D}$ thì $-x\in\mathscr{D}$ nên $\mathscr{D}$ là tập đối xứng.\\
% 		Ta có $f(-x)=\cot(-4x)\cdot\tan(-2x)=\cot 4x\cdot\tan 2x=f(x)$, $\forall x\in\mathscr{D}$.\\
% 		Do đó hàm số $f(x)$ đã cho là hàm số chẵn.
% 	}
% \end{bt}
% \begin{bt}%[Danh Trần - DA2.1]%[1D1K1-3]
% 	Xét tính chẵn, lẻ của hàm số $f(x)=\sin^3(3x+\pi)+\cot(2x-7\pi)$.
% 	\loigiai
% 	{
% 		Ta có $f(x)=\sin^3(3x+\pi)+\cot(2x-7\pi)=-\sin^3 x+\cot 2x$.\\
% 		Điều kiện xác định của hàm số là $\sin 2x\ne0\Leftrightarrow x\ne\dfrac{k\pi}{2}$, $k\in\mathbb{Z}$.\\
% 		Tập xác định của hàm số $\mathscr{D}=\mathbb{R}\setminus\left\{\dfrac{k\pi}{2},\,k\in\mathbb{Z}\right\}$.\\
% 		Với mọi $x\in\mathscr{D}$ thì $-x\in\mathscr{D}$ nên $\mathscr{D}$ là tập đối xứng.\\
% 		Ta có $f(-x)=-\sin^3(-x)+\cot(-2x)=\sin^3 x-\cot 2x=-f(x)$, $\forall x\in\mathscr{D}$.\\
% 		Do đó hàm số $f(x)$ đã cho là hàm số lẻ.
% 	}
% \end{bt}
\begin{bt}%[Danh Trần - DA2.1]%[1D1K1-3]
	Xét tính chẵn, lẻ của hàm số $f(x)=\left|\sin x-\dfrac{1}{2}\right|+\left|\sin x+\dfrac{1}{2}\right|$.
	\loigiai
	{
		Tập xác định của hàm số $\mathscr{D}=\mathbb{R}$.\\
		Với mọi $x\in\mathscr{D}$ thì $-x\in\mathscr{D}$ nên $\mathscr{D}$ là tập đối xứng. Ta có
		\allowdisplaybreaks
		\begin{eqnarray*}
			f(-x)&=&\left|\sin (-x)-\dfrac{1}{2}\right|+\left|\sin (-x)+\dfrac{1}{2}\right|=\left|-\sin x-\dfrac{1}{2}\right|+\left|-\sin x+\dfrac{1}{2}\right|\\
			&=&\left|\sin x+\dfrac{1}{2}\right|+\left|\sin x-\dfrac{1}{2}\right|=f(x),\,\forall x\in\mathscr{D}.
		\end{eqnarray*}
		Do đó hàm số $f(x)$ đã cho là hàm số chẵn.
	}
\end{bt}
\begin{bt}%[Danh Trần - DA2.1]%[1D1K1-3]
	Xét tính chẵn, lẻ của hàm số $f(x)=\dfrac{\sqrt{\cos x+2}+\cot^2 x}{\sin 4x}$.
	\loigiai
	{
		Điều kiện xác định $\heva{&\cos x+2\ge 0\\&\sin x\ne 0\\&\sin 4x\ne 0}\Leftrightarrow\heva{&\cos x\ge -2\\&\sin 4x\ne 0}\Leftrightarrow 4x\ne k\pi\Leftrightarrow x\ne\dfrac{k\pi}{4}$, $k\in\mathbb{Z}$.\\
		Tập xác định của hàm số $\mathscr{D}=\mathbb{R}\setminus\left\{\dfrac{k\pi}{4},\,k\in\mathbb{Z}\right\}$.\\
		Với mọi $x\in\mathscr{D}$ thì $-x\in\mathscr{D}$ nên $\mathscr{D}$ là tập đối xứng.\\
		Ta có $f(-x)=\dfrac{\sqrt{\cos(-x)+2}+\cot^2(-x)}{\sin(-4x)}=\dfrac{\sqrt{\cos x+2}+\cot^2 x}{-\sin 4x}=-f(x)$, $\forall x\in\mathscr{D}$.\\
		Do đó hàm số $f(x)$ đã cho là hàm số lẻ.
	}
\end{bt}