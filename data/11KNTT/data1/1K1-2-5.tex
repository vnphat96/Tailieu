\begin{dang}{Nhận dạng tam giác}
\begin{itemize}
\item Một số lưu ý khi giả thiết cho $A,B,C$ là ba góc của một tam giác
	\begin{itemize}
		\item $A+B+C=180^\circ \Rightarrow (A+B)$ và $C$ bù nhau, tương tự với $(B+C)$ và $A$,...
		\item $\dfrac{A}{2}+\dfrac{B}{2}+\dfrac{C}{2}=90^\circ \Rightarrow \left(\dfrac{A}{2}+\dfrac{B}{2} \right)$ và $\dfrac{C}{2}$ phụ nhau, tương tự với $\left(\dfrac{B}{2}+\dfrac{C}{2} \right)$ và $\dfrac{A}{2}$,...
		\item Các góc $A,B,C$ đều có số đo trong khoảng $\left(0^\circ;180^\circ \right)$.
		\item Các góc $\dfrac{A}{2},\dfrac{B}{2},\dfrac{C}{2}$ đều là các góc nhọn nên có các giá trị lượng giác đều dương.
	\end{itemize}
\item Phương pháp:
	\begin{itemize}
		\item  Biến đổi, dẫn đến $\sin A = 1$ hoặc $\cos A = 0$ sẽ có $A=90^\circ$.
		\item  Nếu $a^2+b^2=c^2$ thì $C=90^\circ$.
		\item  Nếu $\sin (A - B) = 0$ hoặc $\cos (A - B) = 1$ thì $A=B$, suy ra tam giác cân.
		\item Tam giác cân mà có một góc bằng $60^\circ$ là tam giác đều.
	\end{itemize}
\end{itemize}
\end{dang}

\subsubsection{Ví dụ mẫu}
\begin{vd}%[K]%[Câu 2]
Chứng minh rằng $\triangle ABC$ vuông khi $\sin A\sin C = \cos A\cos C$.
\loigiai{Ta có 
		{\allowdisplaybreaks
			\begin{align*}
			&\sin A\sin C = \cos A\cos C \Leftrightarrow \cos A\cos C - \sin A\sin C = 0\\
			\Leftrightarrow& \cos (A + C) = 0 \Leftrightarrow  - \cos B = 0 \Leftrightarrow \cos B = 0 \Leftrightarrow B = {90^\circ}.
			\end{align*}}Vậy tam giác $ABC$ vuông tại $B$.}
\end{vd}

\begin{vd}%[K]%[Câu 3]
Chứng minh rằng $\Delta ABC$ cân khi $2\sin A\sin B = 1 + \cos C$.\hfill$(1)$
\loigiai{Ta có (1) tương đương với
		{\allowdisplaybreaks
			\begin{align*}
			&\cos (A - B) - \cos (A + B) = 1 + \cos C\\
			\Leftrightarrow& \cos (A - B) + \cos C = 1 + \cos C\\
			\Leftrightarrow& \cos (A - B) = 1 \Leftrightarrow A - B = 0 \Leftrightarrow A = B.
			\end{align*}}Vậy tam giác $ABC$ cân tại $C$.}
\end{vd}

\begin{vd}%[G]%[Câu 4]
Tam giác $ABC$ là tam giác gì nếu $\sin A=\dfrac{\sin B+\sin C}{\cos B+\cos C}$ ?
\loigiai{\\
		Ta có $\sin A=\dfrac{\sin B+\sin C}{\cos B+\cos C} \Leftrightarrow \sin A=\dfrac{2\sin \dfrac{B+C}{2} \cos \dfrac{B-C}{2}}{2\cos \dfrac{B+C}{2} \cos \dfrac{B-C}{2}}
		\Leftrightarrow \sin A=\tan \dfrac{B+C}{2}$\\
		$ \Leftrightarrow \sin \left(2\cdot \dfrac{A}{2}\right)=\tan \left(\dfrac{\pi}{2}-\dfrac{A}{2}\right) \Leftrightarrow 2\sin \dfrac{A}{2} \cos \dfrac{A}{2}=\cot \dfrac{A}{2} \Leftrightarrow 2\sin^2\dfrac{A}{2} \cos \dfrac{A}{2}=\cos \dfrac{A}{2}$\\
		Do $0^\circ<\dfrac{A}{2}<90^\circ$ nên $\cos \dfrac{A}{2}\ne 0$ và $\sin \dfrac{A}{2}>0$.\\
		Từ đó
		$2\sin^2\dfrac{A}{2} \cos \dfrac{A}{2}=\cos \dfrac{A}{2} \Leftrightarrow 2\sin^2\dfrac{A}{2}=1 \Leftrightarrow \sin \dfrac{A}{2}=\dfrac{\sqrt{2}}{2} \Leftrightarrow \dfrac{A}{2}=45^\circ \Leftrightarrow A=90^\circ$.\\
		Vậy $ABC$ là tam giác vuông tại $A$.}
\end{vd}
\subsubsection{Bài tập rèn luyện}
% \centerline{\fcolorbox{red}{yellow!50}{\bf {BÀI TẬP TỰ LUẬN }}}
\begin{bt}%[K]%[Câu 5]
Trong tam giác $ABC$, biết: $3\sin A+4\cos B=6$ và $4\sin B+3\cos A=1$. Tính góc $C$.
\loigiai{Bình phương hai vế 2 phương trình rồi cộng lại, ta được: 
$$24(\sin A\cos B+\cos A\sin B)=12 \Leftrightarrow \sin(A+B)=\dfrac12 \Leftrightarrow \sin C=\dfrac12
\Rightarrow \hoac{&C=30^\circ\cr&C=150^\circ.}$$
Nhưng nếu $C=150^\circ \Rightarrow A<30^\circ \Rightarrow 3\sin A+4\cos B<\dfrac32+4<6.$ (Mâu thuẫn). Vậy $C=30^\circ$.}
\end{bt}

\begin{bt}%[G]%[Câu 6]
Chứng minh rằng tam giác $ABC$ đều nếu 
$$\cos A\cos B\cos C=\dfrac18.$$
\loigiai{Ta có đẳng thức đã cho tương đương với
\begin{align*}
&\dfrac12[\cos(A-B)+\cos(A+B)]\cos C=\dfrac18\Leftrightarrow[\cos(A-B)-\cos C]\cos C=\dfrac14\cr
\Leftrightarrow&\dfrac14+\cos^2C-\cos(A-B)\cos C=0\cr
\Leftrightarrow&\cos^2C-\cos(A-B)\cos C+\dfrac{\cos^2(A-B)}4+\dfrac{\sin^2(A-B)}4=0\cr
\Leftrightarrow&\left[\cos C-\dfrac12\cos(A-B)\right]^2+\dfrac14\sin^2(A-B)=0\cr
\Leftrightarrow&\heva{&\sin(A-B)=0\cr&\cos C=\dfrac12\cos(A-B)}
\Leftrightarrow\heva{&A=B\cr&\cos C=\dfrac12}\Leftrightarrow\Delta ABC\text{ đều.}
\end{align*}
}
\end{bt}

\begin{bt}%[K]%[Câu 7]
Chứng minh $\Delta ABC$ cân nếu: $\sin C=2\sin A\sin B\tan\dfrac C2$.
\loigiai{Ta có:
\begin{align*}
\text{Gt }&\Leftrightarrow \left[\cos(A-B)-\cos(A+B)\right]\dfrac{\sin\dfrac C2}{\cos\dfrac C2}=2\sin\dfrac C2\cos\dfrac C2\cr
&\Leftrightarrow \cos(A-B)+\cos C=2\cos^2\dfrac C2=1+\cos C\cr
&\Leftrightarrow \cos(A-B)=1 \Rightarrow A=B.
\end{align*}
Vậy tam giác $ABC$ cân tại $C$.}
\end{bt}

\begin{bt}%[G]%[Câu 9]
Chứng minh điều kiện cần và đủ để $\Delta ABC$ vuông là: $$\sin A = \dfrac{\sin B + \sin C}{\cos B+\cos C}.$$
\loigiai{Ta có $\sin A = \dfrac{\sin B + \sin C}{\cos B+\cos C}\Leftrightarrow \sin A=\dfrac{2\sin\dfrac{B+C}2\cos\dfrac{B-C}2}{2\cos\dfrac{B+C}2\cos\dfrac{B-C}2}$\\
$\Leftrightarrow\sin A=\dfrac{\sin\dfrac{B+C}2}{\cos\dfrac{B+C}2}$ (vì $\cos\dfrac{B-C}2\ne0$)\\
$\Leftrightarrow 2\sin\dfrac A2\cos\dfrac A2=\dfrac{\cos\dfrac A2}{\sin\dfrac A2}$ (vì $\dfrac{B+C}2+\dfrac A2=\dfrac\pi2$)\\
$\Leftrightarrow 2\sin\dfrac A2=\dfrac1{\sin\dfrac A2}$ (vì $\cos\dfrac A2\ne0$)\\
$\Leftrightarrow2\sin^2\dfrac A2=1\Leftrightarrow1-\cos A=1\Leftrightarrow A=\dfrac\pi2$ (vì $0<A<\pi$)\\
$\Leftrightarrow\Delta ABC$ vuông tại $A$.}
\end{bt}

\begin{bt}%[G]%[Câu 9]
Cho $\dfrac{\sin A+\sin B+\sin C}{\sin A+\sin B-\sin C}=\cot\dfrac A2\cot\dfrac B2$. Chứng minh $\Delta ABC$ cân.
\loigiai{Ta có: 
\begin{align*}
&\dfrac{\sin A+\sin B+\sin C}{\sin A+\sin B-\sin C}=\dfrac{2\sin\dfrac{A+B}2\cos\dfrac{A-B}2+2\sin\dfrac C2\cos\dfrac C2}{2\sin\dfrac{A+B}2\cos\dfrac{A-B}2-2\sin\dfrac C2\cos\dfrac C2}\cr
&=\dfrac{2\cos\dfrac C2\left(\cos\dfrac{A-B}2+\cos\dfrac{A+B}2\right)}{2\cos\dfrac C2\left(\cos\dfrac{A-B}2-\cos\dfrac{A+B}2\right)}\cr
&=\dfrac{2\cos\dfrac A2\cos\dfrac B2}{2\sin\dfrac A2\sin\dfrac B2}=\cot\dfrac A2\cot\dfrac B2.
\end{align*}
Do đó, 
$$\cot\dfrac A2\cot\dfrac B2=\cot\dfrac A2\cot\dfrac C2 \Leftrightarrow \cot\dfrac{B}{2}=\cot\dfrac{C}{2} \Leftrightarrow B=C.$$
Vậy tam giác $ABC$ cân đỉnh $A$.}
\end{bt}
\begin{bt}%[G]%[Câu 10]
Chứng minh tam giác $ABC$ vuông nếu: $\sin B+\sin C=\cos B+\cos C$.
\loigiai{Ta có: $\sin B+\sin C=\cos B+\cos C$\\
$\Leftrightarrow 2\sin\dfrac{B+C}2\cos\dfrac{B-C}2=2\cos\dfrac{B+C}2\cos\dfrac{B-C}2$\\
$\Leftrightarrow\cos\dfrac A2=\sin\dfrac A2$ (vì $\cos\dfrac{B-C}2>0$ và $\dfrac{B+C}2=\dfrac\pi2-\dfrac A2$)\\
$\Leftrightarrow \tan\dfrac A2=1\Rightarrow\dfrac A2=\dfrac\pi4$ (vì $0<A<\pi$)\\
$\Rightarrow A=\dfrac\pi2\Rightarrow \Delta ABC$ vuông tại $A$.}
\end{bt}

% \centerline{\fcolorbox{red}{yellow!50}{\bf {CÂU HỎI TRẮC NGHIỆM}}}
% \Opensolutionfile{ans}[ans/ans-1K1-2-Dang6]
% \begin{ex}
% Cho $\Delta ABC$ có các cạnh $ BC=a$, $AC=b$ , $ AB=c$ thỏa mãn hệ thức $\dfrac{1+\cos B}{1-\cos B}=\dfrac{2a+c}{2a-c}$ là tam giác
% \choice
% {\True Cân tại $C$}
% {Vuông tại $B$}
% {Cân tại $A$}
% {Đều}

% \loigiai{Gọi $ R$ là bán kính đường tròn ngoại tiếp $\Delta ABC$. Ta có:\\
% 		$\dfrac{1+\cos B}{1-\cos B}=\dfrac{2a+c}{2a-c}$$\Leftrightarrow\dfrac{1+\cos B}{1-\cos B}=\dfrac{2.2R\sin A+2R\sin C}{2.2R\sin A-2R\sin C}$$\Leftrightarrow\dfrac{1+\cos B}{1-\cos B}=\dfrac{2\sin A+\sin C}{2\sin A-\sin C}$\\
% 		$\Leftrightarrow 2\sin A+2\sin A\cos B-\sin C-\sin C\cos B=2\sin A-2\sin A\cos B+\sin C-\sin C\cos B$\\ $\Leftrightarrow 4\sin A\cos B=2\sin C$
% 		$\Leftrightarrow 4.\dfrac{a}{2R}.\dfrac{a^2+c^2-b^2}{2ac}=2.\dfrac{c}{2R}$
% 		$\Leftrightarrow{a^2}+c^2-b^2=c^2$
% 		$\Leftrightarrow a=b$.\\
% 		Vậy $\Delta ABC$ cân tại $C$.}
% \end{ex}
% \begin{ex}%[0D6K3-6]%[Câu 17]
% 	Tam giác $ ABC $ có $ \cos A = \dfrac{4}{5} $ và $ \cos B = \dfrac{5}{13} $. Khi đó $ \cos C $ bằng
% \choice
% {$ - \dfrac{16}{25}$}
% {$  \dfrac{56}{65}$}
% {\True $  \dfrac{16}{65}$}
% {$ \dfrac{36}{65}$}

% \loigiai{Vì $ 0 < A,B < \pi  $ nên $ \sin A > 0 $, $ \sin B > 0 $. Do đó
% 		\begin{eqnarray*}
% 			\cos C =  - \cos \left( {A + B} \right) &=&  - \left[ {\cos A\cos B - \sin A\sin B} \right] =  - \cos A\cos B + \sqrt {1 - \cos ^2 A} \cdot \sqrt {1 - \cos ^2 B}  \\ 
% 			&=&  - \dfrac{4}{5} \cdot \dfrac{5}{{13}} + \sqrt {1 - \left( {\dfrac{4}{5}} \right)^2 } \cdot \sqrt {1 - \left( {\dfrac{5}{{13}}} \right)^2 }  = \dfrac{{16}}{{65}}.
% 		\end{eqnarray*}}
% \end{ex}
% \begin{ex}%[0D6K3]%[Câu 28]
% Nếu ba góc $A, B, C$ của tam giác $ABC$ thỏa mãn $\sin A = \dfrac{\sin B + \sin C}{\cos B + \cos C}$ thì tam giác này có tính chất gì?
% \choice
% {Không tồn tại tam giác $ABC$}
% {\True Vuông tại $A$}
% {Cân tại $A$ và không đều}
% {Tam giác đều}

% \loigiai{Ta có
% \begin{align*}
% \sin A = \dfrac{\sin B + \sin C}{\cos B + \cos C} & \Leftrightarrow \sin A = \dfrac{2 \sin \left(\dfrac{B+C}{2}\right) \cos \left(\dfrac{B-C}{2}\right)}{2 \cos \left(\dfrac{B+C}{2}\right) \cos \left(\dfrac{B-C}{2}\right)}\\
% & \Leftrightarrow 2 \sin \dfrac{A}{2} \cos \dfrac{A}{2} = \dfrac{\cos \dfrac{A}{2}}{\sin \dfrac{A}{2}} \Leftrightarrow 2 \sin^2 \dfrac{A}{2} = 1 \\
% & \Leftrightarrow \cos A =0 \Leftrightarrow A = 90^{\circ}.
% \end{align*}}
% \end{ex}
% \begin{ex}
% %[0D6K3-6]
% Trong $\Delta ABC$, nếu $\dfrac{{\sin B}}{{\sin C}}=2\cos A$ thì $\Delta ABC$ là tam giác có tính chất nào sau đây?
% \choice
% {\True Cân tại $B$}
% {Cân tại $A$}
% {Cân tại $C$}
% {Vuông tại $B$}

% \loigiai{Ta có $\dfrac{{\sin B}}{{\sin C}}=2\cos A\Rightarrow \sin B=2\sin C\cdot\cos A=\sin \left({C+A}\right)+\sin \left({C-A}\right)$.\\
% Mặt khác $A+B+C=\pi \Rightarrow B=\pi-\left( {A+C} \right)\Rightarrow \sin B=\sin \left({A+C}\right)$. \\
% Do đó, ta được
% $\sin \left({C-A}\right)=0\Rightarrow A=C$.
% }
% \end{ex}
% \begin{ex}
% %[0D6K3-6]
% Trong $\Delta ABC$, nếu $\dfrac{{\tan A}}{{\tan C}}=\dfrac{{{\sin}^2A}}{{{\sin}^2C}}$ thì $\Delta ABC$ là tam giác gì?
% \choice
% {Tam giác vuông}
% {Tam giác cân}
% {Tam giác đều}
% {\True Tam giác vuông hoặc cân}

% \loigiai{Ta có $\dfrac{{\tan A}}{{\tan C}}=\dfrac{{{\sin}^2A}}{{{\sin}^2C}} \Leftrightarrow \dfrac{{\sin A\cos C}}{{\cos A\sin C}}=\dfrac{{{\sin}^2A}}{{{\sin}^2C}}\Leftrightarrow \sin 2C=\sin 2A$ 
% $\Rightarrow \hoac{ &{2C=2A} \\ &{2C=\pi-2A} }\Rightarrow \hoac{ &{C=A} \\ &{A+C=\dfrac{\pi}{2}}.}$
% }
% \end{ex}
