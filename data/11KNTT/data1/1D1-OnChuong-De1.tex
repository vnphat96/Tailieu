\setcounter{deso}{0}
\begin{name}
	{\tenchude}
	{ĐỀ ÔN TẬP CHƯƠNG I}
	{LỚP TOÁN THẦY PHÁT}
	{\thoigian}
\end{name}
\TN
\Opensolutionfile{ans}[ans/ansBONPA-0D1-1-De1]
%C1
\begin{ex}%[Pj31--2-Đề KT Theo Bài--TeamTeXHoa--Khổng Xuân Thạnh]%[1D1N1-1]
	Một cung tròn có độ dài bằng bán kính. Khi đó số đo bằng rađian của cung tròn đó là
	\choice
	{\True $1$}
	{$\pi$}
	{$2$}
	{$3$}
	\loigiai{
	Theo định nghĩa $1$ rađian là số đo của cung có độ dài bằng bán kính.
	}
\end{ex}

%C2
\begin{ex}%[Pj31--2-Đề KT Theo Bài--TeamTeXHoa--Khổng Xuân Thạnh]%[1D1N1-4] 
	Trên đường tròn bán kính bằng $4$, cung có số đo $\dfrac{\pi}{8}$ thì có độ dài là
	\choice
	{\True $\dfrac{\pi}{4}$}
	{$\dfrac{\pi}{3}$}
	{$\dfrac{\pi}{16}$}
	{$\dfrac{\pi}{2}$}
	\loigiai{
Cung có số đo $\alpha$ rad của đường tròn bán kính $R$ có độ dài $l=R\cdot \alpha$.\\
Vậy $\alpha=\dfrac{\pi}{8}$; $R=4$ thì $l=R\cdot \alpha=\dfrac{\pi}{2}$.
	}
\end{ex}

%C3
\begin{ex}%[Pj31--2-Đề KT Theo Bài--TeamTeXHoa--Khổng Xuân Thạnh]%[1D1N1-4] 
Trên đường tròn bán kính $R=6$, cung $60^\circ$  có độ dài bằng bao nhiêu?
	\choice
	{\True $l=\dfrac{\pi}{2}$}
	{$l=4\pi$}
	{$l=2\pi$}
	{$l=\pi$}
	\loigiai{
 Đổi $60^\circ=\dfrac{\pi}{3}$ rad.\\
Ta có cung có số đo $\alpha$ rad của đường tròn có bán kính $R$ có độ dài $l=R\alpha$.\\
Do đó cung $60^\circ$ có độ dài bằng $l=6\cdot\dfrac{\pi}{3}=2\pi$.
	}
\end{ex}
%C4
\begin{ex}%[Pj31--2-Đề KT Theo Bài--TeamTeXHoa--Khổng Xuân Thạnh]%[1D1N1-5] 
	Trên đường tròn lượng giác, điểm $M$ thỏa mãn $(Ox,OM)$ thì nằm ở góc phần tư thứ
	\choice
	{ $I$}
	{$II$}
	{$III$}
	{$IV$}
	\loigiai{
		Điểm $M$ thỏa mãn $(Ox,OM)=500^\circ$ thì nằm ở góc phần tư thứ $II$ vì $500^\circ-360^\circ=140^\circ\in(90^\circ;180^\circ)$.	
	}
\end{ex}
%C5
\begin{ex}%[Pj31--2-Đề KT Theo Bài--TeamTeXHoa--Khổng Xuân Thạnh]%[1D1H1-3]
	Bánh xe của người đi xe đạp quay được $2$ vòng trong $5$ giây. Hỏi trong $1$ giây, bánh xe quay được một góc bao nhiêu độ?
	\choice
	{\True $144^\circ$}
	{$288^\circ$}
	{$36^\circ$}
	{$72^\circ$}
	\loigiai{
	Ta có trong $5$ giây quay được $2\cdot 360^\circ=720^\circ$.\\
	Vậy trong $1$ giây quay được $\dfrac{720^\circ}{5}=144^\circ$.
	}
\end{ex}
%C6
\begin{ex}%[Pj31--2-Đề KT Theo Bài--TeamTeXHoa--Khổng Xuân Thạnh]%[1D1N2-2]
Cho góc $\alpha$ thỏa mãn $2\pi<\alpha<\dfrac{\pi}{2}$. Khẳng định nào sau đây sai?
	\choice
	{$\tan \alpha<0$}
	{$\cot \alpha>0$}
	{$\sin \alpha>0$}
	{$\cos \alpha>0$}
	\loigiai{
	Với $2\pi<\alpha<\dfrac{\pi}{2}$ thì  $\sin \alpha>0$, $\cos \alpha>0$, $\tan \alpha>0$, $\cot \alpha>0$.			
	}
\end{ex}
%C7
\begin{ex}%[Pj31--2-Đề KT Theo Bài--TeamTeXHoa--Khổng Xuân Thạnh]%[1D1H2-2]
	Cho biết $\tan \alpha=\dfrac{1}{2}$. Giá trị $\cot \alpha$ bằng?
	\choice
	{$\cot \alpha=\dfrac{1}{2}$}
	{$\cot \alpha=\sqrt{2}$}
	{$cot \alpha=2$}
	{\True $cot \alpha=\dfrac{1}{4}$}
	\loigiai{
	Ta có $\tan \alpha\cdot \cot \alpha =1 \Leftrightarrow \cot \alpha =\dfrac{1}{\tan \alpha}=2$.
	}
\end{ex}
%C8
\begin{ex}%[Pj31--2-Đề KT Theo Bài--TeamTeXHoa--Khổng Xuân Thạnh]%[1D1N2-2]
	Tìm khẳng định sai trong các khẳng định sau đây?
	\choice
	{$\tan 45^\circ<\tan 60^\circ$}
	{$\cos 45^\circ\leq \sin 45^\circ$}
	{\True $\sin 60\circ<\sin 80^\circ$}
	{$\cos 35^\circ>\cos 10^\circ$}
	\loigiai{
	Khi $\alpha\in(0^\circ;90^\circ)$ thì hàm $\cos \alpha$ là hàm giảm nên $\cos 35^\circ<\cos 10^\circ$. 
	}
\end{ex}
%C9
\begin{ex}%[Pj31--2-Đề KT Theo Bài--TeamTeXHoa--Khổng Xuân Thạnh]%[1D1H2-2]
	Cho $\sin \alpha =\dfrac{1}{3}$ với $\dfrac{\pi}{2}<\alpha<\pi$. Giá trị $\cos \alpha$ bằng?
	\choice
	{$\cos \alpha=\dfrac{2\sqrt{2}}{3}$}
	{$\cos \alpha=-\dfrac{2\sqrt{2}}{3}$}
	{$\cos \alpha=\dfrac{8}{9}$}
	{\True $\cos \alpha=-\dfrac{8}{9}$}
	\loigiai{
	Ta có $\sin^2 \alpha+\cos^2 \alpha=1\Rightarrow \cos^2 \alpha=1-\sin^2\alpha =\dfrac{8}{9}\Rightarrow \cos \alpha =\pm\dfrac{2\sqrt{2}}{3}$.\\
	Vì $\dfrac{\pi}{2}<\alpha<\pi$ nên $\cos \alpha =-\dfrac{2\sqrt{2}}{3}$.
	}
\end{ex}
%C10
\begin{ex}%[Pj31--2-Đề KT Theo Bài--TeamTeXHoa--Khổng Xuân Thạnh]% [1D1N1-5] 
	Biểu diễn các góc lượng giác $\alpha=-\dfrac{5\pi}{6}$, $\beta=\dfrac{\pi}{3}$, $\gamma=\dfrac{25\pi}{3}$, $\sigma=\dfrac{17\pi}{6}$ trên đường tròn lượng giác. Các góc nào có điểm biểu diễn trùng nhau?
	\choice
	{\True $\beta$ và $\gamma$}
	{$\alpha$, $\beta$, $\gamma$}
	{$\beta$, $\gamma$, $\sigma$}
	{$\alpha$ và $\beta$}
	\loigiai{
 Ta có $\gamma=\dfrac{25\pi}{3}=\dfrac{24\pi}{3}+\dfrac{\pi}{3}=4\cdot2\pi+\dfrac{\pi}{3}=\beta +4\cdot2\pi$.\\
Do đó, hai góc $\beta$ và $\gamma$ có điểm biểu diễn trùng nhau.
	}
\end{ex}
%C11
\begin{ex}%[Pj31--2-Đề KT Theo Bài--TeamTeXHoa--Khổng Xuân Thạnh]%[1D1N3-1]
	Câu 11:	Trong các khẳng định sau, khẳng định nào là \textbf{sai}?
	\choice
	{$\sin(\pi-\alpha)=\sin \alpha$}
	{\True $\cos(\pi-\alpha)=\cos \alpha$}
	{$\sin(\pi+\alpha)=-\sin \alpha$}
	{$\cos(\pi+\alpha)=-\cos \alpha$}
	\loigiai{
	Ta có $\cos(\pi-\alpha)=-\cos \alpha$.	
	}
\end{ex}
%C12
\begin{ex}%[Pj31--2-Đề KT Theo Bài--TeamTeXHoa--Khổng Xuân Thạnh]%[1D1H3-2]
	Rút gọn biểu thức $M=\cos(a+b)\cos (a-b)-\sin(a+b)\sin(a-b)$, ta được
	\choice
	{$M=\sin{4a}$}
	{$M=1-2\cos ^2 a$}
	{\True $1-2\sin^2a$}
	{$\cos {4a}$}
	\loigiai{
		Ta có 
		\begin{eqnarray*}
			M=\cos(a+b)\cos(a-b)-\sin(a+b)\sin(a-b)=\cos \left[(a+b)+(a-b)\right]=\cos {2a}=1-2\sin^2a.
		\end{eqnarray*}	
	}
\end{ex}
\Closesolutionfile{ans}
\TNTF
\setcounter{ex}{0}
\Opensolutionfile{ans}[ans/ansDS-0D1-1-De1]
%C13
\begin{ex}%[Pj31--2-Đề KT Theo Bài--TeamTeXHoa--Khổng Xuân Thạnh]% [1D1H1-3]
	Cho số đo $\alpha$ của góc lượng giác $(Ou,Ov)$ với $0\leq\alpha\leq2\pi$,
	\choiceTF
	{\True Biết một góc lượng giác cùng tia đầu, tia cuối với góc $\alpha$ có số đo là $\dfrac{33\pi}{4}$ khi đó $\alpha=\dfrac{\pi}{4}$}
	{Biết một góc lượng giác cùng tia đầu, tia cuối với góc $\alpha$ có số đo là $-\dfrac{291\,983\pi}{3}$ khi đó $\alpha=\dfrac{\pi}{4}$}
	{\True Biết một góc lượng giác cùng tia đầu, tia cuối với góc $\alpha$ có số đo là $-\dfrac{291\,983\pi}{3}$ khi đó $\alpha=\dfrac{\pi}{3}$}
	{Biết một góc lượng giác cùng tia đầu, tia cuối với góc $\alpha$ có số đo là $30$ khi đó $\alpha>5$}
	\loigiai{
		\begin{itemchoice}
			\itemch Mọi góc lượng giác $(Ou,Ov)$ có số đo là $\dfrac{33\pi}{4}+k2\pi$, $k\in\mathbb{Z}$.\\
			Vì $0\leq\alpha\leq2\pi$ nên \[0\leq\dfrac{33\pi}{4}+k2\pi, k\in\mathbb{Z}\Leftrightarrow -\dfrac{33}{8}\leq k\leq-\dfrac{25}{8}, k\in\mathbb{Z}\Leftrightarrow k=-4.\]
			Suy ra $\alpha=\dfrac{33\pi}{4}+(-4)\cdot2\pi=\dfrac{\pi}{4}$.
			\itemch Mọi góc lượng giác $(Ou,Ov)$ có số đo là $-\dfrac{291\,983\pi}{3}+k2\pi$, $k\in\mathbb{Z}$.\\  
			Vì $0\leq2\pi$ nên \[0\leq-\dfrac{291\,983\pi}{3}+k2\pi\leq2\pi,\, k\in\mathbb{Z}\Leftrightarrow \dfrac{291\,983}{6}\leq k\leq\dfrac{291\,989}{6},\,k\in\mathbb{Z}\Leftrightarrow k=48664.\]\\
			Suy ra $\alpha=-\dfrac{291\,983\pi}{3}+48\,664\cdot2\pi=\dfrac{\pi}{3}$.
			\itemch $\alpha=-\dfrac{291\,983\pi}{3}+48\,664\cdot2\pi=\dfrac{\pi}{3}$.
			\itemch Mọi góc lượng giác $(Ou,Ov)$ có số đo là $30+k2\pi$, $k\in\mathbb{Z}$.\\
			Vì $0\leq\alpha\leq2\pi$ nên $0\leq30+k2\pi\leq2\pi$, $k\in\mathbb{Z}\Leftrightarrow0\leq\dfrac{15}{\pi}+k\leq1$, $k\in\mathbb{Z}$\\
			$\Leftrightarrow-\dfrac{15}{\pi}\leq k\leq\dfrac{\pi-15}{\pi}$, $k\in\mathbb{Z}\Leftrightarrow k=-4$.\\
			Suy ra $\alpha=30+(-4)\cdot2\pi=30-8\pi\approx4{,}867$.
		\end{itemchoice}
	}
\end{ex}
%C14
\begin{ex}%[Pj31--2-Đề KT Theo Bài--TeamTeXHoa--Khổng Xuân Thạnh]%[1D1H5-5] 
	Cho phương trình lượng giác $\sin^2 2x+\cos^2 5x=1$, vậy:
	\choiceTF
	{\True Phương trình đã cho tương đương với phương trình $\dfrac{1-\cos4x}{2}+\dfrac{1+\cos10x}{2}=1$}
	{\True Nghiệm dương nhỏ nhất của phương trình là: $x=\dfrac{\pi}{7}$}
	{Nghiệm âm lớn nhất của phương trình nhỏ hơn $-\dfrac{\pi}{3}$}
	{\True Tổng nghiệm âm lớn nhất và nghiệm dương nhỏ nhất bằng $0$}
	\loigiai{
		\begin{itemchoice}
			\itemch Phương trình tương đương với \[\dfrac{1-\cos4x}{2}+\dfrac{1+\cos10x}{2}=1.\]
			\itemch 	Phương trình tương đương với \[\dfrac{1-\cos4x}{2}+\dfrac{1+\cos10x}{2}=1\Leftrightarrow\cos10x=\cos4x\Leftrightarrow\hoac{&10x=4x+k2\pi\\&10x=-4x+k2\pi}\Leftrightarrow\hoac{&x=\dfrac{k\pi}{3}\\&x=\dfrac{k\pi}{7}.}\]
			Vậy nghiệm dương nhỏ nhất của phương trình là $x=\dfrac{\pi}{7}$.
			\itemch Nghiệm âm lớn nhất của phương trình là $x=-\dfrac{\pi}{7}$.
			\itemch  Tổng nghiệm âm lớn nhất và nghiệm dương nhỏ nhất là $-\dfrac{\pi}{7}+\dfrac{\pi}{7}=0$.
		\end{itemchoice}
	}
\end{ex}
%C15
\begin{ex}%[Pj31--2-Đề KT Theo Bài--TeamTeXHoa--Khổng Xuân Thạnh]% [1D1H1-3]
	Cho góc lượng giác $(Ou,Ov)$ có số đo $-\dfrac{\pi}{7}$. Khi đó:
	\choiceTF
	{\True $-\dfrac{29\pi}{7}$ là số đo của một góc lượng giác có cùng tia đầu, tia cuối với góc đã cho}
	{$-\dfrac{22\pi}{7}$ là số đo của một góc lượng giác có cùng tia đầu, tia cuối với góc đã cho}
	{$\dfrac{6\pi}{7}$ là số đo của một góc lượng giác có cùng tia đầu, tia cuối với góc đã cho}
	{\True $\dfrac{41\pi}{7}$ là số đo của một góc lượng giác có cùng tia đầu, tia cuối với góc đã cho}
	\loigiai{
			Hai góc có cùng tia đầu, tia cuối thì sai khác nhau một bội của $2\pi$.
		\begin{itemchoice}
			\itemch Vì $-\dfrac{29\pi}{7}-\left(-\dfrac{\pi}{7}\right)=(-2)\cdot2\pi$. Do đó $-\dfrac{29\pi}{7}$ là số đo của một góc lượng giác có cùng tia đầu, tia cuối với góc đã cho.
			\itemch Vì $-\dfrac{22\pi}{7}-\left(-\dfrac{\pi}{7}\right)=-3\pi$. Do đó $-\dfrac{22\pi}{7}$ là số đo của một góc lượng giác không có cùng tia đầu, tia cuối với góc đã cho.
			\itemch Vì $\dfrac{6\pi}{7}-\left(-\dfrac{\pi}{7}\right)=\pi$. Do đó $\dfrac{6\pi}{7}$ là số đo của một góc lượng giác không có cùng tia đầu, tia cuối với góc đã cho.
			\itemch  Vì $\dfrac{41\pi}{7}-\left(-\dfrac{\pi}{7}\right)=3\cdot2\pi$. Do đó $\dfrac{41\pi}{7}$ là số đo của một góc lượng giác có cùng tia đầu, tia cuối với góc đã cho.
		\end{itemchoice}
	}
\end{ex}
%C16
\begin{ex}%[Pj31--2-Đề KT Theo Bài--TeamTeXHoa--Khổng Xuân Thạnh]%[1D1H5-3]
	Cho phương trình lượng giác $(\sin x+\cos x)^2=2\cos^2 3x$, vậy:
	\choiceTF
	{Phương trình đã cho tương đương với phương trình $1+\sin2x=3+\cos6x$}
	{Nghiệm dương nhỏ nhất của phương trình lớn hơn $\dfrac{\pi}{7}$}
	{\True Nghiệm âm lớn nhất của phương trình là $x=-\dfrac{\pi}{8}$}
	{Tổng nghiệm âm lớn nhất và nghiệm dương nhỏ nhất bằng $0$}
	\loigiai{
		\begin{itemchoice}
			\itemch Ta có \[(\sin x+\cos x)^2=2\cos^2 3x\Leftrightarrow 1+2\sin x\cdot\cos x=1+\cos 6x\Leftrightarrow1+\sin2x=1+\cos6x.\]
			\itemch Ta có 
			\begin{eqnarray*}
				&&(\sin x+\cos x)^2=2\cos^2 3x\Leftrightarrow 1+2\sin x\cdot\cos x=1+\cos 6x\\
				&\Leftrightarrow&1+\sin2x=1+\cos6x\Leftrightarrow\cos6x=\sin2x=\cos\left(\dfrac{\pi}{2}-2x\right)\\
				&\Leftrightarrow&\hoac{&6x=\dfrac{\pi}{2}-2x+k2\pi\\&6x=-\dfrac{\pi}{2}+2x+k2\pi}\Leftrightarrow\hoac{&x=\dfrac{\pi}{16}+\dfrac{k\pi}{4}\\&x=-\dfrac{\pi}{8}+\dfrac{k\pi}{2}}\,(k\in \mathbb{Z}).
			\end{eqnarray*}
		Vậy nghiệm dương nhỏ nhất của phương trình lớn hơn $\dfrac{\pi}{16}$.
			\itemch Nghiệm âm lớn nhất của phương trình là $x=-\dfrac{\pi}{8}$
			\itemch  Tổng nghiệm âm lớn nhất và nghiệm dương nhỏ nhất bằng $\dfrac{\pi}{16}-\dfrac{\pi}{8}=-\dfrac{\pi}{16}$.
		\end{itemchoice}		
	}
\end{ex}
\Closesolutionfile{ans}
\TNSA
\setcounter{ex}{0}
\Opensolutionfile{ans}[ans/ansTLN-0D1-1-De1]
%%%==========Câu 17
\begin{ex}%[Pj31--2-Đề KT Theo Bài--TeamTeXHoa--Khổng Xuân Thạnh]% [1D1H3-3] 
	Với giá trị nào của $n$ thì đẳng thức sau luôn đúng $\sqrt{\dfrac{1}{2}+\dfrac{1}{2}\sqrt{\dfrac{1}{2}+\dfrac{1}{2}\sqrt{\dfrac{1}{2}+\dfrac{1}{2}\cos x}}}=\cos\dfrac{x}{n}$, $0<x<\dfrac{\pi}{2}$.
	\shortans{8}
	\loigiai{
		Vì $0<x<\dfrac{\pi}{2}$ nên $\cos\dfrac{x}{n}>0$, $\forall n\in\mathbb{N}^*$.\\
		$\sqrt{\dfrac{1}{2}+\dfrac{1}{2}\sqrt{\dfrac{1}{2}+\dfrac{1}{2}\sqrt{\dfrac{1}{2}+\dfrac{1}{2}\cos x}}}=\sqrt{\dfrac{1}{2}+\dfrac{1}{2}\sqrt{\dfrac{1}{2}+\dfrac{1}{2}\cos\dfrac{x}{2}}}=\sqrt{\dfrac{1}{2}+\dfrac{1}{2}\cos\dfrac{x}{4}}=\cos\dfrac{x}{8}$.\\
		Vậy $n=8$.
	}
\end{ex}
%%%==========Câu 18
\begin{ex}%[Pj31--2-Đề KT Theo Bài--TeamTeXHoa--Khổng Xuân Thạnh]%[1D1H3-2]
	Cho các góc $\alpha$, $\beta$ thoả mãn $\dfrac{\pi}{2}<\alpha$, $\beta<\pi$, $\sin\alpha=\dfrac{1}{3}$, $\cos\beta=-\dfrac{2}{3}$. Giá trị $\sin(\alpha+\beta)$, kết quả làm tròn đến hàng phần mười.
	\par \shortans{-0{,}9}
	\loigiai{
		Do $\dfrac{\pi}{2}<\alpha$, $\beta<\pi$, suy ra $\heva{&\cos\alpha<0\\&\sin\beta>0}$.\\
		Ta có $\cos\alpha=-\sqrt{1-\sin^2\alpha}=-\sqrt{1-\dfrac{1}{9}}=-\dfrac{2\sqrt{2}}{3}$, $\sin\beta=\sqrt{1-\cos^2\beta}=\sqrt{1-\dfrac{4}{9}}=\dfrac{\sqrt{5}}{3}$.\\
		Suy ra $\sin(\alpha+\beta)=\sin\alpha\cdot\cos\beta+\cos\alpha\cdot\sin\beta=\dfrac{1}{3}\cdot\left(-\dfrac{2}{3}\right)+\left(-\dfrac{2\sqrt{2}}{3}\right)\cdot\dfrac{\sqrt{5}}{3}=-\dfrac{2+2\sqrt{10}}{9}$.\\
		Vậy $\sin(\alpha+\beta)\approx-0{,}9$.
	}
\end{ex}
%%%==========Câu 19
\begin{ex}%[Pj31--2-Đề KT Theo Bài--TeamTeXHoa--Khổng Xuân Thạnh]%[1D1H4-6]
	Tìm giá trị lớn nhất của biểu thức $\sin^4 x+\cos^7 x$.	
	\shortans{1}
	\loigiai{
		Vì $-1\leq\cos x\leq1$, ta có $\sin^4 x+\cos^7 x\leq\sin^4 x+\cos^4 x=1-\dfrac{1}{2}\sin^2 2x\leq1$.\\
		Vậy giá trị lớn nhất của biểu thức $\sin^4 x+\cos^7 x$ là $1$.
	}
\end{ex}
%%%==========Câu 20
\begin{ex}%[Pj31--2-Đề KT Theo Bài--TeamTeXHoa--Khổng Xuân Thạnh]%[1D1V4-6]
		Số giờ có ánh sáng mặt trời của thành phố $A$ trong ngày thứ $t$ (ở đây $t$ là số ngày tính từ ngày $1$ tháng Giêng) của một năm không nhuận được mô hình hóa bởi hàm số $L(t)=12+2{,}76\sin \left(\dfrac{2\pi}{365}(t-90)\right)$ với $t\in\mathbb{Z}$ và $0<t\leq365$. Vào ngày thứ mấy trong năm thì thành phố $A$ có nhiều giờ chiếu sáng nhất. 
		\shortans{181}
	\loigiai{
		Vì $-1\leq \sin \left(\dfrac{2\pi}{365}(t-90)\right)\leq 1$ nên $-2{,}67\leq 2{,}67\sin \left(\dfrac{2\pi}{365}(t-90)\right)\leq 2{,}67$.\\  
		Do đó $9{,}33\leq 12+2{,}67\sin \left(\dfrac{2\pi}{365}(t-90)\right)\leq 14{,}67$.\\
		Ngày thành phố $A$ ứng với nhiều giờ chiếu sáng nhất ứng với 
		\begin{eqnarray*}
			&&\sin \left(\dfrac{2\pi}{365}(t-90)\right)=1\\
			&\Leftrightarrow& \dfrac{2\pi}{365}(t-90)=\dfrac{\pi}{2}+k2\pi\\
			&\Leftrightarrow& t-90=\dfrac{365}{4}+365k\\
			&\Leftrightarrow& t=\dfrac{725}{4}+365k\,(k\in\mathbb{Z}).
		\end{eqnarray*}
		Vì $0<t\leq365$ nên $k=0$, suy ra $t=181{,}25$ .
		Như vậy vào ngày thứ $181$ của năm thì thành phố $A$ sẽ có nhiều giờ ánh sáng mặt trời nhất. 
	}
\end{ex}
%%%==========Câu 21
\begin{ex}%[Pj31--2-Đề KT Theo Bài--TeamTeXHoa--Khổng Xuân Thạnh]%[1D1H3-5] 
	Giá trị của biểu thức $\sin^2 x\cdot\tan^2 x+4\sin^2 x-\tan^2 x+3\cos^2 x$ bằng.
	\par	\shortans{3}
	\loigiai{
		Ta có 
		\[\sin^2 x\cdot\tan^2 x+4\sin^2 x-\tan^2 x+3\cos^2 x=(\sin^2 x-1)\tan^2 x+4\sin^2 x+3\cos^2 x\]
		\[=-\cos^2 x\cdot\tan^2 x+4\sin^2 x+3\cos^2 x=-\sin^2 x+4\sin^2 x+3(1-\sin^2 x)=3.\]	
	}
\end{ex}
%%%==========Câu 22
\begin{ex}%[Pj31--2-Đề KT Theo Bài--TeamTeXHoa--Khổng Xuân Thạnh]%[1D1V1-6] 
\immini[thm]{	
	Khi một tia sáng truyền từ ông khí vào mặt nước thì một phần tia sáng bị phản xạ trên bề mặt, phần còn lại bị khúc xạ như hình vẽ. Góc tới $i$ liên hệ với góc khúc xạ $r$ bởi Định luật khúc xạ ánh sáng $\dfrac{\sin i}{\sin r}=\dfrac{n_2}{n_1}$. Giá trị của $r$ bằng bao nhiêu (kết quả làm tròn đến độ).
\shortans{35}
}{
	\begin{tikzpicture}[>=stealth,line join=round,line cap=round,font=\footnotesize,scale=.7]
		\path
		(0,0)coordinate(I)++(90:3)coordinate(N)++(-90:6)coordinate(N')
		(I)++(0:3)coordinate(B)++(180:6)coordinate(A)
		(I)++(30:3)coordinate(S')
		(I)++(150:3)coordinate(S)
		(I)++(-55:3)coordinate(R)
		(I)++(40:0.8)coordinate(i')
		(I)++(130:0.6)coordinate(i)
		(I)++(-60:0.5)coordinate(r)
		;
		\fill[cyan!20!](-3,-3)rectangle(3,0)
		;
		\draw (A)--(B)
		;
		\draw[dashed](N)--(N')
		;
		\draw[->,midway](S)--(I)
		;
		\draw[->](I)--(S')
		;
		\draw[->](I)--(R)
		;
		\foreach \p/\r in {N/180,N'/180,S/160,S'/90,R/0,I/-135,i'/90,i/90,r/-90}
		\fill (\p) node[shift={(\r:3mm)}]{$\p$}
		;
		\draw pic[angle radius=3mm,draw=red,fill=green!50,angle eccentricity=1.5] {angle = N--I--S}
		;
		\draw pic[angle radius=4mm,draw=orange,fill=orange!50,angle eccentricity=1.5] {angle = S'--I--N}
		;
		\draw pic[angle radius=4mm,draw=blue,fill=blue!50,angle eccentricity=1.5] {angle = N'--I--R}
		;
		\draw (-2.5,.5)circle(7pt)node{$1$}
		(-2.5,-.5)circle(8pt)node{$2$}
		;	
	\end{tikzpicture}
}
	\loigiai{
		Theo bài ra ta có $i=50^\circ$, $n_1=1$, $n_2=1{,}33$, thay vào $\dfrac{\sin i}{\sin r}=\dfrac{n_2}{n_1}$ ta được $\dfrac{\sin50^\circ}{\sin r}=\dfrac{1{,}33}{1}$ (đk $\sin r\ne0$)\\
		Suy ra $\sin r=\dfrac{\sin50^\circ}{1{,}33}\Leftrightarrow\sin r\approx0{,}57597$ (thoả mãn điều kiện)\\
		$\Leftrightarrow\sin r\approx\sin(35^\circ10')\Leftrightarrow\hoac{&r\approx35^\circ10'+k360^\circ\\&r\approx180^\circ-35^\circ10'+k360^\circ}(k\in\mathbb{Z})\Leftrightarrow\hoac{&r\approx35^\circ10'+k360^\circ\\&r\approx144^\circ50'+k360^\circ}(k\in\mathbb{Z})$.\\
		Mà $0^\circ<r<90^\circ$ nên $r\approx35^\circ$.\\
		Vậy góc khúc xạ $r\approx 35^\circ$.
	}
\end{ex}
\Closesolutionfile{ans}
% \begin{center}	
% 	\fontfamily{qag}\selectfont\color{violet} 	\centering{\textbf{BẢNG ĐÁP ÁN}}
% \end{center}
% \inputansbox{12}{ans/ansBONPA-0D1-1-De1}
% \inputansbox{4}{ans/ansDS-0D1-1-De1}
% \inputansbox{6}{ans/ansTLN-0D1-1-De1}


