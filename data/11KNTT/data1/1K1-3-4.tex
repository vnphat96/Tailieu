\begin{dang}{Tìm chu kỳ của hàm số lượng giác}
	\begin{itemize}
		\item Hàm số $y=\sin x$, $y=\cos x$ tuần hoàn với chu kỳ $T_0=2\pi$, nghĩa là $\sin(x+k2\pi)=\sin x$ và $\cos(x+k2\pi)=\cos x$.\\
		Suy ra hàm số $y=\sin(ax+b)$ và $y=\cos(ax+b)$ tuần hoàn với chu kỳ $T_0=\dfrac{2\pi}{|a|}$.
		\item Hàm số $y=\tan x$, $y=\cot x$ tuần hoàn với chu kỳ $T_0=\pi$.\\
		Suy ra hàm số $y=\tan(ax+b)$ và $y=\cot(ax+b)$ tuần hoàn với chu kỳ $T_0=\dfrac{\pi}{|a|}$.
	\end{itemize}
	\begin{note}
		Giả sử hàm số $f(x)=g(x)\pm h(x)$ có hàm $g(x)$ tuần hoàn với chu kỳ $T_1$ và hàm $h(x)$ tuần hoàn với chu kỳ $T_2$. Khi đó hàm số $f(x)$ sẽ tuần hoàn với chu kỳ $T_0$ là bội chung nhỏ nhất của hai chu kỳ $T_1$ và $T_2$.
	\end{note}
\end{dang}
\setcounter{ex}{0}
\Opensolutionfile{ans}[ans/ans-1K1-3-dang4]
\begin{ex}%[THPT Kinh Môn 2 $-$ Hải Dương]%[Danh Trần - DA2.1]%[1D1Y1-4]
	Hàm số $y=\sin 2x$ tuần hoàn với chu kỳ chu kỳ là
	\choice
	{$T_0=2\pi$}
	{$T_0=\dfrac{\pi}{2}$}
	{\True $T_0=\pi$}
	{$T_0=4\pi$}
	\loigiai
	{
		Ta có hàm số $y=\sin 2x$ tuần hoàn với chu kỳ $T_0=\dfrac{2\pi}{2}=\pi$.
	}
\end{ex}
\begin{ex}%[THPT Thạch Thành $-$ Thanh Hóa]%[Danh Trần - DA2.1]%[1D1Y1-4]
	Hàm số $y=\tan 2x$ tuần hoàn với chu kỳ chu kỳ là
	\choice
	{$T_0=\dfrac{\pi}{3}$}
	{\True $T_0=\dfrac{\pi}{2}$}
	{$T_0=2\pi$}
	{$T_0=\pi$}
	\loigiai
	{
		Ta có hàm số $y=\tan 2x$ tuần hoàn với chu kỳ $T=\dfrac{\pi}{2}$.
	}
\end{ex}
\begin{ex}%[THPT Xuân Hòa $-$ Nam Định]%[Danh Trần - DA2.1]%[1D1B1-4]
	Hàm số $y=3\sin\dfrac{x}{2}$ tuần hoàn với chu kỳ chu kỳ là
	\choice
	{$T_0=0$}
	{$T_0=\dfrac{\pi}{2}$}
	{$T_0=2\pi$}
	{\True $T_0=4\pi$}
	\loigiai
	{
		Ta có hàm số $y=3\sin\dfrac{x}{2}$ tuần hoàn với chu kỳ $T=\dfrac{2\pi}{\tfrac{1}{2}}=4\pi$.
	}
\end{ex}
\begin{ex}%[THPT chuyên Hạ Long $-$ Quảng Ninh]%[Danh Trần - DA2.1]%[1D1K1-4]
	Hàm số $f(x)=\sin\dfrac{x}{2}+2\cos\dfrac{3x}{2}$ tuần hoàn với chu kỳ chu kỳ là
	\choice
	{$5\pi$}
	{$\dfrac{\pi}{2}$}
	{$3\pi$}
	{\True $4\pi$}
	\loigiai
	{
		Ta có hàm số $\sin\dfrac{x}{2}$ tuần hoàn với chu kỳ $T_1=\dfrac{2\pi}{\tfrac{1}{2}}=4\pi$, hàm số $\cos\dfrac{3x}{2}$ tuần hoàn với chu kỳ $T_2=\dfrac{2\pi}{\tfrac{3}{2}}=\dfrac{4\pi}{3}$.\\
		Do đó hàm số $f(x)$ tuần hoàn với chu kỳ $T_0$ là bội chung nhỏ nhất của $T_1$ và $T_2$.\\
		Do $T_1$ là bội của $T_2$ $\left(\dfrac{T_1}{T_2}=3\right)$ nên $T_0=T_1$.\\
		Vậy hàm số $f(x)$ đã cho tuần hoàn với chu kỳ $T_0=4\pi$.
	}
\end{ex}
\begin{ex}%[THPT Lê Trọng Tấn $-$ Tp. HCM]%[Danh Trần - DA2.1]%[1D1B1-4]
	Tìm $m$ để hàm số $y=\cos mx$ tuần hoàn với chu kỳ $T_0=\pi$.
	\choice
	{$m=\pm1$}
	{\True $m=\pm2$}
	{$m=\pm\dfrac{\pi}{2}$}
	{$m=\pm\pi$}
	\loigiai
	{
		Ta có hàm số $y=\cos mx$ tuần hoàn với chu kỳ $T=\dfrac{2\pi}{|m|}$.\\
		Để hàm số tuần hoàn với chu kỳ $T_0=\pi$ thì $T=T_0\Leftrightarrow \dfrac{2\pi}{|m|}=\pi\Leftrightarrow |m|=2\Leftrightarrow m=\pm2$.
	}
\end{ex}
\Closesolutionfile{ans}