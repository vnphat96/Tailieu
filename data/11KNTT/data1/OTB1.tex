\section*{BÀI TẬP GTLG}
\setcounter{ex}{0}\setcounter{bt}{0}
\Opensolutionfile{ans}[ans/ans-OTB1]
\begin{ex}%[Dự án Tex Toán 11-new-WTB]%[Đoàn Hùng]%[1K1Y1-2]
Cho hai góc lượng giác có sđ$(Ox,Ou)=-\dfrac{5\pi }{2}+m2\pi,~m\in \mathbb{Z}$ và sđ$(Ox,Ov)=-\dfrac{\pi }{2}+n2\pi,~n\in \mathbb{Z}$. Khẳng định nào sau đây đúng?
\choice
{\True $Ou$ và $Ov$ trùng nhau}
{$Ou$ và $Ov$ đối nhau}
{$Ou$ và $Ov$ vuông góc}
{Tạo với nhau một góc $\dfrac{\pi }{4}$}
\loigiai{
Ta có $(Ox,Ou)-(Ox,Ov)=-\dfrac{5\pi }{2}-\left(-\dfrac{\pi }{2}\right)=-2\pi $.}
\end{ex}

\begin{ex}%[Dự án Toán 11 -WTB-1]%[Hiếu Phan]%[1K1Y1-1]
Tập xác định của hàm số $y=\dfrac{x^2+1}{\cos x}$ là
\choice
{$\mathscr{D}=\mathbb{R}$}
{\True $\mathscr{D}=\mathbb{R}\setminus\left\{\dfrac{\pi}{2}+k\pi ,k\in\mathbb{Z}\right\}$}
{\True $\mathscr{D}=\mathbb{R}\setminus\left\{ k\pi ,k\in\mathbb{Z}\right\}$}
{$\mathscr{D}=\mathbb{R}\setminus\left\{\dfrac{k\pi}{2},k\in\mathbb{Z}\right\}$}
\loigiai{
Điều kiện $\cos x\ne 0\Leftrightarrow x\ne\dfrac{\pi}{2}+k\pi ,\ k\in\mathbb{Z}$.\\
Vậy tập xác định của hàm số là $\mathscr{D}=\mathbb{R}\setminus\left\{\dfrac{\pi}{2}+k\pi ,k\in\mathbb{Z}\right\}$.}
\end{ex}

\begin{ex}%[Dự án Toán 11-WTB-1]%[Vũ Ngọc Phát]%[1K1Y1-6]
Bất đẳng thức nào dưới đây là đúng?
\choice
{$\sin 90^\circ <\sin 150^\circ $}
{$\sin 90^\circ 15'<\sin 90^\circ 30'$}
{\True $\cos 90^\circ 30'>\cos 100^\circ $}
{$\cos 150^\circ >\cos 120^\circ $}
\loigiai{
Các góc trong đề bài đều là góc tù và hàm số $\sin x^\circ$ và $\cos x^\circ$ nghịch biến trên $(90;180)$.\\
Từ đó $\cos 90^\circ 30'>\cos 100^\circ $.
}
\end{ex}

\begin{ex}%[Dự án Tex Toán 11-new-WTB]%[Đoàn Hùng]%[1K1Y1-1]
Khẳng định nào sau đây đúng?
\choice
{$1 ~\mathrm{rad}=1^\circ$}
{\True $1~\mathrm{rad}=\left(\dfrac{180}{\pi }\right)^\circ $}
{$1~\mathrm{rad}=180^\circ$}
{$1~\mathrm{rad}=100^\circ$}
\loigiai{Khẳng định đúng là $1~\mathrm{rad}=\left(\dfrac{180}{\pi }\right)^\circ $.}
\end{ex}

\begin{ex}%[Dự án Tex Toán 11-new-WTB]%[Đoàn Hùng]%[1K1Y1-3]
Trên đường tròn đường kính $8$ cm, tính độ dài cung tròn có số đo bằng $1{,}5$ rad.
\choice
{$12$ cm}
{$4$ cm}
{\True $6$ cm}
{$15$ cm}
\loigiai{
Tính được $l=\alpha \cdot R=1{,}5\cdot \dfrac{8}{2}=6$(cm).}
\end{ex}

\begin{ex}%[Dự án Tex Toán 11-new-WTB]%[Đoàn Hùng]%[1K1Y1-3]
Một đường tròn có bán kính $15$ (cm). Tìm độ dài cung tròn có góc ở tâm bằng $30^\circ$ là
\choice
{\True $\dfrac{5\pi }{2}$}
{$\dfrac{5\pi }{3}$}
{$\dfrac{2\pi }{5}$}
{$\dfrac{\pi }{3}$}
\loigiai{
\[l=\dfrac{\pi a\cdot R}{180}=\dfrac{\pi \cdot 30\cdot 15}{180}=\dfrac{5\pi }{2}.\]}
\end{ex}

\begin{ex}%[Dự án Tex Toán 11-new-WTB]%[Đoàn Hùng]%[1K1Y1-2]
Cho góc lượng giác $\alpha =(OA;OB)=\dfrac{\pi }{5}$. Trong các góc lượng giác sau, góc nào có tia đầu và tia cuối lần lượt trùng với $OA,OB$.
\choice
{$\dfrac{6\pi }{5}$}
{$-\dfrac{11\pi }{5}$}
{\True $\dfrac{31\pi }{5}$}
{$\dfrac{9\pi }{5}$}
\loigiai{
\[\dfrac{31\pi }{5}=\dfrac{\pi }{5}+6\pi =\dfrac{\pi }{5}+3\cdot 2\pi .\] }
\end{ex}

\begin{ex}%[Dự án Toán 11 -WTB-1]%[Hiếu Phan]%[1K1Y1-1]
Tập xác định của hàm số $y=\dfrac{1-\sin x}{\cos x}$ là
\choice
{$\mathscr{D}=\mathbb{R}\setminus\left\{ x\ne k\pi ;k\in\mathbb{Z}\right\}$}
{$\mathscr{D}=\mathbb{R}\setminus\left\{ x\ne k2\pi ;k\in\mathbb{Z}\right\}$}
{\True $\mathscr{D}=\mathbb{R}\setminus\left\{ x\ne\dfrac{\pi}{2}+k\pi ;k\in\mathbb{Z}\right\}$}
{$\mathscr{D}=\mathbb{R}\setminus\left\{ x\ne-\dfrac{\pi}{2}+k2\pi ;k\in\mathbb{Z}\right\}$}
\loigiai{
Hàm số xác định khi $\cos x\ne 0\Leftrightarrow x\ne\dfrac{\pi}{2}+k\pi $ , $k\in\mathbb{Z}$.}
\end{ex}

\begin{ex}%[Dự án Tex Toán 11-new-WTB]%[Đoàn Hùng]%[1K1Y1-2]
Cho hai góc lượng giác có sđ$(Ox,Ou)=45^\circ+m360^\circ,\ m\in \mathbb{Z}$ và sđ$(Ox,Ov)=-135^\circ+n360^\circ,\ n\in \mathbb{Z}$. Ta có hai tia $Ou$ và $Ov$
\choice
{Tạo với nhau góc $45^\circ$}
{Trùng nhau}
{\True Đối nhau}
{Vuông góc}
\loigiai{
Ta có $(Ox,Ou)-(Ox,Ov)=45^\circ-(-135^\circ)=180^\circ$.}
\end{ex}

\begin{ex}%[Dự án Tex Toán 11-new-WTB]%[Đoàn Hùng]%[1K1Y1-3]
Trên đường tròn bán kính $7$ cm, lấy cung có số đo $54^\circ$. Độ dài $l$ của cung tròn bằng
\choice
{\True $\dfrac{21}{10}\pi$(cm)}
{$\dfrac{11}{20}\pi$(cm)}
{$\dfrac{63}{20}\pi$(cm)}
{$\dfrac{20}{11}\pi$(cm)}
\loigiai{
Ta có $l=7\cdot \left(\dfrac{54^\circ}{180^\circ}\cdot \pi \right)=\dfrac{21}{10}\pi$ (cm).}
\end{ex}

\begin{ex}%[Dự án Tex Toán 11-new-WTB]%[Đoàn Hùng]%[1K1Y1-2]
Cho hai góc lượng giác có sđ$(Ox,Ou)=45^\circ+m360^\circ, m\in \mathbb{Z}$ và sđ$(Ox,Ov)=-315^\circ+n360^\circ, n\in \mathbb{Z}$. Ta có hai tia $Ou$ và $Ov$
\choice
{Tạo với nhau góc $45\circ$}
{\True Trùng nhau}
{Đối nhau}
{Vuông góc}
\loigiai{
Ta có $(Ox,Ou)-(Ox,Ov)=45^\circ-(-315^\circ)=360^\circ$.}
\end{ex}

\begin{ex}%[Dự án Toán 11 -WTB-1]%[Hiếu Phan]%[1K1Y1-1]
Tập xác định của hàm số $y=\dfrac{1-\cos x}{\sin x}$ là
\choice
{\True $\mathscr{D}=\mathbb{R}\setminus\left\{ k\pi |k\in\mathbb{Z}\right\}$}
{$\mathscr{D}=\mathbb{R}\setminus\left\{\dfrac{\pi}{2}+k\pi |k\in\mathbb{Z}\right\}$}
{$\mathscr{D}=\mathbb{R}\setminus\left\{ k2\pi |k\in\mathbb{Z}\right\}$}
{$\mathscr{D}=\mathbb{R}\setminus\left\{\dfrac{\pi}{2}+k2\pi |k\in\mathbb{Z}\right\}$}
\loigiai{
Điều kiện $sin x\ne 0\Leftrightarrow x\ne k\pi\,,\,k\in\mathbb{Z}$.\\
Tập xác định$\mathscr{D}=\mathbb{R}\setminus\left\{ k\pi |k\in\mathbb{Z}\right\}$.}
\end{ex}

\begin{ex}%[Dự án Toán 11-WTB-1]%[Võ Thị Thùy Trang]%[1K1Y1-7]
Đơn giản biểu thức $A=\cos \left( \alpha -\dfrac{\pi }{2} \right)$, ta được
\choice
{ $\cos \alpha $}
{\True $\sin \alpha $}
{ $\cos \alpha $}
{ $-\sin \alpha $}
\loigiai{
Ta có $A=\cos \left( \alpha -\dfrac{\pi }{2} \right)=\cos \left( \dfrac{\pi }{2}-\alpha \right)=\sin \alpha $.}
\end{ex}

\begin{ex}%[Dự án Tex Toán 11-new-WTB]%[Đoàn Hùng]%[1K1Y1-3]
Một đường tròn có bán kính $10$, độ dài cung tròn $40^\circ$ trên đường tròn gần bằng
\choice
{\True $7$}
{$9$}
{$11$}
{$13$}
\loigiai{
\[l=\dfrac{\pi a\cdot R}{180}=\dfrac{\pi \cdot 40\cdot 10}{180}=\dfrac{20\pi }{9}\approx 7.\] }
\end{ex}

\begin{ex}%[Dự án Toán 11-WTB-1]%[Vũ Ngọc Phát]%[1K1Y1-5]
Cho góc $\alpha$ có điểm biểu diễn nằm ở góc phần tư (I) của đường tròn lượng giác. Chọn mệnh đề đúng.
\choice
{\True $\sin \alpha >0$}
{$\cos \alpha <0$}
{$\tan \alpha <0$}
{$\cot \alpha <0$}
\loigiai{
Điểm biểu diễn của $\alpha $ thuộc góc phần tư (I) nên $\sin \alpha >0$, $\cos \alpha >0$, $\tan \alpha >0$, $\cot \alpha >0$.
}
\end{ex}

\begin{ex}%[Dự án Toán 11 -WTB-1]%[Hiếu Phan]%[1K1Y1-1]
Điều kiện xác định của hàm số $y=\dfrac{2021-\cos x}{\sin x}$ là
\choice
{$x\ne\dfrac{\pi}{2}+k\pi ,\,k\in\mathbb{Z}$}
{\True $x\ne k\pi ,\,k\in\mathbb{Z}$}
{$x\ne 2k\pi ,\,k\in\mathbb{Z}$}
{$x\ne\dfrac{k\pi}{2},\,k\in\mathbb{Z}$}
\loigiai{
Hàm số đã cho xác định khi $\sin x\ne 0\Leftrightarrow x\ne k\pi,k\in\mathbb{Z}$.}
\end{ex}

\begin{ex}%[Dự án Toán 11-WTB-1]%[Vũ Ngọc Phát]%[1K1Y1-5]
Cho $\dfrac{2021\pi }{4}<x<\dfrac{2023\pi }{4}$. Khẳng định nào sau đây đúng?
\choice
{$\sin x>0$, $\cos 2x>0$}
{$\sin x<0$, $\cos 2x>0$}
{$\sin x>0$, $\cos 2x<0$}
{\True $\sin x<0$, $\cos 2x<0$}
\loigiai{
Ta có $\dfrac{2021\pi }{4}<x<\dfrac{2023\pi }{4}\Leftrightarrow 504\pi +\dfrac{5\pi }{4}<x<504\pi +\dfrac{7\pi }{4}$ nên $\sin x<0$.\\
Lại có $\dfrac{2021\pi }{4}<x<\dfrac{2023\pi }{4}\Leftrightarrow \dfrac{2021\pi }{2}<2x<\dfrac{2023\pi }{2}\Leftrightarrow 1010\pi +\dfrac{\pi }{2}<2x<1010\pi +\dfrac{3\pi }{2}$
nên $\cos 2x<0$.
}
\end{ex}

\begin{ex}%[Dự án Toán 11-WTB-1]%[Vũ Ngọc Phát]%[1K1Y1-5]
Cho $\dfrac{\pi }{2}<\alpha <\pi$. Kết quả đúng là
\choice
{$\sin \alpha >0$, $\cos \alpha >0$}
{$\sin \alpha <0$, $\cos \alpha <0$}
{\True $\sin \alpha >0$, $\cos \alpha <0$}
{$\sin \alpha <0$, $\cos \alpha >0$}
\loigiai{
Vì $\dfrac{\pi }{2}<\alpha <\pi$ nên $\tan \alpha <0$, $\cot \alpha <0$.
}
\end{ex}

\begin{ex}%[Dự án Toán 11 -WTB-1]%[Hiếu Phan]%[1K1Y1-1]
Tập xác định của hàm số $y=\tan x$ là
\choice
{$\mathscr{D}=\mathbb{R}\setminus\left\{ k2\pi ,k\in\mathbb{Z}\right\}$}
{$\mathscr{D}=\mathbb{R}\setminus\left\{\dfrac{\pi}{2}+k2\pi ,k\in\mathbb{Z}\right\}$}
{\True $\mathscr{D}=\mathbb{R}\setminus\left\{\dfrac{\pi}{2}+k\pi ,k\in\mathbb{Z}\right\}$}
{$\mathscr{D}=\mathbb{R}\setminus\left\{ k\pi ,k\in\mathbb{Z}\right\}$}
\loigiai{
Hàm số $y=\tan x$ xác định khi và chỉ khi $\cos x\ne 0$ $\Leftrightarrow x\ne\dfrac{\pi}{2}+k\pi ,k\in\mathbb{Z}.$\\
Vậy TXĐ là $\mathscr{D}=\mathbb{R}\setminus\left\{\dfrac{\pi}{2}+k\pi ,k\in\mathbb{Z}\right\}.$}
\end{ex}

\begin{ex}%[Dự án Toán 11-WTB-1]%[Vũ Ngọc Phát]%[1K1Y1-7]
Cho hai góc nhọn $\alpha $ và $\beta $ phụ nhau. Hệ thức nào sau đây là sai?
\choice
{\True $\sin \alpha =-\cos \beta $}
{$\cos \alpha =\sin \beta $}
{$\cos \beta =\sin \alpha $}
{$\cot \alpha =\tan \beta $}
\loigiai{
Vì hai góc nhọn $\alpha $ và $\beta $ phụ nhau nên $\cos \alpha =\sin \beta $, $\cot \alpha =\tan \beta $ và ngược lại.
}
\end{ex}

\begin{ex}%[Dự án Toán 11-WTB-1]%[Võ Thị Thùy Trang]%[1K1B1-8]
Cho $\cot \alpha =4\tan \alpha$ và $\alpha \in \left(\dfrac{\pi}{2};\pi \right)$. Khi đó $\sin \alpha$ bằng
\choice
{$-\dfrac{\sqrt{5}}{5}$}
{$\dfrac{1}{2}$}
{$\dfrac{2\sqrt{5}}{5}$}
{\True $\dfrac{\sqrt{5}}{5}$}
\loigiai{
Ta có

\begin{eqnarray*}
&&\cot \alpha =4\tan \alpha \Leftrightarrow \dfrac{\cot \alpha}{\tan \alpha}=4\\
& \Leftrightarrow & \cot ^{2} \alpha =4 \Leftrightarrow 1+\cot ^{2} \alpha =5\\
&  \Leftrightarrow & \dfrac{1}{\sin ^{2} \alpha}=5\\
&\Leftrightarrow &\sin ^{2} \alpha =\dfrac{1}{5} \Leftrightarrow \hoac{&\sin \alpha =\dfrac{\sqrt{5}}{5}\\
&\sin \alpha =-\dfrac{\sqrt{5}}{5}.}
\end{eqnarray*}
Vì $\alpha \in \left(\dfrac{\pi}{2};\pi \right)$ nên $\sin \alpha =\dfrac{\sqrt{5}}{5}$.
}
\end{ex}

\begin{ex}%[Dự án Toán 11-WTB-1]%[Vũ Ngọc Phát]%[1K1B1-5]
Cho $a=1500^\circ$. Trong các mệnh đề sau, mệnh đề nào đúng?
\begin{listEX}[3]
\item[I.] $\sin \alpha =\dfrac{\sqrt{3}}{2}$.
\item[II.] $\cos \alpha =\dfrac{1}{2}$.
\item[III.] $\tan \alpha =\sqrt{3}$.
\end{listEX}
\choice
{Chỉ I và II}
{Chỉ II và III}
{\True Cả I, II và III}
{Chỉ I và III}
\loigiai{
Bấm máy ta được $\sin \alpha =\dfrac{\sqrt{3}}{2}$, $\cos \alpha  = \dfrac{1}{2}$, $\tan \alpha =\sqrt{3}$.\\
Cả I, II, III đều đúng
}
\end{ex}

\begin{ex}%[Dự án Tex Toán 11-new-WTB]%[Đoàn Hùng]%[1K1B1-3]
Một đồng hồ treo tường, kim giờ dài $10{,}57$ cm. Trong 30 phút mũi kim giờ vạch lên cung tròn có độ dài là
\choice
{\True $2{,}77$ cm}
{$2{,}78$ cm}
{$2{,}76$ cm}
{$2{,}8$ cm}
\loigiai{
Trong 30 phút mũi kim giờ quét được một góc là $\dfrac{2\pi \cdot 0{,}5}{12}=\dfrac{\pi }{12}\Rightarrow l=\alpha \cdot R=\dfrac{\pi }{12}\cdot 10{,}57\approx 2{,}77$.}
\end{ex}

\begin{ex}%[Dự án Toán 11-WTB-1]%[Vũ Ngọc Phát]%[1K1B1-6]
Nếu $\tan \alpha =\dfrac{3}{4}$ thì $\sin ^2\alpha $ bằng
\choice
{$\dfrac{16}{25}$}
{\True $\dfrac{9}{25}$}
{$\dfrac{25}{16}$}
{$\dfrac{25}{9}$}
\loigiai{
Ta có $\dfrac{1}{\cos ^2\alpha }=1+\tan ^2\alpha =1+\left(\dfrac{3}{4}\right)^2=\dfrac{25}{16}\Rightarrow \cos ^2\alpha =\dfrac{16}{25}\Rightarrow \sin ^2\alpha =1-\cos ^2\alpha =1-\dfrac{16}{25}=\dfrac{9}{25}$.
}
\end{ex}

\begin{ex}%[Dự án Tex Toán 11-new-WTB]%[Đoàn Hùng]%[1K1B1-2]
Sau quãng thời gian $4$ giờ, kim giờ sẽ quay được một góc là
\choice
{$\dfrac{\pi }{3}$}
{\True $\dfrac{2\pi }{3}$}
{$\dfrac{3\pi }{4}$}
{$\dfrac{\pi }{4}$}
\loigiai{
Sau $1$ giờ, kim giờ sẽ quay được một góc là $\dfrac{\pi }{6}$.\\
Sau $4$ giờ, kim giờ sẽ quay được một góc là $\dfrac{\pi }{6}\cdot 4=\dfrac{2\pi }{3}$.}
\end{ex}

\begin{ex}%[Dự án Tex Toán 11-new-WTB]%[Đoàn Hùng]%[1K1B1-4]
Trên đường tròn lượng giác góc $A$ có bao nhiêu điểm $M$ thỏa mãn sđ$\overset\frown{AM}=30^\circ+k45^\circ,\ k\in \mathbb{Z}$?
\choice
{$6$}
{\True $4$}
{$8$}
{$10$}
\loigiai{
Trên đường tròn lượng giác ta có $\widehat{AOM}=30^\circ+k45^\circ,\ k\in \mathbb{Z}$, mà $0<\widehat{AOM}\leqslant 360^\circ\Leftrightarrow 0<30^\circ+k45^\circ\leqslant 360^\circ\Leftrightarrow -\dfrac{2}{3}<k\leqslant \dfrac{22}{3}\Rightarrow $ có $8$ giá trị của $k$.\\
Vậy có $8$ vị trí của $M$ trên đường tròn.}
\end{ex}

\begin{ex}%[Dự án Tex Toán 11-new-WTB]%[Đoàn Hùng]%[1K1B1-2]
Hai góc lượng giác $\dfrac{\pi }{3}$ và $\dfrac{m\pi }{12}$ có cùng tia đầu và tia cuối khi m có giá trị là
\choice
{\True $m=4+24k$}
{$m=4+14k$}
{$m=4+20k$}
{$m=4+22k$}
\loigiai{
Để hai góc lượng giác trùng nhau thì tồn tại một số nguyên $k$ sao cho $\dfrac{m\pi }{12}=\dfrac{\pi }{3}+k2\pi \Leftrightarrow m=4+24k\Leftrightarrow m=4+24k$.}
\end{ex}

\begin{ex}%[Dự án Tex Toán 11-new-WTB]%[Đoàn Hùng]%[1K1B1-4]
Trên đường tròn lượng giác cho ba điểm $A$, $M$, $N$ sao cho số đo cung $AM=\dfrac{\pi }{3}$, số đo cung $AN=\dfrac{3\pi }{4}$. Lấy điểm $P$ trên đường tròn sao cho tam giác $MNP$ cân tại $N$, tìm số đo cung $AP$.
\choice
{$\dfrac{7\pi }{6}+k\pi $}
{\True $\dfrac{7\pi }{6}+k2\pi $}
{$\dfrac{\pi }{3}+k\pi $}
{$\dfrac{\pi }{3}+k2\pi $}
\loigiai{
Ta có sđ$\overset\frown{MN}=\dfrac{5\pi }{12}$.\\
Tam giác $MNP$ cân tại $N\Leftrightarrow NM=NP\Leftrightarrow \text{sđ}\overset\frown{NM}=\text{sđ}\overset\frown{NP}=\dfrac{5\pi }{12}$.\\
Áp dụng hệ thức Chasles: $\text{sđ}(OA,OP)=\text{sđ}(OA,ON)+\text{sđ}(ON,OP)=\text{sđ}\overset\frown{AN}+\text{sđ}\overset\frown{NP}=\dfrac{3\pi }{4}+\dfrac{5\pi }{12}=\dfrac{7\pi }{6}$.\\
Số đo lượng giác $(OA,OP)=\dfrac{7\pi }{6}+k2\pi $.}
\end{ex}

\begin{ex}%[Dự án Tex Toán 11-new-WTB]%[Đoàn Hùng]%[1K1B1-2]
Trên đồng hồ tại thời điểm đang xét kim giờ $OG$ chỉ số $3$, kim phút $OP$ chỉ số $12$. Đến khi kim phút và kim giờ gặp nhau lần đầu tiên, tính số đo góc lượng giác mà kim giờ quét được
\choice
{$\alpha =\dfrac{\pi }{22}+k2\pi $}
{$\alpha =-\dfrac{\pi }{22}+k\pi $}
{$\alpha =\dfrac{\pi }{22}+k\pi $}
{\True $\alpha =-\dfrac{\pi }{22}+k2\pi $}
\loigiai{
Khi kim phút chỉ số $12$, kim giờ chỉ số $3$ thì sđ $(OG,OP)$ là $\dfrac{\pi }{2}+k2\pi $.\\
Trong $1$ giờ, kim phút quét được một góc lượng giác $-2\pi $, kim giờ quét được góc $-\dfrac{\pi }{6}$.\\
Thời gian từ lúc $3$h đến lúc hai kim trùng nhau lần đầu tiên là $\dfrac{\pi }{2}:\left|-2\pi -\left(-\dfrac{\pi }{6}\right)\right|=\dfrac{3}{11}$.\\
Kim giờ đã quét được một góc có số đo là $-\dfrac{\pi }{6}\cdot \dfrac{3}{11}=-\dfrac{\pi }{22}$.\\
Vậy số đo góc lượng giác mà kim phút quét được là $\dfrac{-\pi }{22}+k2\pi $.}
\end{ex}

\begin{ex}%[Dự án Toán 11-WTB-1]%[Võ Thị Thùy Trang]%[1K1B1-7]
Giá trị đúng của biểu thức $F=\sin^2\dfrac{\pi}{6}+\sin^2\dfrac{2\pi}{6}+\cdots+\sin^2\dfrac{5\pi}{6}+\sin^2\pi$ là
\choice
{\True $3$}
{$2$}
{$1$}
{$4$}
\loigiai{
Ta có
\begin{align*}
F&=\sin^2\dfrac{\pi}{6}+\sin^2\dfrac{2\pi}{6}+\cdots+\sin^2\dfrac{5\pi}{6}+\sin^2\pi\\
&=\sin^2\dfrac{\pi}{6}+\sin^2\dfrac{\pi}{3}+\sin^2\dfrac{\pi}{2}+\sin^2\dfrac{2\pi}{3}+\sin^2\dfrac{5\pi}{6}+\sin^2\pi\\
&=2\left(\sin^2\dfrac{\pi}{6}+\cos^2\dfrac{\pi}{3}\right)+1+0=3.
\end{align*}
}
\end{ex}

\begin{ex}%[Dự án Toán 11-WTB-1]%[Vũ Ngọc Phát]%[1K1B1-5]
Cho $3\pi <\alpha <\dfrac{10\pi }{3}$. Chọn mệnh đề đúng?
\choice
{$\cos \alpha >0$}
{\True $\sin \alpha <0$}
{$\tan \alpha <0$}
{$\cot \alpha <0$}
\loigiai{
$3\pi <\alpha <\dfrac{10\pi }{3} \Leftrightarrow 2\pi +\pi <\alpha <2\pi +\pi +\dfrac{\pi }{3}$ nên $\alpha$ có điểm biểu diễn nằm ở góc phần tư thứ (III). Do đó $\sin \alpha <0$, $\cos \alpha <0$, $\tan \alpha >0$, $\cot \alpha >0$.
}
\end{ex}

\begin{ex}%[Dự án Toán 11-WTB-1]%[Chu Hà]%[1K1B1-8]
Rút gọn biểu thức $C=\cos \left(\dfrac{3\pi}{2}-a\right)- \sin \left(\dfrac{3\pi}{2}-a\right) + \cos \left(a- \dfrac{7\pi}{2}\right)- \sin \left(a- \dfrac{7\pi}{2}\right) $, ta được
\choice
{$2\sin a $}
{\True $-2\sin a $}
{$2\cos a $}
{$-2\cos a $}
\loigiai{
$\begin{aligned}[t]
C&=\cos \left(2\pi - \dfrac{\pi}{2}-a\right)- \sin \left(2\pi - \dfrac{\pi}{2}-a\right) + \cos \left(a-4\pi + \dfrac{\pi}{2}\right)- \sin \left(a-4\pi + \dfrac{\pi}{2}\right) \\
&=\cos \left(\dfrac{\pi}{2} + a\right)- \sin \left(- \dfrac{\pi}{2}-a\right) + \cos \left(a + \dfrac{\pi}{2}\right)- \sin \left(a + \dfrac{\pi}{2}\right) \\
&=- \sin a + \cos a- \sin a- \cos a \\
&=-2\sin a.
\end{aligned}$
}
\end{ex}

\begin{ex}%[Dự án Toán 11-WTB-1]%[Vũ Ngọc Phát]%[1K1B1-5]
Cho $0<\alpha <\dfrac{\pi }{2}$. Khẳng định nào sau đây đúng?
\choice
{$\cot \left(\alpha +\dfrac{\pi }{2}\right)>0$}
{$\cot \left(\alpha +\dfrac{\pi }{2}\right)\ge 0$}
{$\tan \left(\alpha +\pi\right)<0$}
{\True $\tan \left(\alpha +\pi\right)>0$}
\loigiai{
Ta có $0<\alpha <\dfrac{\pi }{2} \Rightarrow \dfrac{\pi }{2}<\alpha +\dfrac{\pi }{2}<\pi \Rightarrow \cot \left(\alpha +\dfrac{\pi }{2}\right)<0$.\\
$0<\alpha <\dfrac{\pi }{2} \Rightarrow \pi <\alpha +\pi <\dfrac{3\pi }{2} \Rightarrow \tan \left(\alpha +\pi\right)>0 $.
}
\end{ex}

\begin{ex}%[Dự án Tex Toán 11-new-WTB]%[Đoàn Hùng]%[1K1B1-3]
Bánh xe đạp có bán kính $50$ cm. Một người quay bánh xe $5$ vòng quanh trục thì quãng đường đi được là
\choice
{$250\pi$ cm}
{$1000\pi$ cm}
{\True $500\pi$ cm}
{$200\pi$ cm}
\loigiai{
Ta có $r=50$ cm suy ra $l=50\cdot 2\pi \cdot 5=500\pi$ (cm).}
\end{ex}

\begin{ex}%[Dự án Toán 11-WTB-1]%[Chu Hà]%[1K1B1-8]
Cho $M=5-2\sin^2 x $. Khi đó giá trị lớn nhất của $M $ là
\choice
{$3 $}
{\True $5 $}
{$6 $}
{$7 $}
\loigiai{
Ta có: $\begin{aligned}[t]
&0\le \sin^2 x\le 1, \forall x\in \mathbb{R}\\
\Leftrightarrow& 0\ge -2\sin^2 x\ge -2, \forall x\in \mathbb{R}\\
\Leftrightarrow& 5\ge 5-2\sin^2 x\ge 3, \forall x\in \mathbb{R}.
\end{aligned}$\\
Giá trị lớn nhất của $M$ là $5 $.
}
\end{ex}

\begin{ex}%[Dự án Toán 11-WTB-1]%[Chu Hà]%[1K1K1-8]
Giá trị nhỏ nhất của $M=\sin^4 x + \cos^4 x $ bằng
\choice
{$0 $}
{$\dfrac{1}{4} $}
{\True $\dfrac{1}{2} $}
{$1 $}
\loigiai{
$M=\sin^4 x + \cos^4 x=(\sin^2 x + \cos^2x)^2 - 2 \sin^2 x \cos^2 x =1- \dfrac{1}{2}\sin^2 (2x)$.\\
Ta có $\begin{aligned}[t]
0\le \sin^2(2x) \le 1 &\Leftrightarrow -1\le - \sin^2(2x) \le 0\\
&\Leftrightarrow - \dfrac{1}{2}\le - \dfrac{1}{2}\sin^2(2x) \le 0\\
&\Leftrightarrow 1- \dfrac{1}{2}\le 1- \dfrac{1}{2}\sin^2(2x) \le 1\\
&\Leftrightarrow \dfrac{1}{2}\le 1- \dfrac{1}{2}\sin^2(2x) \le 1.
\end{aligned}$\\
Hay $M \ge \dfrac{1}{2}$. Dấu "=" xảy ra khi $\sin^2(2x) = 0\Leftrightarrow \sin 2x = 0 \Leftrightarrow x=\dfrac{\pi}{4} + k\dfrac{\pi}{2}, k\in \mathbb{Z} $.
}
\end{ex}

\begin{ex}%[Dự án Toán 11-WTB-1]%[Chu Hà]%[1K1K1-8]
Có bao nhiêu đẳng thức đúng trong các đẳng thức sau đây?
\begin{enumEX}{2}
\item $\cos^2 \alpha =\dfrac{1}{\tan^2 \alpha + 1} $.
\item $\sqrt{2}\cos \left(\alpha + \dfrac{\pi}{4}\right)=\cos \alpha + \sin \alpha $.
\item $\sin \left(\alpha - \dfrac{\pi}{2}\right)=- \cos \alpha $.
\item $\cot 2\alpha =2\cot^2 \alpha -1 $.
\end{enumEX}
\choice
{$3 $}
{\True $2 $}
{$4 $}
{$1 $}
\loigiai{
Ta có
\begin{itemize}
\item[$\bullet$] $\dfrac{1}{\cos^2 \alpha}=1 + \tan^2 \alpha \Leftrightarrow \cos^2 \alpha =\dfrac{1}{1 + \tan^2 \alpha} \Rightarrow $ ta có đẳng thức đúng.\\
\item[$\bullet$] $\sin \left(\alpha - \dfrac{\pi}{2}\right)=- \sin \left(\dfrac{\pi}{2}- \alpha\right)=- \cos \alpha \Rightarrow $ đẳng thức đúng.
\item[$\bullet$] $\sqrt{2}\cos \left(\alpha + \dfrac{\pi}{4}\right)=\sqrt{2}\left(\cos \alpha \cos \dfrac{\pi}{4}- \sin \alpha \sin \dfrac{\pi}{4}\right)=\cos \alpha - \sin \alpha $.\\
Vậy $\sqrt{2}\cos \left(\alpha + \dfrac{\pi}{4}\right)=\cos \alpha + \sin \alpha $ \textbf{sai}.
\item[$\bullet$] Với $\cos \alpha =0\Leftrightarrow \sin^2\alpha =1\Rightarrow 2\cot^2 \alpha -1=\dfrac{2\cos^2 \alpha}{\sin^2\alpha}-1=-1 $.\\
$\cot 2\alpha =\dfrac{\cos 2\alpha}{\sin 2\alpha}=\dfrac{\cos 2\alpha}{2\sin \alpha \cos \alpha} $ không xác định khi $\cos \alpha =0 $.\\
Suy ra $\cot 2\alpha =2\cot^2 \alpha -1 $ không đúng với mọi $\alpha $. Vậy $\cot 2\alpha =2\cot^2 \alpha -1 $ \textbf{sai}.
\end{itemize}
Vậy có $2 $ đẳng thức đúng.
}
\end{ex}

\begin{ex}%[Dự án Toán 11-WTB-1]%[Võ Thị Thùy Trang]%[1K1K1-8]
Nếu $\sin x+\cos x=\dfrac{1}{2}$ thì $3\sin x+2\cos x$ bằng
\choice
{\True $\dfrac{5-\sqrt{7}}{4} \text { hay } \dfrac{5+\sqrt{7}}{4}$}
{$\dfrac{5-\sqrt{5}}{7} \text { hay } \dfrac{5+\sqrt{5}}{4}$}
{$\dfrac{2-\sqrt{3}}{5} \text { hay } \dfrac{2+\sqrt{3}}{5}$}
{$\dfrac{3-\sqrt{2}}{5} \text { hay } \dfrac{3+\sqrt{2}}{5}$}
\loigiai{
Ta biến đổi $3\sin x+2\cos x=2\left( \sin x+\cos x\right) +\sin x=1+\sin x$.\\
Từ $\sin x+\cos x=\dfrac{1}{2}\Rightarrow \sin x\cos x=-\dfrac{3}{8}$.\\
Khi đó $\sin x$, $\cos x$ là nghiệm của phương trình $X^2-\dfrac{1}{2}X-\dfrac{3}{8}=0\Leftrightarrow\hoac{& X=\dfrac{1+\sqrt{7}}{4}\\&X=\dfrac{1-\sqrt{7}}{4}}.$
\begin{itemize}
\item Với $\sin x=\dfrac{1+\sqrt{7}}{4}$ suy ra $3\sin x+2\cos x=\dfrac{5+\sqrt{7}}{4}$.
\item Với $\sin x=\dfrac{1-\sqrt{7}}{4}$ suy ra $3\sin x+2\cos x=\dfrac{5-\sqrt{7}}{4}$.
\end{itemize}
}
\end{ex}

\begin{ex}%[Dự án Toán 11-WTB-1]%[Vũ Ngọc Phát]%[1K1K1-6]
Cho $\sin a=\dfrac{1}{3}$. Giá trị của biểu thức $A=\dfrac{\cot a-\tan a}{\tan a+2\cot a}$ bằng
\choice
{$\dfrac{1}{9}$}
{$\dfrac{7}{9}$}
{$\dfrac{17}{81}$}
{\True $\dfrac{7}{17}$}
\loigiai{
Ta có $A=\dfrac{\cot a-\tan a}{\tan a+2\cot a}=\dfrac{\dfrac{\cos a}{\sin a}-\dfrac{\sin a}{\cos a}}{\dfrac{\sin a}{\cos a}+2\dfrac{\cos a}{\sin a}}=\dfrac{\cos ^2a-\sin ^2a}{\sin ^2a+2\cos ^2a}=\dfrac{\left(1-\sin ^2a\right)-\sin ^2a}{\sin ^2a+2\left(1-\sin ^2a\right)}=\dfrac{1-2\sin ^2a}{2-\sin ^2a}=\dfrac{7}{17}$.
}
\end{ex}

\begin{ex}%[Dự án Toán 11-WTB-1]%[Võ Thị Thùy Trang]%[1K1K1-8]
Nếu biết $\dfrac{\sin^4\alpha}{a}+\dfrac{\cos^4\alpha}{b}=\dfrac{1}{a+b}$ thì biểu thức $A=\dfrac{\sin^{8}\alpha}{a^3}+\dfrac{\cos^{8}\alpha}{b^3}$ bằng
\choice
{$\dfrac{1}{\left(a+b\right)^2}$}
{$\dfrac{1}{a^2+b^2}$}
{\True $\dfrac{1}{\left(a+b\right)^3}$}
{$\dfrac{1}{a^3+b^3}$}
\loigiai{
Đặt
$\sin^2x=u, \left(0\le u\le 1\right)$ $\Rightarrow {\cos^2}x=1-u$.\\
Từ $\dfrac{\sin^4\alpha}{a}+\dfrac{\cos^4\alpha}{b}=\dfrac{1}{a+b}$ ta suy ra $\dfrac{u^2}{a}+\dfrac{\left(1-u\right)^2}{b}=\dfrac{1}{a+b}\Rightarrow \dfrac{bu^2+a\left(1-u\right)^2}{ab}=\dfrac{1}{a+b}$.\\
$\dfrac{\left(a+b\right)u^2-2au+a}{ab}=\dfrac{1}{a+b}$ $\Rightarrow {\left(a+b\right)^2}u^2-2a\left(a+b\right)u+a\left(a+b\right)=ab$.\\
$\Rightarrow {\left(a+b\right)^2}u^2-2a\left(a+b\right)u+a^2=0\Rightarrow {\left[\left(a+b\right)u-a\right]^2}=0\Rightarrow u=\dfrac{a}{a+b}$.\\
Suy ra
$\heva{&\sin^2\alpha=\dfrac{a}{a+b} \\&\cos^2\alpha=\dfrac{b}{a+b}.}$\\
Do đó $A=\dfrac{\sin^8\alpha}{a^3}+\dfrac{\cos^8\alpha}{b^3}=\dfrac{\left(\dfrac{a}{a+b}\right)^4}{a^3}+\dfrac{\left(\dfrac{b}{a+b}\right)^4}{b^3}=\dfrac{1}{\left(a+b\right)^3}$.}
\end{ex}

\begin{ex}%[Dự án Toán 11-WTB-1]%[Võ Thị Thùy Trang]%[1K1K1-8]
Nếu $3\cos x+2\sin x=2$ và $\sin x<0$ thì giá trị đúng của $\sin x$ là
\choice
{\True $-\dfrac{5}{13}$}
{$-\dfrac{7}{13}$}
{$-\dfrac{9}{13}$}
{$-\dfrac{12}{13}$}
\loigiai{
Ta có
\allowdisplaybreaks
\begin{eqnarray*}
3\cos x+2\sin x=2&\Leftrightarrow& {\left(3\cos x+2\sin x\right)^2}=4\\
& \Leftrightarrow & 9\cos^2x+12\cos x\cdot \sin x+4\sin^2x=4 \\
& \Leftrightarrow & 5\cos^2x+12\cos x\cdot\sin x=0\\
&\Leftrightarrow & \cos x\left(5\cos x+12\sin x\right)=0\Leftrightarrow \hoac{&\cos x=0 \\&5\cos x+12\sin x=0.}
\end{eqnarray*}
\begin{itemize}
\item 	Với $\cos x=0\Rightarrow \sin x=1$ loại vì $\sin x<0$.
\item  Với $5\cos x+12\sin x=0$, ta có hệ phương trình $\heva{&5\cos x+12\sin x=0 \\&3\cos x+2\sin x=2}\Leftrightarrow \heva{&\sin x=-\dfrac{5}{13} \\&\cos x=\dfrac{12}{13}.}$
\end{itemize}

}
\end{ex}

\begin{ex}%[Dự án Toán 11-WTB-1]%[Chu Hà]%[1K1K1-8]
Giá trị lớn nhất của $M=\sin^6 x- \cos^6 x $ bằng
\choice
{$0 $}
{\True $1 $}
{$2 $}
{$3 $}
\loigiai{
Ta có.\\
$\begin{aligned}
M&=\sin^6 x- \cos^6 x=(\sin^2 x- \cos^2 x)(\sin^4 x + \sin^2 x\cos^2 x + \cos^4 x) \\
& =- \cos 2x(1- \sin^2 x\cdot \cos^2 x)=- \cos 2x\left(1- \dfrac{1}{4}\sin^22x\right) \\
& =- \cos 2x\left(\dfrac{3}{4} + \dfrac{1}{4}\cos^22x\right)\le \dfrac{3}{4} + \dfrac{1}{4}\cos^22x\le \dfrac{3}{4} + \dfrac{1}{4}=1~(\text{do } - \cos 2x\le 1).
\end{aligned} $\\
Nên giá trị lớn nhất của $M$ là $1 $.
}
\end{ex}

\begin{ex}%[Dự án Toán 11 -WTB-1]%[Hiếu Phan]%[1K1K1-1]
Tìm tập xác định của hàm số $y=\tan\left(3x-\dfrac{\pi}{6}\right)$.
\choice
{$\mathscr{D}=\mathbb{R}\setminus\left\{\dfrac{\pi}{3}+\dfrac{k\pi}{3},\,k\in\mathbb{Z}\right\}$}
{$\mathscr{D}=\mathbb{R}\setminus\left\{\dfrac{\pi}{9}+\dfrac{k\pi}{3},\,k\in\mathbb{Z}\right\}$}
{$\mathscr{D}=\mathbb{R}\setminus\left\{\dfrac{4\pi}{9}+\dfrac{k\pi}{3},\,k\in\mathbb{Z}\right\}$}
{\True $\mathscr{D}=\mathbb{R}\setminus\left\{\dfrac{2\pi}{9}+\dfrac{k\pi}{3},\,k\in\mathbb{Z}\right\}$}
\loigiai{
Điều kiện xác định \begin{eqnarray*}
\cos\left(3x-\dfrac{\pi}{6}\right)\ne 0&\Leftrightarrow& 3x-\dfrac{\pi}{6}\ne\dfrac{\pi}{2}+k\pi\\
&\Leftrightarrow& 3x\ne\dfrac{2\pi}{3}+k\pi\Leftrightarrow x\ne\dfrac{2\pi}{9}+\dfrac{k\pi}{3},\,\,k\in\mathbb{Z}
\end{eqnarray*}
Vậy tập xác định của hàm số trên là $\mathscr{D}=\mathbb{R}\setminus\left\{\dfrac{2\pi}{9}+\dfrac{k\pi}{3},\,k\in\mathbb{Z}\right\}$.}
\end{ex}

\begin{ex}%[Dự án Toán 11 -WTB-1]%[Hiếu Phan]%[1K1K1-1]
Tập xác định của hàm số $y=\dfrac{1}{\sqrt{\sin 2x+1}}$ là
\choice
{$\mathscr{D}=\mathbb{R}\setminus\left\{-\dfrac{\pi}{2}+k2\pi |k\in\mathbb{Z}\right\}$}
{$\mathscr{D}=\mathbb{R}\setminus\left\{-\dfrac{\pi}{4}+k2\pi |k\in\mathbb{Z}\right\}$}
{\True $\mathscr{D}=\mathbb{R}\setminus\left\{-\dfrac{\pi}{4}+k\pi |k\in\mathbb{Z}\right\}$}
{$\mathscr{D}=\mathbb{R}$}
\loigiai{
Hàm số $y=\dfrac{1}{\sqrt{\sin 2x+1}}$ xác định khi và chỉ khi \[\sin 2x+1>0\Leftrightarrow\sin 2x>-1\Leftrightarrow\sin 2x\ne-1\Leftrightarrow2x\ne-\dfrac{\pi}{2}+k2\pi\,\,\left(k\in\mathbb{Z}\right)\Leftrightarrow x\ne-\dfrac{\pi}{4}+k\pi\,\left(k\in\mathbb{Z}\right).\]}
\end{ex}

\begin{ex}%[Dự án Toán 11 -WTB-1]%[Hiếu Phan]%[1K1K1-1]
Tìm tập xác định $\mathscr{D}$ của hàm số $y=\dfrac{2\cos x-1}{\sin x}-3\tan x$.
\choice
{\True $\mathscr{D}=\mathbb{R}\setminus\left\{ k\pi ;\dfrac{\pi}{2}+k\pi ,k\in\mathbb{Z}\right\}$}
{$\mathscr{D}=\mathbb{R}\setminus\left\{ k\pi ,k\in\mathbb{Z}\right\}$}
{$\mathscr{D}=\mathbb{R}\setminus\left\{\dfrac{\pi}{2}+k\pi ,k\in\mathbb{Z}\right\}$}
{$\mathscr{D}=\mathbb{R}\setminus\left\{ k\pi ;\dfrac{\pi}{2}+k2\pi ,k\in\mathbb{Z}\right\}$}
\loigiai{
Điều kiện $\left\{\begin{aligned}
&\sin x\ne 0\\
&\cos x\ne 0\\
\end{aligned}\right.\Leftrightarrow\left\{\begin{aligned}
& x\ne k\pi\\
& x\ne\dfrac{\pi}{2}+k\pi\\
\end{aligned}\right.,\left(k\in\mathbb{Z}\right)$.\\
Tập xác định $\mathscr{D}=\mathbb{R}\setminus\left\{ k\pi ;\dfrac{\pi}{2}+k\pi ,k\in\mathbb{Z}\right\}$.}
\end{ex}

\begin{ex}%[Dự án Toán 11 -WTB-1]%[Hiếu Phan]%[1K1G1-1]
Tìm tập xác định $\mathscr{D}$ của hàm số $y=\sqrt{5+2\cot^2x-\sin x}+\cot\left(\dfrac{\pi}{2}+x\right)$.
\choice
{\True $\mathscr{D}=\mathbb{R}\setminus\left\{\dfrac{k\pi}{2},k\in\mathbb{Z}\right\}$}
{$\mathscr{D}=\mathbb{R}\setminus\left\{-\dfrac{\pi}{2}+k\pi ,k\in\mathbb{Z}\right\}$}
{$\mathscr{D}=\mathbb{R}$}
{$\mathscr{D}=\mathbb{R}\setminus\left\{ k\pi ,k\in\mathbb{Z}\right\}$}
\loigiai{
Hàm số xác định khi và chỉ khi các điều kiện sau thỏa mãn đồng thời
\[5+2\cot^2x-\sin x\ge 0 , \cot\left(\dfrac{\pi}{2}+x\right) \text{xác định và }  \cot x  \text{ xác định}.\]
\begin{itemize}
\item Ta có $ \heva{&  2\cot^2x\ge 0 \\ & -1\le\sin x\le 1\Rightarrow 5-\sin x\ge 0} \Rightarrow 5+2\cot^2x-\sin x\ge 0,\forall x\in\mathbb{R}$.
\item $\cot\left(\dfrac{\pi}{2}+x\right)$ xác định $\Leftrightarrow\sin\left(\dfrac{\pi}{2}+x\right)\ne 0\Leftrightarrow\dfrac{\pi}{2}+x\ne k\pi\Leftrightarrow x\ne-\dfrac{\pi}{2}+k\pi ,k\in\mathbb{Z}$.
\item $\cot x$ xác định $\Leftrightarrow\sin x\ne 0\Leftrightarrow x\ne k\pi ,k\in\mathbb{Z}$.\\
Do đó hàm số xác định $ \Leftrightarrow\heva{& x\ne-\dfrac{\pi}{2}+k\pi  \\ & x\ne k\pi} \Leftrightarrow x\ne\dfrac{k\pi}{2},k\in\mathbb{Z}.$
\end{itemize}
Vậy tập xác định $\mathscr{D}=\mathbb{R}\setminus\left\{\dfrac{k\pi}{2},k\in\mathbb{Z}\right\}.$
}
\end{ex}

\begin{ex}%[Dự án Toán 11-WTB-1]%[Chu Hà]%[1K1G1-8]
Tìm hệ thức \textbf{sai} trong bốn hệ thức sau
\choice
{$\dfrac{\tan x + \tan y}{\cot x + \cot y}=\tan x + \tan y $}
{$\left(\sqrt{\dfrac{1 + \sin \alpha}{1- \sin \alpha}}- \sqrt{\dfrac{1- \sin \alpha}{1 + \sin \alpha}}\right)^2 =4\tan^2 \alpha $}
{$\dfrac{\sin \alpha}{\cos \alpha + \sin \alpha}- \dfrac{\sin \alpha}{\cos \alpha - \sin \alpha}=\dfrac{2}{1- \cot^2 \alpha} $}
{\True $\dfrac{\sin \alpha + \cos \alpha}{1- \cos \alpha}=\dfrac{2\cos \alpha}{\sin \alpha - \cos \alpha + 1} $}
\loigiai{
\begin{itemize}
\item[$\bullet$] $\dfrac{\tan x + \tan y}{\cot x + \cot y}
=\dfrac{\dfrac{\sin x}{\cos x} + \dfrac{\sin y}{\cos y}}{\dfrac{\cos x}{\sin x} + \dfrac{\cos y}{\sin y}}
=\dfrac{\dfrac{\sin x\cos y + \sin y \cos x}{\cos x\cos y}}{\dfrac{\sin y\cos x + \sin x\cos y}{\sin x\sin y}}
=\dfrac{\sin x\sin y}{\cos x\cos y}
=\tan x\tan y $.\\
\item[$\bullet$] $\begin{aligned}[t]
\left(\sqrt{\dfrac{1 + \sin \alpha}{1- \sin \alpha}}- \sqrt{\dfrac{1- \sin \alpha}{1 + \sin \alpha}}\right)^2
=&\left(\sqrt{\dfrac{(1 + \sin \alpha)(1 + \sin \alpha)}{\cos^2\alpha}}- \sqrt{\dfrac{(1- \sin \alpha)(1- \sin \alpha)}{\cos^2\alpha}}\right)^2 \\
=&\left(\sqrt{\dfrac{(1 + \sin \alpha)^2}{\cos^2\alpha}}
- \sqrt{\dfrac{(1- \sin \alpha)^2}{\cos^2\alpha}}\right)^2\\
=&\left(\dfrac{1}{|\cos \alpha |}(| 1 + \sin \alpha |-| 1- \sin \alpha |)\right)^2 \\
=&\dfrac{1}{\cos^2\alpha}{(| 1 + \sin \alpha |-| 1- \sin \alpha |)^2}\\
=&\dfrac{4\sin^2\alpha}{\cos^2\alpha}\\
=&4\tan^2 \alpha.
\end{aligned} $
\item[$\bullet$] $\dfrac{\sin \alpha}{\cos \alpha + \sin \alpha}- \dfrac{\sin \alpha}{\cos \alpha - \sin \alpha}
=\dfrac{2\sin^2\alpha}{\cos^2\alpha - \sin^2\alpha}
=\dfrac{2}{1- \cot^2 \alpha}$.
\item[$\bullet$] $\begin{aligned}[t]
\dfrac{\cos \alpha + \sin \alpha}{1- \cos \alpha}- \dfrac{2\cos \alpha}{\sin \alpha - \cos \alpha + 1}
=&\dfrac{(\sin^2\alpha - \cos^2\alpha) + \cos \alpha + \sin \alpha -2\cos \alpha + 2\cos^2\alpha}
{(1- \cos \alpha)(\sin \alpha - \cos \alpha + 1)} \\
=&\dfrac{(\sin^2\alpha + \cos^2\alpha) + (\sin \alpha - \cos \alpha)}{(1- \cos \alpha)(\sin \alpha - \cos \alpha + 1)}\\
=&\dfrac{1}{1- \cos \alpha}\ne 0.
\end{aligned} $
\end{itemize}
}
\end{ex}

\begin{ex}%[Dự án Toán 11-WTB-1]%[Chu Hà]%[1K1G1-8]
Giả sử $3\sin^4x- \cos^4x=\dfrac{1}{2} $ thì $\sin^4x + 3\cos^4x $ có giá trị bằng
\choice
{\True $1 $}
{$2 $}
{$3 $}
{$4 $}
\loigiai{
Ta có $\sin^2 x + \cos^2 x=1 \Rightarrow \cos^2 x=1- \sin^2 x $.\\
Vậy $\begin{aligned}[t]
3\sin^4x- \cos^4x=\dfrac{1}{2}
\Leftrightarrow& 3\sin^4x-(1- \sin^2 x)^2=\dfrac{1}{2}\\
\Leftrightarrow& 3\sin^4x-(1-2\sin^2 x + \sin^4 x)=\dfrac{1}{2}\\
\Leftrightarrow& 3\sin^4x-1 + 2\sin^2 x- \sin^4 x=\dfrac{1}{2}\\
\Leftrightarrow& 2\sin^4x + 2\sin^2 x- \dfrac{3}{2}=0\\
\Leftrightarrow& \left(\sin^2 x - \dfrac{1}{2}\right)
\left(\sin^2 x + \dfrac{3}{2}\right)=0\\
\Leftrightarrow& \sin^2 x=\dfrac{1}{2} ~ (\do -1 \leq \sin x \leq 1) \\
\Leftrightarrow& \sin x=\pm \dfrac{1}{\sqrt{2}}.
\end{aligned} $. \\
Vậy $\sin^4x + 3\cos^4x =\sin^4x + 3 (1- \sin^2 x)^2 =\dfrac{1}{4} + 3{{\left(1- \dfrac{1}{2}\right)}^2} =\dfrac{1}{4} + \dfrac{3}{4} =1$.
}
\end{ex}

\begin{ex}%[Dự án Toán 11-WTB-1]%[Chu Hà]%[1K1G1-8]
Cho biểu thức $M=\dfrac{1 + \tan^3x}{(1 + \tan x)^3}, \left(x\ne - \dfrac{\pi}{4} + k\pi; x\ne \dfrac{\pi}{2} + k\pi; k\in \mathbb{Z}\right) $, trong các mệnh đề sau, mệnh đề nào \textbf{đúng}?
\choice
{$M<1 $}
{$M\le 1 $}
{\True $M\ge \dfrac{1}{4} $}
{$\dfrac{1}{4}\le M\le 1 $}
\loigiai{
Đặt $ t = \tan x, t\in \mathbb{R}\setminus \{ -1 \} $.\\
Ta có $M=\dfrac{1 + t^3}{(1 + t)^3}=\dfrac{t^2-t + 1}{t^2 + 2t + 1} \Rightarrow (M-1)t^2 + (2M + 1)t + M-1=0 $.\\
Với $M=1 $ thì có nghiệm $t=0 $.\\
Với $M\ne 1 $ để phương trình có nghiệm khác $-1 $ thì
$\begin{aligned}[t]
&\Delta \ge 0\\
\Leftrightarrow& (2M-1)^2-4 (M-1)^2\ge 0\\
\Leftrightarrow& 12M-3\ge 0\\
\Leftrightarrow& M\ge \dfrac{1}{4}.
\end{aligned}$\\
Và $(M-1)(-1)^2 + (2M + 1)(-1) + (-1)-1\ne 0\Leftrightarrow M\ne -4 $.
}
\end{ex}

\begin{ex}%[Dự án Toán 11-WTB-1]%[Chu Hà]%[1K1G1-8]
Cho biểu thức $P=\sin \left(\alpha + \dfrac{\pi}{2}\right) + \cos (3\pi -2\alpha) + \cot (\pi - \alpha) $. Tính giá trị của $P$ biết $\sin \alpha =- \dfrac{1}{2} $ và $- \dfrac{\pi}{2}<\alpha <0 $.
\choice
{\True $\dfrac{3\sqrt{3}-1}{2} $}
{$\dfrac{3\sqrt{3}-3}{2} $}
{$\dfrac{3\sqrt{3} + 3}{2} $}
{$\dfrac{3\sqrt{3} + 1}{2} $}
\loigiai{
Ta có: $\begin{aligned}[t]
P=&\sin \left(\alpha + \dfrac{\pi}{2}\right) + \cos (3\pi -2\alpha) + \cot (\pi - \alpha)=\cos (- \alpha)- \cos (-2\alpha) + \cot (- \alpha) \\
=&\cos \alpha - \cos 2\alpha - \cot \alpha =\cos \alpha -(2\cos^2\alpha -1)- \cot \alpha.
\end{aligned} $ \\
Mặt khác $\cos^2\alpha =1- \sin^2\alpha
=1- \left(- \dfrac{1}{2}\right)^2=\dfrac{3}{4} $
mà $- \dfrac{\pi}{2}<\alpha <0 $ nên $\cos \alpha =\dfrac{\sqrt 3}{2} $.\\
Suy ra $\cot \alpha =\dfrac{\cos \alpha}{\sin \alpha}=- \sqrt{3} $.\\
Do đó $P=\cos \alpha -(2\cos^2\alpha -1)- \cot \alpha
=\dfrac{\sqrt{3}}{2}- \left(2\cdot \dfrac{3}{4}-1\right) + \sqrt{3}=\dfrac{3\sqrt{3}-1}{2}$.
}
\end{ex}
\Closesolutionfile{ans}