
\begin{dang}{Điều kiện có nghiệm của phương trình lượng giác cơ bản}
	\begin{itemize}
		\item $\sin x=a$ có tập giá trị $|a|\le 1$.
		\item $\cos x=b$ có tập giá trị $|b|\le 1$.
	\end{itemize}
\end{dang}
\subsubsection{Ví dụ}
\begin{vd}%[NB]%[DCHT Toán 11 - KNTT -Tên GV] %[ID6 chương trình mới]
	Tìm tất cả các giá trị thực của tham số $m$ để phương trình $\sin x=m$ có nghiệm.\dapso{$-1\le m\le 1$.}
	\loigiai{Phương trình $\sin x=m$ có nghiệm $\Leftrightarrow -1\le m\le 1$.}
\end{vd}
\begin{vd}%[TH]%[DCHT Toán 11 - KNTT -Tên GV] %[ID6 chương trình mới]
	Tìm tất cả các giá trị của tham số $m$ để phương trình $\sin x-m=1$ có nghiệm.
	\dapso{$-2\le m\le 0$.}
	\loigiai{
	Ta có $\sin x-m=1\Leftrightarrow \sin x=m+1$.\\
	Vì $-1\le \sin x\le 1$ nên phương trình $\sin x=m+1$ có nghiệm khi:$-1\le m+1\le 1\Leftrightarrow -2\le m\le 0$.
	}
\end{vd}
\begin{vd}%[TH]%[DCHT Toán 11 - KNTT -Tên GV] %[ID6 chương trình mới]
Tìm tất cả các giá trị của tham số $m$ để phương trình $3\sin^2x=2m-1$ có nghiệm.
	\dapso{$\dfrac{1}{2}\le m\le 2$.}
	\loigiai{
	Ta có $3\sin^2x=2m-1\Leftrightarrow{\sin^2}x=\dfrac{2m-1}{3}$.\\
	Để phương trình có nghiệm thì $\heva{&\dfrac{2m-1}{3}\le 1 \\&\dfrac{2m-1}{3}\ge 0}\Leftrightarrow \heva{&m\le 2 \\&m\ge \dfrac{1}{2}}\Leftrightarrow \dfrac{1}{2}\le m\le 2$.
	}
\end{vd}

\begin{vd}%[NB]%[DCHT Toán 11 - KNTT -Tên GV] %[ID6 chương trình mới]
	Tìm $m$ để phương trình $\cos x-m=0$ vô nghiệm.
	\dapso{$\hoac{&m<-1 \\&m>1}$.}
	\loigiai{
		Phương trình $\cos x-m=0\Leftrightarrow \cos x=m$.\\
		Phương trình $\cos x=m$ vô nghiệm khi $\hoac{&m<-1 \\&m>1.}$
	}
\end{vd}
\begin{vd}%[TH]%[DCHT Toán 11 - KNTT -Tên GV] %[ID6 chương trình mới]
	Có bao nhiêu giá trị nguyên của tham số $m$ để phương trình $\cos x=m+1$ có nghiệm?
	\dapso{$3$.}
	\loigiai{
	Phương trình $\cos x=m+1$ có nghiệm $\Leftrightarrow -1\le m+1\le 1 \Leftrightarrow -2\le m\le 0$.\\
	Mà $m\in \mathbb{Z}\Rightarrow m\in \left\{-2;-1;0\right\}$.\\
	Vậy có $3$ giá trị nguyên của tham số $m$ thỏa yêu cầu bài toán.
	}
\end{vd}
\subsubsection{Bài tập tự luận}

\begin{bt}%[TH]%[Câu 1]
	Tìm tất cả các tham số $m$ sao cho trong tập nghiệm của phương trình $\sin 2x=1+2m$ có ít nhất một nghiệm thuộc khoảng $\left(0;\dfrac{\pi}{2}\right)$.
	\dapso{$m\in \left(-\dfrac{1}{2};0\right]$.}
	\loigiai{
		Yêu cầu của bài toán được thỏa mãn khi và chỉ khi $0<1+2m\le 1\Leftrightarrow -1<2m\le 0\Leftrightarrow -\dfrac{1}{2}<m\le 0$.\\
		Vậy $m\in \left(-\dfrac{1}{2};0\right]$.}
\end{bt}

\begin{bt}%[TH]%[Câu 2]
	Tìm m để phương trình $\sin 3x-6-5m=0$ có nghiệm.\dapso{$-\dfrac{7}{5}\le m\le -1$. }
	\loigiai{
		Phương trình có nghiệm khi và chỉ khi:
		$-1\le 6+5m\le 1\Leftrightarrow -\dfrac{7}{5}\le m\le -1$}
\end{bt}
\begin{bt}%[TH]%[Câu 3]
	Có bao nhiêu giá trị nguyên của $m$ để phương trình: $3\sin x+m-1=0$ có nghiệm? \dapso{ $7$.}
	\loigiai{
	Ta có $3\sin x+m-1=0\Leftrightarrow \sin x=\dfrac{1-m}{3}$.\\
	Để phương trình có nghiệm thì $-1\le \dfrac{1-m}{3}\le 1\Leftrightarrow -2\le m\le 4$.\\
		Vậy có $7$ giá trị nguyên của $m$ để phương trình có nghiệm.}
\end{bt}
%%%%%%%%%%%%%%%%%%%%%%%%%%%%%%%%%%%%%%%%%
\subsubsection{Bài tập trắc nghiệm}
\Opensolutionfile{ans}[ans/ans-1K1-4-Dang2]

\begin{ex}%[Câu 1]
	Với giá trị nào của $m$ thì phương trình $\sin x-m=1$ có nghiệm là
	\choice
	{$0\le m\le 1$}
	{$m\le 0$}
	{$m\ge 1$}
	{\True $-2\le m\le 0$}
	\loigiai{
		Ta có $\sin x-m=1\Leftrightarrow \sin x=m+1$.\\
		Vì $-1\le \sin x\le 1\Rightarrow -1\le m+1\le 1\Rightarrow -2\le m\le 0$.\\
		Vậy để phương trình $\sin x-m=1$ có nghiệm thì $-2\le m\le 0$.}
\end{ex}
\begin{ex}%[Câu 3]
	Phương trình $\sin \dfrac{x}{2}=m$ có nghiệm khi và chỉ khi.
	\choice
	{\True $m\in \left[-1;1\right]$}
	{$m\in \left[-2;2\right]$}
	{$m\in \left[-\dfrac{1}{2};\dfrac{1}{2}\right]$}
	{$m\in R$}
	\loigiai{
		Ta có $-1\le \sin \dfrac{x}{2}\le 1$ $\Rightarrow $ $-1\le m\le 1$. Vậy $m\in \left[-1;1\right]$.}
\end{ex}
\begin{ex}%[Câu 4]
	Với giá trị nào của $m$ thì phương trình $\sin x-2m=1$ có nghiệm?
	\choice
	{$0\le m\le 1$}
	{$m\le 0$}
	{$m\ge 1$}
	{\True $-1\le m\le 0$}
	\loigiai{
		Phương trình $\sin x-2m=1\Leftrightarrow \sin x=2m+1$.\\
		Phương trình đã cho có nghiệm khi $-1\le 2m+1\le 1\Leftrightarrow -1\le m\le 0$.}
\end{ex}

\begin{ex}%[Câu 5]
	Tập hợp các giá trị của tham số $m$ để phương trình $\sin 2x+2=m$ có nghiệm là $\left[a;b\right]$. Khi đó $a+b$ bằng
	\choice
	{$3$}
	{$0$}
	{$2$}
	{\True $4$}
	\loigiai{
		Ta có $\sin 2x+2=m\Leftrightarrow \sin 2x=m-2$ có nghiệm khi và chỉ khi 
		\[-1\le m-2\le 1\Leftrightarrow 1\le m\le 3\Leftrightarrow m\in [1;3]. \]
		Vậy $a+b=4$.}
\end{ex}
\begin{ex}%[Câu 6]
	Có bao nhiêu giá trị nguyên của tham số $m$ để phương trình $3\sin 2x-m^2+5=0$ có nghiệm?
	\choice
	{$6$}
	{\True $2$}
	{$1$}
	{$7$}
	\loigiai{
		Phương trình đã cho tương đương với phương trình $\sin 2x=\dfrac{m^2-5}{3}$.\\
		Vì $\sin 2x\in \left[-1;1\right]$ nên $\dfrac{m^2-5}{3}\in \left[-1;1\right]\Rightarrow m^2\in \left[2;8\right]\Rightarrow  \hoac{&-2\sqrt{2}\le m\le -\sqrt{2} \\&\sqrt{2}\le m\le 2\sqrt{2}.}$}
\end{ex}
\begin{ex}%[Câu 7]
	Cho phương trình $4\sin \left(x+\dfrac{\pi}{3}\right)\cos \left(x-\dfrac{\pi}{6}\right)=a^2+\sqrt{3}\sin 2x-\cos 2x\quad(1)$. Có tất cả bao nhiêu giá trị nguyên của tham số $a$ để phương trình $\left(1\right)$ có nghiệm.
	\choice
	{\True $5$}
	{$0$}
	{$2$}
	{$3$}
	\loigiai{
		\begin{eqnarray*}
		\text{Phương trình}\, (1)
			& \Leftrightarrow & 2\left[\sin \dfrac{\pi}{2}+\sin \left(2x+\dfrac{\pi}{6}\right)\right]=a^2+\sqrt{3}\sin 2x-\cos 2x \\
			& \Leftrightarrow & 2\left(1+\sin 2x\cos \dfrac{\pi}{6}+\cos 2x\sin \dfrac{\pi}{6}\right)=a^2+\sqrt{3}\sin 2x-\cos 2x \\
			& \Leftrightarrow & 2\left(1+\dfrac{\sqrt{3}}{2}\sin 2x+\dfrac{1}{2}\cos 2x\right)=a^2+\sqrt{3}\sin 2x-\cos 2x \\
			& \Leftrightarrow & 2+\sqrt{3}\sin 2x+\cos 2x=a^2+\sqrt{3}\sin 2x-\cos 2x \\
			& \Leftrightarrow & 2\cos 2x=a^2-2 \\
			& \Leftrightarrow & \cos 2x=\dfrac{a^2-2}{2}=\dfrac{a^2}{2}-1. \qquad(2)
		\end{eqnarray*}
		Phương trình $(1)$ có nghiệm khi và chỉ khi phương trình $(2)$ có nghiệm
		\[\Leftrightarrow -1\le \dfrac{a^2}{2}-1\le 1\Leftrightarrow 0\le \dfrac{a^2}{2}\le 2\Leftrightarrow a^2\le 4\Leftrightarrow -2\le a\le 2.\]
		Vì $a\in \mathbb{Z}\Rightarrow a\in \left\{-2;-1;0;1;2\right\}$.
		Vậy có $5$ giá trị nguyên của tham số $a$.}
\end{ex}
\begin{ex}%[Câu 8]
	Tìm tất cả giá trị thực của $m$ để phương trình $\cos 2x-m=0$ vô nghiệm.
	\choice
	{\True $m\in (-\infty;-1)\cup (1;+\infty)$}
	{$m\in (1;+\infty)$}
	{$m\in [-1;1]$}
	{$m\in (-\infty;-1)$}
	\loigiai{
		Ta có $\cos 2x-m=0\Leftrightarrow \cos 2x=m$.\\
		Do đó phương trình đã cho vô nghiệm khi và chỉ khi
		$\left| m\right|>1\Leftrightarrow \hoac{&m>1 \\&m<-1}\Leftrightarrow m\in \left(-\infty;-1\right)\cup \left(1;+\infty\right)$.}
\end{ex}

\begin{ex}%[Câu 9]
	Cho phương trình $\cos \left(2x-\dfrac{\pi}{3}\right)-m=2$. Tìm $m$ để phương trình có nghiệm?
	\choice
	{Không tồn tại $m$}
	{$m\in \left[-1;3\right]$}
	{\True $m\in \left[-3;-1\right]$}
	{$m\in \mathbb{R}$}
	\loigiai{
		Ta có $\cos \left(2x-\dfrac{\pi}{3}\right)-m=2\Leftrightarrow \cos \left(2x-\dfrac{\pi}{3}\right)=m+2$.\\
	Phương trình đã cho có nghiệm khi $-1\le m+2\le 1\Leftrightarrow -3\le m\le -1$.}
\end{ex}

\begin{ex}%[Câu 10]
	Tìm tất cả giá trị của $a$ để phương trình sau có nghiệm $\cos^23x=2a^2-3a+1$.
	\choice
	{$a\in \left[0;1\right]$}
	{\True $a\in \left[0;\dfrac{1}{2}\right]\cup \left[1;\dfrac{3}{2}\right]$}
	{$a\in \left[0;\dfrac{3}{2}\right]$}
	{$a\in \left[0;1\right]\cup \left[\dfrac{3}{2};+\infty\right)$}
	\loigiai{
		\begin{eqnarray*}
			\text{Ta có}\, \cos^23x=2a^2-3a+1 &\Leftrightarrow & \dfrac{1+\cos 6x}{2}=2a^2-3a+1 \\
			&\Leftrightarrow & 1+\cos 6x=4a^2-6a+2\\
			&\Leftrightarrow & \cos 6x=4a^2-6a+1.\quad(*)
		\end{eqnarray*}
		Phương trình đã cho có nghiệm khi và chỉ khi phương trình $(*)$ có nghiệm
		\[\Leftrightarrow \heva{&4a^2-6a+1\ge -1 \\&4a^2-6a+1\le 1}\Leftrightarrow \heva{&4a^2-6a+2\ge 0 \\&4a^2-6a\le 0}\Leftrightarrow \heva{&\hoac{&a\le \dfrac{1}{2} \\&a\ge 1} \\&0\le a\le \dfrac{3}{2}}\Leftrightarrow \hoac{&0\le a\le \dfrac{1}{2} \\&1\le a\le \dfrac{3}{2}.}\]}
\end{ex}
\Closesolutionfile{ans}
% \begin{indapan}{10}
% 	{ans/ans-1K1-4-Dang2}
% \end{indapan}
%%%%%%%%%%%%%%%%%%%%%%%%%%%%%%%%%%%%%%%%%%%%%%%%%%%%%%%
