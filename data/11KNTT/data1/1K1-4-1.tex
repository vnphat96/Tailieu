\subsection{Các dạng toán thường gặp}
\begin{dang}{Phương trình tương đương}
\begin{itemize}
	\item 	Hai phương trình được gọi là tương đương nếu chúng có cùng tập nghiệm.
	\item Nếu phương trình $f(x)=0$ tương đương với phương trình $g(x)=0$ thì ta viết 
	\begin{center}
		$f(x)=0\Leftrightarrow g(x)=0$.
	\end{center}
	\item \textbf{Chú ý:}\\
	Để giải phương trình, thông thường ta biến đổi phương trình đó thành một phương trình tương đương đơn giản hơn. Các phép biến đổi như vậy gọi là \textit{các phép biến đổi tương đương}.\\
	Nếu thực hiện các phép biến đổi sau đây trên một phương trình mà không làm thay đổi điều kiện của nó thì ta được một phương trình mới tương đương với phương trình đã cho.
	\begin{enumEX}[a)]{1}
		\item Cộng hay trừ hai vế với cùng một số hoặc một biểu thức:
		\begin{center}
			$f(x)=g(x)\Leftrightarrow f(x)+h(x)=g(x)+h(x)$.
		\end{center}
		\item Nhân hoặc chia hai vế với cùng một số khác $0$ hoặc với cùng một biểu thức luôn có giá trị khác $0$:
		\begin{center}
			$f(x)=g(x)\Leftrightarrow f(x)h(x)=g(x)h(x)$, $(h(x)\neq 0)$.
		\end{center} 
	\end{enumEX}
\end{itemize}
\end{dang}
\subsubsection{Ví dụ}
\begin{vd}%[NB]
	Hai phương trình sau có tương đương không?
	\begin{center}
		$2x-4=0$ và $x^2-4x+4=0$
	\end{center}
	\loigiai{Tập nghiệm của phương trình $2x-4=0$ là $S_1=\{2\}$.\\
		Phương trình $x^2-4x+4=0$ có tập nghiệm là $S_2=\{2\}$.\\
		Vậy hai phương trình trên là tương đương.
	}
\end{vd}
\begin{vd}%[NB]
	Khẳng định $5x+10=0\Leftrightarrow x+2=0$ đúng hay sai?
	\loigiai{Ở phương trình trên ta chia 2 vế cho $5\neq 0$ là phép biến đổi tương nên khẳng định trên là đúng.
	}
\end{vd}
\begin{vd}%[TH]
	Hai phương trình $2x+8=0$ và $\dfrac{x^2-16}{x-4}=0$ có tương đương không? Vì sao?
	\loigiai{Tập nghiệm của phương trình $2x+8=0$ là $S_1=\{-4\}$.\\
		Phương trình $\dfrac{x^2-16}{x-4}=0$ được viết lại thành $x+4=0$ nên có tập nghiệm là $S_2=\{-4\}$.\\
		Vì $S_1=S_2$ nên hai phương trình trên là phương trình tương đương.
	} 
\end{vd}
\begin{vd}%[TH]
	Hãy chỉ ra lỗi sai trong phép biến đổi tương đương sau đây:
	\begin{center}
		$-x^2=3x\Leftrightarrow-\dfrac{x^2}{x}=\dfrac{3x}{x}\Leftrightarrow -x=3\Leftrightarrow x=-3$.
	\end{center}
	\loigiai{Vì đề bài chưa cho $x\neq 0$ nên việc chia 2 vế cho $x$ là chưa đúng.
	}
\end{vd}
\begin{vd}%[TH]
	Phương trình $x^2-9=0$ tương đương với phương trình nào sau đây?
	\begin{enumEX}[a)]{2}
		\item $3x^2=27$.
		\item $x^2-9+\dfrac{5}{x-3}==\dfrac{5}{x-3}$
	\end{enumEX}
	\loigiai{
		\begin{enumEX}[a)]{1}
			\item Hai phương trình $x^2-9=0$ và $3x^2=27$ có cùng tập nghiệm là $\{-3;3\}$ nên hai phương trình này tương đương.
			\item Ta có $x=3$ là một nghiệm của phương trình $x^2-9=0$, nhưng không là nghiệm của phương trình $x^2-9+\dfrac{5}{x-3}=\dfrac{5}{x-3}$. Do đó hai phương trình này không tương đương với nhau. 
		\end{enumEX}
	}
\end{vd}
\subsubsection{Bài tập rèn luyện}
\subsubsection{Bài tập tự luận}
\begin{bt}%[NB]
	Phương trình $x^2=9$ và $|x|=3$ có tương đương với nhau không? Vì sao?
	\loigiai{Phương trình $x^2=9$ có tập nghiệm là $S_1=\{-3;3\}$.\\
		Phương trình $|x=3|$ có tập nghiệm là $S_2=\{-3;3\}$.\\
		Vì $S_1=S_2$ nên hai phương trình $x^2=9$ và $|x=3|$ tương đương nhau.
	}
\end{bt}
\begin{bt}%[NB]
	Phương trình $(x-4)(2+\sqrt{3x-15})=0$ và $4-x=0$ có tương đương với nhau không? Vì sao?
	\loigiai{ Phương trình $(x-4)(2+\sqrt{3x-15})=0$ có tập nghiệm là $S_1=\{4\}$.\\
		Phương trình $4-x=0$ có tập nghiệm là $S_2=\{4\}$.\\
		Vì $S_1=S_2$ nên hai phương trình $(x-4)(2+\sqrt{3x-15}=0)$ và $4-x=0$ tương đương nhau.
	}
\end{bt}
\begin{bt}%[NB]
	Bạn Minh giải phương trình $(x^2-4)(x^2+1)=12(x^2+1)$ theo các bước như sau:\\
	Bước 1. Chia 2 vế cho $x^2+1$ ta được $(x^2-4)=12$.\\
	Bước 2. $x^2=16$.\\
	Bước 3. $x=4$ hay $x=-4$.\\
	Theo em các bước giải của bạn Minh đã đúng hay sai? Giải thích.
	\loigiai{Các bước làm của bạn Minh là đúng vì ta chia 2 vế của phương trình $(x^2-4)(x^2+1)=12(x^2+1)$ cho cùng biểu thức $x^2+1$ luôn có giá trị khác $0$ nên không làm thay đổi điều kiện của phương trình.
	}
\end{bt}
\begin{bt}%[NB]
	Vì sao các phép biến đổi sau là tương đương?
	\begin{center}
		$(3x-4)(x^2+6)=(-2x+6)(x^2+6)\Leftrightarrow3x-4=-2x+6\Leftrightarrow5x=10\Leftrightarrow x=2$.
	\end{center}
	\loigiai{Hai phương trình $(3x-4)(x^2+6)=(-2x=6)(x^2+6)$ và $3x-4=-2x+6$ là tương đương vì ta chia hai vế cho biểu thức $x^2+6$ luốn có giá trị khác $0$ mà không làm thay đổi điều kiện của phương trình.\\
		Hai phương trình $3x-4=-2x+6$ và $5x=10$ là tương đương vì ta cộng hai vế cho cùng biểu thức $2x+4$ mà không làm thay đổi điều kiện của phương trình.\\
		Hai phương trình $5x=10$ và $x=2$ là tương đương vì ta chia 2 vế cho $5$ không làm thay đổi điều kiện của phương trình.	
	}
\end{bt}
\begin{bt}
	Cho hai phương trình $(x-5)(1+\sqrt{x-2})=0$ và $x^2-2mx+6m-5=0$. Tìm $m$ để hai phương trình trên là hai phương trình tương đương.
	\loigiai{Ta có phương trình $(x-5)(1+\sqrt{x-2})=0$ có tập nghiệm là $S_1=\{5\}$, khi đó để hai phương trình là tương đương thì phương trình $x^2-3mx+4=0$ phải có nghiệm duy nhất $x=5$.\\
		Thay $x=5$ vào phương trình $x^2-2mx+6m-5=0$ ta được $-4m+20=0$ hay $m=5$.\\
		Thử lại với $m=5$ ta có phương trình
		\begin{center}
			$\begin{aligned}
				&x^2-10x+25=0\\
				&\Leftrightarrow x=5
			\end{aligned}$
		\end{center}
		Vậy $m=5$ thỏa mãn yêu cầu bài toán. 
	}
\end{bt}
\subsubsection{Bài tập trắc nghiệm}
\Opensolutionfile{ans}[ans/ans-1K1-4-Dang1]
\begin{ex}%%Câu 1
	Hai phương trình được gọi là tương đương khi
	\choice
	{Có cùng bậc}
	{Có cùng tập xác định}
	{\True Có cùng tập nghiệm}
	{Cả 3 ý trên}
	\loigiai{Hai phương trình được gọi là tương đương nếu chúng có cùng tập nghiệm.
	}
\end{ex}
\begin{ex}%%Câu 2
	Phương trình nào tương đương với phương trình $3x-6=0$
	\choice
	{$x^2-4=0$}
	{\True$x^2-4x+4=0$}
	{$|x|=2$}
	{$(x+2)(x+1)=0$}
	\loigiai{Phương trình $3x-6=0$ có tập nghiệm là $S=\{2\}$.\\
		Phương trình $x^2-4=0$ có tập nghiệm là $S=\{-2;2\}$.\\
		Phương trình $x^2-4x+4=0$ có tập nghiệm là $S=\{2\}$\\
		Phương trình $|x|=2$ có tập nghiệm là $S=\{-2;2\}$.\\
		Phương trình $(x+2)(x+1)=0$ có tập nghiệm là $S=\{-2;-1\}$.\\
		Vì hai phương trình $3x-6=0$ và $x^2-4x+4=0$ có cùng tập nghiệm nên hai phương trình đó là phương trình tương đương.
	}
\end{ex}
\begin{ex}%%Câu 3
	Cho các phương trình sau:
	\begin{enumEX}[1.]{2}
		\item $x^2-2x+1=0$.
		\item $(x-1)(x^2+3)=0$
		\item $x(x-1)=x$
		\item $(x^2-4x+4=0$
	\end{enumEX}
	Hỏi có bao nhiêu cặp phương trình là tương đương nhau?
	\choice
	{\True$1$}
	{$2$}
	{$3$}
	{$4$}
	\loigiai{Phương trình $x^2-2x+1=0$ có tập nghiệm là $S=\{1\}$.\\
		Phương trình $(x-1)(x^2+3)=0$ có tập nghiệm là $S=\{1\}$.\\
		Phương trình $x(x-1)=x$ có tập nghiệm là $S=\{0;1\}$.\\
		Phương trình $x^2-4x+4=0$ có tập nghiệm là $S=\{2\}$.\\
		Hai phương trình $x^2-2x+1$ và $(x-1)(x^2+3=0)$ có cùng tập nghiệm nên hai phương trình đó là tương đương.\\
		Vậy có 1 cặp phương trình tương đương.
	}
\end{ex}
\begin{ex}%%Câu 4
	Cho các phép biến đổi của 2 phương trình dưới đây:
	\begin{itemize}
		\item [(1)] $x+\sqrt{x-1}=1+\sqrt{x-1}\Leftrightarrow x=1$.
		\item [(2)] $x^2+2x=x+1\Leftrightarrow(x+1)^2=x+2$.
	\end{itemize}
	Phép biến đổi nào là đúng
	\choice
	{$(1)$}
	{\True $(2)$}
	{Cả $(1)$ và $(2)$ đều sai}
	{Cả $(1)$ và $(2)$ đều đúng}
	\loigiai{
		Hai phương trình $x+\sqrt{x-1}=1+\sqrt{x-1}$ và  $x=1$ không tương đương vì khi trừ hai vế cho biểu thức $\sqrt{x-1}$ làm thay đổi điều kiện của phương trình.\\
		Do đó phép biến đổi $(1)$ là sai.\\
		Hai phương trình $x^2+2x=x+1$ và $(x+1)^2=x+2$ là tương đương vì ta cộng cả  hai vế cho $1$ không làm thay đổi điều kiện của phương trình.\\
		Do đó phép biến đổi $(2)$ là đúng.
	}
\end{ex}
\begin{ex}%%Câu 5
	Cho các phép biến đổi của phương trình $15x(x-3)=3x$ sau:
	\begin{itemize}
		\item[+] Bước 1.  $\dfrac{15x(x-3)}{x}=\dfrac{3x}{x}$.
		\item[+] Bước 2.   $15(x-3)=3$.
		\item[+] Bước 3.  $x-3=\dfrac{1}{5}$.
		\item[+] Bước 4.  $x=\dfrac{16}{5}$.
	\end{itemize}
	Trong các bước biến đổi trên phép biến đổi của bước thứ mấy là \textbf{sai}?
	\choice
	{\True Bước 1}
	{Bước 2}
	{Bước 3}
	{Bước 4}
	\loigiai{Bước 1 là sai vì chưa xác định được $x\neq 0$ nên việc chia 2 vế cho $x$ làm thay đổi điều kiện của phương trình.
	}
\end{ex}
\begin{ex}%%Câu 6
	Tập nghiệm của phương trình $x+\dfrac{1}{x-2}=2+\dfrac{1}{x-2}$ là
	\choice
	{$S=\{2\}$}
	{$S=\{-2\}$}
	{\True$S=\varnothing$}
	{$S=\{-2;2\}$}
	\loigiai{Phương trình $x+\dfrac{1}{x-2}=2+\dfrac{1}{x-2}$ có tập xác định $\mathscr{D}=\mathbb{R}\setminus \{2\}$ có tập nghiệm $S=\varnothing$.
	}
\end{ex}
\begin{ex}%%Câu 7
	Tập nghiệm của phương trình $x+2+\sqrt{x-4}=2x-18+\sqrt{x-4}$ là
	\choice
	{$S=\{4\}$}
	{$S=\{-20\}$}
	{$S=\varnothing$}
	{\True $S=\{20\}$}
	\loigiai{Phương trình $x+2+\sqrt{x-4}=2x-18+\sqrt{x-4}$ với tập xác định $\mathscr{D}=\left[4;+\infty\right)$ có tập nghiệm là $S=\{20\}$
	}
\end{ex}
\begin{ex}%%Câu 8
	Chọn khẳng định \textbf{sai}?
	\choice
	{$3+x^2=\sqrt{x-1}\Leftrightarrow (3+x^2)^2=x-1$}
	{$x+\sqrt{x-5}=3+\sqrt{x-5}\Leftrightarrow x=3$}
	{$\dfrac{x+7}{\sqrt{2x-1}}=\sqrt{2x-1}\Leftrightarrow x+7=2x-1$}
	{\True $x^2-3\sqrt{x+5}=5\Leftrightarrow x^2=5+3\sqrt{x+5}$}
	\loigiai{
		Xét các đáp án:\\
		Đáp án A: Hai phương trình $3+x^2=\sqrt{x-1}$ và $(3+x^2)=x-1$ không cùng điều kiện nên không phải là hai phương trình tương đương.\\
		Đáp án B: Hai phương trình $x+\sqrt{x-5}=3+\sqrt{x-5}$ và $x=3$ không cùng điều kiện nên không phải là hai phương trình tương đương.\\
		Đáp án C: Hai phương trình $\dfrac{x+7}{\sqrt{2x-1}}=\sqrt{2x-1}$ và $x+7=2x-1$ không cùng điều kiện nên không phải là hai phương trình tương đương.\\
		Đáp án D: Hai phương trình $x^2-3\sqrt{x+5}=5$ và $x^2=5+3\sqrt{x+5}$ là tương đương vì ta cộng hai vế cho biểu thức $3\sqrt{x+5}$ không làm thay đổi điều kiện của phương trình.
	}
\end{ex}
\begin{ex}%%Câu 9
	Với giá trị nào của tham số $m$ thì hai phương trình sau tương đương:
	\begin{center}
		$3x-6=0 \quad (1)$ và $x^2-2mx+3m-2=0 \quad (2)$.
	\end{center}
	\choice
	{$m=1$}
	{\True $m=2$}
	{$m=3$}
	{$m=4$}
	\loigiai{Ta có $(1)\Leftrightarrow x=2$. Để hai phương trình tương đương thì phương trình $x^2-2mx+3m-2=0$  phải có nghiệm kép $x_1=x_2=2$.\\
		Thay $x=2$ vào $(2)$ ta được $m=2$.\\
		Với $m=2$, ta có $(2)$ trở thành $x^2-4x+4=0\Leftrightarrow (x-2)^2=0\Leftrightarrow x=2$. Vậy $m=2$ thỏa yêu cầu bài toán
	}
\end{ex}
\begin{ex}%%Câu 10
	Cho hai phương trình $3x^2+2x-5=0\quad (1)$ và $x^3-(2m-1)x^2+(m-3)x+m+1=0\quad (2)$. Biết rằng tại giá trị $m=\dfrac{a}{b} (a,b \in\mathbb{R})$ thì hai phương trình trên tương đương. Giá trị của $T=2a-b$ bằng
	\choice
	{\True $T=1$}
	{$T=-3$}
	{$T=4$}
	{$T=-7$}
	\loigiai{Xét phương trình: 
		\begin{center}
			$3x^2+2x-5=0\Leftrightarrow\hoac{&x=1\\&x=-\dfrac{5}{3}}$.\\
		\end{center}
		Xét phương trình:
		\begin{center}
			$\begin{aligned}
				&x^3-(2m-1)x^2+(m-3)x+m+1=0\\
				&\Leftrightarrow(x-1)[x^2-2(m-1)x-1-m]=0\\
				&\Leftrightarrow\hoac{&x=1\\&x^2-2(m-1)x-1-m=0\quad (*)}
			\end{aligned}$.
		\end{center}
		Phương trình $x^2-2(m-1)-1-m=0$ có $\Delta'= m^2-m+2=\left(m-\dfrac{1}{2}\right)^2+\dfrac{7}{4}>0$, $\forall m$ nên phương trình luôn có 2 nghiệm phân biệt.\\
		Để phương trình $(1)$ và $(2)$ tương đương thì phương trình $x^2-2(m-1)-1-m=0$ có 2 nghiệm phân biệt sao cho có một nghiệm bằng 1, một nghiệm bằng $-\dfrac{5}{3}$.\\
		\begin{itemize}
			\item Với $x=1$ thì $(*)\Leftrightarrow -3m+2=0\Leftrightarrow m=\dfrac{2}{3}$\\
			Thử lại với  $m=\dfrac{2}{3}$, $(2)\Leftrightarrow x^3-\dfrac{1}{3}x^2-\dfrac{7}{3}x+\dfrac{5}{3}=0\Leftrightarrow\hoac{&x=1\\&x=-\dfrac{5}{3}}$(thỏa mãn).
			\item Với $x=-\dfrac{5}{3}$ thì $(*)\Leftrightarrow \dfrac{7}{3}m-\dfrac{14}{9}=0\Leftrightarrow m=\dfrac{2}{3}$ (thỏa mãn).
		\end{itemize}
		Vậy $m=\dfrac{2}{3}$ thỏa yêu cầu bài toán, khi đó $T=4-3=1$
	}
\end{ex}
\Closesolutionfile{ans}
\begin{indapan}{10}
	{ans/ans-1K1-4-Dang1}
\end{indapan}