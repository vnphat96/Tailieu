\begin{name}
	{\tenchude}
	{ĐỀ ÔN TẬP CHƯƠNG I}
	{LỚP TOÁN THẦY PHÁT}
	{\thoigian}
\end{name}
\TN
\setcounter{ex}{0}
\Opensolutionfile{ans}[ans/ans-TN-C1-De1]
\TN
%Câu 1
\begin{ex}
	Rút gọn biểu thức $M=\cos 2x \cdot \cos x+\sin 2x \cdot \sin x$ ta được kết quả là:
	\choice
	{\True $M=\cos x$}
	{$M=\cos 3x$}
	{$M=\sin x$}
	{$M=\sin 3x$}
	\loigiai{
		Ta có: $M=\cos 2x \cdot \cos x+\sin 2x \cdot \sin x=\cos (2x-x)=\cos x$
	}
\end{ex}
%Câu 2
\begin{ex}
	Đẳng thức nào không đúng với mọi $x$?
	\choice
	{$\cos^2 3x=\dfrac{1+\cos 6x}{2}$}
	{$\cos 2x=1-2\sin^2x$}
	{$\sin 2x=2\sin x\cos x$}
	{\True $\sin^2 2x=\dfrac{1+\cos 4x}{2}$}
	\loigiai{
		Ta có $\sin^2 2x=\dfrac{1-\cos 4x}{2}$
	}
\end{ex}
%Câu 3
\begin{ex}
	Góc có số đo $\dfrac{\pi }{24}$ đổi sang độ bằng
	\choice
	{$7^\circ $}
	{\True $7^\circ 3{0}'$}
	{$8^\circ $}
	{$8^\circ 3{0}'$}
	\loigiai{
		Ta có: $\dfrac{\pi }{24}=\dfrac{180^\circ }{24}=7^\circ 30'$
	}
\end{ex}
%Câu 4
\begin{ex}
	Một đường tròn có đường kính là $50$ (cm). Độ dài của cung tròn trên đường tròn có số đo là $\dfrac{\pi }{4}$ bằng (làm tròn đến hàng đơn vị)
	\choice
	{$40$ (cm)}
	{$39$ (cm)}
	{$19$ (cm)}
	{\True $20$ (cm)}
	\loigiai{
		Độ dài của cung tròn $l=\alpha \cdot R=\dfrac{\pi }{4} \cdot 25=\dfrac{25}{4}\pi \approx 20$ (cm)
	}
\end{ex}
%Câu 5
\begin{ex}
	Chọn phát biểu đúng:
	\choice
	{Các hàm số $y=\sin x$, $y=\cos x$, $y=\cot x$ đều là hàm số chẵn}
	{Các hàm số $y=\sin x$, $y=\cos x$, $y=\cot x$ đều là hàm số lẻ}
	{Các hàm số $y=\sin x$, $y=\cot x$, $y=\tan x$ đều là hàm số chẵn}
	{\True Các hàm số $y=\sin x$, $y=\cot x$, $y=\tan x$ đều là hàm số lẻ}
	\loigiai{
		Hàm số $y=\cos x$ là hàm số chẵn, hàm số $y=\sin x$, $y=\cot x$, $y=\tan x$ là các hàm số lẻ
	}
\end{ex}
%Câu 6
\begin{ex}
	Nếu $\sin x+\cos x=\dfrac{1}{2}$ thì $\sin 2x$ bằng
	\choice
	{$\dfrac{3}{4}$}
	{$\dfrac{3}{8}$}
	{$\dfrac{\sqrt{2}}{2}$}
	{\True $\dfrac{-3}{4}$}
	\loigiai{
	Do $\sin x+\cos x=\dfrac{1}{2}\Rightarrow \dfrac{1}{4}={{\left(\sin x+\cos x\right)}^2}={{\left(\sin x\right)}^2}+{{\left(\text{cosx}\right)}^2}+2\sin x \cdot \cos x$\\
	$\Rightarrow \dfrac{1}{4}=1+\sin 2x\Rightarrow \sin 2x =\dfrac{-3}{4}$
	}
\end{ex}
%Câu 7
\begin{ex}
	Một con lắc lò xo sau khi được kéo xuống dưới vị trí cân bằng $4$ cm và thả ra thì nó dao động điều hòa với phương trình: $y=-4\cos 8t$ (cm). Biên độ $A$ cm và chu kỳ $T$ của dao động là
	\choice
	{\True $A=4$ cm, $T=\dfrac{\pi }{4}$}
	{$A=4$ cm, $T=\dfrac{\pi }{2}$}
	{$A=8$ cm, $T=\dfrac{\pi }{4}$}
	{$A=4$ cm, $T=2\pi $}
	\loigiai{
		Biên độ của dao động là: $A=|-4|=4$ (cm).\\
		Chu kỳ của dao động là:$T=\dfrac{2\pi }{|8|}=\dfrac{\pi }{4}$
	}
\end{ex}
%Câu 8
\begin{ex}
	Hãy tìm tập tất cả các giá trị của $m$ để phương trình $\left| \sin x \right|=m$ có nghiệm?
	\choice
	{$-1\le m\le 1$}
	{$-1\le m\le 0$}
	{$-1<m<0$}
	{\True $0\le m\le 1$}
	\loigiai{
		Vì $0<= |\sin x|<=1, \forall x \in \mathbb{R}$ nên phương trình $\left| \sin x \right|=m$ có nghiệm khi và chỉ khi $0\le m\le 1$.
	}
\end{ex}
%Câu 9
\begin{ex}
	Nghiệm của phương trình $2\sin \left(4x-\dfrac{\pi }{3}\right)-1=0$ là:
	\choice
	{$x=\pi +k2\pi ;x=k\dfrac{\pi }{2}\ (k \in \mathbb{Z})$}
	{$x=\dfrac{\pi }{8}+k\dfrac{\pi }{2};x=\dfrac{7\pi }{24}+k\dfrac{\pi }{2}\ (k \in \mathbb{Z})$}
	{$x=k2\pi ;x=\dfrac{\pi }{2}+k2\pi\ (k \in \mathbb{Z})$}
	{$x=k\pi ;x=\pi +k2\pi\ (k \in \mathbb{Z})$}
	\loigiai{
		$2\sin \left(4x-\dfrac{\pi }{3}\right)-1=0\Leftrightarrow \sin \left(4x-\dfrac{\pi }{3}\right)=\dfrac{1}{2}\Leftrightarrow \hoac{& 4x-\dfrac{\pi }{3}=\dfrac{\pi }{6}+k2\pi \\& 4x-\dfrac{\pi }{3}=\pi -\dfrac{\pi }{6}+k2\pi}\Leftrightarrow \hoac{& x=\dfrac{\pi }{8}+k\dfrac{\pi }{2} \\& x=\dfrac{7\pi }{24}+k\dfrac{\pi }{2}}\left(k\in \mathbb{Z}\right)$
	}
\end{ex}
%Câu 10
\begin{ex}
	Biết $\sin \left(\alpha +\dfrac{3\pi }{2}\right)+\cos \left(\alpha +\dfrac{3\pi }{2}\right)=\sqrt{2}$. Tính $\sin \left(\alpha +\pi\right)-2\cos \left(\alpha -\pi\right)$.
	\choice
	{$\dfrac{3}{\sqrt{2}}$}
	{\True $-\dfrac{3}{\sqrt{2}}$}
	{$-\dfrac{1}{\sqrt{2}}$}
	{$\dfrac{1}{\sqrt{2}}$}
	\loigiai{
	Ta có $\sin \left(\alpha +\dfrac{3\pi }{2}\right)=\sin \left(\alpha +2\pi -\dfrac{\pi }{2}\right)=\sin \left(\alpha -\dfrac{\pi }{2}\right)=-\sin \left(\dfrac{\pi }{2}-\alpha\right)=-\cos \alpha $.\\
	$\cos \left(\alpha +\dfrac{3\pi }{2}\right)=\cos \left(\alpha +2\pi -\dfrac{\pi }{2}\right)=\cos \left(\alpha -\dfrac{\pi }{2}\right)=\cos \left(\dfrac{\pi }{2}-\alpha\right)=\sin \alpha $.\\
	Suy ra $\sin \alpha -\cos \alpha =\sqrt{2}\Rightarrow \sin \alpha =\cos \alpha +\sqrt{2}$.\\
	Vì ${{\sin }^2}\alpha +{{\cos }^2}\alpha =1\Rightarrow 2{{\cos }^2}\alpha +2\sqrt{2}\cos \alpha +2=1$\\
	$\Leftrightarrow 2{{\cos }^2}\alpha +2\sqrt{2}\cos \alpha +1=0\Leftrightarrow \cos \alpha =-\dfrac{1}{\sqrt{2}}\Rightarrow \sin \alpha =\dfrac{1}{\sqrt{2}}$.\\
	Do đó $\sin \left(\alpha +\pi\right)-2\cos \left(\alpha -\pi\right)=-\sin \alpha +2\cos \alpha =-\dfrac{3}{\sqrt{2}}$
	}
\end{ex}
%Câu 11
\begin{ex}
	Hằng ngày mực nước của con kênh lên xuống theo thủy triều. Độ sâu $h$(mét) của mực nước trong kênh được tính tại thời điểm $t$ (giờ) trong một ngày bởi công thức $h=3\cos \left(\dfrac{\pi t}{7=8}+\dfrac{\pi }{4}\right)+12$. Mực nước của kênh cao nhất khi:
	\choice
	{$t=13$(giờ)}
	{\True $t=14$(giờ)}
	{$t=15$(giờ)}
	{$t=16$(giờ)}
	\loigiai{
		Mực nước của kênh cao nhất khi $h$ lớn nhất\\
		$\Leftrightarrow \cos \left(\dfrac{\pi t}{8}+\dfrac{\pi }{4}\right)=1\Leftrightarrow \dfrac{\pi t}{8}+\dfrac{\pi }{4}=k2\pi $ với $0<t\le 24$ và $k\in \mathbb{Z}$.\\
		Lần lượt thay các đáp án, ta được đáp án B thỏa mãn.\\
		Vì với $t=14$ thì $\dfrac{\pi t}{8}+\dfrac{\pi }{4}=2\pi $ (đúng với $k=1\in \mathbb{Z}$)
	}
\end{ex}
%Câu 12
\begin{ex}
	Số giờ có ánh sáng mặt trời của một thành phố A ở vĩ độ ${{40}^{\text{o}}}$ bắc trong ngày thứ t của một năm không nhuận được cho bởi hàm số $d(t)=3\sin \left[\dfrac{\pi }{180}(t-80)\right]+12$ với $t\in \mathbb{Z}$ và $0<t\le 365$. Vào ngày nào trong năm thì thành phố A có nhiều giờ có ánh sáng mặt trời nhất?
	\choice
	{\True 170}
	{171}
	{172}
	{173}
	\loigiai{
		Ta có $d(t)=3\sin \left[\dfrac{\pi }{180}(t-80)\right]+12\le 3 \cdot 1+12=15$.\\
		Vậy thành phố A có nhiều giờ có ánh sáng mặt trời nhất khi $\sin \left[\dfrac{\pi }{180}(t-80)\right]=1\Leftrightarrow \dfrac{\pi }{180}(t-80)=\dfrac{\pi }{2}+k2\pi \Leftrightarrow t=170+360k (k\in \mathbb{Z})$.\\
		Vì $0<t\le 365$ nên $0<170+360k\le 365\Leftrightarrow -\dfrac{17}{36}<k\le \dfrac{39}{72}\Rightarrow k=0\Rightarrow t=170$.
	}
\end{ex}

\TNTF
%Câu 13
\begin{ex}
	Cho phương trình $\sin x=a$ (1).
	\choiceTF
	{\True Nếu $a>1$ thì phương trình (1) vô nghiệm}
	{Nếu $a=1$ thì phương trình (1) có nghiệm $\alpha =\dfrac{\pi }{2}+k\pi ,\left(k\in \mathbb{Z}\right)$}
	{\True Nếu $-1\le a\le 1$ thì phương trình (1) có nghiệm $\hoac{& x=\alpha +k2\pi \\ & x=\pi -\alpha +k2\pi} \left(k\in \mathbb{Z}\right)$}
	{Phương trình (1) luôn có hai điểm biểu diễn nghiệm trên đường tròn lượng giác}
	\loigiai{
		Nếu $a=1\Rightarrow \sin \alpha =1\Leftrightarrow \alpha =\dfrac{\pi }{2}+k2\pi ,\left(k\in \mathbb{Z}\right)$
	}
\end{ex}
%Câu 14
\begin{ex}
	Các mệnh đề sau đúng hay sai?
	\choiceTF
	{\True Hàm số $y=\sin \sqrt{x+4}$ có tập xác định là $D=\left[-4;+\infty\right)$}
	{Hàm số $y=\cot \left(\dfrac{\pi }{2}+x\right)$ có tập xác định là $D=\mathbb{R}$}
	{\True Hàm số $y=\sqrt{3-2\cos x}$ có tập xác định là $D=\mathbb{R}$}
	{Hàm số $y=\dfrac{1-3\cos x}{\sin x}$ có tập xác định là $D=\mathbb{R}\backslash \left\{ k\dfrac{\pi }{2},k\in \mathbb{Z} \right\}$}
	\loigiai{
	a) Hàm số xác định khi và chỉ khi $x+4\ge 0\Leftrightarrow x\ge -4$.\\
	Vậy tập xác định của hàm số là $D=\left[-4;+\infty\right)$.\\
	b) Hàm số xác định khi và chỉ khi $\sin \left(x+\dfrac{\pi }{2}\right)\ne 0\Leftrightarrow x+\dfrac{\pi }{2}\ne k\pi \Leftrightarrow x\ne -\dfrac{\pi }{2}+k\pi ;k\in \mathbb{Z}$.\\
	Vậy tập xác định của hàm số là $D=\mathbb{R}\backslash \left\{ -\dfrac{\pi }{2}+k\pi ;k\in \mathbb{Z} \right\}$.\\
	c) Hàm số xác định khi $3-2\cos x\ge 0\Leftrightarrow \cos x\le \dfrac{3}{2}$ (đúng $\forall x\in \mathbb{R}$), vì $-1\le \cos x\le 1,\forall x\in \mathbb{R}$.\\
	Vậy tập xác định của hàm là $D=\mathbb{R}$.\\
	d) Hàm số xác định khi và chỉ khi $\sin x\ne 0\Leftrightarrow x\ne k\pi \left(k\in \mathbb{Z}\right)$.\\
	Vậy tập xác định của hàm số là $D=\mathbb{R}\backslash \left\{ k\pi ,k\in \mathbb{Z} \right\}$
	}
\end{ex}
%Câu 15
\begin{ex}
	Hằng ngày mực nước của con kênh lên xuống theo thủy triều. Độ sâu $h$ (mét) của mực nước trong kênh tính theo thời gian $t$ (giờ) được cho bởi công thức $h(t)=3\cos \left(\dfrac{\pi t}{6}+\dfrac{\pi }{4}\right)+14$.
	\choiceTF
	{Công thức tuần hoàn với chu kì $T=2\pi $}
	{\True Chiều sâu của mực nước thấp nhất là $11 \text{m}$}
	{Chiều sâu của mực nước cao nhất là $14 \text{m}$}
	{\True Thời gian để mực nước cao nhất là $t=9$}
	\loigiai{
		a) Công thức có dạng $y=\cos (ax+b)$ tuần hoàn với chu kì $T=\dfrac{2\pi }{|a|}$ nên chu kì cần tìm là $T=\dfrac{2\pi }{\left| \dfrac{\pi }{6} \right|}=12$.\\
		b) Ta có $\forall t\colon -1\le \cos \left(\dfrac{\pi t}{6}+\dfrac{\pi }{4}\right)\le 1\Leftrightarrow -3\le 3\cos \left(\dfrac{\pi t}{6}+\dfrac{\pi }{4}\right)\le 3\Leftrightarrow 11\le 3\cos \left(\dfrac{\pi t}{6}+\dfrac{\pi }{4}\right)+14\le 17\Leftrightarrow 11\le h\le 17$. Vậy chiều sâu của mực nước thấp nhất là $11 \text{m}$.\\
		c) Ta có $\forall t\colon -1\le \cos \left(\dfrac{\pi t}{6}+\dfrac{\pi }{4}\right)\le 1\Leftrightarrow -3\le 3\cos \left(\dfrac{\pi t}{6}+\dfrac{\pi }{4}\right)\le 3\Leftrightarrow 11\le 3\cos \left(\dfrac{\pi t}{6}+\dfrac{\pi }{4}\right)+14\le 17\Leftrightarrow 11\le h\le 17$. Chiều sâu của mực nước cao nhất là $17 \text{m}$.\\
		d) Ta có $\forall t\colon -1\le \cos \left(\dfrac{\pi t}{6}+\dfrac{\pi }{4}\right)\le 1\Leftrightarrow -3\le 3\cos \left(\dfrac{\pi t}{6}+\dfrac{\pi }{4}\right)\le 3\Leftrightarrow 11\le 3\cos \left(\dfrac{\pi t}{6}+\dfrac{\pi }{4}\right)+14\le 17\Leftrightarrow 11\le h\le 17$. Chiều sâu của mực nước cao nhất là $17 \text{m}$.\\
		Max $h=17\Leftrightarrow \cos \left(\dfrac{\pi t}{6}+\dfrac{\pi }{4}\right)=1\Leftrightarrow \dfrac{\pi t}{6}+\dfrac{\pi }{4}=k2\pi \Leftrightarrow t=-3+12k,k\in \mathbb{Z}$.\\
		Vì thời gian không âm và $k\in \mathbb{Z}$ nên ta chọn $t=1$. Vậy thời gian ngắn nhất $t=-3+12=9$
	}
\end{ex}
%Câu 16
\begin{ex}
	Cho phương trình $\left(2\cos x-1\right)\left(\sin 2x-m\right)=0$ (1).
	\choiceTF
	{\True $x=\dfrac{7\pi }{3}$ là một nghiệm của phương trình $(1)$}
	{Khi $m=2$ thì phương trình $(1)\Leftrightarrow \hoac{& x=\pm\dfrac{\pi }{3}+k2\pi \\& x=\dfrac{\pi }{2}+l2\pi} (k,l \in \mathbb{Z})$}
	{\True Khi $m=1$ thì tập nghiệm của phương trình $(1)$ có tất cả 4 điểm biểu diễn trên đường tròn lượng giác}
	{Chỉ tìm được một giá trị của $m$ để phương trình $(1)$ có đúng hai nghiệm thuộc $\left(-\dfrac{\pi }{4};\dfrac{3\pi }{4}\right]$}
			\loigiai{
			Ta có $\left(2\cos x-1\right)\left(\sin 2x-m\right)=0\Leftrightarrow \hoac{& \cos x=\dfrac{1}{2} \\& \sin 2x=m}\Leftrightarrow \hoac{& x=\dfrac{\pi }{3}+k2\pi \\& x=-\dfrac{\pi }{3}+k2\pi \\& \sin 2x=m}$\\
			a) Thay $x=\dfrac{7\pi }{3}$ phương trình $(1)$ ta thấy thỏa mãn nên $x=\dfrac{7\pi }{3}$ là một nghiệm của phương trình $(1)$.\\
			b) Khi $m=2$ thì phương trình $(1)\Leftrightarrow \hoac{& x=\dfrac{\pi }{3}+k2\pi \\& x=-\dfrac{\pi }{3}+k2\pi} (k \in \mathbb{Z})$\\
			c) Khi $m=1$ phương trình $(1)\Leftrightarrow \hoac{& x=\dfrac{\pi }{3}+k2\pi \\& x=-\dfrac{\pi }{3}+k2\pi \\& \sin 2x=1}\Leftrightarrow \hoac{& x=\dfrac{\pi }{3}+k2\pi \\& x=-\dfrac{\pi }{3}+k2\pi \\& x=\dfrac{\pi }{4}+l\pi}$.\\
			Do đó tập nghiệm của phương trình $(1)$ có tất cả $4$ điểm biểu diễn trên đường tròn lượng giác.\\
			d) Do phương trình $(2)$ có một nghiệm $x=\dfrac{\pi }{3}$ thuộc $\left(-\dfrac{\pi }{4};\dfrac{3\pi }{4}\right]$.\\
			Do đó để phương trình $(1)$ có đúng hai nghiệm thuộc $\left(-\dfrac{\pi }{4};\dfrac{3\pi }{4}\right]$ thì phương trình $\sin 2x=m$ có 1 nghiệm thuộc $\left(-\dfrac{\pi }{4};\dfrac{3\pi }{4}\right]$ khác $\dfrac{\pi }{3}$ (*)\\
			Ta có $x\in \left(-\dfrac{\pi }{4};\dfrac{3\pi }{4}\right]\Rightarrow 2x\in \left(-\dfrac{\pi }{2};\dfrac{3\pi }{2}\right]$ hay $2x\in \left[0;2\pi\right]$\\
			Từ (*) suy ra $m=1$ hoặc $m=-1$\\
	}
\end{ex}

\TNSA
%Câu 17
\begin{ex}
	Cho góc $\alpha $ thỏa mãn $\sin \alpha =\dfrac{1}{5}$. Khi đó giá trị biểu thức $P={{\cos }^2}2x+{{\cos }^2}x$ bằng $\dfrac{a}{b}$. Tính $a+b$. Biết rằng phân số $\dfrac{a}{b}$ là phân số tối giản
	\shortans{1754}
	\loigiai{
		Biến đổi biểu thức $P$ rồi thay giá trị $\sin \alpha =\dfrac{1}{5}$ vào $P$, ta được:\\
		$\begin{aligned}
				& P={{\cos }^2}2x+{{\cos }^2}x \\& \text{ }={{\left(1-2{{\sin }^2}\alpha\right)}^2}+\left(1-{{\sin }^2}\alpha\right)={{\left(1-2 \cdot {{\left(\dfrac{1}{5}\right)}^2}\right)}^2}+\left(1-{{\left(\dfrac{1}{5}\right)}^2}\right)=\dfrac{1129}{625} \end{aligned}$\\
		$\Rightarrow \heva{& a=1129 \\& b=625}\Rightarrow a+b=1754$
	}
\end{ex}
%Câu 18
\begin{ex}
	Số điểm chung của đồ thị hàm số $y=\sin x$ và $y=\cos x$ trên $\left[ -\dfrac{\pi }{2};\dfrac{3\pi }{2} \right]$ là $n$. Giá trị $\sqrt{n}$ (làm tròn đến hàng phần trăm) bằng
	\shortans{1,41}
	\loigiai{
		Số điểm chung của đồ thị hàm số $y=\sin x$ và $y=\cos x$ trên $\left[ -\dfrac{\pi }{2};\dfrac{3\pi }{2} \right]$ bằng số nghiệm phương trình $\sin x = \cos x$ trên $\left[ -\dfrac{\pi }{2};\dfrac{3\pi }{2} \right]$.\\
		Ta có $\sin x = \cos x \Leftrightarrow \sin x - \cos x =0 \Leftrightarrow \sin \left(x-\dfrac{\pi}{4} \right)=0 \Leftrightarrow x-\dfrac{\pi}{4}=k \pi \Leftrightarrow x= \dfrac{\pi}{4} +k\pi \ (k \in \mathbb{Z})$.\\
		$x \in \left[ -\dfrac{\pi }{2};\dfrac{3\pi }{2} \right]$ nên $x \in \left\{ \dfrac{\pi}{4}; \dfrac{5\pi}{4}\right\}$.\\
		Vậy $n=2$ nên $\sqrt{n} \approx 1,41$.
	}
\end{ex}
%Câu 19
\begin{ex}
	Biết có $n$ giá trị nguyên của tham số $m$ để phương trình $\cos x=m$ có nghiệm. Giá trị $\sqrt{n}$ (làm tròn đến hàng phần trăm) bằng
	\shortans{1,73}
	\loigiai{
		$\cos x=m$ có nghiệm $\Leftrightarrow -1\le m\le 1$. Mà $m\in \mathbb{Z}\Rightarrow m\in \left\{ -1;0;1 \right\}$. Vậy $\sqrt{n}\approx 1{,}73$
	}
\end{ex}
%Câu 20
\begin{ex}
	Biết $x=x_0$ là nghiệm duy nhất của phương trình $2\sin \left(x-\dfrac{\pi }{6}\right)+2=0$ trên khoảng $\left(0;2\pi\right)$. Giá trị $x_0$ (làm tròn đến hàng phần trăm) bằng
	\shortans{5,24}
	\loigiai{
		Ta có: $2\sin \left(x-\dfrac{\pi }{6}\right)+2=0\Leftrightarrow \sin \left(x-\dfrac{\pi }{6}\right)=-1\Leftrightarrow x=-\dfrac{\pi }{3}+k2\pi ,k\in \mathbb{Z}$\\
		Do $x\in \left(0;2\pi\right)$ nên $0<-\dfrac{\pi }{3}+k2\pi <2\pi \Leftrightarrow \dfrac{1}{6}<k<\dfrac{7}{6}\Leftrightarrow k=1$.\\
		Vậy phương trình có một nghiệm $x=\dfrac{5\pi }{3}\approx 5{,}24$
	}
\end{ex}
%Câu 21
\begin{ex}
	Gọi $M$ và $m$ lần lượt là giá trị lớn nhất và giá trị nhỏ nhất của hàm số $y=\sin x+\sqrt{3}\cos x+\sqrt{2}$. Tính $M^2m$ (làm tròn đến hàng phần trăm)
	\shortans{6,83}
	\loigiai{
		Ta có $y=\sin x+\sqrt{3}\cos x+\sqrt{2}=2\left(\dfrac{1}{2}\sin x+\dfrac{\sqrt{3}}{2}\cos x\right)+\sqrt{2}=2\sin \left(x+\dfrac{\pi }{3}\right)+\sqrt{2}$.\\
		Suy ra $M=2+\sqrt{2}$, $m=-2+\sqrt{2}$. Nên $M^2m\approx 6{,}83$
	}
\end{ex}
%Câu 22
\begin{ex}
	Mùa xuân ở Hội Lim (tỉnh Bắc Ninh) thường có trò chơi đu. Khi người chơi đu nhún đều, cây đu sẽ đưa người chơi đu dao động qua lại vị trí cân bằng. Nghiên cứu trò chơi này, người ta thấy khoảng cách $h$ (mét) được tính từ vị trí chân người chơi đu đến vị trí cân bằng được biểu diễn bởi hệ thức $h=|d|$ với $d=3\cos \left[\dfrac{\pi }{3}(2t-1)\right]$ ($t\ge 0$ và được tính bằng giây), trong đó ta quy ước $d>0$ khi vị trí cân bằng ở về phía sau lưng người chơi đu và $d<0$ trong trường hợp ngược lại.
	Biết $t_1$, $t_2$ lần lượt là thời điểm đầu tiên người đu ở vị trí phía sau lưng và vị trí phía trước vị trí cân bằng $1{,}5$ mét. Giá trị $t_1+t_2^2$ (làm tròn đến hàng phần trăm) bằng
	\shortans{3,25}
	\loigiai{
	Người chơi cách vị trí cân bằng 1 mét khi $3\cos \left[\dfrac{\pi }{3}(2t-1)\right]=\pm 1{,}5$\\
	$\Leftrightarrow \cos^2\left[\dfrac{\pi }{3}(2t-1)\right]=\dfrac{1}{4}\Leftrightarrow \cos \left[\dfrac{2\pi }{3}(2t-1)\right]=-\dfrac{1}{2}$ $\Leftrightarrow \hoac{& \dfrac{2\pi }{3}(2t-1)=\dfrac{2\pi }{3}+k2\pi \\& \dfrac{2\pi }{3}(2t-1)=-\dfrac{2\pi }{3}+k2\pi} \left(k\in \mathbb{Z}\right)
		\Leftrightarrow \hoac{& t=1+\dfrac{3k}{2} \\& t=\dfrac{3k}{2}}\left(k\in \mathbb{Z}\right)$.\\
	Vì $t>0$ nên $t_1=1$ và $t_2=1{,}5$. Vậy $t_1+t_2^2=3{,}25$
	}
\end{ex}

\Closesolutionfile{ans}

\indapan{10}{ans/ans-SA-C1-De1}