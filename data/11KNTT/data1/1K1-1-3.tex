

\begin{dang}{Độ dài của một cung tròn}
	
\end{dang}
\subsubsection{Ví dụ mẫu}
\begin{vd}%[1K1Y1-3]
	Một đường tròn có bán kính $20$ cm. Tìm độ dài các cung trên đường tròn đó có số đo sau:
	\begin{multicols}{4}
		\begin{enumerate}
			\item $\dfrac{\pi}{12}$;
			\item $1{,}5$;
			\item $35^\circ$;
			\item $315^\circ$.
		\end{enumerate}
	\end{multicols}
	\loigiai{
		\begin{enumerate}
			\item $l=R\alpha=20\cdot\dfrac{\pi}{12}=\dfrac{5\pi}{3}$ cm;
			\item $l=R\alpha=20\cdot1{,}5=30$ cm;
			\item Đổi $35^{\circ}=35 \cdot \dfrac{\pi}{180}=\dfrac{7\pi}{36} \mathrm{rad}$.\\
			Độ dài cung tròn là  $l=R\alpha=20\cdot\dfrac{7\pi}{36}=\dfrac{35\pi}{9}$ cm;
			\item Đổi $315^{\circ}=315 \cdot \dfrac{\pi}{180}=\dfrac{7\pi}{4} \mathrm{rad}$.\\
			Độ dài cung tròn là  $l=R\alpha=20\cdot\dfrac{7\pi}{4}=35$ cm.
		\end{enumerate}
	}
\end{vd}
%D:\DCHT-11-KNTT\DCHT-11-KNTT\DATA\1K1-1.tex
\begin{vd}%[1K1B1-3]
	Một vệ tinh được định vị tại vị trí $A$ trong không gian. Từ vị trí $A$, vệ tinh bắt đầu chuyển động quanh Trái Đất theo quỹ đạo là đường tròn với tâm là tâm $O$ của Trái Đất, bán kính $9~000$ km. Biết rằng vệ tinh chuyển động hết một vòng của quỹ đạo trong $2$ giờ.
	\begin{enumerate}
		\item Hãy tính quãng đường vệ tinh đã chuyển động được sau $1$ giờ; $3$ giờ; $5$ giờ.
		\item Vệ tinh chuyển động được quãng đường $200~000$ km sau bao nhiêu giờ (làm tròn đến kết quả hàng đơn vị)?
	\end{enumerate}
	\loigiai{
		\begin{enumerate}
			\item Sau $1$ giờ, vệ tinh chuyển động hết $\dfrac{1}{2}$ vòng của quỹ đạo.\\
			Suy ra quãng đường vệ tinh đã chuyển động là $$S=\dfrac{1}{2}\cdot 2\pi\cdot  9~000=9~000\pi\approx 28247{,}3\text{ km.}$$
			Sau $3$ giờ, vệ tinh chuyển động hết $\dfrac{3}{2}$ vòng của quỹ đạo.\\
			Suy ra quãng đường vệ tinh đã chuyển động là $$S=\dfrac{3}{2}\cdot 2\pi \cdot 9~000=27~000\pi\approx 84823\text{ km.}$$
			Sau $5$ giờ, vệ tinh chuyển động hết $\dfrac{5}{2}$ vòng của quỹ đạo.\\
			Suy ra quãng đường vệ tinh đã chuyển động là $$S=\dfrac{5}{2}\cdot 2\pi\cdot 9~000=45~000\pi\approx 141371{,}7\text{ km.}$$
			\item Vệ tinh chuyển động được quãng đường $200~000$ km. Gọi $x$ là thời gian vệ tinh chuyển động. Khi đó
			$$200~000=x\cdot 2\pi \cdot 9~000\Leftrightarrow x\approx 11{,}1\text{ giờ}.$$
		\end{enumerate}
	}
\end{vd}

\subsubsection{Bài tập rèn luyện}
\centerline{\fcolorbox{red}{yellow!50}{\bf {BÀI TẬP TỰ LUẬN }}}
\begin{bt}%[1K1Y1-3]
	Một đường tròn có bán kính $R=75$ cm. Độ dài của cung trên đường tròn đó có số đo $\alpha =\dfrac{\pi}{25}$ là \dapso{$3\pi$ cm}
	\loigiai{
		Độ dài của cung trên đường tròn đó có số đo $\alpha =\dfrac{\pi}{25}$ là $\ell=R \cdot \alpha =75 \cdot \dfrac{\pi}{25}=3\pi$ cm.}
\end{bt}
\begin{bt}%[1K1Y1-3]
	Trên đường tròn bán kính bằng $4$, cung có số đo $\dfrac{\pi}{8}$ thì có độ dài là bao nhiêu? \dapso{$\dfrac{\pi}{2}$}
	\loigiai{
		Cung có số đo $\alpha $ rad của đường tròn bán kính $R$ có độ dài $l=R\cdot\alpha $.\\
		Vậy $\alpha =\dfrac{\pi}{8}$; $R=4$ thì $l=R\cdot\alpha =\dfrac{\pi}{2}$.}
\end{bt}
\begin{bt}%[1K1B1-3]
	Trạm vũ trụ Quốc tế ISS (tên Tiếng Anh: International Space Station) nằm trong quỹ đạo tròn cách bề mặt Trái Đất khoảng $400$ km. Nếu trạm mặt đất theo dõi được trạm vũ trụ ISS khi nó nằm trong góc $45^\circ$ ở tâm của quỹ đạo tròn này phía trên ăng-ten theo dõi, thì trạm vũ trụ ISS đã di chuyển được bao nhiêu kilômét trong khi nó đang được trạm mặt đất theo dõi? Giả sử rằng bán kính của Trái Đất là $6400$ km. Làm tròn kết quả đến hàng đơn vị. \dapso{$5341$ km}
	\loigiai{
		Bán kính quỹ đạo của trạm vũ trụ quốc tế là $R=6400+400=6800$ (km).\\
		Đổi $45^{\circ}=45 \cdot \dfrac{\pi}{180}=\dfrac{\pi}{4} \mathrm{rad}$.\\
		Vậy trong khi được trạm mặt đất theo dõi, trạm ISS đã di chuyển một quãng đường có độ dài là $$l=R\alpha=6800\cdot\dfrac{\pi}{4}\approx 5340{,}708\approx 5341~ (\text{km}).$$
	}
\end{bt}
\begin{bt}%[1K1B1-3]
	\immini{Hải lí là một đơn vị chiều dài hàng hải, được tính bằng độ dài một cung chắn một góc $\alpha=\left(\dfrac{1}{60}\right)^{\circ}$ của đường kinh tuyến (Hình 17). Đổi số đo $\alpha$ sang radian và cho biết 1 hải lí bằng khoảng bao nhiêu kilômét, biết bán kính trung bình của Trái Đất là $6371$ km. Làm tròn kết quả đến hàng phần trăm. \dapso{$1{,}85$ km}}
	{{\begin{tikzpicture}[scale=1.2, font=\footnotesize,line join=round, line cap=round, >=stealth];
				\draw[fill=cyan ] (0,0) circle (1.7);		
				\draw (-2,0) -- (2,0) ;
				\draw (0,-2)node[right] {\scriptsize $\text{Cực Nam}$} -- (0,2) node[right] {\scriptsize $\text{Cực Bắc}$};
				\path
				(0,0) coordinate (O)
				(48:1.7) coordinate (B)
				(55:1.7) coordinate (C);
				\draw  (0:0)--(48:1.7) (0:0)--(55:1.7);			
				\draw  (62:2) node[right,rotate=-43]{$\text{hải lí}$};
				\draw (0:1.7) arc (0:360:1.7);
				\draw (15:0.9) node{$\alpha=\left(\tfrac{1}{60}\right)^{\circ}$};		
				\draw (0.5,0) node[below] {Đường xích đạo};
				\pic["$ $"{shift={(1pt,3pt)}},draw,angle radius=6mm,angle eccentricity=1.79] {angle = B--O--C};	
				\path (270:2.8) node{Hình 17};	
	\end{tikzpicture}}}
	\loigiai{
		1 hải lí = $\alpha\cdot R=\dfrac{1\cdot \pi }{60\cdot 180}\cdot 6371 \approx 1{,}85$ km.
		
	}
	
\end{bt}
\begin{bt}%[1K1B1-3]
	Bánh xe của người đi xe đạp quay được $11$ vòng trong $5$ giây. Tính độ dài quãng đường mà người đi xe đã đi được trong $1$ phút, biết rằng đường kính của bánh xe đạp là $680$ mm.\dapso{$89670\pi$ mm}
	\loigiai{ Đổi $1$ phút $=60$ s.\\
			Trong $60$ giây, bánh xe quay được số vòng là $\dfrac{11}{5}\cdot60=132$ (vòng).\\
			Chu vi mỗi vòng xe là $680\pi$ (mm).\\
			Độ dài quãng đường người đó đi trong $1$ phút là $132\cdot680\pi=89760\pi$ (mm).
	}
\end{bt}
\centerline{\fcolorbox{red}{yellow!50}{\bf {CÂU HỎI TRẮC NGHIỆM}}}
\Opensolutionfile{ans}[ans/ans-1K1-1-Dang3]
\begin{ex}%[1K1Y1-3]
	Tính độ dài cung tròn có số đo góc ở tâm bằng $\dfrac{\pi}{6}$ của đường tròn lượng giác.
	\choice
	{\True $\dfrac{\pi}{6}$}
	{$\dfrac{\pi}{12}$}
	{$\dfrac{\pi}{3}$}
	{$\dfrac{\pi}{24}$}
	\loigiai{
		Đường tròn lượng giác có bán kính $R=1$. \\
		Độ dài cung tròn có số đo góc ở tâm bằng $\alpha=\dfrac{\pi}{6}$ là $\ell = R\alpha = 1 \cdot \dfrac{\pi}{6}=\dfrac{\pi}{6}$.
	}
\end{ex}

\begin{ex}%[1K1Y1-3]
	Trên đường tròn lượng giác đường kính $36$ , cung có số đo $\dfrac{\pi}{6}$ thì có độ dài bằng bao nhiêu?
	\choice
	{$l=\dfrac{\pi}{54}$}
	{$l=\dfrac{54}{\pi}$}
	{\True$l=3 \pi$}
	{$l=6 \pi$}
	\loigiai{
		Độ dài cung có số đo $\dfrac{\pi}{6}$ là $l=R\cdot \alpha \Rightarrow l=18 \cdot \dfrac{\pi}{6}=3 \pi$.
	}
\end{ex}
\begin{ex}%[1K1Y1-3]
	Một đường tròn có bán kính $R=3$ cm. Tính độ dài $\ell$ của cung trên đường tròn đó có số đo bằng $60^{\circ}$.
	\choice
	{$\ell=\dfrac{\pi}{2}$ cm}
	{\True $\ell=\pi $ cm}
	{$\ell=\dfrac{\pi}{4}$ cm}
	{$\ell=2\pi$ cm}
	\loigiai{
		Ta có số đo cung $ \alpha = \dfrac{\pi}{3}$.\\
		$\ell=R \cdot \alpha =3 \cdot\dfrac{\pi}{3} =\pi$ cm.}
\end{ex}

\begin{ex}%[1K1Y1-3]
	Một cung tròn có độ dài bằng bán kính. Khi đó số đo bằng rađian của cung tròn đó là
	\choice
	{$2$}
	{$3$}
	{$\pi $}
	{\True $1$}
	\loigiai{
		Theo định nghĩa $1$ rađian là số đo của cung có độ dài bằng bán kính.}
\end{ex}

\begin{ex}%[1K1Y1-3]
	Cung tròn bán kính $ R=4 $ cm và có số đo $ \dfrac{3\pi}{4} $ thì có độ dài là (kết quả làm tròn đến số thập phân thứ $ 2 $)
	\choice
	{$ 0{,59} $ cm}
	{$ 1{,}05 $ cm}
	{$ 17{,}76 $ cm}
	{\True $ 9{,}42 $ cm}
	\loigiai{
		Ta có $ l=\alpha \cdot R = \dfrac{3\pi}{4}\cdot 4=3\pi \approx 9{,}42$.
	}
\end{ex}

\begin{ex}%[1K1Y1-3]
	Trên đường tròn bán kính bằng $4$, cung có số đo $\dfrac{\pi}{8}$ thì có độ dài là
	\choice
	{$\dfrac{\pi}{3}$}
	{\True $\dfrac{\pi}{2}$}
	{$\dfrac{\pi}{16}$}
	{$\dfrac{\pi}{4}$}
	\loigiai{
		Cung có số đo $\alpha $ rad của đường tròn bán kính $R$ có độ dài $l=R\cdot\alpha $.\\
		Vậy $\alpha =\dfrac{\pi}{8}$; $R=4$ thì $l=R\cdot\alpha =\dfrac{\pi}{2}$.}
\end{ex}

\begin{ex}%[1K1Y1-3]
	Cho đường tròn $ (O) $ đường kính bằng $ 10 $ cm. Tính độ dài cung có số đo $ \dfrac{7\pi}{12} $.
	\choice
	{$ \dfrac{17\pi}{3} $ cm}
	{$ \dfrac{35\pi}{2} $ cm}
	{$ \dfrac{35\pi}{6} $ cm}
	{\True $ \dfrac{35\pi}{12} $ cm}
	\loigiai{
		Độ dài cung tròn cần tìm là $ l=\alpha \cdot R=\dfrac{7\pi}{12} \cdot \dfrac{10}{2}=\dfrac{35\pi}{12} $ (cm).
	}
\end{ex}

\begin{ex}%[1K1Y1-3]
	Trên đường tròn có đường kính $20$ cm. Độ dài của một cung tròn có số đo $\dfrac{\pi}{4}$ là
	\choice
	{$\dfrac{5}{2}$ cm}
	{$5$ cm}
	{\True $\dfrac{5\pi}{2}$ cm}
	{$5\pi$ cm}
	\loigiai{
		Đường tròn có bán kính $R=10$ cm.\\
		Gọi $\ell$ là độ dài cung tròn cần tìm, ta có $\ell =R\cdot\alpha=10\cdot\dfrac{\pi}{4}=\dfrac{5\pi}{2}$.
	}
\end{ex}

\begin{ex}%[1K1Y1-3]
	Một đường tròn có bán kính $4 $ cm. Độ dài cung tròn có số đo $45^{\circ}$ là
	\choice
	{$\dfrac{1}{20 \pi} $ cm}
	{$9 $ cm}
	{$180 $ cm}
	{\True $\pi $ cm}
	\loigiai{
		Áp dụng tính độ dài cung tròn công thức $\ell=\dfrac{\pi R\alpha}{180}$, suy ra độ dài cung tròn là $\ell=\dfrac{\pi\cdot 4\cdot 45}{180}= \pi$.}
\end{ex}

\begin{ex}%[1K1Y1-3]
	Một đường tròn có đường kính bằng $10 \text{ cm}$. Tính độ dài $l$ của cung tròn có số đo $\dfrac{\pi}{5}$.
	\choice
	{\True $l=2 \pi\text{ cm}$}
	{$l=\pi\text{ cm}$}
	{$l=5 \pi\text{ cm}$}
	{$l=1\text{ cm}$}
	\loigiai{
		Ta có $l=10\cdot \dfrac{\pi}{5} =2\pi \text{ cm}$.
	}
\end{ex}
\Closesolutionfile{ans}

