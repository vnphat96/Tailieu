\setcounter{deso}{0}
\begin{name}
	{\tenchude}
	{ĐỀ ÔN TẬP CHƯƠNG I}
	{LỚP TOÁN THẦY PHÁT}
	{\thoigian}
\end{name}
\TN
\Opensolutionfile{ans}[ans/ansBONPA-0D1-1-De1]
\begin{ex} %[1D1H4-2]
	Tập xác định của hàm số $y = \dfrac{\cos x}{\sin x - 1}$ là
	\choice
	{$\mathbb{R} \setminus \{k2\pi \mid k \in \mathbb{Z}\}$}
	{\True $\mathbb{R} \setminus \left\{ \dfrac{\pi}{2} + k2\pi \mid k \in \mathbb{Z} \right\}$}
	{$\mathbb{R} \setminus \left\{ \dfrac{\pi}{2} + k\pi \mid k \in \mathbb{Z} \right\}$}
	{$\mathbb{R} \setminus \{k\pi \mid k \in \mathbb{Z}\}$}
	\loigiai{
		Biểu thức $\dfrac{\cos x}{\sin x - 1}$ có nghĩa khi $\sin x - 1 \ne 0 \Leftrightarrow \sin x \ne 1 \Leftrightarrow x \ne \dfrac{\pi}{2} + k2\pi,\ k \in \mathbb{Z}$.\\
		Vậy tập xác định là $D = \mathbb{R} \setminus \left\{ \dfrac{\pi}{2} + k2\pi \mid k \in \mathbb{Z} \right\}$.
	}
\end{ex}

\begin{ex} %[1D1H1-2]
	Cung có số đo $250^\circ$ thì số đo theo đơn vị radian là
	\choice
	{\True $\dfrac{25\pi}{18}$}
	{$\dfrac{25\pi}{12}$}
	{$\dfrac{25\pi}{9}$}
	{$\dfrac{35\pi}{18}$}
	\loigiai{
		Ta có $250^\circ = \dfrac{\pi}{180} \cdot 250 = \dfrac{25\pi}{18}$.
	}
\end{ex}

\begin{ex} %[1D1H1-2]
	Nếu một cung tròn có số đo bằng $\dfrac{5\pi}{4}$ radian thì số đo bằng độ của cung tròn đó là
	\choice
	{172$^\circ$}
	{15$^\circ$}
	{\True 225$^\circ$}
	{5$^\circ$}
	\loigiai{
		Ta có $a^\circ = \dfrac{180^\circ}{\pi} \cdot \dfrac{5\pi}{4} = 225^\circ$.
	}
\end{ex}

\begin{ex} %[1D1H2-2]
	Cho $\sin \alpha = \dfrac{3}{5}$ và $90^\circ < \alpha < 180^\circ$. Tính $\cos \alpha$.
	\choice
	{$\cos \alpha = -\dfrac{5}{4}$}
	{\True $\cos \alpha = -\dfrac{4}{5}$}
	{$\cos \alpha = \dfrac{4}{5}$}
	{$\cos \alpha = \dfrac{5}{4}$}
	\loigiai{
		Ta có: $\sin^2 \alpha + \cos^2 \alpha = 1 \Rightarrow \cos^2 \alpha = 1 - \sin^2 \alpha = 1 - \left( \dfrac{3}{5} \right)^2 = \dfrac{16}{25} \Rightarrow \cos \alpha = \pm \dfrac{4}{5}$.\\
		Mặt khác $90^\circ < \alpha < 180^\circ$ nên $\cos \alpha < 0$.\\
		Vậy $\cos \alpha = -\dfrac{4}{5}$.
	}
\end{ex}

\begin{ex} %[1D1H4-5]
	Trong các hàm số sau đây, hàm số nào là hàm tuần hoàn?
	\choice
	{$y = \tan x + x$}
	{$y = x^2 + 1$}
	{\True $y = \cot x$}
	{$y = \dfrac{\sin x}{x}$}
	\loigiai{
		Hàm số $y = \cot x$ là hàm tuần hoàn với chu kì $\pi$.
	}
\end{ex}

\begin{ex} %[1D1H4-7]
	Đồ thị của các hàm số $y = \sin x$ và $y = \cos x$ cắt nhau tại bao nhiêu điểm có hoành độ thuộc đoạn $\left[-2\pi; \dfrac{5\pi}{2}\right]$?
	\choice
	{\True 5}
	{6}
	{4}
	{7}
	\loigiai{
		Hoành độ giao điểm của hai đồ thị $y = \sin x$ và $y = \cos x$ là nghiệm của phương trình:
		\[
		\sin x = \cos x \Leftrightarrow \tan x = 1 \quad \text{(do } \tan x = \dfrac{\sin x}{\cos x} \text{)}
		\]
		\[
		\Leftrightarrow x = \dfrac{\pi}{4} + k\pi, \quad k \in \mathbb{Z}.
		\]
		Xét điều kiện:
		\[
		-2\pi \le \dfrac{\pi}{4} + k\pi \le \dfrac{5\pi}{2} \Leftrightarrow \dfrac{-9\pi}{4} \le k\pi \le \dfrac{9\pi}{4} \Leftrightarrow -2,25 \le k \le 2,25.
		\]
		Mà $k \in \mathbb{Z}$ nên $k = -2; -1; 0; 1; 2$.\\
		Vậy có $5$ giá trị của $k$ tương ứng với $5$ điểm cắt nhau trên đoạn đã cho.
	}
\end{ex}

\begin{ex} %[1D1H3-3]
	Cho $\sin \alpha = \dfrac{3}{4}$. Khi đó, $\cos 2\alpha$ bằng
	\choice
	{\True $-\dfrac{1}{8}$}
	{$\dfrac{\sqrt{7}}{4}$}
	{$-\dfrac{\sqrt{7}}{4}$}
	{$\dfrac{1}{8}$}
	\loigiai{
		Ta có $\cos 2\alpha = 1 - 2\sin^2 \alpha = 1 - 2 \left( \dfrac{3}{4} \right)^2 = -\dfrac{1}{8}$.
	}
\end{ex}

\begin{ex} %[1D1H3-4]
	Biểu thức $\dfrac{\sin 10^\circ + \sin 20^\circ}{\cos 10^\circ + \cos 20^\circ}$ bằng
	\choice
	{$\tan 10^\circ + \tan 20^\circ$}
	{$\tan 30^\circ$}
	{$\cot 10^\circ + \cot 20^\circ$}
	{\True $\tan 15^\circ$}
	\loigiai{
		$\dfrac{\sin 10^\circ + \sin 20^\circ}{\cos 10^\circ + \cos 20^\circ} = \dfrac{2\sin 15^\circ \cos 5^\circ}{2\cos 15^\circ \cos 5^\circ} = \tan 15^\circ$.
	}
\end{ex}

\begin{ex} %[1D1H4-2]
	Tập xác định của hàm số $y = \tan \left( 2x - \dfrac{\pi}{3} \right)$ là
	\choice
	{\True $\mathbb{R} \setminus \left\{ \dfrac{5\pi}{12} + k \dfrac{\pi}{2} \mid k \in \mathbb{Z} \right\}$}
	{$\mathbb{R} \setminus \left\{ \dfrac{5\pi}{12} + k\pi \mid k \in \mathbb{Z} \right\}$}
	{$\mathbb{R} \setminus \left\{ \dfrac{5\pi}{12} + k\dfrac{\pi}{6} \mid k \in \mathbb{Z} \right\}$}
	{$\mathbb{R} \setminus \left\{ \dfrac{5\pi}{12} + k\dfrac{\pi}{4} \mid k \in \mathbb{Z} \right\}$}
	\loigiai{
		Hàm số xác định khi $\cos \left( 2x - \dfrac{\pi}{3} \right) \ne 0 \Leftrightarrow 2x - \dfrac{\pi}{3} \ne \dfrac{\pi}{2} + k\pi \Leftrightarrow x \ne \dfrac{5\pi}{12} + \dfrac{k\pi}{2},\ k \in \mathbb{Z}$.\\
		Vậy tập xác định là $\mathbb{R} \setminus \left\{ \dfrac{5\pi}{12} + \dfrac{k\pi}{2} \mid k \in \mathbb{Z} \right\}$.
	}
\end{ex}

\begin{ex} %[1D1H4-5]
	Hàm số $y = \sin 2x$ có chu kỳ là
	\choice
	{$T = 2\pi$}
	{$T = \dfrac{\pi}{2}$}
	{\True $T = \pi$}
	{$T = 4\pi$}
	\loigiai{
		Hàm số $y = \sin 2x$ tuần hoàn với chu kỳ $T = \dfrac{2\pi}{2} = \pi$.
	}
\end{ex}

\begin{ex} %[1D1H4-4]
	Khẳng định nào dưới đây là \textbf{sai}?
	\choice
	{\True Hàm số $y = \cos x$ là hàm số lẻ}
	{Hàm số $y = \cot x$ là hàm số lẻ}
	{Hàm số $y = \sin x$ là hàm số lẻ}
	{Hàm số $y = \tan x$ là hàm số lẻ}
	\loigiai{
		\begin{itemize}
			\item Hàm số $y = \cos x$ là hàm số chẵn.
			\item Hàm số $y = \cot x$ là hàm số lẻ.
			\item Hàm số $y = \sin x$ là hàm số lẻ.
			\item Hàm số $y = \tan x$ là hàm số lẻ.
		\end{itemize}
		Vậy khẳng định sai là: Hàm số $y = \cos x$ là hàm số lẻ.
	}
\end{ex}

\begin{ex} %[1D1H5-3]
	Phương trình lượng giác $2\cot x - \sqrt{3} = 0$ có nghiệm là:
	\choice
	{$x = \dfrac{\pi}{6} + k2\pi$ hoặc $x = -\dfrac{\pi}{6} + k2\pi$}
	{\True $x = \text{arccot} \dfrac{\sqrt{3}}{2} + k\pi$}
	{$x = \dfrac{\pi}{6} + k\pi$}
	{$x = \dfrac{\pi}{3} + k\pi$}
	\loigiai{
		Ta có $2\cot x = \sqrt{3} \Leftrightarrow \cot x = \dfrac{\sqrt{3}}{2} \Leftrightarrow x = \text{arccot} \dfrac{\sqrt{3}}{2} + k\pi$, với $k \in \mathbb{Z}$.
	}
\end{ex}
\Closesolutionfile{ans}
\TNTF
\setcounter{ex}{0}
\Opensolutionfile{ans}[ans/ansDS-0D1-1-De1]
\begin{ex} %[1D1V3-5]
	Cho $\cos a = \dfrac{1}{3}$, $\cos b = \dfrac{1}{4}$. Khi đó:
	\choiceTF
	{\True $\sin^2 a = \dfrac{8}{9}$}
	{$\sin^2 a > \sin^2 b$}
	{\True $\sin^2 a + \sin^2 b > 1$}
	{$\cos (a+b) \cdot \cos (a-b) = \dfrac{11}{14}$}
	\loigiai{
		\begin{itemize}
			\item Ta có $\sin^2 a = 1 - \cos^2 a = 1 - \dfrac{1}{9} = \dfrac{8}{9}$, $\sin^2 b = 1 - \cos^2 b = 1 - \dfrac{1}{16} = \dfrac{15}{16}$.
			\item So sánh: $\dfrac{8}{9} = \dfrac{128}{144}$ và $\dfrac{15}{16} = \dfrac{135}{144} \Rightarrow \sin^2 a < \sin^2 b$ nên b) Sai.
			\item $\sin^2 a + \sin^2 b = \dfrac{8}{9} + \dfrac{15}{16} = \dfrac{128}{144} + \dfrac{135}{144} = \dfrac{263}{144} > 1$ nên c) Đúng.
			\item 
			\[
			\cos (a+b) \cdot \cos (a-b) = (\cos a \cos b - \sin a \sin b)(\cos a \cos b + \sin a \sin b) = \cos^2 a \cos^2 b - \sin^2 a \sin^2 b
			\]
			\[
			= \left( \dfrac{1}{3} \right)^2 \cdot \left( \dfrac{1}{4} \right)^2 - \dfrac{8}{9} \cdot \dfrac{15}{16} = \dfrac{1}{9} \cdot \dfrac{1}{16} - \dfrac{8}{9} \cdot \dfrac{15}{16} = \dfrac{1}{144} - \dfrac{120}{144} = -\dfrac{119}{144}.
			\]
			Khác $\dfrac{11}{14}$ nên d) Sai.
		\end{itemize}
	}
\end{ex}

\begin{ex} %[1D1V5-5]
	Cho phương trình lượng giác $\sin \left( 3x + \dfrac{\pi}{3} \right) = \cos \left( 2x - \dfrac{\pi}{4} \right)$, vậy:
	\choiceTF
	{\True Phương trình có nghiệm là
	$x = \dfrac{\pi}{12} + \dfrac{k2\pi}{5} \quad \text{hoặc} \quad x = -\dfrac{\pi}{12} + k2\pi, \quad k \in \mathbb{Z}$}
	{Trong khoảng $(-\pi, \pi)$ phương trình có $3$ nghiệm}
	{\True $x = -\dfrac{\pi}{12}$ là một nghiệm của phương trình thuộc khoảng $(-\pi, \pi)$} 
	{Tổng các nghiệm trong $(-\pi, \pi)$ bằng $\dfrac{\pi}{4}$}
	\loigiai{
		\begin{itemize}
			\item Giải phương trình:
			\[
			\sin \left( 3x + \dfrac{\pi}{3} \right) = \sin \left( \dfrac{3\pi}{4} - 2x \right)
			\]
			\[
			\Leftrightarrow 
			\begin{cases}
				3x + \dfrac{\pi}{3} = \dfrac{3\pi}{4} - 2x + k2\pi \\
				\text{hoặc} \\
				3x + \dfrac{\pi}{3} = \pi - \left( \dfrac{3\pi}{4} - 2x \right) + k2\pi
			\end{cases}
			\]
			Giải ra được:
			\[
			x = \dfrac{\pi}{12} + \dfrac{k2\pi}{5} \quad \text{hoặc} \quad x = -\dfrac{\pi}{12} + k2\pi, \quad k \in \mathbb{Z}.
			\]
			Vậy a) Đúng.
			\item Với $k = -2, -1, 0, 1, 2$, kiểm tra các nghiệm trong khoảng $(-\pi, \pi)$, được tổng cộng $5$ nghiệm nên b) Sai.
			\item Trong đó có nghiệm $x = -\dfrac{\pi}{12}$ thuộc $(-\pi, \pi)$ nên c) Đúng.
			\item Tổng các nghiệm trong $(-\pi, \pi)$ tính được bằng $\dfrac{\pi}{3}$ nên khác $\dfrac{\pi}{4}$ nên d) Sai.
		\end{itemize}
	}
\end{ex}

\begin{ex} %[1D1V1-4]
	Một đường tròn có bán kính $36\,\mathrm{m}$. Khi đó:
	\choiceTF
	{Cung tròn bán kính $R$ có số đo $\alpha\ (0 \le \alpha \le 2\pi)$, có số đo $a^\circ\ (0 \le a \le 360^\circ)$ và có độ dài là $l$ thì:
	$l = R\alpha = \dfrac{a}{180} \cdot \pi R$}
	{\True Độ dài cung tròn trên đường tròn có số đo $\dfrac{3\pi}{4}$ là $84,8\,\mathrm{m}$}
	{\True Độ dài cung tròn trên đường tròn có số đo $51^\circ$ là $32,04\,\mathrm{m}$}
	{Độ dài cung tròn trên đường tròn có số đo $\dfrac{1}{3}$ là $22\,\mathrm{m}$}
	\loigiai{
		\begin{itemize}
			\item Công thức tính độ dài cung tròn: $l = R\alpha = \dfrac{\pi a}{180} \cdot R$ (với $a^\circ$ là số đo góc).
			\item[a)] $l = 36 \cdot \dfrac{3\pi}{4} = 27\pi \approx 84,8\,\mathrm{m}$ $\rightarrow$ Đúng.
			\item[b)] $l = \dfrac{\pi \cdot 51}{180} \cdot 36 = \dfrac{51\pi}{5} \approx 32,04\,\mathrm{m}$ $\rightarrow$ Đúng.
			\item[c)] $l = 36 \cdot \dfrac{1}{3} = 12\,\mathrm{m}$ $\rightarrow$ Sai (vì kết quả không khớp với $22\,\mathrm{m}$).
		\end{itemize}
	}
\end{ex}

\begin{ex} %[1D1V5-5]
	Cho phương trình lượng giác $\sin^2 2x = \cos^2 \left( 3x - \dfrac{\pi}{8} \right)$, vậy:
	\choiceTF
	{\True Phương trình đã cho tương đương với phương trình $\cos \left( 6x - \dfrac{\pi}{4} \right) = \cos (\pi + 4x)$}
	{\True Trong khoảng $(-\pi, \pi)$ phương trình có $11$ nghiệm}
	{\True $x = \dfrac{37\pi}{40}$ là một nghiệm của phương trình thuộc khoảng $(-\pi, \pi)$}
	{Tổng các nghiệm trong $(-\pi, \pi)$ bằng $\dfrac{7\pi}{9}$}
	\loigiai{
		\begin{itemize}
			\item Phương trình đã cho tương đương với
			\[
			\cos \left( 6x - \dfrac{\pi}{4} \right) = \cos (\pi + 4x)
			\]
			nên a) Đúng.
			\item Giải phương trình được các nghiệm:
			\[
			x = \dfrac{5\pi}{8} + k\pi, \quad x = \dfrac{37\pi}{40} + \dfrac{k\pi}{5}, \quad k \in \mathbb{Z}.
			\]
			Đếm được $11$ nghiệm trong $(-\pi, \pi)$ nên b) Đúng.
			\item $x = \dfrac{37\pi}{40}$ là một nghiệm đúng nằm trong $(-\pi, \pi)$ nên c) Đúng.
			\item Tổng các nghiệm đã cho là $\dfrac{7\pi}{8}$ nên khác với $\dfrac{7\pi}{9}$ nên d) Sai.
		\end{itemize}
	}
\end{ex}

\Closesolutionfile{ans}
\TNSA
\setcounter{ex}{0}
\Opensolutionfile{ans}[ans/ansTLN-0D1-1-De1]

\begin{ex} %[1D1H4-5]
	Xét tính tuần hoàn của hàm số $y = \cos x$ và hàm số $y = \cot x$.
	\loigiai{
		Ta có:
		\begin{itemize}
			\item $\cos x = \cos (x + 2\pi),\ \forall x \in \mathbb{R}$.
			\item $\cot x = \cot (x + \pi),\ \forall x \ne k\pi,\ k \in \mathbb{Z}$.
		\end{itemize}
		Do đó, hàm số $y = \cos x$ và $y = \cot x$ là các hàm số tuần hoàn.
	}
\end{ex}

\begin{ex} %[1D1H4-5]
	Chứng minh rằng hàm số $T$ thỏa mãn $\sin (x + T) = \sin x$ với mọi $x \in \mathbb{R}$ phải có dạng $T = k2\pi$, $k$ là một số nguyên nào đó. Từ đó suy ra, số $T$ nhỏ nhất thỏa mãn $\sin (x + T) = \sin x$ với mọi $x \in \mathbb{R}$ là $2\pi$.
	\loigiai{
		Nếu $\sin (x + T) = \sin x$ với mọi $x$, thì khi $x = \dfrac{\pi}{2}$ ta được
		\[
		\sin \left( \dfrac{\pi}{2} + T \right) = \sin 1.
		\]
		Số $U$ mà $\sin U = 1$ thì $U$ phải có dạng $\dfrac{\pi}{2} + k2\pi$.\par
		Nên $\dfrac{\pi}{2} + T = \dfrac{\pi}{2} + k2\pi \Rightarrow T = k2\pi$.
	}
\end{ex}

\begin{ex} %[1D1H1-3]
	Cho góc lượng giác $(Ou, Ov)$ có số đo $\dfrac{\pi}{5}$. Hỏi trong các góc
	\[
	\dfrac{6\pi}{5},\ \dfrac{9\pi}{5},\ \dfrac{11\pi}{5},\ \dfrac{31\pi}{5},\ \dfrac{14\pi}{5}
	\]
	những góc nào là số đo của một góc lượng giác có cùng tia đầu, tia cuối với góc đã cho?
	\loigiai{
		Ta có:
		\[
		\dfrac{6\pi}{5} = \pi + \dfrac{\pi}{5},\quad
		\dfrac{9\pi}{5} = 2\pi - \dfrac{\pi}{5},\quad
		\dfrac{11\pi}{5} = -2\pi + \dfrac{\pi}{5},\quad
		\dfrac{31\pi}{5} = 6\pi + \dfrac{\pi}{5},\quad
		\dfrac{14\pi}{5} = -3\pi + \dfrac{\pi}{5}.
		\]
		Nhận thấy số đo của một góc lượng giác có cùng tia đầu, tia cuối với góc đã cho khi ta quay góc đó chẵn số vòng, nghĩa là số đo hơn kém nhau bội của $2\pi$.\par
		Vậy các góc thỏa mãn là:
		\[
		\dfrac{9\pi}{5},\quad \dfrac{11\pi}{5},\quad \dfrac{31\pi}{5}.
		\]
	}
\end{ex}

\begin{ex} %[1D1H4-2]
	Trong các khoảng thời gian từ $0$ giờ đến $2$ giờ $15$ phút, kim phút quét một góc lượng giác là bao nhiêu độ?
	\loigiai{
		Kim phút quét một góc là:
		\[
		2 \cdot (-360^\circ) + (-90^\circ) = -810^\circ.
		\]
	}
\end{ex}

\begin{ex} %[1D1C3-6]
	Cho $\triangle ABC$ có các cạnh $BC = a$, $AC = b$, $AB = c$ thỏa mãn hệ thức
	\[
	\dfrac{1 + \cos B}{1 - \cos B} = \dfrac{2a + c}{2a - c}.
	\]
	Hãy nhận dạng $\triangle ABC$.
	\loigiai{
		Gọi $R$ là bán kính đường tròn ngoại tiếp $\triangle ABC$. Ta có:
		\[
		\dfrac{1 + \cos B}{1 - \cos B} = \dfrac{2a + c}{2a - c}.
		\]
		Sử dụng công thức đường tròn ngoại tiếp:
		\[
		\cos B = \dfrac{b^2 + c^2 - a^2}{2bc}, \quad \text{và} \quad a = 2R \sin A, \quad b = 2R \sin B, \quad c = 2R \sin C.
		\]
		Biến đổi:
		\[
		\dfrac{1 + \cos B}{1 - \cos B} = \dfrac{2a + c}{2a - c} \Leftrightarrow 2 \sin A \cdot \cos B = 2 \sin C.
		\]
		Dẫn đến:
		\[
		a = b.
		\]
		Vậy $\triangle ABC$ là tam giác cân tại $C$.
	}
\end{ex}

\begin{ex} %[1D1C5-5]
	Số nghiệm của phương trình $\sin (2x - 40^\circ) = \dfrac{\sqrt{3}}{2}$ với $-180^\circ \le x \le 180^\circ$ là bao nhiêu?
	\loigiai{
		Ta có:
		\[
		\sin (2x - 40^\circ) = \dfrac{\sqrt{3}}{2} \Leftrightarrow \sin (2x - 40^\circ) = \sin 60^\circ.
		\]
		\[
		\Leftrightarrow 
		\begin{cases}
			2x - 40^\circ = 60^\circ + k360^\circ\\
			2x - 40^\circ = 180^\circ - 60^\circ + k360^\circ
		\end{cases}
		\Leftrightarrow 
		\begin{cases}
			2x = 100^\circ + k360^\circ\\
			2x = 160^\circ + k360^\circ
		\end{cases}
		\Leftrightarrow 
		\begin{cases}
			x = 50^\circ + k180^\circ\\
			x = 80^\circ + k180^\circ
		\end{cases}
		\]
		Xét nghiệm $x = 50^\circ + k180^\circ$:
		\[
		-180^\circ \le x \le 180^\circ \Leftrightarrow -180^\circ \le 50^\circ + k180^\circ \le 180^\circ \Leftrightarrow -\dfrac{23}{18} \le k \le \dfrac{13}{18}.
		\]
		Vì $k \in \mathbb{Z}$ nên $k = -1$ (nếu $x = -130^\circ$) hoặc $k = 0$ (nếu $x = 50^\circ$).\par
		
		Xét nghiệm $x = 80^\circ + k180^\circ$:
		\[
		-180^\circ \le x \le 180^\circ \Leftrightarrow -180^\circ \le 80^\circ + k180^\circ \le 180^\circ \Leftrightarrow -\dfrac{13}{9} \le k \le \dfrac{5}{9}.
		\]
		Vì $k \in \mathbb{Z}$ nên $k = -1$ (nếu $x = -100^\circ$) hoặc $k = 0$ (nếu $x = 80^\circ$).\par
		
		Vậy có tất cả $4$ nghiệm thỏa mãn bài toán.
	}
\end{ex}

%\begin{ex}

%\shortans{}
%\loigiai{
%}
%\end{ex}
%\Closesolutionfile{ans}
%\begin{center}	
%\fontfamily{qag}\selectfont\color{violet} 	\centering{\textbf{BẢNG ĐÁP ÁN}}
%\end{center}
%\inputansbox{12}{ans/ansBONPA-0D1-1-De1}
%\inputansbox{4}{ans/ansDS-0D1-1-De1}
%\inputansbox{6}{ans/ansTLN-0D1-1-De1}


