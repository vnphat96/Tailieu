\begin{dang}{Phương trình bận $ n $ theo một hàm số lượng giác}
	Quan sát và dùng các công thức biến đổi để đưa phương trình về cùng một hàm lượng giác (cùng $\sin$ hoặc cùng $\cos$ hoặc cùng $\tan$ hoặc cùng $\cot$) với cung góc giống nhau, chẳng hạn:
	\begin{center}
		\begin{tabular}{|c|c|c|}
			\hline
			Dạng & Đặt ẩn phụ & Điều kiện\\
			\hline
			$a\sin^2x+b\sin x+c=0$ & $t=\sin x$ & $-1\le t \le 1$\\
			\hline
			$a\cos^2x+b\cos x+c=0$ & $t=\cos x$ & $-1\le t \le 1$\\
			\hline
			$a\tan^2x+b\tan x+c=0$ & $t=\tan x$ & $x\ne\dfrac{\pi}{2}+k\pi$\\
			\hline
			$a\cot^2x+b\cot x+c=0$ & $t=\cot x$ & $x\ne k\pi$\\
			\hline
		\end{tabular}
	\end{center}
	Nếu đặt $t=\sin^2x,\cos^2x$ hoặc $t=|\sin x|,|\cos x|$ thì điều kiện là $0\le t\le 1$.
	\begin{nx}
		Khi gặp phương trình bậc $ 3 $; $ 4 $\ldots ta có thể làm tương tự.
	\end{nx}
\end{dang}
\subsubsection{Ví dụ}
\begin{vd}% [1K1B4-7]
	Giải các phương trình sau
	\begin{listEX}[2]
		\item $2\cos^2x - 3\cos x + 1 = 0$.
		\item $\sin^2x + 3\sin x + 2 = 0$.
		\item $\tan^2x + (\sqrt{3}  - 1)\tan x - \sqrt{3}  = 0$. 
		% \item $\cot^2x + 4\cot x + 3 = 0$.
	\end{listEX}
	\loigiai{
		\begin{listEX}[2]
			\item!  $2\cos ^2x - 3\cos x + 1 = 0$ (*)\\
			Đặt $t = \cos x$, $- 1 \le t \le 1$. \\
			(*) trở thành $ 2t^2 - 3t + 1 = 0  \Leftrightarrow \left[ \begin{array}{ll}
				t = 1& (N)\\
				t = \dfrac{1}{2}&  (N).
			\end{array} \right.$ \\
			• Với $t = 1\Leftrightarrow \cos x=1\Leftrightarrow x=k2\pi$, $ (k\in\mathbb{Z}) $. \\
			• Với $t = \dfrac{1}{2}\Leftrightarrow \cos x=\dfrac{1}{2}\Leftrightarrow \hoac{&x=\dfrac{\pi}{3}+k2\pi \\ &x=-\dfrac{\pi}{3}+k2\pi}$, $ (k\in \mathbb{Z}) $.\\  
			Vậy nghiệm của phương trình $x = k2\pi $; $x = \dfrac \pi 3 + k2\pi $; $x =  - \dfrac \pi 3 + k2\pi $, $ (k\in \mathbb{Z}) $.
			\item! $\sin^2x + 3\sin x + 2 = 0$ (*)\\
			Đặt $t = \sin x$, $- 1 \le t \le 1$.\\
			(*) trở thành $ t^2 + 3t + 2 = 0$ 
			$ \Leftrightarrow 
			\hoac{
				& t =  - 1 & (N) \\ 
				&t =  - 2 & (L).}$ \\
			• Với $t=-1\Leftrightarrow 
			\sin x=-1\Leftrightarrow x=-\dfrac{\pi}{2}+k2\pi$, 
			$( k\in \mathbb{Z})$.\\
			Vậy nghiệm phương trình $x=-\dfrac{\pi}{2}+k2\pi$, 
			$( k\in \mathbb{Z})$.
			\item! $\tan^2x + (\sqrt{3}  - 1)\tan x - \sqrt{3}  = 0$ (*)\\
			Đặt $t = \tan x$.\\
			(*) trở thành $ t^2 + (\sqrt{3}  - 1)t - \sqrt{3}  = 0 \Leftrightarrow \hoac{
				& t = 1\\
				& t =  - \sqrt{3}.
			}$ \\
			• Với $t=1\Leftrightarrow \tan x=1\Leftrightarrow x=\dfrac{\pi}{4}+k\pi$, $( k\in \mathbb{Z})$.\\
			• Với $t=-\sqrt{3}\Leftrightarrow 
			\tan x=-\sqrt{3}\Leftrightarrow x=-\dfrac{\pi}{3}+k\pi$, 
			$( k\in \mathbb{Z} )$. \\
			Vậy nghiệm phương trình  $x=\dfrac{\pi}{4}+k\pi $; $x=-\dfrac{\pi}{3}+k\pi,( k\in \mathbb{Z}).$ 
			% \item! $\cot^2x + 4\cot x + 3 = 0$ (*)\\
			% Đặt $t = \cot x$. \\
			% (*) trở thành $ t^2 + 4t + 3 = 0 \Leftrightarrow 
			% \left[\begin{array}{l}
			% 	t =  - 1\\
			% 	t =  - 3.
			% \end{array} \right.$ \\
			% • Với $t=-1\Leftrightarrow \cot x=-1\Leftrightarrow x=-\dfrac{\pi}{4}+k\pi$, $( k\in \mathbb{Z})$. \\
			% • Với $t=-3\Leftrightarrow \cot x=-3\Leftrightarrow x=\text{arccot}(-3)+k\pi$, $( k\in \mathbb{Z})$.\\
			% Vậy nghiệm phương trình  $x=-\dfrac{\pi}{4}+k\pi $; $x=\text{arccot}(-3)+k\pi$, $( k\in \mathbb{Z})$. 
		\end{listEX}
	}
\end{vd}
\subsubsection{Bài tập tự luận}
\begin{bt}% [1K1B4-7]
	Giải các phương trình lượng giác sau
	\begin{listEX}[2]
		\item $6\cos^2{x}+5\sin{x}-2=0$.
		%		 \dapso{$\hoac{&x=\dfrac{-\pi}{6}+k2\pi\\&x=\dfrac{7\pi}{6}+k2\pi}(k\in\mathbb{Z})$}
		\item $2\cos^2{x}+5\sin{x}-4=0$.
		%		 \dapso{$\hoac{&x=\dfrac{\pi}{6}+k2\pi\\&x=\dfrac{\pi}{6}+k2\pi}(k\in\mathbb{Z})$}
		\item $3-4\cos^2{x}=\sin{x}(2\sin{x}+1)$.
		%		 \dapso{$\hoac{&x=\dfrac{\pi}{2}+k2\pi\\&x=\dfrac{-5\pi}{6}+k2\pi\\&x=\dfrac{-\pi}{6}+k2\pi}(k\in\mathbb{Z})$}
		\item $-\sin^2{x}-3\cos{x}+3=0$.
		%		 \dapso{$x=k2\pi\,(k\in\mathbb{Z})$}
	\end{listEX}
	\loigiai{
		\begin{listEX}[2]
			\item! Ta có
			$$
			6\cos^2{x}+5\sin{x}-2=0 \Leftrightarrow -6\sin^2{x}+5\sin{x}+4=0.
			$$
			Đặt $t=\sin{x}\,(-1\le t\le 1)$. Khi đó, phương trình trở thành
			$$
			-6t^2+5t+4=0 \Leftrightarrow \hoac{&t=\dfrac{4}{3}\\&t=\dfrac{-1}{2}.}
			$$
			Vì $-1\le t\le 1$ nên $t=\sin{x}=\dfrac{-1}{2}\Leftrightarrow\hoac{&x=\dfrac{-\pi}{6}+k2\pi\\&x=\dfrac{7\pi}{6}+k2\pi}(k\in\mathbb{Z})$.
			\item! Ta có
			\begin{eqnarray*}
				& 2\cos^2{x}+5\sin{x}-4=0 & \Leftrightarrow 2-2\sin^2{x}+5\sin{x}-4=0\\
				& & \Leftrightarrow 2\sin^2{x}-5\sin{x}+2=0.
			\end{eqnarray*}
			Đặt $t=\sin{x}(-1\le t\le 1)$. Khi đó, phương trình trở thành
			$$
			2t^2-5t+2=0\Leftrightarrow \hoac{&t=2\\&t=\dfrac{1}{2}.}
			$$
			Vì $-1\le t\le 1$ nên $t=\sin{x}=\dfrac{1}{2}\Leftrightarrow\hoac{&x=\dfrac{\pi}{6}+k2\pi\\&x=\dfrac{5\pi}{6}+k2\pi}(k\in\mathbb{Z})$.
			\item! Ta có
			\begin{eqnarray*}
				& 3-4\cos^2{x}=\sin{x}(2\sin{x}+1) & \Leftrightarrow 3-4(1-\sin^2{x})-2\sin^2{x}-\sin{x}=0\\
				& & \Leftrightarrow 2\sin^2{x}-\sin{x}-1=0.
			\end{eqnarray*}
			Đặt $t=\sin{x}(-1\le t\le 1)$. Khi đó, phương trình trở thành:
			$$
			2t^2-t-1=0 \Leftrightarrow \hoac{&t=1\\&t=\dfrac{-1}{2}.}
			$$
			Vì $-1\le t\le 1$ nên $\hoac{&t=\sin{x}=1\\&t=\sin{x}=\dfrac{-1}{2}}\Leftrightarrow\hoac{&x=\dfrac{\pi}{2}+k2\pi\\&x=\dfrac{-5\pi}{6}+k2\pi\\&x=\dfrac{-\pi}{6}+k2\pi}(k\in\mathbb{Z})$.
			\item! Ta có
			$$
			-\sin^2{x}-3\cos{x}+3=0 \Leftrightarrow \cos^2{x}-3\cos{x}+2=0.$$
			Đặt $t=\cos{x}\,(-1\le t\le 1)$. Khi đó, phương trình trở thành:
			$$
			t^2-3t+2=0 \Leftrightarrow \hoac{&t=2\\&t=1.}
			$$
			Vì $-1\le t\le 1$ nên $t=\cos{x}=1\Leftrightarrow x=k2\pi(k\in\mathbb{Z})$.
		\end{listEX}
	}
\end{bt}

\begin{bt}%[1K1B4-7]
	Giải các phương trình lượng giác sau:
	\begin{listEX}[2]
		\item $2\cos{2x}-8\cos{x}+5=0$.
		%		 \dapso{$\hoac{&x=\dfrac{-\pi}{3}+k2\pi\\&x=\dfrac{\pi}{3}+k2\pi}(k\in\mathbb{Z})$}
		\item $1+\cos{2x}=2\cos{x}$.
		\item $9\sin{x}+\cos{2x}=8$.
		%		 \dapso{$x=\dfrac{\pi}{2}+k2\pi\,(k\in\mathbb{Z})$}
		\item $2+\cos{2x}+5\sin{x}=0$.
		%		 \dapso{$\hoac{&x=\dfrac{-\pi}{6}+k2\pi\\&x=\dfrac{-5\pi}{6}+k2\pi}(k\in\mathbb{Z})$}
		% \item $3\sin{x}+2\cos{2x}=2$.
		%		 \dapso{$\hoac{&x=k2\pi\\&x=\arcsin{\dfrac{3}{4}}+k2\pi\\&x=-\arcsin{\dfrac{3}{4}}+\pi+k2\pi}(k\in\mathbb{Z})$}
		\item $2\cos{2x}+8\sin{x}-5=0$.
		%		 \dapso{$\hoac{&x=\dfrac{\pi}{6}+k2\pi\\&x=\dfrac{5\pi}{6}+k2\pi}(k\in\mathbb{Z})$}
	\end{listEX}
	\loigiai{
		\begin{listEX}[2]
			\item! Ta có
			$$
			2\cos{2x}-8\cos{x}+5=0 \Leftrightarrow 4\cos^2{x}-8\cos{x}+3=0.
			$$
			Đặt $t=\cos{x}\,(-1\le t\le 1)$. Khi đó, phương trình trở thành:
			$$
			4t^2-8t+3=0 \Leftrightarrow \hoac{&t=\dfrac{3}{2}\\&t=\dfrac{1}{2}.}
			$$
			Vì $-1\le t\le 1$ nên $t=\cos{x}=\dfrac{1}{2}\Leftrightarrow\hoac{&x=\dfrac{-\pi}{3}+k2\pi\\&x=\dfrac{\pi}{3}+k2\pi}(k\in\mathbb{Z})$.
			\item! Ta có
			$$
			1+\cos{2x}=2\cos{x} \Leftrightarrow 2\cos^2{x}-2\cos{x}=0.
			$$
			Đặt $t=\cos{x}\,(-1\le t\le 1)$. Khi đó, phương trình trở thành:
			$$
			2t^2-2t=0 \Leftrightarrow \hoac{&t=0\\&t=1.}
			$$
			Vì $-1\le t\le 1$ nên $\hoac{&t=\cos{x}=0\\&t=\cos{x}=1}\Leftrightarrow\hoac{&x=k2\pi\\&x=\dfrac{-\pi}{2}+k\pi}(k\in\mathbb{Z})$.
			\item!Ta có
			$$
			9\sin{x}+\cos{2x}=8
			\Leftrightarrow -2\sin^2{x}+9\sin{x}-7=0.
			$$
			Đặt $t=\sin{x}(-1\le t\le 1)$. Khi đó, phương trình trở thành:
			$$
			2t^2-9t+7=0 \Leftrightarrow \hoac{&t=1\\&t=\dfrac{7}{2}.}
			$$
			Vì $-1\le t\le 1$ nên $t=\sin{x}=1\Leftrightarrow x=\dfrac{\pi}{2}+k2\pi\,(k\in\mathbb{Z})$.
			\item! Ta có
			$$
			2+\cos{2x}+5\sin{x}=0 \Leftrightarrow -2\sin^2{x}+5\sin{x}+3=0.
			$$
			Đặt $t=\sin{x}(-1\le t\le 1)$. Khi đó, phương trình trở thành:
			$$
			2t^2-5t-3=0 \Leftrightarrow \hoac{&t=3\\&t=\dfrac{-1}{2}.}
			$$
			Vì $-1\le t\le 1$ nên $t=\sin{x}=\dfrac{-1}{2}\Leftrightarrow\hoac{&x=\dfrac{-\pi}{6}+k2\pi\\&x=\dfrac{-5\pi}{6}+k2\pi}(k\in\mathbb{Z})$.
			% \item! Ta có
			% $$
			% 3\sin{x}+2\cos{2x}=2 \Leftrightarrow -4\sin^2{x}+3\sin{x}=0.
			% $$
			% Đặt $t=\sin{x}\,(-1\le t\le 1)$. Khi đó, phương trình trở thành:
			% $$
			% -4t^2+3t=0 \Leftrightarrow \hoac{&t=0\\&t=\dfrac{3}{4}.}
			% $$
			% Vì $-1\le t\le 1$ nên $\hoac{&t=\sin{x}=0\\&t=\sin{x}=\dfrac{3}{4}}\Leftrightarrow \hoac{&x=k\pi\\&x=\arcsin{\dfrac{3}{4}}+k2\pi\\&x=-\arcsin{\dfrac{3}{4}}+\pi+k2\pi}(k\in\mathbb{Z})$.
			\item! Ta có
			$$
			2\cos{2x}+8\sin{x}-5=0 \Leftrightarrow -4\sin^2{x}+8\sin{x}-3=0.
			$$
			Đặt $t=\sin{x}\,(-1\le t\le 1)$. Khi đó, phương trình trở thành:
			$$
			4t^2-8t+3=0 \Leftrightarrow \hoac{&t=\dfrac{3}{2}\\&t=\dfrac{1}{2}.}
			$$
			Vì $-1\le t\le 1$ nên $t=\sin{x}=\dfrac{1}{2}\Leftrightarrow\hoac{&x=\dfrac{\pi}{6}+k2\pi\\&x=\dfrac{5\pi}{6}+k2\pi}(k\in\mathbb{Z})$.
		\end{listEX}
	}
\end{bt}



\begin{bt}%[1K1B4-7]
	Giải các phương trình lượng giác sau:
	\begin{listEX}[2]
		\item $2\cos^2{2x}+5\sin{2x}+1=0$.
		%		 \dapso{$\hoac{&x=\dfrac{-5\pi}{12}+k\pi\\&x=\dfrac{-\pi}{12}+k\pi}(k\in\mathbb{Z})$}
		\item $5\cos{x}-2\sin\dfrac{x}{2}+7=0$.
		%		 \dapso{$x=\pi+4k\pi\,(k\in\mathbb{Z})$}
		\item $\sin^2{x}+\cos{2x}+\cos{x}=2$.
		%		 \dapso{$x=k2\pi\,(k\in\mathbb{Z})$}
		\item $\cos{2x}+\cos^2{x}-\sin{x}+2=0$.
		%		 \dapso{$x=\dfrac{\pi}{2}+k2\pi\,(k\in\mathbb{Z})$}
	\end{listEX}
	\loigiai{
		\begin{listEX}[2]
			\item! Ta có
			$$
			2\cos^2{2x}+5\sin{2x}+1=0 \Leftrightarrow -2\sin^2{2x}+5\sin{2x}+3=0.
			$$
			Đặt $t=\sin{2x}\,(-1\le t\le 1)$. Khi đó, phương trình trở thành:
			$$
			2t^2-5t-3=0 \Leftrightarrow \hoac{&t=3\\&t=\dfrac{-1}{2}.}
			$$
			Vì $-1\le t\le 1$ nên $t=\sin{2x}=\dfrac{-1}{2}\Leftrightarrow \hoac{&x=\dfrac{-5\pi}{12}+k\pi\\&x=\dfrac{-\pi}{12}+k\pi}(k\in\mathbb{Z})$.
			\item! Đặt $y=\dfrac{x}{2}$. Khi đó, phương trình trở thành:
			$$
			5\cos{2y}-2\sin{y}+7=0 \Leftrightarrow -10\sin^2{y}-2\sin{y}+12=0.
			$$
			Đặt $t=\sin{y}\,(-1\le t\le 1)$. Khi đó, phương trình trở thành:
			$$
			10t^2+2t-12=0 \Leftrightarrow \hoac{&t=1\\&t=\dfrac{-6}{5}.}
			$$
			Vì $-1\le t\le 1,y=\dfrac{x}{2}$ nên $t=\sin\dfrac{x}{2}=1\Leftrightarrow x=\pi+4k\pi\,(k\in\mathbb{Z})$.
			\item! Ta có
			\begin{eqnarray*}
				& \sin^2{x}+\cos{2x}+\cos{x}=2 & \Leftrightarrow 1-\cos^2{x}+2\cos^2{x}-1+\cos{x}-2=0\\
				& & \Leftrightarrow \cos^2{x}+\cos{x}-2=0.
			\end{eqnarray*}
			Đặt $t=\cos{x}\,(-1\le t\le 1)$. Khi đó, phương trình trở thành:
			$$
			t^2+t-2=0 \Leftrightarrow \hoac{&t=-2\\&t=1.}
			$$
			Vì $-1\le t\le 1$ nên $t=\cos{x}=1\Leftrightarrow x=k2\pi\,(k\in\mathbb{Z})$.
			\item! Ta có
			\begin{eqnarray*}
				& \cos{2x}+\cos^2{x}-\sin{x}+2=0 & \Leftrightarrow 1-2\sin^2{x}+1-\sin^2{x}-\sin{x}+2=0\\
				& & \Leftrightarrow 3\sin^2{x}+\sin{x}-4=0.
			\end{eqnarray*}
			Đặt $t=\sin{x}\,(-1\le t\le 1)$. Khi đó, phương trình trở thành:
			$$
			3t^2+t-4=0 \Leftrightarrow \hoac{&t=\dfrac{-4}{3}\\&t=1.}
			$$
			Vì $-1\le t\le 1$ nên $t=\sin{x}=1\Leftrightarrow x=\dfrac{\pi}{2}+k2\pi\,(k\in\mathbb{Z})$.
		\end{listEX}
	}
\end{bt}

\begin{bt}%[1K1K4-7]
	Giải các phương trình lượng giác sau
	\begin{listEX}[2]
		\item $3\sin^2{x}+2\cos^4{x}-2=0$.
		%		 \dapso{$\hoac{&x=k2\pi\\&x=\dfrac{-\pi}{4}+k\pi\\&x=\dfrac{\pi}{4}+k\pi}(k\in\mathbb{Z})$}
		\item $4\sin^4{x}+2\cos^2{x}=7$.
		%		 \dapso{$\hoac{&x=\dfrac{\pi}{4}+k\pi\\&x=\dfrac{-\pi}{4}+k\pi}(k\in\mathbb{Z})$}
		\item $4\cos^4{x}=4\sin^2{x}-1$
		%		 \dapso{$\hoac{&x=\dfrac{-3\pi}{4}+k\pi\\&x=\dfrac{3\pi}{4}+k\pi}(k\in\mathbb{Z})$}
		\item $4\sin^4{x}+5\cos^2{x}-4=0$
		%		 \dapso{$\hoac{&x=\dfrac{-\pi}{6}+k2\pi\\&x=\dfrac{\pi}{6}+k2\pi\\&x=k2\pi}(k\in\mathbb{Z})$}
	\end{listEX}
	\loigiai{
		\begin{listEX}[2]
			\item! Ta có
			$$
			3\sin^2{x}+2\cos^4{x}-2=0 \Leftrightarrow 2\cos^4{x}-3\cos^2{x}+1=0.
			$$
			Đặt $t=\cos^2{x}\,(0\le t\le 1)$. Khi đó, phương trình trở thành:
			$$
			2t^2-3t+1=0 \Leftrightarrow \hoac{&t=1\\&t=\dfrac{1}{2}.}
			$$
			Vì $0\le t\le 1$ nên $\hoac{&t=\cos^2{x}=1\\&t=\cos^2{x}=\dfrac{1}{2}}\Leftrightarrow\hoac{&\cos{x}=1\\&\cos{x}=\pm\dfrac{\sqrt{2}}{2}}\Leftrightarrow\hoac{&x=k2\pi\\&x=\dfrac{-\pi}{4}+k\pi\\&x=\dfrac{\pi}{4}+k\pi}(k\in\mathbb{Z})$.
			\item! Ta có
			$$
			4\sin^4{x}+12\cos^2{x}=7 \Leftrightarrow 4\sin^4{x}-12\sin^2{x}+5=0.
			$$
			Đặt $t=\sin^2{x}(0\le t\le 1)$. Khi đó, phương trình trở thành:
			$$
			4t^2-12t+5=0 \Leftrightarrow \hoac{&t=\dfrac{1}{2}\\&t=\dfrac{5}{2}.}
			$$
			Vì $0\le t\le 1$ nên $t=\sin^2{x}=\dfrac{1}{2}\Leftrightarrow\sin{x}=\pm\dfrac{\sqrt{2}}{2}\Leftrightarrow\hoac{&x=\dfrac{\pi}{4}+k\pi\\&x=\dfrac{-\pi}{4}+k\pi}(k\in\mathbb{Z})$.
			\item! Ta có
			$$
			4\cos^4{x}=4\sin^2{x}-1 \Leftrightarrow 4\cos^4{x}+4\cos^2{x}-3=0.
			$$
			Đặt $t=\cos^2{x}\,(0\le t\le 1)$. Khi đó, phương trình trở thành:
			$$
			4t^2+4t-3=0 \Leftrightarrow \hoac{&t=\dfrac{1}{2}\\&t=\dfrac{-3}{2}.}
			$$
			Vì $0\le t\le 1$ nên $t=\cos^2{x}=\dfrac{1}{2}\Leftrightarrow\cos{x}=\pm\dfrac{\sqrt{2}}{2}\Leftrightarrow\hoac{&x=\dfrac{-3\pi}{4}+k\pi\\&x=\dfrac{3\pi}{4}+k\pi}(k\in\mathbb{Z})$.
			\item! Ta có
			$$
			4\sin^4{x}+5\cos^2{x}-4=0 \Leftrightarrow 4\sin^4{x}-5\sin^2{x}+1=0.
			$$
			Đặt $t=\sin^2{x}\,(0\le t\le 1)$. Khi đó, phương trình trở thành:
			$$
			4t^2-5t+1=0 \Leftrightarrow \hoac{&t=\dfrac{1}{4}\\&t=1.}
			$$
			Vì $0\le t\le 1$ nên $\hoac{&t=\sin^2{x}=\dfrac{1}{4}\\&t=\sin^2{x}=1}\Leftrightarrow\hoac{&t=\sin{x}=\dfrac{1}{2}\\&t=\sin{x}=1}\Leftrightarrow\hoac{&x=\dfrac{5\pi}{6}+k2\pi\\&x=\dfrac{\pi}{6}+k2\pi\\&x=\dfrac{\pi}{2}+k2\pi}(k\in\mathbb{Z})$.
		\end{listEX}
	}
\end{bt}



\begin{bt}%[1K1K4-7]
	Giải các phương trình sau
	\begin{listEX}[2]
		\item $\cos^3x + 3\cos^2x + 2\cos x = 0$. 
		\item $23\sin x - \sin 3x = 24$. 
		\item $2\cos 3x\cdot\cos x - 4\sin^22x + 1 = 0$. 
		\item $\sin^6x + \cos^6x = \dfrac{15}{8} \cos 2x - \dfrac{1}{2}$. 
	\end{listEX}
	\loigiai{
		\begin{listEX}[2]
			\item! $\cos^3x + 3\cos^2x + 2\cos x = 0$ (*)
			
			Đặt $t = \cos x$, $- 1 \le t \le 1$. \\
			(*) trở thành $ t^3 + 3t^2 + 2t = 0 \Leftrightarrow 
			\hoac{
				&	t = 0& (N)\\
				&	t =  - 1& (N)\\
				&	t =  - 2& (L).
			}$
			
			• Với $t=0\Leftrightarrow \cos x=0\Leftrightarrow x=\dfrac{\pi}{2}+k\pi$, 
			$( k\in \mathbb{Z})$.
			
			• Với $t=-1\Leftrightarrow \cos x=-1\Leftrightarrow x=\pi +k2\pi$, $( k\in \mathbb{Z})$.
			
			Vậy nghiệm của phương trình 
			$x=\dfrac{\pi}{2}+k\pi $; 
			$x=\pi +k2\pi$, $( k\in \mathbb{Z})$.
			\item! $23\sin x - \sin 3x = 24$
			
			$ \Leftrightarrow 23\sin x - (3\sin x - 4\sin^3x) = 24 \Leftrightarrow 4\sin^3x + 20\sin x - 24 = 0$ (*)
			
			Đặt $t = \cos x$,$ - 1 \le t \le 1$.
			
			(*) trở thành $  4t^3 + 20t - 24 = 0 \Leftrightarrow t = 1$ (N) 
			
			• Với $t=1\Leftrightarrow \sin x=1\Leftrightarrow x=\dfrac{\pi}{2}+k2\pi$, $ ( k\in \mathbb{Z})$.
			
			Vậy nghiệm của phương trình $x=\dfrac{\pi}{2}+k2\pi,( k\in \mathbb{Z})$.
			\item! $2\cos 3x\cdot \cos x - 4\sin^22x + 1 = 0$
			
			$ \Leftrightarrow \cos 4x + \cos 2x - 2(1 - \cos 2x) + 1 = 0 \Leftrightarrow 2\cos^22x + 3\cos 2x - 2 = 0$ (*)
			
			Đặt $t = \cos 2x$, $- 1 \le t \le 1$.
			
			(*) trở thành $ 2t^2 + 3t - 2 = 0 \Leftrightarrow 
			\left[\begin{array}{lr}
				t = \dfrac{1}{2}& (N)\\
				t =  - 2& (L).
			\end{array} \right.$
			
			• Với $t=\dfrac{1}{2}\Leftrightarrow 
			\cos 2x=\dfrac{1}{2}\Leftrightarrow 
			\cos 2x=\cos \dfrac{\pi}{3}\Leftrightarrow 
			x=\pm \dfrac{\pi}{6}+k\pi $, $( k\in \mathbb{Z})$.
			
			Vậy nghiệm của phương trình $x=\pm \dfrac{\pi}{6}+k\pi$, $( k\in \mathbb{Z})$.
			\item! 
			\begin{eqnarray*}
				&\sin^6x + \cos^6x = \dfrac{15}{8} \cos 2x - \dfrac{1}{2}
				&\Leftrightarrow 1-\dfrac{3}{4}\sin^22x= \dfrac{15}{8} \cos 2x - \dfrac{1}{2}\\
				&&\Leftrightarrow 6\cos^2 2x-15\cos 2x+6=0\\
				&&\Leftrightarrow 
				\hoac{ & \cos 2x= 2 & (L) \\ & \cos 2x= \dfrac{1}{2} & (N)}\\
				&&\Leftrightarrow x=\pm \dfrac{\pi}{6}+k\pi, (k\in \mathbb{Z})
			\end{eqnarray*}
			Vậy nghiệm của phương trình  
			$  x=\pm \dfrac{\pi}{6}+k\pi, (k\in \mathbb{Z})$. 
		\end{listEX}
	}
\end{bt}

\subsubsection{Bài tập trắc nghiệm}
\Opensolutionfile{ans}[ans/ans-1K1-4-Dang7]

\begin{ex}%[1K1B4-7]
	Nghiệm của phương trình $\sin^2x-4\sin x+3=0$ là
	\choice
	{$x=-\dfrac{\pi}{2}+k2\pi, k\in\mathbb{Z}$}
	{$x=\pi+k2\pi, k\in\mathbb{Z}$}
	{\True $x=\dfrac{\pi}{2}+k2\pi, k\in\mathbb{Z}$}
	{$x=k2\pi, k\in\mathbb{Z}$}
	\loigiai{
		Ta có  $\sin^2x-4\sin x+3=0\Leftrightarrow \hoac{&\sin x=1 \\ &\sin x=3.}$
		\begin{itemize}
			\item Với $\sin x=1\Leftrightarrow x=\dfrac{\pi}{2}+k2\pi, k\in\mathbb{Z}$.
			\item Với $\sin x=3$ phương trình vô nghiệm.
		\end{itemize}
	}
\end{ex}


% \begin{ex}%[1K1B4-7]
% 	Phương trình $\cos^2 {2x}+\cos {2x}-\dfrac{3}{4}=0$ có họ nghiệm là
% 	\choice
% 	{$x=\pm \dfrac{2\pi}{3}+k\pi$, $k\in \mathbb{Z}$}
% 	{$x=\pm \dfrac{\pi}{3}+k\pi$, $k\in \mathbb{Z}$}
% 	{\True $x=\pm \dfrac{\pi}{6}+k\pi$, $k\in \mathbb{Z}$}
% 	{$x=\pm \dfrac{\pi}{6}+k2\pi$, $k\in \mathbb{Z}$}
% 	\loigiai{
% 		$\cos^2 {2x}+\cos {2x}-\dfrac{3}{4}=0 \Leftrightarrow \hoac{\cos {2x}&=\dfrac{1}{2}\\
% 			\cos {2x}&=-\dfrac{3}{4}.}$ \\
% 		Với $\cos {2x}=\dfrac{1}{2}\Leftrightarrow 2x=\pm \dfrac{\pi}{3}+k2\pi \Leftrightarrow x=\pm \dfrac{\pi}{6}+k\pi$, $k\in \mathbb{Z}$.\\
% 		Với $\cos2x=-\dfrac{3}{4}\Leftrightarrow x=\pm\dfrac{1}{2}\arccos\left(-\dfrac{3}{4}\right)+k\pi,\, k\in\mathbb{Z}.$	
% 	}	
% \end{ex}
\begin{ex}%[1K1B4-7]
	Nghiệm âm lớn nhất của phương trình $2\tan^2x+5\tan x+3=0$ là
	\choice
	{\True $-\dfrac{\pi}{4}$}
	{$-\dfrac{\pi}{3}$}
	{$-\dfrac{\pi}{6}$}
	{$-\dfrac{5\pi}{6}$}
	\loigiai{
		Ta có \[2\tan^2x+5\tan x+3=0\Leftrightarrow \hoac{&\tan x =-1\\&\tan x =-\dfrac{3}{2}}.\] Suy ra nghiệm âm lớn nhất của phương trình là $-\dfrac{\pi}{4}$. 
	}
\end{ex}




% \begin{ex}%[1K1B4-7]
% 	Phương trình $3\tan^2x+\left(6-\sqrt{3}\right)\tan x-2\sqrt{3}=0$ có nghiệm là
% 	\choice
% 	{$\hoac{&x=\dfrac{\pi}{6}+k2\pi \\& x=\arctan (-2)+k2\pi }\left(k \in \mathbb{Z}\right)$}
% 	{$\hoac{&x=\dfrac{\pi}{3}+k\pi \\& x=\arctan (-2)+k\pi }\left(k \in \mathbb{Z}\right)$}
% 	{\True $\hoac{&x=\dfrac{\pi}{6}+k\pi \\& x=\arctan (-2)+k\pi }\left(k \in \mathbb{Z}\right)$}
% 	{$\hoac{&x=\dfrac{\pi}{6}+k\pi \\& x=-\arctan 2+k\pi }\left(k \in \mathbb{Z}\right)$}
% 	\loigiai{
% 		Ta có\\
% 		$3\tan^2x+\left(6-\sqrt{3}\right)\tan x-2\sqrt{3}=0 \Leftrightarrow \hoac{&\tan x=-2 \\& \tan x=\dfrac{\sqrt{3}}{3}}\Leftrightarrow \hoac{&x=\arctan (-2)+k\pi \\& x=\dfrac{\pi}{6}+k\pi }\left(k \in \mathbb{Z}\right)$.}
% \end{ex}

\begin{ex}%[1K1B4-7]
	Cho phương trình $ \cos^2 x+3\sin x-3=0 $. Đặt $ \sin x=t\ (-1\leq t\leq 1) $ ta được phương trình nào sau đây?
	\choice
	{$ t^2+3t+2=0 $}
	{\True $ t^2-3t+2=0 $}
	{$ t^2-3t-2=0 $}
	{$ t^2+3t-3=0 $}
	\loigiai{
		Ta có $ \cos^2 x+3\sin x-3=0\Leftrightarrow 1-\sin^2 x+3\sin x-3=0\Leftrightarrow -\sin^2 x+3\sin x -2=0 $.\\
		Do đó, đặt $ \sin x=t\ (-1\leq t\leq 1) $ thì ta được phương trình $ t^2-3t+2=0 $.
	}
\end{ex}


\begin{ex}%[1K1B4-7]
	Phương trình $\sin^2x-3\cos x-4=0$ có nghiệm là
	\choice
	{$x=-\pi +k2\pi$}
	{\True Vô nghiệm}
	{$x=-\dfrac{\pi}{2} +k2\pi$}
	{$x=\dfrac{\pi}{6} +k\pi$}
	\loigiai{
		Ta có \[\sin^2x-3\cos x-4=0 \Leftrightarrow \cos^2x+3\cos x +3=0\ (\textrm{vô nghiệm}).\]
	}
\end{ex}


\begin{ex}%[1K14-7]
	Giải phương trình $ \cos^2x+\sin x+1=0 $ có nghiệm là
	\choice
	{$ x=-\dfrac{\pi}{2}+k\dfrac{\pi}{2},k\in\mathbb{Z} $}
	{\True $ x=-\dfrac{\pi}{2}+k2\pi,k\in\mathbb{Z} $}
	{$ x=-\dfrac{\pi}{2}+k\pi,k\in\mathbb{Z} $}
	{$ x=\dfrac{\pi}{2}+k2\pi,k\in\mathbb{Z} $}
	\loigiai{
		Ta có $ \cos^2x+\sin x+1=0 \Leftrightarrow -\sin^2x+\sin x+2=0\Leftrightarrow\hoac{&\sin x = -1\\&\sin x=2}\Leftrightarrow x=-\dfrac{\pi}{2}+k2\pi,k\in\mathbb{Z}$.
	}
\end{ex}

\begin{ex}%[1D1B3-1]
	Nghiệm của  phương trình $2\sin^2 x-3\sin x+1=0$ là
	\choice
	{$x=\dfrac{\pi }{2}+k\pi$, $\hoac{& x=\dfrac{\pi }{6}+k\pi \\ & x=\dfrac{5\pi }{6}+k\pi}$ $\left(k\in \mathbb{Z}\right)$}
	{$x=\dfrac{\pi }{2}+k2\pi$, $\hoac{& x=\dfrac{\pi }{6}+k\dfrac{2}{3}\pi \\ & x=\dfrac{5\pi }{6}+k\dfrac{2}{3}\pi}$ $\left(k\in \mathbb{Z}\right)$}
	{$x=\dfrac{\pi }{2}+k\dfrac{5}{2}\pi $, $\hoac{& x=\dfrac{\pi }{6}+k\dfrac{1}{2}\pi\\ & x=\dfrac{5\pi }{6}+k\dfrac{1}{2}\pi}$ $\left(k\in \mathbb{Z}\right)$}
	{\True $x=\dfrac{\pi }{2}+k2\pi$, $\hoac{& x=\dfrac{\pi }{6}+k2\pi \\ & x=\dfrac{5\pi }{6}+k2\pi}$ $\left(k\in \mathbb{Z}\right)$}
	\loigiai{
		Đặt $t=\sin x$, $t \in [-1;1]$, ta có phương trình $2 t^2-3t+1=0 \Rightarrow t=1$, $t=\dfrac{1}{2}$. \\ 
		$\bullet$ $t=1 \Rightarrow \sin x=1 \Leftrightarrow x=\dfrac{\pi }{2}+k2\pi $. \\ 
		$\bullet$ $t=\dfrac{1}{2} \Rightarrow \sin x=\dfrac{1}{2}=\sin \dfrac{\pi }{6} \Leftrightarrow \hoac{& x=\dfrac{\pi }{6}+k2\pi \\ & x=\dfrac{5\pi }{6}+k2\pi}$ $\left(k\in \mathbb{Z}\right)$.
	} 
\end{ex}
\begin{ex}%[1K1B4-7]
	Cho phương trình $3\cos 2x-10\cos x-4=0$. Đặt $t=\cos x$ thì phương trình đã cho trở thành phương trình nào sau đây?
	\choice
	{$6t^2-10t-4=0$}
	{$3t^2-10t-4=0$}
	{$-6t^2-10t-1=0$}
	{\True $6t^2-10t-7=0$}
	\loigiai
	{
		Ta có
		\[3\cos 2x-10\cos x-4=0 \Leftrightarrow 3\left(2\cos^2 x-1\right)-10\cos x-4=0 \Leftrightarrow 6\cos^2 x-10\cos x-7=0.\]
		Đặt $t=\cos x$ phương trình trên trở thành $6t^2-10t-7=0$.
	}
\end{ex}
\begin{ex}%[1K1B4-7]
	Tập nghiệm của phương trình $\sin x+\cos 2x=0$ là
	\choice
	{$x=\dfrac{\pi}{2}+k2\pi, x=-\dfrac{\pi}{2}+\dfrac{k2\pi}{3}$}
	{$x=\dfrac{\pi}{2}+k\pi, x=-\dfrac{\pi}{6}+\dfrac{k\pi}{3}$}
	{\True $x=\dfrac{\pi}{2}+k2\pi, x=-\dfrac{\pi}{6}+\dfrac{k2\pi}{3}$}
	{$x=\dfrac{\pi}{2}+k\pi, x=-\dfrac{\pi}{2}+\dfrac{k\pi}{3}$}
	\loigiai{
		Ta có 
		\begin{eqnarray*}
			&&\sin x+cos2x=0\\
			&\Leftrightarrow& -2\sin^2x+\sin x+1=0\\
			&\Leftrightarrow& \hoac{&\sin x=1 \\& \sin x=-\dfrac{1}{2}}\\
			&\Leftrightarrow& \hoac{&x=\dfrac{\pi}{2}+k2\pi \\& x=-\dfrac{\pi}{6}+k2\pi \vee x=-\dfrac{5\pi}{6}+k2\pi }\\
			&\Leftrightarrow& x=-\dfrac{\pi}{6}+\dfrac{k2\pi}{3}.
		\end{eqnarray*}
	}
\end{ex}





\begin{ex}%[1K1B4-7]
	Nghiệm của phương trình lượng giác $2\sin^2x-3\sin x+1=0$ thỏa điều kiện 
	$0<x<\dfrac{\pi}{2}$ là
	\choice
	{$x=\dfrac{\pi}{2}$}
	{$x=\dfrac{\pi}{3}$}
	{\True $x=\dfrac{\pi}{6}$}
	{$\dfrac{5\pi}{6}$}
	\loigiai{
		$$2\sin^2x-3\sin x+1=0 \Leftrightarrow \hoac{& \sin x = 1 \\ & \sin x = \dfrac{1}{2}}
		\Leftrightarrow \hoac{& x = \dfrac{\pi}{2}+k2\pi \\ & x = \dfrac{\pi}{6}+k2\pi}, (k\in\mathbb{Z}).$$
		Vậy $\dfrac{\pi}{6}$ là nghiệm của phương trình.
	}
\end{ex}


\begin{ex}%[1K1K4-7]
	Tìm nghiệm phương trình $3\sin^22x-7\sin 2x+4=0$ trên đoạn $[0;\pi]$.
	\choice
	{$x=\dfrac{\pi}{3}$}
	{\True $x=\dfrac{\pi}{4}$}
	{$x=\dfrac{\pi}{2}$}
	{$x=\dfrac{\pi}{6}$}
	\loigiai{
		Ta có
		\begin{align*}
			&3\sin^22x-7\sin 2x+4=0\\
			&\Leftrightarrow \left(3\sin 2x-4\right)\left(\sin 2x-1\right)=0\\
			&\Leftrightarrow \sin 2x=1 \text{ hoặc }\sin 2x=\dfrac{4}{3} \text{ (vô nghiệm)}\\
			&\Leftrightarrow \sin 2x=1 \Leftrightarrow x=\dfrac{\pi}{4}+k\pi,k\in \mathbb{Z}.
		\end{align*}
		Mà $x\in [0;\pi]$ nên $x=\dfrac{\pi}{4}$.
	}
\end{ex}













\begin{ex}%[1K1K4-7]
	Tính tổng các nghiệm của phương trình $2\cos^2x+5\sin x-4=0$ trong $[0;2\pi]$.
	\choice
	{$0$}
	{$\dfrac{8\pi}{3}$}
	{\True $\pi$}
	{$\dfrac{5\pi}{6}$}
	\loigiai{
		\begin{align*}
			2\cos^2x+5\sin x-4=0 &\Leftrightarrow 2(1-\sin^2x)+5\sin x-4=0\\
			& \Leftrightarrow 2\sin^2x-5\sin x+2=0 \\
			& \Leftrightarrow \hoac{&\sin x=\dfrac{1}{2}\\&\sin x=2\quad \text{ (vô nghiệm)}}\\
			& \Leftrightarrow \hoac{&x=\dfrac{\pi}{6}+k2\pi\\&x=\dfrac{5\pi}{6}+k2\pi.}
		\end{align*}
		Vậy các nghiệm trong $[0;2\pi]$ của phương trình đã cho là $x=\dfrac{\pi}{6}$, $x=\dfrac{5\pi}{6}$. Nên tổng các nghiệm của phương trình đã cho trong $[0;2\pi]$ bằng $\pi$.
	}
\end{ex}





\begin{ex}%[1K1K4-7]
	Tổng các nghiệm của phương trình $\tan x+\cot x=2$ trong khoảng $\left(-\pi;\pi \right)$ là
	\choice
	{$-\pi $}
	{\True $-\dfrac{\pi}{2}$}
	{$\dfrac{5\pi}{4}$}
	{$\dfrac{\pi}{4}$}
	\loigiai{
		Điều kiện $\heva{& \sin x \ne 0\\& \cos x \ne 0} \Leftrightarrow \sin 2x \ne 0$.\\
		Với $x$ thỏa mãn điều kiện xác định thì
		{\allowdisplaybreaks
			\begin{eqnarray*}
				\tan x+\cot x=2 & \Leftrightarrow & \dfrac{\sin x}{\cos x} + \dfrac{\cos x}{\sin x} = 2 \Leftrightarrow \dfrac{\sin ^2 x + \cos^2 x}{\sin x \cdot \cos x} = 2 \\
				& \Leftrightarrow & \sin 2x=1 \Leftrightarrow 2x=\dfrac{\pi}{2}+k2\pi \Leftrightarrow x=\dfrac{\pi}{4}+k\pi \ (k \in \mathbb{Z}).
			\end{eqnarray*}
		}
		Do $x \in \left(-\pi;\pi \right)$ nên $x \in \left\{-\dfrac{3\pi}{4},\dfrac{\pi}{4}\right\}$. So với điều kiện, thỏa mãn.\\
		Vậy tổng các nghiệm $S=-\dfrac{3\pi}{4}+\dfrac{\pi}{4}=-\dfrac{\pi}{2}$.
	}
\end{ex}

\begin{ex}%[1K1K4-7]
	Số nghiệm của phương trình $\cos 2\left(x+\dfrac{\pi}{3}\right)+4\cos \left(\dfrac{\pi}{6}-x\right)=\dfrac{5}{2}$ thuộc $\left[0;2\pi \right]$ là
	\choice
	{$1$}
	{\True $2$}
	{$3$}
	{$4$}
	\loigiai{
		Đặt $t=x+\dfrac{\pi}{3}$.
		Phương trình trở thành $$\cos 2t+4\cos \left(\dfrac{\pi}{2}-t\right)=\dfrac{5}{2} \Leftrightarrow 1-2\sin^2t+4\sin t=\dfrac{5}{2} \Leftrightarrow -2\sin^2t+4\sin t-\dfrac{3}{2}=0 \Leftrightarrow \hoac{&\sin t=\dfrac{3}{2} \text{ (loại)} \\& \sin t=\dfrac{1}{2}.}$$
		Với $\sin t=\dfrac{1}{2}$, ta có $\sin \left(x+\dfrac{\pi}{3}\right)=\dfrac{1}{2} \Leftrightarrow \hoac{&x+\dfrac{\pi}{3}=\dfrac{\pi}{6}+k2\pi \\& x+\dfrac{\pi}{3}=\dfrac{5\pi}{6}+k2\pi }\Leftrightarrow \hoac{&x=\dfrac{-\pi}{6}+k2\pi \\& x=\dfrac{\pi}{2}+k2\pi. }$\\
		Vậy trong đoạn $\left[0;2\pi \right]$ phương trình có $2$ nghiệm $x=\dfrac{\pi}{2};x=\dfrac{11\pi}{6}$.}
\end{ex}





% \begin{ex}%[1K1K4-7]
% 	Nghiệm của phương trình $3\cos 4x-\sin^2 2x+\cos 2x-2=0$ là 
% 	\choice
% 	{$\hoac{& x=\dfrac{\pi }{2}+k\pi  \\ & x=\pm \arccos \dfrac{6}{7}+k\pi }$ $\left(k\in \mathbb{Z}\right)$}
% 	{$\hoac{& x=\dfrac{\pi }{2}+k2\pi \\ & x=\pm \arccos \dfrac{6}{7}+k2\pi }$ $\left(k\in \mathbb{Z}\right)$}
% 	{$\hoac{& x=\dfrac{\pi }{3}+k\pi \\ & x=\pm \arccos \dfrac{6}{7}+k2\pi }$ $\left(k\in \mathbb{Z}\right)$}
% 	{\True $\hoac{& x=\dfrac{\pi }{2}+k\pi  \\ & x=\pm \dfrac{1}{2} \arccos \dfrac{6}{7}+k\pi }$ $\left(k\in \mathbb{Z}\right)$}
% 	\loigiai{
% 		Phương trình đã cho tương đương với
% 		$3(2\cos^2 2x-1)-(1-\cos^2 2x)+\cos 2x-2=0$ \\ 
% 		$ \Leftrightarrow 7\cos^2 2x+\cos 2x-6=0 \Rightarrow \hoac{& \cos 2x=-1 \\ & \cos 2x=\dfrac{6}{7} } \Leftrightarrow \hoac{& x=\dfrac{\pi }{2}+k\pi  \\ & x=\pm \dfrac{1}{2}\arccos \dfrac{6}{7}+k\pi }$ $\left(k\in \mathbb{Z}\right)$. 
% 	} 
% \end{ex}

% \begin{ex}%[1K1K4-7]
	
% 	Giải phương trình $4 \cos x\cos 2x+1=0$.
% 	\choice
% 	{$\hoac{& x=\pm \dfrac{\pi }{3}+k2\pi  \\ & x=\pm \arccos \dfrac{-1\pm \sqrt{3}}{8}+k2\pi }$ $\left(k\in \mathbb{Z}\right)$}
% 	{\True $\hoac{
% 			& x=\pm \dfrac{\pi }{3}+k2\pi  \\ & x=\pm \arccos \dfrac{-1\pm \sqrt{5}}{8}+k2\pi }$ $\left(k\in \mathbb{Z}\right)$}
% 	{$\hoac{& x=\pm \dfrac{\pi }{3}+k2\pi \\ & x=\pm \arccos \dfrac{-1\pm \sqrt{7}}{8}+k2\pi }$ $\left(k\in \mathbb{Z}\right)$}
% 	{$\hoac{
% 			& x=\pm \dfrac{\pi }{3}+k2\pi \\ & x=\pm \arccos \dfrac{-1\pm \sqrt{6}}{8}+k2\pi }$ $\left(k\in \mathbb{Z}\right)$}
% 	\loigiai{
% 		Phương trình $\Leftrightarrow 4 \cos x(2\cos^2 x-1)+1=0$ \\ 
% 		$ \Leftrightarrow 8\cos^3 x-4\cos x+1=0 \Leftrightarrow (2\cos x-1)(4\cos^2 x+2\cos x-1)=0$ \\ 
% 		$ \Leftrightarrow \hoac{& \cos x=\dfrac{1}{2} \\ & 4\cos^2 x+2\cos x-1=0}  \Rightarrow \hoac{& \cos x=\dfrac{1}{2} \\ & \cos x=\dfrac{-1\pm \sqrt{5}}{8} } \Leftrightarrow \hoac{& x=\pm \dfrac{\pi }{3}+k2\pi \\ & x=\pm \arccos \dfrac{-1\pm \sqrt{5}}{8}+k2\pi }$ $\left(k\in \mathbb{Z}\right)$.
% 	} 
% \end{ex}

\begin{ex}%[1K1K4-7]
	Họ nghiệm của phương trình $16 (\sin^8 x+\cos^8 x)=17\cos^2 2x$ là
	\choice
	{$x=\dfrac{\pi }{8}+k\dfrac{5\pi }{4}$ $\left(k\in \mathbb{Z}\right)$}
	{$x=\dfrac{\pi }{8}+k\dfrac{7\pi }{4}$ $\left(k\in \mathbb{Z}\right)$}
	{$x=\dfrac{\pi }{8}+k\dfrac{9\pi }{4}$ $\left(k\in \mathbb{Z}\right)$}
	{\True $x=\dfrac{\pi }{8}+k\dfrac{\pi }{4}$ $\left(k\in \mathbb{Z}\right)$}
	\loigiai{
		Ta có $\sin^8 x+\cos^8 x=(\sin^4 x+\cos^4 x)^2 -2\sin^4 x \cos^4 x=\left(1-\dfrac{1}{2}\sin^2 2x \right)^2 -\dfrac{1}{8} \sin^4 2x$.\\
		Nên đặt $t=\sin^2 2x$, $0 \le t \le 1$, ta được phương trình \\ 
		$16 \left(1-\dfrac{1}{2}t \right)^2-2t^2=17(1-t) \Leftrightarrow 2t^2+t-1=0 \Leftrightarrow t=\dfrac{1}{2} \Leftrightarrow \sin^2 2x=\dfrac{1}{2} $\\ 
		$\Leftrightarrow 1-2\sin ^2 2x=0 \Leftrightarrow \cos 4x=0 \Leftrightarrow x=\dfrac{\pi }{8}+k\dfrac{\pi }{4}$ $\left(k\in \mathbb{Z}\right)$. 
	} 
\end{ex}

\begin{ex}%[1K1K4-7]
	Nghiệm của phương trình $\cos^4 x-\cos 2x+2\sin^6 x=0$. 
	\choice
	{$x=k2\pi$ $\left(k\in \mathbb{Z}\right)$}
	{$x=k\dfrac{1}{2}\pi$ $\left(k\in \mathbb{Z}\right)$}
	{$x=k\dfrac{2}{3}\pi$ $\left(k\in \mathbb{Z}\right)$}
	{\True $x=k\pi$ $\left(k\in \mathbb{Z}\right)$}
	\loigiai{
		Đặt $t=\cos 2x \Rightarrow -1\le t\le 1 \Rightarrow \cos^4 x=\dfrac{1}{4} (1+t)^2$, $\sin^6 x=\dfrac{1}{8} (1-t)^3$. \\ 
		Nên phương trình đã cho trở thành \\ 
		$\dfrac{1}{4}(1+t)^2-t+\dfrac{1}{4}(1-t)^3=0\Leftrightarrow t^3-4t^2+5t-2=0 \Leftrightarrow t=1, t=2$. \\ 
		$t=1 \Rightarrow \cos 2x=1 \Leftrightarrow x=k\pi$ $\left(k\in \mathbb{Z}\right)$.
	} 
\end{ex}





\begin{ex}%[1K1K4-7]
	Giải phương trình $5(1 + \cos x) = 2 + \sin^4x - \cos^4x$.
	\choice
	{$x =  \pm \dfrac{2\pi}{3} + k\pi $}
	{$x =  \pm \dfrac{2\pi}{3} + k\dfrac{2}{3}\pi $}
	{$x =  \pm \dfrac{2\pi}{3} + k\dfrac{3}{4}\pi $} 
	{\True $x =  \pm \dfrac{2\pi}{3} + k2\pi $}
	\loigiai{
		\begin{eqnarray*}
			&5(1 + \cos x) = 2 + \sin^4x - \cos^4x
			&\Leftrightarrow 3 + 5\cos x = (\sin^2x - \cos^2x)(\sin^2x + \cos^2x)\\
			&&\Leftrightarrow 2\cos^2x + 5\cos x + 2 = 0 \\
			&&\Leftrightarrow \cos x = -\dfrac{1}{2} \\
			&&\Leftrightarrow x =  \pm \dfrac{2\pi}{3} + k2\pi.
		\end{eqnarray*}
	} 
\end{ex}


\begin{ex}%[1K1K4-7]
	Nghiệm của  phương trình 
	$\sin\left(2x + \dfrac{5\pi}{2}\right) - 3\cos\left(x - \dfrac{7\pi}{2}\right) = 1 + 2\sin x$ là 
	\choice
	{$ \hoac{& x=k2\pi\\ & x=\dfrac{\pi}{6}+k2\pi\\ & x=\dfrac{5\pi}{6}+k\pi} (k\in \mathbb{Z}) $}
	{$ \hoac{& x=k\dfrac{1}{2}\pi\\ & x=\dfrac{\pi}{6}+k\pi\\ & x=\dfrac{5\pi}{6}+k2\pi} (k\in \mathbb{Z}) $}
	{\True $ \hoac{& x=k\pi\\ & x=\dfrac{\pi}{6}+k2\pi\\ & x=\dfrac{5\pi}{6}+k2\pi} (k\in \mathbb{Z}) $}
	{$ \hoac{& x=k2\pi\\ & x=\dfrac{\pi}{6}+k2\pi\\ & x=\dfrac{5\pi}{6}+k2\pi} (k\in \mathbb{Z}) $}
	\loigiai{
		\begin{eqnarray*}
			&PT
			&\Leftrightarrow \cos 2x + 3\sin x = 1 + 2\sin x\\ &&\Leftrightarrow 1 - 2\sin^2x + 3\sin x - 1 - 2\sin x = 0\\
			&&\Leftrightarrow -2\sin^2x+\sin x=0\\
			&&\Leftrightarrow \hoac{
				& \sin x=0 \\ 
				& \sin x=\dfrac{1}{2}
			}\\
			&&\Leftrightarrow \hoac{
				& x=k\pi  \\ 
				& x=\dfrac{\pi}{6}+k2\pi  \\ 
				& x=\dfrac{5\pi}{6}+k2\pi
			},\quad ( k\in \mathbb{Z}).
		\end{eqnarray*}
	}
\end{ex}
\begin{ex}%[1K1K4-7]
	Giải phương trình $7\cos x = 4\cos^3x + 4\sin 2x$.
	\choice
	{$\hoac{
			&x = \dfrac{\pi}{2} + k2\pi \\
			&x = \dfrac{\pi}{6} + k\pi\\
			&x = \dfrac{5\pi}{6} + k\pi 
		}$}
	{$\hoac{
			&x = \dfrac{\pi}{2} + k\dfrac{1}{4}\pi \\
			&x = \dfrac{\pi}{6} + k2\pi\\
			&x = \dfrac{5\pi}{6} + k2\pi 
		}$}
	{$\hoac{
			&x = \dfrac{\pi}{2} + k\dfrac{1}{2}\pi \\
			&x = \dfrac{\pi}{6} + k\pi\\
			&x = \dfrac{5\pi}{6} + k2\pi 
		}$} 
	{\True $\hoac{
			&x = \dfrac{\pi}{2} + k\pi \\
			&x = \dfrac{\pi}{6} + k2\pi\\
			&x = \dfrac{5\pi}{6} + k2\pi	 
		}$}
	\loigiai{
		Phương trình $ \Leftrightarrow \cos x(4\cos^2x + 8\sin x - 7) = 0$\\
		$ \Leftrightarrow 
		\cos x(4\sin^2x - 8\sin x + 3) = 0 \Leftrightarrow 
		\hoac{
			&x = \dfrac{\pi}{2} + k\pi \\
			&x = \dfrac{\pi}{6} + k2\pi\\
			&x = \dfrac{5\pi}{6} + k2\pi. 
		}$}
\end{ex}



\begin{ex}%[1K1K4-7]
	Giải phương trình $\cos 4x = \cos^23x$ .
	\choice
	{$ \hoac{
			&x=k2\pi\\ 
			& x=\pm \dfrac{\pi}{12}+k2\pi\\ 
			& x=\pm\dfrac{5\pi}{12}+k\pi} $}
	{$ \hoac{
			&x=k\pi\\ 
			& x=\pm \dfrac{\pi}{12}+k\dfrac{1}{2}\pi\\ 
			& x=\pm\dfrac{5\pi}{12}+k\dfrac{1}{2}\pi} $}
	{$ \hoac{
			&x=k\pi\\ 
			& x=\pm \dfrac{\pi}{12}+k3\pi\\ 
			& x=\pm\dfrac{5\pi}{12}+k3\pi} $}
	{\True $ \hoac{
			&x=k\pi\\ 
			& x=\pm \dfrac{\pi}{12}+k\pi\\ 
			& x=\pm\dfrac{5\pi}{12}+k\pi} $}
	\loigiai{
		Phương trình $ \Leftrightarrow 2\cos 4x = 1 + \cos 6x$ 
		$ \Leftrightarrow 2(2\cos^22x - 1) = 1 + 4\cos^32x - 3\cos 2x$ \\
		$ \Leftrightarrow 4\cos^32x - 4\cos^22x - 3\cos 2x + 3 = 0 \Leftrightarrow 
		\hoac{
			&\cos 2x = 1\\
			&\cos 2x =  \pm \dfrac{\sqrt{3}}{2}
		} \Leftrightarrow 
		\hoac{
			&x = k\pi \\
			&x =  \pm \dfrac{\pi}{12} + k\pi\\
			&x =  \pm \dfrac{5\pi}{12} + k\pi. 
		}$}
\end{ex}





\begin{ex}%[1K1K4-7]
	Cho phương trình: 	$\cos 2x - (2m + 1)\cos x + m + 1 = 0$.
	Tìm $m$ để phương trình có nghiệm $x \in \left( \dfrac{\pi }{2};\dfrac{3\pi}{2} \right).$
	\choice{\True $ - 1 \le m < 0$}
	{$ - 1 \le m \le 0$}
	{$ - 1 < m < 0$}
	{$ - 1 \le m \le 1$}
	\loigiai{Phương trình tương đương
		\[2\cos ^2 x - (2m + 1)\cos x + m = 0 \Leftrightarrow \left[ \begin{array}{l}
			\cos x = m \\ 
			\cos x = \dfrac{1}{2}.
		\end{array} \right.\tag{*}\]
		Với $x \in \left(\dfrac{\pi }{2};\dfrac{3\pi}{2}\right)$, ta có
		$ - 1 \le \cos x < 0$. Từ (*), ta loại trường hợp $\cos x = \dfrac{1}{2}$ và phương trình đã cho có nghiệm 
		$x \in \left(\dfrac{\pi }{2};\dfrac{3\pi}{2} \right)$
		khi và chỉ khi $ - 1 \le m < 0$.}		
\end{ex}

\begin{ex}%[1K1G4-7]
	Cho phương trình $3\cos 4x - 2\cos^23x = 1$. 
	Trên đoạn $[ 0;\pi ]$, tổng các nghiệm của phương trình là
	\choice 
	{$ 0 $}
	{$\pi $} 
	{\True $2\pi $}
	{$3\pi $} 
	\loigiai{
		Phương trình 
		$ \Leftrightarrow 3\cos 4x - 1 - \cos 6x = 1 \Leftrightarrow 3(2\cos^22x - 1) - 2 - (4\cos^32x - 3\cos 2x) = 0$ \\
		$ \Leftrightarrow 4\cos^32x - 6\cos^22x - 3\cos 2x + 5 = 0 \Leftrightarrow 
		\hoac{
			&\cos 2x = 1& (N)\\
			&\cos 2x = \dfrac{1-\sqrt{21}}{4}& (N)\\
			&\cos 2x = \dfrac{1+\sqrt{21}}{4}& (L)
		}$ \\
		$ \Leftrightarrow x = k\pi$; 
		$x =  \pm \dfrac{1}{2}\alpha  + k\pi$, 
		$\left(\cos \alpha  = \dfrac{1-\sqrt{21}}{4} \right)$.  \\
		Vì $\dfrac{1-\sqrt{21}}{4} \in ( - 1;0)$, 
		nên ta chọn $\alpha  \in \left(\dfrac{\pi}{2};\pi \right) \Rightarrow \dfrac{1}{2}\alpha  \in \left(\dfrac{\pi}{4};\dfrac{\pi}{2}\right)$. \\
		Trong đoạn $[ 0,\pi]$, 
		ta có các nghiệm $0$, $\pi$, 
		$\dfrac{1}{2}\alpha$, $\pi  - \dfrac{1}{2}\alpha $. \\
		Tổng các nghiệm là $2\pi $.}
\end{ex}

\Closesolutionfile{ans}