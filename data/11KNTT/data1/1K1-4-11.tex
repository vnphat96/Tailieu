\begin{dang}{Phương trình lượng giác không mẫu mực}
	\begin{enumerate}
		\item PHƯƠNG PHÁP  ĐƯA VỀ PHƯƠNG TRÌNH TÍCH\\
		Phương pháp này nhằm biến đổi phương trình lượng giác về dạng 
		$$
		A\cdot B=0 \Leftrightarrow\hoac{
			A=0 \\
			B=0
		}
		$$	
		\item PHƯƠNG PHÁP TỔNG BÌNH PHƯƠNG\\
		Phương pháp này nhằm biến đổi phương trình lượng giác về dạng một vế là tỗng bình phương các số hạng (hay tổng các số hạng không âm) và vế còn lại bằng không và áp dụng tính chất:
		$$
		A^2+B^2=0 \Leftrightarrow\heva{
			A=0 \\
			B=0
		}
		$$	
		\item PHƯƠNG PHÁP ĐỐI LẬP\\
		Phương pháp này nhằm biến đổi phương trình lượng giác về dạng $f(x)=g(x)$, trong đó  $f(x) \geq A, \forall x \in(a, b)$ và $g(x) \leq A, \forall x \in(a, b)$ thì khi đó:
		$$
		f(x)=g(x) \Leftrightarrow\heva{&
			f(x)=A \\
			&g(x)=A}
		$$
		Nếu  $f(x)>A$ và $g(x)<A, \forall x \in(a, b)$ thì kết luận phương trình vô ngiệm trên $(a, b)$.
		
	\end{enumerate}
	
	
\end{dang}
\subsubsection{Ví dụ}
\begin{vd}%[1K1K4-A]%[DCHT-11-KNTT]%[Vĩ Lê Văn] 
	Giải phương trình $2\sin x+\cos x-\sin 2x-1=0$.
	\dapso{$\hoac{&x=\dfrac{\pi}{6}+k2\pi\\&x=\dfrac{5\pi}{6}+k2\pi\\&x=k2\pi}, \quad k \in \mathbb{Z}$.}
	\loigiai{
		\allowdisplaybreaks\begin{eqnarray*}
			&&	2\sin x+\cos x-\sin 2x-1=0\Leftrightarrow 2\sin x+\cos x-2\sin x\cos x-1=0 \\
			& \Leftrightarrow&(\cos x-1)(1-2\sin x)=0\Leftrightarrow\hoac{&\cos x=1\\&\sin x=\dfrac{1}{2}}\Leftrightarrow\hoac{&x=\dfrac{\pi}{6}+k2\pi\\&x=\dfrac{5\pi}{6}+k2\pi\\&x=k2\pi}, \quad k \in \mathbb{Z}.
	\end{eqnarray*}	}
\end{vd}
\begin{vd}%[1K1K4-A]%[DCHT-11-KNTT]%[Vĩ Lê Văn]
	Giải phương trình: $3\tan ^2\mathrm{x}+4\sin ^2\mathrm{x}-2\sqrt{3} \tan x-4\sin x+2=0$.
	\dapso{$x=\dfrac{\pi}{6}+k 2\pi$.}
	
	\loigiai{
		Điều kiện: $\cos x \neq 0\Leftrightarrow x \neq \dfrac{\pi}{2}+k \pi$.\\
		Ta có: $3\tan ^2\mathrm{x}+4\sin ^2\mathrm{x}-2\sqrt{3} \tan x-4\sin x+2=0$\\ $\Leftrightarrow\left(3\tan ^2\mathrm{x}-2\sqrt{3} \tan \mathrm{x}+1\right)+\left(4\sin ^2\mathrm{x}-4\sin \mathrm{x}+1\right)=0$\\ $\Leftrightarrow(\sqrt{3} \tan x-1)^2+(2\sin x-1)^2=0$ 
		\\
		$\Leftrightarrow\left\{\begin{array}{c}\sqrt{3} \tan x-1=0 \\ 2 \sin x-1=0\end{array} \Leftrightarrow\left\{\begin{array}{l}\tan x=\dfrac{1}{\sqrt{3}} \\ \sin x=\dfrac{1}{2}\end{array}\right.\right.$
		$\Leftrightarrow  
		\heva{& x=\dfrac{\pi }{6}+k\pi  \\ 
			& \hoac{
				&	x=\dfrac{\pi }{6}+k2\pi   \\
				&	x=\dfrac{5\pi }{6}+k2\pi   
			}  
		} \Leftrightarrow x=\dfrac{\pi }{6}+k2\pi$.
		\\
		Kết hợp với điều kiện, ta có nghiệm của  phương trình đã cho là $x=\dfrac{\pi}{6}+k 2\pi$.
	}
\end{vd}
\begin{vd}%[1K1K4-A]%[DCHT-11-KNTT]%[Vĩ Lê Văn]
	Giải phương trình: $\cos ^5 x+x^2=0$ 
	\dapso{vô nghiệm.}
	\loigiai{
		Ta có $ 
		\cos ^5 x+x^2=0\Leftrightarrow x^2=-\cos ^5 x \quad (*)$.\\ 
		Vì $-1\leq \cos x \leq 1\Leftrightarrow -1\leq -\cos^5 x \leq 1$, kết hợp với (*), suy ra  $0\leq x^2\leq 1\Leftrightarrow-1\leq x \leq 1$.\\
		Mà $[-1,1] \subset\left(\dfrac{-\pi}{2}, \dfrac{\pi}{2}\right) \Rightarrow \cos x>0, \forall x \in[-1,1] \Rightarrow-\cos ^5 x<0, \forall x \in[-1,1]$\\\ 
		Do $x^2>0$ và $-\cos ^5 x<0$ nên phương trình vô nghiệm. \\
		Vậy phương trình đã cho vô nghiệm.
	}
\end{vd}
\begin{vd}%[1K1K4-A]%[DCHT-11-KNTT]%[Vĩ Lê Văn]
	Giải phương trình: $\sin ^{2024} x+\cos ^{2024} x=1$
	\dapso{$x=k \dfrac{\pi}{2}, \quad k  \in \mathbb{Z}$.}
	\loigiai{
		$\begin{aligned} (1)& \Leftrightarrow \sin ^{2024} x+\cos ^{2024} x=\sin ^2 x+\cos ^2 x \\ & \Leftrightarrow \sin ^2 x\left(\sin ^{2022} x-1\right)=\cos ^2 x\left(1-\cos ^{2022} x\right)\quad (2).\end{aligned}$\\
		Ta thấy $\left\{\begin{aligned}
			&\sin ^2 x \geq 0\\& \sin ^{2022} x \leq 1\end{aligned} \Rightarrow \sin ^2 x\left(\sin ^{2022} x-1\right) \leq 0, \forall x\right.$.\\
		Và $\left\{\begin{aligned}
			&\cos ^2 x \geq 0\\& 1-\cos ^{2022} x \geq 0\end{aligned} \Rightarrow \cos ^2 x\left(1-\cos ^{2022} x\right) \geq 0, \forall x\right.$.\\
		Do đó $(2)\Leftrightarrow \heva{&\sin ^2 x\left(\sin ^{2022} x-1\right) =0\\&\cos ^2 x\left(1-\cos ^{2022} x\right)=0 }\Leftrightarrow \sin2x=0\Leftrightarrow x=k \dfrac{\pi}{2}, \quad k  \in \mathbb{Z}$.\\
		Vậy nghiệm của phương trình là: $x=k \dfrac{\pi}{2}, \quad k  \in \mathbb{Z}$.
	}
\end{vd}

\subsubsection{Bài tập tự luận}
\begin{bt}%[1K1K4-A]%[DCHT-11-KNTT]%[Vĩ Lê Văn]
	Giải phương trình $\sin^22x+\cos^23x=1$.
	\dapso{$\hoac{&x=\dfrac{k\pi}{5}\\&x=k\pi}, \quad k \in \mathbb{Z}$.}
	
	\loigiai{\allowdisplaybreaks\begin{eqnarray*}
			&&\sin^22x+\cos^23x=1\Leftrightarrow\dfrac{1-\cos 4x}{2}+\dfrac{1+\cos 6x}{2}=1\Leftrightarrow \cos 6x-\cos 4x=0\\
			& \Leftrightarrow&-2\sin 5x\cdot\sin x=0   \Leftrightarrow\hoac{&\sin 5x=0\\&\sin x=0}\Leftrightarrow\hoac{&x=\dfrac{k\pi}{5}\\&x=k\pi}, \quad k \in \mathbb{Z}.		
		\end{eqnarray*}
	}
\end{bt}
\begin{bt}%[1K1K4-A]%[DCHT-11-KNTT]%[Vĩ Lê Văn]
	Giải phương trình $\cos^2x-\sin x\cos x=0$. 
	\dapso{$\hoac{&x=\dfrac{\pi}{2}+k\pi\\&x=\dfrac{\pi}{4}+k\pi}, \quad k \in \mathbb{Z}$.}
	
	\loigiai{
		Ta có\allowdisplaybreaks\begin{eqnarray*}		
			&&\cos^2x-\sin x\cos x=0\Leftrightarrow\cos x\left(\cos x-\sin x\right)=0\\
			&\Leftrightarrow&\sqrt{2}\cos x\cos\left(x+\dfrac{\pi}{4}\right)=0 
			\Leftrightarrow\hoac{&\cos x=0\\&\cos\left(x+\dfrac{\pi}{4}\right)=0}\\
			&\Leftrightarrow&\hoac{&x=\dfrac{\pi}{2}+k\pi\\&x+\dfrac{\pi}{4}=\dfrac{\pi}{2}+k\pi}\Leftrightarrow\hoac{&x=\dfrac{\pi}{2}+k\pi\\&x=\dfrac{\pi}{4}+k\pi}, \quad k \in \mathbb{Z}. 
	\end{eqnarray*}}
\end{bt}
\begin{bt}%[1K1K4-A]%[DCHT-11-KNTT]%[Vĩ Lê Văn]
	Giải phương trình $\cos 4 x \cdot \cos x+1=0$ trên $\left[-\dfrac{3 \pi}{2} ; \pi\right]$.
	\dapso{$x\in \left\{-\pi; \pi\right\}$.}
	
	\loigiai{\begin{eqnarray*}
			&\cos 4 x \cdot \cos x+1=0
			\Leftrightarrow\dfrac{1}{2}(\cos 3x+\cos 5x)+1=0
			\Leftrightarrow\cos 3x+\cos 5x+2=0
		\end{eqnarray*}
		Vì $\heva{&\cos3x \ge -1\\&\cos5x \ge -1}\Rightarrow \cos 3x+\cos 5x \ge -2 \Rightarrow \cos 3x+\cos 5x+2 \ge 0$.\\
		Do đó $\cos 3x+\cos 5x+2=0\Leftrightarrow \heva{&\cos3x=-1\\&\cos5x=-1}\Leftrightarrow \heva{ &x=\dfrac{\pi}{3}+\dfrac{k2\pi}{3}\\&x=\dfrac{\pi}{5}+\dfrac{k2\pi}{5}}, \quad k \in \mathbb{Z}$.\\
		Mà  $x \in \left[-\dfrac{3 \pi}{2} ; \pi\right]$ nên phương trình trên có nghiệm $x\in \left\{-\pi; \pi\right\}$.	
	}
\end{bt}

\begin{bt}%[1K1K4-A]%[DCHT-11-KNTT]%[Vĩ Lê Văn]
	Giải phương trình $(2\cos x-1)(2\cos 2x+2\cos x+3)=3-4\sin^2x$.
	\dapso{$\hoac{&x=\pm \dfrac{\pi}{3}+k2\pi\\&x= \dfrac{\pi}{4}+k\dfrac{\pi}{2}}, \quad k \in \mathbb{Z}$.}
	\loigiai{
		Ta có \\
		$\begin{aligned}
			&(2\cos x-1)\left(2\cos 2x+2\cos x+3\right)=3-4\sin^2x \\
			\Leftrightarrow &(2\cos x-1)(4\cos^2x-2+2\cos x+3)=3-4(1-\cos ^2x)\\
			\Leftrightarrow&(2\cos x-1)(4\cos^2x+2\cos x+1)=(2\cos x-1)(2\cos x+1)\\
			\Leftrightarrow&(2\cos x-1)4\cos^2x=0\\
			\Leftrightarrow&\hoac{&2\cos x-1=0\\ &\cos2x=0}\Leftrightarrow \hoac{&x=\pm \dfrac{\pi}{3}+k2\pi\\&x= \dfrac{\pi}{4}+k\dfrac{\pi}{2}}, \quad k \in \mathbb{Z}
		\end{aligned}$. \\
		
	}
\end{bt}
\begin{bt}%[1K1K4-A]%[DCHT-11-KNTT]%[Vĩ Lê Văn]
	Giải phương trình $2\sqrt{3}\sin 5x\cos 3x=\sin 4x+2\sqrt{3}\sin 3x\cos 5x$. 
	\dapso{$x=\dfrac{k\pi}{2}, \quad k \in \mathbb{Z}$.}
	
	\loigiai{
		Ta có \allowdisplaybreaks\begin{eqnarray*}
			&&2\sqrt{3}\sin 5x\cos 3x=\sin 4x+2\sqrt{3}\sin 3x\cos 5x \\
			& \Leftrightarrow& 2\sqrt{3}\left(\sin 5x\cos 3x-\sin 3x\cos 5x\right)=\sin 4x\Leftrightarrow 2\sqrt{3}\sin 2x=2\sin 2x\cos 2x  \\
			& \Leftrightarrow&\hoac{&\sin 2x=0\\&2\sqrt{3}=2\cos 2x}\Leftrightarrow\hoac{&2x=k\pi\\&\cos 2x=\sqrt{3}>1\quad (\text{vô nghiệm})}\Leftrightarrow x=\dfrac{k\pi}{2}, \quad k \in \mathbb{Z}.
	\end{eqnarray*} }
\end{bt}
\begin{bt}%[1K1K4-A]%[DCHT-11-KNTT]%[Vĩ Lê Văn]
	Giải phương trình $\sin 9x\sin x=\sin 3x\sin 7x$. 
	\dapso{$x=\dfrac{k\pi}{6}, \quad k \in \mathbb{Z}$.}
	\loigiai{
		Phương trình tương đương với 
		\begin{eqnarray*}
			& & \dfrac{1}{2}(\cos 8x-\cos 10) = \dfrac{1}{2}(\cos 4x-\cos 10)\\
			&\Leftrightarrow & \cos 8x -\cos 4x=0 
			\Leftrightarrow  -2\sin 6x \sin 2x =0\\
			&\Leftrightarrow & \hoac{& \sin 6x=0 \\ & \sin 2x=0}
			\Leftrightarrow  \hoac{& x=\dfrac{k\pi}{6} \\ & x=\dfrac{k\pi}{2}}
			\Leftrightarrow 	x=\dfrac{k\pi}{6}, \quad k \in \mathbb{Z}.
		\end{eqnarray*}
	}
\end{bt}
\begin{bt}%[1K1K4-A]%[DCHT-11-KNTT]%[Vĩ Lê Văn]
	Tìm số nghiệm thuộc $\left[\dfrac{\pi}{14};\dfrac{69\pi}{10}\right)$ của phương trình $2\sin 3x\left(1-4\sin^2x\right)=0$.
	\dapso{ $7$ nghiệm.}
	\loigiai{
		Ta có\allowdisplaybreaks\begin{eqnarray*}
			&&2\sin 3x\cdot\left(1-4\sin^2x\right)=0\Leftrightarrow\hoac{&\sin 3x=0\\&1-4\sin^2x=0} \\
			&\Leftrightarrow&\hoac{&\sin 3x=0\\&\cos 2x=\dfrac{1}{2}}\Leftrightarrow\hoac{&3x=k\pi\\&2x=\pm\dfrac{\pi}{3}+l2\pi}\Leftrightarrow\hoac{&x=\dfrac{k\pi}{3}\\&x=\pm\dfrac{\pi}{6}+l\pi}  , \quad k,\, l\in\mathbb{Z}.
		\end{eqnarray*}
		\begin{itemize}
			\item  Với $x=\dfrac{k\pi}{3}$. Vì $x\in\left[\dfrac{\pi}{14};\dfrac{69\pi}{10}\right)$ nên{\allowdisplaybreaks\begin{eqnarray*}
					&&\dfrac{\pi}{14}\leq\dfrac{k\pi}{3}<\dfrac{69\pi}{10}\Leftrightarrow\dfrac{3}{14}\leq k<\dfrac{207}{10} 
					, \quad k \in \mathbb{Z}.
			\end{eqnarray*}} 
			Suy ra: $k\in\left\{1;2;3;\ldots;20\right\}$. Có $20$ giá trị $k$ nên có $20$ nghiệm.
			\item  Với $x=\dfrac{\pi}{6}+l\pi$. Vì $x\in\left[\dfrac{\pi}{14};\dfrac{69\pi}{10}\right)$ nên\begin{eqnarray*}
				&&\dfrac{\pi}{14}\leq\dfrac{\pi}{6}+l\pi<\dfrac{69\pi}{10} \Leftrightarrow-\dfrac{2}{21}\leq l<\dfrac{101}{15},\quad  l\in\mathbb{Z}.
			\end{eqnarray*}  Suy ra: $l\in\left\{0;1;2;3;\ldots;6\right\}$. Có $7$ giá trị $l$ nên có $7$ nghiệm.
			\item  Với $x=-\dfrac{\pi}{6}+l\pi$. Vì $x\in\left[\dfrac{\pi}{14};\dfrac{69\pi}{10}\right)$ nên{\allowdisplaybreaks\begin{eqnarray*}
					&&\dfrac{\pi}{14}\leq-\dfrac{\pi}{6}+l\pi<\dfrac{69\pi}{10}\Leftrightarrow\dfrac{5}{21}\leq l<\dfrac{106}{15},\quad  l\in\mathbb{Z}.
			\end{eqnarray*}}  Suy ra: $l\in\left\{1;2;3;\ldots;7\right\}\Rightarrow$ có $7$ giá trị $l$ nên có $7$ nghiệm.
		\end{itemize}
		
		Vậy số nghiệm của phương trình là $20+7+7=34$.}
\end{bt}
\begin{bt}%[1K1K4-A]%[DCHT-11-KNTT]%[Vĩ Lê Văn]
	Tìm nghiệm dương nhỏ nhất của phương trình $\left(2\sin x-\cos x\right)(1+\cos x)=\sin^2x$.
	\dapso{$x=\dfrac{\pi}{6}$.}
	\loigiai{
		Ta có{\allowdisplaybreaks\begin{eqnarray*}
				&&		\left(2\sin x-\cos x\right)(1+\cos x)=\sin^2x\Leftrightarrow\left(2\sin x-\cos x\right)(1+\cos x)=(1-\cos x)(1+\cos x) \\
				&\Leftrightarrow&(1+\cos x)(2\sin x-1)=0\Leftrightarrow\hoac{&\cos x=-1\\&\sin x=\dfrac{1}{2}}\Leftrightarrow\hoac{&x=\pi+k2\pi\\&x=\dfrac{\pi}{6}+k2\pi\\&x=\dfrac{5\pi}{6}+k2\pi.} 			
		\end{eqnarray*}}
		
		Suy ra nghiệm dương nhỏ nhất của phương trình là $x=\dfrac{\pi}{6}$.}
\end{bt}

\begin{bt}%[1K1K4-A]%[DCHT-11-KNTT]%[Vĩ Lê Văn]
	Giải phương trình $4\cos x-2\cos 2x-\cos 4x=1$.
	\dapso{$\hoac{&x=\dfrac{\pi}{2}+k\pi\\&x=k2\pi}, \quad k \in \mathbb{Z}$.}
	\loigiai{\allowdisplaybreaks\begin{eqnarray*}
			&&4\cos x-2\cos 2x-\cos 4x=1\Leftrightarrow 4\cos x-2\cos 2x=1+\cos 4x \\
			&\Leftrightarrow& 4\cos x=2\cos^22x+2\cos 2x\Leftrightarrow 2\cos x=\cos 2x\cdot (\cos 2x+1)  \\
			&\Leftrightarrow& 2\cos x=\cos 2x\cdot 2\cos^2x\Leftrightarrow\cos x\left(1-\cos 2x\cdot\cos x\right)=0  \\
			&\Leftrightarrow&\cos x\cdot\left[1-\left(2\cos^2x-1\right)\cos x\right]=0\Leftrightarrow\cos x\cdot\left(-2\cos^3x+\cos x+1\right)=0  \\
			&\Leftrightarrow&\hoac{&\cos x=0\\&-2\cos^3x+\cos x+1=0}\Leftrightarrow\hoac{&\cos x=0\\&(\cos x-1)\left(-2\cos^2x-2\cos x-1\right)=0}  \\
			&\Leftrightarrow&\hoac{&\cos x=0\\&\cos x=1\\&2\cos^2x+2\cos x+1=0\,\text{  (vô nghiệm)}}\Leftrightarrow\hoac{&x=\dfrac{\pi}{2}+k\pi\\&x=k2\pi}, \quad k \in \mathbb{Z}.
		\end{eqnarray*}
	}
\end{bt}
\begin{bt}%[1K1K4-A]%[DCHT-11-KNTT]%[Vĩ Lê Văn]
	Tìm nghiệm dương nhỏ nhất của phương trình $2\sin x+2\sqrt{2}\sin x\cos x=0$. 
	\dapso{$x=\dfrac{3\pi}{4}$.}
	
	\loigiai{
		Ta có\allowdisplaybreaks\begin{eqnarray*}
			&&2\sin x+2\sqrt{2}\sin x\cos x=0\Leftrightarrow\sin x\left(1+\sqrt{2}\cos x\right)=0\\&\Leftrightarrow&\hoac{&\sin x=0\\&\cos x=-\dfrac{1}{\sqrt{2}}}\Leftrightarrow\hoac{&x=k\pi\\&x=\pm\dfrac{3\pi}{4}+k2\pi}, \quad k \in \mathbb{Z}.		
		\end{eqnarray*}
		
		Suy ra nghiệm dương nhỏ nhất của phương trình là $x=\dfrac{3\pi}{4}$.}
\end{bt}
\subsubsection{Bài tập trắc nghiệm}
\Opensolutionfile{ans}[ans/ans-1K1-4-Dang11]

\begin{ex}%[1K1K4-A]%[DCHT-11-KNTT]%[Vĩ Lê Văn]
	Nghiệm dương nhỏ nhất của phương trình $\sin x+\sin 2x=\cos x+2\cos^2x$ là 
	\choice
	{$\dfrac{\pi}{6}$}
	{$\dfrac{2\pi}{3}$}
	{\True $\dfrac{\pi}{4}$}
	{$\dfrac{\pi}{3}$}
	\loigiai{
		Ta có\allowdisplaybreaks\begin{eqnarray*}
			&&\sin x+\sin 2x=\cos x+2\cos^2x \\
			&\Leftrightarrow&\sin x(1+2\cos x)-\cos x(1+2\cos x)=0\\
			&\Leftrightarrow&\left(\sin x-\cos x\right)(1+2\cos x)=0  \Leftrightarrow\hoac{&\sin x=\cos x\\&\cos x=-\dfrac{1}{2}}\\
			&\Leftrightarrow&\hoac{&\tan x=1\\&\cos x=\cos\left(\dfrac{2\pi}{3}\right)}\Leftrightarrow\hoac{&x=\dfrac{\pi}{4}+k\pi\\&x=\pm\dfrac{2\pi}{3}+k2\pi} , \quad k \in \mathbb{Z} .\\
		\end{eqnarray*} 
		Vậy nghiệm dương nhỏ nhất là $x=\dfrac{\pi}{4}$.}
\end{ex}
\begin{ex}%[1K1K4-A]%[DCHT-11-KNTT]%[Vĩ Lê Văn]
	Một nghiệm của phương trình lượng giác $\sin^2x+\sin^22x+\sin^23x=2$ là 
	\choice
	{$\dfrac{\pi}{3}$}
	{$\dfrac{\pi}{12}$}
	{\True $\dfrac{\pi}{6}$}
	{$\dfrac{\pi}{8}$}
	\loigiai{
		Ta có\allowdisplaybreaks\begin{eqnarray*}
			&&\sin^2x+\sin^22x+\sin^23x=2\Leftrightarrow\dfrac{1-\cos 2x}{2}+\sin^22x+\dfrac{1-\cos 6x}{2}=2 \\
			& \Leftrightarrow&\sin^22x-\dfrac{\cos 6x+\cos 2x}{2}=1\Leftrightarrow\cos^22x+\cos 4x\cos 2x=0  \\
			&\Leftrightarrow&\cos 2x\left(\cos 4x+\cos 2x\right)=0\Leftrightarrow 2\cos 3x\cos 2x\cos x=0  \\
			& \Leftrightarrow&\hoac{&\cos 3x=0\\&\cos 2x=0\\&\cos x=0}\Leftrightarrow\hoac{&x=\dfrac{\pi}{6}+\dfrac{k\pi}{3}\\&x=\dfrac{\pi}{4}+\dfrac{k\pi}{2}\\&x=\dfrac{\pi}{2}+k\pi} , \quad k \in \mathbb{Z} .
	\end{eqnarray*} }
\end{ex}
\begin{ex}%[1K1K4-A]%[DCHT-11-KNTT]%[Vĩ Lê Văn]
	Nghiệm dương nhỏ nhất của phương trình $2\cos^2x+\cos x=\sin x+\sin 2x$ là
	\choice
	{$x=\dfrac{\pi}{6}$}
	{\True $x=\dfrac{\pi}{4}$}
	{$x=\dfrac{\pi}{3}$}
	{$x=\dfrac{2\pi}{3}$}
	\loigiai{
		\allowdisplaybreaks\begin{eqnarray*}
			&&2\cos^2x+\cos x=\sin x+\sin 2x\Leftrightarrow\cos x(2\cos x+1)-\sin x(2\cos x-1)=0 \\
			& \Leftrightarrow&(2\cos x-1)\left(\cos x-\sin x\right)=0\Leftrightarrow\hoac{&\cos x=\dfrac{1}{2}\\&\cos\left(x+\dfrac{\pi}{4}\right)=0}\Leftrightarrow\hoac{&x=\pm\dfrac{\pi}{3}+k2\pi\\&x=\dfrac{\pi}{4}+k\pi}, \quad k \in \mathbb{Z}.
		\end{eqnarray*}	\\
		Vậy nghiệm dương nhỏ nhất của phương trình là $\dfrac{\pi}{4}$.
	}
\end{ex}
\begin{ex}%[1K1K4-A]%[DCHT-11-KNTT]%[Vĩ Lê Văn]
	Tập nghiệm $S$ của phương trình $\cos 2 x-\sqrt{3} \sin 2 x-\sqrt{3} \sin x-\cos x+4=0$ là
	\choice
	{$S=\left\{\dfrac{\pi}{3}+k 2\pi ,\quad \text{với }k\in \mathbb{Z}\right\}$}
	{$S=\left\{\dfrac{\pi}{6}+k 2\pi ,\quad \text{với }k\in \mathbb{Z}\right\}$}
	{$S=\left\{\dfrac{\pi}{3}+k 2\pi ,\quad \text{với }k\in \mathbb{Z}\right\}$}
	{\True $S=\emptyset$}
	
	\loigiai{
		Ta có: $\cos 2 x-\sqrt{3} \sin 2 x-\sqrt{3} \sin x-\cos x+4=0$\\
		$\Leftrightarrow \dfrac{1}{2} \cdot \cos 2 x-\dfrac{\sqrt{3}}{2} \sin 2 x-\dfrac{\sqrt{3}}{2} \sin x-\dfrac{1}{2} \cos x+2=0$\\
		$ \Leftrightarrow\left(\dfrac{1}{2} \cdot \cos 2 x-\dfrac{\sqrt{3}}{2} \sin 2 x\right)-\left(\dfrac{\sqrt{3}}{2} \sin x+\dfrac{1}{2} \cos x\right)+2=0\\
		\Leftrightarrow\left(\sin \dfrac{\pi}{6} \cdot \cos 2 x-\cos \dfrac{\pi}{6} \cdot \sin 2 x\right)-\left(\cos \dfrac{\pi}{6} \cdot \sin x+\sin \dfrac{\pi}{6} \cdot \cos x\right)=-2\\
		\Leftrightarrow \sin \left(\dfrac{\pi}{6}-2 x\right)-\sin \left(x-\dfrac{\pi}{6}\right)=-2\quad (*)$.\\
		Với mọi $x$ ta có: $-1\leq \sin \left(\dfrac{\pi}{6}-2 x\right) \leq 1$ và $-1\leq-\sin \left(x-\dfrac{\pi}{6}\right) \leq 1$.\\
		Do đó $$
		(*)\Leftrightarrow\heva{&\sin \left(\dfrac{\pi}{6}-2 x\right)=-1\\&\sin \left(x-\dfrac{\pi}{6}\right)=1} \\
		\Leftrightarrow\heva{&
			\dfrac {\pi } {6} - 2 x = -\dfrac {\pi } {2} + k 2\pi \\
			&x - \dfrac {\pi } {6} = \dfrac {\pi } {2} + l 2\pi } \Leftrightarrow \heva{
			&x=\dfrac{\pi}{3}+k \pi \\
			&x=\dfrac{2\pi}{3}+l 2\pi}
		\Leftrightarrow x\in \emptyset.
		$$
	}
\end{ex}
\begin{ex}%[1K1K4-A]%[DCHT-11-KNTT]%[Vĩ Lê Văn]
	Cho
	phương trình: $4\cos ^2 x+\tan ^2 x+4=2\cdot(2\cos x-\tan x)$. Tìm số nghiệm của phương trình trên khoảng (0; $10\pi)?$
	\choice
	{$10$}
	{$16$}
	{$22$}
	{\True $0$}
	\loigiai{
		Điều kiện: $\cos x \neq 0$ hay $x \neq \dfrac{\pi }{2}+k \pi$.\\
		Ta có: $4\cos ^2 x+\tan ^2 x+4=2\cdot(2\cos x-\tan x)$\\
		$\Rightarrow 4\cos ^2 x-4\cos x+1+\tan ^2 x+2\tan x+1+2=0$\\
		$ \Rightarrow(2\cos x-1)^2+(\tan x+1)^2+2=0$.\\
		Với mọi $x$ thỏa mãn điều kiện ta có: $(2\cos x-1)^2\geq 0$ và $(\tan x+1)^2\geq 0$ $$
		\Rightarrow(2\cos x-1)^2+(\tan x+1)^2+2>0$$ 
		Vậy phương trình đã cho vô nghiệm.
	}
\end{ex}
\begin{ex}%[1K1K4-A]%[DCHT-11-KNTT]%[Vĩ Lê Văn] 
	Cho
	phương trình $\sin ^{2022} x+\cos ^{2022} x=2\left(\sin ^{2024} x+\cos ^{2024} x\right)$. Số điểm biểu diễn các nghiệm của phương trình trên đường tròn lượng giác là
	\choice
	{$3$}
	{\True $4$}
	{$6$}
	{$8$}
	\loigiai{
		Ta có: $\sin ^{2022} x+\cos ^{2022} x=2\left(\sin ^{2024} x+\cos ^{2024} x\right) \\
		\Leftrightarrow\left(\sin ^{2022} x-2\sin ^{2024} x\right)+\left(\cos ^{2022} x-2\cos ^{2024} x\right)=0\\
		\Leftrightarrow \sin ^{2022} x \cdot\left(1-2\sin ^2 x\right)+\cos ^{2022} x \cdot\left(1-2\cos ^2 x\right)=0\\
		\Leftrightarrow \sin ^{2022} x \cdot \cos 2 x-\cos ^{2022} x \cdot \cos 2 x=0\\
		\Leftrightarrow \cos 2 x \cdot\left(\sin ^{2022} x-\cos ^{2022} x\right)=0\\
		\Leftrightarrow\hoac{&\cos 2 x=0\\&\sin ^{2022} x=\cos ^{2022} x.}$ \\
		+ Trường hợp 1: Nếu $ \cos 2 x=0$ thì $2 x=\dfrac{\pi}{2}+k \pi \Leftrightarrow x=\dfrac{\pi}{4}+\dfrac{k \pi}{2}$ \\
		+ Trường hợp 2: Nếu  $\sin ^{2022} x=\cos ^{2022} x \Leftrightarrow \tan ^{2022} x=1\\
		\Leftrightarrow \tan x= \pm 1\Leftrightarrow x= \pm \dfrac{\pi}{4}+k \pi $\\
		Hợp hai trường hợp ta được nghiệm của phương trình đã cho là $x=\dfrac{\pi}{4}+\dfrac{k \pi}{2}$.\\
		Vậy có 4 điêm biếu diễn trên đường tròn lượng giác}
\end{ex}



% \begin{ex}%[1K1K4-A]%[DCHT-11-KNTT]%[Vĩ Lê Văn]
% 	Phương trình $2\sqrt{3}\sin 5x\cos 3x=\sin 4x+2\sqrt{3}\sin 3x\cos 5x$ có nghiệm là 
% 	\choice
% 	{$x=\dfrac{k\pi}{4},x=\pm\dfrac{1}{4}\arccos\dfrac{\sqrt{3}}{12}+\dfrac{k\pi}{2},k\in\mathbb{Z}$}
% 	{$x=\dfrac{k\pi}{4},x=\pm\arccos\dfrac{\sqrt{3}}{48}+\dfrac{k\pi}{2},k\in\mathbb{Z}$}
% 	{Vô nghiệm}
% 	{\True $x=\dfrac{k\pi}{2},k\in\mathbb{Z}$}
% 	\loigiai{
% 		Ta có \allowdisplaybreaks\begin{eqnarray*}
% 			&&2\sqrt{3}\sin 5x\cos 3x=\sin 4x+2\sqrt{3}\sin 3x\cos 5x \\
% 			& \Leftrightarrow& 2\sqrt{3}\left(\sin 5x\cos 3x-\sin 3x\cos 5x\right)=\sin 4x\Leftrightarrow 2\sqrt{3}\sin 2x=2\sin 2x\cos 2x  \\
% 			& \Leftrightarrow&\hoac{&\sin 2x=0\\&2\sqrt{3}=2\cos 2x}\Leftrightarrow\hoac{&2x=k\pi\\&\cos 2x=\sqrt{3}>1}\Leftrightarrow x=\dfrac{k\pi}{2}, \quad k \in \mathbb{Z}.
% 	\end{eqnarray*} }
% \end{ex}



\begin{ex}%[1K1K4-A]%[DCHT-11-KNTT]%[Vĩ Lê Văn]
	Tổng tất cả các nghiệm của phương trình $\cos^2x\left(\tan^2x-\cos 2x\right)=\cos^3x-\cos^2x+1$ trên đoạn $[0;43\pi]$ bằng
	\choice
	{$\dfrac{4220}{3}\pi$}
	{\True $\dfrac{4225}{3}\pi$}
	{$\dfrac{4230}{3}\pi$}
	{$\dfrac{4235}{3}\pi$}
	\loigiai{
		Điều kiện $\cos^2x\neq 0\Leftrightarrow x\neq\dfrac{\pi}{2}+k\pi$ $, \quad k \in \mathbb{Z}$.\\
		Phương trình đã cho tương đương 
		\begin{eqnarray*}
			&&\sin^2x-\cos^2x\cos 2x=\cos^3x-\cos^2x+1\\
			&\Leftrightarrow& 1-\cos^2x+\cos^2x\left(1-2\cos^2x\right)=\cos^3x-\cos^2x+1\\
			&\Leftrightarrow& 2\cos^4x+\cos^3x-\cos^2x=0\\
			&\Leftrightarrow& 2\cos^2x+\cos x-1=0\\
			&\Leftrightarrow&\hoac{&\cos x=-1\\&\cos x=\dfrac{1}{2}}\\
			&\Leftrightarrow&\hoac{&x=\pi+k2\pi\\&x=\pm\dfrac{\pi}{3}+k2\pi}, \quad k \in \mathbb{Z}.
		\end{eqnarray*}
		\begin{itemize}
			\item $0\leq\pi+k2\pi\leq 43\pi\Leftrightarrow-\dfrac{1}{2}\leq k\leq 21\xrightarrow{k\in\mathbb{Z}}k\in\left\{0;1;2;\ldots;21\right\}$ \\
			$ \Rightarrow $ tổng các nghiệm là $S_1=22\pi+\left(0+1+2+\cdots +21\right)2\pi=484\pi$.
			\item $0\leq\dfrac{\pi}{3}+k2\pi\leq 43\pi\Leftrightarrow-\dfrac{1}{6}\leq k\leq\dfrac{64}{3}\xrightarrow{k\in\mathbb{Z}}k\in\left\{0;1;2;\ldots;21\right\}$ \\
			$ \Rightarrow $ tổng các nghiệm là $S_2=22\cdot\dfrac{\pi}{3}+\left(0+1+2+\cdots +21\right)2\pi=\dfrac{1408}{3}\pi$.
			\item $0\leq-\dfrac{\pi}{3}+k2\pi\leq 43\pi\Leftrightarrow\dfrac{1}{6}\leq k\leq\dfrac{65}{3}\xrightarrow{k\in\mathbb{Z}}k\in\{1;2;\ldots;21\}$ \\
			$ \Rightarrow $ tổng các nghiệm là $S_3=21\cdot\left(-\dfrac{\pi}{3}\right)+\left(1+2+3+\cdots +21\right)2\pi=455\pi$.
		\end{itemize}
		Vậy tổng tất cả các nghiệm của phương trình đã cho trên đoạn $[0;43\pi]$ là
		$$S=S_1+S_2+S_3=\dfrac{4225}{3}\pi.$$}
\end{ex}
\begin{ex}%[1K1K4-A]%[DCHT-11-KNTT]%[Vĩ Lê Văn]
	Phương trình $\sin^2 3x-\cos^2 4x=\sin^2 5x-\cos^2 6x$ có nghiệm là 
	\choice
	{\True $\hoac{&x=k\dfrac{\pi}{9}\\&x=k\dfrac{\pi}{2}}$, $ \quad k \in \mathbb{Z}$}
	{$\hoac{&x=k\dfrac{\pi}{2}\\&x=k\pi}$, $ \quad k \in \mathbb{Z}$}
	{$\hoac{&x=k\dfrac{\pi}{3}\\&x=k\dfrac{\pi}{4}}$, $ \quad k \in \mathbb{Z}$}
	{ $\hoac{&x=k\dfrac{\pi}{6}\\&x=k\dfrac{\pi}{2}}$, $ \quad k \in \mathbb{Z}$}
	\loigiai{
		\allowdisplaybreaks
		\begin{eqnarray*}
			\sin^2 3x-\cos^2 4x=\sin^2 5x-\cos^2 6x &\Leftrightarrow &1-\cos 6x-1-\cos 8x=1-\cos 10x-1-\cos 12x\\
			&\Leftrightarrow &(\cos 12x-\cos 6x)+(\cos 10x-\cos 8x)=0\\
			&\Leftrightarrow &-\sin 9x \cdot \sin 3x-2\sin 9x \cdot \sin x=0\\
			&\Leftrightarrow &\sin 9x (\sin 3x+\sin x)=0\\
			&\Leftrightarrow &2\sin 9x \cdot \sin 2x \cdot \cos x=0 \\
			&\Leftrightarrow &
			\hoac{&\sin 9x=0\\&\sin 2x=0\\&\cos x=0}
			\Leftrightarrow\hoac{&9x=k\pi\\&2x=k\pi\\&x=\dfrac{\pi}{2}+k\pi}
			\Leftrightarrow\hoac{&x=k\dfrac{\pi}{9}\\&x=k\dfrac{\pi}{2}}, \quad k \in \mathbb{Z}.
	\end{eqnarray*}}
\end{ex}
\begin{ex}%[1K1K4-A]%[DCHT-11-KNTT]%[Vĩ Lê Văn]
	Cho phương trình $x^2-\left(2\cos\alpha-3\right)x+7\cos^2\alpha-3\cos\alpha-\dfrac{9}{4}=0$. Gọi $S$ là tập các giá trị của tham số $\alpha$ thuộc đoạn $[0;4\pi]$ để phương trình có nghiệm kép. Tổng các phần tử của tập $S$ bằng
	\choice
	{$\dfrac{20\pi}{3}$}
	{$15\pi$}
	{\True $16\pi$}
	{$17\pi$}
	\loigiai{
		Phương trình đã cho có nghiệm kép khi và chỉ khi
		\begin{eqnarray*}
			&&\Delta=\left(2\cos\alpha-3\right)^2-4\left(7\cos^2\alpha-3\cos\alpha-\dfrac{9}{4}\right)=0 \\
			&\Leftrightarrow & 6\left(3-4\cos^2\alpha\right)=0\\
			&\Leftrightarrow&\hoac{&\cos\alpha=\dfrac{\sqrt{3}}{2}\xrightarrow{\alpha\in[0;4\pi]}\alpha\in\left\{\dfrac{\pi}{6};\dfrac{11\pi}{6};\dfrac{13\pi}{6};\dfrac{23\pi}{6}\right\}\\&\cos\alpha=-\dfrac{\sqrt{3}}{2}\xrightarrow{\alpha\in[0;4\pi]}\alpha\in\left\{\dfrac{5\pi}{6};\dfrac{7\pi}{6};\dfrac{17\pi}{6};\dfrac{19\pi}{6}\right\}.}
		\end{eqnarray*}
		Tổng các phần tử của tập $S$ bằng
		$$\dfrac{\pi}{6}+\dfrac{11\pi}{6}+\dfrac{13\pi}{6}+\dfrac{23\pi}{6}+\dfrac{5\pi}{6}+\dfrac{7\pi}{6}+\dfrac{17\pi}{6}+\dfrac{19\pi}{6}=16\pi.$$}
\end{ex}
\Closesolutionfile{ans}
% \begin{indapan}{10}
% 	{ans/ans-1K1-4-Dang11}
% \end{indapan}