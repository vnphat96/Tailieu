\subsection{Các dạng toán thường gặp}
\begin{dang}{Áp dụng công thức cộng}
	% Một số công thức cộng cần nhớ (với giả thiết các biểu thức đều có nghĩa):
	% \begin{itemize}
	% 	\item $\cos (a-b) = \cos a \cos b + \sin a\sin b.$
	% 	\item $\cos (a+b) = \cos a \cos b - \sin a\sin b.$
	% 	\item $\sin (a-b) = \sin a \cos b - \sin b \cos a$.
	% 	\item $\sin (a+b) = \sin a \cos b + \sin b \cos a$.
	% 	\item $\tan (a-b) = \dfrac{\tan a - \tan b}{1+\tan a \tan b}$.
	% 	\item $\tan (a+b) = \dfrac{\tan a + \tan b}{1-\tan a \tan b}$.
	% \end{itemize}
	Một số trường hợp rút gọn nên nhớ:
	\begin{itemize}
		\item $\sin x + \cos x = \sqrt{2}\sin\left(x+\dfrac{\pi}{4}\right) = \sqrt{2}\cos \left(x-\dfrac{\pi}{4}\right)$.
		\item $\sqrt{3}\sin x + \cos x = 2\sin \left(x+\dfrac{\pi}{6}\right) = 2\cos \left(x-\dfrac{\pi}{3}\right)$.
		\item $\sin x + \sqrt{3}\cos x = 2\sin \left(x+ \dfrac{\pi}{3}\right) = 2\cos \left(x-\dfrac{\pi}{6}\right)$.
	\end{itemize}
\end{dang}
\subsubsection{Ví dụ mẫu}
\begin{vd}[NB]%[DCHT Toán 11 - KNTT -Phong Lê] %[1K1Y2-1] 
	Không dùng máy tính, hãy tính $\cos 105^{\circ}$ và $\cot\dfrac{\pi}{12}$.
	\dapso{$\cos 105^{\circ} = \dfrac{\sqrt{2}-\sqrt{6}}{4},\cot\dfrac{\pi}{12} = 2+\sqrt{3}.$}
	\loigiai{
		$\cos 105^{\circ} = \cos\left(45^{\circ} + 60^{\circ}\right) = \cos45^{\circ}\cos60^{\circ}- \sin45^{\circ}\sin60^{\circ} = \dfrac{\sqrt{2}}{2}.\dfrac{1}{2} - \dfrac{\sqrt{2}}{2}.\dfrac{\sqrt{3}}{2} = \dfrac{\sqrt{2}-\sqrt{6}}{4}.$\\
		$\tan\dfrac{\pi}{12} = \tan\left(\dfrac{\pi}{3}-\dfrac{\pi}{4}\right) = \dfrac{\tan\dfrac{\pi}{3}-\tan\dfrac{\pi}{4}}{1+\tan\dfrac{\pi}{3}\tan\dfrac{\pi}{4}} = \dfrac{\sqrt{3}-1}{1+\sqrt{3}}= \dfrac{\left(\sqrt{3}-1\right)^2}{\sqrt{3}^2-1^2}=\dfrac{4-2\sqrt{3}}{2} = 2-\sqrt{3}$.\\
		Suy ra $ \cot\dfrac{\pi}{12} = \dfrac{1}{2-\sqrt{3}} = 2+\sqrt{3}.$
	}
\end{vd}
\begin{vd}[NB]%[DCHT Toán 11 - KNTT -Phong Lê] %[1K1Y2-1] 
	Chứng minh rằng $\sin x + \sqrt{3}\cos x = 2\sin\left(x+\dfrac{\pi}{3}\right)$.
	\loigiai{
		\textbf{Cách 1:} Ta có $2\sin\left(x+\dfrac{\pi}{3}\right) = 2\left(\sin x\cos\dfrac{\pi}{3}+ \cos x\sin\dfrac{\pi}{3}\right) = 2\left(\dfrac{1}{2}\sin x +\dfrac{\sqrt{3}}{2}\cos x\right) = \sin x + \sqrt{3}\cos x$.\\
		Đẳng thức được chứng minh.\\
		\textbf{Cách 2:} Ta có $\sin x + \sqrt{3}\cos x = 2\left(\dfrac{1}{2}\sin x + \dfrac{\sqrt{3}}{2}\cos x\right) = 2\left(\sin\dfrac{\pi}{3}\sin x + \cos \dfrac{\pi}{3}\cos x\right) = 2\sin \left(x+\dfrac{\pi}{3}\right)$.\\
		Đẳng thức được chứng minh.
	}
\end{vd}
\begin{vd}[TH]%[DCHT Toán 11 - KNTT -Phong Lê] %[1K1B2-1] 
	Tính $\sin\left(a+\dfrac{\pi}{4}\right)$, biết $\sin a = \dfrac{12}{13}$ và $0<a<\dfrac{\pi}{2}$.
	\dapso{$\sin\left(a+\dfrac{\pi}{4}\right) = \dfrac{17\sqrt{2}}{26}$.}
	\loigiai{
		Ta có $\sin^2a+\cos^2a = 1 \Rightarrow \cos^2a=1-\sin^2a = 1-\left(\dfrac{12}{13}\right)^2 = \dfrac{25}{169}$.\\
		Vì $0<a<\dfrac{\pi}{2}$ nên $\cos a >0$, suy ra $\cos a = \sqrt{\dfrac{25}{169}} = \dfrac{5}{13}$.\\
		Do đó $\sin\left(a+\dfrac{\pi}{4}\right) = \sin a \cos\dfrac{\pi}{4} + \cos a \sin\dfrac{\pi}{4} = \dfrac{12}{13}.\dfrac{\sqrt{2}}{2} + \dfrac{5}{13}.
		\dfrac{\sqrt{2}}{2} = \dfrac{17\sqrt{2}}{26}$.\\
		Vậy $\sin\left(a+\dfrac{\pi}{4}\right) = \dfrac{17\sqrt{2}}{26}$.
		
	}
	
\end{vd}
\begin{vd}[VDT]%[DCHT Toán 11 - KNTT -Phong Lê] %[1K1K2-1]
	Không sử dụng máy tính, hãy tính $ P =\cos 10^{\circ}\cos35^{\circ} - \cos55^{\circ}\cos80^{\circ}.$
	\dapso{$P = \dfrac{\sqrt{2}}{2}$.}
	\loigiai{
		Ta có \\
		$ P =\cos 10^{\circ}\cos35^{\circ} + \cos55^{\circ}\cos80^{\circ} = \cos 10^{\circ}\cos35^{\circ} - \sin 35^{\circ}\sin10^{\circ}  = \cos \left(10^{\circ} + 35^{\circ}\right) = \cos 45^{\circ} = \dfrac{\sqrt{2}}{2}$.\\
		Vậy $P = \dfrac{\sqrt{2}}{2}$.
	}
\end{vd}
\begin{vd}[VDT]%[DCHT Toán 11 - KNTT -Phong Lê] %[1K1K2-1] 
	Chứng minh giá trị của biểu thức
	$$P = \sin \left(\dfrac{\pi}{6}-\alpha\right) + \sin \left(\dfrac{\pi}{6} + \alpha\right) -\cos \alpha$$
	không phụ thuộc vào $\alpha$.
	\dapso{$P=0$.}
	\loigiai{
		Ta có\\
		$\begin{array}{ll}
			P &= \sin \left(\dfrac{\pi}{6}-\alpha\right) + \sin \left(\dfrac{\pi}{6} + \alpha\right) -\cos \alpha\\
			&=\sin \dfrac{\pi}{6} \cos \alpha -\cos \dfrac{\pi}{6}\sin \alpha + \sin \dfrac{\pi}{6} \cos \alpha +\cos \dfrac{\pi}{6}\sin \alpha -\cos \alpha\\
			&=\dfrac{1}{2}\cos \alpha + \dfrac{1}{2}\cos \alpha - \cos \alpha\\
			&=0.
		\end{array}$\\
		Vậy giá trị của biểu thức $P$ không phụ thuộc vào $\alpha$.
		
	}
\end{vd}
\begin{vd}[VDC]%[DCHT Toán 11 - KNTT -Phong Lê] %[1K1G2-1] 
	Một thiết bị trễ kỹ thuật số lặp lại tín hiệu đầu vào bằng cách lặp lại tín hiệu đó trong một khoảng thời gian cố định sau khi nhận được tín hiệu. Nếu một thiết bị như vậy nhận được nốt thuần $f_1(t) = 5\sin t$ và phát lại nốt thuần $f_2(t) = 5\cos t$ thì âm kết hợp là $f(t)=f_1(t)+f_2(t)$, trong đó $t$ là biến thời gian. Chứng tỏ rằng âm kết hợp viết được dưới dạng $f(t) = k\sin(t+\varphi)$, tức là âm kết hợp là sóng hình sin. Hãy xác định biên độ âm $k$ và pha ban đầu $\varphi$ ($-\pi < \varphi <\pi$) của sóng âm.
	\dapso{$k = 5\sqrt{2},\varphi = \dfrac{\pi}{4}$.}
	\loigiai{Ta có $f(t) = f_1(t) + f_2(t) = 5\sin t + 5 \cos t.$\\
		Mà $\sin t + \cos t = \sqrt{2}\left(\dfrac{1}{\sqrt{2}}\sin t + \dfrac{1}{\sqrt{2}}\cos t\right) = \sqrt{2}\left(\sin t\cos\dfrac{\pi}{4} + \cos t\sin\dfrac{\pi}{4}\right) = \sqrt{2}\sin \left(t+\dfrac{\pi}{4}\right)$,\\
		suy ra $f(t) = 5\left(\sin t + \cos t\right) = 5\sqrt{2}\sin\left(t + \dfrac{\pi}{4}\right).$\\
		Vậy biên độ âm của sóng âm là $k = 5\sqrt{2}$ và pha ban đầu của sóng âm là $\varphi = \dfrac{\pi}{4}$.}
\end{vd}

\subsubsection{Bài tập rèn luyện}
% \centerline{\fcolorbox{red}{yellow!50}{\bf {BÀI TẬP TỰ LUẬN }}}
\begin{bt}[NB]%[DCHT Toán 11 - KNTT -Phong Lê] %[1K1Y2-1] 
	Tính các giá trị lượng giác của góc $75^{\circ}$.
	\dapso{$\sin 75^{\circ}=\dfrac{\sqrt{6}+\sqrt{2}}{4}$, $\cos 75^{\circ} = \dfrac{\sqrt{6}-\sqrt{2}}{4}$, $\tan 75^{\circ} = 2+\sqrt{3}$, $\cot 75^{\circ} = 2-\sqrt{3}$.}
	\loigiai{
		Ta có:\\
		$\sin 75^{\circ} = \sin\left(30^{\circ}+45^{\circ}\right) = \sin 30^{\circ} \cos 45^{\circ} + \cos 30^{\circ}\sin 45^{\circ} = \dfrac{1}{2}.\dfrac{\sqrt{2}}{2} + \dfrac{\sqrt{3}}{2}.\dfrac{\sqrt{2}}{2} = \dfrac{\sqrt{6} + \sqrt{2}}{4}$.\\
		$\cos 75^{\circ} = \cos \left(30^{\circ}+45^{\circ}\right) = \cos 30^{\circ}\cos 45^{\circ} - \sin 30^{\circ}\sin 45^{\circ} = \dfrac{\sqrt{3}}{2}.\dfrac{\sqrt{2}}{2}-\dfrac{1}{2}.\dfrac{\sqrt{2}}{2} = \dfrac{\sqrt{6}-\sqrt{2}}{4}$.\\
		$\tan 75^{\circ} = \dfrac{\sin 75^{\circ}}{\cos 75^{\circ}}=\dfrac{\sqrt{6}+\sqrt{2}}{4}:\dfrac{\sqrt{6}-\sqrt{2}}{4} = \dfrac{\sqrt{6}+\sqrt{2}}{\sqrt{6}-\sqrt{2}}= \dfrac{\left(\sqrt{6}+\sqrt{2}\right)^2}{\sqrt{6}^2-\sqrt{2}^2} = \dfrac{8+4\sqrt{3}}{4}= 2+\sqrt{3}$.\\
		$\cot 75^{\circ} = \dfrac{1}{\tan 75^{\circ}} = \dfrac{1}{2+\sqrt{3}} = \dfrac{2-\sqrt{3}}{2^2-\sqrt{3}^2} = 2-\sqrt{3}$.
	}
\end{bt}
\begin{bt}[NB]%[DCHT Toán 11 - KNTT -Phong Lê] %[1K1Y2-1] 
	Chứng minh rằng $\sin x + \cos x = \sqrt{2}\sin\left(x+\dfrac{\pi}{4}\right)$.
	\loigiai{
		\textbf{Cách 1:} Ta có $$\sqrt{2}\sin\left(x+\dfrac{\pi}{4}\right) = \sqrt{2}\left(\sin x \cos \dfrac{\pi}{4} + \cos x \sin \dfrac{\pi}{4}\right) = \sqrt{2}\left(\dfrac{\sqrt{2}}{2}\sin x + \dfrac{\sqrt{2}}{2}\cos x\right) = \sin x+\cos x.$$
		Vậy $\sin x + \cos x = \sqrt{2}\sin\left(x+\dfrac{\pi}{4}\right)$.\\
		\textbf{Cách 2:} Ta có $$\sin x + \cos x = \sqrt{2}\left(\dfrac{1}{\sqrt{2}}\sin x + \dfrac{1}{\sqrt{2}}\cos x\right) = \sqrt{2}\left(\sin x \cos \dfrac{\pi}{4} + \cos x \sin \dfrac{\pi}{4}\right) = \sqrt{2}\sin\left(x+\dfrac{\pi}{4}\right).$$
		Vậy $\sin x + \cos x = \sqrt{2}\sin\left(x+\dfrac{\pi}{4}\right)$.
	}
\end{bt}
\begin{bt}[TH]%[DCHT Toán 11 - KNTT -Phong Lê] %[1K1B2-1] 
	Tính giá trị của biểu thức $P = \dfrac{\cos\dfrac{5\pi}{18}\cos\dfrac{\pi}{9}+\sin \dfrac{5\pi}{18}\sin \dfrac{\pi}{9} }{\sin \dfrac{\pi}{5} \cos \dfrac{3\pi}{10} + \cos \dfrac{\pi}{5}\sin\dfrac{3\pi}{10}}$.
	\dapso{$P = \dfrac{\sqrt{3}}{2}$.}
	\loigiai{Ta có $P = \dfrac{\cos \left(\dfrac{5\pi}{18}-\dfrac{\pi}{9}\right)}{\sin\left(\dfrac{\pi}{5}+\dfrac{3\pi}{10}\right)} = \dfrac{\cos \dfrac{\pi}{6}}{\sin \dfrac{\pi}{2}} = \dfrac{\sqrt{3}}{2}$.\\
		Vậy $P = \dfrac{\sqrt{3}}{2}$.}
\end{bt}
\begin{bt}[TH]%[DCHT Toán 11 - KNTT -Phong Lê] %[1K1B2-1] 
	Tính $\tan \left(x+\dfrac{\pi}{4}\right)$ biết $\cos x = \dfrac{2}{3}$ và $0<x<\pi$.
	\dapso{$\tan \left(x+\dfrac{\pi}{4}\right) = -9-4\sqrt{5}$.}
	\loigiai{
		Ta có $\sin^2 x + \cos^2 x = 1 \Rightarrow \sin^2x = 1-\cos^2x = 1-\left(\dfrac{2}{3}\right)^2 = \dfrac{5}{9}$.\\
		Vì $0<x<\pi$ nên $\sin x >0$, suy ra $\sin x = \sqrt{\dfrac{5}{9}} = \dfrac{\sqrt{5}}{3}$.\\
		Do đó $\tan x = \dfrac{\sin x}{\cos x} = \dfrac{\sqrt{5}}{3}:\dfrac{2}{3} = \dfrac{\sqrt{5}}{2}$.\\
		Suy ra $\tan \left(x + \dfrac{\pi}{4}\right) = \dfrac{\tan x + \tan \dfrac{\pi}{4}}{1-\tan x\tan \dfrac{\pi}{4}} = \dfrac{\dfrac{\sqrt{5}}{2}+1}{1-\dfrac{\sqrt{5}}{2}.1} = \dfrac{\sqrt{5}+2}{2-\sqrt{5}} = \dfrac{\left(\sqrt{5}+2\right)^2}{2^2-\sqrt{5}^2} = -9-4\sqrt{5}$.\\
		Vậy $\tan \left(x+\dfrac{\pi}{4}\right) = -9-4\sqrt{5}$.
	}
\end{bt}
\begin{bt}[TH]%[DCHT Toán 11 - KNTT -Phong Lê] %[1K1B2-1] 
	Tính $\sin \left(x+\dfrac{\pi}{3}\right)$ biết $\sin x + \sqrt{3}\cos x = 1$.
	\dapso{$\sin\left(x + \dfrac{\pi}{3}\right) - \dfrac{1}{2}.$}
	\loigiai{Ta có
		$\sin \left(x+\dfrac{\pi}{3}\right) = \sin x \cos \dfrac{\pi}{3} + \cos x\sin \dfrac{\pi}{3} = \dfrac{1}{2}\sin x + \dfrac{\sqrt{3}}{2}\cos x =\dfrac{1}{2}\left(\sin x + \sqrt{3}\cos x\right) = \dfrac{1}{2}$.\\
		Vậy $\sin \left(x + \dfrac{\pi}{3}\right)$ = $\dfrac{1}{2}$.
	}
\end{bt}
\begin{bt}[VDT]%[DCHT Toán 11 - KNTT -Phong Lê] %[1K1K2-1]
	Không sử dụng máy tính, hãy tính $P = \cos 20^{\circ}\cos 40^{\circ} - \sin 140^{\circ}\sin 160^{\circ}.$
	\dapso{$P = \dfrac{1}{2}$.}
	\loigiai{Ta có \\
		$P = \cos 20^{\circ}\cos 40^{\circ} - \sin 140^{\circ}\sin 160^{\circ} = \cos 20^{\circ}\cos 40^{\circ} -\sin 40^{\circ}\sin 20^{\circ} = \cos\left(40^{\circ}+20^{\circ}\right) = \cos 60^{\circ} = \dfrac{1}{2}$.\\
		Vậy $P= \dfrac{1}{2}$.
	}
\end{bt}
\begin{bt}[VDT]%[DCHT Toán 11 - KNTT -Phong Lê] %[1K1K2-1] 
	Cho tam giác $ABC$ có $\cos B = \dfrac{3}{5}$, $\cos C = \dfrac{\sqrt{21}}{5}$. Chứng minh rằng $$\sin A =\sin B \cos C + \cos B \sin C$$ và tính $\sin A$.
	\dapso{$\sin A = \dfrac{6+4\sqrt{21}}{25}$.}
	\loigiai{
		Theo tính chất tổng ba góc trong một tam giác, ta có $A+B+C = \pi$.\\
		Do đó
		$\sin A = \sin (\pi - A) =\sin (B+C) = \sin B\cos C + \cos B \sin C.$	\\
		Vậy $\sin A =\sin B \cos C + \cos B \sin C$.\\
		Ta có $\sin^2B+\cos^2B = 1$, suy ra $\sin^2B = 1-\cos^2B = 1-\left(\dfrac{3}{5}\right)^2 = \dfrac{16}{25}$.\\
		Mà $0<B<\pi$ nên $\sin B >0$, suy ra $\sin B =\sqrt{\dfrac{16}{25}} = \dfrac{4}{5}$.\\
		Lại có $\sin^2C+\cos^2C = 1$, suy ra $\sin^2C = 1-\cos^2C = 1-\left(\dfrac{\sqrt{21}}{5}\right)^2 = \dfrac{4}{25}$.\\
		Mà $0<C<\pi$ nên $\sin C >0$, suy ra $\sin C =\sqrt{\dfrac{4}{25}} = \dfrac{2}{5}$.\\
		Do đó $\sin A = \sin B \cos C + \cos B \sin C = \dfrac{4}{5}.\dfrac{\sqrt{21}}{5} + \dfrac{3}{5}.\dfrac{2}{5} = \dfrac{6+4\sqrt{21}}{25}$.\\
		Vậy $\sin A = \dfrac{6+4\sqrt{21}}{25}$.
	}
	
\end{bt}

\begin{bt}[VDT]%[DCHT Toán 11 - KNTT -Phong Lê] %[1K1K2-1] 
	Với giả thiết các biểu thức đều có nghĩa, chứng minh rằng
	$$\cot (a+b) = \dfrac{\cot a \cot b -1}{\cot a + \cot b}.$$
	\loigiai{
		Ta có
		$\cot (a+b) = \dfrac{1}{\tan (a+b)} = \dfrac{1}{\dfrac{\tan a+\tan b}{1-\tan a\tan b}} = \dfrac{1-\tan a\tan b}{\tan a+\tan b} = \dfrac{\dfrac{1}{\tan a}.\dfrac{1}{tan b}-1}{\dfrac{1}{\tan b}+\dfrac{1}{\tan a}} = \dfrac{\cot a . \cot b - 1}{\cot a+\cot b}$.\\
		Vậy $\cot (a+b) = \dfrac{\cot a \cot b -1}{\cot a + \cot b}.$
	}
\end{bt}
\begin{bt}[VDC]%[DCHT Toán 11 - KNTT -Phong Lê] %[1K1G2-1]
	Một vật thực hiện đồng thời hai dao động điều hòa có phương trình $x_1(t) =2\sqrt{3} \sin \left(4\pi t+\dfrac{\pi}{6}\right)$ và $x_2(t) = 2\cos \left(4\pi t+\dfrac{\pi}{6}\right)$. Chứng tỏ rằng phương trình dao động tổng hợp của vật đó $x(t) = x_1(t)+x_2(t)$ viết được dưới dạng $x(t) = A\cos (\omega t + \varphi)$, tức là dao động tổng hợp của vật đó là dao động điều hòa. Hãy xác định biên độ $A$, tần số góc $\omega$ và pha ban đầu $\varphi$ ($-\pi<\varphi<\pi$) của dao động tổng hợp.
	\dapso{$A=4, \omega = 4\pi, \varphi = -\dfrac{\pi}{6}$.}
	\loigiai{
		Ta có $x(t) = x_1(t)+x_2(t) = 2\sqrt{3}\sin \left(4\pi t+\dfrac{\pi}{6}\right) + 2\cos \left(4\pi t+\dfrac{\pi}{6}\right)$, đồng thời\\
		$\dfrac{1}{2}\cos\left(4\pi t+\dfrac{\pi}{6}\right) + \dfrac{\sqrt{3}}{2}\sin \left(4\pi t+\dfrac{\pi}{6}\right) =\cos \left(4\pi t+\dfrac{\pi}{6}\right) \cos \dfrac{\pi}{3} + \sin \left(4\pi t+\dfrac{\pi}{6}\right)\sin \dfrac{\pi}{3}  = \cos \left(4\pi t-\dfrac{\pi}{6}\right) $,\\
		suy ra $x(t) = 4\left[\dfrac{1}{2}\cos\left(4\pi t+\dfrac{\pi}{6}\right) + \dfrac{\sqrt{3}}{2}\sin \left(4\pi t+\dfrac{\pi}{6}\right)\right] = 4\cos\left(4\pi t - \dfrac{\pi}{6}\right).$\\
		Vậy dao động tổng hợp $x(t)$ có biên độ $A = 4$, tần số góc $\omega = 4\pi$ và pha ban đầu $\varphi = -\dfrac{\pi}{6}$.
	}
\end{bt}
\subsubsection{Bài tập trắc nghiệm}
\Opensolutionfile{ans}[ans/ans-1K1-2-Dang1]
\begin{ex}%[DCHT Toán 11 - KNTT -Phong Lê] %[1K1Y2-1]
	Với mọi $a,b$, ta có $\sin(a-b)$ bằng
	\choice{$\sin a \sin b-\cos a\cos b$}{$\sin b \cos a- \sin a\cos b$}{\True $\sin a \cos b - \cos a \sin b$}{$\sin a \cos b + \cos a\sin b$}
	\loigiai{Theo công thức cộng, ta có $\sin (a-b) = \sin a \cos b -\cos a \sin b$.	
	}
\end{ex}
\begin{ex}%[DCHT Toán 11 - KNTT -Phong Lê] %[1K1Y2-1]
	Biết rằng $\dfrac{7\pi}{12} = \dfrac{\pi}{3}+\dfrac{\pi}{4}$, khi đó giá trị $\cos\dfrac{7\pi}{12}$ bằng
	\choice{\True $\cos\dfrac{\pi}{3}\cos\dfrac{\pi}{4}-\sin\dfrac{\pi}{3}\sin\dfrac{\pi}{4}$}{$\cos\dfrac{\pi}{3}\cos\dfrac{\pi}{4}+\sin\dfrac{\pi}{3}\sin\dfrac{\pi}{4}$}{$\sin\dfrac{\pi}{3}\cos\dfrac{\pi}{4}-\cos\dfrac{\pi}{3}\sin\dfrac{\pi}{4}$}{$\sin\dfrac{\pi}{3}\cos\dfrac{\pi}{4}+\cos\dfrac{\pi}{3}\sin\dfrac{\pi}{4}$}
	\loigiai{Áp dụng công thức cộng, ta có 
		$\cos\dfrac{7\pi}{12}= \cos\left(\dfrac{\pi}{3}+\dfrac{\pi}{4}\right) =\cos\dfrac{\pi}{3}\cos\dfrac{\pi}{4}-\sin\dfrac{\pi}{3}\sin\dfrac{\pi}{4} $.
	}
\end{ex}
\begin{ex}%[DCHT Toán 11 - KNTT -Phong Lê] %[1K1Y2-1]
	Thu gọn $\sin a \sin b - \cos a \cos b$, ta được
	\choice{\True $-\cos (a+b)$}{$\cos (a-b)$}{$\cos (a+b)$}{$-\cos (a-b)$}
	\loigiai{Ta có $\sin a \sin b - \cos a\cos b = -(\cos a\cos b - \sin a \sin b) = -\cos (a+b)$.}
\end{ex}
\begin{ex}%[DCHT Toán 11 - KNTT -Phong Lê] %[1K1Y2-1]
	Với điều kiện các biểu thức đều xác định, biểu thức nào sau đây bằng $\tan (a-b)$?
	\choice{$\tan a \cot b - \tan b \cot a$}{\True $\dfrac{\tan a - \tan b}{1 + \tan a \tan b}$}{$\dfrac{\tan a + \tan b}{1-\tan a \tan b}$}{$\dfrac{1-\tan a\tan b}{\tan a + \tan b}$}
	\loigiai{
		Áp dụng công thức cộng, ta có $\tan(a-b) = \dfrac{\tan a - \tan b}{1 + \tan a \tan b}$.
	}
\end{ex}
\begin{ex}%[DCHT Toán 11 - KNTT -Phong Lê] %[1K1Y2-1]
	Cho $a,b$ thỏa $\tan a = \tan b = 2$. Tính $\tan (a+b)$.
	\choice{\True $-\dfrac{4}{3}$}{$\dfrac{4}{3}$}{$0$}{$\dfrac{3}{4}$}
	\loigiai{Áp dụng công thức cộng, ta có $\tan (a+b) = \dfrac{\tan a + \tan b}{1 - \tan a \tan b} = \dfrac{2+2}{1-2.2} = -\dfrac{4}{3}$.}
\end{ex}
\begin{ex}%[DCHT Toán 11 - KNTT -Phong Lê] %[1K1Y2-1]
	Với $\tan \left(x+\dfrac{\pi}{4}\right)$ và $\tan x$ xác định, biểu thức nào sau đây bằng $\tan \left(x+\dfrac{\pi}{4}\right)$?
	\choice{$\dfrac{1-\tan x}{1+\tan x}$}{$\dfrac{\tan x-1}{\tan x +1}$}{$\dfrac{\tan x + 1}{\tan x -1}$}{\True $\dfrac{1+\tan x}{1-\tan x}$}
	\loigiai{Áp dụng công thức cộng, ta có $\tan \left(x + \dfrac{\pi}{4}\right) = \dfrac{\tan x + \tan \dfrac{\pi}{4}}{1-\tan x \tan\dfrac{\pi}{4} } = \dfrac{\tan x + 1}{1- \tan x}$.}
\end{ex}
\begin{ex}%[DCHT Toán 11 - KNTT -Phong Lê] %[1K1Y2-1]
	Biểu thức nào sau đây bằng $\cos\left(x-\dfrac{\pi}{3}\right)$?
	\choice{$\dfrac{1}{2} \cos x - \dfrac{\sqrt{3}}{2}\sin x$}{$\dfrac{\sqrt{3}}{2}\cos x - \dfrac{1}{2}\sin x$}{\True $\dfrac{1}{2} \cos x + \dfrac{\sqrt{3}}{2}\sin x$}{$\dfrac{\sqrt{3}}{2}\cos x + \dfrac{1}{2}\sin x$}
	\loigiai{Áp dụng công thức cộng, ta có $\cos \left(x-\dfrac{\pi}{3}\right) = \cos x \cos \dfrac{\pi}{3} + \sin x \sin \dfrac{\pi}{3} = \dfrac{1}{2}\cos x + \dfrac{\sqrt{3}}{2}\sin x$.}
\end{ex}
\begin{ex}%[DCHT Toán 11 - KNTT -Phong Lê] %[1K1Y2-1]
	Thu gọn $\sin x + \cos x$ ta được
	\choice{$\sqrt{2}\sin\left(x-\dfrac{\pi}{4}\right)$}{\True $\sqrt{2}\cos\left(x-\dfrac{\pi}{4}\right)$}{$\sin\left(x-\dfrac{\pi}{4}\right)$}{$\cos\left(x-\dfrac{\pi}{4}\right)$}
	\loigiai{Ta có $\sin x + \cos x = \sqrt{2}\left(\dfrac{1}{\sqrt{2}\sin x} + \dfrac{1}{\sqrt{2}}\cos x\right) = \sqrt{2}\left(\cos x\cos \dfrac{\pi}{4} + \sin x \sin \dfrac{\pi}{4}\right) = \sqrt{2} \cos \left(x-\dfrac{\pi}{4}\right)$.
	}
\end{ex}
\begin{ex}%[DCHT Toán 11 - KNTT -Phong Lê] %[1K1B2-1]
	Cho $x \in \left[0;\dfrac{\pi}{2}\right]$ thỏa $\sin x = \dfrac{7}{25}$, giá trị của $\sqrt{2}\cos\left(x+\dfrac{\pi}{4}\right)$ là
	\choice{$\sqrt{2}\cos\left(x+\dfrac{\pi}{4}\right)=\dfrac{17\sqrt{2}}{50}$}{$\sqrt{2}\cos\left(x+\dfrac{\pi}{4}\right)=-\dfrac{17}{25}$}{$\sqrt{2}\cos\left(x+\dfrac{\pi}{4}\right)=\dfrac{31}{25}$}{\True $\sqrt{2}\cos\left(x+\dfrac{\pi}{4}\right)=\dfrac{17}{25}$}
	\loigiai{
		Ta có $\sin^2x+\cos^2x = 1$, suy ra $\cos^2x = 1-\sin^2 x = 1 -\left(\dfrac{7}{25}\right)^2 = \dfrac{576}{625}$.\\
		Mà $x \in \left[0;\dfrac{\pi}{2}\right]$ nên $\cos x \ge 0$, suy ra $\cos x = \sqrt{\dfrac{576}{625}} = \dfrac{24}{25}$.\\
		Lại có $\sqrt{2}\cos \left(x + \dfrac{\pi}{4}\right) = \sqrt{2}\left(\cos x \cos \dfrac{\pi}{4} - \sin x \sin \dfrac{\pi}{4}\right) = \sqrt{2}\left(\dfrac{\sqrt{2}}{2}\cos x -\dfrac{\sqrt{2}}{2} \sin x\right) = \cos x - \sin x$,\\
		suy ra $\sqrt{2}\cos \left(x + \dfrac{\pi}{4}\right) = \dfrac{24}{25}-\dfrac{7}{25} = \dfrac{17}{25}$.
	}
\end{ex}

\begin{ex}%[DCHT Toán 11 - KNTT -Phong Lê] %[1K1B2-1]
	Cho $x \in [0;\pi]$ thỏa $\cos x = \dfrac{3}{5}$. Tính $\tan \left(x + \dfrac{\pi}{4}\right)$.
	\choice{$7$}{\True $-7$}{$\dfrac{1}{7}$}{$-\dfrac{1}{7}$}
	\loigiai{Ta có $\cos^2x+\sin^2x = 1$, suy ra $\sin^2x = 1-\cos^2x = 1-\left(\dfrac{3}{5}\right)^2 = \dfrac{16}{25}$.\\
		Vì $x\in [0;\pi]$ nên $\sin x \le 0$, suy ra $\sin x = \sqrt{\dfrac{16}{25}} = \dfrac{4}{5}$.\\
		Ta có $\tan x = \dfrac{\sin x}{\cos x} = \dfrac{4}{5}:\dfrac{3}{5} = \dfrac{4}{3}$, suy ra $\tan \left(x+\dfrac{\pi}{4}\right) = \dfrac{\tan x + \tan \dfrac{\pi}{4}}{1 - \tan x\tan \dfrac{\pi}{4}} = \dfrac{1+\dfrac{4}{3}}{1-\dfrac{4}{3}} = -7$.
	}
\end{ex}
\begin{ex}%[DCHT Toán 11 - KNTT -Phong Lê] %[1K1B2-1]
	Giá trị của biểu thức $P=\dfrac{\sin \dfrac{2\pi}{13}\cos \dfrac{\pi}{13}-\cos \dfrac{2\pi}{13}\sin \dfrac{\pi}{13}}{\cos \dfrac{2\pi}{13}\cos \dfrac{\pi}{13}+ \sin \dfrac{2\pi}{13}\sin \dfrac{\pi}{13}}$ là
	\choice{\True $\tan \dfrac{\pi}{13}$}{$\sin \dfrac{\pi}{13}$}{$\tan \dfrac{3\pi}{13}$}{$\sin \dfrac{3\pi}{13}$}
	\loigiai{Ta có $P=\dfrac{\sin \dfrac{2\pi}{13}\cos \dfrac{\pi}{13}-\cos \dfrac{2\pi}{13}\sin \dfrac{\pi}{13}}{\cos \dfrac{2\pi}{13}\cos \dfrac{\pi}{13}+ \sin \dfrac{2\pi}{13}\sin \dfrac{\pi}{13}} = \dfrac{\sin \left(\dfrac{2\pi}{13} - \dfrac{\pi}{13}\right)}{\cos \left(\dfrac{2\pi}{13}-\dfrac{\pi}{13}\right)} = \dfrac{\sin\dfrac{\pi}{13}}{\cos \dfrac{\pi}{13}} = \tan \dfrac{\pi}{13}$.}
\end{ex}
\begin{ex}%[DCHT Toán 11 - KNTT -Phong Lê] %[1K1B2-1]
	Giá trị của biểu thức $P=\sin 10^{\circ}\cos 20^{\circ}+\sin 20^{\circ}\cos 10^{\circ}$ là
	\choice{$\dfrac{\sqrt{3}}{2}$}{$-\dfrac{\sqrt{3}}{2}$}{\True $\dfrac{1}{2}$}{$-\dfrac{1}{2}$}
	\loigiai{Ta có $P=\sin 10^{\circ}\cos 20^{\circ}+\sin 20^{\circ}\cos 10^{\circ} = \sin \left(10^{\circ} + 20^{\circ}\right) = \sin 30 ^{\circ} = \dfrac{1}{2}$.}
\end{ex}
\begin{ex}%[DCHT Toán 11 - KNTT -Phong Lê] %[1K1B2-1]
	Cho $a,b$ thỏa $a +b \ne k\pi$. Biểu thức nào sau đây bằng $P=\cot (a+b)$?
	\choice{\True $\dfrac{\cos a\cos b - \sin a \sin b}{\sin a \cos b + \cos a \sin b}$}{$\dfrac{\cos a\cos b + \sin a \sin b}{\sin a \cos b + \cos a \sin b}$}{$\dfrac{\sin a \cos b + \cos a \sin b}{\cos a\cos b - \sin a \sin b}$}{$\dfrac{\sin a \cos b + \cos a \sin b}{\cos a\cos b + \sin a \sin b}$}
	\loigiai{Ta có $\cot (a+b) = \dfrac{\cos (a+b)}{\sin (a+b)} = \dfrac{\cos a \cos b - \sin a \sin b}{\sin a \cos b + \cos a \sin b}.$}
\end{ex}
\begin{ex}%[DCHT Toán 11 - KNTT -Phong Lê] %[1K1B2-1]
	Cho $a,b \in \left[0;\dfrac{\pi}{2}\right]$ thỏa mãn $\sin a = \cos b = \dfrac{3}{5}$. Khi đó $\sin (a+b)$ bằng
	\choice{$\dfrac{24}{25}$}{\True $1$}{$0$}{$-\dfrac{7}{25}$}
	\loigiai{Ta có $a,b \in \left[0;\dfrac{\pi}{2}\right]$, suy ra $\cos a>0$ và $\sin b >0$.\\
		Lại có $\sin^2a + \cos^2a=\sin^2b+\cos^2b=1$, suy ra $\cos a= \sqrt{1-\left(\dfrac{3}{5}\right)^2}=\dfrac{4}{5}$ và $\sin b = \sqrt{1-\left(\dfrac{3}{5}\right)^2}=\dfrac{4}{5}$.\\
		Suy ra $\sin (a+b) = \sin a \cos b + \cos a \sin b = \dfrac{3}{5}.\dfrac{3}{5} + \dfrac{4}{5}.\dfrac{4}{5} = 1$. 
		
	}
\end{ex}
\begin{ex}%[DCHT Toán 11 - KNTT -Phong Lê] %[1K1K2-1]
	Biểu thức $P = \cos 5^{\circ}\sin 70^{\circ}-\sin175^{\circ}\sin20^{\circ}$ có giá trị bằng với
	\choice{$\sin 25^{\circ}$}{\True $\cos 25^{\circ}$}{$\sin 15^{\circ}$}{$\cos 15^{\circ}$}
	\loigiai{Ta có $P = \cos 5^{\circ}\sin 70^{\circ}-\sin175^{\circ}\sin20^{\circ} = \cos 5^{\circ}\cos 20^{\circ}-\sin5^{\circ}\sin20^{\circ}= \cos \left(5^{\circ}+20^{\circ}\right) = \cos 25^{\circ}$.}
\end{ex}
\begin{ex}%[DCHT Toán 11 - KNTT -Phong Lê] %[1K1K2-1]
	Cho $\alpha + \beta = \dfrac{\pi}{3}$ và $\sin\alpha\cos\beta = \dfrac{1+\sqrt{3}}{4}$. Giá trị của $\sin (\alpha-\beta)$ bằng
	\choice{$\dfrac{\sqrt{3}}{2}$}{$-\dfrac{3}{2}$}{$-\dfrac{1}{2}$}{\True $\dfrac{1}{2}$}
	\loigiai{Ta có $\sin(\alpha+\beta) = \sin \alpha\cos\beta + \cos\alpha\sin \beta$\\
		suy ra $\cos \alpha \sin \beta = \sin(\alpha+\beta) - \sin \alpha\cos\beta  = \sin \dfrac{\pi}{3}-\dfrac{1+\sqrt{3}}{4} = \dfrac{\sqrt{3}}{2}-\dfrac{1+\sqrt{3}}{4} = =\dfrac{\sqrt{3}-1}{4}$.\\
		Do đó $\sin (\alpha-\beta) = \sin \alpha \cos \beta - \cos \alpha \sin \beta = \dfrac{1+\sqrt{3}}{4}-\dfrac{\sqrt{3}-1}{4} = \dfrac{1}{2} $.
	}
\end{ex}
\begin{ex}%[DCHT Toán 11 - KNTT -Phong Lê] %[1K1K2-1]
	Cho tam giác $ABC$ cân tại $A$ có $\cos B = \dfrac{5}{13}$. Tính $\sin A$.
	\choice{$\dfrac{119}{169}$}{$1$}{\True $\dfrac{120}{169}$}{$\dfrac{5}{13}$}
	\loigiai{
		Vì $0<\widehat{B}<\pi$ nên $\sin B >0$. Ta có $\sin^2B+\cos^2B = 1$, suy ra $\sin B = \sqrt{1-\left(\dfrac{5}{13}\right)^2} = \dfrac{12}{13}$.\\
		Vì tam giác $ABC$ cân tại $A$ nên $\widehat{B} = \widehat{C}$, suy ra $\sin C = \dfrac{12}{13}$ và $\cos C=\dfrac{5}{13}$.\\
		Theo tính chất tổng ba góc trong một tam giác, ta có $\widehat{A} + \widehat{B} + \widehat{C} = \pi$.\\
		Do đó
		$\sin A = \sin (\pi - A) =\sin (B+C) = \sin B\cos C + \cos B \sin C = \dfrac{12}{13}.\dfrac{5}{13} + \dfrac{5}{13}.\dfrac{12}{13} = \dfrac{120}{169}$.
	}
\end{ex}
\begin{ex}%[DCHT Toán 11 - KNTT -Phong Lê] %[1K1K2-1]
	Cho $a,b$ thỏa mãn $\sin a = \dfrac{\sqrt{7}}{4}$ và $\sin b = \dfrac{\sqrt{3}}{4}$. Giá trị của $\sin (a+b)\sin(a-b)$ là:
	\choice{\True $\dfrac{1}{4}$}{$-\dfrac{1}{4}$}{$\dfrac{1}{2}$}{$-\dfrac{1}{2}$}
	\loigiai{Ta có $\sin (a+b)\sin(a-b) = (\sin a \cos b + \cos a \sin b)(\sin a \cos b - \cos a \sin b) = \sin^2 a\cos^2b - \cos^2a \sin^2b$.\\
		Lại có $\sin^2 a\cos^2b - \cos^2a \sin^2b = \sin^2a\left(1-\sin^2b\right)-\cos^2a\sin^2b = \sin^2a-\sin^2b(\sin^2a+\sin^2b)\\
		= \sin^2a-\sin^2b = \left(\dfrac{\sqrt{7}}{4}\right)^2 - \left(\dfrac{\sqrt{3}}{4}\right)^2 = \dfrac{1}{4}$.\\
		Suy ra $\sin (a+b)\sin(a-b) = \dfrac{1}{4}$.
	}
\end{ex}
\begin{ex}%[DCHT Toán 11 - KNTT -Phong Lê] %[1K1G2-1]
	Cho $\alpha \in \left[-\dfrac{2\pi}{3},-\dfrac{\pi}{6}\right]$ thỏa mãn $\sin \left(\alpha+\dfrac{\pi}{6}\right) = \dfrac{\sqrt{6}-\sqrt{2}}{4}$. Giá trị của $\tan \alpha$ là
	\choice{$2-\sqrt{3}$}{$2+\sqrt{3}$}{\True $-2+\sqrt{3}$}{$\sqrt{3}$}
	\loigiai{
		Ta có $-\dfrac{2\pi}{3}\le \alpha \le -\dfrac{\pi}{6}$, suy ra $-\dfrac{\pi}{2} \le \alpha \le 0$, do đó $\cos \left(\alpha + \dfrac{\pi}{6}\right) \ge 0$.\\
		Lại có $\sin^2\left(\alpha + \dfrac{\pi}{6}\right) + \cos^2 \left(\alpha + \dfrac{\pi}{6}\right) = 1$, suy ra $$\cos \left(\alpha + \dfrac{\pi}{6}\right) = \sqrt{1-\left(\dfrac{\sqrt{6}-\sqrt{2}}{4}\right)^2} =\sqrt{1-\dfrac{\sqrt{8-2\sqrt{12}}}{16} }= \sqrt{\dfrac{8+2\sqrt{12}}{16}} = \sqrt{\left(\dfrac{\sqrt{6}+\sqrt{2}}{4}\right)^2} = \dfrac{\sqrt{6}+\sqrt{2}}{4}.$$
		Suy ra $\tan \left(\alpha + \dfrac{\pi}{6}\right) = \dfrac{\sin \left(\alpha + \dfrac{\pi}{6}\right)}{\cos \left(\alpha + \dfrac{\pi}{6}\right)} = \dfrac{\sqrt{6}-\sqrt{2}}{4}:\dfrac{\sqrt{6}+\sqrt{2}}{4} =2-\sqrt{3}$.\\
		Do đó $\tan \alpha = \tan \left(\alpha+\dfrac{\pi}{6}-\dfrac{\pi}{6}\right) = \dfrac{\tan \left(\alpha + \dfrac{\pi}{6}\right)-\tan \dfrac{\pi}{6}}{1+\tan \left(\alpha + \dfrac{\pi}{6}\right)\tan \dfrac{\pi}{6}} = \dfrac{2-\sqrt{3}-\dfrac{\sqrt{3}}{3}}{1+(2-\sqrt{3}).\dfrac{\sqrt{3}}{3}} =\dfrac{6-4\sqrt{3}}{2\sqrt{3}}= -2+\sqrt{3}$.
	}
\end{ex}
\begin{ex}%[DCHT Toán 11 - KNTT -Phong Lê] %[1K1G2-1]
	Một vật thực hiện đồng thời hai dao động điều hòa có phương trình $x_1(t) = \sin \left(\pi t+\dfrac{\pi}{3}\right)$ và $x_2(t) = \cos \left(\pi t+\dfrac{\pi}{3}\right)$. Phương trình dao động tổng hợp của vật $x(t) = x_1(t)+x_2(t)$ được viết dưới dạng $x(t) = A\cos (\omega t + \varphi)$, tức là dao động tổng hợp của vật đó là dao động điều hòa. Hãy xác định pha ban đầu $\varphi$ ($-\pi<\varphi<\pi$) của dao động tổng hợp.
	\choice{$\dfrac{\pi}{4}$}{\True $\dfrac{\pi}{12}$}{$\dfrac{5\pi}{12}$}{$-\dfrac{\pi}{4}$}
	\loigiai{Ta có $x(t) = x_1(t)+x_2(t) = \sin \left(\pi t + \dfrac{\pi}{3}\right) + \cos \left(\pi t + \dfrac{\pi}{3}\right)$.\\
		Lại có $\sin \left(\pi t + \dfrac{\pi}{3}\right) + \cos \left(\pi t + \dfrac{\pi}{3}\right) = \sqrt{2}\left[\dfrac{\sqrt{2}}{2}\cos \left(\pi t + \dfrac{\pi}{3}\right) + \dfrac{\sqrt{2}}{2}\sin \left(\pi t + \dfrac{\pi}{3}\right)\right]$\\
		$= \sqrt{2}\left[\cos \dfrac{\pi}{4}\cos \left(\pi t + \dfrac{\pi}{3}\right) + \sin \dfrac{\pi}{4}\sin \left(\pi t + \dfrac{\pi}{3}\right)\right] = \sqrt{2}\cos \left(\pi t + \dfrac{\pi}{3} -\dfrac{\pi}{4}\right) = \sqrt{2}\cos\left(\pi t + \dfrac{\pi}{12}\right)$.\\
		Suy ra $x(t) = \sqrt{2}\cos \left(\pi t + \dfrac{\pi}{12}\right)$. Vậy pha ban đầu của dao động tổng hợp là $\dfrac{\pi}{12}$.
	}
\end{ex}
\Closesolutionfile{ans}
\begin{indapan}{10}
	{ans/ans-1K1-2-Dang1}
\end{indapan}