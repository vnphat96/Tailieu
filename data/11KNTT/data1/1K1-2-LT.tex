%\chapter{Hàm số  lượng giác và phương trình lượng giác}
\setcounter{section}{1}
\section{Công thức lượng giác}
\subsection{Tóm tắt lý thuyết}
\begin{tomtat}
% 	\begin{center}
% 		\begin{tikzpicture}[scale = 2.5]
% 			\path (0,0) coordinate (O) (1.5,0) coordinate (x) (0,1.5) coordinate (y);
% 			\draw[thick,->] (-1.5,0)--(x);
% 			\draw[thick,->] (0,-1.5)--(y);
% 			\draw (O) circle (1);
% 			\path ($(O)+(55:1)$) coordinate (M) 
% 			($(O)+(30:1)$) coordinate (N);
% 			\path ($(O)!(M)!(x)$) coordinate (x_M)
% 			($(O)!(M)!(y)$) coordinate (y_M)
% 			($(O)!(N)!(x)$) coordinate (x_N)
% 			($(O)!(N)!(y)$) coordinate (y_N);
% 			\draw[dashed] (x_M)--(M)--(y_M) (x_N)--(N)--(y_N);
% 			\foreach \x/\g in {O/-135,x/-90,y/180,x_M/-90,x_N/-90,y_M/180,y_N/180}
% 			\fill ($(\g:1mm)+(\x)$) node {$\x$};
% 			\fill 	(M) circle (0.5pt)
% 			($(15:4mm)+(M)$) node {$M\left(x_M,y_M\right)$};
% 			\fill (N) circle (0.5pt)
% 			($(15:4mm)+(N)$) node {$N\left(x_N,y_N\right)$};
% 	\draw (M)--(O)--(N);		
% 	\draw pic[draw,,angle radius=6mm,->,red]{angle=x--O--M};
% 	\fill[red] (45:3mm) node {$\alpha$};
% 	\draw pic[red,draw,,angle radius=10mm,->]{angle=x--O--N};
% 	\fill[red] (15:5mm) node {$\beta$};
% 		\end{tikzpicture}
% 	\end{center}
% Trong mặt phẳng $Oxy$ cho hai điểm $M,N$ trên đường tròn lượng giác.\\ Đặt $\alpha = \text{sđ} (Ox,OM), \beta = \text{sđ} (Ox,ON)$, ta có $M(\cos \alpha,\sin \alpha)$ và $N(\cos \beta, \sin \beta)$. Khi đó ta tính được $\overrightarrow{OM}.\overrightarrow{ON}$ bằng hai cách
% \begin{align*}
% 	\overrightarrow{OM}.\overrightarrow{ON}&=\left|\overrightarrow{OM}\right|.\left|\overrightarrow{ON}\right|.\cos \left(\overrightarrow{OM},\overrightarrow{ON}\right) = \cos (\alpha-\beta),\\
% 	\overrightarrow{OM}.\overrightarrow{ON} &= x_Mx_N+y_My_N= \cos \alpha \cos\beta +\sin\alpha\sin\beta.
% \end{align*}
% Từ đó dẫn tới công thức
% \begin{align*}
% 	\cos (\alpha-\beta) = \cos \alpha \cos \beta + \sin \alpha\sin \beta \tag{$\star$}
% \end{align*}
% Tất cả các công thức trong bài học được xây dựng dựa trên công thức $(\star)$.\\
% Trong suốt bài học, khi không nói gì thêm, chỉ xét các góc lượng giác mà trong đó giá trị lượng giác được để cập có nghĩa. 
	\subsubsection{Công thức cộng}
	\begin{khung4}{Công thức cộng}
	\begin{tasks}[style=itemize](2)
		\task $\cos (a-b) = \cos a \cos b + \sin a\sin b$.
		\task $\cos (a+b) = \cos a \cos b - \sin a\sin b$.
		\task $\sin (a-b) = \sin a \cos b - \sin b \cos a$.
		\task $\sin (a+b) = \sin a \cos b + \sin b \cos a$.
		\task $\tan (a-b) = \dfrac{\tan a - \tan b}{1+\tan a \tan b}$.
		\task $\tan (a+b) = \dfrac{\tan a + \tan b}{1-\tan a \tan b}$.
	\end{tasks}
	\end{khung4}
	\subsubsection{Công thức nhân đôi}
	Công thức nhân đôi được xây dựng bằng cách thay $b=a$ trong công thức cộng.
	\begin{khung4}{Công thức nhân đôi}
		\begin{tasks}[style=itemize]
			\task $\sin 2a = 2\sin a \cos a$.
			\task $\cos 2a = \cos^2a-\sin^2a = 2\cos^2a-1 = 1-2\sin^2a$.
			\task $\tan 2a = \dfrac{2\tan a}{1-\tan^2a}$.
		\end{tasks}
		\end{khung4}
\begin{note}
	Từ công thức nhân đôi, ta có công thức hạ bậc:
\end{note}
	\begin{khung4}{Công thức hạ bậc}
		\begin{tasks}[style=itemize](3)
			\task $\sin^2a= \dfrac{1-\cos 2a}{2}$.
		\task $\cos^2a = \dfrac{1+\cos 2a}{2}$.
		\task $\tan^2a=\dfrac{1-\cos2a}{1+\cos 2a}$.
		\end{tasks}
		\end{khung4}

\begin{note}
	Áp dụng công thức cộng cho $3a = a +2a$, ta có công thức nhân ba:
\end{note}
	\begin{khung4}{Công thức nhân ba}
		\begin{tasks}[style=itemize](2)
			\task $\sin3a= 3\sin a -4\sin^3a$.
		\task $\cos3a= 4\cos^3a-3\cos a$.
		\task $\tan3a = \dfrac{3\tan a - \tan^3 a}{1-3\tan^2a}$.
		\end{tasks}
		\end{khung4}

	\subsubsection{Công thức biến đổi tích thành tổng}
	\begin{khung4}{Công thức tích thành tổng}
		\begin{tasks}[style=itemize]
			\task $\cos a \cos b = \dfrac{1}{2}\left[\cos (a-b) + \cos (a+b)\right]$.
		\task $\sin a \sin b = \dfrac{1}{2}\left[\cos (a-b)-\cos(a+b)\right]$.
		\task $\sin a \cos b = \dfrac{1}{2}\left[\sin (a-b)+\sin (a+b)\right]$.
		\end{tasks}
		\end{khung4}
	\subsubsection{Công thức biến đổi tổng thành tích}
	Công thức biến đổi tổng thành tích được xây dựng bằng cách $a=\dfrac{a+b}{2}, b = \dfrac{a-b}{2}$ trong công thức biến đổi tích thành tổng.
	\begin{khung4}{Công thức tổng thành tích}
		\begin{tasks}[style=itemize](2)
			\task $\cos a+ \cos b = 2\cos\dfrac{a+b}{2}\cos \dfrac{a-b}{2}$.
		\task $\cos a- \cos b = -2\sin\dfrac{a+b}{2}\sin \dfrac{a-b}{2}$.
		\task $\sin a+ \sin b = 2\sin\dfrac{a+b}{2}\cos \dfrac{a-b}{2}$.
		\task $\sin a -\sin b = 2\cos\dfrac{a+b}{2}\sin \dfrac{a-b}{2}$.
		\end{tasks}
		\end{khung4}
\end{tomtat}

