
\begin{dang}{Phương trình lượng giác cơ bản dùng độ}
		\begin{itemize}
			\item $\sin x=\sin\alpha^\circ\Leftrightarrow \left[\begin{aligned}
			&x=\alpha^\circ+k360^\circ\\
			&x=180^\circ-\alpha^\circ+k360^\circ\\
			\end{aligned}\right.\left(k\in\mathbb{Z}\right).$
			\item $\cos x=\cos\alpha^\circ\Leftrightarrow \left[\begin{aligned}
			&x=\alpha^\circ+k360^\circ\\
			&x=-\alpha^\circ+k360^\circ\\
			\end{aligned}\right.\left(k\in\mathbb{Z}\right).$
			\item $\tan x=\tan\alpha^\circ\Leftrightarrow x=\alpha^\circ+k180^\circ\quad\left(k\in\mathbb{Z}\right)$.
			\item $\cot x=\cot\alpha^\circ\Leftrightarrow x=\alpha^\circ+k180^\circ\quad\left(k\in\mathbb{Z}\right)$.
		\end{itemize}
\end{dang}
\subsubsection{Ví dụ}
%%=====Ví dụ 1
\begin{vd}%[DCHT Toán 11 - KNTT -Thọ Bùi] %[1K1Y4-4]
Tìm góc lượng giác $x$ sao cho:
\begin{multicols}{2}
	\begin{enumerate}
		\item $\sin x=\sin 55^\circ$;
		\item  $\cos x=\cos (-87^\circ)$;
		\item $\tan x=\tan 67^\circ$;
		\item $\cot x=\cot (-83^\circ)$.
	\end{enumerate}
\end{multicols}

 
\loigiai{
\begin{listEX}[1]
	\item $\sin x=\sin 55^\circ \Leftrightarrow \hoac{
		&x=55^\circ+k360^\circ\\
		&x=180^\circ -55^\circ+k360^\circ
	} \Leftrightarrow
	\hoac{
		&x=55^\circ+k360^\circ\\
		&x=125^\circ+k360^\circ
	} (k\in \mathbb{Z}).$
	\item $\cos x=\cos (-87^\circ)
	\Leftrightarrow \hoac{
		&x=-87^\circ+k360^\circ\\
		&x=-(-87^\circ)+k360^\circ
	} \Leftrightarrow
	\hoac{
		&x=-87^\circ+k360^\circ\\
		&x=87^\circ+k360^\circ\\
	} (k\in \mathbb{Z}).$
	\item $\tan x=\tan 67^\circ \Leftrightarrow x=67^\circ +k180^\circ, k\in \mathbb{Z}$.
	\item $\cot x=\cot (-83^\circ) \Leftrightarrow x=-83^\circ+k180^\circ\,\, (k\in \mathbb{Z})$.
\end{listEX}

}
\end{vd}


\begin{vd}%[DCHT Toán 11 - KNTT -Thọ Bùi] %[1K1B4-4]
	Giải các phương trình sau:
	\begin{multicols}{2}
		\begin{enumerate}
		\item $\sin\left(x+20^\circ\right)=\dfrac{1}{2}$;
		\item $\sin\left(x+30^\circ\right)=\sin\left(x+60^\circ\right)$.
	\end{enumerate}
\end{multicols}
	\loigiai{\begin{listEX}[1]
			\item Ta có:
			{\allowdisplaybreaks
				\begin{eqnarray*}
					\sin\left(x+20^\circ\right)=\dfrac{1}{2}&\Leftrightarrow &\sin\left(x+20^\circ\right)=\sin 30^\circ\\
					&\Leftrightarrow& \left[\begin{aligned}
						&x+20^\circ=30^\circ+k360^\circ\\
						&x+20^\circ=180^\circ-30^\circ+k360^\circ\\
					\end{aligned}\right.\\
					&\Leftrightarrow&\left[\begin{aligned}
						&x=10^\circ+k360^\circ\\
						&x=130^\circ+k360^\circ\\
					\end{aligned}\right. \left(k\in\mathbb{Z}\right).
\end{eqnarray*}}
\item Ta có:
{\allowdisplaybreaks
	\begin{eqnarray*}
		\sin\left(x+30^\circ\right)=\sin\left(x+60^\circ\right)
		&\Leftrightarrow& \left[\begin{aligned}
			&x+30^\circ=x+60^\circ+k360^\circ\\
			&x+30^\circ=180^\circ-\left(x+60^\circ\right)+k360^\circ\\
		\end{aligned}\right.\\
		&\Leftrightarrow&\left[\begin{aligned}
			&-30^\circ=k360^\circ\;(\text{vô nghiệm})\\
			&2x=90^\circ+k360^\circ\\
		\end{aligned}\right.\\
		&\Leftrightarrow& x=45^\circ+k180^\circ \left(k\in\mathbb{Z}\right).
\end{eqnarray*}}
	\end{listEX}}
\end{vd}
\begin{vd}%[DCHT Toán 11 - KNTT -Thọ Bùi] %[1K1B4-4]
	Giải phương trình $\sin 2x  = \sin (60^{\circ} - 3x)$.
	\loigiai{
		Ta có 
		\begin{align*}
		\sin 2x = \sin (60^{\circ}-3x) & \Leftrightarrow \hoac{& 2x = 60^{\circ}-3x + k360^{\circ} \\ & 2x = 180^{\circ} - (60^{\circ}-3x) + k360^{\circ}}
		\Leftrightarrow \hoac{& 5x = 60^{\circ} + k360^{\circ} \\ & -x = 120^{\circ} + k360^{\circ}} \\
		&\Leftrightarrow \hoac{& x = 12^{\circ} + k72^{\circ} \\ & x = -120^{\circ} - k360^{\circ}} \quad (k \in \mathbb{Z}).
		\end{align*}
	}
\end{vd}
\begin{vd}%[DCHT Toán 11 - KNTT -Thọ Bùi] %[1K1B4-4]
	Giải phương trình $\cos 2 x=\cos \left(45^{\circ}-x\right)$.
	\loigiai{
		$$
		\begin{aligned}
		\cos 2 x=\cos \left(45^{\circ}-x\right) & \Leftrightarrow\left[\begin{array}{l}
		2 x=45^{\circ}-x+k 360^{\circ} \\
		2 x=-\left(45^{\circ}-x\right)+k 360^{\circ}
		\end{array}\right. \\
		& \Leftrightarrow\left[\begin{array} { l } 
		{ 3 x = 4 5 ^ { \circ } + k 3 6 0 ^ { \circ }} \\
		{ x = - 4 5 ^ { \circ } + k 3 6 0 ^ { \circ }}
		\end{array} \Leftrightarrow \left[\begin{array}{l}
		x=15^{\circ}+k 120^{\circ} \\
		x=-45^{\circ}+k 360^{\circ}
		\end{array}(k \in \mathbb{Z}) .\right.\right.
		\end{aligned}
		$$
	}
	
\end{vd}
%%=====Ví dụ 1
\begin{vd}%[DCHT Toán 11 - KNTT -Thọ Bùi] %[1K1B4-4]
Giải phương trình: $\sqrt3\tan\left(\dfrac{x}{2}+15^\circ \right)=1$.
\loigiai{
	Ta có:
	{\allowdisplaybreaks
		\begin{eqnarray*}
			\sqrt3\tan\left(\dfrac{x}{2}+15^\circ \right)=1
			&\Leftrightarrow& \tan \left(\dfrac{x}{2}+15^\circ \right)=\dfrac{1}{\sqrt3}\\
			&\Leftrightarrow&\tan \left(\dfrac{x}{2}+15^\circ \right)=\tan 30^\circ\\
			&\Leftrightarrow& \dfrac{x}{2}+15^\circ=30^\circ+k180^{\circ}\\
			&\Leftrightarrow& x=30^\circ+k360^{\circ}, k \in \mathbb{Z}.
	\end{eqnarray*}}
}
\end{vd}


\subsubsection{Bài tập tự luận}
\begin{bt}%[DCHT Toán 11 - KNTT -Thọ Bùi] %[1K1B4-4]
	Giải phương trình $\cos\left(x-15^\circ\right)=-\dfrac{1}{2}$.
	\loigiai{
		Ta có: $\cos\left(x-15^\circ\right)=-\dfrac{1}{2}=\cos 120^\circ\Leftrightarrow\hoac{&x-15^\circ=120^\circ+k360^\circ\\&x-15^\circ=-120^\circ+k360^\circ}\Leftrightarrow\hoac{&x=135^\circ+k360^\circ\\&x=-105^\circ+k360^\circ} \left(k \in \mathbb{Z}\right)$.
	}
\end{bt}
%%=====Bài 1
\begin{bt}%[DCHT Toán 11 - KNTT -Thọ Bùi] %[1K1B4-4]
Giải phương trình: $\cos\left(2x-60^\circ\right)=\dfrac{1}{3}$.
\loigiai{
Vì $\dfrac{1}{3} \in[-1; 1]$ nên tồn tại $\cos a^\circ=\dfrac{1}{3}$.\\
Khi đó ta có: 
$$\cos \left(2 x-60^\circ\right)=\dfrac{1}{3} \Leftrightarrow \cos \left(2 x-60^\circ\right)=\cos a^\circ\Leftrightarrow 2 x-60^\circ= \pm a^\circ+k 360^\circ\Leftrightarrow x=30^\circ \pm \dfrac{a^\circ}{2}+k 180^\circ, k \in \mathbb{Z}.$$
}
\end{bt}

%%=====Bài 2
\begin{bt}%[DCHT Toán 11 - KNTT -Thọ Bùi] %[1K1K4-4]
Giải phương trình $\tan \left(x+30^\circ\right)+1=0$ với $-90^\circ<x<360^\circ$.
\loigiai{
Ta có: 
{\allowdisplaybreaks
	\begin{eqnarray*}
		\tan \left(x+30^\circ\right)+1=0&\Leftrightarrow &\tan \left(x+30^\circ\right)=-1=\tan\left(-45^\circ\right)\\
		&\Leftrightarrow& x+30^\circ=-45^\circ+\mathrm{k} 180^\circ\\
		&\Leftrightarrow&x=-75^\circ+k 180^\circ \left(k\in\mathbb{Z}\right).
\end{eqnarray*}}
Do $-90^\circ<x<360^\circ$ nên ta có tập nghiệm của phương trình là $S=\left\{-75^\circ; 105^\circ; 285^\circ\right\}$. 
}
\end{bt}

%%=====Bài 3
\begin{bt}%[DCHT Toán 11 - KNTT -Thọ Bùi] %[1K1K4-4]
Giải phương trình $3\cot^2\left(5x+40^\circ\right)=1$.
\loigiai{
Ta có: $3\cot^2\left(5x+40^\circ\right)=1\Leftrightarrow\cot\left(5x+40^\circ\right)=\pm\dfrac{\sqrt{3}}{3}$.
\begin{itemize}
	\item $\cot\left(5x+40^\circ\right)=\dfrac{\sqrt{3}}{3}=\cot 60^\circ\Leftrightarrow 5x+40^\circ=60^\circ+k180^\circ\Leftrightarrow x=4^\circ+k36^\circ, k\in\mathbb{Z}$.
	\item $\cot\left(5x+40^\circ\right)=-\dfrac{\sqrt{3}}{3}=\cot \left(-60^\circ\right)\Leftrightarrow 5x+40^\circ=-60^\circ+k180^\circ\Leftrightarrow x=-20^\circ+k36^\circ, k\in\mathbb{Z}$.
\end{itemize}
}
\end{bt}

%%=====Bài 5
\begin{bt}%[DCHT Toán 11 - KNTT -Thọ Bùi] %[1K1K4-4]
Giải phương trình: $\tan\left(3x-20^\circ\right)-\cot\left(2x+15^\circ\right)=0$.
\loigiai{
Ta có: 
{\allowdisplaybreaks
	\begin{eqnarray*}
		\tan\left(3x-20^\circ\right)-\cot\left(2x+15^\circ\right)=0&\Leftrightarrow &\tan\left(3x-20^\circ\right)=\cot\left(2x+15^\circ\right)\\
		&\Leftrightarrow& \tan\left(3x-20^\circ\right)=\tan\left(90^\circ-2x-15^\circ\right)\\
		&\Leftrightarrow&\tan\left(3x-20^\circ\right)=\tan\left(75^\circ-2x\right)\\
		&\Leftrightarrow&3x-20^\circ=75^\circ-2x+k180^\circ\\
		&\Leftrightarrow&x=19^\circ+k36^\circ, k\in\mathbb{Z}.
\end{eqnarray*}}
}
\end{bt}
%%=====Bài 1
\begin{bt}%[DCHT Toán 11 - KNTT -Thọ Bùi] %[1K1K4-4]
Giải phương trình: $\cot\left(x+30^\circ\right)=\cot\dfrac{x}{2}$ 
\loigiai{
Điều kiện: $ \left\{\begin{aligned}
&\sin\left(x+30^\circ\right)\neq 0\\
&\sin\dfrac{x}{2}\neq 0\\
\end{aligned}\right.\Leftrightarrow \left\{\begin{aligned}
&x+30^{\circ} \neq k \cdot 180^{\circ}\\
&\dfrac{x}{2} \neq n \cdot 180^{\circ}\\
\end{aligned}\right.\Leftrightarrow \left\{\begin{aligned}
&x\neq -30^\circ+k \cdot 180^{\circ}\\
&x\neq n\cdot 360^\circ\\
\end{aligned}\right. \left(k, n\in\mathbf{Z}\right)$.\\
Khi đó: 
$$\cot\left(x+30^\circ\right)=\cot\dfrac{x}{2} \Leftrightarrow x+30^\circ=\dfrac{x}{2}+m \cdot 180^\circ \Leftrightarrow 2 x+60^\circ=x+m \cdot 360^\circ \Leftrightarrow x=-60^\circ+m \cdot 360^\circ, m \in \mathbb{Z}.
$$
Vậy nghiệm của phương trình là $x=-60^\circ+m \cdot 360^\circ, m \in \mathbb{Z}$.
}
\end{bt}


\subsubsection{Bài tập trắc nghiệm}
\Opensolutionfile{ans}[ans/ans-1K1-3-Dang4]
%1
\begin{ex}%[DCHT Toán 11 - KNTT -Thọ Bùi] %[1K1Y4-4]
	Phương trình $\sin x= \sin a^\circ$ tương đương với
	\choice{$\hoac{& x=a^\circ+k 360^\circ \\ & x=-a^\circ+k60^\circ} (k \in \mathbb{Z})$}
	{\True $\hoac{& x=a^\circ+k 60^\circ \\ & x=180^\circ -a^\circ+60^\circ} (k \in \mathbb{Z})$}
	{$x=a^\circ+k180^\circ$ $ (k \in \mathbb{Z})$}
	{$x=-a^\circ+k180^\circ $ $(k \in \mathbb{Z})$}
	\loigiai{Ta có $\sin x=\sin a^\circ\Leftrightarrow \hoac{& x=a^\circ+k 360^\circ \\ & x=180^\circ -a^\circ+k360^\circ} (k \in \mathbb{Z}).$}
\end{ex}

\begin{ex}%[DCHT Toán 11 - KNTT -Thọ Bùi] %[1K1Y4-4]
	Hỏi $x=45^\circ$ là nghiệm của phương trình nào sau đây?
	\choice
	{$\sin x=1$}
	{$\cos x=1$}
	{\True $\sin x\cdot\cos x=\dfrac{1}{2}$}
	{$\sin 2x=0$}
	\loigiai{Ta có $\sin x\cdot\cos x=\dfrac{1}{2}\Leftrightarrow \sin 2x=1\Leftrightarrow 2x=90^\circ+k360^\circ\Leftrightarrow x=45^\circ+k180^\circ$.\\
	Do đó $x=45^\circ$ là nghiệm của phương trình $\sin x\cdot\cos x=\dfrac{1}{2}$.}

\end{ex}
\begin{ex}%[DCHT Toán 11 - KNTT -Thọ Bùi] %[1K1Y4-4]
	Tìm tập nghiệm $S$ của phương trình $\cos 3x = \cos 45^\circ$.
	\choice
	{$S=\{15^\circ + k120^\circ; 45^\circ + k120^\circ, k\in \mathbb{Z}\}$}
	{\True $S=\{-15^\circ + k120^\circ; 15^\circ + k120^\circ, k\in \mathbb{Z}\}$}
	{$S=\{15^\circ + k360^\circ; 45^\circ + k360^\circ, k\in \mathbb{Z}\}$}
	{$S=\{-15^\circ + k360^\circ; 15^\circ + k360^\circ, k\in \mathbb{Z}\}$}
	\loigiai{
		$\cos 3x=\cos 45^\circ \Leftrightarrow \hoac{&3x=45^\circ+k360^\circ\\& 3x=-45^\circ+k360^\circ}\Leftrightarrow \hoac{&x=15^\circ + k120^\circ\\&x=-15^\circ + k120^\circ}\, (k\in \mathbb{Z}).$
	}
\end{ex}
\begin{ex}%[DCHT Toán 11 - KNTT -Thọ Bùi] %[1K1B4-4]
	Tìm tập nghiệm $S$ của phương trình $\cos \left(2x-30^\circ \right)=-\dfrac{1}{2}$.
	\choice
	{$S=\{-45^\circ + k360^\circ; 75^\circ + k360^\circ, k\in \mathbb{Z}\}$}
	{$S=\{-45^\circ + k180^\circ; 45^\circ + k180^\circ, k\in \mathbb{Z}\}$}
	{\True $S=\{-45^\circ + k180^\circ; 75^\circ + k180^\circ, k\in \mathbb{Z}\}$}
	{$S=\{-75^\circ + k180^\circ; 75^\circ + k180^\circ, k\in \mathbb{Z}\}$}
	\loigiai{
		Ta có 
		$$\cos \left(2x-30^\circ \right)=-\dfrac{1}{2}\Leftrightarrow \hoac{&2x-30^\circ=120^\circ+k360^\circ\\& 2x-30^\circ=-120^\circ+k360^\circ}\Leftrightarrow \hoac{&x=75^\circ+k180^\circ\\& x=-45^\circ+k180^\circ}\, (k\in \mathbb{Z}).$$	
	}
\end{ex}
%%=====Câu 1
\begin{ex}%[DCHT Toán 11 - KNTT -Thọ Bùi] %[1K1Y4-4]
Nghiệm của phương trình $\tan x=\tan 25^\circ$ là 
\choice
{$x=25^\circ+\mathrm{k} 360^\circ$ và $x=155^\circ+\mathrm{k} 360^\circ, \mathrm{k} \in \mathbb{Z}$ }
{$x=25^\circ+k 180^\circ$ và $x=155^\circ+\mathrm{k} 180^\circ, \mathrm{k} \in \mathbb{Z}$ }
{$x=25^\circ+k 360^\circ$ và $x=-25^\circ+\mathrm{k} 360^\circ, \mathrm{k} \in \mathbb{Z}$ }
{\True $x=25^\circ+\mathrm{k} 180^\circ, \mathrm{k} \in \mathbb{Z}$ }
\loigiai{
Ta có: 
$$\tan x=\tan 25^\circ\Leftrightarrow x=25^\circ+\mathrm{k} 180^\circ, \mathrm{k} \in \mathbb{Z}.$$
}
\end{ex}


\begin{ex}%[DCHT Toán 11 - KNTT -Thọ Bùi] %[1K1Y4-4]
	Phương trình $\tan \left( 2x+12^\circ \right)=0$ có họ nghiệm là
	\choice
	{$x=-6^\circ+k180^\circ$, $\ k\in \mathbb{Z}$}
	{$x=-6^\circ+k360^\circ$, $\ k\in \mathbb{Z}$}
	{$x=-12^\circ+k90^\circ$, $\ k\in \mathbb{Z}$}
	{\True $x=-6^\circ+k90^\circ$, $\ k\in \mathbb{Z}$}
	\loigiai{
		Ta có 
		\begin{eqnarray*}
			&  & \tan \left( 2x+12^\circ \right)=0 \\
			& \Leftrightarrow  &x=- 6^{\circ}+k90^{\circ}, k\in \mathbb{Z}.
		\end{eqnarray*}
	}
\end{ex}

\begin{ex}%[DCHT Toán 11 - KNTT -Thọ Bùi] %[1K1B4-4]
	Tìm số nghiệm của phương trình $ \sin 3x=0$ thuộc khoảng $\left(0; 180^\circ\right)$.
	\choice{$1 $}
	{\True $2 $}
	{$3 $}
	{$4 $}
	\loigiai{
		Ta có: $\sin 3x=0\Leftrightarrow x = \dfrac{k180^\circ}{3}.$\\ Xét bất phương trình $0<  \dfrac{k180^\circ}{3}< 180^\circ \Leftrightarrow k \in \left\{1;2\right\}.$\\
		Vậy phương trình có $2$ nghiệm trong $\left(0;180^\circ\right)$.}
\end{ex}
\begin{ex}%[DCHT Toán 11 - KNTT -Thọ Bùi] %[1K1B4-4]
	Tìm tập nghiệm $S$ của phương trình $\cos (x+30^\circ) = -\dfrac{\sqrt{3}}{2}$.
	\choice
	{$S=\{120^\circ + k360^\circ; k360^\circ, k\in \mathbb{Z}\}$}
	{\True $S=\{120^\circ + k360^\circ; -180^\circ+k360^\circ, k\in \mathbb{Z}\}$}
	{$S=\{120^\circ + k180^\circ; k180^\circ, k\in \mathbb{Z}\}$}
	{$S=\{120^\circ + k180^\circ; -180^\circ+k180^\circ, k\in \mathbb{Z}\}$}
	\loigiai{Ta có  $\cos (x+30^\circ) = -\dfrac{\sqrt{3}}{2}\Leftrightarrow \hoac{&x+30^\circ=-150^\circ+k 360^\circ\\&x+30^\circ=150^\circ+k 360^\circ}\Leftrightarrow \hoac{&x=-180^\circ+k 360^\circ\\&x=120^\circ+k 360^\circ},(k\in\mathbb{Z})$.
		
	}
\end{ex}
\begin{ex}%[DCHT Toán 11 - KNTT -Thọ Bùi] %[1K1B4-4]
	Tìm nghiệm của phương trình $\sqrt{3}\cot \left(x+60^\circ\right)-1=0$.
	\choice
	{$x=-30^\circ+k360^\circ,k\in\mathbb{Z}$}
	{$x=-30^\circ+k180^\circ,k\in\mathbb{Z}$}
	{$x=k360^\circ,k\in\mathbb{Z}$}
	{\True $x=k180^\circ,k\in\mathbb{Z}$}
	\loigiai{
		\begin{eqnarray*}
			&  & \sqrt{3}\cot \left(x+60^\circ\right)-1=0 \\
			& \Leftrightarrow & \cot \left(x+60^\circ\right)=\dfrac{1}{\sqrt{3}}\\
			& \Leftrightarrow  & x=k180^\circ,k\in\mathbb{Z}.
		\end{eqnarray*}
	}
\end{ex}
\begin{ex}%[DCHT Toán 11 - KNTT -Thọ Bùi] %[1K1K4-4]
	Cho phương trình $\tan \left(2x-15^\circ\right)=1$ biết rằng $-90^\circ <x<90^\circ$. Số nghiệm của phương trình là
	\choice
	{$1$}
	{\True $2$}
	{$3$}
	{$4$}
	\loigiai{Ta có:	$\tan \left(2x-15^\circ\right)=1\Leftrightarrow 2x-15^\circ=45^\circ+k180^\circ\Leftrightarrow x=30^\circ+k90^\circ,k\in\mathbb{Z}$.\\ Do 
		$x\in(-90^\circ;90^\circ)\Leftrightarrow -90^\circ<30^\circ+k90^\circ<90^\circ\Leftrightarrow -\dfrac{4}{3}<k<\dfrac{2}{3}, k\in\mathbb{Z}\Rightarrow k\in\{-1;0\}$.\\
		nên phương trình có hai nghiệm thỏa mãn yêu cầu.
	}
\end{ex}
%%=====Câu 1
\begin{ex}%[DCHT Toán 11 - KNTT -Thọ Bùi] %[1K1K4-4]
Số nghiệm của phương trình $\sin \left(2 x-40^\circ\right)=\dfrac{\sqrt{3}}{2}$ với $-180^\circ \leq x \leq 180^\circ$ là
\choice
{$2$}
{\True $4$}
{$6$}
{$7$}
\loigiai{
Ta có: $$\sin \left(2 x-40^\circ\right)=\dfrac{\sqrt{3}}{2}\Leftrightarrow \sin \left(2 x-40^\circ\right)=\sin 60^\circ\Leftrightarrow\left[\begin{array}{l}
2 x-40^\circ=60^\circ+k 360^\circ \\
2 x-40^\circ=180^\circ-60^\circ+k 360^\circ
\end{array}\right.\Leftrightarrow\left[\begin{array}{l}
x=50^\circ+k 180^\circ \\
x=80^\circ+k 180^\circ.
\end{array}\right.$$
Do $-180^\circ \leq x \leq 180^\circ$ nên $x\in\left\{-130^\circ; 50^\circ; -100^\circ; 80^\circ\right\}$.\\
Vậy có tất cả $4$ nghiệm thỏa mãn bài toán.
}
\end{ex}

\begin{ex}%[DCHT Toán 11 - KNTT -Thọ Bùi] %[1K1K4-4]
	Tìm tập nghiệm $S$ của phương trình $\sin\left(x+30^\circ\right)\cdot\cos\left(x-45^\circ\right)=0.$
	\choice
	{$S=\left\{-30^\circ+k180^\circ,k\in\mathbb{Z}\right\}$}
	{\True $S=\left\{-30^\circ+k180^\circ;135^\circ+k180^\circ,k\in\mathbb{Z}\right\}$}
	{$S=\left\{135^\circ+k180^\circ,k\in\mathbb{Z}\right\}$}
	{$S=\left\{45^\circ+k180^\circ,k\in\mathbb{Z}\right\}$}
	\loigiai{Ta có:  $$\sin\left(x+30^\circ\right)\cdot\cos\left(x-45^\circ\right)=0\Leftrightarrow \hoac{&\sin(x+30^\circ)=0\\&\cos(x-45^\circ)=0}\Leftrightarrow \hoac{&x+30^\circ=k\cdot 180^\circ\\&x-45^\circ=90^\circ+k\cdot 180^\circ}\Leftrightarrow \hoac{&x=-30^\circ+k\cdot 180^\circ\\&x=135^\circ+k\cdot 180^\circ.}$$
		Vậy phương trình có tập nghiệm là $S=\left\{-30^\circ+k180^\circ;135^\circ+k180^\circ,k\in\mathbb{Z}\right\}$.}
\end{ex}

\Closesolutionfile{ans}
% \begin{indapan}{10}
% 	{ans/ans-1K1-3-Dang4}
% \end{indapan}