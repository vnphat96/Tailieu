
\begin{dang}{Phương trình bậc nhất đối với $\sin x$ và $\cos x$ }
	\begin{enumerate}
		\item Dạng phương trình
			\begin{itemize}
				\item [$\bullet$] $a \sin x + b \cos x =c \quad (1)$.
				\item [$\bullet$] Điều kiện có nghiệm $a^2+b^2 \ge c^2$.
			\end{itemize}
 		\item Phương pháp giải: Chia 2 vế phương trình cho $\sqrt{a^2+b^2}$. Khi đó
		\begin{eqnarray*}
			(1)&\Leftrightarrow&
			\dfrac{a}{\sqrt{a^2+b^2}} \sin x \pm \dfrac{b}{\sqrt{a^2+b^2}} \cos x=\dfrac{c}{\sqrt{a^2+b^2}}\\
			&\Leftrightarrow& \cos \phi \cdot \sin x \pm \sin \phi \cdot \cos x =\dfrac{c}{\sqrt{a^2+b^2}}\\
			&\Leftrightarrow& \sin \left(x \pm \phi \right)=\dfrac{c}{\sqrt{a^2+b^2}} \quad(2), \text{ với }\cos \phi=\dfrac{a}{\sqrt{a^2+b^2}} \text{ và } \sin \phi=\dfrac{b}{\sqrt{a^2+b^2}}.
		\end{eqnarray*}
		Phương trình (2) là phương trình cơ bản đã xét ở bài trước.
				Chú ý hai công thức sau:
			\begin{itemize}
				\item [$\bullet$] $\sin a \cos b \pm \cos a \sin b =\sin(a \pm b)$.
				\item [$\bullet$] $\cos a \cos b \pm \sin a \sin b =\cos(a \mp b)$.
			\end{itemize}
	\end{enumerate}
\end{dang}

\subsubsection{Ví dụ}
\begin{vd}%%[TH]%[DCHT Toán 11 - KNTT -Tên Nguyễn Thanh Sang] %[%[1K1B4-8]
      Giải phương trình sau: $\sqrt{3}\cdot\sin x+\cos x=1$.
	\dapso{$x=k2\pi; x=\dfrac{2\pi}{3}+k2\pi, k\in \mathbb{Z}$.}
	\loigiai{
	\begin{eqnarray}
		\sqrt{3}\cdot\sin x+\cos x=1 &\Leftrightarrow &\dfrac{\sqrt{3}}{2}\cdot \sin x+\dfrac{1}{2}\cdot \cos x=\dfrac{1}{2} \notag\\
		&\Leftrightarrow &\cos\dfrac{\pi}{6}\cdot \sin x+\sin\dfrac{\pi}{6}\cdot \cos x=\sin\dfrac{\pi}{6} \notag\\
		&\Leftrightarrow &\sin\left( x+\dfrac{\pi}{6}\right)=\sin\dfrac{\pi}{6} \notag\\
		&\Leftrightarrow &\hoac{&x+\dfrac{\pi}{6}=\dfrac{\pi}{6}+k2\pi\\ &x+\dfrac{\pi}{6}=\dfrac{5\pi}{6}+k2\pi}, k\in \mathbb{Z} \notag\\
		&\Leftrightarrow &\hoac{&x=k2\pi\\ &x=\dfrac{2\pi}{3}+k2\pi}, k\in \mathbb{Z}. \notag
	\end{eqnarray}
	Vậy phương trình có 2 họ nghiệm $\hoac{&x=k2\pi\\ &x=\dfrac{2\pi}{3}+k2\pi}$, $k\in \mathbb{Z}$.
	}
\end{vd}
\begin{vd}%[TH]%[DCHT Toán 11 - KNTT -Tên Nguyễn Thanh Sang] %[1K1B4-8]
	Giải phương trình: $\sin x + \sqrt{3}\cos x=1$. 
	\dapso{$\left\{ -\dfrac{\pi}{6} +  2k\pi; \dfrac{\pi}{2}+2k\pi, k\in\mathbb{Z}  \right\}$.}
	\loigiai{
		Điều kiện $1^2 + \left(\sqrt{3}\right)^2=4>1^2$ (luôn đúng).\\
		Chia $2$ vế cho $\sqrt{a^2 + b^2}=2$ thì phương trình trở thành:
		\begin{eqnarray*}
			& & \dfrac{1}{2}\sin x + \dfrac{\sqrt{3}}{2}\cos x= \dfrac{1}{2}\\
			&\Leftrightarrow & \sin x\cos\dfrac{\pi}{3} + \cos x\sin\dfrac{\pi}{3} = \sin\dfrac{\pi}{6} \\
			&\Leftrightarrow & \sin\left(x + \dfrac{\pi}{3}\right)=\sin \dfrac{\pi}{6}\\
			&	\Leftrightarrow & \hoac{&x+\dfrac{\pi}{3} = \dfrac{\pi}{6}+ k2\pi\\&x + \dfrac{\pi}{3} = \pi - \dfrac{\pi}{6} + 2k\pi  } \\
			&	\Leftrightarrow & \hoac{&x = -\dfrac{\pi}{6} +  2k\pi\\&x = \dfrac{\pi}{2}+2k\pi.}
		\end{eqnarray*}
		Kết luận $S= \left\{ -\dfrac{\pi}{6} +  2k\pi; \dfrac{\pi}{2}+2k\pi \mid k\in\mathbb{Z}  \right\}$.
		}
\end{vd}
\begin{vd}%[TH]%[DCHT Toán 11 - KNTT -Tên Nguyễn Thanh Sang] %[1K1K4-8]
Giải phương trình: $\sqrt{3}\cos x - \sin x=\sqrt{2}$.
\dapso{$S= \left\{\dfrac{\pi}{12} +  2k\pi; -\dfrac{5\pi}{12}+2k\pi\mid k\in\mathbb{Z}  \right\}$.}
\loigiai{
	Điều kiện $\left(\sqrt{3}\right)^2 + (-1)^2 = 4>\left(\sqrt{2}\right)^2$ (luôn đúng).\\
	Chia $2$ vế cho $\sqrt{a^2 + b^2}=2$ thì phương trình trở thành
	\begin{eqnarray*}
		& & \dfrac{\sqrt{3}}{2}\cos x - \dfrac{1}{2}\sin x= \dfrac{\sqrt{2}}{2}\\
		&\Leftrightarrow & \cos x\cdot \cos\dfrac{\pi}{6} - \sin x\cdot \sin\dfrac{\pi}{6} = \cos\dfrac{\pi}{4} \\
		&\Leftrightarrow & \cos\left(x + \dfrac{\pi}{6}\right)=\cos\dfrac{\pi}{4}\\
		&	\Leftrightarrow & \hoac{&x+\dfrac{\pi}{6} = \dfrac{\pi}{4}+ k2\pi\\&x + \dfrac{\pi}{6} = -\dfrac{\pi}{4} + 2k\pi  } \\
		&	\Leftrightarrow & \hoac{&x = \dfrac{\pi}{12} +  2k\pi\\&x = -\dfrac{5\pi}{12}+2k\pi.}
	\end{eqnarray*}
	Kết luận $S= \left\{\dfrac{\pi}{12} +  2k\pi; -\dfrac{5\pi}{12}+2k\pi\mid k\in\mathbb{Z}  \right\}$.
}
\end{vd}
\begin{vd}%[TH]%[DCHT Toán 11 - KNTT -Tên Nguyễn Thanh Sang] %[1K1K4-8]
	 Giải phương trình: $\sqrt{3}\sin{2x} + \cos{2x}=\sqrt{2}$.
	\dapso{$S= \left\{ \dfrac{\pi}{24} +  k\pi; \dfrac{7\pi}{24}+k\pi\mid k\in\mathbb{Z}  \right\}$}
	 \loigiai{
	 	Điều kiện $\left(\sqrt{3}\right)^2 + 1^2 = 4>\left(\sqrt{2}\right)^2$ (luôn đúng).\\
	 	Chia $2$ vế cho $\sqrt{a^2 + b^2}=2$ thì phương trình trở thành
	 	\begin{eqnarray*}
	 		& & \dfrac{\sqrt{3}}{2}\sin 2x + \dfrac{1}{2}\cos 2x= \dfrac{\sqrt{2}}{2}\\
	 		&\Leftrightarrow & \sin 2x\cdot \cos\dfrac{\pi}{6} + \cos 2x\cdot \sin\dfrac{\pi}{6} = \sin\dfrac{\pi}{4} \\
	 		&\Leftrightarrow & \sin\left(2x + \dfrac{\pi}{6}\right)=\sin\dfrac{\pi}{4}\\
	 		&	\Leftrightarrow & \hoac{&2x+\dfrac{\pi}{6} = \dfrac{\pi}{4}+ k2\pi\\&2x + \dfrac{\pi}{6} = \pi -\dfrac{\pi}{4} + 2k\pi  } \\
	 		&	\Leftrightarrow & \hoac{&x = \dfrac{\pi}{24} +  k\pi\\&x = \dfrac{7\pi}{24}+k\pi.}
	 	\end{eqnarray*}
	 	Kết luận $S= \left\{ \dfrac{\pi}{24} +  k\pi; \dfrac{7\pi}{24}+k\pi\mid k\in\mathbb{Z}  \right\}$.
} 	
\end{vd}
\begin{vd}%[VDT]%[DCHT Toán 11 - KNTT -Tên Nguyễn Thanh Sang] %[1K1K4-9]
Tìm tất cả các giá trị nguyên của $m$ để phương trình $(m-1)\sin 2x-4\cos 2x=2-2m$ có nghiệm.
	\dapso{$-1, 0, 1, 2, 3.$}
\loigiai{
		Để phương trình có nghiệm thì \begin{eqnarray*}
		& &(m-1)^2+(-4)^2\ge (2-2m)^2\\
		&\Leftrightarrow& m^2-2m+1+16\ge 4-8m+4m^2\\
		&\Leftrightarrow& 3m^2-6m-13\le 0\\
		&\Leftrightarrow& \dfrac{3-4\sqrt{3}}{3}\le m\le \dfrac{3+4\sqrt{3}}{3}.
	\end{eqnarray*}
	Khi đó các giá trị nguyên của tham số $m$ là $-1$, $0$, $1$, $2$, $3$.
	}
\end{vd}
\begin{vd}%[VDT]%[DCHT Toán 11 - KNTT -Tên Nguyễn Thanh Sang] %[1K1K4-9]
	Tính tổng tất cả các giá trị nguyên của tham số $m$ để phương trình $(2m+1)\sin x-(m+2)\cos x=2m+3$ vô nghiệm.
	\dapso{$10.$}
	\loigiai{
	Phương trình vô nghiệm khi
	\begin{eqnarray*}
		& & (2m+1)^2 + (m+2)^2 < (2m+3)^2\\
		&\Leftrightarrow & 4m^2+4m+1 +m^2+4m+4<4m^2+12m+9\\
		&\Leftrightarrow & m^2-4m-4<0\\
		&\Leftrightarrow & 2-2\sqrt{2} < m < 2+2\sqrt{2}.
	\end{eqnarray*}
	Do $m$ nguyên nên ta nhận $m \in \{0;1;2;3;4\}$.\\
	Vậy tổng tất cả các giá trị nguyên của tham số $m$ bằng: $0+1+2+3+4=10$.
}
\end{vd}
\begin{vd}%[VDT]%[DCHT Toán 11 - KNTT -Tên Nguyễn Thanh Sang] %[1K1K4-9]
	Tìm số nghiệm của phương trình $\sin 5x + \sqrt{3}\cos 5x = 2\sin 7x$ trên khoảng $\left(0; \dfrac{\pi}{2}\right).$
		\dapso{$4.$}
\loigiai{
	$\sin 5x + \sqrt{3}\cos 5x = 2\sin 7x$ $\Leftrightarrow \sin(5x+\dfrac{\pi}{3})=\sin 7x$
	\begin{align*}
		\Leftrightarrow \hoac{& 5x+\dfrac{\pi}{3}=7x+k2\pi \\ & 5x+\dfrac{\pi}{3}=\pi-7x+k2\pi} \Leftrightarrow \hoac{& x=\dfrac{\pi}{6}-k2\pi \qquad (1)\\ &x= \dfrac{\pi}{18}+k\dfrac{\pi}{6}\qquad (2).}
	\end{align*}
	Vì $x\in \left(0; \dfrac{\pi}{2}\right), \ k\in \mathbb{Z}$ nên $\hoac{& (1) \Rightarrow k=0 \\ & (2) \Rightarrow k\in \{0; 1; 2\}.}$\\
	Vậy số nghiệm của phương trình đã cho là $4$.
}

\end{vd}
\subsubsection{Bài tập tự luận}
\begin{bt}%[TH]%[DCHT Toán 11 - KNTT -Tên Nguyễn Thanh Sang] %[1K1K4-8]
	Giải phương trình $\cos x-\sqrt{3}\sin x=-2.$
	\dapso{$x=\dfrac{2\pi}{3}+k2\pi, (k\in \mathbb{Z}).$}
	\loigiai{
	Ta có
	\begin{align*}
		&\quad \cos x-\sqrt{3}\sin x =-2 \\
		& \Leftrightarrow \dfrac{1}{2}\cos x-\dfrac{\sqrt{3}}{2}\sin x=-1 \\
		&\Leftrightarrow \cos \left(\dfrac{\pi}{3}+x\right)=-1 \\
		& \Leftrightarrow \dfrac{\pi}{3}+x=\pi +k2\pi \\
		&\Leftrightarrow x=\dfrac{2\pi}{3}+k2\pi, (k\in \mathbb{Z}).
	\end{align*}
}
\end{bt}
\begin{bt}%[TH]%[DCHT Toán 11 - KNTT -Tên Nguyễn Thanh Sang] %[1K1K4-8]
Tìm tập nghiệm của trình $\sin x +\cos x =\sqrt{2}$.
	\dapso{$x=\dfrac{\pi}{4}+k2\pi,(k\in \mathbb{Z}).$}
	\loigiai{
	Phương trình đã cho tương đương
$	\sin\left(x+\dfrac{\pi}{4}\right) =1 \Leftrightarrow x+\dfrac{\pi}{4}=\dfrac{\pi}{2}+k2\pi \Leftrightarrow x=\dfrac{\pi}{4}+k2\pi, (k\in \mathbb{Z}).$ 
	}
\end{bt}
\begin{bt}%[TH]%[DCHT Toán 11 - KNTT -Tên Nguyễn Thanh Sang] %[1K1K4-8]
	Giải phương trình $\sqrt{3}\sin 2x-\cos 2x=2$.
	\dapso{$x=\dfrac{\pi}{3}+k\pi\quad, (k\in \mathbb{Z}).$}
	\loigiai{
		Phương trình đã cho tương đương với
		\begin{eqnarray*}
			&& \dfrac{\sqrt{3}}{2}\cdot\sin 2x-\dfrac{1}{2}\cdot\cos 2x=1
			\Leftrightarrow \sin\left(2x-\dfrac{\pi}{6}\right)=1\\
			& \Leftrightarrow & 2x-\dfrac{\pi}{6}=\dfrac{\pi}{2}+k2\pi\Leftrightarrow x=\dfrac{\pi}{3}+k\pi\quad (k\in\mathbb{Z}).
		\end{eqnarray*}
	}
\end{bt}
\begin{bt}%[TH]%[DCHT Toán 11 - KNTT -Tên Nguyễn Thanh Sang] %[1K1K4-8]
Giải phương trình $\sin 2x-\sqrt{3}\cos 2x=1.$
	\dapso{$x=\dfrac{\pi}{4}+k\pi,x=\dfrac{7\pi}{12}+k\pi, (k\in \mathbb{Z}).$}
	\loigiai{
		Phương trình đã cho tương đương với
		$\dfrac{1}{2}\sin 2x-\dfrac{\sqrt{3}}{2}\cos 2x=\dfrac{1}{2}$		$\Leftrightarrow \sin \left(2x-\dfrac{\pi}{3}\right)=\dfrac{1}{2}$\\
		$\Leftrightarrow  \sin \left(2x-\dfrac{\pi}{3}\right)=\sin \dfrac{\pi}{6}$
		$\Leftrightarrow  \left[
		\begin{aligned}
			&2x-\dfrac{\pi}{3}=\dfrac{\pi}{6}+k2\pi\\
			&2x-\dfrac{\pi}{3}=\pi-\dfrac{\pi}{6}+k2\pi.\\
		\end{aligned}\right.$
		$\Leftrightarrow  \left[
		\begin{aligned}
			&x=\dfrac{\pi}{4}+k\pi\\
			&x=\dfrac{7\pi}{12}+k\pi\\
		\end{aligned}\right.$
	}
\end{bt}
\begin{bt}%[VDT]%[DCHT Toán 11 - KNTT -Tên Nguyễn Thanh Sang] %[1K1K4-9] 
Tìm tất cả các giá trị của tham số m để phương trình $\sin x+\left(m-1\right)\cos x=2m-1$ có nghiệm.
	\dapso{$-\dfrac{1}{3}\le m\le 1.$}
	\loigiai{
	Phương trình có nghiệm khi: $1^2+(m-1)^2\ge (2m-1)^2 \Leftrightarrow 3m^2-2m-1\le 0 \Leftrightarrow -\dfrac{1}{3}\le m\le 1$.
		}
\end{bt}
\begin{bt}%[VDT]%[DCHT Toán 11 - KNTT -Tên Nguyễn Thanh Sang] %[1K1K4-9]
	Tìm tất cả các giá trị thực của tham số $m$ để phương trình $m\sin 2x+(m-1)\cos 2x=\sqrt{5}m$ vô nghiệm.
	\dapso{$m>\dfrac{1}{3}$ hoặc $m<-1$.}
	\loigiai{
	Phương trình đã cho vô nghiệm khi và chỉ khi:
	\begin{align*}
		&m^2+(m-1)^2<\left(\sqrt{5}m\right)^2 \Leftrightarrow 2m^2-2m+1< 5m^2\\
		\Leftrightarrow\ &3m^2+2m-1>0 \Leftrightarrow \hoac{&m>\dfrac{1}{3}\\ &m<-1.}
	\end{align*}
	}
\end{bt}
\begin{bt}%[VDT]%[DCHT Toán 11 - KNTT -Tên Nguyễn Thanh Sang] %[1K1K4-9]
Giải phương trình $\sin x - \sqrt 3 \cos x = 2\sin 3x.$
	% \dapso{$x = \dfrac{\pi }{3} + k\dfrac{\pi }{2}, (k\in \mathbb{Z})}.$}
	\loigiai{
	Phương trình tương đương $ 2\sin \left(x-\dfrac{\pi}{3} \right)=2\sin 3x  \Leftrightarrow \sin \left(x-\dfrac{\pi}{3}\right)=\sin 3x$.\\
	$\Leftrightarrow \hoac{&x-\dfrac{\pi}{3}=3x+k2\pi\\&x-\dfrac{\pi}{3}=\pi-3x+k2\pi}\Leftrightarrow \hoac{&x=-\dfrac{\pi}{6}-k\pi\\& x=\dfrac{\pi}{3}-k\dfrac{\pi}{2}}$, $ k\in\mathbb{Z} $. \\
	Hợp hai họ nghiệm trên ta được $x$ là $ \dfrac{\pi }{3} + k\dfrac{\pi }{2}, (k\in \mathbb{Z}).$
	}
\end{bt}
\begin{bt}%[VDT]%[DCHT Toán 11 - KNTT -Tên Nguyễn Thanh Sang] %[1K1K4-9]
Tính tổng tất cả các nghiệm thuộc $(0;2\pi)$ của phương trình $\sqrt{2}\cos 3x=\sin x+\cos x$.
	\dapso{$6\pi.$}
\loigiai{
Phương trình tương đương với
$$\cos 3x=\cos\left(x-\dfrac{\pi}{4}\right)\Leftrightarrow \hoac{& 3x=x-\dfrac{\pi}{4}+k2\pi\\ & 3x=-x+\dfrac{\pi}{4}+k2\pi}\Leftrightarrow \hoac{& x=-\dfrac{\pi}{8}+k\pi\\ & x=\dfrac{\pi}{16}+k\dfrac{\pi}{2}.}$$
Vì $x\in(0;2\pi)$ nên $x\in\left\{ \dfrac{7\pi}{8};\dfrac{15\pi}{8};\dfrac{\pi}{16};\dfrac{9\pi}{16};\dfrac{17\pi}{16};\dfrac{25\pi}{16} \right\}$. Tổng các nghiệm là $6\pi$.
}
\end{bt}
\begin{bt}%[VDC]%[DCHT Toán 11 - KNTT -Tên Nguyễn Thanh Sang] %[1K1G4-8]
Gọi $S$ là tập hợp tất cả các nghiệm thuộc khoảng $(0;2023)$ của phương trình lượng giác $\sqrt{3}\left(1-\cos 2x\right)+\sin 2x-4\cos x+8=4\left(\sqrt{3}+1\right)\sin x$. Tính tổng tất cả các phần tử của $S.$
	\dapso{$\dfrac{310408}{3}\pi.$}
 \loigiai{
 	Ta có $\sqrt{3}\left(1-\cos 2x\right)+\sin 2x-4\cos x+8=4\left(\sqrt{3}+1\right)\sin x$ \\
 	$ \Leftrightarrow 2\sqrt{3}\sin^2x+2\sin x\cos x-4\cos x+8=4\left(\sqrt{3}+1\right)\sin x$ \\
 	$ \Leftrightarrow 2\sqrt{3}\sin x\left(\sin x-2\right)+2\cos x\left(\sin x-2\right)=4\left(\sin x-2\right)$ \\
 	$ \Leftrightarrow 2\sqrt{3}\sin x+2\cos x=4$ (vì $\sin x\leqslant 1<2$ ) \\
 	$ \Leftrightarrow \sqrt{3}\sin x+\cos x=2 \Leftrightarrow \sin x\cos \dfrac{\pi}{6}+\cos x\sin \dfrac{\pi}{6}=1$ \\
 	$ \Leftrightarrow \sin \left(x+\dfrac{\pi}{6}\right)=1 \Leftrightarrow x+\dfrac{\pi}{6}=\dfrac{\pi}{2}+k2\pi \Leftrightarrow x=\dfrac{\pi}{3}+k2\pi,  \left(k\in \mathbb{Z}\right)$. \\
 	Theo đề bài $x\in (0;2023) \Rightarrow \dfrac{\pi}{3}+k2\pi \in (0;2023) \Rightarrow 2k+\dfrac{1}{3}\in \left(0;\dfrac{2023}{\pi}\right) \Rightarrow k\in \{0;1;\ldots;321\}$. \\
 	Tổng tất cả các phần tử của $S$ là \\
 	$322\cdot \dfrac{\pi}{3}+(0+1+2+\cdots +321)2\pi =322\cdot \dfrac{\pi}{3}+51681\cdot 2\pi =\dfrac{310408}{3}\pi $.}
\end{bt}
\subsubsection{Bài tập trắc nghiệm}
\Opensolutionfile{ans}[ans/ans-1K1-4-Dang9]

%%==========Câu 1
\begin{ex}%[DCHT Toán 11 - KNTT -Tên Nguyễn Thanh Sang] %[1K1Y4-8]
Điều kiện có nghiệm của phương trình $a\cos x+b\sin x=c\ \left(a^2+b^2 \ne 0\right)$ là
\choice
{$a^2+b^2>c^2$}
{$a^2+b^2<c^2$}
{$a^2+b^2 \geq c$}
{\True $a^2+b^2 \geq c^2$}
\loigiai{
	Vì $a^2+b^2 \ne 0$ nên chia hai vế của phương trình cho $\sqrt{a^2+b^2}$, ta có
	$$a\cos x+b\sin x=c \Leftrightarrow \dfrac{a}{\sqrt{a^2+b^2}}\cos x+\dfrac{b}{\sqrt{a^2+b^2}}\sin x=\dfrac{c}{\sqrt{a^2+b^2}}\quad (1).$$
	Vì $\left(\dfrac{a}{\sqrt{a^2+b^2}}\right)^2+\left(\dfrac{b}{\sqrt{a^2+b^2}}\right)^2=1$ nên có một góc $\alpha$ sao cho $cos\alpha =\dfrac{a}{\sqrt{a^2+b^2}},\ \sin \alpha =\dfrac{b}{\sqrt{a^2+b^2}}$. \\
	Phương trình $(1) \Leftrightarrow \cos\alpha \cos x+\sin \alpha \sin x=\dfrac{c}{\sqrt{a^2+b^2}} \Leftrightarrow \cos\left(\alpha +x\right)=\dfrac{c}{\sqrt{a^2+b^2}}\quad (2)$.\\
	Phương trình đã cho có nghiệm $\Leftrightarrow $  Phương trình (1) có nghiệm $\Leftrightarrow $  Phương trình (2) có nghiệm\\
	$$\Leftrightarrow  \left|\dfrac{c}{\sqrt{a^2+b^2}}\right| \leq 1 \Leftrightarrow a^2+b^2 \geq c^2.$$}
\end{ex}

%%==========Câu 2
\begin{ex}%[DCHT Toán 11 - KNTT -Tên Nguyễn Thanh Sang] %[1K1Y4-8]
Phương trình nào sau đây là phương trình bậc nhất đối với $\sin x$ và $\cos x$?
\choice
{$x^2 - 3\sin x + \cos x = 2$}
{$\sin x + 3x = 1$}
{$3\cos x - \sin 2x = 2$}
{\True $\sqrt 3 \cdot \cos x - \sin x = 1$}
\loigiai{Phương trình bậc nhất đối với $\sin x$ và $\cos x$ có dạng $a\sin x+ b\cos x=c$. Trong đó $a$, $b$, $c  \in \mathbb{R}$ và $ a^2+b^2>0$.}
\end{ex}

%%==========Câu 3
\begin{ex}%[DCHT Toán 11 - KNTT -Tên Nguyễn Thanh Sang]%[1K1Y4-8]
Phương trình $\sqrt{3}\sin3x + \cos 3x =  - 1$ tương đương với phương trình nào sau đây?
\choice{$\sin \left({3x} - \dfrac{\pi }{6}\right) =  - \dfrac{1}{2}$}
{$\sin \left(3x + \dfrac{\pi }{6} \right) =  - \dfrac{\pi }{6}$}
{\True $\sin \left( 3x + \dfrac{\pi }{6}\right) =  - \dfrac{1}{2}$}
{$\sin \left(3x + \dfrac{\pi }{6}\right) = \dfrac{1}{2 }$}
\loigiai{Ta có $
	\begin{aligned}[t]
		&\sqrt{3}\sin3x+\cos3x= -1\Leftrightarrow\dfrac{\sqrt{3}}{2}\sin3x+\dfrac{1}{2}\cos3x=-\dfrac{1}{2}\\
		\Leftrightarrow\,&\cos\dfrac{\pi}{6}\sin3x+\sin\dfrac{\pi}{6}\cos3x=-\dfrac{1}{2}\Leftrightarrow\sin\left(3x+\dfrac{\pi}{6}\right)=-\dfrac{1}{2}.
	\end{aligned}$}
\end{ex}

%%==========Câu 4
\begin{ex}%[DCHT Toán 11 - KNTT -Tên Nguyễn Thanh Sang] %[1K1B4-8]
	Phương trình $\sqrt{3}\sin x+\cos x=-\dfrac{1}{4}$ tương đương với phương trình nào sau đây?
	\choice
	{\True $\sin \left(x+\dfrac{\pi }{6}\right)=-\dfrac{1}{8}$}
	{$\sin \left(x+\dfrac{\pi }{6}\right)=-\dfrac{1}{4}$}
	{$\sin \left(x+\dfrac{\pi }{3}\right)=-\dfrac{1}{4}$}
	{$\sin \left(x-\dfrac{\pi }{3}\right)=-\dfrac{1}{8}$}
	\loigiai{
		$$\sqrt{3}\sin x+\cos x=-\dfrac{1}{4}\Leftrightarrow \dfrac{\sqrt{3}}{2}\sin x+\dfrac{1}{2}\cos x=-\dfrac{1}{8}\Leftrightarrow \sin x\cos \dfrac{\pi }{6}+\sin \dfrac{\pi }{6}\cos x=-\dfrac{1}{8}\Leftrightarrow \sin \left(x+\dfrac{\pi }{6}\right)=-\dfrac{1}{8}.$$}
\end{ex}

%%==========Câu 5
\begin{ex}%[DCHT Toán 11 - KNTT -Tên Nguyễn Thanh Sang] %[1K1Y4-8]
Với giá trị nào của tham số $m$ thì phương trình $3\sin x+m\cos x=5$ vô nghiệm?
\choice
{$m>4$}
{$|m|\ge 4$}
{$m<-4$}
{\True $-4<m<4$}
\loigiai{
	Phương trình đã cho vô nghiệm khi và chỉ khi $3^2+m^2<5^2\Leftrightarrow m^2<16\Leftrightarrow -4<m<4$.
}
\end{ex}


%%==========Câu 6
\begin{ex}%[DCHT Toán 11 - KNTT -Tên Nguyễn Thanh Sang] %[1K1B4-8]
Số giá trị nguyên dương của $ m$ để phương trình $ m\sin x-3\cos x=m+1$ có nghiệm là
\choice
{ $6$}
{\True $4$}
{ $5$}
{ $2$}
\loigiai{
	Để phương trình có nghiệm: $ m^2+(-3)^2\ge (m+1)^2\Leftrightarrow m^2+9\ge m^2+2m+1\Leftrightarrow m\le 4$.\\
	Vậy có $4$ giá trị nguyên dương của $ m$ là $m\in \left\{ 1,2,3,4\right\}.$
}
\end{ex}

%%==========Câu 7
\begin{ex}%[DCHT Toán 11 - KNTT -Tên Nguyễn Thanh Sang] %[1K1B4-8]
Với $k\in \mathbb{Z}$, họ nghiệm của phương trình $\sqrt{3}\cos x+\sin x=-2$ là
\choice
{$x=\dfrac{\pi}{6}+k2\pi$}
{\True $x=-\dfrac{5\pi}{6}+k2\pi$}
{$x=\pm \dfrac{5\pi}{6}+k2\pi$}
{$x=-\dfrac{\pi}{2}+k2\pi$}
\loigiai{
	Ta có
	\begin{eqnarray*}
		\sqrt{3}\cos x+\sin x=-2&\Leftrightarrow& \dfrac{\sqrt{3}}{2}\cos x+\dfrac{1}{2}\sin x=-1\\
		&\Leftrightarrow& \sin\left(\dfrac{\pi}{3}+x\right)=-1\\
		&\Leftrightarrow& \dfrac{\pi}{3}+x=-\dfrac{\pi}{2}+k2\pi\\
		&\Leftrightarrow& x=-\dfrac{5\pi}{6}+k2\pi, k\in \mathbb{Z}.
	\end{eqnarray*}
}
\end{ex}

%%==========Câu 8
\begin{ex}%[DCHT Toán 11 - KNTT -Tên Nguyễn Thanh Sang] %[1K1B4-8]
Nghiệm của phương trình $\cos x-\sqrt{3}\sin x=-2$ là
\choice
{$x=\dfrac{4\pi}{3}+k2\pi; k \in \mathbb{Z}$}
{\True $x=\dfrac{2\pi}{3}+k2\pi; k \in \mathbb{Z}$}
{$x=\dfrac{5\pi}{6}+k2\pi; k \in \mathbb{Z}$}
{$x=\dfrac{7\pi}{6}+k2\pi; k \in \mathbb{Z}$}
\loigiai{
	Ta có
	\begin{align*}
		&\quad \cos x-\sqrt{3}\sin x =-2 \\
		& \Leftrightarrow \dfrac{1}{2}\cos x-\dfrac{\sqrt{3}}{2}\sin x=-1 \\
		&\Leftrightarrow \cos \left(\dfrac{\pi}{3}+x\right)=-1 \\
		& \Leftrightarrow \dfrac{\pi}{3}+x=\pi +k2\pi \\
		&\Leftrightarrow x=\dfrac{2\pi}{3}+k2\pi.
	\end{align*}
}
\end{ex}

%%==========Câu 9
\begin{ex}%[DCHT Toán 11 - KNTT -Tên Nguyễn Thanh Sang] %[1K1B4-8]
Nghiệm của phương trình $\sin x - \sqrt 3 \cos x = 2\sin 3x$ là
\choice
{$x = \dfrac{\pi }{6} + k\pi $ hoặc $x = \dfrac{\pi }{6} + k\dfrac{2\pi}{3}$, $ k\in \mathbb{Z} $}
{$x = \dfrac{\pi }{3} + k2\pi $ hoặc $x = \dfrac{2\pi}{3} + k2\pi $, $ k\in \mathbb{Z} $}
{$x =  - \dfrac{\pi }{3} + k2\pi $ hoặc $x = \dfrac{4\pi}{3} + k2\pi $, $ k\in \mathbb{Z} $}
{\True $x = \dfrac{\pi }{3} + k\dfrac{\pi }{2}$,$  k\in \mathbb{Z} $}
\loigiai{
	Phương trình tương đương $ 2\sin \left(x-\dfrac{\pi}{3} \right)=2\sin 3x  \Leftrightarrow \sin \left(x-\dfrac{\pi}{3}\right)=\sin 3x$.\\
	$\Leftrightarrow \hoac{&x-\dfrac{\pi}{3}=3x+k2\pi\\&x-\dfrac{\pi}{3}=\pi-3x+k2\pi}\Leftrightarrow \hoac{&x=-\dfrac{\pi}{6}-k\pi\\& x=\dfrac{\pi}{3}-k\dfrac{\pi}{2}}$, $ k\in\mathbb{Z} $. \\
	Hợp hai họ nghiệm trên ta được $x = \dfrac{\pi }{3} + k\dfrac{\pi }{2}$,$  k\in\mathbb{Z} $.
}
\end{ex}

%%==========Câu 10
\begin{ex}%[DCHT Toán 11 - KNTT -Tên Nguyễn Thanh Sang] %[1K1B4-8]
	Giải phương trình $\sin x+\cos x=\sqrt{2}\sin 3x$.
	\choice
	{$\left[
		\begin{aligned}&x=\dfrac{\pi}{10}+k\pi\\&x=\dfrac{\pi}{5}+\dfrac{k\pi}{2}\end{aligned}\right. (k\in\mathbb{Z})$}
	{$\left[
		\begin{aligned}&x=\dfrac{\pi}{4}+k\pi\\&x=\dfrac{\pi}{8}+\dfrac{k\pi}{2}\end{aligned}\right. (k\in\mathbb{Z})$}
	{$\left[
		\begin{aligned}&x=\dfrac{\pi}{2}+k\pi\\&x=\dfrac{\pi}{6}+\dfrac{k\pi}{2}\end{aligned}\right. (k\in\mathbb{Z})$}
	{\True $\left[
		\begin{aligned}&x=\dfrac{\pi}{8}+k\pi\\&x=\dfrac{3\pi}{16}+\dfrac{k\pi}{2}\end{aligned}\right. (k\in\mathbb{Z})$}
	\loigiai{
		Ta có $\sin x+\cos x=\sqrt{2}\sin 3x\Leftrightarrow \sin\left(x+\dfrac{\pi}{4}\right)=\sin 3x\Leftrightarrow \left[
		\begin{aligned}&x=\dfrac{\pi}{8}+k\pi\\&x=\dfrac{3\pi}{16}+\dfrac{k\pi}{2}\end{aligned}\right. (k\in\mathbb{Z})$.
	}
\end{ex}

%%==========Câu 11
\begin{ex}%[DCHT Toán 11 - KNTT -Tên Nguyễn Thanh Sang] %[1K1K4-8]
	Phương trình $\sin  2x-\cos 2x=\sqrt{2}\cos x$ có hai họ nghiệm dạng $x=\alpha +k2\pi$ và $x=\beta +\dfrac{k2\pi}{3}$, trong đó $\alpha \in \left(0;\pi\right)$ và $\beta \in \left(0;\dfrac{\pi}{2}\right)$. Khi đó, giá trị $2\alpha -\beta$ là
	\choice
	{$-\dfrac{\pi}{4}$}
	{$\dfrac{7\pi}{4}$}
	{$-\dfrac{11\pi}{4}$}
	{\True $\dfrac{5\pi}{4}$}
	\loigiai{
		Xét $\sin  2x-\cos 2x=\sqrt{2}\cos x$
		\begin{eqnarray*}
			& \Leftrightarrow &\sqrt{2}\sin \left(2x-\dfrac{\pi}{4}\right)=\sqrt{2}\sin \left(\dfrac{\pi}{2}-x\right) \\
			& \Leftrightarrow& \hoac{& 2x-\dfrac{\pi}{4}=\dfrac{\pi}{2}-x+k2\pi \\& 2x-\dfrac{\pi}{4}=\pi -\dfrac{\pi}{2}+x+k2\pi}\\&\Leftrightarrow& \hoac{& x=\dfrac{\pi}{4}+k\dfrac{2\pi}{3} \\& x=\dfrac{3\pi}{4}+k2\pi.}\left(k\in \mathbb{Z}\right) \\
		\end{eqnarray*}
		Theo đề bài ta tìm được $\alpha=\dfrac{3\pi}{4}$, $\beta=\dfrac{\pi}{4}$.\\
		Khi đó $2\alpha -\beta=\dfrac{5\pi}{4}$.
	}
\end{ex}

%%==========Câu 12
\begin{ex}%[DCHT Toán 11 - KNTT -Tên Nguyễn Thanh Sang]  %[1K1K4-8]
Phương trình $\sin^2x+\sqrt{3}\sin x\cos x=1$ có bao nhiêu nghiệm thuộc $[0;2\pi]?$
\choice
{$5$}
{$3$}
{$2$}
{\True $4$}
\loigiai{
	Ta có phương trình đã cho
	\[
	\begin{aligned}
		&\Leftrightarrow -\cos^2x+\sqrt{3}\sin x\cos x=0\\
		& \Leftrightarrow \hoac{&\cos x=0\\&-\cos x+\sqrt{3}\sin x=0}\\
		&\Leftrightarrow \hoac{&x=\dfrac{\pi}{2}+k\pi\\&x=\dfrac{\pi}{6}+l\pi.}
	\end{aligned}
	\]
	Vì $x\in\left[0;2\pi\right]$ nên ta có
	\begin{itemize}
		\item $0\le\dfrac{\pi}{2}+ k\pi\le 2\pi\Leftrightarrow -\dfrac{1}{2}\le k\le \dfrac{3}{2}\Leftrightarrow\left[\begin{aligned}
			&k=0\Rightarrow x=\dfrac{\pi}{2}\\
			&k=1\Rightarrow x=\dfrac{3\pi}{2}.\\
		\end{aligned}\right.$
		\item $0\le\dfrac{\pi}{6}+l\pi\le 2\pi\Leftrightarrow-\dfrac{1}{6}\le m\le\dfrac{11}{6}\Leftrightarrow \hoac{&l=0\Rightarrow x=\dfrac{\pi}{6}\\&l=1\Rightarrow x=\dfrac{7\pi}{6}.}$.
	\end{itemize}
	Vậy phương trình có bốn nghiệm thuộc $\left[0;2\pi\right]$.
}
\end{ex}

%%==========Câu 13
\begin{ex}%[DCHT Toán 11 - KNTT -Tên Nguyễn Thanh Sang] %[1K1K4-8]
	Tìm số nghiệm của phương trình  $4\sin^2x+3\sqrt{3}\sin 2x-2\cos^2x=4$ trong khoảng $\left(0;\dfrac{\pi}{2}\right)$.
	\choice
	{$4$}
	{$0$}
	{ $2$}
	{\True $1$}
	\loigiai{Phương trình đã cho tương đương với
		\begin{eqnarray*}
			&&4\sin^2x+3\sqrt{3}\sin 2x-2\cos^2x=4\left(\sin^2x+\cos^2x\right)\\
			&\Leftrightarrow &3\sqrt{3}\sin 2x-6\cos^2x=0\\
			&\Leftrightarrow &6\cos x\left(\sqrt{3}\sin x-\cos x\right)=0\\
			&\Leftrightarrow &  \left[
			\begin{aligned}
				&\cos x=0\\
				&\sin\left(x-\dfrac{\pi}{6}\right)=0\\
			\end{aligned}\right.\\
			&\Leftrightarrow &  \left[
			\begin{aligned}
				&x=\dfrac{\pi}{2}+k\pi \\
				&x=\dfrac{\pi}{6}+k\pi\\
			\end{aligned}\right.\\
			&\Leftrightarrow &  \left[
			\begin{aligned}
				&x=\dfrac{\pi}{2}\notin \left(0;\dfrac{\pi}{2}\right)\\
				&x=\dfrac{\pi}{6}\in \left(0;\dfrac{\pi}{2}\right).\\
			\end{aligned}\right.\\
		\end{eqnarray*}
		Vậy phương trình có $1$ nghiệm thuộc $\left(0;\dfrac{\pi}{2}\right)$.
	}
\end{ex}

%%==========Câu 14
\begin{ex}%[DCHT Toán 11 - KNTT -Tên Nguyễn Thanh Sang]  %[1K1K4-8]
Số nghiệm của phương trình $\cos^2x-\sin 2x=\sqrt{2}+\cos^2\left(\dfrac{\pi}{2}+x\right)$ trên khoảng $\left(0; 3\pi\right)$ bằng
\choice
{$2$}
{\True $3$}
{$4$}
{$1$}
\loigiai{Ta có\\ $\begin{aligned}
		\cos^2x-\sin 2x=\sqrt{2}+\cos^2\left(\dfrac{\pi}{2}+x\right)&\Leftrightarrow \cos^2x-\sin 2x=\sqrt{2}+\sin^2x\\
		&\Leftrightarrow \cos 2x-\sin 2x=\sqrt{2}\\
		&\Leftrightarrow \sin\left(\dfrac{\pi}{4}-2x\right)=1\\
		&\Leftrightarrow x=-\dfrac{\pi}{8}-k\pi
	\end{aligned}$.\\
	Khi đó: $0<x<3\pi\Leftrightarrow k\in \{-1; -2; -3\}$, suy ra trong khoảng $\left(0; 3\pi\right)$ phương trình đã cho có $3$ nghiệm.}
\end{ex}

%%==========Câu 15
\begin{ex}%[DCHT Toán 11 - KNTT -Tên Nguyễn Thanh Sang] %[1K1G4-8]
Gọi $S$ là tập hợp tất cả các nghiệm thuộc khoảng $(0;2024)$ của phương trình\\
$\sqrt{3}(1-\cos 2x)+\sin 2x-4\cos x+8=4\left(\sqrt{3}+1\right)\sin x.$
 Tìm số các phần tử của tập hợp $S.$
\choice
{\True $322$}
{ $323$}
{$321$}
{$324$}
\loigiai{
	Ta có
	\begin{align*}
		&\sqrt{3}(1-\cos 2x)+\sin 2x-4\cos x+8=4\left(\sqrt{3}+1\right)\sin x\\
		\Leftrightarrow & \sqrt{3}(1-\cos 2x-4\sin x)+\sin 2x-4\cos x+8-4\sin x=0\\
		\Leftrightarrow & \sqrt{3}\left(2\sin^2x-4\sin x\right)+2\sin x\cdot\cos x-4\cos x-4\sin x+8=0\\
		\Leftrightarrow & 2\sqrt{3}\sin x\cdot \left(\sin x-2\right)+2(\sin x-2)\cdot (\cos x-2)=0\\
		\Leftrightarrow & 2(\sin x-2)\cdot \left(\sqrt{3}\sin x+\cos x-2\right)=0\\
		\Leftrightarrow & (\sin x-2)\left(\dfrac{\sqrt{3}}{2}\sin x+\dfrac{1}{2}\cos x-1\right)=0\\
		\Leftrightarrow & (\sin x-2)\left[\sin\left(x+\dfrac{\pi}{6}\right)-1\right]=0\\
		\Leftrightarrow & \sin \left(x+\dfrac{\pi}{6}\right)=1\quad (\text{vì } \sin x-2<0, \forall x\in \Bbb{R})\\
		\Leftrightarrow & x+\dfrac{\pi}{6}=\dfrac{\pi}{2}+k2\pi\\
		\Leftrightarrow & x=\dfrac{\pi}{3}+k2\pi.
	\end{align*}
	Xét $ x=\dfrac{\pi}{3}+k2\pi\in (0;2024)\Rightarrow 0<\dfrac{\pi}{3}+k2\pi<2024\Leftrightarrow -\dfrac{1}{6}<k<\dfrac{1012}{\pi}-\dfrac{1}{6}.$\\
	Vì $k\in \Bbb{Z}$ nên $k\in \{0;1;2;\ldots;321;321\}.$\\
	Vậy tập $S$ có $322$ phần tử.
	}
\end{ex}
\Closesolutionfile{ans}
\begin{indapan}{10}
	{ans/ans-1K1-3-Dang1}
\end{indapan}
