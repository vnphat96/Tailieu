
\begin{dang}{Kết hợp nhiều công thức lượng giác}
\end{dang}
\subsubsection{Ví dụ minh hoạ}
\begin{vd}[VDT]
Chứng minh rằng $4\cos x\cos \left(\dfrac{\pi}{3}-x\right)\cos \left(\dfrac{\pi}{3}+x\right)=\cos 3x$, với mọi $x \in \mathbb{R}$.
\loigiai{Ta có $4\cos x\cos \left(\dfrac{\pi}{3}-x\right)\cos \left(\dfrac{\pi}{3}+x\right)=4\cos x\cdot \dfrac{1}{2}\left[\cos (-2x)+\cos \dfrac{2\pi}{3}\right]$
		$$=2\cos x\cos 2x-\cos x=\cos 3x+\cos (-x)-\cos x=\cos 3x, \forall x \in \mathbb{R}.$$
		Vậy $4\cos x\cos \left(\dfrac{\pi}{3}-x\right)\cos \left(\dfrac{\pi}{3}+x\right)=\cos 3x$, với mọi $x \in \mathbb{R}$.}
\end{vd}
\begin{vd}[VDT]
Chứng minh rằng với mọi $a \in \mathbb{R}$: $\cos^3a\cos 3a-\sin^3a\sin 3a=\dfrac{3}{4}\cos 4a+\dfrac{1}{4}$.
\loigiai{Ta có\\
		$\begin{aligned}\cos^3a\cos 3a-\sin^3a\sin 3a&=\left(\cos 3a\cos a\right)\cos^2a-\left(\sin 3a\sin a\right)\sin^2a \\
		&=\dfrac{1}{2}\left[\cos 2a+\cos 4a\right]\cos^2a-\dfrac{1}{2}\left[\cos 2a-\cos 4a\right]\sin^2a \\
		&=\dfrac{1}{2}\cos 2a\cos^2a+\dfrac{1}{2}\cos 4a\cos^2a-\dfrac{1}{2}\cos 2a\sin^2a+\dfrac{1}{2}\cos 4a\sin^2a \\
		&=\dfrac{1}{2}\cos 2a\left(\cos^2a-\sin^2a\right)+\dfrac{1}{2}\cos 4a\left(\cos^2a+\sin^2a\right)\\
&=\dfrac{1}{2}\cos 2a\cos 2a+\dfrac{1}{2}\cos 4a\\
&=\dfrac{1}{4}\left(\cos 4a+\cos 0\right)+\dfrac{1}{2}\cos 4a\\
&=\dfrac{3}{4}\cos 4a+\dfrac{1}{4}, \forall x \in \mathbb{R}.
\end{aligned}$\\
		Vậy $\cos^3a\cos 3a-\sin^3a\sin 3a=\dfrac{3}{4}\cos 4a+\dfrac{1}{4}$, với mọi $x \in \mathbb{R}.$}
\end{vd}
\begin{vd}[VDT]
Chứng minh rằng giá trị của biểu thức sau đây không phụ thuộc vào biến số $x$:
$$S=\cos^2x+\cos^2\left(\dfrac{2\pi}{3}+x\right)+\cos^2\left(\dfrac{2\pi}{3}-x\right).$$
\loigiai{Ta có
$$\begin{aligned}
S&=\cos^2x+\cos^2\left(\dfrac{2\pi}{3}+x\right)+\cos^2\left(\dfrac{2\pi}{3}-x\right)\cr
&=\dfrac{1+\cos 2x}{2}+\dfrac{1+\cos \left(\dfrac{4\pi}{3}+2x\right)}{2}+\dfrac{1+\cos \left(\dfrac{4\pi}{3}-2x\right)}{2}\cr
&=\dfrac{3}{2}+\dfrac{1}{2}\cos 2x+\dfrac{1}{2}\left[\cos \left(\dfrac{4\pi}{3}+2x\right)+\cos \left(\dfrac{4\pi}{3}-2x\right)\right]\cr
&=\dfrac{3}{2}+\dfrac{1}{2}\cos 2x+\dfrac{1}{2}\cdot 2\cos \dfrac{4\pi}{3}\cos 2x =\dfrac{3}{2}+\dfrac{1}{2}\cos 2x+\dfrac{1}{2}\cdot 2\cdot \left(-\dfrac{1}{2}\right)\cos 2x=\dfrac{3}{2}.
\end{aligned}$$
Vậy $S=\dfrac{3}{2}$ với mọi $x\in \mathbb{R}$ (không phụ thuộc vào biến số $x$).}
\end{vd}
\begin{vd}[VDT]
Rút gọn biểu thức $A=2\sin x(\cos x+\cos 3x+\cos 5x)$. \\
Từ đó tính giá trị biểu thức  $T=\cos\dfrac{\pi}{7}+\cos\dfrac{3\pi}{7}+\cos\dfrac{5\pi }{7}$.
\dapso{$A=\sin6x$; $T=\dfrac12$}
\loigiai{Ta có
$$\begin{aligned}
A&=2\sin x(\cos x+\cos 3x+\cos 5x)=2\sin x\cos x+2\sin x\cos 3x+2\sin x\cos 5x\cr
&=\sin 2x + \sin 4x + \sin (-2x) + \sin 6x + \sin (-4x)=\sin 2x+\sin 4x-\sin 2x+\sin 6x-\sin 4x.
\end{aligned}$$
		Như vậy,  $A=2\sin x(\cos x+\cos 3x+\cos 5x)=\sin 6x$.\\
		Áp dụng kết quả trên, ta có\\
		$T=\cos\dfrac{\pi}{7}+\cos\dfrac{3\pi}{7}+\cos\dfrac{5\pi }{7} \Rightarrow T\cdot 2\sin\dfrac{\pi}{7}=2\sin\dfrac{\pi}{7}\left(\cos\dfrac{\pi}{7}+\cos\dfrac{3\pi}{7}+\cos\dfrac{5\pi }{7} \right)=\sin\dfrac{6\pi}{7}$.\\
		Do $\sin \dfrac{6\pi}{7}=\sin\left(\pi-\dfrac{\pi}{7} \right)=\sin\dfrac{\pi}{7}$ nên $T\cdot 2\sin\dfrac{\pi}{7}=\sin\dfrac{\pi}{7} \Rightarrow T=\dfrac{1}{2}$.}
\end{vd}
\begin{vd}[VDT]
Tính giá trị biểu thức $A=\sin^210^\circ+\cos 70^\circ\cos 50^\circ$.
\dapso{$A=\dfrac14$}
\loigiai{Ta có
$$\begin{aligned}
A&=\sin^210{}^\circ+\cos 70^\circ\cos 50^\circ=\dfrac{1-\cos 20^\circ}{2}+\dfrac{1}{2}\left[\cos 120^\circ+\cos 20^\circ\right]\cr
&=\dfrac{1}{2}-\dfrac{1}{2}\cos 20^\circ+\dfrac{1}{2}\cos 120^\circ+\dfrac{1}{2}\cos 20^\circ=\dfrac{1}{2}+\dfrac{1}{2}\cos 120^\circ=\dfrac{1}{2}-\dfrac{1}{4}=\dfrac{1}{4}.
\end{aligned}
$$
Vậy $A=\sin^210^\circ+\cos 70^\circ\cos 50^\circ=\dfrac{1}{4}$.}
\end{vd}

\subsubsection{Bài tập rèn luyện}
% \subsubsection{Bài tập tự luận}
\begin{bt}[VDT]
	Chứng minh các đẳng thức sau đây:
	\begin{enumerate}
		\item $\cos a+\cos b+\sin (a+b)=4\cos \dfrac{a+b}{2}\cos \left(\dfrac{\pi}{4}-\dfrac{a}{2}\right)\sin \left(\dfrac{\pi}{4}+\dfrac{b}{2}\right)$
		\item $\sin^2a+\sin^2b+2\sin a\sin b\cos (a+b)=\sin^2(a+b)$
		\item $\sin \left(2x+\dfrac{\pi}{3}\right)\cos \left(x-\dfrac{\pi}{6}\right)-\cos \left(2x+\dfrac{\pi}{3}\right)\cos \left(\dfrac{2\pi}{3}-x\right)=\cos x$
	\end{enumerate}
\loigiai{
\begin{enumerate}
			\item Chứng minh $\cos a+\cos b+\sin (a+b)=4\cos \dfrac{a+b}{2}\cos \left(\dfrac{\pi}{4}-\dfrac{a}{2}\right)\sin \left(\dfrac{\pi}{4}+\dfrac{b}{2}\right)$.\\
			Ta có
\begin{align*}
\cos a+\cos b+\sin (a+b)&=2\cos \dfrac{a+b}{2}\cos \dfrac{a-b}{2}+2\sin \dfrac{a+b}{2}\cos \dfrac{a+b}{2}\cr
&=2\cos \dfrac{a+b}{2}\left(\cos \dfrac{a-b}{2}+\sin \dfrac{a+b}{2}\right)=2\cos \dfrac{a+b}{2}\left(\sin \dfrac{\pi -a+b}{2}+\sin \dfrac{a+b}{2}\right) \\
&=2\cos \dfrac{a+b}{2}\cdot 2\sin \left(\dfrac{\pi}{4}+\dfrac{b}{2}\right)\cos \left(\dfrac{\pi}{4}-\dfrac{a}{2}\right).
\end{align*}
			\item Chứng minh $\sin^2a+\sin^2b+2\sin a\sin b\cos (a+b)=\sin^2(a+b)$.\\
			Ta có
\begin{align*}
&\sin^2a+\sin^2b+2\sin a\sin b\cos (a+b)\cr
=&\left[\dfrac{1-\cos 2a}{2}+\dfrac{1-\cos 2b}{2}\right]+2\sin a\sin b\cos (a+b)\cr
=&\left[1-\dfrac{1}{2}\left(\cos 2a+\cos 2b\right)\right]+\left[\cos (a-b)-\cos (a+b)\right]\cos (a+b)\cr
=&\left[1-\cos (a+b)\cos (a-b)\right]+\left[\cos (a-b)\cos (a+b)-\cos^2(a+b)\right]\cr
=&1-\cos^2(a+b)=\sin^2(a+b).
\end{align*}
			\item Chứng minh $\sin \left(2x+\dfrac{\pi}{3}\right)\cos \left(x-\dfrac{\pi}{6}\right)-\cos \left(2x+\dfrac{\pi}{3}\right)\cos \left(\dfrac{2\pi}{3}-x\right) = \cos x$.\\
			Ta có
\begin{align*}
&\sin \left(2x+\dfrac{\pi}{3}\right)\cos \left(x-\dfrac{\pi}{6}\right)-\cos \left(2x+\dfrac{\pi}{3}\right)\cos \left(\dfrac{2\pi}{3}-x\right)\\
=&\dfrac{1}{2}\left[\sin \left(3x+\dfrac{\pi}{6}\right)+\sin \left(x+\dfrac{\pi}{2}\right)\right]-\dfrac{1}{2}\left[\cos\left(x+\pi \right)+\cos \left(3x-\dfrac{\pi}{3} \right) \right]\\
=&\dfrac{1}{2}\left[\sin \left(3x+\dfrac{\pi}{6} \right)-\cos \left(3x-\dfrac{\pi}{3} \right) +\sin \left(x+\dfrac{\pi}{2} \right) -\cos \left(x+\pi \right) \right]\\
=&\dfrac{1}{2}\left[\sin \left(3x+\dfrac{\pi}{6} \right)-\cos \left(\dfrac{\pi}{3} -3x \right) +\cos x - (-\cos x) \right]\\
=&\dfrac{1}{2}\left[\sin \left(3x+\dfrac{\pi}{6} \right) - \sin \left(3x+\dfrac{\pi}{6} \right) + 2\cos x \right] = \cos x.
\end{align*}
\end{enumerate}
}
\end{bt}

\begin{bt}[VDT]
Chứng minh giá trị của biểu thức sau không phụ thuộc vào biến số $x$:
$$A=\cos \left(\dfrac{\pi}{3}-x\right)\cos \left(\dfrac{\pi}{4}+x\right)+\cos \left(\dfrac{\pi}{6}+x\right)\cos \left(\dfrac{3\pi}{4}+x\right).$$
\loigiai{
Ta có
\begin{align*}
A&=\cos \left(\dfrac{\pi}{3}-x\right)\cos \left(\dfrac{\pi}{4}+x\right)+\cos \left(\dfrac{\pi}{6}+x\right)\cos \left(\dfrac{3\pi}{4}+x\right)\cr
&=\dfrac{1}{2}\left[\cos \left(\dfrac{\pi}{12}-2x\right)+\cos \dfrac{7\pi}{12}\right]+\dfrac{1}{2}\left[\cos \left(-\dfrac{7\pi}{12}\right)+\cos \left(\dfrac{11\pi}{12}+2x\right)\right]\cr
&=\dfrac{1}{2}\left[\cos \left(\dfrac{11\pi}{12}+2x\right)+\cos \left(\dfrac{\pi}{12}-2x\right)+\cos \dfrac{7\pi}{12}+\cos \left(-\dfrac{7\pi}{12}\right)\right]\cr
&=\dfrac{1}{2}\left[0+2\cos \dfrac{7\pi}{12}\right]=\cos \dfrac{7\pi}{12} ~\left(\text{do } \dfrac{11\pi}{12}+2x+\dfrac{\pi}{12}-2x=\pi \right).
\end{align*}
Vậy $A=\cos \dfrac{7\pi}{12}$ với mọi $x\in \mathbb{R}$ (không phụ thuộc vào biến số $x$).
}
\end{bt}

\begin{bt}%[TH]
Rút gọn các biểu thức sau đây:
\begin{enumEX}{2}
			\item $A=\dfrac{\cos 4a-\cos 2a}{\sin 4a-\sin 2a}$;
			\item $B=\dfrac{\sin a-2\sin 2a+\sin 3a}{\cos a-2\cos 2a+\cos 3a}$.
\end{enumEX}
\dapso{$A=-\tan3a$; $B=\tan2a$}
\loigiai{
\begin{enumerate}
			\item Ta có $A=\dfrac{\cos 4a-\cos 2a}{\sin 4a-\sin 2a}=\dfrac{-2\sin 3a\sin a}{2\cos 3a\sin a}=-\dfrac{\sin 3a}{\cos 3a}=-\tan 3a$.
			\item Ta có
\begin{align*}
B&=\dfrac{\sin a-2\sin 2a+\sin 3a}{\cos a-2\cos 2a+\cos 3a}=\dfrac{\left(\sin 3a+\sin a\right)-2\sin 2a}{\left(\cos 3a+\cos a\right)-2\cos 2a}\cr
&=\dfrac{2\sin 2a\cos a-2\sin 2a}{2\cos 2a\cos a-2\cos 2a}=\dfrac{2\sin 2a\left(\cos a-1\right)}{2\cos 2a\left(\cos a-1\right)}=\dfrac{\sin 2a}{\cos 2a}=\tan 2a.
\end{align*}
\end{enumerate}
}
\end{bt}

\begin{bt}[VDT]
Rút gọn các biểu thức:
\begin{enumEX}{2}
\item $A=4\sin \dfrac{x}{3} \sin \dfrac{x+\pi}{3} \sin \dfrac{x-\pi}{3}$;
\item $B=\dfrac{\cos^2a-\cos^2b}{\sin (a-b)}$.
\end{enumEX}
\loigiai{
\begin{enumerate}
\item Ta có 
\begin{align*}
A&=4\sin \dfrac{x}{3} \sin \dfrac{x+\pi}{3} \sin \dfrac{x-\pi}{3}=4\sin \dfrac{x}{3}\cdot \dfrac{1}{2}\left[\cos \dfrac{2\pi}{3}-\cos \dfrac{2x}{3}\right]\cr
&=-\sin \dfrac{x}{3}+2\sin \dfrac{x}{3} \cos \dfrac{2x}{3}=-\sin \dfrac{x}{3}+\sin x-\sin \dfrac{x}{3}\cr
&=\sin x.
\end{align*}
\item Ta có
\begin{align*}
B&=\dfrac{\cos^2a-\cos^2b}{\sin (a-b)}=\dfrac{\left(1+\cos 2a\right)-\left(1+\cos 2b\right)}{2\sin (a-b)}\cr
&=\dfrac{\cos 2a-\cos 2b}{2\sin (a-b)}=\dfrac{-2\sin (a+b)\sin (a-b)}{2\sin (a-b)}\cr
&=-\sin (a+b).
\end{align*}
\end{enumerate}
}
\end{bt}

\begin{bt}
Tính giá trị các biểu thức:
\begin{enumEX}{2}
\item $A=\cos \dfrac{2\pi}{7}+\cos \dfrac{4\pi}{7}+\cos \dfrac{6\pi}{7}$;
\item $B=\tan 9^\circ-\tan 27^\circ-\tan 63^\circ+\tan 81^\circ$.
\end{enumEX}
\loigiai{
\begin{enumerate}
\item Ta có
$$ \begin{aligned}
A\cdot\sin \dfrac{\pi}{7} &=\sin \dfrac{\pi}{7} \cos \dfrac{2\pi}{7}+\sin \dfrac{\pi}{7} \cos \dfrac{4\pi}{7}+\sin \dfrac{\pi}{7} \cos \dfrac{6\pi}{7} \\
			&=\dfrac{1}{2}\left[\sin \dfrac{3\pi}{7}-\sin \dfrac{\pi}{7}+\sin \dfrac{5\pi}{7}-\sin \dfrac{3\pi}{7}+\sin \dfrac{7\pi}{7}-\sin \dfrac{5\pi}{7}\right] \\
			&=\dfrac{1}{2}\left(-\sin \dfrac{\pi}{7}\right)=-\dfrac{1}{2}\sin \dfrac{\pi}{7}\\
			\Rightarrow A&=-\dfrac{1}{2}.
\end{aligned}$$
\item Ta có
\begin{align*}
B&=\tan 9^\circ-\tan 27^\circ-\tan 63^\circ+\tan 81^\circ=\tan 9^\circ+\tan 81^\circ-\left(\tan 27^\circ+\tan 63^\circ\right)\cr
&=\dfrac{\sin 9^\circ}{\cos 9^\circ}+\dfrac{\sin 81^\circ}{\cos 81^\circ}-\left(\dfrac{\sin 27^\circ}{\cos 27^\circ}+\dfrac{\sin 63^\circ}{\cos 63^\circ}\right)\cr
&=\dfrac{\sin 9^\circ\cos 81^\circ+\cos 9^\circ\sin 81^\circ}{\cos 9^\circ\cos 81^\circ}-\dfrac{\sin 27^\circ\cos 63^\circ+\cos 27^\circ\sin 63^\circ}{\cos 27^\circ\cos 63^\circ}\cr
&=\dfrac{\sin 90^\circ}{\cos 9^\circ\sin 9^\circ}-\dfrac{\sin 90^\circ}{\cos 27^\circ\sin 27^\circ}=\dfrac{2}{\sin 18^\circ}-\dfrac{2}{\sin 54^\circ}=\dfrac{2\left(\sin 54^\circ-\sin 18^\circ\right)}{\sin 54^\circ\sin 18^\circ}\cr
&=\dfrac{4\cos 36^\circ\sin 18^\circ}{\sin 54^\circ\sin 18^\circ}=4 \left(\text{ do } \cos 36^\circ=\sin 54^\circ\right).
\end{align*}
\end{enumerate}
}
\end{bt}

\subsubsection{Bài tập trắc nghiệm}
\Opensolutionfile{ans}[ans/ans-1K1-2-Dang4]
\begin{ex}
%[0D6B3-2]
Rút gọn biểu thức $M=\cos ^415^{\circ}-\sin ^415^{\circ}.$
\choice
{$M=1$}
{\True $M=\dfrac{{\sqrt{3}}}{2}$}
{$M=\dfrac{1}{4}$}
{$M=0$}

\loigiai{Ta có
\begin{align*}
M&=\cos ^415^{\circ}-\sin ^415^{\circ}=\left({{\cos}^2{15}^{\circ}}\right)^2-\left({{\sin}^2{15}^{\circ}}\right)^2\cr
&=\left({{\cos}^2{15}^{\circ}-{\sin}^2{15}^{\circ}}\right)\left({{\cos}^2{15}^{\circ}+{\sin}^2{15}^{\circ}}\right)\cr
&=\cos ^215^{\circ}-\sin ^215^{\circ}=\cos \left({{2.15}^{\circ}}\right)=\cos 30^{\circ}=\dfrac{{\sqrt{3}}}{2}.
\end{align*}
}
\end{ex}
\begin{ex}
%[0D6B3-2]
Tính giá trị của biểu thức $M=\cos ^415^{\circ}-\sin ^415^{\circ}+\cos ^215^{\circ}-\sin ^215^{\circ}.$
\choice
{\True $M=\sqrt{3}$}
{$M=\dfrac{1}{2}$}
{$M=\dfrac{1}{4}$}
{$M=0$}

\loigiai{Áp dụng công thức nhân đôi $\cos ^2a-\sin ^2a=\cos 2a$.\\
Ta có
\begin{eqnarray*}
 M & = & \left({{\cos}^4{15}^{\circ}-{\sin}^4{15}^{\circ}}\right)+\left({{\cos}^2{15}^{\circ}-{\sin}^2{15}^{\circ}}\right)\\
& =& \left({{\cos}^2{15}^{\circ}-{\sin}^2{15}^{\circ}}\right)\left({{\cos}^2{15}^{\circ}+{\sin}^2{15}^{\circ}}\right)+\left({{\cos}^2{15}^{\circ}-{\sin}^2{15}^{\circ}}\right)\\
& = & \left({{\cos}^2{15}^{\circ}-{\sin}^2{15}^{\circ}}\right)+\left({{\cos}^2{15}^{\circ}-{\sin}^2{15}^{\circ}}\right)\\
& = &\cos 30^{\circ}+\cos 30^{\circ}=\sqrt{3}.
\end{eqnarray*}
}
\end{ex}
\begin{ex}
%[0D6B3-2]
Tính giá trị của biểu thức $M=\cos ^615^{\circ}-\sin ^615^{\circ}.$
\choice
{$M=1$}
{$M=\dfrac{1}{2}$}
{$M=\dfrac{1}{4}$}
{\True $M=\dfrac{{15\sqrt{3}}}{{32}}$}

\loigiai{Ta có 
\begin{eqnarray*}
\cos ^6\alpha-\sin ^6\alpha & = & \left({\cos}^2\alpha-{\sin}^2\alpha\right)\left({{\cos}^4\alpha+{\cos}^2\alpha\cdot {\sin}^2\alpha+{\sin}^4\alpha}\right) \\
&& =\cos 2\alpha\cdot \left[{{\left({{\cos}^2\alpha+{\sin}^2\alpha}\right)}^2-{\cos}^2\alpha \cdot {\sin}^2\alpha}\right] \\
&&=\cos 2\alpha \cdot \left({1-\dfrac{1}{4}{\sin}^22\alpha}\right). 
\end{eqnarray*}
Vậy $M=\cos 30^{\circ}\cdot \left(1-\dfrac{1}{4}{\sin}^2{30}^{\circ}\right)=\dfrac{\sqrt{3}}{2}\cdot \left(1-\dfrac{1}{4}\cdot\dfrac{1}{4}\right)=\dfrac{15\sqrt{3}}{32}.$
}
\end{ex}
\begin{ex}
%[0D6B3-1]
Giá trị của biểu thức $\cos \dfrac{\pi}{{30}}\cos \dfrac{\pi}{5}+\sin \dfrac{\pi}{{30}}\sin \dfrac{\pi}{5}$ là
\choice
{\True $\dfrac{{\sqrt{3}}}{2}$}
{$-\dfrac{{\sqrt{3}}}{2}$}
{$\dfrac{{\sqrt{3}}}{4}$}
{$\dfrac{1}{2}$}

\loigiai{Ta có $\cos \dfrac{\pi}{{30}}\cos \dfrac{\pi}{5}+\sin \dfrac{\pi}{{30}}\sin \dfrac{\pi}{5}=\cos \left({\dfrac{\pi}{{30}}-\dfrac{\pi}{5}}\right)=\cos \left({-\dfrac{\pi}{6}}\right)=\dfrac{{\sqrt{3}}}{2}.$}
\end{ex}
\begin{ex}
%[0D6B3-1]
Giá trị của biểu thức $P=\dfrac{{\sin \dfrac{{5\pi}}{{18}}\cos \dfrac{\pi}{9}-\sin \dfrac{\pi}{9}\cos \dfrac{{5\pi}}{{18}}}}{{\cos \dfrac{\pi}{4}\cos \dfrac{\pi}{{12}}-\sin \dfrac{\pi}{4}\sin \dfrac{\pi}{{12}}}}$ là
\choice
{\True $1$}
{$\dfrac{1}{2}$}
{$\dfrac{{\sqrt{2}}}{2}$}
{$\dfrac{{\sqrt{3}}}{2}$}

\loigiai{Áp dụng công thức $\heva{& \sin a\cdot \cos b-\cos a\cdot \sin b=\sin \left({a-b}\right) \\
& \cos a\cdot\cos b-\sin a\cdot\sin b=\cos \left({a+b}.\right)}$\\
Khi đó $\sin \dfrac{{5\pi}}{{18}}\cos \dfrac{\pi}{9}-\sin \dfrac{\pi}{9}\cos \dfrac{{5\pi}}{{18}}=\sin \left({\dfrac{{5\pi}}{{18}}-\dfrac{\pi}{9}}\right)=\sin \dfrac{\pi}{6}=\dfrac{1}{2}.$\\
Và $\cos \dfrac{\pi}{4}\cos \dfrac{\pi}{{12}}-\sin \dfrac{\pi}{4}\sin \dfrac{\pi}{{12}}=\cos \left({\dfrac{\pi}{4}+\dfrac{\pi}{{12}}}\right)=\cos \dfrac{\pi}{3}=\dfrac{1}{2}.$ Vậy $P=\dfrac{1}{2}:\dfrac{1}{2}=1.$
}
\end{ex}
\begin{ex}
%[0D6B3-1]
Giá trị đúng của biểu thức $\dfrac{{\tan {225}^{\circ}-\cot {81}^{\circ}\cdot \cot{69}^{\circ}}}{{\cot {261}^{\circ}+\tan {201}^{\circ}}}$ bằng
\choice
{$\dfrac{1}{{\sqrt{3}}}$}
{$-\dfrac{1}{{\sqrt{3}}}$}
{\True $\sqrt{3}$}
{$-\sqrt{3}$}

\loigiai{Ta có $\dfrac{{\tan {225}^{\circ}-\cot {81}^{\circ}\cdot \cot{69}^{\circ}}}{{\cot {261}^{\circ}+\tan {201}^{\circ}}}=\dfrac{{\tan \left({{180}^{\circ}+{45}^{\circ}}\right)-\tan 9^{\circ}\cdot \cot{69}^{\circ}}}{{\cot \left({{180}^{\circ}+{81}^{\circ}}\right)+\tan \left({{180}^{\circ}+{21}^{\circ}}\right)}} =\dfrac{{1-\tan 9^{\circ}\cdot \tan {21}^{\circ}}}{{\tan 9^{\circ}+\tan {21}^{\circ}}} $ 
$=\dfrac{1}{{\tan \left({9^{\circ}+{21}^{\circ}}\right)}}=\dfrac{1}{{\tan {30}^{\circ}}}=\sqrt{3}.$
}
\end{ex}
\begin{ex}
%[0D6K3-4]
Giá trị của biểu thức $M=\sin \dfrac{\pi}{{24}}\sin \dfrac{{5\pi}}{{24}}\sin \dfrac{{7\pi}}{{24}}\sin \dfrac{{11\pi}}{{24}}$ bằng
\choice
{$\dfrac{1}{2}$}
{$\dfrac{1}{4}$}
{$\dfrac{1}{8}$}
{\True $\dfrac{1}{{16}}$}

\loigiai{Ta có $\sin \dfrac{{7\pi}}{{24}}=\cos \dfrac{{5\pi}}{{24}}$ và $\sin \dfrac{{11\pi}}{{24}}=\cos \dfrac{\pi}{{24}}$. \\
Do đó $M=\sin \dfrac{\pi}{{24}}\sin \dfrac{{5\pi}}{{24}}\cos \dfrac{{5\pi}}{{24}}\cos \dfrac{\pi}{{24}}=\dfrac{1}{4}\cdot \left({2\sin \dfrac{\pi}{{24}}\cdot \cos \dfrac{\pi}{{24}}}\right)\cdot \left({2\sin \dfrac{{5\pi}}{{24}}\cdot\cos \dfrac{{5\pi}}{{24}}}\right)$\\
$=\dfrac{1}{4}\cdot \sin \dfrac{\pi}{{12}}\cdot\sin \dfrac{{5\pi}}{{12}}=\dfrac{1}{4}\cdot \dfrac{1}{2}\left({\cos \dfrac{{6\pi}}{{12}}+\cos \dfrac{\pi}{3}}\right)=\dfrac{1}{8}\cdot \left({0+\dfrac{1}{2}}\right)=\dfrac{1}{{16}}.$
}
\end{ex}
\begin{ex}
%[0D6K3-4]
Giá trị của biểu thức $M=\sin \dfrac{\pi}{48} \cos\dfrac{\pi}{48} \cos \dfrac{ \pi}{24} \cos \dfrac{\pi}{12} \cos \dfrac{\pi}{6}$ là
\choice
{$\dfrac{1}{32}$}
{$\dfrac{\sqrt{3}}{8}$}
{$\dfrac{\sqrt{3}}{16}$}
{\True $\dfrac{\sqrt{3}}{32}$}

\loigiai{Áp dụng công thức $\sin 2a=2\sin a\cdot \cos a,$ ta có\\
$A=\sin \dfrac{\pi}{{48}}\cdot \cos \dfrac{\pi}{{48}}\cdot \cos \dfrac{\pi}{{24}}\cdot\cos\dfrac{\pi}{{12}}\cdot\cos\dfrac{\pi}{6}=\dfrac{1}{2}\cdot\sin \dfrac{\pi}{{24}}\cdot\cos \dfrac{\pi}{{24}}\cdot \cos\dfrac{\pi}{{12}}\cdot \cos\dfrac{\pi}{6}$\\
$=\dfrac{1}{4}\cdot \sin \dfrac{\pi}{{12}}\cdot\cos \dfrac{\pi}{{12}}\cdot\cos \dfrac{\pi}{6}=\dfrac{1}{8}\cdot\sin \dfrac{\pi}{6}\cdot\cos \dfrac{\pi}{6}=\dfrac{1}{{16}}\cdot\sin \dfrac{\pi}{3}=\dfrac{{\sqrt{3}}}{{32}}.$
}
\end{ex}
\begin{ex}
%[0D6K3-4]
Tính giá trị của biểu thức $M=\cos 10^{\circ}\cos 20^{\circ}\cos 40^{\circ}\cos 80^{\circ}.$
\choice
{$M=\dfrac{1}{{16}}\cos 10^{\circ}$}
{$M=\dfrac{1}{2}\cos 10^{\circ}$}
{$M=\dfrac{1}{4}\cos 10^{\circ}$}
{\True $M=\dfrac{1}{8}\cos 10^{\circ}$}

\loigiai{Vì $\sin 10^{\circ}\ne 0$ nên suy ra \\
$M=\dfrac{{16\sin {10}^{\circ}\cos {10}^{\circ}\cos {20}^{\circ}\cos {40}^{\circ}\cos {80}^{\circ}}}{{16\sin {10}^{\circ}}}=\dfrac{{8\sin {20}^{\circ}\cos {20}^{\circ}\cos {40}^{\circ}\cos {80}^{\circ}}}{{16\sin {10}^{\circ}}}$\\
$\Rightarrow M=\dfrac{{4\sin {40}^{\circ}\cos {40}^{\circ}\cos {80}^{\circ}}}{{16\sin {10}^{\circ}}}=\dfrac{{2\sin {80}^{\circ}\cos {80}^{\circ}}}{{16\sin {10}^{\circ}}}=\dfrac{{\sin {160}^{\circ}}}{{16\sin {10}^{\circ}}}$.\\
$\Rightarrow M=\dfrac{{\sin {20}^{\circ}}}{{16\sin {10}^{\circ}}}=\dfrac{{2\sin {10}^{\circ}\cos {10}^{\circ}}}{{16\sin {10}^{\circ}}}=\dfrac{1}{8}\cos 10^{\circ}$.
}
\end{ex}
\begin{ex}
%[0D6K3-4]
Tính giá trị của biểu thức $M=\cos \dfrac{{2\pi}}{7}+\cos \dfrac{{4\pi}}{7}+\cos \dfrac{{6\pi}}{7}.$
\choice
{$M=0$}
{\True $M=-\dfrac{1}{2}$}
{$M=1$}
{$M=2$}

\loigiai{Áp dụng công thức $\sin a-\sin b=2\cdot\cos \dfrac{{a+b}}{2}\cdot \sin \dfrac{{a-b}}{2}.$ \\
Ta có
\begin{align*}
 2\sin \dfrac{\pi}{7}\cdot M&=2\cdot\cos \dfrac{{2\pi}}{7}\cdot \sin \dfrac{\pi}{7}+2\cdot\cos \dfrac{{4\pi}}{7}\cdot\sin \dfrac{\pi}{7}+2\cdot\cos \dfrac{{6\pi}}{7}\cdot\sin \dfrac{\pi}{7}\\
&=\sin \dfrac{{3\pi}}{7}-\sin \dfrac{\pi}{7}+\sin \dfrac{{5\pi}}{7}-\sin \dfrac{{3\pi}}{7}+\sin \dfrac{{7\pi}}{7}-\sin \dfrac{{5\pi}}{7}\cr
&=-\sin \dfrac{\pi}{7}+\sin \pi =-\sin \dfrac{\pi}{7}.
\end{align*}
Vậy giá trị biểu thức $M=-\dfrac{1}{2}$.}
\end{ex}
\begin{ex}
%Câu 72
	Rút gọn biểu thức $M=\cos^2\left(\dfrac{\pi}{4}+\alpha\right)-\cos^2\left(\dfrac{\pi}{4}-\alpha\right).$
\choice
{$M=\sin 2\alpha $}
{$M=\cos 2\alpha $}
{$M=-\cos 2\alpha $}
{\True $M=-\sin 2\alpha $}
\loigiai{
Ta có
\begin{align*}
M&=\cos^2\left(\dfrac{\pi}{4}+\alpha\right)-\cos^2\left(\dfrac{\pi}{4}-\alpha\right)\cr
&=\dfrac{1-\cos\left(\dfrac\pi2+2\alpha\right)}{2}-\dfrac{1-\cos\left(\dfrac\pi2-2\alpha\right)}{2}\cr
&=\dfrac12(\sin2\alpha+\sin2\alpha)\cr
&=\sin2\alpha.
\end{align*}
}
\end{ex}
\begin{ex}
%Câu 75
	Gọi $ M=\cos x+\cos 2x+\cos 3x$ thì
\choice
{$ M=2\cos 2x\left(\cos x+1\right)$}
{$ M=4\cos 2x.\left(\dfrac{1}{2}+\cos x\right)$}
{$ M=\cos 2x\left(2\cos x-1\right)$}
{\True $ M=\cos 2x\left(2\cos x+1\right)$}
\loigiai{
Ta có
\begin{align*}
M*=\cos x+\cos 2x+\cos 3x=2\cos2x\cos x+\cos2x=\cos2x(2\cos x+1).
\end{align*}
}
\end{ex}
\begin{ex}
%Câu 76
	Rút gọn biểu thức $ M=\dfrac{\sin 3x-\sin x}{2\cos^2x-1}$.
\choice
{$\tan 2x$}
{$\sin x$}
{$ 2\tan x$}
{\True $ 2\sin x$}
\loigiai{
Ta có
\begin{align*}
M&=\dfrac{\sin 3x-\sin x}{2\cos^2x-1}=\dfrac{2\cos2x\sin x}{\cos2x}=2\sin x.
\end{align*}
}
\end{ex}
\begin{ex}
%Câu 77
	Rút gọn biểu thức $A=\dfrac{1+\cos x+\cos 2x+\cos 3x}{2\cos^2x+\cos x-1}$ .
\choice
{$\cos x$}
{$ 2\cos x-1$}
{\True $ 2\cos x$}
{$\cos x-1$}
\loigiai{
Ta có
\begin{align*}
A&=\dfrac{1+\cos x+\cos 2x+\cos 3x}{2\cos^2x+\cos x-1}=\dfrac{2\cos^2x+2\cos2x\cos x}{\cos2x+\cos x}\cr
&=\dfrac{2\cos x(\cos2x+\cos x)}{\cos2x+\cos x}\cr
&=2\cos x.
\end{align*}
}
\end{ex}
\begin{ex}
%Câu 78
	Rút gọn biểu thức $A=\dfrac{\tan x-\cot x}{\tan x+\cot x}+\cos 2 x $ .
\choice
{\True $ 0$}
{$ 2\cos^2x$}
{$ 2$}
{$\cos 2x$}
\loigiai{
Ta có
\begin{align*}
A&=\dfrac{\tan x-\cot x}{\tan x+\cot x}+\cos 2 x=\dfrac{\sin^2x-\cos^2x}{\sin^2x+\cos^2x}+\cos2x=-\cos2x+\cos2x=0.
\end{align*}
}
\end{ex}
\begin{ex}
%Câu 79
	Rút gọn biểu thức $A=\dfrac{1+\sin 4\alpha-\cos4\alpha}{1+\sin 4\alpha+\cos4\alpha}$.
\choice
{$\sin 2\alpha $}
{$\cos 2\alpha $}
{\True $\tan 2\alpha $}
{$\cot 2\alpha $}
\loigiai{
Ta có
\begin{align*}
A&=\dfrac{1+\sin 4\alpha-\cos4\alpha}{1+\sin 4\alpha+\cos4\alpha}=\dfrac{2\sin^22\alpha+2\sin2\alpha\cos2\alpha}{2\cos^22\alpha+2\sin2\alpha\cos2\alpha}\cr
&=\dfrac{2\sin2\alpha}{2\cos2\alpha}=\tan2\alpha.
\end{align*}
}
\end{ex}
\begin{ex}
%Câu 81
	Khi $\alpha=\dfrac{\pi}{6}$ thì biểu thức $A=\dfrac{\sin^22\alpha+4\sin^4\alpha-4\sin^2\alpha\cdot\cos^2\alpha}{4-\sin^22\alpha-4\sin^2\alpha}$ có giá trị bằng:
\choice
{$\dfrac{1}{3}$}
{$\dfrac{1}{6}$}
{\True $\dfrac{1}{9}$}
{$\dfrac{1}{12}$}
\loigiai{
Ta có
\begin{align*}
A&=\dfrac{\sin^22\alpha+4\sin^4\alpha-4\sin^2\alpha\cdot\cos^2\alpha}{4-\sin^22\alpha-4\sin^2\alpha}
=\dfrac{4\sin^2\alpha\cos^2\alpha+4\sin^4\alpha-4\sin^2\alpha\cdot\cos^2\alpha}{4(1-\sin^2\alpha)-4\sin^2\alpha\cos^2\alpha}\cr
&=\dfrac{4\sin^4\alpha}{4\cos^2\alpha(1-\sin^2\alpha)}=\tan^4\alpha.
\end{align*}
Với $\alpha=\dfrac\pi6$ ta có $A=\tan^4\dfrac\pi6=\dfrac19$.
}
\end{ex}
\begin{ex}
%Câu 82
	Rút gọn biểu thức $ A=\dfrac{\sin 2\alpha+\sin\alpha}{1+\cos 2\alpha+\cos\alpha}$.
\choice
{\True $\tan\alpha $}
{$ 2\tan\alpha $}
{$\tan 2\alpha+\tan\alpha $}
{$\tan 2\alpha $}
\loigiai{
Ta có
\begin{align*}
A&=\dfrac{\sin 2\alpha+\sin\alpha}{1+\cos 2\alpha+\cos\alpha}=\dfrac{2\sin\alpha\cos\alpha+\sin\alpha}{2\cos^2\alpha+\cos\alpha}\cr
&=\dfrac{2\sin\alpha(\cos\alpha+1)}{2\cos\alpha(\cos\alpha+1)}=\tan\alpha.
\end{align*}
}
\end{ex}
\begin{ex}
%Câu 83
	Rút gọn biểu thức $ A=\dfrac{1-\sin a-\cos 2a}{\sin 2a-\cos a}$.
\choice
{$ 1$}
{\True $\tan a $}
{$\dfrac{5}{2}$}
{$ 2\tan a $}
\loigiai{
Ta có
\begin{align*}
A&=\dfrac{1-\sin a-\cos 2a}{\sin 2a-\cos a}=\dfrac{2\sin^2a-\sin a}{2\sin a\cos a-\cos a}\cr
&=\dfrac{\sin a(2\sin a-1)}{\cos a(2\sin a-1)}=\tan a.
\end{align*}
}
\end{ex}
\begin{ex}
%Câu 84
	Rút gọn biểu thức $ A=\dfrac{\sin x+\sin\dfrac{x}{2}}{1+\cos x+\cos\dfrac{x}{2}}$ được
\choice
{\True $\tan\dfrac{x}{2}$}
{$\cot x$}
{$\tan^2\left(\dfrac{\pi}{4}-x\right)$}
{$\sin x$}
\loigiai{
Ta có
\begin{align*}
A&=\dfrac{\sin x+\sin\dfrac{x}{2}}{1+\cos x+\cos\dfrac{x}{2}}=\dfrac{2\sin\dfrac x2\cos\dfrac x2+\sin\dfrac x2}{2\cos^2\dfrac x2+\cos\dfrac x2}\cr
&=\dfrac{2\sin\dfrac x2\left(2\cos\dfrac x2+1\right)}{2\cos\dfrac x2\left(2\cos\dfrac x2+1\right)}=\tan\dfrac x2.
\end{align*}
}
\end{ex}
\Closesolutionfile{ans}
\begin{indapan}{10}
	{ans/ans-1K1-2-Dang4}
\end{indapan}