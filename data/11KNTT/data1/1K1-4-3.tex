\begin{dang}{Phương trình lượng giác cơ bản dùng Radian}
	\begin{itemize}
		\item  $\sin x = \sin \alpha \Leftrightarrow \hoac{& x = \alpha + k2\pi \\ & x = \pi - \alpha + k2\pi} \,\, (k \in \mathbb{Z})$.
		\item $\cos x = \cos \alpha  \Leftrightarrow \hoac{& x = \alpha + k2\pi \\ & x = -\alpha + k2\pi} \,\, (k \in \mathbb{Z})$.
		\item $\tan x = \tan \alpha \Leftrightarrow x = \alpha + k\pi \,\, (k \in \mathbb{Z})$.
		\item $\cot x = \cot \alpha   \Leftrightarrow x = \alpha + k\pi \,\, (k \in \mathbb{Z})$.
	\end{itemize}
\end{dang}
\subsubsection{Ví dụ}
\begin{vd}%[NB]%[DCHT Toán 11 - KNTT -Tên GV] %[ID6 chương trình mới]
	Giải phương trình $\sin x=1$.
	\dapso{$x=\dfrac{\pi}{2}+k2\pi\, (k\in \mathbb{Z})$.}
	\loigiai{Ta có $\sin x=1\Leftrightarrow x=\dfrac{\pi}{2}+k2\pi, (k\in \mathbb{Z})$.}
\end{vd}
\begin{vd}%[NB]%[DCHT Toán 11 - KNTT -Tên GV] %[ID6 chương trình mới]
	Giải phương trình $\cos x=1$.
	\dapso{$x=2k\pi\, (k\in \mathbb{Z})$.}
	\loigiai{Ta có $\cos x=1\Leftrightarrow x=2k\pi, (k\in \mathbb{Z})$.}
\end{vd}
\begin{vd}%[TH]%[DCHT Toán 11 - KNTT -Tên GV] %[ID6 chương trình mới]
	Giải phương trình $\sin \left(\dfrac{3x}{4}-\dfrac{\pi}{3}\right)=1$.
	\dapso{$x=\dfrac{10\pi}{9}+\dfrac{k8\pi}{3}, (k\in \mathbb{Z})$.}
	\loigiai{Ta có 
		\[\sin \left(\dfrac{3x}{4}-\dfrac{\pi}{3}\right)=1\Leftrightarrow \dfrac{3x}{4}-\dfrac{\pi}{3}=\dfrac{\pi}{2}+k2\pi \Leftrightarrow \dfrac{3x}{4}=\dfrac{5\pi}{6}+k2\pi \Leftrightarrow x=\dfrac{10\pi}{9}+\dfrac{k8\pi}{3} \left(k\in \mathbb{Z}\right).\]}
\end{vd}
\begin{vd}%[NB]%[DCHT Toán 11 - KNTT -Tên GV] %[ID6 chương trình mới]
	Giải phương trình $\tan x-1=0$.
	\dapso{$x=\dfrac{\pi}{4}+k\pi,(k\in \mathbb{Z})$.}
	\loigiai{Ta có 
		$\tan x-1=0\Leftrightarrow \tan x=1\Leftrightarrow x=\dfrac{\pi}{4}+k\pi,(k\in \mathbb{Z}) $.}
\end{vd}
\begin{vd}%[TH]%[DCHT Toán 11 - KNTT -Tên GV] %[ID6 chương trình mới]
	Giải phương trình $\sqrt{3}\tan x-1=0$.
	\dapso{$x=\dfrac{\pi}{6}+k\pi, (k\in \mathbb{Z})$.}
	\loigiai{Điều kiện: $x\ne \dfrac{\pi}{2}+k\pi \left(k\in \mathbb{Z}\right)$.\\
		Với điều kiện $x\ne \dfrac{\pi}{2}+k\pi \left(k\in \mathbb{Z}\right)$ thì phương trình \[\sqrt{3}\tan x-1=0\Leftrightarrow \tan x=\dfrac{1}{\sqrt{3}}
		\Leftrightarrow \tan x=\tan \dfrac{\pi}{6}\Leftrightarrow x=\dfrac{\pi}{6}+k\pi, (k\in \mathbb{Z}).\]
		Vậy phương trình có nghiệm là $x=\dfrac{\pi}{6}+k\pi\,\, (k\in \mathbb{Z})$.}
\end{vd}
\begin{vd}%[TH]%[DCHT Toán 11 - KNTT -Tên GV] %[ID6 chương trình mới]
	Giải phương trình $\cot 3x=\cot x$.
	\dapso{$x=\dfrac{\pi}{2}+k\pi,(k\in \mathbb{Z})$.}
	\loigiai{Điều kiện xác định $\heva{&\sin 3x\ne 0 \\&\operatorname{s}\text{inx}\ne 0}\Leftrightarrow \heva{&x\ne k\dfrac{\pi}{3} \\&x\ne k\pi}$.\\
		Phương trình đã cho tương đương
		\[\dfrac{\cos 3x}{\sin 3x}=\dfrac{\cos x}{\sin x}\Leftrightarrow \sin x\cos 3x-\cos x\sin 3x=0\Leftrightarrow \sin 2x=0\Leftrightarrow x=k\dfrac{\pi}{2},\, (k\in \mathbb{Z}).\]
		Kết hợp điều kiện ta được các nghiệm của phương trình $x=\dfrac{\pi}{2}+k\pi,\, (k\in \mathbb{Z})$.}
\end{vd}
\subsubsection{Bài tập tự luận}
\begin{bt}%[NB]%[Câu 1]
	Giải phương trình $\sin 2x=1$
	\dapso{$x=\dfrac{\pi}{4}+k\pi$.}
	\loigiai{
		Ta có $\sin 2x=1\Leftrightarrow 2x=\dfrac{\pi}{2}+k2\pi \Leftrightarrow x=\dfrac{\pi}{4}+k\pi $.}
\end{bt}
\begin{bt}%[TH]%[Câu 4]
	Giải phương trình $\cot\left(3x-1\right)=-\sqrt{3}$. \dapso{ $xx=\dfrac{1}{3}-\dfrac{\pi}{18}+k\dfrac{\pi}{3}$.}
	\loigiai{
		Ta có 
		\begin{eqnarray*}
			\cot\left(3x-1\right)=-\sqrt{3} &\Leftrightarrow & \cot\left(3x-1\right)=\cot\left(-\dfrac{\pi}{6}\right)\\
			&\Leftrightarrow & 3x-1=\dfrac{-\pi}{6}+k\pi\\
			&\Leftrightarrow & x=\dfrac{1}{3}-\dfrac{\pi}{18}+k\dfrac{\pi}{3},\, (k\in \mathbb{Z}).
		\end{eqnarray*}
	}
\end{bt}
\begin{bt}%[TH]%[Câu 5]
	Giải phương trình $\cot x=\cot \left(-\dfrac{\pi}{7}\right)$ trên khoảng $\left(0;3\pi\right)$. 
	\dapso{ $x\in \left\{\dfrac{6\pi}{7};\dfrac{13\pi}{7}\right\}$.}
	\loigiai{
		Ta có 
		$ \cot x=\cot \left(-\dfrac{\pi}{7}\right) \Leftrightarrow x=-\dfrac{\pi}{7}+k\pi,k\in \mathbb{Z}$. \\
		Vì $x\in \left(0;3\pi\right)$ nên $x\in \left\{\dfrac{6\pi}{7};\dfrac{13\pi}{7};\dfrac{20\pi}{7} \right\}$.}
\end{bt}
\begin{bt}%[TH]%[Câu 6]
	Phương trình $\cot x=\sqrt{3}$ có bao nhiêu nghiệm thuộc $\left[-2018\pi;2018\pi\right]$?
	\dapso{ $4036$.}
	\loigiai{
		Điều kiện $\sin x\ne 0\Leftrightarrow x\ne k\pi,k\in \mathbb{Z}$.\\
		$\cot x=\sqrt{3} \Leftrightarrow \tan x=\dfrac{1}{\sqrt{3}} 
		\Leftrightarrow x=\dfrac{\pi}{6}+k\pi,k\in \mathbb{Z}$. \\
		Vì $x\in \left[-2018\pi;2018\pi\right]$ nên $k\in \left[-2018;2017\right]$. Do đó có $4036$ nghiệm.}
\end{bt}
\begin{bt}%[TH]%[Câu 7]
	Tổng các nghiệm của phương trình $\tan 5x-\tan x=0$ trên nửa khoảng $\left[0;\pi\right)$ bằng
	\dapso{ $\pi$.}
	\loigiai{
		Điều kiện $\heva{&5x\ne \dfrac{\pi}{2}+k\pi \\&x\ne \dfrac{\pi}{2}+k\pi}\Leftrightarrow \heva{&x\ne \dfrac{\pi}{10}+k\dfrac{\pi}{2} \\&x\ne \dfrac{\pi}{2}+k\pi},k\in \mathbb{Z}$.\\
		Ta có $\tan 5x-\tan x=0 \Leftrightarrow \tan 5x=\tan x$
		$\Leftrightarrow 5x=x+k\pi $
		$\Leftrightarrow x=\dfrac{k\pi}{4}$ ($k\in \mathbb{Z}$).\\
		Do $x\in \left[0;\pi\right)$ và kết hợp với điều kiện suy ra $x\in \left\{0;\dfrac{\pi}{4};\dfrac{3\pi}{4}\right\}$.\\
		Vậy tổng các nghiệm là $0+\dfrac{\pi}{4}+\dfrac{3\pi}{4}=\pi $.}
\end{bt}
%%%%%%%%%%%%%%%%%%%%%%%%%%%%%%%%%%%%%%%%%
\subsubsection{Bài tập trắc nghiệm}
\Opensolutionfile{ans}[ans/ans-1K1-4-Dang3]

\begin{ex}%[Câu 1]
	Phương trình $\sin x=\dfrac{\sqrt{3}}{2}$ có tập nghiệm là
	\choice
	{$S=\left\{\dfrac{\pi}{6}+k2\pi;\dfrac{5\pi}{6}+k2\pi,k\in \mathbb{Z}\right\}$}
	{$S=\left\{\dfrac{\pi}{3}+k2\pi;-\dfrac{\pi}{3}+k2\pi,k\in \mathbb{Z}\right\}$}
	{\True $S=\left\{\dfrac{\pi}{3}+k2\pi;\dfrac{2\pi}{3}+k2\pi,k\in \mathbb{Z}\right\}$}
	{$S=\left\{\dfrac{\pi}{3}+k2\pi;-\dfrac{2\pi}{3}+k2\pi,k\in \mathbb{Z}\right\}$}
	\loigiai{
		Ta có $\sin x=\dfrac{\sqrt{3}}{2}\Leftrightarrow \sin x=\sin \dfrac{\pi}{3}\Leftrightarrow \hoac{&x=\dfrac{\pi}{3}+k2\pi \\&x=\dfrac{2\pi}{3}+k2\pi}\, (k\in \mathbb{Z})$.}
\end{ex}
\begin{ex}%[Câu 2]
	Phương trình $2\sin x-1=0$ có tập nghiệm là
	\choice
	{\True $S=\left\{\dfrac{\pi}{6}+k2\pi;\dfrac{5\pi}{6}+k2\pi,k\in \mathbb{Z}\right\}$}
	{$S=\left\{\dfrac{\pi}{3}+k2\pi;-\dfrac{2\pi}{3}+k2\pi,k\in \mathbb{Z}\right\}$}
	{$S=\left\{\dfrac{\pi}{6}+k2\pi;-\dfrac{\pi}{6}+k2\pi,k\in \mathbb{Z}\right\}$}
	{$S=\left\{\dfrac{1}{2}+k2\pi,k\in \mathbb{Z}\right\}$}
	\loigiai{
		Ta có $2\sin x-1=0\Leftrightarrow \sin x=\dfrac{1}{2}\Leftrightarrow \sin x=\sin \dfrac{\pi}{6}\Leftrightarrow \hoac{&x=\dfrac{\pi}{6}+k2\pi \\&x=\dfrac{5\pi}{6}+k2\pi}\left(k\in \mathbb{Z}\right)$.}
\end{ex}
\begin{ex}%[Câu 3]
	Tập nghiệm của phương trình $\sin x=0$ là
	\choice
	{$x=\dfrac{\pi}{2}+k\pi \, (k\in \mathbb{Z})$}
	{\True $x=k\pi \, (k\in \mathbb{Z})$}
	{$x=\dfrac{\pi}{2}+k2\pi \, (k\in \mathbb{Z})$}
	{$x=k2\pi \, (k\in \mathbb{Z})$}
	\loigiai{
		Ta có	$\sin x=0\Rightarrow x=k\pi,\, (k\in \mathbb{Z}) $.}
\end{ex}
\begin{ex}%[Câu 4]
	Số nghiệm của phương trình $\sin 2x=0$ thỏa mãn $0<x<2\pi$ là?
	\choice
	{$2$}
	{$1$}
	{\True $3$}
	{$0$}
	\loigiai{
		Ta có $\sin 2x=0\Leftrightarrow 2x=k\pi\Leftrightarrow x=k\dfrac{\pi}{2}; \,k\in \mathbb{Z}$.\\
		Do $0<x<2\pi \Rightarrow 0<k\dfrac{\pi}{2}<2\pi \Rightarrow 0<k<4\xrightarrow{k\in \mathbb{Z}}k=\left\{1;2;3\right\}$.}
\end{ex}
\begin{ex}%[Câu 5]
	Nghiệm của phương trình $\sin \dfrac{x}{2}=1$ là
	\choice
	{\True $x=\pi +k4\pi,k\in \mathbb{Z}$}
	{$x=k2\pi,k\in \mathbb{Z}$}
	{$x=\dfrac{\pi}{2} +k\pi,k\in \mathbb{Z}$}
	{$x=\dfrac{\pi}{2}+k2\pi,\, k\in \mathbb{Z}$}
	\loigiai{
		Phương trình $\sin \dfrac{x}{2}=1\Leftrightarrow \dfrac{x}{2}=\dfrac{\pi}{2}+k2\pi \Leftrightarrow x=\pi +k4\pi,k\in \mathbb{Z}$.}
\end{ex}
\begin{ex}%[Câu 6]
	Nghiệm của phương trình $\cos x=\dfrac{1}{2}$ là
	\choice
	{$x=\pm \dfrac{\pi}{2}+k2\pi,\, k\in \mathbb{Z} $}
	{\True $x=\pm \dfrac{\pi}{3}+k2\pi,\, k\in \mathbb{Z} $}
	{$x=\pm \dfrac{\pi}{4}+k2\pi ,\, k\in \mathbb{Z}$}
	{$x=\pm \dfrac{\pi}{6}+k2\pi,\, k\in \mathbb{Z} $}
	\loigiai{
		Ta có $\cos x=\cos \dfrac{\pi}{3}\Leftrightarrow \left[\begin{matrix}
			x=\dfrac{\pi}{3}+k2\pi \\
			x=-\dfrac{\pi}{3}+k2\pi \\
		\end{matrix}\right.\left(k\in \mathbb{Z}\right)$.}
\end{ex}
\begin{ex}%[Câu 7]
	Số nghiệm của phương trình $\cos \left(x+\dfrac{\pi}{4}\right)=1$ với $\pi \le x\le 5\pi $ là
	\choice
	{$0$}
	{$3$}
	{$1$}
	{\True $2$}
	\loigiai{
		Phương trình $\cos \left(x+\dfrac{\pi}{4}\right)=1$ $\Leftrightarrow x+\dfrac{\pi}{4}=k2\pi \Leftrightarrow x=-\dfrac{\pi}{4}+k2\pi,k\in \mathbb{Z}$.\\
		Mà $\pi \le x\le 5\pi $ nên $\pi \le -\dfrac{\pi}{4}+k2\pi \le 5\pi \Leftrightarrow \dfrac{5}{8}\le k\le \dfrac{21}{8}$; $k\in \mathbb{Z}\Rightarrow k\in \left\{1;2\right\}$.\\
		Vậy phương trình đã cho có $2$ nghiệm trên $[\pi;5\pi]$.}
\end{ex}

\begin{ex}%[Câu 8]
	Phương trình $\cos x-1=0$ có nghiệm là
	\choice
	{$x=k\pi, \, k\in \mathbb{Z}$}
	{\True $x=k2\pi, \, k\in \mathbb{Z}$}
	{$x=\dfrac{\pi}{2}+k2\pi, \, k\in \mathbb{Z}$}
	{$x=\pi +k2\pi, \, k\in \mathbb{Z}$}
	\loigiai{
		Ta có $\cos x-1=0\Leftrightarrow \cos x=1\Leftrightarrow x=k2\pi,\, k\in \mathbb{Z}$.}
\end{ex}
\begin{ex}%[Câu 9]
	Tập nghiệm của phương trình $\cos 2x=\dfrac{\sqrt{3}}{2}$ là
	\choice
	{\True $x=\pm \dfrac{\pi}{12}+k\pi,\, k\in \mathbb{Z} $}
	{$x=\pm \dfrac{\pi}{6}+k\pi,\, k\in \mathbb{Z} $}
	{$x=-\dfrac{\pi}{12}+k\pi, \, k\in \mathbb{Z}$}
	{$x=\dfrac{\pi}{12}+k\pi, \, k\in \mathbb{Z}$}
	\loigiai{
		Ta có $\cos 2x=\dfrac{\sqrt{3}}{2}\Leftrightarrow 2x=\pm \dfrac{\pi}{6}+k2\pi \Leftrightarrow x=\pm \dfrac{\pi}{12}+k\pi ,\,k\in \mathbb{Z}$.}
\end{ex}
\begin{ex}%[Câu 10]
	Tập nghiệm của phương trình $\cos 2x=\dfrac{1}{2}$ là
	\choice
	{\True $x=\pm \dfrac{\pi}{6}+k\pi \left(k\in \mathbb{Z}\right)$}
	{$x=\pm \dfrac{\pi}{6}+k\pi \left(k\in \mathbb{R}\right)$}
	{$x=\dfrac{\pi}{6}+k\pi \left(k\in \mathbb{Z}\right)$}
	{$x=\pm \dfrac{\pi}{3}+k2\pi \left(k\in \mathbb{Z}\right)$}
	\loigiai{
		Ta có $\cos 2x=\dfrac{1}{2}=\cos \dfrac{\pi}{3}\Leftrightarrow 2x=\pm \dfrac{\pi}{3}+k2\pi \Leftrightarrow x=\pm \dfrac{\pi}{6}+k\pi,\,(k\in \mathbb{Z})$.}
\end{ex}
\begin{ex}%[Câu 11]
	Tổng nghiệm âm lớn nhất và nghiệm dương nhỏ nhất của phương trình $2\cos x-\sqrt{3}=0$ là
	\choice
	{$\dfrac{5\pi}{3}$}
	{\True $0$}
	{$\dfrac{5\pi}{6}$}
	{$-\dfrac{5\pi}{3}$}
	\loigiai{
		Ta có $2\cos x-\sqrt{3}=0\Leftrightarrow \cos x=\dfrac{\sqrt{3}}{2}\Leftrightarrow \cos x=\cos \dfrac{\pi}{6}\Leftrightarrow \hoac{&x=\dfrac{\pi}{6}+k2\pi \\&x=-\dfrac{\pi}{6}+k2\pi.}$\\
		Suy ra nghiệm dương nhỏ nhất của phương trình là $x=\dfrac{\pi}{6}$,
		nghiệm âm lớn nhất của phương trình là $x=-\dfrac{\pi}{6}$.\\
		Vậy tổng cần tìm là $S=\dfrac{\pi}{6}+\left(-\dfrac{\pi}{6}\right)=0$.}
\end{ex}
% \begin{ex}%[Câu 12]
% 	Số nghiệm của phương trình $\cos x=\dfrac{2}{5}$ trên khoảng $\left(-\dfrac{\pi}{2};2\pi\right)$ là
% 	\choice
% 	{\True $2$}
% 	{$1$}
% 	{$4$}
% 	{$3$}
% 	\loigiai{
% 		Ta có $\cos x=\dfrac{2}{5}\Leftrightarrow x=\pm \arccos \dfrac{2}{5}+k2\pi, \, k\in \mathbb{Z} $. \\
% 		Vì $x\in \left(-\dfrac{\pi}{2};2\pi\right)\Rightarrow x=\arccos \dfrac{2}{3};x=-\arccos \dfrac{2}{3}+2\pi $
% 		là thỏa mãn.}
% \end{ex}
% \begin{ex}%[Câu 13]
% 	Tổng các nghiệm thuộc khoảng $\left(0;2\pi\right)$ của phương trình $5\cos x-2=0$ là
% 	\choice
% 	{$S=3\pi $}
% 	{\True $S=2\pi $}
% 	{$S=0$}
% 	{$S=4\pi $}
% 	\loigiai{
% 		Ta có $5\cos x-2=0\Leftrightarrow \cos x=\dfrac{2}{5}\Leftrightarrow x=\pm \arccos \left(\dfrac{2}{5}\right)+k2\pi,\, k\in \mathbb{Z}$.\\
% 		Xét trên $\left(0;2\pi\right)$ phương trình có hai nghiệm $x=\arccos \left(\dfrac{2}{5}\right)$ và $x=-\arccos \left(\dfrac{2}{5}\right)+2\pi $.\\
% 		Do vậy tổng tất cả các nghiệm của phương trình bằng $\arccos \left(\dfrac{2}{5}\right)-\arccos \left(\dfrac{2}{5}\right)+2\pi =2\pi$.}
% \end{ex}
\begin{ex}%[Câu 14]
	Tính tổng $S$ tất cả các nghiệm trên khoảng $\left(0;3\pi\right)$ của phương trình $2\cos 3x=1$
	\choice
	{\True $S=\dfrac{121\pi}{9}$}
	{$S=\dfrac{120\pi}{9}$}
	{$S=\dfrac{122\pi}{9}$}
	{$S=\dfrac{20\pi}{3}$}
	\loigiai{
		Ta có $2\cos 3x=1\Leftrightarrow \cos 3x=\dfrac{1}{2}\Leftrightarrow \left[\begin{matrix}
			3x=\dfrac{\pi}{3}+k2\pi \\
			3x=-\dfrac{\pi}{3}+k2\pi \\
		\end{matrix}\right.\Leftrightarrow \left[\begin{matrix}
			x=\dfrac{\pi}{9}+\dfrac{k2\pi}{3}, \, k\in \mathbb{Z}  \\
			x=-\dfrac{\pi}{9}+\dfrac{k2\pi}{3}, \, k\in \mathbb{Z}. \\
		\end{matrix}\right.$ \\
		Suy ra $S=\left(\dfrac{\pi}{9}+\dfrac{7\pi}{9}+\dfrac{13\pi}{9}+\dfrac{19\pi}{9}+\dfrac{25\pi}{9}\right)+\left(\dfrac{5\pi}{9}+\dfrac{11\pi}{9}+\dfrac{17\pi}{9}+\dfrac{23\pi}{9}\right)=\dfrac{121\pi}{9}$.}
\end{ex}
\begin{ex}%[Câu 15]
	Tập nghiệm $S$ của phương trình $\sqrt{3}\tan \dfrac{x}{3}+3=0$.
	\choice
	{$S=\left\{-\dfrac{\pi}{9}+k3\pi,k\in \mathbb{Z}\right\}$}
	{$S=\left\{-\dfrac{\pi}{3}+k\pi,k\in \mathbb{Z}\right\}$}
	{\True $S=\left\{-\pi +k3\pi,k\in \mathbb{Z}\right\}$}
	{$S=\left\{\dfrac{\pi}{6}+k\pi,k\in \mathbb{Z}\right\}$}
	\loigiai{
		Ta có $\sqrt{3}\tan \dfrac{x}{3}+3=0\Leftrightarrow \tan \dfrac{x}{3}=\tan \left(-\dfrac{\pi}{3}\right)\Leftrightarrow \dfrac{x}{3}=-\dfrac{\pi}{3}+k\pi \Leftrightarrow x=-\pi +k3\pi,\, k\in \mathbb{Z}$.}
\end{ex}
\begin{ex}%[Câu 16]
	Nghiệm của phương trình $\tan \mathrm{x}=\tan \dfrac{\pi}{3}$ là
	\choice
	{$x=\pm \dfrac{\pi}{3}+k2\pi,k\in \mathbb{Z}$}
	{$x=\dfrac{\pi}{6}+k2\pi,k\in \mathbb{Z}$}
	{\True $x=\dfrac{\pi}{3}+k\pi,k\in \mathbb{Z}$}
	{$x=-\dfrac{\pi}{6}+k2\pi,k\in \mathbb{Z}$}
	\loigiai{
		Ta có $\tan \mathrm{x}=\tan \dfrac{\pi}{3}\Leftrightarrow x=\dfrac{\pi}{3}+k\pi,k\in \mathbb{Z}$.}
\end{ex}
%\begin{ex}%[Câu 17]
%	Nghiệm của phương trình $\tan 3 x=\tan x$ là:
%	\choice
%	{$x=k \pi, \mathrm{k} \in \mathbb{Z}$}
%	{\True $x=\dfrac{k\pi}{2},\mathrm{k}\in \mathbb{Z}$}
%	{$x=\dfrac{k \pi}{6}, \mathrm{k} \in \mathbb{Z}$}
%	{$x=k2\pi,\mathrm{k}\in \mathbb{Z}$}
%	\loigiai{
%		
%		Ta có $\tan 3 x=\tan x \Leftrightarrow 3 x=x+k \pi \Leftrightarrow x=\dfrac{k \pi}{2},\, k \in \mathbb{Z}$.}
%\end{ex}
\begin{ex}%[Câu 18]
	Phương trình $\tan x=1$ có nghiệm là
	\choice
	{$x=\dfrac{\pi}{4}+k2\pi $}
	{$x=-\dfrac{\pi}{4}+k2\pi $}
	{$x=-\dfrac{\pi}{4}+k\pi $}
	{\True $x=\dfrac{\pi}{4}+k\pi $}
	\loigiai{
		Ta có $\tan x=1\Leftrightarrow x=\dfrac{\pi}{4}+k\pi,\, k \in \mathbb{Z}$.}
\end{ex}
\begin{ex}%[Câu 19]
	Phương trình $\sqrt{3}\tan 2x-3=0$ có nghiệm là
	\choice
	{$x=\dfrac{\pi}{3}+k\pi \left(k\in \mathbb{Z}\right)$}
	{\True $x=\dfrac{\pi}{6}+\dfrac{k\pi}{2}\left(k\in \mathbb{Z}\right)$}
	{$x=\dfrac{\pi}{3}+\dfrac{k\pi}{2}\left(k\in \mathbb{Z}\right)$}
	{$x=\dfrac{\pi}{6}+k\pi \left(k\in \mathbb{Z}\right)$}
	\loigiai{
		Ta có $\sqrt{3}\tan 2x-3=0\Leftrightarrow \tan 2x=\tan \dfrac{\pi}{3}$ $\Leftrightarrow 2x=\dfrac{\pi}{3}+k\pi \Leftrightarrow x=\dfrac{\pi}{6}+\dfrac{k\pi}{2},\, k \in \mathbb{Z}$.\\
		Vậy $x=\dfrac{\pi}{6}+\dfrac{k\pi}{2},\, k \in \mathbb{Z}$.}
\end{ex}
\begin{ex}%[Câu 20]
	Cho phương trình $\sqrt{3}\tan 2x=3$ có nghiệm $x_0$ khi đó $\cos x_0$ nhận giá trị là
	\choice
	{$\dfrac{-\sqrt{3}}{2}$}
	{\True $\pm \dfrac{\sqrt{3}}{2};\pm \dfrac{1}{2}$}
	{$\pm \dfrac{\sqrt{3}}{2}$}
	{$\pm \dfrac{1}{2}$}
	\loigiai{
		Ta có $\sqrt{3}\tan 2x=3\Leftrightarrow \tan 2x=\dfrac{3}{\sqrt{3}}\Leftrightarrow 2x=\dfrac{\pi}{3}+k\pi \Leftrightarrow x=\dfrac{\pi}{6}+k\dfrac{\pi}{2}$.\\
		Suy ra $x_0\in \left\{\dfrac{\pi}{6}+2k\pi;\dfrac{2\pi}{3}+2k\pi;\dfrac{7\pi}{6}+2k\pi;\dfrac{5\pi}{3}+2k\pi |k\in \mathbb{Z}\right\}$.\\
		Do vậy $\cos x_0\in \left\{\pm \dfrac{\sqrt{3}}{2};\pm \dfrac{1}{2}\right\}$.}
\end{ex}
\begin{ex}%[Câu 21]
	Tổng các nghiệm của phương trình $\tan 2x=\tan x$ trên $\left[-\pi;2\pi\right]$ là
	\choice
	{$\pi $}
	{$\dfrac{\pi}{2}$}
	{$4\pi $}
	{\True $2\pi $}
	\loigiai{
		Điều kiện xác định $\heva{&\cos 2x\ne 0 \\&\cos x\ne 0}\Leftrightarrow \heva{&x\ne \dfrac{\pi}{4}+\dfrac{k\pi}{2} \\&x\ne \dfrac{\pi}{2}+k\pi},\, (k\in \mathbb{Z})$.\\
		Khi đó $\tan 2x=\tan x\Leftrightarrow 2x=x+k\pi \Leftrightarrow x=k\pi,k\in \mathbb{Z}$.\\
		Do $x\in \left[-\pi;2\pi\right]$ nên $x\in \left\{-\pi;0;\pi;2\pi\right\}$.\\
		Vậy tổng các nghiệm của phương trình trên $\left[-\pi;2\pi\right]$ là $2\pi$.}
\end{ex}
\begin{ex}%[Câu 22]
	Nghiệm của phương trình $\tan 3x=\tan x$ là
	\choice
	{\True $x=k\pi,k\in \mathbb{Z}$}
	{$x=\dfrac{k\pi}{2},k\in \mathbb{Z}$}
	{$x=\dfrac{k\pi}{6},k\in \mathbb{Z}$}
	{$x=k2\pi,k\in \mathbb{Z}$}
	\loigiai{
		Điều kiện $\heva{&x\ne \dfrac{\pi}{2}+k\pi \\&x\ne \dfrac{\pi}{6}+\dfrac{k\pi}{3}} (k\in \mathbb{Z}).$\\
		Ta có $\tan 3x=\tan x\Leftrightarrow 3x=x+k\pi $
		$\Leftrightarrow x=\dfrac{k\pi}{2}$, $k\in \mathbb{Z}$.\\
		Kết hơp điều kiện, khi đó phương trình có nghiệm là $x=k\pi,k\in \mathbb{Z}$.}
\end{ex}
\begin{ex}%[Câu 23]
	Nghiệm của phương trình $\tan 2\mathrm{x}=\tan \left(\dfrac{\pi}{2}-x\right)$ là
	\choice
	{$x=\dfrac{\pi}{6}+k\dfrac{\pi}{2},k\in \mathbb{Z}$}
	{$x=\dfrac{\pi}{4}+k\dfrac{\pi}{3},k\in \mathbb{Z}$}
	{$x=\dfrac{\pi}{3}+k\dfrac{\pi}{2},k\in \mathbb{Z}$}
	{\True $x=\dfrac{\pi}{6}+k\dfrac{\pi}{3},k\in \mathbb{Z}$}
	\loigiai{
		Điều kiện $\heva{&2x\ne \dfrac{\pi}{2}+k\pi \\&\dfrac{\pi}{2}-x\ne \dfrac{\pi}{2}+k\pi }\Leftrightarrow \heva{&x\ne \dfrac{\pi}{2}+\dfrac{k\pi}{2} \\&x\ne k\pi} (k\in \mathbb{Z}).  $\\
		Ta có $\tan 2\mathrm{x}=\tan \left(\dfrac{\pi}{2}-x\right)\Leftrightarrow 2\mathrm{x}=\dfrac{\pi}{2}-x+k\pi \Leftrightarrow x=\dfrac{\pi}{6}+k\dfrac{\pi}{3},k\in \mathbb{Z}$.}
\end{ex}

\begin{ex}%[Câu 34]
	Phương trình lượng giác $\sqrt{3}\cot x-3=0$ có nghiệm là
	\choice
	{$x=\dfrac{\pi}{3}+k\pi,\left(k\in \mathbb{Z}\right)$}
	{$x=\dfrac{\pi}{6}+k2\pi,\left(k\in \mathbb{Z}\right)$}
	{$x=-\dfrac{\pi}{6}+k2\pi,\left(k\in \mathbb{Z}\right)$}
	{\True $x=\dfrac{\pi}{6}+k\pi,\left(k\in \mathbb{Z}\right)$}
	\loigiai{
		Ta có $\sqrt{3}\cot x-3=0\Leftrightarrow \cot x=\sqrt{3}\Leftrightarrow x=\dfrac{\pi}{6}+k\pi,\, k\in \mathbb{Z}$.}
\end{ex}
\begin{ex}%[Câu 35]
	Phương trình $\cot \left(\dfrac{\pi}{4}-2x\right)=1$ có nghiệm
	\choice
	{$x=\dfrac{\pi}{2}+k2\pi,k\in \mathbb{Z}$}
	{$x=\dfrac{\pi}{2}+k\pi,k\in \mathbb{Z}$}
	{$x=k\pi,k\in \mathbb{Z}$}
	{\True $x=k\dfrac{\pi}{2},k\in \mathbb{Z}$}
	\loigiai{
		Ta có $\cot \left(\dfrac{\pi}{4}-2x\right)=1\Leftrightarrow \dfrac{\pi}{4}-2x=\dfrac{\pi}{4}-k\pi \Leftrightarrow x=k\dfrac{\pi}{2},\, k\in \mathbb{Z}$.}
\end{ex}

\Closesolutionfile{ans}
\begin{indapan}{10}
	{ans/ans-1K1-4-Dang3}
\end{indapan}