\begin{dang}{Phương trình đưa về phương trình lượng giác cơ bản}
	
\end{dang}
\subsubsection{Ví dụ}
\begin{vd}%[DCHT Toán 11 - KNTT -Thọ Bùi] %[1K1B4-5]
	Giải phương trình:  $\sin 2x=\cos 3x$.
	\loigiai{Ta có: 
		\begin{eqnarray*}
			\sin 2x=\cos 3x &\Leftrightarrow& \cos 3x =\cos \left(\dfrac{\pi}{2}-2x\right)\\
			&\Leftrightarrow& \hoac{
				&3x=\dfrac{\pi}{2}-2x+k2\pi\\
				&3x=-\left(\dfrac{\pi}{2}-2x\right) +k2\pi}\\
			&\Leftrightarrow&
			\hoac{&5x=\dfrac{\pi}{2}+k2\pi\\&x=-\dfrac{\pi}{2}+k2\pi}\\
			&\Leftrightarrow&
			\hoac{&x=\dfrac{\pi}{12}+\dfrac{k2\pi}{5}\\
				&x=-\dfrac{\pi}{2}+k2\pi} (k\in \mathbb{Z}).
		\end{eqnarray*}
	}
\end{vd}

%%=====Ví dụ 2
\begin{vd}%[DCHT Toán 11 - KNTT -Thọ Bùi] %[1K1B4-5]
	Giải phương trình:  $\sin 4x - \cos\left(x + \dfrac{\pi}{6}\right)=0$.
	\loigiai{
		Ta có:
		\allowdisplaybreaks
		\begin{eqnarray*}
			&&\sin 4x - \cos\left(x + \dfrac{\pi}{6}\right)=0\\
			&\Leftrightarrow& \sin 4x = \cos\left(x + \dfrac{\pi}{6}\right)\\
			&\Leftrightarrow& \sin 4x = \sin \left(\dfrac{\pi}{3}-x\right)\\
			&\Leftrightarrow& \hoac{&4x=\dfrac{\pi}{3}-x+k2\pi\\&4x=\pi - \dfrac{\pi}{3}+x+k2\pi\\}\\
			&\Leftrightarrow& \hoac{&5x=\dfrac{\pi}{3}+k2\pi\\&3x= \dfrac{2\pi}{3}+k2\pi\\}\\
			&\Leftrightarrow& \hoac{&x=\dfrac{\pi}{15}+\dfrac{k2\pi}{5}\\&x= \dfrac{2\pi}{9}+\dfrac{k2\pi}{3}\\} (k\in \mathbb{Z}).
		\end{eqnarray*}
	}
\end{vd}

%%=====Ví dụ 3
\begin{vd}%[DCHT Toán 11 - KNTT -Thọ Bùi] %[1K1B4-5]
	Giải các phương trình sau:
	\begin{multicols}{2}
		\begin{enumerate}
			\item $\sin 2x+\cos 4x=0$.
			\item  $\cos 3x=-\cos 7x$.
		\end{enumerate}
	\end{multicols}
	\loigiai{
		\begin{enumerate}
			\item  $\sin 2x+\cos 4x=0\Leftrightarrow -\sin2x=\cos4x\Leftrightarrow \cos\left(2x+\dfrac{\pi}{2} \right) =\cos4x \Leftrightarrow \hoac{&4x=2x+\dfrac{\pi}{2}+k2\pi\\&4x=-2x-\dfrac{\pi}{2}+k2\pi}$\\$\Leftrightarrow\hoac{&2x=\dfrac{\pi}{2}+k2\pi\\&6x=-\dfrac{\pi}{2}+k2\pi}\Leftrightarrow\hoac{&x=\dfrac{\pi}{4}+k\pi\\&x=-\dfrac{\pi}{12}+\dfrac{k}{3}\pi},k \in \mathbb{Z}.$
			\item $\cos 3x=-\cos 7x\Leftrightarrow \cos 3x=\cos (\pi-7x)\Leftrightarrow\hoac{&3x=\pi-7x+k2\pi\\&3x=7x-\pi+k2\pi}\Leftrightarrow\hoac{&x=\dfrac{\pi}{10}+\dfrac{k}{5}\pi\\&x=\dfrac{\pi}{4}+\dfrac{k}{4}\pi} , k \in \mathbb{Z}$.
		\end{enumerate}
	}
\end{vd}

%%=====Ví dụ 4
\begin{vd}%[DCHT Toán 11 - KNTT -Thọ Bùi] %[1K1K4-5]
	Giải phương trình: $\cos^2 2x =\cos^2 \left(x+\dfrac{\pi}{6}\right)$.
	\loigiai{
		\textbf{Cách 1.}\\ Ta có:
		$\cos^2 2x =\cos^2 \left(x+\dfrac{\pi}{6}\right) \Leftrightarrow 
		\hoac{
			&\cos 2x=\cos \left(x+\dfrac{\pi}{6}\right) \quad (1)\\
			&\cos 2x=-\cos \left(x+\dfrac{\pi}{6}\right). \quad (2)
		}$\\
		+) $(1) \Leftrightarrow \hoac{
			&2x=x+\dfrac{\pi}{6}+k2\pi\\
			&2x=-\left(x+\dfrac{\pi}{6}\right)+k2\pi
		} \Leftrightarrow
		\hoac{
			&x=\dfrac{\pi}{6}+k2\pi\\
			&3x=-\dfrac{\pi}{6}+k2\pi
		}
		\Leftrightarrow
		\hoac{
			&x=\dfrac{\pi}{6}+k2\pi\\
			&x=-\dfrac{\pi}{18}+\dfrac{k2\pi}{3}
		}(k\in \mathbb{Z})$.\\
		+) $(2) \Leftrightarrow 
		\cos 2x=\cos\left[\pi- \left(x+\dfrac{\pi}{6}\right) \right]
		\Leftrightarrow
		\hoac{
			&2x=\pi- \left(x+\dfrac{\pi}{6}\right)+k2\pi\\
			&2x=-\left[\pi- \left(x+\dfrac{\pi}{6}\right)\right]+k2\pi
		} $
		\[\Leftrightarrow
		\hoac{
			&3x=\dfrac{5\pi}{6}+k2\pi\\
			&x=-\dfrac{5\pi}{6}+k2\pi
		} \Leftrightarrow
		\hoac{
			&x=\dfrac{5\pi}{18}+\dfrac{k2\pi}{3}\\
			&x=-\dfrac{5\pi}{6}+k2\pi
		} (k\in\mathbb{Z}).\]\\
		\textbf{Cách 2.} Dùng công thức hạ bậc, ta có:
		\allowdisplaybreaks
		\begin{eqnarray*}
			\cos^2 2x =\cos^2 \left(x+\dfrac{\pi}{6}\right)
			&\Leftrightarrow& \dfrac{1-\cos 4x}{2}=\dfrac{1-\cos\left(2x+\dfrac{\pi}{3}\right)}{2}\\
			&\Leftrightarrow& \cos 4x=\cos\left(2x+\dfrac{\pi}{3}\right)\\
			&\Leftrightarrow& \hoac{&4x=2x+\dfrac{\pi}{3}+k2\pi\\&4x= - \left(2x+\dfrac{\pi}{3}\right)+x+k2\pi\\}\\
			&\Leftrightarrow& \hoac{&2x=\dfrac{\pi}{3}+k2\pi\\&6x= -\dfrac{\pi}{3}+k2\pi\\}\\
			&\Leftrightarrow& \hoac{&x=\dfrac{\pi}{6}+k\pi\\&x= -\dfrac{\pi}{18}+\dfrac{k\pi}{3}\\} (k\in \mathbb{Z}).
		\end{eqnarray*}
	}
\end{vd}

%%=====Ví dụ 5
\begin{vd}%[DCHT Toán 11 - KNTT -Thọ Bùi] %[1K1B4-5]
	Giải phương trình: $\sin x+\sin 2 x=0$.
	\loigiai{
		Ta có: \allowdisplaybreaks
		\begin{eqnarray*}
			\sin x+\sin 2 x=0&\Leftrightarrow& \sin x+2\sin x\cdot \cos x=0 \\
			&\Leftrightarrow& \sin x \cdot \left(1+2\cos x\right)=0 \\
			&\Leftrightarrow& \hoac{&\sin x=0\\&1+2\cos x=0} \\
			&\Leftrightarrow& \hoac{&\sin x=0\\&\cos x=-\dfrac{1}{2}}\Leftrightarrow \hoac{&x=k\pi\\&x=\pm \dfrac{2\pi}{3}+k2\pi} (k\in\mathbb{Z}).
		\end{eqnarray*}	
		Vậy phương trình có các nghiệm là $k\pi, k\in\mathbb{Z}$ và $x=\pm \dfrac{2\pi}{3}+k2\pi, k\in\mathbb{Z}$.
	}
\end{vd}
\subsubsection{Bài tập tự luận}


%%=====Bài 1
\begin{bt}%[DCHT Toán 11 - KNTT -Thọ Bùi] %[1K1B4-5]
	Giải phương trình: $\sin 3 x-\cos 5 x=0$.
	\loigiai{
		Ta có: \begin{eqnarray*}
			\sin 3 x-\cos 5 x=0	&\Leftrightarrow& \sin 3x = \cos 5x \\
			&\Leftrightarrow &\cos \left(\dfrac{\pi}{2}-3x\right) =\cos{5x} \\
			&\Leftrightarrow& \hoac{&\dfrac{\pi}{2}-3x=5x+k2\pi\\&\dfrac{\pi}{2}-3x=-5x+k2\pi}\\
			&\Leftrightarrow& \hoac{&x=-\dfrac{\pi}{16}-k\dfrac{\pi}{4}\\&x=-\dfrac{\pi}{4}+k\pi} (k \in \mathbb{Z}).
		\end{eqnarray*}
	}
\end{bt}
\begin{bt}%[DCHT Toán 11 - KNTT -Thọ Bùi] %[1K1B4-5]
	Giải phương trình $\sin 2x+\sin\left(x+\dfrac{\pi}{6}\right)=0$.
	\loigiai{ Ta có:\allowdisplaybreaks
		\begin{eqnarray*}
			\sin 2x+\sin\left(x+\dfrac{\pi}{6}\right)=0&\Leftrightarrow& \sin 2x=-\sin\left(x+\dfrac{\pi}{6}\right) \\
			&\Leftrightarrow& \sin 2x=\sin\left(-x-\dfrac{\pi}{6}\right)\\
			&\Leftrightarrow& \hoac{&2x=-x-\dfrac{\pi}{6}+k2\pi\\&2x=\dfrac{7\pi}{6}+x+k2\pi}\\
			&\Leftrightarrow&  \hoac{&2x=-x-\dfrac{\pi}{6}+k2\pi\\&2x=\dfrac{7\pi}{6}+x+k2\pi}\\
			&\Leftrightarrow&x= \dfrac{7\pi}{6}+k2\pi. 
		\end{eqnarray*}	
	}
\end{bt}
%%=====Bài 1
\begin{bt}%[DCHT Toán 11 - KNTT -Thọ Bùi] %[1K1K4-5]
	Giải phương trình: $\tan(2x+1) + \cot x = 0$.
	\loigiai{
		Ta có: \begin{eqnarray*}
			\tan(2x+1) + \cot x = 0&\Leftrightarrow& \tan\left(2x+1\right)=-\cot x \\
			&\Leftrightarrow &\tan\left(2x+1\right)=\cot \left(-x\right) \\
			&\Leftrightarrow& \tan\left(2x+1\right)=\tan \left(\dfrac{\pi}{2}+x\right)\\
			&\Leftrightarrow& 2x+1=\dfrac{\pi}{2}+x+k\pi\\
			&\Leftrightarrow& x=\dfrac{\pi}{2}-1+k\pi, k\in\mathbb{Z}.
		\end{eqnarray*}
		
	}
\end{bt}



\begin{bt}%[DCHT Toán 11 - KNTT -Thọ Bùi] %[1K1K4-5]
	Tìm $x \in (-\pi;\pi)$ sao cho $\sin \left( x-\dfrac{\pi}{3}\right)+2 \cos \left( x+\dfrac{\pi}{6}\right)=0$.
	\loigiai{Ta có:\\
		\begin{eqnarray*}
			\sin \left(x-\dfrac{\pi}{3}\right)+2 \cos \left(x+\dfrac{\pi}{6}\right)=0
			\Leftrightarrow &-\cos \left(x+\dfrac{\pi}{6}\right)+2 \cos \left(x+\dfrac{\pi}{6}\right)=0\\
			\Leftrightarrow &\cos \left(x+\dfrac{\pi}{6}\right)=0\\
			\Leftrightarrow &x=\dfrac{\pi}{3}+ k\pi, k \in \mathbb{Z}.
		\end{eqnarray*}\\
		Cho $k=-1, 0$ ta được $x = -\dfrac{2\pi}{3}, \dfrac{\pi}{3}$.
	}
\end{bt}

%%=====Bài 1
\begin{bt}%[DCHT Toán 11 - KNTT -Thọ Bùi] %[1K1K4-5]
	Giải phương trình: $2\sin^2x-1+\cos 3x=0$.
	\loigiai{
		Ta có: \allowdisplaybreaks
		\begin{eqnarray*}
			2\sin^2x-1+\cos 3x=0&\Leftrightarrow&\cos 3x-\cos 2x=0\\
			&\Leftrightarrow&\cos 3x=\cos 2x\\
			&\Leftrightarrow&\hoac{&3x=2x+k2\pi\\&3x=-2x+k2\pi}\\
			&\Leftrightarrow&\hoac{&x=k2\pi\\&5x=k2\pi}\\
			&\Leftrightarrow&\hoac{&x=k2\pi\\&x=\dfrac{k2\pi}{5}},\left(k\in\mathbb{Z}\right).
		\end{eqnarray*}
	}
\end{bt}

\begin{bt}%[DCHT Toán 11 - KNTT -Thọ Bùi] %[1K1K4-5]
	Giải phương trình $\sin 3x+\cos 2x-\sin x=0$.
	\loigiai{
		Ta có: \allowdisplaybreaks
		\begin{eqnarray*}
			\sin 3x+\cos 2x-\sin x=0&\Leftrightarrow&\sin 3x-\sin x+\cos 2x=0\\
			&\Leftrightarrow&2\cos 2x\cdot \sin x+\cos 2x=0\\
			&\Leftrightarrow&\cos 2x\cdot \left(\sin x+1\right)=0\\
			&\Leftrightarrow&\hoac{&\cos 2x=0\\&\sin x=-1}\\
			&\Leftrightarrow&\hoac{&x=\dfrac{\pi}{4}+\dfrac{k\pi}{2}\\& x=-\dfrac{\pi}{2}+k2\pi}\left(k\in\mathbb{Z}\right).
		\end{eqnarray*}
	}
\end{bt}

\begin{bt}%[DCHT Toán 11 - KNTT -Thọ Bùi] %[1K1K4-5]
	Giải phương trình $\sin x\cdot\cos 2x=\sin 2x\cdot\cos 3x$.
	\loigiai{ Áp dụng công thức biến đổi tích thành tổn, ta có: \allowdisplaybreaks
		\begin{eqnarray*}
			\sin x\cdot\cos 2x=\sin 2x\cdot\cos 3x
			&\Leftrightarrow&\dfrac{1}{2}\left(\sin 3x-\sin x\right)=\dfrac{1}{2}\left(\sin 5x-\sin x\right)\\
			&\Leftrightarrow&\sin 5x=\sin 3x\\
			&\Leftrightarrow&\hoac{&5x=3x+k2\pi\\&5x =\pi -3x+k2\pi}\\
			&\Leftrightarrow&\hoac{&x=k\pi\\& x=\dfrac{\pi}{8}+\dfrac{k\pi}{4}}, \left(k\in \mathbb{Z}\right).
		\end{eqnarray*}
	}
\end{bt}
\begin{bt}%[DCHT Toán 11 - KNTT -Thọ Bùi] %[1K1K4-5]
	Giải phương trình: $\sin^4\dfrac{x}{2}+\cos^4 \dfrac{x}{2}=\dfrac{1}{2}$.
	\loigiai{ Ta có:
		\allowdisplaybreaks
		\begin{eqnarray*}
			\sin^4\dfrac{x}{2}+\cos^4 \dfrac{x}{2}=\dfrac{1}{2}
			&\Leftrightarrow&\left(\sin^2\dfrac{x}{2}+\cos^2\dfrac{x}{2}\right)^2-2\sin^2\dfrac{x}{2}\cos^2\dfrac{x}{2}=\dfrac{1}{2}\\
			&\Leftrightarrow&1-\dfrac{1}{2}\sin^2x=\dfrac{1}{2}\\
			&\Leftrightarrow&\sin^2 x=1\\
			&\Leftrightarrow&\cos x=0\\
			&\Leftrightarrow&x=\dfrac{\pi}{2}+k\pi, \left(k\in \mathbb{Z}\right).
		\end{eqnarray*}
		
	}
\end{bt}
\begin{bt}%[DCHT Toán 11 - KNTT -Thọ Bùi] %[1K1K4-5]
	Giải phương trình: $\tan^2 4x-\tan^2\left(3x-\dfrac{\pi}{3}\right)=0$.
	\loigiai{Điều kiện: $ \left\{\begin{aligned}
			&\cos 4x\neq 0\\
			&\cos\left(3x-\dfrac{\pi}{3}\right)\neq 0.\\
		\end{aligned}\right.$\\
		Phương trình đã cho tương đương với\\
		$\left[\tan 4x-\tan\left(3x-\dfrac{\pi}{3}\right)\right]\cdot\left[\tan 4x+\tan \left(3x-\dfrac{\pi}{3}\right)\right]=0\Leftrightarrow \hoac{&\tan 4x=\tan \left(3x-\dfrac{\pi}{3}\right)\\& \tan 4x=-\tan\left(3x-\dfrac{\pi}{3}\right).}$
		\begin{itemize}
			\item $\tan 4x=\tan\left(3x-\dfrac{\pi}{3}\right) \Leftrightarrow x=-\dfrac{\pi}{3}+k\pi, (k\in \mathbb{Z})$.
			\item $\tan 4x=-\tan\left(3x-\dfrac{\pi}{3}\right) \Leftrightarrow x=\dfrac{\pi}{21}+\dfrac{k2\pi}{7}, (k\in \mathbb{Z})$.
		\end{itemize}
		Các nghiệm này thỏa mãn các điều kiện.\\
		Vậy phương trình có nghiệm $x=-\dfrac{\pi}{3}+k\pi, x=-\dfrac{\pi}{21}+\dfrac{k2\pi}{7}, (k\in \mathbb{Z}).$
	}
\end{bt}
\begin{bt}%[DCHT Toán 11 - KNTT -Thọ Bùi] %[1K1K4-5]
	Giải phương trình $\sin^6x+\cos^6 x=\dfrac{7}{16}$.
	\loigiai{Ta có:
		\allowdisplaybreaks
		\begin{eqnarray*}
			\sin^6x+\cos^6 x=\dfrac{7}{16}
			&\Leftrightarrow&\left(\sin^2 x+\cos^2x\right)^3-3\sin^2x\cos^2x\left(\sin^2 x+\cos^2 x\right)=\dfrac{7}{16}\\
			&\Leftrightarrow&1-\dfrac{3}{4}\sin^2 2x=\dfrac{7}{16}\\
			&\Leftrightarrow&1-\dfrac{3}{4}\left(\dfrac{1-\cos 4x}{2}\right)=\dfrac{7}{16}\\
			&\Leftrightarrow&\cos 4x =-\dfrac{1}{2}\\
			&\Leftrightarrow&\cos 4x=\cos\dfrac{2\pi}{3}\\
			&\Leftrightarrow&x=\pm \dfrac{\pi}{6}+\dfrac{k\pi}{2}, \left(k\in \mathbb{Z}\right).
		\end{eqnarray*}
	}
\end{bt}

\subsubsection{Bài tập trắc nghiệm}
\Opensolutionfile{ans}[ans/ans-1K1-3-Dang5]
\begin{ex}%[DCHT Toán 11 - KNTT -Thọ Bùi] %[1K1K4-5]
	Tìm số nghiệm thuộc khoảng $(-\pi;\pi)$ của phương trình $\sin{x}+\sin2x=0.$
	\choice
	{\True $3$}
	{$1$}
	{$2$}
	{$4$}
	\loigiai{
		Phương trình tương đương với 
		$$\sin x\left(1+2\cos x\right)=0\Leftrightarrow \hoac{&\sin x=0\\&\cos x=-\dfrac{1}{2}.}$$
		\begin{itemize}
			\item $\sin x=0\Leftrightarrow x=k\pi\, ,(k\in \mathbb{Z})$.
			\item $\cos x=-\dfrac{1}{2}\Leftrightarrow x=\pm \dfrac{2\pi}{3}+k2\pi\, , (k\in \mathbb{Z})$.
		\end{itemize}
		Do $x\in (-\pi; \pi)\Rightarrow x\in \left\{0, -\dfrac{2\pi}{3}, \dfrac{2\pi}{3}    \right\}$.\\
		Vậy phương trình đã cho có 3 nghiệm thuộc khoảng $(-\pi; \pi)$.
	}
\end{ex}
\begin{ex}%[DCHT Toán 11 - KNTT -Thọ Bùi] %[1K1K4-5]
	Tìm số nghiệm thuộc khoảng $(0;\pi)$ của phương trình $\sin\left(x+\dfrac{\pi}{3}\right) + \sin5x=0$.
	\choice
	{$4$}
	{\True $5$}
	{$6$}
	{$7$}
	\loigiai{
		Phương trình đã cho tương đương với
		$$\sin 5x=\sin \left(-x-\dfrac{\pi}{3}\right)\Leftrightarrow \hoac{&5x=-x-\dfrac{\pi}{3}+k2\pi\\& 5x=x+\dfrac{4\pi}{3}+k2\pi}\Leftrightarrow \hoac{&x=- \dfrac{\pi}{18}+\dfrac{k\pi}{3}\\& x=\dfrac{\pi}{3}+\dfrac{k\pi}{2}}\, (k\in \mathbb{Z}).$$
		Do $x\in (0; \pi)$ nên $x\in \left\{\dfrac{5\pi}{18}, \dfrac{11\pi}{18}, \dfrac{17\pi}{18}, \dfrac{\pi}{3}, \dfrac{5\pi}{6}   \right\}$.\\
		Vậy phương trình có 5 nghiệm thuộc khoảng $(0; \pi)$.
	}
\end{ex}
\begin{ex}%[DCHT Toán 11 - KNTT -Thọ Bùi] %[1K1K4-5]
	Phương trình $\tan 2x + \tan x = 0$ có bao nhiêu nghiệm trong đoạn $[-4\pi;5\pi]$?
	\choice
	{\True $28$}
	{$27$}
	{$19$}
	{$18$}
	\loigiai{
		Điều kiện: $\heva{ 2x \neq \dfrac{\pi}{2}+k\pi \\
			x\neq \dfrac{\pi}{2}+ n\pi}
		\Leftrightarrow \heva{x\neq \dfrac{\pi}{4}+\dfrac{k\pi}{2} \\ 
			x \neq \dfrac{\pi}{2}+ n\pi}$, $k, n\in \mathbb{Z}$. \\
		Khi đó: \\
		$\tan 2x + \tan x= 0 \Leftrightarrow \tan 2x = - \tan x \Leftrightarrow \tan 2x = \tan (-x) \Leftrightarrow 2x = -x + m\pi \\
		\Leftrightarrow 3x = m\pi \Leftrightarrow x = \dfrac{m\pi}{3}, m\in \mathbb{Z}$ (thỏa điều kiện). \\
		Mà $x \in [-4\pi,5\pi]$ nên $-4\pi \leq \dfrac{m\pi}{3} \leq 5\pi \Leftrightarrow -12 \leq m \leq 15$. \\ 
		Vậy số nghiệm của phương trình là $28$.
	}
\end{ex}
\begin{ex}%[DCHT Toán 11 - KNTT -Thọ Bùi] %[1K1K4-5]
	Giải phương trình $\sin x+\cos \left( x-\dfrac{\pi}{2}\right)=2$.
	\choice 
	{$x=k\pi, k \in \mathbb{Z}$}
	{$x=\dfrac{\pi}{2}+k\pi, k \in \mathbb{Z}$}
	{$x=k2\pi, k \in \mathbb{Z}$}
	{\True $x=\dfrac{\pi}{2}+k2\pi, k \in \mathbb{Z}$}
	\loigiai{
		Ta có: $\sin x+\cos \left( x-\dfrac{\pi}{2}\right)=2\Leftrightarrow 2\sin x=2\Leftrightarrow \sin x=1\Leftrightarrow x=\dfrac{\pi}{2}+k2\pi,k\in\mathbb{Z}$.
	}
\end{ex} 
\begin{ex}%[DCHT Toán 11 - KNTT -Thọ Bùi] %[1K1K4-5]
	Họ nghiệm của phương trình $\tan 3x\cdot\tan x=1$ là
	\choice
	{$x=\dfrac{\pi }{8}+k\dfrac{\pi }{8}$, $\ k\in \mathbb{Z}$}
	{$x=\dfrac{\pi }{4}+k\dfrac{\pi }{4}$, $\ k\in \mathbb{Z}$}
	{\True $x=\dfrac{\pi }{8}+k\dfrac{\pi }{4}$, $\ k\in \mathbb{Z}$}
	{$x=\dfrac{\pi }{8}+k\dfrac{\pi }{2}$, $\ k\in \mathbb{Z}$}
	\loigiai{
		Điều kiện: $\heva{&\cos 3x\neq 0 \\&\cos x\neq 0}\Leftrightarrow \heva{&x\neq \dfrac{\pi }{6}+m\dfrac{\pi }{3} \\ &x\neq \dfrac{\pi }{2}+n\pi}, m,n\in \mathbb{Z}$.\\
		Khi đó:$$\tan 3x\cdot\tan x=1\Leftrightarrow \tan 3x=\dfrac{1}{\tan x}\Leftrightarrow\tan 3x=\cot x=\tan\left(\dfrac{\pi}{2}-x\right)\Leftrightarrow 3x=\dfrac{\pi}{2}-x+k\pi \Leftrightarrow x=\dfrac{\pi }{8}+k\dfrac{\pi }{4}.$$
		Nghiệm này thỏa mãn các điều kiện của phương trình.\\
		Vậy nghiệm của phương trình là $x=\dfrac{\pi}{8}+k\dfrac{\pi}{4}$. 
	}
\end{ex}
\begin{ex}%[DCHT Toán 11 - KNTT -Thọ Bùi] %[1K1K4-5]
	Tổng các nghiệm của phương trình $\sin x=\dfrac{-1}{2\sqrt{2}\cos x}$ trên đoạn $\left[0;2\pi\right]$  là
	\choice
	{$\dfrac{9\pi}{8}$}
	{$\dfrac{15\pi}{8}$}
	{\True $5\pi$}
	{$\dfrac{11\pi}{8}$}
	\loigiai{
		Phương trình tương đương với $\sin 2x=-\dfrac{1}{\sqrt{2}}\Leftrightarrow\hoac{&x=-\dfrac{\pi}{8}+k\pi\\ &x=\dfrac{5\pi}{8}+k\pi}\ (k\in\mathbb{Z})$.\\
		Do đó, tổng các nghiệm của phương trình đã cho trên đoạn $\left[0;2\pi\right]$ bằng
		$$\dfrac{7\pi}{8}+\dfrac{15\pi}{8}+\dfrac{5\pi}{8}+\dfrac{13\pi}{8}=5\pi.$$
	}
\end{ex}
\begin{ex}%[DCHT Toán 11 - KNTT -Thọ Bùi] %[1K1K4-5]
	Giải phương trình $\sin^2 2x=\cos^2\left(x-\dfrac{\pi}{4}\right)$.
	\choice 
	{$x=\dfrac{\pi}{4}+k\pi$, $x=\dfrac{\pi}{2}+ \dfrac{k\pi}{3}$, $k \in \mathbb{Z}$}
	{\True $x=\dfrac{\pi}{4}+k\pi$, $x=-\dfrac{\pi}{12}+ k\pi$, $x=\dfrac{7\pi}{12}+k\pi$, $k \in \mathbb{Z}$}
	{$x=-\dfrac{\pi}{4}+k\pi$, $x=-\dfrac{\pi}{12}+ \dfrac{k\pi}{3}$, $k \in \mathbb{Z}$}
	{$x=\dfrac{\pi}{4}+k\pi$, $x=-\dfrac{\pi}{12}+ \dfrac{k\pi}{3}$, $k \in \mathbb{Z}$}
	\loigiai{Ta có:
		{\allowdisplaybreaks
			\begin{align*}
				&\sin^2 2x=\cos^2\left(x-\dfrac{\pi}{4}\right)\Leftrightarrow \dfrac{1-\cos4x}{2}=\dfrac{1+\cos\left(2x-\dfrac{\pi}{2}\right)}{2}\\
				\Leftrightarrow &-\cos 4x=\sin 2x\Leftrightarrow 2\sin^2 2x-\sin 2x-1=0\\
				\Leftrightarrow &\hoac{&\sin 2x=1\\&\sin 2x=-\dfrac{1}{2}}\Leftrightarrow \hoac{& x=\dfrac{\pi}{4}+k\pi \\ & x=-\dfrac{\pi}{12}+k\pi\\ & x=\dfrac{7\pi}{12}+k\pi}\ (k\in\mathbb{Z}).
		\end{align*}}
	}
\end{ex} 
\begin{ex}%[DCHT Toán 11 - KNTT -Thọ Bùi] %[1K1K4-5]
	Có bao nhiêu điểm trên đường tròn lượng giác biểu diễn tất các nghiệm của phương trình $\sin 4x \cos x =\sin5x \cos2x$? 
	\choice{$ 2$ điểm}
	{$ 5$ điểm}
	{$9 $ điểm}
	{\True $14 $ điểm}
	\loigiai{Phương trình đã cho tương đương với  $$ \dfrac{1}{2} \left(\sin5x + \sin3x\right) =  \dfrac{1}{2} \left(\sin7x + \sin3x\right) \Leftrightarrow \sin5x = \sin7x \Leftrightarrow \hoac{&x= k\pi\\& x=\dfrac{\pi}{12}+\dfrac{k\pi}{6}}\, (k\in \mathbb{Z}).$$
		\begin{itemize}
			\item Cung lượng giác $x=k\pi$ có 2 điểm biểu diễn trên đường tròn lượng giác.
			\item Cung lượng giác $x=\dfrac{\pi}{12}+\dfrac{k\pi}{6}$ có 12 điểm biểu diễn trên đường tròn lượng giác, trong đó không có điểm nào trùng với các điểm biểu diễn của cung $x=k\pi$.
		\end{itemize}
	}
\end{ex}
\begin{ex}%[DCHT Toán 11 - KNTT -Thọ Bùi] %[1K1K4-5]
	Có bao nhiêu điểm trên đường tròn lượng giác biểu diễn tất các nghiệm của phương trình $\sin x +\cos x = \sqrt{2}\sin 2x$? 
	\choice{$2 $ điểm}
	{\True $ 3$ điểm}
	{ $ 4$ điểm}
	{$ 1$ điểm}
	\loigiai{ Phương trình đã cho tương đương với $$  \sin\left(x+\dfrac{\pi}{4}\right) = \sin2x \Leftrightarrow \hoac{&x = \dfrac{\pi}{4}+k2\pi \\ &x=\dfrac{\pi}{4}+\dfrac{k2\pi}{3}}\, (k\in \mathbb{Z}).$$
		Vậy có $3$ điểm biểu diễn. }
\end{ex}
\begin{ex}%[DCHT Toán 11 - KNTT -Thọ Bùi] %[1K1K4-5]
	Một vật thể chuyển động với vận tốc thay đổi có phương trình  $v(t) = 2+\sin\left(\pi t +\dfrac{\pi}{4}\right)$  ($t$ tính bằng giây, vận tốc tính bằng $\mathrm{m/s^2}$).
	Trong khoảng $1$ giây đầu chuyển động, thời  điểm vật thể đạt vận tốc $3$ $\mathrm{m/s^2}$ là 
	\choice{$1$ giây}
	{\True $\dfrac{1}{4}$ giây}
	{$\dfrac{1}{2}$ giây}
	{$\dfrac{3}{4}$ giây}
	\loigiai{Ta có: $2+\sin\left(\pi t +\dfrac{\pi}{4}\right) = 3 \Leftrightarrow \sin\left(\pi t +\dfrac{\pi}{4}\right)  = 1 \Leftrightarrow t = \dfrac{1}{4}+2k \, (k \in \mathbb{Z})$.\\
		Ta có, $0 \le t \le 1 \Leftrightarrow k = 0.$ Suy ra $t= \dfrac{1}{4}$.}
\end{ex}
\begin{ex}%[DCHT Toán 11 - KNTT -Thọ Bùi] %[1K1K4-5]
	Tìm số nghiệm thuộc khoảng $(0;2\pi)$ của phương trình $\sin x + 2\sin2x + \sin3x=0.$
	\choice
	{$6$}
	{$5$}
	{$4$}
	{\True $3$}
	\loigiai{
		Phương trình đã cho tương đương với 
		$$ 2\sin 2x\cos x+2\sin 2x=0\Leftrightarrow \sin2x(\cos{x}+1)=0\Leftrightarrow\hoac{&\sin 2x=0\\& \cos x=-1}\Leftrightarrow\hoac{&x=k\dfrac{\pi}{2}\\&x=\pi+k2\pi}\, (k\in\mathbb{Z}).$$
		Do $x\in (0; 2\pi)\Rightarrow x\in \left\{\dfrac{\pi}{2}, \pi, \dfrac{3\pi}{2}\right\}$.
	}
\end{ex}
\begin{ex}%[DCHT Toán 11 - KNTT -Thọ Bùi] %[1K1K4-5]
	Cho phương trình $\sin x + 2\sin 2x+\sin 3x = \cos x+2\cos 2x + \cos3x.$ Tính tổng $S$ tất cả các nghiệm trong đoạn $(0;\pi)$ của phương trình đã cho.
	\choice
	{\True $S=\dfrac{3\pi}{4}$}
	{$S=\dfrac{5\pi}{8}$}
	{$S=\dfrac{17\pi}{12}$}
	{$S=\dfrac{13\pi}{12}$}
	\loigiai{
		Phương trình đã cho tương đương với
		\begin{eqnarray*}
			&&\sin2x\cdot (\cos x+1)=\cos 2x\cdot (\cos x+1)\\
			&\Leftrightarrow &(\cos{x}+1)\cdot(\sin{2x}-\cos{2x})=0\\
			&\Leftrightarrow &\hoac{&\sin\left(2x-\dfrac{\pi}{4}\right)=0\\& \cos x=-1}\\
			&\Leftrightarrow &\hoac{&x=\pi+k2\pi\\&x=\dfrac{\pi}{8}+k\dfrac{\pi}{2}}\, (k\in\mathbb{Z}).
		\end{eqnarray*}
		Do $x\in (0; \pi)\Rightarrow x\in \left\{\dfrac{\pi}{8}, \dfrac{5\pi}{8}\right\}\Rightarrow S=\dfrac{3\pi}{4}.$
	}
\end{ex}
\begin{ex}%[DCHT Toán 11 - KNTT -Thọ Bùi] %[1K1K4-5]
	Cho phương trình $\sin x \cos x =2(\sin^4x+\cos^4x)-\dfrac{3}{2}$. Tính tổng $S$ tất cả các nghiệm thuộc $\left(0;\dfrac{\pi}{2}\right)$ của phương trình đã cho.
	\choice{\True $S=\dfrac{\pi}{2}$}
	{$S=\dfrac{5\pi}{12}$}
	{$S=\dfrac{\pi}{12}$}
	{$S=\dfrac{5\pi}{4}$}
	\loigiai{Ta có $\sin^4x+\cos^4x = \dfrac{3}{4}+\dfrac{\cos 4x}{4}$. Khi đó
		\begin{eqnarray*}
			&&\sin x \cos x =2(\sin^4x+\cos^4x)-\dfrac{3}{2}\\ &\Leftrightarrow& \sin x \cos x = \dfrac{1}{2}\cos 4x\\
			& \Leftrightarrow& \sin2x = \sin\left(\dfrac{\pi}{2}-4x\right)\\ &\Leftrightarrow &\hoac{&x = \dfrac{\pi}{12}+\dfrac{k\pi}{3}\\& x =- \dfrac{\pi}{4}-k\pi} (k \in \mathbb{Z}) .
		\end{eqnarray*}
		Suy ra các nghiệm thuộc $\left(0; \dfrac{\pi}{2}\right)$ là $\dfrac{5\pi}{12};\dfrac{\pi}{12}$. Vậy $S=\dfrac{\pi}{2}$.
	}
	
\end{ex}
\begin{ex}%[DCHT Toán 11 - KNTT -Thọ Bùi] %[1K1K4-5]
	Phương trình  $\tan\left(\dfrac{\pi}{3}-x\right)\cdot\tan\left(\dfrac{\pi}{2}+2x\right)=1$ có nghiệm là 
	\choice
	{$x=-\dfrac{\pi}{6} +\mathrm{k}\pi, k\in \mathbb{Z}$}
	{$x=\dfrac{\pi}{6} +\mathrm{k}\pi, k\in \mathbb{Z}$}
	{\True $x=-\dfrac{\pi}{3}+k\pi,k\in \mathbb{Z}$}
	{ $x=\dfrac{5\pi}{6} +\mathrm{k}\pi, k\in \mathbb{Z}$}
	\loigiai{Điều kiện xác định $ \heva{&\cos \left(\dfrac{\pi}{3}-x\right)\neq 0\\&\cos\left(\dfrac{\pi}{2}+2x\right)\neq 0.}$
		\begin{eqnarray*}
			\tan\left(\dfrac{\pi}{3}-x\right)\cdot\tan\left(\dfrac{\pi}{2}+2x\right)=1 & \Leftrightarrow  & \tan\left(\dfrac{\pi}{3}-x\right)=\cot\left(\dfrac{\pi}{2}+2x\right)\\
			& \Leftrightarrow  & \tan\left(\dfrac{\pi}{3}-x\right)=\tan (-2x)\\
			&\Leftrightarrow & x=-\dfrac{\pi}{3}+k\pi,k\in \mathbb{Z}.
		\end{eqnarray*}
	}
\end{ex} 
\begin{ex}%[DCHT Toán 11 - KNTT -Thọ Bùi] %[1K1K4-5]
	Nghiệm của phương trình $\tan2x-\cot\left( x+\dfrac{\pi}{4}\right) =0$ có dạng $x=\dfrac{\pi}{n}+\dfrac{k\pi}{m}, k\in \mathbb{Z}$. Khi đó $m\cdot n$ bằng
	\choice
	{$8$}
	{$32$}
	{\True $36$}
	{$12$}
	\loigiai{Ta có: $\tan2x-\cot\left( x+\dfrac{\pi}{4}\right) =0\Leftrightarrow \tan2x=\tan\left(\dfrac{\pi}{4}-x\right)\Leftrightarrow 2x=\dfrac{\pi
		}{4}-x+k\pi \Leftrightarrow x=\dfrac{\pi}{12}+k\dfrac{\pi}{3}$.\\
		Suy ra $n=12,m=3\Rightarrow m\cdot n=36$.
	}
\end{ex}
\Closesolutionfile{ans}
\begin{indapan}{10}
	{ans/ans-1K1-3-Dang5}
\end{indapan}