
\setcounter{section}{3}
\section{Phương trình lượng giác cơ bản}
\subsection{Tóm tắt lý thuyết}
\begin{tomtat}
\subsubsection{Phương trình $\sin x=m$}
\begin{itemize}
	\item Với $|m|>1$ thì phương trình $\sin x=m$ vô nghiệm.
	\item Với $|m|\leq 1$, sẽ tồn tại duy nhất $\alpha \in \left[-\dfrac{\pi}{2}; \dfrac{\pi}{2}\right]$ thỏa mãn $\sin\alpha=m$. Khi đó
	\begin{center}
		$\sin x=m\Leftrightarrow\sin x=\sin\alpha\Leftrightarrow\hoac{&x=\alpha+k2\pi\\&x=\pi-\alpha+k2\pi}$ ($k\in \mathbb{Z}$).
	\end{center}
\item Nếu số đo của góc $\alpha$ được đo bằng đơn vị độ thì
\begin{center}
	$\sin x=\sin\alpha^\circ\Leftrightarrow\hoac{&x=\alpha^\circ+k360^\circ\\&x=180^\circ-\alpha^\circ+k360^\circ}$ ($k\in\mathbb{Z}$).
\end{center}
\item Tổng quát,
\begin{center}
	$\sin f(x)=\sin g(x)\Leftrightarrow\hoac{&f(x)=g(x)+k2\pi\\&f(x)=\pi - g(x)+k2\pi}$ ($k\in\mathbb{Z}$).
\end{center}
\item Một số trường hợp đặt biệt:
\begin{enumEX}[\faCheckCircleO]{1}
	\item $\sin x=0\Leftrightarrow x=k\pi$, $k\in\mathbb{Z}$.
	\item $\sin x=1\Leftrightarrow x=\dfrac{\pi}{2}+k2\pi$, $k\in\mathbb{Z}$.
	\item $\sin x=-1\Leftrightarrow x=-\dfrac{\pi}{2}+k2\pi$, $k\in \mathbb{Z}$.
\end{enumEX}
\end{itemize}
	\subsubsection{Phương trình $\cos x=m$}
\begin{itemize}
	\item Với $|m|>1$ thì phương trình $\cos x=m$ vô nghiệm.
	\item Với $|m|\leq 1$, sẽ tồn tại duy nhất $\alpha\in\left[0; \pi\right]$ thỏa mãn $\cos x=m$. Khi đó
	\begin{center}
		$\cos x=m\Leftrightarrow\cos x=\cos \alpha\Leftrightarrow\hoac{&x=\alpha+k2\pi\\&x=-\alpha+k2\pi}$ ($k\in\mathbb{Z}$).
	\end{center}
\item Nếu số đo của góc $\alpha$ được đo bằng đơn vị độ thì
\begin{center}
	$\cos x=\cos\alpha\Leftrightarrow\hoac{&x=\alpha^\circ+k360^\circ\\&x=-\alpha^\circ+k360^\circ}$ ($k\in\mathbb{Z}$).
\end{center}
\item Tổng quát,
\begin{center}
	$\cos f(x)=\cos g(x)\Leftrightarrow\hoac{&f(x)=g(x)+k2\pi\\&f(x)=-g(x)+k2\pi}$ ($k\in \mathbb{Z}$)
\end{center}
\item Một số trường hợp đặc biệt:
\begin{enumEX}[\faCheckCircleO]{1}
\item $\cos x=0\Leftrightarrow x=\dfrac{\pi}{2}+k\pi$, $k\in\mathbb{Z}$.
\item $\cos x=1\Leftrightarrow x=k2\pi$, $k\in\mathbb{Z}$.
\item $\cos x=-1\Leftrightarrow x=\pi+k2\pi$, $k\in\mathbb{Z}$.
\end{enumEX}
\end{itemize}
\subsubsection{Phương trình $\tan x=m$}
\begin{itemize}
	\item Với mọi $m\in\mathbb{R}$, tồn tại duy nhất $\alpha\in\left(-\dfrac{\pi}{2};\dfrac{\pi}{2}\right)$ thỏa mãn $\tan \alpha=m$. Khi đó
	\begin{center}
		$\tan x=m\Leftrightarrow\tan x=\tan \alpha\Leftrightarrow x=\alpha+k\pi$ ($k\in\mathbb{Z}$).
	\end{center}
\item Nếu số đo của góc $\alpha$ được đo bằng đơn vị độ thì
\begin{center}
	$\tan x=\tan\alpha^\circ\Leftrightarrow x=\alpha^\circ+k180^\circ$, $k\in \mathbb{Z}$
\end{center}
\item Tổng quát,
\begin{center}
	$\tan f(x)=\tan g(x)\Leftrightarrow f(x)=g(x)+k\pi$, $k\in\mathbb{Z}$.
\end{center}
\end{itemize}
\subsubsection{Phương trình $\cot x=m$}
\begin{itemize}
	\item Với mọi $m\in\mathbb{R}$, tồn tại duy nhất $\alpha\in\left(0;\pi\right)$ thỏa mãn $\cot\alpha=m$. Khi đó
	\begin{center}
		$\cot x=m\Leftrightarrow\cot x=\cot\alpha\Leftrightarrow x=\alpha+k\pi$ $k\in\mathbb{Z}$.
	\end{center}
\item Nếu số đo của góc $\alpha$ được đo bằng đơn vị độ thì
\begin{center}
	$\cot x=\cot\alpha^\circ\Leftrightarrow x=\alpha^\circ+k180^\circ$, $k\in\mathbb{Z}$.
\end{center}
\item Tổng quát,
\begin{center}
	$\cot f(x)=\cot g(x)\Leftrightarrow f(x)=g(x)+k\pi$, $k\in\mathbb{Z}$.
\end{center}
\end{itemize}
\end{tomtat}

