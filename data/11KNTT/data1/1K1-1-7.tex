\begin{dang}{Giá trị lượng giác của các góc có liên quan đặc biệt}
	% \begin{enumerate}
	% 	\item Góc đối nhau ($\alpha$ và $-\alpha$)
	% 	\immini{\begin{itemize}
	% 			\item $\cos (-\alpha)=\cos \alpha$
	% 			\item $\sin (-\alpha) =-\sin \alpha$
	% 			\item $\tan (-\alpha) =-\tan \alpha$
	% 			\item $\cot (-\alpha) =-\cot \alpha$
	% 	\end{itemize}}
	% 	{\begin{tikzpicture}[line join = round, line cap = round, >=stealth, font=\footnotesize, scale=0.5]
	% 			\tikzset{label style/.style={font=\footnotesize}}
	% 			\path (0,0) coordinate (O)
	% 			(3,0) coordinate (A)
	% 			(0:0)++(120:3) coordinate (M)
	% 			(0:0)++(-120:3) coordinate (N)
	% 			(0,4) coordinate (C)
	% 			(0,-4) coordinate (D)
	% 			($(O)!(M)!(C)$) coordinate (E)
	% 			($(O)!(N)!(D)$) coordinate (F)
	% 			;
	% 			\draw[->] (-4,0) -- (4,0) node[above,blue]{$x$};
	% 			\draw[->] (0,-4) -- (0.,4) node[left,blue]{$y$};
	% 			\draw[orange] (O) circle (3cm);
	% 			\draw[rotate=0,->,red] (0.5,0) arc (0:120:0.5cm);
	% 			\draw[rotate=0,->,green!50!black] (0.6,0) arc (0:-120:0.6cm);
	% 			\draw (0,0) node[above right=2pt,blue] {$\alpha$} (0,-0) node[below right=2pt,blue]{$-\alpha$};
	% 			\draw[dashed] (E)--(M)--(N)--(F);
	% 			\draw[green!50!black] (M)--(O);
	% 			\draw[red] (N)--(O);
	% 			\foreach \p/\r in {A/-45,M/120,N/-120,O/-150}
	% 			\fill (\p) circle (1pt) node[shift={(\r:3mm)},blue]{$\p$};
	% 	\end{tikzpicture}}
	% 	\item Góc bù nhau ($\alpha$ và $\pi-\alpha$)
	% 	\immini{\begin{itemize}
	% 			\item $\sin (\pi -\alpha)=\sin \alpha$
	% 			\item $\cos (\pi -\alpha) =-\cos \alpha$
	% 			\item $\tan (\pi -\alpha) =-\tan \alpha$
	% 			\item $\cot (\pi -\alpha) =-\cot \alpha$
	% 	\end{itemize}}
	% 	{\begin{tikzpicture}[line join = round, line cap = round, >=stealth, font=\footnotesize, scale=0.5]
	% 			\tikzset{label style/.style={font=\footnotesize}}
	% 			\path (0,0) coordinate (O)
	% 			(3,0) coordinate (A)
	% 			(0:0)++(30:3) coordinate (M)
	% 			(0:0)++(150:3) coordinate (N)
	% 			(4,0) coordinate (C)
	% 			(-4,0) coordinate (D)
	% 			($(O)!(M)!(C)$) coordinate (E)
	% 			($(O)!(N)!(D)$) coordinate (F)
	% 			;
	% 			\draw[->] (-4,0) -- (4,0) node[above,blue]{$x$};
	% 			\draw[->] (0,-4) -- (0.,4) node[left,blue]{$y$};
	% 			\draw[orange] (O) circle (3cm);
	% 			\draw[rotate=0,->,red] (0.5,0) arc (0:150:0.5cm);
	% 			\draw[rotate=0,->,green!50!black] (1.6,0) arc (0:30:1.6cm);
	% 			\draw (2,0) node[above,blue] {$\alpha$} (0.3,1.5) node[below,blue]{$\pi-\alpha$};
	% 			\draw[dashed] (E)--(M)--(N)--(F);
	% 			\draw[red] (O)--(N);
	% 			\draw[green!50!black] (O)--(M);
	% 			\foreach \p/\r in {A/-45,M/30,N/150,O/-130}
	% 			\fill (\p) circle (1pt) node[shift={(\r:3mm)},blue]{$\p$};
	% 	\end{tikzpicture}}
	% 	\item Góc phụ nhau ($\alpha$ và $\dfrac{\pi}{2}-\alpha$)
	% 	\immini{\begin{itemize}
	% 			\item $\sin \left( \dfrac{\pi}{2}-\alpha\right)=\cos \alpha$
	% 			\item $\cos \left( \dfrac{\pi}{2}-\alpha\right)=\sin \alpha$
	% 			\item $\tan \left( \dfrac{\pi}{2}-\alpha\right)=\cot \alpha$
	% 			\item $\cot \left( \dfrac{\pi}{2}-\alpha\right)=\tan \alpha$
	% 	\end{itemize}}
	% 	{\begin{tikzpicture}[line join = round, line cap = round, >=stealth, font=\footnotesize, scale=0.5]
	% 			\tikzset{label style/.style={font=\footnotesize}}
	% 			\path (0,0) coordinate (O)
	% 			(3,0) coordinate (A)
	% 			(0:0)++(20:3) coordinate (M)
	% 			(0:0)++(70:3) coordinate (N)
	% 			(0,4) coordinate (C)
	% 			(4,0) coordinate (D)
	% 			($(O)!(M)!(C)$) coordinate (E)
	% 			($(O)!(N)!(D)$) coordinate (F)
	% 			(2.82,0) coordinate (G)
	% 			(0,2.82) coordinate (H)
	% 			;
	% 			\draw[->] (-4,0) -- (4,0) node[above,blue]{$x$};
	% 			\draw[->] (0,-4) -- (0.,4) node[left,blue]{$y$};
	% 			\draw[orange] (O) circle (3cm);
	% 			\draw[rotate=0,->,red] (0.7,0) arc (0:70:0.7cm);
	% 			\draw[rotate=0,->,green!50!black] (1.6,0) arc (0:20:1.6cm);
	% 			\draw (2,0) node[above,blue] {$\alpha$} (1,-.2) node[below,blue]{$\frac{\pi}{2}-\alpha$};
	% 			\draw[dashed] (E)--(M) (F)--(N) (G)--(M) (H)--(N);
	% 			\draw[dashed] (-3,-3)--(3,3);
	% 			\draw[->] (0.8,-.5)--(0.5,0.45);
	% 			\draw[red] (O)--(N);
	% 			\draw[green!50!black] (O)--(M);
	% 			\foreach \p/\r in {A/-45,M/20,N/70,O/-220}
	% 			\fill (\p) circle (1pt) node[shift={(\r:3mm)},blue]{$\p$};
	% 	\end{tikzpicture}}
	% 	\item Góc hơn kém $\pi$ ($\alpha$ và $\pi+\alpha$)
	% 	\immini{\begin{itemize}
	% 			\item $\sin (\pi +\alpha)=-\sin \alpha$
	% 			\item $\cos (\pi +\alpha)=-\cos \alpha$
	% 			\item $\tan (\pi +\alpha)=\tan \alpha$
	% 			\item $\cot (\pi +\alpha)=\cot \alpha$
	% 	\end{itemize}}
	% 	{\begin{tikzpicture}[line join = round, line cap = round, >=stealth, font=\footnotesize, scale=0.5]
	% 			\tikzset{label style/.style={font=\footnotesize}}
	% 			\path (0,0) coordinate (O)
	% 			(3,0) coordinate (A)
	% 			(0:0)++(60:3) coordinate (M)
	% 			(0:0)++(240:3) coordinate (N)
	% 			;
	% 			\draw[->] (-4,0) -- (4,0) node[above,blue]{$x$};
	% 			\draw[->] (0,-4) -- (0.,4) node[left,blue]{$y$};
	% 			\draw[orange] (O) circle (3cm);
	% 			\draw[rotate=0,->,red] (1.7,0) arc (0:240:1.7cm);
	% 			\draw[rotate=0,->,green!50!black] (0.6,0) arc (0:60:0.6cm);
	% 			\draw (1,0) node[above,blue] {$\alpha$};
	% 			\draw (-1.2,1) node[below,blue,rotate=60]{$\pi+\alpha$};
	% 			\draw[red] (O)--(N);
	% 			\draw[green!50!black] (O)--(M);
	% 			\foreach \p/\r in {A/-45,M/60,N/240,O/-150}
	% 			\fill (\p) circle (1pt) node[shift={(\r:3mm)},blue]{$\p$};
	% 	\end{tikzpicture}}
	% \end{enumerate}
\end{dang}
\subsubsection{Ví dụ minh hoạ}
\begin{vd}%[DCHT Toán 11 - KNTT - Nguyễn Hoài Nam]%[1K1B1-7]
	Không dùng máy tính cầm tay, tính giá trị của biểu thức
	\[A=\cos \dfrac{\pi}{9}-\sin \dfrac{13\pi}{36}+\cos \dfrac{5\pi}{36}+\cos \dfrac{8\pi}{9}-\cos \pi.\]
	\loigiai{
		Ta có
		\begin{eqnarray*}
			A&=& \cos \dfrac{\pi}{9}-\sin \dfrac{13\pi}{36}+\cos \dfrac{5\pi}{36}+\cos \dfrac{8\pi}{9}-\cos \pi\\
			&=& \cos \dfrac{\pi}{9}-\sin \dfrac{13\pi}{36}+\sin \left(\dfrac{\pi}{2}-\dfrac{5\pi}{36}\right)+\cos \left(\pi-\dfrac{\pi}{9}\right)+1\\
			&=& \cos \dfrac{\pi}{9}-\sin \dfrac{13\pi}{36}+\sin \dfrac{13\pi}{36}-\cos \dfrac{\pi}{9}+1\\
			&=&1.
		\end{eqnarray*}
		Vậy $A=1$.
	}
\end{vd}
\begin{vd}%[DCHT Toán 11 - KNTT - Nguyễn Hoài Nam]%[1K1B1-7]
	Tính giá trị đúng của các biểu thức sau (không dùng máy tính cầm tay).
	\begin{enumEX}{2}
		\item $A=\sin \dfrac{\pi}{36} + \sin \dfrac{5\pi}{6} - \sin \dfrac{35\pi}{36} + \sin \pi$;
		\item $B=\cos \dfrac{\pi}{12} + \cos \dfrac{7\pi}{36} - \sin \dfrac{5\pi}{12} - \sin \dfrac{11\pi}{36}$;
		\item $C=\tan \dfrac{5\pi}{36}\cdot \tan \dfrac{\pi}{4}\cdot \tan \dfrac{23\pi}{36}$;
		\item $D=\cot \dfrac{\pi}{18}\cdot \cot \dfrac{\pi}{6} \cdot \cot \dfrac{5\pi}{9}$.
	\end{enumEX}
	\loigiai
	{
		\begin{enumerate}[a)]
			\item $\begin{aligned}[t]
				A &=\sin \dfrac{\pi}{36} + \sin \dfrac{5\pi}{6} - \sin \dfrac{35\pi}{36} + \sin \pi = \sin \dfrac{\pi}{36} + \sin\left(\pi-\dfrac{\pi}{6}\right) - \sin\left(\pi-\dfrac{\pi}{36}\right) + \sin \pi\\
				&=\sin \dfrac{\pi}{36} + \sin \dfrac{\pi}{6} - \sin \dfrac{\pi}{36} + \sin \pi = \sin \dfrac{\pi}{6} + \sin \pi = \dfrac{1}{2}+0 = \dfrac{1}{2}.
			\end{aligned}$
			\item $\begin{aligned}[t]
				B &=\cos \dfrac{\pi}{12} + \cos \dfrac{7\pi}{36} - \sin \dfrac{5\pi}{12} - \sin \dfrac{11\pi}{36} = \cos \dfrac{\pi}{12} + \cos \dfrac{7\pi}{36} - \sin\left(\dfrac{\pi}{2}-\dfrac{\pi}{12}\right) - \sin\left(\dfrac{\pi}{2}-\dfrac{7\pi}{36}\right)\\
				&=\cos \dfrac{\pi}{12} + \cos \dfrac{7\pi}{36} - \cos \dfrac{\pi}{12} - \cos \dfrac{7\pi}{36} = 0.
			\end{aligned}$
			\item $\begin{aligned}[t]
				C &=\tan \dfrac{5\pi}{36}\cdot \tan \dfrac{\pi}{4}\cdot \tan \dfrac{23\pi}{36} = \tan \dfrac{5\pi}{36}\cdot \tan \dfrac{\pi}{4}\cdot \tan\left(\pi - \dfrac{13\pi}{36}\right)\\
				&= \tan \dfrac{5\pi}{36}\cdot \tan \dfrac{\pi}{4}\cdot \left(-\tan \dfrac{13\pi}{36}\right)=-\tan \dfrac{5\pi}{36}\cdot \tan \dfrac{\pi}{4}\cdot  \tan \dfrac{13\pi}{36} \\
				&=-\tan \dfrac{5\pi}{36}\cdot \tan \dfrac{\pi}{4}\cdot \tan\left(\dfrac{\pi}{2} - \dfrac{5\pi}{36}\right) = -\tan \dfrac{5\pi}{36}\cdot \tan \dfrac{\pi}{4}\cdot \cot \dfrac{5\pi}{36}\\
				&=-\left(\tan \dfrac{5\pi}{36} \cdot \cot \dfrac{5\pi}{36}\right)\tan \dfrac{\pi}{4} = -1\cdot 1 = -1.
			\end{aligned}$
			\item $\begin{aligned}[t]
				D &=\cot \dfrac{\pi}{18}\cdot \cot \dfrac{\pi}{6} \cdot \cot \dfrac{5\pi}{9} = \cot \dfrac{\pi}{18}\cdot \cot \dfrac{\pi}{6} \cdot \cot \left(\pi - \dfrac{4\pi}{9}\right)\\
				&=\cot \dfrac{\pi}{18}\cdot \cot \dfrac{\pi}{6} \cdot \left(-\cot \dfrac{4\pi}{9}\right) = -\cot \dfrac{\pi}{18}\cdot \cot \dfrac{\pi}{6}\cdot \cot \dfrac{4\pi}{9}\\
				&=-\cot \dfrac{\pi}{18}\cdot \cot \dfrac{\pi}{6}\cdot \cot\left(\dfrac{\pi}{2} - \dfrac{\pi}{18}\right) = -\cot \dfrac{\pi}{18}\cdot \cot \dfrac{\pi}{6} \cdot \tan \dfrac{\pi}{18}\\
				&=-\left(\cot \dfrac{\pi}{18} \cdot \tan \dfrac{\pi}{18}\right) \cot \dfrac{\pi}{6} = -1\cdot \sqrt{3} = -\sqrt{3}.
			\end{aligned}$
		\end{enumerate}
	}
\end{vd}
% \subsubsection{Bài tập rèn luyện}
\subsubsection{Bài tập tự luận}
\begin{bt}%[DCHT Toán 11 - KNTT - Nguyễn Hoài Nam]%[1K1Y1-7]
	Tính $\sin\alpha+\sin\left(\dfrac{\pi}{2}+\alpha\right)-\cos\alpha+\cos\left(\dfrac{\pi}{2}+\alpha\right)$.
	\loigiai{
		Ta có
		$\sin\left(\dfrac{\pi}{2}+\alpha\right)=\sin\left(\dfrac{\pi}{2}-(-\alpha)\right)=\cos (-\alpha)=\cos\alpha$.\\
		$\cos\left(\dfrac{\pi}{2}+\alpha\right)=\cos\left(\dfrac{\pi}{2}-(-\alpha)\right)=\sin(-\alpha)=-\sin\alpha$.\\
		Suy ra 	$\sin\alpha+\sin\left(\dfrac{\pi}{2}+\alpha\right)-\cos\alpha+\cos\left(\dfrac{\pi}{2}+\alpha\right)=\sin\alpha+\cos \alpha-\cos\alpha-\sin\alpha=0$.
	}
\end{bt}
\begin{bt}%[DCHT Toán 11 - KNTT - Nguyễn Hoài Nam]%[1K1B1-7]
	Cho $\cos \alpha=\dfrac{-3}{7},\, \left(\dfrac{\pi}{2}<\alpha<\pi\right)$. Tính $\sin( - \alpha)$.
	\loigiai{
		Do $\dfrac{\pi}{2}<\alpha<\pi$ nên $\sin \alpha >0$.\\
		Ta có $\sin^2 \alpha+\cos^2 \alpha=1$, suy ra $\sin \alpha= \sqrt{1-\cos^2 \alpha}=\sqrt{1-\dfrac{9}{49}}=\dfrac{2\sqrt{10}}{7}$.\\
		Mà $\sin (- \alpha) = - \sin \alpha = -\dfrac{2\sqrt{10}}{7}$.}
\end{bt}
\begin{bt}%[DCHT Toán 11 - KNTT - Nguyễn Hoài Nam]%[1K1K1-7]
	Với mọi tam giác $ABC$, chứng minh $\cos\dfrac{A-B-C}{2} =\sin A$.
	\loigiai{
		Với mọi tam giác $ABC$, ta có 
		$$\cos\dfrac{A-B-C}{2} = \cos\dfrac{2A-(A+B+C)}{2} = \cos\dfrac{2A-180^\circ}{2} = \cos(A-90^\circ) = \sin A.$$
	}
\end{bt}
\begin{bt}%[DCHT Toán 11 - KNTT - Nguyễn Hoài Nam]%[1K1G1-7]
	Với mọi $\alpha\in\mathbb{R}$, tính giá trị của biểu thức $\cos\alpha + \cos\left( \alpha + \dfrac{\pi}{5} \right) + \ldots + \cos\left( \alpha + \dfrac{9\pi}{5} \right)$.
	\loigiai{
		Áp dụng quan hệ hai cung hơn kém $\pi$, ta có
		\allowdisplaybreaks
		\begin{eqnarray*}
			& & \cos\left( \alpha + \dfrac{5\pi}{5} \right) = -\cos\alpha, \\
			& & \cos\left( \alpha + \dfrac{6\pi}{5} \right) = -\cos\left( \alpha + \dfrac{\pi}{5} \right), \\
			& & \cos\left( \alpha + \dfrac{7\pi}{5} \right) = -\cos\left( \alpha + \dfrac{2\pi}{5} \right), \\
			& & \cos\left( \alpha + \dfrac{8\pi}{5} \right) = -\cos\left( \alpha + \dfrac{3\pi}{5} \right), \\
			& & \cos\left( \alpha + \dfrac{9\pi}{5} \right) = -\cos\left( \alpha + \dfrac{4\pi}{5} \right).
		\end{eqnarray*}
		Suy ra $\cos\alpha + \cos\left( \alpha + \dfrac{\pi}{5} \right) + \ldots + \cos\left( \alpha + \dfrac{9\pi}{5} \right) = 0$.
	}
\end{bt}
\subsubsection{Bài tập trắc nghiệm}
\Opensolutionfile{ans}[ans/ans-1K1-1-Dang7]
\begin{ex}%[DCHT Toán 11 - KNTT - Nguyễn Hoài Nam]%[1K1Y1-7]
	Trong các đẳng thức sau, đẳng thức nào đúng?
	\choice
	{$\sin\left(180^\circ-a\right)=-\cos a$}
	{$\sin\left(180^\circ-a\right)=-\sin a$}
	{\True $\sin\left(180^\circ-a\right)=\sin a$}
	{$\sin\left(180^\circ-a\right)=\cos a$}
	\loigiai{Ta có: $\sin\left(180^\circ-a\right)=\sin a$.}
\end{ex}

\begin{ex}%[DCHT Toán 11 - KNTT - Nguyễn Hoài Nam]%[1K1Y1-7]
	Tìm mệnh đề đúng trong các mệnh đề sau.
	\choice
	{$\tan \left(\pi - a\right)=\tan a$}
	{$\cos \left(\dfrac{\pi}{2} - a\right)=-\sin a$}
	{\True $\cot \left(\dfrac{\pi}{2} + a\right)=-\tan a$}
	{$\sin\left(\pi + a\right)= \sin a$}
	\loigiai{ Ta có $\cot \left(\dfrac{\pi}{2} + a\right)=\left(\dfrac{\pi}{2} -(- a)\right)=\tan (-a)=-\tan a$.}
\end{ex}

\begin{ex}%[DCHT Toán 11 - KNTT - Nguyễn Hoài Nam]%[1K1Y1-7]
	Trong các đẳng thức sau, đẳng thức nào \textbf{sai}?
	\choice
	{$\sin\left(\dfrac{\pi}{2}-x\right)=\cos x$}
	{$\sin\left(\dfrac{\pi}{2}+x\right)=\cos x$}
	{$\tan\left(\dfrac{\pi}{2}-x\right)=\cot x$}
	{\True $\tan\left(\dfrac{\pi}{2}+x\right)=\cot x$}
	\loigiai{
		Ta có: $\tan\left(\dfrac{\pi}{2}+x\right)=\tan\left(\dfrac{\pi}{2}-(-x)\right)=\cot(-x)=-\cot x$. }
\end{ex}

\begin{ex}%[DCHT Toán 11 - KNTT - Nguyễn Hoài Nam]%[1K1Y1-7]
	Giá trị của $\sin \dfrac{47\pi}{6}$ là
	\choice
	{$\dfrac{\sqrt{3}}{2}$}
	{$\dfrac{1}{2}$}
	{\True $-\dfrac{1}{2}$}
	{$\dfrac{\sqrt{2}}{2}$}
	\loigiai{
		Ta có $\sin \dfrac{47\pi}{6}=\sin\left(8\pi-\dfrac{\pi}{6}\right)=\sin\left(-\dfrac{\pi}{6}\right)=-\sin\dfrac{\pi}{6} =-\dfrac{1}{2}$.	
	}
\end{ex}

\begin{ex}%[DCHT Toán 11 - KNTT - Nguyễn Hoài Nam]%[1K1Y1-7]
	Tính $\cos 18^{\circ}-\cos 342^{\circ}$
	\choice
	{$1$}
	{\True $0$}
	{$2\cos 18^{\circ}$}
	{$-2\cos 18^{\circ}$}
	\loigiai{
		Ta có: $342^{\circ}=-18^{\circ}+360^{\circ}$. 
		Nên  $\cos 342^{\circ}=\cos (-18^\circ)=\cos 18^{\circ}$.\\
		$\Rightarrow \cos{18^{\circ}}-\cos 342^{\circ}=\cos 18^{\circ}-\cos 18^{\circ}=0$.}
\end{ex}

\begin{ex}%[DCHT Toán 11 - KNTT - Nguyễn Hoài Nam]%[1K1Y1-7]
	Chỉ ra đẳng thức \textbf{sai} trong các đẳng thức sau
	\choice
	{$\sin{\dfrac{2\pi}{3}}=\dfrac{\sqrt{3}}{2}$}
	{$\cos{\dfrac{2\pi}{3}}=\dfrac{\sqrt{3}}{2}=-\dfrac{1}{2}$}
	{$\sin{\dfrac{2\pi}{3}}=\sin{\dfrac{\pi}{3}}$}
	{\True $\cos \dfrac{2\pi}{3}=\cos \dfrac{\pi}{3}$}
	\loigiai{ 
		Ta có: $\cos\dfrac{2\pi}{3}=\cos \left(\pi-\dfrac{\pi}{3}\right)=-\cos\dfrac{\pi}{3}$.
	}
\end{ex}

\begin{ex}%[DCHT Toán 11 - KNTT - Nguyễn Hoài Nam]%[1K1Y1-7]
	$\tan\alpha-(-\tan\alpha)$ bằng 
	\choice
	{$0$}{\True $2\tan\alpha$}{$-2\tan\alpha$}{$\tan2\alpha$}
	\loigiai{Ta có: $\tan\alpha-(-\tan\alpha)=\tan\alpha+\tan\alpha=2\tan\alpha$.}
\end{ex}

%--------------------
\begin{ex}%[DCHT Toán 11 - KNTT - Nguyễn Hoài Nam]%[1K1B1-7]
	Cho tam giác $ABC$ vuông tại $A$. Khẳng định nào sau đây là \textbf{đúng}?
	\choice
	{$\sin \widehat{B} = \sin \widehat{C}$}
	{$\tan \widehat{B} = \sin \widehat{C}$}
	{\True $\cot \widehat{B} = \tan \widehat{C}$}
	{$\sin \widehat{B} = - \cos \widehat{C}$}
	\loigiai{
		Vì $\widehat{B}+\widehat{C}=90^{\circ}$ nên $\cot \widehat{B} =\tan \widehat{C}$.		
	}
\end{ex}
\begin{ex}%[DCHT Toán 11 - KNTT - Nguyễn Hoài Nam]%[1K1B1-7]
	Cho tam giác $ABC$. Khẳng định nào sau đây là \textbf{sai}?
	\choice
	{$\cos\dfrac{B+C}{2} = \sin\dfrac{A}{2}$}
	{\True $\tan\dfrac{B+C}{2} = \cos\dfrac{A}{2}$}
	{$\cot\dfrac{B+C}{2} = \tan\dfrac{A}{2}$}
	{$\sin\dfrac{B+C}{2} = \cos\dfrac{A}{2}$}
	\loigiai
	{
		Ta có $A+ B + C = \pi \Rightarrow \dfrac{B + C}{2} = \dfrac{\pi}{2} - \dfrac{A}{2} \Rightarrow \heva{&\cos\dfrac{B+C}{2} = \sin\dfrac{A}{2}  \\&\cot\dfrac{B+C}{2} = \tan\dfrac{A}{2}  \\&\sin\dfrac{B+C}{2} = \cos\dfrac{A}{2}.  }$\\
		Vậy khẳng định \lq\lq $\tan\dfrac{B+C}{2} = \cos\dfrac{A}{2}$ \rq\rq\; là sai.
	}
\end{ex}
\begin{ex}%[DCHT Toán 11 - KNTT - Nguyễn Hoài Nam]%[1K1B1-7]
	Cho tam giác $ABC$. Khẳng định nào sau đây là \textbf{sai}?
	\choice
	{\True $\sin (A+B)= - \sin C$}
	{$\tan (A+B)= - \tan C$}
	{$\cos (A+B)= - \cos C$}
	{$\cot (A+B)= - \cot C$}
	\loigiai{
		Ta có $\widehat{A}+\widehat{B}=180^{\circ}-\widehat{C} \Rightarrow 	\sin (A+B)= \sin \left (180^{\circ} -C \right )=\sin C$.
	}
\end{ex}

\begin{ex}%[DCHT Toán 11 - KNTT - Nguyễn Hoài Nam]%[1K1B1-7]
	Cho $\cos x=\dfrac{1}{3}$. Tính $\sin\left(\dfrac{\pi}{2}-x\right)+2\cos(-x)$.
	\choice
	{\True $1 $}
	{$3 $}
	{$0 $}
	{$-1 $}
	\loigiai{
		Ta có $\sin\left(\dfrac{\pi}{2}-x\right)+2\cos(-x)=\cos x+2\cos x=1$.
	}
\end{ex}

\begin{ex}%[DCHT Toán 11 - KNTT - Nguyễn Hoài Nam]%[1K1B1-7]
	Cho $\sin \alpha =\dfrac{4}{5}$, $0<\alpha<\dfrac{\pi}{2}$. Tính $\cos (\pi + \alpha)$.
	\choice
	{$\cos \alpha=\dfrac{1}{\sqrt{5}}$}
	{\True $\cos \alpha=-\dfrac{3}{5}$}
	{$\cos \alpha=\dfrac{3}{4}$}
	{$\cos \alpha=\dfrac{3}{5}$}
	\loigiai{
		Ta có $\cos^2 \alpha=1-\sin^2\alpha=1-\left(\dfrac{4}{5}\right)^2=\dfrac{9}{25}$.\\
		Vì $0<\alpha<\dfrac{\pi}{2}$ nên suy ra $\cos \alpha=\dfrac{3}{5}$. \\
		Mà $\cos (\pi + \alpha)= - \cos \alpha = -\dfrac{3}{5}$
	}
\end{ex}
%-------------------
\begin{ex}%[DCHT Toán 11 - KNTT - Nguyễn Hoài Nam]%[1K1K1-7]
	Cho $A=\dfrac{\sin515^\circ \cos(-475^\circ) + \cot222^\circ \cot408^\circ}{\cot415^\circ \cot(-505^\circ) + \tan197^\circ \tan73^\circ}$. Giá trị của $A$ bằng
	\choice
	{\True $\dfrac{1}{2}\cos^2 25^\circ$}
	{$-\dfrac{1}{2}\cos^2 25^\circ$}
	{$\dfrac{1}{2}\sin^2 25^\circ$}
	{$-\dfrac{1}{2}\sin^2 25^\circ$}
	\loigiai{
		Ta có 
		\allowdisplaybreaks
		\begin{eqnarray*}
			& & \sin515^\circ = \sin(-25^\circ + 180^\circ + 360^\circ) = -\sin(-25^\circ) = \sin 25^\circ, \\
			& & \cos(-475^\circ) = \cos(-25^\circ - 90^\circ - 360^\circ) = \sin(-25^\circ) = -\sin 25^\circ, \\
			& & \cot 408^\circ = \cot(720^\circ - 90^\circ - 222^\circ) = -\tan(-222^\circ) = \tan 222^\circ, \\
			& & \cot(-505^\circ) = \cot(-415^\circ - 90^\circ) = -\tan(-415^\circ) = \tan 415^\circ, \\
			& & \tan 197^\circ = \tan(-73^\circ + 90^\circ + 180^\circ) = -\tan(-73^\circ + 90^\circ) = \cot 73^\circ.
		\end{eqnarray*}
		Vậy $A = \dfrac{\sin25^\circ (-\sin25^\circ) + \cot222^\circ \tan222^\circ}{\cot 415^\circ \tan 415^\circ + \cot 73^\circ \tan 73^\circ} = \dfrac{-\sin^2 25^\circ + 1}{1+1} = \dfrac{1}{2}\cos^2 25^\circ$.
	}
\end{ex}


\begin{ex}%[DCHT Toán 11 - KNTT - Nguyễn Hoài Nam]%[1K1K1-7]
	Cho tam giác $ABC$ không có góc $45^\circ$. Mệnh đề nào sau đây là \textbf{sai}?
	\choice
	{$\sin A = \sin(B+C)$}
	{$\sin\dfrac{A+B}{2} = \cos \dfrac{C}{2}$}
	{\True $\cos(3A+B+C) = \cos2A$}
	{$\cos \dfrac{A}{2} = \sin \dfrac{B+C}{2}$}
	\loigiai{
		Ta có 
		$$\cos(3A+B+C) = \cos(2A + A+B+C) = \cos(2A + 180^\circ) = -\cos2A \ne \cos2A.$$
	}
\end{ex}

\begin{ex}%[DCHT Toán 11 - KNTT - Nguyễn Hoài Nam]%[1K1G1-7]
	Giá trị biểu thức $\cos^2 1^\circ + \cos^2 2^\circ + \cos^2 3^\circ + \cdots + \cos^2 87^\circ + \cos^2 88^\circ + \cos^2 89^\circ$ bằng 
	\choice
	{\True $\dfrac{89}{2}$}
	{$44$}
	{$45$}
	{$\dfrac{91}{2}$}
	\loigiai{
		Áp dụng quan hệ hai cung phụ nhau, ta có
		\allowdisplaybreaks
		\begin{eqnarray*}
			& & \cos^2 1^\circ + \cos^2 89^\circ = \cos^2 1^\circ + \sin^2 1^\circ = 1,\\
			& & \cos^2 2^\circ + \cos^2 88^\circ = \cos^2 2^\circ + \sin^2 2^\circ = 1,\\
			& & \cdots \\
			& & \cos^2 44^\circ + \cos^2 46^\circ = \cos^2 44^\circ + \sin^2 44^\circ = 1.\\
		\end{eqnarray*}
		Suy ra 
		$$\cos^2 1^\circ + \cos^2 2^\circ + \cos^2 3^\circ + \cdots + \cos^2 87^\circ + \cos^2 88^\circ + \cos^2 89^\circ = 44 + \cos^2 45^\circ = \dfrac{89}{2}.$$
	}
\end{ex}

\Closesolutionfile{ans}
% \begin{indapan}{10}
% 	{ans/ans-1K1-1-Dang7}
% \end{indapan}