\begin{dang}{Tìm tập xác định của hàm số lượng giác}
	Để tìm tập xác định của hàm số lượng giác ta cần nhớ
	\begin{itemize}
		\item $y=\tan f(x)=\dfrac{\sin f(x)}{\cos f(x)} \text{ xđ khi }\cos f(x)\neq 0 \Leftrightarrow f(x)\neq \dfrac{\pi}{2}+k\pi, k\in \mathbb{Z}$.
		\item $y=\cot f(x)=\dfrac{\cos f(x)}{\sin f(x)} \text{ xđ khi } \sin f(x)\neq 0 \Leftrightarrow f(x)\neq k\pi, k\in \mathbb{Z}$.		
		% \begin{note}
		% 	Khi tìm tập xác định, ta xem nó có mẫu không? có $\tan,\, \cot$ không? có căn không?
		% \end{note}
		\item Với $k\in \mathbb{Z}$, ta cần nhớ những trường hợp đặc biệt:
		\begin{enumEX}{2}
			\item  $\left| \begin{aligned} &+\, \sin x=1\Leftrightarrow x=\dfrac{\pi}{2}+k2\pi.\\&+\, \sin x=-1\Leftrightarrow x=-\dfrac{\pi}{2}+k2\pi.\\& +\, \sin x=0\Leftrightarrow x=k\pi.
			\end{aligned}\right.$
			\item  $\left| \begin{aligned} &+\, \cos x=1\Leftrightarrow x=k2\pi.\\&+\, \cos x=-1\Leftrightarrow x=\pi+k2\pi.\\& +\, \cos x=0\Leftrightarrow x=\dfrac{\pi}{2}+k\pi.
			\end{aligned}\right.$
			\item  $\left| \begin{aligned} &+\, \tan x=1\Leftrightarrow x=\dfrac{\pi}{4}+k\pi.\\&+\, \tan x=-1\Leftrightarrow x=-\dfrac{\pi}{4}+k\pi.\\& +\, \tan x=0\Leftrightarrow x=k\pi.
			\end{aligned}\right.$
			\item  $\left| \begin{aligned} &+\, \cot x=1\Leftrightarrow x=\dfrac{\pi}{4}+k\pi.\\&+\, \cot x=-1\Leftrightarrow x=-\dfrac{\pi}{4}+k\pi.\\& +\, \cot x=0\Leftrightarrow x=\dfrac{\pi}{2}+k\pi.
			\end{aligned}\right.$
		\end{enumEX}
	\end{itemize}
\end{dang}
\Opensolutionfile{ans}[ans/ans-1K1-3-dang1]
\begin{bt}%[1D1B1-1]%[Vĩ Lê Văn  - DA2]
Tìm tập xác định $\mathscr{D}$ của hàm số
$y=\dfrac{\tan 2x}{\cos x+1}+\sin x$.
\dapso{Tập xác định $\mathscr{D}=\mathbb{R}\setminus\left\lbrace\pi +k2\pi;\, \dfrac{\pi}{4}+k\dfrac{\pi}{2},\, k\in \mathbb{Z} \right\rbrace$.}
\loigiai{
Điều kiện: $\heva{&\cos x+1\neq 0\\& \cos 2x\neq 0}\Leftrightarrow \heva{&\cos x\neq -1\\& \cos 2x\neq 0}\Leftrightarrow \heva{& x\neq \pi +k2\pi\\&2x\neq \dfrac{\pi}{2}+k\pi}\Leftrightarrow \heva{& x\neq \pi +k2\pi\\&x\neq \dfrac{\pi}{4}+k\dfrac{\pi}{2}.}$\\
Tập xác định $\mathscr{D}=\mathbb{R}\setminus\left\lbrace\pi +k2\pi;\, \dfrac{\pi}{4}+k\dfrac{\pi}{2},\, k\in \mathbb{Z} \right\rbrace$.\\
}
\end{bt}
\begin{bt}%[1D1B1-1]%[Vĩ Lê Văn  - DA2]
	Tìm tập xác định $\mathscr{D}$ của hàm số
	$y=\dfrac{\cos 3x}{1-\sin x}+\tan x$.
	\dapso{Tập xác định $\mathscr{D}=\mathbb{R}\setminus\left\lbrace  \dfrac{\pi}{2}+k\pi,\, k\in \mathbb{Z} \right\rbrace$.}
	\loigiai{
		Điều kiện: $\heva{&1-\sin x\neq 0\\& \cos x\neq 0}\Leftrightarrow \heva{&\sin x\neq 1\\& \cos x\neq 0}\Leftrightarrow \heva{& x\neq \dfrac{\pi}{2} +k2\pi\\&x\neq \dfrac{\pi}{2}+k\pi}\Leftrightarrow x\neq \dfrac{\pi}{2}+k\pi$.\\
		Tập xác định $\mathscr{D}=\mathbb{R}\setminus\left\lbrace  \dfrac{\pi}{2}+k\pi,\, k\in \mathbb{Z} \right\rbrace$.\\
	}
\end{bt}
\begin{bt}%[1D1B1-1]%[Vĩ Lê Văn  - DA2]
	Tìm tập xác định $\mathscr{D}$ của hàm số
	$y=\dfrac{2\tan 2x-5}{\sin 2x+1}$.
	\dapso{Tập xác định $\mathscr{D}=\mathbb{R}\setminus\left\lbrace \dfrac{\pi}{4}+k\dfrac{\pi}{2}, \, k\in \mathbb{Z} \right\rbrace$.}
	\loigiai{
		Điều kiện: \\
		$\heva{&\sin 2x+1\neq 0\\& \cos 2x\neq 0}\Leftrightarrow \heva{&\sin 2x\neq -1\\& \cos 2x\neq 0}\Leftrightarrow \heva{& 2x\neq -\dfrac{\pi}{2} +k2\pi\\&2x\neq \dfrac{\pi}{2}+k\pi}\Leftrightarrow \heva{& x\neq -\dfrac{\pi}{4} +k\pi\\&x\neq \dfrac{\pi}{4}+k\dfrac{\pi}{2}}\Leftrightarrow x\neq \dfrac{\pi}{4}+k\dfrac{\pi}{2}$.\\
		Tập xác định $\mathscr{D}=\mathbb{R}\setminus\left\lbrace \dfrac{\pi}{4}+k\dfrac{\pi}{2}, \, k\in \mathbb{Z} \right\rbrace$.\\
	}
\end{bt}
\begin{bt}%[1D1K1-1]%[Vĩ Lê Văn  - DA2]
	Tìm tập xác định $\mathscr{D}$ của hàm số
	$y=\dfrac{1}{\tan x-1}$.
	\loigiai{
		Điều kiện: $\heva{&\tan x-1\neq 0\\& \cos x\neq 0}\Leftrightarrow \heva{&\tan x\neq 1\\& \cos x\neq 0}\Leftrightarrow \heva{& x\neq \dfrac{\pi}{4} +k\pi\\&x\neq \dfrac{\pi}{2}+k\pi.}$\\
		Tập xác định $\mathscr{D}=\mathbb{R}\setminus\left\lbrace\dfrac{\pi}{4} +k\pi;\, \dfrac{\pi}{2}+k\pi,\, k\in \mathbb{Z} \right\rbrace$.\\
	}
\end{bt}
\begin{bt}%[1D1B1-1]%[Vĩ Lê Văn  - DA2]
	Tìm tập xác định $\mathscr{D}$ của hàm số
	$y=\dfrac{3}{\cos^2 x-\sin^2 x}+\tan x$.
	\dapso{Tập xác định $\mathscr{D}=\mathbb{R}\setminus\left\lbrace\dfrac{\pi}{2} +k\pi;\, \dfrac{\pi}{4}+k\dfrac{\pi}{2},\, k\in \mathbb{Z} \right\rbrace$.}
	\loigiai{
		Ta có $y=\dfrac{3}{\cos^2 x-\sin^2 x}+\tan x=\dfrac{3}{\cos 2x}+\tan x$.\\
		Điều kiện: $\heva{&\cos x\neq 0\\& \cos 2x\neq 0}\Leftrightarrow \heva{& x\neq \dfrac{\pi}{2} +k\pi\\&2x\neq \dfrac{\pi}{2}+k\pi}\Leftrightarrow \heva{& x\neq \dfrac{\pi}{2} +k\pi\\&x\neq \dfrac{\pi}{4}+k\dfrac{\pi}{2}.}$\\
		Tập xác định $\mathscr{D}=\mathbb{R}\setminus\left\lbrace\dfrac{\pi}{2} +k\pi;\, \dfrac{\pi}{4}+k\dfrac{\pi}{2},\, k\in \mathbb{Z} \right\rbrace$.\\
	}
\end{bt}
\begin{bt}%[1D1K1-1]%[Vĩ Lê Văn  - DA2]
	Tìm tập xác định $\mathscr{D}$ của hàm số
	$y=\dfrac{1}{\sin x}+\dfrac{1}{\cos x}$.
	\dapso{Tập xác định $\mathscr{D}=\mathbb{R}\setminus\left\lbrace k\dfrac{\pi}{2},\, k\in \mathbb{Z} \right\rbrace$.}
	\loigiai{
		Điều kiện: $\heva{&\cos x\neq 0\\& \sin x\neq 0}\Leftrightarrow \heva{& x\neq k\pi\\&x\neq \dfrac{\pi}{2}+k\pi}\Leftrightarrow x\neq k\dfrac{\pi}{2}.$\\
		Tập xác định $\mathscr{D}=\mathbb{R}\setminus\left\lbrace k\dfrac{\pi}{2},\, k\in \mathbb{Z} \right\rbrace$.\\
	}
\end{bt}
\begin{bt}%[1D1K1-1]%[Vĩ Lê Văn  - DA2]
	Tìm tập xác định $\mathscr{D}$ của hàm số
	$y=\sqrt{\dfrac{2-\sin x}{\cos x+1}}$.
	\dapso{ $\mathscr{D}=\mathbb{R}\setminus\left\lbrace \pi +k2\pi, \, k\in \mathbb{Z} \right\rbrace$.}
	\loigiai{
		Điều kiện: $\dfrac{2-\sin x}{\cos x+1}\geq 0$.\\
		Với mọi $x\in \mathbb{R}$, ta có
		$$-1\leq \sin x\leq 1\Leftrightarrow  1\leq 2-\sin x\leq 3;$$
		$$-1\leq \cos x\leq 1\Leftrightarrow 0\leq \cos x+1\leq 2.$$
		Do đó
		$\dfrac{2-\sin x}{\cos x+1}\geq 0\Leftrightarrow \cos x+1\neq 0 \Leftrightarrow \cos x\neq -1\Leftrightarrow
		 x\neq \pi +k2\pi$.\\
		Tập xác định $\mathscr{D}=\mathbb{R}\setminus\left\lbrace \pi +k2\pi, \, k\in \mathbb{Z} \right\rbrace$.\\
	}
\end{bt}
\begin{bt}%[1D1K1-1]%[Vĩ Lê Văn  - DA2]
	Tìm tập xác định $\mathscr{D}$ của hàm số
	$y=\dfrac{1}{\sqrt{1-\sin x}}$.
	\dapso{$\mathscr{D}=\mathbb{R}\setminus\left\lbrace \dfrac{\pi}{2}+k2\pi, \, k\in \mathbb{Z} \right\rbrace$.}
	\loigiai{
		Điều kiện: $1-\sin x> 0\Leftrightarrow \sin x<1\Leftrightarrow \sin x \neq 1\Leftrightarrow x \neq \dfrac{\pi}{2}+k2\pi$.\\
		Tập xác định $\mathscr{D}=\mathbb{R}\setminus\left\lbrace \dfrac{\pi}{2}+k2\pi, \, k\in \mathbb{Z} \right\rbrace$.\\
	}
\end{bt}
\begin{bt}%[1D1K1-1]%[Vĩ Lê Văn  - DA2]
	Tìm tập xác định $\mathscr{D}$ của hàm số
	$y=\sqrt{\dfrac{\cos x+4}{\sin x+1}}$.
	\dapso{ $\mathscr{D}=\mathbb{R}\setminus\left\lbrace -\dfrac{\pi}{2} +k2\pi, \, k\in \mathbb{Z} \right\rbrace$.}
	\loigiai{
		Điều kiện: $\dfrac{\cos x+4}{\sin x+1}\geq 0$.\\
		Với mọi $x\in \mathbb{R}$, ta có
		$$-1\leq \sin x\leq 1\Leftrightarrow  0\leq \sin x+1\leq 2;$$
		$$-1\leq \cos x\leq 1\Leftrightarrow 3\leq \cos x+4\leq 5.$$
		Do đó
		$\dfrac{\cos x+4}{\sin x+1}\geq 0\Leftrightarrow \sin x+1\neq 0 \Leftrightarrow \sin x\neq -1\Leftrightarrow
		x\neq -\dfrac{\pi}{2} +k2\pi$.\\
		Tập xác định $\mathscr{D}=\mathbb{R}\setminus\left\lbrace -\dfrac{\pi}{2} +k2\pi, \, k\in \mathbb{Z} \right\rbrace$.\\
	}
\end{bt}
\begin{bt}%[1D1K1-1]%[Vĩ Lê Văn  - DA2]
	Tìm tập xác định $\mathscr{D}$ của hàm số
	$y=\sqrt{\dfrac{2-\cos x}{1-\sin x}}$.
	\dapso{$\mathscr{D}=\mathbb{R}\setminus\left\lbrace \dfrac{\pi}{2} +k2\pi, \, k\in \mathbb{Z} \right\rbrace$.}
	\loigiai{
		Điều kiện: $\dfrac{2-\cos x}{1-\sin x}\geq 0$.\\
		Với mọi $x\in \mathbb{R}$, ta có
		$$-1\leq \sin x\leq 1\Leftrightarrow  0\leq 1-\sin x\leq 2;$$
		$$-1\leq \cos x\leq 1\Leftrightarrow 1\leq 2-\cos x\leq 3.$$
		Do đó
		$\dfrac{2-\cos x}{1-\sin x}\geq 0\Leftrightarrow 1-\sin x\neq 0 \Leftrightarrow \sin x\neq 1\Leftrightarrow
		x\neq \dfrac{\pi}{2} +k2\pi$.\\
		Tập xác định $\mathscr{D}=\mathbb{R}\setminus\left\lbrace \dfrac{\pi}{2} +k2\pi, \, k\in \mathbb{Z} \right\rbrace$.\\
	}
\end{bt}
\begin{bt}%[1D1K1-1]%[Vĩ Lê Văn  - DA2]
	Tìm tập xác định $\mathscr{D}$ của hàm số
	$y=\sqrt{4\pi^2-x^2}+\cot 2x$.
	\dapso{$\mathscr{D}=(-2\pi;2\pi)\setminus\left\lbrace \pm \dfrac{3\pi}{2};\pm \pi; \pm\dfrac{\pi}{2};0 \right\rbrace$.}
	\loigiai{
		Điều kiện: $\heva{& 4\pi^2-x^2\geq 0\\&\sin 2x\neq 0}\Leftrightarrow \heva{&-2\pi\leq x\leq 2\pi\\&2x\neq k\pi}\Leftrightarrow \heva{&-2\pi\leq x\leq 2\pi\\&x\neq k\dfrac{\pi}{2}.}$\\
		Ta có $-2\pi\leq k\dfrac{\pi}{2}\leq 2\pi\Leftrightarrow -4\leq k\leq 4$.\\	
		Tập xác định $\mathscr{D}=(-2\pi;2\pi)\setminus\left\lbrace \pm \dfrac{3\pi}{2};\pm \pi; \pm\dfrac{\pi}{2};0 \right\rbrace$.\\
	}
\end{bt}
\begin{bt}%[1D1K1-1]%[Vĩ Lê Văn  - DA2]
	Tìm tập xác định $\mathscr{D}$ của hàm số
	$y=\sqrt{\pi^2-x^2}+\cot 2x$.
	\dapso{$\mathscr{D}=(-\pi;\pi)\setminus\left\lbrace  \pm\dfrac{\pi}{2};0 \right\rbrace$.}
	\loigiai{
		Điều kiện: $\heva{& \pi^2-x^2\geq 0\\&\sin 2x\neq 0}\Leftrightarrow \heva{&-\pi\leq x\leq \pi\\&2x\neq k\pi}\Leftrightarrow \heva{&-\pi\leq x\leq \pi\\&x\neq k\dfrac{\pi}{2}.}$\\
		Ta có $-\pi\leq k\dfrac{\pi}{2}\leq \pi\Leftrightarrow -2\leq k\leq 2$.\\	
		Tập xác định $\mathscr{D}=(-\pi;\pi)\setminus\left\lbrace  \pm\dfrac{\pi}{2};0 \right\rbrace$.\\
	}
\end{bt}
\begin{bt}%[1D1K1-1]%[Vĩ Lê Văn  - DA2]
	Tìm tập xác định $\mathscr{D}$ của hàm số
	$y=\dfrac{\sqrt{\pi^2-x^2}}{\sin 2x+1}$.
	\dapso{$\mathscr{D}=[-\pi;\pi]\setminus\left\lbrace  \dfrac{\pi}{4};\dfrac{3\pi}{4} \right\rbrace$.}
	\loigiai{
		Điều kiện: $\heva{& \pi^2-x^2\geq 0\\&\sin 2x+1\neq 0}\Leftrightarrow \heva{& \pi^2-x^2\geq 0\\&\sin 2x\neq -1}\Leftrightarrow \heva{&-\pi\leq x\leq \pi\\&2x\neq -\dfrac{\pi}{2}+ k2\pi}\Leftrightarrow \heva{&-\pi\leq x\leq \pi\\&x\neq -\dfrac{\pi}{4}+k\pi.}$\\
		Ta có $-\pi\leq -\dfrac{\pi}{4}+k\pi\leq \pi\Leftrightarrow -\dfrac{3}{4}\leq k\leq \dfrac{5}{4}$.\\	
		Tập xác định $\mathscr{D}=[-\pi;\pi]\setminus\left\lbrace  \dfrac{\pi}{4};\dfrac{3\pi}{4} \right\rbrace$.\\
	}
\end{bt}
\begin{bt}%[1D1K1-1]%[Vĩ Lê Văn  - DA2]
	Tìm tập xác định $\mathscr{D}$ của hàm số
	$y=\dfrac{\sqrt{4\pi^2-x^2}}{\cos 2x+1}$.
	\dapso{$\mathscr{D}=[-2\pi;2\pi]\setminus\left\lbrace  \pm\dfrac{3\pi}{2};-\dfrac{\pi}{2};\dfrac{\pi}{2} \right\rbrace$.}
	\loigiai{
		Điều kiện: $\heva{& 4\pi^2-x^2\geq 0\\&\cos 2x+1\neq 0}\Leftrightarrow \heva{& 4\pi^2-x^2\geq 0\\&\cos 2x\neq -1}\Leftrightarrow \heva{&-2\pi\leq x\leq 2\pi\\&2x\neq \pi+ k2\pi}\Leftrightarrow \heva{&-2\pi\leq x\leq 2\pi\\&x\neq \dfrac{\pi}{2}+k\pi.}$\\
		Ta có $-2\pi\leq \dfrac{\pi}{2}+k\pi\leq 2\pi\Leftrightarrow -\dfrac{5}{2}\leq k\leq \dfrac{3}{2}$.\\	
		Tập xác định $\mathscr{D}=[-2\pi;2\pi]\setminus\left\lbrace  \pm\dfrac{3\pi}{2};-\dfrac{\pi}{2};\dfrac{\pi}{2} \right\rbrace$.\\
	}
\end{bt}


%%%%%%%%%%% Thầy Ân Trương %%%%%%%%%%%%%%%%%%
% \subsubsection{Câu hỏi trắc nghiệm}
\begin{ex}%[1D1B1-1]
	%Cau1
	Hàm số $ y=\dfrac{2 \sin x+1}{1-\cos x} $ xác định khi
	\choice
	{$ x \neq k \pi $}
	{\True$ x \neq k 2 \pi $}
	{$ x \neq \dfrac{\pi}{2}+k 2 \pi $}
	{$ x \neq \dfrac{\pi}{2}+k \pi $}
	\loigiai
	{
		Điều kiện: $ 1- \cos x \ne 0 \Leftrightarrow \cos x \ne 1 \Leftrightarrow x \ne k2\pi, (k\in \mathbb{Z})$.
	}
\end{ex}
%Cau2
\begin{ex}%[1D1B1-1]
	Hàm số $ y=\dfrac{1-3 \cos x}{\sin x} $ xác định khi
	\choice
	{\True$ x \neq k \pi $}
	{$ x \neq k 2 \pi $}
	{$ x \neq \dfrac{k \pi}{2}  $}
	{$ x \neq \dfrac{\pi}{2}+k \pi $}
	\loigiai
	{
		Điều kiện: $ \sin x \ne 0 \Leftrightarrow x\ne k\pi, (k\in \mathbb{Z})$.
	}
\end{ex}
%Cau3
\begin{ex}%[1D1B1-1]
	Tập xác định của hàm số $ y=\dfrac{1-\cos x}{\sin x-1} $ là 
	\choice
	{$ \mathbb{R} \backslash\left\{\dfrac{\pi}{2}+k \pi\right\}  $}
	{\True$ \mathbb{R} \backslash\left\{\dfrac{\pi}{2}+k 2 \pi\right\}  $}
	{$ \mathbb{R} \backslash\{k \pi\} $}
	{$ \mathbb{R} \backslash\{k 2 \pi\}  $}
	\loigiai
	{
		Điều kiện: $ \sin x -1 \ne 0 \Leftrightarrow \sin x \ne 1 \Leftrightarrow x\ne  \dfrac{\pi}{2}+k 2 \pi, (k\in \mathbb{Z}) $.
	}
\end{ex}
%Cau4
\begin{ex}%[1D1B1-1]
	Tập xác định của hàm số $ y=\dfrac{\cot x}{\cos x-1} $ là
	\choice
	{$ \mathscr{D}=\mathbb{R} \backslash\left\{\dfrac{k \pi}{2}\right\} $}
	{$ \mathscr{D}=\mathbb{R} \backslash\left\{\dfrac{\pi}{2}+k \pi\right\}  $}
	{\True$ \mathscr{D}=\mathbb{R} \backslash\{k \pi\} $}
	{$\mathscr{D}=\mathbb{R} \backslash\{k 2 \pi\}  $}
	\loigiai
	{
		Điều kiện: $ \heva{& \sin x \ne 0 \\ & \cos x \ne 1} \Leftrightarrow \heva{& x\ne k\pi \\ & x\ne k2\pi}\Leftrightarrow x \ne k\pi ,(k \in \mathbb{Z}).$\\
		Vậy, tập xác định là $ \mathscr{D}=\mathbb{R} \backslash\{k \pi\} $.
	}
\end{ex}
%Cau5
\begin{ex}%[1D1B1-1]
	Hàm số $ y=\dfrac{1}{\sin x-\cos x} $ xác định khi
	\choice
	{$ x \neq k 2 \pi  $}
	{$ x \neq \dfrac{\pi}{2}+k \pi  $}
	{$ x \neq k \pi  $}
	{\True$ x \neq \dfrac{\pi}{4}+k \pi  $}
	\loigiai
	{
		Điều kiện: $ \sin x - \cos x \ne 0 \Rightarrow \tan x \ne 1 \Leftrightarrow x \neq \dfrac{\pi}{4}+k \pi, k \in \mathbb{Z} $.
	}
\end{ex}
%Cau6
\begin{ex}%[1D1B1-1]
	Tập xác định của hàm số $ y=\dfrac{\tan 2 x}{\cos x} $ là 
	\choice
	{$\mathbb{R} $}
	{$ \mathbb{R} \backslash\left\{\dfrac{\pi}{2}+k \pi\right\} $}
	{$\mathbb{R} \backslash\left\{\dfrac{\pi}{4}+\dfrac{k \pi}{2}\right\}  $}
	{\True$ \mathbb{R} \backslash\left\{\dfrac{\pi}{4}+\dfrac{k \pi}{2} ; \dfrac{\pi}{2}+k \pi\right\}  $}
	\loigiai
	{
		Điều kiện: $ \heva{& \cos 2x \ne 0 \\ & \cos x \ne 0} \Leftrightarrow \heva{& 2x \ne \dfrac{\pi}{2} + k\pi \\ &  x \ne \dfrac{\pi}{2} + k\pi} \Leftrightarrow \heva{& x \ne \dfrac{\pi}{4} + \dfrac{k\pi}{2} \\ & x \ne \dfrac{\pi}{2} + k\pi},(k \in \mathbb{Z}).$\\
		Vậy, tập xác định là $ \mathbb{R} \backslash\left\{\dfrac{\pi}{4}+\dfrac{k \pi}{2} ; \dfrac{\pi}{2}+k \pi\right\}  $.
	}
\end{ex}
%Cau7
\begin{ex}%[1D1B1-1]
	Tập xác định của hàm số $ y=\dfrac{\tan x-5}{1-\sin ^{2} x} $ là
	\choice
	{\True$ \mathbb{R} \backslash\left\{\dfrac{\pi}{2}+k \pi\right\}  $}
	{$ \mathbb{R}  $}
	{$ \mathbb{R} \backslash\left\{\dfrac{\pi}{2}+k 2 \pi\right\}  $}
	{$ \mathbb{R} \backslash\{\pi+k \pi\} $}
	\loigiai
	{
		Ta có $ y=\dfrac{\tan x-5}{1-\sin ^{2} x}=\dfrac{\tan x-5}{\cos ^{2} x} $.\\
		Điều kiện: $ \cos x \ne 0 \Leftrightarrow x\ne \dfrac{\pi}{2}+k \pi $.\\
		Vậy, tập xác định là $ \mathbb{R} \backslash\left\{\dfrac{\pi}{2}+k \pi\right\}  ,(k \in \mathbb{Z}).$
	}
\end{ex}
%Cau8
\begin{ex}%[1D1B1-1]
	Hàm số $ y=\sqrt{\dfrac{1-\sin x}{1+\sin x}} $ xác định khi
	\choice
	{$ x \neq \pm \dfrac{\pi}{2}+k 2 \pi  $}
	{$ x \neq-k \pi  $}
	{$ x \neq \dfrac{\pi}{2}+k 2 \pi  $}
	{\True$ x \neq-\dfrac{\pi}{2}+k 2 \pi  $}
	\loigiai
	{
		Điều kiện: $ \dfrac{1-\sin x}{1+\sin x} \geq 0.$\\
		Ta có $ \heva{& -1 \leq -\sin x \leq 1 \\ & -1 \leq \sin x \leq 1} \Leftrightarrow \heva{& 0 \leq 1-\sin x \leq 2 \\ & 0 \leq 1+\sin x \leq 2 },(k \in \mathbb{Z}).$\\
		Để $\dfrac{1-\sin x}{1+\sin x} \geq0  \Rightarrow  1+\sin x \ne 0 \Leftrightarrow  x\neq-\dfrac{\pi}{2}+k 2 \pi $.
	}
\end{ex}
%Cau9
\begin{ex}%[1D1B1-1]
	Tập xác định hàm số $ y=\sqrt{\dfrac{\sin 2 x+2}{1-\cos x}} $ là 
	\choice
	{$ \mathscr{D}=\mathbb{R}  $}
	{\True$ \mathscr{D}=\mathbb{R} \backslash\{k 2 \pi\} $}
	{$ \mathscr{D}=\{k 2 \pi\} $}
	{$ \mathscr{D}=\mathbb{R} \backslash\{k \pi\} $}
	\loigiai
	{
		Điều kiện: $ y=\sqrt{\dfrac{\sin 2 x+2}{1-\cos x}} \geq 0 $.\\
		Ta có $ \heva{& -1 \leq \sin 2 x \leq 1 \\ & -1 \leq -\cos x \leq 1} \Leftrightarrow \heva{& 1 \leq \sin 2 x +2\leq 3 \\ & 0 \leq 1-\cos x \leq 2},(k \in \mathbb{Z}).$\\
		Để $\dfrac{\sin 2 x+2}{1-\cos x} \geq0 \Rightarrow 1-\cos x \ne 0 \Leftrightarrow  x \ne k 2 \pi $.\\
		Vậy, tập xác định là $ \mathscr{D}=\mathbb{R} \backslash\{k 2 \pi\} $.
	}
\end{ex}
%Cau10
\begin{ex}%[1D1B1-1]
	Tập xác định $ \mathscr{D} $ của hàm số $ y=\dfrac{\tan 2 x}{\sqrt{\sin x+1}} $ là
	\choice
	{\True$ \mathscr{D}=\mathbb{R} \backslash\left\{-\dfrac{\pi}{2}+k 2 \pi ; \dfrac{\pi}{4}+\dfrac{k \pi}{2}\right\}  $}
	{$ \mathscr{D}=\mathbb{R} \backslash\left\{\dfrac{\pi}{4}+\dfrac{k \pi}{2}\right\}  $}
	{$ \mathscr{D}=\mathbb{R} \backslash\{k 2 \pi\}  $}
	{$ \mathscr{D}=\mathbb{R} \backslash\left\{\dfrac{\pi}{2}+k \pi ; \dfrac{\pi}{4}+\dfrac{k \pi}{2}\right\}  $}
	\loigiai
	{
		Điều kiện: $ \heva{& \cos 2x \ne 0 \\ & \sin x +1 \geq 0} (\ast)$\\
		Ta có $ -1 \leq \sin x \leq 1 \Leftrightarrow 0 \leq \sin x +1 \leq 2.$
		\begin{eqnarray*}
			(\ast) &\Leftrightarrow& \heva{& \cos 2x \ne 0  \\ & \sin x +1 \ne 0}\\
			&\Leftrightarrow& \heva{& x \ne \dfrac{\pi}{4}+\dfrac{k \pi}{2}\\& x \ne -\dfrac{\pi}{2}+k 2 \pi },(k \in \mathbb{Z}).
		\end{eqnarray*}
		Vậy, tập xác định là $ \mathscr{D}=\mathbb{R} \backslash\left\{-\dfrac{\pi}{2}+k 2 \pi ; \dfrac{\pi}{4}+\dfrac{k \pi}{2}\right\}.  $
	}
\end{ex}
%Cau11
\begin{ex}%[1D1B1-1]
	Tập xác định của hàm số $ y=\cot \left(x+\dfrac{\pi}{6}\right)+\sqrt{\dfrac{1+\cos x}{1-\cos x}} $ là
	\choice
	{$ \mathscr{D}=\mathbb{R} \backslash\left\{-\dfrac{\pi}{3}+k \pi ; \dfrac{3 \pi}{4}+k \pi, k \in \mathbb{Z}\right\}  $}
	{$ \mathscr{D}=\mathbb{R} \backslash\{\pi+k \pi, k \in \mathbb{Z}\}  $}
	{$ \mathscr{D}=\mathbb{R} \backslash\{k 2 \pi, k \in \mathbb{Z}\}  $}
	{\True$ \mathscr{D}=\mathbb{R} \backslash\left\{-\dfrac{\pi}{6}+k \pi ; k 2 \pi, k \in \mathbb{Z}\right\}  $}
	\loigiai
	{
		Điều kiện: $ \heva{& \sin\left(x+\dfrac{\pi}{6}\right)\ne 0 \\ & \dfrac{1+\cos x}{1-\cos x}\geq 0 }  (\ast)$\\
		Ta có $ \heva{& -1 \leq \cos x \leq 1 \\ & -1 \leq -\cos x \leq 1} \Leftrightarrow \heva{& 0 \leq 1+\cos x \leq 2 \\ & 0 \leq 1-\cos x \leq 2.}$
		\begin{eqnarray*}
			(\ast) &\Leftrightarrow& \heva{& \sin\left(x+\dfrac{\pi}{6}\right)\ne 0 \\ & 1-\cos x \ne 0}\\
			&\Leftrightarrow& \heva{& x \ne -\dfrac{\pi}{6}+k \pi \\ & x \ne k 2 \pi},(k \in \mathbb{Z}).
		\end{eqnarray*}
		Vậy, tập xác định là $ \mathscr{D}=\mathbb{R} \backslash\left\{-\dfrac{\pi}{6}+k \pi ; k 2 \pi, k \in \mathbb{Z}\right\}  $.
	}
\end{ex}
\begin{ex}%Câu 28.[1D1B1-1]
	Tập xác định của hàm số $y=\sqrt{\dfrac{1-\sin x}{1+\cos x}}$ là
	\choice
	{\True $\mathbb{R}\setminus\left\{\pi+k2\pi, k\in\mathbb{Z}\right\}$}
	{$\mathbb{R}\setminus\left\{k2\pi, k\in\mathbb{Z}\right\}$}
	{$\mathbb{R}\setminus\left\{\dfrac{\pi}{4}+k2\pi, k\in\mathbb{Z}\right\}$}
	{$\mathbb{R}\setminus\left\{\dfrac{\pi}{2}+k2\pi, k\in\mathbb{Z}\right\}$}
	\loigiai{
		Ta có  $1-\sin x\geq 0; 1+\cos x\geq 0, \,\forall x\in\mathbb{R}$.\\
		Nên hàm số xác định khi và chỉ khi $\cos x\ne -1 \Leftrightarrow x\ne \pi+2k\pi, k\in\mathbb{Z}. $\\
		Vậy tập xác định của hàm số là $\mathscr{D}=\mathbb{R}\setminus\left\{\pi+k2\pi,k\in\mathbb{Z}\right\}$}
\end{ex}
\begin{ex}%Câu 29.[1D1Y1-1]
	Tập xác định của hàm số $y=\sqrt{\sin x+2}$. là
	\choice
	{\True $\mathbb{R}$}
	{$[-2;+\infty)$}
	{$(0;2\pi)$}
	{$\left[\arcsin(-2);+\infty\right)$}
	\loigiai{
		Ta có  $\sin x+2>0, \,\forall x\in\mathbb{R}$.\\
		Vậy tập xác định của hàm số là $\mathscr{D}=\mathbb{R}$.}
\end{ex}
\begin{ex}%Câu 30.[1D1Y1-1]
	Tập xác định của hàm số $y=\sqrt{1-\cos 2x}$ là
	\choice
	{\True $\mathscr{D}=\mathbb{R}$}
	{$\mathscr{D}=[0;1]$}
	{$\mathscr{D}=[-1;1]$}
	{$\mathscr{D}=\mathbb{R}\setminus\left\{k\pi, k\in\mathbb{Z}\right\}$}
	\loigiai{
		Ta có  $-1\leq\cos 2x\leq 1\Rightarrow 1-\cos 2x\geq 0, \,\forall x\in\mathbb{R}$.\\
		Vậy tập xác định của hàm số là $\mathscr{D}=\mathbb{R}$.}
\end{ex}
\begin{ex}%Câu 31.[1D1Y1-1]
	Hàm số nào sau đây có tập xác định $\mathbb{R}$?
	\choice
	{\True $y=\sqrt{\dfrac{2+\cos x}{2-\sin x}}$}
	{$y=\tan^2x+\cot^2x$}
	{$y=\dfrac{1+\sin^2x}{1+\cot^2x}$}
	{$y=\dfrac{\sin^3x}{2\cos x+\sqrt{2}}$}
	\loigiai{
		Vì	$-1\leq\sin x;\cos x\leq 1\Rightarrow 2+\cos x>0;2-\sin x>0$ \\
		Nên hàm số	$\dfrac{2+\cos x}{2-\sin x}>0$ xác định $\forall x\in\mathbb{R} $.}
\end{ex}
\Closesolutionfile{ans}
%\begin{indapan}{10}
%	{ans/ans-1K1-3-Dang1}
%\end{indapan}