\begin{dang}{Dấu của các giá trị lượng giác}
	
\end{dang}
\subsubsection{Ví dụ minh hoạ}
\begin{vd}%[1K1B1-5]
	Xét dấu các giá trị lượng giác của góc lượng giác $\alpha=-\dfrac{3 \pi}{4}$.
	\loigiai{
		Do $-\pi<-\dfrac{3 \pi}{4}<-\dfrac{\pi}{2}$ nên $\sin \left(-\dfrac{3 \pi}{4}\right)<0; \cos \left(-\dfrac{3 \pi}{4}\right)<0; \tan \left(-\dfrac{3 \pi}{4}\right)>0; \cot \left(-\dfrac{3 \pi}{4}\right)>0$.
	}
\end{vd}
\begin{vd}%[1K1K1-5]
	Xác định dấu các biểu thức:
	\begin{enumEX}{2}
		\item $A=\sin 50^{\circ} \cdot \cos \left(-100^{\circ}\right)$;
		\item $B=\sin 195^{\circ} \cdot \tan \dfrac{20 \pi}{7}$.
	\end{enumEX}
	\loigiai{
		\begin{enumerate}
			\item $A=\sin 50^{\circ} \cdot \cos \left(-100^{\circ}\right)$.\\
			Ta có điểm cuối của cung $50^{\circ}$ thuộc góc phần tư thứ I nên $\sin 50^{\circ}>0$. Điểm cuối của cung $-100^{\circ}$ thuộc góc phần tư thứ III nên $\cos \left(-100^{\circ}\right)<0$.\\
			Do đó, $A<0$.
			\item $B=\sin 195^{\circ} \cdot \tan \dfrac{20 \pi}{7}$.\\
			Ta có: điểm cuối của cung $195^{\circ}$ thuộc góc phần tư thứ III nên sin $195^{\circ}<0$. Điểm cuối của cung $\dfrac{20 \pi}{7}=\dfrac{6 \pi}{7}+2 \pi$ thuộc góc phần tư thứ II nên $\tan\dfrac{20 \pi}{7}<0$.\\ Do đó, $B>0$.
		\end{enumerate}
		
	}
\end{vd}
\begin{vd}%[1K1K1-5]
	Cho $\pi<\alpha<\dfrac{3 \pi}{2}$. Xét dấu các biểu thức sau:
	\begin{enumEX}{2}
		\item $A=\cos \left(\alpha-\dfrac{\pi}{2}\right)$;
		\item $B=\tan \left(\dfrac{2019 \pi}{2}-\alpha\right)$.
	\end{enumEX}
	\loigiai{
		\begin{enumerate}
			\item $A=\cos \left(\alpha-\dfrac{\pi}{2}\right)=\cos \left(\dfrac{\pi}{2}-\alpha\right)=\sin \alpha<0$.
			\item $B=\tan \left(\dfrac{2019 \pi}{2}-\alpha\right)=\tan \left(\dfrac{\pi}{2}-\alpha+1009 \pi\right)=\tan \left(\dfrac{\pi}{2}-\alpha\right)=\cot \alpha>0$.
		\end{enumerate}
	}
\end{vd}
\subsubsection{Bài tập vận dụng}
\begin{bt}%[1K1B1-5]
	Xét dấu các giá trị lượng giác của góc lượng giác $\alpha=\dfrac{5 \pi}{6}$.
	\loigiai{
		Do $\dfrac{\pi}{2}<\dfrac{5 \pi}{6}<\pi$ nên $\sin \left(\dfrac{5 \pi}{6}\right)>0; \cos \left(\dfrac{5 \pi}{6}\right)<0; \tan \left(\dfrac{5 \pi}{6}\right)<0; \cot \left(\dfrac{5 \pi}{6}\right)<0$.
	}
\end{bt}
\begin{bt}%[1K1B1-5]
	Xác định dấu của $\sin \alpha, \cos \alpha, \tan \alpha$, biết
	\begin{enumEX}{3}
		\item $\dfrac{3 \pi}{2}<\alpha<\dfrac{7 \pi}{4}$;
		\item $3 \pi<\alpha<\dfrac{10 \pi}{3}$;
		\item $\dfrac{5 \pi}{2}<\alpha<\dfrac{11 \pi}{4}$.
	\end{enumEX}
	\loigiai{
		\begin{enumerate}
			\item $\dfrac{3 \pi}{2}<\alpha<\dfrac{7 \pi}{4}$.
			Ta có điểm cuối của cung $\alpha$ thuộc góc phần tư thứ IV nên $\sin \alpha<0, \cos \alpha>0, \tan \alpha<0$.
			\item $3 \pi<\alpha<\dfrac{10 \pi}{3}$.
			Ta có điểm cuối của cung $\alpha$ thuộc góc phần tư thứ III nên $\sin \alpha<0, \cos \alpha<0, \tan \alpha>0$.
			\item $\dfrac{5 \pi}{2}<\alpha<\dfrac{11 \pi}{4}$. 
			Ta có điểm cuối của cung $\alpha$ thuộc góc phần tư thứ II nên $\sin \alpha>0, \cos \alpha<0, \tan \alpha<0$.
		\end{enumerate}
	}
\end{bt}
\begin{bt}%[1K1K1-5]
	Cho $0^{\circ}<\alpha<90^{\circ}$. Xét dấu các biểu thức sau:
	\begin{enumEX}{2}
		\item  $A=\cos \left(\alpha+90^{\circ}\right)$;
		\item  $B=\sin \left(\alpha+80^{\circ}\right)$.
	\end{enumEX}
	\loigiai{
		\begin{enumerate}
			\item $A=\cos \left(\alpha+90^{\circ}\right)=\cos \left(90^{\circ}-(-\alpha)\right)=\sin(-\alpha)=-\sin \alpha$.\\
			Vì $0^{\circ}<\alpha<90^{\circ}$ nên $\sin \alpha>0$.\\
			Do đó $A<0$.
			\item $B=\sin \left(\alpha+80^{\circ}\right)$.\\
			Vì $0^{\circ}<\alpha<90^{\circ}$ nên $80^{\circ}<\alpha+80^{\circ}<170^{\circ}$.\\
			Do đó, điểm cuối của cung $\alpha+80^{\circ}$ thuộc góc phần tư thứ I hoặc thứ II nên $B>0$.
		\end{enumerate}
	}
	
\end{bt}
\begin{bt}%[1K1K1-5]
	Cho $\pi<\alpha<\dfrac{3 \pi}{2}$. Xét dấu các biểu thức sau
	\begin{enumEX}{2}
		\item $A=\sin \left(\alpha+\dfrac{\pi}{2}\right)$
		\item $B=\sin \left(\alpha+\dfrac{1119 \pi}{2}\right)$.
	\end{enumEX}
	\loigiai{
		\begin{enumerate}
			\item $A=\sin \left(\alpha+\dfrac{\pi}{2}\right)=\sin \left(\dfrac{\pi}{2}-(-\alpha)\right)=\cos(-\alpha)=\cos \alpha<0$.
			\item {\allowdisplaybreaks
				\begin{eqnarray*}
					B&=&\sin \left(\alpha+\dfrac{1119 \pi}{2}\right)=\sin \left(\alpha-\dfrac{\pi}{2}+280 \cdot 2 \pi\right) \\
					&=&\sin \left(\alpha-\dfrac{\pi}{2}\right)=-\sin \left(\dfrac{\pi}{2}-\alpha\right)=-\cos \alpha>0	
			\end{eqnarray*}}
		\end{enumerate}
	}
\end{bt}
\subsubsection{Bài tập trắc nghiệm}
\Opensolutionfile{ans}[ans/ans-1K1-1-Dang5]
\begin{ex}%[1K1Y1-5]
	Cho góc lượng giác $\alpha=\dfrac{\pi}{3}$. Khẳng định nào sau đây là sai?
	\choice
	{$\sin\alpha>0$}
	{$\cos\alpha>0$}
	{\True $\tan\alpha<0$}
	{$\cot\alpha>0$}
	\loigiai{
		Với $\alpha=\dfrac{\pi}{3}$, ta có $0<\alpha<\dfrac{\pi}{2}$ thuộc góc phần tư thứ I nên $\sin\alpha>0,\cos\alpha>0,\tan\alpha>0,\cot\alpha>0$.
	}
\end{ex}
\begin{ex}%[1K1Y1-5]
	Cho góc lượng giác $\alpha=-\dfrac{\pi}{6}$. Khẳng định nào sau đây là sai?
	\choice
	{$\sin\alpha<0$}
	{\True $\cos\alpha<0$}
	{$\tan\alpha<0$}
	{$\cot\alpha<0$}
	\loigiai{
		Với $\alpha=-\dfrac{\pi}{6}$, ta có $\alpha$ thuộc góc phần tư thứ IV nên $\sin\alpha<0,\cos\alpha>0,\tan\alpha<0,\cot\alpha<0$.	
	}
\end{ex}
\begin{ex}%[1K1B1-5]
	Cho $0^{\circ}<\alpha<90^{\circ}$. Khẳng định nào sau đây đúng?
	\choice
	{$\sin\left(90^{\circ}+\alpha \right)<0$}
	{\True $\cos\left(90^{\circ}+\alpha \right)<0$}
	{$\sin\left(150^{\circ}+\alpha \right)>0$}
	{$\cos\left(180^{\circ}+\alpha \right)>0$}
	\loigiai{
		Ta có $0^{\circ}<\alpha<90^{\circ}  \Leftrightarrow  90^{\circ}<\alpha+90^{\circ}<180^{\circ}$ thuộc góc phần tư thứ II, nên $\cos\left(90^{\circ}+\alpha \right)<0$. 
	}
\end{ex}
\begin{ex}%[1K1K1-5]
	Cho $0<\alpha<\dfrac{\pi}{2}$. Khẳng định nào sau đây là đúng?
	\choice
	{\True $\sin\left(\dfrac{\pi}{6}+\alpha\right)>0$}
	{$\cos\left(\dfrac{\pi}{3}+\alpha\right)>0$}
	{$\sin\left(\dfrac{3\pi}{2}+\alpha\right)>0$}
	{$\cos\left(\dfrac{2\pi}{3}+\alpha\right)>0$}
	\loigiai{
		Ta có $0<\alpha<\dfrac{\pi}{2} \Leftrightarrow \dfrac{\pi}{6}<\dfrac{\pi}{6}+\alpha<\dfrac{2\pi}{3}$ thuộc góc phần tư thứ I và II, nên $\sin\left(\dfrac{\pi}{6}+\alpha\right)>0$. 	
	}
\end{ex}
\begin{ex}%[1K1K1-5]
	Cho $\pi<\alpha<\dfrac{3\pi}{2}$. Khẳng định nào sau đây là đúng?
	\choice
	{$\sin\left(\alpha-\dfrac{\pi}{2}\right)<0$}
	{$\cos\left(\alpha-\dfrac{\pi}{5}\right)>0$}
	{\True $\sin\left(\dfrac{2\pi}{5}+\alpha\right)<0$}
	{$\sin\left(\dfrac{2\pi}{3}+\alpha\right)>0$}
	\loigiai{
		Ta có $\pi<\alpha<\dfrac{3\pi}{2} \Leftrightarrow \dfrac{7\pi}{5}<\dfrac{2\pi}{5}+\alpha<\dfrac{19\pi}{10}$ thuộc góc phần tư thứ III và IV, nên $\sin\left(\dfrac{2\pi}{5}+\alpha\right)<0$.	
	}
\end{ex}
\Closesolutionfile{ans}