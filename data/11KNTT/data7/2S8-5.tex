% \setcounter{section}{4}
\section{GÓC GIỮA ĐƯỜNG THẲNG VÀ MẶT PHẲNG. GÓC NHỊ DIỆN}
\subsection{Trọng tâm kiến thức}
\begin{tomtat}
\subsubsection{Góc giữa đường thẳng và mặt phẳng}
\begin{boxdn}
	Nếu đường thẳng $a$ vuông góc với mặt phẳng $(P)$ thì ta nói rằng góc giữa đường thẳng $a$ và mặt phẳng $(P)$ bằng $90^{\circ}$.\\	
	Nếu đường thẳng a không vuông góc với mặt phẳng $(P)$ thì góc giữa a và hình chiếu $a'$ của nó trên $(P)$ được gọi là góc giữa đường thẳng a và mặt phẳng $(P)$.
\end{boxdn}
\begin{center}
	\begin{tabular}{cc}
	\begin{tikzpicture}[scale=.6,font=\footnotesize,line join=round,line cap=round,>=stealth]
	\path 
	(0,0)coordinate(A) 
	(7,0)coordinate(B) 
	(60:3.5)coordinate(C) 
	($(B)+(C)-(A)$) coordinate(D)
	(3.5,4.5) coordinate(M) 
	(3.5,-1) coordinate(N) 
	(3.5,1.5) coordinate(K) 
	(intersection of M--N and A--B) coordinate(O)	 
	;
	\draw (C)--(A)--(B)--(D)--cycle 
	pic[draw,angle radius=6mm]{angle=B--A--C}
	;
	\draw[red] (M)--(K) (O)--(N)
	;	
	\draw[dashed,red] (K)--(O);	
	\path (A)+(30:4.5mm)node{$P$}
	(M)+(-30:4mm)node{$a$}
	;
	\fill[black](K)circle(2pt);
	\end{tikzpicture}
	&	\begin{tikzpicture}[scale=.6,font=\footnotesize,line join=round,line cap=round,>=stealth]
	\path 
	(0,0)coordinate(A) 
	(7,0)coordinate(B) 
	(60:3.5)coordinate(C) 
	($(B)+(C)-(A)$) coordinate(D)
	(3.5,4.5) coordinate(M) 
	(3.5,1.5) coordinate(N) 
	(6.5,1.8) coordinate(K) 
	($(K)!1.2!(M)$)coordinate(A1)
	($(K)!1.4!(N)$)coordinate(A2)
	($(M)!1.6!(K)$)coordinate(A3)
	($(N)!1.3!(K)$)coordinate(A4)
	(intersection of M--N and C--D) coordinate(O1)
	(intersection of M--K and C--D) coordinate(O2)
	(intersection of M--A3 and B--D) coordinate(O3)	 
	;
	\draw (C)--(A)--(B)--(D) (C)--(O1) (D)--(O2)
	(A2)--(N) (M)--(N)--(K) (K)--(A4)
	pic[draw,angle radius=6mm]{angle=B--A--C}
	pic[draw,angle radius=4mm,double]{angle=M--K--N}
	pic[draw,angle radius=2mm]{right angle=M--N--K}
	;	
	\draw[red] (A1)--(K) (O3)--(A3);
	\draw[dashed] (O1)--(O2); 
	\draw[dashed,red](K)--(O3);
	\path (A)+(30:4.5mm)node{$P$}
	(A1)+(0:4.5mm)node{$a$}
	(A2)+(40:4.5mm)node{$a'$}
	(M)+(10:4mm)node{$A$}
	(N)+(-90:4mm)node{$H$}
	(K)+(50:4mm)node{$O$}
	;
	\foreach \x in {M,N,K}\fill[black] (\x) circle (1.5pt);
	\end{tikzpicture}
\end{tabular}\\
\begin{tabular}{cc}
	\begin{tikzpicture}[scale=.6,font=\footnotesize,line join=round,line cap=round,>=stealth]
	\path 
	(0,0)coordinate(A) 
	(7,0)coordinate(B) 
	(60:3.5)coordinate(C) 
	($(B)+(C)-(A)$) coordinate(D)
	(3.5,4.5) coordinate(M) 
	(3.5,1.5) coordinate(N) 
	(6.5,1.8) coordinate(K)
	($(M)+(K)-(N)$) coordinate(H)
	($(H)!1.3!(M)$)coordinate(A1)
	($(M)!1.3!(H)$)coordinate(A2)
	($(K)!1.3!(N)$)coordinate(A3)
	($(N)!1.3!(K)$)coordinate(A4)
	(intersection of M--N and C--D) coordinate(O1)
	(intersection of H--K and C--D) coordinate(O2)	 
	;
	\draw (C)--(A)--(B)--(D) (C)--(O1) (D)--(O2)
	(A3)--(A4) (M)--(N) (K)--(H)
	pic[draw,angle radius=6mm]{angle=B--A--C}
	pic[draw,angle radius=2mm]{right angle=M--N--K}
	pic[draw,angle radius=2mm]{right angle=H--K--A4}
	;
	\draw[red] (A1)--(A2);
	\draw[dashed] (O1)--(O2);	
	\path (A)+(30:4.5mm)node{$P$}
	(M)+(70:4mm)node{$A$}
	(N)+(-90:4mm)node{$H$}
	(K)+(-90:4mm)node{$K$}
	(H)+(50:4mm)node{$B$}
	(M)--(A1)node[midway,above]{$a$}
	(N)--(A3)node[midway,above]{$a'$}
	;
	\foreach \x in {M,N,K,H}\fill[black] (\x) circle (1.5pt);
	\end{tikzpicture}
	&\begin{tikzpicture}[scale=.6,font=\footnotesize,line join=round,line cap=round,>=stealth]
	\path 
	(0,0)coordinate(A) 
	(7,0)coordinate(B) 
	(60:3.5)coordinate(C) 
	($(B)+(C)-(A)$) coordinate(D)
	(2,1.5) coordinate(M) 
	(6.5,1.8) coordinate(N)
	;
	\draw (C)--(A)--(B)--(D)--cycle 
	pic[draw,angle radius=6mm]{angle=B--A--C}
	;
	\draw[red] (M)--(N)	;
	\path (A)+(30:4.5mm)node{$P$}
	(M)+(-70:4mm)node{$a'$}
	(N)+(90:4mm)node{$a$}
	;
	\end{tikzpicture}
\end{tabular}
\end{center}
\begin{note}
	Chú ý: Nếu $\alpha$ là góc giữa đường thẳng a và mặt phẳng $(P)$ thì $0 \leq \alpha \leq 90^{\circ}$.
\end{note}
\begin{nx}
	\immini{Cho điểm $A$ có hình chiếu $H$ trên mặt phẳng $(P)$. Lấy điểm $O$ thuộc mặt phẳng $(P)$, $O$ không trùng $H$. Khi đó góc giữa đường thẳng $A O$ và mặt phẳng $(P)$ bằng góc $AOH$}{
	\vspace*{-3mm}
	\begin{tikzpicture}[scale=.5,font=\footnotesize,line join=round,line cap=round,>=stealth]
	\path 
	(0,0)coordinate(A) 
	(7,0)coordinate(B) 
	(60:3.5)coordinate(C) 
	($(B)+(C)-(A)$) coordinate(D)
	(3.5,4.5) coordinate(M) 
	(3.5,1.5) coordinate(N) 
	(6.5,1.8) coordinate(K) 
	($(K)!1.2!(M)$)coordinate(A1)
	($(K)!1.4!(N)$)coordinate(A2)
	($(M)!1.6!(K)$)coordinate(A3)
	($(N)!1.3!(K)$)coordinate(A4)
	(intersection of M--N and C--D) coordinate(O1)
	(intersection of M--K and C--D) coordinate(O2)
	(intersection of M--A3 and B--D) coordinate(O3)	 
	;
	\draw (C)--(A)--(B)--(D) (C)--(O1) (D)--(O2)
	(M)--(N)--(K) 
	pic[draw,angle radius=6mm]{angle=B--A--C}
	pic[draw,angle radius=4mm,double]{angle=M--K--N}
	pic[draw,angle radius=2mm]{right angle=M--N--K}
	;	
	\draw (M)--(K) ;
	\draw[dashed] (O1)--(O2); 
	%	\draw[dashed,red](K)--(O3);
	\path (A)+(30:4.5mm)node{$P$}
	%	(A1)+(0:4.5mm)node{$a$}
	%	(A2)+(40:4.5mm)node{$a'$}
	(M)+(10:4mm)node{$A$}
	(N)+(-90:4mm)node{$H$}
	(K)+(50:4mm)node{$O$}
	;
	\foreach \x in {M,N,K}\fill[black] (\x) circle (1.5pt);
	\end{tikzpicture}
	}
\end{nx}
\subsubsection{Góc nhị diện}
\begin{boxdn}
	\immini
	{
	Hình gồm hai nửa mặt phẳng $(P)$, $(Q)$ có chung bờ $a$ được gọi là một \text{\color{red} góc nhị diện}, kí hiệu là $[P, a, Q]$. Đường thẳng $a$ và các nửa mặt phẳng $(P)$, $(Q)$ tương ứng được gọi là các mặt phẳng của góc nhị diện đó.
	}
	{
	\begin{tikzpicture}[>=stealth,line join=round,line cap=round,font=\footnotesize,scale=.7]
	\path 
	(0,0) coordinate (a1)
	($(a1)+(60:2)$) coordinate (a2)
	($(a1)+(0:4)$) coordinate (q1)
	($(a1)+(150:3)$) coordinate (p1)
	($(p1)+(a2)-(a1)$) coordinate (p2)
	($(q1)+(a2)-(a1)$) coordinate (q2)
	($(a1)!.1!(a2)$) coordinate (a)
	;
	\draw 
	(a1)--(a2)--(p2)--(p1)--(a1)--(q1)--(q2)--(a2)
	;
	\draw (a) node[above left] {$a$};
	\begin{scope}
	\clip (p2)--(p1)--(a1);
	\draw (p1) circle (.6cm);
	\draw ($(p1)+(10:.35)$) node{$P$};
	\end{scope}
	\begin{scope}
	\clip (q2)--(q1)--(a1);
	\draw (q1) circle (.6cm);
	\draw ($(q1)+(130:.3)$) node{$Q$};
	\end{scope}
	\end{tikzpicture}
	}
\end{boxdn}
\begin{boxdn}
	\immini
	{
	Từ một điểm $O$ bất kì thuộc cạnh $a$ của góc nhị diện $[P, a, Q]$, vẽ các tia $Ox$, $Oy$ tương ứng thuộc $(P)$, $(Q)$ và vuông góc với $a$. Góc $xOy$ được gọi là một \text{\color{red} góc phẳng của góc nhị diện} $[P, a, Q]$ (gọi tắt là \text{\color{red} góc phẳng nhị diện}). Số đo của góc $xOy$ không phụ thuộc vào vị trí của $O$ trên $a$, được gọi là số đo của góc nhị diện $[P,a,Q]$.
	}
	{
	\begin{tikzpicture}[>=stealth,line join=round,line cap=round,font=\footnotesize,scale=.8]
	\path 
	(0,0) coordinate (a1)
	($(a1)+(60:2)$) coordinate (a2)
	($(a1)+(0:4)$) coordinate (q1)
	($(a1)+(150:3)$) coordinate (p1)
	($(p1)+(a2)-(a1)$) coordinate (p2)
	($(q1)+(a2)-(a1)$) coordinate (q2)
	($(a1)!.1!(a2)$) coordinate (a)
	($(a1)!.6!(a2)$) coordinate (O)
	($(p1)!.6!(p2)$) coordinate (x1)
	($(O)!.8!(x1)$) coordinate (x)
	($(q1)!.6!(q2)$) coordinate (y1)
	($(O)!.8!(y1)$) coordinate (y)
	;
	\draw 
	(a1)--(a2)--(p2)--(p1)--(a1)--(q1)--(q2)--(a2)
	(x)--(O)--(y)
	;
	\draw (a) node[above left] {$a$}
	(x) node[above] {$x$} (y) node[above] {$y$};
	\draw[fill=black] (O) circle (1pt) node[shift=(-60:3mm)] {$O$} ;
	\begin{scope}
	\clip (p2)--(p1)--(a1);
	\draw (p1) circle (.6cm);
	\draw ($(p1)+(10:.35)$) node{$P$};
	\end{scope}
	\begin{scope}
	\clip (q2)--(q1)--(a1);
	\draw (q1) circle (.6cm);
	\draw ($(q1)+(130:.3)$) node{$Q$};
	\end{scope}
	\tkzMarkRightAngle(x,O,a1)
	\tkzMarkRightAngle(y,O,a2)
	\end{tikzpicture}
	}	
\end{boxdn}
\begin{note}
	\begin{itemize}
	\item Số đo của góc nhị diện có thể nhận từ $0^\circ$ đến $180^\circ$. Góc nhị diện được gọi là vuông, nhọn, tù nếu nó có số đo tương ứng bằng, nhỏ hơn, lớn hơn $90^\circ$.
	\item Đối với hai điểm $M$, $N$ không thuộc đường thẳng $a$, ta kí hiệu $[M,a,N]$ là góc nhị diện có cạnh $a$ và các mặt phẳng tương ứng chứa $M$, $N$.
	\item Hai mặt phẳng cắt nhau tạo thành bốn góc nhị diện. Nếu một trong bốn góc nhị diện đó là góc nhị diện vuông thì các góc nhị diện còn lại cũng là góc nhị diện vuông.
	\end{itemize}
\end{note}
\end{tomtat}
%%%%%%%%%%%%%%%%%%%%
\subsection{Các dạng bài tập}
\begin{dang}{Góc giữa đường thẳng và mặt phẳng}
	Cho đường thẳng $d$ và mặt phẳng $(P)$ cắt nhau.\\
	Nếu $d\perp (P)$ thì $(d,(P))=90^{\circ}$.
	\begin{center}
	\begin{tikzpicture}[scale=0.6]
	\tkzInit[ymin=-5,ymax=0.5,xmin=-4.1,xmax=4.1]
	\tkzClip %cắt bớt phần ko gian dư
	\tkzDefPoints{-4/-4/M, -2/-1.5/Q, 2/-4/N, -0.5/-3/H}
	\tkzDefPointBy[translation = from M to N](Q) \tkzGetPoint{P}%phép tịnh tiến biến Q thành P
	\tkzDefLine[perpendicular=through H,K=0.4](Q,P) \tkzGetPoint{h}%đường qua H vuông góc QP
	\tkzInterLL(H,h)(Q,P) \tkzGetPoint{h_1}
	\coordinate[label={above right}:$A$] (A) at ($(h_1)+(0,1)$);
	\tkzDefLine[parallel=through H](Q,P) \tkzGetPoint{x}%đường qua H song song MQ
	\coordinate[label={above right}:$O$] (O) at ($(H)!0.4!(x)$);
	\tkzInterLL(A,O)(N,P) \tkzGetPoint{O_1}
	\tkzInterLL(A,O)(P,Q) \tkzGetPoint{O_3}
	\coordinate (O_2) at ($(O)!3.3!(O_1)$);
	\tkzLabelPoints[below](H)
	\tkzDrawLines[add=0 and 0.3](O,H O,A)
	\tkzDrawSegments[dashed](h_1,O_3 O,O_1)
	\tkzDrawSegments(M,N N,P P,O_3 h_1,Q Q,M A,H O_1,O_2)
	\tkzLabelLine[below left,pos=1.3](O,A){$d$}
	\tkzLabelLine[above,pos=1.3](O,H){$d'$}
	\tkzMarkRightAngle(A,H,O)
	\tkzMarkAngles[arc=l,size=0.7cm](A,O,H)
	\tkzLabelAngle[pos=0.9](A,O,H){$\varphi$}
	\tkzMarkAngles[arc=l,size=1.1cm](N,M,Q)
	\tkzLabelAngle[pos=0.7](N,M,Q){$\alpha$}
	\end{tikzpicture}
	\end{center}
	Nếu $d\not\perp (P)$ thì để xác định góc giữa $d$ và $(P)$, ta thường làm như sau
	\begin{enumerate}
	\item Xác định giao điểm $O$ của $d$ và $(P)$.
	\item Lấy một điểm $A$ trên $d$ ($A$ khác $O$). Xác định hình chiếu vuông góc (vuông góc) $H$ của $A$ lên $(P)$. Lúc đó $(d,(P))=(d,d')=\widehat{AOH}$.
	\end{enumerate}
\end{dang}
\subsubsection{Ví dụ minh hoạ}
\begin{vd}%[1C8B3-1]
	Cho hình chóp $S.ABC$ có $SA \perp(ABC)$, $SA=a$, $CA=CB=a \sqrt{7}$, $AB=2a$.
	\begin{listEX}[2]
	\item Gọi $\alpha$ là góc giữa $SB$ và $(ABC)$. Tính $\tan \alpha$.
	\item Tính góc giữa $SC$ và $(SAB)$.
	\end{listEX}
	\loigiai{
	\begin{listEX}[1]
	\item \immini{Do $SA \perp(ABC)$ nên $\alpha=\widehat{SBA}$. Tam giác $SAB$ vuông tại $A$ nên
	$$
	\tan \alpha=\tan \widehat{SBA}=\dfrac{SA}{AB}=\dfrac{a}{2a}=\dfrac{1}{2} .
	$$}{\begin{tikzpicture}[scale=.8,>=stealth, font=\footnotesize, line join=round, line cap=round]
	\path 
	(0,0)coordinate(A) 
	(5,0)coordinate(B) 
	(-60:2.5)coordinate(C) 
	($(A)+(0,3)$) coordinate(S)
	($(A)!0.5!(B)$)coordinate(M)	 
	;
	\draw (S)--(A)--(C)--(B)--cycle (C)--(S) 
	pic[draw,angle radius=3mm]{right angle=S--A--B}
	pic[draw,angle radius=3mm]{right angle=S--A--C}
	pic[draw,angle radius=3mm]{right angle=C--M--B}
	;
	\path
	(S)--(A)node[midway,left]{$a$}
	(M)--(A)node[midway,above]{$a$}
	(B)--(M)node[midway,above]{$a$}
	(C)--(A)node[midway,left]{$a\sqrt{7}$}
	(C)--(B)node[midway,right]{$a\sqrt{7}$}
	;
	\draw [dashed](S)--(M)--(C) (B)--(A) 
	;	
	\foreach \x/\g in {S/90,A/180,C/-90,M/40,B/0}\fill[black] (\x) circle (1pt) +(\g:.3)node{$\x$};
	\end{tikzpicture}}
	\item Gọi $M$ là trung điểm của $A B$. Tam giác $ABC$ cân tại $C$ nên $CM \perp AB$.\\	
	Mặt khác, từ $SA \perp(ABC)$ ta có $CM \perp SA$. Do đó $CM \perp(SAB)$.
	Vậy góc giữa $SC$ và $(SAB)$ bằng $\widehat{CSM}$.\\
	Tam giác $SAC$ vuông tại $A$ nên $SC=\sqrt{S A^2+A C^2}=\sqrt{a^2+7 a^2}=a \sqrt{8}$.\\
	Ta có $AM=\dfrac{1}{2} AB=a$. Do đó, tam giác $SAM$ vuông cân tại $A$ và $SM=a \sqrt{2}$.\\
	Tam giác $CMS$ vuông tại $M$ và $\cos \widehat{CSM}=\dfrac{SM}{SC}=\dfrac{a \sqrt{2}}{a \sqrt{8}}=\dfrac{1}{2}$.\\
	Vậy $\widehat{CSM}=60^{\circ}$ và do đó góc giữa $SC$ và $(SAB)$ bằng $60^{\circ}$.
	\end{listEX}
	}
\end{vd}
\begin{vd}%[1H3B3]
	Cho hình chóp $S.ABCD$ có đáy $ABCD$ là hình vuông cạnh $a$, $SA=a\sqrt{6}$ và $SA$ vuông góc $(ABCD)$. Hãy xác định các góc giữa
	\begin{listEX}[4]
	\item $SC$ và $(ABCD)$.
	\item $SC$ và $(SAB)$.
	\item $SB$ và $(SAC)$.
	\item $AC$ và $(SBC)$.
	\end{listEX}
	\loigiai{
	\begin{center}
	\begin{tikzpicture}[scale=0.8]
	%Hình chóp S.ABCD có SA vuông góc đáy, đáy ABCD là hình chữ nhật.
	\tkzDefPoints{0/0/A, -3/-2/B, 6/0/D}
	\coordinate (C) at ($(B)+(D)-(A)$);
	\coordinate (S) at ($(A)+(0,4.5)$);
	\tkzInterLL(A,C)(B,D) \tkzGetPoint{O}
	% Chú ý MN song song BD.
	\coordinate (M) at ($(S)!0.75!(B)$);
	\coordinate (N) at ($(S)!0.6!(D)$);
	% vẽ đoạn thẳng
	\draw[thick] (S)--(C);
	\draw[thick] (S)--(D);
	\draw[thick] (S)--(B);
	\draw[thick] (B)--(C);
	\draw[thick] (C)--(D) (M)--(C);
	\draw[dashed] (O)--(S)--(A);
	\draw[dashed] (D)--(A);
	\draw[dashed] (A)--(B);
	\draw[dashed] (A)--(C);
	\draw[dashed] (D)--(B);
	%\draw[dashed] (A)--(N);
	\draw[dashed] (A)--(M);
	%\draw[dashed] (M)--(N);
	% Gán nhãn các điểm
	\tkzLabelPoints[below](A)
	\tkzLabelPoints[right](D)
	\tkzLabelPoints[below](B)
	\tkzLabelPoints[below](C)
	\tkzLabelPoints[above](S)
	\tkzLabelPoints[right](O)
	%\tkzLabelPoints[right](N)
	\tkzLabelPoints[left](M)
	% Kí hiệu các góc
	\tkzMarkRightAngle(S,A,D)
	\tkzMarkRightAngle(S,A,B)
	\tkzMarkRightAngle(A,M,S)
	%\tkzMarkRightAngle(A,N,S)
	\tkzMarkRightAngle(A,O,B)
	% Vẽ các điểm
	\tkzDrawPoints(S,A,B,C,D,O,M)
	\tkzMarkAngles[mark=|,arc=l,size=0.7cm](S,C,A)
	\tkzMarkAngles[mark=||,arc=l,size=1.2cm](B,S,C)
	\tkzMarkAngles[mark=|||,arc=l,size=0.5cm](B,S,O)
	\tkzMarkAngles[mark=x,arc=l,size=1.3cm](A,C,M)
	%\tkzLabelAngle[pos=0.9](A,O,H){$\varphi$}
	\end{tikzpicture}
	\end{center}
	\begin{enumerate}
	\item Vì $AC$ là hình chiếu vuông góc của $SC$ lên $(ABCD)$ nên góc giữa $SC$ và $(ABCD)$ là $\widehat{SCA}$.\\
	Trong tam giác $SCA$, ta có $\tan\widehat{SCA}=\dfrac{SA}{SC}=\sqrt{3}$ nên $\left(SC,(ABCD)\right)=\widehat{SCA}=60^{\circ}$.
	\item Vì $BC\perp (SAB)$ tại $B$ nên $SB$ là hình chiếu vuông góc của $SC$ lên $(SAB)$.\\
	Do đó $\left(SC,(SAB)\right)=(SC,SB)=\widehat{CSB}$.\\
	Trong tam giác $SCB$, ta có $\tan\widehat{CSB}=\dfrac{BC}{SB}=\dfrac{a}{a\sqrt{7}}$ nên $\left(SC,(SAB)\right)=\arctan\dfrac{1}{\sqrt{7}}$.
	\item Vì $BO\perp (SAC)$ tại $O$ nên $SO$ là hình chiếu vuông góc của $SB$ lên $(SAC)$.\\ Do đó $\left(SB,(SAC)\right)=(SB,SO)=\widehat{BSO}$.\\
	Trong tam giác $SBO$, ta có $\sin\widehat{BSO}=\dfrac{BO}{SB}=\dfrac{\dfrac{a\sqrt{2}}{2}}{a\sqrt{7}}=\dfrac{1}{\sqrt{14}}$ nên $\left(SB,(SAC)\right)=\arcsin\dfrac{1}{\sqrt{14}}$.
	\item Gọi $M$ là hình chiếu vuông góc của $A$ lên $SB$. Lúc đó $AM\perp SB$ và $AM\perp BC$ (vì $BC\perp(SAB)$ và $AM\subset (SAB)$) nên $AM\perp (SBC)$ tại $M$. Do đó $MC$ là hình chiếu vuông góc của $AC$ lên $(SBC)$.\\ 
	Suy ra $\left(AC,(SBC)\right)=(AC,MC)=\widehat{ACM}$.\\
	Trong tam giác $SAB$, ta có $AM=\dfrac{SA.AB}{SB}=\dfrac{a\sqrt{6}}{\sqrt{7}}$ và trong tam giác $ACM$, ta có $\sin\widehat{ACM}=\dfrac{MA}{AC}=\dfrac{\sqrt{21}}{7}$ nên $\left(AC,(SBC)\right)=\arcsin\dfrac{\sqrt{21}}{7}$.
	\end{enumerate}
	}
\end{vd}
\begin{vd}%[1H3G3]
	Cho hình chóp $S.ABCD$ có đáy là hình vuông cạnh $a$, tâm $O$, $SO$ vuông góc $(ABCD)$. Gọi $M, N$ lần lượt là trung điểm $SA, BC$. Biết rằng góc giữa $MN$ và $(ABCD)$ bằng $60^{\circ}$. Tính góc giữa $MN$ và $(SBD)$.
	\loigiai{
	\begin{center}
	\begin{tikzpicture}
	\tkzDefPoints{0/0/O, -4/-1/A, 1/-1/B, 0/5.5/S}
	\tkzDefPointBy[symmetry = center O](A)
	\tkzGetPoint{C}
	\tkzDefPointBy[symmetry = center O](B)
	\tkzGetPoint{D}
	\tkzDefMidPoint(S,A)
	\tkzGetPoint{M}
	\tkzDefMidPoint(B,C)
	\tkzGetPoint{N}
	\tkzDefMidPoint(A,O)
	\tkzGetPoint{H}
	\tkzDefMidPoint(S,D)
	\tkzGetPoint{K}
	\tkzDrawSegments[dashed](A,D A,C D,C D,B S,D S,O M,N M,H N,H C,K O,K M,K)
	\tkzDrawSegments(A,B B,C S,A S,B S,C)
	\tkzMarkRightAngle[size=.2](B,O,C)
	\tkzMarkRightAngle[size=.2](M,H,N)
	\tkzMarkRightAngle[size=.2](K,O,C)
	\tkzLabelPoints[below](O,A,B,N,H)
	\tkzLabelPoints[above left](S,M,K)
	\tkzLabelPoints[right](C)
	\tkzLabelPoints[left](D)
	\tkzMarkAngles[mark=|,arc=l,size=0.5cm](O,K,C)
	\tkzMarkAngles[arc=l,size=0.7cm](M,N,H)
	%\tkzMarkSegments[mark=||](S,A S,B S,C S,D)
	\end{tikzpicture}
	\end{center}
	Gọi $H$ là trung điểm $AO$. Ta có $MH\parallel SO$ nên $MH\perp (ABCD)$, suy ra $HN$ là hình chiếu vuông góc của $MN$ lên $(ABCD)$. Do đó $\left(MN,(ABCD)\right)=(MN,KN)=\widehat{MNK}=60^{\circ}$.\\
	Trong tam giác $HCN$, ta có $HN^2=HC^2+CN^2-2HC.CN.\cos\widehat{HCN}$, suy ra $HN=\dfrac{a\sqrt{10}}{4}$.\\
	Mà trong tam giác $MNH$, ta có $\sqrt{3}=\tan\widehat{MNH}=\dfrac{MH}{HN}$ nên $MH=\dfrac{a\sqrt{30}}{4}$, suy ra $SO=2MH=\dfrac{a\sqrt{30}}{2}$.\\
	Gọi $K$ là trung điểm $SD$.\\
	Ta có $MKCN$ là hình bình hành nên $MN$ song song $KC$. Do đó $\left(MN,(SBD)\right)=\left(KC,(SBD)\right)$.\\
	Mà $CO\perp (SBD)$ tại $O$ (do $CO\perp DO$ và $CO\perp SO$) nên $KO$ là hình chiếu vuông góc của $KC$ lên $(SBD)$. Suy ra $\left(KC,(SBD)\right)=\left(KC,KO\right)=\widehat{CKO}$.\\
	Ta có $OK=\dfrac{1}{2}SD=\dfrac{1}{2}\sqrt{OD^2+OS^2}=a\sqrt{2}$.\\
	Mặt khác, trong tam giác $COK$, ta có $\tan\widehat{CKO}=\dfrac{OC}{OK}=\dfrac{1}{2}$, suy ra $\left(KC,(SBD)\right)=\arctan\widehat{CKO}=\arctan\dfrac{1}{2}\approx 26^{\circ}33'$. 
	}
\end{vd}
\subsubsection{Bài tập áp dụng}
\begin{bt}%[1H3B3]
	Cho hình chóp $S.ABC$ có đáy $ABC$ là tam giác đều cạnh $a$, $SA=2a$ và $SA$ vuông góc với đáy. Tính góc giữa
	\begin{listEX}[2]
	\item $SC$ và $(ABC)$.
	\item $SC$ và $(SAB)$.
	\end{listEX}
	\loigiai{
	\begin{center}
	\begin{tikzpicture}[scale=1]
	%Hình chóp S.ABC có SA vuông góc đáy, đáy ABC là tam giác vuông tại B.
	\tkzDefPoints{0/0/A, 2.5/-2/B, 6/0/C}
	\coordinate (S) at ($(A)+(0,4)$);
	\coordinate (M) at ($(A)!0.5!(B)$);
	%\coordinate (N) at ($(S)!0.4!(C)$);
	\draw[thick] (M)--(S)--(A);
	\draw[thick] (S)--(C);
	\draw[thick] (A)--(B);
	\draw[thick] (B)--(C);
	\draw[thick] (B)--(S);
	%\draw (A)--(M);
	%\draw (N)--(M);
	\draw[dashed] (A)--(C)--(M);
	%\draw[dashed] (A)--(N);
	\tkzLabelPoints[left](A)
	\tkzLabelPoints[right](C)
	\tkzLabelPoints[below](B)
	\tkzLabelPoints[above](S)
	%\tkzLabelPoints[right](N)
	\tkzLabelPoints[left](M)
	\tkzMarkRightAngle(S,A,C)
	\tkzMarkRightAngle(S,A,B)
	\tkzMarkRightAngle(A,M,B)
	\tkzMarkRightAngle(B,M,C)
	\tkzDrawPoints(S,A,B,C)
	\tkzMarkAngles[mark=|,arc=l,size=0.7cm](S,C,A)
	\tkzMarkAngles[mark=||,arc=l,size=0.7cm](M,S,C)
	\end{tikzpicture}
	\end{center}
	\begin{enumerate}
	\item Vì $AC$ là hình chiếu vuông góc của $SC$ lên $(ABC)$ nên $\left(SC,(ABC)\right)=(SC,AC)=\widehat{SCA}$.\\
	Ta có $\tan\widehat{SCA}=\dfrac{SA}{AC}=2$ nên $\left(SC,(ABC)\right)=\arctan2\approx 63^{\circ}$.
	\item Gọi $M$ là trung điểm $AB$.
	Vì $\heva{CM&\perp AB \\ CM&\perp SA \ (\text{vì} \ SA\perp (ABC))}$ nên $CM\perp (SAB)$ tại $M$.\\
	Suy ra $SM$ là hình chiếu vuông góc của $SC$ lên $(SAB)$.\\
	Do đó $\left(SC,(SAB)\right)=(SC,SM)=\widehat{CSM}$.\\
	Trong tam giác $SMC$, ta có $\tan\widehat{CSM}=\dfrac{MC}{SM}=\dfrac{MC}{\sqrt{SA^2+AM^2}}=\dfrac{\dfrac{a\sqrt{3}}{2}}{\sqrt{4a^2+\dfrac{a^2}{4}}}=\dfrac{\sqrt{51}}{17}$.\\
	Vậy $(SC,(SAB))=\arctan\dfrac{\sqrt{51}}{17}\approx 23^{\circ}$.
	\end{enumerate}
	}
\end{bt}
\begin{bt}%[1H3B3]
	Cho hình chóp $S.ABCD$ có đáy $ABCD$ là hình vuông tâm $O$, cạnh $a$, $SO$ vuông góc $(ABCD)$ và $SO=a\sqrt{6}$.
	\begin{enumerate}
	\item Tính góc giữa cạnh bên $SC$ và mặt đáy.
	\item Tính góc giữa $SO$ và $(SAD)$.
	\item Gọi $I$ là trung điểm $BC$. Tính góc giữa $SI$ và $(SAD)$.
	\end{enumerate}
	\loigiai{\begin{center}
	\begin{tikzpicture}
	\tkzDefPoints{0/0/O, -4/-1/A, 1/-1/B, 0/4.5/S}
	\tkzDefPointBy[symmetry = center O](A)
	\tkzGetPoint{C}
	\tkzDefPointBy[symmetry = center O](B)
	\tkzGetPoint{D}
	\tkzDefMidPoint(A,D)
	\tkzGetPoint{K}
	\tkzDefMidPoint(B,C)
	\tkzGetPoint{I}
	\coordinate (H) at ($(S)!0.8!(K)$);
	\tkzDefLine[parallel=through I](O,H) \tkzGetPoint{d}
	\tkzInterLL(d,I)(S,K) \tkzGetPoint{E}
	\tkzDrawSegments[dashed](A,D A,C D,C D,B S,D S,O K,I K,S I,S O,H I,E)
	\tkzDrawSegments(A,B B,C S,A S,B S,C)
	\tkzMarkRightAngle[size=.2](B,O,C)
	\tkzMarkRightAngle[size=.2](S,H,O)
	\tkzMarkRightAngle[size=.2](S,E,I)
	\tkzLabelPoints[below](O,A,B,I,K)
	\tkzLabelPoints[above left](S)
	\tkzLabelPoints[above](E,H)
	\tkzLabelPoints[right](C)
	\tkzLabelPoints[left](D)
	\tkzMarkAngles[mark=|,arc=l,size=0.7cm](S,C,A)
	\tkzMarkAngles[arc=l,size=0.7cm](E,S,I)
	%\tkzMarkSegments[mark=||](S,A S,B S,C S,D)
	\end{tikzpicture}
	\end{center}
	\begin{enumerate}
	\item Vì $OC$ là hình chiếu vuông góc của $SC$ lên $(ABCD)$ nên $(SC,(ABCD))=(SC,OC)=\widehat{SCO}$.\\
	Trong tam giác $SOC$, ta có $\tan\widehat{SCO}=\dfrac{SO}{OC}=2\sqrt{3}$, do đó $(SC,(ABCD))=\arctan2\sqrt{3}\approx 74^{\circ}$.
	\item Gọi $K$ là trung điểm $AD$ và $H$ là hình chiếu vuông góc của $O$ lên $SK$. Ta có $OH\perp SK$ và $OH\perp AD$ (vì $AD\perp (SKO)$) nên $OH\perp (SAD)$, do đó $H$ là hình chiếu vuông góc của $O$ lên $(SAD)$, suy ra $SH$ là hình chiếu vuông góc của $SO$ lên $(SAD)$.\\ 
	Do đó $(SO,(SAD))=(SO,SH)=\widehat{HSO}$.\\
	Trong tam giác $SOK$, ta có $\tan\widehat{HSO}=\tan\widehat{KSO}=\dfrac{OK}{OS}=\dfrac{\sqrt{3}}{6}$. Suy ra $(SO,(SAD))=\arctan\dfrac{\sqrt{6}}{12}\approx 12^{\circ}$.
	\item Trong tam giác $SKI$, kẻ $IE$ vuông góc $SK$ tại $E$. Lúc đó $IE\perp (SAD)$ (do $IE \parallel OH$). Suy ra $SE$ là hình chiếu vuông góc của $SI$ lên $(SAD)$.\\
	Do đó $(SI,(SAD))=(SI,SE)=\widehat{ISE}=2\widehat{HSO}\approx 24^{\circ}$.
	\end{enumerate}
	}
\end{bt}
\begin{bt}%[1H3K3]
	Cho hình chóp $S.ABC$ có đáy $ABC$ là tam giác vuông cân tại $A$, $BC=a$, $SA=SB=SC=\dfrac{a\sqrt{3}}{2}$. Tính góc giữa $SA$ và $(ABC)$.
	\loigiai{
	\immini{
	Gọi $H$ là hình chiếu vuông góc của $S$ lên $(ABC)$. Lúc đó ba tam giác $SAH, SBH$ và $SCH$ bằng nhau (vì chúng là 3 tam giác vuông có chung cạnh $SH$ và có ba cạnh $SA, SB, SC$ bằng nhau).\\
	Suy ra $HA=HB=HC$ nên $H$ là tâm đường tròn ngoại tiếp tam giác vuông $ABC$, suy ra $H$ là trung điểm $BC$.\\
	Do đó $HA$ là hình chiếu vuông góc của $SA$ lên $(ABC)$, suy ra $(SA,(ABC))=(SA,AH)=\widehat{SAH}$.\\
	Ta có $\cos{SAH}=\dfrac{AH}{SA}=\dfrac{\dfrac{BC}{2}}{SA}=\dfrac{1}{\sqrt{3}}
	$,\\
	suy ra $(SA,(ABC))=\arccos\dfrac{\sqrt{3}}{3}\approx 55^{\circ}$.
}{
	\begin{tikzpicture}[scale=0.7]
	%Định nghĩa các điểm
	\tkzDefPoints{0/0/A, 8/0/B}
	\path(A)++(-25:6)coordinate(C);
	\tkzDefMidPoint(B,C) \tkzGetPoint{H}
	\path(H)++(90:6)coordinate(S);
	\tkzLabelPoints[above left](A,S)
	\tkzLabelPoints[below right](C,H,B)
	%Nối các điểm
	\tkzDrawSegments(S,A S,B S,C S,H A,C B,C)
	\tkzDrawSegments[style=dashed](A,B A,H)
	\tkzMarkRightAngle[size=.4](B,A,C)
	\tkzMarkRightAngle[size=.3](S,H,B)
	\tkzMarkAngles[mark=|,arc=l,size=0.7cm](H,A,S)
	\end{tikzpicture}
}
}
\end{bt}
\begin{bt}%[1H3K3]
	Cho hình chóp $S.ABCD$ có đáy là hình thang vuông tại $A$ và $B$, $AB=BC=a$, $AD=2a$. Cạnh bên $SA=a\sqrt{2}$ và vuông góc với đáy. Tính góc giữa đường thẳng $SB$ và mặt phẳng $(SAC)$.
	\loigiai{
	\begin{center}
	\begin{tikzpicture}[scale=0.7]
	%Định nghĩa các điểm
	\tkzDefPoints{0/0/A, 8/0/D}
	\path(A)++(-135:3)coordinate(B);
	\path(B)++(0:4)coordinate(C);
	\tkzDefMidPoint(A,D) \tkzGetPoint{I}
	\path(A)++(90:6)coordinate(S);
	\tkzInterLL(A,C)(B,I) \tkzGetPoint{O}
	\tkzLabelPoints[above left](A,S,I)
	\tkzLabelPoints[below right](C,D,B)
	\tkzLabelPoints[below](O)
	%Nối các điểm
	\tkzDrawSegments(C,D S,B S,C S,D B,C)
	\tkzDrawSegments[style=dashed](A,B A,D S,A A,C B,I I,C S,O)
	\tkzMarkRightAngle[size=.3](B,A,D)
	\tkzMarkRightAngle[size=.3](S,A,D)
	\tkzMarkRightAngle[size=.2](I,O,C)
	\tkzMarkRightAngle[size=.3](S,O,B)
	\tkzMarkAngles[mark=|,arc=l,size=1.5cm](B,S,O)
	\end{tikzpicture}
	\end{center}
	Gọi $I$ là trung điểm $AD$. Lúc đó $ABCI$ là hình vuông, suy ra $BI\perp AC$ (tại $O$).\\
	Mà $SA\perp (ABCD)$ nên $BI\perp SA$. Do đó $BI\perp (SAC)$ tại $O$ nên $SO$ là hình chiếu vuông góc của $SB$ lên $(SAC)$, suy ra $(SB,(SAC))=(SB,SO)=\widehat{BSO}$.\\
	Trong tam giác $SBO$, ta có $\sin\widehat{BSO}=\dfrac{BO}{SB}=\dfrac{\dfrac{BI}{2}}{\sqrt{SA^2+AB^2}}=\dfrac{\sqrt{6}}{6}$, suy ra $(SB,(SAC))=\arcsin\dfrac{\sqrt{6}}{6}\approx 24^{\circ}$.
	}
\end{bt}
\begin{bt}%[1H3B3]
	Cho hình lăng trụ tam giác $ABC.A'B'C'$ có đáy là tam giác đều cạnh $a$ và $AA'$ vuông góc $(ABC)$. Đường chéo $BC'$ của mặt bên $(BCC'B')$ hợp với $(ABB'A')$ một góc $30^{\circ}$.
	\begin{enumerate}
	\item Tính $AA'$.
	\item Gọi $M, N$ lần lượt là trung điểm $AC$ và $BB'$. Tính góc giữa $MN$ và $(ACC'A')$.
	\end{enumerate}
	\loigiai{\begin{center}
	\begin{tikzpicture}[scale=0.7]
	%Định nghĩa các điểm
	\tkzDefPoints{0/0/A, 6/0/C}
	\path(A)++(-25:4.5)coordinate(B)++(90:7)coordinate(B');
	\path(A)++(90:7)coordinate(A')++(0:6)coordinate(C');
	\tkzDefMidPoint(A',B') \tkzGetPoint{I}
	\tkzDefMidPoint(A,C) \tkzGetPoint{M}
	\tkzDefMidPoint(B,B') \tkzGetPoint{N}
	\tkzDefMidPoint(A',C') \tkzGetPoint{M'}
	\tkzDefMidPoint(M,M') \tkzGetPoint{H}
	\tkzLabelPoints[above left](A')
	\tkzLabelPoints[above right](C',H,M')
	\tkzLabelPoints[below right](C,B,B',N)
	\tkzLabelPoints[below left](A,I,M)
	%Nối các điểm
	\tkzDrawSegments(A,A' B,B' C,C' A,B B,C A',B' B',C' C',A' B,C' B,I C',I)
	\tkzDrawSegments[style=dashed](A,C M,M' H,N M,N M,B)
	\tkzMarkRightAngle[size=.4](A',A,B)
	\tkzMarkRightAngle[size=.3](C',I,B')
	\tkzMarkRightAngle[size=.3](B,M,A)
	\tkzMarkRightAngle[size=.3](N,H,M)
	\tkzMarkAngles[mark=|,arc=l,size=0.7cm](N,M,H)
	\end{tikzpicture}
	\end{center}
	\begin{enumerate}
	\item Gọi $I$ là trung điểm $A'B'$. Ta có $C'I\perp A'B'$ và $C'I\perp BB'$ nên $C'I\perp (ABB'A')$ tại $I$. Do đó $IB$ là hình chiếu vuông góc của $C'B$ lên $(ABB'A')$. Suy ra $(BC',(ABB'A'))=\widehat{C'BI}=30^{\circ}$. \\
	Trong tam giác $C'IB$, ta có $\tan\widehat{C'BI}=\dfrac{IC'}{IB}$, suy ra $IB=\dfrac{3a}{2}$. Khi đó $AA'=BB'=\sqrt{IB^2-IB'^2}=a\sqrt{2}$.
	\item Gọi $M'$ là trung điểm $A'C'$ và $H$ là trung điểm $MM'$.\\
	Ta có $BM\perp (ACC'A')$ (vì $BM\perp AC$ và $BM\perp AA'$) mà $HN\parallel BM$ nên $HN\perp (ACC'A')$ tại $H$. Suy ra $MH$ là hình chiếu vuông góc của $MN$ lên $(ACC'A')$.\\
	Do đó $(MN,(ACC'A'))=(MN,MH)=\widehat{NMH}$.
	Mà trong tam giác $NMH$, ta có $\tan\widehat{NMH}=\dfrac{HN}{MH}=\dfrac{\sqrt{6}}{2}$.\\
	Vậy $(MN,(ACC'A'))=\arctan\dfrac{\sqrt{6}}{2}\approx 51^{\circ}$.
	\end{enumerate}
	}
\end{bt}
%%%%%%%%%%%%%%%%%%%%
%==================
\begin{dang}{Tính góc giữa hai mặt phẳng, góc nhị diện}
\end{dang}
\subsubsection{Ví dụ minh hoạ}
\begin{vd}%[1C8T3-2]
	Trong các công trình xây dựng nhà ở, độ dốc mái được hiểu là độ nghiêng của mái khi hoàn thiện so vơi mặt phẳng nằm ngang. Khi thi công, mái nhà cần một độ nghiêng nhất định để đảm bảo thoát nước tốt tránh gây ra tình trạng đọng nươc hay thấm dột. Quan sát hình dưới và cho biết góc nhị diện nào phản ánh độ dốc của mái.
	\begin{center}
	\begin{tikzpicture}[scale=0.7, font=\footnotesize, line join=round, line cap=round, >=stealth]
	\tikzset{Icon-mattroi/.pic={
	\def\r{0.4}
	\fill(0,0)circle(\r);
	\draw[line width=1pt](0,0)circle(\r);
	\foreach \g in{1,...,10}{
	\draw[line width=1 pt](\g*36:1.2*\r)++(\g*36:0.1*\r)--++(\g*36:\r*0.25);
	}
	}}
	\tikzset{Icon-may/.pic={
	\def\r{0.35}
	\fill[white,line width=2.5*\r pt](-\r,0)--(\r,0)..controls++(0:\r) and ++(30:\r)..(0.8*\r,\r)..controls++(100:\r) and ++(80:\r)..(-0.75*\r,0.8*\r)..controls++(140:\r) and++(180:\r)..(-\r,0);
	}}
	\path
	(-0.5,0) coordinate (A)
	(1,3) coordinate (B)
	(-3,2) coordinate (C)
	(-3,0) coordinate (D)
	(3,0) coordinate (E)
	($(A)!4/5!(B)$) coordinate (F)
	(2.5,0) coordinate (G)
	(6,0) coordinate (A')
	(9,2) coordinate (B')
	(10,5) coordinate (C')
	($(A')+(C')-(B')$) coordinate (D')
	(13,2) coordinate (E')
	($(A')+(E')-(B')$) coordinate (F')
	;
	\fill[blue!30] (-3,-0.8)--(3.5,-0.8)--(3.5,4)--(-3,4)--cycle;
	\path
	(3,3.5)pic[yellow,scale=0.5]{Icon-mattroi}
	(-1,3.5)pic[scale=0.5,fill=white]{Icon-may}
	(1,3.2)pic[scale=0.5,fill=white]{Icon-may}
	;
	\draw (A)--(B)--(C)--(D)--cycle 
	(B)--(F)--(G)--(E)--cycle
	(G)--(2.5,-0.8) (A)--(-0.5,-0.8)
	(-3,-0.8)--(3.5,-0.8)--(3.5,4)--(-3,4)--cycle
	(0,-1)node[below]{a)}
	(9,-1)node[below]{b)}
	;
	\fill[blue!80!black] (A)--(B)--(C)--(D)--cycle ;
	\fill[blue!60!black] (B)--(F)--(G)--(E)--cycle ;
	\fill[yellow!50!black] (D)--(A)--(-0.5,-0.8)--(-3,-0.8)--cycle ;
	\fill[yellow!90!black] (-0.5,-0.8)--(A)--(F)--(G)--(2.5,-0.8)--cycle ;
	\draw[red,line width=2pt] (A)--(G)--(F) (A)node[below right]{$x$}
	(G)node[below left]{$O$}
	(F)node[below]{$y$}
	;
	\fill[blue!50] (A')--(B')--(C')--(D')--(A')--cycle;
	\fill[pink!50] (A')--(B')--(E')--(F')--cycle;
	\draw (A')--(B')node[pos=0.5,above]{$d$} (B')--(C')--(D')--(A')
	(A')--(B')--(E')--(F')--cycle
	; 
	\tkzMarkAngles[mark=](D',C',B' B',E',F')
	\tkzLabelAngle[pos=0.7](D',C',B'){$Q$}
	\tkzLabelAngle[pos=0.7](B',E',F'){$P$}
	\end{tikzpicture}
	\end{center}
	\loigiai{
	Giả sử nửa mặt phẳng $(P)$ (minh hoạ mặt phẳng nằm ngang) và nửa mặt phẳng $(Q)$ (minh hoạ mái nhà) cắt nhau theo giao tuyến $d$ (Hình b). Khi đó góc nhị diện có cạnh là đường thẳng $d$, hai mặt lần lượt là $(P)$ và $(Q)$ phản ánh độ dốc của mái. Độ dốc đó cũng được phản ánh bởi góc phẳng nhị diện $xOy$ của góc nhị diện trên (Hình a).
	}
\end{vd}
\begin{vd}%[1C8B3-2]
	\immini
	{
	Cho hình chóp $S.ABC$ có đáy $ABC$ là tam giác vuông cân tại $B$, $AB=a$, $SA \perp(ABC)$, $SA=a\sqrt{3}$ (Hình bên ). Tính số đo theo đơn vị độ của mỗi góc nhị diện sau:
	\begin{enumerate}
	\item $[B, SA, C]$;
	\item $[A, BC, S]$.
	\end{enumerate}
	}
	{
	\begin{tikzpicture}[scale=1, font=\footnotesize, line join=round, line cap=round, >=stealth]
	\path
	(0,0) coordinate (A)
	(4,0) coordinate (B)
	(2,-1) coordinate (C)
	(0,3) coordinate (S)
	;
	\draw (S)--(A)--(C)--(S)--(B)--(C);
	\draw[dashed] (A)--(B);
	\foreach \x/\g in {S/90,A/180,B/0,C/270} \fill[black] (\x) circle (1pt)+(\g:0.3) node{$\x$};
	\draw
	pic[draw,angle radius=5]{right angle=A--B--C};
	\end{tikzpicture}
	}
	\loigiai{
	\begin{enumerate}
	\item Vì $SA \perp(ABC)$ nên $SA \perp AB$, $SA \perp AC$. Do đó, góc $BAC$ là góc phẳng nhị diện của góc nhị diện $[B, SA, C]$. Do tam giác $ABC$ vuông cân tại $B$ nên $\widehat{BAC}=45^\circ$. Vậy số đo của góc nhị diện $[B, SA, C]$ bằng $45^\circ$.
	\item Vì $SA \perp(ABC)$ nên $SA \perp BC$. Như vậy $BC$ vuông góc với hai đường thẳng $AB$ và $SB$, suy ra góc $SBA$ là góc phẳng nhị diện của góc nhị diện $[A, BC, S]$.\\
	Trong tam giác vuông $SAB$, ta có
	$$
	\tan \widehat{SBA}=\dfrac{SA}{AB}=\dfrac{a \sqrt{3}}{a}=\sqrt{3}.
	$$
	Suy ra $\widehat{SBA}=60^\circ$. Vậy số đo của góc nhị diện $[A, BC, S]$ bằng $60^\circ$.
	\end{enumerate}
	}
\end{vd}
\begin{vd}%[1K7BO-4]
	Cho hình chóp $S.ABC$ có $SA\perp(ABC)$. Gọi $H$ là hình chiếu của $A$ trên $BC$.
	\begin{enumerate}
	\item Chứng minh rằng $(ASB) \perp(ABC)$ và $(SAH) \perp(SBC)$.
	\item Giả sử tam giác $ABC$ vuông tại $A$, $\widehat{ABC}=30^\circ$, $AC=a$, $SA=\dfrac{a\sqrt{3}}{2}$.\\
	Tính số đo của góc nhị diện $[S,BC,A]$.
	\end{enumerate}
	\loigiai{
	\immini{
	\begin{enumerate}
	\item Vì $SA\perp(ABC)$ và $SA\subset (ASB)$ nên $(ASB) \perp(ABC)$.\\
	Ta có $\heva{& BC\perp AH\\
	& BC\perp SA \ (\text{do } SA\perp (ABC)}\Rightarrow BC\perp (SAH))$.\\
	Lại có $BC\subset (SBC)$ nên $(SAH) \perp(SBC)$.
	\item Theo chứng minh trên, $BC\perp (SAH))\Rightarrow BC\perp SH$.\\
	Kết hợp với $BC\perp AH$, ta có góc $\widehat{SHA}$ là một góc phẳng của góc nhị diện $[S,BC,A]$.\\
	Vì tam giác $ABC$ vuông tại $A$ và $\widehat{ABC}=30^\circ$ nên $\widehat{ACB}=60^\circ$.
	\end{enumerate}
	}{
	\begin{tikzpicture}[scale=1, font=\footnotesize, line join=round, line cap=round, >=stealth]
	\tkzDefPoints{0/0/A,4/0/C,1.5/-1/B,0/4/x}
	\coordinate (S) at ($(A)+(x)$);
	\coordinate (H) at ($(B)!0.5!(C)$);
	\tkzDrawSegments(S,A S,B S,C A,B B,C S,H)
	\tkzDrawSegments[dashed](A,C A,H)
	\foreach \x/\g in {A/180,B/-90,C/0,S/90,H/-60} \fill[black] (\x) circle (1pt) +(\g:0.3)node{$\x$};
	\tkzMarkRightAngle(A,H,B)
	\end{tikzpicture}
	}
	\noindent
	Ta có $AH=AC\cdot \sin\widehat{ACB}=a\cdot\sin 60^\circ=\dfrac{a\sqrt{3}}{2}$.\\
	Tam giác $SAH$ vuông tại $A$ có $\tan\widehat{SHA}=\dfrac{SA}{AH}=\dfrac{\dfrac{a\sqrt{3}}{2}}{\dfrac{a\sqrt{3}}{2}}=1\Rightarrow\widehat{SHA}=45^\circ$.\\
	Vậy số đo của góc nhị diện $[S,BC,A]$ là $45^\circ$.
	} 
\end{vd}
\subsubsection{Bài tập rèn luyện}
\begin{bt}%[1C8Y3-2]
	\immini
	{
	Trong không gian cho bốn nửa mặt phẳng $(P)$, $(Q)$, $(R)$, $(S)$ cắt nhau theo giao tuyến $d$ (Hình bên). Hãy chỉ ra ba góc nhị diện có cạnh của góc nhị diện là đường thẳng $d$.
	}
	{
	\begin{tikzpicture}[scale=0.7, font=\footnotesize, line join=round, line cap=round, >=stealth]
	\path
	(0,0) coordinate (A)
	(0,4) coordinate (B)
	(1.5,5.8) coordinate (C)
	($(A)+(C)-(B)$) coordinate (D)
	(3,5.5) coordinate (E)
	($(A)+(E)-(B)$) coordinate (F)
	(4,4.5) coordinate (G)
	($(A)+(G)-(B)$) coordinate (H)
	(3.5,3.5) coordinate (I)
	($(A)+(I)-(B)$) coordinate (J)
	(intersection of C--D and B--E)coordinate(M)
	(intersection of E--F and B--G)coordinate(N)
	(intersection of A--H and I--J)coordinate(O)
	;
	\draw (A)--(B)node[left,pos=0.5]{$d$}
	(B)--(C)--(M) 
	(B)--(E)--(N)
	(B)--(G)--(H)--(O)
	(B)--(I)--(J)--(A)
	;
	\fill[violet!50] (A)--(B)--(C)--(D)--cycle;
	\fill[yellow!50] (A)--(B)--(E)--(F)--cycle;
	\fill[pink!50] (A)--(B)--(G)--(H)--cycle;
	\fill[blue!50] (A)--(B)--(I)--(J)--cycle;
	\tkzMarkAngles[mark=](B,C,M B,E,N B,G,H B,I,J)
	\draw[dashed] (A)--(D)--(M)
	(N)--(F)--(A) (A)--(F) (A)--(O)
	;
	\tkzLabelAngle[pos=0.5](B,C,M){$P$}
	\tkzLabelAngle[pos=0.5](B,E,N){$Q$}
	\tkzLabelAngle[pos=0.5](B,G,H){$R$}
	\tkzLabelAngle[pos=0.5](B,I,J){$S$}
	\end{tikzpicture}
	}
	\loigiai{
	Ba góc nhị diện có cạnh của góc nhị diện là đường thẳng $d$, hai mặt lần lượt là $(P)$ và $(Q)$; $(Q)$ và $(R)$; $(R)$ và $(S)$.
	}
\end{bt}
\begin{bt}%[1C8B3-2]
	Cho hình chóp $S.ABCD$ có $SA \perp(ABCD)$, đáy $ABCD$ là hình thoi cạnh $a$ và $AC=a$.
	\begin{enumerate}
	\item Tính số đo của góc nhị diện $[B, SA, C]$.
	\item Tính số đo của góc nhị diện $[B, SA, D]$.
	\item Biết $SA=a$, tính số đo của góc giữa đường thẳng $SC$ và mặt phẳng $(ABCD)$.
	\end{enumerate}
	\loigiai{
	\immini
	{
	\begin{enumerate}
	\item Vì $SA\perp (ABCD)$ nên $SA \perp AB$ và $SA\perp AC$.\\
	Suy ra số đo của góc nhị diện $[B, SA, C]$ bằng số đo của góc $\widehat{BAC}$.\\
	Vì tứ giác $ABCD$ là hình thoi cạnh $a$ và $AC=a$ nên các tam giác $ABC$, $ACD$ là các tam giác đều, suy ra $\widehat{BAC}=60^\circ$.\\
	Vậy góc nhị diện $[B, SA, C]$ có số đo bằng $60^\circ$.
	\item Vì $SA\perp (ABCD)$ nên $SA \perp AB$ và $SA\perp AD$.\\Suy ra số đo của góc nhị diện $[B, SA, D]$ bằng số đo của góc $\widehat{BAD}$.\\
	Mà $\widehat{BAD}=2\widehat{BAC}=120^\circ$.\\
	Vậy góc nhị diện $[B, SA, D]$ có số đo bằng $120^\circ$.
	\end{enumerate}
	}
	{
	\begin{tikzpicture}[scale=1, font=\footnotesize, line join=round, line cap=round, >=stealth]
	\path
	(0,3) coordinate (S)
	(0,0) coordinate (A)
	(4,0) coordinate (B)
	(3,-1) coordinate (C)
	($(A)+(C)-(B)$) coordinate (D)
	;
	\draw (S)--(D)--(C)--(B)--(S)--(C);
	\draw[dashed] (D)--(A)--(B) (S)--(A) (A)--(C);
	\foreach \x/\g in {S/90,A/135,B/0,C/270,D/270} \fill[black] (\x) circle (1pt)+(\g:0.3) node{$\x$};
	\end{tikzpicture}
	}
	\begin{enumerate}
	\setcounter{enumi}{2}
	\item Vì $SA\perp (ABCD)$ nên hình chiếu của $SC$ trên $(ABCD)$ là $AC$.\\
	Suy ra góc giữa $SC$ và $(ABCD)$ bằng góc giữa $SC$ và $AC$, bằng góc $\widehat{SCA}$.\\
	Ta có $\tan \widehat{SCA}=\dfrac{SA}{AC}=1\Rightarrow \widehat{SCA}=45^\circ$.\\
	Vậy góc giữa $SC$ và $(ABCD)$ bằng $45^\circ$.
	\end{enumerate}
	}
\end{bt}
%==================
\begin{bt}%[1K7BO-4]
	Cho hình chóp $S.ABCD$ có $SA \perp (ABCD)$, đáy $ABCD$ là hình thoi cạnh bằng $a$, $AC=a$, $SA= \dfrac{1}{2}a$. Gọi $O$ là giao điểm của hai đường chéo hình thoi $ABCD$ và $H$ là hình chiếu của $O$ trên $SC$.
	\begin{enumerate}
	\item Tính số đo của các góc nhị diện $[B, SA, D]$; $[S, BD, A]$; $[S, BD, C]$.
	\item Chứng minh rằng $\widehat{BHD}$ là một góc phẳng của góc nhị diện $[B,SC,D]$.
	\end{enumerate}
	\loigiai
	{
	\immini
	{
	\begin{enumerate}
	\item Vì $SA \perp (ABCD)$ nên $AB$ và $AD$ vuông góc với $SA$. Vậy $\widehat{BAD}$ là một góc phẳng của góc nhị diện $[B,SA,D]$. Hình thoi $ABCD$ có cạnh bằng $a$ và $AC=a$ nên các tam giác $ABC$, $ABD$ đều. Do đó $\widehat{BAD}=120^\circ$. Vậy số đo của góc nhị diện $[B,SA,D]$ bằng $120^\circ$.\\
	Vì $BD \perp AC$ và $BD \perp SA$ nên $BD \perp (SAC)$. Vậy $AC$ và $SO$ vuông góc với $BD$. Suy ra $\widehat{AOS}$ là một góc phẳng của góc nhị diện $[S,BD,A]$ và $\widehat{COS}$ là một góc phẳng của góc nhị diện $[S,BD,C]$. Tam giác $SAO$ vuông tại $A$ và có $SA=\dfrac{1}{2}a=AO$ nên $\widehat{AOS}=45^\circ$. Suy ra $\widehat{COS}= 180^\circ- \widehat{AOS}= 135^\circ$.
	\item Theo chứng minh trên, $BD \perp (SAC)$ nên $BD \perp SC$. Mặt khác, $OH \perp SC$ nên $SC \perp (BHD)$. Do đó $\widehat{BHD}$ là một góc phẳng của góc nhị diện $[B,SC,D]$. 
	\end{enumerate}
	}
	{
	\begin{tikzpicture}[>=stealth,line join=round,line cap=round,font=\footnotesize,scale=.91]
	\path 
	(0,0) coordinate (A)
	(5,0) coordinate (B)
	(-2,-2) coordinate (D)
	($(B)+(D)-(A)$) coordinate (C)
	($(A)+(90:4)$) coordinate (S)
	($(A)!.5!(C)$) coordinate (O)
	($(S)!.6!(C)$) coordinate (H)
	;
	\draw 
	(S)--(D)--(C)--(B)--(S)--(C) (D)--(H)--(B)
	;
	\draw[dashed]
	(S)--(A)--(D)--(B)--(A)--(C) (S)--(O)--(H)
	;
	\foreach \p/\g in {S/90, A/170, B/0, C/-90, D/-90, O/-90, H/45}
	\draw[fill=black] (\p) circle (1pt) node[shift=(\g:3mm)] {$\p$};
	\end{tikzpicture}
	}
	}
\end{bt}
%----------------
\begin{bt}%[1K7BO-2]%[1K7BO-4]
	Cho hình lập phương $ABCD.A'B'C'D'$ có cạnh bằng $a$.
	\begin{enumerate}
	\item Tính độ dài đường chéo của hình lập phương.
	\item Chứng minh rằng $(ACC'A') \perp(BDD'B')$.
	\item Gọi $O$ là tâm của hình vuông $ABCD$. Chứng minh rằng $\widehat{COC'}$ là một góc phẳng của góc nhị diện $[C,BD, C']$. Tính (gần đúng) số đo của các góc nhị diện $[C,BD,C]$, $[A,BD,C']$.
	\end{enumerate}
	\loigiai{
	\begin{enumerate}
	\item Độ dài đường chéo $AC'$
	\immini{
	\begin{eqnarray*}
	AC'&=&\sqrt{AC^2+AA'^2}\\
	&=&\sqrt{AB^2+AD^2+AA'^2}\\
	&=&\sqrt{a^2+a^2+a^2}\\
	&=&a\sqrt{3}.
	\end{eqnarray*}
	}{
	\begin{tikzpicture}[scale=1, font=\footnotesize, line join=round, line cap=round, >=stealth]
	\tkzDefPoints{0/0/A,3/0/D,-1.5/-1.2/B,0/2.5/x}
	\coordinate (C) at ($(D)-(A)+(B)$);
	\coordinate (A') at ($(A)+(x)$);
	\coordinate (B') at ($(B)+(x)$);
	\coordinate (C') at ($(C)+(x)$);
	\coordinate (D') at ($(D)+(x)$);
	\coordinate (O) at ($(A)!0.5!(C)$);
	\tkzDrawSegments(B,B' C,C' D,D' B,C C,D A',B' B',C' C',D' D',A' A',C' B',D')
	\tkzDrawSegments[dashed](A,A' A,B A,D A,C B,D O,C')
	\foreach \x/\g in {A/150,B/-150,C/-30,D/30,A'/150,B'/-150,C'/-30,D'/30,O/-90} \fill[black] (\x) circle (1pt) +(\g:0.3)node{$\x$};
	\end{tikzpicture}
	}
	\item Ta có $\heva{&AC\perp BD && (\text{do } ABCD \text{ là hình vuông}) \\
	& AC\perp BB' && (\text{tính chất của hình lập phương})}$ nên $AC\perp (BDD'B')$.\\
	Suy ra $(ACC'A') \perp(BDD'B')$.
	\item Ta có $\heva{& BD\perp AC\\ & BD\perp CC'}\Rightarrow BD\perp (ACC'A')\Rightarrow BD\perp C'O$.\\
	Vì $\heva{& BD\perp CO\\ & BD\perp C'O}$ nên $\widehat{COC'}$ là một góc phẳng của góc nhị diện $[C,BD, C']$.\\
	Tam giác $COC'$ vuông tại $C$ có $CC'=a$ và $OC=\dfrac{AC}{2}=\dfrac{a\sqrt{2}}{2}$ nên 
	\[\tan\widehat{COC'}=\dfrac{CC'}{CO}=\dfrac{a}{\dfrac{a\sqrt{2}}{2}}=\sqrt{2}\Rightarrow \widehat{COC'}\approx 54{,}7^\circ.\]
	Ta thấy $\widehat{AOC'}$ là một góc phẳng của góc nhị diện $[A,BD, C']$ và 
	\[\widehat{AOC'}=180^\circ-\widehat{COC'}\approx 180^\circ-54{,}7^\circ=125{,}3^\circ.\]
	Vậy số đo các góc nhị diện $[C,BD,C]$ và $[A,BD,C']$ tương ứng là $54{,}7^\circ$ và $125{,}3^\circ$.
	\end{enumerate}
	} 
\end{bt}
%%%%%%%%%%%%%%%%%%%%
% \subsection{Bài tập rèn luyện}\BTRL
% \begin{bt}%[1K8B2-5]
% 	Cho hình chóp $S.ABCD$ có đáy là hình vuông $ABCD$ cạnh $a$, $SA \perp(ABCD)$ và $SA=a \sqrt{2}$.
% 	\begin{enumerate}
% 	\item Tính góc giữa $SC$ và mặt phẳng $(ABCD)$.
% 	\item Tính góc giữa $BD$ và mặt phẳng $(SAC)$.
% 	\item Tìm hình chiếu của $S B$ trên mặt phẳng $(SAC)$.
% 	\end{enumerate}	
% 	\loigiai{
% 	\begin{center}
% 	\begin{tikzpicture}[scale=1,>=stealth, font=\footnotesize, line join=round, line cap=round]
% 	\path 
% 	(0,0)coordinate(B) 
% 	(5,0)coordinate(C) 
% 	(50:3)coordinate(A) 
% 	($(A)+(C)-(B)$) coordinate(D)
% 	($(A)+(0,4)$) coordinate(S) 
% 	(intersection of A--C and B--D) coordinate(O)
% 	;
% 	\draw (S)--(B)--(C)--(D)--cycle (C)--(S) 
% 	pic[draw,angle radius=2mm]{right angle=S--A--B}
% 	pic[draw,angle radius=2mm]{right angle=S--A--D}
% 	;
% 	\draw [dashed](O)--(S)--(A)--(B) (C)--(A)--(D)--(B)
% 	;
% 	\path
% 	(S)--(A)node[midway,left]{\small$a\sqrt{2}$}
% 	(B)--(A)node[midway,left]{$a$}
% 	;	
% 	\foreach \x/\g in {S/90,A/180,D/0,B/-90,C/-90,O/0}\fill[black] (\x) circle (1pt) +(\g:.3)node{$\x$};
% 	\end{tikzpicture}
% 	\end{center}	
% 	\begin{enumerate}
% 	\item Ta có $AC$ là hình chiếu của $SC$ lên mặt phẳng $(ABCD)$. Suy ra góc giữa đường thẳng $SC$ và mặt phẳng $(ABCD)$ là góc $\widehat{SCA}$.\\
% 	Xét tam giác $SAC$ vuông tại $A$. Ta có\\
% 	$\tan \widehat{SCA}=\dfrac{SA}{AC}=\dfrac{a\sqrt{2}}{a\sqrt{2}}=1\Rightarrow \widehat{SCA}=45^\circ$.
% 	\item Ta có $BD\perp AC$ (giả thiết).\\
% 	$SA\perp(ABCD)\Rightarrow SA\perp BD$.\\
% 	Do đó $BD\perp(SAC)$. Suy ra góc giữa $BD$ và mặt phẳng $(SAC)$ bằng $90^\circ$.
% 	\item Gọi $O$ là giao điểm của $BD$ và $AC$.\\
% 	Ta có $BD\perp (SAC)\Rightarrow BO\perp (SAC)$ nên $O$ là hình chiếu của điểm $B$ lên mặt phẳng $(SAC)$. Do đó hình chiếu của $SB$ lên mặt phẳng $(SAC)$ là $SO$.
% 	\end{enumerate}}
% \end{bt}
% \begin{bt}%%4.16:%[1K4YB-1]
% 	Cho hình chóp $S.ABC$ có $SA \perp(ABC)$, tam giác $ABC$ vuông tại $B$, $SA=AB=BC=a$.
% 	\begin{enumerate}
% 	\item Xác định hình chiếu của $A$ trên mặt phẳng $(S B C)$.
% 	\item Tính góc giữa $S C$ và mặt phẳng $(A B C)$.
% 	\end{enumerate}	
% 	\loigiai{
% 	\begin{center}
% 	\begin{tikzpicture}[scale=0.8,>=stealth, font=\footnotesize, line join=round, line cap=round]
% 	\path 
% 	(0,0)coordinate(A) 
% 	(5,0)coordinate(C) 
% 	(-60:3)coordinate(B) 
% 	($(A)+(0,4)$) coordinate(S)
% 	($(S)!(A)!(B)$) coordinate(H) 
% 	;
% 	\draw (S)--(A)--(B)--(C)--cycle (B)--(S) (A)--(H)
% 	pic[draw,angle radius=3mm]{right angle=S--A--B}
% 	pic[draw,angle radius=3mm]{right angle=S--A--C}
% 	pic[draw,angle radius=3mm]{right angle=A--B--C}
% 	pic[draw,angle radius=3mm]{right angle=A--H--B}
% 	;
% 	\draw [dashed](C)--(A) 
% 	;
% 	\path
% 	(S)--(A)node[midway,left]{$a$}
% 	(B)--(A)node[midway,left]{$a$}
% 	(C)--(B)node[midway,right]{$a$}
% 	;	
% 	\foreach \x/\g in {S/90,A/180,C/0,B/-90,H/0}\fill[black] (\x) circle (1pt) +(\g:.3)node{$\x$};
% 	\end{tikzpicture}
% 	\end{center}
% 	\begin{enumerate}
% 	\item Dựng $AH\perp SB$. Chứng minh được $AH\perp (SBC)$.\\
% 	Do đó $H$ là hình chiếu của điểm $A$ lên mặt phẳng $(SBC)$.
% 	\item Ta có $AC$ là hình chiếu của $SC$ lên mặt phẳng $(ABC)$. Suy ra góc giữa đường thẳng $SC$ và mặt phẳng $ABC$ là góc $\widehat{SCA}$.\\
% 	Xét tam giác $SAC$ vuông tại $A$ ta có\\
% 	$\tan\widehat{SCA}=\dfrac{SA}{AC}=\dfrac{a}{a\sqrt{2}}=\dfrac{1}{\sqrt{2}}\Rightarrow \widehat{SCA}=45^\circ$.
% 	\end{enumerate}}
% \end{bt}
% \begin{bt}%[1C8B3-1]
% 	Cho hình chóp $S.ABC$ có $SA \perp(ABC)$.
% 	\begin{enumerate}
% 	\item Tính góc giữa đường thẳng $SA$ và mặt phẳng $(ABC)$ theo đơn vị độ.
% 	\item Tính góc giữa đường thẳng $SB$ và mặt phẳng $(ABC)$ theo đơn vị độ, biết $SA=\sqrt{3}AB$.
% 	\end{enumerate}
% 	\loigiai{
% 	\immini
% 	{
% 	\begin{enumerate}
% 	\item Vì $SA \perp(ABC)$ nên góc giữa đường thẳng $SA$ và mặt phẳng $(ABC)$ bằng $90^\circ$.
% 	\item Vì $SA \perp(ABC)$ nên $AB$ là hình chiếu của $SB$ trên $(ABC)$. Suy ra góc giữa đường thẳng $SB$ và mặt phẳng $(ABC)$ bằng $\widehat{SBA}$.\\ Vì tan $\widehat{SBA}=\dfrac{SA}{AB}=\sqrt{3}$ nên $\widehat{SBA}=60^\circ$. \\
% 	Vậy góc giữa đường thẳng $SB$ và mặt phẳng $(ABC)$ bằng $60^\circ$.
% 	\end{enumerate}
% 	}
% 	{
% 	\begin{tikzpicture}[scale=1, font=\footnotesize, line join=round, line cap=round, >=stealth]
% 	\path
% 	(0,0) coordinate (A)
% 	(4,0) coordinate (B)
% 	(2,-1) coordinate (C)
% 	(0,3) coordinate (S)
% 	;
% 	\draw (S)--(A)--(C)--(S)--(B)--(C);
% 	\draw[dashed] (A)--(B);
% 	\foreach \x/\g in {S/90,A/180,B/0,C/270} \fill[black] (\x) circle (1pt)+(\g:0.3) node{$\x$};
% 	\end{tikzpicture}
% 	}
% 	}
% \end{bt}
% %%%%%%%%%
% %\subsection{ctst}
% \begin{bt}%[1C8B3-1]
% 	Cho hình chóp $S.ABCD$ có đáy $ABCD$ là hình vuông, hai đường thẳng $AC$ và $BD$ cắt nhau tại $O$, $SO \perp(ABCD)$, tam giác $SAC$ là tam giác đều.
% 	\begin{enumerate}
% 	\item Tính số đo của góc giữa đường thẳng $SA$ và mặt phẳng $(ABCD)$.
% 	\item Chứng minh rằng $AC \perp(SBD)$. Tính số đo của góc giữa đường thẳng $SA$ và mặt phẳng $(SBD)$.
% 	\item Gọi $M$ là trung điểm của cạnh $AB$. Tính số đo của góc nhị diện $[M, SO, D]$.
% 	\end{enumerate}
% 	\loigiai{\immini
% 	{
% 	\begin{enumerate}
% 	\item Vì $SO\perp (ABCD)$ nên hình chiếu của $SA$ trên $(ABCD)$ là $OA$.\\
% 	Suy ra góc giữa $SA$ và $(ABCD)$ bằng góc giữa $SA$ và $OA$, bằng góc $\widehat{SAO}$.\\
% 	Vì tam giác $SAC$ đều nên suy ra $\widehat{SAO}=60^\circ$.\\
% 	Vậy góc giữa $SA$ và $(ABCD)$ bằng $60^\circ$.
% 	\item Vì $SO \perp (ABCD)$ nên suy ra $SO \perp AC$. \quad (1)\\
% 	Mà $AC \perp BD$ (vì tứ giác $ABCD$ là hình vuông). \quad (2)\\
% 	Từ (1) và (2) suy ra $AC \perp(SBD)$.
% 	\item Vì $SO\perp (ABCD)$ nên suy ra $SO \perp OD$ và $SO \perp OM$.\\
% 	Suy ra số đo của góc nhị diện $[M, SO, D]$ bằng số đo của góc $\widehat{MOD}$.\\
% 	Vì tứ giác $ABCD$ là hình vuông nên $AC \perp BD$, suy ra $\widehat{AOD}=90^\circ$ và $\widehat{MOA}=45^\circ$.\\
% 	Khi đó $\widehat{MOD}=\widehat{AOD}+\widehat{MOA}=135^\circ$.\\
% 	Vậy góc nhị diện $[M, SO, D]$ có số đo bằng $135^\circ$.
% 	\end{enumerate}
% 	}
% 	{
% 	\begin{tikzpicture}[scale=1, font=\footnotesize, line join=round, line cap=round, >=stealth]
% 	\path
% 	(0,0) coordinate (A)
% 	(4,0) coordinate (B)
% 	(3,-1) coordinate (C)
% 	($(A)+(C)-(B)$) coordinate (D)
% 	($(A)!1/2!(C)$) coordinate (O)
% 	($(B)!1/2!(C)$) coordinate (Z)
% 	($(A)!1/2!(B)$) coordinate (M)
% 	($(O)!2.0!90:(Z)$) coordinate (S)
% 	;
% 	\draw (S)--(D)--(C)--(B)--(S)--(C);
% 	\draw[dashed] (S)--(A) (D)--(A)--(B) (S)--(O) (A)--(C) (B)--(D) (O)--(M);
% 	\foreach \x/\g in {S/90,A/135,B/0,C/270,D/270,O/270,M/80} \fill[black] (\x) circle (1pt)+(\g:0.3) node{$\x$};
% 	\end{tikzpicture}
% 	}}
% \end{bt}
% %-------------
% \begin{bt}%[1H3B3]
% 	Cho hình lăng trụ $ABCD.A'B'C'D'$ có đáy là hình thoi cạnh $a$, $\widehat{BAD}=60^{\circ}$. Hình chiếu vuông góc của $B'$ lên mặt đáy trùng với giao điểm của hai đường chéo của $ABCD$ và $BB'=a$. Tính góc giữa cạnh bên và mặt đáy.
% 	\loigiai{
% 	\begin{center}
% 	\begin{tikzpicture}[scale=0.55]
% 	%Định nghĩa các điểm
% 	\tkzDefPoints{0/0/A, 6/0/B}
% 	\path(A)++(40:2.5)coordinate(D)++(0:6)coordinate(C);
% 	\tkzDefMidPoint(A,C) \tkzGetPoint{O}
% 	\path(O)++(90:5)coordinate(B')++(180:6)coordinate(A');
% 	\path(A')++(40:2.5)coordinate(D')++(0:6)coordinate(C');
% 	\tkzLabelPoints[above left](A',D')
% 	\tkzLabelPoints[above right](C')
% 	\tkzLabelPoints[below right](C,B,B')
% 	\tkzLabelPoints[below left](A,O)
% 	\tkzLabelPoints[below](D)
% 	%Nối các điểm
% 	\tkzDrawSegments(A,B B,C C,C' C',D' D',A' A',B' B',C' A,A' B,B')
% 	\tkzDrawSegments[style=dashed](A,C B,D D,D' A,D D,C B',O)
% 	\tkzMarkRightAngle[size=.3](B',O,B)
% 	\tkzMarkRightAngle[size=.3](A,O,D)
% 	\tkzMarkAngles[mark=|,arc=l,size=0.7cm](B',B,O)
% 	\end{tikzpicture}
% 	\end{center}
% 	Gọi $O=AC\cap BD$. Ta có $B'O\perp (ABCD)$. Suy ra $BO$ là hình chiếu vuông góc của $BB'$ lên $(ABCD)$. Do đó $(BB',(ABCD))=(BB',BO)=\widehat{B'BO}$.\\
% 	Ta có $AB=AD=a$ và $\widehat{BAD}=60^{\circ}$ nên tam giác $ABD$ đều, suy ra $BD=a$ và $BO=\dfrac{a}{2}$.\\
% 	Mà trong tam giác $B'BO$, ta có $\cos\widehat{B'BO}=\dfrac{BO}{BB}=\dfrac{1}{2}$ nên $(BB',(ABCD))=60^{\circ}$.
% 	}
% \end{bt}
% \begin{bt}%[1H3G3]
% 	Cho hình chóp $S.ABCD$ có đáy $ABCD$ là hình vuông tâm $O$, cạnh $a$ và $SO\perp (ABCD)$. Mặt phẳng $(P)$ qua $A$ vuông góc với $SC$ cắt hình chóp theo một thiết diện có diện tích bằng $\dfrac{a^2}{2}$. Tính góc giữa $SC$ và $(ABCD)$.
% 	\loigiai{
% 	\begin{center}
% 	\begin{tikzpicture}
% 	\tkzDefPoints{0/0/O, -4/-1/A, 1/-1/B, 0/4/S}
% 	\tkzDefPointBy[symmetry = center O](A)
% 	\tkzGetPoint{C}
% 	\tkzDefPointBy[symmetry = center O](B)
% 	\tkzGetPoint{D}
% 	\coordinate (J) at ($(S)!0.45!(C)$);
% 	\tkzInterLL(S,O)(A,J) \tkzGetPoint{I}
% 	\tkzDefLine[parallel=through I](B,D) \tkzGetPoint{d}
% 	\tkzInterLL(d,I)(S,B) \tkzGetPoint{H}
% 	\tkzInterLL(d,I)(S,D) \tkzGetPoint{K}
% 	\tkzDrawSegments[dashed](A,D A,C D,C D,B S,D S,O H,K A,J A,K K,J)
% 	\tkzDrawSegments(A,B B,C S,A S,B S,C J,H A,H)
% 	\tkzMarkRightAngle[size=.2](B,O,C)
% 	\tkzMarkRightAngle[size=.2](A,J,C)
% 	\tkzMarkRightAngle[size=.2](A,I,K)
% 	\tkzLabelPoints[below](O,A,B)
% 	\tkzLabelPoints[above left](S,K)
% 	\tkzLabelPoints[right](C,I,H,J)
% 	\tkzLabelPoints[left](D)
% 	\tkzMarkAngles[mark=|,arc=l,size=0.7cm](S,C,A)
% 	%\tkzMarkSegments[mark=||](S,A S,B S,C S,D)
% 	\end{tikzpicture}
% 	\end{center}
% 	Gọi $J$ là hình chiếu vuông góc của $A$ lên $SC$ (1), $I$ là giao điểm của $SO$ và $AJ$.\\
% 	Qua $I$ kẻ đường thẳng song song $BD$ cắt $SB, SD$ lần lượt tại $H$ và $K$.\\
% 	Vì $BD\perp (SAC)$ nên $HK\perp (SAC)$, suy ra $HK\perp SC$ (2).\\
% 	Từ (1) và (2) suy ra $SC\perp (AHJK)$ nên tứ giác $AHJK$ là thiết diện của hình chóp khi cắt bởi $(P)$.\\
% 	Vì $HK\perp (SAC)$ nên $HK\perp AI$, do đó $S_{AHIK}=\dfrac{1}{2}AJ.KH=\dfrac{a^2}{2}$.\\
% 	Nhận thấy rằng $OC$ là hình chiếu vuông góc của $SC$ lên $(ABCD)$ nên $(SC,(ABCD))=\widehat{SCO}=\widehat{SAC}=\alpha$.\\
% 	Lúc đó $AJ=AC\sin\alpha=a\sqrt{2}\sin\alpha, SO=OC.\tan\alpha=\dfrac{a\sqrt{2}}{2}\tan\alpha$.\\
% 	Ta có $\Delta SOC\sim \Delta SJI$ nên $\widehat{SIJ}=\widehat{SCO}=\alpha$, suy ra $\widehat{AIO}=\widehat{SIJ}=\alpha$, từ đó ta có $OI=OA.\cot\alpha=\dfrac{a\sqrt{2}}{2}\cot\alpha$.\\
% 	Trong tam giác $SBD$, ta có $\dfrac{HK}{BD}=\dfrac{SI}{SO}=1-\dfrac{OI}{SO}=1-\cot^2\alpha$, suy ra $KH=DB\left(1-\cot^2\alpha\right)=a\sqrt{2}\left(1-\cot^2\alpha\right)$.\\
% 	Vậy $$S_{AHJK}=\dfrac{1}{2}a^2=a^2\sin\alpha\left(1-\cot^2\alpha\right)\Leftrightarrow 4\sin^2\alpha-\sin\alpha-2=0\Rightarrow \sin\alpha=\dfrac{1+\sqrt{33}}{8}\Rightarrow \alpha\approx 57,5^{\circ}.$$
% 	}
% \end{bt}
% \begin{bt}%[1H3G3]
% 	Cho hình hộp chữ nhật $ABCD.A'B'C'D'$ có đáy $ABCD$ là hình vuông. Tìm góc lớn nhất giữa đường thẳng $BD'$ và mặt phẳng $(BDC')$.
% 	\loigiai{
% 	\begin{center}
% 	\begin{tikzpicture}[xscale=0.75,yscale=0.5]
% 	%Định nghĩa các điểm
% 	\tkzDefPoints{0/0/A, 6/0/B}
% 	\path(A)++(40:2.5)coordinate(D)++(0:6)coordinate(C);
% 	\tkzDefMidPoint(A,C) \tkzGetPoint{O}
% 	\path(A)++(90:8)coordinate(A')++(0:6)coordinate(B');
% 	\path(A')++(40:2.5)coordinate(D')++(0:6)coordinate(C');
% 	\tkzDefMidPoint(B,D') \tkzGetPoint{I}
% 	\coordinate (H) at ($(C')!0.7!(O)$);
% 	\tkzLabelPoints[above left](A',D',B')
% 	\tkzLabelPoints[above right](C',I)
% 	\tkzLabelPoints[below right](C,B,H)
% 	\tkzLabelPoints[below left](A,O)
% 	\tkzLabelPoints[below](D)
% 	%Nối các điểm
% 	\tkzDrawSegments(A,B B,C C,C' C',D' D',A' A',B' B',C' A,A' B,B' B,C')
% 	\tkzDrawSegments[style=dashed](A,C B,D D,D' A,D D,C C',O B,D' I,O I,H B,H D,C')
% 	\tkzMarkRightAngle[size=.3](I,H,C')
% 	\tkzMarkRightAngle[size=.3](A,O,B)
% 	%\tkzMarkAngles[arc=l,size=1cm](H,B,I)
% 	\end{tikzpicture}
% 	\end{center}
% 	Gọi $O=AC\cap BD$, $I$ là trung điểm $DD'$ và $H$ là hình chiếu vuông góc của $I$ lên $OC'$. Ta có $IH\perp OC'$ (1).\\
% 	Mà $BD\perp AC$ và $BC\perp AA'$ nên $BD\perp (ACC'A')$, suy ra $BD\perp IH$ (2).\\
% 	Từ (1) và (2), sy ra $IH\perp (BDC')$ tại $H$ nên $BH$ là hình chiếu vuông góc của $BI$ lên $(BDC')$. Suy ra $(D'B,(BDC'))=(BI,BH)=\widehat{IBH}$.\\
% 	Đặt $AB=a, AA'=b$. Ta có $BD'=\sqrt{AB^2+AD^2+DD'^2}=\sqrt{2a^2+b^2}$, suy ra $IB=\dfrac{\sqrt{2a^2+b^2}}{2}$.\\
% 	Ta có $\Delta OHI \sim \Delta C'CO$ nên $HI=\dfrac{OI.CO}{C'O}=\dfrac{ab}{2\sqrt{2b^2+a^2}}=\dfrac{1}{2\sqrt{\dfrac{2}{a^2}+\dfrac{1}{b^2}}}$. Từ đó suy ra $$\sin\widehat{IBH}=\dfrac{IH}{BI}=\dfrac{1}{\sqrt{2\left(\dfrac{a^2}{b^2}+\dfrac{b^2}{a^2}\right)+5}}\le \dfrac{1}{3}.$$
% 	Đẳng thức xảy ra khi $a=b$.\\
% 	Khi đó góc lớn nhất cần tìm có số đó là $\arcsin\dfrac{1}{3}\approx 19^{\circ}28'$.
% 	}
% \end{bt}
% %================
% \begin{bt}%[1K7KO-8]
% 	Hai mái nhà trong hình bên là hai hình chữ nhật ($AOPD$ và $BOPC$). Giả sử $AB=4{,}8\mathrm{~m}$, $OA=2{,}8 \mathrm{~m}$, $OB=4\mathrm{~m}$.
% 	\begin{center}
% 	\begin{tikzpicture}[scale=1, font=\footnotesize, line join=round, line cap=round, >=stealth]
% 	\tkzDefPoints{0/0/P,-2/-1.5/D,1/-1/C,5/0/x}
% 	\coordinate (O) at ($(P)+(x)$);
% 	\coordinate (A) at ($(D)+(x)$);
% 	\coordinate (B) at ($(C)+(x)$);
% 	\tkzDrawSegments(P,D O,A O,B A,B P,O D,A)
% 	\tkzDrawSegments[dashed](B,C C,D P,C)
% 	\foreach \x/\g in {A/-120,B/45,O/90,P/90,D/-120,C/45} \fill[black] (\x) circle (1pt) +(\g:0.3)node{$\x$};
% 	\end{tikzpicture}
% 	\end{center}
% 	\begin{enumerate}
% 	\item Tính (gần đúng) số đo của góc nhị diện tạo bởi hai nửa mặt phẳng tương ứng chứa hai mái nhà.
% 	\item Chứng minh rằng mặt phẳng $(OAB)$ vuông góc với mặt đất phẳng. Lưu ý: Đường giao giữa hai mái (đường nóc) song song với mặt đất.
% 	\item Điểm $A$ ở độ cao (so với mặt đất) hơn điểm $B$ là $0{,}5 \mathrm{~m}$. Tính (gần đúng) góc giữa mái nhà (chứa $OB$) so với mặt đất.
% 	\end{enumerate}
% 	\loigiai{
% 	\begin{enumerate}
% 	\item Ta có $\heva{& AO\perp PO \\ & BO\perp PO}$ nên $\widehat{AOB}$ là một góc phẳng của góc nhị diện $[A,PO,B]$.\\
% 	Theo định lí cos, ta có
% 	\[\cos\widehat{AOB}=\dfrac{OA^2+OB^2-AB^2}{2\cdot OA\cdot OB}=\dfrac{2{,}8^2+4^2-4{,}8^2}{2\cdot 2{,}8\cdot 4}=\dfrac{1}{28}.\]
% 	Suy ra $\widehat{AOB}\approx 88^\circ$.
% 	\item Theo chứng minh trên, ta có $PO\perp (OAB)$.\\
% 	Mà $PO$ song với mặt đất nên $(OAB)$ vuông góc với mặt đất phẳng.
% 	\item Gọi $H$ là giao điểm của phương thẳng đứng đi qua $A$ và phương ngang đi qua $B$.
% 	\immini{
% 	Theo giả thiết, $AH=0{,}5$ m.\\
% 	Góc giữa mái nhà chứa $OB$ so với mặt đất chính là góc $\widehat{OBH}$.\\
% 	Ta có $\sin\widehat{ABH}=\dfrac{AH}{AB}=\dfrac{0{,}5}{4{,}8}=\dfrac{5}{48}$.\\
% 	Suy ra $\widehat{ABH}\approx 5{,}98^\circ$.\\
% 	Theo định lí cos, ta có
% 	}{
% 	\begin{tikzpicture}[scale=1, font=\footnotesize, line join=round, line cap=round, >=stealth]
% 	\tkzDefPoints{0/0/A,2/2/O,5/-1/B}
% 	\coordinate (a) at ($(A)+(0,-2)$);
% 	\coordinate (H) at ($(A)!(B)!(a)$);
% 	\tkzDrawSegments(O,A O,B A,B)
% 	\tkzDrawSegments[dashed](A,H H,B)
% 	\foreach \x/\g in {A/180,B/9,O/90,H/180} \fill[black] (\x) circle (1pt) +(\g:0.4)node{$\x$};
% 	\tkzMarkRightAngle(A,H,B)
% 	\end{tikzpicture}
% 	}
% 	\[\cos\widehat{OBA}=\dfrac{OB^2+AB^2-OA^2}{2\cdot OB\cdot AB}=\dfrac{4^2+4{,}8^2-2{,}8^2}{2\cdot 4\cdot 4{,}8}=\dfrac{13}{16}.\]
% 	Suy ra $\widehat{OBA}\approx 35{,}66^\circ$.\\
% 	Vậy $\widehat{OBH}=\widehat{OBA}+\widehat{ABH}\approx 35{,}66^\circ+5{,}98^\circ=41{,}64^\circ$.
% 	\end{enumerate}
% 	} 
% \end{bt}
% \begin{bt}%[1C8T3-2]
% 	\immini
% 	{
% 	Trong hình bên, máy tính xách tay đang mở gợi nên hình ảnh của một góc nhị diện. Ta gọi số đo góc nhị diện đó là độ mở của màn hình máy tính. Tính độ mở của màn hình máy tính theo đơn vị độ, biết tam giác $ABC$ có độ dài các cạnh là $AB=AC=30$ cm và $BC=30\sqrt{3}$ cm.
% 	}
% 	{
% 	\begin{tikzpicture}[scale=0.7, font=\footnotesize, line join=round, line cap=round, >=stealth]
% 	\path
% 	(0,0) coordinate (A)
% 	(-1,5) coordinate (B)
% 	(3,4) coordinate (D)
% 	(4,1) coordinate (E)
% 	(3.5,-1) coordinate (C)
% 	(7,0.2) coordinate (F)
% 	($(A)+(0.5,0)$) coordinate (A')
% 	($(A)!10/11!(E)$) coordinate (M)
% 	($(M)+(0.5,0)$) coordinate (M')
% 	($(A)!1/2!(C)$) coordinate (N)
% 	($(N)+(0.5,0)$) coordinate (N')
% 	($(E)!0.65!(F)$) coordinate (O)
% 	($(O)-(0.3,0)$) coordinate (O')
% 	;
% 	\draw[black!50,line width=2pt] (A)--(C)--(F);
% 	\draw[line width=3pt] (A)--(B)--(D)--(E)--(A)
% 	;
% 	\draw(F)--(E) (B)--(C);
% 	\fill[black!80] (O')--(N')--(A')--(M')--cycle;
% 	\foreach \x/\g in {A/160,B/180,C/270} \fill[black] (\x) circle (2pt)+(\g:0.6) node{$\x$};
% 	\end{tikzpicture}
% 	}
% 	\loigiai{
% 	Góc nhị diện đã cho có số đo bằng số đo của góc $\widehat{BAC}$.\\
% 	Trong tam giác $ABC$ ta có $\cos \widehat{BAC}=\dfrac{AB^2+AC^2-BC^2}{2\cdot AB \cdot AC}=\dfrac{30^2+30^2-\left(30\sqrt{3}\right)^2}{2\cdot 30 \cdot 30}=-\dfrac{1}{2}$.\\
% 	$\Rightarrow \widehat{BAC}=120^\circ$.\\
% 	Vậy độ mở của màn hình máy tính là $120^\circ$.
% 	}
% \end{bt}
% \begin{bt}[Bài toán đo chiều cao của tháp khi không thể lên tới đỉnh tháp.]%[1C8B3-3]
% 	\immini
% 	{
% 	Để ước lượng chiều cao của tháp khi không thể lên tới đỉnh tháp, người ta đo góc giữa tia nắng chiếu qua đỉnh tháp và mặt đất, đo chiều dài của bóng tháp trên mặt đất, từ đó ước lượng được chiều cao của tháp. Giả sử khi tia nắng tạo với mặt đất một góc $40^\circ$, chiều dài của bóng tháp là $80$ m (Hình bên). Tính chiều cao của tháp theo đơn vị mét (làm tròn kết quả đến hàng phần mười).
% 	}
% 	{
% 	\begin{tikzpicture}[scale=0.7, font=\footnotesize, line join=round, line cap=round, >=stealth]
% 	\tikzset{Icon-mattroi/.pic={
% 	\def\r{0.4}
% 	\fill(0,0)circle(\r);
% 	\draw[line width=1pt](0,0)circle(\r);
% 	\foreach \g in{1,...,10}{
% 	\draw[line width=1 pt](\g*36:1.2*\r)++(\g*36:0.1*\r)--++(\g*36:\r*0.25);
% 	}
% 	}}
% 	\path
% 	(0,6) coordinate (I)
% 	(-7,0.3) coordinate (M)
% 	(0,0.3) coordinate (O)
% 	(1,7)pic[yellow,scale=0.5]{Icon-mattroi}
% 	;
% 	\def\originpath{(0.5,0)--(0.5,1)--(0.4,1.2)--(0.4,4)--(0.5,4)--(0.5,5)--(0.2,5.4)--(0.2,5.7)--(0,6)}
% 	\fill[brown] (0.2,5.7)--(0,6)--(-0.2,5.7)--cycle
% 	(0.2,5.4)--(0.5,5)--(-0.5,5)--(-0.2,5.4)--cycle
% 	;
% 	\fill[brown!50] (0.2,5.7)--(0.2,5.4)--(-0.2,5.4)--(-0.2,5.7)--cycle
% 	(0.5,5)--(0.5,4)--(-0.5,4)--(-0.5,5)--cycle
% 	(0.5,0)--(0.5,1)--(0.4,1.2)--(0.4,4)--(-0.4,4)--(-0.4,1.2)--(-0.5,1)--(-0.5,0)--cycle
% 	;
% 	\fill[brown] (-0.3,4.8)--(0.3,4.8)--(0.3,4.2)--(-0.3,4.2)--cycle
% 	;
% 	\fill[brown!70]
% 	(0.4,1.2)--(0.4,4)--(-0.4,4)--(-0.4,1.2)--cycle (0.5,0)--(0.5,1)--(-0.5,1)--(-0.5,0)--cycle;
% 	\fill[white] (0,4.5)circle (6pt);
% 	\fill[gray!50] (-0.5,0.8)--(-1,0.8)--(-1.2,0.7)--(-4,0.7)--(-4.2,0.8)--(-5,0.8)--(-5.5,0.6)--(-6,0.5)--(-7,0.3)--(-6,0.1)--(-5,-0.1)--(-4.5,-0.1)--(-3.5,-0.1)--(-3,0.1)--(-1,0.1)--(-0.8,0)--(-0.5,0)--(-0.5,0.8);
% 	\draw[brown,line width=2pt] (-0.2,4)--(-0.2,0) (0,4)--(0,0) (0.2,4)--(0.2,0) (-0.5,0)--(0.5,0) ;
% 	\draw[blue,line width=2pt] (I)--(M) (M)--(O)node[pos=0.5,below]{$80$ m};
% 	\draw[brown] \originpath;
% 	% đối xứng qua trục tung
% 	\draw[brown,xscale=-1] \originpath;
% 	\tkzMarkAngle[mark=,blue](O,M,I)
% 	\tkzLabelAngle[pos=1.5,blue](O,M,I){$40^\circ$}
% 	\end{tikzpicture}
% 	}
% 	\loigiai{
% 	\immini
% 	{
% 	Xét hình bên, độ dài $AH$ chỉ chiều cao của tháp, độ dài $OH$ chỉ chiều dài của bóng tháp, độ lớn của góc $AOH$ chỉ số đo góc giữa tia nắng và mặt đất. Vì tam giác $OAH$ vuông tại $H$ nên
% 	$$
% 	AH=OH \cdot \tan \widehat{AOH}=80 \cdot \tan 40^\circ \approx 67{,}1(\mathrm{m}).
% 	$$
% 	}
% 	{
% 	\begin{tikzpicture}[scale=0.55, font=\footnotesize, line join=round, line cap=round, >=stealth]
% 	\path
% 	(0,6) coordinate (A)
% 	(-7,0.3) coordinate (O)
% 	(0,0.3) coordinate (H);
% 	\draw (A)--(O) (A)--(H) (H)--(O)node[pos=0.5,below]{$80$ m};
% 	\draw
% 	pic[draw,angle radius=5]{right angle=A--H--O};
% 	\tkzMarkAngle[mark=](H,O,A)
% 	\tkzLabelAngle[pos=1.5](H,O,A){$40^\circ$}
% 	\foreach \x/\g in {A/90,H/0,O/180} \fill[black] (\x) circle (2pt)+(\g:0.6) node{$\x$};
% 	\end{tikzpicture}
% 	}
% 	}
% \end{bt}
% \begin{bt}%[1C8K3-3]
% 	\immini
% 	{
% 	Trong hình bên, xét các góc nhị diện có góc phẳng nhị diện tương ứng là $\widehat{B}$, $\widehat{C}$, $\widehat{D}$, $\widehat{E}$ trong cùng mặt phẳng. Lục giác $ABCDEG$ nằm trong mặt phẳng đó có $AB=GE=2$ m, $BC=DE$, $\widehat{A}=\widehat{G}=90^\circ$, $\widehat{B}=\widehat{E}=x$, $\widehat{C}=\widehat{D}=y$. Biết rằng khoảng cách từ $C$ và $D$ đến $AG$ là $4$ m, $AG=12$ m, $CD=1$ m. Tìm $x$, $y$ (làm tròn kết quả đến hàng đơn vị theo đơn vị độ).
% 	}
% 	{
% 	\begin{tikzpicture}[scale=0.7, font=\footnotesize, line join=round, line cap=round, >=stealth]
% 	\path
% 	(0,0) coordinate (A)
% 	(0,2) coordinate (B)
% 	(4,4) coordinate (C)
% 	(5,4) coordinate (D)
% 	(9,2) coordinate (E)
% 	(9,0) coordinate (G)
% 	($(C)+(-1.5,2)$) coordinate (F)
% 	($(D)+(1.5,2)$) coordinate (I)
% 	($(C)!1!(B)$) coordinate (J)
% 	($(B)!0.5!(C)$) coordinate (L)
% 	($(J)+(-0.2,0)$) coordinate (J')
% 	($(L)+(-0.2,0)$) coordinate (L')
% 	($(L')+(-0.7,0.5)$) coordinate (M)
% 	($(J')+(-0.7,0)$) coordinate (M')
% 	($(D)!1!(E)$) coordinate (O)
% 	($(D)!0.5!(E)$) coordinate (P)
% 	($(O)+(0.2,0)$) coordinate (O')
% 	($(P)+(0.2,0)$) coordinate (P')
% 	($(P')+(0.7,0.5)$) coordinate (N)
% 	($(O')+(0.7,0)$) coordinate (N')
% 	($(C)+(-0.5,-1)$) coordinate (Q)
% 	($(D)+(0.5,-1)$) coordinate (R)
% 	($(Q)+(0,-1.5)$) coordinate (Q')
% 	($(R)+(0,-1.5)$) coordinate (R')
% 	($(Q)!0.5!(R)$) coordinate (S)
% 	($(Q')!0.5!(R')$) coordinate (S')
% 	;
% 	\draw 
% 	(C)--(F) (D)--(I) 
% 	(-1,0)--(-1,6)--(10,6)--(10,0)--cycle
% 	;
% 	\fill[yellow] (-1,0)--(-1,6)--(F)--(C)--(B)--(A)--cycle
% 	(I)--(D)--(E)--(G)--(10,0)--(10,6)--cycle
% 	;
% 	\fill[yellow!70] (F)--(C)--(D)--(I)--cycle;
% 	\fill[yellow!50] (A)--(B)--(C)--(D)--(E)--(G)--cycle;
% 	\fill[white] (J')--(L')--(M)--(M')--cycle
% 	(O')--(P')--(N)--(N')--cycle
% 	(Q)--(R)--(R')--(Q')--cycle
% 	;
% 	\draw[line width=3pt] (Q)--(R)--(R')--(Q')--cycle (S)--(S');
% 	\foreach \x/\g in {A/150,B/220,C/90,D/90,E/-40,G/30} \fill[black] (\x) circle (1pt)+(\g:0.3) node{$\x$};
% 	\draw (A)node[above right]{$90^\circ$}
% 	(B)node[right]{$x$}
% 	(C)node[below]{$y$}
% 	(D)node[below]{$y$}
% 	(E)node[left]{$x$}
% 	(G)node[above left]{$90^\circ$}
% 	(J')--(L')--(M)--(M')--cycle
% 	(O')--(P')--(N)--(N')--cycle
% 	;
% 	\draw[red,line width=2pt] (A)--(B)--(C)--(D)--(E)--(G)--(A);
% 	\end{tikzpicture}
% 	}
% 	\loigiai{
% 	\immini
% 	{
% 	Gọi $H$, $K$ lần lượt là hình chiếu vuông góc của $C$, $D$ trên $AG$; $I$ là hình chiếu vuông góc của $B$ trên $CH$.\\
% 	Ta có $HI=AB=2$ m, mà $CH=4$ m $\Rightarrow CI=2$ m.\\
% 	Mặt khác $HK=CD=1$ m $\Rightarrow AH=KG=5{,}5$ m$=BI$.\\
% 	Trong tam giác $CBI$ vuông tại $I$ ta có
% 	$$\tan \widehat{CBI}=\dfrac{CI}{BI}=\dfrac{2}{5{,}5}\Rightarrow \widehat{CBI}\approx 20^\circ \Rightarrow \widehat{BCI}\approx 70^\circ.$$
% 	$\Rightarrow x\approx 90^\circ+20^\circ=110^\circ$ và $y\approx 90^\circ+70^\circ=160^\circ$.\\
% 	Vậy $x \approx 110^\circ $ và $y \approx 160^\circ$.
% 	}
% 	{
% 	\begin{tikzpicture}[scale=0.7, font=\footnotesize, line join=round, line cap=round, >=stealth]
% 	\path
% 	(0,0) coordinate (A)
% 	(0,2) coordinate (B)
% 	(4,4) coordinate (C)
% 	(5,4) coordinate (D)
% 	(9,2) coordinate (E)
% 	(9,0) coordinate (G)
% 	(4,0) coordinate (H)
% 	(5,0) coordinate (K)
% 	(4,2) coordinate (I)
% 	;
% 	\draw (A)--(B)--(C)--(D)--(E)--(G)--(A)
% 	;
% 	\foreach \x/\g in {A/150,B/220,C/90,D/90,E/-40,G/30,H/135,K/45,I/135} \fill[black] (\x) circle (1pt)+(\g:0.3) node{$\x$};
% 	\draw (A)node[above right]{$90^\circ$}
% 	(B)node[below right]{$x$}
% 	(C)node[below right]{$y$}
% 	(D)node[below right]{$y$}
% 	(E)node[below left]{$x$}
% 	(G)node[above left]{$90^\circ$}
% 	(C)--(H) (D)--(K) (B)--(I)
% 	;
% 	\end{tikzpicture}
% 	}
% 	}
% \end{bt}
% \begin{bt}%[1C8K3-2]
% 	Cho hình chóp $S.ABC$ có $SA \perp(ABC)$. Gọi $\alpha$ là số đo của góc nhị diện $[A, BC, S]$. Chứng minh rằng tỉ số diện tích của hai tam giác $ABC$ và $SBC$ bằng $\cos \alpha$.
% 	\loigiai{
% 	\immini
% 	{
% 	Gọi $H$ là hình chiếu vuông góc của $A$ trên $BC$. Ta có $BC\perp AH$.\quad(1)\\
% 	Ta lại có $SA\perp (ABC)\Rightarrow BC\perp SA$.\quad(2)\\
% 	Từ (1) và (2) suy ra $BC \perp (SAH)\Rightarrow BC\perp SH$.\quad(3)\\
% 	Từ (1) và (3) suy ra $\alpha=\widehat{SHA}$.\\
% 	Trong tam giác $SAH$ vuông tại $A$ ta có $\cos \alpha=\dfrac{AH}{SH}$.\quad (4)\\
% 	Mặt khác ta có $\dfrac{S_{\triangle ABC}}{S_{\triangle SBC}}=\dfrac{\dfrac{1}{2}AH\cdot BC}{\dfrac{1}{2}SH\cdot BC}=\dfrac{AH}{SH}$.\quad(5)\\
% 	Từ (4) và (5) suy ra tỉ số diện tích của hai tam giác $ABC$ và $SBC$ bằng $\cos \alpha$.
% 	}
% 	{
% 	\begin{tikzpicture}[scale=1, font=\footnotesize, line join=round, line cap=round, >=stealth]
% 	\path
% 	(0,0) coordinate (A)
% 	(4,0) coordinate (B)
% 	(2,-1) coordinate (C)
% 	(0,3) coordinate (S)
% 	($(C)!1/3!(B)$) coordinate (H)
% 	;
% 	\draw (S)--(A)--(C)--(S)--(B)--(C) (S)--(H);
% 	\draw[dashed] (A)--(B) (A)--(H);
% 	\foreach \x/\g in {S/90,A/180,B/0,C/270,H/-20} \fill[black] (\x) circle (1pt)+(\g:0.3) node{$\x$};
% 	\end{tikzpicture}
% 	}
% 	}
% \end{bt}