
%=========================
\begin{dang}{Tính góc, cạnh, đường cao, diện tích các hình thông dụng}
	
\end{dang}

\begin{vd}%[Nguyễn Trần Phong]%[1K7BO-7]
	Cho hình chóp đều $S.ABC$ có $AB=2a$, góc hợp bởi mặt bên và mặt đáy bằng $60^\circ$. Tính tổng diện tích các mặt bên của hình chóp.	
	\loigiai{
		\immini{
			Gọi $M$ là trung điểm $BC$ và $O$ là trọng tâm tam giác $ABC$. Khi đó do tam giác $ABC$ đều nên $BC \perp AM$.\\
			Ta có $\heva{& BC \perp AM \\& BC \perp SO \text{ (do } SO \perp (ABC))\\& SO \cap AM = O} \\ \Rightarrow BC \perp (SAM) \Rightarrow BC \perp SM.$\\
			Khi đó $\heva{& BC = (SBC) \cap (ABC) \\& AM \subset (ABC), \, AM \perp BC \\& SM \subset (SBC), \, SM \perp BC} \\ \Rightarrow \left((SBC), (ABC) \right)= (AM, SM) =\widehat{SMA} =60^\circ$.\\
			Ta có $OM = \dfrac{AB\sqrt{3}}{6}= \dfrac{a\sqrt{3}}{3}$.\\
			Xét tam giác $SOM$ vuông tại $O$ có $SM =\dfrac{OM}{\cos 60^\circ}= \dfrac{2a\sqrt{3}}{3}.$ \\
			Vậy $S_{\triangle SBC}=\dfrac{1}{2} \cdot SM \cdot BC =\dfrac{1}{2} \cdot \dfrac{2a\sqrt{3}}{3}\cdot 2a= \dfrac{2a^2\sqrt{3}}{3}$.\\
			Vậy tổng diện tích các mặt bên của hình chóp là $2a^2\sqrt{3}.$}{\begin{tikzpicture}[scale=0.6, font=\footnotesize, line join=round, line cap=round, >=stealth]
				%\draw[color=gray,dash pattern=on 1pt off 1pt,xstep=1.0cm,ystep=1.0cm] (-5.2,-5.2) grid (5.2,5.2);
				\coordinate (A) at (0,0);
				\coordinate (C) at (5,0);
				\coordinate (B) at (2,-1.5);
				\coordinate (M) at ($(B)!1/2!(C)$);
				\coordinate (O) at ($(A)!2/3!(M)$);
				\coordinate (S) at ($(O)+ (90:5)$);
				\draw (A)--(B)--(C)--(S)--cycle (C)--(S)--(B) (S)--(M); 
				\draw[dashed] (C)--(A)--(M) (S)--(O);
				\foreach \p/\r in {A/180,B/-90, C/0,S/90, M/-90, O/-80}
				\fill (\p) circle (1.5pt) node[shift={(\r:3mm)}]{$\p$};
	\end{tikzpicture}}}
\end{vd}

\begin{vd}%[Nguyễn Trần Phong]%[1K7BO-7]
	Cho hình chóp cụt tam giác đều $ABC.A'B'C'$ có cạnh đáy lớn bằng $a$, cạnh đáy nhỏ bằng $\dfrac{a}{2}$, cạnh bên bằng $2a$. Tính chiều cao của hình chóp cụt.
	\loigiai{
		\immini{Gọi $O$, $O'$ lần lượt là trọng tâm tam giác $ABC$ và $A'B'C'$.\\
			Trong hình thang $AOO'A'$, gọi $H$ là hình chiếu của $A'$ lên $AO$.\\
			Khi đó $AH= AO- HO = \dfrac{a\sqrt{3}}{3}- \dfrac{a\sqrt{3}}{6}=\dfrac{a\sqrt{3}}{6}.$\\
			Suy ra $OO'= A'H =\sqrt{A'A^2- AH^2 }= \sqrt{4a^2 -\dfrac{a^2}{12}}=\dfrac{a\sqrt{141}}{6}.$}{\begin{tikzpicture}[scale=0.7, font=\footnotesize, line join=round, line cap=round, >=stealth]
				%\draw[color=gray,dash pattern=on 1pt off 1pt,xstep=1.0cm,ystep=1.0cm] (-5.2,-5.2) grid (5.2,5.2);
				\coordinate (A) at (0,0);
				\coordinate (C) at (5,0);
				\coordinate (B) at (2,-1.5);
				\coordinate (S) at ($(O)+ (90:7)$);
				\coordinate (A') at ($(S)!2/3!(A)$);
				\coordinate (B') at ($(S)!2/3!(B)$);
				\coordinate (C') at ($(S)!2/3!(C)$);
				\coordinate (M) at ($(B)!1/2!(C)$);
				\coordinate (M') at ($(B')!1/2!(C')$);
				\coordinate (O) at ($(A)!2/3!(M)$);
				\coordinate (O') at ($(A')!2/3!(M')$);
				\coordinate (H) at ($(A')-(O')+(O)$);
				\draw (A)--(B)--(C)--(C')--(B')--(A')--cycle (B)--(B') (A')--(C'); 
				\draw[dashed] (A)--(C) (O)--(O') (A')--(H) (A')--(O') (A)--(O);
				\foreach \p/\r in {A/180,B/-90, C/0,A'/90, B'/-120, C'/80, O/-90, O'/0, H/-50}
				\fill (\p) circle (1.5pt) node[shift={(\r:3mm)}]{$\p$};
	\end{tikzpicture}}}	
\end{vd}



\begin{vd}%[Nguyễn Trần Phong]%[1K7BO-7]
	Cho hình chóp $S.ABC$ có $SA \perp (ABC)$, tam giác $ABC$ đều cạnh $a$. Gọi $M$ là điểm trên cạnh $SA$ sao cho góc tạo bởi $(MBC)$ và $(ABC)$ bằng $60^\circ$. Tính diện tích tam giác $MBC$.
	\loigiai{
		\immini{Gọi $H$ là hình chiếu của $A$ lên $BC$. Do tam giác $ABC$ đều nên $H$ cũng là trung điểm $BC$.\\
			Lại do $BC \perp AM$ (do $AM \perp (ABC)$) nên $BC \perp MH$.\\
			Khi đó $\heva{&BC = (ABC) \cap (MBC) \\& AH \subset  (ABC), \, AH \perp BC \\& MH \subset (MBC), \, MH \perp BC} \\ \Rightarrow \left((ABC), (MBC)) \right)= \left(AH, MH \right)=\widehat{MHA}=60^\circ$.\\
			Xét tam giác $MHA$ vuông tại $A$ có $MH =\dfrac{MH}{\cos 60^\circ} = a\sqrt{3}$.\\
			Vậy $S_{MBC} = \dfrac{1}{2} \cdot BC \cdot MH = \dfrac{1}{2} \cdot a \cdot a\sqrt{3}= \dfrac{a^2 \sqrt{3}}{2}$.}{\begin{tikzpicture}[scale=0.6, font=\footnotesize, line join=round, line cap=round, >=stealth]
				%\draw[color=gray,dash pattern=on 1pt off 1pt,xstep=1.0cm,ystep=1.0cm] (-5.2,-5.2) grid (5.2,5.2);
				\coordinate (A) at (0,0);
				\coordinate (C) at (5,0);
				\coordinate (B) at (2,-1.5);
				\coordinate (S) at ($(A)+ (90:5)$);
				\coordinate (M) at ($(S)!1/2!(A)$);
				\coordinate (H) at ($(B)!1/2!(C)$);
				\draw (A)--(B)--(C)--(S)--cycle (C)--(S)--(B) (S)--(M)--(B); 
				\draw[dashed] (A)--(C)--(M) (A)--(H)--(M);
				\foreach \p/\r in {A/180,B/-90, C/0,S/90, M/180, H/0}
				\fill (\p) circle (1.5pt) node[shift={(\r:3mm)}]{$\p$};
		\end{tikzpicture}}
	}	
\end{vd}



\begin{vd}%[Nguyễn Trần Phong]%[1K7BO-7]
	Cho hình chóp $S.ABC$	có $SA \perp (ABC)$, tam giác $ABC $ đều cạnh $a$, $SA=a$. Gọi $I$ là trung điểm $BC$ và $(P)$ là mặt phẳng đi qua $A$ đồng thời vuông góc với $SI$. Mặt phẳng này cắt $SB$, $SC$ lần lượt tại $E$ và $F$. Tính diện tích tam giác $AEF$.
	\loigiai{
		\immini{Trong  $(SBC)$, gọi $K= SI \cap EF$.\\
			Do  $I$ là trung điểm $BC$, tam giác $ABC$ đều nên $AI \perp BC$.\\
			Ta có $\heva{& BC \perp AI\\ & BC \perp SA \text{ (do } SA \perp (ABC))\\& AI \cap BC = I} \Rightarrow BC \perp (SAI).$\\
			Mà $BC \subset (SBC)$ nên $(SBC) \perp (SAI)$.\\
			Ta lại có $\heva{& (SAI) \perp (SBC)\\ & (P) \perp (SBC) \text{ (do } SI \perp (P), SI \subset (SBC)) \\ & (SAI) \cap (P) = AK } \Rightarrow AK \perp (SBC) \\ \Rightarrow AK \perp EF$.\\
			Suy ra $S_{AEF}=\dfrac{1}{2} \cdot EF \cdot AK$.\\
			Ta có $AI=\dfrac{a\sqrt{3}}{2}$.\\
			Xét tam giác $SAI$ vuông tại $A$ có $AK$ là đường cao nên $$AK =\dfrac{SA \cdot AI}{\sqrt{SA^2 + AI^2}}= \dfrac{a\sqrt{21}}{7}.$$
			$SA^2 = SK \cdot SI \Rightarrow \dfrac{SA^2}{SI^2}=\dfrac{SK}{SI}=\dfrac{EF}{BC}$.\\
			Suy ra $EF= BC \cdot \dfrac{SA^2}{SA^2+ AI^2}= \dfrac{4}{7}a$.\\
			Vậy $S_{AEF}= \dfrac{1}{2} \cdot \dfrac{4}{7}a\cdot \dfrac{a\sqrt{21}}{7}= \dfrac{2\sqrt{21}a^2}{49}$.}{\begin{tikzpicture}[scale=0.6, font=\footnotesize, line join=round, line cap=round, >=stealth]
				%\draw[color=gray,dash pattern=on 1pt off 1pt,xstep=1.0cm,ystep=1.0cm] (-5.2,-5.2) grid (5.2,5.2);
				\coordinate (A) at (0,0);
				\coordinate (C) at (5,0);
				\coordinate (B) at (2,-1.5);
				\coordinate (S) at ($(A)+ (90:5)$);
				\coordinate (I) at ($(B)!1/2!(C)$);
				\coordinate (E) at ($(S)!2/5!(B)$);
				\coordinate (F) at ($(S)!2/5!(C)$);
				\coordinate (K) at (intersection of S--I and E--F) ;
				\draw (A)--(B)--(C)--(S)--cycle (C)--(S)--(B) (S)--(I) (E)--(F); 
				\draw[dashed] (A)--(C) (A)--(I) (A)--(K);
				\foreach \p/\r in {A/180,B/-90, C/0,S/90, I/0, E/180, F/0, K/-20}
				\fill (\p) circle (1.5pt) node[shift={(\r:3mm)}]{$\p$};
		\end{tikzpicture}}
	}
\end{vd}


\begin{vd}%[Nguyễn Trần Phong]%[1K7KKO-7]
	Cho tam giác $ABC$ đều, tâm $O$ có đường cao $AH=3a$. Trên đường thẳng vuông góc với mặt phẳng $(ABC)$ tại $O$ lấy điểm $S$ sao cho $SO=a\sqrt{2}$. Một điểm $M$ lưu động trên đoạn $OH$ với $OM=x$, $(0<x<a)$. Mặt phẳng $(\alpha)$ qua $M$ và vuông góc với $OH$.
	\begin{listEX}
		\item Chứng minh $BC \perp SA$ và $SA \perp (SBC)$.
		\item Dựng thiết diện tạo bởi mặt phẳng $(\alpha)$ với hình chóp $S.ABC$. Thiết diện này là hình gì?
		\item Tính diện tích thiết diện tìm được ở câu b. Định $x$ để thiết diện có diện tích lớn nhất?
	\end{listEX}
	\loigiai
	{\begin{listEX}
			\item Chứng minh $BC \perp SA$ và $SA \perp (SBC)$.
			\immini{Ta có $SO\perp (ABC)$ nên $SO\perp BC$.\quad $(1)$\\
				Vì $AH$ là đường cao của tam giác $ABC$ nên $AH\perp BC$.\quad $(2)$\\
				Từ $(1)$ và $(2)$ suy ra $BC\perp (SAH)\Rightarrow BC\perp SA$.\\
				Trong tam giác đều $ABC$ ta tính được
				$$AB=\dfrac{2AH}{\sqrt{3}}=2\sqrt{3}a,
				AO=\dfrac{2}{3}AH=2a.$$
				Tam giác $SAO$ vuông tại $O$ nên $SA=\sqrt{SO^2+AO^2}=\sqrt{6}a$.\\
				Vì $OA=OB$ nên $SA=SB$.\\
			}{\begin{tikzpicture}[scale=1, font=\footnotesize, line join=round, line cap=round,>=stealth]
					\path (0,0)coordinate (A) (5,0) coordinate (C) (1.5,-1.3) coordinate (B) ($(B)!0.5!(C)$) coordinate (H) ($(A)!2/3!(H)$) coordinate (O) ($(O)+(0,3.3)$) coordinate (S) ($(O)!0.3!(H)$) coordinate (M) ($(M)+(S)-(O)$) coordinate (k) (intersection of M--k and S--H) coordinate (K) ($(M)+(B)-(H)$) coordinate (e) (intersection of M--e and A--B) coordinate (E) (intersection of M--e and A--C) coordinate (F) ($(E)+(K)-(M)$) coordinate (i) (intersection of K--i and S--B) coordinate (I) (intersection of K--i and S--C) coordinate (J);
					\draw (S)--(A)--(B)--(C)--cycle (H)--(S)--(B) (E)--(I)--(J);
					\draw[dashed] (H)--(A)--(C) (S)--(O) (E)--(F)--(J) (M)--(K);
					\foreach \p/\g in {A/180,B/-90,C/0,S/90,O/210,H/-30,E/180,F/45,I/180,J/30,K/60,M/-90} \fill[black] (\p) circle(1pt)+(\g:0.3) node{$\p$};
					\draw pic[draw, angle radius=0.2cm]{right angle= S--O--A} pic[draw, angle radius=0.2cm]{right angle= A--H--B};
			\end{tikzpicture}}\noindent
			Để ý $SA^2+SB^2=AB^2$ nên $SA\perp SB$.\\
			Kết hợp với $BC\perp SA$ ở trên thì $SA\perp (SBC)$.
			\item Dựng thiết diện tạo bởi mặt phẳng $(\alpha)$ với hình chóp $S.ABC$. Thiết diện này là hình gì?\\
			Ta có $SO\perp (ABC)$ và $AH\subset (ABC)$ nên $SO\perp AH$.\\
			Vì $(\alpha)\perp AH$ nên $(\alpha)\parallel SO$.\\
			Tương tự, $(\alpha)\parallel BC$.\\
			Ta có $\heva{& M\in (\alpha)\cap (SAH) \\ & (\alpha)\parallel SO\subset (SAH)}\Rightarrow (\alpha)\cap (SAH)=MK\parallel SO$ với $K\in SH$.\\
			Gọi $E$, $F$ lần lượt là giao điểm của $AB$, $AC$ với $(\alpha)$. Vì $(\alpha)\parallel BC$ nên $(\alpha)\cap (ABC)=EF\parallel BC$ và $M\in EF$.\\
			Gọi $I$, $J$ lần lượt là giao điểm của $SB$, $SC$ với $(\alpha)$. Vì $(\alpha)\parallel BC$ nên $(\alpha)\cap (SBC)=IJ\parallel BC$ và $K\in IJ$.\\
			Ta có $IJ\parallel BC\parallel EF$ nên thiết diện tạo bởi $(\alpha)$ và hình chóp $S.ABC$ là hình thang $EFJI$.
			\item Tính diện tích thiết diện tìm được ở câu b. Định $x$ để thiết diện có diện tích lớn nhất?
			Ta có $MK\parallel SO$ và $SO\perp (ABC)$ nên $MK\perp (ABC)\Rightarrow MK\perp EF$.\\
			Tiếp đến, $IJ\parallel EF$ nên $MK\perp IJ$. Do đó $MK$ là đường cao của hình thang $EFJI$.\\
			Vì $\triangle ABC$ đều nên $OH=\dfrac{AH}{3}=a$.\\
			Ta có $\dfrac{MK}{SO}=\dfrac{HM}{HO}=\dfrac{HO-OM}{HO}=\dfrac{a-x}{a}\Rightarrow MK=\dfrac{a-x}{a}\cdot SO=\sqrt{2}(a-x)$.\\
			Từ $EF\parallel BC$ ta có $$\dfrac{EF}{BC}=\dfrac{AM}{AH}=\dfrac{AO+OM}{AH}=\dfrac{2a+x}{3a}\Rightarrow EF=\dfrac{2a+x}{3a}\cdot BC=\dfrac{2\sqrt{3}(2a+x)}{3}.$$
			Từ $IJ\parallel BC$ và $SO\parallel MK$ ta có $$\dfrac{IJ}{BC}=\dfrac{SK}{SH}=\dfrac{OM}{OH}=\dfrac{x}{a}\Rightarrow IJ=\dfrac{x}{a}\cdot BC=2\sqrt{3}x.$$
			Vậy diện tích hình thang $EFJI$ là $S_{EFJI}=\dfrac{(EF+IJ)MK}{2}=\dfrac{2\sqrt{6}(a+2x)(a-x)}{3}$.\\
			Ta có $(a+2x)(a-x)=2\left(\dfrac{a}{2}+x\right)(a-x)\leq 2\cdot\left(\dfrac{\dfrac{a}{2}+x+a-x}{2}\right)^2=\dfrac{9a^2}{8}$.\\
			Dấu \lq\lq  $=$\rq\rq\ xảy ra khi $\dfrac{a}{2}+x=a-x\Leftrightarrow 2x=\dfrac{a}{2}\Leftrightarrow x=\dfrac{a}{4}$.\\
			Vậy với $x=\dfrac{a}{4}$ thì diện tích $S_{EFJI}$ đạt giá trị lớn nhất.
		\end{listEX}
	}	
\end{vd}

\subsubsection{Bài tập trắc nghiệm}

\Opensolutionfile{ans}[ans/ans-1K7-25-Dang8]

\begin{ex} %[Nguyễn Trần Phong]%[1K7BO-7]
	Cho hình chóp $S.ABC$ có mặt bên $SAB$ là tam giác đều và nằm trong mặt phẳng vuông góc với $(ABC)$, $AB=2a$, tam giác $ABC$ đều. Tính diện tích tam giác $SBC$.
	\choice
	{\True $\dfrac{\sqrt{15}a^2}{2} $}
	{$\dfrac{\sqrt{15}a^2}{4} $}
	{$\dfrac{\sqrt{15}a^2}{3} $}
	{$ a^2 \sqrt{15}$}
	\loigiai{\immini{
			Gọi $H$ là trung điểm $AB$, do tam giác $SAB$ đều nên $SH \perp AB$.\\
			Ta có $\heva{& (SAB) \perp (ABC)\\& (SAB) \cap (ABC)= AB\\& SH \subset (SAB), \, SH \perp AB } \\ \Rightarrow SH \perp (ABC).$\\
			Ta có $SH= CH =\dfrac{AB\sqrt{3}}{2}=a\sqrt{3}.$\\
			Xét tam giác $SHC$ vuông cân tại $H$ có $SC= SH \sqrt{2} =a \sqrt{6}$.\\
			$p=\dfrac{SB+ SC+ BC}{2}= \dfrac{4+ \sqrt{6}}{2} \cdot a.$\\
			Vậy $S_{\triangle SBC}=\sqrt{p \cdot \left(p-  BC\right)^2\cdot \left(p- SC \right)}=\dfrac{\sqrt{15}a^2}{2}$.}{\begin{tikzpicture}[scale=0.6, font=\footnotesize, line join=round, line cap=round, >=stealth]
				%\draw[color=gray,dash pattern=on 1pt off 1pt,xstep=1.0cm,ystep=1.0cm] (-5.2,-5.2) grid (5.2,5.2);
				\coordinate (A) at (0,0);
				\coordinate (C) at (5,0);
				\coordinate (B) at (2,-1.5);
				\coordinate (H) at ($(A)!1/2!(B)$);
				\coordinate (S) at ($(H)+ (90:5)$);
				\draw (A)--(B)--(C)--(S)--cycle (C)--(S)--(B) (S)--(H); 
				\draw[dashed] (A)--(C)--(H);
				\foreach \p/\r in {A/180,B/-90, C/0,S/90, H/180}
				\fill (\p) circle (1.5pt) node[shift={(\r:3mm)}]{$\p$};
	\end{tikzpicture}}}
\end{ex}

\begin{ex}%[Nguyễn Trần Phong]%[1K7BO-7]
	Cho hình chóp cụt tứ giác đều $ABCD.A'B'C'D'$ có đáy lớn $ABCD$ có cạnh bằng $a$, đáy nhỏ có cạnh là $b$, chiều cao $OO'=h$ với $O$, $O'$ lần lượt là tâm của đáy lớn, đáy bé. Tínhd độ dài $CC'$.
	\choice
	{\True $\sqrt{h^2 + \dfrac{(a-b)^2}{2}} $}
	{$\sqrt{h^2 + \dfrac{(a-b)^2}{3}} $}
	{$ \sqrt{h^2 + (a-b)^2}$}
	{$\sqrt{h^2 + \dfrac{(a-b)^2}{5}} $}
	\loigiai{\immini{ Trong hình thang $BDD'B'$, gọi $H$ là hình chiếu của $D'$ lên $BD$.
			\\Khi đó ta có $D'H = OO'=h$, $HD= OD- OH = \dfrac{(a- b )\sqrt{2}}{2}$.\\
			Vậy $DD'=\sqrt{D'H^2 + HD}= \sqrt{h^2 + \dfrac{(a-b)^2}{2}}.$}{\begin{tikzpicture}[scale=0.6, font=\footnotesize, line join=round, line cap=round, >=stealth]
				%\draw[color=gray,dash pattern=on 1pt off 1pt,xstep=1.0cm,ystep=1.0cm] (-5.2,-5.2) grid (5.2,5.2);
				\coordinate (A) at (0,0);
				\coordinate (B) at (-3,-2);
				\coordinate (C) at ($(B)+(5,0)$);
				\coordinate (D) at ($(C)-(B)+(A)$);
				\coordinate (O) at ($(A)!0.5!(C)$);
				\coordinate (S) at ($(O)+ (90:7)$);
				\coordinate (A') at ($(S)!2/3.5!(A)$);
				\coordinate (B') at ($(S)!2/3.5!(B)$);
				\coordinate (C') at ($(S)!2/3.5!(C)$);
				\coordinate (D') at ($(S)!2/3.5!(D)$);
				\coordinate (O') at ($(A')!0.5!(C')$);
				\coordinate (H) at ($(D')-(O')+(O)$);
				\draw (B')--(B)--(C)--(D) (C')--(C) (A')--(B')--(C')--(D')--cycle (D)--(D')--(B'); 
				\draw[dashed] (D)--(A)--(B) (A')--(A) (B)--(D) (O)--(O') (D')--(H);
				\foreach \p/\r in {A/180,B/-90, C/0,D/0,O/0, A'/90, B'/90, C'/-20, D'/90, O'/90, H/-120}
				\fill (\p) circle (1.5pt) node[shift={(\r:3mm)}]{$\p$};
		\end{tikzpicture}}
	}
\end{ex}




\begin{ex} %[Nguyễn Trần Phong]%[1K7BO-7]
	Cho hình chóp $S.ABC$ có $SA \perp (ABC)$, tam giác $ABC$ có $AB=a$, $AC=2a$, $\widehat{BAC}= 60^\circ$. Gọi $M$ là điểm trên cạnh $SA$ sao cho góc hợp bởi $(MBC)$ và $(ABC)$ bằng $45^\circ$, tính diện tích tam giác $MBC$.
	\choice
	{$ \sqrt{6}a^2$}
	{$\dfrac{3 \sqrt{6} a^2}{2}$}
	{$\dfrac{\sqrt{6} a^2} {3}$}
	{\True $\dfrac{\sqrt{6} a^2}{2} $}
	\loigiai{\immini{
			Gọi $H$ là hình chiếu của $A$ lên $BC$.\\
			Lại do $BC \perp AM$ (do $AM \perp (ABC)$) nên $BC \perp MH$.\\
			Khi đó $\heva{&BC = (ABC) \cap (MBC) \\& AH \subset  (ABC), \, AH \perp BC \\& MH \subset (MBC), \, MH \perp BC} \\ \Rightarrow \left((ABC), (MBC)) \right)= \left(AH, MH \right)=\widehat{MHA}=45^\circ$.\\
			Ta có $S_{ABC}=\dfrac{1}{2}\cdot AB \cdot AC \cdot \sin \widehat{BAC} =\dfrac{\sqrt{3}}{2}a^2$.\\
			Xét tam giác $MAH$ vuông tại $A$ có $\cos 45^\circ =\dfrac{AH}{MH}= \dfrac{AH \cdot BC}{ MH \cdot BC}=\dfrac{S_{ABC}}{S_{MBC}}$.\\
			Suy ra $$S_{MBC }= \dfrac{  S_{ABC}}{\cos 45^\circ}= \dfrac{\sqrt{6} a^2}{2}.$$}{\begin{tikzpicture}[scale=0.6, font=\footnotesize, line join=round, line cap=round, >=stealth]
				%\draw[color=gray,dash pattern=on 1pt off 1pt,xstep=1.0cm,ystep=1.0cm] (-5.2,-5.2) grid (5.2,5.2);
				\coordinate (A) at (0,0);
				\coordinate (C) at (5,0);
				\coordinate (B) at (2,-1.5);
				\coordinate (S) at ($(A)+ (90:5)$);
				\coordinate (M) at ($(S)!1/2!(A)$);
				\coordinate (H) at ($(B)!1/2!(C)$);
				\draw (A)--(B)--(C)--(S)--cycle (C)--(S)--(B) (S)--(M)--(B); 
				\draw[dashed] (A)--(C)--(M) (A)--(H)--(M);
				\foreach \p/\r in {A/180,B/-90, C/0,S/90, M/180, H/0}
				\fill (\p) circle (1.5pt) node[shift={(\r:3mm)}]{$\p$};
	\end{tikzpicture}}}
\end{ex}

\begin{ex} %[Nguyễn Trần Phong]%[1K7BO-7]
	Cho hình lăng trụ đứng $ABC.A'B'C'$ có đáy là tam giác cân tại $A$, $AB= a$, $\widehat{BAC}=120^\circ$. Trên cạnh $CC'$ lấy điểm $I$ sao cho góc tạo bởi $(ABC)$ và $(ABI)$ có cosin bằng $\dfrac{\sqrt{30}}{10}$. Tính diện tích tam giác $ABI$. 
	\choice
	{$a^2 \sqrt{10} $}
	{\True $\dfrac{a^2 \sqrt{10}}{4} $}
	{$ \dfrac{a^2 \sqrt{10}}{2}$}
	{$ \dfrac{a^2 \sqrt{10}}{8}$}
	\loigiai{\immini{
			Ta có $S_{ABC}=\dfrac{1}{2}\cdot AB^2 \cdot \sin 120^\circ= \dfrac{a^2 \sqrt{3}}{4}$.\\
			Gọi $H$ là hình chiếu của $C$ lên $AB$.\\
			Lại do $AB \perp CI$ nên $AB \perp IH$.\\
			Ta có $\heva{& AB = (ABC) \cap (ABI)\\& CH \subset (ABC), \, CH \perp AB \\& IH \subset (ABI), \, IH \perp AB}\\
			\Rightarrow \left((ABC), (IAB) \right)=\left(CH, IH \right)= \widehat {CHI}$.\\
			Ta có $\cos \widehat{CHI}=\dfrac{CH}{IH}=\dfrac{CH \cdot AB}{ IH \cdot AB}= \dfrac{S_{ABC}}{S_{ABI}}$.\\
			Suy ra $$ S_{ABI}=\dfrac{S_{ABC}}{\cos \widehat{CHI}}=\dfrac{a^2 \sqrt{10}}{4}. $$}{\begin{tikzpicture}[scale=0.5, font=\footnotesize, line join=round, line cap=round, >=stealth]
				%\draw[color=gray,dash pattern=on 1pt off 1pt,xstep=1.0cm,ystep=1.0cm] (-5.2,-5.2) grid (5.2,5.2);
				\coordinate (A) at (0,0);
				\coordinate (C) at (5,0);
				\coordinate (B) at (2,-2);
				\coordinate (A') at ($(A) +(90:5)$);
				\coordinate (B') at ($(B)+(90:5)$);
				\coordinate (C') at ($(C)+(90:5)$);
				\coordinate (I) at ($(C)!2/3!(C')$);
				\coordinate (H) at ($(A)!1/2!(B)$);
				%\draw pic[draw,angle radius=2mm] {right angle = S--H--K}; 
				%\draw pic[draw,angle radius=2mm] {right angle = H--I--K};
				%\draw pic[draw,angle radius=2mm] {right angle = H--K--D};
				%\coordinate (I) at (intersection of O--N and A--B) {};
				\draw (A)--(B)--(C)  (A')--(B')--(C')--cycle (A)--(A') (B)--(B') (C)--(C') (I)--(B); 
				\draw[dashed] (A)--(C) (I)--(A) (C)--(H)--(I);
				\foreach \p/\r in {A/180,B/-90, C/0,A'/90, B'/-30, C'/90, I/0, H/180}
				\fill (\p) circle (1.5pt) node[shift={(\r:3mm)}]{$\p$};
	\end{tikzpicture}} }
\end{ex}

\begin{ex} %[Nguyễn Trần Phong]%[1K7BO-7]
	Cho hình chóp $S.ABCD$ có $SA \perp (ABCD)$, $ABCD$ là hình chữ nhật tâm $O$, $AB=a$, $AD=2a$. Biết $SA \perp (ABCD)$, $SA= a$, mặt phẳng $(P)$ đi qua $SO$ và vuông góc với $(SAD)$, cắt $AD$ và $BC$ lần lượt tãi $E$, $F$. Tính diện tích tam giác $SEF$.
	\choice
	{\True $\dfrac{a^2\sqrt{2}}{2} $}
	{$ a^2$}
	{$\dfrac{a^2}{2} $}
	{$\sqrt{2} a^2 $}
	\loigiai{\immini{
			Ta có $SA \perp (ABCD)$, $SA \subset (SAD)$ nên $(SAD) \perp (ABCD).$\\
			$\heva{& (SEF) \perp (SAD) \\& (ABCD) \perp (SAD) \\ & (SEF ) \cap (ABCD) = EF} \Rightarrow EF \perp (SAD) \Rightarrow EF \perp SE$ hay tam giác $SEF$ vuông tại $E$.\\
			Vậy $S_{SEF}= \dfrac{1}{2}\cdot EF \cdot SE =\dfrac{1}{2} \cdot a \cdot \sqrt{a^2 + a^2}= \dfrac{a^2 \sqrt{2}}{2}.$}{\begin{tikzpicture}[scale=0.5, font=\footnotesize, line join=round, line cap=round, >=stealth]
				%\draw[color=gray,dash pattern=on 1pt off 1pt,xstep=1.0cm,ystep=1.0cm] (-5.2,-5.2) grid (5.2,5.2);
				\coordinate (A) at (0,0);
				\coordinate (B) at (-2,-3);
				\coordinate (C) at ($(B)+(5,0)$);
				\coordinate (D) at ($(C)-(B)+(A)$);
				\coordinate (S) at ($(A)+ (90:5)$);
				\coordinate (O) at ($(A)!0.5!(C)$);
				\coordinate (E) at ($(A)!0.5!(D)$);
				\coordinate (F) at ($(B)!0.5!(C)$);
				\draw (S)--(B)--(C)--(D)--cycle (S)--(C) (S)--(F); 
				\draw[dashed] (D)--(A)--(B) (S)--(A) (F)--(E)--(S);
				\foreach \p/\r in {A/180,B/-90, C/0,D/0,S/90,O/0, E/-90, F/-90}
				\fill (\p) circle (1.5pt) node[shift={(\r:3mm)}]{$\p$};
	\end{tikzpicture}}}
\end{ex}

\begin{ex} %[Nguyễn Trần Phong]%[1K7BO-7]
	Cho hình chóp tam giác đều $S.ABC$ có cạnh bên bằng $a\sqrt{3}$ và tạo với đáy một góc bằng $60^\circ$. Tính diện tích mặt đáy hình chóp.
	\choice
	{$\dfrac{3a^2\sqrt{3}}{16} $}
	{$ \dfrac{a^2\sqrt{3}}{16}$}
	{$\dfrac{27a^2\sqrt{3}}{16} $}
	{\True $\dfrac{9a^2\sqrt{3}}{16} $}
	\loigiai{\immini{
			Gọi $M$ là trung điểm $BC$ và $O$ là trọng tâm tam giác $ABC$.\\
			Khi đó ta có $\left(SA, (ABC) \right)= \widehat{SAO}= 60^\circ$.\\
			Suy ra $AO = \cos 60^\circ \cdot SA = \cos 60^\circ \cdot a\sqrt{3}=\dfrac{a\sqrt{3}}{2}$.\\
			Do đó $AB = \sqrt{3}\cdot AO = \dfrac{3a}{2}$.\\
			Vậy $S_{ABC}= AB^2 \cdot \dfrac{\sqrt{3}}{4}= \dfrac{9a^2\sqrt{3} }{16}.$}{\begin{tikzpicture}[scale=0.6, font=\footnotesize, line join=round, line cap=round, >=stealth]
				%\draw[color=gray,dash pattern=on 1pt off 1pt,xstep=1.0cm,ystep=1.0cm] (-5.2,-5.2) grid (5.2,5.2);
				\coordinate (A) at (0,0);
				\coordinate (C) at (5,0);
				\coordinate (B) at (2,-1.5);
				\coordinate (M) at ($(B)!1/2!(C)$);
				\coordinate (O) at ($(A)!2/3!(M)$);
				\coordinate (S) at ($(O)+ (90:5)$);
				\draw (A)--(B)--(C)--(S)--cycle (C)--(S)--(B) (S)--(M); 
				\draw[dashed] (C)--(A)--(M) (S)--(O);
				\foreach \p/\r in {A/180,B/-90, C/0,S/90, M/-90, O/-80}
				\fill (\p) circle (1.5pt) node[shift={(\r:3mm)}]{$\p$};
	\end{tikzpicture}}}
\end{ex}

\begin{ex} %[Nguyễn Trần Phong]%[1K7BO-7]
	Cho hình chóp $S.ABC$ có đáy là tam giác cân tại $A$. $SA \perp (ABC)$, tam giác $SBC$ đều cạnh $2a$ và nằm trong mặt phẳng hợp với $(ABC)$ một góc $60^\circ$. Tính diện tích tam giác $ABC$.
	\choice
	{$\dfrac{a^2 \sqrt{3}}{3} $}
	{$\dfrac{a^2 \sqrt{3}}{4} $}
	{\True $\dfrac{a^2 \sqrt{3}}{2}$}
	{$\dfrac{3a^2 \sqrt{3}}{2} $}
	\loigiai{\immini{
			Gọi $M$ là trung điểm của $BC$, do tam giác $ABC$ đều nên $AM \perp BC$.\\
			Mà $BC \perp SA $ (do $SA \perp (ABC)$) nên $BC \perp SM$.
			\\
			Ta có $\heva{& (SBC) \cap (ABC) =BC \\ & AM \subset (ABC), \, AM \perp BC\\& SM \subset (SBC), \, SM \perp BC }\\ \Rightarrow \left((SBC), (ABC) \right)= \left(SM, AM \right)=\widehat {SMA}= 60^\circ$.\\
			Xét tam giác $SAM$ vuông tại $A$ có $AM = \cos 60^\circ \cdot SM = \dfrac{a\sqrt{3}}{2}$.\\
			Vậy $S_{ABC} =\dfrac{1}{2} \cdot AM \cdot BC = \dfrac{1}{2} \cdot \dfrac{a\sqrt{3}}{2}\cdot 2a=\dfrac{a^2 \sqrt{3}}{2}$.
		}{\begin{tikzpicture}[scale=0.6, font=\footnotesize, line join=round, line cap=round, >=stealth]
				%\draw[color=gray,dash pattern=on 1pt off 1pt,xstep=1.0cm,ystep=1.0cm] (-5.2,-5.2) grid (5.2,5.2);
				\coordinate (A) at (0,0);
				\coordinate (C) at (5,0);
				\coordinate (B) at (2,-1.5);
				\coordinate (S) at ($(A)+ (90:5)$);
				\coordinate (M) at ($(B)!1/2!(C)$);
				\draw (A)--(B)--(C)--(S)--cycle (C)--(S)--(B) (S)--(M); 
				\draw[dashed] (C)--(A)--(M);
				\foreach \p/\r in {A/180,B/-90, C/0,S/90, M/-90}
				\fill (\p) circle (1.5pt) node[shift={(\r:3mm)}]{$\p$};
	\end{tikzpicture}}}
\end{ex}


\begin{ex} %[Nguyễn Trần Phong]%[1K7KO-7]
	Cho hình lăng trụ đều $ABC.A'B'C'$ có cạnh đáy là $2a$, $AA' = 5a$, điểm $M$ nằm trên cạnh $AA'$ sao cho $(MBC)$ và $(ABC)$ hợp với nhau một góc $60^\circ$. Tính tỉ số $\dfrac{AM}{AA'}$. 
	\choice
	{\True $\dfrac{3}{5} $}
	{$\dfrac{2}{5} $}
	{$\dfrac{1}{2} $}
	{$\dfrac{2}{3} $}
	\loigiai{
		\immini{Gọi $N$ là trung điểm $BC$, do tam giác $ABC$ đều nên $BC \perp AN$.\\
			Lại do $BC \perp AM$ (do $AM \perp (ABC)$) nên $BC \perp (AMN)$ hay $BC \perp MN$.\\
			Khi đó $\heva{& BC = (MBC) \cap (ABC)\\& AN \subset (ABC), \, AN \perp BC\\& MN \subset (MBC), \, MN \perp BC} \\ \Rightarrow \left((MBC), (ABC) \right)= (AM, MN)= \widehat {ANM}= 60^\circ$.\\
			Xét tam giác $AMN$ vuông tại $A$ có $AM = \tan 60^\circ \cdot AN = \sqrt{3} \cdot \dfrac{2a \cdot \sqrt{3}}{2} = 3a.$\\
			Vậy $\dfrac{AM}{AA'}= \dfrac{3}{5}$.
		}{\begin{tikzpicture}[scale=0.7, font=\footnotesize, line join=round, line cap=round, >=stealth]
				%\draw[color=gray,dash pattern=on 1pt off 1pt,xstep=1.0cm,ystep=1.0cm] (-5.2,-5.2) grid (5.2,5.2);
				\coordinate (A) at (0,0);
				\coordinate (C) at (5,0);
				\coordinate (B) at (2,-2);
				\coordinate (A') at ($(A) +(90:5)$);
				\coordinate (B') at ($(B)+(90:5)$);
				\coordinate (C') at ($(C)+(90:5)$);
				\coordinate (M) at ($(A)!2/3!(A')$);
				\coordinate (N) at ($(B)!1/2!(C)$);
				%\draw pic[draw,angle radius=2mm] {right angle = S--H--K}; 
				%\draw pic[draw,angle radius=2mm] {right angle = H--I--K};
				%\draw pic[draw,angle radius=2mm] {right angle = H--K--D};
				%\coordinate (I) at (intersection of O--N and A--B) {};
				\draw (A)--(B)--(C)  (A')--(B')--(C')--cycle (A)--(A') (B)--(B') (C)--(C') (M)--(B); 
				\draw[dashed] (A)--(C) (M)--(C) (A) --(N)--(M);
				\foreach \p/\r in {A/180,B/-90, C/0,A'/90, B'/-30, C'/90, M/180, N/0}
				\fill (\p) circle (1.5pt) node[shift={(\r:3mm)}]{$\p$};
		\end{tikzpicture}}
	}
\end{ex}


\begin{ex} %[Nguyễn Trần Phong]%[1K7KO-7]
	Cho hình chóp $S.ABC$	có $SA \perp (ABC)$, tam giác $ABC $ đều cạnh $a$, $SA=a$. Gọi $(P)$ là mặt phẳng đi qua $S$ và vuông góc với $BC$. Tính diện tích của thiết diện tạo bởi $(P)$ và khối chóp $S.ABC$.
	\choice
	{$  \dfrac{\sqrt{3} a^2}{3}$}
	{$ \dfrac{\sqrt{3} a^2}{5} $}
	{$ \dfrac{a^2}{4} $}
	{\True $ \dfrac{\sqrt{3} a^2}{4} $}
	\loigiai{\immini{
			Gọi $I$ là trung điểm $BC$, do tam giác $ABC$ đều nên $AI \perp BC$.\\
			Ta có $\heva{& BC \perp AI\\ & BC \perp SA \text{ (do } SA \perp (ABC))\\& AI \cap BC = I} \Rightarrow BC \perp (SAI).$\\
			Suy ra  $ (P) \equiv (SAI)$ và tam giác $SAI$ là thiết diện tạo bởi $(P)$ và hình chóp $S.ABC$.\\
			Khi đó $S_{\triangle SAI}=\dfrac{1}{2}\cdot SA \cdot AI = \dfrac{1}{2} \cdot  a \cdot \dfrac{a\sqrt{3}}{2}= \dfrac{\sqrt{3} a^2}{4}$.}{\begin{tikzpicture}[scale=0.6, font=\footnotesize, line join=round, line cap=round, >=stealth]
				%\draw[color=gray,dash pattern=on 1pt off 1pt,xstep=1.0cm,ystep=1.0cm] (-5.2,-5.2) grid (5.2,5.2);
				\coordinate (A) at (0,0);
				\coordinate (C) at (5,0);
				\coordinate (B) at (2,-1.5);
				\coordinate (S) at ($(A)+ (90:5)$);
				\coordinate (I) at ($(B)!1/2!(C)$);
				\draw (A)--(B)--(C)--(S)--cycle (C)--(S)--(B) (S)--(I); 
				\draw[dashed] (C)--(A)--(I);
				\foreach \p/\r in {A/180,B/-90, C/0,S/90, I/-90}
				\fill (\p) circle (1.5pt) node[shift={(\r:3mm)}]{$\p$};
	\end{tikzpicture}}}
	
\end{ex}

\begin{ex}%[Nguyễn Trần Phong]%[1K7KO-7]
	Cho hình chóp $S.ABCD$ có đáy là hình thang vuông tại $A$, $B$, $AB=BC= a$, $AD=2a$, $SA=4a$, $SA \perp (ABCD)$. Gọi $M$ là trung điểm $AB$, mặt phẳng $(P)$ đi qua $M$, vuông góc với $AB$ và tạo hình chóp thành một thiết diện có diện tích bằng
	\choice
	{$\dfrac{5a^2}{2} $}
	{ $\dfrac{7a^2}{2} $}
	{$ \dfrac{a^2}{2}$}
	{\True $ 2a^2$}
	\loigiai{\immini{Gọi $N$, $Q$ lần lượt là trung điểm $SB$ và $DC$.\\
			Do $M$ là trung điểm $AB$ nên $MN \parallel SA$, $MQ \parallel AD \parallel BC $.\\
			Lại do $SA \perp AB$, $AD \perp AB$ nên $ AB \perp MN$, $AB \perp MQ$.\\
			Suy ra $AB \perp (MNQ)$ hay $(MNQ) \equiv (P)$.\\
			Ta có $\heva{& (SBC) \cap (ABCD)= BC\\ & (P) \cap (ABCD)= MQ\\ & N \in (P) \cap (SBC)}\\ \Rightarrow (P) \cap (SBC) = x'Nx \parallel BC \parallel MQ$.\\
			Trong $(SBC)$, gọi $P = x'Nx \cap SC$ thì $MNPQ$ là thiết diện tạo bởi $(P)$ và hình chóp $S.ABCD$.}{\begin{tikzpicture}[scale=1, font=\footnotesize, line join=round, line cap=round,>=stealth]
				\path (0,0)coordinate (B) (3,0) coordinate (C) (4.0,1.5) coordinate (d) ($(B)+(d)-(C)$) coordinate (A) (0,2.5) coordinate (h) ($(A)+(h)$) coordinate (S) ($(A)!2!(d)$) coordinate (D) ($(A)!0.5!(D)$) coordinate (E) ($(A)!0.5!(B)$) coordinate (M) ($(M)+(S)-(A)$) coordinate (n) (intersection of M--n and S--B) coordinate (N) ($(M)+(D)-(A)$) coordinate (q) (intersection of M--q and C--D) coordinate (Q) ($(N)+(Q)-(M)$) coordinate (p) (intersection of N--p and S--C) coordinate (P) (intersection of C--E and M--Q) coordinate (F);
				\draw (S)--(B)--(C)--(D)--cycle (S)--(C) (N)--(P)--(Q);
				\draw[dashed] (S)--(A)--(B) (D)--(A) (N)--(M)--(Q);
				\foreach \p/\g in {A/150,B/180,C/-60, D/0, S/90,M/-60,N/150,P/30,Q/-45} \fill[black] (\p) circle(1pt)+(\g:0.3) node{$\p$};
				\draw pic[draw, angle radius=0.2cm]{right angle= S--A--D}  pic[draw, angle radius=0.2cm]{right angle= B--A--D} pic[draw, angle radius=0.2cm]{right angle= A--B--C};
		\end{tikzpicture}}
		\noindent  Do $\heva{&MN \parallel SA\\& SA \perp MQ \text{ (do } SA \perp (ABCD))}$ nên $MN \perp MQ$.\\
		Mà $NP \parallel MQ$ nên tứ giác $MNPQ$ là hình thang vuông tại $M$, $N$.\\
		Ta có $MQ =\dfrac{BC + AD}{2}=\dfrac{3a}{2}$. \\
		$MN =\dfrac{1}{2} \cdot SA =2a$, $NP=\dfrac{BC}{2}=\dfrac{a}{2}$.\\
		Vậy $S_{MNPQ}=\dfrac{MN \cdot (NP + MQ)}{2}= 2a^2$.}
\end{ex}




\begin{ex}%[Nguyễn Trần Phong]%[1K7KO-7]
	Cho hình chóp $S.ABCD$ có đáy $ABCD$ là hình thang vuông tại $A$ và $B$ với $AD=2a$, $AB=BC=a$, $SA \perp (ABCD)$ và $SA=2a$. Goi $M$ là một điểm trên cạnh $AB$, $(\alpha)$ là mặt phẳng qua $M$ vuông góc với $AB$. Đặt $AM=x$ với $0<x<a$, tính diện tích thiết diện theo $a$ và $x$. 
	\choice{$a(a-x)$}{\True $2a(a-x)$ }{$x(a-x) $}{$ax$}
	
	\loigiai
	{\immini{Ta có $SA\perp AB$, $AD\perp AB$ và $(\alpha)\perp AB$ nên $(\alpha)\parallel SA$, $(\alpha)\parallel AD$.\\
			Gọi $N$, $P$, $Q$ lần lượt là giao điểm của $(\alpha)$ với $SB$, $SC$, $CD$.\\
			Khi đó $\heva{& (\alpha)\cap (SAB)=MN \\ & (\alpha)\parallel SA}\Rightarrow MN\parallel SA$.\\
			Tương tự, $(\alpha)\cap (ABCD)=MQ\parallel AD$.\\
			Ta có $BC\perp AB$, $(\alpha)\perp AB$ nên $(\alpha)\parallel BC$ và  $(\alpha)\cap (SBC)=NP\parallel BC$.}{\begin{tikzpicture}[scale=1, font=\footnotesize, line join=round, line cap=round,>=stealth]
				\path (0,0)coordinate (B) (3,0) coordinate (C) (4.0,1.5) coordinate (d) ($(B)+(d)-(C)$) coordinate (A) (0,2.5) coordinate (h) ($(A)+(h)$) coordinate (S) ($(A)!2!(d)$) coordinate (D) ($(A)!0.5!(D)$) coordinate (E) ($(A)!0.5!(B)$) coordinate (M) ($(M)+(S)-(A)$) coordinate (n) (intersection of M--n and S--B) coordinate (N) ($(M)+(D)-(A)$) coordinate (q) (intersection of M--q and C--D) coordinate (Q) ($(N)+(Q)-(M)$) coordinate (p) (intersection of N--p and S--C) coordinate (P) (intersection of C--E and M--Q) coordinate (F);
				\draw (S)--(B)--(C)--(D)--cycle (S)--(C) (N)--(P)--(Q);
				\draw[dashed] (S)--(A)--(B) (D)--(A) (N)--(M)--(Q) (E)--(C);
				\foreach \p/\g in {A/150,B/180,C/-60, D/0, S/90,E/90,M/-60,N/150,P/30,Q/-45,F/120} \fill[black] (\p) circle(1pt)+(\g:0.3) node{$\p$};
				\draw pic[draw, angle radius=0.2cm]{right angle= S--A--D}  pic[draw, angle radius=0.2cm]{right angle= B--A--D} pic[draw, angle radius=0.2cm]{right angle= A--B--C};
		\end{tikzpicture}}\noindent
		Vì $AD\parallel BC$ nên $NP\parallel MQ$.\\
		Do đó, thiết diện của hình chóp $S.ABCD$ với mặt phẳng $(\alpha)$ là hình thang $MNPQ$.\\
		Ta có $MN\parallel SA$ và $SA\perp (ABCD)$ nên $MN\perp (ABCD)\Rightarrow MN\perp MQ$.\\
		Vậy thiết diện $MNPQ$ là hình thang vuông tại $M$ và $N$.\\ 
		Gọi $E$ là trung điểm của $AD$ và $F=CE\cap MQ$.\\
		Khi đó $ABCE$ là hình vuông và $\triangle CFQ$ vuông cân tại $F$.\\
		Suy ra $MQ=MF+FQ=a+a-x=2a-x$.\\
		Từ $MN\parallel SA$ suy ra $\dfrac{MN}{SA}=\dfrac{BM}{BA}\Rightarrow MN=\dfrac{BM\cdot SA}{BA}=\dfrac{(a-x)\cdot 2a}{a}=2(a-x)$.\\
		Vì $NP\parallel BC$ và $MN\parallel SA$ nên $\dfrac{NP}{BC}=\dfrac{SN}{SB}=\dfrac{AM}{AB}\Rightarrow NP=\dfrac{AM\cdot BC}{AB}=\dfrac{x\cdot a}{a}=x$.\\
		Vậy diện tích thiết diện là $$S_{MNPQ}=\dfrac{(MQ+NP)\cdot MN}{2}=\dfrac{(2a-x+x)\cdot 2(a-x)}{2}=2a(a-x).$$
	}
\end{ex}


\begin{ex}%[Nguyễn Trần Phong]%[1K7GO-7]
	Cho hình chóp $S.ABC$ có $SA \perp (ABC)$, tam giác $ABC$ vuông cân tại $B$, $AB=a$, $SA= a\sqrt{3}$. Gọi $M$ nằm trên cạnh $AB$ sao cho $AM=x$, $0<x<a$. Mặt phẳng $(P)$ đi qua $M$, vuông góc với $AB$. Tìm $x$ để diện tích của thiết diện tạo bởi $(P)$ và hình chóp đạt giá trị lớn nhất.
	\choice
	{\True $\dfrac{a}{2} $}
	{$\dfrac{a}{\sqrt{2}} $}
	{$\dfrac{a}{3} $}
	{$\dfrac{2a}{3} $}
	\loigiai{\immini{
			Gọi $N$, $Q$ lần lượt là trung điểm của $SB$ và $AC$.\\
			Do $M$ là trung điểm $AB$ nên $MQ \parallel BC$, $MN \parallel SA$.\\
			Lại do $SA \perp AB$, $BC \perp AB$ nên $MN, \, MQ \perp AB$ hay $AB \perp (MNQ)$.\\
			Suy ra $(P) \equiv (MNQ)$.\\
			Ta có $\heva{& (P) \cap (ABC) = MQ\\ & (SBC) \cap (ABC) = BC \\& N \in (P) \cap (SBC) \\ & MQ \parallel BC} \Rightarrow (SBC) \cap (P) = x'Nx \parallel BC$.
			\\
			Trong $(SBC)$, gọi $P = x'Nx \cap SC$ thì $MNPQ$ là thiết diện tạo bởi hình chóp và $(P)$.\\
			Thiết diện này là hình chữ nhật.\\
			Ta có $\dfrac{MN}{SA}=\dfrac{BM}{BA} \Rightarrow MN = \dfrac{BM}{BA} \cdot SA = \dfrac{a-x}{a} \cdot a\sqrt{3} =(a-x)\sqrt{3}$.\\
			$\dfrac{MQ}{BC}=\dfrac{AM}{AB}\Rightarrow MQ =\dfrac{AM}{AB}\cdot BC =\dfrac{x}{a}\cdot a= x$.\\
			Vậy $S_{MNPQ}= MN \cdot MQ= (a-x) \cdot x \cdot \sqrt{3} \le \sqrt{3} \dfrac{\left(a-x+ + x\right)^2}{4}= \dfrac{\sqrt{3}}{4}a^2 $.\\
			Vậy GTLN của $S_{MNPQ} $ là $\dfrac{\sqrt{3}a^2}{4}$ khi $a-x=x \Leftrightarrow x =\dfrac{a}{2}$.
		}{\begin{tikzpicture}[scale=0.6, font=\footnotesize, line join=round, line cap=round, >=stealth]
				%\draw[color=gray,dash pattern=on 1pt off 1pt,xstep=1.0cm,ystep=1.0cm] (-5.2,-5.2) grid (5.2,5.2);
				\coordinate (A) at (0,0);
				\coordinate (C) at (5,0);
				\coordinate (B) at (2,-1.5);
				\coordinate (S) at ($(A)+ (90:5)$);
				\coordinate (M) at ($(A)!1/2!(B)$);
				\coordinate (N) at ($(S)!1/2!(B)$);
				\coordinate (P) at ($(S)!1/2!(C)$);
				\coordinate (Q) at ($(A)!1/2!(C)$);
				\draw (A)--(B)--(C)--(S)--cycle (C)--(S)--(B) (M)--(N)--(P); 
				\draw[dashed] (C)--(A)--(M) (M)--(Q)--(P);
				\draw pic[draw, angle radius=0.2cm]{right angle= A--B--C} pic[draw, angle radius=0.2cm]{right angle= S--A--B} ;
				\foreach \p/\r in {A/180,B/-90, C/0,S/90, M/-90, N/180, P/0, Q/30}
				\fill (\p) circle (1.5pt) node[shift={(\r:3mm)}]{$\p$};
	\end{tikzpicture}}}
\end{ex}

\Closesolutionfile{ans}
\begin{indapan}{10}
	{ans/ans-1K7-25-Dang8}
\end{indapan}

