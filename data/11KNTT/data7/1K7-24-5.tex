\Opensolutionfile{ans}[ans/ans-1K1-3-Dang1]
\begin{bt}%[1K7BN-4]
	\immini{Trên một sân phẳng nằm ngang, tại các điểm $A$, $B$, $C$, $D$ người ta dựng các cột thẳng đứng $AM$, $BN$, $CP$, $DQ$ và nối các sợi dây thẳng giữa $M$ và $P$, $N$ và $Q$ như Hình $7.35$. 
		\begin{enumerate}
			\item Hãy chỉ ra hình chiếu của các dây $MP$ và $NQ$ trên sân.
			\item Chứng minh rằng nếu $BD\perp AC$ thì $BD\perp MP$.
			\item Chứng minh rằng nếu $ABCD$ là một hình bình hành thì các trung điểm $E$, $F$ tương ứng của các đoạn thằng $MP$ và $NQ$ có cùng hình chiếu trên sân.
	\end{enumerate}}{
		\begin{tikzpicture}[scale=1, font=\footnotesize, line join=round, line cap=round, >=stealth]
			\path 
			(0,0) coordinate (X)
			(5,0) coordinate (Y)
			(1,2.5) coordinate (Z)
			($(Y)-(X)+(Z)$) coordinate (T)
			(3, 1.2) coordinate (O)
			(1.5, 0.5) coordinate (A)
			(4, 0.5) coordinate (D)
			($(A)!2!(O)$) coordinate (C)
			($(D)!2!(O)$) coordinate (B)
			($(A)+(0,3.7)$) coordinate (M)
			($(C)+(0,2.5)$) coordinate (P)
			($(B)+(0,1)$) coordinate (N)
			($(D)+(0,2.5)$) coordinate (Q)
			($(M)!0.5!(P)$) coordinate (E)
			($(N)!0.5!(Q)$) coordinate (F);
			\fill[gray!50] (X)--(Y)--(T)--(Z)--cycle;
			\draw [red, line width =1.5pt] (A)--(M) (C)--(P) (D)--(Q) (B)--(N);
			\draw[line width =1pt] (A)--(C) (B)--(D);
			\draw[red!50!white, line width =1.5pt] (N)--(Q);
			\draw[green!80!black, line width =1.5pt] (M)--(P);
			\foreach \i/\j in {A/-145, D/-45, O/-90, B/180, C/0, M/135, P/30, E/90, N/135, F/90, Q/30}\draw (\i) circle (1pt) ++ (\j:0.3) node {$\i$};
			\draw (2.5,-0.5) node {Hình $7.35$};
	\end{tikzpicture}}
	\loigiai{
		\begin{enumerate}[a)]
			\item Do các cột có phương thẳng đứng và sân thuộc mặt phẳng nằm ngang nên các cột vuông góc với sân. Vậy $A$, $B$, $C$, $D$ tương ứng là hình chiếu của $M$, $N$, $P$, $Q$ trên sân. Do đó $AC$, $BD$ tương ứng là hình chiếu của $MP$, $NQ$ trên sân.
			\item Nếu $BD\perp AC$, mà $AC$ là hình chiếu của $MP$ trên sân và $BD$ thuộc sân nên theo định lý ba đường vuông góc ta có $BD\perp MP$.
			\item Nếu $ABCD$ là một hình bình hành thì các đoạn thẳng $AC$, $BD$ có chung trung điểm $O$. Do $EO$ là đường trung bình của hình thang $ACPM$ nên $EO\parallel MA$. Mặt khác, $MA$ vuông góc với sân nên $EO$ cũng vuông góc với sân. Vậy $O$ là hình chiếu của $E$ trên sân. Tương tự, $O$ cũng là hình chiếu của $F$ trên sân. Vậy $E$ và $F$ có cùng hình chiếu trên sân.
		\end{enumerate}
	}
\end{bt}
\begin{bt}%[1K7BN-4]
	Trong một khoảng thời gian đầu kể từ khi cất cánh, máy bay bay theo một đường thẳng. Góc cất cánh của nó là góc giữa đường thẳng đó và mặt phẳng nằm ngang nơi cất cánh. Hai máy bay cất cánh và bay thẳng với cùng độ lớn vận tốc trong $5$ phút đầu, với các góc cất cánh lần lượt là $10^\circ, 15^\circ$. Hỏi sau $ 1 $ phút kể từ khi cất cánh, máy bay nào ở độ cao so với mặt đất (phẳng, nằm ngang) lớn hơn?\\
	\begin{note}
		\textbf{Chú ý}. Độ cao của máy bay so với mặt đất là khoảng cách từ máy bay (coi là một điểm) đến hình chiếu của nó trên mặt đất.
	\end{note}
	\loigiai{Sau $1$ phút kể từ khi cất cánh, máy bay với góc cất cánh $15^\circ$ ở độ cao so với mặt đất (phẳng, nằm ngang) lớn hơn}
\end{bt}
\begin{bt}%[1K7BN-4]
	Hãy nêu cách đo góc giữa đường thẳng chứa tia sáng mặt trời và mặt phẳng nằm ngang tại một vị trí và một thời điểm.
	\begin{note}
		\textbf{Chú ý}. Góc giữa đường thẳng chứa tia sáng mặt trời lúc giữa trưa với mặt phẳng nằm ngang tại vị trí đó được gọi là góc Mặt Trời. Giữa trưa là thời điểm ban ngày mà tâm Mặt Trời thuộc mặt phẳng chứa kinh tuyến đi qua điểm đang xét. Góc Mặt Trời ảnh hưởng tới sự hấp thụ nhiệt từ Mặt Trời của Trái Đất, tạo nên các mùa trong năm trên Trái Đất.
	\end{note}
\end{bt}
\begin{bt}%[1K7BN-4]
	Trên mặt đất phẳng, người ta dựng một cây cột $AB$ có chiều dài bằng $10$ m và tạo với mặt đất góc $80^{\circ}$. Tại một thời điểm dưới ánh sáng mặt trời, bóng $BC$ của cây cột trên mặt đất dài $12$ m vào tạo với cây cột một góc bằng $120^{\circ}$ (tức là $\widehat{ABC}=120^{\circ}$). Tính góc giữa mặt đất và đường thẳng chứa tia sáng mặt trời tại thời điểm nói trên.
	\loigiai{
		\immini{
			Gọi $H$ là hình chiếu vuông góc của $A$ trên mặt đất. Theo giả thiết ta có $\widehat{ABH}=80^\circ$. Đường thẳng $BC$ là bóng của $AB$ nên đường thẳng chứa tia sáng là $AC$. Khi đó góc giữa mặt đất và đường thẳng chứa tia sáng là góc $\widehat{ACH}$.\\
			Xét $\triangle ABH$ vuông tại $H,$ có 
			$$AH=AB\cdot \sin \widehat{ABH}=10\cdot \sin 80^\circ.$$ 
		}{
			\begin{tikzpicture}[line join=round,line cap=round,>=stealth,font=\footnotesize,scale=.8]
				\coordinate[label=left:$H$] (H) at (0,0);
				\coordinate[label=below left:$B$] (B) at (1.5,-1.5);
				\coordinate[label=right:$C$] (C) at (6,0);
				\coordinate[label=above left:$A$] (A) at ($(H)+(90:3.5)$);
				\draw (H)--(B)--(C)--(A)--cycle (A)--(B);
				\draw[dashed] (H)--(C);
				%\draw (B)--(H)--(A) pic [draw,"$80^\circ$"] {angle};
				%\draw ($ (A)!5pt!(C)$)--($(A)!2!($($(A)!5pt!(C)$)!.5!($(A)!5pt!(S)$)$)$)--($(A)!5pt!(S)$);
				%\fill (A)circle(1.5pt) (B)circle(1.5pt) (C)circle(1.5pt) (S)circle(1.5pt);
		\end{tikzpicture}}
		\noindent
		Xét $\triangle ABC,$ có 
		$$AC^2=AB^2+BC^2-2AB\cdot BC\cos\widehat{ABC}=364 \Rightarrow AC=2\sqrt{91}.$$
		Xét $\triangle AHC$ vuông tại $H$, có 
		$$\sin\widehat{ACH}=\dfrac{AH}{AC}=\dfrac{10\cdot \sin 80^\circ}{2\sqrt{91}}=\dfrac{5\cdot \sin 80^\circ}{\sqrt{91}}.$$
		Suy ra $\widehat{ACH}\approx 31^\circ$.
	} 
\end{bt}
\begin{bt}%[1K7BN-4]
	\immini
	{
		Để ước lượng chiều cao của tháp khi không thể lên tới đỉnh tháp, người ta đo góc giữa tia nắng chiếu qua đỉnh tháp và mặt đất, đo chiều dài của bóng tháp trên mặt đất, từ đó ước lượng được chiều cao của tháp. Giả sử khi tia nắng tạo với mặt đất một góc $40^\circ$, chiều dài của bóng tháp là $80$ m (Hình bên). Tính chiều cao của tháp theo đơn vị mét (làm tròn kết quả đến hàng phần mười).
	}
	{
		
		\begin{tikzpicture}[scale=0.7, font=\footnotesize, line join=round, line cap=round, >=stealth]
			\tikzset{Icon-mattroi/.pic={
					\def\r{0.4}
					\fill(0,0)circle(\r);
					\draw[line width=1pt](0,0)circle(\r);
					\foreach \g in{1,...,10}{
						\draw[line width=1 pt](\g*36:1.2*\r)++(\g*36:0.1*\r)--++(\g*36:\r*0.25);
					}
			}}
			\path
			(0,6) coordinate (I)
			(-7,0.3) coordinate (M)
			(0,0.3) coordinate (O)
			(1,7)pic[yellow,scale=0.5]{Icon-mattroi}
			;
			\def\originpath{(0.5,0)--(0.5,1)--(0.4,1.2)--(0.4,4)--(0.5,4)--(0.5,5)--(0.2,5.4)--(0.2,5.7)--(0,6)}
			\fill[brown] (0.2,5.7)--(0,6)--(-0.2,5.7)--cycle
			(0.2,5.4)--(0.5,5)--(-0.5,5)--(-0.2,5.4)--cycle
			;
			\fill[brown!50] (0.2,5.7)--(0.2,5.4)--(-0.2,5.4)--(-0.2,5.7)--cycle
			(0.5,5)--(0.5,4)--(-0.5,4)--(-0.5,5)--cycle
			(0.5,0)--(0.5,1)--(0.4,1.2)--(0.4,4)--(-0.4,4)--(-0.4,1.2)--(-0.5,1)--(-0.5,0)--cycle
			;
			\fill[brown] (-0.3,4.8)--(0.3,4.8)--(0.3,4.2)--(-0.3,4.2)--cycle
			;
			\fill[brown!70]
			(0.4,1.2)--(0.4,4)--(-0.4,4)--(-0.4,1.2)--cycle (0.5,0)--(0.5,1)--(-0.5,1)--(-0.5,0)--cycle;
			\fill[white] (0,4.5)circle (6pt);
			\fill[gray!50] (-0.5,0.8)--(-1,0.8)--(-1.2,0.7)--(-4,0.7)--(-4.2,0.8)--(-5,0.8)--(-5.5,0.6)--(-6,0.5)--(-7,0.3)--(-6,0.1)--(-5,-0.1)--(-4.5,-0.1)--(-3.5,-0.1)--(-3,0.1)--(-1,0.1)--(-0.8,0)--(-0.5,0)--(-0.5,0.8);
			\draw[brown,line width=2pt] (-0.2,4)--(-0.2,0) (0,4)--(0,0) (0.2,4)--(0.2,0) (-0.5,0)--(0.5,0) ;
			\draw[blue,line width=2pt] (I)--(M) (M)--(O)node[pos=0.5,below]{$80$ m};
			\draw[brown] \originpath;
			% đối xứng qua trục tung
			\draw[brown,xscale=-1] \originpath;
			\draw
			pic[draw,angle radius=10]{angle=O--M--I};
			\draw (-6.5,0.3) node[above right] {$40^\circ$};
		\end{tikzpicture}
	}
	\loigiai{
		\immini
		{
			Xét hình bên, độ dài $AH$ chỉ chiều cao của tháp, độ dài $OH$ chỉ chiều dài của bóng tháp, độ lớn của góc $AOH$ chỉ số đo góc giữa tia nắng và mặt đất. Vì tam giác $OAH$ vuông tại $H$ nên
			$$
			AH=OH \cdot \tan \widehat{AOH}=80 \cdot \tan 40^\circ \approx 67{,}1(\mathrm{m}).
			$$
		}
		{
			\begin{tikzpicture}[scale=0.7, font=\footnotesize, line join=round, line cap=round, >=stealth]
				\path
				(0,6) coordinate (A)
				(-7,0.3) coordinate (O)
				(0,0.3) coordinate (H);
				\draw (A)--(O) (A)--(H) (H)--(O)node[pos=0.5,below]{$80$ m};
				\draw
				pic[draw,angle radius=5]{right angle=A--H--O};
				\draw
				pic[draw,angle radius=10]{angle=H--O--A};
				\draw (-6.5,0.3) node[above right] {$40^\circ$};
			\end{tikzpicture}
		}
	}
\end{bt}
\begin{bt}%[1K7KN-4]
	Dốc là đoạn đường thẳng nối hai khu vực hay hai vùng có độ cao khác nhau. Độ dốc được xác định bằng góc giữa dốc và mặt phẳng nằm ngang, ở đó độ dốc lốn nhất là $100 \%$, tương ứng với góc $90^\circ$ (độ dốc $10 \%$ tương ứng với góc $9^\circ$). Giả sử có hai điểm $A$, $B$ nằm ở độ cao lần lượt là $200$ m, $220$ m so với mực nước biển và đoạn dốc $AB$ dài $120$ m. Độ dốc đó bằng bao nhiêu phần trăm (làm tròn kết quả đến hàng phần trăm)?
	\loigiai{
		\immini
		{
			Ta thấy điểm $B$ ở vị trí cao hơn điểm $A$ $20$ m.\\
			Xét tam giác $ABC$ như hình vẽ.\\
			Độ dốc bằng góc giữa dốc và mặt phẳng nằm ngang, tức là bằng góc $\widehat{BAC}$.\\
			Ta có $\sin \widehat{BAC}=\dfrac{20}{120}=\dfrac{1}{6}$.\\
			Suy ra $\widehat{BAC}\approx 9{,}6^\circ$.\\
			Vậy độ dốc $\approx 10{,}67\%$.
		}
		{
			\begin{tikzpicture}[scale=1, font=\footnotesize, line join=round, line cap=round, >=stealth]
				\path
				(0,0) coordinate (A)
				(4,3) coordinate (B)
				(4,0) coordinate (C)
				;
				\draw (A)--(B)node[pos=0.5,left]{$120$m}
				(B)--(C)node[pos=0.5,right]{$20$m} (A)--(C);
				\foreach \x/\g in {A/270,B/90,C/270} \fill[black] (\x) circle (1pt)+(\g:0.3) node{$\x$};
				\draw
				pic[draw,angle radius=5]{right angle=B--C--A};
			\end{tikzpicture}
		}
	}
\end{bt}
\Closesolutionfile{ans}