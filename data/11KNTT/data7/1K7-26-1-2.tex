\setcounter{dang}{0}
\section{Khoảng Cách}
\subsection{Tóm tắt lý thuyết}
\begin{tomtat}
	\subsubsection{Khoảng cách từ một điểm đến một đường thẳng}
	Ta đã biết khoảng cách từ một điểm đến một đường thẳng trong mặt phẳng. Trong không gian, khái niệm khoảng cách đó được định nghĩa tương tự như trong mặt phẳng.
	\begin{dn}
		\immini
		{
			Cho đường thẳng $\Delta$ và điểm $M$ không thuộc $\Delta$. Gọi $H$ là hình chiếu của điểm $M$ trên đường thẳng $\Delta$. Độ dài đoạn thẳng $M H$ gọi là khoảng cách từ điểm $M$ đến đường thẳng $\Delta$, kí hiệu $\mathrm{d}(M, \Delta)$.
		}
		{
			\begin{tikzpicture}[scale=.7,font=\footnotesize, line join=round, line cap=round, >=stealth]
				\coordinate (A) at (0,0);
				\coordinate (B) at (5,0);
				\coordinate (H) at ($(A)!.3!(B)$);
				\coordinate (M) at ($(H)+(0,3)$);
				\foreach \x/\g in {H/-90,M/90} \fill[black](\x) circle (1.5pt) ($(\x)+(\g:3mm)$) node{$\x$};
				\draw (A)--(B)node[above left]{$\Delta$}
				(M)--(H);
				\draw pic[draw,angle radius=2mm] {right angle = M--H--B}; 
			\end{tikzpicture}
		}
	\end{dn}
	
	\begin{note}
		Khi điểm $M$ thuộc đường thẳng $\Delta$ thì $\mathrm{d}(M, \Delta)=0$.
	\end{note}
	\subsubsection{Khoảng cách từ một điểm đến một mặt phẳng}
	\begin{dn}
		\immini
		{
			Cho mặt phẳng $(P)$ và điểm $M$ không thuộc mặt phẳng $(P)$. Gọi $H$ là hình chiếu của $M$ trên mặt phẳng $(P)$. Độ dài đoạn thẳng $M H$ gọi là khoảng cách từ điểm $M$ đến mặt phẳng $(P)$, kí hiệu $\mathrm{d}(M,(P))$.
		}
		{
			\begin{tikzpicture}[scale=.7,font=\footnotesize, line join=round, line cap=round, >=stealth]
				\coordinate (A) at (0,0);
				\coordinate (B) at (5,0);
				\coordinate (D) at (2,3);
				\coordinate (C) at ($(B)+(D)-(A)$);
				\coordinate (H) at ($(A)!.5!(C)$);
				\coordinate (M) at ($(H)+(0,3)$);
				\coordinate (P) at ($(A)+(.6,.3)$);
				\foreach \x/\g in {H/180,M/90} \fill[black](\x) circle (1.5pt) ($(\x)+(\g:3mm)$) node{$\x$};
				\draw (A)--(B)--(C)--(D)--(A) (M)--(H);
				\path pic[draw,angle radius=7mm,angle eccentricity=0.5,"$P$"]{angle=B--A--D};
			\end{tikzpicture}
		}
	\end{dn}
	
	\begin{note}
		Khi điểm $M$ thuộc mặt phẳng $(P)$ thì $\mathrm{d}(M,(P))=0$.
	\end{note}
	
	\subsubsection{Khoảng cách giữa hai đường thẳng song song}
	
	\begin{dn}
		Khoảng cách giữa hai đường thẳng song song $\Delta$, $\Delta'$ là khoảng cách từ một điểm bất kì thuộc đường thẳng này đến đường thẳng kia, kí hiệu $\mathrm{d}\left(\Delta,\Delta'\right)$. 
	\end{dn}
	\immini
	{
		Trong hình bên, ta có $\mathrm{d}\left(\Delta, \Delta'\right)=A B$ với $A \in \Delta$, $B \in \Delta'$, $A B \perp \Delta$, $A B \perp \Delta'$ và $\Delta \parallel  \Delta'$.
	}
	{
		\begin{tikzpicture}[scale=.7,font=\footnotesize, line join=round, line cap=round, >=stealth]
			\coordinate (M) at (0,0);
			\coordinate (N) at (5,0);
			\coordinate (B) at ($(M)!.3!(N)$);
			\coordinate (A) at ($(B)+(0,2)$);
			\coordinate (M') at ($(M)+(0,2)$);
			\coordinate (N') at ($(N)+(0,2)$);
			\foreach \x/\g in {B/-90,A/90} \fill[black](\x) circle (1.5pt) ($(\x)+(\g:3mm)$) node{$\x$};
			\draw (M)--(N)node[above left]{$\Delta'$}
			(M')--(N')node[above left]{$\Delta$}
			(A)--(B);
			\draw pic[draw,angle radius=2mm] {right angle = M--B--A}; 
			\draw pic[draw,angle radius=2mm] {right angle = M'--A--B}; 
		\end{tikzpicture}
	}
	\subsubsection{Khoảng cách giữa đường thẳng và mặt phẳng song song}
	\begin{dn}
		Cho đường thẳng $\Delta$ song song song với mặt phẳng $(P)$. Khoảng cách giữa đường thẳng $\Delta$ và mặt phẳng $(P)$ là khoảng cách từ một điểm bất kì thuộc đường thẳng $\Delta$ đến mặt phẳng $(P)$, kí hiệu $\mathrm{d}(\Delta,(P))$.
	\end{dn}
	\immini
	{
		Trong hình bên, ta có $\mathrm{d}(\Delta,(P))=MH=h$, trong đó $M \in \Delta$, $H \in(P)$, $MH \perp(P)$ và $\Delta \parallel  (P)$.
	}
	{
		\begin{tikzpicture}[scale=.7,font=\footnotesize, line join=round, line cap=round, >=stealth]
			\coordinate (A) at (0,0);
			\coordinate (B) at (5,0);
			\coordinate (D) at (2,2);
			\coordinate (C) at ($(B)+(D)-(A)$);
			\coordinate (H) at ($(A)!.5!(C)$);
			\coordinate (M) at ($(H)+(0,3)$);
			\coordinate (P) at ($(A)+(.6,.3)$);
			\coordinate (N) at ($(M)+(-2,0)$);
			\coordinate (N') at ($(M)+(3,0)$);
			\coordinate (h) at ($(M)!0.5!(H)$);
			\foreach \x/\g in {H/-90,M/90} \fill[black](\x) circle (1.5pt) ($(\x)+(\g:3mm)$) node{$\x$};
			\draw (A)--(B)--(C)--(D)--(A)
			(M)--(H)
			(N)--(N')node[above left]{$\Delta$}
			(h)node[right]{$h$};
			\path pic[draw,angle radius=7mm,angle eccentricity=0.5,"$P$"]{angle=B--A--D};
		\end{tikzpicture}
	}
	\subsubsection{Khoảng cách giữa hai mặt phẳng song song}
	\begin{dn}
		Khoảng cách giữa hai mặt phẳng song song $(P)$, $(Q)$ là khoảng cách từ một điểm bất kì thuộc mặt phẳng này đến mặt phẳng kia, kí kiệu $\mathrm{d}((P),(Q))$.
	\end{dn}
	\immini
	{
		Trong hình bên, ta có $\mathrm{d}((P),(Q))=I K=h$ với $I \in(P)$, $K \in(Q)$, $I K \perp(P)$, $I K \perp(Q)$ và $(P) \parallel (Q)$.
	}
	{
		\begin{tikzpicture}[scale=.7,font=\footnotesize, line join=round, line cap=round, >=stealth]
			\coordinate (A) at (0,0);
			\coordinate (B) at (5,0);
			\coordinate (D) at (2,2);
			\coordinate (C) at ($(B)+(D)-(A)$);
			\coordinate (K) at ($(A)!.5!(C)$);
			\coordinate (I) at ($(K)+(0,3)$);
			\coordinate (G) at ($(I)!.32!(K)$);
			\coordinate (Q) at ($(A)+(.6,.3)$);
			\coordinate (h) at ($(I)!0.5!(K)$);
			\coordinate (A') at ($(A)+(0,3)$);
			\coordinate (B') at ($(B)+(0,3)$);
			\coordinate (C') at ($(C)+(0,3)$);
			\coordinate (D') at ($(D)+(0,3)$);			
			\coordinate (P) at ($(A')+(.6,.3)$);
			\foreach \x/\g in {K/-90,I/90} \fill[black](\x) circle (1.5pt) ($(\x)+(\g:4mm)$) node{$\x$};
			\draw (A)--(B)--(C)--(D)--(A)
			(A')--(B')--(C')--(D')--(A')
			(h)node[right]{$h$}
			(K)--(G);
			\draw[dashed] (G)--(I);
			\path pic[draw,angle radius=7mm,angle eccentricity=0.5,"$P$"]{angle=B--A--D};
			\path pic[draw,angle radius=7mm,angle eccentricity=0.5,"$Q$"]{angle=B'--A'--D'};
		\end{tikzpicture}
	}
	
	\subsubsection{Khoảng cách giữa hai đường thẳng chéo nhau}
	\begin{dn}
		Cho hai đường thẳng $a$, $b$ chéo nhau.
		\begin{itemize}
			\item Đường thẳng $c$ vừa vuông góc, vừa cắt cả hai đường thẳng $a$ và $b$ được gọi là đường vuông góc chung của hai đường thẳng đó.
			\item Đoạn thẳng có hai đầu mút là giao điểm của đường thẳng $c$ với hai đường thẳng $a$, $b$ được gọi là đoạn vuông góc chung của hai đường thẳng đó.
			\item Độ dài đoạn vuông góc chung của hai đường thẳng $a$, $b$ gọi là khoảng cách giữa hai đường thẳng đó, kí hiệu $\mathrm{d}(a, b)$.
		\end{itemize}
	\end{dn}
	
	\begin{nx}
		\immini
		{
			Gọi mặt phẳng chứa $b$ và song song với $a$ là $(P)$, hình chiếu của $a$ trên $(P)$ là $a'$, giao điểm của $a'$ và $b$ là $K$. Khi đó, $H K$ là đoạn vuông góc chung của hai đường thẳng chéo nhau $a, b$. Ngoài ra, ta cũng có $\mathrm{d}(a, b)=\mathrm{d}(a,(P))$.
		}
		{
			\begin{tikzpicture}[scale=.7,font=\footnotesize, line join=round, line cap=round, >=stealth]
				
				\coordinate (A) at (-1,0);
				\coordinate (B) at (6,0);
				\coordinate (D) at (2,3);
				\coordinate (C) at ($(B)+(D)-(A)$);
				\coordinate (K) at ($(A)!.5!(C)$);
				\coordinate (H) at ($(K)+(0,3)$);
				\coordinate (P) at ($(A)+(.6,.3)$);
				\coordinate (N) at ($(H)+(-2,0)$);
				\coordinate (N') at ($(H)+(3,0)$);
				\coordinate (n) at ($(K)+(-2.5,0)$);
				\coordinate (n') at ($(K)+(2.5,0)$);
				\coordinate (b) at ($(K)+(-2,1)$);
				\coordinate (b') at ($(b)!2!(K)$);
				\foreach \x/\g in {K/-90,H/90} \fill[black](\x) circle (1.5pt) ($(\x)+(\g:4mm)$) node{$\x$};
				\draw (A)--(B)--(C)--(D)--(A)
				(K)--(H)
				(N)--(N')node[above left]{$a$}
				(n)--(n')node[above left]{$a'$}
				(b)--(b')node[above left]{$b$};
				\path pic[draw,angle radius=7mm,angle eccentricity=0.5,"$P$"]{angle=B--A--D};
				\draw pic[draw,angle radius=2mm] {right angle = H--K--b}; 
				\draw pic[draw,angle radius=2mm] {right angle = H--K--n'}; 
				\draw pic[draw,angle radius=2mm] {right angle = K--H--N'}; 
			\end{tikzpicture}
		}
		\noindent
		\immini
		{
			Khi $a \perp b$, ta có thể làm như sau: Gọi mặt phẳng đi qua $b$ và vuông góc với $a$ là $(P)$, giao điểm của $a$ và $(P)$ là $H$, hình chiếu của $H$ trên $b$ là $K$, $H$ là hình chiếu của $K$ trên $a$. Khi đó $HK$ là đoạn vuông góc chung của hai đường thẳng chéo nhau $a$, $b$.
		}
		{
			\begin{tikzpicture}[scale=.7,font=\footnotesize, line join=round, line cap=round, >=stealth]
				\coordinate (A) at (-1,0);
				\coordinate (B) at (6,0);
				\coordinate (D) at (2,3);
				\coordinate (C) at ($(B)+(D)-(A)$);
				\coordinate (H) at ($(A)!.4!(C)$);
				\coordinate (K) at ($(H)+(1.5,-0.5)$);
				\coordinate (a) at ($(H)+(0,4)$);
				\coordinate (a') at ($(H)+(0,-1)$);
				\coordinate (b) at ($(K)+(1,1.5)$);
				\coordinate (b') at ($(b)!1.2!(K)$);
				\foreach \x/\g in {K/-20,H/190} \fill[black](\x) circle (1.5pt) ($(\x)+(\g:4mm)$) node{$\x$};
				\draw (A)--(B)--(C)--(D)--(A)
				(K)--(H)--(a)node[left]{$a$}
				(b)node[right]{$b$}--(b');
				\draw[dashed] (H)--(a');
				\path pic[draw,angle radius=7mm,angle eccentricity=0.5,"$P$"]{angle=B--A--D};
				\draw pic[draw,angle radius=2mm] {right angle = H--K--b}; 
				\draw pic[draw,angle radius=2mm] {right angle = a--H--K}; 
			\end{tikzpicture}
		}
	\end{nx}
\end{tomtat}

\subsection{Các dạng toán thường gặp}
\begin{dang}{Câu hỏi lí thuyết}
\end{dang}
\subsubsection{Ví dụ minh hoạ}
\begin{vd}%[1K7BP-1]
	Cho hai đường thẳng chéo nhau $a$ và $b$ thoả mãn $a \perp b$. Gọi $(P)$ là mặt phẳng đi qua $b$ và vuông góc với $a$, giao điểm của $a$ và $(P)$ là $H$, hình chiếu của $H$ trên $b$ là $K$. Chứng minh rằng $HK$ là đoạn vuông góc chung của hai đường thẳng $a$, $b$.
	\loigiai{
		\immini{
			Vì $a\perp (P)$ và $HK \in (P)$ nên $HK \perp a$.\\
			Lại có $\heva{& b \perp HK \\& b \perp a} \Rightarrow b \perp (a,K)$.\\
			Hơn nữa, $HK \in (a,K) \Rightarrow b \perp HK$.\\
			Mặt khác $H\in a$, $K\in b$.\\
			Vậy $HK$ là đoạn vuông góc chung của $a$ và $b$.
		}{
			\begin{tikzpicture}[scale=.7,font=\footnotesize, line join=round, line cap=round, >=stealth]
				\coordinate (A) at (-1,0);
				\coordinate (B) at (6,0);
				\coordinate (D) at (2,3);
				\coordinate (C) at ($(B)+(D)-(A)$);
				\coordinate (H) at ($(A)!.4!(C)$);
				\coordinate (K) at ($(H)+(1.5,-0.5)$);
				\coordinate (a) at ($(H)+(0,4)$);
				\coordinate (a') at ($(H)+(0,-1)$);
				\coordinate (b) at ($(K)+(1,1.5)$);
				\coordinate (b') at ($(b)!1.2!(K)$);
				\foreach \x/\g in {K/-20,H/190} \fill[black](\x) circle (1.5pt) ($(\x)+(\g:4mm)$) node{$\x$};
				\draw (A)--(B)--(C)--(D)--(A)
				(K)--(H)--(a)node[left]{$a$}
				(b)node[right]{$b$}--(b');
				\draw[dashed] (H)--(a');
				\path pic[draw,angle radius=7mm,angle eccentricity=0.5,"$P$"]{angle=B--A--D};
				\draw pic[draw,angle radius=2mm] {right angle = H--K--b}; 
				\draw pic[draw,angle radius=2mm] {right angle = a--H--K}; 
			\end{tikzpicture}
		}
	}
\end{vd}
\subsubsection{Bài tập rèn luyện} 
\begin{bt}%[1K7BP-1]
	Cho hai đường thẳng chéo nhau $a$ và $b$. Gọi $(P)$ là mặt phẳng chứa $b$ và song song với $a$; hình chiếu của $a$ trên $(P)$ là $a'$; giao điểm của $a'$ và $b$ là $K$; $H$ là hình chiếu của $K$ trên $a$.
	\begin{enumerate}
		\item Chứng minh $HK$ là đoạn vuông góc chung của hai đường thẳng chéo nhau $a, b$.
		\item Chứng minh $\mathrm{d}(a, b)=\mathrm{d}(a,(P))$.
	\end{enumerate}		
	\loigiai{
		\begin{center}
			\begin{tikzpicture}[scale=.7,font=\footnotesize, line join=round, line cap=round, >=stealth]
				
				\coordinate (A) at (-1,0);
				\coordinate (B) at (6,0);
				\coordinate (D) at (2,3);
				\coordinate (C) at ($(B)+(D)-(A)$);
				\coordinate (K) at ($(A)!.5!(C)$);
				\coordinate (H) at ($(K)+(0,3)$);
				\coordinate (P) at ($(A)+(.6,.3)$);
				\coordinate (N) at ($(H)+(-2,0)$);
				\coordinate (N') at ($(H)+(3,0)$);
				\coordinate (n) at ($(K)+(-2.5,0)$);
				\coordinate (n') at ($(K)+(2.5,0)$);
				\coordinate (b) at ($(K)+(-2,1)$);
				\coordinate (b') at ($(b)!2!(K)$);
				\foreach \x/\g in {K/-90,H/90} \fill[black](\x) circle (1.5pt) ($(\x)+(\g:4mm)$) node{$\x$};
				\draw (A)--(B)--(C)--(D)--(A)
				(K)--(H)
				(N)--(N')node[above left]{$a$}
				(n)--(n')node[above left]{$a'$}
				(b)--(b')node[above left]{$b$};
				\path pic[draw,angle radius=7mm,angle eccentricity=0.5,"$P$"]{angle=B--A--D};
				\draw pic[draw,angle radius=2mm] {right angle = H--K--b}; 
				\draw pic[draw,angle radius=2mm] {right angle = H--K--n'}; 
				\draw pic[draw,angle radius=2mm] {right angle = K--H--N'}; 
			\end{tikzpicture}
		\end{center}
		\begin{enumerate}
			\item Vì $\heva{&HK\perp a\\& a\parallel a'} \Rightarrow HK \perp a'$.\\
			Mặt khác $a'$ là hình chiếu vuông góc của $a$ trên $(P)$ nên $HK\perp (P) \Rightarrow HK \perp b$.\\
			Hơn nữa, $H\in a$, $K\in b$ nên $HK$ là đoạn vuông góc chung của $a$ và $b$.
			\item Vì $HK$ là đoạn đoạn vuông góc chung của $a$ và $b$ nên $\mathrm{d}(a, b)=HK$.\\
			Mặt khác $HK\perp (P), H\in a$ nên $\mathrm{d}(a,(P))=HK$.\\
			Vậy $\mathrm{d}(a, b)=\mathrm{d}(a,(P))$.
		\end{enumerate}
	}
\end{bt}
\subsubsection{Câu hỏi trắc nghiệm}
\Opensolutionfile{ans}[ans/ans-1K7-26-Dang1-2]
\begin{ex}%[1K7YP-1]
	Trong các mệnh đề sau, mệnh đề nào \textbf{sai}?
	\choice
	{Khoảng cách từ một điểm đến một đường thẳng là khoảng cách từ điểm đó đến hình chiếu vuông góc của nó trên đường thẳng kia}
	{Khoảng cách từ một điểm đến một mặt phẳng là khoảng cách từ điểm đó đến hình chiếu vuông góc của nó trên mặt phẳng kia}
	{\True Khoảng cách từ một điểm đến một mặt phẳng là khoảng cách từ điểm đó đến một điểm bất kì thuộc mặt phẳng kia}
	{Khoảng cách giữa đường thẳng $a$ và mặt phẳng $(P)$ song song với $a$ là khoảng cách từ một điểm $A$ bất kì thuộc $a$ tới mặt phẳng $(P)$}
	\loigiai{
		Theo lí thuyết.
	}
\end{ex}
%==================================
\begin{ex}%[1K7YP-1]
	Cho ba điểm $A$, $B$, $C$. Trong các khẳng định sau, khẳng định nào đúng?
	\choice
	{Khoảng cách từ $A$ đến đường thẳng $BC$ là $AB$}
	{Khoảng cách từ $A$ đến đường thẳng $BC$ là $AC$}
	{Khoảng cách từ $A$ đến đường thẳng $BC$ là $AM$, với $M$ là trung điểm $BC$}
	{\True Khoảng cách từ $A$ đến đường thẳng $BC$ là $AH$, với $H$ là hình chiếu vuông góc của $A$ trên $BC$}
	\loigiai{
		Theo lí thuyết.
	}
\end{ex}
%==================================
\begin{ex}%[1K7YP-1]
	Cho tứ diện $ABCD$. Trong các khẳng định sau, khẳng định nào đúng?
	\choice
	{Khoảng cách từ $A$ đến mặt phẳng $(BCD)$ là $AB$}
	{Khoảng cách từ $A$ đến mặt phẳng $(BCD)$ là $AC$}
	{Khoảng cách từ $A$ đến mặt phẳng $(BCD)$ là $AD$}
	{\True Khoảng cách từ $A$ đến mặt phẳng $(BCD)$ là $AH$, với $H$ là hình chiếu vuông góc của $A$ trên $(BCD)$}
	\loigiai{
		Theo lí thuyết.
	}
\end{ex}
%==================================
\begin{ex}%[1K7BP-1]
	Trong các mệnh đề sau, mệnh đề nào \textbf{sai}?
	\choice
	{\True Khoảng cách giữa hai mặt phẳng song song là khoảng cách từ một điểm $M$ bất kỳ trên mặt phẳng này đến mặt phẳng kia}
	{Nếu hai đường thẳng $a$ và $b$ chéo nhau và vuông góc với nhau thì đường vuông góc chung của chúng nằm trong mặt phẳng $(P)$ chứa đường này và $(P)$ vuông góc với đường kia}
	{Khoảng cách giữa hai đường thẳng chéo nhau $a$ và $b$ là khoảng cách từ một điểm $M$ thuộc $(P)$ chứa $a$ và song song với $b$ đến một điểm $N$ bất kì trên $b$}
	{Khoảng cách giữa đường thẳng $a$ và mặt phẳng $(P)$ song song với $a$ là khoảng cách từ một điểm $A$ bất kì thuộc $a$ tới mặt phẳng $(P)$}
	\loigiai{
		Theo lí thuyết.
	}
\end{ex}
%==================================
\begin{ex}%[1K7BP-1]
	Trong các mệnh đề sau, mệnh đề nào đúng?
	\choice
	{\True Đường vuông góc chung của hai đường thẳng chéo nhau thì vuông góc với mặt phẳng chứa đường thẳng này và song song với đường thẳng kia}
	{Một đường thẳng là đường vuông góc chung của hai đường thẳng chéo nhau nếu nó vuông góc với cả hai đường thẳng đó}
	{Đường vuông góc chung của hai đường thẳng chéo nhau thì nằm trong mặt phẳng chứa đường thẳng này và vuông góc với đường thẳng kia}
	{Một đường thẳng là đường vuông góc chung của hai đường thẳng chéo nhau nếu nó cắt cả hai đường thẳng đó}
	\loigiai{
		Theo lí thuyết.
	}
\end{ex}
%==================================
\begin{ex}%[1K7YP-1]
	Trong các mệnh đề sau, mệnh đề nào \textbf{sai}?
	\choice
	{Nếu hai đường thẳng $a$ và $b$ chéo nhau và vuông góc với nhau thì đường thẳng vuông góc chung của chúng nằm trong mặt phẳng $(P)$ chứa đường thẳng này và vuông góc với đường thẳng kia}
	{Khoảng cách giữa đường thẳng $a$ và mặt phẳng $(P)$ song song với $a$ là khoảng cách từ một điểm $A $ bất kỳ thuộc $a$ tới $(P)$}
	{\True Khoảng cách giữa hai đường thẳng chéo nhau $a$ và $b$ là khoảng cách từ một điểm $M$ thuộc mặt phẳng $(P)$ chứa $a$ và song song với $b$ đến một điểm $N$ bất kỳ trên $b$}
	{Khoảng cách giữa hai mặt phẳng song song là khoảng cách từ một điểm $M$ bất kỳ trên mặt phẳng này đến mặt phẳng kia}
	\loigiai{
		Theo lí thuyết.
	}
\end{ex}
%==================================
\begin{ex}%[1K7BP-1]
	Cho điểm $M$ và đường thẳng $\Delta$ thoả mãn $M\not \in \Delta$. Với mọi điểm $N \in\Delta$, khẳng định nào sau đây đúng?
	\choice
	{\True $\mathrm{d}(M,\Delta) \le MN$}
	{$\mathrm{d}(M,\Delta) < MN$}
	{$\mathrm{d}(M,\Delta) \ge MN$}
	{$\mathrm{d}(M,\Delta) > MN$}
	\loigiai{
		Với mọi điểm $N \in\Delta$ ta có $\mathrm{d}(M,\Delta) \le MN$.
	}
\end{ex}
%==================================
\begin{ex}%[1K7BP-1]
	Cho hai điểm cố định $A$ và $B$. Gọi $d$ là đường thẳng đi qua $B$ sao cho khoảng cách từ $A$ đến $d$ là lớn nhất. Khẳng định nào sau đây đúng?
	\choice
	{\True $d\perp AB$}
	{$d$ tạo với đường thẳng $AB$ góc $30^\circ$}
	{$d$ tạo với đường thẳng $AB$ góc $45^\circ$}
	{$d$ tạo với đường thẳng $AB$ góc $60^\circ$}
	\loigiai{
		\immini{
			Gọi $H$ là hình chiếu của $A$ trên $d$. Khi đó, ta luôn có $\mathrm{d} (A,d) =AH \le AB$.\\
			Do đó $\mathrm{d} (A,d)$ đạt giá trị lớn nhất khi và chỉ khi $H\equiv B$, tức là $d \perp AB$.}
		{\begin{tikzpicture}[scale=.7,font=\footnotesize, line join=round, line cap=round, >=stealth]
				\coordinate (M) at (0,0);
				\coordinate (N) at (5,0);
				\coordinate (H) at ($(M)!.3!(N)$);
				\coordinate (B) at ($(M)!.6!(N)$);
				\coordinate (A) at ($(H)+(0,3)$);
				\foreach \x/\g in {H/-90,A/90,B/-90} \fill[black](\x) circle (1.5pt) ($(\x)+(\g:3mm)$) node{$\x$};
				\draw (M)--(N)node[above left]{$d$}
				(A)--(H) (A)--(B);
				\draw pic[draw,angle radius=2mm] {right angle = A--H--B}; 
			\end{tikzpicture}
		}
	}
\end{ex}
%==================================
\begin{ex}%[1K7BP-1]
	Cho hai điểm $M$, $N$ cố định. Gọi $(P)$ là mặt phẳng đi qua $N$ sao cho khoảng cách từ $M$ đến $(P)$ là lớn nhất. Khẳng định nào sau đây đúng?
	\choice
	{\True $(P)\perp MN$}
	{$(P)$ tạo với đường thẳng $MN$ góc $30^\circ$}
	{$(P)$ tạo với đường thẳng $MN$ góc $45^\circ$}
	{$(P)$ tạo với đường thẳng $MN$ góc $60^\circ$}
	\loigiai{
		\immini{
			Gọi $H$ là hình chiếu của $M$ trên $(P)$.\\
			Khi đó, ta luôn có $\mathrm{d} (M,(P)) =MH \le MN$.\\
			Do đó $\mathrm{d} (M,(P))$ đạt giá trị lớn nhất khi và chỉ khi $H\equiv N$, tức là $(P) \perp MN$.
		}{
			\begin{tikzpicture}[scale=.7,font=\footnotesize, line join=round, line cap=round, >=stealth]
				\coordinate (A) at (0,0);
				\coordinate (B) at (5,0);
				\coordinate (D) at (2,3);
				\coordinate (C) at ($(B)+(D)-(A)$);
				\coordinate (H) at ($(A)!.5!(C)$);
				\coordinate (T) at ($(B)!.5!(C)$);
				\coordinate (N) at ($(H)!.7!(T)$);
				\coordinate (M) at ($(H)+(0,3)$);
				\coordinate (P) at ($(A)+(.6,.3)$);
				\foreach \x/\g in {H/180,M/90,N/-90} \fill[black](\x) circle (1.5pt) ($(\x)+(\g:3mm)$) node{$\x$};
				\draw (A)--(B)--(C)--(D)--(A) (M)--(H) (M)--(N)--(H);
				\path pic[draw,angle radius=7mm,angle eccentricity=0.5,"$P$"]{angle=B--A--D};
				\draw pic[draw,angle radius=2mm] {right angle = M--H--N}; 
			\end{tikzpicture}
		}
	}
\end{ex}
%==================================
\begin{ex}%[1K7BP-1]
	Cho điểm $M$ và mặt phẳng $(P)$ thoả mãn $M\not \in (P)$. Với mọi điểm $N \in (P)$, khẳng định nào sau đây đúng?
	\choice
	{\True $\mathrm{d}(M,(P)) \le MN$}
	{$\mathrm{d}(M,(P)) < MN$}
	{$\mathrm{d}(M,(P)) \ge MN$}
	{$\mathrm{d}(M,(P)) > MN$}
	\loigiai{
		Với mọi điểm $N \in\Delta$ ta có $\mathrm{d}\mathrm{d}(M,(P)) \le MN$.
	}
\end{ex}
%==================================
\begin{ex}%[1K7BP-1]
	Cho đường thẳng $\Delta$ và mặt phẳng $(P)$ thoả mãn $\Delta \parallel (P)$. Với mọi điểm $M \in\Delta$ và $N \in (P)$, khẳng định nào sau đây đúng?
	\choice
	{\True $\mathrm{d}(\Delta, (P)) \le MN$}
	{$\mathrm{d}(\Delta, (P)) < MN$}
	{$\mathrm{d}(\Delta, (P)) \ge MN$}
	{$\mathrm{d}(\Delta, (P)) > MN$}
	\loigiai{
		Với mọi điểm $M \in\Delta$ và $N \in (P)$ ta có $\mathrm{d}(\Delta, (P)) \le MN$.
	}
\end{ex}
%==================================
\begin{ex}%[1K7BP-1]
	Cho hai đường thẳng $\Delta_1$ và $\Delta_2$ thoả mãn $\Delta_1 \parallel \Delta_2$. Với mọi điểm $M \in\Delta_1$ và $N \in \Delta_2$, khẳng định nào sau đây đúng?
	\choice
	{\True $\mathrm{d}(\Delta_1,\Delta_2) \le MN$}
	{$\mathrm{d}(\Delta_1,\Delta_2) < MN$}
	{$\mathrm{d}(\Delta_1,\Delta_2) \ge MN$}
	{$\mathrm{d}(\Delta_1,\Delta_2) > MN$}
	\loigiai{
		Với mọi điểm $M \in\Delta_1$ và $N \in \Delta_2$ ta có $\mathrm{d}(\Delta_1,\Delta_2) \le MN$.
	}
\end{ex}
%==================================
\begin{ex}%[1K7BP-1]
	Cho hai mặt phẳng $(P)$ và $(Q)$ thoả mãn $(P) \parallel (Q)$. Với mọi điểm $M\in (P)$ và $N\in (Q)$, khẳng định nào sau đây đúng?
	\choice
	{\True $\mathrm{d}((P),(Q)) \le MN$}
	{$\mathrm{d}((P),(Q)) < MN$}
	{$\mathrm{d}((P),(Q)) \ge MN$}
	{$\mathrm{d}((P),(Q)) > MN$}
	\loigiai{
		Với mọi điểm $M\in (P)$ và $N\in (Q)$ ta có $\mathrm{d}((P),(Q)) \le MN$.
	}
\end{ex}
%==================================
\begin{ex}%[1K7BP-1]
	Cho hai đường thẳng chéo nhau $\Delta_1$ và $\Delta_2$. Với mọi điểm $M \in\Delta_1$ và $N \in \Delta_2$, khẳng định nào sau đây đúng?
	\choice
	{\True $\mathrm{d}(\Delta_1,\Delta_2) \le MN$}
	{$\mathrm{d}(\Delta_1,\Delta_2) < MN$}
	{$\mathrm{d}(\Delta_1,\Delta_2) \ge MN$}
	{$\mathrm{d}(\Delta_1,\Delta_2) > MN$}
	\loigiai{
		Với mọi điểm $M \in\Delta_1$ và $N \in \Delta_2$ ta có $\mathrm{d}(\Delta_1,\Delta_2) \le MN$.
	}
\end{ex}
%==================================
\Closesolutionfile{ans}
\begin{indapan}{10}
	{ans/ans-1K7-26-Dang1-2}
\end{indapan}


