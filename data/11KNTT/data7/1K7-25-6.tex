\Opensolutionfile{ans}[ans/ans-1K7-25-Dang6]
\begin{dang}{Xác định thiết diện với mặt phẳng $(P)$ biết $(P)$ qua điểm  $M$ và $(P)$ vuông góc với $d$}
	Phương pháp
	\begin{itemize}
		\item Tìm $2$ đường thẳng  $a$, $b$ cắt nhau cùng vuông góc với $d$, trong đó có $1$ đường đi qua điểm $M$.
		\item Khi đó $(P)$ là mp chứa $a$, $b$.
		\item Ta tìm thiết diện của $(P)$ với khối đa diện.
	\end{itemize}
\end{dang}
\begin{dang}{ Xác định thiết diện với mặt phẳng  $(P)$ biết $(P)$ chứa đường thẳng $a$  và  $(P)\perp (Q)$}
	Phương pháp
	\begin{itemize}
		\item Tìm $1$ điểm trên thẳng  $a$ ta dựng đường thẳng $b$ vuông góc với $Q$.
		\item Khi đó $(P)$ là mp chứa $a$, $b$.
		\item Ta tìm thiết diện của $(P)$ với khối đa diện.
	\end{itemize}
\end{dang}
\subsubsection{Ví dụ mẫu}
%ví dụ 1
\begin{vd}
	Cho hình chóp $S.ABCD$  có đáy $ABCD$ là hình vuông cạnh $a$, $SA\perp (ABCD)$ và  $SA=a\sqrt{2}$. Gọi $H$ là hình chiếu vuông góc của $A$ lên $SB$. Xác định và tính diện tích thiết diện tạo bởi hình chóp $S.ABCD$ và mặt phẳng $(P)$ biết $(P)$  qua $H$ và  $(P)\perp SB$.
	\loigiai{
		\immini{
			Ta có $AH\perp SB$.\\
			$\heva{&AD\perp AB\\&AD\perp SA}\Rightarrow AD\perp (SAB)\Rightarrow AD\perp SB$.\\
			Suy ra $(P)$ là mặt phẳng $(AHD)$.\\
			Kẻ đường thẳng qua $H$ và song song với $BC$ cắt $SC$ tại $K$.\\
			Thiết diện của hình chóp $S.ABCD$ với mặt phẳng $(P)$ là hình thang $ADKH$ vuông tại $A$ và $H$.\\
			$SB=a\sqrt{3}$, $SC=2a$.\\
			$\dfrac{1}{AH^2}=\dfrac{1}{SA^2}+\dfrac{1}{AB^2}\Rightarrow AH=\dfrac{\sqrt{6}}{3}a $.\\
			$\dfrac{HK}{BC}=\dfrac{SH}{SB}=\dfrac{SA^2}{SB^2}=\dfrac{2}{3}\Rightarrow HK=\dfrac{2}{3}a$.
		}
		{
			\begin{tikzpicture}[declare function={a=2.65;b=4.5;h=3;},line join=round]
				\path (0,0) coordinate (A)
				(-135:a) coordinate (B)
				(b,0) coordinate (D)
				(0,h) coordinate (S)
				($(D)-(A)+(B)$) coordinate (C)
				($(S)!0.3!(B)$)  coordinate (H)
				($(S)!0.3!(C)$)  coordinate (K)
				;
				\fill [blue,opacity=0.3] (D)--(K)--(H)--(A);
				\draw[dashed] (S)--(A)--(B) (A)--(D) (A)--(H);
				\draw (B)--(C)--(D) (B)--(S)  (D)--(S)--(C) (D)--(K)--(H);
				\foreach \t/\g in {A/190,B/-90,C/-90,D/0,S/90,H/180,K/60}{
					\draw[fill=white] (\t) circle (1pt) node[shift={(\g:7pt)},font=\scriptsize]{$ \t $};
				}
			\end{tikzpicture}
		}
		\noindent
		Suy ra $S_{ADKH}=\dfrac{1}{2}AH\cdot (AD+HK)= \dfrac{1}{2}\cdot \dfrac{\sqrt{6}}{3}\left(1+\dfrac{2}{3}\right)=\dfrac{5\sqrt{6}}{18}a^2$.
	}
\end{vd}
%ví dụ 2
\begin{vd}
	Cho hình chóp $S.ABCD$  có đáy $ABCD$ là hình vuông tâm $O$, cạnh $a$, $SA\perp (ABCD)$ và  $SA=a\sqrt{2}$. Xác định và tính diện tích thiết diện tạo bởi hình chóp $S.ABCD$ và mặt phẳng $(P)$ biết  $(P)$ là mặt phẳng qua $B$ và  $(P)\perp SC$.
	\loigiai{
		\immini{
			Ta có 	$\heva{&BD\perp AC\\&BD\perp SA}\Rightarrow BD\perp (SAC)\Rightarrow BD\perp SC$.\\
			Kẻ $OE\perp SC$ tại $E$. Suy ra $(P)$ là mặt phẳng $(EBD)$.\\
			Thiết diện của hình chóp $S.ABCD$ với mặt phẳng $(P)$ là tam giác $EBD$.\\
			Vì $BD\perp (SAC)\Rightarrow BD\perp OE$.\\
			Gọi $E'$ là trung điểm $SC$, mà $\triangle SAC$ vuông cân tại $A$ nên $AE'=\dfrac{SC}{2}=a$,  $OE=\dfrac{AE'}{2}=\dfrac{a}{2}$.\\
			$S_{EBD}=\dfrac{1}{2}OE\cdot BD=\dfrac{\sqrt{2}a^2}{2}$.
		}
		{
			\begin{tikzpicture}[declare function={a=2.65;b=4.5;h=4.5;},line join=round]
				\path (0,0) coordinate (A)
				(-135:a) coordinate (B)
				(b,0) coordinate (D)
				(0,h) coordinate (S)
				($(D)-(A)+(B)$) coordinate (C)
				($(B)!.5!(D)$) coordinate (O)
				($(S)!.5!(C)$)  coordinate (E')
				($(E')!.5!(C)$) coordinate (E)
				;
				\fill [blue,opacity=0.3] (B)--(E)--(D);
				\draw[dashed] (S)--(A)--(B) (A)--(D) (A)--(C) (B)--(D) (O)--(E) (A)--(E');
				\draw (B)--(C)--(D) (B)--(S)  (D)--(S)--(C) (B)--(E)--(D);
				\foreach \t/\g in {A/180,B/-90,C/-90,D/0,S/90,E/160,O/-90,E'/0}{
					\draw[fill=white] (\t) circle (1pt) node[shift={(\g:7pt)},font=\scriptsize]{$ \t $};
				}
			\end{tikzpicture}	
		}
	}
\end{vd}
%ví dụ 3
\begin{vd}
	Cho hình chóp $S.ABCD$.  có đáy $ABCD$ là hình vuông cạnh $a$, $SA\perp (ABCD)$ và  $SA=a\sqrt{2}$. Xác định và tính diện tích thiết diện tạo bởi hình chóp $S.ABCD$ và mặt phẳng $(P)$ biết  $(P)$ là mặt phẳng chứa $AB$ và  $(P)\perp (SCD)$.
	\loigiai{
		\immini{
			Ta có 	$\heva{&CD\perp AD\\&CD\perp SA}\Rightarrow CD\perp (SAD)$.\\
			Kẻ $AE\perp SD$ tại $E$. Mà $AE\perp CD$ (vì $CD\perp (SAD)$) nên $AE\perp (SCD$.\\ Suy ra $(P)$ là mặt phẳng $(AEB)$.\\
			Từ $E$ kẻ đường thẳng song song $AB$, cắt $SC$ tại $F$.
			Thiết diện của hình chóp $S.ABCD$ với mặt phẳng $(P)$ là hình thang $AEFB$ vuông tại $E$.\\
			Vì $BD\perp (SAC)\Rightarrow BD\perp OE$.\\
			Ta có 
			\begin{itemize}
				\item $AE=AH=\dfrac{\sqrt{6}}{3}a$
				\item $\dfrac{EF}{CD}=\dfrac{SE}{SD}=\dfrac{SA^2}{SD^2}=\dfrac{2}{3}\Rightarrow EF=\dfrac{2}{3}a$.
			\end{itemize}
		}
		{
			\begin{tikzpicture}[declare function={a=2.65;b=4.5;h=3;},line join=round]
				\path (0,0) coordinate (A)
				(-135:a) coordinate (D)
				(b,0) coordinate (B)
				(0,h) coordinate (S)
				($(B)-(A)+(D)$) coordinate (C)
				($(S)!0.3!(D)$) coordinate (E)
				($(S)!0.3!(C)$) coordinate (F)
				;
				\fill [blue,opacity=0.3] (A)--(B)--(F)--(E);
				\draw[dashed] (S)--(A)--(D) (A)--(B) (A)--(E);
				\draw (D)--(C)--(B) (D)--(S)  (B)--(S)--(C) (E)--(F)--(B);
				\foreach \t/\g in {A/180,D/-90,C/-90,B/0,S/90,E/180,F/30}{
					\draw[fill=white] (\t) circle (1pt) node[shift={(\g:7pt)},font=\scriptsize]{$ \t $};
				}
			\end{tikzpicture}	
		}
		\noindent
		$S_{AEFB}=\dfrac{1}{2}AE\cdot (EF+AB)=\dfrac{1}{2}\cdot \dfrac{\sqrt{6}}{3}\cdot\left (1+\dfrac{1}{2}\right)=\dfrac{5\sqrt{6}}{18}a^2$.	
	}
\end{vd}
%ví dụ 4
\begin{vd}
	Cho hình chóp $S.ABCD$ có đáy $ABCD$ là hình vuông tâm $O$, cạnh $a$, $SA\perp (ABCD)$ và  $SA=a\sqrt{2}$. Xác định và tính diện tích thiết diện tạo bởi hình chóp $S.ABCD$ và mặt phẳng $(P)$ biết $(P)$ là mặt phẳng qua $A$ và  $(P)\perp SC$.
	\loigiai{
		\immini{
			Gọi $E$ là trung điểm $SC$, mà $\triangle SAC$ vuông cân tại $A$ nên $AE\perp SC$.\\
			Gọi $I=AE\cap SO$.\\
			Kẻ đường thẳng qua  $I$, song song $BD$, cắt $SD$, $SB$ tại $N$, $M$.	\\
			Mà $BD\perp (SAC)\Rightarrow BD\perp SC$ nên $MN\perp SC$.\\
			Suy ra  $(P)$ là mặt phẳng $(ANE)$.\\
			Thiết diện của hình chóp $S.ABCD$ tạo bởi $(P)$ là hình tứ giác $AMEN$ có $2$ đường chéo vuông góc.
			Ta có 
			\begin{itemize}
				\item $MN=\dfrac{2}{3}BD=\dfrac{2a\sqrt{2}}{3}a$.
				\item $S_{AMEN}=\dfrac{1}{2}AE\cdot MN =\dfrac{1}{2}a\cdot \dfrac{2a\sqrt{2}}{3}=\dfrac{\sqrt{2}}{3}a^2$.	
			\end{itemize}
		}
		{
			\begin{tikzpicture}[declare function={a=2.65;b=4.5;h=3;},line join=round]
				\path (0,0) coordinate (A)
				(-135:a) coordinate (B)
				(b,0) coordinate (D)
				(0,h) coordinate (S)
				($(D)-(A)+(B)$) coordinate (C)
				($(B)!.5!(D)$) coordinate (O)
				($(S)!0.3!(C)$) coordinate (E)
				(intersection of S--O and A--E) coordinate (I)
				($(I)+(D)-(O)$) coordinate (x)
				(intersection of I--x and S--D) coordinate (N)
				(intersection of I--x and S--B) coordinate (M)
				;
				\fill [blue,opacity=0.3] (A)--(M)--(E)--(N);
				\draw[dashed] (S)--(A)--(B) (A)--(D) (A)--(C) (B)--(D) (S)--(O) (A)--(E) (M)--(A)--(N)--(M);
				\draw (B)--(C)--(D) (B)--(S)  (D)--(S)--(C) (M)--(E)--(N);
				\foreach \t/\g in {A/190,B/-90,C/-90,D/0,S/90,O/-90,E/60,I/-40,M/180,N/0}{
					\draw[fill=white] (\t) circle (1pt) node[shift={(\g:7pt)},font=\scriptsize]{$ \t $};
				}
			\end{tikzpicture}	
		}
	}
\end{vd}
%ví dụ 5
\begin{vd}
	Cho hình chóp $S.ABCD$  có đáy $ABCD$ là hình vuông tâm $O$, cạnh $a$, $SA\perp (ABCD)$ và  $SA=a\sqrt{2}$. Gọi $M$ là trung điểm $BO$. Xác định và tính diện tích thiết diện tạo bởi hình chóp $S.ABCD$ và mặt phẳng $(P)$ biết  $(P)$ qua $M$ và  $(P)\perp BC$.
	\loigiai{
		\immini{
			Từ $M$ kẻ đường thẳng song song $CD$, cắt $AD$, $BC$ tại $Q$, $P$. 	\\
			Suy ra $QP\perp BC$.\\
			Từ $P$, kẻ đường thẳng song song $SB$, cắt $SC$ tại $F$.\\
			Vì $BC\perp SB $ nên $BC\perp PF$.\\
			Suy ra  $(P)$ là mặt phẳng $(QPF)$.\\
			Từ $F$, kẻ đường thẳng song song $CD$, cắt $SD$ tại $E$.\\
			Thiết diện của hình chóp $S.ABCD$ tạo bởi (P) là hình thang $QPFE$ vuông tại $Q$.
			Ta có 
			\begin{itemize}
				\item $\dfrac{QE}{SA}=\dfrac{DQ}{DA}=\dfrac{DM}{DB}=\dfrac{3}{4}\Rightarrow QE=\dfrac{3a\sqrt{2}}{4}$.
				\item $\dfrac{FE}{CD}=\dfrac{SE}{SD}=\dfrac{1}{4}\Rightarrow EF=\dfrac{a}{4}$.
			\end{itemize}
		}
		{
			\begin{tikzpicture}[declare function={a=2.65;b=4.5;h=3;},line join=round]
				\path (0,0) coordinate (A)
				(-135:a) coordinate (D)
				(b,0) coordinate (B)
				(0,h) coordinate (S)
				($(B)-(A)+(D)$) coordinate (C)
				($(B)!.5!(D)$) coordinate (O)
				($(B)!0.3!(C)$) coordinate (P)
				($(A)!0.3!(D)$) coordinate (Q)
				(intersection of P--Q and D--B) coordinate (M)
				($(P)+(S)-(B)$) coordinate (x)
				(intersection of P--x and S--C) coordinate (F)
				($(F)+(D)-(C)$) coordinate (y)
				(intersection of F--y and S--D) coordinate (E)
				;
				\fill [blue,opacity=0.3] (P)--(Q)--(E)--(F);
				\draw[dashed] (S)--(A)--(D) (A)--(B) (A)--(C) (B)--(D) (P)--(Q)--(E);
				\draw (D)--(C)--(B) (D)--(S)  (B)--(S)--(C) (E)--(F)--(P) ;
				\foreach \t/\g in {A/190,D/-90,C/-90,B/0,S/90,O/-90,P/0,Q/180,M/-90,F/30,E/180}{
					\draw[fill=white] (\t) circle (1pt) node[shift={(\g:7pt)},font=\scriptsize]{$ \t $};
				}
			\end{tikzpicture}	
		}
		\noindent
		$S_{QPFE}=\dfrac{1}{2}QE\cdot (EF+QP)=\dfrac{1}{2}\cdot \dfrac{3a\sqrt{2}}{4}\cdot\left (\dfrac{a}{4}+a\right)=\dfrac{15\sqrt{2}}{32}a^2$.	
	}
\end{vd}
%ví dụ 6
\begin{vd}
	Cho hình chóp $S.ABCD$  có đáy $ABCD$ là hình vuông tâm $O$, cạnh $a$, $SA\perp (ABCD)$ và  $SA=a\sqrt{2}$. Gọi $M$ là trung điểm $AB$. Xác định và tính diện tích thiết diện tạo bởi hình chóp $S.ABCD$ và mặt phẳng $(P)$ biết  $(P)$ qua $M$ và  $(P)\perp AB$.
	\loigiai{
		\immini{
			Gọi $OM\cap CD=T$ suy ra $MT\perp AB$.\\
			Gọi $N$ là trung điểm $SB$. Suy ra  $MN\parallel SA\Rightarrow MN\perp AB$
			Suy ra  là mặt phẳng $(MNT)$.\\
			Qua $N$, kẻ đường thẳng song song $BC$ cắt $SC$ tại $K$.
			Thiết diện của hình chóp $S.ABCD$ tạo bởi $(P)$ là hình thang $MNKT$ vuông tại $M$.
			Ta có 
			\begin{itemize}
				\item $MN=\dfrac{SA}{2}=\dfrac{a\sqrt{2}}{2}$.
				\item $NK=\dfrac{BC}{2}=\dfrac{a}{2}$.
			\end{itemize}	
		}
		{
			\begin{tikzpicture}[declare function={a=2.65;b=4.5;h=3;},line join=round]
				\path (0,0) coordinate (A)
				(-135:a) coordinate (D)
				(b,0) coordinate (B)
				(0,h) coordinate (S)
				($(B)-(A)+(D)$) coordinate (C)
				($(B)!.5!(D)$) coordinate (O)
				($(B)!0.5!(A)$) coordinate (M)
				(intersection of O--M and D--C) coordinate (T)
				($(S)!0.5!(B)$) coordinate (N)
				($(S)!0.5!(C)$) coordinate (K)
				;
				\fill [blue,opacity=0.3] (M)--(N)--(K)--(T);
				\draw[dashed] (S)--(A)--(D) (A)--(B) (A)--(C) (B)--(D) (T)--(M)--(N);
				\draw (D)--(C)--(B) (D)--(S)  (B)--(S)--(C) (T)--(K)--(N);
				\foreach \t/\g in {A/190,D/-90,C/-90,B/0,S/90,O/-90,M/-90,N/30,T/-90,K/190}{
					\draw[fill=white] (\t) circle (1pt) node[shift={(\g:7pt)},font=\scriptsize]{$ \t $};
				}
			\end{tikzpicture}	
		}
		\noindent
		$S_{MNKT}=\dfrac{1}{2}MN\cdot (NK+MT)=\dfrac{1}{2}\cdot \dfrac{a\sqrt{2}}{2}\cdot\left (\dfrac{a}{2}+a\right)=\dfrac{3\sqrt{2}}{8}a^2$.	
	}
\end{vd}
%ví dụ 7
\begin{vd}
	Cho hình chóp $S.ABCD$  có đáy $ABCD$ là hình vuông tâm $O$, cạnh $a$, $SA\perp (ABCD)$ và  $SA=a\sqrt{2}$.  Xác định và tính diện tích thiết diện tạo bởi hình chóp $S.ABCD$ và mặt phẳng $(P)$ biết  $(P)$ qua $A$ và  $(P)\perp SO$.
	\loigiai{
		\immini{
			Kẻ đường thẳng qua $A$, vuông góc $SO$, cắt $SO$ tại $I$, cắt $SC$ tại $N$.\\
			Qua $I$, kẻ đường thẳng song song $BD$, cắt $SB$, $SD$ tại $E$, $F$.\\
			Vì  $BD\perp (SAC)\Rightarrow BD\perp SO\Rightarrow FE\perp SO$.
			Suy ra $(P)$ là mặt phẳng $(AEF)$.\\
			Thiết diện của hình chóp $S.ABCD$ tạo bởi $(P)$ là hình tứ giác $AENF$ có hai đường chéo vuông góc.
			Ta có 
			\begin{itemize}
				\item $\dfrac{1}{AI^2}=\dfrac{1}{SA^2}+\dfrac{1}{AO^2}\Rightarrow AI=\dfrac{\sqrt{10}}{5}a$.
			\end{itemize}	
			
		}
		{
			\begin{tikzpicture}[declare function={a=2.65;b=4.5;h=3;},line join=round]
				\path (0,0) coordinate (A)
				(-135:a) coordinate (D)
				(b,0) coordinate (B)
				(0,h) coordinate (S)
				($(B)-(A)+(D)$) coordinate (C)
				($(B)!.5!(D)$) coordinate (O)
				($(B)!0.5!(A)$) coordinate (M)
				($(S)!0.3!(C)$) coordinate (N)
				(intersection of A--N and S--O) coordinate (I)
				($(I)+(B)-(O)$) coordinate (x)
				(intersection of I--x and S--B) coordinate (E)
				(intersection of I--x and S--D) coordinate (F)
				;
				\fill [blue,opacity=0.3] (A)--(E)--(N)--(F);
				\draw[dashed] (S)--(A)--(D) (A)--(B) (A)--(C) (B)--(D) (S)--(O) (F)--(A)--(E) (A)--(N) (E)--(F);
				\draw (D)--(C)--(B) (D)--(S)  (B)--(S)--(C) (F)--(N)--(E) ;
				\foreach \t/\g in {A/190,D/-90,C/-90,B/0,S/90,O/-90,M/-90,N/30,I/-30,E/30,F/160}{
					\draw[fill=white] (\t) circle (1pt) node[shift={(\g:7pt)},font=\scriptsize]{$ \t $};
				}
			\end{tikzpicture}	
		}
		\begin{itemize}
			\item $SI=\sqrt{SA^2-AI^2}= \dfrac{2\sqrt{10}}{5}a$;  $OI=\sqrt{OA^2-AI^2}= \dfrac{\sqrt{10}}{10}a$.
			\item $\dfrac{AO}{AC}\cdot \dfrac{CN}{SN}\cdot \dfrac{SI}{OI}=1\Rightarrow \dfrac{1}{2}\cdot \dfrac{CN}{SN}\cdot4=1\Rightarrow \dfrac{CN}{SN}=\dfrac{1}{2}$.\\
			Suy ra $CN=\dfrac{1}{3}SC=\dfrac{2}{3}a\Rightarrow AN=\sqrt{AC^2+CN^2-2\cdot AC\cdot CN\cdot\cos 45^\circ}= \dfrac{\sqrt{10}}{3}a$.
			\item $\dfrac{FE}{BD}=\dfrac{SI}{SO}= \dfrac{2\dfrac{\sqrt{10}}{5}a}{\dfrac{\sqrt{10}}{2}a}=\dfrac{4}{5}\Rightarrow FE=\dfrac{4a\sqrt{2}}{5}$.
		\end{itemize}	
		$S_{AENF}=\dfrac{1}{2}FE\cdot AN =\dfrac{1}{2}\cdot \dfrac{4a\sqrt{2}}{5}\cdot \dfrac{\sqrt{10}}{3}a=\dfrac{4\sqrt{5}}{15}a^2$.	
	}
\end{vd}

\subsubsection{Bài tập rèn luyện}
%Câu 1
\begin{ex}
	Cho hình chóp $S.ABCD$, đáy $ABCD$ là hình vuông,  $SA\perp (ABCD)$. Gọi  $(\alpha)$ là mặt phẳng chứa $AB$ và vuông góc với  $(SCD)$,  $(\alpha)$ cắt hình chóp theo thiết diện là hình gì?
	\choice
	{Hình bình hành}
	{\True Hình thang vuông}
	{Hình thang không vuông}
	{Hình chữ nhật}
	\loigiai{
		\immini{
			Ta có 	$\heva{&CD\perp AD\\&CD\perp SA}\Rightarrow CD\perp (SAD)$.\\
			Kẻ $AH\perp SD$ tại $H$. Mà $AH\perp CD$ (vì $CD\perp (SAD)$) nên $AH\perp (SCD$.\\ Suy ra $(P)$ là mặt phẳng $(AHB)$.\\
			Từ $H$ kẻ đường thẳng song song $AB$, cắt $SC$ tại $K$. Vì $AH\perp (SCD)$ nên $AH\perp HK$.
			Thiết diện của hình chóp $S.ABCD$ với mặt phẳng $(P)$ là hình thang $ABKH$ vuông tại $A$, $H$.
		}
		{
			\begin{tikzpicture}[declare function={a=2.65;b=4.5;h=3;},line join=round]
				\path (0,0) coordinate (A)
				(-135:a) coordinate (D)
				(b,0) coordinate (B)
				(0,h) coordinate (S)
				($(B)-(A)+(D)$) coordinate (C)
				($(S)!0.3!(D)$) coordinate (E)
				($(S)!0.3!(C)$) coordinate (F)
				;
				\fill [blue,opacity=0.3] (A)--(B)--(F)--(E);
				\draw[dashed] (S)--(A)--(D) (A)--(B) (A)--(E);
				\draw (D)--(C)--(B) (D)--(S)  (B)--(S)--(C) (E)--(F)--(B);
				\foreach \t/\g in {A/180,D/-90,C/-90,B/0,S/90,E/180,F/30}{
					\draw[fill=white] (\t) circle (1pt) node[shift={(\g:7pt)},font=\scriptsize]{$ \t $};
				}
			\end{tikzpicture}		
		}
	}
\end{ex}
%Câu 2
\begin{ex}
	Cho hình chóp $S.ABCD$, đáy $ABCD$ là hình chữ nhật tâm $O$, có  $AB=a$, $AD=2a$, $SA\perp (ABCD)$   và  $SA=a$. Gọi $(P)$  là mặt phẳng qua $SO$ và vuông góc với  $(SAD)$. Diện tích thiết diện của $ (P)$ và hình chóp bằng bao nhiêu?
	\choice
	{$\dfrac{a^2\sqrt{3}}{2}$}
	{\True $\dfrac{a^2\sqrt{2}}{2}$}
	{$\dfrac{a^2}{2}$}
	{$a^2 $}
	\loigiai{
		\immini{
			Kẻ đường thẳng qua $O$, song song với $AB$, cắt $AD$, $BC$ lần lượt tại $E$, $F$.	\\
			Ta có 	$\heva{&CD\perp AD\\&CD\perp SA}\Rightarrow CD\perp (SAD)$.\\
			Mà $EF\parallel CD $ nên $EF\perp (SAD)$.\\
			Suy ra $(P)$ là mặt phẳng $(SEF)$.\\
			Thiết diện của hình chóp $S.ABCD$ với mặt phẳng $(P)$ là tam giác $SEF$.\\
			$S_{SEF}=\dfrac{1}{2}SE\cdot EF=\dfrac{1}{2}\sqrt{a^2+a^2}\cdot a=\dfrac{\sqrt{2}}{2}a^2$.
		}
		{
			\begin{tikzpicture}[declare function={a=2.65;b=4.5;h=3;},line join=round]
				\path (0,0) coordinate (A)
				(-135:a) coordinate (D)
				(b,0) coordinate (B)
				(0,h) coordinate (S)
				($(B)-(A)+(D)$) coordinate (C)
				($(B)!.5!(D)$) coordinate (O)
				($(B)!0.5!(C)$) coordinate (F)
				($(A)!0.5!(D)$) coordinate (E)
				;
				\fill [blue,opacity=0.3] (S)--(E)--(F);
				\draw[dashed] (S)--(A)--(D) (A)--(B) (A)--(C) (B)--(D) (S)--(O) (S)--(E)--(F);
				\draw (D)--(C)--(B) (D)--(S)  (B)--(S)--(C) (S)--(F);
				\foreach \t/\g in {A/190,D/-90,C/-90,B/0,S/90,O/-90,F/-90,E/190}{
					\draw[fill=white] (\t) circle (1pt) node[shift={(\g:7pt)},font=\scriptsize]{$ \t $};
				}
			\end{tikzpicture}		
		}
	}
\end{ex}
%Câu 3
\begin{ex}
	Cho hai mặt phẳng vuông góc $(P)$ và $(Q)$   có giao tuyến  $\Delta$. Lấy $A$, $B$ cùng thuộc  $\Delta$ và lấy $C$ trên  $(P)$, D trên $(Q)$  sao cho $AC\perp AB$, $BD\perp AB$  và  $AB=AC=BD=a$. Diện tích thiết diện của tứ diện $ABCD$ khi cắt bởi mặt phẳng $\alpha$  đi qua $A$ và vuông góc với $CD$ là
	\choice
	{$\dfrac{a^2\sqrt{2}}{12}$}
	{$\dfrac{a^2\sqrt{2}}{8}$}
	{\True $\dfrac{a^2\sqrt{3}}{12}$}
	{$\dfrac{a^2\sqrt{3}}{8}$}
	\loigiai{
		\immini{
			Ta có $\heva{&(P)\cap (Q)=AB\\&(P)\perp (Q)\\&BD\subset (Q)\\&BD\perp AB}\Rightarrow BD\perp  (P)\equiv (ABC)$.\\
			Gọi $E$ là trung điểm $BC$ và tam giác $ABC$ cân tại $A$ nên  $AE\perp BC$.\\
			Lại có $AE\perp BD$ (vì $BD\perp (ABC)$) suy ra $AE\perp CD$.\\
			Kẻ $AI\perp CD $ tại $I$. Suy ra $(P)\equiv (AEI)$.\\
			Thiết diện của hình chóp S.ABCD tạo bởi $(P)$  là tam giác $AIE$  vuông tại $E$.
			Kẻ đường cao $BF$ của tam giác $DBC$\\
			Ta có $\dfrac{1}{BF^2}= \dfrac{1}{BD^2}+\dfrac{1}{BC^2}\Rightarrow BF=\dfrac{\sqrt{6}}{3}a$.\\
			$S_{SEF}=\dfrac{1}{2}IE\cdot AE=\dfrac{1}{2}\cdot \dfrac{1}{2}BF\cdot \dfrac{\sqrt{2}}{2}a=\dfrac{\sqrt{3}}{12}a^2$.
		}
		{
			\begin{tikzpicture}[declare function={a=2.65;b=4.5;h=4;},line join=round]
				\path (0,0) coordinate (B)
				(-60:a) coordinate (C)
				(b,0) coordinate (A)
				(0,h) coordinate (D)
				($(B)!.5!(C)$) coordinate (E)
				($(D)!0.7!(C)$) coordinate (I)
				($(D)!0.4!(C)$) coordinate (F)
				;
				\fill [blue,opacity=0.3] (A)--(E)--(I);
				\draw[dashed] (B)--(A)--(E);
				\draw (D)--(B)--(C)--(A)--(D)--(C) (B)--(F) (E)--(I)--(A);
				\foreach \t/\g in {A/0,D/90,C/-90,B/180,I/180,F/10,E/180}{
					\draw[fill=white] (\t) circle (1pt) node[shift={(\g:7pt)},font=\scriptsize]{$ \t $};
				}
			\end{tikzpicture}			
		}
	}
\end{ex}
%Câu 4
\begin{ex}
	Cho hình tứ diện $S.ABC$ có $ABC$ là tam giác đều. $SA$ vuông góc với mặt phẳng $(ABC)$. Gọi $E$ là trung điểm $AC$, $M$ là một điểm thuộc $AE$. Thiết diện tạo bởi tứ diện $S.ABC$ và mặt phẳng $(\alpha)$  là hình gì? Biết $(\alpha)$  là mặt phẳng qua điểm $M$ và vuông góc với $AC$.
	\choice
	{Hình tam giác vuông}
	{Hình chữ nhật}
	{Hình thang không vuông}
	{\True Hình thang vuông}
	\loigiai{
		\immini{
			Qua $M$, kẻ đường thẳng song song $SA$ cắt $SC$ tại $D$.	\\
			Vì $SA\perp (ABC)$ nên $DM\perp (ABC)$ và $DM\perp AC$.\\
			Vẽ $MF\parallel BE$ ($F\in AB$).\\
			Vì $BE\perp AC$ (do $\triangle ABC$ đều) nên $FM\perp AC$.\\
			Suy ra $(P)\equiv (FMD)$.\\
			Kẻ $FG\parallel SA$ cắt $SB$ tại $G$.\\
			Thiết diện của hình chóp $S.ABC$ tạo bởi $(\alpha)$  là hình thang $FGDM$ vuông tại $M$.
		}
		{
			\begin{tikzpicture}[declare function={a=2.65;b=4.5;h=4;},line join=round]
				\path (0,0) coordinate (A)
				(-60:a) coordinate (B)
				(b,0) coordinate (C)
				(0,h) coordinate (S)
				($(A)!.5!(C)$) coordinate (E)
				($(A)!0.6!(E)$) coordinate (M)
				($(A)!0.6!(B)$) coordinate (F)
				($(S)!0.3!(C)$) coordinate (D)
				($(S)!0.6!(B)$) coordinate (G)
				;
				\fill [blue,opacity=0.3] (M)--(F)--(G)--(D);
				\draw[dashed] (A)--(C) (B)--(E) (F)--(M)--(D);
				\draw (S)--(A)--(B)--(C)--(S)--(B) (F)--(G)--(D);
				\foreach \t/\g in {A/180,B/-90,C/0,S/90,E/90,M/40,F/-90,D/90,G/180}{
					\draw[fill=white] (\t) circle (1pt) node[shift={(\g:7pt)},font=\scriptsize]{$ \t $};
				}
			\end{tikzpicture}				
		}
	}
\end{ex}
\Closesolutionfile{ans}

% \begin{indapan}{10}
% 	{ans/ans-1K7-25-Dang6}
% \end{indapan}