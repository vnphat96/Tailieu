\begin{dang}{Đường vuông góc chung của hai đường thẳng chéo nhau}
	% \begin{enumerate}
	% 	\item \textbf{Định nghĩa}\\		
	% 	Cho hai đường thẳng chéo nhau $a$ và $b$. Độ dài đoạn vuông góc chung $MN$ của $a$ và $b$ được gọi là khoảng cách giữa hai đường thẳng $a, b$.
	% 	\immini{Khi đó: $ \mathrm{d}(a; b) = MN$ với $ \heva{&MN \perp a\\&MN \perp b.}$
	% 	}{
	% 		\begin{tikzpicture}[line join=round,line cap=round,>=stealth,scale=.7]
	% 			\path 
	% 			(0,0) coordinate (A)
	% 			(4.5,0) coordinate (B)
	% 			(0,-2) coordinate (C)
	% 			(4.5,-1) coordinate (D)
	% 			(1,0) coordinate (M)
	% 			($(C)!.4!(D)$) coordinate (N);		
	% 			\draw (A)--(B)node[pos=.9,above]{$a$};
	% 			\draw (C)--(D)node[pos=.9,below]{$b$};	
	% 			\draw (M)node[above]{$M$}--(N)node[below]{$N$};	
	% 			\pic[draw,angle radius=2mm,angle eccentricity=1.5] {right angle = A--M--N};
	% 			\pic[draw,angle radius=2mm,angle eccentricity=1.5] {right angle = M--N--C};	
	% 	\end{tikzpicture}} 
	% 	\item \textbf{Phương pháp xác định khoảng cách giữa hai đường thẳng chéo nhau.}\\
	% 	Sử dụng song song và đưa về bài toán tìm khoảng cách từ chân đường cao đến mặt bên.\\		
	% 	\textbf{Cho hai đường thẳng chéo nhau $a$ và $b$. Khi $a$ và $b$ vuông góc với nhau, ta có thể làm như sau:}\\
	% 	\immini{\textbf{Bước 1.} Xác định mặt phẳng $(P)$ đi qua $b$ và vuông góc với $a$, giao điểm của $a$ và $(P)$ là $H$, hình chiếu của $H$ trên $b$ là $K$.\\
	% 		\textbf{Bước 2.} Khi đó $HK$ là đoạn vuông góc chung của hai đường thẳng chéo nhau $a$, $b$ và $\mathrm{d}(a, b) = HK$.
	% 	}
	% 	{\begin{tikzpicture}[scale=.65,font=\footnotesize, line join=round, line cap=round, >=stealth]
	% 			\coordinate (A) at (-1,0);
	% 			\coordinate (B) at (6,0);
	% 			\coordinate (D) at (2,3);
	% 			\coordinate (C) at ($(B)+(D)-(A)$);
	% 			\coordinate (H) at ($(A)!.4!(C)$);
	% 			\coordinate (K) at ($(H)+(1.5,-0.5)$);
	% 			\coordinate (a) at ($(H)+(0,4)$);
	% 			\coordinate (a') at ($(H)+(0,-1)$);
	% 			\coordinate (b) at ($(K)+(1,1.5)$);
	% 			\coordinate (b') at ($(b)!1.2!(K)$);
	% 			\foreach \x/\g in {K/-20,H/190} \fill[black](\x) circle (1.5pt) ($(\x)+(\g:4mm)$) node{$\x$};
	% 			\draw (A)--(B)--(C)--(D)--(A) (K)--(H)--(a) node [left]{$a$}
	% 			(b) node [right]{$b$}--(b');
	% 			\draw[dashed] (H)--(a');
	% 			%\tkzMarkAngles[size=1.1cm,arc=l,mark=](B,A,D)
	% 			\pic[draw,angle radius=2mm,angle eccentricity=1.5] {right angle = a--H--K};
	% 			\pic[draw,angle radius=2mm,angle eccentricity=1.5] {right angle = H--K--b};
	% 			\draw pic[draw, angle radius=6mm]{angle=B--A--D};
	% 			\path (A)+(20:11pt) node{$P$};
	% 		\end{tikzpicture}	
	% 	}
	% \end{enumerate}
\end{dang}
\subsubsection{Ví dụ minh hoạ}
\begin{vd}%[1K7KP-5]
	Cho hình chóp $S.ABCD$ có đáy là hình vuông cạnh $a$, $SA=h$ và $SA \perp (ABCD)$. Dựng và tính độ dài đoạn vuông góc chung của:
	\begin{listEX}[3]
		\item $SB$ và $CD$.
		\item $AD$ và $SB$.
		\item $SC$ và $BD$.
		\item $SC$ và $AB$.
		\item $SC$ và $AD$.
	\end{listEX}
	\loigiai{ 
		\begin{listEX}[1]
			\item $SB$ và $CD$.
			\immini{Ta có $\heva{&BC\perp AB\\&BC\perp SA\\&AB\cap SA=A}\Rightarrow BC\perp(SAB).$\\
				Do đó $BC\perp SB$ và $BC\perp CD$.\\
				Suy ra $BC$ là đoạn vuông góc chung của $SB, CD$.\\
				Theo định nghĩa, suy ra: $\mathrm{d}(SB,CD)=BC=a$.
				\begin{nx}
					Nếu hai đường thẳng chéo nhau và vuông góc nhau sẽ sử dụng định nghĩa để tìm khoảng cách, tức là tìm đoạn vuông góc chung. Còn nếu không vuông góc ta sẽ sử dụng đúng phương pháp kẻ song song như lý thyết.	
				\end{nx}
			}{
				\begin{tikzpicture}[scale=1, font=\footnotesize, line join=round, line cap=round, >=stealth]
					\def\bc{4} % cạnh BC
					\def\ba{2} % cạnh BA
					\def\h{4} % đường cao
					\def\gocB{30} % góc B của đáy
					\coordinate[label=below left:$B$] (B) at (0,0);
					\coordinate[label=above left:$A$] (A) at (\gocB:\ba);
					\coordinate[label=below:$C$] (C) at (\bc,0);
					\coordinate[label=right:$D$] (D) at ($(C)-(B)+(A)$);
					\coordinate[label=above:$S$] (S) at ($(A)+(90:\h)$);
					\coordinate[label=left:$H$] (H) at ($(S)!(A)!(B)$);
					\coordinate[label=below:$O$] (O) at ($(D)!.5!(B)$);
					\coordinate[label=right:$E$] (E) at ($(S)!(O)!(C)$);
					\coordinate[label=right:$F$] (F) at ($(S)!(A)!(D)$);
					\coordinate[label=right:$G$] (G) at ($(S)!(A)!(C)$);
					\draw (B)--(C)--(D)--(S)--cycle (S)--(C);
					\draw[dashed] (A)--(D) (S)--(A)--(B) (A)--(H) (A)--(C) (B)--(D) (O)--(E) (A)--(F) (A)--(G);
					\foreach \diem in {A,B,C,D,S,O,E,G,H,F}	\fill (\diem)circle(1.2pt);
					\pic[draw,angle radius=2mm,angle eccentricity=1.5] {right angle = A--B--C};
					\pic[draw,angle radius=2mm,angle eccentricity=1.5] {right angle = B--A--D};
					\pic[draw,angle radius=2mm,angle eccentricity=1.5] {right angle = A--D--C};
					\pic[draw,angle radius=2mm,angle eccentricity=1.5] {right angle = S--A--D};
					\pic[draw,angle radius=2mm,angle eccentricity=1.5] {right angle = S--H--A};
					\pic[draw,angle radius=2mm,angle eccentricity=1.5] {right angle = A--O--B};
					\pic[draw,angle radius=2mm,angle eccentricity=1.5] {right angle = O--E--S};
					\pic[draw,angle radius=2mm,angle eccentricity=1.5] {right angle = A--F--D};
					\pic[draw,angle radius=2mm,angle eccentricity=1.5] {right angle = A--G--C};
				\end{tikzpicture}
			}
			\item $AD$ và $SB$.\\
			Ta có $AD\perp(SAB),SB\subset(SAB)\Rightarrow AD\perp SB$.\\
			Kẻ $AH\perp SB\Rightarrow \mathrm{d}(SB,AD)=AH$.\\
			Xét $\triangle SAB$ vuông tại $A$, ta có:
			$\dfrac{1}{AH^2}=\dfrac{1}{SA^2}+\dfrac{1}{AB^2}\Rightarrow \dfrac{1}{AH^2}=\dfrac{1}{h^2}+\dfrac{1}{a^2}\Rightarrow AH=\dfrac{ah}{\sqrt{a^2+h^2}}$.
			\item $SC$ và $BD$.\\
			Gọi $O=AC\cap BD$.\\
			$\heva{&BD\perp AC\\&BD\perp SA (\text{ do }SA\perp (ABCD))}\Rightarrow BD\perp (SAC)$.\\
			Mà $SC\subset (SAC)$ nên $BD\perp SC$.\\
			Từ $O$ kẻ $OE\perp SC\Rightarrow \mathrm{d}(BD,SC)=OE$.\\
			Xét $\triangle SAC$ vuông tại $A$, ta có:
			$$\dfrac{1}{AG^2} = \dfrac{1}{SA^2} + \dfrac{1}{AC^2}\Rightarrow \dfrac{1}{AG^2}=\dfrac{1}{h^2}+\dfrac{1}{(a\sqrt{2})^2}\Rightarrow AH=\dfrac{ah\sqrt{2}}{\sqrt{2a^2+h^2}}.$$
			Xét $\triangle AGC$, ta có:\\
			$O$ là trung điểm $AC$ và $OE\parallel AG$ (vì $AG\perp SC, OE\perp SC$).\\
			Suy ra $OE$ là đường trung bình của $\triangle AGC$ nên $OE=\dfrac{1}{2}AG=\dfrac{ah}{\sqrt{4a^2+2h^2}}$.
			\item $SC$ và $AB$.\\
			Ta có $\heva{&AB\parallel CD\\&CD\subset(SCD)}\Rightarrow AB\parallel(SCD)$.\\
			Kẻ $AF\perp SD$ và cắt $SD$ tại $F$.\\
			$\heva{&CD\perp AD\\&CD\perp SA}\Rightarrow CD\perp (SAD)$.\\
			Mà $AF\subset(SAD)$ nên $AF\perp CD$.\\
			Suy ra $AF\perp(SCD)$ và $\mathrm{d}(AB,SC)=\mathrm{d}(A,(SCD))=AF$.\\
			Xét $\triangle SAD$ vuông tại $A$, ta có:
			$\dfrac{1}{AF^2}= \dfrac{1}{SA^2} + \dfrac{1}{AD^2} \Rightarrow \dfrac{1}{AF^2}= \dfrac{1}{h^2}+\dfrac{1}{a^2}\Rightarrow AF = \dfrac{ah}{\sqrt{a^2 + h^2}}$.
			\item $SC$ và $AD$.\\
			Ta có $\heva{&AD\parallel BC\\&BC\subset(SBC)}\Rightarrow AD\parallel(SBC)$.\\
			Ta có $\heva{&AH\perp SB\\&AH\perp BC}\Rightarrow AH\perp(SBC)$.\\
			Suy ra $\mathrm{d}(AD,SC)=\mathrm{d}(A,(SBC))=AH=\dfrac{ah}{\sqrt{a^2+h^2}}$.
		\end{listEX}	
	}
\end{vd}

% \begin{vd}%[1K7KP-5]
% 	Cho hình chóp $S.ABCD$ có đáy là hình vuông cạnh $a$, $SA = h$ và $SA \perp (ABCD)$.
% 	\begin{center}
% 		\begin{tikzpicture}[scale=1,>=stealth, font=\footnotesize, line join=round, line cap=round]
% 			\coordinate[label=below:$A$] (A) at (0,0);
% 			\coordinate[label=below:$B$] (B) at (-1.3,-1.6);
% 			\coordinate[label=below:$C$] (C) at (2.5,-1.6);
% 			\coordinate[label=right:$D$] (D) at ($(A)+(C)-(B)$);
% 			\coordinate[label=above:$S$] (S) at ($(A)+(0,3)$);
% 			\coordinate[label=left:$K$] (K) at ($(S)!0.6!(B)$);
% 			\coordinate[label=below:$O$] (O) at ($(B)!0.5!(D)$);
% 			\coordinate[label=right:$L$] (L) at ($(S)!0.7!(C)$);
% 			\draw (S)--(B)--(C)--(D)--cycle (S)--(C);
% 			\draw[dashed] (A)--(S) (A)--(B) (A)--(D) (A)--(K) (A)--(C) (B)--(D) (O)--(L);
% 			\foreach \diem in {A,B,C,D,S,K,O,L}	\fill (\diem)circle(1.5pt);
% 			\foreach \x/\dinh/\y in {S/A/B,S/A/D,A/B/C,A/K/B,O/L/S} \draw[fill = gray!50] ($(\dinh)!5pt!(\x)$)--($(\dinh)!5pt!(\x)+(\dinh)!5pt!(\y)-(\dinh)$)--($(\dinh)!5pt!(\y)$)--(\dinh)--cycle; 
% 		\end{tikzpicture}
% 	\end{center}
% 	\begin{enumEX}{1}
% 		\item  Dựng và tính độ dài đoạn vuông góc chung của $SB$ và $CD$.\\
% 		Ta có $\heva{&BC \perp AB\\	&BC \perp  SA\\&AB \cap SA = A} \Rightarrow BC \perp(SAB).$\\
% 		Do đó $ BC \perp SB$ và $ BC \perp CD$ suy ra $BC$ là đoạn vuông góc chung của $SB, CD$.\\
% 		Theo định nghĩa, suy ra $\mathrm{d}(SB,CD) = BC = a$.
% 		\begin{nx}
% 			Nếu hai đường thẳng chéo nhau và vuông góc nhau sẽ sử dụng định nghĩa để tìm khoảng cách, tức là tìm đoạn vuông góc chung. Còn nếu không vuông góc nhau ta sẽ sử dụng	đúng phương pháp kẻ song song như lý thuyết.
% 		\end{nx}
% 		\item  Dựng và tính độ dài đoạn vuông góc chung của $AD$ và $SB$.\\
% 		$\mathrm{d}(AD, SB) = \dfrac{a \cdot h}{\sqrt{a^2+h^2}}$.\\
% 		Dựng $AK \perp SB$ $(K\in SB) \hfill{(1)}$.\\
% 		$ \heva{&AD \perp SA\\&AD \perp  AB}  \Rightarrow AD \perp (SAB) \Rightarrow AD \perp AK \hfill{(2)}$.\\
% 		Từ (1), (2) ta có $AK$ là đoạn vuông góc chung $AD$, $SB$.\\
% 		$\Rightarrow \mathrm{d}(AD,SB) = AK = \sqrt{\dfrac{AS^2 \cdot AB^2}{AS^2+ AB^2}}=\sqrt{\dfrac{a^2 \cdot h^2}{a^2+h^2}}=\dfrac{a \cdot h}{\sqrt{a^2+h^2}}$.
% 		\item  Dựng và tính độ dài đoạn vuông góc chung của $SC$ và $BD$.\\
% 		Ta có $AC = a\sqrt{2}$.\\
% 		Gọi $O$ là giao điểm $AC$, $BD$. Dựng $OL \perp SC (L \in SC) \hfill{(3)}$.
% 		$\heva{&BD \perp AC\\&BD \perp SA\\&AC \cap SA = A} \Rightarrow BD \perp (SAC) \Rightarrow BD \perp OL \hfill{(4)}$.\\
% 		Từ (3), (4) ta có $OL$ là đoạn vuông góc chung $SC, BD$.\\
% 		Ta có $\heva{&\widehat{SCA} \text{ chung}\\	&\widehat{SAC} = \widehat{OLC} = 90^{\circ}}  \Rightarrow \triangle CLO \backsim \triangle CAS \Rightarrow \dfrac{OL}{SA}=\dfrac{OC}{SC}$\\
% 		$ \Rightarrow OL=\dfrac{OC\cdot SA}{SC}=\dfrac{\dfrac{a\sqrt{2}}{2}\cdot h}{\sqrt{2 a^2+ h^2}}=\dfrac{a \cdot h}{\sqrt{4 a^2+2 h^2}}$.
% 		\item  Tính độ dài đoạn vuông góc chung của $SC$ và $AB$.\\
% 		Ta có $AB  \parallel CD \Rightarrow AB \parallel (SCD)$
% 		$ \Rightarrow \mathrm{d}(AB,SC) = \mathrm{d}(AB,(SCD)) = \mathrm{d}(A, (SCD)) = \dfrac{a \cdot h}{\sqrt{a^2 + h^2}}$ (theo câu 1.).
% 		\item  Tính độ dài đoạn vuông góc chung của $SC$ và $AD$.\\
% 		Ta có $AD  \parallel BC \Rightarrow AD \parallel (SBC)$
% 		$ \Rightarrow \mathrm{d}(AD,SC) = \mathrm{d}(AD,(SBC)) = \mathrm{d}(A,(SBC)) = \dfrac{a \cdot h}{\sqrt{a^2 + h^2}}$.
% 	\end{enumEX}
% \end{vd}

% \begin{vd}%[TeX hóa SGK CTST]%[1K7KP-5]
% 	Cho hai tam giác cân $ABC$ và $ABD$ có đáy chung $AB$ và không cùng nằm trong một mặt phẳng.
% 	\begin{enumerate}
% 		\item[a)] Chứng minh rằng $AB \perp CD$.
% 		\item[b)] Xác định đoạn vuông góc chung của $AB$ và $CD$.
% 	\end{enumerate}
% 	\loigiai{
% 		\immini{
% 			\begin{enumerate}
% 				\item[a)] Gọi $M$ là trung điểm của $AB$, vì $ABC$ và $ABD$ là các tam giác cân nên
% 				$$\heva{&AB\perp CM\\& AB \perp DM\\& CM\cap DM = M\\& CM \subset (MCD) \\ & DM \subset (MCD)}\Rightarrow AB \perp (MCD)\Rightarrow AB \perp CD.$$
% 				\item[b)] Gọi $N$ là trung điểm của $CD$, vì $ABC$ và $ABD$ là các tam giác cân và bằng nhau nên $CM = DM$, do đó tam giác $CMD$ cân tại $M$ suy ra $MN \perp CD$.\\
% 				Ta có $\heva{&MN \perp CD\\&MN \perp AB }$. \\
% 				Vậy $MN$ là đoạn vuông góc chung của $AB$ và $CD$.
% 			\end{enumerate}
% 		}
% 		{	\begin{tikzpicture}[scale=.8, font=\footnotesize, line join=round, line cap=round, >=stealth]
% 				\def\ac{4} % cạnh AC
% 				\def\ab{2} % cạnh AB
% 				\def\h{4} % chiều cao
% 				\def\gocA{50} % góc A của đáy
% 				\coordinate[label=left:$A$] (A) at (0,0);
% 				\coordinate[label=right:$C$] (C) at (\ac,0);
% 				\coordinate[label=below left:$B$] (B) at (-\gocA:\ab);
% 				\coordinate[label=above:$D$] (S) at ($(A)+(60:\h)$);
% 				\coordinate[label={left}:$M$] (M) at ($(A)!0.5!(B)$);
% 				\coordinate[label={above right}:$N$] (N) at ($(C)!0.5!(S)$);
% 				\pic[draw,angle radius=6]{right angle=A--M--S};
% 				\pic[draw,angle radius=5]{right angle=M--N--S};
% 				\pic[draw,angle radius=5]{right angle=N--M--A};
% 				\pic[draw,angle radius=5]{right angle=C--M--B};
% 				\draw (A)--(B)--(C)--(S)--cycle (S)--(B) (S)--(M);
% 				\draw[dashed] (A)--(C) (C)--(M) (M)--(N);
% 				\foreach \diem in {A,B,C,S,M,N}	\fill (\diem)circle(1.5pt);
% 		\end{tikzpicture}}
% 	}
% \end{vd}
% \begin{vd}%[1K7KP-5]
% 	Cho hình lập phương $ABCD.A'B'C'D'$ có cạnh bằng $a$. Xác định và tính độ dài đoạn vuông góc chung của $AB'$ và $BC'$.
% 	\loigiai{
% 		\immini{Gọi $I$ và $J$ lần lượt là tâm của hai hình vuông $BCC'B'$ và $ADD'A'$.\\
% 			Trong mặt phẳng $(A'B'CD)$, dựng $IK\perp JB'$, $K\in JB'$.\\
% 			Trong mặt phẳng $(AB'D')$, dựng $KP\parallel AD'$, $P\in AB'$, suy ra $KP\parallel BC'$.\\
% 			Trong mặt phẳng $(KP,BC')$, dựng $PQ\parallel IK$, $Q\in BC'$. Ta chứng minh $PQ$ là đoạn vuông góc chung của $BC'$ và $AB'$.\\
% 			Ta có $BC'\perp(A'B'CD)$, $IK\subset(A'B'CD)$ nên $IK\perp BC'$.\\
% 			Mà $BC'\parallel AD'\Rightarrow IK\perp AD'$. Ngoài ra ta có $IK\perp JB'$ nên $IK\perp(AB'D')\Rightarrow IK\perp AB'$.\\
% 			Tóm lại ta có $\heva{&IK\perp BC'\\&IK\perp AB'}$, mà $PQ\parallel IK$ nên $PQ$ vuông góc với cả hai đường thẳng $BC'$ và $AB'$.\\
% 			Vậy $PQ$ là đoạn vuông góc chung của $AB'$ và $BC'$.
% 		}{\begin{tikzpicture}[scale=1,font=\footnotesize]
% 				\coordinate[label=left:$D$] (D) at (0,0);
% 				\coordinate[label=below left:$A$] (A) at (-1.3,-2);
% 				\coordinate[label=above right:$C$] (C) at (5,0);
% 				\coordinate[label=above left:$D'$] (D') at (0,5);
% 				\coordinate[label=below right:$B$] (B) at ($(A)+(C)-(D)$);
% 				\coordinate[label=right:$B'$] (B') at ($(B)+(D')-(D)$);
% 				\coordinate[label=above right:$C'$] (C') at ($(C)+(D')-(D)$);
% 				\coordinate[label=left:$A'$] (A') at ($(A)+(D')-(D)$);
% 				\coordinate[label=below:$O$] (O) at ($(A)!0.5!(C)$);
% 				\coordinate[label=below:$O'$] (O') at ($(A')!0.5!(C')$);
% 				\coordinate[label=left:$J$] (J) at ($(A)!0.5!(D')$);
% 				\coordinate[label=right:$I$] (I) at ($(B)!0.5!(C')$);
% 				\coordinate[label=above:$K$] (K) at ($(B')!0.33!(J)$);
% 				\coordinate[label=below:$P$] (P) at ($(B')!0.33!(A)$);
% 				\coordinate[label=right:$Q$] (Q) at ($(P)+(I)-(K)$);
% 				\draw (B)--(B') (C)--(C') (A)--(A') (A')--(B') (B')--(C') (C')--(D') (D')--(A') (B)--(A) (B)--(C) (A')--(B) (B)--(C') (A')--(C') (B')--(D') (A)--(B') (B')--(C);
% 				\draw[dashed] (D)--(D') (A)--(D) (C)--(D) (A)--(C) (B)--(D) (A)--(D') (A')--(D) (B')--(D) (B')--(J) (I)--(K) (K)--(P) (P)--(Q) (I)--(J);
% 				\foreach \diem in {A,B,C,D,A',B',C',D',I,J,O,O',K,P,Q}	\fill (\diem)circle(1.5pt);
% 				\foreach \x/\dinh/\y in {B'/I/J,I/K/B',C'/Q/P,Q/P/B'} \draw[fill = gray!50] ($(\dinh)!5pt!(\x)$)--($(\dinh)!5pt!(\x)+(\dinh)!5pt!(\y)-(\dinh)$)--($(\dinh)!5pt!(\y)$)--(\dinh)--cycle; 
% 			\end{tikzpicture}	
% 		}
% 		Ta có $IJ\parallel A'B'$, mà $A'B'\perp(BCC'B')\Rightarrow IJ\perp(BCC'B')\Rightarrow IJ\perp IB'$.\\
% 		Xét tam giác vuông $B'IJ$ có đường cao $IK$ nên
% 		$$\dfrac{1}{IK^2}=\dfrac{1}{IB'^2}+\dfrac{1}{IJ^2}=\dfrac{2}{a^2}+\dfrac{1}{a^2}=\dfrac{3}{a^2}\Rightarrow IK=\dfrac{a\sqrt{3}}{3}.$$
% 		Theo cách dựng thì $IKPQ$ là hình chữ nhật nên $PQ=IK=\dfrac{a\sqrt{3}}{3}$.	
% 	}
% \end{vd}

\subsubsection{Bài tập rèn luyện}
\begin{bt}%[1K7KP-5]
	Cho hình hộp đứng $ABCD.A'B'C'D'$ có cạnh bên $AA' = 2a$ và đáy $ABCD$ là hình thoi có $AB = a$ và $AC = a\sqrt{3}$. Tính khoảng cách giữa hai đường thẳng $BD$ và $AA'$.
	\loigiai{
		\immini{Ta có $\heva{&AO \perp AA'\\&AO \perp BD}\Rightarrow \mathrm{d}(BD,AA')=\dfrac{1}{2}AC=\dfrac{a\sqrt{3}}{2}$.
		}{\begin{tikzpicture}[>=stealth,line join=round,line cap=round,font=\footnotesize,scale=0.7]
				\coordinate[label=above left:{$A$}] (A) at (0,0);
				\coordinate[label=below left:{$B$}] (B) at (-2.75,-1.5);
				\coordinate[label=right:{$D$}] (D) at (4.5,0);
				\coordinate[label=below right:{$C$}] (C) at ($(B)+(D)-(A)$);
				\coordinate[label=above left:{$A'$}] (A') at (0,4.5);
				\coordinate[label=above right:{$D'$}] (D') at ($(D)+(A')-(A)$);
				\coordinate[label=above left:{$B'$}] (B') at ($(A')+(B)-(A)$);
				\coordinate[label=above left:{$C'$}] (C') at ($(B')+(D')-(A')$);
				\coordinate[label=below:{$O$}]  (O) at ($(A)!0.5!(C)$);				
				\draw (A')--(B')--(C')--(D')--cycle (B)--(B') (C)--(C') (D)--(D')
				(B)--(C)--(D) ;
				\draw[dashed] (A)--(A') (B)--(A)--(D) (A)--(C) (B)--(D);
				\draw pic[draw,angle radius=2mm] {right angle = B--O--A};
				\foreach \p in {A,B,C,D,A',B',C',D',O}
				\fill (\p)	circle (1.2pt);	
			\end{tikzpicture}
		}
	}
\end{bt}


\begin{bt}%[1K7KP-5]
	Cho hình chóp tứ giác đều $S.ABCD$ có tất cả các cạnh đều bằng $a$ và có $O$ là giao điểm hai đường chéo của đáy. Tính khoảng cách giữa hai đường thẳng $AC$ và $SB$.
	\loigiai{
		\immini{Kẻ $OH \perp SB$ tại $H$.\\
			Ta có $\heva{&AC\perp BD\\&AC \perp SO}\Rightarrow AC \perp (SBD)\Rightarrow AC \perp OH$.\\
			Khi đó $\mathrm{d}(AC,SB) = OH$.\\
			Có $\triangle SAC$ vuông tại $S$ suy ra $SO = \dfrac{1}{2}AC=\dfrac{a\sqrt{2}}{2}$.\\
			Khi đó $OH=\dfrac{SO\cdot OB}{SB} = \dfrac{\tfrac{a\sqrt{2}}{2}\cdot \tfrac{a\sqrt{2}}{2}}{a} = \dfrac{a}{2}$.\\
			Vậy $\mathrm{d}(AC,SB) = \dfrac{a}{2}$.	
		}{\begin{tikzpicture}[=>stealth,line join=round,line cap=round, font=\footnotesize, scale=.8]
				\def\a{5}
				\def\goc{210}
				\def\b{3}
				\def\h{4.5}
				\path
				(0,0)coordinate (A)++(0:\a)coordinate (B)++(\goc:\b)coordinate (C)++(180:\a)coordinate(D)
				($(A)!.5!(C)$)coordinate (O)
				(O)++(90:\h)coordinate (S);
				\path 
				($(B)!(O)!(S)$)coordinate (H);
				\draw (S)--(B)--(C)--(D)--cycle (S)--(C); %(M)--(S)--(C);
				\draw[dashed] (O)--(S)--(A)--(C) (B)--(A)--(D)--(B) (O)--(H);
				\draw pic[draw,angle radius=2mm] {right angle = S--O--B};
				\draw pic[draw,angle radius=2mm] {right angle = O--H--B};
				\foreach \x/\g in{A/160,B/0,C/-90,S/90,O/-90,D/-90,H/60}
				\fill[black](\x)circle(1.5pt) ($(\x)+(\g:3mm)$)node{$\x$};
			\end{tikzpicture}
		}		
	}
\end{bt}

\begin{bt}%[1K7KP-5]
	\immini{Cho lăng trụ $ABCD.A'B'C'D'$ có đáy $ABCD$ là hình vuông cạnh $2a$, $O$ là giao điểm của $AC$ và $BD$, $AA' = a$, $AA'$ vuông góc với mặt phẳng chứa đáy. Tính
		\begin{enumerate}
			\item $\mathrm{d}\left(AC, A'B'\right)$;
			\item $\mathrm{d}\left(CC', BD\right)$.
		\end{enumerate}
	}
	{\begin{tikzpicture}[scale=.6,font=\footnotesize, line join=round, line cap=round, >=stealth]
			\coordinate (A) at (0,0);
			\coordinate (D) at (5,0);
			\coordinate (B) at (-2,-2);
			\coordinate (C) at ($(B)+(D)-(A)$);
			\coordinate (O) at ($(A)!.5!(C)$);
			\coordinate (A') at ($(A)+(0,4)$);
			\coordinate (B') at ($(B)+(A')-(A)$);
			\coordinate (C') at ($(C)+(A')-(A)$);
			\coordinate (D') at ($(D)+(A')-(A)$);
			\foreach \x/\g in {A/170,B/-120,C/-60,D/0,A'/90,B'/170,C'/-10,D'/60,O/-90} \fill[black](\x) circle (1.5pt) ($(\x)+(\g:4mm)$) node{$\x$};
			\draw (B')--(B)--(C)--(D)--(D')--(A')--(B')--(C')--(D') (C)--(C');
			\draw[dashed] (A)--(B)--(D)--(A)--(A') (A)--(C);	
		\end{tikzpicture}
	}
	\loigiai{
		\begin{enumerate}
			\item Vì $AA'$ vuông góc với cả hai mặt phẳng $(ABCD)$ và $(A'B'C'D')$ nên $AA' \perp AC$, $AA' \perp A'B'$.\\
			Suy ra đoạn thẳng $AA'$ là đoạn vuông góc chung của $AC$ và $A'B'$.\\
			Vậy $\mathrm{d}\left(AC, A'B'\right) = AA' = a$.
			\item Vì $C C'$ vuông góc với $(ABCD)$ nên $CC' \perp OC$. Do đáy $ABCD$ là hình vuông có $O$ là giao điểm của $AC$ và $BD$ nên $BD \perp OC$. Suy ra đoạn thẳng $OC$ là đoạn vuông góc chung của $CC'$ và $BD$.\\
			Vậy $\mathrm{d}\left(CC', BD\right) = OC = a\sqrt{2}$.
		\end{enumerate}
	}
\end{bt}
\begin{bt}%[1K7KP-5]
	Cho hình chóp $S.ABCD$ có đáy là hình vuông cạnh $a$, $SA\perp (ABCD)$, $SA=a\sqrt{2}$. Xác định đường vuông góc chung và tính khoảng cách giữa $BD$ và $SC$.
	\loigiai{
		\begin{center}
			\begin{tikzpicture}[scale=1,font=\footnotesize,line join = round, line cap = round, >= stealth]
				%\draw[opacity=0.3] (0,0) grid (6,6);
				\def\x{4} \def\y{2} \def\z{3}
				\def\g{-120}
				\coordinate (A) at (0,0);
				\coordinate (D) at ($(A)+(0:\x)$);
				\coordinate (B) at ($(A)+(\g:\y)$);
				\coordinate (C) at ($(B)+(D)-(A)$);
				\coordinate (S) at ($(A)+(90:\z)$);
				\coordinate (O) at ($(B)!1/2!(D)$);
				\coordinate (H) at ($(S)!1/2!(C)$);
				\coordinate (K) at ($(H)!0.4!(C)$);
				\draw (S)--(B)--(C)--(D)--(S)--(C);
				\draw[dashed] (S)--(A)--(C) (B)--(D)
				(B)--(A)--(D)  (H) (O)--(K)
				;
				\draw pic[draw,angle radius=1mm] {right angle = O--K--C};
				\foreach \p/\g in {S/100,A/180,B/-90,C/-70,D/0,O/-90,K/20} \draw[fill] (\p) circle(.5pt)
				node [shift={(\g:.3)}] {$\p$}
				;
			\end{tikzpicture}
		\end{center}
		Gọi $O=AC\cap BD$. Kẻ $OK\perp SC$ tại $K$.\\
		Do $BD\perp (SAC)$ nên $OK\perp BD$.\\
		Vậy $OK$ là đường vuông góc chung giữa $BD$ và $SC$.\\
		Xét $\triangle SAC\backsim \triangle OKC\Rightarrow \dfrac{OK}{SA}=\dfrac{OC}{SC}$  $\Rightarrow OK=\dfrac{OC\cdot SA}{SC}=\dfrac{a\sqrt{2}\cdot a\sqrt{2}}{2\cdot 2a}=\dfrac{a}{2}$.
	}
\end{bt}



\begin{bt}%[1K7KP-5]
	Cho chóp tam giác đều $ S.ABC $ có cạnh đáy bằng $ 3a $, cạnh bên bằng $2a$. Gọi $G$ là trọng tâm giác $ABC$. Dựng và tính đoạn vuông góc chung của hai đường thẳng $SA$ và $BC$.
	\loigiai{
		\immini{
			Trong tam giác $ABC$ đều, kéo dài $AG$ cắt $BC$ tại $M$
			$\Rightarrow AG \perp BC$.\\
			Chóp $S.ABC$ đều, mà $G$ là tâm $\triangle ABC$ nên: $SG \perp (ABC) \Rightarrow SG \perp BC$.\\
			Vì $BC \perp SG$ và $BC \perp AM$ nên $BC \perp (SAM)$.\\
			Trong $\triangle SAM$ kẻ $MN \perp SA \; \left( N \in SA \right) \Rightarrow MN \perp BC$ (vì $MN \subset (SAM)$).\\
			Do vậy, $MN$ là đoạn vuông góc chung của $BC$ và $SA$.\\
			Trong $\triangle SAG$ vuông tại $G$ ta có:
			$SG=\sqrt{SA^2-AG^2}=\sqrt{4a^2 -3a^2} = a$.\\
			Trong $\triangle SAM$ có: 
			$$MN \cdot SA = SG \cdot AM \Leftrightarrow MN \cdot 2a =a \cdot \dfrac{3a\sqrt{3}}{2} \Rightarrow MN = \dfrac{3a\sqrt{3}}{4}.$$
		}{
			\begin{tikzpicture}[scale=0.8, font=\footnotesize, line join=round, line cap=round, >=stealth]
				\coordinate[label=left:{$A$}] (A) at (0,0);
				\coordinate[label=right:{$C$}] (C) at (4,0);
				\coordinate[label=right:{$B$}] (B) at (3,-2);
				\coordinate (E) at (0,4);
				\coordinate[label=below right:{$M$}] (M) at ($(B)!0.5!(C)$);
				\coordinate[label=below:{$G$}] (G) at ($(A)!2/3!(M)$);
				\coordinate[label = above:$S$] (S) at ($(E)+(G)-(A)$); 
				\coordinate[label=left:{$N$}] (N) at ($(S)!0.5!(A)$);
				\draw (A)--(B) (C)--(B) (A)--(S) (S)--(C) (S)--(B) (S)--(M);
				\draw[dashed] (A)--(C) (A)--(M) (S)--(G) (M)--(N);						
				\foreach \x/\dinh/\y in {A/N/M,A/M/B} \draw[fill = gray!50] ($(\dinh)!5pt!(\x)$)--($(\dinh)!5pt!(\x)+(\dinh)!5pt!(\y)-(\dinh)$)--($(\dinh)!5pt!(\y)$)--(\dinh)--cycle;
				\foreach \p in {S,A,B,C,G,M,N} \fill (\p)	circle (1.2pt);	
			\end{tikzpicture}
		}
	}
\end{bt}
\begin{bt}%[1K7KP-5]
	Cho hình chóp tam giác $S.ABC$ có $SA$ vuông góc với $(ABC)$ và $SA=a\sqrt{2}$. Đáy $ABC$ là tam giác vuông tại $B$ với $BA=a$. Gọi $M$ là trung điểm của $AB$. Tìm độ dài đoạn vuông góc chung của hai đường thẳng $SM$ và $BC$.
	\loigiai{
		\immini{
			Ta có $\heva{&SA \perp BC\\& AB \perp BC} \Rightarrow BC \perp (SAB)$ tại $B.$	\\
			Dựng $BH \perp SM \; (H \in SM)$.\\
			Ta thấy  $BC \perp BH \; (BH \subset (SAB))$\\
			Vậy $BH$ chính là đoạn vuông góc chung của $SM$ và $BC$. \\
			Ta có $\triangle MHB \sim \triangle MAS \Rightarrow \dfrac{HB}{AS} = \dfrac{MB}{MS}$ $$\Leftrightarrow \dfrac{HB}{AS}=\dfrac{MB}{\sqrt{AS^2+AM^2}} = \dfrac{1}{3} \Rightarrow HB = \dfrac{AS}{3}=\dfrac{a\sqrt{2}}{3}.$$
		}{
			\begin{tikzpicture}[scale=1, font=\footnotesize, line join=round, line cap=round, >=stealth]
				\coordinate[label=left:{$A$}] (A) at (0,0);
				\coordinate[label=right:{$C$}] (C) at (5,0);
				\coordinate[label=right:{$B$}] (B) at (3.5,-2);
				\coordinate[label=above:{$S$}] (S) at (0,4);
				\coordinate[label=left:{$M$}] (M) at ($(A)!0.5!(B)$);
				\coordinate (M') at ($(S)!1.45!0:(M)$); 
				\coordinate[label = left:$H$] (H) at ($(S)!1.25!0:(M)$); 
				\draw (A)--(B) (C)--(B) (S)--(A) (S)--(C) (S)--(B) (S)--(M') (B)--(H);
				\draw[dashed](A)--(C);			
				\foreach \x/\dinh/\y in {B/H/M',A/B/C} \draw[fill = gray!50] ($(\dinh)!5pt!(\x)$)--($(\dinh)!5pt!(\x)+(\dinh)!5pt!(\y)-(\dinh)$)--($(\dinh)!5pt!(\y)$)--(\dinh)--cycle; 
				\foreach \p in {S,A,B,C,M,H} \fill (\p)	circle (1.2pt);	
			\end{tikzpicture}
		}
	}
\end{bt}
\begin{bt}%[1K7KP-5]
	Trong mặt phẳng $(P)$ cho hình thoi $ABCD$ có tâm là $O$, cạnh $a$ và $OB=\dfrac{a\sqrt{3}}{3}$. Trên đường thẳng vuông góc với $(ABCD)$ tại $O$, lấy điểm $S$ sao cho $SB=a$. Dựng và tính đoạn vuông góc chung của hai đường thẳng
	\begin{enumerate}
		\item $BD$ và $SC$.
		\item  $AB$ và $SD$.
	\end{enumerate}
	\loigiai{
		\begin{enumerate}
			\immini{
				\item Dễ dàng chứng minh được $BD \perp (SAC)$ (vì $BD \perp AC$, $BD \perp SO$ ).\\
				Trong mặt phẳng $(SAC)$, kẻ $OM \perp SC$, ($M \in SC$), suy ra $OM$ là đoạn vuông góc chung của $SC$ và $BD$. \\
				Trong $\triangle SOB$ vuông tại $O$ có: $$SO=\sqrt{ SB^2-BO^2} = \sqrt{a^2 -\dfrac{a^2}{3}}=\dfrac{a\sqrt{6}}{3}.$$
				Trong $\triangle BOC$ vuông tại $O$ có: $$OC=\sqrt{BC^2-BO^2}=\sqrt{a^2-\dfrac{a^2}{3}} = \dfrac{a\sqrt{6}}{3}.$$
			}{
				\begin{tikzpicture}[scale=0.8,font=\footnotesize, line join=round, line cap=round, >=stealth]
					\coordinate[label=left:{$A$}] (A) at (0,0);
					\coordinate[label=right:{$D$}] (D) at (5,0);
					\coordinate[label=below:{$B$}] (B) at (-2.5,-3);
					\coordinate (E) at (0,5);
					\coordinate[label = below:$C$] (C) at ($(D)+(B)-(A)$); 
					\coordinate[label=below:{$O$}] (O) at ($(A)!0.5!(C)$);
					\coordinate[label = above:$S$] (S) at ($(E)+(O)-(A)$); 
					\coordinate[label=below:{$G$}] (G) at ($(A)!0.5!(B)$);
					\coordinate[label=below:{$H$}] (H) at ($(C)!0.5!(D)$);
					\coordinate[label=right:{$M$}] (M) at ($(S)!0.6!(C)$);
					\coordinate[label=right:{$I$}] (I) at ($(S)!0.75!(H)$);
					\coordinate (P) at ($(S)!0.5!(B)$);
					\coordinate (Q) at ($(C)!0.5!(B)$);
					\coordinate[label=right:{$J$}] (J) at ($(S)!0.5!(H)$);
					\coordinate[label=right:{$K$}] (K) at ($(S)!0.5!(D)$);
					\coordinate (R) at ($(S)!0.5!(C)$);
					\coordinate[label=left:{$L$}] (L) at ($(G)!0.5!(A)$);			
					\draw (S)--(B) (S)--(C) (S)--(D) (S)--(H) (B)--(C) (C)--(D) (K)--(R);
					\draw[dashed] (A)--(B) (S)--(A) (A)--(D) (A)--(C) (B)--(D) (S)--(O) (O)--(M) (O)--(I) (G)--(H) (G)--(J) (L)--(K);
					\foreach \x/\dinh/\y in {O/M/S,O/I/H,G/H/C,S/K/L,R/J/H} \draw[fill = gray!50] ($(\dinh)!5pt!(\x)$)--($(\dinh)!5pt!(\x)+(\dinh)!5pt!(\y)-(\dinh)$)--($(\dinh)!5pt!(\y)$)--(\dinh)--cycle; 
					\foreach \p in {S,A,B,C,G,M,O,H,I,J,K,L} \fill (\p)	circle (1.2pt);
				\end{tikzpicture}
			}Trong $\triangle SOC$ vuông tại $O$ có: $SC=\sqrt{SO^2+OC^2}=\dfrac{2a\sqrt{3}}{3}.$ $$OM \cdot SC = OS \cdot OC \Leftrightarrow OM \cdot \dfrac{2a\sqrt{3}}{3} = \dfrac{a\sqrt{6}}{3} \cdot \dfrac{a\sqrt{6}}{3} \Rightarrow OM = \dfrac{a\sqrt{3}}{3}.$$
			\item 	Gọi $G$, $H$ lần lượt là trung điểm của $AB$ và $CD$. Ta có $$\heva{&CD \perp GH\\& CD \perp SO} \Rightarrow CD \perp (SGH) \Rightarrow (SCD) \perp (SGH).$$
			Từ $O$ dựng $OI \perp SH$ là giao tuyến của hai mặt phẳng vuông góc $(SCD)$ và $(SGH)$, suy ra $OI \perp (SCD)$.\\
			Trong mặt phẳng $(SGH)$, kẻ $GJ \parallel OI \; (J \in SH) \Rightarrow GJ \perp (SCD)$.\\
			Từ $J$ dựng đường thẳng song song với $CD$ cắt $SD$ tại $K$. Suy ra $AB$ và $JK$ cùng thuộc một mặt phẳng. Trong mặt phẳng $\left( AB, JK \right)$ dựng $KL \parallel GJ \; (L \in AB)$.\\
			Suy ra $KL$ là đoạn vuông góc chung của $AB$ và $SD$.\\
			Thật vậy: Vì $GJ \perp (SCD) \Rightarrow \heva{&GJ \perp SD\\& GJ \perp CD} \Rightarrow \heva{&GJ \perp SD\\& GJ \perp AB \; (AB \parallel CD)} \Rightarrow \heva{&KL\perp SD\\& KL \perp AB.}$\\
			Trong $\triangle SOH$: $\dfrac{1}{OI^2}=\dfrac{1}{OS^2} + \dfrac{1}{OH^2} = \dfrac{3}{2a^2}+\dfrac{4}{a^2}=\dfrac{11}{2a^2} \Rightarrow OI = \dfrac{a \sqrt{22}}{11}$.\\
			$\Rightarrow GJ = 2OI = \dfrac{2 a \sqrt{22}}{11}$.\\
			Kết luận: $\mathrm{d}(AB,SD)=\dfrac{2a\sqrt{22}}{11}$.
		\end{enumerate}
	}
\end{bt}
\subsubsection{Bài tập trắc nghiệm}
%\paragraph{Câu hỏi trắc nghiệm}
\Opensolutionfile{ans}[ans/ans-1K7-26-Dang6]%
\begin{ex}%[1K7KP-5]
	Cho hình chóp tam giác $S.ABC$ có $SA\perp (ABC)$, $AB=6$, $BC=8$, $AC=10$. Tính khoảng cách $\mathrm{d}$ giữa hai đường thẳng $SA$ và $BC$.
	\choice
	{$\mathrm{d}=0$}
	{$\mathrm{d}=8$}
	{$\mathrm{d}=10$}
	{\True $\mathrm{d}=6$}
	\loigiai{
		\immini{
			Ta có $AB=6$, $BC=8$, $AC=10$ nên $\Delta ABC$ vuông tại B.\\
			Khi đó $SA\perp AB$ và $BC\perp AB$ nên $AB$ là đoạn vuông góc chung của $SA$ và $BC$.\\
			Do hai đường này chéo nhau nên $\mathrm{d}(SA,BC)=AB=6$.
		}
		{
			\begin{tikzpicture}[scale=1, font=\footnotesize, line join=round, line cap=round, >=stealth,yscale=0.8, xscale=1.2]
				\coordinate[label=left:{$A$}] (A) at (0,0);
				\coordinate[label=below:{$B$}] (B) at (2,-1);
				\coordinate[label=right:{$C$}] (C) at (3,0);
				\coordinate[label=above:{$S$}] (S) at (0,3);
				\draw[dashed] (A)--(C);
				\draw (B)--(C) (B)--(A) (S)--(C) (S)--(A) (S)--(B);
				\foreach \x/\dinh/\y in {B/A/S,C/B/A} \draw[fill = gray!50] ($(\dinh)!5pt!(\x)$)--($(\dinh)!5pt!(\x)+(\dinh)!5pt!(\y)-(\dinh)$)--($(\dinh)!5pt!(\y)$)--(\dinh)--cycle; 
				\foreach \p in {S,A,B,C,D} \fill (\p)	circle (1.2pt);	
			\end{tikzpicture}
		}
		
	}
\end{ex}
\begin{ex}%[Tex hoá SGK KNTT]%Thien Tran Xuan]%[1K7KP-5]
	Cho tứ diện đều $ABCD$ có cạnh đáy bằng $2$. Tính khoảng cách giữa hai đường thẳng $AB$ và $CD$.
	\choice
	{$\sqrt{3}$}
	{$1$}
	{$\dfrac{\sqrt{3}}{2}$}
	{\True $\sqrt{2}$}
	\loigiai{\immini{Gọi $M$ là trung điểm của $DC$.\\
			Do $\triangle ACD$ và $\triangle BCD$ đều nên $\heva{&CD \perp AM\\&CD \perp BM}\Rightarrow CD \perp (ABM)$ tại $M$.\\	
			Kẻ $MN \perp AB$ tại $N$ thì $MN=\mathrm{d}(AB,CD)$.\\
			Do $\triangle MAB$ cân tại $M$ nên $N$ là trung điểm của $AB$.\\
			Xét $\triangle ANM$ vuông tại $N$. Ta có $AM = 2\cdot \dfrac{\sqrt{3}}{2}=\sqrt{3}$; 	$AN=\dfrac{AB}{2} = 1$.\\
			Suy ra $MN = \sqrt{AM^2 - AN^2} = \sqrt{2}$.\\
			Vậy $\mathrm{d}(AB,CD) = \sqrt{2}$.}
		{\hspace{0.5cm}
			\begin{tikzpicture}[>=stealth,line join=round,line cap=round,font=\footnotesize,scale=0.7]
				\coordinate[label=left:{$A$}] (A) at (0,0);
				\coordinate[label=above:{$B$}] (B) at (3.5,3);
				\coordinate[label=right:{$C$}] (C) at (5,0);
				\coordinate[label=below:{$D$}] (D) at (4,-3);
				\coordinate[label=right:{$M$}] (M) at ($(C)!0.5!(D)$);
				\coordinate[label=left:{$N$}] (N) at ($(A)!0.5!(B)$);
				\draw[dashed] (A)--(C) (A)--(M) (M)--(N);
				\draw (D)--(B)--(C)--cycle (D)--(A) (A)--(B) (B)--(M);
				\foreach \x/\dinh/\y in {D/M/A,B/N/M} \draw[fill = gray!50] ($(\dinh)!5pt!(\x)$)--($(\dinh)!5pt!(\x)+(\dinh)!5pt!(\y)-(\dinh)$)--($(\dinh)!5pt!(\y)$)--(\dinh)--cycle; 
				\foreach \p in {A,B,C,D,N,M} \fill (\p)	circle (1.2pt);	
		\end{tikzpicture}}
	}
\end{ex}

\begin{ex}%[Tex hoá SGK KNTT]%Thien Tran Xuan]%[1K7KP-5]
	Cho hình chóp $S.ABCD$ có đáy $ABCD$ là hình vuông cạnh $a$. Gọi $M$ và $N$ lần lượt là trung điểm của các cạnh $AB$ và $AD$, $H$ là giao điểm của $CN$ và $DM$. Biết $SH$ vuông góc với mặt phẳng $(ABCD)$ và $SH=a\sqrt{3}$. Tính khoảng cách giữa hai đường thẳng $DM$ và $SC$ theo $a$.
	\choice
	{\True $\dfrac{2\sqrt{3}a}{\sqrt{19}}$}
	{$\dfrac{2\sqrt{3}a}{19}$}
	{$\dfrac{\sqrt{3}a}{19}$}
	{$\dfrac{3\sqrt{3}a}{\sqrt{19}}$}
	\loigiai{
		\immini{Ta chứng minh $(SHC) \perp DM$ tại $H$.\\
			Thật vậy:
			\begin{itemize}
				\item $DM \perp  SH$ (do $SH \perp (ABCD)$).
				\item Mặt khác $\widehat{HCD}+\widehat{HDC}=\widehat{HDN}+\widehat{HDC}=90^\circ$ nên  $DM \perp HC$.
			\end{itemize}
			Suy ra $DM \perp (SHC)$ tại $H$.\\
			Kẻ $HK \perp SC$ tại $K$ thì $HK=\mathrm{d}(DM,SC)$.\\
			Ta có:
			\begin{itemize}
				\item $CN=\sqrt{DN^2+DC^2}=\dfrac{a\sqrt{5}}{2}$.
				\item $HC=\dfrac{DC^2}{CN}=\dfrac{2a\sqrt{5}}{5}$.
				\item $HK=\dfrac{SH\cdot HC}{SC}=\dfrac{SH\cdot HC}{\sqrt{SH^2+HC^2}}=\dfrac{2\sqrt{3}a}{\sqrt{19}}$.
			\end{itemize}
			Vậy $\mathrm{d}(DM,SC)=\dfrac{2\sqrt{3}a}{\sqrt{19}}$.
		}
		{\begin{tikzpicture}[>=stealth,line join=round,line cap=round,font=\footnotesize,scale=1]
				\def\s{5}
				\coordinate[label=above:{$O$}] (O) at (0,0);
				\coordinate[label=below:{$A$}] (A) at (-3,-1);
				\coordinate[label=below:{$D$}] (D) at (0.8,-1);			
				\coordinate[label = right:$C$] (C) at ($(A)!2!(O)$); 
				\coordinate[label = above left:$B$] (B) at ($(D)!2!(O)$); 		
				\coordinate[label=above:{$M$}] (M) at ($(A)!0.5!(B)$);
				\coordinate[label=below:{$N$}] (N) at ($(A)!0.5!(D)$);
				\path[name path=l1] (D)--(M);
				\path[name path=l2] (C)--(N);
				\path[name intersections={of=l1 and l2, by=H}];
				\draw (H) node[above left]{$H$};		
				\coordinate[label=above:{$S$}] (S) at ($(H)+(0,\s)$);
				\coordinate[label=above right:{$K$}] (K) at ($(S)!0.78!(C)$);			
				\draw (S)--(A) (S)--(D) (S)--(C) (A)--(D) (D)--(C);
				\draw[dashed] (S)--(H) (S)--(B) (A)--(B) (C)--(B) (C)--(N) (D)--(M) (H)--(K) (B)--(D);	
				\foreach \x/\dinh/\y in {D/H/C,H/K/C} \draw[fill = gray!50] ($(\dinh)!5pt!(\x)$)--($(\dinh)!5pt!(\x)+(\dinh)!5pt!(\y)-(\dinh)$)--($(\dinh)!5pt!(\y)$)--(\dinh)--cycle; 
				\foreach \p in {S,A,B,C,D,M,N,H,K,O} \fill (\p) circle (1.2pt);	
			\end{tikzpicture}
		}
	}
\end{ex}

\begin{ex}%[1K7KP-5]
	Cho hình hộp chữ nhật $ABCD.A'B'C'D'$ có $AB = a$, $AD = a\sqrt{3}$. Tính khoảng cách giữa hai đường thẳng $BB'$ và $AC'$.
	\choice
	{$ \dfrac{a\sqrt{2}}{2} $}
	{$ a\sqrt{3} $}
	{\True $ \dfrac{a\sqrt{3}}{2}$}
	{$\dfrac{a\sqrt{3}}{4} $}
	\loigiai{
		\immini
		{
			Vì $BB'\parallel AA'$ nên $BB'\parallel (ACC'A').$\\
			Suy ra $\mathrm{d}(BB',CC') = \mathrm{d}(BB',(ACC'A')) = \mathrm{d}(B,(ACC'A'))$.\\
			Kẻ $BH\perp AC$ tại $H$. Mặt khác $BH\perp AA'$ nên $BH \perp (ACC'A')$.\\
			Suy ra $\mathrm{d}(B,(ACC'A')) = BH$.\\
			Xét $\triangle ABC$ có			 $$\dfrac{1}{BH^2} = \dfrac{1}{AB^2} + \dfrac{1}{BC^2} = \dfrac{1}{a^2} + \dfrac{1}{3a^2}=\dfrac{4}{3a^2} \Rightarrow BH = \dfrac{a\sqrt{3}}{2}.$$
			Vậy $\mathrm{d}(B,(ACC'A')) = \dfrac{a\sqrt{3}}{2}.$
		}
		{
			\begin{tikzpicture}[line cap=round,line join=round,scale=0.7]
				\coordinate[label=below:{$D$}] (D) at (0,0);
				\coordinate[label=left:{$A$}] (A) at (2,2);
				\coordinate[label=right:{$B$}] (B) at (6,2);
				\coordinate[label=below:{$C$}] (C) at ($(D)+(B)-(A)$);
				\coordinate[label=above:{$A'$}] (A') at ($(A)+(0,4)$);
				\coordinate[label=above:{$B'$}] (B') at ($(B)+(A')-(A)$);
				\coordinate[label=right:{$C'$}] (C') at ($(C)+(A')-(A)$);
				\coordinate[label=above:{$D'$}] (D') at ($(D)+(A')-(A)$);
				\coordinate (C) at ($(B)+(D)-(A)$);
				\coordinate[label=left:{$H$}] (H) at ($(A)!0.5!(C)$);			
				\draw [dashed] (D)--(A)--(B)--(H) (A)--(A') (C)--(A)--(C');
				\draw (C')--(C)--(D) (D)--(D') (C)--(B) (B)--(B') (A')--(C');
				\draw (A')--(B')--(C')--(D')--cycle;
				\foreach \p in {A,B,C,D,A',B',C',D',H} \fill (\p) circle (1.2pt);
				\foreach \x/\dinh/\y in {A/H/B,A'/A/B,A/B/C} \draw[fill = gray!50] ($(\dinh)!5pt!(\x)$)--($(\dinh)!5pt!(\x)+(\dinh)!5pt!(\y)-(\dinh)$)--($(\dinh)!5pt!(\y)$)--(\dinh)--cycle; 
			\end{tikzpicture}
		}
		
	}
\end{ex}

\begin{ex}%[1K7KP-5]
	% \immini{
		Cho hình chóp $S.ABC$ có đáy $ABC$ là tam giác đều cạnh $a$. $SA$ vuông góc với mặt phẳng đáy và $SA=\dfrac{a}{2}$. Khoảng cách giữa hai đường thẳng $SA$ và $BC$ bằng
		\choice
		{$a\sqrt{3}$}
		{$a$}
		{$\dfrac{a\sqrt{3}}{4}$}
		{\True $\dfrac{a\sqrt{3}}{2}$}
	% }{
	% 	\begin{tikzpicture}[thick, scale=0.8]%15
	% 		\coordinate (A) at (0,0);
	% 		\coordinate (B) at (2,-2);
	% 		\coordinate (C) at (7,0);
	% 		\coordinate (S) at ($(O)+(0,4)$);
	% 		\draw (B) node[below]{$B$}--(S) node[above]{$S$}--(A) node[left]{$A$}--(B)--(C) node[right]{$C$}--(S);
	% 		\draw[dashed] (A)--(C);
	% 	\end{tikzpicture}
	% }
	\loigiai{
		\immini{
			Gọi $H$ là trung điểm $BC\Rightarrow AH\perp BC$ và $AH=\dfrac{a\sqrt{3}}{2}$.\\
			Vậy $\heva{&AH\perp AS\\&AH\perp BC}\Rightarrow AH$ là khoảng cách giữa $SA$ và $BC$.\\ Do đó $\mathrm{d}(SA,BC)=\dfrac{a\sqrt{3}}{2}$.
		}{
			\begin{tikzpicture}[thick, scale=0.8]%15
				\coordinate (A) at (0,0);
				\coordinate (B) at (2,-2);
				\coordinate (C) at (7,0);
				\coordinate (S) at ($(O)+(0,4)$);
				\coordinate (H) at ($(B)!.5!(C)$);
				\draw (B) node[below]{$B$}--(S) node[above]{$S$}--(A) node[left]{$A$}--(B)--(C) node[right]{$C$}--(S);
				\draw[dashed] (H)node[below]{$H$}--(A)--(C);
				%SHA
				\draw ($ (H)!5pt!(A)$)--($(H)!2!($($(H)!5pt!(A)$)!.5!($(H)!5pt!(B)$)$)$)--($(H)!5pt!(B)$);
				\draw ($ (A)!5pt!(A)$)--($(A)!2!($($(A)!5pt!(S)$)!.5!($(A)!5pt!(H)$)$)$)--($(A)!5pt!(H)$);
			\end{tikzpicture}
		}
	}
\end{ex}
\begin{ex}%[1K7KP-5]
	% \immini{
		Cho hình chóp $S.ABCD$ có đáy $ABCD$ là hình chữ nhật $AD=2a$. Cạnh bên $SA=2a$ và vuông góc với đáy. Khoảng cách giữa hai đường thẳng $AB$ và $SD$ bằng
		\choice
		{$a$}
		{$2a$}
		{$\dfrac{2a}{\sqrt{5}}$}
		{\True $a\sqrt{2}$}
	% }{
	% 	\begin{tikzpicture}[thick, scale=0.8]%23
	% 		\coordinate (B) at (0,0);
	% 		\coordinate (A) at (2,2);
	% 		\coordinate (C) at (6,0);
	% 		\coordinate (D) at ($(A)+(C)-(B)$);
	% 		\coordinate (S) at ($(A)+(0,4)$);
	% 		\draw (S) node[above]{$S$}--(C) node[below]{$C$}--(D) node[right]{$D$}--(S)--(B) node[below]{$B$}--(C);
	% 		\draw[dashed] (B)--(A) node[below]{$A$}--(S) (A)--(D)--(B);
	% 	\end{tikzpicture}
	% }
	\loigiai{
		\immini{
			Gọi $H$ là trung điểm $SD$, suy ra $AH\perp SD$.\\
			Ta có $\heva{&BA\perp SA\\& BA\perp AD}\Rightarrow BA\perp AH$.\\
			Vậy $AH$ là đường vuông góc chung của $BA$ và $SD$ nên khoảng cách giữa $BA$ và $SD$ bằng $AH=a\sqrt{2}$.
		}{
			\begin{tikzpicture}[thick, scale=0.8]%23
				\coordinate (B) at (0,0);
				\coordinate (A) at (2,2);
				\coordinate (C) at (6,0);
				\coordinate (D) at ($(A)+(C)-(B)$);
				\coordinate (S) at ($(A)+(0,4)$);
				\draw (S) node[above]{$S$}--(C) node[below]{$C$}--(D) node[right]{$D$}--(S)--(B) node[below]{$B$}--(C);
				\draw[dashed] (B)--(A) node[below]{$A$}--(S) (A)--(D)--(B);
				\coordinate (H) at ($(S)!.5!(D)$);
				\draw[dashed](A)--(H)node[above right]{$H$};
				\draw ($ (H)!5pt!(A)$)--($(H)!2!($($(H)!5pt!(A)$)!.5!($(H)!5pt!(S)$)$)$)--($(H)!5pt!(S)$);
			\end{tikzpicture}
		}
	}
\end{ex}
\begin{ex}%[1K7KP-5]
	% \immini{
		Cho tứ diện $OABC$ có $OA$, $OB$, $OC$ đôi một vuông góc với nhau và $OA=OB=OC=a$ (tham khảo hình vẽ). Khoảng cách giữa hai đường thẳng $OA$ và $BC$ bằng
		\choice
		{$\dfrac{a\sqrt{3}}{2}$}
		{$\dfrac{a}{2}$}
		{\True $\dfrac{a\sqrt{2}}{2}$}
		{$\dfrac{3a}{2}$}
	% }{
	% 	\begin{tikzpicture}[scale=1,font=\footnotesize,line join=round,line cap=round,>=stealth]
	% 		\path
	% 		(0,0) coordinate (O)
	% 		(-1.3,-1.5) coordinate (A)
	% 		(3,0) coordinate (B)
	% 		(0,3)coordinate (C)
	% 		;
	% 		\draw (A)--(B)--(C)--cycle;
	% 		\draw[dashed] (B)--(O)--(A) (O)--(C);
	% 		\foreach \p/\q in {A/180,B/0,C/90,O/180}
	% 		\fill[black] (\p) circle (1.0pt) ($(\p)+(\q:3mm)$) node{$\p$};
	% 		\draw pic[draw=black,angle radius=0.2cm] {right angle = C--O--B};
	% 		\draw pic[draw=black,angle radius=0.2cm] {right angle = A--O--B};			
	% 	\end{tikzpicture}
	% }
	\loigiai{
		\immini{
			Gọi $H$ là trung điểm của $BC$, vì $OB=OC=a\Rightarrow OH\perp BC$.\\
			Ta có $\heva{& OA\perp OB \\ & OA\perp OC}\Rightarrow OA\perp (OBC)\Rightarrow OA\perp OH$.\\
			Do đó $OH$ là đoạn vuông góc chung của $OA$ và $BC$ nên \[\mathrm{d}(OA,BC)=OH=\dfrac{a\sqrt{2}}{2}.\]
		}{
			\begin{tikzpicture}[scale=0.9,font=\footnotesize,line join=round,line cap=round,>=stealth]
				\path
				(0,0) coordinate (O)
				(-1.3,-1.5) coordinate (A)
				(3,0) coordinate (B)
				(0,3)coordinate (C)
				($(C)!0.5!(B)$) coordinate (H)
				;
				\draw (A)--(B)--(C)--cycle;
				\draw[dashed] (B)--(O)--(A) (H)--(O)--(C);
				\foreach \p/\q in {A/180,B/0,C/90,O/180,H/40}
				\fill[black] (\p) circle (1.0pt) ($(\p)+(\q:3mm)$) node{$\p$};
				\draw pic[draw=black,angle radius=0.2cm] {right angle = C--O--B};
				\draw pic[draw=black,angle radius=0.2cm] {right angle = A--O--B};
				\draw pic[draw=black,angle radius=0.2cm] {right angle = C--H--O};			
			\end{tikzpicture}
		}
	}
\end{ex}
\begin{ex}%[1K7BP-5]
	% \immini{
		Cho hình chóp $S.ABCD$ có đáy là hình vuông cạnh $a$, $SA$ vuông góc với đáy, $SA=a$. Khoảng cách giữa hai đường thẳng $SB$ và $CD$ bằng
		\choice
		{\True $a$}
		{$2a$}
		{$a\sqrt{2}$}
		{$a\sqrt{3}$}
	% }{
	% 	\begin{tikzpicture}[thick, scale=0.8]
	% 		\coordinate (B) at (0,0);
	% 		\coordinate (A) at (2,2);
	% 		\coordinate (C) at (6,0);
	% 		\coordinate (D) at ($(A)+(C)-(B)$);
	% 		\coordinate (S) at ($(A)+(0,4)$);
	% 		\draw (S) node[above]{$S$}--(C) node[below]{$C$}--(D) node[right]{$D$}--(S)--(B) node[below]{$B$}--(C);
	% 		\draw[dashed] (B)--(A) node[below]{$A$}--(S) (A)--(D)--(B);
	% 	\end{tikzpicture}
	% }
	\loigiai{
		\immini{
			Do $ABCD$ là hình vuông nên $BC\perp CD$; $BC\perp BA$.\\
			Có $\heva{&BC\perp BA\\&BC\perp SA}\Rightarrow BC\perp SB$.\\
			Vậy $BC$ là đường vuông góc chung của $SB$ và $CD$, do đó khoảng cách giữa $SB$ và $CD$ bằng $BC=a$.
		}{
			\begin{tikzpicture}[thick, scale=0.8]
				\coordinate (B) at (0,0);
				\coordinate (A) at (2,2);
				\coordinate (C) at (6,0);
				\coordinate (D) at ($(A)+(C)-(B)$);
				\coordinate (S) at ($(A)+(0,4)$);
				\draw (S) node[above]{$S$}--(C) node[below]{$C$}--(D) node[right]{$D$}--(S)--(B) node[below]{$B$}--(C);
				\draw[dashed] (B)--(A) node[below]{$A$}--(S) (A)--(D)--(B);
			\end{tikzpicture}
		}
	}
\end{ex}
\begin{ex}%[1K7KP-5]
	Cho tứ diện đều $ABCD$ cạnh $a$. Tính khoảng cách giữa hai đường thẳng $AB$ và $CD.$
	\choice
	{$\dfrac{a\sqrt{3}}{2}$}
	{\True $\dfrac{a\sqrt{2}}{2}$}
	{$\dfrac{a\sqrt{3}}{3}$}
	{$a$}
	\loigiai{
		\immini{
			Gọi $M$, $N$ lần lượt là trung điểm của $AB $ và $CD$.\\
			Ta có
			$\triangle ABC=\triangle ABD \Rightarrow MC=MD$.
			Suy ra $\triangle MCD \text{ cân tại }M \Rightarrow MN\bot CD$. \\
			$\triangle ABD=\triangle BCD\Rightarrow NA=NB$.\\
			Suy ra $\triangle NAB \text{ cân tại } N \Rightarrow MN\bot AB$. \\
			Suy ra $MN$ là đoạn vuông góc chung của $AB$, $CD$ nên $\mathrm{d}\left( AB,CD \right)=MN$. \\
			Trong $\triangle BMN$ có $MN=\sqrt{BN^2-BN^2}=\dfrac{a\sqrt{2}}{2}$.
		}{\begin{tikzpicture}[scale=0.6,font=\footnotesize,line join=round,line cap=round,>=stealth]
				\coordinate[label=left:{$B$}] (B) at (1,3);
				\coordinate[label=left:{$C$}] (C) at (3,0);
				\coordinate[label=right:{$D$}] (D) at (8,3);
				\coordinate[label=below:{$N$}] (N) at (5,1.25);
				\coordinate[label=above:{$A$}] (A) at (4,8);			
				\coordinate (K) at ($(A)!0.45!(N)$);
				\coordinate[label=left:{$M$}] (M) at ($(A)!0.5!(B)$);
				\draw[dashed] (B)--(C) (B)--(D) (M)--(D) (M)--(N);			
				\draw (A)--(B)--(C)--(D)--cycle (A)--(C) (M)--(C);
				\foreach \x/\dinh/\y in {M/N/C,N/M/B} \draw[fill = gray!50] ($(\dinh)!5pt!(\x)$)--($(\dinh)!5pt!(\x)+(\dinh)!5pt!(\y)-(\dinh)$)--($(\dinh)!5pt!(\y)$)--(\dinh)--cycle; 	 
			\end{tikzpicture}
			
		}
	}
\end{ex}
\begin{ex}%[1K7KP-5]
	Cho tứ diện đều $ABCD$ có cạnh bằng $6$. Khoảng cách giữa hai đường thẳng $AB$ và $CD$ bằng
	\choice
	{$3\sqrt{3}$}
	{\True $3\sqrt{2}$}
	{$3$}
	{$4$}
	\loigiai{
		\immini{
			Gọi $H$, $K$ lần lượt là trung điểm của $AB$, $CD$.\\
			Ta có $AK=BK=\dfrac{6\sqrt{3}}{2}=3\sqrt{3}$ (đường cao của hai tam giác đều $ACD$ và $BCD$)\\
			$\Rightarrow$ tam giác $ABK$ cân tại $K\\ \Rightarrow HK \perp AB$. $\quad (1)$\\
			Mặt khác $CD \perp (ABK)$ (do $AK\perp CD$, $BK\perp CD$)\\%(do $HK\subset(ABK)$)
			$\Rightarrow HK \perp CD$. $\quad (2)$\\
			Từ $(1)$, $(2)$ $\Rightarrow HK$ là đoạn vuông góc chung của hai đường thẳng $AB$ và $CD$.\\
			$\Rightarrow \mathrm{d}(AB,CD)=HK=\sqrt{AK^2-AH^2}=3\sqrt{2}.$\\
		}{
			\begin{tikzpicture}[scale=0.6,font=\footnotesize,line join=round,line cap=round,>=stealth]
				\coordinate[label=left:{$B$}] (B) at (1,3);
				\coordinate[label=left:{$C$}] (C) at (3,0);
				\coordinate[label=right:{$D$}] (D) at (8,3);
				\coordinate[label=below:{$K$}] (K) at (5,1.25);
				\coordinate[label=above:{$A$}] (A) at (4,8);			
				\coordinate (K) at ($(A)!0.45!(K)$);
				\coordinate[label=left:{$H$}] (H) at ($(A)!0.5!(B)$);
				\draw[dashed] (B)--(C) (B)--(D) (H)--(D) (H)--(N);			
				\draw (A)--(B)--(C)--(D)--cycle (A)--(C) (H)--(C);
				\foreach \x/\dinh/\y in {H/N/C,N/H/B} \draw[fill = gray!50] ($(\dinh)!5pt!(\x)$)--($(\dinh)!5pt!(\x)+(\dinh)!5pt!(\y)-(\dinh)$)--($(\dinh)!5pt!(\y)$)--(\dinh)--cycle; 	 
			\end{tikzpicture}
		}
	}
\end{ex}
\begin{ex}%[1K7KP-5]
	Cho tứ diện $ABCD$ có $AC=BC=AD=BD=a$, $CD=b$, $AB=c$. Khoảng cách giữa hai đường thẳng $AB$ và $CD$ bằng
	\choice
	{$\dfrac{\sqrt{3a^2-b^2-c^2}}{2}$}
	{\True $\dfrac{\sqrt{4a^2-b^2-c^2}}{2}$}
	{$\dfrac{\sqrt{a^2-b^2-c^2}}{2}$}
	{$\dfrac{\sqrt{2a^2-b^2-c^2}}{2}$}
	\loigiai{
		\immini{Gọi $M, N$ lần lượt là trung điểm của $AB$ và $CD$.\\
			Tam giác $ACD$ cân tại $A$ nên $AN \perp CD$.\\
			Tam giác $BCD$ cân tại $B$ nên $BN \perp CD$.\\
			Suy ra $CD \perp (ABN) \Rightarrow CD \perp MN$.\tagEX{1}
			Tam giác $ABC$ cân tại $C$ nên $CM \perp AB$.\\
			Tam giác $ABD$ cân tại $D$ nên $DM \perp AB$.\\
			Suy ra $AB \perp (CDM) \Rightarrow AB \perp MN$. \tagEX{2}
			Từ $(1)$ và $(2)$ suy ra $\mathrm{d} (AB, CD) = MN$.
		}
		{
			\begin{tikzpicture}[scale=0.8, font=\footnotesize, line join = round, line cap = round, >=stealth]
				\coordinate[label=above:{$A$}] (A) at (4,5);
				\coordinate[label=left:{$B$}] (B) at (1,2);
				\coordinate[label=below:{$C$}] (C) at (3,0);
				\coordinate[label=right:{$D$}] (D) at (7,2);
				\coordinate[label=left:{$M$}] (M) at ($(A)!0.5!(B)$);
				\coordinate[label=below:{$N$}] (N) at ($(C)!0.5!(D)$);
				\foreach \diem in {A,B,C,D,M,N}	\fill (\diem)circle(1.5pt);
				\draw[dashed] (B)--(D) (B)--(N) (M)--(N) (D)--(M);			
				\draw (A)--(B)--(C)--(D)--cycle (A)--(C) (A)--(N) (C)--(M);
			\end{tikzpicture}
		}
		\noindent Xét tam giác $ACM$ vuông tại $M$ nên $CM^2 = AC^2 - AM^2 = a^2 - \dfrac{c^2}{4}$.\\
		Xét tam giác $CMN$ vuông tại $N$ nên $MN = \sqrt{CM^2 - CN^2} = \sqrt{a^2 - \dfrac{c^2}{4} - \dfrac{b^2}{4}} = \dfrac{\sqrt{4a^2-b^2-c^2}}{2}$.
	}
\end{ex}
\begin{ex}%[1K7KP-5]
	\immini{
		Cho hình chóp $S.ABCD$ có đáy là hình vuông cạnh $a$, tam giác $SAB$ đều và nằm trong mặt phẳng vuông góc với đáy. Khoảng cách giữa hai đường thẳng $SA$ và $BC$ bằng
		\choice
		{\True $\dfrac{a\sqrt{3}}{2}$}
		{$a$}
		{$\dfrac{a\sqrt{3}}{4}$}
		{$\dfrac{a}{2}$}
	}{
		\begin{tikzpicture}[thick, scale=0.8]
			\coordinate (B) at (0,0);
			\coordinate (A) at (2,2);
			\coordinate (C) at (5,0);
			\coordinate (D) at ($(A)+(C)-(B)$);
			\coordinate (H) at ($(A)!0.5!(B)$);
			\coordinate (S) at ($(H)+(0,4)$);
			\draw (S) node[above]{$S$}--(C) node[below]{$C$}--(D) node[right]{$D$}--(S)--(B) node[below]{$B$}--(C);
			\draw[dashed] (B)--(A) node[above right]{$A$}--(S)--(H) node[below]{$H$} (A)--(D);
		\end{tikzpicture}
	}
	\loigiai{
		\immini{
			Gọi $K$ là trung điểm $SA$, do $\triangle SAB$ đều nên $BK\perp SA$ và $BK=\dfrac{a\sqrt{3}}{2}$\quad(1).\\
			Do $(SAB)\perp (ABCD)$, gọi $H$ là trung điểm $AB$ có\\ $SH\perp (ABCD)\Rightarrow SH\perp BC$.\\
			Có $\heva{&BC\perp AB\\&BC\perp SH}\Rightarrow BC\perp (SAB)\Rightarrow BC\perp BK\quad (2) $.\\
			Từ $(1)$, $(2)$ suy ra $BK$ là khoảng cách giữa $SA$ và $BC$.\\ Vậy $\mathrm{d}(SA,BC)=\dfrac{a\sqrt{3}}{2}$.
		}{
			\begin{tikzpicture}[thick, scale=0.8]
				\coordinate (B) at (0,0);
				\coordinate (A) at (2,2);
				\coordinate (C) at (5,0);
				\coordinate (D) at ($(A)+(C)-(B)$);
				\coordinate (H) at ($(A)!0.5!(B)$);
				\coordinate (S) at ($(H)+(0,4)$);
				\coordinate (K) at ($(S)!.5!(A)$);
				\draw (S) node[above]{$S$}--(C) node[below]{$C$}--(D) node[right]{$D$}--(S)--(B) node[below]{$B$}--(C);
				\draw[dashed] (B)--(A) node[above right]{$A$}--(S)--(H) node[below]{$H$} (A)--(D);
				\draw[dashed] (B)--(K)node[right]{$K$} ;
				\draw ($ (K)!5pt!(B)$)--($(K)!2!($($(K)!5pt!(B)$)!.5!($(K)!5pt!(S)$)$)$)--($(K)!5pt!(S)$);
				\draw ($ (H)!5pt!(S)$)--($(H)!2!($($(H)!5pt!(S)$)!.5!($(H)!5pt!(A)$)$)$)--($(H)!5pt!(A)$);
				\draw ($ (A)!5pt!(H)$)--($(A)!2!($($(A)!5pt!(H)$)!.5!($(A)!5pt!(D)$)$)$)--($(A)!5pt!(D)$);
			\end{tikzpicture}
		}
	}
\end{ex}
\begin{ex}%[1K7KP-5]
	Cho hình lăng trụ tam giác $ABC.A'B'C'$ có mặt bên $ABB'A'$ là hình thoi cạnh $a$, $\widehat{A'AB}=120^\circ$ và $A'C=BC=a\sqrt{3}$, $AC=\dfrac{\sqrt{10}}{2}a$. Tính khoảng cách giữa hai đường thẳng $A'B$ và $AC$.
	\choice
	{$\dfrac{\sqrt{10}}{20}a$}
	{$\dfrac{\sqrt{10}}{10}a$}
	{\True $\dfrac{3\sqrt{10}}{20}a$}
	{$\dfrac{3\sqrt{10}}{10}a$}
	\loigiai{
		\immini{
			Ta có $A'C=BC=a\sqrt{3}$,\\ $CI=\sqrt{BC^2-\dfrac{A'B^2}{4}}=\sqrt{3a^2-\dfrac{3a^2}{4}}=\dfrac{3a}{2}$; $AI=\dfrac{AB'}{2}=\dfrac{a}{2}$.\\
			$\Rightarrow\triangle A'BC$ cân tại $C$ và $A'B=2A'I=2A'A\cdot \sin 60=a\sqrt{3}$.\\
			Suy ra $\triangle A'BC$ đều, mặt khác ta lại thấy $AC^2=AI^2+IC^2$ nên suy ra $AI\perp IC$.\\
			Mà $AI\perp A'B$ nên suy ra $AI\perp\left(A'CB\right)$\\
			$\Rightarrow\left(A'BC\right)\perp\left(ABB'A'\right)$ và $A'B\perp\left(AIC\right)\Rightarrow A'B\perp AC$.\\
			Dựng $IE\perp AC$, $E\in AC$ thì đoạn $IE$ chính là đoạn vuông góc chung của hai đường thẳng $A'B$ và $AC$.
		}{
			\begin{tikzpicture}[scale=1, font=\footnotesize, line join=round, line cap=round, >=stealth]
				\def\a{4} \def\b{2.6}\def\h{3}
				\path
				(0:0) coordinate (A)
				(0:\a) coordinate (C)
				(-50:\b) coordinate (B)
				;
				\foreach \t in{A,B,C} \path ($(\t)+(75:\h)$) coordinate (\t');
				\foreach \t in{A,B,C} \draw (\t)--(\t');
				\draw[dashed] (A)--(C) (A')--(C)--($(A')!1/2!(B)$) coordinate (I)
				(I)--($(A)!1/2!(C)$) coordinate (E)
				;
				\draw (A')--(B')--(C')--cycle
				(A)--(B)--(C) (A')--(B) (B')--(A);
				\foreach \x/\g in {A/180,A'/180,B/180,B'/-45,C/0,C'/0,I/175,E/-100}
				\fill[black] (\x) circle(1pt) ($(\x)+(\g:3mm)$) node{$\x$};
			\end{tikzpicture}
		}
		\noindent
		Suy ra $\mathrm{d}\left(A'B;AC\right) = IE = \dfrac{CI\cdot IA}{AC} = \dfrac{3a}{2\sqrt{10}}= \dfrac{3\sqrt{10}}{20}a$.
	}
\end{ex}
\begin{ex}%[1K7KP-5]
	Cho hình hộp $ABCD.A'B'C'D'$ có $ABCD$ là hình thoi cạnh $a$, $AA' \perp (ABCD)$, $AA' = 2a$, $AC = a$. Tính khoảng cách giữa hai đường thẳng $BD$ và $A'C$.
	\choice
	{$a\sqrt{5}$}
	{$5a$}
	{\True $\dfrac{a\sqrt{5}}{5}$}
	{$\dfrac{a\sqrt{2}}{2}$}
	\loigiai{
		\immini{Gọi $O$, $E$ lần lượt là trung điểm $AC$ và hình chiếu của $O$ trên $A'C$.\\
			Vì $BD \perp (A'AC)$ nên $BD \perp OE$. Suy ra $\mathrm{d}(BD, A'C) = OE$.\\
			Ta có $A'C = \sqrt{A'A^2 + AC^2} = \sqrt{(2a)^2 + a^2} = a\sqrt{5}$.\\
			Do $\triangle CEO \sim \triangle CAA'$ nên 
			$$\dfrac{OE}{AA'} = \dfrac{OC}{A'C} \Rightarrow OE = \dfrac{AA' \cdot OC}{A'C} = \dfrac{2a \cdot \dfrac{a}{2}}{a\sqrt{5}} = \dfrac{a\sqrt{5}}{5}.$$
			Vậy $ \mathrm{d}(BD, A'C) = \dfrac{a\sqrt{5}}{5}$.
		}{\begin{tikzpicture}[scale=0.8, font=\footnotesize, line join=round, line cap=round, >=stealth]
				\def\bc{4} % cạnh BC
				\def\ba{2} % cạnh BA
				\def\h{4} % đường cao
				\def\gocB{35} % góc B của đáy
				\coordinate[label=below left:$B$] (B) at (0,0);
				\coordinate[label=above left:$A$] (A) at (\gocB:\ba);
				\coordinate[label=below:$C$] (C) at (\bc,0);
				\coordinate[label=right:$D$] (D) at ($(C)-(B)+(A)$);
				\coordinate[label=above left:$A'$] (A') at ($(A)+(90:\h)$);
				\coordinate[label=left:$B'$] (B') at ($(B)-(A)+(A')$);
				\coordinate[label=below right:$C'$] (C') at ($(C)-(A)+(A')$);
				\coordinate[label=right:$D'$] (D') at ($(D)-(A)+(A')$);
				\coordinate[label=below:$O$] (O) at ($(A)!0.5!(C)$);
				\coordinate[label=above right:$E$] (E) at ($(A')!0.7!(C)$);
				\draw (B')--(B)--(C)--(D)--(D')--(A')--(B')--(C')--(D') (C)--(C');
				\draw[dashed] (A')--(A)--(D) (A)--(B) (O)--(E) (B)--(D) (A)--(C) (A')--(C);
				\foreach \x/\dinh/\y in {C/E/O,A'/A/C} \draw[fill = gray!50] ($(\dinh)!5pt!(\x)$)--($(\dinh)!5pt!(\x)+(\dinh)!5pt!(\y)-(\dinh)$)--($(\dinh)!5pt!(\y)$)--(\dinh)--cycle; 
				\foreach \diem in {A,B,C,D,A',B',C',D',O,E}	\fill (\diem)circle(1.5pt);
			\end{tikzpicture}
		}
	}
\end{ex}
\begin{ex}%[1K7KP-5]
	Cho hình chóp tam giác $S.ABC$ có $SA$ vuông góc với $(ABC)$ và $SA = 2a$. Đáy $ABC$ là tam giác vuông cân tại $B$ với $AB = a$. Gọi $N$ là trung điểm của $AC$. Tìm độ dài đoạn vuông góc chung của hai đường thẳng $SN$ và $BC$.
	\choice
	{$2a\sqrt{17}$}
	{$\dfrac{2a\sqrt{17}}{15}$}
	{$\dfrac{2a\sqrt{17}}{9}$}
	{\True $\dfrac{2a\sqrt{17}}{17}$}
	\loigiai{
		\immini{
			Gọi $M$ là trung điểm $AB$, suy ra $BC \parallel MN \Rightarrow BC \parallel (SMN)$.\\
			Ta có $\heva{&MN \perp AB\\&MN \perp SA} \Rightarrow MN \perp (SAB) \Rightarrow (SMN) \perp (SAB)$ và $(SMN) \cap (SAB) = SN$.\\
			Hạ $BH \perp SN$ suy ra $BH \perp (SMN)$.\\
			Từ $H$ dựng $Hx$ song song với $BC$ và cắt $SN$ tại $E$.\\
			Từ $E$ dựng $Ey$ song song với $BH$ và cắt $BC$ tại $F$.\\
			Đoạn $EF$ là đoạn vuông góc chung của $SN$ và $BC$.\\
			Ta có $BN = \dfrac{AB}{2} = \dfrac{a}{2}$;\\ $SN = \sqrt{SA^2 + AN^2} = \sqrt{(2a)^2 + \left(\dfrac{a}{2}\right)^2} = \dfrac{a\sqrt{17}}{2}$.\\
			Mặt khác $\triangle SAM$ và $\triangle BHM$ là hai tam giác đồng dạng.\\
			Suy ra $\dfrac{BH}{SA} = \dfrac{BM}{SM} \Rightarrow BH = \dfrac{BM \cdot SA}{SM} = \dfrac{2a \cdot \dfrac{a}{2}}{\dfrac{a\sqrt{17}}{2}} = \dfrac{2a\sqrt{17}}{17}$.\\
			Vậy khoảng cách giữa $SN$ và $BC$ là $\dfrac{2a\sqrt{17}}{17}$.
		}{
			\begin{tikzpicture}[scale=1, font=\footnotesize, line join=round, line cap=round, >=stealth]
				\coordinate[label=left:{$A$}] (A) at (0,0);
				\coordinate[label=right:{$C$}] (C) at (5,0);
				\coordinate[label=right:{$B$}] (B) at (3.5,-2);
				\coordinate[label=above:{$S$}] (S) at (0,4);
				\coordinate[label=left:{$M$}] (M) at ($(A)!0.6!(B)$);
				\coordinate[label=above right:{$N$}] (N) at ($(A)!0.6!(C)$);
				\coordinate (M') at ($(S)!1.45!0:(M)$); 
				\coordinate[label = left:$H$] (H) at ($(S)!1.25!0:(M)$);
				\coordinate[label = left:$E$] (E) at ($(S)!1.25!0:(N)$); 
				\coordinate[label = right:$F$] (F) at ($(B)+(E)-(H)$); 
				\foreach \x/\dinh/\y in {S/E/F,E/F/C} \draw[fill = gray!50] ($(\dinh)!5pt!(\x)$)--($(\dinh)!5pt!(\x)+(\dinh)!5pt!(\y)-(\dinh)$)--($(\dinh)!5pt!(\y)$)--(\dinh)--cycle; 
				
				\draw (A)--(B) (C)--(B) (S)--(A) (S)--(C) (S)--(B) (S)--(M') (B)--(H);
				\draw[dashed](A)--(C) (M)--(N) (S)--(E)--(F) (H)--(E);			
				\foreach \x/\dinh/\y in {B/H/M',A/B/C} \draw[fill = gray!50] ($(\dinh)!5pt!(\x)$)--($(\dinh)!5pt!(\x)+(\dinh)!5pt!(\y)-(\dinh)$)--($(\dinh)!5pt!(\y)$)--(\dinh)--cycle; 
				\foreach \p in {S,A,B,C,M,H,N,E,F} \fill (\p)	circle (1.2pt);	
			\end{tikzpicture}
		}
	}
\end{ex}
\Closesolutionfile{ans}
\begin{indapan}{10}
	{ans/ans-1K7-26-Dang6}
\end{indapan}