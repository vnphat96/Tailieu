\begin{dang}{Hình chiếu vuông góc của đa giác lên mặt phẳng}

\end{dang}
\Opensolutionfile{ans}[ans/ans-1K7-25-Dang7]
\subsection{Ví dụ minh họa}
\begin{vd}%[1K7BO-6]
Cho tứ diện đều $ABCD$ cạnh $a$. Tính diện tích hình chiếu vuông góc của tam giác $ABC$ lên $(BCD)$.
\loigiai{
\immini{
Gọi $G$ là trọng tâm tam giác $BCD$. Vì $ABCD$ là tứ diện đều, nên $AG$ vuông góc với $(BCD)$, do đó hình chiếu vuông góc của $A$ lên $(BCD)$ là $G$.\\
Vậy hình chiếu của tam giác $ABC$ lên $(BCD)$ là tam giác $GBC$.\\
Diện tích tam giác $GBC$ là $S_{GBC}=\dfrac{1}{3}S_{BCD}=\dfrac{1}{3}\cdot \dfrac{a^2\sqrt{3}}{4}=\dfrac{a^2\sqrt{3}}{12}$.
}
{
\begin{tikzpicture}[scale=1, font=\footnotesize, line join=round, line cap=round, >=stealth]
\path
(0:0) coordinate (B)
(0:3) coordinate (C)
(-50:1.2) coordinate (D)
($(C)!1/2!(D)$) coordinate (M)
($(B)!2/3!(M)$) coordinate (G)
($(G)+(90:2.5)$) coordinate (A)
;
\draw[dashed] (B)--(C) (A)--(G);
\draw
(B)--(D)--(C)--(A)--(B)
(A)--(D)
;
\foreach \x/\g in {A/90,B/180,C/-40,G/-110,D/-90}
\fill[black] (\x) circle(1pt) ($(\x)+(\g:3mm)$) node{$\x$};
%	\draw ($(A')+(120:0.5)$)--++(-120:2) node[xshift=0.5cm, yshift=0.3cm]{$(\alpha)$}--++(0:3.5)--++(60:2);
\end{tikzpicture}
}
}
\end{vd}
\begin{vd}%[1K7BO-6]
Cho hình chóp $S.ABC$ có $SA$ vuông góc với $(ABC)$. Tam giác $ABC$ vuông tại $B$. Xác định các hình chiếu vuông góc của các tam giác lên mặt phẳng.
\begin{enumerate}
\item Tam giác $SBC$ lên $(ABC)$;
\item Tam giác $SAC$ lên $(SAB)$;
\item Tam giác $SAB$ lên $(SAC)$.
\end{enumerate}
\loigiai{
\immini{
\begin{enumerate}
\item Vì $SA \perp (ABC)$ nên hình chiếu của $S$ lên $(ABC)$ là điểm $A$. Do đó hình chiếu của tam giác $SBC$ lên $(ABC)$ là tam giác $ABC$.
\item Ta có $\heva{&CB \perp AB\\&CB \perp SA}$ nên $CB \perp (SAB)$, do đó hình chiếu của $C$ lên $(SAB)$ là điểm $B$. Suy ra hình chiếu của tam giác $SAC$ lên $(SAB)$ là tam giác $SAB$.
\item Kẻ $BH \perp AC$ tại $H$. Lại có $BH \perp SA$, nên $BH \perp (SAC)$. Do đó hình chiếu vuông góc của $B$ lên $(SAC)$ là điểm $H$. Suy ra hình chiếu vuông góc của tam giác $SAB$ lên $(SAC)$ là tam giác $SAH$.
\end{enumerate}
}
{
\begin{tikzpicture}[scale=1, font=\footnotesize, line join=round, line cap=round, >=stealth]
\def\a{3} \def\b{1.6} \def\h{2.2}
\path
(0:0) coordinate (A)
(0:\a) coordinate (C)
(-50:\b) coordinate (B)
($(A)+(90:\h)$) coordinate (S)
($(A)!1/3!(C)$) coordinate (H)
;
\draw (S)--(A)--(B)--(C)--(S)--(B);
\draw[dashed] (A)--(C) (S)--(H)--(B);
\foreach \x/\g in {S/90,A/180,B/-90,C/0,H/-40}
\fill[black] (\x) circle(1pt) ($(\x)+(\g:3mm)$) node{$\x$};
\end{tikzpicture}
}
}
\end{vd}
\begin{vd}%[1K7BO-6]
Cho hình chóp $S.ABCD$ có đáy là hình vuông, $SA$ vuông góc với mặt đáy. Xác định hình chiếu của các mặt hình chóp lên mặt phẳng tương ứng.
\begin{enumerate}
\item $SBC$ và $SCD$ lên $(ABCD)$;
\item $SBC$ lên $(SAD)$;
\item $SAB$ lên $(SAC)$.
\end{enumerate}
\loigiai{
\immini{
\begin{enumerate}
\item Vì $SA\perp (ABCD)$ nên hình chiếu vuông góc của $S$ lên $(ABCD)$ là điểm $A$.\\
Suy ra hình chiếu vuông góc của hai tam giác $SBC$ và $SCD$ lên $(ABCD)$ lần lượt là tam giác $ABC$ và $ACD$.
\item Ta có $\heva{&BA \perp SA\\&BA \perp AD}$ nên $BA \perp (SAD)$. Vì $CD \parallel AB$ nên $CD \perp (SAD)$. Do đó hình chiếu vuông góc của tam giác $SBC$ lên $(SAD)$ là $SAD$.
\item Gọi $O$ là giao điểm của $AC$ và $BD$. Khi đó $\heva{&BO \perp SA\\&BO \perp AC}$ nên $BO \perp (SAC)$. Suy ra hình chiếu vuông góc của tam giác $SAB$ lên $(SAC)$ là tam giác $SAO$.
\end{enumerate}
}
{
\begin{tikzpicture}[scale=1, font=\footnotesize, line join=round, line cap=round, >=stealth]
\def\a{2} \def\b{3} \def\h{2.2}
\path
(0:0) coordinate (A)
(0:\b) coordinate (D)
(-130:\a) coordinate (B)
($(B)+(D)-(A)$) coordinate (C)
($(A)+(90:\h)$) coordinate (S)
($(A)!1/2!(C)$) coordinate (O)
;
\draw (D)--(C)--(B)--(S)--(C) (S)--(D);
\draw[dashed] (O)--(S)--(A)--(B) (C)--(A)--(D)--(B);
\foreach \x/\g in {A/180,B/-90,C/-90,D/0,S/90,O/-90}
\fill[black] (\x) circle(1pt) ($(\x)+(\g:3mm)$) node{$\x$};
\end{tikzpicture}
}
}
\end{vd}
\begin{vd}%[1K7KO-6]
Cho hình chóp $S.ABCD$ có đáy $ABCD$ là hình chữ nhật, $AD=2AB=2a$. Tam giác $SAB$ là tam giác đều và nằm trong mặt phẳng vuông góc với đáy. Tính diện tích hình chiếu vuông góc của các tam giác
\begin{enumerate}
\item $SCD$ lên $(ABCD)$;
\item $SCD$ lên $(SAB)$.
\end{enumerate}
\loigiai{
\immini{
\begin{enumerate}
\item Gọi $H$ là trung điểm $AB$ thì $SH \perp AB$.\\
Vì $(SAB) \perp (ABCD)$ nên $SH \perp (ABCD)$.\\
Suy ra hình chiếu vuông góc của tam giác $SCD$ lên $(ABCD)$ là tam giác $HCD$.\\
Diện tích $S_{HCD}=\dfrac{1}{2}\cdot AD\cdot CD=\dfrac{1}{2}\cdot 2a\cdot a=a^2$.
\item Ta có $BC \perp SH$, lại có $BC \perp AB$ nên $BC \perp (SAB)$. Tương tự, ta có $AD \perp (SAB)$. Suy ra hình chiếu vuông góc của tam giác $SCD$ lên $(SAB)$ là tam giác $SAB$.\\
Diện tích $S_{SAB}=\dfrac{a^2\sqrt{3}}{4}$.
\end{enumerate}
}
{
\begin{tikzpicture}[scale=1, font=\footnotesize, line join=round, line cap=round, >=stealth]
\def\a{2} \def\b{3} \def\h{2.2}
\path
(0:0) coordinate (A)
(0:\b) coordinate (D)
(-130:\a) coordinate (B)
($(B)+(D)-(A)$) coordinate (C)
($(A)!1/2!(B)$) coordinate (H)
($(H)+(90:\h)$) coordinate (S)
;
\draw (D)--(C)--(B)--(S)--(C) (S)--(D);
\draw[dashed] (H)--(S)--(A)--(B) (A)--(D)--(H)--(C);
\foreach \x/\g in {A/180,B/-90,C/-90,D/0,S/90,H/-70}
\fill[black] (\x) circle(1pt) ($(\x)+(\g:3mm)$) node{$\x$};
\end{tikzpicture}
}
}
\end{vd}
\begin{vd}%[1K7KO-6]
Cho lăng trụ đứng $ABC.A'B'C'$ có $ABC$ là tam giác vuông cân tại $B$, $AB=AA'=a$. Xác định và tính diện tích hình chiếu vuông góc của tam giác $AA'C$ lên $(BCC'B')$.
\loigiai{
\immini{
Vì $ABC.A'B'C'$ là lăng trụ đứng nên $BB' \perp (A'B'C')$, suy ra $BB' \perp AB$. Lại có $AB \perp BC$, nên $AB \perp (BB'C'C)$. Tương tự $A'B' \perp (BB'C'C)$. Do đó hình chiếu vuông góc của tam giác $AA'C$ lên $(BCC'B')$ là tam giác $BB'C$.\\
Diện tích $S_{BB'C}=\dfrac{1}{2}\cdot BB'\cdot BC=\dfrac{1}{2}\cdot a\cdot a=\dfrac{a^2}{2}$.
}
{
\begin{tikzpicture}[scale=1, font=\footnotesize, line join=round, line cap=round, >=stealth]
\def\a{4} \def\b{3}\def\h{3}
\path
(0:0) coordinate (A)
(0:\a) coordinate (C)
(-30:\b) coordinate (B)
;
\foreach \t in{A,B,C} \path ($(\t)+(90:\h)$) coordinate (\t');
\foreach \t in{A,B,C} \draw (\t)--(\t');
\draw (A')--(B')--(C')--cycle
(A)--(B)--(C);
\draw[dashed] (A)--(C)--(B') (A')--(C);
\foreach \x/\g in {A/180,A'/180,B/180,B'/-150,C/0,C'/0}
\fill[black] (\x) circle(1pt) ($(\x)+(\g:3mm)$) node{$\x$};
\end{tikzpicture}
}
}
\end{vd}
\subsection{Trắc nghiệm}
\Opensolutionfile{ans}[ans/ans-1]
\setcounter{ex}{0}
\begin{ex}%[1K7BO-6]
\immini{
Cho hình chóp $S.ABC$ có $SA \perp (ABC)$. Hình chiếu vuông góc của tam giác $SBC$ lên $(ABC)$ là
\choice
{\True tam giác $ABC$}
{tam giác $SAB$}
{tam giác $SBC$}
{tam giác $SAC$}
}
{
\begin{tikzpicture}[scale=1, font=\footnotesize, line join=round, line cap=round, >=stealth]
\def\a{3} \def\b{1.8} \def\h{2.2}
\path
(0:0) coordinate (A)
(0:\a) coordinate (C)
(-60:\b) coordinate (B)
($(A)+(90:\h)$) coordinate (S)
;
\draw (S)--(A)--(B)--(C)--(S)--(B);
\draw[dashed] (A)--(C);
\foreach \x/\g in {S/90,A/180,B/-90,C/0}
\fill[black] (\x) circle(1pt) ($(\x)+(\g:3mm)$) node{$\x$};
\end{tikzpicture}
}
\loigiai{
Vì $SA \perp (ABC)$ nên hình chiếu của $S$ lên $(ABC)$ là $A$. Suy ra hình chiếu của $SBC$ lên $(ABC)$ là $ABC$.
}
\end{ex}

\begin{ex}%[1K7BO-6]
\immini{
Cho hình chóp $S.ABC$ có $SA \perp (ABC)$. Hình chiếu vuông góc của tam giác $SAB$ lên $(ABC)$ là
\choice
{tam giác $ABC$}
{tam giác $SAB$}
{tam giác $SBC$}
{\True tam giác $AB$}
}
{
\begin{tikzpicture}[scale=1, font=\footnotesize, line join=round, line cap=round, >=stealth]
\def\a{3} \def\b{1.8} \def\h{2.2}
\path
(0:0) coordinate (A)
(0:\a) coordinate (C)
(-60:\b) coordinate (B)
($(A)+(90:\h)$) coordinate (S)
;
\draw (S)--(A)--(B)--(C)--(S)--(B);
\draw[dashed] (A)--(C);
\foreach \x/\g in {S/90,A/180,B/-90,C/0}
\fill[black] (\x) circle(1pt) ($(\x)+(\g:3mm)$) node{$\x$};
\end{tikzpicture}
}
\loigiai{
Vì $SA \perp (ABC)$ nên hình chiếu của $S$ lên $(ABC)$ là $A$. Suy ra hình chiếu của tam giác $SAB$ lên $(ABC)$ là đoạn thẳng $AB$.
}
\end{ex}
\begin{ex}%[1K7BO-6]
Cho hình chóp $O.ABC$ có $OA$, $OB$, $OC$ đôi một vuông góc. Hình chiếu vuông góc của tam giác $OAB$ lên $(OAC)$ là
\choice
{tam giác $OAC$}
{tam giác $ABC$}
{đoạn thẳng $OB$}
{\True đoạn thẳng $OA$}
\loigiai{
\immini{
Ta có $\heva{&OB \perp OA\\&OB \perp OC}\Rightarrow OB \perp (OAC)$, nên hình chiếu của $B$ lên $(OAC)$ là $O$. Do đó hình chuiếu vuông góc của tam giác $OAB$ lên $(OAC)$ là đoạn thẳng $OA$.
}
{
\begin{tikzpicture}[scale=1, font=\footnotesize, line join=round, line cap=round, >=stealth]
\path
(0:0) coordinate (O)
(0:2) coordinate (B)
(-130:1) coordinate (C)
(90:2) coordinate (A)
;
\draw (A)--(B)--(C)--(A);
\draw[dashed] (B)--(O)--(A) (O)--(C);
\foreach \x/\g in {O/-90,B/0,A/90,C/-90}
\fill[black] (\x) circle(1pt) ($(\x)+(\g:3mm)$) node{$\x$};
\end{tikzpicture}
}
}
\end{ex}
\begin{ex}%[1K7BO-6]
Cho hình chóp $S.ABC$ có $ABC$ là tam giác đều cạnh $a$, $SA=SB=SC$. Tính diện tích hình chiếu vuông góc của tam giác $SAB$ lên $(ABC)$.
\choice
{$\dfrac{a^2}{3}$}
{$\dfrac{a^2\sqrt{3}}{4}$}
{\True $\dfrac{a^2\sqrt{3}}{12}$}
{$\dfrac{a^2\sqrt{3}}{6}$}
\loigiai{
\immini{
Gọi $H$	là hình chiếu của $S$ lên $(ABC)$, khi đó $SH \perp (ABC)$. Ba tam giác $SHA$, $SHB$, $SHC$ bằng nhau, nên $HA=HB=HC$, hay $H$ là trọng tâm tam giác $ABC$.\\
Suy ra hình chiếu vuông góc của tam giác $SAB$ lên $(ABC)$ là tam giác $HAB$.\\
Diện tích $S_{HAB}=\dfrac{1}{3}S_{ABC}=\dfrac{a^2\sqrt{3}}{12}$.
}
{
\begin{tikzpicture}[scale=1, font=\footnotesize, line join=round, line cap=round, >=stealth]
\def\a{3} \def\b{1.6} \def\h{2.2}
\path
(0:0) coordinate (A)
(0:\a) coordinate (B)
(-60:\b) coordinate (C)
($(B)!0.5!(C)$) coordinate (M)
($(A)!2/3!(M)$) coordinate (H)
($(H)+(90:\h)$) coordinate (S)
;
\draw (S)--(A)--(C)--(B)--(S)--(C);
\draw[dashed] (A)--(B)--(H) (S)--(H)--(A);
\foreach \x/\g in {A/180,B/0,C/-90,S/90,H/-90}
\fill[black] (\x) circle(1pt) ($(\x)+(\g:3mm)$) node{$\x$};
\end{tikzpicture}
}
}
\end{ex}
\begin{ex}%[1K7BO-6]
\immini{
Cho hình chóp tứ giác đều $S.ABCD$ có tất cả các cạnh đều bằng $a$. Tính diện tích hình chiếu của tam giác $SAB$ lên $(ABCD)$.
\choice
{$a^2$}
{$\dfrac{a^2}{2}$}
{\True $\dfrac{a^2}{4}$}
{$2a^2$}
}
{
\begin{tikzpicture}[scale=1, font=\footnotesize, line join=round, line cap=round, >=stealth]
\def\a{3} \def\b{2} \def\h{2.2}
\path
(0:0) coordinate (A)
(0:\a) coordinate (B)
(220:\b) coordinate (D)
($(B)+(D)-(A)$) coordinate (C)
($(A)!1/2!(C)$) coordinate (O)
($(O)+(90:\h)$) coordinate (S)
;
\draw (D)--(C)--(B)--(S)--(D) (S)--(C);
\draw[dashed] (S)--(A)--(B)--(D) (C)--(A)--(D);
\foreach \x/\g in {A/160,B/0,C/-90,D/-90,S/90,O/-90}
\fill[black] (\x) circle(1pt) ($(\x)+(\g:3mm)$) node{$\x$};
\end{tikzpicture}
}
\loigiai{
\immini{
Gọi $O$ là giao điểm của $AC$ và $BD$. Khi đó $SO \perp (ABCD)$. Suy ra hình chiếu vuông góc của tam giác $SAB$ lên $ABCD$ là tam giác $OAB$.\\
Diện tích $S_{OAB}=\dfrac{1}{4}S_{ABCD}=\dfrac{a^2}{4}$.
}
{
\begin{tikzpicture}[scale=1, font=\footnotesize, line join=round, line cap=round, >=stealth]
\def\a{3} \def\b{2} \def\h{2.2}
\path
(0:0) coordinate (A)
(0:\a) coordinate (B)
(220:\b) coordinate (D)
($(B)+(D)-(A)$) coordinate (C)
($(A)!1/2!(C)$) coordinate (O)
($(O)+(90:\h)$) coordinate (S)
;
\draw (D)--(C)--(B)--(S)--(D) (S)--(C);
\draw[dashed] (O)--(S)--(A)--(B)--(D) (C)--(A)--(D);
\foreach \x/\g in {A/160,B/0,C/-90,D/-90,S/90,O/-90}
\fill[black] (\x) circle(1pt) ($(\x)+(\g:3mm)$) node{$\x$};
\end{tikzpicture}
}
}
\end{ex}
\begin{ex}%[1K7BO-6]
\immini{
Cho hình chóp tứ giác đều $S.ABCD$ có tất cả các cạnh đều bằng $a$. Tính diện tích hình chiếu của tam giác $SAB$ lên $(SAC)$.
\choice
{$a^2$}
{$\dfrac{a^2}{2}$}
{\True $\dfrac{a^2}{4}$}
{$2a^2$}
}
{
\begin{tikzpicture}[scale=1, font=\footnotesize, line join=round, line cap=round, >=stealth]
\def\a{3} \def\b{2} \def\h{2.2}
\path
(0:0) coordinate (A)
(0:\a) coordinate (B)
(220:\b) coordinate (D)
($(B)+(D)-(A)$) coordinate (C)
($(A)!1/2!(C)$) coordinate (O)
($(O)+(90:\h)$) coordinate (S)
;
\draw (D)--(C)--(B)--(S)--(D) (S)--(C);
\draw[dashed] (S)--(A)--(B) (A)--(D);
\foreach \x/\g in {A/160,B/0,C/-90,D/-90,S/90}
\fill[black] (\x) circle(1pt) ($(\x)+(\g:3mm)$) node{$\x$};
\end{tikzpicture}
}
\loigiai{
\immini{
Gọi $O$ là giao điểm của $AC$ và $BD$.\\
Khi đó $SO \perp (ABCD)\Rightarrow SO \perp OB$.\\
Lại có $OB \perp AC$, nên $OB \perp (SAC)$.\\
Do đó hình chiếu vuông góc của tam giác $SAB$ lên $(SAC)$ là tam giác $SAO$.\\
Tam giác $SAC$ có $SA^2+SC^2=AC^2$ nên $SAC$ là tam giác vuông cân.\\
Diện tích $S_{SAO}=\dfrac{1}{2}S_{SAC}=\dfrac{1}{2}\cdot \dfrac{a^2}{2}=\dfrac{a^2}{4}$.
}
{
\begin{tikzpicture}[scale=1, font=\footnotesize, line join=round, line cap=round, >=stealth]
\def\a{3} \def\b{2} \def\h{2.2}
\path
(0:0) coordinate (A)
(0:\a) coordinate (B)
(220:\b) coordinate (D)
($(B)+(D)-(A)$) coordinate (C)
($(A)!1/2!(C)$) coordinate (O)
($(O)+(90:\h)$) coordinate (S)
;
\draw (D)--(C)--(B)--(S)--(D) (S)--(C);
\draw[dashed] (O)--(S)--(A)--(B)--(D) (C)--(A)--(D);
\foreach \x/\g in {A/160,B/0,C/-90,D/-90,S/90,O/-90}
\fill[black] (\x) circle(1pt) ($(\x)+(\g:3mm)$) node{$\x$};
\end{tikzpicture}
}
}
\end{ex}
\begin{ex}%[1K7BO-6]
Cho hình lăng trụ đều $ABC.A'B'C'$ có cạnh đáy bằng $a$, cạnh bên bằng $2a$. Tính diện tích hình chiếu vuông góc của $A'BC$ lên $(ABC)$
\choice
{$a^2$}
{\True $\dfrac{a^2\sqrt{3}}{4}$}
{$\dfrac{a^2}{2}$}
{$a^2\sqrt{3}$}
\loigiai{
\immini{
Vì $ABC.A'B'C'$	là lăng trụ đều nên $AA' \perp (ABC)$. Do đó hình chiếu vuông góc của tam giác $A'BC$ lên $(ABC)$ là tam giác $ABC$.\\
Diện tích $S_{ABC}=\dfrac{a^2\sqrt{3}}{4}$.
}
{
\begin{tikzpicture}[scale=1, font=\footnotesize, line join=round, line cap=round, >=stealth]
\def\a{4} \def\b{3}\def\h{3}
\path
(0:0) coordinate (A)
(0:\a) coordinate (C)
(-30:\b) coordinate (B)
;
\foreach \t in{A,B,C} \path ($(\t)+(90:\h)$) coordinate (\t');
\foreach \t in{A,B,C} \draw (\t)--(\t');
\draw (A')--(B')--(C')--cycle
(A)--(B)--(C);
\draw[dashed] (A)--(C);
\foreach \x/\g in {A/180,A'/180,B/200,B'/-45,C/0,C'/0}
\fill[black] (\x) circle(1pt) ($(\x)+(\g:3mm)$) node{$\x$};
\end{tikzpicture}
}
}
\end{ex}
\begin{ex}%[1K7KO-6]
Cho hình lăng trụ đều $ABC.A'B'C'$ có cạnh đáy bằng $a$, cạnh bên bằng $2a$. Tính diện tích hình chiếu vuông góc của tam giác $AA'B$ lên $(BCC'B')$
\choice
{$a^2$}
{$\dfrac{a^2\sqrt{3}}{4}$}
{\True $\dfrac{a^2}{2}$}
{$a^2\sqrt{3}$}
\loigiai{
\immini{
Gọi $M$, $M'$ lần lượt là trung điểm của $BC$ và $B'C'$. Ta có $\heva{&AM \perp BC\\&AM \perp BB'}\Rightarrow AM \perp (BB'C'C)$.\\
Chứng minh tương tự ta được $A'M' \perp (BB'C'C)$. Suy ra hình chiếu của tam giác $AA'B$ lên tam giác $(BB'C'C)$ là $MM'B$.\\
Vì $MM' \parallel BB'$ nên $MM' \perp BM$. Diện tích $$S_{MM'B}=\dfrac{1}{2}\cdot MM'\cdot MB=\dfrac{1}{2}\cdot 2a\cdot \dfrac{a}{2}=\dfrac{a^2}{2}.$$
}
{
\begin{tikzpicture}[scale=1, font=\footnotesize, line join=round, line cap=round, >=stealth]
\def\a{4} \def\b{3}\def\h{3}
\path
(0:0) coordinate (A)
(0:\a) coordinate (C)
(-30:\b) coordinate (B)
($(B)!1/2!(C)$) coordinate (M)
($(B')!1/2!(C')$) coordinate (M')
;
\foreach \t in{A,B,C} \path ($(\t)+(90:\h)$) coordinate (\t');
\foreach \t in{A,B,C} \draw (\t)--(\t');
\draw (B)--(A')--(B')--(C')--cycle
(A)--(B)--(C)
(A')--(M')--(M) (B)--(M');
\draw[dashed] (M)--(A)--(C);
\foreach \x/\g in {A/180,A'/180,B/200,B'/-145,C/0,C'/0,M/-60,M'/0}
\fill[black] (\x) circle(1pt) ($(\x)+(\g:3mm)$) node{$\x$};
\end{tikzpicture}
}
}
\end{ex}
\Closesolutionfile{ans}


\begin{indapan}{10}
{ans/ans-1K7-25-Dang7}
\end{indapan}\begin{dang}{Hình chiếu vuông góc của đa giác lên mặt phẳng}
	
\end{dang}

\subsection{Ví dụ minh họa}
\begin{vd}%[1K7BO-6]
	Cho tứ diện đều $ABCD$ cạnh $a$. Tính diện tích hình chiếu vuông góc của tam giác $ABC$ lên $(BCD)$.
	\loigiai{
		\immini{
			Gọi $G$ là trọng tâm tam giác $BCD$. Vì $ABCD$ là tứ diện đều, nên $AG$ vuông góc với $(BCD)$, do đó hình chiếu vuông góc của $A$ lên $(BCD)$ là $G$.\\
			Vậy hình chiếu của tam giác $ABC$ lên $(BCD)$ là tam giác $GBC$.\\
			Diện tích tam giác $GBC$ là $S_{GBC}=\dfrac{1}{3}S_{BCD}=\dfrac{1}{3}\cdot \dfrac{a^2\sqrt{3}}{4}=\dfrac{a^2\sqrt{3}}{12}$.
		}
		{
			\begin{tikzpicture}[scale=1, font=\footnotesize, line join=round, line cap=round, >=stealth]
				\path
				(0:0) coordinate (B)
				(0:3) coordinate (C)
				(-50:1.2) coordinate (D)
				($(C)!1/2!(D)$) coordinate (M)
				($(B)!2/3!(M)$) coordinate (G)
				($(G)+(90:2.5)$) coordinate (A)
				;
				\draw[dashed] (B)--(C) (A)--(G);
				\draw 
				(B)--(D)--(C)--(A)--(B)
				(A)--(D)
				;
				\foreach \x/\g in {A/90,B/180,C/-40,G/-110,D/-90}
				\fill[black] (\x) circle(1pt) ($(\x)+(\g:3mm)$) node{$\x$};
				%	\draw ($(A')+(120:0.5)$)--++(-120:2) node[xshift=0.5cm, yshift=0.3cm]{$(\alpha)$}--++(0:3.5)--++(60:2);
			\end{tikzpicture}
		}
	}
\end{vd}
\begin{vd}%[1K7BO-6]
	Cho hình chóp $S.ABC$ có $SA$ vuông góc với $(ABC)$. Tam giác $ABC$ vuông tại $B$. Xác định các hình chiếu vuông góc của các tam giác lên mặt phẳng.
	\begin{enumerate}
		\item Tam giác $SBC$ lên $(ABC)$;
		\item Tam giác $SAC$ lên $(SAB)$;
		\item Tam giác $SAB$ lên $(SAC)$.
	\end{enumerate}
	\loigiai{
		\immini{
			\begin{enumerate}
				\item Vì $SA \perp (ABC)$ nên hình chiếu của $S$ lên $(ABC)$ là điểm $A$. Do đó hình chiếu của tam giác $SBC$ lên $(ABC)$ là tam giác $ABC$.
				\item Ta có $\heva{&CB \perp AB\\&CB \perp SA}$ nên $CB \perp (SAB)$, do đó hình chiếu của $C$ lên $(SAB)$ là điểm $B$. Suy ra hình chiếu của tam giác $SAC$ lên $(SAB)$ là tam giác $SAB$.
				\item Kẻ $BH \perp AC$ tại $H$. Lại có $BH \perp SA$, nên $BH \perp (SAC)$. Do đó hình chiếu vuông góc của $B$ lên $(SAC)$ là điểm $H$. Suy ra hình chiếu vuông góc của tam giác $SAB$ lên $(SAC)$ là tam giác $SAH$.
			\end{enumerate}
		}
		{
			\begin{tikzpicture}[scale=1, font=\footnotesize, line join=round, line cap=round, >=stealth]
				\def\a{3} \def\b{1.6} \def\h{2.2}
				\path
				(0:0) coordinate (A)
				(0:\a) coordinate (C)
				(-50:\b) coordinate (B)
				($(A)+(90:\h)$) coordinate (S)
				($(A)!1/3!(C)$) coordinate (H)
				;
				\draw (S)--(A)--(B)--(C)--(S)--(B);
				\draw[dashed] (A)--(C) (S)--(H)--(B);
				\foreach \x/\g in {S/90,A/180,B/-90,C/0,H/-40}
				\fill[black] (\x) circle(1pt) ($(\x)+(\g:3mm)$) node{$\x$};
			\end{tikzpicture}
		}	
	}
\end{vd}
\begin{vd}%[1K7BO-6]
	Cho hình chóp $S.ABCD$ có đáy là hình vuông, $SA$ vuông góc với mặt đáy. Xác định hình chiếu của các mặt hình chóp lên mặt phẳng tương ứng.
	\begin{enumerate}
		\item $SBC$ và $SCD$ lên $(ABCD)$;
		\item $SBC$ lên $(SAD)$;
		\item $SAB$ lên $(SAC)$.
	\end{enumerate}
	\loigiai{
		\immini{
			\begin{enumerate}
				\item Vì $SA\perp (ABCD)$ nên hình chiếu vuông góc của $S$ lên $(ABCD)$ là điểm $A$.\\
				Suy ra hình chiếu vuông góc của hai tam giác $SBC$ và $SCD$ lên $(ABCD)$ lần lượt là tam giác $ABC$ và $ACD$.
				\item Ta có $\heva{&BA \perp SA\\&BA \perp AD}$ nên $BA \perp (SAD)$. Vì $CD \parallel AB$ nên $CD \perp (SAD)$. Do đó hình chiếu vuông góc của tam giác $SBC$ lên $(SAD)$ là $SAD$.
				\item Gọi $O$ là giao điểm của $AC$ và $BD$. Khi đó $\heva{&BO \perp SA\\&BO \perp AC}$ nên $BO \perp (SAC)$. Suy ra hình chiếu vuông góc của tam giác $SAB$ lên $(SAC)$ là tam giác $SAO$.
			\end{enumerate}
		}
		{
			\begin{tikzpicture}[scale=1, font=\footnotesize, line join=round, line cap=round, >=stealth]
				\def\a{2} \def\b{3} \def\h{2.2}
				\path
				(0:0) coordinate (A)
				(0:\b) coordinate (D)
				(-130:\a) coordinate (B)
				($(B)+(D)-(A)$) coordinate (C)
				($(A)+(90:\h)$) coordinate (S)
				($(A)!1/2!(C)$) coordinate (O)
				;
				\draw (D)--(C)--(B)--(S)--(C) (S)--(D);
				\draw[dashed] (O)--(S)--(A)--(B) (C)--(A)--(D)--(B);
				\foreach \x/\g in {A/180,B/-90,C/-90,D/0,S/90,O/-90}
				\fill[black] (\x) circle(1pt) ($(\x)+(\g:3mm)$) node{$\x$};
			\end{tikzpicture}
		}	
	}
\end{vd}
\begin{vd}%[1K7KO-6]
	Cho hình chóp $S.ABCD$ có đáy $ABCD$ là hình chữ nhật, $AD=2AB=2a$. Tam giác $SAB$ là tam giác đều và nằm trong mặt phẳng vuông góc với đáy. Tính diện tích hình chiếu vuông góc của các tam giác
	\begin{enumerate}
		\item $SCD$ lên $(ABCD)$;
		\item $SCD$ lên $(SAB)$.
	\end{enumerate}
	\loigiai{
		\immini{
			\begin{enumerate}
				\item Gọi $H$ là trung điểm $AB$ thì $SH \perp AB$.\\
				Vì $(SAB) \perp (ABCD)$ nên $SH \perp (ABCD)$.\\
				Suy ra hình chiếu vuông góc của tam giác $SCD$ lên $(ABCD)$ là tam giác $HCD$.\\
				Diện tích $S_{HCD}=\dfrac{1}{2}\cdot AD\cdot CD=\dfrac{1}{2}\cdot 2a\cdot a=a^2$.
				\item Ta có $BC \perp SH$, lại có $BC \perp AB$ nên $BC \perp (SAB)$. Tương tự, ta có $AD \perp (SAB)$. Suy ra hình chiếu vuông góc của tam giác $SCD$ lên $(SAB)$ là tam giác $SAB$.\\
				Diện tích $S_{SAB}=\dfrac{a^2\sqrt{3}}{4}$.
			\end{enumerate}
		}
		{
			\begin{tikzpicture}[scale=1, font=\footnotesize, line join=round, line cap=round, >=stealth]
				\def\a{2} \def\b{3} \def\h{2.2}
				\path
				(0:0) coordinate (A)
				(0:\b) coordinate (D)
				(-130:\a) coordinate (B)
				($(B)+(D)-(A)$) coordinate (C)
				($(A)!1/2!(B)$) coordinate (H)
				($(H)+(90:\h)$) coordinate (S)
				;
				\draw (D)--(C)--(B)--(S)--(C) (S)--(D);
				\draw[dashed] (H)--(S)--(A)--(B) (A)--(D)--(H)--(C);
				\foreach \x/\g in {A/180,B/-90,C/-90,D/0,S/90,H/-70}
				\fill[black] (\x) circle(1pt) ($(\x)+(\g:3mm)$) node{$\x$};
			\end{tikzpicture}
		}	
	}
\end{vd}
\begin{vd}%[1K7KO-6]
	Cho lăng trụ đứng $ABC.A'B'C'$ có $ABC$ là tam giác vuông cân tại $B$, $AB=AA'=a$. Xác định và tính diện tích hình chiếu vuông góc của tam giác $AA'C$ lên $(BCC'B')$.
	\loigiai{
		\immini{
			Vì $ABC.A'B'C'$ là lăng trụ đứng nên $BB' \perp (A'B'C')$, suy ra $BB' \perp AB$. Lại có $AB \perp BC$, nên $AB \perp (BB'C'C)$. Tương tự $A'B' \perp (BB'C'C)$. Do đó hình chiếu vuông góc của tam giác $AA'C$ lên $(BCC'B')$ là tam giác $BB'C$.\\
			Diện tích $S_{BB'C}=\dfrac{1}{2}\cdot BB'\cdot BC=\dfrac{1}{2}\cdot a\cdot a=\dfrac{a^2}{2}$.
		}
		{
			\begin{tikzpicture}[scale=1, font=\footnotesize, line join=round, line cap=round, >=stealth]
				\def\a{4} \def\b{3}\def\h{3}
				\path
				(0:0) coordinate (A)
				(0:\a) coordinate (C)
				(-30:\b) coordinate (B)
				;
				\foreach \t in{A,B,C} \path ($(\t)+(90:\h)$) coordinate (\t');
				\foreach \t in{A,B,C} \draw (\t)--(\t');
				\draw (A')--(B')--(C')--cycle
				(A)--(B)--(C);
				\draw[dashed] (A)--(C)--(B') (A')--(C);
				\foreach \x/\g in {A/180,A'/180,B/180,B'/-150,C/0,C'/0}
				\fill[black] (\x) circle(1pt) ($(\x)+(\g:3mm)$) node{$\x$};
			\end{tikzpicture}
		}	
	}
\end{vd}
\subsection{Trắc nghiệm}
\Opensolutionfile{ans}[ans/ans-1K7-25-Dang7]
\setcounter{ex}{0}
\begin{ex}%[1K7BO-6]
	\immini{
		Cho hình chóp $S.ABC$ có $SA \perp (ABC)$. Hình chiếu vuông góc của tam giác $SBC$ lên $(ABC)$ là
		\choice
		{\True tam giác $ABC$}
		{tam giác $SAB$}
		{tam giác $SBC$}
		{tam giác $SAC$}
	}
	{
		\begin{tikzpicture}[scale=1, font=\footnotesize, line join=round, line cap=round, >=stealth]
			\def\a{3} \def\b{1.8} \def\h{2.2}
			\path
			(0:0) coordinate (A)
			(0:\a) coordinate (C)
			(-60:\b) coordinate (B)
			($(A)+(90:\h)$) coordinate (S)
			;
			\draw (S)--(A)--(B)--(C)--(S)--(B);
			\draw[dashed] (A)--(C);
			\foreach \x/\g in {S/90,A/180,B/-90,C/0}
			\fill[black] (\x) circle(1pt) ($(\x)+(\g:3mm)$) node{$\x$};
		\end{tikzpicture}
	}
	\loigiai{
		Vì $SA \perp (ABC)$ nên hình chiếu của $S$ lên $(ABC)$ là $A$. Suy ra hình chiếu của $SBC$ lên $(ABC)$ là $ABC$.
	}
\end{ex}

\begin{ex}%[1K7BO-6]
	\immini{
		Cho hình chóp $S.ABC$ có $SA \perp (ABC)$. Hình chiếu vuông góc của tam giác $SAB$ lên $(ABC)$ là
		\choice
		{tam giác $ABC$}
		{tam giác $SAB$}
		{tam giác $SBC$}
		{\True tam giác $AB$}
	}
	{
		\begin{tikzpicture}[scale=1, font=\footnotesize, line join=round, line cap=round, >=stealth]
			\def\a{3} \def\b{1.8} \def\h{2.2}
			\path
			(0:0) coordinate (A)
			(0:\a) coordinate (C)
			(-60:\b) coordinate (B)
			($(A)+(90:\h)$) coordinate (S)
			;
			\draw (S)--(A)--(B)--(C)--(S)--(B);
			\draw[dashed] (A)--(C);
			\foreach \x/\g in {S/90,A/180,B/-90,C/0}
			\fill[black] (\x) circle(1pt) ($(\x)+(\g:3mm)$) node{$\x$};
		\end{tikzpicture}
	}
	\loigiai{
		Vì $SA \perp (ABC)$ nên hình chiếu của $S$ lên $(ABC)$ là $A$. Suy ra hình chiếu của tam giác $SAB$ lên $(ABC)$ là đoạn thẳng $AB$.
	}
\end{ex}
\begin{ex}%[1K7BO-6]
	Cho hình chóp $O.ABC$ có $OA$, $OB$, $OC$ đôi một vuông góc. Hình chiếu vuông góc của tam giác $OAB$ lên $(OAC)$ là
	\choice
	{tam giác $OAC$}
	{tam giác $ABC$}
	{đoạn thẳng $OB$}
	{\True đoạn thẳng $OA$}
	\loigiai{
		\immini{
			Ta có $\heva{&OB \perp OA\\&OB \perp OC}\Rightarrow OB \perp (OAC)$, nên hình chiếu của $B$ lên $(OAC)$ là $O$. Do đó hình chuiếu vuông góc của tam giác $OAB$ lên $(OAC)$ là đoạn thẳng $OA$.
		}
		{
			\begin{tikzpicture}[scale=1, font=\footnotesize, line join=round, line cap=round, >=stealth]
				\path
				(0:0) coordinate (O)
				(0:2) coordinate (B)
				(-130:1) coordinate (C)
				(90:2) coordinate (A)
				;
				\draw (A)--(B)--(C)--(A);
				\draw[dashed] (B)--(O)--(A) (O)--(C);
				\foreach \x/\g in {O/-90,B/0,A/90,C/-90}
				\fill[black] (\x) circle(1pt) ($(\x)+(\g:3mm)$) node{$\x$};
			\end{tikzpicture}
		}
	}
\end{ex}
\begin{ex}%[1K7BO-6]
	Cho hình chóp $S.ABC$ có $ABC$ là tam giác đều cạnh $a$, $SA=SB=SC$. Tính diện tích hình chiếu vuông góc của tam giác $SAB$ lên $(ABC)$.
	\choice
	{$\dfrac{a^2}{3}$}
	{$\dfrac{a^2\sqrt{3}}{4}$}
	{\True $\dfrac{a^2\sqrt{3}}{12}$}
	{$\dfrac{a^2\sqrt{3}}{6}$}
	\loigiai{
		\immini{
			Gọi $H$	là hình chiếu của $S$ lên $(ABC)$, khi đó $SH \perp (ABC)$. Ba tam giác $SHA$, $SHB$, $SHC$ bằng nhau, nên $HA=HB=HC$, hay $H$ là trọng tâm tam giác $ABC$.\\
			Suy ra hình chiếu vuông góc của tam giác $SAB$ lên $(ABC)$ là tam giác $HAB$.\\
			Diện tích $S_{HAB}=\dfrac{1}{3}S_{ABC}=\dfrac{a^2\sqrt{3}}{12}$.
		}
		{
			\begin{tikzpicture}[scale=1, font=\footnotesize, line join=round, line cap=round, >=stealth]
				\def\a{3} \def\b{1.6} \def\h{2.2}
				\path
				(0:0) coordinate (A)
				(0:\a) coordinate (B)
				(-60:\b) coordinate (C)
				($(B)!0.5!(C)$) coordinate (M)
				($(A)!2/3!(M)$) coordinate (H)
				($(H)+(90:\h)$) coordinate (S)
				;
				\draw (S)--(A)--(C)--(B)--(S)--(C);
				\draw[dashed] (A)--(B)--(H) (S)--(H)--(A);
				\foreach \x/\g in {A/180,B/0,C/-90,S/90,H/-90}
				\fill[black] (\x) circle(1pt) ($(\x)+(\g:3mm)$) node{$\x$};		
			\end{tikzpicture}
		}
	}
\end{ex}
\begin{ex}%[1K7BO-6]
	\immini{
		Cho hình chóp tứ giác đều $S.ABCD$ có tất cả các cạnh đều bằng $a$. Tính diện tích hình chiếu của tam giác $SAB$ lên $(ABCD)$.
		\choice
		{$a^2$}
		{$\dfrac{a^2}{2}$}
		{\True $\dfrac{a^2}{4}$}
		{$2a^2$}
	}
	{
		\begin{tikzpicture}[scale=1, font=\footnotesize, line join=round, line cap=round, >=stealth]
			\def\a{3} \def\b{2} \def\h{2.2}
			\path
			(0:0) coordinate (A)
			(0:\a) coordinate (B)
			(220:\b) coordinate (D)
			($(B)+(D)-(A)$) coordinate (C)
			($(A)!1/2!(C)$) coordinate (O)
			($(O)+(90:\h)$) coordinate (S)
			;
			\draw (D)--(C)--(B)--(S)--(D) (S)--(C);
			\draw[dashed] (S)--(A)--(B)--(D) (C)--(A)--(D);
			\foreach \x/\g in {A/160,B/0,C/-90,D/-90,S/90,O/-90}
			\fill[black] (\x) circle(1pt) ($(\x)+(\g:3mm)$) node{$\x$};		
		\end{tikzpicture}
	}
	\loigiai{
		\immini{
			Gọi $O$ là giao điểm của $AC$ và $BD$. Khi đó $SO \perp (ABCD)$. Suy ra hình chiếu vuông góc của tam giác $SAB$ lên $ABCD$ là tam giác $OAB$.\\
			Diện tích $S_{OAB}=\dfrac{1}{4}S_{ABCD}=\dfrac{a^2}{4}$.
		}
		{
			\begin{tikzpicture}[scale=1, font=\footnotesize, line join=round, line cap=round, >=stealth]
				\def\a{3} \def\b{2} \def\h{2.2}
				\path
				(0:0) coordinate (A)
				(0:\a) coordinate (B)
				(220:\b) coordinate (D)
				($(B)+(D)-(A)$) coordinate (C)
				($(A)!1/2!(C)$) coordinate (O)
				($(O)+(90:\h)$) coordinate (S)
				;
				\draw (D)--(C)--(B)--(S)--(D) (S)--(C);
				\draw[dashed] (O)--(S)--(A)--(B)--(D) (C)--(A)--(D);
				\foreach \x/\g in {A/160,B/0,C/-90,D/-90,S/90,O/-90}
				\fill[black] (\x) circle(1pt) ($(\x)+(\g:3mm)$) node{$\x$};		
			\end{tikzpicture}
		}
	}
\end{ex}
\begin{ex}%[1K7BO-6]
	\immini{
		Cho hình chóp tứ giác đều $S.ABCD$ có tất cả các cạnh đều bằng $a$. Tính diện tích hình chiếu của tam giác $SAB$ lên $(SAC)$.
		\choice
		{$a^2$}
		{$\dfrac{a^2}{2}$}
		{\True $\dfrac{a^2}{4}$}
		{$2a^2$}
	}
	{
		\begin{tikzpicture}[scale=1, font=\footnotesize, line join=round, line cap=round, >=stealth]
			\def\a{3} \def\b{2} \def\h{2.2}
			\path
			(0:0) coordinate (A)
			(0:\a) coordinate (B)
			(220:\b) coordinate (D)
			($(B)+(D)-(A)$) coordinate (C)
			($(A)!1/2!(C)$) coordinate (O)
			($(O)+(90:\h)$) coordinate (S)
			;
			\draw (D)--(C)--(B)--(S)--(D) (S)--(C);
			\draw[dashed] (S)--(A)--(B) (A)--(D);
			\foreach \x/\g in {A/160,B/0,C/-90,D/-90,S/90}
			\fill[black] (\x) circle(1pt) ($(\x)+(\g:3mm)$) node{$\x$};		
		\end{tikzpicture}
	}
	\loigiai{
		\immini{
			Gọi $O$ là giao điểm của $AC$ và $BD$.\\
			Khi đó $SO \perp (ABCD)\Rightarrow SO \perp OB$.\\
			Lại có $OB \perp AC$, nên $OB \perp (SAC)$.\\
			Do đó hình chiếu vuông góc của tam giác $SAB$ lên $(SAC)$ là tam giác $SAO$.\\
			Tam giác $SAC$ có $SA^2+SC^2=AC^2$ nên $SAC$ là tam giác vuông cân.\\
			Diện tích $S_{SAO}=\dfrac{1}{2}S_{SAC}=\dfrac{1}{2}\cdot \dfrac{a^2}{2}=\dfrac{a^2}{4}$.
		}
		{
			\begin{tikzpicture}[scale=1, font=\footnotesize, line join=round, line cap=round, >=stealth]
				\def\a{3} \def\b{2} \def\h{2.2}
				\path
				(0:0) coordinate (A)
				(0:\a) coordinate (B)
				(220:\b) coordinate (D)
				($(B)+(D)-(A)$) coordinate (C)
				($(A)!1/2!(C)$) coordinate (O)
				($(O)+(90:\h)$) coordinate (S)
				;
				\draw (D)--(C)--(B)--(S)--(D) (S)--(C);
				\draw[dashed] (O)--(S)--(A)--(B)--(D) (C)--(A)--(D);
				\foreach \x/\g in {A/160,B/0,C/-90,D/-90,S/90,O/-90}
				\fill[black] (\x) circle(1pt) ($(\x)+(\g:3mm)$) node{$\x$};		
			\end{tikzpicture}
		}
	}
\end{ex}
\begin{ex}%[1K7BO-6]
	Cho hình lăng trụ đều $ABC.A'B'C'$ có cạnh đáy bằng $a$, cạnh bên bằng $2a$. Tính diện tích hình chiếu vuông góc của $A'BC$ lên $(ABC)$
	\choice
	{$a^2$}
	{\True $\dfrac{a^2\sqrt{3}}{4}$}
	{$\dfrac{a^2}{2}$}
	{$a^2\sqrt{3}$}
	\loigiai{
		\immini{
			Vì $ABC.A'B'C'$	là lăng trụ đều nên $AA' \perp (ABC)$. Do đó hình chiếu vuông góc của tam giác $A'BC$ lên $(ABC)$ là tam giác $ABC$.\\
			Diện tích $S_{ABC}=\dfrac{a^2\sqrt{3}}{4}$.
		}
		{
			\begin{tikzpicture}[scale=1, font=\footnotesize, line join=round, line cap=round, >=stealth]
				\def\a{4} \def\b{3}\def\h{3}
				\path
				(0:0) coordinate (A)
				(0:\a) coordinate (C)
				(-30:\b) coordinate (B)
				;
				\foreach \t in{A,B,C} \path ($(\t)+(90:\h)$) coordinate (\t');
				\foreach \t in{A,B,C} \draw (\t)--(\t');
				\draw (A')--(B')--(C')--cycle
				(A)--(B)--(C);
				\draw[dashed] (A)--(C);
				\foreach \x/\g in {A/180,A'/180,B/200,B'/-45,C/0,C'/0}
				\fill[black] (\x) circle(1pt) ($(\x)+(\g:3mm)$) node{$\x$};
			\end{tikzpicture}
		}
	}
\end{ex}
\begin{ex}%[1K7KO-6]
	Cho hình lăng trụ đều $ABC.A'B'C'$ có cạnh đáy bằng $a$, cạnh bên bằng $2a$. Tính diện tích hình chiếu vuông góc của tam giác $AA'B$ lên $(BCC'B')$
	\choice
	{$a^2$}
	{$\dfrac{a^2\sqrt{3}}{4}$}
	{\True $\dfrac{a^2}{2}$}
	{$a^2\sqrt{3}$}
	\loigiai{
		\immini{
			Gọi $M$, $M'$ lần lượt là trung điểm của $BC$ và $B'C'$. Ta có $\heva{&AM \perp BC\\&AM \perp BB'}\Rightarrow AM \perp (BB'C'C)$.\\
			Chứng minh tương tự ta được $A'M' \perp (BB'C'C)$. Suy ra hình chiếu của tam giác $AA'B$ lên tam giác $(BB'C'C)$ là $MM'B$.\\
			Vì $MM' \parallel BB'$ nên $MM' \perp BM$. Diện tích $$S_{MM'B}=\dfrac{1}{2}\cdot MM'\cdot MB=\dfrac{1}{2}\cdot 2a\cdot \dfrac{a}{2}=\dfrac{a^2}{2}.$$
		}
		{
			\begin{tikzpicture}[scale=1, font=\footnotesize, line join=round, line cap=round, >=stealth]
				\def\a{4} \def\b{3}\def\h{3}
				\path
				(0:0) coordinate (A)
				(0:\a) coordinate (C)
				(-30:\b) coordinate (B)
				($(B)!1/2!(C)$) coordinate (M)
				($(B')!1/2!(C')$) coordinate (M')
				;
				\foreach \t in{A,B,C} \path ($(\t)+(90:\h)$) coordinate (\t');
				\foreach \t in{A,B,C} \draw (\t)--(\t');
				\draw (B)--(A')--(B')--(C')--cycle
				(A)--(B)--(C)
				(A')--(M')--(M) (B)--(M');
				\draw[dashed] (M)--(A)--(C);
				\foreach \x/\g in {A/180,A'/180,B/200,B'/-145,C/0,C'/0,M/-60,M'/0}
				\fill[black] (\x) circle(1pt) ($(\x)+(\g:3mm)$) node{$\x$};
			\end{tikzpicture}
		}
	}
\end{ex}
\Closesolutionfile{ans}


\begin{indapan}{10}
	{ans/ans-1K7-25-Dang7}
\end{indapan}