
\begin{dang}
	{Thể tích khối chóp cụt đều}
\end{dang}
\subsubsection{Ví dụ}

\begin{vd}%[DCHT Toán 11 - KNTT- Phạm Tuấn]%[1K7BQ-2]
	Tính thể tích của khối chóp cụt tứ giác đều có cạnh đáy nhỏ  $12~\mathrm{cm}$,  cạnh đáy lớn  $18~\mathrm{cm}$ và chiều cao bằng   $15~\mathrm{cm}$.
	\loigiai{
		Diện tích đáy nhỏ là  $S_1= 12^2=144 ~\mathrm{cm^2}$. \\
		Diện tích đáy lớn là  $S_1= 18^2=324 ~\mathrm{cm^2}$. \\
		Thể tích khối chóp cụt tứ giác đều  là
		$$V= \dfrac{h(S_1+S_2+\sqrt{S_1S_2})}{3} = \dfrac{15(144+324+\sqrt{144 \cdot 324})}{3} = 3240 ~\mathrm{cm^3}.$$
	}
\end{vd}

\begin{vd}%[DCHT Toán 11 - KNTT- Phạm Tuấn]%[1K7BQ-2]
	Cho khối chóp cụt tam giác đều $ABC.A'B'C'$ có chiều cao bằng $3a$, $AB=4a$, $A'B'=a$. Tính thể tích của khối chóp cụt đều $ABC.A'B'C'$.
	\loigiai
	{\immini{Diện tích tam giác đều $A B C$ là
			$$S_1=\dfrac{1}{2}AB\cdot AC\cdot\sin \widehat{BAC}=\dfrac{1}{2}\cdot 4a\cdot 4a\cdot\sin 60^{\circ}=4\sqrt{3}a^2.$$
			Diện tích tam giác đều $A'B'C'$ là
			$$S_2=\dfrac{1}{2}A'B'\cdot A'C'\cdot\sin \widehat{B'A'C'}=\dfrac{1}{2}\cdot a\cdot a\cdot\sin 60^{\circ}=\dfrac{\sqrt{3}a^2}{4}.$$
			Thể tích khối chóp cụt đều $ABC.A'B'C'$ là
			$$V=\dfrac{1}{3}\cdot 3a\cdot\left(4\sqrt{3}a^2+\sqrt{4\sqrt{3}a^2\cdot\dfrac{\sqrt{3}a^2}{4}}+\dfrac{\sqrt{3}a^2}{4}\right)
			=\dfrac{21\sqrt{3}a^3}{4}.$$}{
			\begin{tikzpicture}[scale=1, font=\footnotesize, line join=round, line cap=round,>=stealth]
				\def \k{0.6};
				\path (0,0)coordinate (A) (3.5,0) coordinate (C) (2.5,-1) coordinate (B) ($(B)!0.5!(C)$) coordinate (M) ($(A)!2/3!(M)$) coordinate (O) (0,3) coordinate (h) ($(O)+(h)$) coordinate (S);
				\foreach \p in {A,B,C}{\path ($(\p)!\k!(S)$) coordinate (\p');}
				\draw (A')--(A)--(B)--(C)--(C')--cycle (B')--(C') (A')--(B')--(B);
				\draw[dashed] (A)--(C);
				\foreach \p/\g in {A/150,B/-60,C/30, A'/150, B'/-30, C'/30} \fill[black] (\p) circle(1pt)+(\g:0.3) node{$\p$};
		\end{tikzpicture}}
	}
\end{vd}

\begin{vd}%[DCHT Toán 11 - KNTT- Phạm Tuấn]%[1K7BQ-2]
	Cho hình chóp đều $S.ABC$ có tất cả các cạnh bằng $2a$,  $M$ là trung điểm của $SA$. Gọi $(\alpha)$ là mặt phẳng qua $M$ và song song với mặt phẳng $(ABC)$. Mặt phẳng $(\alpha)$ cắt $SB$, $SC$ tại $N,K$. Tính thể tích khối chóp cụt đều $ABC.MNK$.
	\loigiai{
		\immini{
			Gọi $O$, $O'$ lần lượt là trọng tâm các tam giác $ABC$ và $MNK$. Hình chóp $S.ABC$ và  $S.MNK$ đều nên $SO \perp (ABC)$, $SO' \perp (MNK)$ suy ba điểm $S,O,O'$ thẳng hàng và $OO' \perp (ABC)$. \\
			Ta có  $SO = \sqrt{SA^2-AO^2}= \dfrac{2a\sqrt{6}}{3} \Rightarrow OO' = \dfrac{1}{2} SO = \dfrac{a\sqrt{6}}{3}$. \\
			Diện tích tam giác $MNK$ là $S_1 = \dfrac{a^2\sqrt{3}}{4}$. \\
			Diện tích tam giác $ABC$ là $S_2 = \dfrac{4a^2\sqrt{3}}{4} = a^2\sqrt{3}$. \\
			Thể tích khối chóp cụt đều là
			\begin{align*}
				V&= \dfrac{OO' (S_1+S_2+\sqrt{S_1S_2})}{3} \\
				&=  \dfrac{\dfrac{a\sqrt{6}}{3}  \left ( \dfrac{a^2\sqrt{3}}{4}+a^2\sqrt{3} + \sqrt{\dfrac{a^2\sqrt{3}}{4} \cdot a^2\sqrt{3}}\right )}{3} \\
				&= \dfrac{7a^3\sqrt{2}}{3} .
			\end{align*}
		}
		{
			\begin{tikzpicture}[scale=1, font=\footnotesize, line join=round, line cap=round,>=stealth]
				\path 
				(2,4) coordinate (S) 
				(0,0) coordinate (A) 
				(1.3,-1.2) coordinate (B)
				(4.5,0) coordinate (C) 
				($(S)!0.5!(A)$) coordinate (M) 
				($(S)!0.5!(B)$) coordinate (N)  
				($(S)!0.5!(C)$) coordinate (K)  
				($($(A)!0.5!(B)$)!{1/3}!(C)$) coordinate (O)
				($($(M)!0.5!(N)$)!{1/3}!(K)$) coordinate (O')
				;
				\draw (S)--(A)--(B)--(C)--(S)--(B)  (M)--(N)--(K);
				\draw[dashed] (A)--(C) (M)--(K) (S)--(O);
				\foreach \p/\g in {A/150,B/-90,C/0, M/150,N/-130,K/30,O/-40,S/90,O'/-10}
				\fill[black] (\p) circle(1pt)+(\g:0.3) node{$\p$};
			\end{tikzpicture}
		}
	}
\end{vd}

\begin{vd}%[DCHT Toán 11 - KNTT- Phạm Tuấn]%[1K7BQ-2]
	Tính thể tích khối chóp cụt đều  $ABC.A'B'C'$ biết  $AB=3a$, $A'B'=a$, $AA'=2a$. 
	\loigiai{
		\immini{
			Gọi $H, H'$ tương ứng là tâm của các tam giác $A B C, A' B' C'$.
			Khi đó, $H H'$ vuông góc với hai đáy của hình chóp cụt. \\
			Trong tam giác đều $ABC$ cạnh $3a$, ta có $AH=\dfrac{3a\sqrt{3}}{3} = a\sqrt{3}$. \\
			Trong tam giác đều $A' B' C'$ cạnh $a$, ta có $A'H'=\dfrac{a\sqrt{3}}{3}$. \\
			Hình thang $AHH' A'$ vuông tại $H$ và $H'$. Kẻ $A' M \perp H A$ $(M \in H A)$. Dễ thấy $A'MHH'$ là hình chữ nhật.  Suy ra $HH'= A'M$, $MH= A'H'$ và $AM= AH-MH = \dfrac{2a\sqrt{3}}{3}$. \\
			Trong tam giác vuông $AA'M$ ta có 
			\[
			A'M = \sqrt{AA'^2-AM^2} = \sqrt{4a^2 -  \dfrac{4a^2}{3}} = \dfrac{2a\sqrt{6}}{3}.
			\]
			Diện tích tam giác $A'B'C'$ là $S_1 = \dfrac{a^2\sqrt{3}}{4}$. \\
			Diện tích tam giác $ABC$ là $S_2 = \dfrac{9a^2\sqrt{3}}{4}$. 
		}
		{
			\begin{tikzpicture}[scale=1, font=\footnotesize, line join=round, line cap=round,>=stealth]
				\path
				(0,2) coordinate (A)
				(1.8,0) coordinate (B)
				(5,2) coordinate (C)
				(1,5) coordinate (A')
				(3.66,5) coordinate (C')
				(intersection of {$(A')+(B)-(A)$--A'} and {$(C')+(B)-(C)$--C'}) coordinate (B')
				(barycentric cs:A=1,B=1,C=1) coordinate (H)
				(barycentric cs:A'=1,B'=1,C'=1) coordinate (H')
				($(A')+(H)-(H')$) coordinate (M) ;
				\draw (A')--(A)--(B)--(B')--(A')--(C')--(B') (B)--(C)--(C') (A')--(H');
				\draw [dashed] (A)--(C) (H')--(H)--(A)  (A')--(M) ;
				%\draw pic[draw=black,angle radius=0.2cm]{right angle=H--M--A'};
				\foreach \x/\g in{A/190,B/-90,C/0,A'/150,B'/-140,C'/40,H/-10,H'/10,M/-90}
				\fill[black](\x)circle(1.1pt) ($(\x)+(\g:3.5mm)$) node{$\x$}; 
			\end{tikzpicture}
		}
		Thể tích khối chóp cụt đều $ABC.A'B'C'$ là 
		\[
		V= \dfrac{h(S_1+S_2+\sqrt{S_1S_2})}{3} = \dfrac{\dfrac{2a\sqrt{6}}{3} \left ( \dfrac{a^2\sqrt{3}}{4} + \dfrac{9a^2\sqrt{3}}{4} + \sqrt{\dfrac{a^2\sqrt{3}}{4} \cdot \dfrac{9a^2\sqrt{3}}{4}}\right ) }{3} = \dfrac{13a^3\sqrt{2}}{6}.
		\]
	}
\end{vd}

\begin{vd}%[DCHT Toán 11 - KNTT- Phạm Tuấn]%[1K7KQ-2]
	Cho khối chóp cụt tam giác đều $ABC.A'B'C'$ có cạnh đáy nhỏ bằng $a$,  cạnh đáy lớn bằng $2a$. Biết diện tích một mặt bên của khối chóp bằng $\dfrac{3a^2\sqrt{3}}{4}$. Tính thể tích khối chóp cụt  đã cho.
	\loigiai{
		\immini{
			Kẻ $A'E$ và $B'F$ vuông góc với $AB$. Hình thang $ABB'A'$ cân nên $A'B'FE$ là hình chữ nhật và $AE=BF = \dfrac{2a-a}{2} = \dfrac{a}{2}$.\\
			Theo giả thiết $S_{ABB'A'} = \dfrac{3a^2\sqrt{3}}{3} \Rightarrow A'E = \dfrac{2S_{ABB'A'}}{AB+A'B'} = \dfrac{a\sqrt{3}}{2}$. \\
			Trong tam giác vuông $AA'M$ ta có 
			\[
			AA' =\sqrt{AE^2+A'E^2} = \sqrt{\dfrac{a^2}{4} + \dfrac{3a^2}{4}} = a.
			\]
			Gọi $H, H'$ tương ứng là tâm của các tam giác $A B C, A' B' C'$.
			Khi đó, $H H'$ vuông góc với hai đáy của hình chóp cụt. \\
			Trong tam giác đều $ABC$ cạnh $2a$, ta có $AH=\dfrac{2a\sqrt{3}}{3}$. \\
			Trong tam giác đều $A' B' C'$ cạnh $a$, ta có $A'H'=\dfrac{a\sqrt{3}}{3}$. 
			
		}
		{
			\begin{tikzpicture}[scale=1, font=\footnotesize, line join=round, line cap=round,>=stealth]
				\path
				(0,2) coordinate (A)
				(1.8,0) coordinate (B)
				(5,2) coordinate (C)
				(1,5) coordinate (A')
				(3.66,5) coordinate (C')
				(intersection of {$(A')+(B)-(A)$--A'} and {$(C')+(B)-(C)$--C'}) coordinate (B')
				(barycentric cs:A=1,B=1,C=1) coordinate (H)
				(barycentric cs:A'=1,B'=1,C'=1) coordinate (H')
				($(A')+(H)-(H')$) coordinate (M) 
				($(A)!0.24!(B)$) coordinate (E)  (intersection of {$(B')+(E)-(A')$--B'} and {A--B}) coordinate (F) ;
				\draw (A')--(A)--(B)--(B')--(A')--(C')--(B') (B)--(C)--(C') (A')--(H') (A')--(E)  (B')--(F);
				\draw [dashed] (A)--(C) (H')--(H)--(A)  (A')--(M) ;
				%\draw pic[draw=black,angle radius=0.2cm]{right angle=H--M--A'};
				\foreach \x/\g in{A/190,B/-90,C/0,A'/150,B'/-140,C'/40,H/-10,H'/10,M/-90,E/-120,F/-120}
				\fill[black](\x)circle(1.1pt) ($(\x)+(\g:3.5mm)$) node{$\x$}; 
			\end{tikzpicture}
		} 
		\noindent
		Hình thang $AHH' A'$ vuông tại $H$ và $H'$. Kẻ $A' M \perp H A$ $(M \in H A)$. Dễ thấy $A'MHH'$ là hình chữ nhật.  Suy ra $HH'= A'M$, $MH= A'H'$ và $AM= AH-MH = \dfrac{a\sqrt{3}}{3}$. \\
		Trong tam giác vuông $AA'M$ ta có 
		\[
		A'M = \sqrt{AA'^2-AM^2} = \sqrt{a^2 -  \dfrac{a^2}{3}} = \dfrac{a\sqrt{6}}{3}.
		\]
		Diện tích tam giác $A'B'C'$ là $S_1 = \dfrac{a^2\sqrt{3}}{4}$. \\
		Diện tích tam giác $ABC$ là $S_2 = \dfrac{4a^2\sqrt{3}}{4} = a^2\sqrt{3}$.  \\
		Thể tích khối chóp cụt đều $ABC.A'B'C'$ là 
		\[
		V= \dfrac{h(S_1+S_2+\sqrt{S_1S_2})}{3} = \dfrac{\dfrac{a\sqrt{6}}{3} \left ( \dfrac{a^2\sqrt{3}}{4} + a^2\sqrt{3} + \sqrt{\dfrac{a^2\sqrt{3}}{4} \cdot a^2\sqrt{3}}\right ) }{3} = \dfrac{7a^3\sqrt{2}}{12}.
		\]
	}
\end{vd}

\subsubsection{Bài tập rèn luyện}
\Opensolutionfile{ans}[ans/ans-1K7-27-Dang3]

%\noindent \textbf{Bài tập tự luận}
\setcounter{bt}{0}
\begin{bt}%[DCHT Toán 11 - KNTT- Phạm Tuấn]%[1K7BQ-2]
	Tính thể tích của khối chóp cụt tứ giác đều có cạnh đáy nhỏ  $4~\mathrm{cm}$,  cạnh đáy lớn  $25~\mathrm{cm}$ và chiều cao bằng   $12~\mathrm{cm}$.
	\loigiai{
		Diện tích đáy nhỏ là  $S_1= 4^2=16 ~\mathrm{cm^2}$. \\
		Diện tích đáy lớn là  $S_1= 25^2=625 ~\mathrm{cm^2}$. \\
		Thể tích khối chóp cụt tứ giác đều  là
		\[
		V= \dfrac{h(S_1+S_2+\sqrt{S_1S_2})}{3} = \dfrac{12(16+625+\sqrt{16 \cdot 625})}{3} = 250064 ~\mathrm{cm^3}.
		\]
	}
\end{bt}

\begin{bt}%[DCHT Toán 11 - KNTT- Phạm Tuấn]%[1K7BQ-2]
	Một  khối chóp cụt đều có diện tích các đáy lần lượt là $3~\mathrm{cm^2}$ và $12~\mathrm{cm^2}$. Biết thể tích khối chóp cụt đều đã cho bằng $42 ~\mathrm{cm^3}$. Tính chiều cao $h$ của khối chóp cụt. 
	\loigiai{
		Ta có $V = \dfrac{h(S_1+S_2+\sqrt{S_1S_2})}{3} \Rightarrow h= \dfrac{3V}{S_1+S_2+\sqrt{S_1S_2}}= \dfrac{3 \cdot 42}{3+12+6}= 6 ~\mathrm{cm}$. 
	}
\end{bt}

\begin{bt}%[DCHT Toán 11 - KNTT- Phạm Tuấn]%[1K7BQ-2]
	Cho khối chóp cụt đều có diện tích đáy lớn gấp $9$ lần đáy nhỏ. Nếu tăng $4$ lần diện tích đáy nhỏ  nhưng giữ nguyên diện tích đáy lớn và chiều cao của khối chóp cụt đều thì thể tích của khối chóp cụt đều tăng bao nhiêu lần?
	\loigiai{
		Gọi diện tích đáy nhỏ, diện tích đáy lớn, chiều cao của khối chóp cụt đều đã cho là $S$, $9S$, $h$.  \\
		Thể tích của khối chóp cụt đều ban đầu là $V = \dfrac{h(S+9S+\sqrt{S \cdot 9S})}{3} = \dfrac{13hS}{3}$. \\
		Diện tích các đáy và chiều cao của khối chóp cụt mới là $4S$, $9S$, $h$. \\
		Thể tích của khối chóp cụt đều mới là $V' = \dfrac{h(4S+9S+\sqrt{4S \cdot 9S})}{3} = \dfrac{19hS}{3}$. \\
		Ta có $$\dfrac{V'}{V} =  \dfrac{\dfrac{19hS}{3}}{\dfrac{13hS}{3}} = \dfrac{19}{13}.$$
		Vậy thể tích khối chóp cụt tăng $\dfrac{19}{13}$ lần
	}
\end{bt}





\begin{bt}%[DCHT Toán 11 - KNTT- Phạm Tuấn]%[1K7BQ-2] 
	Một chụp đèn hình chóp cụt đều  có chiều cao bằng $24$ cm, đáy là lục giác đều, độ dài cạnh đáy lớn bằng $17{,}5$ cm và độ dài cạnh đáy nhỏ bằng $10{,}5$ cm. Tính thể tích phần không gian bên trong của chụp đèn này.
	\loigiai{
		Diện tích đáy nhỏ là $S_1 = 6 \cdot \dfrac{10{,}5^2\sqrt{3}}{4} = \dfrac{1323\sqrt{3}}{8}$.  \\
		Diện tích đáy lớn là $S_2 = 6 \cdot \dfrac{17{,}5^2\sqrt{3}}{4} = \dfrac{3675\sqrt{3}}{8}$.  \\
		Thể tích phần không gian bên trong của chụp đèn là
		\[
		V= \dfrac{h(S_1+S_2+\sqrt{S_1S_2})}{3} = \dfrac{24 \left (\dfrac{1323\sqrt{3}}{8} + \dfrac{3675\sqrt{3}}{8} +\sqrt{\dfrac{1323\sqrt{3}}{8} \cdot \dfrac{3675\sqrt{3}}{8}}\right )}{3} = 7203\sqrt{3}~\mathrm{cm^3}.
		\]
	}
\end{bt}

\begin{bt}%[DCHT Toán 11 - KNTT- Phạm Tuấn]%[1K7KQ-2]
	Tính thể tích khối chóp cụt đều  $ABC.A'B'C'$ biết  $AB=2a$, $A'B'=AA'=a$. 
	\loigiai{
		\immini{
			Gọi $H, H'$ tương ứng là tâm của các tam giác $A B C, A' B' C'$.
			Khi đó, $H H'$ vuông góc với hai đáy của hình chóp cụt. \\
			Trong tam giác đều $ABC$ cạnh $2a$, ta có $AH=\dfrac{2a\sqrt{3}}{3}$. \\
			Trong tam giác đều $A' B' C'$ cạnh $a$, ta có $A'H'=\dfrac{a\sqrt{3}}{3}$. \\
			Hình thang $AHH' A'$ vuông tại $H$ và $H'$. Kẻ $A' M \perp H A$ $(M \in HA)$. Dễ thấy $A'MHH'$ là hình chữ nhật.  Suy ra $HH'= A'M$, $MH= A'H'$ và $AM= AH-MH = \dfrac{a\sqrt{3}}{3}$. 
		}
		{
			\begin{tikzpicture}[scale=1, font=\footnotesize, line join=round, line cap=round,>=stealth]
				\path
				(0,2) coordinate (A)
				(1.8,0) coordinate (B)
				(5,2) coordinate (C)
				(1,5) coordinate (A')
				(3.66,5) coordinate (C')
				(intersection of {$(A')+(B)-(A)$--A'} and {$(C')+(B)-(C)$--C'}) coordinate (B')
				(barycentric cs:A=1,B=1,C=1) coordinate (H)
				(barycentric cs:A'=1,B'=1,C'=1) coordinate (H')
				($(A')+(H)-(H')$) coordinate (M) ;
				\draw (A')--(A)--(B)--(B')--(A')--(C')--(B') (B)--(C)--(C') (A')--(H');
				\draw [dashed] (A)--(C) (H')--(H)--(A)  (A')--(M) ;
				%\draw pic[draw=black,angle radius=0.2cm]{right angle=H--M--A'};
				\foreach \x/\g in{A/190,B/-90,C/0,A'/150,B'/-140,C'/40,H/-10,H'/10,M/-90}
				\fill[black](\x)circle(1.1pt) ($(\x)+(\g:3.5mm)$) node{$\x$}; 
			\end{tikzpicture}
		}
		\noindent
		Trong tam giác vuông $AA'M$ ta có 
		\[
		A'M = \sqrt{AA'^2-AM^2} = \sqrt{a^2 -  \dfrac{a^2}{3}} = \dfrac{a\sqrt{6}}{3}.
		\]
		Diện tích tam giác $A'B'C'$ là $S_1 = \dfrac{a^2\sqrt{3}}{4}$. \\
		Diện tích tam giác $ABC$ là $S_2 = \dfrac{4a^2\sqrt{3}}{4}= a^2\sqrt{3}$. \\
		Thể tích khối chóp cụt đều $ABC.A'B'C'$ là 
		\[
		V= \dfrac{h(S_1+S_2+\sqrt{S_1S_2})}{3} = \dfrac{\dfrac{a\sqrt{6}}{3} \left ( \dfrac{a^2\sqrt{3}}{4} + a^2\sqrt{3} + \sqrt{\dfrac{a^2\sqrt{3}}{4} \cdot a^2\sqrt{3}} \right ) }{3} = \dfrac{7a^3\sqrt{2}}{12}.
		\]
	}
\end{bt}

\begin{bt}%[DCHT Toán 11 - KNTT- Phạm Tuấn]%[1K7KQ-2]
	Cho khối chóp cụt đều có diện tích đáy lớn gấp $4$ lần đáy nhỏ. Nếu tăng $\dfrac{3}{2}$ lần diện tích đáy lớn  nhưng giữ nguyên  chiều cao của khối chóp cụt đều thì để thể tích của khối chóp cụt không đổi ta cần giảm diện tích đáy nhỏ bao nhiêu lần?
	\loigiai{
		Gọi diện tích đáy nhỏ, diện tích đáy lớn, chiều cao của khối chóp cụt đều đã cho là $S$, $4S$, $h$.  \\
		Thể tích của khối chóp cụt đều ban đầu là $V = \dfrac{h(S+4S+\sqrt{S \cdot 4S})}{3} = \dfrac{7hS}{3}$. \\
		Giả sử diện tích đáy nhỏ giảm $x$ ($x>1$) lần. \\
		Diện tích các đáy và chiều cao của khối chóp cụt mới là $\dfrac{S}{x}$, $6S$, $h$. \\
		Thể tích của khối chóp cụt đều mới là $$V' = \dfrac{h\left (\dfrac{S}{x}+6S+\sqrt{\dfrac{S}{x} \cdot 6S}\right )}{3} = \dfrac{\left (\dfrac{1}{x}+6+\sqrt{\dfrac{6}{x}}\right )hS}{3}.$$
		Thể tích của khối chóp cụt không đổi nên
		\begin{align*}
			V=V' &\Leftrightarrow \dfrac{7hS}{3} = \dfrac{\left (\dfrac{1}{x}+6+\sqrt{\dfrac{6}{x}}\right )hS}{3} \\
			&\Leftrightarrow \dfrac{1}{x}+6+\sqrt{\dfrac{6}{x}} =7  \\
			&\Leftrightarrow x= 4+\sqrt{15}.
		\end{align*}
		Vậy phải giảm diện tích đáy nhỏ $4+\sqrt{15}$ lần.
	}
\end{bt}

\begin{bt}%[DCHT Toán 11 - KNTT- Phạm Tuấn]%[1K7BQ-2] 
	\immini{
		Cho hình chóp cụt tứ giác đều $A B C D. A' B' C' D'$ cạnh bên bằng $8 a$, cạnh đáy lớn bằng $6 a$, cạnh đáy nhỏ bằng $a$. Tính thể tích hình chóp cụt đều này.
	}
	{
		\begin{tikzpicture}[scale=1, font=\footnotesize, line join=round, line cap=round,>=stealth]
			\path
			(0,0) coordinate (A)
			(3,-0.55) coordinate (B)
			(6.88,0) coordinate (C)
			($(A)+(C)-(B)$) coordinate (D)
			(2,3) coordinate (A')
			(5,3) coordinate (C')
			(intersection of {$(A')+(B)-(A)$--A'} and {$(C')+(B)-(C)$--C'}) coordinate (B')
			($(A')+(C')-(B')$) coordinate (D') ;
			
			%(barycentric cs:A=1,B=1,C=1) coordinate (H)
			%(barycentric cs:A'=1,B'=1,C'=1) coordinate (H')
			%($(A')+(H)-(H')$) coordinate (M) ;
			\draw (A')--(A)--(B)--(B')--(A')--(D')--(C')--(B') (B)--(C)--(C');
			\draw [dashed] (A)--(D)--(C) (D)--(D') ;
			%\draw pic[draw=black,angle radius=0.2cm]{right angle=H--M--A'};
			\foreach \x/\g in{A/190,B/-90,C/-20,D/60,A'/180,B'/-140,C'/40,D'/90}
			\fill[black](\x)circle(1.1pt) ($(\x)+(\g:3.5mm)$) node{$\x$}; 
		\end{tikzpicture}
	}
	\loigiai{
		\immini{
			Gọi $H, H'$ tương ứng là tâm của các hình vuông $A BCD$, $A' B'C'D'$.
			Khi đó, $H H'$ vuông góc với hai đáy của hình chóp cụt. \\
			Trong hình vuông $ABCD$ cạnh $6a$, ta có $AH=3a\sqrt{2}$. \\
			Trong hình vuông $A'B'C'D'$ cạnh $a$, ta có $A'H'=\dfrac{a\sqrt{2}}{2}$. \\
			Hình thang $AHH' A'$ vuông tại $H$ và $H'$. Kẻ $A' M \perp H A$ $(M \in HA)$. Dễ thấy $A'MHH'$ là hình chữ nhật.  Suy ra $HH'= A'M$, $MH= A'H'$ và $AM= AH-MH = \dfrac{5a\sqrt{2}}{2}$. 
		}
		{
			\begin{tikzpicture}[scale=1, font=\footnotesize, line join=round, line cap=round,>=stealth]
				\path
				(0,0) coordinate (A)
				(3,-0.55) coordinate (B)
				(6.88,0) coordinate (C)
				($(A)+(C)-(B)$) coordinate (D)
				(2,3) coordinate (A')
				(5,3) coordinate (C')
				(intersection of {$(A')+(B)-(A)$--A'} and {$(C')+(B)-(C)$--C'}) coordinate (B')
				($(A')+(C')-(B')$) coordinate (D') 
				($(A)!0.5!(C)$) coordinate (H)
				($(A')!0.5!(C')$) coordinate (H') 
				(intersection of {$(A')+(H)-(H')$--A'} and A--C) coordinate (M) ;
				%(barycentric cs:A=1,B=1,C=1) coordinate (H)
				%(barycentric cs:A'=1,B'=1,C'=1) coordinate (H')
				%($(A')+(H)-(H')$) coordinate (M) ;
				\draw (A')--(A)--(B)--(B')--(A')--(D')--(C')--(B') (B)--(C)--(C') (A')--(H');
				\draw [dashed] (A)--(D)--(C) (D)--(D') (A)--(H)--(H') (A')--(M);
				%\draw pic[draw=black,angle radius=0.2cm]{right angle=H--M--A'};
				\foreach \x/\g in{A/190,B/-90,C/-20,D/60,A'/180,B'/-140,C'/40,D'/90,H/0,H'/0,M/-30}
				\fill[black](\x)circle(1.1pt) ($(\x)+(\g:3.5mm)$) node{$\x$}; 
			\end{tikzpicture}
		}
		\noindent 
		Trong tam giác vuông $AA'M$ ta có 
		\[
		A'M = \sqrt{AA'^2-AM^2} = \sqrt{64a^2 -  \dfrac{25a^2}{2}} = \dfrac{a\sqrt{206}}{2}.
		\]
		Diện tích hình vuôn $A'B'C'D'$ là $S_1 = a^2$. \\
		Diện tích hình vuông $ABCD$ là $S_2 = 36a^2$. \\
		Thể tích khối chóp cụt đều $ABCD.A'B'C'D'$ là 
		\[
		V= \dfrac{h(S_1+S_2+\sqrt{S_1S_2})}{3} = \dfrac{\dfrac{a\sqrt{206}}{2} \left ( a^2 + 36a^2 + \sqrt{a^2 \cdot 36a^2} \right ) }{3} = \dfrac{43a^3\sqrt{206}}{6}.
		\]
	}
\end{bt}

\begin{bt}%[DCHT Toán 11 - KNTT- Phạm Tuấn]%[1K7KQ-2] 
	Tính thể tích của khối chóp cụt tam giác đều có cạnh đáy nhỏ $a$, cạnh đáy lớn $3a$ và góc giữa mặt bên và mặt đáy bằng $60^\circ$.
	\loigiai{
		\immini{
			Theo tính chất của hình chóp cụt đều, hình chóp cụt đều $ABC.A'B'C'$ là một phần của hình chóp đều $S.ABC$. Gọi $H$, $H'$ là tâm của các tam giác đều $ABC$, $A'B'C'$, ta có $SH\perp (ABC)$ và $HH' \perp (ABC)$. Gọi $K$ là trung điểm của $ABC$, ta có
			\[
			\heva{&AB\perp SH\\&AB\perp SK} \Rightarrow AB\perp (SHK) \Rightarrow \widehat{SKH} =( (SAB),(ABC)) =60^\circ.
			\]
			Do đó $SH=HK \tan 60^\circ = \dfrac{a\sqrt{3}}{2} \cdot \sqrt{3} = \dfrac{a}{2}$. \\
			Ta có $AH= (SAH) \cap (ABC)$, $A'H'= (SAH) \cap (A'B'C')$ và mặt phẳng $(A'B'C') \parallel (ABC)$ suy ra $A'H' \parallel AH$. Khi đó
			\[
			\dfrac{A'B'}{AB} = \dfrac{SA'}{SA} = \dfrac{SH'}{SH} = \dfrac{1}{3} \Rightarrow HH' = \dfrac{2}{3} SH = \dfrac{a}{3}.
			\]
		}
		{
			\begin{tikzpicture}[scale=1, font=\footnotesize, line join=round, line cap=round,>=stealth]
				\path
				(0,2) coordinate (A)
				(1.8,0) coordinate (B)
				(5,2) coordinate (C)
				(1.2,5) coordinate (A')
				(3.3,5) coordinate (C')
				(intersection of {$(A')+(B)-(A)$--A'} and {$(C')+(B)-(C)$--C'}) coordinate (B')
				(intersection of A--A' and B--B') coordinate (S)
				(barycentric cs:A=1,B=1,C=1) coordinate (H)
				(barycentric cs:A'=1,B'=1,C'=1) coordinate (H')
				($(A)!0.5!(B)$) coordinate (K);
				\draw (S)--(A)--(B)--(C)--(S)--(B)  (A')--(B')--(C') (S)--(K);
				\draw [dashed] (A)--(C) (A')--(C')   (S)--(H)--(K) (A')--(H') (A)--(H);
				%\draw pic[draw=black,angle radius=0.2cm]{right angle=H--M--A'};
				\foreach \x/\g in{A/190,B/-90,C/0,A'/150,B'/-140,C'/10,H/-10,H'/10,S/90,K/-130}
				\fill[black](\x)circle(1.1pt) ($(\x)+(\g:3.5mm)$) node{$\x$}; 
			\end{tikzpicture}
		}
		\noindent 
		Thể tích khối chóp cụt đều $ABC.A'B'C'$ là
		\begin{align*}
			V&= \dfrac{HH' (S_{ABC} + S_{A'B'C'} + \sqrt{S_{ABC} \cdot S_{A'B'C'}})}{3} \\
			&=  \dfrac{\dfrac{a}{3} \left (\dfrac{a^2\sqrt{3}}{4} +\dfrac{9a^2\sqrt{3}}{4} +\sqrt{\dfrac{a^2\sqrt{3}}{4} \cdot \dfrac{9a^2\sqrt{3}}{4}}\right )}{3} \\
			&=\dfrac{13a^3\sqrt{3}}{36}.
		\end{align*}
	}
\end{bt}
\subsubsection{Bài tập trắc nghiệm}
%\noindent \textbf{Bài tập trắc nghiệm}
\setcounter{ex}{0}

\begin{ex}%[DCHT Toán 11 - KNTT- Phạm Tuấn]%[1K7BQ-2]
	Tính thể tích của khối chóp cụt đều có chiều cao $h= 10 ~\mathrm{cm}$, diện tích các đáy lần lượt là $16~\mathrm{cm^2}$ và $25~\mathrm{cm^2}$.
	\choice
	{$180 ~\mathrm{cm^3}$}
	{$610 ~\mathrm{cm^3}$}
	{$200 ~\mathrm{cm^3}$}
	{\True $\dfrac{610}{3} ~\mathrm{cm^3}$}
	\loigiai{
		Ta có $V= \dfrac{h(S_1+S_2+\sqrt{S_1S_2})}{3} = \dfrac{10(16+25+\sqrt{16 \cdot 25})}{3} = \dfrac{610}{3} ~\mathrm{cm^3}$.
	}
\end{ex}

\begin{ex}%[DCHT Toán 11 - KNTT- Phạm Tuấn]%[1K7BQ-2]
	Cho khối chóp cụt đều có diện tích đáy lớn gấp $4$ lần đáy nhỏ. Nếu giảm $9$ lần diện tích đáy lớn  nhưng giữ nguyên diện tích đáy nhỏ và chiều cao của một khối chóp cụt đều thì thể tích của khối chóp cụt đều giảm bao nhiêu lần?
	\choice
	{$\dfrac{9}{4}$}
	{$\dfrac{54}{19}$}
	{\True $\dfrac{63}{19}$}
	{$\dfrac{81}{16}$}
	\loigiai{
		Gọi diện tích đáy nhỏ, diện tích đáy lớn, chiều cao của khối chóp cụt đều đã cho là $S$, $4S$, $h$.  \\
		Thể tích của khối chóp cụt đều ban đầu là $V = \dfrac{h(S+4S+\sqrt{S \cdot 4S})}{3} = \dfrac{7hS}{3}$. \\
		Diện tích các đáy và chiều cao của khối chóp cụt mới là $S$, $\dfrac{4S}{9}$, $h$. \\
		Thể tích của khối chóp cụt đều mới là $V' = \dfrac{h(S+\dfrac{4S}{9}+\sqrt{S \cdot \dfrac{4S}{9}})}{3} = \dfrac{19hS}{27}$. \\
		Ta có $$\dfrac{V}{V'} =  \dfrac{\dfrac{7hS}{3}}{\dfrac{19hS}{27}} = \dfrac{63}{19}.$$
		Vậy thể tích khối chóp cụt giảm $\dfrac{63}{19}$ lần.
	}
\end{ex}

\begin{ex}%[DCHT Toán 11 - KNTT- Phạm Tuấn]%[1K7BQ-2]
	Thể tích của khối chóp cụt tam giác đều có cạnh đáy lớn bằng $3a$, cạnh đáy nhỏ bằng $a$ và chiều cao bằng $a\sqrt{3}$ là
	\choice
	{$\dfrac{13a^3}{2}$}
	{\True $\dfrac{13a^3}{3}$}
	{$4a^3$}
	{$\dfrac{14a^3}{3}$}
	\loigiai{
		Diện tích đáy nhỏ là $S_1 = \dfrac{a^2\sqrt{3}}{4}$. \\
		Diện tích đáy lớn là $S21 = \dfrac{(3a)^2\sqrt{3}}{4} = \dfrac{9a^2\sqrt{3}}{4}$. \\
		Thể tích khối chóp cụt tam giác đều là 
		\[
		V= \dfrac{h(S_1+S_2+\sqrt{S_1S_2})}{3} = \dfrac{a\sqrt{3}\left (\dfrac{a^2\sqrt{3}}{4}+ \dfrac{9a^2\sqrt{3}}{4} + \sqrt{\dfrac{a^2\sqrt{3}}{4} \cdot \dfrac{9a^2\sqrt{3}}{4}}\right )}{3} =\dfrac{13a^3}{3}.
		\]
	}
\end{ex}

\begin{ex}%[DCHT Toán 11 - KNTT- Phạm Tuấn]%[1K7BQ-2]
	Thể tích của khối chóp cụt tứ giác đều có cạnh đáy lớn bằng $12~\mathrm{cm}$, cạnh đáy nhỏ bằng $9~\mathrm{cm}$ và chiều cao bằng $6~\mathrm{cm}$ là
	\choice
	{$546 ~\mathrm{cm^3}$}
	{$688 ~\mathrm{cm^3}$}
	{\True $666 ~\mathrm{cm^3}$}
	{$576 ~\mathrm{cm^3}$}
	\loigiai{
		Diện tích đáy nhỏ là  $S_1= 9^2=81 ~\mathrm{cm^2}$. \\
		Diện tích đáy lớn là  $S_1= 12^2=144 ~\mathrm{cm^2}$. \\
		Thể tích khối chóp cụt tứ giác đều  là
		\[
		V= \dfrac{h(S_1+S_2+\sqrt{S_1S_2})}{3} = \dfrac{6(81+144+\sqrt{81 \cdot 144})}{3} = 666 ~\mathrm{cm^3}.
		\]
	}
\end{ex}

\begin{ex}%[DCHT Toán 11 - KNTT- Phạm Tuấn]%[1K7BQ-2]
	Thể tích của khối chóp cụt tam giác đều có cạnh đáy lớn bằng $2a$, cạnh đáy nhỏ bằng $a$ và chiều cao bằng $\dfrac{2a\sqrt{6}}{3}$ là
	\choice
	{$\dfrac{7\sqrt{2}}{8} a^3$}
	{$\dfrac{\sqrt{2}}{4} a^3$}
	{\True $\dfrac{7\sqrt{2}}{6} a^3$}
	{$\dfrac{7\sqrt{3}}{4} a^3$}
	\loigiai{
		Diện tích đáy lớn  là $S_2 = \dfrac{4a^2\sqrt{3}}{4} = a^2\sqrt{3}$. \\
		Diện tích đáy nhỏ  là  $S_1= \dfrac{a^2\sqrt{3}}{4}$. \\
		Vậy thể tích của khối chóp cụt  là
		
		\begin{eqnarray*}
			V &=& \dfrac{h(S_1+S_2+\sqrt{S_1S_2})}{3}\\
			&=&\dfrac{1}{3} \cdot \dfrac{2a\sqrt{6}}{3}\left(a^2\sqrt{3} + \sqrt{a^2\sqrt{3} \cdot \dfrac{a^2\sqrt{3}}{4}} + \dfrac{a^2\sqrt{3}}{4}\right)\\
			&=& \dfrac{7\sqrt{2}}{6} a^3.
		\end{eqnarray*}
	}
\end{ex}

\begin{ex}%[DCHT Toán 11 - KNTT- Phạm Tuấn]%[1K7BQ-2]
	Một  khối chóp cụt đều có diện tích các đáy lần lượt là $4~\mathrm{m^2}$ và $9~\mathrm{m^2}$. Biết thể tích khối chóp cụt đều đã cho bằng $76 ~\mathrm{m^3}$. Tính chiều cao $h$ của khối chóp cụt. 
	\choice
	{$h=8 ~\mathrm{m}$}
	{$h=10 ~\mathrm{m}$}
	{\True $h=12 ~\mathrm{m}$}
	{$h=15 ~\mathrm{m}$}
	\loigiai{
		Ta có $V = \dfrac{h(S_1+S_2+\sqrt{S_1S_2})}{3} \Rightarrow h= \dfrac{3V}{S_1+S_2+\sqrt{S_1S_2}}= \dfrac{3 \cdot 76}{4+9+6}= 12 ~\mathrm{m}$. 
	}
\end{ex}

\begin{ex}%[DCHT Toán 11 - KNTT- Phạm Tuấn]%[1K7KQ-2]
	Tính thể tích khối chóp cụt đều  $ABC.A'B'C'$ biết  $AB=18~\mathrm{cm}$, $A'B' =12~\mathrm{cm}$ và $AA' =6~\mathrm{cm}$. 
	\choice
	{\True $342 \sqrt{2}~\mathrm{cm^3}$}
	{$420 \sqrt{2}~\mathrm{cm^3}$}
	{$545 \sqrt{2}~\mathrm{cm^3}$}
	{$288 \sqrt{2}~\mathrm{cm^3}$}
	\loigiai{
		\immini{
			Gọi $H$, $H'$ tương ứng là tâm của các tam giác $A B C$, $A' B' C'$.
			Khi đó, $H H'$ vuông góc với hai đáy của hình chóp cụt. \\
			Trong tam giác đều $ABC$ cạnh $18~\mathrm{cm}$, ta có $AH=\dfrac{18\sqrt{3}}{3} = 6\sqrt{3}~\mathrm{cm}$. \\
			Trong tam giác đều $A' B' C'$ cạnh $a$, ta có $A'H'=\dfrac{12\sqrt{3}}{3} =4\sqrt{3}~\mathrm{cm} $. \\
			Hình thang $AHH' A'$ vuông tại $H$ và $H'$. Kẻ $A' M \perp H A$ $(M \in HA)$. Dễ thấy $A'MHH'$ là hình chữ nhật.  Suy ra $HH'= A'M$, $MH= A'H'$ và $AM= AH-MH = 2\sqrt{3}~\mathrm{cm}$. 
		}
		{
			\begin{tikzpicture}[scale=1, font=\footnotesize, line join=round, line cap=round,>=stealth]
				\path
				(0,2) coordinate (A)
				(1.8,0) coordinate (B)
				(5,2) coordinate (C)
				(1,5) coordinate (A')
				(3.66,5) coordinate (C')
				(intersection of {$(A')+(B)-(A)$--A'} and {$(C')+(B)-(C)$--C'}) coordinate (B')
				(barycentric cs:A=1,B=1,C=1) coordinate (H)
				(barycentric cs:A'=1,B'=1,C'=1) coordinate (H')
				($(A')+(H)-(H')$) coordinate (M) ;
				\draw (A')--(A)--(B)--(B')--(A')--(C')--(B') (B)--(C)--(C') (A')--(H');
				\draw [dashed] (A)--(C) (H')--(H)--(A)  (A')--(M) ;
				%\draw pic[draw=black,angle radius=0.2cm]{right angle=H--M--A'};
				\foreach \x/\g in{A/190,B/-90,C/0,A'/150,B'/-140,C'/40,H/-10,H'/10,M/-90}
				\fill[black](\x)circle(1.1pt) ($(\x)+(\g:3.5mm)$) node{$\x$}; 
			\end{tikzpicture}
		}
		\noindent 
		Trong tam giác vuông $AA'M$ ta có 
		\[
		A'M = \sqrt{AA'^2-AM^2} = \sqrt{36 -  12} = 2\sqrt{6} ~\mathrm{cm} .
		\]
		Diện tích tam giác $A'B'C'$ là $S_1 = 36\sqrt{3}~\mathrm{cm^2} $. \\
		Diện tích tam giác $ABC$ là $S_2 = 81\sqrt{3}~\mathrm{cm^2} $. \\
		Thể tích khối chóp cụt đều $ABC.A'B'C'$ là 
		\[
		V= \dfrac{h(S_1+S_2+\sqrt{S_1S_2})}{3} = 
		\dfrac{ 2\sqrt{6} \left ( 36\sqrt{3} + 81\sqrt{3} + \sqrt{36\sqrt{3} \cdot 81\sqrt{3}}\right )}{3} = 342 \sqrt{2}~\mathrm{cm^3}.
		\]
	}
\end{ex}

\begin{ex}%[DCHT Toán 11 - KNTT- Phạm Tuấn]%[1K7KQ-2] 
	Tính thể tích của khối chóp cụt tam giác đều có cạnh đáy nhỏ $4~\mathrm{cm}$, cạnh đáy lớn $8~\mathrm{cm}$ và góc giữa mặt bên và mặt đáy bằng $60^\circ$.
	\choice
	{$\dfrac{28\sqrt{3}}{3}~ \mathrm{cm^2}$}
	{\True $\dfrac{56\sqrt{3}}{3}~ \mathrm{cm^2}$}
	{$\dfrac{32\sqrt{3}}{3}~ \mathrm{cm^2}$}
	{$\dfrac{64\sqrt{3}}{3}~ \mathrm{cm^2}$}
	\loigiai{
		\immini{
			Theo tính chất của hình chóp cụt đều, hình chóp cụt đều $ABC.A'B'C'$ là một phần của hình chóp đều $S.ABC$. Gọi $H$, $H'$ là tâm của các tam giác đều $ABC$, $A'B'C'$, ta có $SH\perp (ABC)$ và $HH' \perp (ABC)$. Gọi $K$ là trung điểm của $ABC$, ta có
			\[
			\heva{&AB\perp SH\\&AB\perp SK} \Rightarrow AB\perp (SHK) \Rightarrow \widehat{SKH} =( (SAB),(ABC)) =60^\circ.
			\]
			Do đó $SH=HK \tan 60^\circ = \dfrac{8\sqrt{3}}{6} \cdot \sqrt{3} = 4$. \\
			Ta có $AH= (SAH) \cap (ABC)$, $A'H'= (SAH) \cap (A'B'C')$ và mặt phẳng $(A'B'C') \parallel (ABC)$ suy ra $A'H' \parallel AH$. Khi đó
			\[
			\dfrac{A'B'}{AB} = \dfrac{SA'}{SA} = \dfrac{SH'}{SH} = \dfrac{1}{2} \Rightarrow HH' = \dfrac{1}{2} SH = 2.
			\]
		}
		{
			\begin{tikzpicture}[scale=1, font=\footnotesize, line join=round, line cap=round,>=stealth]
				\path
				(0,2) coordinate (A)
				(1.8,0) coordinate (B)
				(5,2) coordinate (C)
				(1.2,5) coordinate (A')
				(3.3,5) coordinate (C')
				(intersection of {$(A')+(B)-(A)$--A'} and {$(C')+(B)-(C)$--C'}) coordinate (B')
				(intersection of A--A' and B--B') coordinate (S)
				(barycentric cs:A=1,B=1,C=1) coordinate (H)
				(barycentric cs:A'=1,B'=1,C'=1) coordinate (H')
				($(A)!0.5!(B)$) coordinate (K);
				\draw (S)--(A)--(B)--(C)--(S)--(B)  (A')--(B')--(C') (S)--(K);
				\draw [dashed] (A)--(C) (A')--(C')   (S)--(H)--(K) (A')--(H') (A)--(H);
				%\draw pic[draw=black,angle radius=0.2cm]{right angle=H--M--A'};
				\foreach \x/\g in{A/190,B/-90,C/0,A'/150,B'/-140,C'/10,H/-10,H'/10,S/90,K/-130}
				\fill[black](\x)circle(1.1pt) ($(\x)+(\g:3.5mm)$) node{$\x$}; 
			\end{tikzpicture}
		}
		\noindent 
		Thể tích khối chóp cụt đều $ABC.A'B'C'$ là
		\begin{align*}
			V&= \dfrac{HH' (S_{ABC} + S_{A'B'C'} + \sqrt{S_{ABC} \cdot S_{A'B'C'}})}{3} \\
			&=  \dfrac{2 \left (4\sqrt{3}+16\sqrt{3} +\sqrt{4\sqrt{3}\cdot 16\sqrt{3}}\right )}{3} \\
			&=\dfrac{56\sqrt{3}}{3}~ \mathrm{cm^2}.
		\end{align*}
	}
\end{ex}
\Closesolutionfile{ans}
\begin{indapan}{10}
	{ans/ans-1K7-27-Dang3}
\end{indapan}



