\begin{dang}{Bài toán thực tế về khoảng cách}
\end{dang}
\begin{bt}%[1K7BP-3]
	\immini
	{
		Hình bên gợi nên hình ảnh hai mặt phẳng $(P)$ và $(Q)$ song song với nhau. Cột gỗ cao $4{,}2 \mathrm{~m}$. Khoảng cách giữa $(P)$ và $(Q)$ là bao nhiêu mét?
	}
	{
		\includegraphics[scale=1]{HINHVE/H1}
	}
	\loigiai{
		Vì cột gỗ vuông góc với cả trần nhà $(P)$ và sàn nhà $(Q)$ nên khoảng cách giữa $(P)$ và $(Q)$ chính là chiều cao của cột gỗ.\\
		Vậy khoảng cách giữa $(P)$ và $(Q)$ là $4{,}2 \mathrm{~m}$.
	}
\end{bt}

\begin{bt}[1K7BP-3]
	\immini{
		Một cây cầu dành cho người đi bộ (hình bên) có mặt sàn cầu cách mặt đường $3{,}5$m, khoảng cách từ đường thẳng $a$ nằm trên tay vịn của cầu đến mặt sàn cầu là $0{,}8$m. Gọi $b$ là đường thẳng kẻ theo tim đường. Tính khoảng cách giữa hai đường thẳng $a$ và $b$.
	}{
		\includegraphics[scale=.5]{HINHVE/H2}
	}
	\loigiai{
		Ta coi mặt đường là một mặt phẳng $\left(P\right)$ chứa đường thẳng $b$ và song song với đường thẳng $a$.\\
		Suy ra $\mathrm{d}\left(a,b\right) = \mathrm{d}\left(a,\left(P\right)\right) = 0{,}8 + 3{,}5 = 4{,}3$(m).
	}
\end{bt}