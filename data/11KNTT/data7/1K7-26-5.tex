\begin{dang}{Khoảng cách giữa hai đường thẳng chéo nhau}
	\begin{enumerate}
		\item \textbf{Định nghĩa}\\		
		Cho hai đường thẳng chéo nhau $a$ và $b$. Độ dài đoạn vuông góc chung $MN$ của $a$ và $b$ được gọi là khoảng cách giữa hai đường thẳng $a, b$.
		\immini{Khi đó: $ \mathrm{d}(a; b)=MN$ với $ \heva{&MN\perp a\\&MN \perp b.}$
		}{
			\begin{tikzpicture}[line join=round,line cap=round,>=stealth,scale=.7]
				\path 
				(0,0) coordinate (A)
				(4.5,0) coordinate (B)
				(0,-2) coordinate (C)
				(4.5,-1) coordinate (D)
				(1,0) coordinate (M)
				($(C)!.4!(D)$) coordinate (N)	
				;		
				\draw (A)--(B)node[pos=.9,above]{$a$};
				\draw (C)--(D)node[pos=.9,below]{$b$};	
				\draw (M)node[above]{$M$}--(N)node[below]{$N$};	
				\pic[draw,angle radius=2mm,angle eccentricity=1.5] {right angle = A--M--N};
				\pic[draw,angle radius=2mm,angle eccentricity=1.5] {right angle = M--N--C};	
		\end{tikzpicture}} 
		\item \textbf{Phương pháp xác định khoảng cách giữa hai đường thẳng chéo nhau.}\\
		Sử dụng song song và đưa về bài toán tìm khoảng cách từ chân đường cao đến mặt bên.\\		
		\textbf{Cho hai đường thẳng chéo nhau $d$ và $\Delta$. Tính khoảng cách giữa $d$ và $\Delta$.}\\
		\immini
		{
			\textbf{Bước 1.} Tìm một mặt phẳng $(P)$ chứa đường thẳng $\Delta$ và song song với $d$.\\
			\textbf{Bước 2.} Chọn một điểm $A$ thích hợp trên $d$ khi đó $\mathrm{d}\left(d;\Delta\right)=\mathrm{d}\left(A,(P)\right)$.
		}
		{
			\begin{tikzpicture}[>=stealth,line join=round,line cap=round,font=\footnotesize,scale=1]	
				\coordinate (A) at (0,0);
				\coordinate (B) at (4,0);
				\coordinate (C) at (6,1.5);
				\coordinate (D) at (2,1.5);
				\coordinate (E) at (4.5,3);
				\coordinate[label=above right:$d$] (F) at (1.2,3);
				\coordinate (G) at (1.2,0.25);
				\coordinate [label=above:$\Delta$] (H) at (4.5,1);
				\coordinate[label=above right:$A$] (M) at ($(F)!0.2!(E)$);
				\coordinate (Q) at ($(F)!0.7!(E)$);
				\coordinate (K) at ($(G)!0.7!(H)$);
				\draw pic[angle radius=10mm,draw=black,fill=blue!30,"$P$",angle eccentricity=0.7] {angle = B--A--D};
				\foreach \diem in {M}	\fill (\diem)circle(1.5pt);
				\draw (A)--(B)--(C)--(D)--cycle (E)--(F) (G)--(H) (Q)--(K) (M)--(1.86,0.75);	
			\end{tikzpicture}	
		}
		{\nx 
			\ \\
			Tính khoảng cách giữa hai đường thẳng chéo nhau $d$ và $A$ thực chất là tính khoảng cách từ điểm $A$ đến mặt phẳng $(P)$.\\
			Do vậy, việc chọn điểm $A$ trên $d$ sao cho đưa điểm $A$ về chân đường cao thuận lợi.}\\
		% \textbf{\underline{Ví dụ.}}
		% Cho hình chóp $S.ABCD$ có $SA \perp (ABCD)$ và đáy là hình chữ nhật. Xác định và tính khoảng cách giữa cạnh bên $SB$ và cạnh thuộc mặt đáy $AC$.
		% \loigiai{
		% 	\begin{itemize}
		% 		\item[]
		% 		\immini{\item \textbf{Bước 1.} Xác định giao điểm của cạnh bên và mặt đáy.\\
		% 			Ta có $SB\cap (ABCD)=B$.
		% 			\item \textbf{Bước 2.} Từ giao điểm dựng $d$ song song với cạnh đáy.\\
		% 			Qua $B$, dựng $d \parallel AC$ và dựng $AK \perp d, (K \in d)$.\\
		% 			Khi đó $\mathrm{d}(SB,AC)=\mathrm{d}(AC,(SKB))=\mathrm{d}(A,(SKB))$.
		% 		}{
		% 			\begin{tikzpicture}[scale=.6, font=\footnotesize, line join=round, line cap=round, >=stealth]
		% 				\def\bc{4} % cạnh BC
		% 				\def\ba{2} % cạnh BA
		% 				\def\h{4} % đường cao
		% 				\def\gocB{30} % góc B của đáy
		% 				\coordinate[label=below left:$B$] (B) at (0,0);
		% 				\coordinate[label=above left:$A$] (A) at (\gocB:\ba);
		% 				\coordinate[label=below:$C$] (C) at (\bc,0);
		% 				\coordinate[label=right:$D$] (D) at ($(C)-(B)+(A)$);
		% 				\coordinate[label=above:$S$] (S) at ($(A)+(90:\h)$);
		% 				\coordinate (E) at ($(A)+(B)-(C)$);
		% 				\coordinate (F) at ($(C)+(B)-(A)$);
		% 				\coordinate[label=below left:$K$] (K) at ($(E)!.35!(F)$);
		% 				\coordinate[label=left:$H$] (H) at ($(S)!(A)!(K)$);
		% 				\draw (B)--(C)--(D)--(S)--cycle (S)--(C);
		% 				\draw[dashed] (A)--(D) (S)--(A)--(B) (A)--(C) (A)--(K) (A)--(H);
		% 				\draw (E)--(F)node[pos=.9,above right]{$d$} (S)--(K);
		% 				\foreach \diem in {A,B,C,D,S}	\fill (\diem)circle(1.5pt);
		% 				\pic[draw,angle radius=2mm,angle eccentricity=1.5] {right angle = S--A--D};
		% 				\pic[draw,angle radius=2mm,angle eccentricity=1.5] {right angle = S--H--A};
		% 				\pic[draw,angle radius=2mm,angle eccentricity=1.5] {right angle = A--K--B};	
		% 			\end{tikzpicture}
		% 		}
		% 		\item \textbf{Bước 3.} Tìm khoảng cách từ chân đến mặt bên.\\
		% 		Dựng $AH \perp SK, (K \in SK)$.\\
		% 		Ta có $\heva{&KB \perp AK\\&KB \perp SA}\Rightarrow KB \perp (SAK)\Rightarrow KB \perp AH$.\\
		% 		Khi đó, ta có $\heva{&AH \perp KB\\&AH \perp SB}\Rightarrow AH \perp (SKB)\Rightarrow \mathrm{d}(SB,AC)=\mathrm{d}(A,(SKB))=AH$.
		% 		\item \textbf{Bước 4.} Sử dụng kiến thức hình học phẳng để tính độ dài đoạn $AH$.
		% 	\end{itemize}
		% }
	\end{enumerate}
	% \begin{note}
	% 	Cần nhớ bài toán về tam diện vuông, nó thường được áp dụng\\
	% 	Cho tứ diện vuông $O.ABC$ với $OA, OB, OC$ đôi một vuông góc với nhau. Khi đó đường cao $OH$ của tứ diện $O.ABC$ được tính theo công thức \fbox{$\dfrac{1}{OH^2}=\dfrac{1}{OA^2}+\dfrac{1}{OB^2}+\dfrac{1}{OC^2}$}.\\
	% 	\textbf{Chứng minh}
	% 	\immini{
	% 		Dựng $OD \perp BC, (D \in BC)$ và $OH \perp AD, (H \in AD)$.\\
	% 		Có $\heva{&BC \perp OD\\&BC \perp AO}\Rightarrow BC \perp (AOD)\Rightarrow (ABC) \perp (AOD)$.\\
	% 		Khi đó $\heva{&(ABC) \perp (AOD)\\&(ABC)\cap(AOD)=AD\\&OH \perp AD, OH \subset (AOD)}\Rightarrow OH \perp (ABC)$.\\
	% 		Suy ra $\dfrac{1}{OH^2}=\dfrac{1}{OA^2}+\dfrac{1}{OD^2}=\dfrac{1}{OA^2}+\dfrac{1}{OB^2}+\dfrac{1}{OC^2}$ (đpcm).
	% 	}{
	% 		\begin{tikzpicture}[scale=.8, font=\footnotesize, line join=round, line cap=round, >=stealth]
	% 			\def\ac{4} % cạnh AC
	% 			\def\ab{2} % cạnh AB
	% 			\def\h{4} % chiều cao
	% 			\def\gocA{50} % góc A của đáy
	% 			\coordinate[label=left:$O$] (O) at (0,0);
	% 			\coordinate[label=right:$C$] (C) at (\ac,0);
	% 			\coordinate[label=below left:$B$] (B) at (-\gocA:\ab);
	% 			\coordinate[label=above:$A$] (A) at ($(O)+(90:\h)$);
	% 			\coordinate[label=below right:$D$] (D) at ($(B)!2/5!(C)$);
	% 			\coordinate[label=right:$H$] (H) at ($(A)!(O)!(D)$);
	% 			\draw (O)--(B)--(C)--(A)--cycle (A)--(B) (A)--(D);
	% 			\draw[dashed] (O)--(C) (O)--(D) (O)--(H);
	% 			\foreach \diem in {A,B,C,O}	\fill (\diem)circle(1.5pt);
	% 			\pic[draw,angle radius=2mm,angle eccentricity=1.5] {right angle = A--O--C};
	% 			\pic[draw,angle radius=2mm,angle eccentricity=1.5] {right angle = O--H--D};
	% 			\pic[draw,angle radius=2mm,angle eccentricity=1.5] {right angle = O--D--B};	
	% 		\end{tikzpicture}
	% 	}
	% \end{note}
\end{dang}
\subsubsection{Ví dụ mẫu}
\begin{vd}%[Tex hoá SGK KNTT]%[Phạm Văn Long]%[1K7BP-4]
	Cho hình chóp $S.ABC$ có $SA\perp (ABC)$, $AB=a$, $\widehat{ABC}=60^\circ$.	Xác định đường vuông góc chung và tính khoảng cách giữa hai đường thẳng $SA$ và $BC$.
	\loigiai{
		\immini{Gọi $H$ là hình chiếu của $A$ trên $BC$. Tam giác $ABH$ vuông tại $H$ và có $AB=a$, $\widehat{ABH}=60^\circ$ nên $BH=\dfrac{a}{2}$.\\
			Do $SA$ vuông góc với mặt phẳng $(ABC)$ nên $AH$ là đường vuông góc chung của $SA$ và $BC$ ($H$ thuộc tia $BC$ và $BH=\dfrac{a}{2}$).\\
			Khoảng cách giữa hai đường thẳng $SA$ và $BD$ là $\mathrm{d}(SA,BC)=AH=\dfrac{a\sqrt{3}}{2}$.}{\begin{tikzpicture}[scale=1,font=\footnotesize,line join = round, line cap = round, >= stealth]
				%\draw[opacity=0.3] (0,0) grid (6,6);
				\def\x{4} \def\y{2} \def\z{4}
				\def\g{-60}
				\coordinate (A) at (0,0);
				\coordinate (C) at ($(A)+(0:\x)$);
				\coordinate (B) at ($(A)+(\g:\y)$);
				\coordinate (S) at ($(A)+(90:\z)$);
				\coordinate (H) at ($(B)!.4!(C)$);
				\draw pic[draw,angle radius=2mm] {right angle = A--H--C};				
				\begin{scope}
					\clip (C)--(B)--(A);
					\draw (B) circle (0.3);
					\draw ($(B)+(60:0.6)$)node[rotate=0]{$60^\circ$};
				\end{scope}
				\draw (S)--(A)--(B)--(C)--(S)--(B);
				\draw[dashed] (C)--(A)--(H);
				\foreach \p/\g in {S/100,A/180,B/-90,H/-80,C/0} \draw[fill] (\p) circle(.5pt)
				node [shift={(\g:.3)}] {$\p$}
				;
		\end{tikzpicture}}
	}
\end{vd}
\begin{vd}%[Tex hoá SGK KNTT]%[Phạm Văn Long]%[1K7BP-4]
	Cho hình chóp tứ giác đều $S.ABCD$ có tất cả các cạnh đều bằng $a$ và có $O$ là giao điểm hai đường chéo của đáy. Tính khoảng cách giữa hai đường thẳng $AC$ và $SB$.
	\loigiai{
		\begin{center}
			\begin{tikzpicture}[=>stealth,line join=round,line cap=round, font=\footnotesize, scale=.8]
				\def\a{5}
				\def\goc{210}
				\def\b{3}
				\def\h{4.5}
				\path
				(0,0)coordinate (A)++(0:\a)coordinate (B)++(\goc:\b)coordinate (C)++(180:\a)coordinate(D)
				($(A)!.5!(C)$)coordinate (O)
				(O)++(90:\h)coordinate (S);
				\path 
				($(B)!(O)!(S)$)coordinate (H);
				\draw (S)--(B)--(C)--(D)--cycle (S)--(C); %(M)--(S)--(C);
				\draw[dashed] (O)--(S)--(A)--(C) (B)--(A)--(D)--(B) (O)--(H);
				\draw pic[draw,angle radius=2mm] {right angle = S--O--B};
				\draw pic[draw,angle radius=2mm] {right angle = O--H--B};
				\foreach \x/\g in{A/160,B/0,C/-90,S/90,O/-90,D/-90,H/60}
				\fill[black](\x)circle(1pt) ($(\x)+(\g:3mm)$)node{$\x$};
			\end{tikzpicture}
		\end{center}
		Kẻ $OH \perp SB$ tại $H$.\\
		Ta có $\heva{&AC\perp BD\\&AC \perp SO}\Rightarrow AC \perp (SBD)\Rightarrow AC \perp OH$. Khi đó $\mathrm{d}(AC,SB)=OH$.\\
		Có $\triangle SAC$ vuông tại $S$ suy ra $SO=\dfrac{1}{2}AC=\dfrac{a\sqrt{2}}{2}$.\\
		Khi đó $OH=\dfrac{SO\cdot OB}{SB}=\dfrac{\tfrac{a\sqrt{2}}{2}\cdot \tfrac{a\sqrt{2}}{2}}{a}=\dfrac{a}{2}$.\\
		Vậy $\mathrm{d}(AC,SB)=\dfrac{a}{2}$.				
	}
\end{vd}

\begin{vd}%[Tex hoá SGK KNTT]%[Phạm Văn Long]%[1K7BP-4]
	Cho hình chóp $S.ABCD$ có đáy là hình vuông $ABCD$ cạnh $a$, cạnh $SA=a$ và vuông góc với mặt phẳng $(ABCD)$. Tính khoảng cách giữa hai đường thẳng:
	\begin{enumerate}
		\item[a)] $SB$ và $CD$.
		\item[b)] $AB$ và $SC$.
	\end{enumerate}
	\loigiai{
		\begin{enumerate}
			\item[a)] Ta có $BC\perp SA$ và $BC\perp AB$, suy ra $BC\perp SB$.\\
			Mặt khác $BC\perp CD$, suy ra $BC$ là đoạn vuông góc chung của hai đường thẳng $SB$ và $CD$. Ta có $\mathrm{d}(SB,CD)=BC=a$.
			\item[b)] 
			\immini
			{\textbf{Cách 1.} Ta có $AB\perp(SAD)$ và $SD$ là hình chiếu vuông góc của $SC$ lên $(SAD)$.\\
				Vẽ $AK\perp SD$, $KE\parallel AB$, $EF\parallel AK$.\\
				Ta có $AB\perp AK$, $AK\perp SD$, suy ra $AK\perp SC$. Do $EF\parallel AK$, suy ra ta cũng có $EF\perp AB$ và $EF$ cắt $AB$ tại $F$, $EF\perp SC$ và $EF$ cắt $SC$ tại $E$.\\
				Các kết quả trên chứng tỏ $EF$ là đoạn vuông góc chung của $AB$ và $SC$.
			}
			{\begin{tikzpicture}[scale=0.5, font=\footnotesize, line join=round, line cap=round, >=stealth]
					\draw (4,4) node[left]{$A$};
					\draw (0,0) node[below]{$B$};
					\draw (8,0) node[below]{$C$};
					\draw (12,4) node[below]{$D$};
					\draw (2,2) node[left]{$F$};
					\draw (6,5) node[right]{$E$};
					\draw (8,7) node[right]{$K$};
					\draw (6,2) node[below]{$O$};
					\draw (4,10) node[left]{$S$};
					\draw (6.7,3.25) node[right]{$H$};
					\coordinate (A) at (4,4);
					\coordinate (B) at (0,0);
					\coordinate (C) at (8,0);
					\coordinate (D) at (12,4);
					\coordinate (F) at (2,2);
					\coordinate (E) at (6,5);
					\coordinate (K) at (8,7);
					\coordinate (O) at (6,2);
					\coordinate (S) at (4,10);
					\coordinate (H) at (6.7,3.25);
					\draw (S)--(B)--(C)--(D)--(S)--(C) (E)--(K);
					\draw[dashed] (S)--(A)--(B) (C)--(A)--(D) (A)--(K) (F)--(E) (O)--(H)(B)--(D);
					\draw pic[draw,angle radius=1.5mm] {right angle = S--K--A};
					\draw pic[draw,angle radius=1.5mm] {right angle = O--H--C};
					\draw pic[draw,angle radius=1.5mm] {right angle = C--E--F};					
				\end{tikzpicture}
			}
			Trong tam giác $SAD$ vuông cân tại $A$ có $AK=\dfrac{SD}{2}=\dfrac{a\sqrt{2}}{2}$.\\
			Vậy $\mathrm{d}(AB,SC)=EF=AK=\dfrac{a\sqrt{2}}{2}$.\\
			\textbf{Cách 2.} Ta có mặt phẳng $(SCD)$ chứa $SC$ và song song với $AB$, suy ra
			\[\mathrm{d}(AB,SC)=\mathrm{d}(AB,(SCD))=\mathrm{d}(A,(SCD))=\dfrac{a\sqrt{2}}{2}.\]
		\end{enumerate}
	}
\end{vd}

\begin{vd}%[Tex hoá SGK KNTT]%[Phạm Văn Long]%[1K7BP-4]
	Cho hình chóp $S.ABCD$ có đáy $ABCD$ là hình vuông cạnh bằng $a$, $SA\perp (ABCD)$, góc giữa đường thẳng $SC$ và mặt phẳng $(ABCD)$ bằng $45^\circ$. Tính khoảng cách giữa hai đường thẳng $SB, AC$.
	\loigiai{ 
		\begin{itemize}
			\item Xác định góc giữa $SC$ và $(ABCD)$:
			\immini{Ta có $SC\cap (ABCD)=C$ \tagEX{1}
				$SA \perp (ABCD)$ tại $A$ \tagEX{2}
				Từ $(1),(2)\Rightarrow AC$ là hình chiếu của $SC$ lên mặt phẳng $(ABCD)$.\\
				$\Rightarrow \left(SC,(ABCD)\right)=\widehat{(SC,AC)}=\widehat{SCA}=45^\circ$.\\
				Ta có $\triangle SAC$ vuông tại $A$ và $\widehat{SCA}=45^\circ$.\\
				$\Rightarrow\triangle SAC$ vuông cân tại $A$.\\
				$\Rightarrow AC=SA=AB\sqrt{2}=a\sqrt{2}$.
			}{
				\begin{tikzpicture}[scale=.9, font=\footnotesize, line join=round, line cap=round, >=stealth,transform shape]
					\def\bc{4} % cạnh BC
					\def\ba{2} % cạnh BA
					\def\h{4} % đường cao
					\def\gocB{30} % góc B của đáy
					\coordinate[label=below left:$B$] (B) at (0,0);
					\coordinate[label=above left:$A$] (A) at (\gocB:\ba);
					\coordinate[label=below:$C$] (C) at (\bc,0);
					\coordinate[label=right:$D$] (D) at ($(C)-(B)+(A)$);
					\coordinate[label=above:$S$] (S) at ($(A)+(90:\h)$);
					\coordinate (E) at ($(A)+(B)-(C)$);
					\coordinate (F) at ($(C)+(B)-(A)$);
					\coordinate[label=below left:$K$] (K) at ($(E)!.35!(F)$);
					\coordinate[label=left:$H$] (H) at ($(S)!(A)!(K)$);
					\coordinate[label=below:$O$] (O) at ($(B)!.5!(D)$);
					\draw (B)--(C)--(D)--(S)--cycle (S)--(C);
					\draw[dashed] (A)--(D) (S)--(A)--(B) (A)--(C) (A)--(K) (A)--(H) (B)--(D);
					\draw (E)--(F)node[pos=.9,above right]{$d$} (S)--(K);
					\foreach \diem in {A,B,C,D,S}	\fill (\diem)circle(1.5pt);
					\pic[draw,angle radius=2mm,angle eccentricity=1.5] {right angle = S--A--D};
					\pic[draw,angle radius=2mm,angle eccentricity=1.5] {right angle = S--H--A};
					\pic[draw,angle radius=2mm,angle eccentricity=1.5] {right angle = A--K--B};
					\pic[draw,angle radius=2mm,angle eccentricity=1.5] {right angle = A--B--C};
					\pic[draw,angle radius=2mm,angle eccentricity=1.5] {right angle = A--D--C};	
					\pic[draw,angle radius=2mm,angle eccentricity=1.5] {right angle = A--O--B};
					\pic[draw,blue,angle radius=3mm,angle eccentricity=1.5,"$45^\circ$"] {angle = S--C--A};
				\end{tikzpicture}
			} 
			\item Xác định khoảng cách giữa $SB$ và $AC$:\\
			Ta có $SB\cap (ABCD)=B$.\\
			Từ giao điểm $B$, dựng $d\parallel AC$.\\
			Từ chân đường vuông góc, dựng $AK\perp d, (K\in d, KB\subset(ABCD))$.\\
			Khi đó $\heva{&AC\parallel KB\\&KB \subset(SKB)}\Rightarrow AC\parallel (SKB)$.\\
			Suy ra $\mathrm{d}(AC,SB)=\mathrm{d}(AC,(SBK))=\mathrm{d}(A,(SBK))$.
			\item Xác định khoảng cách từ $A$ đến $(SBK)$:\\
			Từ chân đường vuông góc $A$, dựng $AH\perp SK$.\\
			Ta có $\heva{&AH\perp SK\\&AH\perp KB (\text{ do }SA\perp (ABCD)).}$\\
			$\Rightarrow AH \perp (SKB)\Rightarrow \mathrm{d}(A,(SBK))=AH$.\\
			Ta có $\dfrac{1}{AH^2}=\dfrac{1}{SA^2}+\dfrac{1}{AK^2}\quad (*)$ với $SA=a\sqrt{2}$.\\
			Vì $AKBO$ là HCN nên $AK=BO=\dfrac{a\sqrt{2}}{2}$.\\
			Từ $(*)\Rightarrow \mathrm{d}(AC,SB)=AH=\dfrac{a\sqrt{10}}{5}$.
		\end{itemize}
	}
\end{vd}
\begin{vd}%[Tex hoá SGK KNTT]%[Phạm Văn Long]%[1K7BP-4]
	Cho lăng trụ đứng $ABC.A'B'C'$ có đáy $ABC$ là tam giác vuông tại $A$, $BC=2a$, $AB=a$ và mặt bên $BB'C'C$ là hình vuông. Tính khoảng cách giữa hai đường thẳng $AA'$ và $BC'$ theo $a$.
	\loigiai{
		\immini{Tam giác $ABC$ vuông tại $A$ nên $AC=\sqrt{BC^2 - AB^2}=a\sqrt{3}$.\\
			$BB'C'C$ là hình vuông nên $BB' = BC = 2a$.\\
			Vì $\heva{&AA' \parallel BB' \\& BB' \subset (BB'C'C)} \Rightarrow AA' \parallel (BB'C'C)$.\\
			Do đó $\mathrm{d}(AA',BC)=\mathrm{d}(AA',(BB'C'C)) = \mathrm{d}(A,(BB'C'C))$.\\
			Dựng $AH \perp BC$ ($H$ thuộc $BC$).\\
			Khi đó ta có
			$\heva{&AH \perp BC .\\ & AH \perp BB'} \Rightarrow AH \perp (BB'C'C)$.\\
			$\Rightarrow \mathrm{d}(A,(BB'C'C))=AH$.\\
			Xét tam giác vuông $ABC$, ta có 
			\[AH \cdot BC = AB \cdot AC \Rightarrow AH=\dfrac{AB \cdot AC}{BC}=\dfrac{a\sqrt{3}}{2}\]
			Vậy $\mathrm{d}(AA',BC')=\dfrac{a\sqrt{3}}{2}$.
		}
		{\begin{tikzpicture}[scale=0.8, font=\footnotesize, line join=round, line cap=round, >=stealth]
				\def\ac{4} % cạnh AC
				\def\ab{2.5} % cạnh AB
				\def\h{4} % chiều cao
				\def\gocA{50} % góc A của đáy
				\coordinate[label=left:$A$] (A) at (0,0);
				\coordinate[label=right:$C$] (C) at (\ac,0);
				\coordinate[label=below left:$B$] (B) at (-\gocA:\ab);
				\coordinate[label=left:$A'$] (A') at ($(A)+(90:\h)$);
				\coordinate[label=below left:$B'$] (B') at ($(B)-(A)+(A')$);
				\coordinate[label=right:$C'$] (C') at ($(C)-(A)+(A')$);
				\coordinate[label=below right:$H$] (H) at ($(B)!.3!(C)$);
				\draw (A')--(A)--(B)--(C)--(C')--(A')--(B')--(C') (B)--(B') (B)--(C');
				\draw[dashed] (H)--(A)--(C) ;
				\foreach \diem in {A,B,C,A',B',C',H} \fill (\diem)circle(1.5pt);
				\draw pic[draw,angle radius=2mm] {right angle = A--H--C};
		\end{tikzpicture}}	
	}
\end{vd}
\subsubsection{Bài tập rèn luyện}
\begin{bt}%[Tex hoá SGK KNTT]%[Phạm Văn Long]%[1K7KP-4]
	Cho hình chóp $S.ABCD$ có đáy là hình vuông cạnh $a$, $SA\perp (ABCD)$, $SA=a\sqrt{2}$. Xác định đường vuông góc chung và tính khoảng cách giữa $BD$ và $SC$.
	\loigiai{
		\begin{center}
			\begin{tikzpicture}[scale=1,font=\footnotesize,line join = round, line cap = round, >= stealth]
				%\draw[opacity=0.3] (0,0) grid (6,6);
				\def\x{4} \def\y{2} \def\z{3}
				\def\g{-120}
				\coordinate (A) at (0,0);
				\coordinate (D) at ($(A)+(0:\x)$);
				\coordinate (B) at ($(A)+(\g:\y)$);
				\coordinate (C) at ($(B)+(D)-(A)$);
				\coordinate (S) at ($(A)+(90:\z)$);
				\coordinate (O) at ($(B)!1/2!(D)$);
				\coordinate (H) at ($(S)!1/2!(C)$);
				\coordinate (K) at ($(H)!0.4!(C)$);
				\draw (S)--(B)--(C)--(D)--(S)--(C);
				\draw[dashed] (S)--(A)--(C) (B)--(D)
				(B)--(A)--(D)  (H) (O)--(K)
				;
				\draw pic[draw,angle radius=1mm] {right angle = O--K--C};
				\foreach \p/\g in {S/100,A/180,B/-90,C/-70,D/0,O/-90,K/20} \draw[fill] (\p) circle(.5pt)
				node [shift={(\g:.3)}] {$\p$}
				;
			\end{tikzpicture}
		\end{center}
		Gọi $O=AC\cap BD$. Kẻ $OK\perp SC$ tại $K$.\\
		Do $BD\perp (SAC)$ nên $OK\perp BD$.\\
		Vậy $OK$ là đường vuông góc chung giữa $BD$ và $SC$.\\
		Xét $\triangle SAC\backsim \triangle OKC\Rightarrow \dfrac{OK}{SA}=\dfrac{OC}{SC}$  $\Rightarrow OK=\dfrac{OC\cdot SA}{SC}=\dfrac{a\sqrt{2}\cdot a\sqrt{2}}{2\cdot 2a}=\dfrac{a}{2}$.
	}
\end{bt}
\begin{bt}%[Tex hoá SGK KNTT]%[Phạm Văn Long]%[1K7KP-4]
	Cho hình chóp $S.ABCD$ có đáy $ABCD$ là hình vuông cạnh bằng $a$. Biết $\triangle SAD$ vuông cân tại $S$ và $(SAD)\perp(ABCD)$. Tính theo $a$ khoảng cách giữa hai đường thẳng $AD$ và $SC$.	
	\loigiai{
		\immini{
			Ta có $\heva{&AD\parallel BC \\ &BC\subset (SBC)} \Rightarrow AD\parallel (SBC)$.\\
			$\Rightarrow \mathrm{d}(AD,SC) = \mathrm{d}[AD,(SBC)] = \mathrm{d}[H,(SBC)]$.\\
			Kẻ $HI\perp BC$ tại $I$, $HK\perp SI$ tại $K$. Khi đó $HI=AB=a$ (vì $DHIC$ là hình chữ nhật) và 
			$$\dfrac{1}{HK^2} = \dfrac{1}{HI^2} + \dfrac{1}{SH^2} = \dfrac{1}{a^2} + \dfrac{4}{a^2} = \dfrac{5}{a^2} \Rightarrow HK=\dfrac{a\sqrt{5}}{5}.$$
			Ta có $\heva{&BC\perp HI\\ &BC\perp SH \text{ (vì } SH\perp (ABCD)).}$\\
			$\Rightarrow BC\perp (SHI)$.\\
			Mà $HK\subset (SHI)$.\\
			$\Rightarrow HK\perp BC$.\\
			Mặt khác $HK\perp SI$.\\
			Suy ra $HK\perp (SBC)\Rightarrow \mathrm{d}[H,(SBC)] = HK = \dfrac{a\sqrt{5}}{5}$.\\
			Vậy $\mathrm{d}(AD,SC)=\dfrac{a\sqrt{5}}{5}$.
		}
		{
			\begin{tikzpicture}[line join=round, line cap=round, font=\footnotesize, scale=0.8]
				\tikzset{label style/.style={font=\footnotesize}}
				\def\d{5} \def\r{1.8} \def\h{4.5} \def\l{2.2}
				\coordinate[label={below left}:$D$] (D) at (-3,-3);
				\coordinate[label={below right}:$C$] (C) at ($(D)+(\d,0)$);
				\coordinate[label={below right}:$A$] (A) at ($(D)+(\l,\r)$);
				\coordinate[label={above right}:$B$] (B) at ($(A)+(\d,0)$);
				\coordinate[label={below right}:$H$] (H) at ($(A)!0.5!(D)$);
				\coordinate[label={above}:$S$] (S) at ($(H)+(0,\h)$);
				\coordinate[label={below right}:{$I$}] (I) at ($(B)!0.5!(C)$);
				\coordinate[label={right}:{$K$}] (K) at ($(S)!0.4!(I)$);
				\draw (S)--(D)--(C)--(B)--(S)--(C) (S)--(I);
				\draw[dashed] (S)--(A)--(D) (A)--(B) (S)--(H)--(I) (H)--(K);
				\draw
				pic[draw, angle radius=3mm]{right angle=S--H--A}
				pic[draw, angle radius=3mm]{right angle=A--S--D}
				pic[draw, angle radius=3mm]{right angle=H--K--I}
				;
				\foreach \x in {A,B,C,D,S,H,I,K}
				\draw[fill=black] (\x) circle (1pt);
			\end{tikzpicture}
		}
	}
\end{bt}
\begin{bt}%[Tex hoá SGK KNTT]%[Phạm Văn Long]%[1K7KP-4]
	Cho hình chóp $S . A B C D$ có đáy là hình vuông cạnh $a$, $S A=S B=S C=S D=a \sqrt{2}$. Gọi $I$, $J$ lần lượt là trung điểm của $A B$ và $C D$. Tính khoảng cách giữa hai đường thẳng $A B$ và $S C$.
	\loigiai{
		\immini{Vì $AB\parallel CD$ nên $AB\parallel\left(SCD\right)$, suy ra
			\[\mathrm{d}\left(AB,SC\right) = \mathrm{d}\left(A,\left(SCD\right)\right) = 2\mathrm{d}\left(O,\left(SCD\right)\right).\]	
			Kẻ $OH\perp SJ$. Vì $CD\parallel AB$ mà $AB\perp \left(SIJ\right)$ nên ta cũng có $CD\perp \left(SIJ\right)$.\\
			Suy ra $CD\perp OH$. Kết hợp với $SJ\perp OH$ ta được $OH\perp \left(SCD\right)$ hay $OH$ chính là khoảng cách từ $O$ đến $\left(SCD\right)$.\\
			Ta có $\triangle SAC$ đều cạnh $a\sqrt{2}$ nên $SO = \dfrac{a\sqrt{2}\cdot\sqrt{3}}{2} = \dfrac{a\sqrt{6}}{2}$. \\
			Suy ra
			\[OH = \dfrac{SO\cdot OJ}{\sqrt{SO^2 + OJ^2}} = \dfrac{\tfrac{a\sqrt{6}}{2}\cdot\tfrac{a}{2}}{ \sqrt{\left(\tfrac{a\sqrt{6}}{2}\right)^2 + \left(\tfrac{a}{2}\right)^2}} = \dfrac{a\sqrt{42}}{14}.\]
			Vậy khoảng cách giữa hai đường thẳng $A B$ và $S C$ bằng $\dfrac{a\sqrt{42}}{7}$.
		}
		{
			\begin{tikzpicture}[smooth,line join=round,line cap=round,font=\scriptsize,scale=0.8]
				\path 
				(0,0) coordinate (A)
				($(A)+(-135:3)$) coordinate (D)
				($(A)+(0:5)$) coordinate (B)
				($(B)+(D)-(A)$) coordinate (C)
				($(A)!0.5!(C)$) coordinate (O)
				($(O)+(90:5)$) coordinate (S);
				\coordinate (J) at ($(C)!0.5!(D)$);
				\coordinate (I) at ($(A)!0.5!(B)$);
				\coordinate (H) at ($(J)!0.2!(S)$);
				\draw [dashed] (J)--(I)--(S)--(A)--(D)--(B)--(A)--(C) (S)--(O)--(H);
				\draw (J)--(S)--(D)--(C)--(S)--(B)--(C);
				\foreach \x/\g in {A/180,D/180,C/0,B/0,O/-90,S/90, I/45, J/-90, H/180}
				\fill[black] (\x) circle (1pt) + (\g:3mm) node {$\x$};
				\pic[draw,angle radius=5]{right angle=O--H--J};
			\end{tikzpicture}
		}
	}
\end{bt}
\begin{bt}%[Tex hoá SGK KNTT]%[Phạm Văn Long]%[1K7KP-4]
	Cho hình chóp $S.ABCD$ có đáy là hình vuông cạnh $a$, $SA \perp (ABCD)$ và $SA = a$. Gọi $M$, $N$, $P$ lần lượt là trung điểm của $SB$, $SC$ và $SD$. Tính khoảng cách giữa $AM$ và $NP$.
	\loigiai{
		\immini{Gọi $Q$ là trung điểm điểm $SA$. Suy ra $PQ \parallel AD$.\\
			Do $AD \perp (SAB) \Rightarrow PQ \perp (SAB)$.\\
			Ta có $\heva{&NP \parallel CD\\&AB \parallel CD} \Rightarrow NP \parallel AB$.\\
			Suy ra $NP \parallel (SAB)$.\\
			Do $AM \subset (SAB)$ nên \[\mathrm{\,d} (AM,NP) = \mathrm{\,d} (NP,(SAB)) = \mathrm{\,d} (P,(SAB)) = PQ = \dfrac{AD}{2} = \dfrac{a}{2}.\]
			
		}{\begin{tikzpicture}[scale=1,font=\footnotesize,line join=round,line cap=round,>=stealth]
				\def\bc{4} % cạnh BC
				\def\ba{2} % cạnh BA
				\def\h{4} % đường cao
				\def\gocB{30} % góc B của đáy
				\coordinate[label=below left:$B$] (B) at (0,0);
				\coordinate[label=below:$A$] (A) at (\gocB:\ba);
				\coordinate[label=below:$C$] (C) at (\bc,0);
				\coordinate[label=right:$D$] (D) at ($(C)-(B)+(A)$);
				\coordinate[label=above:$S$] (S) at ($(A)+(90:\h)$);
				\coordinate[label=above left:$M$] (M) at ($(S)!0.5!(B)$);
				\coordinate[label=below left:$N$] (N) at ($(S)!0.5!(C)$);
				\coordinate[label=right:$P$] (P) at ($(S)!0.5!(D)$);
				\coordinate[label=above left:$Q$] (Q) at ($(S)!0.5!(A)$);
				\draw (B)--(C)--(D)--(S)--cycle (S)--(C) (N)--(P);
				\draw[dashed] (A)--(D) (A)--(M) (S)--(A)--(B) (M)--(Q) (P)--(Q);
				\foreach \diem in {A,B,C,D,S,M,N,P,Q}	\fill (\diem)circle(1.2pt);
				\newcommand{\gocv}[4][black]{\draw[#1] ($(#3)!5pt!(#2)$)--($(#3)!2!($($(#3)!5pt!(#2)$)!.5!($(#3)!5pt!(#4)$)$)$)--($(#3)!5pt!(#4)$);}
				\gocv{S}{A}{D}
			\end{tikzpicture}
		}
	}
\end{bt}
\begin{bt}%[Tex hoá SGK KNTT]%[Phạm Văn Long]%[1K7KP-4]
	Cho hình hộp đứng $ABCD.A'B'C'D'$ có cạnh bên $AA'=2 a$ và đáy $A B C D$ là hình thoi có $A B=a$ và $A C=a \sqrt{3}$. Tính khoảng cách giữa hai đường thẳng $B D$ và $AA'$.
	\loigiai{
		\begin{center}
			\begin{tikzpicture}[>=stealth,line join=round,line cap=round,font=\footnotesize,scale=0.7]
				\coordinate[label=above left:{$A$}] (A) at (0,0);
				\coordinate[label=below left:{$B$}] (B) at (-2.75,-1.5);
				\coordinate[label=right:{$D$}] (D) at (4.5,0);
				\coordinate[label=below right:{$C$}] (C) at ($(B)+(D)-(A)$);
				\coordinate[label=above left:{$A'$}] (A') at (0,4.5);
				\coordinate[label=above right:{$D'$}] (D') at ($(D)+(A')-(A)$);
				\coordinate[label=above left:{$B'$}] (B') at ($(A')+(B)-(A)$);
				\coordinate[label=above left:{$C'$}] (C') at ($(B')+(D')-(A')$);
				\coordinate[label=below:{$O$}]  (O) at ($(A)!0.5!(C)$);				
				\draw (A')--(B')--(C')--(D')--cycle (B)--(B') (C)--(C') (D)--(D')
				(B)--(C)--(D) ;
				\draw[dashed] (A)--(A') (B)--(A)--(D) (A)--(C) (B)--(D);
				\draw pic[draw,angle radius=2mm] {right angle = B--O--A};
				\foreach \p in {A,B,C,D,A',B',C',D',O}
				\fill (\p)	circle (1.2pt);	
			\end{tikzpicture}
		\end{center}
		Ta có $\heva{&AO \perp AA'\\&AO \perp BD}\Rightarrow \mathrm{d}(BD,AA')=\dfrac{1}{2}AC=\dfrac{a\sqrt{3}}{2}$.	
	}
\end{bt}


\begin{bt}%[Tex hoá SGK KNTT]%[Phạm Văn Long]%[1K7KP-4]
	\immini
	{
		Cho lăng trụ $A B C D. A' B' C' D'$ có đáy $A B C D$ là hình vuông cạnh $2 a$, $O$ là giao điểm của $A C$ và $B D$, $A A'=a$, $A A'$ vuông góc với mặt phẳng chứa đáy. Tính
		\begin{enumerate}
			\item $\mathrm{d}\left(A C, A' B'\right)$;
			\item $\mathrm{d}\left(C C', B D\right)$.
		\end{enumerate}
	}
	{
		\begin{tikzpicture}[scale=.7,font=\footnotesize, line join=round, line cap=round, >=stealth]
			\path
			(0,0) coordinate (A)
			(5,0) coordinate (D)
			(-2,-2) coordinate (B);
			\coordinate (C) at ($(B)+(D)-(A)$);
			\coordinate (O) at ($(A)!.5!(C)$);
			\coordinate (A') at ($(A)+(0,4)$);
			\coordinate (B') at ($(B)+(A')-(A)$);
			\coordinate (C') at ($(C)+(A')-(A)$);
			\coordinate (D') at ($(D)+(A')-(A)$);
			\foreach \x/\g in {A/170,B/-120,C/-60,D/0,A'/90,B'/170,C'/-10,D'/60,O/-90} \fill[black](\x) circle (1.5pt) ($(\x)+(\g:4mm)$) node{$\x$};
			\draw (B')--(B)--(C)--(D)--(D')--(A')--(B')--(C')--(D')
			(C)--(C');
			\draw[dashed] (A)--(B)--(D)--(A)--(A')
			(A)--(C);
		\end{tikzpicture}
	}
	\loigiai{
		\begin{enumerate}
			\item Vì $A A'$ vuông góc với cả hai mặt phẳng $(A B C D)$ và $\left(A' B' C' D'\right)$ nên $A A' \perp A C$, $A A' \perp A' B'$.\\
			Suy ra đoạn thẳng $A A'$ là đoạn vuông góc chung của $A C$ và $A' B'$. Vậy $\mathrm{d}\left(A C, A' B'\right)=A A'=a$.
			\item Vì $C C'$ vuông góc với $(A B C D)$ nên $C C' \perp O C$. Do đáy $A B C D$ là hình vuông có $O$ là giao điểm của $A C$ và $B D$ nên $B D \perp O C$. Suy ra đoạn thẳng $O C$ là đoạn vuông góc chung của $C C'$ và $B D$. Vậy $\mathrm{d}\left(C C', B D\right)=O C=a \sqrt{2}$.
		\end{enumerate}
	}
\end{bt}
% \subsubsection{Bài tập trắc nghiệm}
% %\paragraph{Câu hỏi trắc nghiệm}
% \Opensolutionfile{ans}[ans/ans-1K7-26-Dang5]%
% \begin{ex}%[Tex hoá SGK KNTT]%[Phạm Văn Long]%[1K7KP-4]
% 	\immini{
% 		Cho tứ diện $OABC$ có $OA$, $OB$, $OC$ đôi một vuông góc với nhau và $OA=OB=OC=a$ (tham khảo hình vẽ). Khoảng cách giữa hai đường thẳng $OA$ và $BC$ bằng
% 		\choice
% 		{$\dfrac{a\sqrt{3}}{2}$}
% 		{$\dfrac{a}{2}$}
% 		{\True $\dfrac{a\sqrt{2}}{2}$}
% 		{$\dfrac{3a}{2}$}
% 	}{
% 		\begin{tikzpicture}[scale=1,font=\footnotesize,line join=round,line cap=round,>=stealth]
% 			\path
% 			(0,0) coordinate (O)
% 			(-1.3,-1.5) coordinate (A)
% 			(3,0) coordinate (B)
% 			(0,3)coordinate (C)
% 			;
% 			\draw (A)--(B)--(C)--cycle;
% 			\draw[dashed] (B)--(O)--(A) (O)--(C);
% 			\foreach \p/\q in {A/180,B/0,C/90,O/180}
% 			\fill[black] (\p) circle (1.0pt) ($(\p)+(\q:3mm)$) node{$\p$};
% 			\draw pic[draw=black,angle radius=0.2cm] {right angle = C--O--B};
% 			\draw pic[draw=black,angle radius=0.2cm] {right angle = A--O--B};			
% 		\end{tikzpicture}
% 	}
% 	\loigiai{
% 		\immini{
% 			Gọi $H$ là trung điểm của $BC$, vì $OB=OC=a\Rightarrow OH\perp BC$.\\
% 			Ta có $\heva{& OA\perp OB \\ & OA\perp OC}\Rightarrow OA\perp (OBC)\Rightarrow OA\perp OH$.\\
% 			Do đó $OH$ là đoạn vuông góc chung của $OA$ và $BC$ nên \[\mathrm{d}(OA,BC)=OH=\dfrac{a\sqrt{2}}{2}.\]
% 		}{
% 			\begin{tikzpicture}[scale=0.9,font=\footnotesize,line join=round,line cap=round,>=stealth]
% 				\path
% 				(0,0) coordinate (O)
% 				(-1.3,-1.5) coordinate (A)
% 				(3,0) coordinate (B)
% 				(0,3)coordinate (C)
% 				($(C)!0.5!(B)$) coordinate (H)
% 				;
% 				\draw (A)--(B)--(C)--cycle;
% 				\draw[dashed] (B)--(O)--(A) (H)--(O)--(C);
% 				\foreach \p/\q in {A/180,B/0,C/90,O/180,H/40}
% 				\fill[black] (\p) circle (1.0pt) ($(\p)+(\q:3mm)$) node{$\p$};
% 				\draw pic[draw=black,angle radius=0.2cm] {right angle = C--O--B};
% 				\draw pic[draw=black,angle radius=0.2cm] {right angle = A--O--B};
% 				\draw pic[draw=black,angle radius=0.2cm] {right angle = C--H--O};			
% 			\end{tikzpicture}
% 		}
% 	}
% \end{ex}
% \begin{ex}%[Tex hoá SGK KNTT]%[Phạm Văn Long]%[1K7KP-4]
% 	\immini{
% 		Cho hình chóp $S.ABC$ có đáy $ABC$ là tam giác đều cạnh $a$. $SA$ vuông góc với mặt phẳng đáy và $SA=\dfrac{a}{2}$. Khoảng cách giữa hai đường thẳng $SA$ và $BC$ bằng	
% 		\choice
% 		{$a\sqrt{3}$}
% 		{$a$}
% 		{$\dfrac{a\sqrt{3}}{4}$}
% 		{\True $\dfrac{a\sqrt{3}}{2}$}
% 	}{
% 		\begin{tikzpicture}[scale=0.7,font=\footnotesize,line join=round,line cap=round,>=stealth]
% 			\coordinate (A) at (0,0);
% 			\coordinate (B) at (2,-2);
% 			\coordinate (C) at (7,0);
% 			\coordinate (S) at ($(O)+(0,4)$);
% 			\draw (B) node[below]{$B$}--(S) node[above]{$S$}--(A) node[left]{$A$}--(B)--(C) node[right]{$C$}--(S);
% 			\draw[dashed] (A)--(C);
% 		\end{tikzpicture}
% 	}
% 	\loigiai{
% 		\immini{
% 			Gọi $H$ là trung điểm $BC\Rightarrow AH\perp BC$ và $AH=\dfrac{a\sqrt{3}}{2}$.\\
% 			Vậy $\heva{&AH\perp AS\\&AH\perp BC}\Rightarrow AH$ là khoảng cách giữa $SA$ và $BC$.\\ Do đó $\mathrm{d}(SA,BC)=\dfrac{a\sqrt{3}}{2}$.
% 		}{
% 			\begin{tikzpicture}[scale=0.7,font=\footnotesize,line join=round,line cap=round,>=stealth]
% 				\coordinate (A) at (0,0);
% 				\coordinate (B) at (2,-2);
% 				\coordinate (C) at (7,0);
% 				\coordinate (S) at ($(O)+(0,4)$);
% 				\coordinate (H) at ($(B)!.5!(C)$);
% 				\draw (B) node[below]{$B$}--(S) node[above]{$S$}--(A) node[left]{$A$}--(B)--(C) node[right]{$C$}--(S);
% 				\draw[dashed] (H)node[below]{$H$}--(A)--(C);
% 				%SHA
% 				\draw ($ (H)!5pt!(A)$)--($(H)!2!($($(H)!5pt!(A)$)!.5!($(H)!5pt!(B)$)$)$)--($(H)!5pt!(B)$);
% 				\draw ($ (A)!5pt!(A)$)--($(A)!2!($($(A)!5pt!(S)$)!.5!($(A)!5pt!(H)$)$)$)--($(A)!5pt!(H)$);
% 			\end{tikzpicture}
% 		}	
% 	}
% \end{ex}

% \begin{ex}%[Tex hoá SGK KNTT]%[Phạm Văn Long]%[1K7KP-4]
% 	\immini{
% 		Cho tứ điện đều $ABCD$ cạnh $a$. Khoảng cách giữa hai đường thẳng $AB$ và $CD$ bằng	
% 		\choice
% 		{$a\sqrt{2}$}
% 		{$\dfrac{a}{2}$}
% 		{$a$}
% 		{\True $\dfrac{a\sqrt{2}}{2}$}
% 	}{
% 		\begin{tikzpicture}[scale=0.7,font=\footnotesize,line join=round,line cap=round,>=stealth]
% 			\coordinate (B) at (0,0);
% 			\coordinate (C) at (2,-2);
% 			\coordinate (D) at (7,0);
% 			\coordinate (B1) at ($(C)!0.5!(D)$);
% 			\coordinate (D1) at ($(C)!0.5!(B)$);
% 			\coordinate (G) at ($(B)!2/3!(B1)$);
% 			\coordinate (A) at ($(G)+(0,4)$);
% 			\draw (A) node[above]{$A$}--(C) node[below]{$C$}--(B) node[left]{$B$}--(A)--(D) node[right]{$D$}--(C);
% 			\draw[dashed] (D1)--(D)--(B)--(B1) (G)--(A);
% 			\draw (B1) node[below right]{$I$} (D1) node[below left]{$J$};
% 		\end{tikzpicture}
% 	}
% 	\loigiai{
% 		\immini{
% 			Do $ABCD$ là tứ diện đều. Gọi $I$ là trung điểm $CD$, ta có $AI=BI$ và $CD\perp (IAB)$.\\
% 			Gọi $E$ là trung điểm của $AB$, suy ra $IE\perp AB$ và $IE\perp CD$. Vậy $IE$ là khoảng cách giữa hai đường thẳng $AB$ và $CD$.\\
% 			Ta có $IE=\sqrt{AI^2-AE^2}=\sqrt{\dfrac{3a^2}{4}-\dfrac{a^2}{4}}=\dfrac{a\sqrt{2}}{2}$.\\
% 			Vậy $\mathrm{d}(AB,CD)=\dfrac{a\sqrt{2}}{2}$.
% 		}{
% 			\begin{tikzpicture}[scale=0.7,font=\footnotesize,line join=round,line cap=round,>=stealth]
% 				\coordinate (B) at (0,0);
% 				\coordinate (C) at (2,-2);
% 				\coordinate (D) at (7,0);
% 				\coordinate (B1) at ($(C)!0.5!(D)$);
% 				\coordinate (D1) at ($(C)!0.5!(B)$);
% 				\coordinate (G) at ($(B)!2/3!(B1)$);
% 				\coordinate (A) at ($(G)+(0,4)$);
% 				\draw (A) node[above]{$A$}--(C) node[below]{$C$}--(B) node[left]{$B$}--(A)--(D) node[right]{$D$}--(C);
% 				\draw[dashed] (D1)--(D)--(B)--(B1) (G)--(A);
% 				\coordinate (E) at ($(A)!.5!(B)$);
% 				\coordinate (I) at ($(C)!.5!(D)$);
% 				\draw[dashed](I)--(E);
% 				\draw (I)--(A);
% 				\foreach \p/\g in {I/-30,E/130} \draw[fill] (\p) circle(.5pt)node [shift={(\g:.3)}] {$\p$};
% 				\draw ($ (E)!5pt!(I)$)--($(E)!2!($($(E)!5pt!(I)$)!.5!($(E)!5pt!(B)$)$)$)--($(E)!5pt!(B)$);
% 				\draw ($ (I)!5pt!(D)$)--($(I)!2!($($(I)!5pt!(D)$)!.5!($(I)!5pt!(E)$)$)$)--($(I)!5pt!(E)$);
% 				\draw (D1) node[below left]{$J$};
% 			\end{tikzpicture}
% 		}	
% 	}
% \end{ex}
% \begin{ex}%[Tex hoá SGK KNTT]%[Phạm Văn Long]%[1K7KP-4]
% 	\immini{
% 		Cho hình chóp $S.ABCD$ có đáy $ABCD$ là hình vuông cạnh $a$, cạnh $SA=a$ và vuông góc với mặt đáy $(ABCD)$ (tham khảo hình bên). Khoảng cách giữa hai đường thẳng $SC$ và $BD$ bằng
% 		\choice
% 		{$\dfrac{a\sqrt{3}}{4}$}
% 		{$\dfrac{a\sqrt{6}}{3}$}
% 		{$\dfrac{a}{2}$}
% 		{\True $\dfrac{a\sqrt{6}}{6}$}
% 	}{
% 		\begin{tikzpicture}[scale=0.7, font=\footnotesize, line join=round, line cap=round,>=stealth]
% 			\path
% 			(0,0) coordinate (A)
% 			(-1.3,-1.6) coordinate (B)
% 			(2.5,-1.6)coordinate (C)
% 			($(A)+(C)-(B)$) coordinate (D)
% 			($(A)+(0,3)$) coordinate (S)
% 			;
% 			\draw (S)--(B)--(C)--(D)--cycle (S)--(C);
% 			\draw[dashed] (S)--(A)--(D) (A)--(B)--(D);	
% 			\foreach \p/\q in {S/90,A/-90,B/-90,C/-90,D/0}			
% 			\fill[black] (\p) circle (1.0pt)node[shift={(\q:2.5mm)}]{$\p$};
% 			\draw pic[draw=black,angle radius=0.2cm] {right angle = S--A--B}; 
% 			\draw pic[draw=black,angle radius=0.2cm] {right angle = A--B--C};
% 			\draw pic[draw=black,angle radius=0.2cm] {right angle = A--D--C};
% 			\draw pic[draw=black,angle radius=0.2cm] {right angle = B--C--D};
% 		\end{tikzpicture}
% 	}
% 	\loigiai{
% 		\immini{
% 			Gọi $O=AC\cap BD$, $H$ là hình chiếu của $O$ trên $SC$.\\
% 			Ta có $\heva{& BD\perp AC \\ & BD\perp SA}\Rightarrow BD\perp (SAC)$.\\
% 			Suy ra $BD\perp OH$.\\
% 			Do đó $OH$ là đoạn vuông góc chung của $BD$ và $SC$ nên $\mathrm{d}(BD,SC)=OH$.\\
% 			Ta lại có 
% 		}{
% 			\begin{tikzpicture}[scale=0.7, font=\footnotesize, line join=round, line cap=round,>=stealth]
% 				\path
% 				(0,0) coordinate (A)
% 				(-1.3,-1.6) coordinate (B)
% 				(2.5,-1.6)coordinate (C)
% 				($(A)+(C)-(B)$) coordinate (D)
% 				($(A)+(0,3)$) coordinate (S)
% 				($(A)!0.5!(C)$) coordinate (O)
% 				($(S)!0.7!(C)$) coordinate (H)				
% 				;
% 				\draw (S)--(B)--(C)--(D)--cycle (S)--(C);
% 				\draw[dashed] (S)--(A)--(D) (C)--(A)--(B)--(D) (O)--(H);	
% 				\foreach \p/\q in {S/90,A/-90,B/-90,C/-90,D/0,O/-90,H/10}			
% 				\fill[black] (\p) circle (1.0pt)node[shift={(\q:2.5mm)}]{$\p$};
% 				\draw pic[draw=black,angle radius=0.2cm] {right angle = S--A--B}; 
% 				\draw pic[draw=black,angle radius=0.2cm] {right angle = A--B--C};
% 				\draw pic[draw=black,angle radius=0.2cm] {right angle = A--D--C};
% 				\draw pic[draw=black,angle radius=0.2cm] {right angle = B--C--D};
% 				\draw pic[draw=black,angle radius=0.2cm] {right angle = O--H--S};
% 			\end{tikzpicture}
% 		}\noindent 
% 		$\triangle SAC\backsim \triangle OHC\Rightarrow \dfrac{OH}{SA}=\dfrac{OC}{SC}\Rightarrow OH=\dfrac{OC\cdot SA}{SC}=\dfrac{\dfrac{a\sqrt{2}}{2}\cdot a}{\sqrt{a^2+\left(a\sqrt{2}\right)^2}}=\dfrac{a\sqrt{6}}{6}$.\\
% 		Vậy $\mathrm{d}(BD,SC)=\dfrac{a\sqrt{6}}{6}$.
% 	}
% \end{ex}
% \begin{ex}%[Tex hoá SGK KNTT]%[Phạm Văn Long]%[1K7BP-4]
% 	\immini{
% 		Cho hình chóp $S.ABCD$ có đáy là hình vuông cạnh $a$, $SA=a$ và $SA$ vuông góc với đáy (tham khảo hình bên). Khoảng cách giữa hai đường thẳng $AB$ và $SC$ bằng
% 		\choice
% 		{$a\sqrt{2}$}
% 		{\True $\dfrac{a\sqrt{2}}{2}$}
% 		{$\dfrac{a\sqrt{2}}{3}$}
% 		{$\dfrac{a\sqrt{2}}{4}$}
% 	}{
% 		\begin{tikzpicture}[scale=0.7, font=\footnotesize, line join=round, line cap=round,>=stealth]
% 			\path
% 			(0,0) coordinate (A)
% 			(-1.3,-1.6) coordinate (B)
% 			(2.5,-1.6)coordinate (C)
% 			($(A)+(C)-(B)$) coordinate (D)
% 			($(A)+(0,3)$) coordinate (S)
% 			;
% 			\draw (S)--(B)--(C)--(D)--cycle (S)--(C);
% 			\draw[dashed] (S)--(A)--(D) (A)--(B);	
% 			\foreach \p/\q in {S/90,A/-90,B/-90,C/-90,D/0}			
% 			\fill[black] (\p) circle (1.0pt)node[shift={(\q:2.5mm)}]{$\p$};
% 			\draw pic[draw=black,angle radius=0.2cm] {right angle = S--A--D}; 
% 		\end{tikzpicture}
% 	}
% 	\loigiai{
% 		\immini{
% 			Gọi $H$ là trung điểm của $SD\Rightarrow AH\perp SD$.\\
% 			Ta có $\heva{& CD\perp AD \\ & CD\perp SA}\Rightarrow CD\perp (SAD)$.\\
% 			Do đó $\heva{& AH\perp CD \\ & AH\perp SD}\Rightarrow AH\perp(SCD)$.\\
% 			Ta có $\heva{& AB\parallel CD \\ & CD\subset (SCD)}\Rightarrow AB\parallel (SCD)$.				
% 		}{
% 			\begin{tikzpicture}[scale=0.8, font=\footnotesize, line join=round, line cap=round,>=stealth]
% 				\path
% 				(0,0) coordinate (A)
% 				(-1.3,-1.6) coordinate (B)
% 				(2.5,-1.6)coordinate (C)
% 				($(A)+(C)-(B)$) coordinate (D)
% 				($(A)+(0,3)$) coordinate (S)
% 				($(S)!0.5!(D)$) coordinate (H)
% 				;
% 				\draw (S)--(B)--(C)--(D)--cycle (S)--(C);
% 				\draw[dashed] (S)--(A)--(D) (H)--(A)--(B);	
% 				\foreach \p/\q in {S/90,A/-90,B/-90,C/-90,D/0,H/20}			
% 				\fill[black] (\p) circle (1.0pt)node[shift={(\q:2.5mm)}]{$\p$};
% 				\draw pic[draw=black,angle radius=0.2cm] {right angle = S--A--D}; 
% 			\end{tikzpicture}
% 		}\noindent 
% 		Mà $SC\subset(SCD)$ nên $\mathrm{d}(AB,SC)=\mathrm{d}(AB,(SCD))=\mathrm{d}(A,(SCD))=AH=\dfrac{a\sqrt{2}}{2}$.
% 	}
% \end{ex}
% \begin{ex}%[Tex hoá SGK KNTT]%[Phạm Văn Long]%[1K7BP-4]
% 	\immini{
% 		Cho hình chóp $S.ABC$ có đáy $ABC$ là tam giác vuông tại $B$, $AB=3a$, $BC=4a$. Cạnh bên $SA$ vuông góc với đáy. Góc tạo bởi giữa $SC$ và đáy bằng $60^{\circ}$. Gọi $M$ là trung điểm của $AC$ (tham khảo hình bên). Khoảng cách giữa hai đường thẳng $AB$ và $SM$ bằng
% 		\choice
% 		{$\dfrac{a\sqrt{13}}{2}$}
% 		{\True $\dfrac{10a\sqrt{3}}{\sqrt{79}}$}
% 		{$\dfrac{5 a}{2}$}
% 		{$5 a\sqrt{3}$}
% 	}{
% 		\begin{tikzpicture}[scale=0.7, font=\footnotesize, line join=round, line cap=round,>=stealth]
% 			\path
% 			(0,0) coordinate (A)
% 			(1.2,-1.5) coordinate (C)
% 			(4,0) coordinate (B)
% 			($(A)+(0,3.3)$) coordinate (S)
% 			($(A)!0.5!(C)$) coordinate (M)			
% 			;
% 			\draw (S)--(A)--(C)--(B)--cycle (M)--(S)--(C);	
% 			\draw[dashed] (A)--(B);	
% 			\foreach \p/\q in {S/90,A/-90,B/0,C/-90,M/-135}
% 			\fill[black] (\p) circle (1.0pt) ($(\p)+(\q:3mm)$) node{$\p$};
% 			\draw pic[draw=black,angle radius=0.2cm] {right angle = S--A--B};
% 		\end{tikzpicture}
% 	}
% 	\loigiai{
% 		\immini{
% 			Tam giác $ABC$ vuông tại $B$ nên có $AC=\sqrt{AB^2+BC^2}=5a$.\\
% 			Lại có $SA\perp(ABC)$, do đó hình chiếu vuông góc của $SC$ trên $(ABC)$ là $AC$.\\
% 			Suy ra $60^\circ=\left(SC,(ABC)\right)=(SC,AC)=\widehat{SCA}$ (vì $\triangle SAC$ vuông tại $A$).\\
% 			Trong tam giác $SAC$ vuông tại thì $SA=AC\cdot\tan60^\circ=5a\sqrt{3}$.					
% 		}{
% 			\begin{tikzpicture}[scale=0.8, font=\footnotesize, line join=round, line cap=round,>=stealth]
% 				\path
% 				(0,0) coordinate (A)
% 				(1.2,-1.5) coordinate (C)
% 				(4,0) coordinate (B)
% 				($(A)+(0,3.3)$) coordinate (S)
% 				($(A)!0.5!(C)$) coordinate (M)
% 				($(B)!0.5!(C)$) coordinate (N)
% 				($(N)!2!(M)$) coordinate (P)
% 				($(S)!0.55!(P)$) coordinate (H)			
% 				;
% 				\draw (S)--(P)--(M)--(C)--(B)--cycle (M)--(S)--(C) (N)--(S);	
% 				\draw[dashed] (H)--(A)--(B) (P)--(A)--(S) (A)--(M)--(N);	
% 				\foreach \p/\q in {S/90,A/45,B/0,C/-90,M/-130,N/-40,P/180,H/160}
% 				\fill[black] (\p) circle (1.0pt) ($(\p)+(\q:3mm)$) node{$\p$};
% 				\draw pic[draw=black,angle radius=0.2cm] {right angle = P--H--A};
% 				\draw pic[draw=black,angle radius=0.2cm] {right angle = A--B--C};
% 				\draw pic[draw=black,angle radius=0.2cm] {right angle = A--P--N};
% 			\end{tikzpicture}
% 		}\noindent 
% 		Gọi $N$ là trung điểm của $BC$, gọi $P$, $H$ lần lượt là hình chiếu của $A$ trên $MN$, $SP$.\\
% 		Ta có tứ giác $ABNP$ là hình chữ nhật.\\
% 		$\heva{& PN\perp SA \\ & PN\perp AP}\Rightarrow PN\perp(SAP)$, 
% 		$\heva{& AH\perp PN \\ & AH\perp SP}\Rightarrow AH\perp(SPN)$.\\
% 		Do $PN\parallel AB\Rightarrow \mathrm{d}(AB,SM)=\mathrm{d}(AB,(SPN))=\mathrm{d}(A,(SPN))=AH$.\\
% 		Trong tam giác $SAP$ vuông tại $A$, ta có 
% 		\[AH=\dfrac{AS\cdot AP}{\sqrt{AS^2+AP^2}} =\dfrac{5a\sqrt{3}\cdot2a}{\sqrt{\left(5a\sqrt{3}\right)^2+(2a)^2}}=\dfrac{10a\sqrt{237}}{79}.\] 
% 		Vậy $\mathrm{d}(BM,AM)=\dfrac{10a\sqrt{237}}{79}$.
% 	}
% \end{ex}
% \begin{ex}%[Tex hoá SGK KNTT]%[Phạm Văn Long]%[1K7BP-4]
% 	\immini{
% 		Cho tứ diện $OABC$ có đáy $OBC$ là tam giác vuông tại $O$, $OB=a$, $OC=a\sqrt{3}$ và đường cao của tứ diện là $OA=a\sqrt{3}$. Gọi $M$ là trung điểm $BC$. Khi đó khoảng cách giữa hai đường thẳng $AB$ và $OM$ bằng	
% 		\choice
% 		{$\dfrac{a\sqrt{3}}{5}$}
% 		{$\dfrac{a\sqrt{15}}{15}$}
% 		{\True $\dfrac{a\sqrt{15}}{5}$}
% 		{$\dfrac{a}{5}$}
% 	}{
% 		\begin{tikzpicture}[scale=0.7,font=\footnotesize,line join=round,line cap=round,>=stealth]
% 			\coordinate (B) at (0,0);
% 			\coordinate (O) at (2,2);
% 			\coordinate (C) at (7,2);
% 			\coordinate (A) at ($(O)+(0,4)$);
% 			\coordinate (M) at ($(B)!0.5!(C)$);
% 			\draw (A) node[above]{$A$}--(C) node[right]{$C$}--(M) node[below]{$M$}--(B) node[left]{$B$}--cycle;
% 			\draw[dashed] (M)--(O)--(A) (B)--(O)--(C);
% 			\draw (O) node[left]{$O$};
% 		\end{tikzpicture}
% 	}
% 	\loigiai{
% 		\immini{
% 			Do $\triangle OBC$ vuông tại $O$, $OM$ là trung tuyến  và $OB=a$, $OC=a\sqrt{3}\Rightarrow \widehat{OBC}=60^\circ$ nên $\triangle OBM$ đều.\\
% 			Gọi $D$ là đỉnh hình bình hành $OMBD$, suy ra $\triangle OBD$ đều cạnh $a$. Gọi $H$ là trung điểm $BD\Rightarrow OH\perp BD$. Kẻ $OK\perp AH$ tại $K$.\\
% 			Có $\heva{&BD\perp OH\\&BD\perp AO}$\\$\Rightarrow BD\perp (AOH) \Rightarrow BD\perp OK$.\\
% 			Có $\heva{&OK\perp AH\\&OK\perp BD}\Rightarrow OK\perp (SBD)$\\
% 			$\Rightarrow OK$ là khoảng cách từ $O$ đến $(ABD)$.
% 		}{
% 			\begin{tikzpicture}[scale=0.7,font=\footnotesize,line join=round,line cap=round,>=stealth]
% 				\coordinate (B) at (0,0);
% 				\coordinate (O) at (2,2);
% 				\coordinate (C) at (7,2);
% 				\coordinate (A) at ($(O)+(0,4)$);
% 				\coordinate (M) at ($(B)!0.5!(C)$);
% 				\coordinate (D) at ($(B)+(O)-(M)$);
% 				\coordinate (H) at ($(B)!0.5!(D)$);
% 				\coordinate (K) at ($(A)!0.45!(H)$);
% 				\draw (A) node[above]{$A$}--(C) node[right]{$C$}--(M) node[below]{$M$}--(B)--cycle;
% 				\draw[dashed] (M)--(O)--(A) (B)--(O)--(C) (O)--(H) (O)--(K) (O)--(D);
% 				\draw (A)--(D)--(B) (A)--(H);
% 				\foreach \p/\g in {O/45,B/-90,D/-125,H/-120,K/190} \draw[fill] (\p) circle(.5pt)node [shift={(\g:.3)}] {$\p$}; 
% 				\draw ($ (H)!5pt!(O)$)--($(H)!2!($($(H)!5pt!(O)$)!.5!($(H)!5pt!(B)$)$)$)--($(H)!5pt!(B)$);
% 				\draw ($ (K)!5pt!(O)$)--($(K)!2!($($(K)!5pt!(O)$)!.5!($(K)!5pt!(A)$)$)$)--($(K)!5pt!(A)$);
% 			\end{tikzpicture}
% 		}
% 		Có $\dfrac{1}{OK^2}=\dfrac{1}{OH^2}+\dfrac{1}{OA^2}=\dfrac{4}{3a^2}+\dfrac{1}{3a^2}=\dfrac{5}{3a^2}\Rightarrow OK=\dfrac{a\sqrt{15}}{5}$.\\
% 		Do $BD\parallel OM\Rightarrow \mathrm{d}(MO,AB)=\mathrm{d}(MO,(ABD))=\mathrm{d}(O,(ABD))=OK=\dfrac{a\sqrt{15}}{5}$.			
% 	}
% \end{ex}
% \begin{ex}%[Tex hoá SGK KNTT]%[Phạm Văn Long]%[1K7BP-4]
% 	\immini{
% 		Cho tứ diện đều $ABCD$ cạnh bằng $a$. Gọi $M$ là trung điểm của $CD$ (tham khảo hình bên). Khoảng cách giữa hai đường thẳng $AC$ và $BM$ bằng
% 		\choice
% 		{\True $\dfrac{a\sqrt{22}}{11}$}
% 		{$\dfrac{a\sqrt{2}}{3}$}
% 		{$\dfrac{a\sqrt{3}}{3}$}
% 		{$\dfrac{5 a}{2}$}
% 	}{
% 		\begin{tikzpicture}[scale=0.7,font=\footnotesize,line join=round,line cap=round,>=stealth]
% 			\path
% 			(0,0) coordinate (B)
% 			(1,-1.4) coordinate (D)
% 			(4,0) coordinate (C)					
% 			($(C)!0.5!(D)$) coordinate (M)
% 			($(B)!2/3!(M)$) coordinate (G)
% 			($(G)+(0,3)$)coordinate (A)						
% 			;
% 			\draw (A)--(B)--(D)--(C)--cycle (A)--(D);
% 			\draw[dashed] (B)--(C) (B)--(M);
% 			\foreach \p/\q in {A/90,B/180,C/0,D/-90,M/-40}
% 			\fill[black] (\p) circle (1.0pt) ($(\p)+(\q:3mm)$) node{$\p$};			
% 		\end{tikzpicture}
% 	}
% 	\loigiai{
% 		\immini{
% 			Gọi $E$ là đỉnh thứ tư của hình bình hành $BMCE$, $G$ là trọng tâm tam giác $BCD$.\\
% 			Gọi $K$, $H$ lần lươt là hình chiếu của $G$ trên $CE$, $AK$.\\
% 			Vì $ABCD$ là tứ diện đều nên $AG\perp (BCD)$,\\ $AG=\sqrt{AB^2-BG^2}=\sqrt{a^2-\left(\dfrac{a\sqrt{3}}{3}\right)^2} =a\sqrt{\dfrac{2}{3}}$.\\
% 			Ta có $\heva{& CE\perp AG \\ & CE\perp GK}\Rightarrow CE\perp(AGK)$.\\
% 			Do đó $\heva{& GH\perp CE \\ & GH\perp AK}\Rightarrow GH\perp(ACE)$.			
% 		}{
% 			\begin{tikzpicture}[scale=1,font=\footnotesize,line join=round,line cap=round,>=stealth]
% 				\path
% 				(0,0) coordinate (B)
% 				(0.5,-1.4) coordinate (D)
% 				(4,0) coordinate (C)
% 				($(C)!0.5!(D)$) coordinate (M)
% 				($(B)!2/3!(M)$) coordinate (G)
% 				($(G)+(0,3)$)coordinate (A)
% 				($(B)+(C)-(M)$) coordinate (E)	
% 				($(E)!2/3!(C)$) coordinate (K)	
% 				($(A)!2/3!(K)$) coordinate (H)	
% 				;
% 				\draw (A)--(B)--(D)--(C)--cycle (A)--(D);
% 				\draw[dashed] (B)--(C)--(E)--(B)--(M) (A)--(G)--(K)--(A) (G)--(H);
% 				\foreach \p/\q in {A/90,B/180,C/0,D/-90,M/-40,G/-90,E/90,H/180,K/30}
% 				\fill[black] (\p) circle (1.0pt) ($(\p)+(\q:2.5mm)$) node{$\p$};
% 				\draw pic[draw=black,angle radius=0.2cm] {right angle = G--H--K}; 
% 			\end{tikzpicture}
% 		}\noindent 
% 		Do $BM\parallel CE\Rightarrow \mathrm{d}(BM,AC)=\mathrm{d}(BM,(ACE))=\mathrm{d}(G,(ACE))=GH$.\\
% 		Trong tam giác $AGK$ vuông tại $G$, ta có 
% 		\[GH=\dfrac{GA\cdot GK}{\sqrt{GH^2+GK^2}}=\dfrac{a\sqrt{\dfrac{2}{3}}\cdot\dfrac{a}{2}}{\sqrt{\left(a\sqrt{\dfrac{2}{3}}\right)^2+\left(\dfrac{a}{2}\right)^2}}=\dfrac{a\sqrt{22}}{11}.\] 
% 		Vậy $\mathrm{d}(BM,AC)=\dfrac{a\sqrt{22}}{11}$.
% 	}
% \end{ex}
% \begin{ex}%[Tex hoá SGK KNTT]%[Phạm Văn Long]%[1K7BP-4]
% 	\immini{
% 		Cho hình chóp $S.ABCD$ có đáy $ABCD$ là hình vuông cạnh $2a$, cạnh bên $SA=a\sqrt{5}$, mặt bên $SAB$ là tam giác cân đỉnh $S$ và thuộc mặt phẳng vuông góc với mặt phẳng đáy (tham khảo hình bên). Khoảng cách gữa hai đường thẳng $AD$ và $SC$ bằng
% 		\choice
% 		{$\dfrac{2a\sqrt{5}}{5}$}
% 		{\True $\dfrac{4a\sqrt{5}}{5}$}
% 		{$\dfrac{a\sqrt{15}}{15}$}
% 		{$\dfrac{2a\sqrt{15}}{15}$}
% 	}{
% 		\begin{tikzpicture}[scale=0.8, font=\footnotesize, line join=round, line cap=round,>=stealth]
% 			\path
% 			(0,0) coordinate (A)
% 			(-1.4,-1.6) coordinate (B)
% 			(2.5,-1.6) coordinate (C)
% 			($(A)+(C)-(B)$) coordinate (D)
% 			($(A)!0.5!(B)$) coordinate (H)
% 			($(H)+(0,3.5)$) coordinate (S)			
% 			;
% 			\draw (S)--(B)--(C)--(D)--cycle (S)--(C);
% 			\draw[dashed] (H)--(S)--(A)--(D) (A)--(B);	
% 			\foreach \p/\q in {S/90,A/-100,B/-90,C/-90,D/0}				
% 			\fill[black] (\p) circle (1.0pt) node[shift={(\q:3mm)}]{$\p$};			
% 		\end{tikzpicture}
% 	}
% 	\loigiai{
% 		\immini{
% 			Gọi $H$ là trung điểm của $AB$; gọi $I$ là hình chiếu của $H$ trên $SB$.\\
% 			Ta có $\heva{& (SAB)\perp(ABCD) \\ & (SAB)\cap(ABCD)=AB\\&SH\subset(SAB)\\&SH\perp AB}\Rightarrow SH\perp (ABCD)$.\\
% 			Lại có $\heva{& BC\perp SH\, (SH\perp(ABCD)) \\ & BC\perp AB}\Rightarrow BC\perp (SAB)$.	\\
% 			Do đó $\heva{& HI\perp SB \\ & HI\perp BC}\Rightarrow HI\perp(SBC)$.			
% 		}{
% 			\begin{tikzpicture}[scale=1, font=\footnotesize, line join=round, line cap=round,>=stealth]
% 				\path
% 				(0,0) coordinate (A)
% 				(-1.4,-1.6) coordinate (B)
% 				(2.5,-1.6) coordinate (C)
% 				($(A)+(C)-(B)$) coordinate (D)
% 				($(A)!0.5!(B)$) coordinate (H)
% 				($(H)+(0,3.5)$) coordinate (S)
% 				($(S)!0.7!(B)$) coordinate (I)
% 				;
% 				\draw (S)--(B)--(C)--(D)--cycle (S)--(C);
% 				\draw[dashed] (H)--(S)--(A)--(D) (A)--(B) (H)--(I);	
% 				\foreach \p/\q in {S/90,A/-100,B/-90,C/-90,D/0,H/-80,I/160}				
% 				\fill[black] (\p) circle (1.0pt) node[shift={(\q:3mm)}]{$\p$};	
% 				\draw pic[draw=black,angle radius=0.2cm] {right angle = S--I--H}; 
% 			\end{tikzpicture}
% 		}\noindent 
% 		Vì $AD\parallel BC\Rightarrow AD\parallel (SBC)$ và $SC\subset (SBC)$ nên 
% 		\[\mathrm{d}(AD,SC)=\mathrm{d}\left(AD,(SBC)\right)=\mathrm{d}\left(A,(SBC)\right)=2\cdot\mathrm{d}\left(H,(SBC)\right)=2HI.\]	
% 		Trong tam giác $SAH$ vuông tại $H$, ta có $SH=\sqrt{SA^2-AH^2}=2a$.\\	
% 		Trong tam giác $SHB$ vuông tại $H$
% 		\[HI=\dfrac{HS\cdot HB}{\sqrt{HS^2+HB^2}}=\dfrac{2a\cdot a}{\sqrt{(2a)^2+a^2}}=\dfrac{2a\sqrt{5}}{5}.\]
% 		Vậy $\mathrm{d}(AD,SC)=2\cdot \dfrac{2a\sqrt{5}}{5}=\dfrac{4a\sqrt{5}}{5}$.
% 	}
% \end{ex}
% \begin{ex}%[Tex hoá SGK KNTT]%[Phạm Văn Long]%[1K7BP-4]
% 	\immini{
% 		Cho hình chóp $S.ABC$ có đáy $ABC$ là tam giác vuông tại $B$, $AB=a$, cạnh bên $SA$ vuông góc với mặt đáy và $SA=a\sqrt{2}$. Gọi $M$ là trung điểm của $AB$ (tham khảo hình bên). Khoảng cách giữa hai đường thẳng $SM$ và $BC$ bằng
% 		\choice
% 		{$\dfrac{a}{2}$}
% 		{\True $\dfrac{a\sqrt{2}}{3}$}
% 		{$\dfrac{a\sqrt{3}}{3}$}
% 		{$\dfrac{a\sqrt{2}}{2}$}
% 	}{
% 		\begin{tikzpicture}[scale=0.7, font=\footnotesize, line join=round, line cap=round,>=stealth]
% 			\path
% 			(0,0) coordinate (A)
% 			(1.5,-1.5) coordinate (B)
% 			(4,0) coordinate (C)
% 			($(A)+(0,3.3)$) coordinate (S)
% 			($(A)!0.5!(B)$) coordinate (M)				
% 			;
% 			\draw (S)--(A)--(B)--(C)--cycle (M)--(S)--(B);	
% 			\draw[dashed] (A)--(C);	
% 			\foreach \p/\q in {S/90,A/180,B/-90,C/0,M/-130}
% 			\fill[black] (\p) circle (1.0pt) ($(\p)+(\q:3mm)$) node{$\p$};
% 			\draw pic[draw=black,angle radius=0.2cm] {right angle = A--B--C}; 
% 		\end{tikzpicture}
% 	}
% 	\loigiai{
% 		\immini{
% 			Gọi $N$ là trung điểm của $AC$, gọi $H$ là hình chiếu của $A$ trên $SM$.\\
% 			Ta có $MN$ là đường trung bình của tam giác $ABC$ nên $MN\parallel BC\Rightarrow BC\parallel(SMN)$.\\
% 			Do đó $\heva{& MN\perp SA \\ & MN\perp AB}\Rightarrow MN\perp(SAB)$.\\
% 			$\heva{& AH\perp MN \\ & AH\perp SM}\Rightarrow AH\perp(SMN)$.	
% 		}{
% 			\begin{tikzpicture}[scale=1, font=\footnotesize, line join=round, line cap=round,>=stealth]
% 				\path
% 				(0,0) coordinate (A)
% 				(1.5,-1.5) coordinate (B)
% 				(4,0) coordinate (C)
% 				($(A)+(0,3.3)$) coordinate (S)
% 				($(A)!0.5!(B)$) coordinate (M)
% 				($(A)!0.5!(C)$) coordinate (N)	
% 				($(S)!0.65!(M)$) coordinate (H)		
% 				;
% 				\draw (S)--(A)--(B)--(C)--cycle (M)--(S)--(B) (A)--(H);	
% 				\draw[dashed] (A)--(C) (M)--(N)--(S);	
% 				\foreach \p/\q in {S/90,A/180,B/-90,C/0,M/-130,N/45,H/165}
% 				\fill[black] (\p) circle (1.0pt) ($(\p)+(\q:2.5mm)$) node{$\p$};
% 				\draw pic[draw=black,angle radius=0.2cm] {right angle = A--B--C}; 
% 				\draw pic[draw=black,angle radius=0.2cm] {right angle = A--H--M}; 
% 			\end{tikzpicture}
% 		}\noindent 			
% 		$\mathrm{d}(BC,SM)=\mathrm{d}(BC,(SMN))=\mathrm{d}(B,(SMN))=\mathrm{d}(A,(SMN))=AH$.\\
% 		Trong tam giác $SAM$ vuông tại $A$
% 		\[AH=\dfrac{AS\cdot AM}{\sqrt{AS^2+AM^2}}=\dfrac{a\sqrt{2}\cdot \dfrac{a}{2}}{\sqrt{\left(a\sqrt{2}\right)^2+\left(\dfrac{a}{2}\right)^2}}=\dfrac{a\sqrt{2}}{3}.\]
% 		Vậy $\mathrm{d}(BC,SM)=\dfrac{a\sqrt{2}}{3}$.
% 	}
% \end{ex}
% \begin{ex}%[Tex hoá SGK KNTT]%[Phạm Văn Long]%[1K7BP-4]
% 	\immini{
% 		Cho hình chóp $S.ABCD$ có đáy là hình vuông cạnh $a$ và hai điểm $M$, $N$ lần lượt là trung điểm $AB$, $AD$. Tam giác $SAB$ đều và nằm trong mặt phẳng vuông góc với đáy (tham khảo hình vẽ). Khoảng cách giữa hai đường thẳng $SM$ và $NC$ bằng
% 		\choice
% 		{$\dfrac{3a}{4}$}
% 		{$a$}
% 		{$\dfrac{a\sqrt{5}}{10}$}
% 		{\True $\dfrac{3a\sqrt{5}}{10}$}
% 	}{
% 		\begin{tikzpicture}[scale=0.8, font=\footnotesize, line join=round, line cap=round,>=stealth]
% 			\path
% 			(0,0) coordinate (A)
% 			(-1.4,-1.6) coordinate (B)
% 			(2.5,-1.6) coordinate (C)
% 			($(A)+(C)-(B)$) coordinate (D)
% 			($(A)!0.5!(B)$) coordinate (M)
% 			($(M)+(0,3.5)$) coordinate (S)
% 			($(A)!0.5!(D)$) coordinate (N)
% 			;
% 			\draw (S)--(B)--(C)--(D)--cycle (S)--(C);
% 			\draw[dashed] (M)--(S)--(A)--(D) (A)--(B) (N)--(C);	
% 			\foreach \p/\q in {S/90,A/-100,B/-90,C/-90,D/0,M/-80,N/90}				
% 			\fill[black] (\p) circle (1.0pt) node[shift={(\q:3mm)}]{$\p$};			
% 		\end{tikzpicture}
% 	}
% 	\loigiai{
% 		\immini{
% 			Gọi $I=MD\cap CN$.\\	
% 			Ta có $\heva{& (SAB)\perp(ABCD) \\ & (SAB)\cap(ABCD)=AB\\&SM\subset(SAB)\\&SM\perp AB}\Rightarrow SM\perp (ABCD)$.\\
% 			Suy ra $SM\perp MI$.\\
% 			Lại có $\triangle ADM=\triangle DCN\Rightarrow \widehat{DNC}=\widehat{DMA}$.\\
% 			Mà $\widehat{DMA}+\widehat{ADM}=90^\circ\Rightarrow\widehat{DNC}+\widehat{ADM}=90^\circ	$ hay	$MI\perp NC$.\\
% 			Vậy $MI$ là đoạn vuông góc chung của $SM$ và $CN$.
% 		}{
% 			\begin{tikzpicture}[scale=1, font=\footnotesize, line join=round, line cap=round,>=stealth]
% 				\path
% 				(0,0) coordinate (A)
% 				(-1.4,-1.6) coordinate (B)
% 				(2.5,-1.6) coordinate (C)
% 				($(A)+(C)-(B)$) coordinate (D)
% 				($(A)!0.5!(B)$) coordinate (M)
% 				($(M)+(0,3.5)$) coordinate (S)
% 				($(A)!0.5!(D)$) coordinate (N)
% 				(intersection of M--D and C--N) coordinate (I)
% 				;
% 				\draw (S)--(B)--(C)--(D)--cycle (S)--(C);
% 				\draw[dashed] (M)--(S)--(A)--(D)--cycle (A)--(B) (N)--(C);	
% 				\foreach \p/\q in {S/90,A/-100,B/-90,C/-90,D/0,M/-80,N/90,I/-30}				
% 				\fill[black] (\p) circle (1.0pt) node[shift={(\q:3mm)}]{$\p$};
% 				\draw pic[draw=black,angle radius=0.2cm] {right angle = S--M--B}; 
% 				\draw pic[draw=black,angle radius=0.2cm] {right angle = M--I--C}; 
% 			\end{tikzpicture}
% 		}\noindent
% 		Trong tam giác $CDN$ vuông tại $D$, ta có $DI=\dfrac{DN\cdot DC}{\sqrt{DN^2+DC^2}}=\dfrac{\dfrac{a}{2}\cdot a}{\sqrt{\left(\dfrac{a}{2}\right)^2+a^2}}=\dfrac{a\sqrt{5}}{5}$.\\
% 		Vậy $\mathrm{d}(SM,NC)=MI=MD-DI=\dfrac{a\sqrt{5}}{2}-\dfrac{a\sqrt{5}}{5}=\dfrac{3a\sqrt{5}}{10}$.
% 	}
% \end{ex}
% \begin{ex}%[Tex hoá SGK KNTT]%[Phạm Văn Long]%[1K7BP-4]
% 	\immini{
% 		Cho hình chóp $S.ABCD$ có đáy $ABCD$ là hình chữ nhật, $AB=a$, $AD=2a$, $SA$ vuông góc với mặt phẳng đáy và $SA=a$ (tham khảo hình vẽ). Gọi $M$ là trung điểm của $CD$. Khoảng cách giữa hai đường thẳng $SD$ và $BM$ bằng
% 		\choice
% 		{$\dfrac{a\sqrt{21}}{21}$}
% 		{\True $\dfrac{2a\sqrt{21}}{21}$}
% 		{$\dfrac{2a\sqrt{7}}{7}$}
% 		{$\dfrac{a\sqrt{7}}{7}$}
% 	}{
% 		\begin{tikzpicture}[scale=0.7, font=\footnotesize, line join=round, line cap=round,>=stealth]
% 			\path
% 			(0,0) coordinate (A)
% 			(-1,-1.6) coordinate (B)
% 			(2.5,-1.6) coordinate (C)
% 			($(A)+(C)-(B)$) coordinate (D)
% 			($(A)+(0,3.5)$) coordinate (S)
% 			($(C)!0.5!(D)$) coordinate (M)			
% 			;
% 			\draw (S)--(B)--(C)--(D)--cycle (S)--(C);
% 			\draw[dashed] (S)--(A)--(D) (A)--(B)--(M);	
% 			\foreach \p/\q in {S/90,A/180,B/-90,C/-90,D/0,M/-30}				
% 			\fill[black] (\p) circle (1.0pt) node[shift={(\q:3mm)}]{$\p$};	
% 			\draw pic[draw=black,angle radius=0.2cm] {right angle = S--A--D}; 		
% 		\end{tikzpicture}
% 	}
% 	\loigiai{
% 		\immini{
% 			Gọi $N$ là trung điểm của $AB$, và gọi $K$, $H$ lần lượt là hình chiếu của $A$ trên $DN$, $SK$.\\
% 			Ta có $\heva{& DN\perp AK \\ & DN\perp SA}\Rightarrow DN\perp (SAK)$.\\
% 			Do đó $\heva{& AK\perp DN \\ & AK\perp SK}\Rightarrow AH\perp (SDN)$.\\
% 			Ta lại có $BM\parallel DN\Rightarrow BM\parallel (SDN)$, suy ra 			
% 		}{
% 			\begin{tikzpicture}[scale=1, font=\footnotesize, line join=round, line cap=round,>=stealth]
% 				\path
% 				(0,0) coordinate (A)
% 				(-1,-1.6) coordinate (B)
% 				(2.5,-1.6) coordinate (C)
% 				($(A)+(C)-(B)$) coordinate (D)
% 				($(A)+(0,3.5)$) coordinate (S)
% 				($(C)!0.5!(D)$) coordinate (M)	
% 				($(A)!0.5!(B)$) coordinate (N)	
% 				($(C)!0.5!(B)$) coordinate (P)	
% 				(intersection of A--P and N--D) coordinate (K)
% 				($(S)!0.6!(K)$) coordinate (H)
% 				;
% 				\draw (S)--(B)--(C)--(D)--cycle (S)--(C);
% 				\draw[dashed] (S)--(A)--(D)--(N)--cycle (H)--(A)--(B)--(M) (A)--(P) (S)--(K);	
% 				\foreach \p/\q in {S/90,A/170,B/-90,C/-90,D/0,M/-30,N/-60,K/-120,H/0}				
% 				\fill[black] (\p) circle (1.0pt) node[shift={(\q:3mm)}]{$\p$};			
% 			\end{tikzpicture}
% 		}\noindent
% 		\[\mathrm{d}(BM,SD)=\mathrm{d}(BM,(SDN))=\mathrm{d}(B,(SDN))=\mathrm{d}(A,(SDN))=AH.\]
% 		Ta lại có $AS$, $AN$, $AD$ đôi một vuông góc nên 
% 		\[\dfrac{1}{AH^2}=\dfrac{1}{AS^2}+\dfrac{1}{AN^2}+\dfrac{1}{AD^2}=\dfrac{1}{a^2}+\dfrac{1}{\left(\dfrac{a}{2}\right)^2}+\dfrac{1}{(2a)^2}\Rightarrow AH=\dfrac{2a\sqrt{21}}{21}.\] 
% 		Vậy $\mathrm{d}\left(BM,SD\right)=\dfrac{2a\sqrt{21}}{21}$.	
% 	}
% \end{ex}

% \begin{ex}%[Tex hoá SGK KNTT]%[Phạm Văn Long]%[1K7BP-4]
% 	\immini[thm]
% 	{
% 		Cho hình chóp $S.ABC$ có đáy là tam giác vuông tại $A$, $AB=2a$, $AC=4a$, $SA$ vuông góc với mặt đáy và $SA=a$ (minh họa như hình bên dưới). Gọi $M$ là trung điểm của $AB$. Khoảng cách giữa hai đường thẳng $SM$ và $BC$ bằng
% 		\choice
% 		{\True$\dfrac{2a}{3}$}
% 		{$\dfrac{\sqrt{6}a}{3}$}
% 		{$\dfrac{\sqrt{3}a}{3}$}
% 		{$\dfrac{a}{2}$}
% 	}
% 	{
% 		\begin{tikzpicture}[scale=0.8, font=\footnotesize, line join=round, line cap=round, >=stealth]
% 			\def\ac{4} % cạnh AC
% 			\def\ab{2} % cạnh AB
% 			\def\h{3} % chiều cao
% 			\def\gocA{50} % góc A của đáy
% 			\coordinate[label=left:$A$] (A) at (0,0);
% 			\coordinate[label=right:$C$] (C) at (\ac,0);
% 			\coordinate[label=below left:$B$] (B) at (-\gocA:\ab);
% 			\coordinate[label=above:$S$] (S) at ($(A)+(90:\h)$);
% 			\coordinate[label=below left:$M$] (M) at ($(A)!0.5!(B)$);
% 			\draw (A)--(B)--(C)--(S)--cycle (S)--(B);
% 			\draw[dashed] (A)--(C) (S)--(M);
% 			\foreach \diem in {A,B,C,S,M}	\fill (\diem)circle(1.5pt);
% 			\newcommand{\gocv}[4][black]{\draw[#1] ($(#3)!5pt!(#2)$)--($(#3)!2!($($(#3)!5pt!(#2)$)!.5!($(#3)!5pt!(#4)$)$)$)--($(#3)!5pt!(#4)$);}
% 			\gocv{S}{A}{C};
% 		\end{tikzpicture}
% 	}
	
% 	\loigiai{	
% 		\immini{Gọi $N$ là trung điểm của $AC$. \\
% 			Theo tính chất đường trung bình trong tam giác $ABC$, ta có $MN\parallel BC$ do đó $BC\parallel (SMN)$.\\
% 			Như vậy
% 			$$\mathrm{d}(SM,BC)=\mathrm{d}\left(BC,(SMN)\right)=\mathrm{d}\left(B,(SMN)\right)=\mathrm{d}\left(A,(SMN)\right).$$
% 			Tứ diện $AMNS$ có ba cạnh $AS,\,AM,\,AN$ đôi một vuông góc với nhau nên 
% 			$$\dfrac{1}{\mathrm{d}^2\left(A,(SMN)\right)}=\dfrac{1}{AS^2}+\dfrac{1}{AM^2}+\dfrac{1}{AN^2}=\dfrac{1}{a^2}+\dfrac{1}{a^2}+\dfrac{1}{4a^2}\Rightarrow \mathrm{d}\left(A,(SMN)\right)=\dfrac{2a}{3}.$$}
% 		{\begin{tikzpicture}[scale=1, font=\footnotesize, line join=round, line cap=round, >=stealth]
% 				\def\ac{4} % cạnh AC
% 				\def\ab{2} % cạnh AB
% 				\def\h{3} % chiều cao
% 				\def\gocA{50} % góc A của đáy
% 				\coordinate[label=left:$A$] (A) at (0,0);
% 				\coordinate[label=right:$C$] (C) at (\ac,0);
% 				\coordinate[label=below left:$B$] (B) at (-\gocA:\ab);
% 				\coordinate[label=above:$S$] (S) at ($(A)+(90:\h)$);
% 				\coordinate[label=below left:$M$] (M) at ($(A)!0.5!(B)$);
% 				\coordinate[label=below :$N$] (N) at ($(A)!0.5!(C)$);
% 				\draw (A)--(B)--(C)--(S)--cycle (S)--(B);
% 				\draw[dashed] (A)--(C) (S)--(M)--(N)--cycle;
% 				\foreach \diem in {A,B,C,S,M,N}	\fill (\diem)circle(1.5pt);
% 				\newcommand{\gocv}[4][black]{\draw[#1] ($(#3)!5pt!(#2)$)--($(#3)!2!($($(#3)!5pt!(#2)$)!.5!($(#3)!5pt!(#4)$)$)$)--($(#3)!5pt!(#4)$);}
% 				\gocv{S}{A}{C}
% 		\end{tikzpicture}}
% 	}
	
% \end{ex}
% \begin{ex}%[Tex hoá SGK KNTT]%[Phạm Văn Long]%[1K7BP-4]
% 	\immini{ Cho hình chóp $S.ABCD$ có đáy là hình thang, $AB=2a$, $AD=DC=CB=a$, $SA$ vuông góc với mặt đáy và $SA=3a$ (minh họa như hình bên dưới). Gọi $M$ là trung điểm của $AB$. Khoảng cách giữa hai đường thẳng $SB$ và $DM$ bằng	
% 		\choice
% 		{\True$\dfrac{3a}{4}$}
% 		{$\dfrac{3a}{2}$}
% 		{$\dfrac{3\sqrt{13}a}{13}$}
% 		{$\dfrac{6\sqrt{13}a}{13}$}}
% 	{\begin{tikzpicture}[scale=.8, font=\footnotesize, line join=round, line cap=round, >=stealth]
% 			\def\ad{4} % cạnh AD
% 			\def\ab{2} % cạnh AB
% 			\def\bc{2} % chéo AC
% 			\def\as{3} % cạnh AS
% 			\def\gocA{60} % góc A của đáy
% 			\def\gocB{120} % góc B của đáy
% 			\coordinate[label=left:$A$] (A) at (0,0);
% 			\coordinate[label=below left:$D$] (D) at (-\gocA:\ab);
% 			\coordinate[label=below right:$C$] (C) at ($(D)+(180-\gocA-\gocB:\bc)$);
% 			\coordinate[label=right:$B$] (B) at (\ad,0);
% 			\coordinate[label=above:$S$] (S) at (90:\as); % chỉnh 75 và as để thay đổi S
% 			\coordinate[label=above right:$M$] (M) at ($(A)!0.5!(B)$);
% 			%	\coordinate[label=above left:$N$] (N) at ($(A)!0.5!(S)$);
% 			%	\coordinate[label=right:$P$] (P) at ($(D)!0.5!(M)$);
% 			%	\coordinate[label=left:$H$] (H) at ($(S)!2/3!(P)$);
% 			\draw (A)--(D)--(C)--(B)--(S)--cycle (D)--(S)--(C);
% 			\draw[dashed] (A)--(B) (D)--(M) (S)--(A)--cycle ;
% 			\foreach \diem in {A,B,C,D,S,M}	\fill (\diem)circle(1.5pt);
% 	\end{tikzpicture}}
% 	\loigiai{
% 		\immini{Gọi $N$, $P$ lần lượt là trung điểm của $SA$ và $DM$. \\
% 			Do $MN$ song song với $SB$ nên $SB$ song song với $(DMN)$.\\
% 			Do đó $\mathrm{d}(SB,DM)=\mathrm{d}\left(SB,(DMN)\right)=\mathrm{d}\left(B,(DMN)\right)=\mathrm{d}\left(A,(DMN)\right).$\\
% 			Từ giả thiết bài toán, ta nhận thấy $\triangle ADM$ là tam giác đều nên $AP\perp DM$ , hơn nữa $SA\perp DM$ nên $DM\perp(SAP)$.\\
% 			Gọi $H$ là chân đường cao kẻ từ $A$ trong tam giác $SNP$. Ta có
% 			$$\heva{&AH\perp NP\\&AH\perp DM\,(\text{do }DM\perp(SAP))}\Rightarrow AH\perp (DMN).$$
% 			Ngoài ra $\dfrac{1}{AH^2}=\dfrac{1}{AM^2}+\dfrac{1}{AP^2}\Rightarrow AH=\dfrac{3a}{4}$.\\
% 			Như vậy $\mathrm{d}(SB,DM)=\mathrm{d}\left(A,(DMN)\right)=AH=\dfrac{3a}{4}$.}
% 		{\begin{tikzpicture}[scale=1, font=\footnotesize, line join=round, line cap=round, >=stealth]
% 				\def\ad{4} % cạnh AD
% 				\def\ab{2} % cạnh AB
% 				\def\bc{2} % chéo AC
% 				\def\as{3} % cạnh AS
% 				\def\gocA{60} % góc A của đáy
% 				\def\gocB{120} % góc B của đáy
% 				\coordinate[label=left:$A$] (A) at (0,0);
% 				\coordinate[label=below left:$D$] (D) at (-\gocA:\ab);
% 				\coordinate[label=below right:$C$] (C) at ($(D)+(180-\gocA-\gocB:\bc)$);
% 				\coordinate[label=right:$B$] (B) at (\ad,0);
% 				\coordinate[label=above:$S$] (S) at (90:\as); % chỉnh 75 và as để thay đổi S
% 				\coordinate[label=above right:$M$] (M) at ($(A)!0.5!(B)$);
% 				\coordinate[label=above left:$N$] (N) at ($(A)!0.5!(S)$);
% 				\coordinate[label=right:$P$] (P) at ($(D)!0.5!(M)$);
% 				\coordinate[label=right:$H$] (H) at ($(N)!0.45!(P)$);
% 				\draw pic[draw,angle radius=1mm] {right angle = A--H--P};
% 				\draw (A)--(D)--(C)--(B)--(S)--cycle (N)--(D)--(S)--(C);
% 				\draw[dashed] (H)--(A)--(B) (D)--(M)--(N) (S)--(A)--(P) (N)--(P);
% 				\foreach \diem in {A,B,C,D,S,M,N,P,H}	\fill (\diem)circle(1.2pt);
% 		\end{tikzpicture}}
% 	}
% \end{ex}
% \begin{ex}%[Tex hoá SGK KNTT]%[Phạm Văn Long]%[1K7BP-4]
% 	\immini{
% 		Cho hình chóp $S.ABCD$ có đáy là hình chữ nhật $AB=a$, $BC=2a$, $SA$ vuông góc với mặt phẳng đáy và $SA=a$. Khoảng cách giữa hai đường thẳng $BD$ và $SC$ bằng	
% 		\choice
% 		{$\dfrac{a\sqrt{30}}{6}$}
% 		{$\dfrac{4a\sqrt{21}}{21}$}
% 		{\True $\dfrac{2a\sqrt{21}}{21}$}
% 		{$\dfrac{a\sqrt{30}}{12}$}
% 	}{
% 		\begin{tikzpicture}[scale=0.7,font=\footnotesize,line join=round,line cap=round,>=stealth]
% 			\coordinate (B) at (0,0);
% 			\coordinate (A) at (2,2);
% 			\coordinate (C) at (6,0);
% 			\coordinate (D) at ($(A)+(C)-(B)$);
% 			\coordinate (S) at ($(A)+(0,4)$);
% 			\draw (S) node[above]{$S$}--(C) node[below]{$C$}--(D) node[right]{$D$}--(S)--(B) node[below]{$B$}--(C);
% 			\draw[dashed] (C)--(A) node[below]{$A$}--(S) (D)--(A)--(B);
% 		\end{tikzpicture}
% 	}
% 	\loigiai{
% 		\immini{
% 			Gọi $E$ là trung điểm $SA$, ta có $SC\parallel OE$\\
% 			$\Rightarrow \mathrm{d}(SC,BD)=\mathrm{d}(SC,(EBD))$.\\
% 			Lại có $E$ là trung điểm $SA$ nên $\mathrm{d}(S,(EBD))=\mathrm{d}(A,(EBD))$.\\
% 			Kẻ $AH\perp BD$, kẻ $AK\perp HE$.\\
% 			Có $\heva{&BD\perp AH\\& BD\perp SA}\Rightarrow BD\perp (SAH)\Rightarrow BD\perp AK$.\\
% 			Có $\heva{& AK\perp HE\\&AK\perp BD}\Rightarrow AK\perp (EBD)$.\\ Vậy $AK$ là khoảng cách từ $A$ đến $(EBD)$.
% 		}{
% 			\begin{tikzpicture}[scale=0.7,font=\footnotesize,line join=round,line cap=round,>=stealth]
% 				\coordinate (B) at (0,0);
% 				\coordinate (A) at (2,2);
% 				\coordinate (C) at (6,0);
% 				\coordinate (D) at ($(A)+(C)-(B)$);
% 				\coordinate (S) at ($(A)+(0,4)$);
% 				\draw (S) node[above]{$S$}--(C) node[below]{$C$}--(D) node[right]{$D$}--(S)--(B) node[below]{$B$}--(C);
% 				\draw[dashed] (C)--(A)--(S) (D)--(A)--(B);
% 				\coordinate (E) at ($(S)!.5!(A)$);
% 				\coordinate (O) at ($(A)!.5!(C)$);
% 				\coordinate (H) at ($(B)!.65!(O)$);
% 				\coordinate (K) at ($(E)!.45!(H)$);
% 				\draw[dashed] (A)--(H)--(E) (A)--(K) (B)--(D) (E)--(D) (E)--(O) (B)--(E);
% 				\foreach \p/\g in {A/180,E/145,O/-90,H/-80,K/45} \draw[fill] (\p) circle(.5pt)node [shift={(\g:.3)}] {$\p$};
% 				\draw ($ (K)!5pt!(A)$)--($(K)!2!($($(K)!5pt!(A)$)!.5!($(K)!5pt!(E)$)$)$)--($(K)!5pt!(E)$);
% 				\draw ($ (H)!5pt!(B)$)--($(H)!2!($($(H)!5pt!(B)$)!.5!($(H)!5pt!(A)$)$)$)--($(H)!5pt!(A)$);
% 			\end{tikzpicture}
% 		}
% 		Có $\dfrac{1}{AK^2}=\dfrac{1}{AE^2}+\dfrac{1}{AH^2}=\dfrac{1}{AE^2}+\dfrac{1}{AB^2}+\dfrac{1}{AD^2}=\dfrac{21}{4a^2}\Rightarrow AK=\dfrac{2a\sqrt{21}}{21}$.\\
% 		Vậy khoảng cách giữa $BD$ và $SC$ bằng $\dfrac{2a\sqrt{21}}{21}$. 	
% 	}
% \end{ex}
% \begin{ex}%[Tex hoá SGK KNTT]%[Phạm Văn Long]%[1K7BP-4]
% 	Cho hình lập phương $MNPQ.M'N'P'Q'$ có cạnh bằng $a$. Khoảng cách từ điểm $M$ đến mặt phẳng $(NQQ'N')$ bằng
% 	\choice
% 	{$a$}
% 	{\True $\dfrac{a}{\sqrt{2}}$}
% 	{$a\sqrt{2}$}
% 	{$\dfrac{a}{2}$}
% 	\loigiai{
% 		\immini{
% 			Ta có $\heva{&MP\perp NQ\\&MP\perp NN'}\Rightarrow MP\perp (NQQ'N')$.\\
% 			Trong mặt phẳng $(MNPQ)$ gọi $O=MP\cap NQ$.\\
% 			Khi đó $\mathrm{d}(M,(NQQ'N'))=MO=\dfrac{MP}{2}=\dfrac{a\sqrt{2}}{2}=\dfrac{a}{\sqrt{2}}$.
% 		}{
% 			\begin{tikzpicture}[scale=1, font=\footnotesize, line join=round, line cap=round, >=stealth]
% 				\path
% 				(0,0) coordinate (M)
% 				(-1,-1) coordinate (N)
% 				(3,0) coordinate (Q)
% 				($(N)+(Q)-(M)$) coordinate (P)
% 				($(M)+(0,2.5)$) coordinate (M')
% 				($(M')-(M)+(N)$) coordinate (N')
% 				($(M')-(M)+(P)$) coordinate (P')
% 				($(M')-(M)+(Q)$) coordinate (Q')
% 				(intersection of M--P and N--Q) coordinate (O)
% 				;
% 				\draw (M')--(N')--(P')--(Q')--(M') (N)--(N') (P)--(P') (Q)--(Q') (N)--(P)--(Q) (N')--(Q');
% 				\draw[dashed] (M')--(M)--(N) (M)--(Q) (P)--(M) (N)--(Q);
% 				\foreach \x/\g in {M/180,N/-90,P/-90,Q/0,M'/90,N'/180,P'/90,Q'/0,O/-90} \draw[fill=black] (\x) circle(1pt)++(\g:0.3)node{$\x$};
% 				\draw pic[draw=black, angle eccentricity=2, angle radius=0.25cm]{right angle=P--O--Q};
% 			\end{tikzpicture}
% 		}
% 	}
% \end{ex}
% \begin{ex}%[Tex hoá SGK KNTT]%[Phạm Văn Long]%[1K7BP-4]
% 	Cho hình hộp chữ nhật $MNPQ.M'N'P'Q'$ có $MN=2a$, $MQ=3a$, $MM'=4a$. Khoảng cách giữa hai đường thẳng $NP$ và $M'N'$ bằng
% 	\choice
% 	{$2a$}
% 	{$3a$}
% 	{\True $4a$}
% 	{$5a$}
% 	\loigiai{
% 		\immini{
% 			Ta có $\heva{&NP\perp NN'\\&M'N'\perp NN'.}$\\
% 			Do đó $NN'$ là đoạn vuông góc chung của $NP$ và $M'N'$.\\
% 			Vậy $\mathrm{d}(NP,M'N')=NN'=MM'=4a$.
% 		}{
% 			\begin{tikzpicture}[scale=1, font=\footnotesize, line join=round, line cap=round, >=stealth]
% 				\path
% 				(0,0) coordinate (M)
% 				(-1,-1) coordinate (N)
% 				(3,0) coordinate (Q)
% 				($(N)+(Q)-(M)$) coordinate (P)
% 				($(M)+(0,2.5)$) coordinate (M')
% 				($(M')-(M)+(N)$) coordinate (N')
% 				($(M')-(M)+(P)$) coordinate (P')
% 				($(M')-(M)+(Q)$) coordinate (Q')
% 				(intersection of M--P and N--Q) coordinate (O)
% 				;
% 				\draw (M')--(N')--(P')--(Q')--(M') (N)--(N') (P)--(P') (Q)--(Q') (N)--(P)--(Q);
% 				\draw[dashed] (M')--(M)--(N) (M)--(Q);
% 				\foreach \x/\g in {M/180,N/-90,P/-90,Q/0,M'/90,N'/180,P'/90,Q'/0} \draw[fill=black] (\x) circle(1pt)++(\g:0.3)node{$\x$};
% 			\end{tikzpicture}
% 		}
% 	}
% \end{ex}
% \begin{ex}%[Tex hoá SGK KNTT]%[Phạm Văn Long]%[1K7BP-4]
% 	\immini{Cho hình lập phương $ABCD.A'B'C'D'$ có cạnh bằng $a$. Khoảng cách giữa hai đường thẳng $BD$ và $A'C'$ bằng
% 		\choice
% 		{$\sqrt{3}a$}
% 		{\True $a$}
% 		{$\dfrac{\sqrt{3}a}{2}$}
% 		{$\sqrt{2a}$}}
% 	{	\begin{tikzpicture}[scale=0.5,font=\footnotesize,line join=round,line cap=round,>=stealth]
% 			\def\d{7} \def\r{3} \def\c{6}
% 			\path
% 			(0,0) coordinate (A)
% 			(\d,0) coordinate (B)
% 			(-2,-\r) coordinate (D)
% 			($(B)!.5!(D)$) coordinate (O)
% 			($(A)!2!(O)$) coordinate (C)
% 			($(A)+(0,\c)$) coordinate (A')
% 			($(B)+(0,\c)$) coordinate (B')
% 			($(C)+(0,\c)$) coordinate (C')
% 			($(D)+(0,\c)$) coordinate (D')
% 			($(O)+(0,\c)$) coordinate (O')
% 			;
% 			\draw (A')--(B')--(C')--(D')--(D)--(C)--(B)--(B')
% 			%--(D')
% 			(C)--(C') (A')--(D') 
% 			;
% 			\draw[dashed] (A')--(A)--(B)--(D)--(A)
% 			% --(C)
% 			% (O)--(O')
% 			;
% 			\foreach \i/\j in {A/160,B/0,C/-45,D/200,A'/90,B'/45,C'/-30,D'/135} \fill[black] (\i)circle(1pt) ($(\i)+(\j:3mm)$)node{$\i$};
% 	\end{tikzpicture}}
% 	\loigiai{
% 		\immini{
% 			Gọi $O$ và $O'$ lần lượt là tâm của $ABCD$ và $A'B'C'D'$.
% 			\\
% 			Do $ABCD.A'B'C'D'$ là hình lập phương nên ta có $OO'$ là đường cao.
% 			\\
% 			Nên $OO' \perp (ABCD)$ và $(A'B'C'D')$.
% 			\\
% 			Do đó $\heva{&OO' \perp BD \\ &OO' \perp A'C'} \Rightarrow OO'$ là đoạn vuông góc chung của $BD$ và $A'C'$. 
% 			\\
% 			Vậy $d(A'C', BD) = OO' = a$.
% 		}{
% 			\begin{tikzpicture}[scale=0.5,font=\footnotesize,line join=round,line cap=round,>=stealth]
% 				\def\d{7} \def\r{3} \def\c{6}
% 				\path
% 				(0,0) coordinate (A)
% 				(\d,0) coordinate (B)
% 				(-2,-\r) coordinate (D)
% 				($(B)!.5!(D)$) coordinate (O)
% 				($(A)!2!(O)$) coordinate (C)
% 				($(A)+(0,\c)$) coordinate (A')
% 				($(B)+(0,\c)$) coordinate (B')
% 				($(C)+(0,\c)$) coordinate (C')
% 				($(D)+(0,\c)$) coordinate (D')
% 				($(O)+(0,\c)$) coordinate (O')
% 				;
% 				\draw (A')--(B')--(C')--(D')--(D)--(C)--(B)--(B')--(D')
% 				(C)--(C')--(A')--(D') 
% 				;
% 				\draw[dashed] (A')--(A)--(B)--(D)--(A)--(C)
% 				(O)--(O')
% 				;
% 				\foreach \i/\j in {A/160,B/0,C/-45,D/200,A'/90,B'/45,C'/-30,D'/135,O/-90,O'/90} \fill[black] (\i)circle(1pt) ($(\i)+(\j:3mm)$)node{$\i$};
% 			\end{tikzpicture}
% 		}
% 	}
% \end{ex}

% \begin{ex}%[Tex hoá SGK KNTT]%[Phạm Văn Long]%[1K7BP-4]
% 	\immini{Cho hình lăng trụ đứng $ABC.A'B'C'$ có đáy $ABC$ là tam giác vuông $BA=BC=a$, cạnh bên $AA'=a \sqrt{2}, M$ là trung điểm $BC$. Khoảng cách giữa hai đường thẳng $AM$ và $B'C$ bằng
% 		\choice
% 		{\True $\dfrac{a \sqrt{7}}{7}$}
% 		{$\dfrac{a \sqrt{2}}{2}$}
% 		{$\dfrac{a \sqrt{5}}{5}$}
% 		{$\dfrac{a \sqrt{3}}{3}$}}
% 	{	\begin{tikzpicture}[scale=0.8]
% 			\def\d{6} \def\r{2} \def\c{5}
% 			\path
% 			(0,0) coordinate (A)
% 			(\d,0) coordinate (B)
% 			(1,-\r) coordinate (C)
% 			($(A)+(0,\c)$) coordinate (A')
% 			($(B)+(0,\c)$) coordinate (B')
% 			($(C)+(0,\c)$) coordinate (C')
% 			($(B)!.5!(C)$) coordinate (M)
% 			% ($(B)!.5!(B')$) coordinate (N)
% 			% ($(A)!.7!(M)$) coordinate (K)
% 			% ($(K)!.7!(N)$) coordinate (H)
% 			;
% 			\draw (C)--(A)--(A')--(B')--(C')--(C)--(B)--(B')--(C)
% 			(A')--(C')
% 			% (M)--(N)--cycle
% 			;
% 			\draw[dashed] (M)--(A)--(B)
% 			% (A)--(N)
% 			% (B)--(K)--(N)
% 			% (B)--(H)
% 			;
% 			\foreach \i/\j in {A/160,B/0,C/-45,A'/90,B'/45,C'/-20,M/-30} \fill[black] (\i)circle(1pt) ($(\i)+(\j:3mm)$)node{$\i$};
% 	\end{tikzpicture}}
% 	\loigiai{
% 		\immini{
% 			Gọi $N$ là trung điểm của $BB'$. Suy ra $B'C \parallel (AMN)$. Do đó 
% 			$$d(AM, B'C) = d(B'C, (AMN)) = d(C, (AMN)) = d(B, (AMN)).$$
% 			Kẻ $BK \perp AM$, $BH \perp NK$. Ta có
% 			$$\heva{&AM \perp BK \\ &AM \perp NB} \Rightarrow AM \perp (BNK) \Rightarrow BH \perp AM$$
% 			Vậy $BH \perp (AMN) \Rightarrow BH = d(B, (AMN))$.
% 			\\
% 			Áp dụng hệ thức lượng trong tam giác vuông $AMB$, ta được:
% 			$$\dfrac{1}{BK^2}=\dfrac{1}{BM^2}+\dfrac{1}{AB^2}=\dfrac{5}{a^2}.$$
% 			Áp dụng hệ thức lượng trong tam giác vuông $BKN$, ta được:
% 			$$\dfrac{1}{BH^2}=\dfrac{1}{BK^2}+\dfrac{1}{BN^2}=\dfrac{7}{a^2} \Rightarrow BH = \dfrac{a\sqrt{7}}{7}.$$
% 			Vậy $d(AM, B'C) = \dfrac{a\sqrt{7}}{7}$.
% 		}{
% 			\begin{tikzpicture}[scale=0.8]
% 				\def\d{6} \def\r{2} \def\c{5}
% 				\path
% 				(0,0) coordinate (A)
% 				(\d,0) coordinate (B)
% 				(1,-\r) coordinate (C)
% 				($(A)+(0,\c)$) coordinate (A')
% 				($(B)+(0,\c)$) coordinate (B')
% 				($(C)+(0,\c)$) coordinate (C')
% 				($(B)!.5!(C)$) coordinate (M)
% 				($(B)!.5!(B')$) coordinate (N)
% 				($(A)!.7!(M)$) coordinate (K)
% 				($(K)!.7!(N)$) coordinate (H)
% 				;
% 				\draw (C)--(A)--(A')--(B')--(C')--(C)--(B)--(B')--(C)--(A')--(C')
% 				(M)--(N)--cycle
% 				;
% 				\draw[dashed] (M)--(A)--(B)--(K)--(N)
% 				(A)--(N)
% 				(B)--(H)
% 				;
% 				\foreach \i/\j in {A/160,B/0,C/-45,A'/90,B'/45,C'/-20,M/-30,N/0,H/100,K/-90} \fill[black] (\i)circle(1pt) ($(\i)+(\j:3mm)$)node{$\i$};
% 			\end{tikzpicture}
% 		}
% 	}
% \end{ex}
% \begin{ex}%[Tex hoá SGK KNTT]%[Phạm Văn Long]%[1K7BP-4]
% 	\immini{
% 		Cho hình lăng trụ đều $ABC.A'B'C'$ có tất cả các cạnh bằng $a$. Gọi $M$ là trung điểm của cạnh $BC$ (tham khảo hình bên). Khoảng cách giữa hai đường thẳng $AM$ và $B'C$ là
% 		\choice
% 		{$a\sqrt{2}$}
% 		{$\dfrac{a\sqrt{2}}{2}$}
% 		{$\dfrac{1}{2} a$}
% 		{\True $\dfrac{a\sqrt{2}}{4}$}
% 	}{
% 		\begin{tikzpicture}[scale=1, font=\footnotesize, line join=round, line cap=round,>=stealth]
% 			\path
% 			(0,0)coordinate (A)
% 			(1.1,-1.4)coordinate (B)
% 			(4,0)coordinate (C)
% 			($(A)+(0,3.2)$) coordinate (A')
% 			($(A')+(B)-(A)$)coordinate (B')
% 			($(A')+(C)-(A)$)coordinate (C')	
% 			($(B)!0.5!(C)$) coordinate (M)					
% 			;
% 			\draw (A)--(B)--(C)--(C')--(B')--(A')--cycle (A')--(C') (B)--(B')--(C);
% 			\draw[dashed] (M)--(A)--(C);	
% 			\foreach \p/\q in {A/180,B/-90,C/0,A'/180,B'/73,C'/0,M/-40}
% 			\fill[black] (\p) circle (1.0pt)node[shift={(\q:3mm)}]{$\p$};
% 		\end{tikzpicture}
% 	}
% 	\loigiai{
% 		\immini{
% 			Gọi $N$ là trung điểm của $BB'$.\\
% 			Ta có $\heva{& AM\perp BC \\ & AM\perp BB'}\Rightarrow AM\perp (BCC'B')$.\\
% 			$\heva{& BH\perp AM \\ & BH\perp MN}\Rightarrow BH\perp (AMN)$.\\
% 			Ta lại có $MN$ là đường trung bình của $\triangle BB'C\Rightarrow MN\parallel B'C$.\\
% 			Mà $MN\subset(AMN)\Rightarrow B'C\parallel (AMN)$.\\			
% 			Do đó 
% 		}{
% 			\begin{tikzpicture}[scale=1, font=\footnotesize, line join=round, line cap=round,>=stealth]
% 				\path
% 				(0,0)coordinate (A)
% 				(1.1,-1.4)coordinate (B)
% 				(4,0)coordinate (C)
% 				($(A)+(0,3.3)$) coordinate (A')
% 				($(A')+(B)-(A)$)coordinate (B')
% 				($(A')+(C)-(A)$)coordinate (C')	
% 				($(B)!0.5!(C)$) coordinate (M)
% 				($(B)!0.5!(B')$) coordinate (N)	
% 				($(N)!0.5!(M)$) coordinate (H)			
% 				;
% 				\draw (A)--(B)--(C)--(C')--(B')--(A')--cycle (A')--(C') (H)--(B)--(B')--(C)
% 				(A)--(N)--(M);
% 				\draw[dashed] (M)--(A)--(C);	
% 				\foreach \p/\q in {A/180,B/-90,C/0,A'/180,B'/73,C'/0,M/-40,N/30,H/20}
% 				\fill[black] (\p) circle (1.0pt)node[shift={(\q:3mm)}]{$\p$};
% 				\draw pic[draw=black,angle radius=0.2cm] {right angle = B--H--N};
% 			\end{tikzpicture}
% 		}\noindent 
% 		\[\mathrm{d}(B'C,AM)=\mathrm{d}(B'C,(AMN))=\mathrm{d}(B',(AMN))=\mathrm{d}(B,(AMN))=BH.\]
% 		Trong tam giác $BMN$ vuông cân tại $B$, ta có $BH=\dfrac{MN}{2}=\dfrac{a\sqrt{2}}{4}$.\\
% 		Vậy $\mathrm{d}(B'C,AM)=\dfrac{a\sqrt{2}}{4}$.
% 	}
% \end{ex}
% \Closesolutionfile{ans}
% \begin{indapan}{10}
% 	{ans/ans-1K7-26-Dang5}
% \end{indapan}