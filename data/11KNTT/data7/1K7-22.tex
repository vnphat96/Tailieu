\setcounter{chapter}{6}% Reset lại số đếm chương
\setcounter{dang}{0}
\setcounter{section}{21}
\chap{Quang hệ vuông góc trong không gian}
\section{Hai đường thẳng vuông góc}
\subsection{Tóm tắt lý thuyết}
\begin{tomtat}
	\subsubsection{Góc giữa hai đường thẳng} 
	\begin{itemize}
		\item Góc giữa hai đường thẳng $m$ và $n$ trong không gian, kí hiệu $(m,n)$, là góc giữa hai đường thẳng $a$ và $b$ cùng đi qua một điểm và tương ứng song song với $m$ và $n$.
		\item Để xác định góc giữa hai đường thẳng chéo nhau $a$ và $b$, ta có thể lấy một điểm $O$ thuộc đường thẳng $a$ và qua đó kẻ đường thẳng $b'$ song song với $b$. Khi đó $(a,b)=(a,b')$.
		\item Với hai đường thẳng $a$, $b$ bất kỳ ta luôn có:  $0^\circ \le (a,b) \le 90^\circ$.
		\item Nếu đường thẳng $a$ song song hoặc trùng với đường thẳng $a'$, đường thẳng $b$ song song hoặc trùng với đường thẳng $b'$ thì $(a,b)=(a',b')$.
	\end{itemize}
	\subsubsection{Hai đường thẳng vuông góc} 
	\begin{itemize}
		\item Hai đường thẳng $a$, $b$ được gọi là vuông góc với nhau, kí hiệu $a \perp b$, nếu góc giữa chúng bằng $90^ \circ$.
		\item Nếu đường thẳng $a$ vuông góc với đường thẳng $b$ thì đường thẳng $a$ cũng vuông góc với mọi đường thẳng song song với đường thẳng $b$.
		\item Hai đường thẳng vuông góc có thể cắt nhau hoặc chéo nhau.
		\item Trong không gian, khi có hai đường thẳng phân biệt $a,b$ cùng vuông góc với đường thẳng thứ ba $c$ thì ta chưa kết luận được $a \parallel b$.
	\end{itemize}
\end{tomtat}
\subsection{Các dạng toán thường gặp}
\begin{dang}{Câu hỏi lý thuyết}
	\begin{itemize}
		\item Sử dụng tính chất và quan hệ vuông góc trong hình học phẳng.
		\item Vận dụng kiến thức khái niệm góc giữa hai đường thẳng trong không gian áp dụng vào giải toán.
		\item Vận dụng kiến thức khái niệm hai đường thẳng vuông góc trong không gian và các kiến thức cũ liên quan (nếu có) áp dụng vào giải toán.
	\end{itemize}
\end{dang}

\subsection{Bài tập trắc nghiệm}
\Opensolutionfile{ans}[ans/ans-1K7-1-1]
\begin{ex}%[Lương Như Quỳnh]%[0D1Y1-1]
	Trong không gian cho ba đường thẳng phân biệt $a,b,c$. Khẳng định nào sau đây đúng?
	\choice
	{Nếu $a$ và $b$ cùng vuông góc với $c$ thì $a \parallel b$}
	{\True Nếu $a \parallel b$ và $c \perp a$ thì $c \perp b$}
	{Nếu góc giữa $a$ và $c$ bằng góc giữa $b$ và $c$ thì $a \parallel b$}
	{Nếu $a$ và $b$ cùng nằm trong mặt phẳng $(\alpha) \parallel c$ thì góc giữa $a$ và $c$ bằng góc giữa $b$ và $c$}
	\loigiai{
		\begin{itemize}
			\item Xét phương án A: Nếu $a$ và $b$ cùng vuông góc với $c$ thì $a$ và $b$ hoặc song song hoặc chéo nhau, nên phương án A sai.
			
			\item Xét phương án B: Nếu một đường thẳng vuông góc với một trong hai đường thẳng song song thì vuông góc với đường còn lại, nên B đúng.
			\item Xét phương án C: Giả sử hai đường thẳng $a$ và $b$ chéo nhau, ta dựng đường thẳng $c$ là đường vuông góc chung của $a$ và $b$. Khi đó góc giữa $a$ và $c$ bằng với góc giữa $b$ và $c$ và cùng bằng $90^ \circ$, nhưng hiển nhiên hai đường thẳng a và b không song song, nên C sai.
			\item Xét phương án D: Giả sử $a$ vuông góc với $c$; $b$ song song với $c$, khi đó góc giữa $a$ và $c$ bằng $90^ \circ$, còn góc giữa$b$ và $c$ bằng $0^ \circ$, nên D sai.
		\end{itemize}
	}
\end{ex}
\begin{ex}%[Lương Như Quỳnh]%[0D1Y1-1]
	Trong các mệnh đề sau, mệnh đề nào đúng?
	\choice
	{\True Góc giữa hai đường thẳng $a$ và $b$ bằng góc giữa hai đường thẳng $a$ và $c$ khi $b$ song song với $c$ (hoặc $b$ trùng với $c$)}
	{Góc giữa hai đường thẳng $a$ và $b$ bằng góc giữa hai đường thẳng $a$ và $c$ thì $b$ song song với $c$}
	{Góc giữa hai đường thẳng là góc nhọn}
	{Góc giữa hai đường thẳng bằng góc giữa hai véctơ chỉ phương của hai đường thẳng đó}
	\loigiai{
		\begin{itemize}
			\item Xét phương án A: Theo đúng khái niệm góc giữa hai đường thẳng, nên A đúng
			\item Xét phương án B: Xét hình lập phương $ABCD.A'B'C'D'$ ta có $(AB,AD)=(AB,AA')=90^ \circ$ nhưng $AB$ và $AA'$ lại vuông góc với nhau, nên B sai.
			\item Xét phương án C: $0^\circ \le (a,b) \le 90^\circ$, nên C sai.
			\item Xét phương án D: Góc giữa hai véctơ có thể là góc tù, nên D sai.			
		\end{itemize}
	}
\end{ex}

\begin{ex}
	Trong các mệnh đề dưới đây, mệnh đề đúng là
	\choice
	{\True Cho hai đường thẳng song song, đường thẳng nào vuông góc với đường thẳng thứ nhất thì cũng vuông góc với đường thẳng thứ hai}
	{Trong không gian, hai đường thẳng phân biệt cùng vuông góc với đường thẳng thứ ba thì song song với nhau}
	{Hai đường thẳng phân biệt vuông góc với nhau thì chúng cắt nhau}
	{Hai đường thẳng phân biệt cùng vuông góc với đường thẳng thứ ba thì vuông góc với nhau}
	\loigiai {Theo lý thuyết.}
\end{ex}

\begin{ex}
	Cho hai đường thẳng $a$ và $b$ vuông góc với nhau. Biết $a$ vuông góc với đường thẳng $c$. Tìm mệnh đề đúng?
	\choice
	{$b$ vuông góc với $c$}
	{$b \parallel c$}
	{Cả A và B đúng}
	{\True Tất cả sai}
	\loigiai {Theo lý thuyết.}
\end{ex}
\begin{ex}
	Hai đường thẳng cùng vuông góc với đường thẳng thứ ba thì
	\choice
	{Song song với nhau}
	{Vuông góc với nhau}
	{Chéo nhau}
	{\True Tất cả sai}
	\loigiai {Theo lý thuyết.}
\end{ex}
\begin{ex}
	Chọn mệnh đề \textbf{sai}?
	\choice
	{ Nếu $a \parallel b$ và $b \parallel c$ thì $a \parallel c$}
	{\True Nếu $a \perp b ; b \perp c$ thì $a \parallel c$}
	{Cho $a \parallel b$. Nếu $a$ vuông góc với $c$ thì $b$ vuông góc với $c$}
	{Hai đường thẳng vuông góc với nhau thì tích vô hướng của hai vecto chỉ phương của hai đường thẳng đó bằng $0$}
	\loigiai {Theo lý thuyết.}
\end{ex}
\begin{ex}
	Trong các mệnh đề sau đây, mệnh đề nào là đúng?
	\choice
	{Một đường thẳng cắt hai đường thẳng cho trước thì cả ba đường thẳng đó cùng nằm trong một mặt phẳng}
	{\True Ba đường thẳng cắt nhau từng đôi một và không nằm trong một mặt phẳng thì đồng quy}
	{Một đường thẳng cắt hai đường thẳng cắt nhau cho trước thì cả ba đường thẳng đó cùng nằm trong một mặt phẳng}
	{Ba đường thẳng cắt nhau từng đôi một thì cùng nằm trong một mặt phẳng}
	\loigiai { Gọi $a,b,c$ là ba đường thẳng cắt nhau từng đôi một.\\
		Giả sử $a,b$ cắt nhau tại $A$. Đường thẳng $c$ không cùng mặt phẳng với $a$ và $b$, mà $c$ cắt cả $a$ và $b$. Nên $c$ phải đi qua $A$.\\
		Thật vậy, giả sử $c$ không đi qua $A$ thì $c$ phải cắt $a,b$ lần lượt tại $B,C$.\\
		Suy ra đường thẳng $c$ cắt mặt phẳng $(a,b)$ tại hai điểm $B,C$. Điều này vô lý, vì một đường thẳng không thể cắt một mặt phẳng tại hai điểm phân biệt.
		
	}
\end{ex}
\begin{ex}
	Trong các mệnh đề sau đây, mệnh đề nào là đúng?
	\choice
	{Nếu đường thẳng $a$ vuông góc với đường thẳng $b$ và đường thẳng $b$ vuông góc với đường thẳng $c$ thì $a$ vuông góc với $c$}
	{Cho ba đường thẳng 
		$a,b,c$ vuông góc với nhau từng đôi một. Nếu có một đường thẳng $d$ vuông góc với $a$ thì $d$ song song với $b$ hoặc $c$}
	{\True Nếu đường thẳng $a$ vuông góc với đường thẳng $b$ và đường thẳng $b$ song song với đường thẳng $c$ thì $a$ vuông góc với $c$}
	{Cho hai đường thẳng $a$ và $b$ song song với nhau. Một đường thẳng $c$ vuông góc với $a$ thì $c$ vuông góc với mọi đường thẳng nằm trong mặt phẳng $(a,b)$}
	\loigiai {Theo lý thuyết.}
\end{ex}
\begin{ex}
	Trong các khẳng định sau, khẳng định nào đúng?
	\choice
	{Hai đường thẳng cùng vuông góc với đường thẳng thứ ba thì song song với nhau}
	{Nếu đường thẳng $a$ vuông góc với đường thẳng $b$ và đường thẳng $b$ vuông góc với đường thẳng $c$ thì $a$ vuông góc với $c$}
	{Cho hai đường thẳng phân biệt $a$ và $b$. Nếu đường thẳng $c$ vuông góc với $a$ và $b$ thì $a; b; c$ không đồng phẳng}
	{\True Cho hai đường thẳng $a$ và $b$ song song, nếu $a$ vuông góc với $c$ thì $b$ cũng vuông góc với $c$}
	\loigiai {
		\begin{itemize}
			\item Xét phương án A: Vì hai đường thẳng cùng vuông góc với đường thẳng thứ ba thì song song với nhau hoặc vuông góc với nhau, nên A sai.
			\item Xét phương án B: Vì nếu đường thẳng $a$ vuông góc với đường thẳng $b$ và đường thẳng $b$ vuông góc với đường thẳng $c$ thì $a$ có thể song song với $c$ (khi 3 đường thẳng $a; b; c$ đồng phẳng).
			\item Xét phương án C: Vì với $2$ đường thẳng $a \parallel b$ và $3$ đường thẳng $a; b; c$ đồng phẳng. Nếu đường thẳng $c$ vuông góc với $a$ thì $c$ cũng vuông góc với $b$.
			\item Xét phương án D: Đúng theo định lý thuyết.			
		\end{itemize}
	}
\end{ex}
\begin{ex}
	Mệnh đề nào sau đây là đúng?
	\choice
	{Một đường thẳng vuông góc với một trong hai đường thẳng vuông góc thì song song với đường thẳng còn lại}
	{Hai đường thẳng cùng vuông góc với một đường thẳng thì song song với nhau}
	{Hai đường thẳng cùng vuông góc với một đường thẳng thì vuông góc với nhau}
	{\True Một đường thẳng vuông góc với một trong hai đường thẳng song song thì vuông góc với đường thẳng kia}
	\loigiai {Theo kiến thức phần lý thuyết hai đường thẳng vuông góc}
\end{ex}
\Closesolutionfile{ans}
% \begin{indapan}{10}
% 	{ans/ans-1K7-1-1}
% \end{indapan}
\begin{dang}{Xác định góc giữa hai đường thẳng bằng định nghĩa}
\end{dang}
\subsubsection{Ví dụ mẫu}
\begin{vd}[NB]%[1K7BL-2]
	Cho hình chóp $S.ABC$ có $SA$, $SB$, $SC$ đôi một vuông góc nhau và $SA=SB=SC=a$. Gọi $M$ là trung điểm của $AB$. Tính góc giữa hai đường thẳng $SM$ và $BC$.
	\dapso{$(SM,BC)=60^\circ$}
	\loigiai{
		\immini{
			Gọi $N$ là trung điểm của $AC$.\\
			Do đó $MN\parallel BC$ nên $(SM,BC)=(SM,BC)=\widehat{SNM}$.\\
			Ta có $SA=SB=SC=a\Rightarrow SM=SN=\dfrac{AB}{2}=\dfrac{a\sqrt{2}}{2}$.\\
			$MN=\dfrac{BC}{2}=\dfrac{a\sqrt{2}}{2}$.\\
			Vậy $(SM,BC)=\widehat{SNM}=60^\circ$ (vì $\triangle SMN$ đều).			
		}{
			\begin{tikzpicture}[scale=.8, line join=round, line cap=round]
				\path
				(0,0) coordinate (B)
				(1.6,-1) coordinate (C)
				(4.5,0) coordinate (A)
				(1,3.5) coordinate (S)
				($(A)!0.5!(B)$) coordinate (M)
				($(A)!0.5!(C)$) coordinate (N)
				;
				\draw (S)--(B)--(C)--(A)--cycle (N)--(S)--(C);
				\draw[dashed] (S)--(M)--(N) (A)--(B);
				\foreach \p/\q in {A/0,B/180,C/-90,S/90,M/-100,N/-30}
				\fill[black] (\p) circle (1.0pt) ($(\p)+(\q:3.5mm)$) node{$\p$};
			\end{tikzpicture}
		}		 
	}
\end{vd}
\begin{vd}[TH]%[1K7BL-2]
	Cho hình hộp $ABCD.A'B'C'D'$ có các mặt là các hình vuông. Tính các góc $\left(AA', CD\right)$, $\left(A'C', BD\right)$, $\left(AC, DC'\right)$.
	\dapso{$\left(AA', CD\right)=\left(A'C', BD\right)=90^{\circ}$, $\left(AC,DC'\right)=60^{\circ}$}
	\loigiai{
		\immini{
			Vì $CD \parallel AB$ nên $\left(AA', C D\right)=\left(AA', AB\right)=90^{\circ}$.\\ 
			Tứ giác $ACC'A'$ có các cặp cạnh đối bằng nhau nên nó là một hình bình hành. Do đó, $A'C' \parallel  AC$.\\ 
			Vậy $\left(A'C', BD\right)=(A C, B D)=90^{\circ}$.\\	
			Tương tự, $D C' \parallel  AB'$. Vậy $\left(A C, D C'\right)=\left(AC, AB'\right)$.\\ Tam giác $AB'C$ có ba cạnh bằng nhau (vì là các đường chéo của các hình vuông có độ dài cạnh bằng nhau) nên nó là một tam giác đều. Từ đó ta có, $\left(A C, D C'\right)=\left(AC, AB'\right)=60^{\circ}$.
		}{\begin{tikzpicture}[line cap=round,line join=round, >=stealth,font=\footnotesize,scale=0.8]
				\def \a{-1.5} \def \b{-1}\def \c{3.5} \def \h{4}
				\path (.5,.5)coordinate(A) 
				+(\a,\b)coordinate(B)
				+(\c,0)coordinate(D)
				($(B)+(D)-(A)$)coordinate(C)
				+(0,\h)coordinate(C')
				($(B)+(C')-(C)$)coordinate(B')
				($(A)+(C')-(C)$)coordinate(A')
				($(D)+(C')-(C)$)coordinate(D');
				\coordinate (I) at ($(B)!0.5!(B')$);	
				\coordinate (J) at ($(B)!0.5!(A)$);
				%\draw[ultra thin,color=gray] (-2.5,-1.5) grid (5.5,5.5);
				\draw [dashed] (A)--(B)(D)--(A)--(A')
				(A)--(C) (A)--(B') (B)--(D);
				\draw(B')--(B)--(C)(B')--(C')--(C)--(D)--(D')--(A')--(B')(C')--(D')(C')--(D) (A')--(C');
				\foreach \d/\g in {A/160,B/180,C/-90,D/0,A'/90,B'/180,C'/-130,D'/0}
				\fill[black](\d) circle (1pt)+(\g:.4)node{$\d$};
	\end{tikzpicture}}}
\end{vd}
\begin{vd}[TH]%[1K7BL-2] 
	Cho hình lăng trụ $ABC.A'B'C'$ có các đáy là các tam giác đều. Tính góc $\left(AB, B'C'\right)$.
	\dapso{$(AB,B'C')=60^\circ$}
	\loigiai{
		\immini{
			Vì $B'C' \parallel BC$ nên $(AB,B'C')=(AB,BC)=\widehat{ABC}=60^\circ$.
		}{
			\begin{tikzpicture}[line cap=round,line join=round, >=stealth,font=\footnotesize, scale=1]
				\def \a{1} \def \b{-1} \def \c{3} \def \h{3.0} 
				\path (.5,.5)coordinate(A) 
				+(\a,\b)coordinate(B)
				+(\c,0)coordinate(C)
				($(A)!1/2!(C)$)coordinate(M)
				($(B)!1/2!(M)$)coordinate(H)
				+(0,\h)coordinate(A')
				($(B)+(A')-(A)$)coordinate(B')
				($(C)+(A')-(A)$)coordinate(C');
				\draw (A)--(B)--(C) (A')--(B')--(C')--(A') (A)--(A') (B)--(B') (C)--(C') ;
				\draw [dashed] (A)--(C);
				\foreach \x/\g in{A/180,B/-90, C/0,A'/90, B'/80, C'/40}
				\fill[black](\x) circle (1pt)($(\x)+(\g:2.5mm)$) node{\small $\x$};		
	\end{tikzpicture}}}
\end{vd}

\subsubsection{Bài tập rèn luyện}
% \centerline{\fcolorbox{red}{yellow!50}{\bf {BÀI TẬP TỰ LUẬN}}}
\begin{bt}%[1K7BL-2]
	Cho hình hộp $ABCD.A'B'C'D'$ có $6$ mặt đều là hình vuông và $M$, $N$, $E$, $F$ lần lượt là trung điểm các cạnh $BC$, $BA$, $AA'$, $A'D'$. Tính góc giữa các cặp đường thẳng:
	\begin{listEX}[2]
		\item $A'C'$ và $BC$.
		\item $MN$ và $EF$.
	\end{listEX}
	\dapso{$\left(A'C', BC\right)=45^{\circ}$, $\left(MN,EF\right)=60^{\circ}$}
	\loigiai{
		\immini{
			\begin{enumerate}
				\item Ta có $A C \parallel A'C'$, suy ra $\left(A'C', BC\right)=(AC, BC) =\widehat{ACB}=45^{\circ}$\\
				(tam giác $A C B$ vuông cân tại $B$).
				\item Ta có $AC\parallel  MN$, $AD'\parallel EF$,\\
				suy ra $(MN, EF)=\left(AC, AD'\right)=\widehat{CAD'}=60^{\circ}$ (tam giác $A C D'$ có ba cạnh bằng nhau).
			\end{enumerate}
		}{
			\begin{tikzpicture}[line cap=round,line join=round,every node/.style={scale=0.8}]
				\def\h{-2}
				\path 
				(0,0) coordinate (B)--+(0,\h) coordinate (B')
				(-130:1) coordinate (A)--+(0,\h) coordinate (A')
				(2.5,0) coordinate (C)--+(0,\h) coordinate (C')
				($(A)+(C)-(B)$) coordinate (D)--+(0,\h) coordinate (D')
				($(B)!.5!(C)$) coordinate (M)
				($(A)!.5!(B)$) coordinate (N)
				($(A)!.5!(A')$) coordinate (E)
				($(A')!.5!(D')$) coordinate (F);
				\draw (A)--(B)--(C)--(D)--cycle (A)--(A')--(D')--(C')--(C) (D')--(D)
				(E)--(F) (C)--(A)--(D')--cycle (M)--(N)
				;	
				\draw[dashed] (A')--(B')--(C') (B)--(B')
				(A')--(C');
				\foreach \t/\g in {A'/-90,B'/160,C'/0,D'/-90,A/180,B/90,C/90,D/-40,M/90,N/180,E/180,F/-90}{\draw[fill=red,draw=black] (\t) circle (1pt) node[shift={(\g:7pt)}]{$\t$};
				}
			\end{tikzpicture}
		}
		
	}
\end{bt}

\begin{bt}%[1K7BL-2]
	Cho tứ diện $ABCD$ có $AB=AC=AD=a$ và $\widehat{BAC}=\widehat{BAD}=60^{\circ}$, $\widehat{CAD}=90^{\circ}$. Gọi $M$ là trung điểm của $BC$. Tính góc của hai đường thẳng $AB$ và $DM$. 
	\dapso{$77^{\circ}4'$}
	\loigiai{
		\begin{center}
			\begin{tikzpicture}[scale=1, line join=round, line cap=round]
				\path
				(0,0) coordinate (B)
				(1.3,-1) coordinate (C)
				(4.5,0) coordinate (D)
				(2,3.5) coordinate (A)
				($(C)!0.5!(B)$) coordinate (M)
				($(A)!0.5!(C)$) coordinate (N)				
				;
				\draw (A)--(B)--(C)--(D)--cycle (M)--(N)--(D) (A)--(C);
				\draw[dashed] (B)--(D)--(M);				
				\foreach \p/\q in {A/90,B/180,C/-90,D/0,M/200,N/160}
				\fill[black] (\p) circle (1.0pt) ($(\p)+(\q:2.5mm)$) node{$\p$};		
			\end{tikzpicture}
		\end{center}	
		Theo giả thiết bài toán ta có tam giác $ABC$ và $ABD$ là những tam giác đều cạnh $a$.\\
		Suy ra $\left\{\begin{aligned}
			&BC=BD=a \\
			&AC=AD=a
		\end{aligned}\right. \quad(1)$ \\
		Tam giác $ACD$ vuông tại $A$ nên $CD=\sqrt{AC^2+AD^2}=a\sqrt{2}.\quad (2)$ \\
		Từ $(1)$ và $(2)$, ta có $CD^2=BC^2+BD^2$ nên tam giác $BCD$ vuông tại $B$. \\
		Gọi $N$ là trung điểm của $AC$, suy ra $MN$ là đường \\
		trung bình của tam giác $ABC$ nên $MN\parallel AB$. Do đó $(AB,DM)=(MN,DM)$. \\
		Trong tam giác $DMN$, ta có 
		$MN=\dfrac{AB}{2}=\dfrac{a}{2}$, $DN=DM=\sqrt{BD^2+BM^2}=\dfrac{a\sqrt{5}}{2}$. \\
		Theo định lý hàm số cosin, ta có 
		$\cos \widehat{DMN}=\dfrac{DM^2+MN^2-DN^2}{2DM.MN}=\dfrac{\sqrt{5}}{10}$. \\
		Vậy góc của hai đường thẳng $AB$ và $DM$ bằng $77^{\circ}4'$. 		
	}
\end{bt}
\begin{bt}%[1K7BL-2]
	\immini{Cho hình chóp $S.ABCD$ có đáy $A B C D$ là hình bình hành và $\widehat{SAB}=100^{\circ}$ (Hình bên). Tính góc giữa hai đường thẳng:
		\begin{listEX}
			\item $SA$ và $AB$,
			\item $SA$ và $CD$.
	\end{listEX}}
	{
		\begin{tikzpicture}[scale=0.8, font=\footnotesize, line join=round, line cap=round, >=stealth]
			\def\bc{4} % cạnh BC
			\def\ba{2} % cạnh BA
			\def\h{4} % đường cao
			\def\gocB{35} % góc B của đáy
			\coordinate[label=below left:$B$] (B) at (0,0);
			\coordinate[label=above right:$A$] (A) at (\gocB:\ba);
			\coordinate[label=below:$C$] (C) at (\bc,0);
			\coordinate[label=above right:$D$] (D) at ($(C)-(B)+(A)$);
			\coordinate [label=above:$S$](S) at (60:5);
			\draw (S)--(D)--(C)--(B)--cycle (S)--(C);
			\draw[dashed] (A)--(D)--(B)--cycle (A)--(C) (S)--(A);
			
			\foreach \i in {A,B,C,D,S} \fill[black] (\i) circle (1.5pt);
			\draw pic["$ 100^\circ $",draw=black, angle eccentricity=1.7, angle radius=.4cm]
			{angle=S--A--B}; 
		\end{tikzpicture}
	}
	\dapso{$(SA,Ab)=100^{\circ}$, $(SA,CD)=100^{\circ}$}
	\loigiai{
		\begin{listEX}
			\item Góc giữa hai đường thẳng $SA$ và $AB$ là $\widehat{SAB}=100^{\circ}$.
			\item Vì $CD\parallel AB$ nên góc giữa hai đường thẳng $SA$ và $CD$ bằng góc giữa hai đường thẳng $SA$ và $AB$. Suy ra $ (SA,CD)=100^{\circ} $.
		\end{listEX}
	}
\end{bt}
\begin{bt}%[1K7BL-2]
	Cho hình chóp $S.ABCD$ có đáy là hình chữ nhật với đáy $ABCD$ là hình chữ nhật với $AB=a$, $AD=a\sqrt{2}$. Ba cạnh $SA$, $AB$, $AD$ đôi một vuông góc và $SA=2a$. Gọi $I$ là trung điểm của $SD$. Tính $\cos(AI,SC)$.
	\dapso{$\cos(AI,SC)=\dfrac{\sqrt{42}}{21}$}
	\loigiai{
		\immini{
			Gọi $M$ là trung điểm của $CD$.\\
			Ta có $MI\parallel SC$ nên $(AI,SC)=(AI,IM)=\widehat{AIM}$.\\			
			Ta lại có $AI=\dfrac{SD}{2}=\dfrac{\sqrt{SA^2+AD^2}}{2}=\dfrac{\sqrt{(2a)^2+\left(a\sqrt{2}\right)^2}}{2}=\dfrac{a\sqrt{6}}{2}$,
			$AM=\sqrt{AD^2+DM^2}=\sqrt{(a\sqrt{2})^2+\left(\dfrac{a}{2}\right)^2}=\dfrac{3a}{2}$,
			\begin{eqnarray*}
				SC&=&\sqrt{AS^2+AC^2}=\sqrt{AS^2+AB^2+AD^2}\\
				&=&\sqrt{(2a)^2+(a^2)+\left(a\sqrt{2}\right)^2}=a\sqrt{7}.
			\end{eqnarray*}
			Do đó $IM=\dfrac{SC}{2}=\dfrac{a\sqrt{7}}{2}$.						
		}{
			\begin{tikzpicture}[scale=1, font=\footnotesize, line join=round, line cap=round,>=stealth]
				\path
				(0,0) coordinate (A)
				(-1.3,-1.6) coordinate (B)
				(2.5,-1.6)coordinate (C)
				($(A)+(C)-(B)$) coordinate (D)
				($(A)+(0,3)$) coordinate (S)
				($(S)!0.5!(D)$) coordinate (I)
				($(C)!0.5!(D)$) coordinate (M)
				;
				\draw (S)--(B)--(C)--(D)--cycle (S)--(C) (I)--(M);
				\draw[dashed] (S)--(A)--(D) (I)--(A)--(B) (M)--(A)--(C);	
				\foreach \p/\q in {S/90,A/-90,B/-90,C/-90,D/0,I/60,M/-30}			
				\fill[black] (\p) circle (1.0pt)node[shift={(\q:2.5mm)}]{$\p$};
			\end{tikzpicture}
		}\noindent
		Theo định lý hàm số cosin, ta có 
		$\cos \widehat{AIM}=\dfrac{AI^2+IM^2-AM^2}{2AI\cdot IM}=\dfrac{\sqrt{42}}{21}$. \\
		Vậy $\cos(AI,SC)=\dfrac{\sqrt{42}}{21}$.
	}
\end{bt}
\begin{bt}%[1H3K2-3]
	Cho tứ diện đều $ABCD$ cạnh $a$. Gọi $K$ là trung điểm của $CD$. Tính góc giữa hai đường thẳng $AK$ và $BC$.
	\dapso{$(AK,BC)\approx 73^\circ 13'$}
	\loigiai{
		\immini{
			Gọi $I$ là trung điểm $BD$. Khi đó $IK$ là đường trung bình của $\triangle BCD$ nên $IK\parallel BC$. 
			Do đó $(AK,BC)=(AK,IK)$.\\
			Xét $\triangle AIK$ có: $\heva{&AI=AK=\dfrac{a\sqrt{3}}{2} \text{ (chiều cao của tam giác đều cạnh $a$)}\\&IK=\dfrac{BC}{2}=\dfrac{a}{2}\text{ ($IK$ là đường trung bình của $\triangle BCD$). }}$\\
			$\Rightarrow \cos \widehat{AKI}=\dfrac{KA^2+KI^2-AI^2}{2KI\cdot KA}=\dfrac{\sqrt{3}}{6}\Rightarrow \widehat{AKI}\approx 73^\circ 13'$.
		}{
			\begin{tikzpicture}[line join=round, line cap=round,every node/.style={scale=0.8}] 
				\path 
				(0,0) coordinate (B)
				(2.4,0) coordinate (D)
				(.6,-.7) coordinate (C)
				($(C)!.5!(D)$) coordinate (K)
				($(B)!.5!(K)$) coordinate (G)
				--+(0,2.5) coordinate (A)
				($(B)!.5!(D)$) coordinate (I);
				\draw (A)--(B)--(C)--(D)--cycle (K)--(A)--(C);
				\draw[dashed] (B)--(D) (I)--(A) (I)--(K);
				\foreach \t/\g in {A/90,B/180,C/-90,D/0,K/-90,I/-90}{
					\draw[fill=red,draw=black] (\t) circle (1pt) node[shift={(\g:7pt)}]{$ \t $};
				}
			\end{tikzpicture}
		}
	}
\end{bt}
\begin{bt}%[1H3K2-3]
	Cho tứ diện $ABCD$. Gọi $M$, $N$ lần lượt là trung điểm của $BC$ và $AD$. Biết $AB=CD=2 a$ và $MN=a\sqrt{3}$. Tính góc giữa $AB$ và $CD$.
	\dapso{$(AB,CD)\approx 60^\circ$}
	\loigiai{
		\immini{
			Gọi $P$ là trung điểm $AC$. Ta có $\heva{&MP\parallel AB \text{ ($MP$ là đường trung bình của $\triangle CAB$) }\\&NP\parallel CD \text{ ($NP$ là đường trung bình của $\triangle ACD$). }}$\\
			$\Rightarrow (AB,CD)=(MP,NP)$.\\
			Xét $\triangle MNP$ có $\heva{&NP=\dfrac{CD}{2}=a\\&MP=\dfrac{AB}{2}=a\\&MN=a\sqrt{3}.}$ \\
			$\Rightarrow \cos\widehat{NPM}=\dfrac{PN^2+PM^2-MN^2}{2PN\cdot PM}=-\dfrac{1}{2}\Rightarrow \widehat{NPM}=120^\circ>90^\circ$.\\
			Suy ra $(MP,NP)=180^\circ-\widehat{NPM}=180^\circ-120^\circ=60^\circ$.\\
			Vậy $(AB,CD)=60^\circ$.
		}{
			\begin{tikzpicture}[line join=round, line cap=round, every node/.style={scale=0.8}]
				\path 
				(0,0) coordinate (D)
				(3,0) coordinate (B)
				(1,-.7) coordinate (C)
				(1.3,2.3)coordinate (A)
				($(B)!.5!(C)$) coordinate (M)
				($(A)!.5!(D)$) coordinate (N)
				($(A)!.5!(C)$) coordinate (P)
				;
				\draw (A)--(D)--(C)--(B)--cycle (A)--(C)
				(N)--(P)--(M);
				\draw[dashed] (B)--(D) (M)--(N);
				\foreach \t/\g in {A/90,D/180,C/-90,B/0,M/-90,N/180,P/40}{
					\draw[fill=red,draw=black] (\t) circle (1pt) node[shift={(\g:7pt)}]{$ \t $};
				}				
			\end{tikzpicture}
		}
	}
\end{bt}
% \centerline{\fcolorbox{red}{yellow!50}{\bf {CÂU HỎI TRẮC NGHIỆM}}}
% \Opensolutionfile{ans}[ans/ans-1K7-1-2]
% \begin{ex}%[1K7YL-2]
% 	Cho hình hộp $ABCD.A'B'C'D'$. Khi đó, góc giữa hai đường thẳng $B'C'$ và $AC$ là góc nào dưới đây?
% 	\choice
% 	{$\widehat{DCA}$}
% 	{$\widehat{C'A'B'}$}
% 	{$\widehat{B'C'A'}$}
% 	{\True $\widehat{DAC}$}
% 	\loigiai{
% 		\immini{
% 			Ta có $B'C'\parallel AD$ nên $(B'C',AC)=(AD,AC)=\widehat{DAC}$.
% 		}{
% 			\begin{tikzpicture}[scale=0.7,font=\footnotesize,line join=round,line cap=round,>=stealth]
% 				\path
% 				(0,0) coordinate (B)
% 				(1,0.8) coordinate (A)
% 				(4,0) coordinate (C)
% 				($(C)-(B)+(A)$) coordinate (D)	
% 				($(A)+(90:3.5)$) coordinate (A')
% 				($(B)-(A)+(A')$) coordinate (B')
% 				($(C)-(A)+(A')$) coordinate (C')
% 				($(D)-(A)+(A')$) coordinate (D')
% 				;
% 				\draw (B')--(B)--(C)--(D)--(D')--(A')--(B')--(C')--(D') (C)--(C');
% 				\draw[dashed] (A)--(D) (A')--(A)--(B);
% 				\foreach \p/\q in {A/160,B/-135,C/-45,D/0,A'/90,B'/180,C'/-20,D'/0}
% 				\fill[black] (\p)node[shift={(\q:3mm)}]{$\p$} circle (1.0pt);	
% 			\end{tikzpicture}
% 		}
% 	}
% \end{ex}
% \begin{ex}%[1K7YL-2]
% 	Cho hình lập phương $ABCD.EFGH$, góc giữa hai đường thẳng $AC$, $EH$ là
% 	\choice
% 	{$60^{\circ}$}
% 	{$30^{\circ}$}
% 	{\True $45^{\circ}$}
% 	{$120^{\circ}$}
% 	\loigiai{
% 		\immini{
% 			Ta có $EH \parallel AD$ nên $$(AC,EH)=(AC,AD)=\widehat{CAD}=45^\circ.$$
% 		}{\begin{tikzpicture}[scale=1, font=\footnotesize, line join=round, line cap=round,>=stealth]
% 				\path (0,2) coordinate (h) (0,0)coordinate (B) (2,0) coordinate (C) (3,0.7) coordinate (D) ($(B)+(D)-(C)$) coordinate (A) ($(A)+(h)$) coordinate (E) ($(B)+(h)$) coordinate (F) ($(C)+(h)$) coordinate (G) ($(D)+(h)$) coordinate (H);
% 				\draw (E)--(F)--(B)--(C)--(D)--(H)--cycle (F)--(G)--(H) (C)--(G);
% 				\draw[dashed] (B)--(A)--(E);
% 				\draw[dashed,->] (A)--(C);
% 				\draw[->] (E)--(H);
% 				\draw[dashed,->] (A)--(D);
% 				\foreach \p/\g in {A/150,B/240,C/-60,D/0,E/90,F/180,G/-30,H/45} \fill[black] (\p) circle(1pt)+(\g:0.3) node{$\p$};
% 		\end{tikzpicture}}
% 	}
% \end{ex}
% \begin{ex}%[1K7BL-2]
% 	Cho tứ diện đều $ABCD$. Góc giữa hai đường thẳng $AB$ và $CD$ bằng
% 	\choice
% 	{$45^\circ$}
% 	{\True $90^\circ$}
% 	{$60^\circ$}
% 	{$30^\circ$}
% 	\loigiai{
% 		\immini{			
% 			Gọi $M$ là trung điểm của $AB$, $O$ là trọng tâm của $\triangle ABC$.\\
% 			Do $ABCD$ là tứ diện đều nên $DO\perp (ABC)$.\\
% 			Vì $AB\perp CM$ và $AB\perp DO$ nên $AB\perp (DMC)$.\\
% 			Từ đó suy ra $AB\perp CD$.
% 		}{		
% 			\begin{tikzpicture}[scale=0.8, font=\footnotesize, line join=round, line cap=round, >=stealth]
% 				\path
% 				(0,0) coordinate (A)
% 				(3.5,-1.5) coordinate (B)
% 				(5,0) coordinate (C)
% 				($(A)!0.5!(B)$) coordinate (M)
% 				($(C)!2/3!(M)$) coordinate (O)
% 				($(O)+(0,3)$)	coordinate (D)					
% 				;				
% 				\draw (D)--(A)--(B)--(C)--(D)--(B) (D)--(B) (D)--(M);
% 				\draw[dashed] (A)--(C)--(M) (D)--(O);
% 				\foreach \p/\q in {A/180,B/-90,C/0,D/90,M/-130,O/-90}
% 				\fill[black] (\p) circle (1.0pt) ($(\p)+(\q:2.5mm)$) node{$\p$};
% 				\draw pic[draw=black,angle radius=0.2cm] {right angle = B--M--C}; 
% 			\end{tikzpicture}
% 		}
% 	}
% \end{ex}
% \begin{ex}%[1K7YL-2]
% 	\immini{
% 		Cho hình lăng trụ đứng $ABC.A'B'C'$ có tất cả các cạnh bằng nhau (tham khảo hình bên). Góc giữa đường thẳng $AB'$ và $CC'$ bằng
% 		\choice
% 		{$ 30^\circ$}
% 		{$ 90^\circ$}
% 		{$ 60^\circ$}
% 		{\True$ 45^\circ$}
% 	}{
% 		\begin{tikzpicture}[>=stealth,line join=round,line cap=round,font=\footnotesize,scale=1]
% 			\path 
% 			(0,0) coordinate (A')
% 			(1,-1) coordinate (B')
% 			(2.5,0) coordinate (C')
% 			(0,2) coordinate (A)
% 			(1,1) coordinate (B)
% 			(2.5,2) coordinate (C)
% 			;
% 			\draw (A)--(B)--(C)--cycle
% 			(A)--(A') (B)--(B') (C)--(C')
% 			(A')--(B')--(C')
% 			;
% 			\draw[dashed] (A')--(C');
% 			\foreach \p/\r in {A'/180,B'/-60,C'/0,A/180,B/-60,C/0}
% 			\fill (\p) circle (1pt) node[shift={(\r:3mm)}]{$\p$};
% 		\end{tikzpicture}
% 	}
% 	\loigiai{
% 		Ta có $CC'\parallel BB'$ nên $(AB',CC')=(AB',BB')=\widehat{AB'B}=45^\circ$ (do $ABB'A'$ là hình vuông).
% 	}
% \end{ex}
% \begin{ex}%[1K7BL-2]
% 	Cho hình lập phương $ABCD.EFGH$. Tính góc giữa hai đường thẳng $AF$ và $EG$.	
% 	\choice
% 	{$45^{\circ}$}
% 	{$30^{\circ}$}
% 	{\True $60^{\circ}$}
% 	{$90^{\circ}$}
% 	\loigiai{
% 		\immini{
% 			Vì $EG\parallel AC$ nên $(EG,AF)=(AC,AF)=\widehat{FAC}=60^{\circ}$ (tam giác $FAC$ đều).}{
% 			\begin{tikzpicture}[scale=0.4]
% 				%Định nghĩa các điểm
% 				\path
% 				(0,0) coordinate (B)
% 				(8,0) coordinate (C);
% 				\path(B)++(40:4)coordinate(A)++(0:8)coordinate(D);
% 				\path(B)++(90:8)coordinate(F)++(0:8)coordinate(G);
% 				\path(F)++(40:3.5)coordinate(E)++(0:8)coordinate(H);
% 				%Gán tên các điểm
% 				\foreach \p/\q in {A/50,B/-90,C/-45,D/0,E/90,F/180,G/-135,H/0}
% 				\fill[black] (\p) circle (1.0pt) ($(\p)+(\q:7mm)$) node{$\p$};
% 				%Nối các điểm
% 				\draw (E)--(F)--(B)--(C)--(D)--(H)--cycle (E)--(G)--(F) (C)--(G)--(H);
% 				\draw[dashed] (E)--(A)--(F) (B)--(A)--(C) (A)--(D);				
% 			\end{tikzpicture}
% 		}
% 	}
% \end{ex}
% \begin{ex}%[1K7BL-2]
% 	Cho hình chóp tứ giác $S.ABCD$ có tất cả các cạnh bằng nhau. Góc giữa hai đường thẳng $SA$ và $CD$ bằng
% 	\choice
% 	{$90^{\circ}$}
% 	{$45^{\circ}$}
% 	{$30^{\circ}$}
% 	{\True $60^{\circ}$}
% 	\loigiai{
% 		\immini{
% 			Ta có $AB\parallel CD$ nên $(SA,CD)=(SA,AB)=\widehat{SAB}=60^{\circ}$ (do $\triangle SAB$ đều).
% 		}{
% 			\begin{tikzpicture}[scale=0.7, font=\footnotesize, line join=round, line cap=round,>=stealth]
% 				\path
% 				(0,0) coordinate (A)
% 				(-1.3,-1.6) coordinate (B)
% 				(2.5,-1.6)coordinate (C)
% 				($(A)+(C)-(B)$) coordinate (D)
% 				($(A)+(0.5,3)$) coordinate (S)
% 				;
% 				\draw (S)--(B)--(C)--(D)--cycle (S)--(C);
% 				\draw[dashed] (S)--(A)--(D) (A)--(B);	
% 				\foreach \p/\q in {S/90,A/-90,B/-90,C/-90,D/0}			
% 				\fill[black] (\p) circle (1.0pt)node[shift={(\q:2.5mm)}]{$\p$};
% 		\end{tikzpicture}}
% 	}
% \end{ex}
% \begin{ex}%[1K7BL-2]
% 	Cho tứ diện $ABCD$ có $AB=CD$. Gọi $I$, $J$, $E$, $F$ lần lượt là trung điểm của $AC$, $BC$, $BD$, $AD$. Tính số đo của góc giữa hai đường thẳng $IE$ và $JF$.
% 	\choice
% 	{$60^\circ $}
% 	{\True $90^\circ $}
% 	{$30^\circ $}
% 	{$45^\circ $}
% 	\loigiai{
% 		\begin{center}
% 			\begin{tikzpicture}[font=\footnotesize, line join=round, line cap=round, >=stealth,scale=1]
% 				\coordinate[label=left:$A$] (A) at (0,0);
% 				\coordinate[label=below left:$B$] (B) at (3,-1);
% 				\coordinate[label=right:$C$] (C) at (4,0);
% 				\coordinate[label=above left:$D$] (S) at (1.2,4);
% 				\coordinate[label=below:$I$] (I) at ($(A)!0.5!(C)$);
% 				\coordinate[label=right:$J$] (J) at ($(B)!0.5!(C)$);
% 				\coordinate[label=right:$E$] (E) at ($(B)!0.5!(S)$);
% 				\coordinate[label=left:$F$] (F) at ($(A)!0.5!(S)$);
% 				\draw (A)--(B)--(C)--(S)--cycle (S)--(B) (J)--(E)--(F);
% 				\draw[dashed] (A)--(C) (F)--(I)--(J)--(F) (E)--(I);
% 				\fill (A)circle(1pt) (I)circle(1pt) (J)circle(1pt) (E)circle(1pt) (F)circle(1pt) (B)circle(1pt) (C)circle(1pt) (S)circle(1pt);
% 			\end{tikzpicture}
% 		\end{center}
% 		Ta có $IJ=FE=\dfrac{1}{2}AB$ và $IJ \parallel FE \parallel AB$ nên tứ giác $IJEF$ là hình bình hành.\\
% 		Mặt khác  $IJ=\dfrac{1}{2}AB$, $JE=\dfrac{1}{2}CD$ mà $AB=CD$ nên $IJ=JE\Rightarrow $ tứ giác $IJEF$ là hình thoi.\\
% 		Vậy $\left(IE,JF\right)=90^\circ$ .
% 	}
% \end{ex}
% \begin{ex}%[1K7BL-2]
% 	Cho hình chóp $S.ABCD$ có đáy $ABCD$ là hình bình hành, $SA=SB=a\sqrt{6}$, $CD=2a\sqrt{2}$. Gọi $\varphi$ là góc giữa hai đường thẳng $CD$ và $AS$. Tính $\cos\varphi$.
% 	\choice
% 	{$\cos\varphi=\dfrac{2}{\sqrt{6}}$}
% 	{$\cos\varphi=\dfrac{2}{\sqrt{3}}$}
% 	{$\cos\varphi=\dfrac{3}{\sqrt{6}}$}
% 	{\True  $\cos\varphi=\dfrac{1}{\sqrt{3}}$}
% 	\loigiai{
% 		\immini{
% 			Ta có
% 			\[			
% 				\varphi=(CD, AS)=(AB, AS)=\widehat{SAB}.
% 			\]
% 			$\Rightarrow \cos\varphi=\cos\widehat{SAB}$.\\
% 			Xét tam giác $SAB$ có $SA=SB=a\sqrt{6}$, $AB=CD=2a\sqrt{2}$.\\ Áp dụng định lý cosin ta có:\\
% 			$\cos\widehat{SAB}=\dfrac{SA^2+AB^2-SB^2}{2\cdot SA\cdot AB}=\dfrac{6a^2+8a^2-6a^2}{2\cdot a\sqrt{6}\cdot 2a\sqrt{2}}=\dfrac{1}{\sqrt{3}}$.\\
% 			Vậy $\cos\varphi=\dfrac{1}{\sqrt{3}}$.
% 		}
% 		{
% 			\begin{tikzpicture}[scale=1, font=\footnotesize, line join=round, line cap=round,>=stealth]
% 				\path
% 				(0,0) coordinate (A)
% 				(-1.3,-1.6) coordinate (B)
% 				(2.5,-1.6)coordinate (C)
% 				($(A)+(C)-(B)$) coordinate (D)
% 				($(A)+(0.5,3)$) coordinate (S)
% 				;
% 				\draw (S)--(B)--(C)--(D)--cycle (S)--(C);
% 				\draw[dashed] (S)--(A)--(D) (A)--(B);	
% 				\foreach \p/\q in {S/90,A/-90,B/-90,C/-90,D/0}			
% 				\fill[black] (\p) circle (1.0pt)node[shift={(\q:2.5mm)}]{$\p$};
% 			\end{tikzpicture}
% 		}
% 	}
% \end{ex}
% \begin{ex}%[1K7BL-2]
% 	Cho tứ diện $ABCD$ có $AB=AC=AD$ và $\widehat{BAC}=\widehat{BAD}=60^{\circ}$, $\widehat{CAD}=90^{\circ}$. Gọi $I$, $J$ lần lượt là trung điểm của $AB$ và $CD$. Tính góc giữa hai đường thẳng $IJ$ và $CD$.
% 	\choice
% 	{$45^{\circ}$}
% 	{$120^{\circ}$}
% 	{\True $90^{\circ}$}
% 	{$60^{\circ}$}
% 	\loigiai{
% 		\immini{
% 			Ta có $\triangle ABC$ và $\triangle ABD$ là hai tam giác đều và bằng nhau nên $IC=ID\Rightarrow \triangle ICD$ cân tại $I$.\\
% 			Suy ra $IJ\perp CD\Rightarrow (IJ,CD)=90^{\circ} $.
% 		}{
% 		\begin{tikzpicture}[scale=1, font=\footnotesize, line join=round, line cap=round, >=stealth]
% 				\path
% 				(-1,2.4) coordinate (A)
% 				(-0.2,-1.5) coordinate (B)
% 				(-2,0) coordinate (C)
% 				(1.8,0) coordinate (D)
% 				($(A)!1/2!(B)$) coordinate (I)
% 				($(C)!1/2!(D)$) coordinate (J)
% 				;
% 				\clip (-3,-2.3) rectangle (3,3);
% 				\draw (A)--(C)--(B)--(D)--cycle (A)--(B) (C)--(I)--(D);
% 				\draw[dashed] (I)--(J) (C)--(D);
% 				\foreach \p/\q in {A/90,B/-90,C/180,D/0,I/40,J/-90}
% 				\fill[black] (\p) circle (1.0pt) ($(\p)+(\q:2.5mm)$) node{$\p$};			
% 	\end{tikzpicture}}}
% \end{ex}
% \begin{ex}%[1K7BL-2]
% 	Cho hình chóp tứ giác $S.ABCD$ có đáy $ABCD$ là hình bình hành, tam giác $ SBC $ là tam giác đều. Tính góc giữa hai đường thẳng $ AD $ và $ SB $.
% 	\choice
% 	{\True $60^{\circ}$}
% 	{$120^{\circ}$}
% 	{$30^{\circ}$}
% 	{$90^{\circ}$}
% 	\loigiai{
% 		\immini{
% 			Ta có $ AD\parallel BC \Rightarrow (AD, SB) =(BC,SB) =60^\circ $ (vì tam giác $ SBC $ đều).
% 		}{
% 			\begin{tikzpicture}[scale=0.7, font=\footnotesize, line join=round, line cap=round,>=stealth]
% 				\path
% 				(0,0) coordinate (A)
% 				(-1.3,-1.6) coordinate (B)
% 				(2.5,-1.6)coordinate (C)
% 				($(A)+(C)-(B)$) coordinate (D)
% 				($(A)+(0,3)$) coordinate (S)
% 				;
% 				\draw (S)--(B)--(C)--(D)--cycle (S)--(C);
% 				\draw[dashed] (S)--(A)--(D) (A)--(B);	
% 				\foreach \p/\q in {S/90,A/-90,B/-90,C/-90,D/0}			
% 				\fill[black] (\p) circle (1.0pt)node[shift={(\q:2.5mm)}]{$\p$};			
% 		\end{tikzpicture}}
% 	}
% \end{ex}
% \begin{ex}%[1K7BL-2]
% 	Cho hình chóp $S.ABC$ có tất cả các cạnh đều bằng $a$. Gọi $I$, $J$ lần lượt là trung điểm của $SA$, $BC$. Tính số đo của góc hợp bởi $IJ$ và $SB$.
% 	\choice
% 	{$60^\circ$}
% 	{$90^\circ$}
% 	{\True $45^\circ$}
% 	{$30^\circ$}
% 	\loigiai{
% 		\begin{center}
% 			\begin{tikzpicture}[scale=0.7, font=\footnotesize, line join=round, line cap=round, >=stealth]
% 				\path
% 				(0,0) coordinate (A)
% 				(1.5,-3) coordinate (B)
% 				(6,0) coordinate (C)
% 				($(A)!0.5!(B)$) coordinate (M)
% 				($(C)!0.5!(B)$) coordinate (J)
% 				(intersection of A--J and C--M) coordinate (O)
% 				($(O)+(0,5)$) coordinate (S)
% 				($(A)!0.5!(S)$) coordinate (I)				
% 				;
% 				\draw (S)--(A)--(B)--(C)--cycle (M)--(I)--(B)--(S);
% 				\draw[dashed] (A)--(C) (I)--(J)--(M);
% 				\foreach \p/\q in {A/180,B/-90,C/0,S/90,M/-120,I/140,J/-40}
% 				\fill[black] (\p) circle (1.0pt) ($(\p)+(\q:3.5mm)$) node{$\p$};		
% 			\end{tikzpicture}
% 		\end{center}
% 		Gọi $M$ là trung điểm của $AB$. Khi đó $IM$ là đường trung bình của tam giác $SAB$ nên $IM\parallel SB$ và $IM=\dfrac{SB}{2}=\dfrac{a}{2}$. Tương tự $MJ=\dfrac{a}{2}$.
% 		Mặt khác, dễ dàng chứng minh tam giác $IBJ$ vuông tại $J$ nên
% 		$$IJ=\sqrt{IB^2-IB^2}=\sqrt{\left(\dfrac{a\sqrt{3}}{2}\right)^2-\left(\dfrac{a}{2}\right)^2}=\dfrac{a\sqrt{2}}{2}.$$
% 		Tam giác $IMJ$ có $MI=MJ=\dfrac{a}{2},IJ=\dfrac{a\sqrt{2}}{2}$ nên là tam giác vuông cân tại $M$. Suy ra
% 		$$(IJ,SB)=(IJ,IM)=\widehat{MIJ}=45^\circ \text{ (do }IM\parallel SB).$$}
% \end{ex}
% \Closesolutionfile{ans}
% \begin{indapan}{10}
% 	{ans/ans-1K7-1-2}
% \end{indapan}

\begin{dang}{Xác định hai đường thẳng vuông góc}
\end{dang}
\subsubsection{Ví dụ mẫu}
\begin{vd}[TH]%[1K7BL-3]
	\immini{Cho hình chóp $S.ABCD$ có đáy $ABCD$ là hình thoi. Gọi $M$, $N$ lần lượt là trung điểm của các cạnh $SB$ và $SD$. Chứng minh rằng $AC \perp MN$.
	}
	{
		\begin{tikzpicture}[scale=1, font=\footnotesize, line join=round, line cap=round, >=stealth]
			\def\bc{4} % cạnh BC
			\def\ba{2} % cạnh BA
			\def\h{4} % đường cao
			\def\gocB{35} % góc B của đáy
			\coordinate[label=below left:$B$] (B) at (0,0);
			\coordinate[label=above left:$A$] (A) at (\gocB:\ba);
			\coordinate[label=below:$C$] (C) at (\bc,0);
			\coordinate[label=above right:$D$] (D) at ($(C)-(B)+(A)$);
			\coordinate [label=above:$S$](S) at (60:5);
			\coordinate[label=left:$M$] (M) at ($(S)!1/2!(B)$);
			\coordinate[label=right:$N$] (N) at ($(S)!1/2!(D)$);
			\draw (S)--(D)--(C)--(B)--cycle (S)--(C);
			\draw[dashed] (M)--(N) (A)--(D)--(B)--cycle (A)--(C) (S)--(A);			
			\foreach \i in {S,A,B,C,D,M,N} \fill[black] (\i) circle (1.5pt);
		\end{tikzpicture}
	}
	\loigiai{
		Vì $M$, $N$ lần lượt là trung điểm của $SB$ và $SD$ nên $MN \parallel BD$.\\
		Do tứ giác $ABCD$ là hình thoi nên $AC \perp BD$. Từ các kết quả trên, ta có $AC \perp MN$.
	}
\end{vd}
\begin{vd}[TH]%[1K7BL-3]
	Cho hình hộp $ABCD.A'B'C'D'$.
	\begin{enumerate}
		\item  Xác định vị trí tương đối của hai đường thẳng $AC$ và $B'D'$.
		\item  Chứng minh rằng $AC$ và $B'D'$ vuông góc với nhau khi và chỉ khi $ABCD$ là một hình thoi.
	\end{enumerate}
	\loigiai{
		\immini{
			\begin{enumerate}
				\item Hai đường thẳng $AC$ và $B'D'$ lần lượt thuộc hai mặt phẳng song song $(ABCD)$ và $\left(A' B' C' D'\right)$ nên chúng không có điểm chung, tức là chúng không thể trùng nhau hoặc cắt nhau.\\
				Tứ giác $B D D' B'$ có hai cạnh đối $B B'$ và $D D'$ song song và bằng nhau nên nó là một hình bình hành. Do đó $B'D'$ song song với $B D$. Mặt khác, $B D$ không song song với $AC$ nên $B'D'$ không song song với $AC$.
				Từ những điều trên suy ra $AC$ và $B'D'$ chéo nhau.
				\item Do $B'D'$ song song với $B D$ nên $\left(A C, B' D'\right)=(A C, B D)$. Do đó, $AC$ và $B'D'$ vuông góc với nhau khi và chỉ khi $AC$ và $B D$ vuông góc với nhau. Do $ABCD$ là hình bình hành nên $AC$ vuông góc với $B D$ khi và chỉ khi $ABCD$ là hình thoi.
		\end{enumerate}}{\begin{tikzpicture}[line cap=round,line join=round, >=stealth,font=\footnotesize,scale=0.8]
				\def \a{-2} \def \b{-1}\def \c{4.5} \def \h{4} 
				\path (.5,.5)coordinate(A') 
				+(\a,\b)coordinate(B')
				+(\c,0)coordinate(D')
				($(B')+(D')-(A')$)coordinate(C')
				($(A')!2/3!($(B')!1/2!(C')$)$)coordinate(G)
				+(-1,\h)coordinate(B)
				($(C')+(B)-(B')$)coordinate(C)
				($(A')+(B)-(B')$)coordinate(A)
				($(D')+(B)-(B')$)coordinate(D);
				%	\draw[ultra thin,color=gray] (-2.5,-1.5) grid (7.5,5.5);
				\draw [dashed] (A')--(B')--(D')--(A') (A') -- (A);
				\draw(A) --(C)--(B)--(B')--(C')--(C)--(C')--(D')--(D)--(A)--(B)--(D)(C)--(D);
				\foreach \i/\j in {A/150, B/180,C/-30,D/0,A'/160,B'/-90,C'/-90,D'/0}\fill[black] (\i) circle (1pt) ($(\i)+(\j:4mm)$)node{$\i$};
	\end{tikzpicture}}}
\end{vd}
\begin{vd}[TH]%[1K7BL-3]
	Cho hình hộp $ABCD.A'B'C'D'$ có $6$ mặt đều là hình vuông. Chứng minh rằng $AB\perp CC'$, $AC \perp B'D'$.
	\loigiai{
		Ta có $CC' \parallel BB'$, suy ra $\left(AB, CC'\right)=\left(AB, BB'\right)=\widehat{ABB'}=90^{\circ}$. Vậy $A B \perp C C'$.\\
		Ta có $B'D' \parallel BD$, suy ra $\left(AC, B'D'\right)=(AC, BD)=90^{\circ}$ (hai đường chéo của hình vuông luôn vuông góc với nhau). Vậy $AC \perp B'D'$.
	}
\end{vd}
\subsubsection{Bài tập rèn luyện}
\centerline{\fcolorbox{red}{yellow!50}{\bf {BÀI TẬP TỰ LUẬN}}}
\begin{bt}[TH]%[1K7BL-3] 
	Cho tứ diện $ABCD$ có $\widehat{CBD}=90^{\circ}$.
	\begin{enumerate}
		\item  Gọi $M$, $N$ tương ứng là trung điểm của $AB$, $AD$. Chứng minh rằng $MN$ vuông góc với $BC$.
		\item Gọi $G$, $K$ tương ứng là trọng tâm của các tam giác $ABC$, $ACD$. Chứng minh rằng $GK$ vuông góc với $BC$.
	\end{enumerate}
	\loigiai{\immini{\begin{enumerate}
				\item  Vì $MN$ là đường trung bình của tam giác $ABD$ nên $MN \parallel BD$ và theo giả thiết $BD \perp BC$ nên ta có $MN \perp BC$.
				\item Hai đường thẳng $GK$ và $BD$ cùng nằm trong mặt phẳng $(PBD)$ nên đồng phẳng, đồng thời $\dfrac{PG}{PB}=\dfrac{PK}{PD}=\dfrac{1}{3}$ nên ta có $GK \parallel BD$.\\
				Mặt khác, $BD\perp BC$ nên ta cũng có $GK \perp BC$.\end{enumerate}}{\begin{tikzpicture}[>=stealth,line join=round,line cap=round,font=\footnotesize,scale=1]
				\def \a{1} \def \b{-1} \def \c{5} \def \h{3.5}  \def \d{2}
				\path (.5,.5)coordinate(B) 
				+(\a,\b)coordinate(D)
				+(\d,\h)coordinate(A)
				+(\c,0)coordinate(C);
				\path ($(A)!1/2!(B)$)coordinate(M);	
				\path ($(A)!1/2!(D)$)coordinate(N);	
				\path ($(A)!1/2!(C)$)coordinate(P);	
				\path ($(B)!2/3!(P)$)coordinate(G);	\\
				\path ($(D)!2/3!(P)$)coordinate(K);	
				\draw [dashed] (B)--(C) (B)--(P) (G)--(K);
				\draw (A)--(B)--(D)--(A)--(C)--(D) (M)--(N) (D)--(P);
				\foreach \i/\j in {A/150,P/0,G/90,K/0, D/-135,C/-90,M/180,N/0,B/180}\fill[black] (\i) circle (1pt) ($(\i)+(\j:3mm)$)node{$\i$};
				\foreach \x/\o/\y/\r in {D/B/C/2} \draw ($(\o)!\r mm!(\x)$)--($($(\o)!\r mm!(\x)$)+($(\o)!\r mm!(\y)$)-(\o)$)--($(\o)!\r mm!(\y)$);
	\end{tikzpicture}}}
\end{bt}
\begin{bt}[TH]%[1H3K2-3]
	Cho tứ diện đều $ABCD$. Chứng minh rằng $AB \perp CD$.
	\loigiai{
		\immini{
			Gọi $2x$ là cạnh của tứ diện đều.\\
			Gọi $M$, $N$, $P$ lần lượt là trung điểm của $BC$, $AC$, $BD$.\\
			Các tam giác $ABD$ và $CBD$ đều có cùng cạnh $2x$ nên các đường cao $AP$ và $CP$ của chúng cũng bằng nhau, và
			$AP=CP=\dfrac{2x\sqrt{3}}{2}=x\sqrt{3}$.\\
			Khi đó $\triangle PAC$ cân tại $P$, có $PN$ là đường trung tuyến suy ra $PN\perp AC$.
			Ta có $\heva{&AB\parallel MN \text{ ($MN$ là đường trung bình của $\triangle ABC$). }\\&CD\parallel MP \text{ ($MP$ là đường trung bình của $\triangle BCD$) }}$
			\\$\Rightarrow (AB,CD)=(MN,MP)$.
		}{
			\begin{tikzpicture}[line join=round, line cap=round,every node/.style={scale=0.8}] 
				\path 
				(0,0) coordinate (B)
				(3,0) coordinate (D)
				(1,-1) coordinate (C)
				($(B)!.5!(C)$) coordinate (M)
				($(A)!.5!(C)$) coordinate (N)
				($(B)!.5!(D)$) coordinate (P)
				($(D)!2/3!(M)$) coordinate (G)
				--+(0,2.5) coordinate (A);
				\draw (A)--(B)--(C)--(D)--cycle (A)--(C)
				(M)--(N);
				\draw[dashed] (B)--(D) (M)--(P)--(N)
				(P)--(A) (P)--(C);
				\foreach \t/\g in {A/90,B/180,C/-90,D/0,M/180,N/180,P/60}{
					\draw[fill=red,draw=black] (\t) circle (1pt) node[shift={(\g:7pt)}]{$ \t $};
				}
				\draw pic[draw, angle radius=1.5mm]{right angle=C--N--P}; 
			\end{tikzpicture}
		}
		\noindent Xét $\triangle MNP$ có: $MN=\dfrac{AB}{2}=x$, $MP=\dfrac{CD}{2}=x$; $PN=\sqrt{PA^2-AN^2}=\sqrt{\left(x\sqrt{3}\right)^2-x^2}=x\sqrt{2}$.\\
		Suy ra $\triangle MPN$ vuông cân tại $M$, suy ra $\widehat{MNP}=90^\circ$.\\
		Vậy $(AB,CD)=(MN,MP)=\widehat{MPN}=90^\circ$, suy ra $AB\perp CD$ (đpcm).
	}
\end{bt}
\begin{bt}[TH]%[1H3K2-3]
	Cho hình chóp $S.ABC$ có $SA=SB=SC=a$, $\widehat{BSA}=\widehat{CSA}=60^{\circ}$, $\widehat{BSC}=90^{\circ}$. Cho $I$ và $J$ lần lượt là trung điểm của $S A$ và $B C$. Chứng minh rằng $I J \perp S A$ và $I J \perp B C$.
	\loigiai{
		\immini{
			$\triangle SAB$ và $\triangle SAC$ cân tại $S$, có $\widehat{ASB}=\widehat{ASC}=60^\circ$, suy ra $SAB$ và $SAC$ là các tam giác đều. Suy ra $AB=AC=a$.\\
			$\triangle SBC$ vuông cân tại $S$, suy ra $BC=SA\sqrt{2}=a\sqrt{2}$.\\
			Suy ra $\triangle BAC$ vuông cân tại $A$.\\
			$SBC$ và $ABC$ là các tam giác vuông có cùng cạnh huyền $BC$, $J$ là trung điểm $BC\Rightarrow JS=JA \left(=\dfrac{BC}{2}\right)$.\\
			$\triangle JSA$ cân tại $S$ có $JI$ là đường trung tuyến, suy ra $JI\perp SA$.\\
			$IB$ và $IC$ là các đường cao của tam giác đều có cùng cạnh $a$, suy ra $IB=IC$. \\
			$\triangle IBC$ cân tại $I$, có $IJ$ là đường trung tuyến nên $IJ\perp BC$.
			
		}{
			\begin{tikzpicture}[line join=round, line cap=round, every node/.style={scale=0.8}]
				\path 
				(0,0) coordinate (A)
				(3,0) coordinate (C)
				(1,-.7) coordinate (B)
				(1,2.3)coordinate (S)
				($(S)!.5!(A)$) coordinate (I)
				($(B)!.5!(C)$) coordinate (J)
				;
				\draw pic[draw, angle radius=2mm,fill=gray!20]{right angle=B--S--C}; 
				\draw (S)--(A)--(B)--(C)--cycle (J)--(S)--(B)--(I);
				\draw[dashed] (A)--(C) (I)--(C) (I)--(J)--(A);
				\foreach \t/\g in {S/90,A/180,B/-90,C/0,I/180,J/-90}{
					\draw[fill=red,draw=black] (\t) circle (1pt) node[shift={(\g:7pt)}]{$ \t $};
				}			
				\draw pic[draw, angle radius=2mm]{right angle=B--A--C}; 
			\end{tikzpicture}
		}
	}
\end{bt}
\begin{bt}[VD]%[1K7KL-2] 
	Cho hình hộp $ABCD.A'B'C'D'$ có các cạnh bằng nhau. Chứng minh rằng tứ diện $ACB'D'$ có các cặp cạnh đối diện vuông góc với nhau.
	\loigiai{
		\immini{
			Hình hộp đã cho có các cạnh bằng nhau nên tứ giác $ABCD$ là một hình thoi. Suy ra $AC \perp BD$. Mà $BD\parallel B'D'$ nên $AC \perp B'D'$.\\
			Lập luận tương tự cho hai cặp cạnh đối diện còn lại.\\
			Vậy tứ diện $ACB'D'$ có các cặp cạnh đối diện vuông góc. }{\begin{tikzpicture}[line cap=round,line join=round, >=stealth,font=\footnotesize,scale=0.6]
				\def \a{-2} \def \b{-1}\def \c{4.5} \def \h{4} 
				\path (.5,.5)coordinate(A') 
				+(\a,\b)coordinate(B')
				+(\c,0)coordinate(D')
				($(B')+(D')-(A')$)coordinate(C')
				($(A')!2/3!($(B')!1/2!(C')$)$)coordinate(G)
				+(-1,\h)coordinate(B)
				($(C')+(B)-(B')$)coordinate(C)
				($(A')+(B)-(B')$)coordinate(A)
				($(D')+(B)-(B')$)coordinate(D);
				%	\draw[ultra thin,color=gray] (-2.5,-1.5) grid (7.5,5.5);
				\draw [dashed] (A')--(B')--(D')--(A') (A') -- (A)--(B') (A)--(D');
				\draw(A) --(C)--(B)--(B')--(C')--(C)--(C')--(D')--(D)--(A)--(B)--(D)(C)--(D) (C)--(B') (C)--(D');
				\foreach \i/\j in {A/150, B/180,C/-30,D/0,A'/-50,B'/-90,C'/-90,D'/0}\fill[black] (\i) circle (1pt) ($(\i)+(\j:4mm)$)node{$\i$};
	\end{tikzpicture}}}
\end{bt}
% \centerline{\fcolorbox{red}{yellow!50}{\bf {CÂU HỎI TRẮC NGHIỆM}}}
% \Opensolutionfile{ans}[ans/ans-1K7-1-3]
% \begin{ex}%[1K7BL-3]
% 	Cho tứ diện đều $ABCD$ có độ dài các cạnh bằng $a$, $M$ là trung điểm của cạnh $BC$. Khi đó $\cos(AB,DM)$ bằng
% 	\choice
% 	{\True $\dfrac{\sqrt{3}}{6}$}
% 	{$\dfrac{\sqrt{2}}{2}$}
% 	{$\dfrac{1}{2}$}
% 	{$\dfrac{\sqrt{3}}{2}$}
% 	\loigiai{
% 		\immini{
% 			Gọi $N$ là trung điểm của $AC$.\\
% 			Xét $\triangle MND$ có $MN=\dfrac{a}{2}$, $DM=DN=\dfrac{\sqrt{3}}{2}a$\\
% 			$\Rightarrow \cos\left(MN,DM \right)=\dfrac{MN^2+MD^2-DN^2}{2MN\cdot MD}=\dfrac{\sqrt{3}}{6}$.\\
% 			Ta có $ \cos\left(AB,DM \right) =\cos\left(MN,DM \right)=\dfrac{\sqrt{3}}{6}$.
% 		}{
% 		\begin{tikzpicture}[scale=.7,line join=round, line cap=round,smooth,font=\footnotesize,>=stealth]
% 				\coordinate (a) at (0,3);
% 				\coordinate (b) at (-3.36,0);
% 				\coordinate (c) at (-1,-2);
% 				\coordinate (d) at (4.28,0);
% 				\coordinate (m) at ($(b)!.5!(c)$);
% 				\coordinate (n) at ($(a)!.5!(c)$);
% 				\draw (a)--(b)--(c)--(d)--cycle
% 				(a)--(c) (d)--(n)--(m)--(a)
% 				;
% 				\draw[dashed] (b)--(d)--(m)  ;
% 				\fill (a)node[above]{$A$}circle(.06)
% 				(b)node[left]{$B$}circle(.06)
% 				(c)node[below]{$C$}circle(.06)
% 				(d)node[right]{$D$}circle(.06)
% 				(m)node[left]{$M$}circle(.06)
% 				(n)node[above right]{$N$}circle(.06);
% 	\end{tikzpicture}}}
% \end{ex}
% \begin{ex}%[1K7BL-3]
% 	Cho hình chóp $S.ABCD$ có tất cả các cạnh đều bằng $a$. Gọi $I$ và $J$ lần lượt là trung điểm của $SC$ và $BC$. Số đo của góc $(IJ,CD)$ bằng
% 	\choice
% 	{$30^{\circ}$}
% 	{\True $60^{\circ}$}
% 	{$90^{\circ}$}
% 	{$45^{\circ}$}
% 	\loigiai{
% 		\immini{
% 			Gọi $O$ là tâm của hình thoi $ABCD$\\
% 			$\Rightarrow $ $OJ$ là đường trung bình của $\triangle BCD \Rightarrow \heva{
% 				& OJ\parallel CD \\
% 				& OJ=\dfrac{1}{2}CD.\\}$\\
% 			Vì $CD\parallel OJ\Rightarrow \left(IJ,CD\right)=\left(IJ,OJ\right)$.\\
% 			Xét tam giác $IOJ$, có $\heva{
% 				& IJ=\dfrac{1}{2}SB=\dfrac{a}{2} \\
% 				& OJ=\dfrac{1}{2}CD=\dfrac{a}{2} \\
% 				& IO=\dfrac{1}{2}SA=\dfrac{a}{2} \\}\Rightarrow \triangle IOJ$ đều.\\
% 			Vậy $\left(IJ,CD\right)=\left(IJ,OJ\right)=\widehat{IJO}=60^{\circ}$.}
% 		{
% 			\begin{tikzpicture}[scale=.55,font=\footnotesize, line join=round, line cap=round,>=stealth]
% 				\path
% 				(0,0) coordinate (B)
% 				(5,0) coordinate (C)
% 				(7.5,2.5) coordinate (D)
% 				;
% 				\coordinate (A) at ($(B)+(D)-(C)$);
% 				\coordinate (H) at ($(A)!.5!(B)$);
% 				\coordinate (S) at ($(H)+(3,6)$);
% 				\coordinate (O) at ($(A)!0.5!(C)$);
% 				\coordinate (I) at ($(S)!0.5!(C)$);
% 				\coordinate (J) at ($(B)!0.5!(C)$);
% 				\draw (S)--(B)--(C)--(D)--cycle (S)--(C) (I)--(J);
% 				\draw[dashed] (O)--(S)--(A)--(B)--(D)--(A)--(C) (I)--(O)--(J);
% 				\foreach \p/\q in {A/180,B/-90,C/-90,D/0,S/90,O/-90,I/30,J/-90}
% 				\fill[black] (\p) circle (1.0pt) ($(\p)+(\q:4mm)$) node{$\p$};
% 	\end{tikzpicture}}}
% \end{ex}
% \begin{ex}%[1K7BL-3]
% 	\immini{
% 		Cho tứ diện $OABC$ có $OA$, $OB$, $OC$ đôi một vuông góc với nhau và $OA=OB=OC$. Gọi $M$ là trung điểm của $BC$. Góc giữa hai đường thẳng $OM$ và $AB$ bằng
% 		\choice
% 		{$90^\circ$}
% 		{$45^\circ$}
% 		{$30^\circ$}
% 		{\True $60^\circ$}
% 	}{
% 		\begin{tikzpicture}[scale=1, font=\footnotesize, line join=round, line cap=round, >=stealth]
% 			\path
% 			(0,0) coordinate (O)
% 			(1,-1) coordinate (C)
% 			(2.5,0) coordinate (B)
% 			($(O)+(0,2)$) coordinate (A)
% 			($(B)!0.5!(C)$) coordinate (M);
% 			\draw (A)--(O)--(C)--(B)--(A)--(C);
% 			\draw[dashed] (O)--(B) (O)--(M);
% 			\foreach \i/\j in {O/left,C/below,B/right,A/above,M/below right}
% 			\draw[fill=black] (\i) circle(1pt) node[\j]{$\i$};
% 		\end{tikzpicture}
% 	}
% 	\loigiai{
% 		\immini{
% 			Gọi $N$ là trung điểm $AC$. Suy ra $AB\parallel MN$.\\
% 			Xét tam giác $OMN$ có $OM=\dfrac{BC}{2}$, $ON=\dfrac{AC}{2}$, $MN=\dfrac{AB}{2}$.\\
% 			Mà $BC=AC=AB$ nên $OM=ON=MN$. Suy ra tam giác $OMN$ là tam giác đều.\\
% 			Khi đó $(OM,AB)=(OM,MN)=\widehat{OMN}=60^\circ$.
% 		}{
% 			\begin{tikzpicture}[scale=1, font=\footnotesize, line join=round, line cap=round, >=stealth]
% 				\path
% 				(0,0) coordinate (O)
% 				(1,-1) coordinate (C)
% 				(2.5,0) coordinate (B)
% 				($(O)+(0,2)$) coordinate (A)
% 				($(B)!0.5!(C)$) coordinate (M)
% 				($(A)!0.5!(C)$) coordinate (N);
% 				\draw (A)--(O)--(C)--(B)--(A)--(C) (O)--(N)--(M);
% 				\draw[dashed] (O)--(B) (O)--(M);
% 				\foreach \i/\j in {O/left,C/below,B/right,A/above,M/below right,N/above right}
% 				\draw[fill=black] (\i) circle(1pt) node[\j]{$\i$};
% 			\end{tikzpicture}
% 		}
% 	}
% \end{ex}
% \begin{ex}%[1K7BL-3]
% 	Cho tứ diện $OABC$ có $OA$, $OB$, $OC$ đôi một vuông góc. Khẳng định nào sau đây đúng?
% 	\choice
% 	{\True Tam giác $ABC$ có ba góc nhọn}
% 	{Tam giác $ABC$ có một góc tù và hai góc nhọn}
% 	{Tam giác $ABC$ là tam giác vuông}
% 	{Tam giác $ABC$ là tam giác đều}
% 	\loigiai{
% 		\immini{
% 			Trong $\Delta ABC$, ta có $\cos A=\dfrac{AB^2+AC^2-BC^2}{2\cdot AB\cdot AC}$\\
% 			$=\dfrac{OA^2+OB^2+OA^2+OC^2-\left(OB^2+OC^2\right)}{2\cdot AB\cdot AC}=\dfrac{OA^2}{AB\cdot AC}>0$.\\
% 			Do đó $\widehat{A}$ là góc nhọn.\\
% 			Tương tự ta cũng có $\widehat{B},\widehat{C}$ là các góc nhọn.\\
% 			Vậy tam giác $ABC$ có ba góc nhọn.
% 		}{
% 			\begin{tikzpicture}[>=stealth, line join=round, line cap = round,scale=0.6]
% 				\path
% 				(0,0) coordinate (O)
% 				(2,-2) coordinate (A)
% 				(5,0) coordinate (B)
% 				(0,4) coordinate (C)
% 				;
% 				\draw (C)--(O)--(A)--(B)--cycle (A)--(C);
% 				\draw[dashed] (B)--(O);
% 				\foreach \p/\q in {A/-90,B/0,C/90,O/180}
% 				\fill[black] (\p) circle (1.0pt) ($(\p)+(\q:3.5mm)$) node{$\p$};
% 			\end{tikzpicture}
% 		}
% 	}
% \end{ex}
% \begin{ex}%[1K7YL-3]
% 	Cho hình chóp $S.ABCD$ có đáy là hình vuông cạnh $a$, cạnh bên $SA=a$ và vuông góc với mặt đáy $(ABCD)$. Tính số đo góc giữa hai đường thẳng $SB$ và $CD$.
% 	\choice
% 	{$60^\circ$}
% 	{$90^\circ$}
% 	{\True $45^\circ$}
% 	{$30^\circ$}
% 	\loigiai{
% 		\immini{
% 			Ta có $CD\parallel AB$. \\
% 			Suy ra $(SB,CD)=(SB,AB)=\widehat{SBA}$.\\
% 			Do $\triangle SAB$ vuông tại $A$ có $SA=AB=a$ nên $ \triangle SAB $ vuông cân tại $A$.\\
% 			Suy ra $\widehat{SBA}=45^\circ$.\\
% 			Vậy $(SB,CD)=45^\circ$.
% 		}{
% 			\begin{tikzpicture}[line join=round, line cap=round, scale=0.7]
% 				\def\d{5} \def\r{1.8} \def\h{4} \def\l{2.2}
% 				\coordinate[label={below left}:$B$] (B) at (-3,-3);
% 				\coordinate[label={below right}:$C$] (C) at ($(B)+(\d,0)$);
% 				\coordinate[label={below right}:$A$] (A) at ($(B)+(\l,\r)$);
% 				\coordinate[label={above right}:$D$] (D) at ($(A)+(\d,0)$);
% 				\coordinate[label={above}:$S$] (S) at ($(A)+(0,\h)$);
% 				\draw (S)--(B)--(C)--(D)--(S)--(C);
% 				\draw[dashed] (S)--(A)--(B) (A)--(D);
% 			\end{tikzpicture}
% 		}
% 	}
% \end{ex}
% \begin{ex}%[1K7YL-3]
% 	Cho hình lập phương $ ABCD.A'B'C'D' $. Góc giữa hai đường thẳng $ A'C' $ và $ BD $ bằng
% 	\choice
% 	{$60^\circ$}
% 	{$30^\circ$}
% 	{\True $90^\circ$}
% 	{$45^\circ$}
% 	\loigiai{
% 		\immini{
% 			Ta có \\
% 			$ (A'C',BD)=(A'C',B'D')=90^\circ $.
% 		}{
% 			\begin{tikzpicture}[scale=0.7,font=\footnotesize,line join=round,line cap=round,>=stealth]
% 				\path
% 				(0,0) coordinate (B)
% 				(1,0.8) coordinate (A)
% 				(4,0) coordinate (C)
% 				($(C)-(B)+(A)$) coordinate (D)	
% 				($(A)+(90:3.5)$) coordinate (A')
% 				($(B)-(A)+(A')$) coordinate (B')
% 				($(C)-(A)+(A')$) coordinate (C')
% 				($(D)-(A)+(A')$) coordinate (D')
% 				;
% 				\draw (B')--(B)--(C)--(D)--(D')--(A')--(B')--(C')--(D')--cycle (C)--(C')--(A');
% 				\draw[dashed] (C)--(A)--(D) (A')--(A)--(B);
% 				\foreach \p/\q in {A/160,B/-135,C/-45,D/0,A'/90,B'/180,C'/-20,D'/0}
% 				\fill[black] (\p)node[shift={(\q:3mm)}]{$\p$} circle (1.0pt);	
% 			\end{tikzpicture}
% 		}
% 	}
% \end{ex}
% \begin{ex}%[1K7YL-3]
% 	Cho hình lập phương $ABCD.A'B'C'D'$. Đường thẳng nào sau đây vuông góc với đường thẳng $BC'$?
% 	\choice
% 	{$AD'$}
% 	{$BB'$}
% 	{$AC$}
% 	{\True $A'D$}
% 	\loigiai{
% 		\immini{
% 			Ta có $B'C\perp BC'$ mà $A'D\parallel B'C$ nên $A'D\perp BC'$.
% 		}{
% 			\begin{tikzpicture}[line cap=round, line join=round, font=\footnotesize, >=stealth, scale=0.7]
% 				\tikzset{label style/.style={font=\footnotesize}}
% 				\path (0,0) coordinate (A)
% 				(3,0) coordinate (B)
% 				(1,1) coordinate (D)
% 				($(B)-(A)+(D)$) coordinate (C);
% 				\foreach \x in {A,B,C,D} \path ($(\x)+(0,3)$) coordinate (\x');
% 				\draw (A)--(B)--(C)--(C')--(D')--(A')--(B')--(C')
% 				(A')--(A) (C)--(B')--(B)--(C');
% 				\draw[dashed] (D')--(D)--(C) (A')--(D)--(A) ;
% 				\foreach\x/\y in {A/-135,B/-90,C/-30,D/-80,A'/135,B'/125,C'/45,D'/90} \fill[black] (\x) circle (1pt)+(\y:0.3) node{$\x$};
% 			\end{tikzpicture}
% 		}
% 	}
% \end{ex}
% \begin{ex}%[1K7YL-3]
% 	Cho hình lập phương $ ABCD.A'B'C'D'$. Đường thẳng nào sau đây vuông góc với đường thẳng $A'B?$
% 	\choice
% 	{\True $DC'$}
% 	{$CD$}
% 	{$AC$}
% 	{$CC'$}
% 	\loigiai{
% 		\immini{Ta có $ABA'B'$ là hình vuông nên $ A'B\perp AB'$\\
% 			mà $ AB' \parallel C'D$\\
% 			nên $ A'B\perp C'D.$}
% 		{\begin{tikzpicture}[font=\footnotesize, line join=round, line cap=round, >=stealth]
% 				\def\r{2.5}
% 				\path
% 				(0,0) coordinate (A)
% 				++(0:\r) coordinate (B)
% 				++(30:0.7*\r) coordinate (C)
% 				($(A)+(C)-(B)$) coordinate (D)
% 				\foreach \x in {A,B,C,D}
% 				{($(\x) + (90:\r)$) coordinate (\x')}
% 				;
% 				\draw [dashed]
% 				(A)--(D)--(C) (C')--(D)--(D')
% 				;
% 				\draw
% 				(A)--(A')--(D')--(C')--(C)--(B)--(A)
% 				(B)--(A')--(B')--(B) (A)--(B')--(C')
% 				;
% 				\foreach \x/\y in {A/-150,B/-30,C/0,D/160,A'/180,B'/-30,C'/0,D'/150}
% 				\draw[fill = black] (\x) circle (.03cm) + (\y:.3cm) node {$\x$};
% 		\end{tikzpicture}}
% 	}
% \end{ex}
% \begin{ex}%[1K7BL-3]
% 	Cho hình lập phương $ABCD.A'B'C'D'$ có cạnh bằng $a$. Gọi $M$, $N$ lần lượt là trung điểm của cạnh $AA'$ và $A'B'$. Tính số đo góc giữa hai đường thẳng $MN$ và $BD$.
% 	\choice
% 	{\True $60^\circ$}
% 	{$90^\circ$}
% 	{$30^\circ$}
% 	{$45^\circ$}
% 	\loigiai{
% 		\immini{
% 			Gọi $E$ là trung điểm của $A'D'$.\\
% 			Ta có $\heva{&NE\parallel B'D'\\&B'D'\parallel BD}\Rightarrow NE\parallel BD$.\\
% 			Suy ra $(MN,BD)=(MN,NE)=\widehat{MNE}=60^\circ$ (vì tam giác $MNE$ đều).
% 		}{
% 			\begin{tikzpicture}[scale=1, font=\footnotesize, line join=round, line cap=round, >=stealth]
% 				\path
% 				(0,0) coordinate (A)
% 				(-0.7,-0.7) coordinate (B)
% 				(2.5,0) coordinate (D)
% 				($(B)+(D)-(A)$) coordinate (C)
% 				($(A)+(0,2.5)$) coordinate (A')
% 				($(A')+(B)-(A)$) coordinate (B')
% 				($(A')+(C)-(A)$) coordinate (C')
% 				($(A')+(D)-(A)$) coordinate (D')
% 				($(A)!0.5!(A')$) coordinate (M)
% 				($(A')!0.5!(B')$) coordinate (N)
% 				($(A')!0.5!(D')$) coordinate (E);
% 				\draw (B)--(C)--(D)--(D')--(A')--(B')--(C')--(D') (C)--(C') (B)--(B')--(D') (N)--(E);
% 				\draw[dashed] (A')--(A)--(B) (A)--(D) (M)--(N) (B)--(D) (E)--(M);
% 				\foreach \i/\j in {A/above left,B/below,C/below,D/right,A'/above,B'/left,C'/above,D'/right,M/right,N/above left,E/above}
% 				\draw[fill=black] (\i) circle(1pt) node[\j]{$\i$};
% 			\end{tikzpicture}
% 		}
% 	}
% \end{ex}
% \begin{ex}%[1K7BL-3]
% 	Cho hình chóp $S.ABC$ có $SA=SB=SC=a$, $\widehat{ASB}=\widehat{BSC}$. Khẳng định nào sau đây là đúng?
% 	\choice
% 	{$SA\perp BC$}
% 	{\True $SB\perp AC$}
% 	{$SA\perp SC$}
% 	{$SC\perp AB$}
% 	\loigiai{
% 		$\overrightarrow{SB}\cdot\overrightarrow{AC}=\overrightarrow{SB}\left(\overrightarrow{SC}-\overrightarrow{SA}\right)=\overrightarrow{SB}\cdot\overrightarrow{SC}-\overrightarrow{SB}\cdot\overrightarrow{SA}=SB.SC\cdot\cos\widehat{BSC}-SB\cdot SA\cdot\cos\widehat{ASB}$ \\
% 		$=SA^2\left(\cos \widehat{BSC}-\cos \widehat{ASB}\right)=0 \Rightarrow SB\perp AC$.}
% \end{ex}
% \begin{ex}%[1K7BL-3]
% 	Cho hình lập phương $ABCD.A'B'C'D'$. Góc giữa đường thẳng $AC$ và $B'D'$ bằng
% 	\choice
% 	{$45^\circ$}
% 	{$60^\circ$}
% 	{$120^\circ$}
% 	{\True $90^\circ$}
% 	\loigiai{
% 		Ta có $B'D'\parallel BD$ nên góc giữa $AC$ và $B'D'$ là góc tạo bởi $AC$ và $BD$.\\
% 		Vậy góc giữa $AC$ và $B'D'$ bằng $90^\circ$.
% 	}
% \end{ex}
% \Closesolutionfile{ans}
% \begin{indapan}{10}
% 	{ans/ans-1K7-1-3}
% \end{indapan}
% \begin{dang}{Bài toán thực tế}
% 	\begin{itemize}
% 		\item Đưa bài toán thực tế thành bài toán hình học.
% 		\item Vận dụng các kiến thức để giải quyết vấn đề.
% 	\end{itemize}
% \end{dang}
% \subsubsection{Ví dụ mẫu}
% \begin{vd}[NB]%[DCHT Toán 11 - KNTT -Tên Huỳnh Thanh Chí]%[1K7YL-4]
% 	\immini{Đối với nhà gỗ truyền thống, trong các cấu kiện hoành, quá giang, xà cái, rui, cột tương ứng được đánh số $1$, $2$, $3$, $4$, $5$ như trong vẽ, những cặp cấu kiện nào vuông góc với nhau?}{						\includegraphics[width=0.6\linewidth]{HINHVE/hinh1.png}}	
% 	\loigiai{Các cấu kiện $1$, $2$, $5$ vuông góc với nhau từng đôi một.\\
% 		Các cấu kiện $2$, $3$, $5$ vuông góc với nhau từng đôi một.\\
% 		Cấu kiện $4$ vuông góc với cấu kiện $1$ và cấu kiện $3$.
% 	}
% \end{vd}
% \begin{vd}[NB]%[DCHT Toán 11 - KNTT -Tên Huỳnh Thanh Chí]%[1K7YL-4]
% 	\immini{Hình bên gợi nên hình ảnh $5$ cặp đường thẳng vuông góc. Hãy chỉ ra $5$ cặp đường thẳng đó.}
% 	{\includegraphics[scale=0.5]{HINHVE/hinh2-2.png}}
% 	\loigiai{
% 		Ta có $5$ cặp đường thẳng vuông góc là $a$ và $b$, $a$ và $c$, $b$ và $c$, $a$ và $d$, $c$ và $d$.
% 	}
% \end{vd}
% \subsubsection{Bài tập rèn luyện}
% \centerline{\fcolorbox{red}{yellow!50}{\bf {BÀI TẬP TỰ LUẬN}}}

% \begin{bt}[TH]%[DCHT Toán 11 - KNTT -Tên Huỳnh Thanh Chí]%[1K7BL-4]
% 	\immini{
% 		Hình vẽ bên mô tả một cửa gỗ có dạng hình chữ nhật, ở đó nẹp cửa và mép dưới cửa lần lượt gợi nên hình ảnh hai đường thẳng $d$ và $a$. Điểm $M$ là vị trí giao giữa mép gắn bản lề và mép dưới cánh cửa. Hãy giải thích tại sao khi quay cánh cửa, mép dưới cửa là những đường thẳng $a$ luôn nằm trên mặt phẳng đi qua điểm $M$ cố định và vuông góc với đường thẳng $d$.}{\includegraphics[width=0.5\linewidth]{HINHVE/hinh2-3.jpg}}
% 	\loigiai{
% 		Ta có đường thẳng $d$ luôn vuông góc với sàn nhà. \\
% 		Đường thẳng $a$ nằm sát với sàn nhà nên ta xem như sàn nhà chứa đường thẳng $a$.\\
% 		Do đó suy ra $d\perp a$.
% 	}	
% \end{bt}

% \begin{bt}[TH]%[DCHT Toán 11 - KNTT -Tên Huỳnh Thanh Chí]%[1K7BL-4]
	
% 	\immini{Hình vẽ mô tả một khung thành bóng đá. Cho biết cột của khung thành là đường thẳng $d$ và đường thẳng $a$ là thanh xà ngang của khung thành. Tìm góc tạo bởi đường thẳng $d$ và $a$.}
% 	{
% 		\begin{tikzpicture}[scale=1.2,font=\footnotesize,line join=round,line cap=round,>=stealth]
% 			\def\d{4} %dài
% 			\def\r{3} %rộng
% 			\path 	(0:0) coordinate (B)
% 			++(0:\d) coordinate (C)
% 			++(90:\r) coordinate (D)
% 			++(180:\d) coordinate (A);
% 			\draw[thick] (C)--(D)--(A)--(B)--(-0.2,0)--(-0.2,3.2)--(4.2,3.2)--(4.2,0)--(4,0) ;
% 			\draw[thick] (4.2,3.2)--(5,0)--(4.2,0);
% 			\draw[thin] (0,0)--(4,0);
% 			\fill[pattern=crosshatch] (C)--(D)--(A);
% 			\fill[pattern=north west lines] (4.2,3.2)--(5,0)--(4.2,0);
% 		\end{tikzpicture}
% 	}
% 	\dapso{$(d,a)=90^\circ$}
% 	\loigiai{
% 		Do khung thành bóng đá có dạng hình chữ nhật nên góc giữa cột của khung thành và thanh xà ngang khung thành là $90^\circ$.\\
% 		Vậy ta có góc giữa $d$ và $a$ bằng $90^\circ$.
% 	}
% \end{bt}

% \begin{bt}[TH]%[DCHT Toán 11 - KNTT -Tên Huỳnh Thanh Chí]%[1K7BL-4]
% 	\immini{
% 		Một ô che nắng có viền khung hình lục giác đều $ABCDEF$ song song với mặt bàn và có cạnh $A B$ song song với cạnh bàn $a$. Tính số đo góc hợp bởi đường thẳng $a$ lần lượt với các đường thẳng $AF$, $AE$ và $AD$.
% 	}{\includegraphics[scale=.8]{HINHVE/hinh5}}
% 	\dapso{$(a,AF)=60^\circ$, $(a,AE)=90^\circ$, $(a,AD)=60^\circ$}
% 	\loigiai{
% 		Trong lục giác đều, mỗi góc ở đỉnh bằng $120^\circ$.
% 		\immini{
% 			Vì $a\parallel AB$ nên
% 			\begin{enumerate}[$\bullet$]
% 				\item $(a,AF)=(AB,AF)=180^\circ-\widehat{BAF}=180^\circ-120^\circ=60^\circ$.
% 				\item $(a,AE)=(AB,AE)=90^\circ$ ($\triangle EAB$ có $OE=OB=OA$ nên vuông tại $A$).
% 				\item $(a,AD)=(AB,AD)=\widehat{DAB}=\widehat{OAB}=60^\circ$ ($\triangle OAB$ đều).
% 			\end{enumerate}	 
% 		}{\begin{tikzpicture}[line join=round, line cap=round, every node/.style={scale=0.8}]
% 				\def\a{1.5}
% 				\path 
% 				(0,0)coordinate (O)
% 				\foreach \i/\x in {1/A,2/B,3/C,4/D,5/E,6/F}{
% 					(-180+\i*60:\a) coordinate (\x)
% 				}
% 				;
% 				\draw (A)--(B)--(C)--(D)--(E)--(F)--cycle 
% 				(D)--(A)--(E)--(B)
% 				(-2,-2)--(3,-2)node[below]{$a$}
% 				;
% 				\foreach \t/\g in {A/-90,B/-90,C/0,D/90,E/90,F/180,O/0}{
% 					\draw[fill=red,draw=black] (\t) circle (1pt) node[shift={(\g:7pt)}]{$ \t $};
% 				}
% 				\draw pic[draw, angle radius=2mm]{right angle=B--A--E}; 
% 				%				\tkzMarkSegments[mark=||,size=2pt](O,E O,D O,B O,A)
% 	\end{tikzpicture}}	}
% \end{bt} 

% \begin{bt}[TH]%[DCHT Toán 11 - KNTT -Tên Huỳnh Thanh Chí]%[1K7BL-4]
% 	\immini{Trong hình bên cho $ABB'A'$, $BCC'B'$, $ACA'C'$ là các hình chữ nhật.\\
% 		Chứng minh rằng $AB \perp CC'$, $AA'\perp BC$.}
% 	{\includegraphics[scale=0.6]{HINHVE/hinh2-6.png}}
% 	\loigiai{
% 		\begin{itemize}
% 			\item Ta có $ABB'A'$ là hình chữ nhật nên $ AB\perp BB' $, $BCC'B'$ là hình chữ nhật nên $ BB'\parallel CC'$. \\
% 			Suy ra $AB \perp CC'$.
% 			\item Ta có $BCC'B'$ là hình chữ nhật nên $ BC\perp CC' $, $ACC'A'$ là hình chữ nhật nên $ AA'\parallel CC'$. \\
% 			Suy ra $AA'\perp BC$.
% 		\end{itemize}
% 	}
% \end{bt}