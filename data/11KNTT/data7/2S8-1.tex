\setcounter{section}{21}
\setcounter{dang}{0}
\section{HAI ĐƯỜNG THẲNG VUÔNG GÓC}
\subsection{Trọng tâm kiến thức}
\begin{tomtat}
	\subsubsection{Góc giữa hai đường thẳng}
	\begin{boxdn}
	\immini{
	Góc giữa hai đường thẳng $m$ và $n$ trong không gian, kí hiệu $(m, n)$, là góc giữa hai đường thẳng $a$ và $b$ cùng đi qua một điểm và tương ứng song song với $m$ và $n$.}{	\begin{tikzpicture}[scale=0.6,font=\footnotesize]
	\def \c{0.5}
	\def \d{-0.7}
	\coordinate[label=above:$O$] (O) at (0,0);
	\fill (O)circle(1.5pt);
	\path
	(-2,-2*\c) coordinate (A1)
	(3,3*\c) coordinate (B1)
	(-0.2,-0.2*\c+1) coordinate (A2)
	(2.5,2.5*\c+1) coordinate (B2)
	(-2,-2*\d) coordinate (C1)
	(2,2*\d) coordinate (D1)
	(-0.4,-0.4*\d-0.8) coordinate (C2)
	(1.8,1.8*\d-0.8) coordinate (D2)	
	($(O)!0.8!(B1)$)node[above]{$a$}
	($(O)!0.5!(D1)$)node[above]{$b$}
	($(A2)!0.3!(B2)$)node[above]{$m$}
	($(C2)!0.7!(D2)$)node[above]{$n$}
	;
	\draw (A1)--(B1)
	(A2)--(B2)
	(C1)--(D1)
	(C2)--(D2)
	;
	\end{tikzpicture}}
	\end{boxdn}
	\begin{note}
	\begin{itemize}
	\item Để xác định góc giữa hai đường thẳng chéo nhau $a$ và $b$, ta có thể lấy một điểm $O$ thuộc đường thẳng $a$ và qua đó kẻ đường thẳng $b'$ song song với $b$. Khi đó $(a, b)=\left(a, b'\right)$.
	\item Với hai đường thẳng $a$, $b$ bất kì $0^{\circ} \leq(a, b) \leq 90^{\circ}$.
	\item Nếu a song song hoặc trùng với $a'$ và $b$ song song hoặc trùng với $b'$ thì $(a, b)=\left(a', b'\right)$.
	\end{itemize}
	\end{note}
	\subsubsection{Hai đường thẳng vuông góc}
	\begin{boxdn}
	Hai đường thẳng $a$, $b$ được gọi là vuông góc với nhau, kí hiệu $a \perp b$, nếu góc giữa chúng bằng $90^{\circ}$.
	\end{boxdn}
	\begin{note}
	Nếu đường thẳng $a$ vuông góc với đường thẳng $b$ thì $a$ có vuông góc với các đường thẳng song song với $b$.
	\end{note}
\end{tomtat}
%%%%%%%%%%%%%%
\subsection{Các dạng bài tập}
\begin{dang}{Xác định góc giữa hai đường thẳng trong không gian}
\end{dang}
\subsubsection{Ví dụ minh hoạ}
	\begin{vd}%[1K7BL-2]
	Cho hình hộp $A B C D.A' B' C' D'$ có các mặt là các hình vuông. Tính các góc $\left(A A', C D\right)$, $\left(A' C', B D\right)$, $\left(A C, D C'\right)$.
	\loigiai{
	Vì $C D \parallel AB$ nên $\left(A A', C D\right)=\left(A A', A B\right)=90^{\circ}$.\\ Tứ giác $A C C' A'$ có các cặp cạnh đối bằng nhau nên nó là một hình bình hành. Do đó, $A' C' \parallel AC$. Vậy $\left(A'C', BD\right)=(A C, B D)=90^{\circ}$.\\	
	Tương tự, $D C' \parallel AB'$. Vậy $\left(A C, D C'\right)=\left(AC, AB'\right)$.\\ Tam giác $AB'C$ có ba cạnh bằng nhau (vì là các đường chéo của các hình vuông có độ dài cạnh bằng nhau) nên nó là một tam giác đều. Từ đó ta có, $\left(A C, D C'\right)=\left(AC, AB'\right)=60^{\circ}$.
	\begin{center}
	\begin{tikzpicture}[line cap=round,line join=round, >=stealth,font=\footnotesize,scale=0.7]
	\def \a{-1.5} \def \b{-1}\def \c{3.5} \def \h{4}
	\path (.5,.5)coordinate(A) 
	+(\a,\b)coordinate(B)
	+(\c,0)coordinate(D)
	($(B)+(D)-(A)$)coordinate(C)
	+(0,\h)coordinate(C')
	($(B)+(C')-(C)$)coordinate(B')
	($(A)+(C')-(C)$)coordinate(A')
	($(D)+(C')-(C)$)coordinate(D');
	\coordinate (I) at ($(B)!0.5!(B')$);	
	\coordinate (J) at ($(B)!0.5!(A)$);
	%\draw[ultra thin,color=gray] (-2.5,-1.5) grid (5.5,5.5);
	\draw [dashed] (A)--(B)(D)--(A)--(A')
	(A)--(C) (A)--(B') (B)--(D);
	\draw(B')--(B)--(C)(B')--(C')--(C)--(D)--(D')--(A')--(B')(C')--(D')(C')--(D) (A')--(C');
	\foreach \d/\g in {A/160,B/180,C/-90,D/0,A'/90,B'/180,C'/-130,D'/0}
	\fill[black](\d) circle (1pt)+(\g:.4)node{$\d$};
	\end{tikzpicture}
	\end{center}
	}
\end{vd}
\begin{vd}%[1H3B2]
	Cho hình lập phương $ABCD.A'B'C'D'$ có cạnh là $a$. Tính góc giữa các cặp đường thẳng sau đây
	\begin{multicols}{3}
	\begin{enumerate}
	\item $AB$ và $A'D'$.
	\item $AD$ và $A'C'$.
	\item $BC'$ và $B'D'$.
	\end{enumerate}
	\end{multicols}
	\loigiai{
	\begin{center}
	\begin{tikzpicture}[scale=.35]
	\tkzDefPoints{0/0/A, -3/-2/B, 5/-2/C}
	\coordinate (D) at ($(A)+(C)-(B)$);
	\coordinate (A') at ($(A)+(0,7)$);
	\tkzDefPointsBy[translation=from A to A'](B,C,D){}
	\tkzDrawPolygon(A',B',C',D')
	\tkzDrawSegments[line width=1](B,B' B,C C,D D,D' C,C' B,C' A',C' B',D' C',D)
	\tkzDrawSegments[dashed,line width=1](A,C A,A' A,B A,D B,D)
	\tkzLabelPoints[below right](C,D,C')
	\tkzLabelPoints[above right](D')
	\tkzLabelPoints[above left](A',B',A)
	\tkzLabelPoints[below](B)
	\tkzDrawPoints(A,B,C,D,A',B',C',D')
	\end{tikzpicture}
	\end{center}
	\begin{enumerate}
	\item Ta có $A'D'\parallel AD$ nên $(AB,A'D')=(AB,AD)=\widehat{BAD}=90^\circ$.
	\item Ta có $A'C'\parallel AC$ nên $(AD,A'C')=(AD,AC)=\widehat{DAC}=45^\circ$.
	\item Ta có $B'D'\parallel BD$ nên $(BC',B'D')=(BC',BD)=\widehat{DBC'}$.\\
	Ta có $BD=BC'=C'D=AB\sqrt{2}$ nên $\triangle BDC'$ đều, suy ra $\widehat{DBC'}=60^\circ$.\\
	Vậy $(BC',B'D')=60^\circ$.
	\end{enumerate}}
\end{vd}
\begin{vd}%[1H3K2]
	Cho hình chóp $S.ABC$ có $SA=SB=SC=AB=AC=a\sqrt{2}$ và $BC=2a$. Tính góc giữa hai đường thẳng $AC$ và $SB$.
	\loigiai{\immini{Ta có $SAB$ và $SAC$ là tam giác đều, $ABC$ và $SBC$ là tam giác vuông cân cạnh huyền $BC$.\\
	Gọi $M$, $N$, $P$ lần lượt là trung điểm của $SA$, $AB$, $BC$, ta có $MN\parallel SB$, $NP\parallel AC$ nên $(AC,SB)=(NP,MN)$.\\
	$MN=\dfrac{SB}{2}=\dfrac{a\sqrt{2}}{2}$, $NP=\dfrac{AC}{2}=\dfrac{a\sqrt{2}}{2}$.\\
	$AP=SP=\dfrac{BC}{2}=a$, $SA=a\sqrt{2}$\\Nên $\triangle SAP$ vuông cân tại $P\Rightarrow MP=\dfrac{SA}{2}=\dfrac{a\sqrt{2}}{2}$.\\
	Vậy $\triangle MNP$ đều $\Rightarrow (AC,SB)=(NP,NM)=\widehat{MNP}=60^\circ$.\\
	\textbf{Cách khác:}\\ $\overrightarrow{AC}\cdot\overrightarrow{SB}=(\overrightarrow{SC}-\overrightarrow{SA})\cdot\overrightarrow{SB}=\overrightarrow{SC}\cdot\overrightarrow{SB}-\overrightarrow{SA}\cdot\overrightarrow{SB}$\\
	$=0-SA\cdot SB\cdot\cos\widehat{ASB}=-a^2$.\\
	$\cos(AC,SB)=\dfrac{\big|\overrightarrow{AC}\cdot\overrightarrow{SB}\big|}{AC\cdot SB}=\dfrac{a^2}{a\sqrt{2}\cdot a\sqrt{2}}=\dfrac{1}{2}$ $\Rightarrow (AC,SB)=60^\circ$.
	}{\begin{tikzpicture}
	\tkzDefPoints{0/0/A,2/-1.5/B,5/0/C,3/5/S}
	\coordinate (M) at ($(S)!0.5!(A)$);
	\coordinate (N) at ($(A)!0.5!(B)$);
	\coordinate (P) at ($(B)!0.5!(C)$);	
	\tkzDrawSegments(S,A S,B S,C A,B B,C M,N S,P)
	\tkzDrawSegments[dashed,line width=1](A,C N,P M,P A,P)
	\tkzDrawPoints(S,A,B,C,M,N,P)
	\tkzLabelPoints(B,C,P)
	\tkzLabelPoints[left](S,A,M,N)
	\end{tikzpicture}}}
\end{vd}
%------------
\subsubsection{Bài tập áp dụng}
\begin{bt}%[1H3K2]
	Cho hình chóp $S.ABC$ có đáy $ABC$ là tam giác đều cạnh là $2a$, tam giác $SBC$ vuông cân tại $S$, $SA=2a$.
	\begin{enumerate}
	\item Tính góc giữa hai đường thẳng $SB$ và $AC$.
	\item Gọi $G$ là trọng tâm của tam giác $SBC$. Tính góc tạo bởi $AG$ và $SC$.
	\end{enumerate}
	\loigiai{
	\begin{center}
	\begin{tikzpicture}
	\tkzDefPoints{0/0/A,2/-1.5/B,6/0/C,3/3/S}
	\coordinate (M) at ($(S)!0.5!(A)$);
	\coordinate (N) at ($(A)!0.5!(B)$);
	\coordinate (P) at ($(B)!0.5!(C)$);
	\coordinate (E) at ($(S)!0.5!(C)$);
	\tkzInterLL(B,E)(S,P)	\tkzGetPoint{G}
	\tkzDrawSegments(S,A S,B S,C A,B B,C M,N S,P)
	\tkzDrawSegments[dashed,line width=1](A,C N,P M,P A,P A,G)
	\tkzDrawPoints(S,A,B,C,M,N,P,G)
	\tkzLabelPoints(B,C,P,G)
	\tkzLabelPoints[left](S,A,M,N)
	\end{tikzpicture}
	\end{center}
	\begin{enumerate}
	\item Gọi $M$, $N$, $P$ là trung điểm của $SA$, $AB$, $BC$, ta có $MN\parallel SB$, $NP\parallel AC$ nên $(SB,AC)=(MN,NP)$.\\
	$\triangle ABC$ đều nên $AP=\dfrac{AB\sqrt{3}}{2}=a\sqrt{3}$, $\triangle SBC$ vuông cân tại $S$ nên $SP=\dfrac{BC}{2}=a$, mặt khác có $SA=2a=\sqrt{SP^2+AP^2}$ nên $\triangle SAP$ vuông tại $P$.\\
	$MN=\dfrac{SB}{2}=\dfrac{a\sqrt{2}}{2}$, $NP=\dfrac{AC}{2}=a$, $MP=\dfrac{SA}{2}=a$.\\
	$\cos\widehat{MNP}=\dfrac{MN^2+NP^2-MP^2}{2MN\cdot NP}=\dfrac{\sqrt{2}}{4}\Rightarrow (SB,AC)=(MN,NP)=\widehat{MNP}\approx69^\circ17'$.
	\item Ta có $\overrightarrow{AG}\cdot\overrightarrow{SC}=(\overrightarrow{PG}-\overrightarrow{PA})(\overrightarrow{PC}-\overrightarrow{PS})$\\
	$=\overrightarrow{PG}\cdot\overrightarrow{PC}-\overrightarrow{PG}\cdot\overrightarrow{PS}-\overrightarrow{PA}\cdot\overrightarrow{PC}+\overrightarrow{PA}\cdot\overrightarrow{PS}=0-\dfrac{PS^2}{3}-0+0=\dfrac{a^2}{3}$.\\
	Có $SC=a\sqrt{2}$, $AG=\sqrt{AP^2+PG^2}=\dfrac{2a\sqrt{7}}{3}$.\\
	$\cos(SC,AG)=\dfrac{\big|\overrightarrow{AG}\cdot\overrightarrow{SC}\big|}{SC\cdot AG}=\dfrac{\dfrac{a^2}{3}}{a\sqrt{2}\cdot\dfrac{2a\sqrt{7}}{3}}=\dfrac{1}{2\sqrt{14}}\Rightarrow (SC,AG)\approx82^\circ19'$.
	\end{enumerate}}
\end{bt}
\begin{bt}%[1H3K2]
	Cho tứ diện đều $ABCD$ có cạnh bằng $a$. Gọi $M$ là điểm trên cạnh $AB$ sao cho $BM=3AM$. Tính góc tạo bởi hai đường thẳng $CM$ và $BD$.
	\loigiai{
	Kẻ $MN\parallel BD$, $N\in AD$, ta có $(CM,BD)=(CM,MN)$.\\
	Do $ABCD$ là tứ diện đều nên ta có\\
	$CM=CN=\sqrt{BM^2+BC^2-2\cdot BM\cdot BC\cdot\cos60^\circ}=\dfrac{a\sqrt{13}}{4}$\\
	$MN=\dfrac{BD}{4}=\dfrac{a}{4}$.\\
	$\cos\widehat{CMN}=\dfrac{MC^2+MN^2-CN^2}{2\cdot MC\cdot MN}=\dfrac{1}{2\sqrt{13}}$\\
	$\Rightarrow (CM,BD)=(CM,MN)=\widehat{CMN}\approx82^\circ1'$.
	\begin{center}
	\begin{tikzpicture}[scale=0.75]
	\tkzDefPoints{0/0/B,2/-1.5/C,5/0/D,3/5/A}
	\coordinate (M) at ($(A)!0.3!(B)$);
	\coordinate (N) at ($(A)!0.3!(D)$);
	\tkzDrawSegments(A,B A,C A,D B,C C,D C,M C,N)
	\tkzDrawSegments[dashed,line width=1](B,D M,N)
	\tkzDrawPoints(A,B,C,M,N,D)
	\tkzLabelPoints[left](B)
	\tkzLabelPoints[right](D,N)
	\tkzLabelPoints[left](A,M)
	\tkzLabelPoints[below](C)
	\end{tikzpicture}
	\end{center}
	}
\end{bt}
\begin{bt}%[1H3K2]
	Cho hình chóp $S.ABCD$ có đáy $ABCD$ là hình vuông cạnh bằng $a$, các tam giác $SAB$ và $SAD$ cùng vuông góc tại $A$. Biết rằng $SA=a\sqrt{2}$, gọi $M$ là trung điểm của cạnh $SB$.
	\begin{enumerate}
	\item Tính góc tạo bởi hai vec-tơ $\overrightarrow{AC}$ và $\overrightarrow{SD}$.
	\item Tính góc tạo bởi hai đường thẳng $AM$ và $SC$.
	\end{enumerate}
	\loigiai{
	\begin{center}
	\begin{tikzpicture}[scale=0.85]
	\tkzDefPoints{0/0/A,-2.5/-1.5/B,5/0/D}
	\coordinate (C) at ($(B)+(D)-(A)$);
	\coordinate (S) at ($(A)+(0,4)$);
	\coordinate (M) at ($(B)!0.5!(S)$);
	\tkzDrawSegments(S,D S,B S,C B,C C,D)
	\tkzDrawSegments[dashed,line width=1](S,A A,C A,B A,D A,M)
	\tkzDrawPoints(S,A,B,C,M,D)
	\tkzLabelPoints(B,C,D)
	\tkzLabelPoints[left](S,M)
	\tkzLabelPoints[above right](A)
	\end{tikzpicture}
	\end{center}
	\begin{enumerate}
	\item Có $\overrightarrow{AC}\cdot\overrightarrow{SD}=(\overrightarrow{AB}+\overrightarrow{AD})\cdot(\overrightarrow{AD}-\overrightarrow{AS})=\overrightarrow{AB}\cdot\overrightarrow{AD}-\overrightarrow{AB}\cdot\overrightarrow{AS}+\overrightarrow{AD}^2-\overrightarrow{AD}\cdot\overrightarrow{AS}=a^2$\\
	Do $ABCD$ là hình vuông nên $AC=AB\sqrt{2}=a\sqrt{2}$, $SD=\sqrt{SA^2+AD^2}=a\sqrt{3}$.\\
	Vậy $\cos(\overrightarrow{AC},\overrightarrow{SD})=\dfrac{\overrightarrow{AC}\cdot\overrightarrow{SD}}{AC\cdot SD}=\dfrac{a^2}{a\sqrt{2}\cdot a\sqrt{3}}=\dfrac{1}{\sqrt{6}}\Rightarrow (\overrightarrow{AC},\overrightarrow{SD})\approx65^\circ54'$.
	\item $\triangle SAB$ vuông tại $A$ nên $AM=\dfrac{SB}{2}=\dfrac{\sqrt{AB^2+SA^2}}{2}=\dfrac{a\sqrt{3}}{2}$.\\
	Có $\overrightarrow{SA}\cdot \overrightarrow{AC}=\overrightarrow{SA}\cdot(\overrightarrow{AB}+\overrightarrow{AD})=\overrightarrow{SA}\cdot\overrightarrow{AB}+\overrightarrow{SA}\cdot\overrightarrow{AD}=0$\\
	Nên $SA\perp AC\Rightarrow SC=\sqrt{SA^2+AC^2}=2a$.\\
	$\overrightarrow{AM}\cdot\overrightarrow{SC}=\dfrac{1}{2}(\overrightarrow{AB}+\overrightarrow{AS})\cdot(\overrightarrow{AC}-\overrightarrow{AS})=\dfrac{1}{2}(\overrightarrow{AB}\cdot\overrightarrow{AC}-\overrightarrow{AB}\cdot\overrightarrow{AS}+\overrightarrow{AS}\cdot\overrightarrow{AC}-\overrightarrow{AS}^2)$\\
	$=\dfrac{1}{2}\left(a\cdot a\sqrt{2}\cdot\dfrac{\sqrt{2}}{2}-0+0-2a^2\right)=-\dfrac{a^2}{2}$\\
	$\cos(AM,SC)=\dfrac{\big|\overrightarrow{AM}\cdot\overrightarrow{SC}\big|}{AM\cdot SC}=\dfrac{\dfrac{a^2}{2}}{\dfrac{a\sqrt{3}}{2}\cdot 2a}=\dfrac{1}{2\sqrt{3}}\Rightarrow(AM,SC)\approx73^\circ13'$.
	\end{enumerate}}
\end{bt}
\begin{bt}%[1H3K2]
	Cho hình lập phương $ABCD.A'B'C'D'$ có cạnh bằng $a$, gọi $M$ là trung điểm của $AB$, $N$ là điểm trên cạnh $B'C'$ sao cho $B'N=2C'N$. Tính $\cos$ của góc tạo bởi hai đường thẳng $DM$ và $AN$.
	\loigiai{
	\begin{center}
	\begin{tikzpicture}[scale=.45]
	\tkzDefPoints{0/0/A, -3/-2/B, 5/-2/C}
	\coordinate (D) at ($(A)+(C)-(B)$);
	\coordinate (A') at ($(A)+(0,7)$);
	\tkzDefPointsBy[translation=from A to A'](B,C,D){}
	\tkzDrawPolygon(A',B',C',D')
	\coordinate (M) at ($(A)!0.5!(B)$);
	\coordinate (N) at ($(B')!0.67!(C')$);
	\tkzDrawSegments[line width=1](B,B' B,C C,D D,D' C,C' A',N)
	\tkzDrawSegments[dashed,line width=1](A,A' A,B A,N D,M A,D)
	\tkzLabelPoints[below right](C,D,C',M,N)
	\tkzLabelPoints[above right](D')
	\tkzLabelPoints[above left](A',B',A)
	\tkzLabelPoints[below](B)
	\tkzDrawPoints(A,B,C,D,A',B',C',D',M,N)
	\end{tikzpicture}
	\end{center}
	Ta có $\overrightarrow{AA'}\cdot\overrightarrow{A'N}=\overrightarrow{AA'}\cdot(\overrightarrow{A'B'}+\overrightarrow{B'N})=\overrightarrow{AA'}\cdot\overrightarrow{A'B'}+\overrightarrow{AA'}\cdot\overrightarrow{B'N}=0+\overrightarrow{BB'}\cdot\overrightarrow{B'N}=0$\\
	Vậy $AA'\perp A'N$ nên $AN=\sqrt{AA'^2+A'N^2}=\sqrt{AA'^2+A'B'^2+B'N^2}=\dfrac{a\sqrt{22}}{3}$.\\
	$DM=\sqrt{AM^2+AD^2}=\dfrac{a\sqrt{5}}{2}$.\\
	$\overrightarrow{AN}\cdot\overrightarrow{DM}=(\overrightarrow{AA'}+\overrightarrow{A'B'}+\overrightarrow{B'N})\cdot(\overrightarrow{AM}-\overrightarrow{AD})=\overrightarrow{AA'}\cdot\overrightarrow{AM}+\overrightarrow{A'B'}\cdot\overrightarrow{AM}+\overrightarrow{B'N}\cdot\overrightarrow{AM}-\overrightarrow{AA'}\cdot\overrightarrow{AD}-\overrightarrow{A'B'}\cdot\overrightarrow{AD}-\overrightarrow{B'N}\cdot\overrightarrow{AD}=0+\dfrac{AB^2}{2}+0-0-0-\dfrac{2AD^2}{3}=\dfrac{a^2}{2}-\dfrac{2a^2}{3}=-\dfrac{a^2}{6}$\\
	Có $\cos(AN,DM)=\dfrac{\big|\overrightarrow{AN}\cdot\overrightarrow{DM}\big|}{AN\cdot DM}=\dfrac{\dfrac{a^2}{6}}{\dfrac{a\sqrt{22}}{3}\cdot\dfrac{a\sqrt{5}}{2}}=\dfrac{1}{\sqrt{110}}$}
\end{bt}
\begin{dang}{Sử dụng tính chất vuông góc trong mặt phẳng.}
	Để chứng minh hai đường thẳng $\Delta$ và $\Delta'$ vuông góc với nhau ta có thể sử dụng tính chất vuông góc trong mặt phẳng, cụ thể:
	\begin{itemize}
	\item Tam giác $ABC$ vuông tại $A$ khi và chỉ khi $\widehat{BAC}=90^{\circ} \Leftrightarrow \widehat{ABC}+\widehat{ACB}=90^{\circ}.$
	\item Tam giác $ABC$ vuông tại $A$ khi và chỉ khi $AB^2+AC^2=BC^2.$
	\item Tam giác $ABC$ vuông tại $A$ khi và chỉ khi trung tuyến xuất phát từ $A$ có độ dài bằng nửa cạnh $BC$.
	\item Nếu tam giác $ABC$ cân tại $A$ thì đường trung tuyến xuất phát từ $A$ cũng là đường cao của tam giác.
	\end{itemize}
	Ngoài ra, chúng ta cũng sử dụng tính chất: Nếu $d\perp \Delta$ và $\Delta'\parallel d$ thì $\Delta'$ cũng vuông góc với đường thẳng $\Delta.$
\end{dang}
\subsubsection{Ví dụ minh hoạ}
\begin{vd}%[1H3B2]
	Cho tứ diện $ABCD$ có $AB = AC = AD$, $\widehat{BAC}=\widehat{BAD}=60^{\circ}$. Gọi $M$ và $N$ lần lượt là trung điểm của $AB$ và $CD$, chứng minh rằng $MN$ là đường vuông góc chung của các đường thẳng $AB$ và $CD$.
	\loigiai{
	Từ giả thiết suy ra các tam giác $ABC, ABD$ đều nên $DM=CM$, do đó $\Delta MCD$ cân tại $M$.\\
	Từ đó suy ra $MN\perp CD$.\\
	Mặt khác $\Delta BCD=\Delta ACD$ nên $BN=AN$, do đó $\Delta NAB$ cân tại $N$.\\
	Từ đó suy ra $NM\perp AB$.\\
	Vậy $MN$ là đường vuông góc chung của $AB$ và $CD$.
\begin{center}
	\begin{tikzpicture}[scale=1.0]
	\tkzDefPoints{0/0/B, 1.5/-1.5/C, 5/0/D}
	\coordinate (N) at ($(D)!0.5!(C)$);
	\coordinate (H) at ($(B)!0.65!(N)$); % H thoa man SH=0.4SB
	\coordinate (A) at ($(H)+(0,3.5)$);
	\coordinate (M) at ($(A)!0.5!(B)$);
	\tkzDrawSegments[dashed](B,D B,N M,N D,M)
	\tkzDrawSegments(A,B B,C C,D D,A A,C A,N C,M)
	\tkzLabelPoints[above](A)
	\tkzLabelPoints[left](B,M)
	\tkzLabelPoints[right](D)
	\tkzLabelPoints[below](C)
	\tkzLabelPoints[below](N)
	\tkzMarkRightAngle(A,M,N)
	\tkzMarkRightAngle(M,N,D)
	\tkzDrawPoints(A,B,C,D,M,N)
	\tkzMarkSegments[mark=||](M,A M,B)
	\tkzMarkSegments[mark=|](N,C N,D)
	\end{tikzpicture}
\end{center}
	}
\end{vd}
	\begin{vd}%[1K7BL-3]
	Cho hình hộp $ABCD.A'B'C'D'$.
	\begin{enumerate}[a)]
	\item Xác định vị trí tương đối của hai đường thẳng $AC$ và $B'D'$.
	\item Chứng minh rằng $AC$ và $B'D'$ vuông góc với nhau khi và chỉ khi $ABCD$ là một hình thoi.
	\end{enumerate}
	\loigiai{
	\immini{
	\begin{enumerate}[a)]
	\item Hai đường thẳng $AC$ và $B'D'$ lần lượt thuộc hai mặt phẳng song song $(ABCD)$ và $\left(A' B' C' D'\right)$ nên chúng không có điểm chung, tức là chúng không thể trùng nhau hoặc cắt nhau.\\
	Tứ giác $B D D' B'$ có hai cạnh đối $B B'$ và $D D'$ song song và bằng nhau nên nó là một hình bình hành. Do đó $B'D'$ song song với $B D$. Mặt khác, $B D$ không song song với $AC$ nên $B'D'$ không song song với $AC$.
	Từ những điều trên suy ra $AC$ và $B'D'$ chéo nhau.
	\item Do $B'D'$ song song với $B D$ nên $\left(A C, B' D'\right)=(A C, B D)$. Do đó, $AC$ và $B'D'$ vuông góc với nhau khi và chỉ khi $AC$ và $B D$ vuông góc với nhau. Do $ABCD$ là hình bình hành nên $AC$ vuông góc với $B D$ khi và chỉ khi $ABCD$ là hình thoi.
	\end{enumerate}}{\begin{tikzpicture}[line cap=round,line join=round, >=stealth,font=\footnotesize,scale=0.8]
	\def \a{-2} \def \b{-1}\def \c{4.5} \def \h{4} 
	\path (.5,.5)coordinate(A') 
	+(\a,\b)coordinate(B')
	+(\c,0)coordinate(D')
	($(B')+(D')-(A')$)coordinate(C')
	($(A')!2/3!($(B')!1/2!(C')$)$)coordinate(G)
	+(-1,\h)coordinate(B)
	($(C')+(B)-(B')$)coordinate(C)
	($(A')+(B)-(B')$)coordinate(A)
	($(D')+(B)-(B')$)coordinate(D);
	%	\draw[ultra thin,color=gray] (-2.5,-1.5) grid (7.5,5.5);
	\draw [dashed] (A')--(B')--(D')--(A') (A') -- (A);
	\draw(A) --(C)--(B)--(B')--(C')--(C)--(C')--(D')--(D)--(A)--(B)--(D)(C)--(D);
	\foreach \i/\j in {A/150, B/180,C/-30,D/0,A'/160,B'/-90,C'/-90,D'/0}\fill[black] (\i) circle (1pt) ($(\i)+(\j:4mm)$)node{$\i$};
	\end{tikzpicture}}}
\end{vd}
\begin{vd}%[1H3K2]
	Cho hình chóp $S.ABC$ có $SA=SB=SC=a$, $\widehat{ASB}=60^{\circ}, \widehat{BSC}=90^{\circ}, \widehat{CSA}=120^{\circ}$. Cho $H$ là trung điểm $AC$. Chứng minh rằng:
\begin{listEX}[2]
	\item $SH\perp AC$.
	\item $AB\perp BC$.
\end{listEX}
	\loigiai{
	\immini{
	\begin{enumerate}
	\item Do tam giác $SAC$ cân tại $S$ và $H$ là trung điểm $AC$ nên $SH\perp AC$.
	\item Do $SA=SB=a$ và $\widehat{ASB}=60^{\circ}$ nên $\Delta SAB$ đều. Từ đó suy ra $AB=a$.\hfill (1)\\
	Áp dụng định lý hàm số cos cho các tam giác $SAC$ ta có\\ $AC^2=SA^2+SC^2-2SA.SC.\cos\widehat{ASC}=2a^2-2a^2.\cos120^{\circ}=3a^2.$\hfill (2)\\
	Áp dụng định lý Pitago cho tam giác $SBC$, ta có $BC^2=SB^2+SC^2=2a^2$.\hfill (3)\\
	Từ (1), (2), (3) suy ra $AC^2=AB^2+BC^2 \Rightarrow AB\perp BC$.
	\end{enumerate}
	}
	{
	\begin{tikzpicture}[scale=1.0]
	\tkzDefPoints{0/0/A, 1.5/-1.5/B, 5/0/C}
	\coordinate (H) at ($(A)!0.5!(C)$);
	\coordinate (S) at ($(H)+(0,3.5)$);
	\tkzDrawSegments[dashed](A,C S,H B,H)
	\tkzDrawSegments(A,B B,C S,A S,B S,C)
	\tkzLabelPoints[above](S)
	\tkzLabelPoints[left](A)
	\tkzLabelPoints[right](C)
	\tkzLabelPoints[below](B)
	\tkzLabelPoints[below right](H)
	\tkzMarkRightAngle(A,B,C)
	\tkzDrawPoints(A,B,C,S,H)
	\tkzMarkSegments[mark=||](H,A H,C)
	\end{tikzpicture}
	}
	}
\end{vd}
\begin{vd}%[1H3K2]
	Cho hình chóp $S.ABCD$ có $SA=x$ và tất cả các cạnh còn lại đều bằng 1. Chứng minh rằng $SA\perp SC$.
	\loigiai{
	\immini{Ta có $ABCD$ là hình thoi, gọi $O$ là giao điểm của $AC$ và $BD$ suy ra $O$ là trung điểm của $AC, BD$.\\
	Xét các tam giác $SBD$ và $CBD$, ta có:\\
	$\heva{&SB=CB\\&SD=CD\\&BD \text{ chung}}\Rightarrow \Delta SBD=\Delta CBD$.\\
	Từ đó suy ra $SO=CO=\dfrac{1}{2}AC$.\\
	Vậy tam giác $SAC$ vuông tại $S$ hay $SA\perp SC$.
	}{
	\begin{tikzpicture}[scale=0.8]
	\tkzDefPoints{2/2/D, 0/0/C, 4/0/B}
	\tkzDefPointBy[translation = from C to B](D)
	\tkzGetPoint{A}
	\tkzDefMidPoint(B,D) \tkzGetPoint{O}
	\coordinate (H) at ($(A)!0.7!(C)$);
	\coordinate (x) at ($(A)!0.5!(S)$);
	\coordinate (S) at ($(H)+(0,4)$);
	\tkzDrawSegments(S,A S,B S,C B,C A,B)
	\tkzDrawSegments[dashed](S,D D,C D,A A,C B,D S,O)
	\tkzLabelPoints[right](A,B)
	\tkzLabelPoints[above](x)
	\tkzLabelPoints[below](O)
	\tkzLabelPoints[above](S)
	\tkzLabelPoints[left](C,D)
	\end{tikzpicture}
	}}
\end{vd}
%----------
\subsubsection{Bài tập áp dụng}
\begin{bt}%[1H3B2]
	Cho hình chóp $S.ABCD$ có đáy $ABCD$ là hình vuông tâm $O$ và $SA=SB=SC=SD$. Chứng minh rằng $SO\perp AB$ và $SO\perp AD.$
	\loigiai{
	\immini{Gọi $M,N$ lần lượt là trung điểm của $AB, CD$. Do $\Delta SAB=\Delta SCD$ nên ta suy ra $SM=SN$.\\
	Xét tam giác cân $SMN$ có $O$ là trung điểm $MN$, suy ra $SO\perp MN$.\\
	Mặt khác $AD\parallel MN$ nên $AD\perp SO$.\\
	Tương tự ta chứng minh được $AB\perp SO$.}
	{
	\begin{tikzpicture}[scale=0.8]
	\tkzDefPoints{2.5/2.5/D, 0/0/C, 4/0/B}
	\tkzDefPointBy[translation = from C to B](D)
	\tkzGetPoint{A}
	\tkzDefMidPoint(B,D) \tkzGetPoint{O}
	\tkzDefMidPoint(B,A) \tkzGetPoint{M}
	\tkzDefMidPoint(C,D) \tkzGetPoint{N}
	\coordinate (S) at ($(O)+(0,4)$);
	\tkzDrawSegments(S,A S,B S,C B,C A,B S,M)
	\tkzDrawSegments[dashed](S,D D,C D,A A,C B,D S,O S,N N,M)
	\tkzLabelPoints[right](A,B,M)
	\tkzLabelPoints[below](O)
	\tkzLabelPoints[above](S)
	\tkzLabelPoints[left](C,D,N)
	\tkzMarkSegments[mark=||](S,B S,C S,D S,A)
	\end{tikzpicture}
	}
	}
\end{bt}
\begin{bt}%[1H3B2]
	Cho hình lập phương $ABCD.A'B'C'D'$ có $M, N$ lần lượt là trung điểm $BC, C'D'$. Chứng minh rằng $AM\perp B'N$.
	\loigiai{
	\immini{Gọi $K$ là trung điểm $CD$, khi đó $BK\parallel B'N$. Ta sẽ chứng minh $BK\perp AM$.\\
	Gọi $I$ là giao điểm của $BK$ và $AM$. Do $\Delta ABM=\Delta BCK$ nên:
	$$\widehat{BAI}+\widehat{ABI}=\widehat{IBC}+\widehat{ABI}=90^{\circ}\Rightarrow \widehat{AIB}=90^{\circ}.$$
	Do đó $BK\perp AM$ tại $I$.
	}
	{
	\begin{tikzpicture}[scale=0.85]
	\tkzDefPoints{0/0/A,1.0/1.5/B, 4.5/1.5/C}
	\coordinate (A') at ($(A)+(0,3.5)$);
	\tkzDefPointBy[translation = from B to C](A)
	\tkzGetPoint{D}
	\tkzDefPointsBy[translation = from A to A'](B,C,D){B',C',D'}
	\tkzDefMidPoint(B,C) \tkzGetPoint{M}
	\tkzDefMidPoint(C',D') \tkzGetPoint{N}
	\tkzDefMidPoint(C,D) \tkzGetPoint{K}
	\tkzLabelPoints[left](A,B,A',B')
	\tkzLabelPoints[right](C,D,C',D',N,K)
	\tkzLabelPoints[above](M)
	\tkzDrawSegments(A,A' A,D D,D' D,C C,C' A',B' B',C' C',D' D',A' B',N)
	\tkzDrawSegments[dashed](B,A B,C B,B' A,M B,K)
	\end{tikzpicture}
	}
	}
\end{bt}
\begin{bt}%[1H3B2]
	Cho hình chóp $S.ABCD$ có đáy là hình vuông và có tất cả các cạnh đều bằng $a$. Cho $M$ và $N$ lần lượt là trung điểm của $AD$ và $SD$, chứng minh rằng $MN\perp SC$.
	\loigiai{
	\immini{Từ giả thiết ta có $$\Delta SAC=\Delta DAC \Rightarrow \widehat{ASC}=\widehat{ADC}=90^{\circ}\Rightarrow SA\perp SC.$$
	Mặt khác $MN\parallel SA \Rightarrow MN\perp SC.$
	}
	{
	\begin{tikzpicture}[scale=0.75]
	\tkzDefPoints{2/2/B, 0/0/C, 4/0/D}
	\tkzDefPointBy[translation = from C to D](B)
	\tkzGetPoint{A}
	\tkzDefMidPoint(B,D) \tkzGetPoint{O}
	\tkzDefMidPoint(D,A) \tkzGetPoint{M}
	\tkzDefMidPoint(S,D) \tkzGetPoint{N}
	\coordinate (S) at ($(O)+(0,3.5)$);
	\tkzDrawSegments(S,A M,N S,C A,D D,C S,D)
	\tkzDrawSegments[dashed](S,B B,A A,C B,D B,C)
	\tkzLabelPoints[right](A,D,M)
	\tkzLabelPoints[below](O)
	\tkzLabelPoints[above](S)
	\tkzLabelPoints[left](C,B,N)
	\tkzMarkSegments[mark=||](M,A M,D N,S N,D)
	\end{tikzpicture}
	}
	}
\end{bt}
\begin{bt}%[1H3K2]
	Cho hình chóp $S.ABCD$ có đáy là hình vuông cạnh $2a$, tam giác $SAB$ đều và $SC=2a\sqrt{2}$. Gọi $H, K$ lần lượt là trung điểm của $AB, CD$. Chứng minh rằng $SH\perp AK$.
	\loigiai{
	\immini{Ta có $AK\parallel HC$, do đó chỉ cần chứng minh $SH\perp HC$.\\
	Do $\Delta SAB$ đều cạnh $2a$ nên $SH=\dfrac{AB\sqrt{3}}{2}=a\sqrt{3}$.\\
	Ta có $HC^2=HB^2+BC^2=a^2+4a^2=5a^2$.\\
	Từ đó suy ra $SH^2+HC^2=3a^2+5a^2=8a^2=SC^2.$\\
	Theo định lý Pitago ta có $SH\perp HC$.
	}
	{\vspace*{-3mm}
	\begin{tikzpicture}[scale=0.7]
	\tkzDefPoints{2/2/A, 0/0/B, 4/0/C}
	\tkzDefPointBy[translation = from B to C](A)
	\tkzGetPoint{D}
	\coordinate (H) at ($(A)!0.5!(B)$);
	\coordinate (K) at ($(C)!0.5!(D)$);
	\coordinate (S) at ($(H)+(0,4)$);
	\tkzDrawSegments(S,B S,C S,D B,C C,D)
	\tkzDrawSegments[dashed](A,B H,C A,D S,H A,K S,A)
	\tkzLabelPoints[right](C,D,K)
	\tkzLabelPoints[above](S)
	\tkzLabelPoints[left](A,B,H)
	\end{tikzpicture}
	}
	}
\end{bt}
\begin{bt}%[1H3B2]
	Cho hình chóp $S.ABCD$ có đáy $ABCD$ là hình thang vuông tại $A$ và $B$, $AD=2a, AB=BC=a$. $SA\perp AD$ và $SA\perp AC$. Chứng minh rằng $SC\perp DC$.
	\loigiai{
	\immini{Gọi $I$ là trung điểm $AD\Rightarrow ABCI$ là hình vuông cạnh $a$, do đó $\Delta CID$ vuông cân tại $I$. Từ đó ta có $CD^2=2a^2$. \hfill (1)\\
	Áp dụng định lý Pitago cho các tam giác $SAC, SAD$ ta có:\\
	$SD^2=SA^2+AD^2=SA^2+4a^2$; \hfill (2)\\
	$SC^2=SA^2+AC^2=SA^2+2a^2$. \hfill (3)\\
	Từ (1), (2) và (3) ta suy ra $$SD^2=SC^2+CD^2 \Rightarrow SC\perp CD.$$
	}
	{\vspace*{-3mm}
	\begin{tikzpicture}[scale=0.65]
	\tkzDefPoints{1.5/2/A, 0/0/B, 6/0/K}
	\tkzDefPointBy[translation = from B to K](A)
	\tkzGetPoint{D}
	\coordinate (C) at ($(B)!0.5!(K)$);
	\coordinate (I) at ($(A)!0.5!(D)$);
	\coordinate (S) at ($(A)+(0,4)$);
	\tkzDrawSegments(S,B S,C S,D B,C C,D)
	\tkzDrawSegments[dashed](S,A A,B A,D A,C C,I)
	\tkzLabelPoints[right](C,D)
	\tkzLabelPoints[above](S, I)
	\tkzLabelPoints[left](A,B)
	\tkzMarkRightAngle(S,A,C)
	\tkzMarkRightAngle(S,A,D)
	\end{tikzpicture}
	}
	}
\end{bt}
\begin{bt}%[1H2K1]
	Cho tứ diện $ABCD$ có $AB=x$, tất cả các cạnh còn lại có độ dài bằng $a$. $K$ là trung điểm $AB$ và $I$ là điểm bất kỳ trên cạnh $CD$, chứng minh rằng $IK\perp AB$.
	\loigiai{
	\immini{Xét các tam giác $ACI$ và $BCI$, ta có:\\
	$\heva{&BC=AC\\&CI \text{ chung}\\&\widehat{BCI}=\widehat{ACI}=60^{\circ}}.$\\
	Từ đó suy ra $\Delta ACI=\Delta BCI \Rightarrow IB=IC.$\\
	Xét tam giác cân $IAB$, ta có $K$ là trung điểm $AB$ nên $IK\perp AB$.
	}
	{\vspace*{-3mm}
	\begin{tikzpicture}[scale=0.9]
	\tkzDefPoints{0/0/B, 1.5/-1.5/C, 5/0/D}
	\coordinate (I) at ($(D)!0.4!(C)$);
	\coordinate (H) at ($(B)!0.65!(I)$); % H thoa man SH=0.4SB
	\coordinate (A) at ($(H)+(0,3.5)$);
	\coordinate (K) at ($(A)!0.5!(B)$);
	\tkzDrawSegments[dashed](B,D B,I K,I)
	\tkzDrawSegments(A,B B,C C,D D,A A,C A,I)
	\tkzLabelPoints[above](A)
	\tkzLabelPoints[left](B,K)
	\tkzLabelPoints[right](D)
	\tkzLabelPoints[below](C)
	\tkzLabelPoints[below](I)
	\tkzMarkRightAngle(A,K,I)
	\tkzDrawPoints(A,B,C,D,K,I)
	\tkzMarkSegments[mark=||](K,A K,B)
	\end{tikzpicture}
	}
	}
\end{bt}
\begin{dang}{Hai đường thẳng song song cùng vuông góc với một đường thẳng thứ ba}
	Để chứng minh đường thẳng $a\perp b$, ta chứng minh $a\parallel a'$, ở đó $a'\perp b$. 
\end{dang}
\subsubsection{Ví dụ minh hoạ}
\begin{vd}%[1H2B2]
	Cho hình chóp $S.ABC$ có $AB=AC$. Lấy $M$, $N$ và $P$ lần lượt là trung điểm của các cạnh $BC$, $SB$ và $SC$. Chứng minh rằng $AM$ vuông góc với $NP$.
	\loigiai{
	\hfill
	\immini{
	Do $N$, $P$ lần lượt là trung điểm của các cạnh $SB$ và $SC$ nên $NP$ là đường trung bình của tam giác $SBC$, từ đó suy ra $NP\parallel BC$. \hfill $(1)$\\
	Mặt khác, do tam giác $ABC$ cân tại $A$, suy ra trung tuyến $AM\perp BC$. \hfill $(2)$\\
	Từ $(1)(2)$ suy ra $AM\perp NP$.
	}{\vspace*{-4mm}
	\begin{tikzpicture}[scale=0.75]
	\tkzDefPoints{0/0/A,2/-1.5/B,5/0/C,3/4/S}
	\tkzDefMidPoint(B,C)\tkzGetPoint{M}
	\tkzDefMidPoint(S,B)\tkzGetPoint{N}
	\tkzDefMidPoint(S,C)\tkzGetPoint{P}
	\tkzDrawSegments(S,A S,B S,C A,B B,C N,P)
	\tkzDrawSegments[dashed](A,C A,M)
	\tkzDrawPoints(A,B,C,S,M,N,P)
	\tkzMarkSegments[mark=||](A,B A,C)
	\tkzMarkSegments[mark=|](S,N N,B S,P P,C)
	\tkzMarkSegments[mark=x](B,M M,C)
	\tkzMarkRightAngle(A,M,B)
	\tkzLabelPoints[left](A)
	\tkzLabelPoints[below](B)
	\tkzLabelPoints[right](C)
	\tkzLabelPoints[above](S)
	\tkzLabelPoints(M,N)
	\tkzLabelPoints[right](P)
	\end{tikzpicture}
	}}
\end{vd}
\begin{vd}%[1H3B2]
	Cho hình lăng trụ tam giác $ABC.A'B'C'$ có đáy là tam giác đều. Lấy $M$ là trung điểm của cạnh $BC$. Chứng minh rằng $AM$ vuông góc với $B'C'$.
	\loigiai{
	\hfill
	\immini{
	Do tứ giác $BB'C'C$ là hình bình hành nên $BC\parallel B'C'$. \hfill $(1)$\\
	Mặt khác, do tam giác $ABC$ đều nên $AM\perp BC$. \hfill $(2)$\\
	Từ $(1)(2)$ suy ra $AM\perp B'C'$.
	}{\vspace*{-3mm}
	\begin{tikzpicture}[scale=0.6]
	\tkzDefPoints{0/0/A',2/-1.5/B',5/0/C',1/4/A}
	\coordinate (B) at ($(A)+(B')-(A')$);
	\coordinate (C) at ($(A)+(C')-(A')$);
	\tkzDefMidPoint(B,C)\tkzGetPoint{M}
	\tkzDrawPoints(A',B',C',A,B,C,M)
	\tkzDrawSegments(A',B' B',C' A,B B,C C,A A,A' B,B' C,C' A,M)
	\tkzDrawSegments[dashed](A',C')
	\tkzMarkRightAngle(A,M,B)
	\tkzLabelPoints[left](A')
	\tkzLabelPoints[below right](C,C',M,B',B)
	\tkzLabelPoints[left](A)
	\end{tikzpicture}
	}}
\end{vd}
\begin{vd}%[1H3K2]
	Cho hình lập phương $ABCD.A'B'C'D'$ cạnh $a$. Các điểm $M$, $N$ lần lượt là trung điểm của các cạnh $AB$, $BC$. Trên cạnh $B'C'$ lấy điểm $P$ sao cho $C'P=x$ ($0<x<a$). Trên cạnh $C'D'$ lấy điểm $Q$ sao cho $C'Q=x$. Chứng minh rằng $MN$ vuông góc với $PQ$.
	\loigiai{
	\hfill
	\immini{
	Do tứ giác $BB'D'D$ là hình chữ nhật, suy ra $BD\parallel B'D'$. \hfill $(1)$\\
	Do $ABCD$ là hình vuông, suy ra $BD\perp AC$. \hfill $(2)$\\
	Từ $(1)(2)$ suy ra $B'D'\perp AC$. \hfill $(3)$\\
	Theo bài ra ta có $MN$ là đường trung bình của tam giác $ABC$, suy ra $MN\parallel AC$. \hfill $(4)$\\
	Mặt khác, ta có $\dfrac{C'P}{C'B}=\dfrac{C'Q}{C'D'}=\dfrac{x}{a}$, suy ra $PQ\parallel B'D'$. \hfill $(5)$\\
	Từ $(3)(4)(5)$ ta có $MN\perp PQ$.
	}{\vspace*{-3mm}
	\begin{tikzpicture}[>=stealth, line cap=round, line join=round,scale=0.6]%Hình HĐ 2/ Tr-94
	\tkzDefPoints{0/0/A}
	\tkzDefShiftPoint[A](0:4){B}
	\tkzDefShiftPoint[A](50:2.5){D}
	\coordinate (C) at ($(B)+(D)-(A)$);
	\tkzDefShiftPoint[A](-90:4){A'}
	\tkzDefPointBy[translation = from A to A'](B) \tkzGetPoint{B'}
	\tkzDefPointBy[translation = from A to A'](C) \tkzGetPoint{C'}
	\tkzDefPointBy[translation = from A to A'](D) \tkzGetPoint{D'}
	\tkzDefMidPoint(A,B)\tkzGetPoint{M}
	\tkzDefMidPoint(B,C)\tkzGetPoint{N}
	\coordinate (P) at ($(C')!0.7!(B')$);
	\coordinate (Q) at ($(C')!0.7!(D')$);
	\tkzDrawSegments[dashed](D',D D',A' D',C' P,Q A',C' B',D')
	\tkzDrawSegments(A,B B,C C,D D,A A,A' A',B' B',C' B,B' C,C' M,N B,D A,C)
	\tkzDrawPoints(A,B,C,D,A',B',C',D',M,N,P,Q)
	\tkzLabelPoints[left](A)
	\tkzLabelPoints[right](B)
	\tkzLabelPoints[above](C,D,M,N,Q) 
	\tkzLabelPoints[below](A') 
	\tkzLabelPoints[below](B')
	\tkzLabelPoints[right](C',P) 
	\tkzLabelPoints[left](D')
	\end{tikzpicture}
	}}
\end{vd}
\subsubsection{Bài tập áp dụng}
\begin{bt}%[1H3B2]
	Cho hình lăng trụ đứng $ABC.A'B'C'$. Gọi $G$, $G'$ lần lượt là trọng tâm hai đáy. Chứng minh rằng $GG'$ vuông góc với $BC$.
	\loigiai{
	\hfill
	\immini{
	Dễ dàng chứng minh được $GG'\parallel MM'$ và $MM'\perp BC$. Từ đó suy ra điều phải chứng minh.
	}{
	\begin{tikzpicture}[scale=0.7]
	\tkzDefPoints{0/0/A}
	\tkzDefShiftPoint[A](0:5){C}
	\tkzDefShiftPoint[A](-40:2.5){B}
	\tkzDefShiftPoint[A](-90:4){A'}
	\coordinate (B') at ($(A')+(B)-(A)$);
	\coordinate (C') at ($(A')+(C)-(A)$);
	\tkzDefMidPoint(B,C)\tkzGetPoint{M}
	\tkzDefMidPoint(B',C')\tkzGetPoint{M'}
	\coordinate (G) at ($(A)!0.67!(M)$);
	\coordinate (G') at ($(A')!0.67!(M')$);
	\tkzDrawSegments(A,B B,C C,A A,A' B,B' C,C' A',B' B',C' M,M' A,M)
	\tkzDrawSegments[dashed](A',C' G,G' A',M')
	\tkzDrawPoints(A,B,C,A',B',C',M,M',G,G')
	\tkzLabelPoints[left](A)
	\tkzLabelPoints[below left](B,B',G)
	\tkzLabelPoints[right](C,C') 
	\tkzLabelPoints[left](A')
	\tkzLabelPoints[below](G')
	\tkzLabelPoints[below right](M,M')
	\end{tikzpicture}
	}}
\end{bt}
\begin{bt}%[1H3B2]
	Cho tứ diện đều $ABCD$. Gọi $M$, $N$, $P$ và $Q$ lần lượt là trung điểm các cạnh $AB$, $CD$, $AD$ và $AC$. Chứng minh rằng $MN$ vuông góc với $PQ$.
	\loigiai{
	\hfill
	\immini{
	Theo giả thiết ta có $\triangle ABC=\triangle ABD$, từ đó ta có $MC=MD$, suy ra $\triangle MCD$ cân tại $M$, suy ra $MN\perp CD$. \hfill $(1)$\\
	Cũng theo giả thiết ta có $PQ$ là đường trung bình của tam giác $ACD$, suy ra $PQ\parallel CD$. \hfill $(2)$\\
	Từ $(1)(2)$ suy ra điều phải chứng minh.
	}{
	\begin{tikzpicture}[scale=0.85]
	\tkzDefPoint(0,0){B}
	\tkzDefShiftPoint[B](0:5){D}
	\tkzDefShiftPoint[B](-60:2){C}
	\tkzDefShiftPoint[B](60:4){A}	
	\tkzDefMidPoint(A,B) \tkzGetPoint{M}
	\tkzDefMidPoint(C,D) \tkzGetPoint{N}
	\tkzDefMidPoint(A,D) \tkzGetPoint{P}
	\tkzDefMidPoint(A,C) \tkzGetPoint{Q}
	\tkzInterLL(D,N)(C,P) \tkzGetPoint{G}	
	\tkzDrawSegments(A,B A,C A,D B,C C,D P,Q M,C)
	\tkzDrawSegments[dashed](B,D M,N M,D)
	\tkzDrawPoints(A,B,C,D,M,N,P,Q)
	\tkzLabelPoints[above](A)
	\tkzLabelPoints[left](B)
	\tkzLabelPoints[below](C,N)
	\tkzLabelPoints[right](D)
	\tkzLabelPoints[above left](M)	
	\tkzLabelPoints[right](P)
	\tkzLabelPoints[below left](Q)	
	\end{tikzpicture}
	} }
\end{bt}
\begin{bt}%[1H3K2]
	Cho tứ diện $ABCD$ có $AB=CD=2a$ ($a>0$). Gọi $M$, $N$ lần lượt là trung điểm các cạnh $BC$, $AD$. Biết rằng $MN=a\sqrt{2}$. Chứng minh rằng $AB$ vuông góc với $CD$.
	\loigiai{
	\hfill
	\immini{
	Lấy $P$ là trung điểm của $AC$.\\
	Theo tính chất đường trung bình ta có $PN\parallel=\dfrac{1}{2}CD=a$ và $PM\parallel=\dfrac{1}{2}AB=a$.\hfill $(*)$\\
	Từ đó ta có $MP^2+NP^2=2a^2=MN^2$, vậy tam giác $MNP$ vuông tại $P$ suy ra $MP\perp NP$.\hfill $(**)$\\
	Từ $(*)(**)$ ta có $AB\perp CD$.
	}{
	\begin{tikzpicture}[scale=0.9]
	\tkzDefPoint(0,0){B}
	\tkzDefShiftPoint[B](0:5){D}
	\tkzDefShiftPoint[B](-60:2){C}
	\tkzDefShiftPoint[B](60:4){A}	
	\tkzDefMidPoint(B,C) \tkzGetPoint{M}
	\tkzDefMidPoint(A,D) \tkzGetPoint{N}
	\tkzDefMidPoint(A,C) \tkzGetPoint{P}
	\tkzDrawSegments(A,B A,C A,D B,C C,D M,C M,P N,P)
	\tkzDrawSegments[dashed](B,D M,N)
	\tkzDrawPoints(A,B,C,D,M,N,P)
	\tkzLabelSegment[above left](A,B){$2a$}
	\tkzLabelSegment[below right](C,D){$2a$}
	\tkzLabelPoints[above](A)
	\tkzLabelPoints[left](B)
	\tkzLabelPoints[below](C)
	\tkzLabelPoints[above right](N)
	\tkzLabelPoints[right](D)
	\tkzLabelPoints[left](M)
	\tkzLabelPoints[above left](P)	
	\end{tikzpicture}
	}}
\end{bt}
\begin{bt}%[1H3G2]
	Cho tứ diện $ABCD$, có $AB=CD$. Gọi $G$ là trọng tâm của tam giác $ABD$, $M$ thuộc cạnh $AC$ sao cho $AC=3AM$, các điểm $N$, $P$ lần lượt là trung điểm của các cạnh $AD$, $BC$. Chứng minh rằng $MG$ vuông góc với $NP$.
	\loigiai{
	\hfill
	\immini{
	Lấy $E$, $F$ lần lượt là trung điểm của $AC$, $BD$.\\ Ta có $\dfrac{AM}{AE}=\dfrac{AG}{AF}=\dfrac{2}{3}$, suy ra $MG\parallel EF$. \hfill $(1)$\\
	Mặt khác theo tính chất đường trung bình ta có
	$$EN\parallel=FP=\dfrac{1}{2}CD=a$$
	$$EP\parallel=FN=\dfrac{1}{2}AB=a.$$
	Từ đó suy ra tứ giác $ENFP$ là hình thoi, suy ra $EF\perp NP$.\hfill $(2)$\\
	Từ $(1)$ và $(2)$ suy ra $MG\perp NP$.
	}{
	\begin{tikzpicture}
	\tkzDefPoint(0,0){B}
	\tkzDefShiftPoint[B](0:5){D}
	\tkzDefShiftPoint[B](-60:2){C}
	\tkzDefShiftPoint[B](60:4){A}	
	\tkzDefMidPoint(B,C) \tkzGetPoint{P}
	\tkzDefMidPoint(A,D) \tkzGetPoint{N}
	\tkzDefMidPoint(A,C)\tkzGetPoint{E}
	\tkzDefMidPoint(B,D)\tkzGetPoint{F}
	\coordinate (G) at ($(B)!0.67!(N)$);
	\coordinate (M) at ($(A)!0.33!(C)$);
	\tkzDrawSegments(A,B A,C A,D B,C C,D E,P E,N)
	\tkzDrawSegments[dashed](B,D N,P M,G A,F E,F N,F F,P)
	\tkzDrawPoints(A,B,C,D,M,N,P,G,E,F)
	\tkzLabelSegment[above left](A,B){$2a$}
	\tkzLabelSegment[below right](C,D){$2a$}
	\tkzLabelPoints[above](A)
	\tkzLabelPoints[left](B)
	\tkzLabelPoints[below](C)
	\tkzLabelPoints[above right](N)
	\tkzLabelPoints[right](D)
	\tkzLabelPoints[left](M)
	\tkzLabelPoints[below left](P)	
	\tkzLabelPoints[above right,yshift=5pt](G)	
	\tkzLabelPoints[above left](E)
	\tkzLabelPoints[below](F)
	\end{tikzpicture}
	}
	}
\end{bt}
%%%%%%%%%%%%%%
% \subsection{Bài tập rèn luyện}\BTRL
% \begin{bt}%[1H3B2-3]
% 	Cho hình hộp $A B C D \cdot A' B' C' D'$ có 6 mặt đều là hình vuông và $M$, $N$, $E$, $F$ lần lượt là trung điểm các cạnh $BC$, $BA$, $AA'$, $A'D'$. Tính góc giữa các cặp đường thẳng:
% 	\begin{listEX}[2]
% 	\item $A' C'$ và $B C$;
% 	\item $M N$ và $E F$.
% 	\end{listEX}
% 	\loigiai{
% 	\immini{
% 	\begin{enumerate}
% 	\item Ta có $A C / / A' C'$, suy ra $\left(A' C', B C\right)=(A C, B C)=\widehat{A C B}=45^{\circ}$\\
% 	(tam giác $A C B$ vuông cân tại $B$ ).
% 	\item Ta có $AC\parallel MN$, $AD'\parallel EF$,\\
% 	suy ra $(MN, EF)=\left(AC, AD'\right)=\widehat{CAD'}=60^{\circ}$ (tam giác $A C D'$ có ba cạnh bằng nhau).
% 	\end{enumerate}
% 	}{
% 	\begin{tikzpicture}[line cap=round,line join=round,scale=1.2]
% 	\def\h{-2}
% 	\path 
% 	(0,0) coordinate (B)--+(0,\h) coordinate (B')
% 	(-130:1) coordinate (A)--+(0,\h) coordinate (A')
% 	(2.5,0) coordinate (C)--+(0,\h) coordinate (C')
% 	($(A)+(C)-(B)$) coordinate (D)--+(0,\h) coordinate (D')
% 	($(B)!.5!(C)$) coordinate (M)
% 	($(A)!.5!(B)$) coordinate (N)
% 	($(A)!.5!(A')$) coordinate (E)
% 	($(A')!.5!(D')$) coordinate (F);
% 	\draw (A)--(B)--(C)--(D)--cycle (A)--(A')--(D')--(C')--(C) (D')--(D)
% 	(E)--(F) (C)--(A)--(D')--cycle (M)--(N)
% 	;	
% 	\draw[dashed] (A')--(B')--(C') (B)--(B')
% 	(A')--(C');
% 	\foreach \t/\g in {A'/-90,B'/160,C'/0,D'/-90,A/180,B/90,C/90,D/-40,M/90,N/180,E/180,F/-90}{\draw[fill=red,draw=black] (\t) circle (1pt) node[shift={(\g:7pt)}]{$\t$};
% 	}
% 	\end{tikzpicture}
% 	}
% 	}
% \end{bt}
% \begin{bt}%[1C8B1-2]
% 	\immini{Cho hình hộp $MNPQ.M'N'P'Q'$ có góc giữa hai đường thẳng $MN$ và $MQ$ bằng $70^{\circ}$.
% 	\begin{listEX}
% 	\item Góc giữa hai đường thẳng $M'N'$ và $NP$ bằng góc giữa hai đường thẳng
% 	\choice
% 	{$MN$ và $MP$}
% 	{$MN$ và $MQ$}	
% 	{$MP$ và $NP$}	 
% 	{$NN'$ và $NP$}
% 	\item Tính góc giữa hai đường thẳng $M'N'$ và $NP$.
% 	\end{listEX}}
% 	{
% 	\begin{tikzpicture}[line cap=round,line join=round,font=\footnotesize,>=stealth,scale=.7]
% 	\coordinate[label=left:$M$] (M) at (0,0);
% 	\coordinate[label=below:$Q$] (Q) at(1.5,-1.5);
% 	\coordinate[label=right:$N$] (N) at(5,0);
% 	\coordinate[label=below:$P$] (P) at($(N)+(Q)-(M)$); 
% 	\coordinate[label=above:$M'$] (M') at	($(M)+(1,3)$);
% 	\coordinate[label=left:$Q'$] (Q') at ($(Q)+(M')-(M)$);
% 	\coordinate[label=above:$N'$] (N') at ($(N)+(M')-(M)$);
% 	\coordinate[label=right:$P'$] (P') at ($(Q')+(N')-(M')$);
% 	\draw (M')--(Q')--(P')--(N')--cycle (Q)--(P) (Q')--(P')--(P) (Q) --(M)--(M') (Q')--(Q);
% 	\draw[dashed] (M)--(N)--(N') (N)--(P);
% 	\end{tikzpicture}
% 	}
% 	\loigiai{
% 	\begin{listEX}
% 	\item Vì $M'N'\parallel MN$, $NP\parallel MQ$ nên góc giữa hai đường thẳng $M'N'$ và $NP$ bằng góc giữa hai đường thẳng $MN$ và $MQ$. Chọn phương án B.
% 	\item Vì góc giữa hai đường thẳng $MN$ và $MQ$ bằng $70^{\circ}$ nên góc giữa hai đường thẳng $M'N'$ và $NP$ bằng $70^{\circ}$.
% 	\end{listEX}
% 	}
% \end{bt}
% \begin{bt}%[1H3B2-3]
% 	Cho hình hộp $ABCD.A'B'C'D'$ có $6$ mặt đều là hình vuông. Chứng minh rằng $AB\perp CC'$, $AC \perp B'D'$.
% 	\loigiai{
% 	Ta có $CC' \parallel BB'$, suy ra $\left(AB, CC'\right)=\left(AB, BB'\right)=\widehat{ABB'}=90^{\circ}$. Vậy $A B \perp C C'$.\\
% 	Ta có $B'D' \parallel BD$, suy ra $\left(AC, B'D'\right)=(AC, BD)=90^{\circ}$ (hai đường chéo của hình vuông luôn vuông góc với nhau). Vậy $AC \perp B'D'$.
% 	}
% \end{bt}
% \begin{bt}%[1C8B1-3]
% 	\immini{Cho hình chóp $S.ABCD$ có đáy $ABCD$ là hình thoi. Gọi $M$, $N$ lần lượt là trung điểm của các cạnh $SB$ và $SD$. Chứng minh rằng $AC \perp MN$.}
% 	{
% 	\begin{tikzpicture}[scale=0.7, font=\footnotesize, line join=round, line cap=round, >=stealth]
% 	\def\bc{4} % cạnh BC
% 	\def\ba{2} % cạnh BA
% 	\def\h{4} % đường cao
% 	\def\gocB{35} % góc B của đáy
% 	\coordinate[label=below left:$B$] (B) at (0,0);
% 	\coordinate[label=above left:$A$] (A) at (\gocB:\ba);
% 	\coordinate[label=below:$C$] (C) at (\bc,0);
% 	\coordinate[label=above right:$D$] (D) at ($(C)-(B)+(A)$);
% 	\coordinate [label=above:$S$](S) at (60:5);
% 	\coordinate[label=left:$M$] (M) at ($(S)!1/2!(B)$);
% 	\coordinate[label=right:$N$] (N) at ($(S)!1/2!(D)$);
% 	\draw (S)--(D)--(C)--(B)--cycle (S)--(C);
% 	\draw[dashed] (M)--(N) (A)--(D)--(B)--cycle (A)--(C) (S)--(A);
% 	\foreach \i in {S,A,B,C,D,M,N} \fill[black] (\i) circle (1.5pt);
% 	\end{tikzpicture}
% 	}
% 	\loigiai{
% 	Vì $M$, $N$ lần lượt là trung điểm của $SB$ và $SD$ nên $MN \parallel BD$.\\
% 	Do tứ giác $ABCD$ là hình thoi nên $AC \perp BD$. Từ các kết quả trên, ta có $AC \perp MN$.
% 	}
% \end{bt}
% %%%%%%%%%%%%%%
% \begin{bt}%[1K7BL-2] 
% 	Cho hình lăng trụ $A B C. A' B' C'$ có các đáy là các tam giác đều. Tính góc $\left(A B, B' C'\right)$.
% 	\loigiai{\immini{Vì $B'C' \parallel BC$ nên $(AB,B'C')=(AB,BC)=\widehat{ABC}=60^\circ$.}{\begin{tikzpicture}[line cap=round,line join=round, >=stealth,font=\footnotesize, scale=0.9]
% 	\def \a{1} \def \b{-1} \def \c{3} \def \h{3.0} 
% 	\path (.5,.5)coordinate(A) 
% 	+(\a,\b)coordinate(B)
% 	+(\c,0)coordinate(C)
% 	($(A)!1/2!(C)$)coordinate(M)
% 	($(B)!1/2!(M)$)coordinate(H)
% 	+(0,\h)coordinate(A')
% 	($(B)+(A')-(A)$)coordinate(B')
% 	($(C)+(A')-(A)$)coordinate(C');
% 	%\draw[ultra thin,color=gray] (-.5,-1.5) grid (8.5,5.5);
% 	\draw[thick] (A)--(B)--(C) (A')--(B')--(C')--(A') (A)--(A') (B)--(B') (C)--(C') ;
% 	\draw [dashed] (A)--(C);
% 	\foreach \x/\g in{A/180,B/-90, C/0,A'/90, B'/80, C'/40}
% 	\fill[black](\x) circle (1pt)($(\x)+(\g:2.5mm)$) node{\small $\x$};	
% 	\end{tikzpicture}}}
% \end{bt}
% \begin{bt}%[1K7KL-2] 
% 	Cho hình hộp $A B C D . A' B' C' D'$ có các cạnh bằng nhau. Chứng minh rằng tứ diện $A C B' D'$ có các cặp cạnh đối diện vuông góc với nhau.
% 	\loigiai{\immini{Hình hộp đã cho có các cạnh bằng nhau nên tứ giác $ABCD$ là một hình thoi. Suy ra $AC \perp BD$. Mà $BD\parallel B'D'$ nên $AC \perp B'D'$.\\
% 	Lập luận tương tự cho hai cặp cạnh đối diện còn lại.\\
% 	Vậy tứ diện $ACB'D'$ có các cặp cạnh đối diện vuông góc. }{\begin{tikzpicture}[line cap=round,line join=round, >=stealth,font=\footnotesize,scale=0.6]
% 	\def \a{-2} \def \b{-1}\def \c{4.5} \def \h{4} 
% 	\path (.5,.5)coordinate(A') 
% 	+(\a,\b)coordinate(B')
% 	+(\c,0)coordinate(D')
% 	($(B')+(D')-(A')$)coordinate(C')
% 	($(A')!2/3!($(B')!1/2!(C')$)$)coordinate(G)
% 	+(-1,\h)coordinate(B)
% 	($(C')+(B)-(B')$)coordinate(C)
% 	($(A')+(B)-(B')$)coordinate(A)
% 	($(D')+(B)-(B')$)coordinate(D);
% 	%	\draw[ultra thin,color=gray] (-2.5,-1.5) grid (7.5,5.5);
% 	\draw [dashed] (A')--(B')--(D')--(A') (A') -- (A)--(B') (A)--(D');
% 	\draw(A) --(C)--(B)--(B')--(C')--(C)--(C')--(D')--(D)--(A)--(B)--(D)(C)--(D) (C)--(B') (C)--(D');
% 	\foreach \i/\j in {A/150, B/180,C/-30,D/0,A'/-50,B'/-90,C'/-90,D'/0}\fill[black] (\i) circle (1pt) ($(\i)+(\j:4mm)$)node{$\i$};
% 	\end{tikzpicture}}}
% \end{bt}
% \begin{bt}%[1K7BL-3] 
% 	Cho tứ diện $ABCD$ có $\widehat{CBD}=90^{\circ}$.
% 	\begin{enumerate}[a)]
% 	\item Gọi $M$, $N$ tương ứng là trung điểm của $AB$, $AD$. Chứng minh rằng $MN$ vuông góc với $BC$.
% 	\item Gọi $G$, $K$ tương ứng là trọng tâm của các tam giác $ABC$, $ACD$. Chứng minh rằng $GK$ vuông góc với $BC$.
% 	\end{enumerate}
% 	\loigiai{\immini{\begin{enumerate}[a)]
% 	\item Vì $MN$ là đường trung bình của tam giác $ABD$ nên $MN \parallel BD$ và theo giả thiết $BD \perp BC$ nên ta có $MN \perp BC$.
% 	\item Hai đường thẳng $GK$ và $BD$ cùng nằm trong mặt phẳng $(PBD)$ nên đồng phẳng, đồng thời $\dfrac{PG}{PB}=\dfrac{PK}{PD}=\dfrac{1}{3}$ nên ta có $GK \parallel BD$.\\
% 	Mặt khác, $BD\perp BC$ nên ta cũng có $GK \perp BC$.\end{enumerate}}{\begin{tikzpicture}[>=stealth,line join=round,line cap=round,font=\footnotesize,scale=0.95]
% 	\def \a{1} \def \b{-1} \def \c{5} \def \h{3.5} \def \d{2}
% 	\path (.5,.5)coordinate(B) 
% 	+(\a,\b)coordinate(D)
% 	+(\d,\h)coordinate(A)
% 	+(\c,0)coordinate(C);
% 	\path ($(A)!1/2!(B)$)coordinate(M);	
% 	\path ($(A)!1/2!(D)$)coordinate(N);	
% 	\path ($(A)!1/2!(C)$)coordinate(P);	
% 	\path ($(B)!2/3!(P)$)coordinate(G);	\\
% 	\path ($(D)!2/3!(P)$)coordinate(K);	
% 	\draw [dashed] (B)--(C) (B)--(P) (G)--(K);
% 	\draw (A)--(B)--(D)--(A)--(C)--(D) (M)--(N) (D)--(P);
% 	\foreach \i/\j in {A/150,P/0,G/90,K/0, D/-135,C/-90,M/180,N/0,B/180}\fill[black] (\i) circle (1pt) ($(\i)+(\j:3mm)$)node{$\i$};
% 	\foreach \x/\o/\y/\r in {D/B/C/2} \draw ($(\o)!\r mm!(\x)$)--($($(\o)!\r mm!(\x)$)+($(\o)!\r mm!(\y)$)-(\o)$)--($(\o)!\r mm!(\y)$);
% 	\end{tikzpicture}}}
% \end{bt}
% %----------
% \begin{bt}%[1H3K2-3]
% 	Cho hình chóp $S.ABCD$ có đáy là hình thoi $A B C D$ cạnh $a$. Cho biết $S A=a \sqrt{3}, S A \perp A B$ và $S A \perp A D$. Tính góc giữa $S B$ và $C D$, $S D$ và $C B$.
% 	\loigiai{
% 	\immini{
% 	Vì $CD\parallel AB$ nên $(SB,CD)=(SB,AB)=\widehat{SBA}$.\\
% 	$\triangle SBA$ vuông tại $A$, có $\tan \widehat{SBA}=\dfrac{SA}{AB}=\dfrac{a\sqrt{3}}{a}=\sqrt{3}\Rightarrow \widehat{SBA}=60^\circ$.\\
% 	Tương tự, $CB\parallel AD$ nên $(SD,CB)=(SD,AD)=\widehat{SDA}$. \\
% 	Do $\triangle SAD=\triangle SAB$ (c.g.c) nên $\widehat{SDA}=\widehat{SBA}=60^\circ$.
% 	}{
% 	\begin{tikzpicture}[line join=round, line cap=round,every node/.style={scale=0.8},scale=1.2]
% 	\path 
% 	(0,0) coordinate (A)
% 	(2,0) coordinate (D)
% 	(-130:1) coordinate (B)
% 	($(B)+(D)-(A)$) coordinate (C)
% 	(0,1.5) coordinate (S)
% 	(intersection of A--C and B--D) coordinate (O);
% 	\draw (S)--(B)--(C)--(D)--cycle (S)--(C);
% 	\draw[dashed] (A)--(B) (A)--(D) (A)--(S)
% 	;
% 	\foreach \t/\g in {S/90,A/-90,B/-90,C/-90,D/0}{
% 	\draw[fill=red,draw=black] (\t) circle (1pt) node[shift={(\g:7pt)}]{$ \t $};
% 	}
% 	\draw pic[draw, angle radius=2mm]{right angle=D--A--S}; 
% 	\draw pic[draw, angle radius=2mm]{right angle=S--A--B};
% 	\tkzMarkSegments[mark=x,size=2pt](A,B A,D C,B C,D) 
% 	\end{tikzpicture}
% 	}
% 	}
% \end{bt}
% \begin{bt}%[1H3K2-3]
% 	Cho tứ diện đều $A B C D$. Chứng minh rằng $A B \perp C D$.
% 	\loigiai{
% 	\immini{
% 	Gọi $2x$ là cạnh của tứ diện đều.\\
% 	Gọi $M$, $N$, $P$ lần lượt là trung điểm của $BC$, $AC$, $BD$.\\
% 	Các tam giác $ABD$ và $CBD$ đều có cùng cạnh $2x$ nên các đường cao $AP$ và $CP$ của chúng cũng bằng nhau, và
% 	$AP=CP=\dfrac{2x\sqrt{3}}{2}=x\sqrt{3}$.\\
% 	Khi đó $\triangle PAC$ cân tại $P$, có $PN$ là đường trung tuyến suy ra $PN\perp AC$.
% 	Ta có $\heva{&AB\parallel MN \text{ ($MN$ là đường trung bình của $\triangle ABC$). }\\&CD\parallel MP \text{ ($MP$ là đường trung bình của $\triangle BCD$) }}$
% 	\\$\Rightarrow (AB,CD)=(MN,MP)$.
% 	}{
% 	\begin{tikzpicture}[line join=round, line cap=round,every node/.style={scale=0.8}] 
% 	\path 
% 	(0,0) coordinate (B)
% 	(3,0) coordinate (D)
% 	(1,-1) coordinate (C)
% 	($(B)!.5!(C)$) coordinate (M)
% 	($(A)!.5!(C)$) coordinate (N)
% 	($(B)!.5!(D)$) coordinate (P)
% 	($(D)!2/3!(M)$) coordinate (G)
% 	--+(0,2.5) coordinate (A);
% 	\draw (A)--(B)--(C)--(D)--cycle (A)--(C)
% 	(M)--(N);
% 	\draw[dashed] (B)--(D) (M)--(P)--(N)
% 	(P)--(A) (P)--(C);
% 	\foreach \t/\g in {A/90,B/180,C/-90,D/0,M/180,N/180,P/60}{
% 	\draw[fill=red,draw=black] (\t) circle (1pt) node[shift={(\g:7pt)}]{$ \t $};
% 	}
% 	\draw pic[draw, angle radius=1.5mm]{right angle=C--N--P}; 
% 	\end{tikzpicture}
% 	}
% 	\noindent Xét $\triangle MNP$ có: $MN=\dfrac{AB}{2}=x$, $MP=\dfrac{CD}{2}=x$; $PN=\sqrt{PA^2-AN^2}=\sqrt{\left(x\sqrt{3}\right)^2-x^2}=x\sqrt{2}$.\\
% 	Suy ra $\triangle MPN$ vuông cân tại $M$, suy ra $\widehat{MNP}=90^\circ$.\\
% 	Vậy $(AB,CD)=(MN,MP)=\widehat{MPN}=90^\circ$, suy ra $AB\perp CD$ (đpcm).
% 	}
% \end{bt}
% \begin{bt}%[1H3K2-3]
% 	Cho hình chóp $S . A B C$ có $S A=S B=S C=a$, $\widehat{B S A}=\widehat{C S A}=60^{\circ}$, $\widehat{B S C}=90^{\circ}$. Cho $I$ và $J$ lần lượt là trung điểm của $S A$ và $B C$. Chứng minh rằng $I J \perp S A$ và $I J \perp B C$.
% 	\loigiai{
% 	\immini{
% 	$\triangle SAB$ và $\triangle SAC$ cân tại $S$, có $\widehat{ASB}=\widehat{ASC}=60^\circ$, suy ra $SAB$ và $SAC$ là các tam giác đều. Suy ra $AB=AC=a$.\\
% 	$\triangle SBC$ vuông cân tại $S$, suy ra $BC=SA\sqrt{2}=a\sqrt{2}$.\\
% 	Suy ra $\triangle BAC$ vuông cân tại $A$.\\
% 	$SBC$ và $ABC$ là các tam giác vuông có cùng cạnh huyền $BC$, $J$ là trung điểm $BC\Rightarrow JS=JA \left(=\dfrac{BC}{2}\right)$.\\
% 	$\triangle JSA$ cân tại $S$ có $JI$ là đường trung tuyến, suy ra $JI\perp SA$.\\
% 	$IB$ và $IC$ là các đường cao của tam giác đều có cùng cạnh $a$, suy ra $IB=IC$. \\
% 	$\triangle IBC$ cân tại $I$, có $IJ$ là đường trung tuyến nên $IJ\perp BC$.
% 	}{
% 	\begin{tikzpicture}[line join=round, line cap=round, every node/.style={scale=0.8}]
% 	\path 
% 	(0,0) coordinate (A)
% 	(3,0) coordinate (C)
% 	(1,-.7) coordinate (B)
% 	(1,2.3)coordinate (S)
% 	($(S)!.5!(A)$) coordinate (I)
% 	($(B)!.5!(C)$) coordinate (J)
% 	;
% 	\draw pic[draw, angle radius=2mm,fill=gray!20]{right angle=B--S--C}; 
% 	\draw (S)--(A)--(B)--(C)--cycle (J)--(S)--(B)--(I);
% 	\draw[dashed] (A)--(C) (I)--(C) (I)--(J)--(A);
% 	\foreach \t/\g in {S/90,A/180,B/-90,C/0,I/180,J/-90}{
% 	\draw[fill=red,draw=black] (\t) circle (1pt) node[shift={(\g:7pt)}]{$ \t $};
% 	}
% 	%\draw pic[draw, angle radius=5mm]{angle=A--S--B};
% 	%\draw pic[draw, angle radius=2mm]{right angle=A--S--C}; 
% 	\tkzMarkSegments[mark=x,size=2pt](I,S I,A) 
% 	\tkzMarkSegments[mark=||,size=2pt](J,B J,C)
% 	\draw pic[draw, angle radius=2mm]{right angle=B--A--C}; 
% 	\end{tikzpicture}
% 	}
% 	}
% \end{bt}
% \begin{bt}%[1H3K2-3]
% 	Cho tứ diện đều $A B C D$ cạnh $a$. Gọi $K$ là trung điểm của $C D$. Tính góc giữa hai đường thẳng $A K$ và $B C$.
% 	\loigiai{
% 	\immini{
% 	Gọi $I$ là trung điểm $BD$. Khi đó $IK$ là đường trung bình của $\triangle BCD$ nên $IK\parallel BC$. 
% 	Do đó $(AK,BC)=(AK,IK)$.\\
% 	Xét $\triangle AIK$ có: $\heva{&AI=AK=\dfrac{a\sqrt{3}}{2} \text{ (chiều cao của tam giác đều cạnh $a$)}\\&IK=\dfrac{BC}{2}=\dfrac{a}{2}\text{ ($IK$ là đường trung bình của $\triangle BCD$). }}$\\
% 	$\Rightarrow \cos \widehat{AKI}=\dfrac{KA^2+KI^2-AI^2}{2KI\cdot KA}=\dfrac{\sqrt{3}}{6}\Rightarrow \widehat{AKI}\approx 73^\circ 13'$.
% 	}{
% 	\begin{tikzpicture}[line join=round, line cap=round,every node/.style={scale=0.8}] 
% 	\path 
% 	(0,0) coordinate (B)
% 	(2.4,0) coordinate (D)
% 	(.6,-.7) coordinate (C)
% 	($(C)!.5!(D)$) coordinate (K)
% 	($(B)!.5!(K)$) coordinate (G)
% 	--+(0,2.5) coordinate (A)
% 	($(B)!.5!(D)$) coordinate (I);
% 	\draw (A)--(B)--(C)--(D)--cycle (K)--(A)--(C);
% 	\draw[dashed] (B)--(D) (I)--(A) (I)--(K);
% 	\foreach \t/\g in {A/90,B/180,C/-90,D/0,K/-90,I/-90}{
% 	\draw[fill=red,draw=black] (\t) circle (1pt) node[shift={(\g:7pt)}]{$ \t $};
% 	}
% 	\tkzMarkSegments[mark=||,size=2pt](K,C K,D)
% 	\tkzMarkSegments[mark=x,size=2pt](I,B I,D)
% 	\end{tikzpicture}
% 	}
% 	}
% \end{bt}
% \begin{bt}%[1H3K2-3]
% 	Cho tứ diện $A B C D$. Gọi $M, N$ lần lượt là trung điểm của $B C$ và $A D$. Biết $A B=C D=2 a$ và $M N=a \sqrt{3}$. Tính góc giữa $A B$ và $C D$.
% 	\loigiai{
% 	\immini{
% 	Gọi $P$ là trung điểm $AC$. Ta có $\heva{&MP\parallel AB \text{ ($MP$ là đường trung bình của $\triangle CAB$) }\\&NP\parallel CD \text{ ($NP$ là đường trung bình của $\triangle ACD$). }}$\\
% 	$\Rightarrow (AB,CD)=(MP,NP)$.\\
% 	Xét $\triangle MNP$ có $\heva{&NP=\dfrac{CD}{2}=a\\&MP=\dfrac{AB}{2}=a\\&MN=a\sqrt{3}.}$ \\
% 	$\Rightarrow \cos\widehat{NPM}=\dfrac{PN^2+PM^2-MN^2}{2PN\cdot PM}=-\dfrac{1}{2}\Rightarrow \widehat{NPM}=120^\circ>90^\circ$.\\
% 	Suy ra $(MP,NP)=180^\circ-\widehat{NPM}=180^\circ-120^\circ=60^\circ$.\\
% 	Vậy $(AB,CD)=60^\circ$.
% 	}{
% 	\begin{tikzpicture}[line join=round, line cap=round, every node/.style={scale=0.8}]
% 	\path 
% 	(0,0) coordinate (D)
% 	(3,0) coordinate (B)
% 	(1,-.7) coordinate (C)
% 	(1.3,2.3)coordinate (A)
% 	($(B)!.5!(C)$) coordinate (M)
% 	($(A)!.5!(D)$) coordinate (N)
% 	($(A)!.5!(C)$) coordinate (P)
% 	;
% 	\draw (A)--(D)--(C)--(B)--cycle (A)--(C)
% 	(N)--(P)--(M);
% 	\draw[dashed] (B)--(D) (M)--(N);
% 	\foreach \t/\g in {A/90,D/180,C/-90,B/0,M/-90,N/180,P/40}{
% 	\draw[fill=red,draw=black] (\t) circle (1pt) node[shift={(\g:7pt)}]{$ \t $};
% 	}
% 	\tkzMarkSegments[mark=x,size=2pt](M,B M,C) 
% 	\tkzMarkSegments[mark=||,size=2pt](N,A N,D)
% 	\tkzMarkSegments[mark=|,size=2pt](A,B C,D)
% 	\end{tikzpicture}
% 	}
% 	}
% \end{bt}
% %%%%%%%%%
% \begin{bt}%[1C8B1-3]
% 	\immini{Hình bên gợi nên hình ảnh $5$ cặp đường thẳng vuông góc. Hãy chỉ ra $5$ cặp đường thẳng đó.}
% 	{
% 	\includegraphics[scale=0.4]{hinhve/CD/CD-01-06.png}
% 	}
% 	\loigiai{
% 	Ta có $5$ cặp đường thẳng vuông góc là $a$ và $b$, $a$ và $c$, $b$ và $c$, $a$ và $d$, $c$ và $d$.
% 	}
% \end{bt}
% \begin{bt}%[1H3K2-3]
% 	\immini{
% 	Một ô che nắng có viền khung hình lục giác đều $ABCDEF$ song song với mặt bàn và có cạnh $A B$ song song với cạnh bàn $a$. Tính số đo góc hợp bởi đường thẳng $a$ lần lượt với các đường thẳng $AF$, $AE$ và $AD$.
% 	}{
% 	\includegraphics[scale=.4]{hinhve/CTST/CTST-8_1_1}
% 	}
% 	\loigiai{
% 	Trong lục giác đều, mỗi góc ở đỉnh bằng $120^\circ$.
% 	\immini{
% 	Vì $a\parallel AB$ nên
% 	\begin{enumerate}[$\bullet$]
% 	\item $(a,AF)=(AB,AF)=180^\circ-\widehat{BAF}=180^\circ-120^\circ=60^\circ$.
% 	\item $(a,AE)=(AB,AE)=90^\circ$ ($\triangle EAB$ có $OE=OB=OA$ nên vuông tại $A$).
% 	\item $(a,AD)=(AB,AD)=\widehat{DAB}=\widehat{OAB}=60^\circ$ ($\triangle OAB$ đều).
% 	\end{enumerate}	 
% 	}{
% 	\begin{tikzpicture}[line join=round, line cap=round, every node/.style={scale=0.8}]
% 	\def\a{1.5}
% 	\path 
% 	(0,0)coordinate (O)
% 	\foreach \i/\x in {1/A,2/B,3/C,4/D,5/E,6/F}{
% 	(-180+\i*60:\a) coordinate (\x)
% 	}
% 	;
% 	\draw (A)--(B)--(C)--(D)--(E)--(F)--cycle 
% 	(D)--(A)--(E)--(B)
% 	(-2,-2)--(3,-2)node[below]{$a$}
% 	;
% 	\foreach \t/\g in {A/-90,B/-90,C/0,D/90,E/90,F/180,O/0}{
% 	\draw[fill=red,draw=black] (\t) circle (1pt) node[shift={(\g:7pt)}]{$ \t $};
% 	}
% 	\draw pic[draw, angle radius=2mm]{right angle=B--A--E}; 
% 	\tkzMarkSegments[mark=||,size=2pt](O,E O,D O,B O,A)
% 	\end{tikzpicture}
% 	}	}
% \end{bt} 
% \begin{bt}%[1C8B1-2]
% 	\immini{Cho hình chóp $S.ABCD$ có đáy $A B C D$ là hình bình hành và $\widehat{SAB}=100^{\circ}$ (Hình bên). Tính góc giữa hai đường thẳng:
% 	\begin{listEX}
% 	\item $SA$ và $AB$,
% 	\item $SA$ và $CD$.
% 	\end{listEX}}
% 	{
% 	\begin{tikzpicture}[scale=0.8, font=\footnotesize, line join=round, line cap=round, >=stealth]
% 	\def\bc{4} % cạnh BC
% 	\def\ba{2} % cạnh BA
% 	\def\h{4} % đường cao
% 	\def\gocB{35} % góc B của đáy
% 	\coordinate[label=below left:$B$] (B) at (0,0);
% 	\coordinate[label=above right:$A$] (A) at (\gocB:\ba);
% 	\coordinate[label=below:$C$] (C) at (\bc,0);
% 	\coordinate[label=above right:$D$] (D) at ($(C)-(B)+(A)$);
% 	\coordinate [label=above:$S$](S) at (60:5);
% 	\draw (S)--(D)--(C)--(B)--cycle (S)--(C);
% 	\draw[dashed] (A)--(D)--(B)--cycle (A)--(C) (S)--(A);
% 	\foreach \i in {A,B,C,D,S} \fill[black] (\i) circle (1.5pt);
% 	\draw pic["$ 100^\circ $",draw=black, angle eccentricity=1.7, angle radius=.4cm]
% 	{angle=S--A--B}; 
% 	\end{tikzpicture}
% 	}
% 	\loigiai{
% 	\begin{listEX}
% 	\item Góc giữa hai đường thẳng $SA$ và $AB$ là $\widehat{SAB}=100^{\circ}$.
% 	\item Vì $CD\parallel AB$ nên góc giữa hai đường thẳng $SA$ và $CD$ bằng góc giữa hai đường thẳng $SA$ và $AB$. Suy ra $ (SA,CD)=100^{\circ} $.
% 	\end{listEX}
% 	}
% \end{bt}
% \begin{bt}%[1C8B1-1]
% 	Bạn Hoa nói rằng: \lq\lq  Nếu hai đường thẳng phân biệt $a$ và $b$ cùng vuông góc với đường thẳng $c$ thì $a$ và $b$ vuông góc với nhau\rq\rq. Bạn Hoa nói đúng hay sai? Vì sao?
% 	\loigiai{
% 	Bạn Hoa nói sai vì $a$ và $b$ chưa chắc vuông góc, chúng có thể cắt nhau, chéo nhau hay song song.\\
% 	\immini{Ví dụ. Hình lập phương $ABCD.A'B'C'D'$ có $ AB $ và $ CD $ cùng vuông góc với $ BC $ nhưng $ AB $ và $ CD $ song song.}
% 	{
% 	\begin{tikzpicture}[line cap=round,line join=round,font=\footnotesize,>=stealth,scale=.6]
% 	\coordinate[label=left:$A$] (A) at (0,0);
% 	\coordinate[label=below:$B$] (B) at (-1.5,-1.5);
% 	\coordinate[label=right:$D$] (D) at (4,0);
% 	\coordinate[label=below:$C$] (C) at ($(B)+(D)-(A)$);
% 	\coordinate[label=left:$A'$] (A') at	($(A)+(0,3)$);
% 	\coordinate[label=left:$B'$] (B') at ($(B)+(A')-(A)$);
% 	\coordinate[label=above:$C'$] (D') at ($(D)+(A')-(A)$);
% 	\coordinate[label=above:$D'$] (C') at ($(B')+(D')-(A')$);
% 	\draw (A')--(B')--(C')--(D')--cycle (B)--(C)--(D) (B')--(C')--(C) (D)--(D') (B')--(B);
% 	\draw[dashed] (B)--(A)--(D) (A)--(A');
% 	\end{tikzpicture}
% 	}
% 	}
% \end{bt}