\setcounter{dang}{0}
\section{Phép chiếu vuông góc. Góc giữa đường thẳng và mặt phẳng}
% \subsection{Tóm tắt lý thuyết}
\subsection{Các dạng toán thường gặp}

\begin{dang}{Phép chiếu vuông góc}
	\begin{itemize}
		\item Áp dụng các tính chất đã biết của phép chiếu song song.
		\item Áp dụng định lí ba đường vuông góc.
	\end{itemize}
\end{dang}

\subsubsection{Ví dụ minh hoạ}
\begin{vd}%[DCHT Toán 11 - KNTT -Nguyễn TRần Vũ] %[1K7BN-2]
	Cho hình chóp $S.ABCD$ có $SA\perp (ABC)$, tam giác $ABC$ vuông tại $B$.
	\begin{enumerate}[a)]
		\item Xác định hình chiếu của điểm $S$ trên mặt phẳng $(ABC)$.
		\item Xác định hình chiếu của tam giác $SBC$ trên mặt phẳng $(ABC)$.
		\item Xác định hình chiếu của tam giác $SBC$ trên mặt phẳng $(SAB)$. 
		\item Xác định hình chiếu của điểm $A$ trên mặt phẳng $(SBC)$
	\end{enumerate}
	\loigiai{
		\begin{center}
			\begin{tikzpicture}[scale=1,font=\footnotesize,line join=round,line cap=round,>=stealth] 
				\path 
				(0,0) coordinate (A) 
				(2,-1) coordinate (B) 
				(4,0) coordinate (C) 
				(0,3) coordinate (S)
				($(S)!0.6!(B)$) coordinate (H); 
				\draw (S)--(A)--(B)--(C)--(S)--(B) (A)--(H); 
				\draw[dashed] (A)--(C); 
				\foreach \p/\g in {S/135,A/-135,B/-135,C/-45,H/45} 
				\fill[black](\p) circle (1pt) ($(\p)+(\g:3mm)$) node{$\p$}; 
				\path[dashed] pic[draw,angle radius=3mm]{right angle=A--B--C};
				\path pic[draw,angle radius=3mm]{right angle=A--H--B}; 
			\end{tikzpicture}
		\end{center}
		\begin{enumerate}[a)]
			\item Vì $SA\perp (ABC)$ nên hình chiếu của điểm $S$ trên mặt phẳng $(ABC)$ là điểm $A$.
			\item Vì hình chiếu của các điểm $S$, $B$, $C$ trên mặt phẳng $(ABC)$ lần lượt là $A$, $B$ và $C$ nên hình chiếu của tam giác $SBC$ trên mặt phẳng $(ABC)$ là tam giác $ABC$.
			\item Ta có $\heva{&CB\perp AB\text{ (gt)}\\&CB\perp SA \,(SA\perp (ABC))}\Rightarrow CB\perp (SAB)$.\\
			Do đó hình chiếu của tam giác $SBC$ trên mặt phẳng $(SAB)$ là đoạn thẳng $SB$A.
			\item Trong mặt phẳng $(SAB)$, kẻ $AH\perp SB$, ($H\in SB$).\\
			Từ câu c ta có $CB\perp (SAB)$ mà $AH\subset (SAB)\Rightarrow CB\perp AH$.\\
			Suy ra $AH\perp (SBC)$. Do đó hình chiếu của điểm $A$ trên mặt phẳng $(SBC)$ là điểm $H$.
		\end{enumerate}
	}
\end{vd}

\begin{vd}%[DCHT Toán 11 - KNTT -Nguyễn TRần Vũ] %[1K7KN-2]
	Cho hình chóp $S.ABCD$ có đáy là hình vuông $ABCD$, $SA\perp (ABCD)$. 
	\begin{enumerate}[a)]
		\item Hình chiếu của $SC$ trên mặt phẳng $(ABCD)$.
		\item Hình chiếu của tam giác $SCD$ trên mặt phẳng $(ABCD)$.
		\item Hình chiếu của điểm $C$ trên mặt phẳng $(SAD)$.
		\item Hình chiếu của $SB$ trên mặt phẳng $(SAC)$.
	\end{enumerate}
	\loigiai{
		\begin{center}
			\begin{tikzpicture}[scale=1,font=\footnotesize,line join=round,line cap=round,>=stealth] 
				\path 
				(0,0) coordinate (A) 
				(-1,-2) coordinate (B) 
				($(A)+(4,0)$) coordinate (D) 
				($(B)+(4,0)$) coordinate (C) 
				($(A)+(0,3)$) coordinate (S)
				(intersection of A--C and B--D) coordinate (O); 
				\draw[dashed] (S)--(A)--(C) (B)--(A)--(D)--cycle (S)--(O); 
				\draw (S)--(B)--(C)--(D)--(S)--(C); 
				\foreach \p/\g in {S/135,A/135,B/-135,C/-45,D/45,O/-90} 
				\fill[black](\p) circle (1pt) ($(\p)+(\g:3mm)$) node{$\p$}; 
				\path 
				(A)--(B) node[pos=0.5,sloped,scale=0.7]{$/$}
				(B)--(C) node[pos=0.51,sloped,scale=0.7]{$/$};
				\path[dashed] pic[draw,angle radius=3mm]{right angle=A--B--C}; 
			\end{tikzpicture}
		\end{center}
		\begin{enumerate}[a)]
			\item Do $SA\perp (ABCD)$ nên hình chiếu của $SC$ trên mặt phẳng $(ABCD)$ là $AC$.
			\item Ta thấy $A$, $C$, $D$ lần lượt là hình chiếu của các điểm $S$, $C$ và $D$ nên hình chiếu của tam giác $SCD$ trên mặt phẳng $(ABCD)$ là tam giác $ACD$.
			\item Ta có $\heva{&CD\perp AD\text{ (do $ABCD$ là hình vuông)}\\&CD\perp SA\text{ (do $SA\perp (ABCD)$)}}\Rightarrow CD\perp (SAD)$.\\
			Do đó hình chiếu của điểm $C$ trên mặt phẳng $(SAD)$ là điểm $D$.
			\item Gọi $O=AC\cap BD$. Khi đó $BO\perp AC$.\\
			Mặt khác $SA\perp (ABCD)$ mà $BO\subset (ABCD)\Rightarrow BO\perp SA$.\\
			Do đó $BO\perp (SAC)$. Từ đó suy ra hình chiếu của $SB$ trên mặt phẳng $(SAC)$ là $SO$.
		\end{enumerate}
	}
\end{vd}

\begin{vd}%[DCHT Toán 11 - KNTT -Nguyễn TRần Vũ] %[1K7KN-2]
	Cho hình chóp tứ giác đều $S.ABCD$. Gọi $O$ là giao điểm của $AC$ và $BD$.
	Xác định hình chiếu của điểm $O$ trên mặt phẳng $(SCD)$.
	\loigiai{
		\begin{center}
			\begin{tikzpicture}[scale=1,font=\footnotesize,line join=round,line cap=round,>=stealth] 
				\path 
				(0,0) coordinate (A) 
				(-2.5,-2) coordinate (B) 
				($(A)+(5,0)$) coordinate (D) 
				($(B)+(5,0)$) coordinate (C) 
				($(A)!0.5!(C)$) coordinate (O) 
				($(O)+(0,4)$) coordinate (S)
				($(C)!0.5!(D)$) coordinate (M)
				($(S)!0.7!(M)$) coordinate (H); 
				\draw[dashed] (O)--(S)--(A)--(B) (C)--(A)--(D)--(B) (M)--(O)--(H); 
				\draw (S)--(B)--(C)--(D)--(S)--(C) (S)--(M); 
				\foreach \p/\g in {S/90,A/135,B/-135,C/-45,D/45,O/-90,M/-45,H/45} 
				\fill[black](\p) circle (1pt) ($(\p)+(\g:3mm)$) node{$\p$}; 
				\path 
				(A)--(B) node[pos=0.5,sloped,scale=0.7]{$/$}
				(B)--(C) node[pos=0.5,sloped,scale=0.7]{$/$}
				(C)--(M) node[pos=0.5,sloped,scale=0.7]{$//$}
				(D)--(M) node[pos=0.5,sloped,scale=0.7]{$//$};
				\path[dashed] 
				pic[draw,angle radius=3mm]{right angle=A--B--C}
				pic[draw,angle radius=3mm]{right angle=S--H--O}
				pic[draw,angle radius=3mm]{right angle=O--M--C}; 
			\end{tikzpicture}
		\end{center} 
		\begin{itemize}
			\item Do $S.ABCD$ là hình chóp tứ giác đều nên có $SO\perp (ABCD)$. 
			\item Gọi $M$ là trung điểm của $CD$. Kẻ $OH\perp SM$ tại $H$.
			\item Khi đó $OH\perp SM.\qquad (1)$
			\item $\heva{&CD\perp OM\\&CD\perp SO}\Rightarrow CD\perp (SOM)$ mà $OH\subset (SOM)\Rightarrow OH\perp CD.\qquad (2)$
			\item Từ $(1)$ và $(2)$ suy ra $OH\perp (SCD)$. 
			\item Vậy hình chiếu của điểm $O$ trên mặt phẳng $(SCD)$ là điểm $H$.
		\end{itemize}	
	}
\end{vd}

\subsubsection{Bài tập rèn luyện}

\subsubsection{Bài tập tự luận}

\begin{bt}%[DCHT Toán 11 - KNTT -Nguyễn TRần Vũ] %[1K7BN-2]
	Cho hình chóp $S.ABCD$ có đáy $ABCD$ là hình chữ nhật và $SA\perp (ABCD)$. 
	\begin{enumerate}[a)]
		\item Chứng minh $BC\perp SB$.
		\item Xác định hình chiếu của $A$ trên mặt phẳng $(SCD)$.
	\end{enumerate} 
	\loigiai{
		\begin{center}
			\begin{tikzpicture}[scale=1,font=\footnotesize,line join=round,line cap=round,>=stealth] 
				\path 
				(0,0) coordinate (A) 
				(-1,-2) coordinate (B) 
				($(A)+(4,0)$) coordinate (D) 
				($(B)+(4,0)$) coordinate (C) 
				($(A)+(0,3)$) coordinate (S)
				($(S)!0.6!(D)$) coordinate (H); 
				\draw[dashed] (S)--(A)--(B) (D)--(A)--(H); 
				\draw (S)--(B)--(C)--(D)--(S)--(C); 
				\foreach \p/\g in {S/135,A/135,B/-135,C/-45,D/45,H/45} 
				\fill[black](\p) circle (1pt) ($(\p)+(\g:3mm)$) node{$\p$}; 
				\path[dashed] 
				pic[draw,angle radius=3mm]{right angle=A--B--C}
				pic[draw,angle radius=3mm]{right angle=A--H--D}; 
			\end{tikzpicture}
		\end{center}
		\begin{enumerate}[a)]
			\item Ta thấy $BC\perp AB$ do $ABCD$ là hình chữ nhật.\\
			Mà $AB$ là hình chiếu của $SB$ trên mặt phẳng $(ABCD)$ suy ra $BC\perp SB$ (đpcm).
			\item Kẻ $AH\perp SD$ tại $H$. $\qquad (1)$\\
			Dễ thấy $\heva{&CD\perp AD\\&CD\perp SA}\Rightarrow CD\perp (SAD)$, mà $AH\subset (SAD)\Rightarrow CD\perp AH$. $\qquad (2)$\\
			Từ $(1)$ và $(2)$ suy ra $AH\perp (SCD)$. Hay $H$ là hình chiếu của điểm $A$ trên mặt phẳng $(SCD)$.
		\end{enumerate}
	}
\end{bt}

\begin{bt}%[DCHT Toán 11 - KNTT -Nguyễn TRần Vũ] %[1K7BN-2]
	Cho hình chóp $S.ABCD$, có đáy $ABCD$ là hình thoi tâm $O$ và $SA=SC$, $SB=SD$. 
	\begin{enumerate}[a)]
		\item Xác định hình chiếu của $S$ trên mặt phẳng $(ABCD)$.
		\item Gọi $M$, $N$ lần lượt là trung điểm của $AB$ và $BC$. Chứng minh $MN\perp SD$.
	\end{enumerate}
	\loigiai{
		\begin{center}
			\begin{tikzpicture}[scale=1,font=\footnotesize,line join=round,line cap=round,>=stealth] 
				\path 
				(0,0) coordinate (A) 
				(-2.5,-2) coordinate (B) 
				($(A)+(5,0)$) coordinate (D) 
				($(B)+(5,0)$) coordinate (C) 
				($(A)!0.5!(C)$) coordinate (O) 
				($(O)+(0,4)$) coordinate (S)
				($(A)!0.5!(B)$) coordinate (M)
				($(B)!0.5!(C)$) coordinate (N); 
				\draw[dashed] (O)--(S)--(A)--(B) (C)--(A)--(D)--(B) (M)--(N); 
				\draw (S)--(B)--(C)--(D)--(S)--(C); 
				\foreach \p/\g in {S/90,A/135,B/-135,C/-45,D/45,O/-90,M/90,N/-90} 
				\fill[black](\p) circle (1pt) ($(\p)+(\g:3mm)$) node{$\p$}; 
				\path 
				(A)--(M) node[pos=0.5,sloped,scale=0.7]{$/$}
				(M)--(B) node[pos=0.5,sloped,scale=0.7]{$/$}
				(B)--(N) node[pos=0.5,rotate=30,scale=0.7]{$/$} node[pos=0.51,rotate=30,scale=0.7]{$/$}
				(C)--(N) node[pos=0.5,rotate=30,scale=0.7]{$/$} node[pos=0.52,rotate=30,scale=0.7]{$/$}
				;
				\path[dashed] pic[draw,angle radius=3mm]{right angle=A--O--B}; 
			\end{tikzpicture} 
		\end{center}
		\begin{enumerate}[a)]
			\item Tam giác $SAC$ cân tại $S$ nên $SO\perp AC$.\\
			Tam giác $SBD$ cân tại $S$ nên $SO\perp BD$.\\
			suy ra $SO\perp (ABCD)$. Do đó hình chiếu của điểm $S$ trên mặt phẳng $(ABCD)$ là điểm $O$.
			\item Vì $SO\perp (ABCD)$ nên hình chiếu của $SD$ trên mặt phẳng $(ABCD)$ là $OD$.\\
			Dễ thấy $AC\perp OD$ mà $MN\parallel AC\Rightarrow MN\perp OD$.\\
			Theo định lí ba đường vuông góc suy ra $MN\perp SD$.
		\end{enumerate}	
	}
\end{bt}

\begin{bt}%[DCHT Toán 11 - KNTT -Nguyễn TRần Vũ] %[1K7KN-2]
	Cho tứ diện $ABCD$ có $AB\perp CD$ và $AC\perp BD$.
	\begin{enumerate}[a)]
		\item Xác định hình chiếu của điểm $A$ trên mặt phẳng $(BCD)$.
		\item Chứng minh $BC\perp AD$.
	\end{enumerate}
	\loigiai{
		\begin{center}
			\begin{tikzpicture}[scale=1,font=\footnotesize,line join=round,line cap=round,>=stealth] 
				\path 
				(0,0) coordinate (B) 
				(1.5,-2) coordinate (C) 
				(6,0) coordinate (D)
				(C)--(D) coordinate[pos=0.4] (M) 
				($(B)!0.7!(M)$) coordinate (H)
				(H)--+(0,4) coordinate (A); 
				\draw[dashed] (B)--(D)--(H)--(A) (C)--(H)--(B); 
				\draw (A)--(B)--(C)--(D)--(A)--(C); 
				\foreach \p/\g in {A/90,B/135,C/-90,D/45,H/-90} 
				\fill[black](\p) circle (1pt) ($(\p)+(\g:3mm)$) node{$\p$}; 
				\path pic[draw,angle radius=3mm]{right angle=A--H--M}; 
			\end{tikzpicture}
		\end{center}
		\begin{enumerate}[a)]
			\item Hạ $AH\perp (BCD)$ tại $H$.\\
			Khi đó $\heva{&CD\perp AB\\&CD\perp AH}\Rightarrow CD\perp BH$.\\
			Tương tự $\heva{&BD\perp AC\\&BD\perp AH}\Rightarrow BD\perp CH$.\\
			Do đó hình chiếu của điểm $A$ trên mặt phẳng $(BCD)$ là trực tâm $H$ của tam giác $BCD$.
			\item Vì $H$ là trực tâm của tam giác $BCD$ nên $BC\perp HD$. Mà $HD$ là hình chiếu của $AD$ trên mặt phẳng $(BCD)$ suy ra $BC\perp AD$ (đpcm).
		\end{enumerate}
	}
\end{bt}

\begin{bt}%[DCHT Toán 11 - KNTT -Nguyễn TRần Vũ] %[1K7KN-2]
	Cho tứ diện $ABCD$ có $AB\perp (BCD)$, $BC\perp CD$. Gọi $M$, $N$ lần lượt là hình chiếu vuông góc của $B$ trên $AC$ và $AD$. Chứng minh rằng:
	\begin{enumerate}[a)]
		\item $CD\perp BM$.
		\item $BM\perp MN$.
	\end{enumerate}
	\loigiai{
		\begin{center}
			\begin{tikzpicture}[scale=1,font=\footnotesize,line join=round,line cap=round,>=stealth] 
				\path 
				(0,0) coordinate (B) 
				(2,-1) coordinate (C) 
				(4,0) coordinate (D) 
				(B)--+(0,3) coordinate (A)
				($(A)!0.6!(C)$) coordinate (M)
				($(A)!0.4!(D)$) coordinate (N); 
				\draw (A)--(B)--(C)--(D)--(A)--(C) (B)--(M)--(N); 
				\draw[dashed] (D)--(B)--(N); 
				\foreach \p/\g in {A/135,B/-135,C/-135,D/-45,M/0,N/45} 
				\fill[black](\p) circle (1pt) ($(\p)+(\g:3mm)$) node{$\p$}; 
				\path[dashed] 
				pic[draw,angle radius=3mm]{right angle=B--C--D}
				pic[draw,angle radius=3mm]{right angle=A--N--B};
				\path pic[draw,angle radius=3mm]{right angle=B--M--C}; 
			\end{tikzpicture}
		\end{center}	
		\begin{enumerate}[a)]
			\item Ta có $\heva{&CD\perp AB\\&CD\perp BC}\Rightarrow CD\perp (ABC)$. Mà $BM\subset (ABC)$ suy ra $CD \perp BM$.
			\item Ta có $\heva{&BM\perp AC\\&BM\perp CD}\Rightarrow BM\perp (ACD)\Rightarrow BM\perp MN$.
		\end{enumerate}	
	}
\end{bt}

\begin{bt}%[DCHT Toán 11 - KNTT -Nguyễn TRần Vũ] %[1K7GN-2]
	Cho hình chóp $S.ABCD$, có đáy $ABCD$ là hình thang vuông tại $A$ và $D$; Có $SA\perp (ABCD)$ và $SA=AB=BC=a$, $AD=2a$.
	\begin{enumerate}[a)]
		\item Chứng minh tam giác $SBC$ vuông và tính diện tích của nó.
		\item Chứng minh tam giác $SCD$ vuông và tính diện tích của nó.
	\end{enumerate}
	\loigiai{
		\begin{center}
			\begin{tikzpicture}[scale=1,font=\footnotesize,line join=round,line cap=round,>=stealth] 
				\path 
				(0,0) coordinate (A) 
				(-1,-1.5) coordinate (B) 
				(1,-1.5) coordinate (C)
				(4,0) coordinate (D) 
				(A)--+(0,3) coordinate (S)
				($(A)!0.5!(D)$) coordinate (E);
				\draw[dashed] (S)--(A)--(B) (E)--(C)--(A)--(D); 
				\draw (S)--(B)--(C)--(S)--(D)--(C); 
				\foreach \p/\g in {S/135,A/180,B/-135,C/-45,D/0,E/45} 
				\fill[black](\p) circle (1pt) ($(\p)+(\g:3mm)$) node{$\p$}; 
				\path[dashed] 
				pic[draw,angle radius=3mm]{right angle=B--A--D}
				pic[draw,angle radius=3mm]{right angle=A--B--C};
			\end{tikzpicture}
		\end{center}
		\begin{enumerate}[a)]
			\item Chứng minh tam giác $SBC$ vuông và tính diện tích của nó.\\
			\begin{itemize}
				\item Do $BC\perp AB$ (giả thiết) mà $AB$ là hình chiếu của $SB$ trên mặt phẳng $(ABCD)$ nên $BC\perp SB$. Do đó tam giác $SBC$ vuông tại $B$.
				\item Vì $\triangle SAB$ vuông cân tại $A$ suy ra $SB=a\sqrt{2}$.\\
				Khi đó $S_{\triangle SBC}=\dfrac{1}{2}\cdot SB\cdot BC=\dfrac{1}{2}\cdot a\sqrt{2} \cdot a=\dfrac{a^2\sqrt{2}}{2}$.
			\end{itemize}
			\item Chứng minh tam giác $SCD$ vuông và tính diện tích của nó.\\
			\begin{itemize}
				\item Gọi $E$ là trung điểm của $AD$ khi đó $ABCE$ là hình vuông suy ra $\widehat{ACE}=45^\circ$. $\qquad (1)$
				\item Tam giác $CED$ vuông cân tại $E$ suy ra $\widehat{ECD}=45^\circ$. $\qquad (2)$
				\item Từ $(1)$ và $(2)$ suy ra $\widehat{ACD}=90^\circ$ hay $CD\perp AC$. Mà $AC$ là hình chiếu của $SC$ trên mặt phẳng $(ABCD)$ nên $CD\perp SC$. Hay $\triangle SCD$ vuông tại $C$.
				\item $\triangle SAC$ vuông tại $A$, có $SA=a$ và $AC=a\sqrt{2}$ suy ra $SC=\sqrt{SA^2+AC^2}=\sqrt{a^2+\left(a\sqrt{2}\right)^2}=a\sqrt{3}$.
				\item $\triangle CED$ vuông cân tại $E$, có cạnh góc vuông bằng $a$ nên $CD=a\sqrt{2}$.
				\item Vậy $S_{\triangle SCD}=\dfrac{1}{2} \cdot SC\cdot CD=\dfrac{1}{2} \cdot a\sqrt{3} \cdot a\sqrt{2}=\dfrac{a^2\sqrt{6}}{2}$.
			\end{itemize}
		\end{enumerate}	
	}
\end{bt}

\subsubsection{Bài tập trắc nghiệm}

\Opensolutionfile{ans}[ans/ans-1K7-24-Dang3]

\begin{ex}%[DCHT Toán 11 - KNTT -Nguyễn TRần Vũ] %[1K7YN-2]
	Cho hình chóp $S.ABC$ có $SA$ vuông góc với đáy. Gọi $M$ là trung điểm của $SC$. Khi đó hình chiếu của điểm $M$ trên mặt phẳng $(ABC)$ là
	\choice
	{điểm $A$}
	{trung điểm của $AB$}
	{\True trung điểm của $AC$}
	{là trọng tâm của tam giác $ABC$}
	\loigiai{
		\begin{center}
			\begin{tikzpicture}[scale=1,font=\footnotesize,line join=round,line cap=round,>=stealth] 
				\path 
				(0,0) coordinate (A) 
				(1.5,-1.5) coordinate (B) 
				(4,0) coordinate (C) 
				(0,3) coordinate (S)
				($(S)!0.5!(C)$) coordinate (M)
				($(A)!0.5!(C)$) coordinate (N); 
				\draw (S)--(A)--(B)--(C)--(S)--(B); 
				\draw[dashed] (A)--(C) (M)--(N); 
				\foreach \p/\g in {S/135,A/-135,B/-135,C/-45,M/45,N/-90} 
				\fill[black](\p) circle (1pt) ($(\p)+(\g:3mm)$) node{$\p$}; 
			\end{tikzpicture}
		\end{center}
		Gọi $N$ là trung điểm của $AC$. Khi đó $MN$ là đường trung bình của tam giác $SAC$. Suy ra $MN\parallel SA$.\\
		Mà $SA\perp (ABC)$ nên suy ra $MN\perp (ABC)$.\\
		Vậy hình chiếu của điểm $M$ trên mặt phẳng $(ABC)$ là trung điểm $N$ của cạnh $AC$.	
	}
\end{ex}

\begin{ex}%[DCHT Toán 11 - KNTT -Nguyễn TRần Vũ] %[1K7YN-2]
	Cho hình lập phương $ABCD.A'B'C'D'$. Hình chiếu vuông góc của tam giác $AA'B'$ trên mặt phẳng $(ABCD)$ là
	\choice
	{tam giác $ABC$}
	{tam giác $DD'C'$}
	{\True đoạn thẳng $AB$}
	{đoạn thẳng $AD$}
	\loigiai{
		\begin{center}
			\begin{tikzpicture}[scale=1,font=\footnotesize,line join=round,line cap=round,>=stealth] 
				\path 
				(0:0) coordinate (A) 
				(-1.2,-0.6) coordinate (B) 
				($(B)+(2,0)$) coordinate (C) 
				($(A)+(2,0)$) coordinate (D) 
				($(A)+(0,2)$) coordinate (A') 
				($(B)+(0,2)$) coordinate (B') 
				($(C)+(0,2)$) coordinate (C') 
				($(D)+(0,2)$) coordinate (D'); 
				\draw (A')--(B')--(C')--(D')--cycle (B')--(B)--(C)--(D)--(D') (C)--(C'); 
				\draw[dashed] (B)--(A)--(D) (A)--(A');
				\foreach \p/\g in {A/135,B/-90,C/-90,D/0,A'/90,B'/135,C'/-45,D'/45} \fill[black](\p) circle (1pt) ($(\p)+(\g:3mm)$) node{$\p$}; 
			\end{tikzpicture} 
		\end{center}
		Do $AA'\perp (ABCD)$ nên hình chiếu vuông góc của tam giác $AA'B'$ trên mặt phẳng $(ABCD)$ là đoạn thẳng $AB$.	
	}
\end{ex}

\begin{ex}%[DCHT Toán 11 - KNTT -Nguyễn TRần Vũ] %[1K7YN-2]
	Cho hình hộp chữ nhật $ABCD.A'B'C'D'$. Hình chiếu vuông góc của tam giác $AB'D'$ là
	\choice
	{\True tam giác $ABD$}
	{tam giác $ABC$}
	{tam giác $ACD$}
	{tam giác $BCD$}
	\loigiai{
		\begin{center}
			\begin{tikzpicture}[scale=1,font=\footnotesize,line join=round,line cap=round,>=stealth] 
				\path 
				(0:0) coordinate (A) 
				(-1,-1) coordinate (B) 
				($(B)+(3,0)$) coordinate (C) 
				($(A)+(3,0)$) coordinate (D) 
				($(A)+(0,2)$) coordinate (A') 
				($(B)+(0,2)$) coordinate (B') 
				($(C)+(0,2)$) coordinate (C') 
				($(D)+(0,2)$) coordinate (D'); 
				\draw (A')--(B')--(C')--(D')--cycle (B')--(B)--(C)--(D)--(D') (C)--(C') (B')--(D'); 
				\draw[dashed] (B)--(A)--(D)--(B) (A)--(A') (B')--(A)--(D');
				\foreach \p/\g in {A/-45,B/-90,C/-90,D/0,A'/90,B'/135,C'/-45,D'/45} \fill[black](\p) circle (1pt) ($(\p)+(\g:3mm)$) node{$\p$}; 
			\end{tikzpicture}
		\end{center}
		Do $AA'\perp (ABCD)$ nên hình chiếu vuông góc của tam $AB'D'$ trên mặt phẳng $(ABCD)$ là tam giác $ABD$.	
	}
\end{ex}

\begin{ex}%[DCHT Toán 11 - KNTT -Nguyễn TRần Vũ] %[1K7YN-2]
	Cho hình chóp tứ giác đều $S.ABCD$, có $O$ là giao điểm của $AC$ và $BD$. Hình chiếu của tam giác $SAB$ trên mặt phẳng $(ABCD)$ là?
	\choice
	{\True Tam giác $OAB$}
	{Tam giác $OBC$}
	{Tam giác $OCD$}
	{Tam giác $OAD$}
	\loigiai{
		\begin{center}
			\begin{tikzpicture}[scale=1,font=\footnotesize,line join=round,line cap=round,>=stealth] 
				\path 
				(0,0) coordinate (A) 
				(-2.5,-2) coordinate (B) 
				($(A)+(5,0)$) coordinate (D) 
				($(B)+(5,0)$) coordinate (C) 
				($(A)!0.5!(C)$) coordinate (O) 
				($(O)+(0,4)$) coordinate (S); 
				\draw[dashed] (O)--(S)--(A)--(B) (C)--(A)--(D)--(B); 
				\draw (S)--(B)--(C)--(D)--(S)--(C); 
				\foreach \p/\g in {S/90,A/135,B/-135,C/-45,D/45,O/-90} 
				\fill[black](\p) circle (1pt) ($(\p)+(\g:3mm)$) node{$\p$}; 
				\path 
				(A)--(B) node[pos=0.5,rotate=30,scale=0.7]{$//$}
				(B)--(C) node[pos=0.5,rotate=-30,scale=0.7]{$//$};
				\path[dashed] 
				pic[draw,angle radius=3mm]{right angle=A--B--C} 
				pic[draw,angle radius=3mm]{right angle=S--O--A}; 
			\end{tikzpicture} 
		\end{center}
		Do $S.ABCD$ là hình chóp tứ giác đều nên $SO\perp (ABCD)$.\\
		Suy ra hình chiếu của tam giác $SAB$ trên mặt phẳng $(ABCD)$ là tam giác $OAB$.	
	}
\end{ex}

\begin{ex}%[DCHT Toán 11 - KNTT -Nguyễn TRần Vũ] %[1K7BN-2]
	Cho hình chóp $S.ABC$ có đáy $ABC$ là tam giác vuông cân tại $B$ và $SA$ vuông góc với đáy. Hình chiếu vuông góc của điểm $B$ trên mặt phẳng $(SAC)$ là
	\choice
	{điểm $A$}
	{điểm $S$}
	{trung điểm $E$ của cạnh $SC$}
	{\True trung điểm $M$ của cạnh $AC$}
	\loigiai{
		\begin{center}
			\begin{tikzpicture}[scale=1,font=\footnotesize,line join=round,line cap=round,>=stealth] 
				\path 
				(0,0) coordinate (A) 
				(1.5,-1.5) coordinate (B) 
				(4,0) coordinate (C) 
				(0,3) coordinate (S)
				($(A)!0.5!(C)$) coordinate (M); 
				\draw (S)--(A)--(B)--(C)--(S)--(B); 
				\draw[dashed] (A)--(C) (B)--(M); 
				\foreach \p/\g in {S/135,A/-135,B/-135,C/-45,M/45} 
				\fill[black](\p) circle (1pt) ($(\p)+(\g:3mm)$) node{$\p$}; 
				\path[dashed] 
				pic[draw,angle radius=3mm]{right angle=A--B--C}
				pic[draw,angle radius=3mm]{right angle=A--M--B};
				\path 
				(A)--(B) node[pos=0.5,rotate=-30,scale=0.7]{$/$}
				(B)--(C) node[pos=0.5,rotate=30,scale=0.7]{$/$};
			\end{tikzpicture}
		\end{center}
		\begin{itemize}
			\item Gọi $M$ là trung điểm của $AC$.
			\item Do tam giác $ABC$ vuông cân tại $B$ nên $BM\perp AC$.
			\item Mặt khác $BM\perp SA$ do $SA\perp (ABC)$.
			\item Suy ra $BM\perp (SAC)$.
			\item Vậy hình chiếu vuông góc của điểm $B$ trên mặt phẳng $(SAC)$ là trung điểm $M$ của cạnh $AC$.
		\end{itemize}	
	}
\end{ex}

\begin{ex}%[DCHT Toán 11 - KNTT -Nguyễn TRần Vũ] %[1K7BN-2]
	Cho hình chóp $S.ABCD$, có đáy $ABCD$ là hình thang vuông tại $A$ và $D$; Có $SA\perp (ABCD)$ và $SA=AB=BC=\dfrac{1}{2}AD$. Hình chiếu vuông góc của $SD$ trên mặt phẳng $(SAC)$ là đường
	\choice
	{$SA$}
	{$AC$}
	{\True $SC$}
	{$AD$}
	\loigiai{
		\immini
		{Gọi $E$ là trung điểm của $AD$. Khi đó $CE=AB=\dfrac{1}{2}\cdot AD$.\\
			Suy ra tam giác $ACD$ vuông tại $C$ hay $CD\perp AC$.\\
			Mặt khác $CD\perp SA$ do $SA\perp (ABCD)$. Suy ra $CD\perp (SAC)$.\\
			Vậy hình chiếu vuông góc của $SD$ trên mặt phẳng $(SAC)$ là đường $SC$.
		}{
			\begin{tikzpicture}[scale=0.7,font=\footnotesize,line join=round,line cap=round,>=stealth] 
				\path 
				(0,0) coordinate (A) 
				(-1,-1.5) coordinate (B) 
				(1,-1.5) coordinate (C)
				(4,0) coordinate (D) 
				(A)--+(0,3) coordinate (S)
				($(A)!0.5!(D)$) coordinate (E);
				\draw[dashed] (S)--(A)--(B) (E)--(C)--(A)--(D); 
				\draw (S)--(B)--(C)--(S)--(D)--(C); 
				\foreach \p/\g in {S/135,A/180,B/-135,C/-45,D/0,E/45} 
				\fill[black](\p) circle (1pt) ($(\p)+(\g:3mm)$) node{$\p$}; 
				\path[dashed] 
				pic[draw,angle radius=3mm]{right angle=B--A--D}
				pic[draw,angle radius=3mm]{right angle=A--B--C};
		\end{tikzpicture}}
	}
\end{ex}

\begin{ex}%[DCHT Toán 11 - KNTT -Nguyễn TRần Vũ] %[1K7BN-2]
	Cho hình chóp $S.ABC$ có đáy $ABC$ là tam giác đều và $SA$ vuông góc với đáy. Gọi $G$ là trọng tâm của tam giác $ABC$. Khẳng định nào sau đây \textbf{sai}?
	\choice
	{$SA\perp AC$}
	{\True $SG\perp AC$}
	{$SA\perp AB$}
	{$SG\perp BC$}
	\loigiai{
		\begin{center}
			\begin{tikzpicture}[scale=1,font=\footnotesize,line join=round,line cap=round,>=stealth] 
				\path 
				(0,0) coordinate (A) 
				(1.5,-1.5) coordinate (B) 
				(4,0) coordinate (C) 
				(0,3) coordinate (S)
				($(B)!0.5!(C)$) coordinate (M)
				($(A)!2/3!(M)$) coordinate (G); 
				\draw (S)--(A)--(B)--(C)--(S)--(B); 
				\draw[dashed] (A)--(C) (A)--(G)--(S); 
				\foreach \p/\g in {S/135,A/-135,B/-135,C/-45,G/-90} 
				\fill[black](\p) circle (1pt) ($(\p)+(\g:3mm)$) node{$\p$}; 
				\path 
				(A)--(B) node[pos=0.5,rotate=-30,scale=0.7]{$/$}
				(B)--(C) node[pos=0.7,rotate=30,scale=0.7]{$/$}
				(A)--(C) node[pos=0.5,rotate=30,scale=0.7]{$/$};
			\end{tikzpicture}
		\end{center}
		\begin{itemize}
			\item Do $SA\perp (ABC)$ nên $SA\perp AB$ và $SA\perp AC$. 
			\item Vì $G$ là trọng tâm của tam giác $ABC$ đều nên $BC\perp AG$.
			\item Mặt khác $AG$ là hình chiếu của của $SG$ trên mặt phẳng $(ABC)$ nên suy ra $SG\perp BC$ (theo định lí ba đường cuông góc). 
		\end{itemize}	
	}
\end{ex}

\begin{ex}%[DCHT Toán 11 - KNTT -Nguyễn TRần Vũ] %[1K7KN-2]
	Cho hình chóp $S.ABCD$ có đáy $ABCD$ là hình chữ nhật và $SA$ vuông góc với đáy. Khẳng định nào sau đây \textbf{sai}?
	\choice
	{$SA\perp AC$}
	{$BC\perp SB$}
	{$CD\perp SD$}
	{\True $CD\perp SC$}
	\loigiai{
		\begin{center}
			\begin{tikzpicture}[scale=1,font=\footnotesize,line join=round,line cap=round,>=stealth] 
				\path 
				(0,0) coordinate (A) 
				(-2,-2) coordinate (B) 
				($(A)+(4,0)$) coordinate (D) 
				($(B)+(4,0)$) coordinate (C) 
				($(A)+(0,3)$) coordinate (S); 
				\draw[dashed] (S)--(A)--(B) (C)--(A)--(D); 
				\draw (S)--(B)--(C)--(D)--(S)--(C); 
				\foreach \p/\g in {S/135,A/135,B/-135,C/-45,D/45} 
				\fill[black](\p) circle (1pt) ($(\p)+(\g:3mm)$) node{$\p$}; 
				\path[dashed] pic[draw,angle radius=3mm]{right angle=A--B--C}; 
			\end{tikzpicture}
			
		\end{center}
		Do $SA\perp (ABCD)$ nên $AC$ là hình chiếu của $SC$ trên mặt phẳng $(ABCD)$.\\
		Giả sử $CD\perp SC$, theo định lí ba đường vuông góc suy ra $CD\perp AC$ (vô lý).\\
		Vậy khẳng định $CD\perp SC$ là khẳng định sai.	
	}
\end{ex}

\begin{ex}%[DCHT Toán 11 - KNTT -Nguyễn TRần Vũ] %[1K7KN-2]
	Cho hình chóp $S.ABCD$ có đáy $ABCD$ là hình vuông cạnh $a$. Mặt bên $SAB$ là tam giác đều có đường cao $SH$ vuông góc với $(ABCD)$. Gọi $K$ là hình chiếu vuông góc của điểm $H$ trên mặt phẳng $(SAD)$. Khi đó độ dài đoạn thẳng $HK$ theo $a$ bằng
	\choice
	{\True $\dfrac{a\sqrt{3}}{4}$}
	{$\dfrac{a}{2}$}
	{$\dfrac{a\sqrt{3}}{2}$}
	{$\dfrac{a\sqrt{2}}{3}$}
	\loigiai{
		\begin{center}
			\begin{tikzpicture}[scale=1,font=\footnotesize,line join=round,line cap=round,>=stealth] 
				\path 
				(0,0) coordinate (A) 
				(-1.5,-2) coordinate (B) 
				($(A)+(4,0)$) coordinate (D) 
				($(B)+(4,0)$) coordinate (C)
				($(A)!0.5!(B)$) coordinate (H)
				($(H)+(0,4)$) coordinate (S)
				($(S)!0.5!(A)$) coordinate (M)
				($(A)!0.5!(M)$) coordinate (K); 
				\draw[dashed] (S)--(A)--(B)--(M) (A)--(D) (S)--(H)--(K); 
				\draw (S)--(B)--(C)--(D)--(S)--(C); 
				\foreach \p/\g in {S/135,A/-45,B/-135,C/-45,D/45,H/-45,M/30,K/30} 
				\fill[black](\p) circle (1pt) ($(\p)+(\g:3mm)$) node{$\p$}; 
				\path 
				(A)--(B) node[pos=0.7,right]{$a$} 
				(B)--(C) node[pos=0.5,below]{$a$}
				(S)--(B) node[pos=0.5,left]{$a$}
				(S)--(A) node[pos=0.5,left]{$a$}; 
				\path[dashed] 
				pic[draw,angle radius=3mm]{right angle=A--B--C}
				pic[draw,angle radius=3mm]{right angle=S--H--B}
				pic[draw,angle radius=3mm]{right angle=B--M--A}; 
			\end{tikzpicture}
			
		\end{center}
		Gọi $M$ là trung điểm của $SA$ và $K$ là trung điểm của $AM$.\\
		Khi đó $HK\parallel BM$, mà $BM\perp SA$ (do tam giác $SAB$ đều). Suy ra $HK\perp SA$. $\qquad (1)$\\
		Ta có $\heva{&AD\perp AB\\&AD\perp SH}\Rightarrow AD\perp (SAB)\Rightarrow AD\perp HK$. $\qquad (2)$\\
		Từ $(1)$ và $(2)$ suy ra $HK\perp (SAD)$. Do đó $K$ là hình chiếu của điểm $H$ trên mặt phẳng $(SAD)$.\\
		Vậy $HK=\dfrac{1}{2}\cdot BM=\dfrac{1}{2}\cdot \dfrac{a\sqrt{3}}{2}=\dfrac{a\sqrt{3}}{4}$.
	}
\end{ex}

\begin{ex}%[DCHT Toán 11 - KNTT -Nguyễn TRần Vũ] %[1K7TN-2]
	Cho tứ diện đều $OABC$ có $OA$, $OB$, $OC$ đôi một vuông góc với nhau. Gọi $H$ là hình chiếu vuông góc của điểm $O$ trên mặt phẳng $(ABC)$. Khi đó $H$ là
	\choice
	{trung điểm của cạnh $AB$}
	{trung điểm của $BC$}
	{trọng tâm của tam giác $ABC$}
	{\True trực tâm của tam giác $ABC$}
	\loigiai{
		\begin{center}
			\begin{tikzpicture}[scale=1,font=\footnotesize,line join=round,line cap=round,>=stealth] 
				\path 
				(0,0) coordinate (A) 
				(1.5,-2) coordinate (B) 
				(6,0) coordinate (C)
				(B)--(C) coordinate[pos=0.4] (M) 
				($(A)!0.7!(M)$) coordinate (H)
				(H)--+(0,4) coordinate (O); 
				\draw[dashed] (A)--(C)--(H)--(O) (B)--(H)--(A); 
				\draw (O)--(A)--(B)--(C)--(O)--(B); 
				\foreach \p/\g in {O/90,A/135,B/-90,C/45,H/-90} 
				\fill[black](\p) circle (1pt) ($(\p)+(\g:3mm)$) node{$\p$}; 
				\path pic[draw,angle radius=3mm]{right angle=O--H--M}; 
			\end{tikzpicture}
		\end{center}
		Do $H$ là hình chiếu vuông góc của điểm $O$ trên mặt phẳng $(ABC)$ nên $OH\perp (ABC)$. Suy ra $OH\perp AB$ và $OH\perp BC\qquad (1)$.\\
		Theo đề bài ta có
		\begin{itemize}
			\item $\heva{&OC\perp OA\\&OC\perp OB}\Rightarrow OC\perp (OAB)\Rightarrow OC\perp AB\qquad (2)$.
			\item $\heva{&OA\perp OB\\&OA\perp OC}\Rightarrow OA\perp (OBC)\Rightarrow OA\perp BC\qquad (3)$.
		\end{itemize}	
		Từ $(1)$ và $(2)$ suy ra $AB\perp (OHC)\Rightarrow AB\perp HC.\qquad (*)$\\
		Từ $(1)$ và $(3)$ suy ra $BC\perp (OHA)\Rightarrow BC\perp HA.\qquad (**)$\\
		Từ $(*)$ và $(**)$ suy ra $H$ là trực tâm của tam giác $ABC$.
	}
\end{ex}

\Closesolutionfile{ans}
\begin{indapan}{10}
	{ans/ans-1K7-24-Dang3}
\end{indapan}
