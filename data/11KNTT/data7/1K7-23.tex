\setcounter{dang}{0}
\section{Đường thẳng vuông góc với mặt phẳng}
\subsection{Tóm tắt lý thuyết}
\begin{tomtat}
	\subsubsection{Định nghĩa}
	\begin{dn}
		\immini{Đường thẳng $d$ được gọi là vuông góc với mặt phẳng $(\alpha)$ nếu $d$ vuông góc với mọi đường thẳng $a$ nằm trong mặt phẳng $(\alpha)$. 	Khi đó ta còn nói $(\alpha)$ vuông góc $d$ và kí hiệu $d \perp (\alpha)$ hoặc $(\alpha) \perp d$. 
		}
		{\begin{tikzpicture}[scale=1.2,font=\footnotesize, line join=round, line cap=round,>=stealth]
				\coordinate (A) at (0,0);
				\coordinate (B) at (1,1);
				\coordinate (D) at (3,0);
				\coordinate (C) at ($(D)+(B)-(A)$);
				\coordinate (m1) at (2.5,2);
				\coordinate (m2) at ($(m1)-(0,2.2)$);
				\coordinate (m3) at (intersection of B--C and m1--m2);
				\coordinate (m4) at ($(m1)!0.6!(m2)$); 
				\coordinate (m5) at (intersection of A--D and m1--m2);
				\coordinate (e1) at ($(A)!0.8!(B)$);
				\coordinate (e2) at ($(A)!0.6!(D)$);
				\coordinate (e3) at ($(e1)!0.25!(e2)$);
				\draw (e3) node[above,rotate=-30]{$a$};	
				\draw (A)--(B)--(C)--(D)--cycle;
				\draw (D) -- (A);
				\draw (B) -- (C);
				\draw (m1) -- (m4);
				\draw (m2) -- (m5);
				\draw[dashed] (m4) -- (m5);
				\draw ($(e1)!0.15!(e2)$) -- ($(e2)!0.15!(e1)$);
				\draw pic[draw,angle radius=0.65cm,"$\alpha$",angle eccentricity=0.5] {angle = D--A--B};	
				\draw (m1) node[right]{$d$};
				\draw (e3) node[above,rotate=-30]{$a$};
			\end{tikzpicture}
		}
	\end{dn}
	\subsubsection{Điều kiện để đường thẳng vuông góc với mặt phẳng}
	\begin{dl} Nếu một đường thẳng vuông góc với hai đường thẳng cắt nhau cùng thuộc\immini{ mặt phẳng thì nó vuông góc với mặt phẳng ấy.}
		{\begin{tikzpicture}[scale=1.2,font=\footnotesize, line join=round, line cap=round,>=stealth]
				\coordinate (A) at (0,0);
				\coordinate (B) at (1,1.5);
				\coordinate (D) at (3,0);
				\coordinate (C) at ($(D)+(B)-(A)$);
				
				\draw (A)--(B)--(C)--(D)--cycle;
				\coordinate (m1) at (2.5,2);
				\coordinate (m2) at (2.5,-0.2);
				\coordinate (m3) at ($(B)!(m2)!(C)$);
				\coordinate (m4) at ($(m1)!0.5!(m2)$); 
				\coordinate (m5) at ($(D)!(m1)!(A)$);
				\draw (m1) node[left]{$d$};
				\draw (m1)--(m4);
				\draw[dashed] (m4)--(m5);
				\draw (m2)--(m5);
				
				\coordinate (m6) at ($(m1)!0.15!(m2)$); 
				\coordinate (e1) at ($(A)!0.8!(B)$);
				\coordinate (e2) at ($(A)!0.75!(D)$);
				\coordinate (e3) at ($(e1)!0.2!(e2)$);
				\coordinate (e4) at ($(e1)!0.8!(e2)$);
				
				\draw (e3) node[above, rotate=-30]{$a$};
				
				\coordinate (e5) at ($(B)!0.6!(C)$); 
				\coordinate (e6) at ($(A)!0.3!(e5)$);
				\coordinate (e7) at ($(A)!0.74!(e5)$);
				
				\draw (e6) node[above, rotate=30]{$b$};
				
				
				\coordinate (O) at (intersection of e1--e2 and A--e5);
				\draw[fill=black] (O)  node[below]{$O$};
				\draw ($(e1)!0.15!(e2)$) -- ($(e2)!0.15!(e1)$);
				\draw ($(A)!0.25!(e5)$) -- ($(e5)!0.15!(A)$);
				\draw pic[draw,angle radius=0.65cm,"$\alpha$",angle eccentricity=0.5] {angle = D--A--B};	
			\end{tikzpicture}
		}
	\end{dl}
	\begin{note}
		Tóm tắt định lí.
		$\heva{&a,b \subset (\alpha)\\&a \cap b=O\\&d \perp a, d \perp b}\Rightarrow d \perp (\alpha).$
	\end{note}
	\subsubsection{Tính chất}
	\begin{tc}
		Có duy nhất một mặt phẳng đi qua một điểm cho trước và vuông góc với một đường thẳng cho trước. 
		\begin{center}
			\begin{tikzpicture}[scale=1.2,font=\footnotesize, line join=round, line cap=round,>=stealth]
				\coordinate (A) at (0,0);
				\coordinate (B) at (1,1);
				\coordinate (D) at (3,0);
				\coordinate (C) at ($(D)+(B)-(A)$);
				\coordinate (m1) at (2.5,2);
				\coordinate (m2) at ($(m1)-(0,2.2)$);
				\coordinate (m3) at (intersection of B--C and m1--m2);
				\coordinate (m4) at ($(m1)!0.6!(m2)$); 
				\coordinate (m5) at (intersection of A--D and m1--m2);
				\coordinate (e1) at ($(A)!0.8!(B)$);
				\coordinate (e2) at ($(A)!0.6!(D)$);
				\coordinate (e3) at ($(e1)!0.25!(e2)$);
				\coordinate (O) at ($(e1)!0.5!(e2)$);
				\fill (O) circle (1pt) node[below] {$O$};
				%\draw (e3) node[above,rotate=-30]{$a$};	
				\draw (A)--(B)--(C)--(D)--cycle;
				\draw (D) -- (A);
				\draw (B) -- (C);
				\draw (m1) -- (m4);
				\draw (m2) -- (m5);
				\draw[dashed] (m4) -- (m5);
				\draw pic[draw,angle radius=0.65cm,"$\alpha$",angle eccentricity=0.5] {angle = D--A--B};	
				\draw (m1) node[right]{$d$};
			\end{tikzpicture}			
			\qquad\qquad\qquad
			\begin{tikzpicture}[font=\footnotesize, line join=round, line cap=round,>=stealth,xscale=0.7,yscale=0.7]
				\draw[fill=black] (0,0) circle (1pt) coordinate (A)  node[below]{$A$};
				\draw[fill=black] (3,0) circle (1pt) coordinate (I) node[below]{$I$};
				\draw[fill=black] (4.5,3) circle (1pt) coordinate (M) node[above]{$M$};
				\draw[fill=black] (1.75,-2) circle (1pt) coordinate (X) ;
				\draw[fill=black] (1.75,3.75) circle (1pt) coordinate (Y);
				\draw[fill=black] (5.5,5) circle (1pt) coordinate (Z) ;
				\coordinate (T) at ($(X)+(Z)-(Y)$) ;
				\coordinate (B) at ($(A)!2!(I)$) ;
				\coordinate (O) at (intersection of Z--T and A--B);
				\coordinate (E) at (intersection of M--A and X--Y);
				\coordinate (F) at (intersection of M--B and Z--T);
				\coordinate (G) at (intersection of X--Y and A--B);
				\fill (B) circle (1pt) node[below] {$B$};
				\draw (X)--(Y)--(Z) (T)--(O)(T)--(X)(M)--(B)(A)--(E)(B)--(I)(Z)--(F)(A)--(G)(M)--(I);
				\draw[dashed] (O)--(F)(I)--(G)(M)--(E);
				\path pic[draw,angle radius=5]{right angle=M--I--B};
			\end{tikzpicture}
		\end{center}
	\end{tc}
	\begin{note}
		Mặt phẳng trung trực của đoạn thẳng $AB$ là mặt phẳng đi qua trung điểm $I$ của đoạn thẳng $AB$ và vuông góc với đường thẳng $AB$.
	\end{note}
	
	\begin{tc}\immini
		{Có duy nhất một đường thẳng đi qua một điểm cho trước và vuông góc với một mặt phẳng cho trước. }
		{	\begin{tikzpicture}[scale=1.2,font=\footnotesize, line join=round, line cap=round,>=stealth]
				\coordinate (A) at (0,0);
				\coordinate (B) at (1,1);
				\coordinate (D) at (3,0);
				\coordinate (C) at ($(D)+(B)-(A)$);
				\coordinate (m1) at (2.5,2);
				\draw[fill=black] (2.5,1.5) circle (1pt) coordinate (O) node[above left]{$O$};
				\coordinate (m2) at ($(m1)-(0,2.2)$);
				\coordinate (m3) at (intersection of B--C and m1--m2);
				\coordinate (m4) at ($(m1)!0.6!(m2)$); 
				\coordinate (m6) at ($(m4)+(1,0)$);
				\coordinate (m7) at ($(m4)+(0.15,0)$);
				\coordinate (m5) at (intersection of A--D and m1--m2);
				\coordinate (e1) at ($(A)!0.8!(B)$);
				\coordinate (e2) at ($(A)!0.6!(D)$);
				\coordinate (e3) at ($(e1)!0.25!(e2)$);
				\draw (A)--(B)--(C)--(D)--cycle;
				\draw (D) -- (A);
				\draw (B) -- (C);
				\draw (m1) -- (m4);
				\draw (m2) -- (m5)(m4)--(m7);
				\draw[dashed] (m4) -- (m5);
				\draw pic[draw,angle radius=0.65cm,"$\alpha$",angle eccentricity=0.5] {angle = D--A--B};	
				\path pic[draw,angle radius=5]{right angle=O--m4--m6};
				\path pic[draw,angle radius=5]{right angle=m6--m4--O};
		\end{tikzpicture} }
	\end{tc}
	\subsubsection{Liên hệ giữa quan hệ song song và quan hệ vuông góc của đường thẳng và mặt phẳng}
	\begin{tc}\immini{\begin{enumerate}
				\item Cho hai đường thẳng song song. Mặt phẳng nào vuông góc với đường thẳng này thì cũng vuông góc với đường thẳng kia.
				\item Hai đường thẳng phân biệt cùng vuông góc với một mặt phẳng thì song song với nhau.
		\end{enumerate}}
		{	\begin{tikzpicture}[scale=1.2,font=\footnotesize, line join=round, line cap=round,>=stealth]
				\coordinate (A) at (0,0);
				\coordinate (B) at (1,1);
				\coordinate (D) at (3,0);
				\coordinate (C) at ($(D)+(B)-(A)$);
				\coordinate (m1) at (2.5,2);
				\coordinate (m2) at ($(m1)-(0,2.2)$);
				\coordinate (m3) at (intersection of B--C and m1--m2);
				\coordinate (m4) at ($(m1)!0.6!(m2)$); 
				\coordinate (m5) at (intersection of A--D and m1--m2);
				\coordinate (e1) at ($(A)!0.8!(B)$);
				\coordinate (e2) at ($(A)!0.6!(D)$);
				\coordinate (e3) at ($(e1)!0.25!(e2)$);
				\coordinate (O) at ($(e1)!0.5!(e2)$);
				\coordinate (m1') at ($(m1)-(1,0)$);
				\coordinate (m2') at ($(m2)-(1,0)$);
				\coordinate (m4') at ($(m4)-(1,0)$);
				\coordinate (m5') at ($(m5)-(1,0)$);
				\draw (A)--(B)--(C)--(D)--cycle;
				\draw (D) -- (A);
				\draw (B) -- (C);
				\draw (m1) -- (m4);
				\draw (m2) -- (m5);
				\draw[dashed] (m4) -- (m5);
				\draw (m1') -- (m4');
				\draw (m2') -- (m5');
				\draw[dashed] (m4') -- (m5');
				\draw pic[draw,angle radius=0.65cm,"$\alpha$",angle eccentricity=0.5] {angle = D--A--B};	
				\draw (m1) node[right]{$b$};
				\draw (m1') node[left]{$a$};
		\end{tikzpicture}}
	\end{tc}
	\begin{note}
		\begin{listEX}[2]
			\item $\heva{&a \parallel b\\&(\alpha) \perp a} \Rightarrow (\alpha) \perp b.$
			\item $\heva{&a \parallel b\\&(\alpha) \perp a} \Rightarrow (\alpha) \perp b.$
		\end{listEX}
	\end{note}
	
	\begin{tc}\immini
		{\begin{enumerate}
				\item Cho hai mặt phẳng song song. Đường thẳng nào vuông góc với mặt phẳng này thì cũng vuông góc với mặt phẳng kia.
				\item Hai mặt phẳng phân biệt cùng vuông góc với một đường thẳng thì song song với nhau.
		\end{enumerate}}
		{	\begin{tikzpicture}[scale=1.2,font=\footnotesize, line join=round, line cap=round,>=stealth]
				\coordinate (A) at (0,0);
				\coordinate (B) at (1,1);
				\coordinate (D) at (3,0);
				\coordinate (C) at ($(D)+(B)-(A)$);
				\coordinate (A') at ($(A)-(0,1.5)$);
				\coordinate (B') at ($(B)-(0,1.5)$);
				\coordinate (D') at ($(D)-(0,1.5)$);
				\coordinate (C') at ($(C)-(0,1.5)$);
				\coordinate (m1) at (2.5,2);
				\coordinate (m2) at ($(m1)-(0,2.2)$);
				\coordinate (m3) at (intersection of B--C and m1--m2);
				\coordinate (m4) at ($(m1)!0.6!(m2)$); 
				\coordinate (m5) at (intersection of A--D and m1--m2);
				\coordinate (e1) at ($(A)!0.8!(B)$);
				\coordinate (e2) at ($(A)!0.6!(D)$);
				\coordinate (e3) at ($(e1)!0.25!(e2)$);
				\coordinate (O) at ($(e1)!0.5!(e2)$);
				\coordinate (m1') at ($(m1)-(0,1.5)$);
				\coordinate (m2') at ($(m2)-(0,1.5)$);
				\coordinate (m4') at ($(m4)-(0,1.5)$);
				\coordinate (m5') at ($(m5)-(0,1.5)$);
				\draw (A)--(B)--(C)--(D)--cycle;
				\draw (A')--(B')--(C')--(D')--cycle;
				\draw (D) -- (A);
				\draw (B) -- (C);
				\draw (m1) -- (m4);
				\draw (m2) -- (m5);
				\draw[dashed] (m4) -- (m5);
				\draw (m5) -- (m4');
				\draw (m2') -- (m5');
				\draw[dashed] (m4') -- (m5');
				\draw pic[draw,angle radius=0.65cm,"$\alpha$",angle eccentricity=0.5] {angle = D--A--B};	
				\draw pic[draw,angle radius=0.65cm,"$\beta$",angle eccentricity=0.5] {angle = D'--A'--B'};	
				\draw (m1) node[left]{$a$};
			\end{tikzpicture}
		}
	\end{tc}
	\begin{note}
		\begin{listEX}[2]
			\item $\heva{&(\alpha) \parallel (\beta)\\&a \perp (\alpha)} \Rightarrow a \perp (\beta).$
			\item $\heva{& (\alpha) \perp a, (\beta) \perp a\\&(\alpha) \not \equiv (\beta)} \Rightarrow (\alpha) \parallel (\beta).$
		\end{listEX}
	\end{note}
	\begin{tc} \immini
		{\begin{enumerate}
				\item  Cho đường thẳng $a$ và mặt phẳng $(\alpha)$ song song với nhau. Đường thẳng nào vuông góc với mặt phẳng $(\alpha)$ thì cũng vuông góc với $a$.
				\item Nếu một đường thẳng và một mặt phẳng  (không chứa đường thẳng đó) cùng vuông góc với một đường thẳng khác thì chúng song song với nhau.
		\end{enumerate}}
		{	\begin{tikzpicture}[scale=1.2,font=\footnotesize, line join=round, line cap=round,>=stealth]
				\coordinate (A) at (0,0);
				\coordinate (B) at (1,1);
				\coordinate (D) at (3,0);
				\coordinate (C) at ($(D)+(B)-(A)$);
				\coordinate (m1) at (2,2);
				\coordinate (m1') at (2.3,1.8);
				\coordinate (m2') at (4.3,1.8);
				\coordinate (m3') at ($(m1')!0.5!(m2')$);
				\coordinate (m2) at ($(m1)-(0,2.2)$);
				\coordinate (m3) at (intersection of B--C and m1--m2);
				\coordinate (m4) at ($(m1)!0.6!(m2)$); 
				\coordinate (m5) at (intersection of A--D and m1--m2);
				\coordinate (e1) at ($(A)!0.8!(B)$);
				\coordinate (e2) at ($(A)!0.6!(D)$);
				\coordinate (e3) at ($(e1)!0.25!(e2)$);
				\coordinate (O) at ($(e1)!0.5!(e2)$);
				\draw (A)--(B)--(C)--(D)--cycle;
				\draw (D) -- (A);
				\draw (B) -- (C);
				\draw (m1) -- (m4)(m1')--(m2');
				\draw (m2) -- (m5);
				\draw[dashed] (m4) -- (m5);
				\draw pic[draw,angle radius=0.65cm,"$\alpha$",angle eccentricity=0.5] {angle = D--A--B};	
				\draw (m1) node[left]{$b$};
				\draw (m3') node[above]{$a$};
		\end{tikzpicture}}
	\end{tc}
	\begin{note}
		\begin{listEX}[2]
			\item $\heva{&a \parallel (\alpha)\\&b \perp (\alpha)} \Rightarrow b \perp a.$
			\item $\heva{&a \not \subset (\alpha)\\&a \perp b\\&(\alpha) \perp b} \Rightarrow a \parallel (\alpha).$
		\end{listEX}	
	\end{note}
\end{tomtat}
\subsection{Các dạng toán thường gặp}
\begin{dang}{Câu hỏi lí thuyết}
\end{dang}
\Opensolutionfile{ans}[ans/ans-1K7-2-1]
\begin{ex}%[1K7YM-1]
	Đường thẳng $\Delta$ vuông góc với mặt phẳng $(\alpha)$ khi và chỉ khi
	\choice
	{đường thẳng $\Delta$ vuông góc với 1 đường thẳng nằm trên mặt phẳng $(\alpha)$}
	{\True đường thẳng $\Delta$ vuông góc với 2 đường thẳng cắt nhau nằm trên mặt phẳng $(\alpha)$}
	{đường thẳng $\Delta$ vuông góc với 2 đường thẳng phân biệt nằm trên mặt phẳng $(\alpha)$}
	{đường thẳng $\Delta$ vuông góc với 2 đường thẳng song song nằm trên mặt phẳng $(\alpha)$}
	\loigiai{
		Theo định lý đường thẳng vuông góc với mặt phẳng.}
\end{ex}
\begin{ex}%[1K7YM-1]
	Trong không gian, mệnh đề nào sau đây đúng
	\choice
	{Hai đường thẳng cùng vuông góc với một đường thẳng thì song song với nhau}
	{Hai đường thẳng phân biệt cùng vuông góc với một đường thẳng thì vuông góc với nhau}
	{\True Hai đường thẳng phân biệt cùng vuông góc với một mặt phẳng thì song song với nhau}
	{Hai đường thẳng phân biệt cùng vuông góc với một đường thẳng thì song song với nhau}
	\loigiai{
		Hai đường thẳng phân biệt cùng vuông góc với một mặt phẳng thì song song với nhau.}
\end{ex}
\begin{ex}%[1K7YM-1]
	Trong không gian cho các đường thẳng $ a $, $ b $, $ c $ và mặt phẳng $(P)$. Mệnh đề nào sau đây \textbf{sai} ?
	\choice
	{\True Nếu $ a \perp b $ và $ b \perp c $ thì $ a\parallel c $}
	{Nếu $ a \perp b $, $ c \perp b $ và $ a $ cắt $ c $ thì $ b $ vuông góc với mặt phẳng chứa $ a $ và $ c $}
	{Nếu $ a \parallel b $ và $ b \perp c $ thì $ c \perp a $}
	{Nếu $ a \perp (P) $ và $ b \parallel (P)$ thì $ a \perp b $}
	\loigiai{
		Nếu $ a \perp b $ và $ b \perp c $ thì còn trường hợp $ a $ chéo $ c $.
	}
\end{ex}
\begin{ex}%[1K7YM-1]
	Trong không gian cho đường thẳng $ \Delta $ không nằm trong mặt phẳng $ (P) $. Đường thẳng $ \Delta $ vuông góc với mặt phẳng $ (P) $ nếu:
	\choice
	{$ \Delta $ vuông góc với đường thẳng $ a $ mà $ a \parallel (P) $}
	{$ \Delta $ vuông góc với mặt phẳng $ (Q) $ mà $ (Q) \perp (P) $}
	{\True$ \Delta $ vuông góc với mọi đường thẳng nằm trong mặt phẳng $ (P) $}
	{$ \Delta $ vuông góc với hai đường thẳng phân biệt nằm trong mặt phẳng $ (P) $}
	\loigiai{
		Theo định nghĩa, một đường thẳng gọi là vuông góc với một mặt phẳng nếu đường thẳng vuông góc với mọi đường thẳng nằm trong mặt phẳng đó.
	}
\end{ex}
\begin{ex}%[1K7YM-1]
	Trong không gian cho đường thẳng $\Delta$ và điểm $A$. Có bao nhiêu mặt phẳng đi qua $A$ và vuông góc với đường thẳng $\Delta$ đã cho?
	\choice
	{$2$}
	{Vô số}
	{Không có}
	{\True $1$}
	\loigiai
	{
		Trong không gian có một và chỉ một mặt phẳng đi qua $A$ và vuông góc với đường thẳng $\Delta$ đã cho.
	}
\end{ex}
\begin{ex}%[1K7YM-1]
	Qua điểm $O$ có bao nhiêu mặt phẳng vuông góc với đường thẳng cho trước?
	\choice
	{\True $1$}
	{$0$}
	{Vô số}
	{$2$}
	\loigiai{
		Qua điểm $O$ có duy nhất một mặt phẳng vuông góc với đường thẳng cho trước.
	}
\end{ex}
\begin{ex}%[1K7YM-1]
	Cho $\left( P \right)$ là mặt phẳng trung trực của đoạn thẳng $AB$. Gọi $I$ là trung điểm của $AB$. Khi đó
	\choice
	{$AB \subset \left( P \right)$}
	{\True $
		\begin{cases} I \in \left( P \right) \\ AB \perp \left( P \right) \end{cases}$}
	{$
		\begin{cases} I \in \left( P \right) \\ AB \parallel \left( P \right) \end{cases}$}
	{$AB\parallel \left( P \right)$}
	\loigiai{
		Mặt phẳng trung trực của đoạn thẳng là mặt phẳng đi qua trung điểm và vuông góc với đoạn thẳng đó.
	}
\end{ex}
\begin{ex}%[1K7YM-1]
	Trong không gian, mệnh đề nào sau đây đúng?
	\choice
	{\True Nếu $a\parallel (P)$ và $b\perp (P)$ thì $b\perp a$}
	{Nếu $a\parallel (P)$ và $b\parallel a$ thì $b\parallel (P)$}
	{Một đường thẳng vuông góc với hai đường thẳng phân biệt trong mặt phẳng $(P)$ thì nó vuông góc với mặt phẳng $(P)$}
	{Nếu $a\parallel (P)$ và $b\perp a$ thì $b\perp (P)$}
	\loigiai{
		Theo tính chất liên hệ giữa quan hệ song song và quan hệ vuông góc.}
	%<MyLT>
\end{ex}
\begin{ex}%[1K7YM-1]
	Trong không gian tập hợp các điểm $M$ cách đều hai điểm cố định $A$ và $B$ (phân biệt) là
	\choice{\True Mặt phẳng trung trực của đoạn thẳng $AB$}
	{Đường trung trực của đoạn thẳng $AB$}
	{Mặt phẳng vuông góc với $AB$ tại $A$}
	{Đường thẳng qua $A$ và vuông góc với $AB$}
	\loigiai{
		Tập hợp các điểm $M$ cách đều hai điểm cố định $A$ và $B$ là mặt phẳng trung trực của đoạn thẳng $AB$
	}
\end{ex}
\begin{ex}%[1K7BM-1]
	Chọn mệnh đề \textbf{sai} trong các mệnh đề dưới đây?
	\choice
	{Cho hai đường thẳng $a$, $ b$ song song với nhau. Nếu có một đường thẳng $d$ vuông góc với $a$ thì $d$ vuông góc với $b$}
	{Cho hai đường thẳng $a$, $b$ song song với nhau. Nếu có một mặt phẳng $(P)$ vuông góc với $a$ thì $(P)$ vuông góc với $b$}
	{\True Cho hai đường thẳng $a$, $b$ song song với nhau. Tồn tại một mặt phẳng chứa đường thẳng này và vuông góc với đường thẳng kia}
	{Nếu đường thẳng $d$ vuông góc với $(P)$ thì $d$ vuông góc với mọi đường thẳng nằm trong $(P)$}
	\loigiai{
		Giả sử có một mặt phẳng chứa đường thẳng $a$ và vuông góc với đường thẳng $b$. Khi đó, $b \perp a$ (vô lí vì hai đường thẳng này cho trước có thể không vuông góc nhau).}
\end{ex}

\begin{ex}%[1K7BM-1]
	Cho hai đường thẳng phân biệt $a$, $b$ và mặt phẳng $(\alpha)$. Mệnh đề nào dưới đây là đúng? 
	\choice
	{Nếu $a\parallel (\alpha)$ và $b\parallel (\alpha)$ thì $b\parallel a$ }
	{  Nếu $a\perp (\alpha)$ và $b\perp a$ thì $b\parallel (\alpha)$}
	{\True Nếu $a\parallel (\alpha)$ và $b\perp (\alpha)$ thì $a\perp b$.}
	{Nếu $a\parallel (\alpha)$ và $b\perp a$ thì $b\perp (\alpha)$ }
	\loigiai{
		Khẳng định \lq\lq Nếu $a\parallel (\alpha)$ và $b\parallel (\alpha)$ thì $b\parallel a$ \rq\rq \,là một khẳng định sai, vì ta có thể lấy $a$ và $b$ cùng nằm trên mặt phẳng $(\beta)$ nhưng chúng cắt nhau.\\
		Khẳng định \lq\lq Nếu $a\perp (\alpha)$ và $b\perp a$ thì $b\parallel (\alpha)$\rq\rq \,là khẳng định sai, vì ta có thể lấy $b$ nằm trên $(\alpha)$.\\
		Khẳng định \lq\lq Nếu $a\parallel (\alpha)$ và $b\perp a$ thì $b\perp (\alpha)$\rq\rq\,là một khẳng định sai, vì ta ta có thể lấy $a$ và $b$ cùng nằm trong mặt phẳng $(\beta)$ song song với $(\alpha)$ và $b\perp a$.
	}
\end{ex}

\begin{ex}%[1K7BM-1]
	Cho hai đường thẳng $a$, $b$ và mặt phẳng $(P)$. Mệnh đề nào dưới đây đúng?
	\choice
	{\True Nếu $a\parallel (P)$ và $b\perp (P)$ thì $a\perp b$}
	{Nếu $a\perp (P)$ và $b\perp a$ thì $b\parallel (P)$}
	{Nếu $a\parallel (P)$ và $b\perp a$ thì $b\perp (P)$}
	{Nếu $a\parallel (P)$ và $b\perp a$ thì $b\parallel (P)$}
	\loigiai{
		Ta có
		\begin{itemize}
			\item Mệnh đề ``Nếu $a\perp (P)$ và $b\perp a$ thì $b\parallel (P)$'' sai vì $b$ có thể chứa trong $(P)$.
			\item Mệnh đề ``Nếu $a\parallel (P)$ và $b\perp a$ thì $b\perp (P)$'' sai vì $b$ có thể song song với $(P)$.
			\item Mệnh đề ``Nếu $a\parallel (P)$ và $b\perp a$ thì $b\parallel (P)$'' sai vì $b$ có thể vuông góc với $(P)$.
		\end{itemize}
	}
\end{ex}

\begin{ex}%[1K7KM-1]
	Trong các mệnh đề sau, mệnh đề nào \textbf{sai?}
	\choice
	{Nếu $a\parallel b$ và $a\perp(\alpha)$ thì $b\perp(\alpha)$}
	{Nếu $(\alpha)\parallel (\beta)$ và $a\perp(\alpha)$ thì $a\perp (\beta)$}
	{Nếu $a$ và $b$ là hai đường thẳng phân biệt và $a\perp(\alpha)$, $b\perp(\alpha)$ thì $a\parallel b$}
	{\True Nếu $a\parallel (\alpha)$ và $b\perp a$ thì $b\perp(\alpha)$}
	\loigiai
	{
		\immini
		{Xét hình lập phương $ABCD.A'B'C'D'$.\\
			Ta có $A'B'\parallel (ABCD)$ và $A'D'\perp A'B'$ nhưng $A'D'\parallel(ABCD)$.\\
			Vậy khẳng định \textbf{sai} là: ``Nếu $a\parallel (\alpha)$ và $b\perp a$ thì $b\perp(\alpha)$''.
		}
		{
			\begin{tikzpicture}[line cap=round,line join=round,font=\footnotesize,>=stealth,scale=0.8]
				\fill (0,0) coordinate [label=left:$A$] (A) circle(1pt) 
				(3,0) coordinate [label=below right:$B$] (B) circle(1pt)
				(50:2) coordinate [label=left:$D$] (D) circle(1pt)
				(90:3) coordinate [label=left:$A'$] (A') circle(1pt)
				($(B)+(D)$) coordinate [label=right:$C$] (C) circle(1pt)
				($(A')+(B)$) coordinate [label=right:$B'$] (B') circle(1pt)
				($(A')+(C)$) coordinate [label=right:$C'$] (C') circle(1pt)
				($(A')+(D)$) coordinate [label=above:$D'$] (D') circle(1pt);
				\draw (C')--(B')--(A')--(A)--(B)--(C)--(C')--(D')--(A') (B')--(B);
				\draw [dashed] (A)--(D)--(D') (D)--(C);
		\end{tikzpicture}}
	}
\end{ex}

\Closesolutionfile{ans}
% \begin{indapan}{10}
% 	{ans/ans-1K7-2-1}
% \end{indapan}
\begin{dang}{Đường thẳng vuông góc với mặt phẳng}
	Để chứng minh đường thẳng $d$ vuông góc với mặt phẳng $(\alpha)$, ta cần chứng minh đường thẳng $d$ vuông góc với hai đường thẳng cắt nhau nằm trong mặt phẳng $(\alpha)$.
	\begin{center}
		\fbox{$d\perp (\alpha) \Leftrightarrow \heva{&d\perp a \\ &d\perp b \\ &a, b \subset (\alpha) \\ &a \text{ cắt } b.}$}
	\end{center}
\end{dang}
\subsubsection{Ví dụ mẫu}
\begin{vd}[TH]%[1K7BM-2]
	Cho tứ diện $ABCD$ có hai mặt $ABC$ và $BCD$ là hai tam giác cân có chung đáy $BC$. Điểm $I$ là trung điểm của cạnh $BC$.
	\begin{enumerate}
		\item Chứng minh $BC \perp (ADI)$.
		\item Gọi $AH$ là đường cao trong tam giác $ADI$. Chứng minh rằng $AH \perp (BCD)$.
	\end{enumerate}
	\loigiai{
		\immini{
			\begin{enumerate}
				\item Do các tam giác $ABC$ và $BCD$ là hai tam giác cân nên tại $A$ và $D$ ta có \\ $\heva{&AI \perp BC \\ &DI \perp BC}$ (trong tam giác cân đường trung tuyến đồng thời là đường cao). \\
				Do đó $BC \perp (AID)$.
				\item Do $AH$ là đường cao trong tam giác $ADI$ nên $AH \perp DI$. \\
				Mặt khác $BC \perp (AID) \Rightarrow BC \perp AH$.
				Do đó $AH \perp (BCD)$.
			\end{enumerate}
		}
		{
			\begin{tikzpicture}[scale=1,line join=round,line cap=round,font=\footnotesize,>=stealth]
				\path (0,0) coordinate (B)++(0:4) coordinate (D)++(-150:3) coordinate (C)
				($(B)!.5!(C)$) coordinate (I) ($(D)!2/3!(I)$) coordinate (H)++(90:3) coordinate (A)
				;
				\draw (A)--(B)--(C)--(D)--cycle
				(I)--(A)--(C);
				\draw[dashed] (I)--(D)--(B) (A)--(H);
				\foreach \a/\b in {A/90,B/180,C/-90,D/0,I/200,H/-90}{
					\fill[black] (\a)circle(.7pt) ($(\a)+(\b:2mm)$)node[scale=.8]{$\a$};
				}
			\end{tikzpicture}
		}
	}
\end{vd}
\begin{vd}[VD]%[1K7KM-2]
	Cho hình chóp $S.ABCD$ có đáy là hình vuông cạnh $a$, $SA \perp (ABCD)$. Gọi $M$ và $N$ lần lượt là hình chiếu của điểm $A$ trên các đường thẳng $SB$ và $SD$.
	\begin{enumerate}
		\item Chứng minh rằng $BC \perp (SAB), CD \perp (SAD)$.
		\item Chứng minh rằng $AM \perp(SBC), AN \perp (SCD)$.
		\item Chứng minh rằng $S C \perp (AMN)$ và $MN \parallel BD$.
		\item Gọi $K$ là giao điểm của $SC$ với mặt phẳng $(AMN)$. Chứng minh rằng tứ giác $AMKN$ có hai đường chéo vuông góc.
	\end{enumerate}
	\loigiai{
		\begin{center}
			\begin{tikzpicture}[scale=1,line join=round, line cap=round, font=\footnotesize,>=stealth]
				\path (0,0) coordinate (A)++(0:3) coordinate (B)++(-140:1.5) coordinate (C)
				++(180:3) coordinate (D)
				(90:2) coordinate (S)
				($(S)!1/3!(B)$) coordinate (H)
				($(S)!1/3!(D)$) coordinate (K)
				;
				\draw (S)--(B)--(C)--(D)--cycle
				(S)--(C);
				\draw[dashed] (B)--(A)--(D)--cycle (C)--(A)--(S)
				(H)--(A)--(K)--cycle;
				\foreach \a/\b in {A/150,B/0,C/-30,D/180,S/90,H/20,K/160}{
					\fill[black] (\a)circle(.7pt)
					($(\a)+(\b:2mm)$)node[scale=.8]{$\a$};
				}
			\end{tikzpicture}
		\end{center}
		\begin{enumerate}
			\item Do $SA \perp (ABCD) \Rightarrow SA \perp BC$. \\
			Mặt khác $ABCD$ là hình vuông nên $BC \perp AB$.
			Khi đó $\heva{&BC \perp AB \\ &BC \perp SA} \Rightarrow BC \perp (SAB)$. \\
			Tương tự chứng minh trên ta có $CD \perp (SAD)$.
			\item Do $BC \perp(SAB) \Rightarrow BC \perp AM$. \\
			Mặt khác $AM \perp SB \Rightarrow AM \perp (SBC)$.
			Tương tự ta có $AN \perp (SCD)$.
			\item Do $\heva{&AM \perp(SBC) \\ &AN \perp (SCD)} \Rightarrow \heva{&AM \perp SC \\ &AN \perp SC} \Rightarrow SC \perp (AMN)$. \\
			Hai tam giác vuông $SAB$ và $SAD$ bằng nhau có các đường cao tương ứng là $AM$ và $AN$ nên $CM=DN$. Mặt khác tam giác $SBD$ cân tại đỉnh $S$ nên $MN \parallel BD$.
			\item Do $ABCD$ là hình vuông nên $AC \perp BD$, mặt khác $SA \perp BD \Rightarrow BD \perp(SAC)$. \\
			Do $MN \parallel BD \Rightarrow MN \perp (SAC) \Rightarrow MN \perp AK$.
		\end{enumerate}
	}
\end{vd}
\subsubsection{Bài tập rèn luyện}
% \centerline{\fcolorbox{red}{yellow!50}{\bf {BÀI TẬP TỰ LUẬN}}}
\begin{bt}%[1K7BM-2]
	Cho tứ diện $ABCD$ có ba cạnh $AB$, $AC$, $AD$ đôi một vuông góc.
	\begin{enumerate}
		\item Chứng minh hình chiếu vuông góc của đỉnh $A$ lên mặt phẳng $(BCD)$ trùng với trực tâm của tam giác $BCD$.
		\item Chứng minh rằng $\dfrac{1}{AH^2}=\dfrac{1}{AB^2}+\dfrac{1}{AC^2}+\dfrac{1}{AD^2}$.
		\item Chứng minh rằng tam giác $BCD$ có $3$ góc nhọn.
	\end{enumerate}
	\loigiai{
		\immini{
			\begin{enumerate}
				\item Gọi $H$ là hình chiếu vuông góc của điểm $A$ trên mặt phẳng $(BCD)$ thì $AH \perp (BCD)$. \\
				Ta có $\heva{&AD \perp AB \\ &AD \perp AC} \Rightarrow AD \perp (ABC) 
				\Rightarrow AD \perp BC$.\\
				Mặt khác $AH \perp BC \Rightarrow BC \perp (ADH) \Rightarrow BC \perp DH$. \\
				Tương tự chứng minh trên ta có $BH \perp CD$.
				Do đó $H$ là trực tâm của tam giác $BCD$.
				\item Gọi $E=DH \cap BC$, do $BC \perp (ADH) \Rightarrow BC \perp AE$. \\
				Xét $\triangle ABC$ vuông tại $A$ có đường cao $AE$ ta có: $\frac{1}{AE^2}=\frac{1}{AB^2}+\frac{1}{AC^2}$. \\
				Lại có $\frac{1}{AH^2}=\frac{1}{AD^2}+\frac{1}{AE^2}=\frac{1}{AB^2}+\frac{1}{AC^2}+\frac{1}{AD^2}(\text{đpcm})$.
				\item Đặt $AB=x; AC=y$ và $AD=z$. Ta có $\heva{&BC=\sqrt{x^2+y^2} \\ &BD=\sqrt{x^2+z^2} \\ &CD=\sqrt{y^2+z^2}.}$ \\
				Khi đó $\cos B=\frac{BC^2+BD^2-CD^2}{2\cdot BC \cdot BD}=\frac{x^2}{BC \cdot BD}>0 \Rightarrow \widehat{CBD}<90^\circ$. \\
				Tương tự chứng minh trên ta cũng có $\heva{&\widehat{BDC}<90^\circ \\ &\widehat{BCD}<90^\circ} \Rightarrow$ tam giác $BCD$ có $3$ góc nhọn.
			\end{enumerate}
		}
		{
			\begin{tikzpicture}[scale=1,line join=round, line cap=round, font=\footnotesize,>=stealth]
				\path (0,0) coordinate (A)++(0:3) coordinate (B)
				(-130:1) coordinate (C)
				(90:2) coordinate (D)
				($(B)!0.4!(C)$) coordinate (E)
				($(D)!2/3!(E)$) coordinate (H)
				;
				\draw[dashed] (D)--(A)--(B) (C)--(A)--(E)
				(A)--(H);
				\draw (D)--(B)--(C)--cycle (D)--(E);
				\foreach \a/\b in {A/150,B/0,C/180,D/30,E/-50,H/20}{
					\fill[black] (\a)circle(.7pt)
					($(\a)+(\b:2mm)$)node[scale=.8]{$\a$};
				}
			\end{tikzpicture}
		}
	}
\end{bt}
\begin{bt}%[1K7BM-2]
	Cho hình chóp $S.ABC$ có $SA \perp (ABC)$, các tam giác $ABC$ và $SBC$ là các tam giác nhọn. Gọi $H$ và $K$ lần lượt là trực tâm của các tam giác $ABC$ và $SBC$. Chứng minh rằng:
	\begin{enumEX}{3}
		\item $AH$, $SK$, $BC$ đồng quy.
		\item $SC \perp (BHK)$.
		\item $HK \perp (SBC)$.
	\end{enumEX}
	\loigiai{
		\begin{center}
			\begin{tikzpicture}[scale=1,line join=round, line cap=round, font=\footnotesize,>=stealth]
				\path (0,0) coordinate (A)++(0:5) coordinate (C)
				(-60:1.5) coordinate (B)
				(90:2.5) coordinate (S)
				($(A)!0.5!(C)$) coordinate (E)
				($(B)!0.5!(C)$) coordinate (M)
				($(C)!0.6!(S)$) coordinate (F)
				(intersection of A--M and B--E) coordinate (H)
				(intersection of B--F and S--M) coordinate (K)
				;
				\draw (S)--(A)--(B)--(C)--cycle
				(M)--(S)--(B)--(F);
				\draw[dashed] (C)--(A)--(M)
				(B)--(E)--(F) (H)--(K);
				\foreach \a/\b in {A/180,B/-90,C/0,M/-50,E/40,F/50,H/-90,S/90,K/190}{
					\fill[black] (\a)circle(.7pt)
					($(\a)+(\b:2mm)$)node[scale=.8]{$\a$};
				}
			\end{tikzpicture}
		\end{center}
		\begin{enumerate}
			\item Giả sử $AH \perp BC$ tại $M$.
			Ta có $\heva{&BC \perp AM \\ &BC \perp SA} \Rightarrow BC \perp (SAM) \Rightarrow BC \perp SM$. \\
			Mặt khác $SK \perp BC \Rightarrow S, K, M$ thẳng hàng do đó $AH, SK, BC$ đồng quy tại điểm $M$.
			\item Do $H$ là trực tâm tam giác $ABC$ nên $BH \perp AC$. \\
			Mặt khác $BH \perp SA \Rightarrow BH \perp (SAC) \Rightarrow BH \perp SC$.
			Lại có $BK \perp SC \Rightarrow SC \perp (BHK)$.
			\item Do $SC \perp (BHK) \Rightarrow SC \perp HK$, mặt khác $BC \perp (SAM) \Rightarrow BC \perp HK$.
			Do đó $HK \perp (SBC)$.
		\end{enumerate}
	}
\end{bt}
\begin{bt}%[1K7BM-2]
	Cho hình chóp $S.ABCD$ có đáy $ABCD$ là hình thoi tâm $O$ và có $SA=SC$, $SB=SD$.
	\begin{enumerate}
		\item Chứng minh rằng $SO \perp (ABCD)$.
		\item Gọi $I, K$ lần lượt là trung điểm của $BA$ và $BC$. Chứng minh rằng $IK \perp (SBD)$ và $IK \perp SD$.
	\end{enumerate}
	\loigiai{
		\immini{
			\begin{enumerate}
				\item Do $SA=AC \Rightarrow \triangle SAC$ cân tại $S$ có trung tuyến $SO$ đồng thời là đường cao suy ra $SO \perp AC$. \\
				Tương tự ta có $SO \perp BD \Rightarrow SO \perp (ABCD)$.
				\item Do $ABCD$ là hình thoi nên $AC \perp BD$. \\
				Mặt khác $SO \perp (ABCD) \Rightarrow AC \perp SO$.
				Do vậy $AC \perp (SBD)$.
				Mà $IK$ là đường trung bình trong tam giác $BAC$ nên $I K \parallel AC$ mà $AC \perp (SBD) \Rightarrow IK \perp (SBD)$.
			\end{enumerate}
		}
		{
			\begin{tikzpicture}[scale=1,line join=round, line cap=round, font=\footnotesize,>=stealth]
				\path (0,0) coordinate (A)++(0:4) coordinate (B)++(-150:2) coordinate (C)
				($(A)!.5!(C)$) coordinate (O)++(90:2.5) coordinate (S)
				($(O)!-1!(B)$) coordinate (D)
				($(B)!0.5!(A)$) coordinate (I)
				($(C)!0.5!(B)$) coordinate (K)
				;
				\draw (S)--(B)--(C)--(D)--cycle
				(C)--(S);
				\draw[dashed] (D)--(A)--(B)--cycle
				(O)--(S)--(A)--(C) (I)--(K);
				\foreach \a/\b in {A/150,B/0,C/-30,D/-120,S/40,I/50,K/-90,O/-90}{
					\fill[black] (\a)circle(.7pt)
					($(\a)+(\b:2mm)$)node[scale=.8]{$\a$};
				}
			\end{tikzpicture}
		}
	}
\end{bt}
\begin{bt}%[1K7BM-2]
	Cho hình chóp $S.ABCD$ có đáy là hình vuông cạnh $a$. Mặt bên $SAB$ là tam giác đều, $SCD$ là tam giác vuông cân đỉnh $S$. Gọi $I, J$ lần lượt là trung điểm của $AB$ và $CD$.
	\begin{enumerate}
		\item Tính các cạnh của tam giác $SIJ$, suy ra tam giác $SIJ$ vuông.
		\item Chứng minh rằng $SI \perp (SCD)$; $SJ \perp (SAB)$.
		\item Gọi $H$ là hình chiếu của $S$ lên $IJ$, chứng minh rằng $SH \perp (ABCD)$.
	\end{enumerate}
	\loigiai{
		\immini{
			\begin{enumerate}
				\item Ta có $\triangle SAB$ đều cạnh $a$ nên $SI=\dfrac{a\sqrt{3}}{2}$.\\
				Tứ giác $IBCJ$ là hình chữ nhật nên $IJ=BC=a$. \\
				$\triangle SCD$ là tam giác vuông cân đỉnh $S \Rightarrow SJ=\dfrac{CD}{2}=\dfrac{a}{2}$. \\
				Do đó $SJ^2+SI^2=IJ^2=a^2 \Rightarrow \triangle SIJ$ vuông tại $S$.
				\item Do $\triangle SCD$ cân tại $S$ nên $SJ \perp CD$.
				Do $AB \parallel CD \Rightarrow SJ \perp AB$. \\
				Mặt khác $SJ \perp SI \Rightarrow SJ \perp (SAB)$.
				Chứng minh tương tự ta có $S I \perp(S C D)$.
				\item Do $SI \perp (SCD) \Rightarrow SI \perp CD$. \\
				Mặt khác $CD \perp IJ \Rightarrow CD \perp (SIJ) \Rightarrow CD \perp SH$. \\
				Do $SH \perp IJ \Rightarrow SH \perp (ABCD)$.
			\end{enumerate}
		}
		{
			\begin{tikzpicture}[scale=1,line join=round, line cap=round, font=\footnotesize,>=stealth]
				\path (0,0) coordinate (A)++(0:4) coordinate (D)++(-150:2) coordinate (C)
				($(A)!.5!(C)$) coordinate (O)
				($(O)!-1!(D)$) coordinate (B)
				($(B)!0.5!(A)$) coordinate (I)
				($(C)!0.5!(D)$) coordinate (J)
				($(I)!1/3!(J)$) coordinate (H)++(90:3) coordinate (S)
				;
				\draw (S)--(B)--(C)--(D)--cycle
				(C)--(S)--(J);
				\draw[dashed] (D)--(A)--(B)
				(I)--(S)--(A) (S)--(H) (I)--(J);
				\foreach \a/\b in {A/150,B/180,C/-30,D/0,S/40,I/150,J/-20,H/-90}{
					\fill[black] (\a)circle(.7pt)
					($(\a)+(\b:2mm)$)node[scale=.8]{$\a$};
				}
			\end{tikzpicture}
		}
	}
\end{bt}
\begin{bt}%[1K7BM-2]
	Cho hình chóp $S.ABC$ có đáy $ABC$ là tam giác cân tại $A$, điểm $I$ và $H$ lần lượt là trung điểm của $AB$ và $BC$. Trên đoạn $CI$ và $SA$ lần lượt lấy hai điểm $M$, $N$ sao cho $MC=2MI$, $NA=2NS$. Biết $SH \perp (ABC)$, chứng minh $MN \perp (ABC)$.
	\loigiai{
		\immini{
			Do điểm $M$ thuộc đường trung tuyến $CI$ và $MC=2MI$
			$\Rightarrow M$ là trọng tâm tam giác $ABC \Rightarrow AH \cap CI$. \\
			Ta có $\dfrac{NA}{NS}=\dfrac{MA}{MH}=2 \Rightarrow MN \parallel SH$. \\
			Mặt khác $SH \perp (ABC) \Rightarrow MN  \perp (ABC)$.
		}
		{
			\begin{tikzpicture}[scale=1,line join=round, line cap=round, font=\footnotesize,>=stealth]
				\path (0,0) coordinate (A)++(0:4) coordinate (B)++(-130:2) coordinate (C)
				($(B)!.5!(C)$) coordinate (H)++(90:2.5) coordinate (S)
				($(B)!0.5!(A)$) coordinate (I)
				(intersection of A--H and C--I) coordinate (M)
				($(A)!2/3!(S)$) coordinate (N)
				;
				\draw (S)--(A)--(C)--(B)--cycle
				(H)--(S)--(C);
				\draw[dashed] (I)--(C)--(A)--(H)
				(M)--(N) (A)--(B);
				\foreach \a/\b in {A/150,B/0,C/-30,S/40,I/90,H/-20,M/-120,N/120}{
					\fill[black] (\a)circle(.7pt)
					($(\a)+(\b:2mm)$)node[scale=.8]{$\a$};
				}
			\end{tikzpicture}
		}
	}
\end{bt}
\subsection{Bài tập trắc nghiệm}
\Opensolutionfile{ans}[ans/ans-1K7-2-2]
\begin{ex}%[1K7BM-2]
	Cho hình chóp $S.ABC$ có đáy $ABC$ là tam giác vuông tại $B$ và $SA\perp (ABC)$. Mệnh đề nào sau đây đúng?
	\choice
	{$AC\perp (SAB)$}
	{\True $BC\perp (SAB)$}
	{$AB\perp (SBC)$}
	{$AC\perp (SBC)$}
	\loigiai{
		\immini{
			Ta có $\heva{&BC\perp AB \\&BC\perp SA}\Rightarrow BC\perp (SAB)$.
		}
		{
			\begin{tikzpicture}[scale=1,line join=round, line cap=round, font=\footnotesize,>=stealth]
				\path (0,0) coordinate (A)
				++(0:3) coordinate (C)++(-160:2.5) coordinate (B)
				(90:3) coordinate (S)
				;
				\draw (S)--(A)--(B)--(C)--cycle
				(S)--(B);
				\draw[dashed] (A)--(C);
				\foreach \a/\b in {A/180,B/-90,C/0,S/90}{
					\fill[black] (\a)circle(.7pt)
					($(\a)+(\b:2mm)$)node[scale=.8]{$\a$};
				}
			\end{tikzpicture}
		}
	}
\end{ex}
\begin{ex}%[1K7BM-2]
	Cho tứ diện $ABCD$ có hai tam giác $ABC$ và $ABD$ là hai tam giác đều. Gọi $M$ là trung điểm của $AB$. Mệnh đề nào sau đây đúng?
	\choice
	{$CM\perp (ABD)$}
	{\True $AB\perp (MCD)$}
	{$AB\perp (BCD)$}
	{$MD\perp (ABC)$}
	\loigiai{
		\immini{
			Ta có $\heva{&AB\perp MC \\&AB\perp MD}\Rightarrow AB\perp (MCD)$.
		}
		{
			\begin{tikzpicture}[scale=1,line join=round, line cap=round, font=\footnotesize,>=stealth]
				\path (0,0) coordinate (B)
				++(0:3) coordinate (D)++(-160:2.5) coordinate (C)
				(80:3) coordinate (A)
				($(A)!.5!(B)$) coordinate (M)
				;
				\draw (A)--(B)--(C)--(D)--cycle
				(A)--(C)--(M);
				\draw[dashed] (B)--(D)--(M);
				\foreach \a/\b in {B/180,C/-90,D/0,A/90,M/150}{
					\fill[black] (\a)circle(.7pt)
					($(\a)+(\b:2mm)$)node[scale=.8]{$\a$};
				}
			\end{tikzpicture}
		}
	}
\end{ex}
\begin{ex}%[1K7BM-2]
	Cho hình chóp $S.ABCD$ có đáy $ABCD$ là hình vuông và $SA\perp (ABCD)$. Mệnh đề nào sau đây \textbf{sai}?
	\choice
	{$BC\perp (SAB)$}
	{$CD\perp (SAD)$}
	{\True $AC\perp (SBD)$}
	{$BD\perp (SAC)$}
	\loigiai{
		\immini{
			Ta có $\heva{&BC\perp AB \\&BC\perp SA}\Rightarrow BC\perp (SAB)$.\\
			Ta có $\heva{&CD\perp AD \\&CD\perp SA}\Rightarrow CD\perp (SAD)$.\\
			Ta có $\heva{&BD\perp AC \\&BD\perp SA}\Rightarrow BD\perp (SAC)$.
		}
		{
			\begin{tikzpicture}[scale=1,line join=round, line cap=round, font=\footnotesize,>=stealth]
				\path (0,0) coordinate (A)++(0:3) coordinate (B)++(-140:1.5) coordinate (C)
				++(180:3) coordinate (D)
				(90:2) coordinate (S)
				;
				\draw (S)--(B)--(C)--(D)--cycle
				(S)--(C);
				\draw[dashed] (B)--(A)--(D) (A)--(S);
				\foreach \a/\b in {A/150,B/0,C/-30,D/180,S/90}{
					\fill[black] (\a)circle(.7pt)
					($(\a)+(\b:2mm)$)node[scale=.8]{$\a$};
				}
			\end{tikzpicture}
		} 
	}
\end{ex}
\begin{ex}%[1K7BM-2]
	Cho hình chóp $S.ABC$ có $SA\perp (ABC)$ và đáy $ABC$ là tam giác vuông cân tại $A$. Gọi $M$ là trung điểm $BC$. Mệnh đề nào sau đây đúng?
	\choice
	{$AB\perp (SBC)$}
	{\True $BC\perp (SAM)$}
	{$BC\perp (SAB)$}
	{$AC\perp (SBC)$}
	\loigiai{
		\immini{
			Ta có $\heva{&BC\perp AM \\&BC\perp SA}\Rightarrow BC\perp (SAM)$.
		}
		{
			\begin{tikzpicture}[scale=1,line join=round, line cap=round, font=\footnotesize,>=stealth]
				\path (0,0) coordinate (A)++(0:3) coordinate (C)++(-150:2.5) coordinate (B)
				($(B)!.5!(C)$) coordinate (M)
				(90:2.5) coordinate (S)
				;
				\draw (S)--(A)--(B)--(C)--cycle
				(B)--(S)--(M);
				\draw[dashed] (M)--(A)--(C);
				\foreach \a/\b in {A/180,B/180,C/0,M/-30,S/90}{
					\fill[black] (\a)circle(.7pt)
					($(\a)+(\b:2mm)$)node[scale=.8]{$\a$};
				}
			\end{tikzpicture}
		}
	}
\end{ex}
\begin{ex}%[1K7BM-2]
	Cho hình chóp $S.ABCD$ có đáy $ABCD$ là hình vuông và các cạnh bên bằng nhau. Mệnh đề nào sau đây \textbf{đúng}?
	\choice
	{$SA\perp (ABCD)$}
	{$AC\perp (SBC)$}
	{\True $AC\perp (SBD)$}
	{$AC\perp (SCD)$}
	\loigiai{
		\immini{
			Gọi $O$ là giao điểm của $AC$ và $BD$.\\
			Ta có $\heva{&SA=SB=SC=SD \\&OA=OB=OC=OD}\Rightarrow SO\perp (ABCD)$. \\
			Ta có $\heva{&AC\perp BD \\&AC\perp SO}\Rightarrow AC\perp (SBD)$.
		}
		{
			\begin{tikzpicture}[scale=1,line join=round, line cap=round, font=\footnotesize,>=stealth]
				\path (0,0) coordinate (A)++(0:3) coordinate (B)++(-140:1.5) coordinate (C)
				++(180:3) coordinate (D)
				($(A)!.5!(C)$) coordinate (O)++(90:2.5) coordinate (S)
				;
				\draw (S)--(B)--(C)--(D)--cycle
				(S)--(C);
				\draw[dashed] (B)--(A)--(D)--cycle
				(C)--(A)--(S)--(O);
				\foreach \a/\b in {A/150,B/0,C/-30,D/180,S/90,O/-90}{
					\fill[black] (\a)circle(.7pt)
					($(\a)+(\b:2mm)$)node[scale=.8]{$\a$};
				}
			\end{tikzpicture}
		} 
	}
\end{ex}
\begin{ex}%[1K7BM-2]
	Cho hình chóp $S.ABCD$ có đáy $ABCD$ là hình chữ nhật $AB=a$, $AD=a\sqrt{2}$ và $SA\perp (ABCD)$. Mệnh đề nào sau đây đúng?
	\choice
	{\True $BC\perp SB$}
	{$CD\perp SD$}
	{$BD\perp SC$}
	{$SA\perp AB$}
	\loigiai{
		\immini{
			Ta có $\heva{&BC\perp AB \\&BC\perp SA}\Rightarrow BC\perp (SAB)\Rightarrow BC\perp SB$.
		}
		{
			\begin{tikzpicture}[scale=1,line join=round, line cap=round, font=\footnotesize,>=stealth]
				\path (0,0) coordinate (A)++(0:3) coordinate (B)++(-140:1.5) coordinate (C)
				++(180:3) coordinate (D)
				(90:2) coordinate (S)
				;
				\draw (S)--(B)--(C)--(D)--cycle
				(S)--(C);
				\draw[dashed] (B)--(A)--(D) (A)--(S);
				\foreach \a/\b in {A/150,B/0,C/-30,D/180,S/90}{
					\fill[black] (\a)circle(.7pt)
					($(\a)+(\b:2mm)$)node[scale=.8]{$\a$};
				}
			\end{tikzpicture}
		}
	}
\end{ex}
\begin{ex}%[1K7BM-2]
	Cho hình chóp $S.ABCD$ có $SA=SB=SC=SD$ và đáy $ABCD$ là hình thoi tâm $O$. Mệnh đề nào đúng?
	\choice
	{$BC\perp (SAB)$}
	{\True $SO\perp (ABCD)$}
	{$CD\perp (SAD)$}
	{$SA\perp (ABCD)$}
	\loigiai{
		\immini{
			Ta có $\heva{&SA=SB=SC=SD \\&OA=OB=OC=OD}\Rightarrow SO\perp (ABCD)$.
		}
		{
			\begin{tikzpicture}[scale=1,line join=round, line cap=round, font=\footnotesize,>=stealth]
				\path (0,0) coordinate (A)++(0:3) coordinate (B)++(-140:1.5) coordinate (C)
				++(180:3) coordinate (D)
				($(A)!.5!(C)$) coordinate (O)++(90:2.5) coordinate (S)
				;
				\draw (S)--(B)--(C)--(D)--cycle
				(S)--(C);
				\draw[dashed] (B)--(A)--(D)--cycle
				(C)--(A)--(S)--(O);
				\foreach \a/\b in {A/150,B/0,C/-30,D/180,S/90,O/-90}{
					\fill[black] (\a)circle(.7pt)
					($(\a)+(\b:2mm)$)node[scale=.8]{$\a$};
				}
			\end{tikzpicture}
		}
	}
\end{ex}
\begin{ex}%[1K7BM-2]
	Cho hình chóp $S.ABCD$ có đáy là hình vuông tâm $O$, cạnh $SA$ vuông góc với mặt phẳng đáy. Hỏi đường thẳng $BD$ vuông góc với mặt phẳng nào dưới đây?
	\choice
	{$(SAB)$}
	{$(SAD)$}
	{\True $(SAC)$}
	{$(SCD)$}
	\loigiai{
		\immini{
			Ta có $\heva{&BD\perp AC \\ &BD\perp SA} \Rightarrow BD\perp (SAC)$
		}
		{
			\begin{tikzpicture}[scale=1,line join=round, line cap=round, font=\footnotesize,>=stealth]
				\path (0,0) coordinate (A)++(0:3) coordinate (B)++(-140:1.5) coordinate (C)
				++(180:3) coordinate (D)
				(90:2) coordinate (S)
				;
				\draw (S)--(B)--(C)--(D)--cycle
				(S)--(C);
				\draw[dashed] (B)--(A)--(D)--cycle (C)--(A)--(S);
				\foreach \a/\b in {A/150,B/0,C/-30,D/180,S/90}{
					\fill[black] (\a)circle(.7pt)
					($(\a)+(\b:2mm)$)node[scale=.8]{$\a$};
				}
			\end{tikzpicture}
		}
	}
\end{ex}
\begin{ex}%[1K7BM-2]
	Cho hình chóp $S.ABCD$ có đáy là hình chữ nhật tâm $O$, cạnh $SA$ vuông góc với mặt phẳng đáy. Hỏi đường thẳng $BC$ vuông góc với mặt phẳng nào trong các mặt phẳng sau đây?
	\choice
	{\True $(SAB)$}
	{$(SAC)$}
	{$(SAD)$}
	{$(SCD)$}
	\loigiai{
		\immini{
			Do $SA\perp (ABCD)\Rightarrow SA\perp BC$.\\
			Mặt khác $BC\perp AB\Rightarrow BC\perp (SAB)$.
		}
		{
			\begin{tikzpicture}[scale=1,line join=round, line cap=round, font=\footnotesize,>=stealth]
				\path (0,0) coordinate (A)++(0:3) coordinate (B)++(-140:1.5) coordinate (C)
				++(180:3) coordinate (D)
				(90:2) coordinate (S)
				;
				\draw (S)--(B)--(C)--(D)--cycle
				(S)--(C);
				\draw[dashed] (B)--(A)--(D)--cycle (C)--(A)--(S);
				\foreach \a/\b in {A/150,B/0,C/-30,D/180,S/90}{
					\fill[black] (\a)circle(.7pt)
					($(\a)+(\b:2mm)$)node[scale=.8]{$\a$};
				}
			\end{tikzpicture}
		}
	}
\end{ex}
\begin{ex}%[1K7BM-2]
	Cho hình chóp $S.ABCD$ có đáy là hình chữ nhật tâm $O$, cạnh $SA$ vuông góc với mặt phẳng đáy. Gọi $H$ và $K$ lần lượt là hình chiếu của $A$ lên $SB$ và $SD$. Hỏi đường thẳng $SC$ vuông góc với mặt phẳng nào trong các mặt phẳng sau đây?
	\choice
	{\True $(AHK)$}
	{$(AHD)$}
	{$(AKB)$}
	{$(SBD)$}
	\loigiai{
		\immini{
			Do $SA\perp (ABCD)\Rightarrow SA\perp BC$. \\
			Khi đó $\heva{&BC\perp SA \\&BC\perp AB}\Rightarrow BC\perp (SAB)\Rightarrow BC\perp AH$. \\
			Lại có $AH\perp SB\Rightarrow AH\perp (SBC)\Rightarrow AH\perp SC$ \hfill $(1)$ \\
			Tương tự chứng minh trên ta có: $AK\perp SC$ \hfill $(2)$ \\
			Từ $(1) và $(2) suy ra $SC\perp (AHK)$.
		}
		{
			\begin{tikzpicture}[scale=1,line join=round, line cap=round, font=\footnotesize,>=stealth]
				\path (0,0) coordinate (A)++(0:3) coordinate (B)++(-140:1.5) coordinate (C)
				++(180:3) coordinate (D)
				(90:2) coordinate (S)
				($(S)!1/3!(B)$) coordinate (H)
				($(S)!1/3!(D)$) coordinate (K)
				;
				\draw (S)--(B)--(C)--(D)--cycle
				(S)--(C);
				\draw[dashed] (B)--(A)--(D)--cycle (C)--(A)--(S)
				(H)--(A)--(K)--cycle;
				\foreach \a/\b in {A/150,B/0,C/-30,D/180,S/90,H/20,K/160}{
					\fill[black] (\a)circle(.7pt)
					($(\a)+(\b:2mm)$)node[scale=.8]{$\a$};
				}
			\end{tikzpicture}
		}
	}
\end{ex}
\begin{ex}%[1K7BM-2]
	Cho hình chóp $S.ABC$ có các cạnh $SA$, $SB$, $SC$ bằng nhau. Hỏi trong các mặt phẳng trung trực của các đoạn thẳng $AB$, $BC$, $CA$ có bao nhiêu mặt phẳng chứa điểm $S$?
	\choice
	{$0$}
	{$1$}
	{$2$}
	{\True $3$}
	\loigiai{
		Do $\heva{&SA=SB \\&SB=SC \\&SC=SA}\Rightarrow S$ thuộc mặt phẳng trung trực của $AB$, $BC$ và $AC$.
	}
\end{ex}
\begin{ex}%[1K7BM-2]
	Cho hình chóp $S.ABC$ có các cạnh $SA$, $SB$, $SC$ đôi một vuông góc với nhau. Khi đó hình chiếu của $S$ lên mặt phẳng $(ABC)$ là
	\choice
	{Giao điểm của các đường trung tuyến của tam giác $ABC$}
	{Giao điểm của các đường phân giác của tam giác $ABC$}
	{Giao điểm của các đường trung trực của tam giác $ABC$}
	{\True Giao điểm của các đường cao của tam giác $ABC$}
	\loigiai{
		\immini{
			Do $\heva{&SC\perp SB \\&SC\perp SA}\Rightarrow SC\perp (SAB)\Rightarrow SC\perp AB$. \\
			Dựng $SH\perp (ABC)\Rightarrow SH\perp AB$.
			Do đó $AB\perp (SHC)\Rightarrow AB\perp CH$ \hfill $(1)$ \\
			Tương tự chứng minh trên ta có $AH\perp BC$ \hfill $(2)$ \\
			Từ $(1)$ và $(2) \Rightarrow H$ là trực tâm tam giác $ABC$.
		}
		{
			\begin{tikzpicture}[scale=1,line join=round, line cap=round, font=\footnotesize,>=stealth]
				\path (0,0) coordinate (S)++(0:3) coordinate (B)++(-160:4) coordinate (A)
				(90:2) coordinate (C)
				($(A)!.5!(B)$) coordinate (K)
				($(C)!2/3!(K)$) coordinate (H)
				;
				\draw (C)--(B)--(A)--cycle
				(C)--(K);
				\draw[dashed] (B)--(S)--(A) 
				(C)--(S)--(K) (S)--(H)
				;
				\foreach \a/\b in {A/180,B/0,C/90,K/-40,S/180,H/20}{
					\fill[black] (\a)circle(.7pt)
					($(\a)+(\b:2mm)$)node[scale=.8]{$\a$};
				}
			\end{tikzpicture}
		}
	}
\end{ex}
\begin{ex}%[1K7BM-2]
	Cho hình chóp $S.ABCD$ có đáy $ABCD$ là hình vuông và $SA\perp (ABCD)$. Từ $A$, kẻ $AM\perp SB$ với $M\in SB$. Mệnh đề nào sau đây đúng?
	\choice
	{$SB\perp (MAC)$}
	{$AM\perp (SAD)$}
	{$AM\perp (SBD)$}
	{\True $AM\perp (SBC)$}
	\loigiai{
		\immini{
			Do $SA\perp (ABCD)\Rightarrow SA\perp BC$. \\
			Khi đó $\heva{&BC\perp SA \\&BC\perp AB}\Rightarrow BC\perp (SAB)\Rightarrow BC\perp AM$. \\
			Lại có $AM\perp SB\Rightarrow AM\perp (SBC)$.
		}
		{
			\begin{tikzpicture}[scale=1,line join=round, line cap=round, font=\footnotesize,>=stealth]
				\path (0,0) coordinate (A)++(0:3) coordinate (B)++(-140:1.5) coordinate (C)
				++(180:3) coordinate (D)
				(90:2) coordinate (S)
				($(S)!1/3!(B)$) coordinate (M)
				;
				\draw (S)--(B)--(C)--(D)--cycle
				(S)--(C);
				\draw[dashed] (B)--(A)--(D)--cycle (C)--(A)--(S)
				(A)--(M);
				\foreach \a/\b in {A/150,B/0,C/-30,D/180,S/90,M/30}{
					\fill[black] (\a)circle(.7pt)
					($(\a)+(\b:2mm)$)node[scale=.8]{$\a$};
				}
			\end{tikzpicture}
		}
	}
\end{ex}
\begin{ex}%[1K7BM-2]
	Cho hình chóp $S.ABC$ có $SA\perp (ABC)$ và $\Delta ABC$ vuông ở $B$. Gọi $AH$ là đường cao của $\Delta SAB$. Khẳng định nào sau đây \textbf{sai}?
	\choice
	{$AH\perp SB$}
	{$AH\perp BC$}
	{\True $AH\perp AC$}
	{$AH\perp SC$}
	\loigiai{
		\immini{
			Ta có $SA\perp (ABC)\Rightarrow SA\perp BC$. \\
			Mặt khác $\triangle ABC$ vuông ở $B \Rightarrow AB\perp BC$. \\
			Do đó $BC\perp (SAB)\Rightarrow BC\perp AH$.\\
			Lại có $AH\perp SB\Rightarrow AH\perp (SBC)$.
		}
		{
			\begin{tikzpicture}[scale=1,line join=round, line cap=round, font=\footnotesize,>=stealth]
				\path (0,0) coordinate (A)++(0:3) coordinate (B)++(-155:2.5) coordinate (C)
				(90:2) coordinate (S)
				($(S)!1/3!(B)$) coordinate (H)
				;
				\draw (S)--(B)--(C)--(A)--cycle
				(S)--(C);
				\draw[dashed] (B)--(A)--(H);
				\foreach \a/\b in {A/180,B/0,C/-90,H/30,S/90}{
					\fill[black] (\a)circle(.7pt)
					($(\a)+(\b:2mm)$)node[scale=.8]{$\a$};
				}
			\end{tikzpicture}
		}
	}
\end{ex}
\begin{ex}%[1K7BM-2]
	Cho hình chóp $S.ABC$ có cạnh $SA\perp (ABC)$ và đáy $ABC$ là tam giác cân tại $C$. Gọi $H$ và $K$ lần lượt là trung điểm $AB$ và $SB$. Khẳng định nào sau đây \textbf{sai}?
	\choice
	{$CH\perp AK$}
	{$CH\perp SB$}
	{$CH\perp SA$}
	{\True $AK\perp SB$}
	\loigiai{
		\immini{
			Do $SA\perp (ABC)\Rightarrow SA\perp CH$. \\
			Ta có $\triangle ABC$ là tam giác cân ở $C \Rightarrow CH\perp AB$ (tam giác cân có đường trung tuyến đồng thời là đường cao).\\
			Mặt khác $CH\perp SA\Rightarrow CH\perp (SAB)\Rightarrow \heva{&CH\perp AK \\&CH\perp SA \\&CH\perp SB.}$
		}
		{
			\begin{tikzpicture}[scale=1,line join=round, line cap=round, font=\footnotesize,>=stealth]
				\path (0,0) coordinate (A)++(0:3) coordinate (B)++(-155:2.5) coordinate (C)
				(90:2) coordinate (S)
				($(A)!.5!(B)$) coordinate (H)
				($(S)!.5!(B)$) coordinate (K)
				;
				\draw (S)--(B)--(C)--(A)--cycle
				(S)--(C)--(K);
				\draw[dashed] (B)--(A)--(K)--(H)--(C);
				\foreach \a/\b in {A/180,B/0,C/-90,H/-50,S/90,K/40}{
					\fill[black] (\a)circle(.7pt)
					($(\a)+(\b:2mm)$)node[scale=.8]{$\a$};
				}
			\end{tikzpicture}
		}
	}
\end{ex}
\begin{ex}%[1K7BM-4]
	\immini{Cho hình chóp $S.ABCD$, có $ABCD$ là hình vuông và $SA \perp (ABCD)$. Khẳng định nào sau đây là \textbf{sai}?	
		\choice
		{$BC\perp SB$}
		{$BD\perp SC$}
		{$AB\perp SA$}
		{\True $AB\perp SC$}
	}{
		\begin{tikzpicture}[font=\footnotesize, line join=round, line cap=round, >=stealth, scale=.8]
			\coordinate (A) at (0,0);
			\coordinate (D) at (4,0);
			\coordinate (C) at (2,-2);
			\coordinate (B) at ($(A)+(C)-(D)$);
			\coordinate (S) at ($(A)+(0,4)$);
			\draw (S)--(B)--(C)--(D)--cycle (S)--(C);
			\draw[dashed] (A)--(S) (B)--(A)--(D);
			\foreach \p/\g in {A/180, B/-90, C/-90, D/0, S/90} \draw[fill=black] (\p) circle(1.5pt) node [shift={(\g:.3)}] {$\p$};
		\end{tikzpicture}
	}
	\loigiai{
		$\bullet \ BC\perp (SAB) \Rightarrow BC\perp SB$ -> loại.\\
		$\bullet \ BD\perp (SAC) \Rightarrow BD\perp SC$ -> loại.\\
		$\bullet \ SA\perp (ABCD) \Rightarrow SA \perp AB$ -> loại.	
	}
\end{ex}
\Closesolutionfile{ans}
% \begin{indapan}{10}
% 	{ans/ans-1K7-2-2}
% \end{indapan}

\begin{dang}{Xác định hai đường thẳng vuông góc}
	\begin{dl}
		Nếu đường thẳng vuông góc với mặt phẳng thì đường thẳng đó vuông góc với mọi đường thẳng nằm trong mặt phẳng đó.
	\end{dl}
	\begin{hq}
		Nếu đường thẳng vuông góc với hai cạnh của tam giác thì đường thẳng đó cũng vuông góc với cạnh thứ ba của tam giác.
	\end{hq}
\end{dang}
\subsubsection{Ví dụ mẫu}
\begin{vd}[TH]%[Đỗ Chí Tâm KNTT]%[1K7BM-3]
	Cho hình chóp $S.ABCD$ có đáy là hình vuông, $SA\perp (ABCD)$.
	\begin{listEX}[2]
		\item Chứng minh $BD\perp SC$.
		\item Chứng minh $\Delta SBC$ vuông.
	\end{listEX}	
	\loigiai{
		\immini{
			\begin{enumerate}
				\item Ta có $\heva{&BD \perp AC \text{ (do }ABCD\text{ là hình vuông)}\\& BD\perp SA \text{ (do }SA\perp (ABCD))}$\\
				\hspace*{1cm}$\Rightarrow BD\perp (SAC)\Rightarrow BD\perp SC$.
				\item Ta có $\heva{&BC \perp AB \text{ (do ABCD là hình vuông)}\\& BC\perp SA \text{ (do }SA\perp (ABCD))}$\\
				\hspace*{1cm} $\Rightarrow BC\perp SB$.\\
				Vậy tam giác $SBC$ vuông tại $B$.				
			\end{enumerate}
		}
		{
			\begin{tikzpicture}[scale=0.5,font=\footnotesize, line join=round, line cap=round, >=stealth]
				\coordinate (A) at (0,0);
				\coordinate (B) at (-2,-3);
				\coordinate (D) at (7,0);
				\coordinate (C) at ($(B)+(D)-(A)$);
				\coordinate (S) at ($(A)+(0,5)$);
				\draw(S)--(B) (S)--(C) (S)--(D) (B)--(C)--(D);
				\draw[dashed,thin](S)--(A) (A)--(B) (A)--(D) (B)--(D) (A)--(C);
				\pic[draw,thin,angle radius=2mm] {right angle = S--A--D} pic[draw,thin,angle radius=3mm] {right angle = S--A--B};
				\foreach \i/\g in {S/90,A/-90,B/-90,C/-90,D/0}{\draw[fill=white](\i) circle (1.5pt) ($(\i)+(\g:3mm)$) node[scale=1]{$\i$};}
			\end{tikzpicture}
		}
	}  
\end{vd}

\begin{vd}[TH]%[Đỗ Chí Tâm KNTT]%[1K7BM-3]
	Cho hình chóp đều $S . ABC$, chứng minh $BC\perp SA; AB\perp SC; AC\perp SB.$
	\loigiai{
		\immini
		{
			Gọi $M$ là trung điểm $BC$, $\Delta ABC$ đều và $\Delta SBC$ cân tại $S$.\\
			Ta có $\heva{&BC\perp AM\\&BC\perp SM}\Rightarrow BC\perp (SAM)\Rightarrow BC\perp SA$.\\
			$AC\perp SB$ và $AB\perp SC$ được chứng minh tương tự.
		}{
			\begin{tikzpicture}[scale=0.8,font=\footnotesize, line join=round, line cap=round, >=stealth]
				\path 
				(0,0) coordinate (A)
				(5,0) coordinate (C)
				(1,-1.5) coordinate (B)	
				($(C)!.5!(B)$) coordinate (M)
				($(A)!2/3!(M)$) coordinate (O)
				($(O)+(0,3)$) coordinate (S)
				;
				\draw (A)--(S)--(M) (B)--(S) (S)--(C) (A)--(B)--(C);
				\draw[dashed] (A)--(C) (S)--(O) (C)--(O)--(B) (A)--(M);
				\path (B)--(M) node[midway,sloped]{$|$};
				\path (C)--(M) node[midway,sloped]{$|$};
				\pic[draw,angle radius=2mm,angle eccentricity=1.5] {right angle = A--M--B};
				\foreach \x/\g in {A/180,B/-90,C/0,S/90,M/-90,O/-90} \fill[black] (\x) circle (1pt)+(\g:.3) node {$\x$};
			\end{tikzpicture}
		}		
	}
\end{vd}

\begin{vd}[VD]%[Đỗ Chí Tâm KNTT]%[1K7KM-3]
	Cho hình lập phương $ABCD.A'B'C'D'$, gọi $O=AC\cap BD$, kẻ $AH\perp A'O$. Chứng minh $AH\perp A'B$.
	\loigiai{
		\immini
		{
			Ta có $\heva{&BD\perp AC\\&BD\perp AA'}\Rightarrow BD\perp (ACC'A')\Rightarrow BD\perp AH$.\\
			Mặt khác $\heva{&AH\perp A'O\\&AH\perp BD}\Rightarrow AH\perp (A'BD)$\\
			Suy ra $AH\perp A'B$.
		}
		{
			\begin{tikzpicture}[>=stealth,line join=round,line cap=round,font=\footnotesize,scale=0.7]
				\coordinate[label=above left:{$A$}] (A) at (0,0);
				\coordinate[label=below left:{$B$}] (B) at (-1.75,-1.5);
				\coordinate[label=right:{$D$}] (D) at (4.5,0);
				\coordinate[label=below right:{$C$}] (C) at ($(B)+(D)-(A)$);
				\coordinate[label=above left:{$A'$}] (A') at (0,4.5);
				\coordinate[label=above right:{$D'$}] (D') at ($(D)+(A')-(A)$);
				\coordinate[label=above left:{$B'$}] (B') at ($(A')+(B)-(A)$);
				\coordinate[label=above left:{$C'$}] (C') at ($(B')+(D')-(A')$);
				\coordinate[label=below:{$O$}] (O) at (intersection of A--C and B--D);
				\coordinate[label=right:{$H$}] (H) at ($(A')!.7!(O)$);
				\draw (A')--(B')--(C')--(D')--cycle (B)--(B') (C)--(C') (D)--(D')
				(B)--(C)--(D);
				\draw[dashed] (A)--(A') (B)--(A)--(D) (B)--(A')--(D) (A')--(O) (B)--(D) (A)--(C) (A)--(H);
				\foreach \p in {A,B,C,D,A',B',C',D',O}
				\fill (\p)	circle (1.2pt);	
				\pic[draw,angle radius=2mm,angle eccentricity=1.5] {right angle = D--O--A'};
				\pic[draw,angle radius=2mm,angle eccentricity=1.5] {right angle = A--O--B};
				\pic[draw,angle radius=2mm,angle eccentricity=1.5] {right angle = A--H--O};
			\end{tikzpicture}
		}
	}
\end{vd}

\subsubsection{Bài tập rèn luyện}
% \centerline{\fcolorbox{red}{yellow!50}{\bf {BÀI TẬP TỰ LUẬN}}}
\begin{bt}%[Đỗ Chí Tâm KNTT]%[1K7BM-3]
	Cho hình chóp $S.ABCD$ có $SA \perp(ABCD)$. Cho biết $ABCD$ là hình thang vuông tại $A$ và $D$, $AB=2AD$.
	\begin{enumerate}
		\item Chứng minh $CD \perp SD$;
		\item Gọi $M$ là trung điểm của $AB$. Chứng minh $CM \perp SB$.
	\end{enumerate}
	\loigiai{
		\immini{
			\begin{enumerate}
				\item Ta có $CD \perp AD \quad (1)$.\\
				Mặt khác $CD \perp SA \quad (2)$ do $SA \perp (ABCD) $.\\
				Từ $(1)$ và $(2)$ suy ra $CD \perp (SAD) \Rightarrow CD\perp SD$.
				\item  Ta có $SA \perp CM \quad (3)$ do $SA \perp (ABCD) $.\\
				Mặt khác, $CM \parallel AD \Rightarrow CM \perp AB \quad (4)$.\\
				Từ $(3)$ và $(4)$ suy ra $CM\perp (SAB)\Rightarrow CM\perp SB.$
			\end{enumerate}
		}
		{
			\begin{tikzpicture}[scale=0.5, font=\footnotesize, line join=round, line cap=round, >=stealth]
				\coordinate (A) at (0,0);
				\coordinate (D) at (-2,-3);
				\coordinate (B) at (8,0);
				\coordinate (C) at ($(D)+(4.5,0)-(A)$);
				\coordinate (S) at ($(A)+(0,5)$);
				\coordinate (M) at ($(A)!0.5!(B)$);
				\draw(S)--(D) (S)--(C) (S)--(B) (D)--(C)--(B);
				\draw[dashed,thin](S)--(A) (A)--(D) (A)--(B) (C)--(M);
				\pic[draw,thin,angle radius=2mm] {right angle = S--A--B};
				\foreach \i/\g in {S/90,A/-90,B/-90,C/-90,D/-90,M/90}{\draw[fill=black](\i) circle (1.5pt) ($(\i)+(\g:3mm)$) node[scale=1]{$\i$};}
			\end{tikzpicture}
		}
	}
\end{bt}

\begin{bt}%[Đỗ Chí Tâm KNTT]%[1K7BM-3]
	Cho hình lăng trụ đứng $ABC.A'B'C'$ có đáy $ABC$ là tam giác đều. Gọi $M$ là trung điểm $BC$, $H$ là hình chiếu của $A$ lên $A'M$. Chứng minh $AH\perp A'B$.
	\loigiai{
		\immini{
			Ta có $\heva{&BC\perp AM\\&BC\perp AA'}\Rightarrow BC\perp AH$.\\
			$\heva{&AH\perp BC\\&AH\perp A'M}\Rightarrow AH\perp (A'BC)\Rightarrow AH\perp A'B$.
		}{
			\begin{tikzpicture}[scale=1,line join=round,line cap=round,font=\footnotesize,>=stealth]
				\path (0,0) coordinate (A)  
				(2,-2) coordinate (B)
				(5,0) coordinate (C)
				(0,4) coordinate (A')
				(2,2) coordinate (B')
				(5,4) coordinate (C');
				\path ($(B)!0.5!(C)$) coordinate (M);
				\path ($(A')!(A)!(M)$) coordinate (H);				
				\draw[dashed] (A)--(C) (A')--(M) (A)--(M) (A')--(C) (A)--(H);
				\draw (A)--(B)--(C)--(C')--(B')--(A')--(A) (B)--(B') (A')--(C') (A')--(B);
				\foreach \p/\q in {A/180, B/-90, C/-90, A'/90, B'/70,C'/90, M/-30, H/10}
				\fill (\p) circle(2pt) node[shift={(\q:3mm)}]{$\p$};
				\pic[draw,angle radius=4]{right angle=A--H--M};
				\pic[draw,angle radius=4]{right angle=A--M--B};
			\end{tikzpicture}
		}
	}
	
\end{bt}
\begin{bt}%[Đỗ Chí Tâm KNTT]%[1K7KM-3]
	Cho hai tam giác cân và bằng nhau $ABC$ và $ABD$ có đáy chung $AB$ và không cùng nằm trong một mặt phẳng.
	\begin{listEX}[2]
		\item Chứng minh rằng $AB \perp CD$.
		\item Xác định đoạn vuông góc chung của $AB$ và $CD$.
	\end{listEX}
	\loigiai{
		\immini{
			\begin{enumerate}
				\item[a)] Gọi $M$ là trung điểm của $AB$, vì $ABC$ và $ABD$ là các tam giác cân nên
				$$\heva{&AB\perp CM\\& AB \perp DM\\& CM\cap DM = M\\& CM \subset (MCD) \\ & DM \subset (MCD)}\Rightarrow AB \perp (MCD)\Rightarrow AB \perp CD.$$
				\item[b)] Gọi $N$ là trung điểm của $CD$, vì $ABC$ và $ABD$ là các tam giác cân và bằng nhau nên $CM = DM$, do đó tam giác $CMD$ cân tại $M$ suy ra $MN \perp CD$.\\
				Ta có $\heva{&MN \perp CD\\&MN \perp AB }$. \\
				Vậy $MN$ là đoạn vuông góc chung của $AB$ và $CD$.
			\end{enumerate}
		}{
			\begin{tikzpicture}[scale=.8, font=\footnotesize, line join=round, line cap=round, >=stealth]
				\def\ac{4} % cạnh AC
				\def\ab{2} % cạnh AB
				\def\h{4} % chiều cao
				\def\gocA{50} % góc A của đáy
				\coordinate[label=left:$A$] (A) at (0,0);
				\coordinate[label=right:$C$] (C) at (\ac,0);
				\coordinate[label=below left:$B$] (B) at (-\gocA:\ab);
				\coordinate[label=above:$D$] (S) at ($(A)+(60:\h)$);
				\coordinate[label={left}:$M$] (M) at ($(A)!0.5!(B)$);
				\coordinate[label={above right}:$N$] (N) at ($(C)!0.5!(S)$);
				\pic[draw,angle radius=6]{right angle=A--M--S};
				\pic[draw,angle radius=5]{right angle=M--N--S};
				\pic[draw,angle radius=5]{right angle=N--M--A};
				\pic[draw,angle radius=5]{right angle=C--M--B};
				\draw (A)--(B)--(C)--(S)--cycle (S)--(B) (S)--(M);
				\draw[dashed] (A)--(C) (C)--(M) (M)--(N);
				\foreach \diem in {A,B,C,S,M,N}	\fill (\diem)circle(1pt);
		\end{tikzpicture}}
	}
\end{bt}

\begin{bt}%[Đỗ Chí Tâm KNTT]%[1K7KM-3]
	Cho hình vuông $ABCD$. Gọi $H$, $K$ lần lượt là trung điểm của $AB$, $AD$. Trên đường thẳng vuông góc với $(ABCD)$ tại $H$, lấy điểm $S$. Chứng minh rằng:
	\begin{enumEX}{2}
		\item $AC \perp SK$
		\item $CK \perp SD$.
	\end{enumEX}
	\loigiai{
		\begin{center}
			\begin{tikzpicture}[scale=.65, font=\footnotesize, line join=round, line cap=round, >=stealth]
				\coordinate (A) at (0,0);
				\coordinate (B) at (-2,-3);
				\coordinate (D) at (7,0);
				\coordinate (C) at ($(B)+(D)-(A)$);
				\coordinate (S) at ($(-1,0)+(0,5)$);
				\coordinate (H) at ($(A)!0.5!(B)$);
				\coordinate (K) at ($(A)!0.5!(D)$);
				\coordinate (I) at (intersection of H--D and C--K);
				\draw(S)--(B) (S)--(C) (S)--(D) (B)--(C)--(D);
				\draw[dashed,thin](S)--(A) (A)--(B) (A)--(D) (S)--(H) (S)--(K)--(H) (A)--(C) (H)--(D) (C)--(K);
				\foreach \i/\g in {S/90,A/-90,B/-90,C/-90,D/0,H/180,K/45,I/-30}{\draw[fill=black](\i) circle (1.5pt) ($(\i)+(\g:3mm)$) node[scale=1]{$\i$};}
			\end{tikzpicture}
		\end{center}
		\begin{enumerate}
			\item Theo đề bài ta có $SH \perp (ABCD)$ mà $AC \subset (ABCD)$ nên $SH \perp AC \quad (1)$.\\
			Vì $HK$ là đường trung bình của $\triangle ABD \Rightarrow HK \parallel BD$.\\
			Mà $BD \perp AC$ (Vì $BD$ và $AC$ là hai đường chéo của hình vuông $ABCD$).\\
			Suy ra $HK \perp AC \quad (2)$.\\
			Từ $(1)$ và $(2)$ ta được $\heva{&AC \perp SH \subset (SHK)\\ & AC \perp HK \subset (SHK)\\ & SH \cap HK =H} \Rightarrow AC \perp (SHK)\Rightarrow AC\perp SK$.
			\item Gọi $I=CK \cap DH$.\\
			Suy ra, $\triangle IDC$ có $\widehat{IDC}+\widehat{ICD}=\widehat{IDC}+\widehat{ADH}=90^\circ \Rightarrow CK \perp DH \quad (3)$.\\
			Mà $AH \perp (ABCD) \Rightarrow AH \perp CK \quad (4)$.\\
			Từ $(3)$ và $(4)\Rightarrow CK \perp (SDH)\Rightarrow CK\perp SD$.
		\end{enumerate}
	}
\end{bt}
\subsection{Bài tập trắc nghiệm}
\Opensolutionfile{ans}[ans/ans-1K7-2-3]
\begin{ex}%[Đỗ Chí Tâm KNTT]%[1K7BM-3]
	Chọn khẳng định đúng trong các khẳng định sau
	\choice
	{Trong không gian, hai mặt phẳng cùng vuông góc với một đường thẳng thì song song với nhau}
	{\True Trong không gian, hai đường thẳng vuông góc với nhau có thể cẳt nhau hoặc chéo nhau}
	{Trong không gian, hai đường thẳng không có điểm chung thì song song với nhau}
	{Trong không gian, hai đường thẳng phân biệt cùng vuông góc với một đường thẳng thì song song với nhau}
	\loigiai{
		Khẳng định đúng là khẳng định ``Trong không gian hai đường thẳng vuông góc với nhau có thể cẳt nhau hoặc chéo nhau''.
	}
\end{ex}

\begin{ex}%[Đỗ Chí Tâm KNTT]%[1K7BM-3]
	Chọn khẳng định đúng trong các khẳng định sau
	\choice
	{Trong không gian, một đường thẳng vuông góc với hai cạnh của một hình bình hành thì sẽ vuông góc với tất cả các cạnh của hình bình hành đó.}
	{Trong không gian, hai đường thẳng cùng vuông góc với một đường thẳng thứ ba thì song song nhau.}
	{\True Trong không gian, đường thẳng vuông góc với hai cạnh của một tam giác thì vuông góc với cạnh còn lại của tam giác đó.}
	{Trong không gian, một đường thẳng vuông góc với hai đường thẳng cùng nằm trong mặt phẳng thì vuông góc với mặt phẳng đó.}
	\loigiai{
		Khẳng định đúng là khẳng định ``Trong không gian, đường thẳng vuông góc với hai cạnh của một tam giác thì vuông góc với cạnh còn lại của tam giác đó''.
	}
\end{ex}

\begin{ex}%[Đỗ Chí Tâm KNTT]%[1K7BM-3]
	\immini{Cho hình chóp $S.ABCD$ có đáy $ABCD$ là hình vuông, $SA$ vuông góc với mặt đáy. Đường thẳng $CD$ vuông góc đường thẳng nào sau đây?
		\choice
		{\True $SD$}
		{$SC$}
		{$SB$}
		{$AC$}
	}{\begin{tikzpicture}[scale=0.6, font=\footnotesize, line join=round, line cap=round, >=stealth]
			\def\bc{4} % cạnh BC
			\def\ba{2} % cạnh BA
			\def\h{4} % đường cao
			\def\gocB{30} % góc B của đáy
			\coordinate[label=below left:$B$] (B) at (0,0);
			\coordinate[label=above left:$A$] (A) at (\gocB:\ba);
			\coordinate[label=below:$C$] (C) at (\bc,0);
			\coordinate[label=right:$D$] (D) at ($(C)-(B)+(A)$);
			\coordinate[label=above:$S$] (S) at ($(A)+(90:\h)$);
			\draw (B)--(C)--(D)--(S)--cycle (S)--(C);
			\draw[dashed] (A)--(D) (S)--(A)--(B);
			\foreach \diem in {A,B,C,D,S}	\fill (\diem)circle(1.5pt);
			\pic[draw,angle radius=4]{right angle=S--A--D};
		\end{tikzpicture}
	}
	\loigiai{Ta có $\heva{&CD \perp AD\\&CD \perp SA\\&SA \cap AD = \{A\}\\&SA, AD \subset (SAD)} \Rightarrow CD \perp (SAD)\Rightarrow CD\perp SD$.
	}
\end{ex}

\begin{ex}%[Đỗ Chí Tâm KNTT]%[1K7BM-3]
	Cho hình lập phương $ABCD.A'B'C'D'$. Đường thẳng $A'C$ vuông góc với đường thẳng nào sau đây?
	\choice
	{$CD$}
	{$B'C'$}
	{\True $BD$}
	{$BD'$}
	\loigiai{
		\immini{Ta có $\heva{&BD\perp AC\\&BD\perp AA'}\Rightarrow BD\perp (ACC'A')\Rightarrow BD\perp A'C$.
		}{\begin{tikzpicture}[scale=0.5, font=\footnotesize, line join=round, line cap=round, >=stealth]
				\def\bc{5} % cạnh BC
				\def\ba{3} % cạnh BA
				\def\h{4} % đường cao
				\def\gocB{35} % góc B của đáy
				\coordinate[label=below left:$B$] (B) at (0,0);
				\coordinate[label=above left:$A$] (A) at (\gocB:\ba);
				\coordinate[label=below:$C$] (C) at (\bc,0);
				\coordinate[label=right:$D$] (D) at ($(C)-(B)+(A)$);
				\coordinate[label=above left:$A'$] (A') at ($(A)+(90:\h)$);
				\coordinate[label=left:$B'$] (B') at ($(B)-(A)+(A')$);
				\coordinate[label=below right:$C'$] (C') at ($(C)-(A)+(A')$);
				\coordinate[label=right:$D'$] (D') at ($(D)-(A)+(A')$);
				\draw (B')--(B)--(C)--(D)--(D')--(A')--(B')--(C')--(D') (C)--(C') (A')--(C');
				\draw[dashed] (A')--(A)--(D) (A)--(B) (A')--(C) (A)--(C) (B)--(D);
				\foreach \diem in {A,B,C,D,A',B',C',D'}	\fill (\diem)circle(1.5pt);
			\end{tikzpicture}
		}
	}
\end{ex}

\begin{ex}%[Đỗ Chí Tâm KNTT]%[1K7BM-3]
	Cho chóp $S.ABC$ có $SA\perp (ABC)$, $\triangle ABC$ vuông tại $B$, kẻ $AH\perp SB$. Chọn khẳng định \textbf{sai}?
	\choice
	{$BC\perp SB$}
	{$AH\perp SC$}
	{$AH\perp BC$}
	{\True $BC\perp SC$}
	\loigiai{
		\immini{
			Ta có $\heva{&BC\perp AB\\&BC\perp SA}\Rightarrow BC\perp (SAB)\Rightarrow BC\perp SB$.\\
			$\heva{&AH\perp SB\\&AH\perp BC}\Rightarrow AH\perp (SBC)\Rightarrow \heva{&AH\perp SB\\&AH\perp SC}$.     
		}{
			\begin{tikzpicture}[thick, scale=0.7]
				%\draw[gray!20] (-3,-2) grid (6,5);
				\path (0,0) coordinate (A)  
				(2,-2) coordinate (B)
				(5,0) coordinate (C)
				(0,4) coordinate (S);
				\draw[dashed] (A)--(C);
				\draw (B)--(C)--(S)--(A)--(B) (S)--(B) (A)--(H);
				\coordinate[label=right:$H$] (H) at ($(S)!0.6!(B)$); 
				\foreach \p/\q in {A/180, B/-90, C/-90, S/90}
				\fill[blue] (\p) circle(2pt) node[shift={(\q:3mm)}]{\bfseries \p};
				\pic[draw,angle radius=4]{right angle=A--B--C};
				\pic[draw,angle radius=4]{right angle=S--H--A};
			\end{tikzpicture}
		}		
	}
\end{ex}

\begin{ex}%[Đỗ Chí Tâm KNTT]%[1K7KM-3]
	\immini{Cho hình chóp $S.ABCD$ có đáy $ABCD$ là hình vuông, $SA$ vuông góc với mặt đáy. Tam giác nào dưới đây không phải là tam giác vuông?
		\choice
		{$SCD$}
		{$SBC$}
		{$SAC$}
		{\True $SBD$}
	}{\begin{tikzpicture}[scale=0.6, font=\footnotesize, line join=round, line cap=round, >=stealth]
			\def\bc{4} % cạnh BC
			\def\ba{2} % cạnh BA
			\def\h{4} % đường cao
			\def\gocB{30} % góc B của đáy
			\coordinate[label=below left:$B$] (B) at (0,0);
			\coordinate[label=above left:$A$] (A) at (\gocB:\ba);
			\coordinate[label=below:$C$] (C) at (\bc,0);
			\coordinate[label=right:$D$] (D) at ($(C)-(B)+(A)$);
			\coordinate[label=above:$S$] (S) at ($(A)+(90:\h)$);
			\draw[thick] (B)--(C)--(D)--(S)--cycle (S)--(C);
			\draw[dashed] (A)--(D) (S)--(A)--(B);
			\foreach \diem in {A,B,C,D,S}	\fill (\diem)circle(1.5pt);
			\pic[draw,angle radius=4]{right angle=S--A--D};
		\end{tikzpicture}
	}
	\loigiai{
		Dễ chứng minh được $\Delta SCD$ vuông tại $D$, $\Delta SBC$ vuông tại $B$, $\Delta SAC$ vuông tại $A$.\\
		$\Delta SBD$ cân tại $S$, không phải tam giác vuông.
	}
\end{ex}

\begin{ex}%[Đỗ Chí Tâm KNTT]%[1K7KM-3]
	Cho hình chóp $S.ABC$ có đáy $ABC$ là tam giác cân tại $B$, cạnh bên $SA$ vuông góc với đáy, $I$ là trung điểm $AC$. Khẳng định nào sau đây đúng?
	\choice
	{$SI\perp AC$}
	{$BC\perp SC$}
	{\True $BI\perp SC$}
	{$BC\perp SB$}
	\loigiai{
		\immini{
			Vì $SA\perp (ABC)$ nên $SA\perp BI$ và $\triangle ABC$ cân tại $B$ nên $BI\perp AC$.\\
			Vậy $BI\perp SA,\ BI\perp AC$ nên $BI\perp (SAC)\Rightarrow BI\perp SC$.}
		{	\begin{tikzpicture}[scale=0.8,font=\footnotesize, line join=round, line cap=round, >=stealth]
				\coordinate (A) at (0,0);
				\coordinate (C) at (2,-2);
				\coordinate (B) at (5,0);
				\coordinate (S) at (0,4);
				\coordinate (I) at ($(A)!0.5!(C)$);
				\coordinate (H) at (1.4,-0.2);
				\draw (A) node[left] {$A$} -- (C) node[below] {$C$} -- (B) node[right] {$B$} -- (S) node[above left] {$S$} -- (A) (S) -- (C) (I) -- (S);
				\node[above right] at (H) {$H$};
				\draw[dashed] (A) -- (B) -- (I) node[below left] {$I$};
		\end{tikzpicture}}
	}
\end{ex}

\begin{ex}%[Đỗ Chí Tâm KNTT]%[1K7KM-3]
	Cho hình chóp tam giác đều $S.ABC$, gọi $O$ là trọng tâm của tam giác $ABC$, $M$ là trung điểm $BC$, $H$ là hình chiếu của $O$ lên $SM$. Chọn khẳng định đúng?
	\choice
	{$OH\parallel SA$}
	{\True $OH\perp SB$}
	{$\Delta SAB$ vuông tại $S$}
	{$BC\perp SC$}
	\loigiai{
		\immini{Gọi $M$ là trung điểm $BC$ và $O$ là trọng tâm $\triangle ABC$.\\
			Khi đó $AM \perp BC$ và $SO \perp (ABC)$.\\
			Gọi $H$ là hình chiếu của $O$ trên $SM$\\
			Ta có $\heva{&BC \perp AM\\&BC \perp SO\\&SO, AM \subset (SAM)\\&SO \cap AM = \{O\}} \Rightarrow BC \perp (SAM) \Rightarrow BC \perp OH$.\\
			$\heva{&OH\perp SM\\&OH\perp BC}$\\
			Suy ra  $OH \perp (SBC)\Rightarrow OH\perp SB.$\\
		}{\begin{tikzpicture}[scale=0.8, font=\footnotesize, line join=round, line cap=round, >=stealth]
				\def\ac{4} % cạnh AC
				\def\ab{2} % cạnh AB
				\def\h{4} % chiều cao
				\def\gocA{50} % góc A của đáy
				\coordinate[label=left:$A$] (A) at (0,0);
				\coordinate[label=right:$C$] (C) at (\ac,0);
				\coordinate[label=below left:$B$] (B) at (-\gocA:\ab);
				\coordinate[label=below right:$M$] (M) at ($(B)!.5!(C)$);
				\coordinate[label=below:$O$] (O) at ($(A)!2/3!(M)$);
				\coordinate[label=above:$S$] (S) at ($(O)+(90:\h)$);
				\coordinate[label=right:$H$] (H) at ($(S)!0.7!(M)$);
				\draw[thick] (A)--(B)--(C)--(S)--cycle (S)--(B) (S)--(M);
				\draw[dashed] (M)--(A)--(C) (O)--(S) (O)--(H);
				\foreach \diem in {A,B,C,S,O,M,H}	\fill (\diem)circle(1.5pt);
				\pic[draw,angle radius=4]{right angle=A--M--B};
				\pic[draw,angle radius=4]{right angle=S--O--M};
				\pic[draw,angle radius=4]{right angle=O--H--M};
			\end{tikzpicture}
		}
	}
\end{ex}

\begin{ex}%[Đỗ Chí Tâm KNTT]%[1K7KM-3]
	Cho hình chóp tứ giác đều $S.ABCD$, gọi $M$ là trung điểm của $CD$, $O=AC\cap BD$, $H$ là hình chiếu của $O$ lên $SM$. Chọn khẳng định đúng?
	\choice
	{$OH\perp BD$}
	{\True $OH\perp AB$}
	{$BC\perp SB$}
	{$SB\perp SD$}
	\loigiai{
		\immini{
			Ta có $\heva{&CD\perp OM\\&CD\perp SO}\Rightarrow CD\perp OH$\\
			Mà $AB\parallel CD\Rightarrow AB\perp OH$.
		}
		{	\begin{tikzpicture}[line join=round, line cap=round, font=\footnotesize, scale=0.7]
				\tikzset{label style/.style={font=\footnotesize}}
				\def\d{5} \def\r{1.8} \def\h{5} \def\l{2.2}
				\coordinate[label={below}:$B$] (B) at (-3,-3);
				\coordinate[label={below right}:$C$] (C) at ($(B)+(\d,0)$);
				\coordinate[label={above right}:$A$] (A) at ($(B)+(\l,\r)$);
				\coordinate[label={above right}:$D$] (D) at ($(A)+(\d,0)$);
				\coordinate[label={below right}:$M$] (M) at ($(C)!0.6!(D)$);
				\coordinate[label={above right}:$H$] (H) at ($(S)!0.65!(M)$);
				\path(intersection of A--C and B--D)coordinate(O)node[below]{$O$};
				\coordinate[label={above}:$S$] (S) at ($(O)+(0,\h)$);
				\draw (S)--(B)--(C)node[midway,below]{$a$}--(D)--(S)--(C) (S)--(M);
				\draw[dashed] (O)--(S)--(A)--(B)node[shift={(30:.5)},scale=.5]{$60^\circ$}--(D)--(A)--(C) (O)--(M) (O)--(H);
				\pic[draw,angle radius=9]{angle=C--B--A};
				\pic[draw,angle radius=6]{right angle=A--O--B};
				\pic[draw,angle radius=6]{right angle=S--O--D};
				\pic[draw,angle radius=6]{right angle=S--H--O};
				\pic[draw,angle radius=6]{right angle=O--M--C};
				\foreach\p in{A,B,C,D,S,O,M,H}\draw[fill=black](\p)circle(1pt);
		\end{tikzpicture}}
	}
\end{ex}

\begin{ex}%[Đỗ Chí Tâm KNTT]%[1K7KM-3]
	Cho hình chóp $S.ABCD$ có tất cả các cạnh bằng nhau. Gọi $M$ là trung điểm của $SA$, $O$ là trung điểm của $AC$. Chọn khẳng định đúng?
	\choice
	{$BM\perp MD$}
	{\True $SA\perp SC$}
	{$(SBC)\perp (SCD)$}
	{$OM\perp SB$}
	\loigiai{
		\immini{
			Ta có $\heva{&SA \perp BM\\&SA \perp DM\\&BM, DM \subset (MBD)\\&BM \cap DM = \{M\}} \Rightarrow SA \perp (MBD)\Rightarrow SA\perp OM\,\,\, (1)$\\
			Mặt khác $OM$ là đường trung bình tam giác $SAC\Rightarrow OM\parallel SC.\quad (2)$\\
			Từ $(1)(2)\Rightarrow SA\perp SC$.
		}{\begin{tikzpicture}[scale=0.6, font=\footnotesize, line join=round, line cap=round, >=stealth]
				\def\bc{4} % cạnh BC
				\def\ba{2} % cạnh BA
				\def\h{4} % đường cao
				\def\gocB{30} % góc B của đáy
				\coordinate[label=below left:$B$] (B) at (0,0);
				\coordinate[label=above right:$A$] (A) at (\gocB:\ba);
				\coordinate[label=below:$C$] (C) at (\bc,0);
				\coordinate[label=right:$D$] (D) at ($(C)-(B)+(A)$);
				\coordinate[label=below:$O$] (O) at ($(A)!.5!(C)$);
				\coordinate[label=above:$S$] (S) at ($(O)+(90:\h)$);
				\coordinate[label=right:$M$] (M) at ($(S)!0.5!(A)$);
				\draw (B)--(C)--(D)--(S)--cycle (S)--(C);
				\draw[dashed] (C)--(A)--(D)--(B) (M)--(B)  (M)--(D) (S)--(A)--(B);
				\foreach \diem in {A,B,C,D,S,M}	\fill (\diem)circle(1.5pt);
				\pic[draw,angle radius=4]{right angle=A--M--B};
				\pic[draw,angle radius=4]{right angle=A--M--D};
			\end{tikzpicture}
		}
	}
\end{ex}
\Closesolutionfile{ans}
% \begin{indapan}{10}
% 	{ans/ans-1K7-2-3}
% \end{indapan}
\begin{dang}{Góc giữa hai đường thẳng (có \textit{d} vuông \textit{(P)})}
\end{dang}
\subsubsection{Ví dụ mẫu}
\begin{vd}[NB]%[Dung Phuong KNTT]%[1K7YM-4]
	Cho hình chóp $S.ABCD$ có đáy $ABCD$ là hình thoi, $SA$ vuông góc với mặt đáy $ABCD$. Hỏi góc giữa hai đường thẳng $SA$ và $BC$ là bao nhiêu độ?
	\dapso{$(SA,BC)=90^{\circ}$}
	\loigiai{
		\immini{
			Do $AD \parallel BC$ nên $(SA,BC)=(SA,AD)=90^{\circ}$.
		}{
			\begin{tikzpicture}[scale=1, font=\footnotesize, line join=round, line cap=round,>=stealth]
				\path
				(0,0) coordinate (A)
				(-1.3,-1.6) coordinate (B)
				(2.5,-1.6)coordinate (C)
				($(A)+(C)-(B)$) coordinate (D)
				($(A)+(0,3)$) coordinate (S)
				;
				\draw (S)--(B)--(C)--(D)--cycle (S)--(C);
				\draw[dashed] (S)--(A)--(D) (A)--(B);	
				\foreach \p/\q in {S/90,A/-90,B/-90,C/-90,D/0}			
				\fill[black] (\p) circle (1.0pt)node[shift={(\q:2.5mm)}]{$\p$};
		\end{tikzpicture}}}
\end{vd}
\begin{vd}[TH]%[Dung Phuong KNTT]%[1K7BM-4]
	Cho tứ diện đều $ABCD$ có $N$, $M$ lần lượt là trung điểm của các cạnh $AB$ và $CD$. Góc giữa $MN$ và $AB$ bằng
	\dapso{$(SA,BC)=90^{\circ}$}
	\loigiai{
		\immini{
			Gọi $G$ là hình chiếu của $D$ lên mặt phẳng $(ABC)$ do tứ diện $ABCD$ là tứ diện đều nên $G$ là trọng tâm của $\triangle ABC$.\\
			Ta có $\heva{&AB\perp NC\\&AB\perp DG}\Rightarrow AB\perp (DNC)\Rightarrow AB\perp MN$.
		}{\begin{tikzpicture}[scale=0.7,line join=round, line cap=round,>=stealth]
			\path
			(0,0) coordinate (A)
			(6,0) coordinate (C)
			(2,-2) coordinate (B)
			($(B)!0.5!(A)$) coordinate (N)	
			($(C)!0.6!(N)$) coordinate (G)
			($(G)+(0,5)$) coordinate (D)
			($(C)!0.5!(D)$) coordinate (M)
			;
			\draw (D)--(A)--(B)--(C)--cycle (D)--(B);
			\draw[dashed] (C)--(N)--(M) (A)--(C) (D)--(G);
			\foreach \p/\q in {A/180,B/-90,C/0,D/90,M/40,N/220,G/-90}
			\fill[black] (\p)node[shift={(\q:3mm)}]{$\p$} circle (1.0pt);	
			\draw pic[draw=black,angle radius=0.2cm] {right angle = B--N--C};
	\end{tikzpicture}}
	}
\end{vd}
\begin{vd}[TH]%[Dung Phuong KNTT]%[1K7BM-4]
	\immini{
		Cho hình lập phương $ABCD.A'B'C'D'$ có $I$, $J$ tương ứng là trung điểm của $BC$ và $BB'$. Góc giữa hai đường thẳng $AC$ và $IJ$ bằng
	}{
	\begin{tikzpicture}[scale=0.8,font=\footnotesize,line join=round,line cap=round,>=stealth]
			\path
			(0,0) coordinate (B)
			(1,0.8) coordinate (A)
			(4,0) coordinate (C)
			($(C)-(B)+(A)$) coordinate (D)	
			($(A)+(90:3.5)$) coordinate (A')
			($(B)-(A)+(A')$) coordinate (B')
			($(C)-(A)+(A')$) coordinate (C')
			($(D)-(A)+(A')$) coordinate (D')
			($(C)!0.5!(B)$) coordinate (I)
			($(B')!0.5!(B)$) coordinate (J)
			($(A)!0.5!(B)$) coordinate (K)	
			;
			\draw (B')--(B)--(C)--(D)--(D')--(A')--(B')--(C')--(D') (C)--(C') (I)--(J);
			\draw[dashed] (C)--(A)--(D) (A')--(A)--(B) (I)--(K)--(J);
			\foreach \p/\q in {A/160,B/-135,C/-45,D/0,A'/90,B'/180,C'/-20,D'/0,I/-90,J/180,K/-90}
			\fill[black] (\p)node[shift={(\q:3mm)}]{$\p$} circle (1.0pt);	
	\end{tikzpicture}}
	\dapso{$(SA,BC)=60^{\circ}$}
	\loigiai{
		Gọi $K$ là trung điểm của $AB$. Khi đó $IK\parallel AC$ nên
		\[(AC,IJ)=(IK,IJ)=\widehat{JIK}=60^{\circ}\,\,(\text{do tam giác}\,\, IJK\,\,\text{đều}).\]
	}
\end{vd}
\subsubsection{Bài tập rèn luyện}
% \centerline{\fcolorbox{red}{yellow!50}{\bf {BÀI TẬP TỰ LUẬN}}}
\begin{bt}%[Dung Phuong KNTT]%[1K7YM-4]
	Cho hình chóp $S.ABCD$ có đáy là hình vuông cạnh $a$, cạnh bên $SA=a$ và vuông góc với mặt đáy $(ABCD)$. Tính số đo góc giữa hai đường thẳng $SB$ và $CD$.
	\dapso{$(SB,CD)=45^\circ$}
	\loigiai{
		\immini{
			Ta có $CD\parallel AB$. \\
			Suy ra $(SB,CD)=(SB,AB)=\widehat{SBA}$.\\
			Do $\triangle SAB$ vuông tại $A$ có $SA=AB=a$ nên $ \triangle SAB $ vuông cân tại $A$.\\
			Suy ra $\widehat{SBA}=45^\circ$.\\
			Vậy $(SB,CD)=45^\circ$.
		}{
			\begin{tikzpicture}[scale=1, font=\footnotesize, line join=round, line cap=round,>=stealth]
				\path
				(0,0) coordinate (A)
				(-1.3,-1.6) coordinate (B)
				(2.5,-1.6)coordinate (C)
				($(A)+(C)-(B)$) coordinate (D)
				($(A)+(0,3)$) coordinate (S)
				;
				\draw (S)--(B)--(C)--(D)--cycle (S)--(C);
				\draw[dashed] (S)--(A)--(D) (A)--(B);	
				\foreach \p/\q in {S/90,A/-90,B/-90,C/-90,D/0}			
				\fill[black] (\p) circle (1.0pt)node[shift={(\q:2.5mm)}]{$\p$};
				\draw pic[draw=black,angle radius=0.2cm] {right angle = S--A--B}; 
				\draw pic[draw=black,angle radius=0.2cm] {right angle = S--A--D}; 
				\draw pic[draw,angle radius=6mm]{angle=A--B--S};				
			\end{tikzpicture}
		}
	}
\end{bt}
\begin{bt}%[Dung Phuong KNTT]%[1K7YM-4]
	Cho hình chóp $S.ABCD$ có tất cả các cạnh đều bằng $a$. Gọi $M$ và $N$ lần lượt là trung điểm của $SC$ và $BC$. Số đo của góc giữa hai đường thẳng $MN$ và $CD$ là
	\dapso{$(SB,CD)=60^\circ$}
	\loigiai{
		\immini{
			Ta có $MN\parallel SB$ và $CD\parallel AB$ nên
			$$(MN,CD)=(SB,AB)=\widehat{SBA}=60^\circ.$$
		}{
			\begin{tikzpicture}[line cap=round, line join=round, font=\footnotesize, >=stealth, scale=0.7]
				\tikzset{label style/.style={font=\footnotesize}}
				\path (0,0) coordinate (O)
				(3,0.5) coordinate (C)
				(1,-1) coordinate (D)
				($(C)!2!(O)$) coordinate (A)
				($(D)!2!(O)$) coordinate (B)
				($(O)+(0,3.5)$) coordinate (S)
				($(S)!0.5!(C)$) coordinate (M)
				($(B)!0.5!(C)$) coordinate (N);
				\draw (S)--(A)--(D)--(C)--(S)--(D);
				\draw[dashed] (A)--(B)--(C) (S)--(B) (M)--(N);
				\foreach \x/\y in {A/-150,B/-90,C/0,D/-60,S/90,M/60,N/-90} \fill[black] (\x) circle (1pt)+(\y:0.3) node{$\x$};
			\end{tikzpicture}
		}
	}
\end{bt}
\begin{bt}%[Dung Phuong KNTT]%[1K7BM-4]
	\immini{
		Cho hình hộp chữ nhật $ABCD.A'B'C'D'$ (\textit{tham khảo hình vẽ bên}) có $AD=a$, $BD=2a$. Góc giữa hai đường thẳng $A'C'$ và $BD$ là
	}{
		\begin{tikzpicture}[scale=0.7, font=\footnotesize, line join=round, line cap=round,>=stealth]
			\def\h{4} \def\r{5} \def\x{2.6} \def\y{1.3}
			\coordinate[label={below left}:$A$] (A) at (-3,-3);
			\coordinate[label={below}:$B$] (B) at ($(A)+(\x,\y)$);
			\coordinate[label={right}:$C$] (C) at ($(B)+(\r,0)$);
			\coordinate[label={below right}:$D$] (D) at ($(A)+(\r,0)$);
			\coordinate[label={above left}:$A'$] (A') at ($(A)+(0,\h)$);
			\coordinate[label={above left}:$B'$] (B') at ($(A')+(\x,\y)$);
			\coordinate[label={above right}:$C'$] (C') at ($(B')+(\r,0)$);
			\coordinate[label={above}:$D'$] (D') at ($(A')+(\r,0)$);
			\draw (A')--(B')--(C')--(D')--(A')--(A)--(D)--(C)--(C') (D)--(D') (A')--(C');
			\draw[dashed] (A)--(B)--(C) (B)--(B') (B)--(D);
			\foreach \p in {A,B,C,D,A',B',C',D'}
			\fill[black] (\p) circle (1.0pt);
		\end{tikzpicture}
	}
	\dapso{$(A'C',BD)=60^\circ$}
	\loigiai{
		\immini{
			Ta có $AC\parallel A'C'$ nên góc giữa hai đường thẳng $A'C'$ và $BD$ cũng chính là góc giữa hai đường thẳng $AC$ và $BD$.\\
			Gọi $O$ là giao điểm của $AC$ và $BD$. Khi đó tam giác $AOB$ đều do $3$ cạnh bằng $a$ nên $\widehat{AOB}=60^\circ$.\\
			Vậy $(A'C',BD)=(AC,BD)=\widehat{AOB}=60^\circ$.
		}{
			\begin{tikzpicture}[scale=1, font=\footnotesize, line join=round, line cap=round,>=stealth]
				\path
				(0,0) coordinate (A)
				(0,2) coordinate (B)
				($(A)!1!-60:(B)$) coordinate (O)
				($2*(O)-(A)$) coordinate (C)
				($2*(O)-(B)$) coordinate (D)
				;
				\draw (A)--(B)--(C)--(D)--cycle (A)--(C) (D)--(B);
				\foreach \p/\q in {A/180,B/135,C/45,D/-45,O/-90}
				\fill[black] (\p)node[shift={(\q:3mm)}]{$\p$} circle (1.0pt);
			\end{tikzpicture}
		}
	}
\end{bt}
\begin{bt}%[Dung Phuong KNTT]%[1K7BM-4]
	Cho tứ diện $ABCD$ có $AB$, $AC$, $AD$ đôi một vuông góc với nhau biết $AB=AC=AD=1$. Số đo góc giữa hai đường thẳng $AB$ và $CD$ bằng
	\dapso{$(AB,CD)=90^\circ$}
	\loigiai{
		\begin{center}
			\begin{tikzpicture}[scale=0.9,font=\footnotesize,line join=round,line cap=round,>=stealth]
				\path	
				(-2,-1.5) coordinate (C)
				(4,-1) coordinate (D)
				(0,0) coordinate (A)
				($(A)+(0,3)$) coordinate (B)
				;
				\draw (B)--(C)--(D)--cycle;
				\draw[dashed] (A)--(C) (D)--(A)--(B);
				\foreach \p/\q in {A/170,B/90,C/-90,D/0}
				\fill[black] (\p) circle (1.0pt) ($(\p)+(\q:3.5mm)$) node{$\p$};			
			\end{tikzpicture}
		\end{center}
		Ta có $\left\{\begin{aligned}
			&AB \perp AC\\
			&AB \perp AD\\
		\end{aligned}\right. \Rightarrow AB \perp (ACD) \Rightarrow AB \perp CD \Rightarrow (AB;CD)=90^{\circ}$.}
\end{bt}
\begin{bt}%[Dung Phuong KNTT]%[1K7BM-4]
	Cho hình chóp $S.ABC$ có $SA=SB=SC=\dfrac{a\sqrt{3}}{2}$, đáy là tam giác vuông tại $A$, cạnh $BC=a$. Tính cosin của góc giữa đường thẳng $SA$ và mặt phẳng $(ABC)$.
	\dapso{$\dfrac{1}{\sqrt{3}}$}
	\loigiai{
		\immini{
			Gọi $H$ là chân đường vuông góc hạ từ $S$ tới mặt phẳng $(ABC)$. Ta có
			\begin{align*}
				\heva{& SA^2=SH^2+HA^2 \\ & SB^2=SH^2+HB^2\\ & SC^2=SH^2+HC^2}\Rightarrow HA=HB=HC.
			\end{align*}
			$\Rightarrow H\text{ là tâm đường tròn ngoại tiếp }\triangle ABC$.\\
			Vì $\triangle ABC$ vuông tại $A$ nên $H$ là trung điểm của $BC$.\\
			Ta có $SH\perp (ABC)\Rightarrow AH$ là hình chiếu vuông góc của $SA$ lên $(ABC)\Rightarrow (SA;(ABC))=\widehat{SAH}$.
		}{
			\begin{tikzpicture}[scale=1, font=\footnotesize, line join=round, line cap=round,>=stealth]
				\path
				(0,0) coordinate (A)
				(2,-2) coordinate (B)
				(5,0) coordinate (C)	
				($(C)!0.5!(B)$) coordinate (H)
				($(H)+(0,5)$) coordinate (S)
				;
				\draw (S)--(A)--(B)--(C)--cycle (H)--(S)--(B);
				\draw[dashed] (H)--(A)--(C);	
				\foreach \p/\q in {S/90,A/-100,B/-90,C/0,H/-70}				
				\fill[black] (\p) circle (1.0pt) node[shift={(\q:3mm)}]{$\p$};			
			\end{tikzpicture}
		}\noindent
		Ta có $AH=\dfrac{1}{2}BC=\dfrac{a}{2}\Rightarrow \cos \widehat{SAH}=\dfrac{AH}{SA}=\dfrac{1}{\sqrt{3}}$.
	}
\end{bt}
% \centerline{\fcolorbox{red}{yellow!50}{\bf {CÂU HỎI TRẮC NGHIỆM}}}
% \Opensolutionfile{ans}[ans/ans-1K7-2-4]
% \begin{ex}%[Dung Phuong KNTT]%[1K7BM-4]
% 	\immini{
% 		Cho hình chóp $S.ABCD$, có đáy $ABCD$ là hình vuông cạnh $a$. Đường thẳng $SA$ vuông góc với mặt phẳng đáy $(ABCD)$ và $SA$ bằng $2a$. Tính tang của góc tạo bởi hai đường thẳng $SB$ và $CD$.
% 		\choice
% 		{$\sqrt{2}$}
% 		{\True $2$}
% 		{$\dfrac{\sqrt{2}}{3}$}
% 		{$3$}
% 	}{
% 		\begin{tikzpicture}[scale=0.7, font=\footnotesize, line join=round, line cap=round,>=stealth]
% 			\path
% 			(0,0) coordinate (A)
% 			(-1.3,-1.6) coordinate (B)
% 			(2.5,-1.6)coordinate (C)
% 			($(A)+(C)-(B)$) coordinate (D)
% 			($(A)+(0,3)$) coordinate (S)
% 			;
% 			\draw (S)--(B)--(C)--(D)--cycle (S)--(C);
% 			\draw[dashed] (S)--(A)--(D) (A)--(B);	
% 			\foreach \p/\q in {S/90,A/-90,B/-90,C/-90,D/0}			
% 			\fill[black] (\p) circle (1.0pt)node[shift={(\q:2.5mm)}]{$\p$};
% 			\draw pic[draw=black,angle radius=0.2cm] {right angle = S--A--B}; 
% 			\draw pic[draw=black,angle radius=0.2cm] {right angle = S--A--D}; 
% 		\end{tikzpicture}
% 	}
% 	\loigiai{
% 		Vì $CD\parallel AB$ nên góc giữa hai đường thẳng $CD$ và $SB$ bằng góc giữa $AB$ và $SB$.\\
% 		Tam giác $SAB$ vuông tại $A$, khi đó góc giữa $AB$ và $SB$ bằng góc $\widehat{SBA}$, suy ra
% 		\[\tan{\widehat{SBA}}=\dfrac{SA}{AB}=2.\]
% 	}
% \end{ex}
% \begin{ex}%[Dung Phuong KNTT]%[1K7BM-4]
% 	\immini{Cho hình chóp $S. ABCD$, có $ABCD$ là hình vuông cạnh $a$ và $SA \perp (ABCD)$ và $SA=a\sqrt{3}$. Tính số đo góc giữa hai đường thẳng $SB$ và $AB$.
% 		\choice
% 		{$30^{\circ}$}
% 		{$45^{\circ}$}
% 		{\True $60^{\circ}$}
% 		{$90^{\circ}$}
% 	}{
% 		\begin{tikzpicture}[font=\footnotesize, line join=round, line cap=round, >=stealth, scale=.8]
% 			\coordinate (A) at (0,0);
% 			\coordinate (D) at (4,0);
% 			\coordinate (C) at (2,-2);
% 			\coordinate (B) at ($(A)+(C)-(D)$);
% 			\coordinate (S) at ($(A)+(0,4)$);
% 			\draw (S)--(B)--(C)--(D)--cycle (S)--(C);
% 			\draw[dashed] (A)--(S) (B)--(A)--(D);
% 			\foreach \p/\g in {A/180, B/-90, C/-90, D/0, S/90} \draw[fill=black] (\p) circle(1.0pt) node [shift={(\g:.3)}] {$\p$};
% 		\end{tikzpicture}
% 	}
% 	\loigiai{
% 		Ta có $\widehat{(SB,AB)}=\widehat{SBA}$.\\
% 		Trong tam giác vuông $SAB$, ta có $\tan \widehat{SBA} = \dfrac{SA}{AB}=\dfrac{a\sqrt 3}{a}=\sqrt 3$. \\
% 		Suy ra, $\widehat{SBA}=60^{\circ}$ hay $\widehat{(SB,AB)}=60^ {\circ}$.
% 	}
% \end{ex}
% \begin{ex}%[Dung Phuong KNTT]%[1K7BM-4]
% 	Cho hình chóp $S.ABCD$ có $ABCD$ là hình chữ nhật. Biết $AB=a\sqrt{2}$, $AD=2a$, $SA\perp (ABCD)$ và $SA=a\sqrt{2}$. Góc giữa hai đường thẳng $SC$ và $AB$ bằng
% 	\choice
% 	{$90^\circ$}
% 	{$30^\circ$}
% 	{\True $60^\circ$}
% 	{$45^\circ$}
% 	\loigiai{
% 		\immini{
% 			Góc giữa hai đường thẳng $SC$ và $AB$ bằng góc giữa hai đường thẳng $SC$ và $CD$.
% 			Ta có
% 			\begin{itemize}
% 				\item $AC=\sqrt{AB^2+BC^2}=a\sqrt{6}$.
% 				\item $SC=\sqrt{SA^2+AC^2}=2a\sqrt{2}$.
% 				\item $SD=\sqrt{SA^2+AD^2}=a\sqrt{6}$.
% 			\end{itemize}
% 			Khi đó
% 			$\cos\widehat{SCD}=\dfrac{SC^2+CD^2-SD^2}{2\cdot SC \cdot CD}=\dfrac{8a^2+2a^2-6a^2}{2\cdot 2a\sqrt{2}\cdot a\sqrt{2}}=\dfrac{1}{2}$.\\
% 			Vậy góc giữa $SC$ và $AB$ bằng $60^\circ$.
% 		}{
% 			\begin{tikzpicture}[font=\footnotesize, line join=round, line cap=round, >=stealth, scale=.8]
% 				\coordinate (A) at (0,0);
% 				\coordinate (D) at (4,0);
% 				\coordinate (C) at (2,-2);
% 				\coordinate (B) at ($(A)+(C)-(D)$);
% 				\coordinate (S) at ($(A)+(0,4)$);
% 				\draw (S)--(B)--(C)--(D)--cycle (S)--(C);
% 				\draw[dashed] (A)--(S) (B)--(A)--(D);
% 				\foreach \p/\g in {A/180, B/-90, C/-90, D/0, S/90} \draw[fill=black] (\p) circle(1.0pt) node [shift={(\g:.3)}] {$\p$};
% 				\draw pic[draw,angle radius=6mm]{angle=D--C--S}; 
% 			\end{tikzpicture}
% 		}
% 		\noindent
% 		Cách khác: Có thể chứng minh $\triangle SCD$ vuông tại $D$. Khi đó $\cos \widehat{SCD}=\dfrac{CD}{SC}=\dfrac{a\sqrt{2}}{2a\sqrt{2}}=\dfrac{1}{2}$.
% 	}
% \end{ex}
% \begin{ex}%[Dung Phuong KNTT]%[1K7BM-4]
% 	Cho khối lăng trụ $ABC.A'B'C'$ có đáy $ABC$ là tam giác đều cạnh $a$, hình chiếu vuông góc của $A'$ lên mặt phẳng $(ABC)$ trùng với trung điểm cạnh $AC$, đường thẳng $A'B$ tạo với mặt phẳng $(ABC)$ một góc $30^\circ$. Gọi $\alpha$ là góc giữa hai đường thẳng $AB$ và $CC'$. Tính $\cos \alpha$.
% 	\choice
% 	{\True $\cos \alpha=\dfrac{\sqrt{2}}{4}$}
% 	{$\cos \alpha=\dfrac{\sqrt{2}}{3}$}
% 	{$\cos \alpha =\sqrt{2}$}
% 	{$\cos \alpha=\dfrac{\sqrt{5}}{2}$}
% 	\loigiai{
% 		\immini{
% 			Ta có $A'H\perp (ABC)$ nên $\widehat{A'BH}=(A'B, (ABC))=30^\circ$. Suy ra
% 			\begin{align*}
% 				A'H&=BH\cdot \tan 30^\circ= \dfrac{a}{2},\\
% 				A'B&=\dfrac{BH}{\cos 30^{\circ}}=a,\\
% 				AA'&=\sqrt{AH^2+A'H^2}=\dfrac{a\sqrt{2}}{2}.
% 			\end{align*}
% 		}{
% 			\begin{tikzpicture}[scale=1, font=\footnotesize, line join=round, line cap=round,>=stealth]
% 				\path
% 				(0,0)coordinate (A)
% 				(1.5,-1)coordinate (C)
% 				(3.5,0)coordinate (B)
% 				($(A)!0.5!(C)$) coordinate (H)	
% 				($(H)+(0,3.5)$) coordinate (A')
% 				($(A')+(B)-(A)$)coordinate (B')
% 				($(A')+(C)-(A)$)coordinate (C')		
% 				;
% 				\draw (A)--(C)--(B)--(B')--(C')--(A')--cycle (H)--(A')--(B') (C)--(C');
% 				\draw[dashed] (A)--(B)--(A') (B)--(H);	
% 				\foreach \p/\q in {A/180,C/-90,B/0,A'/180,C'/75,B'/0}
% 				\fill[black] (\p) circle (1.0pt)node[shift={(\q:3mm)}]{$\p$};
% 				\draw pic[draw=black,angle radius=0.2cm] {right angle = A'--H--B}; 
% 			\end{tikzpicture}
% 		}
% 		Do đó $\cos \alpha=\cos(AB,AA')=\dfrac{A'A^2+AB^2-A'B^2}{2A'A\cdot AB}=\dfrac{\sqrt{2}}{4}$.
% 	}
% \end{ex}
% \begin{ex}%[Dung Phuong KNTT]%[1K7KM-4]
% 	\immini
% 	{Cho hình chóp $S.ABCD$ có đáy là hình vuông cạnh $a$, $SA$ vuông góc với mặt phẳng đáy và $SA=a\sqrt{3}$ (minh họa như hình bên). Góc giữa đường thẳng $SB$ và $DC$ bằng
% 		\choice
% 		{$45^\circ$}
% 		{$30^\circ$}
% 		{\True $60^\circ$}
% 		{$90^\circ$}}
% 	{
% 		\begin{tikzpicture}[line cap=round,line join=round,font=\footnotesize,>=stealth,scale=0.75]
% 			\fill (0,0) coordinate [label=above left:$A$] (A) circle(1pt)
% 			(0:3) coordinate [label=below right:$B$] (B) circle(1pt)
% 			(90:2) coordinate [label=above right:$S$] (S) circle(1pt)
% 			(-40:2) coordinate [label=below right:$C$] (C) circle(1pt)
% 			($(A)+(C)-(B)$) coordinate [label=below:$D$] (D) circle(1pt);
% 			\draw (S)--(B)--(C) (D)--(S)--(C)--(D);
% 			\draw[dashed] (D)--(A)--(S) (A)--(B);
% 	\end{tikzpicture}}
% 	\loigiai
% 	{
% 		Ta có $AB$ song song $DC$ nên góc hai giữa đường thẳng $SB$, $DC$ sẽ bằng góc giữa hai đường thẳng $SB$, $AB$.\\
% 		Mà $(SB,AB)=\widehat{SBA}$.\\
% 		Xét tam giác $SAB$ vuông tại $A$, $\tan\widehat{SBA}=\dfrac{SA}{AB}=\dfrac{a\sqrt{3}}{a}=\sqrt{3}$.\\
% 		Suy ra $\widehat{SBA}=60^\circ$.
% 	}
% \end{ex}

% \begin{ex}%[Dung Phuong KNTT]%[1K7YM-4]
% 	Cho hình lập phương $ABCD.A'B'C'D'$. Tính góc giữa hai đường thẳng $A'C'$ và $AD$ bằng
% 	\choice
% 	{$30^\circ$}
% 	{$90^\circ$}
% 	{$60^\circ$}
% 	{\True $45^\circ$}
% 	\loigiai{
% 		\immini{
% 			Vì $ABCD$ là hình vuông nên
% 			\[\widehat{\left(A'C',AD\right)}=\widehat{\left(AC,AD\right)}=\widehat{CAD}=45^\circ.\]
% 		}
% 		{
% 			\begin{tikzpicture}[scale=0.7, font=\footnotesize, line join=round, line cap=round, >=stealth]
% 				\def\a{4}
% 				\def\h{4}
% 				\path 
% 				(0,0) coordinate (A)
% 				(-1.5,-1.5) coordinate (B)
% 				(B)+(\a,0) coordinate (C)
% 				(A)+(\a,0) coordinate (D)
% 				(A)+(0,\h) coordinate (A')
% 				(B)+(0,\h) coordinate (B')
% 				(C)+(0,\h) coordinate (C')
% 				(D)+(0,\h) coordinate (D')
% 				;
% 				\draw 
% 				(A')--(B')--(C')--(D')--cycle
% 				(B)--(C)--(D)
% 				(B')--(B) (C')--(C) (D')--(D);
% 				\draw[dashed]
% 				(A')--(A) 
% 				(B)--(A)--(D)
% 				(A)--(C)
% 				;
% 				\foreach \p/\g in {A/40,B/-90,C/-90,D/0,A'/90,B'/90,C'/90,D'/90}
% 				\fill[black] (\p) circle (1.2pt) 
% 				($(\p)+(\g:3mm)$) node{$\p$};
% 				%	\tkzMarkRightAngles(S,B,D S,D,C)
				
% 		\end{tikzpicture}}	
% 	}
% \end{ex}
% \begin{ex}%[Dung Phuong KNTT]%[1K7BM-4]
% 	\immini
% 	{Cho hình lập phương $ABCD.A'B'C'D'$. Tính góc giữa $AC'$ và $BD$.
% 		\choice
% 		{\True $90^\circ$}
% 		{$45^\circ$}
% 		{$60^\circ$}
% 		{$120^\circ$}}
% 	{
% 		\begin{tikzpicture}[line cap=round,line join=round,font=\footnotesize,>=stealth,scale=0.6]
% 			\fill (0,0) coordinate [label=above left:$A$] (A) circle(1pt)
% 			(4,0) coordinate [label=below right:$B$] (B) circle(1pt)
% 			(-120:2) coordinate [label=below:$D$] (D) circle(1pt)
% 			($(B)+(D)$) coordinate [label=below:$C$] (C) circle(1pt)
% 			(0,3) coordinate [label=above:$A'$] (A') circle(1pt)
% 			($(A')+(B)$) coordinate [label=above:$B'$] (B') circle(1pt)
% 			($(A')+(D)$) coordinate [label=left:$D'$] (D') circle(1pt)
% 			($(A')+(C)$) coordinate [label=right:$C'$] (C') circle(1pt);
% 			\draw (D')--(D)--(C)--(B)--(B')--(A')--(D')--(C')--(B') (C)--(C');
% 			\draw[dashed] (A')--(A)--(D) (A)--(B);
% 	\end{tikzpicture}}
% 	\loigiai
% 	{
% 		\immini
% 		{Ta có $\heva{&BD\perp AC\text{ (do $ABCD$ là hình vuông)}\\&BD\perp CC'}\Rightarrow BD\perp AC'$.\\
% 			Do đó góc giữa $AC'$ và $BD$ bằng $90^\circ$.}
% 		{
% 			\begin{tikzpicture}[line cap=round,line join=round,font=\footnotesize,>=stealth,scale=0.6]
% 				\fill (0,0) coordinate [label=above left:$A$] (A) circle(1pt)
% 				(4,0) coordinate [label=below right:$B$] (B) circle(1pt)
% 				(-120:2) coordinate [label=below:$D$] (D) circle(1pt)
% 				($(B)+(D)$) coordinate [label=below:$C$] (C) circle(1pt)
% 				(0,3) coordinate [label=above:$A'$] (A') circle(1pt)
% 				($(A')+(B)$) coordinate [label=above:$B'$] (B') circle(1pt)
% 				($(A')+(D)$) coordinate [label=left:$D'$] (D') circle(1pt)
% 				($(A')+(C)$) coordinate [label=right:$C'$] (C') circle(1pt);
% 				\draw (D')--(D)--(C)--(B)--(B')--(A')--(D')--(C')--(B') (C)--(C');
% 				\draw[dashed] (A')--(A)--(D) (C)--(A)--(B)--(D) (A)--(C');
% 		\end{tikzpicture}}
% 	}
% \end{ex}
% \begin{ex}%[Dung Phuong KNTT]%[1K7BM-4]
% 	Cho hình chóp $S.ABC$ có tam giác $ABC$ vuông tại $A$, $SA\perp (ABC)$ và $SA=AB=AC=a$. Góc giữa hai đường thẳng $BC$ và $SC$ bằng
% 	\choice
% 	{\True $60^\circ$}
% 	{$30^\circ$}
% 	{$45^\circ$}
% 	{$90^\circ$}
% 	\loigiai{
% 		\immini
% 		{
% 			Do $SA\perp (ABC)$ và tam giác $ABC$ vuông tại $A$ nên $3$ đường thẳng $AB$, $AC$, $SA$ đôi một vuông góc với nhau. \\
% 			Khi đó do $SA=AB=AC=a$ nên $SC=SB=BC=a\sqrt{2}$.\\
% 			Vậy tam giác $SBC$ đều nên $(BC,SC) = \widehat{SCB} = 60^\circ$.
% 		}
% 		{
% 			\begin{tikzpicture}[scale=0.8,font=\footnotesize, line join=round, line cap=round, >=stealth, transform shape]
% 				\foreach \x/\y/\z/\g in
% 				{
% 					0/3/S/90, 0/0/A/180, 3/0/C/0,-1.5/-1.5/B/-90
% 				}
% 				\draw[fill=black] (\x,\y) circle(1pt) coordinate (\z) ($(\z)+(\g:3mm)$) node{$\z$};
% 				\draw (S)--(B)--(C)--(S);
% 				\draw[dashed] (S)--(A)--(C) (A)--(B);
% 				%Đánh dấu góc
% 				\begin{scope}
% 					\clip (S)--(C)--(B);
% 					\draw (C) circle(5mm);
% 				\end{scope}
% 			\end{tikzpicture}
% 		}
% 	}
% \end{ex}	
% \begin{ex}%[Dung Phuong KNTT]%[1K7BM-4]
% 	\immini{
% 		Cho hình lăng trụ đứng $ABC.A'B'C'$ có tất cả các cạnh bằng nhau
% 		(tham khảo hình bên). Góc giữa hai đường thẳng $AA'$ và $BC'$ bằng
% 		\choice
% 		{$30^\circ$}
% 		{$90^\circ$}
% 		{\True $45^\circ$}
% 		{$60^\circ$.}
% 	}{
% 		\begin{tikzpicture}[font=\footnotesize, line join=round, line cap=round, >=stealth]
% 			\def\r{2}
% 			\def\ga{140}
% 			\def\h{2.5}
% 			\path (\ga: {\r} and {\r/4})coordinate(A)+(90:\h)coordinate(A')
% 			(\ga+120:{\r} and {\r/4})coordinate(B)+(90:\h)coordinate(B')
% 			(\ga+240:{\r} and {\r/4})coordinate(C)+(90:\h)coordinate(C');
% 			\draw[dashed](C)--(A);
% 			\draw(A')--(A)--(B)
% 			(B')--(B)--(C)--(C')--(B')--(A')--(C');
% 			\foreach \p/\g in{A'/90,B'/90,C'/90,A/140,B/-120,C/20}
% 			\draw[fill=black](\p)circle(.03)node[shift={(\g:.2)},scale=.8]{$\p$};
% 	\end{tikzpicture}}
% 	\loigiai{
% 		\immini{Do $AA'\parallel BB'$ nên góc giữa hai đường thẳng $AA'$ và $BC'$ bằng góc giữa hai đường thẳng $BB'$ và $B'C$ bằng $\widehat{CB'B}$.\\
% 			Mà tam giác $\triangle B'BC$ vuông tại $B$ có $BC=BB'$ nên $\widehat{CB'B}=45^\circ$.
% 		}{
% 			\begin{tikzpicture}[font=\footnotesize, line join=round, line cap=round, >=stealth]
% 				\def\r{2}
% 				\def\ga{140}
% 				\def\h{2.5}
% 				\path (\ga: {\r} and {\r/4})coordinate(A)+(90:\h)coordinate(A')
% 				(\ga+120:{\r} and {\r/4})coordinate(B)+(90:\h)coordinate(B')
% 				(\ga+240:{\r} and {\r/4})coordinate(C)+(90:\h)coordinate(C');
% 				\draw[dashed](C)--(A);
% 				\draw(A')--(A)--(B)
% 				(C)--(B')--(B)--(C)--(C')--(B')--(A')--(C');
% 				\foreach \p/\g in{A'/90,B'/90,C'/90,A/140,B/-120,C/20}
% 				\draw[fill=black](\p)circle(.03)node[shift={(\g:.2)},scale=.8]{$\p$};
% 		\end{tikzpicture}}
% 	}
% \end{ex}
% \begin{ex}%[Dung Phuong KNTT]%[1K7BM-4]
% 	Cho một hình thoi $ABCD$ cạnh $a$ và một điểm $S$ nằm ngoài mặt phẳng chứa hình thoi sao cho $SA=a$ và $SA$ vuông góc với $(ABCD)$. Tính góc giữa $SD$ và $BC$.
% 	\choice
% 	{$ 60^\circ$}
% 	{$90^\circ $}
% 	{\True $45^\circ $}
% 	{$ 30^\circ$}
% 	\loigiai{
% 		\immini{Ta có $AD\parallel BC \Rightarrow \left(SD,BC\right)=\left(SD,AD\right)=\widehat{ADS}=45^\circ$.
% 		}{
% 			\begin{tikzpicture}[scale=0.6,>=stealth, font=\footnotesize, line join=round, line cap=round]
% 				\path
% 				(0,0) coordinate (A) 
% 				(-2,-2) coordinate (B)
% 				(2,-2) coordinate (C)
% 				(4,0) coordinate (D)
% 				(0,4) coordinate (S)
% 				;
% 				\draw 
% 				(S)--(B)--(C)--(D)--cycle
% 				(S)--(C)
% 				pic[draw,angle radius=1.5mm]{right angle=S--A--B}
% 				pic[draw,angle radius=1.5mm]{right angle=D--A--S}
% 				;
% 				\draw [dashed] (S)--(A)--(D) (A)--(B) (A)--(C); 
% 				\foreach \x/\g in
% 				{A/150,B/180,C/0,D/0,S/90}
% 				\fill[black](\x) circle (1pt)
% 				($(\x)+(\g:3mm)$) node{$\x$};
% 		\end{tikzpicture}}
% 	}
% \end{ex}
% \Closesolutionfile{ans}
% \begin{indapan}{10}
% 	{ans/ans-1K7-2-4}
% \end{indapan}

\begin{dang}{Dựng mặt phẳng, tìm thiết diện}
\end{dang}
\subsubsection{Ví dụ mẫu}
\begin{vd}[TH]%[1K7BM-5]
	Cho hình lăng trụ $ABC.A'B'C'$ có đáy $ABC$ là tam giác vuông cân tại $A$ với $BC=a\sqrt{2}$; $AA'=a$ và vuông góc với đáy. Mặt phẳng $\left(\alpha \right)$ qua $M$ là trung điểm của $BC$ và vuông góc với $AB'$. Thiết diện tạo bởi $\left(\alpha \right)$ với hình lăng trụ $ABC.A'B'C'$ là
	\dapso{Hình thang vuông}
	\loigiai{
		\immini{
			Gọi $N$ là trung điểm $AB\Rightarrow MN\perp AB$. Ta có $$\heva{& MN\perp AB \\ & MN\perp AA' \\}\Rightarrow MN\perp \left(ABB'A'\right)\Rightarrow MN\perp AB'\Rightarrow MN\subset \left(\alpha \right).$$
			Từ giả thiết suy ra $AB=a=AA'\Rightarrow ABB'A'$ là hình vuông, suy ra $BA'\perp AB'$.\\
			Trong mp $\left(ABB'A'\right)$ kẻ $NQ\parallel BA'$ với $Q\in AA'$. \\
			Trong mp $\left(ACC'A'\right)$ kẻ $QR\parallel AC$ với $R\in CC'$. \\
			Vậy thiết diện là hình thang $MNQR$ vuông (do $MN$ và $QR$ cùng song song với $AC$ và $MN\perp NQ$).
		}{
			\begin{tikzpicture}[scale=0.7, font=\footnotesize, line join=round, line cap=round,>=stealth]
				\path
				(0,0) coordinate (B)
				(6,0) coordinate (C)
				(2.5,-2) coordinate (A)
				;	
				\coordinate (B') at ($(B)+(0,5)$);
				\coordinate (C') at ($(C)+(0,5)$);
				\coordinate (A') at ($(A)+(0,5)$);
				\coordinate (N) at ($(A)!.5!(B)$);
				\coordinate (Q) at ($(A)!.5!(A')$);
				\coordinate (M) at ($(C)!.5!(B)$);
				\coordinate (R) at ($(C)!.5!(C')$);
				\draw (B')--(B)--(A)--(C)--(C')--cycle (B)--(A')--(A) (B')--(A')--(C') (B')--(A)
				(N)--(Q)--(R);
				\draw[dashed] (C)--(B) (N)--(M)--(R);
				\foreach \p/\q in {A/-90,B/180,C/0,A'/90,B'/180,C'/0,M/-40,N/240,Q/135,R/0}
				\fill[black] (\p) circle (1.0pt) ($(\p)+(\q:4mm)$) node{$\p$};
			\end{tikzpicture}
		}
	}
\end{vd}
\begin{vd}[TH]%[1K7BM-5]
	Cho hình chóp $S.ABCD$ có đáy $ABCD$ là hình chữ nhật với $AB=a$, $BC=2a$. Tam giác $SAB$ đều và nằm trong mặt phẳng vuông góc với đáy. Mặt phẳng $\left(\alpha \right)$ đi qua $S$ vuông góc với $AB$. Tính diện tích $S$ của thiết diện tạo bởi $\left(\alpha \right)$ với hình chóp đã cho.
	\dapso{$S=\dfrac{a^2\sqrt{3}}{2}$}
	\loigiai{
		\immini{
			Gọi $H$ là trung điểm $AB\Rightarrow SH\perp AB$. \\
			Suy ra $SH\subset \left(\alpha \right)$ và $SH\perp \left(ABCD\right)$ (do $\left(SAB\right)\perp \left(ABCD\right)$ theo giao tuyến $AB$).\\
			Kẻ $HM\perp AB \left(M\in CD\right)\Rightarrow HM\subset \left(\alpha \right)$.\\
			Do đó thiết diện là tam giác $SHM$ vuông tại $H$.\\
			Ta có $SH=\dfrac{a\sqrt{3}}{2}$, $HM=BC=2a$.\\
			Vậy $S_{\triangle SHM}=\dfrac{1}{2}\cdot \dfrac{a\sqrt{3}}{2}\cdot 2a=\dfrac{a^2\sqrt{3}}{2}$.
		}{
			\begin{tikzpicture}[scale=1, font=\footnotesize, line join=round, line cap=round,>=stealth]
				\path
				(0,0) coordinate (A)
				(-1.4,-1.6) coordinate (B)
				(2.5,-1.6) coordinate (C)
				($(A)+(C)-(B)$) coordinate (D)
				($(A)!0.5!(B)$) coordinate (H)
				($(H)+(0,3.5)$) coordinate (S)
				($(C)!0.5!(D)$) coordinate (M)
				;
				\draw (S)--(B)--(C)--(D)--cycle (M)--(S)--(C);
				\draw[dashed] (M)--(H)--(S)--(A)--(D) (A)--(B);	
				\foreach \p/\q in {S/90,A/-100,B/-90,C/-90,D/0,H/170,M/-20}				
				\fill[black] (\p) circle (1.0pt) node[shift={(\q:3mm)}]{$\p$};			
			\end{tikzpicture}
		}
	}
\end{vd}
\begin{vd}[TH]%[1K7BM-5]
	\immini{
		Cho khối lập phương $(H)$ kích thước $3\times 3\times 3$ được tạo thành từ $27$ khối lập phương đơn vị (xem hình vẽ). Mặt phẳng $(P)$ vuông góc với một đường chéo của $(H)$ tại trung điểm của nó. Hỏi $(P)$ cắt qua bao nhiêu khối lập phương đơn vị?
		\dapso{$19$}
	}
	{
		\begin{tikzpicture}[scale=0.95, font=\footnotesize, line join=round, line cap=round,>=stealth]
			\path
			(0,0) coordinate (O)
			(0,1) coordinate (A)
			(-1,0) coordinate (B)
			(55:0.7) coordinate (D)	
			($(O)!2!(B)$) coordinate (B1)
			($(O)!3!(B)$) coordinate (B2)
			($(O)!2!(D)$) coordinate (D1)
			($(O)!3!(D)$) coordinate (D2)
			($(O)!2!(A)$) coordinate (A1)
			($(O)!3!(A)$) coordinate (A2)
			($(A)+(D2)-(O)$) coordinate (C)
			($(A1)+(D2)-(O)$) coordinate (C1)
			($(A2)+(D2)-(O)$) coordinate (C2)	
			($(B)+(0,3)$) coordinate (E)
			($(B1)+(0,3)$) coordinate (E1)
			($(B2)+(0,3)$) coordinate (E2)
			($(A)+(B2)-(O)$) coordinate (F)
			($(A1)+(B2)-(O)$) coordinate (F1)
			($(A2)+(B2)-(O)$) coordinate (F2)	
			($(D)+(0,3)$) coordinate (H)
			($(D1)+(0,3)$) coordinate (H1)
			($(D2)+(0,3)$) coordinate (H2)	
			($(E)+(C2)-(A2)$) coordinate (G)
			($(E1)+(C2)-(A2)$) coordinate (G1)
			($(E2)+(C2)-(A2)$) coordinate (G2)	
			($(H)-(3,0)$) coordinate (K)
			($(H1)-(3,0)$) coordinate (K1)
			;
			\draw (E2)--(B2)--(O)--(A2)--cycle (O)--(D2)--(C2)--(A2) (E2)--(G2)--(C2)
			(B)--(E)--(G)
			(B1)--(E1)--(G1)
			(F)--(A)--(C) (F1)--(A1)--(C1) 
			(D)--(H)--(K) (D1)--(H1)--(K1) 
			;	
		\end{tikzpicture}
	}
	\loigiai{
		\immini{
			Đặt tên các đỉnh của khối lập phương $(H)$ như hình vẽ bên.\\
			Gọi $M,N,P,Q,R,S$ lần lượt là trung điểm của các cạnh $BC$, $CD$, $DD'$, $D'A'$, $A'B'$, $B'B$.\\
			Khi đó mặt phẳng $(P)$ vuông góc với $AC'$ tại trung điểm của $AC'$ đi qua $M,N,P,Q,R,S$.\\
			Xét mỗi mặt của $(H)$ gồm $9$ khối lập phương đơn vị. Khi đó $(P)$ cắt qua $4$ khối lập phương mỗi mặt.\\
			Ngoài ra, $(P)$ cắt khối lập phương đơn vị ở trung tâm (chứa trung điểm của $AC'$).
		}
		{
			\begin{tikzpicture}[scale=1,font=\footnotesize,line join=round,line cap=round,>=stealth]
				\path
				(0,0) coordinate (A)
				(0:4) coordinate (B)
				(50:2) coordinate (D)
				($(B)+(D)-(A)$) coordinate (C)
				($(A)+(-90:3.5)$) coordinate (A')
				($(B)-(A)+(A')$) coordinate (B')
				($(C)-(A)+(A')$) coordinate (C')
				($(D)-(A)+(A')$) coordinate (D')
				($(B)!0.5!(C)$) coordinate (M)
				($(C)!0.5!(D)$) coordinate (N)
				($(D)!0.5!(D')$) coordinate (P)
				($(D')!0.5!(A')$) coordinate (Q)
				($(A')!0.5!(B')$) coordinate (R)
				($(B)!0.5!(B')$) coordinate (S)
				;
				\draw (A)--(A')--(B')--(C')--(C)--(D)--cycle (B')--(B)--(A) (C)--(B)
				(R)--(S)--(M)--(N);
				\draw[dashed] (A')--(D')--(D) (D')--(C')--(A) (R)--(P)--(N)--cycle (M)--(Q) (P)--(S)
				(R)--(Q)--(P);
				\foreach \p/\q in {A/160,B/-35,C/90,D/90,A'/-90,B'/-90,C'/-10,D'/160,M/-30,N/90,P/180,Q/170,R/-90,S/-6}
				\fill[black] (\p)node[shift={(\q:3mm)}]{$\p$} circle (1.0pt);	
			\end{tikzpicture}
		}
		\noindent Khi đếm như vậy, các khối lập phương đơn vị chứa các điểm $M,N,P,Q,R,S$ được tính $2$ lần bị cắt qua.\\
		Vậy $(P)$ cắt qua số khối lập phương đơn vị là $6\cdot 4+1-6=19$.
	}
\end{vd}
\begin{vd}[VD]%[1K7KM-5]
	Cho hình chóp $S.ABC$ có đáy $ABC$ là tam giác đều cạnh $a$, $SA=a$ và vuông góc với đáy. Mặt phẳng $\left(\alpha \right)$ qua trung điểm $E$ của $SC$ và vuông góc với $AB$. Tính diện tích $S$ của thiết diện tạo bởi $\left(\alpha \right)$ với hình chóp đã cho.
	\dapso{$S=\dfrac{5a^2\sqrt{3}}{32}$}
	\loigiai{
		\immini{
			Gọi $F$ là trung điểm $AC$, suy ra $EF\parallel SA$.\\
			Do $SA\perp \left(ABC\right)\Rightarrow SA\perp AB$ nên $EF\perp AB$. \hfill $(1)$\\
			Gọi $J, G$ lần lượt là trung điểm $AC, AJ$.\\
			Suy ra $CJ\perp AB$ và $FG\parallel CJ$ nên $FG\perp AB$.\hfill $(2)$\\
			Trong $\triangle SAB$ kẻ $GH\parallel SA\left(H\in SB\right)$, suy ra $GH\perp AB$.\hfill $(3)$\\
			Từ $(1)$, $(2)$ và $(3)$, suy ra thiết diện cần tìm là hình thang vuông $EFGH$. Do đó $S_{EFGH}=\dfrac{1}{2}\left(EF+GH\right)\cdot FG$. Ta có\\
			$EF=\dfrac{1}{2}SA=\dfrac{a}{2}$; $FG=\dfrac{1}{2}CJ=\dfrac{a\sqrt{3}}{4}$;\\ $\dfrac{GH}{SA}=\dfrac{BG}{BA}\Rightarrow GH=BG=\dfrac{3a}{4}$.\\
			Vậy $S_{EFGH}=\dfrac{1}{2}\left(\dfrac{a}{2}+\dfrac{3a}{4}\right)\cdot \dfrac{a\sqrt{3}}{4}=\dfrac{5a^2\sqrt{3}}{32}$.
		}{
			\begin{tikzpicture}[scale=0.7, font=\footnotesize, line join=round, line cap=round,>=stealth]
				\path
				(0,0) coordinate (A)
				(6,0) coordinate (B)
				(2.5,-3) coordinate (C)
				;
				\coordinate (S) at ($(A)+(0,4)$);
				\coordinate (J) at ($(A)!.5!(C)$);
				\coordinate (E) at ($(S)!.5!(B)$);
				\coordinate (G) at ($(A)!.5!(J)$);
				\coordinate (F) at ($(A)!.5!(B)$);
				\coordinate (H) at ($(S)!.25!(C)$);
				\draw (S)--(A)--(C)--(B)--cycle (G)--(H)--(E) (S)--(C);
				\draw[dashed] (G)--(F)--(E) (J)--(B)--(A);
				\foreach \p/\q in {A/180,B/0,C/-90,S/90,G/190,H/45,E/80,F/45,J/-150}
				\fill[black] (\p) circle (1.0pt) ($(\p)+(\q:3.5mm)$) node{$\p$};
				\draw pic[draw=black,angle radius=0.2cm] {right angle = B--J--C}
				pic[draw=black,angle radius=0.2cm] {right angle = S--A--C}
				pic[draw=black,angle radius=0.2cm] {right angle = S--A--B}; 
			\end{tikzpicture}
		}
	}
\end{vd}
\subsubsection{Bài tập rèn luyện}
% \centerline{\fcolorbox{red}{yellow!50}{\bf {BÀI TẬP TỰ LUẬN }}}
\begin{bt}%[1K7BM-5]
	Cho tứ diện đều $ABCD$ có cạnh bằng $a$. Gọi $G, G'$ là trọng  tâm tam giác $ABC$ và $ABD$. Diện tích thiết diện của tứ diện cắt bởi mặt phẳng $(BGG')$ là
	\dapso{$\dfrac{a^2 \sqrt{11}}{16}$}
	\loigiai{
		\immini{
			Gọi $H$, $K$ lần lượt là trung điểm của $AC$, $AD$.\\
			Ta có $HK = \dfrac{a}{2}; BH = BK = \dfrac{a \sqrt{3}}{2}$ suy ra tam giác $BHK$ cân tại $B$.\\
			Gọi $I$ là trung điểm của $HK$. Khi đó $BI = \sqrt{ BH^2 - IH^2}=\sqrt{\dfrac{3a^2}{4}-\dfrac{a^2}{16}}=\dfrac{a \sqrt{11}}{4}$\\
			Diện tích thiết diện là $S= \dfrac{1}{2}BI \cdot HK = \dfrac{a^2 \sqrt{11}}{16}.$
		}{
			\begin{tikzpicture}[scale=0.8,font=\footnotesize,line join=round,line cap=round,>=stealth]
				\path
				(3,5) coordinate (A)
				(2,-3) coordinate (B)
				(7,0) coordinate (D)
				(0,0) coordinate (C)
				($(A)!0.5!(C)$) coordinate (K)
				($(A)!0.5!(D)$) coordinate (H)
				($(K)!0.5!(H)$) coordinate (I)
				($(B)!2/3!(K)$) coordinate (G)
				($(B)!2/3!(H)$) coordinate (G')
				;	
				\draw (A)--(C)--(B)--(D)--cycle (K)--(B)--(H) (A)--(B);
				\draw[dashed] (D)--(C)--(B) (K)--(H) (B)--(I);
				\foreach \p/\q in {A/90,B/-90,C/180,D/0,G/190,G'/-10,K/170,H/30,I/90}
				\fill[black] (\p)node[shift={(\q:3mm)}]{$\p$} circle (1.0pt);
				\draw pic[draw=black,angle radius=0.2cm] {right angle = B--I--H}; 
			\end{tikzpicture}
		}
	}
\end{bt}
\begin{bt}%[1K7BM-5]
	Cho hình chóp $S.ABC$ có đáy $ABC$ là tam giác đều cạnh $a$, $SA=2a$ và vuông góc với đáy. Gọi $\left(\alpha \right)$ là mặt phẳng đi qua $B$ và vuông góc với $SC$. Tính diện tích $S$ của thiết diện tạo bởi $\left(\alpha \right)$ với hình chóp đã cho.
	\dapso{$S=\dfrac{a^2\sqrt{15}}{20}$}
	\loigiai{
		\immini{
			Gọi $I$ là trung điểm của $AC$, suy ra $BI\perp AC$.\\
			Ta có $\heva{& BI\perp AC \\ & BI\perp SA \\}\Rightarrow BI\perp \left(SAC\right)\Rightarrow BI\perp SC$. \hfill $(1)$\\
			Kẻ $IH\perp SC\left(H\in SC\right)$.\hfill $(2)$\\
			Từ $(1)$ và $(2)$, suy ra $SC\perp \left(BIH\right)$.\\
			Vậy thiết diện cần tìm là tam giác $IBH$.\\
			Do $BI\perp \left(SAC\right)\Rightarrow BI\perp IH$ nên $\triangle IBH$ vuông tại $I$.\\
			Ta có $BI$ đường cao của tam giác đều cạnh $a$ nên $BI=\dfrac{a\sqrt{3}}{2}$.\\
			Tam giác $CHI$ đồng dạng tam giác $CAS$, suy ra
			$$\dfrac{IH}{SA}=\dfrac{CI}{CS}\Rightarrow IH=\dfrac{CI\cdot SA}{CS}=\dfrac{CI\cdot SA}{\sqrt{SA^2+AC^2}}=\dfrac{a\sqrt{5}}{5}.$$
			Vậy $S_{\triangle BIH}=\dfrac{1}{2}BI\cdot IH=\dfrac{a^2\sqrt{15}}{20}$.
		}{
			\begin{tikzpicture}[scale=0.7, font=\footnotesize, line join=round, line cap=round,>=stealth]
				\path
				(0,0) coordinate (A)
				(2.5,-3) coordinate (B)
				(6,0) coordinate (C)
				($(A)+(0,4)$) coordinate (S)
				($(S)!.65!(C)$) coordinate (H)
				($(A)!.5!(C)$) coordinate (I)
				;
				\draw (S)--(A)--(B)--(C)--cycle (B)--(S);
				\draw[dashed] (A)--(C) (B)--(I)--(H);
				\foreach \p/\q in {A/180,B/-90,C/0,S/90,H/30,I/100}
				\fill[black] (\p)node[shift={(\q:3mm)}]{$\p$} circle (1.0pt);
				\draw pic[draw=black,angle radius=0.2cm] {right angle = S--A--C}
				pic[draw=black,angle radius=0.2cm] {right angle = B--I--A}
				pic[draw=black,angle radius=0.2cm] {right angle = I--H--S}; 
			\end{tikzpicture}
		}
	}
\end{bt}
\begin{bt}%[1K7BM-5]
	Cho hình chóp $S.ABCD$ có đáy $ABCD$ là hình thang vuông tại $A$, đáy lớn $AD=8$, $BC=6$, $SA$ vuông góc với mặt phẳng $\left(ABCD\right)$, $SA=6$. Gọi $M$ là trung điểm $AB$. Gọi $(P)$ là mặt phẳng qua $M$ và vuông góc với$AB$. Thiết diện của $(P)$ và hình chóp có diện tích bằng
	\dapso{$15$}
	\loigiai{
		\immini{
			Do $(P)\perp AB\Rightarrow (P)\parallel SA$.\\
			Gọi $I$ là trung điểm của $SB\Rightarrow MI\parallel SA\Rightarrow MI\subset (P)$.\\
			Gọi $N$ là trung điểm của $CD\Rightarrow MN\perp AB\Rightarrow MN\subset (P)$.\\
			Gọi $K$ là trung điểm của $SC\Rightarrow IK\parallel BC$, mà $MN\parallel BC\Rightarrow MN\parallel IK\Rightarrow IK\subset (P)$. \\
			Vậy thiết diện của $P$ và hình chóp là hình thang $MNKI$ vuông tại $M$. Ta có $MI$ là đường trung bình của tam giác $SAB\Rightarrow MI=\dfrac{1}{2}SA=3$; $IK$ là đường trung bình của tam giác $SBC\Rightarrow IK=\dfrac{1}{2}BC=3$; $MN$ là đường trung bình của hình thang $ABCD$ $\Rightarrow MN=\dfrac{1}{2}\left(AD+BC\right)=7$.\\
			Vậy $S_{MNKI}=\dfrac{IK+MN}{2}\cdot MI=15$.
		}{
			\begin{tikzpicture}[scale=0.7, font=\footnotesize, line join=round, line cap=round,>=stealth]
				\path
				(0,0) coordinate (B)
				(5,0) coordinate (P)
				(7,2) coordinate (D)	
				($(B)+(D)-(P)$) coordinate (A)
				($(A)+(0,5)$) coordinate (S)
				($(A)!.5!(B)$) coordinate (M)
				($(S)!.5!(B)$) coordinate (I)
				($(P)!.25!(B)$) coordinate (C)
				($(S)!.5!(C)$) coordinate (K)
				($(C)!.5!(D)$) coordinate (N)
				;	
				\draw (S)--(B)--(C)--(D)--cycle (S)--(C) (I)--(K)--(N);
				\draw[dashed] (S)--(A)--(D) (B)--(A)--(C) (N)--(M)--(I);
				\foreach \p/\q in {A/160,B/-90,C/-90,D/0,M/-60,N/-50,K/60,I/120}
				\fill[black] (\p)node[shift={(\q:3mm)}]{$\p$} circle (1.0pt);
				\draw pic[draw=black,angle radius=0.2cm] {right angle = S--A--D}; 
			\end{tikzpicture}
		}
	}
\end{bt}
\begin{bt}%[1K7KM-5]
	Cho hình chóp đều $S.ABC$ có đáy $ABC$ là tam giác đều cạnh $a$, tâm $O$; $SO=2a$. Gọi $M$ là điểm thuộc đoạn $AO \left(M\ne A;M\ne O\right)$. Mặt phẳng $\left(\alpha \right)$ đi qua $M$ và vuông góc với $AO$. Đặt $AM=x$. Tính diện tích $S$ của thiết diện tạo bởi $\left(\alpha \right)$ với hình chóp $S.ABC$.
	\dapso{$S=2x^2$}
	\loigiai{
		\immini{
			Vì $S.ABC$ là hình chóp đều nên $SO\perp \left(ABC\right)$ ($O$ là tâm của tam giác $ABC$).\\
			Do đó $SO\perp AA'$ mà $\left(\alpha \right)\perp AA'$ suy ra $SO\parallel \left(\alpha \right)$. \\
			Tương tự ta cũng có $BC\parallel \left(\alpha \right)$.\\
			Qua $M$ kẻ $IJ\parallel BC$ với $I\in AB, J\in AC$; kẻ $MK\parallel SO$ với $K\in SA$.\\
			Khi đó thiết diện là tam giác $KIJ$.\\
			Diện tích tam giác $IJK$ là $S_{\triangle IJK}=\dfrac{1}{2}IJ\cdot MK$.\\
			Trong tam giác $ABC$, ta có $\dfrac{IJ}{BC}=\dfrac{AM}{AA'}$ suy ra $$IJ=\dfrac{AM\cdot BC}{AA'}=\dfrac{2x\sqrt{3}}{3}.$$
		}{
			\begin{tikzpicture}[scale=0.8, font=\footnotesize, line join=round, line cap=round,>=stealth]
				\path
				(0,0) coordinate (A)
				(4,-2.5) coordinate (B)	
				(6,0) coordinate (C)
				($(B)!.5!(C)$) coordinate (N)
				($(A)!.67!(N)$) coordinate (O)
				($(O)+(0,5)$) coordinate (S)
				($(A)!.5!(N)$) coordinate (M)
				($(A)!.5!(B)$) coordinate (I)
				($(A)!.7!(S)$) coordinate (K)
				($(A)!.5!(C)$) coordinate (J)
				;	
				\draw (S)--(A)--(B)--(C)--cycle (B)--(S)--(N) (I)--(K);
				\draw[dashed] (A)--(C)--(O)--(B) (S)--(O) (A)--(N) (M)--(K)--(J)--(I);
				\foreach \p/\q in {A/180,B/-90,C/0,S/90,M/-90,N/-40,K/120,J/-90,I/-120,O/-95}
				\fill[black] (\p)node[shift={(\q:3mm)}]{$\p$} circle (1.0pt);
			\end{tikzpicture}
		}
		\noindent Tương tự trong tam giác $SAO$, ta có $\dfrac{MK}{SO}=\dfrac{AM}{AO}$ suy ra $MK=\dfrac{AM\cdot SO}{AO}=2x\sqrt{3}$.\\
		Vậy ${S}_{\triangle IJK}=\dfrac{1}{2}\cdot \dfrac{2x\sqrt{3}}{3}\cdot 2x\sqrt{3}=2{x}^2$.
	}
\end{bt}
% \centerline{\fcolorbox{red}{yellow!50}{\bf {CÂU HỎI TRẮC NGHIỆM}}}
% \Opensolutionfile{ans}[ans/ans-1K7-2-5]
% \begin{ex}%[1K7KM-5]
% 	Cho hình lăng trụ đứng $ABC.A'B'C'$ có đáy $ABC$ là tam giác vuông tại $A$, với $AB=c$, $AC=b$, cạnh bên $AA'=h$. Mặt phẳng $(P)$ đi qua $A'$ và vuông góc với $B'C$. Thiết diện của lăng trụ cắt bởi mặt phẳng $(P)$ có hình:
% 	\begin{center}
% 		\begin{tabular}{ccc}
% 			{\begin{tikzpicture}[scale=.8]
% 					\path
% 					(0,0) coordinate (B)
% 					(2,-1) coordinate (A)
% 					(4,0) coordinate (C)
% 					($(A)-(0,3)$) coordinate (A')
% 					($(B)-(0,3)$) coordinate (B')
% 					($(C)-(0,3)$) coordinate (C')
% 					($(B')!.7!(C')$) coordinate (K')
% 					($(B')!1.4!(B)$) coordinate (I)
% 					(intersection of I--A' and B--A) coordinate (E)
% 					(intersection of I--K' and B--C) coordinate (F)
% 					;				
% 					\draw (I)--(B')--(A')--(C')--(C)--(F)--cycle (F)--(E)
% 					(B)--(A)--(C) (A)--(A')--(I);
% 					\draw[dashed] (C)--(B')--(C') (B)--(F)--(K')--(A');
% 					\foreach \p/\q in {A/90,B/180,C/10,A'/-90,B'/180,C'/0,K'/60}
% 					\fill[black] (\p)node[shift={(\q:3mm)}]{$\p$} circle (1.0pt);		
% 			\end{tikzpicture}}&{\begin{tikzpicture}[scale=.8]
% 					\path
% 					(0,0) coordinate (B)
% 					(2,-1) coordinate (A)
% 					(4,0) coordinate (C)
% 					($(A)-(0,3)$) coordinate (A')
% 					($(B)-(0,3)$) coordinate (B')
% 					($(C)-(0,3)$) coordinate (C')
% 					($(B')!.7!(C')$) coordinate (K')
% 					($(B')!0.8!(B)$) coordinate (I)
% 					;				
% 					\draw (B)--(B')--(A')--(C')--(C)--cycle (B)--(A)--(A')--(I) (A)--(C);
% 					\draw[dashed] (C)--(B')--(C') (I)--(K')--(A');
% 					\foreach \p/\q in {A/90,B/180,C/10,A'/-90,B'/180,C'/0,K'/60}
% 					\fill[black] (\p)node[shift={(\q:3mm)}]{$\p$} circle (1.0pt);
% 			\end{tikzpicture}}&{\begin{tikzpicture}[scale=.8]
% 					\path
% 					(0,0) coordinate (B)
% 					(2,-1) coordinate (A)
% 					(4,0) coordinate (C)
% 					($(A)-(0,3)$) coordinate (A')
% 					($(B)-(0,3)$) coordinate (B')
% 					($(C)-(0,3)$) coordinate (C')
% 					($(B)!.6!(C)$) coordinate (K')
% 					($(B')!0.8!(B)$) coordinate (I)
% 					($(C')!0.8!(C)$) coordinate (J)
% 					;				
% 					\draw (B)--(B')--(A')--(C')--(C)--cycle (B)--(A)--(A')--(I) (A)--(C) (A')--(J);
% 					\draw[dashed] (C)--(B')--(C') (I)--(K')--(A') (K')--(J);
% 					\foreach \p/\q in {A/90,B/180,C/10,A'/-90,B'/180,C'/0,K'/60}
% 					\fill[black] (\p)node[shift={(\q:3mm)}]{$\p$} circle (1.0pt);
% 			\end{tikzpicture}}\\
% 			{Hình 1}&{Hình 2}&{Hình 3}
% 		\end{tabular}
% 	\end{center}
% 	\choice
% 	{Hình 2}
% 	{Hình 2 và Hình 3}
% 	{Hình 1}
% 	{\True Hình 1 và Hình 2}
% 	\loigiai{
% 		Hình số 3 không thỏa mãn giả thiết $(P)\perp B'C$.}
% \end{ex}
% \begin{ex}%[1K7KM-5]
% 	Cho hình chóp $S.ABCD$ có $SA\perp (ABCD)$, $ABCD$ là hình vuông cạnh $a$, $SA=2a$. $M$ là trung điểm $AD$. Gọi $(P)$ là mặt phẳng qua $M$ và vuông góc với $AC$. Thiết diện tạo bởi hình chóp và mặt phẳng $(P)$ là
% 	\choice
% 	{Hình vuông}
% 	{Hình thang vuông}
% 	{\True Hình ngũ giác}
% 	{Hình tam giác}
% 	\loigiai{
% 		\immini{ Vì $(P)$ qua $M$ và vuông góc với $AC$ nên $(P)\parallel BD$ và $(P)\parallel SA$.\\
% 			Qua $M$ vẽ đường thẳng song song với $BD$ cắt $AC$, $AB$ lần lượt tại $K$ va $N$.\\
% 			Qua $M$ vẽ đường thẳng song song với $SA$ cắt $SD$ tại $Q$.\\
% 			Qua $K$ vẽ đường thẳng song song với $SA$ cắt $SD$ tại $I$.\\
% 			Qua $N$ vẽ đường thẳng song song với $SA$ cắt $SD$ tại $E$.\\
% 			Vậy thiết diện là hình ngũ giác $MNPIQ$.
% 		}{
% 			\begin{tikzpicture}[scale=1,font=\footnotesize,line join=round,line cap=round,>=stealth]
% 				\path
% 				(1,0) coordinate (A)
% 				(6,0) coordinate (B)
% 				(-1,-2) coordinate (D)	
% 				(1,4) coordinate (S)
% 				($(B)+(D)-(A)$) coordinate (C)
% 				($(A)!0.5!(D)$) coordinate (M)
% 				($(A)!0.5!(B)$) coordinate (N)
% 				($(S)!0.5!(B)$) coordinate (E)
% 				($(S)!0.5!(D)$) coordinate (Q)
% 				(intersection of A--C and M--N) coordinate (K)
% 				($(S)!0.25!(C)$) coordinate (I)
% 				;
% 				\draw (S)--(D)--(C)--(B)--cycle (Q)--(I)--(E) (S)--(C);
% 				\draw[dashed] (S)--(A)--(B)--(D) (D)--(A)--(C) (Q)--(M)--(N)--(E) (K)--(I);
% 				\foreach \p/\q in {A/45,B/0,C/-90,D/-90,S/90,Q/160,K/-90,M/-70,N/-85,E/70,I/70}
% 				\fill[black] (\p)node[shift={(\q:3mm)}]{$\p$} circle (1.0pt);
% 				\draw pic[draw=black,angle radius=0.2cm] {right angle = S--A--D}
% 				pic[draw=black,angle radius=0.2cm] {right angle = S--A--B}
% 				pic[draw=black,angle radius=0.2cm] {right angle = B--A--D}; 
% 			\end{tikzpicture}
% 	}}
% \end{ex}
% \begin{ex}%[1K7KM-5]
% 	Cho tứ diện đều $ABCD$. Thiết diện của tứ diện $ABCD$ và mặt phẳng trung trực của cạnh $BC$ là
% 	\choice
% 	{Hình thang}
% 	{Tam giác vuông}
% 	{\True Tam giác cân}
% 	{Hình bình hành}
% 	\loigiai{
% 		\immini{Gọi $I$ là trung điểm của $BC$, ta có $\heva{&AI\perp BC\\& DI\perp BC}$ nên\\ $BC\perp (AID)$.\\ Vậy $(ADI)$ chính là mặt trung trực của $BC$ và thiết diện là tam giác $ADI$.\\
% 			Dễ thấy $IA=ID=\dfrac{AB\sqrt{3}}{2}$.\\ Vậy thiết diện là tam giác cân.}{
% 			\begin{tikzpicture}[scale=.7, line join = round, line cap = round,>=stealth,font=\footnotesize]
% 				\path	
% 				(0,0) coordinate (B)
% 				(7,0) coordinate (C)	
% 				(2,-3) coordinate (D)
% 				($(B)!0.5!(C)$) coordinate (I)
% 				($(D)!2/3!(I)$) coordinate (G)	
% 				($(G)+(0,5)$) coordinate (A)	
% 				;
% 				\draw (A)--(B)--(D)--(C)--cycle (A)--(D);
% 				\draw[dashed] (B)--(C) (A)--(I)--(D);
% 				\foreach \p/\q in {A/90,B/180,C/0,D/-90,I/-50}
% 				\fill[black] (\p)node[shift={(\q:3mm)}]{$\p$} circle (1.0pt);
% 		\end{tikzpicture}}
% 	}
% \end{ex}
% \begin{ex}%[1K7KM-5]
% 	Cho tứ diện $ABCD$ có tất các cạnh bằng $6a$. Gọi $M$, $N$ lần lượt là trung điểm của $CA$, $CB$, $P$ là điểm trên cạnh $BD$ sao cho $BP = 2PD$. Diện tích $S$ của thiết diện của tứ diện $ABCD$ bị cắt bởi mặt phẳng $\left(MNP\right)$ là
% 	\choice
% 	{$S = \dfrac{5\sqrt{51}a^2}{2}$}
% 	{$S = \dfrac{5\sqrt{147}a^2}{2}$}
% 	{$S = \dfrac{5\sqrt{147}a^2}{4}$}
% 	{\True $S = \dfrac{5\sqrt{51}a^2}{4}$}
% 	\loigiai{
% 		\immini{
% 			Gọi $E$ là trung điểm của $AB$.\\
% 			Do giả thiết $\heva{&DE\perp AB\\&CE\perp AB}\Rightarrow AB\perp\left(CED\right)$.\\
% 			Dễ thấy $MN$ là đường trung bình trong $\triangle ABC$\\
% 			nên $MN\parallel AB$.\\
% 			Gọi $Q$ là giao điểm của $DA$ với mặt phẳng $\left(MNP\right)$ suy ra  $PQ\parallel AB$ ($Q\in AD$). Nên $MNPQ$ là hình thang.
% 			Mà $BP = 2PD$ nên $\dfrac{DP}{DB} = \dfrac{1}{3}$.
% 			Mặt khác, xét tam giác $DAB$ theo định lý Talét ta có $\dfrac{DQ}{DA} = \dfrac{DP}{DB} = \dfrac{PQ}{AB}$.\\
% 			Suy ra $PQ = \dfrac{1}{3}AB\Leftrightarrow PQ = 2a$.
% 		}{
% 		\begin{tikzpicture}[scale=1, font=\footnotesize,line join=round, line cap=round,>=stealth]
% 			\path
% 			(0,0) coordinate (A)
% 			(2,-2) coordinate (B)
% 			(2,4.5) coordinate (D)	
% 			(5,0) coordinate (C)
% 			($(A)!0.5!(C)$) coordinate (M)
% 			($(B)!0.5!(C)$) coordinate (N)
% 			($(A)!0.5!(B)$) coordinate (E)
% 			($(D)!1/3!(B)$) coordinate (P)
% 			($(D)!1/3!(A)$) coordinate (Q)
% 			(intersection of D--E and P--Q) coordinate (I)
% 			(intersection of C--E and M--N) coordinate (J)
% 			;
% 			\draw (D)--(A)--(B)--(C)--cycle (E)--(D)--(B) (Q)--(P)--(N);
% 			\draw[dashed] (A)--(C)--(E) (N)--(M)--(Q) (J)--(I);
% 			\foreach \p/\q in {A/180,B/-90,C/0,D/90,M/-110,N/-45,E/-140,P/20,Q/150,I/30,J/-100}
% 			\fill[black] (\p)node[shift={(\q:3mm)}]{$\p$} circle (1.0pt);
% 			\draw pic[draw=black,angle radius=0.2cm] {right angle = D--E--A}
% 			pic[draw=black,angle radius=0.2cm] {right angle = C--E--B}; 
% 		\end{tikzpicture}
% 		}\noindent
% 		Gọi $\left\{I\right\} = PQ\cap DE$ và $\left\{J\right\} = MN\cap CE$.
% 		Do $MN\parallel AB$ theo chứng minh trên $MN\perp\left(DEC\right)$.\\ Do đó $MN\perp IJ$ nên $S_{MNPQ} = \dfrac{\left(PQ + MN\right)\cdot IJ}{2}$.\\
% 		Dễ thấy $EJ = \dfrac{CE}{2} = \dfrac{3\sqrt{3}a}{2}$ và $EI = \dfrac{2}{3}DE = 2\sqrt{3}a$.\\
% 		Xét tam giác $IJE$ ta có $IJ^2 = IE^2 + JE^2 - 2 IE\cdot JE\cdot\cos\widehat{IEJ}\quad (1)$.\\
% 		Mặt khác trong tam giác $EDC$ ta có\\
% 		$\cos\widehat{DEC} = \dfrac{DE^2 + EC^2 - DC^2}{2ED\cdot EC}\Leftrightarrow \cos\widehat{DEC} = \dfrac{27a^2 + 27a^2 - 36a^2}{2\cdot 3\sqrt{3}a\cdot 3\sqrt{3}a}\Leftrightarrow \cos\widehat{DEC} = \dfrac{1}{3}\quad (2)$.\\
% 		Từ $(1)$ và $(2)$ suy ra $IJ^2  = \dfrac{27a^2}{4} + 12a^2 - 2\cdot \dfrac{3\sqrt{3}a}{2}\cdot 2\sqrt{3}a\cdot\dfrac{1}{3}\Leftrightarrow IJ^2 = \dfrac{51a^2}{4}\Leftrightarrow IJ = \dfrac{\sqrt{51}a}{2}$.\\
% 		Do đó $S_{MNPQ} = \dfrac{\left(2a + 3a\right)\cdot \dfrac{\sqrt{51}a}{2}}{2} = \dfrac{5\sqrt{51}a^2}{4}$.
% 	}
% \end{ex}
% \begin{ex}%[1K7KM-5]
% 	Cho hình thoi $ ABCD $ có cạnh bằng $ 4 $ cm, $ \widehat{BAD}=60^\circ $ và hình bình hành $ ADEF $ không cùng nằm trên một mặt phẳng. Biết tam giác $ EDB $ là tam giác cân tại $ E $ và $ EC=8 $ cm. Tính diện tích $ S $ của thiết diện của hình chóp $ E.ABCD $ cắt bởi mặt phẳng $ FBD $.
% 	\choice
% 	{$ S=16 $ cm$ ^2 $}
% 	{$ S=32 $ cm$ ^2 $}
% 	{\True $ S=8 $ cm$ ^2 $}
% 	{$ S=4 $ cm$ ^2 $}
% 	\loigiai{
% 		\immini{
% 			Gọi $ I $ là giao điểm của $ AE $ và $ FD $. Suy ra tam giác $ IBD $ là thiết diện của hình chóp $ E.ABCD $ cắt bởi mặt phẳng $ FBD $.\\
% 			Gọi $ O $ là giao điểm của $ AC $ và $ BD $. Vì $ ABCD $ là hình thoi nên $ AO\perp BD $ và $ O  $ là trung điểm của $ BD $. Mặt khác tam giác $ EBD $ cân tại $ E $  nên $ EO\perp BD $. Hệ quả là $ BD\perp (AOE) $, kéo theo $ BD\perp IO $.\\
% 			Ta có $ IO $ là đường trung bình của tam giác $ AEC $ nên $ IO=\dfrac{1}{2}EC=4 $ cm.\\
% 			$ ABCD $ là hình thoi có $ \widehat{BAD}=60^\circ $ nên $ ABD $ là tam giác đều, suy ra $ BD=4 $ cm.\\
% 			Vậy $ S_{IBD}=\dfrac{1}{2}IO\cdot BD=8 $ cm$ ^2 $.
% 		}{
% 			\begin{tikzpicture}[scale=0.8, font=\footnotesize, line join=round, line cap=round, >=stealth]
% 				\path
% 				(0,0) coordinate (A)
% 				(4.5,0) coordinate (B)
% 				(1.7,1.3) coordinate (D)	
% 				(3.1,4) coordinate (E)	
% 				($(B)+(D)-(A)$) coordinate (C)
% 				($(A)+(E)-(D)$) coordinate (F)
% 				($(E)!0.5!(A)$) coordinate (I)
% 				(intersection of A--C and B--D) coordinate (O)	
% 				;
% 				\draw (E)--(A)--(B)--(C)--cycle (E)--(F)--(A) (E)--(B)--(I)--(F);
% 				\draw[dashed] (E)--(D)--(A)--(C) (C)--(D)--(B) (D)--(I) (E)--(O);
% 				\foreach \p/\q in {A/-120,B/-20,C/10,D/-90,E/90,O/-90,F/110,I/20}
% 				\fill[black] (\p)node[shift={(\q:3mm)}]{$\p$} circle (1.0pt);
% 			\end{tikzpicture}
% 		}
% 	}
% \end{ex}
% \begin{ex}%[1K7KM-5]
% 	Cho hình chóp $S.ABC$ có đáy $ABC$ là tam giác đều cạnh bằng $2a$, $SA\perp (ABC)$ và $SA=a\sqrt{3}$. Gọi $M$ là trung điểm $BC$, gọi $(P)$ là mặt phẳng đi qua $A$ và vuông góc với $SM$. Tính diện tích thiết diện của $(P)$ và hình chóp $S.ABC$?
% 	\choice
% 	{\True $\dfrac{a^2\sqrt{6}}{4}$}
% 	{$\dfrac{a^2}{2}$}
% 	{$\dfrac{a^2\sqrt{3}}{4}$}
% 	{$\dfrac{a^2\sqrt{6}}{2}$}
% 	\loigiai{
% 		\immini{
% 			Dễ thấy $\triangle SAB=\triangle SAC\Rightarrow SB=SC\Rightarrow \triangle SBC$ cân tại $S$.\\
% 			Vì $M$ là trung điểm của $BC$ nên $SM\perp BC$.\\
% 			Ta có $\heva{&(P)\perp SM\\&BC\perp SM\\&BC\not \subset (P)}\Rightarrow BC\parallel (P).$\\
% 			Kẻ $AI\perp SM$ tại $I$.\\
% 			Từ $\heva{&BC\parallel (P)\\&BC\subset (SBC)\\&I\in (P)\cap (SBC)}\Rightarrow (P)\cap (SBC)=Ix\parallel BC.$\\
% 			Đường thẳng $Ix$ cắt $SB$, $SC$ lần lượt tại $E$ và $F$.\\
% 			Ta có $AM=AB\sin \widehat{B}=2a\cdot \dfrac{\sqrt{3}}{2}=a\sqrt{3}.$
% 		}
% 		{
% 			\begin{tikzpicture}[scale=0.8,font=\footnotesize,line join=round,line cap=round,>=stealth]
% 				\path
% 				(0,0) coordinate (A)
% 				(3,-3) coordinate (B)
% 				(7,0) coordinate (C)
% 				($(A)+(0,5)$) coordinate (S)
% 				($(B)!0.5!(C)$) coordinate (M)
% 				($(S)!0.57!(M)$) coordinate (I)
% 				($(S)!0.57!(B)$) coordinate (E)
% 				($(S)!0.57!(C)$) coordinate (F)
% 				;
% 				\draw (S)--(A)--(B)--(C)--cycle (B)--(S)--(M) (A)--(E)--(F);
% 				\draw[dashed] (M)--(A)--(C) (I)--(A)--(F);
% 				\foreach \p/\q in {A/180,B/-90,C/0,S/90,E/-10,F/60,M/-30,I/0}
% 				\fill[black] (\p)node[shift={(\q:3mm)}]{$\p$} circle (1.0pt);
% 			\end{tikzpicture}
% 		}
% 		\noindent
% 		Từ đó suy ra $\triangle SAM$ vuông cân tại $A$ nên $I$, $E$, $F$ lần lượt là trung điểm của $SM$, $SB$, $SC$.
% 		Trong tam giác $SBC$ có $EF=\dfrac{1}{2}BC=a.$\\
% 		Trong tam giác vuông cân $SAM$ có $AI=AM\sin \widehat{M}=\dfrac{a\sqrt{6}}{2}.$\\
% 		Thiết diện cần tìm tam giác $SEF.$\\
% 		Diện tích thiết diện $S_{SEF}=\dfrac{1}{2}\cdot SI\cdot EF=\dfrac{1}{2}\cdot \dfrac{a\sqrt{6}}{2}\cdot a=\dfrac{a^2\sqrt{6}}{4}.$
% 	}
% \end{ex}
% \begin{ex}%[1K7KM-5]
% 	Cho hình chóp $S.ABCD$ có đáy $ABCD$ là hình chữ nhật với $AB=a$, $AD=a\sqrt{3}$. Cạnh bên $SA=2a$ và vuông góc với đáy. Mặt phẳng $\left(\alpha \right)$ đi qua $A$ vuông góc với $SC$. Tính diện tích $S$ của thiết diện tạo bởi $\left(\alpha \right)$ với hình chóp đã cho.
% 	\choice
% 	{\True $S=\dfrac{12a^2\sqrt{6}}{35}$}
% 	{$S=\dfrac{a^2\sqrt{6}}{7}$}
% 	{$S=\dfrac{6a^2\sqrt{6}}{35}$}
% 	{$S=\dfrac{a^2\sqrt{6}}{5}$}
% 	\loigiai{
% 		\immini{
% 			Trong tam giác $SAC$, kẻ $AI\perp SC\left(I\in SC\right)$.\\
% 			Trong mp$\left(SBC\right)$, dựng đường thẳng đi qua $I$ vuông góc với $SC$ cắt $SB$ tại $M$.\\
% 			Trong mp$\left(SCD\right)$, dựng đường thẳng qua $I$ vuông góc với $SC$ cắt $SD$ tại $N$.\\
% 			Khi đó thiết diện của hình chóp cắt bởi mp $\left(\alpha \right)$ là tứ giác $AMIN$.\\
% 			Ta có $SC\perp \left(\alpha \right)\Rightarrow SC\perp AM$.\hfill $(1)$\\
% 			Lại có $\heva{& BC\perp AB \\ & BC\perp SA \\}\Rightarrow BC\perp \left(SAB\right)\Rightarrow BC\perp AM$.\hfill $(2)$\\
% 			Từ $(1)$ và $(2)$, suy ra $AM\perp \left(SBC\right)\Rightarrow AM\perp MI$.\\
% 			Chứng minh tương tự, ta được $AN\perp NI$.
% 		}{
% 			\begin{tikzpicture}[scale=0.7, font=\footnotesize, line join=round, line cap=round,>=stealth]
% 				\path	
% 				(0,0) coordinate (B)
% 				(6,0) coordinate (C)	
% 				(8,2.5) coordinate (D)
% 				($(B)+(D)-(C)$) coordinate (A)	
% 				($(A)+(0,5)$) coordinate (S)
% 				($(D)!.5!(B)$) coordinate (O)
% 				($(S)!.5!(B)$) coordinate (M)
% 				($(D)!.5!(S)$) coordinate (N)
% 				(intersection of M--N and S--O) coordinate (P)
% 				(intersection of A--P and S--C) coordinate (I)
% 				;
% 				\draw (S)--(D)--(C)--(B)--cycle (M)--(I)--(N) (S)--(C);
% 				\draw[dashed] (S)--(A)--(B) (D)--(A)--(M) (N)--(A)--(I);
% 				\foreach \p/\q in {A/-90,B/-90,C/-90,D/0,S/90,M/150,N/40,I/70}
% 				\fill[black] (\p)node[shift={(\q:3mm)}]{$\p$} circle (1.0pt);
% 				\draw pic[draw=black,angle radius=0.2cm] {right angle = S--A--D}
% 				pic[draw=black,angle radius=0.2cm] {right angle = A--N--D}
% 				pic[draw=black,angle radius=0.2cm] {right angle = A--I--C}
% 				pic[draw=black,angle radius=0.2cm] {right angle = A--M--B}
% 				pic[draw=black,angle radius=0.2cm] {right angle = S--I--N}
% 				pic[draw=black,angle radius=0.2cm] {right angle = S--I--M}; 
% 			\end{tikzpicture}
% 		}
% 		\noindent Do đó $S_{AMIN}=S_{\triangle AMI}+S_{\triangle ANI}=\dfrac{1}{2}AM\cdot MI+\dfrac{1}{2}AN\cdot NI$.\\
% 		Vì $AM, AI, AN$ là các đường cao của các tam giác vuông $SAB, SAC, SAD$ nên:\\
% 		$AM=\dfrac{SA\cdot AB}{\sqrt{SA^2+AB^2}}=\dfrac{2a}{\sqrt{5}}$; $AI=\dfrac{SA\cdot AC}{\sqrt{SA^2+AC^2}}=a\sqrt{2}$; $AN=\dfrac{SA\cdot AD}{\sqrt{SA^2+AD^2}}=\dfrac{2a\sqrt{21}}{7}$.\\
% 		Suy ra $MI=\sqrt{AI^2-AM^2}=\dfrac{a\sqrt{30}}{5}$ và $NI=\sqrt{AI^2-AN^2}=\dfrac{a\sqrt{14}}{7}$.\\
% 		Vậy $S_{AMIN}=\dfrac{1}{2}\left(\dfrac{2a}{\sqrt{5}}\cdot \dfrac{a\sqrt{30}}{5}+\dfrac{2a\sqrt{21}}{7}\cdot \dfrac{a\sqrt{14}}{7}\right)=\dfrac{12a^2\sqrt{6}}{35}$.
% 	}
% \end{ex}
% \begin{ex}%[1K7KM-5]
% 	Cho hình lập phương $ABCD.A'B'C'D'$ có độ dài mỗi cạnh bằng $1$. Gọi $(P)$ là mặt phẳng chứa $CD'$ và tạo với mặt phẳng $BDD'B'$ một góc $x$ nhỏ nhất, cắt hình lập phương theo một thiết diện có diện tích $S$. Giá trị của $S$ bằng
% 	\choice
% 	{$\dfrac{\sqrt{6}}{6}$}
% 	{$\dfrac{2\sqrt{6}}{3}$}
% 	{\True $\dfrac{\sqrt{6}}{4}$}
% 	{$\dfrac{\sqrt{6}}{12}$}
% 	\loigiai{
% 		\begin{center}
% 			\begin{tikzpicture}[scale=0.7,font=\footnotesize,line join=round,line cap=round,>=stealth]
% 				\path
% 				(0,0) coordinate (A)	
% 				($(A)+(0:4.5)$) coordinate (D)
% 				($(A)+(-130:3)$) coordinate (B)
% 				($(B)+(0:4.5)$) coordinate (C)
% 				($(A)+(90:4)$) coordinate (A')
% 				($(A')+(B)-(A)$) coordinate (B')
% 				($(A')+(C)-(A)$) coordinate (C')
% 				($(A')+(D)-(A)$) coordinate (D')	
% 				($(B)!2!(D)$) coordinate (M)	
% 				(intersection of B--B' and M--D') coordinate (N)
% 				(intersection of C--N and B'--C') coordinate (I)
% 				(intersection of A--C and B--D) coordinate (O)	
% 				;
% 				\fill[color = red!10, opacity = 0.3] (I)--(D')--(C)--cycle;
% 				\draw (N)--(B)--(C)--(D)--(M)--cycle (N)--(C)--(C')--(D')--(I) (B')--(C') (C)--(D')--(D);
% 				\draw[dashed] (B')--(A')--(D') (A')--(A)--(B)--(D)--(A);
% 				\foreach \p/\q in {A/-90,B/-90,C/-90,D/-80,N/90,A'/90,B'/180,C'/90,D'/80,M/0,I/80}
% 				\fill[black] (\p)node[shift={(\q:3mm)}]{$\p$} circle (1.0pt);	
% 			\end{tikzpicture}
% 		\end{center}
% 		\begin{itemize}
% 			\item Góc của $(P)$ qua $CD'$ hợp với $ (BB'D'D)$ một góc nhỏ nhất bằng với góc giữa đường thẳng $CD'$ và $(BB'D'D) $.
% 			\item Ta có $ \heva{& OD = \dfrac{\sqrt{2}}{2}  \\ & OD' = \sqrt{\dfrac{3}{2}}} \Rightarrow \cos \widehat{DOD'} = \dfrac{\sqrt{3}}{3} \Rightarrow OM = \dfrac{3\sqrt{2}}{2} \Rightarrow D$ là trung điểm của $BM$.
% 			\item Kéo dài $MD'$ cắt $BB'$ tại $N$. Đường thẳng $CN$ cắt $B'C'$ tại $I$, ta được $I$ là trung điểm $B'C'$.\\
% 			\item Ta được thiết diện cần tìm là $ \triangle ICD'$.
% 			\item Tính được $S = \dfrac{\sqrt{6}}{4}$.
% 		\end{itemize}
% 	}
% \end{ex}
% \begin{ex}%[1K7KM-5]
% 	Cho hình chóp $S.ABC$ có đáy $ABC$ là tam giác vuông tại $B$, $SA\perp(ABC)$. Cho $AB=a,~BC=a\sqrt{3},~SA=2a$. Mặt phẳng $(P)$ qua $A$ và vuông góc với $SC$. Tính diện tích thiết diện của hình chóp cắt bởi mặt phẳng $(P)$.
% 	\choice
% 	{$\dfrac{a^2\sqrt{6}}{3}$}
% 	{\True $\dfrac{a^2\sqrt{6}}{5}$}
% 	{$\dfrac{a^2\sqrt{6}}{4}$}
% 	{$\dfrac{a^2\sqrt{3}}{3}$}
% 	\loigiai{
% 		\immini{
% 			Gọi $M$ là trung điểm của $SC$. Do $SA=AC=2a$ nên \[AM\perp SC.\tag{1}\]
% 			Trong $(SBC)$, gọi điểm $N$ thuộc cạnh $SB$ sao cho \[MN\perp SC.\tag{2}\]
% 			Từ (1) và (2) suy ra $SC\perp (AMN)$. Khi đó thiết diện của hình chóp cắt bởi mặt phẳng $(P)$ là tam giác $AMN$.\\
% 			Ta có $AM=\dfrac{SC}{2}=a\sqrt{2}$.\hfill(3)
% 		}{
% 			\begin{tikzpicture}[scale=0.75,font=\footnotesize,line join=round,line cap=round,>=stealth]
% 				\path
% 				(0,0) coordinate (A)
% 				(3,-2) coordinate (B)
% 				(5,0) coordinate (C)
% 				($(A)!1!90:(C)$) coordinate (S)
% 				($(S)!0.5!(C)$) coordinate (M)
% 				($(S)!4/5!(B)$) coordinate (N)
% 				;	
% 				\draw (S)--(A)--(B)--(C)--cycle (A)--(N)--(M) (S)--(B);
% 				\draw[dashed] (C)--(A)--(M);
% 				\foreach \p/\q in {A/180,B/-90,C/0,S/90,M/60,N/0}
% 				\fill[black] (\p)node[shift={(\q:3mm)}]{$\p$} circle (1.0pt);	
% 				\draw pic[draw=black,angle radius=0.2cm] {right angle = S--M--A}
% 				pic[draw=black,angle radius=0.2cm] {right angle = A--B--C}
% 				pic[draw=black,angle radius=0.2cm] {right angle = N--M--C}; 
% 			\end{tikzpicture}
% 		}
% 		\noindent Vì $\triangle SMN\sim\triangle SBC$ nên
% 		\begin{align*}
% 			& \dfrac{SM}{SB}=\dfrac{MN}{BC}\Leftrightarrow MN=\dfrac{SM\cdot BC}{SB}=\dfrac{a\sqrt{2}\cdot a\sqrt{3}}{a\sqrt{5}}=\dfrac{a\sqrt{30}}{5}.\tag{4}\\
% 			& \dfrac{SN}{SC}=\dfrac{MN}{BC}\Leftrightarrow SN=\dfrac{SC\cdot MN}{BC}=\dfrac{2a\sqrt{2}\cdot \dfrac{a\sqrt{30}}{5}}{a\sqrt{3}}=\dfrac{4a\sqrt{5}}{5}.
% 		\end{align*}
% 		\textbf{Cách 1.}\\
% 		Xét $\triangle SAB$, $\cos S=\dfrac{SA}{SB}=\dfrac{2a}{a\sqrt{5}}=\dfrac{2\sqrt{5}}{5}$.\\
% 		Xét $\triangle SAB$,
% 		\begin{align*}
% 			& AN^2=SA^2+SN^2-2SA\cdot SN\cdot\cos S=4a^2+\dfrac{16a^2}{5}-2\cdot2a\cdot\dfrac{4a\sqrt{5}}{5}\cdot\dfrac{2\sqrt{5}}{5}=\dfrac{4a^2}{5}.\\
% 			\Leftrightarrow &AN=\dfrac{2a\sqrt{5}}{5}.\tag{5}
% 		\end{align*}
% 		Từ (3),(4) và (5) đặt $p=\dfrac{AM+MN+AN}{2}$. Ta tính được
% 		\[S_{AMN}=\sqrt{p(p-AM)(p-MN)(p-AN)}=\dfrac{a^2\sqrt{6}}{5}.\]
% 		\textbf{Cách 2.}\\
% 		Ta có
% 		\begin{eqnarray*}
% 			& &\dfrac{V_{S.AMN}}{V_{S.ACB}}=\dfrac{SM}{SC}\cdot\dfrac{SN}{SB} \\
% 			& \Leftrightarrow & \dfrac{SM\cdot S_{AMN}}{SA\cdot S_{ACB}}=\dfrac{SM}{SC}\cdot\dfrac{SN}{SB}\\
% 			& \Leftrightarrow & S_{AMN}=\dfrac{S_{ACB}\cdot SA\cdot SN}{SC\cdot SB}=\dfrac{\dfrac{1}{2}a^2\sqrt{3}\cdot 2a\cdot \dfrac{4a\sqrt{5}}{5}}{2a\sqrt{2}\cdot a\sqrt{5}}=\dfrac{a^2\sqrt{6}}{5}.
% 		\end{eqnarray*}
% 	}
% \end{ex}
% \begin{ex}%[1K7KM-5]
% 	Cho hình chóp $S.ABCD$ có đáy $ABCD$ là hình thang cân $(AD \parallel BC)$, $BC=2a$, $AB=AD=DC=a$ với $a>0$. Gọi $O$ là giao điểm của $AC$ và $BD$. Biết $SD$ vuông góc $AC$. $M$ là một điểm thuộc đoạn $OD$; $MD=x$ với $x>0$. $M$ khác $O$ và $D$. Mặt phẳng $(\alpha)$ qua $M$ và song song với hai đường thẳng $SD$ và $AC$ cắt khối chóp $S.ABCD$ theo một thiết diện. Tìm $x$ để diện tích thiết diện là lớn nhất?
% 	\choice
% 	{$a$}
% 	{\True $a\dfrac{\sqrt{3}}{4}$}
% 	{$a\dfrac{\sqrt{3}}{2}$}
% 	{$a\sqrt{3}$}
% 	\loigiai{
% 		\immini{Trong mp$(SBD)$ kẻ đường thẳng qua $M$ song song với $SD$, cắt cạnh $SB$ tại $H$.\\
% 			Trong mp$(ABCD)$ kẻ đường thẳng qua $M$ song song với $AC$, cắt các cạnh $DA$ và $DC$ lần lượt tại $E$ và $F$.\\
% 			Trong mp$(SDA)$ kẻ đường thẳng qua $E$ song song với $SD$, cắt cạnh $SA$ tại $I$.\\
% 			Trong $(SDC)$ kẻ đường thẳng qua $F$ song song với $SD$, cắt cạnh $SC$ tại $G$.\\
% 			Khi đó thiết diện của khối chóp $S.ABCD$ cắt bởi mặt phẳng $(\alpha)$ là ngũ giác $EFGHI$.}
% 		{\begin{tikzpicture}[scale=1,font=\footnotesize,line join=round,line cap=round,>=stealth]
% 				\path
% 				(2,-3) coordinate (A)
% 				(0,0) coordinate (B)
% 				(7,0) coordinate (C)
% 				(6,-3) coordinate (D)	
% 				(3,5) coordinate (S)
% 				($(B)!0.6!(C)$) coordinate (K)
% 				(intersection of A--C and B--D) coordinate (O)
% 				($(O)!0.4!(D)$) coordinate (M)
% 				($(A)!0.4!(D)$) coordinate (E)
% 				($(C)!0.4!(D)$) coordinate (F)
% 				($(S)+(M)-(D)$) coordinate (m)
% 				(intersection of M--m and B--S) coordinate (H)
% 				($(A)!0.4!(S)$) coordinate (I)
% 				($(C)!0.4!(S)$) coordinate (G)
% 				(intersection of M--H and I--G) coordinate (N)	
% 				;
% 				\draw (S)--(B)--(A)--(D)--(C)--cycle (A)--(S)--(D) (E)--(I)--(H) (G)--(F);
% 				\draw[dashed] (S)--(K)--(A)--(C)--(B)--(D) (K)--(D) (E)--(F) (M)--(H)--(G)--(I);
% 				\foreach \p/\q in {A/-90,B/180,C/0,D/-90,S/90,H/160,K/40,M/-80,N/160,E/-90,I/170,F/-20 ,G/50}
% 				\fill[black] (\p)node[shift={(\q:3mm)}]{$\p$} circle (1.0pt);
% 				\draw pic[draw=black,angle radius=0.2cm] {right angle = H--N--G}	
% 				pic[draw=black,angle radius=0.2cm] {right angle = H--M--F}; 
% 		\end{tikzpicture}}
% 		\noindent
% 		Ta có $ABCD$ là nửa lục giác đều có tâm là trung điểm $K$ của $BC$. Do đó $ADCK$ và $ABND$ là hình thoi nên $AC\perp KD$. Mặt khác $AC\perp SD$ nên $AC\perp (SKD)\Rightarrow AC\perp SK$.\\
% 		Lại có $SK\perp BC$ (vì$\triangle SBC$ đều), suy ra $SK\perp (ABCD)\Rightarrow SK\perp KD$.\\
% 		Ta có $IG$ là giao tuyến của $(\alpha)$ với $(SAC)$, mà $AC\parallel (\alpha)$, suy ra $IG\parallel AC$.\\
% 		Mặt khác $HM\parallel SD$ và $SD\perp AC$, suy ra $HM\perp IG$  và $HM\perp EF$ và $IGEF$ là hình chữ nhật.
% 		Diện tích thiết diện $EFGHI$ bằng $S=S_{EFGI}+S_{HGI}=IG\cdot NM +\dfrac{1}{2}IG\cdot HN$. \\
% 		Ta có $AK=KD=AD=a$ nên $\triangle AKD$ đều.\\
% 		Mà $BD\perp AK$, $AC\perp KD$ nên $O$ là trọng tâm tam giác $\triangle ADK$. Suy ra $OD=\dfrac{2}{3}\cdot\dfrac{a\sqrt 3 }{2}=\dfrac{a\sqrt 3}{3}$.\\
% 		$AC=BD=a\sqrt{3}$ ($\triangle BAC$ vuông tại $A$, do $KA=KB=KC$).\\
% 		$SD=\sqrt {SK^2+KD^2}=2a$.\\
% 		Ta có $\dfrac{DM}{DO}=\dfrac{EF}{AC}\Rightarrow EF=\dfrac{DM}{DO}\cdot AC=\dfrac{x}{\frac{a\sqrt 3 }{3}}\cdot a\sqrt 3=3x$.\\
% 		$\dfrac{GF}{SD}=\dfrac{CF}{CD}=\dfrac{OM}{OD}\Rightarrow GF=\dfrac{OM}{OD}\cdot SD=\dfrac{\frac{a\sqrt 3 }{3}- x}{\frac{a\sqrt 3 }{3}}\cdot 2a=2a-2\sqrt 3x$.\\
% 		$\dfrac{HM}{SD}=\dfrac{BM}{BD}\Rightarrow HM=\dfrac{BM}{BD}\cdot SD=\dfrac{a\sqrt 3-x}{a\sqrt 3}\cdot 2a=\dfrac{6a-2x\sqrt 3 }{3}$.\\
% 		Suy ra $HN=HM-NM=HN-GF=\dfrac{6a-2x\sqrt 3}{3}-\left( 2a-2\sqrt 3x \right) =\dfrac{4x\sqrt 3}{3}$.\\
% 		Vậy $S=\dfrac{1}{2}\cdot\dfrac{4x\sqrt 3}{3}\cdot 3x+\left( 2a-2\sqrt 3x\right)\cdot 3x=-4\sqrt 3x^2+6ax=-\sqrt 3 \left( 2x - \dfrac{a\sqrt 3 }{2} \right)^2 + \dfrac{3a^2\sqrt 3 }{4}$.\\
% 		Suy ra $S \leq \dfrac{3a^2\sqrt 3}{4}$. Dấu “=” xảy ra khi và chỉ khi $2x-\dfrac{a\sqrt 3}{2}\Leftrightarrow x=\dfrac{a\sqrt 3 }{4}$.
% 	}
% \end{ex}
% \Closesolutionfile{ans}
% % \begin{indapan}{10}
% % 	{ans/ans-1K7-2-5}
% % \end{indapan}

% \begin{dang}{Bài toán thực tế}
% \end{dang}
% \subsubsection{Ví dụ mẫu}
% \begin{vd}[TH]%[1T8T2-5]
% 	\immini{
% 		Một cái lều có dạng hình lăng trụ $ABC.A' B'C'$ có cạnh bên $AA'$ vuông góc với đáy (Hình bên). Cho biết $AB=AC=2{,}4$ m; $BC=2$ m; $AA'=3$ m.
% 		\begin{enumerate}
% 			\item Tính góc giữa hai đường thẳng $AA'$ và $BC$; $A'B'$ và $AC$.
% 			\item Tính diện tích hình chiếu vuông góc của tam giác $ABB'$ trên mặt phẳng $\left(BB'C'C\right)$.
% 		\end{enumerate}
% 	}
% 	{
% 		\begin{tikzpicture}[scale=0.7, font=\footnotesize, line join=round, line cap=round, >=stealth]
% 			\coordinate (A) at (0,3);
% 			\coordinate (B) at (-0.5,0);
% 			\coordinate (C) at (1,1);
% 			\coordinate (A') at ($(A) + (-4,0.5)$);
% 			\coordinate (B') at ($(B) + (-4,0.5)$);
% 			\coordinate (C') at ($(C) + (-4,0.5)$);
% 			\draw(A)--(B)--(C)--cycle (A)--(A') (A')--(B') (B)--(B');
% 			\draw[dashed,thin] (A')--(C')--(B') (C')--(C);
% 			\foreach \i/\g in {A/90,B/-90,C/-90,A'/180,B'/180,C'/60}{\draw[fill=black](\i) circle (1.5pt) ($(\i)+(\g:3mm)$) node[scale=1]{$\i$};}
% 		\end{tikzpicture}
% 	}
% 	\dapso($(AA',BC)=90^\circ$, $(AB,AC)\approx 49^\circ$, $\dfrac{3}{2}$ m$^2$)
% 	\loigiai{\immini{
% 			\begin{enumerate}
% 				\item Ta có $AA'\perp (ABC) \Rightarrow AA'\perp BC \Rightarrow (AA',BC)=90^\circ$.\\
% 				Do $A'B' \parallel AB \Rightarrow (A'B',AC)=(AB,AC)=\widehat{BAC}$.\\
% 				Áp dụng hệ quả định lý Côsin $$\cos \widehat{BAC}=\dfrac{AC^2+AB^2-BC^2}{2\cdot AB \cdot AC}=\dfrac{2{,}4^2+2{,}4^2-2^2}{2\cdot 2{,}4 \cdot 2{,}4}=0{,}652.$$
% 				Suy ra $\widehat{BAC} \approx 49^\circ$.
% 				\item Gọi $H$ là hình chiếu vuông góc của $A$ lên $BC$.\\
% 				Ta có $AH \perp CB \quad (1)$.\\
% 				Mặt khác $BB' \perp AH \quad (2)$ do $BB'\parallel AA'$ mà $AA' \perp (ABC)$.\\
% 				Từ $(1)$ và $(2)$ suy ra $AH \perp (BB'C'C)$. Do đó, $H$ là hình chiếu của $A$ lên mặt phẳng $(BB'C'C)$.\\
% 				Khi đó hình chiếu  vuông góc của tam giác $ABB'$ trên mặt phẳng $\left(BB'C'C\right)$ là tam giác $BHB'$.\\
% 				Vậy $S_{\triangle BHB'}=\dfrac{1}{2} \cdot BB' \cdot BH=\dfrac{1}{2} \cdot BB' \cdot \dfrac{1}{2}BC=\dfrac{1}{2} \cdot 3 \cdot \dfrac{1}{2}\cdot 2=\dfrac{3}{2}$ m$^2$.
% 			\end{enumerate}
% 		}
% 		{
% 			\begin{tikzpicture}[scale=0.7, font=\footnotesize, line join=round, line cap=round, >=stealth]
% 				\coordinate (A) at (0,3);
% 				\coordinate (B) at (-0.5,0);
% 				\coordinate (C) at (1,1);
% 				\coordinate (A') at ($(A) + (-4,0.5)$);
% 				\coordinate (B') at ($(B) + (-4,0.5)$);
% 				\coordinate (C') at ($(C) + (-4,0.5)$);
% 				\coordinate (H) at ($(B)!0.5!(C)$);
% 				\draw(A)--(B)--(C)--cycle (A)--(A') (A')--(B') (B)--(B') (A)--(H) (A)--(B');
% 				\draw[dashed,thin] (A')--(C')--(B') (C')--(C) (B')--(H);
% 				\foreach \i/\g in {A/90,B/-90,C/-90,A'/180,B'/180,C'/60,H/-45}{\draw[fill=black](\i) circle (1.5pt) ($(\i)+(\g:3mm)$) node[scale=1]{$\i$};}
% 				\draw pic[draw,angle radius=0.2cm]{right angle=A--H--C};
% 			\end{tikzpicture}
% 		}
% 	}
	
% \end{vd}
% \subsubsection{Bài tập rèn luyện}
% \centerline{\fcolorbox{red}{yellow!50}{\bf {BÀI TẬP TỰ LUẬN}}}
% \begin{bt}
% 	Bạn Vinh thả quả dọi chìm vào thùng nước. Hỏi khi dây dọi căng và mặt nước yên lặng thì đường thẳng chứa dây dọi có vuông góc với mặt phẳng chứa mặt nước trong thùng hay không?
% 	\loigiai{Đường thẳng chứa dây dọi vuông góc với mặt phẳng chứa nước trong thùng vì mặt nước lúc yên lặng là một mặt phẳng nằm ngang.}
% \end{bt}
% \begin{bt}
% 	\immini{
% 		Một cột bóng rổ được dựng trên một sân phẳng. Bạn Hùng đo khoảng cách từ một điểm trên sân, cách chân cột $1$ m đến một điểm trên cột, cách chân cột $1$ m được kết quả là $1{,}5\mathrm{~m}$ ($\text{H.7.27}$). Nếu phép đo của Hùng là chính xác thì cột có vuông góc với sân hay không? Có thể kết luận rằng cột không có phương thẳng đứng hay không?
% 	}{\begin{tikzpicture}[scale=0.9, font=\footnotesize, line join=round, line cap=round, >=stealth]			
% 			\path 
% 			(0,0) coordinate (A) 
% 			(0,4) coordinate (S)
% 			(-1.5,4) coordinate (M)
% 			(1.5,4) coordinate (N)
% 			(1.5,6) coordinate (P)
% 			(-1.5,6) coordinate (Q)
% 			(-2,0) coordinate (B) 
% 			(0,2) coordinate (C);
			
% 			\draw [line width=3pt] (S)--(A) ;
% 			\draw [line width=2pt, blue]	  (M)--(N)--(P)--(Q)--cycle ;
% 			\draw  (A)--(C) node[midway,right]{$1$ m}  (B)--(A) node[midway,below]{$1$ m};
% 			\draw [dashed] (B)--(C) node[midway,sloped,above]{$1 {,}5$ m} ;
% 			\node [below] at (current bounding box.south){\it Hình $7.27$};  	
			
% 			\foreach \x/\g in
% 			{A/180,B/0, C/0}
% 			\fill[red](\x) circle (1pt)
% 			($(\x)+(\g:3mm)$) ;
% 	\end{tikzpicture}}
% 	\loigiai{
% 		\immini{Ta có $1^2+1^2=2 \neq 1{,} 5^2$ hay $AB^2+AC^2 \neq BC^2$.\\ Do đó tam giác $ABC$ không vuông tại $A$. Suy ra $AC$ không vuông góc với $AB$.\\
% 			Vậy cột không vuông góc với sân.\\
% 			Ta không thể kết luận cột không có phương thẳng đứng vì mặt sân phẳng chưa chắc nằm ngang (có thể mặt sân không vuông góc với phương thẳng đứng tại điểm đặt chân trụ).
% 		}{
% 			\begin{tikzpicture}[scale=0.9, font=\footnotesize, line join=round, line cap=round, >=stealth]				
% 				\path 
% 				(0,0) coordinate (A) 
% 				(0,4) coordinate (S)
% 				(-1.5,4) coordinate (M)
% 				(1.5,4) coordinate (N)
% 				(1.5,6) coordinate (P)
% 				(-1.5,6) coordinate (Q)
% 				(-2,0) coordinate (B) 
% 				(0,2) coordinate (C);				
% 				\draw [line width=3pt] (S)--(A) ;
% 				\draw [line width=2pt, blue]  (M)--(N)--(P)--(Q)--cycle ;
% 				\draw  (A)--(C) node[midway,right]{$1$ m}  (B)--(A) node[midway,below]{$1$ m};
% 				\draw [dashed] (B)--(C) node[midway,sloped,above]{$1 {,}5$ m};				
% 				\foreach \x/\g in
% 				{A/0,B/180, C/0}
% 				\fill[black](\x) circle (1pt)
% 				($(\x)+(\g:3mm)$) node{\x} ;
% 		\end{tikzpicture}}
% 	}
% \end{bt}
% \begin{bt}%[1K7KN-4]
% 	Trong một khoảng thời gian đầu kể từ khi cất cánh, máy bay bay theo một đường thẳng. Góc cất cánh của nó là góc giữa đường thẳng đó và mặt phẳng nằm ngang nơi cất cánh. Hai máy bay cất cánh và bay thẳng với cùng độ lớn vận tốc trong 5 phút đầu, với các góc cất cánh lần lượt là $10^\circ, 15^\circ$. Hỏi sau $ 1 $ phút kể từ khi cất cánh, máy bay nào ở độ cao so với mặt đất (phẳng, nằm ngang) lớn hơn?
% 	\begin{note}
% 		\textbf{Chú ý}. Độ cao của máy bay so với mặt đất là khoảng cách từ máy bay (coi là một điểm) đến hình chiếu của nó trên mặt đất.
% 	\end{note}
% 	\loigiai{Sau $1$ phút kể từ khi cất cánh, máy bay với góc cất cánh $15^\circ$ ở độ cao so với mặt đất (phẳng, nằm ngang) lớn hơn}
% \end{bt}
% \begin{bt}%[1K7GN-4]
% 	Hãy nêu cách đo góc giữa đường thẳng chứa tia sáng mặt trời và mặt phẳng nằm ngang tại một vị trí và một thời điểm.
% 	\begin{note}
% 		\textbf{Chú ý}. Góc giữa đường thẳng chứa tia sáng mặt trời lúc giữa trưa với mặt phẳng nằm ngang tại vị trí đó được gọi là góc Mặt Trời. Giữa trưa là thời điểm ban ngày mà tâm Mặt Trời thuộc mặt phẳng chứa kinh tuyến đi qua điểm đang xét. Góc Mặt Trời ảnh hưởng tới sự hấp thụ nhiệt từ Mặt Trời của Trái Đất, tạo nên các mùa trong năm trên Trái Đất.
% 	\end{note}
% \end{bt}
% \begin{bt}%[TeX hóa SGK 11 KNTT]%[1K7KN-4]
% 	Trên mặt đất phẳng, người ta dựng một cây cột $AB$ có chiều dài bằng $10$ m và tạo với mặt đất góc $80^{\circ}$. Tại một thời điểm dưới ánh sáng mặt trời, bóng $BC$ của cây cột trên mặt đất dài $12$ m vào tạo với cây cột một góc bằng $120^{\circ}$ (tức là $\widehat{ABC}=120^{\circ}$). Tính góc giữa mặt đất và đường thẳng chứa tia sáng mặt trời tại thời điểm nói trên.
% 	\dapso{$\approx 31^\circ$}
% 	\loigiai{
% 		\immini{
% 			Gọi $H$ là hình chiếu vuông góc của $A$ trên mặt đất. Theo giả thiết ta có $\widehat{ABH}=80^\circ$. Đường thẳng $BC$ là bóng của $AB$ nên đường thẳng chứa tia sáng là $AC$. Khi đó góc giữa mặt đất và đường thẳng chứa tia sáng là góc $\widehat{ACH}$.\\
% 			Xét $\triangle ABH$ vuông tại $H,$ có 
% 			$$AH=AB\cdot \sin \widehat{ABH}=10\cdot \sin 80^\circ.$$ 
% 		}{
% 			\begin{tikzpicture}[line join=round,line cap=round,>=stealth,font=\footnotesize,scale=.8]
% 				\coordinate[label=left:$H$] (H) at (0,0);
% 				\coordinate[label=below left:$B$] (B) at (1.5,-1.5);
% 				\coordinate[label=right:$C$] (C) at (6,0);
% 				\coordinate[label=above left:$A$] (A) at ($(H)+(90:3.5)$);
% 				\draw (H)--(B)--(C)--(A)--cycle (A)--(B);
% 				\draw[dashed] (H)--(C);
% 		\end{tikzpicture}}
% 		\noindent
% 		Xét $\triangle ABC,$ có 
% 		$$AC^2=AB^2+BC^2-2AB\cdot BC\cos\widehat{ABC}=364 \Rightarrow AC=2\sqrt{91}.$$
% 		Xét $\triangle AHC$ vuông tại $H$, có 
% 		$$\sin\widehat{ACH}=\dfrac{AH}{AC}=\dfrac{10\cdot \sin 80^\circ}{2\sqrt{91}}=\dfrac{5\cdot \sin 80^\circ}{\sqrt{91}}.$$
% 		Suy ra $\widehat{ACH}\approx 31^\circ$.
% 	} 
% \end{bt}
% \begin{bt}%[Bài toán đo chiều cao của tháp khi không thể lên tới đỉnh tháp]%[1K7KN-4]
% 	\immini
% 	{
% 		Để ước lượng chiều cao của tháp khi không thể lên tới đỉnh tháp, người ta đo góc giữa tia nắng chiếu qua đỉnh tháp và mặt đất, đo chiều dài của bóng tháp trên mặt đất, từ đó ước lượng được chiều cao của tháp. Giả sử khi tia nắng tạo với mặt đất một góc $40^\circ$, chiều dài của bóng tháp là $80$ m (Hình bên). Tính chiều cao của tháp theo đơn vị mét (làm tròn kết quả đến hàng phần mười).
% 	}
% 	{
		
% 		\begin{tikzpicture}[scale=0.7, font=\footnotesize, line join=round, line cap=round, >=stealth]
% 			\tikzset{Icon-mattroi/.pic={
% 					\def\r{0.4}
% 					\fill(0,0)circle(\r);
% 					\draw[line width=1pt](0,0)circle(\r);
% 					\foreach \g in{1,...,10}{
% 						\draw[line width=1 pt](\g*36:1.2*\r)++(\g*36:0.1*\r)--++(\g*36:\r*0.25);
% 					}
% 			}}
% 			\path
% 			(0,6) coordinate (I)
% 			(-7,0.3) coordinate (M)
% 			(0,0.3) coordinate (O)
% 			(1,7)pic[yellow,scale=0.5]{Icon-mattroi}
% 			;
% 			\def\originpath{(0.5,0)--(0.5,1)--(0.4,1.2)--(0.4,4)--(0.5,4)--(0.5,5)--(0.2,5.4)--(0.2,5.7)--(0,6)}
% 			\fill[brown] (0.2,5.7)--(0,6)--(-0.2,5.7)--cycle
% 			(0.2,5.4)--(0.5,5)--(-0.5,5)--(-0.2,5.4)--cycle
% 			;
% 			\fill[brown!50] (0.2,5.7)--(0.2,5.4)--(-0.2,5.4)--(-0.2,5.7)--cycle
% 			(0.5,5)--(0.5,4)--(-0.5,4)--(-0.5,5)--cycle
% 			(0.5,0)--(0.5,1)--(0.4,1.2)--(0.4,4)--(-0.4,4)--(-0.4,1.2)--(-0.5,1)--(-0.5,0)--cycle
% 			;
% 			\fill[brown] (-0.3,4.8)--(0.3,4.8)--(0.3,4.2)--(-0.3,4.2)--cycle
% 			;
% 			\fill[brown!70]
% 			(0.4,1.2)--(0.4,4)--(-0.4,4)--(-0.4,1.2)--cycle (0.5,0)--(0.5,1)--(-0.5,1)--(-0.5,0)--cycle;
% 			\fill[white] (0,4.5)circle (6pt);
% 			\fill[gray!50] (-0.5,0.8)--(-1,0.8)--(-1.2,0.7)--(-4,0.7)--(-4.2,0.8)--(-5,0.8)--(-5.5,0.6)--(-6,0.5)--(-7,0.3)--(-6,0.1)--(-5,-0.1)--(-4.5,-0.1)--(-3.5,-0.1)--(-3,0.1)--(-1,0.1)--(-0.8,0)--(-0.5,0)--(-0.5,0.8);
% 			\draw[brown,line width=2pt] (-0.2,4)--(-0.2,0) (0,4)--(0,0) (0.2,4)--(0.2,0) (-0.5,0)--(0.5,0) ;
% 			\draw[blue,line width=2pt] (I)--(M) (M)--(O)node[pos=0.5,below]{$80$ m};
% 			\draw[brown] \originpath;
% 			% đối xứng qua trục tung
% 			\draw[brown,xscale=-1] \originpath;
% 		\end{tikzpicture}
% 	}
% 	\dapso{$\approx 67{,}1$ m}
% 	\loigiai{
% 		\immini{
% 			Xét hình bên, độ dài $AH$ chỉ chiều cao của tháp, độ dài $OH$ chỉ chiều dài của bóng tháp, độ lớn của góc $AOH$ chỉ số đo góc giữa tia nắng và mặt đất. Vì tam giác $OAH$ vuông tại $H$ nên
% 			$$
% 			AH=OH \cdot \tan \widehat{AOH}=80 \cdot \tan 40^\circ \approx 67{,}1(\mathrm{m}).
% 			$$
% 		}{
% 			\begin{tikzpicture}[scale=0.7, font=\footnotesize, line join=round, line cap=round, >=stealth]
% 				\path
% 				(0,6) coordinate (A)
% 				(-7,0.3) coordinate (O)
% 				(0,0.3) coordinate (H);
% 				\draw (A)--(O) (A)--(H) (H)--(O)node[pos=0.5,below]{$80$ m};
% 				\draw
% 				pic[draw,angle radius=5]{right angle=A--H--O};
% 				%				\tkzMarkAngle[mark=](H,O,A)
% 				%				\tkzLabelAngle[pos=1.5](H,O,A){$40^\circ$}
% 				\foreach \x/\g in {A/90,H/0,O/180} \fill[black] (\x) circle (2pt)+(\g:0.6) node{$\x$};
% 			\end{tikzpicture}
% 		}
% 	}
% \end{bt}