\chapter{Các quy tắc tính xác suất}
\section{Biến cố hợp, biến cố giao, biến cố độc lập}
\subsection{Tóm tắt lí thuyết}
\begin{tomtat}
	\subsubsection{Biến cố hợp}
	\begin{dn}
		Cho $A$ và $B$ là hai biến cố. Biến cố: \lq\lq $A$ hoặc $B$ xảy ra\rq\rq\ được gọi là biến cố hợp của $A$ và $B$, kí hiệu là $A \cup B$.\\
		Biến cố hợp của $A$ và $B$ là tập con $A \cup B$ của không gian mẫu $\Omega$.
	\end{dn}
	\subsubsection{Biến cố giao}
	\begin{dn}
		Cho $A$ và $B$ là hai biến cố. Biến cố: \lq\lq Cả $A$ và $B$ đều xảy ra\rq\rq\ được gọi là biến cố giao của $A$ và $B$, kí hiệu là $A B$.\\
		Biến cố giao của $A$ và $B$ là tập con $A \cap B$ của không gian mẫu $\Omega$.
	\end{dn}
	\subsubsection{Biến cố độc lập}
	\begin{dn}
		Cặp biến cố $A$ và $B$ được gọi là độc lập nếu việc xảy ra hay không xảy ra của biến cố này không ảnh hưởng tới xác suất xảy ra của biến cố kia. 
	\end{dn}
	\begin{note}
		Nếu cặp biến cố $A$ và $B$ độc lập thì các cặp biến cố: $A$ và $\bar{B}$, $\bar{A}$ và $B$, $\bar{A}$ và $\bar{B}$ cũng độc lập.
	\end{note}
\end{tomtat}
\subsection{Các dạng toán thường gặp}
\begin{dang}{Mô tả không gian mẫu, biến cố}
	Nội dung và phương pháp giải
\end{dang}
\subsubsection{Ví dụ mẫu}
\begin{vd}%[1K8BR-2]
	Một tổ trong lớp 11C có $9$ học sinh. Phỏng vấn $9$ bạn này với câu hỏi: \lq\lq Bạn có biết chơi môn thể thao nào trong hai môn này không?\rq\rq. Nếu biết thì đánh dấu X vào ô ghi tên môn thể thao đó, không biết thì để trống. Kết quả thu được như sau:
	\begin{center}
		\begin{tabular}{|c|c|c|}
			\hline
			\diagbox{Tên học sinh}{Môn thể thao} & Cầu lông & Bóng bàn \\
			\hline
			Bảo & X & \\
			\hline
			Đăng & X & X \\
			\hline
			Giang & & \\
			\hline
			Hoa & X & X \\
			\hline
			Long & & \\
			\hline
			Mai & X & X \\
			\hline
			Phúc & X & X \\
			\hline
			Tuấn & X & \\
			\hline
			Yến & & \\
			\hline
		\end{tabular}
	\end{center}
	Chọn ngẫu nhiên một học sinh trong tổ. Xét các biến cố sau:\\	
	$U$: \lq\lq Học sinh được chọn biết chơi cầu lông\rq\rq;\\
	$V$: \lq\lq Học sinh được chọn biết chơi bóng bàn\rq\rq.
	\begin{enumerate}
		\item Mô tả không gian mẫu.
		\item Nội dung của biến cố giao $T=U V$ là gì? Mỗi biến cố $U$, $V$, $T$ là tập con nào của không gian mẫu?
	\end{enumerate}
	\loigiai{
		\begin{enumerate}
			\item Không gian mẫu $\Omega=\{\text{Bảo; Đăng; Giang; Hoa; Long; Mai; Phúc; Tuấn; Yến} \}$.
			\item $T$ là biến cố \lq\lq Học sinh được chọn biết chơi cả cầu lông và bóng bàn\rq\rq.\\	
			Ta có: $U=\{\text{Bảo; Đăng; Hoa; Mai; Phúc; Tuấn;}\}$;\\ $V=\{\text{Đăng; Hoa; Mai; Phúc}\}$.\\
			Vậy $T=U \cap V=\{\text{Đăng; Hoa; Mai; Phúc}\}$.
		\end{enumerate}
	}
\end{vd}
%%==========Ví dụ 4
\begin{vd}%[1K8BR-2]
	Một hộp đựng $25$ tấm thẻ cùng loại được đánh số từ $1$ đến $25$. Rút ngẫu nhiên một tấm thẻ trong hộp. Xét các biến cố $P$ : \lq\lq Số ghi trên tấm thẻ là số chia hết cho $4$\rq\rq; $Q$: \lq\lq Số ghi trên tấm thẻ là số chia hết cho $6$\rq\rq.
	\begin{enumerate}
		\item Mô tả không gian mẫu.
		\item Nội dung của biến cố giao $S=P Q$ là gì? Mỗi biến cố $P$, $Q$, $S$ là tập con nào của không gian mẫu?
	\end{enumerate} 
	\loigiai{
		\begin{enumerate}
			\item $\Omega=\{1,2,3,4,5,6,7,8,9,10,11,12,13,14,15,16,17,18,19,20,21,22,23,24,25\}$.
			\item $S$ là biến cố \lq\lq Số ghi trên tấm thẻ là số chia hết cho $4$ và chia hết cho $6$\rq\rq.\\	
			Ta có: $P=\{4,8,12,16,20,24\}$; $Q=\{6,12,18,24\}$.\\
			Vậy $T=P \cap Q=\{12;24\}$.
		\end{enumerate} 
	}
\end{vd}
\begin{vd}%[1K8BR-2]
	Gieo hai con xúc xắc cân đối và đồng chất. Gọi $A$ là biến cố \lq\lq Tổng số chấm xuất hiện trên hai con xúc xắc bằng 5\rq\rq, $B$ là biến cố \lq\lq Tích số chấm xuất hiện trên hai con xúc xắc bằng 6\rq\rq. Gọi $C$ là biến cố \lq\lq Có ít nhất một con xúc xắc xuất hiện mặt $1$ chấm\rq\rq. Hãy viết tập hợp mô tả các biến cố giao $AC$ và $BC$.
	\loigiai{
		Biến cố $A=\{(1 ; 4) ;(4 ; 1) ;(2 ; 3) ;(3 ; 2)\}$.\\
		Biến cố $B=\{(1 ; 6) ;(6 ; 1) ;(2 ; 3) ;(3 ; 2)\}$.\\
		Biến cố $C=\{(1 ; 6) ;(6 ; 1) ;(1 ; 5) ;(5 ; 1) ;(1 ; 4) ;(4 ; 1) ;(1 ; 3) ;(3 ; 1) ;(1 ; 2) ;(2 ; 1) ;(1 ; 1)\}$.\\
		Kết hợp tập hợp mô tả biến cố $A$, $B$, $C$ ở trên, ta có biến cố $AC=\{(1 ; 4) ;(4 ; 1)\};$  $B C=\{(1 ; 6) ;(6 ; 1)\}.$
		
	}
\end{vd}
\subsubsection{Bài tập rèn luyện}
%%==========Bài 1
\begin{bt}%[1K8BR-2]
	Một hộp đựng $15$ tấm thẻ cùng loại được đánh số từ $1$ đến $15$. Rút ngẫu nhiên một tấm thẻ và quan sát số ghi trên thẻ. Gọi $A$ là biến cố \lq\lq Số ghi trên tấm thẻ nhỏ hơn $7$\rq\rq; $B$ là biến cố \lq\lq Số ghi trên tấm thẻ là số nguyên tố\rq\rq.
	\begin{enumerate}
		\item Mô tả không gian mẫu.
		\item Mỗi biến cố $A \cup B$ và $A B$ là tập con nào của không gian mẫu?
	\end{enumerate}
	\loigiai{
		\begin{enumerate}
			\item Khi rút 1 thẻ từ 15 thẻ được đánh số thì không gian mẫu là: \\
			$\Omega= \{1;2;3;4;5;6;7;8;9;10;11;12;13;14;15 \}$.
			\item 
			Các kết quả thuận lợi của biến cố $A$: $A= \{1;2;3;4;5;6 \}$.\\
			Các kết quả thuận lợi của biến cố $B$: $B=\{2;3;5;7;11;13 \}$.\\
			$A \cup B$ là biến cố \lq\lq Số ghi trên thẻ nhỏ hơn $7$ hoặc số nguyên tố\rq\rq.\\
			$A \cup B = \{1;2;3;4;5;6;7;11;13 \}$.\\
			$AB$ là biến cố \lq\lq Số ghi trên thẻ nhỏ hơn $7$ và nguyên tố\rq\rq.\\
			$AB=\{2;3;5 \}$.
		\end{enumerate}
	}
\end{bt}
\begin{bt}%[1K8BR-2]
	Gieo hai con xúc xắc cân đối và đồng chất. Gọi $A$ là biến cố \lq\lq Tổng số chấm xuất hiện trên hai con xúc xắc bằng 5\rq\rq, $B$ là biến cố \lq\lq Tích số chấm xuất hiện trên hai con xúc xắc bằng 6\rq\rq.
	\begin{enumerate}
		\item Hãy viết tập hợp mô tả các biến cố trên.
		\item Hãy liệt kê các kết quả của phép thử làm cho cả hai biến cố $A$ và $B$ cùng xảy ra.
	\end{enumerate}
	\loigiai{
		\begin{enumerate}
			\item 	Biến cố $A=\{(1 ; 4) ;(4 ; 1) ;(2 ; 3) ;(3 ; 2)\}$.\\
			Biến cố $B=\{(1 ; 6) ;(6 ; 1) ;(2 ; 3) ;(3 ; 2)\}$.
			\item Kết quả của phép thử làm cho cả hai biến cố $A$ và $B$ cùng xảy ra là $\{(2 ; 3) ;(3 ; 2)\}$.
		\end{enumerate}
	}
\end{bt}
\begin{bt}%[1K8BR-2]
	Gieo hai con xúc xắc cân đối và đồng chất. Gọi $A$ là biến cố \lq\lq Tổng số chấm xuất hiện trên hai con xúc xắc bằng 5\rq\rq, $B$ là biến cố \lq\lq Tích số chấm xuất hiện trên hai con xúc xắc bằng 6\rq\rq. Gọi $C$ là biến cố \lq\lq Có ít nhất một con xúc xắc xuất hiện mặt $1$ chấm\rq\rq.
	\begin{enumerate}
		\item  Gọi $D$ là biến cố \lq\lq Số chấm xuất hiện trên con xúc xắc thứ nhất là 3\rq\rq. Hãy xác định các biến cố $AD$, $BD$ và $CD$.
		\item Gọi $\bar{A}$ là biến cố đối của biến cố $A$. Hãy viết tập hợp mô tả các biến cố giao $\bar{A}B$ và $\bar{A}C$.
	\end{enumerate}
	\loigiai{
		\begin{enumerate}
			\item  Biến cố $D=\{(3 ; 1) ;(3 ; 2) ;(3 ; 3) ;(3 ; 4); (3 ; 5); (3 ; 6)\}$.\\
			Biến cố $AD=\{(3 ; 2)\}$.\\
			Biến cố $BD=\{(3 ; 2) \}$.\\
			Biến cố $CD=\{(3 ; 1) \}$.
			\item Gọi $\bar{A}$ là biến cố đối của biến cố $A$. \\
			Biến cố giao $\bar{A}B=\{(1 ; 6) ;(6 ; 1)\}$.\\
			Biến cố giao $\bar{A}C=\{(1 ; 6) ;(6 ; 1) ;(1 ; 5) ;(5 ; 1) ;(1 ; 3) ;(3 ; 1) ;(1 ; 2) ;(2 ; 1) ;(1 ; 1)\}$.
		\end{enumerate}
	}
\end{bt}

\begin{dang}{Xác định các loại biến cố}% [1K8?R-3] 
	Nội dung và phương pháp giải
\end{dang}
\subsubsection{Ví dụ mẫu}
\begin{vd}%[1K8BR-3]
	Một hộp đựng $15$ tấm thẻ cùng loại được đánh số từ $1$ đến $15$. Rút ngẫu nhiên một tấm thẻ trong hộp. Gọi $E$ là biến cố \lq\lq Số ghi trên tấm thẻ là số lẻ\rq\rq; $F$ là biến cố \lq\lq Số ghi trên tấm thẻ là số nguyên tố\rq\rq.
	\begin{enumerate}
		\item Mô tả không gian mẫu.
		\item Nêu nội dung của biến cố hợp $G=E \cup F$. Hỏi $G$ là tập con nào của không gian mẫu?
	\end{enumerate}
	\loigiai{
		\begin{enumerate}
			\item Không gian mẫu $\Omega=\{1 ; 2 ; 3 ; 4 ; 5 ; 6 ; 7 ; 8 ; 9 ; 10 ; 11 ; 12 ; 13 ; 14 ; 15\}$.
			\item $E \cup F$ là biến cố \lq\lq Số ghi trên tấm thẻ là số lẻ hoặc số nguyên tố\rq\rq.\\
			Ta có $E=\{1 ; 3 ; 5 ; 7 ; 9 ; 11 ; 13 ; 15\} ; F=\{2 ; 3 ; 5 ; 7 ; 11 ; 13\}$.\\
			Vậy $G=E \cup F=\{1 ; 2 ; 3 ; 5 ; 7 ; 9 ; 11 ; 13 ; 15\}$.
		\end{enumerate}
	}
\end{vd}
%%==========Ví dụ 2
\begin{vd}%[1K8BR-3]
	Một tổ trong lớp 11B có $4$ học sinh nữ là Hương, Hồng, Dung, Phương và $5$ học sinh nam là Sơn, Tùng, Hoàng, Tiến, Hải. Trong giờ học, giáo viên chọn ngẫu nhiên một học sinh trong tổ đó lên bảng để kiểm tra bài.\\
	Xét các biến cố sau:\\
	$H$: \lq\lq Học sinh đó là một bạn nữ\rq\rq;\\
	$K$: \lq\lq Học sinh đó có tên bắt đầu là chữ cái $\mathrm{H}$\rq\rq.
	\begin{enumerate}
		\item Mô tả không gian mẫu.
		\item Nêu nội dung của biến cố hợp $M=H \cup K$. Mỗi biến cố $H, K, M$ là tập con nào của không gian mẫu?
	\end{enumerate}
	\loigiai{
		\begin{enumerate}
			\item $\Omega=\{\text{Hương, Hồng, Dung, Phương, Sơn, Tùng, Hoàng, Tiến, Hải}\}$.
			\item $M=H \cup K$ là biến cố \lq\lq Học sinh được chọn là một bạn nữ hoặc một bạn nam\rq\rq. \\
			Ta có $H=\{\text{Hương, Hồng, Dung, Phương}\}$;\\
			$K=\{\text{Sơn, Tùng, Hoàng, Tiến, Hải}\}$.\\
			Vậy $M=\Omega=\{\text{Hương, Hồng, Dung, Phương, Sơn, Tùng, Hoàng, Tiến, Hải}\}$.
		\end{enumerate}
	}
\end{vd}
\begin{vd}%[1K8KR-3]
	Một hộp đựng $4$ viên bi màu đỏ và $5$ viên bi màu xanh, có cùng kích thước và khối lượng.
	\begin{enumerate}
		\item Bạn Minh lấy ngẫu nhiên một viên bi, ghi lại màu của viên bi được lấy ra rồi trả lại viên bi vào hộp. Tiếp theo, bạn Hùng lấy ngẫu nhiên một viên bi từ hộp đó. Xét hai biến cố sau:\\	
		$A$: \lq\lq Minh lấy được viên bi màu đỏ\rq\rq;\\	
		$B$: \lq\lq Hùng lấy được viên bi màu xanh\rq\rq.\\	
		Chứng tỏ rằng hai biến cố $A$ và $B$ độc lập.
		\item Bạn Sơn lấy ngẫu nhiên một viên bi và không trả lại vào hộp. Tiếp theo, bạn Tùng lấy ngẫu nhiên một viên bi từ hộp đó. Xét hai biến cố sau:\\	
		$C$: \lq\lq Sơn lấy được viên bi màu đỏ\rq\rq;\\	
		$D$: \lq\lq Tùng lấy được viên bi màu xanh\rq\rq.\\	
		Chứng tỏ rằng hai biến cố $C$ và $D$ không độc lập.
	\end{enumerate}
	\loigiai{
		\begin{enumerate}
			\item Nếu $A$ xảy ra, tức là Minh lấy được viên bi màu đỏ. Vì Minh trả lại viên bi đã lấy vào hộp nên trong hộp có $4$ viên bi màu đỏ và $5$ viên bi màu xanh. Vậy $P(B)=\dfrac{5}{9}$.\\	
			Nếu $A$ không xảy ra, tức là Minh lấy được viên bi màu xanh. Vì Minh trả lại viên bi đã lấy vào hộp nên trong hộp vẫn có $4$ viên bi màu đỏ và $5$ viên bi màu xanh. Vậy $P(B)=\dfrac{5}{9}$.\\	
			Như vậy, xác suất xảy ra của biến cố $B$ không thay đổi bởi việc xảy ra hay không xảy ra của biến cố $A$.\\	
			Vì Hùng lấy sau Minh nên $P(A)=\dfrac{4}{9}$ dù biến cố $B$ xảy ra hay không xảy ra.\\	
			Vậy $A$ và $B$ độc lập.
			\item Nếu $C$ xảy ra, tức là Sơn lấy được viên bi màu đỏ. Vì Sơn không trả lại viên bi đó vào hộp nên trong hộp có $8$ viên bi với $3$ viên bi màu đỏ và $5$ viên bi màu xanh. Vậy $P(D)=\dfrac{5}{8}$. \\
			Nếu $C$ không xảy ra, tức là Sơn lấy được viên bi màu xanh. Vì Sơn không trả lại viên bi đã lấy vào hộp nên trong hộp có $4$ viên bi màu đỏ và $4$ viên bi màu xanh. Vậy $P(D)=\dfrac{4}{8}$. \\
			Như vậy, xác suất xảy ra của biến cố $D$ đã thay đổi phụ thuộc vào việc biến cố $C$ xảy ra hay không xảy ra. Do đó, hai biến cố $C$ và $D$ không độc lập.
		\end{enumerate}
	}
\end{vd}
\begin{vd}%[1K8BR-3]
	Một hộp có $5$ viên bi xanh, $4$ viên bi đỏ và $2$ viên bi vàng. Lấy ra ngẫu nhiên đồng thời $2$ viên bi từ hộp. Hãy xác định các cặp biến cố xung khắc trong các biến cố sau:\\
	$A$ : \lq\lq Hai viên bi lấy ra cùng màu xanh\rq\rq;\\
	$B$ : \lq\lq Hai viên bi lấy ra cùng màu đỏ\rq\rq;\\
	$C$ : \lq\lq Hai viên bi lấy ra cùng màu\rq\rq;\\
	$D$ : \lq\lq Hai viên bi lấy ra khác màu\rq\rq.
	\loigiai{
		\begin{enumerate}
			\item Ta có hai biến cố $A$ và $B$ xung khắc.
			\item Biến cố $C$ xảy ra khi lấy ra 2 viên bi xanh hoặc 2 viên bi đỏ hoặc 2 viên bi vàng. Khi lấy được 2 viên bi màu xanh thì biến cố $A$ và biến cố $C$ cùng xảy ra. Khi lấy được 2 viên bi màu đỏ thì biến cố $B$ và biến cố $C$ cùng xảy ra. Do đó biến cố $C$ không xung khắc với biến cố $A$ và biến cố $B$.
			\item Biến cố $D$ xảy ra khi lấy ra 1 viên bi xanh, 1 viên bi đỏ; hoặc 1 viên bi xanh, 1 viên bi vàng; hoặc 1 viên bi đỏ, 1 viên bi vàng. Do đó biến cố $D$ xung khắc với biến cố $A$, xung khắc với biến cố $B$ và xung khắc với biến cố $C$.
		\end{enumerate}
		Vậy có 4 cặp biến cố xung khắc là: $A$ và $B ; A$ và $D ; B$ và $D ; C$ và $D$.
	}
\end{vd}
\begin{vd}%[1K8BR-3]
	Trong hộp có 1 quả bóng xanh, 1 quả bóng đỏ, 1 quả bóng vàng. Lấy ra ngẫu nhiên 1 quả bóng, xem màu rồi trả lại hộp. Lặp lại phép thử trên 2 lần và gọi $A_k$ là biến cố quả bóng lấy ra lần thứ $k$ là bóng xanh $(k=1,2)$.
	\begin{enumerate}
		\item [a)] $A_1, A_2$ có là các biến cố độc lập không? Tại sao?
		\item [b)] Nếu trong mỗi phép thử trên ta không trả bóng lại hộp thì $A_1, A_2$ có là các biến cố độc lập không? Tại sao?
	\end{enumerate}
	\loigiai{
		\begin{enumerate}
			\item [a)] Nếu $A_1$ xảy ra thì sau khi trả lại quả bóng thứ nhất vào hộp, trong hộp có 1 quả bóng xanh, 1 quả bóng đỏ và 1 quả bóng vàng, do đó xác suất xảy ra $A_2$ là $\dfrac{1}{3}$.\\
			Ngược lại, nếu $A_1$ không xảy ra thì sau khi trả lại quả bóng thứ nhất vào hộp, trong hộp vẫn có 1 quả bóng xanh, 1 quả bóng đỏ và 1 quả bóng vàng, do đó xác suất xảy ra $A_2$ là $\dfrac{1}{3}$.\\
			Ta thấy khi $A_1$ xảy ra hay không xảy ra thì xác suất của biến cố $A_2$ luôn bằng $\dfrac{1}{3}$. Do quả bóng lấy ra lần thứ nhất được trả lại hộp nên biến cố $A_2$ xảy ra hay không xảy ra không ảnh hưởng đến xác suất xảy ra của $A_1$. \\
			Vậy $A_1$ và $A_2$ là hai biến cố độc lập.
			\item [b)] Giả sử quả bóng lấy ra lần đầu tiên không được trả lại hộp.\\
			Nếu $A_1$ xảy ra thì trước khi bốc quả bóng thứ hai, trong hộp có 1 quả bóng đỏ, 1 quả bóng vàng. Do đó xác suất xảy ra $A_2$ là $0$.\\
			Ngược lại, nếu $A_1$ không xảy ra thì trước khi bốc quả bóng thứ hai, trong hộp có 2 quả bóng,
			trong đó có đúng 1 quả bóng xanh. Do đó xác suất xảy ra $A_2$ là $\dfrac{1}{2}$.\\
			Ta thấy xác suất xảy ra của biến cố $A_2$ phụ thuộc vào sự xảy ra của $A_1$.\\
			Vậy $A_1$ và $A_2$ không là hai biến cố độc lập.
		\end{enumerate}
	}
\end{vd}
\begin{vd}%[1C5Y2-2]
	Trong hộp kín có $10$ quả bóng màu xanh và $8$ quả bóng màu đỏ, các quả bóng có kích thước và khối lượng giống nhau. Lấy ngẫu nhiên đồng thời $2$ quả bóng. Xét các biến cố:\\
	A: \lq\lq Hai quả bóng lấy ra có màu xanh\rq\rq;\\
	B: \lq\lq Hai quả bóng lấy ra có màu đỏ\rq\rq.\\
	Chọn phát biểu đúng trong những phát biểu sau đây?
	\begin{enumerate}
		\item Biến cố hợp của hai biến cố $A$ và $B$ là \lq\lq Hai quả bóng lấy ra có cùng màu đỏ hoặc cùng màu xanh\rq\rq;
		\item Biến cố hợp của hai biến cố $A$ và $B$ là \lq\lq Hai quả bóng lấy ra có màu khác nhau\rq\rq;
		\item Biến cố hợp của hai biến cố $A$ và $B$ là \lq\lq Hai quả bóng lấy ra có cùng màu\rq\rq.
	\end{enumerate}
	\loigiai{
		\begin{enumerate}
			\item Phát biểu đúng;
			\item Phát biểu sai;
			\item Phát biểu đúng.
		\end{enumerate}
	}
\end{vd}
\begin{vd}%[1C5Y2-1]
	Một hộp có $52$ chiếc thẻ cùng loại, mỗi thẻ được ghi một trong các số $1,2,3, \ldots, 52$; hai thẻ khác nhau thì ghi hai số khác nhau. Rút ngẫu nhiên $1$ chiếc thẻ trong hộp. Xét biến cố $A$: \lq\lq Số xuất hiện trên thẻ được rút ra là số chia hết cho 3\rq\rq \, và biến cố $B$: \lq\lq Số xuất hiện trên thẻ được rút ra là số lẻ\rq\rq. Viết các tập con của không gian mẫu tương ứng với các biến cố $A, B, A \cap B$.
	\loigiai{
		Ta có $A=\{3 ; 6 ; 9 ; 12 ; 15 ; \ldots ; 48 ; 51\}$;\\
		$B=\{4 ; 8 ; 12 ; 16 ; 20 ; \ldots ; 48 ; 52\}$;\\
		$A \cap B=\{12 ; 24 ; 36 ; 48\}$.
	}
\end{vd}
\begin{vd}%[1C5Y2-1]
	Tung một đồng xu cân đối và đồng chất hai lần liên tiếp. Xét các biến cố:\\
	A: \lq\lq Đồng xu xuất hiện mặt $S$ ở lần gieo thứ nhất\rq\rq.\\
	B: \lq\lq Đồng xu xuất hiện mặt $N$ ở lần gieo thứ nhất\rq\rq.\\
	Hai biến cố trên có xung khắc không?\\
	\loigiai{
		Ta thấy	$A=\{S S ; S N\} ; B=\{N S ; N N\}$.\\
		Suy ra $A \cap B=\varnothing$. Do đó, $A$ và $B$ là hai biến cố xung khắc.
	}
\end{vd}
\begin{vd}%[1C5Y2-1]
	Một hộp có $3$ quả bóng màu xanh, $4$ quả bóng màu đỏ; các quả bóng có kích thước và khối lượng như nhau. Lấy bóng ngẫu nhiên hai lần liên tiếp, trong đó mỗi lần lấy ngẫu nhiên một quả bóng trong hộp, ghi lại màu của quả bóng lấy ra và bỏ lại quả bóng đó vào hộp.\\
	Xét các biến cố:\\
	$A$: \lq\lq Quả bóng màu xanh được lấy ra ở lần thứ nhất\rq\rq;\\
	$B$: \lq\lq Quả bóng màu đỏ được lấy ra ở lần thứ hai\rq\rq.\\
	Hỏi
	\begin{enumerate}[a)]
		\item Hai biến cố $A$ và $B$ có độc lập không? Vì sao?
		\item Hai biến cố $A$ và $B$ có xung khắc không? Vì sao?
	\end{enumerate}
	\loigiai{
		\begin{enumerate}[a)]
			\item Trước hết, biến cố $B$ xảy ra sau biến cố $A$ nên việc xảy ra hay không xảy ra của biến cố $B$ không làm ảnh hưởng đến xác suất xảy ra của biến cố $A$.\\
			Mặt khác, ta có xác suất của biến cố $B$ khi biến cố $A$ xảy ra bằng $\dfrac{4}{7}$; xác suất của biến cố $B$ khi biến cố $A$ không xảy ra cũng bằng $\dfrac{4}{7}$.\\
			Do đó việc xảy ra hay không xảy ra của biến cố $A$ không làm ảnh hưởng đến xác suất xảy ra của biến cố $B$.\\
			Vậy hai biến cố $A$ và $B$ là độc lập.
			\item Ta thấy kết quả (xanh; đỏ) là kết quả thuận lợi cho cả hai biến cố $A$ và $B$. Vì thế $A$ và $B$ không là hai biến cố xung khắc.
	\end{enumerate}}		
\end{vd}
\subsubsection{Bài tập rèn luyện}
%%==========Bài 2
\begin{bt}%[1K8KR-3]
	Gieo hai con xúc xắc cân đối, đồng chất. Xét các biến cố sau:\\
	$E$: \lq\lq Số chấm xuất hiện trên hai con xúc xắc đều là số chẵn\rq\rq;\\
	$F$: \lq\lq Số chấm xuất hiện trên hai con xúc xắc khác tính chẵn lẻ\rq\rq;\\
	$K$: \lq\lq Tích số chấm xuất hiện trên hai con xúc xắc là số chẵn\rq\rq.\\
	Chứng minh rằng $K$ là biến cố hợp của $E$ và $F$.
	\loigiai{
		Để tích của số chấm xuất hiện trên hai con xúc xắc là số chẵn thì có 2 trường hợp xảy ra.\\
		TH1: 1 con xúc xắc xuất hiện mặt chẵn, con còn lại xuất hiện mặt lẻ.\\
		Khi đó số chấm xuất hiện trên hai con xúc xắc khác tính chẵn lẻ.\\
		TH2: 2 con xúc xắc xuất hiện mặt chẵn.\\
		Do đó $K$ là biến cố hợp của $E$ và $F$.
	}
\end{bt}

%%==========Bài 3
\begin{bt}%[1K8BR-3]
	Chọn ngẫu nhiên một học sinh trong trường em. Xét hai biến cố sau:\\
	$P$: \lq\lq Học sinh đó bị cận thị\rq\rq;\\
	$Q$: \lq\lq Học sinh đó học giỏi môn Toán\rq\rq.\\
	Nêu nội dung của các biến cố $P \cup Q$; $P Q$ và $\bar{P} \bar{Q}$.
	\loigiai{
		$P \cup Q$ là biến cố \lq\lq Học sinh đó bị cận thị hoặc học sinh đó giỏi môn Toán\rq\rq.\\
		$PQ$ là biến cố \lq\lq Học sinh đó bị cận thị và học giỏi môn Toán\rq\rq.\\
		$\bar{P}$ là biến cố \lq\lq Học sinh đó không bị cận thị\rq\rq.\\
		$\bar{Q}$ là biến cố \lq\lq Học sinh đó không giỏi môn Toán\rq\rq.\\
		$\bar{P} \bar{Q}$ là biến cố \lq\lq Học sinh đó không bị cận và không giỏi môn Toán\rq\rq. 
	}
\end{bt}

%%==========Bài 4
\begin{bt}%[1K8KR-3]
	Có hai chuồng nuôi thỏ. Chuồng I có $5$ con thỏ đen và $10$ con thỏ trắng. Chuồng II có $3$ con thỏ trắng và $7$ con thỏ đen. Từ mỗi chuồng bắt ngẫu nhiên ra một con thỏ. Xét hai biến cố sau:\\
	$A$: \lq\lq Bắt được con thỏ trắng từ chuồng I\rq\rq;\\
	$B$: \lq\lq Bắt được con thỏ đen từ chuồng Il\rq\rq.\\
	Chứng tỏ rằng hai biến cố $A$ và $B$ độc lập.
	\loigiai{
		Nếu $A$ xảy ra, tức là bắt được con thỏ trắng ở chuồng I. Khi đó, số thỏ ở chuồng II không bị thay đổi và có $3$ thỏ trắng và $7$ thỏ đen. Vậy $P(B)=\dfrac{7}{10}$.\\	
		Nếu $A$ không xảy ra, tức là bắt được thỏ đen ở chuồng I. Khi đó, số thỏ ở chuồng II không bị thay đổi và có $3$ thỏ trắng và $7$ thỏ đen. Vậy $P(B)=\dfrac{7}{10}$.\\	
		Như vậy, xác suất xảy ra của biến cố $B$ không thay đổi bởi việc xảy ra hay không xảy ra của biến cố $A$.\\	
		Tương tự $P(A)=\dfrac{2}{3}$ dù biến cố $B$ xảy ra hay không xảy ra.\\	
		Vậy $A$ và $B$ độc lập.
	}
\end{bt}

%%==========Bài 5
\begin{bt}%[1K8KR-3]
	Có hai chuồng nuôi gà. Chuồng I có $9$ con gà mái và $3$ con gà trống. Chuồng II có $3$ con gà mái và $6$ con gà trống. Bắt ngẫu nhiên một con gà của chuồng I để đem bán rồi dồn các con gà còn lại của chuồng I vào chuồng II. Sau đó bắt ngẫu nhiên một con gà của chuồng II. Xét hai biến cố sau:\\
	$E$: \lq\lq Bắt được con gà trống từ chuồng I\rq\rq;\\
	$F$: \lq\lq Bắt được con gà mái từ chuồng II\rq\rq.\\
	Chứng tỏ rằng hai biến cố $E$ và $F$ không độc lập.
	\loigiai{
		Nếu $E$ xảy ra, tức là bắt được con gà trống từ chuồng I. Vì con gà trống bị bắt đem đi bán và số gà ở chuồng I dồn vô chuồng II, nên khi bắt gà mái từ chuồng II sẽ có $12$ gà mái và $8$ gà trống. Vậy $P(F)=\dfrac{12}{20}=\dfrac{3}{5}$. \\
		Nếu $E$ không xảy ra, tức là bắt được con gà mái từ chuồng I. Vì con gà mái bị bắt đem đi bán và số gà ở chuồng I dồn vô chuồng II, nên khi bắt gà mái từ chuồng II sẽ có $11$ gà mái và $9$ gà trống. Vậy $P(F)=\dfrac{11}{20}$. \\
		Như vậy, xác suất xảy ra của biến cố $F$ đã thay đổi phụ thuộc vào việc biến cố $E$ xảy ra hay không xảy ra. Do đó, hai biến cố $E$ và $F$ không độc lập.
	}
\end{bt}
\begin{bt}%[1K8BR-2]
	Gieo hai con xúc xắc cân đối và đồng chất. Gọi $A$ là biến cố \lq\lq Tổng số chấm xuất hiện trên hai con xúc xắc bằng 5\rq\rq, gọi $B$ là biến cố \lq\lq Xuất hiện hai mặt có cùng số chấm\rq\rq. Hai biến cố $A$ và $B$ có thể đồng thời cùng xảy ra không?
	\loigiai{
		\begin{itemize}
			\item Biến cố $A=\{(1 ; 4) ;(4 ; 1) ;(2 ; 3) ;(3 ; 2)\}$.
			\item Biến cố $B=\{(1 ; 1) ;(2 ; 2) ;(3 ; 3) ;(4 ; 4);(5;5);(6;6)\}$.\\
			Hai biến cố $A$ và $B$ không thể đồng thời cùng xảy ra.
		\end{itemize}
	}
\end{bt}
\begin{bt}%[1K8BR-3]
	\indent
	\begin{enumerate}
		\item [a)] Hai biến cố đối nhau có xung khắc với nhau không?
		\item [b)] Hai biến cố xung khắc có phải là hai biến cố đối nhau không?
	\end{enumerate}
	\loigiai{
		\begin{enumerate}
			\item [a)] Hai biến cố đối nhau có xung khắc với nhau.
			\item [b)] Hai biến cố xung khắc chưa phải là hai biến cố đối nhau.
		\end{enumerate}
	}
\end{bt}
\begin{bt}%[1K8BR-3]
	An và Bình mỗi người gieo một con xúc xắc cân đối và đồng chất. Gọi $A$ là biến cố \lq\lq An gieo được mặt $6$ chấm\rq\rq $\,$và $B$ là biến cố \lq\lq Bình gieo được mặt $6$ chấm\rq\rq.
	\begin{enumerate}
		\item Tính xác suất của biến cố $B$.
		\item Tính xác suất của biến cố $B$ trong hai trường hợp sau:
		\begin{itemize}
			\item Biến cố $A$ xảy ra;
			\item Biến cố $A$ không xảy ra.
		\end{itemize}
	\end{enumerate}
	\loigiai{
		\begin{enumerate}
			\item Xác suất của biến cố $B$ là $\dfrac{1}{6}$.
			\item 
			\begin{itemize}
				\item  Xác suất của biến cố $A$ xảy ra là $\dfrac{1}{6}$;
				\item  Xác suất của biến cố $A$ không xảy ra là $\dfrac{5}{6}$.
			\end{itemize}
		\end{enumerate}
		Ta thấy dù biến cố $A$ xảy ra hay không thì xác suất của biến cố $B$ vẫn luôn là $\dfrac{1}{6}$. \\
		Ta nói $A$ và $B$ là hai {\bf \textit{biến cố độc lập}}.
	}
\end{bt}


