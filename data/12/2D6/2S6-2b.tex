%%%%%%%
\setcounter{bt} {0}
\subsection{Bài tập tự luận}
%%==========Bài 1
\begin{bt}%%%%[2D5H2-2]
	Trong quân sự, một máy bay chiến đấu của đối phương có thể xuất hiện ở vị trí X với xác suất 0{,}55. Nếu máy bay đó không xuất hiện ở vị trí X thì nó xuât hiện ở vị trí Y. Để phòng thủ, các bệ phóng tên lửa được bố trí tại các vị trí X và Y. Khi máy bay đối phương xuất hiện ở vị trí X hoặc Y thì tên lửa sẽ được phóng để hạ máy bay đó.\\
	Xét phương án tác chiến sau: Nếu máy bay xuất hiện tại X thì bắn 2 quả tên lửa và nếu máy bay xuất hiện tại Y thì bắn 1 quả tên lửa.\\
	Biết rằng, xác suất bắn trúng máy bay của mỗi quả tên lửa là 0{,}8 và các bệ phóng tên lửa hoạt động độc lập. Máy bay bị bắn hạ nếu nó trúng ít nhất 1 quả tên lửa. Tính xác suất bắn hạ máy bay đối phương trong phương án tác chiến nêu trên.
	\loigiai{Xét biến cố $A$: ``Máy bay xuất hiện ở vị trí X'', điều đó có nghĩa là biến cố $\overline{A}$: ``Máy bay xuất hiện ở vị trí Y''. Xét biến cố $B$: ``Máy bay bị bắn hạ''. \\
	Ta có $P(B)=P(A)\cdot P(B \mid A)+P(\overline{A}) \cdot P(B \mid \overline{A})$.
	\begin{itemize}
	\item Tính $P(A)$, $P(\overline{A})$: $P(A)=0{,}55$ và $P(\overline{A})=0{,}35$.
	\item Tính $P(B\mid A)$: Đây là xác suất để máy bay bị bắn hạ tại vị trí X. Máy bay bị bắn hạ nếu nó trúng ít nhất một 1 quả tên lửa (trong 2 quả tên lửa đối với máy bay ở vị trí X), mà xác suất bắn trúng máy bay của mỗi quả tên lửa là $0{,}8$, vậy $P(B\mid A)=1-\left(1-0{,}8\right)\left(1-0{,}8\right)=0{,}96$.
	\item Tính $P(B\mid \overline{A})$: Đây là xác suất để máy bay bị bắn hạ tại vị trí Y. Máy bay bị bắn hạ nếu nó trúng ít nhất một 1 quả tên lửa (trong 1 quả tên lửa đối với máy bay ở vị trí Y), mà xác suất bắn trúng máy bay của mỗi quả tên lửa là $0{,}8$, vậy $P(B\mid \overline{A})=0{,}8$.
	\end{itemize}
	Vậy $P(B)=P(A)\cdot P(B \mid A)+P(\overline{A}) \cdot P(B \mid \overline{A})=0{,}55\cdot 0{,}96+0{,}35\cdot 0{,}8=0{,}808$. \\
	Vậy xác suất để máy bay bị bắn hạ là $0{,}808$.
	}
\end{bt}
%%==========Bài 2
\begin{bt}%%%%[2D5H2-2]
	Có hai chuồng thỏ. Chuồng I có 5 con thỏ đen và 10 con thỏ trắng. Chuồng II có 7 con thỏ đen và 3 con thỏ trắng. Trước tiên, từ chuồng II lấy ra ngẫu nhiên 1 con thỏ rồi cho vào chuồng I. Sau đó, từ chuồng I lấy ra ngẫu nhiên 1 con thỏ. Tính xác suất để con thỏ được lấy ra là con thỏ trắng.
	\loigiai{Xét biến cố $A$: ``Con thỏ được lấy ra từ chuồng II để cho vào chuồng I là con thỏ trắng''. 	Xét biến cố $B$: ``Con thỏ được lấy ra từ chuồng I là con thỏ trắng''. \\	
	Ta có $P(B)=P(A)\cdot P(B \mid A)+P(\overline{A}) \cdot P(B \mid \overline{A})$.
	\begin{itemize}
	\item Tính $P(A)$: Đây là xác suất để lấy ra ngẫu nhiên 1 con thỏ trắng từ chuồng II rồi cho vào chuồng I. Có $n\left(\Omega\right)=\mathrm{C}^1_{10}$, $n\left(A\right)=\mathrm{C}^1_3$. Vậy $P(A)=\dfrac{3}{10}$.
	\item Tính $P(\overline{A})$: $P(\overline{A})=1-P(A)=\dfrac{7}{10}$.
	\item Tính $P(B\mid A)$: Đây là xác suất để lấy ra ngẫu nhiên 1 con thỏ trắng từ chuồng I với điều kiện đã chọn ra 1 con thỏ trắng từ chuồng II rồi cho vào chuồng I, tức là có 5 con thỏ đen và 11 con thỏ trắng ở trong chuồng I. Tương tự như trên ta có $P(B\mid A)=\dfrac{11}{16}$.
	\item Tính $P(B\mid \overline{A})$: Đây là để lấy ra ngẫu nhiên 1 con thỏ trắng từ chuồng I với điều kiện đã chọn ra 1 con thỏ đen từ chuồng II rồi cho vào chuồng I, tức là có 6 con thỏ đen và 10 con thỏ trắng ở trong chuồng I. Tương tự như trên ta có $P(B\mid \overline{A})=\dfrac{10}{16}$.
	\end{itemize}
	Vậy $P(B)=P(A)\cdot P(B \mid A)+P(\overline{A}) \cdot P(B \mid \overline{A})=\dfrac{3}{10}\cdot \dfrac{11}{16}+\dfrac{7}{10}\cdot \dfrac{10}{16}=\dfrac{103}{160}=0{,}64375$. \\
	Vậy xác suất để con thỏ được lấy ra là con thỏ trắng là $0{,}64375$.
	}
\end{bt}
%%==========Bài 3
\begin{bt}%%%%[2D5V2-3]
	Một bộ lọc được sử dụng để chặn thư rác trong các tài khoản thư điện tử. Tuy nhiên, vì bộ lọc không tuyệt đối hoàn hảo nên một thư rác bị chặn với xác suất 0{,}95 và một thư đúng (không phải là thư rác) bị chặn với xác suất 0{,}01. Thống kê cho thấy tỉ lệ thư rác là $3 \%$.
	\begin{listEX}
	\item Chọn ngẫu nhiên một thư bị chặn. Tính xác suất để đó là thư rác.
	\item Chọn ngẫu nhiên một thư không bị chặn. Tính xác suất để đó là thư đúng.
	\item Trong số các thư bị chặn, có bao nhiêu phần trăm là thư đúng? Trong số các thư không bị chặn, có bao nhiêu phần trăm là thư rác?
	\end{listEX}
	\loigiai{Xét biến cố $A$: ``Thư đó là thư rác''. Xét biến cố $B$: ``Thư đó bị chặn''.
	\begin{listEX}
	\item Ta cần tính $P(A\mid B)$.
	Áp dụng công thức Bayes
	$$P(A\mid B) = \dfrac{P(A)\cdot P(B\mid A)}{P(A)\cdot P(B\mid A) + P(\overline{A})\cdot P(B \mid \overline{A})}.$$
	\begin{itemize}
	\item Tính $P(A)$: Đây là xác suất thư đó là thư rác. Vậy $P(A)=0{,}03$.
	\item Tính $P(\overline{A})$: $P(\overline{A})=1-P(A)=0{,}97$.
	\item Tính $P(B \mid A)$: Đây là xác suất thư đó là thư rác bị chặn. Vậy $P(B \mid A)=0{,}95$.
	\item Tính $P(B \mid \overline{A})$: Đây là xác suất thư đó là thư đúng bị chặn. Vậy $P(B \mid \overline{A})=0{,}01$.
	\end{itemize}
	Vậy $P(A\mid B) = \dfrac{P(A)\cdot P(B\mid A)}{P(A)\cdot P(B\mid A) + P(\overline{A})\cdot P(B \mid \overline{A})}=\dfrac{0{,}03\cdot 0{,}95}{0{,}03\cdot 0{,}95+0{,}97\cdot 0{,}01}=\dfrac{285}{382}\approx 0{,}746$.\\
	Vậy xác suất để chọn ngẫu nhiên một thư bị chặn mà thư đó là thư rác là khoảng $0{,}746$.
	\item Ta cần tính $P(\overline{A}\mid \overline{B})$.
	Áp dụng công thức Bayes
	$$P(\overline{A}\mid \overline{B}) = \dfrac{P(\overline{A})\cdot P(\overline{B}\mid \overline{A})}{P(\overline{A})\cdot P(\overline{B}\mid \overline{A}) + P(A)\cdot P(\overline{B} \mid A)}.$$
	\begin{itemize}
	\item Tính $P(\overline{B} \mid A)$: Ta có $(\overline{B} \mid A)=1-P(B \mid A)=0{,}05$.
	\item Tính $P(\overline{B} \mid \overline{A})$: Ta có $P(\overline{B} \mid \overline{A})=1-P(B \mid \overline{A})=0{,}99$.
	\end{itemize}
	Vậy $P(\overline{A}\mid \overline{B}) = \dfrac{P(\overline{A})\cdot P(\overline{B}\mid \overline{A})}{P(\overline{A})\cdot P(\overline{B}\mid \overline{A}) + P(A)\cdot P(\overline{B} \mid A)}=\dfrac{0{,}97\cdot 0{,}99}{0{,}97\cdot 0{,}99+0{,}03\cdot 0{,}05}=\dfrac{3201}{3206}\approx 0{,}998$.\\
	Vậy xác suất để chọn ngẫu nhiên một thư không bị chặn mà thư đó là thư đúng là khoảng $0{,}998$.
	\item Trong số các thư bị chặn, có $74{,}6 \%$ là thư rác, có $25{,}4 \%$ là thư đúng. \\
	Trong số các thư không bị chặn, có $0{,}2 \%$ là thư rác, có $99{,}8 \%$ là thư đúng.
	\end{listEX}
	}
\end{bt}
%%==========Bài 4
\begin{bt}%%%%[2D5N2-2]
	Trong một trường học, tỉ lệ học sinh nữ là $52\%$. Tỉ lệ học sinh nữ và tỉ lệ học sinh nam
	tham gia câu lạc bộ nghệ thuật lần lượt là $18\%$ và $15\%$. Gặp ngẫu nhiên 1 học sinh của trường. 
	\begin{listEX}
	\item Tính xác suất học sinh đó có tham gia câu lạc bộ nghệ thuật. 
	\item Biết rằng học sinh có tham gia câu lạc bộ nghệ thuật. Tính xác suất học sinh đó là nam.
	\end{listEX}
	\loigiai{
	\begin{listEX}
	\item Gọi $A$ là biến cố "Học sinh đó là nữ" và $B$ là biến cố "Học sinh đó tham gia câu lạc bộ nghệ thuật".\\
	Do tỉ lệ học sinh nữ là $52\%$ nên
	\begin{center}
	$P(A)=0{,}52$ và $P(\overline{A})=1-0{,}52=0{,}48$.
	\end{center}
	Do tỉ lệ học sinh nữ và tỉ lệ học sinh nam tham gia câu lạc bộ nghệ thuật lần lượt là $18\%$ và $15\%$ nên
	\begin{center}
	$P(B|A)=0{,}18$ và $P(B|\overline{A})=0{,}15$.
	\end{center}
	Xác suất để học sinh đó có tham gia câu lạc bộ nghệ thuật là
	$$P(B)=P(A)P(B|A)+P(\overline{A})P(B|\overline{A})=0{,}52\cdot0{,}18+0{,}48\cdot0{,}15=0{,}1656.$$
	\item Do học sinh có tham gia câu lạc bộ nghệ thuật nên xác suất học sinh đó là nam là
	$$P(\overline{A}|B)=\dfrac{P(\overline{A})P(B|\overline{A})}{P(B)}=\dfrac{0{,}48\cdot0{,}15}{0{,}1656}=\dfrac{10}{23}.$$
	\end{listEX}}
\end{bt}
%%==========Bài 5
\begin{bt}%%%%[2D5Y2-3]
	Tỉ lệ người dân đã tiêm vắc xin phòng bệnh A ở một địa phương là $65\%$. Trong số những 
	người đã tiêm phòng, tỉ lệ mắc bệnh A là $5\%$ còn trong số những người chưa tiêm, tỉ lệ 
	mắc bệnh A là $17\%$. Gặp ngẫu nhiên một người ở địa phương đó.
	\begin{listEX}
	\item Tính xác suất người đó mắc bệnh A.
	\item Biết rằng người đó mắc bệnh A. Tính xác suất người đó không tiêm vắc xin phòng bệnh A.
	\end{listEX}
	\loigiai{Gọi $H_1$ là biến cố "Gặp được người đã tiêm vắc xin phòng bệnh $A$", $H_2$ là biến cố "Gặp được người chưa tiêm vắc xin phòng bệnh $A$", $A$ là biến cố" người đó mắc bệnh A".
	\begin{listEX}
	\item Theo công thức Bayes, ta có:
	$$\mathrm{P}(A)=\mathrm{P}(H_1).\mathrm{P}(A|H_1)+\mathrm{P}(H_2).\mathrm{P}(A|H_2)$$
	$$ \Leftrightarrow \mathrm{P}(A)= 0,65.0,05+0,35.0,17=0,092. $$
	\item $$\mathrm{P}(H_2|A)=\dfrac{\mathrm{P}(AH_2)}{\mathrm{P}(A)}=\dfrac{\mathrm
	P(H_2)\mathrm{P}(A|H_2)}{\mathrm{P}(A)}$$
	$$\Leftrightarrow \mathrm{P}(H_2|A)= \dfrac{0,35.0,17}{0,092}=\dfrac{119}{184}\approx 0,65.
	$$
	\end{listEX}}
\end{bt}
%%==========Bài 6
\begin{bt}%%%%[2D5N2-3]
	Ở một khu rừng nọ có 7 chú lùn, trong đó có 4 chú luôn nói thật, 3 chú còn lại nói thật với 
	xác suất 0,5. Bạn Tuyết gặp ngẫu nhiên một chú lùn. Gọi $A$ là biến cố “Chú lùn đó luôn
	nói thật” và $B$ là biến cố “Chú lùn đó tự nhận mình luôn nói thật”.
	\begin{listEX}
	\item Tính xác suất của các biến cố $A$ và $B$.
	\item Biết rằng chú lùn mà bạn Tuyết gặp tự nhận mình là người luôn nói thật. Tính xác suất để 
	chú lùn đó luôn nói thật.
	\end{listEX}
	\loigiai{Gọi $C$ là biến cố "bạn Tuyết gặp được chú lùn nói thật với xác suất $0,5$".
	\begin{listEX}
	\item Ta có
	$\mathrm{P}(A)=\dfrac{4}{7}$
	Theo công thức Bayes, ta có
	$\mathrm{P}(B)=\mathrm{P}(A).\mathrm{P}(B|A)+\mathrm{P}(C).\mathrm{P}(B|C)$
	$\Leftrightarrow \mathrm{P}(B)=\dfrac{4}{7}.1+\dfrac{3}{7}.0,5=\dfrac{11}{14}\approx 0,79 $
	\item $$\mathrm{P}(A|B)=\dfrac{\mathrm{P}(AB)}{\mathrm{P(B)}}=\dfrac{\mathrm{P}(A).\mathrm{P}(B|A)}{\mathrm{P}(B)}=\dfrac{\dfrac{4}{7}.1}{\dfrac{11}{14}}=\dfrac{8}{11}\approx 0,73.$$
	\end{listEX}}
\end{bt}
%%==========Bài 7
\begin{bt}%%%
	Giả sử có khoảng $40 \%$ thư điện tử (email) gửi đến một địa chỉ là thư rác. Người ta sử dụng một thuật toán để phân loại thư rác, biết rằng thuật toán này có thể phân loại đến $99 \%$ thư rác và tỉ lệ sai sót khi phân loại thư bình thường thành thư rác là $5 \%$. Tính xác suất một thư điện tử là thư bình thường nếu thư này đã được phân loại đúng.
	\loigiai{
	Ta có công thức
	$$	P(A | B)=\dfrac{P(B | A) \cdot P(A)}{P(B)}$$
	Trong đó
	\begin{itemize}
	\item $A$: Thư điện tử là thư bình thường.
	\item $B$: Thư đã được phân loại đúng.
	\item 	$\mathrm{P}(A)$: Xác suất một thư điện tử là thư bình thường ban đầu.\\
	Vì có $40 \%$ thư rác, nên $\mathrm{P}(A)=$ $1-0{,}4=0{,}6$.
	\item $\mathrm{P}(B | A)$: Xác suất một thư bình thường được phân loại đúng.\\
	Do tỉ lệ sai sót là $5 \%$, nên $\mathrm{P}(B | A)=1-0{,}05=0{,}95$.
	\item $P(B)$: Xác suất một thư nào đó được phân loại đúng, tính bằng tổng xác suất một thư rác được phân loại đúng và xác suất một thư bình thường được phân loại đúng
	\end{itemize}
	$$\begin{aligned}[t]
	\mathrm{P}(B)&=\mathrm{P}(B | A) \cdot \mathrm{P}(A)+\mathrm{P}(B |\overline{A}) \cdot \mathrm{P}(\overline{A}) \\&
	=0{,}95 \cdot 0{,}6+0{,}99 \cdot 0{,}4=0{,}97.
	\end{aligned}$$
	Áp dụng định lý Bayes: $\mathrm{P}(A | B)=\dfrac{\mathrm{P}(B | A) \cdot \mathrm{P}(A)}{\mathrm{P}(B)}=\dfrac{0{,}95 \cdot 0{,}6}{0{,}97}=	\dfrac{57}{97}$.
	}
\end{bt}
%%==========Bài 8
\begin{bt}%%%
	Một chiếc hộp có $20$ chiếc thẻ cùng loại, trong đó có $2$ chiếc thẻ màu xanh và $18$ chiếc thẻ màu trắng. Bạn Châu rút thẻ hai lần một cách ngẫu nhiên, mỗi lần rút một thẻ và thẻ được rút ra không bỏ lại hộp. Tính xác suất để cả hai lần bạn Châu đều rút được thẻ màu xanh.
	\loigiai{
	Xét hai biến cố\\
	$A$: \lq\lq  Thẻ thứ nhất rút được màu xanh\rq\rq.\\
	$B$: \lq\lq  Thẻ thứ hai rút được màu xanh\rq\rq.\\
	Khi đó, ta có xác suất để cả hai lần bạn Châu đều rút được thẻ màu xanh là \[\mathrm{P}(A\cap B)=\dfrac{C_2^2}{C_{20}^2}=\dfrac{1}{190}.\]
	}
\end{bt}