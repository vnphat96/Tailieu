% \newpage
% \setcounter{ex}{0}
\Opensolutionfile{ans}[ans/ans-0-B15]
%\TN
%%==========PHẦN 1=============================================
%%==========Câu 1
\begin{ex}%[2D6H1-2]
	Cho $A$, $B$ là các biến cố của một phép thử $T$. Biết rằng $\mathrm{P}(B)>0$, xác suất của biến cố $A$ với điều kiện biến cố $B$ đã xảy ra được tính theo công thức nào sau đây?
	\choice
	{$\mathrm{P}(A\mid B)=\dfrac{\mathrm{P}(A)}{\mathrm{P}(B)}$}
	{$\mathrm{P}(A\mid B)=\dfrac{\mathrm{P}(A)}{\mathrm{P}(A B)}$}
	{\True $\mathrm{P}(A\mid B)=\dfrac{\mathrm{P}(A B)}{\mathrm{P}(B)}$}
	{$\mathrm{P}(A\mid B)=\dfrac{\mathrm{P}(A B)}{\mathrm{P}(A) \cdot \mathrm{P}(B)}$}
	\loigiai{ Dựa theo công thức tính xác suất biến cố $A$ với điều kiện $B$ thì $\mathrm{P}(A\mid B)=\dfrac{\mathrm{P}(A B)}{\mathrm{P}(B)}$ là đáp án đúng.
	}
\end{ex}
%%==========Câu 2
\begin{ex}%[2D6H1-2]
	Cho hai biến cố độc lập $A$, $B$ với $\mathrm{P}(A)=0{,}3$; $\mathrm{P}(B)=0{,}4$. Khi đó, $\mathrm{P}(A\mid B)$ bằng
	\choice{$0{,}7$}
	{$0{,}12$}
	{$0{,}4$}
	{\True $0{,}3$}
	\loigiai{
	Vì $A$ và $B$ là $2$ biến cố độc lập nên $\mathrm{P}(A\cap B)=\mathrm{P}(A)\cdot \mathrm{P}(B)=0{,}12$.\\
	$$\Rightarrow \mathrm{P}(A \mid B)=\dfrac{\mathrm{P}(A\cap B)}{\mathrm{P}(B)}=\mathrm{P}(A)=\dfrac{0{,}12}{0{,}4}=0{,}3.$$
	}
\end{ex}
%%==========Câu 3
\begin{ex}%[2D6H1-2]
	Cho hai biến cố xung khắc $A$, $B$ với $\mathrm{P}(A)=0{,}2$; $\mathrm{P}(B)=0{,}4$. Khi đó, $\mathrm{P}(A\mid B)$ bằng
	\choice{$0{,}5$}
	{$0{,}2$}
	{$0{,}4$}
	{\True $0$}
	\loigiai{
	Vì $A$ và $B$ là $2$ biến cố xung khắc nên $\mathrm{P}(A\cap B)=0$.\\
	$$\Rightarrow \mathrm{P}(A\mid B)=\dfrac{\mathrm{P}(A\cap B)}{\mathrm{P}(B)}=\dfrac{0}{0{,}4}=0.$$
	}
\end{ex}
%%==========Câu 4
\begin{ex}%[2D6H1-2]
	Cho $\mathrm{P}(A)=\dfrac{2}{5}$; $\mathrm{P}\left( B\mid A\right)=\dfrac{1}{3}$. Giá trị của $\mathrm{P}(AB)$ là 
	\choice
	{\True $\dfrac{2}{15} $}
	{$ \dfrac{3}{16}$}
	{$ \dfrac{1}{5}$}
	{$ \dfrac{4}{15}$}
	\loigiai{
	Ta có $\mathrm{P}(AB)=\mathrm{P}(A)\cdot \mathrm{P}\left( B\mid A\right)=\dfrac{2}{5}\cdot \dfrac{1}{3}=\dfrac{2}{15}$.
	}
\end{ex}
%%==========Câu 5
\begin{ex}%[2D6H1-2]
	Cho $\mathrm{P}(A)=\dfrac{2}{5}$; $\mathrm{P}\left(B\mid \overline{A}\right)=\dfrac{1}{4}$. Giá trị của $\mathrm{P}\left(B\overline{A}\right)$ là 
	\choice
	{$\dfrac{1}{7} $}
	{$ \dfrac{4}{19}$}
	{$ \dfrac{4}{21}$}
	{\True $ \dfrac{3}{20}$}
	\loigiai{
	Ta có $\mathrm{P}\left(B\overline{A}\right)=\mathrm{P}\left( B\mid \overline{A} \right)\cdot \mathrm{P}\left( \overline{A}\right) =\dfrac{1}{4}\cdot \dfrac{3}{5}=\dfrac{3}{20}$.
	}
\end{ex}
%%%%%----------Câu 1
\begin{ex}%[2D6N1-1]%[Võ Thanh Hiệp]
	Cho hai biến cố $A$ và $B$. Xác suất của biến cố $B$, tính trong điều kiện biết rằng biến cố $A$ đã xảy ra, được gọi là xác suất của $B$ với điều kiện $A$ kí hiệu là 
	\choice
	{$\mathrm{P}\left(A\mid B\right)$}
	{\True $\mathrm{P}\left(B\mid A\right)$}
	{$\mathrm{P}\left(AB\right)$}
	{$\mathrm{P}\left(B\right)$}
	\loigiai{
	Theo định nghĩa xác suất có điều kiện. Xác suất của $B$ với điều kiện $A$ kí hiệu là $\mathrm{P}\left(B\mid A\right)$.
	}	
\end{ex}
%%%%%----------Câu 2
\begin{ex}%[2D6N1-2]%[Võ Thanh Hiệp]
	Cho hai biến cố $A$ và $B$. Biết rằng xác suất của biến cố $A$ bằng $0{,}6$; xác suất của biến cố biến cố $B$ trong điều kiện biến cố $A$ đã xảy ra bằng $0{,}2$. Tính xác suất của $A$ và $B$ đều xảy ra. 
	\choice
	{\True $\dfrac{3}{25}$}
	{$\dfrac{3}{10}$}
	{$\dfrac{1}{3}$}
	{$\dfrac{2}{3}$}
	\loigiai{
	Ta có $\mathrm{P}\left(A\right) = 0{,}6$; $\mathrm{P}\left(B\mid A\right) = 0{,}2$.\\
	Suy ra 
	$\mathrm{P}\left(AB\right)=
	\mathrm{P}\left(A\right)\cdot \mathrm{P}\left(B\mid A\right)=0{,}6\cdot 0{,}2 =
	0{,}12 = \dfrac{3}{25} $.\\
	Vậy xác suất của $A$ và $B$ đều xảy ra là $\mathrm{P}\left(AB\right)= \dfrac{3}{25}$.
	}	
\end{ex}
%%%%%----------Câu 3
\begin{ex}%[2D6H1-2]%[Võ Thanh Hiệp]
	Gieo hai con xúc xắc cân đối đồng chất. Tính xác suất để tổng số chấm trên hai con xúc xắc bằng bằng $8$ nếu biết rằng ít nhất có một con xúc xắc xuất hiện mặt $3$ chấm.
	\choice
	{$\dfrac{5}{11}$}
	{$\dfrac{2}{5}$}
	{$\dfrac{11}{5}$}
	{\True $\dfrac{2}{11}$}
	\loigiai{
	Gọi $A$ là biến cố: \lq\lq  Tổng số chấm trên hai con xúc xắc bằng bằng $8$\rq\rq.\\
	Gọi $B$ là biến cố: \lq\lq  Có ít nhất có một con xúc xắc xuất hiện mặt $3$ chấm\rq\rq.\\
	Gieo hai con xúc xắc cân đối đồng chất, ta có $n\left(\Omega\right)=6\cdot 6 = 36$.\\
	$A=\{(2,6); (6,2); (3,5); (5,3); (4;4)\}$	$\Rightarrow n\left(A\right) = 5
	\Rightarrow \mathrm{P}\left(A\right)= \dfrac{5}{36}$.\\	
	$B=\{(3,1);(3,2); (3,3); (3,4); (3,5); (3,6); (1,3); (2,3); (4,3); (5,3); (6,3)\}$.\\
	$\Rightarrow n\left(B\right) =11 \Rightarrow \mathrm{P}\left(B\right)= \dfrac{11}{36}$.\\
	$AB=\{(3,5); (5,3)\}$
	$\Rightarrow n\left(AB\right) =2\Rightarrow \mathrm{P}\left(AB\right)=\dfrac{2}{36}=\dfrac{1}{18}$.\\
	Xác suất cần tính là 	
	$\mathrm{P}\left(A\mid B\right) =\dfrac{\mathrm{P}\left(AB\right) }{\mathrm{P}\left(B\right)}= \dfrac{2}{11}$.
	}	
\end{ex}
%%%%%----------Câu 4
\begin{ex}%[2D6H1-1]%[Võ Thanh Hiệp]
	Cho $A$ và $B$ là hai biến cố. Trong các mệnh đề sau, mệnh đề nào {\bf sai}?
	\choice
	{\True Với $\mathrm{P}\left(B\right)>0$. Khi đó $\mathrm{P}\left(A\mid B\right)=\mathrm{P}\left(B\right)\cdot \mathrm{P}\left(AB\right)$}
	{Nếu $A$ và $B$ là hai biến cố độc lập thì $\mathrm{P}\left(B\right)=\mathrm{P}\left(B \mid A\right)$}
	{Với $\mathrm{P}\left(B\right)>0$. Khi đó $\mathrm{P}\left(\overline{A}\mid B\right)= 1-\mathrm{P}\left(A\mid B\right)$}
	{Nếu $A$ và $B$ là hai biến cố độc lập thì $\mathrm{P}\left(A\mid B\right)=\mathrm{P}\left(A \mid \overline{B}\right)$}
	\loigiai{
	\begin{itemize}[\color{blue}\checkmark]
	\item Theo công thức nhân xác suất ta có $\mathrm{P}\left(AB \right)= \mathrm{P}\left(B\right)\cdot \mathrm{P}\left(A \mid B\right)$.\\
	Vậy $\mathrm{P}\left(A\mid B\right)=\mathrm{P}\left(B\right)\cdot \mathrm{P}\left(AB\right)$ sai.
	\item Nếu $A$ và $B$ là hai biến cố độc lập thì $\mathrm{P}\left(AB\right) =\mathrm{P}\left(A\right) \cdot \mathrm{P}\left(B\right)$.\\
	Suy ra $\mathrm{P}\left(B \mid A\right)=\dfrac{\mathrm{P}\left(BA\right)}{\mathrm{P}\left(A\right) }
	=	\dfrac{\mathrm{P}\left(B\right) \cdot \mathrm{P}\left(A\right) }{\mathrm{P}\left(A\right) }
	=\mathrm{P}\left(B\right)$.
	\item 
	\immini{Vì $\overline{A}B$ và $AB$ là hai biến cố xung khắc và $\overline{A}B \cup AB = B$ nên\\ $\mathrm{P}\left(\overline{A}B\right)=\mathrm{P}\left(B\right)-\mathrm{P}\left(AB\right)$.\\
	Với $\mathrm{P}\left(B\right)>0$.\\
	Ta có $\begin{aligned}[t]
	\mathrm{P}\left(\overline{A}\mid B\right)&=
	\dfrac{\mathrm{P}\left(\overline{A}B\right)}{\mathrm{P}\left(B\right) }\\
	&=\dfrac{\mathrm{P}\left(B\right)-\mathrm{P}\left(AB\right)}{\mathrm{P}\left(B\right) }\\
	&= 1- \dfrac{\mathrm{P}\left(AB\right)}{\mathrm{P}\left(B\right) }\\
	&=
	1-\mathrm{P}\left(A\mid B\right)
	\end{aligned}$.
	}{
	\begin{tikzpicture}[scale=1.5,>=stealth, line join=round, line cap=round]
	\draw[ pattern=north east lines]
	(0,0) to [bend left=90] (2,2) to [bend left=90] (0,0) (0,-.5) node {$A$} ;
	\draw[ pattern=north west lines]
	(1,0) to [bend left=90] (3,2) to [bend left=90] (1,0)
	(2.5,-.5) node {$B$};
	\def\miena{(0,-3.5) to [bend left=90] (2,-1.5) to [bend left=90] (0,-3.5)};
	\def\mienb{(1,-3.5) to [bend left=90] (3,-1.5) to [bend left=90] (1,-3.5)};	
	\node [circle,draw,fill=white] at (2.7,1.5){$\small \overline{A} B$};
	\node [circle,draw,fill=white] at (1.6,1.2){$\small AB$};
	\end{tikzpicture}
	}
	\item Nếu $A$ và $B$ là hai biến cố độc lập thì $A$ và $\overline{B}$ cũng là biến cố độc lập. Do đó\\
	$\mathrm{P}\left(A \mid B\right)=
	\dfrac{\mathrm{P}\left(AB\right)}{\mathrm{P}\left(B\right) }=
	\dfrac{\mathrm{P}\left(A\right) \cdot \mathrm{P}\left(B\right) }{\mathrm{P}\left(B\right) }
	=\mathrm{P}\left(A\right)$.\\
	$\mathrm{P}\left(A \mid \overline{B}\right)=
	\dfrac{\mathrm{P}\left(A\overline{B}\right)}{\mathrm{P}\left(\overline{B}\right) }=
	\dfrac{\mathrm{P}\left(A\right) \cdot \mathrm{P}\left(\overline{B}\right) }{\mathrm{P}\left(\overline{B}\right) }
	=\mathrm{P}\left(A\right)$.\\
	$\Rightarrow \mathrm{P}\left(A\mid B\right)=
	\mathrm{P}\left(A \mid \overline{B}\right)=\mathrm{P}\left(A\right)$.
	\end{itemize}
	}
\end{ex}
%%%%%----------Câu 5
\begin{ex}%[2D6N2-1]%[Võ Thanh Hiệp]
	Cho $A$ và $B$ là hai biến cố. Công thức nào dưới đây là công thức tính xác suất toàn phần?
	\choice
	{$\mathrm{P}\left(A\right)= \mathrm{P}\left(B\right)\cdot \mathrm{P}\left(A\mid B\right) + 
	\mathrm{P}\left(\overline{B}\right)\cdot \mathrm{P}\left(A\mid B\right)$}
	{$\mathrm{P}\left(A\right)= \mathrm{P}\left(B\right)\cdot \mathrm{P}\left(A\mid B\right) + 
	\mathrm{P}\left(\overline{B}\right)\cdot \mathrm{P}\left(\overline{A}\mid \overline{B}\right)$}
	{$\mathrm{P}\left(A\right)= \mathrm{P}\left(B\right)\cdot \mathrm{P}\left(A\mid B\right) + 
	\mathrm{P}\left(\overline{B}\right)\cdot \mathrm{P}\left(\overline{A}\mid B\right)$}
	{\True $\mathrm{P}\left(A\right)= \mathrm{P}\left(B\right)\cdot \mathrm{P}\left(A\mid B\right) + 
	\mathrm{P}\left(\overline{B}\right)\cdot \mathrm{P}\left(A\mid \overline{B}\right)$}
	\loigiai{
	Công thức tính xác suất toàn phần là 	
	$$\mathrm{P}\left(A\right)= \mathrm{P}\left(B\right)\cdot \mathrm{P}\left(A\mid B\right) + 
	\mathrm{P}\left(\overline{B}\right)\cdot \mathrm{P}\left(A\mid \overline{B}\right).$$
	}
\end{ex}
%%%%%----------Câu 6
\begin{ex}%[2D6N2-2]%[Võ Thanh Hiệp]
	Cho hai biến cố $A$ và $B$ có $\mathrm{P}\left(A\right) = 0{,}3$; $\mathrm{P}\left(B\right) = 0{,}5$ và
	$\mathrm{P}\left(B\mid A\right)=0{,}4$. Tính $\mathrm{P}\left(A\mid B\right)$. 
	\choice
	{$0{,}6$}
	{\True $0{,}24$}
	{$0{,}15$}
	{$0{,}5$}
	\loigiai{
	Áp dụng công thức Bayes	ta có
	\begin{eqnarray*}
	\mathrm{P}\left(A\mid B\right)&=&
	\dfrac{\mathrm{P}\left(A\right) \cdot \mathrm{P}\left(B\mid A\right) }{\mathrm{P}\left(A\right)\cdot \mathrm{P}\left(B\mid A\right)
	+ \mathrm{P}\left(\overline{A}\right)\cdot \mathrm{P}\left(B\mid \overline{A}\right)}\\
	&=& \dfrac{\mathrm{P}\left(A\right) \cdot \mathrm{P}\left(B\mid A\right) }{\mathrm{P}\left(B\right)}\\
	&=&\dfrac{0{,}3\cdot 0{,}4}{0{,}5}=0{,}24.
	\end{eqnarray*}
	}
\end{ex}
%%%%%----------Câu 7
\begin{ex}%[2D6N1-2]%[Võ Thanh Hiệp]
	Cho hai biến cố $A$ và $B$ có $\mathrm{P}\left(A\right) = 0{,}7$; $\mathrm{P}\left(B\right) = 0{,}4$ và
	$\mathrm{P}\left(AB\right)=0{,}2$. Tính xác suất biến cố $B$ với điều kiện $A$. 
	\choice
	{$\dfrac{1}{2}$}
	{$\dfrac{4}{7}$}
	{\True $\dfrac{2}{7}$}
	{$\dfrac{7}{10}$}
	\loigiai{
	Xác suất của biến cố $B$ với điều kiện $A$ là $\mathrm{P}\left(B\mid A\right)$.\\
	Ta có $\mathrm{P}\left(B\mid A\right) 
	= \dfrac{\mathrm{P}\left(AB\right) }{\mathrm{P}\left(A\right)}
	=\dfrac{0{,}2}{0{,}7}=\dfrac{2}{7}$.
	}
\end{ex}
%%%%%----------Câu 8
\begin{ex}%[2D6H1-2]%[Võ Thanh Hiệp]
	Cho hai biến cố $A$ và $B$ có $\mathrm{P}\left(A\right) = 0{,}6$; $\mathrm{P}\left(B\right) = 0{,}4$ và $\mathrm{P}\left(AB\right)=0{,}3$. Xác suất biến cố $A$ không xảy ra với điều kiện $B$ là 
	\choice
	{$\dfrac{7}{10}$}
	{$\dfrac{3}{4}$}
	{$\dfrac{1}{2}$}
	{\True $\dfrac{1}{4}$}
	\loigiai{
	Xác suất biến cố $A$ không xảy ra với điều kiện $B$ là 
	$\mathrm{P}\left(\overline{A}\mid B\right)$.\\
	Ta có $\mathrm{P}\left(\overline{A}B\right)= 1- \mathrm{P}\left(A \mid B\right)
	=1-\dfrac{\mathrm{P}\left(AB\right)}{\mathrm{P}\left(B\right)}=1-\dfrac{0{,}3}{0{,}4}=\dfrac{1}{4}$.
	}
\end{ex}
%%%%%----------Câu 9
\begin{ex}%[2D6H1-2]%[Võ Thanh Hiệp]
	Trong hộp có $7$ viên bi màu xanh, $5$ viên bi màu đỏ, các viên bi có cùng kích thước và khối lượng. An lấy ngẫu nhiên một viên bi từ hộp không trả lại, sau đó Bình lấy một viên bi trong các bi còn lại. Tính xác suất để An lấy được bi màu xanh và Bình lấy được bi màu đỏ.
	\choice
	{\True $\dfrac{35}{132}$}
	{$\dfrac{35}{144}$}
	{$\dfrac{1}{11}$}
	{$\dfrac{3}{132}$}
	\loigiai{
	Gọi $A$ là biến cố: \lq\lq  An lấy được bi màu xanh\rq\rq.\\
	$B$ là biến cố: \lq\lq  Bình lấy được bi màu đỏ\rq\rq.\\
	Ta cần tính $\mathrm{P}\left(AB\right)$.\\
	Ta có $\mathrm{P}\left(A\right) = \dfrac{7}{12}$, 
	$\mathrm{P}\left(B\mid A\right) = \dfrac{5}{11}$.\\
	Theo công thức nhân xác suất
	$\mathrm{P}\left(AB\right)= \mathrm{P}\left(A\right)\cdot \mathrm{P}\left(B\mid A\right)
	=\dfrac{7}{12} \cdot \dfrac{5}{11}=\dfrac{35}{132}$.
	}
\end{ex}
%%%%%----------Câu 10
\begin{ex}%[2D6H1-3]%[Võ Thanh Hiệp]
	\immini{ Cho sơ đồ cây như hình vẽ bên. Khẳng định nào sau đây {\bf sai}?
	\choice
	{$\mathrm{P}\left(A\right)= 0{,}3$}
	{$\mathrm{P}\left(B\right)= 0{,}54$}
	{\True $\mathrm{P}\left(A\mid B\right)=0{,}6$}
	{$\mathrm{P}\left(\overline{B}\mid \overline{A}\right)=0{,}2$}
	}{
	\tikzstyle{xs} = [rectangle ,fill=white,draw=black,rounded corners,align=center]
	\tikzstyle{bc} = [circle ,fill=white,draw=black,rounded corners,align=center]
	\begin{tikzpicture}[scale=0.7,>=stealth, font=\footnotesize, line join=round, line cap=round]
	\node[bc,text width=2.5mm] (O) at(0,0) { };
	\node[bc] (A) at ($(O)+(30:3)$) {$A$};
	\node[bc] (A1) at ($(O)+(-30:3)$){$\overline{A}$};
	\node[bc] (B) at ($(A)+(20:3)$) {$B$};
	\node[bc] (B1) at ($(A)+(-20:3)$) {$\overline{B}$};
	\node[bc] (B2) at ($(A1)+(20:3)$) {$B$};
	\node[bc] (B3) at ($(A1)+(-20:3)$) {$\overline{B}$};
	\draw[->] {(O)--node [above,xs,sloped] {$0{,}3$}(A)} ;
	\draw[->] {(O)--node [below,xs,sloped] {$0{,}7$}(A1)} ;
	\draw[->] {(A)--node [above,xs,sloped] {$0{,}4$}(B)} ;
	\draw[->] {(A)--node [below,xs,sloped] {$0{,}6$}(B1)} ;
	\draw[->] {(A1)--node [above,xs,sloped] {$0{,}8$}(B2)} ;
	\draw[->] {(A1)--node [below,xs,sloped] {$0{,}2$}(B3)} ;
	\end{tikzpicture}
	}
	\loigiai{
	\begin{center}	
	\tikzstyle{xs} = [rectangle ,fill=white,draw=black,rounded corners,align=center]
	\tikzstyle{bc} = [circle ,fill=white,draw=black,rounded corners,align=center]
	\begin{tikzpicture}[scale=0.7,>=stealth, font=\footnotesize, line join=round, line cap=round]
	\node[bc,text width=2.5mm] (O) at(0,0) { };
	\node[bc] (A) at ($(O)+(32:3)$) {$A$};
	\node[bc] (A1) at ($(O)+(-32:3)$){$\overline{A}$};
	\node[bc] (B) at ($(A)+(20:3)$) {$B$};
	\node[bc] (B1) at ($(A)+(-20:3)$) {$\overline{B}$};
	\node[bc] (B2) at ($(A1)+(20:3)$) {$B$};
	\node[bc] (B3) at ($(A1)+(-20:3)$) {$\overline{B}$};
	\draw[->] {(O)--node [above,xs,sloped] {$0{,}3$}(A)} ;
	\draw[->] {(O)--node [below,xs,sloped] {$0{,}7$}(A1)} ;
	\draw[->] {(A)--node [above,xs,sloped] {$0{,}4$}(B)} ;
	\draw[->] {(A)--node [below,xs,sloped] {$0{,}6$}(B1)} ;
	\draw[->] {(A1)--node [above,xs,sloped] {$0{,}8$}(B2)} ;
	\draw[->] {(A1)--node [below,xs,sloped] {$0{,}2$}(B3)} ;
	\node[bc] (AB) at ($(B)+(0:2)$) {$AB$};
	\node[bc] (AB1) at ($(B1)+(0:2)$) {$A\overline{B}$};
	\node[bc] (AB2) at ($(B2)+(0:2)$) {$\overline{A}B$};
	\node[bc] (AB3) at ($(B3)+(0:2)$) {$\overline{A}\,\overline{B}$};
	\end{tikzpicture}
	\end{center}
	Từ sơ đồ cây suy ra\\
	$\mathrm{P}\left(A\right)= 0{,}3$; $\mathrm{P}\left(\overline{A}\right)= 0{,}7$;
	$\mathrm{P}\left(A B\right)= 0{,}12$;\\
	$\mathrm{P}\left(B\mid A\right)= 0{,}4$; $\mathrm{P}\left(B\mid \overline{A}\right)= 0{,}8$;
	$\mathrm{P}\left(\overline{B}\mid \overline{A}\right)=0{,}2$.\\
	Suy ra
	$\mathrm{P}\left(B\right)= \mathrm{P}\left(A\right)\cdot \mathrm{P}(B\mid A)
	+ \mathrm{P}\left(\overline{A}\right)\cdot \mathrm{P}(B\mid \overline{A})
	= 0{,}3\cdot 0{,}4 + 0{,}7\cdot 0{,}6 = 0{,}54 $.\\
	$\mathrm{P}\left(A\mid B\right)= \dfrac{\mathrm{P}\left(AB\right)}{\mathrm{P}\left(B\right)}
	=\dfrac{0{,}12}{0{,}54}=\dfrac{2}{9}$.\\
	Vậy $\mathrm{P}\left(A\mid B\right)=0{,}6$ sai.
	}
\end{ex}
%%%%%----------Câu 11
\begin{ex}%[2D6H2-3]%[Võ Thanh Hiệp]
	Cho hai biến cố ngẫu nhiên $A$ và $B$. Biết rằng $\mathrm{P}\left(A\mid B\right)= \dfrac{2}{3} \mathrm{P}\left(B\mid A\right)$ và $P\left(AB\right) \ne 0$. Tính tỉ số $\dfrac{\mathrm{P}\left(A\right)}{\mathrm{P}\left(B\right)}$.
	\choice
	{$\dfrac{3}{2}$}
	{\True $\dfrac{2}{3}$}
	{$\dfrac{1}{3}$}
	{$\dfrac{1}{2}$}
	\loigiai{
	Theo công thức Bayes ta có\\
	$\mathrm{P}\left(B\mid A\right)=
	\dfrac{\mathrm{P}\left(B\right)\cdot \mathrm{P}\left(A\mid B\right)}{\mathrm{P}\left(A\right)}
	\Leftrightarrow 
	\dfrac{\mathrm{P}\left(A\mid B\right)}{\mathrm{P}\left(B\mid A\right)}
	=\dfrac{\mathrm{P}\left(A\right) }{\mathrm{P}\left(B\right)}=\dfrac{2}{3}$. 
	}
\end{ex}
%%%%%----------Câu 12
\begin{ex}%[2D6H2-2]%[Võ Thanh Hiệp]
	Cho hai biến cố ngẫu nhiên $A$ và $B$. Biết rằng $\mathrm{P}\left(A\right)= \dfrac{3}{5}$; $\mathrm{P}\left(B\mid A\right)= \dfrac{1}{4}$ và $\mathrm{P}\left(B\mid \overline{A}\right)= \dfrac{1}{3}$. Tính $\mathrm{P}\left(B\overline{A}\right)$.
	\choice
	{$\dfrac{43}{180}$}
	{\True$\dfrac{2}{15}$}
	{$\dfrac{3}{15}$}
	{$\dfrac{17}{180}$}
	\loigiai{
	Ta có $\mathrm{P}\left(A\right)= \dfrac{3}{5}\Rightarrow \mathrm{P}\left(\overline{A}\right)=\dfrac{2}{5}$.\\
	Áp dụng công thức xác suất toàn phần ta có\\
	$\mathrm{P}\left(B\right)=\mathrm{P}\left(A\right)\cdot \mathrm{P}\left(B\mid A\right)
	+\mathrm{P}\left(\overline{A}\right)\cdot \mathrm{P}\left(B\mid \overline{A}\right)
	=\dfrac{3}{5}\cdot\dfrac{1}{4}+\dfrac{2}{5}\cdot\dfrac{1}{3}=\dfrac{17}{60}$.\\
	Áp dụng công thức nhân xác suất ta có\\
	$\mathrm{P}\left(B\overline{A}\right)=\mathrm{P}\left(\overline{A}\right)\cdot \mathrm{P}\left(B\mid \overline{A}\right)
	=\dfrac{2}{5}\cdot\dfrac{1}{3}=\dfrac{2}{15}$.
	}
\end{ex}
%%==========Câu 6
\begin{ex}%[2D6H1-2]
	Cho $\mathrm{P}(A)=\dfrac{2}{5}$; $\mathrm{P}\left( B\mid A\right)=\dfrac{1}{3}$; $\mathrm{P}\left(B\mid \overline{A}\right)=\dfrac{1}{4}$. Giá trị của $\mathrm{P}(B)$ là 
	\choice
	{$\dfrac{19}{60} $}
	{\True $ \dfrac{17}{60}$}
	{$ \dfrac{9}{20}$}
	{$ \dfrac{7}{30}$}
	\loigiai{Áp dụng công thức Bayes ta có
	$$\mathrm{P}(A\mid B)=\dfrac{\mathrm{P}(A)\cdot \mathrm{P}(B\mid A)}{\mathrm{P}(A)\cdot \mathrm{P}(B\mid A)+\mathrm{P}\left( \overline{A}\right) \cdot \mathrm{P}\left( B\mid \overline{A}\right) }=\dfrac{\dfrac{2}{5}\cdot \dfrac{1}{3}}{\dfrac{2}{5}\cdot \dfrac{1}{3}+\dfrac{3}{5}\cdot \dfrac{1}{4}}=\dfrac{8}{17}.$$
	Khi đó 
	$\mathrm{P}(B)=\dfrac{\mathrm{P}(AB)}{\mathrm{P}(A\mid B)}=\dfrac{2}{15}:\dfrac{8}{17}=\dfrac{17}{60}$.
	}
\end{ex}
%%==========Câu 7
\begin{ex}%[2D6H1-2]
	An có một túi gồm một số chiếc kẹo cùng loại, chỉ khác màu, trong đó có $6$ chiếc kẹo sô-cô-la đen, còn lại là $4$ chiếc kẹo sô-cô-la trắng. An lấy ngẫu nhiên $1$ chiếc kẹo trong túi để cho Bình, rồi lại lấy ngẫu nhiên tiếp $1$ chiếc kẹo nữa trong túi và cũng đưa cho Bình. Xác suất để Bình nhận được $2$ chiếc kẹo sô-cô-la đen là 
	\choice
	{\True $\dfrac{1}{3} $}
	{$ \dfrac{1}{4}$}
	{$ \dfrac{2}{5}$}
	{$ \dfrac{3}{7}$}
	\loigiai{
	Gọi $A$ là biến cố \lq\lq  An lấy lần $1$ được $1$ chiếc kẹo sô-cô-la đen\rq\rq\, và $B$ là biến cố \lq\lq  An lấy lần $2$ được $1$ chiếc kẹo sô-cô-la đen\rq\rq.\\
	Khi đó $AB$ là biến cố \lq\lq  Cả hai lần đều lấy được kẹo sô-cô-la đen\rq\rq.\\
	Ta có $\mathrm{P}(A)=\dfrac{n(A)}{n(\Omega)}=\dfrac{3}{5}$.\\
	Sau khi lấy $1$ chiếc kẹo sô-cô-la đen thì xác suất để chọn $1$ chiếc kẹo sô-cô-la đen trong hộp đựng $5$ chiếc kẹo sô-cô-la đen, còn lại là $4$ chiếc kẹo sô-cô-la trắng là $\mathrm{P}(B\mid A)=\dfrac{5}{9}$.\\
	Khi đó $\mathrm{P}(AB)=\mathrm{P}(A)\cdot \mathrm{P}(B\mid A)=\dfrac{3}{5}\cdot \dfrac{5}{9}=\dfrac{1}{3}$.\\
	Xác suất để Bình nhận được $2$ chiếc kẹo sô-cô-la đen là $\dfrac{1}{3}$.
	}
\end{ex}
%%==========Câu 8
\begin{ex}%[0D0H2-9]
	Người ta nhập hai lô hàng vào kho. Lô thứ nhất chứa $10$ sản phẩm, trong đó có $3$ phế phẩm. Lô thứ hai có $4$ phế phẩm và $8$ sản phẩm tốt. Chọn ngẫu nhiên một sản phẩm. Xác suất chọn được một sản phẩm tốt là
	\choice
	{\True $\dfrac{15}{22}$}
	{$\dfrac{7}{15}$}
	{$\dfrac{7}{22}$}
	{$\dfrac{83}{242}$}
	\loigiai{
	Gọi $A$ là biến cố “chọn được sản phẩm tốt”, theo đề bầi, kho có $22$ sản phẩm, trong đó có $15$ sản phẩm tốt nên $n(A)=15$, $n(\Omega)=22.$\\
	Vậy $\mathrm{P}(A)=\dfrac{n(A)}{n(\Omega)}=\dfrac{15}{22}$.
	}
\end{ex}
%%==========Câu 9
\begin{ex}%[2D6V2-2]
	Cho $A$, $B$ là các biến cố thỏa mãn $\mathrm{P}(\overline{A}\cap \overline{B})=0{,}35$; $\mathrm{P}(A)=0{,}25$; $\mathrm{P}(B)=0{,}6$. Giá trị của $\mathrm{P}(A\mid B)$ bằng
	\choice
	{$\dfrac{1}{5}$}
	{\True$\dfrac{1}{3}$}
	{$\dfrac{7}{15}$}
	{$\dfrac{2}{3}$}
	\loigiai{
	Ta có $$\mathrm{P}(\overline{A}\cap \overline{B}) = \mathrm{P} \left(\overline A \right) \mathrm{P} \left( \overline B \mid \overline A \right) \Rightarrow \mathrm{P} \left(\overline B \mid \overline A \right) = \dfrac{\mathrm{P}(\overline{A}\cap \overline{B})}{\mathrm{P} \left( \overline A\right)} = \dfrac{0{,}35}{0{,}75} = \dfrac{7}{15}.$$
	Suy ra $$\mathrm{P} \left(B\mid \overline A\right) = 1 - \dfrac{7}{15} = \dfrac{8}{15}.$$
	Theo công thức xác suất toàn phần, ta có
	\begin{eqnarray*}
	&&	\mathrm{P}\left( B \right) = \mathrm{P}\left(B\mid A\right)\mathrm{P}\left( A \right) + \mathrm{P}\left( B\mid \overline A \right)\mathrm{P}\left( \overline A \right)\\
	&\Rightarrow& \mathrm{P}\left( B\mid A \right) = \dfrac{\mathrm{P}\left( B \right) - \mathrm{P} \left( {B\mid \overline A } \right)\mathrm{P}\left( {\overline A } \right)}{\mathrm{P}\left( A \right)} = \dfrac{0{,}6 - \dfrac{8}{{15}} \cdot 0{,}75}{0{,}25} = 0{,}8.	
	\end{eqnarray*}
	Theo công thức Bayes, ta được
	$$\mathrm{P}\left( {A\mid B} \right) = \dfrac{{\mathrm{P}\left( A \right)\mathrm{P}\left( {B\mid A} \right)}}{{\mathrm{P}\left( B \right)}} = \dfrac{{0{,}25 \cdot 0{,}8}}{{0{,}6}} = \dfrac{1}{3}.$$}
\end{ex}
%%==========Câu 10
\begin{ex}%[2D6V1-3]
	Một bệnh viện có hai phòng khám là phòng A và phòng B với khả năng lựa chọn của bệnh nhân là như nhau. Tỉ lệ bệnh nhân nam có ở phòng A và phòng B lần lượt là $60\%$ và $40\%$. Một người bệnh được chọn ngẫu nhiêu từ hai phòng khám và biết người này là nam, xác suất để người bệnh được chọn đến từ phòng A là
	\choice
	{\True $0{,}6$}
	{$0{,}5$}
	{$0{,}4$}
	{$0{,}3$}
	\loigiai{Một người bệnh được chọn ngẫu nhiên từ hai phòng khám.\\
	Gọi $X$ là biến cố \lq \lq Người đó đến từ phòng khám A\rq \rq \, và $Y$, $\overline{Y}$ lần lượt là biến cố \lq \lq Người đó là nam\rq \rq \; và \lq \lq Người đó không là nam\rq \rq.\\
	Ta có sơ đồ hình cây sau
	\begin{center}
		\begin{tikzpicture}[>=stealth,scale=0.8]
	%Khung 1
	\draw (-2,-1) rectangle (2.2,0);
	\draw (0.1,-0.5) node{Bệnh nhân được chọn} ;
	%Mui ten 1,2
	\draw [->] (2.2,-0.5)--(3.8,1.6) node[pos=0.5,sloped,above]{$0{,}5$};
	\draw [->] (2.2,-0.5)--(3.8,-2.6) node[pos=0.5,sloped,below]{$0{,}5$};
	%Khung 2.1
	\draw (3.8,1.1) rectangle (5.1,2.1);
	\draw (8.9/2,1.6) node{$X$} ;
	%Khung 2.2
	\draw (3.8,-2.1) rectangle (5.1,-3.1);
	\draw (8.9/2,-2.6) node{$\overline{X}$} ;
	%Mui ten 3,4
	\draw [->] (5.1,1.6)--(6.5,2.6) node[pos=0.5,sloped,above]{$0{,}6$};
	\draw [->] (5.1,1.6)--(6.5,0.6) node[pos=0.5,sloped,below]{$0{,}4$};
	%Mui ten 5,6
	\draw [->] (5.1,-2.6)--(6.5,-1.6) node[pos=0.5,sloped,above]{$0{,}4$};
	\draw [->] (5.1,-2.6)--(6.5,-3.6) node[pos=0.5,sloped,below]{$0{,}6$};
	%Khung 3.1
	\draw (6.5,2.2) rectangle (7.7,3.2);
	\draw (7.1,5.4/2) node{$Y$} ;
	%Khung 3.2
	\draw (6.5,1.2) rectangle (7.7,0.2);
	\draw (7.1,1.4/2) node{$\overline{Y}$} ;
	%Khung 3.3
	\draw (6.5,-1.1) rectangle (7.7,-2.1);
	\draw (7.1,-3.2/2) node{$Y$} ;
	%Khung 3.3
	\draw (6.5,-2.9) rectangle (7.7,-3.9);
	\draw (7.1,-3.4) node{$\overline{Y}$} ;
	%Kết quả
	\draw (9.5,3.7) node{\textbf{Kết quả}};	
	\draw (9.5,2.7) node{$XY$};
	\draw (9.5,0.7) node{$X \overline{Y}$};
	\draw (9.5,-1.6) node{$\overline{X}Y$};
	\draw (9.5,-3.4) node{$\overline{X}\overline{Y}$};
	%Xác suất
	\draw (12.5,3.7) node{\textbf{Xác suất}};	
	\draw (12.5,2.7) node{$0{,}3$};
	\draw (12.5,0.7) node{$0{,}2$};
	\draw (12.5,-1.6) node{$0{,}2$};
	\draw (12.5,-3.4) node{$0{,}3$};	
	\end{tikzpicture}
	\end{center}
	\noindent Theo công thức Bayes, ta có $$\mathrm{P}(X\mid Y)=\dfrac{\mathrm{P}(X)\mathrm{P}(Y\mid X)}{\mathrm{P}(X)\mathrm{P}(Y\mid X)+\mathrm{P}(\overline{X})\mathrm{P}(Y\mid \overline{X})}=\dfrac{0{,}3}{0{,}3+0{,}2}=0{,}6.$$
	Vậy với một người bệnh được chọn ngẫu nhiêu từ hai phòng khám và biết người này là nam, xác suất để người đó đến từ phòng A là $0{,}6$.}
\end{ex}
%%==========Câu 11
\begin{ex}%[2D6V1-3]
	Ở một địa phương $X$, xác suất để một người lớn trên $40$ tuổi mắc bệnh ung thư là $0{,}05$. Xác suất bác sĩ chẩn đoán đúng một người mắc bệnh ung thư là $0{,}78$ và chẩn đoán sai (không bị ung thư nhưng được chẩn đoán mắc bệnh) là $0{,}06$. Xác suất để một người thật sự mắc bệnh ung thư khi nhận được kết quả chẩn đoán bị ung thư bằng
	\choice
	{\True$0{,}40625$}
	{$0{,}096$}
	{$0{,}904$}
	{$0{,}59375$}
	\loigiai{Một bệnh nhân trên 40 tuổi ở địa phương X đến bác sĩ để khám bệnh ung thư.\\
	Gọi $A$ là biến cố \lq \lq Người đó mắc bệnh ung thư\rq \rq \, và $B$, $\overline{B}$ lần lượt là biến cố \lq \lq Bác sĩ chẩn đoán người đó bị ung thư\rq \rq \;và \lq \lq Bác sĩ chẩn đoán người đó không bị ung thư\rq \rq.\\
	Ta xét sơ đồ hình cây như sau
	\begin{center}
		\begin{tikzpicture}[>=stealth,scale=0.8]
	%Khung 1
	\draw (-3.5,-1) rectangle (2.2,0);
	\draw (-1.3/2,-0.5) node{Bệnh nhân được chẩn đoán} ;
	%Mui ten 1,2
	\draw [->] (2.2,-0.5)--(3.8,1.6) node[pos=0.5,sloped,above]{$0{,}05$};
	\draw [->] (2.2,-0.5)--(3.8,-2.6) node[pos=0.5,sloped,below]{$0{,}95$};
	%Khung 2.1
	\draw (3.8,1.1) rectangle (5.1,2.1);
	\draw (8.9/2,1.6) node{$A$} ;
	%Khung 2.2
	\draw (3.8,-2.1) rectangle (5.1,-3.1);
	\draw (8.9/2,-2.6) node{$\overline{A}$} ;
	%Mui ten 3,4
	\draw [->] (5.1,1.6)--(6.5,2.6) node[pos=0.5,sloped,above]{$0{,}78$};
	\draw [->] (5.1,1.6)--(6.5,0.6) node[pos=0.5,sloped,below]{$0{,}22$};
	%Mui ten 5,6
	\draw [->] (5.1,-2.6)--(6.5,-1.6) node[pos=0.5,sloped,above]{$0{,}06$};
	\draw [->] (5.1,-2.6)--(6.5,-3.6) node[pos=0.5,sloped,below]{$0{,}94$};
	%Khung 3.1
	\draw (6.5,2.2) rectangle (7.7,3.2);
	\draw (7.1,5.4/2) node{$B$} ;
	%Khung 3.2
	\draw (6.5,1.2) rectangle (7.7,0.2);
	\draw (7.1,1.4/2) node{$\overline{B}$} ;
	%Khung 3.3
	\draw (6.5,-1.1) rectangle (7.7,-2.1);
	\draw (7.1,-3.2/2) node{$B$} ;
	%Khung 3.3
	\draw (6.5,-2.9) rectangle (7.7,-3.9);
	\draw (7.1,-3.4) node{$\overline{B}$} ;
	%Kết quả
	\draw (9.5,3.7) node{\textbf{Kết quả}};	
	\draw (9.5,2.7) node{$AB$};
	\draw (9.5,0.7) node{$A\overline{B}$};
	\draw (9.5,-1.6) node{$\overline{A}B$};
	\draw (9.5,-3.4) node{$\overline{A}\overline{B}$};
	%Xác suất
	\draw (12.5,3.7) node{\textbf{Xác suất}};	
	\draw (12.5,2.7) node{$0{,}039$};
	\draw (12.5,0.7) node{$0{,}011$};
	\draw (12.5,-1.6) node{$0{,}057$};
	\draw (12.5,-3.4) node{$0{,}893$};
	\end{tikzpicture}
	\end{center}
	Theo công thức Bayes, ta có $$\mathrm{P}(A\mid B)=\dfrac{\mathrm{P}(A)\mathrm{P}(B\mid A)}{\mathrm{P}(A)\mathrm{P}(B\mid A)+\mathrm{P}(\overline{A})\mathrm{P}(B\mid \overline{A})}=\dfrac{0{,}039}{0{,}039+0{,}057}=0{,}40625.$$	 
	Vậy xác suất để một người thật sự mắc bệnh ung thư khi nhận được kết quả chẩn đoán bị ung thư bằng $0{,}40625$.}
\end{ex} 
%%==========Câu 12
\begin{ex}%[2D6V2-3]
	Một loại vaccine được tiêm ở địa phương $X$. Người có bệnh nền thì với xác suất $0{,}35$ có phản ứng phụ sau tiêm, người không có bệnh nền thì chỉ có phản ứng phụ sau tiêm với xác suất $0{,}16$. Chọn ngẫu nhiên một người được tiêm vaccine và người này có phản ứng phụ. Tính xác suất để người này có bệnh nền, biết rằng tỉ lệ người có bệnh nền ở địa phương $X$ là $18\%$.
	\choice
	{\True $\dfrac{315}{971}$ }
	{ $\dfrac{31}{971}$}
	{$0{,}16$} 
	{$0{,}063$}
	\loigiai{
	Gọi $A$ là biến cố \lq\lq  Người được chọn có bệnh nền\rq\rq\, và $B$ là biến cố \lq\lq  Người này có phản ứng phụ sau tiêm\rq\rq.\\
	Ta có $\mathrm{P}(A)=0{,}18$; $\mathrm{P}(\overline{A})=0{,}82$.\\
	$\mathrm{P}(B\mid A)$ là xác suất để một người bệnh có phản ứng sau tiêm với điều kiện có bệnh nền, suy ra $$\mathrm{P}(B\mid A)=0{,}35.$$
	$\mathrm{P}(B\mid \overline{A})$ là xác suất để một người bệnh có phản ứng sau tiêm với điều kiện không có bệnh nền, suy ra $$\mathrm{P}(B\mid \overline{A})=0{,}16.$$
	Theo công thức Bayes, ta được 
	$$\mathrm{P}(A\mid B)=\dfrac{\mathrm{P}(A)\cdot \mathrm{P}(B\mid A)}{\mathrm{P}(A)\cdot \mathrm{P}(B\mid A)+\mathrm{P}(\overline{A})\cdot \mathrm{P}(B\mid \overline{A})}=\dfrac{0{,}18\cdot 0{,}35}{0{,}18\cdot 0{,}35+0{,}82\cdot 0{,}16}=\dfrac{315}{971}.$$
	}
\end{ex}
%==============================================================
\Closesolutionfile{ans}
\indapan{6}{ans/ans-0-B15}
%%==========HẾT PHẦN 1=========================================
\Opensolutionfile{ans}[ans/ans-0-B15-DS]

%%%%%----------Câu 13
\begin{ex}%[2D6V1-2]%[Nguyễn Khánh Trọng]
	Một nhà máy thực hiện khảo sát toàn bộ công nhân về sự hài lòng của họ về điều kiện làm việc tại phân xưởng. Kết quả khảo sát như sau:
	\begin{center}
	\begin{tabular}{|c|ccll|}
	\hline
	\multirow{2}{*}{Khảo sát công nhân} & \multicolumn{4}{c|}{Kết quả khảo sát} \\ \cline{2-5} 
	& \multicolumn{1}{c|}{Hài lòng} & \multicolumn{3}{c|}{Không hài lòng} \\ \hline
	Số công nhân xưởng I & \multicolumn{1}{c|}{23} & \multicolumn{3}{c|}{12} \\ \hline
	Số công nhân xưởng II & \multicolumn{1}{c|}{25} & \multicolumn{3}{c|}{15} \\ \hline
	\end{tabular}
	\end{center}
	Gặp ngẫu nhiên một công nhân của nhà máy. Gọi $A$ là biến cố \lq\lq  Công nhân đó làm việc tại phân xưởng I\rq\rq \, và $B$ là biến cố \lq\lq  Công nhân đó hài lòng với điều kiện làm việc tại phân xưởng\rq\rq. Xét tính đúng sai của các phát biểu sau.
	\choiceTF
	{\True Xác suất của biến cố $A$ là $\dfrac{7}{15}$}
	{Xác suất của biến cố $B$ là $0{,}65$}
	{\True Xác suất gặp được công nhân không hài lòng với điều kiện làm việc tại phân xưởng biết công nhân đó thuộc xưởng I là $\dfrac{12}{35}$}
	{Xác suất gặp được công nhân thuộc xưởng II biết công nhân đó hài lòng với điều kiện làm việc tại phân xưởng là $0{,}52$}
	\loigiai{
	Gặp ngẫu nhiên một công nhân của nhà máy, ta có $n(\Omega)=23+12+25+15=75$. 
	\begin{itemchoice}
	\itemch {\bf Đúng.}\\
	Ta có $n(A)=23+12=35$.\\
	Xác suất của biến cố $A$ là 
	$\mathrm{P}(A)=\dfrac{35}{75}=\dfrac{7}{15}$.
	\itemch {\bf Sai.} \\
	Ta có $n(B)=23+25=48$. \\
	Xác suất của biến cố $B$ là 
	$\mathrm{P}(B)=\dfrac{48}{75}=\dfrac{16}{25}$.
	\itemch Đúng. Ta có $n(\overline{B}A)=12\Rightarrow \mathrm{P}(\overline{B}A)=\dfrac{12}{75}=\dfrac{4}{25}$.\\
	Do đó
	$\mathrm{P}(\overline{B}\mid A)=\dfrac{\mathrm{P}(\overline{B}A)}{\mathrm{P}(A)}=\dfrac{\dfrac{4}{25}}{\dfrac{7}{15}}=\dfrac{12}{35}$.
	\itemch {\bf Sai.}\\
	Ta có $n(\overline{A}B)=25\Rightarrow\mathrm{P}(\overline{A}B)
	=\dfrac{25}{75}=\dfrac{1}{3}$.\\
	Do đó
	$\mathrm{P}(\overline{A}\mid B)=\dfrac{\mathrm{P}(\overline{A}B)}{\mathrm{P}(B)}
	=\dfrac{\dfrac{1}{3}}{\dfrac{16}{25}}=\dfrac{25}{48}$.
	\end{itemchoice}
	}
\end{ex}
%%%%%----------Câu 14
\begin{ex}%[2D6H1-2]%[Nguyễn Khánh Trọng]
	Cho hai biến cố $A$ và $B$ có $P(A) = 0{,}35$; $P(B \mid A)= 0{,}6$ và $P(B \mid \overline{A})= 0{,}2$. Mỗi phát biểu dưới đây đúng hay sai?
	\choiceTF
	{\True $\mathrm{P}(\overline{A})=0{,}65$}
	{$\mathrm{P}(AB)=0{,}2$}
	{\True $\mathrm{P}(\overline{B}\mid A)=0{,}4$}
	{\True $\mathrm{P}(B)=0{,}34$}
	\loigiai{
	\begin{itemchoice}
	\itemch {\bf Đúng.}\\
	Vì $\mathrm{P}(\overline{A})=1-\mathrm{P}(A)=1-0{,}35=0{,}65$.
	\itemch {\bf Sai.}\\
	Vì $\mathrm{P}(AB)=\mathrm{P}(B\mid A)\cdot\mathrm{P}(A)
	= 0{,}35\cdot0{,}6=0{,}21$.
	\itemch {\bf Đúng.}\\
	Vì $\mathrm{P}(\overline{B}\mid A)
	=1-\mathrm{P}(B\mid A)=1-0{,}6=0{,}4$.
	\itemch {\bf Đúng.} \\
	Vì $\mathrm{P}(B)=\mathrm{P}(A)\cdot\mathrm{P}(B\mid A)
	+\mathrm{P}(\overline{A})\cdot\mathrm{P}(B\mid\overline{A})
	=0{,}35\cdot 0{,}6+0{,}65\cdot0{,}2=0{,}34$.
	\end{itemchoice}
	}
\end{ex}
%%%%%----------Câu 15
\begin{ex}%[2D6H1-3]%[Nguyễn Khánh Trọng]
	\immini{Cho sơ đồ cây như hình vẽ. Xét tính đúng sai của các phát biểu sau.
	\choiceTF
	{\True $ \mathrm{P}\left(A\right)= 0{,}25$}
	{$\mathrm{P}\left(A\overline{B}\right)=\dfrac{1}{8}$}
	{$\mathrm{P}\left(B\right)= 0{,}65$}
	{\True $\mathrm{P}\left(A\mid B\right)=0{,}16$}
	}{
	\tikzstyle{xs} = [rectangle ,fill=white,draw=black,rounded corners,align=center]
	\tikzstyle{bc} = [circle ,fill=white,draw=black,rounded corners,align=center]
	\begin{tikzpicture}[scale=1.2,>=stealth, font=\footnotesize, line join=round, line cap=round]
	\node[bc,text width=2.5mm] (O) at(0,0) { };
	\node[bc] (A) at ($(O)+(30:3)$) {$A$};
	\node[bc] (A1) at ($(O)+(-30:3)$){$\overline{A}$};
	\node[bc] (B) at ($(A)+(20:3)$) {$B$};
	\node[bc] (B1) at ($(A)+(-20:3)$) {$\overline{B}$};
	\node[bc] (B2) at ($(A1)+(20:3)$) {$B$};
	\node[bc] (B3) at ($(A1)+(-20:3)$) {$\overline{B}$};
	\draw[->] {(O)--node [above,xs,sloped] {$?$}(A)} ;
	\draw[->] {(O)--node [below,xs,sloped] {$\mathrm{P}(\overline{A})=0{,}75$}(A1)} ;
	\draw[->] {(A)--node [above,xs,sloped] { $\mathrm{P}(B\mid A)=0{,}4$}(B)} ;
	\draw[->] {(A)--node [below,xs,sloped] {$?$}(B1)} ;
	\draw[->] {(A1)--node [above,xs,sloped] {$?$}(B2)} ;
	\draw[->] {(A1)--node [below,xs,sloped] {$\mathrm{P}(\overline{B}\mid \overline{A})=0{,}3$}(B3)} ;
	\end{tikzpicture}
	}
	\loigiai{
	\begin{center}	
	\tikzstyle{xs} = [rectangle ,fill=white,draw=black,rounded corners,align=center]
	\tikzstyle{bc} = [circle ,fill=white,draw=black,rounded corners,align=center]
	\begin{tikzpicture}[scale=1,>=stealth, font=\footnotesize, line join=round, line cap=round]
	\node[bc,text width=2.5mm] (O) at(0,0) { };
	\node[bc] (A) at ($(O)+(32:3)$) {$A$};
	\node[bc] (A1) at ($(O)+(-32:3)$){$\overline{A}$};
	\node[bc] (B) at ($(A)+(20:3)$) {$B$};
	\node[bc] (B1) at ($(A)+(-20:3)$) {$\overline{B}$};
	\node[bc] (B2) at ($(A1)+(20:3)$) {$B$};
	\node[bc] (B3) at ($(A1)+(-20:3)$) {$\overline{B}$};
	\draw[->] {(O)--node [above,xs,sloped] {$0{,}25$}(A)} ;
	\draw[->] {(O)--node [below,xs,sloped] {$0{,}75$}(A1)} ;
	\draw[->] {(A)--node [above,xs,sloped] {$0{,}4$}(B)} ;
	\draw[->] {(A)--node [below,xs,sloped] {$0{,}6$}(B1)} ;
	\draw[->] {(A1)--node [above,xs,sloped] {$0{,}7$}(B2)} ;
	\draw[->] {(A1)--node [below,xs,sloped] {$0{,}3$}(B3)} ;
	\node[bc] (AB) at ($(B)+(0:2)$) {$AB$};
	\node[bc] (AB1) at ($(B1)+(0:2)$) {$A\overline{B}$};
	\node[bc] (AB2) at ($(B2)+(0:2)$) {$\overline{A}B$};
	\node[bc] (AB3) at ($(B3)+(0:2)$) {$\overline{A}\,\overline{B}$};
	\end{tikzpicture}
	\end{center}
	Từ sơ đồ cây suy ra\\
	\begin{itemchoice}
	\itemch {\bf Đúng.}\\
	$\mathrm{P}\left(A\right) = 1 - \mathrm{P}\left(\overline{A}\right) = 1- 0{,}75 = 0{,}25$.
	\itemch {\bf Sai.}\\
	Vì $\mathrm{P}\left(A \overline{B}\right)
	=\mathrm{P}\left(A\right) \cdot \mathrm{P}\left(\overline{B}\mid A\right)
	= 0{,}25\cdot 0{,}6=0{,}15$.
	\itemch {\bf Sai.}\\
	Ta có $\mathrm{P}\left(B\right)= \mathrm{P}\left(A\right)\cdot \mathrm{P}(B\mid A)
	+ \mathrm{P}\left(\overline{A}\right)\cdot \mathrm{P}(B\mid \overline{A})
	= 0{,}25\cdot 0{,}4 + 0{,}75\cdot 0{,}7 = 0{,}625$.
	\itemch {\bf Đúng.}\\
	Ta có $\mathrm{P}\left(A B\right)= \mathrm{P}\left(A\right)\cdot \mathrm{P}\left(B\mid A\right)
	=0{,}25\cdot0{,}4=0{,}1$.\\
	Suy ra $\mathrm{P}\left(A\mid B\right)= \dfrac{\mathrm{P}\left(AB\right)}{\mathrm{P}\left(B\right)}
	=\dfrac{0{,}1}{0{,}625}=0{,}16$.
	\end{itemchoice}
	}
\end{ex}
%%%%%----------Câu 16
\begin{ex}%[2D6V2-2]%[Nguyễn Khánh Trọng]
	Trong một trường học, tỉ lệ học sinh nữ là $55\%$. Tỉ lệ học sinh nữ và tỉ lệ học sinh nam tham gia câu lạc bộ tiếng anh lần lượt là $20\%$ và $15\%$. Gặp ngẫu nhiên $1$ học sinh của trường. 
	Gọi $A$ là biến cố \lq\lq  Học sinh đó là nữ\rq\rq\, và $B$ là biến cố \lq\lq  Học sinh đó tham gia câu lạc bộ tiếng Anh\rq\rq. Xét tính đúng sai của các phát biểu sau.
	\choiceTF
	{\True $\mathrm{P}(\overline{A})=0{,}45$}
	{$\mathrm{P}(B\mid \overline{A}) = 0{,}15$ và $\mathrm{P}(\overline{B}\mid A) = 0{,}2$}
	{Xác suất để học sinh đó có tham gia câu lạc bộ tiếng Anh là $0{,}1675$}
	{\True Biết rằng học sinh có tham gia câu lạc bộ tiếng Anh. Xác suất học sinh đó là nam bằng $\dfrac{27}{71}$}
	\loigiai{
	\begin{itemchoice}
	\itemch {\bf Đúng.}\\
	Do tỉ lệ học sinh nữ là $55\%$ nên
	$\mathrm{P}(A) = 0{,}55$ và $\mathrm{P}(\overline{A}) = 1 - 0{,}55 = 0{,}45$.
	\itemch {\bf Sai.}\\
	Do tỉ lệ học sinh nữ và tỉ lệ học sinh nam tham gia câu lạc bộ tiếng Anh lần lượt là $20\%$ và $15\%$ nên $\mathrm{P}(B\mid A) = 0{,}2$ và $\mathrm{P}(B\mid \overline{A}) = 0{,}15$.\\
	Suy ra $\mathrm{P}(\overline{B}\mid A) =1-\mathrm{P}(B\mid A)=1-0{,}2=0{,}8$.
	\itemch {\bf Sai.}\\
	Xác suất để học sinh đó có tham gia câu lạc bộ tiếng Anh là
	$$\mathrm{P}(B) = \mathrm{P}(A)\cdot\mathrm{P}(B\mid A) + \mathrm{P}(\overline{A})\cdot\mathrm{P}(B\mid\overline{A}) = 0{,}55 \cdot 0{,}2 + 0{,}45 \cdot 0{,}15 = 0{,}1775.$$
	\itemch {\bf Đúng.}\\
	Do học sinh có tham gia câu lạc bộ tiếng Anh nên xác suất học sinh đó là nam là
	$$\mathrm{P}(\overline{A}\mid B) = \dfrac{\mathrm{P}(\overline{A})\cdot \mathrm{P}(B\mid \overline{A})}{\mathrm{P}(B)}=
	\dfrac{0{,}45\cdot 0{,}15}{0{,}1775}=\dfrac{27}{71}.$$
	\end{itemchoice}
	}
\end{ex}
%%==========Câu 13
\begin{ex}%[2D6H1-2]
	Lớp $12A$ có $40$ học sinh, trong đó có $25$ học sinh tham gia câu lạc bộ Tiếng Anh, $16$ học sinh tham gia câu lạc bộ Toán, $12$ học sinh vừa tham gia câu lạc bộ tiếng Anh vừa tham gia câu lạc bộ Toán. Chọn ngẫu nhiên $1$ học sinh. Xét các biến cố sau\\
	$A\colon$ \lq\lq  Học sinh được chọn tham gia câu lạc bộ Tiếng Anh\rq\rq;\\
	$B\colon$ \lq\lq  Học sinh được chọn tham gia câu lạc bộ Toán\rq\rq.\\
	Xét tính đúng, sai của các khẳng định sau
	\choiceTF
	{$\mathrm{P}(A)=0{,}4$}
	{$\mathrm{P}(B)=0{,}625$}
	{\True $\mathrm{P}(A | B)=0{,}75$}
	{\True $\mathrm{P}(B | A)=0{,}48$}
	\loigiai{
	\begin{itemchoice}
	\itemch Sai, vì xác suất của biến cố $A$ là $\mathrm{P}(A)=\dfrac{25}{40}=0{,}625$.
	\itemch Sai, vì xác suất của biến cố $B$ là $\mathrm{P}(B)=\dfrac{16}{40}=0{,}4$.
	\itemch Đúng, vì số học sinh vừa tham gia câu lạc bộ tiếng Anh vừa tham gia câu lạc bộ Toán là $12$, số học sinh tham gia câu lạc bộ Toán là $16$ nên $\mathrm{P}(A | B)=\dfrac{12}{16}=0{,}75$.
	\itemch Đúng, vì số học sinh vừa tham gia câu lạc bộ tiếng Anh vừa tham gia câu lạc bộ Toán là $12$, số học sinh tham gia câu lạc bộ Tiếng Anh là $25$ nên $\mathrm{P}(B | A)=\dfrac{12}{25}=0{,}48$.
	\end{itemchoice}
	}
\end{ex}
%%==========Câu 14
\begin{ex}%[2D6V1-3]
	\immini{Cho sơ đồ hình cây như hình bên. Xét tính đúng, sai của các khẳng định sau
	\choiceTF
	{$\mathrm{P(AB)}=0{,}48$}
	{ $\mathrm{P(A| B)}=0{,}5$}
	{$\mathrm{P(\overline{A}| B)}=0{,}3$}
	{\True $\dfrac{\mathrm{P}(B) \mathrm{P}(\overline{A} | B)}{\mathrm{P}(\overline{A})}=0{,}6$}}
	{\begin{tikzpicture}[scale=.2,>=stealth]
	\tikzstyle{block} = [rectangle, draw, fill=blue!10\text{,} rounded corners, text centered, text width = 10em, minimum height = 2em]
	\node (c1) {};
	\node (c2)[above right = 1.5cm of c1] {$A$};
	\node at (0.5,5){\fbox{$0\text{,}2$}};
	\node at (0.5,-5){\fbox{$0\text{,}8$}};
	\node (c3) [below right= 1.5cm of c1]{$\overline{A}$};
	\node at (12,11.5){\fbox{$0\text{,}7$}};
	\node (c4) at (21.5, 12){$B$};
	\node (c5) at (21.5, 2){$\overline{B}$};
	\node at (12,3){\fbox{$0\text{,}3$}};
	\node (c6) at (21.5, -4){$B$};
	\node at (12,-4){\fbox{$0\text{,}6$}};
	\node (c7) at (21.5, -14){$\overline{B}$};
	\node at (12,-13){\fbox{$0\text{,}4$}};
	\draw[->] (c1.east) -- (c2.west);
	\draw[->] (c1.east) -- (c3.west);
	\draw[->] (c2.east) -- (c4.west);
	\draw[->] (c2.east) -- (c5.west);
	\draw[->] (c3.east) -- (c6.west);
	\draw[->] (c3.east) -- (c7.west);
	\end{tikzpicture}}
	\loigiai{
	\begin{itemchoice}
	\itemch Sai, vì $\mathrm{P}(B)=0\text{,}2\cdot 0\text{,}7+0\text{,}8\cdot 0\text{,}6=0\text{,}62$, $\mathrm{P}(\overline{A})=0\text{,}8$ và $\mathrm{P}(AB)=0\text{,}2\cdot 0\text{,}7=0\text{,}14$.
	\itemch Sai, vì $\mathrm{P}(A| B)=\dfrac{\mathrm{P}(AB)}{\mathrm{P}(B)}=\dfrac{0\text{,}14}{0\text{,}62}=\dfrac{7}{31}$.
	\itemch Sai, vì $\mathrm{P}(\overline{A} | B)=1-\mathrm{P}(A| B)=1-\dfrac{7}{31}=\dfrac{24}{31}$.
	\itemch Đúng, vì $\dfrac{\mathrm{P}(B) \mathrm{P}(\overline{A} | B)}{\mathrm{P}(\overline{A})}=\dfrac{0\text{,}62\cdot \dfrac{24}{31}}{0\text{,}8}=0\text{,}6$.
	\end{itemchoice}	
	}
\end{ex}
%%==========Câu 15
\begin{ex}%[2D6V1-3]
	Trong một hộp có $18$ quả bóng bàn loại I và $2$ quả bóng bàn loại II, các quả bóng bàn có hình dạng và kích thước như nhau. Một học sinh lấy ngẫu nhiên lần lượt $2$ quả bóng bàn (lấy không hoàn lại) trong hộp.\\ Xét tính đúng, sai của các khẳng định sau
	\choiceTF
	{Xác suất để lần thứ nhất lấy được quả bóng bàn loại II là $\dfrac{9}{10}$}
	{\True Xác suất để lần thứ hai lấy được quả bóng bàn loại II, biết lần thứ nhất lấy được quả bóng bàn loại II, là $\dfrac{1}{19}$}
	{Xác suất để cả hai lần đều lấy được quả bóng bàn loại II là $\dfrac{9}{190}$}
	{\True Xác suất để ít nhất $1$ lần lấy được quả bóng bàn loại I là $\dfrac{189}{190}$}
	\loigiai{
	Xét các biến cố\\
	$A\colon$ \lq\lq  Lần thứ nhất lấy được quả bóng bàn loại II\rq\rq;\\
	$B\colon$ \lq\lq  Lần thứ hai lấy được quả bóng bàn loại II\rq\rq. 
	\begin{itemchoice}
	\itemch Sai, vì Xác suất để lần thứ nhất lấy được quả bóng bàn loại II là $\mathrm{P}(A)=\dfrac{2}{20}=\dfrac{1}{10}$.
	\itemch Đúng, vì sau khi lấy $1$ quả bóng bàn loại II thì chỉ còn $1$ quả bóng bàn loại II trong hộp.\\ Suy ra xác suất để lần thứ hai lấy được quả bóng bàn loại II, biết lần thứ nhất lấy được quả bóng bàn loại II là $\mathrm{P}(B | A)=\dfrac{1}{19}$.
	\itemch Sai, vì xác suất để cả hai lần đều lấy được quả bóng bàn loại II là
	$$
	\mathrm{P}(C)=\mathrm{P}(A \cap B)=\mathrm{P}(A) \cdot \mathrm{P}(B | A)=\dfrac{1}{10} \cdot \dfrac{1}{19}=\dfrac{1}{190}.
	$$
	\itemch Đúng, vì để ít nhất $1$ lần lấy được quả bóng bàn loại I là
	$$
	\mathrm{P}(\overline {C})=1-\mathrm{P}(C)=1-\frac{1}{190}=\frac{189}{190} \text {. }
	$$
	\end{itemchoice}
	}
\end{ex}
%%==========Câu 16
\begin{ex}%[2D6V2-3]
	Một xưởng máy sử dụng một loại linh kiện được sản xuất từ hai cơ sở I và II. Số linh kiện do cơ sở I sản xuất chiếm $61 \%$, số linh kiện do cơ sở II sản xuất chiếm $39 \%$. Tỉ lệ linh kiện đạt tiêu chuẩn của cơ sở I, cơ sở II lần lượt là $93 \%$, $82 \%$. Kiểm tra ngẫu nhiên $1$ linh kiện ở xường máy. Xét các biến cố\\
	$A_1\colon$ \lq\lq  Linh kiện được kiểm tra do cơ sở I sản xuất\rq\rq;\\
	$A_2\colon$\lq\lq  Linh kiện được kiểm tra do cơ sở II sản xuất\rq\rq;\\
	$B\colon$ \lq\lq  Linh kiện được kiểm tra đạt tiêu chuẩn\rq\rq.\\
	Xét tính đúng, sai của các khẳng định sau
	\choiceTF
	{$\mathrm{P}\left(A_1\right)=0{,}39$} 
	{\True $\mathrm{P}\left(B | A_2\right)=0{,}82$}
	{$\mathrm{P}(B)=0{,}89$}
	{\True $\mathrm{P}\left(A_1 | B\right)=0{,}55$}
	\loigiai{
	\begin{itemchoice}
	\itemch Sai, vì $\mathrm{P}\left(A_1\right)=0{,}61$.
	\itemch Đúng, vì $\mathrm{P}\left(A_2\right)=0{,}39$;~ $\mathrm{P}\left(B | A_1\right)=0{,}93 ;~\mathrm{P}\left(B | A_2\right)=0{,}82$.
	\itemch Đúng, vì theo công thức xác suất toàn phần, ta có
	$$
	\mathrm{P}(B)=\mathrm{P}\left(A_1\right) \cdot \mathrm{P}\left(B | A_1\right)+\mathrm{P}\left(A_2\right) \cdot \mathrm{P}\left(B | A_2\right)=0{,}61\cdot 0{,}93+0{,}39\cdot 0{,}82=0{,}8871.
	$$
	\itemch Sai, vì theo công thức Bayes, ta có $$\mathrm{P}\left(A_1 | B\right)=\dfrac{\mathrm{P}\left(A_1\right) \cdot \mathrm{P}\left(B | A_1\right)}{\mathrm{P}(B)}=\dfrac{0{,}61 \cdot 0{,}93}{0{,}8871} \approx 0{,}64.$$
	\end{itemchoice}
	}
\end{ex}
%==============================================================
\Closesolutionfile{ans}
\indapan{3}{ans/ans-0-B15-DS}
%%==========HẾT PHẦN 2=========================================
\Opensolutionfile{ans}[ans/ans-0-B15-KQ]
%\TNSA
%%==========PHẦN 3=============================================
%%==========Câu 17
\begin{ex}%[2D6V2-2]
	Có hai hộp đựng các viên bi cùng kích thước và khối lượng. Hộp thứ nhất chứa $5$ viên bi đỏ và $5$ viên bi xanh, hộp thứ hai chứa $6$ viên bi đỏ và $4$ viên bi xanh. Lấy ngẫu nhiên một viên bi từ hộp thứ nhất chuyển sang hộp thứ hai, sau đó lấy ra ngẫu nhiên một viên bi từ hộp thứ hai. Tính xác suất để viên bi được lấy ra từ hộp thứ hai là viên bi đỏ (làm tròn đến hàng phần trăm).
	\shortans{$0{,}59$}
	\loigiai{
	Xét phép thử lấy ngẫu nhiên một viên bi từ hộp thứ nhất chuyển sang hộp thứ hai, sau đó lấy ra ngẫu nhiên một viên bi từ hộp thứ hai. Xét các biến cố sau
	\begin{itemize}
	\item $A$ là biến cố \lq \lq Viên bi được lấy ra từ hộp thứ hai là bi đỏ \rq \rq;
	\item $C$ là biến cố \lq \lq Viên bi được lấy ra từ hộp thứ hai là bi của hộp thứ nhất \rq \rq;
	\item $\overline{C}$ là biến cố \lq \lq Viên bi được lấy ra từ hộp thứ hai là bi của hộp thứ hai\rq \rq.
	\end{itemize}
	Sau khi chuyển một viên bi từ hộp thứ nhất sang hộp thứ hai thì hộp thứ hai có $11$ viên bi. Ta có 
	$$\mathrm{P}(C)=\dfrac{1}{11};\, \mathrm{P}(\overline{C})=\dfrac{10}{11}.$$
	Xác suất để viên bi được lấy ra từ hộp thứ hai là bi đỏ của hộp thứ nhất 
	$$\mathrm{P}(A\mid C)=\dfrac{5}{10}=\dfrac{1}{2}.$$
	Xác suất để viên bi được lấy ra từ hộp thứ hai là bi đỏ của hộp thứ hai 
	$$\mathrm{P}(A\mid \overline{C})=\dfrac{6}{10}=\dfrac{3}{5}.$$
	Áp dụng công thức xác suất toàn phần, ta có
	$$\mathrm{P}(A) = \mathrm{P}(C)\cdot \mathrm{P}(A\mid C) + \mathrm{P}(\overline{C})\cdot P(A\mid \overline{C}) =\dfrac{1}{11}\cdot \dfrac{1}{2}+\dfrac{10}{11}\cdot \dfrac{3}{5}=\dfrac{13}{22}\approx 0{,}59.$$
	}
\end{ex}
%%==========Câu 18
\begin{ex} %[2D6H1-4]
	Trong số $40$ học sinh lớp $12$A, có $22$ em đăng kí thi ngành Kinh tế, $25$ em đăng kí thi ngành Luật, $3$ em không đăng kí thi cả hai ngành này. Chọn ngẫu nhiên một học sinh, biết rằng em đó đăng kí thi ngành luật. Tính xác suất để em đó đăng kí thi ngành kinh tế. 	
	\shortans{$0{,}4$}
	\loigiai{	Gọi $A$ và $B$ lần lượt là tập hợp các học sinh đăng kí thi ngành kinh tế và ngành luật.
	Ta có $|A\cup B| =40-3 = 37$.\\
	Số sinh viên đăng kí cả hai ngành là $ |A\cap B| = |A|+|B|-|A\cup B|=10$.\\
	Vậy chọn ngẫu nhiên một học sinh, biết rằng em đó đăng kí thi ngành luật thì xác suất để em đó đăng kí thi ngành kinh tế là	 $\dfrac{10}{25}=\dfrac{2}{5}=0{,}4$.
	}
\end{ex}
%%==========Câu 19
\begin{ex} %[2D6V2-2]
	Trong một tuần, Sơn chọn ngẫu nhiên ba ngày chạy bộ buổi sáng. Nếu chạy bộ thì xác suất Sơn ăn thêm một quả trứng vào bữa sáng hôm đó là $0{,}7$ . Nếu không chạy bộ thì xác suất Sơn ăn thêm một quả trứng vào bữa sáng hôm đó là $0{,}25$. Chọn ngẫu nhiên một ngày trong tuần của Sơn. Tính xác suất để hôm đó Sơn chạy bộ nếu biết rằng bữa sáng hôm đó Sơn có ăn thêm một quả trứng (làm tròn đến hàng phần trăm).
	\par\shortans{$0{,}68$}	
	\loigiai{
	Gọi $A$	là biến cố \lq \lq Chọn ngày Sơn ăn trứng \rq \rq .\\
	Gọi $B_1$ là biến cố \lq \lq Ngày Sơn chạy bộ \rq \rq và $B_2$ là biến cố \lq \lq Ngày Sơn không chạy bộ \rq \rq.\\
	Ta có $\mathrm{P}(B_1) = \dfrac{3}{7}$ và $\mathrm{P}(B_2)=\dfrac{4}{7}$.\\
	Xác suất ngày ăn trứng và chạy bộ là
	$\mathrm{P}(A\mid B_1) = 0{,}7$.\\
	Xác suất ngày ăn trứng và không chạy bộ là
	$\mathrm{P}(A\mid B_2) = 0{,}25$.\\
	Khi đó ta có $\mathrm{P}(A) =\mathrm{P}(B_1)\cdot \mathrm{P}(A\mid B_1)+ \mathrm{P}(B_2)\cdot \mathrm{P}(A\mid B_2) = \dfrac{3}{7}\cdot 0{,}7+ \dfrac{4}{7}\cdot 0{,}25 = \dfrac{31}{70}$.\\
	Theo công thức Bayes, ta có xác suất hôm chọn Sơn chạy bộ mà trong bữa sáng có ăn một quả trứng là 
	$$\mathrm{P}(B_1\mid A) = \dfrac{\mathrm{P}(B_1A)}{P(A)} = \dfrac{\mathrm{P}(B_1)\cdot \mathrm{P}(A\mid B_1)}{\mathrm{P}(A)} = \dfrac{\dfrac{3}{7}\cdot 0{,}7}{ \dfrac{31}{70}}=\dfrac{21}{31}\approx 0{,}68.$$	
	}
\end{ex}
%%==========Câu 20
\begin{ex}%[2D6V2-2]
	Giả sử có khoảng $40 \%$ thư điện tử (email) gửi đến một địa chỉ là thư rác. Người ta sử dụng một thuật toán để phân loại thư rác, biết rằng thuật toán này có thể phân loại đến $99 \%$ thư rác và tỉ lệ sai sót khi phân loại thư bình thường thành thư rác là $5 \%$. Tính xác suất một thư điện tử là thư bình thường nếu thư này đã được phân loại đúng (làm tròn đến hàng phần trăm).
	\shortans{$0{,}59$}
	\loigiai{
	Ta có công thức
	$$	\mathrm{P}(A \mid B)=\dfrac{\mathrm{P}(B \mid A) \cdot \mathrm{P}(A)}{\mathrm{P}(B)}$$
	Trong đó
	\begin{itemize}
	\item $A$: Thư điện tử là thư bình thường.
	\item $B$: Thư đã được phân loại đúng.
	\item 	$\mathrm{P}(A)$: Xác suất một thư điện tử là thư bình thường ban đầu.\\
	Vì có $40 \%$ thư rác, nên $\mathrm{P}(A)=$ $1-0{,}4=0{,}6$.
	\item $\mathrm{P}(B \mid A)$: Xác suất một thư bình thường được phân loại đúng.\\
	Do tỉ lệ sai sót là $5 \%$, nên $\mathrm{P}(B \mid A)=1-0{,}05=0{,}95$.
	\item $\mathrm{P}(B)$: Xác suất một thư nào đó được phân loại đúng, tính bằng tổng xác suất một thư rác được phân loại đúng và xác suất một thư bình thường được phân loại đúng
	\end{itemize}
	$$\begin{aligned}[t]
	\mathrm{P}(B)&=\mathrm{P}(B \mid A) \cdot \mathrm{P}(A)+\mathrm{P}(B \mid \overline{A}) \cdot \mathrm{P}(\overline{A}) \\&
	=0{,}95 \cdot 0{,}6+0{,}99 \cdot 0{,}4=0{,}97.
	\end{aligned}$$
	Áp dụng định lý Bayes, ta có
	$$\mathrm{P}(A \mid B)=\dfrac{\mathrm{P}(B \mid A) \cdot \mathrm{P}(A)}{\mathrm{P}(B)}=\dfrac{0{,}95 \cdot 0{,}6}{0{,}97}=\dfrac{57}{97}\approx0{,}59.$$
	}
\end{ex}
% \begin{ex}%[2D6V2-1]
% 	\immini{Cho sơ đồ hình cây như hình bên. Tính giá trị của biểu thức $\dfrac{\mathrm{P}(B) \mathrm{P}(\overline{A} \mid B)}{\mathrm{P}(\overline{A})}$.
% 	\par\shortans{$0{,}6$
% 	}
% 	}
% 	{\begin{tikzpicture}[scale=.2,>=stealth]
% 	\tikzstyle{block} = [rectangle, draw, fill=blue!10\text{,} rounded corners, text centered, text width = 10em, minimum height = 2em]
% 	\node (c1) {};
% 	\node (c2)[above right = 1.5cm of c1] {$A$};
% 	\node at (0.5,5){\fbox{$0\text{,}2$}};
% 	\node at (0.5,-5){\fbox{$0\text{,}8$}};
% 	\node (c3) [below right= 1.5cm of c1]{$\overline{A}$};
% 	\node at (12,11.5){\fbox{$0\text{,}7$}};
% 	\node (c4) at (21.5, 12){$B$};
% 	\node (c5) at (21.5, 2){$\overline{B}$};
% 	\node at (12,3){\fbox{$0\text{,}3$}};
% 	\node (c6) at (21.5, -4){$B$};
% 	\node at (12,-4){\fbox{$0\text{,}6$}};
% 	\node (c7) at (21.5, -14){$\overline{B}$};
% 	\node at (12,-13){\fbox{$0\text{,}4$}};
% 	\draw[->] (c1.east) -- (c2.west);
% 	\draw[->] (c1.east) -- (c3.west);
% 	\draw[->] (c2.east) -- (c4.west);
% 	\draw[->] (c2.east) -- (c5.west);
% 	\draw[->] (c3.east) -- (c6.west);
% 	\draw[->] (c3.east) -- (c7.west);
% 	\end{tikzpicture}}
% 	\loigiai{
% 	\begin{itemize}
% 	\item Ta có $\mathrm{P}(B)=0{,}2\cdot 0{,}7+0{,}8\cdot 0{,}6=0{,}62$; $\mathrm{P}(\overline{A})=0{,}8$ và $\mathrm{P}(AB)=0{,}2\cdot 0{,}7=0{,}14$.
% 	\item $\mathrm{P}(A\mid B)=\dfrac{\mathrm{P}(AB)}{\mathrm{P}(B)}=\dfrac{0{,}14}{0{,}62}=\dfrac{7}{31}$.
% 	\item $\mathrm{P}(\overline{A} \mid B)=1-\mathrm{P}(A\mid B)=1-\dfrac{7}{31}=\dfrac{24}{31}$.
% 	\item $\dfrac{\mathrm{P}(B) \mathrm{P}(\overline{A} \mid B)}{\mathrm{P}(\overline{A})}=\dfrac{0{,}62\cdot \dfrac{24}{31}}{0{,}8}=0{,}6$.
% 	\end{itemize}	
% 	}
% \end{ex}
\begin{ex}%[2D6V2-4]
	Năm $2001$, Cộng đồng châu Âu có làm một đợt kiểm tra rất rộng rãi các con bò để phát hiện những con bị bệnh bò điên. Không có xét nghiệm nào cho kết quả chính xác $100 \%$. Một loại xét nghiệm, mà ở đây ta gọi là xét nghiệm A cho kết quả như sau: khi con bò bị bệnh bò điên thì xác suất để có phản ứng dương tính trong xét nghiệm A là $70 \%$ còn khi con bò không bị bệnh thì xác suất để có phản ứng dương tính trong xét nghiệm A là $10 \%$. Biết rằng tỉ lệ bò bị mắc bệnh bò điên ở Hà Lan là $13$ con trên $1~000~000$ con \textit{(Nguồn: F. M. Dekking et al., Amodern introduction to probability and statistics Understanding why and how, Springer, $2005$)}. Khi con bò ở Hà Lan có phản ứng dương tính với xét nghiệm A thì xác suất để nó bị mắc bệnh bò điên là $\mathrm{P}$, tính $1000P$ (lấy gần đúng đến hàng phần trăm).
	\shortans{$ 0{,}09$}
	\loigiai{
	Xét hai biến cố\\	
	$N$: \lq\lq  Con bò được chọn bị nhiễm bệnh\rq\rq.\\	
	$D$: \lq\lq  Con bò được chọn có phản ứng dương tính\rq\rq.\\	
	Khi đó, ta có\\	
	\[\mathrm{P}(N)=\dfrac{13}{1 000 000}=0{,}000013; \qquad \mathrm{P}(\overline{N})=1-\mathrm{P}(N)=0{,}999987;\]	
	\[\mathrm{P}(D|N)=70\%=0{,}7; \qquad \mathrm{P}(D|\overline{N})=10\%=0{,}1.\]
	Áp dụng công thức Bayes, ta có\\
	$\mathrm{P}(N|D)=\dfrac{\mathrm{P}(D|N) \cdot \mathrm{P}(N)}{\mathrm{P}(N) \cdot \mathrm{P}(D|N)+\mathrm{P}(\overline{N})\mathrm{P}(D|\overline{N})}=\dfrac{0{,}7 \cdot 0{,}000013}{0{,}7 \cdot 0{,}000013+0{,}1 \cdot 0{,}999987}\approx 0{,}009\%$.\\
	Do đó $1000\mathrm{P}\approx1000\cdot 0{,}009\%\approx 0{,}09$.
	}
\end{ex}

%\TNSA
%%%%%----------Câu 17
\begin{ex}%[2D6V1-2]%[Võ Thanh Hiệp]
	Lớp 12A có $40$ học sinh, trong đó có $22$ bạn nữ và $18$ bạn nam. Có $3$ tên Hiền gồm hai bạn nam và một bạn nữ. Thầy giáo chọn ngẫu nhiên một bạn lên bảng làm bài tập. Tính xác suất để chọn đúng bạn tên Hiền là bạn nam ({\it kết quả làm tròn đến hàng phần trăm}).
	\shortans{$0{,}25$}
	\loigiai{
	Gọi $A$ là biến cố: \lq\lq  Chọn bạn tên Hiền\rq\rq.\\
	Gọi $B$ là biến cố: \lq\lq  Chọn bạn nam\rq\rq.\\
	Ta có $\mathrm{P}\left(A\right) = \dfrac{3}{40}$,
	$\mathrm{P}\left(B\right) = \dfrac{18}{40}=\dfrac{9}{20}$, 
	$\mathrm{P}\left(AB\right) = \dfrac{2}{18}=\dfrac{1}{9}$.\\
	Xác suất chọn đúng bạn tên Hiền với điều kiện là bạn nam $\mathrm{P}\left(A\mid B\right)$.\\
	Ta có $\mathrm{P}\left(A\mid B\right)=\dfrac{\mathrm{P}\left(AB\right)}{\mathrm{P}\left(B\right)}=
	\dfrac{20}{81} \approx 0{,}25$.
	}
\end{ex}
%%%%%----------Câu 18
\begin{ex}%[2D6V1-2]%[Võ Thanh Hiệp]
	Một hộp đựng $30$ viên bi kích thước, chất liệu như nhau, trong đó có $20$ viên bi xanh và $10$ viên bi trắng. Lấy ngẫu nhiên ra một viên bi không bỏ lại trong hộp, rồi lại lấy ngẫu nhiên ra một viên bi nữa. Tính xác suất để lấy được một viên bi trắng ở lần thứ nhất và một viên bi xanh ở lần thứ hai ({\it kết quả làm tròn đến hàng phần trăm}).
	\shortans{$0{,}23$}
	\loigiai{
	Gọi $A$ là biến cố: \lq\lq  Lấy được một viên bi trắng ở lần thứ nhất\rq\rq.\\
	Gọi $B$ là biến cố: \lq\lq  Lấy được một viên bi xanh ở lần thứ hai\rq\rq.\\
	Ta có $P\left(A\right) = \dfrac{10}{30}=\dfrac{1}{3}$.\\
	Nếu $A$ đã xảy ra, tức là một viên bi trắng đã được lấy ra ở lần thứ nhất, thì còn lại trong hộp $29$ viên bi trong đó số viên bi xanh là 20, do đó $P\left(B\mid A\right)=\dfrac{20}{29}$.\\
	Xác suất để lấy được một viên bi trắng ở lần thứ nhất và một viên bi xanh ở lần thứ hai là $P\left(A B\right)$.\\
	Ta có $P\left(A B\right) = P\left(A\right) \cdot P\left(B\mid A\right)=
	\dfrac{1}{3} \cdot \dfrac{20}{29}=\dfrac{20}{87}\approx 0{,}23$.
	}
\end{ex}
%%%%%----------Câu 19
\begin{ex}%[2D6V2-2]%[Võ Thanh Hiệp]
	Khảo sát tỉ lệ người dân trong một xã nghiện thuốc lá là $20\%$; tỉ lệ người bị bệnh phổi trong số người nghiện thuốc lá là $70\%$, trong số người không nghiện thuốc lá là $15\%$. Hỏi khi ta gặp ngẫu nhiên một người dân của của xã đó thì khả năng mà người đó bị bệnh phổi là bao nhiêu $\%$?
	\shortans{$26$}
	\loigiai{
	Gọi $A$ là biến cố: \lq\lq  Người nghiện thuốc lá\rq\rq.\\
	Suy ra $\overline{A}$ là biến cố: \lq\lq  Người không nghiện thuốc lá\rq\rq. \\
	Gọi $B$ là biến cố: \lq\lq  Người bị bệnh phổi\rq\rq.\\
	Xác suất người nghiện thuốc lá là $\mathrm{P}\left(A\right) = 20\% = 0{,}2$.\\
	Xác suất người không nghiện thuốc lá là $\mathrm{P}\left(\overline{A}\right) =1- \mathrm{P}\left(A\right) = 0{,}8$.\\
	Xác suất người bị bệnh phổi trong số người nghiện thuốc lá là $70\% \Rightarrow \mathrm{P}\left(B\mid A\right) = 0{,}7$.\\
	Xác suất người bị bệnh phổi không nghiện thuốc lá là $15\% \Rightarrow \mathrm{P}\left(B\mid \overline{A}\right) = 0{,}15$.\\
	Xác suất người bị bệnh phổi là\\
	$\mathrm{P}\left(B\right) = \mathrm{P}\left(A\right) \cdot \mathrm{P}\left(B\mid A\right)+
	\mathrm{P}\left(\overline{A}\right) \cdot \mathrm{P}\left(B\mid \overline{A}\right)
	= 0{,}2\cdot 0{,}7+0{,}8\cdot 0{,}15=0{,}26=26\%$.
	}
\end{ex}
%%%%%----------Câu 20
\begin{ex}%[2D6V2-2]%[Võ Thanh Hiệp]
	Có $2$ xạ thủ loại I và $8$ xạ thủ loại II, xác suất bắn trúng đích của các loại xạ thủ loại I là $0{,}9$ và loại II là $0{,}7$. Chọn ngẫu nhiên ra một xạ thủ và xạ thủ đó bắn một viên đạn. Tìm xác suất để viên đạn đó trúng đích.
	\par\shortans{$0{,}74$}
	\loigiai{
	Gọi $A$ là biến cố: \lq\lq  Viên đạn bắn trúng đích\rq\rq.\\	
	Gọi $B$ là biến cố: \lq\lq  Chọn xạ thủ loại I\rq\rq.\\
	Gọi $C$ là biến cố: \lq\lq  Chọn xạ thủ loại II\rq\rq.\\	
	Xác suất biến cố $B$ là $\mathrm{P}\left(B\right)=\dfrac{2}{10}=0{,}2$.
	Xác suất biến cố $C$ là $\mathrm{P}\left(C\right)=\dfrac{8}{10}=0{,}8$.
	Xác suất biến cố viên đạn đó trúng đích với điều kiện là xạ thủ loại I là
	$\mathrm{P}\left(A\mid B\right)= 0{,}9$.\\
	Xác suất biến cố viên đạn đó trúng đích với điều kiện là xạ thủ loại II là
	$\mathrm{P}\left(A\mid C\right)= 0{,}7$.\\
	$\mathrm{P}\left(A\right) = \mathrm{P}\left(B\right) \cdot \mathrm{P}\left(A\mid B\right)+
	\mathrm{P}\left(C\right) \cdot \mathrm{P}\left(A\mid C\right)
	= 0{,}2\cdot 0{,}9+0{,}8\cdot 0{,}7=0{,}74$.
	}
\end{ex}
%%%%%----------Câu 21
\begin{ex}%[2D6V2-3]%[Võ Thanh Hiệp]
	Khảo sát sự yêu thích môn Toán của hai lớp 12 của một trường. Lớp 12A1 có 40 học sinh và có $80\%$ học sinh thích môn Toán, lớp 12A2 có 32 học sinh và có $75\%$ học sinh thích môn Toán. Chọn ngẫu nhiên một học sinh. Biết rằng bạn đó yêu thích môn Toán, tính xác suất bạn đó học lớp 12A1 ({\it kết quả làm tròn đến hàng phần trăm}).
	\shortans{$0{,}57$}
	\loigiai{
	Gọi $A$ là biến cố: \lq\lq  Học sinh yêu thích môn Toán\rq\rq.\\
	Gọi $B$ là biến cố: \lq\lq  Học sinh lớp 12A1\rq\rq.\\
	Gọi $C$ là biến cố: \lq\lq  Học sinh lớp 12A2\rq\rq.\\	
	Theo đề bài ta có\\
	$\mathrm{P}\left(B\right) = \dfrac{40}{72} =\dfrac{5}{9}$;
	$\mathrm{P}\left(C\right) = \dfrac{32}{72} =\dfrac{4}{9}$;
	$\mathrm{P}\left(A\mid B\right) = 80\% =\dfrac{4}{5}$;
	$\mathrm{P}\left(A\mid C\right) = 75\% =\dfrac{3}{4}$.\\ 
	Áp dụng công thức xác suất toàn phần, ta có\\
	$\mathrm{P}\left(A\right) = \mathrm{P}\left(B\right) \cdot \mathrm{P}\left(A\mid B\right)+
	\mathrm{P}\left(C\right) \cdot \mathrm{P}\left(A\mid C\right)
	= \dfrac{5}{9}\cdot \dfrac{4}{5}+ \dfrac{4}{9}\cdot \dfrac{3}{4}=\dfrac{7}{9}$.\\
	Xác suất cần tìm là 
	$\mathrm{P}\left(B\mid A\right)=
	\dfrac{\mathrm{P}\left(A\mid B\right)\cdot \mathrm{P}\left(B\right) }{\mathrm{P}\left(A\right)}=
	\dfrac{\dfrac{4}{5}\cdot\dfrac{5}{9}}{\dfrac{7}{9}}=\dfrac{4}{7}\approx 0{,}57$.
	}
\end{ex}
%%%%%----------Câu 22
\begin{ex}%[2D6V2-3]%[Võ Thanh Hiệp]
	Hộp thứ nhất có 6 viên bi đỏ và 4 viên bi xanh, hộp thứ hai có 4 viên bi đỏ và 6 viên bi xanh, các viên bi có cùng khối lượng và kích thước. Lấy ngẫu nhiên 1 viên bi từ hộp thứ nhất bỏ sang hộp thứ hai. Sau đó từ hộp thứ hai lấy ngẫu nhiên ra một viên bi. Biết rằng viên bi lấy ra từ hộp hai là viên bi màu đỏ. Tính xác suất viên bi bỏ từ hộp thứ nhất sang hộp thứ hai là màu xanh ({\it kết quả làm tròn đến hàng phần trăm}).
	\shortans{$0{,}35$}
	\loigiai{
	Gọi $A$ là biến cố: \lq\lq  Bi bỏ từ hộp thứ nhất sang hộp thứ hai là bi màu xanh \rq\rq.\\	
	Suy ra $\overline{A}$ là biến cố: \lq\lq  Bi bỏ từ hộp thứ nhất sang hộp thứ hai là bi màu đỏ \rq\rq.\\
	Gọi $B$ là biến cố: \lq\lq  Bi lấy từ hộp thứ hai là bi màu đỏ \rq\rq.\\	
	Theo đề bài ta có\\
	$\mathrm{P}\left(A\right)= \dfrac{4}{10}$;
	$\mathrm{P}\left(\overline{A}\right)= \dfrac{6}{10}$;
	$\mathrm{P}\left(B\mid A\right)=\dfrac{4}{11}$;
	$\mathrm{P}\left(B\mid \overline{A}\right)=\dfrac{5}{11}$.\\
	Áp dụng công thức xác suất toàn phần, ta có\\
	$\mathrm{P}\left(B\right) =\mathrm{P}\left(A\right) \cdot \mathrm{P}\left(B\mid A\right)+
	\mathrm{P}\left(\overline{A}\right) \cdot \mathrm{P}\left(B\mid \overline{A}\right)=
	\dfrac{4}{10}\cdot\dfrac{4}{11}+ \dfrac{6}{10}\cdot\dfrac{5}{11}=\dfrac{23}{55}$.\\
	Xác suất cần tìm là 
	$\mathrm{P}\left(A\mid B\right)=
	\dfrac{\mathrm{P}\left(B\mid A\right)\cdot \mathrm{P}\left(A\right) }{\mathrm{P}\left(B\right)}=
	\dfrac{\dfrac{4}{11}\cdot\dfrac{4}{10}}{\dfrac{23}{55}}=\dfrac{8}{23}\approx 0{,}35$.
	}
\end{ex}
\Closesolutionfile{ans}
\indapan{3}{ans/ans-2-B6-De2-TLN}