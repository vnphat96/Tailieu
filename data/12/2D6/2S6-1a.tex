\setcounter{section}{0}
\section{XÁC SUẤT CÓ ĐIỀU KIỆN}
%%%%%%%%%%%%%%%%
\subsection{Trọng tâm kiến thức}
\begin{tomtat}
	\subsubsection{Xác suất có điều kiện}
	\begin{boxdn}
	\begin{itemize}
	\item
	Cho hai biến cố $A$ và $B$. Xác suất của biến cố $A$, tính trong điều kiện biết rằng biến cố $B$ đã xảy ra, được gọi là xác suất của $A$ với điều kiện $B$ và kí hiệu là $\mathrm{P}(A\mid B)$.
	\item
	Cho hai biến cố $A$ và $B$ bất kì, với $\mathrm{P}(B)>0$. Khi đó
	$$\mathrm{P}(A \mid B)=\dfrac{\mathrm{P}(A B)}{\mathrm{P}(B)}.$$
	\end{itemize}
	\end{boxdn}
	\subsubsection{Công thức nhân xác suất}
	\begin{boxdn}
	Vậy với hai biến cố $A$ và $B$ bất kì, ta có
	$$\mathrm{P}(A B)=\mathrm{P}(B) \cdot \mathrm{P}(A \mid B).$$
	Công thức trên được gọi là \textbf{\textit{công thức nhân xác suất}}.
	\end{boxdn}
	\begin{note}
	\begin{itemize}
	\item Vì $AB=BA$ nên với hai biến cố $A$ và $B$ bất kì, ta cũng có
	$$\mathrm{P}(A B)=\mathrm{P}(A) \cdot \mathrm{P}(B \mid A) \text{. }$$
	\item Nếu $A$ và $B$ là hai biến cố độc lập thì
	$$\mathrm{P}(A B)=\mathrm{P}(A) \cdot \mathrm{P}(B).$$
	\end{itemize}
	\end{note}
\end{tomtat}
%%%%%%%%%%%%%%
\subsection{Các dạng bài tập}
%============================
\begin{dang}{Tính xác suất có điều kiện theo định nghĩa}
%	\begin{itemize}
%	\item Cho hai biến cố $A$ và $B$. Xác suất của biến cố $B$ khi biến cố $A$ đã xảy ra được gọi là \textbf{xác suất của $B$ với điều kiện $A$}, kí hiệu là $\mathrm{P}(B \mid A)$.
%	\item Sử dụng định nghĩa để tính xác suất có điều kiện (áp dụng với các bài có thể tìm được số phần tử của các biến cố).
%	\end{itemize}
\end{dang}
%----------------------------
\subsubsection{Ví dụ minh hoạ}
\begin{vd}%[2D5H1-2]
	Có hai hộp chứa các thẻ được đánh số. Hộp thứ nhất có các thẻ được đánh số từ $1$ đến $4$, hộp thứ hai có các thẻ được đánh số từ $5$ đến $6$. Các thẻ có cùng kích thước và khối lượng. Bạn Phương lấy ngẫu nhiên một thẻ từ hộp thứ nhất bỏ vào hộp thứ hai. Sau đó bạn lại lấy ngẫu nhiên một thẻ từ hộp thứ hai. Liệt kê các kết quả của phép thử biết lần thứ nhất bạn Phương lấy được một thẻ đánh số chẵn.
	\loigiai{
	Vì đã biết lần thứ nhất bạn Phương lấy được một thẻ đánh số chẵn. Nghĩa là lúc đó bạn Phương có thể lấy được thẻ đánh số $2$ hoặc $4$.\\
	Nếu bạn Phương lấy được thẻ đánh số $2$ và bỏ vào hộp thứ hai, thì lúc này trong hộp thứ hai có các thẻ đánh số từ $5$ đến $6$ và $2$. Do đó ta có các khả năng $(2;5),(2;6),(2;7),(2;2)$.\\
	Tương tự như vậy nếu bạn Phương lấy được thẻ đánh số $4$, ta có các khả năng $(4;5),(4;6),(4,7),(4,4)$.\\
	Vậy các kết quả của phép thử biết lần thứ nhất bạn Phương lấy được một thẻ đánh số chẵn là 
	$$(2;5),(2;6),(2;7),(2;2),(4;5),(4;6),(4,7),(4,4).$$
	}
\end{vd}
\begin{vd}%[2D5H1-2]
	Một hộp có $5$ viên bi cùng kích thước và khối lượng, trong đó có $3$ viên bi màu đỏ và $2$ viên bi màu xanh. Lấy ngẫu nhiên lần lượt $2$ viên bi và không hoàn lại. Tính xác suất để lấy được viên bi thứ hai có màu xanh, biết rằng viên bi thứ nhất có màu đỏ.
	\loigiai{
	Gọi
	\begin{itemize}
	\item $A$ là biến cố \lq\lq  Lấy được viên bi thứ hai có màu xanh\rq\rq;
	\item $B$ là biến cố \lq\lq  Lấy được viên bi thứ nhất có màu đỏ\rq\rq.
	\end{itemize}
	Khi đó xác suất để lấy được viên bi thứ hai có màu xanh, biết rằng viên bi thứ nhất có màu đỏ chính là xác suất của $A$ với điều kiện $B$.\\
	Vì một viên bi đỏ đã được lấy ra ở lần thứ nhất nên trong hộp còn lại $4$ viên bi, trong đó có $2$ viên bi xanh.\\
	Từ đó ta có $\mathrm{P}(A \mid B)=\dfrac{2}{4}=0{,}5$.\\
	Vậy xác suất để lấy được viên bi thứ hai có màu xanh, biết rằng viên bi thứ nhất có màu đỏ là $0{,}5$.
	}	
\end{vd}
\begin{vd}%[2D5H2-2]
	Một hộp có $20$ viên bi trắng và $10$ viên bi đen, các viên bi có cùng kích thước và khối lượng. Bạn Bình lấy ngẫu nhiên một viên bi trong hộp, không trả lại. Sau đó bạn An lấy ngẫu nhiên một viên bi trong hộp đó.\\
	Gọi $A$ là biến cố: ``An lấy được viên bi trắng''; $B$ là biến cố: ``Bình lấy được viên bi trắng''. Tính $\mathrm{P}(A\mid B)$, $P\left(A\mid\overline{B}\right)$.
	\loigiai{
		Nếu $B$ xảy ra tức là Bình lấy được viên bi trắng. \\
		Khi đó, trong hộp còn lại $29$ viên bi với $19$ viên bi trắng và $10$ viên bi đen.\\ 
		Vậy $\mathrm{P}(A\mid B)=\dfrac{19}{29}\approx 0{,}67$.\\
		Nếu $\overline{B}$ xảy ra tức là Bình lấy được viên bi đen.\\
	Khi đó trong hộp còn lại $29$ viên bi với $20$ viên bi trắng và $9$ viên bi đen.\\
	Vậy $\mathrm{P}(A\mid\overline{B})=\dfrac{20}{29}$.
	}
\end{vd}
\begin{vd}%[2D5H2-2]
	Câu lạc bộ cờ của nhà trường gồm $35$ thành viên, mỗi thành viên biết chơi ít nhất một trong hai môn cờ vua hoặc cờ tướng. Biết rằng có $25$ thành viên biết chơi cờ vua và $20$ thành viên biết chơi cờ tướng. Chọn ngẫu nhiên $1$ thành viên của câu lạc bộ. Tính xác suất thành viên được chọn biết chơi cờ vua, biết rằng thành viên đó biết chơi cờ tướng.
	\loigiai{
	Gọi $A$ là biến cố \lq\lq  Thành viên được chọn biết chơi cờ tướng\rq\rq \,và $B$ là biến cố \lq\lq  Thành viên được chọn biết chơi cờ vua\rq\rq.\\
	Số thành viên của câu lạc bộ biết chơi cả hai môn cờ là $20+25-35=10$.\\
	Do đó, trong số $20$ thành viên biết chơi cờ tướng, có đúng $10$ thành viên biết chơi cờ vua.\\ 
	Vậy nên xác suất thành viên được chọn biết chơi cờ vua, biết rằng thành viên đó biết chơi cờ tướng là 
	$$\mathrm{P}(B \mid A)=\dfrac{10}{20}=0{.}5.$$
	}
\end{vd}
\begin{vd}%[2D5H1-2]
	Hộp thứ nhất chứa $4$ viên bi xanh và $3$ viên bi đỏ. Hộp thứ hai chứa $3$ viên bi xanh và $5$ viên bi đỏ. Các viên bi có cùng kích thước và khối lượng. Bạn Thanh lấy ra ngẫu nhiên $1$ viên bi từ hộp thứ nhất bỏ vào hộp thứ hai, sau đó lại lấy ra ngẫu nhiên $1$ viên bi từ hộp thứ hai. Tính xác suất để viên bi lấy ra ở lần thứ hai là viên bi đỏ, biết viên bi lấy ra ở lần thứ nhất là viên bi đỏ.
	\loigiai{
	Gọi $A$ là biến cố \lq\lq  viên bi lấy ra lần thứ hai là viên bi đỏ\rq\rq; $B$ là biến cố \lq\lq  viên bi lấy ra lần thứ hai là viên bi đỏ\rq\rq.\\
	Biến cố $B$ xảy ra, nghĩa là lần thứ nhất lấy được viên bi đỏ và bỏ vào hộp thứ hai. Khi đó trong hộp thứ hai sẽ có $3$ viên bi xanh và $6$ viên bi đỏ.\\
	Vậy $\mathrm{P}(A \mid B)= \dfrac{6}{9}=\dfrac{2}{3}$.
	}
\end{vd}
%=========================
% \setcounter{subsubsection}{1}
\begin{dang}{Tính xác suất có điều kiện theo công thức}
	Nếu $\mathrm{P}(B)>0$ thì xác suất của biến cố $A$ với điều kiện $B$ được xác định bởi công thức
	$$\mathrm{P}(A \mid B)=\dfrac{\mathrm{P}(A B)}{\mathrm{P}(B)}.$$
\end{dang}
%----------------------------
\subsubsection{Ví dụ minh hoạ}
\begin{vd}%[2D5H1-2]
	Cho hai biến cố $A$, $B$ có $\mathrm{P}(A)=0,4$; $\mathrm{P}(B)=0,6$; $\mathrm{P}(AB)=0,2$. Tính các xác suất $\mathrm{P}(A\mid B)$ và $\mathrm{P}(B\mid A)$.
	\loigiai{
	Ta có
	\begin{itemize}
	\item $\mathrm{P}(A\mid B)=\dfrac{\mathrm{P}(AB)}{\mathrm{P}(B)}=\dfrac{0{,}2}{0{,}6}=\dfrac{1}{3}$.
	\item $\mathrm{P}(B|A)=\dfrac{\mathrm{P}(AB)}{\mathrm{P}(A)}=\dfrac{0{,}2}{0{,}4}=0{,}5$.
	\end{itemize}
	}
\end{vd}
\begin{vd}%[2D5H1-2]
	Cho hai biến độc lập $A,B$ với $\mathrm{P}(A)=0{,}8$. Tính $\mathrm{P}(A\mid B)$.
	\loigiai{
	Vì $A$ và $B$ là hai biến cố độc lập, do đó
	\[\mathrm{P}(A\mid B)=\dfrac{\mathrm{P}(AB)}{\mathrm{P}(B)}=\dfrac{\mathrm{P}(A)\cdot \mathrm{P}(B)}{\mathrm{P}(B)}=\mathrm{P}(A)=0{,}8.\]
	\begin{note} Ta có thể dùng công thức $\mathrm{P}\left(A\mid B\right)=\mathrm{P}(A)$ với $A$ và $B$ là hai biến cố độc lập.
	\end{note}
	}
\end{vd}
\begin{vd}%[2D5H2-2]
	Một lô sản phẩm có $20$ sản phẩm, trong đó có $5$ sản phẩm chất lượng thấp. Lấy liên tiếp $2$ sản phẩm trong lô sản phẩm trên, trong đó sản phẩm lấy ra ở lần thứ nhất không được bỏ lại vào lô sản phẩm. Tính xác suất để cả hai sản phẩm được lấy ra đều có chất lượng thấp.
	\loigiai{
	Gọi $A$ là biến cố \lq\lq  sản phẩm thứ nhất có chất lượng thấp\rq\rq, và $B$ là biến cố \lq\lq  sản phẩm thứ hai có chất lượng thấp\rq\rq.\\
	Xác suất của $A$ là xác suất để lấy ra một sản phẩm chất lượng thấp trong lần đầu tiên:
	$$\mathrm{P}(A)=\dfrac{n(A)}{n(\Omega)}=\dfrac{5}{20}=\dfrac{1}{4}.$$
	Sau khi lấy một sản phẩm chất lượng thấp, số sản phẩm chất lượng thấp giảm còn $4$ trong tổng số $19$ sản phẩm.\\
	Xác suất của $B$ khi đã xảy ra $A$ là xác suất để lấy ra một sản phẩm chất lượng thấp trong lần thứ hai:
	$$\mathrm{P}(B \mid A)=\dfrac{4}{19}.$$
	Áp dụng quy tắc nhân xác suất:
	$$\mathrm{P}(AB)=\mathrm{P}(A)\cdot \mathrm{P}(B \mid A)=\dfrac{1}{4}\cdot\dfrac{4}{19}=\dfrac{1}{19}.$$
	}
\end{vd}
\begin{vd}%[2D5H2-2]
	Trong cuộc khảo sát $300$ gia đình ở một khu vực, người ta nhận thấy rằng có $90\%$ gia đình có tivi và $60\%$ gia đình có máy tính bàn. Mỗi gia đình đều có ít nhất một trong hai thiết bị này. Chọn ngẫu nhiên một gia đình. Tính xác suất gia đình có máy tính bàn trong nhóm các gia đình có tivi.
	\loigiai{
	Gọi $A$ là biến cố \lq\lq  Gia đình được chọn có máy tính bàn\rq\rq; $B$ là biến cố \lq\lq  Gia đình được chọn có tivi\rq\rq. Khi đó $AB$ là biến cố \lq\lq  Gia đình được chọn có cả máy tính bàn và tivi\rq\rq. \\
	Ta có $n(B)=0{,}9\cdot300=270$ và $n(AB)=0{,}9\cdot300+0{,}6\cdot300-300=150$. \\
	Do đó $\mathrm{P}(A\mid B)=\dfrac{n\left(A\cap B\right)}{n(B)}=\dfrac{150}{270}=\dfrac{5}{9}$.
	}
\end{vd}
%----------------------------
\subsubsection{Bài tập áp dụng}

\begin{bt}%[2D5H2-2]
	Một phòng nghiên cứu dược học cho $500$ người bị bệnh $H$ dùng hai loại thuốc $X, Y$ để điều trị. Một số người được điều trị bằng thuốc $X$ và số người còn lại được điều trị bằng thuốc $Y$. Kết quả nghiên cứu được trình bày ở bảng $6.2$.
	\begin{center}
	\begin{tikzpicture}
	\begin{scope}[xscale=4.4,yscale=0.85]
	\path
	(0,0) foreach \i[count=\k] in {$X$,$Y$} {++(1,0)node(1\k){\i}}
	(0,-1) node {Khỏi bệnh} foreach \i[count=\k] in {$180$,$190$} {++(1,0)node(2\k){\i}}
	(0,-2) node{Không khỏi bệnh} foreach \i[count=\k] in {$60$,$70$} {++(1,0)node(3\k){\i}}
	%(0,-3) node{tiện} foreach \i[count=\k] in {,$O_1$,,$O_4$,} {++(1,0)node(4\k){\i}}
	%(0,-4) node{điện} foreach \i[count=\k] in {,$O_1$,,$O_4$,} {++(1,0)node(5\k){\i}}
	%(0,-5) node{nước} foreach \i[count=\k] in {,$O_1$,,$O_4$,} {++(1,0)node(6\k){\i}}
	;
	%\path
	%(23.south east) node{$\times$}
	%;
	\draw[shift={(-0.5,.5)}] (0,0) grid (3.,-3)
	(0,0)--(1.,-1)
	(0,-1) node[above right]{Tình trạng}
	(1,0) node[below left]{Loại thuốc}
	;
	\end{scope}
	\end{tikzpicture}
	\end{center}
	Chọn ngẫu nhiên một người trong số này. Gọi $A$ là biến cố \lq\lq  Người được chọn khỏi bệnh\rq\rq, $B$ là biến cố \lq\lq  Người được chọn điều trị bằng thuốc $X$\rq\rq, $C$ là biến cố \lq\lq  Người được chọn điều trị bằng thuốc $Y$\rq\rq.
	\begin{listEX}
	\item Tính và giải thích ý nghĩa của $\mathrm{P}(A \mid B)$ và $\mathrm{P}(A \mid C)$.
	\item Có thể nói loại thuốc nào có hiệu quả hơn trong việc điều trị bệnh $H$?
	\end{listEX}
	\loigiai{
	\begin{listEX}
	\item $\mathrm{P}(A \mid B)=\dfrac{n\left(AB\right)}{n(B)}=\dfrac{180}{240}=\dfrac{3}{4}$ và $\mathrm{P}(A \mid C)=\dfrac{n\left(A\cap C\right)}{n(C)}=\dfrac{190}{260}=\dfrac{19}{26}$. \\
	Theo các kết quả trên, xác suất để một người khỏi bệnh khi được chọn điều trị bằng thuốc $X$ là $\dfrac{3}{4}$ và xác suất để một người khỏi bệnh khi được chọn điều trị bằng thuốc $Y$ là $\dfrac{19}{26}$.
	\item Do $\dfrac{3}{4}>\dfrac{19}{26}$ nên loại thuốc $X$ có hiệu quả hơn loại thuốc $Y$ trong việc điều trị bệnh $H$.
	\end{listEX}
	}
\end{bt}
\begin{bt}%[2D5H2-2]
	Một xí nghiệp dệt may có những dải của một loại vải đang sản xuất theo một quy trình đặc biệt. Những dải này có thể bị lỗi theo hai hướng: lỗi chiều dài và lỗi kết cấu. Thông qua đợt kiểm tra quy trình sản xuất, người ta thấy rằng có $10\%$ dải không đạt yêu cầu về chiều dài, $5\%$ dải không đạt yêu cầu và kết cấu và chỉ có $0{,}8\%$ dải không đạt yêu cầu về cả chiều dài và kết cấu.
	\begin{listEX}
	\item Nếu chọn ngẫu nhiên một dải từ quy trình này thì xác suất không đạt yêu cầu về kết cấu là bao nhiêu?
	\item Nếu một dải được chọn ngẫu nhiên từ quy trình này và phép đo nhanh xác định dải đó không đạt yêu cầu về chiều dài, tính xác suất để dải đó không đạt yêu cầu về kết cấu.
	\end{listEX}
	\loigiai{
	\begin{listEX}
	\item Nếu chọn ngẫu nhiên một dải từ quy trình này thì xác suất không đạt yêu cầu về kết cấu là $\dfrac{5}{100}+\dfrac{0{,}8}{100}=\dfrac{29}{500}$.
	\item Gọi $A$ là biến cố \lq\lq  Dải được chọn từ quy trình không đạt yêu cầu về kết cấu\rq\rq;\\
	$B$ là biến cố \lq\lq  Dải được chọn từ quy trình không đạt yêu cầu về chiều dài\rq\rq.\\
	Khi đó $AB$ là biến cố \lq\lq  Một dải từ quy trình không đạt yêu cầu về cả kết cấu và chiều dài\rq\rq. Ta có $\mathrm{P}(AB)=0{,}8\%=0{,}008$ và $\mathrm{P}(B)=0{,}1$. \\
	Do đó $\mathrm{P}(A\mid B)=\dfrac{\mathrm{P}\left(AB\right)}{\mathrm{P}(B)}=\dfrac{0{,}008}{0{,}1}=0{,}08$.
	\end{listEX}
	}
\end{bt}
\begin{bt}%[2D5H2-2]
	Trong một lọ có chứa bi đen và bi trắng cùng kích thước và khối lượng, lấy ngẫu nhiên lần lượt hai viên bi ra ngoài và không bỏ vào lại. Biết rằng xác suất để lần đầu lấy được bi đen là $0{,}47$; xác suất để lần đầu lấy được bi đen và lần thứ hai lấy được bi trắng là $0{,34}$. Tính xác suất để lấy được bi trắng ở lần thứ hai với điều kiện lần đầu lấy được bi đen.
	\loigiai{
	Gọi $A$ là biến cố \lq\lq  Lấy được bi trắng ở lần thứ hai\rq\rq; $B$ là biến cố \lq\lq  Lấy được bi đen ở lần đầu\rq\rq. \\
	Do đó $\mathrm{P}(A \mid B)=\dfrac{\mathrm{P}(AB)}{\mathrm{P}(B)}=\dfrac{0{,34}}{0{,47}}=\dfrac{34}{47}$.
	}
\end{bt}
%=======================
\begin{dang}{Tính xác suất có điều kiện nhờ sơ đồ hình cây}
	Trên sơ đồ hình cây:
	\begin{itemize}
	\item Xác suất của các nhánh trong sơ đồ hình cây từ đỉnh thứ hai là xác suất có điều kiện.
	\item Xác suất xảy ra của mỗi kết quả bằng tích các xác suất trên các nhánh của cây đi đến kết quả đó.
	\end{itemize}
\end{dang}
%----------------------------
\subsubsection{Ví dụ minh hoạ}
\begin{vd}%[2D5H2-3]
	Một hộp có $8$ bi màu đỏ và $5$ viên bi màu vàng; các viên bi có kích thước và khối lượng như nhau. Có $5$ viên bi trong hộp được đánh số, trong đó có $3$ viên bi màu đỏ và $2$ viên bi màu vàng. Lấy ngẫu nhiên một viên bi trong hộp. Dùng sơ đồ hình cây, tính xác suất để viên bi được lấy ra, có màu đỏ, biết rằng viên bi đó được đánh số.
	\loigiai{
	Xét các biến cố sau:
	\begin{itemize}
	\item $A \colon$ \lq\lq  Viên bi được lấy ra có đánh số\rq\rq.
	\item $B \colon$ \lq\lq  Viên bi được lấy ra có màu đỏ \rq\rq.
	\end{itemize}
	Khi đó, xác suất để viên bi được lấy ra có màu đỏ, biết ràng viên bi đó được đánh số, chính là xác suất có điều kiện $\mathrm{P}(B\mid A)$.\\
	Sơ đồ hình cây biểu thị cách tính xác suất có điều kiện $\mathrm{P}(B\mid A)$, được vẽ như sau:
	\begin{center}
	\begin{tikzpicture}[scale=0.8]
	\def\gocm{20}
	\def\gocn{10}
	\def\r{4}
	\tikzset{s/.style={outer sep=0.5 mm,draw=magenta,rectangle,minimum width=2.75cm,rounded corners=1mm}}
	\path(0,0)node(O){}++(\gocm:\r)node[s](A1){A}++(\gocn:\r)node[s](A2){$B$}++(0:\r)node[s](A3){$AB$};
	\path(A1)++({-\gocn}:\r)node[s](a2){$\overline{B}$}++(0:\r)node[s](a3){$A\overline{B}$};
	\path(O)++(-\gocm:\r)node[s](B1){$\overline{A}$}++(\gocn:\r)node[s](B2){$B$}++(0:\r)node[s](B3){$\overline{A}B$};
	\path(B1)++({-\gocn}:\r)node[s](b2){$\overline{B}$}++(0:\r)node[s](b3){$\overline{A}\overline{B}$};
	\foreach \x/\y in {
	O/A1,A1/A2,
	O/B1,B1/B2,
	A1/a2,
	B1/b2}
	\draw[-stealth](\x.east)--(\y.west);
	\path(O)--(A1.west)node[pos=0.5,above,sloped]{$\frac{5}{13}$}(O)--(B1.west)node[pos=0.5,below,sloped]{$\frac{8}{13}$}(B1.east)--(B2.west)node[pos=0.5,above,sloped]{$\frac{5}{8}$}(A1.east)--(A2.west)node[pos=0.5,above,sloped]{$\frac{3}{5}$}
	(A1.east)--(a2.west)node[pos=0.5,below,sloped]{$\frac{2}{5}$}
	(B1.east)--(b2.west)node[pos=0.5,below,sloped]{$\frac{3}{8}$};
	\end{tikzpicture}
	\end{center}
	Vậy xác suất để viên bi được lấy ra có màu đỏ, biết rằng viên bi đó có đánh số, là $ 0{,}6$.
	}
\end{vd}
\begin{vd}%[2D5H2-3]
	Ở một sân bay, người ta sử dụng một loại máy soi tự động phát hiện hàng cấm trong hành lí kí gửi. Máy phát chuông cảnh báo với $95 \%$ các kiện hành lí có chứa hàng cấm và $2 \%$ các kiện hành lí không chứa hàng cấm. Tỉ lệ các kiện hành lí có chứa hàng cấm là $1 \%$.\\
	Chọn ngẫu nhiên một kiện hành lí để soi bằng máy trên. Sử dụng sơ đồ hình cây, tính xác suất của các biến cố:\\
	M: \lq\lq  Kiện hành lí có chứa hàng cấm và máy phát chuông cảnh báo \rq\rq;
	N : \lq\lq  Kiện hành lí không chứa hàng cấm và máy phát chuông cảnh báo\rq\rq.
	\loigiai
	{
	Gọi $A$ là biến cố \lq\lq  Kiện hành lí có chứa hàng cấm\rq\rq và $B$ là biến cố \lq\lq  Máy phát chuông cành báo\rq\rq. Ta có
	$$
	\mathrm{P}(B \mid A)=0{,}95 ; \mathrm{P}(B \mid \overline{A})=0{,}02 ; \mathrm{P}(A)=0{,}01.
	$$
	Do đó $\mathrm{P}(\overline{A})=1-\mathrm{P}(A)=0{,}99 ; \mathrm{P}(\overline{B} \mid A)=1-\mathrm{P}(B \mid A)=0{,}05 ; \mathrm{P}(\overline{B} \mid \overline{A})=1-\mathrm{P}(B \mid \overline{A})=0{,}98$.
	Ta có sơ đồ hình cây như sau:
	\begin{center}
	\begin{tikzpicture}[yscale=0.7]
	\def\gocm{20}
	\def\gocn{10}
	\def\r{3.5}
	\tikzset{s/.style={outer sep=0.5 mm,draw=magenta,rectangle,minimum width=3cm,rounded corners=1mm}}
	\path(0,0)node(O){}++(\gocm:\r)node[s](A1){A}++(\gocn:\r)node[s](A2){$B$}++(0:\r)node[s](A3){$AB$}++(0:\r)node[s](A4){$0{,}0095$};
	\path(A1)++({-\gocn}:\r)node[s](a2){$\overline{B}$}++(0:\r)node[s](a3){$A\overline{B}$}++(0:\r)node[s](a4){$0{,}0005$};
	\path(O)++(-\gocm:\r)node[s](B1){$\overline{A}$}++(\gocn:\r)node[s](B2){$B$}++(0:\r)node[s](B3){$\overline{A}B$}++(0:\r)node[s](B4){$0{,}0198$};
	\path(B1)++({-\gocn}:\r)node[s](b2){$\overline{B}$}++(0:\r)node[s](b3){$\overline{A}\overline{B}$}++(0:\r)node[s](b4){$0{,}9702$};
	\foreach \x/\y in {
	O/A1,A1/A2,
	O/B1,B1/B2,
	A1/a2,
	B1/b2}
	\draw[-stealth](\x.east)--(\y.west);
	\path(O)--(A1.west)node[pos=0.5,above,sloped]{$0{,}01$}(O)--(B1.west)node[pos=0.5,below,sloped]{$0{,}99$}(B1.east)--(B2.west)node[pos=0.5,above,sloped]{\tiny$0{,}02$}(A1.east)--(A2.west)node[pos=0.5,above,sloped]{\tiny$0{,}95$}
	(A1.east)--(a2.west)node[pos=0.5,below,sloped]{\tiny$0{,}05$}
	(B1.east)--(b2.west)node[pos=0.5,below,sloped]{\tiny$0{,}98$};
	\end{tikzpicture}
	\end{center}
	Do $M=A B$ nên $\mathrm{P}(M)=\mathrm{P}(A B)=0{,}0095$.\\
	Do $N=\overline{A} B$ nên $\mathrm{P}(N)=\mathrm{P}(\overline{A} B)=0{,}0198$.
	}
\end{vd}
\begin{vd}%[2D5H2-3]
	Theo kết quả từ trạm nghiên cứu khí hậu tại địa phương $T$, xác suất để một ngày có gió là $0{,}6$; nếu ngày có gió thì xác suất có mưa là $0{,}4$; nếu ngày không có gió thì xác suất có mưa là $0{,}2$. Gọi $G$ là biến cố \lq\lq  Ngày có gió\rq\rq~ và $M$ là biến cố \la\la Ngày có mưa\rq\rq.
	\begin{listEX}
	\item Vẽ lại sơ đồ hình cây sau và điền vào ô? các giá trị xác suất tương ứng.
	\begin{center}
	\begin{tikzpicture}[node distance=1.5cm, every node/.style={fill=none}, align=center,scale=0.6]
	\definecolor{diagram_bg_green}{HTML}{d5e8d4}
	\definecolor{diagram_bg_blue}{HTML}{dae8fc}
	\definecolor{diagram_bg_pink}{HTML}{f8cecc}
	\definecolor{diagram_bd_green}{HTML}{82b366}
	\definecolor{diagram_bd_blue}{HTML}{7494c2}
	\definecolor{diagram_bd_pink}{HTML}{b85450}
	\tikzset{>={Latex[width=2mm,length=2mm]}, base/.style = {rectangle, rounded corners, draw=black, text centered, drop shadow={shadow xshift=0.6mm, shadow yshift=-0.6mm}},
	Style1/.style = {base, fill=diagram_bg_blue, draw=diagram_bd_blue},
	Style2/.style = {base, fill=diagram_bg_pink, draw=diagram_bd_pink},
	Style3/.style = {base, fill=diagram_bg_green, draw=diagram_bd_green},
	Style4/.style = {base, minimum width=2.5cm, fill=orange!15, draw=orange},
	}
	\node (B0) [Style1, text width=1cm] {Ngày};
	\node (B1) [Style2, right of=B0, xshift=3cm, yshift=1cm, text width=3.5cm] {Có gió $(G)$};
	\node (B3) [Style2, right of=B0, xshift=3cm, yshift=-1cm, text width=3.5cm] {Không có gió $(\overline{G})$};
	\node (B11) [Style3, right of=B1, xshift=6cm, yshift=0.5cm, text width=3cm] {Có mưa};
	\node (B13) [Style3, right of=B1, xshift=6cm, yshift=-0.5cm, text width=3cm] {Không có mưa};
	\node (B31) [Style3, right of=B3, xshift=6cm, yshift=0.5cm, text width=3cm] {Có mưa};
	\node (B33) [Style3, right of=B3, xshift=6cm, yshift=-0.5cm, text width=3cm] {Không có mưa};
	\draw[->] (B0) -- (B1.west)node[sloped,above,pos=0.5]{$\mathrm{P}(G)=?$};
	\draw[->] (B0) -- (B3.west)node[sloped,below,pos=0.5]{$\mathrm{P}(\overline{G})=?$};
	\draw[->] (B1) -- (B11.west)node[sloped,above,pos=0.5]{$\mathrm{P}(M\mid G)=?$};
	\draw[->] (B1) -- (B13.west)node[sloped,below,pos=0.5]{$\mathrm{P}(\overline{M}\mid G)=?$};
	\draw[->] (B3) -- (B31.west)node[sloped,above,pos=0.5]{$\mathrm{P}(M\mid\overline{G})=?$};
	\draw[->] (B3) -- (B33.west)node[sloped,below,pos=0.5]{$\mathrm{P}(\overline{M}\mid\overline{G})=?$};
	\end{tikzpicture}
	\end{center}
	\item Tính xác suất $\mathrm{P}(G M)$ và $\mathrm{P}(G \overline{M})$. Nêu ý nghĩa của các xác suất này.
	\end{listEX}
	\loigiai{
	\begin{listEX}
	\item Theo đề bài, nếu ngày có gió thì xác suất có mưa là 0{,}4 nên $\mathrm{P}(M \mid G)=0{,}4$.\\
	Suy ra $\mathrm{P}(\overline{M} \mid G)=1-0{,}4=0{,}6$.
	Ngày không có gió thì xác suất có mưa là $0{,}2$ nên $\mathrm{P}(M \mid \overline{G})=0{,}2$.\\
	Suy ra $\mathrm{P}(\overline{M} \mid \overline{G})=1-0{,}2=0{,}8$.
	\begin{center}
	\begin{tikzpicture}[node distance=1.5cm, every node/.style={fill=none}, align=center,scale=0.6]
	\definecolor{diagram_bg_green}{HTML}{d5e8d4}
	\definecolor{diagram_bg_blue}{HTML}{dae8fc}
	\definecolor{diagram_bg_pink}{HTML}{f8cecc}
	\definecolor{diagram_bd_green}{HTML}{82b366}
	\definecolor{diagram_bd_blue}{HTML}{7494c2}
	\definecolor{diagram_bd_pink}{HTML}{b85450}
	\tikzset{>={Latex[width=2mm,length=2mm]}, base/.style = {rectangle, rounded corners, draw=black, minimum width=1cm, text centered, drop shadow={shadow xshift=0.6mm, shadow yshift=-0.6mm}},
	Style1/.style = {base, fill=diagram_bg_blue, draw=diagram_bd_blue},
	Style2/.style = {base, fill=diagram_bg_pink, draw=diagram_bd_pink},
	Style3/.style = {base, fill=diagram_bg_green, draw=diagram_bd_green},
	Style4/.style = {base, minimum width=2.5cm, fill=orange!15, draw=orange},
	}
	\node (B0) [Style1, text width=1cm] {Ngày};
	\node (B1) [Style2, right of=B0, xshift=3cm, yshift=1cm, text width=3cm] {Có gió $(G)$};
	\node (B3) [Style2, right of=B0, xshift=3cm, yshift=-1cm, text width=3cm] {Không có gió $(\overline{G})$};
	\node (B11) [Style3, right of=B1, xshift=6cm, yshift=0.5cm, text width=4cm] {Có mưa};
	\node (B13) [Style3, right of=B1, xshift=6cm, yshift=-0.5cm, text width=4cm] {Không có mưa};
	\node (B31) [Style3, right of=B3, xshift=6cm, yshift=0.5cm, text width=4cm] {Có mưa};
	\node (B33) [Style3, right of=B3, xshift=6cm, yshift=-0.5cm, text width=4cm] {Không có mưa};
	\draw[->] (B0) -- (B1.west)node[sloped,above,pos=0.5]{$\mathrm{P}(G)=0{,}6$};
	\draw[->] (B0) -- (B3.west)node[sloped,below,pos=0.5]{$\mathrm{P}(\overline{G})=0{,}4$};
	\draw[->] (B1) -- (B11.west)node[sloped,above,pos=0.5]{$\mathrm{P}(M\mid G)=0{,}4$};
	\draw[->] (B1) -- (B13.west)node[sloped,below,pos=0.5]{$\mathrm{P}(\overline{M}\mid G)=0{,}6$};
	\draw[->] (B3) -- (B31.west)node[sloped,above,pos=0.5]{$\mathrm{P}(M\mid\overline{G})=0{,}2$};
	\draw[->] (B3) -- (B33.west)node[sloped,below,pos=0.5]{$\mathrm{P}(\overline{M}\mid\overline{G})=0{,}8$};
	\end{tikzpicture}
	\end{center}
	\item $\mathrm{P}(G M)=\mathrm{P}(G) \cdot \mathrm{P}(M \mid G)=0{,}6 \cdot 0{,}4=0{,}24 $.\\
	$\mathrm{P}(G \overline{M})=\mathrm{P}(G) \cdot \mathrm{P}(\overline{M} \mid G)=0{,}6 \cdot 0{,}6=0{,}36$.\\
	Điểu này có nghĩa là tại địa phương $T$, trong một ngày, xác suất để trời vừa có gió và vừa có mưa là $0{,}24$; xác suất để trời có gió nhưng không có mưa là $0{,}36$.
	\end{listEX}
	}
\end{vd}
%----------------------------
\subsubsection{Bài tập áp dụng}
\begin{bt}%[2D5H2-3]
	Bạn Việt chuẩn bị đi tham quan một hòn đảo trong hai ngày thứ Bảy và Chủ nhật. Ở hòn đảo đó, mỗi ngày chỉ có nắng hoặc mưa, nếu một ngày là nắng thì khả năng xảy ra mưa ở ngày tiếp theo là $20 \%$, còn nếu một ngày là mưa thì khả năng ngày hôm sau vẫn mưa là $30 \%$. Theo dự báo thời tiết, xác suất trời sẽ nắng vào thứ Bảy là $0{,}7$.
	Hãy tìm các giá trị thích hợp thay vào \mbox{?} ở sơ đồ hình cây sau:
	\begin{center}
	\begin{tikzpicture}[yscale=0.7]
	\def\gocm{20}
	\def\gocn{10}
	\def\r{4}
	\tikzset{s/.style={outer sep=0.5 mm,draw=magenta,rectangle,minimum width=2.75cm,rounded corners=1mm}}
	\path(0,0)node(O){}++(\gocm:\r)node[s](A1){Nắng}++(\gocn:\r)node[s](A2){Nắng};
	\path(A1)++({-\gocn}:\r)node[s](a2){Mưa};
	\path(O)++(-\gocm:\r)node[s](B1){Mưa}++(\gocn:\r)node[s](B2){Nắng};
	\path(B1)++({-\gocn}:\r)node[s](b2){Mưa};
	\foreach \x/\y in {
	O/A1,A1/A2,
	O/B1,B1/B2,
	A1/a2,
	B1/b2}
	\draw[-stealth](\x.east)--(\y.west);
	\path(O)--(A1.west)node[pos=0.5,above,sloped]{$\mbox{0{,}7}$}(O)--(B1.west)node[pos=0.5,below]{$\mbox{?}$}(B1.east)--(B2.west)node[pos=0.5,above]{$\mbox{?}$}(A1.east)--(A2.west)node[pos=0.5,above]{$\mbox{?}$}
	(A1.east)--(a2.west)node[pos=0.5,below,sloped]{$\mbox{0{,}2}$}
	(B1.east)--(b2.west)node[pos=0.5,below,sloped]{$\mbox{0{,}3}$};
	%%Node dòng trên
	\path(A2)++(0,1)node{\textbf{Chủ nhật}}++(180:4)node{\textbf{Thứ bảy}};
	\end{tikzpicture}
	\end{center}
	\loigiai{
	Gọi $A$ là biến cố \lq\lq  Ngày thứ Bảy trời nắng\rq\rq và $B$ là biến cố \lq\lq  Ngày Chủ nhật trời mưa\rq\rq.\\
	Ta có $\mathrm{P}(A)=0{,}7 ; \mathrm{P}(B \mid A)=0{,}2 ; \mathrm{P}(B \mid \overline{A})=0{,}3$.\\
	Do đó $\mathrm{P}(\overline{A})=1-\mathrm{P}(A)=0{,}3 ;\, \mathrm{P}(\overline{B} \mid A)=1-\mathrm{P}(B \mid A)=0{,}8 ;\, \mathrm{P}(\overline{B} \mid \overline{A})=1-\mathrm{P}(B \mid \overline{A})=0{,}7$.
	Áp dụng công thức nhân xác suất, ta có xác suất trời nắng vào thứ Bảy và trời mưa vào Chủ nhật là
	$$
	\mathrm{P}(A B)=\mathrm{P}(A) \mathrm{P}(B \mid A)=0{,}7\cdot 0{,}2=0{,}14 .
	$$
	Tương tự, ta có
	\allowdisplaybreaks
	\begin{eqnarray*}
	&&\mathrm{P}(A \overline{B})=\mathrm{P}(A) \mathrm{P}(\overline{B} \mid A)=0{,}7\cdot 0{,}8=0{,}56 ; \\
	&&\mathrm{P}(\overline{A} B)=\mathrm{P}(\overline{A}) \mathrm{P}(B \mid \overline{A})=0{,}3\cdot 0{,}3=0{,}09 ; \\
	&&\mathrm{P}(\overline{A} \overline{B})=\mathrm{P}(\overline{A}) \mathrm{P}(\overline{B} \mid \overline{A})=0{,}3\cdot0{,}7=0{,}21.
	\end{eqnarray*}
	Ta có thể biểu diễn các kết quả trên theo sơ đồ hình cây như sau:
	\begin{center}
	\begin{tikzpicture}[yscale=0.7]
	\def\gocm{20}
	\def\gocn{10}
	\def\r{4}
	\tikzset{s/.style={outer sep=0.5 mm,draw=magenta,rectangle,minimum width=2.75cm,rounded corners=1mm}}
	\path(0,0)node(O){}++(\gocm:\r)node[s](A1){A}++(\gocn:\r)node[s](A2){$\overline{B}$}++(0:\r)node[s](A3){$A\overline{B}$}++(0:\r)node[s](A4){$0{,}56$};
	\path(A1)++({-\gocn}:\r)node[s](a2){B}++(0:\r)node[s](a3){$AB$}++(0:\r)node[s](a4){$0{,}14$};
	\path(O)++(-\gocm:\r)node[s](B1){$\overline{A}$}++(\gocn:\r)node[s](B2){$\overline{B}$}++(0:\r)node[s](B3){$\overline{A}\overline{B}$}++(0:\r)node[s](B4){$0{,}21$};
	\path(B1)++({-\gocn}:\r)node[s](b2){$B$}++(0:\r)node[s](b3){$\overline{A}B$}++(0:\r)node[s](b4){$0{,}09$};
	\foreach \x/\y in {
	O/A1,A1/A2,
	O/B1,B1/B2,
	A1/a2,
	B1/b2}
	\draw[-stealth](\x.east)--(\y.west);
	\path(O)--(A1.west)node[pos=0.5,above,sloped]{$0{,}7$}(O)--(B1.west)node[pos=0.5,below,sloped]{$0{,}3$}(B1.east)--(B2.west)node[pos=0.5,above,sloped]{$0{,}7$}(A1.east)--(A2.west)node[pos=0.5,above,sloped]{$0{,}8$}
	(A1.east)--(a2.west)node[pos=0.5,below,sloped]{$0{,}2$}
	(B1.east)--(b2.west)node[pos=0.5,below,sloped]{$0{,}3$};
	%%Node dòng trên
	\path(A2)++(0,1)node{\textbf{Chủ nhật}}++(180:4)node{\textbf{Thứ bảy}}(A3)++(0,1)node{\textbf{Kết quả}}(A4)++(0,1)node{\textbf{Xác suất}};
	\end{tikzpicture}
	\end{center}}
\end{bt}
\begin{bt}%[2D5H2-3]
	Hộp thứ nhất có $4$ viên bi xanh và $6$ viên bi đỏ. Hộp thứ hai có $5$ viên bi xanh và $4$ viên bi đỏ. Các viên bi có cùng kích thước và khối lượng. Lấy ra ngẫu nhiên $1$ viên bi từ hộp thứ nhất chuyển sang hộp thứ hai. Sau đó lại lấy ra ngẫu nhiên $1$ viên bi từ hộp thứ hai.
	Sử dụng sơ đồ hình cây, tính xác suất của các biến cố:
	$A$ : \lq\lq  Viên bi lấy ra từ hộp thứ nhất có màu xanh và viên bi lấy ra từ hộp thứ hai có màu đỏ\rq\rq ;
	$B$ : \lq\lq  Hai viên bi lấy ra có cùng màu\rq\rq.
	\loigiai{
	Gọi $X$ là biến cố: \lq\lq  Viên bi lấy ra từ hộp thứ nhất có màu xanh\rq\rq.\\
	$Y$ là biến cố: \lq\lq  Viên bi lấy ra từ hộp thứ hai có màu đỏ\rq\rq.\\
	Ta có
	$
	\mathrm{P}(Y|X)=0{,}4; \mathrm{P}(Y \mid \overline{X})=0{,}5 ; \mathrm{P}(X)=0{,}4
	$.\\
	Do đó $\mathrm{P}(\overline{X})=1-\mathrm{P}(X)=0{,}6 ; \mathrm{P}(\overline{Y} |X)=1-\mathrm{P}(Y|X)=0{,}6 ; \\\mathrm{P}(\overline{Y} \mid \overline{X})=1-\mathrm{P}(Y \mid \overline{X})=0{,}5$.\\
	Ta có sơ đồ hình cây như sau
	\begin{center}
	\begin{tikzpicture}[yscale=0.7]
	\def\gocm{20}
	\def\gocn{10}
	\def\r{4}
	\tikzset{s/.style={outer sep=0.5 mm,draw=magenta,rectangle,minimum width=2.75cm,rounded corners=1mm}}
	\path(0,0)node(O){}++(\gocm:\r)node[s](A1){X}++(\gocn:\r)node[s](A2){$Y$};
	\path(A1)++({-\gocn}:\r)node[s](a2){$\overline{Y}$};
	\path(O)++(-\gocm:\r)node[s](B1){$\overline{X}$}++(\gocn:\r)node[s](B2){$Y$};
	\path(B1)++({-\gocn}:\r)node[s](b2){$\overline{Y}$};
	\foreach \x/\y in {
	O/A1,A1/A2,
	O/B1,B1/B2,
	A1/a2,
	B1/b2}
	\draw[-stealth](\x.east)--(\y.west);
	\path(O)--(A1.west)node[pos=0.5,above,sloped]{$0{,}4$}(O)--(B1.west)node[pos=0.5,below,sloped]{$0{,}6$}(B1.east)--(B2.west)node[pos=0.5,above,sloped]{$0{,}5$}(A1.east)--(A2.west)node[pos=0.5,above,sloped]{$0{,}4$}
	(A1.east)--(a2.west)node[pos=0.5,below,sloped]{$0{,}6$}
	(B1.east)--(b2.west)node[pos=0.5,below,sloped]{$0{,}5$};
	\end{tikzpicture}
	\end{center}
	Khi đó $\mathrm{P}(A)=\mathrm{P}(XY)=0{,}4\cdot 0{,}4=0{,}16$;\\ $\mathrm{P}(B)=\mathrm{P}(X\overline{Y})+\mathrm{P}(\overline{X}Y)=0{,}4\cdot 0{,}6+0{,}6\cdot 0{,}5=0{,}54$.
	}
\end{bt}
\begin{bt}%[2D5V2-3]
	Một trường đại học tiến hành khảo sát tình trạng việc làm sau khi tốt nghiệp của sinh viên. Kết quả khảo sát cho thấy tỉ lệ người tìm được việc làm đúng chuyên ngành là $85 \%$ đối với sinh viên tốt nghiệp loại giỏi và $70 \%$ đối với sinh viên tốt nghiệp loại khác.
	Tỉ lệ sinh viên tốt nghiệp loại giỏi là $30 \%$. Gặp ngẫu nhiên một sinh viên đã tốt nghiệp của trường.
	Sử dụng sơ đồ hình cây, tính xác suất của các biến cố:
	C: \lq\lq  Sinh viên tốt nghiệp loại giỏi và tìm được việc làm đúng chuyên ngành\rq\rq;
	$D$ : \lq\lq  Sinh viên không tốt nghiệp loại giỏi và tìm được việc làm đúng chuyên ngành\rq\rq.
	\loigiai{
	Gọi $X$ là biến cố: \lq\lq  Sinh viên tốt nghiệp loại Giỏi\rq\rq.\\
	$Y$ là biến cố: \lq\lq  Sinh viên tìm được việc làm đúng chuyên ngành\rq\rq.\\
	Ta có
	$
	\mathrm{P}(Y|X)=0{,}85; \mathrm{P}(Y \mid \overline{X})=0{,}7 ; \mathrm{P}(X)=0{,}3
	$.\\
	Do đó $\mathrm{P}(\overline{X})=1-\mathrm{P}(X)=0{,}7 ; \mathrm{P}(\overline{Y} |X)=1-\mathrm{P}(Y|X)=0{,}15 ; \\\mathrm{P}(\overline{Y} \mid \overline{X})=1-\mathrm{P}(Y \mid \overline{X})=0{,}3$.\\
	Ta có sơ đồ hình cây như sau
	\begin{center}
	\begin{tikzpicture}[yscale=0.7]
	\def\gocm{20}
	\def\gocn{10}
	\def\r{4}
	\tikzset{s/.style={outer sep=0.5 mm,draw=magenta,rectangle,minimum width=2.75cm,rounded corners=1mm}}
	\path(0,0)node(O){}++(\gocm:\r)node[s](A1){X}++(\gocn:\r)node[s](A2){$Y$};
	\path(A1)++({-\gocn}:\r)node[s](a2){$\overline{Y}$};
	\path(O)++(-\gocm:\r)node[s](B1){$\overline{X}$}++(\gocn:\r)node[s](B2){$Y$};
	\path(B1)++({-\gocn}:\r)node[s](b2){$\overline{Y}$};
	\foreach \x/\y in {
	O/A1,A1/A2,
	O/B1,B1/B2,
	A1/a2,
	B1/b2}
	\draw[-stealth](\x.east)--(\y.west);
	\path(O)--(A1.west)node[pos=0.5,above,sloped]{$0{,}3$}(O)--(B1.west)node[pos=0.5,below,sloped]{$0{,}7$}(B1.east)--(B2.west)node[pos=0.5,above,sloped]{$0{,}7$}(A1.east)--(A2.west)node[pos=0.5,above,sloped]{$0{,}85$}
	(A1.east)--(a2.west)node[pos=0.5,below,sloped]{$0{,}15$}
	(B1.east)--(b2.west)node[pos=0.5,below,sloped]{$0{,}3$};
	\end{tikzpicture}
	\end{center}
	Khi đó $\mathrm{P}(C)=\mathrm{P}(XY)=0{,}3\cdot 0{,}85=0{,}255$; $\mathrm{P}(B)=\mathrm{P}(\overline{X}Y)=0{,}7\cdot 0{,}7=0{,}49$.
	}
\end{bt}
%=====================
\begin{dang}{Công thức nhân xác suất}
	Với hai biến cố $A$ và $B$ bất kì, ta có
	$$\mathrm{P}(A B)=\mathrm{P}(B) \cdot \mathrm{P}(A \mid B).$$
\end{dang}
%----------------------------
\subsubsection{Ví dụ minh hoạ}
\begin{vd}%[2D5H2-4]
	Trong một hộp kín có 7 chiếc bút bi xanh và 5 chiếc bút bi đen, các chiếc bút có cùng kích thước và khối lượng. Bạn Sơn lấy ngẫu nhiên một chiếc bút bi từ trong hộp, không trả lại. Sau đó bạn Tùng lấy ngẫu nhiên một trong 11 chiếc bút còn lại. Tính xác suất để Sơn lấy được bút bi đen và Tùng lấy được bút bi xanh.
	\loigiai{
	Gọi $A$ là biến cố: ``Bạn Sơn lấy được bút bi đen'';\\
	$B$ là biến cố: ``Bạn Tùng lấy được bút bi xanh''.\\
	Ta cần tính $\mathrm{P}(AB)$.\\
	Vì $n(A)=5$ nên $\mathrm{P}(A)=\dfrac{5}{12}$.\\
	Nếu $A$ xảy ra tức là bạn Sơn lấy được bút bi đen thì trong hộp có 11 bút bi với 7 bút bi xanh.\\
	Vậy $\mathrm{P}(B\mid A)=\dfrac{7}{11}$.\\
	Theo công thức nhân xác suất: $\mathrm{P}(AB)=\mathrm{P}(A)\cdot \mathrm{P}(B\mid A)=\dfrac{5}{12}\cdot\dfrac{7}{11}=\dfrac{35}{132}$.\\}
\end{vd}
\begin{vd}
	Cho hai biến cố $A$ và $B$ có $\mathrm{P}(A)=0{,}3 ; \mathrm{P}(B)=0{,}5$ và $\mathrm{P}(A \mid B)=0{,}4$. Tính $\mathrm{P}(\overline{A} B)$ và $\mathrm{P}(\overline{A} \mid B)$.
	\loigiai
	{
	\immini
	{
	Theo công thức nhân xác suất, ta có $\mathrm{P}(A B)=\mathrm{P}(B) \cdot \mathrm{P}(A \mid B)=0{,}2$.\\
	Vì $\overline{A} B$ và $A B$ là hai biến cố xung khắc và $\overline{A} B \cup A B=B$ nên theo tính chất của xác suất, ta có $\mathrm{P}(\overline{A} B)=\mathrm{P}(B)-\mathrm{P}(A B)=0{,}3$.\\
	Theo công thức tính xác suất có điều kiện,
	$$
	\mathrm{P}(\overline{A} \mid B)=\dfrac{\mathrm{P}(\overline{A} B)}{\mathrm{P}(B)}=\dfrac{0{,}3}{0{,}5}=0{,}6.$$
	}
	{
	\begin{tikzpicture}[scale=0.54]
	\def\firstven{(0,0) ellipse (3cm and 2cm)}
	\def\secondven{(2.5,1) ellipse (2.8cm and 2cm)}
	\begin{scope}
	\clip \firstven;
	\fill[gray!50,opacity=0.85] \secondven;
	\end{scope}
	\draw \firstven \secondven;
	\node at (-2.2,2) {$A$};
	\node at (5.6,2.2){$B$};
	\node at (1.3,0.5){$AB$};
	\node at (4,1){$\overline{A} B$};
	\end{tikzpicture}
	}
	}
\end{vd}
\begin{vd}%[2D5H2-4]
	Ô cửa bí mật (Let's Make a Deal) là một trò chơi trên truyền hình nổi tiếng ở Mỹ, đã được mua bản quyền và phát sóng ở nhiều nước trên thế giới. Nội dung trò chơi như sau:
	\begin{itemize}
	\item Người chơi được mời lên sân khấu và đứng trước ba cánh cửa đóng kín. Sau một cánh cửa có chiếc ô tô, sau mỗi cánh cửa còn lại là một con lừa. Người chơi được yêu cầu chọn ngẫu nhiên một cánh cửa, nhưng không được mở ra.
	\item Tiếp đó người quản trò tuyên bố sẽ mở ngẫu nhiên một trong hai cánh cửa người chơi không chọn mà sau cửa đó là con lừa. Người quản trò hỏi người chơi muốn giữ nguyên sự lựa chọn ban đầu của mình hay muốn chuyển sang cửa chưa mở còn lại.
	\end{itemize}
	Giả sử người chơi chọn cửa số 1 và người quản trò mở cửa số 3. Kí hiệu $E_1$; $E_2$; $E_3$ tương ứng là các biến cố: ``Sau ở cửa số 1 có ô tô''; ``Sau ở cửa số 2 có ô tô''; ``Sau ở cửa số 3 có ô tô'' và $H$ là biến cố: ``Người quản trò mở ở cửa số 3 thấy con lừa''.
	Sau khi người quản trò mở cánh cửa số 3 thấy con lừa, tức là khi $H$ xảy ra. Để quyết định thay đổi lựa chọn hay không, người chơi cần so sánh hai xác suất có điều kiện: $P\left(E_1\mid H\right)$ và $P\left(E_2\mid H\right)$
	\begin{listEX}
	\item Chứng minh rằng:
	\begin{enumEX}[\itemCI]{2}
	\item $\mathrm{P}(E_1)=\mathrm{P}(E_2)=\mathrm{P}(E_3)=\dfrac{1}{3}$;
	\item $P\left(H \mid E_1\right)=\dfrac{1}{2} ~\text{và}~P\left(H \mid E_2\right)=1$.
	\end{enumEX}
	\item Sử dụng công thức tính xác suất có điều kiện và công thức nhân xác suất, chứng minh rằng:
	$$ \mathrm{P}\left(E_1 \mid H\right)=\dfrac{\mathrm{P}(E_1) \cdot P\left(H \mid E_1\right)}{\mathrm{P}(H)}$$
	%	\item $P\left(E_2 \mid H\right)=\dfrac{\mathrm{P}(E_2) \cdot P\left(H \mid E_2\right)}{\mathrm{P}(H)$.
	\item Từ các kết quả trên hãy suy ra:
	$$P\left(E_2 \mid H\right)=2 P\left(E_1 \mid H\right)$$
	Từ đó hãy đưa ra lời khuyên cho người chơi: Nên giữ nguyên sự lựa chọn ban đầu hay chuyển sang cửa chưa mở còn lại?\\
	\end{listEX}
	\loigiai{
	\begin{listEX}
	\item Không gian mẫu $\Omega$ là tập hợp gồm 3 phần thưởng (1 ô tô + 2 con lừa) $\Rightarrow n(\Omega)=3$.
	Ta có $n(E_1)=n(E_2)=n(E_3)=1$.\\
	Suy ra $\mathrm{P}(E_1)=\mathrm{P}(E_2)=\mathrm{P}(E_3)=\dfrac{1}{3}$.\\
	Nếu $E_1$ xảy ra, tức là sau cửa số 1 có ô tô. Khi đó, sau cửa số 2 và 3 là con lừa. Người quản trò chọn ngẫu nhiên một trong hai cửa số 2 và 3 để mở ra. Do đó, việc chọn cửa số 2 hay cửa số 3 có khả năng như nhau. \\
	Vậy $P\left(H\mid E_1\right)=\dfrac{1}{2}$.\\
	Nếu $E_2$ xảy ra, tức là cửa số 2 có ô tô. Khi đó, người quản trò chắc chắn phải mở cửa số 3.\\
	Do đó $P\left(H\mid E_2\right)=1$.
	\item Ta có $$\begin{aligned} &P\left(E_1\mid H\right)=\dfrac{\mathrm{P}(E_1H)}{\mathrm{P}(H)}\\\Leftrightarrow &\mathrm{P}(E_1H)=\mathrm{P}(E_1)\cdot P\left(H\mid E_1\right)\\\Leftrightarrow &P\left(E_1\mid H\right)=\dfrac{\mathrm{P}(E_1)\cdot P\left(H\mid E_1\right)}{\mathrm{P}(H)}.\end{aligned}$$
	\item Ta có $$P\left(E_2 \mid H\right)=\dfrac{\mathrm{P}(E_2) \cdot P\left(H\mid E_2\right)}{\mathrm{P}(H)}.$$
	Suy ra
	$$\begin{aligned}
	\dfrac{P\left(E_2 \mid H\right)}{P\left(E_1 \mid H\right)}&=\dfrac{\mathrm{P}(E_2) \cdot P\left(H \mid E_2\right)}{\mathrm{P}(E_1) \cdot P\left(H \mid E_1\right)}\\
	&=\dfrac{\dfrac{1}{3} \cdot 1}{\dfrac{1}{3} \cdot\dfrac{1}{2}}=2.
	\end{aligned}$$
	Suy ra $P\left(E_2\mid H\right)=2\cdot P\left(E_1\mid H\right)$.\\
	\textbf{\textit{Nhận xét: }}Từ kết quả trên ta thấy người chơi nên chuyển sang cửa chưa mở còn lại để tăng gấp đôi khả năng trúng thưởng chiếc ô tô.
	\end{listEX}
	}
\end{vd}
\begin{vd}%[2D5H2-4]
	Một nhóm $5$ học sinh nam và $4$ học sinh nữ tham gia lao động trên sân trường. Cô giáo chọn ngẫu nhiên đồng thời $2$ bạn trong nhóm đi tưới cây. Tính xác suất để hai bạn được chọn có cùng giới tính, biết rằng có ít nhất $1$ bạn nam được chọn.
	\loigiai{
	Số phần tử của không gian mẫu là $n(\Omega)=C^2_9=36$.\\
	Gọi A là biến cố \lq\lq  Hai bạn được chọn có cùng giới tính\rq\rq.\\
	B là biến cố \lq\lq  Có ít nhất một bạn nam được chọn\rq\rq.\\
	Ta có $n(B)=C^2_5+C^1_5\cdot C^1_4=30$ suy ra $\mathrm{P}(B)=\dfrac{30}{36}$.\\
	Ta có $n(AB)=C^2_5=10$ suy ra $\mathrm{P}(AB)=\dfrac{10}{36}$.\\
	Vậy $\mathrm{P}(A|B)=\dfrac{\mathrm{P}(AB)}{\mathrm{P}(B)}=\dfrac{10}{30}=\dfrac{1}{3}$.
	}
\end{vd}
%----------------------------
\subsubsection{Bài tập áp dụng}
\begin{bt}%[2D5H1-2]
	Cho hai biến cố $A$ và $B$ có $\mathrm{P}(A)=0{,}4; \mathrm{P}(B)=0{,}8$ và $\mathrm{P}(A|\overline{B})=0{,}5$. Tính $\mathrm{P}(A\overline{B})$ và $\mathrm{P}(A|B)$.
	\loigiai{
	Ta có $\mathrm{P}(A\overline{B})=\mathrm{P}(A|\overline{B})\cdot \mathrm{P}(\overline{B})=0{,}5\cdot 0{,}2=0{,}1$.\\
	Vì $AB$ và $A\overline{B}$ là hai biến cố xung khắc và $AB\cup A\overline{B}=A$ nên theo tính chất của xác suất, ta có\\ $\mathrm{P}(AB)=\mathrm{P}(A)-\mathrm{P}(A\overline{B})=0{,}4-0{,}1=0{,}3$.\\
	Khi đó: $\mathrm{P}(A|B)=\dfrac{\mathrm{P}(AB)}{\mathrm{P}(B)}=\dfrac{0{,}3}{0{,}8}=0{,}375$.
	}
\end{bt}
\begin{bt}%[2D5H2-3]
	\immini{Máy tính và thiết bị lưu điện (UPS) được kết nối như hình 5. Khi xảy ra sự cố điện, UPS bị hỏng với xác suất $0{,}02$. Nếu UPS bị hỏng khi xảy ra sự cố điện, máy tính sẽ bị hỏng với xác suất $0{,}1$; ngược lại, nếu UPS không bị hỏng, máy tính sẽ không bị bỏng.
	\begin{listEX}
	\item Tính xác suất để cả UPS và máy tính đều không bị hỏng khi xảy ra sự cố điện.
	\item Tính xác suất để cả UPS và máy tính đều bị hỏng khi xảy ra sự cố điện.
	\end{listEX}
	}
	{\includegraphics[width=6.5cm,height=4cm]{images/12-SGK-CTST-6-1-5}}
	\loigiai{
	Gọi $A$ là biến cố \lq\lq  UPS bình thường\rq\rq\, và $B$ là biến cố: \lq\lq  Máy tính bình thường\rq\rq.\\
	Ta có
	$
	\mathrm{P}(B|A)=1; \mathrm{P}(B\mid \overline{A})=0{,}9 ; \mathrm{P}(A)=0{,}98
	$.\\
	Do đó $\mathrm{P}(\overline{A})=1-\mathrm{P}(A)=0{,}02 ; \mathrm{P}(\overline{B} \mid \overline{A})=1-\mathrm{P}(B \mid \overline{A})=0{,}1$.\\
	Ta có sơ đồ hình cây như sau
	\begin{center}
	\begin{tikzpicture}[yscale=0.6]
	\def\gocm{20}
	\def\gocn{10}
	\def\r{4}
	\tikzset{s/.style={outer sep=0.5 mm,draw=magenta,rectangle,minimum width=2.75cm,rounded corners=1mm}}
	\path(0,0)node(O){}++(\gocm:\r)node[s](A1){A}++(\gocn:\r)node[s](A2){$B$};
	\path(O)++(-\gocm:\r)node[s](B1){$\overline{A}$}++(\gocn:\r)node[s](B2){$B$};
	\path(B1)++({-\gocn}:\r)node[s](b2){$\overline{B}$};
	\foreach \x/\y in {
	O/A1,A1/A2,
	O/B1,B1/B2,
	B1/b2}
	\draw[-stealth](\x.east)--(\y.west);
	\path(O)--(A1.west)node[pos=0.5,above,sloped]{$0{,}98$}(O)--(B1.west)node[pos=0.5,below,sloped]{$0.02$}(B1.east)--(B2.west)node[pos=0.5,above,sloped]{$0{,}9$}(A1.east)--(A2.west)node[pos=0.5,above,sloped]{$1$}
	(B1.east)--(b2.west)node[pos=0.5,below,sloped]{$0{,}1$};
	\end{tikzpicture}
	\end{center}
	\begin{listEX}
	\item $\mathrm{P}(AB)=0{,}98\cdot 1=0{,}98$
	\item $\mathrm{P}(\overline{A}.\overline{B})=0{,}02\cdot 0{,}1=0{,}002$.
	\end{listEX}
	}
\end{bt}
\begin{bt}%[2D5H2-4]
	Công ty nước giải khát $X$ tổ chức một chương trình khuyến mại như sau: Trong mỗi thùng 24 chai nước giải khát đều có hai chai trúng thưởng (giải thưởng được viết ở dưới nắp chai), người tham gia chương trình được mở nắp một cách ngẫu nhiên lần lượt hai chai trong một thùng. Tính xác suất để một người tham gia chương trình mở được cả hai chai đều trúng thưởng.
	\loigiai{
		Gọi $A$ là biến cố \lq\lq  chai thứ nhất có trúng thưởng\rq\rq, và $B$ là biến cố \lq\lq  chai thứ hai có trúng thưởng\rq\rq.\\
		Xác suất của $A$ là xác suất để lấy ra một chai có trúng thưởng lần đầu tiên là $$\mathrm{P}(A)=\dfrac{n(A)}{n(\Omega)}=\dfrac{2}{24}=\dfrac{1}{12}.$$
		Sau khi lấy một chai trúng thưởng, số chai trúng trưởng còn $1$ trong tổng số $23$ chai nước.
		Xác suất của $B$ khi đã xảy ra $A$ là xác suất để lấy ra một chai trúng thưởng lần thứ hai là $$\mathrm{P}(B \mid A)=\dfrac{1}{23}.$$
		Áp dụng quy tắc nhân xác suất, ta có
		$$\mathrm{P}(A \cap B)=\mathrm{P}(A) \cdot \mathrm{P}(B \mid A)=\dfrac{1}{2} \cdot \dfrac{1}{23}=\dfrac{1}{46}.$$
		Vậy, xác suất để cả hai chai nước đều trúng thưởng là $\dfrac{1}{46}$.
	}
\end{bt}

