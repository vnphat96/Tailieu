\Opensolutionfile{ans}[ans/ansCD2D1-4-2]
\begin{dang}{Tìm đường tiệm cận hàm số}.
\end{dang}
\paragraph{Các ví dụ}

\begin{vd}%[2D1Y4-1]%[SỞ GD-ĐT VĨNH PHÚC-LẦN 1-2018]%Ví dụ 1.
	Cho hàm số $y=\dfrac{3x-1}{2x-1}$ có đồ thị $(C)$. Khẳng định nào sau đây là đúng?
	\choice
	{Đường thẳng $y=-3$ là tiệm cận ngang của đồ thị $(C)$}
	{Đường thẳng $y=\dfrac{3}{2}$ là tiệm cận đứng của đồ thị $(C)$}
	{\True Đường thẳng $x=\dfrac{1}{2}$ là tiệm cận đứng của đồ thị $(C)$}
	{Đường thẳng $y=-\dfrac{1}{2}$ là tiệm cận ngang của đồ thị $(C)$.}
	\loigiai{}
\end{vd}

\begin{vd}%[2D1B4-1]%[SỞ GD-ĐT BẮC GIANG -LẦN 1-2018]%Ví dụ 2.
	Đồ thị của hàm số nào sau đây có tiệm cận ngang?
	\choice
	{\True $y=\dfrac{x}{x^2+1}$}
	{$y=\dfrac{x^2}{x+1}$}
	{$y=\dfrac{x^2-3x+2}{x-1}$}
	{$y=\dfrac{\sqrt{4-x^2}}{1+x}$}
	\loigiai{
	Vì $\lim\limits_{x\to\infty}\dfrac{x}{x^2+1}=0$ nên đồ thị hàm số nhận đường thẳng $y=0$ là tiệm cận ngang.\\
	Vì $\lim\limits_{x\to\pm\infty}\dfrac{x^2}{x+1}=\pm\infty$ nên đồ thị hàm số không có tiệm cận ngang.\\
	Vì $y=\dfrac{x^2-3x+2}{x-1}=x-2$ nên đồ thị hàm số không có tiệm cận ngang.\\
	Vì hàm số $y=\dfrac{\sqrt{4-x^2}}{1+x}$ có tập xác định là $[-2;2]\setminus\{-1\}$ nên đồ thị hàm số không có tiệm cận ngang.
	}
\end{vd}

\begin{vd}%[2D1B4-1]%[THPT Chuyên Hùng Vương-Gia Lai-lần 1 năm 2017-2018]%Ví dụ 3.
	Đồ thị của hàm số $y=\dfrac{3x^2-7x+2}{2x^2-5x+2}$ có bao nhiêu tiệm cận đứng?
	\choice
	{$2$}
	{\True $1$}
	{$3$}
	{$4$}
	\loigiai{
	Tập xác định $\mathbb{R}\setminus\left\{\dfrac{1}{2};2\right\}$.\\
	$\lim\limits_{x\to 2}\dfrac{3x^2-7x+2}{2x^2-5x+2} =\lim\limits_{x\to 2}\dfrac{(3x-1)(x-2)}{(2x-1)(x-2)} =\lim\limits_{x\to 2}\dfrac{3x-1}{2x-1}=\dfrac{5}{3}$ nên $x=2$ không phải là tiệm cận đứng của đồ thị hàm số $y=\dfrac{3x^2-7x+2}{2x^2-5x+2}$.\\
	$\lim\limits_{x\to\left(\dfrac{1}{2}\right)^+}\dfrac{3x^2-7x+2}{2x^2-5x+2}=+\infty$, $\lim\limits_{x\to\left(\dfrac{1}{2}\right)^-}\dfrac{3x^2-7x+2}{2x^2-5x+2}=-\infty$ nên $x=\dfrac{1}{2}$ là tiệm cận đứng của đồ thị hàm số $y=\dfrac{3x^2-7x+2}{2x^2-5x+2}$.
	}
\end{vd}

\begin{vd}%[2D1K4-1]%[THPT Lương Thế Vinh-Hà Nội năm 2017-2018]%Ví dụ 4.
	Đồ thị hàm số $y=\sqrt{4x^2+4x+3}-\sqrt{4x^2+1}$ có bao nhiêu tiệm cận ngang?
	\choice
	{\True $2$}
	{$0$}
	{$1$}
	{$3$}
	\loigiai{
	Tập xác định $\mathscr{D}=\mathbb{R}$.\\
	Ta có \allowdisplaybreaks
	\begin{eqnarray*}
		\lim\limits_{x\to+\infty} y &=&\lim\limits_{x\to+\infty}\left(\sqrt{4x^2+4x+3}-\sqrt{4x^2+1}\right)\\
		&=&\lim\limits_{x\to+\infty}\dfrac{4x+2}{\sqrt{4x^2+4x+3}+\sqrt{4x^2+1}}\\
		&=&\lim\limits_{x\to+\infty}\dfrac{4+\dfrac{2}{x}}{\sqrt{4+\dfrac{4}{x}+\dfrac{3}{x^2}}+\sqrt{4+\dfrac{1}{x^2}}}=1
	\end{eqnarray*}
	suy ra đường thẳng $y=1$ là tiệm cận ngang.\\
	Ta có \allowdisplaybreaks
	\begin{eqnarray*}
	\lim\limits_{x\to-\infty} y&=&\lim\limits_{x\to-\infty}\left(\sqrt{4x^2+4x+3}-\sqrt{4x^2+1}\right)\\
	&=&\lim\limits_{x\to-\infty}\dfrac{4x+2}{\sqrt{4x^2+4x+3}+\sqrt{4x^2+1}}\\
	&=&\lim\limits_{x\to-\infty}\dfrac{4+\dfrac{2}{x}}{-\sqrt{4+\dfrac{4}{x}+\dfrac{3}{x^2}}-\sqrt{4+\dfrac{1}{x^2}}}=-1
	\end{eqnarray*}
 suy ra đường thẳng $y=-1$ là tiệm cận ngang.\\
	Vậy đồ thị hàm số có $2$ tiệm cận ngang.
	}
\end{vd}

\begin{vd}%[2D1K4-1]%[THPT Chuyên Phan Bội Châu-lần 2 năm 2017-2018]%Ví dụ 5.
	Số đường tiệm cận đứng của đồ thị hàm số $y=\dfrac{\sqrt{4x^2-1}+3x^2+2}{x^2-x}$ là
	\choice
	{$2$}
	{$3$}
	{$0$}
	{\True $1$}
	\loigiai{
	Tập xác định $\mathscr{D}=\left(-\infty;-\dfrac{1}{2}\right]\cup\left[\dfrac{1}{2};+\infty\right)\setminus \{1\} $.\\
	Ta có $\lim\limits_{x\to 1^+} y=\lim\limits_{x\to 1^+}\dfrac{\sqrt{4x^2-1}+3x^2+2}{x^2-x}=\lim\limits_{x\to 1^+}\dfrac{\sqrt{4x^2-1}+3x^2+2}{x(x-1)}=+\infty$.\\
	(do $\lim\limits_{x\to 1^+}\dfrac{\sqrt{4x^2-1}+3x^2+2}{x}=5+\sqrt{3}>0$, $\lim\limits_{x\to 1^+}(x-1)=0$ và $x-1>0$ ).\\
	Do đó đồ thị hàm số có một tiệm cận đứng là đường thẳng có phương trình $x=1$.}
\end{vd}

\paragraph{Câu hỏi trắc nghiệm}
\begin{ex}%[2D1Y4-1]%[THPT Triệu Thị Trinh-lần 1 năm 2017-2018]%Câu 1.
	Tiệm cận ngang của đồ thị hàm số $y=\dfrac{x+1}{1-x}$ là
	\choice
	{\True $y=-1$}
	{$x=1$}
	{$y=0$}
	{$x=-1$}
	\loigiai{
	Ta có $\lim\limits_{x\to\infty}\dfrac{x+1}{1-x}=-1$ nên đồ thị hàm số đã cho có tiệm cận ngang là đường thẳng $y=-1$.
	}
\end{ex}

\begin{ex}%[2D1B4-1]%[THPT Hà Huy Tập-Hà Tĩnh-lần 2 năm 2017-2018]%Câu 2.
	Đồ thị hàm số $y=\dfrac{2x-3}{x-1}$ có các đường tiệm cận đứng và tiệm cận ngang lần lượt là
	\choice
	{$x=-1$ và $y=-3$}
	{$x=-1$ và $y=3$}
	{\True $x=1$ và $y=2$}
	{$x=2$ và $y=1$}
	\loigiai{
	Ta có $\lim\limits_{x\to \pm\infty} y=\lim\limits_{x\to\pm\infty}\dfrac{2x-3}{x-1} =\lim\limits_{x\to\pm\infty}\dfrac{2-\dfrac{3}{x}}{1-\dfrac{1}{x}}=2$ nên đường thẳng $y=2$ là tiệm cận ngang.\\
	$\lim\limits_{x\to 1^+} y=\lim\limits_{x\to 1^+}\dfrac{2x-3}{x-1}=-\infty$ và $\lim\limits_{x\to 1^-} y=\lim\limits_{x\to 1^-}\dfrac{2x-3}{x-1}=+\infty$ suy ra đường thẳng $x=1$ là tiệm cận đứng.}
\end{ex}

\begin{ex}%[2D1B4-1]%[THPT Lý Thái Tổ-Bắc Ninh-lần 1 năm 2017-2018]%Câu 3.
	Gọi $I$ là giao điểm của hai đường tiệm cận của đồ thị hàm số $y=\dfrac{2x-3}{x+1}$. Khi đó, điểm $I$ nằm trên đường thẳng có phương trình 
	\choice
	{$x+y+4=0$}
	{\True $2x-y+4=0$}
	{$x-y+4=0$}
	{$2x-y+2=0$}
	\loigiai{
	Đồ thị hàm số đã cho có đường tiệm cận đứng là $x=-1$, tiệm cận ngang là $y=2$, do đó $I(-1;2)$, thay vào các phương trình thì $I$ thuộc đường thẳng $2x-y+4=0$.}
\end{ex}

\begin{ex}%[2D1Y4-1]%[THPT Can Lộc-Hà Tĩnh-lần 1 năm 2017-2018]%Câu 4.
	Cho hàm số $y=\dfrac{2x-3}{x+1}$ có đồ thị là $(C)$. Mệnh đề nào sau đây là đúng?
	\choice
	{\True $(C)$ có tiệm cận ngang là $y=2$}
	{$(C)$ chỉ có một tiệm cận}
	{$(C)$ có tiệm cận ngang là $x=2$}
	{$(C)$ có tiệm cận đứng là $x=1$}
	\loigiai{
	Do $\lim\limits_{x\to-\infty} y=\lim\limits_{x\to-\infty}\dfrac{2x-3}{x+1}=2$ và $\lim\limits_{x\to+\infty} y=\lim\limits_{x\to+\infty}\dfrac{2x-3}{x+1}=2$ nên đường thẳng $y=2$ là đường tiệm cận ngang của $(C)$.}
\end{ex}

\begin{ex}%[2D1Y4-1]%[THPT Quảng Xương I – Thanh Hóa – năm 2017 – 2018]%Câu 5.
	Cho hàm số $y=\dfrac{x-2}{x-1}$. Đường tiệm cận đứng của đồ thị hàm số là
	\choice
	{$y=1$}
	{$x=2$}
	{$y=2$}
	{\True $x=1$}
	\loigiai{
	Ta có $\lim\limits_{x\to 1^+}\dfrac{x-2}{x-1}=-\infty$ và $\lim\limits_{x\to 1^-}\dfrac{x-2}{x-1}=+\infty$ do đó đường thẳng $x=1$ là đường tiệm cận đứng của đồ thị hàm số.}
\end{ex}

\begin{ex}%[2D1Y4-1]%[THPT Chuyên Thoại Ngọc Hầu – An Giang - Lần 3 năm 2017 – 2018]%Câu 6.
	Đồ thị của hàm số nào dưới đây có tiệm cận đứng
	\choice
	{\True $y=\dfrac{x+2}{x-1}$}
	{$y=\dfrac{x^3}{x^2+2}$}
	{$y=\sqrt{x^2+1}$}
	{$y=\dfrac{x^2-5x+6}{x-2}$}
	\loigiai{
	Ta có $\lim\limits_{x\to 1^+}\dfrac{x+2}{x-1}=+\infty$ và $\lim\limits_{x\to 1^-}\dfrac{x+2}{x-1}=-\infty$ nên đồ thị của hàm số $y=\dfrac{x+2}{x-1}$ có tiệm cận đứng là đường thẳng $x=1$.}
\end{ex}

\begin{ex}%[2D1Y4-1]%[THPT Chuyên Ngữ – Hà Nội - Lần 1 năm 2017 – 2018]%Câu 7.
	Đồ thị nào dưới đây có tiệm cận ngang?
	\choice
	{$y=x^3-x-1$}
	{$y=\dfrac{x^3+1}{x^2+1}$}
	{\True $y=\dfrac{3x^2+2x-1}{4x^2+5}$}
	{$y=\sqrt{2x^2+3}$}
	\loigiai{
	Ta có $\lim\limits_{x\to\infty}\dfrac{3x^2+2x-1}{4x^2+5}=\dfrac{3}{4}\Rightarrow y=\dfrac{3}{4}$ là tiệm cận ngang của đồ thị hàm số $y=\dfrac{3x^2+2x-1}{4x^2+5}$.}
\end{ex}

\begin{ex}%[2D1Y4-1]%[THPT Chuyên Nguyễn Quang Diệu – Đồng Tháp – Lần 5 năm 2017 – 2018]%Câu 8.
	Đồ thị hàm số nào dưới đây có tiệm cận đứng?
	\choice
	{$y=2^x$}
	{\True $y=\log_2x$}
	{$y=\dfrac{x^2}{x^2+1}$}
	{$y=\dfrac{x^2-4x+3}{x-1}$}
	\loigiai{
	Hàm số $y=\log_2x$ có tiệm cận đứng là trục $Ox$.
	}
\end{ex}

\begin{ex}%[2D1Y4-1]%[CHUYÊN QUANG TRUNG BÌNH PHƯỚC-LẦN 4-2018]%Câu 9.
	Phương trình tiệm cận đứng của đồ thị hàm số $y=\dfrac{x-1}{x+1}$ là
	\choice
	{\True $x=-1$}
	{$y=1$}
	{$y=-1$}
	{$x=1$}
	\loigiai{
	Tập xác định $\mathscr{D}=\mathbb{R}\setminus\{-1\}$.\\
	Ta có $\lim\limits_{x\to-1^+} y =\lim\limits_{x\to-1^+}\dfrac{x-1}{x+1}=-\infty \Rightarrow x=-1$ là tiệm cận đứng của đồ thị hàm số đã cho.}
\end{ex}

\begin{ex}%[2D1Y4-1]%[SỞ GD VÀ ĐT ĐÀ NẴNG 2017-2018]%Câu 10.
	Cho hàm số $y=\dfrac{2}{x-2}$. Đường tiệm cận ngang của đồ thị hàm số là
	\choice
	{$y=-1$}
	{$x=2$}
	{$y=2$}
	{\True $y=0$}
	\loigiai{}
\end{ex}

\begin{ex}%[2D1B4-1]%[THPT Lục Ngạn-Bắc Ninh-lần 1 năm 2017-2018]%Câu 11.
	Cho đồ thị $(C)$: $y=\dfrac{\sqrt{x^2-4}}{x+1}$. Đồ thị $(C)$ có bao nhiêu đường tiệm cận?
	\choice
	{$0$}
	{$1$}
	{\True $2$}
	{$3$}
	\loigiai{
	Tập xác định $\mathscr{D}=(-\infty;-2]\cup[2;+\infty)$.\\
	Ta có $\lim\limits_{x\to-\infty} y=\lim\limits_{x\to-\infty}\dfrac{\sqrt{x^2-4}}{x+1}=-1\Rightarrow y=-1$ là tiệm cận ngang của đồ thị hàm số.\\
	$\lim\limits_{x\to+\infty} y=\lim\limits_{x\to+\infty}\dfrac{\sqrt{x^2-4}}{x+1}=1 \Rightarrow y=1$ là tiệm cận ngang của đồ thị hàm số.\\
	$\lim\limits_{x\to(-1)^+} y=\lim\limits_{x\to(-1)^+}\dfrac{\sqrt{x^2-4}}{x+1}$ không tồn tại và $\lim\limits_{x\to(-1)^-} y=\lim\limits_{x\to(-1)^-}\dfrac{\sqrt{x^2-4}}{x+1}$ không tồn tại nên đồ thị $(C)$ không có tiệm cận đứng.\\
	Vậy đồ thị $(C)$ có $2$ đường tiệm cận.
	}
\end{ex}

\begin{ex}%[2D1B4-1]%[THPT Chuyên Thái Bình-lần 3 năm 2017-2018]%Câu 12.
	Tổng số các đường tiệm cận đứng và tiệm cận ngang của đồ thị hàm số $y=\dfrac{x+2}{\sqrt{16-x^4}}$ là 
	\choice
	{$3$}
	{$0$}
	{$2$}
	{\True $1$}
	\loigiai{
	Điều kiện: $16-x^4>0\Leftrightarrow-2<x<2$.\\
	Tập xác định $\mathscr{D}=(-2;2)$.\\
	Từ tập xác định $D$ suy ra đồ thị hàm số không có tiệm cận ngang.\\
	Ta có $y=\dfrac{x+2}{\sqrt{(2-x)(2+x)\left(4+x^2\right)}}=\dfrac{\sqrt{x+2}}{\sqrt{(2-x)\left(4+x^2\right)}}\to+\infty$ khi $x\to 2^-$ nên đồ thị hàm số có đường tiệm cận đứng là $x=2$.\\
	Vậy tổng số các đường tiệm cận đứng và tiệm cận ngang là $1$.}
\end{ex}

\begin{ex}%[2D1B4-1]%[SGD Hà Tĩnh – Lần 2 năm 2017 – 2018]%Câu 13.
	Tổng số đường tiệm cận đứng và ngang của đồ thị hàm số $y=\dfrac{3x+1}{x^2-4}$ là 
	\choice
	{\True $3$}
	{$1$}
	{$2$}
	{$4$}
	\loigiai{
	Ta có:
	\begin{enumerate}[•]
		\item $\lim\limits_{x\to+\infty} y=\lim\limits_{x\to-\infty} y=0$ nên đồ thị hàm số có tiệm cận ngang $y=0$.
		\item $\heva{&\lim\limits_{x\to(-2)^+} y=+\infty\\&\lim\limits_{x\to(-2)^-} y=-\infty}$ nên đồ thị hàm số có tiệm cận đứng $x=-2$.
		\item $\heva{&\lim\limits_{x\to 2^+} y=+\infty\\&\lim\limits_{x\to 2^-} y=-\infty}$ nên đồ thị hàm số có tiệm cận đứng $x=2$.
	\end{enumerate}
	Vậy tổng số đường tiệm cận đứng và ngang của đồ thị hàm số $y=\dfrac{3x+1}{x^2-4}$ là $3$.}
\end{ex}

\begin{ex}%[2D1Y4-1]%[THPT YÊN ĐINH THANH HÓA-LẦN 1-2018]%Câu 14.
	Đường thẳng nào dưới đây là tiệm cận đứng của đồ thị hàm số $y=\dfrac{2x-5}{x-3}$ 
	\choice
	{$x=2$}
	{$x=-3$}
	{\True $x=3$}
	{$y=3$}
	\loigiai{
	Ta có $\lim\limits_{x\to 3^+}\dfrac{2x-5}{x-3}=+\infty$, $\lim\limits_{x\to 3^-}\dfrac{2x-5}{x-3}=-\infty\Rightarrow x=3$ là tiệm cận đứng của đồ thị hàm số.}
\end{ex}

\begin{ex}%[2D1B4-1]%[Đề Thử Nghiệm - Mã đề 01 - 2018]%Câu 15.
	Đồ thị hàm số nào dưới đây có tiệm cận ngang?
	\choice
	{$y=\dfrac{\sqrt{4-x^2}}{x}$}
	{\True $y=\dfrac{\sqrt{x-1}}{x+1}$}
	{$y=\dfrac{x^2+1}{x}$}
	{$y=\sqrt{x^2-1}$}
	\loigiai{
	$\lim\limits_{x\to\infty}\dfrac{\sqrt{x-1}}{x+1}=0$ nên đồ thị hàm số $y=\dfrac{\sqrt{x-1}}{x+1}$ có tiệm cận ngang.}
\end{ex}

\begin{ex}%[2D1B4-1]%[Thử nghiệm - MD2 - 2018]%Câu 16.
	Đồ thị hàm số nào dưới đây \textbf{không} có tiệm cận đứng?
	\choice
	{$y=\dfrac{x-1}{x+1}$}
	{$y=\dfrac{x^2+1}{x+1}$}
	{$y=\dfrac{2}{x+1}$}
	{\True $y=\dfrac{x^2+3x+2}{x+1}$}
	\loigiai{
	Ta có $y=\dfrac{x^2+3x+2}{x+1}=x+2, \forall x \ne -1$.\\
	$\lim\limits_{x\to-1^+} y=\lim\limits_{x\to-1^+} (x+2)=1;\lim\limits_{x\to-1^-} y=\lim\limits_{x\to-1^-} (x+2)=1$.\\
	Vậy đồ thị hàm số $y=\dfrac{x^2+3x+2}{x+1}$ không có tiệm cận đứng.}
\end{ex}

\begin{ex}%[2D1B4-1]%[Đề thực nghiệm - 03-2018]%Câu 17.
	Đồ thị của hàm số nào dưới đây \textbf{không} có tiệm cận ngang?
	\choice
	{$y=\dfrac{\cos x}{x}$}
	{$y=\dfrac{\sin x}{x}$}
	{$y=\dfrac{\sqrt{x^3+1}}{x^2}$}
	{\True $y=\dfrac{\sqrt{x^3+1}}{x}$}
	\loigiai{
	Xét hàm số $y=\dfrac{\sqrt{x^3+1}}{x}$ có tập xác định $\mathscr{D}=[-1;+\infty)\setminus\{0\}$.\\
	$\lim\limits_{x\to+\infty} y=\lim\limits_{x\to+\infty}\dfrac{\sqrt{x^3+1}}{x}=\lim\limits_{x\to+\infty}\dfrac{x\sqrt{x}\cdot\sqrt{1+\dfrac{1}{x^3}}}{x}=\lim\limits_{x\to+\infty}\sqrt{x}\cdot\sqrt{1+\dfrac{1}{x^3}}=+\infty$.\\
	Vậy đồ thị hàm số $y=\dfrac{\sqrt{x^3+1}}{x}$ không có tiệm ngang.}
\end{ex}

\begin{ex}%[2D1B4-1]%[THPT CHUYÊN QUANG TRUNG - BP - LẦN 1 - 2018]%Câu 18.
	Tiệm cận đứng của đồ thị hàm số $y=\dfrac{x^3-3x-2}{x^2+3x+2}$ là đường thẳng: 
	\choice
	{\True $x=-2$}
	{Không có tiệm cận đứng}
	{$x=-1$; $x=-2$}
	{$x=-1$}
	\loigiai{
	*Tập xác định $\mathscr{D}=\mathbb{R}\setminus\{-1;-2\}$.\\
	* Ta có $\lim\limits_{x\to-1}\dfrac{x^3-3x-2}{x^2+3x+2}=\lim\limits_{x\to-1}\dfrac{x^2-x-2}{x+2}=0$; $\lim\limits_{x\to-2}\dfrac{x^3-3x-2}{x^2+3x+2}=\infty$ \\
	$ \Rightarrow $ Đồ thị hàm số đã cho có một tiệm cận đứng duy nhất là đường thẳng $x=-2$.}
\end{ex}

\begin{ex}%[2D1B4-1]%[THPT XUÂN HÒA - VP - LẦN 1 - 2018]%Câu 19.
	Đồ thị hàm số $y=\dfrac{2x-3}{x-1}$ có các đường tiệm cận đứng và tiệm cận ngang lần lượt là
	\choice
	{$x=2$ và $y=1$}
	{$x=1$ và $y=-3$}
	{$x=-1$ và $y=2$}
	{\True $x=1$ và $y=2$}
	\loigiai{
	Ta có $\lim\limits_{x\to+\infty} y=\lim\limits_{x\to+\infty}\dfrac{2x-3}{x-1}=\lim\limits_{x\to+\infty}\dfrac{2-\dfrac{3}{x}}{1-\dfrac{1}{x}}=2$, $\lim\limits_{x\to-\infty} y=\lim\limits_{x\to-\infty}\dfrac{2x-3}{x-1}=\lim\limits_{x\to-\infty}\dfrac{2-\dfrac{3}{x}}{1-\dfrac{1}{x}}=2$.\\
	Do đó đường tiệm cận ngang của đồ thị hàm số là $y=2$.\\
	Và $\lim\limits_{x\to 1^+} y=\lim\limits_{x\to 1^+}\dfrac{2x-3}{x-1}=-\infty$, $\lim\limits_{x\to 1^-} y=\lim\limits_{x\to 1^-}\dfrac{2x-3}{x-1}=+\infty$.\\
	Do đó đường tiệm cận đứng của đồ thị hàm số là $x=1$.}
\end{ex}

\begin{ex}%[2D1K4-1]%[AN LÃO HẢI PHÒNG_LẦN 3-2018]%Câu 20.
	Đồ thị của hàm số nào sau đây có tiệm cận ngang?
	\choice
	{$y=\dfrac{\sqrt{4-x^2}}{x}$}
	{\True $y=\dfrac{\sqrt{x-1}}{x+1}$}
	{$y=\dfrac{x^2+1}{x}$}
	{$y=\sqrt{x^2-1}$}
	\loigiai{
	Hàm số $y=\dfrac{\sqrt{4-x^2}}{x}$ có tập xác định $\mathscr{D}=[-2; 2]\setminus\{0\}$ nên nó không có tiệm cận ngang.\\
	Hàm số $y=\dfrac{\sqrt{x-1}}{x+1}$ có tập xác định $\mathscr{D}=[1;+\infty)$ và $\lim\limits_{x\to+\infty} y=0$ nên nó có tiệm cận ngang $y=0$.\\
	Hàm số $y=\dfrac{x^2+1}{x}$ có tập xác định $\mathscr{D}=\mathbb{R}$ và bậc tử lớn hơn bậc mẫu nên nó không có tiệm cận ngang.\\
	Hàm số $y=\sqrt{x^2-1}$ có tập xác định $\mathscr{D}=(-\infty;-1]\cup[1;+\infty)$ và $\lim\limits_{x\to\pm\infty} y=\pm\infty$ nên nó không có tiệm cận ngang.}
\end{ex}

\begin{ex}%[2D1K4-1]%[THPT TỨ KỲ - HẢI DƯƠNG - LẦN 2 - 2018]%Câu 21.
	Tìm tất cả các tiệm cận đứng của đồ thị hàm số $y=\dfrac{2x-\sqrt{4x^2-3x+2}}{3x^2-8x+4}$. 
	\choice
	{$x=-\dfrac{2}{3}$ và $x=-2$}
	{$x=-2$}
	{\True $x=2$}
	{$x=\dfrac{2}{3}$ và $x=2$}
	\loigiai{
	Tập xác định $\mathscr{D}=\mathbb{R}\setminus\left\{\dfrac{2}{3};2\right\}$.\\
	\begin{eqnarray*}
	\lim\limits_{x\to 2}\dfrac{2x-\sqrt{4x^2-3x+2}}{3x^2-8x+4}&=&\lim\limits_{x\to 2}\dfrac{3x-2}{(x-2)(3x-2)\left(2x+\sqrt{4x^2-3x+2}\right)}\\
	&=&\lim\limits_{x\to 2}\dfrac{1}{(x-2)\left(2x+\sqrt{4x^2-3x+2}\right)}=\infty
	\end{eqnarray*}
	$\Rightarrow x=2$ là đường tiệm cận đứng của đồ thị hàm số.\\
	\begin{eqnarray*}
	\lim\limits_{x\to\dfrac{2}{3}}\dfrac{2x-\sqrt{4x^2-3x+2}}{3x^2-8x+4}=\lim\limits_{x\to\dfrac{2}{3}}\dfrac{3x-2}{(x-2)(3x-2)\left(2x+\sqrt{4x^2-3x+2}\right)}\\
	=\lim\limits_{x\to\dfrac{2}{3}}\dfrac{1}{(x-2)\left(2x+\sqrt{4x^2-3x+2}\right)}=-\dfrac{32}{9}
	\end{eqnarray*}
	$\Rightarrow x=\dfrac{2}{3}$ không là đường tiệm cận đứng của đồ thị hàm số.\\
	Vậy đồ thị hàm số có một đường tiệm cận đứng là $x=2$.}
\end{ex}

\begin{ex}%[2D1K4-1]%[THPT CHUYÊN BIÊN HÒA - HÀ NAM - 2018]%Câu 22.
	Đường tiệm cận ngang của đồ thị hàm số $y=\dfrac{\sqrt[3]{-x^3+3x^2}}{x-1}$ có phương trình
	\choice
	{$y=1$}
	{\True $y=-1$}
	{$x=-1$}
	{$y=-1$ và $y=1$}
	\loigiai{
	Ta có\\ $\lim\limits_{x\to+\infty}\dfrac{\sqrt[3]{-x^3+3x^2}}{x-1}=\lim\limits_{x\to+\infty}\dfrac{\sqrt[3]{x^3\left(-1+\dfrac{3}{x}\right)}}{x-1} =\lim\limits_{x\to+\infty}\dfrac{x\sqrt[3]{\left(-1+\dfrac{3}{x}\right)}}{x-1} =\lim\limits_{x\to+\infty}\dfrac{\sqrt[3]{\left(-1+\dfrac{3}{x}\right)}}{1-\dfrac{1}{x}}=-1$.\\
	$\lim\limits_{x\to-\infty}\dfrac{\sqrt[3]{-x^3+3x^2}}{x-1}=\lim\limits_{x\to-\infty}\dfrac{\sqrt[3]{x^3\left(-1+\dfrac{3}{x}\right)}}{x-1} =\lim\limits_{x\to-\infty}\dfrac{x\sqrt[3]{\left(-1+\dfrac{3}{x}\right)}}{x-1} =\lim\limits_{x\to-\infty}\dfrac{\sqrt[3]{\left(-1+\dfrac{3}{x}\right)}}{1-\dfrac{1}{x}}=-1$.\\
	Suy ra $y=-1$ là tiệm cận ngang của đồ thị hàm số.}
\end{ex}

\begin{ex}%[2D1K4-1]%[THPT CHUYÊN HÙNG VƯƠNG - PHÚ THỌ - LẦN 1 – 2018]%Câu 23.
	Số đường tiệm cận đứng của đồ thị hàm số $y=\dfrac{\left(x^2-3x+2\right)\sin x}{x^3-4x}$ là 
	\choice
	{\True $1$}
	{$2$}
	{$3$}
	{$4$}
	\loigiai{
	Tập xác định $\mathscr{D}=\mathbb{R}\setminus\{0;-2;2\}$.\\
	$\bullet \lim\limits_{x\to 0} y=\lim\limits_{x\to 0}\left[\left(\dfrac{x^2-3x+2}{x^2-4}\right)\left(\dfrac{\sin x}{x}\right)\right]=\dfrac{0^2-3\cdot 0+2}{0^2-4}\cdot 1=-\dfrac{1}{2}$.\\
	$\bullet \lim\limits_{x\to-2^{\pm}} y=\lim\limits_{x\to-2^{\pm}}\left[\dfrac{\left(x^2-3x+2\right)\sin x}{x\left(x^2-4\right)}\right]=\lim\limits_{x\to-2^{\pm}}\left[\dfrac{(x-1)(x-2)\sin x}{x(x-2)}\cdot\dfrac{1}{(x+2)}\right]$\\
	$=\lim\limits_{x\to-2^{\pm}}\left[\dfrac{(x-1)\sin x}{x}\cdot\dfrac{1}{(x+2)}\right]$.\\
	Vì $\lim\limits_{x\to-2^+}\left[\dfrac{(x-1)\sin x}{x}\right]=-\dfrac{3\sin 2}{2}<0$ và $\lim\limits_{x\to-2^+}\dfrac{1}{(x+2)}=+\infty$ nên $\lim\limits_{x\to-2^+} y=-\infty$.\\
	Vì $\lim\limits_{x\to-2^-}\left[\dfrac{(x-1)\sin x}{x}\right]=-\dfrac{3\sin 2}{2}<0$ và $\lim\limits_{x\to-2^-}\dfrac{1}{(x+2)}=-\infty$ nên $\lim\limits_{x\to-2^-} y=+\infty$.\\
	Vậy đường thẳng $x=-2$ là tiệm cận đứng của đồ thị hàm số.\\
	$\bullet \lim\limits_{x\to 2} y=\lim\limits_{x\to 2}\dfrac{(x-1)\sin x}{x(x+2)}=\dfrac{\sin 2}{6}$.\\
	Vậy đồ thị hàm số có $1$ đường tiệm cận đứng.}
\end{ex}

\begin{ex}%[2D1K4-1]%[THPT CHUYÊN VĨNH PHÚC - LẦN 4 - 2018]%Câu 24.
	Tìm số đường tiệm cận của đồ thị hàm số $y=\dfrac{\sqrt{x+2}}{|x|-2}$. 
	\choice
	{$1$}
	{$0$}
	{$2$}
	{\True $3$}
	\loigiai{
	Tập xác định của hàm số là $\mathscr{D}=(-2;+\infty)\setminus\{2\}$.\\
	Ta có\\
	$\bullet \lim\limits_{x\to 2^+} y=\lim\limits_{x\to 2^+}\dfrac{\sqrt{x+2}}{|x|-2}=\lim\limits_{x\to 2^+}\dfrac{\sqrt{x+2}}{x-2}=+\infty$ nên đồ thị hàm số có tiệm cận đứng $x=2$.\\
	$\bullet \lim\limits_{x\to(-2)^+} y=\lim\limits_{x\to(-2)^+}\dfrac{\sqrt{x+2}}{|x|-2}=\lim\limits_{x\to(-2)^+}\dfrac{\sqrt{x+2}}{-x-2}=\lim\limits_{x\to(-2)^+}\dfrac{-1}{\sqrt{x+2}}=-\infty$ nên đồ thị hàm số có tiệm cận đứng $x=-2$.\\
	$\bullet \lim\limits_{x\to+\infty} y=\lim\limits_{x\to+\infty}\dfrac{\sqrt{x+2}}{|x|-2}=0$ nên đồ thị hàm số có tiệm cận ngang là $y=0$.\\
	Vậy đồ thị hàm số có tất cả $3$ đường tiệm cận.}
\end{ex}

\begin{ex}%[2D1K4-1]%[CHUYÊN BẮC NINH - LẦN 2 - 2018]%Câu 25.
	Trong bốn hàm số $y=\dfrac{x+1}{x-2}$, $y=3^x$, $y=\log_3x$, $y=\sqrt{x^2+x+1}-x$. Có mấy hàm số mà đồ thị của nó có đường tiệm cận. 
	\choice
	{\True $4$}
	{$3$}
	{$1$}
	{$2$}
	\loigiai{
	$y=\dfrac{x+1}{x-2}$ đồ thị có tiệm cận đứng $x=2$, tiệm cận ngang $y=1$.\\
	$y=3^x$ đồ thị có tiệm cận ngang $y=0$.\\
	$y=\log_3x$ đồ thị có tiệm cận đứng $x=0$.\\
	Kiểm tra hàm số thứ tư $y=\sqrt{x^2+x+1}-x$ có tập xác định là $\mathscr{D}=\mathbb{R}$.\\
	$\lim\limits_{x\to+\infty} y=\lim\limits_{x\to+\infty}\left(\sqrt{x^2+x+1}-x\right)$
	$=\lim\limits_{x\to+\infty}\dfrac{x+1}{\sqrt{x^2+x+1}+x}=\dfrac{1}{2}$.\\
	Suy ra đồ thị hàm số có có đường tiệm cận ngang bên phải $y=\dfrac{1}{2}$.\\
	Vậy cả $4$ hàm số trên đồ thị của chúng đều có đường tiệm cận.}
\end{ex}
\Closesolutionfile{ans}