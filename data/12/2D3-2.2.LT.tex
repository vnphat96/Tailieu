\Opensolutionfile{ans}[ans/ansCD2D3-2.2LT]
\begin{dang}{TÍCH PHÂN HÀM ẨN}
\end{dang}
\paragraph{Vấn đề 1. Tính tích phân theo định nghĩa.}
\begin{ex}%Câu 1%[Nguyễn Diệu Linh]%[2D3K2-4]
	Cho hàm số $f(x)$ có đạo hàm liên tục trên $[0;1],$ thỏa mãn $2f(x)+3f(1-x)=\sqrt{1-x^2}$. Giá trị của tích phân $\displaystyle\int\limits_0^1 f'(x)\mathrm{\,d}x$ bằng
	\choice
	{$0$}
	{$\dfrac{1}{2}$}
	{\True $1$}
	{$\dfrac{3}{2}$}
	\loigiai{
		Ta có $\displaystyle\int\limits_0^1 f'(x)\mathrm{\,d}x=f(x)\bigg|_0^1=f(1)-f(0)$.\\
		Từ $2f(x)+3f(1-x)=\sqrt{1-x^2}\Rightarrow  \heva{&2f(0)+3f(1)=1\\&2f(1)+3f(0)=0}\Leftrightarrow\heva{&f(0)=-\dfrac{2}{5}\\&f(1)=\dfrac{3}{5}.}$ \\
		Vậy $I=\displaystyle\int\limits_0^1 f'(x)\mathrm{\,d}x=f(1)-f(0)=\dfrac{3}{5}+\dfrac{2}{5}=1$.}
\end{ex}
\begin{ex}%Câu 2%[Nguyễn Diệu Linh]%[2D3K2-4]
	Cho hàm số $f(x)$ có đạo hàm liên tục trên $[0;1],$ thỏa mãn $f(0)=f(1)=1$. Biết rằng $\displaystyle\int\limits_0^1\mathrm{e}^x[f(x)+f'(x)]\mathrm{\,d}x=ae+b$. Tính $Q=a^{2018}+b^{2018}$. 
	\choice
	{$Q=2^{2017}+1$}
	{\True $Q=2$}
	{$Q=0$}
	{$Q=2^{2017}-1$}
	\loigiai{
		Ta có $\displaystyle\int\limits_0^1\mathrm{e}^x[f(x)+f'(x)]\mathrm{\,d}x=\displaystyle\int\limits_0^1\left[\mathrm{e}^xf(x)\right]^/\mathrm{\,d}x=\left[\mathrm{e}^xf(x)\right]\bigg|_0^1 =ef(1)-f(0)=e-1$.\\
		Suy ra $\heva{&a=1\\&b=-1}\Rightarrow Q=a^{2018}+b^{2018}=1^{2018}+(-1)^{2018}=2$.}
\end{ex}
\begin{ex}%Câu 3%[Nguyễn Diệu Linh]%[2D3K2-4]
	Cho các hàm số $y=f(x), y=g(x)$ có đạo hàm liên tục trên $[0;2]$ và thỏa mãn $\displaystyle\int\limits_0^2 f'(x)g(x)\mathrm{\,d}x=2,\displaystyle\int\limits_0^2 f(x)g'(x)\mathrm{\,d}x=3$. Tính tích phân $I=\displaystyle\int\limits_0^2[f(x)g(x)]'\mathrm{\,d}x$. 
	\choice
	{$I=-1$}
	{$I=1$}
	{\True $I=5$}
	{$I=6$}
	\loigiai{Ta có $I=\displaystyle\int\limits_0^2[f(x)g(x)]'\mathrm{\,d}x=\displaystyle\int\limits_0^2\left[f'(x)g(x)+f(x)g'(x)\right]\mathrm{\,d}x$.\\
		$=\displaystyle\int\limits_0^2 f'(x)g(x)\mathrm{\,d}x+\displaystyle\int\limits_0^2 f(x)g'(x)\mathrm{\,d}x=2+3=5$.}
\end{ex}
\begin{ex}%Câu 4%[Nguyễn Diệu Linh]%[2D3K2-4]
	Cho hàm số $y=f(x)$ liên tục trên $[0;+\infty)$ và thỏa mãn $\displaystyle\int\limits_0^{x^2} f(t)\mathrm{\,d}t=x\cdot\sin(\pi x)$. Tính $f\left(\dfrac{1}{4}\right)$. 
	\choice
	{$f\left(\dfrac{1}{4}\right)=-\dfrac{\pi}{2}$}
	{$f\left(\dfrac{1}{4}\right)=\dfrac{1}{2}$}
	{\True $f\left(\dfrac{1}{4}\right)=1$}
	{$f\left(\dfrac{1}{4}\right)=1+\dfrac{\pi}{2}$}
	\loigiai{
		Từ $\displaystyle\int\limits_0^{x^2} f(t)\mathrm{\,d}t=x\cdot\sin(\pi x)$, đạo hàm hai vế ta được $2xf(x^2)=\sin(\pi x)+\pi x\cos(\pi x)$.\\
		Cho $x=\dfrac{1}{2}$ ta được $2\cdot\dfrac{1}{2}\cdot f\left(\dfrac{1}{4}\right)=\sin\dfrac{\pi}{2}+\dfrac{\pi}{2}\cos\dfrac{\pi}{2}=1\Rightarrow f\left(\dfrac{1}{4}\right)=1$.}
\end{ex}
\begin{ex}%Câu 5%[Nguyễn Diệu Linh]%[2D3K2-4]
	Cho hàm số $f(x)$ liên tục trên $[a;+\infty)$ với $a>0$ và thỏa mãn $\displaystyle\int\limits_a^x\dfrac{f(t)}{t^2}\mathrm{\,d}t+6=2\sqrt{x}$ với mọi $x>a$. Tính $f(4)$. 
	\choice
	{$f(4)=2$}
	{$f(4)=4$}
	{\True $f(4)=8$}
	{$f(4)=16$}
	\loigiai{
		Từ $\displaystyle\int\limits_a^x\dfrac{f(t)}{t^2}\mathrm{\,d}t+6=2\sqrt{x}$, đạo hàm hai vế ta được $\dfrac{f(x)}{x^2}=\dfrac{1}{\sqrt{x}}$.\\
		Suy ra $f(x)=x\sqrt{x}\Rightarrow f(4)=4\sqrt{4}=8$.}
\end{ex}
\paragraph{Vấn đề 2. Kỹ thuật đổi biến.}
\begin{ex}%Câu 6%[Nguyễn Diệu Linh]%[2D3K2-4]
	Cho $\displaystyle\int\limits_0^{2017} f(x)\mathrm{\,d}x=2$. Tính tích phân $I=\displaystyle\int\limits_0^{\sqrt{\mathrm{e}^{2017}-1}}\dfrac{x}{x^2+1}\cdot f\left[\ln\left(x^2+1\right)\right]\mathrm{\,d}x$. 
	\choice
	{\True $I=1$}
	{$I=2$}
	{$I=4$}
	{$I=5$}
	\loigiai{
		Đặt $t=\ln\left(x^2+1\right),$ suy ra $\mathrm{\,d}t=\dfrac{2x\mathrm{\,d}x}{x^2+1}\Rightarrow \dfrac{x\mathrm{\,d}x}{x^2+1}=\dfrac{\mathrm{\,d}t}{2}$.\\
		Đổi cận $\heva{&x=0\Rightarrow t=0\\&x=\sqrt{\mathrm{e}^{2017}-1}\Rightarrow t=2017.}$ \\
		Khi đó $I=\dfrac{1}{2}\displaystyle\int\limits_0^{2017} f(t)\mathrm{\,d}t =\dfrac{1}{2}\displaystyle\int\limits_0^{2017} f(x)\mathrm{\,d}x=\dfrac{1}{2}\cdot 2=1$.}
\end{ex}
\begin{ex}%Câu 7%[Nguyễn Diệu Linh]%[2D3K2-4]
	Cho hàm số $f(x)$ liên tục trên $\mathbb{R}$ và $\displaystyle\int\limits_1^9\dfrac{f(\sqrt{x})}{\sqrt{x}}\mathrm{\,d}x=4,\displaystyle\int\limits_0^{\tfrac{\pi}{2}} f(\sin x)\cos x\mathrm{\,d}x=2$. Tính tích phân $I=\displaystyle\int\limits_0^3 f(x)\mathrm{\,d}x$. 
	\choice
	{$I=2$}
	{$I=6$}
	{\True $I=4$}
	{$I=10$}
	\loigiai{
		\begin{itemize}
			\item Xét $\displaystyle\int\limits_1^9\dfrac{f(\sqrt{x})}{\sqrt{x}}\mathrm{\,d}x=4$. Đặt $t=\sqrt{x}\Rightarrow t^2=x\Rightarrow 2t\mathrm{\,d}t=\mathrm{\,d}x$.\\
			Đổi cận $\heva{&x=1\to t=1\\&x=9\to t=3}$. Suy ra $4=\displaystyle\int\limits_1^9\dfrac{f(\sqrt{x})}{\sqrt{x}}\mathrm{\,d}x=2\displaystyle\int\limits_1^3 f(t)\mathrm{\,d}t\Rightarrow \displaystyle\int\limits_1^3 f(t)\mathrm{\,d}t=2$.
			\item Xét $\displaystyle\int\limits_0^{\tfrac{\pi}{2}} f(\sin x)\cos x\mathrm{\,d}x=2$. Đặt $u=\sin x,$ suy ra $\mathrm{\,d}u=\cos x\mathrm{\,d}x$.\\
			Đổi cận $\heva{&x=0\to u=0\\&x=\dfrac{\pi}{2}\to u=1}$. Suy ra $2=\displaystyle\int\limits_0^{\tfrac{\pi}{2}} f(\sin x)\cos x\mathrm{\,d}x=\displaystyle\int\limits_0^1 f(t)\mathrm{\,d}t$.
		\end{itemize}
		Vậy $I=\displaystyle\int\limits_0^3 f(x)\mathrm{\,d}x=\displaystyle\int\limits_0^1 f(x)\mathrm{\,d}x+\displaystyle\int\limits_1^3 f(x)\mathrm{\,d}x=4$.}
\end{ex}
\begin{ex}%Câu 8%[Nguyễn Diệu Linh]%[2D3G2-4]
	Cho hàm số $f(x)$ liên tục trên $\mathbb{R}$ và $\displaystyle\int\limits_0^{\tfrac{\pi}{4}} f(\tan x)\mathrm{\,d}x=4,\displaystyle\int\limits_0^1\dfrac{x^2f(x)}{x^2+1}\mathrm{\,d}x=2$. Tính tích phân $I=\displaystyle\int\limits_0^1 f(x)\mathrm{\,d}x$. 
	\choice
	{\True $I=6$}
	{$I=2$}
	{$I=3$}
	{$I=1$}
	\loigiai{
		Xét $\displaystyle\int\limits_0^{\tfrac{\pi}{4}} f(\tan x)\mathrm{\,d}x=4$.\\
		Đặt $t=\tan x,$ suy ra $\mathrm{\,d}t=\dfrac{1}{\cos^2x}\mathrm{\,d}x=\left(\tan^2x+1\right)\mathrm{\,d}x\Rightarrow \mathrm{\,d}x=\dfrac{\mathrm{\,d}t}{1+t^2}$.\\
		Đổi cận $\heva{&x=0\to t=0\\&x=\dfrac{\pi}{4}\to t=1}$. Khi đó $4=\displaystyle\int\limits_0^{\tfrac{\pi}{4}} f(\tan x)\mathrm{\,d}x=\displaystyle\int\limits_0^1\dfrac{f(t)}{t^2+1}d t=\displaystyle\int\limits_0^1\dfrac{f(x)}{x^2+1}\mathrm{\,d}x$.\\
		Từ đó suy ra $I=\displaystyle\int\limits_0^1 f(x)\mathrm{\,d}x=\displaystyle\int\limits_0^1\dfrac{f(x)}{x^2+1}\mathrm{\,d}x+\displaystyle\int\limits_0^1\dfrac{x^2f(x)}{x^2+1}\mathrm{\,d}x=4+2=6$.}
\end{ex}
\begin{ex}%Câu 9%[Nguyễn Diệu Linh]%[2D3G2-4]
	Cho hàm số $f(x)$ liên tục trên $\mathbb{R}$ và thỏa mãn $\displaystyle\int\limits_0^{\tfrac{\pi}{4}}\tan x\cdot f\left(\cos^2x\right)\mathrm{\,d}x=1,\displaystyle\int\limits_\mathrm{e}^{\mathrm{e}^2}\dfrac{f\left(\ln^2x\right)}{x\ln x}\mathrm{\,d}x=1$. Tính tích phân $I=\displaystyle\int\limits_{\tfrac{1}{4}}^2\dfrac{f(2x)}{x}\mathrm{\,d}x$. 
	\choice
	{$I=1$}
	{$I=2$}
	{$I=3$}
	{\True $I=4$}
	\loigiai{
		\begin{itemize}
			\item Xét $A=\displaystyle\int\limits_0^{\tfrac{\pi}{4}}\tan x\cdot f\left(\cos^2x\right)\mathrm{\,d}x=1$. Đặt $t=\cos^2x$.\\
			Suy ra $\mathrm{\,d}t=-2\sin x\cos x\mathrm{\,d}x=-2\cos^2x\tan x\mathrm{\,d}x=-2t\cdot\tan x\mathrm{\,d}x\Rightarrow \tan x\mathrm{\,d}x=-\dfrac{\mathrm{\,d}t}{2t}$.\\
			Đổi cận $\heva{&x=0\Rightarrow t=1\\&x=\dfrac{\pi}{4}\Rightarrow t=\dfrac{1}{2}.}$ \\
			Khi đó $1=A=-\dfrac{1}{2}\displaystyle\int\limits_1^{\tfrac{1}{2}}\dfrac{f(t)}{t}\mathrm{\,d}t=\dfrac{1}{2}\displaystyle\int\limits_{\tfrac{1}{2}}^1\dfrac{f(t)}{t}\mathrm{\,d}t=\dfrac{1}{2}\displaystyle\int\limits_{\tfrac{1}{2}}^1\dfrac{f(x)}{x}\mathrm{\,d}x\Rightarrow \displaystyle\int\limits_{\tfrac{1}{2}}^1\dfrac{f(x)}{x}\mathrm{\,d}x=2$.
			\item Xét $B=\displaystyle\int\limits_\mathrm{e}^{\mathrm{e}^2}\dfrac{f\left(\ln^2x\right)}{x\ln x}\mathrm{\,d}x=1$. Đặt $u=\ln^2x$.\\
			Suy ra $\mathrm{\,d}u=\dfrac{2\ln x}{x}\mathrm{\,d}x=\dfrac{2\ln^2x}{x\ln x}\mathrm{\,d}x=\dfrac{2u}{x\ln x}\mathrm{\,d}x\Rightarrow \dfrac{\mathrm{\,d}x}{x\ln x}=\dfrac{\mathrm{\,d}u}{2u}$.\\
			Đổi cận $\heva{&x=e\Rightarrow u=1\\&x=\mathrm{e}^2\Rightarrow u=4.}$ \\
			Khi đó $1=B=\dfrac{1}{2}\displaystyle\int\limits_1^4\dfrac{f(u)}{u}\mathrm{\,d}u=\dfrac{1}{2}\displaystyle\int\limits_1^4\dfrac{f(x)}{x}\mathrm{\,d}x\Rightarrow \displaystyle\int\limits_1^4\dfrac{f(x)}{x}\mathrm{\,d}x=2$.\\
			\item Xét tích phân cần tính $I=\displaystyle\int\limits_{\tfrac{1}{2}}^2\dfrac{f(2x)}{x}\mathrm{\,d}x$.\\
			Đặt $v=2x,$ suy ra $\heva{&\mathrm{\,d}x=\dfrac{1}{2}\mathrm{\,d}v\\&x=\dfrac{v}{2}}$. Đổi cận $\heva{&x=\dfrac{1}{4}\Rightarrow v=\dfrac{1}{2}\\&x=2\Rightarrow v=4.}$ \\
			Khi đó $I=\displaystyle\int\limits_{\tfrac{1}{2}}^4\dfrac{f(v)}{v}\mathrm{\,d}v=\displaystyle\int\limits_{\tfrac{1}{2}}^4\dfrac{f(x)}{x}\mathrm{\,d}x=\displaystyle\int\limits_{\tfrac{1}{2}}^1\dfrac{f(x)}{x}\mathrm{\,d}x+\displaystyle\int\limits_1^4\dfrac{f(x)}{x}\mathrm{\,d}x=2+2=4$.
		\end{itemize}
	}
\end{ex}
\begin{ex}%Câu 10%[Nguyễn Diệu Linh]%[2D3G2-4]
	Cho hàm số $y=f(x)$ xác định và liên tục trên $\left[\dfrac{1}{2};2\right],$ thỏa $f(x)+f\left(\dfrac{1}{x}\right)=x^2+\dfrac{1}{x^2}+2$. Tính tích phân $I=\displaystyle\int\limits_{\tfrac{1}{2}}^2\dfrac{f(x)}{x^2+1}\mathrm{\,d}x$. 
	\choice
	{\True $I=\dfrac{3}{2}$}
	{$I=2$}
	{$I=\dfrac{5}{2}$}
	{$I=3$}
	\loigiai{
		Đặt $x=\dfrac{1}{t},$ suy ra $\mathrm{\,d}x=-\dfrac{1}{t^2}\mathrm{\,d}t$. Đổi cận $\heva{&x=\dfrac{1}{2}\Rightarrow t=2\\&x=2\Rightarrow t=\dfrac{1}{2}.}$ \\
		Khi đó $I=\displaystyle\int\limits_2^{\tfrac{1}{2}}\dfrac{f\left(\dfrac{1}{t}\right)}{\dfrac{1}{t^2}+1}\cdot\left(-\dfrac{1}{t^2}\right)\mathrm{\,d}t=\displaystyle\int\limits_{\tfrac{1}{2}}^2\dfrac{f\left(\dfrac{1}{t}\right)}{t^2+1}\mathrm{\,d}t=\displaystyle\int\limits_{\tfrac{1}{2}}^2\dfrac{f\left(\dfrac{1}{x}\right)}{x^2+1}\mathrm{\,d}x$.\\
		Suy ra
		\begin{eqnarray*}
	    2I&=&\displaystyle\int\limits_{\tfrac{1}{2}}^2\dfrac{f(x)}{x^2+1}\mathrm{\,d}x+\displaystyle\int\limits_{\tfrac{1}{2}}^2\dfrac{f\left(\dfrac{1}{x}\right)}{x^2+1}\mathrm{\,d}x\\&=&\displaystyle\int\limits_{\tfrac{1}{2}}^2\dfrac{f(x)+f\left(\dfrac{1}{x}\right)}{x^2+1}\mathrm{\,d}x=\displaystyle\int\limits_{\tfrac{1}{2}}^2\dfrac{x^2+\dfrac{1}{x^2}+2}{x^2+1}\mathrm{\,d}x\\
		&=&\displaystyle\int\limits_{\tfrac{1}{2}}^2\dfrac{x^2+1}{x^2}\mathrm{\,d}x=\displaystyle\int\limits_{\tfrac{1}{2}}^2\left(1+\dfrac{1}{x^2}\right)\mathrm{\,d}x=\left(x-\dfrac{1}{x}\right)\bigg|_{\tfrac{1}{2}}^2 =3\Rightarrow I=\dfrac{3}{2}.
	    \end{eqnarray*}
	}
\end{ex}
\begin{ex}%Câu 11%[Nguyễn Diệu Linh]%[2D3G2-4]
	Cho hàm số $f(x)$ liên tục trên $\mathbb{R}$ và thỏa mãn $f(x)+f(-x)=\sqrt{2+2\cos 2x}$ với mọi $x\in\mathbb{R}$.\\
	Tính $I=\displaystyle\int\limits_{-\tfrac{3\pi}{2}}^{\tfrac{3\pi}{2}} f(x) \mathrm{\,d} x$. 
	\choice
	{$I=-6$}
	{$I=0$}
	{$I=-2$}
	{\True $I=6$}
	\loigiai{
		Đặt $t=-x\Rightarrow \mathrm{\,d}x=-\mathrm{\,d}t$. Đổi cận $\heva{&x=-\dfrac{3\pi}{2}\to t=\dfrac{3\pi}{2}\\&x=\dfrac{3\pi}{2}\to t=-\dfrac{3\pi}{2}.}$ \\
		Khi đó $I=-\displaystyle\int\limits_{\tfrac{3\pi}{2}}^{-\tfrac{3\pi}{2}} f(-t)\mathrm{\,d}t=\displaystyle\int\limits_{-\tfrac{3\pi}{2}}^{\tfrac{3\pi}{2}} f(-t)\mathrm{\,d}t=\displaystyle\int\limits_{-\tfrac{3\pi}{2}}^{\tfrac{3\pi}{2}} f(-x)\mathrm{\,d}x$.\\
		Suy ra $2I=\displaystyle\int\limits_{-\tfrac{3\pi}{2}}^{\tfrac{3\pi}{2}}[f(t)+f(-t)]\mathrm{\,d}t=\displaystyle\int\limits_{-\tfrac{3\pi}{2}}^{\tfrac{3\pi}{2}}\sqrt{2+2\cos 2t}\mathrm{\,d}t=\displaystyle\int\limits_{-\tfrac{3\pi}{2}}^{\tfrac{3\pi}{2}} 2|\cos t|\mathrm{\,d}t=12\Rightarrow I=6$.}
\end{ex}
\begin{ex}%Câu 12%[Nguyễn Diệu Linh]%[2D3K2-4]
	Cho hàm số $y=f(x)$ xác định và liên tục trên $\mathbb{R},$ thỏa mãn $f\left(x^5+4x+3\right)=2x+1$ với mọi $x\in\mathbb{R}$. Tích phân $\displaystyle\int\limits_{-2}^8 f(x)\mathrm{\,d}x$ bằng
	\choice
	{$2$}
	{\True $10$}
	{$\dfrac{32}{3}$}
	{$72$}
	\loigiai{
		Đặt $x=t^5+4t+3,$ suy ra $\mathrm{\,d}x=\left(5t^4+4\right)\mathrm{\,d}t$. Đổi cận $\heva{&x=-2\to t=-1\\&x=8\to t=1.}$ \\
		Khi đó $\displaystyle\int\limits_{-2}^8 f(x)\mathrm{\,d}x=\displaystyle\int\limits_{-1}^1 f\left(t^5+4t+3\right)\left(5t^4+4\right)\mathrm{\,d}t=\displaystyle\int\limits_{-1}^1(2t+1)\left(5t^4+4\right)\mathrm{\,d}t=10$.}
\end{ex}
\begin{ex}%Câu 13%[Nguyễn Diệu Linh]%[2D3G2-4]
	Cho các hàm số $f(x), g(x)$ liên tục trên $[0;1],$ thỏa $m\cdot f(x)+n\cdot f(1-x)=g(x)$ với $m, n$ là số thực khác $0$ và $\displaystyle\int\limits_0^1 f(x)\mathrm{\,d}x=\displaystyle\int\limits_0^1 g(x)\mathrm{\,d}x=1$. Tính $m+n$. 
	\choice
	{$m+n=0$}
	{$m+n=\dfrac{1}{2}$}
	{\True $m+n=1$}
	{$m+n=2$}
	\loigiai{Từ giả thiết $m\cdot f(x)+n\cdot f(1-x)=g(x)$, lấy tích phân hai vế ta được.
		$$\displaystyle\int\limits_0^1\left[m\cdot f(x)+n\cdot f(1-x)\right]\mathrm{\,d}x=\displaystyle\int\limits_0^1 g(x)\mathrm{\,d}x\Rightarrow m+n\displaystyle\int\limits_0^1 f(1-x)\mathrm{\,d}x=1. \quad (1)$$
		(do $\displaystyle\int\limits_0^1 f(x)\mathrm{\,d}x=\displaystyle\int\limits_0^1 g(x)\mathrm{\,d}x=1$).\\
		Xét tích phân $\displaystyle\int\limits_0^1 f(1-x)\mathrm{\,d}x$. Đặt $t=1-x$, suy ra $\mathrm{\,d}t=-\mathrm{\,d}x$. Đổi cận $\heva{&x=0\to t=1\\&x=1\to t=0.}$ \\
		Khi đó $\displaystyle\int\limits_0^1 f(1-x)\mathrm{\,d}x=-\displaystyle\int\limits_1^0 f(t)\mathrm{\,d}t=\displaystyle\int\limits_0^1 f(t)\mathrm{\,d}t=\displaystyle\int\limits_0^1 f(x)\mathrm{\,d}x=1$. \quad $(2)$\\
		Từ $(1)$ và $(2),$ suy ra $m+n=1$.
	}
\end{ex}
\begin{ex}%Câu 14%[Nguyễn Diệu Linh]%[2D3G2-4]
	Cho hàm số $f(x)$ xác định và liên tục trên $[0;1],$ thỏa mãn $f'(x)=f'(1-x)$ với mọi $x\in[0;1]$. Biết rằng $f(0)=1, f(1)=41$. Tính tích phân $I=\displaystyle\int\limits_0^1 f(x)\mathrm{\,d}x$. 
	\choice
	{$I=\sqrt{41}$}
	{\True $I=21$}
	{$I=41$}
	{$I=42$}
	\loigiai{
		Ta có $f'(x)=f'(1-x)\Rightarrow f(x)=-f(1-x)+C$.\\
		Suy ra $f(0)=-f(1)+C\Rightarrow C=42$.\\
		Suy ra 
		\begin{eqnarray*}
		f(x)=-f(1-x)+42&\Rightarrow &f(x)+f(1-x)=42.\\
		&\Rightarrow& \displaystyle\int\limits_0^1\left[f(x)+f(1-x)\right]\mathrm{\,d}x=\displaystyle\int\limits_0^1 42\mathrm{\,d}x=42. \quad (1)
		\end{eqnarray*}
		Vì $f'(x)=f'(1-x)\Rightarrow \displaystyle\int\limits_0^1 f(x)\mathrm{\,d}x=\displaystyle\int\limits_0^1 f(1-x)\mathrm{\,d}x$. \quad $(2)$\\
		Từ $(1)$ và $(2),$ suy ra $\displaystyle\int\limits_0^1 f(x)\mathrm{\,d}x=\displaystyle\int\limits_0^1 f(1-x)\mathrm{\,d}x=21$.}
\end{ex}
\begin{ex}%Câu 15%[Nguyễn Diệu Linh]%[2D3G2-4]
	Cho hàm số $y=f(x)$ liên tục trên $\mathbb{R}$ và thỏa mãn $f^3(x)+f(x)=x$ với mọi $x\in\mathbb{R}$. Tính $I=\displaystyle\int\limits_0^2 f(x)\mathrm{\,d}x$. 
	\choice
	{$I=-\dfrac{4}{5}$}
	{$I=\dfrac{4}{5}$}
	{$I=-\dfrac{5}{4}$}
	{\True $I=\dfrac{5}{4}$}
	\loigiai{
		Đặt $u=f(x)$, ta thu được $u^3+u=x$. Suy ra $\left(3u^2+1\right)\mathrm{\,d}u=\mathrm{\,d}x$.\\
		Từ $u^3+u=x$, ta đổi cận $\heva{&x=0\to u=0\\&x=2\to u=1}$. Khi đó $I=\displaystyle\int\limits_0^1 u\left(3u^2+1\right)\mathrm{\,d}u=\dfrac{5}{4}$.\\
		\textbf{Cách khác:} Nếu bài toán cho $f(x)$ có đạo hàm liên tục thì ta làm như sau:\\
		Từ giả thiết $f^3(x)+f(x)=x\Rightarrow \heva{&f^3(0)+f(0)=0\\&f^3(2)+f(2)=2}\Rightarrow\heva{&f(0)=0\\&f(2)=1}$. \quad $(*)$\\
		Cũng từ giả thiết $f^3(x)+f(x)=x$, ta có $f'(x)\cdot f^3(x)+f'(x)\cdot f(x)=x\cdot f'(x)$.\\
		Lấy tích phân hai vế
		\begin{eqnarray*}
        &&\displaystyle\int\limits_0^2\left[f'(x)\cdot f^3(x)+f'(x)\cdot f(x)\right]\mathrm{\,d}x=\displaystyle\int\limits_0^2 x\cdot f'(x)\mathrm{\,d}x.\\
		&\Rightarrow& \left(\dfrac{[f(x)]^4}{4}+\dfrac{[f(x)]^2}{2}\right)\bigg|_0^2 =xf(x)\bigg|_0^2 -\displaystyle\int\limits_0^2 f(x)\mathrm{\,d}x\\&\Rightarrow&  \displaystyle\int\limits_0^2 f(x)\mathrm{\,d}x=\dfrac{5}{4}.
	    \end{eqnarray*}}
\end{ex}
\paragraph{Vấn đề 3. Kỹ thuật tích phân từng phần.}
\begin{ex}%Câu 16%[Nguyễn Diệu Linh]%[2D3K2-4]
	Cho hàm số $f(x)$ thỏa mãn $\displaystyle\int\limits_0^3 x\cdot f'(x)\cdot\mathrm{e}^{f(x)}\mathrm{\,d}x=8$ và $f(3)=\ln 3$. Tính $I=\displaystyle\int\limits_0^3\mathrm{e}^{f(x)}\mathrm{\,d}x$. 
	\choice
	{\True $I=1$}
	{$I=11$}
	{$I=8-\ln 3$}
	{$I=8+\ln 3$}
	\loigiai{
		Đặt $\heva{&u=x\\&\mathrm{\,d}v=f'(x)\cdot\mathrm{e}^{f(x)}\mathrm{\,d}x}\Rightarrow\heva{&\mathrm{\,d}u=\mathrm{\,d}x\\&v=\mathrm{e}^{f(x)}}$.\\
		Khi đó $\displaystyle\int\limits_0^3 x\cdot f'(x)\cdot\mathrm{e}^{f(x)}\mathrm{\,d}x=x\cdot\mathrm{e}^{f(x)}\bigg|_0^3 -\displaystyle\int\limits_0^3\mathrm{e}^{f(x)}\mathrm{\,d}x$.\\
		Suy ra $8=3\cdot\mathrm{e}^{f(3)}-\displaystyle\int\limits_0^3\mathrm{e}^{f(x)}\mathrm{\,d}x\Rightarrow \displaystyle\int\limits_0^3\mathrm{e}^{f(x)}\mathrm{\,d}x=9-8=1$.}
\end{ex}
\begin{ex}%Câu 17%[Nguyễn Diệu Linh]%[2D3K2-4]
	Cho hàm số $f(x)$ có đạo hàm liên tục trên $\left[0;\dfrac{\pi}{2}\right],$ thỏa mãn $\displaystyle\int\limits_0^{\tfrac{\pi}{2}} f'(x)\cos^2x\mathrm{\,d}x=10$ và $f(0)=3$. Tích phân $\displaystyle\int\limits_0^{\tfrac{\pi}{2}} f(x)\sin 2x\mathrm{\,d}x$ bằng
	\choice
	{$I=-13$}
	{$I=-7$}
	{$I=7$}
	{\True $I=13$}
	\loigiai{
		Xét $\displaystyle\int\limits_0^{\tfrac{\pi}{2}} f'(x)\cos^2x\mathrm{\,d}x=10$, đặt $\heva{&u=\cos^2x\\&\mathrm{\,d}v=f'(x)\mathrm{\,d}x}\Rightarrow\heva{&\mathrm{\,d}u=-\sin 2x\mathrm{\,d}x\\&v=f(x).}$ \\
		Khi đó
		\begin{eqnarray*}
	    &&10=\displaystyle\int\limits_0^{\tfrac{\pi}{2}} f'(x)\cos^2x\mathrm{\,d}x=\cos^2xf(x)\bigg|_0^{\tfrac{\pi}{2}} +\displaystyle\int\limits_0^{\tfrac{\pi}{2}} f(x)\sin 2x\mathrm{\,d}x \\
		&\Leftrightarrow& 10=-f(0)+\displaystyle\int\limits_0^{\tfrac{\pi}{2}} f(x)\sin 2x\mathrm{\,d}x\\&\Rightarrow& \displaystyle\int\limits_0^{\tfrac{\pi}{2}} f(x)\sin 2x\mathrm{\,d}x=10+f(0)=13 .
	    \end{eqnarray*}}
\end{ex}
\begin{ex}%Câu 18%[Nguyễn Diệu Linh]%[2D3G2-4]
	Cho hàm số $y=f(x)$ có đạo hàm liên tục trên $[0;1],$ thỏa mãn $\displaystyle\int\limits_1^2 f(x-1)\mathrm{\,d}x=3$ và $f(1)=4$. Tích phân $\displaystyle\int\limits_0^1 x^3f'(x^2)\mathrm{\,d}x$ bằng
	\choice
	{$-1$}
	{$-\dfrac{1}{2}$}
	{\True $\dfrac{1}{2}$}
	{$1$}
	\loigiai{
		Ta có $\displaystyle\int\limits_1^2 f(x-1)\mathrm{\,d}x=3$. Đặt $t=x-1\Rightarrow  \displaystyle\int\limits_0^1 f(t)\mathrm{\,d}t=3$ hay $\displaystyle\int\limits_0^1 f(x)\mathrm{\,d}x=3$.\\
		Xét $\displaystyle\int\limits_0^1 x^3f'(x^2)\mathrm{\,d}x$. Đặt $t=x^2\Rightarrow  \dfrac{1}{2}\displaystyle\int\limits_0^1 tf'(t)\mathrm{\,d}t=\dfrac{1}{2}\displaystyle\int\limits_0^1 xf'(x)\mathrm{\,d}x$. \\
		Đặt $\heva{&u=x\\&\mathrm{\,d}v=f'(x)\mathrm{\,d}x}\Rightarrow\heva{&\mathrm{\,d}u=\mathrm{\,d}x\\&v=f(x).}$ \\
		Khi đó $\displaystyle\int\limits_0^1 x^3f'(x^2)\mathrm{\,d}x=\dfrac{1}{2}\displaystyle\int\limits_0^1 tf'(t)\mathrm{\,d}t=\dfrac{1}{2}\left[xf(x)\bigg|_0^1-\displaystyle\int\limits_0^1 f(x)\mathrm{\,d}x\right]=\dfrac{1}{2}[4-3]=\dfrac{1}{2}$.}
\end{ex}
\begin{ex}%Câu 19%[Nguyễn Diệu Linh]%[2D3G2-4]
	Cho hàm số $f(x)$ nhận giá trị dương, có đạo hàm liên tục trên $[0;2]$. Biết $f(0)=1$ và $f(x)f(2-x)=\mathrm{e}^{2x^2-4x}$ với mọi $x\in[0;2]$. Tính tích phân $I=\displaystyle\int\limits_0^2\dfrac{\left(x^3-3x^2\right)f'(x)}{f(x)}\mathrm{\,d}x$. 
	\choice
	{$I=-\dfrac{14}{3}$}
	{$I=-\dfrac{32}{5}$}
	{$I=-\dfrac{16}{3}$}
	{\True $I=-\dfrac{16}{5}$}
	\loigiai{
		Từ giả thiết $f(x)f(2-x)=\mathrm{e}^{2x^2-4x}\Rightarrow f(2)=1$.\\
		Ta có $I=\displaystyle\int\limits_0^2\dfrac{\left(x^3-3x^2\right)f'(x)}{f(x)}\mathrm{\,d}x$. Đặt $\heva{&u=x^3-3x^2\\&\mathrm{\,d}v=\dfrac{f'(x)}{f(x)}\mathrm{\,d}x}\Rightarrow\heva{&\mathrm{\,d}u=\left(3x^2-6x\right)\mathrm{\,d}x\\&v=\ln\left|f(x)\right|.}$ \\
		Khi đó $I=\left(x^3-3x^2\right)\ln\left|f(x)\right|\bigg|_0^2-\displaystyle\int\limits_0^2\left(3x^2-6x\right)\ln\left|f(x)\right|\mathrm{\,d}x=-3\displaystyle\int\limits_0^2\left(x^2-2x\right)\ln\left|f(x)\right|\mathrm{\,d}x=-3J$.\\
		Ta có
		\begin{eqnarray*}
	    J&=&\displaystyle\int\limits_0^2\left(x^2-2x\right)\ln\left|f(x)\right|\mathrm{\,d}x=\displaystyle\int\limits_2^0\left[(2-t)^2-2(2-t)\right]\ln\left|f(2-t)\right|\mathrm{d}(2-t)\quad (\text{Đặt}\quad x=2-t)\\
		&=&\displaystyle\int\limits_2^0\left[(2-x)^2-2(2-x)\right]\ln\left|f(2-x)\right|\mathrm{d}(2-x)\\&=&\displaystyle\int\limits_0^2\left(x^2-2x\right)\ln\left|f(2-x)\right|\mathrm{\,d}x.
	    \end{eqnarray*}
		Suy ra
		\begin{eqnarray*}
		 2J&=&\displaystyle\int\limits_0^2\left(x^2-2x\right)\ln\left|f(x)\right|\mathrm{\,d}x+\displaystyle\int\limits_0^2\left(x^2-2x\right)\ln\left|f(2-x)\right|\mathrm{\,d}x\\&=&\displaystyle\int\limits_0^2\left(x^2-2x\right)\ln\left|f(x)f(2-x)\right|\mathrm{\,d}x
		 =\displaystyle\int\limits_0^2\left(x^2-2x\right)\ln\mathrm{e}^{2x^2-4x}\mathrm{\,d}x\\&=&\displaystyle\int\limits_0^2\left(x^2-2x\right)\left(2x^2-4x\right)\mathrm{\,d}x=\dfrac{32}{15}\Rightarrow J=\dfrac{16}{15}.
		\end{eqnarray*}
		Vậy $I=-3J=-\dfrac{16}{5}$.}
\end{ex}
\begin{ex}%Câu 20%[Nguyễn Diệu Linh]%[2D3K2-4]
	Cho biểu thức $S=\ln\left(1+\displaystyle\int\limits_{\tfrac{n}{4+m^2}}^{\tfrac{\pi}{2}}(2-\sin 2x)\mathrm{e}^{2\cot x}\mathrm{\,d}x\right),$ với số thực $m\neq 0$. Chọn khẳng định đúng trong các khẳng định sau. 
	\choice
	{$S=5$}
	{$S=9$}
	{\True $S=2\cot\left(\dfrac{\pi}{4+m^2}\right)+2\ln\left(\sin\dfrac{\pi}{4+m^2}\right)$}
	{$S=2\tan\left(\dfrac{\pi}{4+m^2}\right)+2\ln\left(\dfrac{\pi}{4+m^2}\right)$}
	\loigiai{
		Ta có $\displaystyle\int\limits_{\tfrac{\pi}{4+m^2}}^{\tfrac{\pi}{2}}(2-\sin 2x)\mathrm{e}^{2\cot x}\mathrm{\,d}x=2\displaystyle\int\limits_{\tfrac{\pi}{4+m^2}}^{\tfrac{\pi}{2}}\mathrm{e}^{2\cot x}\mathrm{\,d}x-\displaystyle\int\limits_{\tfrac{\pi}{4+m^2}}^{\tfrac{\pi}{2}}\sin 2x\mathrm{e}^{2\cot x}\mathrm{\,d}x$. \quad  $(1)$\\
		Xét
		\begin{eqnarray*}
	    \displaystyle\int\limits_{\tfrac{\pi}{4+m^2}}^{\tfrac{\pi}{2}}\sin 2x\mathrm{e}^{2\cot x}\mathrm{\,d}x&=&\displaystyle\int\limits_{\tfrac{\pi}{4+m^2}}^{\tfrac{\pi}{2}}\mathrm{e}^{2\cot x}\mathrm{d}\left(\sin^2x\right)\\&=&\sin ^2 x\cdot e^{2\cot x}\bigg|_{\tfrac{\pi}{4+m^2}}^{\tfrac{\pi}{2}}-\displaystyle\int\limits_{\tfrac{\pi}{4+m^2}}^{\tfrac{\pi}{2}}\sin^2x\left(-\dfrac{2}{\sin^2x}\right) \mathrm{e}^{2\cot x}\mathrm{\,d}x\\&=&\sin ^2 x\cdot e^{2\cot x}\bigg|_{\tfrac{\pi}{4+m^2}}{\tfrac{\pi}{2}}+2\displaystyle\int\limits_{\tfrac{\pi}{4+m^2}}^{\tfrac{\pi}{2}}\mathrm{e}^{2\cot x}\mathrm{\,d}x. \quad (2)
	    \end{eqnarray*}
		Từ $(1)$ và $(2),$ suy ra $I=\sin^2x\cdot\mathrm{e}^{2\cot x}\bigg|_{\tfrac{\pi}{4+m^2}}^{\tfrac{\pi}{2}} =-1+\sin^2\dfrac{\pi}{4+m^2}\cdot\mathrm{e}^{2\cot\tfrac{\pi}{4+m^2}}$.\\
		$\Rightarrow  S=\ln\left(\sin^2\dfrac{\pi}{4+m^2}\cdot\mathrm{e}^{2\cot\tfrac{\pi}{4+m^2}}\right) =2\cot\left(\dfrac{\pi}{4+m^2}\right)+2\ln\left(\sin\dfrac{\pi}{4+m^2}\right)$.}
\end{ex}
\paragraph{Vấn đề 4. Tính $a$, $b$, $c$ trong tích phân}
\begin{ex}%Câu 21%[Nguyễn Diệu Linh]%[2D3K2-4]
	Biết $\displaystyle\int\limits_1^2\ln\left(9-x^2\right)\mathrm{\,d}x=a\ln 5+b\ln 2+c$ với $a, b, c\in\mathbb{Z}$. Tính $P=|a|+|b|+|c|$. 
	\choice
	{\True $P=13$}
	{$P=18$}
	{$P=26$}
	{$P=34$}
	\loigiai{
		Đặt $\heva{&u=\ln\left(9-x^2\right)\\&\mathrm{\,d}v=\mathrm{\,d}x} \Rightarrow\heva{&\mathrm{\,d}u=\dfrac{-2x}{9-x^2}\mathrm{\,d}x\\&v=x+3.}$ \\
		Khi đó 
		\begin{eqnarray*}
		I&=&(x+3)\ln\left(9-x^2\right)\bigg|_1^2 + 2\displaystyle\int\limits_1^2\dfrac{x(x+3)}{9-x^2}\mathrm{\,d}x\\&=&5\ln 5-4\ln 8+ 2\displaystyle\int\limits_1^2\left(-1+\dfrac{3}{3-x}\right)\mathrm{\,d}x.\\
		&=&5\ln 5-12\ln 2-2\left(x+3\ln|3-x|\right)\bigg|_1^2 =5\ln 5-6\ln 2-2.
	    \end{eqnarray*}
        $\Rightarrow \heva{&a=5\\&b=-6\\&c=-2}\Rightarrow P=13$.\\
		\textbf{Nhận xét.} Ở đây chọn $v=x+3$ thay bởi $x$ để rút gọn cho $9-x^2$, giảm thiểu biến đổi.}
\end{ex}
\begin{ex}%Câu 22%[Nguyễn Diệu Linh]%[2D3K2-4]
	Biết $\displaystyle\int\limits_0^1\dfrac{\pi x^3+2^x+\mathrm{e}x^32^x}{\pi+e{\cdot 2}^x}\mathrm{\,d}x=\dfrac{1}{m}+\dfrac{1}{\mathrm{e}\ln n}\cdot\ln\left(p+\dfrac{\mathrm{e}}{\mathrm{e}+\pi}\right)$ với $m$, $n$, $p$ là các số nguyên dương. Tính tổng $P=m+n+p$. 
	\choice
	{$P=5$}
	{$P=6$}
	{\True $P=7$}
	{$P=8$}
	\loigiai{
		Ta có $I=\displaystyle\int\limits_0^1\dfrac{\pi x^3+2^x+\mathrm{e}x^32^x}{\pi+\mathrm{e}\cdot 2^x}\mathrm{\,d}x=\displaystyle\int\limits_0^1\left(x^3+\dfrac{2^x}{\pi+\mathrm{e}2^x}\right)\mathrm{\,d}x=\dfrac{1}{4}x^4\bigg|_0^1+A=\dfrac{1}{4}+A$.\\
		Tính $A=\displaystyle\int\limits_0^1\dfrac{2^x}{\pi+\mathrm{e}2^x}\mathrm{\,d}x$. Đặt $t=\pi+e\cdot 2^x\Rightarrow \mathrm{\,d}t=\mathrm{e}\cdot\ln 2\cdot 2^x\mathrm{\,d}x\Rightarrow 2^x\mathrm{\,d}x=\dfrac{1}{\mathrm{e}\ln 2}\mathrm{\,d}t$.\\
		Đổi cận $\heva{&x=0\to t=\pi+\mathrm{e}\\&x=1\to t=\pi+2\mathrm{e}.}$ \\
		Khi đó $A=\dfrac{1}{\mathrm{e}\cdot\ln 2}\cdot\displaystyle\int\limits_{\pi+\mathrm{e}}^{\pi+2\mathrm{e}}\dfrac{\mathrm{\,d}t}{t}=\dfrac{1}{\mathrm{e}\cdot\ln 2}\ln|t|\bigg|_{\pi+\mathrm{e}}^{\pi+2\mathrm{e}}=\dfrac{1}{\mathrm{e}\ln 2}\ln\dfrac{\pi+2\mathrm{e}}{\pi+\mathrm{e}}=\dfrac{1}{\mathrm{e}\ln 2}\ln\left(1+\dfrac{\mathrm{e}}{\mathrm{e}+\pi}\right)$.\\
		Vậy $I=\dfrac{1}{4}+\dfrac{1}{\mathrm{e}\ln 2}\ln\left(1+\dfrac{\mathrm{e}}{\mathrm{e}+\pi}\right)\Rightarrow \heva{&m=4\\&n=2\\&p=1}\Rightarrow P=m+n+p=7$.}
\end{ex}
\begin{ex}%Câu 23%[Nguyễn Diệu Linh]%[2D3K2-4]
	Biết $\displaystyle\int\limits_0^{\tfrac{\pi}{2}}\dfrac{x^2+(2x+\cos x)\cos x+1-\sin x}{x+\cos x}\mathrm{\,d}x=a\pi^2+b-\ln\dfrac{c}{\pi}$ với $a, b, c$ là các số hữu tỉ. Tính $P=ac^3+b$. 
	\choice
	{$P=\dfrac{5}{4}$}
	{$P=\dfrac{3}{2}$}
	{\True $P=2$}
	{$P=3$}
	\loigiai{Ta có
		\begin{eqnarray*}
	    I&=&\displaystyle\int\limits_0^{\tfrac{\pi}{2}}\dfrac{\left(x^2+2x\cos x+\cos^2x\right)+(1-\sin x)}{x+\cos x}\mathrm{\,d}x\\
		&=&\displaystyle\int\limits_0^{\tfrac{\pi}{2}}\dfrac{(x+\cos x)^2}{x+\cos x}\mathrm{\,d}x+\displaystyle\int\limits_0^{\tfrac{\pi}{2}}\dfrac{1-\sin x}{x+\cos x}\mathrm{\,d}x\\&=&\displaystyle\int\limits_0^{\tfrac{\pi}{2}}(x+\cos x)\mathrm{\,d}x+\displaystyle\int\limits_0^{\tfrac{\pi}{2}}\dfrac{\mathrm{d}(x+\cos x)}{x+\cos x}.\\
		&=&\left(\dfrac{1}{2}x^2+\sin x+\ln|x+\cos x|\right)\bigg|_0^{\tfrac{\pi}{2}}=\dfrac{1}{8}\pi^2+1+\ln\dfrac{\pi}{2}=\dfrac{1}{8}\pi^2+1-\ln\dfrac{2}{\pi}\\&\Rightarrow& \heva{&a=\dfrac{1}{8}\\&b=1\\&c=2}\Rightarrow P=ac^3+b=2.
	    \end{eqnarray*}
		}
\end{ex}
\begin{ex}%Câu 24%[Nguyễn Diệu Linh]%[2D3K2-4]
	Biết $\displaystyle\int\limits_{\ln\sqrt{3}}^{\ln\sqrt{8}}\dfrac{1}{\sqrt{\mathrm{e}^{2x}+1}-\mathrm{e}^x}\mathrm{\,d}x=1+\dfrac{1}{2}\ln\dfrac{b}{a}+a\sqrt{a}-\sqrt{b}$ với $a, b\in\mathbb{Z}^+$. Tính $P=a+b$. 
	\choice
	{$P=-1$}
	{$P=1$}
	{$P=3$}
	{\True $P=5$}
	\loigiai{
		Ta có $I=\displaystyle\int\limits_{\ln\sqrt{3}}^{\ln\sqrt{8}}\dfrac{1}{\sqrt{\mathrm{e}^{2x}+1}-\mathrm{e}^x}\mathrm{\,d}x=\displaystyle\int\limits_{\ln\sqrt{3}}^{\ln\sqrt{8}}\left(\sqrt{\mathrm{e}^{2x}+1}+\mathrm{e}^x\right)\mathrm{\,d}x=\displaystyle\int\limits_{\ln\sqrt{3}}^{\ln\sqrt{8}}\sqrt{\mathrm{e}^{2x}+1}\mathrm{\,d}x+\displaystyle\int\limits_{\ln\sqrt{3}}^{\ln\sqrt{8}}\mathrm{e}^x\mathrm{\,d}x$.\\
		\begin{itemize}
			\item $\displaystyle\int\limits_{\ln\sqrt{3}}^{\ln\sqrt{8}}\mathrm{e}^x\mathrm{\,d}x=\mathrm{e}^x\bigg|_{\ln\sqrt{3^{\ln\sqrt{8}}}}=2\sqrt{2}-\sqrt{3}$.
			\item $\displaystyle\int\limits_{\ln\sqrt{3}}^{\ln\sqrt{8}}\sqrt{\mathrm{e}^{2x}+1}\mathrm{\,d}x$. Đặt $t=\sqrt{\mathrm{e}^{2x}+1}\Leftrightarrow t^2=\mathrm{e}^{2x}+1$, suy ra $2t\mathrm{\,d}t=2\mathrm{e}^{2x}\mathrm{\,d}x\Leftrightarrow\mathrm{\,d}x=\dfrac{t\mathrm{\,d}t}{\mathrm{e}^{2x}}=\dfrac{t\mathrm{\,d}t}{t^2-1}$.
			Đổi cận $\heva{&x=\ln\sqrt{3}\to t=2\\&x=\ln\sqrt{8}\to t=3.}$ \\
			Khi đó $\displaystyle\int\limits_{\ln\sqrt{3}}^{\ln\sqrt{8}}\sqrt{\mathrm{e}^{2x}+1}\mathrm{\,d}x=\displaystyle\int\limits_2^3\dfrac{t^2\mathrm{\,d}t}{t^2-1}\mathrm{\,d}t=\displaystyle\int\limits_2^3\left(1+\dfrac{1}{t^2-1}\right)\mathrm{\,d}t=\left(t+\dfrac{1}{2}\ln\left|\dfrac{t-1}{t+1}\right|\right)\bigg|_2^3=1+\dfrac{1}{2}\ln\dfrac{3}{2}$.
		\end{itemize}
		Vậy $I=1+\dfrac{1}{2}\ln\dfrac{3}{2}+2\sqrt{2}-\sqrt{3}\Rightarrow \heva{&a=2\\&b=3}\Rightarrow P=a+b=5$.}
\end{ex}
\begin{ex}%Câu 25%[Nguyễn Diệu Linh]%[2D3G2-4]
	Biết $\displaystyle\int\limits_1^2\dfrac{\mathrm{\,d}x}{(x+1)\sqrt{x}+x\sqrt{x+1}}=\sqrt{a}-\sqrt{b}-c$ với $a, b, c\in\mathbb{Z}^+$. Tính $P=a+b+c$. 
	\choice
	{$P=12$}
	{$P=18$}
	{$P=24$}
	{\True $P=46$}
	\loigiai{
		Ta có $I=\displaystyle\int\limits_1^2\dfrac{\mathrm{\,d}x}{\sqrt{x(x+1)}\left(\sqrt{x+1}+\sqrt{x}\right)}=\displaystyle\int\limits_1^2\dfrac{\sqrt{x+1}+\sqrt{x}}{\sqrt{x(x+1)}\left(\sqrt{x+1}+\sqrt{x}\right)^2}\mathrm{\,d}x$.\\
		Đặt $u=\sqrt{x+1}+\sqrt{x}$, suy ra $\mathrm{\,d}u=\left(\dfrac{1}{2\sqrt{x+1}}+\dfrac{1}{2\sqrt{x}}\right)\mathrm{\,d}x\Rightarrow 2\mathrm{\,d}u=\dfrac{\sqrt{x}+\sqrt{x+1}}{\sqrt{x(x+1)}}\mathrm{\,d}x$.\\
		Đổi cận $\heva{&x=2\to u=\sqrt{3}+\sqrt{2}\\&x=1\to u=\sqrt{2}+1}$. Khi đó
		\begin{eqnarray*}
		I&=&2\displaystyle\int\limits_{\sqrt{2}+1}^{\sqrt{3}+\sqrt{2}}\dfrac{\mathrm{\,d}u}{u^2}= -\dfrac{2}{u}\bigg|_{\sqrt{2}+1}^{\sqrt{3}+\sqrt{2}}\\&=&-2\left(\dfrac{1}{\sqrt{3}+\sqrt{2}}-\dfrac{1}{\sqrt{2}+1}\right)=-2\left(\dfrac{\sqrt{3}-\sqrt{2}}{3-2}-\dfrac{\sqrt{2}-1}{2-1}\right)\\&=&\sqrt{32}-\sqrt{12}-2\Rightarrow \heva{&a=32\\&b=12\\&c=2}\Rightarrow  P=46.
	    \end{eqnarray*}}
\end{ex}
\begin{ex}%Câu 26%[Nguyễn Diệu Linh]%[2D3G2-4]
	Biết $\displaystyle\int\limits_0^{\tfrac{\pi}{4}}\dfrac{\sin 4x}{\sqrt{\cos^2x+1}+\sqrt{\sin^2x+1}}\mathrm{\,d}x=\dfrac{a\sqrt{2}+b\sqrt{6}+c}{6}$ với $a, b, c\in\mathbb{Z}$. Tính $P=|a|+|b|+|c|$. 
	\choice
	{$P=10$}
	{$P=12$}
	{$P=14$}
	{\True $P=36$}
	\loigiai{
		Ta có $I=\displaystyle\int\limits_0^{\tfrac{\pi}{4}}\dfrac{\sin 4x}{\sqrt{\cos^2x+1}+\sqrt{\sin^2x+1}}\mathrm{\,d}x=\sqrt{2}\displaystyle\int\limits_0^{\tfrac{\pi}{4}}\dfrac{2\sin 2x\cos 2x}{\sqrt{3+\cos 2x}+\sqrt{3-\cos 2x}}\mathrm{\,d}x$.\\
		Đặt $t=\cos 2x\Rightarrow \mathrm{\,d}t=-2\sin 2x\mathrm{\,d}x$. Đổi cận $\heva{&x=0\to t=1\\&x=\dfrac{\pi}{4}\to t=0.}$ \\
		Khi đó
		\begin{eqnarray*}
	    I&=&-\sqrt{2}\displaystyle\int\limits_1^0\dfrac{t}{\sqrt{3+t}+\sqrt{3-t}}\mathrm{\,d}t=\sqrt{2}\displaystyle\int\limits_0^1\dfrac{t}{\sqrt{3+t}+\sqrt{3-t}}\mathrm{\,d}t=\dfrac{1}{\sqrt{2}}\displaystyle\int\limits_0^1\left(\sqrt{3+t}-\sqrt{3-t}\right)\mathrm{\,d}t\\
		&=&\dfrac{1}{\sqrt{2}}\left[\dfrac{2}{3}\sqrt{(3+t)^3}+\dfrac{2}{3}\sqrt{(3-t)^3}\right]\bigg|_0^1 =\dfrac{16\sqrt{2}-12\sqrt{6}+8}{6}\\&\Rightarrow& \heva{&a=16\\&b=-12\\&c=8}\Rightarrow P=36.
	    \end{eqnarray*}}
\end{ex}
\begin{ex}%Câu 27%[Nguyễn Diệu Linh]%[2D3K2-4]
	Biết $\displaystyle\int\limits_1^4\sqrt{\dfrac{1}{4x}+\dfrac{\sqrt{x}+\mathrm{e}^x}{\sqrt{x}\mathrm{e}^{2x}}}\mathrm{\,d}x=a+\mathrm{e}^b-\mathrm{e}^c$ với $a, b, c\in\mathbb{Z}$. Tính $P=a+b+c$. 
	\choice
	{$P=-5$}
	{\True $P=-4$}
	{$P=-3$}
	{$P=3$}
	\loigiai{
		Ta có
		\begin{eqnarray*}
		\displaystyle\int\limits_1^4\sqrt{\dfrac{1}{4x}+\dfrac{\sqrt{x}+\mathrm{e}^x}{\sqrt{x}\mathrm{e}^{2x}}}\mathrm{\,d}x&=&\displaystyle\int\limits_1^4\sqrt{\dfrac{\mathrm{e}^{2x}+4x+4\mathrm{e}^x\sqrt{x}}{4x\mathrm{e}^{2x}}}\mathrm{\,d}x=\displaystyle\int\limits_1^4\sqrt{\dfrac{\left(\mathrm{e}^x+2\sqrt{x}\right)^2}{\left(2\mathrm{e}^x\sqrt{x}\right)^2}}\mathrm{\,d}x\\
		&=&\displaystyle\int\limits_1^4\dfrac{\mathrm{e}^x+2\sqrt{x}}{2\mathrm{e}^x\sqrt{x}}\mathrm{\,d}x=\displaystyle\int\limits_1^4\left(\dfrac{1}{2\sqrt{x}}+\dfrac{1}{\mathrm{e}^x}\right)\mathrm{\,d}x\\&=&\left(\sqrt{x}-\dfrac{1}{\mathrm{e}^x}\right)\bigg|_1^4=1-\dfrac{1}{\mathrm{e}^4}+\dfrac{1}{e}=1+\mathrm{e}^{-1}-\mathrm{e}^{-4}\\
		&\Rightarrow& \heva{&a=1\\&b=-1\\&c=-4}\Rightarrow P=a+b+c=-4.
	    \end{eqnarray*}
	}
\end{ex}
\begin{ex}%Câu 28%[Nguyễn Diệu Linh]%[2D3G2-4]
	Biết $\displaystyle\int\limits_0^2\sqrt{\dfrac{2+\sqrt{x}}{2-\sqrt{x}}}\mathrm{\,d}x=a\pi+b\sqrt{2}+c$ với $a, b, c\in\mathbb{Z}$. Tính $P=a+b+c$. 
	\choice
	{$P=-1$}
	{$P=2$}
	{\True $P=3$}
	{$P=4$}
	\loigiai{
		Đặt $\sqrt{x}=2\cos u$ với $u\in\left[0;\dfrac{\pi}{2}\right]$. Suy ra $x=4\cos^2u\Rightarrow \mathrm{\,d}x=-4\sin 2u\mathrm{\,d}u$.\\
		Đổi cận $\heva{&x=0\Rightarrow u=\dfrac{\pi}{2}\\&x=2\Rightarrow u=\dfrac{\pi}{4}}$. Khi đó
		\begin{eqnarray*}
	    I&=&4\displaystyle\int\limits_{\tfrac{\pi}{4}}^{\tfrac{\pi}{2}}\sqrt{\dfrac{2+2\cos u}{2-2\cos u}}\sin 2u\mathrm{\,d}u=8\displaystyle\int\limits_{\tfrac{\pi}{4}}^{\tfrac{\pi}{2}}\dfrac{\cos\dfrac{u}{2}}{\sin\dfrac{u}{2}}\cdot\sin u\cdot\cos u\mathrm{\,d}u\\
		&=&16\displaystyle\int\limits_{\tfrac{\pi}{4}}^{\tfrac{\pi}{2}}\cos^2\dfrac{u}{2}\cdot\cos u\mathrm{\,d}u=8\displaystyle\int\limits_{\tfrac{\pi}{4}}^{\tfrac{\pi}{2}}(1+\cos u)\cdot\cos u\mathrm{\,d}u\\&=&8\displaystyle\int\limits_{\tfrac{\pi}{4}}^{\tfrac{\pi}{2}}\cos u\mathrm{\,d}u+4\displaystyle\int\limits_{\tfrac{\pi}{4}}^{\tfrac{\pi}{2}}(1+\cos 2u)\mathrm{\,d}u\\
		&=&8\sin u\bigg|_{\tfrac{\pi}{4}}^{\tfrac{\pi}{2}} +\left(4x+2\cdot\sin 2u\right)\bigg|_{\tfrac{\pi}{4}}^{\tfrac{\pi}{2}} =\pi-4\sqrt{2}+6\\&\Rightarrow& \heva{&a=1\\&b=-4\\&c=6}\Rightarrow P=3.
	    \end{eqnarray*}
	}
\end{ex}
\begin{ex}%Câu 29%[Nguyễn Diệu Linh]%[2D3K2-4]
	Biết $I=\displaystyle\int\limits_1^e\dfrac{\ln^2x+\ln x}{(\ln x+x+1)^3}\mathrm{\,d}x=\dfrac{1}{a}-\dfrac{b}{(e+2)^2}$ với $a, b\in\mathbb{Z}^+$. Tính $P=b-a$. 
	\choice
	{$P=-8$}
	{\True $P=-6$}
	{$P=6$}
	{$P=10$}
	\loigiai{
		Ta có $\displaystyle\int\limits_1^e\dfrac{\ln^2x+\ln x}{(\ln x+x+1)^3}\mathrm{\,d}x=\displaystyle\int\limits_1^e\dfrac{\ln x+1}{\ln x+x+1}\cdot\dfrac{\ln x}{(\ln x+x+1)^2}\mathrm{\,d}x$.\\
		Đặt $t=\dfrac{\ln x+1}{\ln x+x+1}\Rightarrow \mathrm{\,d}t=\left(\dfrac{\ln x+1}{\ln x+x+1}\right)^/\mathrm{\,d}x=-\dfrac{\ln x}{(\ln x+x+1)^2}\mathrm{\,d}x$.\\
		Đổi cận $\heva{&x=1\to t=\dfrac{1}{2}\\&x=e\to t=\dfrac{2}{e+2}}$. Khi đó $I=-\displaystyle\int\limits_{\tfrac{1}{2}}^{\tfrac{2}{e+2}} t\mathrm{\,d}t=-\dfrac{1}{2}t^2\bigg|_{\tfrac{1}{2}}^{\tfrac{2}{e+2}}=\dfrac{1}{8}-\dfrac{2}{(e+2)^2}$.}
\end{ex}
\begin{ex}%Câu 30%[Nguyễn Diệu Linh]%[2D3G2-4]
	Biết $\displaystyle\int\limits_{-\tfrac{\pi}{6}}^{\tfrac{\pi}{6}}\dfrac{x\cos x}{\sqrt{1+x^2}+x}\mathrm{\,d}x=a+\dfrac{{\pi}^2}{b}+\dfrac{\sqrt{3}\pi}{c}$ với $a, b, c$ là các số nguyên. Tính $P=a-b+c$. 
	\choice
	{$P=-37$}
	{$P=-35$}
	{\True $P=35$}
	{$P=41$}
	\loigiai{
		Ta có $I=\displaystyle\int\limits_{-\tfrac{\pi}{6}}^{\tfrac{\pi}{6}}\dfrac{x\cos x}{\sqrt{1+x^2}+x}\mathrm{\,d}x=\displaystyle\int\limits_{-\tfrac{\pi}{6}}^{\tfrac{\pi}{6}} x\cos x\left(\sqrt{1+x^2}-x\right)\mathrm{\,d}x=\displaystyle\int\limits_{-\tfrac{\pi}{6}}^{\tfrac{\pi}{6}} x\left(\sqrt{1+x^2}-x\right)\cos x\mathrm{\,d}x$.\\
		Lại có
		\begin{eqnarray*}
	    I&=&\displaystyle\int\limits_{-\tfrac{\pi}{6}}^{\tfrac{\pi}{6}}\dfrac{x\cos x}{\sqrt{1+x^2}+x}\mathrm{\,d}x=\displaystyle\int\limits_{\tfrac{\pi}{6}}^{-\tfrac{\pi}{6}}\dfrac{(-t)\cos(-t)}{\sqrt{1+(-t)^2}-t}\mathrm{d}(-t)=\displaystyle\int\limits_{\tfrac{\pi}{6}}^{-\tfrac{\pi}{6}}\dfrac{t\cos t}{\sqrt{1+t^2}-t}\mathrm{\,d}t.\\
		&=&-\displaystyle\int\limits_{-\tfrac{\pi}{6}}^{\tfrac{\pi}{6}} t\left(\sqrt{1+t^2}+t\right)\cos t\mathrm{\,d}t=-\displaystyle\int\limits_{-\tfrac{\pi}{6}}^{\tfrac{\pi}{6}} x\left(\sqrt{1+x^2}+x\right)\cos x\mathrm{\,d}x.
	    \end{eqnarray*}
		Suy ra
		\begin{eqnarray*}
	    &&2I=\displaystyle\int\limits_{-\tfrac{\pi}{6}}^{\tfrac{\pi}{6}} x\left(\sqrt{1+x^2}-x\right)\cos x\mathrm{\,d}x-\displaystyle\int\limits_{-\tfrac{\pi}{6}}^{\tfrac{\pi}{6}} x\left(\sqrt{1+x^2}+x\right)\cos x\mathrm{\,d}x=-2\displaystyle\int\limits_{-\tfrac{\pi}{6}}^{\tfrac{\pi}{6}} x^2\cos x\mathrm{\,d}x\\
		&\Rightarrow& I=-\displaystyle\int\limits_{-\tfrac{\pi}{6}}^{\tfrac{\pi}{6}} x^2\cos x\mathrm{\,d}x. 
	    \end{eqnarray*}
		Tích phân từng phần hai lần ta được $I=2+\dfrac{{\pi}^2}{-36}+\dfrac{\sqrt{3}\pi}{-3}
		\Rightarrow \heva{&a=2\\&b=-36\\&c=-3}\Rightarrow P=a-b+c=35$.}
\end{ex}
\paragraph{Vấn đề 5. Tính tích phân hàm phân nhánh.}
\begin{ex}%[2D3B2-1]
	Cho hàm số $f(x)=\heva{&x+1 & \text{ khi } x\geq 0\\&\mathrm{e}^{2x} & \text{ khi } x\leq 0}$. Tính tích phân $I=\displaystyle\int\limits_{-1}^2 f(x)\mathrm{\,d}x$. 
	\choice
	{$I=\dfrac{3\mathrm{e}^2-1}{2\mathrm{e}^2}$}
	{$I=\dfrac{7\mathrm{e}^2+1}{2\mathrm{e}^2}$}
	{\True $I=\dfrac{9\mathrm{e}^2-1}{2\mathrm{e}^2}$}
	{$I=\dfrac{11\mathrm{e}^2-11}{2\mathrm{e}^2}$}
	\loigiai{
		Ta có $I=\displaystyle\int\limits_{-1}^0 f(x)\mathrm{\,d}x+\displaystyle\int\limits_0^2 f(x)\mathrm{\,d}x=\displaystyle\int\limits_{-1}^0\mathrm{e}^{2x}\mathrm{\,d}x+\displaystyle\int\limits_0^2(x+1)\mathrm{\,d}x=\dfrac{9\mathrm{e}^2-1}{2\mathrm{e}^2}$.}
\end{ex}
\begin{ex}%[2D3K1-1]
	Cho hàm số $f(x)$ xác định trên $\mathbb{R}\setminus\left\{\dfrac{1}{2}\right\}$, thỏa $f'(x)=\dfrac{2}{2x-1}, f(0)=1$ và $f(1)=2$. Giá trị của biểu thức $f(-1)+f(3)$ bằng
	\choice
	{$\ln 15$}
	{$2+\ln 15$}
	{\True $3+\ln 15$}
	{$4+\ln 15$}
	\loigiai{
		Ta có $f'(x)=\dfrac{2}{2x-1}$. Suy ra 
		\begin{eqnarray*}
			f(x)=\displaystyle\int\dfrac{2}{2x-1}\mathrm{\,d}x&=&\ln|2x-1|+C\\
			&=&\heva{&\ln(1-2x)+C_1 &&\text{ khi }x<\dfrac{1}{2}\\&\ln(2x-1)+C_2 &&\text{ khi } x>\dfrac{1}{2}.}
		\end{eqnarray*}
		$f(0)=1\Rightarrow \ln(1-2\cdot 0)+C_1=1\Rightarrow C_1=1$.\\
		$f(1)=2\Rightarrow \ln(2\cdot 1-1)+C_2=2\Rightarrow C_2=2$.\\
		Do đó $f(x)=\heva{&\ln(1-2x)+1 &\text{ khi }x<\dfrac{1}{2}\\&\ln(2x-1)+2 &\text{ khi } x>\dfrac{1}{2}} \Rightarrow\heva{&f(-1)=\ln 3+1\\&f(3)=\ln 5+2.}$ \\
		Suy ra $f(-1)+f(3)=3+\ln 5+\ln 3=3+\ln 15$.
	}
\end{ex}
\begin{ex}%[2D3K1-1]
	Cho hàm số $f(x)$ xác định trên $\mathbb{R}\setminus\{-2;1\}$, thỏa mãn $f'(x)=\dfrac{1}{x^2+x-2}$, $f(-3)-f(3)=0$ và $f(0)=\dfrac{1}{3}$. Giá trị biểu thức $f(-4)+f(-1)-f(4)$ bằng
	\choice
	{$\dfrac{1}{3}\ln 20+\dfrac{1}{3}$}
	{\True $\dfrac{1}{3}\ln 2+\dfrac{1}{3}$}
	{$\ln 80+1$}
	{$\dfrac{1}{3}\ln\dfrac{8}{5}+1$}
	\loigiai{
		Ta có $f'(x)=\dfrac{1}{x^2+x-2}=\dfrac{1}{3}\left(\dfrac{1}{x-1}-\dfrac{1}{x+2}\right)$. Suy ra \[f(x)=\displaystyle\int\dfrac{1}{x^2+x-2}\mathrm{\,d}x=\heva{&\dfrac{1}{3}\left[\ln(1-x)-\ln(-x-2)\right]+C_1 &&\text{ khi }x <-2\\&\dfrac{1}{3}\left[\ln(1-x)-\ln(x+2)\right]+C_2 &&\text{ khi } -2<x<1\\&\dfrac{1}{3}\left[\ln(x-1)-\ln(x+2)\right]+C_3&&\text{ khi }x>1.}\]
		$f(0)=\dfrac{1}{3}\Rightarrow  \dfrac{1}{3}\left[\ln(1-0)-\ln(0+2)\right]+C_2=\dfrac{1}{3}\Rightarrow C_2=\dfrac{1}{3}\ln 2+\dfrac{1}{3}$.\\
		$f(-3)-f(3)=0\Rightarrow  C_1-C_3=\dfrac{1}{3}\ln\dfrac{1}{10}$.\\
		Ta có $f(-4)+f(-1)-f(4)=\dfrac{1}{3}\ln\dfrac{5}{2}+\dfrac{1}{3}\ln 2-\dfrac{1}{3}\ln\dfrac{1}{2}+C_2+C_1-C_3=\dfrac{1}{3}\ln 2+\dfrac{1}{3}$.
	}
\end{ex}
\begin{ex}%[2D3K1-2]
	Cho hàm số $f(x)$ xác định trên $(0;+\infty)\setminus\{\mathrm{e}\}$, thỏa mãn $f'(x)=\dfrac{1}{x(\ln x-1)}, f\left(\dfrac{1}{\mathrm{e}^2}\right)=\ln 6$ và $f(\mathrm{e}^2)=3$. Giá trị biểu thức $f\left(\dfrac{1}{\mathrm{e}}\right)+f(\mathrm{e}^3)$ bằng
	\choice
	{\True $3(\ln 2+1)$}
	{$2\ln 2$}
	{$3\ln 2+1$}
	{$\ln 2+3$}
	\loigiai{
		Ta có $f'(x)=\dfrac{1}{x(\ln x-1)}$. Suy ra 
		\begin{eqnarray*}
			f(x)=\displaystyle\int\dfrac{1}{x(\ln x-1)}\mathrm{\,d}x&=&\displaystyle\int\dfrac{\mathrm{d}(\ln x-1)}{(\ln x-1)}=\ln|\ln x-1|+C\\
			&=&\heva{&\ln(1-\ln x)+C_1 && \text{ khi } x\in(0;\mathrm{e})\\&\ln(\ln x-1)+C_2 &&\text{ khi } x\in(\mathrm{e};+\infty).}
		\end{eqnarray*}
		$f\left(\dfrac{1}{\mathrm{e}^2}\right)=\ln 6\Rightarrow  \ln\left(1-\ln\dfrac{1}{\mathrm{e}^2}\right)+C_1=\ln 6\Rightarrow C_1=\ln 2$.\\
		$f(\mathrm{e}^2)=3\Rightarrow  \ln\left(\ln\mathrm{e}^2-1\right)+C_2=3\Rightarrow C_2=3$.\\
		Do đó $f(x)=\heva{&\ln(1-\ln x)+\ln 2 &&\text{ khi }x\in(0;\mathrm{e})\\&\ln(\ln x-1)+3 &&\text{ khi } x\in(\mathrm{e};+\infty)}\Rightarrow\heva{&f\left(\dfrac{1}{\mathrm{e}}\right)=\ln 2+\ln 2\\&f(\mathrm{e}^3)=\ln 2+3.}$ \\
		Suy ra $f\left(\dfrac{1}{\mathrm{e}}\right)+f(\mathrm{e}^3)=3(\ln 2+1)$.
	}
\end{ex}
\begin{ex}%[2D3K1-1]
	Cho $F(x)$ là một nguyên hàm của hàm số $y=\dfrac{1}{1+\sin 2x}$ với $x\in\mathbb{R}\setminus\left\{-\dfrac{\pi}{4}+k\pi,k\in\mathbb{Z}\right\}$. Biết $F(0)=1, F(\pi)=0$, tính giá trị biểu thức $P=F\left(-\dfrac{\pi}{12}\right)-F\left(\dfrac{11\pi}{12}\right)$. 
	\choice
	{$P=0$}
	{$P=2-\sqrt{3}$}
	{\True $P=1$}
	{Không tồn tại $P$}
	\loigiai{
		Với $x$ thuộc vào mỗi khoảng $\left(-\dfrac{\pi}{4}+k\pi;-\dfrac{\pi}{4}+k\pi\right), k\in\mathbb{Z}$ ta có
		\[F(x)=\displaystyle\int\dfrac{\mathrm{\,d}x}{1+\sin 2x}=\displaystyle\int\dfrac{\mathrm{\,d}x}{\left(\sin x+\cos x\right)^2}= \displaystyle\int\dfrac{\mathrm{\,d}x}{2\cos^2\left(x+\dfrac{\pi}{4}\right)}=\dfrac{1}{2}\tan\left(x+\dfrac{\pi}{4}\right)+C.\]
		Ta có $0;-\dfrac{\pi}{12}\in\left(-\dfrac{\pi}{4};\dfrac{\pi}{4}\right)$ nên $F(0)-F\left(-\dfrac{\pi}{12}\right)=\dfrac{1}{2}\tan\left(x-\dfrac{\pi}{4}\right)\bigg|_{-\tfrac{\pi}{12}}^0=-\dfrac{1}{2}+\dfrac{\sqrt{3}}{2}$.\\
		Suy ra $F\left(-\dfrac{\pi}{12}\right)=\dfrac{3}{2}-\dfrac{\sqrt{3}}{2}$.\\
		Ta có $\pi;\dfrac{11\pi}{12}\in\left(\dfrac{\pi}{4};\dfrac{5\pi}{4}\right)$ nên $F(\pi)-F\left(\dfrac{11\pi}{12}\right)=\dfrac{1}{2}\tan\left(x-\dfrac{\pi}{4}\right)\bigg|_{\tfrac{11\pi}{12}}^{\pi}=-\dfrac{1}{2}+\dfrac{\sqrt{3}}{2}$.\\
		Suy ra $F\left(\dfrac{11\pi}{12}\right)=\dfrac{1}{2}-\dfrac{\sqrt{3}}{2}$.\\
		Vậy $P=F\left(-\dfrac{\pi}{12}\right)-F\left(\dfrac{11\pi}{12}\right)=1$.}
\end{ex}
\paragraph{Vấn đề 6. Tính tích phân dựa vào tính chất.}
\begin{ex}%[2D3K2-2]
	Cho hàm số $f(x)$ là hàm số lẻ, liên tục trên $[-4; 4]$. Biết rằng $\displaystyle\int\limits_{-2}^0 f(-x)\mathrm{\,d}x=2$ và $\displaystyle\int\limits_1^2 f(-2x)\mathrm{\,d}x=4$. Tính tích phân $I=\displaystyle\int\limits_0^4 f(x)\mathrm{\,d}x$. 
	\choice
	{$I=-10$}
	{\True $I=-6$}
	{$I=6$}
	{$I=10$}
	\loigiai{
		Do $f(x)$ là hàm lẻ nên $f(-x)=-f(x)$.\\
		Xét $A=\displaystyle\int\limits_{-2}^0 f(-x)\mathrm{\,d}x=2$. Đặt $t=-x\Rightarrow  \mathrm{\,d}t=-\mathrm{\,d}x$. Đổi cận $\heva{&x=-2\Rightarrow t=2\\&x=0\Rightarrow t=0.}$ \\
		Khi đó $A=-\displaystyle\int\limits_2^0 f(t)\mathrm{\,d}t=\displaystyle\int\limits_0^2 f(t)\mathrm{\,d}t=\displaystyle\int\limits_0^2 f(x)\mathrm{\,d}x$.\\
		Xét $B=\displaystyle\int\limits_1^2 f(-2x)\mathrm{\,d}x=-\displaystyle\int\limits_1^2 f(2x)\mathrm{\,d}x$. \\
		Đặt $u=2x\Rightarrow  \mathrm{\,d}u=2\mathrm{\,d}x$. Đổi cận $\heva{&x=1\Rightarrow u=2\\&x=2\Rightarrow u=4.}$ \\
		Khi đó $B=-\dfrac{1}{2}\displaystyle\int\limits_2^4 f(u)\mathrm{\,d}u=-\dfrac{1}{2}\displaystyle\int\limits_2^4 f(x)\mathrm{\,d}x\Rightarrow  \displaystyle\int\limits_2^4 f(x)\mathrm{\,d}x=-2B=-2\cdot 4=-8$.\\
		Vậy $I=\displaystyle\int\limits_0^4 f(x)\mathrm{\,d}x=\displaystyle\int\limits_0^2 f(x)\mathrm{\,d}x+\displaystyle\int\limits_2^4 f(x)\mathrm{\,d}x=2-8=-6$.}
\end{ex}
\begin{ex}%[2D3K2-2]
	Cho hàm số $f(x)$ là hàm số chẵn, liên tục trên $[-1;6]$. Biết rằng $\displaystyle\int\limits_{-1}^2 f(x)\mathrm{\,d}x=8$ và $\displaystyle\int\limits_1^3 f(-2x)\mathrm{\,d}x=3$. Tính tích phân $I=\displaystyle\int\limits_{-1}^6 f(x)\mathrm{\,d}x$. 
	\choice
	{$I=2$}
	{$I=5$}
	{$I=11$}
	{\True $I=14$}
	\loigiai{
		Vì $f(x)$ là hàm số chẵn nên $\displaystyle\int\limits_1^3 f(-2x)\mathrm{\,d}x=\displaystyle\int\limits_1^3 f(2x)\mathrm{\,d}x=3$.\\
		Xét $K=\displaystyle\int\limits_1^3 f(2x)\mathrm{\,d}x=3$. Đặt $t=2x\Rightarrow  \mathrm{\,d}t=2\mathrm{\,d}x$. Đổi cận $\heva{&x=1\Rightarrow t=2\\&x=3\Rightarrow t=6.}$ \\
		Khi đó $K=\dfrac{1}{2}\displaystyle\int\limits_2^6 f(t)\mathrm{\,d}t=\dfrac{1}{2}\displaystyle\int\limits_2^6 f(x)\mathrm{\,d}x\Rightarrow  \displaystyle\int\limits_2^6 f(x)\mathrm{\,d}x=2K=6$.\\
		Vậy $I=\displaystyle\int\limits_{-1}^6 f(x)\mathrm{\,d}x=\displaystyle\int\limits_{-1}^2 f(x)\mathrm{\,d}x+\displaystyle\int\limits_2^6 f(x)\mathrm{\,d}x=8+6=14$.}
\end{ex}
\begin{ex}%[2D3G2-2]
	Cho hàm số $f(x)$ liên tục trên $[3;7],$ thỏa mãn $f(x)=f(10-x)$ với mọi $x\in[3;7]$ và $\displaystyle\int\limits_3^7 f(x)\mathrm{\,d}x=4$. Tính tích phân $I=\displaystyle\int\limits_3^7 xf(x)\mathrm{\,d}x$. 
	\choice
	{\True $I=20$}
	{$I=40$}
	{$I=60$}
	{$I=80$}
	\loigiai{
		Đặt $t=(3+7)-x=10-x\Rightarrow  \mathrm{\,d}t=-\mathrm{\,d}x$. Đổi cận $\heva{&x=7\Rightarrow t=3\\&x=3\Rightarrow t=7.}$ \\
		Khi đó
		\allowdisplaybreaks{
			\begin{eqnarray*}
				I&=&-\displaystyle\int\limits_7^3(10-t)f(10-t)\mathrm{\,d}t\\
				&=&\displaystyle\int\limits_3^7(10-t)f(10-t)\mathrm{\,d}t\\
				&=&\displaystyle\int\limits_3^7(10-x)f(10-x)\mathrm{\,d}x\\
				&=&\displaystyle\int\limits_3^7(10-x)f(x)\mathrm{\,d}x \quad (\text{do } f(x)=f(10-x))\\
				&=&10\displaystyle\int\limits_3^7 f(x)\mathrm{\,d}x-\displaystyle\int\limits_3^7 xf(x)\mathrm{\,d}x\\
				&=&10\displaystyle\int\limits_3^7 f(x)\mathrm{\,d}x-I.
			\end{eqnarray*} 
		}
		Suy ra $2I=10\displaystyle\int\limits_3^7 f(x)\mathrm{\,d}x=10\cdot 4=40\Rightarrow  I=20$.}
\end{ex}
\begin{ex}%[2D3G2-2]
	Cho hàm số $y=f(x)$ là hàm số chẵn và liên tục trên đoạn $[-\pi;\pi]$, thỏa mãn $\displaystyle\int\limits_0^{\pi} f(x)\mathrm{\,d}x=2018$. Giá trị của tích phân $I=\displaystyle\int\limits_{-\pi}^{\pi}\dfrac{f(x)}{{2018}^x+1}\mathrm{\,d}x$ bằng
	\choice
	{$I=0$}
	{$I=\dfrac{1}{2018}$}
	{\True $I=2018$}
	{$I=4036$}
	\loigiai{
		Đặt $x=-t\Rightarrow  \mathrm{\,d}x=-\mathrm{\,d}t$. Đổi cận $\heva{&x=-\pi\Rightarrow t=\pi\\&x=\pi\Rightarrow t=-\pi.}$ \\
		Khi đó \[I=-\displaystyle\int\limits_{\pi}^{-\pi}\dfrac{f(-t)}{{2018}^{-t}+1}\mathrm{\,d}t=\displaystyle\int\limits_{-\pi}^{\pi}\dfrac{f(-t)}{{2018}^{-t}+1}\mathrm{\,d}t=\displaystyle\int\limits_{-\pi}^{\pi}\dfrac{{2018}^tf(-t)}{1+{2018}^t}\mathrm{\,d}t=\displaystyle\int\limits_{-\pi}^{\pi}\dfrac{{2018}^xf(-x)}{1+{2018}^x}\mathrm{\,d}x.\]
		Vì $y=f(x)$ là hàm số chẵn trên đoạn $[-\pi;\pi]$ nên $f(-x)=f(x)\Rightarrow  I=\displaystyle\int\limits_{-\pi}^{\pi}\dfrac{{2018}^xf(x)}{{2018}^x+1}\mathrm{\,d}x$.\\
		Vậy $2I=\displaystyle\int\limits_{-\pi}^{\pi}\dfrac{f(x)}{{2018}^x+1}\mathrm{\,d}x+\displaystyle\int\limits_{-\pi}^{\pi}\dfrac{{2018}^xf(x)}{{2018}^x+1}\mathrm{\,d}x=\displaystyle\int\limits_{-\pi}^{\pi} f(x)\mathrm{\,d}x=2\displaystyle\int\limits_0^{\pi} f(x)\mathrm{\,d}x=2\cdot 2018\Rightarrow I=2018$.}
\end{ex}
\begin{ex}%[2D3G2-2]
	Biết $\displaystyle\int\limits_0^{\pi}\dfrac{x\sin^{2018}x}{\sin^{2018}x+\cos^{2018}x}\mathrm{\,d}x=\dfrac{{\pi}^a}{b}$ với $a,b\in\mathbb{Z}^+$. Tính $P=2a+b$. 
	\choice
	{$P=6$}
	{\True $P=8$}
	{$P=10$}
	{$P=12$}
	\loigiai{
		Gọi $I=\displaystyle\int\limits_0^{\pi}\dfrac{x\sin^{2018}x}{\sin^{2018}x+\cos^{2018}x}\mathrm{\,d}x$.\\
		Đặt $t=\pi-x\Rightarrow  \mathrm{\,d}t=-\mathrm{\,d}x$. Đổi cận $\heva{&x=0\Rightarrow t=\pi\\&x=\pi\Rightarrow t=0.}$ \\
		Khi đó \[I=-\displaystyle\int\limits_{\pi}^0\dfrac{(\pi-t)\sin^{2018}(\pi-t)}{\sin^{2018}(\pi-t)+\cos^{2018}(\pi-t)}\mathrm{\,d}t=\displaystyle\int\limits_0^{\pi}\dfrac{(\pi-t)\sin^{2018}t}{\sin^{2018}t+\cos^{2018}t}\mathrm{\,d}t=\displaystyle\int\limits_0^{\pi}\dfrac{(\pi-x)\sin^{2018}x}{\sin^{2018}x+\cos^{2018}x}\mathrm{\,d}x.\]
		Suy ra \[2I=\displaystyle\int\limits_0^{\pi}\dfrac{x\sin^{2018}x}{\sin^{2018}x+\cos^{2018}x}\mathrm{\,d}x+\displaystyle\int\limits_0^{\pi}\dfrac{(\pi-x)\sin^{2018}x}{\sin^{2018}x+\cos^{2018}x}\mathrm{\,d}x=\displaystyle\int\limits_0^{\pi}\dfrac{\pi\sin^{2018}x}{\sin^{2018}x+\cos^{2018}x}\mathrm{\,d}x.\]
		Suy ra 
		\[I=\dfrac{\pi}{2}\displaystyle\int\limits_0^{\pi}\dfrac{\sin^{2018}x}{\sin^{2018}x+\cos^{2018}x}\mathrm{\,d}x=\dfrac{\pi}{2}\left[\displaystyle\int\limits_0^{\tfrac{\pi}{2}}\dfrac{\sin^{2018}x}{\sin^{2018}x+\cos^{2018}x}\mathrm{\,d}x+\displaystyle\int\limits_{\tfrac{\pi}{2}}^{\pi}\dfrac{\sin^{2018}x}{\sin^{2018}x+\cos^{2018}x}\mathrm{\,d}x\right].\]
		Đặt $x=u+\dfrac{\pi}{2}$ ta suy ra \[\displaystyle\int\limits_{\tfrac{\pi}{2}}^{\pi}\dfrac{\sin^{2018}x}{\sin^{2018}x+\cos^{2018}x}\mathrm{\,d}x=\displaystyle\int\limits_0^{\tfrac{\pi}{2}}\dfrac{\cos^{2018}u}{\sin^{2018}u+\cos^{2018}u}\mathrm{\,d}u=\displaystyle\int\limits_{\tfrac{\pi}{2}}^{\pi}\dfrac{\cos^{2018}x}{\sin^{2018}x+\cos^{2018}x}\mathrm{\,d}x.\]
		Vậy $I=\dfrac{\pi}{2}\displaystyle\int\limits_0^{\tfrac{\pi}{2}}\mathrm{\,d}x=\dfrac{{\pi}^2}{4}\Rightarrow  \heva{&a=2\\&b=4}\Rightarrow  P=8$.}
\end{ex}
\paragraph{Vấn đề 7. Kỹ thuật phương trình hàm}
\begin{ex}%[2D3G2-4]
	Cho hàm số $y=f(x)$ liên tục trên $\left[-\dfrac{\pi}{2};\dfrac{\pi}{2}\right]$ và thỏa mãn $2f(x)+f(-x)=\cos x$. Tính tích phân $I=\displaystyle\int\limits_{-\tfrac{\pi}{2}}^{\tfrac{\pi}{2}} f(x)\mathrm{\,d}x$. 
	\choice
	{$I=-2$}
	{\True $I=\dfrac{2}{3}$}
	{$I=\dfrac{3}{2}$}
	{$I=2$}
	\loigiai{
		Từ giả thiết, thay $x$ bằng $-x$ ta được $2f(-x)+f(x)=\cos x$.\\
		Do đó ta có hệ 
		\[\heva{&2f(x)+f(-x)=\cos x\\&2f(-x)+f(x)=\cos x}\Leftrightarrow\heva{&4f(x)+2f(-x)=2\cos x\\&f(x)+2f(-x)=\cos x}\Rightarrow  f(x)=\dfrac{1}{3}\cos x.\]
		Khi đó $I=\displaystyle\int\limits_{-\tfrac{\pi}{2}}^{\tfrac{\pi}{2}} f(x)\mathrm{\,d}x=\dfrac{1}{3}\displaystyle\int\limits_{-\tfrac{\pi}{2}}^{\tfrac{\pi}{2}}\cos x\mathrm{\,d}x=\dfrac{1}{3}\sin x\bigg|_{-\tfrac{\pi}{2}}^{\tfrac{\pi}{2}} =\dfrac{2}{3}$.}
\end{ex}
\begin{ex}%[2D3G2-4]
	Cho hàm số $y=f(x)$ liên tục trên $[-2;2]$ và thỏa mãn $2f(x)+3f(-x)=\dfrac{1}{4+x^2}$. Tính tích phân $I=\displaystyle\int\limits_{-2}^2 f(x)\mathrm{\,d}x$. 
	\choice
	{$I=-\dfrac{\pi}{10}$}
	{$I=-\dfrac{\pi}{20}$}
	{\True $I=\dfrac{\pi}{20}$}
	{$I=\dfrac{\pi}{10}$}
	\loigiai{
		Từ giả thiết, thay $x$ bằng $-x$ ta được $2f(-x)+3f(x)=\dfrac{1}{4+x^2}$.\\
		Do đó ta có hệ \[\heva{&2f(x)+3f(-x)=\dfrac{1}{4+x^2}\\&2f(-x)+3f(x)=\dfrac{1}{4+x^2}}\Leftrightarrow\heva{&4f(x)+6f(-x)=\dfrac{2}{4+x^2}\\&9f(x)+6f(-x)=\dfrac{3}{4+x^2}}\Rightarrow  f(x)=\dfrac{1}{5\left(4+x^2\right)}.\]
		Khi đó $I=\displaystyle\int\limits_{-2}^2 f(x)\mathrm{\,d}x=\dfrac{1}{5}\displaystyle\int\limits_{-2}^2\dfrac{1}{4+x^2}\mathrm{\,d}x=\dfrac{\pi}{20}$.}
\end{ex}
\begin{ex}%[2D3G2-4]
	Cho hàm số $y=f(x)$ liên tục trên $[0;1]$ và thỏa mãn $x^2f(x)+f(1-x)=2x-x^4$. Tính tích phân $I=\displaystyle\int\limits_0^1 f(x)\mathrm{\,d}x$. 
	\choice
	{$I=\dfrac{1}{2}$}
	{$I=\dfrac{3}{5}$}
	{\True $I=\dfrac{2}{3}$}
	{$I=\dfrac{4}{3}$}
	\loigiai{
		Từ giả thiết, thay $x$ bằng $1-x$ ta được
		\begin{eqnarray*}
			&&(1-x)^2f(1-x)+f(x)=2(1-x)-(1-x)^4 \\
			&\Leftrightarrow& \left(x^2-2x+1\right)f(1-x)+f(x)=1+2x-6x^2+4x^3-x^4. \qquad (1) 
		\end{eqnarray*} 
		Ta có $x^2f(x)+f(1-x)=2x-x^4\Rightarrow  f(1-x)=2x-x^4-x^2f(x)$. \\
		Thay vào $(1)$ ta được
		\begin{eqnarray*}
			&&\left(x^2-2x+1\right)\left[2x-x^4-x^2f(x)\right]+f(x)=1+2x-6x^2+4x^3-x^4 \\
			&\Leftrightarrow&\left(1-x^2+2x^3-x^4\right)f(x)=x^6-2x^5+2x^3-2x^2+1 \\
			&\Leftrightarrow&\left(1-x^2+2x^3-x^4\right)f(x)=\left(1-x^2\right)\left(1-x^2+2x^3-x^4\right)\\
			&\Rightarrow& f(x)=1-x^2.
		\end{eqnarray*}
		Vậy $I=\displaystyle\int\limits_0^1 f(x)\mathrm{\,d}x=\displaystyle\int\limits_0^1\left(1-x^2\right)\mathrm{\,d}x=\left(x-\dfrac{1}{3}x^3\right)\bigg|_0^1=\dfrac{2}{3}$.}
\end{ex}
\begin{ex}%[2D3G2-4]
	Cho hàm số $f(x)$ liên tục trên $\left[\dfrac{1}{2};2\right]$ và thỏa mãn $f(x)+2f\left(\dfrac{1}{x}\right)=3x$. Tính tích phân $I=\displaystyle\int\limits_{\tfrac{1}{2}}^2\dfrac{f(x)}{x}\mathrm{\,d}x$. 
	\choice
	{$I=\dfrac{1}{2}$}
	{\True $I=\dfrac{3}{2}$}
	{$I=\dfrac{5}{2}$}
	{$I=\dfrac{7}{2}$}
	\loigiai{
		Từ giả thiết, thay $x$ bằng $\dfrac{1}{x}$ ta được $f\left(\dfrac{1}{x}\right)+2f(x)=\dfrac{3}{x}$.\\
		Do đó ta có hệ \[\heva{&f(x)+2f\left(\dfrac{1}{x}\right)=3x\\&f\left(\dfrac{1}{x}\right)+2f(x)=\dfrac{3}{x}}
		\Leftrightarrow\heva{&f(x)+2f\left(\dfrac{1}{x}\right)=3x\\&4f(x)+2f\left(\dfrac{1}{x}\right)=\dfrac{6}{x}}
		\Rightarrow  f(x)=\dfrac{2}{x}-x.\]
		Khi đó $I=\displaystyle\int\limits_{\tfrac{1}{2}}^2\dfrac{f(x)}{x}\mathrm{\,d}x =\displaystyle\int\limits_{\tfrac{1}{2}}^2\left(\dfrac{2}{x^2}-1\right)\mathrm{\,d}x =\left(-\dfrac{2}{x}-x\right)\bigg|_{\tfrac{1}{2}}^2 =\dfrac{3}{2}$.\\
		\textbf{Cách khác.}\\
		Từ $f(x)+2f\left(\dfrac{1}{x}\right)=3x\Rightarrow  f(x)=3x-2f\left(\dfrac{1}{x}\right)$.\\
		Khi đó $I=\displaystyle\int\limits_{\tfrac{1}{2}}^2\dfrac{f(x)}{x}\mathrm{\,d}x =\displaystyle\int\limits_{\tfrac{1}{2}}^2\left[3-2\dfrac{f\left(\dfrac{1}{x}\right)}{x}\right]\mathrm{\,d}x =3\displaystyle\int\limits_{\tfrac{1}{2}}^2\mathrm{\,d}x-2\displaystyle\int\limits_{\tfrac{1}{2}}^2\dfrac{f\left(\dfrac{1}{x}\right)}{x}\mathrm{\,d}x$.\\
		Xét $J=\displaystyle\int\limits_{\tfrac{1}{2}}^2\dfrac{f\left(\dfrac{1}{x}\right)}{x}\mathrm{\,d}x$. Đặt $t=\dfrac{1}{x}$, suy ra $\mathrm{\,d}t=-\dfrac{1}{x^2}\mathrm{\,d}x=-t^2\mathrm{\,d}x \Rightarrow  \mathrm{\,d}x=-\dfrac{1}{t^2}\mathrm{\,d}t$.\\
		Đổi cận $\heva{&x=\dfrac{1}{2}\Rightarrow t=2\\&x=2\Rightarrow t=\dfrac{1}{2}}$. Khi đó $J=\displaystyle\int\limits_2^{\tfrac{1}{2}} tf(t)\left(-\dfrac{1}{t^2}\right)\mathrm{\,d}t =\displaystyle\int\limits_{\tfrac{1}{2}}^2\dfrac{f(t)}{t}\mathrm{\,d}t=\displaystyle\int\limits_{\tfrac{1}{2}}^2\dfrac{f(x)}{x}\mathrm{\,d}x=I$.\\
		Vậy $I=3\displaystyle\int\limits_{\tfrac{1}{2}}^2\mathrm{\,d}x-2I\Rightarrow  I=\displaystyle\int\limits_{\tfrac{1}{2}}^2\mathrm{\,d}x=\dfrac{3}{2}$.}
\end{ex}
\begin{ex}%[2D3G2-4]
	Cho hàm số $f(x)$ liên tục trên $[0;1]$ và thỏa mãn $2f(x)+3f(1-x)=\sqrt{1-x^2}$. Tính tích phân $I=\displaystyle\int\limits_0^1 f(x)\mathrm{\,d}x$. 
	\choice
	{\True $\dfrac{\pi}{20}$}
	{$\dfrac{\pi}{16}$}
	{$\dfrac{\pi}{6}$}
	{$\dfrac{\pi}{4}$}	
	\loigiai{
		Từ giả thiết, thay $x$ bằng $1-x$ ta được $2f(1-x)+3f(x)=\sqrt{2x-x^2}$.\\
		Do đó ta có hệ \[\heva{&2f(x)+3f(1-x)=\sqrt{1-x^2}\\&2f(1-x)+3f(x)=\sqrt{2x-x^2}}\Leftrightarrow\heva{&4f(x)+6f(1-x)=2\sqrt{1-x^2}\\&9f(x)+6f(1-x)=3\sqrt{2x-x^2}.}\]
		Suy ra $f(x)=\dfrac{3\sqrt{2x-x^2}-2\sqrt{1-x^2}}{5}$.\\
		Vậy $I=\dfrac{1}{5}\displaystyle\int\limits_0^1\left(3\sqrt{2x-x^2}-2\sqrt{1-x^2}\right)\mathrm{\,d}x=\dfrac{\pi}{20}$. \\
		\textbf{Cách khác.}\\
		Từ $2f(x)+3f(1-x)=\sqrt{1-x^2}\Rightarrow  f(x)=\dfrac{1}{2}\left[\sqrt{1-x^2}-3f(1-x)\right]$.\\
		Khi đó $I=\displaystyle\int\limits_0^1 f(x)\mathrm{\,d}x=\dfrac{1}{2}\left[\displaystyle\int\limits_0^1\sqrt{1-x^2}\mathrm{\,d}x-3\displaystyle\int\limits_0^1 f(1-x)\mathrm{\,d}x\right]$.\\
		Xét $J=\displaystyle\int\limits_0^1 f(1-x)\mathrm{\,d}x$. Đặt $t=1-x\Rightarrow  \mathrm{\,d}t=-\mathrm{\,d}x$.\\
		Đổi cận $\heva{&x=0\Rightarrow t=1\\&x=1\Rightarrow t=0.}$\\
		Khi đó $J=-\displaystyle\int\limits_1^0 f(t)\mathrm{\,d}t=\displaystyle\int\limits_0^1 f(t)\mathrm{\,d}t=\displaystyle\int\limits_0^1 f(x)\mathrm{\,d}x=I$.\\
		Vậy $I=\dfrac{1}{2}\left[\displaystyle\int\limits_0^1\sqrt{1-x^2}\mathrm{\,d}x-3I\right]\Rightarrow  I=\dfrac{1}{5}\displaystyle\int\limits_0^1\sqrt{1-x^2}\mathrm{\,d}x=\dfrac{\pi}{20}$.}
\end{ex}
\paragraph{Vấn đề 8. Kỹ thuật biến đổi.}
\begin{ex}%[2D3G2-4]
	Cho hàm số $f(x)$ thỏa $f(x)f'(x)=3x^5+6x^2$. Biết rằng $f(0)=2$, tính $f^2(2)$. 
	\choice
	{$f^2(2)=64$}
	{$f^2(2)=81$}
	{\True $f^2(2)=100$}
	{$f^2(2)=144$}
	\loigiai{
		Từ giả thiết ta có 
		\[\displaystyle\int f(x)\cdot f'(x)\mathrm{\,d}x=\displaystyle\int\left(3x^5+6x^2\right)\mathrm{\,d}x\Leftrightarrow\dfrac{f^2(x)}{2}=\dfrac{x^6}{2}+2x^3+C.\]
		Thay $x=0$ vào hai vế, ta được $\dfrac{f^2(0)}{2}=C\Rightarrow C=2$.\\
		Suy ra $f^2(x)=x^6+4x^3+4\Rightarrow  f^2(2)=2^6+4\cdot 2^3+4=100$.}
\end{ex}
\begin{ex}%[2D3G2-4]
	Cho hàm số $f(x)$ có đạo hàm $f'(x)$ liên tục và nhận giá trị không âm trên $[1;+\infty)$, thỏa $f(1)=0,\mathrm{e}^{2f(x)}\cdot [f'(x)]^2=4x^2-4x+1$ với mọi $x\in[1;+\infty)$. Mệnh đề nào sau đây đúng?
	\choice
	{$-1<f'(4)<0$}
	{\True $0<f'(4)<1$}
	{$1<f'(4)<2$}
	{$2<f'(4)<3$}
	\loigiai{
		Từ giả thiết suy ra $\mathrm{e}^{f(x)}f'(x)=2x-1$ (do $f'(x)$ không âm trên $[1;+\infty)$).\\
		Suy ra $\displaystyle\int\mathrm{e}^{f(x)}f'(x)\mathrm{\,d}x=\displaystyle\int(2x-1)\mathrm{\,d}x\Leftrightarrow\mathrm{e}^{f(x)}=x^2-x+C$.\\
		Thay $x=1$ vào hai vế, ta được $\mathrm{e}^{f(1)}=1^2-1+C\Rightarrow C=1$.\\
		Suy ra 
		\[\mathrm{e}^{f(x)}=x^2-x+1\Rightarrow f(x)=\ln\left(x^2-x+1\right)\Rightarrow f'(x)=\dfrac{2x-1}{x^2-x+1}\Rightarrow  f'(4)=\dfrac{7}{13}.\]
	}
\end{ex}
\begin{ex}%[2D3G2-4]
	Cho hàm số $f(x)$ thỏa mãn $[f'(x)]^2+f(x)\cdot f''(x)=15x^4+12x$ với mọi $x\in\mathbb{R}$ và $f(0)=f'(0)=1$. Giá trị của $f^2(1)$ bằng
	\choice
	{$\dfrac{5}{2}$}
	{$\dfrac{9}{2}$}
	{\True $8$}
	{$10$}
	\loigiai{
		Nhận thấy được $[f'(x)]^2+f(x)\cdot f''(x)=[f(x)\cdot f'(x)]'$.\\
		Do đó giả thiết tương đương với $[f(x)\cdot f'(x)]'=15x^4+12x$.\\
		Suy ra $f(x)\cdot f'(x)=\displaystyle\int\left(15x^4+12x\right)\mathrm{\,d}x=3x^5+6x^2+C$.\\
		Mà $f(0)=f'(0)=1\Rightarrow C=1$. Do đó $f(x)\cdot f'(x)=3x^5+6x^2+1$.\\
		Suy ra 
		\[\displaystyle\int f(x)\cdot f'(x)\mathrm{\,d}x=\displaystyle\int\left(3x^5+6x^2+1\right)\mathrm{\,d}x\Leftrightarrow\dfrac{f^2(x)}{2}=\dfrac{x^6}{2}+2x^3+x+C'.\]
		Thay $x=0$ vào hai vế ta được $\dfrac{f^2(0)}{2}=C'\Rightarrow C'=\dfrac{1}{2}$.\\
		Vậy $f^2(x)=x^6+4x^3+2x+1\Rightarrow  f^2(1)=8$.
	}
\end{ex}
\begin{ex}%[2D3G2-4]
	Cho hàm số $f(x)$ có đạo hàm liên tục trên đoạn $[1;2]$ và thỏa mãn $f(x)>0,\forall x\in[1;2]$. Biết rằng $\displaystyle\int\limits_1^2 f'(x)\mathrm{\,d}x=10$ và $\displaystyle\int\limits_1^2\dfrac{f'(x)}{f(x)}\mathrm{\,d}x=\ln 2$. Tính $f(2)$. 
	\choice
	{$f(2)=-20$}
	{$f(2)=-10$}
	{$f(2)=10$}
	{\True $f(2)=20$}
	\loigiai{
		Ta có $\displaystyle\int\limits_1^2 f'(x)\mathrm{\,d}x=10\Leftrightarrow f(x)\bigg|_1^2 =10\Leftrightarrow f(2)-f(1)=10$. \qquad $(1)$\\
		Lại có
		\allowdisplaybreaks{
			\begin{eqnarray*}
				\displaystyle\int\limits_1^2 \dfrac{f'(x)}{f(x)}\mathrm{\,d}x=\ln 2 &\Leftrightarrow& \ln\left|f(x)\right|\bigg|_1^2 =\ln 2\\
				&\Leftrightarrow&\ln[f(x)]\bigg|_1^2 =\ln 2 \quad (\text{do } f(x)>0,\forall x\in[1;2])\\
				&\Leftrightarrow& \ln f(2)-\ln f(1)=\ln 2\\
				&\Leftrightarrow& \ln\dfrac{f(2)}{f(1)}=\ln 2\\
				&\Leftrightarrow& \dfrac{f(2)}{f(1)}=2. \qquad\qquad\quad (2)
			\end{eqnarray*}
		}
		Từ $(1)$ và $(2)$, suy ra $f(2)=20$.}
\end{ex}
\begin{ex}%[2D3G2-4]
	Cho hàm số $f(x)$ có đạo hàm liên tục trên $[-1;1]$, thỏa mãn $f(x)>0,\forall x\in\mathbb{R}$ và $f'(x)+2f(x)=0$. Biết rằng $f(1)=1$, giá trị của $f(-1)$ bằng
	\choice
	{$\mathrm{e}^{-2}$}
	{$\mathrm{e}^3$}
	{\True $\mathrm{e}^4$}
	{$3$}
	\loigiai{
		Ta có 
		\begin{eqnarray*}
			f'(x)+2f(x)=0&\Leftrightarrow& f'(x)=-2f(x)\\
			&\Leftrightarrow&\dfrac{f'(x)}{f(x)}=-2 \qquad (\text{do } f(x)>0)\\
			&\Rightarrow& \displaystyle\int\dfrac{f'(x)}{f(x)}\mathrm{\,d}x=\displaystyle\int-2\mathrm{\,d}x\\
			&\Leftrightarrow& \ln f(x)=-2x+C \qquad (\text{do } f(x)>0).		
		\end{eqnarray*}
		Mà $f(1)=1\Rightarrow C=2\Rightarrow\ln f(x)=-2x+2\Rightarrow  f(x)=\mathrm{e}^{-2x+2}\Rightarrow  f(-1)=\mathrm{e}^4$.}
\end{ex}
\begin{ex}%[2D3G2-4]
	Cho hàm số $f(x)$ xác định và liên tục trên $\mathbb{R}$ đồng thời thỏa mãn
	\[\heva{&f(x)>0,\forall x\in\mathbb{R}\\&f'(x)=-\mathrm{e}^xf^2(x),\forall x\in\mathbb{R}\\&f(0)=\dfrac{1}{2}.}\]
	Tính giá trị của $f(\ln 2)$. 
	\choice
	{$f(\ln 2)=\dfrac{1}{4}$}
	{\True $f(\ln 2)=\dfrac{1}{3}$}
	{$f(\ln 2)=\ln 2+\dfrac{1}{2}$}
	{$f(\ln 2)=\ln^22+\dfrac{1}{2}$}
	\loigiai{
		Ta có
		\begin{eqnarray*}
			f'(x)=-\mathrm{e}^xf^2(x)&\Leftrightarrow& \dfrac{f'(x)}{f^2(x)}=-\mathrm{e}^x \qquad (\text{do } f(x)>0).\\
			&\Rightarrow& \displaystyle\int\dfrac{f'(x)}{f^2(x)}\mathrm{\,d}x=\displaystyle\int-\mathrm{e}^x\mathrm{\,d}x\\
			&\Rightarrow& -\dfrac{1}{f(x)}=-\mathrm{e}^x+C\\
			&\Rightarrow& f(x)=\dfrac{1}{\mathrm{e}^x-C}.
		\end{eqnarray*} 
		Thay $x=0$ ta được $f(0)=\dfrac{1}{\mathrm{e}^0-C}\Rightarrow  {{f(0)=\dfrac{1}{2}}}C=-1$.\\
		Vậy $f(x)=\dfrac{1}{\mathrm{e}^x+1}\Rightarrow  f(\ln 2)=\dfrac{1}{\mathrm{e}^{\ln 2}+1}=\dfrac{1}{2+1}=\dfrac{1}{3}$.}
\end{ex}
\begin{ex}%[2D3G2-4]
	Cho hàm số $f(x)$ có đạo hàm liên tục trên $(0;+\infty)$, biết $f'(x)+(2x+3)f^2(x)=0, f(x)>0$ với mọi $x>0$ và $f(1)=\dfrac{1}{6}$. Tính $P=1+f(1)+f(2)+\cdots +f(2018)$. 
	\choice
	{$P=\dfrac{1009}{2020}$}
	{$P=\dfrac{2019}{2020}$}
	{\True $P=\dfrac{3029}{2020}$}
	{$P=\dfrac{4039}{2020}$}
	\loigiai{
		Ta có $f'(x)+(2x+3)f^2(x)=0\Leftrightarrow\dfrac{f'(x)}{f^2(x)}=-(2x+3)$ (do $f(x)>0$). Suy ra 
		\[\displaystyle\int\dfrac{f'(x)}{f^2(x)}\mathrm{\,d}x=-\displaystyle\int(2x+3)\mathrm{\,d}x\Leftrightarrow-\dfrac{1}{f(x)}=-x^2-3x+C\Rightarrow  f(x)=\dfrac{1}{x^2+3x-C}.\]
		Mà $f(1)=\dfrac{1}{6}\Rightarrow  \dfrac{1}{6}=\dfrac{1}{1^2+3\cdot 1-C}\Leftrightarrow C=-2\Rightarrow  f(x)=\dfrac{1}{x^2+3x+2}=\dfrac{1}{x+1}-\dfrac{1}{x+2}$.\\
		Suy ra $P=1+\left(\dfrac{1}{2}-\dfrac{1}{3}\right)+\left(\dfrac{1}{3}-\dfrac{1}{4}\right)+\cdots +\left(\dfrac{1}{2019}-\dfrac{1}{2020}\right)=\dfrac{3029}{2020}$.}
\end{ex}
\begin{ex}%[2D3G2-4]
	Cho hàm số $f(x)$ liên tục trên $[0;\sqrt{3}]$, thỏa mãn $f(x) >-1, f(0)=0$ và $f'(x)\sqrt{x^2+1}=2x\sqrt{f(x)+1}$. Giá trị của $f(\sqrt{3})$ bằng
	\choice
	{$0$}
	{\True $3$}
	{$7$}
	{$9$}
	\loigiai{
		Từ giả thiết suy ra $\dfrac{f'(x)}{\sqrt{f(x)+1}}=\dfrac{2x}{\sqrt{x^2+1}}$. Do đó
		\begin{eqnarray*}
			\displaystyle\int\dfrac{f'(x)}{\sqrt{f(x)+1}}\mathrm{\,d}x=\displaystyle\int\dfrac{2x}{\sqrt{x^2+1}}\mathrm{\,d}x&\Leftrightarrow& 2\displaystyle\int\dfrac{f'(x)}{2\sqrt{f(x)+1}}\mathrm{\,d}x=2\displaystyle\int\dfrac{\left(x^2+1\right)'}{2\sqrt{x^2+1}}\mathrm{\,d}x\\
			&\Leftrightarrow& 2\sqrt{f(x)+1}=2\sqrt{x^2+1}+C.
		\end{eqnarray*} 	
		Mà $f(0)=0\Rightarrow C=0\Rightarrow f(x)=x^2\Rightarrow  f(\sqrt{3})=3$.}
\end{ex}
\begin{ex}%[2D3G2-4]
	Cho hàm số $f(x)$ có đạo hàm và liên tục trên $[1;4],$ đồng biến trên $[1;4],$ thoản mãn $x+2xf(x)=[f'(x)]^2$ với mọi $x\in[1;4]$. Biết rằng $f(1)=\dfrac{3}{2},$ tính tích phân $I=\displaystyle\int\limits_1^4 f(x)\mathrm{\,d}x$. 
	\choice
	{$I=\dfrac{1186}{45}$}
	{$I=\dfrac{1187}{45}$}
	{\True $I=\dfrac{1188}{45}$}
	{$I=\dfrac{9}{2}$}
	\loigiai{
		Do $f(x)$ đồng biến trên $[1;4]$ nên $f'(x)\geq 0,\forall x\in[1;4]$.\\
		Từ giả thiết ta có 
		\begin{eqnarray*}
			x[1+2f(x)]=[f'(x)]^2&\Rightarrow& f'(x)=\sqrt{x}\cdot\sqrt{1+2f(x)},\forall x\in[1;4]\\
			&\Rightarrow& \dfrac{2f'(x)}{2\sqrt{1+2f(x)}}=\sqrt{x}.
		\end{eqnarray*}
		Suy ra \[\displaystyle\int\dfrac{2f'(x)}{2\sqrt{1+2f(x)}}\mathrm{\,d}x=\displaystyle\int\sqrt{x}\mathrm{\,d}x\Leftrightarrow\sqrt{1+2f(x)}=\dfrac{2}{3}x\sqrt{x}+C.\]
		Mà $f(1)=\dfrac{3}{2}\Rightarrow C=\dfrac{4}{3}\Rightarrow  f(x)=\dfrac{\left(\dfrac{2}{3}x\sqrt{x}+\dfrac{4}{3}\right)^2-1}{2}=\dfrac{2}{9}x^3+\dfrac{8}{9}x\sqrt{x}+\dfrac{7}{18}$.\\
		Vậy $I=\displaystyle\int\limits_1^4 f(x)\mathrm{\,d}x=\dfrac{1186}{45}$.}
\end{ex}
\begin{ex}%[2D3G2-4]
	Cho hàm số $f(x)$ liên tục, không âm trên $\left[0;\dfrac{\pi}{2}\right]$, thỏa $f(x)\cdot f'(x)=\cos x\sqrt{1+f^2(x)}$ với mọi $x\in\left[0;\dfrac{\pi}{2}\right]$ và $f(0)=\sqrt{3}$. Giá trị của $f\left(\dfrac{\pi}{2}\right)$ bằng
	\choice
	{$0$}
	{$1$}
	{$2$}
	{\True $2\sqrt{2}$}
	\loigiai{
		Từ giả thiết ta có $\dfrac{2f(x)\cdot f'(x)}{2\sqrt{1+f^2(x)}}=\cos x,\forall x\in\left[0;\dfrac{\pi}{2}\right]$. Suy ra
		\[\displaystyle\int\dfrac{2f(x)\cdot f'(x)}{2\sqrt{1+f^2(x)}}\mathrm{\,d}x=\displaystyle\int\cos x\mathrm{\,d}x\Leftrightarrow\sqrt{1+f^2(x)}=\sin x+C.\]
		Mà $f(0)=\sqrt{3}\Rightarrow C=2\Rightarrow  f(x)=\sqrt{(\sin x+2)^2-1}=\sqrt{\sin^2x+4\sin x+3},\forall x\in\left[0;\dfrac{\pi}{2}\right]$.\\
		Do đó $f\left(\dfrac{\pi}{2}\right)=2\sqrt{2}$.}
\end{ex}
\begin{ex}%[2D3G2-4]
	Cho hàm số $f(x)$ liên tục, không âm trên $[0;3],$ thỏa $f(x)\cdot f'(x)=2x\sqrt{f^2(x)+1}$ với mọi $x\in[0;3]$ và $f(0)=0$. Giá trị của $f(3)$ bằng
	\choice
	{$0$}
	{$1$}
	{$\sqrt{3}$}
	{\True $3\sqrt{11}$}
	\loigiai{
		Từ giả thiết ta có $\dfrac{2f(x)\cdot f'(x)}{2\sqrt{1+f^2(x)}}=2x,\forall x\in[0;3]$. Suy ra
		\[\Rightarrow  \displaystyle\int\dfrac{2f(x)\cdot f'(x)}{2\sqrt{1+f^2(x)}}\mathrm{\,d}x=\displaystyle\int2x\mathrm{\,d}x\Leftrightarrow\sqrt{1+f^2(x)}=x^2+C.\]
		Mà $f(0)=0\Rightarrow C=1\Rightarrow  f(x)=\sqrt{\left(x^2+1\right)^2-1}=\sqrt{x^4+2x^2},\forall x\in[0;3]$.\\
		Do đó $f(3)=3\sqrt{11}$.}
\end{ex}
\begin{ex}%[2D3G2-4]
	Cho hàm số $f(x)$ có đạo hàm không âm trên $[0;1],$ thỏa mãn $f(x)>0$ với mọi $x\in[0;1]$ và $[f(x)]^4\cdot [f'(x)]^2\cdot\left(x^2+1\right)=1+[f(x)]^3$. Biết $f(0)=2,$ hãy chọn khẳng định đúng trong các khẳng định sau đây. 
	\choice
	{$\dfrac{3}{2}<f(1)<2$}
	{$2<f(1)<\dfrac{5}{2}$}
	{\True $\dfrac{5}{2}<f(1)<3$}
	{$3<f(1)<\dfrac{7}{2}$}
	\loigiai{
		Từ giả thiết ta có 
		\[[f(x)]^2\cdot f'(x)\cdot\sqrt{x^2+1}=\sqrt{1+[f(x)]^3}\Leftrightarrow\dfrac{[f(x)]^2\cdot f'(x)}{\sqrt{1+[f(x)]^3}}=\dfrac{1}{\sqrt{x^2+1}}.\]
		Do đó
		\begin{eqnarray*}
			\displaystyle\int\limits_0^1\dfrac{[f(x)]^2\cdot f'(x)}{\sqrt{1+[f(x)]^3}}\mathrm{\,d}x=\displaystyle\int\limits_0^1\dfrac{1}{\sqrt{x^2+1}}\mathrm{\,d}x&\Leftrightarrow&\dfrac{2}{3}\displaystyle\int\limits_0^1\dfrac{\mathrm{d}\left(1+[f(x)]^3\right)}{2\sqrt{1+[f(x)]^3}}=\displaystyle\int\limits_0^1\dfrac{1}{\sqrt{x^2+1}}\mathrm{\,d}x \\
			&\Leftrightarrow& \dfrac{2}{3}\sqrt{1+[f(x)]^3}\bigg|_0^1 =\ln\left(x+\sqrt{x^2+1}\right)\bigg|_0^1\\
			&\Rightarrow& f(1)\approx 2{,}604.
		\end{eqnarray*}
	}
\end{ex}
\begin{ex}%[2D3G2-4]
	Cho hàm số $f(x)$ liên tục trên $\mathbb{R}\setminus\{0;-1\},$ thỏa mãn $x(x+1)\cdot f'(x)+f(x)=x^2+x$ với mọi $x\in\mathbb{R}\setminus\{0;-1\}$ và $f(1)=-2\ln 2$. Biết $f(2)=a+b\ln 3$ với $a, b\in\mathbb{Q}$, tính $P=a^2+b^2$. 
	\choice
	{\True $P=\dfrac{9}{2}$}
	{$P=\dfrac{1}{2}$}
	{$P=\dfrac{3}{4}$}
	{$P=\dfrac{13}{4}$}
	\loigiai{
		Từ giả thiết ta có $\dfrac{x}{x+1}f'(x)+\dfrac{1}{(x+1)^2}f(x)=\dfrac{x}{x+1},\forall x\in\mathbb{R}\setminus\{0;-1\}$.\\
		Nhận thấy $\dfrac{x}{x+1}f'(x)+\dfrac{1}{(x+1)^2}f(x)=\left[f(x)\cdot\dfrac{x}{x+1}\right]'$. Do đó giả thiết tương đương với
		\[\left[f(x)\cdot\dfrac{x}{x+1}\right]'=\dfrac{x}{x+1},\forall x\in\mathbb{R}\setminus\{0;-1\}.\]
		Suy ra \[f(x)\cdot\dfrac{x}{x+1}=\displaystyle\int\dfrac{x}{x+1}\mathrm{\,d}x=\displaystyle\int\left(1-\dfrac{1}{x+1}\right)\mathrm{\,d}x=x-\ln|x+1|+C.\]
		Mà $f(1)=-2\ln 2\Rightarrow C=-1\Rightarrow  f(x)\cdot\dfrac{x}{x+1}=x-\ln|x+1|-1$.\\
		Cho $x=2$ ta được $f(2)\cdot\dfrac{2}{3}=2-\ln 3-1\Rightarrow f(2)=\dfrac{3}{2}-\dfrac{3}{2}\ln 3$.\\
		Vậy $a=\dfrac{3}{2}$, $b=-\dfrac{3}{2}$. Do đó $P=\dfrac{9}{2}$.
	}
\end{ex}
\begin{ex}%[2D3G2-4]
	Cho hàm số $f(x)$ có đạo hàm xác định, liên tục trên $[0;1],$ thỏa mãn $f'(0)=-1$ và $\heva{&[f'(x)]^2=f''(x)\\&f'(x)\neq 0}$ với mọi $x\in[0;1]$. Đặt $P=f(1)-f(0)$, khẳng định nào sau đây đúng?
	\choice
	{$-2\leq P\leq-1$}
	{\True $-1\leq P\leq 0$}
	{$0\leq P\leq 1$}
	{$1\leq P\leq 2$}
	\loigiai{
		Nhận thấy $P=f(1)-f(0)=\displaystyle\int\limits_0^1 f'(x)\mathrm{\,d}x$ nên ta cần tìm $f'(x)$.\\
		Từ giả thiết ta có \[\dfrac{f''(x)}{[f'(x)]^2}=1\Rightarrow  \displaystyle\int\dfrac{f''(x)}{[f'(x)]^2}\mathrm{\,d}x=\displaystyle\int 1\mathrm{\,d}x\Leftrightarrow-\dfrac{1}{f'(x)}=x+C\Leftrightarrow f'(x)=-\dfrac{1}{x+C}.\]
		Mà $f'(0)=-1\Rightarrow C=1\Rightarrow  f'(x)=-\dfrac{1}{x+1}$.\\
		Vậy $P=\displaystyle\int\limits_0^1 f'(x)\mathrm{\,d}x=-\displaystyle\int\limits_0^1\dfrac{1}{x+1}\mathrm{\,d}x=-\ln 2\approx-0{,}69$.}
\end{ex}
\begin{ex}%[2D3G2-4]
	Cho hai hàm số $f(x)$ và $g(x)$ có đạo hàm liên tục trên $[0;2],$ thỏa mãn $f'(0)\cdot f'(2)\neq 0$ và $g(x)\cdot f'(x)=x(x-2)\mathrm{e}^x$. Tính tích phân $I=\displaystyle\int\limits_0^2 f(x)\cdot g'(x)\mathrm{\,d}x$. 
	\choice
	{$I=-4$}
	{\True $I=4$}
	{$I=e-2$}
	{$I=2-e$}
	\loigiai{
		Từ giả thiết $f'(0)\cdot f'(2)\neq 0\Rightarrow  \heva{&f'(0)\neq 0\\&f'(2)\neq 0.}$ \\
		Do đó từ $g(x)\cdot f'(x)=x(x-2)\mathrm{e}^x$, suy ra $\heva{&g(2)=\dfrac{2(2-2)\mathrm{e}^x}{f'(2)}=0\\&g(0)=\dfrac{0(0-2)\mathrm{e}^x}{f'(0)}=0.}$ \\
		Tích phân từng phần ta được 
		\begin{eqnarray*}
			I&=&[f(x)\cdot g(x)]\bigg|_0^2 -\displaystyle\int\limits_0^2 g(x)\cdot f'(x)\mathrm{\,d}x\\
			&=&f(2)\cdot g(2)-f(0)\cdot g(0)-\displaystyle\int\limits_0^2 x(x-2)\mathrm{e}^x\mathrm{\,d}x\\
			&=&-\displaystyle\int\limits_0^2 x(x-2)\mathrm{e}^x\mathrm{\,d}x=4.
		\end{eqnarray*}
	}
\end{ex}
\begin{ex}%[2D3G2-4] %Câu 61.
	Cho hàm số $f(x)>0$ xác định và có đạo hàm trên đoạn $[0;1],$ thỏa mãn $\heva{&g(x)=1+2018\displaystyle\int\limits_0^x f(t)\mathrm{\,d}t\\&g(x)=f^2(x)}$. Tính $I=\displaystyle\int\limits_0^1\sqrt{g(x)}\mathrm{\,d}x$. 
	\choice
	{$I=\dfrac{1009}{2}$}
	{$I=505$}
	{\True $I=\dfrac{1011}{2}$}
	{$I=\dfrac{2019}{2}$}
	\loigiai{Từ giả thiết, ta có 
		\begin{eqnarray*}
			&  \heva{&g'(x)=2018f(x)\\&g'(x)=2f'(x)\cdot f(x)} \\
			\Rightarrow		&  2f(x)[1009-f'(x)]=0\\
			\Leftrightarrow &  \hoac{&f(x)=0\text{ (loại)}\\&f'(x)=1009}\\
			\Rightarrow     &  f(x)=1009x+C.
		\end{eqnarray*}
		Thay ngược lại, ta được 
		\begin{eqnarray*}
			& 1+2018\displaystyle\int\limits_0^x[1009t+C]\mathrm{\,d}t=(1009x+C)^2 \\
			\Leftrightarrow & 1+2018\left(\dfrac{1009}{2}t^2+Ct\right)\bigg|_0^x=(1009x+C)^2\\ 
			\Leftrightarrow & C^2=1
		\end{eqnarray*}
		
		Suy ra $f(x)=1009x+1$ hoặc $f(x)=1009x-1$ (loại vì $f(x)>0\forall x\in[0;1]$).\\
		Khi đó $I=\displaystyle\int\limits_0^1\sqrt{g(x)}\mathrm{\,d}x=\displaystyle\int\limits_0^1 f(x)\mathrm{\,d}x=\displaystyle\int\limits_0^1(1009x+1)\mathrm{\,d}x=\dfrac{1011}{2}$.}
\end{ex}
\begin{ex}%[2D3G2-4] %Câu 62.
	Cho hai hàm $f(x)$ và $g(x)$ có đạo hàm trên $[1;4],$ thỏa mãn $\heva{&f(1)+g(1)=4\\&g(x)=-xf'(x)\\&f(x)=-xg'(x)}$ với mọi $x\in[1;4]$. Tính tích phân $I=\displaystyle\int\limits_1^4[f(x)+g(x)]\mathrm{\,d}x$. 
	\choice
	{$I=3\ln 2$}
	{$I=4\ln 2$}
	{$I=6\ln 2$}
	{\True $I=8\ln 2$}
	\loigiai{
		Từ giả thiết ta có:
		\begin{eqnarray*}
			& f(x)+g(x)=-x\cdot f'(x)-x\cdot g'(x)\\
			\Leftrightarrow & \left[f(x)+x\cdot f'(x)\right]+\left[g(x)+x\cdot g'(x)\right]=0 \\
			\Leftrightarrow & [x\cdot f(x)]'+[x\cdot g(x)]'=0 \\
			\Rightarrow     & x\cdot f(x)+x\cdot g(x)=C\\
			\Rightarrow 	& f(x)+g(x)=\dfrac{C}{x}.
		\end{eqnarray*}
		
		Mà $f(1)+g(1)=4\Rightarrow C=4\Rightarrow  I=\displaystyle\int\limits_1^4[f(x)+g(x)]\mathrm{\,d}x=\displaystyle\int\limits_1^4\dfrac{4}{x}\mathrm{\,d}x=8\ln 2$.}
\end{ex}
\begin{ex}%[2D3G2-4] %Câu 63.
	Cho hai hàm $f(x)$ và $g(x)$ có đạo hàm trên $[1;2],$ thỏa mãn $f(1)=g(1)=0$ và.\\
	$\heva{&\dfrac{x}{(x+1)^2}g(x)+2017x=(x+1)f'(x)\\&\dfrac{x^3}{x+1}g'(x)+f(x)=2018x^2},\forall x\in[1;2]$.\\
	Tính tích phân $I=\displaystyle\int\limits_1^2\left[\dfrac{x}{x+1}g(x)-\dfrac{x+1}{x}f(x)\right]\mathrm{\,d}x$. 
	\choice
	{\True $I=\dfrac{1}{2}$}
	{$I=1$}
	{$I=\dfrac{3}{2}$}
	{$I=2$}
	\loigiai{Từ giả thiết ta có $\heva{&\dfrac{1}{(x+1)^2}g(x)-\dfrac{(x+1)}{x}f'(x)=-2017\\&\dfrac{x}{x+1}g'(x)+\dfrac{1}{x^2}f(x)=2018},\forall x\in[1;2]$.\\
		Suy ra
		\begin{eqnarray*}
			& \left[\dfrac{1}{(x+1)^2}g(x)+\dfrac{x}{x+1}g'(x)\right]-\left[\dfrac{(x+1)}{x}f'(x)-\dfrac{1}{x^2}f(x)\right]=1 \\
			\Leftrightarrow & \left[\dfrac{x}{x+1}g(x)\right]'-\left[\dfrac{(x+1)}{x}f(x)\right]'=1\\
			\Rightarrow     & \dfrac{x}{x+1}g(x)-\dfrac{(x+1)}{x}f(x)=x+C.
		\end{eqnarray*}
		
		Mà $f(1)=g(1)=0\Rightarrow C=-1\Rightarrow  I=\displaystyle\int\limits_1^2\left[\dfrac{x}{x+1}g(x)-\dfrac{x+1}{x}f(x)\right]\mathrm{\,d}x=\displaystyle\int\limits_1^2(x-1)\mathrm{\,d}x=\dfrac{1}{2}$.}
\end{ex}
\begin{ex}%[2D3G2-4] %Câu 64.
	Cho hàm số $y=f(x)$ có đạo hàm trên $[0;3],$ thỏa mãn $\heva{&f(3-x)\cdot f(x)=1\\&f(x)\neq-1}$ với mọi $x\in[0;3]$ và $f(0)=\dfrac{1}{2}$. Tính tích phân $I=\displaystyle\int\limits_0^3\dfrac{xf'(x)}{[1+f(3-x)]^2\cdot f^2(x)}\mathrm{\,d}x$. 
	\choice
	{\True $I=\dfrac{1}{2}$}
	{$I=1$}
	{$I=\dfrac{3}{2}$}
	{$I=\dfrac{5}{2}$}
	\loigiai{
		Từ giả thiết $\heva{&f(3-x)\cdot f(x)=1\\&f(0)=\dfrac{1}{2}}\xrightarrow  {x=3}f(3)=2$.\\
		Ta có $[1+f(3-x)]^2\cdot f^2(x)\overset{f(3-x)\cdot f(x)=1}{=}[1+f(x)]^2$.\\
		Tích phân
		\begin{eqnarray*}
			I & = & \displaystyle\int\limits_0^3\dfrac{xf'(x)}{[1+f(x)]^2}\mathrm{\,d}x \\
			& = & -\displaystyle\int\limits_0^3 x\mathrm{d}\left(\dfrac{1}{1+f(x)}\right)\\
			& = & -\dfrac{x}{1+f(x)}\bigg|_0^3 +\displaystyle\int\limits_0^3\dfrac{1}{1+f(x)}\mathrm{\,d} x\\
			& = & -1+J.
		\end{eqnarray*}
		Tính 
		\begin{eqnarray*}
			J & = & \displaystyle\int\limits_0^3\dfrac{1}{1+f(x)}\mathrm{\,d} x \\
			& \overset{t=3-x}{=} & -\displaystyle\int\limits_3^0\dfrac{1}{1+f(3-t)}\mathrm{\,d} t\\
			& = & \displaystyle\int\limits_0^3\dfrac{1}{1+f(3-t)}\mathrm{\,d}t\\
			& = & \displaystyle\int\limits_0^3\dfrac{1}{1+f(3-x)}\mathrm{\,d} x.
		\end{eqnarray*} 
		Suy ra
		\begin{eqnarray*}
			2J & = & \displaystyle\int\limits_0^3\dfrac{1}{1+f(x)}\mathrm{\,d} x+\displaystyle\int\limits_0^3\dfrac{1}{1+f(3-x)}\mathrm{\,d} x \\
			& \overset{f(3-x)\cdot f(x)=1}{=} & \displaystyle\int\limits_0^3 1\cdot \mathrm{\,d} x\\
			& = & 3.
		\end{eqnarray*}
		$\Rightarrow J=\dfrac{3}{2}$. Vậy $I=\dfrac{1}{2}$.}
\end{ex}
\begin{ex}%[2D3G2-4] %Câu 65.
	Cho hàm số $y=f(x)$ liên tục trên đoạn $[0;1]$ và thỏa mãn $af(b)+bf(a)=1$ với mọi $a, b\in[0;1]$. Tính tích phân $I=\displaystyle\int\limits_0^1 f(x)\mathrm{\,d}x$. 
	\choice
	{$I=\dfrac{1}{2}$}
	{$I=\dfrac{1}{4}$}
	{$I=\dfrac{\pi}{2}$}
	{\True $I=\dfrac{\pi}{4}$}
	\loigiai{
		Đặt $a=\sin x, b=\cos x$ với $x\in\left[0;\dfrac{\pi}{2}\right]$.\\
		Từ giả thiết, suy ra $\sin xf(\cos x)+\cos xf(\sin x)=1$.\\
		$\Rightarrow \displaystyle\int\limits_0^{\tfrac{\pi}{2}}\sin xf(\cos x)\mathrm{\,d}x +\displaystyle\int\limits_0^{\tfrac{\pi}{2}}\cos xf(\sin x)\mathrm{\,d}x =\displaystyle\int\limits_0^{\tfrac{\pi}{2}} 1\mathrm{\,d}x=\dfrac{\pi}{2}. \quad(1)$\\
		Ta có $\heva{&\displaystyle\int\limits_0^{\tfrac{\pi}{2}}\sin xf(\cos x)\mathrm{\,d}x\overset{t=\cos x}{=}-\displaystyle\int\limits_1^0 f(t)\mathrm{\,d}t =\displaystyle\int\limits_0^1 f(x)\mathrm{\,d}x\\&\displaystyle\int\limits_0^{\tfrac{\pi}{2}}\cos xf(\sin x)\mathrm{\,d}x
			\overset{t=\sin x}{=}\displaystyle\int\limits_0^1 f(t)\mathrm{\,d}t
			=\displaystyle\int\limits_0^1 f(x)\mathrm{\,d}x}$. 
		Do đó $(1)\Leftrightarrow\displaystyle\int\limits_0^1 f(x)\mathrm{\,d}x=\dfrac{\pi}{4}$.}
\end{ex}
\paragraph{Vấn đề 9. Kỹ thuật đạo hàm đúng.}
\begin{ex}%[2D3G2-4] %Câu 66.
	Cho hàm số $f(x)$ có đạo hàm liên tục trên $[0; 1],$ thoả mãn $3f(x)+xf'(x)=x^{2018}$ với mọi $x\in[0; 1]$. Tính $I=\displaystyle\int\limits_0^1 f(x)\mathrm{\,d}x$. 
	\choice
	{$I=\dfrac{1}{2018\times 2021}$}
	{$I=\dfrac{1}{2019\times 2020}$}
	{\True $I=\dfrac{1}{2019\times 2021}$}
	{$I=\dfrac{1}{2018\times 2019}$}
	\loigiai{
		Từ giả thiết $3f(x)+xf'(x)=x^{2018},$ nhân hai vế cho $x^2$ ta được:
		\begin{eqnarray*}
			& 3x^2f(x)+x^3f'(x)=x^{2020} \\
			\Leftrightarrow  & \left[x^3f(x)\right]'=x^{2020}
		\end{eqnarray*}
		
		Suy ra $x^3f(x)=\displaystyle\int x^{2020}\mathrm{\,d}x=\dfrac{x^{2021}}{2021}+C$.\\
		Thay $x=0$ vào hai vế ta được $C=0\Rightarrow  f(x)=\dfrac{x^{2018}}{2021}$.\\
		Vậy $\displaystyle\int\limits_0^1 f(x)\mathrm{\,d}x=\displaystyle\int\limits_0^1\dfrac{1}{2021}x^{2018}\mathrm{\,d}x
		=\dfrac{1}{2021}\cdot\dfrac{1}{2019}x^{2019}\bigg|_0^1=\dfrac{1}{2021\times 2019}$.\\
		Nhận xét: Ý tưởng nhân hai vế cho $x^2$ là để thu được đạo hàm đúng dạng $(uv)'=u'v+uv'$.}
\end{ex}
\begin{ex}%[2D3G2-4] %Câu 67.
	Cho hàm số $f(x)$ có đạo hàm liên tục trên $[0;4],$ thỏa mãn $f(x)+f'(x)=\mathrm{e}^{-x}\sqrt{2x+1}$ với mọi $x\in[0; 4]$. Khẳng định nào sau đây là đúng?
	\choice
	{\True $\mathrm{e}^4f(4)-f(0)=\dfrac{26}{3}$}
	{$\mathrm{e}^4f(4)-f(0)=3e$}
	{$\mathrm{e}^4f(4)-f(0)=\mathrm{e}^4-1$}
	{$\mathrm{e}^4f(4)-f(0)=3$}
	\loigiai{Nhân hai vế cho $\mathrm{e}^x$ để thu được đạo hàm đúng, ta được.
		\begin{eqnarray*}
			& \mathrm{e}^xf(x)+\mathrm{e}^xf'(x)=\sqrt{2x+1} \\
			\Leftrightarrow	& \left[\mathrm{e}^xf(x)\right]^/=\sqrt{2x+1}.
		\end{eqnarray*}
		Suy ra $\mathrm{e}^xf(x)=\displaystyle\int\sqrt{2x+1}\mathrm{\,d}x=\dfrac{1}{3}(2x+1)\sqrt{2x+1}+C$.\\
		Vậy $\mathrm{e}^4f(4)-f(0)=\dfrac{26}{3}$.}
\end{ex}
\begin{ex}%[2D3G2-4] %Câu 68.
	Cho hàm số $f(x)$ có đạo hàm trên $\mathbb{R},$ thỏa mãn $f'(x)-2018f(x)=2018x^{2017}\mathrm{e}^{2018x}$ với mọi $x\in\mathbb{R}$ và $f(0)=2018$. Tính giá trị $f(1)$. 
	\choice
	{$f(1)=2018\mathrm{e}^{-2018}$}
	{$f(1)=2017\mathrm{e}^{2018}$}
	{$f(1)=2018\mathrm{e}^{2018}$}
	{\True $f(1)=2019\mathrm{e}^{2018}$}
	\loigiai{Nhân hai vế cho $\mathrm{e}^{-2018x}$ để thu được đạo hàm đúng, ta được.
		\begin{eqnarray*}
			& f'(x)\mathrm{e}^{-2018x}-2018f(x)\mathrm{e}^{-2018x}=2018x^{2017} \\
			\Leftrightarrow	 & \left[f(x)\mathrm{e}^{-2018x}\right]'=2018x^{2017}
		\end{eqnarray*}
		
		Suy ra $f(x)\mathrm{e}^{-2018x}=\displaystyle\int 2018x^{2017}\mathrm{\,d}x=x^{2018}+C$.\\
		Thay $x=0$ vào hai vế ta được $C=2018\Rightarrow  f(x)=\left(x^{2018}+2018\right)\mathrm{e}^{2018x}$.\\
		Vậy $f(1)=2019\mathrm{e}^{2018}$.}
\end{ex}
\begin{ex}%[2D3G2-4] %Câu 69.
	Cho hàm số $f(x)$ có đạo hàm và liên tục trên $\mathbb{R},$ thỏa mãn $f'(x)+xf(x)=2x\mathrm{e}^{-x^2}$ và $f(0)=-2$. Tính $f(1)$. 
	\choice
	{$f(1)=e$}
	{$f(1)=\dfrac{1}{e}$}
	{$f(1)=\dfrac{2}{e}$}
	{\True $f(1)=-\dfrac{2}{e}$}
	\loigiai{Nhân hai vế cho $\mathrm{e}^{\tfrac{x^2}{2}}$ để thu được đạo hàm đúng, ta được.
		\begin{eqnarray*}
			& f'(x)\mathrm{e}^{\tfrac{x^2}{2}}+f(x)x\mathrm{e}^{\tfrac{x^2}{2}}=2x\mathrm{e}^{-\tfrac{x^2}{2}} \\
			\Leftrightarrow  & \left[\mathrm{e}^{\tfrac{x^2}{2}}f(x)\right]'=2x\mathrm{e}^{-\tfrac{x^2}{2}}
		\end{eqnarray*}
		Suy ra $\mathrm{e}^{\tfrac{x^2}{2}}f(x)=\displaystyle\int 2x\mathrm{e}^{-\tfrac{x^2}{2}}\mathrm{\,d}x=-2\mathrm{e}^{-\tfrac{x^2}{2}}+C$.\\
		Thay $x=0$ vào hai vế ta được $C=0\Rightarrow  f(x)=-2\mathrm{e}^{-x^2}$.\\
		Vậy $f(1)=-2\mathrm{e}^{-1}=-\dfrac{2}{e}$.}
\end{ex}
\begin{ex}%[2D3G2-4] %Câu 70.
	Cho hàm số $f(x)$ liên tục và có đạo hàm trên $\left(0;\dfrac{\pi}{2}\right),$ thỏa mãn hệ thức $f(x)+\tan xf'(x)=\dfrac{x}{\cos^3x}$. Biết rằng $\sqrt{3}f\left(\dfrac{\pi}{3}\right)-f\left(\dfrac{\pi}{6}\right)=a\pi\sqrt{3}+b\ln 3$ trong đó $a, b\in\mathbb{Q}$. Tính giá trị của biểu thức $P=a+b$. 
	\choice
	{\True $P=-\dfrac{4}{9}$}
	{$P=-\dfrac{2}{9}$}
	{$P=\dfrac{7}{9}$}
	{$P=\dfrac{14}{9}$}
	\loigiai{
		Từ giả thiết, ta có 
		\begin{eqnarray*}
			& \cos xf(x)+\sin xf'(x)=\dfrac{x}{\cos^2x} \\
			\Leftrightarrow	& [\sin xf(x)]'=\dfrac{x}{\cos^2x}
		\end{eqnarray*}
		Suy ra $\sin xf(x)=\displaystyle\int\dfrac{x}{\cos^2x}\mathrm{\,d}x=x\tan x+\ln|\cos x|+C$.\\
		Với $x=\dfrac{\pi}{3}\Rightarrow  \dfrac{\sqrt{3}}{2}f\left(\dfrac{\pi}{3}\right)=\dfrac{\pi}{3}\cdot\sqrt{3}-\ln 2\Rightarrow  \sqrt{3}f\left(\dfrac{\pi}{3}\right)=\dfrac{2}{3}\cdot\pi\sqrt{3}-2\ln 2+2C$.\\
		Với $x=\dfrac{\pi}{6}\Rightarrow  \dfrac{1}{2}f\left(\dfrac{\pi}{6}\right)=\dfrac{\pi}{6}\cdot\dfrac{\sqrt{3}}{3}+\dfrac{1}{2}\ln 3-\ln 2+C\Rightarrow  f\left(\dfrac{\pi}{6}\right)=\dfrac{1}{9}\cdot\pi\sqrt{3}+\ln 3-2\ln 2+2C$.\\
		Suy ra $\sqrt{3}f\left(\dfrac{\pi}{3}\right)-f\left(\dfrac{\pi}{6}\right)=\dfrac{5}{9}\pi\sqrt{3}-\ln 3\Rightarrow  \heva{&a=\dfrac{5}{9}\\&b=-1}\Rightarrow  P=a+b=-\dfrac{4}{9.}$}
\end{ex}
\paragraph{Vấn đề 10. Kỹ thuật đưa về bình phương loại 1.}
\begin{ex}%[2D3G2-4] %Câu 71.
	Cho hàm số $f(x)$ liên tục trên $\left[0;\dfrac{\pi}{2}\right],$ thỏa $\displaystyle\int\limits_0^{\tfrac{\pi}{2}}\left[f^2(x)-2\sqrt{2}f(x)\sin\left(x-\dfrac{\pi}{4}\right)\right]\mathrm{\,d}x=\dfrac{2-\pi}{2}$. Tính tích phân $I=\displaystyle\int\limits_0^{\tfrac{\pi}{2}} f(x)\mathrm{\,d}x$. 
	\choice
	{\True $I=0$}
	{$I=\dfrac{\pi}{4}$}
	{$I=1$}
	{$I=\dfrac{\pi}{2}$}
	\loigiai{
		Ta có $\displaystyle\int\limits_0^{\tfrac{\pi}{2}} 2\sin^2\left(x-\dfrac{\pi}{4}\right)\mathrm{\,d}x=-\dfrac{2-\pi}{2}$.\\
		Do đó giả thiết tương đương với
		\begin{eqnarray*}
			&\displaystyle\int\limits_0^{\tfrac{\pi}{2}}\left[f^2(x)-2\sqrt{2}f(x)\sin\left(x-\dfrac{\pi}{4}\right)+2\sin^2\left(x-\dfrac{\pi}{4}\right)\right]\mathrm{\,d}x=0\\
			\Leftrightarrow &\displaystyle\int\limits_0^{\tfrac{\pi}{2}}\left[f(x)-\sqrt{2}\sin\left(x-\dfrac{\pi}{4}\right)\right]^2\mathrm{\,d}x=0\\
			\Leftrightarrow & f(x)-\sqrt{2}\sin\left(x-\dfrac{\pi}{4}\right)=0,\forall x\in\left[0;\dfrac{\pi}{2}\right]\\ 
		\end{eqnarray*}
		Suy ra $f(x)=\sqrt{2}\sin\left(x-\dfrac{\pi}{4}\right)\Rightarrow  I=\displaystyle\int\limits_0^{\tfrac{\pi}{2}} f(x)\mathrm{\,d}x=\sqrt{2}\displaystyle\int\limits_0^{\tfrac{\pi}{2}}\sin\left(x-\dfrac{\pi}{4}\right)\mathrm{\,d}x=0$.}
\end{ex}
\begin{ex}%[2D3G2-4] %Câu 72.
	Cho hàm số $f(x)$ liên tục trên $[0;1]$ thỏa $\displaystyle\int\limits_0^1\left[f^2(x)+2\ln^2\dfrac{2}{e}\right]\mathrm{\,d}x=2\displaystyle\int\limits_0^1\left[f(x)\ln(x+1)\right]\mathrm{\,d}x$. Tích phân $I=\displaystyle\int\limits_0^1 f(x)\mathrm{\,d}x$. 
	\choice
	{$I=\ln\dfrac{e}{4}$}
	{\True $I=\ln\dfrac{4}{e}$}
	{$I=\ln\dfrac{e}{2}$}
	{$I=\ln\dfrac{2}{e}$}
	\loigiai{Bằng phương pháp tích phân từng phần ta tính được.\\
		$\displaystyle\int\limits_0^1\ln^2(x+1)\mathrm{\,d}x=2\ln^2\dfrac{2}{e}=\displaystyle\int\limits_0^1 2\ln^2\dfrac{2}{e}\mathrm{\,d}x$.\\
		Do đó giả thiết tương đương với $\displaystyle\int\limits_0^1\left[f(x)-\ln(1+x)\right]^2\mathrm{\,d}x=0\Leftrightarrow f(x)\equiv\ln(1+x),\forall x\in[0;1]$.\\
		Suy ra $\displaystyle\int\limits_0^1 f(x)\mathrm{\,d}x=\displaystyle\int\limits_0^1\ln(1+x)\mathrm{\,d}x=\ln\dfrac{4}{e}$.}
\end{ex}
\begin{ex}%[2D3G2-4] %Câu 73.
	Cho hàm số $f(x)$ có đạo liên tục trên $[0;1], f(x)$ và $f'(x)$ đều nhận giá trị dương trên $[0;1]$ và thỏa mãn $f(0)=2$ và $\displaystyle\int\limits_0^1\left[f'(x)\cdot [f(x)]^2+1\right]\mathrm{\,d}x=2\displaystyle\int\limits_0^1\sqrt{f'(x)}\cdot f(x)\mathrm{\,d}x$ Tính $I=\displaystyle\int\limits_0^1[f(x)]^3\mathrm{\,d}x$ 
	\choice
	{$I=\dfrac{15}{4}$}
	{$I=\dfrac{15}{2}$}
	{$I=\dfrac{17}{2}$}
	{\True $I=\dfrac{19}{2}$}
	\loigiai{Giả thiết tương đương với $\displaystyle\int\limits_0^1\left[\sqrt{f'(x)}\cdot f(x)-1\right]^2\mathrm{\,d}x=0$.
		\begin{eqnarray*}
			\Rightarrow & \sqrt{f'(x)}\cdot f(x)=1,\forall x\in[0;1] \\
			\Rightarrow & f'(x)f^2(x)=1\\
			\Rightarrow & \displaystyle\int f'(x)f^2(x)\mathrm{\,d}x=\displaystyle\int\mathrm{\,d}x
		\end{eqnarray*}
		$\Rightarrow  \dfrac{f^3(x)}{3}=x+C\Rightarrow  {{f(0)=2}}C=\dfrac{8}{3}$.\\
		Vậy $f^3(x)=3x+8\Rightarrow  I=\displaystyle\int\limits_0^1[f(x)]^3\mathrm{\,d}x=\dfrac{19}{2}$.}
\end{ex}
\begin{ex}%[2D3G2-4] %Câu 74.
	Cho hàm số $f(x)$ có đạo hàm dương, liên tục trên đoạn $[0;1]$ và thỏa mãn $f(0)=1$ đồng thời $3\displaystyle\int\limits_0^1\left[f'(x)\cdot [f(x)]^2+\dfrac{1}{9}\right]\mathrm{\,d}x=2\displaystyle\int\limits_0^1\sqrt{f'(x)}\cdot f(x)\mathrm{\,d}x$. Tính $I=\displaystyle\int\limits_0^1[f(x)]^3\mathrm{\,d}x$ 
	\choice
	{$I=\dfrac{3}{2}$}
	{$I=\dfrac{5}{4}$}
	{$I=\dfrac{5}{6}$}
	{\True $I=\dfrac{7}{6}$}
	\loigiai{Giả thiết ta có:
		\begin{eqnarray*}
			& 3\displaystyle\int\limits_0^1\left[\sqrt{f'(x)}\cdot f(x)\right]^2\mathrm{\,d}x+\dfrac{1}{3}=2\displaystyle\int\limits_0^1\sqrt{f'(x)}\cdot f(x)\mathrm{\,d}x\\
			\Leftrightarrow & \displaystyle\int\limits_0^1\left[3\sqrt{f'(x)}\cdot f(x)\right]^2\mathrm{\,d}x-2\displaystyle\int\limits_0^1 3\sqrt{f'(x)}\cdot f(x)\mathrm{\,d}x+\displaystyle\int\limits_0^1\mathrm{\,d}x=0 \\
			\Leftrightarrow &  \displaystyle\int\limits_0^1\left[3\sqrt{f'(x)}\cdot f(x)-1\right]^2\mathrm{\,d}x=0 \\
			\Rightarrow 	& 3\sqrt{f'(x)}\cdot f(x)-1=0,\forall x\in[0;1]\\
			\Rightarrow     & 9f'(x)\cdot f^2(x)=1\\
			\Rightarrow     & \displaystyle\int 9f'(x)\cdot f^2(x)\mathrm{\,d}x=\displaystyle\int\mathrm{\,d}x\\
			\Rightarrow 	& 9\cdot\dfrac{f^3(x)}{3}=x+C\\
			\Rightarrow     & {{f(0)=1}}C=3
		\end{eqnarray*}
		Vậy $f^3(x)=\dfrac{1}{3}x+1\Rightarrow  \displaystyle\int\limits_0^1[f(x)]^3\mathrm{\,d}x=\dfrac{7}{6}$.}
\end{ex}
\begin{ex}%[2D3G2-4] %Câu 75.
	Cho hàm số $y=f(x)$ có đạo hàm dương, liên tục trên đoạn $[0;1],$ thỏa $f(1)-f(0)=1$ và $\displaystyle\int\limits_0^1 f'(x)\left[f^2(x)+1\right]\mathrm{\,d}x=2\displaystyle\int\limits_0^1\sqrt{f'(x)}f(x)\mathrm{\,d}x$. Giá trị của tích phân $\displaystyle\int\limits_0^1[f(x)]^3\mathrm{\,d}x$ bằng
	\choice
	{$\dfrac{3}{2}$}
	{$\dfrac{5\sqrt{33}-27}{18}$}
	{\True $\dfrac{5\sqrt{33}}{18}$}
	{$\dfrac{5\sqrt{33}+54}{18}$}
	\loigiai{Nhóm hằng đẳng thức ta có 
		\begin{eqnarray*}
			& \displaystyle\int\limits_0^1 f'(x)\left[f^2(x)+1\right]\mathrm{\,d}x=2\displaystyle\int\limits_0^1\sqrt{f'(x)}f(x)\mathrm{\,d}x \\
			\Leftrightarrow &  \displaystyle\int\limits_0^1\left[f'(x)f^2(x)+f'(x)\right]\mathrm{\,d}x-2\displaystyle\int\limits_0^1\sqrt{f'(x)}f(x)\mathrm{\,d}x=0\\
			\Leftrightarrow & \displaystyle\int\limits_0^1\left[\sqrt{f'(x)}f(x)-1\right]^2\mathrm{\,d}x+\underbrace{\displaystyle\int\limits_0^1[f'(x)-1]\mathrm{\,d}x}_{=0 \textrm{ vì }f(1)-f(0)=1}=0\\
			\Rightarrow     & \sqrt{f'(x)}\cdot f(x)=1,\forall x\in[0;1]\\
			\Rightarrow     & f'(x)f^2(x)=1\\
			\Rightarrow     & \displaystyle\int f'(x)f^2(x)\mathrm{\,d}x=\displaystyle\int\mathrm{\,d}x\\
			\Rightarrow	 & \dfrac{f^3(x)}{3}=x+C\\
			\Rightarrow     & f^3(x)=3x+3C\\
			\Rightarrow     & {{f(1)-f(0)=1}}C=\dfrac{5\sqrt{33}-27}{54}.
		\end{eqnarray*}
		Vậy $f^3(x)=3x+\dfrac{5\sqrt{33}-27}{18}\Rightarrow  \displaystyle\int\limits_0^1[f(x)]^3\mathrm{\,d}x=\dfrac{5\sqrt{33}}{18}$.\\
	}
\end{ex}
\paragraph{Vấn đề 11. Kỹ thuật đưa về bình phương loại 2. Kỹ thuật Holder.}
\begin{ex}%[2D3G2-4] %Câu 76.
	Cho hàm số $y=f(x)$ liên tục trên đoạn $[0;1],$ thỏa mãn $\displaystyle\int\limits_0^1 f(x)\mathrm{\,d}x=\displaystyle\int\limits_0^1 xf(x)\mathrm{\,d}x=1$ và $\displaystyle\int\limits_0^1[f(x)]^2\mathrm{\,d}x=4$. Giá trị của tích phân $\displaystyle\int\limits_0^1[f(x)]^3\mathrm{\,d}x$ bằng
	\choice
	{$1$}
	{$8$}
	{\True $10$}
	{$80$}
	\loigiai{Ở đây các hàm xuất hiện dưới dấu tích phân là $[f(x)]^2, xf(x), f(x)$ nên ta sẽ liên kết với bình phương $\left[f(x)+\alpha x+\beta\right]^2$.\\
		Với mỗi số thực $\alpha,\beta$ ta có 
		\begin{eqnarray*}
			\displaystyle\int\limits_0^1\left[f(x)+\alpha x+\beta\right]^2\mathrm{\,d}x & = & \displaystyle\int\limits_0^1[f(x)]^2\mathrm{\,d}x+2\displaystyle\int\limits_0^1\left(\alpha x+\beta\right)f(x)\mathrm{\,d}x+\displaystyle\int\limits_0^1\left(\alpha x+\beta\right)^2\mathrm{\,d}x \\
			& = & 4+2\left(\alpha+\beta\right)+\dfrac{{\alpha}^2}{3}+\alpha\beta+\beta^2.
		\end{eqnarray*}
		Ta cần tìm $\alpha,\beta$ sao cho
		\begin{eqnarray*}
			& \displaystyle\int\limits_0^1\left[f(x)+\alpha x+\beta\right]^2\mathrm{\,d}x=0 \\
			\Leftrightarrow	 & 4+2\left(\alpha+\beta\right)+\dfrac{{\alpha}^2}{3}+\alpha\beta+\beta^2=0\\
			\Leftrightarrow  & \alpha^2+(3\beta+6)\alpha+3\beta^2+6\beta+12=0.
		\end{eqnarray*}
		Để tồn tại $\alpha$ thì 
		\begin{eqnarray*}
			& \Delta=(3\beta+6)^2-4\left(3{\beta}^2+6\beta+12\right)\geq 0 \\
			\Leftrightarrow & -3\beta^2+12\beta-12\geq 0\\
			\Leftrightarrow & -3(\beta-2)^2\geq 0\\
			\Leftrightarrow & \beta=2\Rightarrow  \alpha=-6.
		\end{eqnarray*}
		Vậy $\displaystyle\int\limits_0^1[f(x)-6x+2]^2\mathrm{\,d}x=0\Rightarrow  f(x)=6x-2,\forall x\in[0;1]\Rightarrow  \displaystyle\int\limits_0^1[f(x)]^3\mathrm{\,d}x=10$.}
\end{ex}
\begin{ex}%[2D3G2-4] %Câu 77.
	Cho hàm số $y=f(x)$ liên tục trên đoạn $[0;1],$ thỏa mãn $\displaystyle\int\limits_0^1 xf(x)\mathrm{\,d}x=\displaystyle\int\limits_0^1\sqrt{x}f(x)\mathrm{\,d}x=1$ và $\displaystyle\int\limits_0^1[f(x)]^2\mathrm{\,d}x=5$. Giá trị của tích phân $\displaystyle\int\limits_0^1[f(x)]^3\mathrm{\,d}x$ bằng
	\choice
	{\True $\dfrac{5}{6}$}
	{$\dfrac{6}{5}$}
	{$8$}
	{$10$}
	\loigiai{Ở đây các hàm xuất hiện dưới dấu tích phân là $[f(x)]^2, xf(x),\sqrt{x}f(x)$ nên ta sẽ liên kết với bình phương $\left[f(x)+\alpha x+\beta\sqrt{x}\right]^2$.\\
		Với mỗi số thực $\alpha,\beta$ ta có
		\begin{eqnarray*}
			\displaystyle\int\limits_0^1\left[f(x)+\alpha x+\beta\sqrt{x}\right]^2\mathrm{\,d}x & = & \displaystyle\int\limits_0^1[f(x)]^2\mathrm{\,d}x+2\displaystyle\int\limits_0^1\left(\alpha x+\beta\sqrt{x}\right)f(x)\mathrm{\,d}x+\displaystyle\int\limits_0^1\left(\alpha x+\beta\sqrt{x}\right)^2\mathrm{\,d}x \\
			& = & 5+2\left(\alpha+\beta\right)+\dfrac{{\alpha}^2}{3}+\dfrac{4\alpha\beta}{5}+\dfrac{{\beta}^2}{2}.
		\end{eqnarray*}
		Ta cần tìm $\alpha,\beta$ sao cho $\displaystyle\int\limits_0^1\left[f(x)+\alpha x+\beta\sqrt{x}\right]^2\mathrm{\,d}x=0$ hay $5+2\left(\alpha+\beta\right)+\dfrac{{\alpha}^2}{3}+\dfrac{4\alpha\beta}{5}+\dfrac{{\beta}^2}{2}=0$.\\
		Tương tự như bài trước, ta tìm được $\alpha=-15,\beta=10$.\\
		Vậy $\displaystyle\int\limits_0^1\left[f(x)-15x+10\sqrt{x}\right]^2\mathrm{\,d}x=0\Rightarrow  f(x)=15x-10\sqrt{x},\forall x\in[0;1]\Rightarrow  \displaystyle\int\limits_0^1[f(x)]^3\mathrm{\,d}x=\dfrac{5}{6}$.}
\end{ex}
\begin{ex}%[2D3G2-4] %Câu 78.
	Cho hàm số $y=f(x)$ liên tục trên đoạn $[0;1],$ thỏa mãn $\displaystyle\int\limits_0^1 xf^2(x)\mathrm{\,d}x=\displaystyle\int\limits_0^1 x^2f(x)\mathrm{\,d}x-\dfrac{1}{16}$. Giá trị của tích phân $\displaystyle\int\limits_0^1 f(x)\mathrm{\,d}x$ bằng 
	\choice
	{$\dfrac{1}{5}$}
	{\True $\dfrac{1}{4}$}
	{$\dfrac{1}{3}$}
	{$\dfrac{2}{5}$}
	\loigiai{Hàm bình phương không như thông thường là $[f(x)]^2$ hoặc $[f'(x)]^2$.\\
		Ở đây các hàm xuất hiện dưới dấu tích phân là $\left[\sqrt{x}f(x)\right]^2, x^2f(x)$ nên ta sẽ liên kết với bình phương $\left[\sqrt{x}f(x)+???\right]^2=xf^2(x)+2???\sqrt{x}f(x)+(???)^2$. So sánh ta thấy được $???=\dfrac{x\sqrt{x}}{2}$.\\
		Do đó giả thiết được viết lại $\displaystyle\int\limits_0^1\left(\sqrt{x}f(x)-\dfrac{x\sqrt{x}}{2}\right)^2\mathrm{\,d}x=\displaystyle\int\limits_0^1\left(\dfrac{x\sqrt{x}}{2}\right)^2\mathrm{\,d}x-\dfrac{1}{16}=0$.\\
		Suy ra $\sqrt{x}f(x)=\dfrac{x\sqrt{x}}{2},\forall x\in[0;1]\Rightarrow  f(x)=\dfrac{x}{2}\Rightarrow  \displaystyle\int\limits_0^1 f(x)\mathrm{\,d}x=\dfrac{1}{4}$.}
\end{ex}
\begin{ex}%[2D3G2-4] %Câu 79.
	Cho hàm số $f(x)$ có đạo hàm liên tục trên $[1;8]$ và thỏa mãn\\
	$\displaystyle\int\limits_1^2\left[f(x^3)\right]^2\mathrm{\,d}x+2\displaystyle\int\limits_1^2 f(x^3)\mathrm{\,d}x=\dfrac{2}{3}\displaystyle\int\limits_1^8 f(x)\mathrm{\,d}x-\dfrac{38}{15}$.\\
	Tích phân $\displaystyle\int\limits_1^8 f(x)\mathrm{\,d}x$ bằng
	\choice
	{$\dfrac{8\ln 2}{27}$}
	{$\dfrac{\ln 2}{27}$}
	{$\dfrac{4}{3}$}
	{\True $\dfrac{3}{2}$}
	\loigiai{Nhận thấy có một tích phân khác cận là $\displaystyle\int\limits_1^8 f(x)\mathrm{\,d}x$. Bằng cách đổi biến $x=t^3$ ta thu được tích phân $3\displaystyle\int\limits_1^2 t^2f(t^3)\mathrm{\,d}t=3\displaystyle\int\limits_1^2 x^2f(x^3)\mathrm{\,d}x$.\\
		Do đó giả thiết được viết lại $\displaystyle\int\limits_1^2\left[f(x^3)\right]^2\mathrm{\,d}x+2\displaystyle\int\limits_1^2 f(x^3)\mathrm{\,d}x=2\displaystyle\int\limits_1^2 x^2f(x^3)\mathrm{\,d}x-\dfrac{38}{15}$. $(*)$.\\
		Ở đây các hàm xuất hiện dưới dấu tích phân là $\left[f(x^3)\right]^2, f(x^3), x^2f(x^3)$ nên ta sẽ liên kết với bình phương $\left[f(x^3)+\alpha x^2+\beta\right]^2$.\\
		Tương tự như các bài trên ta tìm được $\alpha=-1,\beta=1$.\\
		Do đó $(*)\Leftrightarrow\displaystyle\int\limits_1^2\left[f(x^3)-x^2+1\right]^2\mathrm{\,d}x=-\dfrac{38}{15}+\displaystyle\int\limits_1^2\left(1-x^2\right)^2\mathrm{\,d}x=0$.\\
		$\Rightarrow  f(x^3)=x^2-1,\forall x\in[1;2]\Rightarrow  f(x)=\sqrt[3]{x^2}-1,\forall x\in[1;8]\Rightarrow  \displaystyle\int\limits_1^8 f(x)\mathrm{\,d}x=\dfrac{3}{2}$.}
\end{ex}
\begin{ex}%[2D3G2-4] %Câu 80.
	Cho hàm số $f(x)$ có đạo hàm liên tục trên $[0; 1],$ thỏa mãn $f(1)=0$, $\displaystyle\int\limits_0^1[f'(x)]^2\mathrm{\,d}x=7$ và $\displaystyle\int\limits_0^1 x^2f(x)\mathrm{\,d}x=\dfrac{1}{3}$. Tích phân $\displaystyle\int\limits_0^1 f(x)\mathrm{\,d}x$ bằng
	\choice
	{$1$}
	{\True $\dfrac{7}{5}$}
	{$\dfrac{7}{4}$}
	{$4$}
	\loigiai{Hàm dưới dấu tích phân là $[f'(x)]^2, x^2f(x)$ không có mối liên hệ với nhau.\\
		Dùng tích phân từng phần ta có $\displaystyle\int\limits_0^1 x^2f(x)\mathrm{\,d}x=\dfrac{x^3}{3}f(x)\bigg|_0^1 -\dfrac{1}{3}\displaystyle\int\limits_0^1 x^3f'(x)\mathrm{\,d}x$.\\
		Kết hợp với giả thiết $f(1)=0$, ta suy ra $\displaystyle\int\limits_0^1 x^3f'(x)\mathrm{\,d}x=-1$.\\
		Bây giờ giả thiết được đưa về $\heva{&\displaystyle\int\limits_0^1[f'(x)]^2\mathrm{\,d}x=7\\&\displaystyle\int\limits_0^1 x^3f'(x)\mathrm{\,d}x=-1}$. \\
		Hàm dưới dấu tích phân bây giờ là $[f'(x)]^2, x^3f'(x)$ nên ta sẽ liên kết với bình phương $\left[f'(x)+\alpha x^3\right]^2$.\\
		Với mỗi số thực $\alpha$ ta có
		\begin{eqnarray*}
			\displaystyle\int\limits_0^1\left[f'(x)+\alpha x^3\right]^2\mathrm{\,d}x & = & \displaystyle\int\limits_0^1[f'(x)]^2\mathrm{\,d}x+2\alpha\displaystyle\int\limits_0^1 x^3f'(x)\mathrm{\,d}x+\alpha^2\displaystyle\int\limits_0^1 x^6\mathrm{\,d}x \\
			& = & 7-2\alpha+\dfrac{{\alpha}^2}{7}=\dfrac{1}{7}(\alpha-7)^2
		\end{eqnarray*}
		Ta cần tìm $\alpha$ sao cho $\displaystyle\int\limits_0^1\left[f'(x)+\alpha x^3\right]^2\mathrm{\,d}x=0$ hay $\dfrac{1}{7}(\alpha-7)^2=0\Leftrightarrow\alpha=7$.\\
		Vậy $\displaystyle\int\limits_0^1\left[f'(x)+7x^3\right]^2\mathrm{\,d}x=0\Rightarrow  f'(x)=-7x^3,\forall x\in[0;1]\Rightarrow  f(x)=-\dfrac{7}{4}x^4+C$.\\
		$\Rightarrow  {{f(1)=0}}C=\dfrac{7}{4}\Rightarrow  f(x)=-\dfrac{7}{4}x^4+\dfrac{7}{4}\Rightarrow  \displaystyle\int\limits_0^1 f(x)\mathrm{\,d}x=\dfrac{7}{5}$.\\
		Cách 2. Dùng tích phân từng phần ta có $\displaystyle\int\limits_0^1 x^2f(x)\mathrm{\,d}x=\dfrac{x^3}{3}f(x)\bigg|_0^1 -\dfrac{1}{3}\displaystyle\int\limits_0^1 x^3f'(x)\mathrm{\,d}x$.\\
		Kết hợp với giả thiết $f(1)=0$, ta suy ra $\displaystyle\int\limits_0^1 x^3f'(x)\mathrm{\,d}x=-1$.\\
		Theo Holder.\\
		$(-1)^2=\left(\displaystyle\int\limits_0^1 x^3f'(x)\mathrm{\,d}x\right)^2\leq\displaystyle\int\limits_0^1 x^6\mathrm{\,d}x\cdot\displaystyle\int\limits_0^1[f'(x)]^2\mathrm{\,d}x=\dfrac{x^7}{7}\bigg|_0^1\cdot 7=1$.\\
		Vậy đẳng thức xảy ra nên ta có $f'(x)=kx^3,$ thay vào $\displaystyle\int\limits_0^1 x^3f'(x)\mathrm{\,d}x=-1$ ta được $k=-7$.\\
		Suy ra $f'(x)=-7x^3$ (làm tiếp như trên).}
\end{ex}
\begin{ex}%[2D3G2-4] %Câu 81.
	Cho hàm số $f(x)$ có đạo hàm liên tục trên $[0; 1],$ thỏa mãn $f(1)=1$, $\displaystyle\int\limits_0^1 x^5f(x)\mathrm{\,d}x=\dfrac{11}{78}$ và $\displaystyle\int\limits_0^1 f'(x)\mathrm{d}\left(f(x)\right)=\dfrac{4}{13}$. Tính $f(2)$. 
	\choice
	{$f(2)=2$}
	{$f(2)=\dfrac{251}{7}$}
	{$f(2)=\dfrac{256}{7}$}
	{\True $f(2)=\dfrac{261}{7}$}
	\loigiai{Viết lại $\displaystyle\int\limits_0^1 f'(x)\mathrm{d}\left(f(x)\right)=\dfrac{4}{13}\Leftrightarrow\displaystyle\int\limits_0^1[f'(x)]^2\mathrm{\,d}x=\dfrac{4}{13}$.\\
		Dùng tích phân từng phần ta có $\displaystyle\int\limits_0^1 x^5f(x)\mathrm{\,d}x=\dfrac{x^6}{6}f(x)\bigg|_0^1 -\dfrac{1}{6}\displaystyle\int\limits_0^1 x^6f'(x)\mathrm{\,d}x$.\\
		Kết hợp với giả thiết $f(1)=1$, ta suy ra $\displaystyle\int\limits_0^1 x^6f'(x)\mathrm{\,d}x=\dfrac{2}{13}$.\\
		Bây giờ giả thiết được đưa về $\heva{&\displaystyle\int\limits_0^1[f'(x)]^2\mathrm{\,d}x=\dfrac{4}{13}\\&\displaystyle\int\limits_0^1 x^6f'(x)\mathrm{\,d}x=\dfrac{2}{13}}$.\\
		Hàm dưới dấu tích phân bây giờ là $[f'(x)]^2, x^6f'(x)$ nên ta sẽ liên kết với bình phương $\left[f'(x)+\alpha x^6\right]^2$. Tương tự như bài trên ta tìm được $\alpha=-2\Rightarrow  f'(x)=2x^6\Rightarrow  f(x)=\dfrac{2}{7}x^7+C\Rightarrow  {{f(1)=1}}C=\dfrac{5}{7}$.\\
		Vậy $f(x)=\dfrac{2}{7}x^7+\dfrac{5}{7}\Rightarrow  f(2)=\dfrac{261}{7}$. }
\end{ex}
\begin{ex}%[2D3G2-4] %Câu 82.
	Cho hàm số $f(x)$ có đạo hàm liên tục trên $[0; 1],$ thỏa mãn $f(1)=2, f(0)=0$ và $\displaystyle\int\limits_0^1[f'(x)]^2\mathrm{\,d}x=4$. Tích phân $\displaystyle\int\limits_0^1\left[f^3(x)+2018x\right]\mathrm{\,d}x$. bằng
	\choice
	{$0$}
	{\True $1011$}
	{$2018$}
	{$2022$}
	\loigiai{Từ giả thiết $f(1)=2, f(0)=0$ suy ra $\displaystyle\int\limits_0^1 f'(x)\mathrm{\,d}x=f(x)\bigg|_0^1 =2$.\\
		Hàm dưới dấu tích phân là $[f'(x)]^2, f'(x)$ nên sẽ liên kết với bình phương $\left[f'(x)+\alpha\right]^2$.\\
		Ta tìm được $\alpha=-2\Rightarrow  f'(x)=2\Rightarrow  f(x)=2x+C\Rightarrow  {{f(0)=0}}C=0$.\\
		Vậy $f(x)=2x\Rightarrow  \displaystyle\int\limits_0^1\left[f^3(x)+2018x\right]\mathrm{\,d}x=1011$. \\
		Cách 2. Theo Holder.\\
		$2^2=\left(\displaystyle\int\limits_0^1 f'(x)\mathrm{\,d}x\right)^2\leq\displaystyle\int\limits_0^1\mathrm{\,d}x\cdot\displaystyle\int\limits_0^1[f'(x)]^2\mathrm{\,d}x=1\cdot 4=4$.}
\end{ex}
\begin{ex}%[2D3G2-4] %Câu 83.
	Cho hàm số $f(x)$ có đạo hàm liên tục trên $[1; 2],$ thỏa mãn $\displaystyle\int\limits_1^2(x-1)^2f(x)\mathrm{\,d}x=-\dfrac{1}{3}, f(2)=0$ và $\displaystyle\int\limits_1^2[f'(x)]^2\mathrm{\,d}x=7$. Tích phân $\displaystyle\int\limits_1^2 f(x)\mathrm{\,d}x$ bằng
	\choice
	{$-\dfrac{7}{20}$}
	{$\dfrac{7}{20}$}
	{\True $-\dfrac{7}{5}$}
	{$\dfrac{7}{5}$}
	\loigiai{Chuyển thông tin $\displaystyle\int\limits_1^2(x-1)^2f(x)\mathrm{\,d}x$ sang $f'(x)$ bằng cách tích phân từng phần, ta được $$\displaystyle\int\limits_1^2(x-1)^3f'(x)\mathrm{\,d}x=1$$
		Hàm dưới dấu tích phân là $[f'(x)]^2,(x-1)^3f'(x)$ nên liên kết với $\left[f'(x)+\alpha(x-1)^3\right]^2$.\\
		Ta tìm được $\alpha=-7\Rightarrow  f'(x)=7(x-1)^3\Rightarrow  f(x)=\dfrac{7}{4}(x-1)^4+C\Rightarrow  {{f(2)=0}}C=-\dfrac{7}{4}$.\\
		Vậy $f(x)=\dfrac{7}{4}(x-1)^4-\dfrac{7}{4}\Rightarrow  \displaystyle\int\limits_1^2 f(x)\mathrm{\,d}x=-\dfrac{7}{5}$.\\
		Cách 2. Theo Holder.\\
		$1^1=\left[\displaystyle\int\limits_1^2(x-1)^3f'(x)\mathrm{\,d}x\right]^2\leq\displaystyle\int\limits_1^2(x-1)^6\mathrm{\,d}x\displaystyle\int\limits_1^2[f'(x)]^2\mathrm{\,d}x=\dfrac{1}{7}\cdot 7=1$.}
\end{ex}
\begin{ex}%[2D3G2-4] %Câu 84.
	Cho hàm số $f(x)$ có đạo hàm liên tục trên $[0;1],$ thỏa mãn $f(1)=1,\displaystyle\int\limits_0^1[f'(x)]^2\mathrm{\,d}x=\dfrac{9}{5}$ và $\displaystyle\int\limits_0^1 f(\sqrt{x})\mathrm{\,d}x=\dfrac{2}{5}$. Tích phân $\displaystyle\int\limits_0^1 f(x)\mathrm{\,d}x$ bằng
	\choice
	{$I=\dfrac{1}{5}$}
	{\True $I=\dfrac{1}{4}$}
	{$I=\dfrac{3}{5}$}
	{$I=\dfrac{3}{4}$}
	\loigiai{Chuyển thông tin $\displaystyle\int\limits_0^1 f(\sqrt{x})\mathrm{\,d}x$ sang $f'(x)$ bằng cách:\\
		Đặt $t=\sqrt{x}\Rightarrow  \displaystyle\int\limits_0^1 tf(t)\mathrm{\,d}t=\dfrac{1}{5}$ hay $\displaystyle\int\limits_0^1 xf(x)\mathrm{\,d}x=\dfrac{1}{5}$.\\
		Tích phân từng phần $\displaystyle\int\limits_0^1 xf(x)\mathrm{\,d}x,$ ta được $\displaystyle\int\limits_0^1 x^2f'(x)\mathrm{\,d}x=\dfrac{3}{5}$.\\
		Hàm dưới dấu tích phân là $[f'(x)]^2, x^2f'(x)$ nên liên kết với $\left[f'(x)+\alpha x^2\right]^2$.\\
		Ta tìm được $\alpha=-3\Rightarrow  f'(x)=3x^2\Rightarrow  f(x)=x^3+C\Rightarrow  {{f(1)=1}}C=0$.\\
		Vậy $f(x)=x^3\Rightarrow  \displaystyle\int\limits_0^1 f(x)\mathrm{\,d}x=\dfrac{1}{4}$.\\
		Cách 2. Theo Holder.\\
		$\left(\dfrac{3}{5}\right)^2=\left[\displaystyle\int\limits_0^1 x^2f'(x)\mathrm{\,d}x\right]^2\leq\displaystyle\int\limits_0^1 x^4\mathrm{\,d}x\displaystyle\int\limits_0^1[f'(x)]^2\mathrm{\,d}x=\dfrac{1}{5}\cdot\dfrac{9}{5}=\dfrac{9}{25}$.}
\end{ex}
\begin{ex}%[2D3G2-4] %Câu 85.
	Cho hàm số $f(x)$ có đạo hàm liên tục trên $[0; 1],$ thỏa mãn $f(0)+f(1)=0,\displaystyle\int\limits_0^1 f'(x)\cos(\pi x)\mathrm{\,d}x=\dfrac{\pi}{2}$ và $\displaystyle\int\limits_0^1 f^2(x)\mathrm{\,d}x=\dfrac{1}{2}$. Tích phân $\displaystyle\int\limits_0^1 f(x)\mathrm{\,d}x$ bằng
	\choice
	{$\dfrac{1}{\pi}$}
	{\True $\dfrac{2}{\pi}$}
	{$\pi$}
	{$\dfrac{3\pi}{2}$}
	\loigiai{Hàm dưới dấu tích phân là $f^2(x)$ và $f'(x)\cos(\pi x)$, không thấy liên kết.\\
		Do đó ta chuyển thông tin của $f'(x)\cos(\pi x)$ về $f(x)$ bằng cách tích phân từng phần của 
		$$\displaystyle\int\limits_0^1 f'(x)\cos(\pi x)\mathrm{\,d}x=\dfrac{\pi}{2}$$
		cùng với kết hợp $f(0)+f(1)=0,$ ta được $\displaystyle\int\limits_0^1 f(x)\sin(\pi x)\mathrm{\,d}x=\dfrac{1}{2}$.\\
		Hàm dưới dấu tích phân bây giờ là $f^2(x)$ và $f(x)\sin(\pi x)$ nên ta sẽ liên kết với bình phương $\left[f(x)+\alpha\sin(\pi x)\right]^2$.\\
		Ta tìm được $\alpha=-1\Rightarrow  f(x)=\sin(\pi x)\Rightarrow  \displaystyle\int\limits_0^1 f(x)\mathrm{\,d}x=\dfrac{2}{\pi}$.\\
		Cách 2. Theo Holder.\\
		$\left(\dfrac{1}{2}\right)^2=\left(\displaystyle\int\limits_0^1 f(x)\sin(\pi x)\mathrm{\,d}x\right)^2\leq\displaystyle\int\limits_0^1 f^2(x)\mathrm{\,d}x\cdot\displaystyle\int\limits_0^1[\sin(\pi x)]^2\mathrm{\,d}x=\dfrac{1}{2}\cdot\dfrac{1}{2}$.}
\end{ex}
\begin{ex}%[2D3G2-4] %Câu 86.
	Cho hàm số $f(x)$ có đạo hàm liên tục trên $[0;\pi],$ thỏa mãn $\displaystyle\int\limits_0^{\pi} f'(x)\sin x\mathrm{\,d}x=-1$ và $\displaystyle\int\limits_0^{\pi} f^2(x)\mathrm{\,d}x=\dfrac{2}{\pi}$. Tích phân $\displaystyle\int\limits_0^{\pi} xf(x)\mathrm{\,d}x$ bằng
	\choice
	{$-\dfrac{6}{\pi}$}
	{$-\dfrac{4}{\pi}$}
	{$\dfrac{2}{\pi}$}
	{\True $\dfrac{4}{\pi}$}
	\loigiai{Hàm dưới dấu tích phân là $f^2(x)$ và $f'(x)\sin x$, không thấy liên kết.\\
		Do đó ta chuyển thông tin của $f'(x)\sin x$ về $f(x)$ bằng cách tích phân từng phần của 
		$$\displaystyle\int\limits_0^{\pi} f'(x)\sin x\mathrm{\,d}x=-1$$ 
		ta được $\displaystyle\int\limits_0^{\pi} f(x)\cos x\mathrm{\,d}x=1$.\\
		Hàm dưới dấu tích phân bây giờ là $f^2(x)$ và $f(x)\cos x$ nên ta sẽ liên kết với bình phương $\left[f(x)+\alpha\cos x\right]^2$.\\
		Ta tìm được $\alpha=-\dfrac{2}{\pi}\Rightarrow  f(x)=\dfrac{2}{\pi}\cos x\Rightarrow  \displaystyle\int\limits_0^{\pi} xf(x)\mathrm{\,d}x=\displaystyle\int\limits_0^{\pi}\dfrac{2x\cos x}{\pi}\mathrm{\,d}x=-\dfrac{4}{\pi}$.\\
		Cách 2. Theo Holder.\\
		$(1)^2=\displaystyle\int\limits_0^{\pi} f(x)\cos x\mathrm{\,d}x^2\leq\displaystyle\int\limits_0^{\pi} f^2(x)\mathrm{\,d}x\displaystyle\int\limits_0^{\pi}\cos^2x\mathrm{\,d}x=\dfrac{2}{\pi}\cdot\dfrac{\pi}{2}=1$.}
\end{ex}
\begin{ex}%[2D3G2-4] %Câu 87.
	Cho hàm số $f(x)$ có đạo hàm liên tục trên $[0; 1],$ thỏa $f(1)=0$ đồng thời $\displaystyle\int\limits_0^1[f'(x)]^2\mathrm{\,d}x=\dfrac{{\pi}^2}{8}$ và $\displaystyle\int\limits_0^1\cos\left(\dfrac{\pi x}{2}\right)f(x)\mathrm{\,d}x=\dfrac{1}{2}$. Tích phân $\displaystyle\int\limits_0^1 f(x)\mathrm{\,d}x$ bằng
	\choice
	{$\dfrac{1}{\pi}$}
	{\True $\dfrac{2}{\pi}$}
	{$\dfrac{\pi}{2}$}
	{$\pi$}
	\loigiai{Hàm dưới dấu tích phân là $[f'(x)]^2$ và $\cos\left(\dfrac{\pi x}{2}\right)f(x)$, không thấy liên kết.\\
		Do đó ta chuyển thông tin của $\cos\left(\dfrac{\pi x}{2}\right)f(x)$ về $f'(x)$ bằng cách tích phân từng phần của $$\displaystyle\int\limits_0^1\cos\left(\dfrac{\pi x}{2}\right)f(x)\mathrm{\,d}x=\dfrac{1}{2}$$ cùng với kết hợp $f(1)=0,$ ta được $\displaystyle\int\limits_0^1\sin\left(\dfrac{\pi x}{2}\right)f'(x)\mathrm{\,d}x=-\dfrac{\pi}{4}$.\\
		Hàm dưới dấu tích phân bây giờ là $[f'(x)]^2$ và $\sin\left(\dfrac{\pi x}{2}\right)f'(x)$ nên ta sẽ liên kết với bình phương $\left[f'(x)+\alpha\sin\left(\dfrac{\pi x}{2}\right)\right]^2$.\\
		Ta tìm được $\alpha=\dfrac{\pi}{2}\Rightarrow  f'(x)=-\dfrac{\pi}{2}\sin\left(\dfrac{\pi x}{2}\right)\Rightarrow  f(x)=\cos\left(\dfrac{\pi x}{2}\right)+C\Rightarrow  {{f(1)=0}}\Rightarrow C=0$.\\
		Vậy $f(x)=\cos\left(\dfrac{\pi x}{2}\right)\Rightarrow  \displaystyle\int\limits_0^1 f(x)\mathrm{\,d}x=\dfrac{2}{\pi}$.\\
		Cách 2. Theo Holder.\\
		$\left(-\dfrac{\pi}{4}\right)^2=\left(\displaystyle\int\limits_0^1\sin\left(\dfrac{\pi x}{2}\right)f'(x)\mathrm{\,d}x\right)^2\leq\displaystyle\int\limits_0^1\sin^2\left(\dfrac{\pi x}{2}\right)\mathrm{\,d}x\cdot\displaystyle\int\limits_0^1[f'(x)]^2\mathrm{\,d}x=\dfrac{1}{2}\cdot\dfrac{{\pi}^2}{8}$.}
\end{ex}
\begin{ex}%[2D3G2-4] %Câu 88.
	Cho hàm số $f(x)$ có đạo hàm liên tục trên $[0; 1],$ thỏa mãn $\displaystyle\int\limits_0^1 f'(x)\sin(\pi x)\mathrm{\,d}x=\pi$ và $\displaystyle\int\limits_0^1 f^2(x)\mathrm{\,d}x=2$. Tích phân $\displaystyle\int\limits_0^1 f\left(\dfrac{x}{2}\right)\mathrm{\,d}x$ bằng
	\choice
	{$-\dfrac{6}{\pi}$}
	{\True $-\dfrac{4}{\pi}$}
	{$\dfrac{4}{\pi}$}
	{$\dfrac{6}{\pi}$}
	\loigiai{Chuyển thông tin của $f'(x)\sin(\pi x)$ về $f(x)$ bằng cách tích phân từng phần của $\displaystyle\int\limits_0^1 f'(x)\sin(\pi x)\mathrm{\,d}x=\pi,$ ta được $\displaystyle\int\limits_0^1 f(x)\cos(\pi x)\mathrm{\,d}x=-1$.\\
		Hàm dưới dấu tích phân bây giờ là $f^2(x)$ và $\cos(\pi x)f(x)$ nên ta sẽ liên kết với bình phương $\left[f(x)+\alpha\cos(\pi x)\right]^2$.\\
		Ta tìm được $\alpha=2\Rightarrow  f(x)=-2\cos(\pi x)\Rightarrow  \displaystyle\int\limits_0^1 f\left(\dfrac{x}{2}\right)\mathrm{\,d}x=-2\displaystyle\int\limits_0^1\cos\left(\dfrac{\pi x}{2}\right)\mathrm{\,d}x=-\dfrac{4}{\pi}$.\\
		Cách 2. Theo Holder.\\
		$(-1)^2=\left[\displaystyle\int\limits_0^1 f(x)\cos(\pi x)\mathrm{\,d}x\right]^2\leq\displaystyle\int\limits_0^1\cos^2(\pi x)\mathrm{\,d}x\displaystyle\int\limits_0^1[f(x)]^2\mathrm{\,d}x=\dfrac{1}{2}\cdot 2$.}
\end{ex}
\begin{ex}%[2D3G2-4] %Câu 89.
	Cho hàm số $f(x)$ có đạo hàm liên tục trên $\left[0;\dfrac{\pi}{2}\right],$ thỏa $f\left(\dfrac{\pi}{2}\right)=0,\displaystyle\int\limits_0^{\tfrac{\pi}{2}} f^2(x)\mathrm{\,d}x=3\pi$ và $\displaystyle\int\limits_0^{\pi}(\sin x-x)f'\left(\dfrac{x}{2}\right)\mathrm{\,d}x=6\pi$. Tích phân $\displaystyle\int\limits_0^{\tfrac{\pi}{2}}[f''(x)]^3\mathrm{\,d}x$ bằng
	\choice
	{$-\dfrac{2}{\pi}$}
	{\True $0$}
	{$3\pi$}
	{$9\pi$}
	\loigiai{Tích phân từng phần của $\displaystyle\int\limits_0^{\pi}(\sin x-x)f'\left(\dfrac{x}{2}\right)\mathrm{\,d}x=6\pi,$ kết hợp với $f\left(\dfrac{\pi}{2}\right)=0$ ta được.\\
		ta được $\displaystyle\int\limits_0^{\tfrac{\pi}{2}}\sin^2xf(x)\mathrm{\,d}x=\dfrac{3\pi}{4}$.\\
		Hàm dưới dấu tích phân bây giờ là $f^2(x)$ và $\sin^2xf(x)$ nên ta sẽ liên kết với bình phương $\left[f(x)+\alpha\sin^2x\right]^2$.\\
		Ta tìm được $\alpha=-4\Rightarrow  f(x)=4\sin^2x\Rightarrow f'(x)=4\sin 2x\Rightarrow f''(x)=8\cos 2x$.\\
		Vậy $\displaystyle\int\limits_0^{\tfrac{\pi}{2}}[f''(x)]^3\mathrm{\,d}x=\displaystyle\int\limits_0^{\tfrac{\pi}{2}}[8\cos 2x]^3\mathrm{\,d}x=0$.\\
		Cách 2. Theo Holder.\\
		$\left(\dfrac{3\pi}{4}\right)^2=\left(\displaystyle\int\limits_0^{\tfrac{\pi}{2}}\sin^2xf(x)\mathrm{\,d}x\right)^2\leq\displaystyle\int\limits_0^{\tfrac{\pi}{2}}\sin^4x\mathrm{\,d}x\displaystyle\int\limits_0^{\tfrac{\pi}{2}} f^2(x)\mathrm{\,d}x=\dfrac{3\pi}{16}\cdot 3\pi$.}
\end{ex}
\begin{ex}%[2D3G2-4] %Câu 90.
	Cho hàm số $f(x)$ có đạo hàm liên tục trên đoạn $[0; 1],$ thỏa mãn $f(1)=0$ và $\displaystyle\int\limits_0^1[f'(x)]^2\mathrm{\,d}x=\displaystyle\int\limits_0^1(x+1)\mathrm{e}^xf(x)\mathrm{\,d}x=\dfrac{\mathrm{e}^2-1}{4}$. Tính tích phân $I=\displaystyle\int\limits_0^1 f(x)\mathrm{\,d}x$. 
	\choice
	{$I=\dfrac{e-1}{2}$}
	{$I=\dfrac{\mathrm{e}^2}{4}$}
	{\True $I=e-2$}
	{$I=\dfrac{e}{2}$}
	\loigiai{Tích phân từng phần của $\displaystyle\int\limits_0^1(x+1)\mathrm{e}^xf(x)\mathrm{\,d}x,$ kết hợp với $f(1)=0$ ta được.\\
		$\displaystyle\int\limits_0^1 x\mathrm{e}^xf'(x)\mathrm{\,d}x=-\dfrac{\mathrm{e}^2-1}{4}$.\\
		Hàm dưới dấu tích phân bây giờ là $[f'(x)]^2$ và $x\mathrm{e}^xf'(x)$ nên ta sẽ liên kết với $\left[f(x)+\alpha x\mathrm{e}^x\right]^2$.\\
		Ta tìm được $\alpha=1\Rightarrow  f'(x)=-x\mathrm{e}^x\Rightarrow  f(x)=-\displaystyle\int x\mathrm{e}^x\mathrm{\,d}x=(1-x)\mathrm{e}^x+C\Rightarrow  {{f(1)=0}}C=0$.\\
		Vậy $f(x)=(1-x)\mathrm{e}^x\Rightarrow  \displaystyle\int\limits_0^1 f(x)\mathrm{\,d}x=\displaystyle\int\limits_0^1(1-x)\mathrm{e}^x\mathrm{\,d}x=e-2$.\\ Cách 2. Theo Holder.\\
		$\left(-\dfrac{\mathrm{e}^2-1}{4}\right)^2=\left(\displaystyle\int\limits_0^1 x\mathrm{e}^xf'(x)\mathrm{\,d}x\right)^2\leq\displaystyle\int\limits_0^1 x^2\mathrm{e}^{2x}\mathrm{\,d}x\cdot\displaystyle\int\limits_0^1[f'(x)]^2\mathrm{\,d}x=\dfrac{\mathrm{e}^2-1}{4}\cdot\dfrac{\mathrm{e}^2-1}{4}$.}
\end{ex}
\begin{ex}%[2D3G2-4]%[Trần Ngọc Phú]%Câu 91.
	Cho hàm số $f\left( x\right) $ có đạo hàm liên tục trên $\left[ 0; 1\right] $, thỏa mãn $f\left( 0\right) =0,\, f\left( 1\right) =1$ và $\displaystyle\int\limits_0^1\dfrac{\left[ f'\left( x\right) \right] ^2}{\mathrm{e}^x}\mathrm{\,d}x=\dfrac{1}{e-1}$. Tích phân $\displaystyle\int\limits_0^1 f\left( x\right) \mathrm{\,d}x$ bằng
	\choice
	{$\dfrac{\mathrm{e}-2}{\mathrm{e}-1}$}
	{$\dfrac{\mathrm{e}-1}{\mathrm{e}-2}$}
	{$\dfrac{1}{\left( \mathrm{e}-1\right) \left( \mathrm{e}-2\right) }$}
	{\True $1$}
	\loigiai{Hàm dưới dấu tích phân là $\dfrac{\left[ f'(x)\right] ^2}{\mathrm{e}^x}$ nên ta cần tìm một thông tin liên quan $f'(x)$.\\
		Từ giả thiết $f(0)=0,\, f(1)=1$ ta nghĩ đến $\displaystyle\int\limits_0^1 f'(x)\mathrm{\,d}x=f(x)\bigg|_0^1 =f(1)-f(0)=1$.\\
		Do đó ta có hàm dưới dấu tích phân là $\dfrac{\left[ f'(x)\right] ^2}{\mathrm{e}^x}$ và $f'(x)$ nên sẽ liên kết với bình phương $\left[\dfrac{f'(x)}{\sqrt{\mathrm{e}^x}}+\alpha\sqrt{\mathrm{e}^x}\right]^2$. Với mỗi số thực $\alpha$ ta có
		\begin{eqnarray*}
			\displaystyle\int\limits_0^1\left[\dfrac{f'(x)}{\sqrt{\mathrm{e}^x}}+\alpha\sqrt{\mathrm{e}^x}\right]^2\mathrm{\,d}x&=&\displaystyle\int\limits_0^1\dfrac{\left[ f'(x)\right] ^2}{\mathrm{e}^x}\mathrm{\,d}x+2\alpha\displaystyle\int\limits_0^1 f'(x)\mathrm{\,d}x+\alpha^2\displaystyle\int\limits_0^1\mathrm{e}^x\mathrm{\,d}x.\\
			&=&\dfrac{1}{\mathrm{e}-1}+2\alpha+\alpha^2(\mathrm{e}-1)=\dfrac{1}{\mathrm{e}-1}\left[(\mathrm{e}-1)\alpha+1\right]^2.
		\end{eqnarray*}
		Ta cần tìm $\alpha$ sao cho $\displaystyle\int\limits_0^1\left[\dfrac{f'(x)}{\sqrt{\mathrm{e}^x}}+\alpha\sqrt{\mathrm{e}^x}\right]^2\mathrm{\,d}x=0$ hay $\dfrac{1}{\mathrm{e}-1}\left[(\mathrm{e}-1)\alpha+1\right]^2=0\Leftrightarrow\alpha=-\dfrac{1}{\mathrm{e}-1}$.\\
		Với $\alpha=-\dfrac{1}{\mathrm{e}-1}$ thì $\displaystyle\int\limits_0^1\left[\dfrac{f'(x)}{\sqrt{\mathrm{e}^x}}-\dfrac{1}{\mathrm{e}-1}\sqrt{\mathrm{e}^x}\right]^2\mathrm{\,d}x=0\Rightarrow\dfrac{f'(x)}{\sqrt{\mathrm{e}^x}}\equiv\dfrac{1}{\mathrm{e}-1}\sqrt{\mathrm{e}^x},\,\forall x\in[0;1]$.\\
		Suy ra $f'(x)=\dfrac{\mathrm{e}^x}{\mathrm{e}-1}\Rightarrow  f(x)=\displaystyle\int\dfrac{\mathrm{e}^x}{\mathrm{e}-1}\mathrm{\,d}x=\dfrac{\mathrm{e}^x}{\mathrm{e}-1}+C$.\\
		Ta có  ${{f(0)=0,\, f(1)=1}} $ suy ra $ C=-\dfrac{1}{\mathrm{e}-1}$.\\
		Vậy $f(x)=\dfrac{\mathrm{e}^x-1}{\mathrm{e}-1}\Rightarrow  \displaystyle\int\limits_0^1 f(x)\mathrm{\,d}x=\dfrac{\mathrm{e}-2}{\mathrm{e}-1}$.\\
		Cách 2. Theo Holder.\\
		$1^2=\left[\displaystyle\int\limits_0^1 f'(x)\mathrm{\,d}x\right]^2=\left[\displaystyle\int\limits_0^1\dfrac{f'(x)}{\sqrt{\mathrm{e}^x}}\cdot\sqrt{\mathrm{e}^x}\mathrm{\,d}x\right]^2\leq\displaystyle\int\limits_0^1\dfrac{[f'(x)]^2}{\mathrm{e}^x}\mathrm{\,d}x\displaystyle\int\limits_0^1\mathrm{e}^x\mathrm{\,d}x=\dfrac{1}{\mathrm{e}-1}\cdot (\mathrm{e}-1)=1$.}
\end{ex}
\begin{ex}%[2D3G2-4]%[Trần Ngọc Phú]%Câu 92.
	Cho hàm số $f(x)$ có đạo hàm liên tục trên $[0; 1]$, thỏa mãn $f(0)=0,\, f(1)=1$ và $\displaystyle\int\limits_0^1\sqrt{1+x^2}\left[ f'(x)\right] ^2\mathrm{\,d}x=\dfrac{1}{\ln(1+\sqrt{2})}$. Tích phân $\displaystyle\int\limits_0^1\dfrac{f(x)}{\sqrt{1+x^2}}\mathrm{\,d}x$ bằng
	\choice
	{$\dfrac{1}{2}\ln^2(1+\sqrt{2})$}
	{$\dfrac{\sqrt{2}-1}{2}\ln^2(1+\sqrt{2})$}
	{\True $\dfrac{1}{2}\ln(1+\sqrt{2})$}
	{$(\sqrt{2}-1)\ln(1+\sqrt{2})$}
	\loigiai{Tương tự bài trước, ta có $\displaystyle\int\limits_0^1 f'(x)\mathrm{\,d}x=f(x)\bigg|_0^1 =f(1)-f(0)=1$.\\
		Do đó ta có hàm dưới dấu tích phân là $\sqrt{1+x^2}\left[ f'(x)\right]^2$ và $f'(x)$ nên sẽ liên kết với bình phương $\left[\sqrt[4]{1+x^2}f'(x)+\dfrac{\alpha}{\sqrt[4]{1+x^2}}\right]^2$.\\
		Ta tìm được $\alpha=-\dfrac{1}{\ln(1+\sqrt{2})}\Rightarrow  f'(x)=\dfrac{1}{\ln(1+\sqrt{2})}\cdot\dfrac{1}{\sqrt{1+x^2}}$.\\
		Suy ra $  f(x)=\dfrac{1}{\ln(1+\sqrt{2})}\cdot\displaystyle\int\dfrac{1}{\sqrt{1+x^2}}\mathrm{\,d}x=\dfrac{1}{\ln(1+\sqrt{2})}\ln\left(x+\sqrt{1+x^2}\right)+C$.\\
		Mà $f(0)=0,\,f(1)=1\Rightarrow C=0\Rightarrow  f(x)=\dfrac{\ln\left(x+\sqrt{1+x^2}\right)}{\ln(1+\sqrt{2})}$.\\
		Vậy
		\begin{eqnarray*}
			\displaystyle\int\limits_0^1\dfrac{f(x)}{\sqrt{1+x^2}}\mathrm{\,d}x&=&\dfrac{1}{\ln(1+\sqrt{2})}\displaystyle\int\limits_0^1\dfrac{\ln\left(x+\sqrt{1+x^2}\right)}{\sqrt{1+x^2}}\mathrm{\,d}x\\&=&\dfrac{1}{\ln(1+\sqrt{2})}\displaystyle\int\limits_0^1\ln\left(x+\sqrt{1+x^2}\right)\mathrm{d}\left[\ln\left(x+\sqrt{1+x^2}\right)\right].\\
			&=&\dfrac{1}{\ln(1+\sqrt{2})}\cdot\dfrac{\ln^2\left(x+\sqrt{1+x^2}\right)}{2}\bigg|_0^1\\&=&\dfrac{1}{2}\ln(1+\sqrt{2}).
		\end{eqnarray*}  
		Cách 2. Theo Holder.
		\begin{eqnarray*}
			1^2=\left(\displaystyle\int\limits_0^1 f'(x)\mathrm{\,d}x\right)^2=\displaystyle\int\limits_0^1\sqrt[4]{1+x^2}f'(x)\cdot\dfrac{1}{\sqrt[4]{1+x^2}}\mathrm{\,d}x&\leq&\displaystyle\int\limits_0^1\sqrt{1+x^2}\left[ f'(x)\right]^2\mathrm{\,d}x\cdot\displaystyle\int\limits_0^1\dfrac{\mathrm{\,d}x}{\sqrt{1+x^2}}\\
			&\leq&\dfrac{1}{\ln(1+\sqrt{2})}\cdot\ln(1+\sqrt{2})=1.
		\end{eqnarray*}
	}
\end{ex}
\begin{ex}%[2D3G2-4]%[Trần Ngọc Phú]%Câu 93.
	Cho hàm số $f(x)$ có đạo hàm liên tục trên $[-1;1]$, thỏa mãn $f(-1)=0,\displaystyle\int\limits_{-1}^1\left[ f'(x)\right]^2\mathrm{\,d}x=112$ và $\displaystyle\int\limits_{-1}^1 x^2f(x)\mathrm{\,d}x=\dfrac{16}{3}$. Tính tích phân $I=\displaystyle\int\limits_{-1}^1 f(x)\mathrm{\,d}x$. 
	\choice
	{\True $I=\dfrac{84}{5}$}
	{$I=\dfrac{35}{2}$}
	{$I=\dfrac{35}{4}$}
	{$I=\dfrac{168}{5}$}
	\loigiai{
		Như các bài trước, ta chuyển $\displaystyle\int\limits_{-1}^1 x^2f(x)\mathrm{\,d}x=\dfrac{16}{3}$ về thông tin của $f'(x)$ bằng cách tích phân từng phần. Đặt $\heva{&u=f(x)\\&\mathrm{\,d}v=x^2\mathrm{\,d}x}\Rightarrow\heva{&\mathrm{\,d}u=f'(x)\mathrm{\,d}x\\&v=\dfrac{x^3}{3}.}$ \\
		Khi đó $\displaystyle\int\limits_{-1}^1 x^2f(x)\mathrm{\,d}x=\dfrac{x^3}{3}f(x)\bigg|_{-1}^1 -\dfrac{1}{3}\displaystyle\int\limits_{-1}^1 x^3f'(x)\mathrm{\,d}x=\dfrac{1}{3}f(1)+\dfrac{1}{3}f(-1)-\dfrac{1}{3}\displaystyle\int\limits_{-1}^1 x^3f'(x)\mathrm{\,d}x$.\\
		Tới đây ta bị vướng $f(1)$ vì giả thiết không cho. Do đó ta điều chỉnh lại như sau:\\
		$\heva{&u=f(x)\\&\mathrm{\,d}v=x^2\mathrm{\,d}x}\Rightarrow\heva{&\mathrm{\,d}u=f'(x)\mathrm{\,d}x\\&v=\dfrac{x^3}{3}+k}$ với $k$ là hằng số.\\
		Khi đó 
		\begin{eqnarray*}
			\displaystyle\int\limits_{-1}^1 x^2f(x)\mathrm{\,d}x&=&\left(\dfrac{x^3}{3}+k\right)f(x)\bigg|_{-1}^1 -\displaystyle\int\limits_{-1}^1\left(\dfrac{x^3}{3}+k\right)f'(x)\mathrm{\,d}x.\\
			&=&\left(\dfrac{1}{3}+k\right)f(1)-\underbrace{\left(-\dfrac{1}{3}+k\right)f(-1)}_{=0 \, \text{do}\, f(-1)=0}-\displaystyle\int\limits_{-1}^1\left(\dfrac{x^3}{3}+k\right)f'(x)\mathrm{\,d}x.
		\end{eqnarray*}
		Ta chọn $k$ sao cho $\dfrac{1}{3}+k=0\Leftrightarrow k=-\dfrac{1}{3}$.\\
		Khi đó $\dfrac{16}{3}=\displaystyle\int\limits_{-1}^1 x^2f(x)\mathrm{\,d}x=-\dfrac{1}{3}\displaystyle\int\limits_{-1}^1\left(x^3-1\right)f'(x)\mathrm{\,d}x\Rightarrow  \displaystyle\int\limits_{-1}^1\left(x^3-1\right)f'(x)\mathrm{\,d}x=-16$.\\
		Hàm dưới dấu tích phân là $[f'(x)]^2,\,\left(x^3-1\right)f'(x)$ nên ta liên kết với $\left[f'(x)+\alpha\left(x^3-1\right)\right]^2$.\\
		Ta tìm được $\alpha=7\Rightarrow  f'(x)=-7\left(x^3-1\right)\Rightarrow f(x)=-7\displaystyle\int\left(x^3-1\right)\mathrm{\,d}x=-\dfrac{7}{4}x^4+7x+C$.\\
		Ta có  ${{f(-1)=0}}$ $\Rightarrow C=\dfrac{35}{4}\Rightarrow  f(x)=-\dfrac{7}{4}x^4+7x+\dfrac{35}{4}$.\\ Vậy $I=\displaystyle\int\limits_{-1}^1 f(x)\mathrm{\,d}x=\dfrac{84}{5}$.\\
		Cách 2. Theo Holder.\\
		$(-16)^2=\left(\displaystyle\int\limits_{-1}^1\left(x^3-1\right)f'(x)\mathrm{\,d}x\right)^2\leq\displaystyle\int\limits_{-1}^1\left(x^3-1\right)^2\mathrm{\,d}x\cdot\displaystyle\int\limits_{-1}^1[f'(x)]^2\mathrm{\,d}x=\dfrac{16}{7}\cdot 112=256$.}
\end{ex}
\begin{ex}%[2D3G2-4]%[Trần Ngọc Phú]%Câu 94.
	Cho hàm số $f(x)$ có đạo hàm liên tục trên $[0;1]$, thỏa mãn $f(1)=0,\break \displaystyle\int\limits_0^1[f'(x)]^2\mathrm{\,d}x=\dfrac{3}{2}-2\ln 2$ và $\displaystyle\int\limits_0^1\dfrac{f(x)}{(x+1)^2}\mathrm{\,d}x=2\ln 2-\dfrac{3}{2}$. Tích phân $\displaystyle\int\limits_0^1 f(x)\mathrm{\,d}x$ bằng
	\choice
	{$\dfrac{1-\ln 2}{2}$}
	{\True $\dfrac{1-2\ln 2}{2}$}
	{$\dfrac{3-2\ln 2}{2}$}
	{$\dfrac{3-4\ln 2}{2}$}
	\loigiai{
		Như các bài trước, ta chuyển $\displaystyle\int\limits_0^1\dfrac{f(x)}{(x+1)^2}\mathrm{\,d}x=2\ln 2-\dfrac{3}{2}$ về thông tin của $f'(x)$ bằng cách tích phân từng phần. Đặt $\heva{&u=f(x)\\&\mathrm{\,d}v=\dfrac{1}{(x+1)^2}\mathrm{\,d}x}\Rightarrow\heva{&\mathrm{\,d}u=f'(x)\mathrm{\,d}x\\&v=-\dfrac{1}{x+1}.}$ \\
		Khi đó $\displaystyle\int\limits_0^1\dfrac{f(x)}{(x+1)^2}\mathrm{\,d}x=-\dfrac{f(x)}{x+1}\bigg|_0^1 +\displaystyle\int\limits_0^1\dfrac{f'(x)}{x+1}\mathrm{\,d}x=-\dfrac{f(1)}{2}+\dfrac{f(0)}{1}+\displaystyle\int\limits_0^1\dfrac{f'(x)}{x+1}\mathrm{\,d}x$.\\ Tới đây ta bị vướng $f(0)$ vì giả thiết không cho.\\ Do đó ta điều chỉnh lại như sau:\\
		$\heva{&u=f(x)\\&\mathrm{\,d}v=\dfrac{1}{(x+1)^2}\mathrm{\,d}x}\Rightarrow\heva{&\mathrm{\,d}u=f'(x)\mathrm{\,d}x\\&v=-\dfrac{1}{x+1}+k}$ với $k$ là hằng số.\\
		Khi đó
		\begin{eqnarray*}
			\displaystyle\int\limits_0^1\dfrac{f(x)}{(x+1)^2}\mathrm{\,d}x&=&\left(-\dfrac{1}{x+1}+k\right)f(x)\bigg|_0^1 -\displaystyle\int\limits_0^1\left(-\dfrac{1}{x+1}+k\right)f'(x)\mathrm{\,d}x\\&
			\overset{f(1)=0}{=}&-(-1+k)f(0)-\displaystyle\int\limits_0^1\left(-\dfrac{1}{x+1}+k\right)f'(x)\mathrm{\,d}x.
		\end{eqnarray*} 
		Ta chọn $k$ sao cho $-1+k=0\Leftrightarrow k=1$.\\
		Khi đó $2\ln 2-\dfrac{3}{2}=\displaystyle\int\limits_0^1\dfrac{f(x)}{(x+1)^2}\mathrm{\,d}x=-\displaystyle\int\limits_0^1\dfrac{x}{x+1}f'(x)\mathrm{\,d}x\Rightarrow  \displaystyle\int\limits_0^1\dfrac{x}{x+1}f'(x)\mathrm{\,d}x=\dfrac{3}{2}-2\ln 2$.\\
		Hàm dưới dấu tích phân là $[f'(x)]^2,\,\dfrac{x}{x+1}f'(x)$ nên ta liên kết với $\left[f'(x)+\alpha\dfrac{x}{x+1}\right]^2$.\\
		Ta tìm được $\alpha=-1\Rightarrow  f'(x)=\dfrac{x}{x+1}\Rightarrow f(x)=\displaystyle\int\dfrac{x}{x+1}\mathrm{\,d}x=x-\ln|x+1|+C$.\\
		Ta có  	${{f(1)=0}}\Rightarrow C=\ln 2-1$. Do đó   $f(x)=x-\ln(x+1)+\ln 2-1$.\\ Vậy $\displaystyle\int\limits_0^1 f(x)\mathrm{\,d}x=\dfrac{1-2\ln 2}{2}$.\\ Cách 2. Theo Holder.\\
		$\left(\dfrac{3}{2}-2\ln 2\right)^2=\left[\displaystyle\int\limits_0^1\dfrac{x}{x+1}f'(x)\mathrm{\,d}x\right]^2\leq\displaystyle\int\limits_0^1\left(\dfrac{x}{x+1}\right)^2\mathrm{\,d}x\displaystyle\int\limits_0^1\left[ f'(x)\right] ^2\mathrm{\,d}x=\left(\dfrac{3}{2}-2\ln 2\right)\left(\dfrac{3}{2}-2\ln 2\right)$.}
\end{ex}
\begin{ex}%[2D3G2-4]%[Trần Ngọc Phú]%Câu 95.
	Cho hàm số $f(x)$ có đạo hàm liên tục trên $[1; 2]$, đồng biến trên $[1; 2]$, thỏa mãn $f(1)=0$, $\displaystyle\int\limits_1^2[f'(x)]^2\mathrm{\,d}x=2$ và $\displaystyle\int\limits_1^2 f(x)\cdot f'(x)\mathrm{\,d}x=1$. Tích phân $\displaystyle\int\limits_1^2 f(x)\mathrm{\,d}x$ bằng
	\choice
	{\True $\dfrac{\sqrt{2}}{2}$}
	{$\sqrt{2}$}
	{$2$}
	{$2\sqrt{2}$}
	\loigiai{Hàm dưới dấu tích phân là $[f'(x)]^2,\, f(x)\cdot f'(x)$ nên ta sẽ liên kết với bình phương $\left[f'(x)+\alpha f(x)\right]^2$. Nhưng khi khai triển thì vướng $\displaystyle\int\limits_1^2[f(x)]^2\mathrm{\,d}x$ nên hướng này không khả thi.\\
		Ta có $1=\displaystyle\int\limits_1^2 f(x)\cdot f'(x)\mathrm{\,d}x=\dfrac{f^2(x)}{2}\bigg|_1^2 =\dfrac{f^2(2)-f^2(1)}{2}=\dfrac{f^2(2)-0}{2}\Rightarrow  f(2)=\sqrt{2}$ (do đồng biến trên $[1; 2]$ nên $f(2)>f(1)=0$).\\
		Từ $f(1)=0$ và $f(2)=\sqrt{2}$ ta nghĩ đến $\displaystyle\int\limits_1^2 f'(x)\mathrm{\,d}x=f(x)\bigg|_1^2 =f(2)-f(1)=\sqrt{2}-0=\sqrt{2}$.\\
		Hàm dưới dấu tích phân bây giờ là $[f'(x)]^2,\, f'(x)$ nên ta sẽ liên kết với $\left[f'(x)+\alpha\right]^2$.\\
		Ta tìm được $\alpha=-\sqrt{2}\Rightarrow  f'(x)=\sqrt{2}\Rightarrow  f(x)=\sqrt{2}x+C$.\\
		Ta có  $ {{f(1)=0}} \Rightarrow C=-\sqrt{2}$.\\
		Vậy $f(x)=\sqrt{2}x-\sqrt{2}\Rightarrow  \displaystyle\int\limits_1^2 f(x)\mathrm{\,d}x=\dfrac{\sqrt{2}}{2}$.}
\end{ex}
\begin{ex}%[2D3G2-4]%[Trần Ngọc Phú]%Câu 96.
	Cho hàm số $f(x)$ có đạo hàm liên tục trên $[0;1]$, thỏa mãn $f(1)=0$, $\displaystyle\int\limits_0^1 f^2(x)\mathrm{\,d}x=1$ và $\displaystyle\int\limits_0^1\left[ f'(x)\right] ^2f^2(x)\mathrm{\,d}x=\dfrac{3}{4}$. Giá trị của $f^2(\sqrt{2})$ bằng
	\choice
	{\True $-\dfrac{3}{2}$}
	{$\dfrac{3}{2}$}
	{$\dfrac{3(1-\sqrt{2})}{2}$}
	{$-\dfrac{3(1-\sqrt{2})}{2}$}
	\loigiai{Hàm dưới dấu tích phân là $[f'(x)]^2f^2(x)$ và $f^2(x)$ nên ta sẽ liên kết với bình phương $\left[f'(x)f(x)+\alpha f(x)\right]^2$. Nhưng khi khai triển thì vướng $\displaystyle\int\limits_0^1 f^2(x)f'(x)\mathrm{\,d}x$ nên hướng này không khả thi.\\
		Tích phân từng phần $\displaystyle\int\limits_0^1 f^2(x)\mathrm{\,d}x=1$ kết hợp với $f(1)=0,$ ta được $\displaystyle\int\limits_0^1 xf(x)f'(x)\mathrm{\,d}x=-\dfrac{1}{2}$.\\
		Hàm dưới dấu tích phân bây giờ là $[f'(x)]^2f^2(x)$ và $xf(x)f'(x)$ nên ta sẽ liên kết với bình phương $\left[f(x)f'(x)+\alpha x\right]^2$.\\
		Ta tìm được $\alpha=\dfrac{3}{2}\Rightarrow  f(x)f'(x)=-\dfrac{3}{2}x\Rightarrow\displaystyle\int f(x)f'(x)\mathrm{\,d}x=-\dfrac{3}{2}\displaystyle\int x\mathrm{\,d}x\Rightarrow\dfrac{f^2(x)}{2}=-\dfrac{3}{4}x^2+C$.\\
		Mà ${{f(1)=0}}\Rightarrow C=\dfrac{3}{4}\Rightarrow  f^2(x)=\dfrac{3}{2}\left(1-x^2\right)\Rightarrow  f^2(\sqrt{2})=-\dfrac{3}{2}$.}
\end{ex}
\begin{ex}%[2D3G2-4]%[Trần Ngọc Phú]%Câu 97.
	Cho hàm số $f(x)$ có đạo hàm liên tục trên $[0;2]$, thỏa mãn $f(2)=1$, $\displaystyle\int\limits_0^2 x^2f(x)\mathrm{\,d}x=\dfrac{8}{15}$ và $\displaystyle\int\limits_0^2\left[ f'(x)\right] ^4\mathrm{\,d}x=\dfrac{32}{5}$. Giá trị của tích phân $\displaystyle\int\limits_0^2 f(x)\mathrm{\,d}x$ bằng
	\choice
	{$-\dfrac{3}{2}$}
	{\True $-\dfrac{2}{3}$}
	{$-\dfrac{7}{3}$}
	{$\dfrac{7}{3}$}
	\loigiai{Hàm dưới dấu tích phân $\left[ f'(x)\right] ^4$ và $x^2f(x)$. Lời khuyên là đừng có cố liên kết với bình phương nào, vì có tìm cũng không ra.\\
		Tích phân từng phần $\displaystyle\int\limits_0^2 x^2f(x)\mathrm{\,d}x=\dfrac{8}{15}$ kết hợp với $f(2)=1$, ta được $\displaystyle\int\limits_0^2 x^3f'(x)\mathrm{\,d}x=\dfrac{32}{5}$.\\
		Áp dụng Holder $2$ lần ta được.
		\begin{eqnarray*}
			\left(\dfrac{32}{5}\right)^4=\left(\displaystyle\int\limits_0^2 x^3f'(x)\mathrm{\,d}x\right)^4=\left(\displaystyle\int\limits_0^2 x^2\cdot xf'(x)\mathrm{\,d}x\right)^4&\leq&\left(\displaystyle\int\limits_0^2 x^4\mathrm{\,d}x\right)^2\left(\displaystyle\int\limits_0^2 x^2\left[ f'(x)\right]^2\mathrm{\,d}x\right)^2\\
			&\leq&\left(\displaystyle\int\limits_0^2 x^4\mathrm{\,d}x\right)^2\times\left(\displaystyle\int\limits_0^2 x^4\mathrm{\,d}x\cdot\displaystyle\int\limits_0^2\left[ f'(x)\right]^4\mathrm{\,d}x\right)\\&\leq&\left(\displaystyle\int\limits_0^2 x^4\mathrm{\,d}x\right)^3\times\displaystyle\int\limits_0^2\left[ f'(x)\right]^4\mathrm{\,d}x\\&\leq&\dfrac{1048576}{625}=\left(\dfrac{32}{5}\right)^4.
		\end{eqnarray*}
		Dấu \lq\lq=\rq\rq\,  xảy ra, tức là $xf'(x)=kx^2\Rightarrow f'(x)=kx$ thay vào $\displaystyle\int\limits_0^2[f'(x)]^4\mathrm{\,d}x=\dfrac{32}{5}$ tìm được $k=1$.\\
		$\Rightarrow  f'(x)=x\Rightarrow f(x)=\displaystyle\int x\mathrm{\,d}x=\dfrac{x^2}{2}+C$.\\
		Mà ${{f(2)=1}} \Rightarrow C=-1$.\\
		Vậy $f(x)=\dfrac{x^2}{2}-1\Rightarrow  \displaystyle\int\limits_0^2 f(x)\mathrm{\,d}x=-\dfrac{2}{3}$.\\ Cách 2. Áp dụng bất đẳng thức AM – GM ta có\\
		$[f'(x)]^4+x^4+x^4+x^4\geq 4x^3f'(x)$.\\
		Do vậy $\displaystyle\int\limits_0^2[f'(x)]^4\mathrm{\,d}x+3\displaystyle\int\limits_0^2 x^4\mathrm{\,d}x\geq 4\displaystyle\int\limits_0^2 x^3f'(x)\mathrm{\,d}x$. Mà giá trị của hai vế bằng nhau, có nghĩa là dấu \lq\lq=\rq\rq\, xảy ra nên $f'(x)=x$. (Làm tiếp như trên).}
\end{ex}
\paragraph{Vấn đề 12. Kỹ thuật đánh giá AM-GM}
\begin{ex}%[2D3G2-4]%[Trần Ngọc Phú]%Câu 98.
	Cho hàm số $f(x)$ nhận giá trị dương và có đạo hàm $f'(x)$ liên tục trên $[0;1]$, thỏa mãn $f(1)=\mathrm{e}\cdot f(0)$ và $\displaystyle\int\limits_0^1\dfrac{\mathrm{\,d}x}{f^2(x)}+\displaystyle\int\limits_0^1[f'(x)]^2\mathrm{\,d}x\leq 2$. Mệnh đề nào sau đây đúng?
	\choice
	{$f(1)=\sqrt{\dfrac{2\mathrm{e}}{\mathrm{e}-1}}$}
	{$f(1)=\dfrac{2(\mathrm{e}-2)}{\mathrm{e}-1}$}
	{\True $f(1)=\sqrt{\dfrac{2\mathrm{e}^2}{\mathrm{e}^2-1}}$}
	{$f(1)=\sqrt{\dfrac{2(\mathrm{e}-2)}{\mathrm{e}-1}}$}
	\loigiai{
		Ta có
		\begin{eqnarray*}
			\displaystyle\int\limits_0^1\dfrac{\mathrm{\,d}x}{f^2(x)}+\displaystyle\int\limits_0^1[f'(x)]^2\mathrm{\,d}x=\displaystyle\int\limits_0^1\left[\dfrac{1}{f^2(x)}+[f'(x)]^2\right]\mathrm{\,d}x&\overset{AM-GM}{\geq}& 2\displaystyle\int\limits_0^1\dfrac{f'(x)}{f(x)}\mathrm{\,d}x\\
			&\ge&2\ln\left|f(x)\right|\bigg|_0^1\\&\ge&2\ln\left|f(1)\right|-2\ln\left|f(0)\right|\\&\ge&2\ln\left|\dfrac{f(1)}{f(0)}\right|=2\ln e=2.
		\end{eqnarray*}
		Mà $\displaystyle\int\limits_0^1\dfrac{\mathrm{\,d}x}{f^2(x)}+\displaystyle\int\limits_0^1[f'(x)]^2\mathrm{\,d}x\leq 2$ nên dấu \lq\lq=\rq\rq\, xảy ra, tức là $f'(x)=\dfrac{1}{f(x)}\Leftrightarrow f(x)f'(x)=1$.\\
		Suy ra $\displaystyle\int f(x)f'(x)\mathrm{\,d}x=\displaystyle\int x\mathrm{\,d}x\Leftrightarrow\dfrac{f^2(x)}{2}=x+C\Rightarrow  f(x)=\sqrt{2x+2C}$.\\
		Theo giả thiết $f(1)=\mathrm{e}\cdot f(0)$ nên ta có $\sqrt{2+2C}=\mathrm{e}\sqrt{2C}\Leftrightarrow 2+2C=\mathrm{e}^22C\Leftrightarrow C=\dfrac{1}{\mathrm{e}^2-1}$.\\
		Do đó $f(x)=\sqrt{2x+\dfrac{2}{\mathrm{e}^2-1}}\Rightarrow f(1)=\sqrt{2+\dfrac{2}{\mathrm{e}^2-1}}=\sqrt{\dfrac{2\mathrm{e}^2}{\mathrm{e}^2-1}}$.}
\end{ex}
\begin{ex}%[2D3G2-4]%[Trần Ngọc Phú]%Câu 99.
	Cho hàm số $f(x)$ nhận giá trị dương trên $[0;1]$, có đạo hàm dương và liên tục trên $[0;1]$, thỏa mãn $f(0)=1$ và $\displaystyle\int\limits_0^1\left[f^3(x)+4[f'(x)]^3\right]\mathrm{\,d}x\leq 3\displaystyle\int\limits_0^1 f'(x)f^2(x)\mathrm{\,d}x$. Tính $I=\displaystyle\int\limits_0^1 f(x)\mathrm{\,d}x$. 
	\choice
	{\True $I=2(\sqrt{\mathrm{e}}-1)$}
	{$I=2\left(\mathrm{e}^2-1\right)$}
	{$I=\dfrac{\sqrt{\mathrm{e}}-1}{2}$}
	{$I=\dfrac{\mathrm{e}^2-1}{2}$}
	\loigiai{Áp dụng bất đẳng thức $AM-GM$ cho ba số dương ta có\\
		\centerline{	$f^3(x)+4[f'(x)]^3=4[f'(x)]^3+\dfrac{f^3(x)}{2}+\dfrac{f^3(x)}{2}\geq 3\sqrt[3]{4[f'(x)]^3\cdot\dfrac{f^3(x)}{2}\cdot\dfrac{f^3(x)}{2}}=3f'(x)f^2(x)$.}\\
		Suy ra $\displaystyle\int\limits_0^1\left[f^3(x)+4[f'(x)]^3\right]\mathrm{\,d}x\geq 3\displaystyle\int\limits_0^1 f'(x)f^2(x)\mathrm{\,d}x$.\\
		Mà $\displaystyle\int\limits_0^1\left[f^3(x)+4[f'(x)]^3\right]\mathrm{\,d}x\leq 3\displaystyle\int\limits_0^1 f'(x)f^2(x)\mathrm{\,d}x$ nên dấu \lq\lq=\rq\rq\, xảy ra.\\ Tức là\\
		\centerline{	$4[f'(x)]^3=\dfrac{f^3(x)}{2}=\dfrac{f^3(x)}{2}\Leftrightarrow f'(x)=\dfrac{1}{2}f(x)$.}\\
		Suy ra $ \dfrac{f'(x)}{f(x)}=\dfrac{1}{2}\Rightarrow\displaystyle\int\dfrac{f'(x)}{f(x)}\mathrm{\,d}x=\dfrac{1}{2}\displaystyle\int\mathrm{\,d}x\Rightarrow\ln\left|f(x)\right|=\dfrac{1}{2}x+C\Rightarrow  f(x)=\mathrm{e}^{\tfrac{1}{2}x+C}$.\\
		Theo giả thiết $f(0)=1\Rightarrow C=0\Rightarrow f(x)=\mathrm{e}^{\tfrac{1}{2}x}\Rightarrow  \displaystyle\int\limits_0^1 f(x)\mathrm{\,d}x=2(\sqrt{\mathrm{e}}-1)$.}
\end{ex}
\begin{ex}%[2D3G2-4]%[Trần Ngọc Phú]%Câu 100. 
	Cho hàm số $f(x)$ nhận giá trị dương trên $[0;1]$, có đạo hàm dương liên và tục trên $[0;1]$, thỏa mãn $\displaystyle\int\limits_0^1\sqrt{\dfrac{xf'(x)}{f(x)}}\mathrm{\,d}x\geq 1$ và $f(0)=1,\, f(1)=\mathrm{e}^2$. Tính giá trị của $f\left(\dfrac{1}{2}\right)$. 
	\choice
	{$f\left(\dfrac{1}{2}\right)=1$}
	{$f\left(\dfrac{1}{2}\right)=4$}
	{\True $f\left(\dfrac{1}{2}\right)=\sqrt{\mathrm{e}}$}
	{$f\left(\dfrac{1}{2}\right)=\mathrm{e}$}
	\loigiai{Hàm dưới dấu tích phân là $\sqrt{\dfrac{xf'(x)}{f(x)}}=\sqrt{x}\cdot\sqrt{\dfrac{f'(x)}{f(x)}},\forall x\in[0;1]$.\\ Điều này làm ta liên tưởng đến đạo hàm đúng $\dfrac{f'(x)}{f(x)}$, muốn vậy ta phải đánh giá theo $AM-GM$ như sau:\\
		$\dfrac{f'(x)}{f(x)}+mx\geq 2\sqrt{m}\cdot\sqrt{\dfrac{xf'(x)}{f(x)}}$ với $m\geq 0$ và $x\in[0;1]$.\\
		Do đó ta cần tìm tham số $m\geq 0$ sao cho.\\
		$\displaystyle\int\limits_0^1\left[\dfrac{f'(x)}{f(x)}+mx\right]\mathrm{\,d}x\geq 2\sqrt{m}\cdot\displaystyle\int\limits_0^1\sqrt{\dfrac{xf'(x)}{f(x)}}\mathrm{\,d}x$.\\
		hay
		$\ln\left|f(x)\right|\bigg|_0^1 +m\dfrac{x^2}{2}\bigg|_0^1\geq 2\sqrt{m}\cdot 1\Leftrightarrow\ln\left|f(1)\right|-\ln\left|f(0)\right|+\dfrac{m}{2}\geq 2\sqrt{m}\Leftrightarrow 2-0+\dfrac{m}{2}\geq 2\sqrt{m}$.\\
		Để dấu \lq\lq=\rq\rq\, xảy ra thì ta cần có $2-0+\dfrac{m}{2}=2\sqrt{m}\Leftrightarrow m=4$.\\
		Với $m=4$ thì đẳng thức xảy ra nên $\dfrac{f'(x)}{f(x)}=4x$.\\
		$\Rightarrow  \displaystyle\int\dfrac{f'(x)}{f(x)}\mathrm{\,d}x=\displaystyle\int 4x\mathrm{\,d}x\Leftrightarrow\ln\left|f(x)\right|=2x^2+C\Rightarrow f(x)=\mathrm{e}^{2x^2+C}$.\\
		Theo giả thiết $\heva{&f(0)=1\\&f(1)=\mathrm{e}^2}\Rightarrow C=0\Rightarrow  f(x)=\mathrm{e}^{2x^2}\Rightarrow  f\left(\dfrac{1}{2}\right)=\sqrt{\mathrm{e}}$.\\ Cách 2. Theo Holder.\\
		$1^2\leq\left(\displaystyle\int\limits_0^1\sqrt{\dfrac{xf'(x)}{f(x)}}\mathrm{\,d}x\right)^2=\left(\displaystyle\int\limits_0^1\sqrt{x}\cdot\sqrt{\dfrac{f'(x)}{f(x)}}\mathrm{\,d}x\right)^2\leq\displaystyle\int\limits_0^1 x\mathrm{\,d}x\cdot\displaystyle\int\limits_0^1\dfrac{f'(x)}{f(x)}\mathrm{\,d}x=\dfrac{1}{2}\cdot\ln\dfrac{f(1)}{f(0)}=1$.\\
		Vậy đẳng thức xảy ra nên ta có $\dfrac{f'(x)}{f(x)}=kx,$ thay vào $\displaystyle\int\limits_0^1\sqrt{\dfrac{xf'(x)}{f(x)}}\mathrm{\,d}x=1$ ta được $k=4$.\\
		Suy ra $\dfrac{f'(x)}{f(x)}=4x$. (làm tiếp như trên).}
\end{ex}
\begin{ex}%[2D3G2-4]%[Trần Ngọc Phú]%Câu 101. 
	Cho hàm số $f(x)$ có đạo hàm liên tục trên $[0;1]$, thỏa mãn $\displaystyle\int\limits_0^1[f(x)f'(x)]^2\mathrm{\,d}x\leq 1$ và $f(0)=1,\, f(1)=\sqrt{3}$. Tính giá trị của $f\left(\dfrac{1}{2}\right)$. 
	\choice
	{\True $f\left(\dfrac{1}{2}\right)=\sqrt{2}$}
	{$f\left(\dfrac{1}{2}\right)=3$}
	{$f\left(\dfrac{1}{2}\right)=\sqrt{\mathrm{e}}$}
	{$f\left(\dfrac{1}{2}\right)=\mathrm{e}$}
	\loigiai{
		Nhận thấy bài này ngược dấu bất đẳng thức với bài trên.\\
		Hàm dưới dấu tích phân là $\left[ f(x)f'(x)\right] ^2$. Điều này làm ta liên tưởng đến đạo hàm đúng $f(x)f'(x)$, muốn vậy ta phải đánh giá theo $AM-GM$ như sau:\\
		\centerline{$\left[ f(x)f'(x)\right] ^2+m\geq 2\sqrt{m}\cdot f(x)f'(x)$ với $m\geq 0$.}\\
		Do đó ta cần tìm tham số $m\geq 0$ sao cho.\\
		\centerline{$\displaystyle\int\limits_0^1\left([f(x)f'(x)]^2+m\right)\mathrm{\,d}x\geq 2\sqrt{m}\displaystyle\int\limits_0^1 f(x)f'(x)\mathrm{\,d}x$.}\\
		hay\\
		\centerline{$1+m\geq 2\sqrt{m}\cdot\dfrac{f^2(x)}{2}\bigg|_0^1\Leftrightarrow 1+m\geq 2\sqrt{m}$.}\\
		Để dấu \lq\lq=\rq\rq\, xảy ra thì ta cần có $1+m=2\sqrt{m}\Leftrightarrow m=1$.\\
		Với $m=1$ thì đẳng thức xảy ra nên $[f(x)f'(x)]^2=1\Leftrightarrow\hoac{&f(x)f'(x)=1\\&f(x)f'(x)=-1.}$ \\
		$f(x)f'(x)=-1\Rightarrow  \displaystyle\int\limits_0^1 f(x)f'(x)\mathrm{\,d}x=-\displaystyle\int\limits_0^1\mathrm{\,d}x\Leftrightarrow\dfrac{f^2(x)}{2}\bigg|_0^1 =-x\bigg|_0^1\Leftrightarrow 1=-1$. (vô lý).\\
		$f(x)f'(x)=1\Rightarrow  \displaystyle\int f(x)f'(x)\mathrm{\,d}x=\displaystyle\int\mathrm{\,d}x\Leftrightarrow\dfrac{f^2(x)}{2}=x+C\Rightarrow  f(x)=\sqrt{2x+2C}$.\\
		Theo giả thiết $\heva{&f(0)=1\\&f(1)=\sqrt{3}}\Rightarrow C=\dfrac{1}{2}\Rightarrow  f(x)=\sqrt{2x+1}\Rightarrow  f\left(\dfrac{1}{2}\right)=\sqrt{2}$.\\
		Cách 2. Ta có $\displaystyle\int\limits_0^1 f(x)f'(x)\mathrm{\,d}x=\dfrac{f^2(x)}{2}\bigg|_0^1 =\dfrac{1}{2}\left[f^2(1)-f^2(0)\right]=1$.\\
		Theo Holder\\
		\centerline{$1^2=\left(\displaystyle\int\limits_0^1 1\cdot f(x)f'(x)\mathrm{\,d}x\right)^2\leq\displaystyle\int\limits_0^1 1^2\mathrm{\,d}x\cdot\displaystyle\int\limits_0^1\left[ f(x)f'(x)\right] ^2\mathrm{\,d}x\leq 1\cdot 1=1$.}\\
		Vậy đẳng thức xảy ra nên ta có $f'(x)f(x)=k,$ thay vào $\displaystyle\int\limits_0^1 f(x)f'(x)\mathrm{\,d}x=1$ ta được $k=1$.\\ Suy ra $f'(x)f(x)=1$. (làm tiếp như trên).}
\end{ex}
\begin{ex}%[2D3G2-4]%[Trần Ngọc Phú]%Câu 102. 
	Cho hàm số $f(x)$ nhận giá trị dương và có đạo hàm $f'(x)$ liên tục trên $[1;2]$, thỏa mãn $\displaystyle\int\limits_1^2\dfrac{[f'(x)]^2}{xf(x)}\mathrm{\,d}x\leq 24$ và $f(1)=1,\, f(2)=16$. Tính giá trị của $f(\sqrt{2})$. 
	\choice
	{$f(\sqrt{2})=1$}
	{$f(\sqrt{2})=\sqrt{2}$}
	{$f(\sqrt{2})=2$}
	{\True $f(\sqrt{2})=4$}
	\loigiai{
		Hàm dưới dấu tích phân là $\dfrac{\left[ f'(x)\right] ^2}{xf(x)}=\dfrac{1}{x}\cdot\dfrac{\left[ f'(x)\right] ^2}{f(x)}$.\\ Điều này làm ta liên tưởng đến đạo hàm đúng $\dfrac{f'(x)}{\sqrt{f(x)}}$, muốn vậy ta phải đánh giá theo $AM-GM$ như sau:\\
		\centerline{$\dfrac{[f'(x)]^2}{xf(x)}+mx\geq 2\sqrt{m}\dfrac{f'(x)}{\sqrt{f(x)}}$ với $m\geq 0$ và $x\in[1;2]$.}\\
		Do đó ta cần tìm tham số $m\geq 0$ sao cho\\
		\centerline{$\displaystyle\int\limits_1^2\left(\dfrac{[f'(x)]^2}{xf(x)}+mx\right)\mathrm{\,d}x\geq 2\sqrt{m}\displaystyle\int\limits_1^2\dfrac{f'(x)}{\sqrt{f(x)}}\mathrm{\,d}x$.}\\
		hay\\
		\centerline{$24+\dfrac{2m}{3}\geq 4\sqrt{m}\sqrt{f(x)}\bigg|_1^2\Leftrightarrow 24+\dfrac{2m}{3}\geq 4\sqrt{m}\left[\sqrt{f(2)}-\sqrt{f(1)}\right]\Leftrightarrow 24+\dfrac{2m}{3}\geq 12\sqrt{m}\Leftrightarrow m=16$.}\\
		Để dấu \lq\lq=\rq\rq\, xảy ra thì ta cần có $24+\dfrac{2m}{3}=12\sqrt{m}\Leftrightarrow m=16$.\\
		Với $m=16$ thì đẳng thức xảy ra nên $\dfrac{[f'(x)]^2}{xf(x)}=16x\Rightarrow\dfrac{f'(x)}{2\sqrt{f(x)}}=2x$.\\
		$\Rightarrow  \displaystyle\int\dfrac{f'(x)}{2\sqrt{f(x)}}\mathrm{\,d}x=\displaystyle\int 2x\mathrm{\,d}x\Leftrightarrow\sqrt{f(x)}=x^2+C\Rightarrow  f(x)=\left(x^2+C\right)^2$.\\
		Theo giả thiết $\heva{&f(1)=1\\&f(2)=16}\Rightarrow C=0\Rightarrow  f(x)=x^4\Rightarrow  f(\sqrt{2})=4$.\\ Cách 2. Ta có $\displaystyle\int\limits_1^2\dfrac{f'(x)}{\sqrt{f(x)}}\mathrm{\,d}x=2\cdot\displaystyle\int\limits_1^2\dfrac{f'(x)}{2\sqrt{f(x)}}\mathrm{\,d}x=2\sqrt{f(x)}\bigg|_1^2 =2\left[\sqrt{f(2)}-\sqrt{f(1)}\right]=6$.\\
		Theo Holder.\\
		\centerline{$6^2=\left(\displaystyle\int\limits_1^2\dfrac{f'(x)}{\sqrt{f(x)}}\mathrm{\,d}x\right)^2=\left(\displaystyle\int\limits_1^1\sqrt{x}\cdot\dfrac{f'(x)}{\sqrt{xf(x)}}\mathrm{\,d}x\right)^2\leq\displaystyle\int\limits_1^2 x\mathrm{\,d}x\cdot\displaystyle\int\limits_1^2\dfrac{[f'(x)]^2}{xf(x)}\mathrm{\,d}x\leq\dfrac{x^2}{2}\bigg|_1^2\cdot 24=36$.}\\
		Vậy đẳng thức xảy ra nên ta có $\dfrac{f'(x)}{\sqrt{xf(x)}}=k\sqrt{x}\Leftrightarrow\dfrac{f'(x)}{\sqrt{f(x)}}=kx$.\\
		Thay vào $\displaystyle\int\limits_1^2\dfrac{f'(x)}{\sqrt{f(x)}}\mathrm{\,d}x=6$ ta được $k=4$. Suy ra $\dfrac{f'(x)}{\sqrt{f(x)}}=4x$. (làm tiếp như trên).}
\end{ex}
\paragraph{Vấn đề 13. Tìm GTLN-GTNN của tích phân}
\begin{ex}%[2D3G2-4]%[Trần Ngọc Phú]%Câu 103. 
	Cho hàm số $f(x)$ liên tục trên $\mathbb{R}$, có đạo hàm cấp hai thỏa mãn $x.f''(x)\geq\mathrm{e}^x+x$ và $f'(2)=2\mathrm{e}, \,f(0)=\mathrm{e}^2$. Mệnh đề nào sau đây là đúng?
	\choice
	{\True $f(2)\leq 4\mathrm{e}-1$}
	{$f(2)\leq 2\mathrm{e}+\mathrm{e}^2$}
	{$f(2)\leq\mathrm{e}^2-2\mathrm{e}$}
	{$f(2)>12$}
	\loigiai{Từ giả thiết $x.f''(x)\geq\mathrm{e}^x+x$ ta có $\displaystyle\int\limits_0^2 x\cdot f''(x)\mathrm{\,d}x\geq\displaystyle\int\limits_0^2\left(\mathrm{e}^x+x\right)\mathrm{\,d}x$. $(1)$.\\
		Đặt $\heva{&u=x\\&\mathrm{\,d}v=f''(x)}\Rightarrow\heva{&\mathrm{\,d}u=\mathrm{\,d}x\\&v=f'(x).}$ \\ 		
		Khi đó
		\begin{eqnarray*}
			(1)&\Leftrightarrow& x\cdot f'(x)\bigg|_0^2 -\displaystyle\int\limits_0^2 f'(x)\mathrm{\,d}x\geq\left(\mathrm{e}^x+\dfrac{x^2}{2}\right)\bigg|_0^2.\\
			&\Leftrightarrow& x\cdot f'(x)\bigg|_0^2 -f(x)\bigg|_0^2\geq\left(\mathrm{e}^x+\dfrac{x^2}{2}\right)\bigg|_0^2\\&\Leftrightarrow&\left[2\cdot f'(2)-0\cdot f'(0)\right]-[f(2)-f(0)]\geq\mathrm{e}^2+2-1 \\
			&\Leftrightarrow& f(2)\leq 4\mathrm{e}-1 \, (\text{do}\, f'(2)=2\mathrm{e}, f(0)=\mathrm{e}^2).
		\end{eqnarray*}
	}
\end{ex}
\begin{ex}%[2D3G2-4]%[Trần Ngọc Phú]%Câu 104. 
	Cho hàm số $f(x)$ dương và liên tục trên $[1;3],$ thỏa $\max\limits_{[1;3]} f(x)=2,\min\limits_{[1;3]} f(x)=\dfrac{1}{2}$ và biểu thức $S=\displaystyle\int\limits_1^3 f(x)\mathrm{\,d}x\cdot\displaystyle\int\limits_1^3\dfrac{1}{f(x)}\mathrm{\,d}x$ đạt giá trị lớn nhất, khi đó hãy tính $I=\displaystyle\int\limits_1^3 f(x)\mathrm{\,d}x$. 
	\choice
	{$\dfrac{3}{5}$}
	{$\dfrac{7}{5}$}
	{$\dfrac{7}{2}$}
	{\True $\dfrac{5}{2}$}
	\loigiai{Từ giả thiết ta có $\dfrac{1}{2}\leq f(x)\leq 2$, suy ra $f(x)+\dfrac{1}{f(x)}\leq\dfrac{5}{2}$.\\
		Suy ra $\displaystyle\int\limits_1^3\left[f(x)+\dfrac{1}{f(x)}\right]\mathrm{\,d}x\leq\displaystyle\int\limits_1^3\dfrac{5}{2}\mathrm{\,d}x\Leftrightarrow\displaystyle\int\limits_1^3 f(x)\mathrm{\,d}x+\displaystyle\int\limits_1^3\dfrac{1}{f(x)}\mathrm{\,d}x\leq 5\Leftrightarrow\displaystyle\int\limits_1^3\dfrac{1}{f(x)}\mathrm{\,d}x\leq 5-\displaystyle\int\limits_1^3 f(x)\mathrm{\,d}x$.\\
		Khi đó $S=\displaystyle\int\limits_1^3 f(x)\mathrm{\,d}x\cdot\displaystyle\int\limits_1^3\dfrac{1}{f(x)}\mathrm{\,d}x\leq\displaystyle\int\limits_1^3 f(x)\mathrm{\,d}x\cdot\left(5-\displaystyle\int\limits_1^3 f(x)\mathrm{\,d}x\right)\leq\dfrac{25}{4}$.\\
		(dạng $t(5-t)=-t^2+5t=-\left(t-\dfrac{5}{2}\right)^2+\dfrac{25}{4}\leq\dfrac{25}{4}$).\\
		Dấu \lq\lq=\rq\rq\, xảy ra khi và chỉ khi $\displaystyle\int\limits_1^3 f(x)\mathrm{\,d}x=\dfrac{5}{2}$.}
\end{ex}
\begin{ex}%[2D3G2-4]%[Trần Ngọc Phú]%Câu 105. 
	Cho hàm số $f(x)$ có đạo hàm liên tục trên $\mathbb{R}$, thỏa mãn $f(x)+f'(x)\leq 1$ với mọi $x\in\mathbb{R}$ và $f(0)=0$. Giá trị lớn nhất của $f(1)$ bằng
	\choice
	{$\mathrm{e}-1$}
	{\True $\dfrac{\mathrm{e}-1}{\mathrm{e}}$}
	{$\dfrac{\mathrm{e}}{\mathrm{e}-1}$}
	{$\mathrm{e}$}
	\loigiai{Từ giả thiết $f(x)+f'(x)\leq 1$, nhân thêm hai vế cho $\mathrm{e}^x$ để thu được đạo hàm đúng là
		\begin{center}
			$\mathrm{e}^x\cdot f(x)+\mathrm{e}^x\cdot f'(x)\leq\mathrm{e}^x,\forall x\in\mathbb{R}\Leftrightarrow\left[\mathrm{e}^x\cdot f(x)\right]'\leq\mathrm{e}^x,\forall x\in\mathbb{R}$.
		\end{center}
		Suy ra $\displaystyle\int\limits_0^1\left[\mathrm{e}^x\cdot f(x)\right]'\mathrm{\,d}x\leq\displaystyle\int\limits_0^1\mathrm{e}^x\mathrm{\,d}x\Leftrightarrow\left[\mathrm{e}^xf(x)\right]\bigg|_0^1\leq e-1\Leftrightarrow\left[\mathrm{e}\cdot f(1)-1\cdot f(0)\right]\leq \mathrm{e}-1$.\\
		Ta có   ${{f(0)=0}}\Rightarrow f(1)\leq\dfrac{\mathrm{e}-1}{\mathrm{e}}$.}
\end{ex}
\begin{ex}%[2D3G2-4]%[Trần Ngọc Phú]%Câu 106. 
	Cho hàm số $f(x)$ nhận giá trị dương và có đạo hàm $f'(x)$ liên tục trên $[0;1]$, thỏa mãn $f(1)=2018f(0)$. Giá trị nhỏ nhất của biểu thức $M=\displaystyle\int\limits_0^1\dfrac{1}{[f(x)]^2}\mathrm{\,d}x+\displaystyle\int\limits_0^1[f'(x)]^2\mathrm{\,d}x$ bằng
	\choice
	{$\ln 2018$}
	{\True $2\ln 2018$}
	{$m=2e$}
	{$m=2018e$}
	\loigiai{
		Áp dụng bất đẳng thức Cauchy, ta được\\
		\centerline{$M=\displaystyle\int\limits_0^1\dfrac{1}{[f(x)]^2}\mathrm{\,d}x+ \displaystyle\int\limits_0^1[f'(x)]^2\mathrm{\,d}x\geq 2\displaystyle\int\limits_0^1\dfrac{f'(x)}{f(x)}\mathrm{\,d}x= 2\ln |f(x)|\bigg|_0^1 =2\ln\dfrac{f(1)}{f(0)}=2\ln 2018$.}}
\end{ex}
\begin{ex}%[2D3G2-4]%[Trần Ngọc Phú]%Câu 107. 
	Cho hàm số $f(x)$ có đạo hàm liên tục trên $[0;1]$ và $\displaystyle\int\limits_0^1(1-x)^2f'(x)\mathrm{\,d}x=-\dfrac{1}{3}$. Giá trị nhỏ nhật của biểu thức $\displaystyle\int\limits_0^1[f(x)]^2\mathrm{\,d}x-f(0)$ bằng
	\choice
	{$\dfrac{1}{3}$}
	{$\dfrac{2}{3}$}
	{$-\dfrac{1}{3}$}
	{\True $-\dfrac{2}{3}$}
	\loigiai{
		Tích phân từng phần $\displaystyle\int\limits_0^1(1-x)^2f'(x)\mathrm{\,d}x=-\dfrac{1}{3}$, ta được $f(0)-\dfrac{1}{3}=2\displaystyle\int\limits_0^1(1-x)f(x)\mathrm{\,d}x$.\\
		Áp dụng bất đẳng thức Cauchy, ta được\\
		\centerline{$2\displaystyle\int\limits_0^1(1-x)f(x)\mathrm{\,d}x\leq\displaystyle\int\limits_0^1(1-x)^2\mathrm{\,d}x+\displaystyle\int\limits_0^1[f(x)]^2\mathrm{\,d}x$.}\\
		Từ đó suy ra\\ \centerline{$\displaystyle\int\limits_0^1[f(x)]^2\mathrm{\,d}x\geq 2\displaystyle\int\limits_0^1(1-x)f(x)\mathrm{\,d}x-\displaystyle\int\limits_0^1(1-x)^2\mathrm{\,d}x$
			$ \Leftrightarrow\displaystyle\int\limits_0^1[f(x)]^2\mathrm{\,d}x\geq f(0)-\dfrac{1}{3}+\dfrac{(1-x)^3}{3}\bigg|_0^1 $.}\\
		Vậy $\displaystyle\int\limits_0^1[f(x)]^2\mathrm{\,d}x-f(0)\geq-\dfrac{2}{3}$.}
\end{ex}
\begin{ex}%[2D3G2-4]%[Trần Ngọc Phú]%Câu 108. 
	Cho hàm số $f(x)$ liên tục trên $[0; 1]$ thỏa mãn $\displaystyle\int\limits_0^1 xf(x)\mathrm{\,d}x=0$ và $\max \limits_{[0; 1]}\left|f(x)\right|=1$. Tích phân $\displaystyle\int\limits_0^1\mathrm{e}^xf(x)\mathrm{\,d}x$ thuộc khoảng nào trong các khoảng sau đây?
	\choice
	{$\left(-\infty;-\dfrac{5}{4}\right)$}
	{$\left(\dfrac{3}{2}; e-1\right)$}
	{\True $\left(-\dfrac{5}{4};\dfrac{3}{2}\right)$}
	{$(e-1;+\infty)$}
	\loigiai{
		Với mỗi số thực $\alpha\in\mathbb{R}$ ta có
		\begin{eqnarray*}
			\left|\displaystyle\int\limits_0^1\mathrm{e}^xf(x)\mathrm{\,d}x\right|&=&\left|\displaystyle\int\limits_0^1\mathrm{e}^xf(x)\mathrm{\,d}x-\displaystyle\int\limits_0^1\alpha xf(x)\mathrm{\,d}x\right|.\\
			&=&\left|\displaystyle\int\limits_0^1 f(x)\left(\mathrm{e}^x-\alpha x\right)\mathrm{\,d}x\right|\\&\leq&\displaystyle\int\limits_0^1\left|f(x)\right|\cdot\left|\mathrm{e}^x-\alpha x\right|\mathrm{\,d}x\\&\leq&\displaystyle\int\limits_0^1\left|\mathrm{e}^x-\alpha x\right|\mathrm{\,d}x.
		\end{eqnarray*}
		Suy ra $\left|\displaystyle\int\limits_0^1\mathrm{e}^xf(x)\mathrm{\,d}x\right|\leq\min\limits_{\alpha\in\mathbb{R}}\displaystyle\int\limits_0^1\left|\mathrm{e}^x-\alpha x\right|\mathrm{\,d}x\leq\min\limits_{\alpha\in[0;1]}\displaystyle\int\limits_0^1\left|\mathrm{e}^x-\alpha x\right|\mathrm{\,d}x=\min\limits_{\alpha\in[0;1]}\left\{e-1-\dfrac{\alpha}{2}\right\}=\mathrm{e}-\dfrac{3}{2}$.}
\end{ex}
\begin{ex}%[2D3G2-4]%[Trần Ngọc Phú]%Câu 109. 
	Cho hàm số $f(x)$ nhận giá trị không âm và liên tục trên $[0;1]$. Đặt $g(x)=1+\displaystyle\int\limits_0^x f(t)\mathrm{\,d}t$. Biết $g(x)\leq\sqrt{f(x)}$ với mọi $x\in[0;1]$, tích phân $\displaystyle\int\limits_0^1\dfrac{1}{g(x)}\mathrm{\,d}x$ có giá trị lớn nhất bằng
	\choice
	{$\dfrac{1}{3}$}
	{\True $\dfrac{1}{2}$}
	{$\dfrac{\sqrt{2}}{2}$}
	{$1$}
	\loigiai{Từ giả thiết $g(x)=1+\displaystyle\int\limits_0^x f(t)\mathrm{\,d}t,$ ta có $\heva{&g(0)=1\\&g'(x)=f(x)}$ và $g(x)>0,\forall x\in[0;1]$.\\
		Theo giả thiết $g(x)\leq\sqrt{f(x)}\Rightarrow  g(x)\leq\sqrt{g'(x)}\Leftrightarrow\dfrac{\sqrt{g'(x)}}{g(x)}\geq 1\Leftrightarrow\dfrac{g'(x)}{g^2(x)}\geq 1$.\\
		Suy ra $\displaystyle\int\limits_0^t\dfrac{g'(x)}{g^2(x)}\mathrm{\,d}x\geq\displaystyle\int\limits_0^t 1\mathrm{\,d}x,\forall t\in[0;1]\Leftrightarrow-\dfrac{1}{g(x)}\bigg|_0^t\geq x\bigg|_0^t\Leftrightarrow-\dfrac{1}{g(t)}+\dfrac{1}{g(0)}\geq t\Leftrightarrow\dfrac{1}{g(t)}\leq 1-t$.\\
		Do đó $\displaystyle\int\limits_0^1\dfrac{1}{g(x)}\mathrm{\,d}x\leq\displaystyle\int\limits_0^1(1-x)\mathrm{\,d}x=\dfrac{1}{2}$.}
\end{ex}
\begin{ex}%[2D3G2-4]%[Trần Ngọc Phú]%Câu 110. 
	Cho hàm số $f(x)$ nhận giá trị không âm và liên tục trên đoạn $[0;1],$ thỏa mãn $f^2(x)\leq 1+3\displaystyle\int\limits_0^x f(t)\mathrm{\,d}t=g(x)$ với mọi $x\in[0;1]$, tích phân $\displaystyle\int\limits_0^1\sqrt{g(x)}\mathrm{\,d}x$ có giá trị lớn nhất bằng
	\choice
	{$\dfrac{4}{3}$}
	{\True $\dfrac{7}{4}$}
	{$\dfrac{9}{5}$}
	{$\dfrac{5}{2}$}
	\loigiai{
		Từ giả thiết $g(x)=1+3\displaystyle\int\limits_0^x f(t)\mathrm{\,d}t,$ ta có $\heva{&g(0)=1\\&g'(x)=3f(x)}$ và $g(x)>0,\forall x\in[0;1]$.\\
		Theo giả thiết $g(x)\geq f^2(x)\Rightarrow  g(x)\geq\dfrac{[g'(x)]^2}{9}\Leftrightarrow\dfrac{g'(x)}{2\sqrt{g(x)}}\leq\dfrac{3}{2}$.\\
		Suy ra
		\begin{eqnarray*}
			\displaystyle\int\limits_0^t\dfrac{g'(x)}{2\sqrt{g(x)}}\mathrm{\,d}x\leq \displaystyle\int\limits_0^t\dfrac{3}{2}\mathrm{\,d}x,\forall t\in[0;1]&\Leftrightarrow&\sqrt{g(x)}\bigg|_0^t\leq\dfrac{3}{2}x\bigg|_0^t\\ &\Leftrightarrow&\sqrt{g(t)}-\sqrt{g(0)}\leq\dfrac{3}{2}t\\&\Leftrightarrow&\sqrt{g(t)}\leq\dfrac{3}{2}t+1.
		\end{eqnarray*}
		Do đó $\displaystyle\int\limits_0^1\sqrt{g(x)}\mathrm{\,d}x\leq\displaystyle\int\limits_0^1\left(\dfrac{3}{2}x+1\right)\mathrm{\,d}x=\dfrac{7}{4}$.}
\end{ex}
\begin{ex}%[2D3G2-4]%[Trần Ngọc Phú]%Câu 111. 
	Cho hàm số $f(x)$ nhận giá trị không âm và liên tục trên đoạn $[0;1],$ thỏa mãn $f(x)\leq 2018+2\displaystyle\int\limits_0^x f(t)\mathrm{\,d}t$ với mọi $x\in[0;1]$. Biết giá trị lớn nhất của tích phân $\displaystyle\int\limits_0^1 f(x)\mathrm{\,d}x$ có dạng $a\mathrm{e}^2+b$ với $a, b\in\mathbb{Z}$. Tính $a+b$. 
	\choice
	{\True $0$}
	{$1009$}
	{$2018$}
	{$2020$}
	\loigiai{
		Đặt $g(x)=2018+2\displaystyle\int\limits_0^x f(t)\mathrm{\,d}t,$ ta có $\heva{&g(0)=2018\\&g'(x)=2f(x)}$ và $g(x)>0,\forall x\in[0;1]$.\\
		Theo giả thiết $g(x)\geq f(x)\Rightarrow  g(x)\geq\dfrac{g'(x)}{2}\Leftrightarrow\dfrac{g'(x)}{g(x)}\leq 2$.\\
		Suy ra
		\begin{eqnarray*}
			\displaystyle\int\limits_0^t\dfrac{g'(x)}{g(x)}\mathrm{\,d}x\leq\displaystyle\int\limits_0^t 2\mathrm{\,d}x,\forall t\in[0;1]&\Leftrightarrow&\ln\left|g(x)\right|\bigg|_0^t\leq 2x\bigg|_0^t\\
			&\Leftrightarrow&\ln g(t)-\ln g(0)\leq 2t\\&\Leftrightarrow&\ln g(t)\leq 2t+\ln 2018\\&\Leftrightarrow& g(t)\leq 2018\cdot\mathrm{e}^{2t}.
		\end{eqnarray*}
		Do đó $\displaystyle\int\limits_0^1 f(x)\mathrm{\,d}x\leq\displaystyle\int\limits_0^1 g(x)\mathrm{\,d}x\leq 2018\displaystyle\int\limits_0^1\mathrm{e}^{2x}\mathrm{\,d}x=1009\mathrm{e}^{2x}\bigg|_0^1 =1009\mathrm{e}^2-1009$.\\
		Vậy $a=1009,\,b=-1009$. Do đó $a+b=0$.}
\end{ex}
\begin{ex}%[2D3G2-4]%[Trần Ngọc Phú]%Câu 112. 
	Cho hàm số $f(x)$ nhận giá trị không âm và liên tục trên đoạn $[0;1]$. \break Đặt $g(x)=1+\displaystyle\int\limits_0^{x^2} f(t)\mathrm{\,d}t$. Biết $g(x)\geq 2xf(x^2)$ với mọi $x\in[0;1]$, tích phân $\displaystyle\int\limits_0^1 g(x)\mathrm{\,d}x$ có giá trị lớn nhất bằng
	\choice
	{$1$}
	{\True $e-1$}
	{$2$}
	{$e+1$}
	\loigiai{
		Từ giả thiết $g(x)=1+\displaystyle\int\limits_0^{x^2} f(t)\mathrm{\,d}t,$ ta có $\heva{&g(0)=1\\&g'(x)=2xf(x^2)}$ và $g(x)>0,\forall x\in[0;1]$.\\
		Theo giả thiết $g(x)\geq 2xf(x^2)\Rightarrow  g(x)\geq g'(x)\Leftrightarrow\dfrac{g'(x)}{g(x)}\leq 1$.\\
		Suy ra
		\begin{eqnarray*}
			\displaystyle\int\limits_0^t\dfrac{g'(x)}{g(x)}\mathrm{\,d}x\leq\displaystyle\int\limits_0^t 1\mathrm{\,d}x,\forall t\in[0;1]&\Leftrightarrow&\ln g(x)\bigg|_0^t\leq x\bigg|_0^t \\
			&\Leftrightarrow&\ln g(t)-\ln g(0)\leq t\\&\Leftrightarrow&\ln g(t)\leq t\\&\Leftrightarrow& g(t)\leq\mathrm{e}^t.
		\end{eqnarray*}
		Do đó $\displaystyle\int\limits_0^1 g(x)\mathrm{\,d}x\leq\displaystyle\int\limits_0^1\mathrm{e}^x\mathrm{\,d}x=e-1$.\\ Nhận xét.\\ Gọi $F(t)$ là một nguyên hàm của hàm số $f(t)$ trên đoạn $\left[0;x^2\right]$.\\
		Khi đó $g(x)=1+F(t)\bigg|_0^{x^2}=1+F(x^2)-F(0)\Rightarrow  g'(x)=\left[F(x^2)\right]'=(x^2)'F'(x^2)=2xf(x^2)$.}
\end{ex}
\begin{ex}%[2D3G2-4]%[Trần Ngọc Phú]%Câu 113. 
	Cho hàm số $f(x)$ có đạo hàm liên tục trên $[0;1]$, thỏa $f'(x)\geq f(x)>0,\forall x\in[0;1]$. Giá trị lớn nhất của biểu thức $f(0)\cdot\displaystyle\int\limits_0^1\dfrac{1}{f(x)}\mathrm{\,d}x$ bằng
	\choice
	{$1$}
	{\True $\dfrac{e-1}{e}$}
	{$\dfrac{e+1}{e}$}
	{$e-1$}
	\loigiai{
		Từ giả thiết $f'(x)\geq f(x)>0,\forall x\in[0;1]$ ta có $\dfrac{f'(x)}{f(x)}\geq 1,\forall x\in[0;1]$.\\
		Suy ra $\displaystyle\int\limits_0^t\dfrac{f'(x)}{f(x)}\mathrm{\,d}x\geq\displaystyle\int\limits_0^t 1\mathrm{\,d}x,\forall t\in[0;1]\Leftrightarrow\ln f(x)\bigg|_0^t\geq x\bigg|_0^t\Leftrightarrow\ln f(t)-\ln f(0)\geq t\Leftrightarrow f(t)\geq f(0)\mathrm{e}^t$.\\
		Do đó $f(0)\cdot\displaystyle\int\limits_0^1\dfrac{1}{f(x)}\mathrm{\,d}x\leq\displaystyle\int\limits_0^1\dfrac{1}{\mathrm{e}^x}\mathrm{\,d}x=\dfrac{e-1}{e}$.}
\end{ex}
\begin{ex}%[2D3G2-4]%[Trần Ngọc Phú]%Câu 114. 
	Cho hàm số $f(x)$ liên tục trên $[0;\pi],$ thỏa mãn $\displaystyle\int\limits_0^{\pi} f(x)\mathrm{\,d}x=\displaystyle\int\limits_0^{\pi}\cos xf(x)\mathrm{\,d}x=1$. Giá trị nhỏ nhất của tích phân $\displaystyle\int\limits_0^{\pi} f^2(x)\mathrm{\,d}x$ bằng
	\choice
	{$\dfrac{2}{\pi}$}
	{\True $\dfrac{3}{\pi}$}
	{$\dfrac{4}{\pi}$}
	{$\dfrac{3}{2\pi}$}
	\loigiai{
		Theo Holder\\
		\centerline{$(1)^2=\left[\displaystyle\int\limits_0^{\pi}\cos xf(x)\mathrm{\,d}x\right]^2\leq\displaystyle\int\limits_0^{\pi}\cos^2x\mathrm{\,d}x\cdot\displaystyle\int\limits_0^{\pi} f^2(x)\mathrm{\,d}x=\dfrac{\pi}{2}\cdot\displaystyle\int\limits_0^{\pi} f^2(x)\mathrm{\,d}x$.}\\
		Suy ra $\displaystyle\int\limits_0^{\pi} f^2(x)\mathrm{\,d}x\geq\dfrac{2}{\pi}$. (Đến đây bạn đọc có thể chọn phương án $\dfrac{2}{\pi}$).\\
		Dấu \lq\lq=\rq\rq\, xảy ra khi $f(x)=k\cos x$ thay vào $\displaystyle\int\limits_0^{\pi} f(x)\mathrm{\,d}x=1$ ta được\\
		\centerline{$1=\displaystyle\int\limits_0^{\pi} f(x)\mathrm{\,d}x=k\displaystyle\int\limits_0^{\pi}\cos x\mathrm{\,d}x=k\cdot\sin x\bigg|_0^{\pi}=0$.}\\
		Điều này hoàn toàn vô lý\\
		Ta có $\displaystyle\int\limits_0^{\pi} f(x)\mathrm{\,d}x=\displaystyle\int\limits_0^{\pi}\cos xf(x)\mathrm{\,d}x=1\Rightarrow  \heva{&a=\displaystyle\int\limits_0^{\pi} a\cos xf(x)\mathrm{\,d}x\\&b=\displaystyle\int\limits_0^{\pi} bf(x)\mathrm{\,d}x}$ với $\heva{&a, b\in\mathbb{R}\\&a^2+b^2>0.}$ \\
		Theo Holder\\
		\centerline{$(a+b)^2=\left[\displaystyle\int\limits_0^{\pi}(a\cos x+b)f(x)\mathrm{\,d}x\right]^2\leq\displaystyle\int\limits_0^{\pi}(a\cos x+b)^2\mathrm{\,d}x\displaystyle\int\limits_0^{\pi} f^2(x)\mathrm{\,d}x$.}\\
		Lại có\\
		\centerline{$\displaystyle\int\limits_0^{\pi}(a\cos x+b)^2\mathrm{\,d}x=\dfrac{1}{2}\pi\left(a^2+2b^2\right)$.}\\
		Từ đó suy ra $\displaystyle\int\limits_0^{\pi} f^2(x)\mathrm{\,d}x\geq\dfrac{2(a+b)^2}{\pi\left(a^2+2b^2\right)}$ với mọi $a, b\in\mathbb{R}$ và $a^2+b^2>0$.\\
		Do đó $\displaystyle\int\limits_0^{\pi} f^2(x)\mathrm{\,d}x\geq\dfrac{2}{\pi}\cdot\max\left\{\dfrac{(a+b)^2}{a^2+2b^2}\right\}=\dfrac{3}{\pi}$.\\
		Nhận xét: Ta nhân thêm $a,\,b$ vào giả thiết được gọi là phương pháp biến thiên hằng số.\\
		Cách tìm giá trị lớn nhất của $P=\dfrac{(a+b)^2}{a^2+2b^2}$ ta làm như sau:\\
		Nếu $b=0\Rightarrow  P=1$. (chính là đáp án sai mà mình đã làm ở trên).\\
		Nếu $b\neq 0\Rightarrow  P=\dfrac{(a+b)^2}{a^2+2b^2}=\dfrac{\left(\dfrac{a}{b}\right)^2+2\dfrac{a}{b}+1}{\left(\dfrac{a}{b}\right)^2+2}\overset{t=\dfrac{a}{b}}{=}\dfrac{t^2+2t+1}{t^2+2}$.\\ Tới đây ta khảo sát hàm số hoặc dùng MODE 7 dò tìm.\\ Kết quả thu được GTLN của $P$ bằng $\dfrac{3}{2}$ khi $t=2\Rightarrow  \dfrac{a}{b}=2\Leftrightarrow a=2b$.\\
		Vậy dấu \lq\lq$=$\rq\rq\, để bài toán xảy ra khi $\heva{&a=2b\\&f(x)=b(2\cos x+1)}$ thay ngược lại điều kiện, ta được\\
		\centerline{	$\displaystyle\int\limits_0^{\pi} b(2\cos x+1)\mathrm{\,d}x=1\Leftrightarrow b=\dfrac{1}{\pi}\Rightarrow  f(x)=\dfrac{2\cos x+1}{\pi}$.}\\
		Lúc này $\displaystyle\int\limits_0^{\pi} f^2(x)\mathrm{\,d}x=\displaystyle\int\limits_0^{\pi}\left(\dfrac{2\cos x+1}{\pi}\right)^\mathrm{\,d}x=\dfrac{3}{\pi}$.\\
		Cách khác. Đưa về bình phương.\\
		Hàm dưới dấu tích phân là $f^2(x), f(x),\cos xf(x)$ nên ta liến kết với $\left[f(x)+\alpha\cos x+\beta\right]^2$.\\
		Với mỗi số thực $\alpha,\beta$ ta có
		\begin{eqnarray*}
			\displaystyle\int\limits_0^{\pi}\left[f(x)+\alpha\cos x+\beta\right]^2&=&\displaystyle\int\limits_0^{\pi} f^2(x)\mathrm{\,d}x+2\displaystyle\int\limits_0^{\pi}\left(\alpha\cos x+\beta\right)f(x)\mathrm{\,d}x+\displaystyle\int\limits_0^{\pi}\left(\alpha\cos x+\beta\right)^2\mathrm{\,d}x.\\
			&=&\displaystyle\int\limits_0^{\pi} f^2(x)\mathrm{\,d}x+2\left(\alpha+\beta\right)+\dfrac{\pi}{2}\alpha^2+\pi\beta^2.
		\end{eqnarray*}
		Ta cần tìm $\alpha,\beta$ sao cho $2\left(\alpha+\beta\right)+\dfrac{\pi}{2}\alpha^2+\pi\beta^2$ đạt giá trị nhỏ nhất. Ta có\\
		\centerline{	$2\left(\alpha+\beta\right)+\dfrac{\pi}{2}\alpha^2+\pi\beta^2=\dfrac{\pi}{2}\left(\alpha+\dfrac{2}{\pi}\right)^2+\pi\left(\beta+\dfrac{1}{\pi}\right)^2-\dfrac{3}{\pi}\geq-\dfrac{3}{\pi}$.}\\
		Vậy với $\alpha=-\dfrac{2}{\pi};\beta=-\dfrac{1}{\pi}$ thì ta có\\
		\centerline{$\displaystyle\int\limits_0^{\pi}\left[f(x)-\dfrac{2}{\pi}\cos x-\dfrac{1}{\pi}\right]^2=\displaystyle\int\limits_0^{\pi} f^2(x)\mathrm{\,d}x-\dfrac{3}{\pi}$.}\\
		Suy ra $\displaystyle\int\limits_0^{\pi} f^2(x)\mathrm{\,d}x=\displaystyle\int\limits_0^{\pi}\left[f(x)-\dfrac{2}{\pi}\cos x-\dfrac{1}{\pi}\right]^2+\dfrac{3}{\pi}\geq\dfrac{3}{\pi}$. Dấu \lq\lq=\rq\rq\, xảy ra khi $f(x)=\dfrac{2\cos x+1}{\pi}$.}
	%<MyLT>
\end{ex}
\begin{ex}%[2D3G2-4]%[Trần Ngọc Phú]%Câu 115. 
	Cho hàm số $f(x)$ liên tục trên $[0;\pi],$ thỏa mãn $\displaystyle\int\limits_0^{\pi}\sin xf(x)\mathrm{\,d}x=\displaystyle\int\limits_0^{\pi}\cos xf(x)\mathrm{\,d}x=1$. Giá trị nhỏ nhất của tích phân $\displaystyle\int\limits_0^{\pi} f^2(x)\mathrm{\,d}x$ bằng
	\choice
	{$\dfrac{2}{\pi}$}
	{$\dfrac{3}{\pi}$}
	{\True $\dfrac{4}{\pi}$}
	{$\dfrac{3}{2\pi}$}
	\loigiai{Liên kết với bình phương $\left[f(x)+\alpha\sin x+\beta\cos x\right]^2$.\\
		Ta có\\ $\displaystyle\int\limits_0^{\pi}\left[f(x)+\alpha\sin x+\beta\cos x\right]^2\mathrm{\,d}x$\\
		$\begin{aligned}&=\displaystyle\int\limits_0^{\pi}[f(x)]^2\mathrm{\,d}x+2\displaystyle\int\limits_0^{\pi}\left(\alpha\sin x+\beta\cos x\right)f(x)\mathrm{\,d}x+\displaystyle\int\limits_0^{\pi}\left(\alpha\sin x+\beta\cos x\right)^2\mathrm{\,d}x\\&=\displaystyle\int\limits_0^{\pi}[f(x)]^2\mathrm{\,d}x+2\left(\alpha+\beta\right)+\dfrac{\pi{\alpha}^2}{2}+\dfrac{\pi{\beta}^2}{2}\cdot\end{aligned}$.\\
		Phân tích $2\left(\alpha+\beta\right)+\dfrac{\pi{\alpha}^2}{2}+\dfrac{\pi{\beta}^2}{2}=\dfrac{\pi}{2}\left(\alpha+\dfrac{2}{\pi}\right)^2+\dfrac{\pi}{2}\left(\beta+\dfrac{2}{\pi}\right)^2-\dfrac{4}{\pi}$.}
\end{ex}
\begin{ex}%[2D3G2-4]%[Trần Ngọc Phú]%Câu 116. 
	Cho hàm số $f(x)$ liên tục trên $[0;1],$ thỏa mãn $\displaystyle\int\limits_0^1 f(x)\mathrm{\,d}x=\displaystyle\int\limits_0^1\mathrm{e}^xf(x)\mathrm{\,d}x=1$. Gọi $m$ là giá trị nhỏ nhất của tích phân $\displaystyle\int\limits_0^1[f(x)]^2\mathrm{\,d}x$. Mệnh đề nào sau đây \textbf{đúng}?
	\choice
	{$0<m<1$}
	{$1<m<2$}
	{$2<m<3$}
	{\True $3<m<4$}
	\loigiai{Từ giả thiết, ta có $\heva{&a=\displaystyle\int\limits_0^1 a\mathrm{e}^xf(x)\mathrm{\,d}x\\&b=\displaystyle\int\limits_0^1 bf(x)\mathrm{\,d}x.}$ \\
		Theo Holder\\
		\centerline{	$(a+b)^2=\left[\displaystyle\int\limits_0^1\left(a\mathrm{e}^x+b\right)f(x)\mathrm{\,d}x\right]^2\leq\displaystyle\int\limits_0^1\left(a\mathrm{e}^x+b\right)^2\mathrm{\,d}x\displaystyle\int\limits_0^1 f^2(x)\mathrm{\,d}x$.}\\
		Lại có\\
		\centerline{	$\displaystyle\int\limits_0^1\left(a\mathrm{e}^x+b\right)^2\mathrm{\,d}x=\displaystyle\int\limits_0^1\left(a^2\mathrm{e}^{2x}+2ab\mathrm{e}^x+b^2\right)\mathrm{\,d}x=\dfrac{1}{2}\left(\mathrm{e}^2-1\right)a^2+2(e-1)ab+b^2$.}\\
		Suy ra $\displaystyle\int\limits_0^1 f^2(x)\mathrm{\,d}x\geq\dfrac{(a+b)^2}{\dfrac{1}{2}\left(\mathrm{e}^2-1\right)a^2+2(e-1)ab+b^2}$ với mọi $a, b\in\mathbb{R}$ và $a^2+b^2>0$.\\
		Do đó $\displaystyle\int\limits_0^1 f^2(x)\mathrm{\,d}x\geq\max\left\{\dfrac{(a+b)^2}{\dfrac{1}{2}\left(\mathrm{e}^2-1\right)a^2+2(e-1)ab+b^2}\right\}=-1+\dfrac{1}{3-e}+\dfrac{1}{e-1}\approx 3,1316$.}
\end{ex}
\begin{ex}%[2D3G2-4]%[Trần Ngọc Phú]%Câu 117. 
	Cho hàm số $f(x)$ liên tục trên $[0;1]$ thỏa mãn $\displaystyle\int\limits_0^1 f(x)\mathrm{\,d}x=\displaystyle\int\limits_0^1\sqrt{x}f(x)\mathrm{\,d}x=1$. Giá trị nhỏ nhất của tích phân $\displaystyle\int\limits_0^1 f^2(x)\mathrm{\,d}x$ bằng
	\choice
	{$\dfrac{2}{3}$}
	{$1$}
	{$\dfrac{8}{3}$}
	{\True $3$}
	\loigiai{
		Từ giả thiết, ta có $\heva{&a=\displaystyle\int\limits_0^1 a\sqrt{x}f(x)\mathrm{\,d}x\\&b=\displaystyle\int\limits_0^1 bf(x)\mathrm{\,d}x.}$ \\
		Theo Holder\\
		\centerline{$(a+b)^2=\left(\displaystyle\int\limits_0^1\left(a\sqrt{x}+b\right)f(x)\mathrm{\,d}x\right)^2\leq\displaystyle\int\limits_0^1\left(a\sqrt{x}+b\right)^2\mathrm{\,d}x\cdot\displaystyle\int\limits_0^1 f^2(x)\mathrm{\,d}x$.}\\
		Lại có\\
		\centerline{$\displaystyle\int\limits_0^1\left(a\sqrt{x}+b\right)^2\mathrm{\,d}x=\dfrac{a^2}{2}+\dfrac{4ab}{3}+b^2$.}\\
		Suy ra\\\centerline{ $\displaystyle\int\limits_0^1 f^2(x)\mathrm{\,d}x\geq\dfrac{(a+b)^2}{\dfrac{a^2}{2}+\dfrac{4ab}{3}+b^2}$ với mọi $a, b\in\mathbb{R}$ và $a^2+b^2>0$.}\\
		Do đó $\displaystyle\int\limits_0^1 f^2(x)\mathrm{\,d}x\geq\max\left\{\dfrac{(a+b)^2}{\dfrac{a^2}{2}+\dfrac{4ab}{3}+b^2}\right\}=3$.\\ Cách 2. Liên kết với bình phương $\left[f(x)+\alpha\sqrt{x}+\beta\right]^2$.\\
		Ta có\\ $\displaystyle\int\limits_0^{\pi}\left[f(x)+\alpha\sqrt{x}+\beta\right]^2\mathrm{\,d}x$.\\
		$\begin{aligned}&=\displaystyle\int\limits_0^{\pi}[f(x)]^2\mathrm{\,d}x+2\displaystyle\int\limits_0^{\pi}\left(\alpha\sqrt{x}+\beta\right)f(x)\mathrm{\,d}x+\displaystyle\int\limits_0^{\pi}\left(\alpha\sqrt{x}+\beta\right)^2\mathrm{\,d}x\\&=\displaystyle\int\limits_0^{\pi}[f(x)]^2\mathrm{\,d}x+2\left(\alpha+\beta\right)+\dfrac{{\alpha}^2}{2}+\dfrac{4}{3}\alpha\beta+{\beta}^2\cdot\end{aligned}$.\\
		Phân tích $2\left(\alpha+\beta\right)+\dfrac{{\alpha}^2}{2}+\dfrac{4}{3}\alpha\beta+\beta^2=\left(\beta+\dfrac{2}{3}\alpha+1\right)^2+\dfrac{1}{18}(\alpha+6)^2-3$.}
\end{ex}
\begin{ex}%[2D3G2-4]%[Trần Ngọc Phú]%Câu 118. 
	Cho hàm số $y=f(x)$ có đạo hàm liên tục trên $[1;2],$ thỏa $\displaystyle\int\limits_1^2 x^3f(x)\mathrm{\,d}x=31$. Giá trị nhỏ nhất của tích phân $\displaystyle\int\limits_1^2 f^4(x)\mathrm{\,d}x$ bằng
	\choice
	{$961$}
	{\True $3875$}
	{$148955$}
	{$923521$}
	\loigiai{
		Ta có áp dụng hai lần liên tiếp bất đẳng thức Holder ta được
		\begin{eqnarray*}
			31^4=\left(\displaystyle\int\limits_1^2 x^3f(x)\mathrm{\,d}x\right)^4=\left(\left[\displaystyle\int\limits_1^2 x^2\cdot xf(x)\mathrm{\,d}x\right]^2\right)^2&\leq&\left(\displaystyle\int\limits_1^2 x^4\mathrm{\,d}x\right)^2\left(\displaystyle\int\limits_1^2 x^2f^2(x)\mathrm{\,d}x\right)^2\\&\leq&\left(\displaystyle\int\limits_1^2 x^4\mathrm{\,d}x\right)^3\displaystyle\int\limits_1^2 f^4(x)\mathrm{\,d}x.\\
		\end{eqnarray*}
		Suy ra $\displaystyle\int\limits_1^2 f^4(x)\mathrm{\,d}x\geq\dfrac{{31}^4}{\left(\displaystyle\int\limits_1^2 x^4\mathrm{\,d}x\right)^3}=3875$.\\
		Dấu \lq\lq=\rq\rq\, xảy ra khi $f(x)=kx$ nên $k\displaystyle\int\limits_1^2 x^4\mathrm{\,d}x=31\Leftrightarrow k=5\Rightarrow  f(x)=5x^2$.}
\end{ex}
\begin{ex}%[2D3G2-4]%[Trần Ngọc Phú]%Câu 119. 
	Cho hàm số $f(x)$ liên tục và có đạo hàm đến cấp $2$ trên $[0;2]$ thỏa $f(0)-2f(1)+f(2)=1$. Giá trị nhỏ nhất của tích phân $\displaystyle\int\limits_0^2[f''(x)]^2\mathrm{\,d}x$ bằng
	\choice
	{$\dfrac{2}{3}$}
	{\True $\dfrac{3}{2}$}
	{$\dfrac{4}{5}$}
	{$\dfrac{5}{4}$}
	\loigiai{
		Ta có
		\begin{eqnarray*}
			\displaystyle\int\limits_0^1[f''(x)]^2\mathrm{\,d}x=3\displaystyle\int\limits_0^1 x^2\mathrm{\,d}x\cdot\displaystyle\int\limits_0^1[f''(x)]^2\mathrm{\,d}x&\overset{\text{Holder}}{\geq}& 3\left(\displaystyle\int\limits_0^1 x\cdot f''(x)\mathrm{\,d}x\right)^2\\
			&\overset{\heva{&u=x\\&\mathrm{\,d}v=f''(x)\mathrm{\,d}x}}{\geq}& 3\left[f'(1)+f(0)-f(1)\right]^2.
		\end{eqnarray*}
		\begin{eqnarray*}
			\displaystyle\int\limits_1^2[f''(x)]^2\mathrm{\,d}x=3\displaystyle\int\limits_1^2(x-2)^2\mathrm{\,d}x\cdot\displaystyle\int\limits_1^2[f''(x)]^2\mathrm{\,d}x&\overset{\text{Holder}}{\geq}& 3\left(\displaystyle\int\limits_1^2(x-2)\cdot f''(x)\mathrm{\,d}x\right)^2\\
			&\overset{\heva{&u=x-2\\&\mathrm{\,d}v=f''(x)\mathrm{\,d}x}}{\geq}& 3\left[-f'(1)+f(2)-f(1)\right]^2.
		\end{eqnarray*}
		
		Suy ra
		\begin{eqnarray*}
			\displaystyle\int\limits_0^2[f''(x)]^2\mathrm{\,d}x&\geq& 3\left[f'(1)+f(0)-f(1)\right]^2+3\left[-f'(1)+f(2)-f(1)\right]^2\\
			&\geq& 3\cdot\dfrac{\left[f(0)-2f(1)+f(2)\right]^2}{2}\\&=&\dfrac{3}{2}.
		\end{eqnarray*}
		Nhận xét: Lời giải trên sử dụng bất đẳng thức ở bước cuối là $a^2+b^2\geq\dfrac{(a+b)^2}{2}$.}
\end{ex}
\begin{ex}%[2D3G2-4]%[Trần Ngọc Phú]%Câu 120. 
	Cho hàm số $f(x)$ có đạo hàm trên $[1;3]$ và $f(1)=0,\max\limits_{[1;3]}\left|f(x)\right|=\sqrt{10}$. Giá trị nhỏ nhất của tích phân $\displaystyle\int\limits_1^3[f'(x)]^2\mathrm{\,d}x$ bằng
	\choice
	{$1$}
	{\True $5$}
	{$10$}
	{$20$}
	\loigiai{
		Vì $\max\limits_{[1;3]}\left|f(x)\right|=\sqrt{10}\Rightarrow \exists x_0\in[1;3]$ sao cho $\left|f(x_0)\right|=\sqrt{10}$.\\
		$\xrightarrow  {f(1)=0}\exists x_0\in(1;3]$ sao cho $\left|f(x_0)\right|=\sqrt{10}$.\\
		Theo Holder\\
		\centerline{	$\left(\displaystyle\int\limits_1^{x_0} f'(x)\mathrm{\,d}x\right)^2\leq\displaystyle\int\limits_1^{x_0} 1^2\mathrm{\,d}x\cdot\displaystyle\int\limits_1^{x_0}[f'(x)]^2\mathrm{\,d}x=(x_0-1)\cdot\displaystyle\int\limits_1^{x_0}[f'(x)]^2\mathrm{\,d}x$.}\\
		Mà \\
		\centerline{$\left(\displaystyle\int\limits_1^{x_0} f'(x)\mathrm{\,d}x\right)^2=\left(f(x)\bigg|_1^{x_0}\right)^2=\left(f(x_0)-f(1)\right)^2=10$.}\\
		Từ đó suy ra $\displaystyle\int\limits_1^{x_0}[f'(x)]^2\mathrm{\,d}x\geq\dfrac{10}{x_0-1}$.\\
		Suy ra $  \displaystyle\int\limits_1^3[f'(x)]^2\mathrm{\,d}x\geq\displaystyle\int\limits_1^{x_0}[f'(x)]^2\mathrm{\,d}x\geq\dfrac{10}{x_0-1}\geq\dfrac{10}{3-1}=5$.}
\end{ex}
\Closesolutionfile{ans}

% \inputansbox{10}{ans/ansCD2D3-2.2LT}