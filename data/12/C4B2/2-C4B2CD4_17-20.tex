\Opensolutionfile{ans}[ans/ans-GV40-1]
\setcounter{ex}{19}

\begin{ex}%[2D4C2-5]
    Cho hai hàm $f(x)$ và $g(x)$ có đạo hàm trên $\left[1;2\right]$ thỏa mãn $f(1)=g(1)=0$ và $\heva{& \dfrac{x}{(x+1)^2}g(x)+2023x=(x+1)f'(x) \\ & \dfrac{x^3}{x+1}g'(x)+f(x)=2024x^2}\,,\forall x\in \left[1;2\right]$.\\ 
    Tính tích phân $I=\displaystyle\int\limits_1^2 \left[\dfrac{x}{x+1}g(x)-\dfrac{x+1}{x}f(x) \right]\mathrm{\,d}x$.
    \choice
    {\True $I=\dfrac{1}{2}$}
    {$I=1$} 
    {$I=\dfrac{3}{2}$}
    {$I=2$}
    \loigiai{
    Từ giả thiết ta có $\heva{& \dfrac{1}{(x+1)^2}g(x)-\dfrac{x+1}{x}f'(x)=-2023\\ & \dfrac{x}{x+1}g'(x)+\dfrac{1}{x^2}f(x)=2024}\,,\forall x\in \left[1;2\right]$.\\
    Suy ra
    \allowdisplaybreaks 
    \begin{eqnarray*}
        && \left[\dfrac{1}{(x+1)^2}g(x)+\dfrac{x}{x+1}g'(x) \right]-\left[\dfrac{x+1}{x}f'(x)-\dfrac{1}{x^2}f(x) \right]=1\\  
        &\Leftrightarrow& \left[\dfrac{x}{x+1}g(x) \right]'-\left[\dfrac{x+1}{x}f(x) \right]'=1\\ 
        &\Rightarrow& \dfrac{x}{x+1}g(x)-\dfrac{x+1}{x}f(x)=x+C.
    \end{eqnarray*}
    Mà $f(1)=g(1)=0\Rightarrow C=-1 \Rightarrow \dfrac{x}{x+1}g(x)-\dfrac{x+1}{x}f(x)=x-1$.\\
    Vậy $I=\displaystyle\int\limits_1^2 \left[\dfrac{x}{x+1}g(x)-\dfrac{x+1}{x}f(x) \right]\mathrm{\,d}x=\displaystyle\int\limits_1^2 (x-1)\mathrm{\,d}x=\dfrac{1}{2}$.
    }
\end{ex}

\begin{ex}%[2D4C2-5]
    Cho hàm số $f\left(x \right)$ xác định và liên tục trên $\mathbb{R}\setminus \left\{0\right\}$ thỏa mãn $x^2f^2\left(x \right)+\left(2x-1\right)f\left(x \right)=xf'\left(x \right)-1$, với mọi $x\in \mathbb{R}\setminus \left\{0\right\}$ đồng thời thỏa mãn $f\left(1\right)=-2$. Tính $\displaystyle\int\limits_1^2 f\left(x \right)\mathrm{\,d}x$.
    \choice
    {$-\dfrac{\ln 2}{2}-1$}
    {\True $-\ln 2-\dfrac{1}{2}$}
    {$-\ln 2-\dfrac{3}{2}$}
    {$-\dfrac{\ln 2}{2}-\dfrac{3}{2}$}
    \loigiai{
    Ta có 
    \allowdisplaybreaks 
    \begin{eqnarray*}
        && x^2f^2\left(x \right)+2xf\left(x \right)+1=xf'\left(x \right)+f\left(x \right) \\ 
        &\Leftrightarrow& \left(xf\left(x \right)+1\right)^2=\left(xf\left(x \right)+1\right)'.
    \end{eqnarray*}
    Do đó
    \allowdisplaybreaks 
    \begin{eqnarray*}
        && \dfrac{\left(xf\left(x \right)+1\right)'}{\left(xf\left(x \right)+1\right)^2}=1\\ 
        &\Rightarrow& \displaystyle\int \dfrac{\left(xf\left(x \right)+1\right)'}{\left(xf\left(x \right)+1\right)^2}\mathrm{\,d}x=\displaystyle\int 1\mathrm{\,d}x\\
        &\Rightarrow& -\dfrac{1}{xf\left(x \right)+1}=x+C\\
        &\Rightarrow& xf\left(x \right)+1=-\dfrac{1}{x+C}.
    \end{eqnarray*}
    Mặt khác $f\left(1\right)=-2$ nên $-2+1=-\dfrac{1}{1+C}\Rightarrow C=0$.\\
    Nên suy ra $xf\left(x \right)+1=-\dfrac{1}{x}\Rightarrow f\left(x \right)=-\dfrac{1}{x^2}-\dfrac{1}{x}$.\\
    Vậy $\displaystyle\int\limits_1^2 f\left(x \right)\mathrm{\,d}x=\displaystyle\int\limits_1^2 \left(-\dfrac{1}{x^2}-\dfrac{1}{x} \right)\mathrm{\,d}x=\left.\left(-\ln x+\dfrac{1}{x} \right)\right|_1^2=-\ln 2-\dfrac{1}{2}$.
    }
\end{ex}

\begin{ex}%[2D4C2-5]
    Cho hàm số $y=f(x)$ có đạo hàm liên tục trên $\mathbb{R}$ thỏa mãn $x\cdot f(x)\cdot f'(x)=f^2(x)-x,\,\forall x\in \mathbb{R}$ và có $f(2)=1$. Tích phân $\displaystyle\int\limits_0^2 f^2(x)\mathrm{\,d}x$ bằng
    \choice
    {$\dfrac{3}{2}$}
    {$\dfrac{4}{3}$}
    {\True $2$}
    {$4$}
    \loigiai{
    Ta có
    \allowdisplaybreaks 
    \begin{eqnarray*}
        x\cdot f(x)\cdot f'(x)=f^2(x)-x &\Leftrightarrow& 2x\cdot f(x)\cdot f'(x)=2f^2(x)-2x \\
       &\Leftrightarrow& 2x\cdot f(x)\cdot f'(x)+f^2(x)=3f^2(x)-2x \\ 
       &\Leftrightarrow& \displaystyle\int\limits_0^2 \left(x\cdot f^2(x) \right)'\mathrm{\,d}x=3\displaystyle\int\limits_0^2 f^2(x)\mathrm{\,d}x-\displaystyle\int\limits_0^2 2x\mathrm{\,d}x \\ 
       &\Leftrightarrow& \left.\left(x\cdot f^2(x) \right)\right|_0^2 =3I-4\\ 
       &\Leftrightarrow& 2=3I-4\\ 
       &\Leftrightarrow& I=2.
    \end{eqnarray*}
    }
\end{ex}

\begin{ex}%[2D4C2-5]
    Cho hàm số $f\left(x \right)$ có đạo hàm liên tục trên $\mathbb{R}$, $f\left(0\right)=0$, $f'\left(0\right)\ne 0$ và thỏa mãn hệ thức $f\left(x \right)\cdot f'\left(x \right)+18x^2=\left(3x^2+x \right)f'\left(x \right)+\left(6x+1\right)f\left(x \right),\,\forall x \in \mathbb{R}$. Biết $\displaystyle\int\limits_0^1 \left(x+1\right)\mathrm{e}^{f\left(x \right)}\mathrm{\,d}x=a\mathrm{e}^2+b,\,\left(a,b\in \mathbb{Q} \right)$. Giá trị của $a-b$ bằng
    \choice
    {\True $1$}
    {$2$}
    {$0$}
    {$\dfrac{2}{3}$}
    \loigiai{
    Ta có $f\left(x \right)\cdot f'\left(x \right)+18x^2=\left(3x^2+x \right)f'\left(x \right)+\left(6x+1\right)f\left(x \right)$.\\
    Lấy nguyên hàm hai vế ta được 
    \allowdisplaybreaks 
    \begin{eqnarray*}
        \dfrac{f^2\left(x \right)}{2}+6x^3=\left(3x^2+x \right)f\left(x \right)
        &\Rightarrow& f^2\left(x \right)-2\left(3x^2+x \right)f\left(x \right)+12x^3=0\\
        &\Rightarrow& \hoac{& f\left(x \right)=6x^2 \\ & f\left(x \right)=2x.}
    \end{eqnarray*}
    \begin{enumerate}[\bf TH1:]
        \item $f\left(x \right)=6x^2$ không thoả mãn kết quả $\displaystyle\int\limits_0^1 \left(x+1\right)\mathrm{e}^{f\left(x \right)}\mathrm{\,d}x=a\mathrm{e}^2+b,\,\left(a,b\in \mathbb{Q} \right)$.
        \item $f\left(x \right)=2x \Rightarrow \displaystyle\int\limits_0^1 \left(x+1\right)\mathrm{e}^{f\left(x \right)}\mathrm{\,d}x= \displaystyle\int\limits_0^1 \left(x+1\right)\mathrm{e}^{2x}\mathrm{\,d}x=\dfrac{3}{4}\mathrm{e}^2-\dfrac{1}{4}$.\\ 
        Suy ra $a=\dfrac{3}{4};b=-\dfrac{1}{4}$.
    \end{enumerate}
    Vậy $a-b=1$.
    }
\end{ex}

\begin{ex}%[2D4C2-5]
    Cho hàm số $y=f(x)$ xác định và có đạo hàm $f'\left(x \right)$ liên tục trên $[1;3]$; $f\left(x \right)\ne 0,\,\forall x\in \left[1;3\right]$; $f'\left(x \right)\left[1+f\left(x \right) \right]^2=\left(x-1\right)^2\left[f\left(x \right) \right]^4$ và $f\left(1\right)=-1$. Biết rằng $\displaystyle\int\limits_{\mathrm{e}}^3 f\left(x \right)\mathrm{\,d}x=a\ln 3+b\,\left(a,b\in \mathbb{Z} \right)$. Giá trị của $a+b^2$ bằng
    \choice
    {$4$}
    {\True $0$}
    {$2$}
    {$-1$}
    \loigiai{
    Ta có 
    \allowdisplaybreaks 
    \begin{eqnarray*}
        f'(x)\left[1+f(x)\right]^2=(x-1)^2\left[f(x)\right]^4
        &\Rightarrow& \dfrac{f'(x)}{f^4(x)}+\dfrac{2f'(x)}{f^3(x)}+\dfrac{f'(x)}{f^2(x)}=(x-1)^2\\
        &\Rightarrow& \displaystyle\int \left(\dfrac{f'(x)}{f^4(x)}+\dfrac{2f'(x)}{f^3(x)}+\dfrac{f'(x)}{f^2(x)} \right) \mathrm{\,d}x=\displaystyle\int (x-1)^2\mathrm{\,d}x\\
        &\Rightarrow& -\left(\dfrac{1}{3f^3(x)}+\dfrac{1}{f^2(x)}+\dfrac{1}{f(x)} \right)=\dfrac{1}{3}(x-1)^3+C. \quad (*)
    \end{eqnarray*}
    Do $f(1)=-1$ nên $C=\dfrac{1}{3}$.\\ 
    Thay vào $(*)$ ta được $\left(\dfrac{1}{f(x)}+1\right)^3=-(x-1)^3 \Rightarrow f(x)=\dfrac{-1}{x}$.\\
    Khi đó $\displaystyle\int\limits_{\mathrm{e}}^3 \dfrac{-1}{x}\mathrm{\,d}x=\left.-\ln \left|x \right|\right|_{\mathrm{e}}^3=-\ln 3+1\Rightarrow a=-1,b=1$.\\ 
    Vậy $a+b^2=0$.\\
    }
\end{ex}


\Closesolutionfile{ans}
