\chapter{Nguyên hàm Tích phân}

\Opensolutionfile{ans}[ans/ans-2-C4B2CD3-KQ]
\TNSA

\begin{ex}%[2D4H2-1]
	Cho $\displaystyle\int\limits_0^3f(x)\mathrm{\,d}x=4$. Tính $I=\displaystyle\int\limits_0^33f(x)\mathrm{\,d}x$.\\
	\shortans{$12$}
	\loigiai{
	Ta có $\displaystyle\displaystyle\int\limits_0^3 3 f(x){d}x=3\displaystyle\displaystyle\int\limits_0^3 f(x){d}x=12$.}
\end{ex}
\begin{ex}%[2D4H2-1]
	Cho $\displaystyle\int\limits_1^3f(x)\mathrm{\,d}x=2$. Tính $I=\displaystyle\int\limits_1^3\left[f(x)+2x\right]\mathrm{\,d}x$.\\
	\shortans{$10$}
\loigiai{
	Ta có$\colon $ $\displaystyle\int\limits_1^3\left[f(x)+2x\right]\mathrm{\,d}x=\displaystyle\int\limits_1^3f(x)\mathrm{\,d}x+\displaystyle\int\limits_1^32x\mathrm{\,d}x=2+\left.x^2\right|_1^3=2+3^2-1^2=10$.}
	\end{ex}
	
	\begin{ex}%[2D4H2-1]
Cho $\displaystyle\int\limits_{-1}^2f(x)\mathrm{\,d}x=2$ và $\displaystyle\int\limits_{-1}^2g(x)\mathrm{\,d}x=-1$. Tính $ I=\displaystyle\int\limits_{-1}^2\left[x+2f(x)+3g(x)\right]\mathrm{\,d}x$.\\
\shortans{$2{,}5$}
\loigiai{
Ta có $\displaystyle\int\limits_{-1}^2\left[x+2f(x)+3g(x)\right]\mathrm{\,d}x=\displaystyle\int\limits_{-1}^2x\mathrm{\,d}x+2\displaystyle\int\limits_{-1}^2f(x)\mathrm{\,d}x+3\displaystyle\int\limits_{-1}^2g(x)\mathrm{\,d}x=\dfrac{3}{2}+4-3=\dfrac{5}{2}=2{.}5$.}
\end{ex}

\begin{ex}%[2D4H2-1]
Cho $\displaystyle\int\limits_0^1f(x)\mathrm{\,d}x=1$. Tính tích phân $ I=\displaystyle\int\limits_0^1\left[2f(x)-3x^2\right]\mathrm{\,d}x.$\\
\shortans{$1$}
\loigiai{
$\displaystyle\int\limits_0^1\left[2f(x)-3x^2\right]\mathrm{\,d}x=2\displaystyle\int\limits_0^1f(x)\mathrm{\,d}x-3\displaystyle\int\limits_0^1x^2\mathrm{\,d}x=2-1=1$.}
\end{ex}

\begin{ex}%[2D4H2-1]
Biết $\displaystyle\int\limits_1^3f(x)\mathrm{\,d}x=3$. Tính giá trị của $ I=\displaystyle\int\limits_3^12f(x)\mathrm{\,d}x$.\\
\shortans{$-6$}
\loigiai{
Ta có $\displaystyle\int\limits_3^12f(x)\mathrm{\,d}x=-\displaystyle\int\limits_1^32f(x)\mathrm{\,d}x=-2\displaystyle\int\limits_1^3f(x)\mathrm{\,d}x=-2\cdot3=-6$.}
\end{ex}

\begin{ex}%[2D4H2-1]
Biết $\displaystyle\int\limits_0^1f(x)\mathrm{\,d}x=-2$ và $\displaystyle\int\limits_1^0g(x)\mathrm{\,d}x=-3$.. Tính $ I=\displaystyle\int\limits_0^1\left[f(x)-g(x)\right]\mathrm{\,d}x$.\\
\shortans{$-5$}
\loigiai{
$\displaystyle\int\limits_0^1\left[f(x)-g(x)\right]\mathrm{\,d}x=\displaystyle\int\limits_0^1f(x)\mathrm{\,d}x-\displaystyle\int\limits_0^1g(x)\mathrm{\,d}x=-2-3=-5$.}
\end{ex}

\begin{ex}%[2D4H2-1]
Biết $\displaystyle\int\limits_1^2f(x)\,\mathrm{\,d}x=3$ và $\displaystyle\int\limits_1^2g(x)\mathrm{\,d}x=2$ và $\displaystyle\int\limits_1^2h(x)\mathrm{\,d}x=2022$. Tính $\linebreak I=\displaystyle\int\limits_1^2\left[f(x)-g(x)+h(x)\right]\mathrm{\,d}x$.\\
\shortans{$2023$}
\loigiai{
Ta có $\displaystyle\int\limits_1^2\left[f(x)-g(x)+h(x)\right]\,\mathrm{\,d}x=\displaystyle\int\limits_1^2f(x)\,\mathrm{\,d}x-\displaystyle\int\limits_1^2g(x)\mathrm{\,d}x+\displaystyle\int\limits_1^2h(x)\mathrm{\,d}x$\\
$=3-2+2022=2023$.}
\end{ex}

\begin{ex}%[2D4H2-1]
Cho $\displaystyle\int\limits_{-1}^2f(x)\mathrm{\,d}x=2$ và $\displaystyle\int\limits_2^5f(x)\mathrm{\,d}x=-5$. Tính $ I=\displaystyle\int\limits_{-1}^5f(x)\mathrm{\,d}x$.\\
\shortans{$-3$}
\loigiai{
Ta có $\displaystyle\int\limits_{-1}^5f(x)\mathrm{\,d}x=\displaystyle\int\limits_{-1}^2f(x)\mathrm{\,d}x+\displaystyle\int\limits_2^5f(x)\mathrm{\,d}x=2-5=-3$.}
\end{ex}

\begin{ex}%[2D4H2-1]
Cho $ f$, $ g$ là hai hàm liên tục trên đoạn $\left[1;\,3\right]$ thoả$\colon $ $\displaystyle\int\limits_1^3\left[f(x)+3g(x)\right]\mathrm{\,d}x=10$, $\displaystyle\int\limits_1^3\left[2f(x)-g(x)\right]\mathrm{\,d}x=6$. Tính $I=\displaystyle\int\limits_1^3\left[f(x)+g(x)\right]\mathrm{\,d}x$.\\
\shortans{$6$}
\loigiai{
Đặt $ a=\displaystyle\int\limits_1^3f(x)\mathrm{\,d}x$ và $ b=\displaystyle\int\limits_1^3g(x)\mathrm{\,d}x$.\\
Khi đó, $\displaystyle\int\limits_1^3\left[f(x)+3g(x)\right]\mathrm{\,d}x=a+3b$, $\displaystyle\int\limits_1^3\left[2f(x)-g(x)\right]\mathrm{\,d}x=2a-b$.\\
Theo giả thiết, ta có $\heva{
& a+3b=10\\ 
& 2a-b=6\\ 
}\Leftrightarrow\heva{
& a=4\\ 
& b=2.\\ 
}$\\
Vậy $ I=a+b=6$.}
\end{ex}

\begin{ex}%[2D4H2-1]
Cho hàm số $ f(x)$ liên tục trên $\mathbb{R}$ thoả mãn $\displaystyle\int\limits_1^8f(x)\,\mathrm{\,d}x=9$, $\displaystyle\int\limits_4^{12}{f(x)}\,\mathrm{\,d}x=3$, $\displaystyle\int\limits_4^8f(x)\,\mathrm{\,d}x=5$. Tính $ I=\displaystyle\int\limits_1^{12}{f(x)}\,\mathrm{\,d}x$.\\
\shortans{$7$}
\loigiai{
Ta có $ I=\displaystyle\int\limits_1^{12}{f(x)}\,\mathrm{\,d}x=\displaystyle\int\limits_1^8f(x)\,\mathrm{\,d}x+\displaystyle\int\limits_8^{12}{f(x)}\,\mathrm{\,d}x$ $=\displaystyle\int\limits_1^8f(x)\,\mathrm{\,d}x+\displaystyle\int\limits_4^{12}{f(x)}\,\mathrm{\,d}x-\displaystyle\int\limits_4^8f(x)\,\mathrm{\,d}x$\\$=9+3-5=7$.}
\end{ex}

\begin{ex}%[2D4H2-1]
Cho hàm số $ f(x)$ liên tục trên $\left[0;10\right]$ thỏa mãn $\displaystyle\int\limits_0^{10}{f(x)\mathrm{\,d}x}=7$, $\displaystyle\int\limits_2^6f(x)\mathrm{\,d}x=3$. Tính $ P=\displaystyle\int\limits_0^2f(x)\mathrm{\,d}x+\displaystyle\int\limits_6^{10}{f(x)\mathrm{\,d}x}$.\\
\shortans{$4$}
\loigiai{
Ta có $\displaystyle\int\limits_0^{10}{f(x)\mathrm{\,d}x}=\displaystyle\int\limits_0^2f(x)\mathrm{\,d}x+\displaystyle\int\limits_2^6f(x)\mathrm{\,d}x+\displaystyle\int\limits_6^{10}{f(x)\mathrm{\,d}x}$\\
Suy ra $\displaystyle\int\limits_0^2f(x)\mathrm{\,d}x+\displaystyle\int\limits_6^{10}{f(x)\mathrm{\,d}x}=\displaystyle\int\limits_0^{10}{f(x)\mathrm{\,d}x}-\displaystyle\int\limits_2^6f(x)\mathrm{\,d}x=7-3=4$.}
\end{ex}

\begin{ex}%[2D4H2-1]
Giả sử $\displaystyle\int\limits_0^1f(x)\mathrm{\,d}x=3$ và $\displaystyle\int\limits_0^5f(z)\mathrm{\,d}z=9$. Tổng $I=\displaystyle\int\limits_1^3f(t)\mathrm{\,d}t+\displaystyle\int\limits_3^5f(t)\mathrm{\,d}t$ bằng\\
\shortans{$6$}
\loigiai{
$\displaystyle\int\limits_0^1f(x)\mathrm{\,d}x=3\Leftrightarrow\displaystyle\int\limits_0^1f(t)\mathrm{\,d}t=3\Leftrightarrow\displaystyle\int\limits_1^0f(t)\mathrm{\,d}t=-3.$\\
$\displaystyle\int\limits_0^5f(z)\mathrm{\,d}z=9\Leftrightarrow\displaystyle\int\limits_0^5f(t)\mathrm{\,d}t=9.$\\
$\Rightarrow\displaystyle\int\limits_1^0f(t)\mathrm{\,d}t+\displaystyle\int\limits_0^5f(t)\mathrm{\,d}t=6\Leftrightarrow\displaystyle\int\limits_1^5f(t)\mathrm{\,d}t=6.$\\
$I=\displaystyle\int\limits_1^3f(t)\mathrm{\,d}t+\displaystyle\int\limits_3^5f(t)\mathrm{\,d}t=\displaystyle\int\limits_1^5f(t)\mathrm{\,d}t=6.$}
\end{ex}





\Closesolutionfile{ans}
\indapan{6}{ans/ans-2-C4B2CD3-KQ}