\chude{TÍCH PHÂN HÀM ẨN BIẾN ĐỔI CƠ BẢN}
\begin{itemize}
	\item $\displaystyle \int\limits_{a}^bk\cdot f(x)\mathrm{\, d}x=k\cdot \displaystyle \int\limits_{a}^bf(x)\mathrm{\, d}x$.
	\item $\displaystyle \int\limits_{a}^b\left[f(x)\pm g(x)\right]\mathrm{\, d}x=\displaystyle \int\limits_{a}^bf(x)\mathrm{\, d}x\pm \displaystyle \int\limits_{a}^bg(x)\mathrm{\, d}x$.
	\item $\displaystyle \int\limits_{a}^bf(x)\mathrm{\, d}x=\displaystyle \int\limits_{a}^cf(x)\mathrm{\, d}x+\displaystyle \int\limits_{c}^bf(x)\mathrm{\, d}x$.
\end{itemize}
\TN
\Opensolutionfile{ans}[ans/ans-2C4B2CD3-LC]
\begin{ex}%[Câu 1]%[2D4N2-1]
	Nếu $\displaystyle\int\limits_0^3f(x)\mathrm{\,d}x=6$ thì $\displaystyle\int\limits_0^3\left[\dfrac{1}{3}f(x)+2\right]\mathrm{\,d}x$ bằng
	\choice
	{\True $8$}
	{$5$}
	{$9$}
	{$6$}
	\loigiai{
			Ta có $\displaystyle\int\limits_0^3\left[\dfrac{1}{3}f(x)+2\right]\mathrm{\,d}x=\dfrac{1}{3}\displaystyle\int\limits_0^3f(x)\mathrm{\,d}x+\displaystyle\int\limits_0^32\mathrm{\,d}x=\dfrac{1}{3}\cdot 6+6=8$.}
\end{ex}
\begin{ex}%[Câu 2]%[2D4N2-1]
	Nếu $\displaystyle\int_1^4 f(x) \mathrm{\,d}x=3$ và $\displaystyle\int_1^4 g(x) \mathrm{\,d}x=-2$ thì $\displaystyle\int_1^4\left(f(x)-g(x)\right)\mathrm{\,d}x$ bằng
	\choice
	{$-1$}
	{$-5$}
	{\True $5$}
	{$1$}
	\loigiai{
			Ta có $\displaystyle\int _1^4\left[f(x)-g(x)\right]\mathrm{\,d}x=\displaystyle\int _1^4f(x)\mathrm{\,d}x-\displaystyle\int _1^4g(x)\mathrm{\,d}x=3-(-2)=5$.}
\end{ex}
\begin{ex}%[Câu 3]%[2D4N2-1]
	Nếu $\displaystyle\int\limits_1^4f(x)\mathrm{\,d}x=5$ và $\displaystyle\int\limits_1^4g(x)\mathrm{\,d}x=-4$ thì $\displaystyle\int\limits_1^4\left[f(x)-g(x)\right]\mathrm{\,d}x$ bằng
	\choice
	{$-1$}
	{$-9$}
	{$1$}
	{\True $9$}
	\loigiai{
			Ta có $\displaystyle\int\limits_1^4\left[f(x)-g(x)\right]\mathrm{\,d}x=\displaystyle\int\limits_1^4f(x)\mathrm{\,d}x-\displaystyle\int\limits_1^4g(x)\mathrm{\,d}x=5-(-4)=9$.}
\end{ex}
\begin{ex}%[Câu 4]%[2D4N2-1]
	Biết $\displaystyle\int\limits_1^{2024}f(x)\mathrm{\,d}x=-3$ và $\displaystyle\int\limits_{2024}^1g(x)\mathrm{\,d}x=2$. Khi đó $\displaystyle\int\limits_1^{2024}\left[f(x)-g(x)\right]\mathrm{\,d}x$ bằng
	\choice
	{$6$}
	{$-5$}
	{$5$}
	{\True $-1$}
	\loigiai{
			Ta có $\displaystyle\int\limits_{2024}^1g(x)\mathrm{\,d}x=2\Leftrightarrow \displaystyle\int\limits_1^{2024}g(x)\mathrm{\,d}x=-2$.\\
			Do đó $\displaystyle\int\limits_1^{2024}\left[f(x)-g(x)\right]\mathrm{\,d}x=\displaystyle\int\limits_1^{2024}f(x)\mathrm{\,d}x-\displaystyle\int\limits_1^{2024}g(x)\mathrm{\,d}x=-3-(-2)=-1$.}
\end{ex}
\begin{ex}%[Câu 5]%[2D4N2-1]
	Nếu $\displaystyle\int\limits_0^3f(x)\mathrm{\,d}x=3$ thì $\displaystyle\int\limits_0^34f(x)\mathrm{\,d}x$ bằng
	\choice
	{$3$}
	{\True $12$}
	{$36$}
	{$4$}
	\loigiai{
			Ta có $\displaystyle\int\limits_0^34f(x)\mathrm{\,d}x=4\displaystyle\int\limits_0^3f(x)\mathrm{\,d}x=4\cdot 3=12$.}
\end{ex}
\begin{ex}%[Câu 6]%[2D4N2-1]
	Cho $\displaystyle\int\limits_0^2f(x)\mathrm{\,d}x=\dfrac{1}{2024}$. Tính $I=\displaystyle\int\limits_0^2 2024f(x)\mathrm{\,d}x$.
	\choice
	{$I=5$}
	{$I=\dfrac{1}{2024}$}
	{\True $I=1$}
	{$I=2024$}
	\loigiai{
			Ta có $I=\displaystyle\int\limits_0^2 2024f(x)\mathrm{\,d}x=2024\displaystyle\int\limits_0^2f(x)\mathrm{\,d}x=2024\cdot \dfrac{1}{2024}=1$.}
\end{ex}
\begin{ex}%[Câu 7]%[2D4N2-1]
	Nếu $\displaystyle\int\limits_0^5f(x)\mathrm{\,d}x=5$ thì $\displaystyle\int\limits_5^05f(x)\mathrm{\,d}x$ bằng
	\choice
	{$1$}
	{$-1$}
	{$25$}
	{\True $-25$}
	\loigiai{
			Ta có $\displaystyle\int\limits_5^05f(x)\mathrm{\,d}x=5\displaystyle\int\limits_5^0f(x)\mathrm{\,d}x=-5\cdot\displaystyle\int\limits_0^5f(x)\mathrm{\,d}x=(-5)\cdot 5=-25$.
			
			}
\end{ex}
\begin{ex}%[Câu 8]%[2D4N2-1]
	Nếu $\displaystyle\int\limits_0^2f(x)\mathrm{\,d}x=5$ thì $\displaystyle\int\limits_0^2\left[2f(x)-1\right]\mathrm{\,d}x$ bằng
	\choice
	{\True $8$}
	{$9$}
	{$10$}
	{$12$}
	\loigiai{
			Ta có $\displaystyle\int _0^2\left[2f(x)-1\right]\mathrm{\,d}x=2\displaystyle\int _0^2f(x)\mathrm{\,d}x-\displaystyle\int _0^21\mathrm{\,d}x=2\cdot 5-2=8$.}
\end{ex}
\begin{ex}%[Câu 9]%[2D4N2-1]
	Nếu $\displaystyle\int_0^2 f(x) d x=3$ thì $\displaystyle\int_0^2\left[2f(x)-1\right]\mathrm{\,d}x$ bằng
	\choice
	{$6$}
	{\True $4$}
	{$8$}
	{$5$}
	\loigiai{
			Ta có $\displaystyle\int_0^2\left[2f(x)-1\right]\mathrm{\,d}x=2\displaystyle\int_0^2f(x)\mathrm{\,d}x-\displaystyle\int_0^2\mathrm{\,d}x=2\cdot 3-2=4$.}
\end{ex}
\begin{ex}%[Câu 10]%[2D4N2-1]
	Cho $\displaystyle\int\limits_0^1f(x)\mathrm{\,d}x=2$ và $\displaystyle\int\limits_0^1g(x)\mathrm{\,d}x=5$, khi $\displaystyle\int\limits_0^1\left[f(x)-2g(x)\right]\mathrm{\,d}x$ bằng
	\choice
	{\True $-8$}
	{$1$}
	{$-3$}
	{$12$}
	\loigiai{
			Ta có $\displaystyle\int\limits_0^1\left[f(x)-2g(x)\right]\mathrm{\,d}x=\displaystyle\int\limits_0^1f(x)\mathrm{\,d}x-2\displaystyle\int\limits_0^1g(x)\mathrm{\,d}x=2-2\cdot 5=-8$.}
\end{ex}
\begin{ex}%[Câu 11]%[2D4H2-1]
	Cho $\displaystyle\int\limits_0^{\frac{\pi}{2}}f(x)\mathrm{\,d}x=5$. Tính $I=\displaystyle\int\limits_0^{\frac{\pi}{2}}\left[f(x)+2\sin x\right]\mathrm{\,d}x$.
	\choice
	{\True $I=7$}
	{$I=5+\dfrac{\pi}{2}$}
	{$I=3$}
	{$I=5+\pi $}
	\loigiai{
			Ta có
			\begin{eqnarray*}
				&I&=\displaystyle\int\limits_0^{\frac{\pi}{2}}\left[f(x)+2\sin x\right]\mathrm{\,d}x\\
				&&=\displaystyle\int\limits_0^{\frac{\pi}{2}}f(x)\mathrm{\,d}x\text{+2}\displaystyle\int\limits_0^{\tfrac{\pi}{2}}\sin x\mathrm{\,d}x\\
				&&=\displaystyle\int\limits_0^{\frac{\pi}{2}}f(x)\mathrm{\,d}x-2\cos x\bigg|_0^{\frac{\pi}{2}}\\
				&&=5-2(0-1)=7.
			\end{eqnarray*}
		}
\end{ex}
\begin{ex}%[Câu 12]%[2D4H2-1]
	Cho $\displaystyle\int\limits_1^2\left[4f(x)-2x\right]\mathrm{\,d}x=1$. Khi đó $\displaystyle\int\limits_1^2f(x)\mathrm{\,d}x$ bằng
	\choice
	{\True $1$}
	{$-3$}
	{$3$}
	{$-1$}
	\loigiai{
		Ta có \begin{eqnarray*}
			&&\displaystyle\int\limits_1^2\left[4f(x)-2x\right]\mathrm{\,d}x=1\\
			&\Leftrightarrow&4\displaystyle\int\limits_1^2f(x)\mathrm{\,d}x-2\displaystyle\int\limits_1^2x\mathrm{\,d}x=1\\
			&\Leftrightarrow&4\displaystyle\int\limits_1^2f(x)\mathrm{\,d}x-2\cdot  \dfrac{x^2}{2}\bigg|_1^2=1\\
			&\Leftrightarrow&4\displaystyle\int\limits_1^2f(x)\mathrm{\,d}x=4\\
			&\Leftrightarrow&\displaystyle\int\limits_1^2f(x)\mathrm{\,d}x=1.
		\end{eqnarray*}
			}
\end{ex}
\begin{ex}%[Câu 13]%[2D4H2-1]
	Cho $\displaystyle\int\limits_0^1f(x)\mathrm{\,d}x=1$, tích phân $\displaystyle\int\limits_0^1\left(2f(x)-3x^2\right)\mathrm{\,d}x$ bằng
	\choice
	{\True $1$}
	{$0$}
	{$3$}
	{$-1$}
	\loigiai{Ta có 
			$\displaystyle\int\limits_0^1(2f(x)-3x^2)\mathrm{\,d}x=2\displaystyle\int\limits_0^1f(x)\mathrm{\,d}x-3\displaystyle\int\limits_0^1x^2\mathrm{\,d}x=2-1=1$.}
\end{ex}
\begin{ex}%[Câu 14]%[2D4H2-1]
	Cho $\displaystyle\int\limits_{-1}^2f(x)\mathrm{\,d}x=2$ và $\displaystyle\int\limits_{-1}^2g(x)\mathrm{\,d}x=-1$. Tính $I=\displaystyle\int\limits_{-1}^2\left[x+2f(x)-3g(x)\right]\mathrm{\,d}x$.
	\choice
	{\True $I=\dfrac{17}{2}$}
	{$I=\dfrac{5}{2}$}
	{$I=\dfrac{7}{2}$}
	{$I=\dfrac{11}{2}$}
	\loigiai{
			Ta có 
			\begin{eqnarray*}
				&I&=\displaystyle\int\limits_{-1}^2\left[x+2f(x)-3g(x)\right]\mathrm{\,d}x\\
				&&= \dfrac{x^2}{2}\bigg|_{-1}^2+2\displaystyle\int\limits_{-1}^2f(x)\mathrm{\,d}x-3\displaystyle\int\limits_{-1}^2g(x)\mathrm{\,d}x\\
				&&=\dfrac{3}{2}+2\cdot 2-3(-1)=\dfrac{17}{2}.
			\end{eqnarray*}
			}
\end{ex}
\begin{ex}%[Câu 15]%[2D4H2-1]
	Cho $\displaystyle\int\limits_0^2f(x)\mathrm{\,d}x=3$,$\displaystyle\int\limits_0^2g(x)\mathrm{\,d}x=-1$ thì $\displaystyle\int\limits_0^2\left[f(x)-5g(x)+x\right]\mathrm{\,d}x$ bằng
	\choice
	{$12$}
	{$0$}
	{$8$}
	{\True $10$}
	\loigiai{Ta có 
		$\displaystyle\int\limits_0^2\left[f(x)-5g(x)+x\right]\mathrm{\,d}x=\displaystyle\int\limits_0^2f(x)\mathrm{\,d}x-5\displaystyle\int\limits_0^2\mathrm{g}(x)\mathrm{\,d}x+\displaystyle\int\limits_0^2x\mathrm{\,d}x=3+5+2=10$.}
\end{ex}
\begin{ex}%[Câu 16]%[2D4H2-1]
	Cho $\displaystyle\int\limits_0^5f(x)\mathrm{\,d}x=-2$. Tích phân $\displaystyle\int\limits_0^5\left[4f(x)-3x^2\right]\mathrm{\,d}x$ bằng
	\choice
	{$-140$}
	{$-130$}
	{$-120$}
	{\True $-133$}
	\loigiai{Ta có
		$\displaystyle\int\limits_0^5\left[4f(x)-3x^2\right]\mathrm{\,d}x=4\displaystyle\int\limits_0^5f(x)\mathrm{\,d}x-\displaystyle\int\limits_0^53x^2\mathrm{\,d}x=-8-x^3\bigg|_0^5=-8-125=-133$.}
\end{ex}
\begin{ex}%[Câu 17]%[2D4H2-1]
	Cho $\displaystyle\int\limits_1^2\left[4f(x)-2x\right]\mathrm{\,d}x=1$. Khi đó $\displaystyle\int\limits_1^2f(x)\mathrm{\,d}x$ bằng:
	\choice
	{\True $1$}
	{$-3$}
	{$3$}
	{$-1$}
	\loigiai{Ta có
		\begin{eqnarray*}
			&&\displaystyle\int\limits_1^2\left[4f(x)-2x\right]\mathrm{\,d}x=1\\
			&\Leftrightarrow&4\displaystyle\int\limits_1^2f(x)\mathrm{\,d}x-2\displaystyle\int\limits_1^2x\mathrm{\,d}x=1\\
			&\Leftrightarrow&4\displaystyle\int\limits_1^2f(x)\mathrm{\,d}x-2\cdot  \dfrac{x^2}{2}\bigg|_1^2=1\\
			&\Leftrightarrow&4\displaystyle\int\limits_1^2f(x)\mathrm{\,d}x=4\\
			&\Leftrightarrow&\displaystyle\int\limits_1^2f(x)\mathrm{\,d}x=1.
		\end{eqnarray*}
		}
\end{ex}
\begin{ex}%[Câu 18]%[2D4H2-1]
	Cho $\displaystyle\int\limits_{-2}^2f(x)\mathrm{\,d}x=1$, $\displaystyle\int\limits_{-2}^4f(t)\mathrm{\,d}t=-4$. Tính $\displaystyle\int\limits_2^4f(y)\mathrm{\,d}y$.
	\choice
	{$I=5$}
	{$I=-3$}
	{$I=3$}
	{\True $I=-5$}
	\loigiai{	
		Ta có $\displaystyle\int\limits_{-2}^4f(t)\mathrm{\,d}t=\displaystyle\int\limits_{-2}^4f(x)\mathrm{\,d}x$, $\displaystyle\int\limits_2^4f(y)\mathrm{\,d}y=\displaystyle\int\limits_2^4f(x)\mathrm{\,d}x$.\\
		Khi đó $\displaystyle\int\limits_{-2}^2f(x)\mathrm{\,d}x+\displaystyle\int\limits_2^4f(x)\mathrm{\,d}x=\displaystyle\int\limits_{-2}^4f(x)\mathrm{\,d}x$. Do đó
		$$ \displaystyle\int\limits_2^4f(x)\mathrm{\,d}x=\displaystyle\int\limits_{-2}^4f(x)\mathrm{\,d}x-\displaystyle\int\limits_{-2}^2f(x)\mathrm{\,d}x=-4-1=-5.$$
		Vậy $\displaystyle\int\limits_2^4f(y)\mathrm{\,d}y=-5$.}
\end{ex}
\begin{ex}%[Câu 19]%[2D4H2-1]
	Cho hàm số $f(x)$ liên tục trên $\mathbb{R}$ và có $\displaystyle\int\limits_0^2f(x)\mathrm{\,d}x=9;\displaystyle\int\limits_2^4f(x)\mathrm{\,d}x=4$. Tính $I=\displaystyle\int\limits_0^4f(x)\mathrm{\,d}x$.
	\choice
	{$I=5$}
	{$I=36$}
	{$I=\dfrac{9}{4}$}
	{\True $I=13$}
	\loigiai{
			Ta có $I=\displaystyle\int\limits_0^4f(x)\mathrm{\,d}x=\displaystyle\int\limits_0^2f(x)\mathrm{\,d}x+\displaystyle\int\limits_2^4f(x)\mathrm{\,d}x=9+4=13$.}
\end{ex}
\begin{ex}%[Câu 20]%[2D4H2-1]
	Cho hàm số $f(x)$ liên tục trên $\mathbb{R}$ và $\displaystyle\int\limits_0^4f(x)\mathrm{\,d}x=10$, $\displaystyle\int\limits_3^4f(x)\mathrm{\,d}x=4$. Tích phân $\displaystyle\int\limits_0^3f(x)\mathrm{\,d}x$ bằng
	\choice
	{$4$}
	{$7$}
	{$3$}
	{\True $6$}
	\loigiai{
		Theo tính chất của tích phân, ta có $\displaystyle\int\limits_0^3f(x)\mathrm{\,d}x+\displaystyle\int\limits_3^4f(x)\mathrm{\,d}x=\displaystyle\int\limits_0^4f(x)\mathrm{\,d}x$.\\
		Suy ra  $\displaystyle\int\limits_0^3f(x)\mathrm{\,d}x=\displaystyle\int\limits_0^4f(x)\mathrm{\,d}x-\displaystyle\int\limits_3^4f(x)\mathrm{\,d}x=10-4=6$.\\
		Vậy $\displaystyle\int\limits_0^3f(x)\mathrm{\,d}x=6$.}
\end{ex}
\begin{ex}%[Câu 21]%[2D4H2-1]
	Cho hàm số $f(x)$ liên tục trên đoạn $[0;10]$ và $\displaystyle\int\limits_0^{10}f(x)\mathrm{\,d}x=7$; $\displaystyle\int\limits_2^6f(x)\mathrm{\,d}x=3$.\\
	 Tính $P=\displaystyle\int\limits_0^2f(x)\mathrm{\,d}x+\displaystyle\int\limits_6^{10}f(x)\mathrm{\,d}x$.
	\choice
	{\True $P=4$}
	{$P=10$}
	{$P=7$}
	{$P=-4$}
	\loigiai{
	Ta có $\displaystyle\int\limits_0^{10}f(x)\mathrm{\,d}x=\displaystyle\int\limits_0^2f(x)\mathrm{\,d}x+\displaystyle\int\limits_2^6f(x)\mathrm{\,d}x+\displaystyle\int\limits_6^{10}f(x)\mathrm{\,d}x$ hay $7=P+3\Leftrightarrow P=4$.}
\end{ex}
\begin{ex}%[Câu 22]%[2D4H2-1]
	Cho hàm số $f(x)$ liên tục trên đoạn $[0; 6]$ thỏa mãn $\displaystyle\int\limits_0^6f(x)\mathrm{\,d}x=10$ và $\displaystyle\int\limits_2^4f(x)\mathrm{\,d}x=6$. 	Tính giá trị của biểu thức $P=\displaystyle\int\limits_0^2f(x)\mathrm{\,d}x+\displaystyle\int\limits_4^6f(x)\mathrm{\,d}x$.
	\choice
	{\True$P=4$}
	{$P=16$}
	{$P=8$}
	{$P=10$}
	\loigiai{
	Ta có $\displaystyle\int\limits_0^{6}f(x)\mathrm{\,d}x=\displaystyle\int\limits_0^2f(x)\mathrm{\,d}x+\displaystyle\int\limits_2^4f(x)\mathrm{\,d}x+\displaystyle\int\limits_4^{6}f(x)\mathrm{\,d}x$ hay $7=P+3\Leftrightarrow P=4$.	
		}
\end{ex}
\textbf{PHẦN II. Câu trắc nghiệm đúng sai. Trong mỗi ý A), B), C), D) ở mỗi câu, thí sinh chọn đúng hoặc sai}
\begin{ex}%[Câu 23]%[2D4H2-1]
	Cho hai hàm $f$, $g$ liên tục trên $K$ và $a$, $b$ là các số bất kỳ thuộc $K$.
	\choiceTF
	{\True $\displaystyle\int\limits_a^b\left[f(x)+2g(x)\right]\mathrm{\,d}x=\displaystyle\int\limits_a^bf(x)\mathrm{\,d}x\text{+2}\displaystyle\int\limits_a^bg(x)\mathrm{\,d}x$}
	{$\displaystyle\int\limits_a^b\dfrac{f(x)}{g(x)}\mathrm{\,d}x=\dfrac{\displaystyle\int\limits_a^bf(x)\mathrm{\,d}x}{\displaystyle\int\limits_a^bg(x)\mathrm{\,d}x}$}
	{$\displaystyle\int\limits_a^b\left[f(x)\cdot g(x)\right]\mathrm{\,d}x=\displaystyle\int\limits_a^bf(x)\mathrm{\,d}x \displaystyle\int\limits_a^bg(x)\mathrm{\,d}x$}
	{$\displaystyle\int\limits_a^bf^2(x)\mathrm{\,d}x=\left[\displaystyle\int\limits_a^bf(x)\mathrm{\,d}x\right]^2$}
	\loigiai{
		\begin{itemchoice}
			\itemch Đúng. Theo tính chất tích phân ta có
			$\displaystyle\int\limits_a^b\left[f(x)+g(x)\right]\mathrm{\,d}x=\displaystyle\int\limits_a^bf(x)\mathrm{\,d}x+\displaystyle\int\limits_a^bg(x)\mathrm{\,d}x;\displaystyle\int\limits_a^bkf(x)\mathrm{\,d}x=k\displaystyle\int\limits_a^bf(x)\mathrm{\,d}x$, với $k\in \mathbb{R}$.
			\itemch Sai. Cho $a=1,b=2$ và $f(x)=x+1, g(x)=x$. Khi đó
			$$VT=\displaystyle\int\limits_{1}^2\dfrac{x+1}{x}\mathrm{\,d}x==\displaystyle\int\limits_{1}^2\left(1+\dfrac{1}{x}\right)\mathrm{\,d}x=\left(x+\ln x\right)\bigg|_1^2=1+\ln 2.$$
			và $$VP=\dfrac{\displaystyle\int\limits_1^2(x+1)\mathrm{\,d}x}{\displaystyle\int\limits_1^2x\mathrm{\,d}x}=\dfrac{\left(\dfrac{x^2}{2}+x\right)\bigg|_1^2}{\dfrac{x^2}{2}\bigg|_1^2}=\dfrac{1}{3}.$$
			Do đó $VT\neq VP$.
			\itemch Sai. Cho $a=1, b=2$ và $f(x)=x, g(x)=\dfrac{1}{x}$. Khi đó
			$$VT=\displaystyle\int\limits_1^2\left[x\cdot \dfrac{1}{x}\right]\mathrm{\,d}x=x\bigg|_1^2=1.$$
			 và $$VP=\displaystyle\int\limits_1^2x\mathrm{\,d}x\cdot \displaystyle\int\limits_1^2\dfrac{1}{x}\mathrm{\,d}x=\left(\dfrac{x^2}{2}\right)\bigg|_1^2\cdot \ln x\bigg|_1^2=\dfrac{3}{2}\ln 2.$$
			 Do đó $VT\neq VP$.
			\itemch Sai. Cho $a=1,b=2$ và $f(x)=x$. Khi đó
			$$VT=\displaystyle\int\limits_1^2x^2\mathrm{\,d}x=\left(\dfrac{x^3}{3}\right)\bigg|_1^2=\dfrac{7}{3}.$$
			và $$VP=\left(\displaystyle\int\limits_1^2x\mathrm{\,d}x\right)^2=\left(\dfrac{x^2}{2}\bigg|_1^2\right)^2=\dfrac{9}{4}.$$
			Do đó $VT\neq VP$.
		\end{itemchoice}
		}
	\end{ex}
	\begin{ex}%[Câu 24]%[2D4H2-1]
		Cho hàm số $f(x),g(x)$ liên tục trên $\mathbb{R}$.
		\choiceTF
		{\True Nếu $\displaystyle\int\limits_0^2f(x)\mathrm{\,d}x=4$ thì $\displaystyle\int\limits_0^2\left[\dfrac{1}{2}f(x)+2\right]\mathrm{\,d}x=6$}
		{\True Nếu $\displaystyle\int\limits_2^5f(x)\mathrm{\,d}x=3$ và $\displaystyle\int\limits_2^5g(x)\mathrm{\,d}x=-2$ thì $\displaystyle\int\limits_2^5\left[f(x)+g(x)\right]\mathrm{\,d}x=1$}
		{Nếu $\displaystyle\int\limits_1^4f(x)\mathrm{\,d}x=6$ và $\displaystyle\int\limits_1^4g(x)\mathrm{\,d}x=-5$ thì $\displaystyle\int\limits_1^4\left[f(x)-g(x)\right]\mathrm{\,d}x=1$}
		{\True Nếu $\displaystyle\int\limits_2^3f(x)\mathrm{\,d}x=4$ và$\displaystyle\int\limits_2^3g(x)\mathrm{\,d}x=1$ thì $\displaystyle\int\limits_2^3\left[f(x)-g(x)\right]\mathrm{\,d}x=3$}
		\loigiai{
			\begin{itemchoice}
				\itemch Đúng. Ta có $\displaystyle\int\limits_0^2\left[\dfrac{1}{2}f(x)+2\right]\mathrm{\,d}x=\dfrac{1}{2}\displaystyle\int\limits_0^2f(x)\mathrm{\,d}x+\displaystyle\int\limits_0^22\mathrm{\,d}x=\dfrac{1}{2}\cdot 4+4=6$.
				\itemch Đúng. Ta có $\displaystyle\int\limits_2^5\left[f(x)+g(x)\right]\mathrm{\,d}x=\displaystyle\int\limits_2^5f(x)\mathrm{\,d}x+\displaystyle\int\limits_2^5g(x)\mathrm{\,d}x=3+(-2)=1$.
				\itemch Sai. Ta có $\displaystyle\int\limits_1^4\left[f(x)-g(x)\right]\mathrm{\,d}x=\displaystyle\int\limits_1^4f(x)\mathrm{\,d}x-\displaystyle\int\limits_1^4g(x)\mathrm{\,d}x=6-(-5)=11$.
				\itemch Đúng. Ta có $\displaystyle\int\limits_2^3\left[f(x)-g(x)\right]\mathrm{\,d}x=\displaystyle\int\limits_2^3f(x)\mathrm{\,d}x-\displaystyle\int\limits_2^3g(x)\mathrm{\,d}x=4-1=3$.
			\end{itemchoice}
				}
		\end{ex}
		\begin{ex}%[Câu 25]%[2D4H2-1]
			Cho hàm số $f(x),g(x)$ liên tục trên $\mathbb{R}$.
			\choiceTF
			{Biết $\displaystyle\int\limits_2^3f(x)\mathrm{\,d}x=3$ và $\displaystyle\int\limits_3^2g(x)\mathrm{\,d}x=1$. Khi đó $\displaystyle\int\limits_2^3\left[f(x)+g(x)\right]\mathrm{\,d}x=4$}
			{\True Biết $\displaystyle\int\limits_1^3f(x)\mathrm{\,d}x=2022$ và $\displaystyle\int\limits_3^1g(x)\mathrm{\,d}x=1$. Khi đó $\displaystyle\int\limits_1^3\left[f(x)+g(x)\right]\mathrm{\,d}x=2021$}
			{\True Biết $\displaystyle\int\limits_1^2f(x)\mathrm{\,d}x=3$ và $\displaystyle\int\limits_1^2g(x)\mathrm{\,d}x=2$. Khi đó $\displaystyle\int\limits_1^2\left[f(x)-g(x)\right]\mathrm{\,d}x=1$}
			{Biết $\displaystyle\int\limits_2^5f(x)\mathrm{\,d}x=2$. Khi đó $\displaystyle\int\limits_2^53f(x)\mathrm{\,d}x=2$}
			\loigiai{
				\begin{itemchoice}
					\itemch Sai. Ta có
					$\displaystyle\int\limits_2^3\left[f(x)+g(x)\right]\mathrm{\,d}x=\displaystyle\int\limits_2^3f(x)\mathrm{\,d}x+\displaystyle\int\limits_2^3g(x)\mathrm{\,d}x=\displaystyle\int\limits_2^3f(x)\mathrm{\,d}x-\displaystyle\int\limits_3^2g(x)\mathrm{\,d}x=2$.
					\itemch Đúng. Ta có $\displaystyle\int\limits_3^1g(x)\mathrm{\,d}x=1\Leftrightarrow \displaystyle\int\limits_1^3g(x)\mathrm{\,d}x=-1$. Do đó 
					$$\displaystyle\int\limits_1^3\left[f(x)+g(x)\right]\mathrm{\,d}x=\displaystyle\int\limits_1^3f(x)\mathrm{\,d}x+\displaystyle\int\limits_1^3g(x)\mathrm{\,d}x=2022+(-1)=2021.$$
					\itemch Đúng. Ta có $\displaystyle\int\limits_1^2\left[f(x)-g(x)\right]\mathrm{\,d}x=\displaystyle\int\limits_1^2f(x)\mathrm{\,d}x-\displaystyle\int\limits_1^2g(x)\mathrm{\,d}x=3-2=1$.
					\itemch Sai. Ta có $\displaystyle\int\limits_2^53f(x)\mathrm{\,d}x=3\displaystyle\int\limits_2^5f(x)\mathrm{\,d}x=3\cdot 2=6$.
				\end{itemchoice}
					}
			\end{ex}
			\begin{ex}%[Câu 26]%[2D4H2-1]
				Cho hàm số $f(x)$ liên tục trên $\mathbb{R}$.
				\choiceTF
				{\True Nếu $\displaystyle\int\limits_0^3f(x)\mathrm{\,d}x=3$ thì $\displaystyle\int\limits_0^32f(x)\mathrm{\,d}x=6$}
				{\True Nếu $\displaystyle\int\limits_1^4f(x)\mathrm{\,d}x=2024$ thì $\displaystyle\int\limits_4^1f(x)\mathrm{\,d}x=-2024$}
				{Nếu $\displaystyle\int\limits_6^0f(x)\mathrm{\,d}x=12$ thì $\displaystyle\int\limits_0^62022f(x)\mathrm{\,d}x=24264$}
				{\True Nếu $\displaystyle\int\limits_0^1f(x)\mathrm{\,d}x=4$ thì $\displaystyle\int\limits_0^12f(x)\mathrm{\,d}x=8$}
				\loigiai{
					\begin{itemchoice}
						\itemch Đúng. Ta có $\displaystyle\int\limits_0^32f(x)\mathrm{\,d}x=2\displaystyle\int\limits_0^3f(x)\mathrm{\,d}x=2\cdot 3=6$.
						\itemch Đúng. Ta có $\displaystyle\int\limits_4^1f(x)\mathrm{\,d}x=-\displaystyle\int\limits_1^4f(x)\mathrm{\,d}x=-2024$.
						\itemch Sai. Ta có $\displaystyle\int\limits_0^62022f(x)\mathrm{\,d}x=2022\displaystyle\int\limits_0^6f(x)\mathrm{\,d}x=2022\cdot (-12)=-24264$.
						\itemch Đúng. Ta có $\displaystyle\int\limits_0^12f(x)\mathrm{\,d}x=2\displaystyle\int\limits_0^1f(x)\mathrm{\,d}x=2\cdot 4=8$.
					\end{itemchoice}
						}
				\end{ex}
				\begin{ex}%[Câu 27]%[2D4H2-1]
					Cho hàm số $f(x),g(x)$ liên tục trên $\mathbb{R}$.
					\choiceTF
					{Nếu $\displaystyle\int_0^2 f(x)d x=6$ thì $\displaystyle\int_0^2\left[2f(x)-1\right]\mathrm{\,d}x=-10$}
					{\True Nếu $\displaystyle\int\limits_0^2f(x)\mathrm{\,d}x=4$ thì $\displaystyle\int\limits_0^2\left[2f(x)-1)\right]\mathrm{\,d}x=6$}
					{\True Nếu $\displaystyle\int_0^2f(x)\mathrm{\,d}x=3$ và $\displaystyle\int_0^2g(x)\mathrm{\,d}x=7$ thì $\displaystyle\int_0^2\left[f(x)+3g(x)\right]\mathrm{\,d}x=24$}
					{\True Nếu $\displaystyle\int\limits_0^1\left[f(x)+2x\right]\mathrm{\,d}x=3$ thì $\displaystyle\int\limits_0^1f(x)\mathrm{\,d}x=2$}
					\loigiai{
						\begin{itemchoice}
							\itemch Sai. Ta có $\displaystyle\int _0^2\left[2f(x)-1\right]\mathrm{\,d}x=2\displaystyle\int _0^2f(x)\mathrm{\,d}x-\displaystyle\int _0^2\mathrm{\,d}x=2\cdot 6-2=10$.
							\itemch Đúng. Ta có $\displaystyle\int\limits_0^2\left[2f(x)-1)\right]\mathrm{\,d}x=\displaystyle\int\limits_0^22f(x)\mathrm{\,d}x-\displaystyle\int\limits_0^2\mathrm{\,d}x=2\cdot 4-2=6$.
							\itemch Đúng. Ta có $\displaystyle\int_0^2\left[f(x)+3g(x)\right]\mathrm{\,d}x=\displaystyle\int_0^2f(x)\mathrm{\,d}x+3\displaystyle\int_0^2g(x)\mathrm{\,d}x=3+3\cdot 7=24$.
							\itemch Đúng. Ta có:\\ $\displaystyle\int\limits_0^1\left[f(x)+2x\right]\mathrm{\,d}x=3\Leftrightarrow \displaystyle\int\limits_0^1f(x)\mathrm{\,d}x+2\displaystyle\int\limits_0^1x\mathrm{\,d}x=3\Leftrightarrow \displaystyle\int\limits_0^1f(x)\mathrm{\,d}x+2\cdot \dfrac{x^2}{2}\bigg|_0^1=3$.\\
							Suy ra $\displaystyle\int\limits_0^1f(x)\mathrm{\,d}x=3-x^2\bigg|_0^1=3-(1-0)=2$.
						\end{itemchoice}
							}
					\end{ex}
					\begin{ex}%[Câu 28]%[2D4H2-1]
						Cho hàm số $f(x),g(x)$ liên tục trên $\mathbb{R}$.
						\choiceTF
						{\True Nếu $\displaystyle\int\limits_{-1}^5f(x)\mathrm{\,d}x=-3$ thì $\displaystyle\int\limits_5^{-1}f(x)\mathrm{\,d}x=3$}
						{\True Nếu $\displaystyle\int\limits_2^3f(x)\mathrm{\,d}x=-6$ thì $\displaystyle\int\limits_3^22f(x)\mathrm{\,d}x=12$}
						{Nếu $\displaystyle\int\limits_1^2f(x)\mathrm{\,d}x=2$ và $\displaystyle\int\limits_1^2g(x)\mathrm{\,d}x=6$ thì $\displaystyle\int\limits_2^1\left[f(x)-g(x)\right]\mathrm{\,d}x=-4$}
						{Nếu $\displaystyle\int\limits_0^1f(x)\mathrm{\,d}x=3$ và $\displaystyle\int\limits_0^1g(x)\mathrm{\,d}x=-4$ thì $\displaystyle\int\limits_1^0\left[f(x)+g(x)\right]\mathrm{\,d}x=-1$}
						\loigiai{
							\begin{itemchoice}
								\itemch Đúng. Ta có $\displaystyle\int _5^{-1}f(x)\mathrm{\,d}x=-\displaystyle\int _{-1}^5f(x)\mathrm{\,d}x=-(-3)=3$.
								\itemch Đúng. Ta có $\displaystyle\int\limits_3^22f(x)\mathrm{\,d}x=-\displaystyle\int\limits_2^32f(x)\mathrm{\,d}x=-2\displaystyle\int\limits_2^3f(x)\mathrm{\,d}x=-2\cdot (-6)=12$.
								\itemch Sai. Ta có \\
								$\displaystyle\int\limits_2^1\left[f(x)-g(x)\right]\mathrm{\,d}x=-\displaystyle\int\limits_1^2\left[f(x)-g(x)\right]\mathrm{\,d}x=-\displaystyle\int\limits_1^2f(x)\mathrm{\,d}x+\displaystyle\int\limits_1^2g(x)\mathrm{\,d}x=-2+6=4$.
								\itemch Sai. Ta có\\
								$\displaystyle\int\limits_1^0\left[f(x)+g(x)\right]\mathrm{\,d}x=-\displaystyle\int\limits_0^1\left[f(x)+g(x)\right]\mathrm{\,d}x=-\displaystyle\int\limits_0^1f(x)\mathrm{\,d}x-\displaystyle\int\limits_0^1g(x)\mathrm{\,d}x=-3+4=1$.
							\end{itemchoice}
							}
						\end{ex}
						\begin{ex}%[Câu 29]%[2D4V2-1]
							Cho hàm số $f(x),g(x)$ liên tục trên $\mathbb{R}$.
							\choiceTF
							{\True Nếu $\displaystyle\int\limits_0^1f(x)\mathrm{\,d}x=-1$ và $\displaystyle\int\limits_0^3f(x)\mathrm{\,d}x=5$ thì $\displaystyle\int\limits_1^3f(x)=6$}
							{Nếu $\displaystyle\int\limits_1^2f(x)\mathrm{\,d}x=-3$ và $\displaystyle\int\limits_2^3f(x)\mathrm{\,d}x=4$ thì $\displaystyle\int\limits_1^3f(x)\mathrm{\,d}x=-1$}
							{Nếu $\displaystyle\int\limits_{-1}^0f(x)\mathrm{\,d}x=3, \displaystyle\int\limits_{0}^3f(x)\mathrm{\,d}x=1$ thì $\displaystyle\int\limits_{-1}^3f(x)\mathrm{\,d}x=-4$}
							{Nếu $\displaystyle\int\limits_{-2}^{5}f(x)\mathrm{\,d}x=8$ và $\displaystyle\int\limits_5^{-2}g(x)\mathrm{\,d}x=3$ thì $\displaystyle\int\limits_{-2}^5\left(f(x)-4g(x)-1\right)\mathrm{\,d}x=-13$}
							\loigiai{
								\begin{itemchoice}
									\itemch Đúng. Ta có 
									$\displaystyle\int\limits_0^3f(x)\mathrm{\,d}x =\displaystyle\int\limits_0^1f(x)\mathrm{\,d}x +\displaystyle\int\limits_1^3f(x)\mathrm{\,d}x$.\\
									Do đó $\displaystyle\int\limits_1^3f(x)\mathrm{\,d}x =\displaystyle\int\limits_0^3f(x)\mathrm{\,d}x-\displaystyle\int\limits_0^1f(x)\mathrm{\,d}x = 5+ 1= 6$.
									\itemch Sai. Ta có $\displaystyle\int\limits_1^3f(x)\mathrm{\,d}x=\displaystyle\int\limits_1^2f(x)\mathrm{\,d}x+\displaystyle\int\limits_2^3f(x)\mathrm{\,d}x=-3+4=1$.
									\itemch Sai. Ta có $\displaystyle\int\limits_{-1}^0f(x)\mathrm{\,d}x=3;\displaystyle\int\limits_{0}^3f(x)\mathrm{\,d}x=1;\displaystyle\int\limits_{-1}^3f(x)\mathrm{\,d}x=\displaystyle\int\limits_{-1}^0f(x)\mathrm{\,d}x+\displaystyle\int\limits_{0}^3f(x)\mathrm{\,d}x=3+1=4$.
									\itemch Sai. Ta có 
									\begin{eqnarray*}
										&&\displaystyle\int\limits_{-2}^5\left[f(x)-4g(x)-1\right]\mathrm{\,d}x\\
										&&=\displaystyle\int\limits_{-2}^5f(x)\mathrm{\,d}x-\displaystyle\int\limits_{-2}^54g(x)\mathrm{\,d}x-\displaystyle\int\limits_{-2}^5\mathrm{\,d}x\\
										&&=\displaystyle\int\limits_{-2}^5f(x)\mathrm{\,d}x-4\displaystyle\int\limits_{-2}^5g(x)\mathrm{\,d}x-\displaystyle\int\limits_{-2}^5\mathrm{\,d}x\\
										&&=\displaystyle\int\limits_{-2}^5f(x)\mathrm{\,d}x+4\displaystyle\int\limits_5^{-2}g(x)\mathrm{\,d}x-\displaystyle\int\limits_{-2}^5\mathrm{\,d}x\\
										&&=8+4\cdot 3-x\bigg|_{-2}^5=8+4\cdot 3-7=13.
									\end{eqnarray*}
								\end{itemchoice}
								}
							\end{ex}
\begin{ex}%[Câu 30]%[2D4V2-1]
								Cho hàm số $f(x),g(x)$ liên tục trên $\mathbb{R}$.
								\choiceTF
								{\True Biết $\displaystyle\int\limits_1^2f(x)\mathrm{\,d}x=2$. Giá trị của  $\displaystyle\int\limits_2^13f(x)\mathrm{\,d}x=-6$}
								{Biết $\displaystyle\int\limits_1^2f(x)\mathrm{\,d}x=-1$ và $\displaystyle\int\limits_1^2g(x)\mathrm{\,d}x=3$, khi đó $\displaystyle\int\limits_2^1\left[f(x)-g(x)\right]\mathrm{\,d}x=5$}
								{\True Nếu $\displaystyle\int\limits_1^2f(x)\mathrm{\,d}x=-2$ và $\displaystyle\int\limits_2^3f(x)\mathrm{\,d}x=1$ thì $\displaystyle\int\limits_1^3f(x)\mathrm{\,d}x=-1$}
								{\True Nếu $\displaystyle\int\limits_0^2(f(x)+3x^2)\mathrm{\,d}x=10$ thì $\displaystyle\int\limits_0^2f(x)\mathrm{\,d}x=2$}
								\loigiai{
									\begin{itemchoice}
										\itemch Đúng. Biết $\displaystyle\int\limits_1^2f(x)\mathrm{\,d}x=2$. Giá trị của $\displaystyle\int\limits_2^13f(x)\mathrm{\,d}x=-6$.\\
										Ta có $\displaystyle\int\limits_2^13f(x)\mathrm{\,d}x=-\displaystyle\int\limits_1^23f(x)\mathrm{\,d}x=-3\displaystyle\int\limits_1^2f(x)\mathrm{\,d}x=-3\cdot 2=-6$.
										\itemch Sai. Biết $\displaystyle\int\limits_1^2f(x)\mathrm{\,d}x=-1$ và $\displaystyle\int\limits_1^2g(x)\mathrm{\,d}x=3$.\\
										Ta có $\displaystyle\int\limits_1^2f(x)\mathrm{\,d}x=-1\Leftrightarrow \displaystyle\int\limits_2^1f(x)\mathrm{\,d}x=1$ và $\displaystyle\int\limits_1^2g(x)\mathrm{\,d}x=3\Leftrightarrow \displaystyle\int\limits_2^1g(x)\mathrm{\,d}x=-3$.\\
										Do vậy,  $\displaystyle\int\limits_2^1\left[f(x)-g(x)\right]\mathrm{\,d}x=\displaystyle\int\limits_2^1f(x)\mathrm{\,d}x-\displaystyle\int\limits_2^1g(x)\mathrm{\,d}x=1-(-3)=4$.
										\itemch Đúng. Nếu $\displaystyle\int\limits_1^2f(x)\mathrm{\,d}x=-2$ và $\displaystyle\int\limits_2^3f(x)\mathrm{\,d}x=1$ thì $\displaystyle\int\limits_1^3f(x)\mathrm{\,d}x=-1$.\\
										Ta có $\displaystyle\int\limits_1^3f(x)\mathrm{\,d}x=\displaystyle\int\limits_1^2f(x)\mathrm{\,d}x+\displaystyle\int\limits_2^3f(x)\mathrm{\,d}x=-2+1=-1$.
										\itemch Đúng. Ta có
										\begin{eqnarray*}
											&&\displaystyle\int\limits_0^2(f(x)+3x^2)\mathrm{\,d}x=10\\
											&\Leftrightarrow&\displaystyle\int\limits_0^2f(x)\mathrm{\,d}x+\displaystyle\int\limits_0^23x^2\mathrm{\,d}x=10\\
											&\Leftrightarrow&\displaystyle\int\limits_0^2f(x)\mathrm{\,d}x=10-\displaystyle\int\limits_0^23x^2\mathrm{\,d}x\\
											&\Leftrightarrow&\displaystyle\int\limits_0^2f(x)\mathrm{\,d}x=10-x^3\bigg|_0^2\\
											&\Leftrightarrow&\displaystyle\int\limits_0^2f(x)\mathrm{\,d}x=10-8=2.	\end{eqnarray*}
										\end{itemchoice}
										}
\end{ex}
