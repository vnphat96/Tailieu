\chude{Tích phân hàm ẩn biến đổi phức tạp}
\begin{tomtat}
	Cần nhớ các công thức đạo hàm của hàm hợp
	\begin{itemize}
		\item $\displaystyle\int f'(x)\mathrm{\,d}x=f(x)+C$
		\item $f'(x)\cdot g(x)+f(x)\cdot g'(x)=\left[f(x)\cdot g(x)\right]'$
		\item $\dfrac{f'(x)\cdot g(x)-f(x)\cdot g'(x)}{g^2(x)}=\left[\dfrac{f(x)}{g(x)}\right]'$
		\item $\dfrac{f'(x)}{f(x)}=\left[\ln\left(f(x)\right)\right]'$
		\item $ -\dfrac{f'(x)}{f^2(x)}=\left[\dfrac 1{f(x)}\right]'$
		\item $-\dfrac{f'(x)}{f^n(x)}=\left[\dfrac 1{(n-1)[f(x)]^{n-1}}\right]'$
		\item $n\cdot f'(x)\cdot \left(f(x)\right)^{n-1}=\left[f(x)^n\right]'$
		\item $\dfrac{f'(x)}{\sqrt{f(x)}}=\left[2\sqrt{f(x)}\right]'$
	\end{itemize}	 
\end{tomtat}
\begin{dang}{.}
\begin{enumerate}
		
\item[1.]  Điều kiện hàm ẩn có dạng$\colon $ $\hoac{&f'(x)=g(x) \cdot h\left[f(x)\right] \\ & f'(x) \cdot h[f(x)]=g(x).}$

Phương pháp giải$\colon $
	\begin{itemize}
	\item $\dfrac{f'(x)}{h[f(x)]}=g(x) \Leftrightarrow \displaystyle\int \dfrac{f'(x)}{h[f(x)]} \mathrm{\,d}x=\displaystyle\int g(x) \mathrm{\,d}x \Leftrightarrow \displaystyle\int \dfrac{d[f(x)]}{h[f(x)]}=\displaystyle\int g(x)\mathrm{\,d}x.$
	\item $f'(x) h[f(x)]=g(x) \Leftrightarrow \displaystyle\int f'(x) h[f(x)] \mathrm{\,d}x=\displaystyle\int g(x)\mathrm{\,d}x$\\$ \Leftrightarrow \displaystyle\int h[f(x)]  d\left[f'(x)\right]=\displaystyle\int g(x).$
	\end{itemize}
Chú ý$\colon$ Ngoài việc nguyên hàm hai vế, ta có thể lấy tích phân hai vế (tùy câu hỏi của bài toán).\\
\item[2.] Điều kiện hàm ẩn có dạng$\colon $ $\hoac{&f'(x)+p(x) \cdot f(x)=0 \\ & f'(x)+p(x) \cdot[f(x)]^n=0.}$

Phương pháp giải$\colon $
\begin{itemize}
	\item $f'(x)+p(x) \cdot f(x)=0.$\\
Chia hai vế với $f(x)$ ta đựơc $\dfrac{f'(x)}{f(x)}+p(x)=0 \Leftrightarrow \dfrac{f'(x)}{f(x)}=-p(x).$\\
Suy ra $\displaystyle\int \dfrac{f'(x)}{f(x)} \mathrm{d} x=-\displaystyle\int p(x) \mathrm{d} x \Leftrightarrow \ln |f(x)|=-\displaystyle\int p(x) \mathrm{d} x$.\\
Từ đây ta dễ dàng tính được $f(x).$
	\item $f'(x)+p(x) \cdot[f(x)]^n=0$\\
Chia hai vế với $[f(x)]^n$ ta được $\dfrac{f'(x)}{[f(x)]^n}+p(x)=0 \Leftrightarrow \dfrac{f'(x)}{[f(x)]^n}=-p(x).$
\end{itemize}
\end{enumerate}
\end{dang}
\Opensolutionfile{ans}[ans/ans-2-B1]
\TN  
\begin{ex}%[2D4C2-2]
	Cho hàm số $f(x)$ nhận giá trị không âm và có đạo hàm liên tục trên $\mathbb{R}$ thỏa mãn $f'(x)=(2x+1){{\left[f(x) \right]}^2},\forall x\in \mathbb{R}$ và $f(0)=-1$. Giá trị của tích phân $\displaystyle\int\limits_0^1\left(x^3-1\right)f(x)\mathrm{\,d}x$ bằng
	\choice
	{$1$}
	{$\dfrac{2}{3}$}
	{\True $\dfrac{1}{2}$}
	{$\dfrac{3}{2}$}
	\loigiai{
		Ta có
		$$
		\begin{aligned}
			 &&f'(x)=(2x+1)[f(x)]^2,\forall x\in\mathbb{R}\\
			&\Rightarrow&\dfrac{-f'(x)}{[f(x)]^2}=-(2x+1),\forall x\in\mathbb{R}\\ 			
			&\Rightarrow&\left[\dfrac 1{f(x)}\right]'=-(2x+1),\forall x\in\mathbb{R}.
		\end{aligned}
		$$
		Suy ra $\dfrac{1}{f(x)}=-\displaystyle\int{\left(2x+1\right)}\mathrm{\,d}x=-x^2-x+C\Rightarrow f(x)=\dfrac{1}{-x^2-x+C}$.\\
		Vì  $f(0)=-1\Rightarrow C=-1$.\\
		Suy ra $f(x)=-\dfrac{1}{x^2+x+1}$.\\
		$\displaystyle\int\limits_0^1\left(x^3-1\right)f(x)\mathrm{\,d}x=-\displaystyle\int\limits_0^1\left(x^3-1\right)\left(\dfrac{1}{x^2+x+1}\right)\mathrm{\,d}x=\displaystyle\int\limits_0^1\left(1-x\right)\mathrm{\,d}x$\\
		$=\left.\left(x-\dfrac{x^2}{2}\right)\right|_0^1=\dfrac{1}{2}$.}
\end{ex}

\begin{ex}%[2D4C2-2]
	Cho hàm số $f(x)\ne 0$, liên tục trên đoạn $\left[1;2\right]$ và thỏa mãn $f(1)=\dfrac{1}{3}$; $\linebreak x^2\cdot f'(x)=f^2(x)$ với $\forall x\in\left[1;2\right]$. Tính tích phân $I=\displaystyle\int\limits_1^2\left(2x+1\right)^2f(x)\mathrm{\,d}x$.
	\choice
	{$I=\dfrac{7}{6}$}
	{$I=\dfrac{5}{6}$}
	{\True $I=\dfrac{37}{6}$}
	{$I=\dfrac{1}{6}$}
	\loigiai{
		Ta có
		$$
		\begin{aligned}
		&x^2\cdot f'(x)=f^2(x)\\ 
		\Rightarrow&\dfrac{f'(x)}{f^2(x)}=\dfrac 1{x^2}\\ 
		\Rightarrow&{\left[-\dfrac 1{f(x)}\right]'}=\dfrac 1{x^2}\\ 
		\Rightarrow&-\dfrac 1{f(x)}=\displaystyle\int{\dfrac 1{x^2}}\mathrm{\,d}x\\
		 \Rightarrow&\dfrac 1{f(x)}=-\displaystyle\int{\dfrac 1{x^2}}\mathrm{\,d}x\\
		  \Rightarrow&\dfrac 1{f(x)}=\dfrac 1 x+C.\\ 
		\end{aligned}
		$$
		Mà $f(1)=\dfrac{1}{3}$ $\Rightarrow 3=1+C\Rightarrow C=2.$\\
		Do đó $\dfrac{1}{f(x)}=\dfrac{1}{x}+2 \Rightarrow f(x)=\dfrac{x}{2x+1}.$\\
		Vậy $I=\displaystyle\int\limits_1^2\left(2x+1\right)^2f(x)\mathrm{\,d}x=\displaystyle\int\limits_1^2\left(2x+1\right)^2\dfrac{x}{2x+1}\mathrm{\,d}x=\displaystyle\int\limits_1^2\left(2x^2+x\right)\mathrm{\,d}x=\dfrac{37}{6}$.}
\end{ex}

\begin{ex}%[2D4C2-2]
	Cho hàm số $f(x)$ có đạo hàm trên $\mathbb{R}$ thỏa mãn $3f'(x)\cdot \mathrm{e}^{f^3(x)}-\dfrac{2x}{f^2(x)}=0$ với $\forall x\in\mathbb{R}$. Biết $f(1)=0$, tính tích phân $I=\displaystyle\int\limits_0^{2024}{\dfrac{1}{\sqrt[3]{2\ln x}}\cdot f(x){\mathrm{\,d}}x}$.
	\choice
	{$1$}
	{$\dfrac{1}{2024}$}
	{\True $2024$}
	{$0$}
	\loigiai{
		Ta có
		$$
		\begin{aligned}
		&3f'(x)\cdot\mathrm{e}^{f^3(x)}-\dfrac{2x}{f^2(x)}=0\\ 
		\Rightarrow& 3f^2(x)\cdot f'(x)\cdot\mathrm{e}^{f^3(x)}=2x \\
		\Rightarrow&\left[\mathrm{e}^{f^3(x)}\right]'=2x \\
		\Rightarrow&\mathrm{e}^{f^3(x)}=\displaystyle\int{2x}\mathrm{\,d}x \\
		\Rightarrow&\mathrm{e}^{f^3(x)}=x^2+C.\\ 
		\end{aligned}
		$$
		Mặt khác $f(1)=0\Rightarrow\mathrm{e}^{f^3(1)}=1+C\Rightarrow C=0.$\\
		Suy ra $\mathrm{e}^{f^3(x)}=x^2\Rightarrow{f^3}(x)=\ln{x^2}\Rightarrow f(x)=\sqrt[3]{2\ln x}$.\\
		Vậy $I=\displaystyle\int\limits_0^{2024}\dfrac 1{\sqrt[3]{2\ln x}}\cdot f(x)\mathrm{\,d}x=\displaystyle\int\limits_0^{2024}\dfrac 1{\sqrt[3]{2\ln x}}\cdot \sqrt[3]{2\ln x}\mathrm{\,d}x=\displaystyle\int\limits_0^{2024}\mathrm{\,d}x=2024$}
\end{ex}

\begin{ex}%[2D4C2-2]
	Cho hàm số $f(x)$ đồng biến, có đạo hàm trên đoạn $\left[1;4\right]$ và thoả mãn $x+2x\cdot f(x)=\left[f'(x)\right]^2$ với $\forall x\in\left[1;4\right]$. Biết $f(1)=\dfrac{3}{2}$, tính $I=\displaystyle\int\limits_1^4f(x)\mathrm{\,d}x$.
	\choice
	{\True $I=\dfrac{1186}{45}$}
	{$I=\dfrac{1186}{9}$}
	{$I=\dfrac{1186}{5}$}
	{$I=\dfrac{1186}{41}$}
	\loigiai{
		Do $f(x)$ đồng biến trên đoạn $\left[1;4\right]$ $\Rightarrow f'(x)\ge 0,\forall x\in\left[1;4\right].$\\
		Ta có  $x+2x \cdot f(x)=\left[f'(x)\right]^2
		\Leftrightarrow x\left(1+2\cdot f(x)\right)=\left[f'(x)\right]^2$, \\Do $x\in\left[1;4\right]$ và $f'(x)\ge 0,\forall x\in\left[1;4\right]$
		$\Rightarrow f(x) >\dfrac{-1}{2}$ và
		$$
		\begin{aligned}
			&f'(x)=\sqrt x \cdot \sqrt{1+2f(x)}\\
			\Leftrightarrow&\dfrac{f'(x)}{\sqrt{1+2f(x)}}=\sqrt x\\
			\Leftrightarrow&\left(\sqrt{1+2f(x)}\right)'=\sqrt x \\
			\Leftrightarrow&\sqrt{1+2f(x)}=\displaystyle\int{\sqrt x}\mathrm{\,d}x\\
			\Leftrightarrow&\sqrt{1+2f(x)}=\dfrac{2}{3}x\sqrt x+C.
		\end{aligned}
		$$
		Vì $f(1)=\dfrac{3}{2}\Rightarrow\sqrt{1+2\cdot\dfrac{3}{2}}=\dfrac{2}{3}+C\Leftrightarrow C=\dfrac{4}{3}$.\\
		Suy ra
		$$
		\begin{aligned}
			&\sqrt{1+2f(x)}=\dfrac{2}{3}x\sqrt x+\dfrac{4}{3}\\
			\Leftrightarrow & 1+2f(x)=\left(\dfrac{2}{3}x\sqrt x+\dfrac{4}{3}\right)^2\\
			\Leftrightarrow & f(x)=\dfrac{2}{9}{x^3}+\dfrac{8}{9}{x^{\dfrac{3}{2}}}+\dfrac{7}{18}.
		\end{aligned}
		$$
		Khi đó\\ $I=\displaystyle\int\limits_1^4f(x)\mathrm{\,d}x=\displaystyle\int\limits_1^4\left(\dfrac{2}{9}{x^3}+\dfrac{8}{9}{x^{\tfrac{3}{2}}}+\dfrac{7}{18}\right)\mathrm{\,d}x=\left.\left(\dfrac{1}{18}{x^4}+\dfrac{16}{45}{x^{\tfrac{5}{2}}}+\dfrac{7}{18}x\right)\right|_1^4=\dfrac{1186}{45}$.}
\end{ex}

\begin{ex}%[2D4C2-2]
	Cho hàm số $f(x)$ nhận giá trị dương và thỏa mãn $f(0)=1$, $\left[f'(x)\right]^3=\mathrm{e}^x\left[f(x)\right]^2,\forall x\in\mathbb{R}$.
	Tính $I=\displaystyle\int\limits_1^2f(x)\mathrm{\,d}x$.
	\choice
	{$I=\mathrm{e}^2+1$}
	{$I=\mathrm{e}-1$}
	{\True $I=\mathrm{e}^2-e$}
	{$I=\mathrm{e}$}
	\loigiai{
		Ta có
		$$
		\begin{aligned}
			&\left[f'(x)\right]^3=\mathrm{e}^x\left[f(x)\right]^2\\
		\Leftrightarrow& f'(x)=\sqrt[3]{\mathrm{e}^x}\cdot\sqrt[3]{\left[f(x)\right]^2}\\ 
		\Leftrightarrow&\dfrac{f'(x)}{\sqrt[3]{\left[f(x)\right]^2}}=\sqrt[3]{\mathrm{e}^x}\\
		 \Leftrightarrow&\dfrac{f'(x)}{\sqrt[3]{\left[f(x)\right]^2}}=\sqrt[3]{\mathrm{e}^x}\\
		  \Leftrightarrow &f'(x)\cdot \left[f(x)\right]^{-\tfrac 23}=\sqrt[3]{\mathrm{e}^x}\\ 
		  \Leftrightarrow& 3\left[\left(f(x)\right)^{\tfrac 13}\right]'=\sqrt[3]{\mathrm{e}^x}\\ 
		  \Leftrightarrow&\left[\left(f(x)\right)^{\tfrac 13}\right]'=\dfrac 13\sqrt[3]{\mathrm{e}^x}\\ 
		  \Leftrightarrow&\left[f(x)\right]^{\tfrac 13}=\dfrac 13\displaystyle\int{\sqrt[3]{\mathrm{e}^x}}\mathrm{\,d}x \\
		  \Leftrightarrow&\left[f(x)\right]^{\tfrac 13}=\mathrm{e}^{\tfrac x3}+C.
		\end{aligned}
		$$	
		Mà $f(0)=1\Rightarrow 1=1+C\Rightarrow C=0$.\\
		Do đó $\left[f(x)\right]^{\tfrac{1}{3}}=\mathrm{e}^{\tfrac{x}{3}}\Rightarrow f(x)=\mathrm{e}^x$.\\
		Vậy $I=\displaystyle\int\limits_1^2\mathrm{e}^x\mathrm{\,d}x=\mathrm{e}^2-\mathrm{e}$.}
\end{ex}

\begin{ex}%[2D4C2-2]
	Cho hàm số $y=f(x)$ có đạo hàm liên tục trên $\mathbb{R}$ và thỏa mãn điều kiện ${{x}^6}{{\left[f'(x) \right]}^3}+27{{\left[f(x)-1 \right]}^4}=0\,,\,\forall x\in \mathbb{R}$ và $f(1)=0$. Tính $I=\displaystyle\int\limits_2^3f(x)\mathrm{\,d}x$.
	\choice
	{$I=\dfrac{31}{2}$}
	{$I=-\dfrac{31}{2}$}
	{$I=\dfrac{61}{4}$}
	{\True $I=-\dfrac{61}{4}$}
	\loigiai{
		Ta có
		$$
		\begin{aligned}
		&x^6\left[f'(x)\right]^3+27\left[f(x)-1\right]^4=0\\ 
		\Leftrightarrow&{x^6}{\left[f'(x)\right]^3}=-27\left[f(x)-1\right]^4\\ 
		\Leftrightarrow&\dfrac{\left[f'(x)\right]^3}{\left[f(x)-1\right]^4}=-\dfrac{27}{x^6}\\ 
		\Leftrightarrow&\dfrac{\left[f'(x)\right]^3}{\left[f(x)-1\right]^3\left[f(x)-1\right]}=-\dfrac{27}{x^6}\\ 
		\Leftrightarrow&\dfrac{f'(x)}{\left[f(x)-1\right]\sqrt[3]{f(x)-1}}=-\dfrac 3{x^2}\\
		 \Leftrightarrow&\dfrac{f'(x)}{-3\left[f(x)-1\right]\sqrt[3]{f(x)-1}}=\dfrac 1{x^2}\\ 
		 \Leftrightarrow&{\left[\dfrac 1{\sqrt[3]{f(x)-1}}\right]'}=\dfrac 1{x^2}.\\ 
		\end{aligned}
		$$
		Do đó $\displaystyle\int{\left[\dfrac{1}{\sqrt[3]{f(x)-1}}\right]'}\mathrm{\,d}x=\displaystyle\int{\dfrac{1}{x^2}\mathrm{\,d}x}=-\dfrac{1}{x}+C.$\\
		Suy ra $\dfrac{1}{\sqrt[3]{f(x)-1}}=-\dfrac{1}{x}+C$.\\
		Mà  $f(1)=0\Rightarrow C=0$.\\
		Nên  $f(x)=1-x^3$.\\
		Khi đó $I=\displaystyle\int\limits_2^3f(x)\mathrm{\,d}x=\displaystyle\int\limits_2^3(1-x^3)\mathrm{\,d}x=-\dfrac{61}{4}$.}
\end{ex}

