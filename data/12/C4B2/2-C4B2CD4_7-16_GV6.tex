\begin{ex}%[2D4C2-4]
Cho hàm số $f(x) > 0$ và thỏa mãn $\left[f'(x)\right]^2+f(x)\cdot f''(x)=\mathrm{e}^x$, $\forall x\in \mathbb{R}$ và $f(0)=f'(0)=1$. Tính $I=\displaystyle\int\limits_1^2 f(x) \mathrm{\,d}x$.
	\choice
	{$I=2\sqrt{\mathrm{e}}$}
	{$I=\mathrm{e}-\sqrt{\mathrm{e}}$}
	{\True $I=2\mathrm{e}-2\sqrt{\mathrm{e}}$}
	{$I=2\mathrm{e}+2\sqrt{\mathrm{e}}$}
	\loigiai{
	Ta có
	\allowdisplaybreaks
	\begin{eqnarray*}
		&&\left[f'(x)\right]^2+f(x)\cdot f''(x)=\mathrm{e}^x\\
		&\Leftrightarrow& \left[f(x)\cdot f'(x)\right]'=\mathrm{e}^x\\
		&\Rightarrow& f(x)\cdot f'(x)=\displaystyle\int\limits_{\mathrm{e}}^x \mathrm{e}^x \mathrm{\,d}x\\
		&\Rightarrow& f(x)\cdot f'(x)=\mathrm{e}^x+C.
	\end{eqnarray*}
	Từ $f(0)=f'(0)=1$ ta suy ra $C=0$.\\
	Vậy $f(x)\cdot f'(x)=\mathrm{e}^x$\\
	Tiếp đến có
	\allowdisplaybreaks
	\begin{eqnarray*}
		&&2f(x)\cdot f'(x)=\mathrm{e}^x\\
		&\Leftrightarrow& \left[f^2(x)\right]'=\mathrm{e}^x\\
		&\Rightarrow& f^2(x)=\displaystyle\int\limits_{\mathrm{e}}^x \mathrm{e}^x \mathrm{\,d}x\\
		&\Rightarrow& f^2(x)=\mathrm{e}^x+C
	\end{eqnarray*}
		Từ $f(0)=1$ ta suy ra $C=0$.\\
		Vậy $f^2(x)=\mathrm{e}^x\Rightarrow f(x)=\sqrt{\mathrm{e}^x}$ (do $f(x) > 0$).\\
		Khi đó $I=\displaystyle\int\limits_1^2 f(x) \mathrm{\,d}x = \displaystyle\int\limits_1^2 \sqrt{\mathrm{e}^x}\mathrm{\,d}x = \displaystyle\int\limits_1^2 \mathrm{e}^{\tfrac{x}{2}} \mathrm{\,d}x = \left.2\mathrm{e}^{\tfrac{x}{2}}\right|_1^2 = 2\mathrm{e}-2\sqrt{\mathrm{e}}$.
}
\end{ex}

\begin{ex}%[2D4C2-2]
	Cho hàm số $f(x)$ thỏa mãn $\left[f'(x)\right]^2+f(x)\cdot f''(x)=2x$, và $f(0)=f'(0)=2$. Tính $I=\displaystyle\int\limits_1^2f^2(x)\mathrm{\,d}x$.
	\choice
	{\True $I=\dfrac{15}{2}$}
	{$I=\dfrac{1}{2}$}
	{$I=\dfrac{19}{2}$}
	{$I=15$}
	\loigiai{
		Ta có $\left[f(x)f'(x)\right]'=\left[f'(x)\right]^2+f(x)f''(x)$.\\
		Do đó theo giả thiết ta được $\left[f(x)f'(x)\right]'=2x$.\\
		Suy ra $f(x)f'(x)=x^2+C$.\\
		Hơn nữa $f(0)=f'(0)=2$ suy ra $C=1$.\\
		$\Rightarrow f(x)f'(x)=x^2+1$.\\
		Tương tự vì $\left[f^2(x)\right]'=2f(x)f'(x)$ nên $\left[f^2(x)\right]'=2\left(x^2+1\right)$.\\
		Suy ra $f^2(x)=\displaystyle\int 2\left(x^2+1\right) \mathrm{\,d}x \Rightarrow f^2(x)=\dfrac{2}{3}{x^3}+2x+C$.\\
		Mặt khác $f(0)=2$ nên  suy ra $C=2$.\\
		$\Rightarrow f^2(x)=\dfrac{2}{3}{x^3}+2x+2$.\\
		Vậy $I=\displaystyle\int\limits_1^2 f^2(x)\mathrm{\,d}x=\displaystyle\int\limits_1^2 \left(\dfrac{2}{3}{x^3}+2x+2\right)\mathrm{\,d}x=\dfrac{15}{2}$.
	}
\end{ex}

\begin{ex}%[2D4C2-2]
	Cho hàm số $f(x)$ thỏa mãn: $\left[f'(x)\right]^2+f(x)\cdot f''(x)=15x^4+12x$, $\forall x\in\mathbb{R}$ và $f(0)=f'(0)=1$. Giá trị của $f^2(1)$ bằng
	\choice
	{$\dfrac{5}{2}$}
	{\True $8$}
	{$10$}
	{$4$}
	\loigiai{
	Theo giả thiết
	\allowdisplaybreaks
	\begin{eqnarray*}
		& & \forall x\in\mathbb{R}\colon \left[f'(x)\right]^2+f(x)\cdot f''(x)=15x^4+12x\\
		&\Leftrightarrow& f'(x)\cdot f'(x)+f(x)\cdot f''(x)=15x^4+12x\\
		&\Leftrightarrow& \left[f(x)\cdot f'(x)\right]'=15x^4+12x\\
		&\Leftrightarrow& f(x)\cdot f'(x)=\displaystyle\int \left(15x^4+12x\right)\mathrm{\,d}x=3x^5+6x^2+C.\quad (1)
	\end{eqnarray*}
	Thay $x=0$ vào $(1)$, ta được $f(0)\cdot f'(0)=C \Leftrightarrow C=1$.\\
	Khi đó $(1)$ trở thành $\begin{aligned}[t]& f(x)\cdot f'(x)=3x^5+6x^2+1\\
	&\Rightarrow \displaystyle\int\limits_0^1 f(x)\cdot f'(x) \mathrm{\,d}x = \displaystyle\int\limits_0^1 \left(3x^5+6x^2+1\right) \mathrm{\,d}x\\
	&\Leftrightarrow \left.\left[\dfrac{1}{2} f^2(x)\right] \right|_0^1 = \left.\left(\dfrac{1}{2}{x^6}+2x^3+x\right) \right|_0^1
	\Leftrightarrow \dfrac{1}{2}\left[f^2(1)-f^2(0)\right]=\dfrac{7}{2} \\
	&\Leftrightarrow f^2(1)-1=7\Leftrightarrow f^2(1)=8.\end{aligned}$\\
	Vậy $f^2(1)=8$.
}
\end{ex}

\begin{ex}%[2D4C2-2]
	Cho hàm số $y=f(x)$ thỏa mãn $\left[f'(x)\right]^2+f(x)\cdot f''(x)=x^3-2x,\,\forall x\in \mathbb{R}$ và $f(0)=f'(0)=2$. Tính giá trị của $T=f^2(2)$.
	\choice
	{$\dfrac{160}{15}$}
	{\True $\dfrac{268}{15}$}
	{$\dfrac{4}{15}$}
	{$\dfrac{268}{30}$}
	\loigiai{
		Ta có $\left[f'(x)\right]^2+f(x)\cdot f''(x)=x^3-2x,\,\forall x\in \mathbb{R} \Leftrightarrow \left[f'(x)\cdot f(x)\right]'=x^3-2x,\,\forall x\in \mathbb{R}$.\\
		Lấy nguyên hàm hai vế ta có 
		\allowdisplaybreaks
		\begin{eqnarray*}
			& & \displaystyle\int \left[f'(x)\cdot f(x)\right]' \mathrm{\,d}x= \displaystyle\int \left(x^3-2x\right)\mathrm{\,d}x\\ &\Leftrightarrow& f'(x)\cdot f(x)=\dfrac{x^4}{4}-x^2+C.
		\end{eqnarray*}
		Theo đề ra ta có $f(0)\cdot f(0)=C=4$.\\
		Suy ra $\displaystyle\int\limits_0^2 f'(x)\cdot f(x)\mathrm{\,d}x = \displaystyle\int\limits_0^2 \left(\dfrac{x^4}{4}-x^2+4\right)\mathrm{\,d}x \Leftrightarrow \left.\dfrac{f^2(x)}{2}\right|_0^2=\dfrac{104}{15}$ $\Leftrightarrow f^2(2)=\dfrac{268}{15}$.
}
\end{ex}
\Closesolutionfile{ans}
\indapan{10}{ans/ans-2-B1}
\begin{dang}{}
	\begin{enumerate}
		\item Điều kiện hàm ẩn có dạng: $A(x)f(x)+B(x)f'(x)=h(x)$.\quad$(1)$\\
		\textit{Ý tưởng giải:}
		\begin{itemize}
			\item Ta cần nhân thêm một lượng $u(x)$ vào $(1)$ để tạo thành \break  $u'(x)f(x)+u(x)f'(x)=u(x).h(x)$ và lúc này:
			\allowdisplaybreaks
			\begin{eqnarray*}
				& & u'(x)f(x)+u(x)f'(x)=u(x)\cdot h(x)\\
				&\Leftrightarrow & \left[u(x)f(x)\right]'=u(x)\cdot .h(x)\\
				&\Rightarrow& \displaystyle\int \left[u(x)f(x)\right]'\mathrm{\,d}x= \displaystyle\int u(x)\cdot h(x)\mathrm{\,d}x\\
				&\Rightarrow& u(x)f(x)=\displaystyle\int u(x)\cdot h(x)\mathrm{\,d}x\\
				&\Rightarrow& f(x)=\dfrac{\displaystyle\int u(x)\cdot h(x)\mathrm{\,d}x}{u(x)}
			\end{eqnarray*}
			\item Cách tìm $u(x)$\\
			$u(x)$ được chọn sao cho: $\heva{&u'(x)=A(x)\\&u(x)=B(x)}$\\
			$\Rightarrow \dfrac{u'(x)}{u(x)}=\dfrac{A(x)}{B(x)} \Rightarrow \displaystyle\int \dfrac{u'(x)}{u(x)}\mathrm{\,d}x =\displaystyle\int\dfrac{A(x)}{B(x)}\mathrm{\,d}x$\\ $\Rightarrow \ln \left|u(x)\right|=\displaystyle\int \dfrac{A(x)}{B(x)}\mathrm{\,d}x \Rightarrow u(x)=\mathrm{e}^{\displaystyle\int \dfrac{A(x)}{B(x)}\mathrm{\,d}x}$.\\
		\end{itemize}
		\textbf{Tóm lại phương pháp giải:} $A(x)f(x)+B(x)f'(x)=h(x)$ $(1)$ như sau:
		\begin{itemize}
			\item Tìm $u(x)$: $u(x)=\mathrm{e}^{\displaystyle\int \dfrac{A(x)}{B(x)} \mathrm{\,d}x}$.
			\item Nhân $u(x)$ vào $(1)$ $\Rightarrow f(x)=\dfrac{\displaystyle\int{u(x)\cdot h(x)} \mathrm{\,d}x}{u(x)}$. 
		\end{itemize}
		\item Một số dạng đặc biệt của $(1)$
		\begin{enumerate}
			\item Điều kiện hàm ẩn có dạng: $\heva{&f'(x)+f(x)=h(x)\\&f'(x)-f(x)=h(x).}$\\
			\textbf{Phương pháp giải}
			\begin{itemize}
				\item $f'(x)+f(x)=h(x)$.\\
				Nhân hai vế với $\mathrm e^x$ ta được $$\mathrm e^x\cdot f'(x)+\mathrm e^x\cdot f(x)=\mathrm e^x\cdot h(x)\Leftrightarrow \left[\mathrm e^x\cdot f(x)\right]'=\mathrm e^x\cdot h(x).$$
				Suy ra $\mathrm e^x\cdot f(x)=\displaystyle\int \mathrm e^x\cdot h(x) \mathrm{\,d}x$.\\
				Từ đây ta dễ dàng tính được $f(x)$.
				\item $f'(x)-f(x)=h(x)$.\\
				Nhân hai vế với $\mathrm e^{-x}$ ta được $$\mathrm e^{-x}\cdot f'(x)-\mathrm e^{-x}\cdot f(x)=\mathrm e^{-x}\cdot h(x)\Leftrightarrow \left[e^{-x}\cdot f(x)\right]'=\mathrm e^{-x}\cdot h(x).$$
				Suy ra $\mathrm e^{-x}\cdot f(x)=\displaystyle\int \mathrm e^{-x}\cdot h(x) \mathrm{\,d}x$.\\
				Từ đây ta dễ dàng tính được $f(x)$.
			\end{itemize}
			\item Điều kiện hàm ẩn có dạng: $f'(x)+p(x)\cdot f(x)=h(x)$.\\
			\textbf{Phương pháp giải}\\
			Nhân hai vế với $\mathrm e^{\displaystyle\int p (x)\mathrm{\,d}x}$ ta được
			\allowdisplaybreaks
			\begin{eqnarray*}
				& & f'(x)\cdot \mathrm e^{\displaystyle\int p(x)\mathrm{\,d}x}+p(x)\cdot \mathrm e^{\displaystyle\int p (x)dx}\cdot f(x)=h(x)\cdot{\mathrm e^{\displaystyle\int p (x)dx}}\\
				&\Leftrightarrow& \left[f(x)\cdot{e^{\displaystyle\int p (x)dx}}\right]'=h(x)\cdot \mathrm e^{\displaystyle\int p (x)\mathrm{\,d} x}.
			\end{eqnarray*}		
			Suy ra $f(x)\cdot \mathrm e^{\displaystyle\int p(x)\mathrm{\,d}x}=\displaystyle\int \mathrm e^{\displaystyle\int p (x)\mathrm{\,d}x}h(x) \mathrm{\,d}x$.\\
			Từ đây ta dễ dàng tính được $f(x)$.
		\end{enumerate}
	\end{enumerate}
\end{dang}
\Opensolutionfile{ans}[ans/ans-2-B1-D2]
\TN
\begin{ex}%[2D4C2-4]
	Cho hàm số $f(x)$ thỏa mãn $f(x)+f'(x)=\mathrm{e}^{-x}$, $\forall x\in\mathbb{R}$ và $f(0)=2$. Tính $I=\displaystyle\int\limits_1^2 \dfrac{f(x) \mathrm{e}^x}{x}\mathrm{\,d}x$.
	\choice
	{$I=2\ln 2$}
	{$I=\ln 2$}
	{$I=1+\ln 2$}
	{\True $I=1+2\ln 2$}
	\loigiai{
	Ta có
	\allowdisplaybreaks
	\begin{eqnarray*}
		& & f(x)+f'(x)=\mathrm{e}^{-x}\\
		&\Leftrightarrow& f(x) \mathrm{e}^x+f'(x)\mathrm{e}^x=1\\
		&\Leftrightarrow& \left[f(x) \mathrm{e}^x\right]'=1\\
		&\Rightarrow& f(x)\mathrm{e}^x=\displaystyle\int x \mathrm{\,d}x\\
		&\Leftrightarrow& f(x) \mathrm{e}^x=x+C.
	\end{eqnarray*}
	Vì $f(0)=2$ nên $C=2$.\\
	$\Rightarrow f(x)\mathrm{e}^x=x+2$.\\
	Vậy 
	$I=\displaystyle\int\limits_1^2 \dfrac{f(x) \mathrm{e}^x}{x} \mathrm{\,d}x = \displaystyle\int\limits_1^2 \dfrac{x+2}{x}\mathrm{\,d}x=\displaystyle\int\limits_1^2 \left(1+\dfrac{2}{x}\right)\mathrm{\,d}x= \left(x+2\ln | x|\right)\bigg|_1^2=1+2\ln 2$.
}
\end{ex}

\begin{ex}%[2D4C2-4]
	Cho hàm số $f(x)$ có đạo hàm trên $\mathbb{R}$ thỏa mãn $\left(x+2\right)f(x)+\left(x+1\right)f'(x)=\mathrm{e}^x$ và $f(0)=\dfrac{1}{2}$. Tính $I=\displaystyle\int\limits_1^2 \left(2x+2\right)f(x)\mathrm{\,d}x$.
	\choice
	{$I=\mathrm{e}^2$}
	{$I=1+\mathrm{e}$}
	{$I=1+\mathrm{e}^2$}
	{\True $I=\mathrm{e}^2-\mathrm{e}$}
	\loigiai{
	Ta có
	\allowdisplaybreaks
	\begin{eqnarray*}
		& & \left(x+2\right)f(x)+\left(x+1\right)f'(x)=\mathrm{e}^x\\
		&\Leftrightarrow& \left(x+1\right)f(x)+f(x)+\left(x+1\right)f'(x)=\mathrm{e}^x\\
		&\Leftrightarrow& \left[\left(x+1\right)f(x)\right]+\left[\left(x+1\right)f(x)\right]'=\mathrm{e}^x\\
		&\Leftrightarrow& \mathrm{e}^x\left[\left(x+1\right)f(x)\right]+\mathrm{e}^x\left[\left(x+1\right)f(x)\right]'=\mathrm{e}^{2x}\\
		&\Leftrightarrow& \left[\mathrm{e}^x\left(x+1\right)f(x)\right]'=\mathrm{e}^{2x}\\
		&\Rightarrow& \displaystyle\int \left[\mathrm{e}^x\left(x+1\right)f(x)\right]'\mathrm{\,d}x=\displaystyle\int \mathrm{e}^{2x}\mathrm{\,d}x\\
		&\Leftrightarrow& \mathrm{e}^x\left(x+1\right)f(x)=\dfrac{1}{2}{\mathrm{e}^{2x}}+C.
	\end{eqnarray*}
		Mà $f(0)=\dfrac{1}{2}$ $\Rightarrow C=0$.\\
		Vậy $f(x)=\dfrac{1}{2}\cdot \dfrac{\mathrm{e}^x}{x+1}$.\\
		Do đó 
		$I=\displaystyle\int\limits_1^2 \left(2x+2\right)\dfrac{1}{2}\cdot \dfrac{\mathrm{e}^x}{x+1}\mathrm{\,d}x=\displaystyle\int\limits_1^2 \mathrm{e}^x\mathrm{\,d}x=\mathrm e^2-\mathrm e$.
}
\end{ex}

\begin{ex}%[2D4C2-3]
	Cho hàm số $y=f(x)$ liên tục, có đạo hàm trên $\mathbb{R}$ thỏa mãn điều kiện \break $f(x)+x\left[f'(x)-2\sin x\right]=x^2\cos x$, $x\in \mathbb{R}$ và $f\left(\dfrac{\pi}{2}\right)=\dfrac{\pi}{2}$. Tính $I=\displaystyle\int\limits_0^{\tfrac{\pi}{2}} \dfrac{f(x)}{x}\mathrm{\,d}x$.
	\choice
	{\True $I=1$}
	{$I=\dfrac{\pi}{2}$}
	{$I=-1$}
	{$I=-\pi$}
	\loigiai{
	Từ giả thiết $\begin{aligned}[t] &f(x)+x\left(f'(x)-2\sin x\right)=x^2\cos x\\
		&\Leftrightarrow f(x)+xf'(x)=x^2\cos x+2x\sin x\\
		&\Leftrightarrow \left(xf(x)\right)'=\left(x^2\sin x\right)'\\
		&\Leftrightarrow xf(x)=x^2\sin x+C.\end{aligned}$\\
	Mặt khác $f\left(\dfrac{\pi}{2}\right)=\dfrac{\pi}{2}\Rightarrow C=0\Rightarrow f(x)=x\sin x$.\\
	Vậy
	$I=\displaystyle\int\limits_0^{\tfrac{\pi}{2}}{\dfrac{f(x)}{x}\mathrm{\,d}x}=\displaystyle\int\limits_0^{\tfrac{\pi}{2}}{\dfrac{x\sin x}{x}\mathrm{\,d}x}=\displaystyle\int\limits_0^{\tfrac{\pi}{2}}{\sin x\mathrm{\,d}x}=1$.
}
\end{ex}

\begin{ex}%[2D4C2-4]
	Cho hàm số $y=f(x)$ có đạo hàm trên $(0;+\infty)$ thỏa mãn $2xf'(x)+f(x)=2x$, $\forall x\in(0;+\infty)$, $f(1)=1$. Giá trị của biểu thức $f(4)$ là
	\choice
	{$\dfrac{25}{6}$}
	{$\dfrac{25}{3}$}
	{\True $\dfrac{17}{6}$}
	{$\dfrac{17}{3}$}
	\loigiai{
		Xét phương trình $2xf'(x)+f(x)=2x$ $(1)$ trên $(0;+\infty)$ ta có $$(1)\Leftrightarrow f'(x)+\dfrac{1}{2x}\cdot f(x)=1.\quad(2)$$
		Đặt $g(x)=\dfrac{1}{2x}$, ta tìm một nguyên hàm $G(x)$ của $g(x)$.\\
		Ta có $\displaystyle\int g(x)\mathrm{\,d}x=\displaystyle\int \dfrac{1}{2x}\mathrm{\,d}x=\dfrac{1}{2}\ln x+C=\ln \sqrt x+C$. Ta chọn $G(x)=\ln \sqrt x $.\\
		Nhân cả 2 vế của $(2)$ cho $\mathrm{e}^{G(x)}=\sqrt x$, ta được 
		$$\sqrt x\cdot f'(x)+\dfrac{1}{2\sqrt x}\cdot f(x)=\sqrt x \Leftrightarrow \left[\sqrt x \cdot f(x)\right]'=\sqrt x. \quad(3)$$
		Lấy tích phân 2 vế của $(3)$ từ $1$ đến $4$, ta được \\
		$\displaystyle\int\limits_1^4 \left[\sqrt x \cdot f(x)\right]'\mathrm{\,d}x= \displaystyle\int\limits_1^4 \sqrt x\mathrm{\,d}x$ $\Rightarrow \left[\sqrt x \cdot f(x)\right]\bigg|_1^4=\left.\left(\dfrac{2}{3}\sqrt{x^3}\right)\right|_1^4\Rightarrow 2f(4)-f(1)=\dfrac{14}{3}$\\
		$\Rightarrow f(4)=\dfrac{1}{2}\left(\dfrac{14}{3}+1\right)=\dfrac{17}{6}$ (vì $f(1)=1$).\\
		Vậy $f(4)=\dfrac{17}{6}$.
}
\end{ex}

\begin{ex}%[2D4C2-2]
	Cho hàm số $f(x)$ không âm, có đạo hàm trên đoạn $[0;1]$ và thỏa mãn $f(1)=1$, $\left[2f(x)+1-x^2\right]f'(x)=2x\left[1+f(x)\right]$, $\forall x\in[0;1]$. Tích phân $\displaystyle\int\limits_0^1 f(x)\mathrm{\,d}x$ bằng
	\choice
	{$1$}
	{$2$}
	{\True $\dfrac{1}{3}$}
	{$\dfrac{3}{2}$}
	\loigiai{
		Xét trên đoạn $[0;1]$, theo đề bài ta có
		\allowdisplaybreaks
		\begin{eqnarray*}
			&&\left[2f(x)+1-x^2\right]f'(x)=2x\left[1+f(x)\right]\\
			&\Leftrightarrow& 2f(x)\cdot f'(x)=2x+\left(x^2-1\right)\cdot f'(x)+2x\cdot f(x)\\
			&\Leftrightarrow& \left[f^2(x)\right]'=\left[x^2+\left(x^2-1\right)\cdot f(x)\right]'\\
			&\Leftrightarrow& f^2(x)=x^2+\left(x^2-1\right)\cdot f(x)+C.\quad (1)
		\end{eqnarray*}
		Thay $x=1$ vào $(1)$ ta được $f^2(1)=1+C\Leftrightarrow C=0$ (vì $f(1)=1$).\\
		Do đó, $(1)$ trở thành
		\allowdisplaybreaks
		\begin{eqnarray*} &&f^2(x)=x^2+\left(x^2-1\right)\cdot f(x)\\
		&\Leftrightarrow& f^2(x)-1=x^2-1+\left(x^2-1\right)\cdot f(x)\\
		&\Leftrightarrow& \left[f(x)-1\right]\cdot \left[f(x)+1\right]=\left(x^2-1\right)\cdot \left[f(x)+1\right]\\
		&\Leftrightarrow& f(x)-1=x^2-1 ~(\text{vì } f(x)\ge 0\Rightarrow f(x)+1 > 0,\,\forall x\in[0;1])\\
		&\Leftrightarrow& f(x)=x^2.
	\end{eqnarray*}
		Vậy $\displaystyle\int\limits_0^1 f(x)\mathrm{\,d}x=\displaystyle\int\limits_0^1 x^2\mathrm{\,d}x=\left.\dfrac{x^3}{3}\right|_0^1=\dfrac{1}{3}$.
}
\end{ex}

\begin{ex}%[2D4C2-2]
	Cho hàm số $y=f(x)$ có đạo hàm liên tục trên $[0;1]$, thỏa mãn \break  $\left[f'(x)\right]^2+4f(x)=8x^2+4,\,\forall x\in[0;1]$ và $f(1)=2$. Tính $\displaystyle\int\limits_0^1 f(x) \mathrm{\,d}x$.
	\choice
	{$\dfrac{1}{3}$}
	{$2$}
	{\True $\dfrac{4}{3}$}
	{$\dfrac{21}{4}$}
	\loigiai{
	Ta có
	\allowdisplaybreaks
	\begin{eqnarray*}
	& & \left[f'(x)\right]^2+4f(x)=8x^2+4\\
	&\Rightarrow& \displaystyle\int\limits_0^1\left[f'(x)\right]^2\mathrm{\,d}x+4\displaystyle\int\limits_0^1 f(x)\mathrm{\,d}x =\displaystyle\int\limits_0^1\left(8x^2+4\right)\mathrm{\,d}x=\dfrac{20}{3}.\quad (1)	
	\end{eqnarray*}
	Và 
	\allowdisplaybreaks
	\begin{eqnarray*}
		&&\displaystyle\int\limits_0^1 xf'(x)\mathrm{\,d}x=xf(x)\big|_0^1-\displaystyle\int\limits_0^1 f(x)\mathrm{\,d}x=2-\displaystyle\int\limits_0^1 f(x)\mathrm{\,d}x\\
		&\Rightarrow&-4\displaystyle\int\limits_0^1 xf'(x)\mathrm{\,d}x=-8+4\displaystyle\int\limits_0^1 f(x)\mathrm{\,d}x.\quad (2)
	\end{eqnarray*}
	Lại có $$\displaystyle\int\limits_0^1\left(2x\right)^2\rm{d}x=\dfrac{4}{3}.\quad (3)$$
	Cộng vế với vế của (1), (2), (3) ta được $$\displaystyle\int\limits_0^1\left(f'(x)-2x\right)^2\mathrm{\,d}x=0\Rightarrow f'(x)=2x\Rightarrow f(x)=x^2+C.$$
	Mặt khác $f(1)=C+1=2\Rightarrow C=1\Rightarrow f(x)=x^2+1$.\\
	Do đó $\displaystyle\int\limits_0^1 f(x)\mathrm{\,d}x=\displaystyle\int\limits_0^1 \left(x^2+1\right)\mathrm{\,d}x=\dfrac{4}{3}$.
}
\end{ex}

\begin{ex}%[2D4C2-2]
	Cho hàm số $y=f(x)$ có đạo hàm liên tục trên $[0;1]$ thỏa mãn \break $3f(x)+xf'(x)\ge x^{2018}$, $\forall x\in[0;1]$. Tìm giá trị nhỏ nhất của $\displaystyle\int_0^1 f(x)\mathrm{\,d}x$.
	\choice
	{$\dfrac{1}{2018\cdot2020}$}
	{$\dfrac{1}{2019\cdot2020}$}
	{$\dfrac{1}{2020\cdot2021}$}
	{\True $\dfrac{1}{2019\cdot2021}$}
	\loigiai{
	Ta có
	\allowdisplaybreaks
	\begin{eqnarray*}
		& & 3f(x)+xf'(x)\ge{x^{2018}},\, \forall x\in[0;1]\\
		&\Leftrightarrow& 3x^2f(x)+x^3\cdot f'(x)\ge x^{2020},\, \forall x\in[0;1]\\
		&\Leftrightarrow& \left[x^3f(x)\right]'\ge x^{2020}, \, \forall x\in[0;1]\\
		&\Rightarrow& x^3f(x)\ge\displaystyle\int x^{2020}\mathrm{\,d}x,\, \forall x\in[0;1]\\
		&\Rightarrow& x^3f(x)\ge\dfrac{x^{2021}}{2021}+C,\, \forall x\in[0;1].
	\end{eqnarray*}
	Cho $x=0\Rightarrow C=0\Rightarrow x^3f(x)\ge\dfrac{x^{2021}}{2021}$, $\forall x\in[0;1]$ $\Rightarrow f(x)\ge\dfrac{x^{2018}}{2021},\,\forall x\in[0;1]$.\\
	$\Rightarrow\displaystyle\int_0^1 f(x) \mathrm{\,d}x\ge\displaystyle\int_0^1 \dfrac{x^{2018}}{2021}\mathrm{\,d}x=\left.\left(\dfrac{x^{2019}}{2019\cdot2021}\right)\right|_0^1=\dfrac{1}{2019\cdot2021}$.
}
\end{ex}
\Closesolutionfile{ans}
\indapan{10}{ans/ans-2-B1-D2}
\begin{dang}{MỘT SỐ DẠNG KHÁC}
\end{dang}
\Opensolutionfile{ans}[ans/ans-2-B1-D3]
\TNSA
\begin{ex}%[2D4C2-2]
	Cho hàm số $y=f(x)$ có đạo hàm trên $\mathbb{R}$ thỏa mãn \break $\heva{&f(0)=f'(0)=1\\&		f(x+y)=f(x)+f(y)+3xy(x+y)-1}$, với $x$, $y\in\mathbb{R}$. Tính $\displaystyle\int\limits_0^1 f(x-1)\mathrm{\,d}x$.
	\choice
	{$\dfrac{1}{2}$}
	{$-\dfrac{1}{4}$}
	{\True $\dfrac{1}{4}$}
	{$\dfrac{7}{4}$}
	\loigiai{
		Lấy đạo hàm theo hàm số $y$ ta được $f'(x+y)=f'(y)+3x^2+6xy$, $\forall x\in\mathbb{R}$.\\
		Cho $y=0\Rightarrow f'(x)=f'(0)+3x^2\Rightarrow f'(x)=1+3x^2$\\
		$\Rightarrow f(x)=\displaystyle\int f'(x)\mathrm{\,d}x=x^3+x+C$ mà $f(0)=1 \Rightarrow C=1$.\\
		Do đó $f(x)=x^3+x+1\Rightarrow f(x-1)=(x-1)^3+x-1+1=x^3-3x^2+4x-1$.\\
		Vậy $\displaystyle\int\limits_0^1 f(x-1)\mathrm{\,d}x=\displaystyle\int\limits_0^1 \left(x^3-3x^2+4x-1\right)\mathrm{\,d}x=\dfrac{1}{4} \displaystyle\int\limits_{-1}^0 f(x)\mathrm{\,d}x= \displaystyle\int\limits_{-1}^0 \left(x^3+x+1\right)\mathrm{\,d}x=\dfrac{1}{4}$.
}
\end{ex}

\begin{ex}%[2D4C2-2]
	Cho hai hàm $f(x)$ và $g(x)$ có đạo hàm trên $[1;4]$, thỏa mãn $\heva{&f(1)+g(1)=4\\&g(x)=-xf'(x)\\&f(x)=-xg'(x)}$, 
	với mọi $x\in[1;4]$. Tính tích phân $I=\displaystyle\int\limits_1^4\left[f(x)+g(x)\right]\mathrm{\,d}x$.
	\choice
	{$3\ln 2$}
	{$4\ln 2$}
	{$6\ln 2$}
	{\True $8\ln 2$}
	\loigiai{
	Từ giả thiết ta có 
	\allowdisplaybreaks
	\begin{eqnarray*}
		&& f(x)+g(x)=-x\cdot f'(x)-x\cdot g'(x)\\
		&\Leftrightarrow& \left[f(x)+x\cdot f'(x)\right]+\left[g(x)+x\cdot g'(x)\right]=0\\ &\Leftrightarrow& \left[x\cdot f(x)\right]'+\left[x\cdot g(x)\right]'=0\\
		&\Rightarrow& x\cdot f(x)+x\cdot g(x)=C\\
		&\Rightarrow& f(x)+g(x)=\dfrac{C}{x}
	\end{eqnarray*}
	Mà $f(1)+g(1)=4\Rightarrow C=4\Rightarrow f(x)+g(x)=\dfrac{4}{x}$.\\
	Vậy $I=\displaystyle\int\limits_1^4 \left[f(x)+g(x)\right]\mathrm{\,d}x=\displaystyle\int\limits_1^4\dfrac{4}{x}\mathrm{\,d}x=8\ln 2$.
}
\end{ex}
\Closesolutionfile{ans}
\indapan{10}{ans/ans-2-B1-D3}
