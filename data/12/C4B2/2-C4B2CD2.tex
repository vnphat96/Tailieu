\begin{dang}{Tích phân có điều kiện}
\end{dang}
\TN
\Opensolutionfile{ans}[ans/ans-2C4B2CD2-LC]
\begin{ex}%[2D4H1-1]
	Nếu $F'(x) = \dfrac{1}{2x}$ và $F(1) = 1$ thì giá trị của $F(4)$ bằng
	\choice
	{$\ln 2$}
	{\True $1 + \ln 2$}
	{$1 + \dfrac{1}{2} \ln 2 $}
	{$ \dfrac{1}{2} \ln 2 $}
\loigiai{	
	Ta có
	$$
	\displaystyle\int\limits_{1}^{4} F'(x) \, \mathrm{d}x = 	\displaystyle\int\limits_{1}^{4} \dfrac{1}{2x} \, \mathrm{d}x = \dfrac{1}{2} \ln \left| x \right| \bigg|_{1}^{4} = \ln 2 .
	$$	
	Lại có 
	$$
	\displaystyle\int\limits_{1}^{4} F'(x) \, dx = F(x) \bigg|_{1}^{4} = F(4) - F(1) .
	$$	
	Suy ra 
	$	F(4) - F(1) = \ln 2	$	.
	Do đó 
	$	F(4) = F(1) + \ln 2 = 1 + \ln 2 $.
}
\end{ex}
\begin{ex}%[2D4H1-1]
	Cho $F(x)$ là một nguyên hàm của $f(x) = \dfrac{2}{x}$. Biết $F(-1) = 0$. Tính $F(2)$ kết quả là
	\choice
	{$2 \ln 2 + 1$}
	{$\ln 2$}
	{$ 2 \ln 3 + 2$}
	{\True $2 \ln 2 $}
\loigiai{
	Ta có 
\allowdisplaybreaks
\begin{eqnarray*}
&&	\displaystyle\int\limits_{-1}^{2} f(x) \, \mathrm{d}x = F(x) \bigg|_{-1}^{2} = F(2) - F(-1)\\
&&	\displaystyle\int\limits_{-1}^{2} \dfrac{2}{x} \, dx = 2 \ln \left| x \right| \bigg|_{-1}^{2} = 2 \ln 2 - 2 \ln 1 = 2 \ln 2\\
&	\Rightarrow& F(2) - F(-1) = 2 \ln 2\\
&\Leftrightarrow& F(2) = 2 \ln 2 \, \text{(do } F(-1) = 0).	
\end{eqnarray*}
}
\end{ex}
\begin{ex}%[2D4H1-1]
	Cho hàm số $f(x)$ liên tục, có đạo hàm trên $[-1;2]$, $f(-1) = 8$, $f(2) = -1$. Tích phân $\displaystyle\int\limits_{-1}^{2} f'(x) \, \mathrm{d}x$ bằng
	\choice
	{$1$}
	{$7$}
	{\True $-9$}
	{$9$}
\loigiai{
	Ta có 
	$$
	\displaystyle\int\limits_{-1}^{2} f'(x) \, \mathrm{d}x = f(x) \bigg|_{-1}^{2} = f(2) - f(-1) = -1 - 8 = -9
	.$$
}
\end{ex}
\begin{ex}%[2D4H1-1]
	Biết $F(x) = x^2$ là một nguyên hàm của hàm số $f(x)$ trên $\mathbb{R}$. Giá trị của $\displaystyle\int\limits_{1}^{3} \left[ 1 + f(x) \right] \mathrm{d}x$ bằng
	\choice
	{\True $10$}
	{$8$}
	{$ \dfrac{26}{3}$}
	{$ \dfrac{32}{3}$}


\loigiai{
	Ta có 
	$$\displaystyle\int\limits_{1}^{3} \left[ 1 + f(x) \right] \mathrm{d}x = (x + F(x)) \bigg|_{1}^{3} = (x + x^2) \bigg|_{1}^{3} = 12 - 2 = 10.$$
}
\end{ex}
\begin{ex}%[2D4H2-2]
	Biết $F(x) = x^3$ là một nguyên hàm của hàm số $f(x)$ trên $\mathbb{R}$. Giá trị của $\displaystyle\int\limits_{1}^{3} \left[ 1 + f(x) \right] \mathrm{d}x$ bằng
	\choice
	{$20$}
	{$22$}
	{$26$}
	{\True $28$}
\loigiai{	
	Ta có 
	$$
	\displaystyle\int\limits_{1}^{3} \left[ 1 + f(x) \right] \mathrm{d}x = \left[ x + F(x) \right] \bigg|_{1}^{3} = \left[ x + x^3 \right] \bigg|_{1}^{3} = 30 - 2 = 28
	.$$
}

\end{ex}
\begin{ex}%[2D4H2-2]
	Biết $F(x) = x^2$ là một nguyên hàm của hàm số $f(x)$ trên $\mathbb{R}$. Giá trị của $\displaystyle\int\limits_{1}^{2} \left[ 2 + f(x) \right] \mathrm{d}x$ bằng
	\choice
	{\True$5$}
	{$3$}
	{$\dfrac{13}{3}$}
	{$\dfrac{7}{3}$}
\loigiai{
	Ta có 
	$$	\displaystyle\int\limits_{1}^{2} \left[ 2 + f(x) \right] \mathrm{d}x = (2x + x^2) \bigg|_{1}^{2} = 8 - 3 = 5	.$$
}
\end{ex}
\begin{ex}%[2D4H2-2]
	Biết $F(x) = x^3$ là một nguyên hàm của hàm số $f(x)$ trên $\mathbb{R}$. Giá trị của $\displaystyle\int\limits_{1}^{2} \left[ 2 + f(x) \right] \mathrm{d}x$ bằng
	\choice
	{$\dfrac{23}{4}$}
	{$7$}
	{\True$9$}
	{$\dfrac{15}{4}$}
\loigiai{
	Ta có 
	$$	\displaystyle\int\limits_{1}^{2} \left[ 2 + f(x) \right] \mathrm{d}x = \displaystyle\int\limits_{1}^{2} 2 \, \mathrm{d}x + \displaystyle\int\limits_{1}^{2} f(x) \, \mathrm{d}x = 2x \bigg|_{1}^{2} + F(x) \bigg|_{1}^{2} = 2x \bigg|_{1}^{2} + x^3 \bigg|_{1}^{2} = 9	.$$
}
\end{ex}
\begin{ex}%[2D4H2-3]
	Cho hàm số $f(x)$. Biết $f(0) = 4$ và $f'(x) = 2 \sin^2 \dfrac{x}{2} + 1$, $\forall x \in \mathbb{R}$, khi đó $\displaystyle\int_{0}^{\frac{\pi}{4}} f(x) \mathrm{d}x$ bằng
	\choice
	{\True $\dfrac{\pi^2 + 16\pi + 8\sqrt{2} - 16}{16}$}
	{$\dfrac{\pi^2 + 16\pi + 2\sqrt{2} - 4}{16}$}
	{$\dfrac{\pi^2 + 16\pi + 8\sqrt{2}}{16}$}
	{$\dfrac{\pi^2 + 16\pi - 16}{16}$}
\loigiai{	
	Ta có 	$$
	f(x) = \displaystyle\int \left( 2 \sin^2 \dfrac{x}{2} + 1 \right) \mathrm{d}x = \displaystyle\int (2 - \cos x) \mathrm{d}x = 2x - \sin x + C	.$$	
	Vì $f(0) = 4 \Rightarrow C = 4 \Rightarrow f(x) = 2x - \sin x + 4$.\\	
	Suy ra 
	\allowdisplaybreaks
	\begin{eqnarray*}
&&		\displaystyle\int\limits_{0}^{\frac{\pi}{4}} f(x) \mathrm{d}x = \displaystyle\int\limits_{0}^{\frac{\pi}{4}} (2x - \sin x + 4) \mathrm{d}x\\
&	= &\left( x^2 + \cos x + 4x \right) \bigg|_{0}^{\frac{\pi}{4}} = \dfrac{\pi^2}{16} + \dfrac{\sqrt{2}}{2} + \pi - 1 = \dfrac{\pi^2 + 16\pi + 8\sqrt{2} - 16}{16}.		
	\end{eqnarray*}
}
\end{ex}
\begin{ex}%[2D4H2-3]
	Cho hàm số $f(x)$. Biết $f(0) = 4$ và $f'(x) = 2 \cos^2 \dfrac{x}{2} + 3$, $\forall x \in \mathbb{R}$, khi đó $\displaystyle\int\limits_{0}^{\frac{\pi}{4}} f(x) \mathrm{d}x$ bằng?
	\choice
	{$\dfrac{\pi^2 + 8\pi - 8 - \sqrt{2}}{8}$}
	{\True$\dfrac{\pi^2 + 8\pi - 8 - 4\sqrt{2}}{8}$}
	{$\dfrac{\pi^2 + 6\pi + 8}{8}$}
	{$\dfrac{\pi^2 + 8\pi - 4\sqrt{2}}{8}$}

\loigiai{
	Ta có 
\allowdisplaybreaks
\begin{eqnarray*}
		f(x) &= &\displaystyle\int f'(x) \mathrm{d}x = \displaystyle\int (2 \cos^2 \dfrac{x}{2} + 3) \mathrm{d}x\\
	&=&\displaystyle\int \left( 2 \cdot \dfrac{1 + \cos x}{2} + 3 \right) \mathrm{d}x = \displaystyle\int (\cos x + 4) \mathrm{d}x\\
	&	\Rightarrow& f(x) = \sin x + 4x + C	.
\end{eqnarray*}	
Do $f(0) = 4 \Rightarrow C = 4\Rightarrow  	f(x) = \sin x + 4x + 4$. Vậy\\
$$ \displaystyle\int\limits_{0}^{\frac{\pi}{4}} f(x) \mathrm{d}x = \displaystyle\int\limits_{0}^{\frac{\pi}{4}} (\sin x + 4x + 4) \mathrm{d}x	= \left( -\cos x + 2x^2 + 4x \right) \bigg|_{0}^{\frac{\pi}{4}} = \dfrac{\pi^2 + 8\pi - 8 - 4\sqrt{2}}{8}.$$
}
\end{ex}
\begin{ex}%[2D4H2-4]
	Cho hàm số $f(x) = \heva{
		&e^{2x} \text{ khi } x \geq 0 \\
		&x^2 + x + 2 \text{ khi } x < 0 
	}$. Biết tích phân $\displaystyle\int\limits_{-1}^{1} f(x) \mathrm{d}x = \dfrac{a}{b} + \dfrac{e^2}{c}$ ($\dfrac{a}{b}$ là phân số tối giản). Giá trị $a + b + c$ bằng
	\choice
	{$7$}
	{$8$}
	{\True$9$}
	{$10$}
\loigiai{
	Ta có
	$$
	I = \displaystyle\int\limits_{-1}^{1} f(x) \mathrm{d}x = \displaystyle\int\limits_{-1}^{0} (x^2 + x + 2) \mathrm{d}x + \displaystyle\int\limits_{0}^{1} e^{2x} \mathrm{d}x = \dfrac{4}{3} + \dfrac{e^2}{2}
	.$$	
	Vậy $a + b + c = 9$.
}
\end{ex}
\begin{ex}%[2D4H2-2]
	Cho hàm số $f(x) = \heva{
		&x^2 - 1 \text{ khi } x \geq 2 \\
		&x^2 - 2x + 3 \text{ khi } x < 2 
	}$. Tích phân $I = \dfrac{1}{2} \displaystyle\int\limits_{1}^{3} f(x) \mathrm{d}x$ bằng:
	\choice
	{$\dfrac{23}{3}$}
	{\True $\dfrac{23}{6}$}
	{$\dfrac{17}{6}$}
	{$\dfrac{17}{3}$}
\loigiai{
Ta có
	$$I = \dfrac{1}{2} \displaystyle\int\limits_{1}^{3} f(x) \mathrm{d}x = \dfrac{1}{2} \left[ \displaystyle\int\limits_{1}^{2} (x^2 - 2x + 3) \mathrm{d}x + \displaystyle\int\limits_{2}^{3} (x^2 - 1) \mathrm{d}x \right] = \dfrac{23}{6}.$$
}
\end{ex}
\begin{ex}%[2D4H2-2]
	Cho hàm số $f(x) = \heva{
		&\dfrac{x(1 + x^2)}{x - 4} \text{ khi } x \geq 3 \\
		&\dfrac{1}{x - 4} \text{ khi } x < 3 
	}$. Tích phân $I = \displaystyle\int\limits_{2}^{4} f(t) \mathrm{d}t$ bằng:
	\choice
	{$\dfrac{40}{3} - \ln 2$}
	{$\dfrac{95}{6} + \ln 2$}
	{$\dfrac{189}{4} + \ln 2$}
	{\True $\dfrac{189}{4} - \ln 2$}
\loigiai{
Ta có
$$	I = \displaystyle\int\limits_{2}^{4} f(t) \mathrm{d}t = \displaystyle\int\limits_{2}^{3} \dfrac{1}{x - 4} \mathrm{d}x + \displaystyle\int\limits_{3}^{4} \dfrac{x(1 + x^2)}{x - 4} \mathrm{d}x = \dfrac{189}{4} - \ln 2
	.$$
}
\end{ex}
\begin{ex}%[2D4H2-2]
	Cho số thực $a$ và hàm số $f(x) = \heva{
		&2x \text{ khi } x \leq 0 \\
		&a(x - x^2) \text{ khi } x > 0 
	}$. Tính tích phân $\displaystyle\int\limits_{-1}^{1} f(x) \mathrm{d}x$ bằng:
	\choice
	{\True $\dfrac{a}{6} - 1$}
	{$\dfrac{2a}{3} + 1$}
	{$\dfrac{a}{6} + 1$}
	{$\dfrac{2a}{3} - 1$}
\loigiai{
	Ta có
\allowdisplaybreaks
\begin{eqnarray*}
&&	\displaystyle\int\limits_{-1}^{1} f(x) \mathrm{d}x = \displaystyle\int\limits_{-1}^{0} f(x) \mathrm{d}x + \displaystyle\int\limits_{0}^{1} f(x) \mathrm{d}x = \displaystyle\int\limits_{-1}^{0} 2x \mathrm{d}x + \displaystyle\int\limits_{0}^{1} a(x - x^2) \mathrm{d}x\\
&	=& (x^2) \bigg|_{-1}^{0} + a \left( \dfrac{x^2}{2} - \dfrac{x^3}{3} \right) \bigg|_{0}^{1} = -1 + a \left( \dfrac{1}{6} \right) = \dfrac{a}{6} - 1.	
\end{eqnarray*}
}
\end{ex}
\Closesolutionfile{ans}
\indapan{6}{ans/ans-2C4B2CD2-LC}
\TNTF
\Opensolutionfile{ans}[ans/ans-2C4B2CD2-DS]
\begin{ex}%[2D4H2-2]
	Cho hàm số $f(x) = \heva{
		&2x^2 + 3 \text{ khi } x \geq 1 \\
		&2 - x^3 \text{ khi } x < 1 
	}$.
	\choiceTF
	{\True $\displaystyle\int\limits_{1}^{2024} f(x) \mathrm{d}x = \displaystyle\int\limits_{1}^{2024} (2x^2 + 3) \mathrm{d}x$}
{\True $\displaystyle\int\limits_{-2024}^{1} f(x) \mathrm{d}x = \displaystyle\int\limits_{-2024}^{1} (2 - x^3) \mathrm{d}x$}
	{$\displaystyle\int\limits_{-2024}^{2024} f(x) \mathrm{d}x = \displaystyle\int\limits_{1}^{2024} (2x^2 + 3) \mathrm{d}x + \displaystyle\int\limits_{-2024}^{1} (2 - x^3) \mathrm{d}x$}
	{\True $\displaystyle\int_{-2024}^{2024} f(x) \mathrm{d}x = \displaystyle\int\limits_{1}^{2024} (2x^2 + 3) \mathrm{d}x + \displaystyle\int\limits_{-2024}^{1} (2 - x^3) \mathrm{d}x$}
\loigiai{
Do $f(x) = \heva{
	&2x^2 + 3 \text{ khi } x \geq 1 \\
	&2 - x^3 \text{ khi } x < 1 
}$ nên\\
\begin{itemize}
	\item $\displaystyle\int\limits_{1}^{2024} f(x) \mathrm{d}x = \displaystyle\int\limits_{1}^{2024} (2x^2 + 3) \mathrm{d}x.
	$
	\item $\displaystyle\int\limits_{-2024}^{1} f(x) \mathrm{d}x = \displaystyle\int\limits_{-2024}^{1} (2 - x^3) \mathrm{d}x.
	$
\item $
\displaystyle\int\limits_{-2024}^{2024} f(x) \mathrm{d}x = \displaystyle\int\limits_{1}^{2024} (2x^2 + 3) \mathrm{d}x + \displaystyle\int\limits_{-2024}^{1} (2 - x^3) \mathrm{d}x.
$
\item $
\displaystyle\int\limits_{-2024}^{2024} f(x) \mathrm{d}x = \displaystyle\int\limits_{1}^{2024} (2x^2 + 3) \mathrm{d}x + \displaystyle\int\limits_{-2024}^{1} (2 - x^3) \mathrm{d}x.
$
\end{itemize}
}

\end{ex}

\begin{ex}%[2D4H2-2]
	Cho hàm số $f(x) = \heva{
		&x^2 - 2x + 3 \text{ khi } x \geq 2 \\
		&x + 1 \text{ khi } x < 2 
	}$.
	\choiceTF
	{\True $\displaystyle\int\limits_{1}^{2} f(x) \mathrm{d}x = \displaystyle\int\limits_{1}^{2} (x + 1) \mathrm{d}x$}
	{\True $\displaystyle\int\limits_{2}^{3} f(x) \mathrm{d}x = \displaystyle\int\limits_{2}^{3} (x^2 - 2x + 3) \mathrm{d}x$}
	{\True $\displaystyle\int\limits_{1}^{3} \dfrac{1}{2} f(x) \mathrm{d}x = \dfrac{41}{12}$}
	{$\displaystyle\int\limits_{1}^{2} f(x) \mathrm{d}x = \displaystyle\int\limits_{1}^{2} (x^2 - 2x + 3) \mathrm{d}x$}
\loigiai{
	Do $f(x) = \heva{
		&x^2 - 2x + 3 \text{ khi } x \geq 2 \\
		&x + 1 \text{ khi } x < 2 
	}$ nên\\
\begin{itemize}
	\item $
	\displaystyle\int\limits_{1}^{2} f(x) \mathrm{d}x = \displaystyle\int\limits_{1}^{2} (x + 1) \mathrm{d}x	$.
\item 	$
\displaystyle\int\limits_{2}^{3} f(x) \mathrm{d}x = \displaystyle\int\limits_{2}^{3} (x^2 - 2x + 3) \mathrm{d}x$.
\item 	$
 \displaystyle\int\limits_{1}^{3} \dfrac{1}{2} f(x) \mathrm{d}x = \dfrac{1}{2} \left( \displaystyle\int\limits_{1}^{2} (x + 1) \mathrm{d}x + \displaystyle\int\limits_{2}^{3} (x^2 - 2x + 3) \mathrm{d}x \right) = \dfrac{41}{12}
$.
\end{itemize}
}
\end{ex}
\Closesolutionfile{ans}
\indapan{2}{ans/ans-2C4B2CD2-DS}
\TNSA
\Opensolutionfile{ans}[ans/ans-2C4B2CD2-KQ]
\begin{ex}%[2D4H1-2]
	Cho hàm số $f(x) = \heva{
		&\dfrac{1}{x} \text{ khi } x \geq 1 \\
		&x + 1 \text{ khi } x < 1 
	}$. Tích phân $I = \displaystyle\int\limits_{2}^{0} -3t^2 f(t) \mathrm{d}t$. (\textit{\textit{làm tròn đến hàng phần trăm}})
\shortans{$2{,}08$}
\loigiai{
	Ta có\\
	$
	I = -3 \displaystyle\int\limits_{2}^{0} t^2 f(t) \mathrm{d}t = 3 \displaystyle\int\limits_{0}^{2} t^2 f(t) \mathrm{d}t = 3 \left[ \displaystyle\int\limits_{0}^{1} x^2 (x + 1) \mathrm{d}x + \displaystyle\int\limits_{1}^{2} x^2 \cdot \dfrac{1}{x} \mathrm{d}x \right] = \dfrac{25}{12}\approx 2{,}08
	$.
}
\end{ex}
\begin{ex}%[2D4H1-2]
	Cho hàm số $f(x) = \heva{
		&2x^2 - 1 \text{ khi } x < 0 \\
		&x - 1 \text{ khi } 0 \leq x \leq 2 \\
		&5 - 2x \text{ khi } x > 2 
	}$. Tính tích phân $I = \displaystyle\int\limits_{-5}^{9} \dfrac{1}{7} f(t) \mathrm{d}t$. (\textit{làm tròn đến hàng phần trăm})
\shortans{$5{,}19$}
\loigiai{
	Ta có
\allowdisplaybreaks
\begin{eqnarray*}
		I &=& \dfrac{1}{7} \displaystyle\int\limits_{-5}^{9} f(t) \mathrm{d}t = \dfrac{1}{7} \displaystyle\int\limits_{-5}^{9} f(x) \mathrm{d}x = \dfrac{1}{7} \left( \displaystyle\int\limits_{-5}^{0} f(x) \mathrm{d}x + \displaystyle\int\limits_{0}^{2} f(x) \mathrm{d}x + \displaystyle\int\limits_{2}^{9} f(x) \mathrm{d}x \right)\\
&	=& \dfrac{1}{7} \displaystyle\int\limits_{-5}^{0} (2x^2 - 1) \mathrm{d}x + \dfrac{1}{7} \displaystyle\int\limits_{0}^{2} (x - 1) \mathrm{d}x + \dfrac{1}{7} \displaystyle\int\limits_{2}^{9} (5 - 2x) \mathrm{d}x = \dfrac{109}{21}	\approx 5{,}19.
\end{eqnarray*}
}
\end{ex}
\begin{ex}%[2D4H1-2]
	Cho hàm số $f(x) = \heva{
		&x^2 - x \text{ khi } x \geq 0 \\
		&x \text{ khi } x < 0 
	}$. Khi đó $I = \displaystyle\int\limits_{-1}^{1} f(x) \mathrm{d}x + \displaystyle\int\limits_{-1}^{3} f(x) \mathrm{d}x$ bằng bao nhiêu? (\textit{làm tròn đến hàng phần trăm})
\shortans{$3{,}33$}
\loigiai{
	Đặt 	$	I_1 = \displaystyle\int\limits_{-1}^{1} f(x) \mathrm{d}x$ và 	$	I_2 = \displaystyle\int\limits_{-1}^{3} f(x) \mathrm{d}x
	$. \\	
	Vì  $f(x) = \heva{
		&x^2 - x \text{ khi } x \geq 0 \\
		&x \text{ khi } x < 0 
	}$ nên \\
	$$ I_1 = \displaystyle\int\limits_{-1}^{0} x \mathrm{d}x + \displaystyle\int\limits_{0}^{1} (x^2 - x) \mathrm{d}x = -\dfrac{2}{3}.$$
Và
	$$
 I_2 = \displaystyle\int\limits_{-1}^{0} x \mathrm{d}x + \displaystyle\int_{0}^{3} (x^2 - x) \mathrm{d}x = 4.$$	
	Vậy $I = I_1 + I_2 = \dfrac{10}{3}\approx 3{,}33$.
}
\end{ex}

\begin{ex}%[2D4H1-2]
	Cho hàm số $f(x) = \heva{
		&4x \text{ khi } x > 2 \\
		&-2x + 12 \text{ khi } x \leq 2 
	}$. Tính tích phân $I = \displaystyle\int\limits_{1}^{2} f(t) \mathrm{d}t + \dfrac{1}{2} \displaystyle\int\limits_{5}^{10} f(t) \mathrm{d}t$.
\shortans{$84$}
\loigiai{
Đặt 	$I_1 = \displaystyle\int\limits_{1}^{2} f(t) \mathrm{d}t = \displaystyle\int\limits_{1}^{2} f(x) \mathrm{d}x$ và 	$
I_2 = \dfrac{1}{2} \displaystyle\int\limits_{5}^{10} f(t) \mathrm{d}t = \dfrac{1}{2} \displaystyle\int\limits_{5}^{10} f(x) \mathrm{d}x$.\\
	Vì  $f(x) = \heva{
		&4x \text{ khi } x > 2 \\
		&-2x + 12 \text{ khi } x \leq 2 
	}$ nên\\	
$$I_1 = \displaystyle\int\limits_{1}^{2} (-2x + 12) \mathrm{d}x = 9.$$
Và 
	$$
I_2 = \dfrac{1}{2} \displaystyle\int\limits_{5}^{10} 4x \mathrm{d}x = 75.$$	
	Vậy $I = I_1 + I_2 = 84$.
}
\end{ex}

\begin{ex}%[2D4H1-2]
	Biết rằng hàm số $f(x) = mx + n$ thỏa mãn $\displaystyle\int\limits_{0}^{1} f(x) \mathrm{d}x = 3$, $\displaystyle\int\limits_{0}^{2} f(x) \mathrm{d}x = 8$. Tính $m + n$.
\shortans{$4$}
\loigiai{
	Ta có 
	$
	\displaystyle\int f(x) \mathrm{d}x = \displaystyle\int (mx + n) \mathrm{d}x = \dfrac{m}{2} x^2 + nx + C
	$.\\	
	Lại có
	$
	\displaystyle\int\limits_{0}^{1} f(x) \mathrm{d}x = 3 \Rightarrow \left( \dfrac{m}{2} x^2 + nx \right) \bigg|_{0}^{1} = 3 \Rightarrow \dfrac{1}{2} m + n = 3 \quad (1)
	$.\\
	$	\displaystyle\int\limits_{0}^{2} f(x) \mathrm{d}x = 8 \Rightarrow \left( \dfrac{m}{2} x^2 + nx \right) \bigg|_{0}^{2} = 8 \Rightarrow 2m + 2n = 8 \quad (2)
	$.\\	
	Từ (1) và (2) ta có hệ phương trình
$$
\heva{&\dfrac{1}{2} m + n = 3 \\
		&2m + 2n = 8 }
	\Rightarrow \heva{&	m = 2 \\&	n = 2.}
	$$	
Vậy $ m + n = 4$.
}
\end{ex}
\begin{ex}%[2D4V1-2]
	Biết rằng hàm số $f(x) = ax^2 + bx + c$ thỏa mãn $\displaystyle\int\limits_{0}^{1} f(x) \mathrm{d}x = -\dfrac{7}{2}$, $\displaystyle\int\limits_{0}^{2} f(x) \mathrm{d}x = -2$ và $\displaystyle\int\limits_{0}^{3} f(x) \mathrm{d}x = \dfrac{13}{2}$. Tính $P = a + b + c$. (\textit{làm tròn đến hàng phần trăm}).
	\shortans{$-1{,}33$}
\loigiai{
	Ta có
	$	\displaystyle\int f(x) \mathrm{d}x = \displaystyle\int (ax^2 + bx + c) \mathrm{d}x = \dfrac{a}{3} x^3 + \dfrac{b}{2} x^2 + cx + C
	$.\\	
	Lại có
	$	\displaystyle\int\limits_{0}^{1} f(x) \mathrm{d}x = -\dfrac{7}{2} \Rightarrow \left( \dfrac{a}{3} x^3 + \dfrac{b}{2} x^2 + cx \right) \bigg|_{0}^{1} = -\dfrac{7}{2} \Rightarrow \dfrac{1}{3} a + \dfrac{1}{2} b + c = -\dfrac{7}{2} \quad (1)
	$.	
	$	\displaystyle\int\limits_{0}^{2} f(x) \mathrm{d}x = -2 \Rightarrow \left( \dfrac{a}{3} x^3 + \dfrac{b}{2} x^2 + cx \right) \bigg|_{0}^{2} = -2 \Rightarrow \dfrac{8}{3} a + 2b + 2c = -2 \quad (2)
	$.\\	
	$	\displaystyle\int\limits_{0}^{3} f(x) \mathrm{d}x = \dfrac{13}{2} \Rightarrow \left( \dfrac{a}{3} x^3 + \dfrac{b}{2} x^2 + cx \right) \bigg|_{0}^{3} = \dfrac{13}{2} \Rightarrow 9a + \dfrac{9}{2} b + 3c = \dfrac{13}{2} \quad (3)
	$.\\	
	Từ (1), (2) và (3) ta có hệ phương trình:
	$$\heva{&\dfrac{1}{3} a + \dfrac{1}{2} b + c = -\dfrac{7}{2} \\&
		\dfrac{8}{3} a + 2b + 2c = -2 \\&
		9a + \dfrac{9}{2} b + 3c = \dfrac{13}{2}}
	\Rightarrow \heva{&	a = 1 \\&
		b = 3 \\&
		c = -\dfrac{16}{3}.}	$$	
Vậy	$P = a + b + c = 1 + 3 + \left( -\dfrac{16}{3} \right) = -\dfrac{4}{3}\approx -1{,}33$.
}
\end{ex}

% \begin{ex}%[2D4H1-2]
% 	Có hai giá trị của số thực $a$ là $a_1$, $a_2$ ($0 < a_1 < a_2$) thỏa mãn $\displaystyle\int\limits_{1}^{a} (2x - 3) \mathrm{d}x = 0$. Hãy tính $T = 3^{a_1} + 3^{a_2} + \log_2 \left( \dfrac{a_2}{a_1} \right)$.
% 	\shortans{$13$}
% \loigiai{
% 	Ta có
% 	$\displaystyle\int\limits_{1}^{a} (2x - 3) \mathrm{d}x = \left( x^2 - 3x \right) \bigg|_{1}^{a} = a^2 - 3a + 2	$.\\	
% 	Vì $\displaystyle\int\limits_{1}^{a} (2x - 3) \mathrm{d}x = 0$ nên $a^2 - 3a + 2 = 0$, suy ra $\hoac{&a = 1 \\ &a = 2.}$\\	
% 	Lại có $0 < a_1 < a_2$ nên $a_1 = 1$, $a_2 = 2$.\\	
% 	Như vậy $T = 3^{a_1} + 3^{a_2} + \log_2 \left( \dfrac{a_2}{a_1} \right) = 3^1 + 3^2 + \log_2 \left( \dfrac{2}{1} \right) = 13$.
% }
% \end{ex}
\begin{ex}%[2D4H1-2]
	Cho $\displaystyle\int\limits_{0}^{m} (3x^2 - 2x + 1) \mathrm{d}x = 6$. Tính giá trị của tham số $m$.
	\shortans{$2$}
\loigiai{
	Ta có\\
	$	\displaystyle\int\limits_{0}^{m} (3x^2 - 2x + 1) \mathrm{d}x = \left( x^3 - x^2 + x \right) \bigg|_{0}^{m} = m^3 - m^2 + m	$.\\	
	$
	\displaystyle\int\limits_{0}^{m} (3x^2 - 2x + 1) \mathrm{d}x = 6 \Leftrightarrow m^3 - m^2 + m - 6 = 0 \Leftrightarrow m = 2
	$.
}
\end{ex}
\begin{ex}%[2D4V1-2]
	Cho $I = \displaystyle\int\limits_{0}^{1} (4x - 2m^2) \mathrm{d}x$. Có bao nhiêu giá trị nguyên của $m$ để $I + 6 > 0$?
\shortans{$3$}
\loigiai{
	Theo định nghĩa tích phân ta có:
	$	I = \displaystyle\int\limits_{0}^{1} (4x - 2m^2) \mathrm{d}x = \left( 2x^2 - 2m^2 x \right) \bigg|_{0}^{1} = -2m^2 + 2
	$.\\
		Khi đó $I + 6 > 0 \Leftrightarrow -2m^2 + 2 + 6 > 0 \Leftrightarrow -2m^2 + 8 > 0  \Leftrightarrow -2 < m < 2$.\\	
	Mà $m$ là số nguyên nên $m \in \{-1; 0; 1\}$.\\	
	Vậy có $3$ giá trị nguyên của $m$ thỏa mãn yêu cầu.
}
\end{ex}
\begin{ex}%[2D4V1-2]
	Có bao nhiêu giá trị nguyên dương của $a$ để $\displaystyle\int\limits_{0}^{a} (2x - 3) \mathrm{d}x \leq 4$?
\shortans{$4$}
\loigiai{
	Ta có
	$
	\displaystyle\int\limits_{0}^{a} (2x - 3) \mathrm{d}x = \left( x^2 - 3x \right) \bigg|_{0}^{a} = a^2 - 3a
	$.\\	
	Khi đó
	$
	\displaystyle\int\limits_{0}^{a} (2x - 3) \mathrm{d}x \leq 4 \Leftrightarrow a^2 - 3a \leq 4 \Leftrightarrow -1 \leq a \leq 4
	$.\\	
	Mà $a \in \mathbb{N}^*$ nên $a \in \{1; 2; 3; 4\}$.\\	
	Vậy có $4$ giá trị của $a$ thỏa đề bài.
}
\end{ex}
\begin{ex}%[2D4V1-2]
	Có bao nhiêu số thực $b$ thuộc khoảng $(\pi; 3\pi)$ sao cho $\displaystyle\int\limits_{\pi}^{b} 4 \cos 2x \mathrm{d}x = 1$?
\shortans{$4$}
\loigiai{
	Ta có
	$
	\displaystyle\int\limits_{\pi}^{b} 4 \cos 2x \mathrm{d}x = 1 \Leftrightarrow 2 \sin 2x \bigg|_{\pi}^{b} = 1 \Leftrightarrow \sin 2b - \sin 2\pi = \dfrac{1}{2} \Leftrightarrow \sin 2b = \dfrac{1}{2}
	$.\\	
	$
	\Rightarrow 2b = \dfrac{\pi}{6} + k2\pi \quad \text{ hoặc } \quad 2b = \dfrac{5\pi}{6} + k2\pi
	$\\	
	$
	\Rightarrow b = \dfrac{\pi}{12} + k\pi \quad \text{ hoặc } \quad b = \dfrac{5\pi}{12} + k\pi\qquad (k\in \mathbb{Z})
	$.\\
Khi $	b = \dfrac{\pi}{12} + k\pi$, ta xét\\
\allowdisplaybreaks
\begin{eqnarray*}
&& \pi< \dfrac{\pi}{12} + k\pi<3\pi\\
&\Leftrightarrow& \dfrac{11}{12}<k<\dfrac{35}{12}\\
&\Leftrightarrow& k \in \{1;2\}.
\end{eqnarray*}
Khi $	b = \dfrac{5\pi}{12} + k\pi$, ta xét\\
\allowdisplaybreaks
\begin{eqnarray*}
&& \pi< \dfrac{5\pi}{12} + k\pi<3\pi\\
	&\Leftrightarrow& \dfrac{7}{12}<k<\dfrac{31}{12}\\
	&\Leftrightarrow& k \in \{1;2\}.
\end{eqnarray*}
	Vậy có $4$ số thực $b$ thỏa mãn yêu cầu bài toán.
}
\end{ex}
\Closesolutionfile{ans}
\indapan{2}{ans/ans-2C4B2CD2-KQ}