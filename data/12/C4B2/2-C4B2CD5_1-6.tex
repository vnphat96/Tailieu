\Opensolutionfile{ans}[ans/ans-2C4B2CD3-LC]
\chude{ỨNG DỤNG TÍCH PHÂN TRONG THỰC TIỄN}
    \begin{itemize}
        \item Cho hàm số$f\left(x \right)$ liên tục trên đoạn $\left[a;b \right]$. Khi đó $\dfrac{1}{b-a}\displaystyle\int\limits_a^b{f\left(x \right)dx}$ được gọi là giá trị trung bình của hàm số $f\left(x \right)$ trên đoạn $\left[a;b \right]$.
        \item Đạo hàm của quãng đường di chuyển của vật theo thời gian bằng tốc độ của chuyển động tại mọi thời điểm $v(t)=s'(t)$. Do đó, nếu biết tốc độ $v(t)$ tại mọi thời điểm $t\in \left[a;b \right]$ thì tính được quãng đường di chuyển trong khoảng thời gian từ $a$ đến $b$ theo công thức
        $$s=s\left(b \right)-s\left(a \right)=\displaystyle\int\limits_a^b v(t)\mathrm{\,d}t.$$
        \item Giả sử là vận tốc của vật tại thời điểm và là quãng đường vật đi được sau khoảng thời gian tính từ lúc bắt đầu chuyển động. Ta có mối liên hệ giữa vận tốc và quãng đường như sau
        \begin{itemize}
            \item Đạo hàm của quãng đường là vận tốc $s'(t)=v(t)$.
            \item Nguyên hàm của vận tốc là quãng đường $s(t)= \displaystyle\int v(t)\mathrm{\,d}t$.
        \end{itemize}
        $\Rightarrow$ Từ đây ta cũng có quãng đường vật đi được trong khoảng thời gian từ $a$ đến $b$ là 
        $$\displaystyle\int\limits_a^b v(t)\mathrm{\,d}t=s(b)-s(a).$$ 
        Nếu gọi $a(t)$ là gia tốc của vật thì ta có mối liên hệ giữa gia tốc và vận tốc như sau
        \begin{itemize}
            \item Đạo hàm của vận tốc là gia tốc $v'(t)=a(t)$.
            \item Nguyên hàm của gia tốc là vận tốc $v(t)= \displaystyle\int a(t)\mathrm{\,d}t$.
        \end{itemize}
    \end{itemize}

\TN
\begin{ex}%[2D4H2-6] 
    Một ô tô đang chạy với vận tốc $10\,m/s$ thì gặp chướng ngại vật, người lái xe đạp phanh. Từ thời điểm đó, ô tô chuyển động chậm dần đều với vận tốc $v\,\left(t \right)=-2t+10\,\left(m/s \right)$, trong đó $t$ là khoảng thời gian tính bằng giây, kể từ lúc bắt đầu đạp phanh. Tính quãng đường ô tô di chuyển được trong $8$ giây cuối cùng.
    \choice
    {\True $55\,m$}
    {$25\,m$}
    {$50\,m$}
    {$16\,m$}
    \loigiai{
    Ta có $-2t+10=0\Leftrightarrow t=5\Rightarrow$ thời gian tính từ lúc bắt đầu đạp phanh đến khi dừng hẳn là $5$ giây.\\ 
    Vậy trong $8$ giây cuối cùng thì có $3$ giây ô tô chuyển động với vận tốc $10\,m/s$ và $5$ giây chuyển động chậm dần đều với vận tốc $v\left(t \right)=-2t+10\,\left(m/s \right)$.\\
    Khi đó quãng đường ô tô di chuyển là $$S=3\cdot 10+\displaystyle\int\limits_0^5 \left(-2t+10\right)\mathrm{\,d}t=30+25=55\,m.$$
    }
\end{ex}

\begin{ex}%[2D4H2-6]
    Một ô tô đang chạy với tốc độ $20\,\left(m/s \right)$ thì gặp chướng ngại vật, người lái đạp phanh, từ thời điểm đó ô tô chuyển động chậm dần đều với vận tốc $v\left(t \right)=-5t+20\,\left(m/s \right)$, trong đó $t$ là khoảng thời gian tính bằng giây, kể từ lúc bắt đầu đạp phanh. Hỏi từ lúc đạp phanh đến khi dừng hẳn, ô tô còn di chuyển bao nhiêu mét ($m$)?
    \choice
    {$20\,m$}
    {$30\,m$}
    {$10\,m$}
    {\True $40\,m$}
    \loigiai{
    Khi ô tô dừng hẳn thì $v\left(t \right)=0\Leftrightarrow-5t+20=0\Leftrightarrow t=4\,\left(s \right)$.\\
    Vậy từ lúc đạp phanh đến khi dừng hẳn, ô tô di chuyển được 
    $$s=\displaystyle\int\limits_0^4 \left(-5t+20\right)\mathrm{\,d}t=40\,\left(m \right).$$
}
\end{ex}

\begin{ex}%[2D4V2-6]
    Một chất điểm $A$ xuất phát từ $O$, chuyển động thẳng với vận tốc biến thiên theo thời gian bởi quy luật $v\left(t \right)=\dfrac{1}{120}t^2+\dfrac{58}{45}t\,\left(m/s \right)$, trong đó $t$ (giây) là khoảng thời gian tính từ lúc $A$ bắt đầu chuyển động. Từ trạng thái nghỉ, một chất điểm $B$ cũng xuất phát từ $O$, chuyển động thẳng cùng hướng với $A$ nhưng chậm hơn $3$ giây so với $A$ và có gia tốc bằng $a\,\left(m/s^2 \right)$ ($a$ là hằng số). Sau khi $B$ xuất phát được $15$ giây thì đuổi kịp $A$. Vận tốc của $B$ tại thời điểm đuổi kịp $A$ bằng
    \choice
    {$21\,\left(m/s \right)$}
    {$25\,\left(m/s \right)$}
    {$36\,\left(m/s \right)$}
    {\True $30\,\left(m/s \right)$}
    \loigiai{
    Thời điểm chất điểm $B$ đuổi kịp chất điểm $A$ thì chất điểm $B$ đi được $15$ giây, chất điểm $A$ đi được $18$ giây.\\
    Biểu thức vận tốc của chất điểm $B$ có dạng $v_B\left(t \right)=\displaystyle\int a\mathrm{\,d}t =at+C$ mà $v_B\left(0\right)=0$ nên $v_B\left(t \right)=at$.\\
    Do từ lúc chất điểm $A$ bắt đầu chuyển động cho đến khi chất điểm $B$ đuổi kịp thì quãng đường hai chất điểm đi được bằng nhau.\\
    Do đó $\displaystyle\int\limits_0^{18} \left(\dfrac{1}{120}t^2+\dfrac{58}{45} \right)\mathrm{\,d}t=\displaystyle\int\limits_0^{15} at\mathrm{\,d}t \Leftrightarrow 225=a\cdot\dfrac{225}{2}\Leftrightarrow a=2$.\\
    Vậy vận tốc của chất điểm $B$ tại thời điểm đuổi kịp $A$ bằng 
    $$v_B\left(t \right)=2\cdot 15=30\,\left(m/s \right).$$
    }
\end{ex}

\begin{ex}%[2D4V2-6]
    Một chất điểm $A$ xuất phát từ $O$, chuyển động thẳng với vận tốc biến thiên theo thời gian bởi quy luật $v\left(t \right)=\dfrac{1}{150}t^2+\dfrac{59}{75}t\,\left(m/s \right)$, trong đó $t$ (giây) là khoảng thời gian tính từ lúc $a$ bắt đầu chuyển động. Từ trạng thái nghỉ, một chất điểm $B$ cũng xuất phát từ $O$, chuyển động thẳng cùng hướng với $A$ nhưng chậm hơn $3$ giây so với $A$ và có gia tốc bằng $a\,\left(m/s^2 \right)$ ($a$ là hằng số). Sau khi $B$ xuất phát được $12$ giây thì đuổi kịp $A$. Vận tốc của $B$ tại thời điểm đuổi kịp $A$ bằng
    \choice
    {$15\,\left(m/s \right)$}
    {$20\,\left(m/s \right)$}
    {\True $16\,\left(m/s \right)$}
    {$13\,\left(m/s \right)$}
    \loigiai{
    Quãng đường chất điểm $A$ đi từ đầu đến khi $B$ đuổi kịp là 
    $$S=\displaystyle\int\limits_0^{15} \left(\dfrac{1}{150}t^2+\dfrac{59}{75}t \right)\mathrm{\,d}t=96\,\left(m \right).$$
    Vận tốc của chất điểm $B$ là 
    $$v_B\left(t \right)=\displaystyle\int a\mathrm{\,d}t=at+C.$$
    Tại thời điểm $t=3$ vật $B$ bắt đầu từ trạng thái nghỉ nên $v_B\left(3\right)=0\Leftrightarrow C=-3a$.\\
    Lại có quãng đường chất điểm $B$ đi được đến khi gặp $A$ là 
    $$S_2=\displaystyle\int\limits_3^{15} \left(at-3a \right)\mathrm{\,d}t=\left. \left(\dfrac{at^2}{2}-3at \right) \right|_3^{15}=72a\,\left(m \right).$$
    Vậy $72a=96\Leftrightarrow a=\dfrac{4}{3}\,\left(m/s^2 \right)$.\\
    Tại thời điểm đuổi kịp $A$ thì vận tốc của $B$ là $v_B\left(15\right)=16\,\left(m/s \right)$.
    }
\end{ex}

\begin{ex}%[2D4V2-6]
    Một ô tô bắt đầu chuyển động thẳng đều với vận tốc $v_0$, sau $6$ giây chuyển động thì gặp chướng ngại vật nên bắt đầu giảm tốc độ với vận tốc chuyển động $v(t)=-\dfrac{5}{2}t+a\,(m/s)$ với $t\ge 6$ cho đến khi dừng hẳn. Biết rằng kể từ lúc chuyển động đến lúc dừng hẳn thì ô tô đi được quãng đường là $80\,m$. Tìm $v_0$.
    \choice
    {$v_0=35\,m/s$}
    {$v_0=25\,m/s$}
    {\True $v_0=10\,m/s$}
    {$v_0=20\,m/s$}
    \loigiai{
    Tại thời điểm $t=6$ vật đang chuyển động với vận tốc $v_0$ nên có 
    $$v(6)=v_0 \Leftrightarrow -\dfrac{5}{2}\cdot 6+a=v_0 \Leftrightarrow a=v_0+15 \Rightarrow v(t)=-\dfrac{5}{2}t+v_0+15.$$
    Gọi $k$ là thời điểm vật dừng hẳn, ta có 
    $$v(k)=0 \Leftrightarrow k=\dfrac{2}{5}\cdot\left(v_0+15\right)\Leftrightarrow k=\dfrac{2v_0}{5}+6.$$
    Tổng quãng đường vật đi được là 
    \allowdisplaybreaks 
    \begin{eqnarray*}
        && 80=6\cdot v_0+\displaystyle\int\limits_6^k \left(-\dfrac{5}{2}t+v_0+15\right)\mathrm{\,d}t\\
        &\Leftrightarrow& 80=6\cdot v_0+\left. \left(-\dfrac{5}{4}t^2+v_0\cdot t+15t \right) \right|_6^k \\ 
        &\Leftrightarrow& 80=6\cdot v_0-\dfrac{5}{4}\left(k^2-6^2\right)+v_0\cdot (k-6)+15(k-6) \\ 
        &\Leftrightarrow& 80=6\cdot v_0-\dfrac{5}{4}\left(\dfrac{4\left(v_0 \right)^2}{25}+\dfrac{24v_0}{5} \right)+v_0\cdot\dfrac{2v_0}{5}+15\cdot\dfrac{2v_0}{5} \\
        &\Leftrightarrow& \left(v_0 \right)^2+36\cdot v_0-400=0\\ 
        &\Leftrightarrow& v_0=10. 
    \end{eqnarray*}
    }
\end{ex}

\begin{ex}%[2D4H2-6]
    Để đảm bảo an toàn khi lưu thông trên đường, các xe ô tô khi dừng đèn đỏ phải cách nhau tối thiểu $1\,m$. Một ô tô $A$ đang chạy với vận tốc $16\,m/s$ bỗng gặp ô tô $B$ đang dừng đèn đỏ nên ô tô $A$ hãm phanh và chuyển động chậm dần đều với vận tốc được biểu thị bởi công thức $v_A\left(t \right)=16-4t$ (đơn vị tính bằng $m/s$), thời gian tính bằng giây. Hỏi rằng để hai ô tô $A$ và $B$ đạt khoảng cách an toàn khi dừng lại thì ô tô $A$ phải hãm phanh khi cách ô tô $B$ một khoảng ít nhất là bao nhiêu?
    \choice
    {$33$}
    {$12$}
    {$31$}
    {\True $32$}
    \loigiai{
    Ta có $v_A\left(0\right)=16\,m/s$.\\
    Khi xe $A$ dừng hẳn $v_A\left(t \right)=0 \Leftrightarrow t=4\,s$.\\
    Quãng đường từ lúc xe $A$ hãm phanh đến lúc dừng hẳn là 
    $$s=\displaystyle\int\limits_0^4 \left(16-4t \right)\mathrm{\,d}t=32\,m.$$
    }
\end{ex}

\begin{ex}%[2D4H2-6]
    Do các xe phải cách nhau tối thiểu $1\,m$ để đảm bảo an toàn nên khi dừng lại ô tô $A$ phải hãm phanh khi cách ô tô $B$ một khoảng ít nhất là $33\,m$. Một chất điểm đang chuyển động với vận tốc $v_0=15\,m/s$ thì tăng tốc với gia tốc $a\left(t \right)=t^2+4t\, \left(m/s^2\right)$. Tính quãng đường chất điểm đó đi được trong khoảng thời gian $3$ giây kể từ lúc bắt đầu tăng vận tốc.
    \choice
    {$70{,}25\, {m}$}
    {$68{,}25\, {m}$}
    {$67{,}25\, {m}$}
    {\True $69{,}75\, {m}$}
    \loigiai{
    Ta có 
    $$a\left(t \right)=t^2+4t \Rightarrow v\left(t \right)=\displaystyle\int a\left(t \right)\mathrm{\,d}t=\dfrac{t^3}{3}+2t^2+C,\, \left(C\in \mathbb{R} \right).$$
    Mà $v\left(0\right)=C=15 \Rightarrow v\left(t \right)=\dfrac{t^3}{3}+2t^2+15$.\\
    Vậy $S=\displaystyle\int\limits_0^3 \left(\dfrac{t^3}{3}+2t^2+15\right)\mathrm{\,d}t=69{,}75\, {m}$.
    }
\end{ex}

\begin{ex}%[2D4V2-6]
    Một vật chuyển động với vận tốc $10\,m/s$ thì tăng tốc với gia tốc được tính theo thời gian là $a\left(t \right)=t^2+3t$. Tính quãng đường vật đi được trong khoảng thời gian $6$ giây kể từ khi vật bắt đầu tăng tốc.
    \choice
    {$136\,{m}$}
    {$126\,{m}$}
    {$276\,{m}$}
    {\True $216\,{m}$}
    \loigiai{
    Ta có $v\left(0\right)=10\,m/s$ và 
    $$v\left(t \right)=\displaystyle\int\limits_0^t a\left(t \right)\mathrm{\,d}t=\displaystyle\int\limits_0^t \left(t^2+3t \right)\mathrm{\,d}t=\left. \left(\dfrac{t^3}{3}+\dfrac{3t^2}{2} \right) \right|_0^t=\dfrac{1}{3}t^3+\dfrac{3}{2}t^2.$$
    Quãng đường vật đi được là 
    $$S=\displaystyle\int\limits_0^6 v\left(t \right)\mathrm{\,d}t=\displaystyle\int\limits_0^6 \left(\dfrac{1}{3}t^3+\dfrac{3}{2}t^2 \right)\mathrm{\,d}t=\left. \left(\dfrac{1}{12}t^4+\dfrac{1}{2}t^3 \right) \right|_0^6=216\,{m}.$$
    }
\end{ex}

\begin{ex}%[2D4V2-6]
    Một chiếc máy bay chuyển động trên đường băng với vận tốc $v\left(t \right)=t^2+10t$ $\left(m/s \right)$ với $t$ là thời gian được tính theo đơn vị giây kể từ khi máy bay bắt đầu chuyển động. Biết khi máy bay đạt vận tốc $200\,\left(m/s \right)$ thì rời đường băng. Quãng đường máy bay đã di chuyển trên đường băng là
    \choice
    {\True $\dfrac{2500}{3}\,\left(m \right)$}
    {$2000\,\left(m \right)$}
    {$500\,\left(m \right)$}
    {$\dfrac{4000}{3}\,\left(m \right)$}
    \loigiai{
    Thời điểm máy bay đạt vận tốc $200\,\left(m/s \right)$ là 
    $$v\left(t \right)=200 \Leftrightarrow t^2+10t=200 \Leftrightarrow \hoac{& t=10\\ & t=-20}\Leftrightarrow t=10.$$
    Quãng đường máy bay đã di chuyển trên đường băng là
    $$s=\displaystyle\int\limits_0^{10} \left(t^2+10t \right)\mathrm{\,d}t=\left.\left(\dfrac{t^3}{3}+5t \right)\right|_0^{10}=\dfrac{2500}{3}\,\left(m \right).$$
    }
\end{ex}

\begin{ex}%[2D4V2-6]
    Một ô tô bắt đầu chuyển động nhậnh dần đều với vận tốc $v_1\left(t \right)=7t\,\left(m/s \right)$. Đi được $5\,s$, người lái xe phát hiện chướng ngại vật và phanh gấp, ô tô tiếp tục chuyển động chậm dần đều với gia tốc $a=-70\,\left(m/s^2 \right)$. Tính quãng đường $S$ đi được của ô tô từ lúc bắt đầu chuyển bánh cho đến khi dừng hẳn.
    \choice
    {\True $S=96{,}25\,\left(m\right)$}
    {$S=87{,}5\,\left(m\right)$}
    {$S=94\,\left(m\right)$}
    {$S=95{,}7\,\left(m\right)$}
    \loigiai{
    Chọn gốc thời gian là lúc ô tô bắt đầu đi.\\ 
    Sau $5\,s$ ô tô đạt vận tốc là $v\left(5\right)=35\,\left(m/s\right)$.\\
    Sau khi phanh vận tốc ô tô là $v\left(t\right)=35-70\left(t-5\right)$.\\
    Ô tô dừng tại thời điểm $t=5{,}5\,s$.\\
    Quãng đường ô tô đi được là 
    $$S=\displaystyle\int\limits_0^5 7t\mathrm{\,d}t+\displaystyle\int\limits_5^{5{,}5} \left[35-70\left(t-5\right) \right]\mathrm{\,d}t=96{,}25\,\left(m\right).$$
    }
\end{ex}

\begin{ex}%[2D4V2-6]
    Một ô tô bắt đầu chuyển động nhanh dần đều với vận tốc $v_1\left(t \right)=2t\,\left(m/s\right)$. Đi được $12$ giây, người lái xe gặp chướng ngại vật và phanh gấp, ô tô tiếp tục chuyển động chậm dần đều với gia tốc $a=-12\,\left(m/s^2\right)$. Tính quãng đường $s\left(m\right)$ đi được của ôtô từ lúc bắt đầu chuyển động đến khi dừng hẳn.
    \choice
    {\True $s=168\,\left(m\right)$}
    {$s=166\,\left(m\right)$}
    {$s=144\,\left(m\right)$}
    {$s=152\,\left(m\right)$}
    \loigiai{
    \textbf{Giải đoạn 1:} Xe bắt đầu chuyển động đến khi gặp chướng ngại vật.\\
    Quãng đường xe đi được là
    $$S_1=\displaystyle\int\limits_0^{12} v_1\left(t \right)\mathrm{\,d}t=\displaystyle\int\limits_0^{12} 2t\mathrm{\,d}t =\left. t^2 \right|_0^{12}=144\,\left(m\right).$$
    \textbf{Giải đoạn 2:} Xe gặp chướng ngại vật đến khi dừng hẳn.\\
    Ôtô chuyển động chậm dần đều với vận tốc 
    $$v_2\left(t \right)=\displaystyle\int a\mathrm{\,d}t=-12t+c.$$
    Vận tốc của xe khi gặp chướng ngại vật là $$v_2\left(0\right)=v_1\left(12\right)=2\cdot 12=24\,\left(m/s\right).$$
    Suy ra $-12\cdot 0+c=24 \Rightarrow c=24\Rightarrow v_2\left(t \right)=-12t+24$.\\
    Thời gian khi xe gặp chướng ngại vật đến khi xe dừng hẳn là nghiệm phương trình
    $$-12t+24=0\Leftrightarrow t=2.$$
    Khi đó, quãng đường xe đi được là
    $$S_2=\displaystyle\int\limits_0^2 v_2\left(t \right)\mathrm{\,d}t=\displaystyle\int\limits_0^2 \left(-12t+24\right)\mathrm{\,d}t=\left. \left(-6t^2+24t \right) \right|_0^2=24\,\left(m\right).$$
    Vậy tổng quãng đường xe đi được là $S=S_1+S_2=168\,\left(m\right)$.
    }
\end{ex}


\Closesolutionfile{ans}
