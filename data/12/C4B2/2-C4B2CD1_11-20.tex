

\Opensolutionfile{ans}[ans/ans-C4B2CD1]
\TNSA

%%%==============EX_1============%%%
\begin{ex}%[2D4H2-2]
	Tính tích phân $I=\displaystyle\int\limits\limits_1^2\dfrac{x-1}{x} \mathrm{d}x$ (\textit{\textit{làm tròn đến hàng phần trăm}}).
\shortans{$0{,}31$}	
\loigiai{
\begin{eqnarray*}
	I	&= &\displaystyle\int\limits_1^2\dfrac{x-1}{x} \mathrm{d}x\\
	&=& \displaystyle\int\limits_1^2\left(1-\dfrac{1}{x} \right) \mathrm{d}x\\
	&= & \left(x-\ln|x|\right)\Bigg|_1^2\\
	&=&	\left(2-\ln 2\right)-\left(1-\ln 1\right)=1-\ln 2.
	\end{eqnarray*}}
\end{ex}
%%%==============EX_2============%%%
\begin{ex}%[2D4H2-2]
Tính $I=\displaystyle\int\limits_1^2\left(\dfrac{x-\sqrt[{4}]{x^3}}{x} \right)^2 \mathrm{\,d}x$ (\textit{\textit{làm tròn đến hàng phần trăm}}).
\shortans{$0{,}01$}	
\loigiai{
\begin{eqnarray*}
	I	&= &\displaystyle\int\limits_1^2\left(\dfrac{x-\sqrt[{4}]{x^3}}{x} \right)^2 \mathrm{\,d}x\\
	&=& \displaystyle\int\limits_1^2\left(1-x^{-\tfrac{1}{4}}\right)^2 \mathrm{\,d}x\\
	&= &\displaystyle\int\limits_1^2\left(1-2x^{-\tfrac{1}{4}}+x^{-\tfrac{1}{8}}\right) \mathrm{\,d}x\\
	&=&	\left(x-\dfrac{8}{3}x^{\tfrac{3}{4}}+\dfrac{8}{7}x^{\tfrac{7}{8}} \right)\Bigg|_1^2\\
	&\approx& 0{,}01. 
\end{eqnarray*}
}
\end{ex}
%%%==============EX_3============%%%
\begin{ex}%[2D4H2-2]
Tính $I=\displaystyle\int\limits_1^2\left(\sqrt{x}+1\right)\left(\sqrt[{3}]{x}-1\right)\mathrm{\,d}x$ (\textit{\textit{làm tròn đến hàng phần trăm}}).
\shortans{$0{,}32$}	
\loigiai{
\begin{eqnarray*}
	I&= &\displaystyle\int\limits_1^2\left(\sqrt{x}+1\right)\left(\sqrt[{3}]{x}-1\right)\mathrm{\,d}x\\
	&=& \displaystyle\int\limits_1^2\left(x^{\tfrac{5}{6}}-x^{\tfrac{1}{2}}+x^{\tfrac{1}{3}}-1\right) \mathrm{\,d}x\\
	&=&	\left(\dfrac{6}{11}x^{\tfrac{11}{6}}-\dfrac{2}{3}x^{\tfrac{3}{2}}+\dfrac{3}{4}x^{\tfrac{4}{3}}-x \right)\Bigg|_1^2\\
	&\approx& 0{,}32. 
\end{eqnarray*}
}
\end{ex}
%%%==============EX_4============%%%
\begin{ex}%[2D4H2-2]
Tính $I=\displaystyle\int\limits_1^2\dfrac{(x^2+1)^3}{x^2} \mathrm{\,d}x$ (\textit{làm tròn đến hàng phần chục}).
\shortans{$16{,}7$}	
\loigiai{
	\begin{eqnarray*}
		I&= &\displaystyle\int\limits_1^2\dfrac{(x^2+1)^3}{x^2} \mathrm{\,d}x\\
		&=& \displaystyle\int\limits_1^2\left(x^4+3x^2+3+\dfrac{1}{x^2}\right) \mathrm{\,d}x\\
		&=&	\left(\dfrac{x^5}{5}+x^3+3x-\dfrac{1}{x}\right)\Bigg|_1^2\\
		&=& 16{,}7. 
	\end{eqnarray*}
}
\end{ex}
%%%==============EX_5============%%%
\begin{ex}%[2D4H2-4]
Tính $I=\displaystyle\int\limits _0^15^{x+1}\cdot7^{2x-1} \mathrm{\,d}x$ (\textit{làm tròn đến hàng đơn vị}).
\shortans{$959$}	
\loigiai{
\begin{eqnarray*}
	I&= &\displaystyle\int\limits _0^15^{x+1}\cdot7^{2x-1} \mathrm{\,d}x\\
	&=&\dfrac{5}{7} \displaystyle\int\limits_0^15^x\cdot49^x \mathrm{\,d}x\\
	&=&	\dfrac{5}{7} \displaystyle\int\limits_0^1245^x \mathrm{\,d}x\\
		&=&	\dfrac{5}{7}\left(245^x\ln245\right)\Bigg|_0^1\\
	&=&\dfrac{5}{7}\left(245\ln245-\ln245\right)\approx 959. 
\end{eqnarray*}
}
\end{ex}
%%%==============EX_6============%%%
\begin{ex}%[2D4H2-4]
Tính $I=\displaystyle\int\limits _0^1\left(x+\mathrm{e}^{-x-2} \right)\mathrm{\,d}x$ (\textit{\textit{làm tròn đến hàng phần trăm}}).
\shortans{$0{,}59$}	
\loigiai{
\begin{eqnarray*}
	I&= &\displaystyle\int\limits _0^1\left(x+\mathrm{e}^{-x-2} \right)\mathrm{\,d}x\\
	&=&	\left(\dfrac{x^2}{2}-\mathrm{e}^{-x-2}\right)\Bigg|_0^1\\
	&=&\left(\dfrac{1}{2}+\mathrm{e}^{-2}-\mathrm{e}^{-3}\right)\approx 0{,}59. 
\end{eqnarray*}
}
\end{ex}
%%%==============EX_7============%%%
\begin{ex}%[2D4H2-3]
Tính $I=\displaystyle\int\limits _{\tfrac{\pi}{6}}^{\tfrac{\pi}{3}}x^2 \left(1-\dfrac{\sin x}{x^2} \right)\mathrm{\,d}x$ (\textit{\textit{làm tròn đến hàng phần trăm}}).
\shortans{$-0{,}03$}	
\loigiai{
\begin{eqnarray*}
	I&= &\displaystyle\int\limits _{\tfrac{\pi}{6}}^{\tfrac{\pi}{3}}x^2 \left(1-\dfrac{\sin x}{x^2} \right)\mathrm{\,d}x\\
	&=&\displaystyle\int\limits _{\tfrac{\pi}{6}}^{\tfrac{\pi}{3}}\left(x^2-\sin x \right)\mathrm{\,d}x\\
	&=&\left(\dfrac{x^3}{3}+\cos x\right)\Bigg| _{\tfrac{\pi}{6}}^{\tfrac{\pi}{3}}\approx -0{,}03.  
\end{eqnarray*}
}
\end{ex}
%%%==============EX_8============%%%
\begin{ex}%[2D4H2-3]
Tính $I=\displaystyle\int\limits _{\tfrac{\pi}{6}}^{\tfrac{\pi}{2}}\left(\sin x-\dfrac{1}{\sqrt[{3}]{x^2}} \right) \mathrm{\,d}x$ \textit{(\textit{làm tròn đến hàng phần trăm})}.
\shortans{$0{,}38$}	
\loigiai{
\begin{eqnarray*}
	I&= &\displaystyle\int\limits _{\tfrac{\pi}{6}}^{\tfrac{\pi}{2}}\left(\sin x-\dfrac{1}{\sqrt[{3}]{x^2}} \right) \mathrm{\,d}x\\
	&=&\left(-\cos x-3\sqrt[{3}]{x}\right)\Bigg| _{\tfrac{\pi}{6}}^{\tfrac{\pi}{2}}\approx 0{,}38.  
\end{eqnarray*}
}
\end{ex}
%%%==============EX_9============%%%
\begin{ex}%[2D4H2-4]
Biết $\displaystyle\int\limits _0^1\dfrac{\left(e^{-x}+2\right)^2}{e^{x-1}} \mathrm{\,d}x=ae+b+\dfrac{c}{e}+\dfrac{1}{e^2}$ $\left(a,b,c\in \mathbb{Z}\right)$. Tính giá trị của $P=a+b+c$.
\shortans{$-1$}	
\loigiai{
	\begin{eqnarray*}
	I	&=& \displaystyle\int\limits _0^1\dfrac{\left(e^{-x}+2\right)^2}{e^{x-1}} \mathrm{\,d}x\\
		&= & \displaystyle\int\limits _0^1\dfrac{e^{-2x}+4e^{-x}+4}{e^{x-1}} \mathrm{\,d}x\\
	&=&	\displaystyle\int\limits _0^1\left(e^{-3x+1}+4e^{-2x+1}+4e^{-x+1} \right)\mathrm{\,d}x\\
	&=&\left. \left(\dfrac{e^{-3x+1}}{-3}+\dfrac{4e^{-2x+1}}{-2}+\dfrac{4e^{-x+1}}{-1} \right)\right|_0^1\\
	&=&\dfrac{-9e^3+4e^2+4e+1}{e^2}=-9e+4+\dfrac{4}{e}+\dfrac{1}{e^2}.
	\end{eqnarray*}
Vậy $ P=a+b+c=-1$.
}
\end{ex}
%%%==============EX_10============%%%
\begin{ex}%[2D4H2-3]
Biết $\displaystyle\int\limits _0^{\tfrac{\pi}{3}}\dfrac{1-\cos 2x}{1+\cos 2x} \mathrm{\,d}x=a\sqrt{3}+\dfrac{\pi}{b}$ $\left(a,b\in \mathbb{Z}\right)$. Tính $a+b$.
\shortans{$0$}	
\loigiai{
	\begin{eqnarray*}
		I&= & \displaystyle\int\limits _0^{\tfrac{\pi}{3}}\dfrac{1-\cos 2x}{1+\cos 2x} \mathrm{\,d}x\\
		&= & \displaystyle\int\limits _0^{\tfrac{\pi}{3}}\dfrac{2\sin^2 x}{2\cos^2 x} \mathrm{\,d}x\\
		&=& \displaystyle\int\limits _0^{\tfrac{\pi}{3}}\left(\dfrac{1}{\cos^2 x}-1\right)\mathrm{\,d}x\\
		&=&  \left(\tan x-x\right)\Bigg|_0^{\tfrac{\pi}{3}}=\sqrt{3}-\dfrac{\pi}{3}.
	\end{eqnarray*}
Vậy  $\heva{&a=1\\
	&b=-1}\Rightarrow a+b=0.$
}
\end{ex}
%%%==============EX_11============%%%
\begin{ex}%[2D4H2-4]
Tính $I=\displaystyle\int\limits _0^1\dfrac{\left(2024^x+1\right)^2}{e^{-3x}} \mathrm{\,d}x$ (\textit{làm tròn đến hàng phần trăm}).
\shortans{$0$}	
\loigiai{
\begin{eqnarray*} 
	 I&=&\int_0^1 \frac{\left(2024^x+1\right)^2}{e^{-3 x}} d x\\
	 &=&\int_0^1 \frac{2024^{2 x}+2 \cdot 2024^x+1}{e^{-3 x}} d x\\
	 &=&\left[\left(\frac{2024^2}{e^{-3}}\right)^x+2 \cdot\left(\frac{2024}{e^{-3}}\right)^x+e^{3 x}\right]\Bigg|_0 ^1 \\ 
	 & =&\dfrac{\left(\dfrac{2024^2}{e^{-3}}\right)^x}{\ln \dfrac{2024^2}{e^{-3}}}+\dfrac{2 \cdot\left(\dfrac{2024}{e^{-3}}\right)^x}{\ln \dfrac{2024}{e^{-3}}}+\dfrac{1}{3} e^{3 x}\\
	 &=&\dfrac{2024^{2 x} e^{3 x}}{2 \ln 2024-3}+\dfrac{2.2024^{2 x} e^{3 x}}{\ln 2024-3}+\dfrac{1}{3} e^{3 x} \\ 
	 & =&\left(\dfrac{2024^{2 x}}{2 \ln 2024-3}+\dfrac{2\cdot2024^{2 x}}{\ln 2024-3}+\dfrac{1}{3}\right) e^{3 x}. 
 \end{eqnarray*}
}
\end{ex}
%%%==============EX_12============%%%
\begin{ex}%[2D4H2-4]
Tính $I=\dfrac{1}{1000}\displaystyle\int\limits _0^1\dfrac{\left(e^{-x}+2\right)^2}{e^{x-1}} \mathrm{\,d}x$ (\textit{làm tròn đến hàng đơn vị}).
\shortans{$4522$}	
\loigiai{
\begin{eqnarray*}
	I&= &\dfrac{1}{1000}\displaystyle\int\limits _0^1\dfrac{\left(e^{-x}+2\right)^2}{e^{x-1}} \mathrm{\,d}x\\
	&=& \dfrac{1}{1000}\displaystyle\int\limits _0^1\dfrac{e^{-2x}+4e^{-x}+4}{e^{x-1}} \mathrm{\,d}x\\
	&= &\dfrac{1}{1000} \displaystyle\int\limits _0^1\left(e^{-3x+1}+4e^{-2x+1}+4e^{-x+1} \right)\mathrm{\,d}x\\
	&=& \dfrac{1}{1000}\left(e^{-3x+1}+4e^{-2x+1}+4e^{-x+1} \right)\Bigg|_0^1\\
		&=&\dfrac{1}{1000} \dfrac{-9e^3+4e^2+4e+1}{e^2}\approx 4522.
\end{eqnarray*}
}
\end{ex}
%%%==============EX_13============%%%
\begin{ex}%[2D4H2-4]
Tính $I=\dfrac{1}{100}\displaystyle\int\limits_1^2e^{2x} \left(2023+\dfrac{2024e^{-2x}}{x^3} \right) \mathrm{\,d}x$ (\textit{làm tròn đến hàng phần chục}).
\shortans{$48{,}5$}	
\loigiai{
\begin{eqnarray*}
	I&= &\dfrac{1}{100}\displaystyle\int\limits_1^2e^{2x} \left(2023+\dfrac{2024e^{-2x}}{x^3} \right) \mathrm{\,d}x\\
	&=&\dfrac{1}{100}\displaystyle\int\limits_1^2\left(2023e^{2x} +\dfrac{2024}{x^3} \right) \mathrm{\,d}x\\
	&=&\dfrac{1}{100}\left(2023\dfrac{e^{2x}}{2}-\dfrac{1012}{x}\right)\Bigg|_1^2\\
	&\approx& 48{,}5.
\end{eqnarray*}
}
\end{ex}
%%%==============EX_14============%%%
\begin{ex}%[2D4H2-4]
Tính $I=\displaystyle\int\limits_1^2\left(4x^3-2\cdot3^{x+1}+\dfrac{1}{x^2} \right) \mathrm{\,d}x$ (\textit{làm tròn đến hàng phần chục}).
\shortans{$-17{,}3$}	
\loigiai{
\begin{eqnarray*}
	I&= &\displaystyle\int\limits_1^2\left(4x^3-2\cdot3^{x+1}+\dfrac{1}{x^2} \right) \mathrm{\,d}x\\
	&=&\left(x^4-\dfrac{2\cdot3^{x+1}}{\ln3}-\dfrac{1}{x}\right)\Bigg|_1^2\\
	&\approx&-17{,}3.
\end{eqnarray*}
}
\end{ex}


\Closesolutionfile{ans}
\indapan{6}{ans/ans-C4B2CD1}

\begin{dang}{TÍCH PHÂN HÀM TRỊ TUYỆT ĐỐI}
Tính tích phân $I=\displaystyle\int\limits_a^b|f(x)| \mathrm{\,d}x$?\\
\textbf{Phương pháp}
\begin{itemize}
	\item \textbf{Bước 1.} Xét dấu $f(x)$ trên đoạn $[a ; b]$.
	\item \textbf{Bước 2.} Dựa vào bảng xét dấu trên đoạn $[a ; b]$ để khử $|f(x)|$. Sau đó sử dụng các phương pháp tính tích phân đã học để tính $I=\displaystyle\int\limits_a^b|f(x)| \cdot \mathrm{\,d}x$.
\end{itemize}
\end{dang}

\Opensolutionfile{ans}[ans/ans-C4B2CD1-Dang2]
\TN
%%%==============EX_1============%%%
\begin{ex}%[2D4V2-3]
	Giá trị của $I=\displaystyle\int\limits _0^{2\pi}\sqrt{1-\cos 2x} \mathrm{\,d}x$ bằng
	\choice
	{$\sqrt{3}$}
	{\True $4\sqrt{2}$}
	{$2\sqrt{3}$}
	{$\dfrac{\pi}{2}$}
	\loigiai{
Ta có	$I=\displaystyle\int\limits_0^{2\pi}\sqrt{1-\cos 2x} \mathrm{\,d}x=\displaystyle\int\limits _0^{2\pi}\sqrt{2\sin^2 x} \mathrm{\,d}x=\sqrt{2} \displaystyle\int\limits_0^{2\pi}\left|\sin x\right|\mathrm{\,d}x.$
		\\
Vì $x\in \left[0;\pi \right]\to \sin x > 0\Rightarrow \left|\sin x\right|=\sin x$;\\
		$x\in \left[\pi;2\pi \right]\to \sin x < 0\Rightarrow \left|\sin x\right|=-\sin x$.
		\\		
Vậy $I=\sqrt{2} \left(\displaystyle\int\limits_0^{\pi}\sin x \mathrm{\,d}x+\displaystyle\int\limits_{\pi}^{2\pi}-\sin x\mathrm{\,d}x \right)=\sqrt{2} \left(-\cos x\Bigg|_0^\pi+\cos x\Bigg|_\pi^{2\pi}\right) =4\sqrt{2}$.
	}
\end{ex}
%%%==============EX_2============%%%
\begin{ex}%[2D4H2-2]
	Tính tích phân $I=\displaystyle\int\limits _0^2\left|x-2\right|\mathrm{\,d}x$.
	\choice
	{$I=-2$}
	{$I=4$}
	{\True $I=2$}
	{$I=0$}
	\loigiai{
		Ta có $I=\displaystyle\int\limits _0^2\left|x-2\right|\mathrm{\,d}x.$\\
		Do $x\in \left[0;2\right]\Rightarrow x-2< 0\Leftrightarrow \left|x-2\right|=2-x$.\\
		Vậy $I=\displaystyle\int\limits _0^2\left(2-x\right)\mathrm{\,d}x=\left(2x-\dfrac{1}{2} x^2 \right)\Bigg|_0^2=4-2=2$.
	}
\end{ex}


%%%==============EX_3============%%%
\begin{ex}%[2D4H2-2]
	Tính tích phân $I=\displaystyle\int\limits _0^2\left|x^3-x\right|\mathrm{\,d}x$.
	\choice
	{$I=-\dfrac{1}{2}$}
	{$I=5$}
	{$I=\dfrac{1}{2}$}
	{\True $I=\dfrac{5}{2}$}
	\loigiai{
		Ta có $I=\displaystyle\int\limits _0^2\left|x^3-x\right|\mathrm{\,d}x.$\\
		Ta có $f(x)=x^3-x=x\left(x^2-1\right)=0\leftrightarrow \hoac{&x=0\\&x=-1\\&x=1.}$\\
		\[\Rightarrow f(x) > 0\forall x\in \left[1;2\right];\quad f(x) < 0\forall x\in \left[0;1\right].
		\]
		Vậy $I=\displaystyle\int\limits _0^1\left(x-x^3 \right)\mathrm{\,d}x+\displaystyle\int\limits_1^2\left(x^3-x\right)\mathrm{\,d}x=\left(\dfrac{1}{2} x^2-\dfrac{1}{4} x^4 \right)\Bigg|_0^1+\left(\dfrac{1}{4} x^4-\dfrac{1}{2}^2 \right)\Bigg|_1^2=\dfrac{5}{2}$.
	}
\end{ex}

%%%==============EX_4============%%%
\begin{ex}%[2D4H2-2]
	Tính tích phân $I=\displaystyle\int\limits _0^2\left|x^2+2x-3\right|\mathrm{\,d}x$.
	\choice
	{$I=-2$}
	{$I=4$}
	{\True $I=5$}
	{$I=-4$}
	\loigiai{
		Ta có		$I=\displaystyle\int\limits _0^2\left|x^2+2x-3\right|\mathrm{\,d}x.$\\
		Ta có $f(x)=x^2+2x-3=0\Rightarrow\hoac{&x=1\\&x=-3}\Rightarrow f(x) > 0$, $\forall x\in \left[1;2\right]$; $f(x) < 0$, $\forall x\in \left[0;1\right]$.
		\begin{eqnarray*} 
			I&=&\displaystyle\int\limits _0^1-f(x)\mathrm{\,d}x+\displaystyle\int\limits_1^2f(x)\mathrm{\,d}x\\
			&=&\displaystyle\int\limits _0^1\left(3-2x-x^2 \right)\mathrm{\,d}x+\displaystyle\int\limits_1^2\left(x^2+2x-3\right)\mathrm{\,d}x\\
			&=&	\left(3x-x^2-\dfrac{1}{3} x^3 \right)\Bigg|_0^1+\left(\dfrac{1}{3} x^3+x^2-3x\right)\Bigg|_1^2\\
			&=&\left(3-1-\dfrac{1}{3} \right)+\left[\left(\dfrac{8}{3}+4-6\right)-\left(\dfrac{1}{3}+1-3\right)\right]=5.
		\end{eqnarray*} 	
	}
\end{ex}
%%%==============EX_5============%%%

\begin{ex}%[2D4V2-2]
	Cho tích phân $I=\left(\sqrt{3}+\sqrt{2} \right)\displaystyle\int\limits _{-3}^3\left|x^2-1\right|\mathrm{\,d}x=\dfrac{20}{3}+\dfrac{4}{3}+\dfrac{16}{3}=a\sqrt{3}+b\sqrt{2}$ với $a,b\in \mathbb{Q}$. Tính $P=a+b$.
	\choice
	{$P=\dfrac{40}{3}$}
	{\True $P=\dfrac{80}{3}$}
	{$P=\dfrac{17}{3}$}
	{$P=\dfrac{98}{3}$}
	\loigiai{
		Ta có		$I=\left(\sqrt{3}+\sqrt{2} \right)\displaystyle\int\limits _{-3}^3\left|x^2-1\right|\mathrm{\,d}x$.\\
		Tính $J=\displaystyle\int\limits _{-3}^3\left|x^2-1\right|\mathrm{\,d}x$.\\
		Ta có $f(x)=x^2-1=0\Rightarrow \hoac{&x=1\\&x=-1.}$\\
		$\Rightarrow f(x) > 0$, $\forall x\in \left[-3;-1\right]\cup \left[1;3\right]$; và $f(x) < 0$, $\forall x\in \left[-1;1\right]$.\\
		Vậy
		\begin{eqnarray*} 
			I&=&\displaystyle\int\limits _{-3}^{-1}\left(x^2-1\right)\mathrm{\,d}x+\displaystyle\int\limits _{-1}^1\left(1-x^2 \right)\mathrm{\,d}x+\displaystyle\int\limits_1^3\left(x^2-1\right)\mathrm{\,d}x\\
			&=&\left(\dfrac{1}{3} x^3-x\right)\Bigg|_{-3}^{-1}+\left(x-\dfrac{1}{3} x^3 \right)\Bigg|_{-1}^{1}+\left(\dfrac{1}{3} x^3-x\right)\Bigg|_{1}^3\\
			&=&\dfrac{20}{3}+\dfrac{4}{3}+\dfrac{16}{3}=\dfrac{40}{3}.
		\end{eqnarray*}
		\[\Rightarrow I=\left(\sqrt{3}+\sqrt{2} \right)\displaystyle\int\limits _{-3}^3\left|x^2-1\right|\mathrm{\,d}x=\dfrac{40}{3} \sqrt{3}+\dfrac{40}{3} \sqrt{2}.
		\]
		Khi đó $a=\dfrac{40}{3}$, $b=\dfrac{40}{3}$. Suy ra $P=a+b=\dfrac{80}{3}$.
	}
\end{ex}
%%%==============EX_6============%%%
\begin{ex}%[2D4V2-2]
	Tính tích phân $I=\displaystyle\int\limits _{-2}^5\left(\left|x+2\right|-\left|x-2\right|\right)\mathrm{\,d}x$.
	\choice
	{$I=38$}
	{\True $I=44$}
	{$I=48$}
	{$I=40$}
	\loigiai{
		Ta có $I=\displaystyle\int\limits _{-2}^5\left(\left|x+2\right|-\left|x-2\right|\right)\mathrm{\,d}x.$\\
		Gọi $f(x)=\left|x+2\right|-\left|x-2\right|$ trên $x\in [-2;5]$. Khi đó
		\begin{itemize}
			\item Với $ x\in \left[-2;2\right]$ thì $f(x)=4$.
			\item Với $ x\in \left[2;5\right]$ thì $f(x)=2x$.
		\end{itemize} 
		Vậy $\displaystyle\int\limits _{-2}^5f(x)\mathrm{\,d}x=\displaystyle\int\limits _{-2}^24\mathrm{\,d}x+\leftarrow \displaystyle\int\limits_2^52x\mathrm{\,d}x=4x\Bigg|_{-2}^2+x^2\Bigg|_2^5=16+32-4=44$.
	}
\end{ex}
%%%==============EX_7============%%%
\begin{ex}%[2D4V2-4]
	Cho tích phân $I=\displaystyle\int\limits _0^3\left|2^x-4\right|\mathrm{\,d}x=a+\dfrac{b}{c\ln 2}$ với $a,b,c\in \mathbb{Z}$ và $\dfrac{b}{c}$ là phân số tối giản. Tính $P=a^2+b^2+c^2$.
	\choice
	{$P=15$}
	{$P=10$}
	{$P=5$}
	{\True $P=18$}
	\loigiai{
		Ta có $I=\displaystyle\int\limits _0^3\left|2^x-4\right|\mathrm{\,d}x$.
		Ta có $2^x-4> 0\Leftrightarrow x > 2\Rightarrow f(x) > 0,~\forall x\in \left[2;3\right]$; và $f(x) < 0,~\forall x\in \left[0;2\right]$.\\
		Vậy
		\begin{eqnarray*} 
			I&=&\displaystyle\int\limits _0^2\left(4-2^x \right)\mathrm{\,d}x+\displaystyle\int\limits_2^3\left(2^x-4\right)\mathrm{\,d}x\\
			&=&\left(4x-\dfrac{1}{\ln 2} 2^x \right)\Bigg|_0^2+\left(\dfrac{1}{\ln 2} 2^x-4x\right)\Bigg|_2^3\\
			&=&\left(8-\dfrac{3}{\ln 2} \right)+\left(\dfrac{4}{\ln 2}-4\right)=4+\dfrac{1}{\ln 2}.
		\end{eqnarray*} 
		\[\Rightarrow P=a^2+b^2+c^2=4^2+1^2+1^2=18.
		\]
	}
\end{ex}
%%%==============EX_8============%%%
\begin{ex}%[2D4V2-4]
	Tính tích phân $I=\displaystyle\int\limits _{-1}^1\left|2^x-2^{-x} \right|\mathrm{\,d}x$.
	\choice
	{\True $\dfrac{1}{\ln 2}$}
	{$\ln 2$}
	{$2\ln 2$}
	{$\dfrac{2}{\ln 2}$}
	\loigiai{
		$I=\displaystyle\int\limits _{-1}^1\left|2^x-2^{-x} \right|\mathrm{\,d}x$.\\
		Ta có $2^x-2^{-x}=0$ $\Rightarrow x=0$.
		\begin{eqnarray*} 
			I&=&\displaystyle\int\limits _{-1}^1\left|2^x-2^{-x} \right|\mathrm{\,d}x\\
			&=&\displaystyle\int\limits _{-1}^0\left|2^x-2^{-x} \right|\mathrm{\,d}x+\displaystyle\int\limits _0^1\left|2^x-2^{-x} \right|\mathrm{\,d}x\\
			&=&\left|\displaystyle\int\limits _{-1}^0\left(2^x-2^{-x} \right) \mathrm{\,d}x\right|+\left|\displaystyle\int\limits _0^1\left(2^x-2^{-x} \right) \mathrm{\,d}x\right|\\
			&=&\left|\left(\dfrac{2^x+2^{-x}}{\ln 2} \right)\Bigg|_{-1}^0 \right|+\left| \left(\dfrac{2^x+2^{-x}}{\ln 2} \right)\Bigg|_0^1 \right|=\dfrac{1}{\ln 2}.
		\end{eqnarray*} 		
	}
\end{ex}
%%%==============EX_9============%%%

\begin{ex}%[2D4V2-2]
	Tính tích phân $I=\displaystyle\int\limits _{-1}^2\left(\left|x\right|-\left|x-1\right|\right)\mathrm{\,d}x$.
	\choice
	{\True $I=0$}
	{$I=2$}
	{$I=-2$}
	{$I=-3$}
	\loigiai{
		Ta có $I=\displaystyle\int\limits _{-1}^2\left(\left|x\right|-\left|x-1\right|\right)\mathrm{\,d}x$.
		\begin{eqnarray*} 
			I&=&\displaystyle\int\limits _{-1}^2\left(\left|x\right|-\left|x-1\right|\right)\mathrm{\,d}x\\
			&=&\displaystyle\int\limits _{-1}^2\left|x\right|\mathrm{\,d}x-\displaystyle\int\limits _{-1}^2\left|x-1\right|\mathrm{\,d}x\\
			&=&-\displaystyle\int\limits _{-1}^0x\mathrm{\,d}x+\displaystyle\int\limits _0^2x\mathrm{\,d}x+\displaystyle\int\limits _{-1}^1(x-1)\mathrm{\,d}x-\displaystyle\int\limits_1^2(x-1)\mathrm{\,d}x\\
			&=&-\dfrac{x^2}{2}\Bigg|_{-1}^0+ \dfrac{x^2}{2}\Bigg|_0^2+ \left(\dfrac{x^2}{2}-x\right)\Bigg|_{-1}^1- \left(\dfrac{x^2}{2}-x\right)\Bigg|_1^2=0.
		\end{eqnarray*} 		
	}
\end{ex}


%%%==============EX_10============%%%
\begin{ex}%[2D4V2-2]
	Cho $a$ là số thực dương, tính tích phân $I=\displaystyle\int\limits _{-1}^a\left|x\right|\mathrm{d}x$ theo $a$.
	\choice
	{\True $I=\dfrac{a^2+1}{2}$}
	{$I=\dfrac{a^2+2}{2}$}
	{$I=\dfrac{-2a^2+1}{2}$}
	{$I=\dfrac{\left|3a^2-1\right|}{2}$}
	\loigiai{
		Vì $a > 0$ nên $I=-\displaystyle\int\limits_{-1}^0x \mathrm{\,d}x+\displaystyle\int\limits_0^ax \mathrm{\,d}x=\dfrac{1}{2}+\dfrac{a^2}{2}=\dfrac{1+a^2}{2}$.
	}
\end{ex}
%%%==============EX_11============%%%
\begin{ex}%[2D4V2-2]
	Cho số thực $m > 1$ thỏa mãn $\displaystyle\int\limits_1^m\left|2mx-1\right|\mathrm{\,d}x=1$. Khẳng định nào sau đây đúng?
	\choice
	{$m\in \left(4;6\right)$}
	{$m\in \left(2;4\right)$}
	{$m\in \left(3;5\right)$}
	{\True $m\in \left(1;3\right)$}
	\loigiai{
Do $m > 1\Rightarrow 2m > 2\Rightarrow \dfrac{1}{2m} < 1$. Do đó với $m > 1, x\in \left[1;m\right]\Rightarrow 2mx-1> 0$.\\		
Vậy
\begin{eqnarray*} 
	\displaystyle\int\limits_1^m\left|2mx-1\right|\mathrm{\,d}x&=&\displaystyle\int\limits_1^m\left(2mx-1\right)\mathrm{\,d}x\\
	&=&\left(mx^2-x\right)\Bigg|_1^m\\
	&=&m^3-m-m+1=m^3-2m+1.
\end{eqnarray*}
		Từ đó theo bài ra ta có $m^3-2m+1=1\Leftrightarrow \hoac{&m=0 \\&m=\pm \sqrt{2}.} $\\ Do $m > 1$ vậy $m=\sqrt{2}$.
	}
\end{ex}
%%%==============EX_12============%%%
\begin{ex}%[2D4V2-2]
	Khẳng định nào sau đây là đúng?
	\choice
	{$\displaystyle\int\limits _{-1}^1\left|x\right|^3 \mathrm{d}x=\left|\displaystyle\int\limits _{-1}^1x^3 \mathrm{d}x \right|$}
	{\True $\displaystyle\int\limits _{-1}^{2024}\left|x^4-x^2+1\right|\mathrm{d}x=\displaystyle\int\limits _{-1}^{2024}\left(x^4-x^2+1\right)\mathrm{d}x$}
	{$\displaystyle\int\limits _{-2}^3\left|e^x \left(x+1\right)\mathrm{d}x\right|=\displaystyle\int\limits _{-2}^3e^x \left(x+1\right)\mathrm{d}x$}
	{$\displaystyle\int\limits _{-\tfrac{\pi}{2}}^{\tfrac{\pi}{2}}\sqrt{1-\cos^2 x} \mathrm{d}x=\displaystyle\int\limits _{-\tfrac{\pi}{2}}^{\tfrac{\pi}{2}}\sin x\mathrm{d}x$}
	\loigiai{
		Ta có: $x^4-x^2+1=x^4-2\cdot x^2\cdot\dfrac{1}{2}+\dfrac{1}{4}+\dfrac{3}{4}$ $=\left(x^2-\dfrac{1}{2} \right)^2+\dfrac{3}{4} > 0,\forall x\in {\bf \mathbb{R}}$.\\
		Do đó $\displaystyle\int\limits _{-1}^{2024}\left|x^4-x^2+1\right|\mathrm{d}x=\displaystyle\int\limits _{-1}^{2024}\left(x^4-x^2+1\right)\mathrm{d}x$.
	}
\end{ex}
%%%==============EX_13============%%%
\begin{ex}%[2D4V2-2]
	Tính tích phân $I=\displaystyle\int\limits_1^4\sqrt{x^2-6x+9} \mathrm{\,d}x$.
	\choice
	{\True $I=\dfrac{5}{2}$}
	{$I=-\dfrac{1}{2}$}
	{$I=-2$}
	{$I=\dfrac{1}{2}$}
	\loigiai{
Ta có $I=\displaystyle\int\limits_1^4\sqrt{x^2-6x+9} \mathrm{\,d}x=\displaystyle\int\limits_1^4\left|x-3\right|\mathrm{\,d}x$.\\
Ta có $x-3> 0,~\forall x\in \left[3;4\right];~x-3< 0,~\forall x\in \left[1;3\right]$.\\
Vậy
\begin{eqnarray*} 
I&=&\displaystyle\int\limits_1^3\left(3-x\right)\mathrm{\,d}x+\displaystyle\int\limits_3^4\left(x-3\right)\mathrm{\,d}x\\
	&=&\left(3x-\dfrac{1}{2} x^2 \right)\Bigg|_1^3+\left(\dfrac{1}{2} x^2-3x\right)\Bigg|_3^4\\
	&=&2+\dfrac{1}{2}=\dfrac{5}{2}.
\end{eqnarray*}
	}
\end{ex}
\Closesolutionfile{ans}
\indapan{6}{ans/ans-C4B2CD1-Dang2}



\Opensolutionfile{ans}[ans/ans-C4B2CD1-Dang2-KQ]
\TNSA

\begin{ex}%[2D4V2-2]
Tính tích phân $I=\displaystyle\int\limits _{-3}^{3}\left|x^{2} -1\right|\mathrm{\,d}x $ (tính gần đúng đến hàng phần chục).
	\shortans{$13{,}3$}	
	\loigiai{
	\[I=\displaystyle\int\limits _{-3}^{3}\left|x^{2} -1\right|\mathrm{\,d}x.\] 
	Vì  $f(x)=x^{2} -1=0\to\hoac{&x=-1\\&x=1} \Rightarrow f(x)>0,~\forall x\in \left[-3;-1\right]\cup \left[1;3\right]$; $f(x)<0,~\forall x\in \left[-1;1\right]$.\\
Vậy  
\begin{eqnarray*} 
	I&=&\displaystyle\int\limits _{-3}^{-1}\left(x^{2} -1\right)\mathrm{\,d}x+\displaystyle\int\limits _{-1}^{1}\left(1-x^{2} \right)\mathrm{\,d}x+\displaystyle\int\limits _{1}^{3}\left(x^{2} -1\right)\mathrm{\,d}x\\
	&=&\left(\frac{1}{3} x^{3} -x\right)\Bigg|_{-3}^{-1}+\left(x-\frac{1}{3} x^{3} \right)\Bigg|_{-1}^1+\left(\frac{1}{3} x^{3} -x\right)\Bigg|_1^3\\
	&=&\frac{20}{3} +\frac{4}{3} +\frac{16}{3} =\frac{40}{3}\approx 13{,}3.
\end{eqnarray*}
	}
\end{ex}

\begin{ex}%[2D4V2-2]
	Tính tích phân $I=\displaystyle\int\limits _{-1}^{2}\left|-x^{2} -2x+3\right|\mathrm{\,d}x $ (tính gần đúng đến hàng phần trăm).
	\shortans{$7{,}67$}	
	\loigiai{
Vì  $f(x)=-x^{2} -2x+3=0\Rightarrow\hoac{&x=1\\&x=-3} \Rightarrow f(x)>0,~\forall x\in \left[-1;-1\right]$; $f(x)<0,~\forall x\in \left[1;2\right]$\\
Vậy  
\begin{eqnarray*} 
	I&=&\displaystyle\int\limits _{-1}^{1}\left(-x^{2} -2x+3\right)\mathrm{\,d}x+\displaystyle\int\limits _{1}^{2}\left(x^{2}+2x-3 \right)\mathrm{\,d}x\\
	&=&\left(-\dfrac{1}{3} x^{3}-x^2 +3x\right)\Bigg|_{-1}^{1}+\left(\dfrac{1}{3} x^{3}+x^2 -3x \right)\Bigg|_{1}^2\\
	&=&-\dfrac{1}{3}-1+3-\dfrac{1}{3}+1+3 +\dfrac{8}{3}+4-6-\dfrac{1}{3}-1+3 \approx7{,}67.
\end{eqnarray*}		
	}
\end{ex}

\begin{ex}%[2D4V2-2]
	Tính tích phân $I=\displaystyle\int\limits _{1}^{2}\left|\frac{x+1}{x} \right|\mathrm{\,d}x $ (tính gần đúng đến hàng phần trăm).
	\shortans{$1{,}69$}	
	\loigiai{
Vì $\frac{x+1}{x}>0$, $\forall x\in [1;2]$ nên
\[I=\displaystyle\int\limits _{1}^{2}\left(\dfrac{x+1}{x}\right)\mathrm{\,d}x=\displaystyle\int\limits _{1}^{2}\left(1+\dfrac{1}{x}\right)\mathrm{\,d}x=\left(x+\ln x \right)\Bigg|1^2=2+\ln2-1=1+\ln2\approx1{,}69.  \]		
	}
\end{ex}

\begin{ex}%[2D4V2-2]
	Tính tích phân $I=\displaystyle\int\limits _{2}^{6}\sqrt{x^{2} -8x+16} \mathrm{\,d}x $.
	\shortans{$4$}	
	\loigiai{
Ta có $I=\displaystyle\int\limits _{2}^{6}\left| x-4\right|  \mathrm{\,d}x $.\\
Ta có $x-4\le 0$, $\forall x\in [2;4]$	và 	 $x-4\ge 0$, $\forall x\in [4;6]$. Khi đó
\[I=\displaystyle\int\limits _{2}^{4}\left( 4- x\right)   \mathrm{\,d}x+\displaystyle\int\limits _{4}^{6}\left( x- 4\right)   \mathrm{\,d}x=\left( 4 x-\dfrac{x^2}{2}\right)\Bigg|_{2}^{4}+\left( -4 x+\dfrac{x^2}{2}\right)\Bigg|_{4}^{6}=4.\]
	}
\end{ex}

\begin{ex}%[2D4V2-2]
	Tính tích phân $I=\displaystyle\int\limits _{-2}^{1}\sqrt{4x^{2} +6x+9} \mathrm{\,d}x $ (\textit{làm tròn đến hàng phần trăm}).
	\shortans{$9{,}38$}	
	\loigiai{
		Ta có $I=\displaystyle\int\limits _{-2}^{1}\sqrt{4x^{2} +6x+9} \mathrm{\,d}x=\displaystyle\int\limits _{-2}^{1}\left|2x+3 \right|  \mathrm{\,d}x$.\\
		Ta có $2x+3\le 0$, $\forall x\in \left[-2;-\dfrac{3}{2} \right] $	và 	 $2x+3\ge 0$, $\forall x\in \left[-\dfrac{3}{2};1 \right]$. Khi đó
\begin{eqnarray*} 
	I&=&\displaystyle\int\limits _{-2}^{-\tfrac{3}{2}}\left( -2x-3 \right)   \mathrm{\,d}x+\displaystyle\int\limits _{-\tfrac{3}{2}}^{1}\left( 2x+3 \right)  \mathrm{\,d}x\\
	&=&\left(-x^2-3x\right)\Bigg|_{-2}^{-\tfrac{3}{2}}+\left(x^2+3x \right)\Bigg|_{-\tfrac{3}{2}}^{1}\\
	&\approx&9{,}38.
\end{eqnarray*}				
	}
\end{ex}

\begin{ex}%[2D4V2-2]
	Tính tích phân $I=\displaystyle\int\limits _{0}^{1}\sqrt{9x^{2} -6x+1} \mathrm{\,d}x $ (\textit{làm tròn đến hàng phần trăm}).
	\shortans{$0{,}83$}	
	\loigiai{
Ta có $I=\displaystyle\int\limits _{0}^{1}\sqrt{9x^{2} -6x+1} \mathrm{\,d}x =\displaystyle\int\limits _{0}^{1}\left| 3x-1\right| \mathrm{\,d}x $.\\
Ta có $3x+1\le 0$, $\forall x\in \left[1;\dfrac{1}{3} \right] $	và 	 $3x+1\ge 0$, $\forall x\in \left[\dfrac{1}{3};1 \right]$. Khi đó
\begin{eqnarray*} 
	I&=&\displaystyle\int\limits _{0}^{\tfrac{1}{3}}\left(-3x-1 \right)   \mathrm{\,d}x+\displaystyle\int\limits _{\tfrac{1}{3}}^{1}\left( 3x+1 \right)  \mathrm{\,d}x\\
	&=&\left(-\dfrac{3x^2}{2}-x\right)\Bigg| _{0}^{\tfrac{1}{3}}+\left(\dfrac{3x^2}{2}+x\right)\Bigg|_{\tfrac{1}{3}}^{1}\\
	&\approx&0{,}83.
\end{eqnarray*}		
	}
\end{ex}

\begin{ex}%[2D4V2-3]
	Tính tích phân $I=\displaystyle\int\limits _{0}^{2\pi }\sqrt{1+\cos 2x} \mathrm{\,d}x $ (\textit{làm tròn đến hàng phần trăm}).
	\shortans{$5{,}66$}	
	\loigiai{
Ta có  $I=\displaystyle\int\limits _{0}^{2\pi }\sqrt{1+\cos 2x} \mathrm{\,d}x =\sqrt{2}\displaystyle\int\limits _{0}^{2\pi }|\cos x|\mathrm{\,d}x $.\\
Ta có $\cos x\ge 0, \forall x\in \left[0;\dfrac{\pi}{2} \right]\cup\left[\dfrac{3\pi}{2};2\pi \right] $ và $\cos x\le 0, \forall x\in \left[\dfrac{\pi}{2};\dfrac{3\pi}{2} \right] $. Khi đó
\begin{eqnarray*} 
	I&=&\sqrt{2}\displaystyle\int\limits _{0}^{\tfrac{\pi}{2} }\cos x\mathrm{\,d}x-\sqrt{2}\displaystyle\int\limits _{\tfrac{\pi}{2} }^{\tfrac{3\pi}{2} }\cos x\mathrm{\,d}x+\sqrt{2}\displaystyle\int\limits _{\tfrac{3\pi}{2} }^{2\pi}\cos x\mathrm{\,d}x\\
	&=&\sqrt{2}\sin x\Bigg|_{0}^{\tfrac{\pi}{2} } -\sqrt{2}\sin x\Bigg|_{\tfrac{\pi}{2} }^{\tfrac{3\pi}{2} }+\sqrt{2}\sin x\Bigg|_{\tfrac{3\pi}{2} }^{2\pi}\\
	&=&4\sqrt{2}\approx5{,}66.
	\end{eqnarray*}		}
\end{ex}

\begin{ex}%[2D4V2-3]
	Tính tích phân $I=\displaystyle\int\limits_{0}^{2\pi }\sqrt{1-\cos 2x} \mathrm{\,d}x $ (\textit{làm tròn đến hàng phần trăm}).
	\shortans{$5{,}66$}	
	\loigiai{
Ta có  $I=\displaystyle\int\limits _{0}^{2\pi }\sqrt{1-\cos 2x} \mathrm{\,d}x =2\displaystyle\int\limits _{0}^{2\pi }|\sin  x|\mathrm{\,d}x $.\\
Ta có $\sin x\ge 0, \forall x\in \left[0;\pi \right]$ và $\sin x\le 0, \forall x\in \left[\pi;2\pi \right]$. Khi đó
\begin{eqnarray*} 
	I&=&\sqrt{2}\displaystyle\int\limits _{0}^{\pi}\sin x\mathrm{\,d}x-\sqrt{2}\displaystyle\int\limits _{\pi}^{2\pi}\sin x\mathrm{\,d}x\\
	&=&-\sqrt{2}\cos x\Bigg|_{0}^{\pi } +\sqrt{2}\cos x\Bigg|_{\pi }^{2\pi}\\
	&=&4\sqrt{2}\approx5{,}66.
\end{eqnarray*}			
	}
\end{ex}


\begin{ex}%[2D4V2-3]
Tính tích phân $I=\displaystyle\int\limits _{0}^{2\pi }\sqrt{1-\sin 2x} \mathrm{\,d}x $, (\textit{làm tròn đến hàng phần trăm}).
	\shortans{$0{,}31$}	
	\loigiai{
Ta có  $I=\displaystyle\int\limits _{0}^{2\pi }\sqrt{1-\sin 2x} \mathrm{\,d}x =\displaystyle\int\limits _{0}^{2\pi }|\sin  x-\cos x|\mathrm{\,d}x $.\\
Ta có $\sin x-\cos x\le 0, \forall x\in \left[0;\dfrac{\pi}{4} \right]\cup\left[\dfrac{5\pi}{4};2\pi \right] $ và $\sin x-\cos x\ge 0, \forall x\in \left[\dfrac{\pi}{4};\dfrac{5\pi}{4} \right]$. Khi đó
\begin{eqnarray*} 
	I&=&\displaystyle\int\limits _{0}^{\tfrac{\pi}{4} }\left( \cos x-\sin x\right) \mathrm{\,d}x+\displaystyle\int\limits _{\tfrac{\pi}{4} }^{\tfrac{5\pi}{4} }\left( \sin x-\cos x\right) \mathrm{\,d}x+\displaystyle\int\limits _{\tfrac{5\pi}{4} }^{2\pi}\left( \cos x-\sin x\right) \mathrm{\,d}x\\
	&=&\left( \sin x+\cos x\right) \Bigg|_{0}^{\tfrac{\pi}{4} }+\left( -\cos x-\sin x\right) \Bigg|_{\tfrac{\pi}{4} }^{\tfrac{5\pi}{4} }+\left( \sin x+\cos x\right) \Bigg|_{\tfrac{5\pi}{4} }^{2\pi}\\
	&=&4\sqrt{2}\approx5{,}66.
\end{eqnarray*}				
	}
\end{ex}

\begin{ex}%[2D4V2-3]
	Tính tích phân $I=\displaystyle\int\limits _{0}^{2\pi }\sqrt{1+\sin 2x} \mathrm{\,d}x $ (\textit{làm tròn đến hàng phần trăm}).
	\shortans{$5{,}66$}	
	\loigiai{
Ta có  $I=\displaystyle\int\limits _{0}^{2\pi }\sqrt{1+\sin 2x} \mathrm{\,d}x =\displaystyle\int\limits _{0}^{2\pi }|\sin  x+\cos x|\mathrm{\,d}x $.\\
Ta có $\sin x+\cos x\ge 0, \forall x\in \left[0;\dfrac{3\pi}{4} \right]\cup\left[\dfrac{7\pi}{4};2\pi \right] $ và $\sin x+\cos x\le 0, \forall x\in \left[\dfrac{3\pi}{4};\dfrac{7\pi}{4} \right]$. \\
Khi đó:
\begin{eqnarray*} 
	I&=&\displaystyle\int\limits _{0}^{\tfrac{3\pi}{4} }\left( \cos x+\sin x\right) \mathrm{\,d}x-\displaystyle\int\limits _{\tfrac{3\pi}{4} }^{\tfrac{7\pi}{4} }\left( \sin x+\cos x\right) \mathrm{\,d}x+\displaystyle\int\limits _{\tfrac{7\pi}{4} }^{2\pi}\left( \cos x+\sin x\right) \mathrm{\,d}x\\
	&=&\left( \sin x-\cos x\right) \Bigg|_{0}^{\tfrac{3\pi}{4} }-\left( \sin x-\cos x\right) \Bigg|_{\tfrac{3\pi}{4} }^{\tfrac{7\pi}{4} }+\left( \sin x-\cos x\right) \Bigg|_{\tfrac{7\pi}{4} }^{2\pi}\\
	&=&4\sqrt{2}\approx5{,}66.
\end{eqnarray*}				
	}
\end{ex}


\Closesolutionfile{ans}
\indapan{6}{ans/ans-C4B2CD1-Dang2-KQ}
