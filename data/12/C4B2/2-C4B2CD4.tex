\subsection{Tích phân hàm ẩn biến đổi phức tạp}
% \begin{tomtat}
% 	Cần nhớ các công thức đạo hàm của hàm hợp
% 	\begin{itemize}
% 		\item $\displaystyle\int f'(x)\mathrm{\,d}x=f(x)+C$
% 		\item $f'(x)\cdot g(x)+f(x)\cdot g'(x)=\left[f(x)\cdot g(x)\right]'$
% 		\item $\dfrac{f'(x)\cdot g(x)-f(x)\cdot g'(x)}{g^2(x)}=\left[\dfrac{f(x)}{g(x)}\right]'$
% 		\item $\dfrac{f'(x)}{f(x)}=\left[\ln \left(f(x)\right)\right]'$
% 		\item $ -\dfrac{f'(x)}{f^2(x)}=\left[\dfrac 1{f(x)}\right]'$
% 		\item $-\dfrac{f'(x)}{f^n(x)}=\left[\dfrac 1{(n-1)[f(x)]^{n-1}}\right]'$
% 		\item $n\cdot f'(x)\cdot \left(f(x)\right)^{n-1}=\left[f(x)^n\right]'$
% 		\item $\dfrac{f'(x)}{\sqrt{f(x)}}=\left[2\sqrt{f(x)}\right]'$
% 	\end{itemize}	 
% \end{tomtat}
% \begin{dang}{.}
% \begin{enumerate}
		
% \item[1.]  Điều kiện hàm ẩn có dạng$\colon $ $\hoac{&f'(x)=g(x) \cdot h\left[f(x)\right] \\ & f'(x) \cdot h[f(x)]=g(x).}$

% Phương pháp giải$\colon $
% 	\begin{itemize}
% 	\item $\dfrac{f'(x)}{h[f(x)]}=g(x) \Leftrightarrow \displaystyle\int \dfrac{f'(x)}{h[f(x)]} \mathrm{\,d}x=\displaystyle\int g(x) \mathrm{\,d}x \Leftrightarrow \displaystyle\int \dfrac{d[f(x)]}{h[f(x)]}=\displaystyle\int g(x)\mathrm{\,d}x.$
% 	\item $f'(x) h[f(x)]=g(x) \Leftrightarrow \displaystyle\int f'(x) h[f(x)] \mathrm{\,d}x=\displaystyle\int g(x)\mathrm{\,d}x$\\$ \Leftrightarrow \displaystyle\int h[f(x)]  d\left[f'(x)\right]=\displaystyle\int g(x).$
% 	\end{itemize}
% Chú ý$\colon$ Ngoài việc nguyên hàm hai vế, ta có thể lấy tích phân hai vế (tùy câu hỏi của bài toán).\\
% \item[2.] Điều kiện hàm ẩn có dạng$\colon $ $\hoac{&f'(x)+p(x) \cdot f(x)=0 \\ & f'(x)+p(x) \cdot[f(x)]^n=0.}$

% Phương pháp giải$\colon $
% \begin{itemize}
% 	\item $f'(x)+p(x) \cdot f(x)=0.$\\
% Chia hai vế với $f(x)$ ta đựơc $\dfrac{f'(x)}{f(x)}+p(x)=0 \Leftrightarrow \dfrac{f'(x)}{f(x)}=-p(x).$\\
% Suy ra $\displaystyle\int \dfrac{f'(x)}{f(x)} \mathrm{d} x=-\displaystyle\int p(x) \mathrm{d} x \Leftrightarrow \ln |f(x)|=-\displaystyle\int p(x) \mathrm{d} x$.\\
% Từ đây ta dễ dàng tính được $f(x).$
% 	\item $f'(x)+p(x) \cdot[f(x)]^n=0$\\
% Chia hai vế với $[f(x)]^n$ ta được $\dfrac{f'(x)}{[f(x)]^n}+p(x)=0 \Leftrightarrow \dfrac{f'(x)}{[f(x)]^n}=-p(x).$
% \end{itemize}
% \end{enumerate}
% \end{dang}
\setcounter{ex}{0}
\Opensolutionfile{ans}[ans/ans-2-B1]
\TN  
\begin{ex}%[2D4C2-2]
	Cho hàm số $f(x)$ nhận giá trị không âm và có đạo hàm liên tục trên $\mathbb{R}$ thỏa mãn $f'(x)=(2x+1){{\left[f(x) \right]}^2},\forall x\in \mathbb{R}$ và $f(0)=-1$. Tính tích phân $\displaystyle\int\limits_0^1\left(x^3-1\right)f(x)\mathrm{\,d}x$.
	\choice
	{$1$}
	{$\dfrac{2}{3}$}
	{\True $\dfrac{1}{2}$}
	{$\dfrac{3}{2}$}
	\loigiai{
		Ta có
		$$
		\begin{aligned}
			 &&f'(x)=(2x+1)[f(x)]^2,\forall x\in\mathbb{R}\\
			&\Rightarrow&\dfrac{-f'(x)}{[f(x)]^2}=-(2x+1),\forall x\in\mathbb{R}\\ 			
			&\Rightarrow&\left[\dfrac 1{f(x)}\right]'=-(2x+1),\forall x\in\mathbb{R}.
		\end{aligned}
		$$
		Suy ra $\dfrac{1}{f(x)}=-\displaystyle\int{\left(2x+1\right)}\mathrm{\,d}x=-x^2-x+C\Rightarrow f(x)=\dfrac{1}{-x^2-x+C}$.\\
		Vì  $f(0)=-1\Rightarrow C=-1$.\\
		Suy ra $f(x)=-\dfrac{1}{x^2+x+1}$.\\
		$\displaystyle\int\limits_0^1\left(x^3-1\right)f(x)\mathrm{\,d}x=-\displaystyle\int\limits_0^1\left(x^3-1\right)\left(\dfrac{1}{x^2+x+1}\right)\mathrm{\,d}x=\displaystyle\int\limits_0^1\left(1-x\right)\mathrm{\,d}x$\\
		$=\left.\left(x-\dfrac{x^2}{2}\right)\right|_0^1=\dfrac{1}{2}$.}
\end{ex}

\begin{ex}%[2D4C2-2]
	Cho hàm số $f(x)\ne 0$, liên tục trên đoạn $\left[1;2\right]$ và thỏa mãn $f(1)=\dfrac{1}{3}$; $\linebreak x^2\cdot f'(x)=f^2(x)$ với $\forall x\in\left[1;2\right]$. Tính tích phân $I=\displaystyle\int\limits_1^2\left(2x+1\right)^2f(x)\mathrm{\,d}x$.
	\choice
	{$I=\dfrac{7}{6}$}
	{$I=\dfrac{5}{6}$}
	{\True $I=\dfrac{37}{6}$}
	{$I=\dfrac{1}{6}$}
	\loigiai{
		Ta có
		$$
		\begin{aligned}
		&x^2\cdot f'(x)=f^2(x)\\ 
		\Rightarrow&\dfrac{f'(x)}{f^2(x)}=\dfrac 1{x^2}\\ 
		\Rightarrow&{\left[-\dfrac 1{f(x)}\right]'}=\dfrac 1{x^2}\\ 
		\Rightarrow&-\dfrac 1{f(x)}=\displaystyle\int{\dfrac 1{x^2}}\mathrm{\,d}x\\
		 \Rightarrow&\dfrac 1{f(x)}=-\displaystyle\int{\dfrac 1{x^2}}\mathrm{\,d}x\\
		  \Rightarrow&\dfrac 1{f(x)}=\dfrac 1 x+C.\\ 
		\end{aligned}
		$$
		Mà $f(1)=\dfrac{1}{3}$ $\Rightarrow 3=1+C\Rightarrow C=2.$\\
		Do đó $\dfrac{1}{f(x)}=\dfrac{1}{x}+2 \Rightarrow f(x)=\dfrac{x}{2x+1}.$\\
		Vậy $I=\displaystyle\int\limits_1^2\left(2x+1\right)^2f(x)\mathrm{\,d}x=\displaystyle\int\limits_1^2\left(2x+1\right)^2\dfrac{x}{2x+1}\mathrm{\,d}x=\displaystyle\int\limits_1^2\left(2x^2+x\right)\mathrm{\,d}x=\dfrac{37}{6}$.}
\end{ex}

\begin{ex}%[2D4C2-2]
	Cho hàm số $f(x)$ có đạo hàm trên $\mathbb{R}$ thỏa mãn $3f'(x)\cdot \mathrm{e}^{f^3(x)}-\dfrac{2x}{f^2(x)}=0$ với $\forall x\in\mathbb{R}$. Biết $f(1)=0$, tính tích phân $I=\displaystyle\int\limits_0^{2024}{\dfrac{1}{\sqrt[3]{2\ln x}}\cdot f(x){\mathrm{\,d}}x}$.
	\choice
	{$1$}
	{$\dfrac{1}{2024}$}
	{\True $2024$}
	{$0$}
	\loigiai{
		Ta có
		$$
		\begin{aligned}
		&3f'(x)\cdot\mathrm{e}^{f^3(x)}-\dfrac{2x}{f^2(x)}=0\\ 
		\Rightarrow& 3f^2(x)\cdot f'(x)\cdot\mathrm{e}^{f^3(x)}=2x \\
		\Rightarrow&\left[\mathrm{e}^{f^3(x)}\right]'=2x \\
		\Rightarrow&\mathrm{e}^{f^3(x)}=\displaystyle\int{2x}\mathrm{\,d}x \\
		\Rightarrow&\mathrm{e}^{f^3(x)}=x^2+C.\\ 
		\end{aligned}
		$$
		Mặt khác $f(1)=0\Rightarrow\mathrm{e}^{f^3(1)}=1+C\Rightarrow C=0.$\\
		Suy ra $\mathrm{e}^{f^3(x)}=x^2\Rightarrow{f^3}(x)=\ln {x^2}\Rightarrow f(x)=\sqrt[3]{2\ln x}$.\\
		Vậy $I=\displaystyle\int\limits_0^{2024}\dfrac 1{\sqrt[3]{2\ln x}}\cdot f(x)\mathrm{\,d}x=\displaystyle\int\limits_0^{2024}\dfrac 1{\sqrt[3]{2\ln x}}\cdot \sqrt[3]{2\ln x}\mathrm{\,d}x=\displaystyle\int\limits_0^{2024}\mathrm{\,d}x=2024$}
\end{ex}

\begin{ex}%[2D4C2-2]
	Cho hàm số $f(x)$ đồng biến, có đạo hàm trên đoạn $\left[1;4\right]$ và thoả mãn $x+2x\cdot f(x)=\left[f'(x)\right]^2$ với $\forall x\in\left[1;4\right]$. Biết $f(1)=\dfrac{3}{2}$, tính $I=\displaystyle\int\limits_1^4f(x)\mathrm{\,d}x$.
	\choice
	{\True $I=\dfrac{1186}{45}$}
	{$I=\dfrac{1186}{9}$}
	{$I=\dfrac{1186}{5}$}
	{$I=\dfrac{1186}{41}$}
	\loigiai{
		Do $f(x)$ đồng biến trên đoạn $\left[1;4\right]$ $\Rightarrow f'(x)\ge 0,\forall x\in\left[1;4\right].$\\
		Ta có  $x+2x \cdot f(x)=\left[f'(x)\right]^2
		\Leftrightarrow x\left(1+2\cdot f(x)\right)=\left[f'(x)\right]^2$, \\Do $x\in\left[1;4\right]$ và $f'(x)\ge 0,\forall x\in\left[1;4\right]$
		$\Rightarrow f(x) >\dfrac{-1}{2}$ và
		$$
		\begin{aligned}
			&f'(x)=\sqrt x \cdot \sqrt{1+2f(x)}\\
			\Leftrightarrow&\dfrac{f'(x)}{\sqrt{1+2f(x)}}=\sqrt x\\
			\Leftrightarrow&\left(\sqrt{1+2f(x)}\right)'=\sqrt x \\
			\Leftrightarrow&\sqrt{1+2f(x)}=\displaystyle\int{\sqrt x}\mathrm{\,d}x\\
			\Leftrightarrow&\sqrt{1+2f(x)}=\dfrac{2}{3}x\sqrt x+C.
		\end{aligned}
		$$
		Vì $f(1)=\dfrac{3}{2}\Rightarrow\sqrt{1+2\cdot\dfrac{3}{2}}=\dfrac{2}{3}+C\Leftrightarrow C=\dfrac{4}{3}$.\\
		Suy ra
		$$
		\begin{aligned}
			&\sqrt{1+2f(x)}=\dfrac{2}{3}x\sqrt x+\dfrac{4}{3}\\
			\Leftrightarrow & 1+2f(x)=\left(\dfrac{2}{3}x\sqrt x+\dfrac{4}{3}\right)^2\\
			\Leftrightarrow & f(x)=\dfrac{2}{9}{x^3}+\dfrac{8}{9}{x^{\dfrac{3}{2}}}+\dfrac{7}{18}.
		\end{aligned}
		$$
		Khi đó\\ $I=\displaystyle\int\limits_1^4f(x)\mathrm{\,d}x=\displaystyle\int\limits_1^4\left(\dfrac{2}{9}{x^3}+\dfrac{8}{9}{x^{\tfrac{3}{2}}}+\dfrac{7}{18}\right)\mathrm{\,d}x=\left.\left(\dfrac{1}{18}{x^4}+\dfrac{16}{45}{x^{\tfrac{5}{2}}}+\dfrac{7}{18}x\right)\right|_1^4=\dfrac{1186}{45}$.}
\end{ex}

\begin{ex}%[2D4C2-2]
	Cho hàm số $f(x)$ nhận giá trị dương và thỏa mãn $f(0)=1$, $\left[f'(x)\right]^3=\mathrm{e}^x\left[f(x)\right]^2,\forall x\in\mathbb{R}$.
	Tính $I=\displaystyle\int\limits_1^2f(x)\mathrm{\,d}x$.
	\choice
	{$I=\mathrm{e}^2+1$}
	{$I=\mathrm{e}-1$}
	{\True $I=\mathrm{e}^2-e$}
	{$I=\mathrm{e}$}
	\loigiai{
		Ta có
		$$
		\begin{aligned}
			&\left[f'(x)\right]^3=\mathrm{e}^x\left[f(x)\right]^2\\
		\Leftrightarrow& f'(x)=\sqrt[3]{\mathrm{e}^x}\cdot\sqrt[3]{\left[f(x)\right]^2}\\ 
		\Leftrightarrow&\dfrac{f'(x)}{\sqrt[3]{\left[f(x)\right]^2}}=\sqrt[3]{\mathrm{e}^x}\\
		 \Leftrightarrow&\dfrac{f'(x)}{\sqrt[3]{\left[f(x)\right]^2}}=\sqrt[3]{\mathrm{e}^x}\\
		  \Leftrightarrow &f'(x)\cdot \left[f(x)\right]^{-\tfrac 23}=\sqrt[3]{\mathrm{e}^x}\\ 
		  \Leftrightarrow& 3\left[\left(f(x)\right)^{\tfrac 13}\right]'=\sqrt[3]{\mathrm{e}^x}\\ 
		  \Leftrightarrow&\left[\left(f(x)\right)^{\tfrac 13}\right]'=\dfrac 13\sqrt[3]{\mathrm{e}^x}\\ 
		  \Leftrightarrow&\left[f(x)\right]^{\tfrac 13}=\dfrac 13\displaystyle\int{\sqrt[3]{\mathrm{e}^x}}\mathrm{\,d}x \\
		  \Leftrightarrow&\left[f(x)\right]^{\tfrac 13}=\mathrm{e}^{\tfrac x3}+C.
		\end{aligned}
		$$	
		Mà $f(0)=1\Rightarrow 1=1+C\Rightarrow C=0$.\\
		Do đó $\left[f(x)\right]^{\tfrac{1}{3}}=\mathrm{e}^{\tfrac{x}{3}}\Rightarrow f(x)=\mathrm{e}^x$.\\
		Vậy $I=\displaystyle\int\limits_1^2\mathrm{e}^x\mathrm{\,d}x=\mathrm{e}^2-\mathrm{e}$.}
\end{ex}

\begin{ex}%[2D4C2-2]
	Cho hàm số $y=f(x)$ có đạo hàm liên tục trên $\mathbb{R}$ và thỏa mãn điều kiện ${{x}^6}{{\left[f'(x) \right]}^3}+27{{\left[f(x)-1 \right]}^4}=0\,,\,\forall x\in \mathbb{R}$ và $f(1)=0$. Tính $I=\displaystyle\int\limits_2^3f(x)\mathrm{\,d}x$.
	\choice
	{$I=\dfrac{31}{2}$}
	{$I=-\dfrac{31}{2}$}
	{$I=\dfrac{61}{4}$}
	{\True $I=-\dfrac{61}{4}$}
	\loigiai{
		Ta có
		$$
		\begin{aligned}
		&x^6\left[f'(x)\right]^3+27\left[f(x)-1\right]^4=0\\ 
		\Leftrightarrow&{x^6}{\left[f'(x)\right]^3}=-27\left[f(x)-1\right]^4\\ 
		\Leftrightarrow&\dfrac{\left[f'(x)\right]^3}{\left[f(x)-1\right]^4}=-\dfrac{27}{x^6}\\ 
		\Leftrightarrow&\dfrac{\left[f'(x)\right]^3}{\left[f(x)-1\right]^3\left[f(x)-1\right]}=-\dfrac{27}{x^6}\\ 
		\Leftrightarrow&\dfrac{f'(x)}{\left[f(x)-1\right]\sqrt[3]{f(x)-1}}=-\dfrac 3{x^2}\\
		 \Leftrightarrow&\dfrac{f'(x)}{-3\left[f(x)-1\right]\sqrt[3]{f(x)-1}}=\dfrac 1{x^2}\\ 
		 \Leftrightarrow&{\left[\dfrac 1{\sqrt[3]{f(x)-1}}\right]'}=\dfrac 1{x^2}.\\ 
		\end{aligned}
		$$
		Do đó $\displaystyle\int{\left[\dfrac{1}{\sqrt[3]{f(x)-1}}\right]'}\mathrm{\,d}x=\displaystyle\int{\dfrac{1}{x^2}\mathrm{\,d}x}=-\dfrac{1}{x}+C.$\\
		Suy ra $\dfrac{1}{\sqrt[3]{f(x)-1}}=-\dfrac{1}{x}+C$.\\
		Mà  $f(1)=0\Rightarrow C=0$.\\
		Nên  $f(x)=1-x^3$.\\
		Khi đó $I=\displaystyle\int\limits_2^3f(x)\mathrm{\,d}x=\displaystyle\int\limits_2^3(1-x^3)\mathrm{\,d}x=-\dfrac{61}{4}$.}
\end{ex}
\begin{ex}%[2D4C2-4]
	Cho hàm số $f(x) > 0$ và thỏa mãn $\left[f'(x)\right]^2+f(x)\cdot f''(x)=\mathrm{e}^x$, $\forall x\in \mathbb{R}$ và $f(0)=f'(0)=1$. Tính $I=\displaystyle\int\limits_1^2 f(x) \mathrm{\,d}x$.
	\choice
	{$I=2\sqrt{\mathrm{e}}$}
	{$I=\mathrm{e}-\sqrt{\mathrm{e}}$}
	{\True $I=2\mathrm{e}-2\sqrt{\mathrm{e}}$}
	{$I=2\mathrm{e}+2\sqrt{\mathrm{e}}$}
	\loigiai{
		Ta có
		\allowdisplaybreaks
		\begin{eqnarray*}
			&&\left[f'(x)\right]^2+f(x)\cdot f''(x)=\mathrm{e}^x\\
			&\Leftrightarrow& \left[f(x)\cdot f'(x)\right]'=\mathrm{e}^x\\
			&\Rightarrow& f(x)\cdot f'(x)=\displaystyle\int\limits_{\mathrm{e}}^x \mathrm{e}^x \mathrm{\,d}x\\
			&\Rightarrow& f(x)\cdot f'(x)=\mathrm{e}^x+C.
		\end{eqnarray*}
		Từ $f(0)=f'(0)=1$ ta suy ra $C=0$.\\
		Vậy $f(x)\cdot f'(x)=\mathrm{e}^x$\\
		Tiếp đến có
		\allowdisplaybreaks
		\begin{eqnarray*}
			&&2f(x)\cdot f'(x)=\mathrm{e}^x\\
			&\Leftrightarrow& \left[f^2(x)\right]'=\mathrm{e}^x\\
			&\Rightarrow& f^2(x)=\displaystyle\int\limits_{\mathrm{e}}^x \mathrm{e}^x \mathrm{\,d}x\\
			&\Rightarrow& f^2(x)=\mathrm{e}^x+C
		\end{eqnarray*}
		Từ $f(0)=1$ ta suy ra $C=0$.\\
		Vậy $f^2(x)=\mathrm{e}^x\Rightarrow f(x)=\sqrt{\mathrm{e}^x}$ (do $f(x) > 0$).\\
		Khi đó $I=\displaystyle\int\limits_1^2 f(x) \mathrm{\,d}x = \displaystyle\int\limits_1^2 \sqrt{\mathrm{e}^x}\mathrm{\,d}x = \displaystyle\int\limits_1^2 \mathrm{e}^{\tfrac{x}{2}} \mathrm{\,d}x = \left.2\mathrm{e}^{\tfrac{x}{2}}\right|_1^2 = 2\mathrm{e}-2\sqrt{\mathrm{e}}$.
	}
\end{ex}

\begin{ex}%[2D4C2-2]
	Cho hàm số $f(x)$ thỏa mãn $\left[f'(x)\right]^2+f(x)\cdot f''(x)=2x$, và $f(0)=f'(0)=2$. Tính $I=\displaystyle\int\limits_1^2f^2(x)\mathrm{\,d}x$.
	\choice
	{\True $I=\dfrac{15}{2}$}
	{$I=\dfrac{1}{2}$}
	{$I=\dfrac{19}{2}$}
	{$I=15$}
	\loigiai{
		Ta có $\left[f(x)f'(x)\right]'=\left[f'(x)\right]^2+f(x)f''(x)$.\\
		Do đó theo giả thiết ta được $\left[f(x)f'(x)\right]'=2x$.\\
		Suy ra $f(x)f'(x)=x^2+C$.\\
		Hơn nữa $f(0)=f'(0)=2$ suy ra $C=1$.\\
		$\Rightarrow f(x)f'(x)=x^2+1$.\\
		Tương tự vì $\left[f^2(x)\right]'=2f(x)f'(x)$ nên $\left[f^2(x)\right]'=2\left(x^2+1\right)$.\\
		Suy ra $f^2(x)=\displaystyle\int 2\left(x^2+1\right) \mathrm{\,d}x \Rightarrow f^2(x)=\dfrac{2}{3}{x^3}+2x+C$.\\
		Mặt khác $f(0)=2$ nên  suy ra $C=2$.\\
		$\Rightarrow f^2(x)=\dfrac{2}{3}{x^3}+2x+2$.\\
		Vậy $I=\displaystyle\int\limits_1^2 f^2(x)\mathrm{\,d}x=\displaystyle\int\limits_1^2 \left(\dfrac{2}{3}{x^3}+2x+2\right)\mathrm{\,d}x=\dfrac{15}{2}$.
	}
\end{ex}

\begin{ex}%[2D4C2-2]
	Cho hàm số $f(x)$ thỏa mãn: $\left[f'(x)\right]^2+f(x)\cdot f''(x)=15x^4+12x$, $\forall x\in\mathbb{R}$ và $f(0)=f'(0)=1$. Giá trị của $f^2(1)$ bằng
	\choice
	{$\dfrac{5}{2}$}
	{\True $8$}
	{$10$}
	{$4$}
	\loigiai{
		Theo giả thiết
		\allowdisplaybreaks
		\begin{eqnarray*}
			& & \forall x\in\mathbb{R}\colon \left[f'(x)\right]^2+f(x)\cdot f''(x)=15x^4+12x\\
			&\Leftrightarrow& f'(x)\cdot f'(x)+f(x)\cdot f''(x)=15x^4+12x\\
			&\Leftrightarrow& \left[f(x)\cdot f'(x)\right]'=15x^4+12x\\
			&\Leftrightarrow& f(x)\cdot f'(x)=\displaystyle\int \left(15x^4+12x\right)\mathrm{\,d}x=3x^5+6x^2+C.\quad (1)
		\end{eqnarray*}
		Thay $x=0$ vào $(1)$, ta được $f(0)\cdot f'(0)=C \Leftrightarrow C=1$.\\
		Khi đó $(1)$ trở thành $\begin{aligned}[t]& f(x)\cdot f'(x)=3x^5+6x^2+1\\
			&\Rightarrow \displaystyle\int\limits_0^1 f(x)\cdot f'(x) \mathrm{\,d}x = \displaystyle\int\limits_0^1 \left(3x^5+6x^2+1\right) \mathrm{\,d}x\\
			&\Leftrightarrow \left.\left[\dfrac{1}{2} f^2(x)\right] \right|_0^1 = \left.\left(\dfrac{1}{2}{x^6}+2x^3+x\right) \right|_0^1
			\Leftrightarrow \dfrac{1}{2}\left[f^2(1)-f^2(0)\right]=\dfrac{7}{2} \\
			&\Leftrightarrow f^2(1)-1=7\Leftrightarrow f^2(1)=8.\end{aligned}$\\
		Vậy $f^2(1)=8$.
	}
\end{ex}

\begin{ex}%[2D4C2-2]
	Cho hàm số $y=f(x)$ thỏa mãn $\left[f'(x)\right]^2+f(x)\cdot f''(x)=x^3-2x,\,\forall x\in \mathbb{R}$ và $f(0)=f'(0)=2$. Tính giá trị của $T=f^2(2)$.
	\choice
	{$\dfrac{160}{15}$}
	{\True $\dfrac{268}{15}$}
	{$\dfrac{4}{15}$}
	{$\dfrac{268}{30}$}
	\loigiai{
		Ta có $\left[f'(x)\right]^2+f(x)\cdot f''(x)=x^3-2x,\,\forall x\in \mathbb{R} \Leftrightarrow \left[f'(x)\cdot f(x)\right]'=x^3-2x,\,\forall x\in \mathbb{R}$.\\
		Lấy nguyên hàm hai vế ta có 
		\allowdisplaybreaks
		\begin{eqnarray*}
			& & \displaystyle\int \left[f'(x)\cdot f(x)\right]' \mathrm{\,d}x= \displaystyle\int \left(x^3-2x\right)\mathrm{\,d}x\\ &\Leftrightarrow& f'(x)\cdot f(x)=\dfrac{x^4}{4}-x^2+C.
		\end{eqnarray*}
		Theo đề ra ta có $f(0)\cdot f(0)=C=4$.\\
		Suy ra $\displaystyle\int\limits_0^2 f'(x)\cdot f(x)\mathrm{\,d}x = \displaystyle\int\limits_0^2 \left(\dfrac{x^4}{4}-x^2+4\right)\mathrm{\,d}x \Leftrightarrow \left.\dfrac{f^2(x)}{2}\right|_0^2=\dfrac{104}{15}$ $\Leftrightarrow f^2(2)=\dfrac{268}{15}$.
	}
\end{ex}
\Closesolutionfile{ans}
% \indapan{10}{ans/ans-2-B1}
% \begin{dang}{}
% 	\begin{enumerate}
% 		\item Điều kiện hàm ẩn có dạng: $A(x)f(x)+B(x)f'(x)=h(x)$.\quad$(1)$\\
% 		\textit{Ý tưởng giải:}
% 		\begin{itemize}
% 			\item Ta cần nhân thêm một lượng $u(x)$ vào $(1)$ để tạo thành \break  $u'(x)f(x)+u(x)f'(x)=u(x).h(x)$ và lúc này:
% 			\allowdisplaybreaks
% 			\begin{eqnarray*}
% 				& & u'(x)f(x)+u(x)f'(x)=u(x)\cdot h(x)\\
% 				&\Leftrightarrow & \left[u(x)f(x)\right]'=u(x)\cdot .h(x)\\
% 				&\Rightarrow& \displaystyle\int \left[u(x)f(x)\right]'\mathrm{\,d}x= \displaystyle\int u(x)\cdot h(x)\mathrm{\,d}x\\
% 				&\Rightarrow& u(x)f(x)=\displaystyle\int u(x)\cdot h(x)\mathrm{\,d}x\\
% 				&\Rightarrow& f(x)=\dfrac{\displaystyle\int u(x)\cdot h(x)\mathrm{\,d}x}{u(x)}
% 			\end{eqnarray*}
% 			\item Cách tìm $u(x)$\\
% 			$u(x)$ được chọn sao cho: $\heva{&u'(x)=A(x)\\&u(x)=B(x)}$\\
% 			$\Rightarrow \dfrac{u'(x)}{u(x)}=\dfrac{A(x)}{B(x)} \Rightarrow \displaystyle\int \dfrac{u'(x)}{u(x)}\mathrm{\,d}x =\displaystyle\int\dfrac{A(x)}{B(x)}\mathrm{\,d}x$\\ $\Rightarrow \ln \left|u(x)\right|=\displaystyle\int \dfrac{A(x)}{B(x)}\mathrm{\,d}x \Rightarrow u(x)=\mathrm{e}^{\displaystyle\int \dfrac{A(x)}{B(x)}\mathrm{\,d}x}$.\\
% 		\end{itemize}
% 		\textbf{Tóm lại phương pháp giải:} $A(x)f(x)+B(x)f'(x)=h(x)$ $(1)$ như sau:
% 		\begin{itemize}
% 			\item Tìm $u(x)$: $u(x)=\mathrm{e}^{\displaystyle\int \dfrac{A(x)}{B(x)} \mathrm{\,d}x}$.
% 			\item Nhân $u(x)$ vào $(1)$ $\Rightarrow f(x)=\dfrac{\displaystyle\int{u(x)\cdot h(x)} \mathrm{\,d}x}{u(x)}$. 
% 		\end{itemize}
% 		\item Một số dạng đặc biệt của $(1)$
% 		\begin{enumerate}
% 			\item Điều kiện hàm ẩn có dạng: $\heva{&f'(x)+f(x)=h(x)\\&f'(x)-f(x)=h(x).}$\\
% 			\textbf{Phương pháp giải}
% 			\begin{itemize}
% 				\item $f'(x)+f(x)=h(x)$.\\
% 				Nhân hai vế với $\mathrm e^x$ ta được $$\mathrm e^x\cdot f'(x)+\mathrm e^x\cdot f(x)=\mathrm e^x\cdot h(x)\Leftrightarrow \left[\mathrm e^x\cdot f(x)\right]'=\mathrm e^x\cdot h(x).$$
% 				Suy ra $\mathrm e^x\cdot f(x)=\displaystyle\int \mathrm e^x\cdot h(x) \mathrm{\,d}x$.\\
% 				Từ đây ta dễ dàng tính được $f(x)$.
% 				\item $f'(x)-f(x)=h(x)$.\\
% 				Nhân hai vế với $\mathrm e^{-x}$ ta được $$\mathrm e^{-x}\cdot f'(x)-\mathrm e^{-x}\cdot f(x)=\mathrm e^{-x}\cdot h(x)\Leftrightarrow \left[e^{-x}\cdot f(x)\right]'=\mathrm e^{-x}\cdot h(x).$$
% 				Suy ra $\mathrm e^{-x}\cdot f(x)=\displaystyle\int \mathrm e^{-x}\cdot h(x) \mathrm{\,d}x$.\\
% 				Từ đây ta dễ dàng tính được $f(x)$.
% 			\end{itemize}
% 			\item Điều kiện hàm ẩn có dạng: $f'(x)+p(x)\cdot f(x)=h(x)$.\\
% 			\textbf{Phương pháp giải}\\
% 			Nhân hai vế với $\mathrm e^{\displaystyle\int p (x)\mathrm{\,d}x}$ ta được
% 			\allowdisplaybreaks
% 			\begin{eqnarray*}
% 				& & f'(x)\cdot \mathrm e^{\displaystyle\int p(x)\mathrm{\,d}x}+p(x)\cdot \mathrm e^{\displaystyle\int p (x)dx}\cdot f(x)=h(x)\cdot{\mathrm e^{\displaystyle\int p (x)dx}}\\
% 				&\Leftrightarrow& \left[f(x)\cdot{e^{\displaystyle\int p (x)dx}}\right]'=h(x)\cdot \mathrm e^{\displaystyle\int p (x)\mathrm{\,d} x}.
% 			\end{eqnarray*}		
% 			Suy ra $f(x)\cdot \mathrm e^{\displaystyle\int p(x)\mathrm{\,d}x}=\displaystyle\int \mathrm e^{\displaystyle\int p (x)\mathrm{\,d}x}h(x) \mathrm{\,d}x$.\\
% 			Từ đây ta dễ dàng tính được $f(x)$.
% 		\end{enumerate}
% 	\end{enumerate}
% \end{dang}
\Opensolutionfile{ans}[ans/ans-2-B1-D2]
% \TN
\begin{ex}%[2D4C2-4]
	Cho hàm số $f(x)$ thỏa mãn $f(x)+f'(x)=\mathrm{e}^{-x}$, $\forall x\in\mathbb{R}$ và $f(0)=2$. Tính $I=\displaystyle\int\limits_1^2 \dfrac{f(x) \mathrm{e}^x}{x}\mathrm{\,d}x$.
	\choice
	{$I=2\ln 2$}
	{$I=\ln 2$}
	{$I=1+\ln 2$}
	{\True $I=1+2\ln 2$}
	\loigiai{
		Ta có
		\allowdisplaybreaks
		\begin{eqnarray*}
			& & f(x)+f'(x)=\mathrm{e}^{-x}\\
			&\Leftrightarrow& f(x) \mathrm{e}^x+f'(x)\mathrm{e}^x=1\\
			&\Leftrightarrow& \left[f(x) \mathrm{e}^x\right]'=1\\
			&\Rightarrow& f(x)\mathrm{e}^x=\displaystyle\int x \mathrm{\,d}x\\
			&\Leftrightarrow& f(x) \mathrm{e}^x=x+C.
		\end{eqnarray*}
		Vì $f(0)=2$ nên $C=2$.\\
		$\Rightarrow f(x)\mathrm{e}^x=x+2$.\\
		Vậy 
		$I=\displaystyle\int\limits_1^2 \dfrac{f(x) \mathrm{e}^x}{x} \mathrm{\,d}x = \displaystyle\int\limits_1^2 \dfrac{x+2}{x}\mathrm{\,d}x=\displaystyle\int\limits_1^2 \left(1+\dfrac{2}{x}\right)\mathrm{\,d}x= \left(x+2\ln | x|\right)\bigg|_1^2=1+2\ln 2$.
	}
\end{ex}

\begin{ex}%[2D4C2-4]
	Cho hàm số $f(x)$ có đạo hàm trên $\mathbb{R}$ thỏa mãn $\left(x+2\right)f(x)+\left(x+1\right)f'(x)=\mathrm{e}^x$ và $f(0)=\dfrac{1}{2}$. Tính $I=\displaystyle\int\limits_1^2 \left(2x+2\right)f(x)\mathrm{\,d}x$.
	\choice
	{$I=\mathrm{e}^2$}
	{$I=1+\mathrm{e}$}
	{$I=1+\mathrm{e}^2$}
	{\True $I=\mathrm{e}^2-\mathrm{e}$}
	\loigiai{
		Ta có
		\allowdisplaybreaks
		\begin{eqnarray*}
			& & \left(x+2\right)f(x)+\left(x+1\right)f'(x)=\mathrm{e}^x\\
			&\Leftrightarrow& \left(x+1\right)f(x)+f(x)+\left(x+1\right)f'(x)=\mathrm{e}^x\\
			&\Leftrightarrow& \left[\left(x+1\right)f(x)\right]+\left[\left(x+1\right)f(x)\right]'=\mathrm{e}^x\\
			&\Leftrightarrow& \mathrm{e}^x\left[\left(x+1\right)f(x)\right]+\mathrm{e}^x\left[\left(x+1\right)f(x)\right]'=\mathrm{e}^{2x}\\
			&\Leftrightarrow& \left[\mathrm{e}^x\left(x+1\right)f(x)\right]'=\mathrm{e}^{2x}\\
			&\Rightarrow& \displaystyle\int \left[\mathrm{e}^x\left(x+1\right)f(x)\right]'\mathrm{\,d}x=\displaystyle\int \mathrm{e}^{2x}\mathrm{\,d}x\\
			&\Leftrightarrow& \mathrm{e}^x\left(x+1\right)f(x)=\dfrac{1}{2}{\mathrm{e}^{2x}}+C.
		\end{eqnarray*}
		Mà $f(0)=\dfrac{1}{2}$ $\Rightarrow C=0$.\\
		Vậy $f(x)=\dfrac{1}{2}\cdot \dfrac{\mathrm{e}^x}{x+1}$.\\
		Do đó 
		$I=\displaystyle\int\limits_1^2 \left(2x+2\right)\dfrac{1}{2}\cdot \dfrac{\mathrm{e}^x}{x+1}\mathrm{\,d}x=\displaystyle\int\limits_1^2 \mathrm{e}^x\mathrm{\,d}x=\mathrm e^2-\mathrm e$.
	}
\end{ex}

\begin{ex}%[2D4C2-3]
	Cho hàm số $y=f(x)$ liên tục, có đạo hàm trên $\mathbb{R}$ thỏa mãn điều kiện \break $f(x)+x\left[f'(x)-2\sin x\right]=x^2\cos x$, $x\in \mathbb{R}$ và $f\left(\dfrac{\pi}{2}\right)=\dfrac{\pi}{2}$. Tính $I=\displaystyle\int\limits_0^{\tfrac{\pi}{2}} \dfrac{f(x)}{x}\mathrm{\,d}x$.
	\choice
	{\True $I=1$}
	{$I=\dfrac{\pi}{2}$}
	{$I=-1$}
	{$I=-\pi$}
	\loigiai{
		Từ giả thiết $\begin{aligned}[t] &f(x)+x\left(f'(x)-2\sin x\right)=x^2\cos x\\
			&\Leftrightarrow f(x)+xf'(x)=x^2\cos x+2x\sin x\\
			&\Leftrightarrow \left(xf(x)\right)'=\left(x^2\sin x\right)'\\
			&\Leftrightarrow xf(x)=x^2\sin x+C.\end{aligned}$\\
		Mặt khác $f\left(\dfrac{\pi}{2}\right)=\dfrac{\pi}{2}\Rightarrow C=0\Rightarrow f(x)=x\sin x$.\\
		Vậy
		$I=\displaystyle\int\limits_0^{\tfrac{\pi}{2}}{\dfrac{f(x)}{x}\mathrm{\,d}x}=\displaystyle\int\limits_0^{\tfrac{\pi}{2}}{\dfrac{x\sin x}{x}\mathrm{\,d}x}=\displaystyle\int\limits_0^{\tfrac{\pi}{2}}{\sin x\mathrm{\,d}x}=1$.
	}
\end{ex}

\begin{ex}%[2D4C2-4]
	Cho hàm số $y=f(x)$ có đạo hàm trên $(0;+\infty)$ thỏa mãn $2xf'(x)+f(x)=2x$, $\forall x\in(0;+\infty)$, $f(1)=1$. Giá trị của biểu thức $f(4)$ là
	\choice
	{$\dfrac{25}{6}$}
	{$\dfrac{25}{3}$}
	{\True $\dfrac{17}{6}$}
	{$\dfrac{17}{3}$}
	\loigiai{
		Xét phương trình $2xf'(x)+f(x)=2x$ $(1)$ trên $(0;+\infty)$ ta có $$(1)\Leftrightarrow f'(x)+\dfrac{1}{2x}\cdot f(x)=1.\quad(2)$$
		Đặt $g(x)=\dfrac{1}{2x}$, ta tìm một nguyên hàm $G(x)$ của $g(x)$.\\
		Ta có $\displaystyle\int g(x)\mathrm{\,d}x=\displaystyle\int \dfrac{1}{2x}\mathrm{\,d}x=\dfrac{1}{2}\ln x+C=\ln \sqrt x+C$. Ta chọn $G(x)=\ln \sqrt x $.\\
		Nhân cả 2 vế của $(2)$ cho $\mathrm{e}^{G(x)}=\sqrt x$, ta được 
		$$\sqrt x\cdot f'(x)+\dfrac{1}{2\sqrt x}\cdot f(x)=\sqrt x \Leftrightarrow \left[\sqrt x \cdot f(x)\right]'=\sqrt x. \quad(3)$$
		Lấy tích phân 2 vế của $(3)$ từ $1$ đến $4$, ta được \\
		$\displaystyle\int\limits_1^4 \left[\sqrt x \cdot f(x)\right]'\mathrm{\,d}x= \displaystyle\int\limits_1^4 \sqrt x\mathrm{\,d}x$ $\Rightarrow \left[\sqrt x \cdot f(x)\right]\bigg|_1^4=\left.\left(\dfrac{2}{3}\sqrt{x^3}\right)\right|_1^4\Rightarrow 2f(4)-f(1)=\dfrac{14}{3}$\\
		$\Rightarrow f(4)=\dfrac{1}{2}\left(\dfrac{14}{3}+1\right)=\dfrac{17}{6}$ (vì $f(1)=1$).\\
		Vậy $f(4)=\dfrac{17}{6}$.
	}
\end{ex}

\begin{ex}%[2D4C2-2]
	Cho hàm số $f(x)$ không âm, có đạo hàm trên đoạn $[0;1]$ và thỏa mãn $f(1)=1$, $\left[2f(x)+1-x^2\right]f'(x)=2x\left[1+f(x)\right]$, $\forall x\in[0;1]$. Tích phân $\displaystyle\int\limits_0^1 f(x)\mathrm{\,d}x$ bằng
	\choice
	{$1$}
	{$2$}
	{\True $\dfrac{1}{3}$}
	{$\dfrac{3}{2}$}
	\loigiai{
		Xét trên đoạn $[0;1]$, theo đề bài ta có
		\allowdisplaybreaks
		\begin{eqnarray*}
			&&\left[2f(x)+1-x^2\right]f'(x)=2x\left[1+f(x)\right]\\
			&\Leftrightarrow& 2f(x)\cdot f'(x)=2x+\left(x^2-1\right)\cdot f'(x)+2x\cdot f(x)\\
			&\Leftrightarrow& \left[f^2(x)\right]'=\left[x^2+\left(x^2-1\right)\cdot f(x)\right]'\\
			&\Leftrightarrow& f^2(x)=x^2+\left(x^2-1\right)\cdot f(x)+C.\quad (1)
		\end{eqnarray*}
		Thay $x=1$ vào $(1)$ ta được $f^2(1)=1+C\Leftrightarrow C=0$ (vì $f(1)=1$).\\
		Do đó, $(1)$ trở thành
		\allowdisplaybreaks
		\begin{eqnarray*} &&f^2(x)=x^2+\left(x^2-1\right)\cdot f(x)\\
			&\Leftrightarrow& f^2(x)-1=x^2-1+\left(x^2-1\right)\cdot f(x)\\
			&\Leftrightarrow& \left[f(x)-1\right]\cdot \left[f(x)+1\right]=\left(x^2-1\right)\cdot \left[f(x)+1\right]\\
			&\Leftrightarrow& f(x)-1=x^2-1 ~(\text{vì } f(x)\ge 0\Rightarrow f(x)+1 > 0,\,\forall x\in[0;1])\\
			&\Leftrightarrow& f(x)=x^2.
		\end{eqnarray*}
		Vậy $\displaystyle\int\limits_0^1 f(x)\mathrm{\,d}x=\displaystyle\int\limits_0^1 x^2\mathrm{\,d}x=\left.\dfrac{x^3}{3}\right|_0^1=\dfrac{1}{3}$.
	}
\end{ex}

\begin{ex}%[2D4C2-2]
	Cho hàm số $y=f(x)$ có đạo hàm liên tục trên $[0;1]$, thỏa mãn \break  $\left[f'(x)\right]^2+4f(x)=8x^2+4,\,\forall x\in[0;1]$ và $f(1)=2$. Tính $\displaystyle\int\limits_0^1 f(x) \mathrm{\,d}x$.
	\choice
	{$\dfrac{1}{3}$}
	{$2$}
	{\True $\dfrac{4}{3}$}
	{$\dfrac{21}{4}$}
	\loigiai{
		Ta có
		\allowdisplaybreaks
		\begin{eqnarray*}
			& & \left[f'(x)\right]^2+4f(x)=8x^2+4\\
			&\Rightarrow& \displaystyle\int\limits_0^1\left[f'(x)\right]^2\mathrm{\,d}x+4\displaystyle\int\limits_0^1 f(x)\mathrm{\,d}x =\displaystyle\int\limits_0^1\left(8x^2+4\right)\mathrm{\,d}x=\dfrac{20}{3}.\quad (1)	
		\end{eqnarray*}
		Và 
		\allowdisplaybreaks
		\begin{eqnarray*}
			&&\displaystyle\int\limits_0^1 xf'(x)\mathrm{\,d}x=xf(x)\big|_0^1-\displaystyle\int\limits_0^1 f(x)\mathrm{\,d}x=2-\displaystyle\int\limits_0^1 f(x)\mathrm{\,d}x\\
			&\Rightarrow&-4\displaystyle\int\limits_0^1 xf'(x)\mathrm{\,d}x=-8+4\displaystyle\int\limits_0^1 f(x)\mathrm{\,d}x.\quad (2)
		\end{eqnarray*}
		Lại có $$\displaystyle\int\limits_0^1\left(2x\right)^2\rm{d}x=\dfrac{4}{3}.\quad (3)$$
		Cộng vế với vế của (1), (2), (3) ta được $$\displaystyle\int\limits_0^1\left(f'(x)-2x\right)^2\mathrm{\,d}x=0\Rightarrow f'(x)=2x\Rightarrow f(x)=x^2+C.$$
		Mặt khác $f(1)=C+1=2\Rightarrow C=1\Rightarrow f(x)=x^2+1$.\\
		Do đó $\displaystyle\int\limits_0^1 f(x)\mathrm{\,d}x=\displaystyle\int\limits_0^1 \left(x^2+1\right)\mathrm{\,d}x=\dfrac{4}{3}$.
	}
\end{ex}

\begin{ex}%[2D4C2-2]
	Cho hàm số $y=f(x)$ có đạo hàm liên tục trên $[0;1]$ thỏa mãn \break $3f(x)+xf'(x)\ge x^{2018}$, $\forall x\in[0;1]$. Tìm giá trị nhỏ nhất của $\displaystyle\int_0^1 f(x)\mathrm{\,d}x$.
	\choice
	{$\dfrac{1}{2018\cdot2020}$}
	{$\dfrac{1}{2019\cdot2020}$}
	{$\dfrac{1}{2020\cdot2021}$}
	{\True $\dfrac{1}{2019\cdot2021}$}
	\loigiai{
		Ta có
		\allowdisplaybreaks
		\begin{eqnarray*}
			& & 3f(x)+xf'(x)\ge{x^{2018}},\, \forall x\in[0;1]\\
			&\Leftrightarrow& 3x^2f(x)+x^3\cdot f'(x)\ge x^{2020},\, \forall x\in[0;1]\\
			&\Leftrightarrow& \left[x^3f(x)\right]'\ge x^{2020}, \, \forall x\in[0;1]\\
			&\Rightarrow& x^3f(x)\ge\displaystyle\int x^{2020}\mathrm{\,d}x,\, \forall x\in[0;1]\\
			&\Rightarrow& x^3f(x)\ge\dfrac{x^{2021}}{2021}+C,\, \forall x\in[0;1].
		\end{eqnarray*}
		Cho $x=0\Rightarrow C=0\Rightarrow x^3f(x)\ge\dfrac{x^{2021}}{2021}$, $\forall x\in[0;1]$ $\Rightarrow f(x)\ge\dfrac{x^{2018}}{2021},\,\forall x\in[0;1]$.\\
		$\Rightarrow\displaystyle\int_0^1 f(x) \mathrm{\,d}x\ge\displaystyle\int_0^1 \dfrac{x^{2018}}{2021}\mathrm{\,d}x=\left.\left(\dfrac{x^{2019}}{2019\cdot2021}\right)\right|_0^1=\dfrac{1}{2019\cdot2021}$.
	}
\end{ex}
\Closesolutionfile{ans}
% \indapan{10}{ans/ans-2-B1-D2}
% \begin{dang}{MỘT SỐ DẠNG KHÁC}
% \end{dang}
\Opensolutionfile{ans}[ans/ans-2-B1-D3]
% \TNSA
\begin{ex}%[2D4C2-2]
	Cho hàm số $y=f(x)$ có đạo hàm trên $\mathbb{R}$ thỏa mãn $$\heva{&f(0)=f'(0)=1\\&		f(x+y)=f(x)+f(y)+3xy(x+y)-1} \text{ với }x,y\in\mathbb{R}$$
	Tính $\displaystyle\int\limits_0^1 f(x-1)\mathrm{\,d}x$.
	\choice
	{$\dfrac{1}{2}$}
	{$-\dfrac{1}{4}$}
	{\True $\dfrac{1}{4}$}
	{$\dfrac{7}{4}$}
	\loigiai{
		Lấy đạo hàm theo hàm số $y$ ta được $f'(x+y)=f'(y)+3x^2+6xy$, $\forall x\in\mathbb{R}$.\\
		Cho $y=0\Rightarrow f'(x)=f'(0)+3x^2\Rightarrow f'(x)=1+3x^2$\\
		$\Rightarrow f(x)=\displaystyle\int f'(x)\mathrm{\,d}x=x^3+x+C$ mà $f(0)=1 \Rightarrow C=1$.\\
		Do đó $f(x)=x^3+x+1\Rightarrow f(x-1)=(x-1)^3+x-1+1=x^3-3x^2+4x-1$.\\
		Vậy $\displaystyle\int\limits_0^1 f(x-1)\mathrm{\,d}x=\displaystyle\int\limits_0^1 \left(x^3-3x^2+4x-1\right)\mathrm{\,d}x=\dfrac{1}{4} \displaystyle\int\limits_{-1}^0 f(x)\mathrm{\,d}x= \displaystyle\int\limits_{-1}^0 \left(x^3+x+1\right)\mathrm{\,d}x=\dfrac{1}{4}$.
	}
\end{ex}

\begin{ex}%[2D4C2-2]
	Cho hai hàm $f(x)$ và $g(x)$ có đạo hàm trên $[1;4]$, thỏa mãn $\heva{&f(1)+g(1)=4\\&g(x)=-xf'(x)\\&f(x)=-xg'(x)}$, 
	với mọi $x\in[1;4]$. Tính tích phân $I=\displaystyle\int\limits_1^4\left[f(x)+g(x)\right]\mathrm{\,d}x$.
	\choice
	{$3\ln 2$}
	{$4\ln 2$}
	{$6\ln 2$}
	{\True $8\ln 2$}
	\loigiai{
		Từ giả thiết ta có 
		\allowdisplaybreaks
		\begin{eqnarray*}
			&& f(x)+g(x)=-x\cdot f'(x)-x\cdot g'(x)\\
			&\Leftrightarrow& \left[f(x)+x\cdot f'(x)\right]+\left[g(x)+x\cdot g'(x)\right]=0\\ &\Leftrightarrow& \left[x\cdot f(x)\right]'+\left[x\cdot g(x)\right]'=0\\
			&\Rightarrow& x\cdot f(x)+x\cdot g(x)=C\\
			&\Rightarrow& f(x)+g(x)=\dfrac{C}{x}
		\end{eqnarray*}
		Mà $f(1)+g(1)=4\Rightarrow C=4\Rightarrow f(x)+g(x)=\dfrac{4}{x}$.\\
		Vậy $I=\displaystyle\int\limits_1^4 \left[f(x)+g(x)\right]\mathrm{\,d}x=\displaystyle\int\limits_1^4\dfrac{4}{x}\mathrm{\,d}x=8\ln 2$.
	}
\end{ex}
\begin{ex}%[2D4C2-5]
	Cho hai hàm $f(x)$ và $g(x)$ có đạo hàm trên $\left[1;2\right]$ thỏa mãn $f(1)=g(1)=0$ và $\heva{& \dfrac{x}{(x+1)^2}g(x)+2023x=(x+1)f'(x) \\ & \dfrac{x^3}{x+1}g'(x)+f(x)=2024x^2}\,,\forall x\in \left[1;2\right]$. \\
	Tính tích phân $I=\displaystyle\int\limits_1^2 \left[\dfrac{x}{x+1}g(x)-\dfrac{x+1}{x}f(x) \right]\mathrm{\,d}x$.
	\choice
	{\True $I=\dfrac{1}{2}$}
	{$I=1$} 
	{$I=\dfrac{3}{2}$}
	{$I=2$}
	\loigiai{
		Từ giả thiết ta có $\heva{& \dfrac{1}{(x+1)^2}g(x)-\dfrac{x+1}{x}f'(x)=-2023\\ & \dfrac{x}{x+1}g'(x)+\dfrac{1}{x^2}f(x)=2024}\,,\forall x\in \left[1;2\right]$.\\
		Suy ra
		\allowdisplaybreaks 
		\begin{eqnarray*}
			&& \left[\dfrac{1}{(x+1)^2}g(x)+\dfrac{x}{x+1}g'(x) \right]-\left[\dfrac{x+1}{x}f'(x)-\dfrac{1}{x^2}f(x) \right]=1\\  
			&\Leftrightarrow& \left[\dfrac{x}{x+1}g(x) \right]'-\left[\dfrac{x+1}{x}f(x) \right]'=1\\ 
			&\Rightarrow& \dfrac{x}{x+1}g(x)-\dfrac{x+1}{x}f(x)=x+C.
		\end{eqnarray*}
		Mà $f(1)=g(1)=0\Rightarrow C=-1 \Rightarrow \dfrac{x}{x+1}g(x)-\dfrac{x+1}{x}f(x)=x-1$.\\
		Vậy $I=\displaystyle\int\limits_1^2 \left[\dfrac{x}{x+1}g(x)-\dfrac{x+1}{x}f(x) \right]\mathrm{\,d}x=\displaystyle\int\limits_1^2 (x-1)\mathrm{\,d}x=\dfrac{1}{2}$.
	}
\end{ex}

\begin{ex}%[2D4C2-5]
	Cho hàm số $f\left(x \right)$ xác định và liên tục trên $\mathbb{R}\setminus \left\{0\right\}$ thỏa mãn $x^2f^2\left(x \right)+\left(2x-1\right)f\left(x \right)=xf'\left(x \right)-1$, với mọi $x\in \mathbb{R}\setminus \left\{0\right\}$ đồng thời thỏa mãn $f\left(1\right)=-2$. Tính $\displaystyle\int\limits_1^2 f\left(x \right)\mathrm{\,d}x$.
	\choice
	{$-\dfrac{\ln 2}{2}-1$}
	{\True $-\ln 2-\dfrac{1}{2}$}
	{$-\ln 2-\dfrac{3}{2}$}
	{$-\dfrac{\ln 2}{2}-\dfrac{3}{2}$}
	\loigiai{
		Ta có 
		\allowdisplaybreaks 
		\begin{eqnarray*}
			&& x^2f^2\left(x \right)+2xf\left(x \right)+1=xf'\left(x \right)+f\left(x \right) \\ 
			&\Leftrightarrow& \left(xf\left(x \right)+1\right)^2=\left(xf\left(x \right)+1\right)'.
		\end{eqnarray*}
		Do đó
		\allowdisplaybreaks 
		\begin{eqnarray*}
			&& \dfrac{\left(xf\left(x \right)+1\right)'}{\left(xf\left(x \right)+1\right)^2}=1\\ 
			&\Rightarrow& \displaystyle\int \dfrac{\left(xf\left(x \right)+1\right)'}{\left(xf\left(x \right)+1\right)^2}\mathrm{\,d}x=\displaystyle\int 1\mathrm{\,d}x\\
			&\Rightarrow& -\dfrac{1}{xf\left(x \right)+1}=x+C\\
			&\Rightarrow& xf\left(x \right)+1=-\dfrac{1}{x+C}.
		\end{eqnarray*}
		Mặt khác $f\left(1\right)=-2$ nên $-2+1=-\dfrac{1}{1+C}\Rightarrow C=0$.\\
		Nên suy ra $xf\left(x \right)+1=-\dfrac{1}{x}\Rightarrow f\left(x \right)=-\dfrac{1}{x^2}-\dfrac{1}{x}$.\\
		Vậy $\displaystyle\int\limits_1^2 f\left(x \right)\mathrm{\,d}x=\displaystyle\int\limits_1^2 \left(-\dfrac{1}{x^2}-\dfrac{1}{x} \right)\mathrm{\,d}x=\left.\left(-\ln x+\dfrac{1}{x} \right)\right|_1^2=-\ln 2-\dfrac{1}{2}$.
	}
\end{ex}

\begin{ex}%[2D4C2-5]
	Cho hàm số $y=f(x)$ có đạo hàm liên tục trên $\mathbb{R}$ thỏa mãn $x\cdot f(x)\cdot f'(x)=f^2(x)-x,\,\forall x\in \mathbb{R}$ và có $f(2)=1$. Tích phân $\displaystyle\int\limits_0^2 f^2(x)\mathrm{\,d}x$ bằng
	\choice
	{$\dfrac{3}{2}$}
	{$\dfrac{4}{3}$}
	{\True $2$}
	{$4$}
	\loigiai{
		Ta có
		\allowdisplaybreaks 
		\begin{eqnarray*}
			x\cdot f(x)\cdot f'(x)=f^2(x)-x &\Leftrightarrow& 2x\cdot f(x)\cdot f'(x)=2f^2(x)-2x \\
			&\Leftrightarrow& 2x\cdot f(x)\cdot f'(x)+f^2(x)=3f^2(x)-2x \\ 
			&\Leftrightarrow& \displaystyle\int\limits_0^2 \left(x\cdot f^2(x) \right)'\mathrm{\,d}x=3\displaystyle\int\limits_0^2 f^2(x)\mathrm{\,d}x-\displaystyle\int\limits_0^2 2x\mathrm{\,d}x \\ 
			&\Leftrightarrow& \left.\left(x\cdot f^2(x) \right)\right|_0^2 =3I-4\\ 
			&\Leftrightarrow& 2=3I-4\\ 
			&\Leftrightarrow& I=2.
		\end{eqnarray*}
	}
\end{ex}

\begin{ex}%[2D4C2-5]
	Cho hàm số $f\left(x \right)$ có đạo hàm liên tục trên $\mathbb{R}$, $f\left(0\right)=0$, $f'\left(0\right)\ne 0$ và thỏa mãn hệ thức $f\left(x \right)\cdot f'\left(x \right)+18x^2=\left(3x^2+x \right)f'\left(x \right)+\left(6x+1\right)f\left(x \right),\,\forall x \in \mathbb{R}$. Biết $\displaystyle\int\limits_0^1 \left(x+1\right)\mathrm{e}^{f\left(x \right)}\mathrm{\,d}x=a\mathrm{e}^2+b,\,\left(a,b\in \mathbb{Q} \right)$. Giá trị của $a-b$ bằng
	\choice
	{\True $1$}
	{$2$}
	{$0$}
	{$\dfrac{2}{3}$}
	\loigiai{
		Ta có $f\left(x \right)\cdot f'\left(x \right)+18x^2=\left(3x^2+x \right)f'\left(x \right)+\left(6x+1\right)f\left(x \right)$.\\
		Lấy nguyên hàm hai vế ta được 
		\allowdisplaybreaks 
		\begin{eqnarray*}
			\dfrac{f^2\left(x \right)}{2}+6x^3=\left(3x^2+x \right)f\left(x \right)
			&\Rightarrow& f^2\left(x \right)-2\left(3x^2+x \right)f\left(x \right)+12x^3=0\\
			&\Rightarrow& \hoac{& f\left(x \right)=6x^2 \\ & f\left(x \right)=2x.}
		\end{eqnarray*}
		\begin{enumerate}[\bf TH1:]
			\item $f\left(x \right)=6x^2$ không thoả mãn kết quả $\displaystyle\int\limits_0^1 \left(x+1\right)\mathrm{e}^{f\left(x \right)}\mathrm{\,d}x=a\mathrm{e}^2+b,\,\left(a,b\in \mathbb{Q} \right)$.
			\item $f\left(x \right)=2x \Rightarrow \displaystyle\int\limits_0^1 \left(x+1\right)\mathrm{e}^{f\left(x \right)}\mathrm{\,d}x= \displaystyle\int\limits_0^1 \left(x+1\right)\mathrm{e}^{2x}\mathrm{\,d}x=\dfrac{3}{4}\mathrm{e}^2-\dfrac{1}{4}$.\\ 
			Suy ra $a=\dfrac{3}{4};b=-\dfrac{1}{4}$.
		\end{enumerate}
		Vậy $a-b=1$.
	}
\end{ex}

\begin{ex}%[2D4C2-5]
	Cho hàm số $y=f(x)$ xác định và có đạo hàm $f'\left(x \right)$ liên tục trên $[1;3]$; $f\left(x \right)\ne 0,\,\forall x\in \left[1;3\right]$; $f'\left(x \right)\left[1+f\left(x \right) \right]^2=\left(x-1\right)^2\left[f\left(x \right) \right]^4$ và $f\left(1\right)=-1$. Biết rằng $\displaystyle\int\limits_{\mathrm{e}}^3 f\left(x \right)\mathrm{\,d}x=a\ln 3+b\,\left(a,b\in \mathbb{Z} \right)$. Giá trị của $a+b^2$ bằng
	\choice
	{$4$}
	{\True $0$}
	{$2$}
	{$-1$}
	\loigiai{
		Ta có 
		\allowdisplaybreaks 
		\begin{eqnarray*}
			f'(x)\left[1+f(x)\right]^2=(x-1)^2\left[f(x)\right]^4
			&\Rightarrow& \dfrac{f'(x)}{f^4(x)}+\dfrac{2f'(x)}{f^3(x)}+\dfrac{f'(x)}{f^2(x)}=(x-1)^2\\
			&\Rightarrow& \displaystyle\int \left(\dfrac{f'(x)}{f^4(x)}+\dfrac{2f'(x)}{f^3(x)}+\dfrac{f'(x)}{f^2(x)} \right) \mathrm{\,d}x=\displaystyle\int (x-1)^2\mathrm{\,d}x\\
			&\Rightarrow& -\left(\dfrac{1}{3f^3(x)}+\dfrac{1}{f^2(x)}+\dfrac{1}{f(x)} \right)=\dfrac{1}{3}(x-1)^3+C. \quad (*)
		\end{eqnarray*}
		Do $f(1)=-1$ nên $C=\dfrac{1}{3}$.\\ 
		Thay vào $(*)$ ta được $\left(\dfrac{1}{f(x)}+1\right)^3=-(x-1)^3 \Rightarrow f(x)=\dfrac{-1}{x}$.\\
		Khi đó $\displaystyle\int\limits_{\mathrm{e}}^3 \dfrac{-1}{x}\mathrm{\,d}x=\left.-\ln \left|x \right|\right|_{\mathrm{e}}^3=-\ln 3+1\Rightarrow a=-1,b=1$.\\ 
		Vậy $a+b^2=0$.\\
	}
\end{ex}
\Closesolutionfile{ans}
% \indapan{10}{ans/ans-2-B1-D3}
