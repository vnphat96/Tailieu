\begin{ex} %Cau 56D  %[2H5H1-4]
Trong KG $Oxyz$, điểm $M \left(0;a;0\right)$ thuộc trục $Oy$ và cách đều hai mặt phẳng: $\left(P\right) \colon x+y-z+1=0$ và $\left(Q\right) \colon x-y+z-5=0$. Khi đó $a$ có giá trị bằng
\shortans{$-3$}
\loigiai{
Ta có $M \in Oy \Rightarrow M\left(0;a;0\right)$.\\
Theo giả thiết: $\mathrm{d} \left(M,\left(P\right)\right) = \mathrm{d} \left(M,\left(Q\right)\right) \Leftrightarrow \dfrac{\vert a+1 \vert}{\sqrt{3}} = \dfrac{\vert -a-5 \vert}{\sqrt{3}} \Leftrightarrow a = -3$.\\
Vậy $a = -3$ thì thỏa mãn đề bài.
}
\end{ex}

\begin{ex} %Cau 57D %[2H5V1-5]
Trong không gian với hệ trục tọa độ $Oxy$, cho $A \left(1;2;3\right)$, $B \left(3;4;4\right)$. Khi đó giá trị của tham số $m$ bằng bao nhiêu để khoảng cách từ điểm $A$ đến mặt phẳng $\left(P\right)\colon 2x + y + mz -1=0$ bằng độ dài đoạn thẳng $AB$.
\shortans{$2$}
\loigiai{
Ta có $\overrightarrow{AB} = \left(2;2;1\right) \Rightarrow AB = \sqrt{2^2+2^2+1^2} = 3$ \quad(1)\\
Khoảng cách từ điểm $A$ đến mặt phẳng $\left(P\right)$:\\
$\mathrm{d} \left(A;\left(P\right)\right) = \dfrac{\vert 2 \cdot 1 + 2 + m \cdot 3 -1 \vert}{\sqrt{2^2+1^2+m^2}} = \dfrac{\vert 3m+3 \vert}{\sqrt{5+m^2}}$ \quad(2).\\
Để $AB = \mathrm{d} \left(A;\left(P\right)\right) \Rightarrow 3 = \dfrac{\vert 3m+3 \vert}{\sqrt{5+m^2}} \Leftrightarrow 9 \left(5+m^2\right)= 9 \left(m+1\right)^2 \Leftrightarrow m =2$.
}
\end{ex}

\begin{ex} %Cau 58D %[2H5H1-5]
Gọi điểm $M \left(0;a;0\right)$ trên trục $Oy$ sao cho khoảng cách từ điểm $M$ đến mặt phẳng $\left(P\right) \colon 2x-y+3z-4=0$ nhỏ nhất. Khi đó giá trị của $a$ là
\shortans{$-4$}
\loigiai{
Khoảng cách từ $M$ đến $\left(P\right)$ nhỏ nhất khi $M$ thuộc $\left(P\right)$. Nên $M$ là giao điểm của trục $Oy$ với mặt phẳng $\left(P\right)$.\\
Thay $x=0$, $z=0$ vào phương trình ta được $y = -4$. Khi đó $M \left(0;-4;0\right)$\\
Vậy giá trị của $a = -4$.
}
\end{ex}

\begin{ex} %Cau 59D %[2H5H1-5]
Cho điểm $M \left(0;0;m\right)$ thuộc trục $Oz$ sao cho điểm $M$ cách đều điểm $A \left(2;3;4\right)$ và mặt phẳng $\left(P\right) \colon 2x+3y+z-17=0$. Khi đó giá trị của $m$ là
\shortans{$3$}
\loigiai{
Ta có $MA = \sqrt{2^2+3^2+\left(4-m\right)^2}$; $\mathrm{d} \left(M,\left(P\right)\right) = \dfrac{\vert m-17 \vert}{\sqrt{14}}$.\\
$M$ cách đều điểm $A \left(2;3;4\right)$ và mặt phẳng $\left(P\right) \colon 2x+3y+z-17=0$ khi và chỉ khi\\
$$\sqrt{2^2+3^2+\left(4-m\right)^2} = \dfrac{\vert m-17 \vert}{\sqrt{14}} \Leftrightarrow 13 \left(m-3\right)^2 = 0 \Leftrightarrow m=3$$
Vậy $m=3$.
}
\end{ex}

\begin{ex} %Cau 60D %[2H5V1-5]
Trong không gian với hệ trục tọa độ $Oxyz$, cho hai điểm $A \left(1;2;3\right)$, $B\left(5;-4;-1\right)$ và mặt phẳng $\left(P\right)$ qua $Ox$ sao cho $\mathrm{d} \left(B;\left(P\right)\right) = 2\mathrm{d} \left(A;\left(P\right)\right)$, $\left(P\right)$ cắt $AB$ tại $I\left(a;b;c\right)$ nằm giữa $AB$. Tính $a+b+c$.
\shortans{$4$}
\loigiai{
Vì $\mathrm{d} \left(B;\left(P\right)\right) = 2\mathrm{d} \left(A;\left(P\right)\right)$ và $\left(P\right)$ cắt đoạn $AB$ tại $I$ nên\\
$\overrightarrow{BI} = -2 \overrightarrow{AI} \Leftrightarrow \heva{&a-5 = -2\left(a-1\right)\\&b+4 = -2\left(b-2\right)\\&c+1 = -2\left(c-3\right)} \Leftrightarrow \heva{&a=\dfrac{7}{3}\\&b=0\\&c=\dfrac{5}{3}} \Rightarrow a+b+c = 4$.
}
\end{ex}

\begin{ex} %Cau 61D %[2H5V1-5]
Trong KG $Oxyz$, cho mặt phẳng $\left(P\right) \colon 3x+4y-12z+5=0$ và điểm $A \left(2;4;-1\right)$. Trên mặt phẳng $\left(P\right)$ lấy điểm $M$. Gọi $B$ là điểm sao cho $\overrightarrow{AB} = 3\cdot \overrightarrow{AM}$. Tính khoảng cách $\mathrm{d}$ từ $B$ đến mặt phẳng $\left(P\right)$
\shortans{$6$}
\loigiai{Ta có: $\overrightarrow{AB} = 3 \cdot \overrightarrow{AM} \Rightarrow BM=2\cdot AM \Rightarrow \dfrac{\mathrm{d} \left(B,\left(P\right)\right)}{\mathrm{d} \left(A,\left(P\right)\right)} = \dfrac{BM}{AM} = 2$
\immini{
\begin{eqnarray*}
&\Rightarrow \mathrm{d} \left(B,\left(P\right)\right) &= 2 \cdot \mathrm{d} \left(A,\left(P\right)\right)\\
& &= 2 \cdot \dfrac{\vert 3 \cdot 2 + 4 \cdot 4 -12 \cdot \left(-1\right)+5\vert}{\sqrt{3^2+4^2+\left(-12\right)^2}}\\
& & = 2 \cdot 3 = 6
\end{eqnarray*}
Vậy $\mathrm{d} \left(B,\left(P\right)\right) = 6$.}{
\begin{tikzpicture}[>=stealth,line join=round, line cap=round, scale=0.7]
        \coordinate (A) at (1,3);
		\coordinate (B) at (-1,0);
		\coordinate (C) at (5,0);
		\coordinate (D) at (7,3);
        \coordinate (H) at (1.5,2);
        \coordinate (K) at (4.5,2);
        \coordinate (M) at (4.5,-1.5);
        \coordinate (N) at (1.5,4.5);
        \path[name path=d1] (N)--(M);
        \path[name path=d2] (H)--(K);
        \path[name path=d3] (B)--(C);
        \path[name intersections={of=d1 and d2,by=l}];
        \path[name intersections={of=d1 and d3,by=z}];
  \draw (A)--(B)--(C)--(D)--(A); \draw[dashed] (K)--(4.5,0); \draw (4.5,0)--(M)--(z);
  \draw (l) node[below left]{$M$} circle (1pt)--(N) node[above]{$A$}  circle (1pt)--(H) node[below left]{$H$}  circle (1pt)--(K) node[below right]{$K$} circle (1pt); \draw (M) node[below]{$B$} circle (1pt); \draw [dashed] (l)--(z);
\begin{scope}
\clip (A)--(B)--(C);
\draw (B) circle (1.1);
\draw (-0.8,0) node[above right]{$P$} ;
\end{scope}
\end{tikzpicture}
}}
\end{ex}

\begin{ex} %Cau 62D %[2H5H1-4]
Trong KG $Oxyz$, cho hai mặt phẳng $\left(P\right) \colon 2x+my+2mz-9=0$ và $\left(Q\right) \colon 6x-y-z-10=0$. Tìm $m$ để $\left(P\right) \perp \left(Q\right)$
\shortans{$4$}
\loigiai{
$\left(P\right) \colon 2x+my+2mz-9=0$ có véc-tơ pháp tuyến là $\overrightarrow{a} = \left(2;m;2m\right)$\\
$\left(Q\right) \colon 6x-y-z-10=0$ có véc-tơ pháp tuyến là $\overrightarrow{b} = \left(6;-1;-1\right)$\\
Khi đó $\left(P\right) \perp \left(Q\right) \Leftrightarrow \overrightarrow{a} \cdot \overrightarrow{b} =0 \Leftrightarrow 2 \cdot 6 + m \cdot \left(-1\right)+2m \cdot \left(-1\right) =0 \Leftrightarrow m=4$.
}
\end{ex}

\begin{ex} %Cau 63D %[2H5H1-4]
Trong KG $Oxyz$, cho hai mặt phẳng $\left(P\right) \colon 5x+my+z-5=0$ và $\left(Q\right) \colon nx-3y-2z+7=0$. Để $\left(P\right) \parallel \left(Q\right)$ thì giá trị của $m+n$ là (làm tròn đến chữ số thập phân thứ nhất)
\shortans{$-8{,}5$}
\loigiai{
$\left(P\right) \colon 5x+my+z-5=0$ có véc-tơ pháp tuyến là $\overrightarrow{a} = \left(5;m;1\right)$\\
$\left(Q\right) \colon nx-3y-2z+7=0$ có véc-tơ pháp tuyến là $\overrightarrow{b} = \left(n;-3;-2\right)$\\
Để $\left(P\right) \parallel \left(Q\right) \Leftrightarrow \left[a;b\right] = \overrightarrow{0} \Leftrightarrow \heva{&-2m+3=0\\&n+10=0\\&-15-mn=0} \Leftrightarrow \heva{&m=\dfrac{3}{2}\\&n=-10}$\\
Khi đó $m+n = \dfrac{3}{2} + \left(-10\right)= -8,5$.
}
\end{ex}

\begin{ex} %Cau 64D %[2H5V1-4]
Trong KG $Oxyz$, cho hai mặt phẳng $\left(P\right) \colon 2x-my-4z-6+m=0$ và $\left(Q\right) \colon \left(m+3\right)x +y+\left(5m+1\right)z -7=0$. Tìm $m$ để $\left(P\right) \equiv \left(Q\right)$.
\shortans{$-1$}
\loigiai{
$\left(P\right) \colon 2x-my-4z-6+m=0$ có véc-tơ pháp tuyến là $\overrightarrow{a} = \left(2;-m;-4\right)$\\
$\left(Q\right) \colon \left(m+3\right)x +y+\left(5m+1\right)z -7=0$ có véc-tơ pháp tuyến là $\overrightarrow{b} = \left(m+3;1;5m+1\right)$\\
Khi đó với $m \neq -3$, $m \neq -\dfrac{1}{5}$ ta có $\left(P\right) \equiv \left(Q\right) \Leftrightarrow \dfrac{2}{m+3} = \dfrac{-m}{1} = \dfrac{-4}{5m+1} \Leftrightarrow m = -1$.
}
\end{ex}

\begin{ex} %Cau 65D %[2H5H1-3]
Trong KG $Oxyz$, cho hai mặt phẳng $\left(P\right) \colon x-2y-z+3=0$ và $\left(Q\right) \colon 2x +y+z -1=0$. Mặt phẳng $\left(R\right)$ đi qua điểm $M\left(1;1;1\right)$ chứa giao tuyến của $\left(P\right)$ và $\left(Q\right)$; phương trình của $\left(R\right) \colon m\left(x-2y-z+3\right) + \left(2x+y+z-1\right)=0$. Khi đó giá trị của $m$ là bao nhiêu?
\shortans{$-3$}
\loigiai{
Vì $\left(R\right) \colon m\left(x-2y-z+3\right) + \left(2x+y+z-1\right)=0$ đi qua điểm $M \left(1;1;1\right)$ nên ta có:\\
$m\left(1-2 \cdot 1-1+3\right) + \left(2 \cdot 1+1+1-1\right)=0 \Leftrightarrow m = -3$\\
Vậy $m = -3$.
}
\end{ex}

\begin{ex} %Cau 66D %[2H5V1-4]
Trong KG $Oxyz$, cho $3$ điểm $A\left(1;0;0\right)$,$B\left(0;b;0\right)$,$C\left(0;0;c\right)$ trong đó $b \cdot c \neq 0$ và mặt phẳng $\left(P\right) \colon y-z+1=0$. Giá trị của $\dfrac{2b}{c}$ bằng bao nhiêu để mặt phẳng $\left(ABC\right)$ vuông góc với mặt phẳng $\left(P\right)$.
\shortans{$2$}
\loigiai{
Phương trình mặt phẳng $\left(ABC\right) \colon \dfrac{x}{1}+\dfrac{y}{b}+\dfrac{z}{c}=1$ có véc-tơ pháp tuyến là $\overrightarrow{n} = \left(1;\dfrac{1}{b};\dfrac{1}{c}\right)$.\\
Phương trình mặt phẳng $\left(P\right) \colon y-z+1=0$ có véc-tơ pháp tuyến là $\overrightarrow{n'} = \left(0;1;-1\right)$.\\
Do đó $\left(ABC\right) \perp \left(P\right) \Leftrightarrow \overrightarrow{n} \cdot \overrightarrow{n'} = 0 \Leftrightarrow \dfrac{1}{b}-\dfrac{1}{c} = 0 \Leftrightarrow b = c$.\\
Vậy $\dfrac{2b}{c} = 2$.
}
\end{ex}

\begin{ex} %Cau 67D %[2H5V1-3]
Trong KG $Oxyz$, cho mặt phẳng $\left(\alpha\right) \colon ax-y+2z+b=0$ đi qua giao tuyến của hai mặt phẳng $\left(P\right) \colon x-y-z+1=0$ và $\left(Q\right) \colon x+2y+z-1=0$. Tính $a+4b$
\shortans{$-16$}
\loigiai{
Trên giao tuyến $\Delta$ của hai mặt phẳng $\left(P\right)$, $\left(Q\right)$ ta lấy lần lượt hai điểm $A$, $B$ như sau\\
Lấy $A \left(x;y;1\right) \in \Delta$, ta có hệ phương trình $\heva{&x-y=0\\&x+2y=0} \Rightarrow x=y=0 \Rightarrow A \left(0;0;1\right)$.\\
Lấy $B \left(-1;y;z\right) \in \Delta$, ta có hệ phương trình $\heva{&y+z=0\\&2y+z=0} \Rightarrow \heva{&y=2\\&z=2} \Rightarrow B \left(-1;2;-2\right)$.\\
Vì $\Delta \subset \left(\alpha\right)$ nên $A$, $B \in \left(\alpha\right)$. Do đó ta có: $\heva{&2+b=0\\&-a+b-6=0} \Rightarrow \heva{&a=-8\\&b=-2}$.\\
Vậy $a+4b = -8 + 2 \cdot \left(-2\right) = -16$.
}
\end{ex}

\begin{ex} %Cau 68D %[2H5V1-4]
Gọi $m$, $n$ là hai giá trị thực thỏa mãn giao tuyến của hai mặt phẳng $\left(P_m\right) \colon mx+2y+nz+1=0$ và $\left(Q_m\right) \colon x-my + nz + 2=0$ vuông góc với mặt phẳng $\left(\alpha\right) \colon 4x -y -6z +3=0$. Tính $m+n$
\shortans{$3$}
\loigiai{
\begin{description}
\item[+] $\left(P_m\right) \colon mx+2y+nz+1=0$ có véc-tơ pháp tuyến $\overrightarrow{n}_1 = \left(m;2;n\right)$
\item[+] $\left(Q_m\right) \colon x-my+nz+2=0$ có véc-tơ pháp tuyến $\overrightarrow{n}_2 = \left(1;-m;n\right)$
\item[+] $\left(\alpha \right) \colon 4x-y-6z+3=0$ có véc-tơ pháp tuyến $\overrightarrow{n}_{\alpha} = \left(4;-1;-6\right)$.
\item[+] Giao tuyến của hai mặt phẳng $\left(P_m\right)$ và $\left(Q_m\right)$ vuông góc với mặt phẳng $\left(\alpha\right)$ nên
$$\heva{&\left(P_m\right) \perp \left(\alpha\right)\\&\left(Q_m\right) \perp \left(\alpha\right)} \Leftrightarrow \heva{&\overrightarrow{n}_1 \perp \overrightarrow{n}_{\alpha}\\&\overrightarrow{n}_2 \perp \overrightarrow{n}_{\alpha}} \Leftrightarrow \heva{&\overrightarrow{n}_1 \cdot \overrightarrow{n}_{\alpha} = 0\\&\overrightarrow{n}_2 \cdot \overrightarrow{n}_{\alpha} = 0} \Leftrightarrow \heva{&4m-2-6n=0\\&4+m-6n=0} \Leftrightarrow \heva{&m=2\\&n=1}$$
\end{description}
Vậy $m+n = 3$.
}
\end{ex}

\begin{ex} %Cau 69D %[2H5V1-4]
Trong KG $Oxyz$ có bao nhiêu mặt phẳng song song với mặt phẳng $\left(Q\right) \colon x+y+z+3=0$, cách điểm $M \left(3;2;1\right)$ một khoảng bằng $3\sqrt{3}$ biết rằng tồn tại một điểm $X\left(a;b;c\right)$ trên mặt phẳng đó, khi đó $a+b+c$ có giá trị bằng
\shortans{$15$}
\loigiai{
Ta có mặt phẳng cần tìm là $\left(P\right) \colon x+y+z+d=0$ với $d \neq 3$.\\
Mặt phẳng $\left(P\right)$ cách điểm $M \left(3;2;1\right)$ một khoảng bằng $3\sqrt{3}$ nên\\
$\mathrm{d} \left(M,\left(P\right)\right) = \dfrac{\vert 6+d \vert}{\sqrt{3}} = 3\sqrt{3} \Leftrightarrow \hoac{&d=3 \quad(L)\\&d=-15} \Rightarrow d = -15$.\\
Suy ra $\left(P\right) \colon x+y+z-15=0$.\\
Theo giả thiết $X\left(a;b;c\right) \in \left(P\right) \Leftrightarrow a+b+c = 15$.
}
\end{ex}

\begin{ex} %Cau 70D %[2H5C1-4]
Biết rằng Trong KG $Oxyz$ có hai mặt phẳng $\left(P\right)$ và $\left(Q\right)$ cùng thỏa mãn các điều kiện sau: đi qua hai điểm $A\left(1;1;1\right)$ và $B\left(0;-2;2\right)$, đồng thời cắt các trục tọa độ $Ox$, $Oy$ tại hai điểm cách đều $O$. Giả sử $\left(P\right) \colon x+b_{1}y+c_{1}z+d_1=0$ và $\left(Q\right) \colon x+b_2 y+c_2 z+d_2=0$. Tính giá trị biểu thức $b_1b_2 + c_1c_2$
\shortans{$-9$}
\loigiai{
$\textbf{Cách 1}$\\
Xét mặt phẳng $\left(\alpha\right) \colon x+by+cz+d=0$ thỏa mãn các điều kiện: đi qua hai điểm $A \left(1;1;1\right)$ và $B\left(0;-2;2\right)$, đồng thời cắt các trục tọa độ $Ox$, $Oy$ tại hai điểm cách đều $O$.\\
Vì $\left(\alpha\right)$ đi qua $A \left(1;1;1\right)$ và $B \left(0;-2;2\right)$ nên ta có hệ phương trình:
$$\heva{&1+b+c+d=0\\&-2b+2c+d=0} \quad(*)$$
Mặt phẳng $\left(\alpha\right)$ cắt các trục tọa độ $Ox$, $Oy$ lần lượt tại $M \left(-d;0;0\right)$, $N \left(0;\dfrac{-d}{c};0\right)$.\\
Vì $M$, $N$ cách đều $O$ nên $OM = ON$. Suy ra: $\vert d \vert = \left\vert \dfrac{d}{b} \right\vert$.\\
Nếu $d=0$ thì chỉ tồn tại duy nhất một mặt phẳng thỏa mãn yêu cầu bài toán (mặt phẳng này sẽ đi qua điểm $O$).\\
Do đó để tồn tại hai mặt phẳng thỏa mãn yêu cầu bài toán thì $\vert d \vert = \left\vert \dfrac{d}{b} \right\vert \Leftrightarrow b = \pm 1$.
\begin{description}
\item[$\bullet$] Với $b=1$, $\left(*\right) \Leftrightarrow \heva{&c+d=-2\\&2c+d=2} \Leftrightarrow \heva{&c=4\\&d=-6}$. Ta được mặt phẳng $\left(P\right) \colon x+y+4z-6=0$.
\item[$\bullet$] Với $b=-1$, $\left(*\right) \Leftrightarrow \heva{&c+d=0\\&2c+d=-2} \Leftrightarrow \heva{&c=-2\\&d=2}$. Ta có mặt phẳng $\left(P\right) \colon x-y-2z+2=0$.
\end{description}
Vậy $b_1b_2 + c_1c_2 = 1 \cdot \left(-1\right) + 4 \cdot \left(-2\right) = -9$.\\
$\textbf{Cách 2}$\\
Ta có $\overrightarrow{AB} = \left(-1;-3;1\right)$.\\
Xét mặt phẳng $\left(\alpha\right) \colon x+by+cz+d=0$ thõa mãn các điều kiện: đi qua hai điểm $A \left(1;1;1\right)$ và $B\left(0;-2;2\right)$, đồng thời cắt các trục tọa độ $Ox$, $Oy$ tại hai điểm cách đều $O$ lần lượt tại $M$, $N$. Vì $M$, $N$ cách đều $O$ nên ta có hai trường hợp sau
\begin{description}
\item[TH1] $M \left(a;0;0\right)$, $N \left(0;a;0\right)$ với $a \neq 0$ khi đó $\left(\alpha\right)$ chính là $\left(P\right)$. Ta có $\overrightarrow{MN} = \left(-a;a;0\right)$, chọn $\overrightarrow{u}_1 = \left(-1;1;0\right)$ là một véc-tơ cùng phương với $\overrightarrow{MN}$.\\
Khi đó $\overrightarrow{n}_P = \left[\overrightarrow{AB},\overrightarrow{u}_1\right] = \left(-1;-1;-4\right)$
suy ra $\left(P\right) \colon x+y+4z+d_1 = 0$.
\item[TH2] $M \left(-a;0;0\right)$, $N \left(0;a;0\right)$ với $a \neq 0$ khi đó $\left(\alpha\right)$ chính là $\left(Q\right)$. Ta có $\overrightarrow{MN} = \left(a;a;0\right)$, chọn $\overrightarrow{u}_2 = \left(1;1;0\right)$ là một véc-tơ cùng phương với $\overrightarrow{MN}$.\\
Khi đó $\overrightarrow{n}_Q = \left[\overrightarrow{AB},\overrightarrow{u}_2\right] = \left(-1;1;2\right)$
suy ra $\left(Q\right) \colon x-y-2z+d_2 = 0$.
\end{description}
Vậy $b_1b_2 + c_1c_2 = 1 \cdot \left(-1\right) + 4 \cdot \left(-2\right) = -9$.
}
\end{ex}
%-------------HetCD1---------------
\Closesolutionfile{ans}
% \indapan{2}{ans/ans2C5B1CD1-D2-KQ}