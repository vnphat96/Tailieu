%Câu 70
\begin{ex}%[2H5H1-3]
	Trong không gian với hệ trục tọa độ $Oxyz$, cho ba điểm $A(a;0;0)$, $B(0;b;0)$, $C(0;0;c)$ với $a$, $b$, $c$ là ba số thực dương thay đổi, thỏa mãn điều kiện $\dfrac{1}{a}+\dfrac{1}{b}+\dfrac{1}{c}=2017$. Khi đó, mặt phẳng $(ABC)$ luôn đi qua một điểm cố định có tọa độ là $M(m;m;m)$. Tính giá trị $P=2017m+2$.
	\shortans[\kindSA]{$3$}	
	\loigiai{
	Phương trình mặt phẳng đi qua ba điểm $A(a;0;0)$, $B(0;b;0)$, $C(0;0;c)$ có dạng  $$(ABC) \colon \dfrac{x}{a}+\dfrac{y}{b}+\dfrac{z}{c}=1.$$
	Giả sử $M(m;m;m)$ là một điểm cố định nằm trên $(ABC)$. Khi đó ta có $$ M \in (ABC) \Leftrightarrow \dfrac{m}{a}+\dfrac{m}{b}+\dfrac{m}{c}=1 \Leftrightarrow m \left(\dfrac{1}{a}+\dfrac{1}{b}+\dfrac{1}{c}\right)=1 \Leftrightarrow m \cdot 2017 =1 \Leftrightarrow m =\dfrac{1}{2017}.$$
	Vậy $P=2017m+2=2017 \cdot \dfrac{1}{2017}+2=3$.
	}
\end{ex}
%Câu 71
\begin{ex}%[2H5V1-3]
	Trong không gian với hệ trục tọa độ $Oxyz$, cho ba điểm $M(1;2;5)$. Tính số mặt phẳng $(\alpha)$ đi qua $M$ và cắt các trục $Ox$, $Oy$, $Oz$ lần lượt tại $A$, $B$, $C$ sao cho $OA=OB=OC \neq 0$.
	\shortans[\kindSA]{$4$}	
	\loigiai{
Gọi $A(a;0;0)$, $B(0;b;b)$, $C(0;0;c)$ lần lượt là giao điểm của mặt phẳng $(\alpha)$ với các trục $Ox$, $Oy$ và $Oz$ (với $abc \neq 0$). \\
Khi đó $(\alpha) \equiv (ABC) \colon \dfrac{x}{a}+\dfrac{y}{b}+\dfrac{z}{c}=1$.\\
Ta có $OA=\sqrt{a^2+0^2+0^2}=|a|$. Tương tự $OB=|b|$, $OC=|c|$.\\
Vì $OA=OB=OC$ nên $\heva{&OA=OB\\ &OC=OB} \Leftrightarrow \heva{&|a|=|b|\\&|c|=|b|} \Leftrightarrow \heva{&a= \pm b \\ &c = \pm b.}$
\begin{itemize}
	\item Trường hợp $1$: $a=b$, $c=b$. \\
	Khi đó $\dfrac{x}{b}+\dfrac{y}{b}+\dfrac{z}{b}=1$ mà $M(1;2;5) \in (ABC)$ nên $\dfrac{1}{b}+\dfrac{2}{b}+\dfrac{5}{b}=1 \Leftrightarrow b=8.$
	\item Trường hợp $2$: $a=b$, $c=-b$. \\
	Khi đó $\dfrac{x}{b}+\dfrac{y}{b}-\dfrac{z}{b}=1$ mà $M(1;2;5) \in (ABC)$ nên $\dfrac{1}{b}+\dfrac{2}{b}-\dfrac{5}{b}=1 \Leftrightarrow b=-2.$
	\item Trường hợp $3$: $a=-b$, $c=b$. \\
	Khi đó $\dfrac{x}{b}-\dfrac{y}{b}+\dfrac{z}{b}=1$ mà $M(1;2;5) \in (ABC)$ nên $\dfrac{1}{b}-\dfrac{2}{b}+\dfrac{5}{b}=1 \Leftrightarrow b=4.$
	\item Trường hợp $4$: $a=-b$, $c=-b$. \\
	Khi đó $\dfrac{x}{b}-\dfrac{y}{b}+\dfrac{z}{b}=1$ mà $M(1;2;5) \in (ABC)$ nên $\dfrac{1}{b}-\dfrac{2}{b}-\dfrac{5}{b}=1 \Leftrightarrow b=-6.$
\end{itemize}
Vậy có bốn mặt phẳng $(\alpha)$ thỏa yêu cầu bài toán.
	}
\end{ex}
\begin{ex}%[2H5V1-3]
Trong không gian với hệ trục tọa độ $Oxyz$, có bao nhiêu mặt phẳng $(P)$ đi qua ba điểm $M(2;1;3)$, $A(0;0;4)$ và cắt hai trục $Ox$, $Oy$ lần lượt tại $B$, $C$ khác $O$ thỏa mãn diện tích tam giác $OBC$ bằng $1$?	
	\shortans[\kindSA]{$2$}	
\loigiai{
	Gọi $B(b;0;0)$ và $C(0;c;0)$ lần lượt là giao điểm của $(P)$ với các trục $Ox$, $Oy$.\\
	Khi đó ta có phương trình mặt phẳng $(P) \colon \dfrac{x}{b}+\dfrac{y}{c}+\dfrac{z}{4}=1$.\\
	Vì $M(2;1;3) \in (P)$ nên ta có $\dfrac{2}{b}+\dfrac{1}{c}+\dfrac{3}{4}=1 \Leftrightarrow \dfrac{2}{b}+\dfrac{1}{c} = \dfrac{1}{4} \Leftrightarrow 4b+8c=bc$. \quad (1)\\
	Diện tích tam giác $OBC$ bằng $1$ nên $\dfrac{1}{2} \cdot OB \cdot OC =1 \Leftrightarrow |b| \cdot |c|=2 \Leftrightarrow |bc|=2.$ \quad (2)\\
	Từ (1) và (2), ta có hệ phương trình $\heva{&4b+8c=bc\\&|bc|=2.} \quad (I)$\\
\begin{itemize}
	\item Xét trường hợp $bc>0$. \\
	Khi đó
	$$(I) \Leftrightarrow \heva{&4b+8c=bc\\&bc=2} \Leftrightarrow \heva{&4a+8b=2\\&2bc=4} \Leftrightarrow \heva{&2b=1-4c\\ &(1-4c)c=4}
	\Leftrightarrow \heva{&2b=1-4c\\ &4c^2-c+4=0  \; (\text{pt vô nghiệm}).}$$
	\item Xét trường hợp $bc<0$. \\
	Khi đó
	\begin{align*}
		(I) \Leftrightarrow \heva{&4b+8c=bc\\&bc=-2} \Leftrightarrow \heva{&4a+8b=2\\&2bc=-4} &\Leftrightarrow  \heva{&2b=1-4c\\ &(1-4c)c=-4}\\
		&\Leftrightarrow \heva{&2b=1-4c\\ &4c^2-c-4=0.} \\
		&\Leftrightarrow \heva {&2b=1-4c\\&c=\dfrac{1 \pm \sqrt{65}}{2}}\\
		&\Leftrightarrow \heva{&c=\dfrac{1 + \sqrt{65}}{2} \\ &b=\dfrac{-1-2\sqrt{65}}{2}} \; \text{hay} \;\heva{&c=\dfrac{1 - \sqrt{65}}{2} \\ &b=\dfrac{-1+2\sqrt{65}}{2}.} 
	\end{align*}
\end{itemize}
Vậy có $2$ cặp số $(b;c)$ thỏa yêu cầu bài toán nên có $2$ mặt phẳng $(P)$ thỏa yêu cầu bài tán.
}
\end{ex}
\Closesolutionfile{ans}
\indapan{6}{ans/ans-2-C5B1CD2-D4}