\chude{ỨNG DỤNG MẶT PHẲNG TRONG KHÔNG GIAN}
\newcommand{\gv}[4][black]{\draw[#1,thick] ($(#3)!8pt!(#2)$)--($(#3)!2!($($(#3)!8pt!(#2)$)!.5!($(#3)!8pt!(#4)$)$)$)--($(#3)!8pt!(#4)$);}
\begin{longtable}{|>{\raggedright\arraybackslash}p{5.2cm}|>{\raggedright\arraybackslash}p{5.4cm}|>{\raggedright\arraybackslash}p{5.7cm}|}
		\hline
	    \multicolumn{3}{|>{\centering\arraybackslash}p{16.5cm}|}{\textbf{I. Gắn trục tọa độ đối với hình chóp}} \\ \hline    
	    \multicolumn{3}{|>{\centering\arraybackslash}p{16.5cm}|}{\textbf{1. Hình chóp có cạnh bên (SA) vuông góc với mặt đáy}} \\ \hline                                                                                                                                                                               
		\multicolumn{1}{|>{\raggedright\arraybackslash}p{5.2cm}|}{\begin{tabular}[l]{>{\raggedright\arraybackslash}p{5.2cm}} \textbf{Đáy là tam giác đều}
							\begin{tikzpicture}[>=stealth,font=\footnotesize]
							\def\a{3}
							\def\b{2}
							\def\h{2}
							\path (0:0) coordinate (A)
							++(0:\a) coordinate (C)
							++(-130:\b) coordinate (B)
							($(A)+(90:\h)$) coordinate (S)
							($(B)!1/2!(C)$) coordinate (O)
							($(O)+(90:3.5)$) coordinate (O1)
							($(S)+(O)-(A)$) coordinate (H);
							\draw[dashed,thick] (A)--(C);
							\draw[thick] (S)--(A)--(B)--(C)--(S)--(B);
							\draw[dashed,thick](A)--(O);
							\draw[thick](S)--(H);
							%Ve truc Ox,Oy, Oz
							\draw[thick,->](C)--($(O)!2!(C)$) node [pos=0.9,above ]{$x$};
							\draw[thick,->](A)--($(O)!1.2!(A)$) node [pos=0.9,above right]{$y$};
							\draw[thick,->](O)--(O1) node [pos=0.9,above right]{$z$};
							%Các góc vuông
							\gv{S}{H}{O}
							\gv{C}{O}{H}
							\gv{A}{O}{H}
							\gv{O}{A}{S}
							\foreach \x/\g in {A/-90,B/0,C/0,S/180,O/-10,H/-10}
							\fill[black] (\x) circle (1pt) ($(\g:4mm)+(\x)$) node {$\x$};	
						\end{tikzpicture}
				 
				 - Gọi $O$ là trung điểm $BC$. Chọn hệ trục tọa độ như hình vẽ, $AB=a=1$.
				
				- Tọa độ các điểm là:
						
						$O(0;0;0)$, $A \left(0;\dfrac{\sqrt{3}}{2};0\right)$, $B \left(\dfrac{-1}{2};0;0\right)$, $C \left(\dfrac{1}{2};0;0\right)$, $S \left(0;\dfrac{\sqrt{3}}{2};\underbrace {OH}_{ = SA}\right)$.
				\end{tabular}} &\multicolumn{1}{l|}{\begin{tabular}[l]{>{\raggedright\arraybackslash}p{5.2cm}}\textbf{Đáy là tam giác cân tại A}
				
					\begin{tikzpicture}[>=stealth,font=\footnotesize]
						\def\a{3.5}
						\def\b{2.5}
						\def\h{3.5}
						\path (0:0) coordinate (B)
						++(0:\a) coordinate (C)
						++(-150:\b) coordinate (A)
						($(B)!1/2!(C)$) coordinate (O)
						($(A)+(90:\h)$) coordinate (S)
						($(O)+(90:3.8)$) coordinate (O1)
						($(S)+(O)-(A)$) coordinate (H);
						\draw[dashed,thick] (B)--(C);
						\draw[thick] (S)--(B)--(A)--(C)--(S)--(A);
						\draw[dashed,thick](A)--(O);
						\draw[thick](S)--(H);
						%Ve truc Ox,Oy, Oz
						\draw[thick,->](C)--($(O)!1.4!(C)$) node [pos=0.9,below]{$x$};
						\draw[thick,->](A)--($(O)!1.4!(A)$) node [pos=0.9,below left]{$y$};
						\draw[thick,->](O)--(O1) node [pos=0.9,above right]{$z$};
						%Các góc vuông
						\gv{B}{A}{S}
						\gv{A}{O}{C}
						\foreach \x/\g in {A/-20,B/120,C/-50,S/180,O/40,H/180}
						\fill[black] (\x) circle (1pt) ($(\g:4mm)+(\x)$) node {$\x$};	
					\end{tikzpicture}
					
			- Gọi $O$ là trung điểm $BC$. Chọn hệ trục tọa độ như hình vẽ, $a=1$.
			
			- Tọa độ các điểm là:
				
				$O(0;0;0)$, $A \left(0;OA;0\right)$, $B \left(-OB;0;0\right)$, $C \left(OC;0;0\right)$, $S \left(0;OA;\underbrace {OH}_{ = SA}\right)$.
			\end{tabular}} & \begin{tabular}[l]{>{\raggedright\arraybackslash}p{5.6cm}}\textbf{ Đáy là tam giác cân tại B}
			
				\begin{tikzpicture}[>=stealth,font=\footnotesize]
					\def\a{3}
					\def\b{2}
					\def\h{2}
					\path (0:0) coordinate (A)
					++(0:\a) coordinate (C)
					++(-150:\b) coordinate (B)
					($(A)!1/2!(C)$) coordinate (O)
					($(A)+(90:\h)$) coordinate (S)
					($(O)+(90:2.5)$) coordinate (O1)
					($(S)+(O)-(A)$) coordinate (H);
					\draw[dashed,thick] (A)--(C);
					\draw[thick] (S)--(A)--(B)--(C)--(S)--(B);
					\draw[dashed,thick](B)--(O) ;
					\draw[thick](S)--(H);
					%Ve truc Ox,Oy, Oz
					\draw[thick,->](C)--($(O)!1.4!(C)$) node [pos=0.9,below]{$x$};
					\draw[thick,->](B)--($(O)!1.2!(B)$) node [pos=0.9,below left]{$y$};
					\draw[dashed,thick](O)--($(O)!1/2!(H)$);
					\draw[thick,->]($(O)!1/2!(H)$)--(O1) node [pos=0.9,above right]{$z$};
					%Các góc vuông
					\gv{S}{A}{C}
					\gv{B}{O}{C}
					\foreach \x/\g in {B/-20,A/120,C/-50,S/180,O/40,H/-10}
					\fill[black] (\x) circle (1pt) ($(\g:4mm)+(\x)$) node {$\x$};	
				\end{tikzpicture}
				
			- Gọi $O$ là trung điểm $BC$. Chọn hệ trục tọa độ như hình vẽ, $AB=a=1$.
			
			- Tọa độ các điểm là:
			
			$O(0;0;0)$, $A \left(\dfrac{-1}{2};0;0\right)$, $B \left(0;\dfrac{\sqrt{3}}{2};0\right)$, $C \left(\dfrac{1}{2};0;0\right)$, $S \left(0;\dfrac{\sqrt{3}}{2};\underbrace {OH}_{ = SA}\right)$.
	\end{tabular} \\ \hline
	\multicolumn{1}{|>{\raggedright\arraybackslash}p{5.2cm}|}{\begin{tabular}[l]{>{\raggedright\arraybackslash}p{5.2cm}} \textbf{Đáy là tam giác vuông tại $B$}
			
			\begin{tikzpicture}[>=stealth,font=\footnotesize,scale=1]
				\def\a{4}
				\def\b{3}
				\def\h{2.6}
				\path (0:0) coordinate (A)
				++(0:\a) coordinate (C)
				++(-150:\b) coordinate (B)
				($(C)!1.01!(B)$) coordinate (O)
				($(A)+(90:\h)$) coordinate (S)
				($(O)+(90:3.5)$) coordinate (O1)
				($(S)+(O)-(A)$) coordinate (H);
				\draw[dashed,thick] (A)--(C);
				\draw[thick] (S)--(A)--(B)--(C)--(S)--(B);
				\draw[thick](S)--(H);
				%Ve truc Ox,Oy, Oz
				\draw[thick,->](C)--($(O)!1.1!(C)$) node [pos=0.9,above ]{$x$};
				\draw[thick,->](A)--($(O)!1.2!(A)$) node [pos=0.9, above]{$y$};
				\draw[thick,->](O)--(O1) node [above]{$z$};
				%Các góc vuông
				\gv{H}{O}{A}
				\gv{C}{B}{A}
				\gv{C}{A}{S}
				\gv{S}{H}{B}
				\foreach \x/\g in {A/-90,B/0,C/-40,S/90,O/-110,H/-10}
				\fill[black] (\x) circle (1pt) ($(\g:4mm)+(\x)$) node {$\x$};	
			\end{tikzpicture}
	\end{tabular}} &\multicolumn{1}{l|}{\begin{tabular}[l]{>{\raggedright\arraybackslash}p{5.2cm}}\textbf{Đáy là tam giác vuông tại $A$}
	
	\begin{tikzpicture}[>=stealth,font=\footnotesize,scale=1]
		\def\a{3.5}
		\def\b{4}
		\def\h{3}
		\path (0:0) coordinate (O)
		++(0:\a) coordinate (C)
		++(-165:\b) coordinate (B)
		($(A)+(90:\h)$) coordinate (S)
		($(O)+(90:3.3)$) coordinate (O1)
		($(C)!1!(O)$) coordinate (A);
		\draw[dashed,thick] (B)--(A)--(C) (O)--(S);
		\draw[thick] (S)--(B)--(C)--(S);
		%Ve truc Ox,Oy, Oz
		\draw[thick,->](C)--($(O)!1.2!(C)$) node [pos=0.9,below]{$x$};
		\draw[thick,->](B)--($(O)!1.25!(B)$) node [pos=0.9,right]{$y$};
		\draw[thick,->](S)--(O1) node [pos=0.9,above right]{$z$};
		%Các góc vuông
		\gv{B}{O}{C}
		\gv{A}{O}{C}
		\foreach \x/\g in {A/-45,B/120,C/-50,S/180,O/145}
		\fill[black] (\x) circle (1pt) ($(\g:4mm)+(\x)$) node {$\x$};	
	\end{tikzpicture}
	\end{tabular}} & \begin{tabular}[l]{>{\raggedright\arraybackslash}p{5.6cm}}\textbf{Đáy là tam giác thường}
	\begin{tikzpicture}[>=stealth,font=\footnotesize,scale=1]
	\def\a{3.8}
	\def\b{3}
	\def\h{2}
	\path (0:0) coordinate (A)
	++(0:\a) coordinate (C)
	++(-150:\b) coordinate (B)
	($(A)!1/2!(C)$) coordinate (O)
	($(A)+(90:\h)$) coordinate (S)
	($(O)+(90:2.5)$) coordinate (O1)
	($(S)+(O)-(A)$) coordinate (H);
	\draw[dashed,thick] (A)--(C);
	\draw[thick] (S)--(A)--(B)--(C)--(S)--(B);
	\draw[dashed,thick](B)--(O) ;
	\draw[thick](S)--(H);
	%Ve truc Ox,Oy, Oz
	\draw[thick,->](C)--($(O)!1.4!(C)$) node [pos=0.9,below]{$x$};
	\draw[thick,->](B)--($(O)!1.2!(B)$) node [pos=0.9,below left]{$y$};
	\draw[dashed,thick](O)--($(O)!1/2!(H)$);
	\draw[thick,->]($(O)!1/2!(H)$)--(O1) node [pos=0.9,above right]{$z$};
	%Các góc vuông
	\gv{S}{A}{C}
	\gv{B}{O}{C}
	\foreach \x/\g in {B/-20,A/120,C/-50,S/180,O/40,H/-10}
	\fill[black] (\x) circle (1pt) ($(\g:4mm)+(\x)$) node {$\x$};	
	\end{tikzpicture}
	\end{tabular} \\ \hline
	\end{longtable}
\begin{longtable}{|>{\raggedright\arraybackslash}p{5.4cm}|>{\raggedright\arraybackslash}p{5.2cm}|>{\raggedright\arraybackslash}p{5.8cm}|}
	\hline
	{\begin{tabular}[l]{>{\raggedright\arraybackslash}p{4.8cm}} \textbf{Đáy là tam giác vuông tại $B$}
			
			\begin{tikzpicture}[>=stealth,font=\footnotesize,scale=1]
				\def\a{4}
				\def\b{3}
				\def\h{2.6}
				\path (0:0) coordinate (A)
				++(0:\a) coordinate (C)
				++(-150:\b) coordinate (B)
				($(C)!1.01!(B)$) coordinate (O)
				($(A)+(90:\h)$) coordinate (S)
				($(O)+(90:3.5)$) coordinate (O1)
				($(S)+(O)-(A)$) coordinate (H);
				\draw[dashed,thick] (A)--(C);
				\draw[thick] (S)--(A)--(B)--(C)--(S)--(B);
				\draw[thick](S)--(H);
				%Ve truc Ox,Oy, Oz
				\draw[thick,->](C)--($(O)!1.1!(C)$) node [pos=0.9,above ]{$x$};
				\draw[thick,->](A)--($(O)!1.2!(A)$) node [pos=0.9, above]{$y$};
				\draw[thick,->](O)--(O1) node [above]{$z$};
				%Các góc vuông
				\gv{H}{O}{A}
				\gv{C}{B}{A}
				\gv{C}{A}{S}
				\gv{S}{H}{B}
				\foreach \x/\g in {A/-90,B/0,C/-40,S/90,O/-110,H/-10}
				\fill[black] (\x) circle (1pt) ($(\g:4mm)+(\x)$) node {$\x$};	
			\end{tikzpicture}
		
			- Chọn hệ trục tọa độ như hình vẽ, $a=1$.
			
			- Tọa độ các điểm là:
				
				$B \equiv O(0;0;0)$, $A \left(0;AB;0\right)$, $C \left(BC;0;0\right)$, $S \left(0;AB;\underbrace {BH}_{ = SA}\right)$.
	
	\end{tabular}}&{\begin{tabular}[l]{>{\raggedright\arraybackslash}p{5cm}}\textbf{Đáy là tam giác vuông tại $A$}
			
			\begin{tikzpicture}[>=stealth,font=\footnotesize,scale=1]
				\def\a{3.5}
				\def\b{4}
				\def\h{3}
				\path (0:0) coordinate (O)
				++(0:\a) coordinate (C)
				++(-165:\b) coordinate (B)
				($(A)+(90:\h)$) coordinate (S)
				($(O)+(90:3.3)$) coordinate (O1)
				($(C)!1!(O)$) coordinate (A);
				\draw[dashed,thick] (B)--(A)--(C) (O)--(S);
				\draw[thick] (S)--(B)--(C)--(S);
				%Ve truc Ox,Oy, Oz
				\draw[thick,->](C)--($(O)!1.2!(C)$) node [pos=0.9,below]{$x$};
				\draw[thick,->](B)--($(O)!1.25!(B)$) node [pos=0.9,right]{$y$};
				\draw[thick,->](S)--(O1) node [pos=0.9,above right]{$z$};
				%Các góc vuông
				\gv{B}{O}{C}
				\gv{A}{O}{C}
				\foreach \x/\g in {A/-45,B/120,C/-50,S/180,O/145}
				\fill[black] (\x) circle (1pt) ($(\g:4mm)+(\x)$) node {$\x$};	
			\end{tikzpicture}

			- Chọn hệ trục tọa độ như hình vẽ, $a=1$.
			
			- Tọa độ các điểm là:
				
				$A \equiv O(0;0;0)$, $B \left(0;OB;0\right)$, $C \left(AC;0;0\right)$, $S \left(0;0;SA \right)$.
	
	\end{tabular}}&{\begin{tabular}[l]{>{\raggedright\arraybackslash}p{5cm}} \textbf{Đáy là tam giác thường}
			\begin{tikzpicture}[>=stealth,font=\footnotesize,scale=1]
				\def\a{3.8}
				\def\b{3}
				\def\h{2}
				\path (0:0) coordinate (A)
				++(0:\a) coordinate (C)
				++(-150:\b) coordinate (B)
				($(A)!1/2!(C)$) coordinate (O)
				($(A)+(90:\h)$) coordinate (S)
				($(O)+(90:2.5)$) coordinate (O1)
				($(S)+(O)-(A)$) coordinate (H);
				\draw[dashed,thick] (A)--(C);
				\draw[thick] (S)--(A)--(B)--(C)--(S)--(B);
				\draw[dashed,thick](B)--(O) ;
				\draw[thick](S)--(H);
				%Ve truc Ox,Oy, Oz
				\draw[thick,->](C)--($(O)!1.4!(C)$) node [pos=0.9,below]{$x$};
				\draw[thick,->](B)--($(O)!1.2!(B)$) node [pos=0.9,below left]{$y$};
				\draw[dashed,thick](O)--($(O)!1/2!(H)$);
				\draw[thick,->]($(O)!1/2!(H)$)--(O1) node [pos=0.9,above right]{$z$};
				%Các góc vuông
				\gv{S}{A}{C}
				\gv{B}{O}{C}
				\foreach \x/\g in {B/-20,A/120,C/-50,S/180,O/40,H/-10}
				\fill[black] (\x) circle (1pt) ($(\g:4mm)+(\x)$) node {$\x$};	
			\end{tikzpicture}
			
				- Dựng đường cao $BO$ của $\triangle ABC$. Chọn hệ trục tọa độ như hình vẽ, $a=1$.\\
				- Tọa độ các điểm là:
			
				$O(0;0;0)$, $A \left(-OA;0;0\right)$, $B \left(0;OB;0\right)$, $C \left(OC;0;0\right)$, $S \left(-OA;0;\underbrace {OH}_{ = SA}\right)$.
	\end{tabular}}\\ \hline
	{\begin{tabular}[l]{>{\raggedright\arraybackslash}p{5cm}} \textbf{Đáy là hình vuông, hình chữ nhật}
			
			\begin{tikzpicture}[>=stealth,font=\footnotesize,scale=1]
				\def\a{3}
				\def\b{2}
				\def\h{2}
				\path 	(0:0) coordinate (A)
				++(0:\a) coordinate (D)
				++(-130:\b) coordinate (C)
				($(A)+(C)-(D)$) coordinate (B)
				($(A)+(90:\h)$) coordinate (S);
				\draw[dashed,thick] (S)--(A)--(B) (D)--(A);
				\draw[thick] (S)--(B)--(C)--(D)--(S)--(C);
				%Ve truc Ox,Oy, Oz
				\draw[thick,->](D)--($(A)!1.2!(D)$) node [pos=0.9,above ]{$x$};
				\draw[thick,->](B)--($(A)!1.2!(B)$) node [pos=0.9, above]{$y$};
				\draw[thick,->](S)--($(A)!1.2!(S)$) node [pos=0.9,above ]{$z$};
				%Các góc vuông
				\gv{D}{A}{S}
				\gv{D}{A}{B}
				\foreach \x/\g in {A/-90,B/-40,C/-40,D/-90,S/180}
				\fill[black] (\x) circle (1pt) ($(\g:4mm)+(\x)$) node {$\x$};	
			\end{tikzpicture}
		
			- Chọn hệ trục tọa độ như hình vẽ, $a=1$.
			
			- Tọa độ các điểm là:
			
			$A \equiv O(0;0;0)$, $B \left(0;AB;0\right)$, $C \left(AD;AB;0\right)$, $D(AD;0;0)$, $S \left(0;0;SA\right)$.
\end{tabular}}&{\begin{tabular}[l]{>{\raggedright\arraybackslash}p{5cm}}\textbf{Đáy là hình thoi}
		
		\begin{tikzpicture}[>=stealth,font=\footnotesize,scale=1]
			\def\a{3}
			\def\b{2}
			\def\h{2}
			\path 	(0:0) coordinate (A)
			++(0:\a) coordinate (D)
			++(-130:\b) coordinate (C)
			($(A)+(C)-(D)$) coordinate (B)
			($(A)+(90:\h)$) coordinate (S)
			($(A)!1/2!(C)$) coordinate (O)
			($(S)+(O)-(A)$) coordinate (H);
			\draw[dashed,thick] (S)--(A)--(B) (D)--(A);
			\draw[thick] (S)--(B)--(C)--(D)--(S)--(C) (S)--(H);
			\draw[dashed,thick] (A)--(C) (B)--(D);
			%Ve truc Ox,Oy, Oz
			\draw[thick,->](A)--($(A)!-1/5!(C)$) node [pos=0.9,above ]{$x$};
			\draw[thick,->](B)--($(B)!-1/10!(D)$) node [pos=0.9, above]{$y$};
			\draw[thick,dashed](O)--($(O)!1/2!(H)$);
			\draw[thick,->]($(O)!1/2!(H)$)--($(O)!3/2!(H)$) node [pos=0.9,right ]{$z$};
	
			%Các góc vuông
			\gv{D}{A}{S}
			\gv{S}{H}{O}
			\gv{A}{O}{D}
			\foreach \x/\g in {A/-90,B/-40,C/-40,D/-90,S/180,O/-90,H/-40}
			\fill[black] (\x) circle (1pt) ($(\g:4mm)+(\x)$) node {$\x$};	
		\end{tikzpicture}
	
			- Chọn hệ trục tọa độ như hình vẽ, $a=1$.
			
			- Tọa độ các điểm là:
			
			$O(0;0;0)$, $A(OA;0;0)$, $B \left(0;OB;0\right)$, $C \left(-OC;0;0\right)$, $D(0;-OD;0)$, $S \left(OA;0;\underbrace {OH}_{ = SA} \right)$.
		
\end{tabular}}&{\begin{tabular}[l]{>{\raggedright\arraybackslash}p{5.8cm}} \textbf{Đáy là hình thang vuông}
	\begin{tikzpicture}[>=stealth,font=\footnotesize,scale=1]
		\def\a{3}
		\def\b{2.1}
		\def\h{2.2}
		\path 	(0:0) coordinate (A)
		++(0:\a) coordinate (D)
		($(A)+(-140:\b)$) coordinate (B)
		($(A)+(90:\h)$) coordinate (S)
		($(A)!0.78!(D)$) coordinate (H)
		($(H)+(B)-(A)$) coordinate (C);
		\draw[dashed,thick] (S)--(A)--(B) (D)--(A) (C)--(H);
		\draw[thick] (S)--(B)--(C)--(D)--(S)--(C);
		%Ve truc Ox,Oy, Oz
		\draw[thick,->](D)--($(A)!1.2!(D)$) node [pos=0.9,above ]{$x$};
		\draw[thick,->](B)--($(A)!1.2!(B)$) node [pos=0.9, above]{$y$};
		\draw[thick,->](S)--($(A)!1.2!(S)$) node [pos=0.9,above ]{$z$};
		%Các góc vuông
		\gv{D}{A}{S}
		\gv{D}{A}{B}
		\gv{C}{H}{A}
		\foreach \x/\g in {A/-90,B/-40,C/-40,D/-90,S/180,H/120}
		\fill[black] (\x) circle (1pt) ($(\g:4mm)+(\x)$) node {$\x$};	
	\end{tikzpicture}
	
	 	- Chọn hệ trục tọa độ như hình vẽ, $a=1$.
	 
	 - Tọa độ các điểm là:
	 
	 $A \equiv O(0;0;0)$, $B \left(0;AB;0\right)$, $C \left(AH;AB;0\right)$, $D(AD;0;0)$, $S \left(0;0;SA\right)$.
\end{tabular}}\\ \hline
\end{longtable}
\newpage
	\begin{longtable}{|>{\raggedright\arraybackslash}p{5cm}|>{\raggedright\arraybackslash}p{5cm}|>{\raggedright\arraybackslash}p{5cm}|}
	\hline
	\multicolumn{3}{|>{\centering\arraybackslash}p{16.5cm}|}{\textbf{2. Hình chóp có cạnh mặt bên $(SAB)$ vuông góc với mặt đáy}}                                                                                                                                                                                 \\ \hline
	\multicolumn{1}{|>{\raggedright\arraybackslash}p{5cm}|}{\begin{tabular}[l]{>{\raggedright\arraybackslash}p{4.5cm}} \textbf{Đáy là tam giác, mặt bên là tam giác thường}
			\begin{tikzpicture}[>=stealth,font=\footnotesize,scale=1]
				\def\a{4}
				\def\b{3}
				\def\h{3}
				\path (0:0) coordinate (A)
				++(0:\a) coordinate (C)
				++(-140:\b) coordinate (B)
				($(A)!0.55!(B)$) coordinate (O)
				($(A)!1/3!(B)$) coordinate (H)
				($(O)+(90:3.7)$) coordinate (O1)
				($(H)+(90:\h)$) coordinate (S)
				($(S)+(O)-(H)$) coordinate (K);
				\draw[dashed,thick] (A)--(C) (C)--(O);
				\draw[thick] (S)--(A)--(B)--(C)--(S)--(B);
				\draw[thick](S)--(H) (S)--(K);
				%Ve truc Ox,Oy, Oz
				\draw[thick,->](C)--($(O)!1.1!(C)$) node [pos=0.9,above ]{$x$};
				\draw[thick,->](A)--($(O)!1.3!(A)$) node [pos=0.9, above]{$y$};
				\draw[thick,->](O)--(O1) node [above]{$z$};
				%Các góc vuông
				\gv{C}{O}{B}
				\gv{S}{H}{B}
				\foreach \x/\g in {A/-90,B/0,C/-40,S/90,O/-110,H/-110,K/-45}
				\fill[black] (\x) circle (1pt) ($(\g:4mm)+(\x)$) node {$\x$};	
			\end{tikzpicture}
			
		- Vẽ đường cao $CO$ trong $\triangle ABC$. Chọn hệ trục như hình vẽ, $a=1$.
		
		- Tọa độ các điểm là:
				
				$O(0;0;0)$, $A \left(0;OA;0\right)$, $B \left(0;-OB;0\right)$, $C \left(OC;0;0\right)$, $S \left(0;OH;\underbrace {OK}_{ =SH}\right)$.
	\end{tabular}} &\multicolumn{1}{l|}{\begin{tabular}[l]{>{\raggedright\arraybackslash}p{5cm}}\textbf{Đáy là tam giác cân tại $C$ (hoặc đều), mặt bên là tam giác cân tại $S$ (hoặc đều)}
			\begin{tikzpicture}[>=stealth,font=\footnotesize,scale=1]
				\def\a{4}
				\def\b{3}
				\def\h{3.3}
				\path (0:0) coordinate (A)
				++(0:\a) coordinate (C)
				++(-140:\b) coordinate (B)
				($(A)!1/2!(B)$) coordinate (O)
				($(O)+(90:3.7)$) coordinate (O1)
				($(O)+(90:\h)$) coordinate (S);
				\draw[dashed,thick] (A)--(C) (C)--(O);
				\draw[thick] (S)--(A)--(B)--(C)--(S)--(B);
				\draw[thick](S)--(O);
				%Ve truc Ox,Oy, Oz
				\draw[thick,->](C)--($(O)!1.1!(C)$) node [pos=0.9,above ]{$x$};
				\draw[thick,->](A)--($(O)!1.3!(A)$) node [pos=0.9, above]{$y$};
				\draw[thick,->](O)--(O1) node [above]{$z$};
				%Các góc vuông
				\gv{C}{O}{B}
				\gv{S}{O}{A}
				\foreach \x/\g in {A/-90,B/0,C/-40,S/180,O/-110}
				\fill[black] (\x) circle (1pt) ($(\g:4mm)+(\x)$) node {$\x$};	
			\end{tikzpicture}
			- Gọi $O$ là trung điểm $BC$. Chọn hệ trục như hình vẽ, $a=1$.
			
			- Tọa độ các điểm là:
			
			$O(0;0;0)$, $A \left(0;OA;0\right)$, $B \left(0;-OB;0\right)$, $C \left(OC;0;0\right)$, $S \left(0;0;SO\right)$.
	\end{tabular}} & \begin{tabular}[l]{>{\raggedright\arraybackslash}p{5.4cm}} \textbf{Đáy là hình chữ nhật, hình vuông, mặt bên là tam giác thường}
			\begin{tikzpicture}[>=stealth,font=\footnotesize,scale=1]
			\def\a{3}
			\def\b{2}
			\def\h{2.5}
			\path (0:0) coordinate (A)
			++(0:\a) coordinate (B)
			++(-130:\b) coordinate (C)
			($(A)+(C)-(B)$) coordinate (D)
			($(A)!1/3!(B)$) coordinate (H)
			($(H)+(90:\h)$) coordinate (S)
			($(A)+(S)-(H)$) coordinate (K);
			\draw[dashed,thick] (S)--(A)--(D) (B)--(A) (S)--(H);
			\draw[thick] (S)--(D)--(C)--(B)--(S)--(C) (K)--(S);
			%Ve truc Ox,Oy, Oz
			\draw[thick,->](B)--($(A)!1.2!(B)$) node [pos=0.9,above ]{$x$};
			\draw[thick,->](D)--($(A)!1.2!(D)$) node [pos=0.9, above]{$y$};
			\draw[thick,->](A)--($(A)!1.2!(K)$) node [right]{$z$};
			%Các góc vuông
			\gv{S}{H}{B}
			\gv{D}{A}{B}
			\foreach \x/\g in {A/-90,D/-40,C/-40,B/-90,S/90,H/-90,K/180}
			\fill[black] (\x) circle (1pt) ($(\g:4mm)+(\x)$) node {$\x$};	
		\end{tikzpicture}
		- Chọn hệ trục tọa độ như hình vẽ, $a=1$.
		
		- Tọa độ các điểm là:
			
			$A \equiv O(0;0;0)$, $B \left(AB;0;0\right)$, $C \left(AB;AD;0\right)$, $D \left(0;AD;0\right)$, $S \left(AH;0;\underbrace {AK}_{ = SH}\right)$.
	\end{tabular} \\\hline
\end{longtable}
\newpage
\begin{longtable}{|>{\raggedright\arraybackslash}p{8.5cm}|>{\raggedright\arraybackslash}p{8.5cm}|}
	\hline
	{\begin{tabular}[l]{>{\raggedright\arraybackslash}p{8.5cm}} \textbf{Hình chóp tam giác đều}
		\begin{tikzpicture}[>=stealth,font=\footnotesize,scale=1]
			\def\a{4}
			\def\b{3}
			\def\h{4}
			\path (0:0) coordinate (A)
			++(0:\a) coordinate (C)
			++(-150:\b) coordinate (B)
			($(B)!1/2!(C)$) coordinate (O)
			($(A)!2/3!(O)$) coordinate (H)
			($(H)+(90:\h)$) coordinate (S)
			($(S)+(O)-(H)$) coordinate (K);
			\draw[dashed,thick] (A)--(C) (A)--(O) (S)--(H);
			\draw[thick] (S)--(A)--(B)--(C)--(S)--(B) (S)--(K);
			\foreach \x/\g in {A/180,B/-90,C/0,S/180,H/-120}
				%Ve truc Ox,Oy, Oz
				\draw[thick,->](C)--($(C)!-0.2!(B)$) node [pos=0.9,above ]{$x$};
				\draw[thick,->](A)--($(A)!-0.2!(O)$) node [pos=0.9, above]{$y$};
				\draw[thick,->](O)--($(O)!1.15!(K)$) node [above]{$z$};
				%Các góc vuông
				\gv{S}{H}{O}
				\gv{A}{O}{K}
				\gv{K}{O}{C}
				\foreach \x/\g in {A/-90,B/-90,C/-90,S/180,H/-90,O/0,K/0}
				\fill[black] (\x) circle (1pt) ($(\g:4mm)+(\x)$) node {$\x$};	
			\end{tikzpicture}
			
			Gọi $O$ là trung điểm $BC$. Chọn hệ trục như hình vẽ, $a=1$.
			
			- Tọa độ các điểm là:
			
			$O(0;0;0)$, $A \left(0;\dfrac{AB \sqrt{3}}{2};0\right)$, $B \left(-\dfrac{BC}{2};0;0\right)$, $C \left(0;0;OC\right)$, $S \left(0;\underbrace {\dfrac{AB \sqrt{3}}{6}}_{ =SH};\underbrace {OK}_{ =SH}\right)$.
	\end{tabular}} &{\begin{tabular}[l]{>{\raggedright\arraybackslash}p{8.5cm}}\textbf{Hình chóp tứ giác đều}
			\begin{tikzpicture}[>=stealth,font=\footnotesize,scale=1]
					\def\a{3}
				\def\b{2.5}
				\def\h{3.4}
				\path (0:0) coordinate (D)
				++(0:\a) coordinate (A)
				++(-150:\b) coordinate (B)
				($(D)+(B)-(A)$) coordinate (C)
				($(D)!1/2!(B)$) coordinate (O)
				($(O)+(90:\h)$) coordinate (S);
				\draw[dashed,thick] (C)--(D)--(A) (D)--(S) (D)--(B) (A)--(C);
				\draw[thick] (S)--(C)--(B)--(A)--(S) (B)--(S);
				%Ve truc Ox,Oy, Oz
				\draw[thick,->](A)--($(C)!1.2!(A)$) node [pos=0.9,above ]{$x$};
				\draw[thick,->](B)--($(D)!1.4!(B)$) node [pos=0.9, above right]{$y$};
				\draw[dashed,thick] (S)--(O);
				\draw[thick,->](S)--($(O)!1.15!(S)$) node [above]{$z$};
				%Các góc vuông
				\gv{D}{O}{S}
				\gv{B}{O}{A}
				\foreach \x/\g in {A/-40,C/180,D/180,B/-140,S/180,O/-120}
				\fill[black] (\x) circle (1pt) ($(\g:4mm)+(\x)$) node {$\x$};	
			\end{tikzpicture}
			
			- Chọn hệ trục như hình vẽ, $a=1$.
			
			- Tọa độ các điểm là:
			
			$O(0;0;0)$, $A \left(\underbrace{\dfrac{AB \sqrt{2}}{2}}_{ =OA};0;0\right)$, $B\left(0;\underbrace{\dfrac{AB \sqrt{2}}{2}}_{ =OB};0\right)$, $C \left(\underbrace{-\dfrac{AB \sqrt{2}}{2}}_{ =-OA};0;0 \right)$,  $D \left(0;\underbrace{-\dfrac{AB \sqrt{2}}{2}}_{ =-OA};0 \right)$, $S \left(0;0;SO\right)$.
	\end{tabular}}\\ \hline
\multicolumn{2}{|>{\centering\arraybackslash}p{17cm}|}{\textbf{II. Gắn tọa độ đối với hình lăng trụ}}                                                                                                                                                                                 \\ \hline
\multicolumn{2}{|>{\centering\arraybackslash}p{17cm}|}{\textbf{1. Hình lăng trụ đứng}}                                                                                                                                                            \\ \hline
{\begin{tabular}[l]{>{\raggedright\arraybackslash}p{8.5cm}} \textbf{Hình lập phương, hình hộp chữ nhật}
		\begin{tikzpicture}[>=stealth,font=\footnotesize,scale=0.9]
		\def\a{4} 
		\def\b{1.8}
		\def\h{2}
		\path 	(0:0) coordinate (A)
		++(0:\a) coordinate (D)
		++(-130:\b) coordinate (C)
		($(A)+(C)-(D)$) coordinate (B)
		($(A)+(90:\h)$) coordinate (A')
		($(B)+(90:\h)$) coordinate (B')
		($(C)+(90:\h)$) coordinate (C')
		($(D)+(90:\h)$) coordinate (D');
		\draw[dashed,thick] 	(B)--(A)--(D)	(A)--(A');
		\draw[thick] (C)--(C') 	(D)--(D') 	(B)--(B') 	(B)--(C)--(D) (A')--(B')--(C')--(D')--cycle;
			%Ve truc Ox,Oy, Oz
			\draw[thick,->](D)--($(A)!1.2!(D)$) node [pos=0.9,above ]{$x$};
			\draw[thick,->](B)--($(A)!1.4!(B)$) node [pos=0.9,right]{$y$};
			\draw[thick,->](A')--($(A)!1.2!(A')$) node [right]{$z$};
			%Các góc vuông
			\gv{B}{A}{D}
			\gv{D}{A}{A'}
			\foreach \x/\g in  {A/180,B/180,C/0,D/-85,A'/180,B'/180,C'/0,D'/0}
			\fill[black] (\x) circle (1pt) ($(\g:4mm)+(\x)$) node {$\x$};	
		\end{tikzpicture}
		
		- Chọn hệ trục như hình vẽ, $a=1$.
		
		- Tọa độ các điểm là:
		
	$A \equiv O \left(0;0;0\right)$, $B \left(0;AB;0\right)$, $C \left(AD;AB;0\right)$,  $D \left(AD;0;0\right)$, $A' \left(0;0;AA'\right)$, $B' \left(0;AB;AA'\right)$, $C' \left(AD;AB;AA'\right)$, $D' \left(AD;0;AA'\right)$.
\end{tabular}} &{\begin{tabular}[l]{>{\raggedright\arraybackslash}p{8.5cm}}\textbf{\textbf{Hình lăng trụ đứng đáy là hình thoi}}
		\begin{tikzpicture}[>=stealth,font=\footnotesize,scale=0.9]
			\def\a{4} 
			\def\b{1.5}
			\def\h{1.8}
			\path 	(0:0) coordinate (A)
			++(0:\a) coordinate (D)
			++(-130:\b) coordinate (C)
			($(A)+(C)-(D)$) coordinate (B)
			($(A)+(90:\h)$) coordinate (A')
			($(B)+(90:\h)$) coordinate (B')
			($(C)+(90:\h)$) coordinate (C')
			($(D)+(90:\h)$) coordinate (D')
			($(A)!1/2!(C)$) coordinate (O)
			($(A')!1/2!(C')$) coordinate (O');
			\draw[dashed,thick] 	(B)--(A)--(D)	(A)--(A') (A)--(C) (B)--(D);
			\draw[thick] (C)--(C') 	(D)--(D') 	(B)--(B') 	(B)--(C)--(D) (A')--(B')--(C')--(D')--cycle (A')--(C') (B')--(D');
			%Ve truc Ox,Oy, Oz
			\draw[thick,->](C)--($(A)!1.2!(C)$) node [pos=0.9,below]{$x$};
			\draw[thick,->](B)--($(D)!1.1!(B)$) node [below right]{$y$};
			\draw[thick,dashed](O)--(O');
			\draw[thick,->](O')--($(O)!1.6!(O')$) node [right]{$z$};
			%Các góc vuông
			\gv{A'}{A}{D}
			\gv{B}{A}{D}
			\gv{A}{O}{D}
			\foreach \x/\g in  {A/180,B/159,C/15,D/-85,A'/180,B'/180,C'/0,D'/0,O/-90}
			\fill[black] (\x) circle (1pt) ($(\g:4mm)+(\x)$) node {$\x$};	
		\end{tikzpicture}
		
	- Chọn hệ trục như hình vẽ, $a=1$.
	
	- Tọa độ các điểm là:
	
	$O \left(0;0;0\right)$, $A \left(-OA;0;0\right)$, $B \left(0;OB;0\right)$, $C \left(OC;0;0\right)$,  $D \left(0;-OD:0\right)$, $A' \left(-OA;0;AA'\right)$, $B' \left(0;OB;AA'\right)$, $C' \left(OC;0;CC'\right)$, $D' \left(0;-OD;DD'\right)$.
\end{tabular}}\\ \hline
{\begin{tabular}[l]{>{\raggedright\arraybackslash}p{8.5cm}} \textbf{Lăng trụ tam giác đều}
		\begin{tikzpicture}[>=stealth,font=\footnotesize,scale=0.9]
			\def\a{4.3} 
			\def\b{2}
			\def\h{2.5}
			\path 	(0:0) coordinate (A)
			++(0:\a) coordinate (C)
			(A)	++(-50:\b) coordinate (B)
			($(A)+(90:\h)$) coordinate (A')
			($(B)+(90:\h)$) coordinate (B')
			($(C)+(90:\h)$) coordinate (C')
			($(A)!1/2!(C)$) coordinate (O)
			($(A')!1/2!(C')$) coordinate (O');
			\draw[dashed,thick] 	(A)--(C) (B)--(O) (O)--(O');
			\draw[thick] (C)--(C')--(B')--(A') (B)--(B') (B')--(O')	(B)--(C) (C')--(A')--(A)--(B) ;
			%Ve truc Ox,Oy, Oz
			\draw[thick,->](C)--($(A)!1.2!(C)$) node [pos=0.9,above ]{$x$};
			\draw[thick,->](B)--($(O)!1.4!(B)$) node [pos=0.9,right]{$y$};
			\draw[thick,->](O')--($(O)!1.2!(O')$) node [right]{$z$};
			%Các góc vuông
			\gv{B}{O}{C}
			\gv{C}{O}{O'}
				\foreach \x/\g in {A/180,B/180,C/-50,A'/180,B'/180,C'/0,O/130}
			\fill[black] (\x) circle (1pt) ($(\g:4mm)+(\x)$) node {$\x$};	
		\end{tikzpicture}
		
		- Chọn hệ trục như hình vẽ, $a=1$.
		
		- Tọa độ các điểm là:
		
		$O \left(0;0;0\right)$, $A \left(-\dfrac{AC}{2};0;0\right)$, $B \left(0;OB;0\right)$,  $C \left(\dfrac{AC}{2};0;0\right)$, $A' \left(-\dfrac{AC}{2};0;AA' \right)$, $B' \left(0;OB;AA'\right)$, $C' \left(\dfrac{AC}{2};0;AA'\right)$.
\end{tabular}} &{\begin{tabular}[l]{>{\raggedright\arraybackslash}p{8.4cm}}\textbf{\textbf{Lăng trụ đứng có đáy là tam giác thường}}

\begin{tikzpicture}[>=stealth,font=\footnotesize,scale=0.9]
	\def\a{4.3} 
	\def\b{2}
	\def\h{2.5}
	\path 	(0:0) coordinate (A)
	++(0:\a) coordinate (C)
	(A)	++(-50:\b) coordinate (B)
	($(A)+(90:\h)$) coordinate (A')
	($(B)+(90:\h)$) coordinate (B')
	($(C)+(90:\h)$) coordinate (C')
	($(A)!0.4!(C)$) coordinate (O)
	($(A')!0.4!(C')$) coordinate (O');
	\draw[dashed,thick] 	(A)--(C) (B)--(O) (O)--(O');
	\draw[thick] (C)--(C')--(B')--(A') (B)--(B') (B')--(O')	(B)--(C) (C')--(A')--(A)--(B) ;
	%Ve truc Ox,Oy, Oz
	\draw[thick,->](C)--($(A)!1.2!(C)$) node [pos=0.9,above ]{$x$};
	\draw[thick,->](B)--($(O)!1.4!(B)$) node [pos=0.9,right]{$y$};
	\draw[thick,->](O')--($(O)!1.2!(O')$) node [right]{$z$};
	%Các góc vuông
	\gv{B}{O}{C}
	\gv{C}{O}{O'}
	\foreach \x/\g in {A/180,B/180,C/-50,A'/180,B'/180,C'/0,O/130}
	\fill[black] (\x) circle (1pt) ($(\g:4mm)+(\x)$) node {$\x$};	
\end{tikzpicture}
		
		- Vẽ đường cao $CO$ của $\triangle ABC$. Chọn hệ trục như hình vẽ, $a=1$.
		
		- Tọa độ các điểm là:
		
		$O \left(0;0;0\right)$, $A \left(-OA;0;0\right)$, $B \left(0;OB;0\right)$, $C \left(OC;0;0\right)$, $A' \left(-OA;0;AA'\right)$, $B' \left(0;OB;AA'\right)$, $C' \left(OC;0;AA'\right)$.
\end{tabular}} \\ \hline
\multicolumn{2}{|>{\centering\arraybackslash}p{17cm}|}{\textbf{2. Hình lăng trụ xiên}}                                                                          \\ \hline
{\begin{tabular}[l]{>{\raggedright\arraybackslash}p{8.4cm}} \textbf{Lăng trụ có đáy là tam giác đều, hình chiếu của các đỉnh trên mặt đối diện là trung điểm của một cạnh tam giác đáy}
		\begin{tikzpicture}[>=stealth,font=\footnotesize,scale=0.9]
			\def\a{4.3} 
			\def\b{2}
			\def\h{2.5}
			\path 	(0:0) coordinate (A)
			++(0:\a) coordinate (C)
			(A)	++(-50:\b) coordinate (B)
			($(A)!1/2!(C)$) coordinate (O)
			($(O)+(90:\h)$) coordinate (A')
			($(A')+(B)-(A)$) coordinate (B')
			($(A')+(C)-(A)$) coordinate (C')
			($(A')!1/2!(C')$) coordinate (O');
			\draw[dashed,thick] 	(A)--(C) (B)--(O) (O)--(A');
			\draw[thick] (C)--(C')--(B')--(A') (B)--(B') (B')--(O')	(B)--(C) (C')--(A')--(A)--(B) ;
			%Ve truc Ox,Oy, Oz
			\draw[thick,->](C)--($(A)!1.2!(C)$) node [pos=0.9,above ]{$x$};
			\draw[thick,->](B)--($(O)!1.4!(B)$) node [pos=0.9,right]{$y$};
			\draw[thick,->](A')--($(O)!1.25!(A')$) node [right]{$z$};
			%Các góc vuông
			\gv{A}{O}{A'}
			\gv{A}{O}{B}
			\foreach \x/\g in {A/180,B/180,C/-50,A'/180,B'/180,C'/0,O/30}
			\fill[black] (\x) circle (1pt) ($(\g:4mm)+(\x)$) node {$\x$};	
		\end{tikzpicture}
		
		- Chọn hệ trục như hình vẽ, ta dễ xác định tọa đọ các điểm $O$, $A$, $B$, $C$, $A'$.
		
		- Tìm tọa độ các điểm còn lại thông qua $\overrightarrow{AA'}=\overrightarrow{BB'}=\overrightarrow{CC'}$.
\end{tabular}} &{\begin{tabular}[l]{>{\raggedright\arraybackslash}p{8.4cm}}\textbf{\textbf{Lăng trụ xiên có đáy là hình vuông hoặc hình chữ nhật, hình chiếu của một đỉnh là một điểm thuộc cạnh đáy không chứa đỉnh đó}}
		
		\begin{tikzpicture}[>=stealth,font=\footnotesize,scale=0.9]
			\def\a{4} 
			\def\b{1.5}
			\def\h{2.8}
			\path (0:0) coordinate (A)
			++(0:\a) coordinate (D)
			++(-150:\b) coordinate (C)
			($(A)+(C)-(D)$) coordinate (B)
			($(B)!1/2!(C)$) coordinate (O)
			($(O)+(90:\h)$) coordinate (A')
			($(A')+(B)-(A)$) coordinate (B')
			($(A')+(C)-(A)$) coordinate (C')
			($(A')+(D)-(A)$) coordinate (D')
			($(A)!1/2!(D)$) coordinate (O');
			\draw[dashed,thick] 	(B)--(A)--(D)	(A)--(A') (A')--(O) (O)--(O');
			\draw[thick] (C)--(C') 	(D)--(D') 	(B)--(B') 	(B)--(C)--(D) (A')--(B')--(C')--(D')--cycle;
			%Ve truc Ox,Oy, Oz
			\draw[thick,->](C)--($(C)!-0.3!(B)$) node [pos=0.9,below]{$x$};
			\draw[thick,->,dashed](O')--($(O)!1.5!(O')$) node [right]{$y$};
			\draw[thick,->](A')--($(O)!1.3!(A')$) node [pos=0.9,right]{$z$};
			%Các góc vuông
			\gv{C}{O}{A'}
			\gv{C}{B}{A}
			\foreach \x/\g in  {A/180,B/180,C/-90,D/-85,A'/180,B'/180,C'/0,D'/0,O/-90}
			\fill[black] (\x) circle (1pt) ($(\g:4mm)+(\x)$) node {$\x$};		
		\end{tikzpicture}
		
		- Chọn hệ trục như hình vẽ, ta dễ xác định tọa đọ các điểm $O$, $A$, $B$, $C$, $A'$.
	
		- Tìm tọa độ các điểm còn lại thông qua $\overrightarrow{AA'}=\overrightarrow{BB'}=\overrightarrow{CC'}=\overrightarrow{DD'}$.
\end{tabular}} \\ \hline
\end{longtable}
\TN
\Opensolutionfile{ans}[ans/ans-2C5B1CD3]
\begin{ex}%[2H5H1-5]
	Cho tứ diện $O.ABC$, có $OA$, $OB$, $OC$ đôi một vuông góc và $OA=5$, $OB=2$, $OC=4$. Gọi $M$, $N$ lần lượt là trung điểm của $OB$ và $OC$. Gọi $G$ là trọng tâm của tam giác $ABC$. Khoảng cách từ $G$ đến mặt phẳng $(AMN)$ là
	\choice
	{\True $\dfrac{20}{3\sqrt{129}}$}
	{$\dfrac{20}{\sqrt{129}}$}
	{$\dfrac{1}{4}$}
	{$\dfrac{1}{2}$}
	\loigiai{
		\begin{center}
			\begin{tikzpicture}[>=stealth,font=\footnotesize,scale=1]
				\def\a{3.5}
				\def\b{4.5}
				\def\h{3}
				\path (0:0) coordinate (O)
				++(0:\a) coordinate (C)
				++(-165:\b) coordinate (B)
				($(O)+(90:\h)$) coordinate (A)
				($(O)!1/2!(B)$) coordinate (M)
				($(O)!1/2!(C)$) coordinate (N) ;
				\draw[dashed,thick] (B)--(O)--(C) (O)--(A) (A)--(M)--(N)--(A);
				\draw[thick] (A)--(B)--(C)--(A);
				%Ve truc Ox,Oy, Oz
				\draw[thick,->](B)--($(O)!1.6!(B)$) node [pos=0.9,right]{$x$};
				\draw[thick,->](C)--($(O)!1.2!(C)$) node [pos=0.9,below]{$y$};
				\draw[thick,->](A)--($(O)!1.2!(A)$) node [pos=0.9,above right]{$z$};
				\foreach \x/\g in {B/120,C/-90,A/180,O/45,M/-30,N/60}
				\fill[black] (\x) circle (1pt) ($(\g:4mm)+(\x)$) node {$\x$};	
			\end{tikzpicture}
		\end{center}
	Chọn hệ trục tọa độ $Oxyz$ như hình vẽ.\\
	Ta có $O(0;0;0)$, $A \in Oz$, $B \in Ox$, $C \in Oy$ sao cho $OA=5$, $OB=2$, $OC=4$.\\
	Do đó $A(0;0;5)$, $B(2;0;0)$, $C(0;4;0)$.\\
	Khi đó $G$ là trọng tâm tam giác $ABC$ nên $G\left(\dfrac{2}{3};\dfrac{4}{3};\dfrac{5}{3}\right)$.\\
	Vì $M$ là trung điểm $OB$ nên $M(1;0;0)$.\\
	Vì $N$ là trung điểm $OC$ nên $N(0;2;0)$.\\
	Phương trình mặt phẳng $(AMN)$ là $\dfrac{x}{1}+\dfrac{y}{2}+\dfrac{z}{5}=1$ hay $10x+5y+2z-10=0$.\\
	Vậy khoảng cách từ $G$ đến mặt phẳng $(AMN)$ là
	$$\mathrm{d}(G,(AMN))=\dfrac{\left|\dfrac{20}{3}+\dfrac{20}{3}+\dfrac{10}{3}-10 \right|}{\sqrt{100+25+4}}=\dfrac{20}{3\sqrt{129}}.$$
	}
\end{ex}
\begin{ex}%[2H5V1-5]
Cho hình chóp $S.ABCD$ có đáy là hình thang vuông tại $A$ và $D$, $SA \perp (ABCD)$. Góc giữa $SB$ và mặt phẳng đáy bằng $45^\circ$, $E$ là trung điểm của $SD$, $AB=2a$, $AD=DC=a$. Tính khoảng cách từ điểm $B$ đến mặt phẳng $(ACE)$.
	\choice
	{ $\dfrac{2a}{2}$}
	{\True$\dfrac{4a}{3}$}
	{$a$}
	{$\dfrac{3a}{4}$}
	\loigiai{
\begin{center}
		\begin{tikzpicture}[>=stealth,font=\footnotesize,scale=1]
		\def\a{6.5}
		\def\b{3}
		\def\h{3.4}
		\path (0:0) coordinate (A)
		++(0:\a) coordinate (B)
		($(A)+(-145:\b)$) coordinate (D)
		($(A)+(90:\h)$) coordinate (S)
		($(A)!0.5!(B)$) coordinate (H)
		($(S)!0.5!(D)$) coordinate (E)
		($(H)+(D)-(A)$) coordinate (C);
		\draw[dashed,thick] (S)--(A)--(D) (B)--(A) (C)--(A)--(E);
		\draw[thick] (S)--(D)--(C)--(B)--(S)--(C) (C)--(E);
		\draw [thick]($(A)!7/8!(B)$) arc (180:135:0.5) node [pos=0.5,left]{$45^\circ$};
		%Ve truc Ox,Oy, Oz
		\draw[thick,->](B)--($(A)!1.2!(B)$) node [pos=0.9,above ]{$x$};
		\draw[thick,->](D)--($(A)!1.2!(D)$) node [pos=0.9, above]{$y$};
		\draw[thick,->](S)--($(A)!1.2!(S)$) node [pos=0.9,above ]{$z$};
		\foreach \x/\g in {A/40,D/-40,C/-40,B/-90,S/180,E/180}
		\fill[black] (\x) circle (1pt) ($(\g:4mm)+(\x)$) node {$\x$};	
	\end{tikzpicture}
\end{center}		
Hình chiếu của $SB$ trên mặt phẳng $(ABCD)$ là $AB$ nên góc giữa $SB$ và mặt đáy là góc giữa $SB$ và $AB$ bằng $\widehat{SBA}=45^\circ$.\\
Vì tam giác $SAB$ vuông cân tại $A$ nên $SA=2a$.\\
Chọn hệ trục tọa độ như hình vẽ, ta có $A(0;0;0)$, $B(0;2a;0)$, $C(a;a;0)$, $D(a;0;0)$, $S(0;0;2a)$, $E \left(\dfrac{a}{2}\;0;a \right)$.\\
Ta có $\overrightarrow{AC}=(a;a;0)$, $\overrightarrow{AE}= \left(\dfrac{a}{2};0;a\right)$. Do đó $\left[\overrightarrow{AC},\overrightarrow{AE}\right]=\left(a^2;-a^2;-\dfrac{a^2}{2}\right)$.\\
Mặt phẳng $(ACE)$ có véc-tơ pháp tuyến là $\overrightarrow{n}=(2;-2;-1)$ nên $(ACE) \colon 2x-2y-z=0$.\\
Vậy $\mathrm{d}(B,(ACE))=\dfrac{|2 \cdot 2a|}{\sqrt{4+4+1}}=\dfrac{4a}{3}$.
	}
\end{ex}
\begin{ex}%[2H5V1-5]
Trong KG $Oxyz$, cho hình chóp $SABCD$ có đáy $ABCD$ là hình chữ nhật. Biết $A(0;0;0)$, $D(2;0;0)$, $B(0;4;0)$, $S(0;0;4)$. Gọi $M$ là trung điểm của $SB$. Tính khoảng cách từ $B$ đến mặt phẳng $(CDM)$.
	\choice
	{$\mathrm{d}(B,(CDM))=2$}
	{$\mathrm{d}(B,(CDM))=2 \sqrt{2}$}
	{$\mathrm{d}(B,(CDM))=\dfrac{1}{\sqrt{2}}$}
	{\True$\mathrm{d}(B,(CDM))=\sqrt{2}$}
	\loigiai{
		\begin{center}
		\begin{tikzpicture}[>=stealth,font=\footnotesize,scale=1]
			\def\a{5}
			\def\b{2.5}
			\def\h{4}
			\path 	(0:0) coordinate (A)
			++(0:\a) coordinate (D)
			++(-130:\b) coordinate (C)
			($(A)+(C)-(D)$) coordinate (B)
			($(A)+(90:\h)$) coordinate (S)
			($(S)!1/2!(B)$) coordinate (M);
			\draw[dashed,thick] (S)--(A)--(B) (D)--(A) (C)--(D)--(M)--(C);
			\draw[thick] (S)--(B)--(C)--(D)--(S)--(C);
			%Ve truc Ox,Oy, Oz
			\draw[thick,->](D)--($(A)!1.2!(D)$) node [pos=0.9,above ]{$x$};
			\draw[thick,->](B)--($(A)!1.4!(B)$) node [pos=0.9, above]{$y$};
			\draw[thick,->](S)--($(A)!1.2!(S)$) node [pos=0.9,above ]{$z$};
			
			\foreach \x/\g in {A/-90,B/-40,C/-40,D/-90,S/180,M/180}
			\fill[black] (\x) circle (1pt) ($(\g:4mm)+(\x)$) node {$\x$};	
		\end{tikzpicture}		
		\end{center}
	Tứ giác $ABCD$ là hình chữ nhật nên $\heva{&x_A+x_C=x_B+x_D\\&y_A+y_C=y_B+y_D\\&z_A+z_C=z_B+z_D} \Leftrightarrow \heva{&x_C=2\\&y_C=4\\&z_C=0} \Leftrightarrow C(2;4;0)$.\\
	Vì $M$ là trung điểm $SB$ nên $M(0;2;2)$.\\
	Ta có $\overrightarrow{CD}=(0;-4;0)$, $\overrightarrow{CM}=(-2;-2;2)$. Do đó $\left[\overrightarrow{CD},\overrightarrow{CM}\right]=(-8;0;-8)$.\\
	Mặt phẳng $(CDM)$ có véc-tơ pháp tuyến là $\overrightarrow{n}=(1;0;1)$.\\
	Suy ra $(CDM)$ có phương trình $x+z-2=0$.\\
	Vậy $\mathrm{d}(B,(CDM))=\dfrac{|0+0-2|}{\sqrt{1^2+0^2+1^2}}=\sqrt{2}$.
	}
\end{ex}
\begin{ex}%[2H5V1-5]
Một phần sân trường được định vị bởi các điểm $A$, $B$, $C$, $D$ như hình vẽ.
\begin{center}
	\begin{tikzpicture}[>=stealth,font=\footnotesize,scale=1]
		\def\a{5}
		\def\b{2.5}
		\def\h{4}
		\path 	(0:0) coordinate (A)
		++(0:\a) coordinate (B)
		++(-90:\b*1.4) coordinate (C)
		($(A)+(-90:\b)$) coordinate (D);
		\draw[thick] (A)--(B) node[pos=0.5,above]{$2500$cm};
		\draw[thick] (B)--(C) node[pos=0.5,right]{$1500$cm};
		\draw[thick] (A)--(D) node[pos=0.5,left]{$1600$cm};
		\draw[thick] (D)--(C);
		%Ve truc Ox,Oy, Oz
		%\draw[thick,->](D)--($(A)!1.2!(D)$) node [pos=0.9,above ]{$x$};
		%\draw[thick,->](B)--($(A)!1.4!(B)$) node [pos=0.9, above]{$y$};
		%\draw[thick,->](S)--($(A)!1.2!(S)$) node [pos=0.9,above ]{$z$};
		
		\foreach \x/\g in {A/180,B/-40,C/-40,D/-90}
		\fill[black] (\x) circle (1pt) ($(\g:4mm)+(\x)$) node {$\x$};	
	\end{tikzpicture}
\end{center}
Bước đầu chúng được lấy "thăng bằng" để có cùng độ cao, biết $ABCD$ là hình thang vuông ở $A$ và $B$ với độ dài $AB=25$ m, $AD=15$ m, $BC=18$ m. Do yêu cầu kĩ thuật, khi lát phẳng phần sân trường phải thoát nước về góc sân ở $C$ nên người ta lấy độ cao ở các điểm $B$, $C$, $D$ xuống thấp hơn so với độ cao ở $A$ là $10$ cm, $a$ cm, $6$ cm tương ứng. Giá trị của $a$ là số nào sau đây?
	\choice
	{$15{,}7$ cm}
	{\True $17{,}2$ cm}
	{$18{,}1$ cm}
	{$17{,}5$ cm}
	\loigiai{
	Chọn hệ trục tọa độ $Oxyz$ sao cho $O \equiv A$, tia $Ox \equiv AD$, tia $Oy \equiv AB$.
	\begin{center}
		\begin{tikzpicture}[>=stealth,font=\footnotesize,scale=0.7]
			\def\a{5}
			\def\b{2.5}
			\def\h{4}
			\path 	(0:0) coordinate (A)
			++(0:\a) coordinate (B)
			++(-120:\b*1.4) coordinate (C)
			($(A)+(-123:\b)$) coordinate (D)
			($(B)+(-90:1.2)$) coordinate (B')
			($(C)+(-90:2.2)$) coordinate (C')
			($(D)+(-90:1.8)$) coordinate (D');
			\draw[thick] (A)--(B)--(C)--(D)--(A);
			\draw[dashed] (B)--(B')--(C')--(C)--(B);
			\draw[dashed] (D)--(D')--(C');
			%Ve truc Ox,Oy, Oz
			\draw[thick,->](D)--($(A)!1.8!(D)$) node [pos=0.9, above left]{$x$};
			\draw[thick,->](B)--($(A)!1.4!(B)$) node [pos=0.9,above ]{$y$};
			\draw[thick,->](A)--($(A)+(90:\h)$) node [pos=0.9,right]{$z$};
			
			\foreach \x/\g in {A/180,B/-40,C/140,D/180,B'/0,C'/0,D'/-90}
			\fill[black] (\x) circle (1pt) ($(\g:4mm)+(\x)$) node {$\x$};	
		\end{tikzpicture}
	\end{center}
	Khi đó $A(0;0;0)$, $B(0;2500;0)$, $C(1800;2500;0)$, $D(1500;0;0)$.\\
	Khi hạ độ cao các điểm ở các điểm $B$, $C$, $D$ xuống thấp hơn so với độ cao ở $A$ là $10$ cm, $a$ cm, $6$ cm tương ứng ta có các điểm mới $B'(0;2500;-10)$, $C(1800;2500;-a)$, $D'(1500;0;-6)$.\\
	Theo bài ta có bốn điểm $A$, $B'$, $C'$, $D'$ đồng phẳng.\\
	Phương trình mặt phẳng $(AB'D') \colon x+y+250z=0$.\\
	Do $C'(1800;2500;-a) \in (AB'D')$ nên có $1800+2500-250a=0 \Leftrightarrow a=17{,}2$.\\
	Vậy $a=17{,}2$ cm.
	}
\end{ex}
\Closesolutionfile{ans}
\indapan{10}{ans/ans-2C5B1CD3}
