\begin{ex}%[2H5H1-3]
Trong hệ trục tọa độ $O x y z$ cho $3$ điểm $M(4 ; 2 ; 1)$, $N(0 ; 0 ; 3)$, $Q(2 ; 0 ; 1)$. Viết phương trình mặt phẳng chứa $O Q$ và cách đều $2$ điểm $M$, $N$.
\choice
{$x-2 y-2 z=0$ hoặc $x+4 y-2 z=0$}
{$x+2 y+2 z=0$ hoặc $x-4 y-2 z=0$}
{$x+2 y-2 z=0$ hoặc $x+4 y-2 z=0$}
{\True $x+2 y-2 z=0$ hoặc $x-4 y-2 z=0$}
\loigiai{
	Gọi $(\alpha)\colon A x+B y+C z+D=0$ $\left(A^2+B^2+C^2 \neq 0\right)$.\\
	$O \in(\alpha)$ nên ta có $D=0$, $Q \in(\alpha)$ nên ta có $2 A+C=0 \Rightarrow C=-2 A$.\\
	Theo đề bài
	$$\mathrm{d}(M,(\alpha))=\mathrm{d}(N,(\alpha))	\Leftrightarrow|2 A+2 B|=|-6 A| \Leftrightarrow \hoac{&2 A + 2B = 6 A\\&2 A + 2B = - 6 A}\Leftrightarrow \hoac{&B=2 A& (*)\\&B=-4 A	 & (* *).}$$
	Từ $(*)$ chọn $A=1 \Rightarrow B=2$, $C=-2 \Rightarrow(\alpha)\colon x+2 y-2 z=0$.\\
	Từ $(**)$ chọn $A=1 \Rightarrow B=-4$, $C=-2 \Rightarrow(\alpha)\colon x-4 y-2 z=0$.
}
\end{ex}
\Closesolutionfile{ans}
\indapan{10}{ans/CD3-50-50}
\TNSA
\begin{ex}%[2H5H1-3]
	Trong không gian với hệ toạ độ $O x y z$, biết mặt phẳng $(P)\colon Ax+By+Cz+D=0$ ($A$, $B$, $C \in \mathbb{Z}$, $A$ và $C$ trái dấu) qua $O$, vuông góc với mặt phẳng $(Q)\colon x+y+z=0$ và cách điểm $M(1 ; 2 ;-1)$ một khoảng bằng $\sqrt{2}$. Tính giá trị của $A+B+C$.
	\shortans{$0$}
	\loigiai{
		 $(P)$ qua $O$ nên phương trình có dạng $A x+B y+C z=0$ (với $A^2+B^2+C^2 \neq 0$ ).\\
		Vì $({P}) \perp({Q})$ nên $1 \cdot A+1 \cdot B+1 \cdot C=0 \Leftrightarrow C=-A-B \quad (1)$.\\
		Do $\mathrm{d}(M,(P))=\sqrt{2} \Leftrightarrow \dfrac{|A+2 B-C|}{\sqrt{A^2+B^2+C^2}}=\sqrt{2} \Leftrightarrow(A+2 B-C)^2=2\left(A^2+B^2+C^2\right) \quad (2)$.\\
		Từ $(1)$ và $(2)$ ta được $8 A B+5 B^2=0 \Leftrightarrow\hoac{&B=0 &(3)\\ &8 A+5 B=0&(4).}$\\
		Từ $(3)$, ta có ${B}=0 \Rightarrow {C}=-{A}$ (nhận do $A$ và $C$ trái dấu). \\
		Chọn ${A}=1$, ${C}=-1 \Rightarrow({P})\colon  x-z=0$.\\
		Khi đó $A+B+C=0$.\\
		Từ $(4)$, ta có $8 {A}+5 {B}=0$. \\
		Chọn ${A}=5$, ${B}=-8 \Rightarrow {C}=3 \Rightarrow({P})\colon 5 x-8 y+3 z=0$. (loại  do $A$ và $C$ cùng dấu).
	}
\end{ex}

\begin{ex}%[2H5H1-3]
	Trong không gian với hệ toạ độ $Oxyz$, cho các điểm $M(-1 ; 1 ; 0)$, $N(0 ; 0 ;-2)$, $I(1 ; 1 ; 1)$. Biết mặt phẳng $({P})$ qua ${A}$ và ${B}$, đồng thời khoảng cách từ ${I}$ đến $({P})$ bằng $\sqrt{3}$. Giả sử phương trình mặt phẳng $(P)$ có dạng $ax+by+z+d=0$ với $b>0$. Tính $\dfrac{a}{b}$ viết dưới dạng số thập phân.
	\shortans{$1{,}4$}
	\loigiai{
	Phương trình mặt phẳng $({P})$ có dạng $a x+b y+ z+d=0$ $\left(a^2+b^2+1 \neq 0\right)$.\\
	Ta có $\heva{ &M \in(P) \\ &N \in(P) \\ &\mathrm{d}(I,(P))=\sqrt{3}} \Leftrightarrow \hoac{&a=-b,\ 2 =a-b,\ d=a-b &(1)\\ &5 a=7 b,\ 2 =a-b,\ d=a-b &(2).}$
		\begin{itemize}
		\item Với $(1) \Rightarrow $ Phương trình mặt phẳng $(P)\colon x-y+z+2=0$ (loại do $b<0$).
		\item Với $(2) \Rightarrow $ Phương trình mặt phẳng $(P)\colon 7 x+5 y+z+2=0$ (nhận do $b=5>0$).\\
		Khi đó $\dfrac{a}{b}=\dfrac{7}{5}=1{,}4$.
	\end{itemize}
	}
\end{ex}

\begin{ex}%[2H5H1-3]
	Trong không gian với hệ toạ độ $O x y z$, cho tứ diện $ABCD$ với $A(1 ;-1 ; 2)$, $B(1 ; 3 ; 0)$, $C(-3 ; 4 ; 1)$, $D(1 ; 2 ; 1)$. Mặt phẳng $({P})$ đi qua ${A}$, ${B}$ sao cho khoảng cách từ ${C}$ đến $({P})$ bằng khoảng cách từ ${D}$ đến $({P})$. Biết có hai mặt phẳng $(P)$ thỏa yêu cầu đề bài là $x+b_1y+c_1z+d_1=0$ và $x+b_2y+c_2z+d_2=0$. Tính $S=b_1+c_1+b_2+c_2$.
	\shortans{$9$}
	\loigiai{
		Phương trình mặt phẳng $({P})$ có dạng $a x+b y+c z+d=0$ với $\left(a^2+b^2+c^2 \neq 0\right)$.\\
		Ta có $\heva{&A \in(P) \\ &B \in(P) \\ &\mathrm{d}(C,(P))=\mathrm{d}(D,(P))} \Leftrightarrow\heva{&a-b+2 1+d=0 \\&a+3 b+d=0 \\&\dfrac{|-3 a+4 b+1+d|}{\sqrt{a^2+b^2+1^2}}=\dfrac{|a+2 b+1+d|}{\sqrt{a^2+b^2+1^2}}}$\\
		$ \Leftrightarrow\hoac{&b=2 a,\ c=4 a,\ d=-7 a \\& c=2 a,\ b=a,\ d=-4 a}$
		\begin{itemize}
			\item Với $b=2 a$, $c=4 a$, $d=-7 a$ và ta đã có $a=1$ nên $({P}) \colon x+2 y+4 z-7=0$.\\
			Khi đó $b_1=2$, $c_1=4$.
			\item Với $c=2 a$, $b=a$, $d=-4 a$ và ta đã có $a=1$ nên $({P})\colon x+y+2 z-4=0$.\\
			Khi đó $b_2=1$, $c_2=2$.
		\end{itemize}
		Vậy $S=2+4+1+2=9$.
	}
\end{ex}

\begin{ex}%[2H5H1-3]
	Trong không gian với hệ trục tọa độ $O x y z$, cho các điểm $A(1 ; 2 ; 3)$, $B(0 ;-1 ; 2)$, $C(1 ; 1 ; 1)$. Mặt phẳng $(P)$ đi qua $A$ và gốc tọa độ $O$ sao cho khoảng cách từ $B$ đến $(P)$ bằng khoảng cách từ $C$ đến $(P)$. Biết phương trình mặt phẳng $(P)$ có dạng $ax+by-4z+d=0$. Hỏi $a$ có bao nhiêu ước nguyên?
	\shortans{$12$}
	\loigiai{
		Vì $O \in(P)$ nên $(P)\colon a x+by-4 z=0$, với $a^2+b^2+16 \neq 0$.\\
		Do $A \in(P) \Rightarrow a+2 b -12=0$ $(1)$\\
		Và $\mathrm{d}(B,(P))=\mathrm{d}(C,(P)) \Leftrightarrow|-b-8|=|a+b-4|$ $(2)$.\\
		Từ $(1)$ và $(2) \Rightarrow b=0$.	Khi đó ta được $a=-3 \cdot (-4) =12$.\\
		Các ước nguyên của $12$ là $\{\pm 1;\pm 2; \pm 3; \pm 4; \pm 6; \pm 12\}$ có $12$ ước nguyên.
	}
\end{ex}

\begin{ex}%[2H5H1-3]
	Trong không gian với hệ trục tọa độ $O x y z$, cho ba điểm $A(1 ; 1 ;-1)$, $B(1 ; 1 ; 2)$, $C(-1 ; 2 ;-2)$ và mặt phẳng $({P})\colon x-2 y+2 z+1=0$. Mặt phẳng $(\alpha)$ đi qua ${A}$, vuông góc với mặt phẳng $({P})$, cắt đường thẳng ${BC}$ tại ${I}$ sao cho $I B=2 I C$. Biết có hai mặt phẳng $(\alpha)$ thỏa yêu cầu đề bài có phương trình lần lượt là $4x+b_1y+c_1+d_1=0$ và $2x+b_2y+c_2+d_2=0$ với $b_1<b_2$. Hỏi có bao nhiêu giá trị nguyên thuộc tập $(b_1;b_2)$?
	\shortans{$4$}
	\loigiai{
		Phương trình mặt phẳng $(\alpha)$ có dạng $a x+b y+c z+d=0$, với $a^2+b^2+c^2 \neq 0$.\\
		Do $A(1 ; 1 ;-1) \in(\alpha)$ nên $a+b-c+d=0$. $\quad (1)$;\\
		$(\alpha) \perp(P)$ nên $a-2 b+2 c=0\quad (2)$.
		\begin{eqnarray*}
			&I B=2 I C &\Rightarrow \mathrm{d}(B,(\alpha))=2 \mathrm{d}(C ;(\alpha))\\  & &\Rightarrow \dfrac{|a+b+2 c+d|}{\sqrt{a^2+b^2+c^2}}=2 \dfrac{|-a+2 b-2 c+d|}{\sqrt{a^2+b^2+c^2}}\\
			& &\Leftrightarrow\hoac{&3 a-3 b+6 c-d=0 \\&-a+5 b-2 c+3 d=0.}
		\end{eqnarray*}
		Từ $(1)$, $(2)$, $(3)$ ta có 2 trường hợp sau
		\begin{itemize}
			\item $\heva{&a+b-c+d=0 \\&a-2 b+2 c=0 \\& 3 a-3 b+6 c-d=0} \Leftrightarrow \heva{&b=\dfrac{-1}{2} a \\& c=-a \\& d=\dfrac{-3}{2} a.}$
			\item $\heva{&a+b-c+d=0 \\ &a-2 b+2 c=0 \\ &-a+5 b-2 c+3 d=0} \Leftrightarrow \heva{&b=\dfrac{3}{2} a \\ &c=a \\ &d=\dfrac{-3}{2} a.} \quad (3)$
		\end{itemize}
		Do theo đề bài, ta có $a>0$ nên ta có thể có được $4x+b_1y+c_1+d_1=0$ là mặt phẳng ở trường hợp $1$ và $2x+b_2y+c_2+d_2=0$ là mặt phẳng ở trường hợp $2$.\\
		Khi đó
		\begin{itemize}
			\item Chọn $a=4 \Rightarrow b_1=-2$; $c_1=-4$; $d_1=-6 \Rightarrow(\alpha)\colon 4 x-2y-4 z-6=0$.
			\item Với $a=2 \Rightarrow b=3 $; $c=2 $; $d=-3 \Rightarrow(\alpha)\colon 2 x+3 y+2 z-3=0$.
		\end{itemize}
		Vậy ta có tập $(-2;3)$ có tất cả $4$ giá trị nguyên là $-1$, $0$, $1$, $2$.
	}
\end{ex}
\Closesolutionfile{ans}
\indapan{6}{ans/ans-2-C5B1CD2-D3}
\begin{dang}{MỘT SỐ DẠNG KHÁC}

\end{dang}
\Opensolutionfile{ans}[ans/ans-2-C5B1CD2-D4]
\TN

\begin{ex}%[2H5H1-3]
	Trong không gian $O x y z$ cho điểm $M(1 ; 2 ; 3)$. Viết phương trình mặt phẳng $(P)$ đi qua điểm $M$ và cắt các trục tọa độ $O x$, $O y$, $O z$ lần lượt tại $A$, $B$, $C$ sao cho $M$ là trọng tâm của tam giác $A B C$.
	\choice
	{$(P)\colon 6 x+3 y+2 z+18=0$}
	{$(P)\colon 6 x+3 y+2 z+6=0$}
	{\True $(P)\colon 6 x+3 y+2 z-18=0$}
	{$(P)\colon 6 x+3 y+2 z-6=0$}
	\loigiai{
		Theo giả thiết $A \in O x$, $B \in O y$, $C \in O z$ nên ta có thể đặt $A(a ; 0 ; 0)$, $B(0 ; b ; 0)$, $C(0 ; 0 ; c)$.\\
		Vì $M(1 ; 2 ; 3)$ là trọng tâm tam giác $A B C$ nên $\heva{&a=3 \\ &b=6 \\ &c=9.}$\\
		Từ đó ta có phương trình mặt phẳng theo đoạn chắn là
		$$	(P)\colon \dfrac{x}{3}+\dfrac{y}{6}+\dfrac{z}{9}=1 \Leftrightarrow 6 x+3 y+2 z-18=0.		$$
	}
\end{ex}

\begin{ex}%[2H5H1-3]
	Trong không gian với hệ trục tọa độ $O x y z$, cho điểm $G(1 ; 4 ; 3)$. Mặt phẳng nào sau đây cắt các trục $O x$, $O y$, $O z$ lần lượt tại $A$, $B$, $C$ sao cho $G$ là trọng tâm tứ diện $O A B C$?
	\choice
	{$\dfrac{x}{3}+\dfrac{y}{12}+\dfrac{z}{9}=1$}
	{\True $12 x+3 y+4 z-48=0$}
	{$\dfrac{x}{4}+\dfrac{y}{16}+\dfrac{z}{12}=0$}
	{$12 x+3 y+4 z=0$}
	\loigiai{
		Mặt phẳng $(P)$ cắt các trục $O x$, $O y$, $O z$ lần lượt tại $A$, $B$, $C$ nên $A(a ; 0 ; 0)$, $B(0 ; b ; 0)$, $C(0 ; 0 ; c)$.
		Vì $G$ là trọng tâm tứ diện $O A B C$ nên $$\heva{&x_G=\dfrac{x_A+x_B+x_C+x_O}{4}=\dfrac{a}{4} \\& y_G=\dfrac{y_A+y_B+y_C+y_O}{4}=\dfrac{b}{4} \\& z_G=\dfrac{z_A+z_B+z_C+z_O}{4}=\dfrac{c}{4}} \Rightarrow\heva{&a=4 \\ &b=16 \\ &c=12.}$$
		Khi đó mặt phẳng $(P)$ có phương trình là $\dfrac{x}{4}+\dfrac{y}{16}+\dfrac{z}{12}=1$ hay $12 x+3 y+4 z-48=0$.\\
		Vậy mặt phẳng $(P)$ thỏa mãn là $12 x+3 y+4 z-48=0$.
	}
\end{ex}

\begin{ex}%[2H5V1-3]
	Viết phương trình mặt phẳng $(\alpha)$ đi qua $M(2 ; 1 ;-3)$, biết $(\alpha)$ cắt trục $O x$, $O y$, $O z$ lần lượt tại $A$, $B$, $C$ sao cho tam giác $A B C$ nhận $M$ làm trực tâm.
	\choice
	{$2 x+5 y+z-6=0$}
	{$2 x+y-6 z-23=0$}
	{\True $2 x+y-3 z-14=0$}
	{ $3 x+4 y+3 z-1=0$}
	\loigiai{
		Giả sử $A(a ; 0 ; 0)$, $B(0 ; b ; 0)$, $C(0 ; 0 ; c)$, $a b c \neq 0$.\\
		Khi đó mặt phẳng $(\alpha)$ có dạng $\dfrac{x}{a}+\dfrac{y}{b}+\dfrac{z}{c}=1$.\\
		Do $M \in(\alpha) \Rightarrow \dfrac{2}{a}+\dfrac{1}{b}-\dfrac{3}{c}=1$.\\
		Ta có $\overrightarrow{A M}=(2-a ; 1 ;-3)$, $\overrightarrow{B M}=(2 ; 1-b ;-3)$, $\overrightarrow{B C}=(0 ;-b ; c)$, $\overrightarrow{A C}=(-a ; 0 ; c)$.\\
		Do $M$ là trực tâm tam giác $A B C$ nên $\heva{&\overrightarrow{A M} \cdot \overrightarrow{B C}=0 \\ &\overrightarrow{B M} \cdot \overrightarrow{A C}=0}\Leftrightarrow\heva{&-b-3 c=0 \\ &-2 a-3 c=0} \Leftrightarrow\heva{&b=-3 c \\ &a=-\dfrac{3 c}{2}.}$\\
		Thay $(2)$ vào $(1)$ ta có $-\dfrac{4}{3 c}-\dfrac{1}{3 c}-\dfrac{3}{c}=1 \Leftrightarrow c=-\dfrac{14}{3} \Rightarrow a=7$, $b=14$.\\
		Do đó $(\alpha)\colon \dfrac{x}{7}+\dfrac{y}{14}-\dfrac{3 z}{14}=1 \Leftrightarrow 2 x+y-3 z-14=0$.
	}
\end{ex}

\begin{ex}%[2H5V1-3]
	Trong không gian với hệ trục toạ độ $Oxyz,$ điểm $M\left(a,b,c\right)$ thuộc mặt phẳng $(P)\colon x+y+z-6=0$ và cách đều các điểm $A\left(1;6;0\right)$, $B\left(-2;2;-1\right)$, $C\left(5;-1;3\right).$ Tích $abc$ bằng
	\choice
	{\True $6$}
	{$-6$}
	{$0$}
	{$5$}
	\loigiai{
		Ta có
		\begin{eqnarray*}
			&\heva{&a+b+c=6\\&MA^2=MB^2\\&MA^2=MC^2}&\Leftrightarrow\heva{&a+b+c=6\\&
				\left(a-1\right)^2+\left(b-6\right)^2+b^2=\left(a+2\right)^2+\left(b-2\right)^2+\left(c+1\right)^2\\&
				\left(a-1\right)^2+\left(b-6\right)^2+c^2=\left(a-5\right)^2+\left(b+1\right)^2+\left(c-3\right)^2}\\
				&	&\Leftrightarrow\heva{&	a+b+c=6\\&3a+4b+c=14\\&4a-7b+3b=-1}\\ &&\Leftrightarrow\heva{&a=1\\&b=2\\&c=3}\Rightarrow abc=6.
		\end{eqnarray*}}
\end{ex}

\begin{ex}%[2H5V1-3]
	Trong không gian với hệ tọa độ $Oxyz,$ cho điểm $M\left(3;2;1\right)$. Mặt phẳng $(P)$ đi qua $M$ và cắt các trục tọa độ $Ox$, $Oy$, $Oz$ lần lượt tại các điểm $A$, $B$, $C$ không trùng với gốc tọa độ sao cho $M$ là trực tâm tam giác $ABC$. Trong các mặt phẳng sau, tìm mặt phẳng song song với mặt phẳng $(P)$.
	\choice
	{\True $3x+2y+z+14=0$}
	{$2x+y+3z+9=0$}
	{$3x+2y+z-14=0$}
	{$2x+y+z-9=0$}
	\loigiai{
			Gọi $A\left(a;0;0\right);B\left(0;b;0\right);C\left(0;0;c\right)$.\\
			Phương trình mặt phẳng $(P)$ có dạng $\dfrac{x}{a}+\dfrac{y}{b}+\dfrac{z}{c}=1$ $\left(abc\ne 0\right)$.\\
			Vì $(P)$ qua $M$ nên $\dfrac{3}{a}+\dfrac{2}{b}+\dfrac{1}{c}=1\quad(1)$.\\
			Ta có $\overrightarrow{MA}=\left(a-3;-2;-1\right)$; $\overrightarrow{MB}=\left(-3;b-2;-1\right)$; $\overrightarrow{BC}=\left(0;-b;c\right)$; $\overrightarrow{AC}=\left(-a;0;c\right)$.\\
			Vì $M$ là trực tâm của tam giác $ABC$ nên $$\heva{&				\overrightarrow{MA}\cdot \overrightarrow{BC}=0\\
			&\overrightarrow{MB}\cdot \overrightarrow{AC}=0}\Leftrightarrow\heva{&		2b=c\\&	3a=c} \quad (2).$$
			Từ $(1)$ và $(2)$ suy ra $a=\dfrac{14}{3}$; $b=\dfrac{14}{2}$; $c=14$.\\
			Khi đó phương trình $(P)\colon 3x+2y+z-14=0$.\\
			Vậy mặt phẳng song song với $(P)$ là $3x+2y+z+14=0$.}
\end{ex}

\begin{ex}%[2H5V1-3]
	Trong không gian với hệ tọa độ $O x y z$, cho các điểm $A(0 ; 1 ; 2)$, $B(2 ;-2 ; 0)$, $C(-2 ; 0 ; 1)$. Mặt phẳng $(P)$ đi qua $A$, trực tâm $H$ của tam giác $A B C$ và vuông góc với mặt phẳng $(A B C)$ có phương trình là
	\choice
	{\True $4 x-2 y-z+4=0$}
	{$4 x-2 y+z+4=0$}
	{$4 x+2 y+z-4=0$}
	{$4 x+2 y-z+4=0$}
	\loigiai{
		Ta có $\overrightarrow{A B}=(2 ;-3 ;-2)$, $\overrightarrow{A C}=(-2 ;-1 ;-1)$ nên $\left[\overrightarrow{A B}, \overrightarrow{A C}\right]=(1 ; 6 ;-8)$.\\
		Phương trình mặt phẳng $(A B C)$ là $x+6 y-8 z+10=0$.\\
		Phương trình mặt phẳng qua $B$ và vuông góc với $A C$ là $2 x+y+z-2=0$.\\
		Phương trình mặt phẳng qua $C$ và vuông góc với $A B$ là $2 x-3 y-2 z+6=0$.\\
		Giao điểm của ba mặt phẳng trên là trực tâm $H$ của tam giác $A B C$ nên $H\left(-\dfrac{22}{101} ; \dfrac{70}{101} ; \dfrac{176}{101}\right)$.\\
		Mặt phẳng $(P)$ đi qua $A$, $H$ nên $\overrightarrow{n_P} \perp \overrightarrow{A H}=\left(-\dfrac{22}{101} ;-\dfrac{31}{101} ;-\dfrac{26}{101}\right)=-\dfrac{1}{101}(22 ; 31 ; 26)$.\\
		Mặt phẳng $(P) \perp(A B C)$ nên $\overrightarrow{n_P} \perp \overrightarrow{n}_{(A B C)}=(1 ; 6 ;-8)$.\\
		Vậy $\left[\overrightarrow{n}_{(A B C)} ; \overrightarrow{u}_{A H}\right]=(404 ;-202 ;-101)$ là một vectơ pháp tuyến của $(P)$.\\
		Chọn $\overrightarrow{n}_P=(4 ;-2 ;-1)$ nên phương trình mặt phẳng $(P)$ là $4 x-2 y-z+4=0$.
	}
\end{ex}

\begin{ex}%[2H5V1-3]
	Trong không gian với hệ tọa độ $O x y z$, viết phương trình mặt phẳng $(P)$ đi qua $A(1 ; 1 ; 1)$ và $B(0 ; 2 ; 2)$ đồng thời cắt các tia $O x$, $O y$ lần lượt tại hai điểm $M$, $N$ ( không trùng với gốc tọa độ $O$ ) sao cho $O M=2 O N$.
	\choice
	{$(P)\colon 3x+y+2z-6=0$}
	{$(P)\colon 2x+3y-z-4=0$}
	{$(P)\colon 2x+y+z-4=0$}
	{\True $(P)\colon x+2 y-z-2=0$}
	\loigiai{
		Giả sử $(P)$ đi qua 3 điểm $M(a ; 0 ; 0)$, $N(0 ; b ; 0)$, $P(0 ; 0 ; c)$.\\
		Suy ra $(P)\colon \dfrac{x}{a}+\dfrac{y}{b}+\dfrac{z}{c}=1$.\\
		Mà $(P)$ đi qua $A(1 ; 1 ; 1)$ và $B(0 ; 2 ; 2)$ nên ta có hệ $\heva{&\dfrac{1}{a}+\dfrac{1}{b}+\dfrac{1}{c}=1 \\ &\dfrac{2}{b}+\dfrac{2}{c}=1} \Leftrightarrow\heva{&a=2 \\ &\dfrac{2}{b}+\dfrac{2}{c}=1.}$\\
		Theo giả thuyết ta có $O M=2 O N \Leftrightarrow|a|=2|b| \Leftrightarrow|b|=1$.
		\begin{itemize}
			\item \textbf{TH1.} $b=1 \Rightarrow c=-2$ suy ra $(P)\colon x+2 y-z-2=0$.
			\item \textbf{TH2.} $b=-1 \Rightarrow c=-\dfrac{2}{3}$ suy ra $(P)\colon x-2 y+3 z-2=0$.
		\end{itemize}
	}
\end{ex}

\begin{ex}%[2H5V1-3]
	Trong không gian $O x y z$, cho mặt phẳng $(\alpha)$ đi qua điểm $M(1 ; 2 ; 3)$ và cắt các trục $O x$, $O y$, $O z$ lần lượt tại $A$, $B$, $C$ (khác gốc tọa độ $O$ ) sao cho $M$ là trực tâm tam giác $A B C$. Mặt phẳng $(\alpha)$ có phương trình dạng $a x+b y+c z-14=0$. Tính tổng $T=a+b+c$.
	\choice
	{$8$}
	{$14$}
	{\True $6$}
	{$11$}
	\loigiai{
		Do $M$ là trực tâm tam giác $ABC$, nên ta có
		\begin{itemize}
			\item $OA \perp BC$ và $AM \perp BC$ nên $(OAM) \perp BC \Rightarrow OM \perp BC$.
			\item $OB \perp AC$ và $BM \perp AC$ nên $(OBM) \perp AC \Rightarrow OM \perp AC$.
		\end{itemize}
		Từ đó ta được $OM \perp (ABC)$ nên $\overrightarrow{OM}=(1;2;3)$ là véc-tơ pháp tuyến của $(ABC)$.\\
		Vậy phương trình mặt phẳng $(ABC)$ là $$1\cdot (x-1)+2\cdot (y-2)+3\cdot (x-3)=0 \Leftrightarrow x+2y+3z-14=0.$$
		Dẫn đến $a=1$, $b=2$, $c=3$ nên $T=1+2+3=6$.
	}
\end{ex}

\begin{ex}%[2H5V1-3]
	Trong không gian $O x y z$, cho hai mặt phẳng $(P)\colon x+4 y-2 z-6=0$, $(Q)\colon x-2 y+4 z-6=0$. Mặt phẳng $(\alpha)$ chứa giao tuyến của $(P)$, $(Q)$ và cắt các trục tọa độ tại các điểm $A$, $B$, $C$ sao cho hình chóp $O. A B C$ là hình chóp đều. Phương trình mặt phẳng $(\alpha)$ là
	\choice
	{\True $x+y+z-6=0$}
	{$x+y+z+6=0$}
	{$x+y+z-3=0$}
	{$x+y-z-6=0$}
	\loigiai{
		Mặt phẳng $(P)\colon x+4 y-2 z-6=0$ có véc-tơ pháp tuyến $\overrightarrow{n_P}=(1 ; 4 ;-2)$.\\
		Mặt phẳng $(Q)\colon x-2 y+4 z-6=0$ có véc-tơ pháp tuyến $\overrightarrow{n_Q}=(1 ;-2 ; 4)$.\\
		Ta có $\left[\overrightarrow{n}_P ; \overrightarrow{n}_Q\right]=(12 ;-6 ;-6)$, cùng phương với $\overrightarrow{u}=(2 ;-1 ;-1)$.\\
		Gọi $d=(P) \cap(Q)$. Ta có đường thẳng $d$ có véc-tơ chỉ phương là $\overrightarrow{u}=(2 ;-1 ;-1)$ và đi qua điểm $M(6 ; 0 ; 0)$.\\
		Mặt phẳng $(\alpha)$ cắt các trục tọa độ tại các điểm $A(a ; 0 ; 0)$, $B(0 ; b ; 0)$, $C(0 ; 0 ; c)$ với $a b c \neq 0$.\\
		Phương trình mặt phẳng $(\alpha)\colon \dfrac{x}{a}+\dfrac{y}{b}+\dfrac{z}{c}=1$.\\
		Mặt phẳng $(\alpha)$ có véc-tơ pháp tuyến $\vec{n}=\left(\dfrac{1}{a} ; \dfrac{1}{b} ; \dfrac{1}{c}\right)$.\\
		Mặt phẳng $(\alpha)$ chứa $d$ nên $$\heva{&\vec{n} \perp \vec{u} \\ &M \in(\alpha)} \Leftrightarrow\heva{&\dfrac{2}{a}-\dfrac{1}{b}-\dfrac{1}{c}=0 \\ &\dfrac{6}{a}=1} \Leftrightarrow\heva{&a=6 \\ &\dfrac{1}{b}+\dfrac{1}{c}=\dfrac{1}{3}.\quad (*)}$$
		Ta lại có hình chóp $O.ABC$ là hình chóp đều $$\Leftrightarrow O A=O B=O C \Leftrightarrow|a|=|b|=|c| \Leftrightarrow|b|=|c|=6.$$
		Kêt hợp với điều kiện $(*)$ ta được $b=c=6$.\\
		Vậy phương trình của mặt phẳng $(\alpha)\colon \dfrac{x}{6}+\dfrac{y}{6}+\dfrac{z}{6}=1 \Leftrightarrow x+y+z-6=0$.
	}
\end{ex}
\begin{ex}%[2H5V1-3]
	Trong không gian tọa độ $O x y z$, cho mặt phẳng $(\alpha)$ đi qua $M(1 ;-3 ; 8)$ và chắn trên $O z$ một đoạn dài gấp đôi các đoạn chắn trên các tia $O x$, $O y$. Giả sử $(\alpha)\colon a x+b y+c z+d=0$ ($a$, $b$, $c$, $d$ là các số nguyên). Tính $S=\dfrac{a+b+c}{d}$.
	\choice
	{$3$}
	{$-3$}
	{$\dfrac{5}{4}$}
	{\True $-\dfrac{5}{4}$}
	\loigiai{
		Giả sử mặt phẳng $(\alpha)$ cắt các tia $Ox$, $Oy$, $Oz$ lần lượt tại $A(m ; 0 ; 0)$, $B(0 ; n ; 0)$, $C(0 ; 0 ; p)$ (với $m$, $n$, $p>0$).\\
		Theo giả thiết có $O C=2 O A=2 O B \Rightarrow p=2 m=2 n. \quad(1)$\\
		Phương trình mặt phẳng $(\alpha)$ có dạng $\dfrac{x}{m}+\dfrac{y}{n}+\dfrac{z}{p}=1. \quad (2)$\\
		Do mặt phẳng $(\alpha)$ đi qua $M(1 ;-3 ; 8)$ nên $\dfrac{1}{m}-\dfrac{3}{n}+\dfrac{8}{p}=1$.\\
		Thay $(1)$ vào $(2)$ ta được $\dfrac{1}{m}-\dfrac{3}{m}+\dfrac{8}{2 m}=1 \Leftrightarrow \dfrac{2}{m}=1 \Leftrightarrow m=2 \Rightarrow m=n=2,\ p=4$.
		Phương trình mặt phẳng $(\alpha)$ có dạng $\dfrac{x}{2}+\dfrac{y}{2}+\dfrac{z}{4}=1 \Leftrightarrow 2 x+2 y+z-4=0$.\\
		Từ đó suy ra $a=2 t$, $b=2 t$, $c=t$, $d=-4 t \quad(t \neq 0)$.\\
		Vậy $S=\dfrac{a+b+c}{d}=-\dfrac{5}{4}$.
	}
\end{ex}
\Closesolutionfile{ans}
\indapan{10}{ans/ans-2-C5B1CD2-D4}

\Opensolutionfile{ans}[ans/ans-2-C5B1CD2-D4]
\TNSA
\begin{ex}%[2H5V1-3]
	Trong không gian với hệ trục tọa độ $O x y z$, cho hai điểm $A(3 ; 1 ; 7)$, $B(5 ; 5 ; 1)$ và mặt phẳng $(P)\colon 2 x-y-z+4=0$. Điểm $M$ thuộc $(P)$ sao cho $M A=M B=\sqrt{35}$. Biết $M$ có hoành độ nguyên, tính $O M$ (làm tròn đến chữ số hàng phần trăm).
	\shortans{$2{,}83$}
	\loigiai{
		Gọi $M(a ; b ; c)$ với $a \in \mathbb{Z}$, $b \in \mathbb{R}$, $c \in \mathbb{R}$.\\
		Ta có $\overrightarrow{A M}=(a-3 ; b-1 ; c-7)$ và $\overrightarrow{B M}=(a-5 ; b-5 ; c-1)$.\\
		Vì $\heva{&M \in ( P )\\&M A = M B = \sqrt { 3 5 }}\Leftrightarrow \heva{&M \in(P) \\&M A^2=M B^2\\&M A^2=35}$ nên ta có hệ phương trình sau
		\begin{eqnarray*}
			\allowdisplaybreaks
			& &\heva{&2a - b - c + 4 = 0\\&(a-3)^ {2} + ( b - 1) ^ {2} + (c-7)^{2} = (a-5)^{2} + ( b - 5 ) ^ { 2 } + ( c - 1 ) ^ { 2 }\\&( a - 3 ) ^ { 2 } + ( b - 1 ) ^ { 2 } + ( c - 7 ) ^ { 2 } = 3 5 }\\
			&\Leftrightarrow &\heva{&2 a-b-c=-4 \\&4 a+8 b-12 c=-8 \\&(a-3)^2+(b-1)^2+(c-7)^2=35}\\
			&\Leftrightarrow &\heva{&b=c\\&c=a+2 \\&(a-3)^2+(b-1)^2+(c-7)^2=35}\Leftrightarrow \heva{&b=a+2 \\&c=a+2 \\&3a^2-14=0}
			\Leftrightarrow \heva{&a=0 \\&b=2 \ (\text{do }a \in \mathbb{Z})\\&c=2.}
		\end{eqnarray*}
		Ta có $M(2 ; 2 ; 0)$. Suy ra $O M=2 \sqrt{2}\approx 2{,}83$.\\
	}
\end{ex}
\begin{ex}%[2H5V1-3]
	Trong không gian với hệ tọa độ $O x y z$, mặt phẳng $(P)$ chứa điểm $M(1 ; 3 ;-2)$, cắt các tia $O x$, $O y$, $O z$ lần lượt tại $A$, $B$, $C$ sao cho $\dfrac{O A}{1}=\dfrac{O B}{2}=\dfrac{O C}{4}$. Biết phương trình mặt phẳng $(P)$ có dạng $ax+by+cz-8=0$. Tính $P=\dfrac{a+c}{2b}$ (kết quả được viết dưới dạng số thập phân).
	\shortans{$1{,}25$}
	\loigiai{
		Phương trình mặt chắn cắt tia $O x$ tại $A(a ; 0 ; 0)$, cắt tia $O y$ tại $B(0 ; b ; 0)$, cắt tia $O z$ tại $C(0 ; 0 ; c)$ có dạng là $(P)\colon \dfrac{x}{a}+\dfrac{y}{b}+\dfrac{z}{c}=1$ (với $a>0, b>0, c>0$).\\
		Theo đề $\dfrac{O A}{1}=\dfrac{O B}{2}=\dfrac{O C}{4} \Leftrightarrow \dfrac{a}{1}=\dfrac{b}{2}=\dfrac{c}{4} \Rightarrow\heva{&a=\dfrac{b}{2} \\ &c=2 b.}$\\
		Vì $M(1 ; 3 ;-2)$ nằm trên mặt phẳng $(P)$ nên ta có $$\dfrac{1}{\frac{b}{2}}+\dfrac{3}{b}+\dfrac{-2}{2 b}=1 \Leftrightarrow \dfrac{4}{b}=1 \Leftrightarrow b=4.$$
		Khi đó $a=2$, $c=8$.\\
		Vậy phương trình mặt phẳng $(P)$ là $\dfrac{x}{2}+\dfrac{y}{4}+\dfrac{z}{8}=1 \Leftrightarrow 4 x+2 y+z-8=0$.\\
		Khi đó $=\dfrac{a+c}{2b}=\dfrac{4+1}{2\cdot 2}=1{,}25$
	}
\end{ex}
\begin{ex}%[2H5V1-3]
	Trong không gian với hệ tọa độ $O x y z$ cho mặt phẳng $(P)$ đi qua điểm $M(9 ; 1 ; 1)$ cắt các tia $O x$, $O y$, $O z$ tại $A$, $B$, $C$ ($A$, $B$, $C$ không trùng với gốc tọa độ ). Thể tích tứ diện $O A B C$ đạt giá trị nhỏ nhất là bao nhiêu  (kết quả được viết dưới dạng số thập phân)?
	\shortans{$40{,}5$}
	\loigiai{
		Giả sử $A(a ; 0 ; 0)$, $B(0 ; b ; 0)$, $C(0 ; 0 ; c)$ với $a$, $b$, $c>0$.\\
		Mặt phẳng $(P)$ có phương trình ( theo đoạn chắn) $$\dfrac{x}{a}+\dfrac{y}{b}+\dfrac{z}{c}=1.$$
		Vì mặt phẳng $(P)$ đi qua điểm $M(9 ; 1 ; 1)$ nên $$\dfrac{9}{a}+\dfrac{1}{b}+\dfrac{1}{c}=1.$$
		Ta có $1=\dfrac{9}{a}+\dfrac{1}{b}+\dfrac{1}{c} \geq 3 \sqrt[3]{\dfrac{9}{abc}} \Rightarrow abc\geq 243$.\\
		$$
		V_{O A B C}=\dfrac{1}{6} abc \geq \dfrac{243}{6}=\dfrac{81}{2}.$$
		Vậy thể tích tứ diện $O A B C$ đạt giá trị nhỏ nhất là $\dfrac{81}{2}=40{,}5$.
	}
\end{ex}

