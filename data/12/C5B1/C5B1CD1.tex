\chapter{PHƯƠNG PHÁP TỌA ĐỘ TRONG KHÔNG GIAN}
\section{PHƯƠNG TRÌNH MẶT PHẲNG}
% \chude{Xác định các yếu tố cơ bản liên quan đến mặt phẳng}
\begin{dang}{Xác định véctơ pháp tuyến của mặt phẳng. Xác định điểm thuộc và không thuộc mặt phẳng}
	\begin{enumerate}[label=\bf\arabic*.]
		\item \textbf{véctơ pháp tuyến của mặt phẳng:}
		\begin{itemize}
			\item Mặt phẳng $(\alpha)\colon A x+B y+C z+D=0$ có véctơ pháp tuyến $\overrightarrow{n}=(A; B; C)$.
			\item Nếu mặt phẳng $(\alpha)$ có cặp véctơ chỉ phương là $\overrightarrow{a}, \overrightarrow{b}$ thì $(\alpha)$ có véctơ pháp tuyến là $\overrightarrow{n}=\left[\overrightarrow{a}, \overrightarrow{b}\right]$.
			\item véctơ pháp tuyến của mặt phẳng $(\alpha)$ là véctơ có giá vuông góc với $(\alpha)$.
			\item véctơ chỉ phương của mặt phẳng $(\alpha)$ là véctơ có giá song song hoặc trùng với $(\alpha)$.
			\item Nếu $\overrightarrow{n}$ là một véctơ pháp tuyến của $(\alpha)$ thì $k \cdot \overrightarrow{n}$ cũng là một véctơ pháp tuyến của $(\alpha)$.
			\item Nếu $\overrightarrow{a}$ là một véctơ chỉ phương của $(\alpha)$ thì $k \cdot \overrightarrow{a}$ cũng là một véctơ chỉ phương của $(\alpha)$.
			\item[] \textbf{Chú ý:}
			\item Trục $O x$ có véctơ chỉ phương là $\overrightarrow{i}=(1; 0; 0)$.
			\item Trục $O y$ có véctơ chỉ phương là $\overrightarrow{j}=(0; 1; 0)$.
			\item Trục $O z$ có véctơ chỉ phương là $\overrightarrow{k}=(0; 0; 1)$.
			\item Mặt phẳng $(O x y)$ có véctơ pháp tuyến là $\overrightarrow{k}=(0; 0; 1)$.
			\item Mặt phẳng $(O x z)$ có véctơ pháp tuyến là $\overrightarrow{j}=(0; 1; 0)$.
			\item Mặt phẳng $(O y z)$ có véctơ pháp tuyến là $\overrightarrow{i}=(1; 0; 0)$.
		\end{itemize}
		\item \textbf{Điểm thuộc và không thuộc mặt phẳng:}\\
		Cho mặt phẳng $(\alpha)$ có phương trình $A x+B y+C z+D=0$. Khi đó: 
		\begin{itemize}
			\item $N_0\left(x_0; y_0; z_0\right) \in(\alpha) \Leftrightarrow A x_0+B y_0+C z_0+D=0$.
			\item $N_0\left(x_0; y_0; z_0\right) \notin(\alpha) \Leftrightarrow A x_0+B y_0+C z_0+D \neq 0$.	
		\end{itemize}
	\end{enumerate}
\end{dang}

\TN
\Opensolutionfile{ans}[ans/ans2C5B1CD1]
\begin{ex}%[2H2H2-5]
	Trong không gian $O x y z$, tọa độ một véctơ $\overrightarrow{n}$ vuông góc với cả hai véctơ $\overrightarrow{a}=(1; 1;-2), \overrightarrow{b}=(1; 0; 3)$ là
	\choice
	{$(2; 3;-1)$}
	{$(3; 5;-2)$}
	{$(2;-3;-1)$}
	{\True $(3;-5;-1)$}
	\loigiai{
		véctơ $\overrightarrow{n}$ vuông góc với cả hai véctơ $\overrightarrow{a}, \overrightarrow{b}$.\\
		Do đó $\overrightarrow{n}=\left[\overrightarrow{a}, \overrightarrow{b}\right]$.\\
		Ta có $\left[\overrightarrow{a}, \overrightarrow{b}\right]=(3;-5;-1)$.	
	}
\end{ex}

\begin{ex}%[2H2H2-5]
	Trong không gian với hệ tọa độ $O x y z$, cho hai véctơ $\overrightarrow{a}=(2; 1;-2)$ và véctơ $\overrightarrow{b}=(1; 0; 2)$. Tìm tọa độ véctơ $\overrightarrow{c}$ là tích có hướng của $\overrightarrow{a}$ và $\overrightarrow{b}$.
	\choice
	{$\overrightarrow{c}=(2; 6;-1)$}
	{$\overrightarrow{c}=(4; 6;-1)$}
	{$\overrightarrow{c}=(4;-6;-1)$}
	{\True $\overrightarrow{c}=(2;-6;-1)$}
	\loigiai{
		Áp dụng công thức tính tích có hướng trong hệ trục tọa độ $O x y z$, ta được
		$$\overrightarrow{c}=\left[\overrightarrow{a}, \overrightarrow{b}\right]=(2;-6;-1).$$	
	}
\end{ex}

\begin{ex}%[2H2H2-5]
	Trong không gian với hệ trục tọa độ $O x y z$, cho $A(2; 1;-3), B(0;-2; 5)$ và $C(1; 1; 3)$. Tìm tọa độ véctơ $\overrightarrow{n}$ có phương vuông góc với hai véctơ $\overrightarrow{A B}$ và $\overrightarrow{A C}$.
	\choice
	{$\overrightarrow{n}=(8; 4;-3)$}
	{$\overrightarrow{n}=(-18; 0;-3)$}
	{\True $\overrightarrow{n}=(-18; 4;-3)$}
	{$\overrightarrow{n}=(1; 4;-3)$}
	\loigiai{
		Ta có $\overrightarrow{A B}=(-2;-3; 8)$ và $\overrightarrow{A C}=(-1; 0; 6)$. Suy ra $\left[\overrightarrow{A B}, \overrightarrow{A C}\right]=(-18; 4;-3)$.\\
		Vậy $\overrightarrow{n}=\left[\overrightarrow{A B}, \overrightarrow{A C}\right]=(-18; 4;-3)$.
	}
\end{ex}

\begin{ex}%[2H5N1-3]
	Trong không gian $O x y z$, phương trình nào sau đây là phương trình tổng quát của mặt phẳng?
	\choice
	{$x-3 y^2+z-1=0$}
	{$x^2+2 y+4 z-2=0$}
	{\True $2 x-3 y+4 z-2024=0$}
	{$2 x-3 y+4 z^2-2025=0$}
	\loigiai{
		Phương trình tổng quát của mặt phẳng là $2 x-3 y+4 z-2024=0$.	
	}
\end{ex}

\begin{ex}%[2H5H1-3]
	Trong không gian $O x y z$, cho mặt phẳng $(P)\colon 3 x-y+2 z-1=0$. véctơ nào dưới đây \textbf{không phải} là một véctơ pháp tuyến của $(P)$?
	\choice
	{$\overrightarrow{n}=(-3; 1;-2)$}
	{\True $\overrightarrow{n}=(3; 1; 2)$}
	{$\overrightarrow{n}=(3;-1; 2)$}
	{$\overrightarrow{n}=(6;-2; 4)$}
	\loigiai{
		Véctơ pháp tuyến của $(P)$ là $\overrightarrow{n}=(3;-1; 2)$.\\
		$\overrightarrow{n}=(-3; 1;-2)=-1(3;-1; 2)$ là một véctơ pháp tuyến của $(P)$.\\
		$\overrightarrow{n}=(6;-2; 4)=2(3;-1; 2)$ là một véctơ pháp tuyến của $(P)$. 	
	}
\end{ex}

\begin{ex}%[2H5H1-3]
	Trong không gian với hệ tọa độ $O x y z$, véctơ nào dưới đây là một véctơ pháp tuyến của mặt phẳng $(O x y)$?
	\choice
	{$\overrightarrow{i}=(1; 0; 0)$}
	{$\overrightarrow{m}=(1; 1; 1)$}
	{$\overrightarrow{j}=(0; 1; 0)$}
	{\True $\overrightarrow{k}=(0; 0; 1)$}
	\loigiai{
		Do mặt phẳng $(O x y)$ vuông góc với trục $O z$ nên nhận véctơ $\overrightarrow{k}=(0; 0; 1)$ làm một véctơ pháp tuyến.	
	}
\end{ex}

\begin{ex}%[2H5H1-3]
	Trong không gian $O x y z$, véctơ nào dưới đây có giá vuông góc với mặt phẳng $(\alpha)\colon 2 x-3 y+1=0$?
	\choice
	{$\overrightarrow{a}=(2;-3; 1)$}
	{$\overrightarrow{b}=(2; 1;-3)$}
	{\True $\overrightarrow{c}=(2;-3; 0)$}
	{$\overrightarrow{d}=(3; 2; 0)$}
	\loigiai{
		Mặt phẳng $(\alpha)$ có một véctơ pháp tuyến là $\overrightarrow{n}=(2;-3; 0)=\overrightarrow{c}$.	
	}
\end{ex}

\begin{ex}%[2H5H1-3]
	Trong không gian $O x y z$, một véctơ pháp tuyến của mặt phẳng $\dfrac{x}{-2}+\dfrac{y}{-1}+\dfrac{z}{3}=1$ là
	\choice
	{\True $\overrightarrow{n}=(3; 6;-2)$}
	{$\overrightarrow{n}=(2;-1; 3)$}
	{$\overrightarrow{n}=(-3;-6;-2)$}
	{$\overrightarrow{n}=(-2;-1; 3)$}
	\loigiai{
		Phương trình $\dfrac{x}{-2}+\dfrac{y}{-1}+\dfrac{z}{3}=1 \Leftrightarrow-\dfrac{1}{2} x-y+\dfrac{1}{3} z-1=0 \Leftrightarrow 3 x+6 y-2 z+6=0.$\\
		Do đó mặt phẳng đã cho có một véctơ pháp tuyến là $\overrightarrow{n}=(3; 6;-2)$.	
	}
\end{ex}

\begin{ex}%[2H5H1-3]
	Trong không gian $O x y z$, điểm nào dưới đây nằm trên mặt phẳng $(P)\colon 2 x-y+z-2=0$.
	\choice
	{$Q(1;-2; 2)$}
	{$P(2;-1;-1)$}
	{$M(1; 1;-1)$}
	{\True $N(1;-1;-1)$}
	\loigiai{
		Thay toạ độ điểm $Q$ vào phương trình mặt phẳng $(P)$ ta được $2\cdot 1-(-2)+2-2=4 \neq 0$ nên $Q \notin(P)$.\\
		Thay toạ độ điểm $P$ vào phương trình mặt phẳng $(P)$ ta được $2\cdot2-(-1)+(-1)-2=2 \neq 0$ nên $P \notin(P)$.\\
		Thay toạ độ điểm $M$ vào phương trình mặt phẳng $(P)$ ta được $2\cdot1-1+(-1)-2=-2 \neq 0$ nên $M \notin(P)$.\\
		Thay toạ độ điểm $N$ vào phương trình mặt phẳng $(P)$ ta được $2 \cdot 1-(-1)+(-1)-2=0$ nên $N \in(P)$.	
	}
\end{ex}

\begin{ex}%[2H5H1-3]
	Trong không gian với hệ tọa độ $O x y z$, cho mặt phẳng $(\alpha)\colon x+y+z-6=0$. Điểm nào dưới đây \textbf{không thuộc} $(\alpha)$?
	\choice
	{$Q(3; 3; 0)$}
	{$N(2; 2; 2)$}
	{$P(1; 2; 3)$}
	{\True $M(1;-1; 1)$}
	\loigiai{
		\begin{itemize}
			\item  Thay $Q(3; 3; 0)$  vào phương trình mặt phẳng $(\alpha)$, ta được $3+3+0-6=0 \Rightarrow Q \in(\alpha)$.
			\item  Thay $N(2; 2; 2)$ vào phương trình mặt phẳng $(\alpha)$, ta được  $2+2+2-6=0 \Rightarrow N \in(\alpha)$.
			\item Thay $P(1; 2; 3)$ vào phương trình mặt phẳng $(\alpha)$, ta được $1+2+3-6=0 \Rightarrow P \in(\alpha)$.
			\item  Thay $M(1;-1; 1)$ toạ độ vào phương trình mặt phẳng $(\alpha)$, ta được $1-1+1-6\neq 0 \Rightarrow M \notin(\alpha)$.	 
		\end{itemize}
	}
\end{ex}

\begin{ex}%[2H5H1-3]
	Trong không gian với hệ tọa độ $O x y z$, cho mặt phẳng $(P)\colon x-2 y+z-5=0$. Điểm nào dưới đây thuộc $(P)$?
	\choice
	{$P(0; 0;-5)$}
	{\True $M(1; 1; 6)$}
	{$Q(2;-1; 5)$}
	{$N(-5; 0; 0)$}
	\loigiai{
		Ta có $1-2 \cdot 1+6-5=0$ nên $M(1; 1; 6)$ thuộc mặt phẳng $(P)$.	
	}
\end{ex}

\begin{ex}%[2H5H1-3]
	Trong không gian $O x y z$, mặt phẳng $(P)\colon \dfrac{x}{1}+\dfrac{y}{2}+\dfrac{z}{3}=1$ \textbf{không} đi qua điểm nào dưới đây?
	\choice
	{$P(0; 2; 0)$}
	{\True $N(1; 2; 3)$}
	{$M(1; 0; 0)$}
	{$Q(0; 0; 3)$}
	\loigiai{
		Thế tọa độ điểm $N$ vào phương trình mặt phẳng $(P)$ ta có $\dfrac{1}{1}+\dfrac{2}{2}+\dfrac{3}{3}=1$ (sai).\\
		Vậy mặt phẳng $(P)\colon \dfrac{x}{1}+\dfrac{y}{2}+\dfrac{z}{3}=1$ không đi qua điểm $N(1; 2; 3)$.	
	}
\end{ex}

\begin{ex}%[2H5H1-3]
	Trong không gian $O x y z$, mặt phẳng $(\alpha)\colon x-y+2 z-3=0$ đi qua điểm nào dưới đây?
	\choice
	{\True $M\left(1; 1; \dfrac{3}{2}\right)$}
	{$N\left(1;-1;-\dfrac{3}{2}\right)$}
	{$P(1; 6; 1)$}
	{$Q(0; 3; 0)$}
	\loigiai{
		Xét điểm $M\left(1; 1; \dfrac{3}{2}\right)$, ta có $1-1+2 \cdot \dfrac{3}{2}-3=0$ (đúng) nên $M \in(\alpha)$ .\\
		Xét điểm $N\left(1;-1;-\dfrac{3}{2}\right)$, ta có $1+1+2.\left(-\dfrac{3}{2}\right)-3=0$ (sai) nên $N \notin(\alpha)$.\\
		Xét điểm $P(1; 6; 1)$, ta có $1-6+2.1-3=0$ (sai) nên $P \notin(\alpha)$.\\
		Xét điểm $Q(0; 3; 0)$, ta có $0-3+2.0-3=0$ (sai) nên $Q \notin(\alpha)$.	
	}
\end{ex}
\Closesolutionfile{ans}
\indapan{10}{ans/ans2C5B1CD1}
\TNTF
\Opensolutionfile{ans}[ans/ans2C5B1CD1-DS]
\begin{ex}%[2H5H1-2]
	Trong không gian cho hệ tọa độ $O x y z$. Các mệnh đề sau đây đúng hay sai?
	\choiceTF
	{\True Mặt phẳng $(O x y)$ có một véctơ pháp tuyến là $\overrightarrow{n}=(0; 0; 1)$}
	{\True Mặt phẳng $(O x z)$ có véctơ pháp tuyến là $\overrightarrow{n}=(0; 3; 0)$}
	{\True Mặt phẳng $(O y z)$ có véctơ pháp tuyến là $\overrightarrow{n}=(-2; 0; 0)$}
	{\True Trục $O z$ có véctơ chỉ phương là $\overrightarrow{a}=(0; 0;-2024)$}
	\loigiai{
		\begin{itemchoice}
			\itemch Mặt phẳng $(O x y)$ có một véctơ pháp tuyến là $\overrightarrow{n}=(0; 0; 1)$.
			\itemch Mặt phẳng $(O x z)$ có véctơ pháp tuyến là $\overrightarrow{n}=(0; 3; 0)$.
			\itemch Mặt phẳng $(O y z)$ có véctơ pháp tuyến là $\overrightarrow{n}=(-2; 0; 0)$.
			\itemch Trục $O z$ có véctơ chỉ phương là $\overrightarrow{a}=(0; 0;-2024)$.
		\end{itemchoice}
	}
\end{ex}

\begin{ex}%[2H2H2-5]%[2H2H2-1] 
	Trong không gian với hệ toạ độ $O x y z$, cho $\overrightarrow{a}=(1;-2; 3)$ và $\overrightarrow{b}=(1; 1;-1)$. Các mệnh đề sau đây đúng hay sai? 
	\choiceTF
	{\True $\left|\overrightarrow{a}+\overrightarrow{b}\right|=3$}
	{\True $\overrightarrow{a} \cdot \overrightarrow{b}=-4$}
	{\True $\left|\overrightarrow{a}-\overrightarrow{b}\right|=5$}
	{$\left[\overrightarrow{a}, \overrightarrow{b}\right]=(-1;-4; 3)$}
	\loigiai{
		\begin{itemchoice}
			\itemch $\left|\overrightarrow{a}+\overrightarrow{b}\right|=\left|\overrightarrow{a}+\overrightarrow{b}\right|=\sqrt{(1+1)^2+(-2+1)^2+(3-1)^2}=\sqrt{4+1+4}=3$.
			\itemch $\overrightarrow{a} \cdot \overrightarrow{b}=1 \cdot 1+(-2) \cdot 1+3 \cdot(-1)=1-2-3=-4$.
			\itemch $\left|\overrightarrow{a}+\overrightarrow{b}\right|=\left|\overrightarrow{a}+\overrightarrow{b}\right|=\sqrt{(1-1)^2+(-2-1)^2+(3+1)^2}=\sqrt{0+9+16}=5$.
			\itemch 
			$\left[\overrightarrow{a}, \overrightarrow{b}\right]=\left(\left|\begin{array}{cc}-2 & 3 \\ 1 &-1\end{array}\right|;\left|\begin{array}{cc}3 & 1 \\-1 & 1\end{array}\right|;\left|\begin{array}{cc}1 &-2 \\ 1 & 1\end{array}\right|\right)=(-1; 4; 3)$.
		\end{itemchoice}
	}
\end{ex}
\begin{ex}%[2H2H2-4]%[2H2H2-1]
	Trong không gian với hệ trục tọa độ $O x y z$, cho ba véctơ $\overrightarrow{a}=(1; 2;-1), \overrightarrow{b}=(3;-1; 0), \overrightarrow{c}=(1;-5; 2)$. Các mệnh đề sau đây đúng hay sai?
	\choiceTF
	{$\overrightarrow{a}$ cùng phương với $\overrightarrow{b}$}
	{$\left[\overrightarrow{a}, \overrightarrow{b}\right] \cdot \overrightarrow{c}=0$}
	{$\overrightarrow{a}$ không cùng phương với $\overrightarrow{b}$}
	{$\overrightarrow{a}$ vuông góc với $\overrightarrow{b}$}
	\loigiai{   
		\begin{itemchoice}
			\itemch Ta có:
			$\left[\overrightarrow{a}, \overrightarrow{b}\right]=(-1;-3;-7) \neq \overrightarrow{0}$.
			\itemch Hai véctơ $\overrightarrow{a}, \overrightarrow{b}$ không cùng phương.
			\itemch $\left[\overrightarrow{a}, \overrightarrow{b}\right] \cdot \overrightarrow{c}=-1+15-14=0$.
			\itemch Ba véctơ $\overrightarrow{a}, \overrightarrow{b}, \overrightarrow{c}$ đồng phẳng.
		\end{itemchoice}
	}
\end{ex}
\begin{ex}%[2H5H1-2]
	Trong không gian $O x y z$, cho mặt phẳng $(P)\colon 2 x+3 y+z-2024=0$. Các mệnh đề sau đây đúng hay sai?
	\choiceTF
	{\True Mặt phẳng $(P)$ có một véctơ pháp tuyến là $\overrightarrow{n}=(2; 3; 1)$}
	{\True Mặt phẳng $(P)$ có véctơ pháp tuyến là $\overrightarrow{n}=(6; 9; 3)$}
	{\True Mặt phẳng $(P)$ có véctơ pháp tuyến là $\overrightarrow{n}=(-4;-6;-2)$}
	{Điểm $M(0; 0; 2024)$ không thuộc mặt phẳng $(P)$}
	\loigiai{
		\begin{itemchoice}
			\itemch Véctơ pháp tuyến của $(P)$ là $\overrightarrow{n}=(2; 3; 1)$.
			\itemch $\overrightarrow{n}=(6; 9; 3)=3(2; 3; 1).$
			\itemch $\overrightarrow{n}=(-4;-6;-2)=-2(2; 3; 1).$
			\itemch Thay điểm $M(0; 0; 2024)$ vào mặt phẳng $(P)\colon 2\cdot0+3\cdot 0+2024-2024=0 \Rightarrow M \in(P)$.
		\end{itemchoice}
	}
\end{ex}
\begin{ex}%[2H5H1-3] 
	Trong không gian $O x y z$, cho mặt phẳng $(P)\colon x+y+z-3=0$. Các mệnh đề sau đây đúng hay sai?
	\choiceTF
	{\True Điểm $M(-1;-1;-1)$ \textbf{không thuộc} mặt phẳng $(P)$}
	{\True Điểm $N(1; 1; 1)$ \textbf{thuộc} mặt phẳng $(P)$}
	{\True Điểm $K(-3; 0; 0)$ \textbf{không thuộc} mặt phẳng $(P)$}
	{Điểm $Q(0; 0;-3)$ \textbf{thuộc} mặt phẳng $(P)$}
	\loigiai{
		\begin{itemchoice}
			\itemch Điểm $M(-1;-1;-1)$ có tọa độ không thỏa mãn phương trình mặt phẳng $(P)$ nên $M \notin(P)$.
			\itemch Điểm $N(1; 1; 1)$ có tọa độ thỏa mãn phương trình mặt phẳng $(P)$ nên $N \in(P)$.
			\itemch Điểm $K(-3; 0; 0)$ có tọa độ không thỏa mãn phương trình mặt phẳng $(P)$ nên $K \notin(P)$.
			\itemch Điểm $Q(0; 0;-3)$ có tọa độ không thỏa mãn phương trình mặt phẳng $(P)$ nên $Q \notin(P)$.
		\end{itemchoice}
	}
\end{ex}
\Closesolutionfile{ans}
\indapan{2}{ans/ans2C5B1CD1-DS}
\TNSA
\Opensolutionfile{ans}[ans/ans2C5B1CD1-KQ]
\begin{ex}%[2H2H2-5]
	Trong không gian với hệ trục tọa độ $O x y z$, cho $A(0; 1;-1)$, $B(1; 1; 2)$ và $C(1;-1; 0)$. Biết  $\vec{u}=\left[\overrightarrow{B C}, \overrightarrow{B D}\right]$. Khi đó, độ dài của $\vec{u}$ bằng bao nhiêu?
	\shortans[0]{$4$}
	\loigiai{
		Ta có $\overrightarrow{B C}=(0;-2;-2)$ và  $\overrightarrow{B D}=(-1;-1;-1)$.\\
		Khi đó $\vec{u}=\left[\overrightarrow{B C}, \overrightarrow{B D}\right]=(0; 2;-2)$.\\
		Suy ra $\left|\vec{u}\right|=\sqrt{0^2+2^2+(-2)^2}=4$. 	
	}
\end{ex}

\begin{ex}%[2H2V2-5]
	Trong không gian với hệ trục tọa độ $Oxyz$, cho $A(2; 0; 2)$, $B(1;-1;-2)$ và $C(-1; 1; 0)$. Một véctơ $\overrightarrow{n}=(a; b; 2)$ có phương vuông góc với hai véctơ $\overrightarrow{AB}$ và $\overrightarrow{AC}$. Tính giá trị của $a+b$.
	\shortans[0]{$-8$}
	\loigiai{
		Ta có $\overrightarrow{A C}=(-3; 1;-2)$ và $\overrightarrow{A B}=(-1;-1;-4)$.\\
		Vì $\vec{n}$ có phương vuông góc với $\overrightarrow{AB}$ và $\overrightarrow{AC}$ nên $\vec{n}$ cùng phương với vectơ $\left[\overrightarrow{AB},\overrightarrow{AC}\right]=(-6;-10; 4)$.\\
		Suy ra $\overrightarrow{n}=(-3; -5; 2)$
		Vậy $a+b=-3-5=-8$.
	}
\end{ex}

\begin{ex}%[2H2V2-5]
	Hệ trục tọa độ $Oxyz$, cho bốn điểm $A(1;-2; 0)$, $B(2; 0; 3)$, $C(-2; 1; 3)$ và $D(0; 1; 1)$. Tính giá trị của phép tính $\left[\overrightarrow{AB}, \overrightarrow{AC}\right] \cdot \overrightarrow{AD}$.
	\shortans[0]{$-24$}
	\loigiai
	{
		Ta có $\overrightarrow{AB}=(1; 2; 3)$; $\overrightarrow{AC}=(-3; 3; 3)$; $\overrightarrow{A D}=(-1; 3; 1)$.\\
		Khi đó $\left[\overrightarrow{A B}, \overrightarrow{A C}\right]=(-3;-12; 9)$.\\
		Và $\left[\overrightarrow{A B}, \overrightarrow{A C}\right] \cdot \overrightarrow{A D}=(-3) \cdot(-1)+(-12) \cdot 3+9 \cdot 1=-24$.
	}
\end{ex}
\begin{ex}%[2H5H1-2] 
	Trong mặt phẳng tọa độ $O x y z$, mặt phẳng $(P)\colon 2 x-6 y-8 z+1=0$ có một véctơ pháp tuyến $\vec{n}=(1;a;b)$. Khi đó tổng $a+b$ bằng bao nhiêu? 
	\shortans[0]{$-7$}
	\loigiai
	{
		Phương trình tổng quát của mặt phẳng $(P)\colon 2 x-6 y-8 z+1=0$ nên một véctơ pháp tuyến của mặt phẳng $(P)$ có tọa độ là $(2;-6;-8)=2\cdot (1;-3;-4)$.\\
		Suy ra $\vec{n}=(1;-3;-4)$, nên $a+b=-3-4=-7$.	
	}
\end{ex}

\begin{ex}%[2H2V2-5]
	Trong không gian với hệ tọa độ $O x y z$, cho $\overrightarrow{u}=(1; 1; 2), \overrightarrow{v}=(-1; m; m-2)$. Tìm giá trị của $m$ dương sao cho $|[\overrightarrow{u}, \overrightarrow{v}]|=\sqrt{14}$.
	\shortans[0]{$1$}
	\loigiai
	{ Ta có {\allowdisplaybreaks
			\begin{eqnarray*}
				&& [\overrightarrow{u}, \overrightarrow{v}]=(-m-2;-m; m+1)\\ &\Rightarrow& |[\overrightarrow{u}, \overrightarrow{v}]|=\sqrt{(m+2)^2+m^2+(m+1)^2}=\sqrt{3 m^2+6 m+5}.
		\end{eqnarray*}}
		Khi đó $$|[\overrightarrow{u}, \overrightarrow{v}]|=\sqrt{14} \Leftrightarrow 3 m^2+6 m+5=14 \Leftrightarrow 3 m^2+6 m-9=0 \Leftrightarrow \hoac{&m=1 \\&m=-3.}$$
		
	}
\end{ex}

\begin{ex}%[2H2V2-5]
	Trong không gian với hệ tọa độ $O x y z$, cho hai véctơ $\overrightarrow{m}=(4; 3; 1), \overrightarrow{n}=(0; 0; 1)$. Gọi $\overrightarrow{p}=\left(a;b;c\right)$ là véctơ cùng hướng với $[\overrightarrow{m}, \overrightarrow{n}]$ (tích có hướng của hai véctơ $\overrightarrow{m}$ và $\overrightarrow{n}$). Biết $|\overrightarrow{p}|=15$, giá trị của tổng $a+b+c$ bằng bao nhiêu?
	\shortans[0]{$3$}
	\loigiai
	{
		Ta có  $[\overrightarrow{m}; \overrightarrow{n}]=(3;-4; 0)$, suy ra $|[\overrightarrow{m}; \overrightarrow{n}]|=5$.\\
		Do $\overrightarrow{p}$ là véctơ cùng hướng với $[\overrightarrow{m}; \overrightarrow{n}]$ nên $\overrightarrow{p}=k[\overrightarrow{m}; \overrightarrow{n}]$, $k>0$.\\
		Mặt khác $|\overrightarrow{p}|=15 \Leftrightarrow k \cdot|[\overrightarrow{m}, \overrightarrow{n}]| =15 \Leftrightarrow k\cdot 5=15 \Leftrightarrow k=3$.\\
		Suy ra $\overrightarrow{p}=(9;-12; 0)$.	\\
		Vậy $a+b+c=9-12+0=3$.
	}
\end{ex}
\Closesolutionfile{ans}
\indapan{6}{ans/ans2C5B1CD1-KQ}
\begin{dang}{Hai mặt phẳng song song, vuông góc. Khoảng cách một điểm đến mặt phẳng}
	\begin{enumerate}[label=\bf\arabic*.]
		\item \textbf{Điều kiện hai mặt phẳng song song, vuông góc:}\\
		Cho 2 mặt phẳng $\left(\alpha_1\right)\colon A_1 x+B_1 y+C_1 z+D_1=0$ và $\left(\alpha_2\right)\colon A_2 x+B_2 y+C_2 z+D_2=0$ có vectơ pháp tuyến lần lượt là $\overrightarrow{n}_1=\left(A_1; B_1; C_1\right), \overrightarrow{n}_2=\left(A_2; B_2; C_2\right)$. Khi đó:
		\begin{itemize}
			\item $\left(\alpha_1\right) \parallel \left(\alpha_2\right) \Leftrightarrow\heva{&\overrightarrow{n}_1=k \overrightarrow{n}_2 \\ &D_1 \neq k D_2} \quad (k \in \mathbb{R})$.
			\item $\left(\alpha_1\right) \equiv\left(\alpha_2\right) \Leftrightarrow\heva{&\overrightarrow{n}_1=k \overrightarrow{n}_2 \\& D_1=k D_2} \quad (k \in \mathbb{R})$.
			\item $\left(\alpha_1\right)$ cắt $\left(\alpha_2\right) \Leftrightarrow \overrightarrow{n}_1$ và $\overrightarrow{n}_2$ không cùng phương.
			\item $\left(\alpha_1\right) \perp\left(\alpha_2\right) \Leftrightarrow \overrightarrow{n}_1 \cdot \overrightarrow{n}_2=0 \Leftrightarrow A_1 A_2+B_1 B_2+C_1 C_2=0$. 	
		\end{itemize}
		\begin{tikzpicture}[line cap=round,line join=round,>=stealth,x=1.0cm,y=1.0cm,scale=0.6]
			\path
			(1,1) coordinate (A)
			(3,3) coordinate (B)
			(8,3) coordinate (C)
			($(A)+(C)-(B)$) coordinate (D)
			(2,4) coordinate (A')
			(4,6) coordinate (B')
			(9,6) coordinate (C')
			($(A')+(C')-(B')$) coordinate (D')
			(5,5) coordinate (K)
			(6,5) coordinate (M)
			($(K)+(0,1.5)$) coordinate (N)
			(6,2) coordinate (H)
			($(C)!0.5!(D)$) coordinate (H')
			;
			\draw (A)--(B)--(C)--(D)--cycle (A')--(B')--(C')--(D')--cycle ;
			\draw[->] (K)--(N) node[right]{$\vec{n}_1$};
			\draw[->] (H)--($(H)+(0,1.5	 )$) node[right]{$\vec{n}_2$};	
			\draw pic[draw,blue,"$\alpha_1$",angle radius=8mm]{angle=D--A--B};
			\draw pic[draw,blue,"$\alpha_2$",angle radius=8mm]{angle=D'--A'--B'};
			\draw pic[draw,blue,,angle radius=3mm]{right angle=M--K--N};
			\draw pic[draw,blue,,angle radius=3mm]{right angle=M--H--H'};
		\end{tikzpicture}
		\begin{tikzpicture}[line cap=round,line join=round,>=stealth,x=1.0cm,y=1.0cm,scale=0.6]
			\path
			(1,1) coordinate (A)
			(3,3) coordinate (B)
			(8,3) coordinate (C)
			($(A)+(C)-(B)$) coordinate (D)
			($(A)+(-2,3)$) coordinate (E)
			($(B)+(-2,3)$) coordinate (F)
			($(A)!0.5!(C)$) coordinate (G)
			($(G)+(0,1.5)$) coordinate (H)
			($(A)!0.5!(F)$) coordinate (I)
			($(I)+(1.7,1.5)$) coordinate (J)
			($(C)!0.5!(D)$) coordinate (K)
			($(A)!0.5!(B)$) coordinate (L)
			;
			\draw (A)--(B)--(C)--(D)--cycle (A)--(E)--(F)--(B) ;
			\draw[->] (G)--(H) node[right]{$\vec{n}_1$};
			\draw[->] (I)--(J) node[right]{$\vec{n}_2$};	
			\draw pic[draw,blue,"$\alpha_1$",angle radius=8mm]{angle=B--C--D};
			\draw pic[draw,blue,"$\alpha_2$",angle radius=5mm]{angle=A--E--F};
			\draw pic[draw,blue,,angle radius=3mm]{right angle=L--I--J};
			\draw pic[draw,blue,,angle radius=3mm]{right angle=H--G--K};
		\end{tikzpicture}
		\begin{tikzpicture}[line cap=round,line join=round,>=stealth,x=1.0cm,y=1.0cm,scale=0.6]
			\path
			(1,1) coordinate (A)
			(3,3) coordinate (B)
			(8,3) coordinate (C)
			($(A)+(C)-(B)$) coordinate (D)
			($(A)+(0,4)$) coordinate (E)
			($(B)+(0,4.)$) coordinate (F)
			($(A)!0.5!(C)$) coordinate (G)
			($(G)+(0,1.5)$) coordinate (H)
			($(A)!0.5!(F)$) coordinate (I)
			($(I)+(1.7,0)$) coordinate (J)
			($(C)!0.5!(D)$) coordinate (K)
			($(A)!0.5!(B)$) coordinate (L)
			;
			\draw (A)--(B)--(C)--(D)--cycle (A)--(E)--(F)--(B) ;
			\draw[->] (G)--(H) node[right]{$\vec{n}_1$};
			\draw[->] (I)--(J) node[above]{$\vec{n}_2$};	
			\draw pic[draw,blue,"$\alpha_1$",angle radius=8mm]{angle=B--C--D};
			\draw pic[draw,blue,"$\alpha_2$",angle radius=5mm]{angle=A--E--F};
			\draw pic[draw,blue,,angle radius=3mm]{right angle=L--I--J};
			\draw pic[draw,blue,,angle radius=3mm]{right angle=H--G--K};
		\end{tikzpicture}
		\begin{note}
			\textbf{Chú ý:}
			\begin{itemize}
				\item $\overrightarrow{a}$ cùng phương với $\overrightarrow{b} \Leftrightarrow[\overrightarrow{a}, \overrightarrow{b}]=\overrightarrow{0}$.
				\item Nếu $\overrightarrow{n}=[\overrightarrow{a}, \overrightarrow{b}]$ thì vectơ $\overrightarrow{n}$ vuông góc với cả hai vectơ $\overrightarrow{a}$ và $\overrightarrow{b}$.
			\end{itemize}
		\end{note}
		\item \textbf{Khoảng cách từ một điểm đến một mặt phẳng}
		\immini{
			Trong không gian $O x y z$, cho điểm $M_0\left(x_0; y_0; z_0\right)$ và mặt phẳng $(\alpha)\colon A x+B y+C z+D=0$. Khi đó khoảng cách từ điểm $M_0$ đến mặt phẳng $(\alpha)$ được tính: $$d\left(M_0,(\alpha)\right)=\dfrac{\left|A x_0+B y_0+C z_0+D\right|}{\sqrt{A^2+B^2+C^2}}.$$
		}{
			\begin{tikzpicture}[line cap=round,line join=round,>=stealth,x=1.0cm,y=1.0cm,scale=0.6]
				\path
				(1,1) coordinate (A)
				(3,3) coordinate (B)
				(9,3) coordinate (C)
				($(A)+(C)-(B)$) coordinate (D)
				(4,2) coordinate (E)
				(6,2) coordinate (F)
				($(E)+(0,2.5)$) coordinate (G)
				;
				\draw (A)--(B)--(C)--(D)--cycle ;
				\draw[->] (E)--($(E)+(0,2.5)$) node[right]{$M_0$};
				\draw[->] (F)--($(F)+(0,1.5)$) node[right]{$\vec{n}$};	
				\draw pic[draw,blue,"$\alpha$",angle radius=8mm]{angle=D--A--B};
				\draw pic[draw,blue,,angle radius=3mm]{right angle=F--E--G};
				%			\draw pic[draw,blue,,angle radius=3mm]{right angle=H--G--K};
			\end{tikzpicture}
		}
		\begin{note}
			\textbf{Chú ý:}
			\begin{itemize}
				\item Mặt phẳng $(O x y)$ có phương trình: $z=0$.
				\item Mặt phẳng $(O x z)$ có phương trình: $y=0$.
				\item Mặt phẳng $(O y z)$ có phương trình: $x=0$.
			\end{itemize}
		\end{note}
		\item \textbf{Khoảng cách hai mặt phẳng song song}\\
		Khoảng cách giữa mặt phẳng song song là khoảng cách từ một điểm thuộc mặt phẳng này đến mặt phẳng kia (Thực chất là khoảng cách từ một điểm đến mặt phẳng).\\
		Để tính khoảng cách mặt phẳng $\left(\alpha_1\right)$ song song với $\left(\alpha_2\right)$, ta thực hiện như sau:
		\begin{enumerate}
			\item[] \textbf{Bước 1:} Chọn điểm $M \in\left(\alpha_1\right)$.
			\item[] \textbf{Bước 2:} Tính khoảng cách điểm $M$ đến $\left(\alpha_2\right)$.
			\item[] \textbf{Bước 3:} Kết luận: $d\left(\left(\alpha_1\right),\left(\alpha_2\right)\right)=d\left(M,\left(\alpha_2\right)\right)$.
		\end{enumerate}
		\begin{note}
			\textbf{Chú ý:}
			Cho 2 mặt phẳng $\left(\alpha_1\right)\colon A x+B y+C z+D_1=0$ và $\left(\alpha_2\right)\colon A x+B y+C z+D_2=0$ có cùng vectơ pháp tuyến là $\overrightarrow{n}=(A; B; C)$. Khi đó khoảng cách giữa hai mặt phẳng đó là: $$d\left(\left(\alpha_1\right),(\alpha_2)\right)=\dfrac{\left|D_1-D_2\right|}{\sqrt{A^2+B^2+C^2}}.$$ 
		\end{note}
	\end{enumerate}
\end{dang}
\textbf{Khoảng cách hai mặt phẳng song song}
\begin{itemize}
	\item Khoảng cách giữa mặt phẳng song song là khoảng cách từ một điểm thuộc mặt phẳng này đến mặt phẳng kia (Thực chất là khoảng cách từ một điểm đến mặt phẳng).
	\item Để tính khoảng cách mặt phẳng $(\alpha_1)$ song song với $(\alpha_2)$, ta thực hiện như sau:
	\begin{enumEX}[\hspace*{1cm}\bf Bước 1:]{1}
		\item Chọn điểm $M\in (\alpha_1)$
		\item Tính khoảng cách điểm $M$ đến $(\alpha_2)$
		\item Kết luận $\mathrm{d}\left((\alpha_1),(\alpha_2)\right)=\mathrm{d}\left(M,(\alpha_2)\right)$
	\end{enumEX}
	\textbf{Chú ý:} Cho 2 mặt phẳng $(\alpha_1)\colon Ax+By+Cz+D_1=0$ và $(\alpha_2)\colon Ax+By+Cz+D_2=0$ có cùng vectơ pháp tuyến là $\vec{n}=(A;B;C)$.\\
	Khi đó khoảng cách giữa hai mặt phẳng đó là: $\mathrm{d}((\alpha_1),(\alpha))=\dfrac{|D_1-D_2|}{\sqrt{A^2+B^2+C^2}}$.
\end{itemize}
\TN
\Opensolutionfile{ans}[ans/ans2C5B1CD1-D2]
%%==========Câu 27
\begin{ex}%[Câu 2]%[2H5N1-5]
	Khoảng cách từ điểm $M\left(3;2;1\right)$ đến mặt phẳng $(P)\colon Ax+Cz+D=0$, $A.C.D\ne 0$. Chọn khẳng định đúng trong các khẳng định sau:
	\choice
	{\True $\mathrm{d}(M,(P))=\dfrac{\left| 3A+C+D\right|}{\sqrt{A^2+C^2}}$}
	{$\mathrm{d}(M,(P))=\dfrac{\left| A+2B+3C+D\right|}{\sqrt{A^2+B^2+C^2}}$}
	{$\mathrm{d}(M,(P))=\dfrac{\left| 3A+C\right|}{\sqrt{A^2+C^2}}$}
	{$\mathrm{d}(M,(P))=\dfrac{\left| 3A+C+D\right|}{\sqrt{3^2+1^2}}$}
	\loigiai{
		Áp dung công thức $\mathrm{d}(M_0,(\alpha))=\dfrac{\left |Ax_0+By_0+Cz_0+D\right |}{\sqrt{A^2+B^2+C^2}}$.\\
		Ta được: $\mathrm{d}(M,(P))=\dfrac{\left| 3A+C+D\right|}{\sqrt{A^2+C^2}}$.}
\end{ex}

%%==========Câu 28
\begin{ex}%[Câu 3]%[2H5N1-5]
	Trong không gian với hệ tọa độ $Oxyz$, cho mặt phẳng $(P)$ có phương trình: $3x+4y+2z+4=0$ và điểm $A(1;-2;3)$. Tính khoảng cách $\mathrm{d}$ từ $A$ đến $(P)$.
	\choice
	{$\mathrm{d}=\dfrac{5}{9}$}
	{$\mathrm{d}=\dfrac{5}{29}$}
	{\True $\mathrm{d}=\dfrac{5}{\sqrt{29}}$}
	{$\mathrm{d}=\dfrac{\sqrt{5}}{3}$}
	\loigiai{
		Khoảng cách $\mathrm{d}$ từ $A$ đến $(P)$ là $$\mathrm{d}(A,(P))=\dfrac{\left| 3x_A+4y_A+2z_A+4\right|}{\sqrt{3^2+4^2+2^2}}=\dfrac{\left| 3-8+6+4\right|}{\sqrt{29}}=\dfrac{5}{\sqrt{29}}.$$}
\end{ex}

%%==========Câu 29
\begin{ex}%[Câu 4]%[2H5N1-5]
	Trong không gian $Oxyz$, cho mặt phẳng $(P)\colon 2x-2y+z-1=0$. Khoảng cách từ điểm $M\left(-1;2;0\right)$ đến mặt phẳng $(P)$ bằng
	\choice
	{$5$}
	{$2$}
	{\True $\dfrac{5}{3}$}
	{$\dfrac{4}{3}$}
	\loigiai{
		Ta có: $\mathrm{d}\left(M,(P)\right)=\dfrac{\left| 2\cdot\left(-1\right)-2\cdot2+0-1\right|}{\sqrt{2^2+\left(-2\right)^2+1^2}}=\dfrac{5}{3}$.}
\end{ex}

%%==========Câu 30
\begin{ex}%[Câu 5]%[2H5N1-5]
	Trong không gian $Oxyz$, tính khoảng cách từ $M\left(1;2;-3\right)$ đến mặt phẳng $(P)\colon x+2y+2z-10=0$.
	\choice
	{\True $\dfrac{11}{3}$}
	{$3$}
	{$\dfrac{7}{3}$}
	{$\dfrac{4}{3}$}
	\loigiai{
		Ta có: $\mathrm{d}\left(M;(P)\right)=\dfrac{\left| 1+2\cdot 2+2\cdot\left(-3\right)-10\right|}{\sqrt{1^2+2^2+2^2}}=\dfrac{\left| -11\right|}{3}=\dfrac{11}{3}$.}
\end{ex}

%%==========Câu 31
\begin{ex}%[Câu 6]%[2H5H1-5]
	Trong không gian $Oxyz$, cho mặt phẳng $(P)\colon 2x-y+2z-4=0$. Gọi $H$ là hình chiếu vuông góc của điểm $M\left(3;1;-2\right)$ lên mặt phẳng $(P)$. Độ dài đoạn thẳng $MH$ là
	\choice
	{$2$}
	{$\dfrac{1}{3}$}
	{\True $1$}
	{$3$}
	\loigiai{
		Độ dài đoạn thẳng $MH$ là $MH=\mathrm{d}\left(M,(P)\right)=\dfrac{\left| 2\cdot 3-1+2\cdot (-2)-4\right|}{\sqrt{2^2+(-1)^2+2^2}}=1$.}
\end{ex}

%%==========Câu 32
\begin{ex}%[Câu 7]%[2H5H1-5]
	Trong không gian với hệ trục tọa độ $Oxyz$, gọi $H$ là hình chiếu vuông góc của điểm $A(1;-2;3)$ lên mặt phẳng $(P)\colon 2x-y-2z+5=0$. Độ dài đoạn thẳng $AH$ bằng
	\choice
	{$3$}
	{$7$}
	{$4$}
	{$1$}
	\loigiai{
		Độ dài đoạn thẳng $AH$ là $AH=\mathrm{d}\left(A,(P)\right)=\dfrac{\left| 2+2-6+5\right|}{\sqrt{2^2+(-1)^2+(-2)^2}}=1$.}
\end{ex}

%%==========Câu 33
\begin{ex}%[Câu 8]%[2H5H1-5]
	Khoảng cách từ điểm $M(-4;-5;6)$ đến mặt phẳng $(Oxy)$, $(Oyz)$ lần lượt bằng
	\choice
	{\True $6$ và $4$}
	{$6$ và $5$}
	{$5$ và $4$}
	{$4$ và $6$}
	\loigiai{
		Ta có: $\mathrm{d}\left(M,(Oxy)\right)=\left|z_M\right|=6$ và $\mathrm{d}(M,(Oyz))=\left|x_M\right|=4$.}
\end{ex}

%%==========Câu 34
\begin{ex}%[Câu 9]%[2H5H1-5]
	Tính khoảng cách $\mathrm{d}$ từ điểm $B\left(x_0;y_0;z_0\right)$ đến mặt phẳng $(P)\colon y + 1 = 0$ ta được:
	\choice
	{$y_0$}
	{$\left| y_0\right|$}
	{$\dfrac{\left| y_0+1\right|}{\sqrt{2}}$}
	{\True $\left| y_0+1\right|$}
	\loigiai{
		Ta có: $\mathrm{d}\left (M,(P)\right )=\dfrac{\left |y_0+1\right |}{\sqrt{1^2}}=\left| y_0+1\right|$.
	}
\end{ex}

%%==========Câu 35
\begin{ex}%[Câu 10]%[2H5H1-5]
	Khoảng cách từ điểm $C(-2;0;0)$ đến mặt phẳng $(Oxy)$ bằng
	\choice
	{\True $0$}
	{$2$}
	{$1$}
	{$\sqrt{2}$}
	\loigiai{
		Điểm $C$ thuộc mặt phẳng $(Oxy)$ nên $\mathrm{d}\left(C,(Oxy)\right)=0$.}
\end{ex}

%%==========Câu 36
\begin{ex}%[Câu 11]%[2H5H1-5]
	Trong không gian $Oxyz$, khoảng cách giữa hai mặt phẳng $(P)\colon x+2y+2z-10=0$ và $(Q)\colon x+2y+2z-3=0$ bằng
	\choice
	{$\dfrac{4}{3}$}
	{$\dfrac{8}{3}$}
	{\True $\dfrac{7}{3}$}
	{$3$}
	\loigiai{
		Ta có $\dfrac{1}{1}=\dfrac{2}{2}=\dfrac{2}{2}\ne \dfrac{-10}{-3}$ nên $(P)\parallel (Q)$.\\
		Lấy $A\left(2;1;3\right)\in \left(P\right)$. 
		Ta có: $\mathrm{d}\left(\left(P\right),\left(Q\right)\right)=\mathrm{d}\left(A,\left(Q\right)\right)=\dfrac{\left| 2+2\cdot 1+2\cdot3-3\right|}{\sqrt{1^2+2^2+2^2}}=\dfrac{7}{3}$.}
\end{ex}

%%==========Câu 37
\begin{ex}%[Câu 12]%[2H5H1-5]
	Trong không gian $Oxyz$, khoảng cách giữa hai mặt phẳng $(P)\colon x+2y+3z-1=0$ và $(Q)\colon x+2y+3z+6=0$ là
	\choice
	{\True $\dfrac{7}{\sqrt{14}}$}
	{$\dfrac{8}{\sqrt{14}}$}
	{$14$}
	{$\dfrac{5}{\sqrt{14}}$}
	\loigiai{
		Ta có $\dfrac{1}{1}=\dfrac{2}{2}=\dfrac{3}{3}\ne \dfrac{-1}{6}$ nên $(P)\parallel (Q)$.\\
		Khi đó: $\mathrm{d}\left((P);(Q)\right)$ =$\dfrac{\left| D_2-D_1\right|}{\sqrt{A^2+B^2+C^2}}
		=\dfrac{\left| -1-6\right|}{\sqrt{1^2+2^2+3^2}}=\dfrac{7}{\sqrt{14}}$.}
\end{ex}

%%==========Câu 38
\begin{ex}%[Câu 13]%[2H5H1-5]
	Trong không gian $Oxyz$, khoảng cách giữa hai mặt phẳng $(P)\colon x+2y+2z-8=0$ và $(Q)\colon x+2y+2z-4=0$ bằng
	\choice
	{$1$}
	{\True $\dfrac{4}{3}$}
	{$2$}
	{$\dfrac{7}{3}$}
	\loigiai{
		Ta có $\dfrac{1}{1}=\dfrac{2}{2}=\dfrac{2}{2}\ne \dfrac{-8}{-4}$ nên $(P)\parallel (Q)$.\\
		Khi đó: $\mathrm{d}\left((P);(Q)\right)=\dfrac{\left| -8-(-4)\right|}{\sqrt{1^2+2^2+2^2}}=\dfrac{4}{3}$.\\
	}
\end{ex}

%%==========Câu 39
\begin{ex}%[Câu 14]%[2H5H1-4]
	Trong không gian $Oxyz$, mặt phẳng $(P)\colon 2x+y+z-2=0$ vuông góc với mặt phẳng nào dưới đây?
	\choice
	{$2x-y-z-2=0$}
	{\True $x-y-z-2=0$}
	{$x+y+z-2=0$}
	{$2x+y+z-2=0$}
	\loigiai{
		Mặt phẳng $(P)$ có một vectơ pháp tuyến $\overrightarrow{n_P}=\left(2;1;1\right)$.\\
		Mặt phẳng $(Q)\colon x-y-z-2=0$ có một vectơ pháp tuyến $\overrightarrow{n_Q}=\left(1;-1;-1\right)$.\\
		Mà $\overrightarrow{n_P}\cdot\overrightarrow{n_Q}=2-1-1=0\Rightarrow \overrightarrow{n_P}\perp \overrightarrow{n_Q}\Rightarrow (P)\perp (Q)$.\\
		Vậy mặt phẳng $(Q)\colon x-y-z-2=0$ là mặt phẳng cần tìm.}
\end{ex}

%%==========Câu 40
\begin{ex}%[Câu 15]%[2H5H1-4]
	Trong không gian với hệ tọa độ $Oxyz$, cho hai mặt phẳng $(P)\colon 2x+my+3z-5=0$ và $(Q)\colon nx-8y-6z+2=0$, với $m,n\in \mathbb{R}$. Xác định $m,n$ để $(P)$ song song với $(Q)$.
	\choice
	{$m=n=-4$}
	{\True $m=4;n=-4$}
	{$m=- 4;n=4$}
	{$m=n=4$}
	\loigiai{
		Mặt phẳng $(P)$ có véc tơ pháp tuyến $\vec{n_1}=(2;m;3)$.\\
		Mặt phẳng $(Q)$ có véc tơ pháp tuyến $\vec{n_2}=(n;-8;-6)$.\\
		Mặt phẳng $(P)\parallel (Q)\Rightarrow \vec{n_1}=k\cdot \vec{n_2}\, (k\in \mathbb{R})\Leftrightarrow \heva{&2=kn \\&m=- 8k \\&3=- 6k}\Leftrightarrow \heva{&k=-\dfrac{1}{2} \\&m=4 \\&n=- 4.}$}
\end{ex}

%%==========Câu 41
\begin{ex}%[Câu 16]%[2H5H1-4]
	Trong không gian $Oxyz$, cho hai mặt phẳng $(P)\colon x-2y+2z-3=0$ và $(Q)\colon mx+y-2z+1=0$. Với giá trị nào của $m$ thì hai mặt phẳng đó vuông góc với nhau?
	\choice
	{$m=1$}
	{$m=-1$}
	{$m=-6$}
	{\True $m=6$}
	\loigiai{
		Ta có: $(P)\perp (Q)\Leftrightarrow 1\cdot m-2\cdot 1+2\cdot (-2)=0\Leftrightarrow m=6$.}
\end{ex}

%%==========Câu 42
\begin{ex}%[Câu 17]%[2H5V1-4]
	Trong không gian $Oxyz$, cho ba mặt phẳng $(P)\colon x+y+z-1=0$, $(Q)\colon 2x+my+2z+3=0$ và $(R)\colon -x+2y+nz=0$. Tính tổng $m+2n$, biết rằng $(P)\perp (R)$ và $(P)\parallel (Q)$.
	\choice
	{$-6$}
	{$1$}
	{\True $0$}
	{$6$}
	\loigiai{
		$(P)$ có vectơ pháp tuyến $\vec{a}=(1;1;1)$.\\
		$(Q)$ có vectơ pháp tuyến $\vec{b}=(2;m;2)$.\\
		$(R)$ có vectơ pháp tuyến $\vec{c}=(-1;2;n)$.\\
		Ta có: $(P)\perp (R)\Leftrightarrow \vec{a}\cdot \vec{c}=0\Leftrightarrow n=-1$.\\
		$(P)\parallel (Q)\Leftrightarrow \dfrac{2}{1}=\dfrac{m}{1}=\dfrac{2}{1}\Leftrightarrow m=2$.\\
		Vậy $m+2n=2+2\left(-1\right)=0$}
\end{ex}

%%==========Câu 43
\begin{ex}%[Câu 18]%[2H5V1-4]
	Trong không gian $Oxyz$, cho $(P)\colon x+y-2z+5=0$ và $(Q)\colon 4x+(2-m)y+mz-3=0$, $m$ là tham số thực. Tìm tham số $m$ sao cho mặt phẳng $(Q)$ vuông góc với mặt phẳng $(P)$.
	\choice
	{$m=-3$}
	{$m=-2$}
	{$m=3$}
	{\True $m=2$}
	\loigiai{
		Mặt phẳng $(P)$ có véctơ pháp tuyến là $\vec{n_{P}}=(1;1;-2)$.\\
		Mặt phẳng $(Q)$ có véctơ pháp tuyến là $\vec{n_{Q}}=(4;2-m;m)$.\\
		Ta có $(P)\perp (Q)\Leftrightarrow \vec{n_{P}}\perp \vec{n_{Q}}\Leftrightarrow \vec{n_{P}}\cdot \vec{n_{Q}}=0\Leftrightarrow 4\cdot 1+2-m-2m=0\Leftrightarrow m=2$.
	}
\end{ex}

%%==========Câu 44
\begin{ex}%[Câu 19]%[2H5V1-4]
	Trong không gian $Oxyz$ cho hai mặt phẳng $(\alpha)\colon x+2y-z-1=0$ và $(\beta)\colon 2x+4y-mz-2=0$. Tìm $m$ để hai mặt phẳng $(\alpha)$ và $(\beta)$ song song với nhau.
	\choice
	{$m=1$}
	{\True Không tồn tại $m$}
	{$m=-2$}
	{$m=2$}
	\loigiai{
		Ta có vectơ pháp tuyến của $(\alpha)$ là $\overrightarrow{n_1}=(1;2;-1)$, vectơ pháp tuyến của $(\beta)$ là $\overrightarrow{n_2}=(2;4;-m)$.\\
		Hai mặt phẳng $(\alpha)$ và $(\beta)$ song song khi $\dfrac{2}{1}=\dfrac{4}{2}=\dfrac{-m}{-1}\ne \dfrac{-2}{-1}$.\\
		Vậy không có giá trị nào của $m$ thỏa mãn điều kiện trên.}
\end{ex}

%%==========Câu 45
\begin{ex}%[Câu 20]%[2H5V1-4]
	Trong không gian toạ độ $Oxyz$, cho mặt phẳng $(P)\colon x+2y-2z-1=0$, mặt phẳng nào dưới đây song song với $(P)$ và cách $(P)$ một khoảng bằng $3$.
	\choice
	{\True $(Q)\colon x+2y-2z+8=0$}
	{$(Q)\colon x+2y-2z+5=0$}
	{$(Q)\colon x+2y-2z+1=0$}
	{$(Q)\colon x+2y-2z+2=0$}
	\loigiai{
		+ Chọn $A\left(1;0;0\right)\in (P)$.\\
		+ Xét đáp án \textbf{A.}, ta có $\mathrm{d}\left(A;(Q)\right)=\dfrac{\left| 1+8\right|}{\sqrt{1^2+2^2+\left(-2\right)^2}}=3$.
	}
\end{ex}
\Closesolutionfile{ans}
\indapan{10}{ans/ans2C5B1CD1-D2}
\TNTF
\Opensolutionfile{ans}[ans/ans2C5B1CD1-D2-DS]
%%==========Câu 46
\begin{ex}%[Câu 21]%[2H5N1-5]
	Trong không gian toạ độ $Oxyz$, cho điểm $M\left(1;2;0\right)$ và các mặt phẳng $(Oxy)$, $(Oyz)$, $(Oxz)$. Các mệnh đề sau đây đúng hay \textbf{sai}?
	\choiceTF
	{\True $\mathrm{d}\left(M,(Oxz)\right)=2$}
	{\True $\mathrm{d}\left(M,(Oyz)\right)=1$}
	{$\mathrm{d}\left(M,(Oxy)\right)=1$}
	{\True $\mathrm{d}\left(M,(Oxz)\right)>d\left(M,(Oyz)\right)$}
	\loigiai{
		\begin{itemchoice}
			\itemch $\mathrm{d}\left(M,(Oxz)\right)=|2|=2$.	ĐÚNG
			\itemch $\mathrm{d}\left(M,(Oyz)\right)=|1|=1$. ĐÚNG
			\itemch $\mathrm{d}\left(M,(Oxy)\right)=|0|=0$.	SAI
			\itemch $\mathrm{d}\left(M,(Oxz)\right)>d\left(M,(Oyz)\right)$. ĐÚNG
		\end{itemchoice}
	}
\end{ex}
%%==========Câu 48
\begin{ex}%[Câu 23]%[2H5H1-4]
	Trong không gian $Oxyz$, cho hai mặt phẳng $(P)\colon x+2y-2z-6=0$ và $(Q)\colon x+2y-2z+3=0$. Các mệnh đề sau đây đúng hay \textbf{sai}?
	\choiceTF
	{\True Hai mặt phẳng $(P)$ và $(Q)$ song song với nhau}
	{Hai mặt phẳng $(P)$ và $(Q)$ vuông góc với nhau}
	{Khoảng cách giữa hai mặt phẳng $(P)$ và $(Q)$ bằng $2$}
	{\True Khoảng cách giữa hai mặt phẳng $(P)$ và $(Q)$ bằng $3$}
	\loigiai{
		\begin{itemize}
			\item Ta có: $\dfrac{1}{1}=\dfrac{2}{2}=\dfrac{-2}{-2}\ne \dfrac{-6}{3}$ nên $(P)\parallel (Q)$.
			\item $\mathrm{d}\left ((P),(Q)\right )=\dfrac{|-6-3|}{\sqrt{1^2+2^2+(-2)^2}}=3$.
		\end{itemize}
		\begin{itemchoice}
			\itemch Hai mặt phẳng $(P)$ và $(Q)$ song song với nhau. ĐÚNG
			\itemch Hai mặt phẳng $(P)$ và $(Q)$ vuông góc với nhau. SAI
			\itemch Khoảng cách giữa hai mặt phẳng $(P)$ và $(Q)$ bằng $2$. SAI
			\itemch Khoảng cách giữa hai mặt phẳng $(P)$ và $(Q)$ bằng $3$. ĐÚNG
		\end{itemchoice}
	}
\end{ex}

%%==========Câu 47
\begin{ex}%[2H5N1-5]%[2H5H1-5]
	Trong không gian toạ độ $Oxyz$, Biết khoảng cách từ điểm $O$ đến mặt phẳng $(Q)$ bằng 1. Các mệnh đề sau đây đúng hay \textbf{sai}?
	\choiceTF
	{Mặt phẳng $(Q)$ có phương trình là $x + y + z-3 = 0$}
	{\True Mặt phẳng $(Q)$ có phương trình là $2x + y + 2z-3 = 0$}
	{Mặt phẳng $(Q)$ có phương trình là $2x + y- 2z + 6 = 0$}
	{\True Mặt phẳng $(Q)$ có phương trình là $x + 2y + 2z-3= 0$}
	\loigiai{
		\begin{itemchoice}
			\itemch {Ta có $\mathrm{d}(O,(Q))=\dfrac{|-3|}{\sqrt{1^2+1^2+1^2}}=\sqrt{3}\ne 1$. SAI}
			\itemch {Ta có $\mathrm{d}(O,(Q))=\dfrac{|-3|}{\sqrt{2^2+1^2+2^2}}= 1$. ĐÚNG}
			\itemch {Ta có $\mathrm{d}(O,(Q))=\dfrac{|6|}{\sqrt{2^2+1^2+(-2)^2}}=2\ne 1$. SAI}
			\itemch {Ta có $\mathrm{d}(O,(Q))=\dfrac{|-3|}{\sqrt{1^2+2^1+2^2}}=1$. ĐÚNG
			}
		\end{itemchoice}
	}
\end{ex}

%%==========Câu 49
\begin{ex}%[Câu 24]%[2H5H1-4]
	Trong không gian $Oxyz$, cho điểm $N(0;1;0)$ và hai mặt phẳng $(P)\colon 2x-y-2z-9=0$, $(Q)\colon 4x-2y-4z-6=0$. Các mệnh đề sau đây đúng hay \textbf{sai}?
	\choiceTF
	{\True Hai mặt phẳng $(P)$ và $(Q)$ song song với nhau}
	{Khoảng cách từ điểm $N$ đến mặt phẳng $(Q)$ bằng $\dfrac{1}{2}$}
	{\True Khoảng cách giữa hai mặt phẳng $(P)$ và $(Q)$ bằng $2$}
	{Khoảng cách giữa hai mặt phẳng $(P)$ và $(Q)$ bằng $3$}
	\loigiai{
		\begin{itemize}
			\item Ta có $\dfrac{2}{4}=\dfrac{-1}{-2}=\dfrac{-2}{-4}\ne \dfrac{-9}{-6}$ nên $(P)\parallel (Q)$.	
			\item $\mathrm{d}\left(N,\left(Q\right)\right)=\dfrac{\left| -2\cdot 1-6\right|}{\sqrt{4^2+\left(-2\right)^2+\left(-4\right)^2}}=\dfrac{4}{3}$.
			\item $\mathrm{d}\left((P),(Q)\right)=\dfrac{|-9-(-3)|}{\sqrt{2^2+(-1)^2+(-2)^2}}=2$.
		\end{itemize}
		\begin{itemchoice}
			\itemch Hai mặt phẳng $(P)$ và $(Q)$ song song với nhau.	ĐÚNG
			\itemch Khoảng cách điểm đến mặt phẳng $(Q)$ bằng $\dfrac{1}{2}$.	SAI
			\itemch Khoảng cách giữa hai mặt phẳng $(P)$ và $(Q)$ bằng $2$. ĐÚNG
			\itemch Khoảng cách giữa hai mặt phẳng $(P)$ và $(Q)$ bằng $3$. SAI
		\end{itemchoice}
	}
\end{ex}

%%==========Câu 50
\begin{ex}%[Câu 25]%[2H5H1-5]
	Khoảng cách từ điểm $A(2;4;3)$ đến mặt phẳng $(\alpha)\colon 2x+y+2z+1=0$ và $(\beta)\colon x=0$ lần lượt là $\mathrm{d}(A,(\alpha))$, $\mathrm{d}(A,(\beta))$. Các mệnh đề sau đây đúng hay \textbf{sai}?
	\choiceTF
	{$\mathrm{d}\left(A,(\alpha)\right)=3\cdot \mathrm{d}\left(A,(\beta)\right)$}
	{$\mathrm{d}\left(A,(\alpha)\right)>\mathrm{d}\left(A,(\beta)\right)$}
	{$\mathrm{d}\left(A,(\alpha)\right)=\mathrm{d}\left(A,(\beta)\right)$}
	{\True $2\cdot\mathrm{d}\left(A,(\alpha)\right) = \mathrm{d}\left(A,(\beta)\right)$}
	\loigiai{
		Ta có: $\mathrm{d}\left(A,(\alpha)\right)=\dfrac{\left| 2.x_A+y_A+2.z_A+1\right|}{\sqrt{2^2+1^2+2^2}}=1$ và $\mathrm{d}\left(A,(\beta)\right)=\dfrac{\left|x_A\right|}{\sqrt{1^2}}=2$.\\
		Kết luận: $\mathrm{d}\left(A,(\beta)\right)=2\cdot \mathrm{d}\left(A,(\alpha)\right)$.
		\begin{itemchoice}
			\itemch $\mathrm{d}\left(A,(\alpha)\right)=3\cdot \mathrm{d}\left(A,(\beta)\right)$. SAI
			\itemch $\mathrm{d}\left(A,(\alpha)\right)>\mathrm{d}\left(A,(\beta)\right)$. SAI
			\itemch $\mathrm{d}\left(A,(\alpha)\right) =\mathrm{d}\left(A,(\beta)\right)$. SAI
			\itemch $2\cdot \mathrm{d}\left(A,(\alpha)\right)=\mathrm{d}\left(A,(\beta)\right)$. ĐÚNG
		\end{itemchoice}
	}
\end{ex}

%%==========Câu 51
\begin{ex}%Câu 51.%[2H5H1-4]
	Trong không gian $Oxyz$, cho điểm $I(2; 6;-3)$ và các mặt phẳng: $(\alpha)\colon x-2=0$; $(\beta)\colon y-6=0$; $(\gamma): z-3=0$. Các mệnh đề sau đây đúng hay sai?
	\choiceTF
	{\True $(\alpha) \perp(\beta)$}
	{$(\beta) \parallel (Oyz)$}
	{$(\gamma) \parallel Oz$}
	{\True $(\alpha)$ qua $I$}
	\loigiai{
		Ta có:
		\begin{itemize}
			\item $(\alpha): x-2=0$ có véctơ pháp tuyến $\vec{a}=(1 ; 0 ; 0)$.
			\item $(\beta): y-6=0$ có véctơ pháp tuyến $\vec{b}=(0 ; 1 ; 0)$.
			\item $(\gamma): z+3=0$ có véctơ pháp tuyến $\vec{c}=(0 ; 0 ; 1)$.
		\end{itemize}
		\begin{itemchoice}
			\itemch đúng vì ta có $\vec{a} \cdot \vec{b}=1\cdot 0+0\cdot 1+0=0 \Rightarrow(\alpha) \perp(\beta)$.
			\itemch sai vì $(Oyz)$ có véctơ pháp tuyến $\vec{i}=(1 ; 0 ; 0)$ không cùng phương với $\vec{b}=(0 ; 1 ; 0)$ nên $(\beta)$ không song song với mặt phẳng $(Oyz)$. 
			\itemch sai vì trục $Oz$ có vectơ chỉ phương $\vec{k}=(0 ; 0 ; 1)=\vec{c}$ nên $(\gamma) \perp Oz$.
			\itemch đúng vì thay tọa độ điểm $I$ vào $(\alpha)$ ta thấy thỏa thỏa mãn nên $I \in(\alpha)$.	
		\end{itemchoice}
	}
\end{ex}

%%==========Câu 52
\begin{ex}%Câu 52.%[2H5H1-4]
	Trong không gian $Oxyz$, cho hai mặt phẳng $(P)\colon y-9=0$. Xét các mệnh đề sau:
	\begin{multicols}{2}
		\item \hspace*{1cm}(I) $(P) \parallel (Oxz)$.
		\item (II) $(P) \perp Oy$
	\end{multicols}
	\choiceTF
	{Cả (I) và (II) đều sai}
	{(I) đúng, (II) sai}
	{(I) sai, (II) đúng}
	{\True Cả (I) và (II) đều đúng}
	\loigiai{
		Ta có: mặt phẳng $(Oxz)$ có véctơ pháp tuyến $\vec{j}=(0 ; 1 ; 0)$.\\
		Mặt phẳng $(P)$ có véctơ pháp tuyến là $\vec{a}=(0;1;1)=\vec{j}$ nên $(P)\parallel (Oxz)$.\\
		Trục $Oz$ có vectơ chỉ phương là $\vec{j}=(0;1;0)$ nên $(P)\perp Oy$.
		\begin{itemchoice}
			\itemch Cả (I) và (II) đều sai. SAI
			\itemch (I) đúng, (II) sai. SAI
			\itemch (I) sai, (II) đúng. SAI
			\itemch Cả (I) và (II) đều đúng. ĐÚNG
		\end{itemchoice}
	}
\end{ex}

%%==========Câu 53
\begin{ex}%Câu 53.%[2H5H1-4]
	Trong không gian $Oxyz$, Cho ba mặt phẳng $(\alpha)\colon x+y+2z+1=0;(\beta)\colon x+y-z+2=0$; $(\gamma)\colon x-y+5=0$. Các mệnh đề sau đây đúng hay sai?
	\choiceTF
	{$(\alpha) \parallel (\gamma)$}
	{\True $(\alpha) \perp(\beta)$}
	{\True $(\gamma) \perp(\beta)$}
	{\True $(\alpha) \perp(\gamma)$}
	\loigiai{ Ta có:
		\begin{itemize}
			\item Mặt phẳng $(\alpha)$ có véctơ pháp tuyến là $\vec{a}=(1;1;2)$.
			\item Mặt phẳng $(\beta)$ có có véctơ pháp tuyến là $\vec{b}=(1;1;-1)$.
			\item Mặt phẳng $(\gamma)$ có có véctơ pháp tuyến là $\vec{c}=(1;-1;0)$.
			\item $\left [\vec{a},\vec{c}\right ]=(2;2;-2)\ne \vec{0}$ nên $(\alpha)$ và $(\gamma)$ không song song nhau.
			\item $\vec{a} \cdot \vec{b}=0 \Rightarrow(\alpha) \perp(\beta)$.
			\item $\vec{a} \cdot \vec{c}=0 \Rightarrow(\alpha) \perp(\gamma)$.
			\item $\vec{b} \cdot \vec{c}=0 \Rightarrow(\beta) \perp(\gamma)$.
		\end{itemize}
		\begin{itemchoice}
			\itemch $(\alpha)\parallel(\gamma)$. SAI
			\itemch $(\alpha) \perp(\beta)$. ĐÚNG
			\itemch $(\gamma) \perp(\beta)$. ĐÚNG
			\itemch $(\alpha) \perp(\gamma)$. ĐÚNG
		\end{itemchoice}
	}
\end{ex}
\Closesolutionfile{ans}
\indapan{2}{ans/ans2C5B1CD1-D2-DS}
\TNSA
\Opensolutionfile{ans}[ans/ans2C5B1CD1-D2-KQ]
%%==========Câu 54
\begin{ex}%Câu 54.%[2H5H1-5]
	Trong không gian $Oxyz$, cho điểm $M(-1; 2-3)$ và mặt phẳng $(P)\colon 2 x-2 y+z+5=0$. Tính khoảng cách từ điểm $M$ đến mặt phẳng $(P)$ (kết quả viết dưới dạng số thập phân, lấy gần đúng đến hàng phần mười).
	\shortans[0]{$1{,}3$}
	\loigiai{
		Khoảng cách từ điểm $M$ đến mặt phẳng $(P)$ là 
		$$\mathrm{d}\left(M,(P)\right)=\dfrac{\left| 2\cdot (-1)-2\cdot 2+1\cdot (-3)+5\right|}{\sqrt{2^2+(-2)^2+1^2}}=\dfrac{4}{3}.$$
	}
\end{ex}

%%==========Câu 55
\begin{ex}%Câu 55.%[2H5H1-4]
	Trong không gian $Oxyz$, khoảng cách giữa hai mặt phẳng $(P)\colon x+2 y-2 z-16=0$ và $(Q)\colon x+2 y-2 z-1=0$ bằng bao nhiêu?
	\shortans[0]{$5$}
	\loigiai{
		Ta có $\heva{&(P)\parallel (Q) \\ &A(16;0;0)\in (P)}\Rightarrow \mathrm{d}\left((P),(Q)\right)=\mathrm{d}\left(A,(Q)\right)=\dfrac{\left| 16+2\cdot 0-2\cdot 0-1\right|}{\sqrt{1^2+2^2+2^2}}=5$.
	}
\end{ex}
\Closesolutionfile{ans}
\indapan{2}{ans/ans2C5B1CD1-D2-DS}
\TNSA
\Opensolutionfile{ans}[ans/ans2C5B1CD1-D2-KQ]
\begin{ex} %Cau 56D  %[2H5H1-4]
	Trong không gian $Oxyz$, điểm $M \left(0;a;0\right)$ thuộc trục $Oy$ và cách đều hai mặt phẳng: $\left(P\right) \colon x+y-z+1=0$ và $\left(Q\right) \colon x-y+z-5=0$. Khi đó $a$ có giá trị bằng
	\shortans{$-3$}
	\loigiai{
		Ta có $M \in Oy \Rightarrow M\left(0;a;0\right)$.\\
		Theo giả thiết: $\mathrm{d} \left(M,\left(P\right)\right) = \mathrm{d} \left(M,\left(Q\right)\right) \Leftrightarrow \dfrac{\vert a+1 \vert}{\sqrt{3}} = \dfrac{\vert -a-5 \vert}{\sqrt{3}} \Leftrightarrow a = -3$.\\
		Vậy $a = -3$ thì thỏa mãn đề bài.
	}
\end{ex}

\begin{ex} %Cau 57D %[2H5V1-5]
	Trong không gian với hệ trục tọa độ $Oxy$, cho $A \left(1;2;3\right)$, $B \left(3;4;4\right)$. Khi đó giá trị của tham số $m$ bằng bao nhiêu để khoảng cách từ điểm $A$ đến mặt phẳng $\left(P\right)\colon 2x + y + mz -1=0$ bằng độ dài đoạn thẳng $AB$.
	\shortans{$2$}
	\loigiai{
		Ta có $\overrightarrow{AB} = \left(2;2;1\right) \Rightarrow AB = \sqrt{2^2+2^2+1^2} = 3$ \quad(1)\\
		Khoảng cách từ điểm $A$ đến mặt phẳng $\left(P\right)$:\\
		$\mathrm{d} \left(A;\left(P\right)\right) = \dfrac{\vert 2 \cdot 1 + 2 + m \cdot 3 -1 \vert}{\sqrt{2^2+1^2+m^2}} = \dfrac{\vert 3m+3 \vert}{\sqrt{5+m^2}}$ \quad(2).\\
		Để $AB = \mathrm{d} \left(A;\left(P\right)\right) \Rightarrow 3 = \dfrac{\vert 3m+3 \vert}{\sqrt{5+m^2}} \Leftrightarrow 9 \left(5+m^2\right)= 9 \left(m+1\right)^2 \Leftrightarrow m =2$.
	}
\end{ex}

\begin{ex} %Cau 58D %[2H5H1-5]
	Gọi điểm $M \left(0;a;0\right)$ trên trục $Oy$ sao cho khoảng cách từ điểm $M$ đến mặt phẳng $\left(P\right) \colon 2x-y+3z-4=0$ nhỏ nhất. Khi đó giá trị của $a$ là
	\shortans{$-4$}
	\loigiai{
		Khoảng cách từ $M$ đến $\left(P\right)$ nhỏ nhất khi $M$ thuộc $\left(P\right)$. Nên $M$ là giao điểm của trục $Oy$ với mặt phẳng $\left(P\right)$.\\
		Thay $x=0$, $z=0$ vào phương trình ta được $y = -4$. Khi đó $M \left(0;-4;0\right)$\\
		Vậy giá trị của $a = -4$.
	}
\end{ex}

\begin{ex} %Cau 59D %[2H5H1-5]
	Cho điểm $M \left(0;0;m\right)$ thuộc trục $Oz$ sao cho điểm $M$ cách đều điểm $A \left(2;3;4\right)$ và mặt phẳng $\left(P\right) \colon 2x+3y+z-17=0$. Khi đó giá trị của $m$ là
	\shortans{$3$}
	\loigiai{
		Ta có $MA = \sqrt{2^2+3^2+\left(4-m\right)^2}$; $\mathrm{d} \left(M,\left(P\right)\right) = \dfrac{\vert m-17 \vert}{\sqrt{14}}$.\\
		$M$ cách đều điểm $A \left(2;3;4\right)$ và mặt phẳng $\left(P\right) \colon 2x+3y+z-17=0$ khi và chỉ khi\\
		$$\sqrt{2^2+3^2+\left(4-m\right)^2} = \dfrac{\vert m-17 \vert}{\sqrt{14}} \Leftrightarrow 13 \left(m-3\right)^2 = 0 \Leftrightarrow m=3$$
		Vậy $m=3$.
	}
\end{ex}

\begin{ex} %Cau 60D %[2H5V1-5]
	Trong không gian với hệ trục tọa độ $Oxyz$, cho hai điểm $A \left(1;2;3\right)$, $B\left(5;-4;-1\right)$ và mặt phẳng $\left(P\right)$ qua $Ox$ sao cho $\mathrm{d} \left(B;\left(P\right)\right) = 2\mathrm{d} \left(A;\left(P\right)\right)$, $\left(P\right)$ cắt $AB$ tại $I\left(a;b;c\right)$ nằm giữa $AB$. Tính $a+b+c$.
	\shortans{$4$}
	\loigiai{
		Vì $\mathrm{d} \left(B;\left(P\right)\right) = 2\mathrm{d} \left(A;\left(P\right)\right)$ và $\left(P\right)$ cắt đoạn $AB$ tại $I$ nên\\
		$\overrightarrow{BI} = -2 \overrightarrow{AI} \Leftrightarrow \heva{&a-5 = -2\left(a-1\right)\\&b+4 = -2\left(b-2\right)\\&c+1 = -2\left(c-3\right)} \Leftrightarrow \heva{&a=\dfrac{7}{3}\\&b=0\\&c=\dfrac{5}{3}} \Rightarrow a+b+c = 4$.
	}
\end{ex}

\begin{ex} %Cau 61D %[2H5V1-5]
	Trong không gian $Oxyz$, cho mặt phẳng $\left(P\right) \colon 3x+4y-12z+5=0$ và điểm $A \left(2;4;-1\right)$. Trên mặt phẳng $\left(P\right)$ lấy điểm $M$. Gọi $B$ là điểm sao cho $\overrightarrow{AB} = 3\cdot \overrightarrow{AM}$. Tính khoảng cách $\mathrm{d}$ từ $B$ đến mặt phẳng $\left(P\right)$
	\shortans{$6$}
	\loigiai{Ta có: $\overrightarrow{AB} = 3 \cdot \overrightarrow{AM} \Rightarrow BM=2\cdot AM \Rightarrow \dfrac{\mathrm{d} \left(B,\left(P\right)\right)}{\mathrm{d} \left(A,\left(P\right)\right)} = \dfrac{BM}{AM} = 2$
		\immini{
			\begin{eqnarray*}
				&\Rightarrow \mathrm{d} \left(B,\left(P\right)\right) &= 2 \cdot \mathrm{d} \left(A,\left(P\right)\right)\\
				& &= 2 \cdot \dfrac{\vert 3 \cdot 2 + 4 \cdot 4 -12 \cdot \left(-1\right)+5\vert}{\sqrt{3^2+4^2+\left(-12\right)^2}}\\
				& & = 2 \cdot 3 = 6
			\end{eqnarray*}
			Vậy $\mathrm{d} \left(B,\left(P\right)\right) = 6$.}{
			\begin{tikzpicture}[>=stealth,line join=round, line cap=round, scale=0.7]
				\coordinate (A) at (1,3);
				\coordinate (B) at (-1,0);
				\coordinate (C) at (5,0);
				\coordinate (D) at (7,3);
				\coordinate (H) at (1.5,2);
				\coordinate (K) at (4.5,2);
				\coordinate (M) at (4.5,-1.5);
				\coordinate (N) at (1.5,4.5);
				\path[name path=d1] (N)--(M);
				\path[name path=d2] (H)--(K);
				\path[name path=d3] (B)--(C);
				\path[name intersections={of=d1 and d2,by=l}];
				\path[name intersections={of=d1 and d3,by=z}];
				\draw (A)--(B)--(C)--(D)--(A); \draw[dashed] (K)--(4.5,0); \draw (4.5,0)--(M)--(z);
				\draw (l) node[below left]{$M$} circle (1pt)--(N) node[above]{$A$}  circle (1pt)--(H) node[below left]{$H$}  circle (1pt)--(K) node[below right]{$K$} circle (1pt); \draw (M) node[below]{$B$} circle (1pt); \draw [dashed] (l)--(z);
				\begin{scope}
					\clip (A)--(B)--(C);
					\draw (B) circle (1.1);
					\draw (-0.8,0) node[above right]{$P$} ;
				\end{scope}
			\end{tikzpicture}
	}}
\end{ex}

\begin{ex} %Cau 62D %[2H5H1-4]
	Trong không gian $Oxyz$, cho hai mặt phẳng $\left(P\right) \colon 2x+my+2mz-9=0$ và $\left(Q\right) \colon 6x-y-z-10=0$. Tìm $m$ để $\left(P\right) \perp \left(Q\right)$
	\shortans{$4$}
	\loigiai{
		$\left(P\right) \colon 2x+my+2mz-9=0$ có véc-tơ pháp tuyến là $\overrightarrow{a} = \left(2;m;2m\right)$\\
		$\left(Q\right) \colon 6x-y-z-10=0$ có véc-tơ pháp tuyến là $\overrightarrow{b} = \left(6;-1;-1\right)$\\
		Khi đó $\left(P\right) \perp \left(Q\right) \Leftrightarrow \overrightarrow{a} \cdot \overrightarrow{b} =0 \Leftrightarrow 2 \cdot 6 + m \cdot \left(-1\right)+2m \cdot \left(-1\right) =0 \Leftrightarrow m=4$.
	}
\end{ex}

\begin{ex} %Cau 63D %[2H5H1-4]
	Trong không gian $Oxyz$, cho hai mặt phẳng $\left(P\right) \colon 5x+my+z-5=0$ và $\left(Q\right) \colon nx-3y-2z+7=0$. Để $\left(P\right) \parallel \left(Q\right)$ thì giá trị của $m+n$ là (làm tròn đến chữ số thập phân thứ nhất)
	\shortans{$-8{,}5$}
	\loigiai{
		$\left(P\right) \colon 5x+my+z-5=0$ có véc-tơ pháp tuyến là $\overrightarrow{a} = \left(5;m;1\right)$\\
		$\left(Q\right) \colon nx-3y-2z+7=0$ có véc-tơ pháp tuyến là $\overrightarrow{b} = \left(n;-3;-2\right)$\\
		Để $\left(P\right) \parallel \left(Q\right) \Leftrightarrow \left[a;b\right] = \overrightarrow{0} \Leftrightarrow \heva{&-2m+3=0\\&n+10=0\\&-15-mn=0} \Leftrightarrow \heva{&m=\dfrac{3}{2}\\&n=-10}$\\
		Khi đó $m+n = \dfrac{3}{2} + \left(-10\right)= -8,5$.
	}
\end{ex}

\begin{ex} %Cau 64D %[2H5V1-4]
	Trong không gian $Oxyz$, cho hai mặt phẳng $\left(P\right) \colon 2x-my-4z-6+m=0$ và $\left(Q\right) \colon \left(m+3\right)x +y+\left(5m+1\right)z -7=0$. Tìm $m$ để $\left(P\right) \equiv \left(Q\right)$.
	\shortans{$-1$}
	\loigiai{
		$\left(P\right) \colon 2x-my-4z-6+m=0$ có véc-tơ pháp tuyến là $\overrightarrow{a} = \left(2;-m;-4\right)$\\
		$\left(Q\right) \colon \left(m+3\right)x +y+\left(5m+1\right)z -7=0$ có véc-tơ pháp tuyến là $\overrightarrow{b} = \left(m+3;1;5m+1\right)$\\
		Khi đó với $m \neq -3$, $m \neq -\dfrac{1}{5}$ ta có $\left(P\right) \equiv \left(Q\right) \Leftrightarrow \dfrac{2}{m+3} = \dfrac{-m}{1} = \dfrac{-4}{5m+1} \Leftrightarrow m = -1$.
	}
\end{ex}

\begin{ex} %Cau 65D %[2H5H1-3]
	Trong không gian $Oxyz$, cho hai mặt phẳng $\left(P\right) \colon x-2y-z+3=0$ và $\left(Q\right) \colon 2x +y+z -1=0$. Mặt phẳng $\left(R\right)$ đi qua điểm $M\left(1;1;1\right)$ chứa giao tuyến của $\left(P\right)$ và $\left(Q\right)$; phương trình của $\left(R\right) \colon m\left(x-2y-z+3\right) + \left(2x+y+z-1\right)=0$. Khi đó giá trị của $m$ là bao nhiêu?
	\shortans{$-3$}
	\loigiai{
		Vì $\left(R\right) \colon m\left(x-2y-z+3\right) + \left(2x+y+z-1\right)=0$ đi qua điểm $M \left(1;1;1\right)$ nên ta có:\\
		$m\left(1-2 \cdot 1-1+3\right) + \left(2 \cdot 1+1+1-1\right)=0 \Leftrightarrow m = -3$\\
		Vậy $m = -3$.
	}
\end{ex}

\begin{ex} %Cau 66D %[2H5V1-4]
	Trong không gian $Oxyz$, cho $3$ điểm $A\left(1;0;0\right)$,$B\left(0;b;0\right)$,$C\left(0;0;c\right)$ trong đó $b \cdot c \neq 0$ và mặt phẳng $\left(P\right) \colon y-z+1=0$. Giá trị của $\dfrac{2b}{c}$ bằng bao nhiêu để mặt phẳng $\left(ABC\right)$ vuông góc với mặt phẳng $\left(P\right)$.
	\shortans{$2$}
	\loigiai{
		Phương trình mặt phẳng $\left(ABC\right) \colon \dfrac{x}{1}+\dfrac{y}{b}+\dfrac{z}{c}=1$ có véc-tơ pháp tuyến là $\overrightarrow{n} = \left(1;\dfrac{1}{b};\dfrac{1}{c}\right)$.\\
		Phương trình mặt phẳng $\left(P\right) \colon y-z+1=0$ có véc-tơ pháp tuyến là $\overrightarrow{n'} = \left(0;1;-1\right)$.\\
		Do đó $\left(ABC\right) \perp \left(P\right) \Leftrightarrow \overrightarrow{n} \cdot \overrightarrow{n'} = 0 \Leftrightarrow \dfrac{1}{b}-\dfrac{1}{c} = 0 \Leftrightarrow b = c$.\\
		Vậy $\dfrac{2b}{c} = 2$.
	}
\end{ex}

\begin{ex} %Cau 67D %[2H5V1-3]
	Trong không gian $Oxyz$, cho mặt phẳng $\left(\alpha\right) \colon ax-y+2z+b=0$ đi qua giao tuyến của hai mặt phẳng $\left(P\right) \colon x-y-z+1=0$ và $\left(Q\right) \colon x+2y+z-1=0$. Tính $a+4b$
	\shortans{$-16$}
	\loigiai{
		Trên giao tuyến $\Delta$ của hai mặt phẳng $\left(P\right)$, $\left(Q\right)$ ta lấy lần lượt hai điểm $A$, $B$ như sau\\
		Lấy $A \left(x;y;1\right) \in \Delta$, ta có hệ phương trình $\heva{&x-y=0\\&x+2y=0} \Rightarrow x=y=0 \Rightarrow A \left(0;0;1\right)$.\\
		Lấy $B \left(-1;y;z\right) \in \Delta$, ta có hệ phương trình $\heva{&y+z=0\\&2y+z=0} \Rightarrow \heva{&y=2\\&z=2} \Rightarrow B \left(-1;2;-2\right)$.\\
		Vì $\Delta \subset \left(\alpha\right)$ nên $A$, $B \in \left(\alpha\right)$. Do đó ta có: $\heva{&2+b=0\\&-a+b-6=0} \Rightarrow \heva{&a=-8\\&b=-2}$.\\
		Vậy $a+4b = -8 + 2 \cdot \left(-2\right) = -16$.
	}
\end{ex}

\begin{ex} %Cau 68D %[2H5V1-4]
	Gọi $m$, $n$ là hai giá trị thực thỏa mãn giao tuyến của hai mặt phẳng $\left(P_m\right) \colon mx+2y+nz+1=0$ và $\left(Q_m\right) \colon x-my + nz + 2=0$ vuông góc với mặt phẳng $\left(\alpha\right) \colon 4x -y -6z +3=0$. Tính $m+n$
	\shortans{$3$}
	\loigiai{
		\begin{description}
			\item[+] $\left(P_m\right) \colon mx+2y+nz+1=0$ có véc-tơ pháp tuyến $\overrightarrow{n}_1 = \left(m;2;n\right)$
			\item[+] $\left(Q_m\right) \colon x-my+nz+2=0$ có véc-tơ pháp tuyến $\overrightarrow{n}_2 = \left(1;-m;n\right)$
			\item[+] $\left(\alpha \right) \colon 4x-y-6z+3=0$ có véc-tơ pháp tuyến $\overrightarrow{n}_{\alpha} = \left(4;-1;-6\right)$.
			\item[+] Giao tuyến của hai mặt phẳng $\left(P_m\right)$ và $\left(Q_m\right)$ vuông góc với mặt phẳng $\left(\alpha\right)$ nên
			$$\heva{&\left(P_m\right) \perp \left(\alpha\right)\\&\left(Q_m\right) \perp \left(\alpha\right)} \Leftrightarrow \heva{&\overrightarrow{n}_1 \perp \overrightarrow{n}_{\alpha}\\&\overrightarrow{n}_2 \perp \overrightarrow{n}_{\alpha}} \Leftrightarrow \heva{&\overrightarrow{n}_1 \cdot \overrightarrow{n}_{\alpha} = 0\\&\overrightarrow{n}_2 \cdot \overrightarrow{n}_{\alpha} = 0} \Leftrightarrow \heva{&4m-2-6n=0\\&4+m-6n=0} \Leftrightarrow \heva{&m=2\\&n=1}$$
		\end{description}
		Vậy $m+n = 3$.
	}
\end{ex}

\begin{ex} %Cau 69D %[2H5V1-4]
	Trong không gian với hệ tọa độ $Oxyz$ có bao nhiêu mặt phẳng song song với mặt phẳng $\left(Q\right) \colon x+y+z+3=0$, cách điểm $M \left(3;2;1\right)$ một khoảng bằng $3\sqrt{3}$ biết rằng tồn tại một điểm $X\left(a;b;c\right)$ trên mặt phẳng đó, khi đó $a+b+c$ có giá trị bằng
	\shortans{$15$}
	\loigiai{
		Ta có mặt phẳng cần tìm là $\left(P\right) \colon x+y+z+d=0$ với $d \neq 3$.\\
		Mặt phẳng $\left(P\right)$ cách điểm $M \left(3;2;1\right)$ một khoảng bằng $3\sqrt{3}$ nên\\
		$\mathrm{d} \left(M,\left(P\right)\right) = \dfrac{\vert 6+d \vert}{\sqrt{3}} = 3\sqrt{3} \Leftrightarrow \hoac{&d=3 \quad(L)\\&d=-15} \Rightarrow d = -15$.\\
		Suy ra $\left(P\right) \colon x+y+z-15=0$.\\
		Theo giả thiết $X\left(a;b;c\right) \in \left(P\right) \Leftrightarrow a+b+c = 15$.
	}
\end{ex}

\begin{ex} %Cau 70D %[2H5C1-4]
	Biết rằng trong không gian với hệ tọa độ $Oxyz$ có hai mặt phẳng $\left(P\right)$ và $\left(Q\right)$ cùng thỏa mãn các điều kiện sau: đi qua hai điểm $A\left(1;1;1\right)$ và $B\left(0;-2;2\right)$, đồng thời cắt các trục tọa độ $Ox$, $Oy$ tại hai điểm cách đều $O$. Giả sử $\left(P\right) \colon x+b_{1}y+c_{1}z+d_1=0$ và $\left(Q\right) \colon x+b_2 y+c_2 z+d_2=0$. Tính giá trị biểu thức $b_1b_2 + c_1c_2$
	\shortans{$-9$}
	\loigiai{
		$\textbf{Cách 1}$\\
		Xét mặt phẳng $\left(\alpha\right) \colon x+by+cz+d=0$ thỏa mãn các điều kiện: đi qua hai điểm $A \left(1;1;1\right)$ và $B\left(0;-2;2\right)$, đồng thời cắt các trục tọa độ $Ox$, $Oy$ tại hai điểm cách đều $O$.\\
		Vì $\left(\alpha\right)$ đi qua $A \left(1;1;1\right)$ và $B \left(0;-2;2\right)$ nên ta có hệ phương trình:
		$$\heva{&1+b+c+d=0\\&-2b+2c+d=0} \quad(*)$$
		Mặt phẳng $\left(\alpha\right)$ cắt các trục tọa độ $Ox$, $Oy$ lần lượt tại $M \left(-d;0;0\right)$, $N \left(0;\dfrac{-d}{c};0\right)$.\\
		Vì $M$, $N$ cách đều $O$ nên $OM = ON$. Suy ra: $\vert d \vert = \left\vert \dfrac{d}{b} \right\vert$.\\
		Nếu $d=0$ thì chỉ tồn tại duy nhất một mặt phẳng thỏa mãn yêu cầu bài toán (mặt phẳng này sẽ đi qua điểm $O$).\\
		Do đó để tồn tại hai mặt phẳng thỏa mãn yêu cầu bài toán thì $\vert d \vert = \left\vert \dfrac{d}{b} \right\vert \Leftrightarrow b = \pm 1$.
		\begin{description}
			\item[$\bullet$] Với $b=1$, $\left(*\right) \Leftrightarrow \heva{&c+d=-2\\&2c+d=2} \Leftrightarrow \heva{&c=4\\&d=-6}$. Ta được mặt phẳng $\left(P\right) \colon x+y+4z-6=0$.
			\item[$\bullet$] Với $b=-1$, $\left(*\right) \Leftrightarrow \heva{&c+d=0\\&2c+d=-2} \Leftrightarrow \heva{&c=-2\\&d=2}$. Ta có mặt phẳng $\left(P\right) \colon x-y-2z+2=0$.
		\end{description}
		Vậy $b_1b_2 + c_1c_2 = 1 \cdot \left(-1\right) + 4 \cdot \left(-2\right) = -9$.\\
		$\textbf{Cách 2}$\\
		Ta có $\overrightarrow{AB} = \left(-1;-3;1\right)$.\\
		Xét mặt phẳng $\left(\alpha\right) \colon x+by+cz+d=0$ thõa mãn các điều kiện: đi qua hai điểm $A \left(1;1;1\right)$ và $B\left(0;-2;2\right)$, đồng thời cắt các trục tọa độ $Ox$, $Oy$ tại hai điểm cách đều $O$ lần lượt tại $M$, $N$. Vì $M$, $N$ cách đều $O$ nên ta có hai trường hợp sau
		\begin{description}
			\item[TH1] $M \left(a;0;0\right)$, $N \left(0;a;0\right)$ với $a \neq 0$ khi đó $\left(\alpha\right)$ chính là $\left(P\right)$. Ta có $\overrightarrow{MN} = \left(-a;a;0\right)$, chọn $\overrightarrow{u}_1 = \left(-1;1;0\right)$ là một véc-tơ cùng phương với $\overrightarrow{MN}$.\\
			Khi đó $\overrightarrow{n}_P = \left[\overrightarrow{AB},\overrightarrow{u}_1\right] = \left(-1;-1;-4\right)$
			suy ra $\left(P\right) \colon x+y+4z+d_1 = 0$.
			\item[TH2] $M \left(-a;0;0\right)$, $N \left(0;a;0\right)$ với $a \neq 0$ khi đó $\left(\alpha\right)$ chính là $\left(Q\right)$. Ta có $\overrightarrow{MN} = \left(a;a;0\right)$, chọn $\overrightarrow{u}_2 = \left(1;1;0\right)$ là một véc-tơ cùng phương với $\overrightarrow{MN}$.\\
			Khi đó $\overrightarrow{n}_Q = \left[\overrightarrow{AB},\overrightarrow{u}_2\right] = \left(-1;1;2\right)$
			suy ra $\left(Q\right) \colon x-y-2z+d_2 = 0$.
		\end{description}
		Vậy $b_1b_2 + c_1c_2 = 1 \cdot \left(-1\right) + 4 \cdot \left(-2\right) = -9$.
	}
\end{ex}
%-------------HetCD1---------------
\Closesolutionfile{ans}
\indapan{6}{ans/ans2C5B1CD1-D2-KQ}