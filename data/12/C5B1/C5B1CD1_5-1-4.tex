\chapter{PHƯƠNG PHÁP TỌA ĐỘ TRONG KHÔNG GIAN}
\section{PHƯƠNG TRÌNH MẶT PHẲNG}
\chude{Xác định các yếu tố cơ bản liên quan đến mặt phẳng}
\begin{dang}{Xác định véctơ pháp tuyến của mặt phẳng. Xác định điểm thuộc và không thuộc mặt phẳng}
	\begin{enumerate}[label=\bf\arabic*.]
	\item \textbf{véctơ pháp tuyến của mặt phẳng:}
	\begin{itemize}
		\item Mặt phẳng $(\alpha)\colon A x+B y+C z+D=0$ có véctơ pháp tuyến $\overrightarrow{n}=(A; B; C)$.
		\item Nếu mặt phẳng $(\alpha)$ có cặp véctơ chỉ phương là $\overrightarrow{a}, \overrightarrow{b}$ thì $(\alpha)$ có véctơ pháp tuyến là $\overrightarrow{n}=\left[\overrightarrow{a}, \overrightarrow{b}\right]$.
		\item véctơ pháp tuyến của mặt phẳng $(\alpha)$ là véctơ có giá vuông góc với $(\alpha)$.
		\item véctơ chỉ phương của mặt phẳng $(\alpha)$ là véctơ có giá song song hoặc trùng với $(\alpha)$.
		\item Nếu $\overrightarrow{n}$ là một véctơ pháp tuyến của $(\alpha)$ thì $k \cdot \overrightarrow{n}$ cũng là một véctơ pháp tuyến của $(\alpha)$.
		\item Nếu $\overrightarrow{a}$ là một véctơ chỉ phương của $(\alpha)$ thì $k \cdot \overrightarrow{a}$ cũng là một véctơ chỉ phương của $(\alpha)$.
		\item[] \textbf{Chú ý:}
		\item Trục $O x$ có véctơ chỉ phương là $\overrightarrow{i}=(1; 0; 0)$.
		\item Trục $O y$ có véctơ chỉ phương là $\overrightarrow{j}=(0; 1; 0)$.
		\item Trục $O z$ có véctơ chỉ phương là $\overrightarrow{k}=(0; 0; 1)$.
		\item Mặt phẳng $(O x y)$ có véctơ pháp tuyến là $\overrightarrow{k}=(0; 0; 1)$.
		\item Mặt phẳng $(O x z)$ có véctơ pháp tuyến là $\overrightarrow{j}=(0; 1; 0)$.
		\item Mặt phẳng $(O y z)$ có véctơ pháp tuyến là $\overrightarrow{i}=(1; 0; 0)$.
	\end{itemize}
		\item \textbf{Điểm thuộc và không thuộc mặt phẳng:}\\
		Cho mặt phẳng $(\alpha)$ có phương trình $A x+B y+C z+D=0$. Khi đó: 
		\begin{itemize}
		\item $N_0\left(x_0; y_0; z_0\right) \in(\alpha) \Leftrightarrow A x_0+B y_0+C z_0+D=0$.
		\item $N_0\left(x_0; y_0; z_0\right) \notin(\alpha) \Leftrightarrow A x_0+B y_0+C z_0+D \neq 0$.	
		\end{itemize}
	\end{enumerate}
\end{dang}

\TN
\Opensolutionfile{ans}[ans/ans2C5B1CD1]
\begin{ex}%[2H2H2-5]
	Trong không gian $O x y z$, tọa độ một véctơ $\overrightarrow{n}$ vuông góc với cả hai véctơ $\overrightarrow{a}=(1; 1;-2), \overrightarrow{b}=(1; 0; 3)$ là
	\choice
	{$(2; 3;-1)$}
	{$(3; 5;-2)$}
	{$(2;-3;-1)$}
	{\True $(3;-5;-1)$}
	\loigiai{
	véctơ $\overrightarrow{n}$ vuông góc với cả hai véctơ $\overrightarrow{a}, \overrightarrow{b}$.\\
	Do đó $\overrightarrow{n}=\left[\overrightarrow{a}, \overrightarrow{b}\right]$.\\
	Ta có $\left[\overrightarrow{a}, \overrightarrow{b}\right]=(3;-5;-1)$.	
	}
\end{ex}

\begin{ex}%[2H2H2-5]
	Trong không gian với hệ tọa độ $O x y z$, cho hai véctơ $\overrightarrow{a}=(2; 1;-2)$ và véctơ $\overrightarrow{b}=(1; 0; 2)$. Tìm tọa độ véctơ $\overrightarrow{c}$ là tích có hướng của $\overrightarrow{a}$ và $\overrightarrow{b}$.
	\choice
	{$\overrightarrow{c}=(2; 6;-1)$}
	{$\overrightarrow{c}=(4; 6;-1)$}
	{$\overrightarrow{c}=(4;-6;-1)$}
	{\True $\overrightarrow{c}=(2;-6;-1)$}
	\loigiai{
	Áp dụng công thức tính tích có hướng trong hệ trục tọa độ $O x y z$, ta được
	$$\overrightarrow{c}=\left[\overrightarrow{a}, \overrightarrow{b}\right]=(2;-6;-1).$$	
	}
\end{ex}

\begin{ex}%[2H2H2-5]
	Trong không gian với hệ trục tọa độ $O x y z$, cho $A(2; 1;-3), B(0;-2; 5)$ và $C(1; 1; 3)$. Tìm tọa độ véctơ $\overrightarrow{n}$ có phương vuông góc với hai véctơ $\overrightarrow{A B}$ và $\overrightarrow{A C}$.
	\choice
	{$\overrightarrow{n}=(8; 4;-3)$}
	{$\overrightarrow{n}=(-18; 0;-3)$}
	{\True $\overrightarrow{n}=(-18; 4;-3)$}
	{$\overrightarrow{n}=(1; 4;-3)$}
	\loigiai{
	Ta có $\overrightarrow{A B}=(-2;-3; 8)$ và $\overrightarrow{A C}=(-1; 0; 6)$. Suy ra $\left[\overrightarrow{A B}, \overrightarrow{A C}\right]=(-18; 4;-3)$.\\
	Vậy $\overrightarrow{n}=\left[\overrightarrow{A B}, \overrightarrow{A C}\right]=(-18; 4;-3)$.
	}
\end{ex}

\begin{ex}%[2H5N1-3]
	Trong không gian $O x y z$, phương trình nào sau đây là phương trình tổng quát của mặt phẳng?
	\choice
	{$x-3 y^2+z-1=0$}
	{$x^2+2 y+4 z-2=0$}
	{\True $2 x-3 y+4 z-2024=0$}
	{$2 x-3 y+4 z^2-2025=0$}
	\loigiai{
	Phương trình tổng quát của mặt phẳng là $2 x-3 y+4 z-2024=0$.	
	}
\end{ex}

\begin{ex}%[2H5H1-3]
	Trong không gian $O x y z$, cho mặt phẳng $(P)\colon 3 x-y+2 z-1=0$. véctơ nào dưới đây \textbf{không phải} là một véctơ pháp tuyến của $(P)$?
	\choice
	{$\overrightarrow{n}=(-3; 1;-2)$}
	{\True $\overrightarrow{n}=(3; 1; 2)$}
	{$\overrightarrow{n}=(3;-1; 2)$}
	{$\overrightarrow{n}=(6;-2; 4)$}
	\loigiai{
	Véctơ pháp tuyến của $(P)$ là $\overrightarrow{n}=(3;-1; 2)$.\\
	$\overrightarrow{n}=(-3; 1;-2)=-1(3;-1; 2)$ là một véctơ pháp tuyến của $(P)$.\\
	$\overrightarrow{n}=(6;-2; 4)=2(3;-1; 2)$ là một véctơ pháp tuyến của $(P)$. 	
	}
\end{ex}

\begin{ex}%[2H5H1-3]
	Trong không gian với hệ tọa độ $O x y z$, véctơ nào dưới đây là một véctơ pháp tuyến của mặt phẳng $(O x y)$?
	\choice
	{$\overrightarrow{i}=(1; 0; 0)$}
	{$\overrightarrow{m}=(1; 1; 1)$}
	{$\overrightarrow{j}=(0; 1; 0)$}
	{\True $\overrightarrow{k}=(0; 0; 1)$}
	\loigiai{
	Do mặt phẳng $(O x y)$ vuông góc với trục $O z$ nên nhận véctơ $\overrightarrow{k}=(0; 0; 1)$ làm một véctơ pháp tuyến.	
	}
\end{ex}

\begin{ex}%[2H5H1-3]
	Trong không gian $O x y z$, véctơ nào dưới đây có giá vuông góc với mặt phẳng $(\alpha)\colon 2 x-3 y+1=0$?
	\choice
	{$\overrightarrow{a}=(2;-3; 1)$}
	{$\overrightarrow{b}=(2; 1;-3)$}
	{\True $\overrightarrow{c}=(2;-3; 0)$}
	{$\overrightarrow{d}=(3; 2; 0)$}
	\loigiai{
	Mặt phẳng $(\alpha)$ có một véctơ pháp tuyến là $\overrightarrow{n}=(2;-3; 0)=\overrightarrow{c}$.	
	}
\end{ex}

\begin{ex}%[2H5H1-3]
	Trong không gian $O x y z$, một véctơ pháp tuyến của mặt phẳng $\dfrac{x}{-2}+\dfrac{y}{-1}+\dfrac{z}{3}=1$ là
	\choice
	{\True $\overrightarrow{n}=(3; 6;-2)$}
	{$\overrightarrow{n}=(2;-1; 3)$}
	{$\overrightarrow{n}=(-3;-6;-2)$}
	{$\overrightarrow{n}=(-2;-1; 3)$}
	\loigiai{
	Phương trình $\dfrac{x}{-2}+\dfrac{y}{-1}+\dfrac{z}{3}=1 \Leftrightarrow-\dfrac{1}{2} x-y+\dfrac{1}{3} z-1=0 \Leftrightarrow 3 x+6 y-2 z+6=0.$\\
	Do đó mặt phẳng đã cho có một véctơ pháp tuyến là $\overrightarrow{n}=(3; 6;-2)$.	
	}
\end{ex}

\begin{ex}%[2H5H1-3]
	Trong không gian $O x y z$, điểm nào dưới đây nằm trên mặt phẳng $(P)\colon 2 x-y+z-2=0$.
	\choice
	{$Q(1;-2; 2)$}
	{$P(2;-1;-1)$}
	{$M(1; 1;-1)$}
	{\True $N(1;-1;-1)$}
	\loigiai{
	Thay toạ độ điểm $Q$ vào phương trình mặt phẳng $(P)$ ta được $2\cdot 1-(-2)+2-2=4 \neq 0$ nên $Q \notin(P)$.\\
	Thay toạ độ điểm $P$ vào phương trình mặt phẳng $(P)$ ta được $2\cdot2-(-1)+(-1)-2=2 \neq 0$ nên $P \notin(P)$.\\
	Thay toạ độ điểm $M$ vào phương trình mặt phẳng $(P)$ ta được $2\cdot1-1+(-1)-2=-2 \neq 0$ nên $M \notin(P)$.\\
	Thay toạ độ điểm $N$ vào phương trình mặt phẳng $(P)$ ta được $2 \cdot 1-(-1)+(-1)-2=0$ nên $N \in(P)$.	
	}
\end{ex}

\begin{ex}%[2H5H1-3]
	Trong không gian với hệ tọa độ $O x y z$, cho mặt phẳng $(\alpha)\colon x+y+z-6=0$. Điểm nào dưới đây \textbf{không thuộc} $(\alpha)$?
	\choice
	{$Q(3; 3; 0)$}
	{$N(2; 2; 2)$}
	{$P(1; 2; 3)$}
	{\True $M(1;-1; 1)$}
	\loigiai{
	\begin{itemize}
		\item  Thay $Q(3; 3; 0)$  vào phương trình mặt phẳng $(\alpha)$, ta được $3+3+0-6=0 \Rightarrow Q \in(\alpha)$.
		\item  Thay $N(2; 2; 2)$ vào phương trình mặt phẳng $(\alpha)$, ta được  $2+2+2-6=0 \Rightarrow N \in(\alpha)$.
		\item Thay $P(1; 2; 3)$ vào phương trình mặt phẳng $(\alpha)$, ta được $1+2+3-6=0 \Rightarrow P \in(\alpha)$.
		\item  Thay $M(1;-1; 1)$ toạ độ vào phương trình mặt phẳng $(\alpha)$, ta được $1-1+1-6\neq 0 \Rightarrow M \notin(\alpha)$.	 
	\end{itemize}
	}
\end{ex}

\begin{ex}%[2H5H1-3]
	Trong không gian với hệ tọa độ $O x y z$, cho mặt phẳng $(P)\colon x-2 y+z-5=0$. Điểm nào dưới đây thuộc $(P)$?
	\choice
	{$P(0; 0;-5)$}
	{\True $M(1; 1; 6)$}
	{$Q(2;-1; 5)$}
	{$N(-5; 0; 0)$}
	\loigiai{
	Ta có $1-2 \cdot 1+6-5=0$ nên $M(1; 1; 6)$ thuộc mặt phẳng $(P)$.	
	}
\end{ex}

\begin{ex}%[2H5H1-3]
	Trong không gian $O x y z$, mặt phẳng $(P)\colon \dfrac{x}{1}+\dfrac{y}{2}+\dfrac{z}{3}=1$ \textbf{không} đi qua điểm nào dưới đây?
	\choice
	{$P(0; 2; 0)$}
	{\True $N(1; 2; 3)$}
	{$M(1; 0; 0)$}
	{$Q(0; 0; 3)$}
	\loigiai{
	Thế tọa độ điểm $N$ vào phương trình mặt phẳng $(P)$ ta có $\dfrac{1}{1}+\dfrac{2}{2}+\dfrac{3}{3}=1$ (sai).\\
	Vậy mặt phẳng $(P)\colon \dfrac{x}{1}+\dfrac{y}{2}+\dfrac{z}{3}=1$ không đi qua điểm $N(1; 2; 3)$.	
	}
\end{ex}

\begin{ex}%[2H5H1-3]
	Trong không gian $O x y z$, mặt phẳng $(\alpha)\colon x-y+2 z-3=0$ đi qua điểm nào dưới đây?
	\choice
	{\True $M\left(1; 1; \dfrac{3}{2}\right)$}
	{$N\left(1;-1;-\dfrac{3}{2}\right)$}
	{$P(1; 6; 1)$}
	{$Q(0; 3; 0)$}
	\loigiai{
	Xét điểm $M\left(1; 1; \dfrac{3}{2}\right)$, ta có $1-1+2 \cdot \dfrac{3}{2}-3=0$ (đúng) nên $M \in(\alpha)$ .\\
	Xét điểm $N\left(1;-1;-\dfrac{3}{2}\right)$, ta có $1+1+2.\left(-\dfrac{3}{2}\right)-3=0$ (sai) nên $N \notin(\alpha)$.\\
	Xét điểm $P(1; 6; 1)$, ta có $1-6+2.1-3=0$ (sai) nên $P \notin(\alpha)$.\\
	Xét điểm $Q(0; 3; 0)$, ta có $0-3+2.0-3=0$ (sai) nên $Q \notin(\alpha)$.	
	}
\end{ex}
\Closesolutionfile{ans}
\indapan{10}{ans/ans2C5B1CD1}
\TNTF
\Opensolutionfile{ans}[ans/ans2C5B1CD1-DS]
\begin{ex}%[2H5H1-2]
	Trong không gian cho hệ tọa độ $O x y z$. Các mệnh đề sau đây đúng hay sai?
	\choiceTF
	{\True Mặt phẳng $(O x y)$ có một véctơ pháp tuyến là $\overrightarrow{n}=(0; 0; 1)$}
	{\True Mặt phẳng $(O x z)$ có véctơ pháp tuyến là $\overrightarrow{n}=(0; 3; 0)$}
	{\True Mặt phẳng $(O y z)$ có véctơ pháp tuyến là $\overrightarrow{n}=(-2; 0; 0)$}
	{\True Trục $O z$ có véctơ chỉ phương là $\overrightarrow{a}=(0; 0;-2024)$}
	\loigiai{
		\begin{itemchoice}
			\itemch Mặt phẳng $(O x y)$ có một véctơ pháp tuyến là $\overrightarrow{n}=(0; 0; 1)$.
			\itemch Mặt phẳng $(O x z)$ có véctơ pháp tuyến là $\overrightarrow{n}=(0; 3; 0)$.
			\itemch Mặt phẳng $(O y z)$ có véctơ pháp tuyến là $\overrightarrow{n}=(-2; 0; 0)$.
			\itemch Trục $O z$ có véctơ chỉ phương là $\overrightarrow{a}=(0; 0;-2024)$.
		\end{itemchoice}
	}
\end{ex}

\begin{ex}%[2H2H2-5]%[2H2H2-1] 
	Trong không gian với hệ toạ độ $O x y z$, cho $\overrightarrow{a}=(1;-2; 3)$ và $\overrightarrow{b}=(1; 1;-1)$. Các mệnh đề sau đây đúng hay sai? 
	\choiceTF
	{\True $\left|\overrightarrow{a}+\overrightarrow{b}\right|=3$}
	{\True $\overrightarrow{a} \cdot \overrightarrow{b}=-4$}
	{\True $\left|\overrightarrow{a}-\overrightarrow{b}\right|=5$}
	{$\left[\overrightarrow{a}, \overrightarrow{b}\right]=(-1;-4; 3)$}
	\loigiai{
		\begin{itemchoice}
			\itemch $\left|\overrightarrow{a}+\overrightarrow{b}\right|=\left|\overrightarrow{a}+\overrightarrow{b}\right|=\sqrt{(1+1)^2+(-2+1)^2+(3-1)^2}=\sqrt{4+1+4}=3$.
			\itemch $\overrightarrow{a} \cdot \overrightarrow{b}=1 \cdot 1+(-2) \cdot 1+3 \cdot(-1)=1-2-3=-4$.
			\itemch $\left|\overrightarrow{a}+\overrightarrow{b}\right|=\left|\overrightarrow{a}+\overrightarrow{b}\right|=\sqrt{(1-1)^2+(-2-1)^2+(3+1)^2}=\sqrt{0+9+16}=5$.
			\itemch 
			$\left[\overrightarrow{a}, \overrightarrow{b}\right]=\left(\left|\begin{array}{cc}-2 & 3 \\ 1 &-1\end{array}\right|;\left|\begin{array}{cc}3 & 1 \\-1 & 1\end{array}\right|;\left|\begin{array}{cc}1 &-2 \\ 1 & 1\end{array}\right|\right)=(-1; 4; 3)$.
		\end{itemchoice}
	}
\end{ex}
\begin{ex}%[2H2H2-4]%[2H2H2-1]
	Trong không gian với hệ trục tọa độ $O x y z$, cho ba véctơ $\overrightarrow{a}=(1; 2;-1), \overrightarrow{b}=(3;-1; 0), \overrightarrow{c}=(1;-5; 2)$. Các mệnh đề sau đây đúng hay sai?
	\choiceTF
	{$\overrightarrow{a}$ cùng phương với $\overrightarrow{b}$}
	{$\left[\overrightarrow{a}, \overrightarrow{b}\right] \cdot \overrightarrow{c}=0$}
	{$\overrightarrow{a}$ không cùng phương với $\overrightarrow{b}$}
	{$\overrightarrow{a}$ vuông góc với $\overrightarrow{b}$}
	\loigiai{   
		\begin{itemchoice}
			\itemch Ta có:
			$\left[\overrightarrow{a}, \overrightarrow{b}\right]=(-1;-3;-7) \neq \overrightarrow{0}$.
			\itemch Hai véctơ $\overrightarrow{a}, \overrightarrow{b}$ không cùng phương.
			\itemch $\left[\overrightarrow{a}, \overrightarrow{b}\right] \cdot \overrightarrow{c}=-1+15-14=0$.
			\itemch Ba véctơ $\overrightarrow{a}, \overrightarrow{b}, \overrightarrow{c}$ đồng phẳng.
		\end{itemchoice}
	}
\end{ex}
\begin{ex}%[2H5H1-2]
	Trong không gian $O x y z$, cho mặt phẳng $(P)\colon 2 x+3 y+z-2024=0$. Các mệnh đề sau đây đúng hay sai?
	\choiceTF
	{\True Mặt phẳng $(P)$ có một véctơ pháp tuyến là $\overrightarrow{n}=(2; 3; 1)$}
	{\True Mặt phẳng $(P)$ có véctơ pháp tuyến là $\overrightarrow{n}=(6; 9; 3)$}
	{\True Mặt phẳng $(P)$ có véctơ pháp tuyến là $\overrightarrow{n}=(-4;-6;-2)$}
	{Điểm $M(0; 0; 2024)$ không thuộc mặt phẳng $(P)$}
	\loigiai{
		\begin{itemchoice}
			\itemch Véctơ pháp tuyến của $(P)$ là $\overrightarrow{n}=(2; 3; 1)$.
			\itemch $\overrightarrow{n}=(6; 9; 3)=3(2; 3; 1).$
			\itemch $\overrightarrow{n}=(-4;-6;-2)=-2(2; 3; 1).$
			\itemch Thay điểm $M(0; 0; 2024)$ vào mặt phẳng $(P)\colon 2\cdot0+3\cdot 0+2024-2024=0 \Rightarrow M \in(P)$.
		\end{itemchoice}
	}
\end{ex}
\begin{ex}%[2H5H1-3] 
	Trong không gian $O x y z$, cho mặt phẳng $(P)\colon x+y+z-3=0$. Các mệnh đề sau đây đúng hay sai?
	\choiceTF
	{\True Điểm $M(-1;-1;-1)$ \textbf{không thuộc} mặt phẳng $(P)$}
	{\True Điểm $N(1; 1; 1)$ \textbf{thuộc} mặt phẳng $(P)$}
	{\True Điểm $K(-3; 0; 0)$ \textbf{không thuộc} mặt phẳng $(P)$}
	{Điểm $Q(0; 0;-3)$ \textbf{thuộc} mặt phẳng $(P)$}
	\loigiai{
		\begin{itemchoice}
			\itemch Điểm $M(-1;-1;-1)$ có tọa độ không thỏa mãn phương trình mặt phẳng $(P)$ nên $M \notin(P)$.
			\itemch Điểm $N(1; 1; 1)$ có tọa độ thỏa mãn phương trình mặt phẳng $(P)$ nên $N \in(P)$.
			\itemch Điểm $K(-3; 0; 0)$ có tọa độ không thỏa mãn phương trình mặt phẳng $(P)$ nên $K \notin(P)$.
			\itemch Điểm $Q(0; 0;-3)$ có tọa độ không thỏa mãn phương trình mặt phẳng $(P)$ nên $Q \notin(P)$.
		\end{itemchoice}
	}
\end{ex}
\Closesolutionfile{ans}
\indapan{10}{ans/ans2C5B1CD1-DS}
\TNSA
\Opensolutionfile{ans}[ans/ans2C5B1CD1-KQ]
\begin{ex}%[2H2H2-5]
	Trong không gian với hệ trục tọa độ $O x y z$, cho $A(0; 1;-1)$, $B(1; 1; 2)$ và $C(1;-1; 0)$. Biết  $\vec{u}=\left[\overrightarrow{B C}, \overrightarrow{B D}\right]$. Khi đó, độ dài của $\vec{u}$ bằng bao nhiêu?
	\shortans[0]{$4$}
	\loigiai{
	Ta có $\overrightarrow{B C}=(0;-2;-2)$ và  $\overrightarrow{B D}=(-1;-1;-1)$.\\
	Khi đó $\vec{u}=\left[\overrightarrow{B C}, \overrightarrow{B D}\right]=(0; 2;-2)$.\\
	Suy ra $\left|\vec{u}\right|=\sqrt{0^2+2^2+(-2)^2}=4$. 	
	}
\end{ex}

\begin{ex}%[2H2V2-5]
	Trong không gian với hệ trục tọa độ $Oxyz$, cho $A(2; 0; 2)$, $B(1;-1;-2)$ và $C(-1; 1; 0)$. Một véctơ $\overrightarrow{n}=(a; b; 2)$ có phương vuông góc với hai véctơ $\overrightarrow{AB}$ và $\overrightarrow{AC}$. Tính giá trị của $a+b$.
	\shortans[0]{$-8$}
	\loigiai{
	Ta có $\overrightarrow{A C}=(-3; 1;-2)$ và $\overrightarrow{A B}=(-1;-1;-4)$.\\
	Vì $\vec{n}$ có phương vuông góc với $\overrightarrow{AB}$ và $\overrightarrow{AC}$ nên $\vec{n}$ cùng phương với vectơ $\left[\overrightarrow{AB},\overrightarrow{AC}\right]=(-6;-10; 4)$.\\
	Suy ra $\overrightarrow{n}=(-3; -5; 2)$
	Vậy $a+b=-3-5=-8$.
	}
\end{ex}

\begin{ex}%[2H2V2-5]
	Hệ trục tọa độ $Oxyz$, cho bốn điểm $A(1;-2; 0)$, $B(2; 0; 3)$, $C(-2; 1; 3)$ và $D(0; 1; 1)$. Tính giá trị của phép tính $\left[\overrightarrow{AB}, \overrightarrow{AC}\right] \cdot \overrightarrow{AD}$.
	\shortans[0]{$-24$}
	\loigiai
	{
	Ta có $\overrightarrow{AB}=(1; 2; 3)$; $\overrightarrow{AC}=(-3; 3; 3)$; $\overrightarrow{A D}=(-1; 3; 1)$.\\
	Khi đó $\left[\overrightarrow{A B}, \overrightarrow{A C}\right]=(-3;-12; 9)$.\\
	Và $\left[\overrightarrow{A B}, \overrightarrow{A C}\right] \cdot \overrightarrow{A D}=(-3) \cdot(-1)+(-12) \cdot 3+9 \cdot 1=-24$.
	}
\end{ex}
\begin{ex}%[2H5H1-2] 
	Trong mặt phẳng tọa độ $O x y z$, mặt phẳng $(P)\colon 2 x-6 y-8 z+1=0$ có một véctơ pháp tuyến $\vec{n}=(1;a;b)$. Khi đó tổng $a+b$ bằng bao nhiêu? 
	\shortans[0]{$-7$}
	\loigiai
	{
	Phương trình tổng quát của mặt phẳng $(P)\colon 2 x-6 y-8 z+1=0$ nên một véctơ pháp tuyến của mặt phẳng $(P)$ có tọa độ là $(2;-6;-8)=2\cdot (1;-3;-4)$.\\
	Suy ra $\vec{n}=(1;-3;-4)$, nên $a+b=-3-4=-7$.	
	}
\end{ex}

\begin{ex}%[2H2V2-5]
	Trong không gian với hệ tọa độ $O x y z$, cho $\overrightarrow{u}=(1; 1; 2), \overrightarrow{v}=(-1; m; m-2)$. Tìm giá trị của $m$ dương sao cho $|[\overrightarrow{u}, \overrightarrow{v}]|=\sqrt{14}$.
	\shortans[0]{$1$}
	\loigiai
	{ Ta có {\allowdisplaybreaks
			\begin{eqnarray*}
			&& [\overrightarrow{u}, \overrightarrow{v}]=(-m-2;-m; m+1)\\ &\Rightarrow& |[\overrightarrow{u}, \overrightarrow{v}]|=\sqrt{(m+2)^2+m^2+(m+1)^2}=\sqrt{3 m^2+6 m+5}.
		\end{eqnarray*}}
	Khi đó $$|[\overrightarrow{u}, \overrightarrow{v}]|=\sqrt{14} \Leftrightarrow 3 m^2+6 m+5=14 \Leftrightarrow 3 m^2+6 m-9=0 \Leftrightarrow \hoac{&m=1 \\&m=-3.}$$
	
	}
\end{ex}

\begin{ex}%[2H2V2-5]
	Trong không gian với hệ tọa độ $O x y z$, cho hai véctơ $\overrightarrow{m}=(4; 3; 1), \overrightarrow{n}=(0; 0; 1)$. Gọi $\overrightarrow{p}=\left(a;b;c\right)$ là véctơ cùng hướng với $[\overrightarrow{m}, \overrightarrow{n}]$ (tích có hướng của hai véctơ $\overrightarrow{m}$ và $\overrightarrow{n}$). Biết $|\overrightarrow{p}|=15$, giá trị của tổng $a+b+c$ bằng bao nhiêu?
	\shortans[0]{$3$}
	\loigiai
	{
	Ta có  $[\overrightarrow{m}; \overrightarrow{n}]=(3;-4; 0)$, suy ra $|[\overrightarrow{m}; \overrightarrow{n}]|=5$.\\
	Do $\overrightarrow{p}$ là véctơ cùng hướng với $[\overrightarrow{m}; \overrightarrow{n}]$ nên $\overrightarrow{p}=k[\overrightarrow{m}; \overrightarrow{n}]$, $k>0$.\\
	Mặt khác $|\overrightarrow{p}|=15 \Leftrightarrow k \cdot|[\overrightarrow{m}, \overrightarrow{n}]| =15 \Leftrightarrow k\cdot 5=15 \Leftrightarrow k=3$.\\
	Suy ra $\overrightarrow{p}=(9;-12; 0)$.	\\
	Vậy $a+b+c=9-12+0=3$.
	}
\end{ex}
\Closesolutionfile{ans}
\indapan{6}{ans/ans2C5B1CD1-KQ}
\begin{dang}{Hai mặt phẳng song song, vuông góc. Khoảng cách một điểm đến mặt phẳng}
	\begin{enumerate}[label=\bf\arabic*.]
	\item \textbf{Điều kiện hai mặt phẳng song song, vuông góc:}\\
	Cho 2 mặt phẳng $\left(\alpha_1\right)\colon A_1 x+B_1 y+C_1 z+D_1=0$ và $\left(\alpha_2\right)\colon A_2 x+B_2 y+C_2 z+D_2=0$ có vectơ pháp tuyến lần lượt là $\overrightarrow{n}_1=\left(A_1; B_1; C_1\right), \overrightarrow{n}_2=\left(A_2; B_2; C_2\right)$. Khi đó:
	\begin{itemize}
	\item $\left(\alpha_1\right) \parallel \left(\alpha_2\right) \Leftrightarrow\heva{&\overrightarrow{n}_1=k \overrightarrow{n}_2 \\ &D_1 \neq k D_2} \quad (k \in \mathbb{R})$.
	\item $\left(\alpha_1\right) \equiv\left(\alpha_2\right) \Leftrightarrow\heva{&\overrightarrow{n}_1=k \overrightarrow{n}_2 \\& D_1=k D_2} \quad (k \in \mathbb{R})$.
	\item $\left(\alpha_1\right)$ cắt $\left(\alpha_2\right) \Leftrightarrow \overrightarrow{n}_1$ và $\overrightarrow{n}_2$ không cùng phương.
	\item $\left(\alpha_1\right) \perp\left(\alpha_2\right) \Leftrightarrow \overrightarrow{n}_1 \cdot \overrightarrow{n}_2=0 \Leftrightarrow A_1 A_2+B_1 B_2+C_1 C_2=0$. 	
	\end{itemize}
	\begin{tikzpicture}[line cap=round,line join=round,>=stealth,x=1.0cm,y=1.0cm,scale=0.6]
		\path
		(1,1) coordinate (A)
		(3,3) coordinate (B)
		(8,3) coordinate (C)
		($(A)+(C)-(B)$) coordinate (D)
		(2,4) coordinate (A')
		(4,6) coordinate (B')
		(9,6) coordinate (C')
		($(A')+(C')-(B')$) coordinate (D')
		(5,5) coordinate (K)
		(6,5) coordinate (M)
		($(K)+(0,1.5)$) coordinate (N)
		(6,2) coordinate (H)
		($(C)!0.5!(D)$) coordinate (H')
		;
		\draw (A)--(B)--(C)--(D)--cycle (A')--(B')--(C')--(D')--cycle ;
		\draw[->] (K)--(N) node[right]{$\vec{n}_1$};
		\draw[->] (H)--($(H)+(0,1.5	 )$) node[right]{$\vec{n}_2$};	
		\draw pic[draw,blue,"$\alpha_1$",angle radius=8mm]{angle=D--A--B};
		\draw pic[draw,blue,"$\alpha_2$",angle radius=8mm]{angle=D'--A'--B'};
		\draw pic[draw,blue,,angle radius=3mm]{right angle=M--K--N};
		\draw pic[draw,blue,,angle radius=3mm]{right angle=M--H--H'};
	\end{tikzpicture}
	\begin{tikzpicture}[line cap=round,line join=round,>=stealth,x=1.0cm,y=1.0cm,scale=0.6]
		\path
		(1,1) coordinate (A)
		(3,3) coordinate (B)
		(8,3) coordinate (C)
		($(A)+(C)-(B)$) coordinate (D)
		($(A)+(-2,3)$) coordinate (E)
		($(B)+(-2,3)$) coordinate (F)
		($(A)!0.5!(C)$) coordinate (G)
		($(G)+(0,1.5)$) coordinate (H)
		($(A)!0.5!(F)$) coordinate (I)
		($(I)+(1.7,1.5)$) coordinate (J)
		($(C)!0.5!(D)$) coordinate (K)
		($(A)!0.5!(B)$) coordinate (L)
		;
		\draw (A)--(B)--(C)--(D)--cycle (A)--(E)--(F)--(B) ;
		\draw[->] (G)--(H) node[right]{$\vec{n}_1$};
		\draw[->] (I)--(J) node[right]{$\vec{n}_2$};	
		\draw pic[draw,blue,"$\alpha_1$",angle radius=8mm]{angle=B--C--D};
		\draw pic[draw,blue,"$\alpha_2$",angle radius=5mm]{angle=A--E--F};
		\draw pic[draw,blue,,angle radius=3mm]{right angle=L--I--J};
		\draw pic[draw,blue,,angle radius=3mm]{right angle=H--G--K};
	\end{tikzpicture}
	\begin{tikzpicture}[line cap=round,line join=round,>=stealth,x=1.0cm,y=1.0cm,scale=0.6]
		\path
		(1,1) coordinate (A)
		(3,3) coordinate (B)
		(8,3) coordinate (C)
		($(A)+(C)-(B)$) coordinate (D)
		($(A)+(0,4)$) coordinate (E)
		($(B)+(0,4.)$) coordinate (F)
		($(A)!0.5!(C)$) coordinate (G)
		($(G)+(0,1.5)$) coordinate (H)
		($(A)!0.5!(F)$) coordinate (I)
		($(I)+(1.7,0)$) coordinate (J)
		($(C)!0.5!(D)$) coordinate (K)
		($(A)!0.5!(B)$) coordinate (L)
		;
		\draw (A)--(B)--(C)--(D)--cycle (A)--(E)--(F)--(B) ;
		\draw[->] (G)--(H) node[right]{$\vec{n}_1$};
		\draw[->] (I)--(J) node[above]{$\vec{n}_2$};	
		\draw pic[draw,blue,"$\alpha_1$",angle radius=8mm]{angle=B--C--D};
		\draw pic[draw,blue,"$\alpha_2$",angle radius=5mm]{angle=A--E--F};
		\draw pic[draw,blue,,angle radius=3mm]{right angle=L--I--J};
		\draw pic[draw,blue,,angle radius=3mm]{right angle=H--G--K};
	\end{tikzpicture}
	\begin{note}
	\textbf{Chú ý:}
	\begin{itemize}
		\item $\overrightarrow{a}$ cùng phương với $\overrightarrow{b} \Leftrightarrow[\overrightarrow{a}, \overrightarrow{b}]=\overrightarrow{0}$.
		\item Nếu $\overrightarrow{n}=[\overrightarrow{a}, \overrightarrow{b}]$ thì vectơ $\overrightarrow{n}$ vuông góc với cả hai vectơ $\overrightarrow{a}$ và $\overrightarrow{b}$.
	\end{itemize}
\end{note}
	\item \textbf{Khoảng cách từ một điểm đến một mặt phẳng}
	\immini{
	Trong không gian $O x y z$, cho điểm $M_0\left(x_0; y_0; z_0\right)$ và mặt phẳng $(\alpha)\colon A x+B y+C z+D=0$. Khi đó khoảng cách từ điểm $M_0$ đến mặt phẳng $(\alpha)$ được tính: $$d\left(M_0,(\alpha)\right)=\dfrac{\left|A x_0+B y_0+C z_0+D\right|}{\sqrt{A^2+B^2+C^2}}.$$
}{
	\begin{tikzpicture}[line cap=round,line join=round,>=stealth,x=1.0cm,y=1.0cm,scale=0.6]
		\path
		(1,1) coordinate (A)
		(3,3) coordinate (B)
		(9,3) coordinate (C)
		($(A)+(C)-(B)$) coordinate (D)
		(4,2) coordinate (E)
		(6,2) coordinate (F)
		($(E)+(0,2.5)$) coordinate (G)
		;
		\draw (A)--(B)--(C)--(D)--cycle ;
		\draw[->] (E)--($(E)+(0,2.5)$) node[right]{$M_0$};
		\draw[->] (F)--($(F)+(0,1.5)$) node[right]{$\vec{n}$};	
		\draw pic[draw,blue,"$\alpha$",angle radius=8mm]{angle=D--A--B};
		\draw pic[draw,blue,,angle radius=3mm]{right angle=F--E--G};
		%			\draw pic[draw,blue,,angle radius=3mm]{right angle=H--G--K};
	\end{tikzpicture}
}
	\begin{note}
		\textbf{Chú ý:}
		\begin{itemize}
		\item Mặt phẳng $(O x y)$ có phương trình: $z=0$.
		\item Mặt phẳng $(O x z)$ có phương trình: $y=0$.
		\item Mặt phẳng $(O y z)$ có phương trình: $x=0$.
		\end{itemize}
	\end{note}
	\item \textbf{Khoảng cách hai mặt phẳng song song}\\
	Khoảng cách giữa mặt phẳng song song là khoảng cách từ một điểm thuộc mặt phẳng này đến mặt phẳng kia (Thực chất là khoảng cách từ một điểm đến mặt phẳng).\\
	Để tính khoảng cách mặt phẳng $\left(\alpha_1\right)$ song song với $\left(\alpha_2\right)$, ta thực hiện như sau:
	\begin{enumerate}
		\item[] \textbf{Bước 1:} Chọn điểm $M \in\left(\alpha_1\right)$.
		\item[] \textbf{Bước 2:} Tính khoảng cách điểm $M$ đến $\left(\alpha_2\right)$.
		\item[] \textbf{Bước 3:} Kết luận: $d\left(\left(\alpha_1\right),\left(\alpha_2\right)\right)=d\left(M,\left(\alpha_2\right)\right)$.
	\end{enumerate}
	\begin{note}
		\textbf{Chú ý:}
		Cho 2 mặt phẳng $\left(\alpha_1\right)\colon A x+B y+C z+D_1=0$ và $\left(\alpha_2\right)\colon A x+B y+C z+D_2=0$ có cùng vectơ pháp tuyến là $\overrightarrow{n}=(A; B; C)$. Khi đó khoảng cách giữa hai mặt phẳng đó là: $$d\left(\left(\alpha_1\right),(\alpha_2)\right)=\dfrac{\left|D_1-D_2\right|}{\sqrt{A^2+B^2+C^2}}.$$ 
	\end{note}
	\end{enumerate}
\end{dang}


