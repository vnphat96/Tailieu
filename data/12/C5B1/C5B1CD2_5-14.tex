
\Opensolutionfile{ans}[ans/CD3_17-23]
\TN

\begin{ex}%[2H5H1-3]
	Trong không gian với hệ tọa độ $O x y z$, phương trình nào dưới đây là phương trình mặt phẳng đi qua điểm $M(1 ; 2 ;-3)$ và có một véc-tơ pháp tuyến $\vec{n}=(1 ;-2 ; 3)$.
	\choice
	{\True $x-2 y+3 z+12=0$}
	{$x-2 y-3 z-6=0$}
	{$x-2 y+3 z-12=0$}
	{$x-2 y-3 z+6=0$}
	\loigiai{
	Phương trình mặt phẳng đi qua điểm $M(1 ; 2 ;-3)$ và có một véc-tơ pháp tuyến $\vec{n}=(1 ;-2 ; 3)$ là $$1(x-1)-2(y-2)+3(z+3)=0 \Leftrightarrow x-2y+3 z+12=0.$$
	}
\end{ex}
\begin{ex}%[2H5H1-3] 
	Trong không gian với hệ trục tọa độ $Oxyz$, phương trình mặt phẳng đi qua điểm $A(1 ; 2 ;-3)$ có véc-tơ pháp tuyến $\vec{n}=(2 ;-1 ; 3)$ là
	\choice
	{\True $2 x-y+3 z+9=0$}
	{$2 x-y+3 z-4=0$}
	{$x-2 y-4=0$}
	{$2 x-y+3 z+4=0$}
	\loigiai{Phương trình mặt phẳng đi qua điểm $A(1 ; 2 ;-3)$ có véc-tơ pháp tuyến $\vec{n}=(2 ;-1 ; 3)$ là
		\allowdisplaybreaks
		\begin{eqnarray*}
			&&2(x-1)-1 (y-2)+3 (z+3)=0\\
			&\Leftrightarrow& 2 x-2-y+2+3 z+9=0\\
			&\Leftrightarrow& 2 x-y+3 z+9=0.
		\end{eqnarray*}
	}
\end{ex}

\begin{ex}%[2H5H1-3] 
	Trong không gian $O x y z$, phương trình của mặt phẳng đi qua điểm $A(3 ; 0 ;-1)$ và có véc-tơ pháp tuyến $\vec{n}=(4 ;-2 ;-3)$ là
	\choice
	{$4x-2 y+3z-9=0$}
	{\True $4x-2y-3z-15=0$}
	{$3x-z-15=0$}
	{$4x-2y-3z+15=0$}
	\loigiai{
	Mặt phẳng đi qua điểm $A(3 ; 0 ;-1)$ và có véc-tơ  pháp tuyến $\vec{n}=(4 ;-2 ;-3)$ có phương trình:
	$$4(x-3)-2(y-0)-3(z+1)=0 \Leftrightarrow 4 x-2 y-3 z-15=0.$$
	}
\end{ex}

\begin{ex}%[2H5H1-3]
	Trong KG $Oxyz$, phương trình mặt phẳng qua $A(-1 ; 1 ;-2)$ và có véc-tơ  pháp tuyến $\vec{n}=(1 ;-2 ;-2)$ là
	\choice
	{\True $x-2 y-2 z-1=0$}
	{$-x+y-2z-1=0$}
	{$x-2y-2z+7=0$}
	{$-x+y-2z+1=0$}
	\loigiai{
	Mặt phẳng $(P)$ đi qua $A(-1 ; 1 ;-2)$ và có véc-tơ  pháp tuyến $\vec{n}=(1 ;-2 ;-2)$ nên có phương trình
	$$1(x+1)-2(y-1)-2(z+2)=0 \Leftrightarrow x-2y-2z-1=0.$$
	}
\end{ex}

\begin{ex}%[2H5N1-1] 
	Trong KG $Oxyz$, phương trình mặt phẳng $(Oyz)$ là
	\choice
	{$z=0$}
	{\True $x=0$}
	{$x+y+z=0$}
	{$y=0$}
	\loigiai{
	Mặt phẳng $(Oyz)$ nhận $\vec{i}=(1;0;0)$ làm véc-tơ  pháp tuyến và đi qua gốc tọa độ $O(0;0;0)$ có phương trình là $x=0$.
	}
\end{ex}

\begin{ex}%[2H5N1-1]
	Trong KG $Oxyz$, phương trình của mặt phẳng $(Oxy)$ là
	\choice
	{\True $z=0$}
	{$x=0$}
	{$y=0$}
	{$x+y=0$}
	\loigiai{
	Phương trình của mặt phẳng $(Oxy)$ là $z=0$.
	}
\end{ex}

\begin{ex}%[2H5N1-1] 
	Trong không gian với hệ toạ độ $Oxyz$, phương trình nào dưới đây là phương trình của mặt phẳng $(Oyz)$?
	\choice
	{$y=0$}
	{\True $x=0$}
	{$y-z=0$}
	{$z=0$}
	\loigiai{
	Mặt phẳng $(Oyz)$ đi qua điểm $O(0 ; 0 ; 0)$ và có véc-tơ  pháp tuyến là $\vec{i}=(1 ; 0 ; 0)$ nên ta có phương trình mặt phẳng $(O y z)$ là  $1(x-0)+0(y-0)+0(z-0)=0 \Leftrightarrow x=0$.
	}
\end{ex}
\begin{ex}%[2H5N1-1] 
	Trong không gian với hệ tọa độ $O x y z$, phương trình nào sau đây là phương trình của mặt phẳng $O z x$ ?
	\choice
	{$x=0$}
	{$y-1=0$}
	{\True $y=0$}
	{$z=0$}
	\loigiai{
		Ta có mặt phẳng $(Oxz)$ đi qua điểm $O(0 ; 0 ; 0)$ và vuông góc với trục $O y$ nên có VTPT $\vec{n}=(0 ; 1 ; 0)$.\\
		Do đó phương trình của mặt phẳng $(Oxz)$ là $y=0$.
	}
\end{ex}

\begin{ex}%[2H5H1-3] 
	Trong không gian với hệ tọa độ $O x y z$, phương trình mặt phẳng $(P)$ qua $M(0 ;-2 ; 1)$ và có cặp véc-tơ  chỉ phương $\vec{a}=(1 ; 1 ;-2),$ $ \vec{b}=(1 ; 0 ; 3)$ là
	\choice
	{\True $3 x-5 y-z-6=0$}
	{$3 x-5 y-z+6=0$}
	{$3 x+5 y-z+6=0$}
	{$3 x-5 y+z-6=0$}
	\loigiai{
		Ta có $\vec{n}=[\vec{a}, \vec{b}]=(3 ;-5 ;-1)$.\\
		Mặt phẳng $(P)$ đi qua $M(0 ;-2 ; 1)$ và có véc-tơ  pháp tuyến $\vec{n}=(3 ;-5 ;-1)$ nên có phương trình $$3(x-0)-5(y+2)-(z-1)=0 \Leftrightarrow 3 x-5 y-z-6=0.$$
	}
\end{ex}
\begin{ex}%[2H5H1-3] 
	Trong không gian với hệ tọa độ $O x y z$, cặp véc-tơ  $\vec{a}=(2 ; 1 ;-2), $ $\vec{b}=(1 ; 0 ; 2)$ có giá song song với mặt phẳng $(P)$. Phương trình mặt phẳng $(P)$ qua $C(1 ; 1 ; 3)$ là
	\choice
	{$2 x+6 y-z-7=0$}
	{$2 x-6 y-z+5=0$}
	{$2 x+6 y+z+5=0$}
	{\True $2 x-6 y-z+7=0$}
	\loigiai{
		Ta có $\vec{n}=[\vec{a}, \vec{b}]=(2 ;-6 ;-1)$.\\
		Mặt phẳng $(P)$ đi qua $C(1 ; 1 ; 3)$ và có véc-tơ  pháp tuyến $\vec{n}=(2 ;-6 ;-1)$ nên có phương trình $$2(x-1)-6(y-1)-1(z-3)=0 \Leftrightarrow 2 x-6 y-z+7=0.$$
	}
\end{ex}

\begin{ex}%[2H5H1-3] 
	Trong không gian $O x y z$, cho ba điểm $A(3 ; 0 ; 0),$ $ B(0 ; 1 ; 0)$ và $C(0 ; 0 ;-2)$. Mặt phẳng $(A B C)$ có phương trình là
	\choice
	{$\dfrac{x}{3}+\dfrac{y}{-1}+\dfrac{z}{2}=1$}
	{\True $\dfrac{x}{3}+\dfrac{y}{1}+\dfrac{z}{-2}=1$}
	{$\dfrac{x}{3}+\dfrac{y}{1}+\dfrac{z}{2}=1$}
	{$\dfrac{x}{-3}+\dfrac{y}{1}+\dfrac{z}{2}=1$}
	\loigiai{Theo công thức phương trình mặt chắn, ta có
		$(A B C)\colon  \dfrac{x}{3}+\dfrac{y}{1}+\dfrac{z}{-2}=1$.}
\end{ex}
\begin{ex}%[2H5H1-3] 
	Trong không gian với hệ tọa độ $O x y z$, cho ba điểm $A(0 ; 1 ; 2), $ $B(2 ;-2 ; 1),$ $ C(-2 ; 1 ; 0)$. Khi đó, phương trình mặt phẳng $(A B C)$ là $a x+y-z+d=0$. Hãy xác định $a$ và $d$.
	\choice
	{\True $a=1,$ $ d=1$}
	{$a=6, $ $d=-6$}
	{$a=-1, $ $d=-6$}
	{$a=-6, $ $d=6$}
	\loigiai{
		Ta có $\overrightarrow{A B}=(2 ;-3 ;-1) ; \overrightarrow{A C}=(-2 ; 0 ;-2)$.
		
		$$[\overrightarrow{A B}, \overrightarrow{A C}]=\left(\left|\begin{array}{cc}-3 & -1 \\ 0 & -2\end{array}\right| ;\left|\begin{array}{cc}-1 & 2 \\ -2 & -2\end{array}\right| ;\left|\begin{array}{cc}2 & -3 \\ -2 & 0\end{array}\right|\right)=(6 ; 6 ;-6).$$
		Chọn $\vec{n}=\dfrac{1}{6}[\overrightarrow{A B} ; \overrightarrow{A C}]=(1 ; 1 ;-1)$ là một VTPT của mp$(A B C)$. Ta có 
		$$(A B C)\colon x+y-1-z+2=0 \Leftrightarrow x+y-z+1=0.$$ Vậy $a=1,$ $ d=1$.
	}
\end{ex}

\begin{ex}%[2H5H1-3] 
	Trong không gian $O x y z$, cho điểm $A(0 ;-3 ; 2)$ và mặt phẳng $(P)\colon 2 x-y+3 z+5=0$. Mặt phẳng đi qua $A$ và song song với $(P)$ có phương trình là
	\choice
	{$2 x-y+3 z+9=0$}
	{$2 x+y+3 z-3=0$}
	{$2 x+y+3 z+3=0$}
	{\True $2 x-y+3 z-9=0$}
	\loigiai{
		Gọi $(Q)$ là mặt phẳng cần tìm.\\
		Theo bài $(Q) \parallel (P) \Rightarrow(Q)\colon 2 x-y+3 z+m=0\,(m \neq 5)$.\\
		Mà $(Q)$ qua $A \Leftrightarrow 2\cdot 0-(-3)+3\cdot 2+m=0 \Leftrightarrow m=-9$.\\
		Vậy $(Q)\colon 2 x-y+3 z-9=0$.
	}
\end{ex}
\begin{ex}%[2H5H1-3] 
	Trong không gian $O x y z$, cho hai điểm $A(0 ; 0 ; 1)$ và $B(1 ; 2 ; 3)$. Mặt phẳng đi qua $A$ và vuông góc với $A B$ có phương trình là
	\choice
	{$x+2 y+2 z-11=0$}
	{\True $x+2 y+2 z-2=0$}
	{$x+2 y+4 z-4=0$}
	{$x+2 y+4 z-17=0$}
	\loigiai{
		Ta có $\overrightarrow{A B}=(1 ; 2 ; 2)$.\\
		Mặt phẳng đi qua $A$ và vuông góc với $A B$ nên nhận $\overrightarrow{A B}=(1 ; 2 ; 2)$ làm véc-tơ pháp tuyến có phương trình $$1(x-0)+2(y-0)+2(z-1)=0 \Leftrightarrow x+2 y+2 z-2=0.$$
	}
\end{ex}

\begin{ex}%[2H5H1-3] 
	Trong mặt phẳng $O x y z$, cho hai điểm $A(1 ; 0 ; 0)$ và $B(3 ; 2 ; 1)$. Mặt phẳng đi qua $A$ và vuông góc với $A B$ có phương trình là
	\choice
	{\True $2 x+2 y+z-2=0$}
	{$4 x+2 y+z-17=0$}
	{$4 x+2 y+z-4=0$}
	{$2 x+2 y+z-11=0$}
	\loigiai{
		Mặt phẳng đi qua $A$ và vuông góc với $A B$ nên nhận $\overrightarrow{A B}=(2 ; 2 ; 1)$ làm véc-tơ pháp tuyến.\\
		Vậy phương trình mặt phẳng cần tìm là $$2(x-1)+2 y+z=0 \Leftrightarrow 2 x+2 y+z-2=0.$$
	}
\end{ex}
\begin{ex}%[2H5H1-3] 
	Trong KG $Oxyz$, cho hai điểm $A(0 ; 1 ; 1)$  và $B(1 ; 2 ; 3)$. Viết phương trình của mặt phẳng $(P)$ đi qua $A$ và vuông góc với đường thẳng $A B$.
	\choice
	{\True $x+y+2 z-3=0$}
	{$x+y+2z-6=0$}
	{$x+3y+4z-7=0$}
	{$x+3y+4z-26=0$}
	\loigiai{
	Mặt phẳng $(P)$ đi qua $A(0 ; 1 ; 1)$ và nhận véc-tơ $\overrightarrow{A B}=(1 ; 1 ; 2)$ là véc-tơ pháp tuyến
		$$(P)\colon 1(x-0)+1(y-1)+2(z-1)=0 \Leftrightarrow x+y+2 z-3=0.$$
	}
\end{ex}

\begin{ex}%[2H5H1-3] 
	Trong không gian $O x y z$, cho ba điểm $A(-1 ; 1 ; 1),$ $ B(2 ; 1 ; 0),$ $ C(1 ;-1 ; 2)$. Mặt phẳng đi qua $A$ và vuông góc với đường thẳng $B C$ có phương trình là
	\choice
	{$3 x+2 z+1=0$}
	{\True $x+2 y-2 z+1=0$}
	{$x+2 y-2 z-1=0$}
	{$3 x+2 z-1=0$}
	\loigiai{
		Ta có $\overrightarrow{B C}=(-1 ;-2 ; 2)$ là một véc-tơ  pháp tuyến của mặt phẳng $(P)$ cần tìm.\\
		$\vec{n}=-\overrightarrow{B C}=(1 ; 2 ;-2)$ cũng là một véc-tơ  pháp tuyến của mặt phẳng $(P)$.\\
		Vậy phương trình mặt phẳng $(P)$ là $x+2 y-2 z+1=0$.
	}
\end{ex}
\begin{ex}%[2H5H1-3] 
	Trong không gian với hệ tọa độ $O x y z$, cho các điểm $A(0 ; 1 ; 2), $ $B(2 ;-2 ; 1)$, $C(-2 ; 0 ; 1)$. Phương trình mặt phẳng đi qua $A$ và vuông góc với $B C$ là
	\choice
	{$y+2 z-5=0$}
	{$2 x-y-1=0$}
	{\True $2 x-y+1=0$}
	{$-y+2 z-5=0$}
	\loigiai{
		Ta có véc-tơ  pháp tuyến của mặt phẳng $(P)$ là $\overrightarrow{B C}=(-4 ; 2 ; 0)$.\\
		Phương trình mặt phẳng $(P)$ là
		$$-4(x-0)+2(y-1)+0(z-2)=0 \Leftrightarrow-4 x+2 y-2=0 \Leftrightarrow 2 x-y+1=0.$$}
\end{ex}

\begin{ex}%[2H5H1-3] 
	Trong không gian $O x y z$, mặt phẳng $(P)$ đi qua hai điểm $A(0 ; 1 ; 0)$, $B(2 ; 3 ; 1)$ và vuông góc với mặt phẳng $(Q)\colon x+2 y-z=0$ có phương trình là
	\choice
	{$4x-3y+2z+3=0$}
	{\True $4 x-3 y-2 z+3=0$}
	{$2 x+y-3 z-1=0$}
	{$4 x+y-2 z-1=0$}
	\loigiai{
	Ta có $\overrightarrow{A B}=(2 ; 2 ; 1)$, véc-tơ  pháp tuyến mặt phẳng $(Q)\colon \overrightarrow{n}_Q=(1 ; 2 ;-1)$.\\
	Theo đề bài ta có véc-tơ  pháp tuyến mặt phẳng $(P)\colon \overrightarrow{n}_P=\left[\overrightarrow{n}_Q , \overrightarrow{A B}\right]=(4 ;-3 ;-2)$.\\
	Phương trình mặt phẳng $(P)$ có dạng $4 x-3 y-2 z+C=0$.\\
		Mặt phẳng $(P)$ đi qua $A(0 ; 1 ; 0)$ nên $-3+C=0 \Leftrightarrow C=3$.\\
		Vậy phương trình mặt phẳng $(P)$ là $4 x-3 y-2 z+3=0$.
	}
\end{ex}
\begin{ex}%[2H5H1-3] 
	Cho hai mặt phẳng $(\alpha)\colon  3 x-2 y+2 z+7=0,$ $(\beta)\colon 5 x-4 y+3 z+1=0$. Phương trình mặt phẳng đi qua gốc tọa độ $O$ đồng thời vuông góc với cả $(\alpha)$ và $(\beta)$ là
	\choice
	{$2 x-y-2 z=0$}
	{$2 x-y+2 z=0$}
	{\True $2 x+y-2 z=0$}
	{$2 x+y-2 z+1=0$}
	\loigiai{
		Véc-tơ pháp tuyến của hai mặt phẳng lần lượt là $\overrightarrow{n}_\alpha=(3 ;-2 ; 2), \overrightarrow{n}_\beta=(5 ;-4 ; 3)$.\\
		Suy ra $\left[\overrightarrow{n}_\alpha ; \overrightarrow{n}_\beta\right]=(2 ; 1 ;-2)$ là véc-tơ pháp tuyến của mặt phẳng cần tìm.\\
		Phương trình mặt phẳng đi qua gốc tọa độ $O, $ có véc-tơ pháp tuyến $\vec{n}=(2 ; 1 ;-2)$ là $2 x+y-2 z=0$.
	}
\end{ex}

\begin{ex}%[2H5H1-3] 
	Trong không gian với hệ tọa độ $O x y z$, cho điểm $A(2 ; 4 ; 1) ;$ $ B(-1 ; 1 ; 3)$ và mặt phẳng $(P)\colon x-3 y+2 z-5=0$. Một mặt phẳng $(Q)$ đi qua hai điểm $A, B$ và vuông góc với mặt phẳng $(P)$ có dạng $a x+b y+c z-11=0$. Khẳng định nào sau đây là đúng?
	\choice
	{\True $a+b+c=5$}
	{$a+b+c=15$}
	{$a+b+c=-5$}
	{$a+b+c=-15$}
	\loigiai{Vì $(Q)$ vuông góc với $(P)$ nên $(Q)$ nhận véc-tơ pháp tuyến $\vec{n}=(1 ;-3 ; 2)$ của $(P)$ làm véc-tơ chỉ phương.\\
		Mặt khác $(Q)$ đi qua $A$ và $B$ nên $(Q)$ nhận $\overrightarrow{A B}=(-3 ;-3 ; 2)$ làm véc-tơ chỉ phương.\\
		$(Q)$ nhận $\overrightarrow{n}_Q=[\vec{n}, \overrightarrow{A B}]=(0 ; 8 ; 12)$ làm véc-tơ pháp tuyến.\\
		Vậy phương trình mặt phẳng $(Q)\colon  0(x+1)+8(y-1)+12(z-3)=0\Leftrightarrow 2 y+3 z-11=0$.\\
		Vậy $a+b+c=5$.}
\end{ex}

\begin{ex}%[2H5V1-3]
	Trong không gian $O x y z$, cho hai mặt phẳng $(P)\colon  x-3 y+2 z-1=0$, 
	$(Q)\colon  x-z+2=0$. Mặt phẳng $(\alpha)$ vuông góc với cả $(P)$ và $(Q)$ đồng thời cắt trục $O x$ tại điểm có hoành độ bằng 3 . Phương trình của $(\alpha)$ là
	\choice
	{\True $x+y+z-3=0$}
	{$x+y+z+3=0$}
	{$-2 x+z+6=0$}
	{$-2 x+z-6=0$}
	\loigiai{
		$(P)$ có véc-tơ pháp tuyến $\overrightarrow{n}_P=(1 ;-3 ; 2),(Q)$ có véc-tơ pháp tuyến $\overrightarrow{n}_Q=(1 ; 0 ;-1)$.\\
		Vì mặt phẳng $(\alpha)$ vuông góc với cả $(P)$ và $(Q)$ nên $(\alpha)$ có một véc-tơ pháp tuyến là $\left[\overrightarrow{n}_P, \overrightarrow{n}_Q\right]=(3 ; 3 ; 3)=3(1 ; 1 ; 1)$.\\
		Vì mặt phẳng $(\alpha)$ cắt trục $O x$ tại điểm có hoành độ bằng $3$ nên $(\alpha)$ đi qua điểm $M(3 ; 0 ; 0)$.\\
		Vậy $(\alpha)$ đi qua điểm $M(3 ; 0 ; 0)$ và có véc-tơ pháp tuyến $\overrightarrow{n}_\alpha=(1 ; 1 ; 1)$ nên $(\alpha)$ có phương trình: $x+y+z-3=0$.
	}
\end{ex}

\begin{ex}%[2H5H1-3] 
	Trong không gian với hệ trục tọa độ $O x y z$, cho mặt phẳng $(P)\colon  a x+b y+c z-9=0$ chứa hai điểm $A(3 ; 2 ; 1),$ $ B(-3 ; 5 ; 2)$ và vuông góc với mặt phẳng $(Q)\colon  3 x+y+z+4=0$. Tính tổng $S=a+b+c$?
	\choice
	{$S=-12$}
	{$S=2$}
	{\True $S=-4$}
	{$S=-2$}
	\loigiai{
		$\overrightarrow{A B}=(-6 ; 3 ; 1)$.\\
		$\overrightarrow{n}_{(Q)}=(3 ; 1 ; 1)$ là véc-tơ pháp tuyến  của $(Q)$.\\
		Mặt phẳng $(P)$ chứa hai điểm $A(3 ; 2 ; 1),$ $ B(-3 ; 5 ; 2)$ và vuông góc với mặt phẳng $(Q)$.\\
		Suy ra $ \overrightarrow{n}_{(P)}=\left[\overrightarrow{A B}, \overrightarrow{n}_{(Q)}\right]=(2 ; 9 ;-15)$ là véc-tơ pháp tuyến  của $(P)$.\\
		$A(3 ; 2 ; 1) \in(P)\Rightarrow(P)\colon 2 x+9 y-15 z-9=0$ hoặc $(P)\colon -2 x-9 y+15 z+9=0$.\\
		Mặt khác $(P)\colon a x+b y+c z-9=0 \Rightarrow a=2 ; $ $b=9 ;$ $ c=-15$.\\
		Vậy $S=a+b+c=2+9+(-15)=-4$.
	}
\end{ex}
\begin{ex}%[2H5H1-3] 
	Trong không gian $O x y z$, phương trình của mặt phẳng $(P)$ đi qua điểm $B(2 ; 1 ;-3)$, đồng thời vuông góc với hai mặt phẳng $(Q)\colon x+y+3 z=0,$ $(R)\colon 2 x-y+z=0$ là
	\choice
	{$4 x+5 y-3 z+22=0$}
	{$4 x-5 y-3 z-12=0$}
	{$2 x+y-3 z-14=0$}
	{\True $4 x+5 y-3 z-22=0$}
	\loigiai{
		Mặt phẳng $(Q)\colon x+y+3 z=0,$ $(R)\colon 2 x-y+z=0$ có các véc-tơ pháp tuyến lần lượt là $\overrightarrow{n}_1=(1 ; 1 ; 3)$ và $\overrightarrow{n}_2=(2 ;-1 ; 1)$.\\
		Vì $(P)$ vuông góc với hai mặt phẳng $(Q),$ $(R)$ nên $(P)$ có véc-tơ pháp tuyến là $\vec{n}=\left[\overrightarrow{n}_1, \overrightarrow{n}_2\right]=(4 ; 5 ;-3)$.\\
		Ta lại có $(P)$ đi qua điểm $B(2 ; 1 ;-3)$ nên $$(P)\colon 4(x-2)+5(y-1)-3(z+3)=0\Leftrightarrow 4 x+5 y-3 z-22=0.$$
	}
\end{ex}
\Closesolutionfile{ans}
\indapan{10}{ans/CD3_17-23}

\Opensolutionfile{ans}[ans/CD3_17-23DS]
\TNTF

\begin{ex}%[2H5H1-3]
	Trong KG $Oxyz$, cho điểm $A(1; -2; 3)$ và hai véc-tơ  $\overrightarrow{v}=(-1; 2; 3)$, $\overrightarrow{u}=(-2; 0; 1)$.
	\choiceTF
		{\True $\overrightarrow{v}=-\overrightarrow{i}+2\overrightarrow{j}+3\overrightarrow{k}$}
		{$\overrightarrow{u}\perp \overrightarrow{v}$}
		{\True Phương trình mặt phẳng đi qua điểm $A(1; -2; 3)$ và vuông góc với giá của véc-tơ  $\overrightarrow{v}=(-1; 2; 3)$ là $x-2y-3z+4=0$}
		{Phương trình mặt phẳng đi qua điểm $A(1; -2; 3)$ và vuông góc với giá của véc-tơ $\overrightarrow{u}=(-2; 0; 1)$ là $2x-y+1=0$}
	\loigiai{
		\begin{itemchoice}
		\itemch Đúng. \\Ta có $\overrightarrow{v}=(-1; 2; 3) \Leftrightarrow \overrightarrow{v}=-\overrightarrow{i}+2\overrightarrow{j}+3\overrightarrow{k}$.
		\itemch Sai.\\ Ta có $\overrightarrow{u}\cdot \overrightarrow{v} = 2 + 0 + 3 = 5\neq 0 \Rightarrow \overrightarrow{u}\not \perp \overrightarrow{v}$.
		\itemch Đúng.\\ Mặt phẳng đi qua điểm $A(1; -2; 3)$ và vuông góc với giá của véc-tơ  $\overrightarrow{v}=(-1; 2; 3)$ có phương trình
		\[-1(x-1)+2(y+2)+3(z-3)=0\Leftrightarrow x-2y-3z+4=0.\]
		\itemch Sai.\\ Mặt phẳng đi qua điểm $A(1; -2; 3)$ và vuông góc với giá của véc-tơ $\overrightarrow{u}=(-2; 0; 1)$ có phương trình
		\[ -2(x-1)+0(y+2)+1(z-3)=0\Leftrightarrow 2x -z +1=0.\]
		\end{itemchoice}
		}
\end{ex}
\begin{ex}%[2H5H1-3]
	Trong KG $Oxyz$, cho ba điểm $A(1;1;4)$, $B(2;7;9)$, $C(0;9;13)$.
	\choiceTF
	{\True $\overrightarrow{AB}=\overrightarrow{i}+6\overrightarrow{j}+5\overrightarrow{k}$}
	{$\overrightarrow{AB}\perp \overrightarrow{AC}$}
	{\True Phương trình mặt phẳng đi qua ba điểm $A$, $B$, $C$ là $x-y+z-4=0$}
	{Phương trình mặt phẳng đi qua ba điểm $A$, $B$, $C$ là $2x+y-z-2=0$}
	\loigiai{
		\begin{itemchoice}
			\itemch $\overrightarrow{AB}=(1; 6; 5) \Rightarrow \overrightarrow{AB}=\overrightarrow{i}+6\overrightarrow{j}+5\overrightarrow{k}$.
			\itemch Ta có $\overrightarrow{AC}=(-1; 8; 9)$, khi đó $\overrightarrow{AB}\cdot \overrightarrow{AC}= -1 + 48 + 45 = 92 \neq 0 \Rightarrow \overrightarrow{AB}\not\perp \overrightarrow{AC}$.
			\itemch Ta có $[\overrightarrow{AB}, \overrightarrow{AC}]= (14;-14;14)=14(1;-1;1)$.\\
			Mặt phẳng $(ABC)$ đi qua điểm $A$ và có véc-tơ pháp tuyến $\overrightarrow{n}=(1;-1;1)$ là $x - y + z - 4 = 0$.
			\itemch Phương trình mặt phẳng đi qua ba điểm $A,B,C$ là $x - y + z - 4 = 0$.			
		\end{itemchoice}
	}
\end{ex}
\begin{ex}%[2H5H1-3]
	Trong KG $Oxyz$, cho điểm $M(2; -1; 4)$ và mặt phẳng $(P)\colon 3x - 2y+z+1=0$.
	\choiceTF
	{\True Mặt phẳng $(P)$ có một vec-tơ pháp tuyến là $\overrightarrow{n}=(-3; 2; -1)$}
	{Mặt phẳng $(P)$ đi qua điểm $B(-1; 1; 2)$}
	{\True Phương trình của mặt phẳng $(Q)$ đi qua điểm $M$ và song song với mặt phẳng $(P)$ là $3x-2y+z-12=0$}
	{Phương trình của mặt phẳng $(R)$ đi qua điểm $O$, $M$ và vuông góc với mặt phẳng $(P)$ là $7x+my+nz=0$. Khi đó $m+n=8$}
	\loigiai{
		\begin{itemchoice}
			\itemch Mặt phẳng $(P)$ có vec-tơ pháp tuyến là $\overrightarrow{n}=(3;-2;1)=-(-3;2;-1)$.
			\itemch Ta có $3\cdot (-1)-2\cdot(1)+2+1=-2\neq 0$. Suy ra mặt phẳng $(P)$ không đi qua điểm $B$.
			\itemch Mặt phẳng $(Q)$ song song với mặt phẳng $(P)$ có dạng $3x-2y+z+d=0$.\\
			Vì $M\in (Q) \Rightarrow d = -12$. Vậy phương trình mặt phẳng $(Q)\colon 3x-2y+z-12=0$.
			\itemch Ta có mặt phẳng $(R)$ đi qua điểm $O$, $M$ và vuông góc với mặt phẳng $(P)$ cho nên mặt phẳng $(R)$ có vec-tơ pháp tuyến là $\overrightarrow{n}_R=\left[\overrightarrow{OM}, \overrightarrow{n}_P\right] =(7;10;-1)$.\\
			Mặt phẳng $(R)$ đi qua điểm $O$ và có vec-tơ pháp tuyến $\overrightarrow{n}_R=(7;10;-1)$ có phương trình $7x+10y-z=0$. Khi đó $m+n=9$.
	\end{itemchoice}}
\end{ex}
\begin{ex}%[2H5H1-3]
	Trong KG $Oxyz$, cho hai điểm $A(1;0;0)$, $B(4;1;2)$.
	\choiceTF
	{$\overrightarrow{AB}=(5;1;2)$}
	{\True Nếu $I$ là trung điểm đoạn thẳng $AB$ thì $I\left(\dfrac{5}{2};\dfrac{1}{2};1\right)$}
	{\True Mặt phẳng $(\alpha) $ đi qua $A$ và vuông góc với $AB$ có phương trình là $3x+y+2z-3=0$}
	{Mặt phẳng trung trực của đoạn thẳng $AB$ có phương trình là $3x+y+2z-12=0$}
	\loigiai{
		\begin{itemchoice}			
			\itemch Ta có $\overrightarrow{AB}=(3;1;2)$.
			\itemch Nếu $I$ là trung điểm đoạn thẳng $AB$ thì $I\left(\dfrac{5}{2};\dfrac{1}{2};1\right)$.
			\itemch Mặt phẳng $(\alpha) $ vuông góc với $AB$ cho nên mặt phẳng $(\alpha)$ có vec-tơ pháp tuyến $\overrightarrow{n}=\overrightarrow{AB}=(3;1;2)$.\\
			Mặt phẳng $(\alpha)$ đi qua $A$ và có vec-tơ pháp tuyến $\overrightarrow{n}=(3;1;2)$ có phương trình là $3x+y+2z-3=0$.
			\itemch Mặt phẳng trung trực của đoạn thẳng $AB$ là mặt phẳng đi qua điểm $I$ và vuông góc $AB$ nên có phương trình là
			\[\begin{array}{l} {3\left(x-\dfrac{5}{2} \right)+y-\dfrac{1}{2}+2\left(z-1\right)=0} \\ {\Leftrightarrow 3x+y+2z-10=0} \end{array}\]
	\end{itemchoice}}
\end{ex}
\begin{ex}%[2H5H1-3]
	Trong không gian với hệ trục tọa độ $Oxyz$, cho điểm $M(1;2;3)$. Gọi $A$, $B$, $C$ lần lượt là hình chiếu vuông góc của $M$ trên các trục $Ox$, $Oy$, $Oz$.
	\choiceTF
	{\True Điểm $A$ có tọa độ là $A\left(1;0;0\right)$}
	{Điểm $B$ có tọa độ là $B\left(1;2;0\right)$}
	{$\overrightarrow{BC}=(-1;-2;3)$}
	{Phương trình mặt phẳng $(ABC)$ là $\dfrac{x}{1}+\dfrac{y}{2}+\dfrac{z}{3}=0$}
	\loigiai{
		\begin{itemchoice}
			\itemch Điểm $A$ có tọa độ là $A\left(1;0;0\right)$.
			\itemch Điểm $B$ có tọa độ là $B\left(0;2;0\right)$.
			\itemch Ta có $C(0;0;3)$. Suy ra $\overrightarrow{BC}=(0;-2;3)$.
			\itemch Mặt phẳng $(ABC)$ là $\dfrac{x}{1}+\dfrac{y}{2}+\dfrac{z}{3}=1$.
		\end{itemchoice}
	}
\end{ex}