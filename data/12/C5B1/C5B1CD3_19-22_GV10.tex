%%%==============Bai_BT1==============%%%
\begin{ex}%[2H5C1-5]
	Cho hình chóp $S.ABCD$ có đáy $ABCD$ là hình vuông cạnh $a$, cạnh bên $SA=a$ và vuông góc với mặt phẳng đáy. Gọi $M$, $N$ lần lượt là trung điểm của $SB$ và $SD$ và $G$ là trọng tâm của tam giác $AMN$.  Biết độ dài đoạn $BG$ có dạng $x\cdot a$. Hỏi giá trị $x$ bằng bao nhiêu? (Kết quả được làm tròn đến hàng phần trăm).
	
\shortans{$0{,}87$}
\loigiai{
	\immini{Đặt hệ trục tọa độ $Oxyz$ như hình vẽ. Khi đó\\ 
	$A\equiv O (0;0;0)$, $B\left(a;0;0\right)$, $D\left(0;a;0\right)$, $S\left(0;0;a\right)$.\\ 
	Suy ra $M\left(\dfrac{a}{2};0;\dfrac{a}{2} \right)$ và $N\left(0;\dfrac{a}{2};\dfrac{a}{2} \right)$.\\
Vì $G$ là trọng tâm của tam giác $AMN$ nên $G\left(\dfrac{a}{6};\dfrac{a}{6};\dfrac{a}{3} \right)$.\\
Khi đó độ dài đoạn $BG$ là
$$BG=\sqrt{\left(\dfrac{5a}{6}\right)^2+\left(\dfrac{a}{6}\right)^2+\left(\dfrac{a}{6}\right)^2}=\dfrac{\sqrt{3}}{2}a\approx 0{,}87 a.$$}{\begin{tikzpicture}[scale=0.9]
	\def\a{3.5}
	\def\h{3.5}
	\path 	(0:0) coordinate (A)
			++(0:\a) coordinate (D)
			++(-130:\a/2) coordinate (C)
			($(A)+(C)-(D)$) coordinate (B)
			($(A)+(90:\h)$) coordinate (S)
			(intersection of A--C and B--D) coordinate (O)
			($(S)!0.5!(B)$) coordinate (M)
			($(S)!0.5!(D)$) coordinate (N)
			($(M)!0.5!(N)$) coordinate (I)
			($(A)!2/3!(I)$) coordinate (G);
	\draw[dashed,thick] 	(B)--(A)--(D)	(A)--(S) (A)--(M)--(N)--(A) (B)--(D);
	\draw [-stealth,thick]  (B) -- ($(B)!-1/2!(A)$)node[left, below]{$x$};	
	\draw [-stealth,thick]  (S) -- ($(S)!-1/3!(A)$)node[above]{$z$};
	\draw [-stealth,thick]  (D) -- ($(D)!-1/4!(A)$)node[right]{$y$};
	\draw[thick] 			(B)--(C)--(D)
			(B)--(S)	(C)--(S)	(D)--(S);
	\foreach \x/\g in {A/-65,B/150,C/-45,D/45,S/180,M/150,N/20,G/0}
			\fill[black] 	(\x) circle (1.5pt)
			($(\g:3mm)+(\x)$) node {$\x$};
\end{tikzpicture}}
	}
\end{ex}
%%%==============HetBai_BT1==============%%%

%%%==============Bai_BT2==============%%%
\begin{ex}%[2H5C1-5]
	Cho hình chóp $S.ABCD$ có đáy $ABCD$ là hình vuông cạnh $a$, cạnh bên $SA=a$ và vuông góc với mặt phẳng đáy. Gọi $M$, $N$ lần lượt là trung điểm của $SB$ và $SD$ và $G$ là trọng tâm của tam giác $AMN$. Khoảng cách từ điểm $G$ đến mặt phẳng $\left(SBC\right)$ là bao nhiêu nếu $a=6\sqrt{3}$?
	
\shortans{$2$}
\loigiai{
	\immini{Chọn hệ trục tọa độ $Oxyz$ thỏa mãn:\\ 
	$A\equiv O(0;0;0)$, $B\left(a;0;0\right)$, $D\left(0;a;0\right)$, $S\left(0;0;a\right)$.\\ 
	Do đó $C(a;a;0)$.\\ 
	Suy ra $M\left(\dfrac{a}{2};0;\dfrac{a}{2} \right)$ và $N\left(0;\dfrac{a}{2};\dfrac{a}{2} \right)$.\\
	Vì $G$ là trọng tâm của tam giác $AMN$ nên $G\left(\dfrac{a}{6};\dfrac{a}{6};\dfrac{a}{3} \right)$.\\
	Phương trình mặt phẳng $(SBD)$ là 
	$$\dfrac{x}{a}+\dfrac{y}{a}+\dfrac{z}{a}=1.$$
	Do đó khoảng cách từ $G$ đến mặt phẳng $\left(SBC\right)$ là
	$$\mathrm{d}\left(G,(SBD)\right)=\dfrac{\left| \dfrac{1}{a}\cdot\dfrac{a}{6}+\dfrac{1}{a}\cdot\dfrac{a}{6}+\dfrac{1}{a}\cdot\dfrac{a}{3}-1\right|}{\sqrt{\left(\dfrac{1}{a}\right)^2+\left(\dfrac{1}{a}\right)^2+\left(\dfrac{1}{a}\right)^2}}=\dfrac{a}{3\sqrt{3}}=2.$$}{\begin{tikzpicture}[scale=0.9]
	\def\a{3.5}
	\def\h{3.5}
	\path 	(0:0) coordinate (A)
			++(0:\a) coordinate (D)
			++(-130:\a/2) coordinate (C)
			($(A)+(C)-(D)$) coordinate (B)
			($(A)+(90:\h)$) coordinate (S)
			(intersection of A--C and B--D) coordinate (O)
			($(S)!0.5!(B)$) coordinate (M)
			($(S)!0.5!(D)$) coordinate (N)
			($(M)!0.5!(N)$) coordinate (I)
			($(A)!2/3!(I)$) coordinate (G);
	\draw[dashed,thick] 	(B)--(A)--(D)	(A)--(S) (A)--(M)--(N)--(A) (B)--(D);
	\draw [-stealth,thick]  (B) -- ($(B)!-1/2!(A)$)node[left, below]{$x$};	
	\draw [-stealth,thick]  (S) -- ($(S)!-1/3!(A)$)node[above]{$z$};
	\draw [-stealth,thick]  (D) -- ($(D)!-1/4!(A)$)node[right]{$y$};
	\draw[thick] 			(B)--(C)--(D)
			(B)--(S)	(C)--(S)	(D)--(S);
	\foreach \x/\g in {A/-65,B/150,C/-45,D/45,S/180,M/150,N/20,G/10}
			\fill[black] 	(\x) circle (1.5pt)
			($(\g:3mm)+(\x)$) node {$\x$};
\end{tikzpicture}}
	}
\end{ex}
%%%==============HetBai_BT2==============%%%

%%%==============Bai_BT3==============%%%
\begin{ex}%[2H5C1-5]
	Cho hình chóp $S.ABCD$ có đáy $ABCD$ là hình vuông cạnh $a$, cạnh bên $SA=a$ và vuông góc với mặt phẳng đáy. Gọi $M$, $N$ lần lượt là trung điểm của $SB$ và $SD$ và $G$ là trọng tâm của tam giác $AMN$. Tính khoảng cách từ điểm $C$ đến mặt phẳng $\left(AMN\right)$ biết $a=\sqrt{3}$.
	
\shortans{$2$}
\loigiai{
	\immini{Chọn hệ trục tọa độ $Oxyz$ thỏa mãn: $A\equiv O$, $B\left(a;0;0\right)$, $D\left(0;a;0\right)$, $S\left(0;0;a\right)$. Do đó $C(a;a;0)$.\\ 
	Suy ra $M\left(\dfrac{a}{2};0;\dfrac{a}{2} \right)$ và $N\left(0;\dfrac{a}{2};\dfrac{a}{2} \right)$.\\
	Vì $G$ là trọng tâm của tam giác $AMN$ nên $G\left(\dfrac{a}{6};\dfrac{a}{6};\dfrac{a}{3} \right)$.\\
	Ta có $AC$ là hình chiếu vuông góc của $SC$ lên mặt phẳng $(ABCD)$. Mà $AC \perp BD$ nên $SC \perp BD$.\\
	Hơn nữa vì $MN \parallel BD$ (tính chất đường trung bình) nên $SC \perp MN$. \quad(1)\\
	Lại có do $\triangle SAB$ cân tại $A$ có $M$ là trung điểm $SB$ nên $AM \perp SB$.\\
	Hơn nữa vì $BC \perp (SAB)$ nên $BC \perp AM$.\\ 
	Do đó $AM \perp (SBC)$.\\
	Suy ra $AM \perp SC$. \quad(2)}{\begin{tikzpicture}[scale=1]
	\def\a{3.5}
	\def\h{3.5}
	\path 	(0:0) coordinate (A)
			++(0:\a) coordinate (D)
			++(-130:\a/2) coordinate (C)
			($(A)+(C)-(D)$) coordinate (B)
			($(A)+(90:\h)$) coordinate (S)
			(intersection of A--C and B--D) coordinate (O)
			($(S)!0.5!(B)$) coordinate (M)
			($(S)!0.5!(D)$) coordinate (N)
			($(M)!0.5!(N)$) coordinate (I)
			($(A)!2/3!(I)$) coordinate (G);
	\draw[dashed,thick] 	(B)--(A)--(D)	(A)--(S) (A)--(M)--(N)--(A) (B)--(D);
	\draw [-stealth,thick]  (B) -- ($(B)!-1/2!(A)$)node[left, below]{$x$};	
	\draw [-stealth,thick]  (S) -- ($(S)!-1/3!(A)$)node[above]{$z$};
	\draw [-stealth,thick]  (D) -- ($(D)!-1/4!(A)$)node[right]{$y$};
	\draw[thick] 			(B)--(C)--(D)
			(B)--(S)	(C)--(S)	(D)--(S);
	\foreach \x/\g in {A/-65,B/150,C/-45,D/45,S/180,M/150,N/20,G/0}
			\fill[black] 	(\x) circle (1.5pt)
			($(\g:3mm)+(\x)$) node {$\x$};
\end{tikzpicture}}
\noindent Từ (1) và (2) ta có $SC \perp (AMN)$, hay $\overrightarrow{SC}$ là véc-tơ pháp tuyến của mặt phẳng $(AMN)$.\\ 
	Hay mặt phẳng $(AMN)$ có một véc-tơ pháp tuyến $\overrightarrow{n} = (1;1;-1)$.\\
	Phương trình mặt phẳng $(AMN)$ là 
	$$x+y-z=0.$$
	Do đó khoảng cách từ $C$ đến mặt phẳng $\left(AMN\right)$ là
	$$\mathrm{d}\left(C,(AMN)\right)=\dfrac{\left| a+a-0\right|}{\sqrt{1^2+1^2+\left(-1\right)^2}}=\dfrac{2a}{\sqrt{3}}=2.$$
	}
\end{ex}
%%%==============HetBai_BT3==============%%%

%%%==============Bai_BT4==============%%%
\begin{ex}%[2H5C1-5]
Cho hình chóp $S.ABCD$ có đáy $ABCD$ là hình chữ nhật, $AB=a$, $BC=a\sqrt{3} $, $SA=a$ và $SA$ vuông góc với đáy $ABCD$. Tính khoảng cách từ điểm $C$ đến mặt phẳng $\left(SBD\right)$ biết $a=\sqrt{21}$.

\shortans{$6$}
\loigiai{
\immini{
 Đặt hệ trục tọa độ $Oxyz$ như hình vẽ.\\ 
 Khi đó, ta có
\[A\left(0;0;0\right), B\left(a;0;0\right), C\left(a;a\sqrt{3} ;0\right), D\left(0;a\sqrt{3} ;0\right), S\left(0;0;a\right).\] 
Phương trình mặt phẳng $(SBD)$ là
$$\dfrac{x}{a}+\dfrac{y}{a\sqrt{3}} + \dfrac{z}{a}=1.$$
Do đó khoảng cách từ $C$ đến mặt phẳng $\left(SBD\right)$ là
$$\mathrm{d}\left(C,(SBD)\right)=\dfrac{\left| \dfrac{1}{a}\cdot a+\dfrac{1}{a\sqrt{3}}\cdot a\sqrt{3}+\dfrac{1}{a}\cdot0\right|}{\sqrt{\left(\dfrac{1}{a}\right)^2+\left(\dfrac{1}{a\sqrt{3}}\right)^2+\left(\dfrac{1}{a}\right)^2}}=\dfrac{2\sqrt{21}a}{7}=6.$$}{\begin{tikzpicture}[scale=0.85]
	\def\a{3.5}
	\def\h{\a}
	\path 	(0:0) coordinate (A)
			++(0:\a) coordinate (D)
			++(-130:\a/2) coordinate (C)
			($(A)+(C)-(D)$) coordinate (B)
			($(A)+(90:\h)$) coordinate (S)
			(intersection of A--C and B--D) coordinate (O)
			($(S)!2/3!(O)$) coordinate (G);
	\draw[dashed,thick] 	(B)--(A)--(D)	(A)--(S) (B)--(D) (A)--(C);
	\draw [-stealth,thick]  (B) -- ($(B)!-1/2!(A)$)node[left]{$x$};	
	\draw [-stealth,thick]  (S) -- ($(S)!-1/3!(A)$)node[above]{$z$};
	\draw [-stealth,thick]  (D) -- ($(D)!-1/4!(A)$)node[right]{$y$};
	\draw[thick] 			(B)--(C)--(D) (B)--(S)	(C)--(S)	(D)--(S);
	\foreach \x/\g in {A/180,B/150,C/-45,D/45,S/180}
	\fill[black] 	(\x) circle (1.5pt) ($(\g:3mm)+(\x)$) node {$\x$};
\end{tikzpicture}}
}
\end{ex}
 %%%==============HetBai_BT4==============%%%
 
 %%%==============Bai_BT5==============%%%
\begin{ex}%[2H5C1-5]
 Cho hình chóp $S.ABCD$ có đáy $ABCD$ là hình chữ nhật, $AB=a$, $BC=a\sqrt{3} $, $SA=a$ và $SA$ vuông góc với đáy $ABCD$. Gọi $G$ là trọng tâm của tam giác $SBD$. Tính khoảng cách từ điểm $G$ đến mặt phẳng $\left(SCD\right)$ biết $a=\sqrt{3}$.

\shortans{$0{,}5$}
\loigiai{
\immini{ Đặt hệ trục tọa độ $Oxyz$ như hình vẽ.\\ 
Khi đó, ta có
\[A\left(0;0;0\right), B\left(a;0;0\right), C\left(a;a\sqrt{3} ;0\right), D\left(0;a\sqrt{3} ;0\right), S\left(0;0;a\right).\] 
$G$ là trọng tâm của tam giác $SBD$ $\Rightarrow G\left(\dfrac{a}{3} ;\dfrac{a\sqrt{3} }{3} ;\dfrac{a}{3} \right)$.\\
Gọi phương trình mặt phẳng $(SCD)$ có dạng
$$Ax+By+Cz+D=0.$$
Vì $S,C,D\in (SCD)$ nên ta có hệ
}{\begin{tikzpicture}[scale=0.9]
	\def\a{3.5}
	\def\h{\a}
	\path 	(0:0) coordinate (A)
			++(0:\a) coordinate (D)
			++(-130:\a/2) coordinate (C)
			($(A)+(C)-(D)$) coordinate (B)
			($(A)+(90:\h)$) coordinate (S)
			(intersection of A--C and B--D) coordinate (O)
			($(S)!2/3!(O)$) coordinate (G);
	\draw[dashed,thick] 	(B)--(A)--(D)	(A)--(S) (B)--(D) (A)--(C) (S)--(O);
	\draw [-stealth,thick]  (B) -- ($(B)!-1/2!(A)$)node[left, below]{$x$};	
	\draw [-stealth,thick]  (S) -- ($(S)!-1/3!(A)$)node[above]{$z$};
	\draw [-stealth,thick]  (D) -- ($(D)!-1/4!(A)$)node[right]{$y$};
	\draw[thick] 			(B)--(C)--(D) (B)--(S)	(C)--(S)	(D)--(S);
	\foreach \x/\g in {A/180,B/150,C/-45,D/45,S/180,G/180,O/-90}
	\fill[black] 	(\x) circle (1.5pt) ($(\g:3mm)+(\x)$) node {$\x$};
\end{tikzpicture}}
$$\heva{
	& 	Ca		+D	=0\\
&Aa 	+a\sqrt{3}B +D =0\\
 &a\sqrt{3}B +D=0
} \Leftrightarrow \heva{&A=0\\&C=B\sqrt{3}\\
&Ca+D=0.}$$
Vì vậy phương trình mặt phẳng $(SCD)$ là
$$y+\sqrt{3}z-a\sqrt{3}=0.$$
Vậy khoảng cách từ $G$ đến mặt phẳng $\left(SCD\right)$ là
$$\mathrm{d}\left(C,(SBD)\right)=\dfrac{\left|\dfrac{a\sqrt{3}}{3}+ \dfrac{a\sqrt{3}}{3}-a\sqrt{3}\right|}{\sqrt{1^2+\left(\sqrt{3}\right)^2}}=\dfrac{a\sqrt{3}}{6}=0{,}5.$$
}
\end{ex}
 %%%==============HetBai_BT5==============%%%
 
%%%==============Bai_BT6==============%%%
\begin{ex}%[2H5C1-5]
 Cho hình chóp $S.ABCD$ có đáy $ABCD$ là hình vuông tâm $I$, có độ dài đường chéo bằng $a\sqrt{2} $ và $SA$ vuông góc với mặt phẳng $\left(ABCD\right)$. Gọi $\alpha $ là góc giữa hai mặt phẳng $\left(SBD\right)$ và $\left(ABCD\right)$ và $\tan \alpha =\sqrt{2} $. Khoảng cách từ điểm $I$ đến mặt phẳng $\left(SAB\right)$ có dạng $x\cdot a$. Tìm giá trị của $x$.

\shortans{$0{,}5$}
\loigiai{
\immini{ Hình vuông $ABCD$ có độ dài đường chéo bằng $a\sqrt{2} $ suy ra hình vuông đó có cạnh bằng $a$.\\
 Ta có 
 $\heva{&\left(SBD\right)\cap \left(ABCD\right)=BD \\ &{SI\bot BD} \\ &{AI\bot BD} } $\\
 $\Rightarrow {\left(\left(SBD\right); \left(ABCD\right)\right)}={\left(SI; AI\right)}=\widehat{SIA}$
 Ta có $\tan \alpha =\tan \widehat{SIA}=\dfrac{SA}{AI} \Leftrightarrow SA=a$.\\
 Ta xét hệ trục tọa độ $Oxyz$ như hình vẽ với\\ 
 $A\left(0; 0; 0\right)$, $B\left(a; 0; 0\right)$, $C\left(a; a; 0\right)$, $D(0;a;0)$, $S\left(0; 0; a\right)$.\\
 Suy ra $I\left(\dfrac{a}{2} ;\dfrac{a}{2};0 \right)$.\\
 Phương trình mặt phẳng $(SAB)$ là $y=0$.\\
 Vì vậy khoảng cách từ $I$ đến mặt phẳng $\left(SAB\right)$ là $\dfrac{a}{2}=0{,}5a$.}{\begin{tikzpicture}[scale=1]
	\def\a{3.5}
	\def\h{\a}
	\path 	(0:0) coordinate (A)
			++(0:\a) coordinate (D)
			++(-130:\a/2) coordinate (C)
			($(A)+(C)-(D)$) coordinate (B)
			($(A)+(90:\h)$) coordinate (S)
			(intersection of A--C and B--D) coordinate (I);
	\draw[dashed,thick] 	(B)--(A)--(D)	(A)--(S) (B)--(D) (A)--(C) (S)--(I);
	\draw [-stealth,thick]  (B) -- ($(B)!-1/2!(A)$)node[left, below]{$x$};	
	\draw [-stealth,thick]  (S) -- ($(S)!-1/3!(A)$)node[above]{$z$};
	\draw [-stealth,thick]  (D) -- ($(D)!-1/4!(A)$)node[right]{$y$};
	\draw[thick] 			(B)--(C)--(D) (B)--(S)	(C)--(S)	(D)--(S);
	\foreach \x/\g in {A/180,B/150,C/-45,D/45,S/180,I/-90}
	\fill[black] 	(\x) circle (1.5pt) ($(\g:3mm)+(\x)$) node {$\x$};
\end{tikzpicture}}
 }
\end{ex}
 %%%==============HetBai_BT6==============%%%
 
 %%%==============Bai_BT7==============%%%
\begin{ex}%[2H5C1-5]
 Cho hình chóp $S.ABCD$ có đáy $ABCD$ là hình vuông tâm $I$, có độ dài đường chéo bằng $a\sqrt{2} $ và $SA$ vuông góc với mặt phẳng $\left(ABCD\right)$. Gọi $\alpha $ là góc giữa hai mặt phẳng $\left(SBD\right)$ và $\left(ABCD\right)$ và $\tan \alpha =\sqrt{2} $. Tính khoảng cách từ điểm $I$ đến mặt phẳng $\left(SCD\right)$ biết $a=2\sqrt{2}$.

\shortans{$1$}
\loigiai{
\immini{ Hình vuông $ABCD$ có độ dài đường chéo bằng $a\sqrt{2} $ suy ra hình vuông đó có cạnh bằng $a$.\\
 Ta có 
 $\heva{&\left(SBD\right)\cap \left(ABCD\right)=BD \\ &{SI\bot BD} \\ &{AI\bot BD} }$\\
 $ \Rightarrow {\left(\left(SBD\right); \left(ABCD\right)\right)}={\left(SI; AI\right)}=\widehat{SIA}.$
 Ta có $\tan \alpha =\tan \widehat{SIA}=\dfrac{SA}{AI} \Leftrightarrow SA=a$.\\
 Ta có $A\left(0; 0; 0\right)$, $B\left(a; 0; 0\right)$, $C\left(a; a; 0\right)$, $D(0;a;0)$, $S\left(0; 0; a\right)\Rightarrow I\left(\dfrac{a}{2} ;\dfrac{a}{2};0 \right)$.\\
 Phương trình mặt phẳng $(SCD)$ có dạng
$$Ax+By+Cz+D=0.$$}{\begin{tikzpicture}[scale=1]
	\def\a{3.5}
	\def\h{\a}
	\path 	(0:0) coordinate (A)
			++(0:\a) coordinate (D)
			++(-130:\a/2) coordinate (C)
			($(A)+(C)-(D)$) coordinate (B)
			($(A)+(90:\h)$) coordinate (S)
			(intersection of A--C and B--D) coordinate (I);
	\draw[dashed,thick] 	(B)--(A)--(D)	(A)--(S) (B)--(D) (A)--(C) (S)--(I);
	\draw [-stealth,thick]  (B) -- ($(B)!-1/2!(A)$)node[left, below]{$x$};	
	\draw [-stealth,thick]  (S) -- ($(S)!-1/3!(A)$)node[above]{$z$};
	\draw [-stealth,thick]  (D) -- ($(D)!-1/4!(A)$)node[right]{$y$};
	\draw[thick] 			(B)--(C)--(D) (B)--(S)	(C)--(S)	(D)--(S);
	\foreach \x/\g in {A/180,B/150,C/-45,D/45,S/180,I/-90}
	\fill[black] 	(\x) circle (1.5pt) ($(\g:3mm)+(\x)$) node {$\x$};
\end{tikzpicture}}
\noindent Vì $S,C,D\in (SCD)$ nên ta có hệ
$$\heva{
	& 	Ca		+D	=0\\
&aA 	aB+D =0\\
 &aB +D=0
} \Leftrightarrow \heva{&A=0\\&C=B\\
&Ca+D=0.}$$
Vì vậy phương trình mặt phẳng $(SCD)$ là
$$y+z-a=0.$$
Vậy khoảng cách từ $I$ đến mặt phẳng $\left(SCD\right)$ là
$$\mathrm{d}\left(C,(SCD)\right)=\dfrac{\left|\dfrac{a}{2}+0-a\right|}{\sqrt{1^2+1^2}}=\dfrac{a\sqrt{2}}{4}=1.$$
 }
\end{ex}
 %%%==============HetBai_BT7==============%%%
 
%%%==============Bai_BT8==============%%%
\begin{ex}%[2H5C1-5]
Cho hình chóp $S.ABCD$ có đáy $ABCD$ là hình vuông cạnh $a$, mặt bên $SAB$ là tam giác đều và nằm trong mặt phẳng vuông góc với mặt phẳng $\left(ABCD\right)$. Tính khoảng cách từ điểm $A$ đến mặt phẳng $\left(SBD\right)$ biết $a=\sqrt{21}$.

\shortans{$3$}
\loigiai{
\begin{center}
\begin{tikzpicture}
	\def\a{4}
	\def\h{4.5}
	\path 	(0:0) coordinate (A)
			++(0:\a) coordinate (D)
			++(-130:\a/2) coordinate (C)
			($(A)+(C)-(D)$) coordinate (B)
			($(A)!0.5!(B)$) coordinate (H)
			($(C)!0.5!(D)$) coordinate (K)
			($(H)+(90:\h)$) coordinate (S)
			(intersection of A--C and B--D) coordinate (O);%giao điểm O
	\draw[dashed,thick] 	(B)--(A)--(D)	(A)--(S) (S)--(H)	(H)--(K);
	\draw [-stealth,thick]  (B) -- ($(B)!-1/2!(A)$)node[left, below]{$x$};	
	\draw [-stealth,thick]  (S) -- ($(S)!-1/3!(H)$)node[above]{$z$};
	\draw [-stealth,thick]  (K) -- ($(K)!-1/3!(H)$)node[right]{$y$};
	\draw[thick] 			(B)--(C)--(D)
			(B)--(S)	(C)--(S)	(D)--(S);
	\foreach \x/\g in {A/45,B/185,C/-45,D/45,S/180}
			\fill[black] 	(\x) circle (1.5pt)
			($(\g:4mm)+(\x)$) node {$\x$};
	\fill[black] 	(H) circle (1.5pt)
			($(-30:5mm)+(H)$) node {\footnotesize{$H\equiv O$}};
\end{tikzpicture}
\end{center}
 Chọn hệ trục tọa độ $Oxyz$ như hình vẽ. Khi đó
\[S\left(0; 0; \dfrac{a\sqrt{3} }{2} \right); A\left(\dfrac{-a}{2} ;0;0\right); B\left(\dfrac{a}{2} ;0; 0\right);C\left(\dfrac{a}{2} ;a; 0\right); D\left(\dfrac{-a}{2} ;a; 0\right).\] 
Phương trình mặt phẳng $(SBD)$ có dạng
$$Ax+By+Cz+D=0.$$
Vì $S,B,D\in (SBD)$ nên ta có hệ
$$\heva{
	& \dfrac{a\sqrt{3} }{2}C		+D	=0\\
&\dfrac{a}{2} A 	  +D=0\\
&-\dfrac{a}{2} A  +aB +D=0
} \Leftrightarrow \heva{&A=-\dfrac{2}{a}D\\ &B=-\dfrac{2}{a} D\\
&C=-\dfrac{2\sqrt{3}}{3a}D.}$$
Vì vậy phương trình mặt phẳng $(SBD)$ là
$$x+y+\dfrac{\sqrt{3}}{3}z-\dfrac{a}{2}=0.$$
Vậy khoảng cách từ $A$ đến mặt phẳng $\left(SBD\right)$ là
$$\mathrm{d}\left(A,(SBD)\right)=\dfrac{\left|-\dfrac{a}{2}-\dfrac{a}{2}\right|}{\sqrt{1^2+1^2+\left(\dfrac{\sqrt{3}}{3}\right)^2 }}=\dfrac{a\sqrt{21}}{7}=3.$$
}
\end{ex}
 %%%==============HetBai_BT8==============%%%
 
%%%==============Bai_BT9==============%%%
\begin{ex}%[2H5C1-5]  
Cho hình chóp $S.ABCD$ có đáy $ABCD$ là hình vuông cạnh $a$, mặt bên $SAB$ là tam giác đều và nằm trong mặt phẳng vuông góc với mặt phẳng $\left(ABCD\right)$. Gọi $G$ là trọng tâm của tam giác $SAB$ và $M$, $N$ lần lượt là trung điểm của $SC$, $SD$. Tính khoảng cách từ điểm $S$ đến mặt phẳng $\left(GMN\right)$ biết $a=\sqrt{14}$.

\shortans{$2$}
\loigiai{\;
\begin{center}
\begin{tikzpicture}[scale=1]
	\def\a{4}
	\def\h{4.5}
	\path 	(0:0) coordinate (A)
			++(0:\a) coordinate (D)
			++(-130:\a/2) coordinate (C)
			($(A)+(C)-(D)$) coordinate (B)
			($(A)!0.5!(B)$) coordinate (H)
			($(C)!0.5!(D)$) coordinate (K)
			($(H)+(90:\h)$) coordinate (S)
			(intersection of A--C and B--D) coordinate (O)
			($(S)!2/3!(H)$) coordinate (G)
			($(S)!1/2!(D)$) coordinate (N)
			($(S)!1/2!(C)$) coordinate (M);
	\draw[dashed,thick] 	(B)--(A)--(D)	(A)--(S) (S)--(H)	(H)--(K) (G)--(N);
	\draw [-stealth,thick]  (B) -- ($(B)!-1/2!(A)$)node[left, below]{$x$};	
	\draw [-stealth,thick]  (S) -- ($(S)!-1/3!(H)$)node[above]{$z$};
	\draw [-stealth,thick]  (K) -- ($(K)!-1/3!(H)$)node[right]{$y$};
	\draw[thick] 			(B)--(C)--(D)
			(B)--(S)	(C)--(S)	(D)--(S) (G)--(M)--(N);
	\foreach \x/\g in {A/45,B/185,C/-45,D/45,S/180,M/-10,N/0,G/-45}
			\fill[black] 	(\x) circle (1.5pt)
			($(\g:4mm)+(\x)$) node {$\x$};
	\fill[black] 	(H) circle (1.5pt)
			($(-30:5mm)+(H)$) node {\footnotesize{$H\equiv O$}};
\end{tikzpicture}
\end{center}
 Chọn hệ trục tọa độ $Oxyz$ như hình vẽ.\\ 
 Khi đó
$S\left(0; 0; \dfrac{a\sqrt{3} }{2} \right)$, $A\left(\dfrac{-a}{2} ;0;0\right)$, $ B\left(\dfrac{a}{2} ;0; 0\right)$, $C\left(\dfrac{a}{2} ;a; 0\right)$ và $D\left(\dfrac{-a}{2} ;a; 0\right)$.\\
Suy ra $G\left(0; 0; \dfrac{a\sqrt{3} }{6} \right)$, $M\left(\dfrac{a}{4} ;\dfrac{a}{2} ; \dfrac{a\sqrt{3} }{4} \right)$, $N\left(-\dfrac{a}{4} ;\dfrac{a}{2} ; \dfrac{a\sqrt{3} }{4} \right)$.\\
Phương trình mặt phẳng $(GMN)$ có dạng
$$Ax+By+Cz+D=0.$$
Vì $G,M,N\in (GMN)$ nên ta có hệ
$$\heva{
	& 	\dfrac{a\sqrt{3} }{6}C		+D	=0\\
&\dfrac{a}{4} A 	+\dfrac{a}{2}B + \dfrac{a\sqrt{3} }{4}C +D=0\\
&-\dfrac{a}{4} A  +\dfrac{a}{2} B+\dfrac{a\sqrt{3} }{4}C +D=0
} \Leftrightarrow \heva{&A=0\\&B=\dfrac{1}{a} D\\
&C=-\dfrac{2\sqrt{3}}{a} D.}$$
Vì vậy phương trình mặt phẳng $(GMN)$ là
$$y-2\sqrt{3}z+a=0.$$
Vậy khoảng cách từ $S$ đến mặt phẳng $\left(GMN\right)$ là
$$\mathrm{d}\left(S,(GMN)\right)=\dfrac{\left|-2a\right|}{\sqrt{1^2+1^2+\left(-2\sqrt{3}\right)^2 }}=\dfrac{2a\sqrt{14}}{14}=2.$$
}
\end{ex}
\Closesolutionfile{ans}
% \indapan{6}{ans/ans-0-B15-KQ}
 %%%==============HetBai_BT9==============%%%