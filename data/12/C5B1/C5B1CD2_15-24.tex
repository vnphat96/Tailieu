\begin{ex}%[2H5H2-3]%Câu 28
	Trong KG $Oxyz$, cho hai điểm $A(1 ; 0 ; 0), B(4 ; 1 ; 2)$. Mệnh đề nào sau đây đúng hay sai?
	\choiceTF
	{\True $\overrightarrow{AB}=(3 ; 1 ; 2)$}
	{\True Mặt phẳng đi qua $\mathrm{A}$ và vuông góc với $AB$ có phương trình là $3x+y+2z-3=0$}
	{\True Nếu $I$ là trung điểm đoạn thẳng $AB$ thì $I\left(\dfrac{5}{2} ; \dfrac{1}{2} ; 1\right)$}
	{\True Mặt phẳng trung trực đoạn thẳng $AB$ có phương trình là $3x+y+2z-12=0$}
	\loigiai{
		\begin{itemchoice}
			\itemch Đúng.\\Do $A(1 ; 0 ; 0), B(4 ; 1 ; 2)$ nên ta có $\overrightarrow{A B}=(3 ; 1 ; 2)$.
			\itemch Đúng.\\Gọi $(Q)$ là mặt phẳng đi qua $A(1 ; 0 ; 0)$ và vuông góc với $A B$ suy ra mặt phẳng $(Q)$ nhận véc-tơ $\overrightarrow{A B}=(3 ; 1 ; 2)$ làm véc-tơ pháp tuyến.\\ 
			Vậy phương trình mặt phẳng $(Q)$ cần tìm có dạng: $3(x-1)+y+2 z=0 \Leftrightarrow 3 x+y+2 z-3=0$.
			\itemch Đúng.\\$I$ là trung điểm đoạn thẳng $A B$ nên $I\left(\dfrac{5}{2} ; \dfrac{1}{2} ; 1\right)$.
			\itemch Đúng. \\Mặt phẳng trung trực đoạn thẳng $AB$ là mặt phẳng đi qua $\mathrm{I}$ và vuông góc $AB$ nên có phương trình là
			$$3\left(x-\dfrac{5}{2}\right)+y-\dfrac{1}{2}+2(z-2)=0 \\
			\Leftrightarrow 3x+y+2z-12=0..$$
		\end{itemchoice}
	}
\end{ex}
\begin{ex} %[2H5H2-3]%Cau 29
	Trong không gian với hệ trục tọa độ $Oxyz$, cho điểm $M(1 ; 2 ; 3)$. Gọi $A, B, C$ lần lượt là hình chiếu vuông góc của $M$ trên các trục $Ox, Oy, Oz$. Mệnh đề nào sau đây đúng hay sai?
	\choiceTF
	{\True Điểm $A$ có tọa độ là $A(1 ; 0 ; 0)$}
	{Điểm $B$ có tọa độ là $B(1 ; 2 ; 0)$}
	{Phương trình mặt phẳng $(A B C)$ là $\dfrac{x}{1}+\dfrac{y}{2}+\dfrac{z}{3}=0$}
	{\True Phương trình mặt phẳng $(A B C)$ là $\dfrac{x}{1}+\dfrac{y}{2}+\dfrac{z}{3}=1$}
	\loigiai{
		\begin{itemchoice}
			\itemch Đúng.\\Do $A$ là hình chiếu vuông góc của $M$ trên trục $Ox \Rightarrow A(1 ; 0 ; 0)$.
			\itemch Sai.\\Do $B$ là hình chiếu vuông góc của $M$ trên trục $Oy \Rightarrow B(0 ; 2 ; 0)$.
			\itemch Sai.\\$C$ là hình chiếu vuông góc của $M$ trên trục $Oz \Rightarrow C(0 ; 0 ; 3)$.
			\itemch Đúng.\\Vì 3 điểm $A(1;0;0);B(0;2;0);C(0;0;3)$ thuộc $Ox;Oy;Oz$ nên phương trình mặt phẳng $(A B C)$ là $\dfrac{x}{1}+\dfrac{y}{2}+\dfrac{z}{3}=1$.
		\end{itemchoice}
	}
\end{ex}	

\begin{ex}%[2H5H2-3]%Cau 30
	Trong KG $Oxyz$, cho điểm $A(3 ; 5 ; 2)$.  Gọi $A_{1}, A_{2}, A_{3}$ lần lượt là hình chiếu của điểm $A$ lên các mặt phẳng $(Oxy),(Oyz),(Oxz)$. Mệnh đề nào sau đây đúng hay sai?
	\choiceTF
	{\True Điểm $A_{1}$ có tọa độ là $(3 ; 5 ; 0)$}
	{\True Phương trình mặt phẳng đi qua các điểm $A_{1}, A_{2}, A_{3}$ là $10 x+6 y+15z-60=0$}
	{Phương trình mặt phẳng đi qua các điểm $A_{1}, A_{2}, A_{3}$ là $10 x+6 y+15 z-90=0$}
	{Phương trình mặt phẳng đi qua các điểm $A_{1}, A_{2}, A_{3}$ là $\dfrac{x}{3}+\dfrac{y}{5}+\dfrac{z}{2}=1$}
	\loigiai{
		\begin{enumerate}
			\itemch Đúng.\\Vì $A_1$ là hình chiếu của $A$ trên mặt phẳng $(Oxy)$ nên $A_1$ có tọa độ là $(3;5;0)$.
			\itemch Đúng.\\Mặt phẳng đi qua $A_1(3;5;0);A_2(0;5;2),A_3(3;0;2)$ có véc-tơ pháp tuyến được tính từ tích có hướng của hai véc-tơ
			$$\overrightarrow{A_1A_2}=(-3;0;2)$$
			$$\overrightarrow{A_1A_3}=(0;-5;2).$$
			Tích có hướng của hai véc-tơ này là
			$$\overrightarrow{n}=\left[ \overrightarrow{A_1A_2}, \overrightarrow{A_1A_3}\right]=(10;6;15).$$
			Phương trình mặt phẳng là
			$10(x-3)+6(y-5)+15(z-10)=0$\\$\Rightarrow 10x+6y+15-60=0$.
			\itemch Sai.\\Vì phương trình mặt phẳng là $10(x-3)+6(y-5)+15(z-10)=0$\\ $\Rightarrow 10x+6y+15-60=0$.
			\itemch Sai.\\Phương trình mặt phẳng đi qua các điểm $A_{1}, A_{2}, A_{3}$ là $\dfrac{x}{3}+\dfrac{y}{5}+\dfrac{z}{2}=1$\\
			Để kiểm tra phương trình này, ta nhân cả hai vế phương trình $\dfrac{x}{3}+\dfrac{y}{5}+\dfrac{z}{2}=1$ với 30 ta được
			$$10x+6y+15z-30=0 \neq 10x+6y+15-60=0.$$
		\end{enumerate}
	}
\end{ex}
\begin{ex} %[2H5H2-3]%Cau 31
	Trong KG $Oxyz$, cho hai điểm $A(4 ; 0 ; 1)$ và $B(-2 ; 2 ; 3)$. Mệnh đề nào sau đây đúng hay sai?
	\choiceTF
	{\True $\overrightarrow{AB}=(-6 ; 2 ; 2)$}
	{\True Nếu $I$ là trung điểm đoạn thẳng $AB$ thì $I(1 ; 1 ; 2)$}
	{ Mặt phẳng trung trực của đoạn thẳng $AB$ có phương trình là $x+y+2z-6=0$}
	{\True Mặt phẳng trung trực của đoạn thẳng $AB$ có phương trình là $3x-y-z=0$}
	
	\loigiai{
		\begin{enumerate}
			\itemch Đúng.\\Vì $\overrightarrow{AB}=(-6;2;2)$.
			\itemch Đúng.\\Vì tọa độ trung điểm $I=\left(\dfrac{4-2}{2};\dfrac{0+2}{2};\dfrac{1+3}{2}\right)=\left(1;1;2\right)$.
			\itemch Sai.\\
			Mặt phẳng trung trực của đoạn thẳng $AB$ là mặt phẳng đi qua trung điểm $I$ và vuông góc với $\overrightarrow{AB}$.\\
			Phương trình mặt phẳng có dạng
			$$a(x-1)+b(y-1)+c(z-2)=0.$$
			Với $\overrightarrow{n}=(a;b;c)$ là các véc-tơ pháp tuyến của mặt phẳng trung trực.\\
			Vì mặt phẳng trung trực vuông góc với $\overrightarrow{AB}=(-6;2;2)$ nên ta chọn véc-tơ pháp tuyến là $(-6;2;2)$.\\
			Do đó phương trình mặt phẳng là
			$$-6(x-1)+2(y-1)+2(z-2)=0 \Leftrightarrow 3x-y-z=0.$$
			\itemch Đúng.\\Vì phương trình mặt phẳng là
			$-6(x-1)+2(y-1)+2(z-2)=0 \Leftrightarrow 3x-y-z=0$.
		\end{enumerate}
	}
\end{ex}
\begin{ex} %[2H5H2-6]%Cau 32
	Trong không gian hệ tọa độ $Oxyz$, cho $A(1 ; 2 ;-1) ; B(-1 ; 0 ; 1)$ và mặt phẳng $(P)\colon x+2y-z+1=0$. Mệnh đề nào sau đây đúng hay sai?
	\choiceTF
	{\True $\overrightarrow{AB}=(1 ; 1;-1)$}
	{\True Phương trình mặt phẳng $(Q)$ qua $A,B$ và vuông góc với $(P)$ là $x+z=0$}
	{\True Khoảng cách từ điểm $A$ đến mặt phẳng $(P)$ là: $\mathrm{d}(A,(P))=\dfrac{7 \sqrt{6}}{6}$}
	{ Phương trình mặt phẳng $(Q)$ qua $A, B$ và vuông góc với $(P)$ là $3x-y+z=0$}
	\loigiai{
		\begin{enumerate}
			\itemch Đúng.\\Vì $\overrightarrow{AB}=(-2;-2;2)=-\dfrac{1}{2}(-2;-2;2)=(1 ; 1;-1)$.
			\itemch Đúng.\\
			Véc-tơ pháp tuyến của mặt phẳng $(P)$ là $(1;2;-1)$\\
			Mặt phẳng $(Q$ chứa $\overrightarrow{AB}$ và vuông góc với $(P)$ nên véc-tơ pháp tuyến của $(Q)$ là tích có hướng của $\overrightarrow{AB}$ và véc-tơ pháp tuyến của $(P)$
			$$\overrightarrow{n_Q}=\left[ \overrightarrow{AB}, \overrightarrow{n_P}\right] =(-2;0;-2)=(1;0;1).$$
			Vậy phương trình mặt phẳng $(Q)$ qua $A,B$ và vuông góc với $(P)$ là $1(x-1)+0+1(z+1)=x+z=0$.
			\itemch Đúng.\\Khoảng cách từ điểm $A(x_1;y_1;z_1)$ đến mặt phẳng $(P)=ax+by+cz+d=0$ là
			$$\mathrm{d}(A,P)=\dfrac{\left|1\cdot 1+2\cdot 2-(-1)+1\right|}{\sqrt{1^2+2^2+(-1)^2}}=\dfrac{7}{\sqrt{6}}=\dfrac{7\sqrt{6}}{6}.$$
			\itemch Sai.\\Vì phương trình mặt phẳng $(Q)$ qua $A,B$ và vuông góc với $(P)$ là $x+z=0$.
		\end{enumerate}
	}
\end{ex}
\Closesolutionfile{ans}
% \indapan{3}{ans/CD3_17-23DS}

\Opensolutionfile{ans}[ans/CD3-14-25-KQ]
\TNSA

\begin{ex} %[2H5H2-3]%câu 33
	Trong KG $Oxyz$, phương trình tổng quát mặt phẳng $(P)\colon ax+by+cz+d=0$ đi qua điểm $M(3 ;-1 ; 4)$ đồng thời vuông góc với giá của véc-tơ $\overrightarrow{a}=(1 ;-1 ; 2)$. Tính $a+b+c$.
	\shortans{$2$}
	\loigiai{
		Mặt phẳng $(P)$ đi qua điểm $M(3 ;-1 ; 4)$ đồng thời vuông góc với giá của $\overrightarrow{a}=(1 ;-1 ; 2)$ nên nhận $\overrightarrow{a}=(1 ;-1 ; 2)$ làm véc-tơ pháp tuyến. \\Do đó, $(P)$ có phương trình là
		$$1(x-3)-1(y+1)+2(z-4)=0 \Leftrightarrow x-y+2 z-12=0.$$
		Suy ra $a+b+c=2$.
	}
\end{ex}
\begin{ex} %[2H5H2-3]%Câu 34.
	Trong KG $Oxyz$, phương trình mặt phẳng $(P)\colon ax+by+cz+d=0$ qua $M(0 ;-2 ; 1)$ và có cặp véc-tơ chỉ phương $\overrightarrow{a}=(-2 ;-3 ; 8), \overrightarrow{b}=(-1 ; 0 ; 6)$. Tính $a+b+c$.
	\shortans{$17$} 
	\loigiai{
		Ta có $\overrightarrow{n}=\left[\overrightarrow{a}, \overrightarrow{b}\right]=(-18 ; 4 ;-3)$. \\
		Mặt phẳng $(P)$ đi qua $M(0 ;-2 ; 1)$ và có véc-tơ pháp tuyến $\overrightarrow{n}=(-18 ; 4 ;-3)$ nên có phương trình $-18(x-0)+4(y+2)-3(z-1)=0 \Leftrightarrow 18 x-4 y+3 z-11=0$.\\
		Vậy mặt phẳng cần tìm có phương trình: $18x-4y+3z-11=0$.\\
		Suy ra $a+b+c=17$.
	}
\end{ex}
\begin{ex} %[2H5H2-3] %Câu 35
	Trong KG $Oxyz$, cho $A(1 ; 1 ; 0), B(0 ; 2 ; 1), C(1 ; 0 ; 2), D(1 ; 1 ; 1)$. Mặt phẳng $(\alpha)\colon ax+by+cz+d=0$ đi qua $A(1 ; 1 ; 0), B(0 ; 2 ; 1),(\alpha)$ song song với đường thẳng $CD$. Tính $a+b+c$.
	\shortans{$4$}
	\loigiai{
		$\overrightarrow{AB}=(-1 ; 1 ; 1), \overrightarrow{CD}=(0 ; 1 ;-1) \Rightarrow \left[ \overrightarrow{ B}, \overrightarrow{D}\right] =(-2 ;-1 ;-1)$.\\
		$(\alpha)$ đi qua $A(1 ; 1 ; 0)$ và có một VTPT là $\overrightarrow{n}=(2 ; 1 ; 1) \Rightarrow(\alpha)\colon 2 x+y+z-3=0$.\\
		Suy ra $a+b+c=4$.
	}
\end{ex}
\begin{ex} %[2H5H2-3]%Câu 36 
	Trong KG $Oxyz$, cho điểm $M(2 ; 1 ;-3)$ và mặt phẳng $(P)\colon 3 x-2 y+z-3=0$. Phương trình của mặt phẳng đi qua $M$ và song song với $(P)$ có dạng $(Q)\colon ax+by+cz+d=0$. Tính $a+b+c$.
	\shortans{$2$}
	\loigiai{
		Mặt phẳng $(Q)$ cần tìm song song với mặt phẳng $(P)\colon 3 x-2 y+z-3=0$ nên có phương trình dạng
		$$(Q)\colon 3x-2y+z+m=0, m \neq -3.$$
		Vì $M$ $\in(Q)$ nên $(Q)\colon 3\cdot2-2\cdot1+(-3)+m=0 \Leftrightarrow m=-1$.\\
		Vậy $(Q)\colon 3x-2y+z-1=0$.\\
		Suy ra $a+b+c=2$.
	}
\end{ex}
\begin{ex} %[2H5H2-3]%Câu 37.
	Trong KG $Oxyz$, cho ba điểm $A(3 ;-2 ;-2), B(3 ; 2 ; 0), C(0 ; 2 ; 1)$. Phương trình mặt phẳng $(ABC)$ có dạng $=ax+by+cz+d=0$. Tính $a+b+c$.
	\shortans{$5$}
	\loigiai{
		Ta có $\overrightarrow{AB}=(0 ; 4 ; 2), \overrightarrow{AC}=(-3 ; 4 ; 3), \overrightarrow{n}=\left[ \overrightarrow{B} ; \overrightarrow{C}\right]=(4 ;-6 ; 12)$.\\
		Ta có $\overrightarrow{n}=(4 ;-6 ; 12)$ cùng phương $\overrightarrow{n}_{1}=(2 ;-3 ; 6)$.\\
		Mặt phẳng $(ABC)$ đi qua điểm $C(0 ; 2 ; 1)$ và có một véc-tơ pháp tuyến $\overrightarrow{n}_{1}=(2 ;-3 ; 6)$ nên $(ABC)$ có phương trình là
		$$2(x-0)-3(y-2)+6(z-1)=0 \Leftrightarrow 2 x-3 y+6 z=0.$$
		Vậy phương trình mặt phẳng cần tìm là $2x-3y+6z=0$.\\
		Suy ra $a+b+c=5$. 
	}
\end{ex}
\begin{ex}  %[2H5H2-4]%Câu 38
	Trong không gian, cho hai điểm $A(0 ; 0 ; 1)$ và $B(2 ; 1 ; 3)$. Phương trình mặt phẳng đi qua $A$ và vuông góc với $ABC\colon ax+by+cz+d=0$. Tính $a+b+c$.
	\shortans{$5$}
	\loigiai{
		Mặt phẳng đi qua $A(0 ; 0 ; 1)$ và nhận véc-tơ $\overrightarrow{AB}=(2 ; 1 ; 2)$ làm véc-tơ pháp tuyến nên có phương trình là
		$$2(x-0)+(y-0)+2(z-1)=0 \Leftrightarrow 2x+y+2z-2=0.$$
		Suy ra $a+b+c=5$.}
\end{ex}
\begin{ex} %[2H5H2-4]%Câu 39
	Trong KG $Oxyz$, cho hai điểm $A(2 ; 4 ; 1), B(-1 ; 1 ; 3)$ và mặt phẳng $(P)\colon x-3y+2z-5=0$. Lập phương trình mặt phẳng $(Q)$ đi qua hai điểm $A, B$ và vuông góc với mặt phẳng $(P)\colon ax+by+cz+d=0$. Tính $a+b+c$.
	\shortans{$5$}
	\loigiai{
		Ta có: $\overrightarrow{AB}=(-3 ;-3 ; 2)$, véc-tơ pháp tuyến của $(P)$ là $\overrightarrow{n}_{P}=(1 ;-3 ; 2)$.\\
		Từ giả thiết suy ra $\overrightarrow{n}=\left[\overrightarrow{AB}, \overrightarrow{n}_{P}\right]=(0 ; 8 ; 12)$ là véc-tơ pháp tuyến của $(Q)$. \\
		$(Q)$ đi qua điểm $A(2 ; 4 ; 1)$ suy ra phương trình tổng quát của $(Q)$ là
		$$0(x-2)+8(y-4)+12(z-1)=0 \Leftrightarrow 2y+3z-11=0.$$
		Suy ra $a+b+c=5$.
	}
\end{ex}
\begin{ex}%[2H5H2-3]%Câu 40
	Trong KG $Oxyz$, gọi $M, N, P$ lần lượt là hình chiếu vuông góc của $A(2 ;-3 ; 1)$ lên các mặt phẳng tọa độ. Tính $a+b+c$ của phương trình mặt phẳng $(MNP)\colon ax+by+cz+d=0$. 
	\shortans{$7$}
	\loigiai{
		Không mất tính tổng quát, ta giả sử $M, N, P$ lần lượt là hình chiếu vuông góc của $A(2 ;-3 ; 1)$ lên các mặt phẳng tọa độ $(Oxy),(Oxz),(Oyz)$. \\
		Khi đó $M(2 ;-3 ; 0), N(2 ; 0 ; 1)$ và $P(0 ;-3 ; 1).$\\
		$\overrightarrow{MN}=(0 ; 3 ; 1)$ và $\overrightarrow{MP}=(-2 ; 0 ; 1)$. \\
		Ta có $\overrightarrow{MN}$ và $\overrightarrow{MP}$ là cặp véc-tơ không cùng phương và có giá nằm trong $(MNP)$.\\
		Do đó $(MNP)$ có một véc-tơ pháp tuyến là $\overrightarrow{n}=\left[\overrightarrow{M N}, \overrightarrow{MP}\right]=(3 ;-2 ; 6)$.\\
		Mặt khác $(MNP)$ đi qua $M(2 ;-3 ; 0)$ nên có phương trình là
		$$3(x-2)-2(y+3)+6(z-0)=0 \Leftrightarrow 3x-2y+6z-12=0.$$
		Suy ra $a+b+c=7$.
	}
\end{ex}
\Closesolutionfile{ans}
% \indapan{8}{ans/CD3-14-25-KQ}
\begin{dang}{VIẾT PHƯƠNG TRÌNH TỔNG QUÁT MẶT PHẲNG KHI BIẾT MỘT VÉC-TƠ PHÁP TUYẾN HOẶC HAI VÉC-TƠ CHỈ PHƯƠNG MÀ KHÔNG BIẾT ĐIỂM THUỘC MẶT PHẲNG}
\end{dang}
\begin{tomtat}
	Khi viết phương trình tổng quát của mặt phẳng $(\alpha)$ mà có một véc-tơ pháp tuyến $\overrightarrow{n}=(A ; B ; C)$ hoặc có hai véc-tơ chỉ phương $\overrightarrow{a}, \overrightarrow{b}$ (với $\overrightarrow{n}=\left[\overrightarrow{a}, \overrightarrow{b}\right]$) và không tìm được điểm $M_0\left(x_0 ; y_0 ; z_0\right) \in(\alpha)$ thì ta thực hiện các bước sau
	\begin{itemize}
		\item Viết phương trình mặt phẳng $(\alpha)$ dưới dạng
		$$
		Ax+By+Cz+D=0
		.$$
		\item Sau đó dựa vào giả thiết bài toán để tìm giá trị $D$.\\
		Chú ý: Dạng này giả thiết có liên quan đến khoảng cách và góc liên quan đến mặt phẳng.
	\end{itemize}
\end{tomtat}
\Opensolutionfile{ans}[ans/CD3-B46-B49-KQ]
\TN
\begin{ex}%[2H5V2-5]%Câu 41
	Trong KG $Oxyz$, cho mặt phẳng $(P)\colon 2 x+2y-z-1=0$ Mặt phẳng nào sau đây song song với $(P)$ và cách $(P)$ một khoảng bằng $3$?
	\choice
	{$(Q)\colon 2x+2y-z+10=0$}
	{$(Q)\colon 2x+2y-z+4=0$}
	{\True $(Q)\colon 2x+2y-z+8=0$}
	{$(Q)\colon 2x+2y-z-8=0$}
	\loigiai{
		Mặt phẳng $(P)$ đi qua điểm $M(0 ; 0 ;-1)$ và có một véc-tơ pháp tuyến $\overrightarrow{n}=(2 ; 2 ;-1)$.\\
		Mặt phẳng $(Q)$ song song với $(P)$ và cách $(P)$ một khoảng bằng $3$ nên có dạng
		$$(Q)\colon 2x+2y-z+d=0,\quad(d \neq -1).$$
		Mặt khác ta có $\mathrm{d}(M,(Q))=3$ 
		\begin{align*}
			\Leftrightarrow & \dfrac{|1+d|}{\sqrt{4+4+1}}=3\\
			\Leftrightarrow &|d+1|=9\\
			\Leftrightarrow &\hoac{d&=8\\d&=-10} \text{(thỏa mãn)}.
		\end{align*}
		Do đó $(Q)\colon 2x+2y-z+8=0$ hoặc $(Q)\colon 2x+2y-z-10=0$. 
	}
\end{ex}
\begin{ex} %[2H5V2-5]%Câu 42
	Trong KG $Oxyz$, cho ba điểm $A(2 ; 0 ; 0), B(0 ; 3 ; 0), C(0 ; 0 ;-1)$. Phương trình của mặt phẳng $(P)$ qua $D(1 ; 1 ; 1)$ và song song với mặt phẳng $(ABC)$ là
	\choice
	{$2x+3y-6z+1=0$}
	{\True $3x+2y-6z+1=0$}
	{$3x+2y-5z=0$}
	{$6x+2y-3z-5=0$}
	\loigiai{
		Phương trình đoạn chắn của mặt phẳng $(ABC)$ là $\dfrac{x}{2}+\dfrac{y}{3}+\dfrac{z}{-1}=1$.\\
		Mặt phẳng $(P)$ song song với mặt phẳng $(ABC)$ nên\\
		$(P)\colon \dfrac{1}{2} x+\dfrac{1}{3} y-z+m=0\quad(m \neq-1)$.\\
		Do $D(1 ; 1 ; 1) \in(P)$ có $\dfrac{1}{2}\cdot 1+\dfrac{1}{3} \cdot 1-1+m=0 \Leftrightarrow m-\dfrac{1}{6}=0 \Leftrightarrow m=\dfrac{1}{6}$.\\
		Vậy $(P)\colon \dfrac{1}{2}x+\dfrac{1}{3}y-z+\dfrac{1}{6}=0 \Leftrightarrow (P)\colon 3x+2y-6z+1=0$.
	}
\end{ex}
\begin{ex} %[2H5V2-5]%Câu 43.
	Trong KG $Oxyz$ cho $A(2 ; 0 ; 0), B(0 ; 4 ; 0), C(0 ; 0 ; 6), D(2 ; 4 ; 6)$. Gọi $(P)$ là mặt phẳng song song với mặt phẳng $(A B C),(P)$ cách đều $D$ và mặt phẳng $(ABC)$. Phương trình của $(P)$ là
	\choice
	{\True $6x+3y+2z-24=0$}
	{$6x+3y+2z-12=0$}
	{$6x+3y+2z=0$}
	{$6x+3y+2z-36=0$}
	\loigiai{
		$(ABC)\colon \dfrac{x}{2}+\dfrac{y}{4}+\dfrac{z}{6}=1 \Leftrightarrow 6x+3y+2z-12=0$.\\
		$(P)\parallel(ABC) \Rightarrow(P)\colon 6x+3y+2z+m=0\quad(m\neq-12)$.\\
		$(P)$ cách đều $D$ và mặt phẳng $(ABC) \Rightarrow \mathrm{d}(D,(P))=\mathrm{d}(A,(P))$.
		\begin{align*}
			\Leftrightarrow& \dfrac{|6\cdot 2+3\cdot 4+2\cdot 6+m|}{\sqrt{6^{2}+3^{2}+2^{2}}}=\dfrac{|6\cdot 2+3\cdot 0+2\cdot 0+m|}{\sqrt{6^{2}+3^{2}+2^{2}}}\\
			\Leftrightarrow&|36+m|=|12+m|\\ \Leftrightarrow& \hoac{36+m=12+m \\ 36+m=-12-m}\\
			\Leftrightarrow& m=-24 \text{(cách)  (nhận).}
		\end{align*}
		Vậy phương trình của $(P)$ là $6x+3y+2z-24=0$.
	}
\end{ex}
\begin{ex} %[2H5V2-5]%Cau 44 
	Trong không gian với hệ trục tọa độ $Oxyz$, cho mặt phẳng $(Q)\colon x+2y+2z-3=0$, mặt phẳng $(P)$ không qua $O$, song song với mặt phẳng $(Q)$ và $\mathrm{d}((P),(Q))=1$. Phương trình mặt phẳng $(P)$ là
	\choice
	{$x+2y+2z+1=0$}
	{$ x+2y+2z=0$}
	{\True $ x+2y+2z-6=0$}
	{$ x+2y+2z+3=0$}
	\loigiai{
		Vì mặt phẳng $(P)$ song song với mặt phẳng $(Q)$.\\
		$\Rightarrow$ vtpt $\overrightarrow{n}_{P}=$ vtpt $\overrightarrow{n}_{Q}=(1 ; 2 ; 2)$.\\
		Phương trình mặt phẳng $(P)$ có dạng $x+2y+2z+d=0\quad(d \ne 0).$\\
		Gọi $A(3 ; 0 ; 0) \in (Q)$\\
		$\Rightarrow \mathrm{d}((P),(Q))=\mathrm{d}(A,(P))=1$\\
		$\Leftrightarrow \dfrac{|3+D|}{3}=1 \Leftrightarrow\hoac{3+d&=3 \\ 3+d&=-3} \Leftrightarrow\hoac{d&=0 &(\text{loại})&O\\ d&=-6 &(\text{nhận})&}.$
	}
\end{ex}
\begin{ex} %[2H5V2-5]%Cau 45
	Trong KG $Oxyz$, cho mặt phẳng $(P)\colon 2 x-2 y+z-5=0$.  Viết phương trình mặt phẳng $(Q)$ song song với mặt phẳng $(P)$, cách $(P)$ một khoảng bằng $3$ và cắt trục $Ox$ tại điểm có hoành độ dương. 
	\choice
	{$(Q)\colon 2x-2y+z+4=0$}
	{\True $(Q)\colon 2x-2y+z-14=0$}
	{$(Q)\colon 2x-2y+z-19=0$}
	{$(Q)\colon 2x-2y+z-8=0$}
	\loigiai{
		Ta có, $(Q)$ song song $(P)$ nên phương trình mặt phẳng $(Q)\colon 2x-2y+z+d=0$; $d\ne -5$.\\
		Chọn $M(0 ; 0 ; 5)\in(P)$.\\
		Ta có $\mathrm{d}((P),(Q))=\mathrm{d}(M),(Q))=\dfrac{|5+d|}{\sqrt{2^{2}+(-2)^{2}+1^{2}}}=3 \Leftrightarrow\hoac{d&=4 \\ d&=-14.}\\
		\\d=4 \Rightarrow(Q)\colon 2x-2y+z+4=0$ khi đó $(Q)$ cắt $Ox$ tại điểm $M_{1}(-2 ; 0 ; 0)$ có hoành độ âm nên trường hợp này $(Q)$ không thỏa đề bài.\\
		$d=-14 \Rightarrow(Q)\colon 2x-2y+z-14=0$ khi đó $(Q)$ cắt $Ox$ tại điểm $M_{2}(7 ; 0 ; 0)$ có hoành độ dương do đó $(Q)\colon 2x-2y+z-14=0$ thỏa đề bài.\\
		Vậy phương trình mặt phẳng $(Q)\colon 2x-2y+z-14=0$.
	}
\end{ex}
\Closesolutionfile{ans}
% \indapan{10}{ans/CD3-B46-B49-KQ}

\TNSA
\Opensolutionfile{ans}[ans/CD3-14-25-KQ2]
\begin{ex} %[2H5V2-5]%Câu 46
	Trong không gian hệ toạ độ $Oxyz$, lập phương trình các mặt phẳng song song với mặt phẳng $(\beta)\colon x+y-z+3=0$ và cách $(\beta)$ một khoảng bằng $\sqrt{3}$ có dạng $ax+by+cz+d=0\quad (d\neq 0)$. Tính $a+b+c$.
	\shortans{$1$}
	\loigiai{
		Gọi mặt phẳng $(\alpha)$ cần tìm.\\
		Vì $(\alpha)\parallel(\beta)$ nên phương trình $(\alpha)$ có dạng: $x+y-z+c=0$ với $c$ khác $\backslash\{3\}$.\\
		Lấy điểm $I(-1 ;-1 ; 1) \in(\beta)$.\\
		Vì khoảng cách từ $(\alpha)$ đến $(\beta)$ bằng $\sqrt{3}$ nên ta có
		$$\mathrm{d}(I,(\alpha))=\sqrt{3} \Leftrightarrow \dfrac{|-1-1-1+c|}{\sqrt{3}}=\sqrt{3} \Leftrightarrow \dfrac{|c-3|}{\sqrt{3}}=\sqrt{3} \Leftrightarrow\hoac{c=0 \\ c=6}. (\text{thỏa điều kiện } c \in \mathbb{R} \backslash\{3\} ).$$
		Vậy phương trình $(\alpha)\colon x+y-z+6=0$ hoặc $(\alpha)\colon x+y-z=0$.\\
		Suy ra $a+b+c=1$.
	}
\end{ex}
\begin{ex} %[2H5V2-5]%Câu 47.
	Trong không gian với hệ trục tọa độ $Oxyz$, cho hai mặt phẳng $\left(Q_{1}\right)\colon 3x-y+4z+2=0$ và $\left(Q_{2}\right)\colon 3x-y+4z+8=0$. Viết phương trình mặt phẳng $(P)\colon ax+by+cz=0$ song song và cách đều hai mặt phẳng $\left(Q_{1}\right)$ và $\left(Q_{2}\right)$. Tính $a+b+c$.
	\shortans{$6$}
	\loigiai{
		Mặt phẳng $(P)$ có dạng $3x-y+4z+d=0$. \\
		Lấy $M(0 ; 2 ; 0) \in\left(Q_{1}\right)$ và $N(0 ; 8 ; 0) \in\left(Q_{2}\right)$. Do $\left(Q_{1}\right)\parallel\left(Q_{2}\right)$ trung điểm $I(0 ; 5 ; 0)$ của $MN$ phải thuộc vào $(P)$ nên ta tìm được $D=5$. Vậy $(P)\colon 3x-y+4z+5=0$.\\
		Suy ra $a+b+c=6$.
	}
\end{ex}
\begin{ex}  %[2H5V2-5]%Câu 48
	Trong KG $Oxyz$, gọi $(\gamma)$ là mặt phẳng cách đều hai mặt phẳng sau đây:
	$4x-y-2z-3=0$, $4x-y-2z-5=0$. lập mặt phẳng $(\gamma)$ có dạng $ax+by+cz=0$. Tính $a+b+c+d$.
	\shortans{$-3$}
	\loigiai{
		Gọi điểm $A(0;-3; 0) \in (\alpha)\colon4x-y-2z-3=0$ và $B(0 ;-5 ; 0) \in (\beta)\colon4x-y-2z-5=0$.\\
		Mặt phẳng cách đều hai mặt phẳng trên có dạng: $(\gamma)\colon 4x-y-2z+m=0$.\\
		Để mặt phẳng $(\gamma)$ cách đều hai mặt phẳng trên thì
		$$\mathrm{d}(A\colon(\beta))=2 \mathrm{d}(A\colon(\gamma))
		\Leftrightarrow|m+3|=1 \Leftrightarrow\hoac{m=-2 \\ m=-4.}$$ 
		Mặt khác điểm hai điểm $A, B$ phải nằm về hai phía của mặt phẳng $(\gamma)$.\\
		Do đó:
		\begin{itemize}
			\item Với $m=-2$ ta có: $(4\cdot0+3-2\cdot0-2)(4\cdot0+5-2\cdot0-2)>0$ nên $A, B$ cùng phía.
			\item Với $ m=-4$ ta có: $(4\cdot0+3-2\cdot0-4)(4\cdot 0+5-2\cdot 0-4)<0$ nên $A, B$ khác phía.
		\end{itemize}
		Vậy phương trình mặt phẳng cần tìm là $(\gamma)\colon 4x-y-2z-4=0$.\\
		Suy ra $a+b+c+d=-3$.
	}
\end{ex}
\begin{ex} %[2H5V2-5]%Câu 49.
	Trong KG $Oxyz$ cho các điểm $A(2 ; 0 ; 0), B(0 ; 4 ; 0), C(0 ; 0 ; 6), D(2 ; 4 ; 6)$. Gọi $(P)$ là mặt phẳng song song với mặt phẳng $(A B C),(P)$ cách đều $D$ và mặt phẳng $(A B C)$. Viêt phương trình của mặt phẳng $(P)\colon ax+by+cz+d=0$. Tính $a+b+c$.
	\shortans{$11$}
	\loigiai{
		Phương trình mặt phẳng $(ABC)$ là $\dfrac{x}{2}+\dfrac{y}{4}+\dfrac{z}{6}=1 \Leftrightarrow 6x+3y+2z-12=0$
		\begin{itemize}
			\item $(P)$ song song với mặt phẳng $(ABC)$ nên $(P)$ có dạng $$6x+3y+2z+d=0\quad(d \ne q-12).$$
			\item Khoảng cách từ $D$ đến mặt phẳng $(P)$ là
			\begin{align*}
				&\mathrm{d}(D),(P))=\mathrm{d}((ABC),(P))\\
				&\Leftrightarrow \mathrm{d}(D),(P))=\mathrm{d}(A,(P))\\
				&\Leftrightarrow|36+d|=|12+d|\\
				&\Leftrightarrow d=-24.
			\end{align*}
		\end{itemize}
		Vậy $(P)\colon 6x+3y+2z-24=0$.\\
		Suy ra $a+b+c=11$.
	}
\end{ex}
\Closesolutionfile{ans}
% \indapan{4}{ans/CD3-14-25-KQ2}

\begin{dang}{VIẾT PHƯƠNG TRÌNH TỔNG QUÁT MẶT PHẲNG KHI BIẾT ĐIỂM THUỘC MẶT PHẲNG VÀ KHÔNG BIẾT VÉC-TƠ PHÁP TUYẾN HOẶC KHÔNG BIẾT HAI VÉC-TƠ CHỈ PHƯƠNG}
\end{dang}
\begin{tomtat}
	Khi bài toán cho biết mặt phẳng $(\alpha)$ đi qua điềm $M_0\left(x_0 ; y_0 ; z_0\right)$ và giả thiết bài toán không cho véc-tơ pháp tuyến $\overrightarrow{n}$ hoặc không cho hai véc-tơ chỉ phương $\overrightarrow{a}, \overrightarrow{b}$ thì ta thực hiện các bước sau:
	\begin{itemize}
		\item Gọi véc-tơ pháp tuyến của mặt phẳng $(\alpha)$ là $\overrightarrow{n}=(A ; B ; C)$ với $A^2+B^2+C^2 \neq 0$.
		\item Viết phương trình mặt phẳng $(\alpha)$ dưới dạng:
		$$
		(\alpha)\colon A\left(x-x_0\right)+B\left(y-y_0\right)+C\left(z-z_0\right)=0.
		$$
		\item Sau đó dựa vào giả thiết bài toán để tìm hai phương trình chứa 3 ẩn $A, B, C$.
	\end{itemize}
	Chú ý:
	\begin{itemize}
		\item Dạng này, giả thiết có liên quan đến khoảng cách và góc liên quan đến mặt phẳng.
		\item Để giải tìm véc-tơ pháp tuyến của mặt phẳng đơn giàn hơn thì gọi véc-tơ pháp tuyến của mặt phẳng là $\overrightarrow{n}=(1 ; B ; C)$.
	\end{itemize}
\end{tomtat}

\Opensolutionfile{ans}[ans/CD3-50-50]
\TN

\begin{ex} %[2H5V2-5]%Câu 50
	Trong KG $Oxyz$, cho $3$ điểm $A(1 ; 0 ; 0), B(0 ;-2 ; 3), C(1 ; 1 ; 1)$. Gọi $(P)$ là mặt phẳng chứa $A, B$ sao cho khoảng cách từ $C$ tới mặt phẳng $(P)$ bằng $\dfrac{2}{\sqrt{3}}$.  Phương trình mặt phẳng $(P)$ là
	\choice
	{$\hoac{&2x+3y+z-1=0 \\& 3x+y+7z+6=0}$}
	{$\hoac{&x+2y+z-1=0 \\ &-2x+3y+6z+13=0}$}
	{$\hoac{&x+y+2z-1=0 \\& -2x+3y+7z+23=0}$}
	{\True $\hoac{&x+y+z-1=0 \\& -23x+37y+17z+23=0}$} 
	\loigiai{
		Gọi $(P)\colon \heva{&\text{ qua } A(1 ; 0 ; 0)\\ &\text{ VTPT } \overrightarrow{n}=(A ; B ; C) \neq \overrightarrow{0}}$\\
		$(P)\colon A \cdot(x-1)+By+Cz=0$.\\
		$B\in(P)\colon -A-2B+3=0 \Leftrightarrow A=-2B+3C$.\\
		$ \mathrm{d}(C\colon(P))=\dfrac{2}{\sqrt{3}} \Leftrightarrow \dfrac{|B+C|}{\sqrt{A^{2}+B^{2}+C^{2}}}=\dfrac{2}{\sqrt{3}}$\\
		$\Leftrightarrow 3\left(B^{2}+C^{2}+2BC\right)=4\left(A^{2}+B^{2}+C^{2}\right)$\\
		$\Leftrightarrow B^{2}+C^{2}-6BC+4A^{2}=0$.\\
		Thay $A=-2B+3C$ vào $B^{2}+C^{2}-6BC+4A^{2}=0$\\ 
		Ta có: $B^{2}+C^{2}-6BC+4(-2B+3C)^{2}=0 \Leftrightarrow 17B^{2}-54BC+37C^{2}=0$\\
		Cho $C=1$ từ đó suy ra $17 B^{2}-54 B+37=0 \Leftrightarrow\hoac{&B=1& &\Rightarrow& &A=1&\\ &B=\dfrac{37}{17}& &\Rightarrow& &A=\dfrac{-23}{17}.&}$\\
		Suy ra $\hoac{&(P)\colon x+y+x-1=0\\&(P)\colon-23x+37y+17z+23=0.}$
	}
\end{ex}