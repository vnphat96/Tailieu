\textbf{Khoảng cách hai mặt phẳng song song}
\begin{itemize}
\item Khoảng cách giữa mặt phẳng song song là khoảng cách từ một điểm thuộc mặt phẳng này đến mặt phẳng kia (Thực chất là khoảng cách từ một điểm đến mặt phẳng).
\item Để tính khoảng cách mặt phẳng $(\alpha_1)$ song song với $(\alpha_2)$, ta thực hiện như sau:
\begin{enumEX}[\hspace*{1cm}\bf Bước 1:]{1}
	\item Chọn điểm $M\in (\alpha_1)$
	\item Tính khoảng cách điểm $M$ đến $(\alpha_2)$
	\item Kết luận $\mathrm{d}\left((\alpha_1),(\alpha_2)\right)=\mathrm{d}\left(M,(\alpha_2)\right)$
\end{enumEX}
\textbf{Chú ý:} Cho 2 mặt phẳng $(\alpha_1)\colon Ax+By+Cz+D_1=0$ và $(\alpha_2)\colon Ax+By+Cz+D_2=0$ có cùng vectơ pháp tuyến là $\vec{n}=(A;B;C)$.\\
Khi đó khoảng cách giữa hai mặt phẳng đó là: $\mathrm{d}((\alpha_1),(\alpha))=\dfrac{|D_1-D_2|}{\sqrt{A^2+B^2+C^2}}$.
\end{itemize}
\TN
\Opensolutionfile{ans}[ans/ans2C5B1CD1-D2]
%%==========Câu 27
\begin{ex}%[Câu 2]%[2H5N1-5]
	Khoảng cách từ điểm $M\left(3;2;1\right)$ đến mặt phẳng $(P)\colon Ax+Cz+D=0$, $A.C.D\ne 0$. Chọn khẳng định đúng trong các khẳng định sau:
\choice
{\True $\mathrm{d}(M,(P))=\dfrac{\left| 3A+C+D\right|}{\sqrt{A^2+C^2}}$}
{$\mathrm{d}(M,(P))=\dfrac{\left| A+2B+3C+D\right|}{\sqrt{A^2+B^2+C^2}}$}
{$\mathrm{d}(M,(P))=\dfrac{\left| 3A+C\right|}{\sqrt{A^2+C^2}}$}
{$\mathrm{d}(M,(P))=\dfrac{\left| 3A+C+D\right|}{\sqrt{3^2+1^2}}$}
\loigiai{
Áp dung công thức $\mathrm{d}(M_0,(\alpha))=\dfrac{\left |Ax_0+By_0+Cz_0+D\right |}{\sqrt{A^2+B^2+C^2}}$.\\
Ta được: $\mathrm{d}(M,(P))=\dfrac{\left| 3A+C+D\right|}{\sqrt{A^2+C^2}}$.}
\end{ex}

%%==========Câu 28
\begin{ex}%[Câu 3]%[2H5N1-5]
	Trong không gian với hệ tọa độ $Oxyz$, cho mặt phẳng $(P)$ có phương trình: $3x+4y+2z+4=0$ và điểm $A(1;-2;3)$. Tính khoảng cách $\mathrm{d}$ từ $A$ đến $(P)$.
\choice
{$\mathrm{d}=\dfrac{5}{9}$}
{$\mathrm{d}=\dfrac{5}{29}$}
{\True $\mathrm{d}=\dfrac{5}{\sqrt{29}}$}
{$\mathrm{d}=\dfrac{\sqrt{5}}{3}$}
\loigiai{
Khoảng cách $\mathrm{d}$ từ $A$ đến $(P)$ là $$\mathrm{d}(A,(P))=\dfrac{\left| 3x_A+4y_A+2z_A+4\right|}{\sqrt{3^2+4^2+2^2}}=\dfrac{\left| 3-8+6+4\right|}{\sqrt{29}}=\dfrac{5}{\sqrt{29}}.$$}
\end{ex}

%%==========Câu 29
\begin{ex}%[Câu 4]%[2H5N1-5]
	Trong không gian $Oxyz$, cho mặt phẳng $(P)\colon 2x-2y+z-1=0$. Khoảng cách từ điểm $M\left(-1;2;0\right)$ đến mặt phẳng $(P)$ bằng
\choice
{$5$}
{$2$}
{\True $\dfrac{5}{3}$}
{$\dfrac{4}{3}$}
\loigiai{
Ta có: $\mathrm{d}\left(M,(P)\right)=\dfrac{\left| 2\cdot\left(-1\right)-2\cdot2+0-1\right|}{\sqrt{2^2+\left(-2\right)^2+1^2}}=\dfrac{5}{3}$.}
\end{ex}

%%==========Câu 30
\begin{ex}%[Câu 5]%[2H5N1-5]
	Trong không gian $Oxyz$, tính khoảng cách từ $M\left(1;2;-3\right)$ đến mặt phẳng $(P)\colon x+2y+2z-10=0$.
\choice
{\True $\dfrac{11}{3}$}
{$3$}
{$\dfrac{7}{3}$}
{$\dfrac{4}{3}$}
\loigiai{
Ta có: $\mathrm{d}\left(M;(P)\right)=\dfrac{\left| 1+2\cdot 2+2\cdot\left(-3\right)-10\right|}{\sqrt{1^2+2^2+2^2}}=\dfrac{\left| -11\right|}{3}=\dfrac{11}{3}$.}
\end{ex}

%%==========Câu 31
\begin{ex}%[Câu 6]%[2H5H1-5]
	Trong không gian $Oxyz$, cho mặt phẳng $(P)\colon 2x-y+2z-4=0$. Gọi $H$ là hình chiếu vuông góc của điểm $M\left(3;1;-2\right)$ lên mặt phẳng $(P)$. Độ dài đoạn thẳng $MH$ là
\choice
{$2$}
{$\dfrac{1}{3}$}
{\True $1$}
{$3$}
\loigiai{
Độ dài đoạn thẳng $MH$ là $MH=\mathrm{d}\left(M,(P)\right)=\dfrac{\left| 2\cdot 3-1+2\cdot (-2)-4\right|}{\sqrt{2^2+(-1)^2+2^2}}=1$.}
\end{ex}

%%==========Câu 32
\begin{ex}%[Câu 7]%[2H5H1-5]
	Trong không gian với hệ trục tọa độ $Oxyz$, gọi $H$ là hình chiếu vuông góc của điểm $A(1;-2;3)$ lên mặt phẳng $(P)\colon 2x-y-2z+5=0$. Độ dài đoạn thẳng $AH$ bằng
\choice
{$3$}
{$7$}
{$4$}
{$1$}
\loigiai{
Độ dài đoạn thẳng $AH$ là $AH=\mathrm{d}\left(A,(P)\right)=\dfrac{\left| 2+2-6+5\right|}{\sqrt{2^2+(-1)^2+(-2)^2}}=1$.}
\end{ex}

%%==========Câu 33
\begin{ex}%[Câu 8]%[2H5H1-5]
	Khoảng cách từ điểm $M(-4;-5;6)$ đến mặt phẳng $(Oxy)$, $(Oyz)$ lần lượt bằng
\choice
{\True $6$ và $4$}
{$6$ và $5$}
{$5$ và $4$}
{$4$ và $6$}
\loigiai{
Ta có: $\mathrm{d}\left(M,(Oxy)\right)=\left|z_M\right|=6$ và $\mathrm{d}(M,(Oyz))=\left|x_M\right|=4$.}
\end{ex}

%%==========Câu 34
\begin{ex}%[Câu 9]%[2H5H1-5]
	Tính khoảng cách $\mathrm{d}$ từ điểm $B\left(x_0;y_0;z_0\right)$ đến mặt phẳng $(P)\colon y + 1 = 0$ ta được:
\choice
{$y_0$}
{$\left| y_0\right|$}
{$\dfrac{\left| y_0+1\right|}{\sqrt{2}}$}
{\True $\left| y_0+1\right|$}
\loigiai{
	Ta có: $\mathrm{d}\left (M,(P)\right )=\dfrac{\left |y_0+1\right |}{\sqrt{1^2}}=\left| y_0+1\right|$.
}
\end{ex}

%%==========Câu 35
\begin{ex}%[Câu 10]%[2H5H1-5]
	Khoảng cách từ điểm $C(-2;0;0)$ đến mặt phẳng $(Oxy)$ bằng
\choice
{\True $0$}
{$2$}
{$1$}
{$\sqrt{2}$}
\loigiai{
Điểm $C$ thuộc mặt phẳng $(Oxy)$ nên $\mathrm{d}\left(C,(Oxy)\right)=0$.}
\end{ex}

%%==========Câu 36
\begin{ex}%[Câu 11]%[2H5H1-5]
	Trong không gian $Oxyz$, khoảng cách giữa hai mặt phẳng $(P)\colon x+2y+2z-10=0$ và $(Q)\colon x+2y+2z-3=0$ bằng
\choice
{$\dfrac{4}{3}$}
{$\dfrac{8}{3}$}
{\True $\dfrac{7}{3}$}
{$3$}
\loigiai{
Ta có $\dfrac{1}{1}=\dfrac{2}{2}=\dfrac{2}{2}\ne \dfrac{-10}{-3}$ nên $(P)\parallel (Q)$.\\
Lấy $A\left(2;1;3\right)\in \left(P\right)$. 
Ta có: $\mathrm{d}\left(\left(P\right),\left(Q\right)\right)=\mathrm{d}\left(A,\left(Q\right)\right)=\dfrac{\left| 2+2\cdot 1+2\cdot3-3\right|}{\sqrt{1^2+2^2+2^2}}=\dfrac{7}{3}$.}
\end{ex}

%%==========Câu 37
\begin{ex}%[Câu 12]%[2H5H1-5]
	Trong không gian $Oxyz$, khoảng cách giữa hai mặt phẳng $(P)\colon x+2y+3z-1=0$ và $(Q)\colon x+2y+3z+6=0$ là
\choice
{\True $\dfrac{7}{\sqrt{14}}$}
{$\dfrac{8}{\sqrt{14}}$}
{$14$}
{$\dfrac{5}{\sqrt{14}}$}
\loigiai{
Ta có $\dfrac{1}{1}=\dfrac{2}{2}=\dfrac{3}{3}\ne \dfrac{-1}{6}$ nên $(P)\parallel (Q)$.\\
Khi đó: $\mathrm{d}\left((P);(Q)\right)$ =$\dfrac{\left| D_2-D_1\right|}{\sqrt{A^2+B^2+C^2}}
=\dfrac{\left| -1-6\right|}{\sqrt{1^2+2^2+3^2}}=\dfrac{7}{\sqrt{14}}$.}
\end{ex}

%%==========Câu 38
\begin{ex}%[Câu 13]%[2H5H1-5]
	Trong không gian $Oxyz$, khoảng cách giữa hai mặt phẳng $(P)\colon x+2y+2z-8=0$ và $(Q)\colon x+2y+2z-4=0$ bằng
\choice
{$1$}
{\True $\dfrac{4}{3}$}
{$2$}
{$\dfrac{7}{3}$}
\loigiai{
Ta có $\dfrac{1}{1}=\dfrac{2}{2}=\dfrac{2}{2}\ne \dfrac{-8}{-4}$ nên $(P)\parallel (Q)$.\\
Khi đó: $\mathrm{d}\left((P);(Q)\right)=\dfrac{\left| -8-(-4)\right|}{\sqrt{1^2+2^2+2^2}}=\dfrac{4}{3}$.\\
}
\end{ex}

%%==========Câu 39
\begin{ex}%[Câu 14]%[2H5H1-4]
	Trong không gian $Oxyz$, mặt phẳng $(P)\colon 2x+y+z-2=0$ vuông góc với mặt phẳng nào dưới đây?
\choice
{$2x-y-z-2=0$}
{\True $x-y-z-2=0$}
{$x+y+z-2=0$}
{$2x+y+z-2=0$}
\loigiai{
Mặt phẳng $(P)$ có một vectơ pháp tuyến $\overrightarrow{n_P}=\left(2;1;1\right)$.\\
Mặt phẳng $(Q)\colon x-y-z-2=0$ có một vectơ pháp tuyến $\overrightarrow{n_Q}=\left(1;-1;-1\right)$.\\
Mà $\overrightarrow{n_P}\cdot\overrightarrow{n_Q}=2-1-1=0\Rightarrow \overrightarrow{n_P}\perp \overrightarrow{n_Q}\Rightarrow (P)\perp (Q)$.\\
Vậy mặt phẳng $(Q)\colon x-y-z-2=0$ là mặt phẳng cần tìm.}
\end{ex}

%%==========Câu 40
\begin{ex}%[Câu 15]%[2H5H1-4]
	Trong không gian với hệ tọa độ $Oxyz$, cho hai mặt phẳng $(P)\colon 2x+my+3z-5=0$ và $(Q)\colon nx-8y-6z+2=0$, với $m,n\in \mathbb{R}$. Xác định $m,n$ để $(P)$ song song với $(Q)$.
\choice
{$m=n=-4$}
{\True $m=4;n=-4$}
{$m=- 4;n=4$}
{$m=n=4$}
\loigiai{
Mặt phẳng $(P)$ có véc tơ pháp tuyến $\vec{n_1}=(2;m;3)$.\\
Mặt phẳng $(Q)$ có véc tơ pháp tuyến $\vec{n_2}=(n;-8;-6)$.\\
Mặt phẳng $(P)\parallel (Q)\Rightarrow \vec{n_1}=k\cdot \vec{n_2}\, (k\in \mathbb{R})\Leftrightarrow \heva{&2=kn \\&m=- 8k \\&3=- 6k}\Leftrightarrow \heva{&k=-\dfrac{1}{2} \\&m=4 \\&n=- 4.}$}
\end{ex}

%%==========Câu 41
\begin{ex}%[Câu 16]%[2H5H1-4]
	Trong không gian $Oxyz$, cho hai mặt phẳng $(P)\colon x-2y+2z-3=0$ và $(Q)\colon mx+y-2z+1=0$. Với giá trị nào của $m$ thì hai mặt phẳng đó vuông góc với nhau?
\choice
{$m=1$}
{$m=-1$}
{$m=-6$}
{\True $m=6$}
\loigiai{
Ta có: $(P)\perp (Q)\Leftrightarrow 1\cdot m-2\cdot 1+2\cdot (-2)=0\Leftrightarrow m=6$.}
\end{ex}

%%==========Câu 42
\begin{ex}%[Câu 17]%[2H5V1-4]
	Trong không gian $Oxyz$, cho ba mặt phẳng $(P)\colon x+y+z-1=0$, $(Q)\colon 2x+my+2z+3=0$ và $(R)\colon -x+2y+nz=0$. Tính tổng $m+2n$, biết rằng $(P)\perp (R)$ và $(P)\parallel (Q)$.
\choice
{$-6$}
{$1$}
{\True $0$}
{$6$}
\loigiai{
$(P)$ có vectơ pháp tuyến $\vec{a}=(1;1;1)$.\\
$(Q)$ có vectơ pháp tuyến $\vec{b}=(2;m;2)$.\\
$(R)$ có vectơ pháp tuyến $\vec{c}=(-1;2;n)$.\\
Ta có: $(P)\perp (R)\Leftrightarrow \vec{a}\cdot \vec{c}=0\Leftrightarrow n=-1$.\\
$(P)\parallel (Q)\Leftrightarrow \dfrac{2}{1}=\dfrac{m}{1}=\dfrac{2}{1}\Leftrightarrow m=2$.\\
Vậy $m+2n=2+2\left(-1\right)=0$}
\end{ex}

%%==========Câu 43
\begin{ex}%[Câu 18]%[2H5V1-4]
	Trong không gian $Oxyz$, cho $(P)\colon x+y-2z+5=0$ và $(Q)\colon 4x+(2-m)y+mz-3=0$, $m$ là tham số thực. Tìm tham số $m$ sao cho mặt phẳng $(Q)$ vuông góc với mặt phẳng $(P)$.
\choice
{$m=-3$}
{$m=-2$}
{$m=3$}
{\True $m=2$}
\loigiai{
Mặt phẳng $(P)$ có véctơ pháp tuyến là $\vec{n_{P}}=(1;1;-2)$.\\
Mặt phẳng $(Q)$ có véctơ pháp tuyến là $\vec{n_{Q}}=(4;2-m;m)$.\\
Ta có $(P)\perp (Q)\Leftrightarrow \vec{n_{P}}\perp \vec{n_{Q}}\Leftrightarrow \vec{n_{P}}\cdot \vec{n_{Q}}=0\Leftrightarrow 4\cdot 1+2-m-2m=0\Leftrightarrow m=2$.
}
\end{ex}

%%==========Câu 44
\begin{ex}%[Câu 19]%[2H5V1-4]
	Trong không gian $Oxyz$ cho hai mặt phẳng $(\alpha)\colon x+2y-z-1=0$ và $(\beta)\colon 2x+4y-mz-2=0$. Tìm $m$ để hai mặt phẳng $(\alpha)$ và $(\beta)$ song song với nhau.
\choice
{$m=1$}
{\True Không tồn tại $m$}
{$m=-2$}
{$m=2$}
\loigiai{
Ta có vectơ pháp tuyến của $(\alpha)$ là $\overrightarrow{n_1}=(1;2;-1)$, vectơ pháp tuyến của $(\beta)$ là $\overrightarrow{n_2}=(2;4;-m)$.\\
Hai mặt phẳng $(\alpha)$ và $(\beta)$ song song khi $\dfrac{2}{1}=\dfrac{4}{2}=\dfrac{-m}{-1}\ne \dfrac{-2}{-1}$.\\
Vậy không có giá trị nào của $m$ thỏa mãn điều kiện trên.}
\end{ex}

%%==========Câu 45
\begin{ex}%[Câu 20]%[2H5V1-4]
	Trong không gian toạ độ $Oxyz$, cho mặt phẳng $(P)\colon x+2y-2z-1=0$, mặt phẳng nào dưới đây song song với $(P)$ và cách $(P)$ một khoảng bằng $3$.
\choice
{\True $(Q)\colon x+2y-2z+8=0$}
{$(Q)\colon x+2y-2z+5=0$}
{$(Q)\colon x+2y-2z+1=0$}
{$(Q)\colon x+2y-2z+2=0$}
\loigiai{
+ Chọn $A\left(1;0;0\right)\in (P)$.\\
+ Xét đáp án \textbf{A.}, ta có $\mathrm{d}\left(A;(Q)\right)=\dfrac{\left| 1+8\right|}{\sqrt{1^2+2^2+\left(-2\right)^2}}=3$.
}
\end{ex}
\Closesolutionfile{ans}
\indapan{10}{ans/ans2C5B1CD1-D2}
\TNTF
\Opensolutionfile{ans}[ans/ans2C5B1CD1-D2-DS]
%%==========Câu 46
\begin{ex}%[Câu 21]%[2H5N1-5]
	Trong không gian toạ độ $Oxyz$, cho điểm $M\left(1;2;0\right)$ và các mặt phẳng $(Oxy)$, $(Oyz)$, $(Oxz)$. Các mệnh đề sau đây đúng hay \textbf{sai}?
\choiceTF
{\True $\mathrm{d}\left(M,(Oxz)\right)=2$}
{\True $\mathrm{d}\left(M,(Oyz)\right)=1$}
{$\mathrm{d}\left(M,(Oxy)\right)=1$}
{\True $\mathrm{d}\left(M,(Oxz)\right)>d\left(M,(Oyz)\right)$}
\loigiai{
\begin{itemchoice}
\itemch $\mathrm{d}\left(M,(Oxz)\right)=|2|=2$.	ĐÚNG
\itemch $\mathrm{d}\left(M,(Oyz)\right)=|1|=1$. ĐÚNG
\itemch $\mathrm{d}\left(M,(Oxy)\right)=|0|=0$.	SAI
\itemch $\mathrm{d}\left(M,(Oxz)\right)>d\left(M,(Oyz)\right)$. ĐÚNG
\end{itemchoice}
}
\end{ex}
%%==========Câu 48
\begin{ex}%[Câu 23]%[2H5H1-4]
	Trong không gian $Oxyz$, cho hai mặt phẳng $(P)\colon x+2y-2z-6=0$ và $(Q)\colon x+2y-2z+3=0$. Các mệnh đề sau đây đúng hay \textbf{sai}?
\choiceTF
{\True Hai mặt phẳng $(P)$ và $(Q)$ song song với nhau}
{Hai mặt phẳng $(P)$ và $(Q)$ vuông góc với nhau}
{Khoảng cách giữa hai mặt phẳng $(P)$ và $(Q)$ bằng $2$}
{\True Khoảng cách giữa hai mặt phẳng $(P)$ và $(Q)$ bằng $3$}
\loigiai{
\begin{itemize}
\item Ta có: $\dfrac{1}{1}=\dfrac{2}{2}=\dfrac{-2}{-2}\ne \dfrac{-6}{3}$ nên $(P)\parallel (Q)$.
\item $\mathrm{d}\left ((P),(Q)\right )=\dfrac{|-6-3|}{\sqrt{1^2+2^2+(-2)^2}}=3$.
\end{itemize}
\begin{itemchoice}
\itemch Hai mặt phẳng $(P)$ và $(Q)$ song song với nhau. ĐÚNG
\itemch Hai mặt phẳng $(P)$ và $(Q)$ vuông góc với nhau. SAI
\itemch Khoảng cách giữa hai mặt phẳng $(P)$ và $(Q)$ bằng $2$. SAI
\itemch Khoảng cách giữa hai mặt phẳng $(P)$ và $(Q)$ bằng $3$. ĐÚNG
\end{itemchoice}
}
\end{ex}

%%==========Câu 47
\begin{ex}%[2H5N1-5]%[2H5H1-5]
	Trong không gian toạ độ $Oxyz$, Biết khoảng cách từ điểm $O$ đến mặt phẳng $(Q)$ bằng 1. Các mệnh đề sau đây đúng hay \textbf{sai}?
\choiceTF
{Mặt phẳng $(Q)$ có phương trình là $x + y + z-3 = 0$}
{\True Mặt phẳng $(Q)$ có phương trình là $2x + y + 2z-3 = 0$}
{Mặt phẳng $(Q)$ có phương trình là $2x + y- 2z + 6 = 0$}
{\True Mặt phẳng $(Q)$ có phương trình là $x + 2y + 2z-3= 0$}
\loigiai{
	\begin{itemchoice}
	\itemch {Ta có $\mathrm{d}(O,(Q))=\dfrac{|-3|}{\sqrt{1^2+1^2+1^2}}=\sqrt{3}\ne 1$. SAI}
	\itemch {Ta có $\mathrm{d}(O,(Q))=\dfrac{|-3|}{\sqrt{2^2+1^2+2^2}}= 1$. ĐÚNG}
	\itemch {Ta có $\mathrm{d}(O,(Q))=\dfrac{|6|}{\sqrt{2^2+1^2+(-2)^2}}=2\ne 1$. SAI}
	\itemch {Ta có $\mathrm{d}(O,(Q))=\dfrac{|-3|}{\sqrt{1^2+2^1+2^2}}=1$. ĐÚNG
		}
	\end{itemchoice}
	}
\end{ex}

%%==========Câu 49
\begin{ex}%[Câu 24]%[2H5H1-4]
	Trong không gian $Oxyz$, cho điểm $N(0;1;0)$ và hai mặt phẳng $(P)\colon 2x-y-2z-9=0$, $(Q)\colon 4x-2y-4z-6=0$. Các mệnh đề sau đây đúng hay \textbf{sai}?
\choiceTF
{\True Hai mặt phẳng $(P)$ và $(Q)$ song song với nhau}
{Khoảng cách từ điểm $N$ đến mặt phẳng $(Q)$ bằng $\dfrac{1}{2}$}
{\True Khoảng cách giữa hai mặt phẳng $(P)$ và $(Q)$ bằng $2$}
{Khoảng cách giữa hai mặt phẳng $(P)$ và $(Q)$ bằng $3$}
\loigiai{
\begin{itemize}
\item Ta có $\dfrac{2}{4}=\dfrac{-1}{-2}=\dfrac{-2}{-4}\ne \dfrac{-9}{-6}$ nên $(P)\parallel (Q)$.	
\item $\mathrm{d}\left(N,\left(Q\right)\right)=\dfrac{\left| -2\cdot 1-6\right|}{\sqrt{4^2+\left(-2\right)^2+\left(-4\right)^2}}=\dfrac{4}{3}$.
\item $\mathrm{d}\left((P),(Q)\right)=\dfrac{|-9-(-3)|}{\sqrt{2^2+(-1)^2+(-2)^2}}=2$.
\end{itemize}
\begin{itemchoice}
\itemch Hai mặt phẳng $(P)$ và $(Q)$ song song với nhau.	ĐÚNG
\itemch Khoảng cách điểm đến mặt phẳng $(Q)$ bằng $\dfrac{1}{2}$.	SAI
\itemch Khoảng cách giữa hai mặt phẳng $(P)$ và $(Q)$ bằng $2$. ĐÚNG
\itemch Khoảng cách giữa hai mặt phẳng $(P)$ và $(Q)$ bằng $3$. SAI
\end{itemchoice}
}
\end{ex}

%%==========Câu 50
\begin{ex}%[Câu 25]%[2H5H1-5]
	Khoảng cách từ điểm $A(2;4;3)$ đến mặt phẳng $(\alpha)\colon 2x+y+2z+1=0$ và $(\beta)\colon x=0$ lần lượt là $\mathrm{d}(A,(\alpha))$, $\mathrm{d}(A,(\beta))$. Các mệnh đề sau đây đúng hay \textbf{sai}?
\choiceTF
{$\mathrm{d}\left(A,(\alpha)\right)=3\cdot \mathrm{d}\left(A,(\beta)\right)$}
{$\mathrm{d}\left(A,(\alpha)\right)>\mathrm{d}\left(A,(\beta)\right)$}
{$\mathrm{d}\left(A,(\alpha)\right)=\mathrm{d}\left(A,(\beta)\right)$}
{\True $2\cdot\mathrm{d}\left(A,(\alpha)\right) = \mathrm{d}\left(A,(\beta)\right)$}
\loigiai{
Ta có: $\mathrm{d}\left(A,(\alpha)\right)=\dfrac{\left| 2.x_A+y_A+2.z_A+1\right|}{\sqrt{2^2+1^2+2^2}}=1$ và $\mathrm{d}\left(A,(\beta)\right)=\dfrac{\left|x_A\right|}{\sqrt{1^2}}=2$.\\
Kết luận: $\mathrm{d}\left(A,(\beta)\right)=2\cdot \mathrm{d}\left(A,(\alpha)\right)$.
\begin{itemchoice}
\itemch $\mathrm{d}\left(A,(\alpha)\right)=3\cdot \mathrm{d}\left(A,(\beta)\right)$. SAI
\itemch $\mathrm{d}\left(A,(\alpha)\right)>\mathrm{d}\left(A,(\beta)\right)$. SAI
\itemch $\mathrm{d}\left(A,(\alpha)\right) =\mathrm{d}\left(A,(\beta)\right)$. SAI
\itemch $2\cdot \mathrm{d}\left(A,(\alpha)\right)=\mathrm{d}\left(A,(\beta)\right)$. ĐÚNG
\end{itemchoice}
}
\end{ex}

%%==========Câu 51
\begin{ex}%Câu 51.%[2H5H1-4]
	Trong không gian $Oxyz$, cho điểm $I(2; 6;-3)$ và các mặt phẳng: $(\alpha)\colon x-2=0$; $(\beta)\colon y-6=0$; $(\gamma): z-3=0$. Các mệnh đề sau đây đúng hay sai?
	\choiceTF
	{\True $(\alpha) \perp(\beta)$}
	{$(\beta) \parallel (Oyz)$}
	{$(\gamma) \parallel Oz$}
	{\True $(\alpha)$ qua $I$}
	\loigiai{
	Ta có:
	\begin{itemize}
	\item $(\alpha): x-2=0$ có véctơ pháp tuyến $\vec{a}=(1 ; 0 ; 0)$.
	\item $(\beta): y-6=0$ có véctơ pháp tuyến $\vec{b}=(0 ; 1 ; 0)$.
	\item $(\gamma): z+3=0$ có véctơ pháp tuyến $\vec{c}=(0 ; 0 ; 1)$.
	\end{itemize}
\begin{itemchoice}
\itemch đúng vì ta có $\vec{a} \cdot \vec{b}=1\cdot 0+0\cdot 1+0=0 \Rightarrow(\alpha) \perp(\beta)$.
\itemch sai vì $(Oyz)$ có véctơ pháp tuyến $\vec{i}=(1 ; 0 ; 0)$ không cùng phương với $\vec{b}=(0 ; 1 ; 0)$ nên $(\beta)$ không song song với mặt phẳng $(Oyz)$. 
\itemch sai vì trục $Oz$ có vectơ chỉ phương $\vec{k}=(0 ; 0 ; 1)=\vec{c}$ nên $(\gamma) \perp Oz$.
\itemch đúng vì thay tọa độ điểm $I$ vào $(\alpha)$ ta thấy thỏa thỏa mãn nên $I \in(\alpha)$.	
\end{itemchoice}
}
\end{ex}

%%==========Câu 52
\begin{ex}%Câu 52.%[2H5H1-4]
	Trong không gian $Oxyz$, cho hai mặt phẳng $(P)\colon y-9=0$. Xét các mệnh đề sau:
	\begin{multicols}{2}
	\item \hspace*{1cm}(I) $(P) \parallel (Oxz)$.
	\item (II) $(P) \perp Oy$
	\end{multicols}
	\choiceTF[t]
	{Cả (I) và (II) đều sai}
	{(I) đúng, (II) sai}
	{(I) sai, (II) đúng}
	{\True Cả (I) và (II) đều đúng}
	\loigiai{
	Ta có: mặt phẳng $(Oxz)$ có véctơ pháp tuyến $\vec{j}=(0 ; 1 ; 0)$.\\
	Mặt phẳng $(P)$ có véctơ pháp tuyến là $\vec{a}=(0;1;1)=\vec{j}$ nên $(P)\parallel (Oxz)$.\\
	Trục $Oz$ có vectơ chỉ phương là $\vec{j}=(0;1;0)$ nên $(P)\perp Oy$.
\begin{itemchoice}
\itemch Cả (I) và (II) đều sai. SAI
\itemch (I) đúng, (II) sai. SAI
\itemch (I) sai, (II) đúng. SAI
\itemch Cả (I) và (II) đều đúng. ĐÚNG
\end{itemchoice}
}
\end{ex}

%%==========Câu 53
\begin{ex}%Câu 53.%[2H5H1-4]
	Trong không gian $Oxyz$, Cho ba mặt phẳng $(\alpha)\colon x+y+2z+1=0;(\beta)\colon x+y-z+2=0$; $(\gamma)\colon x-y+5=0$. Các mệnh đề sau đây đúng hay sai?
	\choiceTF[t]
	{$(\alpha) \parallel (\gamma)$}
	{\True $(\alpha) \perp(\beta)$}
	{\True $(\gamma) \perp(\beta)$}
	{\True $(\alpha) \perp(\gamma)$}
\loigiai{ Ta có:
	\begin{itemize}
	\item Mặt phẳng $(\alpha)$ có véctơ pháp tuyến là $\vec{a}=(1;1;2)$.
	\item Mặt phẳng $(\beta)$ có có véctơ pháp tuyến là $\vec{b}=(1;1;-1)$.
	\item Mặt phẳng $(\gamma)$ có có véctơ pháp tuyến là $\vec{c}=(1;-1;0)$.
	\item $\left [\vec{a},\vec{c}\right ]=(2;2;-2)\ne \vec{0}$ nên $(\alpha)$ và $(\gamma)$ không song song nhau.
	\item $\vec{a} \cdot \vec{b}=0 \Rightarrow(\alpha) \perp(\beta)$.
	\item $\vec{a} \cdot \vec{c}=0 \Rightarrow(\alpha) \perp(\gamma)$.
	\item $\vec{b} \cdot \vec{c}=0 \Rightarrow(\beta) \perp(\gamma)$.
	\end{itemize}
	\begin{itemchoice}
	\itemch $(\alpha)\parallel(\gamma)$. SAI
	\itemch $(\alpha) \perp(\beta)$. ĐÚNG
	\itemch $(\gamma) \perp(\beta)$. ĐÚNG
	\itemch $(\alpha) \perp(\gamma)$. ĐÚNG
	\end{itemchoice}
	}
\end{ex}

\TNSA

%%==========Câu 54
\begin{ex}%Câu 54.%[2H5H1-5]
	Trong không gian $Oxyz$, cho điểm $M(-1; 2-3)$ và mặt phẳng $(P)\colon 2 x-2 y+z+5=0$. Tính khoảng cách từ điểm $M$ đến mặt phẳng $(P)$ (kết quả viết dưới dạng số thập phân, lấy gần đúng đến hàng phần mười).
	\shortans[0]{$1{,}3$}
	\loigiai{
Khoảng cách từ điểm $M$ đến mặt phẳng $(P)$ là 
$$\mathrm{d}\left(M,(P)\right)=\dfrac{\left| 2\cdot (-1)-2\cdot 2+1\cdot (-3)+5\right|}{\sqrt{2^2+(-2)^2+1^2}}=\dfrac{4}{3}.$$
}
\end{ex}

%%==========Câu 55
\begin{ex}%Câu 55.%[2H5H1-4]
	Trong không gian $Oxyz$, khoảng cách giữa hai mặt phẳng $(P)\colon x+2 y-2 z-16=0$ và $(Q)\colon x+2 y-2 z-1=0$ bằng bao nhiêu?
\shortans[0]{$5$}
\loigiai{
Ta có $\heva{&(P)\parallel (Q) \\ &A(16;0;0)\in (P)}\Rightarrow \mathrm{d}\left((P),(Q)\right)=\mathrm{d}\left(A,(Q)\right)=\dfrac{\left| 16+2\cdot 0-2\cdot 0-1\right|}{\sqrt{1^2+2^2+2^2}}=5$.
}
\end{ex}
\Closesolutionfile{ans}
\indapan{10}{ans/ans2C5B1CD1-D2-DS}