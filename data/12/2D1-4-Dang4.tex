\Opensolutionfile{ans}[ans/ansCD2D1-4]

\begin{dang}{Bài toán xác định các đường tiệm cận của hàm số có chứa tham số}
\end{dang}
\paragraph{Các ví dụ}
\begin{vd}%[2D1B4-1]
	Đồ thị hàm số $y=\dfrac{ax+b}{2x+c}$ có tiệm cận ngang $y=2$ và tiệm cận đứng $x=1$ thì $a+c$ bằng 
	\choice
	{$1$}
	{\True $2$}
	{$4$}
	{$6$}
	\loigiai{
		Xét hàm số $y=\dfrac{ax+b}{2x+c}$ có tập xác định $\mathscr{D}=\mathbb{R}\setminus\left\{-\dfrac{c}{2}\right\}$. \\
		Tiệm cận ngang $y=\dfrac{a}{2}$ và tiệm cận đứng $x=-\dfrac{c}{2}$.\\
		Theo đề bài, ta có $\heva{&\dfrac{a}{2}=2\\&-\dfrac{c}{2}=1}\Leftrightarrow\heva{&a=4\\&c=-2}\Rightarrow a+c=2$.}
\end{vd}
\begin{vd}%[2D1B4-2]
	Tìm tất cả các giá trị của tham số $m$ để đồ thị hàm số $y=\dfrac{x-m}{x^2-3x+2}$ có đúng hai đường tiệm cận.
	\choice
	{$m=1$}
	{$m=-1$}
	{\True $m=1, m=2$}
	{Mọi $m\in\mathbb{R}$}
	\loigiai{
		Ta có $x^2-3x+2=0\Leftrightarrow\hoac{&x=1\\&x=2.}$\\
		Do $\lim\limits_{x\to\pm\infty} y=0$ nên đồ thị hàm số có một tiệm cận ngang là $y=0$.\\
		Đồ thị hàm số có đúng hai tiệm khi và chỉ khi nó có đúng một tiệm cận đứng, khi đó $m=1 $ hoặc $m=2$.}
\end{vd}
\begin{vd}%[2D1B4-2]
	Cho hàm số $y=\dfrac{x+1}{x^2-2mx+4}$. Tìm tất cả các giá trị thực của tham số $m$ để đồ thị hàm số có ba đường tiệm cận. 
	\choice
	{$m\in\emptyset$}
	{$\hoac{&m <-2\\&m>2}$}
	{$m>2$}
	{\True $\heva{&\hoac{&m>2\\&m <-2}\\&m\neq-\dfrac{5}{2}}$}
	\loigiai{
	Ta có $\lim\limits_{x\to+\infty} y=0$ suy ra  $y=0$ là tiệm cận ngang của đồ thị hàm số.\\
	Để đồ thị hàm số có $3$ tiệm cận thì phương trình $x^2-2mx+4=0$ có hai nghiệm phân biệt khác $-1$, hay
	\[\heva{&m^2-4>0\\&1+2m+4\neq 0}\Leftrightarrow\heva{&\hoac{&m>2\\&m <-2}\\&m\neq-\dfrac{5}{2}.}\]}
\end{vd}
\paragraph{Câu hỏi trắc nghiệm}
\begin{ex}%[2D1B4-2]
	Cho hàm số $y=\dfrac{x+8}{2x-a}$. Đồ thị hàm số có tiệm cận đứng $x=5$ thì giá trị của $a$ bằng
	\choice
	{$-5$}
	{\True $10$}
	{$-10$}
	{$5$}
	\loigiai{
		Đồ thị hàm số nhận đường thẳng $x=5$ làm tiệm cận đứng $\Leftrightarrow \heva{&\dfrac{a}{2}\neq-8\\&\dfrac{a}{2}=5}\Leftrightarrow a=10$.}
\end{ex}
\begin{ex}%[2D1K4-2]
	Cho hàm số $y=\dfrac{2mx+m}{x-1}$. Tìm tất cả các giá trị của tham số $m$ để đường tiệm cận đứng, tiệm cận ngang của đồ thị hàm số cùng hai trục tọa độ tạo thành một hình chữ nhật có diện tích bằng $8$.
	\choice
	{$m\neq\pm 2$}
	{$m=\pm\dfrac{1}{2}$}
	{$m=2$}
	{\True $m=\pm 4$}
	\loigiai{
		Đồ thị hàm số có đường TCĐ là $x=1$ và đường TCN là $y=2m$.\\
		Diện tích hình chữ nhật tạo bởi hai đường tiện cận và hai trục tọa độ có diện tích bằng $8$ khi và chỉ khi \[1\cdot|2m|=8\Leftrightarrow m=\pm 4.\]}
\end{ex}
\begin{ex}%[2D1K4-2]
	Tìm tất cả các giá trị thực của tham số $m$ để đồ thị hàm số $y=\dfrac{mx+3}{\sqrt{mx^2-5}}$ có hai đường tiệm cận ngang. 
	\choice
	{$m\geq 0$}
	{$m>\sqrt{5}$}
	{$m<0$}
	{\True $m>0$}
	\loigiai{
		Ta có $\lim\limits_{x\to+\infty}\dfrac{mx+3}{\sqrt{mx^2-5}}=\lim\limits_{x\to+\infty}\dfrac{m+\dfrac{3}{x}}{\sqrt{m-\dfrac{5}{x^2}}}$ và $\lim\limits_{x\to-\infty}\dfrac{mx+3}{\sqrt{mx^2-5}}=\lim\limits_{x\to-\infty}\dfrac{m+\dfrac{3}{x}}{-\sqrt{m-\dfrac{5}{x^2}}}$.\\
		Để đồ thị hàm số $y=\dfrac{mx+3}{\sqrt{mx^2-5}}$ có hai đường tiệm cận ngang thì $m>0$.\\
		Khi đó hai đường tiệm cận ngang là $y=\pm\sqrt{m}$.}
\end{ex}
\begin{ex}%[2D1K4-2]
	Biết đồ thị hàm số $y=\dfrac{(4a-b)x^2+ax+1}{x^2+ax+b-12}$ nhận trục hoành và trục tung làm hai tiệm cận thì giá trị $a+b$ bằng
	\choice
	{$-10$}
	{\True $15$}
	{$2$}
	{$10$}
	\loigiai{
		Xét $\lim\limits_{x\to\pm\infty} y=\lim\limits_{x\to\pm\infty}\dfrac{(4a-b)x^2+ax+1}{x^2+ax+b-12}=\lim\limits_{x\to\pm\infty}\dfrac{4a-b+\dfrac{a}{x}+\dfrac{1}{x^2}}{1+\dfrac{a}{x}+\dfrac{b-12}{x^2}}=4a-b$.\\
		Đồ thị đã cho nhận $y=0$ làm tiệm cận ngang nên $4a-b=0$.\\
		Đồ thị nhận $x=0$ làm tiệm cận đứng nên $x=0 $ là nghiệm của phương trình $x^2+ax+b-12=0$. \\
		Khi đó  $b=12$ nên $a=3$.\\
		Thử lại ta có $y=\dfrac{3x+1}{x^2+3x}$, xét $\lim\limits_{x\to 0^+} y=+\infty$ suy ra $x=0$ là tiệm cận đứng của đồ thị hàm số.\\
		Vậy $a+b=15$.}
\end{ex}
\begin{ex}%[2D1K4-2]
	Cho hàm số $y=\dfrac{2x^2-3x+m}{x-m}$ có đồ thị $(C)$. Tìm tất cả các giá trị của tham số $m$ để $(C)$ không có tiệm cận đứng. 
	\choice
	{\True $m=0$ hoặc $m=1$}
	{$m=2$}
	{$m=1$}
	{$m=0$}
	\loigiai{
		Đồ thị $(C)$ không có tiệm cận đứng khi tử số chứa nhân tử $x-m$.\\
		Tức là $x=m$ là nghiệm phương trình $2x^2-3x+m=0$. Khi đó 
		\[2m^2-3m+m=0\Leftrightarrow\hoac{&m=0\\&m=1.}\]}
\end{ex}
% \begin{ex}%[2D1G4-2]
% 	Cho đồ thị hàm số $y=\dfrac{\sqrt{4x+1}+ax+3b}{(x-2)^2}$ không có tiệm cận đứng. Tìm giá trị $3a+9b$. 
% 	\choice
% 	{$7$}
% 	{$11$}
% 	{\True $-7$}
% 	{$8$}
% 	\loigiai{
% 		Vì đồ thị hàm số $y=\dfrac{\sqrt{4x+1}+ax+3b}{(x-2)^2}$ không có tiệm cận đứng nên tử phải có nhân tử là $(x-2)^2$.\\
% 		Ta có
% 		\allowdisplaybreaks{
% 		\begin{eqnarray*}
% 		\sqrt{4x+1}+ax+3b&=&\sqrt{4x+1}-3+ax+3b+3\\
% 		&=&\dfrac{4(x-2)}{\sqrt{4x+1}+3}+a\left(x+\dfrac{3+3b}{a}\right).
% 		\end{eqnarray*} 
% 		}
% 		Để có nhân tử $(x-2)$ thì $\dfrac{3+3b}{a}=-2\Leftrightarrow 2a+3b=-3. \qquad (1)$\\
% 		Khi đó,
% 		\allowdisplaybreaks{
% 		\begin{eqnarray*}
% 			\sqrt{4x+1}+ax+3b&=&\dfrac{4(x-2)}{\sqrt{4x+1}+3}+a(x-2)\\
% 			&=&(x-2)\left(\dfrac{4}{\sqrt{4x+1}+3}+a\right)\\
% 			&=&(x-2)\left(\dfrac{a\sqrt{4x+1}+3a+4}{\sqrt{4x+1}+3}\right)\\
% 			&=&a(x-2)\dfrac{\sqrt{4x+1}+\dfrac{3a+4}{a}}{\sqrt{4x+1}+3}.
% 		\end{eqnarray*} 
% 		}
% 		Để có thêm một nhân tử $(x-2)$ thì
% 		\[\dfrac{\sqrt{4x+1}+\dfrac{3a+4}{a}}{\sqrt{4x+1}+3}=\dfrac{\sqrt{4x+1}-3}{\sqrt{4x+1}+3}\Rightarrow\dfrac{3a+4}{a}=-3\Leftrightarrow a=-\dfrac{2}{3}.\qquad (2)\]
% 		Từ (1) và (2) suy ra $\heva{&a=-\dfrac{2}{3}\\&b=-\dfrac{5}{9}.}$ \\
% 		Vậy $3a+9b=-7$.}
% \end{ex}
\begin{ex}%[2D1B4-2]
	Tìm giá trị thực của tham số $m$ để đồ thị hàm số $y=\dfrac{(m+1)x-2}{1-x}$ có đường tiệm cận ngang đi qua điểm $A(3;1)$. 
	\choice
	{$m=2$}
	{$m=0$}
	{\True $m=-2$}
	{$m=-4$}
	\loigiai{
		Ta có đồ thị hàm số có tiệm cận ngang $y=-m-1$.\\
		Tiệm cận ngang đi qua điểm $A(3;1)$ khi $1=-m-1\Rightarrow m=-2$.}
\end{ex}
\begin{ex}%[2D1B4-2]
	Cho hàm số $y=\dfrac{2x+2m-1}{x+m}$. Tìm tất cả các giá trị thực của tham số $m$ để đường tiệm cận đứng của đồ thị hàm số đi qua điểm $M(3;1)$. 
	\choice
	{$m=1$}
	{$m=3$}
	{\True $m=-3$}
	{$m=2$}
	\loigiai{
		Tiệm cận đứng của đồ thị hàm số là $x=-m$.\\
		Do đó, đường tiệm cận đứng của đồ thị hàm số đi qua điểm $M(3;1)$ khi $3=-m\Rightarrow m=-3$.}
\end{ex}
\begin{ex}%[2D1K4-2]
	Tìm $m$ để tiệm cận ngang của đồ thị hàm số $y=\dfrac{(m-1)x+2}{3x+4}$ cắt đường thẳng $2x-3y+5=0$ tại điểm có hoành độ bằng $2$. 
	\choice
	{$m=2$}
	{$m=1$}
	{\True $m=10$}
	{$m=7$}
	\loigiai{
		Ta có $\lim\limits_{x\to\pm\infty}\dfrac{(m-1)x+2}{3x+4}=\dfrac{m-1}{3}$ suy ra tiệm cận ngang của đồ thị là $\Delta\colon y=\dfrac{m-1}{3}$.\\
		Với $M(2;y)\in d\colon 2x-3y+5=0$ suy ra $M(2; 3)$.\\
		Theo đề $M(2; 3)\in \Delta\colon y=\dfrac{m-1}{3}$ suy ra $\dfrac{m-1}{3}=3\Leftrightarrow m=10$.}
\end{ex}
\begin{ex}%[2D1B4-2]
	Tìm tất cả các giá trị thực của tham số $m$ để đồ thị hàm số $y=\dfrac{mx-8}{x+2}$ có tiệm cận đứng. 
	\choice
	{$m=4$}
	{$m=-4$}
	{$m\neq 4$}
	{\True $m\neq-4$}
	\loigiai{
		Hàm số có tiệm cận đứng khi và chỉ khi phương trình $mx-8=0$ không có nghiệm $x=-2$.\\
		Suy ra $-2m-8\neq 0\Leftrightarrow m\neq-4$.}
\end{ex}
\begin{ex}%[2D1B4-2]
	Cho hàm số $y=\dfrac{x^2-2x+m^2+1}{x-1}\,\,(C)$. Tìm tất cả các giá trị thực của tham số $m$ để đồ thị $(C)$ có tiệm cận đứng. 
	\choice
	{\True $m\neq 0$}
	{$m=0$}
	{$m\in\emptyset$}
	{$m\in\mathbb{R}$}
	\loigiai{
		Đồ thị $(C)$ có tiệm cận đứng khi và chỉ khi $x=1$ không phải là nghiệm phương trình $x^2-2x+m^2+1=0$. \\
		Khi đó $1-2+m^2+1\neq 0\Leftrightarrow m\neq 0$.}
\end{ex}
\begin{ex}%[2D1K4-2]
	Biết đồ thị hàm số $y=\dfrac{(2m-n)x^2+mx+1}{x^2+mx+n-6}$ ($m, n$ là tham số) nhận trục hoành và trục tung làm hai đường tiệm cận. Tính $m+n$. 
	\choice
	{$6$}
	{$-6$}
	{$8$}
	{\True $9$}
	\loigiai{
		Đồ thị nhận trục $Ox$ làm tiệm cận ngang, suy ra
		\[\lim\limits_{x\to\pm\infty} y=\lim\limits_{x\to\pm\infty}\dfrac{(2m-n)x^2+mx+1}{x^2+mx+n-6}=2m-n=0 \text{ hay } n=2m.\]
		Đồ thị nhận $Oy$ làm tiệm cận đứng, suy ra
		\[\heva{&0^2+m\cdot 0+n-6=0\\&(2m-n)0^2+m\cdot 0+1\neq 0}\Leftrightarrow\heva{&n=6\\&1\neq 0}\Leftrightarrow n=6.\]
		Vậy $n=6, m=3$.}
\end{ex}
\begin{ex}%[2D1B4-2]
	Tìm tất cả các giá trị của tham số thực $m$ để đồ thị hàm số $y=\dfrac{mx+2}{1-x}$ luôn có tiệm cận ngang. 
	\choice
	{\True $\forall m\in\mathbb{R}$}
	{$\forall m\neq 2$}
	{$\forall m\neq-2$}
	{$\forall m\neq\dfrac{1}{2}$}
	\loigiai{
		Ta có $\lim\limits_{x\to\pm\infty} y=\lim\limits_{x\to\pm\infty}\dfrac{mx+2}{1-x}=-m$.\\
		Do đó đồ thị hàm số luôn có tiệm cận ngang với mọi giá trị thực của $m$.}
\end{ex}
\begin{ex}%[2D1K4-2]
	Tìm tập hợp tất cả các giá trị của tham số $m$ để đồ thị hàm số $y=\dfrac{1+\sqrt{x+1}}{\sqrt{x^2-mx-3m}}$ có đúng hai đường tiệm cận đứng. 
	\choice
	{\True $\left(0;\dfrac{1}{2}\right]$}
	{$(0;+\infty)$}
	{$\left[\dfrac{1}{4};\dfrac{1}{2}\right]$}
	{$\left(0;\dfrac{1}{2}\right)$}
	\loigiai{
		Điều kiện xác định là $\heva{&x\geq-1\\&x^2-mx-3m>0.}$ \\
		Do đó nếu đường thẳng $x=x_0$ là một tiệm cận đứng của đồ thị hàm số thì $x_0\geq-1$.\\
		Suy ra đồ thị hàm số đã cho có đúng hai tiệm cận đứng khi và chỉ khi phương trình $x^2-mx-3m=0$ có hai nghiệm phân biệt $(x\geq-1)$. Hay
		\allowdisplaybreaks{
		\begin{eqnarray*}
		\heva{&\Delta=m^2+12m>0\\&(x_1+1)(x_2+1)\geq 0\\&(x_1+1)+(x_2+1)>0} &\Leftrightarrow& \heva{&m^2+12m>0\\&x_1x_2+(x_1+x_2)+1\geq 0\\&x_1+x_2+2>0}\\
		&\Leftrightarrow& \heva{&m^2+12m>0\\&-3m+m+1\geq 0\\&m+2>0}\\
		&\Leftrightarrow& \heva{&\hoac{&m>0\\&m <-12}\\&m\leq\dfrac{1}{2}\\&m >-2}\\
		&\Leftrightarrow& 0<m\leq\dfrac{1}{2}.
		\end{eqnarray*} 
		}
		Suy ra $m\in\left(0;\dfrac{1}{2}\right]$.}
\end{ex}
\begin{ex}%[2D1K4-2]
	Số giá trị nguyên dương của tham số $a$ để đồ thị hàm số $y=ax+\sqrt{4x^2+1}$ có đường tiệm cận ngang là
	\choice
	{3}
	{2}
	{4}
	{\True 1}
	\loigiai{
		Với $a>0$ ta có 
		\begin{itemize}
			\item $\lim\limits_{x\to+\infty} y=\lim\limits_{x\to+\infty}\left(ax+\sqrt{4x^2+1}\right) =\lim\limits_{x\to+\infty} x\left(a+\sqrt{4+\dfrac{1}{x^2}}\right)=+\infty$.
			\item $\lim\limits_{x\to-\infty} y=\lim\limits_{x\to-\infty}\left(ax+\sqrt{4x^2+1}\right) =\lim\limits_{x\to-\infty}\dfrac{a^2x^2-4x^2-1}{ax-\sqrt{4x^2+1}} =\lim\limits_{x\to-\infty}\dfrac{\left(a^2-4\right)x^2-1}{ax-\sqrt{4x^2+1}}$.
		\end{itemize}
		Để hàm số có tiệm cận ngang thì một trong các giới hạn ở trên phải hữu hạn, suy ra $a^2-4=0\Rightarrow a=2$.}
\end{ex}
\begin{ex}%[2D1K4-2]
	Biết đồ thị hàm số $y=\dfrac{(a-2b)x^2+bx+1}{x^2+x-b}$ có tiệm cận đứng là $x=1$ và tiệm cận ngang $y=0$. Tính $a+2b$. 
	\choice
	{$7$}
	{$6$}
	{\True $8$}
	{$10$}
	\loigiai{
		Đồ thị có tiệm cận đứng là $x=1$ và tiệm cận ngang $y=0$ nên 
		\[\heva{&a-2b=0\\&1+1-b=0}\Leftrightarrow \heva{& a=4 \\ & b=2.}\]
		Thử lại, lúc này $y=\dfrac{2x+1}{x^2+x-4}$ có tiệm cận đứng là $x=1$ và tiệm cận ngang $y=0$ nên nhận $b=2$ và $a=4$.\\
		Suy ra $a+2b=8$.}
\end{ex}
\begin{ex}%[2D1K4-2]
	Có bao nhiêu giá trị của tham số $m$ để đồ thị hàm số $y=\dfrac{mx+\sqrt{x^2-2x+3}}{2x-1}$ có một tiệm cận ngang là $y=2$?
	\choice
	{$1$}
	{Vô số}
	{$0$}
	{\True $2$}
	\loigiai{
		Xét các giới hạn
		\allowdisplaybreaks{
		\begin{eqnarray*}
		\hoac{&\lim\limits_{x\to+\infty}\dfrac{mx+\sqrt{x^2-2x+3}}{2x-1}=2\\&\lim\limits_{x\to-\infty}\dfrac{mx+\sqrt{x^2-2x+3}}{2x-1}=2} &\Leftrightarrow& \hoac{&\lim\limits_{x\to+\infty}\dfrac{x\left(m+\sqrt{1-\dfrac{2}{x}+\dfrac{3}{x^2}}\right)}{x\left(2-\dfrac{1}{x}\right)}=2\\&\lim\limits_{x\to-\infty}\dfrac{x\left(m-\sqrt{1-\dfrac{2}{x}+\dfrac{3}{x^2}}\right)}{x\left(2-\dfrac{1}{x}\right)}=2}\\
		&\Leftrightarrow& \hoac{&\dfrac{m+1}{2}=2\\&\dfrac{m-1}{2}=2}\\
		&\Leftrightarrow&\hoac{&m=3\\&m=5.}
		\end{eqnarray*} 
		}
		Vậy $m=3$, $m=5$ thì đồ thị hàm số đã cho có một tiệm cận ngang là $y=2$.}
\end{ex}
\begin{ex}%[2D1K4-2]
	Cho hàm số $y=\dfrac{2x+1}{x-m}$ có đồ thị là $(C_m)$. Tìm tổng tất cả các giá trị $m$ nguyên dương sao cho diện tích hình chữ nhật tạo bởi các trục tọa độ và hai đường tiệm cận của đồ thị $(C_m)$ không vượt quá $2018$ (đvdt). 
	\choice
	{\True $509545$}
	{$1009$}
	{$2018!$}
	{$2018$}
	\loigiai{
		Do $m\in\mathbb{Z}^+$ nên $(C_m)$ luôn có hai đường tiệm cận là $x=m$ và $y=2$.\\
		Khi đó diện tích hình thang cần tìm là $S=2m$.\\
		Có $S\leq 2018\Leftrightarrow 2m\leq 2018\Leftrightarrow m\leq 1009\Leftrightarrow m\in\left\{1;2;3;\ldots;1009\right\}$.
		\[1+2+3+\cdots +1009=\dfrac{1009\cdot 1010}{2}=509545.\]}
\end{ex}
\begin{ex}%[2D1K4-2]
	Có bao nhiêu giá trị của tham số $m$ thoả mãn đồ thị hàm số $y=\dfrac{x+3}{x^2-x-m}$ có đúng hai đường tiệm cận?
	\choice
	{$1$}
	{$4$}
	{\True $2$}
	{$3$}
	\loigiai{
		Đồ thị hàm số có đúng hai đường tiệm cận khi phương trình $x^2-x-m=0$ có nghiệm kép hoặc có hai nghiệm phân biệt với một nghiệm bằng $-3$. Khi đó
		\[\hoac{&\Delta=0\\&\heva{&\Delta>0\\&g(-3)=0}}\Leftrightarrow\hoac{&4m+1=0\\&\heva{&4m+1>0\\&m=12}}\hoac{&m=-\dfrac{1}{4}\\&m=12.}\]
		Vậy có hai giá trị của m.}
\end{ex}
\begin{ex}%[2D1K4-2]
	Tìm tất cả các giá trị thực của tham số $m$ để đồ thị hàm số $y=\dfrac{x+1}{\sqrt{m(x-1)^2+4}}$ có hai tiệm cận đứng.
	\choice
	{$m<1$}
	{\True $\heva{&m<0\\&m\neq-1}$}
	{$m=0$}
	{$m<0$}
	\loigiai{
		Điều kiện $m(x-1)^2+4>0$.\\
		Đồ thị hàm số $y=\dfrac{x+1}{\sqrt{m(x-1)^2+4}}$ có hai tiệm cận đứng khi và chỉ khi
		\begin{eqnarray*}
		&& m(x-1)^2+4=0 \text{ có 2 nghiệm phân biệt và khác } -1 \\
		&\Leftrightarrow& \heva{&m(-1-1)^2+4\neq 0\\&-\dfrac{4}{m}>0}\\
		&\Leftrightarrow& \heva{&m\neq-1\\&m<0.}
		\end{eqnarray*}
		}
\end{ex}
\begin{ex}%[2D1K4-2]
	Cho hàm số $y=\dfrac{3mx+1}{nx+n-1}$ với $n\neq 0$ và $3m(n-1)\neq n$. Đồ thị hàm số nhận hai trục tọa độ làm tiệm cận đứng, tiệm cận ngang. Khi đó $(m-n)^{2019}$ bằng bao nhiêu?
	\choice
	{$2^{2019}$}
	{\True $-1$}
	{$1$}
	{$2019$}
	\loigiai{
		Tiệm cận ngang là $y=\dfrac{3m}{n}=0\Rightarrow m=0$.\\
		Tiệm cận đứng là $x=\dfrac{1-n}{n}=0\Rightarrow n=1$.\\
		Suy ra $(m-n)^{2019}=(-1)^{2019}=-1$.}
\end{ex}
\begin{ex}%[2D1K4-2]
	Tìm tất cả các giá trị thực $m$ sao cho đồ thị hàm số $y=\dfrac{5x-3}{x^2-2mx+1}$ không có tiệm cận đứng. 
	\choice
	{\True $-1<m<1$}
	{$m=1$}
	{$m=-1$}
	{$m <-1$ hoặc $m>1$}
	\loigiai{
		Xét $f(x)=5x-3$, có $f(x)=0\Leftrightarrow x=\dfrac{3}{5}$; $g(x)=x^2-2mx+1$ có $\Delta’=m^2-1$.\\
		Đồ thị hàm số không có tiệm cận đứng khi phương trình $g(x)=0$ vô nghiệm $\Leftrightarrow m^2-1<0\Leftrightarrow-1<m<1$.\\
		Vậy với $-1<m<1$ thì đồ thị hàm số đã cho không có tiệm cận đứng.}
\end{ex}
\Closesolutionfile{ans}