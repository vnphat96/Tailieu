\begin{name}
	{\tenchude}
	{TOÁN 12}
	{LỚP TOÁN THẦY PHÁT}
	{Thời gian: 90 phút - Không kể thời gian phát đề}
\end{name}
\Opensolutionfile{ansbook}[ans/ansbookDe2]
\TN
\Opensolutionfile{ans}[ans/ansDe2-TN1]
\begin{ex}%[Dự án 2025 - đề cấu trúc mới, Nguyễn Kiều Nhã Tú]%[2D4N1-1]
Hàm số $F(x)$ là một nguyên hàm của hàm số $f(x)$ trên khoảng $K$ nếu
\choice
{$F'(x)=-f(x)$, $\forall x\in K$}
{$f'(x)=F(x)$, $\forall x\in K$}
{\True $F'(x)=f(x)$, $\forall x \in K$}
{$f'(x)=-F(x)$, $\forall x\in K$}
\loigiai{
Theo tính chất của nguyên hàm có $F'(x)=f(x)$, $\forall x\in K$.
}
\end{ex}

\begin{ex}%[2D4N1-2]
Tìm nguyên hàm của hàm số $f(x)=\sqrt {2x-1}$.
\choice
{ $\displaystyle\int{f(x)\mathrm{d}x=\dfrac{2}{3}( 2x-1 )\sqrt {2x-1}+C}$}
{\True $\displaystyle\int{f(x)\mathrm{d}x=\dfrac{1}{3}( 2x-1 )\sqrt {2x-1}+C}$}
{ $\displaystyle\int{f(x)\mathrm{d}x=-\dfrac{1}{3}\sqrt {2x-1}+C}$}
{$\displaystyle\int{f(x)\mathrm{d}x=\dfrac{1}{2}\sqrt {2x-1}+C}$}
\loigiai{
$\displaystyle\int{f\left( x \right)\mathrm{d}x=\displaystyle\int{\sqrt {2x-1}\mathrm{d}x=\dfrac{1}{2}\displaystyle\int{{\left( 2x-1 \right)}^{\frac{1}{2}}\mathrm{d}( 2x-1 )}}}=\dfrac{1}{3}( 2x-1 )\sqrt {2x-1}+C$.}
\end{ex}

\begin{ex}%[2D4N2-1]
Nếu $\displaystyle\int_0^2 f(x) d x=3$ thì $\displaystyle\int_0^2\left[2f(x)-1\right]\mathrm{\,d}x$ bằng
\choice
{$6$}
{\True $4$}
{$8$}
{$5$}
\loigiai{
Ta có $\displaystyle\int_0^2\left[2f(x)-1\right]\mathrm{\,d}x=2\displaystyle\int_0^2f(x)\mathrm{\,d}x-\displaystyle\int_0^2\mathrm{\,d}x=2\cdot 3-2=4$.}
\end{ex}

\begin{ex}%[2H5N1-1]
Trong không gian với hệ toạ độ $Oxyz$, phương trình nào dưới đây là phương trình của mặt phẳng $(Oyz)$?
\choice
{$y=0$}
{\True $x=0$}
{$y-z=0$}
{$z=0$}
\loigiai{
Mặt phẳng $(Oyz)$ đi qua điểm $O(0 ; 0 ; 0)$ và có véc-tơ  pháp tuyến là $\vec{i}=(1 ; 0 ; 0)$ nên ta có phương trình mặt phẳng $(O y z)$ là  $1(x-0)+0(y-0)+0(z-0)=0 \Leftrightarrow x=0$.
}
\end{ex}

\begin{ex}%[Mức độ 1]%[BG-12-New-4in1, Hiệp Hà]%[2H5N1-2]
Vectơ nào dưới đây là một vectơ pháp tuyến của $(Oxy)$?
\choice
{$\vec{n_1}=(2;0;0)$}
{$\vec{n_2}=(1;1;0)$}
{$\vec{n_3}=(0;3;0)$}
{\True $\vec{n_4}=(0;0;-1)$}
\loigiai{
Ta có $Oz \perp (Oxy)$ nên $\vec{k}=(0;0;1)$ là một vectơ pháp tuyến của $(\alpha)$.\\
Khi đó, $\vec{n_4}=-\vec{k}$ là một vectơ pháp tuyến của $(\alpha)$.
}
\end{ex}

\begin{ex}%[12-MH-2-MH2025]%[MH-2025,Chu Hà]%[2H5N2-1]
Trong không gian tọa độ $Oxyz$, phương trình nào sau đây là phương trình tham số của đường thẳng?
\def\dotEX{}
\choice
{$\heva{&2x+3y-z=0\\&x+y+z=8.}$}
{$\heva{&-3x+z=0\\&x+2y+z-7=0.}$}
{$\heva{&x=2+t\\&y=3-t\\&z =t^2.}$}
{\True  $\heva{&x=-4-3t\\&y=2-5t\\&z=-1+6t.}$}
\loigiai{
Phương trình đường thẳng có dạng $\heva{&x=-4-3t\\&y=2-5t\\&z=-1+6t.}$ với $t$ là tham số.
}
\end{ex}

\begin{ex}%[Ex-Ôn tập 2025, Nguyễn Văn Nay]%[2H5N2-2]
Trong không gian với hệ tọa độ $Oxyz$, vectơ nào sau đây là vectơ chỉ phương của đường thẳng $\Delta\colon\heva{&x=-4+2t \\ &y=7-3t \\&z=8-9t}$?
\choice
{$\vec{u}_1=(4;7;8)$}
{$\vec{u}_2=(-4;7;8)$}
{$\vec{u}_3=(2;3;9)$}
{\True $\vec{u}_4=(2;-3;-9)$}
\loigiai{Một vectơ chỉ phương của đường thẳng là $\vec{u}_4=(2;-3;-9)$.}
\end{ex}

\begin{ex}%[NB]giảng 12 New - 4in1, Nguyễn Vân Trường]%[2H5N3-2]
Trong hệ trục toạ độ $Oxyz$, tìm $m$ để phương trình $(S_m) \colon x^2+y^2+z^2+2x-2my+2mz-5m^2 = 0$ xác định một mặt cầu có bán kính nhỏ nhất.
\choice
{$m=1$}
{$m=7$}
{\True $m=0$}
{$m=\dfrac{1}{7}$}
\loigiai{
Để phương trình $(S_m)$ xác định một mặt cầu thì $1^2+m^2+m^2+5m^2 =1+7m^2>0$ với mọi $m \in \mathbb{R}$.
Khi đó bán kính của mặt cầu $(S_m)$ là
$R=\sqrt{1+7m^2} \ge 1$. \\
Dấu đẳng thức xảy ra khi $m = 0$. Vậy $m=0$.
}
\end{ex}

\begin{ex}%[2D6N1-1]
Cho hai biến cố xung khắc $A$, $B$ thoả mãn $\mathrm{P}(A)=0{,}35$ và $\mathrm{P}(B)=0{,}55$. Khi đó $\mathrm{P}(A\mid B)$ bằng
\choice
{\True $0$}
{$0{,}9$}
{$0{,}1$}
{$0{,}4$}
\loigiai{
Do $A$, $B$ xung khắc nên $\mathrm{P}(A\mid B)=0$.
}
\end{ex}

\begin{ex}%[2D6N2-1]%[Lê Công Trường]
Cho hai biến cố $C$ và $D$ với $0\leq \mathrm{P}(D)\leq 1$. Công thức xác suất toàn phần là
\choice
{\True $\mathrm{P}(C)=\mathrm{P}(D)\cdot\mathrm{P}(C\mid D)+\mathrm{P}(\overline{D})\cdot\mathrm{P}(C\mid \overline{D})$}
{$\mathrm{P}(C)=\mathrm{P}(\overline{D})\cdot\mathrm{P}(C\mid D)+\mathrm{P}(D)\cdot\mathrm{P}(C\mid \overline{D})$}
{$\mathrm{P}(D)=\mathrm{P}(\overline{D})\cdot\mathrm{P}(C\mid D)+\mathrm{P}(D)\cdot\mathrm{P}(C\mid \overline{D})$}
{$\mathrm{P}(D)=\mathrm{P}(\overline{D})\cdot\mathrm{P}(C\mid D)+\mathrm{P}(C)\cdot\mathrm{P}(C\mid \overline{D})$}
\loigiai{Cho hai biến cố $C$ và $D$ với $0\leq \mathrm{P}(D)\leq 1$. Công thức xác suất toàn phần là \[\mathrm{P}(C)=\mathrm{P}(D)\cdot\mathrm{P}(C\mid D)+\mathrm{P}(\overline{D})\cdot\mathrm{P}(C\mid \overline{D})\] }
\end{ex}

\begin{ex}%[2D6N2-3]%[Dự án EX-TF-TLN lần 4 - Quan Ón]
Cho các biến cố $A$ và $B$ thỏa mãn $\mathrm{P}(A) > 0$, $\mathrm{P}(B) > 0$. Khi đó $\mathrm{P}(A\mid B)$ bằng biểu thức nào dưới đây?
\choice
{\True $\dfrac{\mathrm{P}(A)\cdot \mathrm{P}(B\mid A)}{\mathrm{P}(B)}$}
{$\dfrac{\mathrm{P}(B)\cdot \mathrm{P}(B\mid A)}{\mathrm{P}(A)}$}
{$\dfrac{\mathrm{P}(B)}{\mathrm{P}(A)\cdot \mathrm{P}(B\mid A)}$}
{$\dfrac{\mathrm{P}(A)}{\mathrm{P}(B)\cdot \mathrm{P}(B\mid A)}$}
\loigiai{
Theo công thức xác suất có điều kiện, ta có $\mathrm{P}(A\mid B) = \dfrac{\mathrm{P}(AB)}{\mathrm{P}(B)}$.\\
Theo công thức nhân, ta có $\mathrm{P}(AB) = \mathrm{P}(A)\cdot \mathrm{P}(B\mid A)$.\\
Do đó $\mathrm{P}(A\mid B) = \dfrac{\mathrm{P}(AB)}{\mathrm{P}(B)} = \dfrac{\mathrm{P}(A)\cdot \mathrm{P}(B\mid A)}{\mathrm{P}(B)}$.
}
\end{ex}

\begin{ex}%[2H5N3-3]
Trong không gian $Oxyz$, mặt cầu có tâm $I(1;-1; 2)$ và bán kính $R=5$ có phương trình là
\choice
{\True $(x-1)^2+(y+1)^2+(z-2)^2=25$}
{$(x-1)^2+(y+1)^2+(z-2)^2=5$}
{$(x+1)^2+(y-1)^2+(z+2)^2=25$}
{$(x+1)^2+(y-1)^2+(z+2)^2=5$}
\loigiai{
Mặt cầu cần tìm có phương trình $(x-1)^2+(y+1)^2+(z-2)^2=25$.
}
\end{ex}
\Closesolutionfile{ans}

\TNTF
\Opensolutionfile{ans}[ans/ansDe2-TN2]
\begin{ex}%[2D4H2-2]
Cho hàm số $f(x) = \heva{
&2x^2 + 3 \text{ khi } x \geq 1 \\
&2 - x^3 \text{ khi } x < 1
}$.
\choiceTF
{\True Trên $[1;+\infty)$ hàm số $f(x)$ có nguyên hàm là $F_1(x) = \dfrac{2}{3}x^3 + 3x + C_1$}

{\True Trên $(-\infty;1)$ hàm số $f(x)$ có nguyên hàm là $F_2(x) = 2x - \dfrac{1}{4}x^4 + C_2$}
{$\displaystyle\int\limits_{-2025}^{2025} f(x) \mathrm{d}x = \displaystyle\int\limits_{1}^{2025} (2x^2 + 3) \mathrm{d}x + \displaystyle\int\limits_{-2025}^{1} (2 - x^3) \mathrm{d}x$}
{\True Diện tích hình phẳng tạo bởi đồ thị hàm số $f(x)$, trục hoành và hai đường thẳng $x=-1$ và $x=2$ là $S=\displaystyle\int\limits_{-1}^{1} (2 - x^3) \mathrm{d}x + \displaystyle\int\limits_{1}^{2} (2x^2 + 3) \mathrm{d}x$}

\loigiai{
Do $f(x) = \heva{
&2x^2 + 3 \text{ khi } x \geq 1 \\
&2 - x^3 \text{ khi } x < 1
}$ nên
\begin{itemchoice}
\itemch Trên $[1;+\infty)$ hàm số $f(x)$ có nguyên hàm là $\displaystyle\int (2x^2 + 3) \mathrm{d}x = \dfrac{2}{3}x^3 + 3x + C_1$.
\itemch Trên $(-\infty;1)$ hàm số $f(x)$ có nguyên hàm là $\displaystyle\int (2 - x^3) \mathrm{d}x = 2x - \dfrac{1}{4}x^4 + C_2$.
\itemch Ta có $\displaystyle\int\limits_{-2025}^{2025} f(x) \mathrm{d}x = \displaystyle\int\limits_{1}^{2025} (2x^2 + 3) \mathrm{d}x + \displaystyle\int\limits_{-2025}^{1} (2 - x^3) \mathrm{d}x$
\itemch Diện tích hình phẳng tạo bởi đồ thị hàm số $f(x)$, trục hoành và hai đường thẳng $x=-1$ và $x=2$ là $$S=\displaystyle\int\limits_{-1}^{1} |f(x)| \mathrm{d}x + \displaystyle\int\limits_{1}^{2} |f(x)| \mathrm{d}x = \displaystyle\int\limits_{-1}^{1} |2 - x^3| \mathrm{d}x + \displaystyle\int\limits_{1}^{2} |2x^2 + 3| \mathrm{d}x = \displaystyle\int\limits_{-1}^{1} (2 - x^3) \mathrm{d}x + \displaystyle\int\limits_{1}^{2} (2x^2 + 3) \mathrm{d}x$$
\end{itemchoice}
}

\end{ex}

\begin{ex}%[Mức độ 2]%[BG-12-New-4in1, Hiệp Hà]%[2H5H1-2]
Cho mặt phẳng $(\alpha)$ đi qua các điểm $M(1; -2; 3)$, $N(2; 1; -1)$, $P(0;-2;4)$. Mỗi khẳng định dưới đây đúng hay sai?
\choiceTF
{\True $\vec{MN}$, $\vec{PN}$ là cặp vectơ chỉ phương của mặt phẳng $(\alpha)$}
{$\vec{n_1}=(3;-1;2)$ là một vectơ pháp tuyến của mặt phẳng $(\alpha)$}
{\True Đường thẳng $d$ đi qua $M$, $N$ có phương trình $\heva{&x=1+t\\&y=-2+3t\\&z=3-4t}$}
{Mặt cầu $(S)$ có đường kính $NP$ có phương trình là $(x-1)^2+(y+2)^2+(z-4)^2=25$}
\loigiai{
\begin{itemchoice}
\itemch
$\vec{MN}=(1;3;-4)$, $\vec{PN}=(2;3;-5)$ là hai vectơ không cùng phương và có giá nằm trong $(\alpha)$.\\
Do đó, $\vec{MN}$, $\vec{PN}$ là cặp vectơ chỉ phương của mặt phẳng $(\alpha)$.
\itemch Ta có
\[
\begin{aligned}
\vec{n}=[\vec{MN}, \vec{PN}] & =\left(\left|\begin{array}{cc}
3 & -4 \\ 3 & -5
\end{array}\right| ;\left|\begin{array}{cc}
-4 & 1 \\ -5 & 2
\end{array}\right| ;\left|\begin{array}{cc}
1 & 3 \\ 2 & 3
\end{array}\right|\right) \\
& =(-3; -3; -3) .
\end{aligned}
\]
Do đó, $\vec{n_1}=(3;-1;2)$ không phải là vectơ pháp tuyến của mặt phẳng $(\alpha)$.
\itemch Đường thẳng $d$ đi qua $M$, $N$ nên có vectơ chỉ phương là $\vec{MN}=(1;3;-4)$ và có phương trình tham số là $\heva{&x=1+t\\&y=-2+3t\\&z=3-4t}$.
\itemch Mặt cầu $(S)$ có đường kính $NP$ có tâm là trung điểm của đoạn thẳng $NP$ là $I(1; -\dfrac{1}{2}; \dfrac{3}{2})$ và bán kính là $R=\dfrac{1}{2}NP=\dfrac{1}{2}\sqrt{38}$.\\
Vậy phương trình mặt cầu $(S)$ là $(x-1)^2+(y+\dfrac{1}{2})^2+(z-\dfrac{3}{2})^2=\dfrac{19}{2}$.
\end{itemchoice}
}
\end{ex}
\Closesolutionfile{ans}

\TNSA
\Opensolutionfile{ans}[ans/ansDe2-TN3]
\begin{ex}%[2D4H1-4]
Cho $F(x)$ là một nguyên hàm của hàm số $f(x)=x\sqrt{x}+\dfrac{1}{\sqrt{x}}$. Biết $F(1)=-2$. Tính $F(0)$.
\shortans{$-4{,}4$}
\loigiai{
Hàm số $f(x)=x\sqrt{x}+\dfrac{1}{\sqrt{x}}=x^{\tfrac{3}{2}}+x^{-\tfrac{1}{2}}$.\\
Có $F(x)=\displaystyle \int f(x)\mathrm{\,d}x=\displaystyle \int\left(x^{\tfrac{3}{2}}+x^{-\tfrac{1}{2}}\right)\mathrm{\,d}x=\dfrac{2}{5}x^{\tfrac{5}{2}}+2\sqrt{x}+C$.\\
Do $F(1)=-2\Rightarrow -2=\dfrac{2}{5}\cdot 1^{\tfrac{5}{2}}+2\sqrt{1} +C \Rightarrow C=-\dfrac{22}{5}$.\\
Suy ra $F(x)=\dfrac{2}{5}x^{\tfrac{5}{2}}+2\sqrt{x} -\dfrac{22}{5}$.\\
Vậy $F(0)=-4{,}4$.
}
\end{ex}

\begin{ex}%[2D4V2-6]
Một chất điểm $A$ xuất phát từ $O$, chuyển động thẳng với vận tốc biến thiên theo thời gian bởi quy luật $v \left(t\right) = \dfrac{1}{100}t^2 + \dfrac{13}{30}t$ (m/s), trong đó $t$ (giây) là khoảng thời gian tính từ lúc $A$ bắt đầu chuyển động. Từ trạng thái nghỉ, một chất điểm $B$ cũng xuất phát từ $O$, chuyển động thẳng cùng hướng với $A$ nhưng chậm hơn $10$ giây so với $A$ và có gia tốc bằng $a$ (m/s$^2$ ) ( $a$ là hằng số). Sau khi $B$ xuất phát được $15$ giây thì đuổi kịp $A$. Vận tốc của $B$ tại thời điểm đuổi kịp $A$ bằng bao nhiêu m/s?
\shortans{$25$}
\loigiai{
Ta có $v_{B}(t) = \displaystyle\int a \cdot \mathrm{\,d}t = at + C$, $v_{B} (0) = 0 \Rightarrow C = 0 \Rightarrow v_{B} \left(t\right) = at$.\\
Quãng đường chất điểm $A$ đi được trong $25$ giây là
\[S_{A} = \displaystyle\int\limits_0^{25} \left(\dfrac{1}{100}t^2 + \dfrac{13}{30}t \right) \mathrm{\,d}t = \left(\dfrac{1}{300}t^3 + \dfrac{13}{60}t^2 \right) \Big|_0^{25} = \dfrac{375}{2}.\]
Quãng đường chất điểm $B$ đi được trong $15$ giây là
\[S_{B} = \displaystyle\int\limits_0^{15} at \cdot \mathrm{\,d}t = \dfrac{at^2}{2} \Big|_0^{15} = \dfrac{225a}{2}.\]
Ta có $\dfrac{375}{2} = \dfrac{225a}{2} \Leftrightarrow a = \dfrac{5}{3}$.\\
Vận tốc của $B$ tại thời điểm đuổi kịp $A$ là $v_{B} \left(15\right) = \dfrac{5}{3} \cdot 15 = 25$ (m/s).
}
\end{ex}

\begin{ex}%[2H5C2-7]
	Cho biết kim tự tháp Memphis tại bang Tennessee (Mỹ) có dạng hình chóp tứ giác đều $S.ABCD$ với chiều cao $98$ m và cạnh đáy $180$ m. Số đo góc giữa hai mặt bên bằng bao nhiêu độ (làm tròn kết quả đến hàng đơn vị)?
	\shortans{$63$}
	\loigiai{
	\immini{
	Gọi $O=AC \cap BD$. Vì $S.ABCD$ là hình chóp đều nên $S O \perp(A B C D)$.\\
	Ta có $AC=BD=AB \sqrt{2}=180 \sqrt{2}$.\\
	Chọn hệ trục $Oxyz$ như hình vẽ với $O(0;0;0)$, $C(90 \sqrt{2} ; 0 ; 0)$, $D(0 ; 90 \sqrt{2} ; 0)$, $B(0;-900\sqrt{2};0)$ và $S(0 ; 0 ; 98)$.
	}
	{
	\begin{tikzpicture}[scale=0.55, font=\footnotesize, line join=round, line cap=round, >=stealth]
	\def\bc{5} % cạnh BC
	\def\ba{3.5} % cạnh BA
	\def\h{5} % đường cao
	\def\gocB{40} % góc B của đáy
	\coordinate (B) at (0,0);
	\coordinate (A) at (\gocB:\ba);
	\coordinate (C) at (\bc,0);
	\coordinate (D) at ($(C)-(B)+(A)$);
	\coordinate (O) at ($(A)!.5!(C)$);
	\coordinate (S) at ($(O)+(90:\h)$);
	\coordinate (z) at ($(O)+(90:\h+1)$);
	\coordinate (y) at ($(O)!1.25!(D)$);
	\coordinate (x) at ($(O)!1.5!(C)$);
	\draw (B)--(C)--(D)--(S)--cycle (S)--(C);
	\draw[dashed] (C)--(A)--(D)--(B) (O)--(S)--(A)--(B);
	\draw [->] (S)--(z);
	\draw [->] (D)--(y);
	\draw [->] (C)--(x);
	\foreach \p/\i in {S/180, A/180,B/-60,D/40,C/-120,O/170}
	\fill (\p) circle (1.5pt) node[shift={(\i:3mm)}]{$\p$};
	\foreach \p/\i in {z/180,y/-90,x/40}
	\fill (\p)  node[shift={(\i:3mm)}]{$\p$};
	\end{tikzpicture}}
	\noindent
	Phương trình mặt phẳng $(SBC)$ dưới dạng đoạn chắn là
	\[ \dfrac{x}{90\sqrt{2}}+\dfrac{y}{-90 \sqrt{2}}+\dfrac{z}{98}=1\qquad \text{hay } 49x-49y+45\sqrt{2} z-4410 \sqrt{2}=0.\]
	Phương trình mặt phẳng $(SCD)$ dưới dạng đoạn chắn là\\ $ \dfrac{x}{90 \sqrt{2}}+\dfrac{y}{90 \sqrt{2}}+\dfrac{z}{98}=1$ hay $49x+49y+45\sqrt{2} z-a \sqrt{2}=0$.\\
	Khi đó hai mặt phẳng $(SBC)$ và $(SCD)$ có véc-tơ pháp tuyến lần lượt là  $\overrightarrow{n}_1=(49 ; -49 ; 45\sqrt{2})$, $\overrightarrow{n}_2=(49 ; 49 ; 45\sqrt{2})$.\\
	Suy ra \[\cos ((S BC),(SCD))=\dfrac{|\overrightarrow{n}_1 \cdot \overrightarrow{n}_2|}{|\overrightarrow{n}_1| \cdot|\overrightarrow{n}_2|}=\dfrac{\left|49\cdot 49-49\cdot 49+45\sqrt{2}\cdot 45\sqrt{2} \right| }{\sqrt{49^2+(-49)^2+(45\sqrt{2})^2}\cdot \sqrt{49^2+49^2+(45\sqrt{2})^2}}=\dfrac{\sqrt{2025}}{4426}.\]
	Do đó $((SBC),(SCD))\approx 63^{\circ}$.
	}
	\end{ex}

	\begin{ex}%[2D6C1-4]
		Kết quả một cuộc khảo sát các vụ tai nạn giao thông ô tô về mối quan hệ giữa việc thắt dây an toàn của người lái xe khi xảy ra tai nạn giao thông và nguy cơ tử vong của người lái xe khi xảy ra tai nạn giao thông cho thấy:
		\begin{itemize}
		\item Tỉ lệ người lái xe tử vong khi xảy ra tai nạn giao thông là $0{,4}\%$.
		\item Tỉ lệ người lái xe không thắt dây an toàn giao thông khi xảy ra tai nạn giao thông là $28 \%$.
		\item Tỉ lệ người lái xe tử vong khi xảy ra tai nạn giao thông trong trường hợp không thắt dây an toàn là $0{,}3 \%$.
		\end{itemize}
		Hỏi theo kết quả khảo sát trên, việc thắt dây an toàn của người lái xe ô tô sẽ làm giảm khả năng tử vong là bao nhiêu lần? (làm tròn đến hàng phần mười).
		\shortans{$7{,}7$}
		\loigiai{
		Chọn ngẫu nhiên một một vụ tai nạn giao thông của cuộc khảo sát trên. Xét các biến cố:\\
		$A$: "Người lái xe đó tử vong khi xảy ra tai nạn giao thông."\\
		$B$: "Người lái xe đó không thắt dây an toàn khi xảy ra tai nạn giao thông."\\
		Ta có $P(A)= 0{,4}\%$; $P(B)= 28\%$; $P(A\cap B)=0{,}3\%$.\\
		Xác suất người lái xe đó tử vong khi xảy ra tai nạn giao thông trong trường hợp không thắt dây an toàn là
		\[P(A|B)=\dfrac{P(A\cap B)}{P(B)}=\dfrac{3}{280}.\]
		Xác suất người lái xe đó có thắt dây an toàn giao thông là $P(\overline{B})=72\%$.\\
		Xác suất người lái xe đó tử vong khi xảy ra tai nạn giao thông trong trường hợp có thắt dây an toàn là
		\[P(A|\overline{B})=\dfrac{P(A\cap \overline{B})}{P(\overline{B})}=\dfrac{P(A)-P(A\cap B)}{P(\overline{B}}=\dfrac{1}{720}.\]
		Ta có
		\[\dfrac{P(A|B)}{P(A|\overline{B})}=\dfrac{54}{7}\approx7{,}7.\]
		Vậy theo khảo sát trên, việc thắt dây an toàn của người lái xe ô tô sẽ làm giảm khả năng tử vong khoảng $7{,}7$ lần.
		}
		\end{ex}
\TL
		\begin{ex}%[2H5V2-4]
			Trong không gian với hệ trục tọa độ $O x y z$, cho điểm $M(3 ; 3 ;-2)$ và hai đường thẳng $d_1\colon \dfrac{x-1}{1}=\dfrac{y-2}{3}=\dfrac{z}{1} ;$ $ d_2\colon \dfrac{x+1}{-1}=\dfrac{y-1}{2}=\dfrac{z-2}{4}$. Viết phương trình đường thẳng $d$ đi qua $M$ vuông góc $d_1,$ $ d_2$.
			% \shortans{$3$}
			\loigiai{
			Ta có $d_1$ có véc-tơ chỉ phương là $\overrightarrow{u_1}=(1;3;1)$, $d_2$ có véc-tơ chỉ phương là $\overrightarrow{u_2}=(-1;2;4)$.\\
			Đường thẳng $d$ đi qua $M$ vuông góc với $d_1$, $d_2$ có véc-tơ chỉ phương là $\overrightarrow{u}=\dfrac15[\overrightarrow{u_1}, \overrightarrow{u_2}] = \left(2; -1; 1\right)$ nên có phương trình tham số là $\heva{&x=3+2t\\&y=3-t\\&z=-2+t}$.
			}
			\end{ex}

\begin{ex}%[2D6C2-4]
Tỉ lệ người dân đã tiêm vắc xin phòng bệnh A ở một địa phương là $65\%$. Trong số những  người đã tiêm phòng, tỉ lệ mắc bệnh A là $5\%$ còn trong số những người chưa tiêm, tỉ lệ mắc bệnh A là $17\%$. Gặp ngẫu nhiên một người ở địa phương đó. Biết rằng người đó mắc bệnh X. Khi đó xác suất người đó không tiêm vắc xin phòng bệnh X là bao nhiêu?
% \shortans{$65$}
\loigiai{
Gọi $A$ là biến cố \lq\lq người đó mắc bệnh $X$\rq\rq\,và $B$ là biến cố \lq\lq Gặp được người đã tiêm vắc xin phòng bệnh X\rq\rq.\\
Theo công thức xác suất toàn phần, ta có
\begin{eqnarray*}
\mathrm{P}(A) & = &\mathrm{P}(B) \cdot \mathrm{P}(A \mid B)+\mathrm{P}(\overline{B}) \cdot \mathrm{P}(A \mid \overline{B}) \\
& = &0,65 \cdot 0{,}05+0{,}35 \cdot 0{,}17=0{,}092.
\end{eqnarray*}
Suy ra
\begin{eqnarray*}
\mathrm{P}(\overline{B} \mid A) & =& \dfrac{\mathrm{P}(A \overline{B})}{\mathrm{P}(A)}=\dfrac{\mathrm{P}(\overline{B}) \mathrm{P}(A \mid \overline{B})}{\mathrm{P}(A)} \\
& =& \dfrac{0,35 \cdot 0{,}17}{0{,}092}=\dfrac{119}{184}.
\end{eqnarray*}
% Khi đó $a=119$ và $b=184$, suy ra $b-a=65$.
}
\end{ex}

\begin{ex}%[2H5C1-7]
Trong không gian $Oxyz$, cho hai điểm $A(2;1;0)$, $B(1;2;0)$ và điểm $M$ di động trên tia $Oz$. Gọi $H$, $K$ lần lượt là hình chiếu vuông góc của $A$ lên $OB$ và $MB$. Đường thẳng $HK$ cắt trục $Oz$ tại điểm $N$. Khi thể tích khối tứ diện $ABMN$ nhỏ nhất thì mặt phẳng $(AHK)$ có dạng $ax+by+cz-4=0$. Giá trị của $a+b+c$ bằng
\shortans{$1$}
\loigiai{
\immini
{Ta có $A(2;1;0)$, $B(1;2;0)$ $\Rightarrow A,~ B \in (Oxy)$.\\
Có $\heva{&AK\perp MB\\&AH\perp OB;~AH\perp OM \Rightarrow AH \perp (OBM) \Rightarrow AH \perp MB}$ mà $AK\perp MB$ nên
$ MB \perp (AHK)$.\\
Gọi $M(0;0;m)$, $(m>0)$ thuộc tia $Oz$ khi đó $\overrightarrow{MB}=(1;2;-m)$\\
$\Rightarrow (AHK)\colon 1(x-2)+2(y-1)-mz=0$.\\
$\Rightarrow N=HK \cap Oz =(AHK) \cap Oz \Rightarrow N\left(0;0;-\dfrac{4}{m}\right)$.\\
Ta có
\allowdisplaybreaks
$\begin{aligned}[t]
V_{ABMN}&=V_{M.OAB}+V_{N.OAB}=\dfrac{1}{3}S_{OAB}\cdot OM +\dfrac{1}{3}S_{OAB}\cdot ON\\
&=\dfrac{1}{3}S_{OAB}(OM+ON)=\dfrac{1}{3}S_{OAB}\cdot MN
\end{aligned}$\\
Vì $S_{OAB}$ không đổi nên $V_{ABMN}$ nhỏ nhất khi và chỉ khi $MN$ nhỏ nhất.\\
Ta có $MN=m+\dfrac{4}{m}\geq 2\sqrt{m\cdot \dfrac{4}{m}}=4$.\\
Dấu bằng xảy ra khi $m=\dfrac{4}{m}\Rightarrow m=2$.\\
Vậy $(AHK)\colon x+2y-2z-4=0 \Rightarrow a+b+c=1+2-2=1$.}
{\begin{tikzpicture}[scale=0.7, font=\footnotesize, line join=round, line cap=round,>=stealth]
\def\xmin{-2};\def\ymin{-3};\def\xmax{6};\def\ymax{5};
\coordinate (O) at (0,0);
\coordinate (B) at (2,-1);
\coordinate (A) at (3,0);
\coordinate (M) at (0,2);
\coordinate (K) at ($(M)+2/5*(B)-2/5*(M)$);
\coordinate (H) at ($(O)+1/4*(B)-1/4*(O)$);
\coordinate (N) at (intersection of M--O and K--H);
\draw (A)--(B)--(O)--(M)--cycle (M)--(B)--(N) (A)--(K)--(N)--(O);
\draw [dashed] (O)--(A)--(H) (A)--(N);
\draw[black] pic[draw, angle radius=2mm, angle eccentricity=1.5]{right angle=B--K--A};
\draw[black] pic[draw, angle radius=2mm, angle eccentricity=1.5]{right angle=B--H--A};
\foreach \t/\g in {O/180,B/-90,A/0,M/70,K/90,H/200,N/-90}{\draw[fill=white] (\t) circle (1pt) node[shift={(\g:7pt)},font=\scriptsize]{$\t$};}
\end{tikzpicture}}
}
\end{ex}
\Closesolutionfile{ans}


\Closesolutionfile{ansbook}
