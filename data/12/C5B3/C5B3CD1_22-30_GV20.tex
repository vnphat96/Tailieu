%PHẦN III. Câu trắc nghiệm đúng sai. Trong mỗi ý A), B), C), D) ở mỗi câu, thí sinh chọn đúng hoặc sai.
\TNTF
\Opensolutionfile{ans}[ans/ans-C5B3CD1_22-30-DS]
%Câu55
\begin{ex}%[2H5N3-3]
Trong KG $Oxyz$, cho mặt cầu $(S)$ có tâm $I(0;-2;1)$, bán kính bằng $ 2 $. Các mệnh đề sau đây đúng hay sai?
\choiceTF
{Phương trình của mặt cầu $(S)$ là $x^2+(y+2)^2+(z-1)^2=2$}
{Phương trình của mặt cầu $(S)$ là $x^2+(y-2)^2+(z+1)^2=2$}
{Phương trình của mặt cầu $(S)$ là $x^2+(y-2)^2+(z+1)^2=4$}
{\True Phương trình của mặt cầu $(S)$ là $x^2+(y+2)^2+(z-1)^2=4$}
\loigiai{
	Vì phương trình mặt cầu tâm $I(a ; b ; c)$ và bán kính bằng $R$ là $$(x-a)^2+(y-b)^2+(z-c)^2=R^2.$$
	Vậy phương trình mặt cầu $(S)$ có tâm $I(0 ;-2 ; 1)$ và bán kính bằng $ 2 $ là $$x^2+(y+2)^2+(z-1)^2=4.$$
	\begin{itemchoice}
		\itemch Sai.
		\itemch Sai.
		\itemch Sai.
		\itemch Đúng.	
\end{itemchoice}
}
\end{ex}
%câu 56
\begin{ex}%[2H5H3-3]
Trong KG $Oxyz$, cho hai điểm $I(1;1;1)$ và $A(1;2;3)$. Gọi $(S)$ là mặt cầu tâm $ I $ và đi qua điểm $ A $. Các mệnh đề sau đây đúng hay sai?
\choiceTF[t]
{Phương trình mặt cầu $ (S)$ là $(x+1)^{2}+(y+1)^{2}+(z+1)^{2}=5$}
{Phương trình mặt cầu $ (S)$ là $(x+1)^{2}+(y+1)^{2}+(z+1)^{2}=29$}
{\True Phương trình mặt cầu $ (S)$ là $(x-1)^{2}+(y-1)^{2}+(z-1)^{2}=5$}
{Phương trình mặt cầu $ (S) $ là $(x-1)^2+(y-1)^2+(z-1)^2=25$}
\loigiai{Vì $R=I A=\sqrt{(1-1)^2+(2-1)^2+(3-1)^2}=\sqrt5$.\\
	Vậy phương trình mặt cầu tâm $I$ và đi qua điểm $A$ có phương trình là
	
	$\left(x-x_{I}\right)^2+\left(y-y_{I}\right)^2+\left(z-z_{I}\right)^2=R^2 \Rightarrow(x-1)^2+(y-1)^2+(z-1)^2=5$.
\begin{itemchoice}
\itemch Sai. 
\itemch Sai.
\itemch Đúng.
\itemch Sai.
\end{itemchoice}}
\end{ex}
%Câu 57
\begin{ex}%[2H5H3-3]
Trong KG $Oxyz$, cho hai điểm $A(2;-1;-3)$; $B(0;3;-1)$. Gọi $ (S) $ là mặt cầu đường kính $ AB $. Các mệnh đề sau đây đúng hay sai?
\choiceTF[t]
{\True Phương trình của mặt cầu $ (S) $ là $(x-1)^2+(y-1)^2+(z+2)^2=6$}
{Phương trình của mặt cầu $ (S) $ là $(x-1)^2+(y-1)^2+(z+2)^2=24$}
{Phương trình của mặt cầu $ (S) $ là $(x+1)^2+(y+1)^2+(z-2)^2=24$}
{\True Phương trình của mặt cầu có tâm là trung điểm $AB$ và đi qua hai điểm $A$, $B$ là $(x-1)^2+(y-1)^2+(z+2)^2=6$}
\loigiai{
	Vì tâm $I$ của mặt cầu $ (S) $ là trung điểm của $A B$ suy ra	$I(1 ; 1 ;-2)$ và bán kính $R=\dfrac{1}{2}AB=\dfrac{1}{2} \sqrt{4+16+4}=\dfrac{1}{2} \sqrt{24}$.\\
	Vậy phương trình của mặt cầu đường kính $ AB $ là	
	$$(x-1)^2+(y-1)^2+(z+2)^2=6.$$
	\begin{itemchoice}
	\itemch Đúng.
\itemch Sai.
\itemch Sai.
\itemch Đúng. Vì mặt cầu có tâm là trung điểm $AB$ và đi qua hai điểm $A$, $B$ là mặt cầu đường kính $ AB $.
\end{itemchoice}}
\end{ex}
%Câu 58
\begin{ex}%[2H5H3-2]
Gọi $(S)$ là mặt cầu đi qua bốn điểm $A(2;0;0)$, $B(1;3;0)$, $C(-1;0;3)$, $D(1;2;3)$. Các mệnh đề sau đây đúng hay sai?
\choiceTF[t]
{Mặt cầu $(S)$ có tọa độ tâm là $(1;-1;1)$}
{\True Mặt cầu $(S)$ có tọa độ tâm là $(0;1;1)$}
{Bán kính $R$ của mặt cầu $(S)$ là $R=6$}
{\True Bán kính $R$ của mặt cầu $(S)$ là $R=\sqrt{6}$}
\loigiai{
	Vì giả sử $I(a;b;c)$ là tâm mặt cầu đi qua bốn điểm $A$, $B$, $C$, $D$. Khi đó\\
\begin{eqnarray*}
& 				 & \heva{&AI^2 = BI^2\\&AI^2=CI^2\\&AI^2=DI^2} \\
&\Leftrightarrow & \heva{&(a-2)^2+b^2+c^2=(a-1)^2+(b-3)^2+c^2\\&(a-2)^2+b^2+c^2=(a+1)^2+b^2+(c-3)^2\\&(a-2)^2+b^2+c^2=(a-1)^2+(b-2)^2+(c-3)^2} \\
& \Leftrightarrow & \heva{&a-3b=-3\\&a-c=-1\\& a-2b-3c=-5} \\
& \Leftrightarrow & \heva{&a=0\\&b=1\\&c=1.}
\end{eqnarray*}
Vậy $I(0; 1; 1)$; bán kính $R=I A=\sqrt{2^2+1^2+1^2}=\sqrt6$.
	\begin{itemchoice}
		\itemch Sai. 
		\itemch Đúng.
		\itemch Sai.
		\itemch Đúng.
	\end{itemchoice}} 
\end{ex}
%Câu 59
\begin{ex}%[2H5V3-2]
Trong không gian với hệ trục tọa độ $Oxyz$, cho mặt cầu $(S)$ có tâm nằm trên mặt phẳng $(O x y)$ và đi qua ba điểm $A(1;2;-4)$, $B(1;-3;1)$, $C(2;2;3)$. Các mệnh đề sau đây đúng hay sai?
\choiceTF[t]
{Tọa độ tâm $(I)$ của mặt cầu $(S)$ là $(2;-1;0)$}
{\True Tọa độ tâm $(I)$ của mặt cầu $(S)$ là $(-2;1;0)$}
{\True Bán kính $R$ của mặt cầu $(S)$ là $R=\sqrt{26}$}
{Bán kính $R$ của mặt cầu $(S)$ là $R=26$}
\loigiai{
	Vì giả sử tâm $I(a;b;c)$ và phương trình mặt cầu $(S)$ là $$x^2+y^2+z^2-2ax-2by-2cz+d=0.$$
	Do $I \in (Oxy) \Leftrightarrow c=0 \Leftrightarrow(S)\colon x^2+y^2+z^2-2ax-2by+d=0$.\\
	Ta có $\heva{&A\in (S)\\&B\in (S)\\& C \in (S)} \Leftrightarrow \heva{&2a+4b-d=21 \\&2a-6b-d=11 \\& 4a+4b-d=17}\Leftrightarrow\heva{&a=-2 \\& b=1 \\ &d=-21.}$\\
	Vậy $I(-2;1;0)$ và $R=\sqrt{26}$.
	\begin{itemchoice}
	\itemch	Sai.
	\itemch	Đúng.
	\itemch	Đúng.
	\itemch	Sai.
\end{itemchoice}}
\end{ex}
%Câu 60
\begin{ex}%[2H5H3-3]
Trong không gian với hệ trục tọa độ $Oxyz$, cho điểm $A(1;1;2)$, $B(3;2;-3)$. Mặt cầu $(S)$ có tâm $I$ thuộc $Ox$ và đi qua hai điểm $A$, $B$. Các mệnh đề sau đây đúng hay sai?
\choiceTF[t]
{\True Tọa độ tâm $(I)$ của mặt cầu $(S)$ là $I(4;0;0)$}
{Bán kính $R$ của mặt cầu $(S)$ là $R=14$}
{\True Mặt cầu $(S)$ có phương trình $x^2+y^2+z^2-8x+2=0$}
{Mặt cầu $(S)$ có phương trình $x^2+y^2+z^2-8x-2=0$}
\loigiai{
	Vì giả sử $I(a;0;0) \in Ox \Rightarrow \overrightarrow{IA}(1-a;1;2)$; $ \overrightarrow{IB}(3-a;2;-3)$.\\
	Do $(S)$ đi qua hai điểm $A$, $B$ nên
	\begin{eqnarray*}
		& & IA=IB \\
		& \Leftrightarrow &  \sqrt{(1-a)^2+5}=\sqrt{(3-a)^2+13} \\
		& \Leftrightarrow & 4 a=16 \\
		& \Leftrightarrow & a=4.
	\end{eqnarray*} 
	Suy ra mặt cầu $(S)$ có tâm $I(4;0;0)$, bán kính $R=IA=\sqrt{14}$.\\
	Vậy	$(S)$ là $(x-4)^2+y^2+z^2=14 \Leftrightarrow x^2+y^2+z^2-8x+2=0$.
	\begin{itemchoice}
	\itemch Đúng. 
	\itemch Sai.
	\itemch Đúng.
	\itemch Sai.
\end{itemchoice}
}
\end{ex}
%Câu 61
\begin{ex}%[2H5V3-3]
Trong KG $Oxyz$, mặt cầu $(S)$ đi qua điểm $A(1;-1;4)$ và tiếp xúc với các mặt phẳng tọa độ. Các mệnh đề sau đây đúng hay sai?
\choiceTF[t]
{Mặt cầu $(S)$ có phương trình $(x-3)^2+(y+3)^2+(z+3)^2=16$}
{\True Mặt cầu $(S)$ có phương trình $(x-3)^2+(y+3)^2+(z-3)^2=9$}
{Mặt cầu $(S)$ có phương trình $(x+3)^2+(y-3)^2+(z+3)^2=36$}
{Mặt cầu $(S)$ có phương trình $(x+3)^2+(y-3)^2+(z-3)^2=49$}
\loigiai{
	Vì giả sử $I(a;b;c)$ là tâm của mặt cầu $(S)$. Mặt cầu $(S)$ tiếp xúc với các mặt phẳng tọa độ $\mathrm{d}\Big(I,(Oxy)\Big)=\mathrm{d}\Big(I,(Oyz)\Big)=\mathrm{d}\Big(I,(Oxz)\Big) \Leftrightarrow|a|=|b|=|c|=R$.\\
	Mặt cầu $(S)$ đi qua $A(1;-1;4)$. Ta có
	\begin{eqnarray*}
	& \Rightarrow & \heva{& IA=R \\& a>0;c>0;b<0}\\
	& \Leftrightarrow & \heva{& IA^2=R^2\\& a>0; c>0; b < 0} \\
	& \Leftrightarrow & \heva{& (a-1)^2+(b+1)^2+(c-4)^2=R^2\\& a=c=-b=R>0} \\
	& \Leftrightarrow & \heva{& (a-1)^2+(-a+1)^2+(a-4)^2=a^2\\& a=c=-b=R>0 } \\
	& \Leftrightarrow & \heva{& 2a^2-12a+18=0 \\& a=c=-b=R>0} \\
	& \Leftrightarrow & \heva{& a=c=3\\& b=-3 \\& R=3.}
	\end{eqnarray*}
	Vậy $(S)\colon (x-3)^2+(y+3)^2+(z-3)^2=9$.
	\begin{itemchoice}
\itemch Sai.
\itemch Đúng.
\itemch Sai.
\itemch Sai.
\end{itemchoice}}
\end{ex}
\Closesolutionfile{ans}
\indapan{3}{ans/ans-C5B3CD1_22-30-DS}

%PHẦN III. Câu trắc nghiệm trả lời ngắn. Mỗi câu hỏi thí sinh chỉ trả lời đáp án.
\TNSA
\Opensolutionfile{ans}[ans/ans-C5B3CD1_22-30-KQ]
%CÂU 62
\begin{ex}%[2H5N3-3]
Trong KG $Oxyz$, mặt cầu $(S)$ có tâm $I(0; 1;-2)$ và bán kính bằng $ 3 $. Phương trình của $(S)$ có dạng $ x^2+y^2+z^2-2ax-2by-2cz+d=0 $. Tìm $ d $.
\shortans{$ -4 $}
\loigiai{
	Phương trình mặt cầu tâm $I(a; b; c)$ và bán kính bằng $R$ là $(x-a)^2+(y-b)^2+(z-c)^2=R^2$.\\
	Ta có mặt cầu tâm $I(0;1;-2)$, bán kính bằng $ 3 $ có phương trình là $$x^2+(y-1)^2+(z+2)^2=9\Leftrightarrow x^2+y^2+z^2-2y+4z-4=0.$$}
\end{ex}
%Câu 63
\begin{ex}%[2H5N3-2]
	Trong KG $Oxyz$, cho mặt cầu có tâm $I(1;-4;3)$ và đi qua điểm $A(5;-3;2)$. Tính bán kính của mặt cầu đã cho (làm tròn kết quả đến hàng phần nghìn).
	\shortans{$ 4{,}24 $}
\loigiai{Mặt cầu tâm $I(1;-4;3)$ và đi qua điểm $A(5;-3;2)$ nên bán kính $R=IA=3\sqrt{2}\approx 4{,}24$.}
\end{ex}
%Câu 64
\begin{ex}%[2H5H3-3]
Trong KG $Oxyz$, cho hai điểm $A(1;1;1)$ và $B(1;-1;3)$. Phương trình mặt cầu có đường kính $AB$ có dạng $ x^2+y^2+z^2-2ax-2by-2cz+d=0$. Tính tổng $S= a+b+c+d$.
\shortans{$ 6 $}
\loigiai{Gọi $I$ là tâm của mặt cầu đường kính $AB$.\\
		Khi đó $ I $ là trung điểm của đoạn $ AB \Rightarrow I(1;0;2)$.\\
		Bán kính của mặt cầu là $R=\dfrac{1}{2}AB=\dfrac{1}{2}\sqrt{(1-1)^2+(-1-1)^2+(3-1)^2}=\sqrt2$.\\	
	Suy ra phương trình mặt cầu là $(x-1)^2+y^2+(z-2)^2=2 \Leftrightarrow x^2+y^2+z^2-2x-4z+3=0$.\\
Vậy $S= a+b+c+d=1+2+3=6 $.}
\end{ex}
%Câu 65
\begin{ex}%[2H5H3-2]
Trong KG $Oxyz$, cho $A(-1; 0; 0)$, $B(0; 0; 2)$, $C(0;-3; 0)$. Tính bán kính mặt cầu ngoại tiếp tứ diện $OABC$ (làm tròn đến hàng phần nghìn).
\shortans{$ 1{,}87 $}
\loigiai{Gọi $(S)$ là mặt cầu ngoại tiếp tứ diện $OABC$.\\	
	Phương trình mặt cầu $(S)$ có dạng $x^2+y^2+z^2-2ax-2by-2cz+d=0$.\\	
	Vì $O$, $ A$, $B$, $C$ thuộc $(S)$ nên ta có
	$ \heva{& d=0 \\& 1+2a+d = 0\\& 4-4c+d=0\\& 9+6b+d=0}\Leftrightarrow \heva{& a=-\dfrac{1}{2}\\ & b=-\dfrac{3}{2}\\& c=1 \\& d=0.} $\\
	Vậy bán kính mặt cầu $(S)$ là $R=\sqrt{a^2+b^2+c^2-d}=\sqrt{\dfrac{1}{4}+\dfrac{9}{4}+1}=\dfrac{\sqrt{14}}{2}\approx 1{,}87$.
}
\end{ex}
%Câu 66
\begin{ex}%[2H5V3-2]
Trong KG $Oxyz$, gọi $I(a;b;c)$ là tâm mặt cầu đi qua điểm $A(1;-1;4)$ và tiếp xúc với tất cả các mặt phẳng tọa độ. Tính $P=a-b+c$.
\shortans{$ 9 $}
\loigiai{Vì mặt cầu tâm $I$ tiếp xúc với các mặt phẳng tọa độ nên\\
	$\mathrm{d}\Big(I,(Oxy)\Big)=\mathrm{d}\Big(I,(Oyz)\Big)=\mathrm{d}\Big(I,(Oxz)\Big) \Leftrightarrow |a|=|b|=|c| \Leftrightarrow\hoac{&  a=b=c \\& a=b=-c \\& a=-b=c \\& a=-b=-c.}$\\	
	Nhận thấy chỉ có trường hợp $a=-b=c$ thì phương trình $AI=\mathrm{d}\Big(I,(Oxy)\Big)$ có nghiệm, các trường hợp còn lại vô nghiệm.\\	
	Thật vậy, với $a=-b=c$ thì $I(a;-a; a)$.\\
	$AI=\mathrm{d}\Big(I,(Oxy)\Big) \Leftrightarrow \left(a-1\right)^2+\left(a-1\right)^2+\left(a-4\right)^2=a^2 \Leftrightarrow a^2-6a+9=0 \Leftrightarrow a=3$.\\
	Khi đó $P=a-b+c=9$.
}
\end{ex}
%Câu 67
\begin{ex}%[2H5V3-2]
Trong không gian $O x y z$, tìm giá trị dương của $m$ (làm tròn đến hàng phần nghìn) sao cho mặt phẳng $(Oxy)$ tiếp xúc với mặt cầu $\left(x-3\right)^2+y^2\left(z-2\right)^2=m^2+1$.
\shortans{$ 1{,}73 $}
\loigiai{Mặt cầu $(S)\colon \left(x-3\right)^2+y^2\left(z-2\right)^2=m^2+1$ có tâm $I(3;0;2)$, bán kính $R=\sqrt{m^2+1}$.\\	
	$(S)$ tiếp xúc với $(Oxy) \Leftrightarrow \mathrm{d}\left(I,(Oxy)\right)=R \Leftrightarrow 2=\sqrt{m^2+1} \Leftrightarrow m^2=3 \Leftrightarrow m=\sqrt3 \approx 1{,}73$ (do $m>0$).}

\end{ex}
%Câu 68
\begin{ex}%[2H5V3-2]
Trong không gian với hệ trục tọa độ $O x y z$, cho ba điểm $A(1 ; 2 ;-4), B(1 ;-3 ; 1), C(2 ; 2 ; 3)$. Tính đường kính của mặt cầu $(S)$ đi qua ba điểm trên và có tâm nằm trên mặt phẳng $(O x y)$ (Làm tròn kết quả đến hàng phần chục).
\shortans{$ 10{,}2 $}
\loigiai{
Gọi tâm mặt cầu là $I(x;y;0)$. Ta có
\begin{eqnarray*}
& 				  & \heva{&IA=IB \\ &IA=IC} \\
& \Leftrightarrow & \heva{&\sqrt{(x-1)^2+(y-2)^2+4^2}=\sqrt{(x-1)^2+(y+3)^2+1^2}\\&	\sqrt{(x-1)^2+(y-2)^2+4^2}=\sqrt{(x-2)^2+(y-2)^2+3^2}}\\
& \Leftrightarrow & \heva{&(y-2)^2+4^2=(y+3)^2+1^2 \\&	x^2-2 x+1+16=x^2-4x+4+9}\\
& \Leftrightarrow & \heva{&10y=10 \\& 2x=-4} \Leftrightarrow \heva{&x=-2\\&y=1.}
\end{eqnarray*}
$\Rightarrow l=2R=2\sqrt{(-3)^2+(-1)^2+4^2}=2\sqrt{26}\approx 10{,}2$.
}
\end{ex}
%Câu 69
\begin{ex}%[2H5V3-2]
Trong không gian $O x y z$, gọi $(S)$ là mặt cầu đi qua điểm $D(0 ; 1 ; 2)$ và tiếp xúc với các trục $O x, O y, O z$ tại các điểm $A(a ; 0 ; 0)$, $B(0 ; b ; 0)$, $C(0 ; 0 ; c)$ trong đó $a, b, c \in \mathbb{R} \backslash\{0 ; 1\}$. Tính bán kính của $(S)$ (làm tròn kết quả đến hàng phần chục).
\shortans{$7{,}1$}
\loigiai{
Gọi $I$ là tâm của mặt cầu $(S)$.\\
Vì $(S)$ tiếp xúc với các trục $O x, O y, O z$ tại các điểm $A(a ; 0 ; 0)$, $B(0 ; b ; 0)$, $C(0 ; 0 ; c)$ nên ta có $I A \perp Ox, I B \perp Oy, I C \perp Oz$ hay $A, B, C$ tương ứng là hình chiếu của $I$ trên $Ox, Oy, Oz \Rightarrow I(a;b;c)$.\\
$\Rightarrow$ Mặt cầu $(S)$ có phương trình $x^2+y^2+z^2-2 ax-2 by-2cz+d=0$ với $a^2+b^2+c^2-d>0$.\\
Vì $(S)$ đi qua $A$, $B$, $C$, $D$ nên ta có $ \heva{&a^2=b^2=c^2=d\\&5-2b-4c+d=0.} $\\
Vì $a, b, c \in \mathbb{R} \backslash\{0 ; 1\}$ nên $0<d \neq 1$.\\
Mặt khác, từ $(1) \Rightarrow R=\sqrt{a^2+b^2+c^2-d}=\sqrt{2d}$.
\begin{itemize}
	\item TH1. Từ $(1) \Rightarrow b=c=\sqrt d$. Thay vào $(*)$ ta có $ 5-6 \sqrt d +d=0 \Leftrightarrow d=25$ (nhận).
$\Rightarrow R=\sqrt{2.25}=5 \sqrt{2}$.
	\item TH2. Từ $(1) \Rightarrow b=c=-\sqrt d$. Thay vào $(*)$ ta có $ 5+6 \sqrt d+d=0$ (vô nghiệm).
	\item TH3. Từ $(1) \Rightarrow b=\sqrt{d}, c=-\sqrt d$. Thay vào $(*)$ ta có $ 5+2 \sqrt d +d=0$ (vô nghiệm).
	\item TH4. Từ $(1) \Rightarrow b=-\sqrt{d}, c=\sqrt d $. Thay vào $(*)$ ta có $ 5-2 \sqrt d +d=0$ (vô nghiệm).
\end{itemize}
Vậy mặt cầu $(S)$ có bán kính $R=5 \sqrt{2} \approx 7{,}1$.
}
\end{ex}
%Câu 70
\begin{ex}%[2H5V3-3]
Trong không gian với hệ trục tọa độ $Oxyz$, cho ba điểm $A(1; 0; 0), C(0; 0; 3), B(0; 2; 0)$. Tập hợp các điểm $M$ thỏa mãn $MA^2=MB^2+MC^2$ là mặt cầu có bán kính là bao nhiêu? (làm tròn kết quả đến hàng phần nghìn).
\shortans{$ 1{,}41 $}
\loigiai{
Giả sử $M(x;y;z)$.\\
Ta có $MA^2=(x-1)^2+y^2+z^2$; $MB^2=x^2+(y-2)^2+z^2$; $MC^2=x^2+y^2+(z-3)^2$.
\begin{eqnarray*}
& & MA^2 = MB^2 + MC^2\\
& \Leftrightarrow & (x-1)^2+y^2+z^2=x^2+(y-2)^2+z^2+x^2+y^2+(z-3)^2\\
& \Leftrightarrow & -2x+1=(y-2)^2+x^2+(z-3)^2 \\
& \Leftrightarrow & (x+1)^2+(y-2)^2+(z-3)^2=2.
\end{eqnarray*}
Vậy tập hợp các điểm $M$ thỏa mãn $MA^2=MB^2+MC^2$ là mặt cầu có bán kính là $R=\sqrt 2 \approx 1{,}41$.
}
\end{ex}
%Câu 71
\begin{ex}%[2H5H3-2]
Trong không gian với hệ trục tọa độ $Oxyz$, xét mặt cầu $(S)$ có phương trình dạng $x^2+y^2+z^2-4x+2y-2az+10a=0$. Có tất cả bao nhiêu giá trị thực của $a$ để $(S)$ có chu vi đường tròn lớn bằng $8 \pi$.
\shortans{$ 2 $}
\loigiai{
Đường tròn lớn có chu vi bằng $8 \pi$ nên bán kính của $(S)$ là $\dfrac{8\pi}{2 \pi}=4$.\\
Từ phương trình của $(S)$ suy ra bán kính của $(S)$ là $\sqrt{2^2+1^2+a^2-10 a}$.\\
Do đó $\sqrt{2^2+1^2+a^2-10a}=4 \Leftrightarrow \hoac{&a=-1\\&a=11.}$
}
\end{ex}
%Câu 72
\begin{ex}%[2H5C3-2]
Trong KG $Oxyz$, cho mặt cầu $(S)\colon (x-1)^2+(y-2)^2+(z-3)^2=25$ và hình nón $(H)$ có đỉnh $A(3; 2;-2)$ và nhận $AI$ làm trục đối xứng với $I$ là tâm mặt cầu. Một đường sinh của hình nón $(H)$ cắt mặt cầu tại $M, N$ sao cho $AM=3AN$. Tìm bán kính của mặt cầu đồng tâm với mặt cầu $(S)$ và tiếp xúc với các đường sinh của hình nón $(H)$ (làm tròn kết quả đến hàng phần nghìn).
\shortans{$ 4{,}86 $}
\loigiai{
	\begin{center}
		\begin{tikzpicture}[>=stealth,line join=round,line cap=round,font=\footnotesize,scale=.61]
			\path
			(0,0) coordinate (I)
			($(I)+(0,5)$) coordinate (A)
			($(I)-(1.2,0)$) coordinate (x)
			($(I)+(1.2,0)$) coordinate (y)
			(-.6,-.4) coordinate (z)
			(-.3,2.3) coordinate (N)
			(-.7,-1.3) coordinate (M)
			(-.45,.95) coordinate (K)
			%($(N)!.5!(M)$) coordinate (H)
			%($(N)!.7!(M)$) coordinate (A)
			;
			\draw (I) circle (4cm);
			\draw[dashed] (I) ellipse (1.2 and 0.5);
			\draw (4,0) arc (0:-180:4 and 1);
			\draw[dashed] (4,0) arc (0:180:4 and 1);
			\draw[dashed] (x)--(-.5,2.9) (y)--(.5,2.9) (N)--(M) (z)--(I)--(A) (I)--(K);
			\draw (N)--(A) (A)--(-.5,2.9) (A)--(.5,2.9);
			\foreach \p/\g in {I/0, N/135, M/-75, A/135, K/-130}
			\draw[fill=black] (\p) circle (1pt) node[shift=(\g:3mm)] {$\p$};
		
		\end{tikzpicture}
	\end{center}
Gọi hình chiếu vuông góc của $I$ trên $M N$ là $K$.\\
Dễ thấy $AN=NK=\dfrac{1}{3} AM$, mặt cầu $(S)$ có tâm $I(1; 2; 3)$ và bán kính $R=5$.\\	
Ta có $AM \cdot AN=AI^2-R^2=4 \Rightarrow AN^2=\dfrac{4}{3} \Rightarrow KN=AN=\dfrac{2 \sqrt{3}}{3}$\\
$\Rightarrow IK=\sqrt{IN^2-KN^2}=\dfrac{\sqrt{213}}{3}$.\\	
Nhận thấy mặt cầu đồng tâm với mặt cầu $(S)$ và tiếp xúc với các đường sinh của hình nón $(H)$ chính là mặt cầu tâm $I(1; 2; 3)$ có bán kính $IK=\dfrac{\sqrt{213}}{3} \approx 4{,}86$.\\
}
\end{ex}
%Câu 73
\begin{ex}%[2H5C3-2]
Trong KG $Oxyz$, cho bốn điểm $A(0;-1;2)$, $B(2;-3; 0)$, $C(-2; 1; 1)$, $D(0;-1;3)$. Gọi $(L)$ là tập hợp tất cả các điểm $M$ trong không gian thỏa mãn đẳng thức $\overrightarrow{MA} \cdot \overrightarrow{MB}=\overrightarrow{MC} \cdot \overrightarrow{MD}=1$. Biết rằng $(L)$ là một đường tròn, đường tròn đó có bán kính $r$ bằng bao nhiêu? (Làm tròn kết quả đến hàng phần nghìn).
\shortans{$ 1{,}66 $}
\loigiai{
Gọi $M(x; y; z)$ là tập hợp các điểm thỏa mãn yêu cầu bài toán. Ta có\\
$\overrightarrow{AM}=(x; y+1; z-2)$, $\overrightarrow{BM}=(x-2; y+3; z)$, $ \overrightarrow{CM}=(x+2; y-1; z-1)$, $\overrightarrow{DM}=(x; y+1; z-3)$.\\
Từ giả thiết 
\begin{eqnarray*}
&                 & \overrightarrow{MA}\cdot\overrightarrow{MB}=\overrightarrow{MC}\cdot\overrightarrow{MD}=1 \\
& \Leftrightarrow & \heva{&\overrightarrow{MA} \cdot \overrightarrow{MB}=1 \\ & \overrightarrow{MC} \cdot \overrightarrow{MD}=1}\\
& \Leftrightarrow & \heva{&x(x-2)+(y+1)(y+3)+z(z-2)=1 \\& x(x+2)+(y+1)(y-1)+(z-1)(z-3)=1}\\
& \Leftrightarrow & \heva{& x^2+y^2+z^2-2x+4y-2z+2=0 \\ & x^2+y^2+z^2+2x-4z+1=0.}
\end{eqnarray*}
Suy ra quỹ tích điểm $M$ là đường tròn giao tuyến của mặt cầu tâm $I_1 (1;-2; 1), R_1=2$ và mặt cầu tâm $I_2 (-1; 0; 2), R_2=2$.
\begin{center}
	\begin{tikzpicture}[scale=0.7, font=\footnotesize, line join=round, line cap=round, >=stealth]
		\path
		(0,0) coordinate (I_1)
		(2.2,0) coordinate (I_2)
		($ (I_1)!.5!(I_2)$) coordinate (K)
		($(I_2)-(2.2,0)$) coordinate (N)
		($(K)+(0,1.65)$) coordinate (M)
		;
		\fill[white]  (I_1)circle(3);
		\draw (I_1)circle(2) (I_2)circle(2)
		(I_1)--(I_2)--(M)--(I_1) (M)--(K) 
		;
		\foreach \x/\g in {I_1/250,I_2/-45,M/90,K/-90} \fill[black] (\x) circle (1pt)+(\g:0.3) node{$\x$};
	\end{tikzpicture}
\end{center}
Ta có $I_1I_2=\sqrt{5}$.\\
Dễ thấy $r=\sqrt{R_{1}^{2}-\left(\dfrac{I_1I_2}{2}\right)^{2}}=\sqrt{4-\dfrac{5}{4}}=\dfrac{\sqrt{11}}{2}$.}
\end{ex}

\Closesolutionfile{ans}
\indapan{6}{ans/ans-C5B3CD1_22-30-KQ}