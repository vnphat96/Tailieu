\part{véc-tơ và hệ trục tọa độ trong không gian}
\section{véc-tơ trong không gian}
\Opensolutionfile{ans}[ans/C5B3CD7-17-26]
\TN
\setcounter{ex}{19}
\begin{ex}%[2H5C3-2]
	Trong KG $Oxyz$, cho mặt cầu $(S)\colon x^2+y^2+z^2-2x-4y-4=0$ và hai điểm $A(4;2;4)$, $B(1;4;2)$. $MN$ là dây cung của mặt cầu thỏa mãn $\overrightarrow{MN}$ cùng hướng với $\overrightarrow{u}=(0;1;1)$ và $MN=4\sqrt{2}$. Tính giá trị lớn nhất của $\left|AM-BN\right|$.
	\choice
	{$\sqrt{41}$}
	{$4\sqrt{2}$}
	{\True $7$}
	{$\sqrt{17}$}
	\loigiai{
		\begin{center}
			\begin{tikzpicture}[scale=0.75, font=\footnotesize, line join=round, line cap=round, >=stealth]
				\def\R{2} % Bán kính
				\coordinate (I) at (0,0);
				\coordinate (x) at (\R,0);
				\draw[name path=DTO](I) circle (\R) (x) arc (0:-180: {\R} and {\R/2});
				\draw[dashed] (x) arc (0:180: {\R} and {\R/2});
				\coordinate (M) at ($({\R*cos(-30)},{\R*sin(-30)})$);
				\coordinate (N) at ($({\R*cos(-85)},{\R*sin(-85)})$);
				\coordinate (M') at ($({\R*cos(100)},{\R*sin(100)})$);
				\coordinate (N') at ($({\R*cos(155)},{\R*sin(155)})$);
				\coordinate (B) at ($(N)!0.25!(N')$);
				\coordinate (A') at ($(N)!3/2!(N')$);
				\coordinate (A) at ($(M)!3/2!(M')$);
				\path[name path=AI] (A)--(I);
				\path[name path=AB] (A)--(B);
				\path[name intersections={of= AI and DTO,by={I'}}];
				\path[name intersections={of= AB and DTO,by={B'}}];
				\draw[dashed](M')--(M)--(N)--(N')(A)--(I)--(B)--(M)--(I)(B)--(B');
				\draw(M')--(A)--(A')--(N')(A)--(B');
				\foreach \x/\g in {I/20,M/0,N/-90,B/-120,A'/-120,A/90} \fill[black](\x)circle(1pt) +(\g:.4)node[scale=1]{$\x$};
			\end{tikzpicture}
		\end{center}
		Tâm $I(1;2;0)$, bán kính $R=3$.\\
		Ta có $\overrightarrow{IA}=(3;0;4)\Rightarrow IA=5$, $\overrightarrow{IB}=(0;2;2)\Rightarrow IB=2\sqrt{2}$ nên điểm $A(4;2;4)$nằm ngoài mặt cầu $(S)$ và điểm $B(1;4;2)$ nằm trong mặt cầu $(S)$.\\
		Do $\overrightarrow{MN}$ cùng hướng với $\overrightarrow{u}=(0;1;1)$ suy ra $\overrightarrow{MN}=\left( 0;k;k \right),\,k>0$ do $MN=4\sqrt{2}$ suy ra $\overrightarrow{MN}=\left( 0;4;4 \right)$.\\
		Gọi $A'={T_{\overrightarrow{MN}}}(A)$, suy ra $A'=(4;6;8)$.\\
		Khi đó $AMNA'$ là hình bình hành nên $AM=A'N$\\
		Ta có $\left|AM-BN\right|=\left|A'N-BN\right|\le A'B$.\\
		Dấu \lq\lq =\rq\rq xảy ra khi $A'$, $N$, $B$ thẳng hàng $\Leftrightarrow N$ là giao điểm của mặt cầu với đường thẳng $A'B$. (Điểm $N$ luôn tồn tại).\\
		$\overrightarrow{A'B}=(-3;-2;-6)$ suy ra $A'B=\sqrt{(-3)^2+(-2)^2+(-6)^2}=7$.\\
		Vậy ${{\left|AM-BN\right|}_{\min}}=A'B=7$.}
\end{ex}
\begin{ex}%[2H5C3-3]
	Trong KG $Oxyz$, gọi điểm $M(a;b;c)$ (với $a$, $b$, $c$ là các phân số tối giản) thuộc mặt cầu $(S)\colon x^2+y^2+z^2-2x-4y-4z-7=0$ sao cho biểu thức $T=2a+3b+6c$ đạt giá trị lớn nhất. Khi đó giá trị biểu thức $P=2a-b+c$ bằng
	\choice
	{$\dfrac{12}{7}$}
	{$8$}
	{\True $6$}
	{$\dfrac{51}{7}$}
	\loigiai{
		$x^2+y^2+z^2-2x-4y-4z-7=0\Leftrightarrow (x-1)^2+(y-2)^2+(z-2)^2=16$.\\
		$M(a;b;c)\in (S)\Leftrightarrow (a-1)^2+(b-2)^2+(c-2)^2=16$.\\
		Ta có $\left| 2(a-1)+3(b-2)+6(c-2) \right|\le \sqrt{\left( {2^2}+3^2+6^2 \right)\cdot \left[ (a-1)^2+(b-2)^2+(c-2)^2 \right]}$.\\
		$\Leftrightarrow \left| 2a+3b+6c-20 \right|\le 28\Rightarrow 2a+3b+6c-20\le 28\Rightarrow 2a+3b+6c\le 48$.\\
		Dấu \lq\lq =\rq\rq xảy ra khi
		$\heva{&2a+3b+6c=48\\&\dfrac{a-1}{2}=\dfrac{b-2}{3}\\&\dfrac{a-1}{2}=\dfrac{c-2}{6}}\Leftrightarrow\heva{&2a+3b+6c=48\\&3a-2b=-1\\&3a-c=1}\Leftrightarrow\heva{&a=\dfrac{15}{7}\\&b=\dfrac{26}{7}\\&c=\dfrac{38}{7}.}$\\
		Vậy $P=2a-b+c=2\cdot\dfrac{15}{7}-\dfrac{26}{7}+\dfrac{38}{7}=6$.}
\end{ex}
\begin{ex}%[2H5C3-3]
	Cho $x$, $y$, $z$, $a$, $b$, $c$ là các số thực thay đổi thỏa mãn $(x+1)^2+(y+1)^2+(z-2)^2=1$ và $a+b+c=3.$ Tìm giá trị nhỏ nhất của $P=(x-a)^2+(y-b)^2+(z-c)^2$.
	\choice
	{$\sqrt{3}-1$}
	{$\sqrt{3}+1$}
	{\True $4-2\sqrt{3}$}
	{$4+2\sqrt{3}$}
	\loigiai{
		\immini{Gọi $M(x;y;z)\Rightarrow M$ thuộc mặt cầu $(S)$tâm $I(-1;-1;2)$ bán kính $R=1$.\\
			Gọi $H(a;b;c)\Rightarrow H$ thuộc mặt phẳng $(P)\colon x+y+z-3=0$\\
			Ta có $\mathrm{d}(I,(P))=\dfrac{\left| -1-1+2-3 \right|}{\sqrt{3}}=\sqrt{3}>R\Rightarrow (P)$và $(S)$ không có điểm chung.\\
			$P=(x-a)^2+(y-b)^2+(z-c)^2=MH^2$ đạt giá trị nhỏ nhất khi vị trí của $M$ và $H$ như hình vẽ\\
			Khi đó $HI=\mathrm{d}(I,(P))=\sqrt{3}\Rightarrow HM=HI-R=\sqrt{3}-1$\\
			Do đó ${P_{\min }}={{\left( \sqrt{3}-1 \right)}^2}=4-2\sqrt{3}$.}
		{\begin{tikzpicture}[scale=0.75, font=\footnotesize, line join=round, line cap=round, >=stealth]
				\def\R{2.5} % Bán kính
				\coordinate (I) at (0,0);
				\coordinate (M) at (0,-\R);
				\coordinate (H) at (0,-2*\R);
				\coordinate (A) at (-4,-6);
				\coordinate (B) at (-3,-4);
				\coordinate (C) at (2,-6);
				\draw (-4,-6)--(-3,-4)--(3,-4)--(2,-6)--(-4,-6);
				\draw[dashed](I)--(H);
				\coordinate[label=below right:$(P)$] (P) at (-3.9,-5.3);
				\coordinate[label=below right:$(S)$] (S) at (2,2);
				\draw(I) circle (\R);
				\draw pic[angle eccentricity=2.2,draw,angle radius=25]{angle=C--A--B};
				\foreach \y/\g in {I/90,M/-40,H/0}
				\fill[black] (\y) circle(1.5pt) ($(\y)+(\g:3.5mm)$) node{$\y$};
		\end{tikzpicture}}
		}
\end{ex}
\begin{ex}%[2H5C3-3]
	Trong KG $Oxyz$, cho hai điểm $A(-2;2;-2)$; $B(3;-3;3)$. Điểm $M$ trong không gian thỏa mãn $\dfrac{MA}{MB}=\dfrac{2}{3}$. Khi đó độ dài $OM$ lớn nhất bằng
	\choice
	{$6\sqrt{3}$}
	{\True $12\sqrt{3}$}
	{$\dfrac{5\sqrt{3}}{2}$}
	{$5\sqrt{3}$}
	\loigiai{
		Gọi $M(x;y;z)$. Ta có\\
		\begin{eqnarray*}
			\dfrac{MA}{MB}=\dfrac{2}{3}
			&\Leftrightarrow & 9MA^2=4MB^2\\
			&\Leftrightarrow & 9\left[ (x+2)^2+(y-2)^2+(z+2)^2 \right]=4\left[ (x-3)^2+(y+3)^2+(z-3)^2 \right]\\
			&\Leftrightarrow & x^2+y^2+z^2+12x-12y+12z=0\\
			&\Leftrightarrow & (x+6)^2+(y-6)^2+(z+6)^2=108.
		\end{eqnarray*}
		Như vậy, điểm $M$ thuộc mặt cầu $(S)$ tâm $I(-6;6;-6)$ và bán kính $R=\sqrt{108}=6\sqrt{3}$.\\
		Do đó $OM$ lớn nhất bằng $OI+R=\sqrt{(-6)^2+6^2+(-6)^2}+6\sqrt{3}=12\sqrt{3}$.}
\end{ex}
\begin{ex}%[2H5C3-3]
	Trong KG $Oxyz$, cho mặt cầu $(S)\colon x^2+y^2+z^2+2x-4y-2z+\dfrac{9}{2}=0$ và hai điểm $A(0;2;0)$, $B(2;-6;-2)$. Điểm $M(a;b;c)$ thuộc $(S)$ thỏa mãn $\overrightarrow{MA}\cdot\overrightarrow{MB}$ có giá trị nhỏ nhất. Tổng $a+b+c$ bằng
	\choice
	{$-1$}
	{\True $1$}
	{$3$}
	{$2$}
	\loigiai{
		$(x^2+y^2+z^2+2x-4y-2z+\dfrac{9}{2}=0\Leftrightarrow (x+1)^2+(y-2)^2+(z-1)^2=\dfrac{3}{2}$.\\
		Mặt cầu $(S)$ có tâm $I(-1;2;1)$, bán kính $R=\dfrac{\sqrt{6}}{2}$.\\
		Vì $IA=\sqrt{2}>R$ và $IB=\sqrt{82}>R$ nên hai điểm $A$, $B$ nằm ngoài mặt cầu $(S)$.\\
		Gọi $K$ là trung điểm đoạn thẳng $AB$ thì $K(1;-2;-1)$ và $K$ nằm ngoài mặt cầu $(S)$.\\
		Ta có
		\begin{eqnarray*}
			\overrightarrow{MA}\cdot\overrightarrow{MB}
			&= &\left( \overrightarrow{MK}+\overrightarrow{KA} \right) \cdot\left( \overrightarrow{MK}+\overrightarrow{KB} \right)\\
			&= & MK^2+\overrightarrow{MK}\cdot\left( \overrightarrow{KA}+\overrightarrow{KB} \right)+\overrightarrow{KA}\cdot\overrightarrow{KB}\\
			&= & MK^2-KA^2.
		\end{eqnarray*}
		Suy ra $\overrightarrow{MA}\cdot \overrightarrow{MB}$ nhỏ nhất khi $MK^2$ nhỏ nhất, tức là $MK$ nhỏ nhất.\\
		$IM+MK\ge IK\Rightarrow R+MK\ge IK\Rightarrow MK\ge IK-R$.\\
		Suy ra $MK$ nhỏ nhất bằng $IK-R$, xảy ra khi $I$, $M$, $K$ thẳng hàng và $M$ nằm giữa hai điểm $I$, $K$. Như vậy $M$ là giao điểm của đoạn thẳng $IK$ và mặt cầu $(S)$.\\
		Có $\overrightarrow{IK}=(2;-4;-2)$, $IK=\sqrt{2^2+(-4)^2+(-2)^2}=2\sqrt{6}=4R=4IM$.\\
		Suy ra $\overrightarrow{IK}=4\overrightarrow{IM}\Leftrightarrow
		\heva{&2=4\left( a+1 \right)\\&-4=4\left( b-2 \right)\\&-2=4\left( c-1 \right)}\Leftrightarrow \heva{&a=-\dfrac{1}{2}\\&b=1\\&c=\dfrac{1}{2}.}$\\
		Vậy $a+b+c=1$.}
\end{ex}
\begin{ex}%[2H5C1-3]
	Trong không gian với hệ trục tọa độ $Oxyz$, cho mặt cầu $(S)\colon x^2+y^2+z^2=3$. Một mặt phẳng $(\alpha)$ tiếp xúc với mặt cầu $(S)$ và cắt các tia $Ox$, $Oy$, $Oz$ lần lượt tại $A$, $B$, $C$ thỏa mãn $OA^2+OB^2+OC^2=27$. Phương trình mặt phẳng $(\alpha)$ là
	\choice
	{$x+y+z+3=0$}
	{\True $x+y+z-3=0$}
	{$x+2y+3z-3=0$}
	{$x+2y+3z+3=0$}
	\loigiai{
		Gọi $H(a;b;c)$ là tiếp điểm của mặt phẳng $(\alpha)$ và mặt cầu $(S)$.\\
		Từ giả thiết ta có $a$, $b$, $c$ là các số dương.\\
		Mặt khác, $H\in (S)$ nên $a^2+b^2+c^2=3$ hay $OH^2=3\Leftrightarrow OH=\sqrt{3}.\quad(1)$ \\
		Mặt phẳng $(\alpha)$ đi qua điểm $H$ và vuông góc với đường thẳng $OH$ nên nhận $\overrightarrow{OH}=(a;b;c)$ làm véctơ pháp tuyến. Do đó, mặt phẳng $(\alpha)$ có phương trình là
		$$a(x-a)+b(y-b)+c(z-c)=0\Leftrightarrow ax+by+cz-\left( a^2+b^2+c^2 \right)=0\Leftrightarrow ax+by+cz-3=0.$$
		Suy ra $A\left(\dfrac{3}{a};0;0\right)$, $B\left(0;\dfrac{3}{b};0\right)$, $C\left(0;0;\dfrac{3}{c}\right)$.\\
		Theo đề $OA^2+OB^2+OC^2=27\Leftrightarrow $ $\dfrac{9}{a^2}+\dfrac{9}{b^2}+\dfrac{9}{c^2}=27$ $\Leftrightarrow $ $\dfrac{1}{a^2}+\dfrac{1}{b^2}+\dfrac{1}{c^2}=3.\quad(2)$\\
		Từ $(1)$ và $(2)$ ta có $\left(a^2+b^2+c^2\right)\left(\dfrac{1}{a^2}+\dfrac{1}{b^2}+\dfrac{1}{c^2}\right)=9$.\\
		Mặt khác, ta có $\left(a^2+b^2+c^2\right)\left(\dfrac{1}{a^2}+\dfrac{1}{b^2}+\dfrac{1}{c^2}\right)\ge 9$ và dấu \lq\lq =\rq\rq xảy ra khi $a=b=c=1$.\\
		Suy ra phương trình mặt phẳng $(\alpha)$ là $x+y+z-3=0$.}
\end{ex}
\begin{ex}%[2H5C3-3]
	Trong KG $Oxyz$, cho mặt phẳng $(P)\colon x+y+z-1=0$, đường thẳng $d\colon \dfrac{x-15}{1}=\dfrac{y-22}{2}=\dfrac{z-37}{2}$ và mặt cầu $(S)\colon x^2+y^2+z^2-8x-6y+4z+4=0$. Một đường thẳng $(\Delta)$ thay đổi cắt mặt cầu $(S)$ tại hai điểm $A, B$ sao cho $AB=8$. Gọi $A'$, $B'$ là hai điểm lần lượt thuộc mặt phẳng $(P)$ sao cho $AA'$, $BB'$ cùng song song với $d$. Giá trị lớn nhất của biểu thức $AA'+BB'$ là
	\choice
	{$\dfrac{8+30\sqrt{3}}{9}$}
	{\True $\dfrac{24+18\sqrt{3}}{5}$}
	{$\dfrac{12+9\sqrt{3}}{5}$}
	{$\dfrac{16+60\sqrt{3}}{9}$}
	\loigiai{
		\begin{center}
			\begin{tikzpicture}[scale=0.7, font=\footnotesize, line join=round, line cap=round, >=stealth]
				\def\R{2.5} % Bán kính
				\coordinate (I) at (0,0);
				\coordinate (P) at (0,2.5);
				\coordinate (J) at (0,-2);
				\coordinate (K) at (-1,-1.5);
				\coordinate (H) at (-1,2);
				\coordinate (M) at (2,-2.5);
				\coordinate (Z) at (-3,-1);
				\coordinate (N) at (-5,-4);
				\coordinate (E) at (2,-4);
				\coordinate (F) at (4,-1);
				\path[name path=KM] (K)--(M);
				\path[name path=HM] (H)--(M);
				\path[name path=ZF] (Z)--(F);
				\coordinate (x) at (\R,0);
				\coordinate[label=below right:$(P)$] (P') at (-4.9,-3.44);
				\coordinate[label=above:$(S)$] (S) at (1.5,2.2);
				\draw[name path=DTO](I) circle (\R) (x) arc (0:-180: {\R} and {\R/2})(6,-2.5)--(3,2)node[right]{$d$};
				\draw[dashed] (P)--(J)(H)--(K)(x) arc (0:180: {\R} and {\R/2});
				\path[name intersections={of= KM and DTO,by={N'}}];
				\path[name intersections={of= HM and DTO,by={H'}}];
				\path[name intersections={of= ZF and DTO,by={Q,R}}];
				\draw(Z)--(Q)(R)--(F)(Z)--(N)--(E)--(F)(N')--(M)(H')--(M);
				\draw [dashed](Q)--(R)(K)--(N')(H)--(H');
				\begin{scope}
					\clip (Z)--(N)--(E);
					\draw (N)circle (1);
				\end{scope}
				\foreach \x/\g in {P/90,J/-120,K/-180,H/230,M/0,I/-50,Z/-90,N/-90,E/-90,F/-90} \fill[black](\x)circle(1pt) +(\g:.4)node[scale=1]{$\x$};
			\end{tikzpicture}
		\end{center}
		Mặt cầu $(S)$ có tâm $I(4;3;-2)$ và bán kính $R=5$.\\
		Gọi $H$ là trung điểm của $AB$ thì $IH\bot AB$ và $IH=3$ nên $H$ thuộc mặt cầu $(S')$ tâm $I$ bán kính $R'=3$.\\
		Gọi $M$ là trung điểm của $A'B'$ thì $AA'+BB'=2HM$, $M$ nằm trên mặt phẳng $(P)$.\\
		Mặt khác ta có $\mathrm{d}(I;(P))=\dfrac{4}{\sqrt{3}}<R$ nên $(P)$ cắt mặt cầu $(S)$ và $\sin (d;(P))=\sin \alpha =\dfrac{5}{3\sqrt{3}}$. Gọi $K$ là hình chiếu của $H$ lên $(P)$ thì $HK=HM.\sin \alpha $.\\
		Vậy để $AA'+BB'$ lớn nhất thì $HK$ lớn nhất\\
		$\Leftrightarrow HK$ đi qua $I$ nên $H{K_{\max }}=R'+\mathrm{d}(I;(P))=3+\dfrac{4}{\sqrt{3}}=\dfrac{4+3\sqrt{3}}{\sqrt{3}}$.\\
		Vậy $AA'+BB'$ lớn nhất bằng $2\left( \dfrac{4+3\sqrt{3}}{\sqrt{3}} \right)\cdot \dfrac{3\sqrt{3}}{5}=\dfrac{24+18\sqrt{3}}{5}$.
		}
\end{ex}
\begin{ex}%[2H5C1-3]
	Trong KG $Oxyz$ cho mặt cầu $(S)\colon x^2+y^2+z^2=1$. Điểm $M\in (S)$ có tọa độ dương; mặt phẳng $(P)$ tiếp xúc với $(S)$ tại $M$ cắt các tia $Ox$; $Oy$; $Oz$ tại các điểm $A$, $B$, $C$. Giá trị nhỏ nhất của biểu thức $T=\left(1+OA^2\right)\left(1+OB^2\right)\left(1+OC^2\right)$ là
	\choice
	{$24$}
	{$27$}
	{\True $64$}
	{$8$}
	\loigiai{
		$(S)$ có tâm $(O)$ và bán kính $R=1$.\\
		Theo đề bài ta có $A(a;0;0)$, $B(0;b;0)$, $C(0;0;c)$, $(a;b;c>0)$ khi đó phương trình mặt phẳng $(P)$ là $\dfrac{x}{a}+\dfrac{y}{b}+\dfrac{z}{c}=1$.\\
		$(P)$ tiếp xúc với $(S)$ tại $M\in (S)$ khi
		\begin{eqnarray*}
			\mathrm{d}(O;(P))=1
			&\Leftrightarrow &\dfrac{1}{\sqrt{\dfrac{1}{a^2}+\dfrac{1}{b^2}+\dfrac{1}{c^2}}}=1\\
			&\Leftrightarrow & abc=\sqrt{a^2{b^2}+b^2{c^2}+c^2{a^2}}\ge \sqrt{3\sqrt[3]{a^4{b^4}{c^4}}}\\
			&\Leftrightarrow & abc\ge 3\sqrt{3} \text{  }(\text{do } a;b;c>0)\quad(1)
		\end{eqnarray*}
		Khi đó $T=\left( 1+OA^2 \right)\left( 1+OB^2 \right)\left( 1+OC^2 \right)=\left( 1+a^2 \right)\left( 1+b^2 \right)\left( 1+c^2 \right)$\\
		$\Rightarrow T=1+a^2+b^2+c^2+a^2{b^2}+b^2{c^2}+c^2{a^2}+a^2{b^2}{c^2}=1+a^2+b^2+c^2+2a^2{b^2}{c^2}.$\\
		Mặt khác $1+a^2+b^2+c^2+2a^2{b^2}{c^2}\ge 1+3\sqrt[3]{a^2{b^2}{c^2}}+2a^2{b^2}{c^2}\ge 64\quad(2)$\\
		$\Rightarrow T\ge 64$.\\
		Vậy giá trị nhỏ nhất của $T$ là $64$ khi $(1)$ và $(2)$ xảy ra dấu \lq\lq =\rq\rq $\Leftrightarrow a=b=c=\sqrt{3}$.}
\end{ex}
\begin{ex}%[2H5C3-3]
	Cho $a$, $b$, $c$, $d$, $e$, $f$ là các số thực thỏa mãn 
	$\heva{&(d-1)^2+(e-2)^2+(f-3)^2=1\\&(a+3)^2+(b-2)^2+c^2=9.}$ Gọi giá trị lớn nhất, giá trị nhỏ nhất của biểu thức $F=\sqrt{(a-d)^2+(b-e)^2+(c-f)^2}$ lần lượt là $M$, $m$. Khi đó, $M-m$ bằng
	\choice
	{$10$}
	{$\sqrt{10}$}
	{\True $8$}
	{$2\sqrt{2}$}
	\loigiai{
		\begin{center}
			\begin{tikzpicture}[scale=0.6, font=\footnotesize, line join=round, line cap=round, >=stealth]
				\def\R{4.5} % Bán kính
				\coordinate (I_1) at (-5,0);
				\coordinate (A_1) at (-6.5,0);
				\coordinate (A_2) at (-3.5,0);
				\coordinate (B_1) at (7.5,0);
				\coordinate (B_2) at (-1.5,0);
				\draw(I_1) circle (\R/3);
				\coordinate (I_2) at (3,0);
				\coordinate (A) at ($({\R/3*cos(-130)-5},{\R/3*sin(-130)})$);
				\coordinate (B) at ($({\R*cos(150)+3},{\R*sin(150)})$);
				\draw(I_2) circle (\R);
				\draw(A_1)--(B_1)(A)--(B);
				\foreach \x/\g in {I_1/90,I_2/90,A_1/180,A_2/-50,B_1/0,B_2/-40,A/-180,B/180} \fill[black](\x)circle(1pt) +(\g:.4)node[scale=1]{$\x$};
			\end{tikzpicture}
		\end{center}
		Gọi $A(d,e,f)$ thì $A$ thuộc mặt cầu $(S_1)\colon ( x-1)^2+(y-2)^2+(z-3)^2=1$ có tâm $I_1(1;2;3)$, bán kính $R_1=1$, $B(a;b;c)$ thì $B$ thuộc mặt cầu $(S_2)\colon (x+3)^2+(y-2)^2+z^2=9$ có tâm $I_2(-3;2;0)$, bán kính $R_2=3$.\\ 
		Ta có $I_1{I_2}=5>R_1+R_2\Rightarrow (S_1)$ và $(S_2)$ không cắt nhau và ở ngoài nhau.\\
		Dễ thấy $F=AB$, $AB$ đạt giá trị lớn nhất khi khi $A\equiv {A_1},B\equiv {B_1}$\\
		$\Rightarrow $ Giá trị lớn nhất bằng $I_1{I_2}+R_1+R_2=9$.\\
		$AB$ đạt giá trị nhỏ nhất khi $A\equiv {A_2},B\equiv {B_2}$\\
		$\Rightarrow $ Giá trị nhỏ nhất bằng $I_1{I_2}-R_1-R_2=1$.\\
		Vậy $M-m=8$}
\end{ex}
\begin{ex}%[2H5C3-3]
	Trong KG $Oxyz$, Cho điểm $A(2t;2t;0)$, $B(0;0;t)$ (với $t>0$). Điểm $P$ di động thỏa mãn $\overrightarrow{OP}\cdot\overrightarrow{AP}+\overrightarrow{OP}\cdot\overrightarrow{BP}+\overrightarrow{AP}\cdot\overrightarrow{BP}=3$. Biết rằng có giá trị $t=\dfrac{a}{b}$ với $a, b$ nguyên dương và $\dfrac{a}{b}$ tối giản sao cho $OP$ đạt giá trị lớn nhất bằng $3$. Khi đó giá trị của $Q=2a+b$ bằng
	\choice
	{$5$}
	{$13$}
	{\True $11$}
	{$9$}
	\loigiai{
		Gọi $P(x;y;z)$, ta có $\overrightarrow{OP}=(x;y;z)$, $\overrightarrow{AP}=(x-2t;y-2t;z)$, $\overrightarrow{BP}=(x;y;z-t)$.\\
		Ta có
		\begin{eqnarray*}
			& & \overrightarrow{OP}\cdot\overrightarrow{AP}+\overrightarrow{OP}\cdot\overrightarrow{BP}+\overrightarrow{AP}\cdot\overrightarrow{BP}=3\\
			&\Leftrightarrow & 3x^2+3y^2+3z^2-4tx-4ty-2tz-3=0\\
			&\Leftrightarrow &x^2+y^2+z^2-\dfrac{4}{3}tx-\dfrac{4}{3}ty-\dfrac{2}{3}tz-1=0
		\end{eqnarray*}
		Nên $P$ thuộc mặt cầu tâm $I\left(\dfrac{2t}{3};\dfrac{2t}{3};\dfrac{t}{3}\right)$, $R=\sqrt{t^2+1}$.\\
		Ta có $OI=t<R$ nên O thuộc phần không gian phía trong mặt cầu.\\
		Để $OP_{\max }$ thì $P$, $I$, $O$ thẳng hàng và $OP=OI+R$.\\
		Suy ra $O{P_{\max }}=OI+R\Leftrightarrow 3=t+\sqrt{t^2+1} \Leftrightarrow t=\dfrac{4}{3}$. \\
		Suy ra $a=4,b=3$.\\
		Vậy, $Q=2a+b=11$.}
\end{ex}
\begin{ex}%[2H5C3-3]
	Cho $x$, $y$, $z$ là ba số thực thỏa $x^2+y^2+z^2-4x+6y-2z-11=0$. Tìm giá trị lớn nhất của $P=2x+2y-z$.
	\choice
	{$\max P=20$}
	{$\max P=-18$}
	{$\max P=18$}
	{\True $\max P=12$}
	\loigiai{
		Ta có $P=2x+2y-z\Leftrightarrow 2x+2y-z-P=0.\quad (1)$\\
		Lại có $x^2+y^2+z^2-4x+6y-2z-11=0\Leftrightarrow (x-2)^2+(y+3)^2+(z-1)^2=25.\quad (2)$\\
		Xét trong hệ trục tọa độ $Oxyz$, ta thấy $(1)$ là phương trình của một mặt phẳng, gọi là $(\alpha)$ và $(2)$ là phương trình của một mặt cầu $(S)$ tâm $I(2;-3;1)$, bán kính $R=5$.\\
		Giá trị lớn nhất của $P=2x+2y-z$ là giá trị lớn nhất của $P$ để $(\alpha)$ và $(S)$ có điểm chung, điều này tương đương với $$\mathrm{d}\left( I,(\alpha) \right)\le R\Leftrightarrow \dfrac{\left|2\cdot 2+2\cdot\left(-3\right)-1\cdot 1-P\right|}{\sqrt{2^2+2^2+(-1)^2}}\le 5\Leftrightarrow \left|P+3\right|\le 15\Leftrightarrow -18\le P\le 12.$$
		Vậy $\max P=12$.}
\end{ex}
\begin{ex}%[2H5C2-3]
	Trong KG $Oxyz$, cho $d\colon \heva{&x=2\\&y=t\\&z=1-t}$ và mặt cầu $(S)\colon x^2+y^2+z^2-2x-4y+2z+5=0.$ Tọa độ điểm $M$ trên $(S)$ sao cho $\mathrm{d}(M,d)$ đạt giá trị lớn nhất là
	\choice
	{$(1;2;-1)$}
	{$(2;2;-1)$}
	{\True $(0;2;-1)$}
	{$(-3;-2;1)$}
	\loigiai{
		Ta có $\mathrm{d}(I,d)=1=R$ suy ra $(S)$ tiếp xúc với $d$ và tiếp điểm là $H(2;2;-1)$\\
		Suy ra $H$ là hình chiếu vuông góc của tâm $I$ trên $d$.\\
		Đường thẳng $IH$ có phương trình
		$\heva{&x=1+t\\&y=2\\&z=-1}$, $t\in \mathbb{R}$.\\
		Tọa độ giao điểm của $IH$ và $(S)$ là $A(0;2;-1)$ và $B\equiv H(2;2;-1)$.\\
		Ta có $d(A,(d))=AH=2\ge d(B,(P))=BH=0$.\\
		$\Rightarrow d(A,(d))=2\ge d(M,(d))\ge d(B,(d))=0$.\\
		Vậy $M(0;2;-1)$.}
\end{ex}
\begin{ex}%[2H5C2-3]
	Trong KG $Oxyz$, cho điểm $A(-3;3;-3)$ thuộc mặt phẳng $(\alpha)\colon 2x-2y+z+15=0$ và mặt cầu $(S)\colon (x-2)^2+(y-3)^2+(z-5)^2=100$. Đường thẳng $\Delta$ qua $A$, nằm trên mặt phẳng $(\alpha)$ cắt $(S)$ tại $A$, $B$. Để độ dài $AB$ lớn nhất thì PTĐT $\Delta$ là
	\choice
	{\True $\dfrac{x+3}{1}=\dfrac{y-3}{4}=\dfrac{z+3}{6}$}
	{$\dfrac{x+3}{16}=\dfrac{y-3}{11}=\dfrac{z+3}{-10}$}
	{$\heva{&x=-3+5t\\&y=3\\&z=-3+8t}$}
	{$\dfrac{x+3}{1}=\dfrac{y-3}{1}=\dfrac{z+3}{3}$}
	\loigiai{
		Mặt cầu $(S)$ có tâm $I(2;3;5)$, bán kính $R=10$. Do $d(I,(\alpha ))<R$ nên $\Delta $ luôn cắt $(S)$ tại $A$, $B$.\\
		Khi đó $AB=\sqrt{R^2-{\left( \mathrm{d}(I,\Delta ) \right)^2}}$. Do đó, $AB$ lớn nhất thì $\mathrm{d}(I,(\Delta))$ nhỏ nhất nên $\Delta$ qua $H$, với $H$ là hình chiếu vuông góc của $I$ lên $(\alpha)$. Phương trình $BH\colon \heva{&x=2+2\\&y=3-2t\\&z=5+t.}$
		$H\in (\alpha)\Rightarrow 2(2+2t)-2(3-2t)+5+t+15=0\Leftrightarrow t=-2\Rightarrow H(-2;7;3)$.\\
		Do vậy $\overrightarrow{AH}=(1;4;6)$ là véc tơ chỉ phương của $\Delta$.\\
		Phương trình của $\Delta $ là $\dfrac{x+3}{1}=\dfrac{y-3}{4}=\dfrac{z+3}{6}$.}
\end{ex}
\begin{ex}%[2H5C2-3]
	Trong KG $Oxyz$, cho điểm $A(-3;3;-3)$ thuộc mặt phẳng $(\alpha)\colon 2x2y+z+15=0$và mặt cầu $(S)\colon (x-2)^2+(y-3)^2+(z-5)^2=100$. Đường thẳng $\Delta$ qua $A$, nằm trên mặt phẳng $(\alpha)$ cắt $(S)$ tại $A$, $B$. Để độ dài $AB$ nhỏ nhất thì PTĐT $\Delta$ là
	\choice
	{\True $\dfrac{x+3}{16}=\dfrac{y-3}{11}=\dfrac{z+3}{-10}$}
	{$\dfrac{x+3}{1}=\dfrac{y-3}{4}=\dfrac{z+3}{6}$}
	{$\heva{&x=-3+5t\\&y=3\\&z=-3+8t}$}
	{$\dfrac{x+3}{16}=\dfrac{y-3}{-11}=\dfrac{z+3}{10}$}
	\loigiai{
		Mặt cầu $(S)$ có tâm $I(2;3;5)$, bán kính $R=10$. Do $d(I,(\alpha ))<R$ nên $\Delta $ luôn cắt $(S)$ tại $A$, $B$.\\
		Khi đó $AB=\sqrt{R^2-(\mathrm{d}(I,\Delta))^2}$. Do đó, $AB$ nhỏ nhất thì $\mathrm{d}(I,\Delta )$ lớn nhất nên $\Delta $ là đường thẳng nằm trong $(\alpha)$, qua $A$ và vuông góc với $AI$. Do đó $\Delta $ có véctơ chỉ phương $\overrightarrow{u_{\Delta }}=\left[\overrightarrow{AI},\overrightarrow{n}_{\alpha}\right]=(16;11;-10)$.\\
		Vậy, phương trình của $\Delta$ là $ \dfrac{x+3}{16}=\dfrac{y-3}{11}=\dfrac{z+3}{-10}$.}
\end{ex}
\begin{ex}%[2H5C1-3]
	Trong KG $Oxyz$, cho hai điểm $A(3;0;2)$, $B(3;0;2)$ và mặt cầu $x^2+(y+2)^2+(z-1)^2=25$. Phương trình mặt phẳng $(\alpha)$ đi qua hai điểm $A$, $B$ và cắt mặt cầu $(S)$ theo một đường tròn bán kính nhỏ nhất là
	\choice
	{$x-4y-5z+17=0$}
	{$3x-2y+z-7=0$}
	{$x-4y+5z-13=0$}
	{\True $3x+2y+z-11=0$}
	\loigiai{
		Mặt cầu $(S)$ có tâm $I(0;-2;1)$, bán kính $R=5$. Do $IA=\sqrt{17}<R$ nên $AB$ luôn cắt $(S)$. Do đó $(\alpha )$ luôn cắt $(S)$ theo đường tròn $(C)$ có bán kính $r=\sqrt{R^2-{{\left( d\left( I,(\alpha) \right) \right)}^2}}$. Đề bán kính $r$nhỏ nhất $\Leftrightarrow d\left( I,\left( P \right) \right)$ lớn nhất.\\
		Mặt phẳng $(\alpha)$ đi qua hai điểm $A$, $B$ và vuông góc với mp $(ABC)$.\\
		Ta có $\overrightarrow{AB}=(1;-1;-1)$, $\overrightarrow{AC}=(-2;-3;-2)$ suy ra mặt phẳng $(ABC)$ có véctơ pháp tuyến $\overrightarrow{n}=\left[ \overrightarrow{AB},\overrightarrow{AC}\right]=(-1;4;-5)$.\\
		$(\alpha)$ có véctơ pháp tuyến $\overrightarrow{n}_{\alpha}=\left[\overrightarrow{n},\overrightarrow{AB}\right]=(-9-6;-3)=-3(3;2;1)$.\\
		Phương trình $(\alpha)\colon 3(x-2)+2(y-1)+1(z-3)=0\Leftrightarrow 3x+2y+z-11=0$.\\
	}
\end{ex}
\Closesolutionfile{ans}
% \indapan{6}{ans/C5B3CD7-17-26}

