\section{GIÁ TRỊ LỚN NHẤT, GIÁ TRỊ NHỎ NHẤT LIÊN QUAN ĐẾN ĐƯỜNG THẲNG}
\begin{dang}{Bài toán 1}
Trong không gian $Oxyz$, cho các điểm $A_1, A_2, \ldots, A_n$ và đường thẳng
$\Delta\colon\heva{&x=x_0+a t \\ &y=y_0+b t\\ &z=z_0+c t}$.
Tìm tọa độ điểm $M(x ; y ; z)$ thuộc đường thẳng $\Delta$ sao cho
$T=\left|\alpha_1 \vec{MA_1}+\alpha_2 \vec{MA_2}+\ldots+\alpha_n \vec{MA_n}\right|$ nhỏ nhất (với $\alpha_1 ; \alpha_2 \ldots \alpha_n$ là các số thực cho trước thỏAMãn $\left.\alpha_1+\alpha_2+\ldots+\alpha_n \neq 0\right)$.
\\
\textbf{Phương pháp giải}
\\
\textbf{Cách 1:} Phương pháp hình học (Chọn điểm phụ).
\\
\textbf{Bước 1:} Tìm tọa độ điểm phụ $I$.
\\
Gọi $I$ là điểm thỏa mãn $\alpha_1 \vec{IA_1}+\alpha_2 \vec{IA_2}+\ldots+\alpha_n \vec{IA_n}=\vec{0}$.
\\
Dựa vào đẳng thức
$\alpha_1 \vec{IA_1}+\alpha_2 \vec{IA_2}+\ldots+\alpha_n \vec{IA_n}=\vec{0}$,
ta tìm được tọa độ điểm $I$.
Ta có:
\begin{align*}
	&\alpha_1\left(\vec{MI}+\vec{IA_1}\right)+\alpha_2\left(\vec{MI}+\vec{IA_2}\right)+\ldots+\alpha_n\left(\vec{MI}+\vec{IA_n}\right)\\
	&= \left(\alpha_1+\alpha_2+\ldots+\alpha_n\right) \vec{MI}+\alpha_1 \vec{IA_1}+\alpha_2 \vec{IA_2}+\ldots+\alpha_n \vec{IA_n} \\
	&= \left(\alpha_1+\alpha_2+\ldots+\alpha_n\right) \vec{MI} \quad \left(\text {do } \alpha_1 \vec{IA_1}+\alpha_2 \vec{IA_2}+\ldots+\alpha_n \vec{IA_n}=\vec{0}\right) 
\end{align*}
Suy ra
$$
	T=\left|\alpha_1 \vec{MA_1}+\alpha_2 \vec{MA_2}+\ldots+\alpha_n \vec{MA_n}\right|=\left|\alpha_1+\alpha_2+\ldots+\alpha_n\right||\vec{MI}|
$$
Vì $\alpha_1+\alpha_2+\ldots+\alpha_n$ là hằng số khác không nên $T_{\min } \Leftrightarrow|\vec{MI}|_{\min }$.
\\
Mà $M \in \Delta$ nên $MI$ nhỏ nhất khi điểm $M$ cần tìm là hình chiếu của $I$ trên đường thẳng $\Delta$.
\\
\textbf{Bước 2:} Tìm tọa độ điểm $M$
\begin{itemize}
	\item $M \in \Delta \Rightarrow M\left(x_0+a t ; y_0+b t ; z_0+c t\right)$. Tính $\vec{I M}$.
	\item Do $I M \perp \Delta$ nên $\vec{I M} \cdot \vec{u}_{\Delta}=0 \Rightarrow t \Rightarrow$ độ điểm $M$ cần tìm.
\end{itemize}

\textbf{Cách 2:} Phương pháp đại số
\begin{itemize}
	\item $M \in \Delta \Rightarrow M\left(x_0+a t ; y_0+b t ; z_0+c t\right)$.
	\item $T=\left|\alpha_1 \vec{MA_1}+\alpha_2 \vec{MA_2}+\ldots+\alpha_n \vec{MA_n}\right|=f(t)$ theo hàm $t$.
	\item Xét $f(t)$ để tìm giá trị nhỏ nhất, suy ra $t$.
\end{itemize}
\end{dang}

\begin{dang}{Bài toán 2}
Trong không gian $Oxyz$, cho các điểm $A_1, A_2, \ldots, A_n$ và đường thẳng
$\heva{&x=x_0+a t \\ &y=y_0+b t \\&z=z_0+c t}$.
\\
Tìm tọa độ điểm $M(x ; y ; z)$ thuộc đường thẳng $\Delta$ sao cho $T=\alpha_1 MA_1^2+\alpha_2 MA_2^2+\ldots+\alpha_n MA_n^2$ nhỏ nhất (hoặc lớn nhất) (với $\alpha_1 ; \alpha_2 \ldots \alpha_n$ là các số thực cho trước thỏAMãn $\alpha_1+\alpha_2+\ldots+\alpha_n \neq 0$).
\\
\textbf{Chú ý:}
\begin{align*}
	& T_{\min } \Leftrightarrow \alpha_1+\alpha_2+\ldots+\alpha_n>0 \\
	& T_{\max } \Leftrightarrow \alpha_1+\alpha_2+\ldots+\alpha_n<0.
\end{align*}
\textbf{Phương pháp giải}
\\
\textbf{Cách 1:} Phương pháp hình học (Chọn điểm phụ).
\\
\textbf{Bước 1:} Tìm tọa độ điểm phụ $I$
\begin{itemize}
	\item Gọi $I$ là điểm thỏAMãn $\alpha_1 \vec{IA_1}+\alpha_2 \vec{IA_2}+\ldots+\alpha_n \vec{IA_n}=\vec{0}$.
	\item Dựa vào đẳng thức $\alpha_1 \vec{IA_1}+\alpha_2 \vec{IA_2}+\ldots+\alpha_n \vec{IA_n}=\vec{0}$ ta tìm được tọa độ điểm $I$.
	\item Ta có
	\begin{align*}
		T & =\alpha_1 MA_1^2+\alpha_2 MA_2^2+\ldots+\alpha_n MA_n^2 \\
		& =\alpha_1\left(\vec{MA_1}\right)^2+\alpha_2\left(\vec{MA_2}\right)^2+\ldots+\alpha_n\left(\vec{MA_n}\right)^2 \\
		& =\alpha_1\left(\vec{MI}+\vec{IA_1}\right)^2+\alpha_2\left(\vec{MI}+\vec{IA_2}\right)^2+\ldots+\alpha_n\left(\vec{MI}+\vec{IA_n}\right)^2 \\
		& =\left(\alpha_1+\alpha_2+\ldots+\alpha_n\right) MI^2+\alpha_1 I A_1^2+\alpha_2 I A_2^2+\ldots+\alpha_n I A_n^2 \\
		&\qquad\qquad+2 \vec{MI}\left(\alpha_1 \vec{A_1}+\alpha_2 \vec{IA_2}+\ldots+\alpha_n \vec{IA_n}\right) \\
		& =\left(\alpha_1+\alpha_2+\ldots+\alpha_n\right) \vec{MI}+\alpha_1 I A_1^2+\alpha_2 I A_2^2+\ldots+\alpha_n I A_n^2 \\
		&\qquad\qquad\left(\text{do } \alpha_1 \vec{IA_1}+\alpha_2 \vec{IA_2}+\ldots+\alpha_n \vec{IA_n}=\vec{0}\right) \\
\Rightarrow T & =\left(\alpha_1+\alpha_2+\ldots+\alpha_n\right) \vec{MI}+\alpha_1 I A_1^2+\alpha_2 I A_2^2+\ldots+\alpha_n I A_n^2
	\end{align*}
	
	\item Vì $\alpha_1 I A_1^2+\alpha_2 I A_2^2+\ldots+\alpha_n I A_n^2$ không đổi nên
	\begin{itemize}
		\item với $\alpha_1+\alpha_2+\ldots+\alpha_n>0$ thì $T$ đậ giá trị nhỏ nhất khi và chỉ khi $MI$ nhỏ nhất.
		\item với $\alpha_1+\alpha_2+\ldots+\alpha_n<0$ thì $T$ đạt giá trị lớn nhất khi và chỉ khi $MI$ nhỏ nhất.
	\end{itemize}
	\item Mà $M \in \Delta$ nên $MI$ nhỏ nhất khi điểm $M$ cần tìm là hình chiếu của $I$ trên đường thẳng $\Delta$.
\end{itemize}
\textbf{Bước 2:} Tìm tọa độ điểm $M$
\begin{itemize}
	\item $M \in \Delta \Rightarrow M\left(x_0+a t ; y_0+b t ; z_0+c t\right)$. Tính $\vec{I M}$
	\item Do $I M \perp \Delta$ nên $\vec{I M} \cdot \vec{u}_{\Delta}=0 \Rightarrow t \Rightarrow$ độ điểm $M$ cần tìm.
\end{itemize}
\textbf{Cách 2:} Phương pháp đại số
\begin{itemize}
	\item $M \in \Delta \Rightarrow M\left(x_0+a t ; y_0+b t ; z_0+c t\right)$.
	\item $T=\left|\alpha_1 \vec{MA_1}+\alpha_2 \vec{MA_2}+\ldots+\alpha_n \vec{MA_n}\right|=f(t)$ theo hàm $t$.
	\item Xét $f(t)$ để tìm giá trị nhỏ nhất, suy ra $t$.
\end{itemize}
\end{dang}

\begin{dang}{Bài toán 3}
Trong không gian $Oxyz$, cho điểm $A\left(x_0 ; y_0 ; z_0\right)$ cố định và điểm $M$ di động trên đường thẳng $\Delta\colon\heva{&x=x_0+a t \\ &y=y_0+b t \\ &z=z_0+c t}$. Tìm tọa độ điểm $M$ để $AM$ có giá trị nhỏ nhất.
\textbf{Phương pháp giải}
\\
\textbf{Cách 1:} Phương pháp hình học.
\\
\textbf{Bước 1:}
\begin{itemize}
	\item Gọi $H$ là hình chiếu vuông góc của $A$ trên đường thẳng $\Delta$.
	\item Khi đó, tam giác $AH M$ vuông tại $H$ do đó $AM \geq AH$.
	\item Đẳng thức xảy ra khi $M \equiv H$.
	\item Do đó $AM$ nhỏ nhất khi $M$ là hình chiếu của $A$ trên đường thẳng $\Delta$.
\end{itemize}
\textbf{Bước 2:} Tìm tọa độ điểm $M$.
\begin{itemize}
	\item $+M \in \Delta \Rightarrow M\left(x_0+a t ; y_0+b t ; z_0+c t\right)$. Tính $\vec{AM}$.
	\item Do $AM \perp \Delta$ nên $\vec{AM} \cdot \vec{u}_{\Delta}=0 \Rightarrow t \Rightarrow$ độ điểm $M$ cần tìm.
\end{itemize}

\textbf{Cách 2:} Phương pháp đại số
\begin{itemize}
	\item $M \in \Delta \Rightarrow M\left(x_0+a t ; y_0+b t ; z_0+c t\right)$.
	\item $T=\left|\alpha_1 \vec{MA_1}+\alpha_2 \vec{MA_2}+\ldots+\alpha_n \vec{MA_n}\right|=f(t)$ theo hàm $t$.
	\item Xét $f(t)$ để tìm giá trị nhỏ nhất, suy ra $t$
\end{itemize}
\end{dang}
\Opensolutionfile{ans}[ans/ans-2-C3B5CD5]
\TN
%%------------ 
\begin{ex}%[GVBS: Mai Trung Hiếu]%[2H5V2-6]
Trong không gian $Oxyz$, cho 2 điểm $A(3 ;-2 ; 3), B(1 ; 0 ; 5)$ và đường thẳng $d \colon \dfrac{x-1}{1}=\dfrac{y-2}{-2}=\dfrac{z-3}{2}$. Tìm tọa độ điểm $M$ trên đường thẳng $d$ để $MA^2+MB^2$ đạt giá trị nhỏ nhất.
\choice
{$M(1 ; 2 ; 3)$}
{\True $M(2 ; 0 ; 5)$}
{$M(3 ;-2 ; 7)$}
{$M(3 ; 0 ; 4)$}
\loigiai{
Gọi $I$ là trung điểMCủa $AB$, ta có $I=(2 ;-1 ; 4)$.
\\
Khi đó
\begin{align*}
	MA^2+MB^2
	&=\vec{MA}^2+\vec{MB}^2=(\vec{MI}+\vec{IA})^2+(\vec{MI}+\vec{I B})^2 \\
	&=2 \vec{MI}^2+\vec{IA}^2+\vec{I B}^2+2 \vec{MI} \cdot(\vec{IA}+\vec{I B}) \\
	&=2 MI^2+I A^2+I B^2=MI^2+6
\end{align*}
Do đó $MA^2+MB^2$ đạt giá trị nhỏ nhất khi và chỉ khi $MI$ có độ dài ngắn nhất, điều này xảy ra khi và chỉ khi $M$ là hình chiếu vuông góc của $I$ trên đường thẳng $d$.
\\
Phương trình mặt phẳng $(P)$ đi qua $I$ và vuông góc với đường thẳng $d$ là 
$$
	1 \cdot (x-2) - 2 \cdot(y+1) + 2 \cdot (y-4)=0
$$
hay
$$
	(P) \colon x-2 y+2 z-12=0.
$$
Phương trình tham số của đường thẳng $d$ là $\heva{&x=1+t \\ &y=2-2 t \\ &z=3+2 t.}$
\\
Tọa độ điểm $M$ cần tìm là nghiệm $(x ; y ; z)$ củAHệ phương trình
$$
	\heva{&x=1+t \\ &y=2-2 t \\ &z=3+2 t \\ &x-2 y+2 z-12=0}
	\Leftrightarrow
	\heva{&x=2 \\ &y=0 \\ &z=5 \\ &t=1.}
$$
Vậy $M(2 ; 0 ; 5)$.
}
\end{ex}

%%------------ 
\begin{ex}%[GVBS: Mai Trung Hiếu]%[2H5V2-6]
Trong không gian $Oxyz$, cho đường thẳng $d$ có phương trình $\heva{&x=1-t \\ &y=2+t \\ &z=-t}$ và ba điểm $A(6 ; 0 ; 0)$, $B(0 ; 3 ; 0)$, $C(0 ; 0 ; 4)$. Gọi $M(a ; b ; c)$ là điểm thuộc $d$ sao cho biểu thức $P=MA^2+2 MB^2+3 MC^2$ đạt giá trị nhỏ nhất, khi đó $a+b+c$ bằng
\choice
{$-3$}
{\True $4$}
{$1$}
{$2$}
\loigiai{
Vì $M \in d$ nên giả sử $M(1-t ; 2+t ;-t)$.
\\
Ta có
$$
	MA^2=3 t^2+14 t+29;
	\quad
	MB^2=3 t^2-4 t+2;
	\quad
	MC^2=3 t^2+10 t+21.
$$
Suy ra
$$
	P=MA^2+2 MB^2+3 MC^2=18 t^2+36 t+96=18(t+1)^2+78 \geq 78.
$$
Do đó $P=MA^2+2 MB^2+3 MC^2$ đạt giá trị nhỏ nhất khi và chỉ khi $t=-1$, khi đó
$$
M(2 ; 1 ; 1) \Rightarrow a+b+c=4.
$$
}
\end{ex}

%%------------ 
\begin{ex}%[GVBS: Mai Trung Hiếu]%[2H5V2-6]
Trong không gian $Oxyz$, cho đường thẳng $d \colon \dfrac{x-1}{2}=\dfrac{y+1}{1}=\dfrac{z}{-1}$, $M(2 ; 1 ; 0)$. Gọi $H(a ; b ; c)$ là điểm thuộc $d$ sao cho $MH$ có độ dài nhỏ nhất. Tính $T=a^2+b^2+c^2$.
\choice
{\True $T=6$}
{$T=12$}
{$T=\sqrt{5}$}
{$T=21$}
\loigiai{
Phương trình tham số của đường thẳng
$d \colon \heva{&x=1+2 t \\ &y=-1+t \\ &z=-t.}$
\\
Vì $H \in d$ nên tọa độ điểm $H$ có dạng $H(1+2 t ;-1+t ;-t)$. Độ dài $MH$ nhỏ nhất khi và chỉ khi $H$ là hình chiếu của $M$ lên $d$.
\\
Ta có $\vec{u}(2 ; 1 ;-1)$ là vec-tơ chỉ phương của $d$, $\vec{MH}=(-1+2 t ;-2+t ;-t)$.
\\
$H$ là hình chiếu của $M$ lên $d$ nên
$$
	\vec{MH} \cdot \vec{u}=0 \Leftrightarrow 2(-1+2 t)+(-2+t)+t=0 \Leftrightarrow 6 t-4=0 \Leftrightarrow t=\dfrac{2}{3}.
$$
Tọa độ điểm $H\left(\dfrac{7}{3} ;-\dfrac{1}{3} ;-\dfrac{2}{3}\right)$ suy ra $a=\dfrac{7}{3}$, $b=-\dfrac{1}{3}$, $c=-\dfrac{2}{3}$.
\\
Vậy $T=a^2+b^2+c^2=\left(\dfrac{7}{3}\right)^2+\left(-\dfrac{1}{3}\right)^2+\left(\dfrac{2}{3}\right)^2=6$.
}
\end{ex}

%%------------ 
\begin{ex}%[GVBS: Mai Trung Hiếu]%[2H5V2-6]
Trong không gian $Oxyz$, cho hai điểm $A(1 ; 1 ; 1)$, $B(2 ; 0 ; 1)$ và mặt phẳng $(P) \colon x+y+2 z+2=0$. Viết phương trình chính tắc của đường thẳng $d$ đi qua $A$, song song với mặt phẳng $(P)$ sao cho khoảng cách từ $B$ đến $d$ lớn nhất.
\choice
{$d \colon \dfrac{x-1}{3}=\dfrac{y-1}{1}=\dfrac{z-1}{-2}$}
{$d \colon \dfrac{x}{2}=\dfrac{y}{2}=\dfrac{z+2}{-2}$}
{\True $d \colon \dfrac{x-2}{1}=\dfrac{y-2}{1}=\dfrac{z}{-1}$}
{$d \colon \dfrac{x-1}{3}=\dfrac{y-1}{-1}=\dfrac{z-1}{-1}$}
\loigiai{
Gọi $\left(P'\right)$ chứa $A$ và song song $(P)$ suy ra $\left(P'\right): x+y+2 z-4=0$.
\\
Ta thấy $B \in\left(P'\right)$ do đó $d(B, d)$ đạt giá trị lớn nhất là $AB$.
\\
Khi đó $d$ vuông góc với $AB$ và $d$ vuông góc với giá của $\vec{n}$ là vec-tơ pháp tuyến của $(P)$.
\\
Suy ra một vec-tơ chỉ phương của $d$ là $\vec{u}=[\vec{n}, \vec{AB}]=(2 ; 2 ;-2)$.
\\
Kết hợp với điểm $A$ thuộc $d$ nên ta chọn đáp án
$d \colon \dfrac{x-2}{1}=\dfrac{y-2}{1}=\dfrac{z}{-1}$.
}
\end{ex}

%%------------ 
\begin{ex}%[GVBS: Mai Trung Hiếu]%[2H5V2-3]
Trong không gian $Oxyz$, cho hai điểm $A(1 ; 2 ;-3), B(-2 ;-2 ; 1)$ và mặt phẳng $(\alpha): 2 x+2 y-z+9=0$. Gọi $M$ là điểm thay đổi trên mặt phẳng $(\alpha)$ sao cho $M$ luôn nhìn đoạn $AB$ dưới một góc vuông. Xác định phương trình đường thẳng $MB$ khi $MB$ đạt giá trị lớn nhất.
\choice
{$\heva{&x=-2-t \\ &y=-2+2 t \\ &z=1+2 t}$}
{$\heva{&x=-2+2 t \\ &y=-2-t \\ &z=1+2 t}$}
{\True $\heva{&x=-2+t \\ &y=-2 \\ &z=1+2 t}$}
{$\heva{&x=-2+t \\ &y=-2-t \\ &z=1}$}
\loigiai{
Ta có
$$
	2 \cdot(-2)+2 \cdot(-2)-1+9=0 \Rightarrow B \in(\alpha).
$$
Gọi $H$ là hình chiếu của $A$ trên $(\alpha)$ thì $AH \perp MB$, $AM \perp MB$. Suy ra
$$
	MH \perp MB \Rightarrow MB \leq BH.
$$
Dấu bằng xảy ra khi $M \equiv H$, lúc đó $M$ là hình chiếu của $A$ trên $(\alpha)$.
\\
Gọi $H(x ; y ; z)$, $\vec{AH}=(x-1 ; y-2 ; z+3)$.
\\
Ta có hệ phương trình
$$
	\heva{&2 x+2 y-z+9=0 \\ &\dfrac{x-1}{2}=\dfrac{y-2}{2}=\dfrac{z+3}{-1}}
	\Leftrightarrow
	\heva{&2 x+2 y-z=-9 \\ &x-y=-1 \\ &x+2 z=-5} 
	\Leftrightarrow
	\heva{&x=-3 \\ &y=-2 \\ &z=-1}
	\Rightarrow M(-3 ;-2 ;-1)
$$
Suy ra $\vec{MB}=(1 ; 0 ; 2)$ và phương trình đường thẳng $MB$ là
$\heva{&x=-2+t \\ &y=-2 \\&z=1+2 t}$.
}
\end{ex}

%%------------ 
\begin{ex}%[GVBS: Mai Trung Hiếu]%[2H5V2-3]
Viết phương trình đường thẳng $a$ đi qua $M(4 ;-2 ; 1)$, song song với mặt phẳng $(\alpha) \colon 3 x-4 y+z-12=0$ và cách $A(-2 ; 5 ; 0)$ một khoảng lớn nhất.
\choice
{$\heva{&x=4-t \\ &y=-2+t \\ &z=1+t}$}
{$\heva{&x=4+t \\ &y=-2-t \\ &z=-1+t}$}
{$\heva{&x=1+4 t \\ &y=1-2 t \\ &z=-1+t}$}
{\True $\heva{&x=4+t \\ &y=-2+t \\ &z=1+t}$}
\loigiai{
Ta thấy
$\vec{AM}=(6 ;-7 ; 1)$, vec-tơ pháp tuyến của $(\alpha)$ là $\vec{n}=(3 ;-4 ; 1)$.
\\
Gọi $H$ là hình chiếu vuông góc của $A$ trên $a$ thì
$$
	\mathrm{d}(A ; a)=AH \leq AM=\sqrt{86}.
$$
nên $\mathrm{d}(A ; a)$ lớn nhất khi $H \equiv M$.
\\
Khi đó $a$ là đường thẳng đi qua $M$, song song với $(\alpha)$ và vuông góc với $AM$.
\\
Gọi $\vec{u}$ là vec-tơ chỉ phương của $a$, ta có $\vec{u} \perp \vec{n}$ và
$\vec{u} \perp \vec{AM}$ nên
$$
	\left[\vec{AM}, \vec{n}\right]=(-3 ;-3 ;-3)=-3(1 ; 1 ; 1)
$$
Chọn $\vec{u}=(1 ; 1 ; 1)$, ta thấy đáp án $\heva{&x=4+t \\ &y=-2+t \\ &z=1+t}$ thỏa mãn.
}
\end{ex}

%%------------ 
\begin{ex}%[GVBS: Mai Trung Hiếu]%[2H5V2-3]
Trong không gian $Oxyz$, cho hai điểm $A(-3 ; 0 ; 1)$, $B(1 ;-1 ; 3)$ và mặt phẳng $(P) \colon x-2 y+2 z-5=0$. Viết phương trình chính tắc của đường thẳng $\mathrm{d}$ đi qua $A$, song song với mặt phẳng $(P)$ sao cho khoảng cách từ $B$ đến $d$ nhỏ nhất.
\choice
{\True $d \colon \dfrac{x+3}{26}=\dfrac{y}{11}=\dfrac{z-1}{-2}$}
{$d \colon \dfrac{x+3}{26}=\dfrac{y}{-11}=\dfrac{z-1}{2}$}
{$d \colon \dfrac{x+3}{26}=\dfrac{y}{11}=\dfrac{z-1}{2}$}
{$d \colon \dfrac{x+3}{-26}=\dfrac{y}{11}=\dfrac{z-1}{-2}$}
\loigiai{
Gọi mặt phẳng $(Q)$ là mặt phẳng đi qua $A$ và song song với mặt phẳng $(P)$. Khi đó phương trình của mặt phẳng $(Q)$ là
$$
	1(x+3)-2(y-0)+2(z-1)=0 \Leftrightarrow x-2 y+2 z+1=0.
$$
Gọi $H$ là hình chiếu của điểm $B$ lên mặt phẳng $(Q)$, khi đó đường thẳng $BH$ đi qua $B(1 ;-1 ; 3)$ và nhận $\vec{n}_{(Q)}=(1 ;-2 ; 2)$ làm vec-tơ chỉ phương có phương trình tham số là $\heva{&x=1+t \\ &y=-1-2 t \\ &z=3+2 t}$.
\\
Vì $ H=BH \cap(Q) \Rightarrow H \in BH \Rightarrow H(1+t ;-1-2 t ; 3+2 t) \quad$ và $ H \in(Q) \quad$ nên ta có
$$
	(1+t)-2(-1-2 t)+2(3+2 t)+1=0 
	\Leftrightarrow 
	t=-\dfrac{10}{9} 
	\Rightarrow 
	H\left(-\dfrac{1}{9} ; \dfrac{11}{9} ; \dfrac{7}{9}\right)
$$
Suy ra
$$
	\vec{AH}=\left(\dfrac{26}{9} ; \dfrac{11}{9} ; \dfrac{-2}{9}\right)=\dfrac{1}{9}(26 ; 11 ;-2)
$$
Gọi $K$ là hình chiếu của $B$ lên đường thẳng $d$ thì
$$
	\mathrm{d}(B ; d)=BK \geq BH.
$$
nên khoảng cách từ $B$ đến $d$ nhỏ nhất khi $BK=BH$. Do đó đường thẳng $d$ đi qua $A$ và có vec-tơ chỉ phương $\vec{u}=(26 ; 11 ;-2)$ có phương trình chính tắc là $d \colon \dfrac{x+3}{26}=\dfrac{y}{11}=\dfrac{z-1}{-2}$.
}
\end{ex}