%%==============Cau_EX4==============%%%
\begin{ex}
	Trong không gian với hệ tọa độ $Oxyz$, cho phương trình mặt cầu $(S_m)\colon x^2+y^2+z^2+(m+2) x+2m y-2m z-m-3=0$. Biết rằng với mọi số thực $m$ thì $(S_m)$ luôn chứa một đường tròn cố định. Tính bán kính $r$ của đường tròn đó (làm tròn kết quả đến hàng phần mười).
	\shortans{$1{,}9$}
\loigiai{
	Mặt cầu $(S_m)$ có tâm $I\left(-\dfrac{m+2}{2};-m; m\right)$ và bán kính $R=\dfrac{\sqrt{9m^2+8m+16}}{2}$.\\
	Với $m_1$, $m_2$ tùy ý và khác nhau, ta được hai phương trình mặt cầu tương ứng:
	$$\heva{&x^2+y^2+z^2+\left(m_1+2\right) x+2m_1 y-2m_1 z-m_1-3=0\\ &x^2+y^2+z^2+\left(m_2+2\right) x+2m_2 y-2m_2 z-m_2-3=0.}$$
	Lấy $(1)$ trừ $(2)$ theo vế, ta được
	\allowdisplaybreaks
	\begin{eqnarray*}
		&&\left(m_1-m_2\right) x+2\left(m_1-m_2\right) y-2\left(m_1-m_2\right) z-\left(m_1-m_2\right)=0
		\\
		&\Leftrightarrow&\left(m_1-m_2\right) \cdot(x+2y-2z-1)=0
		\\
		&\Leftrightarrow& x+2y-2z-1=0. \qquad(3)		
	\end{eqnarray*}
	Dễ thấy $(3)$ là phương trình tổng quát của mặt phẳng.\\
	Suy ra họ mặt cầu $(S_m)$ có giao tuyến là đường tròn nằm trên mặt phẳng $(P)\colon x+2y-2z-1=0$ cố định.\\
	Mặt khác, đặt $d=\mathrm{d}\left[I,(P)\right]=\dfrac{\left|-\dfrac{m+2}{2}-2m-2m-1\right|}{\sqrt{1^2+2^2+(-2)^2}}=\dfrac{|-9m-4|}{6}$.\\
	$\Rightarrow r^2=R^2-d^2=\dfrac{9m^2+8m+16}{4}-\dfrac{(-9m-4)^2}{36}=\dfrac{32}{9} \quad \forall m \in \mathbb{R}$.\\
	Vậy $r=\dfrac{4\sqrt{2}}{3}\approx 1{,}9$.
}
\end{ex}
%%%==============HetCau_EX4==============%%%

%%%==============Cau_EX5==============%%%
\begin{ex}
	Trong không gian với hệ trục tọa độ $Oxyz$, cho đường thẳng $\Delta\colon \heva{&x=3+t \\ &y=-1-t, \\ &z=-2+t}(t \in \mathbb{R})$, điểm $M(1; 2;-1)$ và mặt cầu $(S)\colon x^2+y^2+z^2-4x+10y+14z+64=0$. Gọi $\Delta'$ là đường thẳng đi qua $M$ cắt
	đường thẳng $\Delta$ tại $A$, cắt mặt cầu tại $B$ sao cho $\dfrac{AM}{AB}=\dfrac{1}{3}$ và điểm $B$ có hoành độ là số nguyên. Biết  phương trình mặt phẳng trung trực đoạn $AB$ có dạng $ax+by+cz+d=0$. Tình $2a+b-12c+d$.
	\shortans{$5$}
\loigiai{
	$\Delta'$ là đường thẳng đi qua $M$ cắt đường thẳng $\Delta$ tại $A$ suy ra tọa độ $A(3+a;-1-a;-2+a)$.\\
	$\dfrac{AM}{AB}=\dfrac{1}{3} \Leftrightarrow 3\overrightarrow{AM}=\pm \overrightarrow{AB}$.
	\begin{itemize}
		\item Trường hợp $1$:\\
		$3\overrightarrow{AM}=\overrightarrow{AB} \Leftrightarrow\heva{&3(-2-a)=x-3-a \\ &3(3+a)=y+1+a \\ &3(1-a)=z+2-a} \Leftrightarrow\heva{&x=-3-2a \\ &y=8+2a \\ &z=1-2a.}$\\
		 Suy ra $B(-3-2a; 8+2a; 1-2a)$.\\
		Do $B\in(S)$ nên
		\allowdisplaybreaks
		\begin{eqnarray*}
			&&(-3-2a)^2+(8+2a)^2+(1-2a)^2-4(-3-2a)+10(8+2a)+14(1-2a)+64=0
			\\
			&\Leftrightarrow& 12a^2+40a+244=0,~\text{phương trình vô nghiệm}			
		\end{eqnarray*}
		\item Trường hợp $2$:\\
		$3\overrightarrow{AM}=-\overrightarrow{AB} \Leftrightarrow\heva{&3(-2-a)=-(x-3-a) \\& 3(3+a)=-(y+1+a) \\ &3(1-a)=-(z+2-a)} \Leftrightarrow\heva{&x=9+4a \\ &y=-10-4a \\ &z=-5+4a.}$\\
		Suy ra $B(9+4a;-10-4a;-5+4a)$.\\
		Do $B\in(S)$ nên
		\allowdisplaybreaks
		\begin{eqnarray*}
			&&(9+4a)^2+(-10-4a)^2+(-5+4a)^2-4(9+4a)+10(-10-4a)+14(-5+4a)+64=0
			\\
			&\Leftrightarrow& 48a^2+112a+64=0
			\\
			&\Leftrightarrow&\hoac{&a=-1\\ &a=-\dfrac{4}{3}.}			
		\end{eqnarray*}
		Điểm $B$ có hoành độ là số nguyên nên $B(5;-6;-9)$; $A(2; 0;-3)$.\\
		Mặt phẳng trung trực đoạn $AB$ đi qua trung điểm $I\left(\dfrac{7}{2};-3;-6\right)$ và có một véc-tơ pháp tuyến $\vec{n}=(-1; 2; 2)$ nên có phương trình \[\left(x-\dfrac{7}{2}\right)-2(y+3)-2(z+6)=0\Leftrightarrow 2x-4y-4z-43=0.\]
		Suy ra $a=2$, $b=-4$, $c=-4$, $d=-43$.
	\end{itemize}
		Vậy $2a+b-12c+d=2\cdot 2+(-4)-12\cdot (-4)+(-43)=5$.
}
\end{ex}
%%%==============HetCau_EX5==============%%%

%%%==============Cau_EX6==============%%%
\begin{ex}
	Trong không gian $Oxyz$, cho $(S)\colon (x+3)^2+(y-2)^2+(z-5)^2=36$, điểm $M(7;1;3)$. Gọi $\Delta$ là đường thẳng di động luôn đi qua $M$ và tiếp xúc với mặt cầu $(S)$ tại $N$. Tiếp điểm $N$ di động trên đường tròn $(T)$ có tâm $J(a;b;c)$. Gọi $k=2a-5b+10c$, tính giá trị của $k$.
	\shortans{$50$}
	\loigiai{
\immini{	Mặt cầu $(S)\colon (x+3)^2+(y-2)^2+(z-5)^2=36$ có tâm $I(-3;2;5)$, bán kính $R=6$.\\
	Có $IM=\sqrt{25+16+4}=3\sqrt{5} > 6=R$, nên $M$ thuộc miền ngoài của mặt cầu $(S)$.\\
	Có $MN$ tiếp xúc mặt cầu $(S)$ tại $N$, nên $MN\perp IN$ tại $N$.\\
	Gọi $J$ là điểm chiếu của $N$ lên $MI$.\\
	Có $IN^2=IJ\cdot IM$. Suy ra $IJ=\dfrac{IN^2}{IM}=\dfrac{36}{3\sqrt{5}}=\dfrac{12\sqrt{5}}{5}$ (không đổi), $I$ cố định.\\
	Suy ra $N$ thuộc $(P)$ cố định và mặt cầu $(S)$, nên $N$ thuộc đường tròn $(C)$ tâm $J$.}{	\begin{tikzpicture}[scale=0.7]
		\tikzset{declare function={%
				R=3;d=-1.5;theta=65;
				r=sqrt(R*R-d*d);betacrit=asin(d/r*cot(theta));},samples=200,smooth}
		\tdplotsetmaincoords{theta}{0}
		\draw (0,0) coordinate (I) circle (R);
		\path (-90:R) coordinate (A);
		\begin{scope}[tdplot_main_coords,line join=round]
			\draw[dashed,dash pattern = on 2pt off 1.5pt]
			(I)--(0,0,d) coordinate (J)
			--({r*cos(betacrit)},{r*sin(betacrit)},d) coordinate (N) -- cycle
			(J)--(A)
			plot[variable=\t,domain=betacrit:180-betacrit]({r*cos(\t)},{r*sin(\t)},d)
			plot[variable=\t,domain=0:180]({R*cos(\t)},{R*sin(\t)},0);
			\draw plot[variable=\t,domain=betacrit:-180-betacrit]({r*cos(\t)},{r*sin(\t)},d);
			\draw plot[variable=\t,domain=180:360]({R*cos(\t)},{R*sin(\t)},0);
		\end{scope}
		\path ($(A)!-0.56!(I)$) coordinate (M);
		\draw (A)--(M)--(N);
		\foreach \t/\g in {I/90,N/-30,J/220,M/-90}{
			\draw[fill=black] (\t) circle (1pt) node[shift={(\g:7pt)},font=\scriptsize]{$ \t $};
		}
\end{tikzpicture}}\noindent
	Gọi $N(x;y;z)$, có $\overrightarrow{IJ}=\dfrac{IJ}{IM}\cdot \overrightarrow{IM}=\dfrac{12\sqrt{5}}{5}\cdot \dfrac{1}{3\sqrt{5}} \overrightarrow{IM}=\dfrac{4}{5} \overrightarrow{IM} \Leftrightarrow\heva{&x+3=8\\ &y-2=-\dfrac{4}{5} \\ &z-5=-\dfrac{2}{5}.}$\\
	Suy ra $ N\left(5; \dfrac{6}{5}; \dfrac{23}{5}\right)$, $k=2a-5b+10c=50$. \\
	Vậy $k=50$.
}
\end{ex}
%%%==============HetCau_EX6==============%%%

%%%==============Cau_EX7==============%%%
\begin{ex}
	Trong không gian với hệ tọa độ $Oxyz$, cho mặt cầu $(S)\colon x^2+y^2+z^2-4x+4y-2z-7=0$ và đường thẳng $d_m$ là giao tuyến của hai mặt phẳng $x+(1-2m) y+4m z-4=0$ và $2x+m y-(2m+1) z-8=0$. Khi đó $m$ thay đổi các giao điểm của $d_m$ và $(S)$ nằm trên một đường tròn cố định. Tính bán kính $r$ của đường tròn đó (làm tròn kết quả đến hàng phần mười).
	\shortans{$3{,}1$}
\loigiai{
	%Hình vẽ
	Giả sử đường thẳng $d_m$ cắt mặt cầu tại hai điểm $A$, $B$.\\
	Mặt cầu $(S)$ có tâm $I(2;-2; 1)$, bán kính $R=4$.\\
	Đường thẳng $M(x; y) \in d_m$ thỏa $\heva{&x+(1-2m) y+4m z-4=0\\ &2x+m y-(2m+1) z-8=0} \Rightarrow 5x+y-2z-20=0$ nên các giao điểm của $(S)$ và $d_m$ thuộc đường tròn giao tuyến giữa $(S)$ và $(P)\colon 5x+y-2z-20=0$.\\
	$\mathrm{d}\left(I,(P)\right)=\dfrac{14}{\sqrt{30}}$ nên $r=\sqrt{R^2-\mathrm{d}^2(I,(P))}=\sqrt{4^2-\dfrac{14^2}{30}}=\sqrt{\dfrac{142}{15}}\approx 3{,}1$.
}
\end{ex}
%%%==============HetCau_EX7==============%%%
\Closesolutionfile{ans}
\indapan{6}{ans/C5B3CD4_22-31-kq}