\chude{GIÁ TRỊ LỚN NHẤT, GIÁ TRỊ NHỎ NHẤT LIÊN QUAN ĐẾN MẶT PHẲNG}
\begin{dang}{GIÁ TRỊ LỚN NHẤT, GIÁ TRỊ NHỎ NHẤT LIÊN QUAN ĐẾN BIỂU THỨC}
	\begin{bt}%[2H5C1-5]
		Trong KG $Oxyz$, cho các điểm $A_1$, $A_2,\ldots$, $A_n$ và mặt phẳng $(P)\colon Ax+By+Cz+D=0$. Tìm tọa độ điểm $M(x_0 ; y_0 ; z_0)$ thuộc mặt phẳng $(P)$ sao cho $T=\left|\alpha_1 \vec{M A_1}+\alpha_2 \vec{M A_2}+\ldots+\alpha_n \vec{M A_n}\right|$ nhỏ nhất (với $\alpha_1$; $\alpha_2$;$\ldots$; $\alpha_n$ là các số thực cho trước thỏa mãn $\alpha_1+\alpha_2+\ldots+\alpha_n \neq 0$).
		\loigiai{
			\begin{itemize}
				\item \textit{Cách 1: Phương pháp hình học (Chọn điểm phụ)}
				\begin{itemize}
					\item \textit{Bước 1:} Tìm tọa độ điểm phụ $I$.\\
					Gọi $I$ là điểm thỏa mãn $\alpha_1 \vec{I A_1}+\alpha_2 \vec{I A_2}+\ldots+\alpha_n \vec{I A_n}=\vec{0}$.\\
					Dựa vào đẳng thức $\alpha_1 \vec{I A_1}+\alpha_2 \vec{I A_2}+\ldots+\alpha_n \vec{I A_n}=\vec{0}$ ta tìm được tọa độ điểm $I$.\\
					Ta có
					\begin{eqnarray*}
						& & \alpha_1\left(\vec{M I}+\vec{I A_1}\right)+\alpha_2\left(\vec{M I}+\vec{I A_2}\right)+\ldots+\alpha_n\left(\vec{M I}+\vec{I A_n}\right) \\
						&= & \left(\alpha_1+\alpha_2+\ldots+\alpha_n\right) \vec{M I}+\alpha_1 \vec{I A_1}+\alpha_2 \vec{I A_2}+\ldots+\alpha_n \vec{I A_n} \\
						&= & \left(\alpha_1+\alpha_2+\ldots+\alpha_n\right) \vec{M I} \quad\left(\text { do } \alpha_1 \vec{I A_1}+\alpha_2 \vec{I A_2}+\ldots+\alpha_n \vec{I A_n}=\vec{0}\right) \\
						\Rightarrow T&= & \left|\alpha_1 \vec{M A_1}+\alpha_2 \vec{M A_2}+\ldots+\alpha_n \vec{M A_n}\right|\\
						&=&\left|\alpha_1+\alpha_2+\ldots+\alpha_n\right||\vec{M I}|.
					\end{eqnarray*}
					Vì $\alpha_1+\alpha_2+\ldots+\alpha_n$ là hằng số khác không nên $T_{\min } \Leftrightarrow|\vec{M I}|_{\min }$.\\
					Mà $M \in(P)$ nên $MI$ nhỏ nhất khi điểm $M$ cần tìm là hình chiếu của $I$ trên mặt phẳng $(P)$.
					\item \textit{Bước 2:} Tìm tọa độ điểm $M$.\\
					Lập PTTS đường thẳng $I M$ với $\heva{&\text{qua } I \\&\vec{u}_{I M}=\vec{n}_P=(A ; B ; C).}$\\
					Ta có $M=I M \cap (P) \Rightarrow$ tọa độ điểm $M$ cần tìm.
				\end{itemize}
				\item \textit{Cách 2: Phương pháp đại số}\\
				Dùng bất đẳng thức bộ $3$ của Bunhiacốpxki.\\
				Với $a$, $b$, $c$, $x$, $y$, $z$ $\in \mathbb{R}$, ta có
				\[(a x+b y+c z)^2  \leq(a^2+b^2+c^2)(x^2+y^2+z^2).\]
				Dấu \lq\lq  =\rq\rq\,xảy ra khi $\dfrac{a}{x}=\dfrac{b}{y}=\dfrac{c}{z}.$
			\end{itemize}	
		}
	\end{bt}
	
	\begin{bt}%[2H5C1-5]
		Trong KG $Oxyz$, cho các điểm $A_1$, $A_2$,$\ldots$, $A_n$ và mặt phẳng $(P)\colon Ax+By+Cz+D=0$. Tìm tọa độ điểm $M(x_0 ; y_0 ; z_0)$ thuộc mặt phẳng $(P)$ sao cho $T=\alpha_1 M A_1^2+\alpha_2 M A_2^2+\ldots+\alpha_n M A_n^2$ nhỏ nhất (hoặc lớn nhất) (với $\alpha_1 ; \alpha_2 \ldots \alpha_n$ là các số thực cho trước thỏa mãn $\alpha_1+\alpha_2+\ldots+\alpha_n \neq 0$).\\
		\textbf{Chú ý:}
		\begin{align*}
			& T_{\min } \Leftrightarrow \alpha_1+\alpha_2+\ldots+\alpha_n>0 \\
			& T_{\max } \Leftrightarrow \alpha_1+\alpha_2+\ldots+\alpha_n<0
		\end{align*}
		\loigiai{
			\begin{itemize}
				\item \textit{Cách 1: Phương pháp hình học (Chọn điểm phụ)}
				\begin{itemize}
					\item \textit{Bước 1:} Tìm tọa độ điểm phụ $I$.\\
					Gọi $I$ là điểm thỏa mãn $\alpha_1\vec{IA_1}+\alpha_2\vec{IA_2}+\ldots+\alpha_n\vec{IA_n}=\vec{0}$.\\
					Dựa vào đẳng thức $\alpha_1 \vec{I A_1}+\alpha_2 \vec{I A_2}+\ldots+\alpha_n \vec{I A_n}=\vec{0}$ ta tìm được tọa độ điểm $I$.\\
					Ta có
					\begin{eqnarray*}
						T& & \alpha_1 M A_1^2+\alpha_2 M A_2^2+\ldots+\alpha_n M A_n^2 \\
						&= & \alpha_1\left(\vec{M A_1}\right)^2+\alpha_2\left(\vec{M A_2}\right)^2+\ldots+\alpha_n\left(\vec{M A_n}\right)^2 \\
						&= & \alpha_1\left(\vec{M I}+\vec{I A_1}\right)^2+\alpha_2\left(\vec{M I}+\vec{I A_2}\right)^2+\ldots+\alpha_n\left(\vec{M I}+\vec{I A_n}\right)^2 \\
						&= & \left(\alpha_1+\alpha_2+\ldots+\alpha_n\right) M I^2+\alpha_1 I A_1^2+\alpha_2 I A_2^2+\ldots+\alpha_n I A_n^2+2 \vec{M I} \\
						&= & \left(\alpha_1+\alpha_2+\ldots+\alpha_n\right) \vec{M I}+\alpha_1 I A_1^2+\alpha_2 I A_2^2+\ldots+\alpha_n I A_n^2\\
						\Rightarrow T&= & \left(\alpha_1+\alpha_2+\ldots+\alpha_n\right) \vec{M I}+\alpha_1 I A_1^2+\alpha_2 I A_2^2+\ldots+\alpha_n I A_n^2.
					\end{eqnarray*}
					Vì  $\alpha_1 I A_1^2+\alpha_2 I A_2^2+\ldots+\alpha_n I A_n^2$ không đổi nên
					\begin{itemize}
						\item Với $\alpha_1+\alpha_2+\ldots+\alpha_n>0$ thì $T$ đạt giá trị nhỏ nhất khi và chỉ khi $M I$ nhỏ nhất.
						\item Với $\alpha_1+\alpha_2+\ldots+\alpha_n<0$ thì $T$ đạt giá trị lớn nhất khi và chỉ khi $M I$ nhỏ nhất.
					\end{itemize}
					Mà $M \in(P)$ nên $M I$ nhỏ nhất khi điểm $M$ cần tìm là hình chiếu của $I$ trên mặt phẳng $(P)$
					\item \textit{Bước 2:} Tìm tọa độ điểm $M$.\\
					Lập PTTS đường thẳng $I M$ với $\heva{& \text{qua } I \\& \vec{u}_{I M}=\vec{n}_P=(A ; B ; C).}$\\
					Ta có $M=I M \cap(P) \Rightarrow$ tọa độ điểm $M$ cần tìm.
				\end{itemize}
				\item \textit{Cách 2: Phương pháp đại số}\,\\
				Dùng bất đẳng thức bộ $3$ của Bunhiacốpxki.\\
				Với $a$, $b$, $c$, $x$, $y$, $z \in \mathbb{R}$, ta có 
				\[(a x+b y+c z)^2 \leq(a^2+b^2+c^2)(x^2+y^2+z^2).\]
				Dấu \lq\lq =\rq\rq\,xảy ra $\Leftrightarrow \dfrac{a}{x}=\dfrac{b}{y}=\dfrac{c}{z}$.
			\end{itemize}
		}
	\end{bt}
\end{dang}
\TN
\Opensolutionfile{ans}[ans/ans-C5B3CD5-LC]
\begin{ex}%[2H5C1-5]
		Trong KG $Oxyz$, cho hai điểm $A(1 ; 0 ; 2)$, $B(3 ; 1 ;-1)$ và mặt phẳng $(P)\colon x+y+z-1=0$. Gọi $M(a ; b ; c) \in(P)$ sao cho $\left|3 \vec{M A}-2 \vec{M B}\right|$ đạt giá trị nhỏ nhất. Tính $S=9a+3b+6c$.
		\choice
		{$4$}
		{\True $3$}
		{$2$}
		{$1$}
	\loigiai{
	\begin{itemize}
		\item \textit{Cách 1: Phương pháp hình học}\,\\
			Gọi $I(m ; n ; p)$ là điểm thỏa mãn  $3 \vec{I A}-2 \vec{I B}=\vec{0}$.\\
			Ta có $\vec{I A}=(1-m ;-n ; 2-p)$; $\vec{I B}=(3-m ; 1-n ;-1-p)$.
			\[3 \vec{I A}-2 \vec{I B}=\vec{0} \Leftrightarrow\heva{
				&3(1-m)-2(3-m)=0\\
				&3(-n)-2(1-n)=0\\
				&3(2-p)-2(-1-p)=0}
			 \Leftrightarrow \heva{&m=-3 \\&n=-2\\&p=8} \Rightarrow I(-3 ;-2 ; 8).\]
		Lại có $\left|3\vec{MA}-2\vec{MB}\right|=\left|3\left(\vec{MI}+\vec{IA}\right)-2\left(\vec{MI}+\vec{IB}\right)\right|=\left|\vec{MI}\right|=MI$.\\
		Khi đó, $\left|3\vec{MA}-2\vec{MB}\right|$ đạt giá trị nhỏ nhất, $M \in(P)$.\\
		Hay $ MI$ nhỏ nhất, $M \in(P)$.\\
		Khi đó $M$ là hình chiếu vuông góc của $I$ trên $(P)$.\\
		Gọi $\Delta$ là đường thẳng qua $I$ và vuông góc với $(P)$. Khi đó $\Delta$ nhận véc-tơ pháp tuyến của $(P)$ là $\vec{n}=(1 ; 1 ; 1)$  làm véc-tơ chỉ phương.\\
		Suy ra phương trình $\Delta$ có dạng $\Delta\colon \heva{&x=-3+t \\&y=-2+t \\&z=8+t.}$\\
		Tọa độ $M$ là nghiệm của hệ\\
		$$\heva{
				&x=-3+t\\
				&y=-2+t\\
				&z=8+t\\
				&x+y+z-1=0}
			\Leftrightarrow\heva{
				&t=-\dfrac{2}{3}\\
				&x=-\dfrac{11}{3}\\
				&y=-\dfrac{8}{3}\\
				&z=\dfrac{22}{3}}
				\Rightarrow\heva{
				&a=-\dfrac{11}{3}\\
				&b=-\dfrac{8}{3}\\
				&c=\dfrac{22}{3}}
				\Rightarrow S=9a+3b+6c=3.
		$$
	\item \textit{Cách 2: Phương pháp đại số}\,\\
	$M(a ; b ; c) \in(P) \Rightarrow a+b+c-1=0.$
	\begin{eqnarray*}
		& & 3 \vec{M A}-2 \vec{M B}=(-3-a ;-2-b ; 8-c)\\
		&\Leftrightarrow & |3 \vec{M A}-2 \vec{M B}|=\sqrt{(a+3)^2+(b+2)^2+(c-8)^2}
	\end{eqnarray*}
		Ta có $a+b+c-1=0 \Leftrightarrow(a+3)+(b+2)+(c-8)=-1$.
		\begin{eqnarray*}
			|(a+3)+(b+2)+(c-8)|
			&=& 1 \cdot(a+3)+1 \cdot(b+2)+1 \cdot(c-8)| \\
			&\leq & \sqrt{\left(1^2+1^2+1^2\right)\left[(a+3)^2+(b+2)^2+(c-8)^2\right]} \\
			\Rightarrow \sqrt{(a+3)^2+(b+2)^2+(c-8)^2} &\geq& \dfrac{1}{\sqrt{3}}.
		\end{eqnarray*}
		Dấu \lq\lq =\rq\rq\,xảy ra $\dfrac{a+3}{1}=\dfrac{b+2}{1}=\dfrac{c-8}{1} \Leftrightarrow a+3=b+2=c-8 \Leftrightarrow\heva{&a-b=-1 \\& a-c=-11.}$\\
		Ta có hệ $\heva{&a-b=-1 \\&a-c=-11 \\& a+b+c-1=0} \Leftrightarrow\heva{&a=-\dfrac{11}{3} \\&b=-\dfrac{8}{3} \\&c=\dfrac{22}{3}}\Rightarrow S=9a+3b+6c=3.$		
	\end{itemize}
	}
\end{ex}
%4-6

%%%=============EX_2=============%%%
\begin{ex}%[2H2C2-2]
	Trong KG $Oxyz$, cho ba điểm $A(4;2; 2)$, $B(1;1;-1)$, $C(2;-2;-2)$. Tìm tọa độ điểm $M$ thuộc $(Oxy)$ sao cho $\left|\vec{MA}+2\vec{MB}-\vec{MC}\right|$ nhỏ nhất.
	\choice
	{\True $M(2;3;0)$}
	{$M(1;3;0)$}
	{$M(2;-3;0)$}
	{$M(2;3;1)$} 
	\loigiai{
		\begin{enumerate}[\bf Cách 1.]
			\item Phương pháp hình học.\\
			Gọi $I$ là điểm thỏa mãn $\vec{IA}+2\vec{IB}-\vec{IC}=\vec{0}$.\\
			Ta có \begin{eqnarray*}
				\bullet&&\vec{IA}+2\vec{IB}-\vec{IC}=\vec{0}\\
				&\Leftrightarrow& \vec{IO}+\vec{OA}+2\left(\vec{IO}+\vec{OB}\right)-\left(\vec{IO}+\vec{OC}\right)=\vec{0}\\
				&\Leftrightarrow& \vec{OI}=\dfrac{1}{2} (\vec{OA}+2\vec{OB}-\vec{OC})\Rightarrow I(2;3;1).\\
				\bullet&& \left|\vec{MA}+2\vec{MB}-\vec{MC}\right|=\left|2\vec{MI}+\vec{IA}+2\vec{IB}-\vec{IC}\right|=2\cdot MI.
			\end{eqnarray*}
			$\left|\vec{MA}+2\vec{MB}-\vec{MC}\right|$ nhỏ nhất $\Leftrightarrow$ $MI$ nhỏ nhất $\Leftrightarrow$ $M$ là hình chiếu của $I$ trên mp$(Oxy)$.\\
			Vì $I(2;3;1)\Rightarrow M(2;3;0)$.
			\item Phương pháp hình học.
			\begin{center}
				\begin{tikzpicture}
					\def\a{4}
					\path 	(0:0) coordinate (A)
					++(0:\a) coordinate (B)
					++(-140:.5*\a) coordinate (C)
					($(A)!.5!(B)$) coordinate (D)
					($(A)!.5!(C)$) coordinate (E)
					($(D)+(120:.5*\a)$) coordinate (M)
					($(M)!.5!(E)$) coordinate (F)
					(intersection of A--B and M--E) coordinate (G);
					\draw[dashed] 	(G)--(D);
					\draw	(G)--(A)--(C)--(B)--(D)	(D)--(M)--(E)--cycle	(D)--(F);
					\foreach \x/\g in {A/180,B/0,C/-45,D/45,E/-100,F/175,M/90}
					\fill[black] 	(\x) circle (1pt)
					($(\g:3mm)+(\x)$) node {$\x$};
				\end{tikzpicture}
			\end{center}
			Gọi $D$; $E$; $F$ lần lượt là trung điểm của $AB$; $AC$; $ME$.\\
			Ta có
			\begin{eqnarray*}
				\left|\vec{MA}+2\vec{MB}-\vec{MC}\right|&=&\left|\vec{MA}+\vec{MB}+\vec{MB}-\vec{MC}\right|\\
				&=&\left|2\cdot \vec{MD}+\vec{CB}\right|\\
				&=&\left|2\cdot \vec{MD}+2\cdot \vec{ED}\right|\\
				&=&2\left|2\cdot \vec{FD}\right|\\
				&=&4\cdot FD.
			\end{eqnarray*}
			Ta lại có 
			\begin{itemize}
				\item $M(x;y;0)$, $D\left(\dfrac{5}{2};\dfrac{3}{2};\dfrac{1}{2}\right)$, $E(3;0;0)$, $F\left(\dfrac{x+3}{2};\dfrac{y}{2};0\right)$;
				\item $FD_{\min} \Leftrightarrow$ $F$ là hình chiếu của $D$ trên mp$(Oxy)\Leftrightarrow \heva{&x=2\\&y=3}\Leftrightarrow M(2;3;0)$.
			\end{itemize}
			\item Phương pháp đại số.\\
			$M\in (Oxy)\Rightarrow M(x;y;0)$;
			\begin{eqnarray*}
				\vec{MA}+2\vec{MB}-\vec{MC}&=&(4-2x;6-2y;-1)\\
				\Rightarrow \left|\vec{MA}+2\vec{MB}-\vec{MC}\right|&=&\sqrt{\left(4-2x\right)^2+\left(6-2y\right)^2+1}\\
				& \ge & 1.
			\end{eqnarray*}
			Dấu \lq\lq $=$\rq\rq, xảy ra $\Leftrightarrow x=2$; $y=3$. Khi đó $M(2;3;0)$.
			\item Trắc nghiệm, thay đáp án.\\
			$M\in (Oxy)\Rightarrow M(x;y;0)$. Ta có
			\begin{eqnarray*}
				\vec{MA}+2\vec{MB}-\vec{MC}&=&(4-2x;6-2y;-1)\\
				\Rightarrow \left|\vec{MA}+2\vec{MB}-\vec{MC}\right|&=&\sqrt{(4-2x)^2+(6-2y)^2+1}.
			\end{eqnarray*}
			\begin{itemize}
				\item Thế tọa độ điểm $M(2;3;0)$ vào ta được $\left|\vec{MA}+2\vec{MB}-\vec{MC}\right|=1$.
				\item Thế tọa độ điểm $M(1;3;0)$ vào ta được $\left|\vec{MA}+2\vec{MB}-\vec{MC}\right|=\sqrt{17}$.
				\item Thế tọa độ điểm $M(2;-3;0)$ vào ta được $\left|\vec{MA}+2\vec{MB}-\vec{MC}\right|=\sqrt{145}$.
				\item Điểm $M(2;3;1)$ không thuộc $\left(Oxy\right)$ nên bị loại.
			\end{itemize}
		\end{enumerate}
	}
\end{ex}

%%%=============EX_3=============%%%
\begin{ex}%[2H2C2-2]
	Trong hệ trục $Oxyz$, cho điểm $A(-1;3;5)$, $B(2;6;-1)$, $C(-4;-12;5)$ và mặt phẳng $(P)\colon x+2y-2z-5=0$. Gọi $M$ là điểm di động trên $(P)$. Gía trị nhỏ nhất của biểu thức $S=\left|\vec{MA}+\vec{MB}+\vec{MC}\right|$ là 
	\choice
	{$42$}
	{\True $14$}
	{$14\sqrt{3}$}
	{$\dfrac{14}{\sqrt{3}}$}
	\loigiai{
		Gọi $G\left(x_1;y_1;z_1 \right)$ là trọng tâm tam giác $ABC$.\\
		Vì $G$ là trọng tâm tam giác $ABC$ và $M$ là điểm tùy ý nên $\vec{MA}+\vec{MB}+\vec{MG}=3\vec{MG}$.\\
		Vậy $S=\left|\vec{MA}+\vec{MB}+\vec{MC}\right|=\left|3\vec{MG}\right|=3MG$.\\
		Do $G$ là trọng tâm $\triangle ABC$ nên $\heva{&{x_1=\dfrac{x_A+x_B+x_C}{3}=\dfrac{-1+2-4}{3}=-1} \\
			&{y_1=\dfrac{y_A+y_B+y_C}{3}=\dfrac{3+6-12}{3}=-1} \\
			&{z_1=\dfrac{z_A+z_B+z_C}{3}=\dfrac{5-1+5}{3}=3}
			&} \Rightarrow G(-1;-1;3)$.\\
		Vì $G$ cố định nên $S=3MG$ đạt giá trị nhỏ nhất khi và chỉ khi $MG$ nhỏ nhất. \\
		Tức là $MG\perp (P)$.\\
		Ta có $\mathrm{d}\big(G,(P)\big)=\dfrac{\left|-1\cdot 1+2\cdot(-1)-2\cdot 3-5\right|}{\sqrt{1^2+2^2+(-2)^2}}=\dfrac{14}{3}=MG$.\\
		Vậy giá trị nhỏ nhất $S=\left|\vec{MA}+\vec{MB}+\vec{MC}\right|=\left|3\vec{MG}\right|=3MG=3\cdot\dfrac{14}{3}=14$.
	}
\end{ex}

%%%=============EX_4=============%%%
\begin{ex}%[2H2C2-2]
	Trong KG $Oxyz$, cho ba điểm $A(4;2;2)$, $B(1;1;-1)$, $C(2;-2;-2)$. Tìm tọa độ điểm $M$ thuộc mặt phẳng $(Oyz)$ sao cho $\left|\vec{MA}+2\vec{MB}-\vec{MC}\right|$ nhỏ nhất
	\choice
	{$M(2; 3; 1)$}
	{\True $M(0; 3; 1)$}
	{$M(0;-3; 1)$}
	{$M(0; 1; 2)$}
	\loigiai{
		Gọi $I(x; y; z)$ là điểm thỏa $\vec{IA}+2\vec{IB}-\vec{IC}=\vec0$.\\
		Khi đó 
		\begin{eqnarray*}
			& &\vec{IA}+2\vec{IB}-\vec{IC}=\vec{0}\\
			&\Leftrightarrow& \big(\vec{OA}-\vec{OI}\big)+2\big(\vec{OB}-\vec{OI}\big)-\big(\vec{OC}-\vec{OI}\big)=\vec{0}\\
			&\Leftrightarrow& \vec{OI}=\dfrac{1}{2} \big(\vec{OA}+2\vec{OB}-\vec{OC}\big)=(2; 3; 1)\Rightarrow I(2; 3; 1).
		\end{eqnarray*}
		Ta có
		\begin{eqnarray*}
			\left|\vec{MA}+2\vec{MB}-\vec{MC}\right|&=&\left|(\vec{MI}+\vec{IA})+2(\vec{MI}+\vec{IB})-(\vec{MI}+\vec{IC})\right|\\
			&=&\left|2\vec{MI}+\vec{IA}+2\vec{IB}-\vec{IC}\right|\\
			&=&2\left|\vec{MI}\right|=2MI.
		\end{eqnarray*}
		$\left|\vec{MA}+2\vec{MB}-\vec{MC}\right|$ nhỏ nhất khi và chỉ khi $MI$ ngắn nhất.\\
		Khi đó $M$ là hình chiếu của $I(2; 3; 1)$ lên mặt phẳng $(Oyz)$. \\
		Suy ra $M(0; 3; 1)$. 
	}
\end{ex}

%%%=============EX_5=============%%%
\begin{ex}%[2H2C2-2]
	Trong KG $Oxyz$, cho các điểm $A(1;-1;3)$, $B(2;1;0)$, $C(-3;-1;-3)$ và mặt phẳng $(P)\colon x+y-z-4=0$. Gọi $M(a,b,c)$ là điểm thuộc mặt phẳng $(P)$ sao cho biểu thức $T=\left|3\vec{MA}-2\vec{MB}+\vec{MC}\right|$ đạt giá trị nhỏ nhất. Tính giá trị của biểu thức $S=a+b+c$. 
	\choice
	{$S=3$}
	{$S=-1$}
	{\True $S=2$}
	{$S=1$}
	\loigiai{
		Gọi $I(x;y;z)$ là điểm thỏa mãn $3\vec{IA}-2\vec{IB}+\vec{IC}=\vec{0}$.\\
		Ta có 
		\begin{itemize}
			\item $\vec{IA}=(1-x;-1-y;3-z)\Rightarrow 3\vec{IA}=(3-3x;-3-3y;9-3z)$.
			\item $\vec{IB}=(2-x;1-y;-z)\Rightarrow 2\vec{IB}=(4-2x;2-2y;-2z)$.
			\item $\vec{IC}=(-3-x;-1-y;-3-z)$.
		\end{itemize}
		Khi đó 
		\begin{eqnarray*}
			& &3\vec{IA}-2\vec{IB}+\vec{IC}=(-2x-4;-2y-6;-2z+6)=\vec{0}\\
			&\Leftrightarrow& \heva{&-2x-4=0\\&-2y-6=0\\&-2z+6=0}\Leftrightarrow \heva{&x=-2\\&y=-3\\&z=3.}
		\end{eqnarray*}
		Vậy $I(-2;-3;3)$.\\
		Ta có
		\begin{eqnarray*}
			T&=&\left|3\vec{MA}-2\vec{MB}+\vec{MC}\right|\\
			&=&\left|3(\vec{MI}+\vec{IA})-2(\vec{MI}+\vec{IB})+(\vec{MI}+\vec{IC})\right|\\
			&=&2\left|\vec{MI}\right|.
		\end{eqnarray*}
		Suy ra $T_{\min} \Leftrightarrow \left|\vec{MI}\right|_{\min}$ khi và chỉ khi $M$ là hình chiếu của $I$ lên mặt phẳng $(P)$.\\
		Đường thẳng $MI$ đi qua $I(-2;-3;3)$ và vuông góc với mặt phẳng $(P)$ có PTTS là $MI\colon \heva{&x=-2+t\\&y=-3+t\\&z=3-t.}$\\
		Lấy $M(-2+t;-3+t;3-t)\in MI$.\\
		Mặt khác $M\in (P)\Rightarrow (-2+t)+(-3+t)-(3-t)-4=0\Rightarrow t=4$. Suy ra $M(2;1;-1)$.\\
		Vậy $a+b+c=2$ 
	}
\end{ex}