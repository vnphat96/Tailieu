\begin{ex}%[2H5C1-5]
	Trong không gian $O x y z$ cho mặt phẳng $(P)\colon x+y+z-3=0$ và ba điểm $A(3; 1; 1), B(7; 3; 9)$ và $C(2; 2; 2)$. Điểm $M(a; b; c)$ trên $(P)$ sao cho $|\overrightarrow{MA}+2 \overrightarrow{MB}+3 \overrightarrow{MC}|$ đạt giá trị nhỏ nhất. Tính $2a-10b+c$.
	\choice
	{\True$\dfrac{62}{9}$}
	{$\dfrac{27}{9}$}
	{$\dfrac{46}{9}$}
	{$\dfrac{43}{9}$}	
	\loigiai{
		\begin{itemize}
			\item Gọi $I(x; y; z)$ là điểm thỏa mãn $\overrightarrow{IA}+2 \overrightarrow{IB}+3 \overrightarrow{I C}=\overrightarrow{0}$.\\
			Ta có $\heva{&\overrightarrow{I A}=(3-x; 1-y; 1-z) \\& \overrightarrow{I B}=(7-x; 3-y; 9-z) \\& \overrightarrow{I C}=(2-x; 2-y; 2-z).}$\\
			\item $\overrightarrow{I A}+2 \overrightarrow{I B}+3 \overrightarrow{I C}=\overrightarrow{0} \Leftrightarrow\heva{& 
				{2 3-6 x=0} \\&
				{1 3-6 y=0} \\&
				{2 5-6 z=0}
			} \Leftrightarrow \heva{&
				x=\dfrac{23}{6} \\&
				y=\dfrac{13}{6} \\&
				z=\dfrac{25}{6}.
			}$\\
			Suy ra
			$I\left(\dfrac{23}{6}; \dfrac{13}{6}; \dfrac{25}{6}\right).\\
			|\overrightarrow{MA}+2 \overrightarrow{MB}+3 \overrightarrow{MC}|=|6 \overrightarrow{MI}+(\overrightarrow{IA}+2 \overrightarrow{IB}+3 \overrightarrow{IC})|=|6 \overrightarrow{MI}|=6 MI.$\\
			$|\overrightarrow{M A}+2 \overrightarrow{M B}+3 \overrightarrow{M C}|$ đạt giá trị nhỏ nhất $\Leftrightarrow M$ là hình chiếu của $I$ lên mặt phẳng $(P)$.
			\item Gọi đường thẳng $(d)$ đi qua $I$ và vuông góc $(P)$.
			Ta có $(d)$ đi qua $I\left(\dfrac{23}{6}; \dfrac{13}{6}; \dfrac{25}{6}\right)$ và nhận $\overline{n_p}=(1; 1; 1)$ làm vectơ chỉ phương.\\
			Suy ra phương trình $(d)\colon\heva{&x=\dfrac{23}{6}+t \\& y=\dfrac{13}{6}+t \\& z=\dfrac{25}{6}+t.}$\\
			$M \in(d) \Leftrightarrow M\left(\dfrac{23}{6}+t; \dfrac{13}{6}+t; \dfrac{25}{6}+t\right)$.\\
			$M \in(P) \Leftrightarrow \dfrac{23}{6}+t+\dfrac{13}{6}+t+\dfrac{25}{6}+t-3=0 \Leftrightarrow t=\dfrac{-43}{18} \Leftrightarrow M\left(\dfrac{13}{9}; \dfrac{-2}{9}; \dfrac{16}{9}\right)$.\\
			Do đó $\heva{&a=\dfrac{13}{9} \\& b=\dfrac{-2}{9} \\& c=\dfrac{16}{9}} \Rightarrow 2 a-10 b+c=\dfrac{62}{9}$.
		\end{itemize}
	}
\end{ex}

\begin{ex}%[2H5C1-5]
	Trong KG $Oxyz$, cho ba điểm $A\left(1;1;1 \right),B\left(-1;2;0 \right),C\left(3;-1;2 \right)$ và điểm $M$ thuộc mặt phẳng $(\alpha)\colon2x-y+2z+7=0$. Tính giá trị nhỏ nhất của $\break P=\left| 3\overrightarrow{MA}+5\overrightarrow{MB}-7\overrightarrow{MC} \right|$.
	\choice
	{${{P}_{\min}}=20$}
	{${{P}_{\min}}=5$}
	{${{P}_{\min}}=25$}
	{\True${{P}_{\min}}=27$}
	\loigiai{
		Gọi $I$ là điểm thỏa mãn
		\begin{eqnarray*}
			&& 3\overrightarrow{IA}+5\overrightarrow{IB}-7\overrightarrow{IC}=\overrightarrow{0}\\
			&\Leftrightarrow&3\left(\overrightarrow{OA}-\overrightarrow{OI} \right)+5\left(\overrightarrow{OB}-\overrightarrow{OI} \right)-7\left(\overrightarrow{OC}-\overrightarrow{OI} \right)=\overrightarrow{0}\\
			&\Leftrightarrow&\overrightarrow{OI}=3\overrightarrow{OA}+5\overrightarrow{OB}-7\overrightarrow{OC}.
		\end{eqnarray*}
		Suy ra tọa độ điểm $I=\left(-23;20;-11 \right).$\\
		Khi đó 
		\begin{eqnarray*}
			\overrightarrow{u}&=&3\overrightarrow{MA}+5\overrightarrow{MB}-7\overrightarrow{MC}\\
			&=&3\left(\overrightarrow{IA}-\overrightarrow{IM} \right)+5\left(\overrightarrow{IB}-\overrightarrow{IM} \right)-7\left(\overrightarrow{IC}-\overrightarrow{IM} \right)\\
			&=&-\overrightarrow{IM}+\left(3\overrightarrow{IA}+5\overrightarrow{IB}-7\overrightarrow{IC} \right)\\
			&=&-\overrightarrow{IM}.
		\end{eqnarray*}
		Nên $P=\left| 3\overrightarrow{MA}+5\overrightarrow{MB}-7\overrightarrow{MC} \right|$ ${=\left|-\overrightarrow{IM} \right|}$ $=IM$ $\ge d\left(I,(\alpha) \right)$.\\
		Vậy ${{P}_{\min}}=d\left(I,(\alpha) \right)$ $=\dfrac{\left| 2\cdot\left(-23 \right)-20+2\cdot\left(-11 \right)+7 \right|}{\sqrt{{{2}^2}+{{1}^2}+{{2}^2}}}=27$.
	}
\end{ex}


\begin{ex}%[2H5C1-5]
	Trong không gian với hệ trục tọa độ ${Oxyz}$, cho bốn điểm ${A\left(2;-3;7 \right)}$, ${B\left(0;4;1 \right)}$, ${C\left(3;0;5 \right)}$ và ${D\left(3;3;3 \right)}$. Gọi ${M}$ là điểm nằm trên mặt phẳng ${\left(Oyz \right)}$ sao cho biểu thức ${\left| \overrightarrow{MA}+\overrightarrow{MB}+\overrightarrow{MC}+\overrightarrow{MD} \right|}$ đạt giá trị nhỏ nhất. Khi đó tọa độ của ${M}$ là
	\choice
	{$M(0;1;-4)$}
	{$M(2;1;0)$}
	{$M(0;1;-2)$}
	{\True$M(0;1;4)$}
	\loigiai{
		Ta có ${\overrightarrow{AB}=\left(-2;7;-6 \right)}$, ${\overrightarrow{AC}=\left(1;3;-2 \right)}$, ${\overrightarrow{AD}=\left(1;6;-4 \right)}$ nên ${\left[\overrightarrow{AB},\overrightarrow{AC} \right]\cdot\overrightarrow{AD}=-4\ne 0}$.
		Suy ra ${\overrightarrow{AB}}$, ${\overrightarrow{AC}}$, ${\overrightarrow{AD}}$ không đồng phẳng.\\
		Gọi ${G}$ là trọng tâm tứ diện ${ABCD}$. Khi đó ${G\left(2;1;4 \right)}$.\\
		Ta có ${\left| \overrightarrow{MA}+\overrightarrow{MB}+\overrightarrow{MC}+\overrightarrow{MD} \right|=\left| 4\overrightarrow{MG} \right|=4MG}$.\\
		Do đó, ${\left| \overrightarrow{MA}+\overrightarrow{MB}+\overrightarrow{MC}+\overrightarrow{MD} \right|}$ nhỏ nhất khi và chỉ khi ${MG}$ ngắn nhất.\\
		Vậy ${M}$ là hình chiếu vuông góc của ${G}$ lên mặt phẳng ${\left(Oyz \right)}$ nên ${M\left(0;1;4 \right)}$.
	}
\end{ex}

\begin{ex}%[2H5C1-5]
	Trong không gian ${Oxyz}$, cho ba điểm ${A(1;1;1)}$, ${B(-2;3;4)}$ và ${C(-2;5;1)}$. Điểm ${M(a;b;0)}$ thuộc mặt phẳng ${(Oxy)}$ sao cho $M{{A}^2}+M{{B}^2}+M{{C}^2}$ đạt giá trị nhỏ nhất. Tổng $T={{a}^2}+{{b}^2}$ bằng
	\choice
	{\True$T=10$}
	{$T=25$}
	{$T=13$}
	{$T=17$}
	\loigiai{
		Ta có $G\left(-1;3;2 \right)$ là trọng tâm tam giác $ABC$.
		Khi đó
		\begin{eqnarray*}
			M{{A}^2}+M{{B}^2}+M{{C}^2}&=&{{\overrightarrow{MA}}^2}+{{\overrightarrow{MB}}^2}+{{\overrightarrow{MC}}^2} \\
			&=&{{\left(\overrightarrow{MG}+\overrightarrow{GA} \right)}^2}+{{\left(\overrightarrow{MG}+\overrightarrow{GB} \right)}^2}+{{\left(\overrightarrow{MG}+\overrightarrow{GC} \right)}^2}\\
			&=&3M{{G}^2}+G{{A}^2}+G{{B}^2}+G{{C}^2}+2\overrightarrow{MG}\left(\overrightarrow{GA}+\overrightarrow{GB}+\overrightarrow{GC} \right)\\
			&=&3M{{G}^2}+G{{A}^2}+G{{B}^2}+G{{C}^2}.
		\end{eqnarray*}
		Do đó, ${M{{A}^2}+M{{B}^2}+M{{C}^2}}$ nhỏ nhất khi và chỉ khi ${MG}$ nhỏ nhất ${\Leftrightarrow} M$ là hình chiếu của G lên mặt phẳng ${\left(Oxy \right)}$. Do hình chiếu vuông góc của G lên mặt phẳng${\left(Oxy \right)}$ có tọa độ $\left(-1;3;0 \right)$Vậy $M\left(-1;3;0 \right)$. Từ đó $T={{(-1)}^2}+{{3}^2}=10$.
	}
\end{ex}

\begin{ex}%[2H5C1-5]
	Trong không gian với hệ trục tọa độ $O x y z$, cho tam giác $A B C$ với $A(2; 1; 3), B(1;-1; 2)$, $C(3;-6; 1)$. Điểm $M(x; y; z)$ thuộc mặt phẳng $(O y z)$ sao cho $M A^2+M B^2+M C^2$ đạt giá trị nhỏ nhất. Tính giá trị biểu thức $P=x+y+z$.
	\choice
	{\True$P=0$}
	{$P=2$}
	{$P=6$}
	{$P=-2$}
	\loigiai{
		Gọi $I$ là điểm thỏa $\overrightarrow{I A}+\overrightarrow{I B}+\overrightarrow{I C}=\overrightarrow{0} \Leftrightarrow I(2;-2; 2)$.
		\begin{eqnarray*}
			M A^2+M B^2+M C^2&=&(\overrightarrow{M I}+\overrightarrow{I A})^2+(\overrightarrow{M I}+\overline{I B})^2+(\overrightarrow{M I}+\overrightarrow{I C})^2\\
			&=&3 M I^2+I A^2+I B^2+I C^2+2 \overrightarrow{M I} \cdot(\overline{I A}+\overrightarrow{I B}+\overrightarrow{I C})\\
			&=&3 M I^2+I A^2+I B^2+I C^2.
		\end{eqnarray*}
		Mà $M\in(Oyz)$, suy ra $M A^2+M B^2+M C^2$ đạt giá trị nhỏ nhất, suy ra $M$ là hình chiếu của $I$ lên $(Oyz)$ hay $M(0;-2; 2)$. Vậy $P=0-2+2=0$.
	}
\end{ex}

\begin{ex}%[2H5C1-5]
	Trong không gian ${Oxyz}$, cho hai điểm ${A\left(2;-2;4 \right)}$, ${B\left(-3;3;-1 \right)}$ và mặt phẳng ${(P)\colon 2x-y+2z-8=0}$. Xét M là điểm thay đổi thuộc ${(P)}$, tính giá trị nhỏ nhất của ${2M{{A}^2}+3M{{B}^2}}$.
	\choice
	{\True $135$}
	{$90$}
	{$45$}
	{$210$}
	\loigiai{
		Gọi ${I(x;y;z)}$ là điểm thỏa mãn	
		${2\overrightarrow{MA}+3\overrightarrow{MB}=\overrightarrow{0}}$. Suy ra ${I\left(-1;1;1 \right)}$.\\
		${I{{A}^2}=27}$; ${I{{B}^2}=12}$; ${d\left(I,(P) \right)=3.}$\\
		${2M{{A}^2}+3M{{B}^2}}$ ${=2{{\left(\overrightarrow{MI}+\overrightarrow{IA} \right)}^2}+3{{\left(\overrightarrow{MI}+\overrightarrow{IB} \right)}^2}}$ ${=5{{\overrightarrow{MI}}^2}+2{{\overrightarrow{IA}}^2}+3{{\overrightarrow{IB}}^2}}$ ${=5M{{I}^2}+90}$.\\
		Mà ${2M{{A}^2}+3M{{B}^2}}$ nhỏ nhất ${\Leftrightarrow}$ ${MI}$ nhỏ nhất.\\
		Suy ra ${MI\ge d\left(I,(P) \right)=3}$.\\
		Vậy ${2M{{A}^2}+3M{{B}^2}\ge 5.9+90=135}.$
	}
\end{ex}

\begin{ex}%[2H5C1-5]
	Trong KG $Oxyz$, cho ba điểm $A\left(1;4;5 \right)$, $B\left(3;4;0 \right)$, $C\left(2;-1;0 \right)$. Gọi $M\left(a;b;c \right)$ là điểm sao cho $M{{A}^2}+M{{B}^2}+3M{{C}^2}$ đạt giá trị nhỏ nhất. Tổng $ a+b+c$ có giá trị bằng
	\choice
	{2}
	{3}
	{\True 4}
	{$-4$}
	\loigiai{
		Gọi $I$ là điểm thỏa mãn
		$$\overrightarrow{IA}+\overrightarrow{IB}+3\overrightarrow{IC}=\overrightarrow{0}\Leftrightarrow \overrightarrow{OI}=\dfrac{1}{5}\overrightarrow{OA}+\dfrac{1}{5}\overrightarrow{OB}+\dfrac{3}{5}\overrightarrow{OC}=(2;1;1).$$\\
		Suy ra $I(2;1;1)$.
		Khi đó
		\begin{eqnarray*}
			T &=&M{{A}^2}+M{{B}^2}+3M{{C}^2}\\
			&=&{{\left(\overrightarrow{MI}+\overrightarrow{IA} \right)}^2}+{{\left(\overrightarrow{MI}+\overrightarrow{IB} \right)}^2}+3{{\left(\overrightarrow{MI}+\overrightarrow{IC} \right)}^2}\\
			&=& 5M{{I}^2}+2\overrightarrow{MI}.\left(\overrightarrow{IA}+\overrightarrow{IB}+3\overrightarrow{IC} \right)+I{{A}^2}+I{{B}^2}+3I{{C}^2}\\
			&=& 5M{{I}^2}+I{{A}^2}+I{{B}^2}+3I{{C}^2}. 
		\end{eqnarray*}
		Vì $I$, $A$, $B$, $C$ cố định, suy ra $I{{A}^2}+I{{B}^2}+3I{{C}^2}$ không đổi nên $T$ nhỏ nhất $\Leftrightarrow MI$ nhỏ nhất $\Leftrightarrow M\equiv I\left(2;1;1 \right)$. 
		Suy ra $a=2, b=c=1$.\\ Vậy $ a+b+c=4$.}
\end{ex}

\begin{ex}%[2H5C1-5]
	Trong KG $Oxyz$, cho ba điểm $A\left(1;4;5 \right)$, $B\left(3;4;0 \right)$, $C\left(2;-1;0 \right)$ và mặt phẳng $(P)\colon 3x+3y-2z-29=0$. Gọi $M\left(a;b;c \right)$ là điểm thuộc $(P)$ sao cho biểu thức $T=M{{A}^2}+M{{B}^2}+3M{{C}^2}$ đạt GTNN. Tính tổng $ a+b+c$.
	\choice
	{\True 8}
	{10}
	{$-10$}
	{$-8$}
	\loigiai{
		Gọi $I$ là điểm thỏa mãn hệ thức $\overrightarrow{IA}+\overrightarrow{IB}+3\overrightarrow{IC}=\overrightarrow{0}$ $\left(* \right)$.\\
		Khi đó, $(*)\Leftrightarrow \overrightarrow{OI}=\dfrac{1}{5}\overrightarrow{OA}+\dfrac{1}{5}\overrightarrow{OB}+\dfrac{3}{5}\overrightarrow{OC}=(2;1;1)\Leftrightarrow I(2;1;1)$.\\
		Mặt khác, áp dụng tính chất tâm tỉ cự của hệ điểm suy ra $ T=5M{{I}^2}+I{{A}^2}+I{{B}^2}+3I{{C}^2}$.\\
		Vì $I{{A}^2}+I{{B}^2}+3I{{C}^2}$ là hằng số nên suy ra $T$ đạt GTNN khi và chỉ khi 
		\begin{eqnarray*}
			& & MI \text{ đạt GTNN}\\
			&\Leftrightarrow & M \text{ là hình chiếu vuông góc của } I \text{ trên }(P)\\
			&\Leftrightarrow & \heva{
				& M\in(P) \\ 
				& \overrightarrow{IM}\text{ cùng phương }{{\overrightarrow{n}}_{(P)}}}\\
			&\Leftrightarrow &\heva{
				& 3a+3b-2c=29 \\ 
				& \dfrac{a-2}{3}=\dfrac{b-1}{3}=\dfrac{c-1}{-2}} \\ 
			&\Leftrightarrow & \heva{
				& a=5 \\ 
				& b=4 \\ 
				& c=-1. \\ 
			}
		\end{eqnarray*}
		Vậy $ a+b+c=8$.}
\end{ex}


\begin{ex}%[2H5C1-5]
	Trong không gian với hệ tọa độ $O x y z$, cho 3 điểm $A(1; 2; 3), B(0; 1; 1), C(1; 0;-2)$ và mặt phẳng $(P):  x+y+z+2=0$. Gọi $M$ là điểm thuộc mặt phẳng $(P)$ sao cho giá trị của biểu thức $T=M A^2+2 M B^2+3 M C^2$ nhỏ nhất. Tính khoảng cách từ $M$ đến mặt phẳng $(Q): 2 x-y-2 z+3=0$?
	\choice
	{$\dfrac{2\sqrt{5}}{3}$}
	{$\dfrac{121}{54}$}
	{${24}$}
	{\True$\dfrac{91}{54}$}
	\loigiai{
		Gọi $I\left(a;b;c \right)$ là điểm thỏa mãn $\overrightarrow{IA}+2\overrightarrow{IB}+3\overrightarrow{IC}=\overrightarrow{0}$.\\
		Ta có $\overrightarrow{IA}\,\left(1-a;2-b;3-c \right),\overrightarrow{IB}\,\left(-a;1-b;1-c \right),\overrightarrow{IC}\,\left(1-a;-b;-2-c \right)$.\\
		$\overrightarrow{IA}+2\overrightarrow{IB}+3\overrightarrow{IC}=\overrightarrow{0}\,\Leftrightarrow \,\heva{
			& 1-a-2a+3-3a=0 \\ 
			& 2-b+2-2b-3b=0 \\ 
			& 3-c+2-2c-6-3c=0 \\ 
		} \,\Leftrightarrow \,\heva{
			& 6a=4 \\ 
			& 6b=4 \\ 
			& 6c=-1 \\ 
		} \,\Leftrightarrow \,\heva{
			& a=\dfrac{2}{3} \\ 
			& b=\dfrac{2}{3} \\ 
			& c=-\dfrac{1}{6}. \\ 
		} $\\
		Suy ra $I\left(\dfrac{2}{3};\dfrac{2}{3};-\dfrac{1}{6} \right)$.\\
		Ta chứng minh được $T=6M{{I}^2}+I{{A}^2}+2I{{B}^2}+3I{{C}^2}$.\\
		Do đó $T$ đạt GTNN khi $MI$ đạt GTNN $\Leftrightarrow M$ là hình chiếu vuông góc của $I$ trên mặt phẳng $(P)$.\\
		Ta có $MI\colon\,\heva{
			& x=\dfrac{2}{3}+t \\ 
			& y=\dfrac{2}{3}+t \\ 
			& z=-\dfrac{1}{6}+t, \\ 
		} $\\ $M \in MI \Rightarrow M\left(t+\dfrac{2}{3};t+\dfrac{2}{3};t-\dfrac{1}{6} \right),M \in (P) \Rightarrow 3t+\dfrac{19}{6}=0\,\Leftrightarrow \,t=-\dfrac{19}{18}.$\\
		$ \Rightarrow M\left(-\dfrac{7}{18};-\dfrac{7}{18};-\dfrac{11}{9} \right) \Rightarrow d\left(M;(P) \right)=\dfrac{\left|-\dfrac{7}{9}+\dfrac{7}{18}+\dfrac{22}{9}+3 \right|}{3}=\dfrac{91}{54}$.
	}
	
\end{ex}

\begin{ex}%[2H5C1-5]
	Trong KG $Oxyz$, cho ba điểm $A\left(2;-2;4 \right)$, $B\left(-3;3;-1 \right)$, $C\left(-1;-1;-1 \right)$ và mặt phẳng $(P)\colon2x-y+2z+8=0$. Xét điểm $M$ thay đổi thuộc $(P)$, tìm giá trị nhỏ nhất của biểu thức $T=2M{{A}^2}+M{{B}^2}-M{{C}^2}$.	
	\choice
	{\True $102$}
	{$105$}
	{$30$}
	{$35$}
	\loigiai{
		Gọi $I$ là điểm thỏa mãn
		\begin{eqnarray*}
			&                &2\overrightarrow{IA}+\overrightarrow{IB}-\overrightarrow{IC}=\overrightarrow{0}\\
			&\Leftrightarrow &2\left(\overrightarrow{OA}-\overrightarrow{OI} \right)+\left(\overrightarrow{OB}-\overrightarrow{OI} \right)-\left(\overrightarrow{OC}-\overrightarrow{OI} \right)=\overrightarrow{0}\\
			&\Leftrightarrow & \overrightarrow{OI}=\overrightarrow{OA}+\dfrac{1}{2}\overrightarrow{OB}-\dfrac{1}{2}\overrightarrow{OC}=\left(1;0;4 \right)\\
			&\Leftrightarrow &I\left(1;0;4 \right).
		\end{eqnarray*}
		Khi đó, với mọi điểm $M\left(x;y;z \right)\in(P)$, ta luôn có
		\begin{eqnarray*}
			&T&=2{{\left(\overrightarrow{MI}+\overrightarrow{IA} \right)}^2}+{{\left(\overrightarrow{MI}+\overrightarrow{IB} \right)}^2}-{{\left(\overrightarrow{MI}+\overrightarrow{IC} \right)}^2}\\
			&&=2{{\overrightarrow{MI}}^2}+2\overrightarrow{MI}.\left(2\overrightarrow{IA}+\overrightarrow{IB}-\overrightarrow{IC} \right)+2{{\overrightarrow{IA}}^2}+{{\overrightarrow{IB}}^2}-{{\overrightarrow{IC}}^2}\\
			&&=2M{{I}^2}+2I{{A}^2}+I{{B}^2}-I{{C}^2}.
		\end{eqnarray*}
		Ta tính được $2I{{A}^2}+I{{B}^2}-I{{C}^2}=30$.\\
		Do đó, $T$ đạt GTNN $\Leftrightarrow MI$ đạt GTNN $\Leftrightarrow MI\bot(P)$.\\
		Lúc này, $IM=d\left(I,(P) \right)=\dfrac{\left| 2\cdot1-0+2\cdot4+8 \right|}{\sqrt{{{2}^2}+{{\left(-1 \right)}^2}+{{2}^2}}}=6$.\\
		Vậy ${{T}_{\min}}={{2\cdot6}^2}+30=102$.
	}	
\end{ex}

\begin{ex}%[2H5C1-5]
	Trong không gian ${Oxyz}$ cho các điểm $A(1;2;0),B(1;-1;3),C(1;-1;-1)$ và mặt phẳng $\left(ABC \right)$. Xét $\dfrac{x}{a}+\dfrac{y}{b}+\dfrac{z}{c}=1$ thuộc mặt phẳng $M(2;-2;1)\in \left(ABC \right)$ sao cho $\overrightarrow{OM}=\left(2;-2;1 \right),\text{}OM=3$ nhỏ nhất. Giá trị của $d\left(O;(ABC) \right)=OH\le OM$ bằng
	\choice
	{\True$3$}
	{$7$}
	{$2$}
	{$-1$}
	\loigiai{
		Gọi $I$ là điểm thỏa mãn
		\begin{eqnarray*}
			&                &2 \overrightarrow{I A}+\overrightarrow{I C}-\overrightarrow{I B}=\overrightarrow{0}\\
			&\Leftrightarrow & 2(\overrightarrow{O A}-\overrightarrow{O I})+(\overrightarrow{O C}-\overrightarrow{O I})-(\overrightarrow{O B}-\overrightarrow{O I})=\overrightarrow{0}\\
			&\Leftrightarrow & \overrightarrow{O I}=\dfrac{2 \overrightarrow{O A}+\overrightarrow{O C}-\overrightarrow{O B}}{2}\\
			&\Rightarrow & I(1; 2;-2).
		\end{eqnarray*}
		Ta có
		\begin{eqnarray*}
			&2 M A^2=2 \overrightarrow{M A}^2=2\left(\overrightarrow{M I}+\overrightarrow{I A}\right)^2=2 \overrightarrow{M I}^2+2 \overrightarrow{I A}^2+4 \overrightarrow{M I} \cdot \overrightarrow{I A}\\
			& M B^2=\overrightarrow{M B}^2=\left(\overrightarrow{M I}+\overrightarrow{I B}\right)^2=\overrightarrow{M I}^2+\overrightarrow{I B}^2+2 \overrightarrow{M I} \cdot \overrightarrow{I B} \\
			& M C^2=\overrightarrow{M C}^2=\left(\overrightarrow{M I}+\overrightarrow{I C}\right)^2=\overrightarrow{M I}^2+\overrightarrow{I C}^2+2 \overrightarrow{M I} \cdot \overrightarrow{I C}.
		\end{eqnarray*}
		Suy ra $2M{{A}^2}-M{{B}^2}+M{{C}^2}=2M{I}^2+2I{{A}^2}+I{{C}^2}-I{{B}^2}+2\overrightarrow{MI}\left(2\overrightarrow{IA}+\overrightarrow{IC}-\overrightarrow{IB} \right)$.\\
		Suy ra $2M{{A}^2}-M{{B}^2}+M{{C}^2}=2M{{I}^2}+2I{{A}^2}+I{{C}^2}-I{{B}^2}$.\\ Do $I$ cố định nên $2I{{A}^2}+I{{C}^2}-I{{B}^2}$ không đổi. Vậy $2M{{A}^2}-M{{B}^2}+M{{C}^2}$ nhỏ nhất $\Leftrightarrow MI$ nhỏ nhất $\Leftrightarrow M$ là hình chiếu của $I$ trên $(P)$.\\
		Đường thẳng $\Delta$ qua $I\left(1;2;-2 \right)$ và vuông góc với $(P)$ là $\heva{
			& x=1+3t \\ 
			& y=2-3t \\ 
			& z=-2+2t. \\ 
		}$\\
		Suy ra tọa độ điểm $M$ là nghiệm của hệ $\heva{
			& x=1+3t \\ 
			& y=2-3t \\ 
			& z=-2+2t \\ 
			& 3x-3y+2z-15=0 \\ 
		} \Leftrightarrow \heva{
			& x=4 \\ 
			& y=-1 \\ 
			& z=0 \\ 
			& t=1. \\ 
		}$\\
		Suy ra $M(4;-1;0)$.
		Vậy $a+b+c=3$.	
	}	
\end{ex}

\begin{ex}%[2H5C1-5]
	Câu 17. Trong không gian $O x y z$, cho hai điểm $A(1; 2; 1), B(2;-1; 3), C(3; 1;-5)$. Tìm điểm $M$ trên mặt phẳng $(O y z)$ sao cho $M A^2-2 M B^2-M C^2$ lớn nhất.
	\choice
	{$M\left(\dfrac{3}{2}; \dfrac{1}{2}; 0\right)$}
	{$M\left(\dfrac{1}{2};-\dfrac{3}{2}; 0\right)$}
	{$M(0; 0; 5)$}
	{\True $M(3;-4; 0)$}
	\loigiai{
		Gọi điểm $E$ thỏa $\overrightarrow{E A}-2 \overrightarrow{E B}=\overrightarrow{0}$. Suy ra $B$ là trung điểm của $A E$, suy ra $E(3;-4; 5)$.\\
		Khi đó $M A^2-2 M B^2=(\overrightarrow{M E}+\overrightarrow{E A})^2-2(\overrightarrow{M E}+\overrightarrow{E B})^2=-M E^2+E A^2-2 E B^2$.\\
		Do đó $M A^2-2 M B^2$ lớn nhất $\Leftrightarrow M E$ nhỏ nhất $\Leftrightarrow M$ là hình chiếu của $E(3;-4; 5)$ lên $(O x y) \Leftrightarrow M(3;-4; 0)$.\\
		\textbf{Chú ý:} Ta có thể làm trắc nghiệm như sau
		\begin{itemize}
			\item Loại C vì $M(0; 0; 5)$ không thuộc $(O x y)$.
			\item Lần lượt thay $M\left(\dfrac{3}{2}; \dfrac{1}{2}; 0\right), M\left(\dfrac{1}{2};-\dfrac{3}{2}; 0\right), M(3;-4; 0)$ vào biểu thức $M A^2-2 M B^2$ thì $M(3;-4; 0)$ cho giá trị lớn nhất nên ta chọn $M(3;-4; 0)$.
		\end{itemize}
	}	
\end{ex}

\begin{ex}%[2H5C1-5]
	Trong không gian ${Oxyz}$, cho ba điểm ${A\left(-10;-5;8 \right)}$, ${B\left(2;1;-1 \right)}$, ${C\left(2;3;0 \right)}$ và mặt phẳng ${(P)\colon x+2y-2z-9=0}$. Xét ${M}$ là điểm thay đổi trên ${(P)}$ sao cho $\linebreak T=M{{A}^2}+2M{{B}^2}+3M{{C}^2}$ đạt giá trị nhỏ nhất. Tính $T=M{{A}^2}+2M{{B}^2}+3M{{C}^2}$.
	\choice
	{$54$}
	{\True $282$      }
	{$256$}
	{$328$}
	\loigiai{
		Gọi $I\left(x;y;z \right)$ là điểm thỏa mãn $\overrightarrow{IA}+2\overrightarrow{IB}+3\overrightarrow{IC}=\overrightarrow{0}$.
		Ta có $\overrightarrow{IA}=\left(-10-x;-5-y;8-z \right)$, $\overrightarrow{IB}=\left(2-x;1-y;-1-z \right)$, $\overrightarrow{IC}=\left(2-x;3-y;-z \right)$.\\
		Khi đó, $\heva{
			& \left(-10-x \right)+2\left(2-x \right)+3\left(2-x \right)=0 \\ 
			& \left(-5-y \right)+2\left(1-y \right)+3\left(3-y \right)=0 \\ 
			& \left(8-z \right)+2\left(-1-z \right)+3\left(-z \right)=0 \\ 
		} \Leftrightarrow \heva{
			& x=0 \\ 
			& y=1 \\ 
			& z=1 \\ 
		} \Rightarrow I\left(0;1;1 \right)$.\\
		Với điểm $M$ thay đổi trên $(P)$, ta có
		\begin{eqnarray*}
			&M{{A}^2}+2M{{B}^2}+3M{{C}^2}&={{\left(\overrightarrow{MI}+\overrightarrow{IA} \right)}^2}+2{{\left(\overrightarrow{MI}+\overrightarrow{IB} \right)}^2}+3{{\left(\overrightarrow{MI}+\overrightarrow{IC} \right)}^2}\\
			&&=6M{{I}^2}+I{{A}^2}+2I{{B}^2}+3I{{C}^2}+2\overrightarrow{MI}\left(\overrightarrow{IA}+2\overrightarrow{IB}+3\overrightarrow{IC} \right)\\
			&& =6M{{I}^2}+I{{A}^2}+2I{{B}^2}+3I{{C}^2} \,\text{(do } \overrightarrow{IA}+2\overrightarrow{IB}+3\overrightarrow{IC}=\overrightarrow{0}).
		\end{eqnarray*}
		Ta lại có $IA^2+2IB^2+3IC^2 = 185 + 2\cdot8 + 3\cdot9 = 228$.\\
		Do đó, $M{{A}^2}+2M{{B}^2}+3M{{C}^2}$ đạt giá trị nhỏ nhất $\Leftrightarrow MI$ đạt giá trị nhỏ nhất
		$\Leftrightarrow M$ là hình chiếu vuông góc của $I$ trên $(P)$.\\
		Khi đó, $MI=d\left(I,(P) \right)=3$.\\
		Vậy giá trị nhỏ nhất của $MA^2+2MB^2+3MC^2$ bằng
		$6M{{I}^2}+228=6\cdot9+228=282$.\\
		Giá trị nhỏ nhất của $M{{A}^2}+2M{{B}^2}+3M{{C}^2}$ đạt được khi và chỉ khi $M$ là hình chiếu vuông góc của $I$ trên $(P)$.\\
		Lưu ý thêm cách tìm điểm $M$ như sau
		Gọi $\Delta $ là đường thẳng qua $I$ và vuông góc với $(P)$. Phương trình của $\Delta\colon\heva{
			& x=t \\ 
			& y=1+2t \\ 
			& z=1-2t. \\ 
		} $\\
		Ta có $M=\Delta \cap (P)$.\\
		Xét phương trình
		$t+2\left(1+2t \right)-2\left(1-2t \right)-9=0\Leftrightarrow 9t-9=0\Leftrightarrow t=1\Rightarrow M\left(1;3;-1 \right)$.	
	}	
\end{ex}

\begin{ex}%[2H5C1-5]
	Trong không gian $O x y z$, cho ba điểm $A(1;-2; 1), B(5; 0;-1), C(3; 1; 2)$ và mặt phẳng $(Q)\colon 3 x+y-z+3=0$. Gọi $M(a; b; c)$ là điểm thuộc $(Q)$ thỏa mãn $\linebreak M A^2+M B^2+2 M C^2$ nhỏ nhất. Tính tồng $a+b+5 c$.
	\choice
	{$11$}
	{\True$9$}
	{$15$}
	{$14$}
	\loigiai{
		Gọi $E$ là điểm thỏa mãn $\overrightarrow{E A}+\overrightarrow{E B}+2 \overrightarrow{E C}=\overrightarrow{0}$. Suy ra $E(3; 0; 1)$.
		Ta có
		\begin{eqnarray*}
			&S&=M A^2+M B^2+2 M C^2=\overrightarrow{M A}^2+\overrightarrow{M B}^2+2 \overrightarrow{M C}^2\\
			&&=(\overrightarrow{M E}+\overrightarrow{E A})^2+(\overrightarrow{M E}+\overrightarrow{E B})^2+2(\overrightarrow{M E}+\overrightarrow{E C})^2\\
			&&=4 M E^2+E A^2+E B^2+2 E C^2.
		\end{eqnarray*}
		Vì $E A^2+E B^2+2 E C^2$ không đổi nên $S$ nhỏ nhất khi và chỉ khi $M E$ nhỏ nhất. Suy ra
		$M$ là hình chiếu vuông góc của $E$ lên $(Q)$.\\
		PTĐT $M E$ $\colon\heva{&x=3+3 t \\& y=t \\& z=1-t.}$\\
		Tọa độ điểm $M$ là nghiệm của hệ phương trình $\heva{&y=1-t \\& z=y-z+3=0} \Leftrightarrow\heva{&x=0 \\& y=-1 \\& z=2 \\& t=-1.}$\\
		$\begin{aligned}
			& \Rightarrow M(0;-1; 2) \Rightarrow a=0, b=-1, c=2. \\
			& \Rightarrow a+b+5 c=0-1+5.2=9.
		\end{aligned}$
	}	
\end{ex}

\begin{ex}%[2H5C1-5]
	Trong không gian với hệ trục tọa độ $O x y z$ cho 3 điểm $A(1; 1; 1), B(0; 1; 2), C(-2; 1; 4)$ và mặt phẳng $(P)\colon x-y+z+2=0$. Tìm điểm $N \in(P)$ sao cho $S=2 N A^2+N B^2+N C^2$ đạt giá trị nhỏ nhất.
	\choice
	{ $N\left(-\dfrac{4}{3}; 2; \dfrac{4}{3}\right)$}
	{ $N(-2; 0; 1)$}
	{ $N\left(-\dfrac{1}{2}; \dfrac{5}{4}; \dfrac{3}{4}\right)$}
	{\True $N(-1; 2; 1)$}
	\loigiai{
		Với mọi điểm $I$ ta có
		\begin{eqnarray*}
			&S& =2 N A^2+N B^2+N C^2=2(\overrightarrow{N I}+\overrightarrow{I A})^2+(\overrightarrow{N I}+\overrightarrow{I B})^2+(\overrightarrow{N I}+\overrightarrow{I C})^2 \\
			&&=4 N I^2+2 \overrightarrow{N I}(2 \overrightarrow{I A}+\overrightarrow{I B}+\overrightarrow{I C})+2 I A^2+I B^2+I C^2.
		\end{eqnarray*}
		Chọn điểm $I$ sao cho $2 \overrightarrow{I A}+\overrightarrow{I B}+\overrightarrow{I C}=\overrightarrow{0}$. Suy ra tọa độ điểm $I$ là $I(0; 1; 2)$.\\
		Khi đó $S=4 N I^2+2 I A^2+I B^2+I C^2$. Do đó $S$ nhỏ nhất khi $N$ là hình chiếu của $I$ lên mặt phẳng $(P)$.\\
		PTĐT đi qua $I$ và vuông góc với mặt phẳng $(P)$ là $\heva{&x=0+t \\& y=1-t \\& z=2+t.}$\\
		Tọa độ điểm $N(t; 1-t; 2+t) \in(P) \Rightarrow t-1+t+2+t+2=0 \Leftrightarrow t=-1 \Rightarrow N(-1; 2; 1)$.	
	}	
\end{ex}

\begin{ex}%[2H5C1-5]
	Trong không gian $O x y z$ cho $A(4;-2; 6), B(2; 4; 2), M \in(\alpha)\colon x+2 y-3 z-7=0$ sao cho $\overrightarrow{M A} \cdot \overrightarrow{M B}$ nhỏ nhất. Tọa độ của $M$ bằng	
	\choice
	{$\left(\dfrac{29}{13}; \dfrac{58}{13}; \dfrac{5}{13}\right)$}
	{\True$(4; 3; 1)$}
	{$(1; 3; 4)$}
	{$\left(\dfrac{37}{3}; \dfrac{-56}{3}; \dfrac{68}{3}\right)$}
	\loigiai{
		Gọi $I$ là điểm thỏa mãn $\overrightarrow{I A}+\overrightarrow{I B}=\overrightarrow{0}$. Suy ra $I$ là trung điểm $A B \Rightarrow I(3; 1; 4)$.\\
		Ta có $\overrightarrow{M A} \cdot \overrightarrow{M B}=(\overrightarrow{M I}+\overrightarrow{I A})(\overrightarrow{M I}+\overrightarrow{I B})=M I^2+\overrightarrow{M I}(\overrightarrow{I A}+\overrightarrow{I B})=M I^2$.\\
		Gọi $H$ là hình chiếu của $I$ xuống mặt phẳng $(\alpha)$.\\
		Do $I A$ không đổi nên $\overrightarrow{M A} \cdot \overrightarrow{M B}$ nhỏ nhất khi $M I$ nhỏ nhất $\Leftrightarrow M I=I H \Leftrightarrow M \equiv H$.\\
		Gọi $\Delta$ là đường thẳng đi qua $I$ và vuông góc với mặt phẳng $(\alpha)$.\\
		Khi đó $\Delta$ nhận $\overrightarrow{n_{(\alpha)}}=(1; 2;-3)$ làm vectơ chỉ phương.\\
		Do đó $\Delta$ có phương trình $\heva{&x=3+t \\& y=1+2 t \\& z=4-3 t.}$
		\begin{eqnarray*}
			&H \in \Delta \Leftrightarrow H(3+t; 1+2 t; 4-3 t). \\
			&H \in(\alpha) \Leftrightarrow(3+t)+2(1+2 t)-3(4-3 t)-7=0 \Leftrightarrow t=1 \Leftrightarrow H(4; 3; 1).
		\end{eqnarray*}
		Vậy $M(4; 3; 1)$.\\
		\textbf{Cách 2}. Phương pháp đại số\\
		Gọi $M(x; y; z) \in(\alpha) \Rightarrow x+2 y-3 z-7=0$\\
		$\overrightarrow{M A}=(4-x;-2-y; 6-z), \overrightarrow{M B}=(2-x; 4-y; 2-z)$\\
		$\overrightarrow{M A} \cdot \overrightarrow{M B}=(4-x)(2-x)+(-2-y)(4-y)+(6-z)(2-z)$\\
		$=x^2+y^2+z^2-6 x-2 y-8 z+12=(x-3)^2+(y-1)^2+(z-4)^2-12$.\\
		Áp dụng bất đẳng thức B.C.S
		\begin{eqnarray*}
			&&{\left[1^2+2^2+(-3)^2\right]\left[(x-3)^2+(y-1)^2+(z-4)^2\right] \geq[x-3+2(y-1)-3(z-4)]^2} \\
			&\Leftrightarrow& 14\left[(x-3)^2+(y-1)^2+(z-4)^2\right] \geq[x+2 y-3 z+7]^2 \\
			&\Leftrightarrow&(x-3)^2+(y-1)^2+(z-4)^2 \geq \dfrac{(7+7)^2}{14} \\
			&\Leftrightarrow&(x-3)^2+(y-1)^2+(z-4)^2-12 \geq 2.
		\end{eqnarray*}
		$\operatorname{min}(\overrightarrow{M A} \cdot \overrightarrow{M B})=2$ xảy ra khi và chỉ khi 
		$\heva{&x+2 y-3 z-7=0 \\ &\dfrac{x-3}{1}=\dfrac{y-1}{2}=\dfrac{z-4}{-3}}$ $ \Leftrightarrow\heva{&x=4 \\& y=3 \\& z=1.}$\\
		Suy ra $ M(4; 3; 1)$.		
	}	
\end{ex}
\begin{ex}%[2H5C1-5]
	Trong không gian $O x y z$ cho $A(1;-1; 2), B(-2; 0; 3), C(0; 1;-2)$. Gọi $M(a; b; c)$ là điểm thuộc mặt phẳng $(O x y)$ sao cho biểu thức $S=\overrightarrow{M A} \cdot \overrightarrow{M B}+2 \overrightarrow{M B} \cdot \overrightarrow{M C}+3 \overrightarrow{M C} \cdot \overrightarrow{M A}$ đạt giá trị nhỏ nhất. Khi đó $T=12 a+12 b+c$ có giá trị là
	\choice
	{$T=3$}
	{$T=-3$}
	{$T=1$}
	{\True $T=-1$ }
	\loigiai{
		Xét
		\begin{eqnarray*}
			&S&=\overrightarrow{M A} \cdot \overrightarrow{M B}+2 \overrightarrow{M B} \cdot \overrightarrow{M C}+3 \overrightarrow{M C} \cdot \overrightarrow{M A} \\
			&&=(\overrightarrow{M I}+\overrightarrow{I A})(\overrightarrow{M I}+\overrightarrow{I B})+2(\overrightarrow{M I}+\overrightarrow{I B})(\overrightarrow{M I}+\overrightarrow{I C})+3(\overrightarrow{M I}+\overrightarrow{I C})(\overrightarrow{M I}+\overrightarrow{I A}) \\
			&&=6 M I^2+\overrightarrow{M I}(4 \overrightarrow{I A}+3 \overrightarrow{I B}+5 \overrightarrow{I C})+\overrightarrow{I A}\overrightarrow{I B}+2 \overrightarrow{I B}\overrightarrow{I C}+3 \overrightarrow{I C} \overrightarrow{I A}
		\end{eqnarray*}
		Gọi $I$ là điểm thỏa mãn
		$4 \overrightarrow{I A}+3 \overrightarrow{I B}+5 \overrightarrow{I C}=\overrightarrow{0}$ 
		$\Rightarrow
		\heva{&x_I=\dfrac{4 x_A+3 x_B+5 x_C}{12} \\
			&y_I=\dfrac{4 y_A+3 y_B+5 y_C}{12} \\
			&z_I=\dfrac{4 x_A+3 z_B+5 z_C}{12}.}$\\
		Suy ra $I\left(\dfrac{-2}{12}; \dfrac{1}{12}; \dfrac{7}{12}\right).$\\
		Mà $(4 \overrightarrow{I A}+3 \overrightarrow{I B}+5 \overrightarrow{I C})=\overrightarrow{0}$ và $ \overrightarrow{I A} \overrightarrow{I B}+2 \overrightarrow{I B} \overrightarrow{I C}+3 \overrightarrow{I C} \overrightarrow{I A}=const$.\\
		Nên $S_{\min} \Leftrightarrow M I_{\min}$.\\
		Suy ra $M$ là hình chiếu của $I$ lên mặt $(Oxy)$ hay
		$ M\left(-\dfrac{2}{12}; \dfrac{1}{12}; 0\right).$\\ Vậy $T=12 a+12 b+c=-1$.
	}	
\end{ex}