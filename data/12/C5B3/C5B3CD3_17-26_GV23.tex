\begin{ex}%[2H5H3-2]
Trong KG $Oxyz$, cho mặt phẳng $(P)\colon x+2 y-2 z+3=0$ và mặt cầu $(S)$ có tâm $I(0 ;-2 ; 1)$. Biết mặt phẳng $(P)$ cắt mặt cầu $(S)$ theo giao tuyến là một đường tròn có diện tích $2 \pi$. Tính bán kính mặt cầu (\textit{kết quả làm tròn đến hàng phần trăm}).
\shortans{$1{,}73 $}
\loigiai{
Gọi $R,$ $r$ lần lượt là bán kính của mặt cầu và đường tròn giao tuyến. Theo giải thiết ta có
$$
\pi r^{2}=2 \pi \Leftrightarrow r^{2}=2.
$$
Mặt khác $\mathrm{d}(I,(P))=\dfrac{\left| 2\cdot (-2) -2 \cdot 1 +3 \right|}{\sqrt{1^2+ 2^2 +(-2)^2}}=1$ nên $R^{2}=r^{2}+\left[\mathrm d(I,(P))\right]^{2}=3$.\\
Vậy $R=\sqrt{3}\approx 1{,} 73$.}
\end{ex}

\begin{ex}%[2H5H3-2]
Trong không gian, cho bốn mặt cầu có bán kính lần lượt là $2$, $3$, $3$, $2$ (đơn vị độ dài) tiếp xúc ngoài với nhau. Mặt cầu nhỏ nhất tiếp xúc ngoài với cả bốn mặt cầu nói trên có bán kính bằng bao nhiêu? (\textit{kết quả làm tròn đến hàng phần trăm}).
\shortans{$0{,} 55$}
\loigiai{\immini{
Gọi $A$, $B$, $C$, $D$ là tâm bốn mặt cầu.\\
Do các mặt cầu tiếp xúc ngoài với nhau nên không mất tính tổng quát ta giả sử $A B=4$, $A C=B D=A D=B C=5$.\\ Gọi $M, N$ lần lượt là trung điểm của $A B, C D$. Dễ dàng tính được $M N=2 \sqrt{3}$.\\
 Gọi $I$ là tâm mặt cầu nhỏ nhất với bán kính $r$ tiếp xúc với bốn mặt cầu trên. Vì $I A=I B, I C=I D$ nên $I$ nằm trên đoạn $M N$.\\
 Đặt $I N=x$, ta có $I C=\sqrt{3^{2}+x^{2}}=3+r,\break  I A=\sqrt{2^{2}+(2 \sqrt{3}-x)^{2}}=2+r$.
}{\begin{tikzpicture}[scale=0.8, font=\footnotesize, line join=round, line cap=round, >=stealth]
%\draw[color=gray,dash pattern=on 1pt off 1pt,xstep=1.0cm,ystep=1.0cm] (-5.2,-5.2) grid (5.2,5.2);
\coordinate (A) at (0,0);
\coordinate (D) at ($(A)+ (5,0)$);
\coordinate (C) at ($(A)+(-60:3)$);
\coordinate (B) at ($(A)+(60:5)$);
\coordinate (M) at ($(A)!1/2!(B)$);
\coordinate (N) at ($(C)!1/2!(D)$);
\coordinate (I) at ($(M)!1/2!(N)$);
%\draw pic[draw,angle radius=2mm] {right angle = S--H--K}; 
%\draw pic[draw,angle radius=2mm] {right angle = H--I--K};
%\draw pic[draw,angle radius=2mm] {right angle = H--K--D};
%\coordinate (I) at (intersection of O--N and A--B) {};
\draw (B)--(A)--(C)--(D)--cycle (C)--(B); 
\draw[dashed] (A)--(D) (B)--(I)--(A) (C)--(I)--(D) (M)--(I)--(N);

\foreach \p/\r in {A/180,B/90, C/0, D/0,  I/40, M/180, N/0}
\fill (\p) circle (1.5pt) node[shift={(\r:3mm)}]{$\p$};
\end{tikzpicture}	
}
\noindent
Từ đó suy ra $\sqrt{3^{2}+x^{2}}-\sqrt{2^{2}+(2 \sqrt{2}-x)^{2}}=1 \Leftrightarrow x=\dfrac{12 \sqrt{3}}{11}$.\\
Suy ra $r=\sqrt{3^{2}+\left(\dfrac{12 \sqrt{3}}{11}\right)^{2}}-3=\dfrac{6}{11}\approx 0{,} 55$.}
\end{ex}

\begin{ex}%[2H5V3-3]
Trong KG $Oxyz$, cho mặt cầu $(S)\colon x^{2}+y^{2}+(z-\sqrt{2})^{2}=3$. Có tất cả bao nhiêu điểm $A(a ; b ; c)$, $(a, b, c$ là các số nguyên) thuộc mặt phẳng $(O x y)$ sao cho có ít nhất hai tiếp tuyến của $(S)$ đi qua $A$ và hai tiếp tuyến đó vuông góc với nhau?
\shortans{$12$}
\loigiai{
Mặt cầu $(S)$ có tâm $I\left(0 ; 0 ; \sqrt{2}\right)$ và bán kính $R=\sqrt{3}$.\\
 Do $ A \in(O x y) \Rightarrow A(a ; b ; 0)$.
\begin{itemize}
  \item Xét trường hợp $A \in(S)$, ta có $a^{2}+b^{2}=1$.\\
   Lúc này các tiếp tuyến của $(S)$ thuộc tiếp diện của $(S)$ tại $A$ nên có vô số các tiếp tuyến vuông góc nhau.
   \\
   Trường hợp này ta có $4$ cặp giá trị của $(a ; b)$ là  $(0;1)$; $(0;-1)$; $(-1;0)$; $(1;0)$.
\item Xét trường hợp $A$ ở ngoài $(S)$. Khi đó, các tiếp tuyến của $(S)$ đi qua $A$ thuộc mặt nón đỉnh $A$. Nên các tiếp tuyến này chỉ có thể vuông góc với nhau tại $A$.
\immini{Giả sử $A N ; A M$ là các tiếp tuyến của $(S)$ thỏa mãn $A$, $M$,  $N$,  $I$ đồng phẳng ($N$, $M$ là các tiếp điểm).\\
Điều kiện để $AM \perp AN$ là góc ở đỉnh của mặt nón lớn hơn hoặc bằng $90^{\circ}$ hay $\widehat{MAN} \ge 90^\circ$.
}{
\begin{tikzpicture}[line join=round,scale=.5, line cap=round,thick]
				\pgfmathsetmacro\a{sqrt(13)}
				\def\b{5}
				\coordinate (I) at (2,0);
				\coordinate (A) at (-4,0);
				\coordinate (J) at ($(A)!1/2!(I)$);
				\path[name path=c1] (I) circle (\a);
				\path[name path=c2] (J) let \p1=($(J)-(A)$) in circle ({veclen(\x1,\y1)});
				\draw[dashed, name path=i] (I)--(A);
				\path[name intersections={of=c1 and c2, by={M,N}}];
				\draw[dashed, name path=mn](N)--(M);
				\path[name intersections={of=mn and i, by={K}}];
				%\draw[dashed](-1.5,0) ellipse (0.5 and 1.5);
				%\draw[dashed]  (C) arc (90:-90:{0.5} and {3});
				\draw (M) arc (90:270:{0.5} and {2.9});
			%	\coordinate (B) at (150:{0.5} and {2.9});
				\draw[dashed]  (M) arc (90:-90:{0.5} and {2.9});
			\draw[fill=black] (A) circle(1pt);
				\path (I) circle (\a);
				\begin{scope}
				%\clip (I') circle (\b);
				\draw[dashed] (I) circle (\a);
				\end{scope}
				%\draw (I) circle (\a);
				\begin{scope}
			%	\clip (I') circle (\b);
				\draw[dashed] (I) circle (\a);
				\end{scope}
				\path (3,0) circle (4.25);
				\begin{scope}
				\draw (I) circle (\a);
				\end{scope}
				\draw[dashed] (I)--(M);
				\draw  (M)--(A)--(N);
				\foreach \i/\g in {I/0,M/90,N/-90, A/180}{\draw[fill=black](\i) circle (1pt) ($(\i)+(\g:3mm)$) node[scale=.6]{$\i $};};
				\end{tikzpicture}}
				\noindent
				$$\Leftrightarrow \heva{&I A>R \\ & 90^\circ\ge  \widehat{MAI} \ge 45^\circ} \Leftrightarrow \heva{&IA > \sqrt{3} \\ &\sin \widehat{MAI}= \dfrac{R}{IA}\ge \dfrac{\sqrt{2}}{2}}\Leftrightarrow\heva{&IA > \sqrt{3}\\& IA \le \sqrt{6}}\Leftrightarrow \heva{&a^2 +b^2 > 1 \\& a^2 + b^2 \le 4} $$
		\noindent	 Vì $a, b$ là các số nguyên nên ta có các cặp nghiệm $(a ; b)$ là
$(0 ; 2)$, $(0 ;-2),$ $(2 ; 0),$ $(-2 ; 0),$ $(1 ; 1)$, $(-1 ;-1)$, $(-1 ; 1)$, $(1 ;-1)$.
\end{itemize}
Vậy có $12$ điểm $A$ thỏa mãn yêu cầu.}
\end{ex}

\begin{ex}%[2H5V3-3]
Trong KG $Oxyz$, cho mặt cầu $(S)\colon x^{2}+y^{2}+(z-1)^{2}=5$. Có tất cả bao nhiêu điểm $A(a, b, c)$ ($a, b, c$ là các số nguyên) thuộc mặt phẳng $(Oxy)$ sao cho có ít nhất hai tiếp tuyến của $(S)$ đi qua $A$ và hai tiếp tuyến đó vuông góc với nhau?
\shortans{$20$}
\loigiai{
Mặt cầu có tâm $I(0 ; 0 ; 1)$, bán kính $R=\sqrt{5}$.
\\
 Do $ A \in(O x y) \Rightarrow A(a ; b ; 0)$.
\begin{itemize}
  \item Xét trường hợp $A \in(S)$, ta có $a^{2}+b^{2}=4$.\\
   Lúc này các tiếp tuyến của $(S)$ thuộc tiếp diện của $(S)$ tại $A$ nên có vô số các tiếp tuyến vuông góc nhau.
   \\
   Trường hợp này ta có $4$ cặp giá trị của $(a ; b)$ là  $(0;2)$; $(0;-2)$; $(-2;0)$; $(2;0)$.
\item Xét trường hợp $A$ ở ngoài $(S)$. Khi đó, các tiếp tuyến của $(S)$ đi qua $A$ thuộc mặt nón đỉnh $A$. Nên các tiếp tuyến này chỉ có thể vuông góc với nhau tại $A$.
\immini{Giả sử $A N ; A M$ là các tiếp tuyến của $(S)$ thỏa mãn $A, \ M,\  N,  \ I$ đồng phẳng ($N ; M$ là các tiếp điểm).\\
Điều kiện để $AM \perp AN$ là góc ở đỉnh của mặt nón lớn hơn hoặc bằng $90^{\circ}$ hay $\widehat{MAN} \ge 90^\circ$.
}{
\begin{tikzpicture}[line join=round,scale=.5, line cap=round,thick]
				\pgfmathsetmacro\a{sqrt(13)}
				\def\b{5}
				\coordinate (I) at (2,0);
				\coordinate (A) at (-4,0);
				\coordinate (J) at ($(A)!1/2!(I)$);
				\path[name path=c1] (I) circle (\a);
				\path[name path=c2] (J) let \p1=($(J)-(A)$) in circle ({veclen(\x1,\y1)});
				\draw[dashed, name path=i] (I)--(A);
				\path[name intersections={of=c1 and c2, by={M,N}}];
				\draw[dashed, name path=mn](N)--(M);
				\path[name intersections={of=mn and i, by={K}}];
				%\draw[dashed](-1.5,0) ellipse (0.5 and 1.5);
				%\draw[dashed]  (C) arc (90:-90:{0.5} and {3});
				\draw (M) arc (90:270:{0.5} and {2.9});
			%	\coordinate (B) at (150:{0.5} and {2.9});
				\draw[dashed]  (M) arc (90:-90:{0.5} and {2.9});
			\draw[fill=black] (A) circle(1pt);
				\path (I) circle (\a);
				\begin{scope}
				%\clip (I') circle (\b);
				\draw[dashed] (I) circle (\a);
				\end{scope}
				%\draw (I) circle (\a);
				\begin{scope}
			%	\clip (I') circle (\b);
				\draw[dashed] (I) circle (\a);
				\end{scope}
				\path (3,0) circle (4.25);
				\begin{scope}
				\draw (I) circle (\a);
				\end{scope}
				\draw[dashed] (I)--(M);
				\draw  (M)--(A)--(N);
				\foreach \i/\g in {I/0,M/90,N/-90, A/180}{\draw[fill=black](\i) circle (1pt) ($(\i)+(\g:3mm)$) node[scale=.6]{$\i $};};
				\end{tikzpicture}}
				\noindent
				$$\Leftrightarrow \heva{&I A>R \\ & 90^\circ\ge  \widehat{MAI} \ge 45^\circ} \Leftrightarrow \heva{&IA > \sqrt{3} \\ &\sin \widehat{MAI}= \dfrac{R}{IA}\ge \dfrac{\sqrt{2}}{2}}\Leftrightarrow\heva{&IA > \sqrt{5}\\& IA \le \sqrt{10}}\Leftrightarrow \heva{&a^2 +b^2 > 4 \\& a^2 + b^2 \le 9.} $$
		\noindent	 Vì $a, b$ là các số nguyên nên ta có các cặp nghiệm $(a ; b)$ là
$(0 ; \pm 3)$, $(\pm 1 ;\pm 2),$ $(\pm 2 ; \pm 2),$ $(\pm 2 ; \pm 1),$ $(\pm 3 ; 0)$.
\end{itemize}
Vậy có $20$ bộ số thỏa mãn yêu cầu.}
\end{ex}

\begin{ex}%[2H5H3-2]
Trong KG $Oxyz$, cho điểm $H(1 ; 2 ;-2)$. Mặt phẳng $(\alpha)$ đi qua $H$ và cắt các trục $O x$, $O y$, $O z$ lần lượt tại các điểm $A$, $B$, $C$ sao cho $H$ là trực tâm của tam giác $ABC$. Tính bán kính mặt cầu ngoại tiếp tứ diện $OABC$ (\textit{làm tròn kết quả đến hàng phần trăm}).
\shortans{$5{,} 51$}
\loigiai{
Mặt phẳng $(\alpha)$ cắt các trục $Ox$, $Oy$, $O z$ lần lượt tại các điểm $A(a ; 0 ; 0)$, $B(0 ; b ; 0)$, $C(0 ; 0 ; c)$. Do $H$ là trực tâm tam giác $A B C$ nên $a, b, c \neq 0$ và $OH \perp (ABC)$.\\
Khi đó $( \alpha) \colon \heva{& \text{qua } H(1;2;-2)\\& \text{một véc-tơ pháp tuyến } \vec{OH}= (1;2;-2)}$ có phương trình $$x + 2x - 2z -9=0.$$
Suy ra $A(9 ; 0 ; 0)$, $B\left(0 ; \dfrac{9}{2} ; 0\right), C\left(0 ; 0 ;-\dfrac{9}{2}\right)$.
\\
Khi đó, giả sử mặt cầu ngoại tiếp tứ diện $O A B C$ có phương trình là:
$$x^{2}+y^{2}+z^{2}-2 a' x-2 b' y-2 c' z+d=0$$ 
Với $\left(a'\right)^{2}+\left(b'\right)^{2}+\left(c'\right)^{2}-d>0$.\\
Vì 4 điểm $O,$ $A$, $B$, $C$ thuộc mặt cầu nên ta có hệ phương trình
$$
\heva{& 
 d = 0  \\&
 - 1 8 a ^ { \prime } + d = - 8 1  \\&
 - 9 b ^ { \prime } + d = - \dfrac { 8 1 } { 4 } \\&
 9 c ^ { \prime } + d = - \dfrac { 8 1 } { 4 } } \Leftrightarrow \heva{& 
d=0 \\&
a'=\dfrac{9}{2} \\&
b'=\dfrac{9}{4} \\&
c'=-\dfrac{9}{4}.}
$$
Vậy bán kính  của mặt cầu là $R=\sqrt{\left(\dfrac{9}{2}\right)^{2}+\left(\dfrac{9}{4}\right)^{2}+\left(\dfrac{9}{4}\right)^{2}-0}=\dfrac{9 \sqrt{6}}{4} \approx 5{,}51$.}
\end{ex}

\begin{ex}%[2H5C3-2]
Trong không gian $O x y z$, mặt cầu $(S)$ đi qua điểm $A(2 ;-2 ; 5)$ và tiếp xúc với ba mặt phẳng $(P)\colon x=1$, $(Q)\colon y=-1$ và $(R)\colon z=1$ có bán kính bằng bao nhiêu?
\shortans{$3$}
\loigiai{
Gọi $I(a ; b ; c)$ và $R$ là tâm và bán kính của $(S)$. Khi đó ta có
$$R=I A=\mathrm d(I ;(P))= \mathrm d(I ;(Q))=\mathrm d(I ;(R)) \Leftrightarrow R=|a-1|=|b+1|=|c-1| \quad (*).$$ 
Lại do mặt cầu đi qua $A(2;-2;5)$ nên ta suy ra $a>1$, $b<-1$ và $c>1$.\\
Do đó $(*) \Leftrightarrow \heva{& a-1= -b-1\\ & a-1 = c-1}\Leftrightarrow \heva{& b= -a\\& c=a.}$\\
Suy ra $(S) \colon (x-a)^2 + (y+a)^2 + (z-a)^2 = (a-1)^2$.\\
Mà $A(2;-2;5) \in (S)$ nên $(2-a)^2 + (-2+a)^2 + (5-a)^2 = (a-1)^2\Leftrightarrow a=4$.\\
Vậy $R= |a-1|=3.$}
\end{ex}

\begin{ex}%[2H5C3-2]
Trong KG $Oxyz$, xét số thực $m \in(0 ; 1)$ và hai mặt phẳng $(\alpha)\colon 2 x-y+2 z+10=0$ và $(\beta)\colon \dfrac{x}{m}+\dfrac{y}{1-m}+\dfrac{z}{1}=1$. Biết rằng, khi $m$ thay đổi có hai mặt cầu cố định tiếp xúc đồng thời với cả hai mặt phẳng $(\alpha)$, $(\beta)$. Tổng bán kính của hai mặt cầu đó bằng bao nhiêu?
\shortans{$ 9$}
\loigiai{
Gọi $I(a ; b ; c)$ là tâm mặt cầu.\\
Ta có $(\beta) \colon \dfrac{x}{m}+ \dfrac{y}{1-m}+ z=1 \Leftrightarrow  (1-m)x + m y + (m-m^2) z + m^2 -m=0$.\\
Theo giả thiết 
\allowdisplaybreaks
\begin{eqnarray*}
	 & R  =\mathrm d (I, (\beta))&=\dfrac{\left|(1-m)a + mb + (m-m^2) c + m^2 -m \right|}{\sqrt{(1-m)^2 + m^2 +(m-m^2)^2 }}\\
	 &&=\dfrac{\left|(1-m)a + mb + (m-m^2) c + m^2 -m \right|}{m^2 -m +1}.
\end{eqnarray*}
Do $R$ cố định với mọi $m$ nên tồn tại hằng số $k$ thỏa mãn 
\allowdisplaybreaks
\begin{eqnarray*}
	 &&(1-m)a + mb + (m-m^2) c + m^2 -m  = k \cdot (m^2 -m +1), \, \forall m \\
	 &\Leftrightarrow & (1-c)m^2 + (-a + b +c-1) m + a = km^2 - m k + k, \, \forall m\\
	 & \Leftrightarrow & \heva{& 1-c = k \\ & -a+b + c-1= -m \\& a = k}\\
	 & \Leftrightarrow & \heva{& a =k \\ & b= k \\& c= 1-k.}
\end{eqnarray*}
Suy ra $I (k; k; 1-k)$.\\
Mặt khác $R = \mathrm d (I, (\alpha))$ nên $$R=\dfrac{\left| 2k - k + 2 (1-k) +10\right|}{\sqrt{2^2 + 1^2 + 2^2}}= \dfrac{\left| -k +12\right|}{3}.$$
Khi đó $\dfrac{\left|-k +2 \right|}{3} = k \Leftrightarrow \hoac{& k=-6 &&\Rightarrow  R= 6\\& k=3 &&\Rightarrow R=3.}$\\
Vậy $6+3 =9. $}
\end{ex}

\begin{ex}%[2H5C3-2]
Trong KG $Oxyz$, cho điểm $A(2 ; 11 ;-5)$ và mặt phẳng $(P)\colon 2 m x+\left(m^{2}+1\right) y+\left(m^{2}-1\right) z-10=0$. Biết rằng khi $m$ thay đổi, tồn tại hai mặt cầu cố định tiếp xúc với mặt phẳng $(P)$ và cùng đi qua $A$. Tổng bán kính của hai mặt cầu đó bằng bao nhiêu? (\textit{kết quả làm tròn đến hàng phần chục}).
\shortans{$17$}
\loigiai{
Gọi $I\left(x_{0} ; y_{0} ; z_{0}\right)$ là tâm của mặt cầu $(S)$ cố định và $R$ là bán kính của mặt cầu $(S)$.
\\
Ta có
\allowdisplaybreaks
\begin{eqnarray*}
	&R=\mathrm  d(I,(P))& =\dfrac{\left|2 m x_{0}+\left(m^{2}+1\right) y_{0}+\left(m^{2}-1\right) z_{0}-10\right|}{\sqrt{4 m^{2}+\left(m^{2}+1\right)^{2}+\left(m^{2}-1\right)^{2}}}\\
	&&=\dfrac{\left|2 m x_{0}+\left(m^{2}+1\right) y_{0}+\left(m^{2}-1\right) z_{0}-10\right|}{\sqrt{2}\cdot \left(m^{2}+1\right)}
\end{eqnarray*}
Do tại hai mặt cầu cố định tiếp xúc với $(P)$ nên tồn tại số thực $k$ thỏa mãn \allowdisplaybreaks
\begin{eqnarray*}
	&& 2m x_0 + (m^2 +1 ) y_0 + (m^2 -1) z_0 -10 = k  \cdot (m^2 +1) \, \forall m\\
	& \Leftrightarrow & (y_0 + z_0) m^2 + 2x_0 m + y_0 -z_0 -10 = k m^2 + k \, \forall m\\
	& \Leftrightarrow & \heva{& y_0+ z_0 =  k \\ & 2x_0 =0 \\& y_0-z_0-10=  k }\\
	& \Leftrightarrow & \heva{ &x_0=0 \\ & y_0= 5 +k \\
	& z_0 = -5.}
\end{eqnarray*} 
Suy ra $I \left( 0; 5 + k ; -5 \right)$ và $R = \dfrac{|k|}{\sqrt{2}}$.\\
Mặt khác 
\allowdisplaybreaks
\begin{eqnarray*}
	&R= IA &\Leftrightarrow \dfrac{|k|}{\sqrt{2}}= \sqrt{2^2 + \left(6-k \right)^2}\\
	&&\Leftrightarrow k^2 = 2 \cdot \left(40 + k^2 - 12 k \right) \\
	& &\Leftrightarrow  k^2 -24k + 80=0\\
	&  &\Leftrightarrow  \hoac{& k=4  &&\Rightarrow R= 2\sqrt{2} \\& k=20 &&\Rightarrow R= 10\sqrt{2}.}
\end{eqnarray*}
Vậy tổng hai bán kính của hai mặt cầu là $12 \sqrt{2}\approx 17.$}\end{ex}



\begin{ex}%[2H5C3-2]
 Trong KG $Oxyz$ cho $A(-3 ; 1 ; 1)$,  $B(1 ;-1 ; 5)$ và mặt phẳng $(P)\colon 2 x-y+2 z+11=0$. Mặt cầu $(S)$ đi qua hai điểm $A$, $B$ và tiếp xúc với $(P)$ tại điểm $C$. Biết $C$ luôn thuộc một đường tròn $(T)$ cố định. Tính bán kính $r$ của đường tròn $(T)$.
\shortans{$4$}
\loigiai{\immini{Ta có $\overrightarrow{A B}=(4 ;-2 ; 4)$ và $(P)$ có véc-tơ pháp tuyến \break $\vec{n}=(2 ;-1 ; 2)$. Do $\vec{AB}= 2 \cdot \vec{n}$ nên $AB$ vuông góc với $(P)$.
\\
Gọi $M$ là trung điểm $AB$. Do $A$, $B \in (S)$ nên $IM \perp AB$ hay $I$ thuộc mặt phẳng trung trực $(Q)$ của đoạn thẳng $AB$.\\
Ta có $M(-1; 0; 3)$ và $(Q) \colon 2x - y + 2z -4=0$.}{
\begin{tikzpicture}[line join=round,scale=.5, line cap=round,thick]
				\coordinate (H) at (0,0);
				\coordinate (C) at (5,0);
				\coordinate (I) at ($(C)+(90:3)$);
				\draw[name path=c1] (I) circle (3);
				\coordinate (A) at ($(I)+(140:3)$);
				\coordinate (B) at ($(I)+(-140:3)$);
				\coordinate (M) at ($(A)!1/2!(B)$);
				\draw  (A)--(B) (M)--(I)--(C);
				\draw (1,0)--(9,0);
				\node[above]  at (9,0){$(P)$}; 
				\draw[dashed] ;
				\foreach \i/\g in {I/90,C/-90, B/180, A/90,M/180}{\draw[fill=black](\i) circle (1pt) ($(\i)+(\g:3mm)$) node[scale=.6]{$\i $};};
				\end{tikzpicture}} \noindent
Suy ra $R = \mathrm d ((P), (Q))= \dfrac{|11+4|}{\sqrt{2^2 +(-1)^2 + 2^2}}=3$.
\\
Do $(S)$ tiếp xúc $(P)$ tại $C$ nên $\mathrm d (I, AB) = \mathrm d (C, AB)\Leftrightarrow \mathrm d (C, AB)= \sqrt{R^2 - \dfrac{AB^2}{4}}= 4$.\\
 Vậy $C$ luôn thuộc một đường tròn $(T)$ cố định có bán kính $r=4$.}
\end{ex}



\begin{ex}%[2H5C3-3]
Trong KG $Oxyz$, cho các điểm $M(2 ; 1 ; 4)$, $N(5 ; 0 ; 0)$, $P(1 ;-3 ; 1)$. Gọi $I(a ; b ; c)$ là tâm của mặt cầu tiếp xúc với mặt phẳng $(Oyz)$ đồng thời đi qua các điểm $M$, $N$, $P$. Tìm $c$ biết rằng $a+b+c<5$.
\shortans{$2$}
\loigiai{
Phương trình mặt cầu $(S)$ tâm $I(a ; b ; c)$.\\
Do $(S)$ tiếp xúc với $(Oyz)$ nên $R= |a|$.\\
Suy ra $(S) \colon (x-a)^2 + (y-b)^2 + (z-c)^2 =a^2$.\\
Mặt khác, $M$, $N$, $P \in (S)$ nên 
\allowdisplaybreaks
\begin{eqnarray*}
	&& \heva{&  (2-a)^2 + (1-b)^2 + (4-c)^2 = a^2\\
& (5-a)^2 + b^2 + c^2 = a^2 \\
& (1-a)^2 + (3+b)^2 + (1-c)^2 =a^2}\\
& \Leftrightarrow & \heva{& b^2 + c^2 -2b - 8c - 4a =-21\\
& b^2 + c^2 -10a = -25\\
& b^2 + c^2 + 6b -2c -2a = -11}	\\
& \Leftrightarrow & \heva{& 6a -2b - 8c = 4\\& 8a + 6b -2c = 14\\ & b^2 + c^2 -10a =-25}\\
& \Leftrightarrow & \heva{& c=a-1\\&b= 2-a \\& (2-a)^2 + (a-1)^2 -10=-25 }\\
& \Leftrightarrow & \hoac{& \heva{& a= 3\\&b=-1\\&c=2}\\&\heva{&a=5 \\&b=-3\\&c=4.}}
\end{eqnarray*}Do $a+b+c <5$ nên $\heva{&a=3\\& b=-1\\&c =2.}$ 
}
\end{ex}
\Closesolutionfile{ans}
% \indapan{6}{ans/ans-2C5B3CD3-D2-KQ}
	