
\Opensolutionfile{ans}[ans/C5B3CD7_37-40]
\TN
\begin{ex}%[2H5V3-4]
	Trong không gian với hệ tọa độ $Oxyz$, cho hai đường thẳng $d_1\colon \heva{&x=2t \\&y=t\\&z=4}$ và $d_2\colon \heva{&x=3-t'\\&y=t'\\&z=0}$.  Viết phương trình mặt cầu có bán kính nhỏ nhất tiếp xúc với cả hai đường thẳng $d_1$ và $d_2$.
	\choice[2]
	 {$(S)\colon (x+2)^2+(y+1)^2+(z+2)^2=4$}
	{$(S)\colon (x-2)^2+(y-1)^2+(z-2)^2=16$}
	{\True $(S)\colon (x-2)^2+(y-1)^2+(z-2)^2=4$}
	{$(S)\colon (x+2)^2+(y+1)^2+(z+2)^2=16$} 
	\loigiai{
		Đường thẳng $d_1$ có véc-tơ chỉ phương $\vec{u}_1=(2;1;0)$.\\
		Đường thẳng $d_2$ có véc-tơ chỉ phương $\vec{u}_2=(1;-1;0)$.\\
		Để phương trình mặt cầu $(S)$ có bán kính nhỏ nhất và đồng thời tiếp xúc với cả hai đường thẳng $d_1$ và $d_2$ khi và chỉ khi tâm mặt cầu $(S)$ nằm trên đoạn thẳng vuông góc chung của $2$ đường thẳng $d_1$ và $d_2$, đồng thời là trung điểm của đoạn thẳng vuông góc chung. \\
		Gọi điểm $M(2t;t;4)$ thuộc $d_1$; \\
		Gọi $N(3-t';t';0)$ điểm thuộc $d_2$ với $MN$ là đoạn vuông góc chung của $d_1$ và $d_2$. \\
		Ta có $\overrightarrow{MN}=(3-t'-2t;t'-t;-4)$.\\
		$MN$ là đoạn thẳng vuông góc chung
		\allowdisplaybreaks
		\begin{eqnarray*}
			&\Leftrightarrow& \heva{&\overrightarrow{MN}\cdot\vec{u}_1 =0\\ &\overrightarrow{MN}\cdot\vec{u}_2=0}\\
			&\Leftrightarrow & \heva{&2\cdot(3-t'-2t)+t'-t=0\\&(-1)\cdot(3-t'-2t)+t'-t =0)}\\
			&\Leftrightarrow &\heva{&t'+5t=0\\&2t'+t=3} \Leftrightarrow \heva{&t=1\\ &t'=1} \Rightarrow \heva{&M(2;1;4)\\& N(2;1;0).}
		\end{eqnarray*}\noindent
		 Gọi điểm $I$ là tâm mặt cầu $(S)$,  do đó điểm $I$ là trung điểm $MN$.\\
		 Suy ra mặt cầu $(S)\colon (x-2)^2+(y-1)^2+(z-2)^2=4$.} 
\end{ex} 
\begin{ex}%[2H5V3-4]
	Trong không gian $Oxyz$ cho hai đường thẳng chéo nhau 
	$d_1\colon \heva{&x=4-2t\\&y=t\\&z=3}$ và 
	$d_2\colon \heva{&x=1\\&y=t'\\ z&=-t'}$ $(t$, $t'\in \mathbb{R})$. Phương trình mặt cầu có bán kính nhỏ nhất tiếp xúc với cả hai đường thẳng $(d_1)$, $(d_2)$ là
	\choice
	 {$\left(x+\dfrac{3}{2}\right)^{2}+y^{2}+(z+2)^{2}=\dfrac{9}{4}$}
	{$\left(x+\dfrac{3}{2}\right)^{2}+y^{2}+(z+2)^{2}=\dfrac{3}{2}$}
	{\True $\left(x-\dfrac{3}{2}\right)^{2}+y^{2}+(z-2)^{2}=\dfrac{9}{4}$}
	{$\left(x-\dfrac{3}{2}\right)^{2}+y^{2}+(z-2)^{2}=\dfrac{3}{2}$} 
	\loigiai{
		Mặt cầu có bán kính nhỏ nhất tiếp xúc với $(d_1)$, $(d_2)$ là mặt cầu có đường kính là đoạn vuông góc chung của $(d_1)$, $(d_2)$.\\
		Lấy $A(4-2t;t; 3)\in d_1$;  $B\left(1;t';-t'\right)\in d_2$.\\
		 $AB$ là đoạn vuông góc chung khi và chỉ khi\\
		  $\heva{&\overrightarrow{AB}\cdot \overrightarrow{u}_{d_1}=0\\& \overrightarrow{AB}\cdot \overrightarrow{u}_{d_2}=0} \Leftrightarrow\heva{&-5t+t'=-6\\&-t+2t'=-3} \Leftrightarrow \heva{&t=1\\& t'=-1.}$\\
		Khi đó $A(2;1;3)$;  $B(1;-1;1)$. Suy ra tâm $I\left(\dfrac{3}{2};0;2\right)$, bán kính $R=\dfrac{3}{2}$.
		} 
\end{ex} 
\begin{ex}%[2H5V3-4]
	Trong không gian $Oxyz$, cho hai đường thẳng $\Delta_1\colon \dfrac{x-4}{3}=\dfrac{y-1}{-1}=\dfrac{z+5}{-2}$ và $\Delta_2\colon \dfrac{x-2}{1}=\dfrac{y+3}{3}=\dfrac{z}{1}$. Trong tất cả mặt cầu tiếp xúc với cả hai đường thẳng $\Delta_1$ và $\Delta_2$. Gọi $(S)$ là mặt cầu có bán kính nhỏ nhất. Bán kính của mặt cầu $(S)$ là
	\choice
	{$\sqrt{12}$}
	{\True $\sqrt{6}$}
	{$\sqrt{24}$}
	{$\sqrt{3}$} 
	\loigiai{
		\begin{center}
			\begin{tikzpicture}[scale=0.8, font=\footnotesize,line join=round, line cap=round, >=stealth]
			\def\R{2}
			\def\S{4}
			\def\h{0}
			\pgfmathsetmacro\a{sqrt(\R^2-\h^2)}
			\def\b{0.8}
			\coordinate (O) at (0,0);
			\coordinate (J) at (5,0);
			\coordinate (M) at (0,\R);
			\coordinate (N) at (0,-\R);
			\coordinate (A) at (3.5,3.2);
			\coordinate (B) at (3.5,-3.2);
			\coordinate (E) at ($(M)!-1!(A)$);\coordinate (X) at ($(M)!1.4!(A)$);
			\coordinate (G) at ($(N)!-1!(B)$);\coordinate (Y) at ($(N)!1.4!(B)$);
			\coordinate (I) at ($(O)-(0,\h)$); % \coordinate (I) at ($(O)-(0,\h)$);
			\coordinate (C) at ($(I)-(\a,0)$);
			\coordinate (D) at ($(I)+(\a,0)$);
			\draw[dashed,thin] (C) arc (180:0:\a cm and \b cm) (O)--(I);
			\draw (O) circle (\R cm)
			(C) arc (-180:0:\a cm and \b cm);
			\draw[dashed,thin] (M)--(N);
			\draw[dashed,thin] (A)--(J)--(B)--(A);
			\draw (J) circle (\S cm);
			\foreach \i/\g in {I/30,M/90,N/-90,J/180,A/10,B/-90}{\draw[fill=black](\i) circle (1.5pt) ($(\i)+(\g:3mm)$) node[scale=1]{$\i$};}
			\draw (E)--(X);
			\draw (G)--(Y);
			\pic[draw,thin,angle radius=2mm] {right angle = A--M--I};
			\pic[draw,thin,angle radius=2mm] {right angle = B--N--I};
			\node at (E){$\Delta_1$};
			\node at (G){$\Delta_2$};
		\end{tikzpicture}
		\end{center}
		Ta có $\Delta_1 \colon \heva{&x=4+3t_1\\&y=1-t_1\\&z=-5-2t_1}$, $\Delta_2 \colon\heva{&x=2+t_2\\&y=-3+3t_2\\&z=t_2}$ $(t_1$, $t_2 \in \mathbb{R})$.\\
		Gọi $\vec{u}_1=(3; -1; -2)$, $\vec{u}_2=(1;3;1)$ lần lượt là véc-tơ chỉ phương của hai đường thẳng.\\
		Gọi $M \in \Delta_1 \Rightarrow M\left(4+3t_1;1-t_1;-5-2t_1\right)$;  $N\in \Delta_2 \Rightarrow N\left(2+t_2;3t_2-3;t_2\right)$.\\
		Suy $\overrightarrow{MN}=\left(t_2-3t_1-2;3t_2+t_1-4;t_2+2t_1+5\right)$.\\
		$MN$ là đoạn vuông góc chung khi và chỉ khi \\
		$$\heva{&\overrightarrow{MN}\cdot \vec{u}_1=0\\&\overrightarrow{MN}\cdot \vec{u}_2=0} \Leftrightarrow\heva{&7t_1+t_2=-6\\&2t_1+11t_2=9} \Leftrightarrow\heva{&t_1=-1\\&t_2=1.}$$
		$\overrightarrow{MN}=(2;-2;4) \Rightarrow MN=\sqrt{6}$.\\
		Giả sử $(S)$ là mặt cầu tâm $J$ đường kính $d$ tiếp xúc với lần lượt $\Delta_1$, $\Delta_2$ tại $A$, $B$.\\ Khi đó $JA+JB\ge AB$, hay $d\ge AB \ge MN \Rightarrow d\ge MN$.\\ Vậy đường kính $d$ nhỏ nhất khi $d=MN$.\\
		Suy ra mặt cầu $(S)$ có bán kính nhỏ nhất $r=\dfrac{MN}{2}=\sqrt{6}$.
	} 
\end{ex} 
\begin{ex}%[2H5H3-4]
	Trong không gian $Oxyz$ cho mặt cầu $(x-3)^{2}+(y-1)^{2}+z^{2}=4$ và đường thẳng $d\colon\heva{&x=1+2t\\&y=-1+t\\&z=-t}$, $t\in \mathrm{R}$. Mặt phẳng chứa  $d$  và cắt $(S)$ theo một đường tròn có bán kính nhỏ nhất có phương trình là
	\choice
	{\True $y+z+1=0$}
	{$x+3y+5z+2=0$}
	{$x-2y-3=0$}
	{$3x-2y-4z-8=0$}
	\loigiai{
	 Gọi $H$ là hình chiếu vuông góc của tâm mặt cầu $I(3;1;0)$ lên $d$, từ đó ta tìm được $H(3;0;-1)$. Thấy $IH\le R$ nên $d$ cắt $(S)$.\\
	 Vậy mặt phẳng cần tìm nhận $\overrightarrow{IH}=(0;-1;-1)$ làm véc-tơ pháp tuyến nên phương trình mặt phẳng là $y+z+1=0$.} 
\end{ex} 
\begin{ex}%[2H5V3-4]
	Trong không gian với hệ trục tọa độ $Oxyz$, cho mặt phẳng $(P)\colon 2x-y+2z+2=0$ và mặt cầu $(S)\colon (x-1)^2+(y+2)^2+z^2=4$ có tâm $I$. Gọi tọa độ điểm $M\left(x_0; y_0;z_0\right)$ thuộc $(P)$ sao cho đoạn $IM$ ngắn nhất. Tổng $T=x_0^2+y_0^2+z_0^2$ bằng
	\choice
	{$\dfrac{7}{3}$}
	{\True $\dfrac{11}{3}$}
	{$14$}
	{$\dfrac{16}{3}$} 
	\loigiai{
		Ta có tâm $I(1;-2;0)$ và bán kính $R=2$. Khoảng cách từ $I$ đến mặt phẳng $(P)$ ngắn nhất khi $M$ là hình chiếu của $I$ lên mặt phẳng $(P)$.\\
		Đường thẳng đi qua $I$ và vuông góc với mặt phẳng $(P)$ có phương trình tham số là $\heva{&x=1+2t\\&y=-2-t\\&z=2t.}$\\
		Khi đó tọa độ của $M$ là nghiệm của hệ phương trình\\
		$\heva{&x=1+2t\\&y=-2-t\\&z=2t\\&2x-y+2 z+2=0} \Leftrightarrow\heva{&x=1+2t\\ &y=-2-t\\&z=2t\\&2(1+2t)-(-2-t)+2(2t)+2=0} \Leftrightarrow \heva{&x=-\dfrac{1}{3}\\&y=-\dfrac{4}{3}\\&z=-\dfrac{4}{3}\\& t=-\dfrac{2}{3}}.$\\
		Tổng $T=x_0^2+y_0^2+z_0^2=\dfrac{11}{3}$.} 
\end{ex} 
\begin{ex}%[2H5V1-5]
	Trong không gian với hệ tọa độ $Oxyz$, cho mặt cầu $(S)$ tâm $I(1;-2;1)$; bán kính $R=4$ và đường thẳng $d\colon \dfrac{x}{2}=\dfrac{y-1}{-2}=\dfrac{z+1}{-1}$. Mặt phẳng $(P)$ chứa $d$ và cắt mặt cầu $(S)$ theo một đường
	tròn có diện tích nhỏ nhất. Hỏi trong các điểm sau điểm nào có khoảng cách đến mặt phẳng $(P)$ lớn nhất?
	\choice
	{\True $O(0;0;0)$}
	{$A\left(1;\dfrac{3}{5};-\dfrac{1}{4}\right)$}
	{$B(-1;-2;-3)$ }
	{$C(2;1;0)$} 
	\loigiai{
		Gọi $H(2t;1-2t;-1-t)$ là hình chiếu của $I$ lên đường thẳng $d$.\\
		Ta có $\overrightarrow{IH}\cdot \overrightarrow{u}_{d}=0 \Rightarrow 2(2 t-1)-2(3-2t)-(-2-t)=0 \Leftrightarrow t=\dfrac{2}{3} \Rightarrow H\left(\dfrac{4}{3};-\dfrac{1}{3};-\dfrac{5}{3}\right)$.\\
		Vì $IH=\sqrt{10}<4=R\Rightarrow d$ cắt mặt cầu $(S)$ tại $2$ điểm phân biệt.\\
		Mặt phẳng $(Q)$ bất kì chứa $d$ luôn cắt $(S)$ theo một đường tròn bán kính $r$.\\
		Khi đó $r^2=R^2-\mathrm{d}^2\big(I,(Q)\big)\ge R^2-\mathrm{d}^2\big(I,d\big)=16-10=6$.\\
		Do vậy mặt phẳng $(P)$ chứa $d$ cắt mặt cầu theo một đường tròn có diện tích nhỏ nhất khi và chỉ khi $d(I,(P))=d(I, d)$ hay mặt phẳng $(P)$ đi qua $H$ nhận $\overrightarrow{IH}=\left(\dfrac{1}{3};  \dfrac{5}{3}; -\dfrac{8}{3}\right)$ làm véc-tơ pháp tuyến, do đó $(P)$ có phương trình $x+5 y-8 z-13=0$.\\
		Khi đó điểm $O(0;0;0)$ có khoảng cách đến $(P)$ lớn nhất.
		} 
\end{ex} 
\Closesolutionfile{ans}
\indapan{6}{ans/C5B3CD7_37-40}
