\section{Phương trình mặt cầu}
\chude{Xác định các yếu tố cơ bản mặt cầu, lập phương trình mặt cầu}
	
\begin{dang}{XÁC ĐỊNH CÁC YẾU TỐ CƠ BẢN MẶT CẦU}
	\begin{itemize}
		\item Phương trình mặt cầu $(S)$ có dạng $(x-a)^2+(y-b)^2+(z-c)^2=R^2$ thì mặt cầu có tâm $I(a;b;c)$ và có bán kính $R$.
		\item Phương trình mặt cầu $(S)$ có dạng $x^2+y^2+z^2-2ax-2by-2cz+d=0$ với $a^2+b^2+c^2-d>0$ thì để xác định tọa độ tâm $I(a;b;c)$ và bán kính $R$ ta thức hiện như sau:
		\begin{itemize}
			\item Xác định tọa độ tâm $I$: $\heva{& -2a=\ldots \\ & -2b=\ldots \\ & -2c=\ldots}$
			\item Xác định bán kính: $R=\sqrt{a^2+b^2+c^2-d}$.
		\end{itemize}
	\end{itemize}
	\begin{note} \textbf{Chú ý:}
		\begin{itemize}
			\item Có thể xác định tọa độ tâm $I(a;b;c)$ và bán kính $R$ của phương trình mặt cầu $(S)$ có dạng $x^2+y^2+z^2-2ax-2by-2cz+d=0$ bằng cách nhóm nhân tử để đưa về dạng $(x-a)^2+(y-b)^2+(z-c)^2=R^2$.
			\item Để một phương trình là một phương trình mặt cầu, cần thỏa mãn hai điều kiện: Hệ số trước $x^2$, $y^2$, $z^2$ phải bằng $1$ và $a^2+b^2+c^2-d>0$.
			\item Nếu $IM=R$ thì $M$ nằm trên mặt cầu.
			\item Nếu $IM<R$ thì $M$ nằm trong mặt cầu.
			\item Nếu $IM>R$ thì $M$ nằm ngoài mặt cầu.
		\end{itemize}
	\end{note}
\end{dang}
\Opensolutionfile{ans}[ans/ans-C5B3CD1_2-11]
\TN

\begin{ex}%[2H5N3-1]
	Cho điểm $M$ nằm ngoài mặt cầu $S(O;R)$. Khẳng định nào dưới đây đúng?
	\choice
	{$OM<R$}
	{$OM=R$}
	{\True $OM>R$}
	{$OM\le R$}
	\loigiai{$M$ nằm ngoài mặt cầu $S(O;R)\Rightarrow OM>R$.}
\end{ex}

\begin{ex}%[2H5N3-2]
	Trong KG $Oxyz$, cho mặt cầu $(S)\colon x^2+(y-2)^2+(z+1)^2=6$. Đường kính của $(S)$ bằng
	\choice
	{$\sqrt{6}$}
	{$12$}
	{\True $2\sqrt{6}$}
	{$3$}
	\loigiai{
		Ta có bán kính của $(S)$ là $\sqrt{6}$ nên đường kính của $(S)$ bằng $2\sqrt{6}$.}
\end{ex}

\begin{ex}%[2H5H3-2]
	Mặt cầu $(S)\colon3x^2+3y^2+3z^2-6x+12y+2=0$ có bán kính bằng
	\choice
	{$\dfrac{\sqrt{7}}{3}$}
	{$\dfrac{2\sqrt{7}}{3}$}
	{$\dfrac{\sqrt{21}}{3}$}
	{\True $\sqrt{\dfrac{13}{3}}$}
	\loigiai{
		Ta có $3x^2+3y^2+3z^2-6x+12y+2=0\Leftrightarrow  x^2+y^2+z^2-2x+4y+\dfrac{2}{3}=0$.\\
		Do đó mặt cầu $(S)$ có tâm $I\left(1;-2;0\right)$, bán kính $R=\sqrt{\dfrac{13}{3}}$.}
\end{ex}

\begin{ex}%[2H5N3-2]
	Trong KG $Oxyz$, cho mặt cầu $(S)\colon(x-2)^2+(y+1)^2+(z-3)^2=4$. Tâm của $(S)$ có tọa độ là
	\choice
	{$(-2;1;-3)$}
	{$(-4;2;-6)$}
	{$(4;-2;6)$}
	{\True $(2;-1;3)$}
	\loigiai{
		Mặt cầu $(S)\colon (x-2)^2+(y+1)^2+(z-3)^2=4$ có tâm $I(2;-1;3)$.}
\end{ex}

\begin{ex}%[2H5N3-2]
	Trong KG $Oxyz$, mặt cầu $(S)\colon (x+1)^2+(y-2)^2+z^2=9$ có bán kính bằng
	\choice
	{\True $3$}
	{$81$}
	{$9$}
	{$6$}
	\loigiai{Mặt cầu $(S)$ có bán kính $R=3$.}
\end{ex}
	
\begin{ex}%[2H5N3-3]
	Trong KG $Oxyz$, cho mặt cầu $(S)$ có tâm $I(1;-4;0)$ và bán kính bằng $3$. Phương trình của $(S)$ là
	\choice
	{$(x+1)^2+(y-4)^2+z^2=9$}
	{$(x-1)^2+(y+4)^2+z^2=9$}
	{\True $(x-1)^2+(y+4)^2+z^2=9$}
	{$(x+1)^2+(y-4)^2+z^2=3$}
	\loigiai{
		Mặt cầu $(S)$ có tâm $I(1;-4;0)$ có bán kính $3$ có phương trình là $(x-1)^2+(y+4)^2+z^2=9$.}
\end{ex}	
	
\begin{ex}%[2H5N3-2]
	Trong KG $Oxyz$, cho mặt cầu $(S)\colon x^2+y^2+(z-1)^2=16$. Bán kính của $(S)$ là
	\choice
	{$32$}
	{$8$}
	{$4$}
	{$16$}
	\loigiai{
		Từ phương trình mặt cầu $(S)\colon x^2+y^2+(z-1)^2=16\Rightarrow$ bán kính $R=\sqrt{16}=4$.}
\end{ex}	
	
\begin{ex}%[2H5N3-2]
	Trong KG $Oxyz$, cho mặt cầu $(S)\colon (x+1)^2+(y+2)^2+(z-3)^2=9$. Tâm của $(S)$ có tọa độ là
	\choice
	{$(-2;-4;6)$}
	{$(2;4;-6)$}
	{\True $(-1;-2;3)$}
	{$(1;2;-3)$}
	\loigiai{
		Tâm của $(S)$ có tọa độ là $(-1;-2;3)$.}
\end{ex}
	
\begin{ex}%[2H5H3-2]
	Trong KG $Oxyz$, cho mặt cầu $(S)\colon x^2+y^2+z^2-8x+10y-6z+49=0$. Tính bán kính $R$ của mặt cầu $(S)$.
	\choice
	{\True $R=1$}
	{$R=7$}
	{$R=\sqrt{151}$}
	{$R=\sqrt{99}$}
	\loigiai{
		Ta có $a=4$, $b=-5$, $c=3$, $d=49$. Do đó $R=\sqrt{a^2+b^2+c^2-d}=1$.}
\end{ex}
		
\begin{ex}%[2H5N3-2]
	Trong KG $Oxyz$, cho mặt cầu có phương trình $(x-1)^2+(y+2)^2+(z-3)^2=4$. Tìm tọa độ tâm $I$ và bán kính $R$ của mặt cầu đó.
	\choice
	{$I(-1;2;-3)$; $R=2$}
	{$I(-1;2;-3)$; $R=4$}
	{\True $I(1;-2;3)$; $R=2$}
	{$I(1;-2;3)$; $R=4$}
	\loigiai{
		Mặt cầu đã cho có tâm $I\left(1;-2;3\right)$ và bán kính $R=2$.}
\end{ex}		
		
\begin{ex}%[2H5H3-2]
	Trong KG $Oxyz$, trong các mặt cầu dưới đây, mặt cầu nào có bán kính $R=2$?
	\choice
	{$(S)\colon x^2+y^2+z^2-4x+2y+2z-3=0$}
	{$(S)\colon x^2+y^2+z^2-4x+2y+2z-10=0$}
	{\True $(S)\colon x^2+y^2+z^2-4x+2y+2z+2=0$}
	{$(S)\colon x^2+y^2+z^2-4x+2y+2z+5=0$}
	\loigiai{
		Ta có mặt cầu $(S)\colon x^2+y^2+z^2-2ax-2by-2cz+d=0$ có bán kính là $R=\sqrt{a^2+b^2+c^2-d}$.\\
		Với $(S)\colon x^2+y^2+z^2-4x+2y+2z+2=0$, ta có $\heva{
			& a=2\\ 
			& b=-1\\ 
			& c=-1\\ 
			& d=2.}$\\
	Suy ra $R=\sqrt{a^2+b^2+c^2-d}=\sqrt{4}=2$.}
\end{ex}
		
\begin{ex}%[2H5H3-3]
	Cho các phương trình sau
	\begin{listEX}[2]
		\item $\left(x-1\right)^2+y^2+z^2=1$;
		\item $x^2+\left(2y-1\right)^2+z^2=4$;
		\item $x^2+y^2+z^2+1=0$;
		\item $\left(2x+1\right)^2+\left(2y-1\right)^2+4z^2=16$.
	\end{listEX}
	Số phương trình là phương trình mặt cầu là
	\choice
	{$4$}
	{$3$}
	{\True $2$}
	{$1$}
	\loigiai{
		Ta có $\left(2x+1\right)^2+\left(2y-1\right)^2+4z^2=16\Leftrightarrow{\left(x+\dfrac{1}{2}\right)^2}+\left(y-\dfrac{1}{2}\right)^2+z^2=4$ là phương trình mặt cầu.\\
		$\left(x-1\right)^2+y^2+z^2=1$ là phương trình của một mặt cầu.}
\end{ex}
		
\begin{ex}%[2H5H3-2]
	Trong không gian với hệ trục tọa độ $Oxyz$, gọi $I$ là tâm mặt cầu $(S)\colon x^2+y^2+(z-2)^2=4$. Độ dài $\left|\overrightarrow{OI}\right|$ bằng
	\choice
	{\True $2$}
	{$4$}
	{$1$}
	{$\sqrt{2}$}
	\loigiai{
		Mặt cầu $(S)$ có tâm $I(0;0;2)\Rightarrow\overrightarrow{OI}=(0;0;2)\Rightarrow\left|\overrightarrow{OI}\right|=2$.}
\end{ex}
			
\begin{ex}%[2H5H3-3]
	Trong KG $Oxyz$ có tất cả bao nhiêu giá trị nguyên $m$ để phương trình $x^2+y^2+z^2+4mx+2my-2mz+9m^2-28=0$ là phương trình mặt cầu?
	\choice
	{\True $7$}
	{$8$}
	{$9$}
	{$6$}
	\loigiai{
		Ta có 
		$$\begin{aligned}
			&x^2+y^2+z^2+4mx+2my-2mz+9m^2-28=0\\
			\Leftrightarrow\ &(x+2m)^2+(y+m)^2+(z-m)^2=28-3m^2.\quad(1)
		\end{aligned}$$
		$(1)$ là phương trình mặt cầu $\Leftrightarrow 28-3m^2>0\Leftrightarrow-\sqrt{\dfrac{28}{3}}<m<\sqrt{\dfrac{28}{3}}$.\\
		Do $m$ nguyên nên $m\in\left\{-3;-2;-1;0;1;2;3\right\}$.\\
		Vậy có $7$ giá trị của $m$ thỏa mãn yêu cầu bài toán.}
\end{ex}			
				
\begin{ex}%[2H5H3-3]
	Trong KG $Oxyz$, có tất cả bao nhiêu giá nguyên của $m$ để
	$x^2+y^2+z^2+2\left(m+2\right)x-2\left(m-1\right)z+3m^2-5=0$ là phương trình một mặt cầu?
	\choice
	{$4$}
	{$6$}
	{$5$}
	{\True $7$}
\loigiai{
	Phương trình đã cho là phương trình mặt cầu khi và chỉ khi
	$$\begin{aligned}
		(m+2)^2+(m-1)^2-3m^2+5>0 
		&\Leftrightarrow m^2-2m-10<0\\ 
		&\Leftrightarrow-1-\sqrt{11}<m<1+\sqrt{11}.
	\end{aligned}$$
	Vì $m\in\mathbb{Z}\Rightarrow m=\{-2;-1;0;1;2;3;4\}\Rightarrow $ có $7$ giá trị của $m$ nguyên thỏa mãn bài toán.}
\end{ex}	

\begin{ex}%[2H5H3-3]
	Cho phương trình $x^2+y^2+z^2-4x+2my+3m^2-2m=0$ với $m$ là tham số. Tính tổng tất cả các giá trị nguyên của $m$ để phương trình đã cho là phương trình mặt cầu.
	\choice
	{$0$}
	{\True $1$}
	{$2$}
	{$3$}
	\loigiai{
		Giả sử $x^2+y^2+z^2-4x+2my+3m^2-2m=0$ là phương trình mặt cầu.\\
		Khi đó tâm mặt cầu là $I(2;-m;0)$, và bán kính $$R=\sqrt{4+m^2-\left(3m^2-2m\right)}=\sqrt{-2m^2+2m+4} \text{ với }-2m^2+2m+4>0\Leftrightarrow m\in(-1;2).$$
		Do $m\in\mathbb{Z}\Rightarrow m\in\{ 0;1\}$.\\
		Vậy tổng tất cả các giá trị nguyên của $m$ bằng $1$.}
\end{ex}

\Closesolutionfile{ans}
\indapan{10}{ans/ans-C5B3CD1_2-11}		
\TNTF
\Opensolutionfile{ans}[ans/ans-C5B3CD1_2-11-DS]

\begin{ex}%[2H5N3-2]
	Trong KG $Oxyz$, cho mặt cầu $(S)\colon x^2+y^2+(z+2)^2=9$ có tâm $I$ và bán kính $R$. Các mệnh đề sau đây đúng hay sai?
	\choiceTF
	{Tọa độ tâm mặt cầu $(S)$ là $I(0;0;2)$}
	{Bán kính mặt cầu $(S)$ là $R=9$}
	{\True Khoảng cách từ tâm mặt cầu đến mặt phẳng $(P)\colon x+y+z=0$ bằng $\dfrac{2\sqrt{3}}{3}$}
	{\True Diện tích mặt cầu $(S)$ bằng $36\pi$}
	\loigiai{
		\begin{itemchoice}
			\itemch \textbf{Sai.} Tọa độ tâm mặt cầu là $I\left(0;0;-2\right)$.
			\itemch \textbf{Sai.} Bán kính của mặt cầu là $R=3$.
			\itemch \textbf{Đúng.} Ta có $\mathrm{d}(I,(P))=\dfrac{|0+0+2|}{\sqrt{1^2+1^2+1^2}}=\dfrac{2\sqrt{3}}{3}$.
			\itemch \textbf{Đúng.} Diện tích mặt cầu $S=4\pi R^2=4\pi\cdot 3^2=36\pi$.
		\end{itemchoice}
		}
\end{ex}

\begin{ex}%[2H5H3-2]
	Trong KG $Oxyz$, cho mặt cầu $(S)\colon(x+3)^2+y^2+(z-2)^2=16$ có tâm $I$ và bán kính $R$. Các mệnh đề sau đây đúng hay sai?
	\choiceTF
	{\True Điểm $M(-1;0;3)$ nằm trong mặt cầu $(S)$}
	{\True Bán kính mặt cầu $(S)$ là $R=4$}
	{\True Tọa độ tâm mặt cầu $(S)$ là $I(-3;0;2)$}
	{Thể tích mặt cầu $(S)$ là $V=\dfrac{16384\pi}{3}$}
	\loigiai{
		\begin{itemchoice}
			\itemch \textbf{Đúng.} Thay tọa độ điểm $M$ vào vế trái của phương trình $(S)$ ta có $$(-1+3)^2+0^2+(3-2)^2=5<16.$$
			Suy ra $M$ nằm trong mặt cầu $(S)$.
			\itemch \textbf{Đúng.} Bán kính mặt cầu là $R=4$.
			\itemch \textbf{Đúng.} Tọa độ tâm mặt cầu là $I(-3;0;2)$.
			\itemch \textbf{Sai.} Thể tích mặt cầu $(S)$ là $V=\dfrac{256\pi}{3}$.
		\end{itemchoice}
		}
\end{ex}

\begin{ex}%[2H5H3-2]
	Trong không gian với hệ toạ độ $Oxyz$, cho điểm $M(2;0;2)$ và mặt cầu $(S)\colon x^2+(y+2)^2+(z-2)^2=8$. Các mệnh đề sau đây đúng hay sai?
	\choiceTF
	{\True Điểm $M\left(2;0;2\right)$ thuộc mặt cầu $(S)$}
	{\True Bán kính mặt cầu $(S)$ là $R=2\sqrt{2}$}
	{\True Tọa độ tâm mặt cầu $(S)$ là $I\left(0;-2;2\right)$}
	{Hình chiếu của tâm mặt cầu lên trục $Ox$ là điểm có tọa độ $(0;0;2)$}
	\loigiai{
		\begin{itemchoice}
			\itemch \textbf{Đúng.} Thay tọa độ điểm $M\left(2;0;2\right)$ vào mặt cầu, ta có $2^2+2^2+(2-2)^2=8\Rightarrow M(2;0;2)\in(S)$.
			\itemch \textbf{Đúng.} Mặt cầu $(S)$ có bán kính $R=2\sqrt{2}$.
			\itemch \textbf{Đúng.} Mặt cầu $(S)$ có tâm $I(0;-2;2)$.
			\itemch \textbf{Sai.} Hình chiếu của tâm mặt cầu lên trục $Ox$ là $O(0;0;0)$.
		\end{itemchoice}
		}
\end{ex}

\begin{ex}%[2H5N3-2]
	Trong KG $Oxyz$, mặt cầu $(S)\colon(x-1)^2+(y+2)^2+(z-4)^2=20$. Các mệnh đề sau đây đúng hay sai?
	\choiceTF
	{Bán kính mặt cầu $(S)$ là $20$}
	{Diện tích mặt cầu $(S)$ là $1600\pi$}
	{Tọa độ tâm mặt cầu $(S)$ là $I\left(-1;2;-4\right)$}
	{\True Điểm đối xứng của tâm mặt cầu $(S)$ qua mặt phẳng $(Oyz)$ là $I\left(-1;-2;4\right)$}
	\loigiai{
		\begin{itemchoice}
			\itemch \textbf{Sai.} Mặt cầu $(x-1)^2+(y+2)^2+(z-4)^2=20$ có bán kính là $R=2\sqrt{5}$.
			\itemch \textbf{Sai.} Diện tích mặt cầu $(S)$ bằng $S=4\pi R^2=4\pi \left(2\sqrt{5}\right)^2=80\pi$.
			\itemch \textbf{Sai.} Mặt cầu $(x-1)^2+(y+2)^2+(z-4)^2=20$ có tâm là $I(1;-2;4)$.
			\itemch \textbf{Đúng.} Điểm đối xứng của tâm $I(1;-2;4)$ của mặt cầu $(S)$ qua mặt phẳng $(Oyz)$ là $I\left(-1;-2;4\right)$.
		\end{itemchoice}
		}
\end{ex}

\begin{ex}%[2H5H3-3]
	Trong KG $Oxyz$, cho các phương trình sau
	\begin{listEX}
		\item $(S_1)\colon x^2+y^2+z^2+x-2y+4z-3=0$, 
		\item $(S_2)\colon 2x^2+2y^2+2z^2-x-y-z=0$,
		\item $(S_3)\colon 2x^2+2y^2+2z^2+4x+8y+6z+3=0$, 
		\item $(S_4)\colon x^2+y^2+z^2-2x+4y-4z+10=0$.
	\end{listEX}
	Các mệnh đề sau đây đúng hay sai?
	\choiceTF
	{\True $(S_1)$ là phương trình của một mặt cầu}
	{\True $(S_2)$ là phương trình của một mặt cầu}
	{$(S_3)$ không phải là phương trình của một mặt cầu}
	{\True $(S_4)$ không phải là phương trình của một mặt cầu}
	\loigiai{
		 Phương trình $x^2+y^2+z^2-2ax-2by-2cz+d=0$ là phương trình của một mặt cầu nếu $a^2+b^2+c^2-d>0$.
		\begin{itemchoice}
			\itemch \textbf{Đúng.}
			$(S_1)\colon x^2+y^2+z^2+x-2y+4z-3=0\Rightarrow \heva{
				& a=-\dfrac{1}{2}\\ 
				& b=1\\ 
				& c=-2\\ 
				& d=-1}\Rightarrow a^2+b^2+c^2-d>0$.
			\itemch \textbf{Đúng.} $(S_2)\colon 2x^2+2y^2+2z^2-x-y-z=0\Leftrightarrow x^2+y^2+z^2-\dfrac{1}{2}x-\dfrac{1}{2}y-\dfrac{1}{2}z=0$.\\
			Suy ra $\heva{
				& a=\dfrac{1}{4}\\ 
				& b=\dfrac{1}{4}\\ 
				& c=\dfrac{1}{4}\\ 
				& d=0}\Rightarrow a^2+b^2+c^2-d>0$.
			\itemch \textbf{Sai.} $(S_3)\colon 2x^2+2y^2+2z^2+4x+8y+6z+3=0\Leftrightarrow x^2+y^2+z^2+2x+4y+3z+\dfrac{3}{2}=0$.\\
			Suy ra $\heva{
				& a=-1\\ 
				& b=-2\\ 
				& c=-\dfrac{3}{2}\\ 
				& d=\dfrac{3}{2}}\Rightarrow a^2+b^2+c^2-d>0$.
			\itemch \textbf{Đúng.} $(S_4)\colon x^2+y^2+z^2-2x+4y-4z+10=0 \Rightarrow\heva{
				& a=1\\ 
				& b=-2\\ 
				& c=2\\ 
				& d=10}\Rightarrow a^2+b^2+c^2-d<0$.
		\end{itemchoice}
		}
\end{ex}

\begin{ex}%[2H5H3-3]
	Trong KG $Oxyz$, cho các phương trình sau
	\begin{listEX}[2]
		\item $(S_1)\colon x^2+y^2+z^2-2x=0$, 
		\item $(S_2)\colon x^2+y^2-z^2+2x-y+1=0$,
		\item $(S_3)\colon 2x^2+2y^2=(x+y)^2-z^2+2x-1$,
		\item $(S_4)\colon (x+y)^2=2xy-z^2-1$.
	\end{listEX}
	Các mệnh đề sau đây đúng hay sai?
	\choiceTF
	{\True $(S_1)$ là phương trình của một mặt cầu}
	{$(S_2)$ là phương trình của một mặt cầu}
	{$(S_3)$ là phương trình của một mặt cầu}
	{\True $(S_4)$ không phải là phương trình của một mặt cầu}
	\loigiai{
		Phương trình mặt cầu $(S)$ có hai dạng là\\
		(1) $(x-a)^2+(y-b)^2+(z-c)^2=R^2$;\\
		(2) $x^2+y^2+z^2-2ax-2by-2cz+d=0$ với $a^2+b^2+c^2-d>0$.
		\begin{itemchoice}
			\itemch \textbf{Đúng.} Ta có $(S_1)\colon x^2+y^2+z^2-2x=0\Leftrightarrow (x-1)^2+y^2+z^2=1$.\\
			Do đó $(S_1)$ là phương trình mặt cầu.
			\itemch \textbf{Sai.} Ta có $(S_2)$ không là phương trình mặt cầu vì các hệ số của $x^2$, $y^2$, $z^2$ không bằng nhau.
			\itemch \textbf{Sai.} Ta có $(S_3)\colon 2x^2+2y^2=(x+y)^2-z^2+2x-1\Leftrightarrow x^2+y^2+z^2-2xy-2x+1=0$.\\
			Vì phương trình có chứa $xy$ nên $(S_3)$ không phải phương trình mặt cầu.
			\itemch \textbf{Đúng.} Ta có $(S_4)\colon (x+y)^2=2xy-z^2-1\Leftrightarrow x^2+y^2+z^2=1$.\\
			Do đó $(S_4)$ là phương trình mặt cầu.
		\end{itemchoice}
		}
\end{ex}

\begin{ex}%[2H5H3-3]
	Trong KG $Oxyz$, cho các phương trình sau
	\begin{listEX}[2]
		\item $(S_1)\colon x^2+y^2+z^2-2x=0$,
		\item $(S_2)\colon 2x^2+2y^2=(x+y)^2-z^2+2x-1$,
		\item $(S_3)\colon x^2+y^2+z^2+2x-2y+1=0$,
		\item $(S_4)\colon (x+y)^2=2xy-z^2+1-4x$.
	\end{listEX}
	Các mệnh đề sau đây đúng hay sai?
	\choiceTF
	{\True $(S_1)$ là phương trình của một mặt cầu}
	{$(S_2)$ là phương trình của một mặt cầu}
	{\True $(S_3)$ là phương trình của một mặt cầu}
	{\True $(S_4)$ là phương trình của một mặt cầu}
	\loigiai{
		Phương trình mặt cầu $(S)$ có hai dạng là\\
		(1) $(x-a)^2+(y-b)^2+(z-c)^2=R^2$;\\
		(2) $x^2+y^2+z^2-2ax-2by-2cz+d=0$ với $a^2+b^2+c^2-d>0$.
		\begin{itemchoice}
			\itemch \textbf{Đúng.} Ta có $(S_1)\colon x^2+y^2+z^2-2x=0\Leftrightarrow (x-1)^2+y^2+z^2=1$.\\
			Do đó $(S_1)$ là phương trình mặt cầu.
			\itemch \textbf{Sai.} Ta có $(S_2)\colon 2x^2+2y^2=(x+y)^2-z^2+2x-1\Leftrightarrow x^2+y^2+z^2-2xy-2x+1=0$.\\
			Vì phương trình có chứa $xy$ nên $(S_2)$ không phải phương trình mặt cầu.
			\itemch \textbf{Đúng.} Ta có $(S_3)\colon x^2+y^2+z^2+2x-2y+1=0\Leftrightarrow (x+1)^2+(y-1)^2+z^2=1$.\\
			Do đó $(S_3)$ là phương trình mặt cầu.
			\itemch \textbf{Đúng.} Ta có $(S_4)\colon (x+y)^2=2xy-z^2+1-4x\Leftrightarrow (x+2)^2+y^2+z^2=5$.\\
			Do đó $(S_4)$ là phương trình mặt cầu.
		\end{itemchoice}
		}
\end{ex}

\begin{ex}%[2H5H3-3]
	Trong KG $Oxyz$, cho các phương trình sau
	\begin{listEX}[2]
		\item $(S_1)\colon (x-1)^2+(2y-1)^2+(z-1)^2=6$, 
		\item $(S_2)\colon (x-1)^2+(y-1)^2+(z-1)^2=6$,
		\item $(S_3)\colon (2x-1)^2+(2y-1)^2+(2z+1)^2=6$,
		\item $(S_4)\colon (x+y)^2=2xy-z^2+3-6x$.
	\end{listEX}
	Các mệnh đề sau đây đúng hay sai?
	\choiceTF
	{\True $(S_1)$ không phải là phương trình của một mặt cầu}
	{$(S_2)$ không phải là phương trình của một mặt cầu}
	{$(S_3)$ không phải là phương trình của một mặt cầu}
	{$(S_4)$ không phải là phương trình của một mặt cầu}
	\loigiai{
		Phương trình mặt cầu $(S)$ có hai dạng là\\
		(1) $(x-a)^2+(y-b)^2+(z-c)^2=R^2$;\\
		(2) $x^2+y^2+z^2-2ax-2by-2cz+d=0$ với $a^2+b^2+c^2-d>0$.
		\begin{itemchoice}
			\itemch \textbf{Đúng.} Ta có hệ số của $x$, $y$, $z$ trong phương trình không bằng nhau nên $(S_1)$ không phải là phương trình của một mặt cầu.
			\itemch \textbf{Sai.} Vì $(S_2)\colon (x-1)^2+(y-1)^2+(z-1)^2=6$ là phương trình mặt cầu.
			\itemch \textbf{Sai.} Vì $$(S_3)\colon (2x-1)^2+(2y-1)^2+(2z+1)^2=6\Leftrightarrow \left(x-\dfrac{1}{2}\right)^2+\left(y-\dfrac{1}{2}\right)^2+\left(z+\dfrac{1}{2}\right)^2=\dfrac{3}{2}.$$
			Do đó $(S_3)$ là phương trình mặt cầu.
			\itemch \textbf{Sai.} $(S_4)\colon (x+y)^2=2xy-z^2+3-6x\Leftrightarrow (x+3)^2+y^2+z^2=12.$\\
			Do đó $(S_4)$ là phương trình mặt cầu.
		\end{itemchoice}
		}
\end{ex}

\begin{ex}%[2H5H3-3]
	Trong không gian với hệ trục $Oxyz$, cho phương trình $(S)\colon x^2+y^2+z^2-2(m+2)x+4my-2mz+5m^2+9=0$. Các mệnh đề sau đây đúng hay sai?
	\choiceTF
	{Với $m=0$ thì $(S)$ là phương trình của một mặt cầu}
	{Với $m=1$ thì $(S)$ là phương trình của một mặt cầu có tâm $I(3;-2;1)$}
	{\True Với $m=3$ thì $(S)$ là phương trình của một mặt cầu có bán kính là $R=4$}
	{\True Với $m<-5$ hoặc $m>1$ thì $(S)$ là phương trình của một mặt cầu}
\loigiai{
	
	\begin{itemchoice}
		\itemch \textbf{Sai.} Với $m=0$, ta có $\heva{&a=2\\&b=0\\&c=0\\&d=9}$. Khi đó $a^2+b^2+c^2-d=-5<0$. \\
		Suy ra $(S)$ không là phương trình mặt cầu.
		\itemch \textbf{Sai.} Với $m=1$, ta có $\heva{&a=3\\&b=-2\\&c=1\\&d=14}$. Khi đó $a^2+b^2+c^2-d=0$. \\
		Suy ra $(S)$ không là phương trình mặt cầu.
		\itemch \textbf{Đúng.} Với $m=3$, ta có $\heva{&a=5\\&b=-6\\&c=3\\&d=54}$. Khi đó $a^2+b^2+c^2-d=16>0$. \\
		Suy ra $(S)$ là phương trình mặt cầu có bán kính $R=\sqrt{a^2+b^2+c^2-d}=4$.
		\itemch \textbf{Đúng.} Ta có điều kiện xác định mặt cầu là $a^2+b^2+c^2-d>0$, khi đó
		$$\left(m+2\right)^2+4m^2+m^2-5m^2-9>0\Leftrightarrow m^2+4m-5>0\Leftrightarrow\hoac{
			& m<-5\\ 
			& m>1.}$$
	\end{itemchoice}
	}
\end{ex}

\Closesolutionfile{ans}
\indapan{3}{ans/ans-C5B3CD1_2-11-DS}

\Opensolutionfile{ans}[ans/ans-C5B3CD1_2-11-KQ]
\TNSA
\begin{ex}%[2H5N3-2]
	Trong KG $Oxyz$, cho mặt cầu $(S)\colon (x-1)^2+(y+2)^2+(z-3)^2=16$. Tọa độ tâm của mặt cầu $(S)$ là $(a;b;c)$. Khi đó $a+b+c$ bằng bao nhiêu?
	\shortans{$2$}
	\loigiai{
	Mặt cầu $(S)\colon (x-1)^2+(y+2)^2+(z-3)^2=16$ có tâm là $I(1;-2;3)$.\\
	Suy ra $a=1$, $b=-2$, $c=3$. Khi đó $a+b+c=2$.
	}
\end{ex}
