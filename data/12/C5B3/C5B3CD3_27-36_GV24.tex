\begin{dang}
	{LẬP PHƯƠNG TRÌNH MẶT PHẲNG LIÊN QUAN ĐẾN MẶT PHẲNG, MẶT CẦU}
\end{dang}
\Opensolutionfile{ans}[ans/ans-2-C5B3CD3-LC]
\TN
\begin{ex}%[2H5H1-3]
	Trong không gian tọa độ $Oxyz$, cho mặt cầu $(S)$ có đường kính $AB$ với $A(6;2;-5)$, $B(-4;0;7)$. Viết phương trình mặt phẳng $(P)$ tiếp xúc với mặt cầu $(S)$ tại $A$.
	\choice
	{$(P)\colon 5x+y-6z+62=0$}
	{\True $(P)\colon 5x+y-6z-62=0$}
	{$(P)\colon 5x-y-6z-62=0$}
	{$(P)\colon 5x+y+6z+62=0$}
	\loigiai{
		Gọi $I$ là trung điểm của $AB$ $\Rightarrow I(1;1;1)$.\\
		Mặt cầu $(S)$ có đường kính $AB$ nên có tâm là điểm $I$.\\
		Mặt phẳng $(P)$ tiếp xúc với mặt cầu $(S)$ tại $A$ nên mặt phẳng $(P)$ đi qua $A$ và nhận $\overrightarrow{IA}=(5;1;-6)$ là véc-tơ pháp tuyến.\\
		Phương trình mặt phẳng $(P)\colon 5(x-6)+1(y-2)-6(z+5)=0\Leftrightarrow 5x+y-6z-62=0$.
		}
\end{ex}
\begin{ex}%[2H5H1-3]
	Trong KG $Oxyz$, cho mặt phẳng $(Q)$ song song với mặt phẳng $(P)\colon 2x-2y+z+7=0$. Biết mp$(Q)$ cắt mặt cầu $(S)\colon x^2+(y+2)^2+(z-1)^2=25$ theo một đường tròn có bán kính $r=3$. Khi đó mặt phẳng $(Q)$ có phương trình là
	\choice
	{$x-y+2z-7=0$}
	{$2x-2y+z+17=0$}
	{$2x-2y+z+7=0$}
	{\True $2x-2y+z-17=0$}
	\loigiai{
		$(S)$ có tâm $I(0;-2;1)$ và bán kính $R=5$.\\
		Gọi $M$ là hình chiếu vuông góc của $I$ lên $(Q)$.\\
		$(Q)$ cắt mặt cầu $(S)$ theo một đường tròn có bán kính $r=3$.\\
		$\Rightarrow IM=\sqrt{R^2-r^2}=\sqrt{5^2-3^2}=4$.\\
		$(Q)\parallel (P)\colon 2x-2y+z+7=0 \Rightarrow (Q)\colon 2x-2y+z+m=0 \quad (m\ne 7)$.
		\begin{eqnarray*}
			\mathrm{d}(I,(Q))&=&\dfrac{|2\cdot 0-2\cdot (-2)+1\cdot 1 +m|}{\sqrt{2^2+(-2)^2+1^2}}=IM=4.\\
			&\Leftrightarrow& |m+5|=12\Leftrightarrow \hoac{&m=7\\ &m=-17.}
		\end{eqnarray*}
		Vậy phương trình của $(Q)$ là $2x-2y+z-17=0$.
		}
\end{ex}
\begin{ex}%[2H5H1-3]
	Trong KG $Oxyz$ cho mặt cầu $(S)\colon x^2+y^2+z^2+2x-4y-6z+5=0$. Mặt phẳng tiếp xúc với $(S)$ và song song với mặt phẳng $(P)\colon 2x-y+2z-11=0$ có phương trình là
	\choice
	{$2x-y+2z-7=0$}
	{$2x-y+2z+9=0$}
	{\True $2x-y+2z+7=0$}
	{$2x-y+2z-9=0$}
	\loigiai{
		Ta gọi phương trình mặt phẳng song song với mặt phẳng $(P)\colon 2x-y+2z-11=0$ có dạng là $(Q)\colon 2x-y+2z+D=0 \quad (D\ne -11)$.\\
		Mặt cầu $(S)$ có tâm $I(-1;2;3)$, bán kính $R=\sqrt{(-1)^2+2^2+3^2-5}=3$.\\
		Vì mặt phẳng tiếp xúc với $(S)$ nên ta có:
		\begin{eqnarray*}
			\mathrm{d}(I,(Q))=R&\Leftrightarrow& \dfrac{|2\cdot(-1)-2+2\cdot 3+D|}{\sqrt{2^2+(-1)^2+2^2}}=3\\
			&\Leftrightarrow& \dfrac{|2+D|}{3}=3 \Leftrightarrow \hoac{&2+D=9\\ &2+D=-9} \Leftrightarrow \hoac{&D=7\\ &D=-11.}
		\end{eqnarray*}
		Do $D\ne -11\Rightarrow D=7$.\\
		Vậy mặt phẳng cần tìm là $2x-y+2z+7=0$.
		}
\end{ex}
\begin{ex}%[2H5H1-3]
	Trong KG $Oxyz$, mặt phẳng $(P)$ chứa trục $Ox$ và cắt mặt cầu $(S)\colon x^2+y^2+z^2-2x+4y+2z-3=0$ theo giao tuyến là đường tròn có bán kính bằng $3$ có phương trình là
	\choice
	{\True $y-2z=0$}
	{$y+2z=0$}
	{$y+3z=0$}
	{$y-3z=0$}
	\loigiai{
		$(S)\colon x^2+y^2+z^2-2x+4y+2z-3=0$ có tâm $I(1;-2;-1)$ và bán kính $R=3$.\\
		$(P)$ cắt mặt cầu $(S)$ theo giao tuyến là đường tròn có bán kính $r=3=R$.\\
		$\Rightarrow I\in (P)$.\\
		Chọn điểm $M(1;0;0)\in Ox\Rightarrow \overrightarrow{IM}=(0;2;1)$.\\
		$\overrightarrow{n}=\left[\overrightarrow{i},\overrightarrow{IM}\right]=(0;-1;2)$.\\
		$(P)$ qua $O(0;0;0)$ và có VTPT $\overrightarrow{n}=(0;-1;2)\Rightarrow (P)\colon y-2z=0$.
		}
\end{ex}
\begin{ex}%[2H5V1-3]
	Trong KG $Oxyz$, cho mặt cầu $(S)\colon x^2+y^2+z^2+2z-2=0$ và điểm $K(2;2;0)$. Viết phương trình mặt phẳng chứa tất cả các tiếp điểm của các tiếp tuyến vẽ từ $K$ đến mặt cầu $(S)$.
	\choice
	{$2x+2y+z-4=0$}
	{$6x+6y+3z-8=0$}
	{\True $2x+2y+z+2=0$}
	{$6x+6y+3z-3=0$}
	\loigiai{
		$(S)\colon x^2+y^2+(z+1)^2=3 \Rightarrow $ mặt cầu tâm $I(0;0;-1)$, $R=\sqrt{3}$.\\
		Do $\overrightarrow{IK}=(2;2;1)$, $IK=3>R\Rightarrow K$ nằm ngoài mặt cầu. Suy ra từ $K$ vẽ được vô số tiếp tuyến đến mặt cầu và khoảng cách từ $K$ đến các tiếp điểm bằng nhau.\\
		Gọi $E$ là $1$ tiếp điểm $\Rightarrow IE\perp EK\Rightarrow \triangle IKE$ vuông tại $E\Rightarrow KE=\sqrt{IK^2-IE^2}=\sqrt{6} \Rightarrow E$ thuộc mặt cầu tâm $K$ bán kính $R'=\sqrt{6}$.\\
		Tọa độ điểm $E$ thỏa mãn hệ\\ $\heva{&x^2+y^2+z^2+2z-2=0\\ &(x-2)^2+(y-2)^2+z^2=6} \Rightarrow x^2+y^2+z^2+2z-2=(x-2)^2+(y-2)^2+z^2=6$.\\
		$\Leftrightarrow 4x+4y+2z+4=0\Leftrightarrow 2x+2y+z+2=0$.
		}
\end{ex}
\begin{ex}%[2H5V1-3]
	Trong KG $Oxyz$, cho mặt cầu $(S)\colon x^2+y^2+z^2-2x+6y-4z-2=0$ và mặt phẳng $(\alpha)\colon x+4y+z-11=0$. Viết phương trình mặt phẳng $(P)$, biết $(P)$ song song với giá của véc-tơ $\overrightarrow{v}=(1;6;2)$, vuông góc với $(\alpha)$ và tiếp xúc với $(S)$.
	\choice
	{$\hoac{&x-2y+z+3=0\\ &x-2y+z-21=0}$}
	{$\hoac{&3x+y+4z+1=0\\ &3x+y+4z-2=0}$}
	{$\hoac{&4x-3y-z+5=0\\ &4x-3y-z-27=0}$}
	{\True $\hoac{&2x-y+2z+3=0\\ &2x-y+2z-21=0}$}
	\loigiai{
		Mặt cầu $(S)$ có tâm $I(1;-3;2)$ và bán kính $R=4$.\\
		Vì mặt phẳng $(P)$ song song với giá của véc-tơ $\overrightarrow{v}=(1;6;2)$, vuông góc với $(\alpha)$ nên có véc-tơ pháp tuyến $\overrightarrow{n}=\left[\overrightarrow{n}_{(\alpha)},\overrightarrow{v}\right]=(2;-1;2)$.\\
		Mặt phẳng $(P)\colon 2x-y+2z+D=0$.\\
		Vì $(P)$ tiếp xúc với mặt cầu $(S)$ nên ta có:
		\begin{eqnarray*}
			\mathrm{d}(I,(P))=R&\Leftrightarrow& \dfrac{|2\cdot 1 +3+ 2\cdot 2+D|}{\sqrt{2^2+(-1)^2+2^2}}=4\\
			&\Leftrightarrow& |D+9|=12\Leftrightarrow \hoac{&D=-21\\ &D=3.}
		\end{eqnarray*}
		Vậy phương trình mặt phẳng $(P)$ là $\hoac{&2x-y+2z+3=0\\ &2x-y+2z-21=0.}$
		}
\end{ex}
\begin{ex}%[2H5V1-3]
	Trong không gian với hệ trục tọa độ $Oxyz$, cho mặt phẳng $(P)$ có phương trình $x-2y-2z-5=0$ và mặt cầu $(S)$ có phương trình $(x-1)^2+(y+2)^2+(z+3)^2=4$. Tìm phương trình mặt phẳng song song với mặt phẳng $(P)$ đồng thời tiếp xúc với mặt cầu $(S)$.
	\choice
	{$x-2y-2z+1=0$}
	{$-x+2y+2z+5=0$}
	{$x-2y-2z-23=0$}
	{\True $-x+2y+2z+17=0$}
	\loigiai{
		Mặt cầu $(S)$ có tâm $I(1;-2;-3)$ và bán kính $R=2$.\\
		Gọi $(Q)$ là mặt phẳng song song với mặt phẳng $(P)$ đồng thời tiếp xúc với mặt cầu $(S)$.\\
		Phương trình $(Q)$ có dạng: $x-2y-2z+D=0$ \quad $(D\ne -5)$.\\
		$(Q)$ tiếp xúc với $(S)$ khi và chỉ khi
		\begin{eqnarray*}
			\mathrm{d}(I,(Q))=R &\Leftrightarrow& \dfrac{|1-2\cdot (-2)-2\cdot (-3)+D|}{\sqrt{1^2+2^2+2^2}}=2\\
			&\Leftrightarrow& |D+11|=6 \Leftrightarrow \hoac{&D+11=6\\ &D+11=-6}\\
			&\Leftrightarrow& \hoac{&D=-5\\ &D=-17.}
		\end{eqnarray*}
		Đối chiếu điều kiện suy ra $D=-17$.\\
		Vậy phương trình của $(Q)$ là $x-2y-2z-17=0 \Leftrightarrow -x+2y+2z+17=0$.
		}
\end{ex}
\begin{ex}%[2H5V1-3]
	Trong không gian với hệ trục tọa độ $Oxyz$ cho mặt cầu $(S)\colon x^2+y^2+z^2-2x+6y-4z-2=0$, mặt phẳng $(\alpha)\colon x+4y+z-11=0$. Gọi $(P)$ là mặt phẳng vuông góc với $(\alpha)$, $(P)$ song song với giá của véc-tơ $\overrightarrow{v}=(1;6;2)$ và $(P)$ tiếp xúc với $(S)$. Lập phương trình mặt phẳng $(P)$.
	\choice
	{$2x-y+2z-2=0$ và $x-2y+z-21=0$}
	{$x-2y+2z+3=0$ và $x-2y+z-21=0$}
	{\True $2x-y+2z+3=0$ và $2x-y+2z-21=0$}
	{$2x-y+2z+5=0$ và $2x-y+2z-2=0$}
	\loigiai{
		$(S)$ có tâm $I(1;-3;2)$ và bán kính $R=4$. Véc-tơ pháp tuyến của $(\alpha)$ là $\overrightarrow{n}_{\alpha}=(1;4;1)$.\\
		Suy ra VTPT của $(P)$ là $\overrightarrow{n}_P=\left[\overrightarrow{n}_{\alpha},\overrightarrow{v}\right]=(2;-1;2)$.\\
		Do đó $(P)$ có dạng: $2x-y+2z+d=0$.\\
		Mặt khác $(P)$ tiếp xúc với $(S)$ nên $\mathrm{d}(I,(P))=4$.\\
		Hay $\dfrac{|2+3+4+d|}{\sqrt{2^2+(-1)^2+2^2}}=4 \Rightarrow \hoac{&d=-21\\ &d=3.}$
		}
\end{ex}
\begin{ex}%[2H5V1-3]
	Trong KG $Oxyz$, viết phương trình mặt phẳng tiếp xúc với mặt cầu $(x-1)^2+y^2+(z+2)^2=6$ đồng thời song song với hai đường thẳng $d_1\colon \dfrac{x-2}{3}=\dfrac{y-1}{-1}=\dfrac{z}{-1}$, $d_2\colon \dfrac{x}{1}=\dfrac{y+2}{1}=\dfrac{z-2}{-1}$.
	\choice
	{$\hoac{&x-y+2z-3=0\\ &x-y+2z+9=0}$}
	{\True $\hoac{&x+y+2z-3=0\\ &x+y+2z+9=0}$}
	{$x+y+2z+9=0$}
	{$x-y+2z+9=0$}
	\loigiai{
		Đường thẳng $d_1$ có vtcp $\overrightarrow{u}_1(3;-1;-1)$, đường thẳng $d_2$ có vtcp $\overrightarrow{u}_2(1;1;-1)$. Gọi $\overrightarrow{n}$ là vtpt của mặt phẳng $(\alpha)$ cần tìm. Do $(\alpha)$ song song với hai đường thẳng $d_1$, $d_2$ nên $\overrightarrow{n}\perp \overrightarrow{u}_1$ và $\overrightarrow{n}\perp \overrightarrow{u}_2$, từ đó ta chọn $\overrightarrow{n}=\left[\overrightarrow{u}_1,\overrightarrow{u}_2\right]=(2;2;4)$. Suy ra $(\alpha)\colon x+y+2z+c=0$.\\
		Mặt cầu $(S)$ có tâm $I(1;0;-2)$, bán kính $R=\sqrt{6}$.\\
		$(\alpha)$ tiếp xúc với $(S)\Leftrightarrow \mathrm{d}(I,(\alpha))=\sqrt{6} \Leftrightarrow \dfrac{|c-3|}{\sqrt{6}}=\sqrt{6} \Leftrightarrow \hoac{&c-3=6\\ &c-3=-6} \Leftrightarrow \hoac{&c=9\\ &c=-3.}$
		}
\end{ex}
\begin{ex}%[2H5V1-3]
	Trong KG $Oxyz$, cho mặt phẳng $(P)$ chứa đường thẳng $d\colon \dfrac{x-4}{3}=\dfrac{y}{1}=\dfrac{z+4}{-4}$ và tiếp xúc với mặt cầu $(S)\colon (x-3)^2+(y+3)^2+(z-1)^2=9$. Khi đó $(P)$ song song với mặt phẳng nào sau đây?
	\choice
	{$3x-y+2z=0$}
	{$-2x+2y-z+4=0$}
	{$x+y+z=0$}
	{\True Đáp án khác}
	\loigiai{
		Véc-tơ chỉ phương của $d$ là $\overrightarrow{u}=(3;1;-4)$, véc-tơ pháp tuyến của mặt phẳng $(P)$ là $\overrightarrow{n}$.\\
		Mặt cầu $(S)$ có tâm $I(3;-3;1)$ và bán kính $R=3$.\\
		Vì $(P)$ chứa $d$ nên $\overrightarrow{u}\cdot \overrightarrow{n}=0$ và $(P)$ tiếp xúc với $(S)$ nên $\mathrm{d}(I,(P))=3$.\\
		Ta chỉ xét phương trình $\overrightarrow{u}\cdot \overrightarrow{n}=0$. Lấy hai điểm nằm trên đường thẳng $d$ là $M(4;0;-4)$ và $N(1;-1;0)$.\\
		Ta nhận thấy: $M(4;0;-4)$ và $N(1;-1;0)$ không thỏa mãn đáp án $A$, $B$, $C$.
		}
\end{ex}
\begin{ex}%[2H5V1-3]
	Trong KG $Oxyz$, cho mặt cầu $(S)\colon (x+1)^2+(y-1)^2+(z+2)^2=2$ và hai đường thẳng $d\colon \dfrac{x-2}{1}=\dfrac{y}{2}=\dfrac{z-1}{-1}$; $\Delta \colon \dfrac{x}{1}=\dfrac{y}{1}=\dfrac{z-1}{-1}$. Phương trình nào dưới đây là phương trình của một mặt phẳng tiếp xúc với $(S)$, song song với $d$ và $\Delta$?
	\choice
	{$y+z+3=0$}
	{\True $x+z+1=0$}
	{$x+y+1=0$}
	{$x+z-1=0$}
	\loigiai{
		Mặt cầu $(S)$ có tâm $I(-1;1;-2)$; $R=\sqrt{2}$.\\
		Véc-tơ chỉ phương của $d\colon \overrightarrow{u}_d=(1;2;-1)$. Véc-tơ chỉ phương của $\Delta$ là $\overrightarrow{u}_{\Delta}=(1;1;-1)$.\\
		Gọi $(P)$ là mặt phẳng cần viết phương trình.\\
		Ta có $\left[\overrightarrow{u}_d,\overrightarrow{u}_{\Delta}\right]=(-1;0;-1)$ nên chọn một véc-tơ pháp tuyến của $(P)$ là $\overrightarrow{n}=(1;0;1)$.\\
		Mặt phẳng $(P)$ có phương trình tổng quát dạng: $x+z+D=0$.\\
		Do $(P)$ tiếp xúc với $(S)$ nên
		\begin{eqnarray*}
			\mathrm{d}(I,(P))=R &\Leftrightarrow& \dfrac{|-1-2+D|}{\sqrt{2}}=\sqrt{2}\\
			&\Leftrightarrow& |D-3|=2 \Leftrightarrow \hoac{&D=5\\ &D=1.}
		\end{eqnarray*}
		Vậy phương trình của $(P)$ là $x+z+1=0$.
		}
\end{ex}
\begin{ex}%[2H5V1-3]
	Trong KG $Oxyz$, cho mặt cầu $(S)\colon (x-1)^2+(y-2)^2+(z-3)^2=1$, đường thẳng $\Delta \colon \dfrac{x-6}{-3}=\dfrac{y-2}{2}=\dfrac{z-2}{2}$ và điểm $M(4;3;1)$. Trong các mặt phẳng sau mặt phẳng nào đi qua $M$, song song với $\Delta$ và tiếp xúc với mặt cầu $(S)$?
	\choice
	{$2x-2y+5z-22=0$}
	{\True $2x+y+2z-13=0$}
	{$2x+y-2z-1=0$}
	{$2x-y+2z-7=0$}
	\loigiai{
		\textbf{Cách 1:}\\
		Gọi $\overrightarrow{n}=(2a;b;c)$ là véc-tơ pháp tuyến của mặt phẳng $(P)$ cần lập, $a^2+b^2+c^2\ne 0$.\\
		Đường thẳng $\Delta$ có véc-tơ chỉ phương là $\overrightarrow{u}=(-3;2;2)$.\\
		Mặt phẳng $(P)$ song song với $\Delta$ nên ta có $\overrightarrow{n}\cdot \overrightarrow{u}=0 \Leftrightarrow -6a+2b+2c=0 \Leftrightarrow c=3a-b$.\\
		Mặt phẳng $(P)$ đi qua $M$ và có véc-tơ pháp tuyến $\overrightarrow{n}$ nên phương trình có dạng:\\
		$2a(x-4)+b(y-3)+(3a-b)(z-1)=0 \Leftrightarrow 2ax+by+(3a-b)z-11a-2b=0 \quad (*)$\\
		Mặt cầu $(S)$ có tâm $I(1;2;3)$ và bán kính $R=1$.\\
		Mặt phẳng $(P)$ tiếp xúc với mặt cầu $(S)$ nên
		\begin{eqnarray*}
			\mathrm{d}(I,(P))=1 &\Leftrightarrow& \dfrac{3|b|}{\sqrt{4a^2+b^2+(3a-b)^2}}=1\\
			&\Leftrightarrow& \dfrac{3|b|}{\sqrt{13a^2+2b^2-6ab}}=1\\
			&\Leftrightarrow& 3|b|=\sqrt{13a^2+2b^2-6ab}\\
			&\Leftrightarrow& 9b^2=13a^2+2b^2-6ab\\
			&\Leftrightarrow& 13a^2-6ab-7b^2=0\\
			&\Leftrightarrow& (a-b)(13a+7b)=0\\
			&\Leftrightarrow& \hoac{&a=b\\ &13a=-7b.}
		\end{eqnarray*}
		Với $a=b$, chọn $a=1$, $b=1$ thay vào $(*)$ ta được pt $(P_1)\colon 2x+y+2z-13=0$.\\
		Ta có $N(6;2;2)\in \Delta$. Dễ thấy $N\notin (P_1)$, suy ra $(P_1)\colon 2x+y+2z-13=0$ song song với $\Delta$.\\
		Với $13a=-7b$, chọn $a=7$, $b=-13$ thay vào $(*)$ ta được pt$(P_2)\colon 14x-13y+34z-51=0$.\\
		Ta có $N(6;2;2)\in \Delta$, dễ thấy $N\notin (P_2)$, suy ra $(P_2)\colon 14x-13y+34z-51=0$ song song với $\Delta$.\\
		\textbf{Cách 2:} (Trắc nghiệm)\\
		Gọi $(P)$ là mặt phẳng thỏa mãn yêu cầu bài toán và có véc-tơ pháp tuyến là $\overrightarrow{n}$.\\
		Vì $(P)$ đi qua $M(4;3;1)$ nên phương án $A$, $C$ bị loại.\\
		Đường thẳng $\Delta$ có véc-tơ chỉ phương $\overrightarrow{u}=(-3;2;2)$. $(P)$ song song với đường thẳng $\Delta$ nên $\overrightarrow{n}\cdot \overrightarrow{u}=0$. Do đó phương án $D$ bị loại.\\
		Vậy phương án $B$ là phương án thỏa mãn yêu cầu bài toán.
		}
\end{ex}
\begin{ex}%[2H5V1-3]
	Trong KG $Oxyz$ cho mặt cầu $(S)\colon x^2+y^2+z^2-2x-4y-6z-2=0$ và mặt phẳng $(\alpha)\colon 4x+3y-12z+10=0$. Lập phương trình mặt phẳng $(\beta)$ thỏa mãn đồng thời các điều kiện: Tiếp xúc với $(S)$; song song với $(\alpha)$ và cắt trục $Oz$ ở điểm có cao độ dương.
	\choice
	{$4x+3y-12z-78=0$}
	{$4x+3y-12z-26=0$}
	{\True $4x+3y-12z+78=0$}
	{$4x+3y-12z+26=0$}
	\loigiai{
		Mặt cầu $(S)$ có tâm $I(1;2;3)$, bán kính $R=4$.\\
		Mặt phẳng $(\beta)$ song song với $(\alpha)$ nên có phương trình dạng $4x+3y-12z+c=0 \quad (c\ne 10)$.\\
		$(\beta)$ tiếp xúc với $(S)$ nên
		\begin{eqnarray*}
			\Leftrightarrow \mathrm{d}(I,(\beta))=R &\Leftrightarrow& \dfrac{|4\cdot 1+3\cdot 2-12\cdot 3+c|}{\sqrt{4^2+3^2+12^2}}=4\\
			&\Leftrightarrow& \dfrac{|-26+c|}{13}=4\\
			&\Leftrightarrow& \hoac{&-26+c=52\\ &-26+c=-52}\\
			&\Leftrightarrow& \hoac{&c=78\\ &c=-26.}
		\end{eqnarray*} 
		Nếu $c=78$ thì $(\beta)\colon 4x+3y-12z+78=0$. Mặt phẳng $(\beta)$ cắt trục $Oz$ ở điểm $M\left(0;0;\dfrac{13}{2}\right)$ có cao độ dương.\\
		Nếu $c=-26$ thì $(\beta)\colon 4x+3y-12z-26=0$. Mặt phẳng $(\beta)$ cắt trục $Oz$ ở điểm $M\left(0;0;-\dfrac{13}{6}\right)$ có cao độ âm.\\
		Vậy phương trình của $(\beta)$ là $4x+3y-12z+78=0$.
		}
\end{ex}
\begin{ex}%[2H5C1-4]
	Trong không gian với hệ trục tọa độ $Oxyz$, cho mặt cầu $(S)\colon x^2+y^2+z^2-2x+2z+1=0$ và đường thẳng $d\colon \dfrac{x}{1}=\dfrac{y-2}{1}=\dfrac{z}{-1}$. Hai mặt phẳng $(P)$, $(P')$ chứa $d$ và tiếp xúc với $(S)$ tại $T$, $T'$. Tìm tọa độ trung điểm $H$ của $TT'$.
	\choice
	{$H\left(-\dfrac{7}{6};\dfrac{1}{3};\dfrac{7}{6}\right)$}
	{$H\left(\dfrac{5}{6};\dfrac{2}{3};-\dfrac{7}{6}\right)$}
	{\True $H\left(\dfrac{5}{6};\dfrac{1}{3};-\dfrac{5}{6}\right)$}
	{$H\left(-\dfrac{5}{6};\dfrac{1}{3};\dfrac{5}{6}\right)$}
	\loigiai{
		\immini{Mặt cầu $(S)$ tâm $I(1;0;-1)$, bán kính \hfill \break $R=\sqrt{1^2+0^2+(-1)^2-1}=1$.\\
			Gọi $K$ là hình chiếu vuông góc của $I$ lên $d$.\\
			$K\in d$ nên ta có thể giả sử $K(t;2+t;-t)$.\\
			$\overrightarrow{IK}=(t-1;2+t;-t+1)$, $\overrightarrow{u}_d=(1;1;-1)$ là một véc-tơ chỉ phương của đường thẳng $d$.\\
			$IK\perp d \Leftrightarrow \overrightarrow{IK}\cdot \overrightarrow{u}_d=0$\hfill \break
			$\Leftrightarrow t-1+2+t+t-1=0 \Leftrightarrow t=0 \Rightarrow K(0;2;0)$.}
		{\begin{tikzpicture}[line join=round,scale=.45, line cap=round,thick]
			\coordinate (I) at (0,0);
			\coordinate (K) at (-6,0);
			\coordinate (F) at (-5,1);
			\coordinate (G) at (-7,-1); \coordinate (P) at (1.5,-4);\coordinate (Q) at ($(F)+(P)-(G)$);
			\coordinate (P1) at (1.5,2);
			\coordinate (F1) at ($(F)+(P1)-(G)$);
			\path[name path=c] (I) circle (2);
			\coordinate (X) at ($(K)!2.2!(F)$);
			\coordinate (Y) at ($(K)!1.5!(G)$);
			\draw (X)node[above]{$d$}--(Y);
			\coordinate (T) at (-0.72,1.87);\coordinate (T') at (-0.72,-1.87);
			\coordinate (H) at ($(T)!0.5!(T')$);
			%\path[name intersections={of=c1 and c2, by={C,D}}];
			\path[name path=ki](K)--(I);\path[name path=g](G)--(P1);\path[name path=kt'](K)--(T');
			\path[name path=f](F)--(Q);
			\path[name intersections={of=c and f, by={A,B}}];
			\path[name intersections={of=c and ki, by={C}}];
			\path[name intersections={of=c and g, by={A1,B1}}];
			\path[name intersections={of=g and f, by={D}}];
			\path[name intersections={of=g and ki, by={D1}}];
			\path[name intersections={of=g and kt', by={D2}}];
			\path (T) let \p1=($(T)-(B1)$) in circle ({veclen(\x1,\y1)});
			\begin{scope}
				\clip (T) let \p1=($(T)-(B1)$) in circle ({veclen(\x1,\y1)});
				\draw[dashed] (I) circle (2);
			\end{scope}
			\path (P) let \p2=($(P)-(B1)$) in circle ({veclen(\x2,\y2)});
			\begin{scope}
				\clip (P) let \p2=($(P)-(B1)$) in circle ({veclen(\x2,\y2)});
				\draw (I) circle (2);
			\end{scope}
			\draw(A)--(D) (F)--(G)--(P)--(Q)--(B) (F)--(F1)--(P1)--(G) (D2)--(T') (K)--(T) (C)--(D1);
			\draw[dashed](C)--(I)--(T)--(T')--(I) (F)--(D) (A)--(B) (D2)--(K)--(D1);
			\foreach \i/\g in {I/0,K/180, T/70,H/130, T'/-90}{\draw[fill=black](\i) circle (1pt) ($(\i)+(\g:4mm)$) node[scale=.6]{$\i $};};
			\draw[dashed] 
			(180:2) arc (180:0:{2} and {0.4*2});
			\draw 	
			(180:2) arc (180:360:{2} and {0.4*2});
			\clip (-8,-4.5) rectangle (4,4.5);
		\end{tikzpicture}}
	$\triangle ITK$ vuông tại $T$ có $TH$ là đường cao nên $IT^2=IH\cdot IK$.
	$\Leftrightarrow IH=\dfrac{1}{\sqrt{6}} (IK=\sqrt{6})$\\
	$\Rightarrow \overrightarrow{IH}=\dfrac{1}{6}\overrightarrow{IK}$. Giả sử $H(x;y;z)$\\
	$\Leftrightarrow \heva{&x-1=\dfrac{1}{6}\cdot (-1)\\ &y-0=\dfrac{1}{6}\cdot 2\\ &z+1=\dfrac{1}{6}\cdot 1} \Leftrightarrow \heva{&x=\dfrac{5}{6}\\ &y=\dfrac{1}{3}\\ &z=\dfrac{-5}{6}.}$\\
	Vậy $H\left(\dfrac{5}{6};\dfrac{1}{3};\dfrac{-5}{6}\right)$.
		}
\end{ex}
\begin{ex}%[2H5C1-5]
	Trong KG $Oxyz$, cho hai điểm $A(1;0;0)$, $B(0;0;2)$ và mặt cầu $(S)\colon x^2+y^2+z^2-2x-2y+1=0$. Số mặt phẳng chứa hai điểm $A$, $B$ và tiếp xúc với mặt cầu $(S)$ là
	\choice
	{\True $1$ mặt phẳng}
	{$2$ mặt phẳng}
	{$0$ mặt phẳng}
	{ vô số mặt phẳng}
	\loigiai{
		Gọi phương trình mặt phẳng là $(P)\colon Ax+By+Cz+D=0 \quad (A^2+B^2+C^2\ne 0)$.\\
		Theo đề bài, mặt phẳng qua $A$, $B$ nên ta có:\\
		$\heva{&A+D=0\\ &2C+D=0} \Leftrightarrow \heva{&A=2C\\ &D=-2C.}$\\
		Vậy mặt phẳng $(P)$ có dạng $2Cx+By+Cz-2C=0$.\\
		$(S)$ có tâm $I(1;1;0)$ và $R=1$.\\
		Vì $(P)$ tiếp xúc với $(S)$ nên\\
		$\mathrm{d}(I,(P))=R \Leftrightarrow \dfrac{2C+B-2C}{\sqrt{5C^2+B^2}}=1 \Leftrightarrow B^2=5C^2+B^2 \Leftrightarrow C=0$.\\
		Suy ra $A=D=0$.\\
		Vậy phương trình của $(P)$ là $y=0$.
		}
\end{ex}
\Closesolutionfile{ans}
\indapan{10}{ans/ans-2-C5B3CD3-LC}
\Opensolutionfile{ans}[ans/ans-C5B3CD3-KQ]
\TNSA
\begin{ex}%[2H5V1-3]
	Trong KG $Oxyz$, cho mặt phẳng $(P)\colon x-2y+z+7=0$ và mặt cầu $(S)\colon x^2+y^2+z^2-2x+4z-10=0$. Gọi $(Q)$ là mặt phẳng song song với mặt phẳng $(P)$ và cắt mặt cầu $(S)$ theo một giao tuyến là đường tròn có chu vi bằng $6\pi$. Biết phương trình của $(Q)$ có dạng $ax+by+cz+d=0$, giá trị của $a+b+c+d$ là
	\shortans{$-5$}
	\loigiai{
		Mặt cầu $(S)$ có tâm $I(1;0;-2)$, bán kính $R=\sqrt{15}$.\\
		Gọi $r$ là bán kính của đường tròn giao tuyến. Ta có $2\pi r=6\pi \Leftrightarrow r=3$.\\
		Do $(Q)\parallel (P) \Rightarrow (Q)\colon x-2y+z+d=0 \quad (d\ne 7)$.\\
		Ta có: $\mathrm{d}(I,(Q))=\sqrt{R^2-r^2}=\sqrt{6} \Leftrightarrow \dfrac{|d-1|}{\sqrt{6}}=\sqrt{6} \Leftrightarrow \hoac{&d=7\\ &d=-5.}$\\
		Suy ra $(Q)\colon x-2y+z-5=0$.\\
		Vậy $a+b+c+d=-5$.
		}
\end{ex}
\begin{ex}%[2H5C1-3]
	Trong KG $Oxyz$, cho mặt cầu $(S)\colon x^2+y^2+z^2-2x-4y-6z-2=0$ và mặt phẳng $(\alpha)\colon 4x+3y-12z+10=0$. Lập phương trình mặt phẳng $(\beta)$ thỏa mãn đồng thời các điều kiện: tiếp xúc với $(S)$; song song với $(\alpha)$ và cắt trục $Oz$ ở điểm có cao độ dương. Biết $(\beta)$ có dạng $ax+by+cz+d=0$, giá trị của $a+b+c+d$ là
	\shortans{$73$}
	\loigiai{
		Mặt cầu $(S)$ có tâm $I(1;2;3)$, bán kính $R=\sqrt{1^2+2^2+3^2+2}=4$.\\
		Vì $(\alpha)\parallel(\beta)$ nên phương trình mp $(\beta)$ có dạng: $4x+3y-12z+d=0 \quad (d\ne 10)$.\\
		Vì $(\beta)$ tiếp xúc mặt cầu $(S)$ nên\\
		$\mathrm{d}(I,(\beta))=R \Leftrightarrow \dfrac{|4\cdot 1+3\cdot 2-12\cdot 3+d|}{\sqrt{4^2+3^2+(-12)^2}}=4	\Leftrightarrow |d-26|=52 \Leftrightarrow \hoac{&d=-26\\ &d=78.}$\\
		Do $(\beta)$ cắt trục $Oz$ ở điểm có cao độ dương nên $d=78$.\\
		Suy ra mp $(\beta)\colon 4x+3y-12z+78=0$.\\
		Vậy $a+b+c+d=73$.
		}
\end{ex}
\Closesolutionfile{ans}