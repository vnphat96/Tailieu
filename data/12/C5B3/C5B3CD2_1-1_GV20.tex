%\setcounter{chude}{1}
\chude{ỨNG DỤNG MẶT CẦU TRONG KHÔNG GIAN}
%Câu 1
\begin{bt}%[2H5H3-4]
	Trong không gian hệ trục tọa độ $Oxyz$ (đơn vị trên mỗi trục là kilômét) một trạm phát sóng điện thoại của nhà mạng Vinaphone được đặt ở vị trí $I(1;-2;-3)$ và được thiết kế bán kính phủ sóng là $ 5000 $ m.
	\begin{enumerate}
		\item Sử dụng phương trình mặt cầu để mô tả ranh giới bên ngoài vùng phủ sóng trong không gian.		
		\item Nhà bạn Minh Hiền và bạn Trúc Linh có vị trí tọa độ lần lượt là $M(1;2;0)$ và $N(-3;1;0)$. Hỏi Minh Hiền và Trúc Linh dùng điện thoại tại nhà thì có thể sử dụng dịch vụ của trạm này không?
	\end{enumerate}
	\loigiai{
		\begin{enumerate}
		\item 	Phương trình mặt cầu để mô tả ranh giới bên ngoài vùng phủ sóng trong không gian là
		$$(x-1)^2+(y+2)^2+(z+3)^2=25.$$
		\item Ta có\\
		$IM=\sqrt{(1-1)^2+(2+2)^2+(0+3)^2}=5$;\\ $IN=\sqrt{(-3-1)^2+(1+2)^2+(0-3)^2}=\sqrt{34}>5$.\\
		Vì $IM=R=5$ nên điểm $M$ nằm trên mặt cầu. Vậy bạn Minh Hiền có thể sử dụng dịch vụ của trạm này.\\
		Vì $IN>R$ nên điểm $N$ nằm ngoài mặt cầu. Vậy bạn Trúc Linh không thể sử dụng dịch vụ của trạm này.
		\end{enumerate}	
	}
\end{bt}