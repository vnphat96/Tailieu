\begin{dang}{Giá trị lớn nhất, giá trị nhỏ nhất liên quan đến khoảng cách}
\end{dang}
\noindent{\bf Bài toán 1. Trong không gian hệ tọa độ $Oxyz$, cho điểm $A\left(x_0;y_0;z_0\right)$ cố định và điểm $M$ di động trên mặt phẳng $(P)\colon Ax+By+Cz+D=0$. Tìm tọa độ điểm $M$ để $AM$ có giá trị nhỏ nhất}\\
{\bf Phương pháp giải}\\
{\bf Cách 1. Phương pháp hình học}
\begin{itemize}
\item Bước 1:
\begin{center}
	\begin{tikzpicture}[scale=1, font=\footnotesize, line join=round, line cap=round, >=stealth]
	\coordinate (O) at (0,0);
	\coordinate (B) at (0.6,2);
	\coordinate (C) at (6,2);
	\coordinate (D) at ($(O)+(C)-(B)$);
	\coordinate (H) at (4,1);
	\coordinate (A) at ($(H)+(0,2)$);\coordinate (M) at ($(H)-(2,0)$);
	\coordinate (F) at (intersection of B--C and A--H);
	\coordinate (E) at (intersection of B--C and A--M);
	\draw (O)--(B)--(E) (F)--(C)--(D)--(O) (M)--(A)--(H)--(M);
	\draw[dashed] (F)--(E);
	\draw pic[draw]{angle = D--O--B} node[above right=-0.1]{$P$};
	\pic[draw,angle radius=2mm] {right angle = A--H--M};
	\foreach \i/\g in {M/-90,A/90,H/-90}{\draw[fill=black](\i) circle (1.5pt) ($(\i)+(\g:3mm)$) node[scale=1]{$\i$};}
\end{tikzpicture}
\end{center}
+ Gọi $H$ là hình chiếu vuông góc của $A$ trên mặt phẳng $(P)$.\\
+ Khi đó, tam giác $AHM$ vuông tại $H$, suy ra $AM\ge AH$. \\
+ Đẳng thức xảy ra khi $M\equiv H$. \\
+ Do đó $AM$ nhỏ nhất khi $M$ là hình chiếu của $A$ trên mặt phẳng $(P)$.
\item Bước 2: Tìm tọa độ điểm $M$ ($M\equiv H$).\\
+ Lập phương trình tham số đường thẳng $AH$ với $\heva{
	& \text{ đi qua điểm }A \\ 
	&\overrightarrow{u}_{AH}=\overrightarrow{n}_P=(a;b;c).}$\\
+ Ta có $M\equiv H=AH\cap (P)\Rightarrow $ tọa độ điểm $M$ cần tìm. 
\end{itemize}
{\bf Cách 2. Phương pháp đại số}\\
Dùng bất đẳng thức bộ $3$ của Bunhiacốpxki.\\
Với $a$, $b$, $c$, $x$, $y$, $z\in \mathbb{R}$, ta có
$$(ax+by+cz)^2\le (a^2+b^2+c^2)(x^2+y^2+z^2).$$
Dấu bằng xảy ra $\Leftrightarrow \dfrac{a}{x}=\dfrac{b}{y}=\dfrac{c}{z}$.\\
\noindent{\bf Bài toán $2$. Trong không gian $Oxyz$, cho hai điểm $A$, $B$ cố định. Lập phương trình mặt phẳng $(P)$ đi qua $A$ và cách $B$ một khoảng lớn nhất.}\\
{\bf Phương pháp giải}\\
{\bf Cách 1: Phương pháp hình học}
\begin{center}
	\begin{tikzpicture}[scale=1, font=\footnotesize, line join=round, line cap=round, >=stealth]
		\coordinate (O) at (0,0);
		\coordinate (I) at (1,2);
		\coordinate (J) at (6,2);
		\coordinate (Q) at ($(O)+(J)-(I)$);
		\coordinate (H) at (4,1);
		\coordinate (B) at ($(H)+(0,2)$);\coordinate (A) at ($(H)-(2,0)$);
		\coordinate (F) at (intersection of I--J and B--H);
		\coordinate (E) at (intersection of I--J and B--A);
		\draw (O)--(I)--(E) (F)--(J)--(Q)--(O) (A)--(B)--(H);
		\draw[dashed] (F)--(E);
		\draw pic[draw]{angle = Q--O--I} node[above right=-0.1]{$P$};
		\pic[draw,angle radius=2mm] {right angle = A--H--B};
		\foreach \i/\g in {B/90,A/-90,H/-90}{\draw[fill=black](\i) circle (1pt) ($(\i)+(\g:3mm)$) node[scale=1]{$\i$};}
	\end{tikzpicture}
\end{center}
\begin{itemize}
\item Bước 1:\\
+ Gọi $H$ là hình chiếu của $B$ lên mặt phẳng $(P)$.\\
+ Khi đó $\mathrm{d}(B,(P))=BH\le BA$.\\
+ Do đó $(P)$ là mặt phẳng đi qua $A$ vuông góc với $AB$.
\item Bước 2: Lập phương trình mặt phẳng $(P)$ với $\heva{
	& \text{đi qua điểm } A \\ 
	& \overrightarrow{n}_P=\overrightarrow{AB}.}$
\end{itemize}
{\bf Cách 2: Phương pháp đại số}
\begin{itemize}
\item Gọi $\overrightarrow{n}_P=(a;b;c)$, $(a^2+b^2+c^2\ne 0 $ là một véc-tơ pháp tuyến của mặt phẳng $(P)$. 
\item Khi đó phương trình mặt phẳng $(P)$ đi qua điểm $A$ là
$$a(x-x_A)+b(y-y_A)+c(z-z_A)=0.$$
\item Khi đó $\overrightarrow{n}_P\cdot \overrightarrow{AB}=0$ từ đây ta rút được theo $a$ theo $b$, $c$ (hoặc $b$ theo $a$, $c$ hoặc $c$ theo $a$, $b$).
\item Ta có $\mathrm{d}(B;(P))=f(t)$, với $t=\dfrac{b}{c}$, $c\neq 0$.\\
+ Khảo sát $f(t)$ ta tìm được max của $f(t)$.
\begin{note}
Để tìm véc-tơ pháp tuyến $\overrightarrow{n}_P$ của $(P)$ đơn giản hơn thì nên gọi $\overrightarrow{n}_P=(1;b;c)$.
\end{note}
\end{itemize}
{\bf Bài toán 3. Trong không gian hệ tọa độ $Oxyz$, cho điểm $A$ và đường thẳng $\Delta $ cố định. Lập phương trình mặt phẳng $(P)$ đi qua đường thẳng $\Delta $ và cách $A$ một khoảng lớn nhất.}\\
{\bf Phương pháp giải}\\
{\bf Cách 1: Phương pháp hình học}

\begin{itemize}
\item Bước 1
\begin{center}
	\begin{tikzpicture}[scale=1, font=\footnotesize, line join=round, line cap=round, >=stealth]
		\coordinate (O) at (0,0);
		\coordinate (B) at (0.6,2);
		\coordinate (C) at (6,2);
		\coordinate (D) at ($(O)+(C)-(B)$);
		\coordinate (H) at (2,1);\coordinate (K) at (4,1);
		\coordinate (A) at ($(H)+(0,2)$);
		\coordinate (T) at ($(C)!5/3!(K)$);\coordinate (U) at ($(K)!1/3!(C)$);
		\coordinate (E) at (intersection of B--C and A--H);
		\coordinate (F) at (intersection of B--C and A--K);
		\draw (O)--(B)--(E) (F)--(C)--(D)--(O) (M)--(A)--(H) (A)--(K) (T)--(U);
		\draw[dashed] (F)--(E);
		\draw pic[draw]{angle = D--O--B} node[above right=-0.1]{$P$};
		\pic[draw,angle radius=2mm] {right angle = A--H--K};
		\pic[draw,angle radius=2mm] {right angle = A--K--U};
		\foreach \i/\g in {K/-90,A/90,H/-90}{\draw[fill=black](\i) circle (1.5pt) ($(\i)+(\g:3mm)$) node[scale=1]{$\i$};}
		\draw (U) node[above]{$\Delta$};
	\end{tikzpicture}
\end{center}
+ Gọi $H$, $K$ lần lượt là hình chiếu của $A$ lên mặt phẳng $(P)$ và đường thẳng $\Delta$.\\   
+ Khi đó $\mathrm{d}(A,(P))=AH\le AK$.\\
+ Do đó $(P)$ là mặt phẳng đi qua $K$ và vuông góc với $AK$. 
\item Bước 2.\\
+ Tìm tọa độ điểm $K$.\\
+ Lập phương trình mặt phẳng $(P)$ với $\heva{
	& \text{đi qua điểm } K \\ 
	& \overrightarrow{n}_P=\overrightarrow{AK}.}$
\end{itemize}
{\bf Cách 2: Phương pháp đại số}\\
+ Gọi $\overrightarrow{n}_P=(a;b;c)$, $(a^2+b^2+c^2\ne 0)$ là một véc-tơ pháp tuyến của mặt phẳng $(P)$.\\
+ Khi đó $\overrightarrow{n}_P\cdot \overrightarrow{u}_{\Delta }=0$ từ đây ta rút được $a$ theo $b$, $c$ (hoặc $b$ theo $a, c$ hoặc $c$ theo $a, b$).\\
+ Ta có $\mathrm{d}(A,(P))=f(t)$, với $t=\dfrac{b}{c}$, $c\neq 0$.\\ 
+ Khảo sát $f(t)$ ta tìm được max của $f(t)$.
\begin{note}
Để tìm véc-tơ pháp tuyến $\overrightarrow{n}_P$ của $(P)$ đơn giản hơn thì nên gọi $\overrightarrow{n}_P=(1;b;c)$.
\end{note}
{\bf Bài toán 4. Trong không gian hệ tọa độ $Oxyz$, cho mặt phẳng $(P)$ và hai điểm phân biệt $A$, $B$. Tìm điểm $M$ thuộc $(P)$ sao cho 
\begin{enumEX}{2}
\item $MA+MB$ nhỏ nhất.
\item $|MA-MB|$ lớn nhất.
\end{enumEX}
}
{\bf Phương pháp giải}
\begin{enumEX}{1}
\item $MA+MB$ nhỏ nhất.\\
Ta xét hai trường hợp sau:
\begin{itemize}
\item TH1: Nếu $A$ và $B$ nằm về hai phía so với $(P)$. 
Khi đó $AM+BM\ge AB$.\\
Đẳng thức xảy ra khi $M$ là giao điểm của $AB$ với $(P)$.
\item TH2: Nếu $A$ và $B$ nằm cùng một phía so với $(P)$.\\
Gọi $A'$ đối xứng với $A$ qua $(P)$.\\
Khi đó $AM+BM=A'M+BM\ge A'B$.\\
Đẳng thức xảy ra khi $M$ là giao điểm của $A'B$ với $(P)$.
\begin{center}
	\begin{tikzpicture}[scale=1, font=\footnotesize, line join=round, line cap=round, >=stealth]
		\coordinate (O) at (0,0);
		\coordinate (P) at (0.6,2);
		\coordinate (Q) at (6,2);
		\coordinate (R) at ($(O)+(Q)-(P)$);
		\coordinate (H) at (2,1);\coordinate (M) at (4,1);
		\coordinate (A) at ($(H)+(-1,2)$);
		\coordinate (B) at ($(A)!2!(H)$);
		\coordinate (U) at (intersection of A--B and O--R);
		\coordinate (V) at (intersection of M--B and O--R);
		\coordinate (E) at (intersection of P--Q and A--H);
		\coordinate (F) at (intersection of P--Q and A--M);
		\draw (O)--(P) (Q)--(R)--(O) (H)--(A)--(M) (V)--(B)--(U) (P)--(E) (F)--(Q);
		\draw[dashed] (H)--(U) (M)--(V) (E)--(F);
		\draw pic[draw]{angle = R--O--P} node[above right=-0.1]{$P$};
		\draw[fill=black] (H) circle (1.5pt);
		\foreach \i/\g in {B/-90,A/90,M/0}{\draw[fill=black](\i) circle (1.5pt) ($(\i)+(\g:3mm)$) node[scale=1]{$\i$};}
	\end{tikzpicture}
		\begin{tikzpicture}[scale=1, font=\footnotesize, line join=round, line cap=round, >=stealth]
		\coordinate (O) at (0,0);
		\coordinate (P) at (0.6,2);
		\coordinate (Q) at (6,2);
		\coordinate (R) at ($(O)+(Q)-(P)$);
		\coordinate (H) at (2,1);\coordinate (M) at (3,1.5);
		\coordinate (K) at (4,1);
		\coordinate (A) at ($(H)+(0,2)$);
		\coordinate (A') at ($(A)!2!(H)$);
		\coordinate (B) at ($(A')!1.8!(K)$);
		\coordinate (U) at (intersection of A--A' and O--R);
		\coordinate (V) at (intersection of A'--B and O--R);
		\coordinate (E) at (intersection of P--Q and A--H);
		\coordinate (F) at (intersection of P--Q and A--M);
		\coordinate (I) at (intersection of P--Q and B--M);
		\coordinate (J) at (intersection of P--Q and A'--B);
		\draw (O)--(P) (Q)--(R)--(O) (H)--(A)--(M)--(B)--(K) (V)--(A')--(U) (P)--(E) (F)--(I) (J)--(Q);
		\draw[dashed] (H)--(U) (K)--(V) (M)--(A') (E)--(F) (I)--(J);
		\draw pic[draw]{angle = R--O--P} node[above right=-0.1]{$P$};
		\pic[draw,angle radius=2mm] {right angle = A--H--M};
		\draw[fill=black] (H) circle (1.5pt);
		\foreach \i/\g in {B/90,A/90,M/180,A'/-90,H/180}{\draw[fill=black](\i) circle (1.5pt) ($(\i)+(\g:3mm)$) node[scale=1]{$\i$};}
	\end{tikzpicture}
\end{center}
\end{itemize}
\item $|MA-MB|$ lớn nhất.\\
Ta xét hai trường hợp sau:
\begin{itemize}
\item TH1: Nếu $A$ và $B$ nằm cùng một phía so với $(P)$.\\
+ Khi đó $|AM-BM|\le AB$.\\
+ Đẳng thức xảy ra khi $M$ là giao điểm của $AB$ với $(P)$.
\item TH2: Nếu $A$ và $B$ nằm khác phía so với $(P)$.\\
+ Gọi $A'$ đối xứng với $A$ qua $(P)$. \\
+ Khi đó $|AM-BM|=|A'M-BM|\le AB$.\\
+ Đẳng thức xảy ra khi $M$ là giao điểm của $A'B$ với $(P)$.
\end{itemize}
\end{enumEX}
{\bf Bài toán 5. Trong không gian hệ tọa độ $Oxyz$, cho các số thực dương $\alpha$, $\beta$ và ba điểm $A$, $B$, $C$. Viết phương trình mặt phẳng $(P)$ đi qua $C$ và thỏa mãn $T=\alpha \mathrm{d}(A,(P))+\beta \mathrm{d}(B,(P))$  nhỏ nhất.}\\
{\bf Phương pháp giải}
\begin{itemize}
\item TH1. Xét  $A$, $B$ nằm về cùng phía so với $(P)$.\\
+ Nếu $AB\parallel (P)$ thì $P=\left(\alpha+\beta\right)\mathrm{d}(A,(P))\le \left(\alpha+\beta\right)AC$.\\
+ Nếu đường thẳng $AB$ cắt $(P)$ tại $I$. Gọi $D$ là điểm thỏa mãn $\overrightarrow{IB}=\dfrac{\alpha}{\beta}\overrightarrow{ID}$ và $E$ là trung điểm $BD$.\\
Khi đó $P=\alpha \mathrm{d}(A;(P))+\beta\cdot \dfrac{IB}{ID}\cdot \mathrm{d}(D,(P))=2\alpha \mathrm{d}(E,(P))=2\left(\alpha+\beta\right)EC$.
\item TH2. Xét $A$, $B$ nằm về hai phía so với $(P)$.\\
Gọi $I$ là giao điểm của $AB$ và $(P)$, $B'$ là điểm đối xứng với $B$ qua $I$.\\ 
Khi đó $P=\alpha \mathrm{d}(A,(P))+\beta \mathrm{d}(B',(P))$.\\
Đến đây ta chuyển về {\bf Bài toán 4} trên.
\end{itemize}
{\bf Bài toán 6. Trong không gian hệ tọa độ $Oxyz$, cho $n$ điểm $A_1$, $A_2$,\ldots, $A_n$ và điểm $A$. Viết phương trình mặt phẳng $(P)$ đi qua $A$ và tổng khoảng cách từ các điểm $A_i$ ($i=\overline{1,n}$) lớn nhất.}\\
{\bf Phương pháp giải}
\begin{itemize}
\item Xét $n$ điểm $A_1$, $A_2$,\ldots, $A_n$ nằm cùng phía so với $(P)$. Gọi $G$ là trọng tâm của $n$ điểm đã cho.\\ 
Khi đó $\sum\limits_{i=1}^{n}\mathrm{d}(A_i,(P))=n\mathrm{d}(G,(P))\le n GA$.
\item Trong $n$ điểm trên có $m$ điểm nằm về một phía và $k$ điểm nằm về phía khác ($m+k=n$). Khi đó, gọi $G_1$ là trọng tâm của $m$ điểm, $G_2$ là trọng tâm của  điểm $G_3$ đối xứng với $G_1$ qua $A$.\\
Khi dó $P=m \mathrm{d}(G_3,(P))+k\mathrm{d}(G_2,(P))$.
\end{itemize}
Đến đây ta chuyển về {\bf Bài toán 5} trên.
\Opensolutionfile{ans}[ans/ans-C5B3CD5]
\TN
%Câu 37
\begin{ex}%[2H5V1-5] 
	Trong không gian hệ tọa độ $Oxyz$, cho điểm $A(1;2;-2)$. Gọi $(P)$ là mặt phẳng chứa trục $Ox$ sao cho khoảng cách từ $A$ đến $(P)$ lớn nhất. Phương trình của $(P)$ là
	\choice
	{$2y+z=0$}
	{$2y-z=0$ }
	{$y+z=0$}
	{\True $y-z=0$}
	\loigiai{
		\begin{center}
			\begin{tikzpicture}[scale=1, font=\footnotesize, line join=round, line cap=round, >=stealth]
				\coordinate (O) at (0,0);
				\coordinate (B) at (0.6,2);
				\coordinate (C) at (6,2);
				\coordinate (D) at ($(O)+(C)-(B)$);
				\coordinate (H) at (2,1);\coordinate (K) at (4,1);
				\coordinate (A) at ($(H)+(0,2)$);
				\coordinate (T) at ($(C)!5/3!(K)$);\coordinate (U) at ($(K)!1/3!(C)$);
				\coordinate (E) at (intersection of B--C and A--H);
				\coordinate (F) at (intersection of B--C and A--K);
				\draw (O)--(B)--(E) (F)--(C)--(D)--(O) (A)--(H) (A)--(K) (T)--(U);
				\draw[dashed] (F)--(E);
				\draw pic[draw]{angle = D--O--B} node[above right=-0.1]{$P$};
				\pic[draw,angle radius=2mm] {right angle = A--H--K};
				\foreach \i/\g in {K/-90,A/90,H/-90}{\draw[fill=black](\i) circle (1.5pt) ($(\i)+(\g:3mm)$) node[scale=1]{$\i$};}
				\draw (U) node[right]{$Ox$};
			\end{tikzpicture}
		\end{center}
		Gọi $K$ là hình chiếu vuông góc của $A(1;2;-2)$ trên $Ox$, suy ra $K(1;0;0),$
		$\overrightarrow{AK}=(0;-2;2)$.\\
		Gọi $H$ là điểm chiếu của $A$ lên mặt phẳng $(P)$.
		Ta có $\mathrm{d}(A,(P))=AH\le AK=2\sqrt{2}$.\\
		Suy ra $\mathrm{d}(A,(P))=2\sqrt{2}$, đạt được khi $H\equiv K(1;0;0)$.\\
		Khi đó mặt phẳng $(P)$ qua $O(0;0;0)$ có một véc-tơ pháp tuyến là $\overrightarrow{AK}=(0;-2;2)$.\\
		Phương trình mặt phẳng $(P)$ là $0(x-1)-2(y-0)+2(z-0)=0\Leftrightarrow y-z=0$.\\
		Vậy $(P)\colon y-z=0$.
	} \end{ex} 
%Câu 38
\begin{ex}%[2H5V1-5]
	Trong không gian hệ trục tọa độ $Oxyz$, mặt phẳng $(P)$ đi qua điểm $A(1;7;2)$ và cách $M(-2;4;-1)$ một khoảng lớn nhất có phương trình là
	\choice
	{$(P)\colon 3x+3y+3z-10=0$}
	{$(P)\colon x+y+z-1=0$}
	{\True $(P)\colon x+y+z-10=0$}
	{$(P)\colon x+y+z+10=0$}
	\loigiai{
	Ta có $\mathrm{d}(M,(P))\le MA$.\\
	Khi đó $\mathrm{d}(M,(P))$ lớn nhất là bằng $MA$ khi $A$ là hình chiếu của $M$ trên mặt phẳng $(P)$.\\
	Suy ra $AM\perp (P)\Rightarrow \overrightarrow{AM}=(-3;-3;-3)$ là véc-tơ pháp tuyến của $(P)$.\\
	Mặt phẳng $(P)$ đi qua $A(1;7;2)$ và nhận $\overrightarrow{AM}=(-3;-3;-3)$ là véc-tơ pháp tuyến nên có phương trình $-3(x-1)-3(y-7)-3(z-2)=0\Leftrightarrow x+y+z-10=0$.	
	}
\end{ex}
%Câu 39
\begin{ex}%[2H5V1-5]
	Trong không gian với hệ tọa độ $Oxyz$, cho điểm $A(2;-1;-2)$ và đường thẳng $d$ có phương trình $\dfrac{x-1}{1}=\dfrac{y-1}{-1}=\dfrac{z-1}{1}$. Gọi $(P)$ là mặt phẳng đi qua điểm $A$, song song với đường thẳng $d$ và khoảng cách từ $d$ tới mặt phẳng $(P)$ là lớn nhất. Khi đó mặt phẳng $(P)$ vuông góc với mặt phẳng nào sau đây?
	\choice
	{$x-y-6=0$}
	{$x+3y+2z+10$}
	{$x-2y-3z-1=0$}
	{\True $3x+z+2=0$}
	\loigiai{
	\begin{center}
	\begin{tikzpicture}[scale=1, font=\footnotesize, line join=round, line cap=round, >=stealth]
		\coordinate (O) at (0,0);
		\coordinate (B) at (0.6,2);
		\coordinate (C) at (6,2);
		\coordinate (D) at ($(O)+(C)-(B)$);
		\coordinate (A) at (1,1);\coordinate (K) at (3,1);
		\coordinate (H) at ($(K)+(0,2)$);
		\coordinate (T) at ($(H)+(-2,0)$);\coordinate (U) at ($(T)!2!(H)$);
		\coordinate (E) at (intersection of B--C and A--H);
		\coordinate (F) at (intersection of B--C and H--K);
		\draw (O)--(B)--(E) (F)--(C)--(D)--(O) (A)--(H) (A)--(K) (T)--(U) (H)--(K);
		\draw[dashed] (F)--(E);
		\draw pic[draw]{angle = D--O--B} node[above right=-0.1]{$P$};
		\pic[draw,angle radius=2mm] {right angle = A--H--T};
		\foreach \i/\g in {K/-90,A/180,H/90}{\draw[fill=black](\i) circle (1.5pt) ($(\i)+(\g:3mm)$) node[scale=1]{$\i$};}
		\draw (U) node[right]{$d$};
	\end{tikzpicture}
\end{center}
	Gọi $H$ là hình chiếu của $A$ lên đường thẳng $d$, suy ra $H(1;1;1)$.\\
	Gọi $(P)$ là mặt phẳng đi qua điểm $A$ và $(P)$ song song với đường thẳng $d$.\\
	Gọi $K$ là hình chiếu của $H$ lên mặt phẳng $(P)$.\\
	Do $d\parallel (P)$ nên ta có $\mathrm{d}(d,(P))=\mathrm{d}(H,(P))=HK$.\\
	Ta luôn có bất đẳng thức $HK\le HA$.\\
	Như vậy khoảng cách từ $d$ đến $(P)$ lớn nhất bằng $AH$.\\
	Và khi đó $(P)$ nhận $\overrightarrow{AH}=(-1;2;3)$ làm véc-tơ pháp tuyến.\\
	Do $(P)$ đi qua $A(2;-1;-2)$ nên ta có phương trình của $(P)$ là $x-2y-3z-10=0$.\\
	Do đó $(P)$ vuông góc với mặt phẳng có phương trình $3x+z+2=0$.
	}
\end{ex}
%Câu 40
\begin{ex}%[2H5V1-5]
	Trong không gian với hệ toạ độ $Oxyz$, gọi $(P)$ là mặt phẳng đi qua hai điểm $A(1;-7;-8)$, $B(2;-5;-9)$ sao cho khoảng cách từ điểm $M(7;-1;-2)$ đến $(P)$ đạt giá trị lớn nhất. Biết $(P)$ có một véc-tơ pháp tuyến là $\overrightarrow{n}=(a;b;4)$, khi đó giá trị của tổng $a+b$ là
	\choice
	{$-1$}
	{\True $3$}
	{$6$}
	{$2$}
	\loigiai{
		\begin{center}
		\begin{tikzpicture}[scale=1, font=\footnotesize, line join=round, line cap=round, >=stealth]
			\coordinate (O) at (0,0);
			\coordinate (B) at (0.6,2);
			\coordinate (C) at (6,2);
			\coordinate (D) at ($(O)+(C)-(B)$);
			\coordinate (H) at (2,1);\coordinate (K) at (4,1);
			\coordinate (M) at ($(H)+(0,2)$);
			\coordinate (T) at ($(C)!5/3!(K)$);\coordinate (U) at ($(K)!1/3!(C)$);
			\coordinate (E) at (intersection of B--C and M--H);
			\coordinate (F) at (intersection of B--C and M--K);
			\draw (O)--(B)--(E) (F)--(C)--(D)--(O) (M)--(H) (M)--(K) (T)--(U);
			\draw[dashed] (F)--(E);
			\draw pic[draw]{angle = D--O--B} node[above right=-0.1]{$P$};
			\pic[draw,angle radius=2mm] {right angle = M--H--K};
			\pic[draw,angle radius=2mm] {right angle = M--K--U};
			\foreach \i/\g in {K/-90,M/90,H/-90}{\draw[fill=black](\i) circle (1.5pt) ($(\i)+(\g:3mm)$) node[scale=1]{$\i$};}
			\draw (U) node[right]{$AB$};
		\end{tikzpicture}
	\end{center}
	Phương trình tham số của đường thẳng $AB$ là $\heva{&x=1+t\\&y=-7+2t\\&z=-8-t.}$\\
	Gọi $H$, $K$ lần lượt là hình chiếu của $M$ trên $(P)$ và đường thẳng $AB$.\\
	Vì $K\in AB$ nên $K(1+t;-7+2t;-8-t)\Rightarrow \vec{MK}=(6-t;6-2t;6+t)$.\\
	Đường thẳng $AB$ có véc-tơ chỉ phưng là $\vec{u}=(1;2;-1)$.\\
	Ta có $\vec{MK}\cdot \vec{u}=0\Leftrightarrow 6-t+2(6-2t)-(6+t)=0\Leftrightarrow t=2$.\\
	Ta tìm được điểm $K(3;-3;-10)$.\\
	Ta luôn có bất đẳng thức $\mathrm{d}(M;(P))=MH\le AK$.\\
	Dấu bằng xảy ra khi và chỉ khi $H\equiv K$. Khi đó $\overrightarrow{MH}=(-4;-2;-8)=-2(2;1;4)$.\\
	Mặt phẳng $(P)$ có một véc-tơ pháp tuyến là $\overrightarrow{n}=(2;1;4)$.\\
	Vậy ta có $a+b=3$.
	}
\end{ex}
%Câu 41
\begin{ex}%[2H5V1-5]
	Trong không gian với hệ tọa độ $Oxyz$, cho điểm $A(3;-1;0)$ và đường thẳng $d\colon \dfrac{x-2}{-1}=\dfrac{y+1}{2}=\dfrac{z-1}{1}$. Mặt phẳng $(\alpha)$ chứa $d$ sao cho khoảng cách từ $A$ đến $(\alpha)$ lớn nhất có phương trình là
	\choice
	{$x+y-z-2=0$}
	{\True $x+y-z=0$}
	{$x+y-z+1=0$}
	{$-x+2y+z+5=0$}
	\loigiai{
		\begin{center}
		\begin{tikzpicture}[scale=1, font=\footnotesize, line join=round, line cap=round, >=stealth]
			\coordinate (O) at (0,0);
			\coordinate (B) at (0.6,2);
			\coordinate (C) at (6,2);
			\coordinate (D) at ($(O)+(C)-(B)$);
			\coordinate (H) at (2,1);\coordinate (K) at (4,1);
			\coordinate (A) at ($(H)+(0,2)$);
			\coordinate (T) at ($(C)!5/3!(K)$);\coordinate (U) at ($(K)!1/3!(C)$);
			\coordinate (E) at (intersection of B--C and A--H);
			\coordinate (F) at (intersection of B--C and A--K);
			\draw (O)--(B)--(E) (F)--(C)--(D)--(O) (A)--(H) (A)--(K) (T)--(U);
			\draw[dashed] (F)--(E);
			\draw pic[draw]{angle = D--O--B} node[above right=-0.1]{$P$};
			\pic[draw,angle radius=2mm] {right angle = A--H--K};
			\pic[draw,angle radius=2mm] {right angle = A--K--U};
			\foreach \i/\g in {K/-90,A/90,H/-90}{\draw[fill=black](\i) circle (1.5pt) ($(\i)+(\g:3mm)$) node[scale=1]{$\i$};}
			\draw (U) node[right]{$d$};
		\end{tikzpicture}
	\end{center}
	Gọi $H$, $K$ lần lượt là hình chiếu của $A$ lên $(\alpha)$ và $d$, suy ra $AH\le AK$.\\
	Vì $H\in d$ nên $H(2-t;-1+2t;1+t)\Rightarrow \overrightarrow{AH}=(-1-t;2t;1+t)$.\\
	Do $AH\perp d$ nên ta có $-(-1-t)+2\cdot 2t+1+t=0\Leftrightarrow t=-\dfrac{1}{3}$.\\
	Khi đó $\overrightarrow{AH}=\left(-\dfrac{2}{3};-\dfrac{2}{3};\dfrac{2}{3}\right)$.\\
	Khoảng cách từ $A$ đến $(\alpha)$ lớn nhất khi và chỉ khi $AH=AK$.\\
	Do đó $(\alpha)$ có véc-tơ pháp tuyến là $\overrightarrow{n}=(1;1;-1)$.\\
	Vậy $(\alpha)\colon 1\cdot (x-2)+1\cdot (y+1)-1\cdot (z-1)=0\Leftrightarrow x+y-z=0$.
	\begin{note}
	Vẫn là đánh giá bất đẳng thức $AH\le AK$ nói trên, nhưng bài toán sau đây lại phát biểu hơi khác một chút.
	\end{note}
	}
\end{ex}
%Câu 42
\begin{ex}%[2H5V1-5]
	Trong không gian hệ tọa độ $Oxyz$, cho đường thẳng $d\colon \dfrac{x+1}{-2}=\dfrac{y}{1}=\dfrac{z-1}{1}$ và điểm $A(1;2;3)$. Gọi $(P)$ là mặt phẳng chứa $d$ và cách điểm $A$ một khoảng cách lớn nhất. Véc-tơ nào dưới đây là một véc-tơ pháp tuyến của $(P)$.
	\choice
	{$\overrightarrow{n}=(1;0;2)$}
	{$\overrightarrow{n}=(1;0;-2)$}
	{\True $\overrightarrow{n}=(1;1;1)$}
	{$\overrightarrow{n}=(1;1;-1)$}
	\loigiai{
	\begin{center}
	\begin{tikzpicture}[scale=1, font=\footnotesize, line join=round, line cap=round, >=stealth]
		\coordinate (O) at (0,0);
		\coordinate (B) at (0.6,2);
		\coordinate (C) at (6,2);
		\coordinate (D) at ($(O)+(C)-(B)$);
		\coordinate (K) at (2,1);\coordinate (H) at (4,1);
		\coordinate (A) at ($(K)+(0,2)$);
		\coordinate (T) at ($(C)!5/3!(H)$);\coordinate (U) at ($(H)!1/3!(C)$);
		\coordinate (E) at (intersection of B--C and A--H);
		\coordinate (F) at (intersection of B--C and A--K);
		\draw (O)--(B)--(E) (F)--(C)--(D)--(O) (A)--(H) (A)--(K) (T)--(U);
		\draw[dashed] (F)--(E);
		\draw pic[draw]{angle = D--O--B} node[above right=-0.1]{$P$};
		\pic[draw,angle radius=2mm] {right angle = A--H--U};
		\pic[draw,angle radius=2mm] {right angle = A--K--H};
		\foreach \i/\g in {K/-90,A/90,H/-90}{\draw[fill=black](\i) circle (1.5pt) ($(\i)+(\g:3mm)$) node[scale=1]{$\i$};}
		\draw (U) node[right]{$d$};
	\end{tikzpicture}
\end{center}
	Gọi $H$ là hình chiếu vuông góc của $A$ lên đường thẳng $d$, gọi $K$ là hình chiếu vuông góc của $A$ lên $(P)$.\\
	Do đó khoảng cách từ $A$ đến $(P)$ là $\mathrm{d}(A;(P))=AK$.\\
	Ta có $d\colon \heva{&x=-1-2t\\&y=t\\&z=1+t}$. Vì $H\in d$ nên $H(-2t-1;t;t+1)$.\\
	Suy ra $\overrightarrow{AH}=(-2t-2;t-2;t-2)$, véc-tơ chỉ phương của đường thẳng $d$ là $\overrightarrow{u}_d=(-2;1;1)$.\\
	Ta có $\overrightarrow{AH}\cdot \overrightarrow{u}_d=0\Leftrightarrow -2(-2t-2)+t-2+t-2=0\Leftrightarrow t=0$.\\
	Do đó $H(-1;0;1)$ và $\overrightarrow{AH}=(-2;-2;-2)\Rightarrow AH=2\sqrt{3}$ (không đổi).\\
	Vì $AK\le AH$ (đường vuông góc luôn ngắn hơn đường xiên) nên $AK$ lớn nhất khi $AK=AH$ hay $K\equiv H$.\\
	Suy ra $\overrightarrow{AK}=\overrightarrow{AH}=(-2;-2;-2)=-2(1;1;1)$.\\
	Vậy một véc-tơ pháp tuyến của $(P)$ là $\overrightarrow{n}=(1;1;1)$.
	}
\end{ex}
%Câu 43
\begin{ex}%[2H5V1-5]
	Trong không gian hệ tọa độ $Oxyz$, cho điểm $A(2;1;3)$ và mặt phẳng $(P)$ có phương trình $x+my+(2m+1)z-m-2=0$, $m$ là tham số. Gọi $H(a;b;c)$ là hình chiếu vuông góc của điểm $A$ trên $(P)$. Tính $a+b$ khi khoảng cách từ điểm $A$ đến  $(P)$ lớn nhất?
	\choice
	{$a+b=-\dfrac{1}{2}$}
	{$a+b=2$}
	{$a+b=0$}
	{\True $a+b=\dfrac{3}{2}$}
	\loigiai{
	Ta có $x+my+(2m+1)z-m-2=0\Leftrightarrow m(y+2z-1)+x+z-2=0\, (*)$.\\
	Phương trình $(*) $ có nghiệm với $\forall m\Leftrightarrow \heva{&y+2z-1=0\\&x+z-2=0.}$\\
	Suy ra $(P)$ luôn đi qua đường thẳng $d\colon \heva{&x=2-t\\&y=1-2t\\&z=t.}$
			\begin{center}
		\begin{tikzpicture}[scale=1, font=\footnotesize, line join=round, line cap=round, >=stealth]
			\coordinate (O) at (0,0);
			\coordinate (B) at (0.6,2);
			\coordinate (C) at (6,2);
			\coordinate (D) at ($(O)+(C)-(B)$);
			\coordinate (H) at (2,1);\coordinate (K) at (4,1);
			\coordinate (A) at ($(H)+(0,2)$);
			\coordinate (T) at ($(C)!5/3!(K)$);\coordinate (U) at ($(K)!1/3!(C)$);
			\coordinate (E) at (intersection of B--C and A--H);
			\coordinate (F) at (intersection of B--C and A--K);
			\draw (O)--(B)--(E) (F)--(C)--(D)--(O) (A)--(H) (A)--(K) (T)--(U);
			\draw[dashed] (F)--(E);
			\draw pic[draw]{angle = D--O--B} node[above right=-0.1]{$P$};
			\pic[draw,angle radius=2mm] {right angle = A--H--K};
			\pic[draw,angle radius=2mm] {right angle = A--K--U};
			\foreach \i/\g in {K/-90,A/90,H/-90}{\draw[fill=black](\i) circle (1.5pt) ($(\i)+(\g:3mm)$) node[scale=1]{$\i$};}
			\draw (V) node[above]{$d$};
			\draw[->] (K)--(U) node[right]{$\overrightarrow{u}$};
		\end{tikzpicture}
	\end{center}
	Gọi $K$ là hình chiếu vuông góc của $A$ trên $d$.\\
	Vì $K\in d\Rightarrow K(2-t;1-2t;t)\Rightarrow \overrightarrow{AK}=(-t;-2t;t-3)$.\\
	Đường thẳng $d$ có véc-tơ chỉ phương $\overrightarrow{u}=(-1;-2;1)$.\\
	Ta có $\overrightarrow{AK}\cdot \overrightarrow{u}=0\Leftrightarrow t+4t+t-3=0\Leftrightarrow t=\dfrac{1}{2}\Rightarrow K\left(\dfrac{3}{2};0;\dfrac{1}{2}\right)$.\\
	Ta có $AH\le AK\Rightarrow AH$ lớn nhất khi $AH=AK$ khi $ H\equiv K$.\\
	Vậy $a+b=\dfrac{3}{2}$.
	}
\end{ex}
%Câu 44
\begin{ex}%[2H5V1-5]
	Trong không gian với hệ tọa độ $Oxyz$, cho điểm $A(2;5;3)$ và đường thẳng $d\colon \dfrac{x-1}{2}=\dfrac{y}{1}=\dfrac{z-2}{2}$. Biết rằng $(P)\colon ax+by+cz-3=0$ ($a$, $b$, $c\in \mathbb{Z}$) là mặt phẳng chứa $d$ và khoảng cách từ $A$ đến $(P)$ lớn nhất. Khi đó tổng $T=a+b+c$ bằng
	\choice
	{$3$}
	{$-3$}
	{\True $-2$}
	{$-5$}
	\loigiai{
	\begin{center}
	\begin{tikzpicture}[scale=1, font=\footnotesize, line join=round, line cap=round, >=stealth]
		\coordinate (O) at (0,0);
		\coordinate (B) at (0.6,2);
		\coordinate (C) at (6,2);
		\coordinate (D) at ($(O)+(C)-(B)$);
		\coordinate (H) at (2,1);\coordinate (K) at (4,1);
		\coordinate (A) at ($(H)+(0,2)$);
		\coordinate (T) at ($(C)!5/3!(K)$);\coordinate (U) at ($(K)!1/3!(C)$);
		\coordinate (E) at (intersection of B--C and A--H);
		\coordinate (F) at (intersection of B--C and A--K);
		\draw (O)--(B)--(E) (F)--(C)--(D)--(O) (A)--(H) (A)--(K) (T)--(U);
		\draw[dashed] (F)--(E);
		\draw pic[draw]{angle = D--O--B} node[above right=-0.1]{$P$};
		\pic[draw,angle radius=2mm] {right angle = A--H--K};
		\foreach \i/\g in {K/-90,A/90,H/-90}{\draw[fill=black](\i) circle (1.5pt) ($(\i)+(\g:3mm)$) node[scale=1]{$\i$};}
		\draw (V) node[above]{$d$};
		\draw[->] (K)--(U) node[right]{$\overrightarrow{u}$};
	\end{tikzpicture}
	\end{center}
	Đường thẳng $d$ đi qua $M(1;0;2)$, có một véc-tơ chỉ phương $\overrightarrow{u}=(2;1;2)$.\\
	Gọi $H$, $K$ lần lượt là hình chiếu của $A$ trên $(P)$ và trên $d$ thì $AH\le AK$ (cố định).\\
	Do đó, khoảng cách từ $A$ đến $(P)$ lớn nhất khi $H\equiv K$ hay $(P)\perp AK$.\\
	Gọi $K(2t+1;t;2t+2)\in d$ là hình chiếu của $A$ trên $d$, suy ra $\overrightarrow{AK}=(2t-1;t-5;2t-1)$.\\
	Ta có $\overrightarrow{AK}\cdot \overrightarrow{u}=0\Leftrightarrow 2(2t-1)+(t-5)+2(2t-1)=0\Leftrightarrow t=1$.\\
	 Khi đó $(P)$ đi qua $M(1;0;2)$, có một véc-tơ pháp tuyến $\overrightarrow{AK}=(1;-4;1)$ nên $$(P)\colon x-4y+z-3=0.$$
	 Vậy $T=a+b+c=1+(-4)+1=-2$.
	}
\end{ex}
%Câu 45
\begin{ex}%[2H5V1-5]
	Trong không gian hệ tọa độ $Oxyz$, cho đường thẳng $d\colon \dfrac{x+1}{-2}=\dfrac{y}{1}=\dfrac{z-1}{1}$ và điểm $A(1;2;3)$. Gọi $(P)$ là mặt phẳng chứa $d$ và cách điểm $A$ một khoảng cách lớn nhất. Véc-tơ nào dưới đây là một véc-tơ pháp tuyến của $(P)$?
	\choice
	{$\overrightarrow{n}=(1;0;2)$}
	{$\overrightarrow{n}=(1;0;-2)$}
	{\True $\overrightarrow{n}=(1;1;1)$}
	{$\overrightarrow{n}=(1;1;-1)$}
	\loigiai{
	\begin{center}
	\begin{tikzpicture}[scale=1, font=\footnotesize, line join=round, line cap=round, >=stealth]
		\coordinate (O) at (0,0);
		\coordinate (B) at (0.6,2);
		\coordinate (C) at (6,2);
		\coordinate (D) at ($(O)+(C)-(B)$);
		\coordinate (H) at (2,1);\coordinate (M) at (4,1);
		\coordinate (A) at ($(H)+(0,2)$);
		\coordinate (T) at ($(C)!5/3!(M)$);\coordinate (U) at ($(M)!1/3!(C)$);
		\coordinate (E) at (intersection of B--C and A--H);
		\coordinate (F) at (intersection of B--C and A--M);
		\draw (O)--(B)--(E) (F)--(C)--(D)--(O) (A)--(H) (A)--(M) (T)--(U) (H)--(M);
		\draw[dashed] (F)--(E);
		\draw pic[draw]{angle = D--O--B} node[above right=-0.1]{$P$};
		\pic[draw,angle radius=2mm] {right angle = A--H--K};
		\foreach \i/\g in {M/-90,A/90,H/-90}{\draw[fill=black](\i) circle (1.5pt) ($(\i)+(\g:3mm)$) node[scale=1]{$\i$};}
		\draw (U) node[right]{$d$};
	\end{tikzpicture}
\end{center}
	Gọi $H$ là hình chiếu của $A$ xuống mặt phẳng $(P)$.\\
	Từ $H$ kẻ $HM\perp d$. Dễ thấy $AM\perp d$.\\
	Ta có $AH\le AM$. Suy ra khoảng cách từ $A$ đến $(P)$ lớn nhất khi $M\equiv H$, hay $IM\perp (P)$.\\
	Phương trình tham số của $d\colon \heva{&x=-1-2t\\&y=t\\&z=1+t} (t\in \mathbb{R})$, véc-tơ chỉ phương là $\overrightarrow{u}=(-2;1;1)$.\\
	Vì $M\in d$ nên $M(-1-2t;t;1+t)\Rightarrow \overrightarrow{MA}=(2-2t;2-t;2-t)$.\\
	Ta có $\overrightarrow{MA}\perp \overrightarrow{u}\Leftrightarrow \overrightarrow{MA}\cdot \overrightarrow{u}=0\Leftrightarrow (-2)\cdot (2+2t)+1\cdot (2-t)+1\cdot (2-t)=0\Leftrightarrow t=0$.\\
	Suy ra $M(-1;0;1)\Rightarrow \overrightarrow{MA}=(2;2;2)$.\\
	Do $\overrightarrow{n}=(1;1;1)$ cùng hướng với $\overrightarrow{MA}$ nên $\overrightarrow{n}=(1;1;1)$ là một véc-tơ pháp tuyến của $(P)$. 
	}
\end{ex}
%Câu 46
\begin{ex}%[2H5V1-5]
Trong không gian với hệ trục tọa độ $Oxyz$, cho hai điểm $A(1;2;3)$, $B(5;-4;-1)$ và mặt phẳng $(P)$ qua $Ox$ sao cho $\mathrm{d}(B,(P))=2\mathrm{d}(A,(P))$ cắt $AB$ tại $I(a;b;c)$ nằm giữa $AB$. Tính $a+b+c$.
	\choice
	{$8$}
	{$6$}
	{$12$}
	{\True $4$}
	\loigiai{
Vì mặt phẳng $(P)$ qua $Ox$ nên phương trình mặt phẳng $(P)$ có dạng $by+cz=0$  trong đó $b^2+c^2>0$. Ta có 
\begin{eqnarray*}
&&\mathrm{d}(B,(P))=2\mathrm{d}(A,(P))\\
&\Leftrightarrow& \dfrac{|-4b-c|}{\sqrt{b^2+c^2}}=2\cdot \dfrac{|2b+3c|}{\sqrt{b^2+c^2}}\\
&\Leftrightarrow&\hoac{&-4b-c=4b+6c\\&-4b-c=-4b-6c}\\
&\Leftrightarrow&\hoac{&8b+7c=0\\&c=0.}
\end{eqnarray*}
\begin{itemize}
\item TH1. $8b+7c=0$. Chọn $b=7\Rightarrow c=-8\Rightarrow (P)\colon 7y-8z=0$.\\
Xét $f(x,y,z)=7y-8z$.\\
Thay tọa độ $A$, $B$ vào ta được $\left(7\cdot 2-8\cdot 3\right)\cdot \left[7\cdot (-4)-8\cdot (-1)\right]>0$.\\
Suy ra $A$, $B$ nằm cùng phía so với $(P)$ (loại).
\item TH2. $c=0$, chọn $b=1\Rightarrow (P)\colon y=0$.\\
Xét $f(x,y,z)=y$. Thay tọa độ $A$, $B$ vào ta được $2\cdot (-4)<0$.\\
Suy ra $A$, $B$ nằm khác phía so với $(P)$.\\
Do đó đường thẳng $AB$ cắt $(P)$ tại $I$ nằm giữa $AB$.\\
Phương trình tham số của đường thẳng $AB\colon \heva{&x=1+4t\\&y=2-6t\\&z=3-4t} (t\in \mathbb{R}).$\\
Tọa độ điểm $I$ là nghiệm hệ phương trình $\heva{&x=1+4t\\&y=2-6t\\&z=3-4t\\&y=0}\Leftrightarrow \heva{&t=\dfrac{1}{3}\\&x=\dfrac{7}{3}\\&y=0\\&z=\dfrac{5}{3}}\Rightarrow I\left(\dfrac{7}{3};0;\dfrac{5}{3}\right)$.\\
Vậy $a+b+c=\dfrac{7}{3}+0+\dfrac{5}{3}=4$.
\end{itemize}
	}
\end{ex}