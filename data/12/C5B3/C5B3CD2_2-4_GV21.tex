\chude{ỨNG DỤNG MẶT CẦU TRONG KHÔNG GIAN}
\begin{bt}%[2H5V3-4]
	Trong không gian với hệ trục tọa độ $Oxyz$ (đơn vị trên mỗi trục là kilômét) một trạm phát sóng rađa của Nga được đặt trên bán đảo Crimea ở vị trí $I(-2;1;-1)$ và được thiết kế phát hiện máy bay của địch ở khoảng cách tối đa $500$\,km.
	\begin{center}
		\includegraphics[scale=0.8]{images/Hinh1_C5B3CD2_2-4}
	\end{center}
	\begin{enumerate}
		\item Sử dụng phương trình mặt cầu để mô tả ranh giới bên ngoài vùng phát sóng của rađa trong không gian.
		\item Hai chiếc máy bay do thám của Mỹ và Anh đang bay ở vị trí có tọa độ lần lượt là $M(-200;100;-250)$ và $N(350;-100;300)$. Hỏi rađa của Nga có thể phát hiện ra hai chiếc máy bay do thám của Mỹ và Anh không?
	\end{enumerate}
	\loigiai{
		\begin{enumerate}
			\item Phương trình mặt cầu để mô tả ranh giới bên ngoài vùng phát sóng của rađa trong không gian là
			\[(x+2)^2+(y-1)^2+(z+1)^2=250\,000.\]
			\item \quad
				\begin{itemize}
					\item Ta có $IM=\sqrt{(-200+2)^2+(100-1)^2+(-250+1)^2} \approx 335{,}6<500$.\\
					Vì $IM<R$ nên điểm $M$ nằm trong mặt cầu.\\
					Vậy chiếc máy bay do thám của Mỹ có thể bị phát hiện bởi trạm rađa này.
					\item Ta có $IN=\sqrt{(350+2)^2+(-100-1)^2+(300+1)^2} \approx 474<500$.\\
					Vì $IN<R$ nên điểm $N$ nằm trong mặt cầu.\\
					Vậy chiếc máy bay do thám của Anh có thể bị phát hiện bởi trạm rađa này.
				\end{itemize}
		\end{enumerate}	
}
\end{bt}

\begin{bt}%[2H5V3-4]
	Trong không gian với hệ trục tọa độ $Oxyz$ (đơn vị trên mỗi trục là kilômét), đài kiểm soát không lưu sân bay Cam Ranh - Khánh Hòa ở vị trí $O(0;0;0)$ và được thiết kế phát hiện máy bay ở khoảng cách tối đa $600$\,km. Một máy bay của hãng Việt Nam Airlines đang ở vị trí $A(-1\,000;-200;10)$, chuyển động theo đường thẳng $d$ có phương trình $\heva{&x=-1000+100t\\&y=-200+80t\\&z=10} \quad(t \in \mathbb{R})$ và hướng về đài kiểm soát không lưu (như hình vẽ).
		\begin{center}
		\begin{tikzpicture}[line join=round,line cap=round,>=stealth,scale=0.7,font=\footnotesize]
			\tikzset{every node/.style={scale=0.8}}
			\def \R{2}
			\def \r{0.5}
			\coordinate (O) at (0,0);
			\coordinate (M1) at (0,\R);
			\coordinate (M2) at (0,-\R-0.8);
			\coordinate (B) at (-\R+0.7,\r+0.6);
			\coordinate (C) at (\R-0.4,\r+0.4);
			\coordinate (A) at ($(B)!-0.3!(C)$);
			\coordinate (d) at ($(B)!1.3!(C)$);
			\coordinate (D) at (\R,0);
			\coordinate (P1) at (-4,\R-0.5);
			\coordinate (p2) at (-5.5,\R-3.5);
			\coordinate (p3) at ($2*(O)-(P1)$);
			\coordinate (P4) at ($2*(O)-(p2)$);
			\coordinate (P2) at ($(p2)!-0.5!(P1)$);
			\coordinate (P3) at ($(p3)!-0.5!(P4)$);
			\draw [name path=i](D) arc (0:180: \R cm and \R cm);
			\draw [dashed](D) arc (0:-180: \R cm and \R cm);
			\draw  (D) arc (0:-180: \R cm and \r cm);
			\draw [dashed] (D) arc (0:180: \R cm and \r cm);
			\draw [->](M1)--(0,\R+0.8) node [right] {$z$};
			\draw[dashed] (D)--(-\R-1,0) (M1)--(M2) (-\R*0.5,-\R*0.2)--(\R*0.5,\R*0.2)--(\R*0.75,\R*0.3) (B)--(C);
			\draw[->] (\R,0)--(\R+1,0) node [below] {$y$};
			\draw [->] (-\R*0.5,-\R*0.2)--(-\R*0.85,-\R*0.375) node[below right] {$x$};
			\path [name path=p1p4] (P1)--(P4);
			\path [name intersections={of=i and p1p4,by={G1,G2}}];
			\draw (P1)--(P2)--(P3)--(P4) (P1)--(G2) (G1)--(P4) (B)--(A) (C)--(d) node[above] {$d$};
			\draw[dashed] (G1)--(G2);
			\foreach \i/\g in {O/-45,A/180,B/45,C/45} \draw[fill=black] (\i) circle(1.0 pt) node[shift={(\g:9pt)}]{$\i$};
		\end{tikzpicture}
	\end{center}
	\begin{enumerate}
		\item Sử dụng phương trình mặt cầu để mô tả ranh giới bên ngoài vùng phát sóng của đài kiểm soát không lưu trong không gian.
		\item Xác định tọa độ vị trí sớm nhất mà máy bay xuất hiện trên màn hình ra đa và tọa độ vị trí mà máy bay bay ra khỏi màn hình ra đa.
		\item Tính khoảng cách ngắn nhất giữa máy bay với đài kiểm soát không lưu.
	\end{enumerate}	
	\loigiai{
		\begin{enumerate}
			\item Ranh giới bên ngoài vùng phát sóng của đài kiểm soát không lưu trong không gian là mặt cầu tâm $O$ bán kính $R=600$.\\
			Vậy phương trình của mặt cầu là $x^2+y^2+z^2=360\,000$.
			\item Gọi $B$ là vị trí sớm nhất mà máy bay xuất hiện trên màn hình ra đa.\\
			Vì $B\in d$ nên $B(-1\,000+100t;-200+80t;10)$.\\
			$B$ là vị trí sớm nhất mà máy bay xuất hiện trên màn hình ra đa khi $OB=600$, tức là
			\allowdisplaybreaks
			\begin{eqnarray*}
				&&(-1\,000+100t)^2+(-200+80t)^2+10^2=600^2\\
				&\Leftrightarrow& 16\,400t^2-232\,000t+680\,100=0\\
				&\Leftrightarrow& \hoac{&t\approx 4{,}15\\&t\approx 10.}
			\end{eqnarray*}			
			\begin{itemize}
				\item Với $t=4{,}15$, ta có $B(-585;132;10)$.\\
				Khi đó $AB\approx 531{,}46$.
				\item Với $t=10$, ta có $B(0;600;10)$.\\
				Khi đó $AB\approx 1\,077{,}03$.				
			\end{itemize}			
			Vì $531{,}46<1\,077{,}03$ nên tọa độ vị trí sớm nhất mà máy bay xuất hiện trên màn hình ra đa là $(-585;132;10)$.\\
			Suy ra, tọa độ vị trí mà máy bay bay ra khỏi màn hình ra đa là $(0;600;10)$.
			\item Ta có vectơ chỉ phương của đường thẳng $d$ là $\overrightarrow{u}=(100;80;0)$.\\
			Gọi $H$ là vị trí mà máy bay bay gần đài kiểm soát không lưu nhất.\\
			Khi đó, khoảng $OH$ phải ngắn nhất, điều này xảy ra khi và chỉ khi $OH\perp d$.\\
			Vì $H\in d$ nên $H(-1\,000+100t;-200+80t;10)$.\\
			Ta có $\overrightarrow{OH}=(-1\,000+100t;-200+80t;10)$.\\
			Khi đó
			\allowdisplaybreaks
			\begin{eqnarray*}
				OH\perp d &\Leftrightarrow& \overrightarrow{OH} \cdot \overrightarrow{u}=0\\
				&\Leftrightarrow&(-1\,000+100t)\cdot 100+(-200+80t)\cdot 80+10\cdot 0=0\\
				&\Leftrightarrow&16\,400t-116\,000=0\\
				&\Leftrightarrow&t=\dfrac{116\,000}{16\,400}\approx 7{,}07.
			\end{eqnarray*}
			Suy ra $OH=\sqrt{(-293)^2+(365{,}6)^2+10^2}\approx 468{,}63$.\\
			Vậy khoảng cách ngắn nhất giữa máy bay với đài kiểm soát không lưu là $468{,}63$\,km.
		\end{enumerate}
	}
\end{bt}

\begin{bt}%[2H5V3-3]
	Trong KG $Oxyz$, cho hình chóp $S.ABCD$, đáy $ABCD$ là hình chữ nhật. Biết $A(0;0;0)$, $D(2;0;0)$, $B(0;4;0)$, $S(0; 0;4)$. Viết phương trình mặt cầu ngoại tiếp hình chóp $S.ABCD$.
	\loigiai{
		\begin{center}
			\begin{tikzpicture}[line join=round,line cap=round,>=stealth,scale=1.0,font=\footnotesize]
			\tikzset{every node/.style={scale=0.9}}
			\coordinate (A) at (0,0);
			\coordinate (D) at (4,0);
			\coordinate (B) at (-150:2);
			\coordinate (C) at ($(B)+(D)-(A)$);
			\coordinate (S) at ($(A)+(0,4)$);
			\coordinate (I) at ($(S)!0.5!(C)$);
			\draw (B)--(C)--(D) (S)--(B) (S)--(C) (S)--(D) (I)--(B) (I)--(D);
			\draw[dashed] (B)--(A)--(D) (A)--(S) (I)--(A);
			\foreach \i/\g in {A/180,B/180,C/0,D/0,S/90,I/30} \draw[fill=black] (\i) circle(1.0 pt) node[shift={(\g:9pt)}]{$\i$};
		\end{tikzpicture}
		\end{center}
		Gọi $I$ là trung điểm $SC$.\\
		Ta có $\triangle SBC$, $\triangle SAC$, $\triangle SAD$ là các tam giác vuông có chung cạnh huyền $SC$ nên suy ra $IA=IB=IC=ID=IS$ hay $I$ là tâm mặt cầu ngoại tiếp hình chóp $S.ABCD$.\\
		Gọi $C(x;y;z)$.\\
		Ta có $\overrightarrow{AD}=(2;0;0)$ và $\overrightarrow{BC}=(x;y-4;z)$.\\
		Do $ABCD$ là hình chữ nhật nên $\overrightarrow{AD}=\overrightarrow{BC}$ hay $\heva{&2=x\\&0=y-4\\&0=z}\Leftrightarrow \heva{&x=2\\&y=4\\&z=0.}$\\
		Suy ra $C(2;4;0)$.\\
		Khi đó mặt cầu ngoại tiếp $S.ABCD$ có $\heva{&\text{tâm } I(1;2;2)\\&\text{Bán kính }R=\dfrac{SC}{2}=\dfrac{\sqrt{2^2+4^2+(-4)^2}}{2}=3.}$\\
		Vậy phương trình của mặt cầu là $(x-1)^2+(y-2)^2+(z-2)^2=9$.
	}
\end{bt}

\begin{bt}%[2H5V3-3]
	\immini{Cho tứ diện $SABC$, có $SA$, $SB$, $SC$ đôi một vuông góc và $SA=5$, $SB=2$, $SC=4$. Chọn hệ tọa độ $Oxyz$ như hình vẽ, viết phương trình mặt cầu ngoại tiếp hình chóp $SABC$.
	}{
		\begin{tikzpicture}[line join=round,line cap=round,>=stealth,scale=0.7,font=\footnotesize]
		\tikzset{every node/.style={scale=0.8}}
		\coordinate (O) at (0,0);
		\coordinate (S) at (0,0);
		\coordinate (C) at ($(O)+(3,0)$);
		\coordinate (A) at (90:4);
		\coordinate (B) at (-135:2);
		\coordinate (x) at ($(C)!-0.4!(O)$);
		\coordinate (y) at ($(B)!-0.4!(O)$);
		\coordinate (z) at ($(A)!-0.4!(O)$);
		\draw[dashed] (S)--(B) (A)--(S) (S)--(C);
		\draw (A)--(B)--(C)--cycle;
		\draw[->] (C)--(x) node[above] {$x$};
		\draw[->] (B)--(y) node [right] {$y$};
		\draw[->] (A)--(z) node [right] {$z$};
		\foreach \i/\g in {O/180,A/45,B/180,C/-90,S/45} \draw[fill=black] (\i) circle(1.0 pt) node[shift={(\g:9pt)}]{$\i$};
	\end{tikzpicture}
	}
	\loigiai{
		Theo đề bài ta có $A(0,0,5)$; $B(0;2;0)$; $C(4;0;0)$ và $S(0;0;0)$.\\
		Gọi phương trình mặt cầu ngoại tiếp hình chóp là $x^2+y^2+z^2-2ax-2by-2cz+d=0$.\\
		Do mặt cầu đi qua các điểm $A$, $B$, $C$, $D$, $S$ nên ta có hệ phương trình
		\allowdisplaybreaks
		\begin{eqnarray*}
			&&\heva{&0^2+0^2+5^2-2\cdot a \cdot 0-2\cdot b\cdot 0-2\cdot c\cdot 5+d=0\\&0^2+2^2+0^2-2\cdot a\cdot 0-2\cdot b\cdot 2-2\cdot c\cdot 0+d=0\\&4^2+0^2+0^2-2\cdot a\cdot 4-2\cdot b\cdot 0-2\cdot c\cdot 0+d=0\\&0^2+0^2+0^2-2\cdot a\cdot 0-2\cdot b\cdot 0-2\cdot c\cdot 0+d=0}\\
			&\Leftrightarrow&\heva{&-10c+d=-25\\&-4b+d=-4\\&-8a+d=-16\\&d=0}\\
			&\Leftrightarrow&\heva{&c=\dfrac{5}{2}\\&b=1\\&a=2\\&d=0.}
		\end{eqnarray*}
		Vậy phương trình mặt cầu ngoại tiếp hình chóp là
		\[x^2+y^2+z^2-4x-2y-5z=0.\]
	}
\end{bt}
\vspace{-1.0cm}
\begin{bt}%[2H5V3-3]
	\immini{Cho hình lập phương $ABCD.A'B'C'D'$ có độ dài cạnh bằng $1$. Chọn hệ tọa độ $Oxyz$ như hình vẽ, viết phương trình mặt cầu ngoại tiếp hình lập phương $ABCD.A'B'C'D'$.
	}{
	\begin{tikzpicture}[line join=round,line cap=round,>=stealth,scale=0.7,font=\footnotesize]
		\tikzset{every node/.style={scale=0.8}}
		\def \dai{3}
		\def \rong{2}
		\def \cao{3}
		\coordinate (A) at (0,0);
		\coordinate (B) at (-150:\rong);
		\coordinate (D) at (0:\dai);
		\coordinate (C) at ($(B)+(D)-(A)$);
		\coordinate (A') at (0,\cao);
		\coordinate (B') at ($(A')+(-150:\rong)$);
		\coordinate (D') at ($(A')+(0:\dai)$);
		\coordinate (C') at ($(B')+(D')-(A')$);
		\coordinate (x) at ($(D)!-0.4!(A)$);
		\coordinate (y) at ($(B)!-0.4!(A)$);
		\coordinate (z) at ($(A')!-0.4!(A)$);
		\draw (B)--(C)--(D) (A')--(B')--(C')--(D')--cycle (B)--(B') (C)--(C') (D)--(D');
		\draw[dashed] (A')--(A)--(B) (A)--(D);
		\draw[->] (D)--(x) node[above] {$x$};
		\draw[->] (B)--(y) node [below] {$y$};
		\draw[->] (A')--(z) node [right] {$z$};
		\foreach \i/\g in {A/180,B/180,C/-45,D/-90,A'/180,B'/180,C'/-45,D'/0} \draw[fill=black] (\i) circle(1.0 pt) node[shift={(\g:9pt)}]{$\i$};
	\end{tikzpicture}
	}
	\loigiai{
		Gọi $I$ là tâm hình lập phương.\\
		Khi đó ta có $IA=IB=IC=ID=IA'=IB'=IC'=ID'$ hay $I$ là tâm của mặt cầu ngoại tiếp hình lập phương.\\
		Do hình lập phương có cạnh bằng $1$ nên suy ra $A'(0;0;1)$; $C(1;1;0)$ và $A'C=\sqrt 3$.\\
		Suy ra $\heva{&\text{Tâm } I\left(\dfrac{1}{2};\dfrac{1}{2};\dfrac{1}{2}\right)\\&\text{Bán kính }R=\dfrac{A'C}{2}=\dfrac{\sqrt 3}{2}.}$\\
		Vậy phương trình mặt cầu ngoại tiếp hình lập phương là
		\[\left(x-\dfrac{1}{2}\right)^2+\left(y-\dfrac{1}{2}\right)^2+\left(z-\dfrac{1}{2}\right)^2=\dfrac{3}{4}.\]
	}
\end{bt}

\begin{bt}%[2H5V3-3]
	\immini{Cho hình lập phương $ABCD.A'B'C'D'$ có độ dài cạnh bằng $2a$. Chọn hệ tọa độ $Oxyz$ như hình vẽ.
	\begin{enumerate}
		\item Viết phương trình mặt cầu ngoại tiếp hình lập phương $ABCD.A'B'C'D'$.
		\item Tính khoảng cách từ $C'$ đến mặt phẳng $\left(CB'D'\right)$.
		\item Tính góc tạo bởi hai đường thẳng $AC'$ và $B'D$.
	\end{enumerate}
	}{
	\begin{tikzpicture}[line join=round,line cap=round,>=stealth,scale=0.7,font=\footnotesize]
		\tikzset{every node/.style={scale=0.8}}
		\def \dai{3}
		\def \rong{2}
		\def \cao{3}
		\coordinate (A) at (0,0);
		\coordinate (B) at (-150:\rong);
		\coordinate (D) at (0:\dai);
		\coordinate (C) at ($(B)+(D)-(A)$);
		\coordinate (A') at (0,\cao);
		\coordinate (B') at ($(A')+(-150:\rong)$);
		\coordinate (D') at ($(A')+(0:\dai)$);
		\coordinate (C') at ($(B')+(D')-(A')$);
		\coordinate (x) at ($(D)!-0.4!(A)$);
		\coordinate (y) at ($(B)!-0.4!(A)$);
		\coordinate (z) at ($(A')!-0.4!(A)$);
		\draw (B)--(C)--(D) (A')--(B')--(C')--(D')--cycle (B)--(B') (C)--(C') (D)--(D');
		\draw[dashed] (A')--(A)--(B) (A)--(D);
		\draw[->] (D)--(x) node[above] {$x$};
		\draw[->] (B)--(y) node [below] {$y$};
		\draw[->] (A')--(z) node [right] {$z$};
		\foreach \i/\g in {A/180,B/180,C/-45,D/-90,A'/180,B'/180,C'/-45,D'/0} \draw[fill=black] (\i) circle(1.0 pt) node[shift={(\g:9pt)}]{$\i$};
	\end{tikzpicture}
	}
	\loigiai{
		\begin{center}
				\begin{tikzpicture}[line join=round,line cap=round,>=stealth,scale=1.0,font=\footnotesize]
				\tikzset{every node/.style={scale=0.8}}
				\def \dai{3}
				\def \rong{2}
				\def \cao{3}
				\coordinate (A) at (0,0);
				\coordinate (B) at (-150:\rong);
				\coordinate (D) at (0:\dai);
				\coordinate (C) at ($(B)+(D)-(A)$);
				\coordinate (A') at (0,\cao);
				\coordinate (B') at ($(A')+(-150:\rong)$);
				\coordinate (D') at ($(A')+(0:\dai)$);
				\coordinate (C') at ($(B')+(D')-(A')$);
				\coordinate (x) at ($(D)!-0.4!(A)$);
				\coordinate (y) at ($(B)!-0.4!(A)$);
				\coordinate (z) at ($(A')!-0.4!(A)$);
				\coordinate (I) at ($(A')!0.5!(C)$);
				\coordinate (M) at ($(D')!0.5!(C)$);
				\coordinate (H) at ($(B')!(C')!(M)$);
				\draw (B)--(C)--(D) (A')--(B')--(C')--(D')--cycle (B)--(B') (C)--(C') (D)--(D') (B')--(C)--(D')--cycle (C')--(M);
				\draw[dashed] (A')--(A)--(B) (A)--(D) (A)--(C') (A')--(C) (B)--(D') (B')--(D) (B')--(M) (C')--(H);
				\draw[->] (D)--(x) node[above] {$x$};
				\draw[->] (B)--(y) node [below] {$y$};
				\draw[->] (A')--(z) node [right] {$z$};
				\pic [draw, angle radius=8] {right angle=C'--M--D'};
				\pic [draw, angle radius=8] {right angle=C'--H--M};
				\foreach \i/\g in {A/180,B/180,C/-45,D/-90,A'/180,B'/180,C'/90,D'/0,I/-90,M/0,H/-90} \draw[fill=black] (\i) circle(1.0 pt) node[shift={(\g:9pt)}]{$\i$};
			\end{tikzpicture}
		\end{center}
		\begin{enumerate}
			\item Gọi $I$ là tâm hình lập phương.\\
			Khi đó ta có $IA=IB=IC=ID=IA'=IB'=IC'=ID'$ hay $I$ là tâm của mặt cầu ngoại tiếp hình lập phương.\\
			Do hình lập phương có cạnh bằng $2a$ nên suy ra $A'(0;0;2a)$; $C(2a;2a;0)$ và \break $A'C=2a\sqrt 3$.\\
			Suy ra $\heva{&\text{Tâm } I(a;a;a)\\&\text{Bán kính }R=\dfrac{A'C}{2}=a\sqrt 3.}$\\
			Vậy phương trình mặt cầu ngoại tiếp hình lập phương là
			\[(x-a)^2+(y-a)^2+(z-a)^2=3a^2.\]
			\item Gọi $M$ là tâm hình vuông $CDD'C'$ và kẻ $CH\perp BM$ tại $H$ ta suy ra khoảng cách từ $C'$ đến $(CB'D')$ là $CH$.\\
			Ta có $C'D=2a\sqrt 2 \Rightarrow CM=a\sqrt 2$.\\
			Trong tam giác vuông $B'C'M$ ta có
			\allowdisplaybreaks
			\begin{eqnarray*}
				\dfrac{1}{CH^2}&=&\dfrac{1}{B'C'^2}+\dfrac{1}{C'M^2}\\
				&=&\dfrac{1}{4a^2}+\dfrac{1}{2a^2}\\
				&=&\dfrac{3}{4a^2}
			\end{eqnarray*}
			Suy ra $CH^2=\dfrac{4a^2}{3}\Rightarrow CH=\dfrac{2a\sqrt 3}{3}$.
			\item Ta có $A'(0;0;2a)$ và $C(2a;2a;0)$ suy ra $\overrightarrow{A'C}=(2a;2a;-2a)$ và $\left|\overrightarrow{A'C}\right|=2a\sqrt 3$.\\
			Ta lại có $B'(0;2a;2a)$ và $D(2a;0;0)$ suy ra $\overrightarrow{B'D}=(2a;-2a;-2a)$ và $\left|\overrightarrow{B'D}\right|=2a\sqrt 3$.\\
			Khi đó ta có,
			\allowdisplaybreaks
			\begin{eqnarray*}
				\cos\Big(\overrightarrow{A'C},\overrightarrow{B'D}\Big)&=&\dfrac{\overrightarrow{A'C}\cdot \overrightarrow{B'D}}{\left|\overrightarrow{A'C}\right|\cdot \left|\overrightarrow{B'D}\right|}\\
				&=&\dfrac{2a\cdot 2a-2a\cdot 2a+2a\cdot 2a}{2a\sqrt 3\cdot 2a\sqrt 3}\\
				&=&\dfrac{\sqrt 3}{3}.
			\end{eqnarray*}
			Suy ra $\Big(\overrightarrow{A'C},\overrightarrow{B'D}\Big)=54^\circ44'8{,}20''$.\\
			Suy ra góc giữa $A'C$ và $B'D$ bằng $54^\circ44'8{,}20''$.
		\end{enumerate}
	}
\end{bt}
