\begin{ex}%[2D4H1-2][Lê Công Trường]
	Cho	các mệnh đề sau đây 
	\choiceTF
	{\True $F(x)=\dfrac{x^4}{4}-\dfrac{3}{2}{x^2}+\ln \left| x\right|+C$ là nguyên hàm của hàm số $f(x)=x^3-3x+\dfrac{1}{x}$}
	{$F(x)=\dfrac{(5x+3)^6}{6}+C$ là nguyên hàm của hàm số $f(x)=\left(5x+3\right)^5$}
	{$F(x)=\dfrac{3}{2}x\sqrt x+\dfrac{4}{3}x\sqrt[3]{x}+\dfrac{5}{4}x\sqrt[4]{x}+C$ là nguyên hàm của hàm số $f(x)=\sqrt x+\sqrt[3]{x}+\sqrt[4]{x}$}
	{\True $F(x)=\dfrac{1}{3}{x^3}-2024x+C$ là nguyên hàm của hàm số $f(x)=\dfrac{x^3-2024x}{x}$}
	\loigiai{
		\begin{itemchoice}
			\itemch {\bf Đúng}. Vì $f(x)=x^3-3x+\dfrac{1}{x}$\\
			$\Rightarrow F(x)=\displaystyle\int f(x)dx=\displaystyle\int{(x^3-3x{\rm}+\dfrac{1}{x})dx}$\\
			$=\displaystyle\int{x^3dx}-3\displaystyle\int{xdx}+\displaystyle\int{\dfrac{1}{x}dx}=\dfrac{x^4}{4}-\dfrac{3}{2}{x^2}+\ln \left| x\right|+C$.
			\itemch {\bf Sai.} Vì $f(x)=\left(5x+3\right)^5$ \\
			$\Rightarrow F(x)=\displaystyle\int{f(x)dx=}\displaystyle\int(5x+3)^{5}dx$\\
			$=\displaystyle\int{\rm{(5x+3)}^{\rm{5}}\dfrac{d(5x+3)}{5}=\dfrac{(5x+3)^6}{30}+C}$.
			\itemch {\bf Sai.} Vì $f(x)=\sqrt x+\sqrt[3]{x}+\sqrt[4]{x}$\\
			$\Rightarrow F(x)=\displaystyle\int{\left(\sqrt x+\sqrt[3]{x}+\sqrt[4]{x}\right)}dx=\displaystyle\int{\left(x^{\frac{1}{2}}+x^{\frac{1}{3}}+x^{\frac{1}{4}}\right)}dx$\\
			$=\dfrac{2}{3}{x^{\frac{3}{2}}}+\dfrac{3}{4}{x^{\frac{4}{3}}}+\dfrac{4}{5}{x^{\frac{5}{4}}}+C=\dfrac{2}{3}x\sqrt x+\dfrac{3}{4}x\sqrt[3]{x}+\dfrac{4}{5}x\sqrt[4]{x}+C$.
			\itemch {\bf Đúng.} $f(x)=\dfrac{x^3-2024x}{x}\Rightarrow F(x)=\displaystyle\int{\dfrac{x^3-2024x}{x}dx}=\displaystyle\int\left(x^2-2024\right)dx$\\
			$=\dfrac{1}{3}{x^3}-2024x+C$.
		\end{itemchoice}
	}
\end{ex}
\Closesolutionfile{ans}
\indapan{3}{ans/ans-2-B1-D2-DS}
	\Opensolutionfile{ans}[ans/ans-2-B1-D1-KQ]
	\TNSA
	\begin{ex}%[2D4H1-2][Lê Công Trường]
		Hệ số của $x^2$ trong nguyên hàm $F(x)$ của hàm số $f(x)=\dfrac{2}{\sqrt{x}}+3^x+3x-2$ là
		\shortans{$1{,}5$}
		\loigiai{
			$F(x)=\displaystyle\int{\left(\dfrac{2}{\sqrt{x}}+3^x+3x-2\right)\mathrm{\,d}x}=4\sqrt{x}+\dfrac{3^x}{\ln 3}+\dfrac{3}{2}{x^2}-2x+C$.
		}
	\end{ex}
	
	\begin{ex}%[2D4H1-2][Lê Công Trường]
		Hệ số của $x^3$ trong nguyên hàm $F(x)$ của hàm số $f(x)=m{x^3}-3x^2+\dfrac{4m}{x^3}+\dfrac{5}{2x}-7m$ ($m$ là tham số) là
		\shortans{$-1$}
		\loigiai{
				$F(x)=\displaystyle\int{\left(m{x^3}-3x^2+\dfrac{4m}{x^3}+\dfrac{5}{2x}-7m\right)\mathrm{\,d}x}=\dfrac{m}{4}{x^4}-x^3-\dfrac{2m}{x^2}-\dfrac{5}{2}\ln {|x|}-7mx+C$
		}
	\end{ex}
	
	\begin{ex}% [2D4H1-2][Lê Công Trường]
		Tìm nguyên hàm $F(x)$ của hàm số $f(x)=\dfrac{1}{\sqrt{x}}-\dfrac{2}{\sqrt[3]{x}}$. Tổng hệ số của biến $x$ là
		\shortans{$-1$}
		\loigiai{
			$F(x)=\displaystyle\int f(x)\mathrm{\,d}x=\displaystyle\int\left(\dfrac{1}{\sqrt{x}}-\dfrac{2}{\sqrt[3]{x}}\right)\mathrm{\,d}x=\displaystyle\int\dfrac{1}{\sqrt{x}}\mathrm{\,d}x-\displaystyle\int\dfrac{2}{\sqrt[3]{x}}=\displaystyle\int{x^{\frac{-1}{2}}}\mathrm{\,d}x-\displaystyle\int{2x^{\frac{-1}{3}}}\mathrm{\,d}x$\\
			$=\dfrac{x^{\frac{1}{2}}}{\dfrac{1}{2}}-2.\dfrac{x^{\frac{2}{3}}}{\dfrac{2}{3}}+C=2{x^{\frac{1}{2}}}-3x^{\frac{2}{3}}+C=2\sqrt{x}-3\sqrt[3]{x^2}+C$.
		}
	\end{ex}
	
	\begin{ex}%%[2D4H1-2][Lê Công Trường]
		Tìm nguyên hàm $F(x)$ của hàm số $f(x)=\dfrac{(x^2-1)^2}{x^2}$. Tổng hệ số của bậc $3$ và bậc $1$ là (làm tròn đến hàng phần chục).
		\shortans{$-1{,}6$}
		\loigiai{
			$\displaystyle\int  f(x)\mathrm{\,d}x=\displaystyle\int\dfrac{(x^2-1)^2}{x^2}\mathrm{\,d}x=\displaystyle\int\dfrac{x^4-2x^2+1}{x^2}\mathrm{\,d}x=\displaystyle\int\left(x^2-2+\dfrac{1}{x^2}\right)\mathrm{\,d}x$\\
			$=\dfrac{x^3}{3}-2x-\dfrac{1}{x}+C$.
		}
	\end{ex}
	
	\begin{ex}%%[2D4H1-2][Lê Công Trường]
		Tính $\displaystyle\int{\left(\dfrac{\left(1-x\right)^3}{\sqrt[3]{x}}\right)\mathrm{\,d}x}$. Giá trị tổng hệ số chứa biến là (làm tròn đến hàng phần trăm).
		\shortans{$0{,}55$}
		\loigiai{$\displaystyle\int\left(\dfrac{\left(1-x\right)^3}{\sqrt[3]{x}}\right)\mathrm{\,d}x=\displaystyle\int\dfrac{1-3x+3x^2-x^3}{x^{\frac{1}{3}}}\mathrm{\,d}x=\displaystyle\int\left(x^{\frac{-1}{3}}-3x^{\frac{2}{3}}+3x^{\frac{5}{3}}-x^{\frac{8}{3}}\right)\mathrm{\,d}x$\\
			$=\dfrac{x^{\frac{2}{3}}}{\dfrac{2}{3}}-3\dfrac{x^{\frac{5}{3}}}{\dfrac{5}{3}}+3\dfrac{x^{\frac{8}{3}}}{\dfrac{8}{3}}-\dfrac{x^{\frac{11}{3}}}{\dfrac{11}{3}}+C=\dfrac{3}{2}{x^{\frac{2}{3}}}-\dfrac{9}{5}{x^{\frac{5}{3}}}+\dfrac{9}{8}{x^{\frac{8}{3}}}-\dfrac{3}{11}{x^{\frac{11}{3}}}+C$.
			
		}
	\end{ex}
	
	\begin{ex}%[2D4H1-2][Lê Công Trường]
		Tính $\displaystyle\int{\left(\sqrt[3]{x^2}-\sqrt[4]{x^3}+\sqrt[5]{x^4}\right)\mathrm{\,d}x}$. Giá trị tổng hệ số chứa biến là (làm tròn đến hàng phần trăm).
		\shortans{$0{,}58$}
		\loigiai{
			$\displaystyle\int\left(\sqrt[3]{x^2}-\sqrt[4]{x^3}+\sqrt[5]{x^4}\right)\mathrm{\,d}x=\displaystyle\int\left(x^{\frac{2}{3}}-x^{\frac{3}{4}}+x^{\frac{4}{5}}\right)\mathrm{\,d}x=\dfrac{x^{\frac{5}{3}}}{\dfrac{5}{3}}-\dfrac{x^{\frac{7}{4}}}{\dfrac{7}{4}}+\dfrac{x^{\frac{9}{5}}}{\dfrac{9}{5}}+C$\\
			$=\dfrac{3}{5}{x^{\frac{5}{3}}}-\dfrac{4}{7}{x^{\frac{7}{4}}}+\dfrac{5}{9}{x^{\frac{9}{5}}}+C$.
		}
	\end{ex}
	
	\begin{ex}%%[2D4H1-2][Lê Công Trường]
		Tính $\displaystyle\int\left(\sqrt{x}+1\right)\left(x-\sqrt{x}+1\right)\mathrm{\,d}x$. Giá trị tổng hệ số chứa biến là (làm tròn đến hàng phần chục).
		\shortans{$1{,}4 $}
		\loigiai{
			$\left(\sqrt{x}+1\right)\left(x-\sqrt{x}+1\right)=\left(\sqrt{x}+1\right)\left[x-\left(\sqrt{x}-1\right)\right]=x\left(\sqrt{x}+1\right)-\left(\sqrt{x}+1\right)\left(\sqrt{x}-1\right)$\\
			$=x\sqrt{x}+x-\left(x-1\right)=x\sqrt{x}+1$.\\
			Do đó $\displaystyle\int\left(\sqrt{x}+1\right)\left(x-\sqrt{x}+1\right)\mathrm{\,d}x=\displaystyle\int\left(x\sqrt{x}+1\right)\mathrm{\,d}x=\displaystyle\int\left(x^{\frac{3}{2}}+1\right)\mathrm{\,d}x$\\
			$=\dfrac{2}{5}x^{\frac{5}{2}}+x+C$.
			}
	\end{ex}
	
	\begin{ex}%%[2D4H1-2][Lê Công Trường]
		Tính $\displaystyle\int{\left(2\sqrt{x}-\dfrac{3}{\sqrt[3]{x}}\right)\mathrm{\,d}x}$. Giá trị tổng hệ số chứa biến là (làm tròn đến hàng phần chục).
		\shortans{$-3{,}1$}
		\loigiai{$\displaystyle\int{\left(2\sqrt {x}-\dfrac{3}{\sqrt[3]{x}}\right)\mathrm{\,d}x=\displaystyle\int{\left(2x^{\frac{1}{2}}-3x^{\frac{-1}{3}}\right)}}\mathrm{\,d}x=\dfrac{4}{3}{x^{\frac{3}{2}}}-\dfrac{9}{2}{x^{\frac{2}{3}}}+C=\dfrac{4}{3}\sqrt[2]{x^3}-\dfrac{9}{2}\sqrt[3]{x^2}+C$.\\
		}
	\end{ex}	
	
	\begin{ex}%[2D4H1-2][Lê Công Trường]
		Tính $\displaystyle\int{\dfrac{1}{\sqrt{2x}+\sqrt{3x}}\mathrm{\,d}x}=a\left(\sqrt {b}-\sqrt {c}\right)\sqrt {x}$. Giá trị của tổng $a+b+c$ là
		\shortans{$7$}
		\loigiai{
			Ta có: $\dfrac{1}{\sqrt{2x}+\sqrt{3x}}=\dfrac{\sqrt{3x}-\sqrt{2x}}{\left(\sqrt{3x}-\sqrt{2x}\right)\left(\sqrt{3x}+\sqrt{2x}\right)}=\dfrac{\sqrt{3x}-\sqrt{2x}}{x}=\dfrac{\sqrt {x}}{x}\left(\sqrt 3-\sqrt 2\right)$\\
			$=\left(\sqrt 3-\sqrt 2\right){x^{\frac{-1}{2}}}.$\\
			$\displaystyle\int{\dfrac{1}{\sqrt{2x}+\sqrt{3x}}\mathrm{\,d}x}=\displaystyle\int{\left(\sqrt 3-\sqrt 2\right){x^{\frac{-1}{2}}}\mathrm{\,d}x=}\left(\sqrt 3-\sqrt 2\right)\dfrac{x^{\frac{1}{2}}}{\dfrac{1}{2}}=2\left(\sqrt 3-\sqrt 2\right)\sqrt {x}$.}
	\end{ex}
	
	\begin{ex}%%[2D4H1-2][Lê Công Trường]
		Tính $\displaystyle\int{\dfrac{1}{\sqrt{5x}-\sqrt{3x}}\mathrm{\,d}x=\left(\sqrt{a}+\sqrt{b}\right)\sqrt {x}+C}$. Giá trị $a+b$ bằng\\
		\shortans{$8$}
		\loigiai{
			$\dfrac{1}{\sqrt{5x}-\sqrt{3x}}=\dfrac{\sqrt{5x}+\sqrt{3x}}{\left(\sqrt{5x}-\sqrt{3x}\right)\left(\sqrt{5x}+\sqrt{3x}\right)}=\dfrac{\sqrt{5x}+\sqrt{3x}}{2x}=\dfrac{\sqrt {x}}{2x}\left(\sqrt 5+\sqrt 3\right).$\\
			$\displaystyle\int{\dfrac{1}{\sqrt{5x}-\sqrt{3x}}\mathrm{\,d}x}=\displaystyle\int{\dfrac{\sqrt {x}}{2x}\left(\sqrt 5+\sqrt 3\right)\mathrm{\,d}x}=\dfrac{\left(\sqrt 5+\sqrt 3\right)}{2}\displaystyle\int{x^{\frac{-1}{2}}}\mathrm{\,d}x=\dfrac{\left(\sqrt 5+\sqrt 3\right)}{2}\cdot\dfrac{x^{\frac{1}{2}}}{\dfrac{1}{2}}$\\
			$=\left(\sqrt 5+\sqrt 3\right)\sqrt {x}+C$.}
	\end{ex}
	
	\begin{ex}%%[2D4H1-2][Lê Công Trường]
		Tính $\displaystyle\int{\left(x^2-1\right)^3\mathrm{\,d}x}$. Giá trị tổng hệ số chứa biến là (làm tròn đến hàng phần chục).
		\shortans{$-0{,}5$}
		\loigiai{
			$\displaystyle\int\left(x^2-1\right)^3\mathrm{\,d}x=\displaystyle\int\left(x^6-3x^4+3x^2-1\right)\mathrm{\,d}x=\dfrac{x^7}{7}-3\dfrac{x^5}{5}+x^3-x+C$.}
	\end{ex}
	
	\begin{ex}%%[2D4H1-2][Lê Công Trường]
		Tính $\displaystyle\int{\left(2-x^2\right)^4\mathrm{\,d}x}$. Giá trị tổng hệ số chứa biến là (làm tròn đến hàng phần chục).
		\shortans{$9{,}1 $}
		\loigiai{
			Sử dụng khai triển theo nhị thức Newton, ta có:\\
			$\left(2-x^2\right)^4=x^8-8x^6+24x^4-32x^2+16$.\\
			Do đó\\
			$\displaystyle\int{\left(2-x^2\right)^4\mathrm{\,d}x}=\displaystyle\int{\left(x^8-8x^6+24x^4-32x^2+16\right)}\mathrm{\,d}x$\\
			$=\dfrac{x^8}{9}-\dfrac{8}{7}{x^7}+\dfrac{24}{5}{x^5}-\dfrac{32}{3}{x^3}+16x+C$.}
	\end{ex}
	
	\begin{ex}%%[2D4H1-2][Lê Công Trường]
		Tính $\displaystyle\int{\left(x-\sqrt[3]{x}\right)^2\mathrm{\,d}x}$. Giá trị tổng hệ số chứa biến là (làm tròn đến hàng phần chục).
		\shortans{$-1{,}1 $}
		\loigiai{
			$\left(x-\sqrt[3]{x}\right)^2=x^2-2x\sqrt[3]{x}-\sqrt[3]{x^2}=x^2-2x^{\frac{4}{3}}-x^{\frac{2}{3}}$.\\
			$\displaystyle\int{\left(x-\sqrt[3]{x}\right)^2\mathrm{\,d}x}=\displaystyle\int{\left(x^2-2x^{\frac{4}{3}}-x^{\frac{2}{3}}\right)\mathrm{\,d}x=}\dfrac{x^3}{3}-\dfrac{6}{7}{x^{\frac{7}{3}}}-\dfrac{3}{5}{x^{\frac{5}{3}}}+C.$}
	\end{ex}
	
	\begin{ex}%%[2D4H1-2][Lê Công Trường]
		Tính $\displaystyle\int{\left(\dfrac{x^2+2\sqrt[3]{x}}{x}\right)^2\mathrm{\,d}x}$.  Giá trị tổng hệ số chứa biến là (làm tròn đến hàng phần chục).
		\shortans{$-8{,}7$}
		\loigiai{
			Ta có: $\left(\dfrac{x^2+2\sqrt[3]{x}}{x}\right)^2=\dfrac{x^4+4x^2\sqrt[3]{x}+4\sqrt[3]{x^2}}{x^2}=x^2+4x^{\frac{1}{3}}+4x^{\frac{-4}{3}}$.\\
			$\displaystyle\int{\left(\dfrac{x^2+2\sqrt[3]{x}}{x}\right)^2\mathrm{\,d}x}=\displaystyle\int{\left(x^2+4x^{\frac{1}{3}}+4x^{\frac{-4}{3}}\right)\mathrm{\,d}x=\dfrac{x^3}{3}+4\dfrac{x^{\frac{4}{3}}}{\dfrac{4}{3}}}+4\dfrac{x^{\frac{-1}{3}}}{\dfrac{-1}{3}}+C$\\
			$=\dfrac{1}{3}{x^3}+3x^{\frac{4}{3}}-12x^{\frac{-1}{3}}+C$.}
	\end{ex}
	
	\begin{ex}%[2D4H1-2][Lê Công Trường]
		Tìm $m$ để $F(x)=m{x^3}+(3m+2){x^2}-4x+3$ là một nguyên hàm của hàm số $f(x)=3x^2+10x-4$.
		\shortans{$1$}
		\loigiai{
			$\displaystyle\int{f(x)\mathrm{\,d}x}=\displaystyle\int{\left(3x^2+10x-4\right)}\mathrm{\,d}x=x^3+5x^2-4x+C.$ Suy ra $ m=1$.
			}
	\end{ex}
	
	\begin{ex}%%[2D4V1-2][Lê Công Trường]
		Tìm $a,b,c$ để $F(x)=(a{x^2}+bx+c)\sqrt{x^2-4x}$ là một nguyên hàm của hàm số $f(x)=(x-2)\sqrt{x^2-4x}$. Giá trị biểu thức $a+b+c$ bằng.
		\shortans{$-1$}
		\loigiai{
			Đặt $ t=\sqrt{x^2-4x}\Rightarrow{t^2}=x^2-4x\Rightarrow 2t\mathrm{\,d}t=\left(2x-4\right)\mathrm{\,d}x=2\left(x-2\right)\mathrm{\,d}x$.\\
			$\Rightarrow \mathrm{\,d}x=\dfrac{2t\mathrm{\,d}t}{2\left(x-2\right)}=\dfrac{t\mathrm{\,d}t}{x-2}$.\\
			$\displaystyle\int{(x-2)\sqrt{x^2-4x}}\mathrm{\,d}x=\displaystyle\int{t.t.\mathrm{\,d}t=\displaystyle\int{t^2\mathrm{\,d}t}=\dfrac{1}{3}}{t^3}+C=\dfrac{1}{3}\sqrt{\left(x^2-4x\right)^3}+C$\\$=\dfrac{1}{3}\left(x^2-4x\right)\sqrt{x^2-4x}+C=\left(\dfrac{1}{3}{x^2}-\dfrac{4}{3}x\right)\sqrt{x^2-4x}+C$.\\
			Vậy $ a=\dfrac{1}{3};\,\,b=-\dfrac{4}{3};\,\,c=0$.}
	\end{ex}
	
	\begin{ex}%%[2D4V1-2][Lê Công Trường]
		Tìm $a,b,c$ để $F(x)=(a{x^2}+bx+c)\sqrt{2x-3}$ là một nguyên hàm của hàm số $f(x)=\dfrac{20x^2-30x+7}{\sqrt{2x-3}}$.  Giá trị biểu thức $a+b+c$ bằng\\
		\shortans{$3$}
		\loigiai{
 Theo định nghĩa nguyên hàm thì $ F'(x)=f(x)$.\\
Ta có 
\begin{eqnarray*}
	F'(x) & =& \left(2ax+b\right)\sqrt{2x-3}+(a{x^2}+bx+c)\dfrac{2}{2\sqrt{2x-3}}\\
	&= & \dfrac{\left(2ax+b\right)\left(2x-3\right)+a{x^2}+bx+c}{\sqrt{2x-3}}\\
	&= & \dfrac{5a{x^2}+\left(-6a+3b\right)x-3b+c}{\sqrt{2x-3}}.
\end{eqnarray*}
Từ đó ta có $\dfrac{5a{x^2}+\left(-6a+3b\right)x-3b+c}{\sqrt{2x-3}}=\dfrac{20x^2-30x+7}{\sqrt{2x-3}}$.\\
 Sử dụng phương pháp đồng nhất hệ số, ta được\\
$\heva{
	&5a=20\\
	&-6a+3b=-30\\
	&-3b+c=7.
}\Leftrightarrow
\heva{
	&a=4\\
	&b=-2\\
	&c=1.
}$
}
\end{ex}
\Closesolutionfile{ans}
\indapan{6}{ans/ans-2-B1-D1-KQ}

\begin{dang}{NGUYÊN HÀM HÀM LƯỢNG GIÁC}
	Bảng nguyên hàm hàm số lượng giác
	\begin{itemize}
		\item $\displaystyle\int{\cos{x} \mathrm{d}x} =\sin{x}+C$
		\item $\displaystyle\int{\sin{x} \mathrm{d}x} =-\cos{x}+C$
		\item $\displaystyle\int{\dfrac{1}{\cos^2{x}} \mathrm{d}x} =\tan{x}+C$
		\item $\displaystyle\int{\dfrac{1}{\sin^2{x}} \mathrm{d}x} =-\cot{x}+C$
	\end{itemize}
	\textbf{Chú ý:}
	$1+\tan ^2x=\dfrac{1}{\cos^2x}$; $1+\cot ^2x=\dfrac{1}{\sin^2x}$
\end{dang}
\Opensolutionfile{ans}[ans/ans-2-B1-D2-TN]
\TN
\begin{ex}%[2D4H1-3][Lê Công Trường]
	Hàm số $F(x)=\cot x$ là một nguyên hàm của hàm số nào dưới đây trên khoảng $\left(0;\dfrac{\pi}{2}\right)$
	\choice
	{$f_2(x)=\dfrac{1}{\sin^2x}$}
	{$f_1(x)=-\dfrac{1}{\cos^2x}$}
	{$f_4(x)=\dfrac{1}{\cos^2x}$}
	{\True $f_3(x)=-\dfrac{1}{\sin^2x}$}
	\loigiai{
		Có $\displaystyle\int{\dfrac{1}{\sin^2x}\mathrm{\,d}x}=-\cot x+C$ suy ra $F(x)=\cot x$ trên khoảng $\left(0;\dfrac{\pi}{2}\right)$ là một nguyên hàm của hàm số $f_3(x)=-\dfrac{1}{\sin^2x}$.}
\end{ex}

\begin{ex}%[2D4H1-3][Lê Công Trường]
	Cho hàm số $f(x)=1+\sin x$. Khẳng định nào dưới đây đúng?
	\choice
	{\True $\displaystyle\int{f(x){\rm{d}}x}=x-\cos x+C$}
	{$\displaystyle\int{f(x){\rm{d}}x}=x+\sin x+C$}
	{$\displaystyle\int{f(x){\rm{d}}x}=x+\cos x+C$}
	{$\displaystyle\int{f(x){\rm{d}}x}=\cos x+C$}
	\loigiai
	{Ta có $\displaystyle\int{f(x){\rm{d}}x=\displaystyle\int{\left(1+\sin x\right){\rm{d}}x}=\displaystyle\int{1\rm{d}x}+\displaystyle\int{\sin x{\rm{d}}x}=x-\cos x+C}$.}
\end{ex}

\begin{ex}%[2D4H1-3][Lê Công Trường]
	Tìm nguyên hàm $F(x)$ của hàm số $f(x)=\cos ^2\dfrac{x}{2}$
	\choice
	{$F(x)=2\cos\dfrac{x}{2}+C$}
	{\True $F(x)=\dfrac{1}{2}\left(1+\sin x\right)+C$}
	{$F(x)=2\sin\dfrac{x}{2}+C$}
	{$F(x)=\dfrac{1}{2}\left(1-\sin x\right)+C$}
	\loigiai{
		Ta có:$f(x)=\cos ^2\dfrac{x}{2}\Rightarrow F(x)=\displaystyle\int{\cos^2\dfrac{x}{2}\mathrm{\,d}x}=\displaystyle\int{\dfrac{1+\cos x}{2}\mathrm{\,d}x}=\dfrac{1}{2}\displaystyle\int{\left(1+\cos x\right)\mathrm{\,d}x}$\\
		$=\dfrac{1}{2}\left(1+\sin x\right)+C$.}
\end{ex}

\begin{ex}%[2D4H1-3][Lê Công Trường]
	Cho hàm số $f(x)=1-\dfrac{1}{\cos^2x}$. Khẳng định nào dưới đây đúng?
	\choice
	{$\displaystyle\int{f(x){\rm{d}}x}=x+\tan x+C$}
	{$\displaystyle\int{f(x){\rm{d}}x}=x+\cot x+C$}
	{\True $\displaystyle\int{f(x){\rm{d}}x}=x-\tan x+C$}
	{$\displaystyle\int{f(x){\rm{d}}x}=x-\cot x+C$}
	\loigiai
	{
		$\displaystyle\int{f(x){\rm{d}}x}=\displaystyle\int{\left(1-\dfrac{1}{\cos^2x}\right){\rm{d}}x}=x-\tan x+C$.}
\end{ex}

\begin{ex}%%[2D4H1-3][Lê Công Trường]
	Họ nguyên hàm của hàm số $f(x)=\cos x+6x$ là
	\choice
	{\True $\sin x+3x^2+C$}
	{$-\sin x+3x^2+C$}
	{$\sin x+6x^2+C$}
	{$-\sin x+C$}
	\loigiai
	{
		Ta có $\displaystyle\int{f(x){\rm{d}}x=\displaystyle\int{\left(\cos x+6x\right){\rm{d}}x=\sin x+3x^2+C}}$.}
\end{ex}

\begin{ex}%%[2D4H1-3][Lê Công Trường]
	Tìm nguyên hàm của hàm số $f(x)=2\sin x+3x$.
	\choice
	{\True $\displaystyle\int{\left(2\sin x+3x\right)\mathrm{\,d}x=-2\cos x+\dfrac{3}{2}{x^2}+C}$}
	{$\displaystyle\int{\left(2\sin x+3x\right)\mathrm{\,d}x=2\cos x+3x^2+C}$}
	{$\displaystyle\int{\left(2\sin x+3x\right)\mathrm{\,d}x=\sin^2x+\dfrac{3}{2}x+C}$}
	{$\displaystyle\int{\left(2\sin x+3x\right)\mathrm{\,d}x=\sin 2x+\dfrac{3}{2}{x^2}+C}$}
	\loigiai
	{
		$\displaystyle\int{\left(2\sin x+3x\right)\mathrm{\,d}x}=-2\cos x+\dfrac{3}{2}{x^2}+C$}
\end{ex}

\begin{ex}%%[2D4H1-3][Lê Công Trường]
	Tính$\displaystyle\int{\left(x-\sin x\right)}{\rm{d}}x$.
	\choice
	{$\dfrac{x^2}{2}+\sin x+C$}
	{$\dfrac{x^2}{2}-\cos x+C$}
	{$\dfrac{x^2}{2}-\sin x+C$}
	{\True $\dfrac{x^2}{2}+\cos x+C$}
\loigiai
{
	Ta có $\displaystyle\int{\left(x-\sin x\right){\rm{d}}x\,\rm{=}\,}\dfrac{x^2}{2}+\cos x+C$.}
	\end{ex}
\begin{ex}%[2D4H1-3][Lê Công Trường]
	Họ nguyên hàm của hàm số $f(x)=3x^2+\sin x$ là
	\choice
	{$x^3+\cos x+C$}
	{$6x+\cos x+C$}
	{\True $x^3-\cos x+C$}
	{$6x-\cos x+C$}
	\loigiai
	{
		Ta có $\displaystyle\int\left(3x^2+\sin x\right){\rm{d}}x=x^3-\cos x+C$.}
\end{ex}
\begin{ex}%[2D4H1-3][Lê Công Trường]
	Họ nguyên hàm của hàm số $ f(x)=\dfrac{1}{x}+\sin x$ là
	\choice
	{$\ln x-\cos x+C$}
	{$-\dfrac{1}{x^2}-\cos x+C$}
	{$\ln \left| x\right|+\cos x+C$}
	{\True $\ln \left| x\right|-\cos x+C$}
	\loigiai
	{
		Ta có $\displaystyle\int{f(x){\rm{d}}x}=\displaystyle\int{\left(\dfrac{1}{x}+\sin x\right){\rm{d}}x}=\displaystyle\int{\dfrac{1}{x}{\rm{d}}x}+\displaystyle\int{\sin x{\rm{d}}x}=\ln \left| x\right|-\cos x+C$.}
\end{ex}
\begin{ex}%%[2D4H1-3][Lê Công Trường]
	Cho $\displaystyle\int{f(x)}\,\rm{d}x=-\cos x+C$. Khẳng định nào dưới đây đúng?
	\choice
	{$ f(x)=-\sin x$}
	{$ f(x)=-\cos x$}
	{\True $ f(x)=\sin x$}
	{$ f(x)=\cos x$}
	\loigiai{
		Áp dụng công thức $\smallint{\rm{sin}}x{\rm{\;d}}x=-\rm{cos}x+C$. Suy ra $ f(x)=\rm{sin}x$.}
\end{ex}
\begin{ex}%[2D4H1-3][Lê Công Trường]
	Cho hàm số $ f(x)=\displaystyle\int{\cos\dfrac{x}{2}\sin\dfrac{x}{2}}$. Khẳng định nào dưới đây đúng?
	\choice
	{$\displaystyle\int{\cos\dfrac{x}{2}\sin\dfrac{x}{2}}=\dfrac{1}{2}\sin+C$}
	{$\displaystyle\int{\cos\dfrac{x}{2}\sin\dfrac{x}{2}}=\dfrac{1}{2}\cos x+C$}
	{$\displaystyle\int{\cos\dfrac{x}{2}\sin\dfrac{x}{2}}=-\dfrac{1}{2}\sin x+C$}
	{\True $\displaystyle\int{\cos\dfrac{x}{2}\sin\dfrac{x}{2}}=-\dfrac{1}{2}\cos x+C$}
	\loigiai{
		$\displaystyle\int{\cos\dfrac{x}{2}\sin\dfrac{x}{2}}=\dfrac{1}{2}\displaystyle\int{\sin x}\mathrm{\,d}x=-\dfrac{1}{2}\cos x+C$.}
\end{ex}
\Closesolutionfile{ans}
\indapan{10}{ans/ans-2-B1-D2-TN}
\Opensolutionfile{ans}[ans/ans-2-B1-D2-DS]
\TNTF
	\begin{ex}%%[2D4H1-3][Lê Công Trường]
		Các mệnh đề sau đây đúng hay sai?
		\choiceTF
		{\True $\displaystyle\int{\left(2+\cot^2x\right)\mathrm{\,d}x}=x-\cot x+C$}
		{ $\displaystyle\int{\left(1-\cos^2\dfrac{x}{2}\right)\mathrm{\,d}x}=\dfrac{1}{2}\left(x+\sin x\right)+C$}
		{$\displaystyle\int{\left(\sin\dfrac{x}{2}+\cos\dfrac{x}{2}\right)^2}\mathrm{\,d}x=x+\cos x+C$}
		{ $\displaystyle\int{\left(\sin\dfrac{x}{2}-\cos\dfrac{x}{2}\right)^2}\mathrm{\,d}x=x-\cos x+C$}
		\loigiai{
				\begin{itemchoice}
				\itemch {\bf Đúng}. Vì
			$\displaystyle\int{\left(2+\cot^2x\right)\mathrm{\,d}x}$\\
			$=\displaystyle\int{\left(1+1+\cot^2x\right)\mathrm{\,d}x}=\displaystyle\int{\left(1+\dfrac{1}{\sin^2x}\right)\mathrm{\,d}x}=x-\cot x+C$.
			\itemch {\bf Sai}. Vì $\displaystyle\int{\left(1-\cos^2\dfrac{x}{2}\right)\mathrm{\,d}x}=\displaystyle\int{\sin^2\dfrac{x}{2}\mathrm{\,d}x}=\displaystyle\int{\dfrac{1-\cos x}{2}\mathrm{\,d}x}=\dfrac{1}{2}\left(x-\sin x\right)+C$.
			\itemch {\bf Sai}. Vì $\displaystyle\int{\left(\sin\dfrac{x}{2}+\cos\dfrac{x}{2}\right)^2}\mathrm{\,d}x=\displaystyle\int{\left(1+\sin x\right)}\mathrm{\,d}x=x-\cos x+C$.
			\itemch {\bf Sai}. Vì $\displaystyle\int{\left(\sin\dfrac{x}{2}-\cos\dfrac{x}{2}\right)^2}\mathrm{\,d}x=\displaystyle\int{\left(1-\sin x\right)}\mathrm{\,d}x=x+\cos x+C$.
			\end{itemchoice}
		}
	\end{ex}
\Closesolutionfile{ans}
\indapan{2}{ans/ans-2-B1-D2-DS}
\Opensolutionfile{ans}[ans/ans-2-B1-D2-KQ]
\TNSA
\begin{ex}%%[2D4H1-3][Lê Công Trường]
	Tìm nguyên hàm $ F(x)$của hàm số $f(x)=2024-2\sin ^2\dfrac{x}{2}$. Hệ số của biến $x$ là
	\shortans{$2023$}
\loigiai{
	$\Rightarrow F(x)=\displaystyle\int{\left(2024-2\sin^2\dfrac{x}{2}\right)}\mathrm{\,d}x=\displaystyle\int{\left(2023+\cos x\right)}\mathrm{\,d}x=2023x-\sin x+C$.}
	\end{ex}
\begin{ex}%%[2D4H1-3][Lê Công Trường]
	Tìm nguyên hàm $ F(x)$của hàm số $f(x)=\dfrac{1}{\sin^2\dfrac{x}{2}\cdot\cos^2\dfrac{x}{2}}==a\cot x+C$. Giá trị $a$ là
		\shortans{$-4$}
\loigiai{
Ta có $\dfrac{1}{\sin^2\dfrac{x}{2}\cdot\cos^2\dfrac{x}{2}}=\dfrac{1}{\left(\sin\dfrac{x}{2}\cdot\cos\dfrac{x}{2}\right)^2}=\dfrac{1}{\left(\dfrac{\sin x}{2}\right)^2}=\dfrac{4}{\sin^2x}\cdot$\\
	$ F(x)=\displaystyle\int{f(x)\mathrm{\,d}x=\displaystyle\int{\dfrac{1}{\sin^2\dfrac{x}{2}\cdot\cos^2\dfrac{x}{2}}}}\mathrm{\,d}x=\displaystyle\int{\dfrac{4}{\sin^2x}=-4\cot x+C}$.}
	\end{ex}
\begin{ex}%%[2D4H1-3][Lê Công Trường]
	Tìm nguyên hàm $ F(x)$ của hàm số $f(x)=\dfrac{1}{3}{x^2}-2x+\dfrac{1}{2}{\tan ^2}x=\dfrac{x^3}{a}+bx^2+\dfrac{1}{c}x+d\tan x+C$. Giá trị của $a+b+c+d$ là
	\shortans{$6{,}5$}
\loigiai{$F(x)=\displaystyle\int{f(x)\mathrm{\,d}x}$\\
	$=\displaystyle\int{\left(\dfrac{1}{3}{x^2}-2x+\dfrac{1}{2}{\tan^2}x\right)}\mathrm{\,d}x=\displaystyle\int{\left(\dfrac{1}{3}{x^2}-2x+\dfrac{1}{2}\dfrac{\sin^2x}{\cos^2x}\right)}\mathrm{\,d}x\\
		=\displaystyle\int{\left[\dfrac{1}{3}{x^2}-2x+\dfrac{1}{2}\left(\dfrac{1-\cos^2x}{\cos^2x}\right)\right]}\mathrm{\,d}x=\displaystyle\int{\left[\dfrac{1}{3}{x^2}-2x+\dfrac{1}{2}\left(\dfrac{1}{\cos^2x}-1\right)\right]}\mathrm{\,d}x\\
		=\dfrac{x^3}{9}-x^2+\dfrac{1}{2}\left(\tan x-x\right)+C=\dfrac{x^3}{9}-x^2-\dfrac{1}{2}x+\dfrac{1}{2}\tan x+C$.
		}
	\end{ex}
\begin{ex}%%[2D4V1-3][Lê Công Trường]
	Tính $I=\displaystyle\int{x\left(1-\dfrac{\sin^2\dfrac{x}{2}}{2}\right)\mathrm{\,d}x}$. Hệ số của hạng tử $\cos {x}$ của $I$ là
		\shortans{$-1$} 
\loigiai{
	Đáp án: Ta có $x\left(1-\dfrac{\sin^2\dfrac{x}{2}}{2}\right)=x\left(1-\dfrac{1-cox}{4}\right)=\dfrac{3}{4}x+\dfrac{1}{4}x\cos x$.\\
	$\displaystyle\int x\left(1-\dfrac{\sin^2\dfrac{x}{2}}{2}\right)\mathrm{\,d}x=\displaystyle\int\left(\dfrac{3}{4}x+\dfrac{1}{4}x\cos x\right)\mathrm{\,d}x=\displaystyle\int\dfrac{3}{4}x\mathrm{\,d}x+\displaystyle\int\dfrac{1}{4}x\cos x\mathrm{\,d}x$\\
	$=\dfrac{3}{8}{x^2}+C_1+\dfrac{1}{4}\displaystyle\int{x\cos x\mathrm{\,d}x.}$\\
	Đặt $\heva{
		&u=x\Rightarrow \mathrm{\,d}u=\mathrm{\,d}x\\
		&dv=\cos x\mathrm{\,d}x\Rightarrow v=\sin x.
}$\\
 Sử dụng phương pháp tích phân từng phần, ta có\\
	$\displaystyle\int{x\cos x\mathrm{\,d}x}=x\sin x+\displaystyle\int{\sin x\mathrm{\,d}x=x\sin x-\cos x+C_2}$.\\
	Vậy $\displaystyle\int{x\left(1-\dfrac{\sin^2\dfrac{x}{2}}{2}\right)\mathrm{\,d}x}=\dfrac{3}{8}{x^2}+x\sin x-\cos x+C.$}
	\end{ex}	
	\begin{ex}%%[2D4H1-3][Lê Công Trường]
		Tính $\displaystyle\int{x^2\left(1+\dfrac{1}{x}-\dfrac{\tan^2x}{x^2}\right)\mathrm{\,d}x}=\dfrac{x^m}{n}+\dfrac{x^p}{q}+x+r\tan x+C$. Giá trị biểu thức $P=\dfrac{m}{n}+\dfrac{p}{q}+2r$ là	
		\shortans{$0$} 
	\loigiai{
 $\displaystyle\int{x^2\left(1+\dfrac{1}{x}-\dfrac{\tan^2x}{x^2}\right)\mathrm{\,d}x}=\displaystyle\int{\left(x^2+x-\tan^2x\right)\mathrm{\,d}x}=\dfrac{x^3}{3}+\dfrac{x^2}{2}-(\tan x-x)+C$\\
 $=\dfrac{x^3}{3}+\dfrac{x^2}{2}+x-\tan x+C$.}
		\end{ex}
	\begin{ex}%%[2D4V1-3][Lê Công Trường]
			Tính $T=\displaystyle\int{x\left(2024-\dfrac{1}{x^3}+\dfrac{\sin x}{x}\right)\mathrm{\,d}x}$. Hệ số của hạng tử $\cos {x}$ của $T$ là
					\shortans{$-1$} 
		\loigiai{
			$\displaystyle\int{x\left(2024-\dfrac{1}{x^3}+\dfrac{\sin x}{x}\right)\mathrm{\,d}x}=\displaystyle\int{\left(2024x-\dfrac{1}{x^2}+\sin x\right)}\mathrm{\,d}x=1012x^2+\dfrac{1}{x}-\cos x+C.$}
	\end{ex}
\begin{ex}%Câu 27%[2D4H1-5]
	Tính $R=\displaystyle\int{x^3\left[\dfrac{\left(\sin\dfrac{x}{2}+\cos\dfrac{x}{2}\right)^2}{x^3}-2x+\dfrac{1}{x^{2024}}\right]}\mathrm{\,d}x= ax+b\cos x+c{x^5}-\dfrac{1}{d\cdot x^{2020}}+C$. Giá trị $a+b+c+d+7$ là (làm tròn đến hàng đơn vị)
			\shortans{$2025$} 
\loigiai{
	Ta có
	\begin{eqnarray*}
	{x^3}\left[\dfrac{\left(\sin\dfrac{x}{2}+\cos\dfrac{x}{2}\right)^2}{x^3}-2x+\dfrac{1}{x^{2024}}\right] &=& \left(\sin\dfrac{x}{2}+\cos\dfrac{x}{2}\right)^2-2x^4+x^{-2021}\\
		&=& \sin ^2\dfrac{x}{2}+\cos^2\dfrac{x}{2}+2\sin\dfrac{x}{2}\cos\dfrac{x}{2}-2x^4+x^{-2021}\\
		&=&1+2\sin x-2x^4+x^{-2021}.
	\end{eqnarray*}
	Khi đó\\
	\begin{eqnarray*}
		\displaystyle\int{x^3\left[\dfrac{\left(\sin\dfrac{x}{2}+\cos\dfrac{x}{2}\right)^2}{x^3}-2x+\dfrac{1}{x^{2024}}\right]}\mathrm{\,d}x&=& \displaystyle\int{\left(1+2\sin x-2x^4+x^{-2021}\right)\mathrm{\,d}x}\\
		&= & x-2\cos x-\dfrac{2}{5}{x^5}-\dfrac{1}{2020x^{2020}}+C.
	\end{eqnarray*}
}
	\end{ex}	
	\begin{ex}%[2D4V1-3][Lê Công Trường]
		Tính $\displaystyle\int{x^2\left[\dfrac{1}{x^2\sin^2\dfrac{x}{2}\cdot\cos^2\dfrac{x}{2}}+\dfrac{3}{x^3}-\dfrac{4}{x^4}\right]}\mathrm{\,d}x=a\cot{x}+b\ln \left| x\right|+\dfrac{c}{x}+C$. Giá trị $a+b+c$ là
		\shortans{$3$} 
	\loigiai{
		Ta có\\
		$\dfrac{1}{\sin^2\dfrac{x}{2}\cdot\cos^2\dfrac{x}{2}}=\dfrac{1}{\left(\sin\dfrac{x}{2}\cdot\,\cos\dfrac{x}{2}\right)^2}=\dfrac{1}{\left(\dfrac{\sin x}{2}\right)^2}=\dfrac{4}{\sin^2x}.$\\
		$x^2\left[\dfrac{1}{x^2\sin^2\dfrac{x}{2}\cdot\cos^2\dfrac{x}{2}}+\dfrac{3}{x^3}-\dfrac{4}{x^4}\right]=\dfrac{1}{\sin^2\dfrac{x}{2}\cdot\cos^2\dfrac{x}{2}}+\dfrac{3}{x}-\dfrac{4}{x^2}=\dfrac{4}{\sin^2x}+\dfrac{3}{x}-\dfrac{4}{x^2}$.\\
Khi đó
\begin{eqnarray*}
\displaystyle\int{x^2\left[\dfrac{1}{x^2\sin^2\dfrac{x}{2}\cdot\cos^2\dfrac{x}{2}}+\dfrac{3}{x^3}-\dfrac{4}{x^4}\right]}\mathrm{\,d}x	&= & \displaystyle\int{\left(\dfrac{4}{\sin^2x}+\dfrac{3}{x}-\dfrac{4}{x^2}\right)}\mathrm{\,d}x\\
	&= & -4\cot x+3\ln \left| x\right|+\dfrac{4}{x}+C.
\end{eqnarray*}
}
\end{ex}
\Closesolutionfile{ans}
\indapan{6}{ans/ans-2-B1-D2-KQ}