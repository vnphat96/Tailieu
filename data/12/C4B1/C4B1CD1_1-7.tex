\chapter{NGUYÊN HÀM VÀ TÍCH PHÂN}
\section{NGUYÊN HÀM}
\subsection{Tóm tắt lý thuyết}
\subsubsection{Khái niệm nguyên hàm}
Cho hàm số $f(x)$ xác định trên $K$. Hàm số $F(x)$ được gọi là nguyên hàm của hàm số $f(x)$ nếu $F'(x)=f(x)$, với mọi $x\in K$.\\
Cho $F(x)$ là một nguyên hàm của hàm số $f(x)$ trên $K$. Khi đó
\begin{itemize}
	\item Với mỗi hằng số $C$, hàm số $F(x)+C$ cũng là một nguyên hàm của hàm số $f(x)$ trên $K$.
	\item Nếu $G(x)$ là một nguyên hàm của hàm số $f(x)$ trên $K$ thì tồn tại hằng số $C$ sao cho $G(x)=F(x)+C$ với mọi $x\in K$.
\end{itemize}
Như vậy, mọi nguyên hàm của hàm số $f(x)$ trên $K$ đều có dạng $F(x)+C$, với $C$ là hằng số. Ta gọi $F(x)+C$, $C\in \mathbb{R}$ là họ tất cả các nguyên hàm của hàm số $f(x)$ trên $K$, kí hiệu $\displaystyle  \displaystyle\int{f(x)\mathrm{\,d}x}$ và viết
$$\displaystyle  \displaystyle\int{f(x)\mathrm{\,d}x}=F(x)+C.$$
\begin{note}\textbf{Chú ý}
	\begin{itemize}
		\item Biểu thức $f(x)\mathrm{\,d}x$ được gọi là vi phân của nguyên hàm $F(x)$ của $f(x)$, kí hiệu là $dF(x)$.
		Vậy, $\mathrm{d}F(x)=F'\mathrm{\,d}x=f(x)\mathrm{\,d}x$.
		\item Mọi hàm số $f(x)$ liên tục trên $K$ đều có nguyên hàm trên $K$.
		\item Khi tìm nguyên hàm của một hàm số mà không chỉ rõ tập $K$ thì ta hiểu là tìm nguyên hàm của hàm số đó trên tập xác định của nó.
		\item $\displaystyle\int{f'(x)\mathrm{\,d}x}=f(x)+C$.
	\end{itemize}
\end{note}
\subsubsection{Các tính chất của nguyên hàm}
\begin{itemize}
	\item $\displaystyle\int{kf(x)\mathrm{\,d}x}=k\displaystyle\int{f(x)\mathrm{\,d}x}$, với $k$ là hằng số khác $0$.
	\item $\displaystyle\int{\left[f(x)+g(x)\right]\mathrm{\,d}x}=\displaystyle\int{f(x)\mathrm{\,d}x}+\displaystyle\int{g(x)\mathrm{\,d}x}$.
	\item $\displaystyle\int{\left[f(x)-g(x)\right]\mathrm{\,d}x}=\displaystyle\int{f(x)\mathrm{\,d}x}-\displaystyle\int{g(x)\mathrm{\,d}x}$.
\end{itemize}
\subsubsection{Nguyên hàm của một hàm số sơ cấp}
%\begin{center}
%	\begin{tabular}{|>{\centering\arraybackslash}p{7cm}|>{\centering\arraybackslash}p{6cm}|}
%		\hline
%		&$\displaystyle\int{0\mathrm{\,d}x}=C$\\
%		Nguyên hàm của hàm số lũy thừa&$\displaystyle\int{\mathrm{\,d}x}=x+C$\\
%		&$\displaystyle\int{x^\alpha\mathrm{\,d}x}=\dfrac{x^{\alpha+1}}{\alpha+1}+C$ $(\alpha\ne-1)$
%		\\ \hline
%		Nguyên hàm của hàm số $y=\dfrac{1}{x}$&$\displaystyle\int{\dfrac{\mathrm{\,d}x}{x}}=\ln |x|+C$ $(x\ne0)$\\ \hline
%		&\\
%		&\\
%		Nguyên hàm của hàm số lượng giác&$a$\\
%		&\\ \hline
%		Nguyên hàm của hàm số lũy thừa&$8$\\ \hline
%	\end{tabular}
%\end{center}
\begin{enumerate}[1.]
	\item Nguyên hàm của hàm số lũy thừa
	\begin{itemize}
		\item $\displaystyle\int{0\mathrm{\,d}x}=C$;
		\item $\displaystyle\int{\mathrm{\,d}x}=x+C$;
		\item $\displaystyle\int{x^\alpha\mathrm{\,d}x}=\dfrac{x^{\alpha+1}}{\alpha+1}+C$ $(\alpha\ne-1)$.
	\end{itemize}
\item Nguyên hàm của hàm số $y=\dfrac{1}{x}$
$$\displaystyle\int{\dfrac{\mathrm{\,d}x}{x}}=\ln |x|+C\,(x\ne0).$$
\item Nguyên hàm của hàm số lượng giác
\begin{itemize}
	\item $\displaystyle\int{\cos x\mathrm{\,d}x}=\sin x+C$;
	\item $\displaystyle\int{\sin x\mathrm{\,d}x}=-\cos x+C$;
	\item $\displaystyle\int{\dfrac{1}{\cos^2x}\mathrm{\,d}x}=\tan x+C$;
	\item $\displaystyle\int{\dfrac{1}{\sin^2x}\mathrm{\,d}x}=-\cot x+C$.
\end{itemize}
\item Nguyên hàm của hàm số mũ
\begin{itemize}
	\item $\displaystyle\int{\mathrm{e}^x\mathrm{\,d}x}=\mathrm{e}^x+C$;
	\item $\displaystyle\int{a^x\mathrm{\,d}x}=\dfrac{a^x}{\ln a}+C$ $(0<a\ne1)$.
\end{itemize}
\end{enumerate}
\subsection{Bài tập}
\subsubsection{TÍNH NGUYÊN HÀM MỘT SỐ HÀM SỐ SƠ CẤP}
\begin{dang}{NGUYÊN HÀM HÀM LŨY THỪA}
	\begin{enumerate}[1.]
		\item Nguyên hàm hàm số lũy thừa
		\begin{itemize}
			\item $\displaystyle\int{0\mathrm{\,d}x}=C$;
			\item $\displaystyle\int{\mathrm{\,d}x}=x+C$;
			\item $\displaystyle\int{x^\alpha\mathrm{\,d}x}=\dfrac{x^{\alpha+1}}{\alpha+1}+C$ $(\alpha\ne-1)$;
			\item $\displaystyle\int{\dfrac{1}{x^2}\mathrm{\,d}x}=-\dfrac{1}{x}+C$;
			\item $\displaystyle\int{\dfrac{1}{\sqrt{x}}\mathrm{\,d}x}=2\sqrt{x}+C$;
			\item $\displaystyle\int{\dfrac{1}{x}\mathrm{\,d}x}=\ln |x|+C$ $(x\ne0)$.
		\end{itemize}
\begin{note}\textbf{Chú ý}
	Dùng công thức sau làm trắc nghiệm cho nhanh $$\displaystyle\int(ax+b)^n\mathrm{\,d}x=\dfrac{1}{a}\dfrac{(ax+b)^{n+1}}{n+1}+C.$$
\end{note}
	
		\item Lũy thừa với số mũ thực\\
	Cho $a$, $b$ là những số thực dương, $\alpha$, $\beta$ là những số thực bất kì. Khi đó
	\begin{itemize}
		\item $a^{\alpha}a^{\beta}=a^{\alpha +\beta}$;
		\item $\dfrac{a^{\alpha}}{a^{\beta}}=a^{\alpha -\beta}$;
		\item $(a^{\alpha})^{\beta}=a^{\alpha\cdot \beta}$;
		\item $(ab)^{\alpha}=a^{\alpha}b^{\alpha}$;
		\item $\left(\dfrac{a}{b}\right)^{\alpha}=\dfrac{a^{\alpha}}{b^{\alpha}}$.
	\end{itemize}
	\end{enumerate}
\end{dang}
\TN
\Opensolutionfile{ans}[ans/ans-2-B1-D2-LC]
\begin{ex}%[2D4N1-1]
	Cho hàm số $F(x)$ là một nguyên hàm của hàm số $f(x)$ trên $K$. Các mệnh đề sau, mệnh đề nào \textbf{sai}.
	\choice
	{$\displaystyle\int{f(x)\mathrm{\,d}x=}F(x)+C$}
	{$\displaystyle{\left(\displaystyle\int{f(x)\mathrm{\,d}x}\right)'}=f(x)$}
	{\True $\displaystyle{\left(\displaystyle\int{f(x)\mathrm{\,d}x}\right)'}=f'(x)$}
	{$\displaystyle{\left(\displaystyle\int{f(x)\mathrm{\,d}x}\right)'}=F'(x)$}
	\loigiai{
		Ta có $\displaystyle\int{f(x)\mathrm{\,d}x=}F(x)+C\Leftrightarrow F'(x)=f(x)$ nên phương án $\left(\displaystyle\int{f(x)\mathrm{\,d}x}\right)'=f'(x)$ sai.}
\end{ex}

\begin{ex}%[2D4N1-2]
	Họ tất cả các nguyên hàm của hàm số $f(x)=2x+6$ là
	\choice
	{$x^2+C$}
	{\True $x^2+6x+C$}
	{$2x^2+C$}
	{$2x^2+6x+C$}
	\loigiai{
		$\displaystyle\int{(2x+6)\mathrm{\,d}x=x^2+6x+C}$.}
\end{ex}

\begin{ex}%[2D4N1-2]
	$\displaystyle\int{x^2\mathrm{\,d}x}$ bằng
	\choice
	{$2x+C$}
	{\True $\dfrac{1}{3}x^3+C$}
	{$x^3+C$}
	{$3x^3+C$}
	\loigiai{
		Ta có $\displaystyle\int{x^2\mathrm{\,d}x}=\dfrac{1}{3}x^3+C$.}
\end{ex}

\begin{ex}%[2D4N1-2]
	Họ nguyên hàm của hàm số $f(x)=3x^2+1$ là
	\choice
	{$x^3+C$}
	{$\dfrac{x^3}{3}+x+C$}
	{$6x+C$}
	{\True $x^3+x+C$}
	\loigiai{
		$\displaystyle\int{(3x^2+1)\mathrm{\,d}x=x^3+x+C}$.}
\end{ex}

\begin{ex}%[2D4N1-2]
	Nguyên hàm của hàm số $f(x)=x^3+x$ là
	\choice
	{\True $\dfrac{1}{4}x^4+\dfrac{1}{2}x^2+C$}
	{$3x^2+1+C$}
	{$x^3+x+C$}
	{$x^4+x^2+C$}
	\loigiai{
		$\displaystyle\int{(x^3+x^2)\mathrm{\,d}x}=\dfrac{1}{4}x^4+\dfrac{1}{2}x^2+C$.}
\end{ex}

\begin{ex}%[2D4N1-2]
	Nguyên hàm của hàm số $f(x)=x^4+x^2$ là
	\choice
	{\True $\dfrac{1}{5}x^5+\dfrac{1}{3}x^3+C$}
	{$x^4+x^2+C$}
	{$x^5+x^3+C$}
	{$4x^3+2x+C$}
	\loigiai{
		$\displaystyle\int{f(x)\mathrm{\,d}x}=\displaystyle\int{(x^4+x^2)\mathrm{\,d}x}$ $=\dfrac{1}{5}x^5+\dfrac{1}{3}x^3+C$.}
\end{ex}

\begin{ex}%[2D4H1-2]
	Hàm số nào trong các hàm số sau đây không là nguyên hàm của hàm số $y=x^{2022}$?
	\choice
	{$\dfrac{x^{2023}}{2023}+1$}
	{$\dfrac{x^{2023}}{2023}$}
	{\True $y=2022x^{2021}$}
	{$\dfrac{x^{2023}}{2023}-1$}
	\loigiai{
		Ta có $\displaystyle\int{x^{2022}\mathrm{\,d}}x=\dfrac{x^{2023}}{2023}+C$, $C$ là hằng số nên $y=2022x^{2021}$ không là nguyên hàm của hàm số $y=x^{2022}$.}
\end{ex}

\begin{ex}%[2D4H1-2]
	Nguyên hàm của hàm số $f(x)=$ $\dfrac{1}{3}x^3-2x^2+x-2024$ là
	\choice
	{$\dfrac{1}{12}x^4-\dfrac{2}{3}x^3+\dfrac{x^2}{2}+C$}
	{$\dfrac{1}{9}x^4-\dfrac{2}{3}x^3+\dfrac{x^2}{2}-2024x+C$}
	{\True $\dfrac{1}{12}x^4-\dfrac{2}{3}x^3+\dfrac{x^2}{2}-2024x+C$}
	{$\dfrac{1}{9}x^4+\dfrac{2}{3}x^3-\dfrac{x^2}{2}-2024x+C$}
	\loigiai{
		Sử dụng công thức $\displaystyle\int{x^n\mathrm{\,d}x=\dfrac{x^{n+1}}{n+1}+C}$ ta được
		\begin{eqnarray*}
		\displaystyle\int\left(\dfrac{1}{3}x^3-2x^2+x-2024\right)\mathrm{\,d}x&=&\dfrac{1}{3}\cdot \dfrac{x^4}{4}-2\cdot \dfrac{x^3}{3}+\dfrac{x^2}{2}-2024x+C\\&=&\dfrac{1}{12}x^4-\dfrac{2}{3}x^3+\dfrac{1}{2}x^2-2024x+C.
		\end{eqnarray*}
		}
\end{ex}

\begin{ex}%[2D4H1-2]
	Tìm nguyên $F(x)$ của hàm số $f(x)=(x+1)(x+2)(x+3)?$
	\choice
	{$F(x)=\dfrac{x^4}{4}-6x^3+\dfrac{11}{2}x^2-6x+C$}
	{$F(x)=x^4+6x^3+11x^2+6x+C$}
	{\True $F(x)=\dfrac{x^4}{4}+2x^3+\dfrac{11}{2}x^2+6x+C$}
	{$F(x)=x^3+6x^2+11x^2+6x+C$}
	\loigiai{
		Ta có $f(x)=(x+1)(x+2)(x+3)=x^3+6x^2+11x+6$ nên\\
		$\displaystyle F(x)=\displaystyle\int{(x^3+6x^2+11x+6)}\mathrm{\,d}x=\dfrac{x^4}{4}+2x^3+\dfrac{11}{2}x^2+6x+C$.}
\end{ex}

\begin{ex}%[2D4H1-2]
	Tìm nguyên hàm của hàm số $f(x)=(5x+3)^5$.
	\choice
	{$(5x+3)^6+C$}
	{$(5x+3)^4+C$}
	{\True $\dfrac{(5x+3)^6}{30}+C$}
	{$\dfrac{(5x+3)^4}{30}+C$}
	\loigiai{
		$f(x)=(5x+3)^5$ $\displaystyle \Rightarrow \displaystyle\int{f(x)\mathrm{\,d}x=}\displaystyle\int{(5x+ 3)^5\mathrm{\,d}x=}\dfrac{1}{5}\cdot \dfrac{(5x+3)^6}{6}+C=\dfrac{(5x+3)^6}{30}+C$.}
\end{ex}

\begin{ex}%[2D4H1-2]
	Tìm nguyên hàm của hàm số $f(x)=x^2+\dfrac{2}{x^2}$.
	\choice
	{\True $\displaystyle\int{f(x)\mathrm{\,d}x}=\dfrac{x^3}{3}+\dfrac{1}{x}+C$}
	{$\displaystyle\int{f(x)\mathrm{\,d}x}=\dfrac{x^3}{3}-\dfrac{2}{x}+C$}
	{$\displaystyle\int{f(x)\mathrm{\,d}x}=\dfrac{x^3}{3}-\dfrac{1}{x}+C$}
	{$\displaystyle\int{f(x)\mathrm{\,d}x}=\dfrac{x^3}{3}+\dfrac{2}{x}+C$}
	\loigiai{
		Ta có $\displaystyle\int{\left(x^2+\dfrac{2}{x^2}\right)\mathrm{\,d}x}=\dfrac{x^3}{3}-\dfrac{2}{x}+C$.}
\end{ex}

\begin{ex}%[2D4H1-4]
	Tính $\displaystyle\int{\sqrt{x\sqrt{x\sqrt{x}}}\mathrm{\,d}x}$.
	\choice
	{$\dfrac{4}{15}x\sqrt[15]x^7+C$}
	{\True $\dfrac{8}{15}x\sqrt[15]x^7+C$}
	{$\dfrac{8}{15}x\sqrt[15]x+C$}
	{$\dfrac{4}{15}x\sqrt[15]x+C$}
	\loigiai{
		\begin{eqnarray*}
		\displaystyle\int{\sqrt{x\sqrt{x\sqrt{x}}}\mathrm{\,d}x}&=&\displaystyle\int{\sqrt{x\sqrt{x\cdot{x^{\frac{1}{2}}}}}\mathrm{\,d}x}=\displaystyle\int{\sqrt{x\cdot{x^{\frac{3}{4}}}}\mathrm{\,d}x}=\displaystyle\int{x^{\frac{7}{8}}\mathrm{\,d}x}\\&=&\dfrac{x^{\frac{7}{8}+1}}{\dfrac{7}{8}+1}+C=\dfrac{8}{15}x\sqrt[15]x^7+C.	
		\end{eqnarray*}
		}
\end{ex}

\begin{ex}%[2D4H1-4]
	Tính $\displaystyle\int{\dfrac{\sqrt{x}-2\sqrt[3]x^2+1}{\sqrt[4]x}\mathrm{\,d}x}$.
	\choice
	{$x\sqrt[5]x-2x\sqrt[17]x^5+\sqrt[4]x^3+C$}
	{\True $\dfrac{4}{5}x\sqrt[5]x-\dfrac{24}{17}x\sqrt[17]x^5+\dfrac{4}{3}\sqrt[4]x^3+C$}
	{$x\sqrt[5]x-\dfrac{24}{17}x\sqrt[17]x^5+\sqrt[4]x^3+C$}
	{$\dfrac{4}{5}x\sqrt[5]x-2x\sqrt[17]x^5+\dfrac{4}{3}\sqrt[4]x^3+C$}
	\loigiai{
		\begin{eqnarray*}
			\displaystyle\int{\dfrac{\sqrt{x}-2\sqrt[3]x^2+1}{\sqrt[4]x}\mathrm{\,d}x}&=&\displaystyle\int{\dfrac{x^{\frac{1}{2}}-2x^{\frac{2}{3}}+1}{x^{\frac{1}{4}}}\mathrm{\,d}x=}\displaystyle\int{\left(\dfrac{x^{\frac{1}{2}}}{x^{\frac{1}{4}}}-2\dfrac{x^{\frac{2}{3}}}{x^{\frac{1}{4}}}+\dfrac{1}{x^{\frac{1}{4}}}\right)\mathrm{\,d}x}\\
			&=&\displaystyle\int{\left(x^{\frac{1}{4}}-2x^{\frac{5}{12}}+x^{-\frac{1}{4}}\right)\mathrm{\,d}x=\dfrac{4}{5}}x\sqrt[5]x-\dfrac{24}{17}x\sqrt[17]x^5+\dfrac{4}{3}\sqrt[4]x^3+C.
		\end{eqnarray*}
	}
\end{ex}

\begin{ex}%[2D4N1-2]
	Cho hàm số $f(x)=x^2+4$. Mệnh đề nào sau đây đúng?
	
	\choice
	{$\displaystyle{\displaystyle\int f(x)\mathrm{\,d}x=2 x+C}$}
	{$\displaystyle{\displaystyle\int f(x)\mathrm{\,d}x=x^2+4 x+C}$}
	{\True $\displaystyle{\displaystyle\int f(x)\mathrm{\,d}x=\dfrac{x^3}{3}+4 x+C}$}
	{$\displaystyle{\displaystyle\int f(x)\mathrm{\,d}x=x^3+4 x+C}$}
	\loigiai{
		Ta có $f(x)=x^2+4 $ nên $ \displaystyle\int f(x)\mathrm{\,d}x=\dfrac{x^3}{3}+4 x+C$.}
\end{ex}
\begin{ex}%[2D4N1-4]
	Trên khoảng $(0;+\infty)$, cho hàm số $f(x)=x^{\frac{3}{2}}$. Mệnh đề nào sau đây đúng?
	\choice
	{$\displaystyle\int{f(x)}\mathrm{\,d}x=\dfrac{3}{2}x^{\frac{1}{2}}+C$}
	{$\displaystyle\int{f(x)}\mathrm{\,d}x=\displaystyle\int{\sqrt{x^3}}\mathrm{\,d}x$}
	{\True $\displaystyle\int{f(x)}\mathrm{\,d}x=\dfrac{2}{5}x^{\frac{5}{2}}+C$}
	{$\displaystyle\int{f(x)}\mathrm{\,d}x=\dfrac{2}{3}x^{\frac{1}{2}}+C$}
	\loigiai{
		Ta có $\displaystyle\int{f(x)}\mathrm{\,d}x=\displaystyle\int{x^{\frac{3}{2}}}\mathrm{\,d}x=\dfrac{2}{5}x^{\frac{5}{2}}+C$.}
\end{ex}

\begin{ex}%[2D4H1-2]
	Cho hàm số $f(x)=\dfrac{x^4+2}{x^2}$. Mệnh đề nào sau đây đúng?
	\choice
	{$\displaystyle\int{f(x)\mathrm{\,d}x=}\dfrac{x^3}{3}-\dfrac{1}{x}+C$}
	{$\displaystyle\int{f(x)\mathrm{\,d}x=}\dfrac{x^3}{3}+\dfrac{2}{x}+C$}
	{$\displaystyle\int{f(x)\mathrm{\,d}x=}\displaystyle\int{\left(x^2+\dfrac{2}{x^2}\right)}\mathrm{\,d}x$}
	{\True $\displaystyle\int{f(x)\mathrm{\,d}x=}\dfrac{x^3}{3}-\dfrac{2}{x}+C$}
	\loigiai{
		Ta có $\displaystyle\int{f(x)\mathrm{\,d}x=}\displaystyle\int{\dfrac{x^4+2}{x^2}}\mathrm{\,d}x=\displaystyle\int{\left(x^2+\dfrac{2}{x^2}\right)}\mathrm{\,d}x=\dfrac{x^3}{3}-\dfrac{2}{x}+C$.}
\end{ex}

\Closesolutionfile{ans}
\indapan{10}{ans/ans-2-B1-D2-LC}
\TNTF
\Opensolutionfile{ans}[ans/ans-2-B1-D2-DS]
\begin{ex}%[2D4H1-4]
	Các mệnh đề sau đây đúng hay sai
	\choiceTF
	{\True $\displaystyle\int{(\sqrt[3]x^2+x-2)\mathrm{\,d}x}=\dfrac{3}{5}\sqrt[3]x^5+\dfrac{1}{2}x^2-2x+C$}
	{\True $\displaystyle\int{\dfrac{1}{2023x^{2024}}\mathrm{\,d}x}=\dfrac{1}{2023^2x^{2023}}+C$}
	{$\displaystyle\int{(2x-2024)^2\mathrm{\,d}x}=x-1012+C$}
	{\True $\displaystyle\int{\left(\dfrac{1}{4}x^4+4x^3\right)\mathrm{\,d}x}=\dfrac{1}{20}x^5+\dfrac{4}{3}x^4+C$}
	\loigiai{
	$\displaystyle\int{(\sqrt[3]x^2+x-2)\mathrm{\,d}x}=\dfrac{3}{5}\sqrt[3]x^5+\dfrac{1}{2}x^2-2x+C$.\\
	$\displaystyle\int{\dfrac{1}{2023x^{2024}}\mathrm{\,d}x}=\dfrac{1}{2023}\displaystyle\int{x^{-2024}\mathrm{\,d}x}=\dfrac{1}{2023^2x^{2023}}+C$.\\
	$\displaystyle\int{(2x-2024)^2\mathrm{\,d}x}=\dfrac{(2x-2024)^3}{3}+C$.\\
	$\displaystyle\int{\left(\dfrac{1}{4}x^4+4x^3\right)\mathrm{\,d}x}=\dfrac{1}{20}x^5+\dfrac{4}{3}x^4+C$.}
\end{ex}