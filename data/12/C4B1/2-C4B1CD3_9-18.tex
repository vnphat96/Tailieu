\begin{ex}%[2D4V1-2]
	Cho hàm số $ f(x)$ có đạo hàm cấp hai trên đoạn $\left[0;1\right]$ đồng thời thỏa mãn các điều kiện $f'(0)=-1$, $f'(x)<0$, $\left[f'(x)\right]^2=f''(x)$, $\forall x\in\left[0;1\right]$. Giá trị $ f'(2)$ thuộc khoảng
	\choice
	{$(2;3)$}
	{\True $(-2;0)$}
	{$(0;2)$}
	{$(-3;-2)$}
	\loigiai
	{
		Ta có
		\begin{align*}
		\left[f'(x)\right]^2=f''(x) \Leftrightarrow\dfrac{f''(x)}{\left[f'(x)\right]^2}=1\Leftrightarrow-\left(\dfrac{1}{f'(x)}\right)'=1\Rightarrow\dfrac{1}{f'(x)}=-\displaystyle\int{\mathrm{d} x} \Leftrightarrow\dfrac{1}{f'(x)}=-x+C.
		\end{align*}
		Mà $f'(0)=-1\Rightarrow C=-1$ suy ra
		$$\dfrac{1}{f'(x)}=-x-1\Rightarrow f'(x)=-\dfrac{1}{x+1} \Rightarrow{f}'(2)=-\dfrac{1}{3}.$$
		}
\end{ex}

\begin{ex}%[2D4V2-2]
	Cho hàm số $ f(x)$ đồng biến có đạo hàm đến cấp hai trên đoạn $\left[0;2\right]$ và thỏa mãn $\left[f(x)\right]^2-f(x)\cdot f''(x)+\left[f'(x)\right]^2=0$. Biết $ f(0)=1$, $ f(2)={\mathrm{e}}^6$. Khi đó $ f(1)$ bằng
	\choice
	{$\mathrm{e}^{\tfrac{3}{2}}$}
	{$\mathrm{e}^3$}
	{\True $\mathrm{e}^{\tfrac{5}{2}}$}
	{$\mathrm{e}^2$}
	\loigiai
	{
		Theo đề bài, ta có
		\begin{align*}
			\left[f(x)\right]^2-f(x)\cdot f''(x)+\left[f'(x)\right]^2=0 &\Rightarrow\dfrac{f(x)\cdot f''(x)-\left[f'(x)\right]^2}{\left[f(x)\right]^2}=1\\
			&\Rightarrow{\left[\dfrac{f'(x)}{f(x)}\right]'}=1\\& \Rightarrow\dfrac{f'(x)}{f(x)}=x+C\\
			&\Rightarrow\ln f(x)=\dfrac{x^2}{2}+Cx+D.
		\end{align*}
		Mà $\heva{
			& f(0)=1\\ 
			& f(2)=\mathrm{e}^6} \Leftrightarrow\heva{
			& C=2\\ 
			& D=0.}$\\
		Suy ra $f(x)=\mathrm{e}^{\tfrac{x^2}{2}+2x}\Rightarrow f(1)=\mathrm{e}^{\tfrac{5}{2}}$.}
\end{ex}
\begin{ex}%[2D4V1-2]
	Cho hàm số $ f(x)$ thỏa mãn $(f'(x))^2+f(x)\cdot f''(x)=x^3-2x$, $\forall x\in\mathbb{R}$ và \break $ f(0)=f'(0)=1$. Giá trị của $ T=f^2(2)$ bằng
	\choice
	{$\dfrac{43}{30}$}
	{$\dfrac{16}{15}$}
	{\True $\dfrac{43}{15}$}
	{$\dfrac{26}{15}$}
	\loigiai
	{
		Ta có
		\begin{align*}
		\left( f'(x)\right)^2+f(x)\cdot f''(x)=x^3-2x & \Leftrightarrow \left( f(x)\cdot f'(x)\right)'=x^3-2x\\& \Rightarrow f(x)\cdot f'(x)=\displaystyle\int{(x^3-2x})\mathrm{\,d} x\\&\Rightarrow f(x)\cdot f'(x)=\dfrac{1}{4}{x^4}-x^2+C.
		\end{align*}
		Từ $ f(0)=f'(0)=1$ suy ra $C=1$. Do đó $ f(x)\cdot f'(x)=\dfrac{1}{4}{x^4}-x^2+1$.\\
	Lại có
	\begin{align*}
		2f(x)\cdot f'(x)=\dfrac{1}{2}{x^4}-2x^2+2 & \Leftrightarrow \left( f^2(x)\right) '=\dfrac{1}{2}{x^4}-2x^2+2\\& \Rightarrow{f^2}(x)=\displaystyle\int{\left(\dfrac{1}{2}{x^4}-2x^2+2\right)}\mathrm{\,d} x\\& \Rightarrow{f^2}(x)=\dfrac{1}{10}{x^5}-\dfrac{2}{3}{x^3}+2x+C.
	\end{align*}
		Vì $ f(0)=1$ nên $C=1$. Do đó $f^2(x)=\dfrac{1}{10}{x^5}-\dfrac{2}{3}{x^3}+2x+1$.\\
	Vậy $T=\dfrac{43}{15}$.}
\end{ex}

\begin{ex}%[2D4V1-2]
	Cho hàm số $ f(x)$ thỏa mãn $\left[f'(x)\right]^2+f(x)\cdot f''(x)=2x^2-x+1$, $\forall x\in\mathbb{R}$ và $ f(0)=f'(0)=3$. Giá trị của $\left[f(1)\right]^2$ bằng
	\choice
	{\True $ 28$}
	{$ 22$}
	{$\dfrac{19}{2}$}
	{$ 10$}
	\loigiai
	{
		Ta có $\left[f(x){f}'(x)\right]'=\left[f'(x)\right]^2+f(x)\cdot f''(x)$.\\
		Do đó theo giả thiết ta được $\left[f(x){f}'(x)\right]'=2x^2-x+1$.\\
		Suy ra $f(x){f}'(x)=\dfrac{2}{3}{x^3}-\dfrac{x^2}{2}+x+C$.\\
		 Hơn nữa $ f(0)=f'(0)=3$ suy ra $ C=9$.\\
		Tương tự vì $\left[f^2(x)\right]'=2f(x){f}'(x)$ nên $\left[f^2(x)\right]'=2\left(\dfrac{2}{3}{x^3}-\dfrac{x^2}{2}+x+9\right)$.\\
		Suy ra $f^2(x)=\displaystyle\int{2\left(\dfrac{2}{3}{x^3}-\dfrac{x^2}{2}+x+9\right)\mathrm{\,d}x}=\dfrac{1}{3}{x^4}-\dfrac{x^3}{3}+x^2+18x+C$.\\
		Vì $ f(0)=3$ nên $C=9$ suy ra $f^2(x)=\dfrac{1}{3}{x^4}-\dfrac{x^3}{3}+x^2+18x+9$.\\
		Do đó $\left[f(1)\right]^2=28$.
		}
\end{ex}

\Closesolutionfile{ans}
\indapan{10}{ans/ans-LC-2-C4B1CD3_1-8}
\TNSA
\Opensolutionfile{ans}[ans/ans-KQ-2-C4B1CD3]
\begin{ex}%[2D4H1-2]
	Cho hàm số $ y=f(x)$ thỏa mãn $y'=x{y^2}$ và $ f\left(-1\right)=1$. Tính giá trị $f(2)$. (\textit{Kết quả làm tròn đến hàng phần mười}).
	\shortans{$20{,}1$}
	\loigiai{
		Ta có $y'=x{y^2} \Rightarrow\dfrac{y'}{y}=x^2\Rightarrow\displaystyle\int{\dfrac{y'}{y}\mathrm{\,d}x}=\displaystyle\int{x^2\mathrm{\,d}x}\Leftrightarrow\ln y=\dfrac{x^3}{3}+C\Leftrightarrow y=\mathrm{e}^{\tfrac{x^3}{3}+C}$.\\
		Theo giả thiết $ f(-1)=1$ nên $\mathrm{e}^{-\tfrac{1}{3}+C}=1\Leftrightarrow C=\dfrac{1}{3}$.\\
	Do đó	 $ y=f(x)= \mathrm{e}^{\tfrac{x^3}{3}+\tfrac{1}{3}}$.\\
		Vậy $f(2)=\mathrm{e}^3\approx 20{,}1$.}
		\end{ex}
		
\begin{ex}%[2D4V2-2]
			Cho hàm số $ f(x)\ne 0$, liên tục trên đoạn $\left[1;2\right]$ và thỏa mãn $ f(1)=3$, \break $x^2\cdot f'(x)=f^2(x)$ với $\forall x\in\left[1;2\right]$. Tính $f(2)$.
			\shortans{$-6$}
			\loigiai{
				Ta có
				\begin{align*}
				x^2\cdot f'(x)=f^2(x)& \Rightarrow\dfrac{f'(x)}{f^2(x)}=\dfrac{1}{x^2} \Rightarrow{\left(-\dfrac{1}{f(x)}\right)'}=\dfrac{1}{x^2}\\& \Rightarrow\displaystyle\int\limits_1^2\left(-\dfrac{1}{f(x)}\right)'\mathrm{\,d} x=\displaystyle\int\limits_1^2\dfrac{1}{x^2}\mathrm{\,d} x\\&\Rightarrow\left.\left(-\dfrac{1}{f(x)}\right)\right|_1^2=-\left.\dfrac{1}{x}\right|_1^2\\& \Rightarrow-\dfrac{1}{f(2)}+\dfrac{1}{f(1)}=-\dfrac{1}{2}+1\\& \Rightarrow-\dfrac{1}{f(2)}+\dfrac{1}{f(1)}=\dfrac{1}{2}.
				\end{align*}
				Vì $f(1)=3\Rightarrow-\dfrac{1}{f(2)}+\dfrac{1}{3}=\dfrac{1}{2}\Rightarrow f(2)=-6$.}
		\end{ex}
		\begin{ex}%Câu 7%[2D4C1-2]
			Cho hàm số $ y=f(x)$ thỏa mãn $ f(x)<0$, $\forall x>0$ và có đạo hàm $f'(x)$ liên tục trên khoảng $\left( 0;+\infty\right) $ thỏa mãn $f'(x)=(2x+1){f^2}(x)$, $\forall x>0$ và $ f(1)=-\dfrac{1}{2}$. Tính giá trị của biểu thức $ T=f(1)+f(2)+\ldots +f\left(2023\right)+f\left(2024\right)$. (\textit{Kết quả làm tròn đến hàng đơn vị}).
			\shortans{$-1$}
			\loigiai{
				Ta có
			\begin{align*}
				f'(x)=(2x+1){f^2}(x) &\Rightarrow \dfrac{f'(x)}{f^2(x)}=2x+1\\
				&\Rightarrow\displaystyle\int\dfrac{f'(x)}{f^2(x)}\mathrm{\,d}x=\displaystyle\int(2x+1)\mathrm{\,d}x \\&\Rightarrow-\dfrac{1}{f(x)}=x^2+x+C.
				\end{align*}
				Mà $ f(1)=-\dfrac{1}{2}$ $\Rightarrow C=0$ $\Rightarrow f(x)=\dfrac{-1}{x^2+x}$ $=\dfrac{1}{x+1}-\dfrac{1}{x}$.\\
				Ta có $\heva{
					& f(1)=\dfrac{1}{2}-1\\ 
					& f(2)=\dfrac{1}{3}-\dfrac{1}{2}\\ 
					& f(3)=\dfrac{1}{4}-\dfrac{1}{3}\\ 
					&\ldots\\ 
					& f\left(2024\right)=\dfrac{1}{2023}-\dfrac{1}{2024}.}$\\
				$
				\Rightarrow T=f(1)+f(2)+\ldots+f\left(2024\right)=-1+\dfrac{1}{2025}=-\dfrac{2024}{2025}\approx -1$.}
		\end{ex}
		
		
		\begin{ex}%[2D4V1-4]
			Cho hàm số $f(x)$ thỏa mãn $ f(0)=1-\ln 2$ và $\mathrm{e}^ xf'(x)=2^x\left[f(x)\right]^2$ với mọi $x\in\mathbb{R}$. Giá trị của $f(1)$ bằng bao nhiêu? (\textit{Kết quả làm tròn đến hàng phần trăm}).
			\shortans{$0{,}42$}
			\loigiai{
				Từ giả thiết ta có $f'(x)=\dfrac{2^x}{\mathrm{e}^ x}{\left[f(x)\right]^2}$ với mọi $ x\in\left(1;2\right]$.\\
				Do đó $ f(x)\ge f(1)=1>0$ với mọi $ x\in\left[1;2\right]$.\\
				Xét với mọi $ x\in [1 ; 2]$ ta có
				\begin{align*}
					\mathrm{e}^ x{f}'(x)=2^x{\left[f(x)\right]^2}&\Rightarrow{f}'(x)=\dfrac{2^x}{\mathrm{e}^ x}{\left[f(x)\right]^2}\\&\Rightarrow\dfrac{f'(x)}{\left[f(x)\right]^2}=\left(\dfrac{2}{\mathrm{e}}\right)^x\\&\Rightarrow-\left(\dfrac{1}{f(x)}\right)'=\left(\dfrac{2}{\mathrm{e}}\right)^x \\&\Rightarrow{\left(\dfrac{1}{f(x)}\right)'}=-\left(\dfrac{2}{\mathrm{e}}\right)^x\\&\Rightarrow\dfrac{1}{f(x)}=-\displaystyle\int\left(\dfrac{2}{\mathrm{e}}\right)^x\mathrm{\,d} x\\&\Rightarrow\dfrac{1}{f(x)}=-\dfrac{\left(\dfrac{2}{\mathrm{e}}\right)^x}{\ln\dfrac{2}{\mathrm{e}}}+C\\&\Rightarrow\dfrac{1}{f(x)}=\dfrac{\left(\dfrac{2}{\mathrm{e}}\right)^x}{1-\ln 2}+C.
				\end{align*}
				Mà $ f(0)=1-\ln 2\Rightarrow C=0$. \\Do đó
				$\dfrac{1}{f(x)}=\dfrac{\left(\dfrac{2}{\mathrm{e}}\right)^x}{1-\ln 2}$
				$\Rightarrow f(x)=\dfrac{1-\ln 2}{\left(\dfrac{2}{\mathrm{e}}\right)^x}=\dfrac{(1-\ln 2)\mathrm{e}^x}{2^x}$.\\
				Vậy $ f(1)=\dfrac{\mathrm{e}-\mathrm{e}\ln 2}{2}\approx 0{,}42$.}
		\end{ex}
		\begin{ex}%[2D4V1-4]
			Cho hàm số $ y=f(x)$ đồng biến và có đạo hàm liên tục trên $\mathbb{R}$ thỏa mãn $\left(f'(x)\right)^2=f(x)\cdot\mathrm{e}^x$, $\forall x\in\mathbb{R}$ và $f(0)=2$. Tính $ f(2)$. (Kết quả làm tròn đến hàng phần trăm).
			\shortans{$9{,}81$}
			\loigiai{
				Vì hàm số $ y=f(x)$ đồng biến và có đạo hàm liên tục trên $\mathbb{R}$ đồng thời $ f(0)=2$ nên $f'(x)\ge 0$ và $ f(x)>0$ với mọi $ x\in\left[0;+\infty\right)$.\\
				Từ giả thiết $\left(f'(x)\right)^2=f(x)\cdot \mathrm{e}^x$, $\forall x\in\mathbb{R}$ suy ra $f'(x)=\sqrt{f(x)}\cdot\mathrm{e}^{\tfrac{x}{2}}$, $\forall x\in\left[0;+\infty\right).$\\
				Do đó $\dfrac{f'(x)}{2\sqrt{f(x)}}=\dfrac{1}{2}{\mathrm{e}^{\tfrac{x}{2}}}$, $\forall x\in\left[0;+\infty\right).$\\
				Lấy nguyên hàm hai vế, ta được $\sqrt{f(x)}=e^{\tfrac{x}{2}}+C$, $\forall x\in\left[0;+\infty\right)$ với $C$ là hằng số.\\
				Kết hợp với $ f(0)=2$, ta được $C=\sqrt{2}-1$.\\
				Suy ra $ f(2)=\left(\mathrm{e}+\sqrt{2}-1\right)^2\approx 9{,}81$.}
		\end{ex}
		\begin{ex}%[2D4H1-2]
			Giả sử hàm số $ y=f(x)$ liên tục, nhận giá trị dương trên $\left(0;+\infty\right)$ và thỏa mãn $ f(1)=1$, $ f(x)=f'(x)\cdot \sqrt{3x}$, với mọi $x>0$. Tính $f(5)$ \textit{(kết quả làm tròn đến hàng phần trăm}).
			\shortans{$4{,}17$}
			\loigiai{
				Ta có 
				\begin{align*}
				f(x)=f'(x)\cdot\sqrt{3x}&\Rightarrow\dfrac{f'(x)}{f(x)}=\dfrac{1}{\sqrt{3x}}\\&
		\Rightarrow\ln f(x)=\dfrac{1}{\sqrt{3}}\displaystyle\int \dfrac{1}{\sqrt{x}}\mathrm{\,d} x \\ &\Rightarrow\ln f(x)=\dfrac{2}{\sqrt{3}}\sqrt{x}+C\\&\Rightarrow f(x)=e^{\tfrac{2}{\sqrt{3}}\sqrt{x}+C}.
				\end{align*}
				Mà $ f(1)=1$ nên $1=e^{\tfrac{2}{\sqrt{3}}+C}\Rightarrow C=-\dfrac{2}{\sqrt{3}}$
				$\Rightarrow f(x)=e^{\tfrac{2}{\sqrt{3}}\sqrt{x}-\tfrac{2}{\sqrt{3}}}$.\\
				Suy ra $ f(5)=e^{\tfrac{2}{\sqrt{3}}\sqrt{5}-\tfrac{2}{\sqrt{3}}}=e^{\tfrac{2\sqrt{5}-2}{\sqrt{3}}}\approx 4{,}17$.}
		\end{ex}
		
		\begin{ex}%[2D4V1-2]
			Cho hàm số $ f(x)$ có đạo hàm trên $\mathbb{R}$ thỏa mãn $\mathrm{e}^{f(x)}-\dfrac{x}{f'(x)}=0$, $\forall x\in\mathbb{R}$. Biết $f(1)=1$, tính $f\left(\mathrm{e}^2\right)$ (\textit{kết quả làm tròn đến hàng phần trăm}).
			\shortans{$ 3{,}38$}
			\loigiai{
				Ta có 
				\begin{align*}
				\mathrm{e}^{f(x)}-\dfrac{x}{f'(x)}=0&\Rightarrow f'(x)\mathrm{e}^{f(x)}=x\\&\Leftrightarrow \left(\mathrm{e}^{f(x)} \right)'=x\\&\Leftrightarrow
				\mathrm{e}^{f(x)}=\displaystyle\int x\mathrm{\,d} x\\& \Leftrightarrow \mathrm{e}^{f(x)}=\dfrac{x^2}{2}+C.
				\end{align*}
		Mà 	$f(1)=1$ nên $\mathrm{e}=\dfrac{1}{2}+C\Rightarrow C=\mathrm{e}-\dfrac{1}{2}$.\\
		Do đó $\mathrm{e}^{f(x)}=\dfrac{x^2}{2}+ \mathrm{e}-\dfrac{1}{2} \Rightarrow \mathrm{e}^{f\left(\mathrm{e}^2\right)}= \dfrac{\mathrm{e}^4}{2}+ \mathrm{e}-\dfrac{1}{2}\Rightarrow f\left(\mathrm{e}^2\right)=\ln \left(\dfrac{\mathrm{e}^4}{2}+ \mathrm{e}-\dfrac{1}{2}\right) \approx 3{,}38$.}
		\end{ex}
		
		\begin{ex}%[2D4V1-4]
			Cho hàm số $ f(x)$ nhận giá trị dương và thỏa mãn $ f(0)=1$, $\left(f'(x)\right)^3=\mathrm{\mathrm{e}}^ x{\left(f(x)\right)^2}$, $\forall x\in\mathbb{R}$. Tính $ f(3)$ (\textit{kết quả làm tròn đến hàng phần mười}).
			\shortans{$20{,}1$}
			\loigiai{
				Ta có
				 
				 \begin{align*}
				 \left(f'(x)\right)^3=\mathrm{e}^x{\left(f(x)\right)^2},\,\forall x\in\mathbb{R}
				 &\Leftrightarrow{f}'(x)=\sqrt[3]{\mathrm{e}^x}\cdot \sqrt[3]{\left(f(x)\right)^2}\Leftrightarrow\dfrac{f'(x)}{\sqrt[3]{\left(f(x)\right)^2}}=\sqrt[3]{\mathrm{e}^x}\\
				 & \Leftrightarrow\dfrac{f'(x)}{\sqrt[3]{\left(f(x)\right)^2}}=\sqrt[3]{\mathrm{e}^x}\Leftrightarrow{f}'(x)\cdot \left(f(x)\right)^{-\tfrac{2}{3}}=\sqrt[3]{\mathrm{e}^x}\\&\Leftrightarrow 3\left[\left(f(x)\right)^{\tfrac{1}{3}}\right]'=\sqrt[3]{\mathrm{e}^x}\Leftrightarrow{\left[\left(f(x)\right)^{\tfrac{1}{3}}\right]'}=\dfrac{1}{3}\sqrt[3]{\mathrm{e}^x}\\&\Leftrightarrow{\left(f(x)\right)^{\tfrac{1}{3}}}=\dfrac{1}{3}\displaystyle\int{\sqrt[3]{\mathrm{e}^x}}\mathrm{\,d} x \Leftrightarrow{\left(f(x)\right)^{\tfrac{1}{3}}}=e^{\tfrac{x}{3}}+C.
				 \end{align*}
		Vì	$f(0)=1$ nên $1=1+C\Rightarrow C=0\Rightarrow{\left(f(x)\right)^{\tfrac{1}{3}}}=e^{\tfrac{x}{3}}\Rightarrow f(x)=\mathrm{e}^x$.\\
Vậy	$f(3)=e^3\approx 20{,}1$.
}
		\end{ex}
		
		\begin{ex}%Câu 13%[2D4V1-2]
			Cho hàm số $ y=f(x)$ có đạo hàm liên tục trên $\mathbb{R}$ và thỏa mãn điều kiện $x^6\left( f'(x)\right) ^3+27\left[f(x)-1\right]^4=0$, $\forall x\in\mathbb{R}$ và $ f(1)=0$. Tính giá trị của $f(2)$.
			\shortans{$-7$}
			\loigiai{
				Ta có
				\begin{align*}
				x^6\left( f'(x)\right)^3+27\left[f(x)-1\right]^4=0&\Leftrightarrow{x^6}{\left( f'(x)\right)^3}=-27\left( f(x)-1\right)^4\\&\Leftrightarrow\dfrac{\left( f'(x)\right) ^3}{\left( f(x)-1\right)^4}=-\dfrac{27}{x^6}\\
				&\Leftrightarrow\dfrac{\left( f'(x)\right) ^3}{\left( f(x)-1\right) ^3\left( f(x)-1\right)}=-\dfrac{27}{x^6}\\&\Leftrightarrow\dfrac{f'(x)}{\left(f(x)-1\right)\sqrt[3]{f(x)-1}}=-\dfrac{3}{x^2}\\&\Leftrightarrow\dfrac{f'(x)}{-3\left(f(x)-1\right)\sqrt[3]{f(x)-1}}=\dfrac{1}{x^2}\\&\Leftrightarrow{\left[\dfrac{1}{\sqrt[3]{f(x)-1}}\right]'}=\dfrac{1}{x^2}
				\end{align*}
				Do đó $\displaystyle\int{\left( \dfrac{1}{\sqrt[3]{f(x)-1}}\right)'}\mathrm{\,d}x=\displaystyle\int{\dfrac{1}{x^2}\mathrm{\,d}x}=-\dfrac{1}{x}+C.$
				\\
				Suy ra $\dfrac{1}{\sqrt[3]{f(x)-1}}=-\dfrac{1}{x}+C$.\\
				Ta có $ f(1)=0\Rightarrow C=0 \Rightarrow f(x)=1-x^3$.\\
				Khi đó $ f(2)=-7$.}
		\end{ex}
		\begin{ex}%[2D4V1-2]
			Cho hàm số $f(x)$ thỏa mãn $\left[x{f}'(x)\right]^2+1=x^2\left[1-f(x).f''(x)\right]$ với mọi $x$ dương. Biết $f(1)=f'(1)=1$. Tính giá trị $f^2(2)$ (\textit{kết quả làm tròn đến hàng phần trăm}).
			\shortans{$3{,}39$}
			\loigiai{
			Với mọi $x$ dương, ta có 
				\begin{align*}
					\left[x{f}'(x)\right]^2+1=x^2\left[1-f(x)\cdot f''(x)\right]; x>0&\Leftrightarrow{x^2}\cdot\left[f'(x)\right]^2+1=x^2\left[1-f(x)\cdot f''(x)\right]\\
					&\Leftrightarrow{\left[f'(x)\right]^2}+\dfrac{1}{x^2}=1-f(x)\cdot f''(x)\\ 
					&\Leftrightarrow{\left[f'(x)\right]^2}+f(x)\cdot f''(x)=1-\dfrac{1}{x^2}\\ 
					&\Leftrightarrow\left[f(x)\cdot f'(x)\right]'=1-\dfrac{1}{x^2}. 
				\end{align*}
				Do đó $\displaystyle\int\left[f(x)\cdot f'(x)\right]'\mathrm{\, d}x=\displaystyle\int\left(1-\dfrac{1}{x^2}\right)\mathrm{\, d}x\Rightarrow f(x)\cdot f'(x)=x+\dfrac{1}{x}+C.$\\
				Vì $ f(1)=f'(1)=1\Rightarrow 1=2+C\Leftrightarrow C=-1.$\\
				Nên $\displaystyle\int f(x)\cdot f'(x)\mathrm{\, d}x=\displaystyle\int\left(x+\dfrac{1}{x}-1\right) \mathrm{\, d}x$ $\Leftrightarrow\displaystyle\int f(x)\mathrm{\, d}\left(f(x)\right)=\displaystyle\int{\left(x+\dfrac{1}{x}-1\right)}\mathrm{\, d}x$.\\
Suy ra				$\dfrac{f^2(x)}{2}=\dfrac{x^2}{2}+\ln x-x+C.$\\
				Vì $ f(1)=1\Rightarrow\dfrac{1}{2}=\dfrac{1}{2}-1+C\Leftrightarrow C=1.$\\
				Vậy $\dfrac{f^2(x)}{2}=\dfrac{x^2}{2}+\ln x-x+1\Rightarrow{f^2}(2)=2\ln 2+2\approx 3{,}39$.
}
\end{ex}
\Closesolutionfile{ans}
\indapan{6}{ans/ans-KQ-2-C4B1CD3}