\chap{NGUYÊN HÀM VÀ TÍCH PHÂN}
\section{NGUYÊN HÀM}
\subsection{Tóm tắt lý thuyết}
\subsection{Kiến thức cần nắm}
% \subsubsection{ĐỊNH NGHĨA VÀ TÍNH CHẤT}
\subsubsection{Định nghĩa nguyên hàm}
Cho hàm số $f(x)$ xác định trên khoảng $K$. Hàm số $F(x)$ được gọi là nguyên hàm của hàm số $f(x)$ nếu $F'(x)=f(x)$ với mọi $x\in K$.\\
\textbf{Nhận xét:} Nếu $F(x)$ là một nguyên hàm của $f(x)$ thì $F(x)+C$, $(C\in\mathbb{R})$ cũng là nguyên hàm của $f(x)$.\\
Ký hiệu $\displaystyle\int f(x)\mathrm{\,d}x=F(x)+C$.\\
\subsubsection{Một số tính chất của nguyên hàm}
\begin{itemize}
	\item $\left(\displaystyle\int f(x)\mathrm{\,d}x\right)'=f(x)$.
	\item $\displaystyle\int a\cdot f(x)\mathrm{\,d}x=a\cdot\displaystyle\int f(x)\mathrm{\,d}x\quad\left(a\in\mathbb{R}, a\neq 0\right)$.
	\item $\displaystyle\int\left[f(x)\pm g(x)\right]\mathrm{\,d}x=\displaystyle\int f(x)\mathrm{\,d}x\pm\displaystyle\int g(x)\mathrm{\,d}x$.
\end{itemize}
\subsubsection{Một số nguyên hàm cơ bản}
\begin{longtable}{|c|c|}
	\hline
	 Nguyên hàm của hàm số cơ bản & Nguyên hàm mở rộng \\
	\hline
	$\displaystyle\int a\cdot\mathrm{\,d}x=ax+C, a\in\mathbb{R}$ & \\
	\hline
	$\displaystyle\int x^{\alpha}\mathrm{\,d}x=\dfrac{x^{\alpha+1}}{\alpha+1}+C,\alpha\neq-1$ & $\displaystyle\int(ax+b)^{\alpha}\mathrm{\,d}x=\dfrac{1}{a}\cdot\dfrac{(ax+b)^{\alpha+1}}{\alpha+1}+C$ \\
	\hline
	$\displaystyle\int\dfrac{\mathrm{\,d}x}{x}=\ln |x|+C, x\neq 0$ & $\displaystyle\int\dfrac{\mathrm{\,d}x}{ax+b}=\dfrac{1}{a}\cdot\ln |ax+b|+C$ \\
	\hline
	$\displaystyle\int\dfrac{\mathrm{\,d}x}{\sqrt{x}}=2\sqrt{x}+C, x>0$ & $\displaystyle\int\dfrac{\mathrm{\,d}x}{\sqrt{ax+b}}=\dfrac2a\sqrt{ax
	+b}+C, x>0$ \\
	\hline
	$\displaystyle\int\dfrac{\mathrm{\,d}x}{x^2}=-\dfrac{1}{x}+C, x\neq 0$ & $\displaystyle\int\dfrac{\mathrm{\,d}x}{(ax+b)^2}=-\dfrac{1}{a}\cdot \dfrac{1}{ax+b}+C$ \\
	\hline
	$\displaystyle\int\dfrac{\mathrm{\,d}x}{x^{\alpha}}=-\dfrac{1}{(\alpha-1)x^{\alpha-1}}+C$ & $\displaystyle\int\dfrac{\mathrm{\,d}x}{(ax+b)^{\alpha}}=-\dfrac{1}{a}\cdot \dfrac{1}{(\alpha-1)}\cdot (ax+b)^{\alpha-1}+C$ \\
	\hline
	$\displaystyle\int\mathrm{e}^x\mathrm{\,d}x=\mathrm{e}^x+C$ & $\displaystyle\int\mathrm{e}^{ax+b}\mathrm{\,d}x=\dfrac{1}{a}\cdot\mathrm{e}^{ax+b}+C$ \\
	\hline
	$\displaystyle\int a^x\mathrm{\,d}x=\dfrac{a^x}{\ln a}+C$ & $\displaystyle\int a^{\alpha x+\beta}\mathrm{\,d}x=\dfrac{1}{\alpha}\cdot\dfrac{a^{\alpha x+\beta}}{\ln a}+C$ \\
	\hline
	$\displaystyle\int\cos x\mathrm{\,d}x=\sin x+C$ & $\displaystyle\int\cos (ax+b)\mathrm{\,d}x=\dfrac{1}{a}\cdot\sin (ax+b)+C$ \\
	\hline
	$\displaystyle\int\sin x\mathrm{\,d}x=-\cos x+C$ & $\displaystyle\int\sin (ax+b)\mathrm{\,d}x=-\dfrac{1}{a}\cdot\cos (ax+b)+C$ \\
	\hline
	$\displaystyle\int\dfrac{1}{\cos^2x}\mathrm{\,d}x=\tan x+C$ & $\displaystyle\int\dfrac{1}{\cos^2(ax+b)}\mathrm{\,d}x=\dfrac{1}{a}\cdot \tan (ax+b)+C$ \\
	\hline
	$\displaystyle\int\dfrac{1}{\sin^2x}\mathrm{\,d}x=-\cot x+C$ & $\displaystyle\int\dfrac{1}{\sin^2(ax+b)}\mathrm{\,d}x=-\dfrac{1}{a}\cdot \cot(ax+b)+C$ \\
	\hline
\end{longtable}
\textit{\textbf{Nhận xét:} $[F(ax+b)]'=af(ax+b) \Rightarrow \int f(ax+b) \mathrm{\,d}x = \dfrac{1}{a} F(ax+b)+C$}.
\subsection{Phân loại và phương pháp giải bài tập}
\begin{dang}{Sử dụng định nghĩa nguyên hàm và bảng nguyên hàm}
\end{dang}
\subsubsection{Các ví dụ}
\begin{vd}%Câu 1  %[2D3Y1-1]
	Tìm họ nguyên hàm của các hàm số sau
    \begin{listEX}[2]
        \item $f(x)=4x^3+x+5$.
        \item $f(x)=3x^2-2x$.
        \item $f(x)=\dfrac{1}{x^5}+x^2$.
        \item $f(x)=\dfrac{1}{x^3}+x^2-1$.
    \end{listEX}
	\loigiai{
        \begin{listEX}[1]
            \item Ta có $F(x)=\displaystyle\int f(x)\mathrm{\,d}x =\displaystyle\int{(4x^3+x+5)\textrm{ d}x=x^4+\dfrac{x^2}{2}+5x+C}$.
            \item Ta có $F(x)=\displaystyle\int f(x)\mathrm{\,d}x =\displaystyle\int{(3x^2-2x)\textrm{ d}x=x^3-x^2+C}$.
            \item Ta có $F(x)=\displaystyle\int f(x)\mathrm{\,d}x=\displaystyle\int ({x^{-5}}+x^2)\mathrm{\,d}x =-\dfrac{{x^{-4}}}{4}+\dfrac{x^3}{3}+C$.
            \item Ta có $F(x)=\displaystyle\int{f(x)\mathrm{\,d}x}=\displaystyle\int{\left( {x^{-3}}+x^2-1 \right)\mathrm{\,d}x}=-\dfrac{{x^{-2}}}{2}+\dfrac{x^3}{3}-x$.
        \end{listEX}
		}
    \end{vd}
\begin{vd}%Câu 5 %[2D3Y1-1]
	Tính
    \begin{listEX}[3]
        \item $I=\displaystyle\int{(x^2-3x)(x+1)\mathrm{\,d}x}$.
        \item $I=\displaystyle\int{(x-1)(x^2+2)\mathrm{\,d}x}$.
        \item $I=\displaystyle\int{{{(2x+1)}^5}\mathrm{\,d}x}$
        \item $I=\displaystyle\int{{{(2x-10)}^{2020}}\mathrm{\,d}x}$.
        \item $I=\displaystyle\int{\left( 3x^2+\dfrac{1}{x}-2 \right)\mathrm{\,d}x}$.
        \item $I=\displaystyle\int{\left( 3x^2-\dfrac{2}{x}-\dfrac{1}{x^2} \right)\mathrm{\,d}x}$.
        \item $I=\displaystyle\int{\dfrac{x^2-3x+1}{x}\mathrm{\,d}x}$.
        \item $I=\displaystyle\int{\dfrac{2x^2-6x+3}{x}\mathrm{\,d}x}$.
        \item $I=\displaystyle\int{\dfrac{1}{2x-1}\mathrm{\,d}x}$.
        \item $I=\displaystyle\int{\dfrac{2}{3-4x}\mathrm{\,d}x}$.
        \item $I=\displaystyle\int{\dfrac{1}{{{\left( 2x-1 \right)}^2}}\mathrm{\,d}x}$.
        \item $I=\displaystyle\int{\left[ \dfrac{12}{{{\left( x-1 \right)}^2}}+\dfrac{2}{2x-3} \right]\mathrm{\,d}x}$.
        \item $I=\displaystyle\int{\dfrac{3}{4x^2+4x+1}\textrm{ d}x}$.
        \item $I=\displaystyle\int{\dfrac{4}{x^2+6x+9}\textrm{ d}x}$.
            \item (*) $I=\displaystyle\int{\dfrac{2x-1}{{{\left( x+1 \right)}^2}}\textrm{ d}x}$.
    \end{listEX}
	\loigiai{
        \begin{listEX}[1]
            \item Phân phối được: $I=\displaystyle\int{(x^3-2x^2-3x)\mathrm{\,d}x} =\dfrac{x^4}{4}-\dfrac{2}{3}x^3-\dfrac{3}{2}x^2+C$.
            \item Phân phối được: $I=\displaystyle\int{(x^3-x^2+2x-2)\mathrm{\,d}x} =\dfrac{x^4}{4}-\dfrac{x^3}{3}+x^2-2x+C$.
            \item $I=\displaystyle\int{{{(2x+1)}^5}\mathrm{\,d}x}=\dfrac{1}{2}\dfrac{{{(2x+1)}^6}}{6}+C$.
            \item $I=\displaystyle\int{{{(2x-10)}^{2020}}\mathrm{\,d}x}=\dfrac{1}{2}\dfrac{{{(2x-10)}^{2021}}}{2021}+C$.
            \item Ta có $I=\displaystyle\int{\left( 3x^2+\dfrac{1}{x}-2 \right)\mathrm{\,d}x}=x^3+\ln \left| x \right|-2x+C$.
            \item Ta có $I=\displaystyle\int{\left( 3x^2-\dfrac{2}{x}-\dfrac{1}{x^2} \right)\mathrm{\,d}x}=x^3-2\ln \left| x \right|+\dfrac{1}{x}+C$.
            \item Ta có $I=\displaystyle\int{\dfrac{x^2-3x+1}{x}\mathrm{\,d}x}=\displaystyle\int{\left( x-3+\dfrac{1}{x} \right)\mathrm{\,d}x}=x^2-3x+\ln \left| x \right|+C$.
            \item Ta có $I=\displaystyle\int{\dfrac{2x^2-6x+3}{x}\mathrm{\,d}x}=\displaystyle\int{\left( 2x-6+\dfrac{3}{x} \right)\mathrm{\,d}x}=x^2-6x+3\ln \left| x \right|+C$.
            \item Ta có $I=\displaystyle\int{\dfrac{1}{2x-1}\mathrm{\,d}x}=\dfrac{1}{2}\ln \left| 2x-1 \right|+C$.
            \item Ta có $I=\displaystyle\int{\dfrac{2}{3-4x}\mathrm{\,d}x}=2.\dfrac{1}{-4}.\ln \left| 3-4x \right|+C=-\dfrac{1}{2}\ln \left| 3-4x \right|+C$.
            \item $I=\displaystyle\int{\dfrac{1}{{{\left( 2x-1 \right)}^2}}\mathrm{\,d}x=-\dfrac{1}{2}}.\dfrac{1}{2x-1}+C=\dfrac{-1}{4x-2}+C$.
            \item $I=\displaystyle\int{\left[ \dfrac{12}{{{\left( x-1 \right)}^2}}+\dfrac{2}{2x-3} \right]\mathrm{\,d}x=-\dfrac{12}{1}}.\dfrac{1}{x-1}+\dfrac{2}{2}\ln \left| 2x-3 \right|+C=\dfrac{-12}{x-1}+\ln \left| 2x-3 \right|+C$.
            \item $I=\displaystyle\int{\dfrac{1}{4x^2+4x+1}\textrm{ d}x=}\displaystyle\int{\dfrac{1}{{{\left( 2x+1 \right)}^2}}\textrm{ d}x=-\dfrac{1}{2}}.\dfrac{1}{2x+1}+C=\dfrac{-1}{4x+2}+C$.
            \item $I=\displaystyle\int{\dfrac{4}{x^2+6x+9}\textrm{ d}x=}\displaystyle\int{\dfrac{4}{{{\left( x+3 \right)}^2}}\textrm{ d}x=-\dfrac{4}{1}}.\dfrac{1}{x+3}+C=\dfrac{-4}{x+3}+C$.
            \item $I=\displaystyle\int{\dfrac{2x+2-3}{{{\left( x+1 \right)}^2}}\textrm{ d}x=\displaystyle\int{\left[ \dfrac{2(x+1)}{{{\left( x+1 \right)}^2}}-\dfrac{3}{{{\left( x+1 \right)}^2}} \right]}}\textrm{ d}x=\displaystyle\int{\dfrac{2}{x+1}\textrm{ d}x-\displaystyle\int{\dfrac{3}{{{\left( x+1 \right)}^2}}\textrm{ d}x}}$.\\
            $I=2\ln \left| x+1 \right|-\dfrac{-3}{x+1}+C=2\ln \left| x+1 \right|+\dfrac{3}{x+1}+C$.
            \item $I=\displaystyle\int{\dfrac{2x-2}{{{\left( 2x+1 \right)}^2}}\textrm{ d}x=\displaystyle\int{\left[ \dfrac{2x+1}{{{\left( 2x+1 \right)}^2}}-\dfrac{3}{{{\left( 2x+1 \right)}^2}} \right]}}\textrm{ d}x = \displaystyle\int{\dfrac{1}{2x+1}\textrm{ d}x-\displaystyle\int{\dfrac{3}{{{\left( 2x+1 \right)}^2}}\textrm{ d}x}}$.\\
            $I=\dfrac{1}{2}\ln \left| 2x+1 \right|-\dfrac{-3}{2\left( 2x+1 \right)}+C \Rightarrow I=\dfrac{1}{2}\ln \left| 2x+1 \right|+\dfrac{3}{2\left( 2x+1 \right)}+C$.
        \end{listEX}
		}
\end{vd}
\begin{vd}%[2D3B1-1]%BT3.
    Tìm họ nguyên hàm của các hàm số sau
    \begin{listEX}[3]
        \item $I=\displaystyle\int(\sin x-\cos x) \mathrm{\,d}x$.
        \item $I=\displaystyle\int (3 \cos x-2 \sin x) \mathrm{\,d}x$.
        \item $I=\displaystyle\int (2 \sin 2x-3 \cos 6x) \mathrm{\,d}x$.
        \item $I=\displaystyle\int \sin x \cos x \mathrm{\,d}x$.
        \item $I=\displaystyle\int \cos \left(\dfrac{x}{2}+\dfrac{\pi}{6}\right)\mathrm{\,d}x$.
        \item $I=\displaystyle\int \sin \left(\dfrac{\pi}{3}-\dfrac{x}{3}\right)\mathrm{\,d}x$.
        \item $I=\displaystyle\int (\sin x-\cos x)^2 \mathrm{\,d}x$.
        \item $I=\displaystyle\int (\cos x+\sin x)^2 \mathrm{\,d}x$.
        %    \item $I=\displaystyle\int \left(\cos ^2x-\sin ^2x\right) \mathrm{\,d}x$.
        %    \item $I=\displaystyle\int \left(\cos ^{4}x-\sin ^{4}x\right) \mathrm{\,d}x$.
    \end{listEX}
    \loigiai{
        \begin{listEX}[1]
            \item $I=\displaystyle\int(\sin x-\cos x) \mathrm{\,d}x=-\cos x-\sin x +C$.
            \item $I=\displaystyle\int (3 \cos x-2 \sin x) \mathrm{\,d}x=3\sin x + 2\cos x+C $.
            \item $I=\displaystyle\int (2 \sin 2x-3 \cos 6x) \mathrm{\,d}x=-\cos 2x -\dfrac{1}{2} \sin 6x+C$.
            \item $I=\dfrac{1}{2}\displaystyle\int \sin 2x \mathrm{\,d}x=-\dfrac{1}{4}\cos 2x+C$.
            \item $I=\displaystyle\int \cos \left(\dfrac{x}{2}+\dfrac{\pi}{6}\right)\mathrm{\,d}x=\displaystyle\int \left(\dfrac{\sqrt{3}}{2}\cos\dfrac{x}{2} -\dfrac{1}{2}\sin \dfrac{x}{2}\right) \mathrm{\,d}x = \sqrt{3}\sin \dfrac{x}{2}+\cos \dfrac{x}{2}+C$.
            \item $I=\displaystyle\int \sin \left(\dfrac{\pi}{3}-\dfrac{x}{3}\right)\mathrm{\,d}x= \displaystyle\int  \left(\dfrac{\sqrt{3}}{2}\cos \dfrac{x}{3}-\dfrac{1}{2}\sin \dfrac{x}{3} \right)\mathrm{\,d}x =\dfrac{3\sqrt{3}}{2}\sin \dfrac{x}{3}+\dfrac{3}{2}\cos\dfrac{x}{3}+C$.
            \item $I=\displaystyle\int (\sin x-\cos x)^2 \mathrm{\,d}x=\displaystyle\int (1-\sin 2x)\mathrm{\,d}x=x+\dfrac{1}{2}\cos 2x+C$.
            \item $I=\displaystyle\int (\cos x+\sin x)^2 \mathrm{\,d}x=\displaystyle\int(1+\sin 2x)\mathrm{\,d}x=x-\dfrac{1}{2}\cos 2x+C$.
            \item $I=\displaystyle\int \left(\cos ^2x-\sin ^2x\right) \mathrm{\,d}x= \displaystyle\int \cos 2x \mathrm{\,d}x=\dfrac{1}{2}\sin 2x+C$.
            \item $I=\displaystyle\int \left(\cos ^{4}x-\sin ^{4}x\right) \mathrm{\,d}x=\displaystyle\int \left(\cos ^2x-\sin ^2x\right) \mathrm{\,d}x= \displaystyle\int \cos 2x \mathrm{\,d}x=\dfrac{1}{2}\sin 2x+C$.
        \end{listEX}
    }
\end{vd}

\begin{vd} %[2D3B1-1]
    Tìm họ nguyên hàm của các hàm số sau
    \begin{listEX}[3]
        \item $I=\displaystyle\int \dfrac{1}{\sin ^2x} \mathrm{\,d}x$.
        \item $I=\displaystyle\int \dfrac{6}{\cos ^2 3x} \mathrm{\,d}x$.
        \item $I=\displaystyle\int (\tan x+\cot x)^2 \mathrm{\,d}x$.
        \item $I=\displaystyle\int \sin ^2x \mathrm{\,d}x$.
        \item $I=\displaystyle\int \cos ^2 2x \mathrm{\,d}x$.
        \item $I=\displaystyle\int \sin 4x \cos x \mathrm{\,d}x$.
        \item $I=\displaystyle\int \dfrac{1}{\sin x \cos x} \mathrm{\,d}x$.
    \end{listEX}

    \loigiai{
        \begin{listEX}[1]
            \item $I=\displaystyle\int\left( \dfrac{1}{\cos ^2x}-\dfrac{1}{\sin ^2x}\right) \mathrm{\,d}x=\tan x+\cot x +C$.
            \item $I=\displaystyle\int \dfrac{6}{\cos ^2 3x} \mathrm{\,d}x=2\tan 3x+C$.
            \item $I=\displaystyle\int (\tan x+\cot x)^2 \mathrm{\,d}x=\displaystyle\int (\tan^2 x+\cot^2x+2) \mathrm{\,d}x            =\displaystyle\int (\tan^2 x+1+\cot^2x+1) \mathrm{\,d}x=\tan x-\cot x+C$.
			\item $I=\displaystyle\int \sin ^2x \mathrm{\,d}x = \displaystyle\int \dfrac{1-\cos 2x}{2} \mathrm{\,d}x = \dfrac{1}{2}x-\dfrac{1}{4}\sin 2x+C$.
			\item $I=\displaystyle\int \cos ^2 2x \mathrm{\,d}x = \displaystyle\int \dfrac{1+\cos 4x}{2} \mathrm{\,d}x = \dfrac{1}{2}x+\dfrac{1}{8}\sin 4x+C$.
        \end{listEX}
    }
\end{vd}
\begin{vd} %[2D3B1-1]
    Tìm họ nguyên hàm của các hàm số sau
    \begin{listEX}[3]
        \item $I=\displaystyle\int \mathrm{e} ^{2x} \mathrm{\,d}x$.
        \item $I=\displaystyle\int \mathrm{e}^{1-2x} \mathrm{\,d}x$.
        \item $I=\displaystyle\int \left(2x-\mathrm{e}^{-x}\right) \mathrm{\,d}x$.
        \item $I=\displaystyle\int \mathrm{e}^x\left(1-3 \mathrm{e}^{-2x}\right) \mathrm{\,d}x$.
        \item $I=\displaystyle\int \left(3-\mathrm{e}^x\right)^2 \mathrm{\,d}x$.
        \item $I=\displaystyle\int \left(2+\mathrm{e}^{3x}\right)^2 \mathrm{\,d}x$.
        \item $I=\displaystyle\int 2^{2x+1} \mathrm{\,d}x$.
        \item $I=\displaystyle\int 4^{1-2x} \mathrm{\,d}x$.
        \item $I=\displaystyle\int 3^x \cdot 5^x \mathrm{\,d}x$.
        \item $I=\displaystyle\int 4^x \cdot 3^{x-1} \mathrm{\,d}x$.
        \item $I=\displaystyle\int \dfrac{\mathrm{\,d}x}{\mathrm{e}^{2-5x}}$.
        \item $I=\displaystyle\int \dfrac{\mathrm{\,d}x}{2^{3-2x}}$.
        \item $I=\displaystyle\int \dfrac{4^{x+1} \cdot 3^{x-1}}{2^x} \mathrm{\,d}x$.
        \item $I=\displaystyle\int \dfrac{4^{2x-1} \cdot 6^{x-1}}{3^x} \mathrm{\,d}x$.
    \end{listEX}
    \loigiai{
        \begin{listEX}[1]
            \item Ta có $I=\displaystyle\int \mathrm{e} ^{2x} \mathrm{\,d}x=\dfrac{1}{2} \mathrm{e}^{2x}+C$.
            \item Ta có $I=\displaystyle\int \mathrm{e}^{1-2x} \mathrm{\,d}x=-\dfrac{1}{2}\mathrm{e}^{1-2x}+C$.
            \item $I=\displaystyle\int \left(2x-\mathrm{e}^{-x}\right) \mathrm{\,d}x=x^2+\mathrm{e}^{-x}+C$.
            \item Ta có $I=\displaystyle\int \mathrm{e}^x\left(1-3 \mathrm{e}^{-2x}\right) \mathrm{\,d}x=\displaystyle\int \left(e^x-3e^{-x}\right) \mathrm{\,d}x=e^x+3e^{-x}+C$.
            \item $I=\displaystyle\int \left(3-\mathrm{e}^x\right)^2 \mathrm{\,d}x=\displaystyle\int\left( 9-6\mathrm{e}^x+\mathrm{e}^{2x}\right) \mathrm{\,d}x=9x-6\mathrm{e}^x+\dfrac{1}{2}\mathrm{e}^{2x}+C$.
            \item Ta có $I=\displaystyle\int \left(2+\mathrm{e}^{3x}\right)^2 \mathrm{\,d}x= \displaystyle\int \left(4+4\mathrm{e}^{3x}+ \mathrm{e}^{6x}\right) \mathrm{\,d}x=4x+\dfrac{4}{3}\mathrm{e}^{3x}+\dfrac{1}{6}\mathrm{e}^{6x}+C$.
            \item Ta có $I=\displaystyle\int 2^{2x+1} \mathrm{\,d}x=\dfrac{2^{2x+1}}{2\ln 2}+C$.
            \item Ta có $I=\displaystyle\int 4^{1-2x} \mathrm{\,d}x=-\dfrac{4^{1-2x}}{2\ln 4}+C$.
            \item Ta có $I=\displaystyle\int 15^x \mathrm{\,d}x = \dfrac{15^x}{\ln 15}+C$.
            \item Ta có $I=\dfrac{1}{3}\displaystyle\int 12^x \mathrm{\,d}x=\dfrac{12^x}{3\ln 12}+C$.
            \item $I=\displaystyle\int \mathrm{e}^{5x-2}\mathrm{\,d}x=\dfrac{\mathrm{e}^{5x-2}}{5}+C$.
            \item Ta có $I=\displaystyle\int 2^{2x-3} \mathrm{\,d}x =\dfrac{ 2^{2x-3}}{2\ln 2}+C$.
            \item Ta có $I=\displaystyle\int \dfrac{4^{x+1} \cdot 3^{x-1}}{2^x} \mathrm{\,d}x=\dfrac{4}{3}\displaystyle\int 6^x\mathrm{\,d}x= \dfrac{4\cdot 6^x}{3\cdot \ln 6}+C$.
            \item Ta có $I=\displaystyle\int \dfrac{4^{2x-1} \cdot 6^{x-1}}{3^x} \mathrm{\,d}x=\dfrac{1}{24}\displaystyle\int 32^x \mathrm{\,d}x=\dfrac{32^x}{24\ln 32}+C=\dfrac{2^{5x}}{120\ln 2}+C$.
        \end{listEX}
    }
\end{vd}
\subsubsection{Câu hỏi trắc nghiệm}
% \TN
\Opensolutionfile{ans}[ans/ans-2-B1-D2-LC]
\begin{ex}%[2D4N1-1]
	Cho hàm số $F(x)$ là một nguyên hàm của hàm số $f(x)$ trên $K$. Các mệnh đề sau, mệnh đề nào \textbf{sai}.
	\choice
	{$\displaystyle\int{f(x)\mathrm{\,d}x=}F(x)+C$}
	{$\displaystyle{\left(\displaystyle\int{f(x)\mathrm{\,d}x}\right)'}=f(x)$}
	{\True $\displaystyle{\left(\displaystyle\int{f(x)\mathrm{\,d}x}\right)'}=f'(x)$}
	{$\displaystyle{\left(\displaystyle\int{f(x)\mathrm{\,d}x}\right)'}=F'(x)$}
	\loigiai{
		Ta có $\displaystyle\int{f(x)\mathrm{\,d}x=}F(x)+C\Leftrightarrow F'(x)=f(x)$ nên phương án $\left(\displaystyle\int{f(x)\mathrm{\,d}x}\right)'=f'(x)$ sai.}
\end{ex}

\begin{ex}%[2D4N1-2]
	Họ tất cả các nguyên hàm của hàm số $f(x)=2x+6$ là
	\choice
	{$x^2+C$}
	{\True $x^2+6x+C$}
	{$2x^2+C$}
	{$2x^2+6x+C$}
	\loigiai{
		$\displaystyle\int{(2x+6)\mathrm{\,d}x=x^2+6x+C}$.}
\end{ex}

\begin{ex}%[2D4N1-2]
	$\displaystyle\int{x^2\mathrm{\,d}x}$ bằng
	\choice
	{$2x+C$}
	{\True $\dfrac{1}{3}x^3+C$}
	{$x^3+C$}
	{$3x^3+C$}
	\loigiai{
		Ta có $\displaystyle\int{x^2\mathrm{\,d}x}=\dfrac{1}{3}x^3+C$.}
\end{ex}

\begin{ex}%[2D4N1-2]
	Họ nguyên hàm của hàm số $f(x)=3x^2+1$ là
	\choice
	{$x^3+C$}
	{$\dfrac{x^3}{3}+x+C$}
	{$6x+C$}
	{\True $x^3+x+C$}
	\loigiai{
		$\displaystyle\int{(3x^2+1)\mathrm{\,d}x=x^3+x+C}$.}
\end{ex}

\begin{ex}%[2D4N1-2]
	Nguyên hàm của hàm số $f(x)=x^3+x$ là
	\choice
	{\True $\dfrac{1}{4}x^4+\dfrac{1}{2}x^2+C$}
	{$3x^2+1+C$}
	{$x^3+x+C$}
	{$x^4+x^2+C$}
	\loigiai{
		$\displaystyle\int{(x^3+x^2)\mathrm{\,d}x}=\dfrac{1}{4}x^4+\dfrac{1}{2}x^2+C$.}
\end{ex}

\begin{ex}%[2D4N1-2]
	Nguyên hàm của hàm số $f(x)=x^4+x^2$ là
	\choice
	{\True $\dfrac{1}{5}x^5+\dfrac{1}{3}x^3+C$}
	{$x^4+x^2+C$}
	{$x^5+x^3+C$}
	{$4x^3+2x+C$}
	\loigiai{
		$\displaystyle\int{f(x)\mathrm{\,d}x}=\displaystyle\int{(x^4+x^2)\mathrm{\,d}x}$ $=\dfrac{1}{5}x^5+\dfrac{1}{3}x^3+C$.}
\end{ex}

\begin{ex}%[2D4H1-2]
	Hàm số nào trong các hàm số sau đây không là nguyên hàm của hàm số $y=x^{2022}$?
	\choice
	{$\dfrac{x^{2023}}{2023}+1$}
	{$\dfrac{x^{2023}}{2023}$}
	{\True $y=2022x^{2021}$}
	{$\dfrac{x^{2023}}{2023}-1$}
	\loigiai{
		Ta có $\displaystyle\int{x^{2022}\mathrm{\,d}}x=\dfrac{x^{2023}}{2023}+C$, $C$ là hằng số nên $y=2022x^{2021}$ không là nguyên hàm của hàm số $y=x^{2022}$.}
\end{ex}

\begin{ex}%[2D4H1-2]
	Nguyên hàm của hàm số $f(x)=$ $\dfrac{1}{3}x^3-2x^2+x-2024$ là
	\choice
	{$\dfrac{1}{12}x^4-\dfrac{2}{3}x^3+\dfrac{x^2}{2}+C$}
	{$\dfrac{1}{9}x^4-\dfrac{2}{3}x^3+\dfrac{x^2}{2}-2024x+C$}
	{\True $\dfrac{1}{12}x^4-\dfrac{2}{3}x^3+\dfrac{x^2}{2}-2024x+C$}
	{$\dfrac{1}{9}x^4+\dfrac{2}{3}x^3-\dfrac{x^2}{2}-2024x+C$}
	\loigiai{
		Sử dụng công thức $\displaystyle\int{x^n\mathrm{\,d}x=\dfrac{x^{n+1}}{n+1}+C}$ ta được
		\begin{eqnarray*}
		\displaystyle\int\left(\dfrac{1}{3}x^3-2x^2+x-2024\right)\mathrm{\,d}x&=&\dfrac{1}{3}\cdot \dfrac{x^4}{4}-2\cdot \dfrac{x^3}{3}+\dfrac{x^2}{2}-2024x+C\\&=&\dfrac{1}{12}x^4-\dfrac{2}{3}x^3+\dfrac{1}{2}x^2-2024x+C.
		\end{eqnarray*}
		}
\end{ex}

\begin{ex}%[2D4H1-2]
	Tìm nguyên $F(x)$ của hàm số $f(x)=(x+1)(x+2)(x+3)?$
	\choice
	{$F(x)=\dfrac{x^4}{4}-6x^3+\dfrac{11}{2}x^2-6x+C$}
	{$F(x)=x^4+6x^3+11x^2+6x+C$}
	{\True $F(x)=\dfrac{x^4}{4}+2x^3+\dfrac{11}{2}x^2+6x+C$}
	{$F(x)=x^3+6x^2+11x^2+6x+C$}
	\loigiai{
		Ta có $f(x)=(x+1)(x+2)(x+3)=x^3+6x^2+11x+6$ nên\\
		$\displaystyle F(x)=\displaystyle\int{(x^3+6x^2+11x+6)}\mathrm{\,d}x=\dfrac{x^4}{4}+2x^3+\dfrac{11}{2}x^2+6x+C$.}
\end{ex}

\begin{ex}%[2D4H1-2]
	Tìm nguyên hàm của hàm số $f(x)=(5x+3)^5$.
	\choice
	{$(5x+3)^6+C$}
	{$(5x+3)^4+C$}
	{\True $\dfrac{(5x+3)^6}{30}+C$}
	{$\dfrac{(5x+3)^4}{30}+C$}
	\loigiai{
		$f(x)=(5x+3)^5$ $\displaystyle \Rightarrow \displaystyle\int{f(x)\mathrm{\,d}x=}\displaystyle\int{(5x+ 3)^5\mathrm{\,d}x=}\dfrac{1}{5}\cdot \dfrac{(5x+3)^6}{6}+C=\dfrac{(5x+3)^6}{30}+C$.}
\end{ex}

\begin{ex}%[2D4H1-2]
	Tìm nguyên hàm của hàm số $f(x)=x^2+\dfrac{2}{x^2}$.
	\choice
	{\True $\displaystyle\int{f(x)\mathrm{\,d}x}=\dfrac{x^3}{3}+\dfrac{1}{x}+C$}
	{$\displaystyle\int{f(x)\mathrm{\,d}x}=\dfrac{x^3}{3}-\dfrac{2}{x}+C$}
	{$\displaystyle\int{f(x)\mathrm{\,d}x}=\dfrac{x^3}{3}-\dfrac{1}{x}+C$}
	{$\displaystyle\int{f(x)\mathrm{\,d}x}=\dfrac{x^3}{3}+\dfrac{2}{x}+C$}
	\loigiai{
		Ta có $\displaystyle\int{\left(x^2+\dfrac{2}{x^2}\right)\mathrm{\,d}x}=\dfrac{x^3}{3}-\dfrac{2}{x}+C$.}
\end{ex}

\begin{ex}%[2D4H1-4]
	Tính $\displaystyle\int{\sqrt{x\sqrt{x\sqrt{x}}}\mathrm{\,d}x}$.
	\choice
	{$\dfrac{4}{15}x\sqrt[8]x^7+C$}
	{\True $\dfrac{8}{15}x\sqrt[8]x^7+C$}
	{$\dfrac{8}{15}x\sqrt[8]x+C$}
	{$\dfrac{4}{15}x\sqrt[8]x+C$}
	\loigiai{
		\begin{eqnarray*}
		\displaystyle\int{\sqrt{x\sqrt{x\sqrt{x}}}\mathrm{\,d}x}&=&\displaystyle\int{\sqrt{x\sqrt{x\cdot{x^{\frac{1}{2}}}}}\mathrm{\,d}x}=\displaystyle\int{\sqrt{x\cdot{x^{\frac{3}{4}}}}\mathrm{\,d}x}=\displaystyle\int{x^{\frac{7}{8}}\mathrm{\,d}x}\\&=&\dfrac{x^{\frac{7}{8}+1}}{\dfrac{7}{8}+1}+C=\dfrac{8}{15}x\sqrt[8]x^7+C.	
		\end{eqnarray*}
		}
\end{ex}

\begin{ex}%[2D4H1-4]
	Tính $\displaystyle\int{\dfrac{\sqrt{x}-2\sqrt[3]x^2+1}{\sqrt[4]x}\mathrm{\,d}x}$.
	\choice
	{$x\sqrt[5]x-2x\sqrt[12]x^5+\sqrt[4]x^3+C$}
	{\True $\dfrac{4}{5}x\sqrt[4]x-\dfrac{24}{17}x\sqrt[12]x^5+\dfrac{4}{3}\sqrt[4]x^3+C$}
	{$x\sqrt[5]x-\dfrac{24}{17}x\sqrt[12]x^5+\sqrt[4]x^3+C$}
	{$\dfrac{4}{5}x\sqrt[5]x-2x\sqrt[12]x^5+\dfrac{4}{3}\sqrt[4]x^3+C$}
	\loigiai{
		\begin{eqnarray*}
			\displaystyle\int{\dfrac{\sqrt{x}-2\sqrt[3]x^2+1}{\sqrt[4]x}\mathrm{\,d}x}&=&\displaystyle\int{\dfrac{x^{\frac{1}{2}}-2x^{\frac{2}{3}}+1}{x^{\frac{1}{4}}}\mathrm{\,d}x=}\displaystyle\int{\left(\dfrac{x^{\frac{1}{2}}}{x^{\frac{1}{4}}}-2\dfrac{x^{\frac{2}{3}}}{x^{\frac{1}{4}}}+\dfrac{1}{x^{\frac{1}{4}}}\right)\mathrm{\,d}x}\\
			&=&\displaystyle\int\left(x^{\frac{1}{4}}-2x^{\frac{5}{12}}+x^{-\frac{1}{4}}\right)\mathrm{\,d}x
			=\dfrac{4}{5}x\sqrt[4]x-\dfrac{24}{17}x\sqrt[12]x^5+\dfrac{4}{3}\sqrt[4]x^3+C.
		\end{eqnarray*}
	}
\end{ex}

\begin{ex}%[2D4N1-2]
	Cho hàm số $f(x)=x^2+4$. Mệnh đề nào sau đây đúng?
	
	\choice
	{$\displaystyle{\displaystyle\int f(x)\mathrm{\,d}x=2 x+C}$}
	{$\displaystyle{\displaystyle\int f(x)\mathrm{\,d}x=x^2+4 x+C}$}
	{\True $\displaystyle{\displaystyle\int f(x)\mathrm{\,d}x=\dfrac{x^3}{3}+4 x+C}$}
	{$\displaystyle{\displaystyle\int f(x)\mathrm{\,d}x=x^3+4 x+C}$}
	\loigiai{
		Ta có $f(x)=x^2+4 $ nên $ \displaystyle\int f(x)\mathrm{\,d}x=\dfrac{x^3}{3}+4 x+C$.}
\end{ex}
\begin{ex}%[2D4N1-4]
	Trên khoảng $(0;+\infty)$, cho hàm số $f(x)=x^{\frac{3}{2}}$. Mệnh đề nào sau đây đúng?
	\choice
	{$\displaystyle\int{f(x)}\mathrm{\,d}x=\dfrac{3}{2}x^{\frac{1}{2}}+C$}
	{$\displaystyle\int{f(x)}\mathrm{\,d}x=\displaystyle\int{\sqrt{x^3}}\mathrm{\,d}x$}
	{\True $\displaystyle\int{f(x)}\mathrm{\,d}x=\dfrac{2}{5}x^{\frac{5}{2}}+C$}
	{$\displaystyle\int{f(x)}\mathrm{\,d}x=\dfrac{2}{3}x^{\frac{1}{2}}+C$}
	\loigiai{
		Ta có $\displaystyle\int{f(x)}\mathrm{\,d}x=\displaystyle\int{x^{\frac{3}{2}}}\mathrm{\,d}x=\dfrac{2}{5}x^{\frac{5}{2}}+C$.}
\end{ex}

\begin{ex}%[2D4H1-2]
	Cho hàm số $f(x)=\dfrac{x^4+2}{x^2}$. Mệnh đề nào sau đây đúng?
	\choice
	{$\displaystyle\int{f(x)\mathrm{\,d}x=}\dfrac{x^3}{3}-\dfrac{1}{x}+C$}
	{$\displaystyle\int{f(x)\mathrm{\,d}x=}\dfrac{x^3}{3}+\dfrac{2}{x}+C$}
	{$\displaystyle\int{f(x)\mathrm{\,d}x=}\displaystyle\int{\left(x^2+\dfrac{2}{x^2}\right)}\mathrm{\,d}x$}
	{\True $\displaystyle\int{f(x)\mathrm{\,d}x=}\dfrac{x^3}{3}-\dfrac{2}{x}+C$}
	\loigiai{
		Ta có $\displaystyle\int{f(x)\mathrm{\,d}x=}\displaystyle\int{\dfrac{x^4+2}{x^2}}\mathrm{\,d}x=\displaystyle\int{\left(x^2+\dfrac{2}{x^2}\right)}\mathrm{\,d}x=\dfrac{x^3}{3}-\dfrac{2}{x}+C$.}
\end{ex}

\Closesolutionfile{ans}
\indapan{10}{ans/ans-2-B1-D2-LC}
% \TNTF
\Opensolutionfile{ans}[ans/ans-2-B1-D2-DS]
\begin{ex}%[2D4H1-4]
	Các mệnh đề sau đây đúng hay sai
	\choiceTF
	{\True $\displaystyle\int{(\sqrt[3]x^2+x-2)\mathrm{\,d}x}=\dfrac{3}{5}\sqrt[3]x^5+\dfrac{1}{2}x^2-2x+C$}
	{\True $\displaystyle\int{\dfrac{1}{2023x^{2024}}\mathrm{\,d}x}=\dfrac{1}{2023^2x^{2023}}+C$}
	{$\displaystyle\int{(2x-2024)^2\mathrm{\,d}x}=x-1012+C$}
	{\True $\displaystyle\int{\left(\dfrac{1}{4}x^4+4x^3\right)\mathrm{\,d}x}=\dfrac{1}{20}x^5+\dfrac{4}{3}x^4+C$}
	\loigiai{
	$\displaystyle\int{(\sqrt[3]x^2+x-2)\mathrm{\,d}x}=\dfrac{3}{5}\sqrt[3]x^5+\dfrac{1}{2}x^2-2x+C$.\\
	$\displaystyle\int{\dfrac{1}{2023x^{2024}}\mathrm{\,d}x}=\dfrac{1}{2023}\displaystyle\int{x^{-2024}\mathrm{\,d}x}=\dfrac{1}{2023^2x^{2023}}+C$.\\
	$\displaystyle\int{(2x-2024)^2\mathrm{\,d}x}=\dfrac{(2x-2024)^3}{3}+C$.\\
	$\displaystyle\int{\left(\dfrac{1}{4}x^4+4x^3\right)\mathrm{\,d}x}=\dfrac{1}{20}x^5+\dfrac{4}{3}x^4+C$.}
\end{ex}
\begin{ex}%[2D4H1-2][Lê Công Trường]
	Cho	các mệnh đề sau đây 
	\choiceTF
	{\True $F(x)=\dfrac{x^4}{4}-\dfrac{3}{2}{x^2}+\ln \left| x\right|+C$ là nguyên hàm của hàm số $f(x)=x^3-3x+\dfrac{1}{x}$}
	{$F(x)=\dfrac{(5x+3)^6}{6}+C$ là nguyên hàm của hàm số $f(x)=\left(5x+3\right)^5$}
	{$F(x)=\dfrac{3}{2}x\sqrt x+\dfrac{4}{3}x\sqrt[3]{x}+\dfrac{5}{4}x\sqrt[4]{x}+C$ là nguyên hàm của hàm số $f(x)=\sqrt x+\sqrt[3]{x}+\sqrt[4]{x}$}
	{\True $F(x)=\dfrac{1}{3}{x^3}-2024x+C$ là nguyên hàm của hàm số $f(x)=\dfrac{x^3-2024x}{x}$}
	\loigiai{
		\begin{itemchoice}
			\itemch {\bf Đúng}. Vì $f(x)=x^3-3x+\dfrac{1}{x}$\\
			$\Rightarrow F(x)=\displaystyle\int f(x)dx=\displaystyle\int{(x^3-3x{\rm}+\dfrac{1}{x})dx}$\\
			$=\displaystyle\int{x^3dx}-3\displaystyle\int{xdx}+\displaystyle\int{\dfrac{1}{x}dx}=\dfrac{x^4}{4}-\dfrac{3}{2}{x^2}+\ln \left| x\right|+C$.
			\itemch {\bf Sai.} Vì $f(x)=\left(5x+3\right)^5$ \\
			$\Rightarrow F(x)=\displaystyle\int{f(x)dx=}\displaystyle\int(5x+3)^{5}dx$\\
			$=\displaystyle\int{\rm{(5x+3)}^{\rm{5}}\dfrac{d(5x+3)}{5}=\dfrac{(5x+3)^6}{30}+C}$.
			\itemch {\bf Sai.} Vì $f(x)=\sqrt x+\sqrt[3]{x}+\sqrt[4]{x}$\\
			$\Rightarrow F(x)=\displaystyle\int{\left(\sqrt x+\sqrt[3]{x}+\sqrt[4]{x}\right)}dx=\displaystyle\int{\left(x^{\frac{1}{2}}+x^{\frac{1}{3}}+x^{\frac{1}{4}}\right)}dx$\\
			$=\dfrac{2}{3}{x^{\frac{3}{2}}}+\dfrac{3}{4}{x^{\frac{4}{3}}}+\dfrac{4}{5}{x^{\frac{5}{4}}}+C=\dfrac{2}{3}x\sqrt x+\dfrac{3}{4}x\sqrt[3]{x}+\dfrac{4}{5}x\sqrt[4]{x}+C$.
			\itemch {\bf Đúng.} $f(x)=\dfrac{x^3-2024x}{x}\Rightarrow F(x)=\displaystyle\int{\dfrac{x^3-2024x}{x}dx}=\displaystyle\int\left(x^2-2024\right)dx$\\
			$=\dfrac{1}{3}{x^3}-2024x+C$.
		\end{itemchoice}
	}
\end{ex}
\Closesolutionfile{ans}
\indapan{3}{ans/ans-2-B1-D2-DS}
\Opensolutionfile{ans}[ans/ans-2-B1-D1-KQ]
% \TN
\begin{ex}%[2D4H1-2][Lê Công Trường]
	Hệ số của $x^2$ trong nguyên hàm $F(x)$ của hàm số $f(x)=\dfrac{2}{\sqrt{x}}+3^x+3x-2$ là
	\shortans{$1{,}5$}
	\loigiai{
		$F(x)=\displaystyle\int{\left(\dfrac{2}{\sqrt{x}}+3^x+3x-2\right)\mathrm{\,d}x}=4\sqrt{x}+\dfrac{3^x}{\ln 3}+\dfrac{3}{2}{x^2}-2x+C$.
	}
\end{ex}

\begin{ex}%[2D4H1-2][Lê Công Trường]
	Hệ số của $x^3$ trong nguyên hàm $F(x)$ của hàm số $f(x)=m{x^3}-3x^2+\dfrac{4m}{x^3}+\dfrac{5}{2x}-7m$ ($m$ là tham số) là
	\shortans{$-1$}
	\loigiai{
		$F(x)=\displaystyle\int{\left(m{x^3}-3x^2+\dfrac{4m}{x^3}+\dfrac{5}{2x}-7m\right)\mathrm{\,d}x}=\dfrac{m}{4}{x^4}-x^3-\dfrac{2m}{x^2}-\dfrac{5}{2}\ln {|x|}-7mx+C$
	}
\end{ex}

\begin{ex}% [2D4H1-2][Lê Công Trường]
	Tìm nguyên hàm $F(x)$ của hàm số $f(x)=\dfrac{1}{\sqrt{x}}-\dfrac{2}{\sqrt[3]{x}}$. Tổng hệ số của biến $x$ là
	\shortans{$-1$}
	\loigiai{
		$F(x)=\displaystyle\int f(x)\mathrm{\,d}x=\displaystyle\int\left(\dfrac{1}{\sqrt{x}}-\dfrac{2}{\sqrt[3]{x}}\right)\mathrm{\,d}x=\displaystyle\int\dfrac{1}{\sqrt{x}}\mathrm{\,d}x-\displaystyle\int\dfrac{2}{\sqrt[3]{x}}=\displaystyle\int{x^{\frac{-1}{2}}}\mathrm{\,d}x-\displaystyle\int{2x^{\frac{-1}{3}}}\mathrm{\,d}x$\\
		$=\dfrac{x^{\frac{1}{2}}}{\dfrac{1}{2}}-2.\dfrac{x^{\frac{2}{3}}}{\dfrac{2}{3}}+C=2{x^{\frac{1}{2}}}-3x^{\frac{2}{3}}+C=2\sqrt{x}-3\sqrt[3]{x^2}+C$.
	}
\end{ex}

\begin{ex}%%[2D4H1-2][Lê Công Trường]
	Tìm nguyên hàm $F(x)$ của hàm số $f(x)=\dfrac{(x^2-1)^2}{x^2}$. Tổng hệ số của bậc $3$ và bậc $1$ là (làm tròn đến hàng phần chục).
	\shortans{$-1{,}6$}
	\loigiai{
		$\displaystyle\int  f(x)\mathrm{\,d}x=\displaystyle\int\dfrac{(x^2-1)^2}{x^2}\mathrm{\,d}x=\displaystyle\int\dfrac{x^4-2x^2+1}{x^2}\mathrm{\,d}x=\displaystyle\int\left(x^2-2+\dfrac{1}{x^2}\right)\mathrm{\,d}x$\\
		$=\dfrac{x^3}{3}-2x-\dfrac{1}{x}+C$.
	}
\end{ex}

\begin{ex}%%[2D4H1-2][Lê Công Trường]
	Tính $\displaystyle\int{\left(\dfrac{\left(1-x\right)^3}{\sqrt[3]{x}}\right)\mathrm{\,d}x}$. Giá trị tổng hệ số chứa biến là (làm tròn đến hàng phần trăm).
	\shortans{$0{,}55$}
	\loigiai{$\displaystyle\int\left(\dfrac{\left(1-x\right)^3}{\sqrt[3]{x}}\right)\mathrm{\,d}x=\displaystyle\int\dfrac{1-3x+3x^2-x^3}{x^{\frac{1}{3}}}\mathrm{\,d}x=\displaystyle\int\left(x^{\frac{-1}{3}}-3x^{\frac{2}{3}}+3x^{\frac{5}{3}}-x^{\frac{8}{3}}\right)\mathrm{\,d}x$\\
		$=\dfrac{x^{\frac{2}{3}}}{\dfrac{2}{3}}-3\dfrac{x^{\frac{5}{3}}}{\dfrac{5}{3}}+3\dfrac{x^{\frac{8}{3}}}{\dfrac{8}{3}}-\dfrac{x^{\frac{11}{3}}}{\dfrac{11}{3}}+C=\dfrac{3}{2}{x^{\frac{2}{3}}}-\dfrac{9}{5}{x^{\frac{5}{3}}}+\dfrac{9}{8}{x^{\frac{8}{3}}}-\dfrac{3}{11}{x^{\frac{11}{3}}}+C$.
		
	}
\end{ex}

\begin{ex}%[2D4H1-2][Lê Công Trường]
	Tính $\displaystyle\int{\left(\sqrt[3]{x^2}-\sqrt[4]{x^3}+\sqrt[5]{x^4}\right)\mathrm{\,d}x}$. Giá trị tổng hệ số chứa biến là (làm tròn đến hàng phần trăm).
	\shortans{$0{,}58$}
	\loigiai{
		$\displaystyle\int\left(\sqrt[3]{x^2}-\sqrt[4]{x^3}+\sqrt[5]{x^4}\right)\mathrm{\,d}x=\displaystyle\int\left(x^{\frac{2}{3}}-x^{\frac{3}{4}}+x^{\frac{4}{5}}\right)\mathrm{\,d}x=\dfrac{x^{\frac{5}{3}}}{\dfrac{5}{3}}-\dfrac{x^{\frac{7}{4}}}{\dfrac{7}{4}}+\dfrac{x^{\frac{9}{5}}}{\dfrac{9}{5}}+C$\\
		$=\dfrac{3}{5}{x^{\frac{5}{3}}}-\dfrac{4}{7}{x^{\frac{7}{4}}}+\dfrac{5}{9}{x^{\frac{9}{5}}}+C$.
	}
\end{ex}

\begin{ex}%%[2D4H1-2][Lê Công Trường]
	Tính $\displaystyle\int\left(\sqrt{x}+1\right)\left(x-\sqrt{x}+1\right)\mathrm{\,d}x$. Giá trị tổng hệ số chứa biến là (làm tròn đến hàng phần chục).
	\shortans{$1{,}4 $}
	\loigiai{
		$\left(\sqrt{x}+1\right)\left(x-\sqrt{x}+1\right)=\left(\sqrt{x}+1\right)\left[x-\left(\sqrt{x}-1\right)\right]=x\left(\sqrt{x}+1\right)-\left(\sqrt{x}+1\right)\left(\sqrt{x}-1\right)$\\
		$=x\sqrt{x}+x-\left(x-1\right)=x\sqrt{x}+1$.\\
		Do đó $\displaystyle\int\left(\sqrt{x}+1\right)\left(x-\sqrt{x}+1\right)\mathrm{\,d}x=\displaystyle\int\left(x\sqrt{x}+1\right)\mathrm{\,d}x=\displaystyle\int\left(x^{\frac{3}{2}}+1\right)\mathrm{\,d}x$\\
		$=\dfrac{2}{5}x^{\frac{5}{2}}+x+C$.
	}
\end{ex}

\begin{ex}%%[2D4H1-2][Lê Công Trường]
	Tính $\displaystyle\int{\left(2\sqrt{x}-\dfrac{3}{\sqrt[3]{x}}\right)\mathrm{\,d}x}$. Giá trị tổng hệ số chứa biến là (làm tròn đến hàng phần chục).
	\shortans{$-3{,}1$}
	\loigiai{$\displaystyle\int{\left(2\sqrt {x}-\dfrac{3}{\sqrt[3]{x}}\right)\mathrm{\,d}x=\displaystyle\int{\left(2x^{\frac{1}{2}}-3x^{\frac{-1}{3}}\right)}}\mathrm{\,d}x=\dfrac{4}{3}{x^{\frac{3}{2}}}-\dfrac{9}{2}{x^{\frac{2}{3}}}+C=\dfrac{4}{3}\sqrt[2]{x^3}-\dfrac{9}{2}\sqrt[3]{x^2}+C$.\\
	}
\end{ex}	

\begin{ex}%[2D4H1-2][Lê Công Trường]
	Tính $\displaystyle\int{\dfrac{1}{\sqrt{2x}+\sqrt{3x}}\mathrm{\,d}x}=a\left(\sqrt {b}-\sqrt {c}\right)\sqrt {x}$. Giá trị của tổng $a+b+c$ là
	\shortans{$7$}
	\loigiai{
		Ta có: $\dfrac{1}{\sqrt{2x}+\sqrt{3x}}=\dfrac{\sqrt{3x}-\sqrt{2x}}{\left(\sqrt{3x}-\sqrt{2x}\right)\left(\sqrt{3x}+\sqrt{2x}\right)}=\dfrac{\sqrt{3x}-\sqrt{2x}}{x}=\dfrac{\sqrt {x}}{x}\left(\sqrt 3-\sqrt 2\right)$\\
		$=\left(\sqrt 3-\sqrt 2\right){x^{\frac{-1}{2}}}.$\\
		$\displaystyle\int{\dfrac{1}{\sqrt{2x}+\sqrt{3x}}\mathrm{\,d}x}=\displaystyle\int{\left(\sqrt 3-\sqrt 2\right){x^{\frac{-1}{2}}}\mathrm{\,d}x=}\left(\sqrt 3-\sqrt 2\right)\dfrac{x^{\frac{1}{2}}}{\dfrac{1}{2}}=2\left(\sqrt 3-\sqrt 2\right)\sqrt {x}$.}
\end{ex}

\begin{ex}%%[2D4H1-2][Lê Công Trường]
	Tính $\displaystyle\int{\dfrac{1}{\sqrt{5x}-\sqrt{3x}}\mathrm{\,d}x=\left(\sqrt{a}+\sqrt{b}\right)\sqrt {x}+C}$. Giá trị $a+b$ bằng\\
	\shortans{$8$}
	\loigiai{
		$\dfrac{1}{\sqrt{5x}-\sqrt{3x}}=\dfrac{\sqrt{5x}+\sqrt{3x}}{\left(\sqrt{5x}-\sqrt{3x}\right)\left(\sqrt{5x}+\sqrt{3x}\right)}=\dfrac{\sqrt{5x}+\sqrt{3x}}{2x}=\dfrac{\sqrt {x}}{2x}\left(\sqrt 5+\sqrt 3\right).$\\
		$\displaystyle\int{\dfrac{1}{\sqrt{5x}-\sqrt{3x}}\mathrm{\,d}x}=\displaystyle\int{\dfrac{\sqrt {x}}{2x}\left(\sqrt 5+\sqrt 3\right)\mathrm{\,d}x}=\dfrac{\left(\sqrt 5+\sqrt 3\right)}{2}\displaystyle\int{x^{\frac{-1}{2}}}\mathrm{\,d}x=\dfrac{\left(\sqrt 5+\sqrt 3\right)}{2}\cdot\dfrac{x^{\frac{1}{2}}}{\dfrac{1}{2}}$\\
		$=\left(\sqrt 5+\sqrt 3\right)\sqrt {x}+C$.}
\end{ex}

\begin{ex}%%[2D4H1-2][Lê Công Trường]
	Tính $\displaystyle\int{\left(x^2-1\right)^3\mathrm{\,d}x}$. Giá trị tổng hệ số chứa biến là (làm tròn đến hàng phần chục).
	\shortans{$-0{,}5$}
	\loigiai{
		$\displaystyle\int\left(x^2-1\right)^3\mathrm{\,d}x=\displaystyle\int\left(x^6-3x^4+3x^2-1\right)\mathrm{\,d}x=\dfrac{x^7}{7}-3\dfrac{x^5}{5}+x^3-x+C$.}
\end{ex}

\begin{ex}%%[2D4H1-2][Lê Công Trường]
	Tính $\displaystyle\int{\left(2-x^2\right)^4\mathrm{\,d}x}$. Giá trị tổng hệ số chứa biến là (làm tròn đến hàng phần chục).
	\shortans{$9{,}1 $}
	\loigiai{
		Sử dụng khai triển theo nhị thức Newton, ta có:\\
		$\left(2-x^2\right)^4=x^8-8x^6+24x^4-32x^2+16$.\\
		Do đó\\
		$\displaystyle\int{\left(2-x^2\right)^4\mathrm{\,d}x}=\displaystyle\int{\left(x^8-8x^6+24x^4-32x^2+16\right)}\mathrm{\,d}x$\\
		$=\dfrac{x^8}{9}-\dfrac{8}{7}{x^7}+\dfrac{24}{5}{x^5}-\dfrac{32}{3}{x^3}+16x+C$.}
\end{ex}

\begin{ex}%%[2D4H1-2][Lê Công Trường]
	Tính $\displaystyle\int{\left(x-\sqrt[3]{x}\right)^2\mathrm{\,d}x}$. Giá trị tổng hệ số chứa biến là (làm tròn đến hàng phần chục).
	\shortans{$-1{,}1 $}
	\loigiai{
		$\left(x-\sqrt[3]{x}\right)^2=x^2-2x\sqrt[3]{x}-\sqrt[3]{x^2}=x^2-2x^{\frac{4}{3}}-x^{\frac{2}{3}}$.\\
		$\displaystyle\int{\left(x-\sqrt[3]{x}\right)^2\mathrm{\,d}x}=\displaystyle\int{\left(x^2-2x^{\frac{4}{3}}-x^{\frac{2}{3}}\right)\mathrm{\,d}x=}\dfrac{x^3}{3}-\dfrac{6}{7}{x^{\frac{7}{3}}}-\dfrac{3}{5}{x^{\frac{5}{3}}}+C.$}
\end{ex}

\begin{ex}%%[2D4H1-2][Lê Công Trường]
	Tính $\displaystyle\int{\left(\dfrac{x^2+2\sqrt[3]{x}}{x}\right)^2\mathrm{\,d}x}$.  Giá trị tổng hệ số chứa biến là (làm tròn đến hàng phần chục).
	\shortans{$-8{,}7$}
	\loigiai{
		Ta có: $\left(\dfrac{x^2+2\sqrt[3]{x}}{x}\right)^2=\dfrac{x^4+4x^2\sqrt[3]{x}+4\sqrt[3]{x^2}}{x^2}=x^2+4x^{\frac{1}{3}}+4x^{\frac{-4}{3}}$.\\
		$\displaystyle\int{\left(\dfrac{x^2+2\sqrt[3]{x}}{x}\right)^2\mathrm{\,d}x}=\displaystyle\int{\left(x^2+4x^{\frac{1}{3}}+4x^{\frac{-4}{3}}\right)\mathrm{\,d}x=\dfrac{x^3}{3}+4\dfrac{x^{\frac{4}{3}}}{\dfrac{4}{3}}}+4\dfrac{x^{\frac{-1}{3}}}{\dfrac{-1}{3}}+C$\\
		$=\dfrac{1}{3}{x^3}+3x^{\frac{4}{3}}-12x^{\frac{-1}{3}}+C$.}
\end{ex}

\begin{ex}%[2D4H1-2][Lê Công Trường]
	Tìm $m$ để $F(x)=m{x^3}+(3m+2){x^2}-4x+3$ là một nguyên hàm của hàm số $f(x)=3x^2+10x-4$.
	\shortans{$1$}
	\loigiai{
		$\displaystyle\int{f(x)\mathrm{\,d}x}=\displaystyle\int{\left(3x^2+10x-4\right)}\mathrm{\,d}x=x^3+5x^2-4x+C.$ Suy ra $ m=1$.
	}
\end{ex}

\begin{ex}%%[2D4V1-2][Lê Công Trường]
	Tìm $a,b,c$ để $F(x)=(a{x^2}+bx+c)\sqrt{x^2-4x}$ là một nguyên hàm của hàm số $f(x)=(x-2)\sqrt{x^2-4x}$. Giá trị biểu thức $a+b+c$ bằng.
	\shortans{$-1$}
	\loigiai{
		Đặt $ t=\sqrt{x^2-4x}\Rightarrow{t^2}=x^2-4x\Rightarrow 2t\mathrm{\,d}t=\left(2x-4\right)\mathrm{\,d}x=2\left(x-2\right)\mathrm{\,d}x$.\\
		$\Rightarrow \mathrm{\,d}x=\dfrac{2t\mathrm{\,d}t}{2\left(x-2\right)}=\dfrac{t\mathrm{\,d}t}{x-2}$.\\
		$\displaystyle\int{(x-2)\sqrt{x^2-4x}}\mathrm{\,d}x=\displaystyle\int{t.t.\mathrm{\,d}t=\displaystyle\int{t^2\mathrm{\,d}t}=\dfrac{1}{3}}{t^3}+C=\dfrac{1}{3}\sqrt{\left(x^2-4x\right)^3}+C$\\$=\dfrac{1}{3}\left(x^2-4x\right)\sqrt{x^2-4x}+C=\left(\dfrac{1}{3}{x^2}-\dfrac{4}{3}x\right)\sqrt{x^2-4x}+C$.\\
		Vậy $ a=\dfrac{1}{3};\,\,b=-\dfrac{4}{3};\,\,c=0$.}
\end{ex}

\begin{ex}%%[2D4V1-2][Lê Công Trường]
	Tìm $a,b,c$ để $F(x)=(a{x^2}+bx+c)\sqrt{2x-3}$ là một nguyên hàm của hàm số $f(x)=\dfrac{20x^2-30x+7}{\sqrt{2x-3}}$.  Giá trị biểu thức $a+b+c$ bằng\\
	\shortans{$3$}
	\loigiai{
		Theo định nghĩa nguyên hàm thì $ F'(x)=f(x)$.\\
		Ta có 
		\begin{eqnarray*}
			F'(x) & =& \left(2ax+b\right)\sqrt{2x-3}+(a{x^2}+bx+c)\dfrac{2}{2\sqrt{2x-3}}\\
			&= & \dfrac{\left(2ax+b\right)\left(2x-3\right)+a{x^2}+bx+c}{\sqrt{2x-3}}\\
			&= & \dfrac{5a{x^2}+\left(-6a+3b\right)x-3b+c}{\sqrt{2x-3}}.
		\end{eqnarray*}
		Từ đó ta có $\dfrac{5a{x^2}+\left(-6a+3b\right)x-3b+c}{\sqrt{2x-3}}=\dfrac{20x^2-30x+7}{\sqrt{2x-3}}$.\\
		Sử dụng phương pháp đồng nhất hệ số, ta được\\
		$\heva{
			&5a=20\\
			&-6a+3b=-30\\
			&-3b+c=7.
		}\Leftrightarrow
		\heva{
			&a=4\\
			&b=-2\\
			&c=1.
		}$
	}
\end{ex}
\Closesolutionfile{ans}
\indapan{6}{ans/ans-2-B1-D1-KQ}

\begin{ex}%[2D4H1-3][Lê Công Trường]
	Hàm số $F(x)=\cot x$ là một nguyên hàm của hàm số nào dưới đây trên khoảng $\left(0;\dfrac{\pi}{2}\right)$
	\choice
	{$f_2(x)=\dfrac{1}{\sin^2x}$}
	{$f_1(x)=-\dfrac{1}{\cos^2x}$}
	{$f_4(x)=\dfrac{1}{\cos^2x}$}
	{\True $f_3(x)=-\dfrac{1}{\sin^2x}$}
	\loigiai{
		Có $\displaystyle\int{\dfrac{1}{\sin^2x}\mathrm{\,d}x}=-\cot x+C$ suy ra $F(x)=\cot x$ trên khoảng $\left(0;\dfrac{\pi}{2}\right)$ là một nguyên hàm của hàm số $f_3(x)=-\dfrac{1}{\sin^2x}$.}
\end{ex}

\begin{ex}%[2D4H1-3][Lê Công Trường]
	Cho hàm số $f(x)=1+\sin x$. Khẳng định nào dưới đây đúng?
	\choice
	{\True $\displaystyle\int{f(x){\rm{d}}x}=x-\cos x+C$}
	{$\displaystyle\int{f(x){\rm{d}}x}=x+\sin x+C$}
	{$\displaystyle\int{f(x){\rm{d}}x}=x+\cos x+C$}
	{$\displaystyle\int{f(x){\rm{d}}x}=\cos x+C$}
	\loigiai
	{Ta có $\displaystyle\int{f(x){\rm{d}}x=\displaystyle\int{\left(1+\sin x\right){\rm{d}}x}=\displaystyle\int{1\rm{d}x}+\displaystyle\int{\sin x{\rm{d}}x}=x-\cos x+C}$.}
\end{ex}

\begin{ex}%[2D4H1-3][Lê Công Trường]
	Tìm nguyên hàm $F(x)$ của hàm số $f(x)=\cos ^2\dfrac{x}{2}$
	\choice
	{$F(x)=2\cos\dfrac{x}{2}+C$}
	{\True $F(x)=\dfrac{1}{2}\left(1+\sin x\right)+C$}
	{$F(x)=2\sin\dfrac{x}{2}+C$}
	{$F(x)=\dfrac{1}{2}\left(1-\sin x\right)+C$}
	\loigiai{
		Ta có:$f(x)=\cos ^2\dfrac{x}{2}\Rightarrow F(x)=\displaystyle\int{\cos^2\dfrac{x}{2}\mathrm{\,d}x}=\displaystyle\int{\dfrac{1+\cos x}{2}\mathrm{\,d}x}=\dfrac{1}{2}\displaystyle\int{\left(1+\cos x\right)\mathrm{\,d}x}$\\
		$=\dfrac{1}{2}\left(1+\sin x\right)+C$.}
\end{ex}

\begin{ex}%[2D4H1-3][Lê Công Trường]
	Cho hàm số $f(x)=1-\dfrac{1}{\cos^2x}$. Khẳng định nào dưới đây đúng?
	\choice
	{$\displaystyle\int{f(x){\rm{d}}x}=x+\tan x+C$}
	{$\displaystyle\int{f(x){\rm{d}}x}=x+\cot x+C$}
	{\True $\displaystyle\int{f(x){\rm{d}}x}=x-\tan x+C$}
	{$\displaystyle\int{f(x){\rm{d}}x}=x-\cot x+C$}
	\loigiai
	{
		$\displaystyle\int{f(x){\rm{d}}x}=\displaystyle\int{\left(1-\dfrac{1}{\cos^2x}\right){\rm{d}}x}=x-\tan x+C$.}
\end{ex}

\begin{ex}%%[2D4H1-3][Lê Công Trường]
	Họ nguyên hàm của hàm số $f(x)=\cos x+6x$ là
	\choice
	{\True $\sin x+3x^2+C$}
	{$-\sin x+3x^2+C$}
	{$\sin x+6x^2+C$}
	{$-\sin x+C$}
	\loigiai
	{
		Ta có $\displaystyle\int{f(x){\rm{d}}x=\displaystyle\int{\left(\cos x+6x\right){\rm{d}}x=\sin x+3x^2+C}}$.}
\end{ex}

\begin{ex}%%[2D4H1-3][Lê Công Trường]
	Tìm nguyên hàm của hàm số $f(x)=2\sin x+3x$.
	\choice
	{\True $\displaystyle\int{\left(2\sin x+3x\right)\mathrm{\,d}x=-2\cos x+\dfrac{3}{2}{x^2}+C}$}
	{$\displaystyle\int{\left(2\sin x+3x\right)\mathrm{\,d}x=2\cos x+3x^2+C}$}
	{$\displaystyle\int{\left(2\sin x+3x\right)\mathrm{\,d}x=\sin^2x+\dfrac{3}{2}x+C}$}
	{$\displaystyle\int{\left(2\sin x+3x\right)\mathrm{\,d}x=\sin 2x+\dfrac{3}{2}{x^2}+C}$}
	\loigiai
	{
		$\displaystyle\int{\left(2\sin x+3x\right)\mathrm{\,d}x}=-2\cos x+\dfrac{3}{2}{x^2}+C$}
\end{ex}

\begin{ex}%%[2D4H1-3][Lê Công Trường]
	Tính$\displaystyle\int{\left(x-\sin x\right)}{\rm{d}}x$.
	\choice
	{$\dfrac{x^2}{2}+\sin x+C$}
	{$\dfrac{x^2}{2}-\cos x+C$}
	{$\dfrac{x^2}{2}-\sin x+C$}
	{\True $\dfrac{x^2}{2}+\cos x+C$}
	\loigiai
	{
		Ta có $\displaystyle\int{\left(x-\sin x\right){\rm{d}}x\,\rm{=}\,}\dfrac{x^2}{2}+\cos x+C$.}
\end{ex}
\begin{ex}%[2D4H1-3][Lê Công Trường]
	Họ nguyên hàm của hàm số $f(x)=3x^2+\sin x$ là
	\choice
	{$x^3+\cos x+C$}
	{$6x+\cos x+C$}
	{\True $x^3-\cos x+C$}
	{$6x-\cos x+C$}
	\loigiai
	{
		Ta có $\displaystyle\int\left(3x^2+\sin x\right){\rm{d}}x=x^3-\cos x+C$.}
\end{ex}
\begin{ex}%[2D4H1-3][Lê Công Trường]
	Họ nguyên hàm của hàm số $ f(x)=\dfrac{1}{x}+\sin x$ là
	\choice
	{$\ln x-\cos x+C$}
	{$-\dfrac{1}{x^2}-\cos x+C$}
	{$\ln \left| x\right|+\cos x+C$}
	{\True $\ln \left| x\right|-\cos x+C$}
	\loigiai
	{
		Ta có $\displaystyle\int{f(x){\rm{d}}x}=\displaystyle\int{\left(\dfrac{1}{x}+\sin x\right){\rm{d}}x}=\displaystyle\int{\dfrac{1}{x}{\rm{d}}x}+\displaystyle\int{\sin x{\rm{d}}x}=\ln \left| x\right|-\cos x+C$.}
\end{ex}
\begin{ex}%%[2D4H1-3][Lê Công Trường]
	Cho $\displaystyle\int{f(x)}\,\rm{d}x=-\cos x+C$. Khẳng định nào dưới đây đúng?
	\choice
	{$ f(x)=-\sin x$}
	{$ f(x)=-\cos x$}
	{\True $ f(x)=\sin x$}
	{$ f(x)=\cos x$}
	\loigiai{
		Áp dụng công thức $\smallint{\rm{sin}}x{\rm{\;d}}x=-\rm{cos}x+C$. Suy ra $ f(x)=\rm{sin}x$.}
\end{ex}
\begin{ex}%[2D4H1-3][Lê Công Trường]
	Cho hàm số $ f(x)=\displaystyle\int{\cos\dfrac{x}{2}\sin\dfrac{x}{2}}$. Khẳng định nào dưới đây đúng?
	\choice
	{$\displaystyle\int{\cos\dfrac{x}{2}\sin\dfrac{x}{2}}=\dfrac{1}{2}\sin+C$}
	{$\displaystyle\int{\cos\dfrac{x}{2}\sin\dfrac{x}{2}}=\dfrac{1}{2}\cos x+C$}
	{$\displaystyle\int{\cos\dfrac{x}{2}\sin\dfrac{x}{2}}=-\dfrac{1}{2}\sin x+C$}
	{\True $\displaystyle\int{\cos\dfrac{x}{2}\sin\dfrac{x}{2}}=-\dfrac{1}{2}\cos x+C$}
	\loigiai{
		$\displaystyle\int{\cos\dfrac{x}{2}\sin\dfrac{x}{2}}=\dfrac{1}{2}\displaystyle\int{\sin x}\mathrm{\,d}x=-\dfrac{1}{2}\cos x+C$.}
\end{ex}
\Closesolutionfile{ans}
\indapan{10}{ans/ans-2-B1-D2-TN}
\Opensolutionfile{ans}[ans/ans-2-B1-D2-DS]
% \TNTF
\begin{ex}%%[2D4H1-3][Lê Công Trường]
	Các mệnh đề sau đây đúng hay sai?
	\choiceTF
	{\True $\displaystyle\int{\left(2+\cot^2x\right)\mathrm{\,d}x}=x-\cot x+C$}
	{ $\displaystyle\int{\left(1-\cos^2\dfrac{x}{2}\right)\mathrm{\,d}x}=\dfrac{1}{2}\left(x+\sin x\right)+C$}
	{$\displaystyle\int{\left(\sin\dfrac{x}{2}+\cos\dfrac{x}{2}\right)^2}\mathrm{\,d}x=x+\cos x+C$}
	{ $\displaystyle\int{\left(\sin\dfrac{x}{2}-\cos\dfrac{x}{2}\right)^2}\mathrm{\,d}x=x-\cos x+C$}
	\loigiai{
		\begin{itemchoice}
			\itemch {\bf Đúng}. Vì
			$\displaystyle\int{\left(2+\cot^2x\right)\mathrm{\,d}x}$\\
			$=\displaystyle\int{\left(1+1+\cot^2x\right)\mathrm{\,d}x}=\displaystyle\int{\left(1+\dfrac{1}{\sin^2x}\right)\mathrm{\,d}x}=x-\cot x+C$.
			\itemch {\bf Sai}. Vì $\displaystyle\int{\left(1-\cos^2\dfrac{x}{2}\right)\mathrm{\,d}x}=\displaystyle\int{\sin^2\dfrac{x}{2}\mathrm{\,d}x}=\displaystyle\int{\dfrac{1-\cos x}{2}\mathrm{\,d}x}=\dfrac{1}{2}\left(x-\sin x\right)+C$.
			\itemch {\bf Sai}. Vì $\displaystyle\int{\left(\sin\dfrac{x}{2}+\cos\dfrac{x}{2}\right)^2}\mathrm{\,d}x=\displaystyle\int{\left(1+\sin x\right)}\mathrm{\,d}x=x-\cos x+C$.
			\itemch {\bf Sai}. Vì $\displaystyle\int{\left(\sin\dfrac{x}{2}-\cos\dfrac{x}{2}\right)^2}\mathrm{\,d}x=\displaystyle\int{\left(1-\sin x\right)}\mathrm{\,d}x=x+\cos x+C$.
		\end{itemchoice}
	}
\end{ex}
\Closesolutionfile{ans}
\indapan{2}{ans/ans-2-B1-D2-DS}
\Opensolutionfile{ans}[ans/ans-2-B1-D2-KQ]
% \TN
\begin{ex}%%[2D4H1-3][Lê Công Trường]
	Tìm nguyên hàm $ F(x)$của hàm số $f(x)=2024-2\sin ^2\dfrac{x}{2}$. Hệ số của biến $x$ là
	\shortans{$2023$}
	\loigiai{
		$\Rightarrow F(x)=\displaystyle\int{\left(2024-2\sin^2\dfrac{x}{2}\right)}\mathrm{\,d}x=\displaystyle\int{\left(2023+\cos x\right)}\mathrm{\,d}x=2023x-\sin x+C$.}
\end{ex}
\begin{ex}%%[2D4H1-3][Lê Công Trường]
	Tìm nguyên hàm $ F(x)$của hàm số $f(x)=\dfrac{1}{\sin^2\dfrac{x}{2}\cdot\cos^2\dfrac{x}{2}}==a\cot x+C$. Giá trị $a$ là
	\shortans{$-4$}
	\loigiai{
		Ta có $\dfrac{1}{\sin^2\dfrac{x}{2}\cdot\cos^2\dfrac{x}{2}}=\dfrac{1}{\left(\sin\dfrac{x}{2}\cdot\cos\dfrac{x}{2}\right)^2}=\dfrac{1}{\left(\dfrac{\sin x}{2}\right)^2}=\dfrac{4}{\sin^2x}\cdot$\\
		$ F(x)=\displaystyle\int{f(x)\mathrm{\,d}x=\displaystyle\int{\dfrac{1}{\sin^2\dfrac{x}{2}\cdot\cos^2\dfrac{x}{2}}}}\mathrm{\,d}x=\displaystyle\int{\dfrac{4}{\sin^2x}=-4\cot x+C}$.}
\end{ex}
\begin{ex}%%[2D4H1-3][Lê Công Trường]
	Tìm nguyên hàm $ F(x)$ của hàm số $f(x)=\dfrac{1}{3}{x^2}-2x+\dfrac{1}{2}{\tan ^2}x=\dfrac{x^3}{a}+bx^2+\dfrac{1}{c}x+d\tan x+C$. Giá trị của $a+b+c+d$ là
	\shortans{$6{,}5$}
	\loigiai{$F(x)=\displaystyle\int{f(x)\mathrm{\,d}x}$\\
		$=\displaystyle\int{\left(\dfrac{1}{3}{x^2}-2x+\dfrac{1}{2}{\tan^2}x\right)}\mathrm{\,d}x=\displaystyle\int{\left(\dfrac{1}{3}{x^2}-2x+\dfrac{1}{2}\dfrac{\sin^2x}{\cos^2x}\right)}\mathrm{\,d}x\\
		=\displaystyle\int{\left[\dfrac{1}{3}{x^2}-2x+\dfrac{1}{2}\left(\dfrac{1-\cos^2x}{\cos^2x}\right)\right]}\mathrm{\,d}x=\displaystyle\int{\left[\dfrac{1}{3}{x^2}-2x+\dfrac{1}{2}\left(\dfrac{1}{\cos^2x}-1\right)\right]}\mathrm{\,d}x\\
		=\dfrac{x^3}{9}-x^2+\dfrac{1}{2}\left(\tan x-x\right)+C=\dfrac{x^3}{9}-x^2-\dfrac{1}{2}x+\dfrac{1}{2}\tan x+C$.
	}
\end{ex}
% \begin{ex}%%[2D4V1-3][Lê Công Trường]
% 	Tính $I=\displaystyle\int{x\left(1-\dfrac{\sin^2\dfrac{x}{2}}{2}\right)\mathrm{\,d}x}$. Hệ số của hạng tử $\cos {x}$ của $I$ là
% 	\shortans{$-1$} 
% 	\loigiai{
% 		Đáp án: Ta có $x\left(1-\dfrac{\sin^2\dfrac{x}{2}}{2}\right)=x\left(1-\dfrac{1-cox}{4}\right)=\dfrac{3}{4}x+\dfrac{1}{4}x\cos x$.\\
% 		$\displaystyle\int x\left(1-\dfrac{\sin^2\dfrac{x}{2}}{2}\right)\mathrm{\,d}x=\displaystyle\int\left(\dfrac{3}{4}x+\dfrac{1}{4}x\cos x\right)\mathrm{\,d}x=\displaystyle\int\dfrac{3}{4}x\mathrm{\,d}x+\displaystyle\int\dfrac{1}{4}x\cos x\mathrm{\,d}x$\\
% 		$=\dfrac{3}{8}{x^2}+C_1+\dfrac{1}{4}\displaystyle\int{x\cos x\mathrm{\,d}x.}$\\
% 		Đặt $\heva{
% 			&u=x\Rightarrow \mathrm{\,d}u=\mathrm{\,d}x\\
% 			&dv=\cos x\mathrm{\,d}x\Rightarrow v=\sin x.
% 		}$\\
% 		Sử dụng phương pháp tích phân từng phần, ta có\\
% 		$\displaystyle\int{x\cos x\mathrm{\,d}x}=x\sin x+\displaystyle\int{\sin x\mathrm{\,d}x=x\sin x-\cos x+C_2}$.\\
% 		Vậy $\displaystyle\int{x\left(1-\dfrac{\sin^2\dfrac{x}{2}}{2}\right)\mathrm{\,d}x}=\dfrac{3}{8}{x^2}+x\sin x-\cos x+C.$}
% \end{ex}	
\begin{ex}%%[2D4H1-3][Lê Công Trường]
	Tính $\displaystyle\int{x^2\left(1+\dfrac{1}{x}-\dfrac{\tan^2x}{x^2}\right)\mathrm{\,d}x}=\dfrac{x^m}{n}+\dfrac{x^p}{q}+x+r\tan x+C$. Giá trị biểu thức $P=\dfrac{m}{n}+\dfrac{p}{q}+2r$ là	
	\shortans{$0$} 
	\loigiai{
		$\displaystyle\int{x^2\left(1+\dfrac{1}{x}-\dfrac{\tan^2x}{x^2}\right)\mathrm{\,d}x}=\displaystyle\int{\left(x^2+x-\tan^2x\right)\mathrm{\,d}x}=\dfrac{x^3}{3}+\dfrac{x^2}{2}-(\tan x-x)+C$\\
		$=\dfrac{x^3}{3}+\dfrac{x^2}{2}+x-\tan x+C$.}
\end{ex}
\begin{ex}%%[2D4V1-3][Lê Công Trường]
	Tính $T=\displaystyle\int{x\left(2024-\dfrac{1}{x^3}+\dfrac{\sin x}{x}\right)\mathrm{\,d}x}$. Hệ số của hạng tử $\cos {x}$ của $T$ là
	\shortans{$-1$} 
	\loigiai{
		$\displaystyle\int{x\left(2024-\dfrac{1}{x^3}+\dfrac{\sin x}{x}\right)\mathrm{\,d}x}=\displaystyle\int{\left(2024x-\dfrac{1}{x^2}+\sin x\right)}\mathrm{\,d}x=1012x^2+\dfrac{1}{x}-\cos x+C.$}
\end{ex}
\begin{ex}%Câu 27%[2D4H1-5]
	Tính $R=\displaystyle\int{x^3\left[\dfrac{\left(\sin\dfrac{x}{2}+\cos\dfrac{x}{2}\right)^2}{x^3}-2x+\dfrac{1}{x^{2024}}\right]}\mathrm{\,d}x= ax+b\cos x+c{x^5}-\dfrac{1}{d\cdot x^{2020}}+C$. Giá trị $a+b+c+d+7$ là (làm tròn đến hàng đơn vị)
	\shortans{$2025$} 
	\loigiai{
		Ta có
		\begin{eqnarray*}
			{x^3}\left[\dfrac{\left(\sin\dfrac{x}{2}+\cos\dfrac{x}{2}\right)^2}{x^3}-2x+\dfrac{1}{x^{2024}}\right] &=& \left(\sin\dfrac{x}{2}+\cos\dfrac{x}{2}\right)^2-2x^4+x^{-2021}\\
			&=& \sin ^2\dfrac{x}{2}+\cos^2\dfrac{x}{2}+2\sin\dfrac{x}{2}\cos\dfrac{x}{2}-2x^4+x^{-2021}\\
			&=&1+2\sin x-2x^4+x^{-2021}.
		\end{eqnarray*}
		Khi đó\\
		\begin{eqnarray*}
			\displaystyle\int{x^3\left[\dfrac{\left(\sin\dfrac{x}{2}+\cos\dfrac{x}{2}\right)^2}{x^3}-2x+\dfrac{1}{x^{2024}}\right]}\mathrm{\,d}x&=& \displaystyle\int{\left(1+2\sin x-2x^4+x^{-2021}\right)\mathrm{\,d}x}\\
			&= & x-2\cos x-\dfrac{2}{5}{x^5}-\dfrac{1}{2020x^{2020}}+C.
		\end{eqnarray*}
	}
\end{ex}	
\begin{ex}%[2D4V1-3][Lê Công Trường]
	Tính $\displaystyle\int{x^2\left[\dfrac{1}{x^2\sin^2\dfrac{x}{2}\cdot\cos^2\dfrac{x}{2}}+\dfrac{3}{x^3}-\dfrac{4}{x^4}\right]}\mathrm{\,d}x=a\cot{x}+b\ln \left| x\right|+\dfrac{c}{x}+C$. Giá trị $a+b+c$ là
	\shortans{$3$} 
	\loigiai{
		Ta có\\
		$\dfrac{1}{\sin^2\dfrac{x}{2}\cdot\cos^2\dfrac{x}{2}}=\dfrac{1}{\left(\sin\dfrac{x}{2}\cdot\,\cos\dfrac{x}{2}\right)^2}=\dfrac{1}{\left(\dfrac{\sin x}{2}\right)^2}=\dfrac{4}{\sin^2x}.$\\
		$x^2\left[\dfrac{1}{x^2\sin^2\dfrac{x}{2}\cdot\cos^2\dfrac{x}{2}}+\dfrac{3}{x^3}-\dfrac{4}{x^4}\right]=\dfrac{1}{\sin^2\dfrac{x}{2}\cdot\cos^2\dfrac{x}{2}}+\dfrac{3}{x}-\dfrac{4}{x^2}=\dfrac{4}{\sin^2x}+\dfrac{3}{x}-\dfrac{4}{x^2}$.\\
		Khi đó
		\begin{eqnarray*}
			\displaystyle\int{x^2\left[\dfrac{1}{x^2\sin^2\dfrac{x}{2}\cdot\cos^2\dfrac{x}{2}}+\dfrac{3}{x^3}-\dfrac{4}{x^4}\right]}\mathrm{\,d}x	&= & \displaystyle\int{\left(\dfrac{4}{\sin^2x}+\dfrac{3}{x}-\dfrac{4}{x^2}\right)}\mathrm{\,d}x\\
			&= & -4\cot x+3\ln \left| x\right|+\dfrac{4}{x}+C.
		\end{eqnarray*}
	}
\end{ex}
\Closesolutionfile{ans}
\indapan{6}{ans/ans-2-B1-D2-KQ}

\Opensolutionfile{ans}[ans/ans-C4B1CD1-LC]
% \TN
\begin{ex}%[2D4N2-4]
	Họ nguyên hàm của hàm số $f(x)=e^{3x}$ là hàm số nào sau đây?
	\choice
	{$3e^x+C$}
	{\True $\dfrac{1}{3}e^{3x}+C$}
	{$\dfrac{1}{3}e^{x}+C$}
	{$3e^{3x}+C$}
	\loigiai{
		\textbf{Cách 1:} $\displaystyle\int e^{3x} \mathrm{\,d}x=\displaystyle\int (e^{3})^x \mathrm{\,d}x=\dfrac{(e^3)^x}{\ln e^3}+C=\dfrac{e^{3x}}{3}+C$.\\
		\textbf{Cách 2 (Trắc nghiệm): } $\displaystyle\int e^{3x} \mathrm{\,d}x=\dfrac{1}{3}e^{3x}+C$, với $C$ là hằng số bất kì.
	}
\end{ex}

\begin{ex}%[2D4N2-4]
	Nguyên hàm của hàm số $y=e^{2x-1}$  là
	\choice
	{$2e^{2x-1}+C$}
	{$e^{2x-1}+C$}
	{\True $\dfrac{1}{2}e^{2x-1}+C$}
	{$\dfrac{1}{2}e^{x}+C$}
	\loigiai{
		\textbf{Cách 1:} $\displaystyle\int e^{2x-1} \mathrm{\,d}x=\displaystyle\int e^{-1}(e^{2})^x \mathrm{\,d}x=e^{-1}\dfrac{(e^2)^x}{\ln e^2}+C=\dfrac{e^{2x-1}}{2}+C$.\\
		\textbf{Cách 2:} $\displaystyle\int e^{2x-1} \mathrm{\,d}x=\dfrac{1}{2}\displaystyle\int e^{2x-1} \mathrm{\,d}(2x-1)=\dfrac{1}{2}e^{2x-1}+C$.
	}
\end{ex}

\begin{ex}%[2D4N2-4]
	Cho hàm số $f(x)=e^x+2$. Khẳng định nào dưới đây là \textbf{đúng}?
	\choice
	{$\displaystyle\int f(x) \mathrm{\,d}x=e^{x-2}+C$}
	{\True $\displaystyle\int f(x) \mathrm{\,d}x=e^{x}+2x+C$}
	{$\displaystyle\int f(x) \mathrm{\,d}x=e^{x}+C$}
	{$\displaystyle\int f(x) \mathrm{\,d}x=e^{x}-2x+C$}
	\loigiai{
		Ta có $\displaystyle\int f(x) \mathrm{\,d}x=\displaystyle\int (e^x+2) \mathrm{\,d}x=e^x+2x+C$.
	}
\end{ex}

\begin{ex}%[2D4N2-4]
	Cho hàm số $f(x)=e^x+2x$. Khẳng định nào dưới đây \textbf{đúng}?
	\choice
	{\True $\displaystyle\int f(x) \mathrm{\,d}x=e^{x}+x^2+C$}
	{$\displaystyle\int f(x) \mathrm{\,d}x=e^{x}+C$}
	{$\displaystyle\int f(x) \mathrm{\,d}x=e^{x}-x^2+C$}
	{$\displaystyle\int f(x) \mathrm{\,d}x=e^{x}+2x^2+C$}
	\loigiai{
		Ta có $\displaystyle\int f(x) \mathrm{\,d}x=\displaystyle\int (e^x+2x) \mathrm{\,d}x=e^x+x^2+C$.
	}
\end{ex}

\begin{ex}%[2D4N2-4]
	Tìm nguyên hàm của hàm số  $f(x)=7^x$.
	\choice
	{\True $\displaystyle\int 7^x \mathrm{\,d}x=\dfrac{7^x}{\ln 7}+C$}
	{$\displaystyle\int 7^x \mathrm{\,d}x=7^{x+1}+C$}
	{$\displaystyle\int 7^x \mathrm{\,d}x=\dfrac{7^{x+1}}{x+1}+C$}
	{$\displaystyle\int 7^x \mathrm{\,d}x=7^x\ln 7+C$}
	\loigiai{
		Ta có $\displaystyle\int 7^x \mathrm{\,d}x=\dfrac{7^x}{\ln 7}+C$.
	}
\end{ex}

\begin{ex}%[2D4N2-4]
	Nguyên hàm của hàm số  $f(x)=2^x$ là
	\choice
	{$\displaystyle\int 2^x \mathrm{\,d}x=\ln 2\cdot 2^x+C$}
	{$\displaystyle\int 2^x \mathrm{\,d}x=2^x+C$}
	{\True $\displaystyle\int 2^x \mathrm{\,d}x=\dfrac{2^{x}}{\ln 2}+C$}
	{$\displaystyle\int 2^x \mathrm{\,d}x=\dfrac{2^x}{x+1}\ln 7+C$}
	\loigiai{
		Ta có $\displaystyle\int 2^x \mathrm{\,d}x=\dfrac{2^x}{\ln 2}+C$.
	}
\end{ex}

\begin{ex}%[2D4N2-4]
	Tất cả các nguyên hàm của hàm số  $f(x)=3^{-x}$ là
	\choice
	{\True $-\dfrac{3^{-x}}{\ln 3}+C$}
	{$-3^{-x}+C$}
	{$-3^{-x}\ln 3+C$}
	{$\dfrac{3^{-x}}{\ln 3}+C$}
	\loigiai{
		Ta có $\displaystyle\int 3^{-x} \mathrm{\,d}x=\displaystyle\int (3^{-1})^{x} \mathrm{\,d}x=-\dfrac{3^{-x}}{\ln 3}+C$.
	}
\end{ex}

\begin{ex}%[2D4N2-4]
	Tìm nguyên hàm của hàm số $f(x)=3^x+2x$.
	\choice
	{\True $\displaystyle\int (3^x+2x) \mathrm{\,d}x=\dfrac{3^x}{\ln 3}+x^2+C$}
	{$\displaystyle\int (3^x+2x) \mathrm{\,d}x=3^x\ln 3+x^2+C$}
	{$\displaystyle\int (3^x+2x) \mathrm{\,d}x=\dfrac{3^x}{\ln 3}+x+C$}
	{$\displaystyle\int (3^x+2x) \mathrm{\,d}x=3^x\ln 3+x+C$}
	\loigiai{
		Ta có $\displaystyle\int (3^x+2x) \mathrm{\,d}x=\dfrac{3^x}{\ln 3}+x^2+C$.
	}
\end{ex}

\begin{ex}%[2D4N2-4]
	Họ nguyên hàm của hàm số $f(x)=e^x-2x$ là
	\choice
	{$e^x+x^2+C$}
	{\True $e^x-x^2+C$}
	{$\dfrac{1}{x+1}e^x-x^2+C$}
	{$e^x-2+C$}
	\loigiai{
		Ta có $\displaystyle\int (e^x-2x) \mathrm{\,d}x=e^x-x^2+C$.
	}
\end{ex}

\begin{ex}%[2D4H2-4]
	Tìm nguyên hàm của hàm số $f(x)=e^x\left(2017-\dfrac{2018e^{-x}}{x^5}\right) $.
	\choice
	{$\displaystyle\int f(x) \mathrm{\,d}x=2017e^x-\dfrac{2018}{x^4}+C$}
	{$\displaystyle\int f(x) \mathrm{\,d}x=2017e^x+\dfrac{2018}{x^4}+C$}
	{\True $\displaystyle\int f(x) \mathrm{\,d}x=2017e^x+\dfrac{504{,}5}{x^4}+C$}
	{$\displaystyle\int f(x) \mathrm{\,d}x=2017e^x-\dfrac{504{,}5}{x^4}+C$}
	\loigiai{
		\begin{eqnarray*}
			\displaystyle\int f(x) \mathrm{\,d}x
			&=&\displaystyle\int e^x\left(2017-\dfrac{2018e^{-x}}{x^5}\right)\mathrm{\,d}x\\
			&=&\displaystyle\int \left(2017e^x-\dfrac{2018}{x^5}\right)\mathrm{\,d}x\\
			&=&2017e^x+\dfrac{504{,}5}{x^4}+C
		\end{eqnarray*}
	}
\end{ex}

\begin{ex}%[2D4H2-4]
	Họ nguyên hàm của hàm số $y=e^x\left(2+\dfrac{e^{-x}}{\cos^2x}\right) $ là
	\choice
	{\True $2e^x+\tan x+C$}
	{$2e^x-\tan x+C$}
	{$2e^x-\dfrac{1}{\cos x}+C$}
	{$2e^x+\dfrac{1}{\cos x}+C$}
	\loigiai{
		Ta có $\displaystyle\int y \mathrm{\,d}x=\displaystyle\int e^x\left(2+\dfrac{e^{-x}}{\cos^2x}\right)\mathrm{\,d}x=\displaystyle\int \left(2e^x+\dfrac{1}{\cos^2x}\right)\mathrm{\,d}x=2e^x+\tan x+C$.
	}
\end{ex}

\begin{ex}%[2D4N2-4]
	Tìm họ nguyên hàm của hàm số $y=x^2-3^x+\dfrac{1}{x}$.
	\choice
	{$\dfrac{x^3}{3}-\dfrac{3^x}{\ln 3}-\dfrac{1}{x^2}+C,\,C\in \mathbb{R}$}
	{$\dfrac{x^3}{3}-3^x+\dfrac{1}{x^2}+C,\,C\in \mathbb{R}$}
	{\True $\dfrac{x^3}{3}-\dfrac{3^x}{\ln 3}+\ln \left|x\right|+C,\,C\in \mathbb{R}$}
	{$\dfrac{x^3}{3}-\dfrac{3^x}{\ln 3}-\ln \left|x\right|+C,\,C\in \mathbb{R}$}
	\loigiai{
		Ta có $\displaystyle\int \left( x^2-3^x+\dfrac{1}{x}\right)  \mathrm{\,d}x=\dfrac{x^3}{3}-\dfrac{3^x}{\ln 3}+\ln \left|x\right|+C,\,C\in \mathbb{R}$.
	}
\end{ex}

\begin{ex}%[2D4N2-4]
	Khẳng định nào dưới đây \textbf{đúng}?
	\choice
	{$\displaystyle\int e^x \mathrm{\,d}x=xe^x+C$}
	{$\displaystyle\int e^x \mathrm{\,d}x=e^{x+1}+C$}
	{$\displaystyle\int e^x \mathrm{\,d}x=-e^{x+1}+C$}
	{\True $\displaystyle\int e^x \mathrm{\,d}x=e^x+C$}
	\loigiai{
		Ta có $\displaystyle\int e^x \mathrm{\,d}x=e^x+C$.
	}
\end{ex}

\begin{ex}%[2D4N2-4]
	Cho hàm số $f(x)=1+e^{2x}$. Khẳng định nào dưới đây \textbf{đúng}?
	\choice
	{$\displaystyle\int f(x) \mathrm{\,d}x=x+\dfrac{1}{2}e^x+C$}
	{$\displaystyle\int f(x) \mathrm{\,d}x=x+2e^{2x}+C$}
	{\True $\displaystyle\int f(x) \mathrm{\,d}x=x+\dfrac{1}{2}e^{2x}+C$}
	{$\displaystyle\int f(x) \mathrm{\,d}x=x+e^{2x}+C$}
	\loigiai{
		Ta có $\displaystyle\int (1+e^{2x}) \mathrm{\,d}x=x+\dfrac{1}{2}e^{2x}+C$.
	}
\end{ex}
\Closesolutionfile{ans}
\indapan{6}{ans/ans-C4B1CD1-LC}
% \TNTF
\Opensolutionfile{ans}[ans/ans-C4B1CD1-DS]
\begin{ex}%[2D4N2-4]
	Các mệnh đề sau đây \textbf{đúng} hay \textbf{sai}?
	\choiceTF
	{$\displaystyle\int \dfrac{1}{x} \mathrm{\,d}x=\ln x+C$}
	{\True $\displaystyle\int \dfrac{1}{\cos^2x} \mathrm{\,d}x=\tan x+C$}
	{\True $\displaystyle\int \sin x \mathrm{\,d}x=-\cos x+C$}
	{\True $\displaystyle\int e^x \mathrm{\,d}x=e^x+C$}
	\loigiai{
		\begin{itemchoice}
			\itemch Ta có $\displaystyle\int \dfrac{1}{x} \mathrm{\,d}x=\ln \left|x\right|+C$.
			\itemch Ta có $\displaystyle\int \dfrac{1}{\cos^2x} \mathrm{\,d}x=\tan x+C$
			\itemch Ta có $\displaystyle\int \sin x \mathrm{\,d}x=-\cos x+C$.
			\itemch Ta có $\displaystyle\int e^x \mathrm{\,d}x=e^x+C$.
		\end{itemchoice}
	}
\end{ex}

\begin{ex}%[2D4N2-4]
	Các mệnh đề sau đây \textbf{đúng} hay \textbf{sai}?
	\choiceTF
	{\True $\displaystyle\int \cos x \mathrm{\,d}x=\sin x+C$}
	{\True $\displaystyle\int x^e \mathrm{\,d}x=\dfrac{x^{e+1}}{e+1}+C$}
	{\True $\displaystyle\int \dfrac{1}{x} \mathrm{\,d}x=\ln \left|x\right|+C$}
	{$\displaystyle\int e^x \mathrm{\,d}x=\dfrac{e^{x+1}}{x+1}+C$}
	\loigiai{
		\begin{itemchoice}
			\itemch Ta có $\displaystyle\int \cos x \mathrm{\,d}x=\sin x+C$.
			\itemch Ta có $\displaystyle\int x^e \mathrm{\,d}x=\dfrac{x^{e+1}}{e+1}+C$
			\itemch Ta có $\displaystyle\int \dfrac{1}{x} \mathrm{\,d}x=\ln \left|x\right|+C$.
			\itemch Ta có $\displaystyle\int e^x \mathrm{\,d}x=e^x+C$.
		\end{itemchoice}
	}
\end{ex}

\begin{ex}%[2D4H2-4]
	Các mệnh đề sau đây \textbf{đúng} hay \textbf{sai}?
	\choiceTF
	{$\displaystyle\int 2^x \mathrm{\,d}x=2^x\ln 2+C$}
	{\True $\displaystyle\int e^{2x} \mathrm{\,d}x=\dfrac{e^{2x}}{2}+C$}
	{$\displaystyle\int e^x(e^x-1) \mathrm{\,d}x=\dfrac{1}{2}e^{2x}+e^x+C$}
	{\True $\displaystyle\int e^{3x}\cdot 3^x \mathrm{\,d}x=\dfrac{(3e^{3})^x}{3+\ln 3}+C$}
	\loigiai{
		\begin{itemchoice}
			\itemch Ta có $\displaystyle\int 2^x \mathrm{\,d}x=\dfrac{2^x}{\ln 2}+C$.
			\itemch Ta có $\displaystyle\int e^{2x} \mathrm{\,d}x=\dfrac{e^{2x}}{2}+C$
			\itemch Ta có $\displaystyle\int e^x(e^x-1) \mathrm{\,d}x=\displaystyle\int (e^{2x}-e^x) \mathrm{\,d}x=\dfrac{1}{2}e^{2x}-e^x+C$.
			\itemch Ta có $\displaystyle\int e^{3x}\cdot 3^x \mathrm{\,d}x=\displaystyle\int (3e^{3})^x \mathrm{\,d}x=\dfrac{(3e^{3})^x}{\ln (3e^3)}+C=\dfrac{(3e^{3})^x}{3+\ln (3)}+C$.
		\end{itemchoice}
	}
\end{ex}
\Closesolutionfile{ans}
\indapan{3}{ans/ans-C4B1CD1-DS}
% \TNSA
\Opensolutionfile{ans}[ans/ans-C4B1CD1-KQ]
\begin{ex}%[2D4H2-4]
	Biết rằng $\displaystyle\int (2^x+3^x) \mathrm{\,d}x=\dfrac{2^x}{\ln a}+\dfrac{3^x}{\ln b}+C,\,a,b\in \mathbb{Z}$. Tính $P=a+b$.
	\shortans[4]{$5$}
	\loigiai{
		Ta có $\displaystyle\int (2^x+3^x) \mathrm{\,d}x=\dfrac{2^x}{\ln 2}+\dfrac{3^x}{\ln 3}+C$.\\
		Do đó $a=2$, $b=3\Rightarrow P=a+b=2+3=5$.
	}
\end{ex}

\begin{ex}%[2D4H2-4]
	Cho $\displaystyle\int e^{3x+2024} \mathrm{\,d}x=\dfrac{a}{b}e^{cx+d}+C$ với $a,b,c,d\in \mathbb{Z}$ và $\dfrac{a}{b}$ là phân số tối giãn . Tính giá trị của biểu thức $P=a+b-c+d$.
	\shortans[4]{$2025$}
	\loigiai{
		Ta có $\displaystyle\int e^{3x+2024} \mathrm{\,d}x=\dfrac{1}{3}e^{3x+2024}+C$.\\
		Do đó $a=1$, $b=3$, $c=3$, $d=2024\Rightarrow P=a+b-c+d=1+3-3+2024=2025$.
	}
\end{ex}

\begin{ex}%[2D4H2-4]
	Biết rằng $\displaystyle\int 3^{x+2}\cdot 2^{2x+1} \mathrm{\,d}x=\dfrac{a\cdot 12^x}{b\ln 2+c\ln 3}+C$ với $a,b,c\in \mathbb{Z}$. Tính giá trị của biểu thức $P=\dfrac{a}{b+c}$.
	\shortans[4]{$6$}
	\loigiai{
		Ta có $\displaystyle\int 3^{x+2}\cdot 2^{2x+1} \mathrm{\,d}x=\displaystyle\int 3^2\cdot 3^x\cdot 2\cdot4^x \mathrm{\,d}x=\displaystyle\int 18\cdot 12^x \mathrm{\,d}x=18\cdot \dfrac{12^x}{\ln 12}+C=\dfrac{18\cdot 12^x}{2\ln 2+\ln 3}+C$.\\
		Do đó $a=18$, $b=2$, $c=1\Rightarrow P=\dfrac{a}{b+c}=\dfrac{18}{2+1}=6$.
	}
\end{ex}

\begin{ex}%[2D4H2-4]
	Biết rằng $\displaystyle\int (3^{x}+5^{x})^2\mathrm{\,d}x=\dfrac{9^x}{a\ln 3}+\dfrac{30^x}{b\ln 5+c\ln 2+d\ln 3}+\dfrac{25^x}{e\ln 5}+C$. Tính giá trị của biểu thức $P=a+b+c+d+e$.
	\shortans[4]{$7$}
	\loigiai{
		\begin{eqnarray*}
			\displaystyle\int (3^{x}+5^{x})\mathrm{\,d}x&=&\displaystyle\int (9^{x}+30^{x}+25^{x})\mathrm{\,d}x\\
			&=&
			\dfrac{9^x}{\ln 9}+\dfrac{30^x}{\ln 30+\ln 25}+C\\
			&=&\dfrac{9^x}{2\ln 3}+\dfrac{30^x}{\ln 5+\ln 2+\ln 3}+\dfrac{25^x}{2\ln 5}+C.
		\end{eqnarray*}
		Do đó $a=2$, $b=c=d=1$, $e=2\Rightarrow P=a+b+c+d+e=7$.
	}
\end{ex}

\begin{ex}%[2D4H2-4]
	Cho $\displaystyle\int \dfrac{e^{3x}+1}{e^x+1}\mathrm{\,d}x=\dfrac{a}{b}e^{2x}+ce^x+dx+C$ với $a,b,c,d\in \mathbb{Z}$ và $\dfrac{a}{b}$ là phân số tối giãn. Tính giá trị của biểu thức $P=a^2+b^2+c^2+d^2$.
	\shortans[4]{$7$}
	\loigiai{
		Ta có $\displaystyle\int \dfrac{e^{3x}+1}{e^x+1}\mathrm{\,d}x=\displaystyle\int \dfrac{(e^{x}+1)(e^{2x}-e^x+1)}{e^x+1}\mathrm{\,d}x=\displaystyle\int (e^{2x}-e^x+1)\mathrm{\,d}x=\dfrac{1}{2}e^{2x}-e^x+x+C$.\\
		Do đó $a=d=1$, $b=2$, $c=-1\Rightarrow P=a^2+b^2+c^2+d^2=7$.
	}
\end{ex}

\begin{ex}%[2D4H2-4]
	Biết rằng $\displaystyle\int (e^x+e^{-x})^2\mathrm{\,d}x=\dfrac{1}{m}e^{2x}+\dfrac{1}{n}e^{-2x}+px+C$ với $m,m,p\in \mathbb{Z}$. Tính giá trị của biểu thức $P=m+n+p$.
	\shortans[4]{$2$}
	\loigiai{
		Ta có $\displaystyle\int (e^x+e^{-x})^2\mathrm{\,d}x=\displaystyle\int (e^{2x}+e^{-2x}+2)\mathrm{\,d}x=\dfrac{1}{2}e^{2x}-\dfrac{1}{2}e^{2x}+2x+C$.\\
		Do đó $m=p=2$, $n=-2\Rightarrow P=m+n+p=2$.
	}
\end{ex}

\begin{ex}%[2D4H2-4]
	Biết rằng $\displaystyle\int \dfrac{e^{2x}-1}{1-e^{-x}}\mathrm{\,d}x=\dfrac{1}{m}e^{nx}+pe^x+C$ với $m,m,p\in \mathbb{Z}$. Tính giá trị của biểu thức $P=m+n-p$.
	\shortans[4]{$5$}
	\loigiai{
		Ta có $\displaystyle\int \dfrac{e^{2x}-1}{1-e^{-x}}\mathrm{\,d}x=\displaystyle\int \dfrac{e^x(e^x-1)(e^x+1)}{e^x-1}\mathrm{\,d}x=\displaystyle\int e^x(e^x-1) \mathrm{\,d}x$\\$=\displaystyle\int (e^{2x}-e^x) \mathrm{\,d}x=\dfrac{1}{2}e^{2x}-e^x+C$.\\
		Do đó $m=n=2$, $p=-1\Rightarrow P=m+n-p=5$.
	}
\end{ex}

\begin{ex}%[2D4H2-4]
	Biết rằng $F(x)=(ax+b)\cdot e^x$ là một nguyên hàm của hàm số $f(x)=(4x-1)\cdot e^x$. Tính giá trị biểu thức $P=a+b$.
	\shortans[4]{$-1$}
	\loigiai{
		Ta có $F'(x)=a\cdot e^x+(ax+b)\cdot e^x=e^x(ax+a+b)$.\\
		Mà $F'(x)=f(x)\Rightarrow \heva{&a=4\\&a+b=-1}\Rightarrow \heva{&a=4\\&b=-5.}$\\
		Vậy $P=a+b=-1$.
	}
\end{ex}

\begin{ex}%[2D4H2-4]
	Biết rằng $F(x)=8e^x+\dfrac{na^x}{\ln a}+p\cos x$ (với $m,n,p\in \mathbb{Z}$) là một nguyên hàm của hàm số $f(x)=me^x+2a^x-2\sin x$. Tính giá trị của biểu thức $P=m+n+p$.
	\shortans[4]{$12$}
	\loigiai{
		Ta có $F'(x)=8e^x+\dfrac{na^x}{\ln a}\cdot \ln a-p\sin x=8e^x+na^x-p\sin x$.\\
		Mà $F'(x)=f(x)\Rightarrow m=8, n=2, p=2$.\\
		Vậy $P=m+n+p=12$.
	}
\end{ex}

\begin{ex}%[2D4H2-4]
	Biết rằng  $F(x)=(ax^2+bx+c)e^{-2x}$ (với $a,b,c\in \mathbb{R}$) là một nguyên hàm của hàm số $f(x)=(-2x^2+8x-7)e^{-2x}$. Tính giá trị biểu thức $P=a+b+c$.
	\shortans[4]{$-7$}
	\loigiai{
		Ta có $F'(x)=(2ax+b)e^{-2x}-2(ax^2+bx+c)e^{-2x}=\left[-2ax^2+2(a-b)x+b-2c \right]e^{-2x}$.\\
		Mà $F'(x)=f(x)\Rightarrow \heva{&-2a=-2\\&2(a-b)=8\\&b-2c=7}\Rightarrow \heva{&a=1\\&b=-3\\&c=-5.}$\\
		Vậy $P=a+b+c=1-3-5=-7$.
	}
\end{ex}
\Closesolutionfile{ans}
\indapan{6}{ans/ans-C4B1CD1-KQ}
