\setcounter{demsochude}{2}
\chude{NGUYÊN HÀM HÀM ẨN}

\subsubsection*{Cần nhớ các công thức đạo hàm của hàm hợp}
\begin{itemize}[\color{blue}\faPencilSquare]
	\item $\int{f'(x)\mathrm{d}x}=f(x)+C$
	\item $f'(x)\cdot g(x)+f(x)\cdot g'(x)=\left[f(x)\cdot g(x)\right]'$	
	\item $\dfrac{f'(x)\cdot g(x)-f(x)\cdot g'(x)}{g^2(x)} =\left[\dfrac{f(x)}{g(x)}\right]'$
	\item $\dfrac{f'(x)}{f(x)}=\left[\ln f(x) \right]'$
	\item $-\dfrac{f'(x)}{f^2(x)}=\left[ \dfrac{1}{f(x)} \right]'$				
	\item $-\dfrac{f'(x)}{f^n(x)}=\left[ \dfrac{1}{(n-1)\left[ f(x) \right]^{n-1}} \right]'$
	\item $n\cdot f'(x)\cdot f^{n-1}(x)=\left[ f^n(x) \right]'$
	\item $\dfrac{f'(x)}{\sqrt{f(x)}}=\left[ 2\sqrt{f(x)} \right]'$
\end{itemize}

\begin{dang}{Dạng 1}
\subsubsection{Điều kiện hàm ẩn có dạng}
	$$\left[ \begin{aligned}
		& f'(x)=g(x)\cdot h\left[ f(x) \right] \\ 
		& f'(x)\cdot h\left[ f(x) \right]=g(x). 
	\end{aligned} \right.$$
\subsubsection*{Phương pháp giải}
	\begin{itemize}[\color{blue}\faPencilSquareO]
		\item $\dfrac{f'(x)}{h[f(x)]}=g(x) \Leftrightarrow \displaystyle\int\!\dfrac{f'(x)}{h[f(x)]}\mathrm{d}x =\int\!\!{g(x)}\mathrm{d}x \Leftrightarrow \int\dfrac{\mathrm{d}\left[ f(x) \right]}{h\left[ f(x) \right]} =\int\!\!{g(x)\mathrm{d}x}$.
		\item $f'(x)h[f(x)]=g(x)
		\Leftrightarrow 
		\displaystyle\int\!f'(x)h[f(x)]\mathrm{d}x=\int\!\!g(x)\mathrm{d}x
		\Leftrightarrow 
		\int\!h[f(x)]\mathrm{d}\left[ f'(x) \right]=\int\!\!g(x)\mathrm{d}x$.
	\end{itemize}
	Chú ý: Ngoài việc nghuyên hàm hai vế, ta có thể lấy tích phân hai vế (tùy câu hỏi của bài toán)
\subsubsection{Điều kiện hàm ẩn có dạng}
	$$\hoac{&f'(x)+p(x)\cdot f(x)=0\\&f'(x)+p'(x)\cdot\left[f(x)\right]^n=0.}$$
\subsubsection*{Phương pháp giải}
	\begin{itemize}[\color{magenta}\faPencilSquareO]
		\item $f'(x)+p(x)\cdot f(x)=0$\\
		Chia hai vế với $f(x)$ ta đựơc $\dfrac{f'(x)}{f(x)}+p(x)=0 \Leftrightarrow \dfrac{f'(x)}{f(x)}=-p(x)$.\\
		Suy ra $\displaystyle\int\!\dfrac{f'(x)}{f(x)}\mathrm{d}x=-\int\!\!p(x)\mathrm{d}x \Leftrightarrow \ln |f(x)|=-\int\!\!p(x)\mathrm{d}x$.\\
		Từ đây ta dễ dàng tính được $f(x)$.
		\item $f'(x)+p(x)\cdot \left[ f(x) \right]^n=0$ \\
		Chia hai vế với $\left[ f(x) \right]^n$ ta được $\dfrac{f'(x)}{\left[ f(x) \right]^n}+p(x)=0 \Leftrightarrow \dfrac{f'(x)}{\left[ f(x) \right]^n}=-p(x)$.\\
		Suy ra $\displaystyle\int\!\dfrac{f'(x)}{\left[ f(x) \right]^n}\mathrm{d}x =-\int\!\!p(x)\mathrm{d}x \Leftrightarrow \dfrac{\left[ f(x) \right]^{-n+1}}{-n+1} =-\int\!\!p(x)\mathrm{d}x$.
	\end{itemize}
\subsubsection{Điều kiện hàm ẩn có dạng}
	\centerline{$u(x)f'(x)+u'(x)f(x)=h(x)$}
\subsubsection*{Phương pháp giải}
	Dễ dàng thấy rằng $u(x)f'(x)+u'(x)f(x)=[u(x)f(x)]'$.\\
	Do dó $u(x)f'(x)+u'(x)f(x)=h(x) \Leftrightarrow [u(x)f(x)]'=h(x)$.\\
	Suy ra $u(x)f(x)=\displaystyle\int\!\! h(x)\mathrm{d}x$.\\
	Từ đây ta dễ dàng tính được $f(x)$.
\end{dang}

%PHẦN I. Câu trắc nghiệm nhiều phương án lựa chọn. Mỗi câu hỏi thí sinh chỉ chọn một phương án.
\TN
\Opensolutionfile{ans}[ans/ans-LC-2-C4B1CD3_1-8]

%%%==============EX_1============%%%
\begin{ex}%[2D4V1-3]
	Cho hàm số $f(x)$ thỏa mãn $f\left(\dfrac{\pi}{4} \right)=0$ và $f'(x)\sin^2\dfrac{x}{2}\cos^2\dfrac{x}{2}=1$. Tính $f\left(\dfrac{\pi}{2} \right)$.
	\choice
	{$f\left(\dfrac{\pi}{2} \right)=1$}
	{$f\left(\dfrac{\pi}{2} \right)=-1$}
	{$f\left(\dfrac{\pi}{2} \right)=2$}
	{\True $f\left(\dfrac{\pi}{2} \right)=4$}
	\loigiai{
		Ta có
		$\begin{aligned}[t]
			&\quad f'(x)\sin^2\dfrac{x}{2}\cos^2\dfrac{x}{2}=1\\
			\Rightarrow&\quad f'(x)=\dfrac{1}{\sin^2\dfrac{x}{2}\cos^2\dfrac{x}{2}}\\
			\Rightarrow&\quad f'(x)=\dfrac{1}{\tfrac{1}{4}\sin^2}x\\
			\Rightarrow&\quad f(x)=4\displaystyle\int\!\!\dfrac{1}{\sin^2x}\mathrm{d}x.
		\end{aligned}$\\
		Tìm được $f(x)=-4\cot x+C$.\\ 
		Với $f\left(\dfrac{\pi}{4} \right)=0$ thì $C=4$.\\
		Suy ra $f(x)=-4\cot x+4$.\\ 
		Vậy $f\left(\dfrac{\pi}{2}\right) =-4\cot\dfrac{\pi}{2}+4=4$.
	}
\end{ex}
%%%==============EX_2============%%%
\begin{ex}%[2D4V1-2]
	Cho hàm số $y=f(x)$ thỏa mãn $f'(x)\cdot f(x)=x^4+x^2$. Biết $f(0)=2$. Tính $f^2(2)$.
	\choice
	{$f^2(2)=\dfrac{313}{15}$}
	{\True $f^2(2)=\dfrac{332}{15}$}
	{$f^2(2)=\dfrac{324}{15}$}
	{$f^2(2)=\dfrac{323}{15}$}
	\loigiai{
		Ta có
		$\begin{aligned}[t]
			f'(x)\cdot f(x)=x^4+x^2
			& \Leftrightarrow f(x)\mathrm{d}f(x)=(x^4+x^3)\mathrm{d}x\\
			& \Leftrightarrow \displaystyle\int\!f'(x)\cdot f(x)\mathrm{d}x =\int{(x^4+x^2)\mathrm{d}x}+C\\
			& \Rightarrow \dfrac{f^2(x)}{2} =\dfrac{x^5}{5}+\dfrac{x^3}{3}+C.
		\end{aligned}$\\
		Do 
		$\begin{aligned}[t]
			f(0)=2 & \Rightarrow \dfrac{f^2(0)}{2}=\dfrac{0^5}{5}+\dfrac{0^3}{3}+C
			& \Rightarrow C=2.
		\end{aligned}$\\
		Vậy $f^2(2)=2\left(\dfrac{32}{5}+\dfrac{8}{3}+2 \right) =\dfrac{332}{15}$.
	}
\end{ex}
%%%==============EX_3============%%%
\begin{ex}%[2D4V1-2]
	Cho hàm số $y=f(x)$ có đạo hàm liên tục trên đoạn $[-2;1]$ thỏa mãn $f(0)=3$ và $\left(f(x)\right)^2\cdot f'(x)=3x^2+4x+2$. Giá trị $f(1)$ là
	\choice
	{$2\sqrt[3]{42}$}
	{$2\sqrt[3]{15}$}
	{\True $\sqrt[3]{42}$}
	{$\sqrt[3]{15}$}
	\loigiai{
		Ta có $\left(f(x)\right)^2\cdot f'(x)=3x^2+4x+2$ (*).\\
		Lấy nguyên hàm 2 vế của phương trình trên ta được
		\begin{eqnarray*}
			\displaystyle\int\! f(x)^2\cdot f'(x)\mathrm{d}x =\int\left(3x^2+4x+2\right)\mathrm{d}x
			& \Leftrightarrow & \int \left(f(x)\right)^2\mathrm{d}f(x) =x^3+2x^2+2x+C\\
			& \Leftrightarrow & \dfrac{\left(f(x)\right)^3}{3} =x^3+2x^2+2x+C\\
			& \Leftrightarrow & \left(f(x)\right)^3 =3(x^3+2x^2+2x+C) \quad(1).
		\end{eqnarray*}
		Theo đề bài 
		$\begin{aligned}[t]
			f(0)=3 &\overset{(1)}{\Rightarrow} \left(f(0)\right)^3 =3(0^3+2.0^2+2\cdot0+C)\\
			&\Leftrightarrow 27=3C\\
			&\Leftrightarrow C=9.
		\end{aligned}$\\
		Suy ra
		$\begin{aligned}[t]
			& \left(f(x)\right)^3=3(x^3+2x^2+2x+9) 
			\Rightarrow&\ f(x)=\sqrt[3]{3(x^3+2x^2+2x+9)}.
		\end{aligned}$\\
		Vậy $f(1)=\sqrt[3]{42}$.
	}
\end{ex}
%%%==============EX_4============%%%
\begin{ex}%[2D4C1-2]
	Cho hàm số $f(x)$ thỏa mãn $f(2)=-\dfrac{1}{3}$ và $f'(x)=x\left[f(x)\right]^2$ với mọi $x\in \mathbb{R}$. Giá trị của $f(1)$ bằng
	\choice
	{\True $f(1)=-\dfrac{2}{3}$}
	{$f(1)=-\dfrac{2}{9}$}
	{$f(1)=-\dfrac{7}{6}$}
	{$f(1)=-\dfrac{11}{6}$}
	\loigiai{
		Từ hệ thức đề cho: $f'(x)=x\left[f(x)\right]^2$ (1), suy ra $f'(x)\ge 0$ với mọi $x\in [1;2]$ \\
		Do đó $f(x)$ là hàm không giảm trên đoạn $[1;2]$, ta có $f(x)\le f(2) < 0$ với mọi $x\in [1;2]$
		\begin{enumerate}[\color{blue}\bf Cách 1.]
			\item Lấy nguyên hàm\\
			Ta có 
			$\begin{aligned}[t]
				f'(x)=x\left[f(x)\right]^2
				&\Rightarrow \dfrac{f'(x)}{\left[f(x)\right]^2}=x\\
				&\Rightarrow \left(-\dfrac{1}{f(x)} \right)'=x\\
				&\Rightarrow \left(\dfrac{1}{f(x)} \right)'=-x\\
				&\Rightarrow \dfrac{1}{f(x)}=\displaystyle\int(-x)dx\\
				&\Rightarrow \dfrac{1}{f(x)}=-\dfrac{x^2}{2}+C.
			\end{aligned}$\\
			Mà
			$\begin{aligned}[t]
				f(2)=-\dfrac{1}{3}
				&\Rightarrow \dfrac{1}{f(2)}=-2+C\\
				&\Leftrightarrow \dfrac{1}{-\tfrac{1}{3}}=-2+C\\
				&\Rightarrow C=-1.
			\end{aligned}$\\
			Tìm được $\dfrac{1}{f(x)}=-\dfrac{x^2}{2}-1$.\\
			Cho nên $\dfrac{1}{f(1)}=-\dfrac{1}{2}-1 \Leftrightarrow f(1)=-\dfrac{2}{3}$.
			\item Chia 2 vế hệ thức $(1)$ cho $\left[f(x)\right]^2$, ta được $ \dfrac{f'(x)}{\left[f(x)\right]^2}=x,\forall x\in[1;2]$. \\
			Lấy tích phân 2 vế trên đoạn $[1;2]$ hệ thức vừa tìm được, ta được:
			\begin{align*}
				\displaystyle\int\limits_1^2\dfrac{f'(x)}{\left[f(x)\right]^2}\mathrm{d}x =\displaystyle\int\limits_1^2x\mathrm{d}x
				&\Rightarrow \int\limits_1^2\dfrac{1}{\left[f(x)\right]^2}\mathrm{d}f(x)=\dfrac{3}{2}\\
				&\Leftrightarrow \dfrac{-1}{f(x)} \bigg|_1^2=\dfrac{3}{2}\\
				&\Leftrightarrow \dfrac{1}{f(1)}-\dfrac{1}{f(2)}=\dfrac{3}{2}\\
				&\Leftrightarrow \dfrac{1}{f(1)}=\dfrac{1}{f(2)}+\dfrac{3}{2}\\
				&\Leftrightarrow f(1)=\dfrac{2f(2)}{2+3f(2)}.
			\end{align*}
			Với $f(2)=-\dfrac{1}{3}$ thì $f(1)=\dfrac{2\cdot\left(-\frac{1}{3}\right)}{2+3\cdot\left(-\frac{1}{3}\right)}=-\dfrac{2}{3}$.
		\end{enumerate}
	}
\end{ex}
%%%==============EX_5============%%%
\begin{ex}%[2D4V1-2]
	Cho hàm số $f(x)$ thỏa mãn $f(2)=-\dfrac{1}{25}$ và $f'(x)=4x^3\left[f(x)\right]^2$ với mọi $x\in\mathbb{R}$. Giá trị của $f(1)$ bằng
	\choice
	{$-\dfrac{391}{400}$}
	{$-\dfrac{1}{40}$}
	{$-\dfrac{41}{400}$}
	{\True $-\dfrac{1}{10}$}
	\loigiai{
		Ta có 
		$\begin{aligned}[t]
			f'(x)=4x^3\left[f(x)\right]^2
			&\Rightarrow-\dfrac{f'(x)}{\left[f(x)\right]^2}=-4x^3\\
			&\Rightarrow \left[\dfrac{1}{f(x)}\right]'=-4x^3\\
			&\Rightarrow \dfrac{1}{f(x)}=-x^4+C.
		\end{aligned}$\\
		Với $f(2)=-\dfrac{1}{25}$ thì $\dfrac{1}{f(2)}=-2^4+C \Leftrightarrow -25=-16+C \Leftrightarrow C=-9$. \\
		Suy ra $f(x)=-\dfrac{1}{x^4+9}$.\\
		Vậy $f(1)=-\dfrac{1}{10}$.
	}
\end{ex}
%%%==============EX_6============%%%
\begin{ex}%[2D4V1-2]
	Cho hàm số $f(x)$ thỏa mãn $f(2)=-\dfrac{1}{5}$ và $f'(x)=x^3\left[f(x)\right]^2$ với mọi $x\in\mathbb{R}$. Giá trị của $f(1)$ bằng
	\choice
	{$-\dfrac{4}{35}$}
	{$-\dfrac{71}{20}$}
	{$-\dfrac{79}{20}$}
	{\True $-\dfrac{4}{5}$}
	\loigiai{
		Ta có
		$\begin{aligned}[t]
			f'(x)=x^3\left[f(x)\right]^2
			&\Rightarrow \dfrac{f'(x)}{f^2(x)}=x^3\\
			&\Rightarrow \displaystyle\int\limits_1^2\dfrac{f'(x)}{f^2(x)}\mathrm{d}x =\int\limits_1^2x^3\mathrm{d}x\\
			&\Leftrightarrow -\dfrac{1}{f(x)} \bigg|_1^2 =\dfrac{1}{4}\cdot(2^4-1^4)\\
			&\Leftrightarrow -\dfrac{1}{f(2)}+\dfrac{1}{f(1)} =\dfrac{15}{4}\\
			&\Leftrightarrow \dfrac{1}{f(1)}=\dfrac{4+15f(2)}{4f(2)}\\
			&\Leftrightarrow f(1)=\dfrac{4f(2)}{4+15f(2)}=\dfrac{4\cdot\left(-\frac{1}{5}\right)}{4+15\cdot\left(-\frac{1}{5}\right)}=-\dfrac{4}{5}.
		\end{aligned}$\\
		Vậy $f(1)=-\dfrac{4}{5}$.
	}
\end{ex}
%%%==============EX_7============%%%
\begin{ex}%[2D4V1-2]
	Cho hàm số $y=f(x)$ thỏa mãn $f(2)=-\dfrac{4}{19}$ và $f'(x)=x^3f^2(x)\, \forall x\in \mathbb{R}$. Giá trị của $f(1)$ bằng
	\choice
	{$-\dfrac{2}{3}$}
	{$-\dfrac{1}{2}$}
	{\True $-1$}
	{$-\dfrac{3}{4}$}
	\loigiai{
		Ta có 
		$\begin{aligned}[t]
			f'(x)=x^3f^2(x) &\Leftrightarrow \dfrac{f'(x)}{f^2(x)}=x^3\\
			&\Rightarrow \displaystyle\int\dfrac{f'(x)}{f^2(x)}\mathrm{d}x =\int x^3\mathrm{d}x\\
			&\Leftrightarrow -\dfrac{1}{f(x)} =\dfrac{x^4}{4}+C.
		\end{aligned}$\\
		Với $f(2)=-\dfrac{4}{9}$ thì $\dfrac{19}{4} =\dfrac{16}{4}+C \Rightarrow C=\dfrac{3}{4}$.\\ 
		Tìm được $f(x)=-\dfrac{4}{x^4+3}$.\\
		Vậy $f(1)=-1$.
	}
\end{ex}
%%%==============EX_8============%%%
\begin{ex}%[2D4V1-4]
	Cho hàm số $f(x)>0$ xác định và liên tục trên $\mathbb{R}$ đồng thời thỏa mãn $f(0)=\dfrac{1}{2}$, $f'(x)=-\mathrm{e}^xf^2(x),\, \forall x\in \mathbb{R}$. Tính giá trị của $f(\ln 2)$.
	\choice
	{$f(\ln 2)=\dfrac{1}{4}$}
	{\True $f(\ln 2)=\dfrac{1}{3}$}
	{$f(\ln 2)=\ln 2+\dfrac{1}{2}$}
	{$f(\ln 2)=\ln ^22+\dfrac{1}{2}$}
	\loigiai{
		Ta có 
		$\begin{aligned}[t]
			f'(x)=-\mathrm{e}^xf^2(x)
			&\Leftrightarrow \dfrac{f'(x)}{f^2(x)}=-\mathrm{e}^x\ (\text{do } f(x)>0)\\
			&\Leftrightarrow \displaystyle\int\dfrac{f'(x)}{f^2(x)}\mathrm{d}x =\int(-\mathrm{e}^x)\mathrm{d}x\\
			&\Rightarrow -\dfrac{1}{f(x)}=-\mathrm{e}^x+C\\
			&\Rightarrow f(x)=\dfrac{1}{\mathrm{e}^x-C}.
		\end{aligned}$\\
		Với $f(0)=\dfrac{1}{2}$ thì $\dfrac{1}{\mathrm{e}^0-C} =\dfrac{1}{2} \Rightarrow C=-1$.\\
		Suy ra $f(x)=\dfrac{1}{\mathrm{e}^x+1}$.\\
		Vậy $f(\ln 2) =\dfrac{1}{\mathrm{e}^{\ln 2}+1}=\dfrac{1}{3}$.
	}
\end{ex}
%%%==============EX_9============%%%
\begin{ex}%[2D4V1-2]
	Cho hàm số $f(x)\ne0$ thỏa mãn điều kiện $f'(x)=(2x+3)f^2(x)$ và $f(0)=-\dfrac{1}{2}$. Biết rằng tổng $f(1)+f(2)+f(3)+\cdots+f(2024)+f(2025) =\dfrac{a}{b}$ với $\left(a\in \mathbb{Z}, b\in \mathbb{N}^{*} \right)$ và $\dfrac{a}{b}$ là phân số tối giản. Mệnh đề nào sau đây đúng?
	\choice
	{$\dfrac{a}{b} <-1$}
	{$\dfrac{a}{b} > 1$}
	{$a+b=1010$}
	{\True $b-a=1519$}
	\loigiai{
		Ta có
		$\begin{aligned}[t]
			f'(x)=(2x+3)f^2(x)
			&\Leftrightarrow \dfrac{f'(x)}{f^2(x)}=2x+3\\
			&\Leftrightarrow \displaystyle\int\dfrac{f'(x)}{f(x)}\mathrm{d}x =\int(2x+3)\mathrm{d}x\\
			&\Leftrightarrow -\dfrac{1}{f(x)}=x^2+3x+C.
		\end{aligned}$\\
		Với $f(0)=-\dfrac{1}{2}$, thì $-\dfrac{1}{-\frac{1}{2}}=0^2+3\cdot0+C \Rightarrow C=2$\\
		Suy ra $f(x)=-\dfrac{1}{(x+1)(x+2)}=\dfrac{1}{x+2}-\dfrac{1}{x+1}$\\
		Ta có: $\left\{\begin{aligned}
			& f(1)=\dfrac{1}{3}-\dfrac{1}{2} \\ 
			& f(2)=\dfrac{1}{4}-\dfrac{1}{3} \\ 
			& f(3)=\dfrac{1}{5}-\dfrac{1}{4} \\ 
			& \vdots \\ 
			& f(2025)=\dfrac{1}{2026}-\dfrac{1}{2025} \\ 
		\end{aligned} \right.$ \\
		Tổng $f(1)+f(2)+f(3)+\cdots+f(2024)+f(2025) =-\dfrac{1}{2}+\dfrac{1}{2026}=-\dfrac{506}{1013}$\\
		Do $a\in \mathbb{Z}, b\in \mathbb{N}^{*}
		\Rightarrow a=-506, b=1013\Rightarrow b-a=1519$.
	}
\end{ex}
%%%==============EX_10============%%%
\begin{ex}%[2D4V1-2]
	Cho hàm số $y=f(x)$ đồng biến trên $(0;+\infty)$; $y=f(x)$ liên tục, nhận giá trị dương trên $(0;+\infty)$ và thỏa mãn $f(3)=\dfrac{4}{9}$ và $\left[f'(x)\right]^2=xf(x)$. Tính $f(8)$.
	\choice
	{\True $f(8)=\dfrac{43-24\sqrt{3}}{9}$}
	{$f(8)=\dfrac{43+24\sqrt{3}}{9}$}
	{$f(8)=\dfrac{43-\sqrt{3}}{3}$}
	{$f(8)=\dfrac{43+\sqrt{3}}{3}$}
	\loigiai{
		Với $\forall x\in (0;+\infty)$ thì $y=f(x) > 0$; $x+1 > 0$\\
		Hàm số $y=f(x)$ đồng biến trên $(0;+\infty)$ nên $f'(x)\ge0, \forall x\in (0;+\infty)$\\
		Ta có
		$\begin{aligned}[t]
			\left[f'(x)\right]^2=xf(x)
			&\Rightarrow f'(x)=\sqrt{xf(x)}\\
			&\Rightarrow \dfrac{f'(x)}{\sqrt{f(x)}}=\sqrt{x}\\
			&\Rightarrow 2\left(\sqrt{f(x)}\right)'=\sqrt{x}\\
			&\Rightarrow \left(\sqrt{f(x)}\right)'=\dfrac{1}{2}\sqrt{x}\\
			&\Rightarrow \sqrt{f(x)}=\dfrac{1}{2}\displaystyle\int\sqrt{x}dx\\
			&\Rightarrow \sqrt{f(x)}=\dfrac{1}{3}\sqrt{x^3}+C.
		\end{aligned}$\\
		Với $f(3)=\dfrac{4}{9}$, thì $\sqrt{f(3)}=\dfrac{1}{3}\cdot\sqrt{3^3}+C \Leftrightarrow \dfrac{2}{3}=\sqrt{3}+C \Leftrightarrow C=\dfrac{2-\sqrt{3}}{3}$.\\
		Tìm được $\sqrt{f(x)}=\dfrac{1}{3}\sqrt{x^3}+\dfrac{2-3\sqrt{3}}{3}$
		$\Rightarrow f(x)=\left(\dfrac{1}{3}\sqrt{x^3}+\dfrac{2-3\sqrt{3}}{3}\right)^2$.\\
		Vậy $f(8)=\dfrac{43-24\sqrt{3}}{9}$.
	}
\end{ex}
%%%==============EX_11============%%%
\begin{ex}%[2D4V1-4]%[2D4V1-4]%[2D4H1-4]%[2D4H1-4]
	Cho hàm số $f(x)>0$ với mọi $x\in\mathbb{R}$, $f(0)=1$ và $f(x)=\sqrt{x}\cdot f'(x)$ với mọi $x\in\mathbb{R}$. Mệnh đề nào dưới đây đúng?
	\choice
	{$f(3)<2$}
	{$2<f(3)<4$}
	{\True $f(3)>6$}
	{$4<f(3)<6$}
	\loigiai{
		Ta có
		$\begin{aligned}[t]
			f(x)=\sqrt{x}\cdot f'(x)
			&\Rightarrow \dfrac{f'(x)}{f(x)}=\dfrac{1}{\sqrt{x}}\\
			&\Rightarrow \ln f(x)=\displaystyle\int\dfrac{1}{\sqrt{x}}\mathrm{d}x\\
			&\Leftrightarrow \ln f(x)=2\sqrt{x}+C\\
			&\Leftrightarrow f(x)=\mathrm{e}^{2\sqrt{x}+C}.
		\end{aligned}$\\
		Với $f(0)=1$ thì $\ln f(0)=2\sqrt{0}+C \Leftrightarrow C=0$.\\
		Suy ra $f(x)=\mathrm{e}^{2\sqrt{x}}$.\\
		Vậy $f(3)=\mathrm{e}^{2\sqrt{3}}>6$.
	}
\end{ex}
