\begin{dang}{NGUYÊN HÀM HÀM MŨ}
Nguyên hàm hàm số mũ:
\begin{enumerate}[$\bullet$]
\item  $\displaystyle\int e^x \mathrm{\,d}x=e^x+C$  
\item  $\displaystyle\int a^x \mathrm{\,d}x=\dfrac{a^x}{\ln a}+C,\,(0<a\ne 1)$  
\end{enumerate}
\textbf{\textit{Chú ý:}}
\begin{enumerate}[$\bullet$]
\item $\displaystyle\int e^{ax+b} \mathrm{\,d}x=\dfrac{1}{a}e^{ax+b}+C$. 
\item  $\displaystyle\int a^{\alpha x+\beta} \mathrm{\,d}x=\dfrac{1}{\alpha}\dfrac{a^{\alpha x+\beta}}{\ln a}+C,\,(0<a\ne 1)$   
\end{enumerate}

\end{dang}
\Opensolutionfile{ans}[ans/ans-C4B1CD1-LC]
\TN
\begin{ex}%[2D4N2-4]
Họ nguyên hàm của hàm số $f(x)=e^{3x}$ là hàm số nào sau đây?
\choice
{$3e^x+C$}
{\True $\dfrac{1}{3}e^{3x}+C$}
{$\dfrac{1}{3}e^{x}+C$}
{$3e^{3x}+C$}
\loigiai{
\textbf{Cách 1:} $\displaystyle\int e^{3x} \mathrm{\,d}x=\displaystyle\int (e^{3})^x \mathrm{\,d}x=\dfrac{(e^3)^x}{\ln e^3}+C=\dfrac{e^{3x}}{3}+C$.\\
\textbf{Cách 2 (Trắc nghiệm): } $\displaystyle\int e^{3x} \mathrm{\,d}x=\dfrac{1}{3}e^{3x}+C$, với $C$ là hằng số bất kì.
}
\end{ex}

\begin{ex}%[2D4N2-4]
Nguyên hàm của hàm số $y=e^{2x-1}$  là
\choice
{$2e^{2x-1}+C$}
{$e^{2x-1}+C$}
{\True $\dfrac{1}{2}e^{2x-1}+C$}
{$\dfrac{1}{2}e^{x}+C$}
\loigiai{
\textbf{Cách 1:} $\displaystyle\int e^{2x-1} \mathrm{\,d}x=\displaystyle\int e^{-1}(e^{2})^x \mathrm{\,d}x=e^{-1}\dfrac{(e^2)^x}{\ln e^2}+C=\dfrac{e^{2x-1}}{2}+C$.\\
\textbf{Cách 2:} $\displaystyle\int e^{2x-1} \mathrm{\,d}x=\dfrac{1}{2}\displaystyle\int e^{2x-1} \mathrm{\,d}(2x-1)=\dfrac{1}{2}e^{2x-1}+C$.
}
\end{ex}

\begin{ex}%[2D4N2-4]
Cho hàm số $f(x)=e^x+2$. Khẳng định nào dưới đây là \textbf{đúng}?
\choice
{$\displaystyle\int f(x) \mathrm{\,d}x=e^{x-2}+C$}
{\True $\displaystyle\int f(x) \mathrm{\,d}x=e^{x}+2x+C$}
{$\displaystyle\int f(x) \mathrm{\,d}x=e^{x}+C$}
{$\displaystyle\int f(x) \mathrm{\,d}x=e^{x}-2x+C$}
\loigiai{
Ta có $\displaystyle\int f(x) \mathrm{\,d}x=\displaystyle\int (e^x+2) \mathrm{\,d}x=e^x+2x+C$.
}
\end{ex}

\begin{ex}%[2D4N2-4]
Cho hàm số $f(x)=e^x+2x$. Khẳng định nào dưới đây \textbf{đúng}?
\choice
{\True $\displaystyle\int f(x) \mathrm{\,d}x=e^{x}+x^2+C$}
{$\displaystyle\int f(x) \mathrm{\,d}x=e^{x}+C$}
{$\displaystyle\int f(x) \mathrm{\,d}x=e^{x}-x^2+C$}
{$\displaystyle\int f(x) \mathrm{\,d}x=e^{x}+2x^2+C$}
\loigiai{
Ta có $\displaystyle\int f(x) \mathrm{\,d}x=\displaystyle\int (e^x+2x) \mathrm{\,d}x=e^x+x^2+C$.
}
\end{ex}

\begin{ex}%[2D4N2-4]
Tìm nguyên hàm của hàm số  $f(x)=7^x$.
\choice
{\True $\displaystyle\int 7^x \mathrm{\,d}x=\dfrac{7^x}{\ln 7}+C$}
{$\displaystyle\int 7^x \mathrm{\,d}x=7^{x+1}+C$}
{$\displaystyle\int 7^x \mathrm{\,d}x=\dfrac{7^{x+1}}{x+1}+C$}
{$\displaystyle\int 7^x \mathrm{\,d}x=7^x\ln 7+C$}
\loigiai{
Ta có $\displaystyle\int 7^x \mathrm{\,d}x=\dfrac{7^x}{\ln 7}+C$.
}
\end{ex}

\begin{ex}%[2D4N2-4]
Nguyên hàm của hàm số  $f(x)=2^x$ là
\choice
{$\displaystyle\int 2^x \mathrm{\,d}x=\ln 2\cdot 2^x+C$}
{$\displaystyle\int 2^x \mathrm{\,d}x=2^x+C$}
{\True $\displaystyle\int 2^x \mathrm{\,d}x=\dfrac{2^{x}}{\ln 2}+C$}
{$\displaystyle\int 2^x \mathrm{\,d}x=\dfrac{2^x}{x+1}\ln 7+C$}
\loigiai{
Ta có $\displaystyle\int 2^x \mathrm{\,d}x=\dfrac{2^x}{\ln 2}+C$.
}
\end{ex}

\begin{ex}%[2D4N2-4]
Tất cả các nguyên hàm của hàm số  $f(x)=3^{-x}$ là
\choice
{\True $-\dfrac{3^{-x}}{\ln 3}+C$}
{$-3^{-x}+C$}
{$-3^{-x}\ln 3+C$}
{$\dfrac{3^{-x}}{\ln 3}+C$}
\loigiai{
Ta có $\displaystyle\int 3^{-x} \mathrm{\,d}x=\displaystyle\int (3^{-1})^{x} \mathrm{\,d}x=-\dfrac{3^{-x}}{\ln 3}+C$.
}
\end{ex}

\begin{ex}%[2D4N2-4]
Tìm nguyên hàm của hàm số $f(x)=3^x+2x$.
\choice
{\True $\displaystyle\int (3^x+2x) \mathrm{\,d}x=\dfrac{3^x}{\ln 3}+x^2+C$}
{$\displaystyle\int (3^x+2x) \mathrm{\,d}x=3^x\ln 3+x^2+C$}
{$\displaystyle\int (3^x+2x) \mathrm{\,d}x=\dfrac{3^x}{\ln 3}+x+C$}
{$\displaystyle\int (3^x+2x) \mathrm{\,d}x=3^x\ln 3+x+C$}
\loigiai{
Ta có $\displaystyle\int (3^x+2x) \mathrm{\,d}x=\dfrac{3^x}{\ln 3}+x^2+C$.
}
\end{ex}

\begin{ex}%[2D4N2-4]
Họ nguyên hàm của hàm số $f(x)=e^x-2x$ là
\choice
{$e^x+x^2+C$}
{\True $e^x-x^2+C$}
{$\dfrac{1}{x+1}e^x-x^2+C$}
{$e^x-2+C$}
\loigiai{
Ta có $\displaystyle\int (e^x-2x) \mathrm{\,d}x=e^x-x^2+C$.
}
\end{ex}

\begin{ex}%[2D4H2-4]
Tìm nguyên hàm của hàm số $f(x)=e^x\left(2017-\dfrac{2018e^{-x}}{x^5}\right) $.
\choice
{$\displaystyle\int f(x) \mathrm{\,d}x=2017e^x-\dfrac{2018}{x^4}+C$}
{$\displaystyle\int f(x) \mathrm{\,d}x=2017e^x+\dfrac{2018}{x^4}+C$}
{\True $\displaystyle\int f(x) \mathrm{\,d}x=2017e^x+\dfrac{504{,}5}{x^4}+C$}
{$\displaystyle\int f(x) \mathrm{\,d}x=2017e^x-\dfrac{504{,}5}{x^4}+C$}
\loigiai{
\begin{eqnarray*}
\displaystyle\int f(x) \mathrm{\,d}x
&=&\displaystyle\int e^x\left(2017-\dfrac{2018e^{-x}}{x^5}\right)\mathrm{\,d}x\\
&=&\displaystyle\int \left(2017e^x-\dfrac{2018}{x^5}\right)\mathrm{\,d}x\\
&=&2017e^x+\dfrac{504{,}5}{x^4}+C
\end{eqnarray*}
}
\end{ex}

\begin{ex}%[2D4H2-4]
Họ nguyên hàm của hàm số $y=e^x\left(2+\dfrac{e^{-x}}{\cos^2x}\right) $ là
\choice
{\True $2e^x+\tan x+C$}
{$2e^x-\tan x+C$}
{$2e^x-\dfrac{1}{\cos x}+C$}
{$2e^x+\dfrac{1}{\cos x}+C$}
\loigiai{
Ta có $\displaystyle\int y \mathrm{\,d}x=\displaystyle\int e^x\left(2+\dfrac{e^{-x}}{\cos^2x}\right)\mathrm{\,d}x=\displaystyle\int \left(2e^x+\dfrac{1}{\cos^2x}\right)\mathrm{\,d}x=2e^x+\tan x+C$.
}
\end{ex}

\begin{ex}%[2D4N2-4]
Tìm họ nguyên hàm của hàm số $y=x^2-3^x+\dfrac{1}{x}$.
\choice
{$\dfrac{x^3}{3}-\dfrac{3^x}{\ln 3}-\dfrac{1}{x^2}+C,\,C\in \mathbb{R}$}
{$\dfrac{x^3}{3}-3^x+\dfrac{1}{x^2}+C,\,C\in \mathbb{R}$}
{\True $\dfrac{x^3}{3}-\dfrac{3^x}{\ln 3}+\ln \left|x\right|+C,\,C\in \mathbb{R}$}
{$\dfrac{x^3}{3}-\dfrac{3^x}{\ln 3}-\ln \left|x\right|+C,\,C\in \mathbb{R}$}
\loigiai{
Ta có $\displaystyle\int \left( x^2-3^x+\dfrac{1}{x}\right)  \mathrm{\,d}x=\dfrac{x^3}{3}-\dfrac{3^x}{\ln 3}+\ln \left|x\right|+C,\,C\in \mathbb{R}$.
}
\end{ex}

\begin{ex}%[2D4N2-4]
Khẳng định nào dưới đây \textbf{đúng}?
\choice
{$\displaystyle\int e^x \mathrm{\,d}x=xe^x+C$}
{$\displaystyle\int e^x \mathrm{\,d}x=e^{x+1}+C$}
{$\displaystyle\int e^x \mathrm{\,d}x=-e^{x+1}+C$}
{\True $\displaystyle\int e^x \mathrm{\,d}x=e^x+C$}
\loigiai{
Ta có $\displaystyle\int e^x \mathrm{\,d}x=e^x+C$.
}
\end{ex}

\begin{ex}%[2D4N2-4]
Cho hàm số $f(x)=1+e^{2x}$. Khẳng định nào dưới đây \textbf{đúng}?
\choice
{$\displaystyle\int f(x) \mathrm{\,d}x=x+\dfrac{1}{2}e^x+C$}
{$\displaystyle\int f(x) \mathrm{\,d}x=x+2e^{2x}+C$}
{\True $\displaystyle\int f(x) \mathrm{\,d}x=x+\dfrac{1}{2}e^{2x}+C$}
{$\displaystyle\int f(x) \mathrm{\,d}x=x+e^{2x}+C$}
\loigiai{
Ta có $\displaystyle\int (1+e^{2x}) \mathrm{\,d}x=x+\dfrac{1}{2}e^{2x}+C$.
}
\end{ex}
\Closesolutionfile{ans}
% \indapan{6}{ans/ans-C4B1CD1-LC}
\TNTF
\Opensolutionfile{ans}[ans/ans-C4B1CD1-DS]
\begin{ex}%[2D4N2-4]
Các mệnh đề sau đây \textbf{đúng} hay \textbf{sai}?
\choiceTF
{$\displaystyle\int \dfrac{1}{x} \mathrm{\,d}x=\ln x+C$}
{\True $\displaystyle\int \dfrac{1}{\cos^2x} \mathrm{\,d}x=\tan x+C$}
{\True $\displaystyle\int \sin x \mathrm{\,d}x=-\cos x+C$}
{\True $\displaystyle\int e^x \mathrm{\,d}x=e^x+C$}
\loigiai{
\begin{itemchoice}
\itemch Ta có $\displaystyle\int \dfrac{1}{x} \mathrm{\,d}x=\ln \left|x\right|+C$.
\itemch Ta có $\displaystyle\int \dfrac{1}{\cos^2x} \mathrm{\,d}x=\tan x+C$
\itemch Ta có $\displaystyle\int \sin x \mathrm{\,d}x=-\cos x+C$.
\itemch Ta có $\displaystyle\int e^x \mathrm{\,d}x=e^x+C$.
\end{itemchoice}
}
\end{ex}

\begin{ex}%[2D4N2-4]
Các mệnh đề sau đây \textbf{đúng} hay \textbf{sai}?
\choiceTF
{\True $\displaystyle\int \cos x \mathrm{\,d}x=\sin x+C$}
{\True $\displaystyle\int x^e \mathrm{\,d}x=\dfrac{x^{e+1}}{e+1}+C$}
{\True $\displaystyle\int \dfrac{1}{x} \mathrm{\,d}x=\ln \left|x\right|+C$}
{$\displaystyle\int e^x \mathrm{\,d}x=\dfrac{e^{x+1}}{x+1}+C$}
\loigiai{
\begin{itemchoice}
\itemch Ta có $\displaystyle\int \cos x \mathrm{\,d}x=\sin x+C$.
\itemch Ta có $\displaystyle\int x^e \mathrm{\,d}x=\dfrac{x^{e+1}}{e+1}+C$
\itemch Ta có $\displaystyle\int \dfrac{1}{x} \mathrm{\,d}x=\ln \left|x\right|+C$.
\itemch Ta có $\displaystyle\int e^x \mathrm{\,d}x=e^x+C$.
\end{itemchoice}
}
\end{ex}

\begin{ex}%[2D4H2-4]
Các mệnh đề sau đây \textbf{đúng} hay \textbf{sai}?
\choiceTF
{$\displaystyle\int 2^x \mathrm{\,d}x=2^x\ln 2+C$}
{\True $\displaystyle\int e^{2x} \mathrm{\,d}x=\dfrac{e^{2x}}{2}+C$}
{$\displaystyle\int e^x(e^x-1) \mathrm{\,d}x=\dfrac{1}{2}e^{2x}+e^x+C$}
{\True $\displaystyle\int e^{3x}\cdot 3^x \mathrm{\,d}x=\dfrac{(3e^{3})^x}{3+\ln 3}+C$}
\loigiai{
\begin{itemchoice}
\itemch Ta có $\displaystyle\int 2^x \mathrm{\,d}x=\dfrac{2^x}{\ln 2}+C$.
\itemch Ta có $\displaystyle\int e^{2x} \mathrm{\,d}x=\dfrac{e^{2x}}{2}+C$
\itemch Ta có $\displaystyle\int e^x(e^x-1) \mathrm{\,d}x=\displaystyle\int (e^{2x}-e^x) \mathrm{\,d}x=\dfrac{1}{2}e^{2x}-e^x+C$.
\itemch Ta có $\displaystyle\int e^{3x}\cdot 3^x \mathrm{\,d}x=\displaystyle\int (3e^{3})^x \mathrm{\,d}x=\dfrac{(3e^{3})^x}{\ln (3e^3)}+C=\dfrac{(3e^{3})^x}{3+\ln (3)}+C$.
\end{itemchoice}
}
\end{ex}
\Closesolutionfile{ans}
% \indapan{3}{ans/ans-C4B1CD1-DS}
\TNSA
\Opensolutionfile{ans}[ans/ans-C4B1CD1-KQ]
\begin{ex}%[2D4H2-4]
Biết rằng $\displaystyle\int (2^x+3^x) \mathrm{\,d}x=\dfrac{2^x}{\ln a}+\dfrac{3^x}{\ln b}+C,\,a,b\in \mathbb{Z}$. Tính $P=a+b$.
\shortans[4]{$5$}
\loigiai{
Ta có $\displaystyle\int (2^x+3^x) \mathrm{\,d}x=\dfrac{2^x}{\ln 2}+\dfrac{3^x}{\ln 3}+C$.\\
Do đó $a=2$, $b=3\Rightarrow P=a+b=2+3=5$.
}
\end{ex}

\begin{ex}%[2D4H2-4]
Cho $\displaystyle\int e^{3x+2024} \mathrm{\,d}x=\dfrac{a}{b}e^{cx+d}+C$ với $a,b,c,d\in \mathbb{Z}$ và $\dfrac{a}{b}$ là phân số tối giãn . Tính giá trị của biểu thức $P=a+b-c+d$.
\shortans[4]{$2025$}
\loigiai{
Ta có $\displaystyle\int e^{3x+2024} \mathrm{\,d}x=\dfrac{1}{3}e^{3x+2024}+C$.\\
Do đó $a=1$, $b=3$, $c=3$, $d=2024\Rightarrow P=a+b-c+d=1+3-3+2024=2025$.
}
\end{ex}

\begin{ex}%[2D4H2-4]
Biết rằng $\displaystyle\int 3^{x+2}\cdot 2^{2x+1} \mathrm{\,d}x=\dfrac{a\cdot 12^x}{b\ln 2+c\ln 3}+C$ với $a,b,c\in \mathbb{Z}$. Tính giá trị của biểu thức $P=\dfrac{a}{b+c}$.
\shortans[4]{$6$}
\loigiai{
Ta có $\displaystyle\int 3^{x+2}\cdot 2^{2x+1} \mathrm{\,d}x=\displaystyle\int 3^2\cdot 3^x\cdot 2\cdot4^x \mathrm{\,d}x=\displaystyle\int 18\cdot 12^x \mathrm{\,d}x=18\cdot \dfrac{12^x}{\ln 12}+C=\dfrac{18\cdot 12^x}{2\ln 2+\ln 3}+C$.\\
Do đó $a=18$, $b=2$, $c=1\Rightarrow P=\dfrac{a}{b+c}=\dfrac{18}{2+1}=6$.
}
\end{ex}

\begin{ex}%[2D4H2-4]
Biết rằng $\displaystyle\int (3^{x}+5^{x})^2\mathrm{\,d}x=\dfrac{9^x}{a\ln 3}+\dfrac{30^x}{b\ln 5+c\ln 2+d\ln 3}+\dfrac{25^x}{e\ln 5}+C$. Tính giá trị của biểu thức $P=a+b+c+d+e$.
\shortans[4]{$7$}
\loigiai{
\begin{eqnarray*}
\displaystyle\int (3^{x}+5^{x})\mathrm{\,d}x&=&\displaystyle\int (9^{x}+30^{x}+25^{x})\mathrm{\,d}x\\
&=&
\dfrac{9^x}{\ln 9}+\dfrac{30^x}{\ln 30+\ln 25}+C\\
&=&\dfrac{9^x}{2\ln 3}+\dfrac{30^x}{\ln 5+\ln 2+\ln 3}+\dfrac{25^x}{2\ln 5}+C.
\end{eqnarray*}
Do đó $a=2$, $b=c=d=1$, $e=2\Rightarrow P=a+b+c+d+e=7$.
}
\end{ex}

\begin{ex}%[2D4H2-4]
Cho $\displaystyle\int \dfrac{e^{3x}+1}{e^x+1}\mathrm{\,d}x=\dfrac{a}{b}e^{2x}+ce^x+dx+C$ với $a,b,c,d\in \mathbb{Z}$ và $\dfrac{a}{b}$ là phân số tối giãn. Tính giá trị của biểu thức $P=a^2+b^2+c^2+d^2$.
\shortans[4]{$7$}
\loigiai{
Ta có $\displaystyle\int \dfrac{e^{3x}+1}{e^x+1}\mathrm{\,d}x=\displaystyle\int \dfrac{(e^{x}+1)(e^{2x}-e^x+1)}{e^x+1}\mathrm{\,d}x=\displaystyle\int (e^{2x}-e^x+1)\mathrm{\,d}x=\dfrac{1}{2}e^{2x}-e^x+x+C$.\\
Do đó $a=d=1$, $b=2$, $c=-1\Rightarrow P=a^2+b^2+c^2+d^2=7$.
}
\end{ex}

\begin{ex}%[2D4H2-4]
Biết rằng $\displaystyle\int (e^x+e^{-x})^2\mathrm{\,d}x=\dfrac{1}{m}e^{2x}+\dfrac{1}{n}e^{-2x}+px+C$ với $m,m,p\in \mathbb{Z}$. Tính giá trị của biểu thức $P=m+n+p$.
\shortans[4]{$2$}
\loigiai{
Ta có $\displaystyle\int (e^x+e^{-x})^2\mathrm{\,d}x=\displaystyle\int (e^{2x}+e^{-2x}+2)\mathrm{\,d}x=\dfrac{1}{2}e^{2x}-\dfrac{1}{2}e^{2x}+2x+C$.\\
Do đó $m=p=2$, $n=-2\Rightarrow P=m+n+p=2$.
}
\end{ex}

\begin{ex}%[2D4H2-4]
Biết rằng $\displaystyle\int \dfrac{e^{2x}-1}{1-e^{-x}}\mathrm{\,d}x=\dfrac{1}{m}e^{nx}+pe^x+C$ với $m,m,p\in \mathbb{Z}$. Tính giá trị của biểu thức $P=m+n-p$.
\shortans[4]{$5$}
\loigiai{
Ta có $\displaystyle\int \dfrac{e^{2x}-1}{1-e^{-x}}\mathrm{\,d}x=\displaystyle\int \dfrac{e^x(e^x-1)(e^x+1)}{e^x-1}\mathrm{\,d}x=\displaystyle\int e^x(e^x-1) \mathrm{\,d}x$\\$=\displaystyle\int (e^{2x}-e^x) \mathrm{\,d}x=\dfrac{1}{2}e^{2x}-e^x+C$.\\
Do đó $m=n=2$, $p=-1\Rightarrow P=m+n-p=5$.
}
\end{ex}

\begin{ex}%[2D4H2-4]
Biết rằng $F(x)=(ax+b)\cdot e^x$ là một nguyên hàm của hàm số $f(x)=(4x-1)\cdot e^x$. Tính giá trị biểu thức $P=a+b$.
\shortans[4]{$-1$}
\loigiai{
Ta có $F'(x)=a\cdot e^x+(ax+b)\cdot e^x=e^x(ax+a+b)$.\\
Mà $F'(x)=f(x)\Rightarrow \heva{&a=4\\&a+b=-1}\Rightarrow \heva{&a=4\\&b=-5.}$\\
Vậy $P=a+b=-1$.
}
\end{ex}

\begin{ex}%[2D4H2-4]
Biết rằng $F(x)=8e^x+\dfrac{na^x}{\ln a}+p\cos x$ (với $m,n,p\in \mathbb{Z}$) là một nguyên hàm của hàm số $f(x)=me^x+2a^x-2\sin x$. Tính giá trị của biểu thức $P=m+n+p$.
\shortans[4]{$12$}
\loigiai{
Ta có $F'(x)=8e^x+\dfrac{na^x}{\ln a}\cdot \ln a-p\sin x=8e^x+na^x-p\sin x$.\\
Mà $F'(x)=f(x)\Rightarrow m=8, n=2, p=2$.\\
Vậy $P=m+n+p=12$.
}
\end{ex}

\begin{ex}%[2D4H2-4]
Biết rằng  $F(x)=(ax^2+bx+c)e^{-2x}$ (với $a,b,c\in \mathbb{R}$) là một nguyên hàm của hàm số $f(x)=(-2x^2+8x-7)e^{-2x}$. Tính giá trị biểu thức $P=a+b+c$.
\shortans[4]{$-7$}
\loigiai{
Ta có $F'(x)=(2ax+b)e^{-2x}-2(ax^2+bx+c)e^{-2x}=\left[-2ax^2+2(a-b)x+b-2c \right]e^{-2x}$.\\
Mà $F'(x)=f(x)\Rightarrow \heva{&-2a=-2\\&2(a-b)=8\\&b-2c=7}\Rightarrow \heva{&a=1\\&b=-3\\&c=-5.}$\\
Vậy $P=a+b+c=1-3-5=-7$.
}
\end{ex}
\Closesolutionfile{ans}
% \indapan{6}{ans/ans-C4B1CD1-KQ}
