%%%%%----------Câu 34
\begin{ex}%[2D4H1-4]
	Hàm số $f(x)$ có đạo hàm liên tục trên $\mathbb{R}$ và $f'(x)=\mathrm{e}^{3x+2024}$, $\forall x $ thoả mã $f(-675)=1$. Giá trị của $f(-674)$ bằng
	\shortans[3]{$3{,}34$}
	\loigiai{
		Hàm số $f(x)$ có đạo hàm $f'(x)=\mathrm{e}^{3x+2024}$.\\
		Ta có $f(x)=\displaystyle\int\!\!\mathrm{e}^{3x+2024}\mathrm{d}x =\dfrac{1}{3}\mathrm{e}^{3x+2024}+C$.\\ 
		Suy ra $f(x)=\dfrac{1}{3}\mathrm{e}^{3x+2024}+C$.\\
		Với $f(-675)=1 \Rightarrow 1 =\dfrac{1}{3}\mathrm{e}^{3\cdot(-675)+2024}+C \Rightarrow C=1-\dfrac{1}{3\mathrm{e}}$.\\
		Vậy $f(x)=\dfrac{1}{3}\mathrm{e}^{3x+2024}+1-\dfrac{1}{3\mathrm{e}}$.\\
		Giá trị $f(-274)=\dfrac{1}{3}\mathrm{e}^2+1-\dfrac{1}{3\mathrm{e}}=3{,}34$.
	}
\end{ex}
%%%%%----------Câu 35
\begin{ex}%[2D4H1-4]
	Hàm số $f(x)$ có đạo hàm liên tục trên $\mathbb{R}$ và $f'(x)=3^{x+2}\cdot2^{2x+1}$, $\forall x$ thoả mãn $f(0)~=~\dfrac{1}{2\ln 2}$.\break Giá trị của $f(1)$ bằng
	\shortans[3]{$80{,}4$}
	\loigiai{
		Hàm số $f(x)$ có đạo hàm $f'(x)=3^{x+2}\cdot2^{2x+1}$.\\
		Ta có $f(x)=\displaystyle\int3^{x+2}\cdot2^{2x+1}\mathrm{d}x =\int3^2\cdot3^x\cdot2\cdot4^x\mathrm{d}x=18\int12^x\mathrm{d}x =18\cdot\dfrac{12^x}{\ln 12}+C$.\\
		Suy ra $f(x)=18\cdot\dfrac{12^x}{\ln 12}+C$.\\
		Với $f(0)=\dfrac{1}{2\ln 2}$
		$\Rightarrow \dfrac{1}{2\ln 2}=18\dfrac{1}{\ln 12}+C \Rightarrow C=\dfrac{1}{2\ln 2}-\dfrac{18}{2\ln 2+\ln 3}$.\\
		Vậy $f(x)=18\cdot\dfrac{12^x}{\ln 12}+\dfrac{1}{2\ln 2}-\dfrac{18}{2\ln 2+\ln 3}$.\\
		Giá trị $f(1)=18\cdot\dfrac{12}{\ln 12}+\dfrac{1}{2\ln 2}-\dfrac{18}{2\ln 2+\ln 3}=\dfrac{216}{\ln 12}+\dfrac{1}{\ln 4}-\dfrac{18}{\ln 4+\ln 3}=80{,}4$.
	}
\end{ex}
%%%%%----------Câu 36
\begin{ex}%[2D4H1-4]
	Hàm số $f(x)$ có đạo hàm liên tục trên $\mathbb{R}$ và $f'(x)=\left( 3^x+5^x \right)^2$, $\forall x$ thoả mãn\break $f(0)=\dfrac{1}{\ln 5+\ln 3+\ln 2}$. Giá trị của $f(1)$ bằng
	\shortans[3]{$19{,}9$}
	\loigiai{
		Hàm số $f(x)$ có đạo hàm $f'(x)=\left( 3^x+5^x \right)^2$.\\
		Ta có 
		$\begin{aligned}[t]
			f(x)&=\displaystyle\int(3^x+5^x)^2\mathrm{d}x\\
			&=\int(9^x+30^x+25^x)\mathrm{d}x\\
			&=\dfrac{9^x}{\ln 9}+\dfrac{30^x}{\ln 30}+\frac{25^x}{\ln 25}+C\\
			&=\dfrac{9^x}{2\ln 3}+\dfrac{30^x}{\ln 5+\ln 3+\ln 2}+\dfrac{25^x}{2\ln 5}+C.
		\end{aligned}$\\
		Suy ra $f(x)=\dfrac{9^x}{2\ln 3}+\dfrac{30^x}{\ln 5+\ln 3+\ln 2}+\dfrac{25^x}{2\ln 5}+C$.\\
		Với 
		$\begin{aligned}[t]
			f(0)=&\ \dfrac{1}{\ln 5+\ln 3+\ln 2}\\
			\Rightarrow &\ \dfrac{1}{\ln 5+\ln 3+\ln 2} =\dfrac{1}{2\ln 3}+\dfrac{1}{\ln 5+\ln 3+\ln 2}+\dfrac{1}{2\ln 5}+C\\
			\Leftrightarrow &\ C=-\dfrac{1}{2\ln 3}-\dfrac{1}{2\ln 5}.
		\end{aligned}$\\
		Vậy
		$\begin{aligned}[t]
			f(x)&=\dfrac{9^x}{2\ln 3}+\dfrac{30^x}{\ln 5+\ln 3+\ln 2}+\dfrac{25^x}{2\ln 5}-\dfrac{1}{2\ln 3}-\dfrac{1}{2\ln 5}\\
			&=\dfrac{9^x}{\ln 9}+\dfrac{30^x}{\ln 30}+\dfrac{25^x}{\ln 25}-\dfrac{1}{\ln 9}-\dfrac{1}{\ln 25}.
		\end{aligned}$\\
		Giá trị của
		$\begin{aligned}[t]
			f(1)&=\dfrac{9}{\ln 9}+\dfrac{30}{\ln 30}+\dfrac{25}{\ln 25}-\dfrac{1}{\ln 9}-\dfrac{1}{\ln 25}\\
			&=\dfrac{8}{\ln 9}+\dfrac{30}{\ln 30}+\dfrac{24}{\ln 25}\\
			&=19{,}9.
		\end{aligned}$
	}
\end{ex}
\Closesolutionfile{ans}
% \indapan{6}{ans/ans-C4B1CD2-CAU31_33-KQ}
