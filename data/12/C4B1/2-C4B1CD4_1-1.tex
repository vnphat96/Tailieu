\chude{ỨNG DỤNG NGUYÊN HÀM TRONG THỰC TIỄN}
\begin{dang}{Ứng Dụng Nguyên Hàm Trong Bài Toán Chuyển Động}
	Giả sử $v(t)$ là vận tốc của vật $ {M}$ tại thời điểm $t$ và $s(t)$ là quãng đường vật đi được sau khoảng thời gian $t$ tính từ lúc bắt đầu chuyển động. Ta có mối liên hệ giữa $s(t)$ và $v(t)$ như sau.
	\begin{itemize}
		\item  Đạo hàm của quãng đường là vận tốc $s'(t)=v(t)$.
		\item  Nguyên hàm của vận tốc là quãng đường $s(t)=\displaystyle\int v(t)  \mathrm{\,d} t$.
	\end{itemize}
	Nếu gọi $a(t)$ là gia tốc của vật M thì ta có mối liên hệ giữa $v(t)$ và $a(t)$ như sau.
	\begin{itemize}
		\item Đạo hàm của vận tốc là gia tốc $v'(t)=a(t)$.
		\item Nguyên hàm của gia tốc là vận tốc $v(t)=\displaystyle\int\limits a(t)  \mathrm{\,d} t$.
	\end{itemize}
\end{dang}

\setcounter{ex}{0}
\begin{ex}%[2D4V1-6]
	Một ô tô đang chạy với vận tốc $20$ m/s thì người lái đạp phanh. Sau khi đạp phanh, ô tô chuyển động chậm dần đều với vận tốc $v(t)=-40t+20$ m/s, trong đó $t$ là khoảng thời gian tính bằng giây kể từ lúc bắt đầu đạp phanh. Gọi  $s(t)$ là quãng đường xe ô tô đi được trong thời gian $t$  (giây) kể từ lúc đạp phanh. Hỏi từ lúc đạp phanh đến khi dừng hẳn, ô tô còn di chuyển bao nhiêu mét?
	\choice{$5$ cm}{$7{,}5$ m}{$\dfrac{5}{2}$ m}{\True $5$ m}
	\loigiai{
		Ta có $v(t)=-40t+20$.\\
		Suy ra $s(t)=\displaystyle\int v(t)\mathrm{\,d}t=\displaystyle\int (-40t+20)\mathrm{\,d}t=-20t^2+20t+C$.\\
		Chọn $t=0$ suy ra $s(0)=0\Rightarrow C=0$.\\
		Khi đó $s(t)=-20t^2+20t$.\\
		Khi xe dừng hẳn $v(t)=0\Leftrightarrow -40t+20=0\Leftrightarrow t=0{,}5$.\\
		Từ lúc đạp phanh đến khi dừng hẳn, ô tô còn di chuyển được\\ $s(0{,}5)=-20\cdot (0{,}5)^2+20\cdot 0{,}5=5$ m.
	}
\end{ex}