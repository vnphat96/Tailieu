\chude{NGUYÊN HÀM CÓ ĐIỀU KIỆN}
\begin{dang}{BÀI TOÁN CHO HÀM $f(x)$, TÌM NGUYÊN HÀM CỦA $f(x)$}
\end{dang}
\Opensolutionfile{ans}[ans/ans-C4B1CD2-LC]
\TN
\begin{ex}%[2D4H2-2]
Hàm số $F(x)$ là một nguyên hàm của hàm số $f(x)=\dfrac{1}{x}$ trên $(-\infty;0)$ thỏa mãn $F(-2)=0$. Khẳng định nào sau đây \textbf{đúng}?
\choice
{\True $F(x)=\ln \left(-\dfrac{x}{2} \right),\,\forall x\in (-\infty;0)$}
{$F(x)=\ln \left|x\right|+C,\,\forall x\in (-\infty;0)$ với $C$ là một số thực bất kì}
{$F(x)=\ln \left|x\right|+\ln 2,\,\forall x\in (-\infty;0)$}
{$F(x)=\ln \left(-x\right)+C,\,\forall x\in (-\infty;0)$ với $C$ là một số thực bất kì}
\loigiai{
Ta có $F(x)=\displaystyle\int \dfrac{1}{x}\mathrm{\,d}x=\ln \left|x\right|+C=\ln (-x)+C,\,\forall x\in (-\infty;0)$.\\
Lại có $F(-2)=0\Rightarrow \ln 2+C=0\Rightarrow C=-\ln 2$.\\
Do đó $F(x)=\ln (-x)-\ln 2=\ln \left(-\dfrac{x}{2}\right)$.
}
\end{ex}

\begin{ex}%[2D4H2-4]
Biết $F(x)$ là một nguyên hàm của hàm số $f(x)=e^{2x}$ và $F(0)=0$. Giá trị của $F(\ln 3)$ bằng
\choice
{$2$}
{$6$}
{$8$}
{\True $4$}
\loigiai{
Ta có $F(x)=\displaystyle\int e^{2x}\mathrm{\,d}x=\dfrac{1}{2}e^{2x}+C$.\\
Lại có $F(0)=0\Rightarrow \dfrac{1}{2}+C=0\Rightarrow C=-\dfrac{1}{2}$.\\
Do đó $F(\ln 3)=\dfrac{1}{2}e^{2\ln 3}-\dfrac{1}{2}=4$.
}
\end{ex}

\begin{ex}%[2D4H2-4]
Cho $F(x)$ là một nguyên hàm của $f(x)=2^x+x+1$. Biết $F(0)=1$. Giá trị của $F(-1)$ bằng
\choice
{$F(-1)=\dfrac{1}{2\ln 2}$}
{\True $F(-1)=\dfrac{1}{2}-\dfrac{1}{2\ln 2}$}
{$F(-1)=1+\dfrac{1}{2\ln 2}$}
{$F(-1)=\dfrac{1}{2}-\dfrac{1}{\ln 2}$}
\loigiai{
Ta có $F(x)=\displaystyle\int (2^x+x+1)\mathrm{\,d}x=\dfrac{2^x}{\ln 2}+\dfrac{x^2}{2}+x+C$.\\
Lại có $F(0)=1\Rightarrow \dfrac{1}{\ln 2}+C=1\Rightarrow C=1-\dfrac{1}{\ln 2}$.\\
Do đó $F(-1)=\dfrac{1}{2\ln 2}+\dfrac{1}{2}-1+1-\dfrac{1}{\ln 2}=\dfrac{1}{2}-\dfrac{1}{2\ln 2}$.
}
\end{ex}

\begin{ex}%[2D4H2-3]
Tìm nguyên hàm $F(x)$ của hàm số $f(x)=\sin x+\cos x$ thoả mãn $F\left(\dfrac{\pi}{2}\right)=2$.
\choice
{$F(x)=-\cos x+\sin x+3$}
{$F(x)=-\cos x+\sin x-1$}
{\True $F(x)=-\cos x+\sin x+1$}
{$F(x)=\cos x-\sin x+3$}
\loigiai{
Ta có $F(x)=\displaystyle\int (\sin x+\cos x)\mathrm{\,d}x=-\cos x+\sin x+C$.\\
Lại có $F\left(\dfrac{\pi}{2}\right)=2\Rightarrow -\cos\dfrac{\pi}{2}+\sin\dfrac{\pi}{2}+C=2\Rightarrow C=1$.\\
Do đó $F(x)=-\cos x+\sin x+1$.
}
\end{ex}

\begin{ex}%[2D4H2-4]
Cho $F(x)$ là một nguyên hàm của hàm số $f(x)=e^x+2x$ thỏa mãn $F(0)=\dfrac{3}{2}$. Tìm $F(x)$.
\choice
{\True $F(x)=e^x+x^2+\dfrac{1}{2}$}
{$F(x)=e^x+x^2+\dfrac{5}{2}$}
{$F(x)=e^x+x^2+\dfrac{3}{2}$}
{$F(x)=e^x+x^2-\dfrac{1}{2}$}
\loigiai{
Ta có $F(x)=\displaystyle\int (e^x+2x)\mathrm{\,d}x=e^x+x^2+C$.\\
Lại có $F(0)=\dfrac{3}{2}\Rightarrow 1+C=\dfrac{3}{2}\Rightarrow C=\dfrac{1}{2}$.\\
Do đó $F(x)=e^x+x^2+\dfrac{1}{2}$.
}
\end{ex}

\begin{ex}%[2D4H2-2]
Cho hàm số $f(x)=\heva{&2x-1&\text{khi}\quad&x\ge 1\\&3x^2-2&\text{khi}\quad&x<1}$, giả sử $F$ là nguyên hàm của  $f$ trên $\mathbb{R}$ thỏa mãn $F(0)=2$. Giá trị của $F(-1)+2F(2)$ bằng
\choice
{\True $9$}
{$15$}
{$11$}
{$6$}
\loigiai{
Ta có $\displaystyle\int (2x-1)\mathrm{\,d}x=x^2-x+C_1$ và $\displaystyle\int (3x^2-2)\mathrm{\,d}x=x^3-2x+C_2$.\\
Suy ra $F(x)=\displaystyle\int f(x)\mathrm{\,d}x=\heva{&x^2-x+C_1&\text{khi}\quad&x\ge 1\\&x^3-2x+C_2&\text{khi}\quad&x<1.}$ 
Lại có $F(0)=2\Rightarrow C_2=2$.\\
Mặt khác hàm số $F$ là nguyên hàm của $f$ trên $\mathbb{R}$ nên $y=F(x)$ liên tục tại $x=1$.\\
Suy ra  $\lim\limits_{ x\to 1^{+}} F(x)=\lim\limits_{ x\to 1^{-}} F(x)\Rightarrow C_1=1$.\\
Khi đó ta có $F(x)=\heva{&x^2-x+1&\text{khi}\quad&x\ge 1\\&x^3-2x+2&\text{khi}\quad&x<1}\Rightarrow \heva{&F(-1)=3\\&F(2)=3.}$  \\
Vậy $F(-1)+2F(2)=9$.  
}
\end{ex}

